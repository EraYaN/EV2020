%!TEX program = xelatex+makeindex+bibtex
\documentclass[final]{scrreprt} %scrreprt of scrartcl
% Include all project wide packages here.
\usepackage{fullpage}
\usepackage{polyglossia}
\setmainlanguage{english}
\usepackage{csquotes}
\usepackage{graphicx}
\usepackage{epstopdf}
\usepackage{pdfpages}
\usepackage{caption}
\usepackage[list=true]{subcaption}
\usepackage{float}
\usepackage{standalone}
\usepackage{import}
\usepackage{tocloft}
\usepackage{wrapfig}
\usepackage{authblk}
\usepackage{array}
\usepackage{booktabs}
\usepackage[toc,page,title,titletoc]{appendix}
\usepackage{xunicode}
\usepackage{fontspec}
\usepackage{pgfplots}
\usepackage{SIunits}
\usepackage{units}
\pgfplotsset{compat=newest}
\pgfplotsset{plot coordinates/math parser=false}
\newlength\figureheight 
\newlength\figurewidth
\usepackage{amsmath}
\usepackage{mathtools}
\usepackage{unicode-math}
\usepackage[
    backend=bibtexu,
	texencoding=utf8,
bibencoding=utf8,
    style=ieee,
    sortlocale=en_US,
    language=auto
]{biblatex}
\usepackage{listings}
\newcommand{\includecode}[3][c]{\lstinputlisting[caption=#2, escapechar=, style=#1]{#3}}
\newcommand{\superscript}[1]{\ensuremath{^{\textrm{#1}}}}
\newcommand{\subscript}[1]{\ensuremath{_{\textrm{#1}}}}


\newcommand{\chapternumber}{\thechapter}
\renewcommand{\appendixname}{Bijlage}
\renewcommand{\appendixtocname}{Bijlagen}
\renewcommand{\appendixpagename}{Bijlagen}

\usepackage[hidelinks]{hyperref} %<--------ALTIJD ALS LAATSTE

\renewcommand{\familydefault}{\sfdefault}

\setmainfont[Ligatures=TeX]{Myriad Pro}
\setmathfont{Asana Math}
\setmonofont{Lucida Console}

\usepackage{titlesec, blindtext, color}
\definecolor{gray75}{gray}{0.75}
\newcommand{\hsp}{\hspace{20pt}}
\titleformat{\chapter}[hang]{\Huge\bfseries}{\chapternumber\hsp\textcolor{gray75}{|}\hsp}{0pt}{\Huge\bfseries}
\renewcommand{\familydefault}{\sfdefault}
\renewcommand{\arraystretch}{1.2}
\setlength\parindent{0pt}

%For code listings
\definecolor{black}{rgb}{0,0,0}
\definecolor{browntags}{rgb}{0.65,0.1,0.1}
\definecolor{bluestrings}{rgb}{0,0,1}
\definecolor{graycomments}{rgb}{0.4,0.4,0.4}
\definecolor{redkeywords}{rgb}{1,0,0}
\definecolor{bluekeywords}{rgb}{0.13,0.13,0.8}
\definecolor{greencomments}{rgb}{0,0.5,0}
\definecolor{redstrings}{rgb}{0.9,0,0}
\definecolor{purpleidentifiers}{rgb}{0.01,0,0.01}


\lstdefinestyle{csharp}{
language=[Sharp]C,
showspaces=false,
showtabs=false,
breaklines=true,
showstringspaces=false,
breakatwhitespace=true,
escapeinside={(*@}{@*)},
columns=fullflexible,
commentstyle=\color{greencomments},
keywordstyle=\color{bluekeywords}\bfseries,
stringstyle=\color{redstrings},
identifierstyle=\color{purpleidentifiers},
basicstyle=\ttfamily\small}

\lstdefinestyle{c}{
language=C,
showspaces=false,
showtabs=false,
breaklines=true,
showstringspaces=false,
breakatwhitespace=true,
escapeinside={(*@}{@*)},
columns=fullflexible,
commentstyle=\color{greencomments},
keywordstyle=\color{bluekeywords}\bfseries,
stringstyle=\color{redstrings},
identifierstyle=\color{purpleidentifiers},
}

\lstdefinestyle{matlab}{
language=Matlab,
showspaces=false,
showtabs=false,
breaklines=true,
showstringspaces=false,
breakatwhitespace=true,
escapeinside={(*@}{@*)},
columns=fullflexible,
commentstyle=\color{greencomments},
keywordstyle=\color{bluekeywords}\bfseries,
stringstyle=\color{redstrings},
identifierstyle=\color{purpleidentifiers}
}

\lstdefinestyle{vhdl}{
language=VHDL,
showspaces=false,
showtabs=false,
breaklines=true,
showstringspaces=false,
breakatwhitespace=true,
escapeinside={(*@}{@*)},
columns=fullflexible,
commentstyle=\color{greencomments},
keywordstyle=\color{bluekeywords}\bfseries,
stringstyle=\color{redstrings},
identifierstyle=\color{purpleidentifiers}
}

\lstdefinestyle{xaml}{
language=XML,
showspaces=false,
showtabs=false,
breaklines=true,
showstringspaces=false,
breakatwhitespace=true,
escapeinside={(*@}{@*)},
columns=fullflexible,
commentstyle=\color{greencomments},
keywordstyle=\color{redkeywords},
stringstyle=\color{bluestrings},
tagstyle=\color{browntags},
morestring=[b]",
  morecomment=[s]{<?}{?>},
  morekeywords={xmlns,version,typex:AsyncRecords,x:Arguments,x:Boolean,x:Byte,x:Char,x:Class,x:ClassAttributes,x:ClassModifier,x:Code,x:ConnectionId,x:Decimal,x:Double,x:FactoryMethod,x:FieldModifier,x:Int16,x:Int32,x:Int64,x:Key,x:Members,x:Name,x:Object,x:Property,x:Shared,x:Single,x:String,x:Subclass,x:SynchronousMode,x:TimeSpan,x:TypeArguments,x:Uid,x:Uri,x:XData,Grid.Column,Grid.ColumnSpan,Click,ClipToBounds,Content,DropDownOpened,FontSize,Foreground,Header,Height,HorizontalAlignment,HorizontalContentAlignment,IsCancel,IsDefault,IsEnabled,IsSelected,Margin,MinHeight,MinWidth,Padding,SnapsToDevicePixels,Target,TextWrapping,Title,VerticalAlignment,VerticalContentAlignment,Width,WindowStartupLocation,Binding,Mode,OneWay,xmlns:x}
}

%defaults
\lstset{
basicstyle=\ttfamily\small,
extendedchars=false,
numbers=left,
numberstyle=\ttfamily\tiny,
stepnumber=1,
tabsize=4,
numbersep=5pt
}
\addbibresource{../../library/bibliography.bib}
\title{Module 3 - Control and Trajectory Tracking}
\author{Erwin {de Haan} \and Xenia {Wesdijk}}
\begin{document}

\chapter{Task 1 \& 2}
\label{ch:mod3-tsk1-2}

\section{Task 1}
\label{sec:mod3-tsk1}

\begin{enumerate}
\item \textit{Derive a corresponding state-space model. The states are $x_{1}$: position, $x_{2}$: velocity. The input is u: the motor force, and the measured output is y = $x_{1}$: the position.}\\
\\
Starting with equation \ref{eq:motion} we first transformed it into equation \ref{eq:switch} from which we were able to derive a state-space model. This state-space model is showed in \ref{eq:matrix}. $\ddot{x_{1}}$ and $\dot{x_{1}}$ from \ref{eq:motion} and \ref{eq:switch} are replaced by $\ddot{x}$ and $\dot{x}$ in the state-space model in equation \ref{eq:matrix}.

\begin{equation}
	m \ddot{x_{1}} = -\rho_f \dot{x_{1}} +\textit{u}
	\label{eq:motion}
\end{equation}
\begin{equation}
	\ddot{x_{1}} = -\frac{\rho_f}{m} \dot{x_{1}} +(\frac{1}{m})\textit{u}
	\label{eq:switch}
\end{equation}
\begin{equation}
	\begin{bmatrix}
		\dot{x} \\
		\ddot{x}
	\end{bmatrix} =
	\begin{bmatrix}
		0 & 1 \\
		0 & -\frac{\rho_f}{m}
	\end{bmatrix}
	\begin{bmatrix}
		x \\
		\dot{x}
	\end{bmatrix} +
	\begin{bmatrix}
		0 \\
		\frac{1}{m}
	\end{bmatrix}
	\textit{u}
	\label{eq:matrix}
\end{equation}
\\
\item \textit{What is the order of this system? What are the coefficients for the characteristic equation? Is this system stable? Controllable? Observable? If the system is not stable, is it stabilizable?}
\label{ref:task12}\\
\\
 In the model of equation \ref{eq:matrix} $x$ equals the position and $\dot{x}$ equals the velocity (which is the derivative of $x$). Our derived state-space model is thus a second order model. \\
In order to find the characteristic equation we have to use equation \ref{eq:labda} on the model of equation \ref{eq:matrix}. This leads to the matrix in \ref{eq:fmatrix}. When calculating the determinant from \ref{eq:fmatrix} we find the characteristic equation as displayed in \ref{eq:characteristic}. Solving this equation for $\lambda$ we find the two eigenvalues $\lambda$ =  $0$ and $\lambda$ = $-\frac{\rho_f}{m}$. In order for our system to be stable it needs to have two poles on the left half of the complex plane. It now, however, has one on the imaginary axis ($\lambda$ = $0$) and the other one on the left half of the complex plane ($\lambda$ = $-\frac{\rho_f}{m}$), so our system is not stable.\\
\begin{equation}
	\left| I \lambda - A \right| = 0
	\label{eq:labda}
\end{equation}

\begin{equation}
	\begin{vmatrix}
		\lambda & -1 \\
		0 & \lambda + \frac{\rho_f}{m}
	\end{vmatrix} = 0
	\label{eq:fmatrix}
\end{equation}

\begin{equation}
	\lambda(\lambda + \frac{\rho_f}{m}) = 0
	\label{eq:characteristic}
\end{equation}
\\
Because our A matrix is not diagonal it means that there is coupling between the states $\dot{x}$ and $\ddot{x}$. Due to the coupling between the states we know that our system is controllable.\\
Our system is also observable since the rank is two, which is the same as the order of the system.\\
Lastly we need to check if our system is stabilizable. A system is stabilizable when there is a matrix $K$ that is able to get the eigenvalues on the left side of the complex plane. Equation \ref{eq:K} can be used to find such a $K$ matrix. Since we can't find a $K$ matrix who can do this or system is not stabilizable.

\begin{equation}
	(A + BK)
\label{eq:K}
\end{equation}


\end{enumerate}

\section{Task 2}
\label{sec:mod3-tsk2}
\begin{enumerate}
\item \textit{Combine the previously found state-space model of the car position with the DC motor model into a single state-space model. Make sure you relate the velocity of the motor to the forward velocity of the car and compute the parameter $k_{w}$ given the wheel radius.}\\
\\
We started out with the following equation. Which was given in our manual (\cite{epo4-manual}).
\begin{equation}
V = iR + L\frac{d}{dt}i + e
\label{eq:start}
\end{equation}
Also given was:
\begin{equation}
e = k_t \omega
\label{eq:e}
\end{equation}
Filling equation \ref{eq:e} in to equation \ref{eq:start} gives: 
\begin{equation}
V = iR + L\frac{d}{dt}i + k_t \omega
\label{eq:1}
\end{equation}

\begin{equation}
i = \frac{\tau_m}{k_t}
\label{eq:2}
\end{equation}

\begin{equation}
V = \frac{\tau_m}{k_t}R + L\frac{d}{dt}\frac{\tau_m}{k_t} +  k_t \omega
\label{eq:3}
\end{equation}

\begin{equation}
\tau_m = \frac{\tau_w}{k_g}
\label{eq:4}
\end{equation}

\begin{equation}
V = \frac{\tau_w}{k_t k_g}R + L\frac{d}{dt}\frac{\tau_w}{k_t k_g} +  k_t \omega
\label{eq:5}
\end{equation}

\begin{equation}
\tau_w = \frac{\tau}{k_w}
\label{eq:6}
\end{equation}

\begin{equation}
V = \frac{\tau}{k_t k_g k_w}R + L\frac{d}{dt}\frac{\tau}{k_t k_g k_w} +  k_t \omega
\label{eq:final}
\end{equation}\\
By subsituting all our given equations (step by step) into euqation \ref{eq:start}, as is shown above, we finally find equation \ref{eq:final}. We still have $\omega$ in this equation. We know that $\omega$ = $\frac{\dot{x}}{r}$. Using this in equation \ref{eq:final} gives us equation \ref{eq:withradius}. In this case $\tau$ represents the force, which is equal to $ma$. $a$ is the acceleration, same as $\ddot{x}$. So $\tau$ = $m\ddot{x}$. Using this to replace $\tau$ in equation \ref{eq:withradius} gives us equation \ref{eq:filled}. 

\begin{equation}
V = \frac{\tau}{k_t k_g k_w}R + L\frac{d}{dt}\frac{\tau}{k_t k_g k_w} +  \frac{k_t \dot{x}}{r}
\label{eq:withradius}
\end{equation}

\begin{equation}
V = \frac{m\ddot{x}}{k_t k_g k_w}R + L\frac{d}{dt}\frac{m\ddot{x}}{k_t k_g k_w} +  \frac{k_t \dot{x}}{r}
\label{eq:filled}
\end{equation}\\
\\ 
Equation \ref{eq:filled} can also be written as equation \ref{eq:new}. Equation \ref{eq:new} is a third order system. In \ref{sec:vraag2} of \ref{sec:mod3-tsk2} we can neglect L and we will see that we're left with a second order system, also see equation \ref{eq:tau}. 
\begin{equation}
V = \frac{m\ddot{x}}{k_t k_g k_w}R + \frac{Lm\dddot{x}}{k_t k_g k_w} +  \frac{k_t \dot{x}}{r}
\label{eq:new}
\end{equation}\\
\\
The last thing we still had to do was compute de parameter $k_{w}$ given the radius of the wheel. From the definition of $\tau$ ($\tau$ = $F$ $\cdot$ $r$) and from equation \ref{eq:6} we can find that $k_{w}$ =  $\frac{1}{r_w}$ in which $r_{w}$ stands for the radius of the wheel. 


\item \textit{What is the order of this system?\\ Simplify  by setting the motor coil inductance L to zero. What are the coefficients for the characteristic equation? Is this system stable? Controllable? Observable? If the system is not stable, is it stabilizable?}
\label{sec:vraag2}\\
\\
When we set the motor coil inductance to zero and use equation \ref{eq:withradius} we get equation \ref{eq:Lzero}.
\begin{equation}
V = \frac{\tau}{k_t k_g k_w}R +  \frac{k_t \dot{x}}{r}
\label{eq:Lzero}
\end{equation}\\
Rewriting this equation we get equation \ref{eq:tau}. 
\begin{equation}
\tau = \frac{k_t k_g k_w}{R} V +  \frac{k_t² k_g k_w \dot{x}}{Rr}
\label{eq:tau}
\end{equation}
\\
Putting equation \ref{eq:tau} into equation \ref{eq:motion} we get \ref{eq:motiontau}.

\begin{equation}
m\ddot{x} = -( \rho_f + \frac{k_t² k_g k_w}{Rr}) \dot{x} + \frac{k_t k_g k_w}{R}V
\label{eq:motiontau}
\end{equation}
\\
Solving this for $\ddot{x}$ we get equation \ref{eq:finito}.
\\
\begin{equation}
\ddot{x} = -( \frac{\rho_f}{m} + \frac{k_t² k_g k_w}{mRr}) \dot{x} + \frac{k_t k_g k_w}{mR}V
\label{eq:finito}
\end{equation}\\

And finally equation \ref{eq:finito} leads us to \ref{eq:total} which is a simplified model for the DC motor model. 
\begin{equation}
	\begin{bmatrix}
		\dot{x} \\
		\ddot{x}
	\end{bmatrix} =
	\begin{bmatrix}
		0 & 1 \\
		0 & -\frac{\rho_f}{m} - \frac{k_t² k_g k_w}{mRr}
	\end{bmatrix}
	\begin{bmatrix}
		x \\
		\dot{x}
	\end{bmatrix} +
	\begin{bmatrix}
		0 \\
		\frac{k_t k_g k_w}{mR}
	\end{bmatrix}
	V
	\label{eq:total}
\end{equation}
\\
Now we again need to find the coefficients of the characteristic equation. By using the same methods as in section \ref{ref:task12} from \ref{sec:mod3-tsk1} we find that the eigenvalues are $\lambda$ = $0$ and $\lambda$ = $ -\frac{\rho_f}{m} - \frac{k_t² k_g k_w}{mRr}$. Since the matrix has the same form as in section \ref{ref:task12} from \ref{sec:mod3-tsk1}, the same things apply to this matrix. 
So it is not stable, but stablizable. It's observable and controllable.


\end{enumerate}





\end{document}