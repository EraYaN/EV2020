%!TEX program = xelatex+makeindex+bibtex
\documentclass[final]{scrreprt} %scrreprt of scrartcl
\input{../../library/preamble.tex}
\input{../../library/style.tex}
\addbibresource{../../library/bibliography.bib}
\begin{document}

\chapter{Assignment 1: System identification}
\label{ch:mod3-ass1}
\section{Task 3: Model parameter identification}
\label{sec:mod3-tsk3}
The goal here is to determine the A and B matrix to be used in the model created in chapter \ref{ch:mod3-tsk1-2}.
In figure \ref{fig:KITT-input-output-model-output} are displayed the input and outputsiganls and the fitted model.
The fitted model does not look quite right.
Firstly this is because the input signal was not normalized for the offset needed in the driving signal.
Secondly the model does not have the same “slowness” as KITT, so the model’s output moved left.

\begin{figure}[H]
	\centering
    	\setlength\figureheight{4cm}
    	\setlength\figurewidth{0.8\linewidth}
    	% This file was created by matlab2tikz v0.4.6 running on MATLAB 8.3.
% Copyright (c) 2008--2014, Nico Schlömer <nico.schloemer@gmail.com>
% All rights reserved.
% Minimal pgfplots version: 1.3
% 
% The latest updates can be retrieved from
%   http://www.mathworks.com/matlabcentral/fileexchange/22022-matlab2tikz
% where you can also make suggestions and rate matlab2tikz.
% 
\begin{tikzpicture}

\begin{axis}[%
width=\figurewidth,
height=\figureheight,
scale only axis,
xmin=0,
xmax=20,
xlabel={t (s)},
ymin=-250,
ymax=200,
ylabel={x (cm)},
legend style={at={(0.01,0.01)},anchor=south west,draw=black,fill=white,legend cell align=left}
]
\addplot [color=blue,solid]
  table[row sep=crcr]{
0	0	\\
0.1	0	\\
0.2	0	\\
0.3	0	\\
0.4	0	\\
0.5	0	\\
0.6	0	\\
0.7	0	\\
0.8	0	\\
0.9	0	\\
1	0	\\
1.1	0	\\
1.2	0	\\
1.3	0	\\
1.4	0	\\
1.5	0	\\
1.6	0	\\
1.7	0	\\
1.8	0	\\
1.9	0	\\
2	0	\\
2.1	0	\\
2.2	0	\\
2.3	0	\\
2.4	0	\\
2.5	0	\\
2.6	0	\\
2.7	0	\\
2.8	0	\\
2.9	0	\\
3	0	\\
3.1	0	\\
3.2	0	\\
3.3	0	\\
3.4	0	\\
3.5	5	\\
3.6	5	\\
3.7	5	\\
3.8	5	\\
3.9	5	\\
4	5	\\
4.1	5	\\
4.2	5	\\
4.3	5	\\
4.4	5	\\
4.5	5	\\
4.6	5	\\
4.7	5	\\
4.8	5	\\
4.9	5	\\
5	0	\\
5.1	0	\\
5.2	0	\\
5.3	0	\\
5.4	-9	\\
5.5	-9	\\
5.6	-9	\\
5.7	-9	\\
5.8	-9	\\
5.9	-9	\\
6	-9	\\
6.1	-9	\\
6.2	-9	\\
6.3	-9	\\
6.4	-9	\\
6.5	-9	\\
6.6	-9	\\
6.7	-9	\\
6.8	-9	\\
6.9	-9	\\
7	0	\\
7.1	0	\\
7.2	0	\\
7.3	0	\\
7.4	5	\\
7.5	5	\\
7.6	5	\\
7.7	5	\\
7.8	5	\\
7.9	5	\\
8	5	\\
8.1	5	\\
8.2	5	\\
8.3	5	\\
8.4	5	\\
8.5	5	\\
8.6	5	\\
8.7	5	\\
8.8	5	\\
8.9	5	\\
9	0	\\
9.1	0	\\
9.2	0	\\
9.3	0	\\
9.4	-9	\\
9.5	-9	\\
9.6	-9	\\
9.7	-9	\\
9.8	-9	\\
9.9	-9	\\
10	-9	\\
10.1	-9	\\
10.2	-9	\\
10.3	-9	\\
10.4	-9	\\
10.5	-9	\\
10.6	-9	\\
10.7	-9	\\
10.8	-9	\\
10.9	-9	\\
11	0	\\
11.1	0	\\
11.2	0	\\
11.3	0	\\
11.4	5	\\
11.5	5	\\
11.6	5	\\
11.7	5	\\
11.8	5	\\
11.9	5	\\
12	5	\\
12.1	5	\\
12.2	5	\\
12.3	5	\\
12.4	5	\\
12.5	5	\\
12.6	5	\\
12.7	5	\\
12.8	5	\\
12.9	5	\\
13	0	\\
13.1	0	\\
13.2	0	\\
13.3	0	\\
13.4	-9	\\
13.5	-9	\\
13.6	-9	\\
13.7	-9	\\
13.8	-9	\\
13.9	-9	\\
14	-9	\\
14.1	-9	\\
14.2	-9	\\
14.3	-9	\\
14.4	-9	\\
14.5	-9	\\
14.6	-9	\\
14.7	-9	\\
14.8	-9	\\
14.9	-9	\\
15	0	\\
15.1	0	\\
15.2	0	\\
15.3	0	\\
15.4	0	\\
15.5	0	\\
15.6	0	\\
15.7	0	\\
15.8	0	\\
15.9	0	\\
16	0	\\
16.1	0	\\
16.2	0	\\
16.3	0	\\
16.4	0	\\
16.5	0	\\
16.6	0	\\
16.7	0	\\
16.8	0	\\
16.9	0	\\
17	0	\\
17.1	0	\\
17.2	0	\\
17.3	0	\\
17.4	0	\\
17.5	0	\\
17.6	0	\\
17.7	0	\\
17.8	0	\\
17.9	0	\\
18	0	\\
18.1	0	\\
18.2	0	\\
18.3	0	\\
18.4	0	\\
18.5	0	\\
18.6	0	\\
18.7	0	\\
18.8	0	\\
18.9	0	\\
19	0	\\
19.1	0	\\
19.2	0	\\
19.3	0	\\
19.4	0	\\
19.5	0	\\
19.6	0	\\
19.7	0	\\
19.8	0	\\
19.9	0	\\
};
\addlegendentry{Input signal to KITT};

\addplot [color=black!50!green,solid]
  table[row sep=crcr]{
0	0	\\
0.1	0	\\
0.2	0	\\
0.3	0	\\
0.4	0	\\
0.5	0	\\
0.6	0	\\
0.7	0	\\
0.8	0	\\
0.9	0	\\
1	0	\\
1.1	0	\\
1.2	0	\\
1.3	0	\\
1.4	0	\\
1.5	0	\\
1.6	0	\\
1.7	0	\\
1.8	0	\\
1.9	0	\\
2	0	\\
2.1	0	\\
2.2	0	\\
2.3	0	\\
2.4	0	\\
2.5	0	\\
2.6	0	\\
2.7	0	\\
2.8	0	\\
2.9	0	\\
3	0	\\
3.1	0	\\
3.2	0	\\
3.3	0	\\
3.4	0	\\
3.5	9.11087121541924	\\
3.6	25.3534840780933	\\
3.7	38.7688246949345	\\
3.8	49.568991885911	\\
3.9	57.9868627869341	\\
4	64.2669745562494	\\
4.1	68.6578109357305	\\
4.2	71.4054121643154	\\
4.3	72.7482114943602	\\
4.4	72.9129913153442	\\
4.5	72.1118459606455	\\
4.6	70.5400360060863	\\
4.7	68.3746196443829	\\
4.8	65.7737499623663	\\
4.9	62.8765321318145	\\
5	50.6924699606184	\\
5.1	31.3030245943278	\\
5.2	14.7524704336722	\\
5.3	0.898081625401357	\\
5.4	-26.8377727287332	\\
5.5	-65.0955231741023	\\
5.6	-96.1674386226642	\\
5.7	-120.661574067365	\\
5.8	-139.224582031806	\\
5.9	-152.51989260781	\\
6	-161.209509004461	\\
6.1	-165.93917129906	\\
6.2	-167.326609580218	\\
6.3	-165.95258830363	\\
6.4	-162.354434660808	\\
6.5	-157.021743399617	\\
6.6	-150.393957254573	\\
6.7	-142.859534493779	\\
6.8	-134.756431754988	\\
6.9	-126.373650154526	\\
7	-101.554046393708	\\
7.1	-64.0589079068669	\\
7.2	-31.9731917455498	\\
7.3	-5.03746156793844	\\
7.4	26.1885038426314	\\
7.5	60.1045168891617	\\
7.6	87.162368540216	\\
7.7	108.001618177111	\\
7.8	123.286458617355	\\
7.9	133.686054812869	\\
8	139.858371757058	\\
8.1	142.437221083857	\\
8.2	142.022232003806	\\
8.3	139.17143946511	\\
8.4	134.396179012151	\\
8.5	128.157982106113	\\
8.6	120.867176162175	\\
8.7	112.882908870231	\\
8.8	104.514335268398	\\
8.9	96.0227274418803	\\
9	78.5134184621729	\\
9.1	54.140001530851	\\
9.2	32.9978829631104	\\
9.3	14.9771504835299	\\
9.4	-16.4827233862562	\\
9.5	-58.0182873687928	\\
9.6	-91.9288160435566	\\
9.7	-118.83810283952	\\
9.8	-139.415312419358	\\
9.9	-154.351406135789	\\
10	-164.339390030463	\\
10.1	-170.058135424395	\\
10.2	-172.159486471466	\\
10.3	-171.258346420364	\\
10.4	-167.925422880161	\\
10.5	-162.682310323793	\\
10.6	-155.998593718654	\\
10.7	-148.290669001495	\\
10.8	-139.921992715062	\\
10.9	-131.204493243325	\\
11	-106.001330429326	\\
11.1	-68.0914885150842	\\
11.2	-35.5750889979044	\\
11.3	-8.20544579099765	\\
11.4	23.4471746814363	\\
11.5	57.7741843916346	\\
11.6	85.2208702221904	\\
11.7	106.421984672605	\\
11.8	122.038409524689	\\
11.9	132.73729596383	\\
12	139.175700709382	\\
12.1	141.987450764933	\\
12.2	141.772944761635	\\
12.3	139.091585404612	\\
12.4	134.456533503191	\\
12.5	128.331477833711	\\
12.6	121.129125134175	\\
12.7	113.211129482742	\\
12.8	104.889198927728	\\
12.9	96.427138419615	\\
13	78.9327396767285	\\
13.1	54.5619378562078	\\
13.2	33.4123325084251	\\
13.3	15.3760331041572	\\
13.4	-16.1056522633789	\\
13.5	-57.6676306685863	\\
13.6	-91.6077308725896	\\
13.7	-118.548493406731	\\
13.8	-139.15801618098	\\
13.9	-154.126370160794	\\
14	-164.145835356239	\\
14.1	-169.894707854946	\\
14.2	-172.024392863189	\\
14.3	-171.149476006422	\\
14.4	-167.840453607938	\\
14.5	-162.618800526402	\\
14.6	-155.954059717105	\\
14.7	-148.262649445702	\\
14.8	-139.908100626208	\\
14.9	-131.202456824213	\\
15	-106.009023770078	\\
15.1	-68.1069533454231	\\
15.2	-35.5965486304815	\\
15.3	-8.23131236865223	\\
15.4	14.3074273574179	\\
15.5	32.3900246710477	\\
15.6	46.4205996765769	\\
15.7	56.8216350123012	\\
15.8	64.0209760492101	\\
15.9	68.4410907823224	\\
16	70.4904203036806	\\
16.1	70.5566340548308	\\
16.2	69.0015946611792	\\
16.3	66.1578339284359	\\
16.4	62.3263434747673	\\
16.5	57.7754895007412	\\
16.6	52.7408704675214	\\
16.7	47.4259481569913	\\
16.8	42.0032960125309	\\
16.9	36.6163231843546	\\
17	31.3813477958045	\\
17.1	26.3899081586282	\\
17.2	21.711215627311	\\
17.3	17.3946671994757	\\
17.4	13.4723496123079	\\
17.5	9.96147938458205	\\
17.6	6.86673489337649	\\
17.7	4.1824470827708	\\
17.8	1.89462474635437	\\
17.9	-0.0172014936435971	\\
18	-1.5783253437446	\\
18.1	-2.81740401147831	\\
18.2	-3.76518834153176	\\
18.3	-4.45343788548287	\\
18.4	-4.91401176215261	\\
18.5	-5.17812383180956	\\
18.6	-5.27574906302974	\\
18.7	-5.23516692714137	\\
18.8	-5.08262712563312	\\
18.9	-4.84212285851119	\\
19	-4.53525709939142	\\
19.1	-4.18118788514366	\\
19.2	-3.79663938972609	\\
19.3	-3.39596647580941	\\
19.4	-2.99126145311291	\\
19.5	-2.59249287504074	\\
19.6	-2.20766733772959	\\
19.7	-1.84300637669858	\\
19.8	-1.50313166039922	\\
19.9	-1.19125273685779	\\
};
\addlegendentry{Model output};

\addplot [color=red,solid]
  table[row sep=crcr]{
0	0	\\
0.1	0	\\
0.2	0	\\
0.3	0	\\
0.4	0	\\
0.5	1	\\
0.6	0	\\
0.7	0	\\
0.8	1	\\
0.9	0	\\
1	1	\\
1.1	0	\\
1.2	0	\\
1.3	0	\\
1.4	0	\\
1.5	0	\\
1.6	0	\\
1.7	0	\\
1.8	0	\\
1.9	0	\\
2	0	\\
2.1	0	\\
2.2	0	\\
2.3	0	\\
2.4	1	\\
2.5	0	\\
2.6	0	\\
2.7	1	\\
2.8	1	\\
2.9	0	\\
3	0	\\
3.1	0	\\
3.2	0	\\
3.3	0	\\
3.4	0	\\
3.5	0	\\
3.6	0	\\
3.7	0	\\
3.8	1	\\
3.9	0	\\
4	0	\\
4.1	5	\\
4.2	8	\\
4.3	14	\\
4.4	14	\\
4.5	20	\\
4.6	28	\\
4.7	36	\\
4.8	45	\\
4.9	55	\\
5	65	\\
5.1	77	\\
5.2	77	\\
5.3	86	\\
5.4	95	\\
5.5	103	\\
5.6	110	\\
5.7	113	\\
5.8	113	\\
5.9	111	\\
6	111	\\
6.1	107	\\
6.2	103	\\
6.3	97	\\
6.4	91	\\
6.5	91	\\
6.6	76	\\
6.7	67	\\
6.8	67	\\
6.9	58	\\
7	49	\\
7.1	39	\\
7.2	28	\\
7.3	28	\\
7.4	10	\\
7.5	10	\\
7.6	3	\\
7.7	2	\\
7.8	3	\\
7.9	7	\\
8	12	\\
8.1	18	\\
8.2	26	\\
8.3	26	\\
8.4	34	\\
8.5	43	\\
8.6	52	\\
8.7	63	\\
8.8	63	\\
8.9	85	\\
9	97	\\
9.1	97	\\
9.2	110	\\
9.3	122	\\
9.4	133	\\
9.5	142	\\
9.6	150	\\
9.7	156	\\
9.8	156	\\
9.9	156	\\
10	154	\\
10.1	150	\\
10.2	146	\\
10.3	141	\\
10.4	134	\\
10.5	127	\\
10.6	120	\\
10.7	120	\\
10.8	111	\\
10.9	102	\\
11	93	\\
11.1	83	\\
11.2	72	\\
11.3	63	\\
11.4	54	\\
11.5	48	\\
11.6	48	\\
11.7	46	\\
11.8	48	\\
11.9	52	\\
12	56	\\
12.1	62	\\
12.2	70	\\
12.3	70	\\
12.4	78	\\
12.5	86	\\
12.6	95	\\
12.7	105	\\
12.8	116	\\
12.9	128	\\
13	128	\\
13.1	152	\\
13.2	152	\\
13.3	163	\\
13.4	173	\\
13.5	173	\\
13.6	191	\\
13.7	195	\\
13.8	196	\\
13.9	196	\\
14	194	\\
14.1	190	\\
14.2	186	\\
14.3	180	\\
14.4	175	\\
14.5	167	\\
14.6	160	\\
14.7	160	\\
14.8	152	\\
14.9	143	\\
15	153	\\
15.1	124	\\
15.2	124	\\
15.3	105	\\
15.4	96	\\
15.5	89	\\
15.6	89	\\
15.7	84	\\
15.8	78	\\
15.9	75	\\
16	72	\\
16.1	70	\\
16.2	69	\\
16.3	69	\\
16.4	69	\\
16.5	69	\\
16.6	70	\\
16.7	70	\\
16.8	70	\\
16.9	69	\\
17	69	\\
17.1	69	\\
17.2	70	\\
17.3	70	\\
17.4	69	\\
17.5	69	\\
17.6	69	\\
17.7	69	\\
17.8	69	\\
17.9	69	\\
18	70	\\
18.1	70	\\
18.2	70	\\
18.3	70	\\
18.4	70	\\
18.5	70	\\
18.6	70	\\
18.7	70	\\
18.8	70	\\
18.9	70	\\
19	69	\\
19.1	69	\\
19.2	69	\\
19.3	70	\\
19.4	70	\\
19.5	70	\\
19.6	70	\\
19.7	70	\\
19.8	70	\\
19.9	70	\\
};
\addlegendentry{Output signal from KITT};

\end{axis}
\end{tikzpicture}%
    	\caption{The input and output signals of KITT and the fitted model.}
    	\label{fig:KITT-input-output-model-output}    	
\end{figure}
\begin{equation}
A=
\begin{bmatrix}
  0 & 1 \\
  -2,5245 & -2,0240
 \end{bmatrix}
\end{equation}
\begin{equation}
eig(A)=
\begin{bmatrix}
 -1.0120 + 1.2249i \\
  -1.0120 - 1.2249i
 \end{bmatrix}
\end{equation}
\end{document}