\documentclass[final]{scrreprt} %scrreprt of scrartcl
\input{../library/preamble.tex}
\input{../library/style.tex}
\addbibresource{../library/bibliography.bib}

\begin{document}

\section{Task 5}
As discussed in the previous sections, bit error correction is not likely to provide better results in our setup.
The BER is too high to be corrected properly, since Hamming and BCH are not made for BER's of over 0.05.
Therefore, our solution does not use any bit error correction.
\\ \\
From the findings in the previous sections, FSK performs better at the default 10 cm range when the default decision threshold of zero is set.
At the lowest bit rate of 5 bits per second, no errors occur.
This BER however increases when the bit rate increases, e.g. a BER of 0.05 at a bit rate of 200 bits/s.
For this reason, we choose the lowest possible bit rate of 5 bits/s.
\\ \\
ASK is proven to perform well with a coherent decision threshold, but since this parameter may not be changed.
The parameter that will be changed is the modulation index, which will be set to 1.00.
With this setting, the ASK will operate with OOK, which has maximal difference between the binary zero and one which are to be transmitted.
With the modulation index of 1.00 and a distance of 10 cm, a BER of 0.13 is found, which is worse than the FSK results on 10 cm.
\\ \\
When looking at larger distances, FSK's BER increases rapidly when ASK's BER increases at a much slower rate.
For this reason we decided to pick FSK for smaller distances (under 50 cm) and ASK for larger distances.
No exact measurement data was saved however at larger distances.

\end{document}