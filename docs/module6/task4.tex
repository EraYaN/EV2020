\documentclass[final]{scrreprt} %scrreprt of scrartcl
\input{../library/preamble.tex}
\input{../library/style.tex}
\addbibresource{../library/bibliography.bib}

\begin{document}

\section{Task 4}
Running the program with the default ASK settings plus Hamming-code and a decision threshold of 1.00 results in $BER_{bc} = 0.114$ and $BER_{ac} = 0.04$.
For BCH-coding this resulted in $BER_{bc} = 0.347$ and $BER_{ac} = 0.32$.
From this data, Hamming-code resulted in a 65\% better BER while BCH-coding only gave an 8\% improvement.
However, the $BER_{bc}$ increases more than the correction fixes, making both correction methods useless in our test setup.
\\ \\
Again using threshold 1.00, we measured the time duration of the audio signal for no coding, Hamming-coding and BCH-coding.
For this we use a stopwatch on a smartphone.

\begin{table}[H]
\centering
\begin{tabular}{c | c}
Coding & Time duration (s)\\
\hline
No coding & 2.62\\
Hamming & 4.18\\
BCH & 6.69\\
\end{tabular}
\caption{Time duration of generated audio signal for different codings.}
\end{table}

No coding results in the shortest audio signal since no error correction bits are included.
From this results it can be concluded that BCH includes more correction bits than Hamming.
\\ \\
Next up set $\mu = 0.66$ and $\sigma = 1.10$ to gain a BER around 0.3 with a deviation of around 0.02 due to varying environment factors.
Subsequently we run the program with Hamming-coding and get $BER_{bc} = 0.36$ and $BER_{ac} = 0.286$.
For BCH-coding we get $BER_{bc} =   0.367$ and $BER_{ac} = 0.36$.
Apparently Hamming- and BCH-coding cannot lower the BER anymore after a certain threshold.
In case the BER is already rather high, coding will actually result in a worse BER or not have a lot of effect.

\end{document}