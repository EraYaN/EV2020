\documentclass[final]{scrreprt} %scrreprt of scrartcl
\input{../../library/preamble.tex}
\input{../../library/style.tex}
\addbibresource{../../library/bibliography.bib}

\begin{document}

\chapter{Labday 2: Channel Estimation}
\label{ch:ass1}

\section{Channel estimation using a training sequence}

We have to look at four different test signals:

\begin{itemize}
\item 	A "minimium-phase" sequence, 
\begin{equation}
X_1(z) = 1 - \dfrac{1}{2}Z^{-1} \leftrightarrow x_1 = [1, -\dfrac{1}{2}, 0, 0,  \dots]^T
\end{equation}
\item A "maximum-phase" sequence,
\begin{equation}
X_2(z) = 1 - 2z^{-1} \leftrightarrow x_2 = [1, -2, 0, 0, \dots]^T
\end{equation}
\item A sinusoidal signal (N samples) followed by zeros,
\[ x_3[n] = \left\{ 
  \begin{array}{l l}
    \cos(\omega n), & \quad n = 0, \dots ,N - 1\\
    0, & \quad \textrm{elsewhere}
  \end{array} \right.\]
\item A random BPSK sequence (i.e., with entries randomly selected from {-1,1}: in Matlab),
\begin{equation}
x_4 = sign(randn(N,1))
\end{equation}
\end{itemize}

These four signals ..

Two Matlab functions, \texttt{ch1.m} and \texttt{ch2.m} respectively, are written. 
\texttt{Ch1.m} implements the time-domain channel estimation using inversion and \texttt{ch2.m} implements a matched filter.




\end{document} 