\documentclass[final]{scrreprt} %scrreprt of scrartcl
\input{../../library/preamble.tex}
\input{../../library/style.tex}
\addbibresource{../../library/bibliography.bib}

\begin{document}

\chapter{Labday 2: Channel Estimation}
\label{ch:labday2}

\section{Channel estimation using a training sequence}

The audio beacon will transmit a known sequence of pulses (\textit{training sequence}).
This pulse train will be deformes by the channeland we receive $y[n]=h[n]*x[n]$.\\


We have to look at four different test signals:

\begin{itemize}
\item 	A "minimium-phase" sequence, 
\begin{equation}
X_1(z) = 1 - \dfrac{1}{2}Z^{-1} \leftrightarrow x_1 = [1, -\dfrac{1}{2}, 0, 0,  \dots]^T
\end{equation}
\item A "maximum-phase" sequence,
\begin{equation}
X_2(z) = 1 - 2z^{-1} \leftrightarrow x_2 = [1, -2, 0, 0, \dots]^T
\end{equation}
\item A sinusoidal signal (N samples) followed by zeros,
\[ x_3[n] = \left\{ 
  \begin{array}{l l}
    \cos(\omega n), & \quad n = 0, \dots ,N - 1\\
    0, & \quad \textrm{elsewhere}
  \end{array} \right.\]
\item A random BPSK sequence (i.e., with entries randomly selected from {-1,1}: in Matlab),
\begin{equation}
x_4 = sign(randn(N,1))
\end{equation}
\end{itemize}

These four signals will be convolved with an audio channel impulse response $h[n]$ to obtain measurement data $y[n]$.
Our channel has a short random sequence $h = [3, 1, 2, -4]$ and length $L = 4$.
In order to generate the corresponding output signals we use: $y_{i}[n]=h[n]*x_{i}[n]$.
The outputs of the script are the sequences $y_{i}[n]$ along with the input sequences $x_{i}[n]$. 
The \texttt{datagen.m} is in Appendix \ref{lst:datagen.m}. 
\\
\\

Two Matlab functions, \texttt{ch1.m} and \texttt{ch2.m} respectively, are written. 
\texttt{ch1.m} implements the time-domain channel estimation using inversion and \texttt{ch2.m} implements a matched filter.
The scripts for these functions can be found in Appendix \ref{lst:ch1.m} and Appendix \ref{lst:ch2.m}.
\\
With the help of the input $x[n]$, the output $y[n]$ and the channel length $L$ these two functions approach the $\hat{h}[n]$ filter. 
For the implementation of these two functions we also need the Toeplitz-matrix. 
We can call upon the Toeplitz matrix by using \texttt{toep.m}. 
The script for the Toeplitz-matrix can be found in Appendix \ref{lst:toep.m}.

\subsection{Autocorrelation of the input signals}

With the help of a 'Matched Filter' we were able to check how the autocorrelation plots of all the four signals were correlated with themself.
The Matched Filter, $r[n]=x[n]*x[-n]$, is implemented in Matlab as folllows: \texttt{r = conv(x, flipud(x))}.
\\
The plots of the autocorrelation can be seen in Figure \ref{fig:auto1}, Figure \ref{fig:auto2}, Figure \ref{fig:auto3} and Figure \ref{fig:auto4}.
\\
As has been said above, the autocorrelation shows how a signal is correlated with itself. 
When a signal contains a certain values, then it will also have an autocorrelation that shows that the signal repeats itself.
In this way we can, for example, take the beat out of the signal with an algorithm.
When, however, the signal is completely uncorrelated we will not find any 'interesting' values.
This happens when, for example, the signal (after a certain value) remains zero in Figure \ref{fig:auto1}.


\begin{figure}[H]
	\centering
    	\setlength\figureheight{6cm}
    	\setlength\figurewidth{10cm}
    	% This file was created by matlab2tikz v0.4.2.
% Copyright (c) 2008--2013, Nico Schlömer <nico.schloemer@gmail.com>
% All rights reserved.
% 
% The latest updates can be retrieved from
%   http://www.mathworks.com/matlabcentral/fileexchange/22022-matlab2tikz
% where you can also make suggestions and rate matlab2tikz.
% 
% 
% 
\begin{tikzpicture}

\begin{axis}[%
width=\figurewidth,
height=\figureheight,
scale only axis,
xmin=0,
xmax=10,
ymin=-1,
ymax=1
]
\addplot [
color=red,
solid,
forget plot
]
table[row sep=crcr]{
1 1\\
2 -1\\
3 0.25\\
4 0\\
5 0\\
6 0\\
7 0\\
8 0\\
9 0\\
10 0\\
11 0\\
12 0\\
13 0\\
14 0\\
15 0\\
16 0\\
17 0\\
18 0\\
19 0\\
};
\end{axis}
\end{tikzpicture}%    	
    	\caption{Autocorrelation of \texttt{x1}}
    	\label{fig:auto1}
\end{figure}
\begin{figure}[H]
	\centering
    	\setlength\figureheight{6cm}
    	\setlength\figurewidth{10cm}
    	% This file was created by matlab2tikz v0.4.2.
% Copyright (c) 2008--2013, Nico Schlömer <nico.schloemer@gmail.com>
% All rights reserved.
% 
% The latest updates can be retrieved from
%   http://www.mathworks.com/matlabcentral/fileexchange/22022-matlab2tikz
% where you can also make suggestions and rate matlab2tikz.
% 
% 
% 
\begin{tikzpicture}

\begin{axis}[%
width=\figurewidth,
height=\figureheight,
scale only axis,
xmin=0,
xmax=10,
ymin=-4,
ymax=4
]
\addplot [
color=blue,
solid,
forget plot
]
table[row sep=crcr]{
1 1\\
2 -4\\
3 4\\
4 0\\
5 0\\
6 0\\
7 0\\
8 0\\
9 0\\
10 0\\
11 0\\
12 0\\
13 0\\
14 0\\
15 0\\
16 0\\
17 0\\
18 0\\
19 0\\
};
\end{axis}
\end{tikzpicture}%    	
    	\caption{Autocorrelation of \texttt{x2}}
    	\label{fig:auto2}
\end{figure}
\begin{figure}[H]
	\centering
    	\setlength\figureheight{6cm}
    	\setlength\figurewidth{10cm}
    	% This file was created by matlab2tikz v0.4.2.
% Copyright (c) 2008--2013, Nico Schlömer <nico.schloemer@gmail.com>
% All rights reserved.
% 
% The latest updates can be retrieved from
%   http://www.mathworks.com/matlabcentral/fileexchange/22022-matlab2tikz
% where you can also make suggestions and rate matlab2tikz.
% 
% 
% 
\begin{tikzpicture}

\begin{axis}[%
width=\figurewidth,
height=\figureheight,
scale only axis,
xmin=0,
xmax=200,
ymin=-50,
ymax=50
]
\addplot [
color=black,
solid,
forget plot
]
table[row sep=crcr]{
1 0.960530497001443\\
2 1.80540219275092\\
3 2.46612105802363\\
4 2.8860068044872\\
5 3.02366208954173\\
6 2.85560592731978\\
7 2.37792712977221\\
8 1.60686326274695\\
9 0.578265171063294\\
10 -0.654036039450516\\
11 -2.02088633649856\\
12 -3.44136409064282\\
13 -4.82705983090923\\
14 -6.0868678000012\\
15 -7.13204777569226\\
16 -7.88129836103421\\
17 -8.26557827186661\\
18 -8.23242042020773\\
19 -7.74950454756482\\
20 -6.8072869382994\\
21 -5.42052890499753\\
22 -3.6286173425187\\
23 -1.4946283306489\\
24 0.896854152390563\\
25 3.44309073449162\\
26 6.02757749954932\\
27 8.52528837442566\\
28 10.8085362254162\\
29 12.7531840494778\\
30 14.2449139605273\\
31 15.1852514204499\\
32 15.4970460735387\\
33 15.1291286892282\\
34 14.0598955179969\\
35 12.2996155931204\\
36 9.8913113448431\\
37 6.91012599110989\\
38 3.4611597667688\\
39 -0.324171913347549\\
40 -4.2931350230356\\
41 -8.27783869728025\\
42 -12.102313658063\\
43 -15.5902097443305\\
44 -18.5727784731256\\
45 -20.8967844376602\\
46 -22.4319829403572\\
47 -23.0778111886327\\
48 -22.7689665460967\\
49 -21.4795868956379\\
50 -19.2258035883674\\
51 -16.0665045267356\\
52 -12.1022208997821\\
53 -7.47213274083488\\
54 -2.34927228480901\\
55 3.06591362879184\\
56 8.5534038252426\\
57 13.8825391859822\\
58 18.8216923245839\\
59 23.1482661499714\\
60 26.6585922350795\\
61 29.1772935456007\\
62 30.5656887074462\\
63 30.7288464478435\\
64 29.6209481488223\\
65 27.2486818173863\\
66 23.6724696811307\\
67 19.0054208988926\\
68 13.4099968576873\\
69 7.09247517233551\\
70 0.295395583025821\\
71 -6.71173778321524\\
72 -13.6431419283969\\
73 -20.208404621688\\
74 -26.124623504744\\
75 -31.1285283397109\\
76 -34.9880694835393\\
77 -37.5129670293833\\
78 -38.5637483047719\\
79 -38.058855557764\\
80 -35.9794787820763\\
81 -32.3718579672675\\
82 -27.346901103051\\
83 -21.0770749034273\\
84 -13.7906398913626\\
85 -5.76341538937022\\
86 2.69163176567737\\
87 11.2365830309881\\
88 19.5220734913716\\
89 27.2015669637819\\
90 33.9458179346064\\
91 39.4569179986714\\
92 43.481330116958\\
93 45.8213449088545\\
94 46.3444484894742\\
95 44.9901691522377\\
96 41.7740675984296\\
97 36.7886486696399\\
98 30.2010971556602\\
99 22.2478711862416\\
100 12.4264233982307\\
101 2.30343893241976\\
102 -7.71274818829983\\
103 -17.2259802931516\\
104 -25.8679437958399\\
105 -33.3124161945913\\
106 -39.2875417588306\\
107 -43.5856756034281\\
108 -46.0704433242277\\
109 -46.680783327641\\
110 -45.431866293205\\
111 -42.4129155061348\\
112 -37.7820777410492\\
113 -31.7586118518339\\
114 -24.6127665471982\\
115 -16.6538059518045\\
116 -8.21670819524078\\
117 0.351893934262051\\
118 8.70794167043402\\
119 16.5237647048212\\
120 23.5009267370196\\
121 29.3815987039963\\
122 33.9580271832643\\
123 37.0797528037611\\
124 38.6583333099809\\
125 38.6694338269356\\
126 37.152258228458\\
127 34.2064056249564\\
128 29.9863403051705\\
129 24.6937577645033\\
130 18.5682100044732\\
131 11.8764170149218\\
132 4.90073593829726\\
133 -2.07271660341556\\
134 -8.76579073706677\\
135 -14.9193978903988\\
136 -20.3035311399789\\
137 -24.7258241820856\\
138 -28.0383320532459\\
139 -30.1422986888873\\
140 -30.9907663236552\\
141 -30.5889756357419\\
142 -28.9925993632876\\
143 -26.303941903779\\
144 -22.6663194320894\\
145 -18.2569060320191\\
146 -13.2783884639069\\
147 -7.94981337013574\\
148 -2.49703456130013\\
149 2.85682608898103\\
150 7.90050251676767\\
151 12.4429202501507\\
152 16.3204152375243\\
153 19.4025638251038\\
154 21.596419160037\\
155 22.8490235258887\\
156 23.1481440192249\\
157 22.5212570782421\\
158 21.0328822732744\\
159 18.7804342757974\\
160 15.8888211985766\\
161 12.5040651737292\\
162 8.78625529892408\\
163 4.90216275680719\\
164 1.01785249136873\\
165 -2.70838452077976\\
166 -6.13247880599182\\
167 -9.13014723242292\\
168 -11.6013604260095\\
169 -13.4735540913293\\
170 -14.7034871700997\\
171 -15.2777140106537\\
172 -15.2117016264164\\
173 -14.5476837976716\\
174 -13.3513985342662\\
175 -11.707901908481\\
176 -9.71668755490443\\
177 -7.48636579241951\\
178 -5.12916848980545\\
179 -2.75554520082084\\
180 -0.469103061415463\\
181 1.63788163495448\\
182 3.48818697483563\\
183 5.02245384446739\\
184 6.20097803981933\\
185 7.00443751874631\\
186 7.43355991324791\\
187 7.50778752208875\\
188 7.26304507236958\\
189 6.74875733373826\\
190 6.02429728434697\\
191 5.15506947537244\\
192 4.20844651654334\\
193 3.24977872995001\\
194 2.33868805747079\\
195 1.52583785671358\\
196 0.850341380513053\\
197 0.337935048891218\\
198 0\\
199 0\\
};
\end{axis}
\end{tikzpicture}%    	
    	\caption{Autocorrelation of \texttt{x3}}
    	\label{fig:auto3}
\end{figure}
\begin{figure}[H]
	\centering
    	\setlength\figureheight{6cm}
    	\setlength\figurewidth{10cm}
    	% This file was created by matlab2tikz v0.4.2.
% Copyright (c) 2008--2013, Nico Schlömer <nico.schloemer@gmail.com>
% All rights reserved.
% 
% The latest updates can be retrieved from
%   http://www.mathworks.com/matlabcentral/fileexchange/22022-matlab2tikz
% where you can also make suggestions and rate matlab2tikz.
% 
% 
% 
%
% defining custom colors
\definecolor{mycolor1}{rgb}{1,0,1}%
%
\begin{tikzpicture}

\begin{axis}[%
width=\figurewidth,
height=\figureheight,
scale only axis,
xmin=0,
xmax=200,
ymin=-40,
ymax=30
]
\addplot [
color=mycolor1,
solid,
forget plot
]
table[row sep=crcr]{
1 1\\
2 -2\\
3 3\\
4 -4\\
5 5\\
6 -6\\
7 7\\
8 -4\\
9 5\\
10 -6\\
11 3\\
12 -4\\
13 5\\
14 -6\\
15 11\\
16 -4\\
17 5\\
18 -14\\
19 3\\
20 -4\\
21 9\\
22 -2\\
23 11\\
24 -4\\
25 1\\
26 -18\\
27 3\\
28 -4\\
29 21\\
30 -6\\
31 7\\
32 -20\\
33 1\\
34 -6\\
35 7\\
36 4\\
37 5\\
38 2\\
39 -9\\
40 -16\\
41 9\\
42 6\\
43 19\\
44 -8\\
45 -3\\
46 -14\\
47 -5\\
48 -4\\
49 5\\
50 10\\
51 -1\\
52 -16\\
53 -3\\
54 -6\\
55 15\\
56 0\\
57 13\\
58 -2\\
59 -9\\
60 -8\\
61 -7\\
62 10\\
63 -5\\
64 0\\
65 -3\\
66 2\\
67 -1\\
68 -20\\
69 -7\\
70 10\\
71 7\\
72 28\\
73 5\\
74 2\\
75 -21\\
76 12\\
77 -7\\
78 18\\
79 -5\\
80 8\\
81 -11\\
82 6\\
83 -25\\
84 16\\
85 -3\\
86 26\\
87 -13\\
88 16\\
89 -11\\
90 22\\
91 -1\\
92 24\\
93 -7\\
94 -2\\
95 -5\\
96 16\\
97 -11\\
98 10\\
99 -17\\
100 24\\
101 7\\
102 26\\
103 5\\
104 28\\
105 -17\\
106 -10\\
107 -15\\
108 20\\
109 3\\
110 14\\
111 -3\\
112 4\\
113 -1\\
114 14\\
115 21\\
116 12\\
117 -9\\
118 -2\\
119 -3\\
120 12\\
121 -17\\
122 10\\
123 13\\
124 4\\
125 -13\\
126 -6\\
127 -15\\
128 4\\
129 11\\
130 -2\\
131 1\\
132 4\\
133 -5\\
134 6\\
135 17\\
136 -12\\
137 -5\\
138 6\\
139 13\\
140 -8\\
141 -21\\
142 -14\\
143 9\\
144 -8\\
145 7\\
146 -10\\
147 -7\\
148 -12\\
149 7\\
150 -2\\
151 13\\
152 0\\
153 -1\\
154 -2\\
155 -3\\
156 -32\\
157 -1\\
158 -6\\
159 13\\
160 0\\
161 7\\
162 -2\\
163 9\\
164 -8\\
165 -5\\
166 -10\\
167 -3\\
168 -12\\
169 7\\
170 6\\
171 1\\
172 -4\\
173 11\\
174 -2\\
175 5\\
176 -4\\
177 -1\\
178 -10\\
179 5\\
180 -8\\
181 7\\
182 10\\
183 1\\
184 0\\
185 3\\
186 -10\\
187 -7\\
188 0\\
189 -1\\
190 2\\
191 5\\
192 0\\
193 3\\
194 2\\
195 1\\
196 0\\
197 3\\
198 -2\\
199 1\\
};
\end{axis}
\end{tikzpicture}%    	
    	\caption{Autocorrelation of \texttt{x4}}
    	\label{fig:auto4}
\end{figure}

\subsection{Singular values of the input signals}

For each of the four given signals we needed to make plots of the singular values for various values of the channel length $L$. 
The plots can be found in Figures \ref{fig:svd1}, \ref{fig:svd2}, \ref{fig:svd3} and \ref{fig:svd4}. 
Since we invert \texttt{X}, we want the singular values not to be to close to 0, because the noise will get magnified otherwise and the results will get unstable. 
\\
When looking closely at the figures we see that we get more singular values that point away from 0 when the channel length increases. 
The only exception is \texttt{x3} (see Figure \ref{fig:svd3}).
Here it is the other way around, the singular value approach 0 when the channel length increases. 
The Matlab code used for the calculations of the singular values can be found in Appendix \ref{lst:svd.m}.


\begin{figure}[H]
	\centering
    	\setlength\figureheight{6cm}
    	\setlength\figurewidth{10cm}
    	% This file was created by matlab2tikz v0.4.2.
% Copyright (c) 2008--2013, Nico Schlömer <nico.schloemer@gmail.com>
% All rights reserved.
% 
% The latest updates can be retrieved from
%   http://www.mathworks.com/matlabcentral/fileexchange/22022-matlab2tikz
% where you can also make suggestions and rate matlab2tikz.
% 
% 
% 
%
% defining custom colors
\definecolor{mycolor1}{rgb}{0,0.6,0.6}%
\definecolor{mycolor2}{rgb}{0,0.7,0.7}%
\definecolor{mycolor3}{rgb}{0,0.8,0.8}%
\definecolor{mycolor4}{rgb}{0,0.9,0.9}%
\definecolor{mycolor5}{rgb}{0,1,1}%
%
\begin{tikzpicture}

\begin{axis}[%
width=\figurewidth,
height=\figureheight,
scale only axis,
xmin=0,
xmax=50,
ymin=0.5,
ymax=1.5,
title={SVD van x1},
legend style={draw=black,fill=white,legend cell align=left}
]
\addplot [
color=teal!20!black,
solid,
mark=+,
mark options={solid}
]
table[row sep=crcr]{
1 1.3228756555323\\
2 0.866025403784438\\
};
\addlegendentry{L 2};

\addplot [
color=teal!40!black,
solid,
mark=+,
mark options={solid}
]
table[row sep=crcr]{
1 1.39896632596591\\
2 1.11803398874989\\
3 0.73681287910395\\
};
\addlegendentry{L 3};

\addplot [
color=teal!60!black,
solid,
mark=+,
mark options={solid}
]
table[row sep=crcr]{
1 1.4546564555882\\
2 1.3228756555323\\
3 1.11803398874989\\
4 0.866025403784438\\
5 0.619656837463738\\
};
\addlegendentry{L 5};

\addplot [
color=teal!80!black,
solid,
mark=+,
mark options={solid}
]
table[row sep=crcr]{
1 1.47976100123834\\
2 1.41987479839561\\
3 1.3228756555323\\
4 1.19316728821525\\
5 1.03747376946748\\
6 0.866025403784438\\
7 0.695669143257786\\
8 0.557052402574562\\
};
\addlegendentry{L 8};

\addplot [
color=teal,
solid,
mark=+,
mark options={solid}
]
table[row sep=crcr]{
1 1.49028246229567\\
2 1.46131996005434\\
3 1.41368693428605\\
4 1.34835631297189\\
5 1.26672999768796\\
6 1.17069922706702\\
7 1.06276211813589\\
8 0.946253197065914\\
9 0.825793711085792\\
10 0.708159058283447\\
11 0.6037747712076\\
12 0.528259578781065\\
};
\addlegendentry{L 12};

\addplot [
color=mycolor1,
solid,
mark=+,
mark options={solid}
]
table[row sep=crcr]{
1 1.49492734037886\\
2 1.47976100123834\\
3 1.4546564555882\\
4 1.41987479839561\\
5 1.37578617876709\\
6 1.3228756555323\\
7 1.26175280595118\\
8 1.19316728821525\\
9 1.11803398874989\\
10 1.03747376946748\\
11 0.952879770314351\\
12 0.866025403784439\\
13 0.779238339863652\\
14 0.695669143257786\\
15 0.619656837463738\\
16 0.557052402574562\\
17 0.514968199977233\\
};
\addlegendentry{L 17};

\addplot [
color=mycolor2,
solid,
mark=+,
mark options={solid}
]
table[row sep=crcr]{
1 1.49714557120335\\
2 1.48859861154344\\
3 1.47440819738337\\
4 1.4546564555882\\
5 1.42945910759673\\
6 1.39896632596591\\
7 1.36336401192371\\
8 1.3228756555323\\
9 1.27776501453322\\
10 1.22833995502162\\
11 1.17495795338389\\
12 1.11803398874989\\
13 1.05805189276327\\
14 0.995580712397282\\
15 0.931298323650864\\
16 0.866025403784439\\
17 0.800773732705612\\
18 0.73681287910395\\
19 0.675756361204809\\
20 0.619656837463738\\
21 0.571069582002678\\
22 0.532986091479817\\
23 0.508483174378651\\
};
\addlegendentry{L 23};

\addplot [
color=mycolor3,
solid,
mark=+,
mark options={solid}
]
table[row sep=crcr]{
1 1.49828879839365\\
2 1.49316105670236\\
3 1.48463438475607\\
4 1.47273820199662\\
5 1.45751384766821\\
6 1.43901474669556\\
7 1.41730664384698\\
8 1.39246792389473\\
9 1.36459004193482\\
10 1.33377809635897\\
11 1.3001515879149\\
12 1.26384542284443\\
13 1.22501123760508\\
14 1.18381914898543\\
15 1.14046006893653\\
16 1.09514877124585\\
17 1.04812796093579\\
18 0.999673680628474\\
19 0.950102492973879\\
20 0.89978100026749\\
21 0.849138380756068\\
22 0.798682676319161\\
23 0.749021415531167\\
24 0.700886493882718\\
25 0.655161475357582\\
26 0.612905689201379\\
27 0.575362658138125\\
28 0.543930826116659\\
29 0.520067359817462\\
30 0.50510461946819\\
};
\addlegendentry{L 30};

\addplot [
color=mycolor4,
solid,
mark=+,
mark options={solid}
]
table[row sep=crcr]{
1 1.4989187129842\\
2 1.49567719198961\\
3 1.49028246229567\\
4 1.48274625001959\\
5 1.47308500897227\\
6 1.46131996005434\\
7 1.44747714508513\\
8 1.43158749764152\\
9 1.41368693428605\\
10 1.39381647052602\\
11 1.37202236701716\\
12 1.34835631297189\\
13 1.3228756555323\\
14 1.29564368612789\\
15 1.26672999768796\\
16 1.23621093018807\\
17 1.20417012659177\\
18 1.17069922706702\\
19 1.13589873673203\\
20 1.09987911148934\\
21 1.06276211813589\\
22 1.0246825392403\\
23 0.985790310402546\\
24 0.946253197065914\\
25 0.906260138479535\\
26 0.866025403784439\\
27 0.825793711085792\\
28 0.785846438182819\\
29 0.746508972813054\\
30 0.708159058283447\\
31 0.67123560438679\\
32 0.636246740232282\\
33 0.6037747712076\\
34 0.574474156373614\\
35 0.549056971591159\\
36 0.528259578781065\\
37 0.512786249193645\\
38 0.503232244461531\\
};
\addlegendentry{L 38};

\addplot [
color=mycolor5,
solid,
mark=+,
mark options={solid}
]
table[row sep=crcr]{
1 1.49928613787983\\
2 1.49714557120335\\
3 1.49358136049002\\
4 1.48859861154344\\
5 1.48220448302355\\
6 1.47440819738337\\
7 1.4652210555178\\
8 1.4546564555882\\
9 1.44272991661729\\
10 1.42945910759673\\
11 1.41486388302161\\
12 1.39896632596591\\
13 1.38179080004901\\
14 1.36336401192371\\
15 1.3437150862514\\
16 1.3228756555323\\
17 1.30087996764459\\
18 1.27776501453322\\
19 1.25357068620129\\
20 1.22833995502162\\
21 1.20211909643601\\
22 1.17495795338389\\
23 1.14691025334598\\
24 1.11803398874989\\
25 1.08839187371546\\
26 1.05805189276327\\
27 1.02708796019809\\
28 0.995580712397282\\
29 0.963618459088886\\
30 0.931298323650864\\
31 0.898727605996944\\
32 0.866025403784439\\
33 0.833324526808372\\
34 0.800773732705612\\
35 0.768540294909728\\
36 0.73681287910395\\
37 0.70580464189535\\
38 0.675756361204809\\
39 0.646939245754541\\
40 0.619656837463738\\
41 0.594245116485876\\
42 0.571069582002678\\
43 0.550517820333633\\
44 0.532986091479817\\
45 0.518859055617968\\
46 0.508483174378651\\
47 0.502136512077539\\
};
\addlegendentry{L 47};

\end{axis}
\end{tikzpicture}%    	
    	\caption{Singular values of \texttt{x1}}
    	\label{fig:svd1}
\end{figure}

\begin{figure}[H]
	\centering
    	\setlength\figureheight{6cm}
    	\setlength\figurewidth{10cm}
    	% This file was created by matlab2tikz v0.4.2.
% Copyright (c) 2008--2013, Nico Schlömer <nico.schloemer@gmail.com>
% All rights reserved.
% 
% The latest updates can be retrieved from
%   http://www.mathworks.com/matlabcentral/fileexchange/22022-matlab2tikz
% where you can also make suggestions and rate matlab2tikz.
% 
% 
% 
%
% defining custom colors
\definecolor{mycolor1}{rgb}{0,0.6,0.6}%
\definecolor{mycolor2}{rgb}{0,0.7,0.7}%
\definecolor{mycolor3}{rgb}{0,0.8,0.8}%
\definecolor{mycolor4}{rgb}{0,0.9,0.9}%
\definecolor{mycolor5}{rgb}{0,1,1}%
%
\begin{tikzpicture}

\begin{axis}[%
width=\figurewidth,
height=\figureheight,
scale only axis,
xmin=0,
xmax=50,
ymin=1,
ymax=3,
title={SVD van x2},
legend style={draw=black,fill=white,legend cell align=left}
]
\addplot [
color=teal!20!black,
solid,
mark=+,
mark options={solid}
]
table[row sep=crcr]{
1 2.64575131106459\\
2 1.73205080756888\\
};
\addlegendentry{L 2};

\addplot [
color=teal!40!black,
solid,
mark=+,
mark options={solid}
]
table[row sep=crcr]{
1 2.79793265193181\\
2 2.23606797749979\\
3 1.4736257582079\\
};
\addlegendentry{L 3};

\addplot [
color=teal!60!black,
solid,
mark=+,
mark options={solid}
]
table[row sep=crcr]{
1 2.90931291117641\\
2 2.64575131106459\\
3 2.23606797749979\\
4 1.73205080756888\\
5 1.23931367492748\\
};
\addlegendentry{L 5};

\addplot [
color=teal!80!black,
solid,
mark=+,
mark options={solid}
]
table[row sep=crcr]{
1 2.95952200247669\\
2 2.83974959679122\\
3 2.64575131106459\\
4 2.3863345764305\\
5 2.07494753893497\\
6 1.73205080756888\\
7 1.39133828651557\\
8 1.11410480514912\\
};
\addlegendentry{L 8};

\addplot [
color=teal,
solid,
mark=+,
mark options={solid}
]
table[row sep=crcr]{
1 2.98056492459135\\
2 2.92263992010867\\
3 2.82737386857211\\
4 2.69671262594379\\
5 2.53345999537592\\
6 2.34139845413404\\
7 2.12552423627177\\
8 1.89250639413183\\
9 1.65158742217158\\
10 1.41631811656689\\
11 1.2075495424152\\
12 1.05651915756213\\
};
\addlegendentry{L 12};

\addplot [
color=mycolor1,
solid,
mark=+,
mark options={solid}
]
table[row sep=crcr]{
1 2.98985468075772\\
2 2.95952200247669\\
3 2.90931291117641\\
4 2.83974959679122\\
5 2.75157235753417\\
6 2.64575131106459\\
7 2.52350561190235\\
8 2.3863345764305\\
9 2.23606797749979\\
10 2.07494753893497\\
11 1.9057595406287\\
12 1.73205080756888\\
13 1.5584766797273\\
14 1.39133828651557\\
15 1.23931367492748\\
16 1.11410480514912\\
17 1.02993639995447\\
};
\addlegendentry{L 17};

\addplot [
color=mycolor2,
solid,
mark=+,
mark options={solid}
]
table[row sep=crcr]{
1 2.9942911424067\\
2 2.97719722308689\\
3 2.94881639476675\\
4 2.90931291117641\\
5 2.85891821519346\\
6 2.79793265193181\\
7 2.72672802384743\\
8 2.64575131106459\\
9 2.55553002906645\\
10 2.45667991004324\\
11 2.34991590676777\\
12 2.23606797749979\\
13 2.11610378552655\\
14 1.99116142479456\\
15 1.86259664730173\\
16 1.73205080756888\\
17 1.60154746541122\\
18 1.4736257582079\\
19 1.35151272240962\\
20 1.23931367492748\\
21 1.14213916400536\\
22 1.06597218295963\\
23 1.0169663487573\\
};
\addlegendentry{L 23};

\addplot [
color=mycolor3,
solid,
mark=+,
mark options={solid}
]
table[row sep=crcr]{
1 2.99657759678731\\
2 2.98632211340471\\
3 2.96926876951215\\
4 2.94547640399324\\
5 2.91502769533641\\
6 2.87802949339111\\
7 2.83461328769396\\
8 2.78493584778945\\
9 2.72918008386963\\
10 2.66755619271794\\
11 2.6003031758298\\
12 2.52769084568886\\
13 2.45002247521015\\
14 2.36763829797085\\
15 2.28092013787306\\
16 2.19029754249169\\
17 2.09625592187159\\
18 1.99934736125695\\
19 1.90020498594776\\
20 1.79956200053498\\
21 1.69827676151214\\
22 1.59736535263832\\
23 1.49804283106233\\
24 1.40177298776544\\
25 1.31032295071516\\
26 1.22581137840276\\
27 1.15072531627625\\
28 1.08786165223332\\
29 1.04013471963492\\
30 1.01020923893638\\
};
\addlegendentry{L 30};

\addplot [
color=mycolor4,
solid,
mark=+,
mark options={solid}
]
table[row sep=crcr]{
1 2.9978374259684\\
2 2.99135438397921\\
3 2.98056492459135\\
4 2.96549250003917\\
5 2.94617001794453\\
6 2.92263992010867\\
7 2.89495429017026\\
8 2.86317499528304\\
9 2.82737386857211\\
10 2.78763294105203\\
11 2.74404473403433\\
12 2.69671262594379\\
13 2.64575131106459\\
14 2.59128737225577\\
15 2.53345999537592\\
16 2.47242186037614\\
17 2.40834025318354\\
18 2.34139845413404\\
19 2.27179747346405\\
20 2.19975822297868\\
21 2.12552423627177\\
22 2.04936507848061\\
23 1.97158062080509\\
24 1.89250639413183\\
25 1.81252027695907\\
26 1.73205080756888\\
27 1.65158742217158\\
28 1.57169287636564\\
29 1.49301794562611\\
30 1.41631811656689\\
31 1.34247120877358\\
32 1.27249348046456\\
33 1.2075495424152\\
34 1.14894831274723\\
35 1.09811394318232\\
36 1.05651915756213\\
37 1.02557249838729\\
38 1.00646448892306\\
};
\addlegendentry{L 38};

\addplot [
color=mycolor5,
solid,
mark=+,
mark options={solid}
]
table[row sep=crcr]{
1 2.99857227575965\\
2 2.9942911424067\\
3 2.98716272098005\\
4 2.97719722308689\\
5 2.9644089660471\\
6 2.94881639476675\\
7 2.9304421110356\\
8 2.90931291117641\\
9 2.88545983323459\\
10 2.85891821519346\\
11 2.82972776604321\\
12 2.79793265193181\\
13 2.76358160009801\\
14 2.72672802384743\\
15 2.6874301725028\\
16 2.64575131106459\\
17 2.60175993528919\\
18 2.55553002906645\\
19 2.50714137240257\\
20 2.45667991004324\\
21 2.40423819287202\\
22 2.34991590676777\\
23 2.29382050669196\\
24 2.23606797749979\\
25 2.17678374743093\\
26 2.11610378552655\\
27 2.05417592039618\\
28 1.99116142479456\\
29 1.92723691817777\\
30 1.86259664730173\\
31 1.79745521199389\\
32 1.73205080756888\\
33 1.66664905361675\\
34 1.60154746541122\\
35 1.53708058981946\\
36 1.4736257582079\\
37 1.4116092837907\\
38 1.35151272240962\\
39 1.29387849150908\\
40 1.23931367492748\\
41 1.18849023297175\\
42 1.14213916400536\\
43 1.10103564066727\\
44 1.06597218295963\\
45 1.03771811123594\\
46 1.0169663487573\\
47 1.00427302415508\\
};
\addlegendentry{L 47};

\end{axis}
\end{tikzpicture}%    	
    	\caption{Singular values of \texttt{x2}}
    	\label{fig:svd2}
\end{figure}

\begin{figure}[H]
	\centering
    	\setlength\figureheight{6cm}
    	\setlength\figurewidth{10cm}
    	% This file was created by matlab2tikz v0.4.2.
% Copyright (c) 2008--2013, Nico Schlömer <nico.schloemer@gmail.com>
% All rights reserved.
% 
% The latest updates can be retrieved from
%   http://www.mathworks.com/matlabcentral/fileexchange/22022-matlab2tikz
% where you can also make suggestions and rate matlab2tikz.
% 
% 
% 
%
% defining custom colors
\definecolor{mycolor1}{rgb}{0,0.6,0.6}%
\definecolor{mycolor2}{rgb}{0,0.7,0.7}%
\definecolor{mycolor3}{rgb}{0,0.8,0.8}%
\definecolor{mycolor4}{rgb}{0,0.9,0.9}%
\definecolor{mycolor5}{rgb}{0,1,1}%
%
\begin{tikzpicture}

\begin{axis}[%
width=\figurewidth,
height=\figureheight,
scale only axis,
xmin=0,
xmax=50,
ymin=0,
ymax=35,
title={SVD van x3},
legend style={draw=black,fill=white,legend cell align=left}
]
\addplot [
color=teal!20!black,
solid,
mark=+,
mark options={solid}
]
table[row sep=crcr]{
1 9.95285583676809\\
2 1.26964917736308\\
};
\addlegendentry{L 2};

\addplot [
color=teal!40!black,
solid,
mark=+,
mark options={solid}
]
table[row sep=crcr]{
1 12.060622083884\\
2 2.26428958765277\\
3 0.649161092933712\\
};
\addlegendentry{L 3};

\addplot [
color=teal!60!black,
solid,
mark=+,
mark options={solid}
]
table[row sep=crcr]{
1 15.1045637525194\\
2 4.65942750968351\\
3 0.984175135237272\\
4 0.710576136833144\\
5 0.588847041000678\\
};
\addlegendentry{L 5};

\addplot [
color=teal!80!black,
solid,
mark=+,
mark options={solid}
]
table[row sep=crcr]{
1 17.8983737713697\\
2 8.68277410767051\\
3 2.01379763463592\\
4 1.13874732848848\\
5 0.719374449483667\\
6 0.68986889455142\\
7 0.619988097314667\\
8 0.461641827962347\\
};
\addlegendentry{L 8};

\addplot [
color=teal,
solid,
mark=+,
mark options={solid}
]
table[row sep=crcr]{
1 19.5286291523445\\
2 14.2128524427863\\
3 3.3677346964395\\
4 2.14828852017767\\
5 1.30569688635871\\
6 0.845150194549077\\
7 0.772323414895649\\
8 0.762212596030754\\
9 0.597980952663657\\
10 0.568300086509949\\
11 0.523496407268271\\
12 0.384199113773418\\
};
\addlegendentry{L 12};

\addplot [
color=mycolor1,
solid,
mark=+,
mark options={solid}
]
table[row sep=crcr]{
1 20.3036091875123\\
2 19.8142549725524\\
3 5.14862903293536\\
4 3.3504402066948\\
5 2.23123032177431\\
6 1.66040084574758\\
7 1.15286412711635\\
8 0.832037944385342\\
9 0.796036588592387\\
10 0.774154503988542\\
11 0.69582786127613\\
12 0.68418649944391\\
13 0.617144870911628\\
14 0.509400492746358\\
15 0.478314944813601\\
16 0.456974990150378\\
17 0.333821336223177\\
};
\addlegendentry{L 17};

\addplot [
color=mycolor2,
solid,
mark=+,
mark options={solid}
]
table[row sep=crcr]{
1 25.1256433962137\\
2 20.5988320962424\\
3 6.81851344481744\\
4 5.05243444299119\\
5 3.21284379181429\\
6 2.52948417119775\\
7 1.95234749529575\\
8 1.55350850108753\\
9 1.18019974879595\\
10 0.900778244746346\\
11 0.80700627440335\\
12 0.802983842107775\\
13 0.74519010205154\\
14 0.744713600002843\\
15 0.706658358902859\\
16 0.653442771701689\\
17 0.59778431784204\\
18 0.571230911599932\\
19 0.538610175784234\\
20 0.43278041685894\\
21 0.430258002764683\\
22 0.414342989996429\\
23 0.302182061063531\\
};
\addlegendentry{L 23};

\addplot [
color=mycolor3,
solid,
mark=+,
mark options={solid}
]
table[row sep=crcr]{
1 27.0722709530202\\
2 24.6799274940252\\
3 7.53815414363198\\
4 7.15380888035294\\
5 4.37634514817807\\
6 3.47230735082011\\
7 2.7614172500829\\
8 2.29387830166935\\
9 1.90622685466131\\
10 1.58188454140558\\
11 1.29290012372539\\
12 1.03975897806336\\
13 0.833599368577181\\
14 0.812086452885808\\
15 0.811033374133514\\
16 0.775469493344693\\
17 0.76769042839351\\
18 0.732392024185621\\
19 0.718193114291542\\
20 0.642997593440278\\
21 0.642486337468609\\
22 0.637619622189828\\
23 0.554076582409295\\
24 0.525045954838372\\
25 0.506593248725806\\
26 0.457744115998321\\
27 0.408796185492018\\
28 0.402082873122768\\
29 0.362631995501486\\
30 0.283661369776406\\
};
\addlegendentry{L 30};

\addplot [
color=mycolor4,
solid,
mark=+,
mark options={solid}
]
table[row sep=crcr]{
1 29.9433959255552\\
2 27.3270838835163\\
3 10.0245742221241\\
4 8.14800615608267\\
5 5.09116779869429\\
6 4.64061225774925\\
7 3.6937620496035\\
8 3.04882048833355\\
9 2.61732798867029\\
10 2.24364213123147\\
11 1.95248661099829\\
12 1.67588841637062\\
13 1.44189045519489\\
14 1.21510524457017\\
15 1.02087970457435\\
16 0.846195861904077\\
17 0.815189350986369\\
18 0.814766718943388\\
19 0.796124598887018\\
20 0.789995382756228\\
21 0.755828508614871\\
22 0.75322705254398\\
23 0.728001199543725\\
24 0.698079678416728\\
25 0.671484352477995\\
26 0.65528423476511\\
27 0.63173338034682\\
28 0.570028728996948\\
29 0.565218672736307\\
30 0.551467693504156\\
31 0.482113689361132\\
32 0.472562010877007\\
33 0.460137824180596\\
34 0.402424607168119\\
35 0.390810018202283\\
36 0.38654911998188\\
37 0.328422834619845\\
38 0.272280250283023\\
};
\addlegendentry{L 38};

\addplot [
color=mycolor5,
solid,
mark=+,
mark options={solid}
]
table[row sep=crcr]{
1 32.297884841761\\
2 30.4497712723113\\
3 11.1740511932905\\
4 10.5415500550308\\
5 6.07560822004736\\
6 5.25141857704429\\
7 4.61185941083028\\
8 3.97933558839313\\
9 3.36092264655982\\
10 2.93177303237162\\
11 2.57409763238441\\
12 2.29120467197087\\
13 2.03244658656069\\
14 1.81064988660675\\
15 1.59831089932251\\
16 1.40931468657857\\
17 1.22629944014367\\
18 1.06293918377447\\
19 0.909292323211862\\
20 0.818804672897627\\
21 0.818317648393006\\
22 0.804117171783335\\
23 0.801791393712339\\
24 0.792007448425228\\
25 0.779448235063106\\
26 0.761925522278605\\
27 0.744655849003806\\
28 0.742255158384029\\
29 0.70611752375003\\
30 0.688258517080443\\
31 0.667070127582774\\
32 0.656204223238995\\
33 0.615541694112002\\
34 0.595512655419004\\
35 0.58996762276948\\
36 0.547349777988468\\
37 0.513768079039538\\
38 0.513196833897382\\
39 0.482881131179012\\
40 0.43946530739859\\
41 0.434458056570783\\
42 0.419615005919826\\
43 0.380325181771795\\
44 0.378810039739182\\
45 0.359871108806428\\
46 0.305272279402934\\
47 0.265207859912736\\
};
\addlegendentry{L 47};

\end{axis}
\end{tikzpicture}%    	
    	\caption{Singular values of \texttt{x3}}
    	\label{fig:svd3}
\end{figure}

\begin{figure}[H]
	\centering
    	\setlength\figureheight{6cm}
    	\setlength\figurewidth{10cm}
    	% This file was created by matlab2tikz v0.4.2.
% Copyright (c) 2008--2013, Nico Schlömer <nico.schloemer@gmail.com>
% All rights reserved.
% 
% The latest updates can be retrieved from
%   http://www.mathworks.com/matlabcentral/fileexchange/22022-matlab2tikz
% where you can also make suggestions and rate matlab2tikz.
% 
% 
% 
%
% defining custom colors
\definecolor{mycolor1}{rgb}{0,0.6,0.6}%
\definecolor{mycolor2}{rgb}{0,0.7,0.7}%
\definecolor{mycolor3}{rgb}{0,0.8,0.8}%
\definecolor{mycolor4}{rgb}{0,0.9,0.9}%
\definecolor{mycolor5}{rgb}{0,1,1}%
%
\begin{tikzpicture}

\begin{axis}[%
width=\figurewidth,
height=\figureheight,
scale only axis,
xmin=0,
xmax=50,
ymin=4,
ymax=16,
title={SVD van x4},
legend style={draw=black,fill=white,legend cell align=left}
]
\addplot [
color=teal!20!black,
solid,
mark=+,
mark options={solid}
]
table[row sep=crcr]{
1 10.0498756211209\\
2 9.9498743710662\\
};
\addlegendentry{L 2};

\addplot [
color=teal!40!black,
solid,
mark=+,
mark options={solid}
]
table[row sep=crcr]{
1 10.4880884817015\\
2 10.0098028163749\\
3 9.47648920103291\\
};
\addlegendentry{L 3};

\addplot [
color=teal!60!black,
solid,
mark=+,
mark options={solid}
]
table[row sep=crcr]{
1 10.6511136504464\\
2 10.6452140754812\\
3 9.97766395531916\\
4 9.49176453237701\\
5 9.14252827991725\\
};
\addlegendentry{L 5};

\addplot [
color=teal!80!black,
solid,
mark=+,
mark options={solid}
]
table[row sep=crcr]{
1 11.4183924530491\\
2 10.9173700307344\\
3 10.4416463345871\\
4 10.1379738896747\\
5 9.82456269050008\\
6 9.28196125855007\\
7 9.05926617072205\\
8 8.59521449523354\\
};
\addlegendentry{L 8};

\addplot [
color=teal,
solid,
mark=+,
mark options={solid}
]
table[row sep=crcr]{
1 11.7268741490082\\
2 11.6354690038852\\
3 11.2953506612392\\
4 10.7175943091234\\
5 10.3149916611399\\
6 10.1903878126125\\
7 10.1439536547092\\
8 9.59162437140307\\
9 9.2087436093546\\
10 8.28365555901508\\
11 8.22491493594733\\
12 7.64416541285124\\
};
\addlegendentry{L 12};

\addplot [
color=mycolor1,
solid,
mark=+,
mark options={solid}
]
table[row sep=crcr]{
1 12.3945514531839\\
2 12.2927712892226\\
3 11.921563905821\\
4 11.6865433019188\\
5 11.053748926928\\
6 10.8777427144836\\
7 10.7326735946181\\
8 10.2810344636566\\
9 10.1628682496213\\
10 9.52077399792071\\
11 9.17693960530558\\
12 8.4885310486471\\
13 8.33080407562692\\
14 7.97633115143627\\
15 7.82422595181956\\
16 7.63464607655389\\
17 7.24102118891637\\
};
\addlegendentry{L 17};

\addplot [
color=mycolor2,
solid,
mark=+,
mark options={solid}
]
table[row sep=crcr]{
1 12.9805119923037\\
2 12.7257546584571\\
3 12.707023462756\\
4 12.334546126301\\
5 11.9654641719581\\
6 11.6447500542855\\
7 11.3984341154357\\
8 11.1036544577627\\
9 10.8273676540194\\
10 10.5600594902923\\
11 10.5577862427649\\
12 8.83610921471103\\
13 8.70430040354982\\
14 8.57871861326801\\
15 8.52661393759142\\
16 8.51917926950877\\
17 8.28854977607639\\
18 8.19658399256999\\
19 7.87084127466865\\
20 7.5390692352286\\
21 7.36114631079257\\
22 7.23076703735638\\
23 7.06346663498857\\
};
\addlegendentry{L 23};

\addplot [
color=mycolor3,
solid,
mark=+,
mark options={solid}
]
table[row sep=crcr]{
1 13.9333465896602\\
2 13.736780201653\\
3 13.4412109452467\\
4 13.0431034961793\\
5 12.7612101163622\\
6 12.5672293350179\\
7 12.5549879497571\\
8 12.1167086036652\\
9 11.920515412058\\
10 10.8236748813663\\
11 10.8105292604215\\
12 9.88554532210962\\
13 9.51542469431348\\
14 9.1781946921673\\
15 8.79997807271775\\
16 8.72075248614358\\
17 8.6659672013139\\
18 8.51862555392614\\
19 8.44763564438232\\
20 8.42646506734828\\
21 8.20890432274554\\
22 7.75065046697032\\
23 7.61592757891181\\
24 7.55566528332892\\
25 7.51811248058148\\
26 7.33727786755608\\
27 7.23550573209854\\
28 7.2074586328329\\
29 7.03017830469414\\
30 6.50907303465795\\
};
\addlegendentry{L 30};

\addplot [
color=mycolor4,
solid,
mark=+,
mark options={solid}
]
table[row sep=crcr]{
1 14.4910479950132\\
2 14.3336124218288\\
3 13.9305307040418\\
4 13.9206825861217\\
5 13.5913750366233\\
6 13.4056996342284\\
7 13.281552844837\\
8 13.1724564935959\\
9 13.0309863370876\\
10 11.6573770340034\\
11 11.5680639458993\\
12 10.8258288833549\\
13 10.745514652846\\
14 10.3328997720342\\
15 10.1975934990033\\
16 10.1133104487996\\
17 9.99528011122364\\
18 9.38517346627529\\
19 9.07287799999728\\
20 8.69681815804016\\
21 8.50940660644188\\
22 8.37788198894617\\
23 8.27983953933104\\
24 8.01655266771983\\
25 7.90250158911349\\
26 7.8594012151312\\
27 7.36095486625134\\
28 7.34577929716095\\
29 7.26781249168411\\
30 7.25729920246919\\
31 7.09047790187523\\
32 7.04730481914266\\
33 6.93797758989325\\
34 6.54291952814545\\
35 6.42648566066511\\
36 6.12727229414706\\
37 5.85571144438218\\
38 5.82294363400318\\
};
\addlegendentry{L 38};

\addplot [
color=mycolor5,
solid,
mark=+,
mark options={solid}
]
table[row sep=crcr]{
1 15.242172190519\\
2 15.1933169515616\\
3 14.6550911068914\\
4 14.4451938491258\\
5 14.3873763444478\\
6 14.0516384653649\\
7 13.8597345877894\\
8 13.7508877405701\\
9 13.5826230417544\\
10 13.5813339917081\\
11 13.180259780216\\
12 11.1825139150326\\
13 11.1404111995141\\
14 11.0098137028177\\
15 10.9688407235163\\
16 10.7073777378866\\
17 10.4767010158087\\
18 10.4081544367832\\
19 10.3019115397203\\
20 9.64339290391026\\
21 9.3928031597742\\
22 9.37664687018981\\
23 9.3327606017036\\
24 9.0840062511775\\
25 8.7393379911651\\
26 8.50105660501031\\
27 8.32688023545744\\
28 8.17213460564718\\
29 7.96119862220187\\
30 7.69340834637919\\
31 7.43299955571599\\
32 7.38436011572583\\
33 7.38238194123052\\
34 7.28274922142471\\
35 7.10664348514616\\
36 6.66386680322498\\
37 6.56186767333718\\
38 6.55888249707756\\
39 6.38414829623482\\
40 6.07837814254579\\
41 6.02246681178525\\
42 6.01958453252228\\
43 5.92626188254214\\
44 5.74871124004205\\
45 5.65671255623652\\
46 5.6254965922766\\
47 5.39824436792588\\
};
\addlegendentry{L 47};

\end{axis}
\end{tikzpicture}%    	
    	\caption{Singular values of \texttt{x4}}
    	\label{fig:svd4}
\end{figure}

\subsection{Error criterion}
As mentioned at the beginning of this chapter, we have written two Matlab codes (\texttt{ch1.m} and \texttt{ch2.m}) that each use an individual method to retrieve $\hat{h}$ from the input- and outputsignals.
This will be compared to the real value of $h$ to find the difference between them. \\
Firstly we add a small amount of Gaussian noise to the measured data vector, which are also computed in the \texttt{ch1.m} and \texttt{ch2.m} files, so that we can determine the $\hat{h}$ filter.
The plots for \texttt{ch1.m} and \texttt{ch2.m}, with a noise variance of 0.01, can be seen in Figure \ref{fig:error1} and Figure \ref{fig:error2}.\\
%As we know \texttt{ch1.m} uses a time-domain equalization to determine $\hat{h}$.
In Figure \ref{fig:error1} we see that, for all test signals, the error decreases with an increasing channel length.
Figure \ref{fig:error2} is a little different.
In this figure the error alsod decreases for all signals only till a certain channel length.
We see that there is a value for which the error increases for a few test signals or stays constant.\\
The Matlab codes we used in this section can be found in Appendix \ref{lst:noise.m} and Appendix \ref{lst:error_criterion.m}.


\begin{figure}[H]
	\centering
    	\setlength\figureheight{6cm}
    	\setlength\figurewidth{10cm}
    	% This file was created by matlab2tikz v0.4.2.
% Copyright (c) 2008--2013, Nico Schlömer <nico.schloemer@gmail.com>
% All rights reserved.
% 
% The latest updates can be retrieved from
%   http://www.mathworks.com/matlabcentral/fileexchange/22022-matlab2tikz
% where you can also make suggestions and rate matlab2tikz.
% 
% 
% 
%
% defining custom colors
\definecolor{mycolor1}{rgb}{0,0.75,0.75}%
%
\begin{tikzpicture}

\begin{axis}[%
width=\figurewidth,
height=\figureheight,
scale only axis,
xmin=0,
xmax=50,
xlabel={L},
ymin=0,
ymax=6,
ylabel={Error},
title={Equalizer 1},
axis x line*=bottom,
axis y line*=left,
legend style={draw=black,fill=white,legend cell align=left}
]
\addplot [
color=blue,
solid
]
table[row sep=crcr]{
1 4.58257569495584\\
2 4.57194958126574\\
3 4.12906725365577\\
4 0.00846907204537622\\
5 0.00851460716326357\\
6 0.0110494278428847\\
7 0.0121443202305918\\
8 0.0130714787812108\\
9 0.0209843331538699\\
10 0.0341374002953432\\
11 0.031375659694929\\
12 0.0373284176343878\\
13 0.0359198728895848\\
14 0.0380001811280385\\
15 0.0375519123791628\\
16 0.0379964883461758\\
17 0.0440662472943406\\
18 0.0426786589009142\\
19 0.0426890538933982\\
20 0.0520259157021509\\
21 0.0498732068432703\\
22 0.0504492121062356\\
23 0.058557628353939\\
24 0.0595744330997801\\
25 0.0592905300017986\\
26 0.0595777734610265\\
27 0.059739879565699\\
28 0.0596746776176169\\
29 0.0624982481531707\\
30 0.0618799650267735\\
31 0.0622594111522694\\
32 0.0625170614677228\\
33 0.0676350465358103\\
34 0.0693556057021231\\
35 0.069429684963434\\
36 0.0692753287363963\\
37 0.0712462809383827\\
38 0.0710399035026003\\
39 0.071059590264511\\
40 0.0710548324199111\\
41 0.0710933290203598\\
42 0.0716679629846236\\
43 0.0722188476397238\\
44 0.0728493807884501\\
45 0.0736717604905123\\
46 0.074769856459\\
47 0.0774367162443554\\
48 0.0776966333674472\\
49 0.0852919863684175\\
50 0.0869256243695956\\
};
\addlegendentry{$\text{x}_\text{1}$};

\addplot [
color=green!50!black,
solid
]
table[row sep=crcr]{
1 4.58257569495584\\
2 4.57099656874317\\
3 4.12976562669009\\
4 0.00364254466297076\\
5 0.00576750429752668\\
6 0.0053249305237694\\
7 0.00696767189701657\\
8 0.0137911031369147\\
9 0.0133364555564858\\
10 0.0164074509610277\\
11 0.0184519001921292\\
12 0.0180309560666662\\
13 0.0185817610595656\\
14 0.0187036851238959\\
15 0.0187780743544558\\
16 0.0209857750880717\\
17 0.0214839601682421\\
18 0.0216881678661052\\
19 0.0234482169745784\\
20 0.0260213711777327\\
21 0.0275272630423451\\
22 0.0301217610712139\\
23 0.0293753830119231\\
24 0.0294726215963606\\
25 0.0295785902423613\\
26 0.0299573502229435\\
27 0.0298677430816435\\
28 0.0301990177130401\\
29 0.0316069971910551\\
30 0.0319804519561704\\
31 0.0328298573357462\\
32 0.0351405373935555\\
33 0.0347105416343785\\
34 0.0346628516416558\\
35 0.0346628218380643\\
36 0.0354962811055621\\
37 0.0353913591587502\\
38 0.0355252133359937\\
39 0.0354855020489982\\
40 0.0354856636993716\\
41 0.0355154977497734\\
42 0.0371209593263205\\
43 0.0373966978475259\\
44 0.0381491644885588\\
45 0.0391698525061158\\
46 0.0405487597241221\\
47 0.0411010402722124\\
48 0.0438761770274433\\
49 0.0432106164812708\\
50 0.0432303772458281\\
};
\addlegendentry{$\text{x}_\text{2}$};

\addplot [
color=red,
solid
]
table[row sep=crcr]{
1 4.58257569495584\\
2 5.09355748421886\\
3 5.29718496416727\\
4 0.0173546673352357\\
5 0.0178520740198129\\
6 0.0219238454182359\\
7 0.0231883679972411\\
8 0.0290187529839107\\
9 0.027291632865252\\
10 0.0357893983504313\\
11 0.041025784133887\\
12 0.0464282937473927\\
13 0.0453314403307087\\
14 0.058935260602203\\
15 0.0651669932081087\\
16 0.0668878407226711\\
17 0.0669790920998374\\
18 0.0752133510671172\\
19 0.0947925887689373\\
20 0.0938996994592183\\
21 0.0982390315178459\\
22 0.123409247455465\\
23 0.121768379743388\\
24 0.123217228608497\\
25 0.134313416482015\\
26 0.12852435277247\\
27 0.130249069587746\\
28 0.132883538162238\\
29 0.136255522260245\\
30 0.1373091953999\\
31 0.146658199568053\\
32 0.145739232646805\\
33 0.147272950758063\\
34 0.14482059812748\\
35 0.14371466289702\\
36 0.145396615996139\\
37 0.144570312809358\\
38 0.145180052775839\\
39 0.14849659128416\\
40 0.148701161285409\\
41 0.148589846094139\\
42 0.147858467228171\\
43 0.148057498470468\\
44 0.148791545390609\\
45 0.149309341664631\\
46 0.149381456528526\\
47 0.1492422731604\\
48 0.149518843435043\\
49 0.150029134521147\\
50 0.151714282281496\\
};
\addlegendentry{$\text{x}_\text{3}$};

\addplot [
color=mycolor1,
solid
]
table[row sep=crcr]{
1 4.58257569495584\\
2 4.49325095062434\\
3 4.01009126233169\\
4 0.00214644887851757\\
5 0.00236762792333101\\
6 0.00235967137388885\\
7 0.0026909330295423\\
8 0.00268012374335532\\
9 0.00305515832328524\\
10 0.00318170769583711\\
11 0.0038249764697791\\
12 0.0038592252872042\\
13 0.00385974092495088\\
14 0.00395090710327331\\
15 0.00432048932115256\\
16 0.00440240577231248\\
17 0.00444358082589232\\
18 0.00456650637042098\\
19 0.0045482370416397\\
20 0.00455179107050542\\
21 0.00464559797987417\\
22 0.00469396693812394\\
23 0.00472345201556239\\
24 0.00548494027833629\\
25 0.00576242524764012\\
26 0.00595127239198961\\
27 0.00599571106942761\\
28 0.00609330610923894\\
29 0.00611677558488291\\
30 0.00608754154919309\\
31 0.00625131545872601\\
32 0.00629658365162574\\
33 0.00651793603194086\\
34 0.00653628709490829\\
35 0.00653396532468221\\
36 0.00650860762415995\\
37 0.00677151576312152\\
38 0.00677270121567967\\
39 0.00655241457348738\\
40 0.00655597471630741\\
41 0.00653686141591312\\
42 0.00655471997249566\\
43 0.00681703543904122\\
44 0.00681400679606526\\
45 0.00668217451766442\\
46 0.00668579862436936\\
47 0.00729665474292111\\
48 0.00744289725747299\\
49 0.0075526774816822\\
50 0.00762001590106233\\
};
\addlegendentry{$\text{x}_\text{4}$};

\end{axis}
\end{tikzpicture}%    	
    	\caption{Singular values of \texttt{x3}}
    	\label{fig:error1}
\end{figure}

\begin{figure}[H]
	\centering
    	\setlength\figureheight{6cm}
    	\setlength\figurewidth{10cm}
    	% This file was created by matlab2tikz v0.4.2.
% Copyright (c) 2008--2013, Nico Schlömer <nico.schloemer@gmail.com>
% All rights reserved.
% 
% The latest updates can be retrieved from
%   http://www.mathworks.com/matlabcentral/fileexchange/22022-matlab2tikz
% where you can also make suggestions and rate matlab2tikz.
% 
% 
% 
%
% defining custom colors
\definecolor{mycolor1}{rgb}{0,0.75,0.75}%
%
\begin{tikzpicture}

\begin{axis}[%
width=\figurewidth,
height=\figureheight,
unbounded coords=jump,
scale only axis,
xmin=0,
xmax=50,
xlabel={L},
ymin=4.2,
ymax=5.8,
ylabel={Error},
title={Equalizer 2},
axis x line*=bottom,
axis y line*=left,
legend style={draw=black,fill=white,legend cell align=left}
]
\addplot [
color=blue,
solid
]
table[row sep=crcr]{
1 NaN\\
2 4.58257569495584\\
3 5.476961867668\\
4 5.23566322604688\\
5 4.8166123732242\\
6 4.81661324706011\\
7 4.81661326475354\\
8 4.81661328137434\\
9 4.81661728310931\\
10 4.8166182195105\\
11 4.81662183127145\\
12 4.81662464505535\\
13 4.81662901637467\\
14 4.81663196427311\\
15 4.81663295242449\\
16 4.81663338432237\\
17 4.81664052883779\\
18 4.81664476860841\\
19 4.81664497039328\\
20 4.81665363437399\\
21 4.81666124849162\\
22 4.8166612710187\\
23 4.81666572262502\\
24 4.81666655256794\\
25 4.81666668765062\\
26 4.81666751318322\\
27 4.81666938523454\\
28 4.81666986843945\\
29 4.81667351708509\\
30 4.81667781509935\\
31 4.8166778192044\\
32 4.81667782070716\\
33 4.81668191998955\\
34 4.81668208465341\\
35 4.81668217067326\\
36 4.81668277032392\\
37 4.81668717214302\\
38 4.81668828116679\\
39 4.81668858647544\\
40 4.81668870435874\\
41 4.81668873331251\\
42 4.81668919290822\\
43 4.81669235181312\\
44 4.81669237248576\\
45 4.81669251529112\\
46 4.81669271189582\\
47 4.81669367951533\\
48 4.81669394702486\\
49 4.81670149537591\\
50 4.81670210165476\\
};
\addlegendentry{$\text{x}_\text{1}$};

\addplot [
color=green!50!black,
solid
]
table[row sep=crcr]{
1 NaN\\
2 4.58257569495584\\
3 5.47691458819111\\
4 5.23648129234321\\
5 4.81822364030136\\
6 4.81822383093787\\
7 4.81822390557123\\
8 4.81822445567981\\
9 4.81822479238028\\
10 4.81822606476716\\
11 4.81822608086283\\
12 4.81822704226325\\
13 4.81822797682888\\
14 4.81822866116008\\
15 4.81822869995237\\
16 4.81822927698659\\
17 4.81823146892154\\
18 4.81823165601055\\
19 4.81823203155387\\
20 4.81823579762862\\
21 4.81823591835411\\
22 4.81823620350476\\
23 4.81823713613694\\
24 4.81823713725364\\
25 4.81823713769832\\
26 4.81823760316302\\
27 4.81823781204045\\
28 4.81823802773732\\
29 4.81823979031671\\
30 4.81823992822897\\
31 4.8182400559149\\
32 4.81824047052237\\
33 4.81824098377973\\
34 4.81824098401906\\
35 4.81824098754848\\
36 4.81824154437982\\
37 4.81824252746587\\
38 4.81824254139272\\
39 4.81824261004067\\
40 4.81824261911835\\
41 4.8182426419486\\
42 4.81824344250846\\
43 4.81824354860128\\
44 4.8182436933597\\
45 4.818243790323\\
46 4.8182439260834\\
47 4.81824392669196\\
48 4.81824484674139\\
49 4.81824602637947\\
50 4.81824602681912\\
};
\addlegendentry{$\text{x}_\text{2}$};

\addplot [
color=red,
solid
]
table[row sep=crcr]{
1 NaN\\
2 4.58257569495584\\
3 4.60242082100129\\
4 4.20130272640053\\
5 4.84551952241539\\
6 4.84732832196317\\
7 4.85583485364227\\
8 4.90023521373826\\
9 4.98277620490546\\
10 5.07884344437178\\
11 5.16419829621037\\
12 5.22980385216272\\
13 5.27690354030139\\
14 5.30974729936956\\
15 5.33230651640054\\
16 5.34756372577384\\
17 5.35758102589821\\
18 5.36379046025333\\
19 5.36722975891541\\
20 5.36874593729478\\
21 5.36911627610636\\
22 5.36911641286737\\
23 5.36949398761131\\
24 5.37085786183125\\
25 5.37352930759981\\
26 5.37745237593046\\
27 5.3822503497645\\
28 5.38739179229893\\
29 5.39237247648858\\
30 5.39681730457732\\
31 5.40051179745194\\
32 5.40337219679938\\
33 5.40541478754479\\
34 5.40672377109776\\
35 5.40743103956456\\
36 5.40770690979068\\
37 5.40774821465364\\
38 5.40776158044445\\
39 5.40793772140463\\
40 5.40841944710453\\
41 5.40927427081665\\
42 5.41048453593686\\
43 5.41195935413062\\
44 5.4135640414409\\
45 5.41515641678042\\
46 5.41661251230515\\
47 5.41784351666298\\
48 5.41880078625128\\
49 5.41947435871979\\
50 5.41988723714711\\
};
\addlegendentry{$\text{x}_\text{3}$};

\addplot [
color=mycolor1,
solid
]
table[row sep=crcr]{
1 NaN\\
2 4.58257569495584\\
3 5.27008661820202\\
4 5.37897509224547\\
5 5.31858023410944\\
6 5.32121159864545\\
7 5.32374445523661\\
8 5.32504401429373\\
9 5.3287816303772\\
10 5.33305371327043\\
11 5.33562827139044\\
12 5.34137074916295\\
13 5.34594541028371\\
14 5.34636058174911\\
15 5.34731195353347\\
16 5.35030241797829\\
17 5.35048246964565\\
18 5.35227127856193\\
19 5.35335820593416\\
20 5.35338946378376\\
21 5.35380587331268\\
22 5.35387024629357\\
23 5.35387063005885\\
24 5.35456734136043\\
25 5.35476466707714\\
26 5.35533656692786\\
27 5.35534028744303\\
28 5.35596294376201\\
29 5.35596645440182\\
30 5.35604849997894\\
31 5.35685752284905\\
32 5.35954894103885\\
33 5.36312735307729\\
34 5.36336577596497\\
35 5.36480956800506\\
36 5.36665885732016\\
37 5.36665911172139\\
38 5.36665939284668\\
39 5.36717159950919\\
40 5.36868635564309\\
41 5.36871077309639\\
42 5.36871107366538\\
43 5.3698777027893\\
44 5.36989189975463\\
45 5.36995574472738\\
46 5.37164480122178\\
47 5.37179228103854\\
48 5.37180567192564\\
49 5.37258849289273\\
50 5.37270900404566\\
};
\addlegendentry{$\text{x}_\text{4}$};

\end{axis}
\end{tikzpicture}%    	
    	\caption{Singular values of \texttt{x4}}
    	\label{fig:error2}
\end{figure}


\subsection{Cross-correlation}

Next we had to look at the cross-correlation of the different signals. 
What happens if we have an interfering signal from a neighboring car?
In order to find out how big the influence of an interfering signal is, we plotted the cross-correlation of different signals with each other in Figure \ref{fig:cross1}, \ref{fig:cross2}, \ref{fig:cross3} and \ref{fig:cross4}.
The Matlab script can again be found in Appendix \ref{lst:crosscorrelation.m}.
\\
In an ideal situation you want the cross-correlation between the different signals to be zero. 
In this way you are able to prevent most errors that occur when we have interfering signals. 
From our figures we see that only the signals in Figure \ref{fig:cross1} and Figure \ref{fig:cross2} go to zero at a certain moment, so the signals \texttt{x1} and \texttt{x2} are the most suitable.

\begin{figure}[H]
	\centering
    	\setlength\figureheight{6cm}
    	\setlength\figurewidth{10cm}
    	% This file was created by matlab2tikz v0.4.2.
% Copyright (c) 2008--2013, Nico Schlömer <nico.schloemer@gmail.com>
% All rights reserved.
% 
% The latest updates can be retrieved from
%   http://www.mathworks.com/matlabcentral/fileexchange/22022-matlab2tikz
% where you can also make suggestions and rate matlab2tikz.
% 
% 
% 
%
% defining custom colors
\definecolor{mycolor1}{rgb}{0,0.75,0.75}%
%
\begin{tikzpicture}

\begin{axis}[%
width=\figurewidth,
height=\figureheight,
scale only axis,
xmin=0,
xmax=200,
ymin=-2.5,
ymax=1.5,
title={Cross correlation x1 with:},
axis x line*=bottom,
axis y line*=left,
legend style={draw=black,fill=white,legend cell align=left}
]
\addplot [
color=blue,
solid
]
table[row sep=crcr]{
1 0\\
2 0\\
3 0\\
4 0\\
5 0\\
6 0\\
7 0\\
8 0\\
9 0\\
10 0\\
11 0\\
12 0\\
13 0\\
14 0\\
15 0\\
16 0\\
17 0\\
18 0\\
19 0\\
20 0\\
21 0\\
22 0\\
23 0\\
24 0\\
25 0\\
26 0\\
27 0\\
28 0\\
29 0\\
30 0\\
31 0\\
32 0\\
33 0\\
34 0\\
35 0\\
36 0\\
37 0\\
38 0\\
39 0\\
40 0\\
41 0\\
42 0\\
43 0\\
44 0\\
45 0\\
46 0\\
47 0\\
48 0\\
49 0\\
50 0\\
51 0\\
52 0\\
53 0\\
54 0\\
55 0\\
56 0\\
57 0\\
58 0\\
59 0\\
60 0\\
61 0\\
62 0\\
63 0\\
64 0\\
65 0\\
66 0\\
67 0\\
68 0\\
69 0\\
70 0\\
71 0\\
72 0\\
73 0\\
74 0\\
75 0\\
76 0\\
77 0\\
78 0\\
79 0\\
80 0\\
81 0\\
82 0\\
83 0\\
84 0\\
85 0\\
86 0\\
87 0\\
88 0\\
89 0\\
90 0\\
91 0\\
92 0\\
93 0\\
94 0\\
95 0\\
96 0\\
97 0\\
98 0\\
99 0\\
100 0\\
101 0\\
102 0\\
103 0\\
104 0\\
105 0\\
106 0\\
107 0\\
108 0\\
109 0\\
110 0\\
111 0\\
112 0\\
113 0\\
114 0\\
115 0\\
116 0\\
117 0\\
118 0\\
119 0\\
120 0\\
121 0\\
122 0\\
123 0\\
124 0\\
125 0\\
126 0\\
127 0\\
128 0\\
129 0\\
130 0\\
131 0\\
132 0\\
133 0\\
134 0\\
135 0\\
136 0\\
137 0\\
138 0\\
139 0\\
140 0\\
141 0\\
142 0\\
143 0\\
144 0\\
145 0\\
146 0\\
147 0\\
148 0\\
149 0\\
150 0\\
151 0\\
152 0\\
153 0\\
154 0\\
155 0\\
156 0\\
157 0\\
158 0\\
159 0\\
160 0\\
161 0\\
162 0\\
163 0\\
164 0\\
165 0\\
166 0\\
167 0\\
168 0\\
169 0\\
170 0\\
171 0\\
172 0\\
173 0\\
174 0\\
175 0\\
176 0\\
177 0\\
178 0\\
179 0\\
180 0\\
181 0\\
182 0\\
183 0\\
184 0\\
185 0\\
186 0\\
187 0\\
188 0\\
189 0\\
190 0\\
191 0\\
192 0\\
193 0\\
194 0\\
195 0\\
196 0\\
197 0\\
198 0\\
199 0\\
};
\addlegendentry{2};

\addplot [
color=green!50!black,
solid
]
table[row sep=crcr]{
1 1\\
2 -2.5\\
3 1\\
4 0\\
5 0\\
6 0\\
7 0\\
8 0\\
9 0\\
10 0\\
11 0\\
12 0\\
13 0\\
14 0\\
15 0\\
16 0\\
17 0\\
18 0\\
19 0\\
20 0\\
21 0\\
22 0\\
23 0\\
24 0\\
25 0\\
26 0\\
27 0\\
28 0\\
29 0\\
30 0\\
31 0\\
32 0\\
33 0\\
34 0\\
35 0\\
36 0\\
37 0\\
38 0\\
39 0\\
40 0\\
41 0\\
42 0\\
43 0\\
44 0\\
45 0\\
46 0\\
47 0\\
48 0\\
49 0\\
50 0\\
51 0\\
52 0\\
53 0\\
54 0\\
55 0\\
56 0\\
57 0\\
58 0\\
59 0\\
60 0\\
61 0\\
62 0\\
63 0\\
64 0\\
65 0\\
66 0\\
67 0\\
68 0\\
69 0\\
70 0\\
71 0\\
72 0\\
73 0\\
74 0\\
75 0\\
76 0\\
77 0\\
78 0\\
79 0\\
80 0\\
81 0\\
82 0\\
83 0\\
84 0\\
85 0\\
86 0\\
87 0\\
88 0\\
89 0\\
90 0\\
91 0\\
92 0\\
93 0\\
94 0\\
95 0\\
96 0\\
97 0\\
98 0\\
99 0\\
100 0\\
101 0\\
102 0\\
103 0\\
104 0\\
105 0\\
106 0\\
107 0\\
108 0\\
109 0\\
110 0\\
111 0\\
112 0\\
113 0\\
114 0\\
115 0\\
116 0\\
117 0\\
118 0\\
119 0\\
120 0\\
121 0\\
122 0\\
123 0\\
124 0\\
125 0\\
126 0\\
127 0\\
128 0\\
129 0\\
130 0\\
131 0\\
132 0\\
133 0\\
134 0\\
135 0\\
136 0\\
137 0\\
138 0\\
139 0\\
140 0\\
141 0\\
142 0\\
143 0\\
144 0\\
145 0\\
146 0\\
147 0\\
148 0\\
149 0\\
150 0\\
151 0\\
152 0\\
153 0\\
154 0\\
155 0\\
156 0\\
157 0\\
158 0\\
159 0\\
160 0\\
161 0\\
162 0\\
163 0\\
164 0\\
165 0\\
166 0\\
167 0\\
168 0\\
169 0\\
170 0\\
171 0\\
172 0\\
173 0\\
174 0\\
175 0\\
176 0\\
177 0\\
178 0\\
179 0\\
180 0\\
181 0\\
182 0\\
183 0\\
184 0\\
185 0\\
186 0\\
187 0\\
188 0\\
189 0\\
190 0\\
191 0\\
192 0\\
193 0\\
194 0\\
195 0\\
196 0\\
197 0\\
198 0\\
199 0\\
};
\addlegendentry{3};

\addplot [
color=red,
solid
]
table[row sep=crcr]{
1 0\\
2 0\\
3 0\\
4 0\\
5 0\\
6 0\\
7 0\\
8 0\\
9 0\\
10 0\\
11 0\\
12 0\\
13 0\\
14 0\\
15 0\\
16 0\\
17 0\\
18 0\\
19 0\\
20 0\\
21 0\\
22 0\\
23 0\\
24 0\\
25 0\\
26 0\\
27 0\\
28 0\\
29 0\\
30 0\\
31 0\\
32 0\\
33 0\\
34 0\\
35 0\\
36 0\\
37 0\\
38 0\\
39 0\\
40 0\\
41 0\\
42 0\\
43 0\\
44 0\\
45 0\\
46 0\\
47 0\\
48 0\\
49 0\\
50 0\\
51 0\\
52 0\\
53 0\\
54 0\\
55 0\\
56 0\\
57 0\\
58 0\\
59 0\\
60 0\\
61 0\\
62 0\\
63 0\\
64 0\\
65 0\\
66 0\\
67 0\\
68 0\\
69 0\\
70 0\\
71 0\\
72 0\\
73 0\\
74 0\\
75 0\\
76 0\\
77 0\\
78 0\\
79 0\\
80 0\\
81 0\\
82 0\\
83 0\\
84 0\\
85 0\\
86 0\\
87 0\\
88 0\\
89 0\\
90 0\\
91 0\\
92 0\\
93 0\\
94 0\\
95 0\\
96 0\\
97 0\\
98 0\\
99 0\\
100 0\\
101 0\\
102 0\\
103 0\\
104 0\\
105 0\\
106 0\\
107 0\\
108 0\\
109 0\\
110 0\\
111 0\\
112 0\\
113 0\\
114 0\\
115 0\\
116 0\\
117 0\\
118 0\\
119 0\\
120 0\\
121 0\\
122 0\\
123 0\\
124 0\\
125 0\\
126 0\\
127 0\\
128 0\\
129 0\\
130 0\\
131 0\\
132 0\\
133 0\\
134 0\\
135 0\\
136 0\\
137 0\\
138 0\\
139 0\\
140 0\\
141 0\\
142 0\\
143 0\\
144 0\\
145 0\\
146 0\\
147 0\\
148 0\\
149 0\\
150 0\\
151 0\\
152 0\\
153 0\\
154 0\\
155 0\\
156 0\\
157 0\\
158 0\\
159 0\\
160 0\\
161 0\\
162 0\\
163 0\\
164 0\\
165 0\\
166 0\\
167 0\\
168 0\\
169 0\\
170 0\\
171 0\\
172 0\\
173 0\\
174 0\\
175 0\\
176 0\\
177 0\\
178 0\\
179 0\\
180 0\\
181 0\\
182 0\\
183 0\\
184 0\\
185 0\\
186 0\\
187 0\\
188 0\\
189 0\\
190 0\\
191 0\\
192 0\\
193 0\\
194 0\\
195 0\\
196 0\\
197 0\\
198 0\\
199 0\\
};
\addlegendentry{4};

\addplot [
color=mycolor1,
solid,
forget plot
]
table[row sep=crcr]{
1 0\\
2 0\\
3 0\\
4 0\\
5 0\\
6 0\\
7 0\\
8 0\\
9 0\\
10 0\\
11 0\\
12 0\\
13 0\\
14 0\\
15 0\\
16 0\\
17 0\\
18 0\\
19 0\\
20 0\\
21 0\\
22 0\\
23 0\\
24 0\\
25 0\\
26 0\\
27 0\\
28 0\\
29 0\\
30 0\\
31 0\\
32 0\\
33 0\\
34 0\\
35 0\\
36 0\\
37 0\\
38 0\\
39 0\\
40 0\\
41 0\\
42 0\\
43 0\\
44 0\\
45 0\\
46 0\\
47 0\\
48 0\\
49 0\\
50 0\\
51 0\\
52 0\\
53 0\\
54 0\\
55 0\\
56 0\\
57 0\\
58 0\\
59 0\\
60 0\\
61 0\\
62 0\\
63 0\\
64 0\\
65 0\\
66 0\\
67 0\\
68 0\\
69 0\\
70 0\\
71 0\\
72 0\\
73 0\\
74 0\\
75 0\\
76 0\\
77 0\\
78 0\\
79 0\\
80 0\\
81 0\\
82 0\\
83 0\\
84 0\\
85 0\\
86 0\\
87 0\\
88 0\\
89 0\\
90 0\\
91 0\\
92 0\\
93 0\\
94 0\\
95 0\\
96 0\\
97 0\\
98 0\\
99 0\\
100 0\\
101 0\\
102 0\\
103 0\\
104 0\\
105 0\\
106 0\\
107 0\\
108 0\\
109 0\\
110 0\\
111 0\\
112 0\\
113 0\\
114 0\\
115 0\\
116 0\\
117 0\\
118 0\\
119 0\\
120 0\\
121 0\\
122 0\\
123 0\\
124 0\\
125 0\\
126 0\\
127 0\\
128 0\\
129 0\\
130 0\\
131 0\\
132 0\\
133 0\\
134 0\\
135 0\\
136 0\\
137 0\\
138 0\\
139 0\\
140 0\\
141 0\\
142 0\\
143 0\\
144 0\\
145 0\\
146 0\\
147 0\\
148 0\\
149 0\\
150 0\\
151 0\\
152 0\\
153 0\\
154 0\\
155 0\\
156 0\\
157 0\\
158 0\\
159 0\\
160 0\\
161 0\\
162 0\\
163 0\\
164 0\\
165 0\\
166 0\\
167 0\\
168 0\\
169 0\\
170 0\\
171 0\\
172 0\\
173 0\\
174 0\\
175 0\\
176 0\\
177 0\\
178 0\\
179 0\\
180 0\\
181 0\\
182 0\\
183 0\\
184 0\\
185 0\\
186 0\\
187 0\\
188 0\\
189 0\\
190 0\\
191 0\\
192 0\\
193 0\\
194 0\\
195 0\\
196 0\\
197 0\\
198 0\\
199 0\\
};
\addplot [
color=blue,
solid,
forget plot
]
table[row sep=crcr]{
1 0\\
2 0\\
3 0\\
4 0\\
5 0\\
6 0\\
7 0\\
8 0\\
9 0\\
10 0\\
11 0\\
12 0\\
13 0\\
14 0\\
15 0\\
16 0\\
17 0\\
18 0\\
19 0\\
20 0\\
21 0\\
22 0\\
23 0\\
24 0\\
25 0\\
26 0\\
27 0\\
28 0\\
29 0\\
30 0\\
31 0\\
32 0\\
33 0\\
34 0\\
35 0\\
36 0\\
37 0\\
38 0\\
39 0\\
40 0\\
41 0\\
42 0\\
43 0\\
44 0\\
45 0\\
46 0\\
47 0\\
48 0\\
49 0\\
50 0\\
51 0\\
52 0\\
53 0\\
54 0\\
55 0\\
56 0\\
57 0\\
58 0\\
59 0\\
60 0\\
61 0\\
62 0\\
63 0\\
64 0\\
65 0\\
66 0\\
67 0\\
68 0\\
69 0\\
70 0\\
71 0\\
72 0\\
73 0\\
74 0\\
75 0\\
76 0\\
77 0\\
78 0\\
79 0\\
80 0\\
81 0\\
82 0\\
83 0\\
84 0\\
85 0\\
86 0\\
87 0\\
88 0\\
89 0\\
90 0\\
91 0\\
92 0\\
93 0\\
94 0\\
95 0\\
96 0\\
97 0\\
98 0\\
99 0\\
100 0\\
101 0\\
102 0\\
103 0\\
104 0\\
105 0\\
106 0\\
107 0\\
108 0\\
109 0\\
110 0\\
111 0\\
112 0\\
113 0\\
114 0\\
115 0\\
116 0\\
117 0\\
118 0\\
119 0\\
120 0\\
121 0\\
122 0\\
123 0\\
124 0\\
125 0\\
126 0\\
127 0\\
128 0\\
129 0\\
130 0\\
131 0\\
132 0\\
133 0\\
134 0\\
135 0\\
136 0\\
137 0\\
138 0\\
139 0\\
140 0\\
141 0\\
142 0\\
143 0\\
144 0\\
145 0\\
146 0\\
147 0\\
148 0\\
149 0\\
150 0\\
151 0\\
152 0\\
153 0\\
154 0\\
155 0\\
156 0\\
157 0\\
158 0\\
159 0\\
160 0\\
161 0\\
162 0\\
163 0\\
164 0\\
165 0\\
166 0\\
167 0\\
168 0\\
169 0\\
170 0\\
171 0\\
172 0\\
173 0\\
174 0\\
175 0\\
176 0\\
177 0\\
178 0\\
179 0\\
180 0\\
181 0\\
182 0\\
183 0\\
184 0\\
185 0\\
186 0\\
187 0\\
188 0\\
189 0\\
190 0\\
191 0\\
192 0\\
193 0\\
194 0\\
195 0\\
196 0\\
197 0\\
198 0\\
199 0\\
};
\addplot [
color=green!50!black,
solid,
forget plot
]
table[row sep=crcr]{
1 0\\
2 0\\
3 0\\
4 0\\
5 0\\
6 0\\
7 0\\
8 0\\
9 0\\
10 0\\
11 0\\
12 0\\
13 0\\
14 0\\
15 0\\
16 0\\
17 0\\
18 0\\
19 0\\
20 0\\
21 0\\
22 0\\
23 0\\
24 0\\
25 0\\
26 0\\
27 0\\
28 0\\
29 0\\
30 0\\
31 0\\
32 0\\
33 0\\
34 0\\
35 0\\
36 0\\
37 0\\
38 0\\
39 0\\
40 0\\
41 0\\
42 0\\
43 0\\
44 0\\
45 0\\
46 0\\
47 0\\
48 0\\
49 0\\
50 0\\
51 0\\
52 0\\
53 0\\
54 0\\
55 0\\
56 0\\
57 0\\
58 0\\
59 0\\
60 0\\
61 0\\
62 0\\
63 0\\
64 0\\
65 0\\
66 0\\
67 0\\
68 0\\
69 0\\
70 0\\
71 0\\
72 0\\
73 0\\
74 0\\
75 0\\
76 0\\
77 0\\
78 0\\
79 0\\
80 0\\
81 0\\
82 0\\
83 0\\
84 0\\
85 0\\
86 0\\
87 0\\
88 0\\
89 0\\
90 0\\
91 0\\
92 0\\
93 0\\
94 0\\
95 0\\
96 0\\
97 0\\
98 0\\
99 0\\
100 0\\
101 0\\
102 0\\
103 0\\
104 0\\
105 0\\
106 0\\
107 0\\
108 0\\
109 0\\
110 0\\
111 0\\
112 0\\
113 0\\
114 0\\
115 0\\
116 0\\
117 0\\
118 0\\
119 0\\
120 0\\
121 0\\
122 0\\
123 0\\
124 0\\
125 0\\
126 0\\
127 0\\
128 0\\
129 0\\
130 0\\
131 0\\
132 0\\
133 0\\
134 0\\
135 0\\
136 0\\
137 0\\
138 0\\
139 0\\
140 0\\
141 0\\
142 0\\
143 0\\
144 0\\
145 0\\
146 0\\
147 0\\
148 0\\
149 0\\
150 0\\
151 0\\
152 0\\
153 0\\
154 0\\
155 0\\
156 0\\
157 0\\
158 0\\
159 0\\
160 0\\
161 0\\
162 0\\
163 0\\
164 0\\
165 0\\
166 0\\
167 0\\
168 0\\
169 0\\
170 0\\
171 0\\
172 0\\
173 0\\
174 0\\
175 0\\
176 0\\
177 0\\
178 0\\
179 0\\
180 0\\
181 0\\
182 0\\
183 0\\
184 0\\
185 0\\
186 0\\
187 0\\
188 0\\
189 0\\
190 0\\
191 0\\
192 0\\
193 0\\
194 0\\
195 0\\
196 0\\
197 0\\
198 0\\
199 0\\
};
\addplot [
color=red,
solid,
forget plot
]
table[row sep=crcr]{
1 0.980066577841242\\
2 0.431027705082264\\
3 0.364805117908236\\
4 0.284038901892326\\
5 0.191948951194557\\
6 0.0922066015426035\\
7 -0.0112117343380959\\
8 -0.114183093751409\\
9 -0.212602333542443\\
10 -0.302545789200599\\
11 -0.380427698981775\\
12 -0.443143156913573\\
13 -0.488191895598324\\
14 -0.513777963984185\\
15 -0.518881326266116\\
16 -0.50329852749453\\
17 -0.467650804682084\\
18 -0.413359320044417\\
19 -0.342588503747343\\
20 -0.258159764906404\\
21 -0.163439010908893\\
22 -0.0622024593080696\\
23 0.0415139080541557\\
24 0.143575246906974\\
25 0.239912693743503\\
26 0.326685578568764\\
27 0.400434540292446\\
28 0.458219440538933\\
29 0.497736577686194\\
30 0.517410528179706\\
31 0.516456953698034\\
32 0.494913870246584\\
33 0.453640132579433\\
34 0.39428119437056\\
35 0.319203509168392\\
36 0.231400187360602\\
37 0.134371670308263\\
38 0.0319861787950597\\
39 -0.0716745007284786\\
40 -0.172477744089938\\
41 -0.266404844079529\\
42 -0.349711223624767\\
43 -0.41907572026167\\
44 -0.47173299040165\\
45 -0.505583754853849\\
46 -0.519278490461826\\
47 -0.512271231333124\\
48 -0.484841334776523\\
49 -0.43808234420769\\
50 -0.373858393024076\\
51 -0.294729887488973\\
52 -0.203851431413629\\
53 -0.104846062058214\\
54 -0.00166081106942079\\
55 0.101590651215718\\
56 0.200792014824727\\
57 0.291988434438515\\
58 0.37154419659403\\
59 0.436287664106851\\
60 0.483637719237073\\
61 0.511706664708392\\
62 0.519375480241545\\
63 0.506338434361573\\
64 0.473115272946933\\
65 0.421030498601479\\
66 0.352160566915352\\
67 0.269251104733243\\
68 0.175607450676413\\
69 0.0749628817224732\\
70 -0.0286702207666899\\
71 -0.131160332028\\
72 -0.228421494751714\\
73 -0.316576213305389\\
74 -0.392110037248587\\
75 -0.452011671381461\\
76 -0.49389302658167\\
77 -0.51608442538164\\
78 -0.517701166740225\\
79 -0.498678796281381\\
80 -0.45977567588674\\
81 -0.402542750200542\\
82 -0.329261715360952\\
83 -0.242854054975351\\
84 -0.14676456978817\\
85 -0.0448240443459157\\
86 0.0589034742939583\\
87 0.160282697294398\\
88 0.255271954955004\\
89 0.340084325328794\\
90 0.411338606849872\\
91 0.466194116169883\\
92 0.502463937238808\\
93 0.518702106746665\\
94 0.514261260117685\\
95 0.489318439893064\\
96 0.444868037603535\\
97 0.382682150517028\\
98 0.305239933712765\\
99 0.215628763991687\\
100 -0.290660905907218\\
101 0\\
102 0\\
103 0\\
104 0\\
105 0\\
106 0\\
107 0\\
108 0\\
109 0\\
110 0\\
111 0\\
112 0\\
113 0\\
114 0\\
115 0\\
116 0\\
117 0\\
118 0\\
119 0\\
120 0\\
121 0\\
122 0\\
123 0\\
124 0\\
125 0\\
126 0\\
127 0\\
128 0\\
129 0\\
130 0\\
131 0\\
132 0\\
133 0\\
134 0\\
135 0\\
136 0\\
137 0\\
138 0\\
139 0\\
140 0\\
141 0\\
142 0\\
143 0\\
144 0\\
145 0\\
146 0\\
147 0\\
148 0\\
149 0\\
150 0\\
151 0\\
152 0\\
153 0\\
154 0\\
155 0\\
156 0\\
157 0\\
158 0\\
159 0\\
160 0\\
161 0\\
162 0\\
163 0\\
164 0\\
165 0\\
166 0\\
167 0\\
168 0\\
169 0\\
170 0\\
171 0\\
172 0\\
173 0\\
174 0\\
175 0\\
176 0\\
177 0\\
178 0\\
179 0\\
180 0\\
181 0\\
182 0\\
183 0\\
184 0\\
185 0\\
186 0\\
187 0\\
188 0\\
189 0\\
190 0\\
191 0\\
192 0\\
193 0\\
194 0\\
195 0\\
196 0\\
197 0\\
198 0\\
199 0\\
};
\addplot [
color=mycolor1,
solid,
forget plot
]
table[row sep=crcr]{
1 0\\
2 0\\
3 0\\
4 0\\
5 0\\
6 0\\
7 0\\
8 0\\
9 0\\
10 0\\
11 0\\
12 0\\
13 0\\
14 0\\
15 0\\
16 0\\
17 0\\
18 0\\
19 0\\
20 0\\
21 0\\
22 0\\
23 0\\
24 0\\
25 0\\
26 0\\
27 0\\
28 0\\
29 0\\
30 0\\
31 0\\
32 0\\
33 0\\
34 0\\
35 0\\
36 0\\
37 0\\
38 0\\
39 0\\
40 0\\
41 0\\
42 0\\
43 0\\
44 0\\
45 0\\
46 0\\
47 0\\
48 0\\
49 0\\
50 0\\
51 0\\
52 0\\
53 0\\
54 0\\
55 0\\
56 0\\
57 0\\
58 0\\
59 0\\
60 0\\
61 0\\
62 0\\
63 0\\
64 0\\
65 0\\
66 0\\
67 0\\
68 0\\
69 0\\
70 0\\
71 0\\
72 0\\
73 0\\
74 0\\
75 0\\
76 0\\
77 0\\
78 0\\
79 0\\
80 0\\
81 0\\
82 0\\
83 0\\
84 0\\
85 0\\
86 0\\
87 0\\
88 0\\
89 0\\
90 0\\
91 0\\
92 0\\
93 0\\
94 0\\
95 0\\
96 0\\
97 0\\
98 0\\
99 0\\
100 0\\
101 0\\
102 0\\
103 0\\
104 0\\
105 0\\
106 0\\
107 0\\
108 0\\
109 0\\
110 0\\
111 0\\
112 0\\
113 0\\
114 0\\
115 0\\
116 0\\
117 0\\
118 0\\
119 0\\
120 0\\
121 0\\
122 0\\
123 0\\
124 0\\
125 0\\
126 0\\
127 0\\
128 0\\
129 0\\
130 0\\
131 0\\
132 0\\
133 0\\
134 0\\
135 0\\
136 0\\
137 0\\
138 0\\
139 0\\
140 0\\
141 0\\
142 0\\
143 0\\
144 0\\
145 0\\
146 0\\
147 0\\
148 0\\
149 0\\
150 0\\
151 0\\
152 0\\
153 0\\
154 0\\
155 0\\
156 0\\
157 0\\
158 0\\
159 0\\
160 0\\
161 0\\
162 0\\
163 0\\
164 0\\
165 0\\
166 0\\
167 0\\
168 0\\
169 0\\
170 0\\
171 0\\
172 0\\
173 0\\
174 0\\
175 0\\
176 0\\
177 0\\
178 0\\
179 0\\
180 0\\
181 0\\
182 0\\
183 0\\
184 0\\
185 0\\
186 0\\
187 0\\
188 0\\
189 0\\
190 0\\
191 0\\
192 0\\
193 0\\
194 0\\
195 0\\
196 0\\
197 0\\
198 0\\
199 0\\
};
\addplot [
color=blue,
solid,
forget plot
]
table[row sep=crcr]{
1 0\\
2 0\\
3 0\\
4 0\\
5 0\\
6 0\\
7 0\\
8 0\\
9 0\\
10 0\\
11 0\\
12 0\\
13 0\\
14 0\\
15 0\\
16 0\\
17 0\\
18 0\\
19 0\\
20 0\\
21 0\\
22 0\\
23 0\\
24 0\\
25 0\\
26 0\\
27 0\\
28 0\\
29 0\\
30 0\\
31 0\\
32 0\\
33 0\\
34 0\\
35 0\\
36 0\\
37 0\\
38 0\\
39 0\\
40 0\\
41 0\\
42 0\\
43 0\\
44 0\\
45 0\\
46 0\\
47 0\\
48 0\\
49 0\\
50 0\\
51 0\\
52 0\\
53 0\\
54 0\\
55 0\\
56 0\\
57 0\\
58 0\\
59 0\\
60 0\\
61 0\\
62 0\\
63 0\\
64 0\\
65 0\\
66 0\\
67 0\\
68 0\\
69 0\\
70 0\\
71 0\\
72 0\\
73 0\\
74 0\\
75 0\\
76 0\\
77 0\\
78 0\\
79 0\\
80 0\\
81 0\\
82 0\\
83 0\\
84 0\\
85 0\\
86 0\\
87 0\\
88 0\\
89 0\\
90 0\\
91 0\\
92 0\\
93 0\\
94 0\\
95 0\\
96 0\\
97 0\\
98 0\\
99 0\\
100 0\\
101 0\\
102 0\\
103 0\\
104 0\\
105 0\\
106 0\\
107 0\\
108 0\\
109 0\\
110 0\\
111 0\\
112 0\\
113 0\\
114 0\\
115 0\\
116 0\\
117 0\\
118 0\\
119 0\\
120 0\\
121 0\\
122 0\\
123 0\\
124 0\\
125 0\\
126 0\\
127 0\\
128 0\\
129 0\\
130 0\\
131 0\\
132 0\\
133 0\\
134 0\\
135 0\\
136 0\\
137 0\\
138 0\\
139 0\\
140 0\\
141 0\\
142 0\\
143 0\\
144 0\\
145 0\\
146 0\\
147 0\\
148 0\\
149 0\\
150 0\\
151 0\\
152 0\\
153 0\\
154 0\\
155 0\\
156 0\\
157 0\\
158 0\\
159 0\\
160 0\\
161 0\\
162 0\\
163 0\\
164 0\\
165 0\\
166 0\\
167 0\\
168 0\\
169 0\\
170 0\\
171 0\\
172 0\\
173 0\\
174 0\\
175 0\\
176 0\\
177 0\\
178 0\\
179 0\\
180 0\\
181 0\\
182 0\\
183 0\\
184 0\\
185 0\\
186 0\\
187 0\\
188 0\\
189 0\\
190 0\\
191 0\\
192 0\\
193 0\\
194 0\\
195 0\\
196 0\\
197 0\\
198 0\\
199 0\\
};
\addplot [
color=green!50!black,
solid,
forget plot
]
table[row sep=crcr]{
1 0\\
2 0\\
3 0\\
4 0\\
5 0\\
6 0\\
7 0\\
8 0\\
9 0\\
10 0\\
11 0\\
12 0\\
13 0\\
14 0\\
15 0\\
16 0\\
17 0\\
18 0\\
19 0\\
20 0\\
21 0\\
22 0\\
23 0\\
24 0\\
25 0\\
26 0\\
27 0\\
28 0\\
29 0\\
30 0\\
31 0\\
32 0\\
33 0\\
34 0\\
35 0\\
36 0\\
37 0\\
38 0\\
39 0\\
40 0\\
41 0\\
42 0\\
43 0\\
44 0\\
45 0\\
46 0\\
47 0\\
48 0\\
49 0\\
50 0\\
51 0\\
52 0\\
53 0\\
54 0\\
55 0\\
56 0\\
57 0\\
58 0\\
59 0\\
60 0\\
61 0\\
62 0\\
63 0\\
64 0\\
65 0\\
66 0\\
67 0\\
68 0\\
69 0\\
70 0\\
71 0\\
72 0\\
73 0\\
74 0\\
75 0\\
76 0\\
77 0\\
78 0\\
79 0\\
80 0\\
81 0\\
82 0\\
83 0\\
84 0\\
85 0\\
86 0\\
87 0\\
88 0\\
89 0\\
90 0\\
91 0\\
92 0\\
93 0\\
94 0\\
95 0\\
96 0\\
97 0\\
98 0\\
99 0\\
100 0\\
101 0\\
102 0\\
103 0\\
104 0\\
105 0\\
106 0\\
107 0\\
108 0\\
109 0\\
110 0\\
111 0\\
112 0\\
113 0\\
114 0\\
115 0\\
116 0\\
117 0\\
118 0\\
119 0\\
120 0\\
121 0\\
122 0\\
123 0\\
124 0\\
125 0\\
126 0\\
127 0\\
128 0\\
129 0\\
130 0\\
131 0\\
132 0\\
133 0\\
134 0\\
135 0\\
136 0\\
137 0\\
138 0\\
139 0\\
140 0\\
141 0\\
142 0\\
143 0\\
144 0\\
145 0\\
146 0\\
147 0\\
148 0\\
149 0\\
150 0\\
151 0\\
152 0\\
153 0\\
154 0\\
155 0\\
156 0\\
157 0\\
158 0\\
159 0\\
160 0\\
161 0\\
162 0\\
163 0\\
164 0\\
165 0\\
166 0\\
167 0\\
168 0\\
169 0\\
170 0\\
171 0\\
172 0\\
173 0\\
174 0\\
175 0\\
176 0\\
177 0\\
178 0\\
179 0\\
180 0\\
181 0\\
182 0\\
183 0\\
184 0\\
185 0\\
186 0\\
187 0\\
188 0\\
189 0\\
190 0\\
191 0\\
192 0\\
193 0\\
194 0\\
195 0\\
196 0\\
197 0\\
198 0\\
199 0\\
};
\addplot [
color=red,
solid,
forget plot
]
table[row sep=crcr]{
1 0\\
2 0\\
3 0\\
4 0\\
5 0\\
6 0\\
7 0\\
8 0\\
9 0\\
10 0\\
11 0\\
12 0\\
13 0\\
14 0\\
15 0\\
16 0\\
17 0\\
18 0\\
19 0\\
20 0\\
21 0\\
22 0\\
23 0\\
24 0\\
25 0\\
26 0\\
27 0\\
28 0\\
29 0\\
30 0\\
31 0\\
32 0\\
33 0\\
34 0\\
35 0\\
36 0\\
37 0\\
38 0\\
39 0\\
40 0\\
41 0\\
42 0\\
43 0\\
44 0\\
45 0\\
46 0\\
47 0\\
48 0\\
49 0\\
50 0\\
51 0\\
52 0\\
53 0\\
54 0\\
55 0\\
56 0\\
57 0\\
58 0\\
59 0\\
60 0\\
61 0\\
62 0\\
63 0\\
64 0\\
65 0\\
66 0\\
67 0\\
68 0\\
69 0\\
70 0\\
71 0\\
72 0\\
73 0\\
74 0\\
75 0\\
76 0\\
77 0\\
78 0\\
79 0\\
80 0\\
81 0\\
82 0\\
83 0\\
84 0\\
85 0\\
86 0\\
87 0\\
88 0\\
89 0\\
90 0\\
91 0\\
92 0\\
93 0\\
94 0\\
95 0\\
96 0\\
97 0\\
98 0\\
99 0\\
100 0\\
101 0\\
102 0\\
103 0\\
104 0\\
105 0\\
106 0\\
107 0\\
108 0\\
109 0\\
110 0\\
111 0\\
112 0\\
113 0\\
114 0\\
115 0\\
116 0\\
117 0\\
118 0\\
119 0\\
120 0\\
121 0\\
122 0\\
123 0\\
124 0\\
125 0\\
126 0\\
127 0\\
128 0\\
129 0\\
130 0\\
131 0\\
132 0\\
133 0\\
134 0\\
135 0\\
136 0\\
137 0\\
138 0\\
139 0\\
140 0\\
141 0\\
142 0\\
143 0\\
144 0\\
145 0\\
146 0\\
147 0\\
148 0\\
149 0\\
150 0\\
151 0\\
152 0\\
153 0\\
154 0\\
155 0\\
156 0\\
157 0\\
158 0\\
159 0\\
160 0\\
161 0\\
162 0\\
163 0\\
164 0\\
165 0\\
166 0\\
167 0\\
168 0\\
169 0\\
170 0\\
171 0\\
172 0\\
173 0\\
174 0\\
175 0\\
176 0\\
177 0\\
178 0\\
179 0\\
180 0\\
181 0\\
182 0\\
183 0\\
184 0\\
185 0\\
186 0\\
187 0\\
188 0\\
189 0\\
190 0\\
191 0\\
192 0\\
193 0\\
194 0\\
195 0\\
196 0\\
197 0\\
198 0\\
199 0\\
};
\addplot [
color=mycolor1,
solid,
forget plot
]
table[row sep=crcr]{
1 1\\
2 -1.5\\
3 -0.5\\
4 -0.5\\
5 1.5\\
6 -1.5\\
7 -0.5\\
8 1.5\\
9 -1.5\\
10 1.5\\
11 -1.5\\
12 -0.5\\
13 1.5\\
14 -1.5\\
15 -0.5\\
16 -0.5\\
17 1.5\\
18 -1.5\\
19 -0.5\\
20 -0.5\\
21 -0.5\\
22 1.5\\
23 -1.5\\
24 1.5\\
25 0.5\\
26 -1.5\\
27 1.5\\
28 0.5\\
29 -1.5\\
30 1.5\\
31 -1.5\\
32 1.5\\
33 -1.5\\
34 -0.5\\
35 1.5\\
36 -1.5\\
37 1.5\\
38 -1.5\\
39 -0.5\\
40 -0.5\\
41 1.5\\
42 -1.5\\
43 -0.5\\
44 1.5\\
45 -1.5\\
46 -0.5\\
47 1.5\\
48 0.5\\
49 -1.5\\
50 -0.5\\
51 -0.5\\
52 1.5\\
53 0.5\\
54 0.5\\
55 -1.5\\
56 1.5\\
57 -1.5\\
58 -0.5\\
59 -0.5\\
60 -0.5\\
61 -0.5\\
62 -0.5\\
63 -0.5\\
64 -0.5\\
65 -0.5\\
66 -0.5\\
67 1.5\\
68 0.5\\
69 0.5\\
70 -1.5\\
71 -0.5\\
72 -0.5\\
73 1.5\\
74 0.5\\
75 0.5\\
76 -1.5\\
77 -0.5\\
78 1.5\\
79 0.5\\
80 -1.5\\
81 -0.5\\
82 -0.5\\
83 -0.5\\
84 1.5\\
85 -1.5\\
86 1.5\\
87 -1.5\\
88 1.5\\
89 -1.5\\
90 -0.5\\
91 -0.5\\
92 1.5\\
93 0.5\\
94 -1.5\\
95 1.5\\
96 -1.5\\
97 1.5\\
98 0.5\\
99 0.5\\
100 -1.5\\
101 0.5\\
102 0\\
103 0\\
104 0\\
105 0\\
106 0\\
107 0\\
108 0\\
109 0\\
110 0\\
111 0\\
112 0\\
113 0\\
114 0\\
115 0\\
116 0\\
117 0\\
118 0\\
119 0\\
120 0\\
121 0\\
122 0\\
123 0\\
124 0\\
125 0\\
126 0\\
127 0\\
128 0\\
129 0\\
130 0\\
131 0\\
132 0\\
133 0\\
134 0\\
135 0\\
136 0\\
137 0\\
138 0\\
139 0\\
140 0\\
141 0\\
142 0\\
143 0\\
144 0\\
145 0\\
146 0\\
147 0\\
148 0\\
149 0\\
150 0\\
151 0\\
152 0\\
153 0\\
154 0\\
155 0\\
156 0\\
157 0\\
158 0\\
159 0\\
160 0\\
161 0\\
162 0\\
163 0\\
164 0\\
165 0\\
166 0\\
167 0\\
168 0\\
169 0\\
170 0\\
171 0\\
172 0\\
173 0\\
174 0\\
175 0\\
176 0\\
177 0\\
178 0\\
179 0\\
180 0\\
181 0\\
182 0\\
183 0\\
184 0\\
185 0\\
186 0\\
187 0\\
188 0\\
189 0\\
190 0\\
191 0\\
192 0\\
193 0\\
194 0\\
195 0\\
196 0\\
197 0\\
198 0\\
199 0\\
};
\end{axis}
\end{tikzpicture}%    	
    	\caption{The cross-correlation of \texttt{x1} with other signals.}
    	\label{fig:cross1}
\end{figure}

\begin{figure}[H]
	\centering
    	\setlength\figureheight{6cm}
    	\setlength\figurewidth{10cm}
    	% This file was created by matlab2tikz v0.4.2.
% Copyright (c) 2008--2013, Nico Schlömer <nico.schloemer@gmail.com>
% All rights reserved.
% 
% The latest updates can be retrieved from
%   http://www.mathworks.com/matlabcentral/fileexchange/22022-matlab2tikz
% where you can also make suggestions and rate matlab2tikz.
% 
% 
% 
%
% defining custom colors
\definecolor{mycolor1}{rgb}{0,0.75,0.75}%
%
\begin{tikzpicture}

\begin{axis}[%
width=\figurewidth,
height=\figureheight,
scale only axis,
xmin=0,
xmax=200,
ymin=-3,
ymax=3,
title={Kruiscorrelatie x2 met:},
axis x line*=bottom,
axis y line*=left,
legend style={draw=black,fill=white,legend cell align=left}
]
\addplot [
color=blue,
solid
]
table[row sep=crcr]{
1 1\\
2 -2.5\\
3 1\\
4 0\\
5 0\\
6 0\\
7 0\\
8 0\\
9 0\\
10 0\\
11 0\\
12 0\\
13 0\\
14 0\\
15 0\\
16 0\\
17 0\\
18 0\\
19 0\\
20 0\\
21 0\\
22 0\\
23 0\\
24 0\\
25 0\\
26 0\\
27 0\\
28 0\\
29 0\\
30 0\\
31 0\\
32 0\\
33 0\\
34 0\\
35 0\\
36 0\\
37 0\\
38 0\\
39 0\\
40 0\\
41 0\\
42 0\\
43 0\\
44 0\\
45 0\\
46 0\\
47 0\\
48 0\\
49 0\\
50 0\\
51 0\\
52 0\\
53 0\\
54 0\\
55 0\\
56 0\\
57 0\\
58 0\\
59 0\\
60 0\\
61 0\\
62 0\\
63 0\\
64 0\\
65 0\\
66 0\\
67 0\\
68 0\\
69 0\\
70 0\\
71 0\\
72 0\\
73 0\\
74 0\\
75 0\\
76 0\\
77 0\\
78 0\\
79 0\\
80 0\\
81 0\\
82 0\\
83 0\\
84 0\\
85 0\\
86 0\\
87 0\\
88 0\\
89 0\\
90 0\\
91 0\\
92 0\\
93 0\\
94 0\\
95 0\\
96 0\\
97 0\\
98 0\\
99 0\\
100 0\\
101 0\\
102 0\\
103 0\\
104 0\\
105 0\\
106 0\\
107 0\\
108 0\\
109 0\\
110 0\\
111 0\\
112 0\\
113 0\\
114 0\\
115 0\\
116 0\\
117 0\\
118 0\\
119 0\\
120 0\\
121 0\\
122 0\\
123 0\\
124 0\\
125 0\\
126 0\\
127 0\\
128 0\\
129 0\\
130 0\\
131 0\\
132 0\\
133 0\\
134 0\\
135 0\\
136 0\\
137 0\\
138 0\\
139 0\\
140 0\\
141 0\\
142 0\\
143 0\\
144 0\\
145 0\\
146 0\\
147 0\\
148 0\\
149 0\\
150 0\\
151 0\\
152 0\\
153 0\\
154 0\\
155 0\\
156 0\\
157 0\\
158 0\\
159 0\\
160 0\\
161 0\\
162 0\\
163 0\\
164 0\\
165 0\\
166 0\\
167 0\\
168 0\\
169 0\\
170 0\\
171 0\\
172 0\\
173 0\\
174 0\\
175 0\\
176 0\\
177 0\\
178 0\\
179 0\\
180 0\\
181 0\\
182 0\\
183 0\\
184 0\\
185 0\\
186 0\\
187 0\\
188 0\\
189 0\\
190 0\\
191 0\\
192 0\\
193 0\\
194 0\\
195 0\\
196 0\\
197 0\\
198 0\\
199 0\\
};
\addlegendentry{1};

\addplot [
color=green!50!black,
solid
]
table[row sep=crcr]{
1 0\\
2 0\\
3 0\\
4 0\\
5 0\\
6 0\\
7 0\\
8 0\\
9 0\\
10 0\\
11 0\\
12 0\\
13 0\\
14 0\\
15 0\\
16 0\\
17 0\\
18 0\\
19 0\\
20 0\\
21 0\\
22 0\\
23 0\\
24 0\\
25 0\\
26 0\\
27 0\\
28 0\\
29 0\\
30 0\\
31 0\\
32 0\\
33 0\\
34 0\\
35 0\\
36 0\\
37 0\\
38 0\\
39 0\\
40 0\\
41 0\\
42 0\\
43 0\\
44 0\\
45 0\\
46 0\\
47 0\\
48 0\\
49 0\\
50 0\\
51 0\\
52 0\\
53 0\\
54 0\\
55 0\\
56 0\\
57 0\\
58 0\\
59 0\\
60 0\\
61 0\\
62 0\\
63 0\\
64 0\\
65 0\\
66 0\\
67 0\\
68 0\\
69 0\\
70 0\\
71 0\\
72 0\\
73 0\\
74 0\\
75 0\\
76 0\\
77 0\\
78 0\\
79 0\\
80 0\\
81 0\\
82 0\\
83 0\\
84 0\\
85 0\\
86 0\\
87 0\\
88 0\\
89 0\\
90 0\\
91 0\\
92 0\\
93 0\\
94 0\\
95 0\\
96 0\\
97 0\\
98 0\\
99 0\\
100 0\\
101 0\\
102 0\\
103 0\\
104 0\\
105 0\\
106 0\\
107 0\\
108 0\\
109 0\\
110 0\\
111 0\\
112 0\\
113 0\\
114 0\\
115 0\\
116 0\\
117 0\\
118 0\\
119 0\\
120 0\\
121 0\\
122 0\\
123 0\\
124 0\\
125 0\\
126 0\\
127 0\\
128 0\\
129 0\\
130 0\\
131 0\\
132 0\\
133 0\\
134 0\\
135 0\\
136 0\\
137 0\\
138 0\\
139 0\\
140 0\\
141 0\\
142 0\\
143 0\\
144 0\\
145 0\\
146 0\\
147 0\\
148 0\\
149 0\\
150 0\\
151 0\\
152 0\\
153 0\\
154 0\\
155 0\\
156 0\\
157 0\\
158 0\\
159 0\\
160 0\\
161 0\\
162 0\\
163 0\\
164 0\\
165 0\\
166 0\\
167 0\\
168 0\\
169 0\\
170 0\\
171 0\\
172 0\\
173 0\\
174 0\\
175 0\\
176 0\\
177 0\\
178 0\\
179 0\\
180 0\\
181 0\\
182 0\\
183 0\\
184 0\\
185 0\\
186 0\\
187 0\\
188 0\\
189 0\\
190 0\\
191 0\\
192 0\\
193 0\\
194 0\\
195 0\\
196 0\\
197 0\\
198 0\\
199 0\\
};
\addlegendentry{3};

\addplot [
color=red,
solid
]
table[row sep=crcr]{
1 0\\
2 0\\
3 0\\
4 0\\
5 0\\
6 0\\
7 0\\
8 0\\
9 0\\
10 0\\
11 0\\
12 0\\
13 0\\
14 0\\
15 0\\
16 0\\
17 0\\
18 0\\
19 0\\
20 0\\
21 0\\
22 0\\
23 0\\
24 0\\
25 0\\
26 0\\
27 0\\
28 0\\
29 0\\
30 0\\
31 0\\
32 0\\
33 0\\
34 0\\
35 0\\
36 0\\
37 0\\
38 0\\
39 0\\
40 0\\
41 0\\
42 0\\
43 0\\
44 0\\
45 0\\
46 0\\
47 0\\
48 0\\
49 0\\
50 0\\
51 0\\
52 0\\
53 0\\
54 0\\
55 0\\
56 0\\
57 0\\
58 0\\
59 0\\
60 0\\
61 0\\
62 0\\
63 0\\
64 0\\
65 0\\
66 0\\
67 0\\
68 0\\
69 0\\
70 0\\
71 0\\
72 0\\
73 0\\
74 0\\
75 0\\
76 0\\
77 0\\
78 0\\
79 0\\
80 0\\
81 0\\
82 0\\
83 0\\
84 0\\
85 0\\
86 0\\
87 0\\
88 0\\
89 0\\
90 0\\
91 0\\
92 0\\
93 0\\
94 0\\
95 0\\
96 0\\
97 0\\
98 0\\
99 0\\
100 0\\
101 0\\
102 0\\
103 0\\
104 0\\
105 0\\
106 0\\
107 0\\
108 0\\
109 0\\
110 0\\
111 0\\
112 0\\
113 0\\
114 0\\
115 0\\
116 0\\
117 0\\
118 0\\
119 0\\
120 0\\
121 0\\
122 0\\
123 0\\
124 0\\
125 0\\
126 0\\
127 0\\
128 0\\
129 0\\
130 0\\
131 0\\
132 0\\
133 0\\
134 0\\
135 0\\
136 0\\
137 0\\
138 0\\
139 0\\
140 0\\
141 0\\
142 0\\
143 0\\
144 0\\
145 0\\
146 0\\
147 0\\
148 0\\
149 0\\
150 0\\
151 0\\
152 0\\
153 0\\
154 0\\
155 0\\
156 0\\
157 0\\
158 0\\
159 0\\
160 0\\
161 0\\
162 0\\
163 0\\
164 0\\
165 0\\
166 0\\
167 0\\
168 0\\
169 0\\
170 0\\
171 0\\
172 0\\
173 0\\
174 0\\
175 0\\
176 0\\
177 0\\
178 0\\
179 0\\
180 0\\
181 0\\
182 0\\
183 0\\
184 0\\
185 0\\
186 0\\
187 0\\
188 0\\
189 0\\
190 0\\
191 0\\
192 0\\
193 0\\
194 0\\
195 0\\
196 0\\
197 0\\
198 0\\
199 0\\
};
\addlegendentry{4};

\addplot [
color=mycolor1,
solid,
forget plot
]
table[row sep=crcr]{
1 0\\
2 0\\
3 0\\
4 0\\
5 0\\
6 0\\
7 0\\
8 0\\
9 0\\
10 0\\
11 0\\
12 0\\
13 0\\
14 0\\
15 0\\
16 0\\
17 0\\
18 0\\
19 0\\
20 0\\
21 0\\
22 0\\
23 0\\
24 0\\
25 0\\
26 0\\
27 0\\
28 0\\
29 0\\
30 0\\
31 0\\
32 0\\
33 0\\
34 0\\
35 0\\
36 0\\
37 0\\
38 0\\
39 0\\
40 0\\
41 0\\
42 0\\
43 0\\
44 0\\
45 0\\
46 0\\
47 0\\
48 0\\
49 0\\
50 0\\
51 0\\
52 0\\
53 0\\
54 0\\
55 0\\
56 0\\
57 0\\
58 0\\
59 0\\
60 0\\
61 0\\
62 0\\
63 0\\
64 0\\
65 0\\
66 0\\
67 0\\
68 0\\
69 0\\
70 0\\
71 0\\
72 0\\
73 0\\
74 0\\
75 0\\
76 0\\
77 0\\
78 0\\
79 0\\
80 0\\
81 0\\
82 0\\
83 0\\
84 0\\
85 0\\
86 0\\
87 0\\
88 0\\
89 0\\
90 0\\
91 0\\
92 0\\
93 0\\
94 0\\
95 0\\
96 0\\
97 0\\
98 0\\
99 0\\
100 0\\
101 0\\
102 0\\
103 0\\
104 0\\
105 0\\
106 0\\
107 0\\
108 0\\
109 0\\
110 0\\
111 0\\
112 0\\
113 0\\
114 0\\
115 0\\
116 0\\
117 0\\
118 0\\
119 0\\
120 0\\
121 0\\
122 0\\
123 0\\
124 0\\
125 0\\
126 0\\
127 0\\
128 0\\
129 0\\
130 0\\
131 0\\
132 0\\
133 0\\
134 0\\
135 0\\
136 0\\
137 0\\
138 0\\
139 0\\
140 0\\
141 0\\
142 0\\
143 0\\
144 0\\
145 0\\
146 0\\
147 0\\
148 0\\
149 0\\
150 0\\
151 0\\
152 0\\
153 0\\
154 0\\
155 0\\
156 0\\
157 0\\
158 0\\
159 0\\
160 0\\
161 0\\
162 0\\
163 0\\
164 0\\
165 0\\
166 0\\
167 0\\
168 0\\
169 0\\
170 0\\
171 0\\
172 0\\
173 0\\
174 0\\
175 0\\
176 0\\
177 0\\
178 0\\
179 0\\
180 0\\
181 0\\
182 0\\
183 0\\
184 0\\
185 0\\
186 0\\
187 0\\
188 0\\
189 0\\
190 0\\
191 0\\
192 0\\
193 0\\
194 0\\
195 0\\
196 0\\
197 0\\
198 0\\
199 0\\
};
\addplot [
color=blue,
solid,
forget plot
]
table[row sep=crcr]{
1 0\\
2 0\\
3 0\\
4 0\\
5 0\\
6 0\\
7 0\\
8 0\\
9 0\\
10 0\\
11 0\\
12 0\\
13 0\\
14 0\\
15 0\\
16 0\\
17 0\\
18 0\\
19 0\\
20 0\\
21 0\\
22 0\\
23 0\\
24 0\\
25 0\\
26 0\\
27 0\\
28 0\\
29 0\\
30 0\\
31 0\\
32 0\\
33 0\\
34 0\\
35 0\\
36 0\\
37 0\\
38 0\\
39 0\\
40 0\\
41 0\\
42 0\\
43 0\\
44 0\\
45 0\\
46 0\\
47 0\\
48 0\\
49 0\\
50 0\\
51 0\\
52 0\\
53 0\\
54 0\\
55 0\\
56 0\\
57 0\\
58 0\\
59 0\\
60 0\\
61 0\\
62 0\\
63 0\\
64 0\\
65 0\\
66 0\\
67 0\\
68 0\\
69 0\\
70 0\\
71 0\\
72 0\\
73 0\\
74 0\\
75 0\\
76 0\\
77 0\\
78 0\\
79 0\\
80 0\\
81 0\\
82 0\\
83 0\\
84 0\\
85 0\\
86 0\\
87 0\\
88 0\\
89 0\\
90 0\\
91 0\\
92 0\\
93 0\\
94 0\\
95 0\\
96 0\\
97 0\\
98 0\\
99 0\\
100 0\\
101 0\\
102 0\\
103 0\\
104 0\\
105 0\\
106 0\\
107 0\\
108 0\\
109 0\\
110 0\\
111 0\\
112 0\\
113 0\\
114 0\\
115 0\\
116 0\\
117 0\\
118 0\\
119 0\\
120 0\\
121 0\\
122 0\\
123 0\\
124 0\\
125 0\\
126 0\\
127 0\\
128 0\\
129 0\\
130 0\\
131 0\\
132 0\\
133 0\\
134 0\\
135 0\\
136 0\\
137 0\\
138 0\\
139 0\\
140 0\\
141 0\\
142 0\\
143 0\\
144 0\\
145 0\\
146 0\\
147 0\\
148 0\\
149 0\\
150 0\\
151 0\\
152 0\\
153 0\\
154 0\\
155 0\\
156 0\\
157 0\\
158 0\\
159 0\\
160 0\\
161 0\\
162 0\\
163 0\\
164 0\\
165 0\\
166 0\\
167 0\\
168 0\\
169 0\\
170 0\\
171 0\\
172 0\\
173 0\\
174 0\\
175 0\\
176 0\\
177 0\\
178 0\\
179 0\\
180 0\\
181 0\\
182 0\\
183 0\\
184 0\\
185 0\\
186 0\\
187 0\\
188 0\\
189 0\\
190 0\\
191 0\\
192 0\\
193 0\\
194 0\\
195 0\\
196 0\\
197 0\\
198 0\\
199 0\\
};
\addplot [
color=green!50!black,
solid,
forget plot
]
table[row sep=crcr]{
1 0\\
2 0\\
3 0\\
4 0\\
5 0\\
6 0\\
7 0\\
8 0\\
9 0\\
10 0\\
11 0\\
12 0\\
13 0\\
14 0\\
15 0\\
16 0\\
17 0\\
18 0\\
19 0\\
20 0\\
21 0\\
22 0\\
23 0\\
24 0\\
25 0\\
26 0\\
27 0\\
28 0\\
29 0\\
30 0\\
31 0\\
32 0\\
33 0\\
34 0\\
35 0\\
36 0\\
37 0\\
38 0\\
39 0\\
40 0\\
41 0\\
42 0\\
43 0\\
44 0\\
45 0\\
46 0\\
47 0\\
48 0\\
49 0\\
50 0\\
51 0\\
52 0\\
53 0\\
54 0\\
55 0\\
56 0\\
57 0\\
58 0\\
59 0\\
60 0\\
61 0\\
62 0\\
63 0\\
64 0\\
65 0\\
66 0\\
67 0\\
68 0\\
69 0\\
70 0\\
71 0\\
72 0\\
73 0\\
74 0\\
75 0\\
76 0\\
77 0\\
78 0\\
79 0\\
80 0\\
81 0\\
82 0\\
83 0\\
84 0\\
85 0\\
86 0\\
87 0\\
88 0\\
89 0\\
90 0\\
91 0\\
92 0\\
93 0\\
94 0\\
95 0\\
96 0\\
97 0\\
98 0\\
99 0\\
100 0\\
101 0\\
102 0\\
103 0\\
104 0\\
105 0\\
106 0\\
107 0\\
108 0\\
109 0\\
110 0\\
111 0\\
112 0\\
113 0\\
114 0\\
115 0\\
116 0\\
117 0\\
118 0\\
119 0\\
120 0\\
121 0\\
122 0\\
123 0\\
124 0\\
125 0\\
126 0\\
127 0\\
128 0\\
129 0\\
130 0\\
131 0\\
132 0\\
133 0\\
134 0\\
135 0\\
136 0\\
137 0\\
138 0\\
139 0\\
140 0\\
141 0\\
142 0\\
143 0\\
144 0\\
145 0\\
146 0\\
147 0\\
148 0\\
149 0\\
150 0\\
151 0\\
152 0\\
153 0\\
154 0\\
155 0\\
156 0\\
157 0\\
158 0\\
159 0\\
160 0\\
161 0\\
162 0\\
163 0\\
164 0\\
165 0\\
166 0\\
167 0\\
168 0\\
169 0\\
170 0\\
171 0\\
172 0\\
173 0\\
174 0\\
175 0\\
176 0\\
177 0\\
178 0\\
179 0\\
180 0\\
181 0\\
182 0\\
183 0\\
184 0\\
185 0\\
186 0\\
187 0\\
188 0\\
189 0\\
190 0\\
191 0\\
192 0\\
193 0\\
194 0\\
195 0\\
196 0\\
197 0\\
198 0\\
199 0\\
};
\addplot [
color=red,
solid,
forget plot
]
table[row sep=crcr]{
1 0.980066577841242\\
2 -1.0390721616796\\
3 -1.01678637309609\\
4 -0.953964520472191\\
5 -0.853111112826191\\
6 -0.718246857259606\\
7 -0.554748366053106\\
8 -0.36913380810177\\
9 -0.168803050090509\\
10 0.0382573528390318\\
11 0.243792555838939\\
12 0.439608518969446\\
13 0.617898677713544\\
14 0.771555166069236\\
15 0.894452184736871\\
16 0.981690217406138\\
17 1.02979135901005\\
18 1.03683796882477\\
19 1.00254912075388\\
20 0.928291802965221\\
21 0.817026420386525\\
22 0.67318877270298\\
23 0.502513213021785\\
24 0.311804037309555\\
25 0.108664218584332\\
26 -0.0988076996260754\\
27 -0.30234046665812\\
28 -0.493819873375019\\
29 -0.665612240079181\\
30 -0.810868747232273\\
31 -0.923798476277515\\
32 -0.999899275288242\\
33 -1.03613724555786\\
34 -1.03106769356723\\
35 -0.984892726356345\\
36 -0.899453194154355\\
37 -0.778155301490119\\
38 -0.625834812566526\\
39 -0.448564264601861\\
40 -0.253410874933911\\
41 -0.0481547933666091\\
42 0.159021067850987\\
43 0.359857260913359\\
44 0.546347080578369\\
45 0.711055766238635\\
46 0.84741690236519\\
47 0.949994200773122\\
48 1.01469822827629\\
49 1.0389494394835\\
50 1.02178101513305\\
51 0.963877406125706\\
52 0.867547046627169\\
53 0.73663032408263\\
54 0.576346475088321\\
55 0.39308551089872\\
56 0.194153467842651\\
57 -0.0125188612896134\\
58 -0.218692102927807\\
59 -0.416146780545111\\
60 -0.597010999249185\\
61 -0.754074273390346\\
62 -0.881074985870412\\
63 -0.972950019056723\\
64 -1.02603660530457\\
65 -1.03821834994467\\
66 -1.00900960525994\\
67 -0.939574831727431\\
68 -0.832682174653789\\
69 -0.692593106957248\\
70 -0.52489253769026\\
71 -0.336266159339751\\
72 -0.134233910365589\\
73 0.0731498210152445\\
74 0.277617299869814\\
75 0.47101705305062\\
76 0.645638842706562\\
77 0.794521049134982\\
78 0.911728208590545\\
79 0.992587641494342\\
80 1.03387573742319\\
81 1.03394647028453\\
82 0.9927970201824\\
83 0.912067885837753\\
84 0.794977483081409\\
85 0.646193836771128\\
86 0.471648481371354\\
87 0.278299989392158\\
88 0.073856555062288\\
89 -0.133531307050071\\
90 -0.335595697332751\\
91 -0.524280946196237\\
92 -0.692064768199076\\
93 -0.83225815183049\\
94 -0.93927202929089\\
95 -1.00884009498775\\
96 -1.03818888967647\\
97 -1.02614836952828\\
98 -0.973198552085428\\
99 -0.881450379476558\\
100 -1.16264362362887\\
101 0\\
102 0\\
103 0\\
104 0\\
105 0\\
106 0\\
107 0\\
108 0\\
109 0\\
110 0\\
111 0\\
112 0\\
113 0\\
114 0\\
115 0\\
116 0\\
117 0\\
118 0\\
119 0\\
120 0\\
121 0\\
122 0\\
123 0\\
124 0\\
125 0\\
126 0\\
127 0\\
128 0\\
129 0\\
130 0\\
131 0\\
132 0\\
133 0\\
134 0\\
135 0\\
136 0\\
137 0\\
138 0\\
139 0\\
140 0\\
141 0\\
142 0\\
143 0\\
144 0\\
145 0\\
146 0\\
147 0\\
148 0\\
149 0\\
150 0\\
151 0\\
152 0\\
153 0\\
154 0\\
155 0\\
156 0\\
157 0\\
158 0\\
159 0\\
160 0\\
161 0\\
162 0\\
163 0\\
164 0\\
165 0\\
166 0\\
167 0\\
168 0\\
169 0\\
170 0\\
171 0\\
172 0\\
173 0\\
174 0\\
175 0\\
176 0\\
177 0\\
178 0\\
179 0\\
180 0\\
181 0\\
182 0\\
183 0\\
184 0\\
185 0\\
186 0\\
187 0\\
188 0\\
189 0\\
190 0\\
191 0\\
192 0\\
193 0\\
194 0\\
195 0\\
196 0\\
197 0\\
198 0\\
199 0\\
};
\addplot [
color=mycolor1,
solid,
forget plot
]
table[row sep=crcr]{
1 0\\
2 0\\
3 0\\
4 0\\
5 0\\
6 0\\
7 0\\
8 0\\
9 0\\
10 0\\
11 0\\
12 0\\
13 0\\
14 0\\
15 0\\
16 0\\
17 0\\
18 0\\
19 0\\
20 0\\
21 0\\
22 0\\
23 0\\
24 0\\
25 0\\
26 0\\
27 0\\
28 0\\
29 0\\
30 0\\
31 0\\
32 0\\
33 0\\
34 0\\
35 0\\
36 0\\
37 0\\
38 0\\
39 0\\
40 0\\
41 0\\
42 0\\
43 0\\
44 0\\
45 0\\
46 0\\
47 0\\
48 0\\
49 0\\
50 0\\
51 0\\
52 0\\
53 0\\
54 0\\
55 0\\
56 0\\
57 0\\
58 0\\
59 0\\
60 0\\
61 0\\
62 0\\
63 0\\
64 0\\
65 0\\
66 0\\
67 0\\
68 0\\
69 0\\
70 0\\
71 0\\
72 0\\
73 0\\
74 0\\
75 0\\
76 0\\
77 0\\
78 0\\
79 0\\
80 0\\
81 0\\
82 0\\
83 0\\
84 0\\
85 0\\
86 0\\
87 0\\
88 0\\
89 0\\
90 0\\
91 0\\
92 0\\
93 0\\
94 0\\
95 0\\
96 0\\
97 0\\
98 0\\
99 0\\
100 0\\
101 0\\
102 0\\
103 0\\
104 0\\
105 0\\
106 0\\
107 0\\
108 0\\
109 0\\
110 0\\
111 0\\
112 0\\
113 0\\
114 0\\
115 0\\
116 0\\
117 0\\
118 0\\
119 0\\
120 0\\
121 0\\
122 0\\
123 0\\
124 0\\
125 0\\
126 0\\
127 0\\
128 0\\
129 0\\
130 0\\
131 0\\
132 0\\
133 0\\
134 0\\
135 0\\
136 0\\
137 0\\
138 0\\
139 0\\
140 0\\
141 0\\
142 0\\
143 0\\
144 0\\
145 0\\
146 0\\
147 0\\
148 0\\
149 0\\
150 0\\
151 0\\
152 0\\
153 0\\
154 0\\
155 0\\
156 0\\
157 0\\
158 0\\
159 0\\
160 0\\
161 0\\
162 0\\
163 0\\
164 0\\
165 0\\
166 0\\
167 0\\
168 0\\
169 0\\
170 0\\
171 0\\
172 0\\
173 0\\
174 0\\
175 0\\
176 0\\
177 0\\
178 0\\
179 0\\
180 0\\
181 0\\
182 0\\
183 0\\
184 0\\
185 0\\
186 0\\
187 0\\
188 0\\
189 0\\
190 0\\
191 0\\
192 0\\
193 0\\
194 0\\
195 0\\
196 0\\
197 0\\
198 0\\
199 0\\
};
\addplot [
color=blue,
solid,
forget plot
]
table[row sep=crcr]{
1 0\\
2 0\\
3 0\\
4 0\\
5 0\\
6 0\\
7 0\\
8 0\\
9 0\\
10 0\\
11 0\\
12 0\\
13 0\\
14 0\\
15 0\\
16 0\\
17 0\\
18 0\\
19 0\\
20 0\\
21 0\\
22 0\\
23 0\\
24 0\\
25 0\\
26 0\\
27 0\\
28 0\\
29 0\\
30 0\\
31 0\\
32 0\\
33 0\\
34 0\\
35 0\\
36 0\\
37 0\\
38 0\\
39 0\\
40 0\\
41 0\\
42 0\\
43 0\\
44 0\\
45 0\\
46 0\\
47 0\\
48 0\\
49 0\\
50 0\\
51 0\\
52 0\\
53 0\\
54 0\\
55 0\\
56 0\\
57 0\\
58 0\\
59 0\\
60 0\\
61 0\\
62 0\\
63 0\\
64 0\\
65 0\\
66 0\\
67 0\\
68 0\\
69 0\\
70 0\\
71 0\\
72 0\\
73 0\\
74 0\\
75 0\\
76 0\\
77 0\\
78 0\\
79 0\\
80 0\\
81 0\\
82 0\\
83 0\\
84 0\\
85 0\\
86 0\\
87 0\\
88 0\\
89 0\\
90 0\\
91 0\\
92 0\\
93 0\\
94 0\\
95 0\\
96 0\\
97 0\\
98 0\\
99 0\\
100 0\\
101 0\\
102 0\\
103 0\\
104 0\\
105 0\\
106 0\\
107 0\\
108 0\\
109 0\\
110 0\\
111 0\\
112 0\\
113 0\\
114 0\\
115 0\\
116 0\\
117 0\\
118 0\\
119 0\\
120 0\\
121 0\\
122 0\\
123 0\\
124 0\\
125 0\\
126 0\\
127 0\\
128 0\\
129 0\\
130 0\\
131 0\\
132 0\\
133 0\\
134 0\\
135 0\\
136 0\\
137 0\\
138 0\\
139 0\\
140 0\\
141 0\\
142 0\\
143 0\\
144 0\\
145 0\\
146 0\\
147 0\\
148 0\\
149 0\\
150 0\\
151 0\\
152 0\\
153 0\\
154 0\\
155 0\\
156 0\\
157 0\\
158 0\\
159 0\\
160 0\\
161 0\\
162 0\\
163 0\\
164 0\\
165 0\\
166 0\\
167 0\\
168 0\\
169 0\\
170 0\\
171 0\\
172 0\\
173 0\\
174 0\\
175 0\\
176 0\\
177 0\\
178 0\\
179 0\\
180 0\\
181 0\\
182 0\\
183 0\\
184 0\\
185 0\\
186 0\\
187 0\\
188 0\\
189 0\\
190 0\\
191 0\\
192 0\\
193 0\\
194 0\\
195 0\\
196 0\\
197 0\\
198 0\\
199 0\\
};
\addplot [
color=green!50!black,
solid,
forget plot
]
table[row sep=crcr]{
1 0\\
2 0\\
3 0\\
4 0\\
5 0\\
6 0\\
7 0\\
8 0\\
9 0\\
10 0\\
11 0\\
12 0\\
13 0\\
14 0\\
15 0\\
16 0\\
17 0\\
18 0\\
19 0\\
20 0\\
21 0\\
22 0\\
23 0\\
24 0\\
25 0\\
26 0\\
27 0\\
28 0\\
29 0\\
30 0\\
31 0\\
32 0\\
33 0\\
34 0\\
35 0\\
36 0\\
37 0\\
38 0\\
39 0\\
40 0\\
41 0\\
42 0\\
43 0\\
44 0\\
45 0\\
46 0\\
47 0\\
48 0\\
49 0\\
50 0\\
51 0\\
52 0\\
53 0\\
54 0\\
55 0\\
56 0\\
57 0\\
58 0\\
59 0\\
60 0\\
61 0\\
62 0\\
63 0\\
64 0\\
65 0\\
66 0\\
67 0\\
68 0\\
69 0\\
70 0\\
71 0\\
72 0\\
73 0\\
74 0\\
75 0\\
76 0\\
77 0\\
78 0\\
79 0\\
80 0\\
81 0\\
82 0\\
83 0\\
84 0\\
85 0\\
86 0\\
87 0\\
88 0\\
89 0\\
90 0\\
91 0\\
92 0\\
93 0\\
94 0\\
95 0\\
96 0\\
97 0\\
98 0\\
99 0\\
100 0\\
101 0\\
102 0\\
103 0\\
104 0\\
105 0\\
106 0\\
107 0\\
108 0\\
109 0\\
110 0\\
111 0\\
112 0\\
113 0\\
114 0\\
115 0\\
116 0\\
117 0\\
118 0\\
119 0\\
120 0\\
121 0\\
122 0\\
123 0\\
124 0\\
125 0\\
126 0\\
127 0\\
128 0\\
129 0\\
130 0\\
131 0\\
132 0\\
133 0\\
134 0\\
135 0\\
136 0\\
137 0\\
138 0\\
139 0\\
140 0\\
141 0\\
142 0\\
143 0\\
144 0\\
145 0\\
146 0\\
147 0\\
148 0\\
149 0\\
150 0\\
151 0\\
152 0\\
153 0\\
154 0\\
155 0\\
156 0\\
157 0\\
158 0\\
159 0\\
160 0\\
161 0\\
162 0\\
163 0\\
164 0\\
165 0\\
166 0\\
167 0\\
168 0\\
169 0\\
170 0\\
171 0\\
172 0\\
173 0\\
174 0\\
175 0\\
176 0\\
177 0\\
178 0\\
179 0\\
180 0\\
181 0\\
182 0\\
183 0\\
184 0\\
185 0\\
186 0\\
187 0\\
188 0\\
189 0\\
190 0\\
191 0\\
192 0\\
193 0\\
194 0\\
195 0\\
196 0\\
197 0\\
198 0\\
199 0\\
};
\addplot [
color=red,
solid,
forget plot
]
table[row sep=crcr]{
1 0\\
2 0\\
3 0\\
4 0\\
5 0\\
6 0\\
7 0\\
8 0\\
9 0\\
10 0\\
11 0\\
12 0\\
13 0\\
14 0\\
15 0\\
16 0\\
17 0\\
18 0\\
19 0\\
20 0\\
21 0\\
22 0\\
23 0\\
24 0\\
25 0\\
26 0\\
27 0\\
28 0\\
29 0\\
30 0\\
31 0\\
32 0\\
33 0\\
34 0\\
35 0\\
36 0\\
37 0\\
38 0\\
39 0\\
40 0\\
41 0\\
42 0\\
43 0\\
44 0\\
45 0\\
46 0\\
47 0\\
48 0\\
49 0\\
50 0\\
51 0\\
52 0\\
53 0\\
54 0\\
55 0\\
56 0\\
57 0\\
58 0\\
59 0\\
60 0\\
61 0\\
62 0\\
63 0\\
64 0\\
65 0\\
66 0\\
67 0\\
68 0\\
69 0\\
70 0\\
71 0\\
72 0\\
73 0\\
74 0\\
75 0\\
76 0\\
77 0\\
78 0\\
79 0\\
80 0\\
81 0\\
82 0\\
83 0\\
84 0\\
85 0\\
86 0\\
87 0\\
88 0\\
89 0\\
90 0\\
91 0\\
92 0\\
93 0\\
94 0\\
95 0\\
96 0\\
97 0\\
98 0\\
99 0\\
100 0\\
101 0\\
102 0\\
103 0\\
104 0\\
105 0\\
106 0\\
107 0\\
108 0\\
109 0\\
110 0\\
111 0\\
112 0\\
113 0\\
114 0\\
115 0\\
116 0\\
117 0\\
118 0\\
119 0\\
120 0\\
121 0\\
122 0\\
123 0\\
124 0\\
125 0\\
126 0\\
127 0\\
128 0\\
129 0\\
130 0\\
131 0\\
132 0\\
133 0\\
134 0\\
135 0\\
136 0\\
137 0\\
138 0\\
139 0\\
140 0\\
141 0\\
142 0\\
143 0\\
144 0\\
145 0\\
146 0\\
147 0\\
148 0\\
149 0\\
150 0\\
151 0\\
152 0\\
153 0\\
154 0\\
155 0\\
156 0\\
157 0\\
158 0\\
159 0\\
160 0\\
161 0\\
162 0\\
163 0\\
164 0\\
165 0\\
166 0\\
167 0\\
168 0\\
169 0\\
170 0\\
171 0\\
172 0\\
173 0\\
174 0\\
175 0\\
176 0\\
177 0\\
178 0\\
179 0\\
180 0\\
181 0\\
182 0\\
183 0\\
184 0\\
185 0\\
186 0\\
187 0\\
188 0\\
189 0\\
190 0\\
191 0\\
192 0\\
193 0\\
194 0\\
195 0\\
196 0\\
197 0\\
198 0\\
199 0\\
};
\addplot [
color=mycolor1,
solid,
forget plot
]
table[row sep=crcr]{
1 1\\
2 -3\\
3 1\\
4 1\\
5 3\\
6 -3\\
7 1\\
8 3\\
9 -3\\
10 3\\
11 -3\\
12 1\\
13 3\\
14 -3\\
15 1\\
16 1\\
17 3\\
18 -3\\
19 1\\
20 1\\
21 1\\
22 3\\
23 -3\\
24 3\\
25 -1\\
26 -3\\
27 3\\
28 -1\\
29 -3\\
30 3\\
31 -3\\
32 3\\
33 -3\\
34 1\\
35 3\\
36 -3\\
37 3\\
38 -3\\
39 1\\
40 1\\
41 3\\
42 -3\\
43 1\\
44 3\\
45 -3\\
46 1\\
47 3\\
48 -1\\
49 -3\\
50 1\\
51 1\\
52 3\\
53 -1\\
54 -1\\
55 -3\\
56 3\\
57 -3\\
58 1\\
59 1\\
60 1\\
61 1\\
62 1\\
63 1\\
64 1\\
65 1\\
66 1\\
67 3\\
68 -1\\
69 -1\\
70 -3\\
71 1\\
72 1\\
73 3\\
74 -1\\
75 -1\\
76 -3\\
77 1\\
78 3\\
79 -1\\
80 -3\\
81 1\\
82 1\\
83 1\\
84 3\\
85 -3\\
86 3\\
87 -3\\
88 3\\
89 -3\\
90 1\\
91 1\\
92 3\\
93 -1\\
94 -3\\
95 3\\
96 -3\\
97 3\\
98 -1\\
99 -1\\
100 -3\\
101 2\\
102 0\\
103 0\\
104 0\\
105 0\\
106 0\\
107 0\\
108 0\\
109 0\\
110 0\\
111 0\\
112 0\\
113 0\\
114 0\\
115 0\\
116 0\\
117 0\\
118 0\\
119 0\\
120 0\\
121 0\\
122 0\\
123 0\\
124 0\\
125 0\\
126 0\\
127 0\\
128 0\\
129 0\\
130 0\\
131 0\\
132 0\\
133 0\\
134 0\\
135 0\\
136 0\\
137 0\\
138 0\\
139 0\\
140 0\\
141 0\\
142 0\\
143 0\\
144 0\\
145 0\\
146 0\\
147 0\\
148 0\\
149 0\\
150 0\\
151 0\\
152 0\\
153 0\\
154 0\\
155 0\\
156 0\\
157 0\\
158 0\\
159 0\\
160 0\\
161 0\\
162 0\\
163 0\\
164 0\\
165 0\\
166 0\\
167 0\\
168 0\\
169 0\\
170 0\\
171 0\\
172 0\\
173 0\\
174 0\\
175 0\\
176 0\\
177 0\\
178 0\\
179 0\\
180 0\\
181 0\\
182 0\\
183 0\\
184 0\\
185 0\\
186 0\\
187 0\\
188 0\\
189 0\\
190 0\\
191 0\\
192 0\\
193 0\\
194 0\\
195 0\\
196 0\\
197 0\\
198 0\\
199 0\\
};
\end{axis}
\end{tikzpicture}%    	
    	\caption{The cross-correlation of \texttt{x2} with other signals.}
    	\label{fig:cross2}
\end{figure}

\begin{figure}[H]
	\centering
    	\setlength\figureheight{6cm}
    	\setlength\figurewidth{10cm}
    	\input{resources/crossx3.tikz}    	
    	\caption{The cross-correlation of \texttt{x3} with other signals.}
    	\label{fig:cross3}
\end{figure}

\begin{figure}[H]
	\centering
    	\setlength\figureheight{6cm}
    	\setlength\figurewidth{10cm}
    	% This file was created by matlab2tikz v0.4.2.
% Copyright (c) 2008--2013, Nico Schlömer <nico.schloemer@gmail.com>
% All rights reserved.
% 
% The latest updates can be retrieved from
%   http://www.mathworks.com/matlabcentral/fileexchange/22022-matlab2tikz
% where you can also make suggestions and rate matlab2tikz.
% 
% 
% 
%
% defining custom colors
\definecolor{mycolor1}{rgb}{0,0.75,0.75}%
%
\begin{tikzpicture}

\begin{axis}[%
width=\figurewidth,
height=\figureheight,
scale only axis,
xmin=0,
xmax=200,
ymin=-8,
ymax=10,
title={Cross correlation x4 with:},
axis x line*=bottom,
axis y line*=left,
legend style={draw=black,fill=white,legend cell align=left}
]
\addplot [
color=blue,
solid
]
table[row sep=crcr]{
1 1\\
2 -1.5\\
3 -0.5\\
4 -0.5\\
5 1.5\\
6 -1.5\\
7 -0.5\\
8 1.5\\
9 -1.5\\
10 1.5\\
11 -1.5\\
12 -0.5\\
13 1.5\\
14 -1.5\\
15 -0.5\\
16 -0.5\\
17 1.5\\
18 -1.5\\
19 -0.5\\
20 -0.5\\
21 -0.5\\
22 1.5\\
23 -1.5\\
24 1.5\\
25 0.5\\
26 -1.5\\
27 1.5\\
28 0.5\\
29 -1.5\\
30 1.5\\
31 -1.5\\
32 1.5\\
33 -1.5\\
34 -0.5\\
35 1.5\\
36 -1.5\\
37 1.5\\
38 -1.5\\
39 -0.5\\
40 -0.5\\
41 1.5\\
42 -1.5\\
43 -0.5\\
44 1.5\\
45 -1.5\\
46 -0.5\\
47 1.5\\
48 0.5\\
49 -1.5\\
50 -0.5\\
51 -0.5\\
52 1.5\\
53 0.5\\
54 0.5\\
55 -1.5\\
56 1.5\\
57 -1.5\\
58 -0.5\\
59 -0.5\\
60 -0.5\\
61 -0.5\\
62 -0.5\\
63 -0.5\\
64 -0.5\\
65 -0.5\\
66 -0.5\\
67 1.5\\
68 0.5\\
69 0.5\\
70 -1.5\\
71 -0.5\\
72 -0.5\\
73 1.5\\
74 0.5\\
75 0.5\\
76 -1.5\\
77 -0.5\\
78 1.5\\
79 0.5\\
80 -1.5\\
81 -0.5\\
82 -0.5\\
83 -0.5\\
84 1.5\\
85 -1.5\\
86 1.5\\
87 -1.5\\
88 1.5\\
89 -1.5\\
90 -0.5\\
91 -0.5\\
92 1.5\\
93 0.5\\
94 -1.5\\
95 1.5\\
96 -1.5\\
97 1.5\\
98 0.5\\
99 0.5\\
100 -1.5\\
101 0.5\\
102 0\\
103 0\\
104 0\\
105 0\\
106 0\\
107 0\\
108 0\\
109 0\\
110 0\\
111 0\\
112 0\\
113 0\\
114 0\\
115 0\\
116 0\\
117 0\\
118 0\\
119 0\\
120 0\\
121 0\\
122 0\\
123 0\\
124 0\\
125 0\\
126 0\\
127 0\\
128 0\\
129 0\\
130 0\\
131 0\\
132 0\\
133 0\\
134 0\\
135 0\\
136 0\\
137 0\\
138 0\\
139 0\\
140 0\\
141 0\\
142 0\\
143 0\\
144 0\\
145 0\\
146 0\\
147 0\\
148 0\\
149 0\\
150 0\\
151 0\\
152 0\\
153 0\\
154 0\\
155 0\\
156 0\\
157 0\\
158 0\\
159 0\\
160 0\\
161 0\\
162 0\\
163 0\\
164 0\\
165 0\\
166 0\\
167 0\\
168 0\\
169 0\\
170 0\\
171 0\\
172 0\\
173 0\\
174 0\\
175 0\\
176 0\\
177 0\\
178 0\\
179 0\\
180 0\\
181 0\\
182 0\\
183 0\\
184 0\\
185 0\\
186 0\\
187 0\\
188 0\\
189 0\\
190 0\\
191 0\\
192 0\\
193 0\\
194 0\\
195 0\\
196 0\\
197 0\\
198 0\\
199 0\\
};
\addlegendentry{1};

\addplot [
color=green!50!black,
solid
]
table[row sep=crcr]{
1 0\\
2 0\\
3 0\\
4 0\\
5 0\\
6 0\\
7 0\\
8 0\\
9 0\\
10 0\\
11 0\\
12 0\\
13 0\\
14 0\\
15 0\\
16 0\\
17 0\\
18 0\\
19 0\\
20 0\\
21 0\\
22 0\\
23 0\\
24 0\\
25 0\\
26 0\\
27 0\\
28 0\\
29 0\\
30 0\\
31 0\\
32 0\\
33 0\\
34 0\\
35 0\\
36 0\\
37 0\\
38 0\\
39 0\\
40 0\\
41 0\\
42 0\\
43 0\\
44 0\\
45 0\\
46 0\\
47 0\\
48 0\\
49 0\\
50 0\\
51 0\\
52 0\\
53 0\\
54 0\\
55 0\\
56 0\\
57 0\\
58 0\\
59 0\\
60 0\\
61 0\\
62 0\\
63 0\\
64 0\\
65 0\\
66 0\\
67 0\\
68 0\\
69 0\\
70 0\\
71 0\\
72 0\\
73 0\\
74 0\\
75 0\\
76 0\\
77 0\\
78 0\\
79 0\\
80 0\\
81 0\\
82 0\\
83 0\\
84 0\\
85 0\\
86 0\\
87 0\\
88 0\\
89 0\\
90 0\\
91 0\\
92 0\\
93 0\\
94 0\\
95 0\\
96 0\\
97 0\\
98 0\\
99 0\\
100 0\\
101 0\\
102 0\\
103 0\\
104 0\\
105 0\\
106 0\\
107 0\\
108 0\\
109 0\\
110 0\\
111 0\\
112 0\\
113 0\\
114 0\\
115 0\\
116 0\\
117 0\\
118 0\\
119 0\\
120 0\\
121 0\\
122 0\\
123 0\\
124 0\\
125 0\\
126 0\\
127 0\\
128 0\\
129 0\\
130 0\\
131 0\\
132 0\\
133 0\\
134 0\\
135 0\\
136 0\\
137 0\\
138 0\\
139 0\\
140 0\\
141 0\\
142 0\\
143 0\\
144 0\\
145 0\\
146 0\\
147 0\\
148 0\\
149 0\\
150 0\\
151 0\\
152 0\\
153 0\\
154 0\\
155 0\\
156 0\\
157 0\\
158 0\\
159 0\\
160 0\\
161 0\\
162 0\\
163 0\\
164 0\\
165 0\\
166 0\\
167 0\\
168 0\\
169 0\\
170 0\\
171 0\\
172 0\\
173 0\\
174 0\\
175 0\\
176 0\\
177 0\\
178 0\\
179 0\\
180 0\\
181 0\\
182 0\\
183 0\\
184 0\\
185 0\\
186 0\\
187 0\\
188 0\\
189 0\\
190 0\\
191 0\\
192 0\\
193 0\\
194 0\\
195 0\\
196 0\\
197 0\\
198 0\\
199 0\\
};
\addlegendentry{2};

\addplot [
color=red,
solid
]
table[row sep=crcr]{
1 0\\
2 0\\
3 0\\
4 0\\
5 0\\
6 0\\
7 0\\
8 0\\
9 0\\
10 0\\
11 0\\
12 0\\
13 0\\
14 0\\
15 0\\
16 0\\
17 0\\
18 0\\
19 0\\
20 0\\
21 0\\
22 0\\
23 0\\
24 0\\
25 0\\
26 0\\
27 0\\
28 0\\
29 0\\
30 0\\
31 0\\
32 0\\
33 0\\
34 0\\
35 0\\
36 0\\
37 0\\
38 0\\
39 0\\
40 0\\
41 0\\
42 0\\
43 0\\
44 0\\
45 0\\
46 0\\
47 0\\
48 0\\
49 0\\
50 0\\
51 0\\
52 0\\
53 0\\
54 0\\
55 0\\
56 0\\
57 0\\
58 0\\
59 0\\
60 0\\
61 0\\
62 0\\
63 0\\
64 0\\
65 0\\
66 0\\
67 0\\
68 0\\
69 0\\
70 0\\
71 0\\
72 0\\
73 0\\
74 0\\
75 0\\
76 0\\
77 0\\
78 0\\
79 0\\
80 0\\
81 0\\
82 0\\
83 0\\
84 0\\
85 0\\
86 0\\
87 0\\
88 0\\
89 0\\
90 0\\
91 0\\
92 0\\
93 0\\
94 0\\
95 0\\
96 0\\
97 0\\
98 0\\
99 0\\
100 0\\
101 0\\
102 0\\
103 0\\
104 0\\
105 0\\
106 0\\
107 0\\
108 0\\
109 0\\
110 0\\
111 0\\
112 0\\
113 0\\
114 0\\
115 0\\
116 0\\
117 0\\
118 0\\
119 0\\
120 0\\
121 0\\
122 0\\
123 0\\
124 0\\
125 0\\
126 0\\
127 0\\
128 0\\
129 0\\
130 0\\
131 0\\
132 0\\
133 0\\
134 0\\
135 0\\
136 0\\
137 0\\
138 0\\
139 0\\
140 0\\
141 0\\
142 0\\
143 0\\
144 0\\
145 0\\
146 0\\
147 0\\
148 0\\
149 0\\
150 0\\
151 0\\
152 0\\
153 0\\
154 0\\
155 0\\
156 0\\
157 0\\
158 0\\
159 0\\
160 0\\
161 0\\
162 0\\
163 0\\
164 0\\
165 0\\
166 0\\
167 0\\
168 0\\
169 0\\
170 0\\
171 0\\
172 0\\
173 0\\
174 0\\
175 0\\
176 0\\
177 0\\
178 0\\
179 0\\
180 0\\
181 0\\
182 0\\
183 0\\
184 0\\
185 0\\
186 0\\
187 0\\
188 0\\
189 0\\
190 0\\
191 0\\
192 0\\
193 0\\
194 0\\
195 0\\
196 0\\
197 0\\
198 0\\
199 0\\
};
\addlegendentry{3};

\addplot [
color=mycolor1,
solid,
forget plot
]
table[row sep=crcr]{
1 0\\
2 0\\
3 0\\
4 0\\
5 0\\
6 0\\
7 0\\
8 0\\
9 0\\
10 0\\
11 0\\
12 0\\
13 0\\
14 0\\
15 0\\
16 0\\
17 0\\
18 0\\
19 0\\
20 0\\
21 0\\
22 0\\
23 0\\
24 0\\
25 0\\
26 0\\
27 0\\
28 0\\
29 0\\
30 0\\
31 0\\
32 0\\
33 0\\
34 0\\
35 0\\
36 0\\
37 0\\
38 0\\
39 0\\
40 0\\
41 0\\
42 0\\
43 0\\
44 0\\
45 0\\
46 0\\
47 0\\
48 0\\
49 0\\
50 0\\
51 0\\
52 0\\
53 0\\
54 0\\
55 0\\
56 0\\
57 0\\
58 0\\
59 0\\
60 0\\
61 0\\
62 0\\
63 0\\
64 0\\
65 0\\
66 0\\
67 0\\
68 0\\
69 0\\
70 0\\
71 0\\
72 0\\
73 0\\
74 0\\
75 0\\
76 0\\
77 0\\
78 0\\
79 0\\
80 0\\
81 0\\
82 0\\
83 0\\
84 0\\
85 0\\
86 0\\
87 0\\
88 0\\
89 0\\
90 0\\
91 0\\
92 0\\
93 0\\
94 0\\
95 0\\
96 0\\
97 0\\
98 0\\
99 0\\
100 0\\
101 0\\
102 0\\
103 0\\
104 0\\
105 0\\
106 0\\
107 0\\
108 0\\
109 0\\
110 0\\
111 0\\
112 0\\
113 0\\
114 0\\
115 0\\
116 0\\
117 0\\
118 0\\
119 0\\
120 0\\
121 0\\
122 0\\
123 0\\
124 0\\
125 0\\
126 0\\
127 0\\
128 0\\
129 0\\
130 0\\
131 0\\
132 0\\
133 0\\
134 0\\
135 0\\
136 0\\
137 0\\
138 0\\
139 0\\
140 0\\
141 0\\
142 0\\
143 0\\
144 0\\
145 0\\
146 0\\
147 0\\
148 0\\
149 0\\
150 0\\
151 0\\
152 0\\
153 0\\
154 0\\
155 0\\
156 0\\
157 0\\
158 0\\
159 0\\
160 0\\
161 0\\
162 0\\
163 0\\
164 0\\
165 0\\
166 0\\
167 0\\
168 0\\
169 0\\
170 0\\
171 0\\
172 0\\
173 0\\
174 0\\
175 0\\
176 0\\
177 0\\
178 0\\
179 0\\
180 0\\
181 0\\
182 0\\
183 0\\
184 0\\
185 0\\
186 0\\
187 0\\
188 0\\
189 0\\
190 0\\
191 0\\
192 0\\
193 0\\
194 0\\
195 0\\
196 0\\
197 0\\
198 0\\
199 0\\
};
\addplot [
color=blue,
solid,
forget plot
]
table[row sep=crcr]{
1 0\\
2 0\\
3 0\\
4 0\\
5 0\\
6 0\\
7 0\\
8 0\\
9 0\\
10 0\\
11 0\\
12 0\\
13 0\\
14 0\\
15 0\\
16 0\\
17 0\\
18 0\\
19 0\\
20 0\\
21 0\\
22 0\\
23 0\\
24 0\\
25 0\\
26 0\\
27 0\\
28 0\\
29 0\\
30 0\\
31 0\\
32 0\\
33 0\\
34 0\\
35 0\\
36 0\\
37 0\\
38 0\\
39 0\\
40 0\\
41 0\\
42 0\\
43 0\\
44 0\\
45 0\\
46 0\\
47 0\\
48 0\\
49 0\\
50 0\\
51 0\\
52 0\\
53 0\\
54 0\\
55 0\\
56 0\\
57 0\\
58 0\\
59 0\\
60 0\\
61 0\\
62 0\\
63 0\\
64 0\\
65 0\\
66 0\\
67 0\\
68 0\\
69 0\\
70 0\\
71 0\\
72 0\\
73 0\\
74 0\\
75 0\\
76 0\\
77 0\\
78 0\\
79 0\\
80 0\\
81 0\\
82 0\\
83 0\\
84 0\\
85 0\\
86 0\\
87 0\\
88 0\\
89 0\\
90 0\\
91 0\\
92 0\\
93 0\\
94 0\\
95 0\\
96 0\\
97 0\\
98 0\\
99 0\\
100 0\\
101 0\\
102 0\\
103 0\\
104 0\\
105 0\\
106 0\\
107 0\\
108 0\\
109 0\\
110 0\\
111 0\\
112 0\\
113 0\\
114 0\\
115 0\\
116 0\\
117 0\\
118 0\\
119 0\\
120 0\\
121 0\\
122 0\\
123 0\\
124 0\\
125 0\\
126 0\\
127 0\\
128 0\\
129 0\\
130 0\\
131 0\\
132 0\\
133 0\\
134 0\\
135 0\\
136 0\\
137 0\\
138 0\\
139 0\\
140 0\\
141 0\\
142 0\\
143 0\\
144 0\\
145 0\\
146 0\\
147 0\\
148 0\\
149 0\\
150 0\\
151 0\\
152 0\\
153 0\\
154 0\\
155 0\\
156 0\\
157 0\\
158 0\\
159 0\\
160 0\\
161 0\\
162 0\\
163 0\\
164 0\\
165 0\\
166 0\\
167 0\\
168 0\\
169 0\\
170 0\\
171 0\\
172 0\\
173 0\\
174 0\\
175 0\\
176 0\\
177 0\\
178 0\\
179 0\\
180 0\\
181 0\\
182 0\\
183 0\\
184 0\\
185 0\\
186 0\\
187 0\\
188 0\\
189 0\\
190 0\\
191 0\\
192 0\\
193 0\\
194 0\\
195 0\\
196 0\\
197 0\\
198 0\\
199 0\\
};
\addplot [
color=green!50!black,
solid,
forget plot
]
table[row sep=crcr]{
1 1\\
2 -3\\
3 1\\
4 1\\
5 3\\
6 -3\\
7 1\\
8 3\\
9 -3\\
10 3\\
11 -3\\
12 1\\
13 3\\
14 -3\\
15 1\\
16 1\\
17 3\\
18 -3\\
19 1\\
20 1\\
21 1\\
22 3\\
23 -3\\
24 3\\
25 -1\\
26 -3\\
27 3\\
28 -1\\
29 -3\\
30 3\\
31 -3\\
32 3\\
33 -3\\
34 1\\
35 3\\
36 -3\\
37 3\\
38 -3\\
39 1\\
40 1\\
41 3\\
42 -3\\
43 1\\
44 3\\
45 -3\\
46 1\\
47 3\\
48 -1\\
49 -3\\
50 1\\
51 1\\
52 3\\
53 -1\\
54 -1\\
55 -3\\
56 3\\
57 -3\\
58 1\\
59 1\\
60 1\\
61 1\\
62 1\\
63 1\\
64 1\\
65 1\\
66 1\\
67 3\\
68 -1\\
69 -1\\
70 -3\\
71 1\\
72 1\\
73 3\\
74 -1\\
75 -1\\
76 -3\\
77 1\\
78 3\\
79 -1\\
80 -3\\
81 1\\
82 1\\
83 1\\
84 3\\
85 -3\\
86 3\\
87 -3\\
88 3\\
89 -3\\
90 1\\
91 1\\
92 3\\
93 -1\\
94 -3\\
95 3\\
96 -3\\
97 3\\
98 -1\\
99 -1\\
100 -3\\
101 2\\
102 0\\
103 0\\
104 0\\
105 0\\
106 0\\
107 0\\
108 0\\
109 0\\
110 0\\
111 0\\
112 0\\
113 0\\
114 0\\
115 0\\
116 0\\
117 0\\
118 0\\
119 0\\
120 0\\
121 0\\
122 0\\
123 0\\
124 0\\
125 0\\
126 0\\
127 0\\
128 0\\
129 0\\
130 0\\
131 0\\
132 0\\
133 0\\
134 0\\
135 0\\
136 0\\
137 0\\
138 0\\
139 0\\
140 0\\
141 0\\
142 0\\
143 0\\
144 0\\
145 0\\
146 0\\
147 0\\
148 0\\
149 0\\
150 0\\
151 0\\
152 0\\
153 0\\
154 0\\
155 0\\
156 0\\
157 0\\
158 0\\
159 0\\
160 0\\
161 0\\
162 0\\
163 0\\
164 0\\
165 0\\
166 0\\
167 0\\
168 0\\
169 0\\
170 0\\
171 0\\
172 0\\
173 0\\
174 0\\
175 0\\
176 0\\
177 0\\
178 0\\
179 0\\
180 0\\
181 0\\
182 0\\
183 0\\
184 0\\
185 0\\
186 0\\
187 0\\
188 0\\
189 0\\
190 0\\
191 0\\
192 0\\
193 0\\
194 0\\
195 0\\
196 0\\
197 0\\
198 0\\
199 0\\
};
\addplot [
color=red,
solid,
forget plot
]
table[row sep=crcr]{
1 0\\
2 0\\
3 0\\
4 0\\
5 0\\
6 0\\
7 0\\
8 0\\
9 0\\
10 0\\
11 0\\
12 0\\
13 0\\
14 0\\
15 0\\
16 0\\
17 0\\
18 0\\
19 0\\
20 0\\
21 0\\
22 0\\
23 0\\
24 0\\
25 0\\
26 0\\
27 0\\
28 0\\
29 0\\
30 0\\
31 0\\
32 0\\
33 0\\
34 0\\
35 0\\
36 0\\
37 0\\
38 0\\
39 0\\
40 0\\
41 0\\
42 0\\
43 0\\
44 0\\
45 0\\
46 0\\
47 0\\
48 0\\
49 0\\
50 0\\
51 0\\
52 0\\
53 0\\
54 0\\
55 0\\
56 0\\
57 0\\
58 0\\
59 0\\
60 0\\
61 0\\
62 0\\
63 0\\
64 0\\
65 0\\
66 0\\
67 0\\
68 0\\
69 0\\
70 0\\
71 0\\
72 0\\
73 0\\
74 0\\
75 0\\
76 0\\
77 0\\
78 0\\
79 0\\
80 0\\
81 0\\
82 0\\
83 0\\
84 0\\
85 0\\
86 0\\
87 0\\
88 0\\
89 0\\
90 0\\
91 0\\
92 0\\
93 0\\
94 0\\
95 0\\
96 0\\
97 0\\
98 0\\
99 0\\
100 0\\
101 0\\
102 0\\
103 0\\
104 0\\
105 0\\
106 0\\
107 0\\
108 0\\
109 0\\
110 0\\
111 0\\
112 0\\
113 0\\
114 0\\
115 0\\
116 0\\
117 0\\
118 0\\
119 0\\
120 0\\
121 0\\
122 0\\
123 0\\
124 0\\
125 0\\
126 0\\
127 0\\
128 0\\
129 0\\
130 0\\
131 0\\
132 0\\
133 0\\
134 0\\
135 0\\
136 0\\
137 0\\
138 0\\
139 0\\
140 0\\
141 0\\
142 0\\
143 0\\
144 0\\
145 0\\
146 0\\
147 0\\
148 0\\
149 0\\
150 0\\
151 0\\
152 0\\
153 0\\
154 0\\
155 0\\
156 0\\
157 0\\
158 0\\
159 0\\
160 0\\
161 0\\
162 0\\
163 0\\
164 0\\
165 0\\
166 0\\
167 0\\
168 0\\
169 0\\
170 0\\
171 0\\
172 0\\
173 0\\
174 0\\
175 0\\
176 0\\
177 0\\
178 0\\
179 0\\
180 0\\
181 0\\
182 0\\
183 0\\
184 0\\
185 0\\
186 0\\
187 0\\
188 0\\
189 0\\
190 0\\
191 0\\
192 0\\
193 0\\
194 0\\
195 0\\
196 0\\
197 0\\
198 0\\
199 0\\
};
\addplot [
color=mycolor1,
solid,
forget plot
]
table[row sep=crcr]{
1 0\\
2 0\\
3 0\\
4 0\\
5 0\\
6 0\\
7 0\\
8 0\\
9 0\\
10 0\\
11 0\\
12 0\\
13 0\\
14 0\\
15 0\\
16 0\\
17 0\\
18 0\\
19 0\\
20 0\\
21 0\\
22 0\\
23 0\\
24 0\\
25 0\\
26 0\\
27 0\\
28 0\\
29 0\\
30 0\\
31 0\\
32 0\\
33 0\\
34 0\\
35 0\\
36 0\\
37 0\\
38 0\\
39 0\\
40 0\\
41 0\\
42 0\\
43 0\\
44 0\\
45 0\\
46 0\\
47 0\\
48 0\\
49 0\\
50 0\\
51 0\\
52 0\\
53 0\\
54 0\\
55 0\\
56 0\\
57 0\\
58 0\\
59 0\\
60 0\\
61 0\\
62 0\\
63 0\\
64 0\\
65 0\\
66 0\\
67 0\\
68 0\\
69 0\\
70 0\\
71 0\\
72 0\\
73 0\\
74 0\\
75 0\\
76 0\\
77 0\\
78 0\\
79 0\\
80 0\\
81 0\\
82 0\\
83 0\\
84 0\\
85 0\\
86 0\\
87 0\\
88 0\\
89 0\\
90 0\\
91 0\\
92 0\\
93 0\\
94 0\\
95 0\\
96 0\\
97 0\\
98 0\\
99 0\\
100 0\\
101 0\\
102 0\\
103 0\\
104 0\\
105 0\\
106 0\\
107 0\\
108 0\\
109 0\\
110 0\\
111 0\\
112 0\\
113 0\\
114 0\\
115 0\\
116 0\\
117 0\\
118 0\\
119 0\\
120 0\\
121 0\\
122 0\\
123 0\\
124 0\\
125 0\\
126 0\\
127 0\\
128 0\\
129 0\\
130 0\\
131 0\\
132 0\\
133 0\\
134 0\\
135 0\\
136 0\\
137 0\\
138 0\\
139 0\\
140 0\\
141 0\\
142 0\\
143 0\\
144 0\\
145 0\\
146 0\\
147 0\\
148 0\\
149 0\\
150 0\\
151 0\\
152 0\\
153 0\\
154 0\\
155 0\\
156 0\\
157 0\\
158 0\\
159 0\\
160 0\\
161 0\\
162 0\\
163 0\\
164 0\\
165 0\\
166 0\\
167 0\\
168 0\\
169 0\\
170 0\\
171 0\\
172 0\\
173 0\\
174 0\\
175 0\\
176 0\\
177 0\\
178 0\\
179 0\\
180 0\\
181 0\\
182 0\\
183 0\\
184 0\\
185 0\\
186 0\\
187 0\\
188 0\\
189 0\\
190 0\\
191 0\\
192 0\\
193 0\\
194 0\\
195 0\\
196 0\\
197 0\\
198 0\\
199 0\\
};
\addplot [
color=blue,
solid,
forget plot
]
table[row sep=crcr]{
1 0\\
2 0\\
3 0\\
4 0\\
5 0\\
6 0\\
7 0\\
8 0\\
9 0\\
10 0\\
11 0\\
12 0\\
13 0\\
14 0\\
15 0\\
16 0\\
17 0\\
18 0\\
19 0\\
20 0\\
21 0\\
22 0\\
23 0\\
24 0\\
25 0\\
26 0\\
27 0\\
28 0\\
29 0\\
30 0\\
31 0\\
32 0\\
33 0\\
34 0\\
35 0\\
36 0\\
37 0\\
38 0\\
39 0\\
40 0\\
41 0\\
42 0\\
43 0\\
44 0\\
45 0\\
46 0\\
47 0\\
48 0\\
49 0\\
50 0\\
51 0\\
52 0\\
53 0\\
54 0\\
55 0\\
56 0\\
57 0\\
58 0\\
59 0\\
60 0\\
61 0\\
62 0\\
63 0\\
64 0\\
65 0\\
66 0\\
67 0\\
68 0\\
69 0\\
70 0\\
71 0\\
72 0\\
73 0\\
74 0\\
75 0\\
76 0\\
77 0\\
78 0\\
79 0\\
80 0\\
81 0\\
82 0\\
83 0\\
84 0\\
85 0\\
86 0\\
87 0\\
88 0\\
89 0\\
90 0\\
91 0\\
92 0\\
93 0\\
94 0\\
95 0\\
96 0\\
97 0\\
98 0\\
99 0\\
100 0\\
101 0\\
102 0\\
103 0\\
104 0\\
105 0\\
106 0\\
107 0\\
108 0\\
109 0\\
110 0\\
111 0\\
112 0\\
113 0\\
114 0\\
115 0\\
116 0\\
117 0\\
118 0\\
119 0\\
120 0\\
121 0\\
122 0\\
123 0\\
124 0\\
125 0\\
126 0\\
127 0\\
128 0\\
129 0\\
130 0\\
131 0\\
132 0\\
133 0\\
134 0\\
135 0\\
136 0\\
137 0\\
138 0\\
139 0\\
140 0\\
141 0\\
142 0\\
143 0\\
144 0\\
145 0\\
146 0\\
147 0\\
148 0\\
149 0\\
150 0\\
151 0\\
152 0\\
153 0\\
154 0\\
155 0\\
156 0\\
157 0\\
158 0\\
159 0\\
160 0\\
161 0\\
162 0\\
163 0\\
164 0\\
165 0\\
166 0\\
167 0\\
168 0\\
169 0\\
170 0\\
171 0\\
172 0\\
173 0\\
174 0\\
175 0\\
176 0\\
177 0\\
178 0\\
179 0\\
180 0\\
181 0\\
182 0\\
183 0\\
184 0\\
185 0\\
186 0\\
187 0\\
188 0\\
189 0\\
190 0\\
191 0\\
192 0\\
193 0\\
194 0\\
195 0\\
196 0\\
197 0\\
198 0\\
199 0\\
};
\addplot [
color=green!50!black,
solid,
forget plot
]
table[row sep=crcr]{
1 0\\
2 0\\
3 0\\
4 0\\
5 0\\
6 0\\
7 0\\
8 0\\
9 0\\
10 0\\
11 0\\
12 0\\
13 0\\
14 0\\
15 0\\
16 0\\
17 0\\
18 0\\
19 0\\
20 0\\
21 0\\
22 0\\
23 0\\
24 0\\
25 0\\
26 0\\
27 0\\
28 0\\
29 0\\
30 0\\
31 0\\
32 0\\
33 0\\
34 0\\
35 0\\
36 0\\
37 0\\
38 0\\
39 0\\
40 0\\
41 0\\
42 0\\
43 0\\
44 0\\
45 0\\
46 0\\
47 0\\
48 0\\
49 0\\
50 0\\
51 0\\
52 0\\
53 0\\
54 0\\
55 0\\
56 0\\
57 0\\
58 0\\
59 0\\
60 0\\
61 0\\
62 0\\
63 0\\
64 0\\
65 0\\
66 0\\
67 0\\
68 0\\
69 0\\
70 0\\
71 0\\
72 0\\
73 0\\
74 0\\
75 0\\
76 0\\
77 0\\
78 0\\
79 0\\
80 0\\
81 0\\
82 0\\
83 0\\
84 0\\
85 0\\
86 0\\
87 0\\
88 0\\
89 0\\
90 0\\
91 0\\
92 0\\
93 0\\
94 0\\
95 0\\
96 0\\
97 0\\
98 0\\
99 0\\
100 0\\
101 0\\
102 0\\
103 0\\
104 0\\
105 0\\
106 0\\
107 0\\
108 0\\
109 0\\
110 0\\
111 0\\
112 0\\
113 0\\
114 0\\
115 0\\
116 0\\
117 0\\
118 0\\
119 0\\
120 0\\
121 0\\
122 0\\
123 0\\
124 0\\
125 0\\
126 0\\
127 0\\
128 0\\
129 0\\
130 0\\
131 0\\
132 0\\
133 0\\
134 0\\
135 0\\
136 0\\
137 0\\
138 0\\
139 0\\
140 0\\
141 0\\
142 0\\
143 0\\
144 0\\
145 0\\
146 0\\
147 0\\
148 0\\
149 0\\
150 0\\
151 0\\
152 0\\
153 0\\
154 0\\
155 0\\
156 0\\
157 0\\
158 0\\
159 0\\
160 0\\
161 0\\
162 0\\
163 0\\
164 0\\
165 0\\
166 0\\
167 0\\
168 0\\
169 0\\
170 0\\
171 0\\
172 0\\
173 0\\
174 0\\
175 0\\
176 0\\
177 0\\
178 0\\
179 0\\
180 0\\
181 0\\
182 0\\
183 0\\
184 0\\
185 0\\
186 0\\
187 0\\
188 0\\
189 0\\
190 0\\
191 0\\
192 0\\
193 0\\
194 0\\
195 0\\
196 0\\
197 0\\
198 0\\
199 0\\
};
\addplot [
color=red,
solid,
forget plot
]
table[row sep=crcr]{
1 0.980066577841242\\
2 -0.0590055838383565\\
3 -1.07579195693445\\
4 -2.02975647740664\\
5 -0.922734434550347\\
6 -1.75899245948667\\
7 -2.50519158372619\\
8 -1.1714500472705\\
9 -1.77107307199577\\
10 -0.320022424443604\\
11 -0.836280070559269\\
12 -1.29926444713735\\
13 0.269615327567125\\
14 -0.152320187859379\\
15 -0.54824975591132\\
16 -0.90238891413847\\
17 0.759513903829435\\
18 0.411070721395138\\
19 0.0661728686662518\\
20 -0.261429665357037\\
21 -0.558676401452799\\
22 1.14641610541327\\
23 0.825738042040495\\
24 2.45220698653988\\
25 3.96098075471379\\
26 3.33177614195361\\
27 4.54981070628259\\
28 5.56652525335127\\
29 4.38125342691383\\
30 5.00138143003127\\
31 3.4420535604639\\
32 3.72556845531019\\
33 1.88049011471314\\
34 -0.0196240103690475\\
35 0.0611107897562764\\
36 -1.84065728230098\\
37 -1.68897757920159\\
38 -3.45003024773765\\
39 -5.05360767533766\\
40 -6.43578029034442\\
41 -5.58131207661303\\
42 -6.48440114107708\\
43 -7.10904417259872\\
44 -5.47020546900378\\
45 -5.59335351343248\\
46 -5.47357878196957\\
47 -3.15552315950497\\
48 -0.731600208658836\\
49 -0.258577244032671\\
50 0.244687801484151\\
51 0.758131338671571\\
52 3.22141714964918\\
53 5.53634180248119\\
54 7.61061655642636\\
55 7.40141346711493\\
56 8.87720595722169\\
57 8.01902568161613\\
58 6.86108557974216\\
59 5.44954906937021\\
60 3.84068965680778\\
61 2.09864748978424\\
62 0.292872292166886\\
63 -1.50464537736844\\
64 -3.22224416173239\\
65 -4.79144881758876\\
66 -6.14970010722021\\
67 -5.28271568223647\\
68 -4.22505947573405\\
69 -3.01889690303905\\
70 -3.67244701534121\\
71 -4.15965483205967\\
72 -4.46109691501504\\
73 -2.60462256393292\\
74 -0.664243552747333\\
75 1.28268333058577\\
76 1.1984070994285\\
77 1.08628758116807\\
78 2.91092778286639\\
79 4.59958505766691\\
80 4.12480481320697\\
81 3.50551503957754\\
82 2.76640486577133\\
83 1.93694028202093\\
84 3.01032257943612\\
85 1.98362623739019\\
86 2.85791555479532\\
87 1.63820221986359\\
88 2.33324550991317\\
89 0.955203086623259\\
90 -0.440986847253912\\
91 -1.79966260498677\\
92 -1.10652491598134\\
93 -0.389206992575782\\
94 -1.63643919243115\\
95 -0.838365147925551\\
96 -1.9869347084276\\
97 -1.07622487439848\\
98 -0.142542773009808\\
99 0.776888636760268\\
100 -0.722800691458064\\
101 -0.204270363409612\\
102 0.149163829414516\\
103 0.323411581072547\\
104 -0.504637940050361\\
105 -0.323165265352743\\
106 -0.302048761333515\\
107 -1.25829439999775\\
108 -1.17497193808382\\
109 -2.03421092646469\\
110 -1.82294847090338\\
111 -1.71225056245474\\
112 -2.52269450102799\\
113 -2.24316269704024\\
114 -2.04744282503179\\
115 -1.94333761866982\\
116 -2.75116154763579\\
117 -2.45990147386201\\
118 -2.24381264099438\\
119 -2.1115098288916\\
120 -2.06826753316653\\
121 -2.93197381131756\\
122 -2.68938767236194\\
123 -3.32898800739064\\
124 -3.66263234579278\\
125 -2.86085521704506\\
126 -2.93442869237228\\
127 -2.71777600585879\\
128 -1.40337029280203\\
129 -1.02242050839012\\
130 0.388693828884645\\
131 0.794908296168244\\
132 2.15883615178963\\
133 3.26345827263956\\
134 3.24857273696963\\
135 4.09358073136862\\
136 3.78598670642113\\
137 4.31686121248837\\
138 4.50239633465599\\
139 4.33519537309307\\
140 3.0057599788772\\
141 2.54589829315543\\
142 1.81129992653117\\
143 0.0150868740957284\\
144 -0.7923237707727\\
145 -1.74138671722371\\
146 -3.61042994415039\\
147 -5.16229697257329\\
148 -5.51895563764199\\
149 -5.82883070751058\\
150 -6.07956844101073\\
151 -7.07733683958325\\
152 -7.61971440218843\\
153 -7.68507824697658\\
154 -6.45465840009675\\
155 -5.95631556528533\\
156 -4.23110935149887\\
157 -2.51046190990685\\
158 -0.862970024188571\\
159 0.645686003133688\\
160 1.95536081708997\\
161 3.01384181576571\\
162 3.77893070197364\\
163 4.22012579619826\\
164 4.31983804230447\\
165 4.0740922277002\\
166 2.67652133889117\\
167 1.3454857405531\\
168 0.134049621665997\\
169 -0.0933267589910516\\
170 -0.490222246277786\\
171 -1.04081386959223\\
172 -2.53931540203195\\
173 -3.76334269266476\\
174 -4.66409763605458\\
175 -4.38950585147829\\
176 -4.11315807049134\\
177 -4.66223553068259\\
178 -4.85220462279948\\
179 -3.85932775499466\\
180 -2.88583141841225\\
181 -1.97052583994644\\
182 -1.14990136459689\\
183 -1.27283782419227\\
184 -0.355626382781308\\
185 -0.41364111336063\\
186 0.534238595557544\\
187 0.471416023930389\\
188 1.37920345669625\\
189 2.05876664997078\\
190 2.48301356372378\\
191 1.81886668866569\\
192 1.2554470884973\\
193 1.63138064833073\\
194 0.952872336206724\\
195 1.22577988462853\\
196 0.46041558359447\\
197 -0.150064283831061\\
198 -0.581321811814436\\
199 0\\
};
\addplot [
color=mycolor1,
solid,
forget plot
]
table[row sep=crcr]{
1 0\\
2 0\\
3 0\\
4 0\\
5 0\\
6 0\\
7 0\\
8 0\\
9 0\\
10 0\\
11 0\\
12 0\\
13 0\\
14 0\\
15 0\\
16 0\\
17 0\\
18 0\\
19 0\\
20 0\\
21 0\\
22 0\\
23 0\\
24 0\\
25 0\\
26 0\\
27 0\\
28 0\\
29 0\\
30 0\\
31 0\\
32 0\\
33 0\\
34 0\\
35 0\\
36 0\\
37 0\\
38 0\\
39 0\\
40 0\\
41 0\\
42 0\\
43 0\\
44 0\\
45 0\\
46 0\\
47 0\\
48 0\\
49 0\\
50 0\\
51 0\\
52 0\\
53 0\\
54 0\\
55 0\\
56 0\\
57 0\\
58 0\\
59 0\\
60 0\\
61 0\\
62 0\\
63 0\\
64 0\\
65 0\\
66 0\\
67 0\\
68 0\\
69 0\\
70 0\\
71 0\\
72 0\\
73 0\\
74 0\\
75 0\\
76 0\\
77 0\\
78 0\\
79 0\\
80 0\\
81 0\\
82 0\\
83 0\\
84 0\\
85 0\\
86 0\\
87 0\\
88 0\\
89 0\\
90 0\\
91 0\\
92 0\\
93 0\\
94 0\\
95 0\\
96 0\\
97 0\\
98 0\\
99 0\\
100 0\\
101 0\\
102 0\\
103 0\\
104 0\\
105 0\\
106 0\\
107 0\\
108 0\\
109 0\\
110 0\\
111 0\\
112 0\\
113 0\\
114 0\\
115 0\\
116 0\\
117 0\\
118 0\\
119 0\\
120 0\\
121 0\\
122 0\\
123 0\\
124 0\\
125 0\\
126 0\\
127 0\\
128 0\\
129 0\\
130 0\\
131 0\\
132 0\\
133 0\\
134 0\\
135 0\\
136 0\\
137 0\\
138 0\\
139 0\\
140 0\\
141 0\\
142 0\\
143 0\\
144 0\\
145 0\\
146 0\\
147 0\\
148 0\\
149 0\\
150 0\\
151 0\\
152 0\\
153 0\\
154 0\\
155 0\\
156 0\\
157 0\\
158 0\\
159 0\\
160 0\\
161 0\\
162 0\\
163 0\\
164 0\\
165 0\\
166 0\\
167 0\\
168 0\\
169 0\\
170 0\\
171 0\\
172 0\\
173 0\\
174 0\\
175 0\\
176 0\\
177 0\\
178 0\\
179 0\\
180 0\\
181 0\\
182 0\\
183 0\\
184 0\\
185 0\\
186 0\\
187 0\\
188 0\\
189 0\\
190 0\\
191 0\\
192 0\\
193 0\\
194 0\\
195 0\\
196 0\\
197 0\\
198 0\\
199 0\\
};
\end{axis}
\end{tikzpicture}%    	
    	\caption{The cross-correlation of \texttt{x4} with other signals.}
    	\label{fig:cross4}
\end{figure}\\ 


Lastely we had to discuss how the training  sequence and its length $N_x$ should be designed and how $\hat{L}$ should be chosen in the estimation algorithm.
$N_x$ is a known value that was given to us at the beginning of this labday. 
$N_y$ can be chosen by looking at the length of the recieved signal.
When we know $N_x$ and $N_y$, we can easily find $\hat{L}$ by using Equation \ref{eq:channellength}.

\begin{equation}
\hat{L}=N_y - N_x +1
\label{eq:channellength}
\end{equation}


\end{document} 