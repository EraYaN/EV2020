\documentclass[final]{scrreprt} %scrreprt of scrartcl
\input{../../library/preamble.tex}
\input{../../library/style.tex}
\addbibresource{../../library/bibliography.bib}

\begin{document}

\chapter{Labday 2: Channel Estimation}
\label{ch:labday2}

\section{Channel estimation using a training sequence}

We have to look at four different test signals:

\begin{itemize}
\item 	A "minimium-phase" sequence, 
\begin{equation}
X_1(z) = 1 - \dfrac{1}{2}Z^{-1} \leftrightarrow x_1 = [1, -\dfrac{1}{2}, 0, 0,  \dots]^T
\end{equation}
\item A "maximum-phase" sequence,
\begin{equation}
X_2(z) = 1 - 2z^{-1} \leftrightarrow x_2 = [1, -2, 0, 0, \dots]^T
\end{equation}
\item A sinusoidal signal (N samples) followed by zeros,
\[ x_3[n] = \left\{ 
  \begin{array}{l l}
    \cos(\omega n), & \quad n = 0, \dots ,N - 1\\
    0, & \quad \textrm{elsewhere}
  \end{array} \right.\]
\item A random BPSK sequence (i.e., with entries randomly selected from {-1,1}: in Matlab),
\begin{equation}
x_4 = sign(randn(N,1))
\end{equation}
\end{itemize}

These four signals ..

Two Matlab functions, \texttt{ch1.m} and \texttt{ch2.m} respectively, are written. 
\texttt{ch1.m} implements the time-domain channel estimation using inversion and \texttt{ch2.m} implements a matched filter.
The scripts for these functions can be found in Appendix \ref{lst:ch1.m} and Appendix \ref{lst:ch2.m}.
With the help of the input $x[n]$, the output $y[n]$ and the channel length $L$ these two functions approach the $\hat{h}[n]$ filter. 
For the implementation of these two functions we also need the Toeplitz-matrix. 
We can call upon the Toeplitz matrix by using \texttt{toep.m}. 
The script for the Toeplitz-matrix can be found in Appendix \ref{lst:toep.m}.

\subsection{Autocorrelation of the input signal}

\begin{figure}[H]
	\centering
    	\setlength\figureheight{6cm}
    	\setlength\figurewidth{10cm}
    	% This file was created by matlab2tikz v0.4.2.
% Copyright (c) 2008--2013, Nico Schlömer <nico.schloemer@gmail.com>
% All rights reserved.
% 
% The latest updates can be retrieved from
%   http://www.mathworks.com/matlabcentral/fileexchange/22022-matlab2tikz
% where you can also make suggestions and rate matlab2tikz.
% 
% 
% 
\begin{tikzpicture}

\begin{axis}[%
width=\figurewidth,
height=\figureheight,
scale only axis,
xmin=0,
xmax=10,
ymin=-1,
ymax=1
]
\addplot [
color=red,
solid,
forget plot
]
table[row sep=crcr]{
1 1\\
2 -1\\
3 0.25\\
4 0\\
5 0\\
6 0\\
7 0\\
8 0\\
9 0\\
10 0\\
11 0\\
12 0\\
13 0\\
14 0\\
15 0\\
16 0\\
17 0\\
18 0\\
19 0\\
};
\end{axis}
\end{tikzpicture}%    	
    	\caption{The plots with $a=0.95$}
    	\label{fig:a0.95-plots}
\end{figure}
\begin{figure}[H]
	\centering
    	\setlength\figureheight{6cm}
    	\setlength\figurewidth{10cm}
    	% This file was created by matlab2tikz v0.4.2.
% Copyright (c) 2008--2013, Nico Schlömer <nico.schloemer@gmail.com>
% All rights reserved.
% 
% The latest updates can be retrieved from
%   http://www.mathworks.com/matlabcentral/fileexchange/22022-matlab2tikz
% where you can also make suggestions and rate matlab2tikz.
% 
% 
% 
\begin{tikzpicture}

\begin{axis}[%
width=\figurewidth,
height=\figureheight,
scale only axis,
xmin=0,
xmax=10,
ymin=-4,
ymax=4
]
\addplot [
color=blue,
solid,
forget plot
]
table[row sep=crcr]{
1 1\\
2 -4\\
3 4\\
4 0\\
5 0\\
6 0\\
7 0\\
8 0\\
9 0\\
10 0\\
11 0\\
12 0\\
13 0\\
14 0\\
15 0\\
16 0\\
17 0\\
18 0\\
19 0\\
};
\end{axis}
\end{tikzpicture}%    	
    	\caption{The plots with $a=0.95$}
    	\label{fig:a0.95-plots}
\end{figure}
\begin{figure}[H]
	\centering
    	\setlength\figureheight{6cm}
    	\setlength\figurewidth{10cm}
    	% This file was created by matlab2tikz v0.4.2.
% Copyright (c) 2008--2013, Nico Schlömer <nico.schloemer@gmail.com>
% All rights reserved.
% 
% The latest updates can be retrieved from
%   http://www.mathworks.com/matlabcentral/fileexchange/22022-matlab2tikz
% where you can also make suggestions and rate matlab2tikz.
% 
% 
% 
\begin{tikzpicture}

\begin{axis}[%
width=\figurewidth,
height=\figureheight,
scale only axis,
xmin=0,
xmax=200,
ymin=-50,
ymax=50
]
\addplot [
color=black,
solid,
forget plot
]
table[row sep=crcr]{
1 0.960530497001443\\
2 1.80540219275092\\
3 2.46612105802363\\
4 2.8860068044872\\
5 3.02366208954173\\
6 2.85560592731978\\
7 2.37792712977221\\
8 1.60686326274695\\
9 0.578265171063294\\
10 -0.654036039450516\\
11 -2.02088633649856\\
12 -3.44136409064282\\
13 -4.82705983090923\\
14 -6.0868678000012\\
15 -7.13204777569226\\
16 -7.88129836103421\\
17 -8.26557827186661\\
18 -8.23242042020773\\
19 -7.74950454756482\\
20 -6.8072869382994\\
21 -5.42052890499753\\
22 -3.6286173425187\\
23 -1.4946283306489\\
24 0.896854152390563\\
25 3.44309073449162\\
26 6.02757749954932\\
27 8.52528837442566\\
28 10.8085362254162\\
29 12.7531840494778\\
30 14.2449139605273\\
31 15.1852514204499\\
32 15.4970460735387\\
33 15.1291286892282\\
34 14.0598955179969\\
35 12.2996155931204\\
36 9.8913113448431\\
37 6.91012599110989\\
38 3.4611597667688\\
39 -0.324171913347549\\
40 -4.2931350230356\\
41 -8.27783869728025\\
42 -12.102313658063\\
43 -15.5902097443305\\
44 -18.5727784731256\\
45 -20.8967844376602\\
46 -22.4319829403572\\
47 -23.0778111886327\\
48 -22.7689665460967\\
49 -21.4795868956379\\
50 -19.2258035883674\\
51 -16.0665045267356\\
52 -12.1022208997821\\
53 -7.47213274083488\\
54 -2.34927228480901\\
55 3.06591362879184\\
56 8.5534038252426\\
57 13.8825391859822\\
58 18.8216923245839\\
59 23.1482661499714\\
60 26.6585922350795\\
61 29.1772935456007\\
62 30.5656887074462\\
63 30.7288464478435\\
64 29.6209481488223\\
65 27.2486818173863\\
66 23.6724696811307\\
67 19.0054208988926\\
68 13.4099968576873\\
69 7.09247517233551\\
70 0.295395583025821\\
71 -6.71173778321524\\
72 -13.6431419283969\\
73 -20.208404621688\\
74 -26.124623504744\\
75 -31.1285283397109\\
76 -34.9880694835393\\
77 -37.5129670293833\\
78 -38.5637483047719\\
79 -38.058855557764\\
80 -35.9794787820763\\
81 -32.3718579672675\\
82 -27.346901103051\\
83 -21.0770749034273\\
84 -13.7906398913626\\
85 -5.76341538937022\\
86 2.69163176567737\\
87 11.2365830309881\\
88 19.5220734913716\\
89 27.2015669637819\\
90 33.9458179346064\\
91 39.4569179986714\\
92 43.481330116958\\
93 45.8213449088545\\
94 46.3444484894742\\
95 44.9901691522377\\
96 41.7740675984296\\
97 36.7886486696399\\
98 30.2010971556602\\
99 22.2478711862416\\
100 12.4264233982307\\
101 2.30343893241976\\
102 -7.71274818829983\\
103 -17.2259802931516\\
104 -25.8679437958399\\
105 -33.3124161945913\\
106 -39.2875417588306\\
107 -43.5856756034281\\
108 -46.0704433242277\\
109 -46.680783327641\\
110 -45.431866293205\\
111 -42.4129155061348\\
112 -37.7820777410492\\
113 -31.7586118518339\\
114 -24.6127665471982\\
115 -16.6538059518045\\
116 -8.21670819524078\\
117 0.351893934262051\\
118 8.70794167043402\\
119 16.5237647048212\\
120 23.5009267370196\\
121 29.3815987039963\\
122 33.9580271832643\\
123 37.0797528037611\\
124 38.6583333099809\\
125 38.6694338269356\\
126 37.152258228458\\
127 34.2064056249564\\
128 29.9863403051705\\
129 24.6937577645033\\
130 18.5682100044732\\
131 11.8764170149218\\
132 4.90073593829726\\
133 -2.07271660341556\\
134 -8.76579073706677\\
135 -14.9193978903988\\
136 -20.3035311399789\\
137 -24.7258241820856\\
138 -28.0383320532459\\
139 -30.1422986888873\\
140 -30.9907663236552\\
141 -30.5889756357419\\
142 -28.9925993632876\\
143 -26.303941903779\\
144 -22.6663194320894\\
145 -18.2569060320191\\
146 -13.2783884639069\\
147 -7.94981337013574\\
148 -2.49703456130013\\
149 2.85682608898103\\
150 7.90050251676767\\
151 12.4429202501507\\
152 16.3204152375243\\
153 19.4025638251038\\
154 21.596419160037\\
155 22.8490235258887\\
156 23.1481440192249\\
157 22.5212570782421\\
158 21.0328822732744\\
159 18.7804342757974\\
160 15.8888211985766\\
161 12.5040651737292\\
162 8.78625529892408\\
163 4.90216275680719\\
164 1.01785249136873\\
165 -2.70838452077976\\
166 -6.13247880599182\\
167 -9.13014723242292\\
168 -11.6013604260095\\
169 -13.4735540913293\\
170 -14.7034871700997\\
171 -15.2777140106537\\
172 -15.2117016264164\\
173 -14.5476837976716\\
174 -13.3513985342662\\
175 -11.707901908481\\
176 -9.71668755490443\\
177 -7.48636579241951\\
178 -5.12916848980545\\
179 -2.75554520082084\\
180 -0.469103061415463\\
181 1.63788163495448\\
182 3.48818697483563\\
183 5.02245384446739\\
184 6.20097803981933\\
185 7.00443751874631\\
186 7.43355991324791\\
187 7.50778752208875\\
188 7.26304507236958\\
189 6.74875733373826\\
190 6.02429728434697\\
191 5.15506947537244\\
192 4.20844651654334\\
193 3.24977872995001\\
194 2.33868805747079\\
195 1.52583785671358\\
196 0.850341380513053\\
197 0.337935048891218\\
198 0\\
199 0\\
};
\end{axis}
\end{tikzpicture}%    	
    	\caption{The plots with $a=0.95$}
    	\label{fig:a0.95-plots}
\end{figure}
\begin{figure}[H]
	\centering
    	\setlength\figureheight{6cm}
    	\setlength\figurewidth{10cm}
    	% This file was created by matlab2tikz v0.4.2.
% Copyright (c) 2008--2013, Nico Schlömer <nico.schloemer@gmail.com>
% All rights reserved.
% 
% The latest updates can be retrieved from
%   http://www.mathworks.com/matlabcentral/fileexchange/22022-matlab2tikz
% where you can also make suggestions and rate matlab2tikz.
% 
% 
% 
%
% defining custom colors
\definecolor{mycolor1}{rgb}{1,0,1}%
%
\begin{tikzpicture}

\begin{axis}[%
width=\figurewidth,
height=\figureheight,
scale only axis,
xmin=0,
xmax=200,
ymin=-40,
ymax=30
]
\addplot [
color=mycolor1,
solid,
forget plot
]
table[row sep=crcr]{
1 1\\
2 -2\\
3 3\\
4 -4\\
5 5\\
6 -6\\
7 7\\
8 -4\\
9 5\\
10 -6\\
11 3\\
12 -4\\
13 5\\
14 -6\\
15 11\\
16 -4\\
17 5\\
18 -14\\
19 3\\
20 -4\\
21 9\\
22 -2\\
23 11\\
24 -4\\
25 1\\
26 -18\\
27 3\\
28 -4\\
29 21\\
30 -6\\
31 7\\
32 -20\\
33 1\\
34 -6\\
35 7\\
36 4\\
37 5\\
38 2\\
39 -9\\
40 -16\\
41 9\\
42 6\\
43 19\\
44 -8\\
45 -3\\
46 -14\\
47 -5\\
48 -4\\
49 5\\
50 10\\
51 -1\\
52 -16\\
53 -3\\
54 -6\\
55 15\\
56 0\\
57 13\\
58 -2\\
59 -9\\
60 -8\\
61 -7\\
62 10\\
63 -5\\
64 0\\
65 -3\\
66 2\\
67 -1\\
68 -20\\
69 -7\\
70 10\\
71 7\\
72 28\\
73 5\\
74 2\\
75 -21\\
76 12\\
77 -7\\
78 18\\
79 -5\\
80 8\\
81 -11\\
82 6\\
83 -25\\
84 16\\
85 -3\\
86 26\\
87 -13\\
88 16\\
89 -11\\
90 22\\
91 -1\\
92 24\\
93 -7\\
94 -2\\
95 -5\\
96 16\\
97 -11\\
98 10\\
99 -17\\
100 24\\
101 7\\
102 26\\
103 5\\
104 28\\
105 -17\\
106 -10\\
107 -15\\
108 20\\
109 3\\
110 14\\
111 -3\\
112 4\\
113 -1\\
114 14\\
115 21\\
116 12\\
117 -9\\
118 -2\\
119 -3\\
120 12\\
121 -17\\
122 10\\
123 13\\
124 4\\
125 -13\\
126 -6\\
127 -15\\
128 4\\
129 11\\
130 -2\\
131 1\\
132 4\\
133 -5\\
134 6\\
135 17\\
136 -12\\
137 -5\\
138 6\\
139 13\\
140 -8\\
141 -21\\
142 -14\\
143 9\\
144 -8\\
145 7\\
146 -10\\
147 -7\\
148 -12\\
149 7\\
150 -2\\
151 13\\
152 0\\
153 -1\\
154 -2\\
155 -3\\
156 -32\\
157 -1\\
158 -6\\
159 13\\
160 0\\
161 7\\
162 -2\\
163 9\\
164 -8\\
165 -5\\
166 -10\\
167 -3\\
168 -12\\
169 7\\
170 6\\
171 1\\
172 -4\\
173 11\\
174 -2\\
175 5\\
176 -4\\
177 -1\\
178 -10\\
179 5\\
180 -8\\
181 7\\
182 10\\
183 1\\
184 0\\
185 3\\
186 -10\\
187 -7\\
188 0\\
189 -1\\
190 2\\
191 5\\
192 0\\
193 3\\
194 2\\
195 1\\
196 0\\
197 3\\
198 -2\\
199 1\\
};
\end{axis}
\end{tikzpicture}%    	
    	\caption{The plots with $a=0.95$}
    	\label{fig:a0.95-plots}
\end{figure}

\end{document} 