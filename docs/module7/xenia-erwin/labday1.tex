\documentclass[final]{scrreprt} %scrreprt of scrartcl
\input{../../library/preamble.tex}
\input{../../library/style.tex}
\usepackage{pgfplots}
\addbibresource{../../library/bibliography.bib}

\begin{document}

\chapter{Labday 1: Basic filtering}
\label{ch:labday1}
\section{Channel inpulse response}
In a 2-dimentional room of 4x4 m, a transmitter and a receiver are placed at (1.2, 0.3) and (3.1, 3.3) respectively.
In this section the propagation of a delta impulse, from TX to RX, is being investagated.
During propagation, the signal is damped as function of the propagated distance $R$. 
The speed of sound varies as function of temperature, humidity and air pressure, but we can assume is it 340m/s. 
For the attenuation $\alpha$(r) of the signal can be found with the use of the following equation:

\begin{equation}
	\alpha (r) = \frac{\beta}{r^2}
\label{eq:damping}
\end{equation}

In Eqaution \ref{eq:damping} $\beta$ is the damping over a reference distance of 1 meter. 
\\
\\
The direct path to the RX point is the fastest and the strongest and can easily be calculated.
The indirect paths, however, are more difficult to calculate.
The indirect paths are the results of reflections and can be calculated by looking at the mirror images of the TX and RX over the edges of the room. 
To find the first order reflections only one mirror image of either the TX or the RX is needed. 
We choose to mirror the TX.
The mirrored image of the TX and the original RX are connected, which always results in an intersection with a wall.
Since there are four walls, four virtual sources are created this way when looking at first order reflections.
This location at the wall is the point where the single bounce is made, connecting this point with both the RX and TX will be the total path.
\\ 
\\
In order to find the second order reflections we needed to mirror both the TX and the RX as can be seen in Figure \ref{fig:mirror}.
In this figure also the first order reflections are displayed. 
This is because the TX is not only present as the mirrored image, but also as the non-mirrored version. 
The connection between the mirror image might intersect with the room.
It either intersects with two walls or none. 
This is because both mirror images are outside the room.
When there is an intersection, we have two point on two different walls. 
Via these two point the delta pulse will bounce from the TX to the RX. 
When we mirrored the TX and RX over all the walls, we got 16 combinations, but not all of these intersect with the room. 
Some of them connected outside the wall, this means that it is not possible to bounce via the two walls. 
When looking closely at Figure \ref{fig:connected}, we see that only 8 of the 16 paths are possible, meaning that we get 8 virutal sources form the second order reflections. 
\\ 
\\
As explained before the second oreder reflection paths were drawn by connecting the two wall intersections. 
When we calculate the length of all the direct and indirect (reflected) paths and we use the speed of sound, we can calculate the delay of each path.
With the help of Equation \ref{eq:damping} we are able to calculate the resulting amplitude. 
We used a $\beta$ of 1.00 since the outcome of all amplitudes would give us the total received signal compared to a infinitely large room with the TX and RX 1 meter away from each other.
In Figure \ref{fig:response} we displayed the resulting delta impulse response over time. 
With the Matlab script in Appendix \ref{lst:report1.m}, we got a resulting amplitude of 0.5420. 
This is equal to the sum of all delta impulse response amplitudes in Figure \ref{fig:response}.

\begin{figure}[H]
	\centering
	\setlength\figureheight{6cm}
    	\setlength\figurewidth{6cm}
	% This file was created by matlab2tikz v0.4.6 running on MATLAB 8.2.
% Copyright (c) 2008--2014, Nico Schlömer <nico.schloemer@gmail.com>
% All rights reserved.
% Minimal pgfplots version: 1.3
% 
% The latest updates can be retrieved from
%   http://www.mathworks.com/matlabcentral/fileexchange/22022-matlab2tikz
% where you can also make suggestions and rate matlab2tikz.
% 
\begin{tikzpicture}

\begin{axis}[%
width=\figurewidth,
height=\figureheight,
scale only axis,
xmin=-4,
xmax=8,
xlabel={x location (m)},
ymin=-4,
ymax=8,
ylabel={y location (m)},
axis x line*=bottom,
axis y line*=left
]
\addplot [color=blue,only marks,mark=asterisk,mark options={solid},forget plot]
  table[row sep=crcr]{
1.2	0.3	\\
};
\addplot [color=blue,only marks,mark=asterisk,mark options={solid},forget plot]
  table[row sep=crcr]{
3.1	3.3	\\
};
\addplot [color=blue,solid,forget plot]
  table[row sep=crcr]{
0	0	\\
0	4	\\
};
\addplot [color=blue,solid,forget plot]
  table[row sep=crcr]{
4	4	\\
0	4	\\
};
\addplot [color=blue,solid,forget plot]
  table[row sep=crcr]{
4	0	\\
0	0	\\
};
\addplot [color=blue,solid,forget plot]
  table[row sep=crcr]{
4	4	\\
4	0	\\
};
\addplot [color=red,only marks,mark=asterisk,mark options={solid},forget plot]
  table[row sep=crcr]{
3.1	4.7	\\
};
\addplot [color=red,only marks,mark=asterisk,mark options={solid},forget plot]
  table[row sep=crcr]{
4.9	3.3	\\
};
\addplot [color=red,only marks,mark=asterisk,mark options={solid},forget plot]
  table[row sep=crcr]{
3.1	-3.3	\\
};
\addplot [color=red,only marks,mark=asterisk,mark options={solid},forget plot]
  table[row sep=crcr]{
-3.1	3.3	\\
};
\addplot [color=red,only marks,mark=asterisk,mark options={solid},forget plot]
  table[row sep=crcr]{
4.9	3.3	\\
};
\addplot [color=red,only marks,mark=asterisk,mark options={solid},forget plot]
  table[row sep=crcr]{
1.2	7.7	\\
};
\addplot [color=red,only marks,mark=asterisk,mark options={solid},forget plot]
  table[row sep=crcr]{
3.1	-3.3	\\
};
\addplot [color=red,only marks,mark=asterisk,mark options={solid},forget plot]
  table[row sep=crcr]{
1.2	7.7	\\
};
\addplot [color=red,only marks,mark=asterisk,mark options={solid},forget plot]
  table[row sep=crcr]{
-3.1	3.3	\\
};
\addplot [color=red,only marks,mark=asterisk,mark options={solid},forget plot]
  table[row sep=crcr]{
1.2	7.7	\\
};
\addplot [color=red,only marks,mark=asterisk,mark options={solid},forget plot]
  table[row sep=crcr]{
3.1	4.7	\\
};
\addplot [color=red,only marks,mark=asterisk,mark options={solid},forget plot]
  table[row sep=crcr]{
6.8	0.3	\\
};
\addplot [color=red,only marks,mark=asterisk,mark options={solid},forget plot]
  table[row sep=crcr]{
3.1	-3.3	\\
};
\addplot [color=red,only marks,mark=asterisk,mark options={solid},forget plot]
  table[row sep=crcr]{
6.8	0.3	\\
};
\addplot [color=red,only marks,mark=asterisk,mark options={solid},forget plot]
  table[row sep=crcr]{
-3.1	3.3	\\
};
\addplot [color=red,only marks,mark=asterisk,mark options={solid},forget plot]
  table[row sep=crcr]{
6.8	0.3	\\
};
\addplot [color=red,only marks,mark=asterisk,mark options={solid},forget plot]
  table[row sep=crcr]{
3.1	4.7	\\
};
\addplot [color=red,only marks,mark=asterisk,mark options={solid},forget plot]
  table[row sep=crcr]{
1.2	-0.3	\\
};
\addplot [color=red,only marks,mark=asterisk,mark options={solid},forget plot]
  table[row sep=crcr]{
4.9	3.3	\\
};
\addplot [color=red,only marks,mark=asterisk,mark options={solid},forget plot]
  table[row sep=crcr]{
1.2	-0.3	\\
};
\addplot [color=red,only marks,mark=asterisk,mark options={solid},forget plot]
  table[row sep=crcr]{
-3.1	3.3	\\
};
\addplot [color=red,only marks,mark=asterisk,mark options={solid},forget plot]
  table[row sep=crcr]{
1.2	-0.3	\\
};
\addplot [color=red,only marks,mark=asterisk,mark options={solid},forget plot]
  table[row sep=crcr]{
3.1	4.7	\\
};
\addplot [color=red,only marks,mark=asterisk,mark options={solid},forget plot]
  table[row sep=crcr]{
-1.2	0.3	\\
};
\addplot [color=red,only marks,mark=asterisk,mark options={solid},forget plot]
  table[row sep=crcr]{
4.9	3.3	\\
};
\addplot [color=red,only marks,mark=asterisk,mark options={solid},forget plot]
  table[row sep=crcr]{
-1.2	0.3	\\
};
\addplot [color=red,only marks,mark=asterisk,mark options={solid},forget plot]
  table[row sep=crcr]{
3.1	-3.3	\\
};
\addplot [color=red,only marks,mark=asterisk,mark options={solid},forget plot]
  table[row sep=crcr]{
-1.2	0.3	\\
};
\end{axis}
\end{tikzpicture}%
	\caption{Mirror images of the TX and RX.}
	\label{fig:mirror}
\end{figure}

\begin{figure}[H]
	\centering
	\setlength\figureheight{6cm}
    	\setlength\figurewidth{6cm}
	% This file was created by matlab2tikz v0.4.6 running on MATLAB 8.2.
% Copyright (c) 2008--2014, Nico Schlömer <nico.schloemer@gmail.com>
% All rights reserved.
% Minimal pgfplots version: 1.3
% 
% The latest updates can be retrieved from
%   http://www.mathworks.com/matlabcentral/fileexchange/22022-matlab2tikz
% where you can also make suggestions and rate matlab2tikz.
% 
\begin{tikzpicture}

\begin{axis}[%
width=\figurewidth,
height=\figureheight,
scale only axis,
xmin=-4,
xmax=8,
xlabel={x location (m)},
ymin=-4,
ymax=8,
ylabel={y location (m)},
axis x line*=bottom,
axis y line*=left
]
\addplot [color=blue,only marks,mark=asterisk,mark options={solid},forget plot]
  table[row sep=crcr]{
1.2	0.3	\\
};
\addplot [color=blue,only marks,mark=asterisk,mark options={solid},forget plot]
  table[row sep=crcr]{
3.1	3.3	\\
};
\addplot [color=blue,solid,forget plot]
  table[row sep=crcr]{
0	0	\\
0	4	\\
};
\addplot [color=blue,solid,forget plot]
  table[row sep=crcr]{
4	4	\\
0	4	\\
};
\addplot [color=blue,solid,forget plot]
  table[row sep=crcr]{
4	0	\\
0	0	\\
};
\addplot [color=blue,solid,forget plot]
  table[row sep=crcr]{
4	4	\\
4	0	\\
};
\addplot [color=red,only marks,mark=asterisk,mark options={solid},forget plot]
  table[row sep=crcr]{
3.1	4.7	\\
};
\addplot [color=red,dotted,forget plot]
  table[row sep=crcr]{
1.2	0.3	\\
3.1	4.7	\\
};
\addplot [color=red,only marks,mark=asterisk,mark options={solid},forget plot]
  table[row sep=crcr]{
1.2	0.3	\\
};
\addplot [color=red,only marks,mark=asterisk,mark options={solid},forget plot]
  table[row sep=crcr]{
2.79772727272727	4	\\
};
\addplot [color=red,only marks,mark=asterisk,mark options={solid},forget plot]
  table[row sep=crcr]{
4.9	3.3	\\
};
\addplot [color=red,dotted,forget plot]
  table[row sep=crcr]{
1.2	0.3	\\
4.9	3.3	\\
};
\addplot [color=red,only marks,mark=asterisk,mark options={solid},forget plot]
  table[row sep=crcr]{
1.2	0.3	\\
};
\addplot [color=red,only marks,mark=asterisk,mark options={solid},forget plot]
  table[row sep=crcr]{
4	2.57027027027027	\\
};
\addplot [color=red,only marks,mark=asterisk,mark options={solid},forget plot]
  table[row sep=crcr]{
3.1	-3.3	\\
};
\addplot [color=red,dotted,forget plot]
  table[row sep=crcr]{
1.2	0.3	\\
3.1	-3.3	\\
};
\addplot [color=red,only marks,mark=asterisk,mark options={solid},forget plot]
  table[row sep=crcr]{
1.2	0.3	\\
};
\addplot [color=red,only marks,mark=asterisk,mark options={solid},forget plot]
  table[row sep=crcr]{
1.35833333333333	0	\\
};
\addplot [color=red,only marks,mark=asterisk,mark options={solid},forget plot]
  table[row sep=crcr]{
-3.1	3.3	\\
};
\addplot [color=red,dotted,forget plot]
  table[row sep=crcr]{
1.2	0.3	\\
-3.1	3.3	\\
};
\addplot [color=red,only marks,mark=asterisk,mark options={solid},forget plot]
  table[row sep=crcr]{
1.2	0.3	\\
};
\addplot [color=red,only marks,mark=asterisk,mark options={solid},forget plot]
  table[row sep=crcr]{
0	1.13720930232558	\\
};
\addplot [color=red,only marks,mark=asterisk,mark options={solid},forget plot]
  table[row sep=crcr]{
4.9	3.3	\\
};
\addplot [color=red,only marks,mark=asterisk,mark options={solid},forget plot]
  table[row sep=crcr]{
1.2	7.7	\\
};
\addplot [color=red,dotted,forget plot]
  table[row sep=crcr]{
1.2	7.7	\\
4.9	3.3	\\
};
\addplot [color=red,only marks,mark=asterisk,mark options={solid},forget plot]
  table[row sep=crcr]{
3.1	-3.3	\\
};
\addplot [color=red,only marks,mark=asterisk,mark options={solid},forget plot]
  table[row sep=crcr]{
1.2	7.7	\\
};
\addplot [color=red,dotted,forget plot]
  table[row sep=crcr]{
1.2	7.7	\\
3.1	-3.3	\\
};
\addplot [color=red,only marks,mark=asterisk,mark options={solid},forget plot]
  table[row sep=crcr]{
1.83909090909091	4	\\
};
\addplot [color=red,only marks,mark=asterisk,mark options={solid},forget plot]
  table[row sep=crcr]{
2.53	0	\\
};
\addplot [color=red,only marks,mark=asterisk,mark options={solid},forget plot]
  table[row sep=crcr]{
-3.1	3.3	\\
};
\addplot [color=red,only marks,mark=asterisk,mark options={solid},forget plot]
  table[row sep=crcr]{
1.2	7.7	\\
};
\addplot [color=red,dotted,forget plot]
  table[row sep=crcr]{
1.2	7.7	\\
-3.1	3.3	\\
};
\addplot [color=red,only marks,mark=asterisk,mark options={solid},forget plot]
  table[row sep=crcr]{
3.1	4.7	\\
};
\addplot [color=red,only marks,mark=asterisk,mark options={solid},forget plot]
  table[row sep=crcr]{
6.8	0.3	\\
};
\addplot [color=red,dotted,forget plot]
  table[row sep=crcr]{
6.8	0.3	\\
3.1	4.7	\\
};
\addplot [color=red,only marks,mark=asterisk,mark options={solid},forget plot]
  table[row sep=crcr]{
4	3.62972972972973	\\
};
\addplot [color=red,only marks,mark=asterisk,mark options={solid},forget plot]
  table[row sep=crcr]{
3.68863636363636	4	\\
};
\addplot [color=red,only marks,mark=asterisk,mark options={solid},forget plot]
  table[row sep=crcr]{
3.1	-3.3	\\
};
\addplot [color=red,only marks,mark=asterisk,mark options={solid},forget plot]
  table[row sep=crcr]{
6.8	0.3	\\
};
\addplot [color=red,dotted,forget plot]
  table[row sep=crcr]{
6.8	0.3	\\
3.1	-3.3	\\
};
\addplot [color=red,only marks,mark=asterisk,mark options={solid},forget plot]
  table[row sep=crcr]{
-3.1	3.3	\\
};
\addplot [color=red,only marks,mark=asterisk,mark options={solid},forget plot]
  table[row sep=crcr]{
6.8	0.3	\\
};
\addplot [color=red,dotted,forget plot]
  table[row sep=crcr]{
6.8	0.3	\\
-3.1	3.3	\\
};
\addplot [color=red,only marks,mark=asterisk,mark options={solid},forget plot]
  table[row sep=crcr]{
4	1.14848484848485	\\
};
\addplot [color=red,only marks,mark=asterisk,mark options={solid},forget plot]
  table[row sep=crcr]{
0	2.36060606060606	\\
};
\addplot [color=red,only marks,mark=asterisk,mark options={solid},forget plot]
  table[row sep=crcr]{
3.1	4.7	\\
};
\addplot [color=red,only marks,mark=asterisk,mark options={solid},forget plot]
  table[row sep=crcr]{
1.2	-0.3	\\
};
\addplot [color=red,dotted,forget plot]
  table[row sep=crcr]{
1.2	-0.3	\\
3.1	4.7	\\
};
\addplot [color=red,only marks,mark=asterisk,mark options={solid},forget plot]
  table[row sep=crcr]{
1.314	0	\\
};
\addplot [color=red,only marks,mark=asterisk,mark options={solid},forget plot]
  table[row sep=crcr]{
2.834	4	\\
};
\addplot [color=red,only marks,mark=asterisk,mark options={solid},forget plot]
  table[row sep=crcr]{
4.9	3.3	\\
};
\addplot [color=red,only marks,mark=asterisk,mark options={solid},forget plot]
  table[row sep=crcr]{
1.2	-0.3	\\
};
\addplot [color=red,dotted,forget plot]
  table[row sep=crcr]{
1.2	-0.3	\\
4.9	3.3	\\
};
\addplot [color=red,only marks,mark=asterisk,mark options={solid},forget plot]
  table[row sep=crcr]{
1.50833333333333	0	\\
};
\addplot [color=red,only marks,mark=asterisk,mark options={solid},forget plot]
  table[row sep=crcr]{
4	2.42432432432432	\\
};
\addplot [color=red,only marks,mark=asterisk,mark options={solid},forget plot]
  table[row sep=crcr]{
-3.1	3.3	\\
};
\addplot [color=red,only marks,mark=asterisk,mark options={solid},forget plot]
  table[row sep=crcr]{
1.2	-0.3	\\
};
\addplot [color=red,dotted,forget plot]
  table[row sep=crcr]{
1.2	-0.3	\\
-3.1	3.3	\\
};
\addplot [color=red,only marks,mark=asterisk,mark options={solid},forget plot]
  table[row sep=crcr]{
0.841666666666667	0	\\
};
\addplot [color=red,only marks,mark=asterisk,mark options={solid},forget plot]
  table[row sep=crcr]{
0	0.704651162790698	\\
};
\addplot [color=red,only marks,mark=asterisk,mark options={solid},forget plot]
  table[row sep=crcr]{
3.1	4.7	\\
};
\addplot [color=red,only marks,mark=asterisk,mark options={solid},forget plot]
  table[row sep=crcr]{
-1.2	0.3	\\
};
\addplot [color=red,dotted,forget plot]
  table[row sep=crcr]{
-1.2	0.3	\\
3.1	4.7	\\
};
\addplot [color=red,only marks,mark=asterisk,mark options={solid},forget plot]
  table[row sep=crcr]{
0	1.52790697674419	\\
};
\addplot [color=red,only marks,mark=asterisk,mark options={solid},forget plot]
  table[row sep=crcr]{
2.41590909090909	4	\\
};
\addplot [color=red,only marks,mark=asterisk,mark options={solid},forget plot]
  table[row sep=crcr]{
4.9	3.3	\\
};
\addplot [color=red,only marks,mark=asterisk,mark options={solid},forget plot]
  table[row sep=crcr]{
-1.2	0.3	\\
};
\addplot [color=red,dotted,forget plot]
  table[row sep=crcr]{
-1.2	0.3	\\
4.9	3.3	\\
};
\addplot [color=red,only marks,mark=asterisk,mark options={solid},forget plot]
  table[row sep=crcr]{
0	0.89016393442623	\\
};
\addplot [color=red,only marks,mark=asterisk,mark options={solid},forget plot]
  table[row sep=crcr]{
4	2.85737704918033	\\
};
\addplot [color=red,only marks,mark=asterisk,mark options={solid},forget plot]
  table[row sep=crcr]{
3.1	-3.3	\\
};
\addplot [color=red,only marks,mark=asterisk,mark options={solid},forget plot]
  table[row sep=crcr]{
-1.2	0.3	\\
};
\addplot [color=red,dotted,forget plot]
  table[row sep=crcr]{
-1.2	0.3	\\
3.1	-3.3	\\
};
\end{axis}
\end{tikzpicture}%
	\caption{Connected mirror images of the TX and RX.}
	\label{fig:connected}
\end{figure}

\begin{figure}[H]
	\centering
	\setlength\figureheight{6cm}
    	\setlength\figurewidth{6cm}
	% This file was created by matlab2tikz v0.4.6 running on MATLAB 8.2.
% Copyright (c) 2008--2014, Nico Schlömer <nico.schloemer@gmail.com>
% All rights reserved.
% Minimal pgfplots version: 1.3
% 
% The latest updates can be retrieved from
%   http://www.mathworks.com/matlabcentral/fileexchange/22022-matlab2tikz
% where you can also make suggestions and rate matlab2tikz.
% 
\begin{tikzpicture}

\begin{axis}[%
width=\figurewidth,
height=\figureheight,
scale only axis,
xmin=0,
xmax=4,
xlabel={x location (m)},
ymin=0,
ymax=4,
ylabel={y location (m)},
axis x line*=bottom,
axis y line*=left
]
\addplot [color=blue,only marks,mark=asterisk,mark options={solid},forget plot]
  table[row sep=crcr]{
1.2	0.3	\\
};
\addplot [color=blue,only marks,mark=asterisk,mark options={solid},forget plot]
  table[row sep=crcr]{
3.1	3.3	\\
};
\addplot [color=red,solid,forget plot]
  table[row sep=crcr]{
1.2	0.3	\\
3.1	3.3	\\
};
\addplot [color=blue,solid,forget plot]
  table[row sep=crcr]{
0	0	\\
0	4	\\
};
\addplot [color=blue,solid,forget plot]
  table[row sep=crcr]{
4	4	\\
0	4	\\
};
\addplot [color=blue,solid,forget plot]
  table[row sep=crcr]{
4	0	\\
0	0	\\
};
\addplot [color=blue,solid,forget plot]
  table[row sep=crcr]{
4	4	\\
4	0	\\
};
\addplot [color=red,only marks,mark=asterisk,mark options={solid},forget plot]
  table[row sep=crcr]{
1.2	0.3	\\
};
\addplot [color=red,only marks,mark=asterisk,mark options={solid},forget plot]
  table[row sep=crcr]{
2.79772727272727	4	\\
};
\addplot [color=red,solid,forget plot]
  table[row sep=crcr]{
1.2	0.3	\\
1.2	0.3	\\
};
\addplot [color=red,solid,forget plot]
  table[row sep=crcr]{
3.1	3.3	\\
2.79772727272727	4	\\
};
\addplot [color=red,solid,forget plot]
  table[row sep=crcr]{
1.2	0.3	\\
2.79772727272727	4	\\
};
\addplot [color=red,only marks,mark=asterisk,mark options={solid},forget plot]
  table[row sep=crcr]{
1.2	0.3	\\
};
\addplot [color=red,only marks,mark=asterisk,mark options={solid},forget plot]
  table[row sep=crcr]{
4	2.57027027027027	\\
};
\addplot [color=red,solid,forget plot]
  table[row sep=crcr]{
1.2	0.3	\\
1.2	0.3	\\
};
\addplot [color=red,solid,forget plot]
  table[row sep=crcr]{
3.1	3.3	\\
4	2.57027027027027	\\
};
\addplot [color=red,solid,forget plot]
  table[row sep=crcr]{
1.2	0.3	\\
4	2.57027027027027	\\
};
\addplot [color=red,only marks,mark=asterisk,mark options={solid},forget plot]
  table[row sep=crcr]{
1.2	0.3	\\
};
\addplot [color=red,only marks,mark=asterisk,mark options={solid},forget plot]
  table[row sep=crcr]{
1.35833333333333	0	\\
};
\addplot [color=red,solid,forget plot]
  table[row sep=crcr]{
1.2	0.3	\\
1.2	0.3	\\
};
\addplot [color=red,solid,forget plot]
  table[row sep=crcr]{
3.1	3.3	\\
1.35833333333333	0	\\
};
\addplot [color=red,solid,forget plot]
  table[row sep=crcr]{
1.2	0.3	\\
1.35833333333333	0	\\
};
\addplot [color=red,only marks,mark=asterisk,mark options={solid},forget plot]
  table[row sep=crcr]{
1.2	0.3	\\
};
\addplot [color=red,only marks,mark=asterisk,mark options={solid},forget plot]
  table[row sep=crcr]{
0	1.13720930232558	\\
};
\addplot [color=red,solid,forget plot]
  table[row sep=crcr]{
1.2	0.3	\\
1.2	0.3	\\
};
\addplot [color=red,solid,forget plot]
  table[row sep=crcr]{
3.1	3.3	\\
0	1.13720930232558	\\
};
\addplot [color=red,solid,forget plot]
  table[row sep=crcr]{
1.2	0.3	\\
0	1.13720930232558	\\
};
\addplot [color=red,only marks,mark=asterisk,mark options={solid},forget plot]
  table[row sep=crcr]{
1.83909090909091	4	\\
};
\addplot [color=red,only marks,mark=asterisk,mark options={solid},forget plot]
  table[row sep=crcr]{
2.53	0	\\
};
\addplot [color=red,solid,forget plot]
  table[row sep=crcr]{
1.2	0.3	\\
1.83909090909091	4	\\
};
\addplot [color=red,solid,forget plot]
  table[row sep=crcr]{
3.1	3.3	\\
2.53	0	\\
};
\addplot [color=red,solid,forget plot]
  table[row sep=crcr]{
1.83909090909091	4	\\
2.53	0	\\
};
\addplot [color=red,only marks,mark=asterisk,mark options={solid},forget plot]
  table[row sep=crcr]{
4	3.62972972972973	\\
};
\addplot [color=red,only marks,mark=asterisk,mark options={solid},forget plot]
  table[row sep=crcr]{
3.68863636363636	4	\\
};
\addplot [color=red,solid,forget plot]
  table[row sep=crcr]{
1.2	0.3	\\
4	3.62972972972973	\\
};
\addplot [color=red,solid,forget plot]
  table[row sep=crcr]{
3.1	3.3	\\
3.68863636363636	4	\\
};
\addplot [color=red,solid,forget plot]
  table[row sep=crcr]{
4	3.62972972972973	\\
3.68863636363636	4	\\
};
\addplot [color=red,only marks,mark=asterisk,mark options={solid},forget plot]
  table[row sep=crcr]{
4	1.14848484848485	\\
};
\addplot [color=red,only marks,mark=asterisk,mark options={solid},forget plot]
  table[row sep=crcr]{
0	2.36060606060606	\\
};
\addplot [color=red,solid,forget plot]
  table[row sep=crcr]{
1.2	0.3	\\
4	1.14848484848485	\\
};
\addplot [color=red,solid,forget plot]
  table[row sep=crcr]{
3.1	3.3	\\
0	2.36060606060606	\\
};
\addplot [color=red,solid,forget plot]
  table[row sep=crcr]{
4	1.14848484848485	\\
0	2.36060606060606	\\
};
\addplot [color=red,only marks,mark=asterisk,mark options={solid},forget plot]
  table[row sep=crcr]{
1.314	0	\\
};
\addplot [color=red,only marks,mark=asterisk,mark options={solid},forget plot]
  table[row sep=crcr]{
2.834	4	\\
};
\addplot [color=red,solid,forget plot]
  table[row sep=crcr]{
1.2	0.3	\\
1.314	0	\\
};
\addplot [color=red,solid,forget plot]
  table[row sep=crcr]{
3.1	3.3	\\
2.834	4	\\
};
\addplot [color=red,solid,forget plot]
  table[row sep=crcr]{
1.314	0	\\
2.834	4	\\
};
\addplot [color=red,only marks,mark=asterisk,mark options={solid},forget plot]
  table[row sep=crcr]{
1.50833333333333	0	\\
};
\addplot [color=red,only marks,mark=asterisk,mark options={solid},forget plot]
  table[row sep=crcr]{
4	2.42432432432432	\\
};
\addplot [color=red,solid,forget plot]
  table[row sep=crcr]{
1.2	0.3	\\
1.50833333333333	0	\\
};
\addplot [color=red,solid,forget plot]
  table[row sep=crcr]{
3.1	3.3	\\
4	2.42432432432432	\\
};
\addplot [color=red,solid,forget plot]
  table[row sep=crcr]{
1.50833333333333	0	\\
4	2.42432432432432	\\
};
\addplot [color=red,only marks,mark=asterisk,mark options={solid},forget plot]
  table[row sep=crcr]{
0.841666666666667	0	\\
};
\addplot [color=red,only marks,mark=asterisk,mark options={solid},forget plot]
  table[row sep=crcr]{
0	0.704651162790698	\\
};
\addplot [color=red,solid,forget plot]
  table[row sep=crcr]{
1.2	0.3	\\
0.841666666666667	0	\\
};
\addplot [color=red,solid,forget plot]
  table[row sep=crcr]{
3.1	3.3	\\
0	0.704651162790698	\\
};
\addplot [color=red,solid,forget plot]
  table[row sep=crcr]{
0.841666666666667	0	\\
0	0.704651162790698	\\
};
\addplot [color=red,only marks,mark=asterisk,mark options={solid},forget plot]
  table[row sep=crcr]{
0	1.52790697674419	\\
};
\addplot [color=red,only marks,mark=asterisk,mark options={solid},forget plot]
  table[row sep=crcr]{
2.41590909090909	4	\\
};
\addplot [color=red,solid,forget plot]
  table[row sep=crcr]{
1.2	0.3	\\
0	1.52790697674419	\\
};
\addplot [color=red,solid,forget plot]
  table[row sep=crcr]{
3.1	3.3	\\
2.41590909090909	4	\\
};
\addplot [color=red,solid,forget plot]
  table[row sep=crcr]{
0	1.52790697674419	\\
2.41590909090909	4	\\
};
\addplot [color=red,only marks,mark=asterisk,mark options={solid},forget plot]
  table[row sep=crcr]{
0	0.89016393442623	\\
};
\addplot [color=red,only marks,mark=asterisk,mark options={solid},forget plot]
  table[row sep=crcr]{
4	2.85737704918033	\\
};
\addplot [color=red,solid,forget plot]
  table[row sep=crcr]{
1.2	0.3	\\
0	0.89016393442623	\\
};
\addplot [color=red,solid,forget plot]
  table[row sep=crcr]{
3.1	3.3	\\
4	2.85737704918033	\\
};
\addplot [color=red,solid,forget plot]
  table[row sep=crcr]{
0	0.89016393442623	\\
4	2.85737704918033	\\
};
\end{axis}
\end{tikzpicture}%
	\caption{All path within two reflections from TX to RX.}
	\label{fig:reflections}
\end{figure}

\begin{figure}[H]
	\centering
	\setlength\figureheight{4cm}
    	\setlength\figurewidth{0.8\linewidth}
	% This file was created by matlab2tikz v0.4.6 running on MATLAB 8.2.
% Copyright (c) 2008--2014, Nico Schlömer <nico.schloemer@gmail.com>
% All rights reserved.
% Minimal pgfplots version: 1.3
% 
% The latest updates can be retrieved from
%   http://www.mathworks.com/matlabcentral/fileexchange/22022-matlab2tikz
% where you can also make suggestions and rate matlab2tikz.
% 
\begin{tikzpicture}

\begin{axis}[%
width=\figurewidth,
height=\figureheight,
scale only axis,
xmin=0.01,
xmax=0.035,
xlabel={time (s)},ymin=0,
ymax=0.08
,ylabel={received signal}]
\addplot [color=blue,solid,forget plot]
  table[row sep=crcr]{
0.0104442828850381	0	\\
0.0104442828850381	0.0793021411578113	\\
};
\addplot [color=blue,solid,forget plot]
  table[row sep=crcr]{
0.0139124333624039	0	\\
0.0139124333624039	0.0446926361231556	\\
};
\addplot [color=blue,solid,forget plot]
  table[row sep=crcr]{
0.0127484308485471	0	\\
0.0127484308485471	0.053226597002272	\\
};
\addplot [color=blue,solid,forget plot]
  table[row sep=crcr]{
0.0107035849233847	0	\\
0.0107035849233847	0.0755063796891677	\\
};
\addplot [color=blue,solid,forget plot]
  table[row sep=crcr]{
0.0106646435224903	0	\\
0.0106646435224903	0.0760588017963182	\\
};
\addplot [color=blue,solid,forget plot]
  table[row sep=crcr]{
0.0328686301560052	0	\\
0.0328686301560052	0.00800716829460444	\\
};
\addplot [color=blue,solid,forget plot]
  table[row sep=crcr]{
0.0227073149783245	0	\\
0.0227073149783245	0.0167768567971839	\\
};
\addplot [color=blue,solid,forget plot]
  table[row sep=crcr]{
0.0295845241897351	0	\\
0.0295845241897351	0.0098835505574921	\\
};
\addplot [color=blue,solid,forget plot]
  table[row sep=crcr]{
0.0156153450778331	0	\\
0.0156153450778331	0.0354763551513238	\\
};
\addplot [color=blue,solid,forget plot]
  table[row sep=crcr]{
0.014220668159795	0	\\
0.014220668159795	0.0427761960378267	\\
};
\addplot [color=blue,solid,forget plot]
  table[row sep=crcr]{
0.0123088257920767	0	\\
0.0123088257920767	0.0570964248975245	\\
};
\addplot [color=blue,solid,forget plot]
  table[row sep=crcr]{
0.0193021030815244	0	\\
0.0193021030815244	0.0232184329257581	\\
};
\addplot [color=blue,solid,forget plot]
  table[row sep=crcr]{
0.0208054986486501	0	\\
0.0208054986486501	0.0199841604438821	\\
};
\end{axis}
\end{tikzpicture}%
	\caption{Response at RX point.}
	\label{fig:response}
\end{figure}

\section{More general filters}
We now have to look at a filter that can damp different frequencies. 
An example of such a filter can be seen in Equation \ref{eq:filters}.

\begin{equation}
H(z) = \dfrac{1}{1 + az^{-1}}, \quad (|a| < 1)
\label{eq:filters}
\end{equation}

The Matlab scripts we used to plot the figures are included in Appendix Listing \ref{lst:report2.m}.
\\
x(t) consists of the first five seconds ($44100\cdot5 ~\mathrm{samples}$) of the file /bin/Sound/music/interface/Noddinagushpa.mp3 from a Age of Empires III installation, this title screen theme.
It is kind of a pompous drum roll.
\\
\\

In Figure \ref{fig:a0.95-plots} the impulse response of the filter in Equation \ref{eq:filters} is plotted for a = 0.95. 
In Figure \ref{fig:a0.-95-plots} shows the impulse response of the filter for a value of a = -0.95.
From the figures we can clearly see that for a value of a = 0.95 the filter enhanceses the signal at frequencies $\omega = \pi + 2n \pi$ and for a = -0.95 the signals are enhanced for the frequencies $\omega = 2n \pi$. 
When we give the input a different, more general signal, than this signal will also be enhanced at the previously named frequencies. 
\\
\\
For the question why there is no Matlab function "FT" or "DTFT" to compute a (discrete-time) Fourier transform, we can best look at Equation \ref{eq:DTFT} and \ref{eq:DFT}.
Matlab uses numeric calculations, so when assuming we have an infinite amount of frequencies, which is the case with the "DTFT", this can cause problems. 
In the formula for the "DFT", however, we use a finite amount of frequencies, Matlab does support this, so that is why Matlab does have a function that can do the DFT, but not the "FT" or "DTFT".


\begin{equation}
X_{DFT} = \sum_{N} x_N[n]e^{-j2\pi \dfrac{kn}{N}} \quad \quad k = 0, 1, \dots, N - 1 \quad \text{DFT}\\
\label{eq:DFT}
\end{equation}

\begin{equation}
X(\omega) = \sum_{n= -\infty}^{\infty} x[n]e^{-j2\pi fn} \quad \text{DTFT}\\
\label{eq:DTFT}
\end{equation}




\begin{figure}[H]
	\centering
    	\setlength\figureheight{3.5cm}
    	\setlength\figurewidth{0.4\linewidth}
    	% This file was created by matlab2tikz v0.4.6 running on MATLAB 8.3.
% Copyright (c) 2008--2014, Nico Schlömer <nico.schloemer@gmail.com>
% All rights reserved.
% Minimal pgfplots version: 1.3
% 
% The latest updates can be retrieved from
%   http://www.mathworks.com/matlabcentral/fileexchange/22022-matlab2tikz
% where you can also make suggestions and rate matlab2tikz.
% 
\begin{tikzpicture}

\begin{axis}[%
width=\figurewidth,
height=\figureheight,
scale only axis,
xmin=0,
xmax=22050,
xlabel={Frequency (Hz)},
ymin=0,
ymax=0.397631788311866,
ylabel={|IMPULSE(f)|},
name=plot3,
title={Single-Sided Amplitude Spectrum of imp(t)},
legend style={draw=black,fill=white,legend cell align=left}
]
\addplot [color=blue,solid]
  table[row sep=crcr]{
0	0.02	\\
344.53125	0.02	\\
689.0625	0.02	\\
1033.59375	0.02	\\
1378.125	0.02	\\
1722.65625	0.02	\\
2067.1875	0.02	\\
2411.71875	0.02	\\
2756.25	0.02	\\
3100.78125	0.02	\\
3445.3125	0.02	\\
3789.84375	0.02	\\
4134.375	0.02	\\
4478.90625	0.02	\\
4823.4375	0.02	\\
5167.96875	0.02	\\
5512.5	0.02	\\
5857.03125	0.02	\\
6201.5625	0.02	\\
6546.09375	0.02	\\
6890.625	0.02	\\
7235.15625	0.02	\\
7579.6875	0.02	\\
7924.21875	0.02	\\
8268.75	0.02	\\
8613.28125	0.02	\\
8957.8125	0.02	\\
9302.34375	0.02	\\
9646.875	0.02	\\
9991.40625	0.02	\\
10335.9375	0.02	\\
10680.46875	0.02	\\
11025	0.02	\\
11369.53125	0.02	\\
11714.0625	0.02	\\
12058.59375	0.02	\\
12403.125	0.02	\\
12747.65625	0.02	\\
13092.1875	0.02	\\
13436.71875	0.02	\\
13781.25	0.02	\\
14125.78125	0.02	\\
14470.3125	0.02	\\
14814.84375	0.02	\\
15159.375	0.02	\\
15503.90625	0.02	\\
15848.4375	0.02	\\
16192.96875	0.02	\\
16537.5	0.02	\\
16882.03125	0.02	\\
17226.5625	0.02	\\
17571.09375	0.02	\\
17915.625	0.02	\\
18260.15625	0.02	\\
18604.6875	0.02	\\
18949.21875	0.02	\\
19293.75	0.02	\\
19638.28125	0.02	\\
19982.8125	0.02	\\
20327.34375	0.02	\\
20671.875	0.02	\\
21016.40625	0.02	\\
21360.9375	0.02	\\
21705.46875	0.02	\\
22050	0.02	\\
};
\addlegendentry{IMP};

\end{axis}

\begin{axis}[%
width=\figurewidth,
height=\figureheight,
scale only axis,
xmin=0,
xmax=0.00226757369614512,
xlabel={t (s)},
ymin=-0.95,
ymax=1,
ylabel={impulse},
name=plot1,
at=(plot3.above north west),
anchor=below south west,
title={impulse(t)},
legend style={draw=black,fill=white,legend cell align=left}
]
\addplot [color=blue,solid]
  table[row sep=crcr]{
0	1	\\
2.29047848095467e-05	0	\\
4.58095696190934e-05	0	\\
6.87143544286401e-05	0	\\
9.16191392381869e-05	0	\\
0.000114523924047734	0	\\
0.00013742870885728	0	\\
0.000160333493666827	0	\\
0.000183238278476374	0	\\
0.00020614306328592	0	\\
0.000229047848095467	0	\\
0.000251952632905014	0	\\
0.000274857417714561	0	\\
0.000297762202524107	0	\\
0.000320666987333654	0	\\
0.000343571772143201	0	\\
0.000366476556952747	0	\\
0.000389381341762294	0	\\
0.000412286126571841	0	\\
0.000435190911381388	0	\\
0.000458095696190934	0	\\
0.000481000481000481	0	\\
0.000503905265810028	0	\\
0.000526810050619574	0	\\
0.000549714835429121	0	\\
0.000572619620238668	0	\\
0.000595524405048215	0	\\
0.000618429189857761	0	\\
0.000641333974667308	0	\\
0.000664238759476855	0	\\
0.000687143544286402	0	\\
0.000710048329095948	0	\\
0.000732953113905495	0	\\
0.000755857898715042	0	\\
0.000778762683524588	0	\\
0.000801667468334135	0	\\
0.000824572253143682	0	\\
0.000847477037953228	0	\\
0.000870381822762775	0	\\
0.000893286607572322	0	\\
0.000916191392381869	0	\\
0.000939096177191415	0	\\
0.000962000962000962	0	\\
0.000984905746810509	0	\\
0.00100781053162006	0	\\
0.0010307153164296	0	\\
0.00105362010123915	0	\\
0.0010765248860487	0	\\
0.00109942967085824	0	\\
0.00112233445566779	0	\\
0.00114523924047734	0	\\
0.00116814402528688	0	\\
0.00119104881009643	0	\\
0.00121395359490598	0	\\
0.00123685837971552	0	\\
0.00125976316452507	0	\\
0.00128266794933462	0	\\
0.00130557273414416	0	\\
0.00132847751895371	0	\\
0.00135138230376326	0	\\
0.0013742870885728	0	\\
0.00139719187338235	0	\\
0.0014200966581919	0	\\
0.00144300144300144	0	\\
0.00146590622781099	0	\\
0.00148881101262054	0	\\
0.00151171579743008	0	\\
0.00153462058223963	0	\\
0.00155752536704918	0	\\
0.00158043015185872	0	\\
0.00160333493666827	0	\\
0.00162623972147782	0	\\
0.00164914450628736	0	\\
0.00167204929109691	0	\\
0.00169495407590646	0	\\
0.001717858860716	0	\\
0.00174076364552555	0	\\
0.0017636684303351	0	\\
0.00178657321514464	0	\\
0.00180947799995419	0	\\
0.00183238278476374	0	\\
0.00185528756957328	0	\\
0.00187819235438283	0	\\
0.00190109713919238	0	\\
0.00192400192400192	0	\\
0.00194690670881147	0	\\
0.00196981149362102	0	\\
0.00199271627843056	0	\\
0.00201562106324011	0	\\
0.00203852584804966	0	\\
0.0020614306328592	0	\\
0.00208433541766875	0	\\
0.0021072402024783	0	\\
0.00213014498728784	0	\\
0.00215304977209739	0	\\
0.00217595455690694	0	\\
0.00219885934171648	0	\\
0.00222176412652603	0	\\
0.00224466891133558	0	\\
0.00226757369614512	0	\\
};
\addlegendentry{impulse};

\end{axis}

\begin{axis}[%
width=\figurewidth,
height=\figureheight,
scale only axis,
xmin=0,
xmax=0.00226757369614512,
xlabel={t (s)},
ymin=-0.95,
ymax=1,
ylabel={h},
name=plot2,
at=(plot1.right of south east),
anchor=left of south west,
title={h(t)},
legend style={draw=black,fill=white,legend cell align=left}
]
\addplot [color=blue,solid]
  table[row sep=crcr]{
0	1	\\
2.29047848095467e-05	-0.95	\\
4.58095696190934e-05	0.9025	\\
6.87143544286401e-05	-0.857375	\\
9.16191392381869e-05	0.81450625	\\
0.000114523924047734	-0.7737809375	\\
0.00013742870885728	0.735091890625	\\
0.000160333493666827	-0.69833729609375	\\
0.000183238278476374	0.663420431289062	\\
0.00020614306328592	-0.630249409724609	\\
0.000229047848095467	0.598736939238379	\\
0.000251952632905014	-0.56880009227646	\\
0.000274857417714561	0.540360087662637	\\
0.000297762202524107	-0.513342083279505	\\
0.000320666987333654	0.487674979115529	\\
0.000343571772143201	-0.463291230159753	\\
0.000366476556952747	0.440126668651765	\\
0.000389381341762294	-0.418120335219177	\\
0.000412286126571841	0.397214318458218	\\
0.000435190911381388	-0.377353602535307	\\
0.000458095696190934	0.358485922408542	\\
0.000481000481000481	-0.340561626288115	\\
0.000503905265810028	0.323533544973709	\\
0.000526810050619574	-0.307356867725023	\\
0.000549714835429121	0.291989024338772	\\
0.000572619620238668	-0.277389573121834	\\
0.000595524405048215	0.263520094465742	\\
0.000618429189857761	-0.250344089742455	\\
0.000641333974667308	0.237826885255332	\\
0.000664238759476855	-0.225935540992565	\\
0.000687143544286402	0.214638763942937	\\
0.000710048329095948	-0.20390682574579	\\
0.000732953113905495	0.193711484458501	\\
0.000755857898715042	-0.184025910235576	\\
0.000778762683524588	0.174824614723797	\\
0.000801667468334135	-0.166083383987607	\\
0.000824572253143682	0.157779214788227	\\
0.000847477037953228	-0.149890254048815	\\
0.000870381822762775	0.142395741346375	\\
0.000893286607572322	-0.135275954279056	\\
0.000916191392381869	0.128512156565103	\\
0.000939096177191415	-0.122086548736848	\\
0.000962000962000962	0.115982221300006	\\
0.000984905746810509	-0.110183110235005	\\
0.00100781053162006	0.104673954723255	\\
0.0010307153164296	-0.0994402569870922	\\
0.00105362010123915	0.0944682441377376	\\
0.0010765248860487	-0.0897448319308507	\\
0.00109942967085824	0.0852575903343082	\\
0.00112233445566779	-0.0809947108175928	\\
0.00114523924047734	0.0769449752767131	\\
0.00116814402528688	-0.0730977265128775	\\
0.00119104881009643	0.0694428401872336	\\
0.00121395359490598	-0.0659706981778719	\\
0.00123685837971552	0.0626721632689783	\\
0.00125976316452507	-0.0595385551055294	\\
0.00128266794933462	0.0565616273502529	\\
0.00130557273414416	-0.0537335459827403	\\
0.00132847751895371	0.0510468686836033	\\
0.00135138230376326	-0.0484945252494231	\\
0.0013742870885728	0.0460697989869519	\\
0.00139719187338235	-0.0437663090376043	\\
0.0014200966581919	0.0415779935857241	\\
0.00144300144300144	-0.0394990939064379	\\
0.00146590622781099	0.037524139211116	\\
0.00148881101262054	-0.0356479322505602	\\
0.00151171579743008	0.0338655356380322	\\
0.00153462058223963	-0.0321722588561306	\\
0.00155752536704918	0.0305636459133241	\\
0.00158043015185872	-0.0290354636176579	\\
0.00160333493666827	0.027583690436775	\\
0.00162623972147782	-0.0262045059149362	\\
0.00164914450628736	0.0248942806191894	\\
0.00167204929109691	-0.0236495665882299	\\
0.00169495407590646	0.0224670882588184	\\
0.001717858860716	-0.0213437338458775	\\
0.00174076364552555	0.0202765471535836	\\
0.0017636684303351	-0.0192627197959044	\\
0.00178657321514464	0.0182995838061092	\\
0.00180947799995419	-0.0173846046158038	\\
0.00183238278476374	0.0165153743850136	\\
0.00185528756957328	-0.0156896056657629	\\
0.00187819235438283	0.0149051253824748	\\
0.00190109713919238	-0.014159869113351	\\
0.00192400192400192	0.0134518756576835	\\
0.00194690670881147	-0.0127792818747993	\\
0.00196981149362102	0.0121403177810593	\\
0.00199271627843056	-0.0115333018920064	\\
0.00201562106324011	0.010956636797406	\\
0.00203852584804966	-0.0104088049575357	\\
0.0020614306328592	0.00988836470965895	\\
0.00208433541766875	-0.009393946474176	\\
0.0021072402024783	0.0089242491504672	\\
0.00213014498728784	-0.00847803669294384	\\
0.00215304977209739	0.00805413485829665	\\
0.00217595455690694	-0.00765142811538181	\\
0.00219885934171648	0.00726885670961272	\\
0.00222176412652603	-0.00690541387413209	\\
0.00224466891133558	0.00656014318042548	\\
0.00226757369614512	-0.00623213602140421	\\
};
\addlegendentry{h};

\end{axis}

\begin{axis}[%
width=\figurewidth,
height=\figureheight,
scale only axis,
xmin=0,
xmax=22050,
xlabel={Frequency (Hz)},
ymin=0,
ymax=0.397631788311866,
ylabel={|H(f)|},
name=plot4,
at=(plot2.below south west),
anchor=above north west,
title={Single-Sided Amplitude Spectrum of h(t)},
legend style={draw=black,fill=white,legend cell align=left}
]
\addplot [color=blue,solid]
  table[row sep=crcr]{
0	0.0101956868797915	\\
344.53125	0.0102478212500977	\\
689.0625	0.0103249662150964	\\
1033.59375	0.0103182094518941	\\
1378.125	0.0102629489357736	\\
1722.65625	0.0102832602374374	\\
2067.1875	0.010392206761738	\\
2411.71875	0.0104700148553275	\\
2756.25	0.0104573924752851	\\
3100.78125	0.0104506193076926	\\
3445.3125	0.0105492633439215	\\
3789.84375	0.0106940771871071	\\
4134.375	0.0107625582779568	\\
4478.90625	0.010765801705408	\\
4823.4375	0.010833169425015	\\
5167.96875	0.0110053660550897	\\
5512.5	0.0111665551828533	\\
5857.03125	0.0112313025095185	\\
6201.5625	0.0112828611051312	\\
6546.09375	0.0114438752641003	\\
6890.625	0.0116773128272677	\\
7235.15625	0.0118446408928542	\\
7579.6875	0.0119293186329924	\\
7924.21875	0.012067791482453	\\
8268.75	0.0123336858872925	\\
8613.28125	0.0126155621282175	\\
8957.8125	0.0127961236774446	\\
9302.34375	0.0129434640638808	\\
9646.875	0.0132097851648812	\\
9991.40625	0.0135864156097986	\\
10335.9375	0.0139142157802257	\\
10680.46875	0.0141411483832134	\\
11025	0.0144141410652619	\\
11369.53125	0.0148519526687185	\\
11714.0625	0.0153499749923172	\\
12058.59375	0.0157473281460711	\\
12403.125	0.016091880901799	\\
12747.65625	0.0165798899857881	\\
13092.1875	0.0172463908217395	\\
13436.71875	0.0179037805783351	\\
13781.25	0.0184477961418328	\\
14125.78125	0.019044031593775	\\
14470.3125	0.0198870269364149	\\
14814.84375	0.0208895334934134	\\
15159.375	0.0218236713777593	\\
15503.90625	0.0227075263729579	\\
15848.4375	0.0238223772717253	\\
16192.96875	0.0252967619770434	\\
16537.5	0.0269084624960711	\\
16882.03125	0.0284689084487852	\\
17226.5625	0.0302005075785837	\\
17571.09375	0.032491991898594	\\
17915.625	0.0353543971677623	\\
18260.15625	0.0384795835479073	\\
18604.6875	0.0418969513891136	\\
18949.21875	0.0462390333004053	\\
19293.75	0.0521421134813846	\\
19638.28125	0.0596925515142661	\\
19982.8125	0.0690352850227611	\\
20327.34375	0.0816295699416506	\\
20671.875	0.100840550278553	\\
21016.40625	0.13212595358944	\\
21360.9375	0.186318281696946	\\
21705.46875	0.288688597925754	\\
22050	0.397631788311866	\\
};
\addlegendentry{H};

\end{axis}

\begin{axis}[%
width=\figurewidth,
height=\figureheight,
scale only axis,
xmin=0,
xmax=5,
xlabel={t (s)},
ymin=-0.887386679649353,
ymax=0.892788469791412,
ylabel={y},
name=plot6,
at=(plot4.below south west),
anchor=above north west,
title={y(t)},
legend style={draw=black,fill=white,legend cell align=left}
]
\addplot [color=blue,solid]
  table[row sep=crcr]{
0	0	\\
0	0	\\
0	0	\\
0.00777781305130635	0	\\
0.00777781305130635	0	\\
0.0155556261026127	0	\\
0.0155556261026127	0	\\
0.0233334391539191	0	\\
0.0233334391539191	0	\\
0.0311112522052254	0	\\
0.0311112522052254	0	\\
0.0461226581526447	-0.00013913696209495	\\
0.046349416550642	0.000226859824353896	\\
0.0529707617721622	0.00112547119158476	\\
0.0544446913591445	-0.00566890798419156	\\
0.0579367706883024	-0.0298560771973848	\\
0.0622225044104508	0.0962336029663559	\\
0.0625626420074467	0.0999768797813114	\\
0.0696375040249616	-0.187388178771308	\\
0.0700003174617572	-0.172499085876412	\\
0.0726080390387258	0.219467478231748	\\
0.0812702098422215	0.136094043799499	\\
0.0840366622977882	-0.232617501131109	\\
0.0883677476995361	-0.142174894506987	\\
0.0932203774166776	0.140680734907001	\\
0.0940367076494678	0.213432273197069	\\
0.096508374187638	-0.0844700590701263	\\
0.101587762302777	0.0798941561045421	\\
0.105147869151334	-0.151906005225159	\\
0.112064000290251	-0.08999893292199	\\
0.114286232590624	0.112907154264879	\\
0.116735223288995	0.088951978264977	\\
0.118390559594375	-0.070188706311392	\\
0.125964290087483	-0.0665178417108565	\\
0.129479045256441	0.0676580027777285	\\
0.133606048099991	0.0849405567124083	\\
0.138549381176332	-0.0528803826231581	\\
0.143470038412873	-0.0876984958154694	\\
0.146644655984834	0.0653254028985184	\\
0.151928126658171	-0.0627413248146394	\\
0.152767132730761	0.0780432884658441	\\
0.156735404695713	0.0866336469502963	\\
0.160817055859664	-0.0651349728688567	\\
0.167574456119982	-0.0851971036532501	\\
0.170091474337752	0.105428041862286	\\
0.171247942167538	0.0515493121479222	\\
0.178640265942249	-0.0406780649438895	\\
0.181973614392809	-0.0530681773308068	\\
0.185420342042368	0.0421179406616641	\\
0.191792253026091	-0.0429393930040333	\\
0.194218567884662	0.0607523952993385	\\
0.194672084680656	0.0676743523088733	\\
0.201180050703178	-0.032549330244663	\\
0.202858062848358	-0.0547848640090569	\\
0.209411380550479	0.0129439856240408	\\
0.210839958457861	-0.017603135642811	\\
0.212109805486646	0.0537899083552565	\\
0.217733413756978	0.0171301053881748	\\
0.225239116730688	-0.0559182242928929	\\
0.225669957686883	-0.0425136186890117	\\
0.231860461952208	0.0215782956185626	\\
0.235148458723169	0.0528519727149747	\\
0.239411516605517	-0.0408310479241139	\\
0.241951210663087	-0.042976838860227	\\
0.248595231724407	0.0154228068775323	\\
0.254082784955941	0.0499351509856891	\\
0.256168962217516	-0.0246262552490396	\\
0.262903686638035	-0.0333223520820025	\\
0.26412818198722	0.0259251211453125	\\
0.266078304209996	-0.019904453552481	\\
0.270227982893346	0.0415167578251382	\\
0.273742738062304	0.0132196322476193	\\
0.279343670492837	-0.0244484004405553	\\
0.284468410287575	-0.0201280345198422	\\
0.287008104345144	0.0262411532394399	\\
0.288550061451526	-0.0288371827545261	\\
0.291611299824489	0.0306677002719239	\\
0.295806330187439	-0.0248335736875856	\\
0.298867568560402	0.0241677741252976	\\
0.310250840139865	-0.0486520176363457	\\
0.311044494532855	0.0887511515618802	\\
0.315987827609196	0.130456066864542	\\
0.317620488074776	-0.113203522785613	\\
0.318867659263761	-0.0956022246681817	\\
0.324173805776897	0.12714201233071	\\
0.330817826838217	-0.0918777988051955	\\
0.333244141696788	0.0736468234074403	\\
0.33718973782194	-0.0844396997740356	\\
0.341928988340083	0.156714025912666	\\
0.345126281751845	-0.160352037964619	\\
0.348731740280001	0.142294460242937	\\
0.349978911468986	0.128576809001054	\\
0.352382550487757	-0.152429453335286	\\
0.362881464315031	0.158707777070022	\\
0.365149048295004	-0.0924915038044499	\\
0.369026616900757	0.069956877604629	\\
0.373017564705509	-0.125522638139044	\\
0.373266998943306	-0.0914512837400594	\\
0.377462029306255	0.137706408362172	\\
0.38455956716357	0.0823295903170899	\\
0.387462074657935	-0.10026484493891	\\
0.389865713676706	0.0495044290796293	\\
0.395784107864435	-0.0464307666497305	\\
0.398618587839401	0.0747851516767824	\\
0.400455330863179	-0.0645361031528988	\\
0.404650361226128	0.0441082832471983	\\
0.411861278282441	-0.0859504823140502	\\
0.413267180350024	-0.0613234800859509	\\
0.419503036294949	0.0712811411893376	\\
0.420047256450143	0.0774826628446821	\\
0.427620986943251	-0.0674684645985713	\\
0.428051827899446	-0.0769272907967167	\\
0.433244595213584	0.0520054387033568	\\
0.436441888625345	-0.0547098317444652	\\
0.438845527644116	0.0740626093489068	\\
0.44680474741382	0.0320013680454919	\\
0.449026979714194	-0.0394839035706582	\\
0.452813844960748	-0.032710321856475	\\
0.458686887468877	0.0439785592705443	\\
0.459072376745473	0.0471661082135991	\\
0.464401199098409	-0.0328843971949058	\\
0.468709608660357	-0.0468905843736013	\\
0.473720969256096	0.0452905879782281	\\
0.477031641866857	-0.0258572992406136	\\
0.479095143288632	0.0472692173774077	\\
0.483970448845573	-0.0463549126839899	\\
0.487054363058336	0.022772841138426	\\
0.489956870552701	-0.0342161728295191	\\
0.494945555308641	0.0350127947311509	\\
0.500365081020776	0.0418706769322977	\\
0.505081655699119	-0.0309718583693827	\\
0.508483031669078	-0.0260854455154101	\\
0.510024988775459	0.00580076143572961	\\
0.513244958027021	-0.0127891904031988	\\
0.519027297175951	0.0342593381136496	\\
0.521476287874321	0.0042560984142442	\\
0.528755232450034	-0.0435540295573166	\\
0.528777908289834	-0.0417866129591578	\\
0.535308550152155	0.0248445282139256	\\
0.538823305321113	0.0263219023182344	\\
0.54233806049007	-0.0233337042482627	\\
0.548324482197198	0.0103915154805643	\\
0.550728121215969	-0.0282451978882674	\\
0.555512723413712	0.0209204838108265	\\
0.559208885301067	-0.00686611290346879	\\
0.561567172640239	0.025426575599867	\\
0.565444741245992	-0.0525561032987236	\\
0.569322309851745	-0.0575279780486475	\\
0.573335933496297	0.0641051688923812	\\
0.579571789441222	0.172966309522553	\\
0.582451621095787	-0.143434168842371	\\
0.586011727944344	-0.162807920219073	\\
0.588301987764117	0.139264820342618	\\
0.591091116059483	-0.105137299040126	\\
0.592837155724062	0.143185868871868	\\
0.599571880144581	-0.0957497475873898	\\
0.601816788284754	0.110299811496941	\\
0.610252200690253	0.101348889141767	\\
0.612020916194631	-0.12097312367102	\\
0.62041097692053	-0.186289420199448	\\
0.621748851468714	0.133361411442775	\\
0.625354309996871	-0.172559585903169	\\
0.629708071238418	0.122650034108936	\\
0.630229615553812	0.153062751620148	\\
0.633245502247176	-0.123208999404681	\\
0.638460945401113	-0.0965035507834979	\\
0.643608361035651	0.115423786751968	\\
0.645603834938027	-0.131313810544956	\\
0.651340822407358	0.133190512446031	\\
0.657055134036889	0.182139579951946	\\
0.660683268404845	-0.144128586254276	\\
0.663744506777809	0.112253202098832	\\
0.667531372024363	-0.157616764988186	\\
0.671363588950517	0.120493886645077	\\
0.675535943473667	-0.124196663680906	\\
0.678007610011837	0.127152399460625	\\
0.680547304069406	-0.116916004801354	\\
0.685037120349752	0.125532501384614	\\
0.688347792960512	-0.158817020110376	\\
0.695898847613821	-0.135263135053568	\\
0.698370514151992	0.133877313562754	\\
0.701635835083152	-0.101917779808361	\\
0.707100712474886	0.0883538275756916	\\
0.709005483018064	-0.0951559254184154	\\
0.715264014802788	0.104005876098502	\\
0.720751568034322	0.0776074843360687	\\
0.722361552660103	-0.13746838270683	\\
0.727304885736443	0.109953303924213	\\
0.730819640905401	-0.130985673309945	\\
0.733177928244573	0.141602474631111	\\
0.736239166617536	-0.12835398881821	\\
0.742361643363462	0.102969893968604	\\
0.744538523984236	-0.158706846698043	\\
0.747214273080604	0.109629669181776	\\
0.752701826312137	-0.126988035332514	\\
0.755286872049306	0.121106115724152	\\
0.758642896339666	-0.0756038628494884	\\
0.765559027478583	-0.130211525249413	\\
0.768098721536152	0.10377697122905	\\
0.77161347670511	-0.0880657493928145	\\
0.774402605000476	0.102363378614902	\\
0.782928720765174	0.0616102091888266	\\
0.785196304745146	-0.0952581994618407	\\
0.785581794021742	-0.0685676389502425	\\
0.787849378001714	0.0818495951625693	\\
0.797554637435997	-0.0653181849784566	\\
0.798575050226985	0.0702081006329033	\\
0.804946961210708	0.080989930910742	\\
0.808620447258264	-0.0871175997197856	\\
0.809210019093057	-0.115016287816045	\\
0.812475340024218	0.117581145789019	\\
0.818461761731346	0.122978642492232	\\
0.822588764574896	-0.0952900484088247	\\
0.824606914317072	0.0860096752582214	\\
0.830638687703799	-0.0779726165550496	\\
0.832951623363371	0.0648771167107668	\\
0.833994711994159	-0.0484664046914902	\\
0.840888167293276	0.061295291487654	\\
0.845899527889015	-0.0601261037104418	\\
0.850548075047959	-0.0594961037690699	\\
0.852906362387131	0.0583603757922033	\\
0.859074190812657	0.0497801863800732	\\
0.862725001020413	-0.0602281553811227	\\
0.867577630737554	-0.0630804276120801	\\
0.870230703994122	0.103425939833811	\\
0.875536850507258	-0.071590255377888	\\
0.87705613177384	0.117198218450614	\\
0.87873414391902	0.11365459881989	\\
0.880344128544801	-0.123056536880317	\\
0.891636696765065	0.0725431242964242	\\
0.893836253225638	-0.0717472819572377	\\
0.896693409040404	0.0773808299729845	\\
0.90202223139334	-0.0969646868356851	\\
0.90202223139334	-0.0969646868356851	\\
0.90449389793151	0.0839905000409413	\\
0.912634524419612	0.0984404336937616	\\
0.915922521190572	-0.114141398311539	\\
0.917872643413349	0.0972616392411829	\\
0.921296695223108	-0.0868461271161534	\\
0.925537077265656	0.0623993684745171	\\
0.929165211633613	-0.0673348409820557	\\
0.9385076576311	0.0568883692274955	\\
0.940094966417081	-0.0703728896722266	\\
0.944970271974023	-0.123750230501598	\\
0.947101800915197	0.0701147295580914	\\
0.952362595748734	-0.0652488272559467	\\
0.954811586447104	0.100004703981779	\\
0.959119996009052	-0.0567932095801899	\\
0.963541784769999	0.0627434645684214	\\
0.966194858026567	-0.0792096774478817	\\
0.971319597821305	0.075798241510422	\\
0.97671644769364	0.0790776504823414	\\
0.97923346591141	-0.0556806147351806	\\
0.981092884774988	-0.079897533466395	\\
0.98424482650715	0.0535119532115102	\\
0.989528297180486	-0.0458075609673952	\\
0.994244871858829	0.049637191036378	\\
0.999052149896371	0.096995721190622	\\
1.00193198155094	-0.0885987675812125	\\
1.00315647690012	0.0460012713277605	\\
1.00946036036445	-0.0688198034239854	\\
1.01336060481	0.0963499546205192	\\
1.01678465661976	-0.0722232369765037	\\
1.02061687354591	0.10331403187528	\\
1.02358740855968	-0.111671002755802	\\
1.02737427380623	0.0876016048063069	\\
1.0299819953832	-0.0795884315867253	\\
1.03426772910535	0.085177825714314	\\
1.03891627626429	-0.0877014816014432	\\
1.04442650533563	-0.0603063813612934	\\
1.04923378337317	0.0528307108272855	\\
1.05281656606152	-0.0739457142087214	\\
1.0552202050803	0.0778621892896562	\\
1.05814538841446	-0.0641290237347112	\\
1.06132000598642	0.0475882958924829	\\
1.06605925650456	-0.0472154376247427	\\
1.06891641231933	0.0596659715323086	\\
1.07567381257965	-0.0912615731016223	\\
1.0799141946222	0.111608884053916	\\
1.08385979074735	-0.169956476264	\\
1.0882362278287	0.256627218680666	\\
1.09070789436687	-0.242749337288509	\\
1.09356505018163	0.12973701442831	\\
1.10143356659214	0.343505525713943	\\
1.10374650225171	-0.14925210839009	\\
1.10961954475984	0.115819503764573	\\
1.11132023274482	-0.159745512435539	\\
1.1130889482492	-0.200081486634816	\\
1.11764679204894	0.122032585068237	\\
1.12234069088749	0.156935813935922	\\
1.12519784670225	-0.100901951111989	\\
1.13247679127797	-0.13511429092323	\\
1.13526591957333	0.0830146241015446	\\
1.13556070549073	0.10652603658478	\\
1.1381457512279	-0.058713856609479	\\
1.14705735626919	-0.112783934263892	\\
1.14866734089497	0.0654632629733874	\\
1.15311180549572	-0.0624679487739983	\\
1.15757894593626	0.0893461849571201	\\
1.15864471040685	0.0651201058856205	\\
1.16347466428419	-0.0497509865588108	\\
1.16805518392374	-0.0868510386745039	\\
1.17168331829169	0.072161928389859	\\
1.17467652914526	-0.0654062587505399	\\
1.17866747695001	0.0831724372936165	\\
1.18213688043937	0.0433552934123406	\\
1.18898498405888	-0.0559468679017711	\\
1.19249973922784	-0.0672210105521669	\\
1.19537957088241	0.086317626263479	\\
1.19782856158078	-0.0298275881827903	\\
1.20211429530293	0.0427203009327684	\\
1.20889437140304	-0.0518820089256214	\\
1.21145674130041	0.0469139713158138	\\
1.21698964621155	-0.0521969380898763	\\
1.21959736778852	0.0717398047728176	\\
1.22281733704008	0.0437377330697839	\\
1.22828221443181	-0.04080805541044	\\
1.22991487489739	-0.0534273503802231	\\
1.23640016508011	0.0233987334204009	\\
1.23907591417648	0.0422456716475182	\\
1.24401924725282	-0.0180704452454752	\\
1.24674034802879	-0.0340558264972458	\\
1.24912131120776	0.0041809401592561	\\
1.25279479725532	-0.0316109709806111	\\
1.25957487335543	0.0419672008353297	\\
1.26043655526782	0.04779682348544	\\
1.26683114209135	-0.0568085899311739	\\
1.26760212064454	-0.0488758412636346	\\
1.27517585113765	0.0317131194322641	\\
1.27821441367081	0.0503261385788516	\\
1.28134367956317	-0.0401791229931213	\\
1.28401942865954	-0.0278374332254689	\\
1.28887205837668	0.0164485080505212	\\
1.29120766987605	-0.021260949023953	\\
1.29490383176341	0.0320043466138831	\\
1.29884942788856	0.0202174531670934	\\
1.3042689536007	-0.0281053165292459	\\
1.31007396858943	-0.038956843729551	\\
1.3120240908122	0.0276469546496614	\\
1.32002866226151	0.104996579002358	\\
1.32177470192609	-0.113819996237819	\\
1.32662733164323	0.113864730130313	\\
1.32971124585599	-0.156143051186693	\\
1.33327135270455	0.152902908737936	\\
1.33653667363571	-0.208827953337814	\\
1.34284055710003	0.155859766436102	\\
1.34445054172581	-0.261219501147527	\\
1.34712629082218	0.229744386891516	\\
1.35304468500991	-0.142127788996553	\\
1.35803336976585	-0.203601029475943	\\
1.36064109134282	0.121655140644839	\\
1.36138939405621	0.140613087763677	\\
1.36660483721015	-0.12355804576298	\\
1.36878171783092	0.0969054599186952	\\
1.3708452192527	-0.122190171420057	\\
1.3790765491	-0.113454133547088	\\
1.38186567739536	0.121677575566005	\\
1.3874439339861	0.0881513171008972	\\
1.39191107442664	-0.092448073662523	\\
1.39193375026644	-0.0853973933703269	\\
1.39413330672701	0.0450718309408208	\\
1.40179774057932	0.081257564632405	\\
1.40195647145792	-0.0403607018867964	\\
1.40980231202863	0.0355605222301342	\\
1.41381593567318	-0.0679193380085312	\\
1.41549394781836	-0.0553579745897636	\\
1.41746674588093	0.0569669825033324	\\
1.42302232663187	-0.045400327884249	\\
1.42547131733024	0.0603685375246601	\\
1.43138971151797	0.0469495221440187	\\
1.43406546061433	-0.0623873883951654	\\
1.44123102599105	-0.0365022664152631	\\
1.4449045120386	0.0606345008671768	\\
1.44687731010118	0.0401523807660294	\\
1.45408822715749	-0.0354358452102564	\\
1.45481385403108	-0.0458300018792998	\\
1.45953042870943	0.0433507346246534	\\
1.46572093297475	0.0510284897045702	\\
1.46930371566311	-0.0473923557480907	\\
1.47061891437149	0.01338086812247	\\
1.47581168168563	-0.0290725898260289	\\
1.47787518310741	-0.0250270919848536	\\
1.48470061088712	0.0473717670949288	\\
1.48792058013869	0.0382128909011459	\\
1.49207025882204	-0.0509668196484677	\\
1.495131497195	-0.0327821046484231	\\
1.49787527381077	0.0226319602264317	\\
1.5031133928045	0.035242523104131	\\
1.50837418763804	-0.0273144471114623	\\
1.51336287239398	-0.0322167212996343	\\
1.51608397316995	0.00829502031038139	\\
1.51764860611613	0.0272448737704318	\\
1.52141279552288	-0.0147034319361715	\\
1.52567585340523	0.0325008367787035	\\
1.53102735159797	-0.0312748300965787	\\
1.53438337588833	0.0458377990661885	\\
1.5369230699459	-0.0570689751438359	\\
1.54093669359045	0.0862282197810442	\\
1.54540383403099	-0.0989293589373196	\\
1.54866915496215	0.0843650917295318	\\
1.55188912421372	-0.0958521095829383	\\
1.55948553054662	-0.0663960968166665	\\
1.5612995977306	0.0902538768870076	\\
1.56324971995338	0.0455384384436593	\\
1.56769418455413	-0.0635428890127464	\\
1.57476904657164	-0.0501899638094294	\\
1.57739944398841	0.0676321084291925	\\
1.58159447435136	0.175473015183519	\\
1.58431557512732	-0.126639131197452	\\
1.58987115587826	0.14025737105812	\\
1.59297774593082	-0.209358322085758	\\
1.59567617086699	0.139113665502083	\\
1.59907754683695	-0.0875632941840882	\\
1.6028870879233	0.138510651453859	\\
1.60656057397086	-0.0981627863665168	\\
1.60982589490202	-0.0812273281503222	\\
1.61626583340514	0.0945943746063352	\\
1.62275112358786	0.0901039466995612	\\
1.62508673508723	-0.0648935961124373	\\
1.63061963999837	-0.120758192843551	\\
1.63175343198835	0.16400461187241	\\
1.63486002204092	-0.166472021592618	\\
1.63844280472927	0.142455786580317	\\
1.64302332436882	-0.175187106844673	\\
1.64592583186318	0.190207467078465	\\
1.6526832321235	0.136344323820202	\\
1.65529095370047	-0.176891064580944	\\
1.65764924103964	0.123707779298758	\\
1.66359031106717	-0.105907113175761	\\
1.66576719168794	0.105511232163889	\\
1.66826153406591	-0.109576617416032	\\
1.67422527993324	0.081632229125612	\\
1.67647018807342	-0.115520897147701	\\
1.68125479027116	0.0879517018583181	\\
1.68465616624112	-0.0671336633428513	\\
1.69193511081683	-0.115355916379371	\\
1.6940212880784	0.112869301024913	\\
1.69567662438378	-0.0940956508457431	\\
1.70136826017352	0.0842680185260933	\\
1.70476963614347	0.0805049354038287	\\
1.70885128730743	-0.107524906693038	\\
1.7110961954476	0.17952046277889	\\
1.71495108821355	-0.120229848005608	\\
1.7190780910571	0.112711003233075	\\
1.72125497167788	-0.101080240841404	\\
1.72871532297199	-0.0824756440643307	\\
1.73213937478175	0.0778279902083949	\\
1.73470174467911	-0.0911352585834232	\\
1.73944099519726	0.0736949729362452	\\
1.74855668279675	0.0702148312327265	\\
1.74957709558774	-0.101042475417309	\\
1.75055215669912	-0.122680122358012	\\
1.75313720243629	0.0876772160280362	\\
1.76141388396319	0.11477486572265	\\
1.76463385321475	-0.123664630770872	\\
1.76982662052889	0.058814020019538	\\
1.77220758370786	-0.0643242076609682	\\
1.77526882208083	-0.100602712584875	\\
1.7775590819006	0.0739361226763027	\\
1.78245706329734	0.130654883985615	\\
1.78708293461648	-0.107032230163184	\\
1.7940897691146	-0.118443650018195	\\
1.79628932557517	0.128176637994942	\\
1.79690157324977	0.140155586978789	\\
1.79901042635114	-0.0757405603598796	\\
1.80442995206327	0.0676752821300805	\\
1.80599458500946	-0.0951775340654736	\\
1.81467943165275	-0.0810647270535758	\\
1.81692433979292	0.057438407231156	\\
1.82084726007828	0.0917946221507969	\\
1.82365906421344	-0.0839124718974508	\\
1.82769536369779	-0.109566300740886	\\
1.82966816176037	0.0818566747991433	\\
1.83794484328727	0.0876447792798085	\\
1.84087002662144	-0.0608110666731441	\\
1.84322831396061	0.0849934041522133	\\
1.85068866525472	-0.0881316932015742	\\
1.85526918489426	0.0962783335477347	\\
1.85835309910703	-0.0968019681657014	\\
1.85923745685921	-0.118825686691592	\\
1.86132363412079	0.131368717890853	\\
1.86814906190051	0.12066621152544	\\
1.87043932172028	-0.112620722510917	\\
1.88028063619336	-0.0808569720596186	\\
1.88132372482415	0.113319432555116	\\
1.88234413761514	0.110926573239311	\\
1.88447566655631	-0.0813727082912588	\\
1.8932965682384	0.0790117571942087	\\
1.89633513077157	-0.106517015391877	\\
1.89855736307194	0.140521407303485	\\
1.90513335661386	-0.0899590435809569	\\
1.90515603245366	-0.0923702603551406	\\
1.90796783658883	0.0627378261525091	\\
1.91347806566016	-0.0647095157455819	\\
1.92046222431848	0.0775709588448409	\\
1.92143728542987	0.0775549310936156	\\
1.9265620252246	-0.0577181161176335	\\
1.92880693336478	-0.0552131522607914	\\
1.92912439512197	0.0435422615568169	\\
1.93862557199806	0.0970196124002167	\\
1.94186821708942	-0.06554123944352	\\
1.94522424137978	0.111165035987538	\\
1.95028095365512	-0.109547543597681	\\
1.95316078530968	0.174802067920147	\\
1.95554174848866	-0.139372012311566	\\
1.96411321593295	-0.094580950125959	\\
1.96656220663132	0.120902511191229	\\
1.96737853686411	0.0623525638851418	\\
1.9685350046939	-0.103772362363946	\\
1.97733323053619	-0.0735305981742717	\\
1.97882983596297	0.0794774908755856	\\
1.98656229733468	0.0738824379082226	\\
1.98966888738724	-0.0785887111477724	\\
1.99075732769763	-0.0617367056689284	\\
1.9929568841582	0.0682335152587984	\\
1.99860316826834	0.0854383469595707	\\
2.00502043093166	-0.0775044010154487	\\
2.00776420754743	0.0839784107488437	\\
2.00917010961501	-0.115866974926663	\\
2.01690257098672	-0.117760002304371	\\
2.02059873287407	0.104877456164339	\\
2.02211801414065	0.0842609330538747	\\
2.02474841155742	-0.0881884097391206	\\
2.03116567422074	-0.101666380059142	\\
2.0350885945061	0.0748504718187775	\\
2.04125642293162	0.0775552333265446	\\
2.04504328817818	-0.0736072874767352	\\
2.04991859373512	0.0560408695525959	\\
2.05257166699169	-0.0869851031411725	\\
2.05517938856866	0.0592130066769387	\\
2.06041750756239	-0.107876238307284	\\
2.06454451040594	-0.153386558343321	\\
2.0679685622157	0.184269250935745	\\
2.07254908185525	-0.219861879567619	\\
2.07524750679141	0.200730362780673	\\
2.0762225679028	0.145183362070665	\\
2.07835409684398	-0.214314783284346	\\
2.08595050317689	-0.168623419075119	\\
2.08699359180767	0.136561852632161	\\
2.09361493702919	-0.236209289654547	\\
2.09649476868376	0.213251768218915	\\
2.10443131261366	0.168316979760756	\\
2.10690297915183	-0.224766957546195	\\
2.10746987514683	-0.165073426686473	\\
2.11189166390777	0.16931633191429	\\
2.11735654129951	0.16890215095711	\\
2.11985088367748	-0.29710013827854	\\
2.12300282540964	-0.0737931335845727	\\
2.12459013419562	0.188846281778888	\\
2.13066725926195	0.0823354964164559	\\
2.13644959841088	-0.123532635946301	\\
2.14062195293403	-0.142259192071639	\\
2.14592809944716	0.14292938150715	\\
2.14629091288396	0.127823873816822	\\
2.15032721236831	-0.0395825822443846	\\
2.1594428999678	0.022309290769173	\\
2.16155175306917	-0.0905275303635699	\\
2.16445426056354	-0.0451606661473748	\\
2.16629100358732	0.0890500867875762	\\
2.17005519299407	0.0620019381307025	\\
2.17359262400283	-0.0851689854398405	\\
2.17828652284137	-0.0471237497348032	\\
2.18493054390269	0.072568935326918	\\
2.18565617077628	0.0653589846289802	\\
2.19071288305162	-0.0382857657783478	\\
2.19656324971995	-0.0662868598586192	\\
2.19914829545712	0.0564633155962145	\\
2.20291248486388	-0.0892438347167524	\\
2.20581499235824	0.112680224896814	\\
2.20855876897401	0.0380378837650895	\\
2.21100775967238	-0.0428078260047103	\\
2.22001006807287	-0.0541721848417171	\\
2.22239103125184	0.0708348015631994	\\
2.22413707091642	-0.0367146309904125	\\
2.226223248178	0.0477024757911011	\\
2.23241375244332	0.0516906812656285	\\
2.23581512841328	-0.0494557496142375	\\
2.24393307906158	-0.0671739189054787	\\
2.24692628991515	0.0826993023045546	\\
2.24728910335194	0.0679706343524588	\\
2.25035034172491	-0.0115432301395228	\\
2.25737985206282	-0.0784300900026466	\\
2.26261797105656	0.020619398345883	\\
2.2670851114971	0.0596893841445798	\\
2.27055451498646	-0.0218596648229963	\\
2.27629150245579	0.00756822293922908	\\
2.27737994276618	-0.0451379798704926	\\
2.28028245026055	-0.023177077643134	\\
2.28617816860847	0.0256264129919229	\\
2.28713055388006	0.0567911563465892	\\
2.29372922326178	-0.0523509897291071	\\
2.29400133333938	-0.0508366288463092	\\
2.30123492623549	0.0323403705529634	\\
2.30617825931183	0.0409251158328194	\\
2.3087859808888	-0.0352408213693725	\\
2.31502183683373	-0.0307339996346193	\\
2.31617830466351	0.0301931133322681	\\
2.31826448192509	-0.0189453605366823	\\
2.32248218812784	0.0367768385015952	\\
2.33164322740693	-0.0512220905840573	\\
2.33277701939691	0.0483540915413925	\\
2.33586093360968	-0.0883877937430649	\\
2.33948906797763	0.0981909467003822	\\
2.34411493929678	-0.0941893418576561	\\
2.34835532133933	0.175078980193531	\\
2.34837799717913	0.173712195493086	\\
2.3502147402029	-0.204873257032786	\\
2.36112181914657	0.157173765579701	\\
2.36348010648574	-0.162290907517048	\\
2.36395629912154	-0.144455394748687	\\
2.3672442958925	0.121841882535106	\\
2.37191551889124	-0.137482787152478	\\
2.37427380623041	0.11840779821664	\\
2.38239175687872	0.0709972998212735	\\
2.38470469253829	-0.125911441842929	\\
2.38828747522665	0.0825634885899519	\\
2.39268658814779	-0.0503909421877769	\\
2.39681359099134	0.0463151744245017	\\
2.39828752057832	-0.0750375060878553	\\
2.40429661812525	0.0736519486014525	\\
2.40724447729922	-0.0658675025353026	\\
2.41273203053075	0.0656135047799354	\\
2.41518102122912	-0.0535571064688756	\\
2.42225588324664	-0.0668793796004651	\\
2.42470487394501	0.0345954452794767	\\
2.42787949151697	-0.0399235782676919	\\
2.43100875740933	0.0722624633063804	\\
2.43461421593749	-0.0478694167626326	\\
2.43724461335426	0.0283950336581966	\\
2.44352582097878	-0.0331073213820614	\\
2.44493172304636	0.0397423996428735	\\
2.45531725767464	0.0656033777550759	\\
2.45685921478102	-0.0412568257714412	\\
2.46223338881355	0.0505154357001717	\\
2.46477308287112	-0.114209248753471	\\
2.47048739450066	0.176352163744773	\\
2.47196132408764	-0.139614201980219	\\
2.4751359416596	0.130658347477399	\\
2.47747155315897	-0.160986334315387	\\
2.48073687409013	-0.161525703492333	\\
2.48268699631291	0.100515782317141	\\
2.4915986013542	0.0726833860970554	\\
2.49425167461077	-0.105404324852568	\\
2.49710883042553	0.0720379173184376	\\
2.50234694941927	-0.0889565343935144	\\
2.50676873818022	-0.0703154564343688	\\
2.51021546582978	0.0672193591104438	\\
2.51259642900875	0.0709559132759368	\\
2.51924045007007	-0.0638413416453139	\\
2.51976199438546	-0.0887296809075019	\\
2.52645136712638	0.0598679930634496	\\
2.52842416518896	0.0698460979258979	\\
2.53069174916893	-0.0217022066630518	\\
2.53529494464828	0.0263895647375717	\\
2.53719971519145	-0.0535554728043585	\\
2.54524963832036	0.0406439643498877	\\
2.55037437811509	-0.0317987050526216	\\
2.55320885809006	0.0721334475395843	\\
2.5582882462052	-0.0802145174471467	\\
2.56243792488855	0.0862838337055998	\\
2.5660887350963	-0.101428710981882	\\
2.5663608451739	-0.135048235065653	\\
2.56996630370206	0.14009067823761	\\
2.57611145628778	0.0834619153333892	\\
2.58023845913133	-0.135777716472881	\\
2.58252871895111	0.0939820183927399	\\
2.58819767890104	-0.0738697513779155	\\
2.59184848910879	0.20612824934108	\\
2.59441085900616	-0.200404339637095	\\
2.59722266314133	0.164117775737842	\\
2.60252880965447	-0.182629733330078	\\
2.60511385539163	0.148237640041471	\\
2.60849255552179	-0.163505286452325	\\
2.61824316663568	0.141971168842086	\\
2.62032934389725	-0.266317007562827	\\
2.62078286069325	-0.24904663920309	\\
2.62672393072077	0.178535359008046	\\
2.62985319661314	0.165486572906855	\\
2.63343597930149	-0.146992718739399	\\
2.63615708007746	-0.116908798483236	\\
2.64377616225017	0.0243979873348621	\\
2.64672402142413	0.151225200303961	\\
2.65078299674828	-0.049690422000698	\\
2.65391226264065	0.0463539653677205	\\
2.65824334804239	-0.072623863783778	\\
2.66096444881836	-0.0792572183600272	\\
2.66686016716629	0.107944334126377	\\
2.66720030476329	0.0751671272394772	\\
2.674025732543	-0.0821319406453485	\\
2.67790330114876	-0.0689806716625281	\\
2.6825291724679	0.0285219087283973	\\
2.68586252091846	0.133369161506252	\\
2.68835686329643	-0.072184698370861	\\
2.69327752053297	0.094281123936078	\\
2.69613467634774	-0.147762013929001	\\
2.70302813164686	-0.129559523563612	\\
2.70588528746162	0.148417873129841	\\
2.70624810089842	0.155166296663327	\\
2.70958144934898	-0.0423480986850837	\\
2.72107810012744	-0.104262641624351	\\
2.72159964444283	0.109141106049452	\\
2.72470623449539	-0.17529086773194	\\
2.72679241175697	0.210386708104811	\\
2.7326881301049	0.0970024463161345	\\
2.73674710542905	-0.143061122772935	\\
2.73940017868562	0.13275969135523	\\
2.74180381770439	-0.131784484437607	\\
2.74495575943655	-0.116228929810113	\\
2.74699658501853	0.134153579578954	\\
2.75518256318623	0.0683026101087612	\\
2.75824380155919	-0.124076234344243	\\
2.76345924471313	-0.0511020274273757	\\
2.7654093669359	0.0764051700080343	\\
2.76935496306106	0.0854053813482456	\\
2.77218944303602	-0.0738739169146762	\\
2.77767699626756	-0.0597297520067685	\\
2.78286976358169	0.0494482887741151	\\
2.78570424355666	-0.0360440971393392	\\
2.78769971745904	0.0618315302798735	\\
2.79196277534138	0.0191193650527461	\\
2.79325529820997	-0.0654671364897454	\\
2.80178141397467	-0.047588273378149	\\
2.80359548115864	0.0859629267728279	\\
2.81042090893836	0.0947506125596345	\\
2.81400369162672	-0.114638445819365	\\
2.81713295751908	0.14399080551127	\\
2.82033025093084	-0.159074376102808	\\
2.82318740674561	0.136171038991326	\\
2.82779060222495	-0.10327037749956	\\
2.83488814008227	-0.100759234892164	\\
2.83613531127125	0.128510372076786	\\
2.84287003569177	-0.131781915290158	\\
2.84418523440016	0.138626892028319	\\
2.84781336876811	-0.137397411451885	\\
2.85189501993206	0.0945965506543088	\\
2.85611272613481	-0.143404654985007	\\
2.85858439267298	0.134352990690851	\\
2.86620347484569	0.086604620217129	\\
2.86849373466546	-0.129461979979577	\\
2.87067061528624	0.0938010099337093	\\
2.87296087510601	-0.0705417367532194	\\
2.87715590546896	0.0702292753812577	\\
2.88087474319611	-0.0874050740678369	\\
2.88522850443766	-0.0611780125529106	\\
2.88767749513603	0.0752817902284764	\\
2.89407208195956	0.0674491799394158	\\
2.89631699009973	-0.0621333250551058	\\
2.90062539966168	0.0382635566089487	\\
2.90298368700085	-0.0493484857139447	\\
2.90883405366918	-0.0335340563170033	\\
2.91429893106091	0.0574575576159837	\\
2.91887945070046	0.0624868535288505	\\
2.92146449643763	-0.106728615248499	\\
2.92887949605214	-0.0855889769666353	\\
2.93146454178931	0.109182510115543	\\
2.93314255393449	0.107729585940774	\\
2.93622646814725	-0.0775143338787501	\\
2.94268908249017	-0.0690910701588398	\\
2.94599975510093	0.0536678799925105	\\
2.95232631440505	-0.0587258671638335	\\
2.95336940303584	0.185821402749496	\\
2.95599980045261	-0.146556917330492	\\
2.95838076363158	0.154601904296879	\\
2.9650021088531	-0.137941296424531	\\
2.97030825536624	0.125359938266012	\\
2.97225837758901	0.118047143376773	\\
2.97767790330115	-0.0844708318194963	\\
2.98216771958149	-0.0846887074164005	\\
2.98312010485308	0.0556619349018737	\\
2.98611331570665	-0.0695988141062886	\\
2.98799541041002	0.102575887000524	\\
2.99697504297072	0.0371328063386308	\\
2.99965079206708	-0.0592547572877577	\\
3.00182767268786	0.0541066968128016	\\
3.00287076131865	-0.0558455424063895	\\
3.00969618909836	0.0736923752302316	\\
3.0152971215289	-0.0334922059872791	\\
3.01779146390687	0.0293038180828532	\\
3.02037650964403	-0.0718506897675335	\\
3.02622687631236	-0.0164622682569684	\\
3.0313289402673	0.0548220991251974	\\
3.03790493380922	0.0578233306965215	\\
3.0399004077116	-0.116815573263935	\\
3.04522923006454	0.0720948255015599	\\
3.0482224409181	-0.121460819031831	\\
3.0483584959569	-0.122194065539988	\\
3.05160114104826	0.141354547083513	\\
3.05804107955138	0.0785964134848869	\\
3.06117034544374	-0.0775760998359309	\\
3.06872140009705	-0.108346165419821	\\
3.07010462632484	0.141820224046936	\\
3.07640850978916	-0.23097973329917	\\
3.07854003873033	0.155399904177781	\\
3.08214549725849	-0.142503818411518	\\
3.08318858588928	0.152086727602838	\\
3.0899233103098	-0.173850693117044	\\
3.09234962516837	0.157448826419334	\\
3.09527480850253	-0.128748379040946	\\
3.09699817232731	0.119367073051456	\\
3.10454922698062	0.151628048989081	\\
3.10715694855759	-0.176067307034769	\\
3.11042226948875	0.126885058992842	\\
3.11271252930852	-0.175726268705649	\\
3.12073977659763	-0.197691754850068	\\
3.12368763577159	0.182636863452969	\\
3.12749717685795	0.0957774133269189	\\
3.13375570864267	-0.121746723582614	\\
3.13377838448247	-0.128488688177113	\\
3.139152558515	0.203503837526566	\\
3.14327956135855	-0.307192374079366	\\
3.14892584546869	0.193061914452584	\\
3.15327960671023	0.275954313628193	\\
3.156136762525	-0.165128331035583	\\
3.16173769495553	0.0968763954069772	\\
3.1645268232509	-0.186512874861617	\\
3.16527512596429	-0.104347808200051	\\
3.16983296976404	0.148681414856622	\\
3.17568333643236	0.110284437971217	\\
3.17822303048993	-0.154465132773149	\\
3.18534324418705	-0.143336259688561	\\
3.18752012480782	0.179209150589005	\\
3.18822307584161	0.168206230116899	\\
3.19110290749618	-0.160486182874197	\\
3.19842720375149	-0.210324620489454	\\
3.20182857972145	0.111817152825383	\\
3.20377870194423	-0.145702366100785	\\
3.20935695853496	0.173105500751389	\\
3.21711209574647	0.114023104539382	\\
3.21894883877024	-0.14102183708986	\\
3.2229624624148	0.152499563235343	\\
3.22649989342355	-0.215130188894434	\\
3.22706678941855	-0.123360869712917	\\
3.23008267611191	0.179015723420101	\\
3.23731626900802	0.0682793915597514	\\
3.24069496913818	-0.0919303967265872	\\
3.24625054988912	0.116131202422008	\\
3.24872221642729	-0.114132710044793	\\
3.25273584007184	-0.108075912510903	\\
3.25711227715318	0.147027358050219	\\
3.26191955519073	-0.0940266926650193	\\
3.26464065596669	0.0914103273057687	\\
3.26752048762126	-0.105446446583471	\\
3.27126200118821	0.119849436283239	\\
3.27482210803677	-0.0951982707729352	\\
3.27708969201674	0.0864255389002534	\\
3.28198767341349	-0.105571385208511	\\
3.28520764266505	0.071284573084105	\\
3.29257729059996	0.10635756401785	\\
3.2970671068803	-0.0711777131118992	\\
3.30085397212686	0.104762262554254	\\
3.30443675481521	-0.168130370726862	\\
3.30482224409181	-0.113009555251251	\\
3.30767939990658	0.116474667999561	\\
3.31520777872008	0.105526785623459	\\
3.31883591308804	-0.0958485685970266	\\
3.32457290055737	-0.0636270546475755	\\
3.32740738053234	0.104874771190147	\\
3.32983369539091	0.0761607910006865	\\
3.33414210495286	-0.0657777680297101	\\
3.3388360037914	-0.11707699720259	\\
3.34130767032957	0.074084438187798	\\
3.3471126853183	0.0792113630317702	\\
3.35121701232205	-0.0491936132339919	\\
3.35448233325321	0.0649740501548473	\\
3.35872271529576	-0.069366680528961	\\
3.3635980208527	-0.0903316546144121	\\
3.36668193506547	0.103184091130842	\\
3.37151188894281	-0.142574473512428	\\
3.37402890716058	0.12121530910361	\\
3.37482256155357	0.0896492892471836	\\
3.37795182744593	-0.108123311065995	\\
3.38643259153103	-0.0798283024554313	\\
3.388473417113	0.0575371212783512	\\
3.39375688778634	0.059975161471459	\\
3.39552560329072	-0.046369239824422	\\
3.39872289670248	0.0649518169641096	\\
3.40339411970122	-0.0738567378366512	\\
3.40872294205416	-0.083597291730867	\\
3.4133488133733	0.126446702223748	\\
3.41865495988644	0.119633378467368	\\
3.42033297203162	-0.135567663332883	\\
3.42430124399657	-0.0787452571729398	\\
3.42650080045714	0.0549811203099112	\\
3.43146680937329	0.0645163927344581	\\
3.43688633508542	-0.0755789032416121	\\
3.43913124322559	0.11496331049759	\\
3.44219248159856	-0.0891935189530681	\\
3.44987959129066	-0.164299771207769	\\
3.45226055446964	0.163632748187786	\\
3.45300885718303	0.152445935516689	\\
3.455163061964	-0.0994987994241043	\\
3.46300890253471	-0.128761897430819	\\
3.46618352010667	0.137025826065964	\\
3.46836040072744	-0.124315573547818	\\
3.47257810693019	0.102499245504123	\\
3.47631962049714	-0.105225053384788	\\
3.47953958974871	0.104348810145206	\\
3.48693191352342	0.0541793001767815	\\
3.49003850357598	-0.104047900755225	\\
3.49609295280251	0.0992990277965785	\\
3.49847391598148	-0.0525197573710104	\\
3.50090023084005	0.0605634070531299	\\
3.50378006249461	-0.111473129496914	\\
3.51008394595894	0.0512370882302255	\\
3.51396151456469	-0.0322995008556636	\\
3.51647853278246	0.108672467863899	\\
3.51856471004404	-0.105430969953221	\\
3.52645590229434	-0.182595853007299	\\
3.52888221715291	0.136381757691952	\\
3.53230626896267	-0.112055030399432	\\
3.53688678860222	0.130462763233908	\\
3.54031084041197	-0.119988816069228	\\
3.54323602374614	0.0653674227587249	\\
3.54768048834689	-0.0884826945115922	\\
3.54978934144826	0.120725896490825	\\
3.55566238395639	0.107036233721424	\\
3.55944924920294	-0.0972377911039503	\\
3.56663749041946	-0.11713627462855	\\
3.56899577775863	0.120615353215499	\\
3.57080984494261	0.11901647107017	\\
3.57466473770856	-0.06047426837308	\\
3.57883709223171	0.0970344843812744	\\
3.58321352931306	-0.122277796535616	\\
3.58602533344822	0.0955943120029592	\\
3.5880208073506	-0.0797661892146266	\\
3.59355371226173	0.0955215362495884	\\
3.5961841096785	-0.0424553073609965	\\
3.604279384487	-0.0769555351665982	\\
3.60695513358337	0.0892952266448724	\\
3.61339507208649	0.0515262177588062	\\
3.61550392518787	-0.0637574190702661	\\
3.61940416963342	-0.0840103545949088	\\
3.62178513281239	0.101499744024116	\\
3.62448355774856	-0.103668865626973	\\
3.62545861885995	0.0829000480421831	\\
3.63679653875981	-0.0960264944162224	\\
3.63908679857959	0.0984418206330773	\\
3.64366731821913	0.108586810479444	\\
3.64516392364591	-0.131960447281579	\\
3.64983514664466	0.100664278144993	\\
3.65403017700761	-0.158099389585838	\\
3.65595762339058	0.15605656551652	\\
3.65895083424415	-0.133755509028576	\\
3.66677399897505	-0.147847150950185	\\
3.66960847895002	0.160962612892993	\\
3.6710597326972	-0.0940726034333545	\\
3.67797586383612	0.0972395746628551	\\
3.68393960970345	-0.131713220619668	\\
3.68586705608642	0.133483388902238	\\
3.69071968580356	-0.122214969680242	\\
3.69253375298754	0.260800221451971	\\
3.6959124531177	-0.187615895456082	\\
3.69940453244686	0.15650818816624	\\
3.7042118104844	-0.142869307461945	\\
3.70917781940054	0.167255113814779	\\
3.71214835441431	0.182161276450434	\\
3.71489213103007	-0.204198339049001	\\
3.71756788012644	0.226114938848043	\\
3.72332754343557	-0.202430223091448	\\
3.72691032612393	0.246411213547743	\\
3.73140014240427	-0.191089262777772	\\
3.73534573852943	0.226921653785222	\\
3.73811219098499	-0.264039908466164	\\
3.74183102871215	0.202969927225855	\\
3.74657027923029	-0.163740477134351	\\
3.74915532496746	-0.215933818054447	\\
3.75074263375344	0.179668138563131	\\
3.75842974344555	-0.205819293630083	\\
3.76001705223153	0.176761974248105	\\
3.76895133311262	0.160579775337646	\\
3.7711508895732	-0.157397151784854	\\
3.77414410042676	-0.195519763139654	\\
3.77473367226155	0.181523046109306	\\
3.78412146993864	0.17240256678873	\\
3.78568610288482	-0.154902170797641	\\
3.79291969578093	0.138601940075248	\\
3.79448432872711	-0.166097410624444	\\
3.79480179048431	-0.171442812718479	\\
3.80013061283725	0.170471666167782	\\
3.80287438945301	-0.205878192316712	\\
3.80475648415639	0.233018430072732	\\
3.81675200341045	-0.237125118787314	\\
3.81804452627903	0.171659975883551	\\
3.820425489458	0.23532508952733	\\
3.82362278286976	-0.231268506807639	\\
3.83040285896988	-0.12845605429507	\\
3.83180876103747	0.15311055535971	\\
3.83498337860943	-0.180828691958242	\\
3.84081106943796	0.167199003127402	\\
3.84414441788852	-0.139336003905312	\\
3.84799931065447	0.272049951234036	\\
3.85087914230904	-0.218865656710569	\\
3.8539630565218	0.152751755493116	\\
3.86083383598112	0.204606142594523	\\
3.86260255148549	-0.215736971129141	\\
3.86613998249425	0.252418299841617	\\
3.87090190885219	-0.246458497644385	\\
3.87339625123016	0.19538787354474	\\
3.87614002784593	-0.182572591031411	\\
3.88110603676207	0.155771496336786	\\
3.88641218327521	-0.189127834161853	\\
3.89375915537032	0.127910102726377	\\
3.89577730511249	-0.120020168980668	\\
3.90022176971324	-0.132986159882218	\\
3.90203583689722	0.20677680310022	\\
3.90468891015379	-0.210896836200369	\\
3.90772747268695	0.157301467823304	\\
3.91396332863188	-0.157285997161162	\\
3.91493838974326	0.145185364845631	\\
3.92226268599858	-0.15387483388206	\\
3.92321507127016	0.164936108840136	\\
3.9283624869047	0.125879516361527	\\
3.93335117166064	-0.147085085923879	\\
3.9364577617132	0.256943900312382	\\
3.94013124776076	-0.221470250804238	\\
3.94675259298228	-0.216952259712569	\\
3.94790906081207	0.21573680228786	\\
3.9537821033202	-0.219956541793152	\\
3.95536941210618	0.205190085973943	\\
3.96394087955047	0.213581141641173	\\
3.96536945745786	-0.197443438960422	\\
3.96722887632143	-0.161746352897542	\\
3.97103841740779	0.202664811812423	\\
3.97459852425635	-0.216120487012608	\\
3.97775046598851	0.199949869987217	\\
3.9868888294278	0.186321311633973	\\
3.98811332477698	-0.269552447818254	\\
3.99260314105733	-0.234008197544395	\\
3.99548297271189	0.155139401296296	\\
3.99720633653667	-0.235709574801124	\\
4.00439457775319	0.231026906242133	\\
4.00564174894217	-0.226945257988876	\\
4.0118322532075	0.211744297232961	\\
4.01496151909986	-0.227007687449278	\\
4.01636742116744	0.158941106900636	\\
4.02394115166055	-0.144503753666965	\\
4.02530170204854	0.136425975818038	\\
4.0284082921011	-0.14443952849202	\\
4.03119742039646	0.167959892321435	\\
4.04013170127756	0.175041239794296	\\
4.04283012621372	-0.218659562439796	\\
4.04697980489707	0.169882266993804	\\
4.04967822983324	-0.209584112765173	\\
4.05670774017116	-0.18908796835843	\\
4.05752407040395	0.229834944996111	\\
4.06366922298967	-0.200736635732161	\\
4.06682116472184	0.268790957375642	\\
4.06818171510982	-0.24020225787298	\\
4.07135633268178	0.254151972147451	\\
4.07550601136513	0.320656995752616	\\
4.07729740270931	-0.303052657637163	\\
4.0863450627894	-0.347112178714717	\\
4.08949700452156	0.272079842822581	\\
4.09403217248151	0.283407086936368	\\
4.09530201951029	-0.245038344353945	\\
4.09976915995084	-0.32484665849896	\\
4.10067619354283	0.314801085934681	\\
4.11004131538011	0.316470075256506	\\
4.11335198799088	-0.339583391011536	\\
4.11353339470927	-0.251565381886269	\\
4.11455380750026	0.327565127678949	\\
4.12466723205094	-0.276577196486316	\\
4.1280459321811	0.315680838286387	\\
4.13029084032127	0.368078041987	\\
4.13351080957283	-0.246220305327629	\\
4.13707091642139	0.303631524796646	\\
4.14031356151275	-0.268508648330374	\\
4.1448714053125	-0.20383225133199	\\
4.15110726125742	0.248395043016835	\\
4.15357892779559	-0.221867845093127	\\
4.15532496746017	0.229452862563019	\\
4.16060843813351	-0.206754715620876	\\
4.16457671009846	0.239525247320457	\\
4.169497367335	-0.256724477330621	\\
4.17344296346015	0.230770832598099	\\
4.17636814679432	-0.21554368681244	\\
4.17947473684688	0.166523813772619	\\
4.18766071501458	-0.448145351835288	\\
4.19085800842634	0.483104323932591	\\
4.19620950661908	-0.428933087101562	\\
4.19800089796326	0.416762387673427	\\
4.19997369602583	0.392647572306668	\\
4.20550660093697	-0.470582332017887	\\
4.20725264060154	0.477720424668133	\\
4.20965627962032	-0.437337326247615	\\
4.21584678388564	0.349449858519895	\\
4.21713930675423	-0.509223422980502	\\
4.22292164590316	0.366454482302011	\\
4.22552936748012	-0.310738990991135	\\
4.23292169125484	-0.468374095737419	\\
4.23373802148763	0.440004191339999	\\
4.24287638492692	0.271237182183051	\\
4.2444863695527	-0.293090933581932	\\
4.24859069655645	-0.371858756186329	\\
4.24992857110463	0.298204012549346	\\
4.25403289810838	0.327939298833447	\\
4.25793314255393	-0.275971451965365	\\
4.26312590986807	0.378957529674375	\\
4.26716220935242	-0.208569357769415	\\
4.27069964036118	-0.225302976153476	\\
4.27602846271412	0.25984172355249	\\
4.28029152059647	-0.344874736362712	\\
4.28371557240622	0.147952756426237	\\
4.28471330935741	-0.187859294170592	\\
4.28786525108957	0.372380144330473	\\
4.2940784311947	-0.253864849083112	\\
4.29548433326228	0.378334498236913	\\
4.30119864489181	0.160065845659853	\\
4.30791069347253	-0.296441394466928	\\
4.30793336931233	-0.26481460088508	\\
4.31407852189806	0.221044506717307	\\
4.3186817173774	-0.156332593100338	\\
4.32296745109955	0.254459943465489	\\
4.32507630420093	0.297387900398327	\\
4.32995160975787	-0.231572025255174	\\
4.33734393353258	-0.307394637269299	\\
4.33895391815836	0.256473178824054	\\
4.34133488133733	0.258448002487898	\\
4.34260472836611	-0.350972618313594	\\
4.34893128767024	0.307287002824959	\\
4.35076803069402	-0.152167266590696	\\
4.35825105782793	0.287076843308081	\\
4.3605639934875	-0.223394808618132	\\
4.36512183728724	0.119629511214532	\\
4.368999405893	-0.295535532566781	\\
4.37142572075157	0.255369843266649	\\
4.37580215783291	-0.212583471504364	\\
4.37865931364768	0.257439297951863	\\
4.38355729504442	-0.122393079661296	\\
4.38843260060136	0.153486406040783	\\
4.38931695835355	-0.357565929140059	\\
4.39611971029347	-0.24895967844195	\\
4.40113107088921	0.098723593462398	\\
4.40287711055379	-0.315176331419792	\\
4.40886353226092	0.197666690715328	\\
4.40924902153751	0.258341223421733	\\
4.41648261443363	-0.137628919380381	\\
4.41793386818081	0.225121041985519	\\
4.42020145216078	-0.256639087623819	\\
4.42843278200808	-0.250463407947839	\\
4.43153937206064	0.36488830955264	\\
4.4351901822684	-0.178491522428571	\\
4.43655073265638	0.219795880149646	\\
4.44503149674148	0.246444164115656	\\
4.44639204712947	-0.191762237552561	\\
4.44997482981782	0.132261126579969	\\
4.45469140449617	-0.319293002368613	\\
4.45861432478152	0.206181800185063	\\
4.4598614959705	-0.209869149147703	\\
4.46521299416324	-0.193658804124096	\\
4.46759395734221	0.225193426607782	\\
4.47217447698176	0.216770763554675	\\
4.47342164817074	-0.218091073127149	\\
4.48312690760502	0.15285472323717	\\
4.48668701445358	-0.233750100493672	\\
4.48963487362755	-0.279795030917059	\\
4.4932403321557	0.212733115560708	\\
4.49795690683404	-0.239299745387011	\\
4.50224264055619	0.336356627156976	\\
4.50788892466633	-0.25677250796352	\\
4.5097029918503	0.311030762993228	\\
4.51051932208309	0.22912790348598	\\
4.51466900076644	-0.336556540075192	\\
4.51970303720198	-0.273107421125251	\\
4.52346722660874	0.37729597005358	\\
4.52834253216568	-0.267349770178033	\\
4.52954435167506	0.3892835881274	\\
4.53630175193538	0.328632377296192	\\
4.53977115542474	-0.366075485960737	\\
4.54383013074889	0.282329574158132	\\
4.54872811214563	-0.363009455491429	\\
4.55065555852861	0.302764512974758	\\
4.55435172041597	-0.267005889916042	\\
4.55868280581771	0.248961762925465	\\
4.56095038979769	-0.199134440243062	\\
4.56940847804298	-0.248962525998643	\\
4.57042889083397	0.167235149803861	\\
4.57258309561495	-0.313534153603978	\\
4.5765967192595	0.129960289273296	\\
4.58301398192282	-0.266089505304302	\\
4.58618859949478	0.190152528408466	\\
4.59403444006549	0.30264351456271	\\
4.59510020453607	-0.201760080552005	\\
4.59675554084146	-0.241714591829301	\\
4.59750384355485	0.178914604788548	\\
4.60777599898412	0.325026510713719	\\
4.61017963800289	-0.256045938085013	\\
4.61396650324945	-0.251733995999327	\\
4.6175492859378	0.323069007894703	\\
4.62131347534456	0.376733294180167	\\
4.62548582986771	-0.311744951718425	\\
4.62720919369249	-0.256472282371633	\\
4.6344881382682	0.254976940638652	\\
4.6348056000254	0.211971589039822	\\
4.64135891772752	-0.310173259512107	\\
4.64326368827069	-0.215454529998977	\\
4.64709590519685	0.277723266434915	\\
4.6509507979628	0.262676326471294	\\
4.65510047664615	-0.338259396141091	\\
4.66074676075628	0.289586254994703	\\
4.66358124073125	-0.202383210254092	\\
4.66872865636579	-0.323581532552442	\\
4.67296903840834	0.186067764083689	\\
4.67432958879632	0.344228205273812	\\
4.68029333466365	-0.230897449384435	\\
4.68253824280382	-0.346181606110458	\\
4.68829790611295	0.332647608369813	\\
4.69110971024812	0.182218139157476	\\
4.69605304332446	-0.340346456110225	\\
4.69705078027565	-0.153703770900671	\\
4.70176735495399	0.211704304789572	\\
4.70884221697151	-0.234329988251955	\\
4.71124585599028	0.195125008206644	\\
4.71591707898902	0.227037607399681	\\
4.71922775159978	-0.104193998710682	\\
4.72138195638075	0.112096249136986	\\
4.7258717726611	-0.251921632783868	\\
4.72986272046585	0.243826849648251	\\
4.73553168041578	-0.154365804220448	\\
4.73720969256096	-0.201520008524411	\\
4.7425611907537	0.283663046058261	\\
4.74342287266609	0.242211670790475	\\
4.74909183261602	-0.154106103456202	\\
4.75145011995519	-0.184825147721902	\\
4.75684696982753	0.150305530705894	\\
4.76149551698647	0.139602460721892	\\
4.76262930897646	-0.253649330674176	\\
4.76936403339698	-0.100190441729533	\\
4.77033909450837	0.216348979282698	\\
4.77773141828308	-0.162330355467944	\\
4.78217588288382	0.197296383891226	\\
4.783853895029	0.176309770959424	\\
4.78918271738194	-0.209356569302939	\\
4.7922212799151	-0.145650020582942	\\
4.79752742642824	0.116370661027977	\\
4.80217597358718	0.200198568405794	\\
4.80312835885877	-0.16293053038426	\\
4.80575875627554	-0.167112149471589	\\
4.80730071338192	0.180680810186358	\\
4.81659780769981	-0.159241021165924	\\
4.81807173728679	0.109836397437991	\\
4.82791305175987	0.125386459720606	\\
4.82882008535186	-0.146564062250746	\\
4.82918289878866	-0.10986208570423	\\
4.83648451920417	0.166634597246386	\\
4.83718747023796	0.159195572343878	\\
4.8424029133919	-0.160118833473408	\\
4.84646188871605	-0.155292821864732	\\
4.8507476224382	0.133136785050037	\\
4.85621249982993	-0.166658401103499	\\
4.85936444156209	0.189233203997696	\\
4.86403566456084	0.190238112956294	\\
4.866847468696	-0.133600261465098	\\
4.86990870706897	-0.135689995215033	\\
4.87485204014531	0.170657672129398	\\
4.87786792683867	0.0810246823817159	\\
4.87854820203266	-0.261462031551658	\\
4.8836729418274	-0.192524430625466	\\
4.88643939428297	0.177180450132326	\\
4.89151878239811	0.172076784562932	\\
4.89634873627545	-0.121585364279144	\\
4.90444401108395	0.116209262825325	\\
4.90553245139434	-0.189359439198503	\\
4.90709708434052	0.120316645374123	\\
4.9089338273643	-0.222497971906072	\\
4.91600868938181	0.141040815215464	\\
4.9168476954544	-0.142423643036258	\\
4.92253933124413	-0.148387994805073	\\
4.92811758783487	0.226662588186058	\\
4.93619018680357	-0.181188351063498	\\
4.93657567608016	0.140810918015109	\\
4.93827636406514	0.162878261973216	\\
4.93925142517653	-0.203887915580602	\\
4.94673445231044	0.122221479805335	\\
4.94823105773722	-0.182626052960024	\\
4.96043065954948	-0.136951321762204	\\
4.96108825890367	0.169982552181325	\\
4.96340119456324	-0.238249764414305	\\
4.96448963487363	0.210530043943099	\\
4.97197266200754	-0.207055445492522	\\
4.97655318164708	0.185636999415771	\\
4.98077088784983	-0.265760087945503	\\
4.98265298255321	0.187853360026739	\\
4.98566886924657	0.288451616573222	\\
4.99063487816271	-0.30488979038583	\\
5	-0.167728681489477	\\
};
\addlegendentry{y};

\end{axis}

\begin{axis}[%
width=\figurewidth,
height=\figureheight,
scale only axis,
xmin=0,
xmax=5,
xlabel={t (s)},
ymin=-0.887386679649353,
ymax=0.892788469791412,
ylabel={x},
name=plot5,
at=(plot6.left of south west),
anchor=right of south east,
title={x(t)},
legend style={draw=black,fill=white,legend cell align=left}
]
\addplot [color=blue,solid]
  table[row sep=crcr]{
0	0	\\
0	0	\\
0	0	\\
0.00777781305130635	0	\\
0.00777781305130635	0	\\
0.0155556261026127	0	\\
0.0155556261026127	0	\\
0.0233334391539191	0	\\
0.0233334391539191	0	\\
0.0311112522052254	0	\\
0.0311112522052254	0	\\
0.0461226581526447	-0.000244148075580597	\\
0.0463720923904417	0.00039674062281847	\\
0.0529707617721622	0.00186162907630205	\\
0.0544446913591445	-0.0108951078727841	\\
0.0579367706883024	-0.0578936114907265	\\
0.0622225044104508	0.187170013785362	\\
0.0625626420074467	0.19455549120903	\\
0.0696601798647613	-0.364268928766251	\\
0.0700003174617572	-0.337382107973099	\\
0.0726080390387258	0.427442252635956	\\
0.0813382373616207	0.2613605260849	\\
0.0840593381375879	-0.451551854610443	\\
0.0883677476995361	-0.268318742513657	\\
0.0932203774166776	0.271248519420624	\\
0.0940367076494678	0.401867747306824	\\
0.0965310500274378	-0.160069584846497	\\
0.101587762302777	0.140079960227013	\\
0.105170544991134	-0.289162874221802	\\
0.11208667613005	-0.174993127584457	\\
0.1162590306532	0.212408825755119	\\
0.116757899128794	0.167394027113914	\\
0.118413235434174	-0.134128853678703	\\
0.125964290087483	-0.129520550370216	\\
0.129479045256441	0.129978328943253	\\
0.133606048099991	0.164159059524536	\\
0.138572057016132	-0.101229898631573	\\
0.143492714252672	-0.169988095760345	\\
0.146667331824634	0.125217437744141	\\
0.151928126658171	-0.118564411997795	\\
0.15365149048295	0.148167356848717	\\
0.156735404695713	0.164799958467484	\\
0.160839731699464	-0.123661004006863	\\
0.167574456119982	-0.164647355675697	\\
0.170114150177552	0.205359056591988	\\
0.171247942167538	0.0993072316050529	\\
0.178640265942249	-0.0755027905106544	\\
0.181973614392809	-0.101260416209698	\\
0.185443017882167	0.0809045657515526	\\
0.191792253026091	-0.0836817547678947	\\
0.194241243724461	0.117679372429848	\\
0.194694760520456	0.131168559193611	\\
0.201180050703178	-0.0618915371596813	\\
0.204740157551735	-0.104892119765282	\\
0.209411380550479	0.024964140728116	\\
0.210862634297661	-0.0330515466630459	\\
0.212109805486646	0.102664269506931	\\
0.217756089596778	0.0330515466630459	\\
0.225261792570488	-0.108035527169704	\\
0.225669957686883	-0.0822473838925362	\\
0.231883137792008	0.0411999896168709	\\
0.235148458723169	0.101901300251484	\\
0.239434192445317	-0.0787682756781578	\\
0.241951210663087	-0.0827661976218224	\\
0.248617907564207	0.0290841404348612	\\
0.254105460795741	0.096865750849247	\\
0.256168962217516	-0.047853022813797	\\
0.262903686638035	-0.0649433881044388	\\
0.26412818198722	0.0503860600292683	\\
0.266078304209996	-0.0386974699795246	\\
0.270250658733146	0.0807519778609276	\\
0.273742738062304	0.0254829563200474	\\
0.279343670492837	-0.0471510961651802	\\
0.284491086127375	-0.0390331745147705	\\
0.287008104345144	0.0507217645645142	\\
0.288550061451526	-0.0558488741517067	\\
0.291611299824489	0.0593584999442101	\\
0.295829006027238	-0.0474562831223011	\\
0.298890244400201	0.0469069480895996	\\
0.309411834067275	-0.0910061970353127	\\
0.311044494532855	0.156315803527832	\\
0.315987827609196	0.253425717353821	\\
0.317643163914576	-0.219885855913162	\\
0.318867659263761	-0.175359353423119	\\
0.324196481616697	0.227759629487991	\\
0.330817826838217	-0.178991064429283	\\
0.333176114177389	0.121738336980343	\\
0.336781572705545	-0.160740986466408	\\
0.341928988340083	0.298471033573151	\\
0.345126281751845	-0.297769099473953	\\
0.348754416119801	0.275490581989288	\\
0.350001587308786	0.249061554670334	\\
0.352382550487757	-0.294015318155289	\\
0.362904140154831	0.30622273683548	\\
0.365149048295004	-0.178533285856247	\\
0.369049292740557	0.126255080103874	\\
0.373040240545309	-0.241248816251755	\\
0.373266998943306	-0.173467203974724	\\
0.377484705146055	0.268379777669907	\\
0.38458224300337	0.154545739293098	\\
0.387462074657935	-0.194219797849655	\\
0.389888389516506	0.0884731560945511	\\
0.395784107864435	-0.0889004170894623	\\
0.398618587839401	0.144077882170677	\\
0.400478006702978	-0.121036410331726	\\
0.404650361226128	0.0840174555778503	\\
0.411883954122241	-0.161900699138641	\\
0.413289856189824	-0.115939818322659	\\
0.419503036294949	0.135807365179062	\\
0.420069932289942	0.150730922818184	\\
0.427620986943251	-0.129337444901466	\\
0.428051827899446	-0.149723812937737	\\
0.433244595213584	0.101138338446617	\\
0.436441888625345	-0.105014190077782	\\
0.438868203483916	0.143101289868355	\\
0.44680474741382	0.0610980577766895	\\
0.449026979714194	-0.0764793828129768	\\
0.452813844960748	-0.0622577592730522	\\
0.458686887468877	0.0842310860753059	\\
0.459095052585273	0.0905789360404015	\\
0.464401199098409	-0.0619830936193466	\\
0.469412559694148	-0.0906399711966515	\\
0.473720969256096	0.0874660462141037	\\
0.477031641866857	-0.0494094677269459	\\
0.479095143288632	0.0914334580302238	\\
0.483993124685373	-0.0891140475869179	\\
0.487077038898136	0.0434278398752213	\\
0.489979546392501	-0.066591389477253	\\
0.494945555308641	0.0679647177457809	\\
0.500365081020776	0.0802331641316414	\\
0.505081655699119	-0.0603045746684074	\\
0.508505707508878	-0.0508438386023045	\\
0.510047664615259	0.0106509597972035	\\
0.513244958027021	-0.0242622159421444	\\
0.519049973015751	0.0662862062454224	\\
0.521498963714121	0.0079348124563694	\\
0.528755232450034	-0.0844752341508865	\\
0.528777908289834	-0.0831629410386086	\\
0.535308550152155	0.0479750968515873	\\
0.538823305321113	0.0511795394122601	\\
0.54236073632987	-0.0450758375227451	\\
0.548347158036998	0.0195928830653429	\\
0.550728121215969	-0.0545976124703884	\\
0.555535399253511	0.0404065065085888	\\
0.559231561140867	-0.0125431073829532	\\
0.561567172640239	0.047883540391922	\\
0.565444741245992	-0.0972624868154526	\\
0.569344985691545	-0.111575670540333	\\
0.573358609336097	0.123416855931282	\\
0.579594465281022	0.333933532238007	\\
0.582474296935587	-0.279030740261078	\\
0.586011727944344	-0.312723159790039	\\
0.588324663603917	0.267830431461334	\\
0.591091116059483	-0.182897433638573	\\
0.592837155724062	0.276650279760361	\\
0.599594555984381	-0.171056240797043	\\
0.601816788284754	0.207098603248596	\\
0.610252200690253	0.190893277525902	\\
0.612043592034431	-0.231238752603531	\\
0.62041097692053	-0.360118418931961	\\
0.621975609866711	0.253151029348373	\\
0.625354309996871	-0.334543913602829	\\
0.629730747078218	0.234290599822998	\\
0.630252291393612	0.291268646717072	\\
0.633268178086975	-0.239326149225235	\\
0.641499507934276	-0.181218907237053	\\
0.643925822792847	0.217230752110481	\\
0.645626510777827	-0.246101260185242	\\
0.651340822407358	0.253364652395248	\\
0.657055134036889	0.347819447517395	\\
0.660683268404845	-0.274849683046341	\\
0.664832947088195	0.211615338921547	\\
0.667531372024363	-0.296182125806808	\\
0.671386264790316	0.232306897640228	\\
0.675535943473667	-0.233588665723801	\\
0.678007610011837	0.247810304164886	\\
0.680569979909206	-0.221686452627182	\\
0.685263878747749	0.239997565746307	\\
0.688370468800312	-0.299630731344223	\\
0.695921523453621	-0.255439937114716	\\
0.698370514151992	0.243873402476311	\\
0.701658510922952	-0.189245283603668	\\
0.707100712474886	0.164952546358109	\\
0.709005483018064	-0.178319647908211	\\
0.715286690642588	0.19611194729805	\\
0.720751568034322	0.145756408572197	\\
0.722384228499902	-0.266640216112137	\\
0.727304885736443	0.212744534015656	\\
0.730819640905401	-0.253364652395248	\\
0.733200604084372	0.271706283092499	\\
0.736239166617536	-0.243842884898186	\\
0.742384319203262	0.19623401761055	\\
0.744538523984236	-0.297738581895828	\\
0.747214273080604	0.210882902145386	\\
0.752724502151937	-0.240546897053719	\\
0.755309547889106	0.227393418550491	\\
0.757395725150681	-0.143314927816391	\\
0.765581703318382	-0.244148075580597	\\
0.768121397375952	0.201910465955734	\\
0.771636152544909	-0.171239361166954	\\
0.775695127869061	0.193762019276619	\\
0.782928720765174	0.116306036710739	\\
0.785218980584946	-0.183812975883484	\\
0.785604469861541	-0.123447373509407	\\
0.787872053841514	0.156468391418457	\\
0.797554637435997	-0.125125885009766	\\
0.798575050226985	0.134800255298615	\\
0.804946961210708	0.157811209559441	\\
0.808620447258264	-0.169225141406059	\\
0.809232694932857	-0.223456531763077	\\
0.812475340024218	0.229224517941475	\\
0.818484437571145	0.236610010266304	\\
0.822611440414696	-0.183690905570984	\\
0.824606914317072	0.161015659570694	\\
0.830661363543599	-0.151829585433006	\\
0.832974299203171	0.120059818029404	\\
0.838371149075506	-0.0909146368503571	\\
0.840910843133075	0.118381299078465	\\
0.845922203728815	-0.116885893046856	\\
0.848575276985383	0.112430192530155	\\
0.850548075047959	-0.113498337566853	\\
0.859074190812657	0.0962858945131302	\\
0.862725001020413	-0.111819818615913	\\
0.867600306577354	-0.118625447154045	\\
0.870230703994122	0.199438452720642	\\
0.875559526347058	-0.138157293200493	\\
0.87707880761364	0.226203188300133	\\
0.87873414391902	0.215247049927711	\\
0.880344128544801	-0.232367932796478	\\
0.891636696765065	0.13489180803299	\\
0.893858929065438	-0.139072850346565	\\
0.896716084880203	0.149021878838539	\\
0.90202223139334	-0.185705125331879	\\
0.90202223139334	-0.185705125331879	\\
0.90451657377131	0.161961734294891	\\
0.915945197030372	-0.221839040517807	\\
0.917555181656153	0.191351056098938	\\
0.917872643413349	0.18793298304081	\\
0.921296695223108	-0.158757284283638	\\
0.925559753105456	0.118686482310295	\\
0.929187887473413	-0.126468703150749	\\
0.9385303334709	0.110904261469841	\\
0.940117642256881	-0.13464766740799	\\
0.944992947813822	-0.237678155303001	\\
0.947124476754997	0.135288551449776	\\
0.952362595748734	-0.125675216317177	\\
0.954834262286904	0.19220557808876	\\
0.959732243683645	-0.108676411211491	\\
0.963564460609799	0.117954038083553	\\
0.966194858026567	-0.15311136841774	\\
0.971319597821305	0.146855071187019	\\
0.97673912353344	0.153080850839615	\\
0.97925614175121	-0.10223700851202	\\
0.981115560614787	-0.15515610575676	\\
0.98424482650715	0.103915527462959	\\
0.989528297180486	-0.0868861973285675	\\
0.994267547698629	0.0953703448176384	\\
0.999052149896371	0.182561725378037	\\
1.00193198155094	-0.170537427067757	\\
1.00315647690012	0.0881679728627205	\\
1.00946036036445	-0.13068026304245	\\
1.01336060481	0.185277864336967	\\
1.01678465661976	-0.140690326690674	\\
1.02063954938571	0.19449445605278	\\
1.02361008439948	-0.213995784521103	\\
1.02737427380623	0.169072538614273	\\
1.0299819953832	-0.150334179401398	\\
1.03429040494515	0.163853883743286	\\
1.03893895210409	-0.170537427067757	\\
1.04444918117542	-0.113864555954933	\\
1.04925645921297	0.102267526090145	\\
1.05206826334813	-0.137272253632545	\\
1.05524288092009	0.147892698645592	\\
1.05862158105025	-0.124057739973068	\\
1.06132000598642	0.0921659022569656	\\
1.06546968466977	-0.0882290080189705	\\
1.06893908815913	0.116031371057034	\\
1.07569648841945	-0.173833429813385	\\
1.079936870462	0.212469860911369	\\
1.08385979074735	-0.325968205928802	\\
1.0882589036685	0.49916073679924	\\
1.08866706878489	0.252723783254623	\\
1.09070789436687	-0.472426533699036	\\
1.10145624243194	0.659688115119934	\\
1.10376917809151	-0.274819165468216	\\
1.10964222059964	0.220374152064323	\\
1.11132023274482	-0.301675468683243	\\
1.113111624089	-0.382030695676804	\\
1.11764679204894	0.228125855326653	\\
1.12234069088749	0.302407920360565	\\
1.12522052254205	-0.190313428640366	\\
1.13249946711776	-0.256569117307663	\\
1.13526591957333	0.160435810685158	\\
1.13556070549073	0.207373276352882	\\
1.1381457512279	-0.110324412584305	\\
1.14705735626919	-0.216223642230034	\\
1.14869001673477	0.126377150416374	\\
1.15313448133552	-0.112735375761986	\\
1.15701204994127	0.168797880411148	\\
1.15864471040685	0.124576553702354	\\
1.16347466428419	-0.093295082449913	\\
1.16807785976354	-0.168706327676773	\\
1.17168331829169	0.13705863058567	\\
1.17469920498506	-0.123783074319363	\\
1.17866747695001	0.162083804607391	\\
1.18215955627917	0.0833155289292336	\\
1.18900765989868	-0.107119970023632	\\
1.19249973922784	-0.129917293787003	\\
1.19513013664461	0.167210906744003	\\
1.19782856158078	-0.0579241327941418	\\
1.20211429530293	0.083040863275528	\\
1.20889437140304	-0.100039675831795	\\
1.21147941714021	0.0911893099546433	\\
1.21698964621155	-0.101748712360859	\\
1.21959736778852	0.138493001461029	\\
1.22281733704008	0.0844447165727615	\\
1.22828221443181	-0.0788598284125328	\\
1.22991487489739	-0.103244118392468	\\
1.23603735164332	0.0447401367127895	\\
1.23909859001628	0.0823694542050362	\\
1.24401924725282	-0.0348216183483601	\\
1.24676302386859	-0.0656758323311806	\\
1.24914398704756	0.00674459058791399	\\
1.25281747309512	-0.0611285753548145	\\
1.25959754919523	0.0814844220876694	\\
1.26043655526782	0.0924405679106712	\\
1.26683114209135	-0.10837122797966	\\
1.26760212064454	-0.0949736014008522	\\
1.27519852697745	0.0615558326244354	\\
1.27821441367081	0.0977813079953194	\\
1.28134367956317	-0.0777001231908798	\\
1.28404210449934	-0.0538041330873966	\\
1.28887205837668	0.0319833979010582	\\
1.29123034571585	-0.0405896194279194	\\
1.29492650760321	0.0607928708195686	\\
1.29887210372836	0.0389416180551052	\\
1.3042689536007	-0.0542313903570175	\\
1.31007396858943	-0.0758995339274406	\\
1.3120240908122	0.0537125766277313	\\
1.32016471730031	0.203344821929932	\\
1.32177470192609	-0.22046571969986	\\
1.32665000748303	0.221594899892807	\\
1.32971124585599	-0.295846432447433	\\
1.33327135270455	0.296945095062256	\\
1.33655934947551	-0.403546243906021	\\
1.34286323293983	0.295754879713058	\\
1.34445054172581	-0.489516884088516	\\
1.3453122236382	-0.271279036998749	\\
1.34714896666198	0.444563120603561	\\
1.35803336976585	-0.39426863193512	\\
1.35977940943043	0.227546006441116	\\
1.36138939405621	0.271187484264374	\\
1.36660483721015	-0.230689406394959	\\
1.36878171783092	0.187536239624023	\\
1.37086789509249	-0.232123777270317	\\
1.3790765491	-0.212164685130119	\\
1.38186567739536	0.229285567998886	\\
1.38426931641413	-0.174687951803207	\\
1.3874439339861	0.168858915567398	\\
1.39193375026644	-0.173223063349724	\\
1.39413330672701	0.0847498998045921	\\
1.40182041641912	0.150151073932648	\\
1.40195647145792	-0.075624868273735	\\
1.40980231202863	0.063081756234169	\\
1.41383861151298	-0.130405589938164	\\
1.41549394781836	-0.102725304663181	\\
1.41746674588093	0.110568560659885	\\
1.42302232663187	-0.0867946445941925	\\
1.42549399317004	0.11673329770565	\\
1.43141238735777	0.0893276780843735	\\
1.43406546061433	-0.120120853185654	\\
1.44123102599105	-0.0710165724158287	\\
1.44417888516501	0.116824857890606	\\
1.44687731010118	0.0778527185320854	\\
1.45408822715749	-0.0684835389256477	\\
1.45481385403108	-0.0881069377064705	\\
1.45955310454923	0.0831934586167336	\\
1.46574360881455	0.0990325659513474	\\
1.46932639150291	-0.0911587849259377	\\
1.47064159021129	0.0256050303578377	\\
1.47583435752543	-0.056031983345747	\\
1.47787518310741	-0.0479750968515873	\\
1.48470061088712	0.092043824493885	\\
1.48794325597849	0.0737632364034653	\\
1.49209293466184	-0.0986053049564362	\\
1.495131497195	-0.0639057606458664	\\
1.49789794965057	0.0437330231070518	\\
1.5031360686443	0.066530354321003	\\
1.50839686347784	-0.0532242804765701	\\
1.51336287239398	-0.0627460554242134	\\
1.51610664900975	0.0154728842899203	\\
1.51767128195593	0.0529801324009895	\\
1.52143547136268	-0.027802363038063	\\
1.52567585340523	0.063081756234169	\\
1.53105002743777	-0.0587786510586739	\\
1.53440605172813	0.0878933072090149	\\
1.5369230699459	-0.108249150216579	\\
1.54093669359045	0.166936248540878	\\
1.54542650987079	-0.191869869828224	\\
1.54873718248155	0.162053287029266	\\
1.55191180005352	-0.186346024274826	\\
1.55948553054662	-0.127781003713608	\\
1.5613222735704	0.175420388579369	\\
1.56324971995338	0.0876186415553093	\\
1.56769418455413	-0.123538926243782	\\
1.57200259411607	-0.0924405679106712	\\
1.57742211982821	0.130222484469414	\\
1.58159447435136	0.340128779411316	\\
1.58431557512732	-0.246131777763367	\\
1.58989383171806	0.271004378795624	\\
1.59300042177062	-0.402661204338074	\\
1.59567617086699	0.269417405128479	\\
1.59907754683695	-0.165654465556145	\\
1.6028870879233	0.255256801843643	\\
1.60656057397086	-0.177739799022675	\\
1.60982589490202	-0.153202921152115	\\
1.61628850924494	0.179815053939819	\\
1.62275112358786	0.169804990291595	\\
1.62508673508723	-0.125125885009766	\\
1.62928176545018	-0.234015926718712	\\
1.63177610782815	0.319345682859421	\\
1.63488269788072	-0.318643748760223	\\
1.63844280472927	0.265877246856689	\\
1.64304600020862	-0.335398405790329	\\
1.64592583186318	0.359996348619461	\\
1.6526832321235	0.258796960115433	\\
1.65529095370047	-0.344553977251053	\\
1.65767191687944	0.23929563164711	\\
1.66361298690697	-0.206091493368149	\\
1.66578986752774	0.200201421976089	\\
1.66828420990571	-0.205877870321274	\\
1.67424795577304	0.158696249127388	\\
1.67649286391322	-0.212530896067619	\\
1.68127746611096	0.165868103504181	\\
1.68467884208092	-0.126743376255035	\\
1.69193511081683	-0.221015051007271	\\
1.6940212880784	0.207159638404846	\\
1.70018911650393	0.156620994210243	\\
1.7028875414401	-0.177800834178925	\\
1.70476963614347	0.155125588178635	\\
1.70885128730743	-0.203711047768593	\\
1.7111188712874	0.347575306892395	\\
1.71497376405335	-0.228217408061028	\\
1.71862457426111	0.21655935049057	\\
1.72127764751768	-0.195287942886353	\\
1.72871532297199	-0.152012690901756	\\
1.73216205062154	0.147648543119431	\\
1.73472442051891	-0.177007362246513	\\
1.73944099519726	0.143589586019516	\\
1.74857935863655	0.133518472313881	\\
1.74957709558774	-0.196966454386711	\\
1.75009863990313	-0.233649715781212	\\
1.75315987827609	0.157719656825066	\\
1.76143655980299	0.215552225708961	\\
1.76411230889936	-0.231910154223442	\\
1.76982662052889	0.113925598561764	\\
1.77220758370786	-0.122104555368423	\\
1.77529149792063	-0.191839352250099	\\
1.7775590819006	0.13718070089817	\\
1.78245706329734	0.251411467790604	\\
1.78708293461648	-0.20639668405056	\\
1.7941124449544	-0.228705704212189	\\
1.79628932557517	0.22498245537281	\\
1.79633467725477	0.271645247936249	\\
1.79901042635114	-0.147282332181931	\\
1.80442995206327	0.128147214651108	\\
1.80601726084926	-0.178868979215622	\\
1.81470210749255	-0.155430763959885	\\
1.81692433979292	0.110690630972385	\\
1.82086993591808	0.172124400734901	\\
1.82368174005324	-0.162297427654266	\\
1.82769536369779	-0.211554303765297	\\
1.82969083760017	0.156712546944618	\\
1.83626683114209	-0.111392557621002	\\
1.83794484328727	0.164586320519447	\\
1.84322831396061	0.162846773862839	\\
1.85071134109452	-0.162694171071053	\\
1.85526918489426	0.186986908316612	\\
1.85837577494683	-0.186407059431076	\\
1.85923745685921	-0.231696531176567	\\
1.86134630996059	0.255592525005341	\\
1.86817173774031	0.233161419630051	\\
1.87046199756008	-0.21692556142807	\\
1.87941895428097	-0.151646479964256	\\
1.88134640066395	0.214423045516014	\\
1.88236681345494	0.211706906557083	\\
1.88449834239611	-0.151066616177559	\\
1.8932965682384	0.148106321692467	\\
1.89635780661137	-0.199621573090553	\\
1.89855736307194	0.270729690790176	\\
1.90513335661386	-0.156224250793457	\\
1.90515603245366	-0.17783135175705	\\
1.90667531372024	0.114169746637344	\\
1.91347806566016	-0.119083225727081	\\
1.92046222431848	0.149754330515862	\\
1.92145996126967	0.14523757994175	\\
1.9265847010644	-0.104922637343407	\\
1.92882960920458	-0.105716116726398	\\
1.92912439512197	0.0796533077955246	\\
1.93864824783786	0.185186311602592	\\
1.94186821708942	-0.123081147670746	\\
1.94524691721958	0.215460672974586	\\
1.95043968453372	-0.209906309843063	\\
1.95316078530968	0.330576509237289	\\
1.95556442432845	-0.25812554359436	\\
1.96413589177275	-0.170720547437668	\\
1.96658488247112	0.234748378396034	\\
1.96737853686411	0.113681450486183	\\
1.9685350046939	-0.19833979010582	\\
1.97733323053619	-0.138340398669243	\\
1.97885251180277	0.149784848093987	\\
1.98658497317448	0.143955811858177	\\
1.98969156322704	-0.144688248634338	\\
1.99075732769763	-0.116580709815025	\\
1.992979559998	0.129978328943253	\\
1.99860316826834	0.166203796863556	\\
2.00502043093166	-0.141544848680496	\\
2.00778688338723	0.162755206227303	\\
2.00919278545481	-0.221808522939682	\\
2.01692524682652	-0.228370010852814	\\
2.02059873287407	0.199133276939392	\\
2.02207266246105	0.162816241383553	\\
2.02477108739722	-0.170964688062668	\\
2.03118835006054	-0.190984830260277	\\
2.0351112703459	0.139774769544601	\\
2.04125642293162	0.150273144245148	\\
2.04504328817818	-0.140263065695763	\\
2.04991859373512	0.106173895299435	\\
2.05259434283149	-0.167668685317039	\\
2.05517938856866	0.114719077944756	\\
2.06044018340219	-0.209051787853241	\\
2.06454451040594	-0.294717252254486	\\
2.0679912380555	0.358409374952316	\\
2.07254908185525	-0.424542993307114	\\
2.07524750679141	0.386211723089218	\\
2.0762225679028	0.303323477506638	\\
2.07835409684398	-0.413708925247192	\\
2.08597317901668	-0.323923468589783	\\
2.08699359180767	0.263801991939545	\\
2.09361493702919	-0.446089059114456	\\
2.09649476868376	0.41489914059639	\\
2.10445398845346	0.313150435686111	\\
2.10692565499163	-0.429151266813278	\\
2.10746987514683	-0.319864511489868	\\
2.11189166390777	0.320688486099243	\\
2.11735654129951	0.322275459766388	\\
2.11987355951728	-0.568102061748505	\\
2.12300282540964	-0.139805287122726	\\
2.12459013419562	0.345957815647125	\\
2.13066725926195	0.159306615591049	\\
2.13644959841088	-0.23911252617836	\\
2.14064462877383	-0.273049116134644	\\
2.14592809944716	0.275002300739288	\\
2.14629091288396	0.245704516768456	\\
2.15034988820811	-0.0754417553544044	\\
2.1594655758076	0.040925320237875	\\
2.16157442890897	-0.168218031525612	\\
2.16447693640334	-0.0868861973285675	\\
2.16629100358732	0.168187499046326	\\
2.17005519299407	0.118778042495251	\\
2.17359262400283	-0.162541583180428	\\
2.17828652284137	-0.0918607115745544	\\
2.18493054390269	0.139652699232101	\\
2.18565617077628	0.125309005379677	\\
2.19071288305162	-0.0725119784474373	\\
2.19658592555975	-0.127719968557358	\\
2.19914829545712	0.107364118099213	\\
2.20291248486388	-0.171269878745079	\\
2.20583766819804	0.217780083417892	\\
2.20858144481381	0.0735801234841347	\\
2.21100775967238	-0.0822473838925362	\\
2.22003274391267	-0.10440382361412	\\
2.22241370709164	0.135654777288437	\\
2.22413707091642	-0.0704977586865425	\\
2.2262459240178	0.0926236733794212	\\
2.23241375244332	0.0997039675712585	\\
2.23583780425308	-0.0954008623957634	\\
2.24393307906158	-0.130954921245575	\\
2.24692628991515	0.160466328263283	\\
2.24728910335194	0.135319069027901	\\
2.25493086136445	-0.0202337726950645	\\
2.25737985206282	-0.151982173323631	\\
2.26264064689636	0.0393993966281414	\\
2.2671077873369	0.116306036710739	\\
2.27055451498646	-0.0420239865779877	\\
2.27631417829559	0.0133976256474853	\\
2.27740261860598	-0.0864894539117813	\\
2.28030512610035	-0.0446485802531242	\\
2.28155229728933	0.0491653196513653	\\
2.28715322971986	0.11059907823801	\\
2.29375189910158	-0.101901300251484	\\
2.29400133333938	-0.0988799706101418	\\
2.30123492623549	0.0621356852352619	\\
2.30617825931183	0.0795922726392746	\\
2.3088086567286	-0.0686056092381477	\\
2.31502183683373	-0.0599078349769115	\\
2.31620098050331	0.0582598336040974	\\
2.31826448192509	-0.0362865068018436	\\
2.32250486396764	0.0710470899939537	\\
2.33164322740693	-0.0976897478103638	\\
2.33279969523671	0.0902432352304459	\\
2.33586093360968	-0.172093868255615	\\
2.33951174381743	0.189703062176704	\\
2.34411493929678	-0.181524097919464	\\
2.34837799717913	0.340037226676941	\\
2.34837799717913	0.340037226676941	\\
2.3502374160427	-0.39500105381012	\\
2.36114449498637	0.296060055494308	\\
2.36348010648574	-0.314920485019684	\\
2.3672442958925	0.227240815758705	\\
2.37153002961465	-0.266884356737137	\\
2.37193819473104	-0.261818289756775	\\
2.37429648207021	0.22690512239933	\\
2.38241443271852	0.136143073439598	\\
2.38472736837809	-0.236610010266304	\\
2.38831015106644	0.160496845841408	\\
2.39268658814779	-0.0964995250105858	\\
2.39681359099134	0.0855433791875839	\\
2.39828752057832	-0.144840851426125	\\
2.40429661812525	0.14108707010746	\\
2.40724447729922	-0.118930630385876	\\
2.41275470637055	0.122714929282665	\\
2.41518102122912	-0.10223700851202	\\
2.42227855908644	-0.129795223474503	\\
2.42472754978481	0.0661336109042168	\\
2.42787949151697	-0.0777916833758354	\\
2.43100875740933	0.139713734388351	\\
2.43461421593749	-0.0912198275327683	\\
2.43726728919406	0.0551469475030899	\\
2.44354849681858	-0.0623187981545925	\\
2.44493172304636	0.0744346454739571	\\
2.45533993351444	0.125705733895302	\\
2.45685921478102	-0.0798669382929802	\\
2.46225606465335	0.0983611568808556	\\
2.46479575871092	-0.220160529017448	\\
2.47048739450066	0.341563165187836	\\
2.47196132408764	-0.267159044742584	\\
2.4751586174994	0.24927519261837	\\
2.47747155315897	-0.313516646623611	\\
2.48075954992993	-0.309366136789322	\\
2.4830951614293	0.186986908316612	\\
2.4915986013542	0.139683216810226	\\
2.49425167461077	-0.19833979010582	\\
2.49710883042553	0.138370916247368	\\
2.50236962525907	-0.169591352343559	\\
2.50679141402002	-0.133640557527542	\\
2.51023814166958	0.12714010477066	\\
2.51261910484855	0.132450327277184	\\
2.51924045007007	-0.123844109475613	\\
2.51978467022526	-0.164494767785072	\\
2.52645136712638	0.115115821361542	\\
2.52842416518896	0.130832850933075	\\
2.53261919555191	-0.0403454713523388	\\
2.53531762048807	0.0482192449271679	\\
2.53722239103125	-0.103885009884834	\\
2.54524963832036	0.0786461979150772	\\
2.55037437811509	-0.0611590929329395	\\
2.55323153392986	0.13922543823719	\\
2.5582882462052	-0.152745142579079	\\
2.56246060072835	0.162968844175339	\\
2.5661114109361	-0.196996971964836	\\
2.5663835210137	-0.260597556829453	\\
2.56996630370206	0.267769396305084	\\
2.57611145628778	0.159337133169174	\\
2.58026113497113	-0.2620929479599	\\
2.58252871895111	0.180059209465981	\\
2.58822035474084	-0.140995517373085	\\
2.59184848910879	0.400860607624054	\\
2.59443353484596	-0.386700034141541	\\
2.59724533898113	0.319254130125046	\\
2.60252880965447	-0.346201986074448	\\
2.60513653123143	0.285287022590637	\\
2.6078349561676	-0.297799617052078	\\
2.61339053691853	0.272896498441696	\\
2.62035201973705	-0.516495227813721	\\
2.62080553653305	-0.484511852264404	\\
2.62672393072077	0.339152187108994	\\
2.62831123950675	0.313180953264236	\\
2.63343597930149	-0.271309554576874	\\
2.6414405507508	-0.222022160887718	\\
2.64377616225017	0.0466933213174343	\\
2.64672402142413	0.291512817144394	\\
2.65078299674828	-0.0928067862987518	\\
2.65391226264065	0.0890530124306679	\\
2.65826602388219	-0.139713734388351	\\
2.66096444881836	-0.150242626667023	\\
2.66686016716629	0.209021270275116	\\
2.66722298060309	0.142521440982819	\\
2.6740484083828	-0.157689139246941	\\
2.67792597698856	-0.134495064616203	\\
2.6825518483077	0.0536515414714813	\\
2.68586252091846	0.259407341480255	\\
2.68837953913623	-0.139255955815315	\\
2.69327752053297	0.183050021529198	\\
2.69615735218754	-0.286416202783585	\\
2.70305080748666	-0.248207032680511	\\
2.70588528746162	0.287759035825729	\\
2.70624810089842	0.30075991153717	\\
2.70958144934898	-0.0796227902173996	\\
2.72107810012744	-0.197851493954659	\\
2.72159964444283	0.212683498859406	\\
2.72472891033519	-0.341563165187836	\\
2.72681508759677	0.408185064792633	\\
2.7327108059447	0.180822163820267	\\
2.73681513294845	-0.269997239112854	\\
2.73940017868562	0.25763726234436	\\
2.74180381770439	-0.255592525005341	\\
2.74497843527635	-0.220801413059235	\\
2.74715531589712	0.259407341480255	\\
2.75273357248786	0.142490923404694	\\
2.75824380155919	-0.23731192946434	\\
2.76348192055293	-0.0964079722762108	\\
2.7654320427757	0.145451217889786	\\
2.76937763890086	0.165349289774895	\\
2.77218944303602	-0.143284395337105	\\
2.77769967210736	-0.115421004593372	\\
2.78286976358169	0.0954008623957634	\\
2.78572691939646	-0.0663472414016724	\\
2.78769971745904	0.118259221315384	\\
2.79196277534138	0.0345469526946545	\\
2.79327797404977	-0.12683492898941	\\
2.80173606229507	-0.0895107910037041	\\
2.80359548115864	0.165807068347931	\\
2.80999006798217	0.183294162154198	\\
2.81400369162672	-0.223120823502541	\\
2.81715563335888	0.279305398464203	\\
2.82033025093084	-0.309183031320572	\\
2.82321008258541	0.263985097408295	\\
2.82779060222495	-0.200476095080376	\\
2.83488814008227	-0.196020379662514	\\
2.83613531127125	0.249732956290245	\\
2.84289271153157	-0.244361698627472	\\
2.84418523440016	0.26142156124115	\\
2.84781336876811	-0.264992207288742	\\
2.85067052458288	0.167241439223289	\\
2.85611272613481	-0.266335040330887	\\
2.85860706851278	0.248420670628548	\\
2.86622615068549	0.166295364499092	\\
2.86849373466546	-0.241523489356041	\\
2.87067061528624	0.176610618829727	\\
2.87296087510601	-0.132389292120934	\\
2.87715590546896	0.154454171657562	\\
2.88087474319611	-0.160527363419533	\\
2.88525118027746	-0.112247079610825	\\
2.88770017097583	0.137882620096207	\\
2.89409475779936	0.128421887755394	\\
2.89767754048771	-0.115909300744534	\\
2.90062539966168	0.0698873847723007	\\
2.90298368700085	-0.0929593816399574	\\
2.90883405366918	-0.0608233883976936	\\
2.91429893106091	0.111239969730377	\\
2.91890212654026	0.120151370763779	\\
2.92148717227743	-0.207190155982971	\\
2.92890217189194	-0.166020691394806	\\
2.93146454178931	0.212530896067619	\\
2.93314255393449	0.20868556201458	\\
2.93622646814725	-0.150028988718987	\\
2.94271175832997	-0.130588695406914	\\
2.94606778262033	0.103518784046173	\\
2.95232631440505	-0.114200264215469	\\
2.95336940303584	0.357921093702316	\\
2.95602247629241	-0.284707188606262	\\
2.95840343947138	0.299691766500473	\\
2.9650247846929	-0.26190984249115	\\
2.97030825536624	0.231910154223442	\\
2.97225837758901	0.224433124065399	\\
2.97767790330115	-0.164494767785072	\\
2.98216771958149	-0.16171757876873	\\
2.98312010485308	0.0963469371199608	\\
2.98613599154645	-0.129856258630753	\\
2.98801808624982	0.197180092334747	\\
2.99699771881052	0.0687887221574783	\\
2.99965079206708	-0.110049746930599	\\
3.00185034852766	0.093264564871788	\\
3.00289343715845	-0.10620441287756	\\
3.00971886493816	0.142185732722282	\\
3.0153197973687	-0.0643635392189026	\\
3.01779146390687	0.0561235398054123	\\
3.02037650964403	-0.137577444314957	\\
3.02624955215216	-0.0309762880206108	\\
3.0313516161071	0.105594046413898	\\
3.03792760964902	0.10849329829216	\\
3.0399004077116	-0.226264223456383	\\
3.04525190590434	0.139835804700851	\\
3.0482224409181	-0.234473705291748	\\
3.0483584959569	-0.23725089430809	\\
3.05162381688806	0.274117261171341	\\
3.05758756275539	0.14920499920845	\\
3.06169188975914	-0.147648543119431	\\
3.06874407593685	-0.20465712249279	\\
3.07010462632484	0.276100963354111	\\
3.07640850978916	-0.447859138250351	\\
3.07854003873033	0.298745691776276	\\
3.07953777568152	0.284981846809387	\\
3.08214549725849	-0.246284365653992	\\
3.0899459861496	-0.335398405790329	\\
3.09234962516837	0.295266568660736	\\
3.09699817232731	0.231727048754692	\\
3.09876688783169	-0.239570304751396	\\
3.10457190282042	0.284188359975815	\\
3.10715694855759	-0.340891748666763	\\
3.11044494532855	0.2474135607481	\\
3.11273520514832	-0.334879606962204	\\
3.12073977659763	-0.380077511072159	\\
3.12368763577159	0.353282272815704	\\
3.12751985269774	0.174169138073921	\\
3.13375570864267	-0.207983642816544	\\
3.13380106032227	-0.24747459590435	\\
3.1391752343548	0.381511896848679	\\
3.14327956135855	-0.593737602233887	\\
3.14894852130849	0.365672767162323	\\
3.15330228255003	0.526749491691589	\\
3.156363520923	-0.313913375139236	\\
3.16173769495553	0.166570022702217	\\
3.1645268232509	-0.356456190347672	\\
3.16527512596429	-0.199774160981178	\\
3.16985564560383	0.28644672036171	\\
3.17568333643236	0.211554303765297	\\
3.17824570632973	-0.297982722520828	\\
3.18536592002685	-0.277932077646255	\\
3.18752012480782	0.346140921115875	\\
3.18822307584161	0.32276377081871	\\
3.19112558333598	-0.309213548898697	\\
3.19844987959129	-0.396893203258514	\\
3.20182857972145	0.204626604914665	\\
3.20377870194423	-0.278603464365005	\\
3.20935695853496	0.328012943267822	\\
3.21711209574647	0.215948969125748	\\
3.21894883877024	-0.271065413951874	\\
3.2229624624148	0.295113980770111	\\
3.22652256926335	-0.414563447237015	\\
3.22706678941855	-0.248268067836761	\\
3.23010535195171	0.343974113464355	\\
3.23733894484782	0.128391370177269	\\
3.24003736978399	-0.174962610006332	\\
3.24627322572892	0.225196078419685	\\
3.24874489226709	-0.220130011439323	\\
3.25275851591164	-0.209967344999313	\\
3.25713495299298	0.28608050942421	\\
3.26194223103053	-0.179143652319908	\\
3.26466333180649	0.1778923869133	\\
3.26754316346106	-0.197393715381622	\\
3.27128467702801	0.232581555843353	\\
3.27482210803677	-0.185338914394379	\\
3.27711236785654	0.16544084250927	\\
3.28198767341349	-0.202978610992432	\\
3.28520764266505	0.136417731642723	\\
3.29259996643976	0.205999940633774	\\
3.2970671068803	-0.131626337766647	\\
3.30085397212686	0.203741565346718	\\
3.30443675481521	-0.326456487178802	\\
3.30484491993161	-0.218085274100304	\\
3.30770207574638	0.224707782268524	\\
3.31520777872008	0.199377417564392	\\
3.31885858892784	-0.177800834178925	\\
3.32457290055737	-0.117221593856812	\\
3.32743005637214	0.203131198883057	\\
3.32983369539091	0.143436998128891	\\
3.33414210495286	-0.127933591604233	\\
3.3388586796312	-0.222327336668968	\\
3.34130767032957	0.13892026245594	\\
3.3471126853183	0.147434920072556	\\
3.35121701232205	-0.0897244215011597	\\
3.35450500909301	0.125766783952713	\\
3.35874539113556	-0.13284707069397	\\
3.3635980208527	-0.175237283110619	\\
3.36668193506547	0.200537130236626	\\
3.37151188894281	-0.276467174291611	\\
3.37402890716058	0.234260082244873	\\
3.37482256155357	0.186864838004112	\\
3.37795182744593	-0.209814757108688	\\
3.38645526737083	-0.150639370083809	\\
3.3884960929528	0.111209452152252	\\
3.39037818765618	-0.0831324234604836	\\
3.39375688778634	0.115207374095917	\\
3.39874557254228	0.12485121935606	\\
3.40341679554102	-0.140140995383263	\\
3.40915378301035	-0.159245580434799	\\
3.4133488133733	0.245551928877831	\\
3.41867763572624	0.231360822916031	\\
3.42033297203162	-0.259407341480255	\\
3.42430124399657	-0.1496322453022	\\
3.42652347629694	0.105624563992023	\\
3.43579789477503	0.116763815283775	\\
3.43688633508542	-0.136082038283348	\\
3.43913124322559	0.222785115242004	\\
3.44221515743836	-0.172429576516151	\\
3.44990226713046	-0.312997847795486	\\
3.45226055446964	0.315591901540756	\\
3.45303153302283	0.291634887456894	\\
3.455163061964	-0.193639948964119	\\
3.46303157837451	-0.250526458024979	\\
3.46618352010667	0.265785694122314	\\
3.46838307656724	-0.242011785507202	\\
3.47260078276999	0.197119057178497	\\
3.47631962049714	-0.198431342840195	\\
3.47956226558851	0.200292974710464	\\
3.48695458936322	0.09820856153965	\\
3.49003850357598	-0.189733579754829	\\
3.49609295280251	0.189031645655632	\\
3.49847391598148	-0.101382486522198	\\
3.50090023084005	0.111239969730377	\\
3.50378006249461	-0.216711938381195	\\
3.51010662179874	0.0964995250105858	\\
3.51396151456469	-0.0627460554242134	\\
3.51647853278246	0.209082305431366	\\
3.51858738588384	-0.20242927968502	\\
3.52645590229434	-0.354655593633652	\\
3.52890489299271	0.262550741434097	\\
3.53232894480247	-0.215887933969498	\\
3.53690946444202	0.250404357910156	\\
3.54031084041197	-0.222022160887718	\\
3.54323602374614	0.123661004006863	\\
3.54770316418669	-0.166875213384628	\\
3.54978934144826	0.235297709703445	\\
3.55568505979619	0.207525864243507	\\
3.55944924920294	-0.187414169311523	\\
3.56663749041946	-0.224524676799774	\\
3.56901845359843	0.232215344905853	\\
3.57080984494261	0.225623339414597	\\
3.57421122091257	-0.113925598561764	\\
3.57883709223171	0.186498612165451	\\
3.58323620515286	-0.237342447042465	\\
3.58602533344822	0.185064241290092	\\
3.5880434831904	-0.154057443141937	\\
3.59357638810153	0.184728533029556	\\
3.5962067855183	-0.0819422006607056	\\
3.6043020603268	-0.148197889328003	\\
3.60697780942317	0.170049130916595	\\
3.61339507208649	0.0985747873783112	\\
3.61552660102767	-0.118015073239803	\\
3.61942684547322	-0.161564990878105	\\
3.62178513281239	0.197119057178497	\\
3.62450623358836	-0.198309272527695	\\
3.62545861885995	0.158879354596138	\\
3.63681921459961	-0.180944249033928	\\
3.63908679857959	0.191534161567688	\\
3.64366731821913	0.20447401702404	\\
3.64518659948571	-0.249244660139084	\\
3.64985782248446	0.195074319839478	\\
3.65405285284741	-0.304147452116013	\\
3.65595762339058	0.300210565328598	\\
3.65897351008395	-0.253669857978821	\\
3.66679667481485	-0.284707188606262	\\
3.66960847895002	0.305581837892532	\\
3.6710597326972	-0.173284098505974	\\
3.67797586383612	0.188268691301346	\\
3.68396228554325	-0.253547787666321	\\
3.68586705608642	0.253456234931946	\\
3.69071968580356	-0.238013848662376	\\
3.69253375298754	0.507919549942017	\\
3.6959351289575	-0.360972940921783	\\
3.69942720828666	0.299356073141098	\\
3.7042344863242	-0.274941265583038	\\
3.70917781940054	0.316263318061829	\\
3.71217103025411	0.35175633430481	\\
3.71491480686987	-0.39085054397583	\\
3.71759055596624	0.440137952566147	\\
3.72332754343557	-0.39463484287262	\\
3.72691032612393	0.477553635835648	\\
3.73140014240427	-0.366740942001343	\\
3.73534573852943	0.434064745903015	\\
3.73813486682479	-0.505722224712372	\\
3.74183102871215	0.395733505487442	\\
3.74657027923029	-0.315134137868881	\\
3.74917800080726	-0.414685517549515	\\
3.75076530959324	0.343913078308105	\\
3.75845241928535	-0.397045820951462	\\
3.76001705223153	0.34382152557373	\\
3.76895133311262	0.303750723600388	\\
3.7711508895732	-0.306802570819855	\\
3.77414410042676	-0.374980926513672	\\
3.77475634810135	0.35187840461731	\\
3.78414414577844	0.335856199264526	\\
3.78570877872462	-0.297891169786453	\\
3.79294237162073	0.268471330404282	\\
3.79448432872711	-0.321787178516388	\\
3.79480179048431	-0.332071900367737	\\
3.80013061283725	0.326883763074875	\\
3.80289706529281	-0.394116044044495	\\
3.80475648415639	0.449049353599548	\\
3.81675200341045	-0.460036009550095	\\
3.81804452627903	0.330912202596664	\\
3.8204481652978	0.452619999647141	\\
3.82364545870956	-0.440504163503647	\\
3.83040285896988	-0.24103519320488	\\
3.83183143687726	0.295937985181808	\\
3.83500605444923	-0.344889670610428	\\
3.84081106943796	0.317819744348526	\\
3.84414441788852	-0.26407665014267	\\
3.84799931065447	0.5240638256073	\\
3.85087914230904	-0.423902094364166	\\
3.8539857323616	0.294076353311539	\\
3.86085651182092	0.39414656162262	\\
3.86260255148549	-0.414868623018265	\\
3.86613998249425	0.488021492958069	\\
3.87090190885219	-0.47456282377243	\\
3.87341892706996	0.379711300134659	\\
3.87616270368573	-0.348185688257217	\\
3.88110603676207	0.30268257856369	\\
3.88641218327521	-0.365672767162323	\\
3.89375915537032	0.246681109070778	\\
3.89579998095229	-0.224463641643524	\\
3.90024444555304	-0.25788140296936	\\
3.90205851273702	0.398236036300659	\\
3.90468891015379	-0.406628608703613	\\
3.90775014852675	0.30616170167923	\\
3.91398600447168	-0.306680500507355	\\
3.91493838974326	0.275765240192413	\\
3.92228536183838	-0.299478143453598	\\
3.92323774710996	0.312082290649414	\\
3.9283624869047	0.239814445376396	\\
3.93337384750044	-0.28217414021492	\\
3.9364577617132	0.498947113752365	\\
3.94015392360056	-0.43131810426712	\\
3.94677526882208	-0.420392453670502	\\
3.94793173665187	0.413251131772995	\\
3.95380477916	-0.427106529474258	\\
3.95536941210618	0.398297071456909	\\
3.96396355539027	0.413129061460495	\\
3.96539213329766	-0.384624779224396	\\
3.96725155216123	-0.315164655447006	\\
3.97106109324759	0.389782398939133	\\
3.97462120009615	-0.420758694410324	\\
3.97775046598851	0.387096762657166	\\
3.9868888294278	0.361827462911606	\\
3.98811332477698	-0.52220219373703	\\
3.99260314105733	-0.454725801944733	\\
3.99548297271189	0.299172937870026	\\
3.99720633653667	-0.455397188663483	\\
4.00441725359299	0.448927283287048	\\
4.00564174894217	-0.441969066858292	\\
4.0118322532075	0.410657078027725	\\
4.01498419493966	-0.43745231628418	\\
4.01636742116744	0.305948048830032	\\
4.02394115166055	-0.279061257839203	\\
4.02532437788834	0.264595478773117	\\
4.0284309679409	-0.281624794006348	\\
4.03119742039646	0.327158421278	\\
4.04013170127756	0.335703611373901	\\
4.04285280205352	-0.40894803404808	\\
4.04697980489707	0.324839025735855	\\
4.04970090567304	-0.390697956085205	\\
4.05754674624375	0.441999584436417	\\
4.05897532415113	-0.355204939842224	\\
4.06366922298967	-0.39017915725708	\\
4.06682116472184	0.519150376319885	\\
4.06820439094962	-0.467055261135101	\\
4.07135633268178	0.484359264373779	\\
4.07552868720493	0.623706758022308	\\
4.07732007854911	-0.590472102165222	\\
4.0863677386292	-0.675099968910217	\\
4.08951968036136	0.530106484889984	\\
4.09405484832131	0.550675988197327	\\
4.09530201951029	-0.472273945808411	\\
4.09976915995084	-0.633198022842407	\\
4.10067619354283	0.61107212305069	\\
4.11004131538011	0.616016089916229	\\
4.11337466383068	-0.661519229412079	\\
4.11353339470927	-0.528305888175964	\\
4.11455380750026	0.637928426265717	\\
4.12131120776058	0.614062905311584	\\
4.12466723205094	-0.537369906902313	\\
4.13031351616107	0.710074186325073	\\
4.13353348541263	-0.47456282377243	\\
4.13707091642139	0.588915705680847	\\
4.14033623735255	-0.51664787530899	\\
4.1448714053125	-0.396923720836639	\\
4.15110726125742	0.464033931493759	\\
4.15360160363539	-0.428724020719528	\\
4.15534764329997	0.439069807529449	\\
4.16060843813351	-0.397198408842087	\\
4.16459938593826	0.463209927082062	\\
4.1695200431748	-0.492019414901733	\\
4.17346563929995	0.444868326187134	\\
4.17643617431372	-0.41508224606514	\\
4.18015501204087	0.320047616958618	\\
4.18772874253398	-0.863277077674866	\\
4.19085800842634	0.892788469791412	\\
4.19709386437127	-0.712820827960968	\\
4.19840906307965	0.751640379428864	\\
4.19974693762784	0.712363064289093	\\
4.20550660093697	-0.838343441486359	\\
4.20727531644134	0.805566549301147	\\
4.20965627962032	-0.850581347942352	\\
4.21584678388564	0.532456457614899	\\
4.21698057587563	-0.887386679649353	\\
4.22555204331992	-0.509170830249786	\\
4.22929355688688	0.652668833732605	\\
4.23294436709464	-0.801049828529358	\\
4.23376069732743	0.723166584968567	\\
4.24287638492692	0.498947113752365	\\
4.2439875010771	-0.548387110233307	\\
4.24859069655645	-0.705557405948639	\\
4.24992857110463	0.491714239120483	\\
4.25405557394818	0.620166659355164	\\
4.25800117007333	-0.450056463479996	\\
4.26337534410587	0.683492541313171	\\
4.26795586374541	-0.375774413347244	\\
4.27230962498696	-0.43510240316391	\\
4.27602846271412	0.479842513799667	\\
4.28029152059647	-0.619891941547394	\\
4.28285389049383	0.262520223855972	\\
4.28786525108957	0.688405990600586	\\
4.29131197873913	-0.344828635454178	\\
4.29548433326228	0.695486307144165	\\
4.29997414954263	-0.440900892019272	\\
4.30119864489181	0.289773255586624	\\
4.30793336931233	-0.546433925628662	\\
4.30793336931233	-0.546433925628662	\\
4.31407852189806	0.414685517549515	\\
4.3189991791346	-0.272011488676071	\\
4.32299012693935	0.473433643579483	\\
4.32509898004073	0.56163215637207	\\
4.32995160975787	-0.445387125015259	\\
4.33736660937238	-0.591601312160492	\\
4.33895391815836	0.469283133745193	\\
4.34133488133733	0.491317480802536	\\
4.34262740420592	-0.659688115119934	\\
4.34893128767024	0.584276854991913	\\
4.35076803069402	-0.295815914869308	\\
4.35827373366773	0.544755399227142	\\
4.3605639934875	-0.420728176832199	\\
4.36600619503943	0.210974454879761	\\
4.368999405893	-0.547929346561432	\\
4.37190191338736	0.485549479722977	\\
4.37580215783291	-0.385998100042343	\\
4.37868198948748	0.48286384344101	\\
4.38357997088422	-0.219641715288162	\\
4.38843260060136	0.298135310411453	\\
4.38933963419335	-0.686941146850586	\\
4.39614238613327	-0.467513054609299	\\
4.40113107088921	0.19244971871376	\\
4.40289978639359	-0.590807795524597	\\
4.40888620810072	0.37250891327858	\\
4.40924902153751	0.491988897323608	\\
4.41650529027343	-0.256538599729538	\\
4.41795654402061	0.43769645690918	\\
4.42020145216078	-0.475325793027878	\\
4.42843278200808	-0.473220020532608	\\
4.43156204790044	0.705252230167389	\\
4.4352128581082	-0.342387169599533	\\
4.43655073265638	0.416608184576035	\\
4.44505417258128	0.478316605091095	\\
4.44639204712947	-0.371013522148132	\\
4.44999750565762	0.253944516181946	\\
4.45471408033597	-0.60951566696167	\\
4.45861432478152	0.397930830717087	\\
4.4598841718103	-0.395458847284317	\\
4.46523567000304	-0.360698252916336	\\
4.46761663318201	0.424298822879791	\\
4.47217447698176	0.41267129778862	\\
4.47312686225334	-0.41303750872612	\\
4.48312690760502	0.296304196119308	\\
4.48668701445358	-0.451185643672943	\\
4.48963487362755	-0.544846951961517	\\
4.4932403321557	0.400860607624054	\\
4.49795690683404	-0.433881640434265	\\
4.50226531639599	0.642536699771881	\\
4.50791160050612	-0.492904454469681	\\
4.5097256676901	0.601855516433716	\\
4.51054199792289	0.438550978899002	\\
4.51469167660624	-0.6541947722435	\\
4.51972571304178	-0.531601905822754	\\
4.52346722660874	0.735465586185455	\\
4.52834253216568	-0.510666191577911	\\
4.52956702751486	0.752861082553864	\\
4.53632442777518	0.629505276679993	\\
4.53977115542474	-0.713370144367218	\\
4.54383013074889	0.530594825744629	\\
4.54872811214563	-0.685903489589691	\\
4.55067823436841	0.588305294513702	\\
4.55435172041597	-0.508316278457642	\\
4.55868280581771	0.48081910610199	\\
4.56097306563749	-0.380291134119034	\\
4.56943115388278	-0.483260601758957	\\
4.57045156667377	0.319864511489868	\\
4.57260577145475	-0.604846358299255	\\
4.5765967192595	0.246406450867653	\\
4.58301398192282	-0.505600154399872	\\
4.58618859949478	0.365733832120895	\\
4.59403444006549	0.573168098926544	\\
4.59510020453607	-0.38843959569931	\\
4.59675554084146	-0.463423579931259	\\
4.59750384355485	0.34748375415802	\\
4.60779867482392	0.627979397773743	\\
4.61020231384269	-0.479995131492615	\\
4.61396650324945	-0.486587107181549	\\
4.6175492859378	0.627887785434723	\\
4.62133615118436	0.728202164173126	\\
4.62550850570751	-0.604235947132111	\\
4.62723186953229	-0.496780306100845	\\
4.634510814108	0.487716287374496	\\
4.6345334899478	0.441908031702042	\\
4.64135891772752	-0.588671505451202	\\
4.64326368827069	-0.412945955991745	\\
4.64711858103665	0.528794229030609	\\
4.6509734738026	0.502334654331207	\\
4.65510047664615	-0.653370797634125	\\
4.66074676075628	0.552415549755096	\\
4.66360391657105	-0.384044915437698	\\
4.66875133220559	-0.626239836215973	\\
4.67299171424814	0.348551899194717	\\
4.67435226463612	0.664571046829224	\\
4.68029333466365	-0.437116622924805	\\
4.68256091864362	-0.669606626033783	\\
4.68829790611295	0.644367814064026	\\
4.69113238608792	0.349375903606415	\\
4.69605304332446	-0.65376752614975	\\
4.69705078027565	-0.28577533364296	\\
4.70179003079379	0.408093512058258	\\
4.70884221697151	-0.451307713985443	\\
4.71126853183008	0.375377655029297	\\
4.71593975482882	0.440504163503647	\\
4.71925042743958	-0.190740689635277	\\
4.72138195638075	0.217139199376106	\\
4.7258717726611	-0.488052010536194	\\
4.72986272046585	0.472151845693588	\\
4.73553168041578	-0.295052945613861	\\
4.73723236840076	-0.384929955005646	\\
4.7425838665935	0.532242834568024	\\
4.74344554850589	0.459181487560272	\\
4.74909183261602	-0.300424218177795	\\
4.75147279579499	-0.357036054134369	\\
4.75684696982753	0.285470128059387	\\
4.76151819282627	0.26154363155365	\\
4.76262930897646	-0.494125187397003	\\
4.76938670923678	-0.190282911062241	\\
4.77036177034817	0.414136171340942	\\
4.77773141828308	-0.314218580722809	\\
4.78217588288382	0.384228020906448	\\
4.7838765708688	0.338816493749619	\\
4.78920539322174	-0.395916610956192	\\
4.7922212799151	-0.283516943454742	\\
4.79752742642824	0.225531786680222	\\
4.80219864942698	0.378521084785461	\\
4.80315103469857	-0.315469831228256	\\
4.80575875627554	-0.313272505998611	\\
4.80732338922172	0.350596636533737	\\
4.81659780769981	-0.30820643901825	\\
4.81809441312659	0.213751643896103	\\
4.82791305175987	0.236793115735054	\\
4.82884276119166	-0.275399029254913	\\
4.82918289878866	-0.206488236784935	\\
4.83648451920417	0.315683454275131	\\
4.83677930512157	0.307748645544052	\\
4.8424255892317	-0.308633685112	\\
4.84648456455585	-0.290444642305374	\\
4.8507476224382	0.249916076660156	\\
4.85621249982993	-0.310373246669769	\\
4.85936444156209	0.366100043058395	\\
4.86403566456084	0.370128482580185	\\
4.866847468696	-0.248908966779709	\\
4.86993138290877	-0.254402309656143	\\
4.87485204014531	0.32541885972023	\\
4.87786792683867	0.15509507060051	\\
4.87854820203266	-0.507034540176392	\\
4.8836729418274	-0.374645233154297	\\
4.88643939428297	0.341257959604263	\\
4.89151878239811	0.332316040992737	\\
4.89634873627545	-0.219672232866287	\\
4.90444401108395	0.220191046595573	\\
4.90555512723414	-0.353282272815704	\\
4.90709708434052	0.222113713622093	\\
4.9089565032041	-0.429334402084351	\\
4.91603136522161	0.272133558988571	\\
4.9168703712942	-0.267189562320709	\\
4.92256200708393	-0.28186896443367	\\
4.92811758783487	0.436384171247482	\\
4.93621286264337	-0.348338276147842	\\
4.93657567608016	0.267799913883209	\\
4.93827636406514	0.309823900461197	\\
4.93925142517653	-0.396404922008514	\\
4.94825373357702	-0.355784773826599	\\
4.95061202091619	0.233100369572639	\\
4.96045333538928	-0.263435781002045	\\
4.96108825890367	0.330271303653717	\\
4.96340119456324	-0.454084903001785	\\
4.96451231071343	0.397900313138962	\\
4.97197266200754	-0.402172923088074	\\
4.97655318164708	0.357921093702316	\\
4.98079356368963	-0.515213489532471	\\
4.98267565839301	0.359050273895264	\\
4.98566886924657	0.548051416873932	\\
4.99063487816271	-0.586107969284058	\\
5	-0.336344480514526	\\
};
\addlegendentry{x};

\end{axis}

\begin{axis}[%
width=\figurewidth,
height=\figureheight,
scale only axis,
xmin=0,
xmax=22050,
xlabel={Frequency (Hz)},
ymin=0,
ymax=0.0178725135586674,
ylabel={|X(f)|},
name=plot7,
at=(plot5.below south west),
anchor=above north west,
title={Single-Sided Amplitude Spectrum of x(t)},
legend style={draw=black,fill=white,legend cell align=left}
]
\addplot [color=blue,solid]
  table[row sep=crcr]{
0	0.000285974608991895	\\
23.0472564697266	1.44574996146449e-05	\\
33.1409454345703	0.0136641261415982	\\
49.7955322265625	0.0151445397044262	\\
61.2350463867188	0.000185201986899367	\\
72.1698760986328	0.0140560239691797	\\
77.8896331787109	0.000154798619294018	\\
129.703903198242	0.00965047015588607	\\
131.890869140625	6.4781587715486e-05	\\
143.498611450195	0.000302589730832128	\\
145.853805541992	0.0178725135586674	\\
199.350357055664	0.00015452762502492	\\
204.733657836914	0.00556231830316185	\\
226.93977355957	0.0138349599591935	\\
235.855865478516	0.000206436247812676	\\
245.949554443359	0.00744278668377071	\\
249.818801879883	0.000198967134868559	\\
291.034698486328	8.18711843172348e-05	\\
298.436737060547	0.00658732946231171	\\
325.185012817383	0.000100317656682763	\\
332.92350769043	0.00520485234745481	\\
350.75569152832	0.000208988108037394	\\
369.429016113281	0.00668081400821382	\\
385.242462158203	5.84178804048836e-05	\\
410.813140869141	0.00309157769728083	\\
420.065689086914	0.000121092942815582	\\
445.468139648438	0.0128942765442175	\\
445.804595947266	0.0101956356981313	\\
462.627410888672	6.17529312811602e-05	\\
496.10481262207	4.30353304274976e-05	\\
510.57243347168	0.00203833815110943	\\
514.946365356445	0.00314003334083228	\\
541.694641113281	4.38055079577798e-05	\\
558.180999755859	9.32176733981672e-05	\\
580.050659179688	0.00420476262511735	\\
583.919906616211	0.00428652070691963	\\
602.593231201172	3.34298656033203e-05	\\
622.948837280273	8.90636394599453e-05	\\
642.967987060547	0.00338028184578015	\\
655.92155456543	0.000107793309327184	\\
669.043350219727	0.00203943505962753	\\
703.530120849609	0.00278466601746104	\\
720.184707641602	1.17925375874389e-05	\\
720.184707641602	1.17925375874389e-05	\\
745.923614501953	0.00114780311180155	\\
758.540725708008	0.00104715825188733	\\
762.914657592773	1.83018153513858e-05	\\
801.775360107422	1.67298485826127e-05	\\
808.672714233398	0.00105595936524638	\\
838.449096679688	0.00130091107722288	\\
846.01936340332	2.0908550559123e-05	\\
858.972930908203	7.13613657337512e-05	\\
874.618148803711	0.00555075789686029	\\
892.282104492188	0.00157892168365651	\\
908.600234985352	5.5254834390149e-05	\\
939.049530029297	1.2812132877606e-05	\\
946.956253051758	0.00115066083077035	\\
984.639358520508	0.00101001273121012	\\
989.181518554688	1.63268533328282e-05	\\
997.256469726563	3.66469265147312e-05	\\
1002.97622680664	0.00161872567743191	\\
1052.60353088379	2.94701530019615e-05	\\
1056.97746276855	0.00174380855615071	\\
1071.78153991699	9.06005907628696e-06	\\
1082.04345703125	0.00136812406687266	\\
1100.38032531738	1.33318370128573e-05	\\
1112.99743652344	0.000808662413127315	\\
1142.60559082031	2.24354193730288e-06	\\
1154.04510498047	0.00180495400790483	\\
1170.0267791748	5.80830564999509e-05	\\
1177.42881774902	0.00343638172897228	\\
1209.3921661377	0.00135000027434775	\\
1226.55143737793	1.64381060490064e-05	\\
1241.35551452637	0.00125339409113619	\\
1263.89808654785	9.03026375809221e-06	\\
1272.8141784668	2.13870255800837e-05	\\
1301.5811920166	0.0017945123185436	\\
1317.73109436035	0.00576779702530577	\\
1322.77793884277	7.00197190843918e-05	\\
1343.13354492188	0.00183416346259219	\\
1357.09648132324	6.75618534635879e-05	\\
1375.93803405762	0.0011071790966938	\\
1386.36817932129	9.55152356491985e-06	\\
1422.20077514648	0.000784953165421207	\\
1427.58407592773	2.5959095018638e-05	\\
1457.69691467285	0.000820185893576994	\\
1468.63174438477	1.04913771564171e-05	\\
1495.04356384277	0.000825388203736847	\\
1505.64193725586	9.78258496416086e-06	\\
1522.96943664551	0.000869057599139045	\\
1531.71730041504	1.17705826795837e-05	\\
1553.58695983887	0.000916355265803494	\\
1573.1014251709	5.69592044353957e-06	\\
1589.9242401123	1.19027671391214e-05	\\
1607.08351135254	0.00087826561722034	\\
1613.98086547852	4.03446966665685e-05	\\
1620.03707885742	0.00140370486649229	\\
1662.43057250977	0.0014500855002241	\\
1672.3560333252	7.808575442713e-06	\\
1689.34707641602	1.72344045461173e-05	\\
1690.3564453125	0.0011636937523721	\\
1717.94586181641	5.62475290407825e-05	\\
1748.73161315918	0.00287526607411744	\\
1756.80656433105	0.00326705778284667	\\
1766.56379699707	5.21003904843216e-05	\\
1789.6110534668	5.93948589380979e-05	\\
1808.45260620117	0.00169346367895257	\\
1825.44364929199	0.000813241831767059	\\
1843.78051757813	2.21224756412082e-05	\\
1872.88398742676	1.05368850948091e-05	\\
1881.63185119629	0.000718634388467552	\\
1888.69743347168	0.000645268946609335	\\
1914.0998840332	1.82065168170544e-06	\\
1941.18461608887	2.27396608857701e-05	\\
1952.62413024902	0.000879504132384473	\\
1961.8766784668	0.000995512052268712	\\
1967.59643554688	2.28799943371952e-05	\\
1997.03636169434	4.82015853452377e-05	\\
2004.10194396973	0.00141142808001938	\\
2023.78463745117	0.00142411408482736	\\
2054.90684509277	8.01675860399154e-06	\\
2061.13128662109	0.00125256228318929	\\
2068.02864074707	4.66953302383435e-05	\\
2093.26286315918	2.18310793990156e-05	\\
2101.33781433105	0.000855332430627615	\\
2133.63761901855	0.00106511745746675	\\
2141.3761138916	3.17094658526365e-05	\\
2176.19934082031	3.35138071050726e-05	\\
2184.61074829102	0.00140336525608684	\\
2200.25596618652	0.00155404010197964	\\
2216.06941223145	7.36653220353047e-06	\\
2233.56513977051	4.58759513523657e-05	\\
2263.00506591797	0.00196549714789431	\\
2264.01443481445	0.00215836489277684	\\
2293.79081726074	3.61285427677847e-05	\\
2324.07188415527	0.00010991384442101	\\
2328.61404418945	0.00269094008944206	\\
2343.41812133789	4.72114795305983e-05	\\
2363.43727111816	0.00382434809379723	\\
2370.67108154297	0.00342017290330222	\\
2390.01731872559	5.81629235595233e-05	\\
2402.63442993164	0.00143659089618823	\\
2417.94319152832	1.11680989071309e-05	\\
2440.14930725098	0.00129160849814629	\\
2451.42059326172	3.30926931389583e-05	\\
2469.25277709961	0.00153775258050903	\\
2484.7297668457	1.32628442166874e-05	\\
2516.02020263672	1.24752476918879e-06	\\
2527.29148864746	0.000704577648960403	\\
2540.41328430176	0.000769803773303224	\\
2550.67520141602	5.86994816800804e-06	\\
2595.25566101074	2.43714063055731e-05	\\
2597.27439880371	0.000985954115459278	\\
2631.76116943359	5.32428550244758e-05	\\
2634.62104797363	0.00194610807991245	\\
2643.36891174316	0.00241264195621981	\\
2673.6499786377	6.70755068065598e-05	\\
2676.00517272949	0.0013843916043155	\\
2690.13633728027	2.8566362738526e-05	\\
2721.93145751953	1.82841992015304e-05	\\
2740.60478210449	0.00154430309534419	\\
2767.35305786133	0.00130450771824477	\\
2774.75509643555	1.52597562654939e-05	\\
2787.20397949219	0.00103551034460242	\\
2792.0825958252	9.02132158513128e-06	\\
2826.90582275391	1.75059087763563e-05	\\
2832.28912353516	0.000556976223840329	\\
2852.30827331543	2.48938088078607e-05	\\
2873.33679199219	0.00102657814201139	\\
2888.30909729004	2.85492624904974e-05	\\
2910.34698486328	0.0012389184446509	\\
2929.18853759766	2.21305701206205e-05	\\
2946.34780883789	0.00291105880694283	\\
2955.60035705566	0.00169795268958336	\\
2980.66635131836	4.30023881808905e-05	\\
2989.75067138672	8.0166108514126e-06	\\
2992.61054992676	0.00105887112624302	\\
3029.95719909668	3.72728123865324e-05	\\
3045.26596069336	0.00111790390944654	\\
3067.80853271484	5.93653870358879e-05	\\
3078.0704498291	0.00122151678043101	\\
3109.86557006836	0.00173889507898381	\\
3116.59469604492	2.54570196922649e-05	\\
3135.26802062988	0.000584428758519469	\\
3144.18411254883	1.58527620601887e-05	\\
3155.7918548584	1.84973842399845e-05	\\
3172.95112609863	0.000775116496890352	\\
3189.2692565918	0.000618667213242457	\\
3195.49369812012	1.00719190979857e-05	\\
3240.24238586426	3.14329260336388e-05	\\
3254.37355041504	0.000714558628435508	\\
3261.77558898926	9.93749222942484e-06	\\
3269.85054016113	0.000903205061542628	\\
3295.58944702148	4.29527041703118e-05	\\
3311.73934936523	0.00138518301027909	\\
3328.39393615723	2.03225962832831e-05	\\
3355.6468963623	0.00135940379671089	\\
3362.7124786377	0.00110317584222026	\\
3384.41390991211	1.37919948958595e-05	\\
3403.59191894531	9.42647921476621e-06	\\
3413.68560791016	0.00100513431764185	\\
3437.91046142578	0.000998775907435779	\\
3442.45262145996	3.00354581270803e-05	\\
3464.32228088379	2.37128937229099e-05	\\
3489.5565032959	0.00163170733106384	\\
3503.18298339844	0.00118421948340393	\\
3526.56669616699	3.3344839355269e-05	\\
3541.87545776367	0.0012447669825837	\\
3545.40824890137	3.41292863178165e-05	\\
3583.93249511719	0.00121285973886014	\\
3594.02618408203	3.52415414805929e-05	\\
3613.87710571289	0.00112012343275908	\\
3623.12965393066	6.49393562056157e-05	\\
3652.40135192871	0.00113271696403707	\\
3666.19606018066	3.88116897856073e-05	\\
3670.90644836426	0.00111348547899443	\\
3680.32722473145	3.92785052863159e-05	\\
3706.73904418945	0.000886993937961824	\\
3735.33782958984	3.40031576713427e-05	\\
3751.65596008301	0.00103128625302982	\\
3764.44129943848	1.47459529131878e-05	\\
3784.79690551758	0.000813473324301724	\\
3790.01197814941	1.73301760726017e-05	\\
3818.77899169922	0.000498498949080998	\\
3830.5549621582	1.31543856546458e-05	\\
3841.99447631836	4.13830425360039e-06	\\
3843.6767578125	0.00073114023578958	\\
3890.78063964844	0.00114734169153986	\\
3900.70610046387	1.32886173878592e-05	\\
3915.5101776123	0.000909363475143503	\\
3941.92199707031	2.61681864412508e-05	\\
3955.88493347168	0.000890035919169219	\\
3969.51141357422	3.25946243414482e-05	\\
3984.48371887207	0.00125638248938498	\\
4008.20388793945	1.04785319091237e-05	\\
4013.92364501953	0.00106540556077702	\\
4042.52243041992	2.88078830258501e-07	\\
4058.16764831543	0.000992242483541499	\\
4068.26133728027	2.29461254684637e-05	\\
4081.88781738281	0.00108897251462597	\\
4103.92570495605	1.99751404546258e-05	\\
4120.58029174805	3.2706345282469e-05	\\
4123.27194213867	0.00148168578741321	\\
4152.54364013672	0.000548752647639823	\\
4166.00189208984	3.98822241554904e-06	\\
4188.71269226074	0.000639220702951094	\\
4197.62878417969	2.59921211351435e-06	\\
4240.69519042969	1.66831720478058e-06	\\
4249.2748260498	0.000607818034137455	\\
4273.83613586426	1.4213502464645e-05	\\
4284.43450927734	0.000870182981284846	\\
4289.48135375977	0.00108929050103423	\\
4300.07972717285	1.64826183405334e-05	\\
4325.14572143555	2.64932329256771e-05	\\
4327.33268737793	0.000900236654804006	\\
4356.09970092773	0.000899331239522625	\\
4377.63290405273	2.68795373131099e-05	\\
4401.68952941895	0.00143588771207381	\\
4418.17588806152	1.03895332778126e-05	\\
4424.56855773926	0.00068093247657327	\\
4445.59707641602	1.03807666326279e-05	\\
4465.27976989746	0.000657420108516742	\\
4486.98120117188	2.39306443061077e-05	\\
4519.95391845703	1.17873683498571e-05	\\
4522.98202514648	0.000462207940091823	\\
4530.3840637207	0.000660438322612606	\\
4552.59017944336	1.11096823621254e-05	\\
4561.33804321289	0.000586302660686563	\\
4569.24476623535	1.70512849297634e-05	\\
4610.62889099121	3.34026842499803e-05	\\
4620.21789550781	0.000819636343412381	\\
4629.80690002441	0.000735322271475459	\\
4657.39631652832	2.39462471574035e-05	\\
4687.84561157227	8.89491604625578e-06	\\
4689.35966491699	0.000820841046896115	\\
4719.3042755127	0.00119430298222713	\\
4722.66883850098	5.30449835720895e-05	\\
4750.25825500488	0.000790381244426637	\\
4752.78167724609	1.98925532194778e-05	\\
4793.99757385254	2.98396201290613e-05	\\
4795.51162719727	0.00071545096660171	\\
4802.24075317383	0.000665168946294428	\\
4811.32507324219	1.19051477290318e-05	\\
4854.22325134277	4.73990870899078e-06	\\
4866.67213439941	0.000535890447498911	\\
4888.71002197266	0.00046117670941628	\\
4896.28028869629	1.23947460274463e-05	\\
4923.8697052002	0.000580620812417867	\\
4935.81390380859	8.17180601670132e-06	\\
4949.1039276123	9.30222499733025e-06	\\
4960.71166992188	0.000736791035707281	\\
4978.71208190918	0.000854750323171286	\\
4997.21717834473	1.49426411024891e-05	\\
5014.37644958496	0.000791592363806461	\\
5034.05914306641	1.85703376140131e-05	\\
5044.32106018066	0.000583816870688952	\\
5046.17156982422	2.17683407110288e-05	\\
5076.28440856934	1.40628575663287e-05	\\
5086.54632568359	0.000537991038598737	\\
5110.77117919922	0.000397559569667123	\\
5119.35081481934	7.31837285625364e-06	\\
5171.33331298828	9.95408429637556e-06	\\
5171.66976928711	0.000456643484204288	\\
5187.31498718262	1.23819167644486e-05	\\
5198.41804504395	0.000582487852826768	\\
5216.92314147949	8.59566186512746e-06	\\
5239.29748535156	0.000505890362973268	\\
5258.30726623535	2.1031321840437e-05	\\
5261.3353729248	0.000612198607631082	\\
5292.28935241699	0.00068522867477235	\\
5301.37367248535	1.85953960054229e-05	\\
5321.0563659668	1.91891415438739e-05	\\
5341.91665649414	0.000632379717362783	\\
5353.52439880371	7.28517933000051e-06	\\
5374.72114562988	0.000932021911264887	\\
5385.99243164063	1.26407766549231e-05	\\
5399.78713989258	0.000566712339281258	\\
5442.34886169434	1.74262981077269e-05	\\
5445.88165283203	0.000648340950963156	\\
5452.77900695801	0.000449799519184816	\\
5478.85437011719	1.72087077388938e-06	\\
5502.74276733398	0.000353541482215354	\\
5517.37861633301	9.08249002361197e-06	\\
5536.72485351563	0.000370627171072953	\\
5539.24827575684	7.95451919090994e-06	\\
5570.03402709961	0.00035849024367965	\\
5576.59492492676	3.93120372631287e-06	\\
5590.05317687988	4.44504894948727e-06	\\
5603.67965698242	0.000408367707540974	\\
5635.13832092285	0.000397092267160308	\\
5650.27885437012	1.95625609489378e-05	\\
5673.83079528809	0.000442495951496692	\\
5675.1766204834	1.54547647543729e-05	\\
5701.25198364258	2.30544825834232e-05	\\
5703.60717773438	0.000548809605414545	\\
5738.93508911133	3.49410674301476e-05	\\
5741.12205505371	0.000581974726183992	\\
5774.59945678711	0.000911482487635224	\\
5791.59049987793	1.33928283900078e-06	\\
5807.06748962402	0.000355983739588596	\\
5816.48826599121	8.49648327500026e-06	\\
5843.06831359863	0.000514170641591764	\\
5862.24632263184	2.08782130119714e-06	\\
5865.94734191895	1.43580764109231e-05	\\
5887.14408874512	0.000392389359021435	\\
5900.09765625	1.42993093398365e-05	\\
5923.64959716797	0.000502192911370199	\\
5946.36039733887	2.2708117680121e-05	\\
5956.95877075195	0.000479221426437542	\\
5971.25816345215	5.61471365022118e-06	\\
6000.02517700195	0.000432556711423553	\\
6014.66102600098	0.000590456681297658	\\
6028.96041870117	2.75729284841852e-05	\\
6043.09158325195	0.000470143705055265	\\
6068.66226196289	1.35226336080948e-05	\\
6074.21379089355	2.61283674280576e-06	\\
6083.46633911133	0.000424351589241602	\\
6119.46716308594	0.000491970732308265	\\
6130.90667724609	1.0899223099527e-05	\\
6138.98162841797	0.000448865985461375	\\
6164.04762268066	3.70446613134825e-06	\\
6178.68347167969	9.27309450558417e-06	\\
6181.54335021973	0.000389084626850481	\\
6212.49732971191	1.14361685218285e-05	\\
6238.06800842285	0.000344618728234032	\\
6241.43257141113	6.8045385245739e-06	\\
6243.78776550293	0.000303839302949228	\\
6283.32138061523	0.000407268972888994	\\
6306.03218078613	1.41902111301357e-05	\\
6318.8175201416	0.000439299176649894	\\
6337.65907287598	8.29311990019659e-06	\\
6345.73402404785	0.000518017467901363	\\
6366.08963012695	1.05386287718591e-05	\\
6387.11814880371	0.000426954730802616	\\
6399.90348815918	1.25587463722978e-05	\\
6415.54870605469	0.000387418504668479	\\
6429.00695800781	2.82256205158688e-06	\\
6457.26928710938	0.000399215896070542	\\
6459.62448120117	4.38107381035466e-06	\\
6504.70962524414	4.45697286367178e-06	\\
6510.59761047363	0.000238237691408202	\\
6518.67256164551	0.000287675480472032	\\
6523.3829498291	3.92376552361333e-06	\\
6569.30923461914	3.73956650294935e-06	\\
6583.77685546875	0.00025297499799195	\\
6588.99192810059	0.000408052799221594	\\
6603.4595489502	8.1108497982701e-06	\\
6630.03959655762	1.78327240544233e-05	\\
6635.59112548828	0.000395159095178401	\\
6656.1149597168	0.00037070942986254	\\
6673.10600280762	2.10918078524322e-05	\\
6689.42413330078	0.000275324268510665	\\
6695.14389038086	8.21013375495682e-06	\\
6721.55570983887	0.000258447397946469	\\
6739.38789367676	2.30294687627724e-06	\\
6770.51010131836	8.36094888901166e-06	\\
6780.77201843262	0.000318261946822724	\\
6794.06204223633	6.14459174208808e-06	\\
6800.28648376465	0.00028113640032481	\\
6849.57733154297	4.60849829578201e-06	\\
6855.29708862305	0.000309504319145701	\\
6860.51216125488	0.000219363248806889	\\
6878.6808013916	4.04415667251127e-06	\\
6897.8588104248	0.000321469217019112	\\
6911.31706237793	5.72419595208466e-07	\\
6941.42990112305	4.07308339199007e-06	\\
6952.53295898438	0.000264596032329742	\\
6964.30892944336	6.24602620427858e-06	\\
6990.88897705078	0.000284719356506459	\\
6995.9358215332	9.68980073327125e-06	\\
7020.32890319824	0.00030320950406638	\\
7053.97453308105	3.03110707903425e-06	\\
7061.04011535645	0.00049501902599767	\\
7065.2458190918	0.000407771469244315	\\
7079.20875549316	1.19697755137245e-05	\\
7111.17210388184	6.949839811374e-06	\\
7115.37780761719	0.000305686369283745	\\
7139.60266113281	0.00029217472089252	\\
7159.95826721191	3.9680214130157e-06	\\
7188.22059631348	0.000205140673562629	\\
7188.5570526123	8.04187231468424e-06	\\
7205.2116394043	0.000227549928368787	\\
7216.81938171387	2.89128935795482e-06	\\
7237.84790039063	5.56041796289489e-06	\\
7267.96073913574	0.000237346765010443	\\
7275.02632141113	7.02359328281751e-06	\\
7294.03610229492	0.000200630915607011	\\
7304.46624755859	1.02963643925022e-05	\\
7325.99945068359	0.000242317629272491	\\
7339.2894744873	0.000260496277510049	\\
7370.24345397949	1.66746390722329e-06	\\
7394.3000793457	5.57282482940589e-06	\\
7397.49641418457	0.000213335003215314	\\
7419.87075805664	1.25817805980953e-05	\\
7439.88990783691	0.000300193144393491	\\
7452.00233459473	9.87659211003612e-06	\\
7461.92779541016	0.000366996264557391	\\
7483.46099853516	5.17383246834683e-06	\\
7486.65733337402	0.000175766297374381	\\
7523.33106994629	3.20799140423886e-06	\\
7531.74247741699	0.000180908846534996	\\
7570.0984954834	0.000203523122429164	\\
7575.81825256348	7.3763418240543e-07	\\
7579.51927185059	7.12014082498027e-06	\\
7604.92172241211	0.000247847495506065	\\
7614.17427062988	5.22985855942867e-06	\\
7640.7543182373	0.000187427687507834	\\
7666.15676879883	0.000182962938689767	\\
7667.50259399414	5.28565296292386e-06	\\
7681.97021484375	4.54037567687488e-06	\\
7697.95188903809	0.000200968709900229	\\
7731.76574707031	0.000234056703463955	\\
7736.81259155273	7.4857651776021e-06	\\
7755.14945983887	3.38497801902854e-06	\\
7780.38368225098	0.000249084242286721	\\
7787.61749267578	3.41278031660304e-06	\\
7807.46841430664	0.000274438066047688	\\
7838.92707824707	7.83011004806554e-06	\\
7846.32911682129	0.000236873083751635	\\
7855.07698059082	0.00022265680066876	\\
7886.19918823242	8.27384162673063e-06	\\
7892.76008605957	0.000134744449690899	\\
7920.01304626465	3.67237595772809e-06	\\
7923.71406555176	1.89470909097409e-06	\\
7949.11651611328	0.000156059188554813	\\
7959.37843322754	0.00016920673067499	\\
7987.6407623291	5.67819794616493e-06	\\
8017.08068847656	6.20304069428437e-06	\\
8023.97804260254	0.00017716486665207	\\
8033.56704711914	5.35912819952213e-06	\\
8042.6513671875	0.000241844191324368	\\
8080.33447265625	0.000258964263732751	\\
8091.77398681641	2.84440736183482e-06	\\
8107.08274841309	0.000219780734740063	\\
8119.1951751709	4.85440700381593e-06	\\
8138.54141235352	0.000196479184937573	\\
8143.0835723877	1.51380678951707e-06	\\
8162.59803771973	5.0045258368629e-06	\\
8172.18704223633	0.00018220254379367	\\
8204.31861877441	0.000201556595519526	\\
8221.98257446289	5.20099473624265e-06	\\
8250.24490356445	0.000201313614348805	\\
8254.28237915039	1.00816169385315e-05	\\
8272.95570373535	0.000186796531476167	\\
8277.83432006836	6.52416477131187e-06	\\
8302.39562988281	0.000186193124283294	\\
8310.97526550293	3.52211637698318e-06	\\
8349.66773986816	0.000267791707510701	\\
8355.38749694824	1.03205898076592e-05	\\
8376.07955932617	2.3388125312973e-06	\\
8382.64045715332	0.000241165780097213	\\
8409.05227661133	0.000180919140879021	\\
8416.45431518555	1.7631055022316e-06	\\
8455.3150177002	0.000170472234860606	\\
8461.03477478027	6.35624683938302e-06	\\
8476.00708007813	0.000153870602916062	\\
8494.8486328125	5.10962178306559e-06	\\
8505.44700622559	0.000145162279971668	\\
8526.98020935059	5.15403192770101e-06	\\
8554.73785400391	0.000183350057360208	\\
8561.97166442871	3.7996652373407e-06	\\
8578.96270751953	0.000105203118134224	\\
8590.73867797852	2.34572080156793e-06	\\
8616.64581298828	0.000173647621157046	\\
8628.42178344727	3.60148176322635e-06	\\
8660.55335998535	4.31398667447443e-06	\\
8668.62831115723	0.000187422514442143	\\
8678.38554382324	0.000165992786883115	\\
8705.13381958008	3.23147673844241e-06	\\
8713.7134552002	0.000187664590575898	\\
8730.03158569336	6.98964404240817e-06	\\
8758.96682739258	0.00016969317777764	\\
8764.85481262207	4.24250345350123e-06	\\
8783.69636535645	3.48555312867717e-06	\\
8792.61245727539	0.000141123420426463	\\
8819.52896118164	8.46409486482464e-07	\\
8838.37051391602	0.000168161587269911	\\
8847.95951843262	0.000130864303647316	\\
8852.83813476563	9.09003041310729e-07	\\
8893.212890625	1.48873076100327e-06	\\
8910.03570556641	0.000127197670883208	\\
8932.24182128906	4.73803246655247e-06	\\
8949.06463623047	0.000172910519653477	\\
8965.0463104248	3.33621674924482e-06	\\
8971.94366455078	0.000159359589371505	\\
8991.96281433105	3.20711419129829e-06	\\
9016.52412414551	0.000193813369120459	\\
9022.24388122559	0.000194916780629513	\\
9039.06669616699	6.21641040035762e-06	\\
9060.43167114258	0.000148915613883975	\\
9065.478515625	4.58628905773659e-06	\\
9098.9559173584	0.000180816514077713	\\
9110.05897521973	3.05358827422094e-06	\\
9129.74166870117	3.04916243835557e-06	\\
9149.92904663086	0.000154370637627841	\\
9161.53678894043	0.000141143968208604	\\
9165.9107208252	5.60097293692854e-06	\\
9210.99586486816	0.000127510492734912	\\
9224.45411682129	2.82445553067934e-06	\\
9225.63171386719	3.8267127910422e-06	\\
9253.05290222168	0.000149334527472243	\\
9271.72622680664	3.72970184988643e-06	\\
9281.48345947266	0.000149242539667766	\\
9322.6993560791	6.5024097017343e-06	\\
9326.2321472168	0.000150477090395858	\\
9343.89610290527	1.79505316401159e-06	\\
9357.3543548584	0.00017623192709476	\\
9370.64437866211	6.99233536921634e-06	\\
9386.1213684082	0.000199389507828367	\\
9397.56088256836	0.000171797129118641	\\
9404.12178039551	6.3896742131557e-06	\\
9433.56170654297	5.43049269784147e-06	\\
9457.78656005859	0.000295977280595198	\\
9480.66558837891	4.26882378315948e-06	\\
9496.142578125	0.000136052695008739	\\
9510.4419708252	3.12268396470695e-06	\\
9521.88148498535	0.000145986086759156	\\
9553.34014892578	0.000107361513263962	\\
9565.28434753418	2.3773501703944e-06	\\
9572.85461425781	1.48450694697579e-06	\\
9588.66806030273	0.000116797475116975	\\
9623.65951538086	1.72852665698381e-06	\\
9636.1083984375	0.000129981627441578	\\
9662.01553344727	0.000226046783031856	\\
9667.73529052734	4.03936370937007e-06	\\
9676.98783874512	1.97769136780574e-06	\\
9701.8856048584	0.000161099192752345	\\
9715.68031311035	0.000186403021476581	\\
9722.57766723633	3.95108807558822e-06	\\
9760.93368530273	0.000147598902123663	\\
9770.69091796875	4.07962493716759e-06	\\
9774.39193725586	4.18195792850229e-06	\\
9798.11210632324	0.000119917155697095	\\
9816.28074645996	1.88290391800475e-06	\\
9830.07545471191	0.000121238070186444	\\
9853.96385192871	0.000136276742568563	\\
9870.28198242188	4.69127449609066e-06	\\
9884.07669067383	4.09824828983886e-06	\\
9887.77770996094	0.000128826699109924	\\
9920.41397094727	3.91651361923645e-06	\\
9941.27426147461	0.000102450727591471	\\
9971.72355651855	0.00012752763306989	\\
9978.95736694336	4.82269084644245e-06	\\
9989.89219665527	2.70395910572955e-06	\\
10001.163482666	0.000183211706226942	\\
10025.7247924805	0.000168465051969536	\\
10044.9028015137	1.39916684474324e-06	\\
10059.3704223633	0.000118863340671514	\\
10072.3239898682	2.12911542194441e-06	\\
10088.4738922119	5.06730630622738e-06	\\
10103.950881958	0.000136670538091691	\\
10118.586730957	4.17835598733498e-06	\\
10137.5965118408	0.000118161152155857	\\
10159.4661712646	3.42128167895502e-06	\\
10165.0177001953	0.000130623019048068	\\
10199.6726989746	0.000135977825400299	\\
10207.9158782959	4.06590984094111e-06	\\
10237.0193481445	0.000133585476446805	\\
10253.5057067871	5.4959688471132e-06	\\
10255.0197601318	9.92995730492344e-05	\\
10260.7395172119	2.29044166503317e-06	\\
10302.2918701172	0.000184419892714202	\\
10318.7782287598	1.39106948172342e-06	\\
10323.4886169434	7.27104406214882e-06	\\
10331.9000244141	0.000263661282329536	\\
10374.7982025146	0.000148157998107113	\\
10377.9945373535	1.55958889778852e-06	\\
10401.5464782715	0.000166283403768979	\\
10414.8365020752	7.38746690631119e-06	\\
10444.9493408203	2.30438925980146e-06	\\
10453.0242919922	0.000118365557082102	\\
10462.1086120605	2.7905845892978e-06	\\
10469.0059661865	0.000116447047196955	\\
10503.1562805176	0.000126936736056465	\\
10526.5399932861	2.13779688255689e-06	\\
10540.1664733887	0.000109126983451043	\\
10551.26953125	1.78295529024958e-06	\\
10570.2793121338	0.000120966272007208	\\
10575.662612915	4.35585924814859e-06	\\
10599.214553833	0.000104484604518726	\\
10621.2524414063	2.06394620613808e-06	\\
10632.1872711182	5.10941506751533e-06	\\
10662.3001098633	0.000150135801539242	\\
10689.5530700684	1.85879295360106e-06	\\
10697.7962493896	0.000198404049705569	\\
10711.5909576416	0.000178924462810971	\\
10716.4695739746	2.1530125325661e-06	\\
10737.4980926514	0.000154948668737386	\\
10741.1991119385	7.30207768349064e-06	\\
10776.0223388672	0.000126181504713817	\\
10802.0977020264	2.85851954642752e-06	\\
10813.2007598877	0.000234970981847081	\\
10827.5001525879	4.609567352557e-06	\\
10843.8182830811	0.000142028141695231	\\
10849.0333557129	2.67263010250125e-06	\\
10880.6602478027	5.98274793955001e-06	\\
10903.539276123	0.000106473873355576	\\
10933.1474304199	1.08588048859899e-06	\\
10936.6802215576	0.000121552821821431	\\
10943.5775756836	0.000126238885732362	\\
10951.1478424072	2.50645036502586e-06	\\
10974.5315551758	9.48975454449463e-05	\\
10975.7091522217	2.76774046142672e-06	\\
11010.5323791504	0.000130974431651656	\\
11034.757232666	1.34494705070881e-07	\\
11046.7014312744	0.000144993864928297	\\
11074.7955322266	2.71605199658034e-06	\\
11091.4501190186	0.000139779292869755	\\
11109.2823028564	2.13892260213897e-06	\\
11118.030166626	0.000145903144280154	\\
11125.4322052002	6.34400429695181e-06	\\
11159.9189758301	3.03542930890853e-06	\\
11160.9283447266	0.000116972796324577	\\
11193.9010620117	0.000121487846823352	\\
11212.2379302979	1.18377962170291e-07	\\
11237.8086090088	7.67202089515134e-06	\\
11242.6872253418	0.000123983710199236	\\
11277.8469085693	2.87591601332484e-06	\\
11280.2021026611	0.000166306018845032	\\
11284.2395782471	8.04638317308946e-07	\\
11313.1748199463	0.000128498284082516	\\
11335.0444793701	3.30823197854161e-06	\\
11341.6053771973	0.000169149865297439	\\
11352.5402069092	0.000142937864010511	\\
11370.8770751953	3.1255584152116e-06	\\
11392.4102783203	2.86297444196193e-06	\\
11412.4294281006	0.000142283347669331	\\
11439.1777038574	0.000151481967442143	\\
11441.3646697998	1.39202029935859e-06	\\
11454.6546936035	0.000138480739841893	\\
11468.1129455566	1.42498793674678e-06	\\
11498.5622406006	9.28124165116016e-07	\\
11521.9459533691	9.29985265352326e-05	\\
11544.9932098389	0.000111204956711464	\\
11551.554107666	3.73537423753592e-06	\\
11559.6290588379	1.62567798267873e-06	\\
11575.9471893311	0.000119448905595563	\\
11602.5272369385	1.34226893031257e-06	\\
11617.8359985352	0.000114740268810099	\\
11633.4812164307	5.45339788188181e-07	\\
11641.3879394531	0.000176783500725708	\\
11661.7435455322	3.00427500961711e-06	\\
11684.2861175537	0.000153170970855983	\\
11694.8844909668	4.0255059741135e-06	\\
11713.0531311035	0.000148569013590293	\\
11738.1191253662	5.00816777639493e-06	\\
11754.1007995605	0.000180447023234936	\\
11789.4287109375	4.15575178258648e-06	\\
11794.6437835693	0.00016693408596908	\\
11800.8682250977	1.39797984485374e-06	\\
11821.3920593262	0.000176485477958722	\\
11831.8222045898	0.000108024939409201	\\
11854.8694610596	2.8883828001244e-06	\\
11874.3839263916	5.52189963258315e-06	\\
11887.1692657471	0.00011340511707676	\\
11919.3008422852	0.000120897273108574	\\
11924.0112304688	4.97763491231645e-06	\\
11943.6939239502	0.000114325942986383	\\
11967.582321167	4.02788598149188e-06	\\
11980.7041168213	0.00012492732664764	\\
11988.4426116943	1.7241721663076e-06	\\
12003.4149169922	8.8137378875653e-07	\\
12022.5929260254	0.000168379754064772	\\
12051.5281677246	1.33464564048058e-06	\\
12055.73387146	0.000116974867784506	\\
12089.7159576416	3.75963837017946e-06	\\
12094.762802124	0.000153637885578958	\\
12113.7725830078	2.82393834454751e-06	\\
12135.8104705811	0.000150735423865786	\\
12151.4556884766	1.60567811849924e-06	\\
12162.7269744873	0.000106981121975107	\\
12178.7086486816	1.43846027563559e-06	\\
12179.2133331299	0.000115417150297274	\\
12224.6349334717	4.74754964768136e-06	\\
12232.878112793	0.000110124193060248	\\
12261.9815826416	9.62102737929833e-05	\\
12265.3461456299	1.68082854818757e-06	\\
12278.2997131348	7.88823471447447e-07	\\
12306.3938140869	0.000105276143075611	\\
12330.1139831543	0.00011122632617557	\\
12338.1889343262	6.49670204126152e-06	\\
12353.4976959229	6.77124339857133e-06	\\
12377.5543212891	0.000141421182276276	\\
12383.4423065186	0.000142901559418971	\\
12391.6854858398	4.97516402741015e-06	\\
12418.2655334473	8.15105961525022e-07	\\
12427.3498535156	0.000109940969929202	\\
12450.3971099854	2.63990815581137e-06	\\
12476.6407012939	0.00013501845310373	\\
12503.3889770508	0.00014174389438733	\\
12511.4639282227	3.57901987797717e-06	\\
12517.0154571533	4.68068212795726e-06	\\
12540.9038543701	0.000116743196179707	\\
12571.8578338623	0.000167740659732743	\\
12579.2598724365	3.69344515881581e-06	\\
12588.0077362061	0.000115845002143586	\\
12604.8305511475	1.00625674086777e-06	\\
12636.1209869385	0.000130546674550473	\\
12648.5698699951	1.99857987328843e-06	\\
12672.2900390625	1.39469348175017e-06	\\
12681.3743591309	7.64811304699585e-05	\\
12691.2998199463	2.96258295689363e-06	\\
12701.0570526123	0.000125751675935617	\\
12725.1136779785	0.00013837789764417	\\
12741.0953521729	6.59718417309738e-06	\\
12763.1332397461	0.000156387945094336	\\
12783.9935302734	2.10379910362784e-06	\\
12791.0591125488	6.36586814698902e-06	\\
12798.6293792725	0.000130052143945726	\\
12835.1348876953	0.000146341159615044	\\
12839.6770477295	4.46107628942224e-06	\\
12877.1919250488	0.000147868368414002	\\
12890.9866333008	1.29681223161782e-06	\\
12904.9495697021	4.08243213375319e-06	\\
12905.1177978516	9.58324620604852e-05	\\
12928.6697387695	3.42370274127169e-07	\\
12948.0159759521	0.000118558192914317	\\
12969.2127227783	0.000104654085282999	\\
12983.1756591797	3.65184022812722e-06	\\
13000.6713867188	0.000129814492717495	\\
13019.849395752	2.44611938499202e-06	\\
13043.0648803711	0.000230272629306893	\\
13051.3080596924	6.56740841291592e-06	\\
13071.8318939209	0.000213618719642632	\\
13091.6828155518	2.84631849904773e-06	\\
13108.6738586426	0.000126656197763355	\\
13126.0013580322	4.78105316885696e-07	\\
13145.6840515137	2.73869544720909e-06	\\
13149.5532989502	0.000106093802260873	\\
13176.301574707	0.000104661994240345	\\
13180.6755065918	2.70503171936464e-06	\\
13208.4331512451	0.00010443932848791	\\
13231.3121795654	3.13497715579414e-06	\\
13244.6022033691	0.000159217693638307	\\
13267.8176879883	4.64013151160323e-06	\\
13281.6123962402	0.000146355050637704	\\
13291.0331726074	1.38075972119266e-06	\\
13321.9871520996	1.84190560494057e-06	\\
13328.8845062256	0.000116808411197185	\\
13349.4083404541	9.95712709133138e-05	\\
13355.9692382813	5.80039991362647e-06	\\
13378.8482666016	1.56325781278787e-06	\\
13407.2788238525	0.000122317677530788	\\
13413.5032653809	9.97437508947741e-05	\\
13422.5875854492	3.30478912324577e-06	\\
13448.1582641602	0.000137447738980715	\\
13467.6727294922	6.24238321821764e-07	\\
13508.3839416504	0.000123238193118553	\\
13509.056854248	5.36572685991875e-06	\\
13528.7395477295	4.1334537197462e-06	\\
13536.9827270508	0.000107738211051058	\\
13546.9081878662	7.88119233431581e-06	\\
13567.7684783936	0.000131279763337311	\\
13602.2552490234	4.07904793271191e-06	\\
13604.6104431152	0.000104405221763054	\\
13616.8910980225	0.000157346045825829	\\
13645.4898834229	5.89202724541812e-07	\\
13658.948135376	2.22684118698831e-06	\\
13677.2850036621	7.54575777986952e-05	\\
13691.4161682129	1.09138467116338e-06	\\
13710.9306335449	0.000116941519775705	\\
13723.0430603027	1.32860087374067e-05	\\
13725.9029388428	0.000155743345573499	\\
13767.2870635986	0.000145916743717322	\\
13779.231262207	5.11699998234643e-06	\\
13790.1660919189	0.000115034385279676	\\
13796.3905334473	1.613729468152e-06	\\
13823.8117218018	0.000124836756026327	\\
13847.7001190186	1.64151787168881e-06	\\
13865.1958465576	0.000102992942105428	\\
13875.2895355225	3.28665184729819e-06	\\
13911.1221313477	8.98250299526183e-05	\\
13915.6642913818	5.52294583829601e-06	\\
13936.861038208	9.28765395738033e-05	\\
13949.9828338623	1.09497899761954e-06	\\
13962.4317169189	1.36680528526661e-06	\\
13987.3294830322	7.19693671782054e-05	\\
13994.8997497559	1.55040137044626e-06	\\
14025.5172729492	0.000111057925338424	\\
14041.1624908447	6.02616857043018e-06	\\
14046.3775634766	0.000155480858276327	\\
14061.8545532227	3.463215669793e-06	\\
14078.8455963135	9.36457874524817e-05	\\
14103.5751342773	0.000106796923747377	\\
14119.0521240234	5.61451363600336e-06	\\
14138.3983612061	1.34176831707045e-06	\\
14153.7071228027	0.000105878617722416	\\
14174.3991851807	3.09455337770224e-06	\\
14187.0162963867	0.00016736534544739	\\
14198.7922668457	0.000119123843655892	\\
14218.3067321777	4.68399456471944e-06	\\
14241.3539886475	5.65116003242862e-06	\\
14258.0085754395	0.000114117707919588	\\
14271.635055542	3.93811625816121e-06	\\
14291.6542053223	0.000109238385336832	\\
14302.2525787354	1.62310929000185e-06	\\
14315.8790588379	8.43646402997756e-05	\\
14348.851776123	2.34649855244305e-06	\\
14358.9454650879	0.00010142032469098	\\
14370.2167510986	8.67517813786419e-07	\\
14380.9833526611	0.000102590273247181	\\
14421.8627929688	8.86594822006654e-05	\\
14428.2554626465	1.04581192532011e-06	\\
14463.9198303223	1.6282672290291e-06	\\
14468.9666748047	0.000109968628255221	\\
14488.6493682861	0.000114608756482938	\\
14499.2477416992	3.10311742939112e-06	\\
14520.1080322266	1.0653683897742e-06	\\
14523.3043670654	9.83810569210878e-05	\\
14559.9781036377	7.50080514822985e-05	\\
14563.8473510742	3.45968639552803e-06	\\
14600.3528594971	8.61465777614842e-05	\\
14604.7267913818	2.9308182635536e-07	\\
14624.9141693115	4.0324394756809e-06	\\
14631.9797515869	0.000105301310765732	\\
14655.6999206543	4.06115884920212e-06	\\
14667.9805755615	8.47342511507929e-05	\\
14681.4388275146	4.10265570364361e-06	\\
14711.0469818115	6.6740589862933e-05	\\
14719.9630737305	2.12178599033904e-05	\\
14732.7484130859	0.000148717000926879	\\
14761.1789703369	9.15809164863569e-07	\\
14769.0856933594	9.70531653816293e-05	\\
14795.6657409668	0.0001177423497769	\\
14812.6567840576	1.28916175945096e-05	\\
14833.0123901367	2.43202066404147e-06	\\
14836.5451812744	5.86133900571506e-05	\\
14848.6576080322	4.34200703243281e-06	\\
14874.564743042	0.000139621715715235	\\
14891.0511016846	1.2503979444248e-06	\\
14915.7806396484	0.000121088956611396	\\
14943.2018280029	0.000126831413021986	\\
14949.7627258301	3.84628279394173e-06	\\
14958.3423614502	0.000102284477198021	\\
14967.0902252197	5.09672230110346e-06	\\
14997.5395202637	2.67839480747673e-06	\\
15014.3623352051	8.78953482164524e-05	\\
15025.9700775146	7.0836350010358e-05	\\
15034.2132568359	2.41127026490983e-06	\\
15056.9240570068	8.60847730587713e-05	\\
15066.0083770752	4.52514982922039e-06	\\
15109.411239624	6.96190255960518e-05	\\
15119.5049285889	4.07930054594026e-06	\\
15138.0100250244	6.9119912118608e-05	\\
15145.7485198975	2.99918649340193e-06	\\
15179.3941497803	2.37225958451348e-06	\\
15185.4503631592	7.97789893609946e-05	\\
15196.5534210205	2.0038229605301e-06	\\
15201.2638092041	7.71530887372488e-05	\\
15232.0495605469	3.26284689763491e-06	\\
15251.9004821777	9.62212636370231e-05	\\
15271.0784912109	7.1622352885095e-05	\\
15278.4805297852	1.71309794545972e-06	\\
15306.0699462891	8.24227594857446e-05	\\
15324.2385864258	2.19917053877833e-06	\\
15334.3322753906	1.46626873603374e-06	\\
15352.1644592285	7.2600781702984e-05	\\
15363.9404296875	8.46253805685523e-05	\\
15380.7632446289	2.56196680455333e-06	\\
15401.2870788574	9.87068193613653e-05	\\
15415.5864715576	2.82582083703031e-06	\\
15446.7086791992	0.000139388173076241	\\
15460.3351593018	2.53936631460109e-06	\\
15470.4288482666	9.14567814463971e-05	\\
15491.7938232422	4.44456254248445e-06	\\
15502.2239685059	9.27611207532851e-05	\\
15526.6170501709	2.86049114013247e-06	\\
15539.4023895264	1.29848006676412e-06	\\
15558.5803985596	9.00393573897022e-05	\\
15583.4781646729	5.77170576779362e-05	\\
15602.6561737061	3.59309166887591e-06	\\
15627.0492553711	1.68346371386899e-06	\\
15637.1429443359	5.92249774144666e-05	\\
15648.4142303467	2.33637448100185e-06	\\
15665.2370452881	7.57475423866847e-05	\\
15690.1348114014	7.14437769629188e-05	\\
15698.3779907227	1.16083769849797e-05	\\
15709.6492767334	3.5507494299574e-06	\\
15734.5470428467	7.5675084630278e-05	\\
15750.8651733398	5.6426410799429e-05	\\
15757.9307556152	2.68570917573333e-06	\\
15782.8285217285	0.000101860988742951	\\
15798.6419677734	4.18881409007718e-06	\\
15813.2778167725	4.95958531042097e-05	\\
15822.1939086914	5.36574118426725e-07	\\
15849.1104125977	5.85961022464405e-05	\\
15871.8212127686	1.76115063545528e-06	\\
15886.9617462158	4.50379311914251e-06	\\
15894.0273284912	6.46016893554383e-05	\\
15914.3829345703	7.45878906314808e-05	\\
15939.1124725342	5.58190556392341e-06	\\
15949.7108459473	6.54603168745306e-05	\\
15960.8139038086	8.48668993618494e-06	\\
15991.431427002	1.40094192193746e-06	\\
16000.6839752197	4.2721001463551e-05	\\
16021.7124938965	1.39583806763206e-05	\\
16047.7878570557	3.80999874259205e-05	\\
16049.133682251	3.7237350225974e-05	\\
16082.6110839844	2.38801032295939e-05	\\
16083.283996582	2.62749953010864e-05	\\
16114.0697479248	2.43167268551964e-07	\\
16118.611907959	5.42294617931694e-07	\\
16151.2481689453	2.77320746411577e-05	\\
16151.921081543	2.50956298061365e-05	\\
16173.2860565186	4.03273089797137e-05	\\
16186.071395874	3.49367054448361e-05	\\
16220.3899383545	4.45428895001593e-06	\\
16220.8946228027	7.11083348459183e-06	\\
16244.1101074219	1.57896292575999e-07	\\
16274.3911743164	3.29105201542596e-07	\\
16288.6905670166	1.08016962583465e-05	\\
16289.1952514648	8.27390467290141e-06	\\
16322.3361968994	3.81284872345201e-05	\\
16337.8131866455	7.80566733059889e-06	\\
16354.6360015869	5.19763635915309e-05	\\
16357.495880127	4.97533501140843e-05	\\
16391.4779663086	2.16687475713392e-05	\\
16391.9826507568	2.41248788618159e-05	\\
16419.7402954102	2.5771794671793e-07	\\
16433.8714599609	7.92688116095568e-06	\\
16443.9651489258	6.96212416355871e-07	\\
16465.4983520508	5.52783583827453e-07	\\
16494.0971374512	4.24513931594009e-05	\\
16516.6397094727	6.43887886654752e-05	\\
16528.2474517822	1.10940055496171e-06	\\
16529.0885925293	7.92537932570332e-06	\\
16547.4254608154	5.79975765976418e-05	\\
16564.9211883545	5.87996096508311e-05	\\
16588.304901123	4.34896823830847e-05	\\
16598.3985900879	4.92871697641046e-05	\\
16631.2030792236	1.41111783259183e-05	\\
16631.8759918213	1.69415168370504e-05	\\
16648.8670349121	3.74424222481955e-07	\\
16670.0637817383	4.82358710829236e-07	\\
16693.9521789551	5.65243503428907e-05	\\
16710.1020812988	4.67962719918515e-06	\\
16726.2519836426	5.78836775776678e-05	\\
16746.1029052734	1.40899151502601e-05	\\
16768.1407928467	5.68198653369037e-05	\\
16773.1876373291	6.21432817116935e-05	\\
16802.795791626	3.1546794397656e-05	\\
16809.0202331543	2.50488690834265e-05	\\
16826.5159606934	5.89353542880568e-05	\\
16858.6475372314	3.70415780120962e-06	\\
16871.9375610352	0.00011412189423775	\\
16876.4797210693	0.000127808999510142	\\
16892.4613952637	1.15079959409961e-05	\\
16913.321685791	1.15685989888466e-05	\\
16925.9387969971	6.09633177694546e-05	\\
16940.5746459961	2.69178869277952e-05	\\
16951.509475708	4.60357084025804e-05	\\
16981.2858581543	2.2469775195064e-05	\\
17008.7070465088	5.38519554523834e-05	\\
17011.23046875	5.4684197790133e-05	\\
17042.6891326904	2.0859465539003e-05	\\
17055.8109283447	2.12917557251229e-06	\\
17076.8394470215	2.60720865718675e-05	\\
17094.1669464111	3.38600953590548e-05	\\
17111.157989502	2.31553376798597e-05	\\
17112.8402709961	2.49475564317119e-05	\\
17144.6353912354	6.68946287036642e-07	\\
17154.0561676025	1.55100100047491e-07	\\
17160.9535217285	4.36670635110704e-06	\\
17200.150680542	2.47219568620742e-06	\\
17202.0011901855	6.24825719740191e-07	\\
17237.4973297119	4.15802999675841e-07	\\
17239.3478393555	2.77610808433859e-06	\\
17275.3486633301	3.10985887582198e-07	\\
17276.0215759277	2.86322860324851e-06	\\
17308.3213806152	2.5677856195408e-07	\\
17315.2187347412	4.25507222689568e-06	\\
17322.2843170166	3.76067268520526e-08	\\
17351.8924713135	7.90867963582815e-06	\\
17360.4721069336	1.25954403703847e-06	\\
17383.0146789551	5.686294839183e-05	\\
17399.1645812988	1.80932329649662e-06	\\
17415.3144836426	5.59349164470291e-05	\\
17420.5295562744	5.04262550486564e-05	\\
17435.1654052734	1.17653009859135e-05	\\
17457.2032928467	2.71458132807791e-05	\\
17475.8766174316	3.7992684252309e-05	\\
17490.0077819824	3.36022883625063e-05	\\
17508.8493347168	1.36852344951187e-05	\\
17524.3263244629	2.43194939294115e-05	\\
17556.6261291504	3.87231330876825e-05	\\
17557.9719543457	3.84741155587304e-05	\\
17591.4493560791	9.46503506188558e-06	\\
17592.1222686768	1.20895361242917e-05	\\
17613.6554718018	3.45341270829566e-07	\\
17627.1137237549	5.04819485656066e-06	\\
17637.8803253174	2.40365014559721e-07	\\
17674.7222900391	9.1092085836517e-08	\\
17694.0685272217	5.739568260026e-06	\\
17695.9190368652	4.02859153200499e-06	\\
17728.8917541504	3.24736935609758e-05	\\
17745.5463409424	3.85839109492959e-05	\\
17763.0420684814	2.7878236034273e-05	\\
17763.5467529297	3.04612536800715e-05	\\
17797.6970672607	6.91714016900748e-07	\\
17813.1740570068	4.6782479467852e-06	\\
17821.4172363281	5.2177224175536e-08	\\
17832.35206604	9.62171149494404e-07	\\
17865.661239624	2.03563250422169e-05	\\
17866.3341522217	1.78924774342771e-05	\\
17897.2881317139	4.11649385069475e-05	\\
17901.1573791504	4.08767785763995e-05	\\
17934.4665527344	3.17072228154515e-06	\\
17935.6441497803	3.39885657168163e-06	\\
17968.9533233643	4.04889345530803e-05	\\
17971.4767456055	4.08214073790641e-05	\\
18003.6083221436	1.8166662663303e-05	\\
18004.4494628906	1.98772889904343e-05	\\
18028.0014038086	1.69947938814699e-07	\\
18038.9362335205	2.25279405256501e-06	\\
18072.0771789551	3.18194230061927e-05	\\
18072.5818634033	2.92678561172373e-05	\\
18105.7228088379	4.30572510500855e-05	\\
18108.2462310791	4.3157789297327e-05	\\
18140.0413513184	3.12569864323257e-05	\\
18140.714263916	3.37678881965487e-05	\\
18175.0328063965	4.15142135860679e-06	\\
18185.9676361084	9.97186167840881e-08	\\
18208.6784362793	1.97777556780449e-05	\\
18209.351348877	1.73815990519962e-05	\\
18242.1558380127	3.76873877570245e-05	\\
18244.6792602539	3.7325220274742e-05	\\
18277.6519775391	1.24154738536957e-05	\\
18278.1566619873	1.48261094993215e-05	\\
18304.0637969971	3.00233919228992e-07	\\
18314.998626709	4.29271784237808e-06	\\
18325.9334564209	1.86981817329945e-07	\\
18353.0181884766	2.61475744895969e-06	\\
18353.6911010742	5.15882080182538e-07	\\
18409.3746185303	2.64361474280542e-06	\\
18410.0475311279	4.49773532190402e-07	\\
18415.2626037598	4.71137785873375e-07	\\
18415.9355163574	2.55196889638686e-06	\\
18459.6748352051	3.70985910488452e-07	\\
18466.7404174805	2.68479130283329e-06	\\
18506.4422607422	2.33243365677286e-07	\\
18517.3770904541	4.24502115104593e-06	\\
18525.1155853271	4.47381654760328e-06	\\
18541.2654876709	1.04047774777809e-07	\\
18558.5929870605	1.19813672703249e-07	\\
18577.4345397949	4.49245616636359e-06	\\
18608.5567474365	6.90282528999811e-08	\\
18620.8374023438	1.06727831123229e-05	\\
18621.5103149414	8.24749188215943e-06	\\
18654.3148040771	3.71421156319555e-05	\\
18662.0532989502	3.85663836588957e-05	\\
18688.4651184082	2.11141707057014e-05	\\
18690.4838562012	2.22827300526549e-05	\\
18715.7180786133	3.19092409002136e-07	\\
18724.129486084	7.47223528576879e-07	\\
18757.1022033691	2.98073230558932e-05	\\
18757.9433441162	2.78480533694433e-05	\\
18776.4484405518	3.88371482309431e-05	\\
18793.2712554932	3.24313303283503e-05	\\
18826.0757446289	2.84941295976163e-06	\\
18826.7486572266	5.48869045719988e-06	\\
18851.1417388916	1.65537413668703e-07	\\
18880.4134368896	2.10208937594243e-07	\\
18895.0492858887	4.98184355508404e-06	\\
18914.0590667725	2.97480540258974e-07	\\
18928.6949157715	1.20175253838451e-05	\\
18930.545425415	1.05802937605456e-05	\\
18962.8452301025	3.72426073917195e-05	\\
18967.892074585	3.7879163643508e-05	\\
18997.6684570313	1.87032841110144e-05	\\
18998.3413696289	2.10921192588895e-05	\\
19028.9588928223	2.38128516464724e-07	\\
19032.8281402588	2.445326506615e-06	\\
19066.4737701416	3.21285611371778e-05	\\
19082.9601287842	3.81000394317892e-05	\\
19100.1194000244	2.89843789696516e-05	\\
19101.969909668	3.06949856373187e-05	\\
19134.2697143555	1.3728976510024e-06	\\
19136.7931365967	1.98925499261027e-07	\\
19157.1487426758	4.38683891533017e-06	\\
19169.7658538818	3.17947171032464e-06	\\
19189.9532318115	3.06204178797627e-07	\\
19231.5055847168	1.23035717178056e-07	\\
19237.2253417969	3.43625580266873e-06	\\
19247.82371521	2.75583380600053e-07	\\
19271.5438842773	1.02620836445242e-05	\\
19273.3943939209	8.44590560378682e-06	\\
19306.5353393555	3.16983976549355e-05	\\
19312.7597808838	3.25330237958285e-05	\\
19340.8538818359	1.67698400554906e-05	\\
19341.3585662842	1.93441734222128e-05	\\
19365.7516479492	2.86820618565217e-07	\\
19383.2473754883	3.45106990444726e-06	\\
19393.8457489014	1.68999409831489e-07	\\
19427.4913787842	2.73002116236514e-06	\\
19428.1642913818	2.41796765678355e-07	\\
19456.4266204834	1.1011982167659e-06	\\
19477.6233673096	1.94320779381379e-06	\\
19511.100769043	2.18306604807944e-06	\\
19511.7736816406	8.42793987294732e-07	\\
19513.6241912842	2.20624077935212e-06	\\
19514.2971038818	8.18516643828381e-07	\\
19549.1203308105	1.9729253563903e-06	\\
19550.9708404541	1.09886940876353e-06	\\
19611.3647460938	1.66384037532128e-06	\\
19614.5610809326	1.33344757146483e-06	\\
19617.7574157715	1.69311299634792e-06	\\
19620.7855224609	1.31154463156103e-06	\\
19650.2254486084	1.3950795387898e-06	\\
19655.9452056885	1.63217166752626e-06	\\
19692.4507141113	1.62981968413605e-06	\\
19694.6376800537	1.36954772422784e-06	\\
19723.0682373047	1.64009915614597e-06	\\
19729.9655914307	1.3946463534185e-06	\\
19776.2283325195	1.58915711113631e-06	\\
19778.0788421631	1.40453432367986e-06	\\
19804.3224334717	1.58470574265839e-06	\\
19804.8271179199	1.42053350868722e-06	\\
19824.8462677002	1.42260594261517e-06	\\
19849.7440338135	1.56546933682618e-06	\\
19874.1371154785	1.57864462133212e-06	\\
19877.3334503174	1.4396520979013e-06	\\
19892.8104400635	1.55702454527469e-06	\\
19908.455657959	1.39096776282948e-06	\\
19934.6992492676	1.57176150592167e-06	\\
19944.624710083	1.41695265195979e-06	\\
19978.7750244141	1.57191005857535e-06	\\
19986.6817474365	1.42565379724306e-06	\\
20002.326965332	1.59064576988119e-06	\\
20021.000289917	1.41544381810132e-06	\\
20029.7481536865	1.41407046010187e-06	\\
20044.3840026855	1.592903679345e-06	\\
20073.9921569824	1.41214179990769e-06	\\
20095.357131958	1.58233817669647e-06	\\
20101.2451171875	1.41371869367413e-06	\\
20105.7872772217	1.59382619964629e-06	\\
20152.7229309082	1.40433037241316e-06	\\
20161.975479126	1.57918331837226e-06	\\
20164.3306732178	1.57003720357004e-06	\\
20186.5367889404	1.42157997473501e-06	\\
20206.3877105713	1.41912987315441e-06	\\
20231.9583892822	1.57834599229426e-06	\\
20239.192199707	1.41785293831089e-06	\\
20250.4634857178	1.56725082061314e-06	\\
20267.1180725098	1.4184306609461e-06	\\
20290.1653289795	1.57565187557602e-06	\\
20302.6142120361	1.42980590848158e-06	\\
20328.8578033447	1.56835413483297e-06	\\
20361.8305206299	1.56750853950004e-06	\\
20361.9987487793	1.39534550278294e-06	\\
20377.8121948242	1.41491465022364e-06	\\
20399.0089416504	1.58224584099886e-06	\\
20415.8317565918	1.58315053248285e-06	\\
20427.6077270508	1.41883695936457e-06	\\
20439.2154693604	1.42703960511488e-06	\\
20441.2342071533	1.56715615341882e-06	\\
20499.4411468506	1.41660167606617e-06	\\
20505.8338165283	1.58045412842968e-06	\\
20523.3295440674	1.40605261527821e-06	\\
20532.7503204346	1.58649418624988e-06	\\
20545.3674316406	1.41919138315749e-06	\\
20568.0782318115	1.56469826859731e-06	\\
20587.0880126953	1.4127527070025e-06	\\
20598.3592987061	1.57851524285664e-06	\\
20621.7430114746	1.38781915718153e-06	\\
20631.5002441406	1.57670895941456e-06	\\
20673.7255096436	1.41984631604602e-06	\\
20674.0619659424	1.56275364538671e-06	\\
20682.6416015625	1.56241535755222e-06	\\
20709.5581054688	1.41943203109351e-06	\\
20727.3902893066	1.54664979009663e-06	\\
20736.3063812256	1.42227474555868e-06	\\
20772.4754333496	1.39175357565527e-06	\\
20774.9988555908	1.56016333789757e-06	\\
20783.7467193604	1.56214566212214e-06	\\
20812.1772766113	1.41063963026336e-06	\\
20818.9064025879	1.4031323100507e-06	\\
20844.8135375977	1.58742007954811e-06	\\
20863.3186340332	1.56552900567302e-06	\\
20871.8982696533	1.42070050352467e-06	\\
20885.188293457	1.40043133865341e-06	\\
20898.983001709	1.57733823120529e-06	\\
20941.3764953613	1.39765209676105e-06	\\
20951.3019561768	1.55768710622133e-06	\\
20968.9659118652	1.57855657244164e-06	\\
20973.6763000488	1.39303537611537e-06	\\
20995.7141876221	1.41363345224225e-06	\\
21003.2844543457	1.55878147523298e-06	\\
21043.8274383545	1.42285827899973e-06	\\
21046.6873168945	1.5675333397041e-06	\\
21060.8184814453	1.40535825996163e-06	\\
21072.9309082031	1.5774214228625e-06	\\
21120.5394744873	1.41083817641148e-06	\\
21120.7077026367	1.54813585980447e-06	\\
21137.0258331299	1.41016221022307e-06	\\
21141.7362213135	1.56712585516648e-06	\\
21160.4095458984	1.54565731412293e-06	\\
21191.53175354	1.40620720113215e-06	\\
21198.9337921143	1.398940157518e-06	\\
21220.4669952393	1.55342379945432e-06	\\
21240.8226013184	1.57358979674372e-06	\\
21255.7949066162	1.41857382149708e-06	\\
21262.6922607422	1.55940708639566e-06	\\
21270.5989837646	1.39677688362194e-06	\\
21317.7028656006	1.56418330913005e-06	\\
21321.7403411865	1.38865492023743e-06	\\
21336.544418335	1.39700168446726e-06	\\
21353.5354614258	1.57274866511436e-06	\\
21372.713470459	1.40562200541494e-06	\\
21388.1904602051	1.54835605131682e-06	\\
21403.6674499512	1.56787600433771e-06	\\
21407.2002410889	1.41219461554963e-06	\\
21433.7802886963	1.38806704269031e-06	\\
21460.3603363037	1.56124912046317e-06	\\
21485.0898742676	1.55801278439629e-06	\\
21486.9403839111	1.42340442896055e-06	\\
21511.3334655762	1.41008042077585e-06	\\
21518.2308197021	1.55831432578637e-06	\\
21555.4092407227	1.54467439363423e-06	\\
21563.3159637451	1.38568816255771e-06	\\
21576.9424438477	1.55820476733514e-06	\\
21598.6438751221	1.40773588622738e-06	\\
21612.1021270752	1.55774962037414e-06	\\
21622.8687286377	1.3989114091488e-06	\\
21639.3550872803	1.39742873275729e-06	\\
21655.504989624	1.57111316847413e-06	\\
21684.7766876221	1.40190250087889e-06	\\
21692.8516387939	1.55250159240259e-06	\\
21708.6650848389	1.39581866414007e-06	\\
21720.7775115967	1.559832137517e-06	\\
21742.1424865723	1.5663439681618e-06	\\
21759.1335296631	1.40961136050104e-06	\\
21792.7791595459	1.56953365430866e-06	\\
21801.358795166	1.41006492921481e-06	\\
21828.1070709229	1.55279269700586e-06	\\
21832.3127746582	1.40368445242879e-06	\\
21872.0146179199	1.41091362376942e-06	\\
21875.0427246094	1.54416940695793e-06	\\
21892.2019958496	1.57800235818748e-06	\\
21894.5571899414	1.40368062924209e-06	\\
21921.8101501465	1.41391429661439e-06	\\
21938.9694213867	1.55782811619502e-06	\\
21948.3901977539	1.56066704080121e-06	\\
21978.3348083496	1.38111987033486e-06	\\
21994.9893951416	1.41379948481447e-06	\\
21997.8492736816	1.56132309490049e-06	\\
22050	1.49118760917462e-06	\\
};
\addlegendentry{X};

\end{axis}

\begin{axis}[%
width=\figurewidth,
height=\figureheight,
scale only axis,
xmin=0,
xmax=22050,
xlabel={Frequency (Hz)},
ymin=0,
ymax=0.0178725135586674,
ylabel={|Y(f)|},
at=(plot7.right of south east),
anchor=left of south west,
title={Single-Sided Amplitude Spectrum of y(t)},
legend style={draw=black,fill=white,legend cell align=left}
]
\addplot [color=blue,solid]
  table[row sep=crcr]{
0	0.000145912475813438	\\
23.0472564697266	7.41901475717461e-06	\\
33.1409454345703	0.00700799592302918	\\
49.7955322265625	0.00776607882639981	\\
61.2350463867188	9.46748569133109e-05	\\
72.1698760986328	0.00720898895717755	\\
77.8896331787109	7.87185825750723e-05	\\
129.703903198242	0.0049497201455911	\\
131.890869140625	3.30342413196869e-05	\\
143.498611450195	0.000154477481201072	\\
145.853805541992	0.00916659039844648	\\
199.350357055664	7.86275570861302e-05	\\
204.733657836914	0.00285222872477367	\\
226.93977355957	0.00709510969693795	\\
235.855865478516	0.000105139188689271	\\
245.949554443359	0.00381665824737881	\\
249.818801879883	0.000101444382580025	\\
291.034698486328	4.16265104669416e-05	\\
298.436737060547	0.00337919724135175	\\
325.185012817383	5.14855458508052e-05	\\
332.92350769043	0.00266984131803347	\\
350.75569152832	0.000107622875221906	\\
369.429016113281	0.00342656214407951	\\
385.242462158203	3.06084348637448e-05	\\
410.813140869141	0.001586819525384	\\
420.065689086914	6.15334193060193e-05	\\
445.468139648438	0.00661645245776536	\\
445.804595947266	0.0052313451231551	\\
462.627410888672	3.11322778181385e-05	\\
496.10481262207	2.16264113493643e-05	\\
510.57243347168	0.0010463995522296	\\
514.946365356445	0.00161083008631501	\\
541.694641113281	2.25330759521757e-05	\\
558.180999755859	4.77585322195078e-05	\\
580.050659179688	0.00215842428186239	\\
583.919906616211	0.00219937599827315	\\
602.593231201172	1.6484197714834e-05	\\
622.948837280273	4.61406625534286e-05	\\
642.967987060547	0.00173597782724728	\\
655.92155456543	5.47781928668006e-05	\\
669.043350219727	0.00104768944541753	\\
703.530120849609	0.00142981015600628	\\
720.184707641602	6.64435571491026e-06	\\
720.184707641602	6.64435571491026e-06	\\
745.923614501953	0.000590032341475398	\\
758.540725708008	0.000537505668917364	\\
762.914657592773	1.00007602348519e-05	\\
801.775360107422	9.1954067928971e-06	\\
808.672714233398	0.000542704309037582	\\
838.449096679688	0.000667891834973798	\\
846.01936340332	1.07076350562304e-05	\\
858.972930908203	3.7249296130346e-05	\\
874.618148803711	0.0028528069021667	\\
892.282104492188	0.000811824246417981	\\
908.600234985352	2.90681360985248e-05	\\
939.049530029297	7.22663994865166e-06	\\
946.956253051758	0.000590881200078847	\\
984.639358520508	0.000519960875495376	\\
989.181518554688	7.82906146537342e-06	\\
997.256469726563	1.81772254568387e-05	\\
1002.97622680664	0.000832957825334496	\\
1052.60353088379	1.44123862663253e-05	\\
1056.97746276855	0.000897207299276942	\\
1071.78153991699	4.66083181191237e-06	\\
1082.04345703125	0.000704399115324653	\\
1100.38032531738	6.11490043723363e-06	\\
1112.99743652344	0.000416697361657949	\\
1142.60559082031	4.28600779824416e-07	\\
1154.04510498047	0.000929412653076868	\\
1170.0267791748	2.9333962574224e-05	\\
1177.42881774902	0.00176919919583209	\\
1209.3921661377	0.000695619695279023	\\
1226.55143737793	7.86283068345028e-06	\\
1241.35551452637	0.00064602086378613	\\
1263.89808654785	4.06717930158637e-06	\\
1272.8141784668	1.07201456870152e-05	\\
1301.5811920166	0.000924968629410077	\\
1317.73109436035	0.00297020237013396	\\
1322.77793884277	3.64341905711054e-05	\\
1343.13354492188	0.000944437812646105	\\
1357.09648132324	3.44046092775044e-05	\\
1375.93803405762	0.000569781933780601	\\
1386.36817932129	4.51695104460281e-06	\\
1422.20077514648	0.000405350268009496	\\
1427.58407592773	1.40031084287783e-05	\\
1457.69691467285	0.000422148756311903	\\
1468.63174438477	6.08397296229829e-06	\\
1495.04356384277	0.00042505196826938	\\
1505.64193725586	5.65693889929572e-06	\\
1522.96943664551	0.000447961206800367	\\
1531.71730041504	6.1281656451356e-06	\\
1553.58695983887	0.000473140788299295	\\
1573.1014251709	3.1280855491212e-06	\\
1589.9242401123	6.50710493582854e-06	\\
1607.08351135254	0.000453139843848772	\\
1613.98086547852	2.13475498798351e-05	\\
1620.03707885742	0.000724721235867309	\\
1662.43057250977	0.000748129633680444	\\
1672.3560333252	3.28846922690687e-06	\\
1689.34707641602	8.51027315643791e-06	\\
1690.3564453125	0.000600659247635459	\\
1717.94586181641	2.96532067080013e-05	\\
1748.73161315918	0.00148528881392953	\\
1756.80656433105	0.00168909881672707	\\
1766.56379699707	2.62911961384485e-05	\\
1789.6110534668	3.06613954744026e-05	\\
1808.45260620117	0.000876316739736448	\\
1825.44364929199	0.000420586166996933	\\
1843.78051757813	1.18979918700234e-05	\\
1872.88398742676	5.06282349333829e-06	\\
1881.63185119629	0.000372531051116495	\\
1888.69743347168	0.000334414502541979	\\
1914.0998840332	1.58915279522477e-06	\\
1941.18461608887	1.21521819286218e-05	\\
1952.62413024902	0.000455031621407513	\\
1961.8766784668	0.00051564050043036	\\
1967.59643554688	1.25918702913935e-05	\\
1997.03636169434	2.43357393969924e-05	\\
2004.10194396973	0.000731985741951364	\\
2023.78463745117	0.000737232174516352	\\
2054.90684509277	4.21168945269928e-06	\\
2061.13128662109	0.000649907466471288	\\
2068.02864074707	2.47088688963174e-05	\\
2093.26286315918	1.05723556144575e-05	\\
2101.33781433105	0.00044286974590892	\\
2133.63761901855	0.000553320086872307	\\
2138.51623535156	1.70232963250284e-05	\\
2181.58264160156	1.79698000040559e-05	\\
2184.61074829102	0.000728502759217385	\\
2200.25596618652	0.000807108413227543	\\
2216.06941223145	3.94424997776247e-06	\\
2233.56513977051	2.30843350669419e-05	\\
2263.00506591797	0.00102042888273795	\\
2264.01443481445	0.00112090293791677	\\
2293.79081726074	1.82262358413861e-05	\\
2324.07188415527	5.78518252953505e-05	\\
2328.61404418945	0.00139977565242331	\\
2343.41812133789	2.45468805696581e-05	\\
2363.43727111816	0.00198963324824207	\\
2370.67108154297	0.0017797178133981	\\
2390.01731872559	2.95205298831024e-05	\\
2402.63442993164	0.00074807363089549	\\
2417.94319152832	5.79094048221921e-06	\\
2440.14930725098	0.000671781941663794	\\
2451.42059326172	1.77284515431777e-05	\\
2469.25277709961	0.000800236350132901	\\
2484.7297668457	7.32668373044126e-06	\\
2516.02020263672	1.39798138281417e-06	\\
2527.29148864746	0.000366501030784863	\\
2540.41328430176	0.000400571679407067	\\
2550.67520141602	3.26539656870366e-06	\\
2595.25566101074	1.33091861326102e-05	\\
2597.27439880371	0.000514010695720907	\\
2631.76116943359	2.7532146269854e-05	\\
2634.62104797363	0.00101637582031653	\\
2643.36891174316	0.00125879975181613	\\
2673.6499786377	3.5707744130515e-05	\\
2676.00517272949	0.000723700795334382	\\
2690.13633728027	1.56720818838045e-05	\\
2735.38970947266	1.00048043429375e-05	\\
2740.60478210449	0.000806876880948564	\\
2767.35305786133	0.000681515879608709	\\
2774.75509643555	7.9909691455367e-06	\\
2787.20397949219	0.000541928516355636	\\
2792.0825958252	5.27716112998436e-06	\\
2826.90582275391	8.40908097216436e-06	\\
2832.28912353516	0.000290917429224284	\\
2852.30827331543	1.34071202613765e-05	\\
2873.33679199219	0.00053765242561152	\\
2888.30909729004	1.55282469648933e-05	\\
2910.34698486328	0.000648751137840595	\\
2929.18853759766	1.14272918654591e-05	\\
2946.34780883789	0.0015263589901621	\\
2955.60035705566	0.000891102034732108	\\
2980.66635131836	2.27086567053352e-05	\\
2992.61054992676	0.000554968895737065	\\
2995.80688476563	3.58550889486944e-06	\\
3029.95719909668	1.88417522668787e-05	\\
3045.26596069336	0.000586975055127671	\\
3067.80853271484	3.13300507447319e-05	\\
3078.0704498291	0.000642482500008655	\\
3109.86557006836	0.000913962960135699	\\
3116.59469604492	1.32981906137589e-05	\\
3135.26802062988	0.000307849923141126	\\
3142.6700592041	7.94836284854829e-06	\\
3155.7918548584	9.67010682404268e-06	\\
3172.95112609863	0.000408372252879376	\\
3189.2692565918	0.000325222207455281	\\
3195.49369812012	6.0582536184703e-06	\\
3240.24238586426	1.69588904926914e-05	\\
3254.37355041504	0.000375745029993319	\\
3261.77558898926	4.47900389808294e-06	\\
3269.85054016113	0.000475302878743753	\\
3295.58944702148	2.29094352515244e-05	\\
3311.73934936523	0.000730333462543214	\\
3328.39393615723	9.98407692281653e-06	\\
3355.6468963623	0.000717743664685935	\\
3362.7124786377	0.000582069008466173	\\
3384.41390991211	6.81583058203349e-06	\\
3403.59191894531	4.91872525271201e-06	\\
3413.68560791016	0.000530666418094145	\\
3437.91046142578	0.000528561546511735	\\
3442.45262145996	1.6503595394656e-05	\\
3464.32228088379	1.27647921024859e-05	\\
3489.5565032959	0.000864047231281616	\\
3503.18298339844	0.000626140590487689	\\
3526.56669616699	1.72468350757618e-05	\\
3541.87545776367	0.000659962943034613	\\
3545.40824890137	1.76314369142984e-05	\\
3583.93249511719	0.000642043365393029	\\
3594.02618408203	1.79344998676322e-05	\\
3613.87710571289	0.000593785420305106	\\
3623.12965393066	3.5024896014407e-05	\\
3652.40135192871	0.000600336451554046	\\
3666.19606018066	2.12626639810046e-05	\\
3670.90644836426	0.000590709996631033	\\
3680.32722473145	2.05589672330876e-05	\\
3706.73904418945	0.000471299529963305	\\
3735.33782958984	1.79961720042925e-05	\\
3751.65596008301	0.000547954913416263	\\
3764.44129943848	7.3410729447201e-06	\\
3784.79690551758	0.000432216715023835	\\
3790.01197814941	9.63172859342908e-06	\\
3818.77899169922	0.000266106900487602	\\
3830.5549621582	7.75679765405743e-06	\\
3841.99447631836	2.29603961201531e-06	\\
3843.6767578125	0.000389979783851306	\\
3890.78063964844	0.000610999381141089	\\
3900.70610046387	6.32067011909121e-06	\\
3915.5101776123	0.00048444979571906	\\
3941.92199707031	1.38283494399713e-05	\\
3955.88493347168	0.000474692944417632	\\
3969.51141357422	1.67352630538889e-05	\\
3984.48371887207	0.000670363922227594	\\
4008.20388793945	6.31567994181295e-06	\\
4013.92364501953	0.000568740036921019	\\
4042.52243041992	9.25709116850075e-07	\\
4058.16764831543	0.00053149274605912	\\
4068.26133728027	1.26506567890911e-05	\\
4081.88781738281	0.000583528549659762	\\
4103.92570495605	1.00166557915445e-05	\\
4120.58029174805	1.69660869600042e-05	\\
4123.27194213867	0.000793314907896333	\\
4152.54364013672	0.000293468564900561	\\
4166.00189208984	2.05107413960588e-06	\\
4188.71269226074	0.000342450759316741	\\
4197.62878417969	1.72699135633571e-06	\\
4240.69519042969	1.03003563702592e-06	\\
4249.2748260498	0.000325766746403958	\\
4273.83613586426	8.41514664395222e-06	\\
4284.43450927734	0.000468125688561544	\\
4289.48135375977	0.000586482171425398	\\
4300.07972717285	8.90440880142788e-06	\\
4325.14572143555	1.43813933262633e-05	\\
4327.33268737793	0.0004840604933569	\\
4356.09970092773	0.000484530417448546	\\
4377.63290405273	1.45737068716246e-05	\\
4401.68952941895	0.000774524843442457	\\
4418.17588806152	4.82771532038629e-06	\\
4424.56855773926	0.000367105338927777	\\
4445.59707641602	5.41889595924331e-06	\\
4465.27976989746	0.000355136479857534	\\
4486.98120117188	1.21541047604644e-05	\\
4519.95391845703	5.67064263294756e-06	\\
4522.98202514648	0.00025004945653668	\\
4530.3840637207	0.000356430325757123	\\
4552.59017944336	6.38030000119516e-06	\\
4561.33804321289	0.000317729710178336	\\
4569.24476623535	9.2348067653213e-06	\\
4610.62889099121	1.79912853073162e-05	\\
4620.21789550781	0.000444765038223735	\\
4629.80690002441	0.000397872538882766	\\
4657.39631652832	1.29359729252064e-05	\\
4687.84561157227	4.08464146967592e-06	\\
4689.35966491699	0.000445365227699075	\\
4719.3042755127	0.000649345537214976	\\
4722.66883850098	2.95369053749427e-05	\\
4750.25825500488	0.000429821756145891	\\
4752.78167724609	1.0175259962e-05	\\
4793.99757385254	1.62749446185814e-05	\\
4795.51162719727	0.000388860652850102	\\
4802.24075317383	0.000361603322913454	\\
4811.32507324219	5.94568213241513e-06	\\
4866.67213439941	0.000291405080837547	\\
4867.00859069824	2.47798018375242e-06	\\
4888.71002197266	0.000252072510969257	\\
4896.28028869629	7.55107077438106e-06	\\
4921.5145111084	0.000316724030976084	\\
4935.81390380859	4.10662365463961e-06	\\
4949.1039276123	4.64655712046177e-06	\\
4960.71166992188	0.000402978758248765	\\
4978.71208190918	0.000468180897837151	\\
4997.21717834473	8.37672546406342e-06	\\
5014.37644958496	0.000433635844203512	\\
5034.05914306641	1.06090999260733e-05	\\
5044.32106018066	0.000319545565487124	\\
5046.17156982422	1.11395078315428e-05	\\
5086.54632568359	0.000294922481043896	\\
5097.31292724609	7.61309654225947e-06	\\
5110.77117919922	0.000218834218534099	\\
5119.35081481934	4.46953220748858e-06	\\
5171.33331298828	4.84028611539952e-06	\\
5171.66976928711	0.00025171854811649	\\
5187.31498718262	6.24899495896131e-06	\\
5198.41804504395	0.000321169223986502	\\
5216.92314147949	5.38549358835418e-06	\\
5239.29748535156	0.000278295134923834	\\
5261.3353729248	0.000336957980763167	\\
5270.25146484375	1.1692863141711e-05	\\
5292.28935241699	0.000378488876778073	\\
5301.37367248535	1.10421681130885e-05	\\
5321.0563659668	1.03876364958088e-05	\\
5341.91665649414	0.000349694483887901	\\
5353.52439880371	4.73575395169071e-06	\\
5374.72114562988	0.000515607579173664	\\
5385.99243164063	6.41450067772535e-06	\\
5399.78713989258	0.000312720323559007	\\
5442.34886169434	1.01572755753305e-05	\\
5445.88165283203	0.000359639521747176	\\
5452.77900695801	0.000249051536237267	\\
5478.85437011719	1.70230064536994e-06	\\
5502.74276733398	0.000195467337440893	\\
5517.37861633301	5.62595319896183e-06	\\
5536.72485351563	0.000205381866187438	\\
5540.93055725098	5.03125441339611e-06	\\
5570.03402709961	0.000199124084767379	\\
5576.59492492676	1.91570817014157e-06	\\
5590.05317687988	1.83606993354424e-06	\\
5603.67965698242	0.000226475049935203	\\
5635.13832092285	0.00022117056216924	\\
5650.27885437012	1.14233392652408e-05	\\
5673.83079528809	0.000247339317553751	\\
5675.1766204834	8.39181770418131e-06	\\
5701.25198364258	1.32927387600714e-05	\\
5703.60717773438	0.000306963689857095	\\
5738.93508911133	1.89218334531614e-05	\\
5741.12205505371	0.00032581818125273	\\
5774.59945678711	0.000510647018006421	\\
5791.59049987793	7.87271037919699e-08	\\
5807.06748962402	0.000199119594622807	\\
5816.48826599121	4.1474735773535e-06	\\
5843.06831359863	0.000288361501778332	\\
5862.24632263184	1.2849471984479e-06	\\
5865.94734191895	7.24954310250965e-06	\\
5887.14408874512	0.000220274084051855	\\
5900.09765625	8.21614197505147e-06	\\
5923.64959716797	0.000282572885653791	\\
5946.36039733887	1.21838671670929e-05	\\
5956.95877075195	0.000269157247636584	\\
5971.25816345215	2.38697507898145e-06	\\
6000.02517700195	0.000244546211458977	\\
6014.66102600098	0.000332922567885367	\\
6028.96041870117	1.48050856697783e-05	\\
6043.09158325195	0.000266084893079135	\\
6068.66226196289	6.99776750201772e-06	\\
6074.21379089355	2.08917395453983e-06	\\
6083.46633911133	0.000240181916409041	\\
6119.46716308594	0.000277622421691482	\\
6130.90667724609	5.69818189098662e-06	\\
6138.98162841797	0.000254824099814011	\\
6164.04762268066	1.44399787738979e-06	\\
6178.68347167969	5.42520952924464e-06	\\
6181.54335021973	0.000220385457105483	\\
6212.49732971191	7.20831684209006e-06	\\
6238.06800842285	0.000195070931615136	\\
6241.43257141113	4.57229771280171e-06	\\
6243.78776550293	0.000173281807255649	\\
6283.32138061523	0.000231822640887475	\\
6306.03218078613	7.3412656550572e-06	\\
6318.8175201416	0.000249417152874284	\\
6337.65907287598	5.52954391835854e-06	\\
6345.73402404785	0.00029446919461319	\\
6366.08963012695	6.2100208103641e-06	\\
6387.11814880371	0.000244458678436388	\\
6390.65093994141	7.10534345055929e-06	\\
6415.54870605469	0.000221946364104978	\\
6429.00695800781	1.22616619584406e-06	\\
6457.26928710938	0.000229269269983129	\\
6458.7833404541	1.95014109586611e-06	\\
6504.70962524414	1.7307515031445e-06	\\
6510.59761047363	0.000136816160768036	\\
6518.67256164551	0.000164208935769965	\\
6524.39231872559	2.01395199319872e-06	\\
6559.72023010254	1.95433193943848e-06	\\
6583.77685546875	0.00014485093426676	\\
6588.99192810059	0.000233834116005443	\\
6605.14183044434	3.97376895104089e-06	\\
6630.03959655762	1.02945615070397e-05	\\
6635.59112548828	0.000226806539644565	\\
6656.1149597168	0.000213688696317678	\\
6658.80661010742	1.14074961250696e-05	\\
6689.42413330078	0.000158791452945193	\\
6695.14389038086	4.95530719738092e-06	\\
6721.55570983887	0.000149829325769743	\\
6739.38789367676	2.07439743448376e-06	\\
6770.51010131836	5.59224202749519e-06	\\
6780.77201843262	0.000183447999698219	\\
6793.05267333984	3.77914620834116e-06	\\
6800.28648376465	0.000163741016853516	\\
6849.57733154297	2.3440799220716e-06	\\
6855.29708862305	0.000179139549180843	\\
6860.51216125488	0.000126682913869811	\\
6878.6808013916	3.12744318197685e-06	\\
6897.8588104248	0.000187767889492368	\\
6911.31706237793	5.2533065470392e-07	\\
6949.50485229492	1.76566637137338e-06	\\
6952.53295898438	0.000154484545594462	\\
6964.30892944336	3.21125115325149e-06	\\
6990.88897705078	0.000165458629526491	\\
6995.9358215332	6.32843776865978e-06	\\
7020.32890319824	0.000177462929868539	\\
7053.97453308105	2.40545736774005e-06	\\
7061.04011535645	0.000289510903326997	\\
7065.2458190918	0.000238201834883649	\\
7075.00305175781	7.73520048527063e-06	\\
7111.17210388184	4.71987076222518e-06	\\
7115.37780761719	0.000178892784429499	\\
7139.60266113281	0.000170853377178024	\\
7144.14482116699	1.61880885682995e-06	\\
7188.22059631348	0.000120269987057654	\\
7188.5570526123	5.56244893688862e-06	\\
7205.2116394043	0.000134474611435783	\\
7216.81938171387	2.14660610532611e-06	\\
7237.84790039063	2.44038838167499e-06	\\
7267.96073913574	0.000140413246459606	\\
7275.02632141113	4.12531234303937e-06	\\
7294.03610229492	0.00011832497021107	\\
7325.99945068359	0.000143099410848117	\\
7338.44833374023	5.9861429141395e-06	\\
7339.2894744873	0.000154919185410985	\\
7370.24345397949	1.76459947850269e-06	\\
7394.3000793457	4.15410926482878e-06	\\
7397.49641418457	0.000127166528165422	\\
7419.87075805664	6.68092682547761e-06	\\
7439.88990783691	0.000178900004061308	\\
7452.00233459473	5.01442668947566e-06	\\
7461.92779541016	0.000217541966762959	\\
7483.46099853516	3.7048318782635e-06	\\
7486.65733337402	0.000105227489775132	\\
7523.33106994629	2.11018845830096e-06	\\
7531.74247741699	0.00010742045393501	\\
7570.0984954834	0.000121640171609953	\\
7575.81825256348	7.56536759843873e-07	\\
7579.51927185059	4.94284301552118e-06	\\
7604.92172241211	0.00014800517901115	\\
7640.7543182373	0.000111474071975215	\\
7644.96002197266	3.25290830641091e-06	\\
7666.15676879883	0.000110642908830881	\\
7667.50259399414	2.41791975934162e-06	\\
7681.97021484375	1.94475905204497e-06	\\
7697.95188903809	0.000121451929478697	\\
7717.80281066895	4.49068671029372e-06	\\
7731.76574707031	0.000141712854971058	\\
7755.14945983887	1.30236541808407e-06	\\
7780.38368225098	0.000150132380071851	\\
7787.61749267578	1.19264137098689e-06	\\
7807.46841430664	0.000165074401648792	\\
7838.92707824707	5.437291114808e-06	\\
7846.32911682129	0.000142393866155043	\\
7855.07698059082	0.000134193675916856	\\
7886.19918823242	5.11250625294106e-06	\\
7892.76008605957	8.13112984579071e-05	\\
7920.01304626465	3.04375188463146e-06	\\
7923.71406555176	1.30571691029287e-06	\\
7949.11651611328	9.39443422951607e-05	\\
7959.37843322754	0.000103496159087142	\\
7964.42527770996	3.05301384846298e-06	\\
8017.08068847656	2.94210087666008e-06	\\
8023.97804260254	0.000107229667115201	\\
8033.56704711914	3.00497652668578e-06	\\
8042.6513671875	0.000147684779649185	\\
8080.33447265625	0.000158768635782146	\\
8091.77398681641	1.58048832117782e-06	\\
8107.08274841309	0.000135387600697452	\\
8119.1951751709	3.84739948393009e-06	\\
8138.54141235352	0.000121299693126801	\\
8143.0835723877	1.69126494919333e-06	\\
8162.59803771973	2.97176344126522e-06	\\
8172.18704223633	0.000112711057525541	\\
8204.31861877441	0.000123050860187822	\\
8221.98257446289	3.01568915655815e-06	\\
8250.24490356445	0.000123365347784653	\\
8254.28237915039	5.6184394571063e-06	\\
8272.95570373535	0.000114643778805393	\\
8277.83432006836	3.93375370453526e-06	\\
8302.39562988281	0.000115896419392334	\\
8310.97526550293	2.16455352254312e-06	\\
8349.66773986816	0.000165240041238861	\\
8355.38749694824	6.90538694025355e-06	\\
8376.07955932617	2.32976912920683e-06	\\
8382.64045715332	0.000150244897053948	\\
8409.05227661133	0.000112504800918197	\\
8416.45431518555	1.70750411879829e-06	\\
8455.3150177002	0.000105656654586177	\\
8461.03477478027	3.38583383495721e-06	\\
8476.00708007813	9.6615418887431e-05	\\
8494.8486328125	2.29479529021714e-06	\\
8505.44700622559	9.06012537404549e-05	\\
8526.98020935059	2.58954080984928e-06	\\
8554.73785400391	0.000114046235366599	\\
8561.97166442871	3.27652580429448e-06	\\
8578.96270751953	6.67164702269745e-05	\\
8590.73867797852	2.15615940850517e-06	\\
8616.64581298828	0.000109820366043356	\\
8628.42178344727	2.89190831949735e-06	\\
8660.55335998535	2.64064232384453e-06	\\
8668.62831115723	0.000117117211720217	\\
8678.38554382324	0.000104899969080354	\\
8706.14318847656	2.05826777687546e-06	\\
8712.87231445313	0.00011900581787977	\\
8730.87272644043	5.16098647225289e-06	\\
8758.96682739258	0.000108115121929665	\\
8764.85481262207	1.81934396990994e-06	\\
8783.69636535645	2.24790636151395e-06	\\
8792.61245727539	8.91506339240554e-05	\\
8819.52896118164	9.55943599604307e-07	\\
8838.37051391602	0.000106474894567009	\\
8847.95951843262	8.2986789300899e-05	\\
8852.83813476563	3.41863804965355e-07	\\
8900.61492919922	1.83265771692025e-06	\\
8910.03570556641	8.129024128681e-05	\\
8921.13876342773	3.51335414206656e-06	\\
8949.06463623047	0.000109625034977859	\\
8965.0463104248	1.70824334736081e-06	\\
8971.94366455078	0.000102207007154051	\\
9007.77626037598	1.57994773056672e-06	\\
9016.52412414551	0.000123199468463581	\\
9022.24388122559	0.000124998943349055	\\
9039.06669616699	3.26578208597723e-06	\\
9060.43167114258	9.62618044214177e-05	\\
9065.478515625	3.50418895838624e-06	\\
9098.9559173584	0.000115367932542553	\\
9108.20846557617	2.28452637566779e-06	\\
9129.74166870117	2.3347209893284e-06	\\
9149.92904663086	9.92951672358615e-05	\\
9161.53678894043	9.04348096111623e-05	\\
9165.9107208252	4.14154239716152e-06	\\
9210.99586486816	8.31973326292631e-05	\\
9224.45411682129	1.4062621864873e-06	\\
9225.63171386719	1.66595062307425e-06	\\
9253.05290222168	9.66382909498146e-05	\\
9263.48304748535	2.81302037682386e-06	\\
9281.48345947266	9.78822714365751e-05	\\
9322.6993560791	4.01919728912102e-06	\\
9326.2321472168	9.73684118656326e-05	\\
9343.89610290527	1.17045829426019e-06	\\
9357.3543548584	0.000115900821024952	\\
9370.64437866211	3.62239992664395e-06	\\
9386.1213684082	0.000131155570221182	\\
9397.56088256836	0.000112217957760551	\\
9404.12178039551	4.80374480225714e-06	\\
9433.56170654297	4.5044357356467e-06	\\
9457.78656005859	0.000193632945548996	\\
9480.66558837891	3.7232289089081e-06	\\
9496.142578125	8.97802468281491e-05	\\
9510.4419708252	2.21292290098421e-06	\\
9521.88148498535	9.65854147799537e-05	\\
9553.34014892578	6.98701185915947e-05	\\
9565.28434753418	2.42546720050682e-06	\\
9572.85461425781	3.71547286117445e-07	\\
9588.66806030273	7.66817834969366e-05	\\
9623.65951538086	1.07128698112696e-06	\\
9636.1083984375	8.6396586661101e-05	\\
9662.01553344727	0.000151019091842914	\\
9667.73529052734	2.79730619723974e-06	\\
9676.98783874512	8.97636078146489e-07	\\
9701.8856048584	0.000106993461314891	\\
9715.68031311035	0.000123171686510578	\\
9722.57766723633	2.66800337643512e-06	\\
9758.91494750977	3.60491238324549e-06	\\
9760.93368530273	9.89485806957414e-05	\\
9774.39193725586	3.72815051657666e-06	\\
9792.22412109375	7.97739063654341e-05	\\
9816.28074645996	1.65316342759997e-06	\\
9830.07545471191	8.20681313792511e-05	\\
9853.96385192871	9.09247774711294e-05	\\
9870.28198242188	2.73522742305026e-06	\\
9884.07669067383	2.58516488939433e-06	\\
9887.77770996094	8.57066230250243e-05	\\
9920.41397094727	3.41331300940038e-06	\\
9941.27426147461	7.00676267788871e-05	\\
9953.72314453125	2.77234527771633e-06	\\
9971.72355651855	8.72013809888073e-05	\\
9989.89219665527	2.80186856297328e-06	\\
10001.163482666	0.000123889494729719	\\
10025.7247924805	0.000113381570611808	\\
10044.9028015137	2.94570679906253e-07	\\
10059.3704223633	8.13789634981808e-05	\\
10072.3239898682	2.4321855303961e-06	\\
10088.4738922119	4.42401517091347e-06	\\
10103.950881958	9.40415872834688e-05	\\
10118.586730957	1.89578455097701e-06	\\
10137.5965118408	8.12763791397729e-05	\\
10157.4474334717	1.39913597259059e-06	\\
10165.0177001953	8.99406155265251e-05	\\
10199.6726989746	9.31379800686205e-05	\\
10207.9158782959	2.35790683867854e-06	\\
10237.0193481445	9.2009773234757e-05	\\
10253.5057067871	3.20374333948582e-06	\\
10255.0197601318	6.81736515476473e-05	\\
10260.7395172119	6.30983513148635e-07	\\
10302.2918701172	0.000127330711669897	\\
10318.7782287598	8.46508719612895e-07	\\
10323.4886169434	6.02609331472078e-06	\\
10331.9000244141	0.000181464099524852	\\
10374.7982025146	0.000101968877517669	\\
10377.9945373535	1.49415143610703e-06	\\
10401.5464782715	0.000114546311044652	\\
10414.8365020752	5.19859971682261e-06	\\
10429.3041229248	1.37172285564633e-06	\\
10453.0242919922	8.17720554559183e-05	\\
10462.1086120605	9.67793310216161e-07	\\
10469.0059661865	8.20589794512891e-05	\\
10503.1562805176	8.816934381661e-05	\\
10526.5399932861	1.00707343398554e-06	\\
10532.7644348145	1.11923136050336e-06	\\
10540.1664733887	7.73990393852696e-05	\\
10570.2793121338	8.56067154007713e-05	\\
10582.8964233398	2.89221139165466e-06	\\
10599.214553833	7.25396822300584e-05	\\
10621.2524414063	2.35414604242945e-06	\\
10632.1872711182	3.16656953356571e-06	\\
10662.3001098633	0.000105315460300503	\\
10689.5530700684	1.6275250653779e-06	\\
10697.7962493896	0.000139902215137073	\\
10711.5909576416	0.000127541118368442	\\
10716.4695739746	1.47568898212682e-06	\\
10737.4980926514	0.000110663627051602	\\
10741.1991119385	5.12660599276592e-06	\\
10776.0223388672	9.0869554669868e-05	\\
10793.5180664063	2.20690266257743e-06	\\
10813.2007598877	0.000166827054110465	\\
10827.5001525879	2.95688173271703e-06	\\
10849.0333557129	1.90044834376617e-06	\\
10853.2390594482	0.000101871201413619	\\
10880.6602478027	3.92381311798719e-06	\\
10903.539276123	7.66860648073604e-05	\\
10933.1474304199	1.59241370364235e-06	\\
10936.6802215576	8.73346927077649e-05	\\
10941.5588378906	1.53366844024806e-06	\\
10943.5775756836	9.05150615301284e-05	\\
10974.5315551758	6.87510085746866e-05	\\
10975.7091522217	2.27559616479903e-06	\\
11010.5323791504	9.45681195405214e-05	\\
11020.121383667	2.01119190302564e-07	\\
11046.7014312744	0.000104745354976248	\\
11074.7955322266	1.08715551689987e-06	\\
11091.4501190186	0.000100797895423457	\\
11102.2167205811	2.01407291733633e-06	\\
11118.030166626	0.000107395367146227	\\
11125.4322052002	4.22491411748837e-06	\\
11159.9189758301	2.89466515863062e-06	\\
11160.9283447266	8.59137285979591e-05	\\
11193.9010620117	9.01532955932016e-05	\\
11212.2379302979	1.11409357343594e-06	\\
11230.7430267334	5.46326092799415e-06	\\
11243.3601379395	9.08626480011916e-05	\\
11277.8469085693	1.34478282495436e-06	\\
11280.2021026611	0.000123468534923438	\\
11284.2395782471	1.13469426379893e-06	\\
11313.1748199463	9.44344819264283e-05	\\
11335.0444793701	3.41682408096205e-06	\\
11341.6053771973	0.00012587905250968	\\
11352.5402069092	0.00010648656728346	\\
11377.6062011719	2.07836879210174e-06	\\
11392.4102783203	1.69938738660693e-06	\\
11412.4294281006	0.000105927653378306	\\
11439.1777038574	0.000112353657508569	\\
11441.3646697998	1.07656797450955e-06	\\
11454.6546936035	0.000102562055947252	\\
11468.1129455566	1.92213726515302e-06	\\
11499.7398376465	2.88003720037217e-07	\\
11505.9642791748	6.94401064274853e-05	\\
11536.918258667	2.80697542378792e-06	\\
11544.9932098389	8.37594499130504e-05	\\
11559.6290588379	2.30222435885501e-06	\\
11575.9471893311	8.91555280052032e-05	\\
11602.5272369385	1.35367788953623e-06	\\
11617.8359985352	8.71672626579757e-05	\\
11633.4812164307	1.1283957103056e-06	\\
11641.3879394531	0.000133468907080812	\\
11661.7435455322	2.36194405453506e-06	\\
11684.2861175537	0.000115697882681996	\\
11713.0531311035	0.000113442316869433	\\
11716.92237854	2.91373584713904e-06	\\
11738.1191253662	4.04256124846937e-06	\\
11754.1007995605	0.000139276355627836	\\
11789.4287109375	3.00320929627732e-06	\\
11794.6437835693	0.000129204220554165	\\
11812.1395111084	1.2573962890404e-06	\\
11821.3920593262	0.000136918215998398	\\
11831.8222045898	8.22124341235811e-05	\\
11854.8694610596	2.13718707783932e-06	\\
11874.3839263916	3.67887892473811e-06	\\
11887.1692657471	8.69406667556427e-05	\\
11919.3008422852	9.3032301284722e-05	\\
11924.0112304688	2.74209800652838e-06	\\
11943.6939239502	8.99468888335807e-05	\\
11967.582321167	2.92578019146009e-06	\\
11980.7041168213	9.85408904129585e-05	\\
11988.4426116943	2.27555513627415e-06	\\
12003.4149169922	1.81227890332636e-06	\\
12022.5929260254	0.00013154361819483	\\
12051.5281677246	2.11799172514479e-06	\\
12055.73387146	9.28123328545402e-05	\\
12089.7159576416	1.86534136348222e-06	\\
12094.762802124	0.000121023811102944	\\
12113.7725830078	2.01711370830773e-06	\\
12135.8104705811	0.000120045901129792	\\
12151.4556884766	1.90910702314854e-06	\\
12162.7269744873	8.5095654465423e-05	\\
12179.2133331299	9.23677618692165e-05	\\
12203.6064147949	1.99599080586967e-06	\\
12224.6349334717	2.62839484594665e-06	\\
12232.878112793	8.86341940709944e-05	\\
12261.9815826416	7.59696650665558e-05	\\
12265.3461456299	1.26516952150393e-06	\\
12278.2997131348	9.82264796406364e-07	\\
12306.3938140869	8.38874918154707e-05	\\
12330.1139831543	8.88449122296404e-05	\\
12338.1889343262	5.29753641030952e-06	\\
12363.2549285889	6.00571660363418e-06	\\
12377.5543212891	0.000113040069652099	\\
12383.4423065186	0.000115401997723585	\\
12391.6854858398	2.89766344527582e-06	\\
12418.2655334473	1.24366559637851e-06	\\
12427.3498535156	8.84489368633576e-05	\\
12466.3787841797	2.91358667302028e-06	\\
12476.6407012939	0.000110609186708134	\\
12499.3515014648	2.55548888359e-06	\\
12503.3889770508	0.000114353605809609	\\
12529.2961120605	2.79726494506512e-06	\\
12540.9038543701	9.43225981153692e-05	\\
12571.8578338623	0.000136395284212874	\\
12579.2598724365	4.1266717026945e-06	\\
12588.0077362061	9.40485693728163e-05	\\
12604.8305511475	1.92317716890626e-06	\\
12636.1209869385	0.000106699340288559	\\
12648.5698699951	1.81143565985577e-06	\\
12672.2900390625	3.21929091545876e-07	\\
12681.3743591309	6.34198205851478e-05	\\
12691.2998199463	2.55799530700506e-06	\\
12701.0570526123	0.000105516685152605	\\
12725.1136779785	0.000115531201682085	\\
12741.0953521729	4.6721161123509e-06	\\
12763.1332397461	0.000129264455810979	\\
12783.9935302734	1.47015776611816e-06	\\
12791.0591125488	4.43677672920354e-06	\\
12798.6293792725	0.00010799219518958	\\
12835.1348876953	0.000123352735501294	\\
12839.6770477295	2.65002881477953e-06	\\
12877.1919250488	0.000123702945548807	\\
12890.9866333008	5.28187134871897e-07	\\
12904.9495697021	2.22913109562683e-06	\\
12905.1177978516	8.05264681303718e-05	\\
12928.6697387695	1.36532169274588e-06	\\
12948.0159759521	0.000101508295042862	\\
12969.2127227783	8.77906096015392e-05	\\
12983.1756591797	2.6836719396993e-06	\\
13019.849395752	9.40305605308991e-07	\\
13030.2795410156	0.000110703454481681	\\
13043.0648803711	0.000195982944719259	\\
13051.3080596924	5.4119381396413e-06	\\
13071.8318939209	0.000182395134405724	\\
13091.6828155518	1.3196416944533e-06	\\
13108.6738586426	0.000108579624900231	\\
13126.0013580322	8.71507962994707e-07	\\
13145.6840515137	1.20015433673772e-06	\\
13149.5532989502	9.26075543978786e-05	\\
13176.301574707	9.20142958322713e-05	\\
13180.6755065918	1.27555092338822e-06	\\
13206.0779571533	3.05550492897683e-06	\\
13208.4331512451	9.15697492278375e-05	\\
13244.6022033691	0.000137796527368295	\\
13267.8176879883	2.88757804686518e-06	\\
13281.6123962402	0.000128242005477825	\\
13291.0331726074	1.02742274291824e-06	\\
13321.9871520996	1.67673662710251e-06	\\
13328.8845062256	0.000103744767874908	\\
13349.4083404541	8.67307420872129e-05	\\
13355.9692382813	5.58171768537204e-06	\\
13378.8482666016	1.23069532243373e-06	\\
13407.2788238525	0.000108159588090774	\\
13413.5032653809	8.97193321174731e-05	\\
13422.5875854492	2.22967243327145e-06	\\
13448.1582641602	0.000122744590621294	\\
13467.6727294922	1.78947975359706e-06	\\
13495.5986022949	4.00403894533053e-06	\\
13508.3839416504	0.000110255645653307	\\
13528.7395477295	3.21439358974541e-06	\\
13536.9827270508	9.81114148885112e-05	\\
13555.1513671875	6.73950118392422e-06	\\
13567.7684783936	0.000117741583476572	\\
13602.2552490234	4.69971005167582e-06	\\
13604.6104431152	9.48220557343191e-05	\\
13616.8910980225	0.00014246783619124	\\
13645.4898834229	9.19779139074193e-07	\\
13658.948135376	1.00376745349963e-06	\\
13677.2850036621	7.01364593723386e-05	\\
13699.995803833	7.48400133362566e-07	\\
13710.9306335449	0.000108132520289233	\\
13723.0430603027	1.17077171638261e-05	\\
13725.9029388428	0.000143393637652147	\\
13767.2870635986	0.000135308951620024	\\
13777.3807525635	4.49683366078078e-06	\\
13796.3905334473	2.14228611110611e-06	\\
13814.8956298828	0.00010536745586697	\\
13822.9705810547	0.000115161219040752	\\
13847.7001190186	2.23750168110691e-06	\\
13865.1958465576	9.68813190994713e-05	\\
13875.2895355225	2.12567740732531e-06	\\
13911.1221313477	8.51679182057978e-05	\\
13915.6642913818	3.81785154110306e-06	\\
13936.3563537598	8.75485152557856e-05	\\
13949.9828338623	9.51288570475737e-07	\\
13962.4317169189	1.77644941922068e-06	\\
13987.3294830322	6.69557647696316e-05	\\
13994.8997497559	2.17393540767999e-06	\\
14025.5172729492	0.000104565442874035	\\
14041.1624908447	4.3843531713359e-06	\\
14046.3775634766	0.000148693851285405	\\
14061.8545532227	2.03129572369198e-06	\\
14081.8737030029	8.92010254622674e-05	\\
14103.5751342773	0.000102779908186664	\\
14112.1547698975	5.60747013301211e-06	\\
14138.3983612061	1.91132039356801e-06	\\
14153.7071228027	0.000102148181013918	\\
14177.0908355713	3.64996306739753e-06	\\
14187.0162963867	0.000160363569026039	\\
14198.7922668457	0.000115529960505627	\\
14218.3067321777	5.14232917260467e-06	\\
14241.3539886475	4.11026660061059e-06	\\
14258.0085754395	0.000110024455135954	\\
14271.635055542	2.54645684477532e-06	\\
14291.6542053223	0.000107001173674196	\\
14302.2525787354	2.86020895911967e-06	\\
14315.8790588379	8.28377012816056e-05	\\
14348.851776123	1.53597273527329e-06	\\
14358.9454650879	9.86063184245991e-05	\\
14370.2167510986	1.80103266107986e-06	\\
14380.9833526611	0.000102524611748173	\\
14421.8627929688	8.92510382551202e-05	\\
14428.2554626465	2.03293078579812e-06	\\
14463.9198303223	2.75934618848047e-06	\\
14468.9666748047	0.000108159028084507	\\
14488.6493682861	0.000115891289355168	\\
14499.2477416992	4.20436727113333e-06	\\
14520.1080322266	1.13747579498191e-06	\\
14523.3043670654	9.7252909580313e-05	\\
14553.4172058105	3.23876885591968e-06	\\
14559.9781036377	7.60065739669702e-05	\\
14600.3528594971	8.86146332478846e-05	\\
14604.7267913818	1.71000860781302e-06	\\
14614.6522521973	4.68048109201212e-06	\\
14631.9797515869	0.000106755779423577	\\
14644.4286346436	4.30415028287435e-06	\\
14667.9805755615	8.72146576498056e-05	\\
14681.4388275146	2.74042551104023e-06	\\
14711.0469818115	6.89394999758702e-05	\\
14719.9630737305	2.18897040917771e-05	\\
14732.7484130859	0.000152152362097062	\\
14761.1789703369	6.45865406055413e-07	\\
14769.0856933594	9.90584206664837e-05	\\
14795.6657409668	0.000122654876125561	\\
14812.6567840576	1.47831400088304e-05	\\
14822.7504730225	2.40649132002932e-06	\\
14836.5451812744	5.96056500071714e-05	\\
14848.6576080322	6.0307729462584e-06	\\
14874.564743042	0.000146894885861285	\\
14883.3126068115	2.55770015396711e-06	\\
14915.7806396484	0.000128267137832273	\\
14943.2018280029	0.000135012790585359	\\
14949.7627258301	2.61530054208763e-06	\\
14958.3423614502	0.00010845700852071	\\
14967.0902252197	4.52811772883259e-06	\\
14997.5395202637	1.68680668732336e-06	\\
15014.3623352051	9.52000903624662e-05	\\
15025.9700775146	7.65540352702504e-05	\\
15034.2132568359	3.12461988221265e-06	\\
15056.9240570068	9.08097483468898e-05	\\
15066.0083770752	6.32382502358569e-06	\\
15111.2617492676	7.56965218346227e-05	\\
15119.5049285889	3.048042125803e-06	\\
15138.8511657715	7.51951142623217e-05	\\
15157.0198059082	4.49466679273639e-06	\\
15159.7114562988	2.13084177132667e-06	\\
15185.4503631592	8.81320364481381e-05	\\
15196.5534210205	7.75235304678089e-07	\\
15201.2638092041	8.45718867885678e-05	\\
15235.0776672363	3.53113537555235e-06	\\
15251.9004821777	0.000104303030560439	\\
15271.0784912109	7.84693658741796e-05	\\
15278.4805297852	5.11827625080137e-07	\\
15301.5277862549	9.11156696706072e-05	\\
15324.2385864258	3.9826389796378e-06	\\
15334.3322753906	1.5301253095448e-06	\\
15352.1644592285	8.24048459626661e-05	\\
15363.9404296875	9.37212976352716e-05	\\
15380.7632446289	3.15721643538037e-06	\\
15409.5302581787	0.000109406686547432	\\
15415.5864715576	2.76452417891262e-06	\\
15446.7086791992	0.000156233475737303	\\
15460.3351593018	4.38839498567452e-06	\\
15470.4288482666	0.000104721845050041	\\
15491.7938232422	3.45265333472374e-06	\\
15502.2239685059	0.000106063784827306	\\
15526.1123657227	3.60940778910491e-06	\\
15539.4023895264	3.05001758949698e-06	\\
15558.5803985596	0.000102685728022575	\\
15583.4781646729	6.75705451993132e-05	\\
15602.6561737061	2.51608784959435e-06	\\
15627.0492553711	1.2152829002179e-06	\\
15637.1429443359	6.71561329618166e-05	\\
15648.4142303467	3.63822202366223e-06	\\
15665.9099578857	8.86759089524084e-05	\\
15690.1348114014	8.48562752763532e-05	\\
15698.3779907227	1.33332642065911e-05	\\
15709.6492767334	2.70157521773103e-06	\\
15734.5470428467	8.92970030093687e-05	\\
15746.6594696045	6.67436716494727e-05	\\
15757.9307556152	1.9496553105622e-06	\\
15782.8285217285	0.000120590037615283	\\
15798.6419677734	3.33774224749784e-06	\\
15813.2778167725	5.95957833498137e-05	\\
15841.3719177246	5.86080431546065e-07	\\
15849.1104125977	6.96853225008203e-05	\\
15871.8212127686	2.17196415147748e-06	\\
15886.9617462158	3.89497655191505e-06	\\
15892.3450469971	7.72110895567336e-05	\\
15914.3829345703	8.95784926398964e-05	\\
15939.1124725342	5.13179421817013e-06	\\
15949.7108459473	7.95477924028645e-05	\\
15960.8139038086	1.17499542489241e-05	\\
15991.431427002	2.83231447542915e-06	\\
16000.6839752197	5.14480278851469e-05	\\
16021.7124938965	1.81443897850268e-05	\\
16047.7878570557	4.53943273902544e-05	\\
16059.0591430664	4.48883937123483e-05	\\
16082.442855835	3.02435269685134e-05	\\
16083.283996582	3.10012765156146e-05	\\
16116.2567138672	3.70761094190489e-07	\\
16118.4436798096	3.79109374477234e-07	\\
16151.2481689453	3.36840847369106e-05	\\
16151.7528533936	3.25325203824397e-05	\\
16172.9496002197	4.94025365028499e-05	\\
16186.071395874	4.28350759875604e-05	\\
16220.2217102051	6.88896157339265e-06	\\
16220.7263946533	8.02351173363226e-06	\\
16232.3341369629	2.82101921692963e-08	\\
16268.5031890869	3.45546772410503e-07	\\
16288.5223388672	1.2264917723773e-05	\\
16289.0270233154	1.15891114855529e-05	\\
16322.16796875	4.77899143103889e-05	\\
16337.6449584961	1.12349990658427e-05	\\
16354.4677734375	6.6274698566132e-05	\\
16358.337020874	6.33080772535391e-05	\\
16391.3097381592	2.97716164674373e-05	\\
16391.8144226074	3.06035299280681e-05	\\
16422.095489502	2.75384059795566e-07	\\
16433.7032318115	9.04328779412463e-06	\\
16443.628692627	2.060881942592e-06	\\
16465.3301239014	2.12718368564098e-06	\\
16494.7700500488	5.62496564528801e-05	\\
16517.6490783691	8.46655242887048e-05	\\
16528.2474517822	8.78801294824377e-07	\\
16529.0885925293	8.94546030422103e-06	\\
16548.4348297119	7.63157481521354e-05	\\
16565.930557251	7.79134978813453e-05	\\
16573.3325958252	5.98816352457849e-05	\\
16598.0621337891	6.55340900905499e-05	\\
16631.0348510742	2.07114873115258e-05	\\
16632.8853607178	2.11520173722657e-05	\\
16651.2222290039	2.00382027339517e-07	\\
16671.4096069336	1.15127152773373e-06	\\
16693.7839508057	7.63696934861605e-05	\\
16710.1020812988	8.42989691012288e-06	\\
16726.0837554932	7.87687654563534e-05	\\
16747.1122741699	2.09184628869102e-05	\\
16768.9819335938	7.9660753656141e-05	\\
16773.0194091797	8.51684860219654e-05	\\
16803.3004760742	4.55726254143591e-05	\\
16808.8520050049	3.65232167485859e-05	\\
16827.5253295898	8.15427870142951e-05	\\
16858.3110809326	5.0296474525431e-06	\\
16871.9375610352	0.000163949929445141	\\
16876.3114929199	0.000180084652772452	\\
16892.2931671143	1.80219870544405e-05	\\
16913.1534576416	1.85915539199559e-05	\\
16926.9481658936	8.583852320795e-05	\\
16940.5746459961	4.04058447734503e-05	\\
16951.1730194092	6.49552134657412e-05	\\
16982.2952270508	3.40022130560517e-05	\\
17008.37059021	7.69052888999114e-05	\\
17011.0622406006	7.81446425415687e-05	\\
17043.3620452881	3.2194664539226e-05	\\
17055.4744720459	4.4197530888453e-06	\\
17077.6805877686	3.71867586613222e-05	\\
17095.1763153076	4.85569140797494e-05	\\
17111.999130249	3.49666764759111e-05	\\
17112.6720428467	3.59054017130169e-05	\\
17145.6447601318	1.93216499095921e-06	\\
17151.3645172119	1.88074735986186e-07	\\
17161.962890625	5.04578363302127e-06	\\
17201.1600494385	2.27843291122818e-06	\\
17213.9453887939	5.83830861671608e-08	\\
17239.1796112061	2.5866777371513e-06	\\
17247.4227905273	3.81101321131422e-08	\\
17268.7877655029	1.27160097796923e-07	\\
17277.0309448242	2.72177545573758e-06	\\
17301.0875701904	5.98869610725691e-08	\\
17316.2281036377	4.97406287848567e-06	\\
17326.1535644531	1.03910283187672e-07	\\
17351.7242431641	1.06261160052904e-05	\\
17360.3038787842	3.591601174595e-06	\\
17382.6782226563	8.74157058670362e-05	\\
17398.9963531494	4.49975914250747e-06	\\
17416.3238525391	8.64540274750197e-05	\\
17420.5295562744	7.75491735980576e-05	\\
17436.1747741699	2.05253932788428e-05	\\
17456.8668365479	4.48637669659943e-05	\\
17476.8859863281	5.90924731760985e-05	\\
17489.6713256836	5.2260899299e-05	\\
17508.512878418	2.38242836655388e-05	\\
17523.9898681641	4.07797800746858e-05	\\
17556.2896728516	6.10842730410334e-05	\\
17557.8037261963	6.07196185776614e-05	\\
17591.2811279297	1.73468205779118e-05	\\
17592.1222686768	1.7570310341363e-05	\\
17608.9450836182	4.30516747766359e-07	\\
17626.7772674561	6.70095059406924e-06	\\
17651.1703491211	1.63789152691768e-07	\\
17685.657119751	3.63167665020011e-07	\\
17694.9096679688	8.39601751329611e-06	\\
17695.7508087158	7.98087155194497e-06	\\
17728.5552978516	5.27448804431084e-05	\\
17745.378112793	6.344171559572e-05	\\
17762.873840332	4.9447840266175e-05	\\
17763.5467529297	4.98437023016734e-05	\\
17797.5288391113	2.40906117578569e-06	\\
17800.0522613525	2.10427156297903e-07	\\
17811.6600036621	6.32309063902313e-06	\\
17832.1838378906	3.12538006955646e-06	\\
17866.3341522217	3.36545629424727e-05	\\
17866.3341522217	3.36545629424727e-05	\\
17898.2975006104	7.02090119830067e-05	\\
17900.8209228516	6.98434267312268e-05	\\
17934.1300964355	7.62833805720063e-06	\\
17935.3076934814	7.64464835232786e-06	\\
17968.7850952148	7.01752935698117e-05	\\
17972.486114502	7.08679130358288e-05	\\
18003.6083221436	3.31687029610768e-05	\\
18004.1130065918	3.40942761177951e-05	\\
18031.029510498	2.4790290538122e-07	\\
18038.7680053711	5.79238114339243e-06	\\
18071.9089508057	5.58519970434012e-05	\\
18072.2454071045	5.50664396788256e-05	\\
18105.3863525391	7.72970923761319e-05	\\
18108.0780029297	7.74732539898315e-05	\\
18140.714263916	6.00573538707556e-05	\\
18140.714263916	6.00573538707556e-05	\\
18174.8645782471	1.00871384266298e-05	\\
18184.4535827637	3.59904272980232e-07	\\
18209.351348877	3.55787036240028e-05	\\
18209.351348877	3.55787036240028e-05	\\
18241.9876098633	6.96604702914628e-05	\\
18244.5110321045	6.91360334110852e-05	\\
18277.4837493896	2.63805252992988e-05	\\
18277.9884338379	2.68931081568207e-05	\\
18299.8580932617	3.73806360453173e-07	\\
18313.4845733643	6.22487199186307e-06	\\
18340.5693054199	2.05295680620683e-07	\\
18350.1583099365	2.86762512802847e-06	\\
18379.9346923828	1.63480881695987e-07	\\
18383.8039398193	1.70988359644511e-07	\\
18413.0756378174	2.98071727784951e-06	\\
18416.9448852539	2.93667244114947e-06	\\
18431.7489624023	1.26727764829971e-07	\\
18461.3571166992	3.12402818981434e-06	\\
18483.7314605713	2.72727283985148e-07	\\
18485.0772857666	1.84926294542194e-07	\\
18517.2088623047	6.46348096791653e-06	\\
18523.6015319824	6.81403375666036e-06	\\
18549.0039825439	4.16031153781468e-08	\\
18553.0414581299	1.05113248161974e-06	\\
18582.3131561279	6.92766092490355e-06	\\
18605.6968688965	7.03216235855266e-08	\\
18620.6691741943	1.97381076354024e-05	\\
18621.1738586426	1.90786659502198e-05	\\
18655.1559448242	7.68937232484095e-05	\\
18663.0626678467	7.97999219023481e-05	\\
18689.4744873047	4.60087768426035e-05	\\
18689.9791717529	4.61641275349937e-05	\\
18720.092010498	3.33645003029177e-07	\\
18723.7930297852	3.82062112726162e-06	\\
18757.9433441162	6.38421992640402e-05	\\
18757.9433441162	6.38421992640402e-05	\\
18777.4578094482	8.30813291515133e-05	\\
18792.7665710449	6.98358909216751e-05	\\
18825.9075164795	9.393945685177e-06	\\
18826.5804290771	9.76772519962672e-06	\\
18835.4965209961	2.5761942979365e-07	\\
18875.0301361084	7.60980448917666e-08	\\
18893.5352325439	8.76930138637239e-06	\\
18911.3674163818	4.719348398742e-07	\\
18929.5360565186	2.66496829449985e-05	\\
18929.5360565186	2.66496829449985e-05	\\
18963.6863708496	8.44870066408127e-05	\\
18968.9014434814	8.59281741401351e-05	\\
18997.6684570313	4.72017716048067e-05	\\
18998.0049133301	4.76508263327522e-05	\\
19026.0990142822	3.98973152412023e-07	\\
19032.49168396	7.88130156771638e-06	\\
19066.3055419922	7.4207329625517e-05	\\
19082.7919006348	8.94254315911899e-05	\\
19100.9605407715	7.21846790800012e-05	\\
19101.6334533691	7.24309888191787e-05	\\
19135.279083252	5.01180244629999e-06	\\
19140.998840332	3.53431263028196e-07	\\
19158.1581115723	7.91323532435738e-06	\\
19169.5976257324	4.83577696910091e-06	\\
19180.8689117432	1.30563147911645e-07	\\
19208.4583282471	1.65365630358544e-07	\\
19238.0664825439	5.96631541480844e-06	\\
19254.0481567383	1.34206077986734e-06	\\
19272.3850250244	2.46533296748342e-05	\\
19273.0579376221	2.45284771168051e-05	\\
19306.1988830566	8.01387434185447e-05	\\
19312.423324585	8.21550719347928e-05	\\
19340.6856536865	4.80094579190329e-05	\\
19341.0221099854	4.8495658595149e-05	\\
19371.1349487305	1.19033293958395e-07	\\
19381.7333221436	5.94011672161388e-06	\\
19407.3040008545	3.06969589478184e-07	\\
19424.7997283936	4.20134065914325e-06	\\
19442.9683685303	1.41619301406833e-07	\\
19459.2864990234	1.92969584064552e-06	\\
19472.912979126	3.60271532817424e-08	\\
19496.6331481934	1.08939397175361e-07	\\
19510.7643127441	2.7166533107301e-06	\\
19515.979385376	2.76809174745948e-06	\\
19545.7557678223	6.11695175625885e-07	\\
19547.6062774658	1.96586591946667e-06	\\
19554.5036315918	6.37632269553643e-08	\\
19583.9435577393	2.85126481381481e-07	\\
19613.7199401855	1.27611493217834e-06	\\
19615.5704498291	1.41387848616523e-07	\\
19617.5891876221	1.34143105609163e-06	\\
19661.1602783203	4.0267978332846e-07	\\
19683.5346221924	1.12501321286899e-06	\\
19692.4507141113	1.11191801830544e-06	\\
19694.1329956055	3.17521030174844e-07	\\
19723.5729217529	4.32425065714452e-07	\\
19739.2181396484	1.03255468213812e-06	\\
19775.8918762207	1.0354126671194e-06	\\
19777.9106140137	4.4470850866028e-07	\\
19800.789642334	4.43578616928047e-07	\\
19816.2666320801	1.01944667796173e-06	\\
19824.341583252	4.41205519421623e-07	\\
19847.0523834229	1.00495708276415e-06	\\
19862.6976013184	1.01838511980161e-06	\\
19878.5110473633	5.07625387099132e-07	\\
19900.5489349365	4.77029027277636e-07	\\
19915.0165557861	1.02667372276704e-06	\\
19939.4096374512	9.97653762698062e-07	\\
19940.0825500488	4.34094689371151e-07	\\
19972.3823547363	5.15859432376865e-07	\\
19973.3917236328	1.00381112994084e-06	\\
20021.5049743652	9.7735625526025e-07	\\
20023.860168457	5.23770231151e-07	\\
20039.0007019043	1.09409137156133e-06	\\
20059.8609924316	4.95498119281038e-07	\\
20089.3009185791	4.9712166256526e-07	\\
20094.6842193604	1.08648489240423e-06	\\
20108.3106994629	1.06255149882862e-06	\\
20111.0023498535	4.66380178477212e-07	\\
20141.9563293457	1.0774897621047e-06	\\
20146.3302612305	4.40353041215308e-07	\\
20176.611328125	1.03995131379698e-06	\\
20181.8264007568	4.58777122602412e-07	\\
20225.0610351563	4.61738246123749e-07	\\
20229.0985107422	1.03970543035293e-06	\\
20255.5103302002	1.10121636847112e-06	\\
20261.0618591309	4.72946606053985e-07	\\
20267.2863006592	4.75954953781087e-07	\\
20290.1653289795	1.10137131133224e-06	\\
20321.2875366211	4.52460965992448e-07	\\
20323.9791870117	1.00554600211952e-06	\\
20354.5967102051	1.02612860049564e-06	\\
20355.4378509521	4.37005486754494e-07	\\
20378.9897918701	1.01285796891329e-06	\\
20395.6443786621	4.27134377608219e-07	\\
20429.7946929932	4.11545948192514e-07	\\
20431.4769744873	1.19333248721544e-06	\\
20442.4118041992	1.0540549317408e-06	\\
20454.6924591064	4.53954189882139e-07	\\
20480.094909668	1.17263448329724e-06	\\
20484.4688415527	3.98697781552868e-07	\\
20518.6191558838	1.18617843231509e-06	\\
20528.5446166992	2.37417638789621e-07	\\
20548.9002227783	3.79096158314052e-07	\\
20549.9095916748	1.09323281038587e-06	\\
20602.5650024414	3.59823285333485e-07	\\
20603.7425994873	1.13943723352514e-06	\\
20612.6586914063	4.26394915735622e-07	\\
20632.3413848877	1.15024404113564e-06	\\
20647.6501464844	1.08639403768919e-06	\\
20673.2208251953	3.71676582547696e-07	\\
20693.2399749756	1.160785253809e-06	\\
20704.6794891357	2.54828086695183e-07	\\
20737.1475219727	3.98588248633647e-07	\\
20743.7084197998	1.15433688008473e-06	\\
20759.1854095459	3.48292055283613e-07	\\
20770.4566955566	1.16132306572014e-06	\\
20792.9992675781	1.18773154371669e-06	\\
20807.2986602783	9.96299546842038e-08	\\
20817.728805542	1.19383114644322e-06	\\
20834.8880767822	2.94015106447154e-07	\\
20857.5988769531	4.01618008549469e-07	\\
20861.8045806885	1.19290681591152e-06	\\
20884.5153808594	3.50278701699084e-07	\\
20898.8147735596	1.1830650971281e-06	\\
20923.0396270752	1.29152715484378e-06	\\
20942.8905487061	3.36118535267786e-07	\\
20968.461227417	1.24793451528407e-06	\\
20970.3117370605	2.23559655781982e-07	\\
20998.2376098633	1.15800697352975e-06	\\
21004.1255950928	2.29203736852796e-07	\\
21021.6213226318	1.72468831266187e-07	\\
21046.6873168945	1.30221224201692e-06	\\
21080.3329467773	2.50384490259944e-07	\\
21089.0808105469	1.19680188420697e-06	\\
21095.641708374	2.3263129111913e-07	\\
21123.9040374756	1.40231201145232e-06	\\
21147.7924346924	1.4412297976037e-06	\\
21149.3064880371	3.26712121363274e-07	\\
21162.7647399902	1.31068381410171e-06	\\
21190.5223846436	2.24827341644588e-07	\\
21211.7191314697	1.67518195409364e-07	\\
21220.4669952393	1.32334278327343e-06	\\
21250.0751495361	1.41334647257786e-06	\\
21260.168838501	2.9952987281822e-07	\\
21268.2437896729	1.29245691870063e-06	\\
21285.5712890625	6.65704203851653e-08	\\
21298.1884002686	1.41121025283652e-06	\\
21322.4132537842	2.26440513096235e-07	\\
21354.8812866211	1.55218951080671e-06	\\
21357.7411651611	5.53994632231354e-08	\\
21376.7509460449	7.86710388598469e-08	\\
21379.1061401367	1.55187973935677e-06	\\
21413.5929107666	8.84827266621658e-08	\\
21418.8079833984	1.55126595983658e-06	\\
21448.2479095459	9.81985223453768e-08	\\
21460.3603363037	1.46595627692842e-06	\\
21468.0988311768	1.20614886822884e-07	\\
21480.2112579346	1.60185881163176e-06	\\
21508.6418151855	1.75690911181127e-06	\\
21524.9599456787	4.49642272042721e-08	\\
21556.9232940674	1.69681684963418e-07	\\
21567.3534393311	1.47385241269805e-06	\\
21578.288269043	2.46357998380326e-08	\\
21594.4381713867	1.60885661204966e-06	\\
21615.9713745117	2.14601249243035e-06	\\
21619.3359375	5.99572376868416e-08	\\
21642.7196502686	2.18588339543504e-06	\\
21653.9909362793	8.38784058489788e-08	\\
21685.7860565186	1.85692531014768e-06	\\
21694.7021484375	8.48649476553272e-09	\\
21717.5811767578	1.97409883763779e-06	\\
21725.8243560791	2.65969927511044e-08	\\
21742.1424865723	2.06469195586274e-06	\\
21742.6471710205	7.54269072897109e-08	\\
21792.7791595459	2.07971640731417e-06	\\
21797.9942321777	7.23115585112475e-08	\\
21818.3498382568	1.02686680609586e-07	\\
21843.7522888184	2.03193100183317e-06	\\
21868.1453704834	9.64118497962041e-08	\\
21876.5567779541	1.9769206827692e-06	\\
21892.2019958496	2.47664270227691e-06	\\
21910.0341796875	7.10753565434576e-08	\\
21944.1844940186	2.23988413241724e-06	\\
21944.352722168	2.43341164898447e-08	\\
21947.7172851563	1.6732376977069e-07	\\
21948.3901977539	2.24256747345921e-06	\\
21994.3164825439	3.85045718869016e-08	\\
21997.8492736816	2.32982513625368e-06	\\
22050	9.18129069660042e-07	\\
};
\addlegendentry{Y};

\end{axis}
\end{tikzpicture}%    	
    	\caption{The plots with $a=0.95$}
    	\label{fig:a0.95-plots}
\end{figure}
\begin{figure}[H]
	\centering
    	\setlength\figureheight{3.5cm}
    	\setlength\figurewidth{0.4\linewidth}
    	% This file was created by matlab2tikz v0.4.6 running on MATLAB 8.3.
% Copyright (c) 2008--2014, Nico Schlömer <nico.schloemer@gmail.com>
% All rights reserved.
% Minimal pgfplots version: 1.3
% 
% The latest updates can be retrieved from
%   http://www.mathworks.com/matlabcentral/fileexchange/22022-matlab2tikz
% where you can also make suggestions and rate matlab2tikz.
% 
\begin{tikzpicture}

\begin{axis}[%
width=\figurewidth,
height=\figureheight,
scale only axis,
xmin=0,
xmax=22050,
xlabel={Frequency (Hz)},
ymin=0,
ymax=0.397631788311866,
ylabel={|IMPULSE(f)|},
name=plot3,
title={Single-Sided Amplitude Spectrum of imp(t)},
legend style={draw=black,fill=white,legend cell align=left}
]
\addplot [color=blue,solid]
  table[row sep=crcr]{
0	0.02	\\
344.53125	0.02	\\
689.0625	0.02	\\
1033.59375	0.02	\\
1378.125	0.02	\\
1722.65625	0.02	\\
2067.1875	0.02	\\
2411.71875	0.02	\\
2756.25	0.02	\\
3100.78125	0.02	\\
3445.3125	0.02	\\
3789.84375	0.02	\\
4134.375	0.02	\\
4478.90625	0.02	\\
4823.4375	0.02	\\
5167.96875	0.02	\\
5512.5	0.02	\\
5857.03125	0.02	\\
6201.5625	0.02	\\
6546.09375	0.02	\\
6890.625	0.02	\\
7235.15625	0.02	\\
7579.6875	0.02	\\
7924.21875	0.02	\\
8268.75	0.02	\\
8613.28125	0.02	\\
8957.8125	0.02	\\
9302.34375	0.02	\\
9646.875	0.02	\\
9991.40625	0.02	\\
10335.9375	0.02	\\
10680.46875	0.02	\\
11025	0.02	\\
11369.53125	0.02	\\
11714.0625	0.02	\\
12058.59375	0.02	\\
12403.125	0.02	\\
12747.65625	0.02	\\
13092.1875	0.02	\\
13436.71875	0.02	\\
13781.25	0.02	\\
14125.78125	0.02	\\
14470.3125	0.02	\\
14814.84375	0.02	\\
15159.375	0.02	\\
15503.90625	0.02	\\
15848.4375	0.02	\\
16192.96875	0.02	\\
16537.5	0.02	\\
16882.03125	0.02	\\
17226.5625	0.02	\\
17571.09375	0.02	\\
17915.625	0.02	\\
18260.15625	0.02	\\
18604.6875	0.02	\\
18949.21875	0.02	\\
19293.75	0.02	\\
19638.28125	0.02	\\
19982.8125	0.02	\\
20327.34375	0.02	\\
20671.875	0.02	\\
21016.40625	0.02	\\
21360.9375	0.02	\\
21705.46875	0.02	\\
22050	0.02	\\
};
\addlegendentry{IMP};

\end{axis}

\begin{axis}[%
width=\figurewidth,
height=\figureheight,
scale only axis,
xmin=0,
xmax=0.00226757369614512,
xlabel={t (s)},
ymin=0,
ymax=1,
ylabel={impulse},
name=plot1,
at=(plot3.above north west),
anchor=below south west,
title={impulse(t)},
legend style={draw=black,fill=white,legend cell align=left}
]
\addplot [color=blue,solid]
  table[row sep=crcr]{
0	1	\\
2.29047848095467e-05	0	\\
4.58095696190934e-05	0	\\
6.87143544286401e-05	0	\\
9.16191392381869e-05	0	\\
0.000114523924047734	0	\\
0.00013742870885728	0	\\
0.000160333493666827	0	\\
0.000183238278476374	0	\\
0.00020614306328592	0	\\
0.000229047848095467	0	\\
0.000251952632905014	0	\\
0.000274857417714561	0	\\
0.000297762202524107	0	\\
0.000320666987333654	0	\\
0.000343571772143201	0	\\
0.000366476556952747	0	\\
0.000389381341762294	0	\\
0.000412286126571841	0	\\
0.000435190911381388	0	\\
0.000458095696190934	0	\\
0.000481000481000481	0	\\
0.000503905265810028	0	\\
0.000526810050619574	0	\\
0.000549714835429121	0	\\
0.000572619620238668	0	\\
0.000595524405048215	0	\\
0.000618429189857761	0	\\
0.000641333974667308	0	\\
0.000664238759476855	0	\\
0.000687143544286402	0	\\
0.000710048329095948	0	\\
0.000732953113905495	0	\\
0.000755857898715042	0	\\
0.000778762683524588	0	\\
0.000801667468334135	0	\\
0.000824572253143682	0	\\
0.000847477037953228	0	\\
0.000870381822762775	0	\\
0.000893286607572322	0	\\
0.000916191392381869	0	\\
0.000939096177191415	0	\\
0.000962000962000962	0	\\
0.000984905746810509	0	\\
0.00100781053162006	0	\\
0.0010307153164296	0	\\
0.00105362010123915	0	\\
0.0010765248860487	0	\\
0.00109942967085824	0	\\
0.00112233445566779	0	\\
0.00114523924047734	0	\\
0.00116814402528688	0	\\
0.00119104881009643	0	\\
0.00121395359490598	0	\\
0.00123685837971552	0	\\
0.00125976316452507	0	\\
0.00128266794933462	0	\\
0.00130557273414416	0	\\
0.00132847751895371	0	\\
0.00135138230376326	0	\\
0.0013742870885728	0	\\
0.00139719187338235	0	\\
0.0014200966581919	0	\\
0.00144300144300144	0	\\
0.00146590622781099	0	\\
0.00148881101262054	0	\\
0.00151171579743008	0	\\
0.00153462058223963	0	\\
0.00155752536704918	0	\\
0.00158043015185872	0	\\
0.00160333493666827	0	\\
0.00162623972147782	0	\\
0.00164914450628736	0	\\
0.00167204929109691	0	\\
0.00169495407590646	0	\\
0.001717858860716	0	\\
0.00174076364552555	0	\\
0.0017636684303351	0	\\
0.00178657321514464	0	\\
0.00180947799995419	0	\\
0.00183238278476374	0	\\
0.00185528756957328	0	\\
0.00187819235438283	0	\\
0.00190109713919238	0	\\
0.00192400192400192	0	\\
0.00194690670881147	0	\\
0.00196981149362102	0	\\
0.00199271627843056	0	\\
0.00201562106324011	0	\\
0.00203852584804966	0	\\
0.0020614306328592	0	\\
0.00208433541766875	0	\\
0.0021072402024783	0	\\
0.00213014498728784	0	\\
0.00215304977209739	0	\\
0.00217595455690694	0	\\
0.00219885934171648	0	\\
0.00222176412652603	0	\\
0.00224466891133558	0	\\
0.00226757369614512	0	\\
};
\addlegendentry{impulse};

\end{axis}

\begin{axis}[%
width=\figurewidth,
height=\figureheight,
scale only axis,
xmin=0,
xmax=0.00226757369614512,
xlabel={t (s)},
ymin=0,
ymax=1,
ylabel={h},
name=plot2,
at=(plot1.right of south east),
anchor=left of south west,
title={h(t)},
legend style={draw=black,fill=white,legend cell align=left}
]
\addplot [color=blue,solid]
  table[row sep=crcr]{
0	1	\\
2.29047848095467e-05	0.95	\\
4.58095696190934e-05	0.9025	\\
6.87143544286401e-05	0.857375	\\
9.16191392381869e-05	0.81450625	\\
0.000114523924047734	0.7737809375	\\
0.00013742870885728	0.735091890625	\\
0.000160333493666827	0.69833729609375	\\
0.000183238278476374	0.663420431289062	\\
0.00020614306328592	0.630249409724609	\\
0.000229047848095467	0.598736939238379	\\
0.000251952632905014	0.56880009227646	\\
0.000274857417714561	0.540360087662637	\\
0.000297762202524107	0.513342083279505	\\
0.000320666987333654	0.487674979115529	\\
0.000343571772143201	0.463291230159753	\\
0.000366476556952747	0.440126668651765	\\
0.000389381341762294	0.418120335219177	\\
0.000412286126571841	0.397214318458218	\\
0.000435190911381388	0.377353602535307	\\
0.000458095696190934	0.358485922408542	\\
0.000481000481000481	0.340561626288115	\\
0.000503905265810028	0.323533544973709	\\
0.000526810050619574	0.307356867725023	\\
0.000549714835429121	0.291989024338772	\\
0.000572619620238668	0.277389573121834	\\
0.000595524405048215	0.263520094465742	\\
0.000618429189857761	0.250344089742455	\\
0.000641333974667308	0.237826885255332	\\
0.000664238759476855	0.225935540992565	\\
0.000687143544286402	0.214638763942937	\\
0.000710048329095948	0.20390682574579	\\
0.000732953113905495	0.193711484458501	\\
0.000755857898715042	0.184025910235576	\\
0.000778762683524588	0.174824614723797	\\
0.000801667468334135	0.166083383987607	\\
0.000824572253143682	0.157779214788227	\\
0.000847477037953228	0.149890254048815	\\
0.000870381822762775	0.142395741346375	\\
0.000893286607572322	0.135275954279056	\\
0.000916191392381869	0.128512156565103	\\
0.000939096177191415	0.122086548736848	\\
0.000962000962000962	0.115982221300006	\\
0.000984905746810509	0.110183110235005	\\
0.00100781053162006	0.104673954723255	\\
0.0010307153164296	0.0994402569870922	\\
0.00105362010123915	0.0944682441377376	\\
0.0010765248860487	0.0897448319308507	\\
0.00109942967085824	0.0852575903343082	\\
0.00112233445566779	0.0809947108175928	\\
0.00114523924047734	0.0769449752767131	\\
0.00116814402528688	0.0730977265128775	\\
0.00119104881009643	0.0694428401872336	\\
0.00121395359490598	0.0659706981778719	\\
0.00123685837971552	0.0626721632689783	\\
0.00125976316452507	0.0595385551055294	\\
0.00128266794933462	0.0565616273502529	\\
0.00130557273414416	0.0537335459827403	\\
0.00132847751895371	0.0510468686836033	\\
0.00135138230376326	0.0484945252494231	\\
0.0013742870885728	0.0460697989869519	\\
0.00139719187338235	0.0437663090376043	\\
0.0014200966581919	0.0415779935857241	\\
0.00144300144300144	0.0394990939064379	\\
0.00146590622781099	0.037524139211116	\\
0.00148881101262054	0.0356479322505602	\\
0.00151171579743008	0.0338655356380322	\\
0.00153462058223963	0.0321722588561306	\\
0.00155752536704918	0.0305636459133241	\\
0.00158043015185872	0.0290354636176579	\\
0.00160333493666827	0.027583690436775	\\
0.00162623972147782	0.0262045059149362	\\
0.00164914450628736	0.0248942806191894	\\
0.00167204929109691	0.0236495665882299	\\
0.00169495407590646	0.0224670882588184	\\
0.001717858860716	0.0213437338458775	\\
0.00174076364552555	0.0202765471535836	\\
0.0017636684303351	0.0192627197959044	\\
0.00178657321514464	0.0182995838061092	\\
0.00180947799995419	0.0173846046158038	\\
0.00183238278476374	0.0165153743850136	\\
0.00185528756957328	0.0156896056657629	\\
0.00187819235438283	0.0149051253824748	\\
0.00190109713919238	0.014159869113351	\\
0.00192400192400192	0.0134518756576835	\\
0.00194690670881147	0.0127792818747993	\\
0.00196981149362102	0.0121403177810593	\\
0.00199271627843056	0.0115333018920064	\\
0.00201562106324011	0.010956636797406	\\
0.00203852584804966	0.0104088049575357	\\
0.0020614306328592	0.00988836470965895	\\
0.00208433541766875	0.009393946474176	\\
0.0021072402024783	0.0089242491504672	\\
0.00213014498728784	0.00847803669294384	\\
0.00215304977209739	0.00805413485829665	\\
0.00217595455690694	0.00765142811538181	\\
0.00219885934171648	0.00726885670961272	\\
0.00222176412652603	0.00690541387413209	\\
0.00224466891133558	0.00656014318042548	\\
0.00226757369614512	0.00623213602140421	\\
};
\addlegendentry{h};

\end{axis}

\begin{axis}[%
width=\figurewidth,
height=\figureheight,
scale only axis,
xmin=0,
xmax=22050,
xlabel={Frequency (Hz)},
ymin=0,
ymax=0.397631788311866,
ylabel={|H(f)|},
name=plot4,
at=(plot2.below south west),
anchor=above north west,
title={Single-Sided Amplitude Spectrum of h(t)},
legend style={draw=black,fill=white,legend cell align=left}
]
\addplot [color=blue,solid]
  table[row sep=crcr]{
0	0.397631788311866	\\
344.53125	0.288688597925754	\\
689.0625	0.186318281696946	\\
1033.59375	0.13212595358944	\\
1378.125	0.100840550278553	\\
1722.65625	0.0816295699416506	\\
2067.1875	0.0690352850227611	\\
2411.71875	0.0596925515142661	\\
2756.25	0.0521421134813846	\\
3100.78125	0.0462390333004053	\\
3445.3125	0.0418969513891136	\\
3789.84375	0.0384795835479073	\\
4134.375	0.0353543971677623	\\
4478.90625	0.032491991898594	\\
4823.4375	0.0302005075785837	\\
5167.96875	0.0284689084487852	\\
5512.5	0.0269084624960711	\\
5857.03125	0.0252967619770434	\\
6201.5625	0.0238223772717253	\\
6546.09375	0.0227075263729579	\\
6890.625	0.0218236713777593	\\
7235.15625	0.0208895334934134	\\
7579.6875	0.0198870269364149	\\
7924.21875	0.019044031593775	\\
8268.75	0.0184477961418328	\\
8613.28125	0.0179037805783351	\\
8957.8125	0.0172463908217395	\\
9302.34375	0.0165798899857881	\\
9646.875	0.016091880901799	\\
9991.40625	0.0157473281460711	\\
10335.9375	0.0153499749923172	\\
10680.46875	0.0148519526687185	\\
11025	0.0144141410652619	\\
11369.53125	0.0141411483832134	\\
11714.0625	0.0139142157802257	\\
12058.59375	0.0135864156097986	\\
12403.125	0.0132097851648812	\\
12747.65625	0.0129434640638808	\\
13092.1875	0.0127961236774446	\\
13436.71875	0.0126155621282175	\\
13781.25	0.0123336858872925	\\
14125.78125	0.012067791482453	\\
14470.3125	0.0119293186329924	\\
14814.84375	0.0118446408928542	\\
15159.375	0.0116773128272677	\\
15503.90625	0.0114438752641003	\\
15848.4375	0.0112828611051312	\\
16192.96875	0.0112313025095185	\\
16537.5	0.0111665551828533	\\
16882.03125	0.0110053660550897	\\
17226.5625	0.010833169425015	\\
17571.09375	0.010765801705408	\\
17915.625	0.0107625582779568	\\
18260.15625	0.0106940771871071	\\
18604.6875	0.0105492633439215	\\
18949.21875	0.0104506193076926	\\
19293.75	0.0104573924752851	\\
19638.28125	0.0104700148553275	\\
19982.8125	0.010392206761738	\\
20327.34375	0.0102832602374374	\\
20671.875	0.0102629489357736	\\
21016.40625	0.0103182094518941	\\
21360.9375	0.0103249662150964	\\
21705.46875	0.0102478212500977	\\
22050	0.0101956868797915	\\
};
\addlegendentry{H};

\end{axis}

\begin{axis}[%
width=\figurewidth,
height=\figureheight,
scale only axis,
xmin=0,
xmax=1,
xlabel={t (s)},
ymin=-6.53222006299869,
ymax=6.99720337275687,
ylabel={y},
name=plot6,
at=(plot4.below south west),
anchor=above north west,
title={y(t)},
legend style={draw=black,fill=white,legend cell align=left}
]
\addplot [color=blue,solid]
  table[row sep=crcr]{
0	0	\\
0	0	\\
0	0	\\
0.00156466133018889	0	\\
0.00156466133018889	0	\\
0.00312932266037779	0	\\
0.00312932266037779	0	\\
0.00467130773940452	0	\\
0.00467130773940452	0	\\
0.00623596906959341	0	\\
0.00623596906959341	0	\\
0.00777795414862015	0	\\
0.00777795414862015	0	\\
0.00934261547880904	0	\\
0.00934261547880904	0	\\
0.0109072768089979	0	\\
0.0109072768089979	0	\\
0.0124492618880247	0	\\
0.0124492618880247	0	\\
0.0140139232182136	0	\\
0.0140139232182136	0	\\
0.0155559082972403	0	\\
0.0155559082972403	0	\\
0.0171205696274292	0	\\
0.0171205696274292	0	\\
0.0186625547064559	0	\\
0.0186625547064559	0	\\
0.0202272160366448	0	\\
0.0202272160366448	0	\\
0.0217918773668337	0	\\
0.0217918773668337	0	\\
0.0233338624458605	0	\\
0.0233338624458605	0	\\
0.0248985237760493	0	\\
0.0248985237760493	0	\\
0.0264405088550761	0	\\
0.0264405088550761	0	\\
0.028005170185265	0	\\
0.028005170185265	0	\\
0.0295698315154539	0	\\
0.0295698315154539	0	\\
0.0311118165944806	0	\\
0.0311118165944806	0	\\
0.0326764779246695	0	\\
0.0326764779246695	0	\\
0.0342184630036962	0	\\
0.0342184630036962	0	\\
0.0357831243338851	0	\\
0.0357831243338851	0	\\
0.0373251094129119	0	\\
0.0373251094129119	0	\\
0.0388897707431008	0	\\
0.0388897707431008	0	\\
0.0404544320732896	0	\\
0.0404544320732896	0	\\
0.0419964171523164	0	\\
0.0419964171523164	0	\\
0.0447402435429375	-0.000224952749285368	\\
0.045103063561532	0.000260931499974259	\\
0.045103063561532	0.000260931499974259	\\
0.0458967323522075	0.00205175905272759	\\
0.0470078686591533	-7.6592580545527e-05	\\
0.0480056237102882	0.00341732170829188	\\
0.0486405587428286	-0.00287498044465705	\\
0.0495022562869906	0.00322825115953176	\\
0.0503412775799905	-0.00408871219859267	\\
0.0512029751241525	0.00257128976102746	\\
0.0514070613846119	-0.000497972327891939	\\
0.0527222839520171	0.0120590411131693	\\
0.0529943989659629	0.0115114054174856	\\
0.0544456790403411	-0.129180997915564	\\
0.0544456790403411	-0.129180997915564	\\
0.0559876641193678	-0.404628644580496	\\
0.0559876641193678	-0.404628644580496	\\
0.0575523254495567	-0.843363597317562	\\
0.0581419079797728	-1.03815968610916	\\
0.0591169867797456	-0.0496313934185652	\\
0.0591169867797456	-0.0496313934185652	\\
0.0603641805936643	2.70061054382341	\\
0.0606589718587723	2.50334408475504	\\
0.0615660219052586	0.895062564473676	\\
0.0628358919703395	3.36446284798397	\\
0.0637656182679879	1.16685168632709	\\
0.0637656182679879	1.16685168632709	\\
0.0645139345563392	-0.449327721993541	\\
0.0653302795981768	0.0242002011113315	\\
0.0668949409283657	-1.87633452561567	\\
0.0668949409283657	-1.87633452561567	\\
0.0684369260073925	-4.47015504794335	\\
0.0684369260073925	-4.47015504794335	\\
0.0700015873375814	-6.53064796525947	\\
0.0700242635887435	-6.53222006299869	\\
0.0715435724166081	-1.8683872961162	\\
0.0715435724166081	-1.8683872961162	\\
0.073108233746797	6.99720337275687	\\
0.073108233746797	6.99720337275687	\\
0.0746502188258237	4.54772168228532	\\
0.0746502188258237	4.54772168228532	\\
0.0755345926211479	2.93228173175284	\\
0.0762148801560126	3.25952026195282	\\
0.0777795414862015	0.875987429299268	\\
0.0779836277466609	0.892645330996429	\\
0.0792534978117418	-0.945971192987262	\\
0.0793215265652282	-0.779895246630027	\\
0.0808861878954171	2.15588715040966	\\
0.0814077416721468	3.4033481980693	\\
0.0824281729744439	-0.396135120584102	\\
0.0824281729744439	-0.396135120584102	\\
0.083743395541849	-5.90834218324445	\\
0.0841515680627679	-6.33936546110927	\\
0.0853987618766865	-2.20144957879627	\\
0.0855574956348217	-2.39150064083157	\\
0.0870994807138484	2.22742335662565	\\
0.0870994807138484	2.22742335662565	\\
0.088618789541713	-2.24209361668641	\\
0.0895258395881993	-2.552731008087	\\
0.090206127123064	-1.54606634923773	\\
0.0909090909090909	-2.68963147996774	\\
0.0917707884532529	-0.895233274450608	\\
0.0917707884532529	-0.895233274450608	\\
0.0932900972811175	2.50532598525912	\\
0.0934941835415769	2.32350978714701	\\
0.0948547586113064	5.58331266865583	\\
0.0948774348624685	5.5636764470158	\\
0.0964420961926574	-0.0769871440535383	\\
0.0964420961926574	-0.0769871440535383	\\
0.09709970747636	-1.15779173607392	\\
0.0980521100251706	-0.389427987879944	\\
0.0991178938297921	0.427020260853626	\\
0.0996621238576838	-0.184700773910363	\\
0.100070296378603	0.440605206368721	\\
0.1010907276809	0.297344897180579	\\
0.101634957708792	0.909177835536514	\\
0.102950180276197	0.607827509337691	\\
0.104220050341278	-0.730201866931079	\\
0.104220050341278	-0.730201866931079	\\
0.105648654164494	-3.50915170089049	\\
0.105762035420304	-3.34791166518406	\\
0.107258667997007	-0.0424178085118573	\\
0.107780221773736	-0.166780390297911	\\
0.108619243066736	-1.54161711484305	\\
0.10886868182952	-1.33490487018097	\\
0.110274609401574	-0.0800443450487859	\\
0.110433343159709	-0.253650380587622	\\
0.111408421959682	1.69832833233804	\\
0.111975328238736	0.535400797239981	\\
0.112564910768952	-1.39250400157352	\\
0.113539989568924	-0.889805745662112	\\
0.114877888387492	2.72738350890432	\\
0.115195355903762	1.80447756152152	\\
0.116306492210708	2.58494602590824	\\
0.116828045987437	2.55853040007958	\\
0.118165944806005	0.117061832577876	\\
0.118483412322275	-0.639250173398335	\\
0.119707929885031	0.340259721763941	\\
0.119934692396653	0.672960876198388	\\
0.121317943717545	-1.27469766081495	\\
0.122383727522166	-1.54113695739247	\\
0.122655842536112	-1.06069394513649	\\
0.123086691308193	-1.44314025002287	\\
0.124175151363977	-0.416981889630499	\\
0.124560647633733	-0.226645556240157	\\
0.125989251456949	-1.01122052345645	\\
0.127077711512733	-1.36188117759361	\\
0.127531236535976	-0.798879883883899	\\
0.127689970294111	-0.942741868535236	\\
0.128846459103381	0.00614352699904236	\\
0.129095897866165	-0.156484831686293	\\
0.130637882945191	0.645232009133048	\\
0.131499580489354	0.258997404912908	\\
0.131794371754462	0.716221355413349	\\
0.132315925531191	0.392082048690818	\\
0.133721853103245	1.99041450951021	\\
0.133767205605569	1.95835050123436	\\
0.135309190684596	0.742585911586183	\\
0.135467924442731	0.989302448120481	\\
0.136533708247353	0.0696796902193538	\\
0.137644844554298	0.542194386275592	\\
0.138347808340325	-0.636282467734808	\\
0.138846685865893	-1.04162870339654	\\
0.139413592144947	-0.33391512968515	\\
0.140025850926325	-0.555517282995685	\\
0.140978253475135	1.32584851447959	\\
0.141545159754189	0.411584509786881	\\
0.143087144833216	-1.29019934958375	\\
0.143087144833216	-1.29019934958375	\\
0.144651806163405	-2.11015783576641	\\
0.144651806163405	-2.11015783576641	\\
0.145944352479648	0.376621397773061	\\
0.146261819995918	0.153731220123427	\\
0.147282251298215	1.10947832427116	\\
0.147781128823783	0.863332192377531	\\
0.149300437651647	-0.66998033035388	\\
0.149300437651647	-0.66998033035388	\\
0.150865098981836	-1.34704095430632	\\
0.151001156488809	-1.52434322395531	\\
0.152316379056214	-0.650273142577256	\\
0.152429760312025	-0.650497628442838	\\
0.153971745391052	2.11376654049318	\\
0.154107802898025	2.16827022637793	\\
0.155536406721241	1.16682032058369	\\
0.155649787977052	1.10338733872781	\\
0.156806276786322	2.01977066472617	\\
0.157078391800268	1.81558480727255	\\
0.158643053130456	0.147071752010728	\\
0.158643053130456	0.147071752010728	\\
0.160207714460645	-1.14787083896379	\\
0.16088800199551	-1.41673036802402	\\
0.161704347037348	-1.02105649374736	\\
0.162339282069888	-1.24363166667249	\\
0.163314360869861	-0.529798373830564	\\
0.163314360869861	-0.529798373830564	\\
0.164652259688428	0.199554506056049	\\
0.16571804349305	0.230165383576199	\\
0.166421007279077	-0.320491245138736	\\
0.166421007279077	-0.320491245138736	\\
0.167781582348806	-2.40842777538137	\\
0.168189754869725	-2.42782082863823	\\
0.169527653688292	0.763330860755922	\\
0.169527653688292	0.763330860755922	\\
0.170593437492914	2.85076931065628	\\
0.171092315018481	2.40062532456638	\\
0.172634300097508	-0.448613540879222	\\
0.17297444386494	-0.568438942539548	\\
0.174153608925372	0.529673726913944	\\
0.174584457697453	0.537045428296165	\\
0.175242068981156	0.0488777972124587	\\
0.176239824032291	0.184385949490582	\\
0.177305607836912	0.927669697085139	\\
0.177645751604345	0.95209681275817	\\
0.178824916664777	-0.510447124928451	\\
0.178870269167101	-0.506795049874157	\\
0.180366901743804	0.435719678416409	\\
0.180729721762398	0.585255958841245	\\
0.181976915576317	-1.12588433190749	\\
0.182362411846074	-1.40597837233844	\\
0.183518900655344	-0.950804390898796	\\
0.183518900655344	-0.950804390898796	\\
0.185083561985533	-0.151595075906556	\\
0.185083561985533	-0.151595075906556	\\
0.185922583278532	1.08221164562396	\\
0.186693575818046	0.708360211904914	\\
0.187827388376154	-0.152119452109729	\\
0.18821288464591	-0.0153094331045757	\\
0.189754869724937	-0.949139223350192	\\
0.189754869724937	-0.949139223350192	\\
0.191228826050477	-1.15820205994217	\\
0.192045171092315	-1.35298894891633	\\
0.192861516134153	-0.875668789866161	\\
0.192861516134153	-0.875668789866161	\\
0.194403501213179	1.503979774936	\\
0.194902378738747	2.00779772077758	\\
0.195968162543368	1.23768964503827	\\
0.196353658813125	1.0229026714955	\\
0.197396766366584	1.66395071897373	\\
0.197532823873557	1.5589843222066	\\
0.199074808952584	0.533863669375413	\\
0.199301571464206	0.588780307065792	\\
0.200639470282773	0.366435650500292	\\
0.200639470282773	0.366435650500292	\\
0.202045397854827	-0.59931555235153	\\
0.202226807864124	-0.560614669134054	\\
0.203746116691988	-1.25841327306906	\\
0.204018231705934	-1.14229664229971	\\
0.205106691761718	-1.64092258149281	\\
0.205809655547745	-1.6308566147178	\\
0.206852763101204	-0.950050533511201	\\
0.206852763101204	-0.950050533511201	\\
0.207691784394204	-0.438687883973614	\\
0.208576158189528	-0.656194100303348	\\
0.209483208236014	-0.0131506146135902	\\
0.21091181205923	-0.266457160217854	\\
0.211524070840609	0.090927153738125	\\
0.211524070840609	0.090927153738125	\\
0.212612530896392	1.27284954479636	\\
0.213066055919635	1.24277762179662	\\
0.213632962198689	1.52671276702975	\\
0.215174947277716	1.59197463666698	\\
0.216195378580013	0.938014689599503	\\
0.216422141091635	1.03395142255457	\\
0.217714687407878	0.53247969439455	\\
0.217782716161364	0.544264584001452	\\
0.219302024989229	-1.10930122261436	\\
0.219392729993877	-1.16531771843764	\\
0.220753305063607	-0.172454441991167	\\
0.221093448831039	-0.132202302989891	\\
0.222408671398444	-0.44845224039865	\\
0.223134311435633	-0.38664577952441	\\
0.223814598970498	-0.53220841165466	\\
0.223950656477471	-0.508156877405685	\\
0.225469965305336	-1.30027878972958	\\
0.22583278532393	-1.38572562366456	\\
0.226943921630876	-0.459709754821044	\\
0.227079979137849	-0.49302258248739	\\
0.228100410440146	-0.0656275519256314	\\
0.2286673167192	-0.257216068691294	\\
0.229460985509876	0.114937751942923	\\
0.230934941835416	-0.482525685788183	\\
0.231728610626091	0.193304632455014	\\
0.231728610626091	0.193304632455014	\\
0.232499603165605	0.493527620540571	\\
0.233542710719064	0.212163988235078	\\
0.234857933286469	1.35251216949683	\\
0.234857933286469	1.35251216949683	\\
0.235334134560874	1.64931305751244	\\
0.236467947118982	1.42215334770397	\\
0.237352320914306	1.67638865147382	\\
0.237964579695685	1.48207578546702	\\
0.239506564774711	-0.838508266817109	\\
0.239506564774711	-0.838508266817109	\\
0.240345586067711	-1.17253230925565	\\
0.24127531236536	-0.932933709944699	\\
0.242613211183927	-1.39411864392445	\\
0.242635887435089	-1.39506305830539	\\
0.244177872514116	-0.497851803203167	\\
0.245175627565251	-0.158295955685453	\\
0.245742533844305	-0.564763689556324	\\
0.245833238848954	-0.610218600474974	\\
0.247148461416359	-0.0823846641384334	\\
0.247284518923332	-0.0850509645446131	\\
0.24884918025352	0.326734088567126	\\
0.24884918025352	0.326734088567126	\\
0.250187079072088	0.924781658193279	\\
0.250391165332547	0.777224880387528	\\
0.251955826662736	-0.287576114677397	\\
0.252137236672033	-0.299952731179114	\\
0.253520487992925	0.993513781781581	\\
0.254246128030114	1.50807542122321	\\
0.255062473071952	0.891445826205429	\\
0.255062473071952	0.891445826205429	\\
0.256627134402141	-0.602471462825565	\\
0.2568312206626	-0.607389771270986	\\
0.258146443230005	-0.117717043571537	\\
0.258169119481167	-0.123519780511381	\\
0.259438989546248	-0.819622622244432	\\
0.259733780811356	-0.763684281206503	\\
0.261275765890383	0.028035835575663	\\
0.261842672169437	0.268341584298304	\\
0.262840427220572	-0.475230325701957	\\
0.263225923490329	-0.724229956544857	\\
0.264405088550761	0.49067347977999	\\
0.264427764801923	0.491714316708058	\\
0.265947073629788	-0.372324939240263	\\
0.266491303657679	-0.576591030296663	\\
0.267489058708814	-0.144915781123326	\\
0.268123993741355	-0.446971025431032	\\
0.269053720039003	0.0136058194408365	\\
0.269053720039003	0.0136058194408365	\\
0.270436971359895	1.28204366133788	\\
0.270618381369192	1.25880868626051	\\
0.272183042699381	0.0795819930171316	\\
0.272817977731921	-0.450908623698391	\\
0.273725027778408	0.113302911576929	\\
0.273725027778408	0.113302911576929	\\
0.275130955350461	0.309884851182625	\\
0.275289689108597	0.273145903757581	\\
0.276400825415542	-0.338858916298059	\\
0.276831674187623	-0.209312120503969	\\
0.278396335517812	0.200569105136226	\\
0.278419011768974	0.204335016149718	\\
0.279915644345677	-0.690710051313313	\\
0.279938320596839	-0.688981947335112	\\
0.281502981927028	0.0504258143172085	\\
0.282137916959568	0.236864297405529	\\
0.283067643257217	-0.199222438805024	\\
0.283430463275811	-0.149123061543567	\\
0.284609628336243	-0.54176811443711	\\
0.284768362094379	-0.568376632637397	\\
0.286174289666432	0.033802423447569	\\
0.286174289666432	0.033802423447569	\\
0.287172044717567	0.753184975974324	\\
0.287716274745459	0.307800692963558	\\
0.289031497312864	-0.739821981236572	\\
0.289280936075648	-0.679815608463589	\\
0.290845597405837	0.342520704104011	\\
0.290845597405837	0.342520704104011	\\
0.291843352456972	0.892321378099558	\\
0.292387582484864	0.6208458125632	\\
0.293952243815052	0.0665478271606949	\\
0.293952243815052	0.0665478271606949	\\
0.295494228894079	-0.468589595074545	\\
0.295879725163836	-0.621106534127894	\\
0.297013537721944	-0.0234651965820966	\\
0.297058890224268	-0.0245511611347715	\\
0.298600875303295	0.459343247348317	\\
0.299190457833511	0.655280708217918	\\
0.300165536633484	0.343597027131966	\\
0.300165536633484	0.343597027131966	\\
0.301730197963673	-0.0528311392408174	\\
0.301730197963673	-0.0528311392408174	\\
0.302705276763645	-0.521180917649192	\\
0.30338556429851	-0.273095869806212	\\
0.304247261842672	-0.438678872589108	\\
0.305358398149618	-0.323410304037839	\\
0.306378829451915	0.157906165154461	\\
0.306378829451915	0.157906165154461	\\
0.307263203247239	0.432702629542907	\\
0.307943490782104	0.212998956315446	\\
0.309508152112293	-1.20093008648409	\\
0.309825619628563	-1.38345546559116	\\
0.31105013719132	1.58877594759474	\\
0.311390280958752	2.67059711878071	\\
0.312614798521508	-0.481524095416741	\\
0.312886913535454	-0.458953656549762	\\
0.314088754847049	-2.76272017561283	\\
0.314156783600535	-2.7402889809745	\\
0.315721444930724	1.4725975556648	\\
0.316356379963264	3.06467296610981	\\
0.317263430009751	-0.255329818352309	\\
0.317263430009751	-0.255329818352309	\\
0.318170480056237	-2.43215734352704	\\
0.318918796344588	-1.74442328387552	\\
0.320392752670129	1.97168350401647	\\
0.320732896437561	2.14373543235449	\\
0.321844032744507	-0.300613737963569	\\
0.32209347150729	0.354130449454177	\\
0.323181931563074	-1.15929990080087	\\
0.323544751581669	-1.03864491031946	\\
0.324587859135128	1.87308242935836	\\
0.325064060409533	1.51177991758468	\\
0.326424635479263	-0.83505914263499	\\
0.326651397990884	-0.499507997388452	\\
0.328034649311776	-1.19661913330604	\\
0.328170706818749	-0.813742759200223	\\
0.329032404362911	0.0180231161847545	\\
0.330324950679154	0.202857805004537	\\
0.331277353227964	-1.92722096805349	\\
0.331322705730289	-1.96434667979706	\\
0.332819338306991	-0.403757566645319	\\
0.333272863330234	0.861676821471964	\\
0.334315970883694	-0.633934521394994	\\
0.33438399963718	-0.36079536936984	\\
0.335472459692964	2.49026041381914	\\
0.335925984716207	1.87986315031932	\\
0.337309236037098	-1.58486327735069	\\
0.337490646046396	-1.16277937622145	\\
0.33900995487426	-0.0565021201123657	\\
0.33932742239053	0.0619420777083381	\\
0.340053062427719	-0.957450912949923	\\
0.340597292455611	-0.562985878567113	\\
0.342025896278827	4.30949503520867	\\
0.3421619537858	4.21534164068416	\\
0.343703938864827	-0.295599045726546	\\
0.343703938864827	-0.295599045726546	\\
0.345268600195016	-4.24414126506692	\\
0.34531395269734	-4.29236330399115	\\
0.346515794008934	-3.42612385029696	\\
0.346833261525205	-3.47423295321838	\\
0.348375246604231	1.668052169027	\\
0.348375246604231	1.668052169027	\\
0.349735821673961	4.20590853131792	\\
0.350098641692555	4.31362655892087	\\
0.351481893013447	-1.76958276985707	\\
0.351481893013447	-1.76958276985707	\\
0.352842468083176	-4.50775044727094	\\
0.353159935599447	-4.04471868275766	\\
0.354588539422663	0.623685562527243	\\
0.354701920678473	0.421669252985842	\\
0.355676999478446	2.88006826825073	\\
0.356221229506338	2.02978727913267	\\
0.357468423320257	-0.27430837185123	\\
0.358058005850473	0.445951001424704	\\
0.359237170910905	-1.94046270153044	\\
0.359259847162067	-1.92492398854159	\\
0.360257602213202	-1.3374634364659	\\
0.360824508492256	-1.79750988258087	\\
0.362366493571283	2.57500759991087	\\
0.362956076101499	4.26620631446102	\\
0.363931154901472	1.71057181998042	\\
0.363931154901472	1.71057181998042	\\
0.365495816231661	-1.92410669110555	\\
0.365881312501417	-1.89082709592852	\\
0.366924420054877	-2.31419227082016	\\
0.367037801310687	-2.01778640451608	\\
0.368602462640876	0.125854278081442	\\
0.368602462640876	0.125854278081442	\\
0.36916936891993	0.918580724386649	\\
0.37084741150593	1.51886615807354	\\
0.371709109050092	0.165152036746951	\\
0.371709109050092	0.165152036746951	\\
0.373137712873308	-3.00644414141056	\\
0.373455180389578	-2.98416306044526	\\
0.374815755459307	-1.43585012570992	\\
0.374815755459307	-1.43585012570992	\\
0.376244359282523	2.11066296961552	\\
0.376425769291821	1.95233334174942	\\
0.377536905598766	2.9719061847241	\\
0.377922401868523	2.75611509259728	\\
0.379419034445226	0.355548187087362	\\
0.379532415701036	0.437507760247632	\\
0.380484818249847	-1.16400579669802	\\
0.381029048277739	-0.798871964664263	\\
0.382434975849793	0.0988408140997059	\\
0.382593709607928	-0.0909601322394751	\\
0.38325132089163	1.00544731358562	\\
0.384657248463684	1.09677736336017	\\
0.385700356017143	0.193123134370525	\\
0.385836413524116	0.201769825263311	\\
0.387174312342684	-2.28397070242121	\\
0.387491779858954	-2.58594246326703	\\
0.388807002426359	-1.2946552135445	\\
0.388807002426359	-1.2946552135445	\\
0.389940814984467	0.535535718970354	\\
0.390779836277467	0.619391140186676	\\
0.391913648835575	-0.408424327702043	\\
0.392752670128574	-0.538701994396947	\\
0.393047461393682	-0.115674007247703	\\
0.393614367672736	-0.562316590951915	\\
0.394612122723871	0.650046034710746	\\
0.395292410258736	0.167357626315158	\\
0.39585931653779	-0.309145235952073	\\
0.396584956574979	0.117755554298093	\\
0.39785482664006	1.12811222080015	\\
0.39869384793306	1.47511234050024	\\
0.399555545477222	-0.148325206559434	\\
0.399691602984195	-0.0363973859796967	\\
0.400666681784167	-0.90084485627365	\\
0.401256264314384	-0.419867608591688	\\
0.40279824939341	1.09022512636957	\\
0.403410508174789	1.67895479948877	\\
0.404317558221275	0.876651502722991	\\
0.404408263225923	0.90494903368686	\\
0.405043198258464	-0.0473592291520768	\\
0.405927572053788	0.299799895516703	\\
0.407356175877004	-0.94706566762003	\\
0.407741672146761	-0.85746457403039	\\
0.408558017188598	-1.28772699323052	\\
0.409034218463004	-1.09819343027868	\\
0.41057620354203	0.0743635329176802	\\
0.410666908546679	0.175943996384635	\\
0.412118188621057	-0.904462363652549	\\
0.412549037393138	-0.668044961691746	\\
0.4134107349373	-1.24522665028976	\\
0.413841583709381	-1.0986761880989	\\
0.414930043765165	-0.0192819259505958	\\
0.41540624503957	-0.134392155819031	\\
0.416154561327921	0.415591929839205	\\
0.416948230118597	-0.188810645344487	\\
0.418195423932515	0.823778467239676	\\
0.418354157690651	0.808058919263007	\\
0.419692056509218	1.81265845785965	\\
0.420168257783623	2.03154122917507	\\
0.421483480351028	1.00053606070823	\\
0.421732919113812	1.05966192006575	\\
0.423025465430055	-0.0227598648496725	\\
0.423025465430055	-0.0227598648496725	\\
0.423977867978866	-0.520054068427471	\\
0.424590126760244	-0.45803590271872	\\
0.425746615569514	0.141658336240083	\\
0.426630989364838	0.0971417153370195	\\
0.42769677316946	-1.56988931712674	\\
0.428104945690378	-1.84553201613515	\\
0.429238758248486	-1.38558214269998	\\
0.429374815755459	-1.53374614682109	\\
0.430803419578675	-0.439654055120835	\\
0.430803419578675	-0.439654055120835	\\
0.432368080908864	0.30208794379437	\\
0.432481462164675	0.260009761060065	\\
0.433320483457675	1.26918572490051	\\
0.434499648518107	1.05506032796321	\\
0.43547472731808	0.224404227714295	\\
0.43547472731808	0.224404227714295	\\
0.437016712397106	-1.00710597936542	\\
0.437039388648269	-1.00933407310664	\\
0.438581373727295	0.983524489720989	\\
0.438581373727295	0.983524489720989	\\
0.439375042517971	1.94624400979818	\\
0.440146035057484	1.13502532892957	\\
0.441007732601646	0.17521530479462	\\
0.441688020136511	0.395213524204072	\\
0.443207328964376	-0.47366887144959	\\
0.443366062722511	-0.36501657857763	\\
0.444363817773646	-0.996697255706522	\\
0.444817342796889	-0.849189785752381	\\
0.446359327875916	0.206198628123149	\\
0.447221025420078	0.64989388977309	\\
0.447901312954942	0.200991743708427	\\
0.447901312954942	0.200991743708427	\\
0.449465974285131	-0.852981777520951	\\
0.449715413047915	-0.999878835453897	\\
0.450690491847888	-0.550139093109123	\\
0.45103063561532	-0.570809733771577	\\
0.452572620694347	-0.316462190507054	\\
0.452912764461779	-0.562131075787725	\\
0.454137282024536	0.0715430550938834	\\
0.454137282024536	0.0715430550938834	\\
0.455135037075671	0.311199647327591	\\
0.455679267103562	0.111444025183085	\\
0.456903784666319	-0.356944320693589	\\
0.457243928433751	-0.0320963907652751	\\
0.458785913512778	1.16907681153458	\\
0.459420848545319	1.46869154907855	\\
0.460350574842967	1.09904460269198	\\
0.460894804870859	1.22713009609747	\\
0.461915236173156	0.987264105714222	\\
0.461915236173156	0.987264105714222	\\
0.463457221252183	0.0362486329655453	\\
0.463457221252183	0.0362486329655453	\\
0.464863148824236	-0.902269694882146	\\
0.465021882582371	-0.745098409710544	\\
0.465974285131182	-0.250259796729443	\\
0.466677248917209	-0.353745908817725	\\
0.468128528991587	-1.2901831298834	\\
0.468627406517155	-1.22197210679405	\\
0.469511780312479	-1.43241612346965	\\
0.469897276582235	-1.40877830387475	\\
0.471235175400803	-0.630904709404487	\\
0.471235175400803	-0.630904709404487	\\
0.472799836730992	0.232809314898186	\\
0.472799836730992	0.232809314898186	\\
0.474341821810018	1.17839209713305	\\
0.474341821810018	1.17839209713305	\\
0.475906483140207	0.274804648048451	\\
0.475906483140207	0.274804648048451	\\
0.477131000702964	-0.409480182124426	\\
0.477471144470396	-0.159612517775282	\\
0.479013129549423	0.938128747144929	\\
0.479738769586612	1.206832144704	\\
0.480509762126125	0.863473612779583	\\
0.480577790879612	0.864565359960091	\\
0.482097099707476	0.534706072924312	\\
0.482165128460963	0.549515354670463	\\
0.483684437288827	-0.742837903534397	\\
0.483684437288827	-0.742837903534397	\\
0.484727544842287	-1.3592815468748	\\
0.485226422367854	-1.09926857988158	\\
0.486791083698043	0.285247437465225	\\
0.487131227465475	0.517800194840278	\\
0.488355745028232	-0.677631017462051	\\
0.488446450032881	-0.667010969178013	\\
0.489376176330529	-1.00864864980022	\\
0.49012449261888	-1.04135044248951	\\
0.491462391437448	-0.216586264096134	\\
0.491462391437448	-0.216586264096134	\\
0.493004376516474	0.615335199783839	\\
0.493321844032744	0.837569232757851	\\
0.494569037846663	0.479574134583602	\\
0.495045239121069	0.834624175299608	\\
0.496133699176852	0.179838426734905	\\
0.496133699176852	0.179838426734905	\\
0.497539626748906	-0.724824256100706	\\
0.497675684255879	-0.634434459052347	\\
0.499240345586068	0.513464666761708	\\
0.499240345586068	0.513464666761708	\\
0.50073697816277	1.21374099477687	\\
0.500782330665094	1.20352985450043	\\
0.502346991995283	0.454860746234016	\\
0.502346991995283	0.454860746234016	\\
0.503820948320824	-0.401498298386679	\\
0.503911653325472	-0.3891040266725	\\
0.505136170888229	-0.887485133296903	\\
0.505453638404499	-0.806229708214381	\\
0.506927594730039	-0.352380440381188	\\
0.507494501009093	-0.294421087500027	\\
0.508560284813715	-0.650765832305071	\\
0.508582961064877	-0.652530345308892	\\
0.510124946143903	-0.105826772683523	\\
0.510646499920633	-0.0734553364369257	\\
0.51166693122293	-0.27513853900325	\\
0.51166693122293	-0.27513853900325	\\
0.512483276264768	-0.106782896537075	\\
0.513299621306606	-0.260261448194026	\\
0.514796253883308	0.0990105637093787	\\
0.514796253883308	0.0990105637093787	\\
0.515748656432119	0.56911126811281	\\
0.517086555250686	0.368791516831314	\\
0.517902900292524	0.643019697507932	\\
0.517902900292524	0.643019697507932	\\
0.519104741604118	1.14301417926452	\\
0.51964897163201	1.1591507711285	\\
0.521009546701739	0.172744522797377	\\
0.521009546701739	0.172744522797377	\\
0.522392798022631	-0.199388586911395	\\
0.522551531780766	-0.169298134825069	\\
0.524116193110955	-0.421715624768418	\\
0.524410984376063	-0.402139816807369	\\
0.525567473185333	-0.785246711411976	\\
0.525839588199279	-0.858965023014189	\\
0.527222839520171	-0.227486834097448	\\
0.527358897027143	-0.19789374834153	\\
0.528787500850359	-1.31650883502784	\\
0.529082292115467	-1.35451408699447	\\
0.530329485929386	-0.282295091913438	\\
0.530329485929386	-0.282295091913438	\\
0.531894147259575	0.401968509578573	\\
0.532733168552575	0.521033112541175	\\
0.533322751082791	0.358991769107926	\\
0.534275153631602	0.56152867302817	\\
0.534796707408331	0.392433528114425	\\
0.535953196217601	0.70561487254996	\\
0.53656545499898	0.401940203530539	\\
0.53656545499898	0.401940203530539	\\
0.537676591305925	-0.123491061232838	\\
0.538107440078006	0.0425353330516756	\\
0.539014490124493	0.655949776481323	\\
0.540216331436087	0.64804666224041	\\
0.541214086487222	0.00223648492317213	\\
0.541214086487222	0.00223648492317213	\\
0.542778747817411	-0.678653216486314	\\
0.542914805324384	-0.706882001645509	\\
0.54382185537087	-0.38922202103403	\\
0.544638200412708	-0.702020412272755	\\
0.545885394226626	-0.463530179920109	\\
0.545885394226626	-0.463530179920109	\\
0.547450055556815	0.10154070061744	\\
0.547835551826572	0.0941629951510233	\\
0.548538515612599	0.240504610336824	\\
0.548992040635842	0.107926708523527	\\
0.550556701966031	-0.716579916021185	\\
0.550987550738112	-0.858609382751827	\\
0.55212136329622	-0.185175290550419	\\
0.55212136329622	-0.185175290550419	\\
0.553663348375247	0.441647211208783	\\
0.55379940588222	0.432093607926936	\\
0.555228009705436	0.649065452298571	\\
0.55527336220776	0.655545358349032	\\
0.556769994784462	0.243784288950044	\\
0.556769994784462	0.243784288950044	\\
0.557745073584435	-0.0517255158985563	\\
0.558561418626273	0.00261950422289247	\\
0.559287058663462	-0.0881035240018099	\\
0.559876641193678	-0.0213937825864238	\\
0.561441302523867	0.536395108335736	\\
0.562393705072677	0.684339859519601	\\
0.562824553844758	0.468705873292232	\\
0.563119345109866	0.515127211647815	\\
0.564547948933082	-0.325290595016757	\\
0.564547948933082	-0.325290595016757	\\
0.566112610263271	-1.32923351312094	\\
0.566180639016758	-1.36333303621969	\\
0.567654595342298	0.106347923570024	\\
0.568130796616703	0.267333748035027	\\
0.569219256672487	-1.25633863140932	\\
0.569219256672487	-1.25633863140932	\\
0.57019433547246	-1.78803966389113	\\
0.570806594253838	-1.3324821422499	\\
0.571917730560784	0.0435714184730695	\\
0.572325903081703	-0.128971800751159	\\
0.573867888160729	1.89098612161773	\\
0.574865643211864	2.04961049543426	\\
0.575432549490918	1.689124961213	\\
0.575432549490918	1.689124961213	\\
0.576997210821107	-0.0959497936012409	\\
0.576997210821107	-0.0959497936012409	\\
0.578471167146647	-3.83232066121023	\\
0.578539195900134	-3.75880113390967	\\
0.580058504727998	4.72528063057926	\\
0.580103857230323	4.71202357064841	\\
0.581668518560512	-0.0559762380749383	\\
0.581668518560512	-0.0559762380749383	\\
0.582802331118619	-3.1786176203732	\\
0.583346561146511	-2.82631398867183	\\
0.584729812467403	0.873718710277414	\\
0.585296718746457	1.74939767422181	\\
0.586317150048754	-3.21548812800514	\\
0.586407855053403	-3.37690105027298	\\
0.587881811378943	0.50802888597206	\\
0.587881811378943	0.50802888597206	\\
0.588426041406835	2.93625187289038	\\
0.589446472709132	1.2865413826228	\\
0.590784371527699	-1.49587318476822	\\
0.59121522029978	-1.71232403140118	\\
0.592553119118347	-0.0185184525693587	\\
0.592553119118347	-0.0185184525693587	\\
0.593233406653212	1.92812370515097	\\
0.594231161704347	1.43228803101059	\\
0.595478355518266	-0.0993574713015142	\\
0.595886528039185	0.717540912418254	\\
0.596476110569401	-0.581386657940107	\\
0.597269779360076	-0.388129972367697	\\
0.598403591918184	0.564577110020312	\\
0.598902469443752	0.613331728913998	\\
0.599809519490238	-1.39255234297842	\\
0.600603188280913	-1.27225431141335	\\
0.601873058345994	1.96491304216204	\\
0.601895734597156	2.00674735028095	\\
0.603415043425021	-0.573131207636342	\\
0.60350574842967	-0.459081066545317	\\
0.60497970475521	-1.3208044160587	\\
0.605002381006372	-1.32343083965016	\\
0.606544366085399	0.461212690194597	\\
0.606544366085399	0.461212690194597	\\
0.608063674913263	-1.46345494509395	\\
0.608267761173723	-1.72831381050387	\\
0.609651012494614	0.90620700277532	\\
0.610331300029479	2.03774789405886	\\
0.611215673824803	0.228851586038508	\\
0.611215673824803	0.228851586038508	\\
0.61275765890383	-2.88546846364644	\\
0.612780335154992	-2.89729722094508	\\
0.614322320234019	0.399544699787798	\\
0.614322320234019	0.399544699787798	\\
0.615456132792127	2.93898274569173	\\
0.616589945350235	3.01518460492116	\\
0.617428966643235	1.10161430868645	\\
0.618154606680424	1.23560032206125	\\
0.618993627973423	-0.000329630171428223	\\
0.618993627973423	-0.000329630171428223	\\
0.62053561305245	-4.68487386088496	\\
0.620558289303612	-4.68955957364916	\\
0.622077598131477	2.9815922642096	\\
0.622553799405882	3.54008274362933	\\
0.623642259461666	1.57552705408994	\\
0.623642259461666	1.57552705408994	\\
0.625206920791855	-2.85220893505897	\\
0.625456359554638	-3.80577467345569	\\
0.626771582122044	-0.351238024348937	\\
0.62736116465226	0.369062982254444	\\
0.628222862196422	-2.07299072861615	\\
0.628404272205719	-2.56835792178954	\\
0.629878228531259	1.66926558454594	\\
0.629878228531259	1.66926558454594	\\
0.630762602326583	3.85699692284413	\\
0.631420213610286	2.35771207650527	\\
0.632984874940475	-1.82388670873225	\\
0.633415723712556	-2.64817493202877	\\
0.634526860019502	-0.286742340353942	\\
0.634526860019502	-0.286742340353942	\\
0.635864758838069	3.15934949629104	\\
0.63609152134969	2.66829378068086	\\
0.637542801424069	-0.127723783760849	\\
0.637656182679879	0.0404268552902303	\\
0.638653937731014	-1.61496804009377	\\
0.639198167758906	-1.06468387145116	\\
0.640445361572825	0.159648891312113	\\
0.640762829089095	0.0341956698026487	\\
0.641669879135581	-1.65356066548926	\\
0.642304814168122	-1.39476849798566	\\
0.643778770493662	2.30604429573792	\\
0.644232295516905	2.6818203739576	\\
0.645411460577337	-0.682639953311419	\\
0.645502165581986	-0.584234765274602	\\
0.646976121907526	-2.91800193860121	\\
0.647157531916823	-2.99930626954469	\\
0.648472754484229	1.85967953638213	\\
0.648540783237715	1.80640683569159	\\
0.649606567042337	-0.892235815716406	\\
0.650082768316742	0.212433053694942	\\
0.651397990884147	2.72283237684329	\\
0.651647429646931	2.5292304989234	\\
0.653030680967822	-1.38621676386175	\\
0.653733644753849	-0.88093315840972	\\
0.654754076056146	-2.36933681956989	\\
0.654844781060795	-2.5609490022511	\\
0.656318737386335	1.2705925096408	\\
0.656318737386335	1.2705925096408	\\
0.657339168688632	4.73181001373367	\\
0.657860722465362	4.4328623205978	\\
0.659425383795551	0.191924596337386	\\
0.659425383795551	0.191924596337386	\\
0.660967368874578	-3.80340507206536	\\
0.661058073879226	-4.01201779200158	\\
0.662532030204767	-1.69945321691412	\\
0.662532030204767	-1.69945321691412	\\
0.664096691534955	2.16850040012477	\\
0.664595569060523	1.8534228322482	\\
0.665230504093063	2.73783431519803	\\
0.665638676613982	2.21382868937472	\\
0.667112632939522	-1.840168005307	\\
0.667770244223225	-3.15165363408622	\\
0.668745323023198	-1.64911668996015	\\
0.66928955305109	-1.76163162786073	\\
0.670309984353387	1.04617851054943	\\
0.670309984353387	1.04617851054943	\\
0.671489149413819	3.0330416120245	\\
0.671851969432413	2.34733787430042	\\
0.672622961971927	1.25640990896644	\\
0.673643393274224	2.27456646369605	\\
0.674981292092791	-1.15718180766135	\\
0.674981292092791	-1.15718180766135	\\
0.675979047143926	-2.86148420215064	\\
0.676523277171818	-1.52916637621406	\\
0.678087938502007	3.179622010499	\\
0.678337377264791	3.50537463814838	\\
0.679629923581034	0.933628121008878	\\
0.679629923581034	0.933628121008878	\\
0.680695707385655	-2.68706197132513	\\
0.681489376176331	-2.12907319116099	\\
0.682577836232114	-0.782596821912957	\\
0.683983763804168	-0.890383937501348	\\
0.684301231320438	-0.043506117806346	\\
0.684301231320438	-0.043506117806346	\\
0.685344338873897	2.82964868333459	\\
0.685865892650627	1.51817741229155	\\
0.687407877729654	-0.912778525775605	\\
0.687407877729654	-0.912778525775605	\\
0.688519014036599	-3.79683612049791	\\
0.688972539059843	-2.55180505492884	\\
0.690491847887707	2.58879454985954	\\
0.690514524138869	2.56873316022667	\\
0.691943127962085	0.00727577331440485	\\
0.692464681738815	-0.166652285762794	\\
0.693439760538788	1.29391420565985	\\
0.693643846799247	1.07351816112268	\\
0.695185831878274	-0.0655838990046337	\\
0.695185831878274	-0.0655838990046337	\\
0.696659788203814	-2.28318856014083	\\
0.696999931971247	-2.83474579619188	\\
0.69829247828749	1.75532624409702	\\
0.69829247828749	1.75532624409702	\\
0.699834463366516	2.67549384218125	\\
0.699857139617678	2.66086341095552	\\
0.701421800947867	-0.839697583178082	\\
0.7017619447153	-1.78710971153677	\\
0.702805052268759	-0.574961629308397	\\
0.703417311050137	-0.966590203500806	\\
0.704142951087326	-0.000140074124297947	\\
0.705186058640786	-1.12584473100437	\\
0.706002403682623	0.581445014568488	\\
0.70607043243611	0.571865599570917	\\
0.707204244994218	1.50126134140602	\\
0.707884532529082	0.834491663799748	\\
0.709131726343001	-1.72810331838493	\\
0.709585251366244	-1.77827340997671	\\
0.710741740175514	-0.121400472931168	\\
0.710900473933649	-0.175337343248312	\\
0.712306401505703	1.46108985952576	\\
0.712329077756865	1.48236807645796	\\
0.71384838658473	0.1582143279496	\\
0.713893739087054	0.114197164513797	\\
0.715413047914919	1.10101295431655	\\
0.715413047914919	1.10101295431655	\\
0.716955032993945	-0.290885021767572	\\
0.71711376675208	-0.0550549943833214	\\
0.717998140547405	-1.29755119517843	\\
0.718610399328783	-0.551651148519833	\\
0.720061679403161	1.61265543489223	\\
0.720832671942674	1.9993921752241	\\
0.72162634073335	-0.266436468550513	\\
0.72162634073335	-0.266436468550513	\\
0.722556067030998	-2.87052056069559	\\
0.723372412072836	-2.91861516368039	\\
0.724596929635593	0.10220280717553	\\
0.72561736093789	-0.507614407706181	\\
0.726116238463457	0.349265404037926	\\
0.726297648472754	0.135419706691522	\\
0.727431461030862	2.97224812470706	\\
0.727839633551781	2.34331608865554	\\
0.729358942379646	0.0101053629273042	\\
0.729426971133132	0.0458093087173012	\\
0.730923603709835	-3.38490358840439	\\
0.731535862491213	-3.44933569632411	\\
0.732510941291186	-0.733605722079084	\\
0.732510941291186	-0.733605722079084	\\
0.734052926370212	3.21146211164524	\\
0.735277443932969	3.34121794139314	\\
0.735572235198077	2.22512715901895	\\
0.735617587700401	2.2335076227385	\\
0.73686478151432	-2.22702744636158	\\
0.73718224903059	-1.81244557624954	\\
0.738043946574752	0.0562039815457024	\\
0.739200435384022	-2.21445692965403	\\
0.739926075421211	-0.530259124611677	\\
0.740288895439806	-0.598618281821888	\\
0.741853556769995	2.58682544593468	\\
0.742579196807184	2.65047954103358	\\
0.743395541849022	0.782120035166109	\\
0.743395541849022	0.782120035166109	\\
0.744733440667589	-3.92868926082564	\\
0.74496020317921	-3.28981529713834	\\
0.746502188258237	-0.34368752329844	\\
0.746502188258237	-0.34368752329844	\\
0.747363885802399	2.26471189838198	\\
0.748066849588426	2.12998697502443	\\
0.749631510918615	-0.293832926832419	\\
0.75010771219302	-1.18783853410297	\\
0.750969409737182	0.174040483670702	\\
0.751173495997642	-0.0141078055605206	\\
0.752738157327831	-1.53478560357028	\\
0.752806186081317	-1.74387396661242	\\
0.754189437402209	0.685023227950358	\\
0.754280142406857	0.640071203328006	\\
0.755413954964965	2.34613159490506	\\
0.755844803737046	1.96986414035049	\\
0.757409465067235	-0.440009767426124	\\
0.757409465067235	-0.440009767426124	\\
0.75872468763464	-1.5130498585422	\\
0.75904215515091	-0.939090751154864	\\
0.760516111476451	0.461238363201783	\\
0.761105694006667	0.433863147941765	\\
0.761400485271775	1.08073209102738	\\
0.762511621578721	0.591379454431637	\\
0.763577405383342	-0.508870299639705	\\
0.763668110387991	-0.372199740406856	\\
0.765096714211207	-1.70809775635349	\\
0.765187419215855	-1.6189685786681	\\
0.765777001746071	-2.84335486721417	\\
0.766729404294882	-1.88029527703608	\\
0.76818068436926	2.45335152344474	\\
0.768997029411098	2.56623801618042	\\
0.769836050704098	0.45381249389088	\\
0.769836050704098	0.45381249389088	\\
0.771310007029638	-1.31127361895105	\\
0.771740855801719	-1.99212842777133	\\
0.772897344610989	-0.275766697377124	\\
0.773078754620286	-0.370617057699406	\\
0.774507358443502	1.36000364064822	\\
0.774847502210934	1.15447989567697	\\
0.775777228508583	1.91702561094351	\\
0.77627610603415	1.63667029237494	\\
0.776774983559718	0.759426179703292	\\
0.777795414862015	1.06439154937455	\\
0.779133313680582	-1.47404345422885	\\
0.779564162452663	-1.71407781157538	\\
0.780720651261933	-1.08667785575664	\\
0.780720651261933	-1.08667785575664	\\
0.782285312592122	0.670194918461516	\\
0.782285312592122	0.670194918461516	\\
0.783351096396744	1.69103231463919	\\
0.783827297671149	0.706205793031839	\\
0.785278577745527	-1.80720828956856	\\
0.785459987754824	-1.78071364195439	\\
0.786956620331527	-0.114893572151244	\\
0.786956620331527	-0.114893572151244	\\
0.787909022880337	1.22899094778547	\\
0.789156216694256	0.563943791321275	\\
0.79004059048958	-0.302164822118731	\\
0.79035805800585	-0.50597653770251	\\
0.791219755550012	0.324430474871502	\\
0.791605251819769	0.107813836261293	\\
0.793169913149958	-0.689654110680549	\\
0.793373999410417	-0.812020550620105	\\
0.794258373205742	0.702048218220014	\\
0.794825279484796	0.368124179918493	\\
0.795823034535931	0.994689546296888	\\
0.796707408331255	0.722727464860681	\\
0.797637134628903	-0.636849249507933	\\
0.797954602145173	-0.500695218262893	\\
0.799315177214903	1.21789713205467	\\
0.799677997233497	1.54960310519676	\\
0.800947867298578	-0.360662102579005	\\
0.800947867298578	-0.360662102579005	\\
0.801832241093902	-1.96284138675199	\\
0.802671262386902	-1.85212500794374	\\
0.804054513707794	0.919387460036625	\\
0.804054513707794	0.919387460036625	\\
0.805029592507767	1.96986617432576	\\
0.805755232544956	0.948446020821501	\\
0.807161160117009	-0.318998574037099	\\
0.807546656386766	0.0862562598762776	\\
0.808725821447198	-1.85806412680319	\\
0.808725821447198	-1.85806412680319	\\
0.809542166489036	-3.30058336147365	\\
0.810267806526225	-2.34135964310844	\\
0.811832467856414	2.81474275948957	\\
0.812558107893603	3.6413916439574	\\
0.813397129186603	2.41238857504421	\\
0.813397129186603	2.41238857504421	\\
0.81493911426563	-1.99691734886287	\\
0.814961790516792	-2.01914551921003	\\
0.816503775595819	-0.713054895382825	\\
0.816503775595819	-0.713054895382825	\\
0.818045760674845	0.891496419605334	\\
0.818045760674845	0.891496419605334	\\
0.819179573232953	3.04282199170597	\\
0.819610422005034	2.26943471317957	\\
0.820993673325926	-0.987104638563736	\\
0.821197759586385	-0.733839943417317	\\
0.82271706841425	-2.59014053797726	\\
0.822762420916574	-2.59527265826408	\\
0.824236377242114	0.490708910231632	\\
0.82482595977233	1.26394859904885	\\
0.825823714823465	-1.02969508358042	\\
0.82586906732579	-1.06326081471494	\\
0.827388376153654	0.188169156383657	\\
0.827909929930384	0.377659348381276	\\
0.828907684981519	-0.437978834956968	\\
0.829542620014059	0.0213727181815392	\\
0.830336288804735	-0.825107039413423	\\
0.830721785074491	-0.900004640756382	\\
0.831810245130275	0.760123730332662	\\
0.832513208916302	0.366915645827321	\\
0.833080115195356	0.941104816602018	\\
0.83364702147441	0.612887088257865	\\
0.834758157781356	-0.223631944451819	\\
0.835461121567383	0.0419492427061642	\\
0.836685639130139	0.770334003382816	\\
0.83743395541849	0.802466192327104	\\
0.83827297671149	0.0183450488726099	\\
0.83827297671149	0.0183450488726099	\\
0.83879453048822	-0.729222686730408	\\
0.839973695548652	-0.351167172969861	\\
0.841039479353273	1.15656747359559	\\
0.841379623120706	0.809850690272344	\\
0.842944284450895	-0.937651193395401	\\
0.843352456971813	-1.01025714977304	\\
0.844486269529921	-0.411701683553786	\\
0.844486269529921	-0.411701683553786	\\
0.845302614571759	0.532984070016689	\\
0.846118959613597	0.0540547777510471	\\
0.847094038413569	-0.772153227799641	\\
0.847683620943786	-0.454933936123155	\\
0.848658699743758	0.594004467677904	\\
0.849157577269326	0.226584531658751	\\
0.850654209846028	-1.07209008230848	\\
0.850722238599515	-1.01322526166975	\\
0.852196194925055	0.492885703375023	\\
0.852264223678541	0.49087731922751	\\
0.853012539966893	1.19549965445431	\\
0.854055647520352	0.978343041621492	\\
0.855348193836595	-0.667190362376579	\\
0.855370870087757	-0.665478537623562	\\
0.856935531417946	0.229604973352372	\\
0.856935531417946	0.229604973352372	\\
0.85813737272954	0.85898637231722	\\
0.859157804031838	1.01256130802118	\\
0.860019501575999	-0.0415669821178043	\\
0.860200911585297	0.352522840231914	\\
0.86160683915735	-0.206892360452145	\\
0.861901630422459	0.0682592401452779	\\
0.86296741422708	-1.33955986151457	\\
0.863148824236377	-1.24350625365755	\\
0.864169255538674	0.140158461486309	\\
0.86496292432935	0.516076102070673	\\
0.866255470645593	-1.09885254196416	\\
0.867094491938593	-1.07857343524363	\\
0.867661398217647	-1.63474097359996	\\
0.868387038254836	-1.72125555401441	\\
0.869384793305971	0.198270493128061	\\
0.869384793305971	0.198270493128061	\\
0.870926778384997	2.7945488367826	\\
0.870994807138484	2.81855840017856	\\
0.872491439715186	-0.426271783668877	\\
0.872922288487267	-0.34891094810392	\\
0.874010748543051	-1.6301326159513	\\
0.874396244812808	-1.6864222103424	\\
0.875416676115105	-0.239081477803209	\\
0.875666114877888	-0.907478612231167	\\
0.877140071203429	1.9919000092286	\\
0.877140071203429	1.9919000092286	\\
0.878387265017347	2.82362920625067	\\
0.878818113789428	2.79869179940478	\\
0.880269393863806	-1.69132783604187	\\
0.880269393863806	-1.69132783604187	\\
0.881471235175401	-2.85695599852425	\\
0.881811378942833	-2.27050764107313	\\
0.883194630263725	0.513360471863442	\\
0.883398716524184	0.169460190864145	\\
0.884555205333454	1.75190068357622	\\
0.885008730356697	1.31182505521727	\\
0.886414657928751	-0.96514076261823	\\
0.886482686682238	-0.946868876527112	\\
0.888047348012427	-0.373809634959589	\\
0.888977074310075	-0.2832056627669	\\
0.889430599333318	-0.818329767631799	\\
0.889589333091453	-0.791616496113268	\\
0.891153994421642	0.440896038240773	\\
0.891153994421642	0.440896038240773	\\
0.892015691965804	1.58987682847444	\\
0.892695979500669	0.574924533898821	\\
0.893943173314588	-1.45290265497796	\\
0.894260640830858	-1.29963167885626	\\
0.89543980589129	0.436499889166033	\\
0.895802625909885	0.371679782748214	\\
0.896823057212182	1.44519295248853	\\
0.89775278350983	1.13100165061936	\\
0.898750538560965	0.517254273002983	\\
0.898954624821425	0.65151407008688	\\
0.900292523639992	0.0299012822811332	\\
0.900768724914397	0.390406746586765	\\
0.902038594979478	-1.41673972952979	\\
0.902401414998072	-2.05580120267589	\\
0.903580580058505	-0.354054003732469	\\
0.903580580058505	-0.354054003732469	\\
0.90469171636545	1.71214261116554	\\
0.905145241388694	1.36684478831498	\\
0.906709902718883	-1.33280358269814	\\
0.907458219007234	-2.03656917874762	\\
0.908251887797909	-0.952294401353096	\\
0.908637384067666	-1.28225450755307	\\
0.909680491621125	-0.0746211660749929	\\
0.910587541667611	-1.31969977631624	\\
0.911358534207125	0.664276164578528	\\
0.911449239211773	0.606507899506636	\\
0.912741785528017	2.35357241178488	\\
0.913059253044287	2.03663337196303	\\
0.914465180616341	0.485524382782455	\\
0.914465180616341	0.485524382782455	\\
0.916029841946529	-2.69069710187159	\\
0.916120546951178	-2.78070268952437	\\
0.917594503276718	1.53336579146535	\\
0.917594503276718	1.53336579146535	\\
0.918025352048799	2.28465931727833	\\
0.919136488355745	1.84861667517359	\\
0.920565092178961	-1.02473328527091	\\
0.921358760969637	-1.25558846553273	\\
0.922016372253339	-0.289625072865905	\\
0.922243134764961	-0.664927483139288	\\
0.923490328578879	-1.67307721297125	\\
0.92380779609515	-1.59050586581003	\\
0.925372457425338	0.661260404631122	\\
0.926143449964852	1.3899786800901	\\
0.926914442504365	0.0802676430370062	\\
0.926982471257852	0.115794553767392	\\
0.928184312569446	-1.21602263742655	\\
0.929250096374067	-1.29975702594929	\\
0.930021088913581	-0.0847954704391513	\\
0.930021088913581	-0.0847954704391513	\\
0.930769405201932	1.21853007040635	\\
0.93158575024377	0.800133868183897	\\
0.932923649062337	-0.000655090145002903	\\
0.933127735322796	0.136446538109592	\\
0.933603936597202	-0.260558635872082	\\
0.935576770448309	0.22604152798796	\\
0.936257057983174	-0.191050888968721	\\
0.936687906755255	-0.368337014216617	\\
0.937776366811039	0.912184584398558	\\
0.938660740606363	1.3001399673769	\\
0.93936370439239	0.683831509109267	\\
0.939499761899363	0.738766882105996	\\
0.940905689471417	-1.25152749928593	\\
0.940928365722579	-1.2668953955881	\\
0.942402322048119	0.888472359200624	\\
0.942946552076011	1.13828659387957	\\
0.944035012131794	-0.0619297995666535	\\
0.944035012131794	-0.0619297995666535	\\
0.945078119685254	-2.58204302028037	\\
0.945576997210821	-2.00098301116944	\\
0.94714165854101	1.52019059444481	\\
0.947187011043334	1.58457616203501	\\
0.948524909861902	-1.08930877383859	\\
0.94893308238282	-1.30476926643074	\\
0.950225628699063	-0.039740975867001	\\
0.950747182475793	0.597719678537819	\\
0.951790290029252	-0.308959072417162	\\
0.952969455089685	-0.736869162357652	\\
0.953354951359441	0.132271811307672	\\
0.953354951359441	0.132271811307672	\\
0.95491961268963	1.92648531872149	\\
0.954942288940792	1.96795212014725	\\
0.95628018775936	-0.144362571366274	\\
0.956529626522143	0.288538905480318	\\
0.957459352819792	1.24229142499341	\\
0.958026259098846	0.719350314837944	\\
0.95922810041044	-1.02537095725897	\\
0.960135150456926	-0.94257753619809	\\
0.960792761740629	-0.262522558588836	\\
0.961132905508061	-0.63492689266143	\\
0.961631783033629	0.129393159219658	\\
0.962810948094061	-0.180609396131434	\\
0.963786026894034	1.46061515696523	\\
0.964239551917277	1.19179444603328	\\
0.965804213247466	-1.20189029549786	\\
0.965804213247466	-1.20189029549786	\\
0.966439148280006	-1.73285507274948	\\
0.967346198326493	-1.49968419631785	\\
0.967981133359033	-1.81846161911797	\\
0.968910859656682	-1.56834560084445	\\
0.970452844735708	1.16950177353503	\\
0.970452844735708	1.16950177353503	\\
0.971904124810086	2.15484816059423	\\
0.972017506065897	2.04766296920024	\\
0.973536814893762	-0.536452799857877	\\
0.973650196149573	-0.412311221751946	\\
0.974262454930951	-0.976638953809874	\\
0.975124152475113	-0.915415538723234	\\
0.976439375042518	0.75176473105189	\\
0.976688813805302	0.59094014812793	\\
0.978049388875031	1.31036431910576	\\
0.978321503888977	1.11481995488302	\\
0.979568697702896	-0.793379643355849	\\
0.979795460214517	-0.242688771241813	\\
0.981201387786571	-1.05022361510463	\\
0.981518855302841	-1.00596043538611	\\
0.982652667860949	-0.264852192376414	\\
0.982902106623733	-0.411740695723951	\\
0.984308034195787	1.02950645488239	\\
0.984489444205084	0.805148481683322	\\
0.985872695525976	-0.655627641757847	\\
0.98643960180503	-0.369160843059561	\\
0.987233270595705	-1.15162758490213	\\
0.987686795618948	-0.925512210562022	\\
0.987981586884056	-0.375565262998244	\\
0.989206104446813	-0.668586037400613	\\
0.990657384521191	0.705743130255259	\\
0.991360348307218	0.977638406417177	\\
0.992085988344407	0.522699824748745	\\
0.992244722102542	0.645762815047256	\\
0.99306106714438	-0.105203128023357	\\
0.994330937209461	0.95034541755385	\\
0.995351368511758	-0.19961104968579	\\
0.996439828567541	0.0815363210314492	\\
0.996893353590784	-0.631263355052664	\\
0.996961382344271	-0.70174244109599	\\
0.998299281162838	0.96147732841689	\\
1	0.780192067086254	\\
};
\addlegendentry{y};

\end{axis}

\begin{axis}[%
width=\figurewidth,
height=\figureheight,
scale only axis,
xmin=0,
xmax=1,
xlabel={t (s)},
ymin=-6.53222006299869,
ymax=6.99720337275687,
ylabel={x},
name=plot5,
at=(plot6.left of south west),
anchor=right of south east,
title={x(t)},
legend style={draw=black,fill=white,legend cell align=left}
]
\addplot [color=blue,solid]
  table[row sep=crcr]{
0	0	\\
0	0	\\
0	0	\\
0.00156466133018889	0	\\
0.00156466133018889	0	\\
0.00312932266037779	0	\\
0.00312932266037779	0	\\
0.00467130773940452	0	\\
0.00467130773940452	0	\\
0.00623596906959341	0	\\
0.00623596906959341	0	\\
0.00777795414862015	0	\\
0.00777795414862015	0	\\
0.00934261547880904	0	\\
0.00934261547880904	0	\\
0.0109072768089979	0	\\
0.0109072768089979	0	\\
0.0124492618880247	0	\\
0.0124492618880247	0	\\
0.0140139232182136	0	\\
0.0140139232182136	0	\\
0.0155559082972403	0	\\
0.0155559082972403	0	\\
0.0171205696274292	0	\\
0.0171205696274292	0	\\
0.0186625547064559	0	\\
0.0186625547064559	0	\\
0.0202272160366448	0	\\
0.0202272160366448	0	\\
0.0217918773668337	0	\\
0.0217918773668337	0	\\
0.0233338624458605	0	\\
0.0233338624458605	0	\\
0.0248985237760493	0	\\
0.0248985237760493	0	\\
0.0264405088550761	0	\\
0.0264405088550761	0	\\
0.028005170185265	0	\\
0.028005170185265	0	\\
0.0295698315154539	0	\\
0.0295698315154539	0	\\
0.0311118165944806	0	\\
0.0311118165944806	0	\\
0.0326764779246695	0	\\
0.0326764779246695	0	\\
0.0342184630036962	0	\\
0.0342184630036962	0	\\
0.0357831243338851	0	\\
0.0357831243338851	0	\\
0.0373251094129119	0	\\
0.0373251094129119	0	\\
0.0388897707431008	0	\\
0.0388897707431008	0	\\
0.0404544320732896	0	\\
0.0404544320732896	0	\\
0.0419964171523164	0	\\
0.0419964171523164	0	\\
0.0444908047801537	-3.05185094475746e-05	\\
0.045103063561532	9.15555283427238e-05	\\
0.0461234948638291	-0.000244148075580597	\\
0.0463729336266129	0.00039674062281847	\\
0.0472346311707748	0.000671407207846642	\\
0.047688156194018	-0.000366222113370895	\\
0.0486178824916665	-0.000671407207846642	\\
0.049411551282342	0.000823999755084515	\\
0.0503186013288283	-0.000823999755084515	\\
0.0508628313567201	0.00119022186845541	\\
0.0516791763985578	0.00164799951016903	\\
0.0527903127055035	-0.000823999755084515	\\
0.0529717227148008	0.00186162907630205	\\
0.0544456790403411	-0.0108951078727841	\\
0.0544910315426654	-0.0106509597972035	\\
0.0558969591147192	-0.0232551041990519	\\
0.0559876641193678	-0.0227362904697657	\\
0.0575523254495567	-0.0501113943755627	\\
0.0579378217193134	-0.0578936114907265	\\
0.0591169867797456	0.0403149500489235	\\
0.0591169867797456	0.0403149500489235	\\
0.0598879793192589	0.16153447329998	\\
0.0613165831424749	0.0169377736747265	\\
0.0622236331889612	0.187170013785362	\\
0.0625637769563936	0.19455549120903	\\
0.0637656182679879	-0.0259102154523134	\\
0.0643098482958797	-0.0499587990343571	\\
0.0650354883330688	0.0468459129333496	\\
0.0653756321005012	0.00155644398182631	\\
0.0668949409283657	-0.182622760534287	\\
0.0668949409283657	-0.182622760534287	\\
0.0682555159980952	-0.254585415124893	\\
0.0684369260073925	-0.252235472202301	\\
0.069661443570149	-0.364268928766251	\\
0.0700015873375814	-0.337382107973099	\\
0.0715435724166081	0.0390942096710205	\\
0.0716569536724189	-0.0195318460464478	\\
0.0726093562212295	0.427442252635956	\\
0.073108233746797	0.359447002410889	\\
0.0745821900723372	0.161748096346855	\\
0.0754212113653371	0.0801416039466858	\\
0.0757160026304451	0.219367042183876	\\
0.0763282614118234	0.13910336792469	\\
0.077711512732715	-0.0538956895470619	\\
0.0779155989931744	0.0702536106109619	\\
0.078573210276877	-0.179967656731606	\\
0.0802059003605524	0.208807647228241	\\
0.0803646341186875	-0.0730002745985985	\\
0.0813397129186603	0.2613605260849	\\
0.0824281729744439	-0.162236392498016	\\
0.082586906732579	-0.150059506297112	\\
0.0836753667883625	-0.393749803304672	\\
0.0840608630581192	-0.451551854610443	\\
0.0851493231139028	-0.00485244300216436	\\
0.085897639402254	-0.0751976072788239	\\
0.086895394453389	0.164952546358109	\\
0.0873035669743078	0.151890620589256	\\
0.0883693507789292	-0.268318742513657	\\
0.0894351345835507	-0.234351634979248	\\
0.0896618970951722	0.0402539148926735	\\
0.0904102133835234	-0.230445265769959	\\
0.0917707884532529	0.0761131644248962	\\
0.0921562847230096	-0.0607928708195686	\\
0.093222068527631	0.271248519420624	\\
0.0934715072904147	0.0450148023664951	\\
0.0940384135694687	0.401867747306824	\\
0.0948774348624685	0.259529411792755	\\
0.0964420961926574	-0.130130931735039	\\
0.0965328011973061	-0.160069584846497	\\
0.0975985850019275	0.060518205165863	\\
0.0987097213088732	-0.0882900506258011	\\
0.0994353613460623	0.0999786406755447	\\
0.0996167713553595	-0.095675528049469	\\
0.099820857615819	0.110019229352474	\\
0.101589605206467	0.140079960227013	\\
0.101702986462278	-0.0639362782239914	\\
0.102836799020386	0.100863672792912	\\
0.104061316583142	-0.0973235294222832	\\
0.104582870359872	0.012695699930191	\\
0.105172452890088	-0.289162874221802	\\
0.106578380462142	-0.153874322772026	\\
0.107235991745845	0.0753196850419044	\\
0.107734869271412	0.0591143518686295	\\
0.10802966053652	-0.133396402001381	\\
0.109163473094628	-0.115115821361542	\\
0.110183904396925	0.11041596531868	\\
0.110977573187601	0.153752252459526	\\
0.111929975736411	-0.0575579106807709	\\
0.112088709494546	-0.174993127584457	\\
0.1126556157736	0.0729087218642235	\\
0.113630694573573	-0.0328379161655903	\\
0.114583097122384	0.211157560348511	\\
0.115104650899113	-0.0180974770337343	\\
0.116261139708383	0.212408825755119	\\
0.116760017233951	0.167394027113914	\\
0.118143268554843	-0.0692770183086395	\\
0.118415383568788	-0.134128853678703	\\
0.119209052359464	0.10199286043644	\\
0.119889339894329	0.125125885009766	\\
0.121113857457085	-0.112887963652611	\\
0.122247670015193	-0.115543074905872	\\
0.122474432526815	0.00610370188951492	\\
0.123517540080274	-0.124088257551193	\\
0.123653597587247	0.0280770286917686	\\
0.124447266377922	0.0383617654442787	\\
0.125966575205787	-0.129520550370216	\\
0.125989251456949	-0.127445295453072	\\
0.127485884033652	0.0292062144726515	\\
0.127621941540625	-0.0875576063990593	\\
0.128597020340597	0.0603045746684074	\\
0.129481394135921	0.129978328943253	\\
0.129594775391732	-0.0571306496858597	\\
0.131454227987029	-0.0418103598058224	\\
0.131726343000975	0.112674340605736	\\
0.132565364293975	-0.0167241431772709	\\
0.133608471847434	0.164159059524536	\\
0.134674255652056	0.135502189397812	\\
0.135195809428785	-0.0438550971448421	\\
0.135399895689245	0.107242040336132	\\
0.135581305698542	-0.0579241327941418	\\
0.137372729540352	0.0860927179455757	\\
0.138030340824055	-0.0967436730861664	\\
0.138574570851947	-0.101229898631573	\\
0.139368239642622	0.0372936204075813	\\
0.139980498424001	-0.0512405782938004	\\
0.140570080954217	0.114047668874264	\\
0.142157418535568	0.0125431073829532	\\
0.142769677316946	-0.124423965811729	\\
0.143495317354135	-0.169988095760345	\\
0.143676727363432	-0.0181585140526295	\\
0.144651806163405	-0.111056856811047	\\
0.145853647474999	0.0898770093917847	\\
0.146239143744756	-0.0344248786568642	\\
0.146669992516837	0.125217437744141	\\
0.148189301344702	0.0661641284823418	\\
0.148982970135377	-0.0887783467769623	\\
0.149595228916755	-0.109530933201313	\\
0.149685933921404	-0.0108951078727841	\\
0.151930882786458	-0.118564411997795	\\
0.152202997800404	0.0251472517848015	\\
0.152429760312025	-0.0159916989505291	\\
0.153654277874782	0.148167356848717	\\
0.154379917911971	0.139835804700851	\\
0.155377672963106	-0.00607318338006735	\\
0.155581759223565	0.0359202846884727	\\
0.156738248032835	0.164799958467484	\\
0.15710106805143	0.0838038250803947	\\
0.15857502437697	-0.0447401367127895	\\
0.15920995940951	-0.101077303290367	\\
0.159550103176943	-0.0182500686496496	\\
0.160842649493186	-0.123661004006863	\\
0.161568289530375	-0.0218512527644634	\\
0.162157872060591	-0.0941190868616104	\\
0.163314360869861	0.0222174748778343	\\
0.163926619651239	-0.0354625098407269	\\
0.164425497176807	0.032685324549675	\\
0.16487902220005	0.0296334736049175	\\
0.166421007279077	-0.064241461455822	\\
0.166806503548833	-0.0208136234432459	\\
0.167577496088347	-0.164647355675697	\\
0.168121726116238	-0.152500987052917	\\
0.169527653688292	0.116336561739445	\\
0.170117236218508	0.205359056591988	\\
0.171092315018481	0.0477614663541317	\\
0.171251048776616	0.0993072316050529	\\
0.171908660060319	-0.0613116845488548	\\
0.172929091362616	-0.0585650205612183	\\
0.174108256423048	0.0541093163192272	\\
0.174561781446291	0.0628376081585884	\\
0.174675162702102	-0.026947844773531	\\
0.176194471529967	-0.0101626636460423	\\
0.17728293158575	0.0872524157166481	\\
0.177305607836912	0.0751976072788239	\\
0.17864350665548	-0.0755027905106544	\\
0.179187736683372	-0.035096287727356	\\
0.180298872990317	0.0634784996509552	\\
0.180661693008912	0.0663777589797974	\\
0.181976915576317	-0.101260416209698	\\
0.181976915576317	-0.101260416209698	\\
0.183473548153019	-0.0158391073346138	\\
0.184289893194857	-0.0524308010935783	\\
0.185083561985533	0.0494399853050709	\\
0.185446382004127	0.0809045657515526	\\
0.186580194562235	-0.00350962858647108	\\
0.186670899566884	0.0400402843952179	\\
0.187668654618019	-0.0389721356332302	\\
0.188190208394748	0.0127262184396386	\\
0.189596135966802	-0.0746177583932877	\\
0.190072337241207	-0.0744956806302071	\\
0.190185718497018	-0.0187078472226858	\\
0.191795732329531	-0.0836817547678947	\\
0.192748134878342	0.00186162907630205	\\
0.192861516134153	-0.0256660673767328	\\
0.194244767455044	0.117679372429848	\\
0.194698292478287	0.131168559193611	\\
0.195559990022449	0.0419324338436127	\\
0.196308306310801	0.0332346558570862	\\
0.197192680106125	0.111270487308502	\\
0.19787296764099	0.0800500512123108	\\
0.19902945645026	0.00939970090985298	\\
0.199278895213043	0.0385753959417343	\\
0.200639470282773	-0.0142826624214649	\\
0.201070319054854	0.0151982177048922	\\
0.201183700310665	-0.0618915371596813	\\
0.202634980385043	-0.0172734763473272	\\
0.202884419147826	-0.102755822241306	\\
0.203904850450124	-0.0367137677967548	\\
0.204743871743123	-0.104892119765282	\\
0.205786979296583	-0.09213537722826	\\
0.206852763101204	-0.0237739197909832	\\
0.207079525612826	0.00201422162353992	\\
0.207895870654663	-0.0558488741517067	\\
0.208553481938366	-0.0411389507353306	\\
0.209415179482528	0.024964140728116	\\
0.210866459556906	-0.0330515466630459	\\
0.211501394589446	0.0291146580129862	\\
0.211569423342933	0.023316141217947	\\
0.212113653370825	0.102664269506931	\\
0.213451552189392	0.0999481156468391	\\
0.213905077212635	0.0484633930027485	\\
0.215129594775392	0.0970793813467026	\\
0.216195378580013	0.0137638477608562	\\
0.216354112338148	0.0775780528783798	\\
0.217601306152067	0.00283822137862444	\\
0.217760039910202	0.0330515466630459	\\
0.219120614979932	-0.0949430838227272	\\
0.219302024989229	-0.079226054251194	\\
0.219846255017121	0.0163579210639	\\
0.221909793872877	0.0173650328069925	\\
0.222113880133336	-0.0415051728487015	\\
0.223088958933309	-0.000885036773979664	\\
0.223225016440282	-0.0371105074882507	\\
0.223950656477471	-0.0234687346965075	\\
0.225265879044876	-0.108035527169704	\\
0.225674051565795	-0.0822473838925362	\\
0.226762511621579	-0.00415051728487015	\\
0.227646885416903	0.0187078472226858	\\
0.228327172951768	-0.033967100083828	\\
0.228621964216876	-0.0304879918694496	\\
0.229370280505227	0.0276802890002728	\\
0.230549445565659	-0.0413525812327862	\\
0.231637905621443	0.0368663594126701	\\
0.231887344384226	0.0411999896168709	\\
0.232681013174902	0.00729392375797033	\\
0.233429329463253	-0.00521866511553526	\\
0.234608494523685	0.0845362693071365	\\
0.235152724551577	0.101901300251484	\\
0.236037098346901	0.0480666533112526	\\
0.237238939658496	0.0923795253038406	\\
0.237624435928252	0.0599993914365768	\\
0.237964579695685	0.0606402792036533	\\
0.239438536021225	-0.0787682756781578	\\
0.240186852309576	-0.0726950913667679	\\
0.240889816095603	-0.0395519882440567	\\
0.241207283611873	-0.0268868077546358	\\
0.241955599900225	-0.0827661976218224	\\
0.242613211183927	-0.0726950913667679	\\
0.243996462504819	-0.00549333170056343	\\
0.244518016281548	0.00860621966421604	\\
0.245719857593143	-0.0549638345837593	\\
0.245742533844305	-0.0517899096012115	\\
0.246762965146602	0.0148930326104164	\\
0.247760720197737	-0.0099795525893569	\\
0.248622417741899	0.0290841404348612	\\
0.249234676523277	0.0589922778308392	\\
0.250300460327899	0.0201727356761694	\\
0.250595251593007	0.0307931769639254	\\
0.251479625388331	-0.0365306548774242	\\
0.252091884169709	-0.0213934760540724	\\
0.253475135490601	0.0788903459906578	\\
0.254110070523141	0.096865750849247	\\
0.255062473071952	0.01535081025213	\\
0.255062473071952	0.01535081025213	\\
0.256173609378897	-0.047853022813797	\\
0.256808544411438	-0.0347911007702351	\\
0.25735277443933	0.00302133243530989	\\
0.258169119481167	-0.0116885891184211	\\
0.258667997006735	-0.052369762212038	\\
0.259892514569491	-0.0381176173686981	\\
0.26113970838341	0.0232551041990519	\\
0.261389147146194	0.0277718435972929	\\
0.262840427220572	-0.0554521307349205	\\
0.262908455974058	-0.0649433881044388	\\
0.264132973536815	0.0503860600292683	\\
0.264405088550761	0.0271919928491116	\\
0.265856368625139	-0.037263099104166	\\
0.26608313113676	-0.0386974699795246	\\
0.266967504932085	0.00885036773979664	\\
0.267783849973922	-0.0353709533810616	\\
0.269008367536679	0.0223700683563948	\\
0.269099072541327	0.0141605883836746	\\
0.270255561350598	0.0807519778609276	\\
0.270618381369192	0.0657368674874306	\\
0.272183042699381	-0.0211188085377216	\\
0.272568538969138	-0.0405896194279194	\\
0.273725027778408	0.0248115491122007	\\
0.27374770402957	0.0254829563200474	\\
0.275289689108597	-0.00244148075580597	\\
0.27542574661557	0.0018005920574069	\\
0.276355472913218	-0.0322885848581791	\\
0.276877026689948	-0.0038453321903944	\\
0.278328306764326	0.0199896246194839	\\
0.278396335517812	0.0165410321205854	\\
0.279348738066623	-0.0471510961651802	\\
0.279938320596839	-0.0328073985874653	\\
0.281502981927028	0.0205999948084354	\\
0.281502981927028	0.0205999948084354	\\
0.282840880745595	-0.0218512527644634	\\
0.283362434522325	-0.00308236945420504	\\
0.284496247080433	-0.0390331745147705	\\
0.284700333340892	-0.032746359705925	\\
0.286174289666432	0.023316141217947	\\
0.287013310959432	0.0507217645645142	\\
0.287716274745459	-0.0179143659770489	\\
0.287852332252432	-0.00714133121073246	\\
0.288555296038459	-0.0558488741517067	\\
0.289371641080297	-0.0247505120933056	\\
0.29077756865235	0.0334482863545418	\\
0.29161658994535	0.0593584999442101	\\
0.292387582484864	0.00900296028703451	\\
0.293226603777863	0.0250251777470112	\\
0.293861538810404	-0.0137943662703037	\\
0.293952243815052	-0.00491348002105951	\\
0.295448876391755	-0.0398876927793026	\\
0.295834372661512	-0.0474562831223011	\\
0.296673393954511	0.00958281196653843	\\
0.297557767749836	-0.00262459181249142	\\
0.298600875303295	0.038453321903944	\\
0.298895666568403	0.0469069480895996	\\
0.300074831628835	0.00198370311409235	\\
0.300845824168348	0.0146794030442834	\\
0.30170752171251	-0.0186773277819157	\\
0.302115694233429	-0.0373546555638313	\\
0.303090773033402	-0.00518814660608768	\\
0.303317535545024	-0.00500503554940224	\\
0.303725708065943	-0.0287484358996153	\\
0.305313045647294	-0.0223700683563948	\\
0.30622009569378	0.0231330301612616	\\
0.30705911698678	0.0293893255293369	\\
0.307807433275131	-0.00567644275724888	\\
0.307943490782104	0.00958281196653843	\\
0.309417447107644	-0.0910061970353127	\\
0.310279144651806	-0.0901516750454903	\\
0.31105013719132	0.156315803527832	\\
0.311118165944806	0.226752519607544	\\
0.312433388512211	-0.0753196850419044	\\
0.312864237284292	-0.00537125766277313	\\
0.313703258577292	-0.200842306017876	\\
0.314202136102859	-0.13104647397995	\\
0.315721444930724	0.145878478884697	\\
0.31599355994467	0.253425717353821	\\
0.317036667498129	-0.184392839670181	\\
0.317648926279507	-0.219885855913162	\\
0.318692033832967	0.0328379161655903	\\
0.318873443842264	-0.175359353423119	\\
0.320324723916642	0.203436389565468	\\
0.320392752670129	0.150700405240059	\\
0.321753327739858	-0.156743064522743	\\
0.322025442753804	0.130558177828789	\\
0.323000521553777	-0.148777738213539	\\
0.323499399079344	-0.13266396522522	\\
0.324202362865371	0.227759629487991	\\
0.325041384158371	0.108188115060329	\\
0.326243225469965	-0.155644401907921	\\
0.327581124288533	-0.149357587099075	\\
0.328170706818749	0.0694601312279701	\\
0.328193383069911	0.0839869379997253	\\
0.329213814372208	-0.102145448327065	\\
0.330030159414046	0.0702230930328369	\\
0.330823828204721	-0.178991064429283	\\
0.332048345767478	-0.164586320519447	\\
0.332751309553505	0.05725272372365	\\
0.333182158325586	0.121738336980343	\\
0.334270618381369	-0.167699202895164	\\
0.335268373432504	0.225409716367722	\\
0.33554048844645	-0.0373851731419563	\\
0.336039365972018	0.102084413170815	\\
0.336787682260369	-0.160740986466408	\\
0.338284314837071	-0.0706198289990425	\\
0.338443048595206	0.0702536106109619	\\
0.339259393637044	0.0891445651650429	\\
0.339735594911449	-0.128055661916733	\\
0.340597292455611	-0.0462965779006481	\\
0.341935191274179	0.298471033573151	\\
0.342411392548584	0.267708361148834	\\
0.343499852604367	-0.116916410624981	\\
0.343998730129935	-0.0611896105110645	\\
0.345132542688043	-0.297769099473953	\\
0.345268600195016	-0.266182452440262	\\
0.346447765255448	-0.107669301331043	\\
0.346833261525205	-0.150517284870148	\\
0.348307217850745	0.202063053846359	\\
0.348443275357718	0.114993743598461	\\
0.348760742873988	0.275490581989288	\\
0.350007936687907	0.249061554670334	\\
0.351368511757636	-0.209875792264938	\\
0.351481893013447	-0.0902737528085709	\\
0.352388943059933	-0.294015318155289	\\
0.353114583097122	-0.246864229440689	\\
0.354339100659879	0.114169746637344	\\
0.354656568176149	-0.0435499139130116	\\
0.3554048844645	0.225928530097008	\\
0.356221229506338	0.116580709815025	\\
0.35740039456677	-0.140537738800049	\\
0.357876595841176	0.0679342001676559	\\
0.359169142157419	-0.159001439809799	\\
0.359418580920202	0.00579851679503918	\\
0.360733803487607	-0.15323343873024	\\
0.360824508492256	-0.0434583574533463	\\
0.362366493571283	0.283761113882065	\\
0.362910723599175	0.30622273683548	\\
0.363795097394499	-0.0079348124563694	\\
0.364089888659607	0.105624563992023	\\
0.365155672464228	-0.178533285856247	\\
0.365972017506066	-0.172795802354813	\\
0.367037801310687	-0.0112918484956026	\\
0.367196535068822	-0.140598773956299	\\
0.368602462640876	0.102511674165726	\\
0.369055987664119	0.126255080103874	\\
0.369260073924579	-0.0748008638620377	\\
0.370552620240822	0.123111665248871	\\
0.371368965282659	-0.103366188704968	\\
0.372026576566362	0.0259712524712086	\\
0.373047007868659	-0.241248816251755	\\
0.373273770380281	-0.173467203974724	\\
0.374815755459307	0.0282906591892242	\\
0.375201251729064	-0.0736411660909653	\\
0.376199006780199	0.205969423055649	\\
0.377491553096442	0.268379777669907	\\
0.377627610603415	0.0344248786568642	\\
0.37830789813828	0.193365275859833	\\
0.379033538175469	-0.04901272803545	\\
0.379487063198712	0.0476393923163414	\\
0.38041678949636	-0.13928647339344	\\
0.382185537087009	-0.0904263406991959	\\
0.382366947096306	0.110934779047966	\\
0.3835687884079	0.149540692567825	\\
0.383682169663711	-0.0634174644947052	\\
0.384589219710197	0.154545739293098	\\
0.385677679765981	-0.0763878300786018	\\
0.385791061021792	0.0428785048425198	\\
0.386992902333386	-0.183812975883484	\\
0.387469103607792	-0.194219797849655	\\
0.388466858658926	0.0849940478801727	\\
0.389147146193791	-0.0604266487061977	\\
0.389895462482142	0.0884731560945511	\\
0.391119980044899	0.0738242715597153	\\
0.391845620082088	-0.0737632364034653	\\
0.392299145105331	-0.0564592443406582	\\
0.392866051384385	0.0618610195815563	\\
0.394113245198304	0.0865199714899063	\\
0.394974942742466	-0.0700094625353813	\\
0.395791287784304	-0.0889004170894623	\\
0.396539604072655	0.0516068004071712	\\
0.397129186602871	-0.0287484358996153	\\
0.397809474137736	0.106173895299435	\\
0.398625819179573	0.144077882170677	\\
0.399396811719087	-0.0725424960255623	\\
0.40048527177487	-0.121036410331726	\\
0.400734710537654	0.0399182103574276	\\
0.401823170593437	-0.0363170281052589	\\
0.402707544388762	0.10229804366827	\\
0.403274450667816	0.127994626760483	\\
0.404204176965464	-0.00848414562642574	\\
0.404498968230572	-0.107455670833588	\\
0.404657701988707	0.0840174555778503	\\
0.406358420825869	0.0191656239330769	\\
0.407288147123518	-0.114230781793594	\\
0.407696319644436	0.0211493279784918	\\
0.407787024649085	-0.117313154041767	\\
0.409487743486247	-0.052308727055788	\\
0.410553527290868	0.0567033924162388	\\
0.41057620354203	0.0479140616953373	\\
0.411891426109436	-0.161900699138641	\\
0.412186217374544	0.0119022186845541	\\
0.413297353681489	-0.115939818322659	\\
0.413750878704733	-0.081820122897625	\\
0.414862015011678	0.0418408773839474	\\
0.415564978797705	0.0616779066622257	\\
0.41624526633257	-0.0516983568668365	\\
0.416812172611624	-0.0282296221703291	\\
0.418104718927867	0.118778042495251	\\
0.419419941495272	0.0279244370758533	\\
0.419510646499921	0.135807365179062	\\
0.420077552778975	0.150730922818184	\\
0.420258962788272	-0.00161748100072145	\\
0.421687566611488	0.0855433791875839	\\
0.422957436676569	-0.0395519882440567	\\
0.423932515476541	-0.0513626523315907	\\
0.424045896732352	0.0259407330304384	\\
0.42486224177419	0.0454725809395313	\\
0.425837320574163	-0.0925931558012962	\\
0.426154788090433	0.044129766523838	\\
0.427628744415973	-0.129337444901466	\\
0.428059593188054	-0.149723812937737	\\
0.428195650695027	-0.00720236822962761	\\
0.429329463253135	-0.120242930948734	\\
0.430531304564729	0.0217291787266731	\\
0.43096215333681	0.0361339151859283	\\
0.431256944601918	-0.0280465111136436	\\
0.432458785913513	-0.00485244300216436	\\
0.433252454704188	0.101138338446617	\\
0.434295562257648	0.0992767140269279	\\
0.435202612304134	-0.0253913998603821	\\
0.43547472731808	-0.00991851557046175	\\
0.436449806118053	-0.105014190077782	\\
0.437424884918025	-0.0599383525550365	\\
0.438536021224971	0.0939359739422798	\\
0.438876164992403	0.143101289868355	\\
0.440146035057484	-0.0220343638211489	\\
0.440690265085376	-0.0274666585028172	\\
0.441483933876052	0.0398266538977623	\\
0.441869430145808	0.0447706542909145	\\
0.443116623959727	-0.0860011577606201	\\
0.443320710220186	0.0075685903429985	\\
0.444295789020159	-0.0760216042399406	\\
0.444794666545727	-0.0448011718690395	\\
0.446336651624753	0.0545365773141384	\\
0.446812852899159	0.0610980577766895	\\
0.447901312954942	-0.0275887325406075	\\
0.448694981745618	-0.0107730338349938	\\
0.44903512551305	-0.0764793828129768	\\
0.450032880564185	-0.0726340562105179	\\
0.450622463094401	0.0161137729883194	\\
0.451937685661806	-0.0531937628984451	\\
0.452527268192022	0.00842310860753059	\\
0.452822059457131	-0.0622577592730522	\\
0.454137282024536	0.0283516962081194	\\
0.455067008322184	0.0409863591194153	\\
0.455225742080319	-0.0166325885802507	\\
0.456858432163995	-0.0473342090845108	\\
0.456994489670968	0.0319528803229332	\\
0.457266604684913	0.00921658985316753	\\
0.458695208508129	0.0842310860753059	\\
0.459103381029048	0.0905789360404015	\\
0.459874373568562	0.0181585140526295	\\
0.460826776117372	0.0874965637922287	\\
0.461439034898751	0.0224005859345198	\\
0.46196058867548	0.0467238388955593	\\
0.463389192498696	-0.0261238440871239	\\
0.463457221252183	-0.00610370188951492	\\
0.464409623800993	-0.0619830936193466	\\
0.465679493866074	0.0119632557034492	\\
0.466337105149777	-0.052339244633913	\\
0.46658654391256	-0.0022888882085681	\\
0.467720356470668	-0.0836817547678947	\\
0.468536701512506	-0.0405896194279194	\\
0.46942107530783	-0.0906399711966515	\\
0.469761219075262	-0.081881158053875	\\
0.471189822898478	-0.00173955503851175	\\
0.4714165854101	-0.0385448783636093	\\
0.472618426721694	0.0407727286219597	\\
0.473117304247262	0.010040589608252	\\
0.47372956302864	0.0874660462141037	\\
0.474341821810018	0.0601825006306171	\\
0.475793101884397	-0.0121768852695823	\\
0.476904238191342	0.0143436994403601	\\
0.477040295698315	-0.0494094677269459	\\
0.477584525726207	0.000152592547237873	\\
0.478491575772693	0.0726950913667679	\\
0.479103834554072	0.0914334580302238	\\
0.480441733372639	0.0210882909595966	\\
0.480827229642395	0.0458693206310272	\\
0.481530193428422	0.0115665150806308	\\
0.482165128460963	0.0343028046190739	\\
0.483684437288827	-0.0764793828129768	\\
0.484001904805098	-0.0891140475869179	\\
0.485203746116692	-0.0245979186147451	\\
0.485362479874827	-0.0412305071949959	\\
0.486791083698043	0.0399182103574276	\\
0.487085874963151	0.0434278398752213	\\
0.488197011270097	-0.0552079826593399	\\
0.488423773781718	-0.0289315469563007	\\
0.489308147577043	-0.0680257603526115	\\
0.489988435111907	-0.066591389477253	\\
0.491439715186285	0.00308236945420504	\\
0.491553096442096	0.00115970335900784	\\
0.492528175242069	0.043336283415556	\\
0.493117757772285	0.0664998292922974	\\
0.494092836572258	0.00177007354795933	\\
0.49495453411642	0.0679647177457809	\\
0.495884260414068	-0.0127567369490862	\\
0.496179051679176	-0.00381481368094683	\\
0.497244835483798	-0.0507217645645142	\\
0.497743713009365	-0.0288399923592806	\\
0.499240345586068	0.0627155378460884	\\
0.499421755595365	0.0370494723320007	\\
0.500374158144176	0.0802331641316414	\\
0.501031769427878	0.0606402792036533	\\
0.502324315744121	-0.00469985045492649	\\
0.502573754506905	0.0228278450667858	\\
0.503752919567337	-0.0461745038628578	\\
0.50388897707431	-0.01535081025213	\\
0.505090818385904	-0.0603045746684074	\\
0.505680400916121	-0.0432447269558907	\\
0.506315335948661	-0.00112918484956026	\\
0.507449148506769	0.00283822137862444	\\
0.50851493231139	-0.0508438386023045	\\
0.508560284813715	-0.0415662117302418	\\
0.510056917390417	0.0106509597972035	\\
0.510397061157849	0.0018005920574069	\\
0.511326787455498	-0.0280465111136436	\\
0.512347218757795	0.00970488600432873	\\
0.512936801288011	-0.0244453269988298	\\
0.513254268804281	-0.0242622159421444	\\
0.514750901380984	0.0177922919392586	\\
0.515363160162362	0.0453199855983257	\\
0.515998095194902	0.00775170139968395	\\
0.516542325222794	0.00170903652906418	\\
0.517834871539037	0.0444349497556686	\\
0.517902900292524	0.0429700613021851	\\
0.519059389101794	0.0662862062454224	\\
0.519558266627361	0.0661946460604668	\\
0.520669402934307	-0.00796533096581697	\\
0.521508424227307	0.0079348124563694	\\
0.522143359259847	-0.0248725861310959	\\
0.522687589287739	-0.0285348072648048	\\
0.522914351799361	-0.00305185094475746	\\
0.524342955622576	-0.00741599779576063	\\
0.525386063176036	-0.0556047260761261	\\
0.525771559445792	-0.0562456138432026	\\
0.526406494478333	0.00451673939824104	\\
0.527313544524819	-0.00390636920928955	\\
0.528764824599197	-0.0844752341508865	\\
0.528787500850359	-0.0831629410386086	\\
0.530148075920089	0.0120853297412395	\\
0.530624277194494	0.00146488845348358	\\
0.530783010952629	0.0303659178316593	\\
0.532687816050251	0.0341807305812836	\\
0.532914578561872	0.00122074037790298	\\
0.533708247352548	0.0394909530878067	\\
0.534728678654845	0.00512710958719254	\\
0.535318261185061	0.0479750968515873	\\
0.536474749994331	0.00326548051089048	\\
0.537449828794304	-0.0196539200842381	\\
0.538107440078006	0.0219428092241287	\\
0.538107440078006	0.0219428092241287	\\
0.538833080115195	0.0511795394122601	\\
0.540148302682601	0.04089480265975	\\
0.541214086487222	-0.0234687346965075	\\
0.541372820245357	-0.012054811231792	\\
0.542370575296492	-0.0450758375227451	\\
0.543753826617384	-0.00494399853050709	\\
0.544298056645275	-0.043275248259306	\\
0.544366085398762	-0.0432142093777657	\\
0.545862717975464	-0.0152592547237873	\\
0.545953422980113	-0.0186773277819157	\\
0.54717794054287	0.0142216254025698	\\
0.548357105603302	0.0195928830653429	\\
0.548719925621896	-0.00479140598326921	\\
0.549354860654437	0.00494399853050709	\\
0.550397968207896	-0.0495010241866112	\\
0.550738111975328	-0.0545976124703884	\\
0.55212136329622	0.0177312549203634	\\
0.552506859565977	0.00927762687206268	\\
0.553005737091544	0.0387279875576496	\\
0.553754053379895	0.0186468102037907	\\
0.555205333454273	0.0390942096710205	\\
0.555545477221706	0.0404065065085888	\\
0.556565908524003	0.00164799951016903	\\
0.557042109798408	0.0162968840450048	\\
0.558062541100705	-0.0108951078727841	\\
0.559241706161137	-0.0125431073829532	\\
0.559876641193678	0.00881984923034906	\\
0.559967346198326	-0.000610370188951492	\\
0.561418626272705	0.0386974699795246	\\
0.56157736003084	0.047883540391922	\\
0.562756525091272	0.0109866634011269	\\
0.56305131635638	0.0355845838785172	\\
0.564502596430758	-0.0346995443105698	\\
0.564638653937731	-0.0211188085377216	\\
0.565454998979569	-0.0972624868154526	\\
0.566135286514433	-0.0814233869314194	\\
0.566951631556271	0.0274666585028172	\\
0.568085444114379	0.0267036966979504	\\
0.569219256672487	-0.105288855731487	\\
0.56935531417946	-0.111575670540333	\\
0.570670536746865	-0.0371410250663757	\\
0.570806594253838	-0.0669270902872086	\\
0.571532234291027	0.0292062144726515	\\
0.572416608086351	-0.00595110934227705	\\
0.573369010635162	0.123416855931282	\\
0.574208031928162	0.117099523544312	\\
0.575409873239756	0.0692770183086395	\\
0.575568606997891	0.08185064047575	\\
0.576906505816458	-0.0663472414016724	\\
0.576997210821107	-0.064241461455822	\\
0.578040318374566	-0.276009410619736	\\
0.578675253407107	-0.172948390245438	\\
0.579604979704755	0.333933532238007	\\
0.580103857230323	0.224127933382988	\\
0.581645842309349	-0.102389596402645	\\
0.582008662327944	-0.0357982106506824	\\
0.582484863602349	-0.279030740261078	\\
0.583323884895349	-0.15329447388649	\\
0.584548402458106	0.103823967278004	\\
0.58522868999297	0.164189577102661	\\
0.586022358783646	-0.312723159790039	\\
0.586339826299916	-0.226264223456383	\\
0.587655048867321	0.156285285949707	\\
0.588335336402186	0.267830431461334	\\
0.589446472709132	-0.110324412584305	\\
0.589650558969591	0.103793449699879	\\
0.590262817750969	-0.203405871987343	\\
0.591101839043969	-0.182897433638573	\\
0.592553119118347	0.0663167238235474	\\
0.592847910383455	0.276650279760361	\\
0.593822989183428	0.00259407330304384	\\
0.595070182997347	0.142307803034782	\\
0.595433003015941	-0.100558489561081	\\
0.59584117553686	0.199163794517517	\\
0.595977233043833	-0.123447373509407	\\
0.597496541871698	-0.0668660551309586	\\
0.597882038141454	0.124546036124229	\\
0.598857116941427	0.0754112377762794	\\
0.599605433229778	-0.171056240797043	\\
0.600421778271616	-0.11865596473217	\\
0.60182770584367	0.207098603248596	\\
0.601873058345994	0.177800834178925	\\
0.602848137145967	-0.106479078531265	\\
0.604072654708724	0.0281075481325388	\\
0.604548855983129	-0.132236704230309	\\
0.60497970475521	-0.0740989446640015	\\
0.606476337331912	0.0915555283427238	\\
0.60697521485748	0.0688497573137283	\\
0.607632826141182	-0.192236095666885	\\
0.608222408671398	-0.168065428733826	\\
0.609560307489966	0.143711656332016	\\
0.610263271275993	0.190893277525902	\\
0.611215673824803	-0.0880764201283455	\\
0.611646522596884	0.00628681294620037	\\
0.612054695117803	-0.231238752603531	\\
0.61275765890383	-0.164586320519447	\\
0.614322320234019	0.120273448526859	\\
0.614458377740992	0.0658589452505112	\\
0.614639787750289	0.240516379475594	\\
0.616091067824667	0.224219486117363	\\
0.617088822875802	-0.0148930326104164	\\
0.617633052903694	0.121097445487976	\\
0.618630807954829	-0.10837122797966	\\
0.619356447992018	0.0124210333451629	\\
0.620422231796639	-0.360118418931961	\\
0.62053561305245	-0.25788140296936	\\
0.621986893126828	0.253151029348373	\\
0.622508446903558	0.227881714701653	\\
0.623483525703531	-0.0357982106506824	\\
0.623732964466314	0.0451368764042854	\\
0.625093539536044	-0.222357854247093	\\
0.62536565454999	-0.334543913602829	\\
0.626544819610422	0.0828272327780724	\\
0.626998344633665	0.0800805687904358	\\
0.627769337173179	-0.301675468683243	\\
0.628358919703395	-0.273171186447144	\\
0.629742171024286	0.234290599822998	\\
0.630263724801016	0.291268646717072	\\
0.631352184856799	-0.0164189580827951	\\
0.632281911154448	0.10000915825367	\\
0.632395292410259	-0.221106603741646	\\
0.633279666205583	-0.239326149225235	\\
0.634322773759042	0.172887355089188	\\
0.635275176307853	0.229316085577011	\\
0.636000816345042	0.0613116845488548	\\
0.637111952651988	0.145207062363625	\\
0.637474772670582	-0.0978118255734444	\\
0.638472527721717	-0.176519066095352	\\
0.638767318986825	0.0871303454041481	\\
0.640354656568176	0.09201330691576	\\
0.640536066577473	-0.108615376055241	\\
0.641511145377446	-0.181218907237053	\\
0.641805936642554	0.0252388082444668	\\
0.642735662940203	-0.0568865016102791	\\
0.643642712986689	0.211737424135208	\\
0.643937504251797	0.217230752110481	\\
0.645366108075013	-0.147038176655769	\\
0.645456813079662	0.00570696126669645	\\
0.645638223088959	-0.246101260185242	\\
0.647134855665661	-0.202093571424484	\\
0.648155286967958	0.236854150891304	\\
0.649561214540012	-0.220496237277985	\\
0.649946710809769	0.0863979011774063	\\
0.650468264586499	0.0021668141707778	\\
0.651352638381823	0.253364652395248	\\
0.651647429646931	0.129856258630753	\\
0.65237306968412	-0.144779816269875	\\
0.653665616000363	0.0801110863685608	\\
0.65384702600966	-0.229956969618797	\\
0.654799428558471	-0.185766160488129	\\
0.656205356130525	0.205114901065826	\\
0.656341413637497	0.147007659077644	\\
0.657067053674687	0.347819447517395	\\
0.658631715004875	0.227057710289955	\\
0.658813125014173	-0.103640861809254	\\
0.659788203814145	0.000274666585028172	\\
0.660695253860632	-0.274849683046341	\\
0.661012721376902	-0.258430749177933	\\
0.662532030204767	0.0459608770906925	\\
0.662600058958253	-0.0489516891539097	\\
0.663756547767523	0.205542162060738	\\
0.664504864055874	0.0186468102037907	\\
0.664845007823307	0.211615338921547	\\
0.665638676613982	0.0498367249965668	\\
0.666296287897685	-0.179906606674194	\\
0.667543481711603	-0.296182125806808	\\
0.668745323023198	0.0213629566133022	\\
0.669221524297603	-0.148625135421753	\\
0.670196603097576	0.176610618829727	\\
0.67139844440917	0.232306897640228	\\
0.671851969432413	-0.000122074037790298	\\
0.6719199981859	-0.0542924292385578	\\
0.673167191999819	0.218329414725304	\\
0.673530012018413	0.152775660157204	\\
0.674890587088143	-0.152287364006042	\\
0.675548198371845	-0.233588665723801	\\
0.676137780902061	0.0308542139828205	\\
0.676727363432277	-0.076815091073513	\\
0.67801990974852	0.247810304164886	\\
0.678269348511304	0.232886746525764	\\
0.679403161069412	-0.049043245613575	\\
0.679629923581034	-0.00250251777470112	\\
0.680582326129844	-0.221686452627182	\\
0.681716138687952	-0.168492689728737	\\
0.682464454976303	0.0520645789802074	\\
0.683439533776276	-0.111575670540333	\\
0.684097145059979	0.110995821654797	\\
0.685276310120411	0.239997565746307	\\
0.685503072632032	-0.0169072542339563	\\
0.686659561441302	0.0903347879648209	\\
0.687113086464546	-0.12057863175869	\\
0.688382956529627	-0.299630731344223	\\
0.68894986280868	-0.018768884241581	\\
0.689131272817978	-0.082918792963028	\\
0.690446495385383	0.20657978951931	\\
0.69062790539468	0.14294870197773	\\
0.691444250436518	-0.143498033285141	\\
0.692419329236491	-0.0982696041464806	\\
0.693394408036463	0.222907200455666	\\
0.694324134334112	0.106540113687515	\\
0.695027098120139	-0.076998196542263	\\
0.695321889385247	0.0185552537441254	\\
0.695934148166625	-0.255439937114716	\\
0.696954579468922	-0.250648528337479	\\
0.697952334520057	0.237220376729965	\\
0.698383183292138	0.243873402476311	\\
0.69892741332003	0.037171546369791	\\
0.699857139617678	0.119144260883331	\\
0.701217714687408	-0.156224250793457	\\
0.701671239710651	-0.189245283603668	\\
0.702351527245516	0.0336008779704571	\\
0.703371958547813	-0.156224250793457	\\
0.703644073561759	0.0976287126541138	\\
0.704732533617542	-0.128391370177269	\\
0.705616907412867	0.0997955277562141	\\
0.707113539989569	0.164952546358109	\\
0.70754438876165	-0.0622272416949272	\\
0.707816503775596	0.0892361253499985	\\
0.70901834508719	-0.178319647908211	\\
0.70953989886392	-0.158085882663727	\\
0.710719063924352	0.0929593816399574	\\
0.710877797682487	-0.0430616177618504	\\
0.711580761468514	0.173833429813385	\\
0.712306401505703	0.115848265588284	\\
0.713304156556838	-0.0624713897705078	\\
0.714982199142838	-0.0527054667472839	\\
0.715299666659108	0.19611194729805	\\
0.716025306696297	0.0843226388096809	\\
0.716887004240459	-0.0951567143201828	\\
0.717091090500918	0.064088873565197	\\
0.717204471756729	-0.183599352836609	\\
0.718519694324134	-0.0489822067320347	\\
0.719970974398512	0.127719968557358	\\
0.720764643189188	0.145756408572197	\\
0.721558311979864	-0.164128541946411	\\
0.722193247012404	-0.0467238388955593	\\
0.722397333272863	-0.266640216112137	\\
0.72330438331935	-0.174138620495796	\\
0.723961994603052	0.081759087741375	\\
0.725526655933241	-0.0822473838925362	\\
0.726025533458809	0.0761741995811462	\\
0.726297648472754	0.00619525741785765	\\
0.727318079775052	0.212744534015656	\\
0.727839633551781	0.0943327099084854	\\
0.729154856119186	-0.0810266435146332	\\
0.72940429488197	0.0179143659770489	\\
0.730832898705186	-0.253364652395248	\\
0.731422481235402	-0.224768817424774	\\
0.732284178779564	0.0440382100641727	\\
0.732510941291186	0.016327403485775	\\
0.733213905077213	0.271706283092499	\\
0.735232091430645	0.20242927968502	\\
0.73539082518878	0.00253303628414869	\\
0.735617587700401	0.117709890007973	\\
0.736252522732942	-0.243842884898186	\\
0.737613097802671	0.0741294622421265	\\
0.738746910360779	-0.165532395243645	\\
0.739109730379374	-0.189397871494293	\\
0.73938184539332	0.0513016134500504	\\
0.740424952946779	-0.0347911007702351	\\
0.741536089253725	0.193090602755547	\\
0.742397786797887	0.19623401761055	\\
0.743395541849022	-0.105777151882648	\\
0.743667656862967	-0.0361949540674686	\\
0.744552030658292	-0.297738581895828	\\
0.745073584435021	-0.19226661324501	\\
0.746116691988481	0.0914639756083488	\\
0.747227828295426	0.210882902145386	\\
0.747454590807048	-0.000946073792874813	\\
0.748588403365156	0.18390454351902	\\
0.749223338397696	-0.103152565658093	\\
0.749744892174426	-0.165257722139359	\\
0.750742647225561	0.102694787085056	\\
0.752080546044128	0.0562456138432026	\\
0.752738157327831	-0.240546897053719	\\
0.752738157327831	-0.240546897053719	\\
0.754030703644074	0.10217597335577	\\
0.754280142406857	0.0274361409246922	\\
0.755323249960317	0.227393418550491	\\
0.756366357513776	0.19809564948082	\\
0.757409465067235	-0.143314927816391	\\
0.757409465067235	-0.143314927816391	\\
0.758906097643937	0.0357066579163074	\\
0.759200888909046	-0.0700704976916313	\\
0.759473003922991	0.116306036710739	\\
0.761309780267126	0.164036989212036	\\
0.762058096555477	-0.0713217556476593	\\
0.762488945327558	0.075563833117485	\\
0.763441347876369	-0.0963164195418358	\\
0.763645434136829	0.0288094729185104	\\
0.765051361708882	-0.157872244715691	\\
0.765595591736774	-0.244148075580597	\\
0.766321231773963	-0.00265511032193899	\\
0.766729404294882	-0.0467848740518093	\\
0.768135331866936	0.201910465955734	\\
0.768951676908773	0.174413278698921	\\
0.769699993197125	-0.0831934586167336	\\
0.77069774824826	0.0312814712524414	\\
0.771264654527314	-0.128971219062805	\\
0.77165015079707	-0.171239361166954	\\
0.772829315857502	0.0839869379997253	\\
0.773441574638881	-0.0607318356633186	\\
0.774416653438853	0.193670466542244	\\
0.775709199755096	0.193762019276619	\\
0.775981314769042	-6.10370188951492e-05	\\
0.776185401029502	0.16574601829052	\\
0.776684278555069	-0.0386364348232746	\\
0.777772738610853	0.080660417675972	\\
0.778838522415474	-0.168950468301773	\\
0.779518809950339	-0.134800255298615	\\
0.780357831243339	0.00885036773979664	\\
0.781310233792149	-0.0383312478661537	\\
0.782194607587474	0.109378337860107	\\
0.782942923875825	0.116306036710739	\\
0.783532506406041	-0.0563982054591179	\\
0.784190117689744	0.0296029541641474	\\
0.785233225243203	-0.183812975883484	\\
0.78561872151296	-0.123447373509407	\\
0.786956620331527	0.0693685710430145	\\
0.787886346629175	0.156468391418457	\\
0.787977051633824	-0.0340586565434933	\\
0.78908818794077	0.140568256378174	\\
0.790017914238418	-0.0567949451506138	\\
0.790947640536067	0.0697958320379257	\\
0.791287784303499	-0.0693380534648895	\\
0.791968071838364	0.0318308062851429	\\
0.792580330619742	-0.112491227686405	\\
0.793328646908093	-0.0748924240469933	\\
0.794213020703417	0.121433146297932	\\
0.795437538266174	0.0967131555080414	\\
0.796208530805687	-0.038453321903944	\\
0.79654867457312	0.0754417553544044	\\
0.797569105875417	-0.125125885009766	\\
0.797886573391687	-0.0501724295318127	\\
0.798589537177714	0.134800255298615	\\
0.799632644731173	0.120456553995609	\\
0.800879838545092	-0.106875821948051	\\
0.801446744824146	-0.154026910662651	\\
0.802421823624119	-0.0103457747027278	\\
0.802535204879929	-0.13699759542942	\\
0.803895779949659	0.103762932121754	\\
0.80496156375428	0.157811209559441	\\
0.805528470033334	-0.0353709533810616	\\
0.805732556293793	0.0706198289990425	\\
0.806662282591442	-0.0858180522918701	\\
0.807319893875145	0.0422681346535683	\\
0.80863511644255	-0.169225141406059	\\
0.809247375223928	-0.223456531763077	\\
0.810267806526225	-0.0214850306510925	\\
0.810267806526225	-0.0214850306510925	\\
0.811832467856414	0.200659200549126	\\
0.812490079140117	0.229224517941475	\\
0.813374452935441	0.0626239851117134	\\
0.813397129186603	0.0636616125702858	\\
0.814304179233089	-0.176854759454727	\\
0.81493911426563	-0.132175669074059	\\
0.815460668042359	0.0237434003502131	\\
0.816753214358602	-0.0729697570204735	\\
0.817977731921359	0.164860993623734	\\
0.818499285698088	0.236610010266304	\\
0.81956506950271	0.020874660462141	\\
0.819678450758521	0.0708334594964981	\\
0.820880292070115	-0.140079960227013	\\
0.821152407084061	0.00790429394692183	\\
0.822626363409601	-0.183690905570984	\\
0.82271706841425	-0.135563224554062	\\
0.824054967232817	0.0966216027736664	\\
0.824621873511871	0.161015659570694	\\
0.825596952311844	-0.127903074026108	\\
0.825823714823465	-0.0743736103177071	\\
0.827388376153654	0.0482802838087082	\\
0.827841901176897	0.0521561317145824	\\
0.828658246218735	-0.0621967241168022	\\
0.830018821288465	0.0778527185320854	\\
0.830200231297762	-0.123752556741238	\\
0.830676432572167	-0.151829585433006	\\
0.830903195083789	0.113376259803772	\\
0.832467856413978	-0.0341196954250336	\\
0.832989410190707	0.120059818029404	\\
0.833896460237194	0.0553605780005455	\\
0.834032517744167	-0.0869167149066925	\\
0.835393092813896	-0.0737937539815903	\\
0.836640286627815	0.078005313873291	\\
0.837615365427787	0.080599382519722	\\
0.838023537948706	-0.0902127176523209	\\
0.838386357967301	-0.0909146368503571	\\
0.839202703009139	0.0439161360263824	\\
0.839928343046328	-0.056062500923872	\\
0.840926098097463	0.118381299078465	\\
0.841901176897435	0.035157322883606	\\
0.842853579446246	-0.0827051624655724	\\
0.843329780720651	-0.0748008638620377	\\
0.844214154515975	0.0296639911830425	\\
0.845053175808975	0.0831324234604836	\\
0.845937549604299	-0.116885893046856	\\
0.84605093086011	0.0721152350306511	\\
0.846640513390326	-0.0720541998744011	\\
0.847660944692623	-0.0544755384325981	\\
0.848590670990272	0.112430192530155	\\
0.850427447334407	0.000457777641713619	\\
0.85056350484138	-0.113498337566853	\\
0.851062382366947	-0.0698263496160507	\\
0.851266468627407	0.100772120058537	\\
0.852944511213406	0.110171817243099	\\
0.853783532506406	-0.00231940671801567	\\
0.853987618766865	0.0737937539815903	\\
0.855121431324973	-0.0913418978452683	\\
0.856005805120298	-0.0679647177457809	\\
0.856754121408649	0.0555131696164608	\\
0.857071588924919	-0.00567644275724888	\\
0.857457085194676	0.0893887132406235	\\
0.859089775278351	0.0962858945131302	\\
0.859951472822513	-0.075624868273735	\\
0.86013288283181	0.0784936100244522	\\
0.860314292841108	-0.0543534643948078	\\
0.86181092541781	0.0542313903570175	\\
0.862740651715458	-0.111819818615913	\\
0.863330234245674	-0.0677816122770309	\\
0.864078550534026	0.0635090172290802	\\
0.86475883806889	0.0671102032065392	\\
0.866051384385133	-0.114597000181675	\\
0.866504909408377	-0.0177617724984884	\\
0.867616045715322	-0.118625447154045	\\
0.868319009501349	-0.112033449113369	\\
0.869362117054808	0.12677389383316	\\
0.869475498310619	0.0750450119376183	\\
0.870246490850133	0.199438452720642	\\
0.870926778384997	0.154271066188812	\\
0.872151295947754	-0.0892666429281235	\\
0.872672849724484	-0.0021668141707778	\\
0.873489194766321	-0.134464547038078	\\
0.875325971110456	0.044160284101963	\\
0.87557540987324	-0.138157293200493	\\
0.875598086124402	-0.126377150416374	\\
0.877094718701104	0.226203188300133	\\
0.877140071203429	0.202032536268234	\\
0.87852332252432	0.0586260557174683	\\
0.878750085035942	0.215247049927711	\\
0.879861221342888	-0.205145418643951	\\
0.880360098868455	-0.232367932796478	\\
0.881629968933536	-0.0323191024363041	\\
0.881811378942833	-0.113406784832478	\\
0.883081249007914	0.108920559287071	\\
0.883557450282319	0.128971219062805	\\
0.884781967845076	-0.0178533289581537	\\
0.884963377854373	0.136387214064598	\\
0.885779722896211	-0.120365001261234	\\
0.886913535454319	-0.0689107924699783	\\
0.888024671761264	0.0622272416949272	\\
0.888047348012427	0.0481887273490429	\\
0.88933989432867	-0.127170622348785	\\
0.89060976439375	-0.0719626471400261	\\
0.890768498151886	0.093173012137413	\\
0.89165287194721	0.13489180803299	\\
0.892514569491372	-0.0296945106238127	\\
0.892764008254155	0.0241706594824791	\\
0.893875144561101	-0.139072850346565	\\
0.894646137100615	0.0822168663144112	\\
0.895099662123858	-0.0631122812628746	\\
0.896732352207533	0.149021878838539	\\
0.897095172226128	0.00473036896437407	\\
0.898433071044695	0.100680559873581	\\
0.898659833556317	-0.0676595345139503	\\
0.899068006077235	-0.0589312426745892	\\
0.899770969863262	0.0618610195815563	\\
0.900723372412073	0.0913113802671433	\\
0.902038594979478	-0.185705125331879	\\
0.902038594979478	-0.185705125331879	\\
0.90342184630037	0.0447706542909145	\\
0.903603256309667	0.0147404400631785	\\
0.904532982607315	0.161961734294891	\\
0.905394680151477	0.0660115331411362	\\
0.90668722646772	-0.119327373802662	\\
0.907435542756072	-0.153202921152115	\\
0.907662305267693	0.00891140475869179	\\
0.909544434114152	0.0563066489994526	\\
0.909771196625774	-0.11059907823801	\\
0.910519512914125	-0.167699202895164	\\
0.911267829202476	0.152317881584167	\\
0.911403886709449	0.0116275520995259	\\
0.91267375677453	0.190954312682152	\\
0.913535454318692	0.126529738306999	\\
0.914465180616341	-0.0266426596790552	\\
0.915009410644232	0.0195928830653429	\\
0.915961813193043	-0.221839040517807	\\
0.916029841946529	-0.173039942979813	\\
0.917571827025556	0.191351056098938	\\
0.917889294541826	0.18793298304081	\\
0.918682963332502	0.010620441287756	\\
0.919363250867367	0.115146338939667	\\
0.920497063425475	-0.129337444901466	\\
0.921313408467312	-0.158757284283638	\\
0.921540170978934	0.07565538585186	\\
0.922379192271934	0.0118717001751065	\\
0.922787364792852	-0.135532706975937	\\
0.92380779609515	-0.0843836814165115	\\
0.925327104923014	0.110568560659885	\\
0.925576543685798	0.118686482310295	\\
0.926846413750879	-0.0462355427443981	\\
0.926959795006689	0.0212714020162821	\\
0.927481348783419	-0.101626634597778	\\
0.929204743871743	-0.126468703150749	\\
0.929680945146148	0.0659810155630112	\\
0.930678700197283	0.116916410624981	\\
0.931290958978662	-0.0180364400148392	\\
0.931631102746094	-0.0422070994973183	\\
0.933037030318148	0.0544755384325981	\\
0.933558584094877	-0.0575273893773556	\\
0.933808022857661	0.0511185042560101	\\
0.93505521667158	-0.0712302029132843	\\
0.935418036690174	0.0496230982244015	\\
0.936551849248282	-0.0428479872643948	\\
0.937731014308715	0.0897244215011597	\\
0.938547359350552	0.110904261469841	\\
0.939341028141228	-0.00711081270128489	\\
0.939477085648201	0.0591143518686295	\\
0.940134696931903	-0.13464766740799	\\
0.940905689471417	-0.0800500512123108	\\
0.942356969545795	0.104373306035995	\\
0.942787818317876	0.0858485698699951	\\
0.944035012131794	-0.0658894628286362	\\
0.944216422141092	-0.0199591051787138	\\
0.945010090931767	-0.237678155303001	\\
0.94566770221547	-0.0679036825895309	\\
0.94714165854101	0.135288551449776	\\
0.94714165854101	0.135288551449776	\\
0.94829814735028	-0.167882323265076	\\
0.94881970112701	-0.117130041122437	\\
0.949930837433955	0.0672322735190392	\\
0.95061112496882	0.0712912380695343	\\
0.951540851266469	-0.0701315328478813	\\
0.952379872559468	-0.125675216317177	\\
0.953082836345495	0.0876491591334343	\\
0.953581713871063	-0.00244148075580597	\\
0.954851583936144	0.19220557808876	\\
0.95491961268963	0.174565881490707	\\
0.955872015238441	-0.0517288744449615	\\
0.956643007777954	0.117831967771053	\\
0.958026259098846	-0.0290841404348612	\\
0.958411755368602	0.0438856184482574	\\
0.959160071656954	-0.102969452738762	\\
0.95974965418717	-0.108676411211491	\\
0.960724732987143	0.0773644223809242	\\
0.961382344270845	0.0941801220178604	\\
0.96238009932198	-0.0990325659513474	\\
0.962788271842899	-0.033906064927578	\\
0.963581940633575	0.117954038083553	\\
0.964420961926574	0.0972624868154526	\\
0.965804213247466	-0.144413590431213	\\
0.966212385768385	-0.15311136841774	\\
0.966938025805574	-0.00079348124563694	\\
0.967913104605547	-0.135074928402901	\\
0.968343953377628	-0.0404065065085888	\\
0.96900156466133	-0.0563982054591179	\\
0.970407492233384	0.127689450979233	\\
0.971155808521735	0.0326242856681347	\\
0.971337218531033	0.146855071187019	\\
0.97215356357287	0.100985750555992	\\
0.972924556112383	-0.0908536016941071	\\
0.973582167396086	0.0165105145424604	\\
0.974013016168167	-0.0939970090985298	\\
0.975124152475113	-0.0594500564038754	\\
0.976688813805302	0.0926847159862518	\\
0.976756842558788	0.153080850839615	\\
0.977414453842491	-0.0224311053752899	\\
0.978253475135491	0.076906643807888	\\
0.979273906437788	-0.10223700851202	\\
0.980226308986598	0.0538041330873966	\\
0.981133359033085	-0.15515610575676	\\
0.981428150298193	-0.123294778168201	\\
0.982063085330733	0.0426953956484795	\\
0.983219574140003	-0.0320444367825985	\\
0.984262681693462	0.103915527462959	\\
0.985373818000408	-0.0950041189789772	\\
0.985509875507381	0.0603656135499477	\\
0.986008753032949	0.0129703665152192	\\
0.986643688065489	-0.11828974634409	\\
0.988729903172408	-0.0754112377762794	\\
0.988979341935191	0.0520340576767921	\\
0.989546248214245	-0.0868861973285675	\\
0.99054400326538	0.0914029330015183	\\
0.991791197079299	0.0807214602828026	\\
0.99201795959092	-0.0309457685798407	\\
0.992630218372299	-0.0802331641316414	\\
0.993514592167623	0.0797448679804802	\\
0.994285584707136	0.0953703448176384	\\
0.994988548493163	-0.0680867955088615	\\
0.996303771060568	0.0675679817795753	\\
0.996802648586136	-0.0898770093917847	\\
0.996893353590784	-0.0745261982083321	\\
0.998027166148892	0.125888854265213	\\
1	-0.0115359965711832	\\
};
\addlegendentry{x};

\end{axis}

\begin{axis}[%
width=\figurewidth,
height=\figureheight,
scale only axis,
xmin=0,
xmax=22050,
xlabel={Frequency (Hz)},
ymin=0,
ymax=0.549134354559093,
ylabel={|X(f)|},
name=plot7,
at=(plot5.below south west),
anchor=above north west,
title={Single-Sided Amplitude Spectrum of x(t)},
legend style={draw=black,fill=white,legend cell align=left}
]
\addplot [color=blue,solid]
  table[row sep=crcr]{
0	0.000373549288362603	\\
1.3458251953125	1.72304889296254e-05	\\
31.6268920898438	0.00115677722843967	\\
34.991455078125	0.000450136353389119	\\
49.7955322265625	0.022535224243801	\\
72.0016479492188	0.00962516606613554	\\
102.955627441406	0.000724652330477434	\\
102.955627441406	0.000724652330477434	\\
129.872131347656	0.0129014721283839	\\
141.984558105469	0.0294879176134795	\\
150.059509277344	0.00038751752259323	\\
187.069702148438	0.000419624714906407	\\
200.527954101563	0.00640306621396154	\\
206.584167480469	0.000610793464227858	\\
238.211059570313	0.0177841584693326	\\
246.286010742188	0.00850298378437354	\\
251.669311523438	0.000629387375641223	\\
277.239990234375	0.000147157871595493	\\
287.333679199219	0.00497317337479226	\\
333.091735839844	0.00371344780253294	\\
342.512512207031	0.00042454548512006	\\
361.354064941406	0.00576453529911535	\\
374.139404296875	0.000694521003066869	\\
390.962219238281	0.000122247389839036	\\
405.766296386719	0.00442386192655753	\\
436.720275878906	0.00197621529186735	\\
441.4306640625	0.000113820272600369	\\
460.945129394531	5.43853852258583e-05	\\
469.692993164063	0.0021368603028137	\\
495.936584472656	0.000161259918623026	\\
510.740661621094	0.00293774633212591	\\
539.675903320313	0.00326701342083149	\\
541.694641113281	0.000335402727719258	\\
556.498718261719	0.000108286531373907	\\
573.994445800781	0.00205063680226689	\\
585.433959960938	0.000110418354225598	\\
613.6962890625	0.00347977415874561	\\
623.117065429688	0.000267767457486475	\\
644.650268554688	0.00390284318741868	\\
670.220947265625	9.91538225758145e-05	\\
681.660461425781	0.00192853395417459	\\
689.0625	1.61822783881218e-05	\\
703.866577148438	0.00204009328375838	\\
733.474731445313	2.71646349854322e-05	\\
746.260070800781	0.00101282130793972	\\
755.680847167969	0.000922888738821517	\\
768.466186523438	1.11544689886171e-05	\\
789.999389648438	0.000977790534107685	\\
803.457641601563	4.09754513016208e-05	\\
831.719970703125	2.02095811188952e-05	\\
841.813659667969	0.0012852344424262	\\
866.038513183594	7.43695643647894e-05	\\
879.496765136719	0.000931259480020877	\\
913.815307617188	0.000760633010207535	\\
925.927734375	6.28168718542915e-05	\\
925.927734375	6.28168718542915e-05	\\
954.190063476563	0.0010693777733275	\\
981.779479980469	0.00102073765296644	\\
991.200256347656	2.55066620767549e-05	\\
996.583557128906	0.000104177568740171	\\
1003.98559570313	0.00199877830895694	\\
1034.93957519531	0.000118916507057929	\\
1055.79986572266	0.000955965785792011	\\
1075.31433105469	0.00010383468473195	\\
1084.73510742188	0.00105989544915646	\\
1100.88500976563	4.01728828936165e-05	\\
1112.32452392578	0.000689923305637986	\\
1133.18481445313	3.87569439821206e-05	\\
1139.91394042969	0.000771164653511544	\\
1169.52209472656	4.00039641884633e-05	\\
1174.90539550781	0.00100698664935381	\\
1220.66345214844	0.00120074700703464	\\
1230.08422851563	9.32466039003785e-05	\\
1250.27160644531	0.000192987600392769	\\
1252.29034423828	0.00136651728653077	\\
1283.91723632813	0.00224170706348583	\\
1288.62762451172	0.000101718950304817	\\
1311.50665283203	0.000120436760618598	\\
1329.00238037109	0.00214959137827297	\\
1345.15228271484	0.00139654300021779	\\
1361.97509765625	4.47713661791985e-05	\\
1374.76043701172	0.00160852853124347	\\
1405.04150390625	7.67060372325086e-05	\\
1419.84558105469	0.0011649421801082	\\
1429.93927001953	2.63314659782802e-05	\\
1448.10791015625	0.00122616581396228	\\
1469.64111328125	3.88797438086615e-05	\\
1485.791015625	0.00102607868551421	\\
1501.94091796875	3.18953153374669e-05	\\
1521.45538330078	0.000614770906426653	\\
1532.22198486328	1.41697956885959e-05	\\
1546.35314941406	0.0010205699190703	\\
1574.61547851563	0.000103077141159594	\\
1599.51324462891	7.90701817869408e-05	\\
1606.91528320313	0.00145964665501428	\\
1627.77557373047	0.00210575401075969	\\
1637.86926269531	0.000114752989831392	\\
1658.056640625	0.00212809892778009	\\
1676.89819335938	0.000195853619091688	\\
1697.75848388672	3.73551369528822e-05	\\
1711.8896484375	0.00118019584068095	\\
1722.65625	0.00159992399427213	\\
1730.73120117188	6.10891348390475e-05	\\
1761.01226806641	0.00230439749997447	\\
1778.50799560547	1.27698310728514e-05	\\
1786.58294677734	0.00199440938697854	\\
1812.82653808594	1.68221820936637e-05	\\
1822.92022705078	0.00122960386512927	\\
1826.95770263672	7.30016694095507e-05	\\
1855.89294433594	0.00129853829766638	\\
1882.80944824219	9.65788447247233e-06	\\
1892.23022460938	0.00116257480048077	\\
1909.72595214844	3.16501305970168e-05	\\
1940.00701904297	3.48366879249253e-05	\\
1952.79235839844	0.0014509361815701	\\
1962.21313476563	0.00180809114922978	\\
1977.69012451172	4.749556744986e-05	\\
1997.87750244141	0.00201276029685578	\\
2020.08361816406	0.000104944708959128	\\
2024.12109375	0.00192139635451644	\\
2041.61682128906	0.000180684485757729	\\
2061.80419921875	0.00244300706097618	\\
2079.29992675781	5.70152423691361e-05	\\
2104.87060546875	0.0011704119735599	\\
2113.61846923828	0.000110358357916855	\\
2132.46002197266	0.000135296250522927	\\
2153.3203125	0.0015463463666927	\\
2184.94720458984	0.00291107050718418	\\
2193.02215576172	0.000116347015305322	\\
2202.44293212891	6.43150901293467e-05	\\
2227.34069824219	0.00214806904043965	\\
2232.05108642578	0.000331698536214737	\\
2256.27593994141	0.00255982642752609	\\
2271.08001708984	9.33246799896631e-05	\\
2292.61322021484	0.00265279940547152	\\
2318.85681152344	0.00200793298051316	\\
2326.25885009766	6.7870023675875e-05	\\
2334.33380126953	0.000204719771167814	\\
2358.55865478516	0.00193173716904177	\\
2367.97943115234	0.00268680950588447	\\
2387.49389648438	0.000134519330222223	\\
2405.66253662109	0.00168054276417268	\\
2422.4853515625	0.000171992598822292	\\
2462.86010742188	0.000131962819601555	\\
2468.91632080078	0.00204637724925198	\\
2469.58923339844	0.00188531836196443	\\
2496.50573730469	3.1980063664711e-05	\\
2504.58068847656	0.00136927224835113	\\
2512.65563964844	7.29103639849282e-05	\\
2543.60961914063	8.62180481153504e-05	\\
2565.81573486328	0.00136573979315979	\\
2578.60107421875	6.19877236500982e-05	\\
2603.49884033203	0.00224293792286424	\\
2613.59252929688	9.8357559646883e-05	\\
2633.10699462891	0.00239167123402143	\\
2663.38806152344	0.000131840177434047	\\
2666.07971191406	0.00224311750123632	\\
2682.90252685547	2.69287700805489e-05	\\
2683.57543945313	0.00182807353745425	\\
2727.31475830078	0.000144604789085463	\\
2740.10009765625	0.00190921385780692	\\
2754.90417480469	6.19501681840885e-05	\\
2760.28747558594	0.00136729205352383	\\
2800.66223144531	0.0012768784291684	\\
2811.42883300781	2.24726251547032e-05	\\
2828.92456054688	7.56212446941278e-06	\\
2839.69116210938	0.00103478841136152	\\
2854.49523925781	4.63980938514257e-05	\\
2871.99096679688	0.00150104253989959	\\
2898.90747070313	0.00126607298319911	\\
2909.00115966797	3.76220872784307e-05	\\
2915.73028564453	4.95987372008456e-05	\\
2944.66552734375	0.00158542940577387	\\
2958.79669189453	0.000158900104210494	\\
2973.60076904297	0.00132654075910048	\\
2990.42358398438	4.80779109138962e-05	\\
3012.62969970703	0.00169978696255259	\\
3035.50872802734	8.4998203373372e-05	\\
3040.21911621094	0.00134452305461287	\\
3053.00445556641	3.35196521692932e-05	\\
3077.90222167969	0.00155891965199061	\\
3097.41668701172	0.000958630166598293	\\
3100.10833740234	3.75069594422374e-05	\\
3131.73522949219	0.00100479088027672	\\
3143.17474365234	2.43621121447249e-05	\\
3163.36212158203	1.83565969388903e-05	\\
3188.93280029297	0.000791523773078717	\\
3194.31610107422	4.86496044769407e-05	\\
3223.92425537109	0.00147604112241487	\\
3236.03668212891	0.00150653009318394	\\
3250.16784667969	2.23705989593258e-05	\\
3259.58862304688	0.000103183044050253	\\
3283.8134765625	0.00163333795488845	\\
3297.27172851563	0.00158241614168636	\\
3304.00085449219	0.000135390041172979	\\
3340.33813476563	0.00143159843459796	\\
3345.04852294922	5.44360940675219e-05	\\
3374.65667724609	0.00156822267495859	\\
3381.38580322266	6.76151860970928e-05	\\
3399.55444335938	0.000115572814510583	\\
3410.99395751953	0.00194132293551032	\\
3446.65832519531	0.00109060403151808	\\
3462.80822753906	2.8964836932169e-05	\\
3480.97686767578	0.00127511347196419	\\
3485.01434326172	2.26345584954002e-05	\\
3511.93084716797	0.00128232469231315	\\
3528.75366210938	5.74529494472794e-05	\\
3534.13696289063	1.38705764707635e-05	\\
3549.61395263672	0.0012157605959614	\\
3577.20336914063	0.00101178231830919	\\
3579.22210693359	2.64875122443095e-05	\\
3616.90521240234	0.00177193155861459	\\
3622.28851318359	1.81707680597547e-05	\\
3650.55084228516	0.0016428242573688	\\
3666.70074462891	0.000113606086452205	\\
3682.85064697266	0.00182565800135206	\\
3697.65472412109	0.000203846307083869	\\
3704.38385009766	9.07726168412116e-05	\\
3711.11297607422	0.00145333097640254	\\
3751.48773193359	0.0018197737383357	\\
3767.63763427734	9.37650189919118e-05	\\
3784.46044921875	0.00144180171109787	\\
3790.51666259766	2.12318465956243e-05	\\
3810.70404052734	8.2148174328171e-05	\\
3816.08734130859	0.000831688536377905	\\
3851.75170898438	0.00084479475593915	\\
3853.09753417969	3.32537883747996e-05	\\
3898.85559082031	3.67617380927141e-05	\\
3909.62219238281	0.000687968891324924	\\
3923.75335693359	9.08933232905804e-05	\\
3943.26782226563	0.00136297809718798	\\
3954.03442382813	0.00175437276652479	\\
3966.81976318359	0.000105275973296404	\\
3989.02587890625	0.00168235640863651	\\
4003.15704345703	5.33833806322681e-05	\\
4013.25073242188	0.00126338339828009	\\
4030.07354736328	9.68210891311352e-05	\\
4055.64422607422	8.29030203094526e-05	\\
4070.44830322266	0.00102491795495821	\\
4091.30859375	0.000940001344103554	\\
4108.13140869141	0.000116126033231523	\\
4127.64587402344	4.80112866700874e-05	\\
4149.17907714844	0.000995550430199041	\\
4164.65606689453	0.000840465474781534	\\
4184.17053222656	3.79360039812912e-05	\\
4184.17053222656	3.79360039812912e-05	\\
4210.41412353516	0.000866726266594384	\\
4224.54528808594	0.00119779875343359	\\
4252.13470458984	2.35759559231652e-05	\\
4257.51800537109	0.00107693024409916	\\
4268.95751953125	5.82794863706195e-05	\\
4288.47198486328	0.00143347915654415	\\
4293.85528564453	0.000148691962862853	\\
4336.24877929688	0.000158733868914849	\\
4347.01538085938	0.00105495607807146	\\
4367.20275878906	9.76122276395434e-05	\\
4379.31518554688	0.00124182065895672	\\
4412.96081542969	9.88891021560674e-05	\\
4422.38159179688	0.00106921967956399	\\
4425.74615478516	0.00122800234550547	\\
4450.64392089844	8.28142460841822e-05	\\
4466.12091064453	0.00105411153708894	\\
4484.28955078125	8.32062935672168e-05	\\
4509.18731689453	0.000633730255142118	\\
4510.53314208984	8.96110454397759e-05	\\
4529.37469482422	5.60372797562207e-05	\\
4556.29119873047	0.00110746105742349	\\
4561.67449951172	0.00101987729963548	\\
4562.34741210938	8.5576033797069e-05	\\
4611.47003173828	5.57179493236293e-05	\\
4617.52624511719	0.00100022117031969	\\
4631.65740966797	0.00122899878971536	\\
4641.75109863281	9.05874907532866e-05	\\
4667.99468994141	0.000982271434319753	\\
4688.85498046875	4.9393784357338e-05	\\
4707.02362060547	7.23608175659395e-05	\\
4730.57556152344	0.00108640408059996	\\
4750.76293945313	3.97007979584897e-05	\\
4761.52954101563	0.000760118425196299	\\
4768.25866699219	6.83168392215595e-05	\\
4799.88555908203	0.0011164042214606	\\
4813.34381103516	4.68923186651893e-05	\\
4830.16662597656	0.00108995596337774	\\
4853.71856689453	7.27632096537638e-05	\\
4861.79351806641	0.00082420350084386	\\
4879.96215820313	7.79884456293704e-05	\\
4894.76623535156	0.000593022983177536	\\
4924.37438964844	3.36492564155812e-05	\\
4929.08477783203	0.000949580824285946	\\
4949.27215576172	0.000898623154400651	\\
4967.44079589844	0.000139226634861398	\\
4976.86157226563	1.94652772592099e-05	\\
5005.12390136719	0.000849288039857161	\\
5020.60089111328	7.69683068454871e-05	\\
5025.31127929688	0.000904328306833735	\\
5042.80700683594	0.00116317611360241	\\
5053.57360839844	5.07933766375388e-05	\\
5103.369140625	0.000918287456704797	\\
5109.42535400391	4.63472421448929e-05	\\
5139.70642089844	0.00079545888234018	\\
5143.74389648438	5.48500767161822e-05	\\
5147.78137207031	1.50333762715011e-05	\\
5155.18341064453	0.000529938316705256	\\
5191.52069091797	0.000497373053868211	\\
5206.99768066406	6.05843168425095e-06	\\
5229.87670898438	0.000655562695166392	\\
5245.35369873047	6.64578434024123e-05	\\
5272.94311523438	0.0005359897316074	\\
5276.98059082031	3.90971884880729e-05	\\
5298.51379394531	0.000945621020143032	\\
5311.97204589844	8.24956739628664e-06	\\
5337.54272460938	0.000649568872822327	\\
5348.98223876953	2.62894289476951e-05	\\
5365.80505371094	3.61601293255067e-05	\\
5374.55291748047	0.000780497072595358	\\
5401.46942138672	0.000845565508128864	\\
5408.19854736328	8.2415371821859e-05	\\
5436.46087646484	6.27289210348498e-05	\\
5445.88165283203	0.000648855440554833	\\
5453.28369140625	0.000628717744565102	\\
5463.37738037109	6.88042156327108e-06	\\
5503.75213623047	4.36861874921521e-05	\\
5506.44378662109	0.0004213518438311	\\
5544.12689208984	0.000558571182361061	\\
5554.22058105469	8.55317410701163e-05	\\
5570.37048339844	0.000561614557863386	\\
5589.21203613281	2.07506312312101e-05	\\
5601.32446289063	2.60405698133123e-05	\\
5606.03485107422	0.000534450627378547	\\
5652.4658203125	0.000752394480900123	\\
5654.48455810547	4.51240854484503e-05	\\
5676.69067382813	1.52235539414763e-05	\\
5678.70941162109	0.000780774627173429	\\
5698.89678955078	4.0394343613342e-05	\\
5715.04669189453	0.000645846184559231	\\
5729.17785644531	3.94338510549919e-05	\\
5745.32775878906	0.000706587672924241	\\
5770.22552490234	7.97766847461967e-05	\\
5778.30047607422	0.000757037165412745	\\
5795.79620361328	0.000531710152162373	\\
5827.42309570313	4.81418631698784e-06	\\
5840.88134765625	0.0011550290656673	\\
5862.41455078125	2.94675718707966e-05	\\
5883.27484130859	0.000336296177206225	\\
5884.62066650391	1.55457242708488e-05	\\
5899.42474365234	2.73811880137222e-05	\\
5908.17260742188	0.000535895110294884	\\
5962.00561523438	2.28714729736616e-05	\\
5966.04309082031	0.000578278768800591	\\
5967.38891601563	0.00049524509192391	\\
5972.09930419922	3.27218550407203e-05	\\
6012.47406005859	0.000696061901417359	\\
6020.54901123047	5.40409090477512e-05	\\
6043.42803955078	0.000735152454553944	\\
6065.63415527344	1.76746326792628e-05	\\
6075.72784423828	3.62399810268064e-05	\\
6095.91522216797	0.000609384625590867	\\
6116.77551269531	0.0006432709369864	\\
6133.59832763672	2.83959789617651e-05	\\
6143.69201660156	0.000638100210973087	\\
6165.22521972656	3.56745892311262e-05	\\
6178.68347167969	0.000702999873049254	\\
6190.12298583984	2.78649882788808e-05	\\
6219.05822753906	0.000480311041726465	\\
6227.13317871094	6.25528829201196e-05	\\
6249.33929443359	0.000632353937670499	\\
6271.54541015625	4.85183959734003e-05	\\
6278.94744873047	0.000512007904197817	\\
6291.73278808594	3.18218994088888e-05	\\
6336.14501953125	0.000612894195451293	\\
6340.85540771484	4.97382066458337e-05	\\
6369.11773681641	4.40735593351294e-05	\\
6377.19268798828	0.000679297197245679	\\
6379.21142578125	0.000597350345931895	\\
6412.85705566406	3.20169308013569e-05	\\
6412.85705566406	3.20169308013569e-05	\\
6447.17559814453	0.000591752799914424	\\
6447.17559814453	0.000591752799914424	\\
6479.47540283203	4.06947741213235e-05	\\
6488.22326660156	1.16273333043079e-05	\\
6510.42938232422	0.000371881974596872	\\
6540.71044921875	3.43133884365629e-05	\\
6547.43957519531	0.000276876364784165	\\
6573.68316650391	0.000297667498988575	\\
6575.70190429688	1.09507872084536e-05	\\
6600.59967041016	2.72154499776455e-05	\\
6601.94549560547	0.000307580727201003	\\
6651.06811523438	0.000455810374589724	\\
6652.41394042969	4.4405958714231e-05	\\
6658.47015380859	3.12479820801368e-05	\\
6661.16180419922	0.000381361589927628	\\
6693.46160888672	0.000547098731785676	\\
6702.88238525391	5.1387376683924e-06	\\
6728.45306396484	0.000415902242042887	\\
6749.31335449219	1.39140207380766e-05	\\
6758.73413085938	0.000469370816245405	\\
6770.17364501953	5.02236419962289e-05	\\
6803.81927490234	1.50015598955041e-05	\\
6821.98791503906	0.000302803359853811	\\
6834.10034179688	1.74895493823235e-05	\\
6857.65228271484	0.00037082866958334	\\
6861.01684570313	0.000444838104940949	\\
6887.26043701172	1.12746873874333e-05	\\
6900.04577636719	3.11482578184096e-05	\\
6912.83111572266	0.000388071627898825	\\
6932.34558105469	1.53820113824656e-05	\\
6958.58917236328	0.000406585347865467	\\
6990.21606445313	0.000378480528355949	\\
6992.90771484375	1.42841377398381e-05	\\
7013.09509277344	0.00042334594679211	\\
7025.88043212891	5.29897804871027e-05	\\
7040.01159667969	3.80681191143706e-05	\\
7047.41363525391	0.00037115300830595	\\
7081.05926513672	5.66044548792096e-05	\\
7089.80712890625	0.00052885840461159	\\
7111.34033203125	0.000302282967252354	\\
7118.74237060547	4.43709609331923e-06	\\
7154.40673828125	2.78944842693424e-05	\\
7164.50042724609	0.000361825662985782	\\
7190.74401855469	3.16242718843304e-05	\\
7201.51062011719	0.000372768188798487	\\
7201.51062011719	0.000372768188798487	\\
7210.25848388672	1.69734947002327e-05	\\
7244.57702636719	3.04529265687089e-05	\\
7268.80187988281	0.000427139444440253	\\
7280.24139404297	1.2445409149073e-05	\\
7299.08294677734	0.000359227357025511	\\
7317.92449951172	1.53833937515029e-05	\\
7333.40148925781	0.000382605164939204	\\
7367.04711914063	2.92522541587135e-05	\\
7369.06585693359	0.000295221053490597	\\
7377.81372070313	2.65886727897984e-05	\\
7397.32818603516	0.000330361268575474	\\
7409.44061279297	0.000335897769416647	\\
7414.15100097656	4.97267449473538e-06	\\
7450.48828125	2.31060434970146e-05	\\
7470.00274658203	0.000260210080038649	\\
7484.13391113281	9.40939348413206e-06	\\
7491.53594970703	0.000266580408627397	\\
7522.48992919922	2.20875298045971e-05	\\
7533.25653076172	0.000232222098130065	\\
7567.57507324219	0.000237543006578658	\\
7578.34167480469	4.8750632708966e-06	\\
7597.18322753906	1.59805663909385e-05	\\
7599.20196533203	0.000295010047869667	\\
7632.17468261719	0.000338274649948968	\\
7645.63293457031	1.99598339893017e-05	\\
7659.09118652344	0.000231582325379456	\\
7671.87652587891	1.18842347493211e-05	\\
7681.97021484375	3.2498328978656e-05	\\
7691.39099121094	0.000321884696986986	\\
7720.32623291016	2.65847631006622e-05	\\
7731.76574707031	0.000294341715030541	\\
7750.60729980469	4.64268412647066e-05	\\
7768.10302734375	0.00036492341774557	\\
7787.61749267578	3.08903021315279e-06	\\
7804.44030761719	0.000350074704507403	\\
7834.72137451172	0.000286372769564195	\\
7838.0859375	7.38582865536219e-06	\\
7865.00244140625	7.37480921938487e-06	\\
7882.49816894531	0.00019155282546263	\\
7904.03137207031	0.000222879649994515	\\
7920.18127441406	7.65452517368561e-06	\\
7922.87292480469	2.49312820748323e-05	\\
7947.77069091797	0.00020599655347264	\\
7956.5185546875	0.000168120645617775	\\
7972.66845703125	2.10435115799882e-05	\\
8007.65991210938	0.000206959547358898	\\
8019.77233886719	1.66948280582639e-05	\\
8033.90350341797	0.000341296305598837	\\
8048.03466796875	1.36854478352947e-05	\\
8083.69903564453	0.000289413374493623	\\
8090.42816162109	1.41103135814084e-05	\\
8120.03631591797	0.000268453987987074	\\
8126.76544189453	2.08270452323592e-05	\\
8130.80291748047	1.78825546910401e-05	\\
8151.66320800781	0.000222904314122694	\\
8183.96301269531	1.02907428475318e-05	\\
8186.65466308594	0.000185162513876119	\\
8205.49621582031	0.000237936654758615	\\
8225.01068115234	6.06633099383257e-06	\\
8237.79602050781	0.00016310418936933	\\
8250.58135986328	1.56589586819854e-05	\\
8270.76873779297	0.000275713851787937	\\
8282.88116455078	1.96452840755316e-05	\\
8303.74145507813	0.000244316595389436	\\
8311.14349365234	1.54002441672694e-05	\\
8345.46203613281	0.00021650072837581	\\
8360.93902587891	6.60087944827559e-06	\\
8371.70562744141	6.92619917467695e-06	\\
8382.47222900391	0.000256624768381601	\\
8418.13659667969	0.000239948155440867	\\
8426.88446044922	8.03324240219271e-06	\\
8443.03436279297	1.41825133591504e-05	\\
8445.72601318359	0.000165973292744435	\\
8482.06329345703	0.000142818577280181	\\
8484.08203125	1.69891593954644e-05	\\
8517.72766113281	0.000164328338887664	\\
8528.49426269531	1.48329547450697e-05	\\
8550.02746582031	0.00019352227598568	\\
8570.21484375	1.20277361096405e-05	\\
8576.94396972656	0.000177754628059428	\\
8580.9814453125	4.41416122884527e-06	\\
8611.93542480469	1.15333301417778e-05	\\
8632.79571533203	0.00016709682973587	\\
8645.5810546875	1.71469512195689e-05	\\
8675.86212158203	0.000168572872471613	\\
8679.89959716797	3.48276531626265e-06	\\
8708.16192626953	0.000213210991526641	\\
8714.89105224609	0.000433946510013072	\\
8725.65765380859	7.49014616925657e-06	\\
8754.59289550781	0.000240916436625641	\\
8771.41571044922	1.25094816296961e-05	\\
8792.27600097656	0.000181625378862497	\\
8807.75299072266	3.42587155580521e-06	\\
8815.15502929688	1.01093464583482e-05	\\
8825.24871826172	0.00012605834864131	\\
8850.81939697266	2.49636564910595e-06	\\
8851.49230957031	0.000144002416459529	\\
8897.92327880859	0.000161362038232661	\\
8903.9794921875	2.70723654852136e-05	\\
8944.35424804688	1.09757570989547e-05	\\
8945.70007324219	0.000135197503562786	\\
8973.28948974609	1.69693081627544e-05	\\
8978.67279052734	0.000183757307634149	\\
9000.20599365234	0.000204835977773595	\\
9009.62677001953	3.83211332426943e-06	\\
9029.81414794922	0.000263366379177768	\\
9052.69317626953	1.57117554480263e-05	\\
9060.09521484375	0.000199322157154271	\\
9069.51599121094	2.56770965265892e-05	\\
9111.90948486328	0.000177756125819794	\\
9114.60113525391	7.85142019727918e-06	\\
9123.34899902344	0.00020589609112882	\\
9136.13433837891	5.33893360793543e-06	\\
9170.45288085938	1.49180171315642e-05	\\
9187.27569580078	0.000143704609176821	\\
9194.00482177734	2.02080396404203e-05	\\
9195.35064697266	0.000231419539431304	\\
9241.10870361328	8.65035039612241e-06	\\
9247.83782958984	0.000137074750515156	\\
9268.02520751953	0.0001657648206524	\\
9276.10015869141	1.28199042256771e-06	\\
9306.38122558594	0.000162274990828278	\\
9308.39996337891	4.67858724081032e-06	\\
9338.00811767578	5.36317471710187e-06	\\
9352.13928222656	0.000212577056665805	\\
9368.28918457031	0.000242116510955532	\\
9387.13073730469	1.28887606586984e-05	\\
9397.22442626953	1.38767994895947e-05	\\
9399.91607666016	0.000181533394766687	\\
9441.63665771484	0.000198265284531645	\\
9459.13238525391	7.96641721187147e-06	\\
9467.20733642578	0.000177235625966681	\\
9484.70306396484	6.38667798924089e-06	\\
9502.19879150391	1.5701811371255e-05	\\
9527.76947021484	0.000209235166697991	\\
9538.53607177734	0.000182167356339393	\\
9550.64849853516	3.14246799844236e-06	\\
9592.36907958984	0.000182360497218389	\\
9601.11694335938	1.62443380964836e-05	\\
9604.48150634766	1.32268817006021e-05	\\
9626.01470947266	0.000130722995928809	\\
9638.12713623047	0.000145811673734183	\\
9658.31451416016	1.28072835899827e-05	\\
9683.88519287109	2.76587261333176e-05	\\
9691.96014404297	0.000209774634189942	\\
9720.22247314453	1.82581079999253e-05	\\
9737.04528808594	0.000257385581017993	\\
9749.15771484375	4.99261340154932e-06	\\
9755.21392822266	0.000265946799065455	\\
9774.05548095703	1.56318749829509e-05	\\
9792.89703369141	0.000210638812532318	\\
9833.94470214844	0.000185234364428547	\\
9841.34674072266	7.81766292227245e-06	\\
9842.01965332031	1.75880184542916e-05	\\
9843.36547851563	0.000161798927761467	\\
9884.41314697266	0.000149810521431789	\\
9890.46936035156	6.78993344836745e-06	\\
9916.71295166016	7.39707614275818e-06	\\
9926.13372802734	0.000113442610666373	\\
9955.74188232422	1.43961332155595e-05	\\
9963.81683349609	0.000187476473345388	\\
9989.38751220703	1.22589608460121e-05	\\
10000.8270263672	0.000154987029712657	\\
10013.6123657227	3.05293143213523e-05	\\
10042.5476074219	0.000223607572321141	\\
10054.6600341797	1.03651704472722e-05	\\
10074.8474121094	0.000189515222608685	\\
10099.072265625	1.01812146848128e-05	\\
10105.8013916016	0.000171935454334135	\\
10125.9887695313	0.000143713647084893	\\
10140.119934082	5.32685350573127e-06	\\
10151.5594482422	0.000129834735654343	\\
10154.9240112305	2.04094807364696e-05	\\
10195.2987670898	2.72118008136349e-06	\\
10200.0091552734	0.00010574303989863	\\
10237.6922607422	0.00014020483129196	\\
10245.7672119141	1.04896694860619e-05	\\
10259.8983764648	1.10873696835986e-05	\\
10276.7211914063	0.000140210779916164	\\
10297.5814819336	1.27749227584202e-05	\\
10301.6189575195	0.00015617298228889	\\
10328.5354614258	0.000138169517007768	\\
10350.0686645508	4.32837006696687e-06	\\
10374.9664306641	0.000195414795878018	\\
10387.0788574219	2.40579521876066e-05	\\
10397.8454589844	1.98224495272897e-05	\\
10405.9204101563	0.000142223076391611	\\
10436.8743896484	1.27735392640257e-05	\\
10446.9680786133	0.000181386403304918	\\
10461.0992431641	0.000157749704286349	\\
10471.8658447266	1.22129712340157e-05	\\
10496.0906982422	2.53666011564761e-05	\\
10512.2406005859	0.000180000761541323	\\
10534.4467163086	0.000177652929380846	\\
10539.1571044922	1.00367067605549e-05	\\
10566.7465209961	0.000153960447112193	\\
10596.354675293	2.09383519850438e-06	\\
10596.354675293	2.09383519850438e-06	\\
10621.2524414063	0.000149995369539856	\\
10649.5147705078	0.000156944777574028	\\
10652.2064208984	1.3797101805653e-05	\\
10666.3375854492	3.42165342825525e-06	\\
10667.0104980469	0.000140140869318074	\\
10704.6936035156	0.000173449475378448	\\
10710.0769042969	2.50221803159325e-06	\\
10736.9934082031	0.000136030091821742	\\
10753.8162231445	1.47666114481697e-05	\\
10771.3119506836	0.00012732792715221	\\
10788.8076782227	8.26297837983551e-06	\\
10813.0325317383	1.19002409130862e-06	\\
10827.1636962891	0.000149592541513372	\\
10855.4260253906	0.000171194075646989	\\
10863.5009765625	1.39489314779869e-05	\\
10896.4736938477	0.000130606209115592	\\
10899.1653442383	1.75599590761865e-05	\\
10911.2777709961	7.66362478297477e-06	\\
10918.0068969727	0.000194967109156901	\\
10944.9234008789	0.000200144175250533	\\
10953.6712646484	1.47018185837653e-05	\\
10981.93359375	0.000172392486616709	\\
11008.1771850586	3.5012081474502e-06	\\
11008.1771850586	3.5012081474502e-06	\\
11022.981262207	0.000131394655656187	\\
11064.7018432617	1.19602944221844e-05	\\
11073.4497070313	0.000177943928200362	\\
11079.5059204102	0.000219694019866792	\\
11091.618347168	8.31216346799529e-06	\\
11127.2827148438	0.000138271333008243	\\
11137.3764038086	5.02870536532902e-06	\\
11147.4700927734	8.00100421339032e-06	\\
11152.180480957	0.00014171301827455	\\
11207.3593139648	3.36467461788613e-06	\\
11212.0697021484	0.000112996906087644	\\
11236.9674682617	0.000114866618767995	\\
11246.3882446289	1.67797992009685e-05	\\
11265.2297973633	4.60493488916827e-06	\\
11282.0526123047	0.00012751362693997	\\
11290.1275634766	1.11881944220184e-05	\\
11315.6982421875	0.000127589730303245	\\
11344.6334838867	5.40145382442855e-06	\\
11350.016784668	0.000123557770440204	\\
11373.5687255859	1.45002569276903e-05	\\
11375.5874633789	0.000210855433177461	\\
11407.2143554688	9.24092830931654e-06	\\
11417.3080444336	0.000151284301813267	\\
11422.6913452148	0.000170535938378402	\\
11442.8787231445	9.96838529933598e-06	\\
11457.682800293	0.000232066479244217	\\
11478.5430908203	5.11424364739308e-06	\\
11506.8054199219	0.00017556254537671	\\
11512.8616333008	1.87884567389578e-05	\\
11548.5260009766	8.525049698622e-06	\\
11550.5447387695	0.000145967666145376	\\
11559.9655151367	0.000130716360652849	\\
11580.8258056641	4.29452025054191e-06	\\
11609.0881347656	0.000129770555267615	\\
11612.4526977539	5.9320998670495e-06	\\
11628.6026000977	4.60139007762305e-06	\\
11653.5003662109	0.000124779274535305	\\
11668.977355957	6.89889872250932e-06	\\
11693.8751220703	0.000107871305343705	\\
11712.043762207	6.69762066924452e-06	\\
11722.1374511719	0.000129589669584447	\\
11745.0164794922	1.67943898219569e-05	\\
11747.7081298828	0.000147333601229743	\\
11788.0828857422	7.18832556960198e-06	\\
11790.1016235352	0.000198466386454729	\\
11819.709777832	0.000169578938720185	\\
11831.1492919922	1.84142127449086e-05	\\
11838.5513305664	4.25974281432224e-06	\\
11862.1032714844	0.000111571365811747	\\
11869.5053100586	1.38717213979582e-05	\\
11887.6739501953	0.000124934796479189	\\
11919.9737548828	0.00010976258051687	\\
11927.375793457	6.48481624335594e-06	\\
11941.5069580078	7.45430563414073e-06	\\
11958.3297729492	0.000186144209281086	\\
11987.2650146484	8.26904496330148e-06	\\
11989.9566650391	0.000144471854915347	\\
12018.2189941406	0.000147118345774583	\\
12024.2752075195	8.45207556037852e-06	\\
12041.0980224609	1.10081675307983e-05	\\
12053.8833618164	0.000184959106398647	\\
12083.4915161133	0.000173133346733031	\\
12103.678894043	4.76495922303758e-06	\\
12117.8100585938	2.18996129749982e-05	\\
12131.2683105469	0.000156500177487778	\\
12149.4369506836	1.03929810752218e-05	\\
12164.241027832	0.000187480179087261	\\
12185.774230957	0.000122300373005854	\\
12190.4846191406	7.31590513393595e-06	\\
12211.344909668	0.000111330554301274	\\
12220.7656860352	1.26657922993673e-05	\\
12245.6634521484	9.31500453091233e-06	\\
12265.8508300781	0.000141241265291759	\\
12283.3465576172	7.70149753739624e-06	\\
12298.8235473633	0.000154589229774842	\\
12318.3380126953	5.53266333219491e-07	\\
12326.4129638672	0.000138654966926727	\\
12354.6752929688	0.00016554403060094	\\
12365.4418945313	1.27713234988325e-05	\\
12386.9750976563	0.000154859664774533	\\
12394.3771362305	1.30802973397898e-05	\\
12430.0415039063	0.000173369144191919	\\
12447.5372314453	2.02879805186596e-05	\\
12456.2850952148	8.13498356617358e-06	\\
12465.705871582	0.000167137781840182	\\
12507.4264526367	0.000125417148004636	\\
12510.791015625	2.57883059686251e-06	\\
12528.2867431641	8.85639142653714e-06	\\
12537.7075195313	0.000138618353876923	\\
12555.876159668	5.47593117431126e-06	\\
12557.2219848633	0.000149621497999302	\\
12601.6342163086	6.61740053282215e-06	\\
12616.438293457	0.000143047760058545	\\
12631.2423706055	0.000138152140972127	\\
12639.990234375	8.805887106053e-06	\\
12674.9816894531	0.000111216244309885	\\
12687.0941162109	9.37550413494595e-06	\\
12697.8607177734	0.000164795397500826	\\
12703.2440185547	9.45697381347492e-06	\\
12728.8146972656	0.00014336932835344	\\
12742.2729492188	2.80188583782564e-06	\\
12759.0957641602	1.73980766845546e-05	\\
12773.2269287109	0.000192877167432717	\\
12800.1434326172	0.000171317030888102	\\
12802.1621704102	3.04779371303408e-05	\\
12845.9014892578	0.000183968371548256	\\
12856.6680908203	1.77046087981043e-05	\\
12861.3784790039	0.000189335197208543	\\
12870.7992553711	8.00355190387376e-06	\\
12899.0615844727	0.000176433403303887	\\
12927.9968261719	1.29374017860527e-05	\\
12936.0717773438	0.000148351670841584	\\
12942.8009033203	7.60709290032836e-06	\\
12980.4840087891	7.43639895544282e-06	\\
12985.8673095703	0.000128718297679282	\\
13022.8775024414	1.84856390330171e-05	\\
13028.9337158203	0.000179965707278308	\\
13045.0836181641	0.000155322981927822	\\
13046.4294433594	1.07338948720245e-05	\\
13077.3834228516	8.07853878945437e-06	\\
13097.5708007813	0.000128497053572086	\\
13104.2999267578	1.47589677897438e-05	\\
13133.235168457	0.000149443117634359	\\
13157.4600219727	1.07673792543357e-05	\\
13163.5162353516	0.000154971168580089	\\
13188.4140014648	0.000141048489891483	\\
13193.1243896484	1.50712582076805e-05	\\
13207.2555541992	0.000131128424244805	\\
13230.1345825195	1.69049178478526e-06	\\
13239.5553588867	1.00463203518494e-05	\\
13244.938659668	0.000107231851895981	\\
13279.2572021484	0.000113920675221794	\\
13293.3883666992	9.26776479428722e-06	\\
13314.2486572266	0.000129260410983023	\\
13320.9777832031	5.37861616335226e-06	\\
13343.1838989258	0.000129683396094304	\\
13371.4462280273	7.2169786626502e-06	\\
13394.3252563477	0.00016054069536161	\\
13403.0731201172	8.36651802109735e-06	\\
13421.9146728516	0.000192058620121388	\\
13440.7562255859	6.13971730985227e-06	\\
13451.5228271484	0.0001737387233208	\\
13457.5790405273	4.28207943873435e-06	\\
13479.78515625	1.72199295421369e-05	\\
13505.3558349609	0.000151757937789503	\\
13515.4495239258	0.000147095052066876	\\
13535.6369018555	1.05123104348989e-05	\\
13545.7305908203	0.000110844532931952	\\
13553.8055419922	8.58691121654067e-06	\\
13596.1990356445	4.93088119691815e-06	\\
13612.3489379883	0.000120620894215575	\\
13636.5737915039	0.000104172205174409	\\
13647.3403930664	3.30735321305755e-06	\\
13654.7424316406	8.92234949741949e-05	\\
13670.2194213867	3.35755420376318e-06	\\
13686.3693237305	8.86530232540351e-05	\\
13705.2108764648	8.51577061964887e-06	\\
13720.0149536133	0.000108355780096311	\\
13740.202331543	1.11512757703831e-05	\\
13759.716796875	0.000154264659921563	\\
13773.1750488281	9.32672807582274e-06	\\
13789.3249511719	1.38127471264566e-05	\\
13807.4935913086	0.000113388066173488	\\
13822.9705810547	0.000148778468160623	\\
13839.1204833984	1.01504484960031e-05	\\
13856.6162109375	9.23318924805657e-05	\\
13885.5514526367	8.73828455545762e-07	\\
13910.44921875	1.35345382834441e-05	\\
13921.8887329102	0.00011365655268045	\\
13923.2345581055	0.000113395983614408	\\
13935.3469848633	6.71835897411495e-06	\\
13983.1237792969	9.43261217870602e-06	\\
13989.1799926758	0.00013593560238402	\\
13993.2174682617	8.17361647821147e-06	\\
13999.2736816406	0.000115260887081872	\\
14046.3775634766	0.00017833625533441	\\
14049.7421264648	4.48271836545188e-06	\\
14060.5087280273	0.000102831992711239	\\
14069.2565917969	1.62444251645645e-06	\\
14110.3042602539	4.20956373909123e-06	\\
14128.4729003906	9.80317195121091e-05	\\
14128.4729003906	9.80317195121091e-05	\\
14150.0061035156	6.37563347881713e-06	\\
14180.2871704102	9.23168592702769e-06	\\
14191.0537719727	0.000131321834002501	\\
14221.3348388672	1.21097753158875e-05	\\
14223.3535766602	9.45177559300211e-05	\\
14238.1576538086	0.00013687600216344	\\
14239.5034790039	1.22610163756891e-05	\\
14281.2240600586	8.73821501157479e-06	\\
14290.6448364258	9.85390594691495e-05	\\
14318.2342529297	0.000101793080686007	\\
14331.6925048828	1.05618987876928e-05	\\
14354.5715332031	2.38231103285695e-06	\\
14364.665222168	0.000119533192468343	\\
14368.7026977539	0.000101989900712045	\\
14383.5067749023	6.1252983489809e-06	\\
14429.9377441406	0.000102393631901491	\\
14432.6293945313	4.31636131565756e-06	\\
14454.1625976563	4.26658060874366e-06	\\
14461.5646362305	8.5485453666448e-05	\\
14491.1727905273	4.35014866865126e-06	\\
14498.5748291016	7.83612797924894e-05	\\
14529.5288085938	3.48014840896299e-06	\\
14534.2391967773	8.3794769368066e-05	\\
14554.426574707	5.89044879302617e-06	\\
14573.9410400391	0.000148009679538831	\\
14574.6139526367	8.76762439175253e-05	\\
14580.6701660156	1.03150430823233e-05	\\
14638.5406494141	5.02008525919312e-05	\\
14640.559387207	2.76344728647381e-06	\\
14645.2697753906	2.6219422106342e-06	\\
14666.130065918	5.41436760356096e-05	\\
14697.0840454102	4.02267497942475e-06	\\
14707.8506469727	6.50415293472675e-05	\\
14711.8881225586	6.70592747378647e-06	\\
14736.1129760742	6.68352770011354e-05	\\
14763.0294799805	5.63365880951595e-06	\\
14777.1606445313	7.99030103543393e-05	\\
14785.9085083008	6.55122206198713e-05	\\
14804.7500610352	8.88142781850254e-06	\\
14824.2645263672	5.18907395377494e-06	\\
14846.4706420898	7.13818506839952e-05	\\
14874.0600585938	5.90454084409529e-05	\\
14876.0787963867	3.20285630065193e-06	\\
14889.5370483398	1.78388793929562e-06	\\
14894.2474365234	5.28459311764669e-05	\\
14918.4722900391	3.41899741358318e-05	\\
14927.8930664063	3.78556847843745e-06	\\
14958.8470458984	3.47343715513244e-06	\\
14984.4177246094	3.35702612608453e-05	\\
14993.1655883789	6.62528378736302e-06	\\
15010.661315918	5.17294880125157e-05	\\
15030.8486938477	5.30244604544508e-06	\\
15052.3818969727	5.60302245286434e-05	\\
15063.8214111328	5.77749513987802e-05	\\
15064.4943237305	1.91249895160623e-06	\\
15089.3920898438	3.27677173091569e-05	\\
15116.30859375	3.03931023932046e-06	\\
15124.3835449219	4.12964505124517e-06	\\
15143.2250976563	6.17035230963585e-05	\\
15176.1978149414	5.7555832685459e-05	\\
15178.889465332	3.75391015312048e-06	\\
15192.3477172852	5.33061460533402e-05	\\
15193.0206298828	3.08598137067613e-06	\\
15226.6662597656	4.00118370821357e-05	\\
15233.3953857422	2.30146913555943e-06	\\
15264.3493652344	3.66530673915019e-05	\\
15285.2096557617	2.83283071909324e-06	\\
15305.3970336914	3.41566388833687e-05	\\
15327.6031494141	1.27653899151715e-06	\\
15332.3135375977	3.77361053839639e-05	\\
15363.2675170898	2.60456361497887e-06	\\
15376.725769043	2.51840273465407e-06	\\
15395.5673217773	4.33559814213042e-05	\\
15398.9318847656	4.6227409543976e-05	\\
15415.754699707	4.06350439508193e-06	\\
15431.9046020508	4.92273331865611e-06	\\
15458.1481933594	5.91605473066492e-05	\\
15474.9710083008	4.231104182496e-05	\\
15498.5229492188	4.35164160787502e-06	\\
15504.5791625977	4.80289148350333e-06	\\
15513.3270263672	5.51126820723796e-05	\\
15547.6455688477	5.02833754316449e-05	\\
15554.3746948242	3.94851737241043e-07	\\
15584.6557617188	3.24861309153838e-05	\\
15586.0015869141	6.35034381329048e-06	\\
15610.8993530273	3.19016149651678e-05	\\
15626.3763427734	1.56203862096587e-06	\\
15642.5262451172	1.74489441039285e-05	\\
15671.4614868164	1.91307582400708e-06	\\
15679.5364379883	3.57326944138098e-07	\\
15706.4529418945	1.58004879693879e-05	\\
15730.6777954102	1.73946786199804e-05	\\
15738.0798339844	1.85084093029679e-06	\\
15746.1547851563	5.16232827878984e-08	\\
15775.0900268555	1.65279132802591e-05	\\
15776.4358520508	3.22852464674469e-06	\\
15795.2774047852	2.38668569249608e-05	\\
15811.4273071289	3.42498794730221e-06	\\
15834.3063354492	2.27559241078025e-05	\\
15859.2041015625	2.90405477169664e-06	\\
15874.0081787109	4.32284849875089e-05	\\
15878.0456542969	4.19984156402493e-05	\\
15892.8497314453	2.66105235610178e-06	\\
15917.7474975586	2.15915296352507e-05	\\
15923.8037109375	1.04810361336841e-07	\\
15949.3743896484	1.77927065153031e-05	\\
15954.084777832	1.77059443587653e-06	\\
15980.3283691406	1.19598564186288e-05	\\
15985.7116699219	3.59085521995788e-06	\\
16014.6469116211	8.34820507122879e-06	\\
16019.3572998047	4.41329660987941e-06	\\
16050.3112792969	5.73482328819753e-06	\\
16083.283996582	3.80665022359134e-06	\\
16084.6298217773	4.39869238487087e-06	\\
16116.9296264648	2.21188178258546e-06	\\
16118.2754516602	2.73135024281996e-06	\\
16150.5752563477	9.04737314463388e-07	\\
16153.9398193359	1.31461559638535e-06	\\
16182.2021484375	1.61610787957383e-07	\\
16202.3895263672	2.65503294156071e-08	\\
16220.5581665039	5.70409258463374e-07	\\
16223.9227294922	4.06609834858876e-08	\\
16237.3809814453	6.4083445141654e-07	\\
16257.568359375	1.78126319494325e-08	\\
16261.6058349609	5.60449787493364e-07	\\
16296.5972900391	7.99214148737326e-08	\\
16315.4388427734	4.14460580336144e-07	\\
16339.6636962891	4.49679705368095e-07	\\
16350.4302978516	4.87260518224642e-08	\\
16378.0197143555	5.08028058810572e-07	\\
16386.0946655273	7.90883106857315e-08	\\
16423.7777709961	3.99712353150374e-07	\\
16425.7965087891	7.87081054402131e-08	\\
16429.1610717773	7.53498037330122e-08	\\
16434.5443725586	4.14693867091902e-07	\\
16481.6482543945	6.36097115131176e-08	\\
16487.7044677734	4.11794970014013e-07	\\
16497.1252441406	3.54548519643409e-07	\\
16522.6959228516	1.422451849628e-07	\\
16547.5936889648	3.53474055194259e-07	\\
16548.2666015625	1.1991982378313e-07	\\
16564.4165039063	5.98969033850076e-08	\\
16575.8560180664	4.05816085839876e-07	\\
16610.8474731445	4.93797019880455e-08	\\
16611.5203857422	4.01816590355943e-07	\\
16649.8764038086	9.6504938831708e-08	\\
16655.9326171875	4.38894622132837e-07	\\
16688.9053344727	4.15870379585596e-07	\\
16694.9615478516	1.14118499367926e-07	\\
16709.0927124023	4.16441754347175e-07	\\
16719.8593139648	1.06287860670789e-07	\\
16738.0279541016	3.73751596403448e-08	\\
16768.9819335938	3.72550686988028e-07	\\
16768.9819335938	3.72550686988028e-07	\\
16788.4963989258	6.95424122281892e-08	\\
16808.0108642578	6.37030854398885e-08	\\
16816.0858154297	3.9756246362258e-07	\\
16857.1334838867	3.69721186936926e-07	\\
16858.479309082	1.00799830760569e-07	\\
16882.03125	4.05866750606623e-07	\\
16885.3958129883	8.426679261309e-08	\\
16926.4434814453	1.46160759280656e-07	\\
16930.4809570313	4.29951315367795e-07	\\
16971.5286254883	1.5032269588197e-07	\\
16973.5473632813	3.80738403774251e-07	\\
16990.3701782227	3.62673225062512e-07	\\
16997.0993041992	7.65624837656518e-08	\\
17011.23046875	3.6327236409219e-07	\\
17023.3428955078	9.8150038510459e-08	\\
17065.0634765625	4.1673376675826e-07	\\
17071.7926025391	5.96474996310552e-08	\\
17089.9612426758	1.17072108639903e-07	\\
17098.0361938477	3.66295126079808e-07	\\
17118.896484375	3.97178806376956e-07	\\
17130.3359985352	1.27751035760043e-07	\\
17162.6358032227	3.84764304858661e-07	\\
17166.6732788086	1.4260604988912e-07	\\
17182.8231811523	1.18899606561696e-07	\\
17184.8419189453	3.84185229974216e-07	\\
17226.5625	3.90545831655902e-07	\\
17230.5999755859	8.52792803044327e-08	\\
17265.5914306641	4.03805211690101e-07	\\
17267.610168457	1.46633571927001e-07	\\
17304.6203613281	3.59631282983559e-07	\\
17318.0786132813	1.22601299910634e-07	\\
17332.209777832	4.02555469823762e-07	\\
17347.6867675781	9.56275314820556e-08	\\
17358.4533691406	3.52496482239382e-07	\\
17386.0427856445	8.28339524895692e-08	\\
17406.2301635742	3.72094999153424e-07	\\
17412.9592895508	1.03496439124013e-07	\\
17422.380065918	1.51677372585111e-07	\\
17434.4924926758	3.87102577214378e-07	\\
17474.1943359375	1.10565577987438e-07	\\
17488.3255004883	4.01282269271608e-07	\\
17493.0358886719	1.10391324742682e-07	\\
17503.1295776367	3.81748398294875e-07	\\
17529.3731689453	9.94049404611242e-08	\\
17535.4293823242	3.37114179954362e-07	\\
17568.4020996094	9.20458041346227e-08	\\
17587.9165649414	3.83751308377717e-07	\\
17601.3748168945	9.18965631712265e-08	\\
17617.5247192383	3.59055870935087e-07	\\
17628.9642333984	3.54186898037009e-07	\\
17652.5161743164	1.05751709838246e-07	\\
17684.1430664063	8.16066106000566e-08	\\
17687.5076293945	4.42919694621716e-07	\\
17707.0220947266	3.93490394223866e-07	\\
17725.8636474609	1.65759075345562e-07	\\
17733.2656860352	3.86615437761736e-07	\\
17748.0697631836	1.065211519058e-07	\\
17774.9862670898	4.33031550831711e-07	\\
17790.4632568359	6.91559185047358e-08	\\
17818.0526733398	1.40732575329642e-07	\\
17825.4547119141	3.68177397182437e-07	\\
17851.025390625	3.92578104382434e-07	\\
17857.0816040039	1.04209791985224e-07	\\
17884.6710205078	4.07853591741859e-08	\\
17888.0355834961	3.86111458663194e-07	\\
17908.2229614258	4.03244055876925e-07	\\
17931.1019897461	8.37184275474932e-08	\\
17941.8685913086	4.08621610188638e-07	\\
17945.9060668945	1.10608459494232e-07	\\
17970.1309204102	9.05626516696175e-08	\\
17980.8975219727	3.47675177815387e-07	\\
18009.8327636719	3.66748047659523e-07	\\
18021.272277832	4.79467244758693e-08	\\
18057.6095581055	4.40302000552998e-07	\\
18067.7032470703	6.89651641770298e-08	\\
18092.6010131836	1.2108128570204e-07	\\
18095.9655761719	3.30058703217133e-07	\\
18124.9008178711	1.00090027743805e-07	\\
18135.6674194336	3.73981646275967e-07	\\
18143.0694580078	4.06422246912364e-07	\\
18153.8360595703	7.96626933019395e-08	\\
18196.9024658203	3.59297623405518e-07	\\
18202.9586791992	8.32921528127118e-08	\\
18223.8189697266	1.5377648826348e-07	\\
18226.5106201172	3.90462347834168e-07	\\
18256.7916870117	1.47126224537233e-07	\\
18277.6519775391	4.11349318069753e-07	\\
18291.1102294922	4.17795636124358e-07	\\
18311.9705200195	9.09214813060613e-08	\\
18316.6809082031	1.23563106868846e-07	\\
18339.5599365234	3.66983415385228e-07	\\
18366.4764404297	3.98693680471337e-07	\\
18368.4951782227	1.2090393168761e-07	\\
18389.35546875	3.6073838890719e-07	\\
18395.4116821289	1.02793101448045e-07	\\
18419.6365356445	1.34146200852958e-07	\\
18429.7302246094	3.75106522388946e-07	\\
18476.1611938477	3.26240757268608e-07	\\
18480.1986694336	1.22101124844732e-07	\\
18514.5172119141	9.6299771096295e-08	\\
18518.5546875	3.9641285461776e-07	\\
18518.5546875	3.9641285461776e-07	\\
18549.5086669922	1.19496057551868e-07	\\
18562.2940063477	1.01274773902701e-07	\\
18575.7522583008	3.32654105345351e-07	\\
18610.7437133789	1.37419321432321e-07	\\
18620.8374023438	4.36611058730555e-07	\\
18620.8374023438	4.36611058730555e-07	\\
18636.3143920898	8.4983278161478e-08	\\
18671.305847168	4.63087988223235e-08	\\
18688.801574707	3.40260349776734e-07	\\
18698.2223510742	3.46120741038344e-07	\\
18717.7368164063	1.07340935216931e-07	\\
18724.4659423828	7.15297273217085e-08	\\
18731.867980957	3.76856714868987e-07	\\
18769.5510864258	4.64779642769242e-08	\\
18791.7572021484	4.24934178138767e-07	\\
18820.01953125	1.1019840804396e-07	\\
18824.7299194336	4.6501492324814e-07	\\
18850.9735107422	4.23141645383215e-07	\\
18851.6464233398	6.91709898855068e-08	\\
18874.5254516602	4.57722319812458e-08	\\
18890.0024414063	3.65954134948714e-07	\\
18916.2460327148	3.82429933006219e-07	\\
18919.6105957031	1.14366027227786e-07	\\
18934.4146728516	3.97100574372569e-07	\\
18946.5270996094	1.51430455772121e-07	\\
18964.6957397461	3.43934158266196e-07	\\
18990.266418457	1.10439382069854e-07	\\
19004.3975830078	3.55911118616172e-07	\\
19013.1454467773	1.11084021010242e-07	\\
19050.8285522461	1.11033172618566e-07	\\
19062.9409790039	4.45106775300468e-07	\\
19067.6513671875	3.90618282470202e-07	\\
19084.4741821289	8.21232267330172e-08	\\
19107.3532104492	3.94431269555155e-07	\\
19129.5593261719	5.54725220123038e-08	\\
19143.6904907227	1.30438259448859e-07	\\
19167.9153442383	3.56204756042888e-07	\\
19193.4860229492	1.08486166435863e-07	\\
19198.1964111328	3.48930940453586e-07	\\
19210.3088378906	1.09490371358159e-07	\\
19227.8045654297	3.46541147312045e-07	\\
19260.7772827148	3.65893504924329e-07	\\
19270.8709716797	1.11826428489704e-07	\\
19284.3292236328	8.41600724736396e-08	\\
19288.3666992188	3.90418264381988e-07	\\
19317.9748535156	6.59321201716806e-08	\\
19319.9935913086	3.52625349522346e-07	\\
19361.0412597656	3.77759269912734e-07	\\
19367.0974731445	1.4157485734447e-07	\\
19385.2661132813	3.54494859355899e-07	\\
19396.7056274414	1.26769261552634e-07	\\
19418.9117431641	3.43685186550353e-07	\\
19419.5846557617	7.29554242513802e-08	\\
19459.9594116211	3.72584115970152e-07	\\
19476.1093139648	4.03060815357187e-08	\\
19485.530090332	4.35367066211331e-07	\\
19507.7362060547	3.62019267780021e-08	\\
19525.2319335938	9.39108712523909e-08	\\
19545.4193115234	3.73762613665303e-07	\\
19550.8026123047	1.15692239830504e-07	\\
19552.8213500977	3.76100739785527e-07	\\
19586.4669799805	1.08066697077763e-07	\\
19603.2897949219	3.39186351991148e-07	\\
19640.299987793	7.26635464449281e-08	\\
19645.6832885742	3.75765547256686e-07	\\
19667.2164916992	3.39391314879332e-07	\\
19682.0205688477	8.44424042877478e-08	\\
19701.5350341797	1.28601931213246e-07	\\
19711.6287231445	4.04054900307515e-07	\\
19722.395324707	1.01081182281235e-07	\\
19737.1994018555	3.72093810592267e-07	\\
19775.5554199219	3.43440231523023e-07	\\
19777.5741577148	1.16842041614795e-07	\\
19793.0511474609	1.03537354237669e-07	\\
19807.1823120117	3.69248738410609e-07	\\
19824.6780395508	3.25014273629999e-07	\\
19834.098815918	1.01753421607045e-07	\\
19863.0340576172	3.57348438399691e-07	\\
19885.2401733398	9.91546323032478e-08	\\
19889.9505615234	1.41025148469736e-07	\\
19912.1566772461	3.19371246719156e-07	\\
19938.4002685547	1.36695809664077e-07	\\
19943.1106567383	3.48548554193303e-07	\\
19972.7188110352	3.90216140255276e-07	\\
19982.1395874023	1.08691890234109e-07	\\
20021.1685180664	1.14006736244986e-07	\\
20025.2059936523	3.69037790976786e-07	\\
20036.6455078125	1.12162708804804e-07	\\
20060.8703613281	3.7188295811386e-07	\\
20068.2723999023	4.14432339964358e-07	\\
20084.4223022461	8.38608000150474e-08	\\
20105.2825927734	7.05598193748253e-08	\\
20105.9555053711	4.07054446488978e-07	\\
20134.2178344727	1.23721282560179e-07	\\
20160.4614257813	4.08255606963125e-07	\\
20186.7050170898	6.0415456329682e-08	\\
20188.7237548828	3.60726749940999e-07	\\
20206.2194824219	3.7447891062985e-07	\\
20214.9673461914	1.31656307785339e-08	\\
20262.0712280273	9.9603703402689e-08	\\
20266.7816162109	4.65598342032292e-07	\\
20271.4920043945	1.22656005786413e-07	\\
20276.8753051758	3.10316273458091e-07	\\
20311.1938476563	3.73036077514851e-07	\\
20330.7083129883	1.44606380373816e-07	\\
20338.1103515625	1.1114084256101e-07	\\
20367.7185058594	3.89958896270642e-07	\\
20378.4851074219	8.16900077044538e-08	\\
20399.3453979492	4.7760893986232e-07	\\
20416.8411254883	6.90782863872105e-08	\\
20424.9160766602	3.63411498996303e-07	\\
20449.1409301758	1.20550394055524e-07	\\
20455.8700561523	3.36961279338074e-07	\\
20481.4407348633	7.2423601227242e-08	\\
20482.1136474609	3.65631697886002e-07	\\
20530.5633544922	3.55081003231078e-07	\\
20533.9279174805	5.1395000698029e-08	\\
20547.3861694336	3.41594554382022e-07	\\
20553.4423828125	9.97962188506782e-08	\\
20593.1442260742	1.09713273061425e-07	\\
20603.9108276367	3.00695650778287e-07	\\
20619.3878173828	3.3445956231734e-08	\\
20631.5002441406	4.11819267410811e-07	\\
20657.7438354492	3.80384650144777e-07	\\
20673.2208251953	1.13312196420522e-07	\\
20688.0249023438	3.4971220629894e-07	\\
20704.8477172852	1.11243325197348e-07	\\
20731.0913085938	1.38727485483372e-07	\\
20743.2037353516	3.55490611398962e-07	\\
20752.6245117188	4.74085596904803e-08	\\
20766.7556762695	3.81753752372583e-07	\\
20792.3263549805	1.945748895837e-08	\\
20811.1679077148	3.38970391963084e-07	\\
20822.607421875	3.82604897274254e-07	\\
20844.140625	5.86695767055402e-08	\\
20856.9259643555	1.0169880519938e-07	\\
20868.3654785156	3.84157509529605e-07	\\
20890.5715942383	3.75169563993196e-07	\\
20914.1235351563	4.12095477097117e-08	\\
20928.9276123047	3.8336395142524e-07	\\
20953.1524658203	5.98816118812052e-08	\\
20953.1524658203	5.98816118812052e-08	\\
20982.0877075195	3.6763858325643e-07	\\
21015.7333374023	1.27037933088316e-07	\\
21016.40625	3.42580374342395e-07	\\
21044.6685791016	7.04889912892505e-08	\\
21049.3789672852	3.76069048203345e-07	\\
21057.453918457	3.95385871007277e-07	\\
21084.3704223633	8.79558613101962e-08	\\
21109.2681884766	8.89681797484878e-08	\\
21118.0160522461	3.93204812866944e-07	\\
21147.624206543	1.11487296165618e-07	\\
21157.7178955078	3.15770168249618e-07	\\
21165.119934082	3.61558960454163e-07	\\
21191.3635253906	9.02417269791105e-08	\\
21197.4197387695	1.19690809588826e-07	\\
21199.4384765625	3.95890479311101e-07	\\
21241.1590576172	3.96390204163434e-07	\\
21257.9818725586	8.01976704691912e-08	\\
21270.0942993164	3.68414380350201e-07	\\
21274.1317749023	8.05053898859007e-08	\\
21315.1794433594	9.25537582160719e-08	\\
21329.3106079102	4.26834443838997e-07	\\
21329.9835205078	3.58588472391269e-07	\\
21358.2458496094	1.08954379949566e-07	\\
21377.0874023438	3.42450288247693e-08	\\
21395.9289550781	4.0742092523983e-07	\\
21410.0601196289	8.09494661829417e-08	\\
21421.4996337891	3.61627239743451e-07	\\
21434.2849731445	4.02619327301719e-08	\\
21446.3973999023	3.94976506708238e-07	\\
21483.4075927734	7.29830390265067e-08	\\
21487.4450683594	4.09318556881895e-07	\\
21502.9220581055	6.79694468139198e-08	\\
21517.0532226563	3.76939123301005e-07	\\
21542.6239013672	7.53186633750308e-08	\\
21561.4654541016	3.98825422571368e-07	\\
21575.5966186523	4.0712634142078e-07	\\
21585.0173950195	1.2984520060257e-07	\\
21608.5693359375	1.04198410715128e-07	\\
21619.3359375	4.38274164175424e-07	\\
21652.3086547852	3.72254656578291e-07	\\
21652.9815673828	3.02544919231868e-08	\\
21673.8418579102	9.61123287459883e-08	\\
21678.5522460938	3.61289186231312e-07	\\
21716.9082641602	1.07618022618989e-07	\\
21734.4039916992	3.66111765056784e-07	\\
21750.553894043	3.69269374696748e-07	\\
21755.9371948242	6.02821198026819e-08	\\
21781.5078735352	9.44900786364385e-08	\\
21795.6390380859	3.84747867370065e-07	\\
21828.6117553711	3.7278656531743e-07	\\
21832.649230957	1.08977346862599e-07	\\
21860.9115600586	3.47743158763715e-07	\\
21874.3698120117	7.97151370782496e-08	\\
21883.1176757813	3.975471434322e-07	\\
21899.9404907227	1.28937396431633e-07	\\
21931.5673828125	7.13870313059976e-08	\\
21936.9506835938	3.32534303241198e-07	\\
21974.6337890625	3.51363878842842e-07	\\
21980.0170898438	1.18122949727492e-08	\\
21988.0920410156	3.50190891106147e-07	\\
22007.6065063477	1.0284903357073e-07	\\
22050	4.04151845005364e-07	\\
};
\addlegendentry{X};

\end{axis}

\begin{axis}[%
width=\figurewidth,
height=\figureheight,
scale only axis,
xmin=0,
xmax=22050,
xlabel={Frequency (Hz)},
ymin=0,
ymax=0.549134354559093,
ylabel={|Y(f)|},
at=(plot7.right of south east),
anchor=left of south west,
title={Single-Sided Amplitude Spectrum of y(t)},
legend style={draw=black,fill=white,legend cell align=left}
]
\addplot [color=blue,solid]
  table[row sep=crcr]{
0	0.006798711423731	\\
1.3458251953125	0.000978872787980698	\\
31.6268920898438	0.0226608884646245	\\
34.991455078125	0.00944308993454663	\\
49.7955322265625	0.44588581931481	\\
72.0016479492188	0.189197504911061	\\
102.955627441406	0.0144212122434849	\\
102.955627441406	0.0144212122434849	\\
129.872131347656	0.243152144863286	\\
141.984558105469	0.549134354559093	\\
150.059509277344	0.00660355384917735	\\
187.069702148438	0.00721833211316051	\\
200.527954101563	0.112246851694893	\\
206.584167480469	0.0102047621448549	\\
238.211059570313	0.296335147822798	\\
246.286010742188	0.140507019640307	\\
270.510864257813	0.00977229472610137	\\
277.239990234375	0.00269539847675821	\\
287.333679199219	0.0781464206702811	\\
333.091735839844	0.0544119177989673	\\
342.512512207031	0.00607863094629569	\\
361.354064941406	0.0809134257938663	\\
374.139404296875	0.0100768540630196	\\
390.962219238281	0.00188214620638587	\\
405.766296386719	0.0591570588690396	\\
436.720275878906	0.0255483867121229	\\
441.4306640625	0.00102393799957861	\\
460.945129394531	0.000431768578595226	\\
469.692993164063	0.0263467769555899	\\
495.936584472656	0.00150823968270338	\\
510.740661621094	0.0334783402289915	\\
517.469787597656	0.00366202924511672	\\
539.675903320313	0.0361303767689421	\\
556.498718261719	0.00153098315641341	\\
573.994445800781	0.0218219379105264	\\
585.433959960938	0.000814483173669339	\\
613.6962890625	0.035514352677406	\\
623.117065429688	0.00255403291348167	\\
644.650268554688	0.0378936129721541	\\
670.220947265625	0.000728191218297736	\\
681.660461425781	0.0179335811737962	\\
689.0625	0.000243081333105277	\\
703.866577148438	0.0188493336650731	\\
733.474731445313	0.000363576286367586	\\
746.260070800781	0.00878315508141051	\\
755.680847167969	0.00820750923319436	\\
768.466186523438	0.000379998343274864	\\
789.999389648438	0.00838748184180105	\\
798.074340820313	0.000193136436038421	\\
831.719970703125	0.000427080720862511	\\
841.813659667969	0.0100315021201092	\\
866.038513183594	0.000313526085706034	\\
879.496765136719	0.00693318994312942	\\
913.815307617188	0.00561876080168671	\\
925.927734375	0.000333900688061041	\\
925.927734375	0.000333900688061041	\\
954.190063476563	0.00775597838319722	\\
981.779479980469	0.00712700926666545	\\
991.200256347656	0.00016838906281366	\\
1003.98559570313	0.0135732302851584	\\
1026.86462402344	0.000668402719961452	\\
1034.93957519531	0.000883249605994311	\\
1055.79986572266	0.00603457396451408	\\
1075.31433105469	0.000470787039040081	\\
1084.73510742188	0.00689355029528311	\\
1105.59539794922	0.000145128160320397	\\
1112.32452392578	0.0040725911530705	\\
1133.18481445313	0.000238106287088724	\\
1139.24102783203	0.00472484501193727	\\
1169.52209472656	0.000138406488350894	\\
1174.90539550781	0.0057121250855238	\\
1220.66345214844	0.00678520624159159	\\
1228.06549072266	0.000590846797766929	\\
1250.27160644531	0.000887229453209424	\\
1252.29034423828	0.00770287989869293	\\
1283.91723632813	0.0122969595834607	\\
1288.62762451172	0.000559670807182876	\\
1311.50665283203	0.000651931368824982	\\
1329.00238037109	0.011196287233785	\\
1345.15228271484	0.00716052822012629	\\
1361.97509765625	0.000270603101250299	\\
1374.76043701172	0.00813234409244722	\\
1405.04150390625	0.000429036706718411	\\
1419.84558105469	0.00557981979122162	\\
1429.93927001953	0.000266536448403471	\\
1448.10791015625	0.00581159931905329	\\
1474.35150146484	0.00012198563771293	\\
1485.791015625	0.00494661290785696	\\
1495.88470458984	8.37621552631557e-05	\\
1524.14703369141	0.00270164140396851	\\
1532.22198486328	0.000159545840252116	\\
1545.00732421875	0.000404290366390128	\\
1546.35314941406	0.00468174164861576	\\
1606.91528320313	0.00651434781977402	\\
1612.29858398438	0.000420617565051333	\\
1612.29858398438	0.000420617565051333	\\
1627.77557373047	0.00920558386182302	\\
1658.056640625	0.00894594054305706	\\
1676.89819335938	0.000729955120186908	\\
1697.75848388672	0.000287067563800221	\\
1711.8896484375	0.00473118909915422	\\
1722.65625	0.00665340925257676	\\
1730.73120117188	0.000184116926794132	\\
1761.01226806641	0.00912552169814212	\\
1778.50799560547	0.000168919010590941	\\
1786.58294677734	0.00800740895566811	\\
1812.82653808594	8.66065079646033e-05	\\
1822.92022705078	0.00464788502154816	\\
1835.03265380859	0.000222377133855616	\\
1855.89294433594	0.00507609924675146	\\
1882.80944824219	0.000159754137017605	\\
1890.21148681641	0.000215849916875136	\\
1891.55731201172	0.00444554033443985	\\
1940.00701904297	0.000239732839313122	\\
1952.79235839844	0.00536827647136805	\\
1962.21313476563	0.0064394722785391	\\
1986.43798828125	0.00012943084694586	\\
1997.87750244141	0.00722111954135698	\\
2020.08361816406	0.000346534536174641	\\
2024.12109375	0.00665851967230336	\\
2041.61682128906	0.000513151151385467	\\
2061.80419921875	0.00832874760736051	\\
2079.97283935547	0.000229895476897578	\\
2104.87060546875	0.00387283676076092	\\
2113.61846923828	0.000485133217714324	\\
2132.46002197266	0.000362290238351591	\\
2153.3203125	0.00503000739808281	\\
2184.94720458984	0.00948457488938696	\\
2193.02215576172	0.000270486872983085	\\
2209.17205810547	0.000178411507524341	\\
2227.34069824219	0.00695919958607194	\\
2232.05108642578	0.00100303660935564	\\
2256.27593994141	0.00808897379183358	\\
2271.08001708984	0.000211608342242908	\\
2292.61322021484	0.00821360966668338	\\
2318.85681152344	0.00617941544619292	\\
2326.25885009766	0.000262258200725127	\\
2341.06292724609	0.000632782726998152	\\
2358.55865478516	0.00577249725837544	\\
2367.97943115234	0.00806354311663559	\\
2387.49389648438	0.000303956433920151	\\
2405.66253662109	0.0049409617166818	\\
2422.4853515625	0.000409044918711114	\\
2462.86010742188	0.000335859499285266	\\
2468.91632080078	0.00585369002404003	\\
2469.58923339844	0.00546744563003558	\\
2485.73913574219	7.53563639865148e-05	\\
2504.58068847656	0.00399719670256667	\\
2512.65563964844	0.000135467669648227	\\
2543.60961914063	0.000154870615325256	\\
2565.81573486328	0.00378492488835514	\\
2578.60107421875	8.3845027356169e-05	\\
2603.49884033203	0.00627043795458916	\\
2613.59252929688	0.000357857863801537	\\
2633.10699462891	0.00660902820068553	\\
2663.38806152344	0.000420827601163867	\\
2666.07971191406	0.00611051841431698	\\
2683.57543945313	0.00489303617890414	\\
2700.39825439453	5.79423060711011e-05	\\
2727.31475830078	0.000292344482785101	\\
2740.10009765625	0.00507793986674186	\\
2754.90417480469	0.000121676933634435	\\
2776.43737792969	0.00348886017789779	\\
2800.66223144531	0.00319142349118144	\\
2811.42883300781	8.73980292670375e-05	\\
2828.92456054688	7.22232725447007e-05	\\
2839.69116210938	0.00267915941065135	\\
2854.49523925781	0.000155917018297904	\\
2871.99096679688	0.00378375558337752	\\
2890.15960693359	0.000163887235785705	\\
2898.90747070313	0.00320976785535577	\\
2915.73028564453	0.000189061697964588	\\
2944.66552734375	0.00380162280152044	\\
2958.79669189453	0.000467246712895959	\\
2966.19873046875	0.0031919446590026	\\
2990.42358398438	4.25987536897985e-05	\\
3012.62969970703	0.00414316374947861	\\
3035.50872802734	0.000152555126911696	\\
3040.21911621094	0.00313782750386949	\\
3053.00445556641	0.000138955525647364	\\
3077.90222167969	0.00357901789952368	\\
3097.41668701172	0.00221660603669509	\\
3100.10833740234	0.000157950912112558	\\
3131.73522949219	0.00232203471186462	\\
3143.17474365234	8.20883917307384e-05	\\
3163.36212158203	0.000111776927418703	\\
3188.93280029297	0.00179960031434614	\\
3194.31610107422	0.000183625687616447	\\
3223.92425537109	0.00336762475048971	\\
3236.03668212891	0.00343588692771165	\\
3250.16784667969	4.14037731989168e-05	\\
3259.58862304688	0.00029495113252279	\\
3283.8134765625	0.00353611211045007	\\
3297.27172851563	0.00352914161192906	\\
3304.00085449219	0.000292048268700255	\\
3340.33813476563	0.00307600114690548	\\
3345.04852294922	9.20419780235618e-05	\\
3367.92755126953	0.000189601395060169	\\
3374.65667724609	0.00335285430152935	\\
3399.55444335938	0.000178456208784531	\\
3410.99395751953	0.00404573158002699	\\
3434.5458984375	6.30355188868163e-05	\\
3446.65832519531	0.00235824151181751	\\
3480.97686767578	0.00258428516640851	\\
3485.01434326172	4.94366273286566e-05	\\
3511.93084716797	0.00271022998273936	\\
3528.75366210938	0.000138297283770262	\\
3549.61395263672	0.0024660232635466	\\
3556.34307861328	4.49956196622812e-05	\\
3577.20336914063	0.00211628674078702	\\
3579.22210693359	0.000117820111435362	\\
3616.90521240234	0.00359054047499519	\\
3622.28851318359	9.91177894461693e-05	\\
3650.55084228516	0.00330168878729172	\\
3666.70074462891	0.000279785864459468	\\
3682.85064697266	0.00354172208840238	\\
3697.65472412109	0.000425116276817417	\\
3704.38385009766	0.000233667218653758	\\
3711.11297607422	0.00280876171487155	\\
3751.48773193359	0.00353579950916752	\\
3766.96472167969	0.000127529177967332	\\
3784.46044921875	0.00282138815892594	\\
3790.51666259766	2.52624157181863e-05	\\
3812.04986572266	0.00019297380403349	\\
3816.08734130859	0.00159235592353624	\\
3851.75170898438	0.00153281086722136	\\
3872.61199951172	4.31248742677336e-05	\\
3880.01403808594	6.89942380541485e-05	\\
3909.62219238281	0.00121843441733861	\\
3923.75335693359	0.000170122651422042	\\
3943.26782226563	0.0025193998730768	\\
3954.03442382813	0.00318066598539075	\\
3966.81976318359	0.00021799479494792	\\
3989.02587890625	0.00301126298927535	\\
4003.15704345703	0.000144794282807169	\\
4013.25073242188	0.00223335859803565	\\
4046.89636230469	0.00015044155516339	\\
4046.89636230469	0.00015044155516339	\\
4070.44830322266	0.00177177739090315	\\
4091.30859375	0.00172512344851614	\\
4108.13140869141	0.000225656892777377	\\
4127.64587402344	9.38621971995048e-05	\\
4149.17907714844	0.00173352234096939	\\
4159.94567871094	8.52141868496667e-05	\\
4164.65606689453	0.0014618149016907	\\
4198.30169677734	1.88698144188912e-05	\\
4210.41412353516	0.00148982015781048	\\
4224.54528808594	0.00200721930289006	\\
4252.13470458984	2.11356827695239e-05	\\
4257.51800537109	0.00179754527804156	\\
4282.41577148438	9.83399634393721e-05	\\
4288.47198486328	0.00247242782152562	\\
4293.85528564453	0.000309238656537613	\\
4332.88421630859	0.000227653244092505	\\
4347.01538085938	0.00171771794136445	\\
4367.20275878906	0.000134004854283125	\\
4379.31518554688	0.00201257351427779	\\
4412.96081542969	0.00021679867940451	\\
4422.38159179688	0.0017572397252715	\\
4425.74615478516	0.00203304125874709	\\
4457.373046875	9.22090304682305e-05	\\
4466.12091064453	0.0017210566003608	\\
4484.28955078125	8.95258045366167e-05	\\
4508.51440429688	0.000974025770134367	\\
4510.53314208984	0.000128530008778165	\\
4529.37469482422	0.000139326765476642	\\
4556.29119873047	0.00176591402353953	\\
4561.67449951172	0.0016857866672295	\\
4562.34741210938	0.000114352920057329	\\
4611.47003173828	8.4245467154349e-05	\\
4617.52624511719	0.0016220781273006	\\
4631.65740966797	0.00191122710535098	\\
4644.44274902344	0.000122648219007382	\\
4667.99468994141	0.00156109761512966	\\
4688.85498046875	2.46245002041199e-05	\\
4707.02362060547	0.000162887403815669	\\
4730.57556152344	0.00166951883319502	\\
4744.70672607422	6.21086234298021e-05	\\
4760.18371582031	0.00117652870728806	\\
4768.25866699219	7.56224832607204e-05	\\
4799.88555908203	0.00173943663725331	\\
4816.70837402344	0.000104465129943201	\\
4830.16662597656	0.00163822786013566	\\
4853.71856689453	9.90317648947021e-05	\\
4861.79351806641	0.00123855908411323	\\
4879.96215820313	0.000107974911203166	\\
4894.09332275391	0.000875513029163197	\\
4929.08477783203	0.00146131675724318	\\
4930.43060302734	4.1819593393623e-05	\\
4949.27215576172	0.00128149580983933	\\
4970.80535888672	0.000196514434518928	\\
4976.86157226563	5.96803316526732e-05	\\
5005.12390136719	0.00129407331062563	\\
5007.14263916016	0.000136484127809626	\\
5025.31127929688	0.00130513483168428	\\
5042.80700683594	0.00172388534371215	\\
5053.57360839844	8.60392633779006e-05	\\
5107.40661621094	0.00130408353500214	\\
5109.42535400391	7.37145932311878e-05	\\
5118.84613037109	0.000102159203421977	\\
5139.70642089844	0.00116426591409139	\\
5147.78137207031	5.23560939660609e-05	\\
5155.18341064453	0.000801770068211226	\\
5188.15612792969	2.36592652236188e-05	\\
5191.52069091797	0.000680451620322607	\\
5229.87670898438	0.000915700535026698	\\
5233.91418457031	6.16997665377097e-05	\\
5272.94311523438	0.000786083349042773	\\
5276.98059082031	3.43320490916908e-05	\\
5298.51379394531	0.00135804420399849	\\
5311.97204589844	5.79790900071027e-05	\\
5337.54272460938	0.000888606784189182	\\
5348.98223876953	1.88970017012866e-05	\\
5374.55291748047	0.00111221256342866	\\
5379.26330566406	4.66595064524158e-05	\\
5401.46942138672	0.0011086542877382	\\
5408.19854736328	8.94641273856649e-05	\\
5424.34844970703	0.000883909860186904	\\
5441.84417724609	7.18927444170742e-05	\\
5453.28369140625	0.000819652488619425	\\
5475.48980712891	3.13511596159855e-05	\\
5492.31262207031	4.08120842651921e-05	\\
5506.44378662109	0.000548604756984735	\\
5544.12689208984	0.000786528877504827	\\
5554.22058105469	8.46961603991477e-05	\\
5570.37048339844	0.000719517476772696	\\
5581.80999755859	1.85640207061136e-05	\\
5601.32446289063	6.30479340321774e-05	\\
5606.03485107422	0.000743109487741436	\\
5652.4658203125	0.000994689010382961	\\
5654.48455810547	2.28870684869573e-05	\\
5676.69067382813	2.43751273731498e-05	\\
5678.70941162109	0.000998979223701898	\\
5698.89678955078	1.3809380076754e-05	\\
5715.04669189453	0.000853246407036976	\\
5729.17785644531	8.98275764457675e-05	\\
5745.32775878906	0.000945144982928802	\\
5778.30047607422	0.000970534331477223	\\
5789.73999023438	7.60620191271968e-05	\\
5795.79620361328	0.000657542299460545	\\
5824.05853271484	3.1977746549571e-05	\\
5840.88134765625	0.001498086777515	\\
5845.59173583984	6.65063770112615e-05	\\
5883.27484130859	0.000465309393868272	\\
5892.02270507813	2.6265479214384e-05	\\
5899.42474365234	3.2304885555903e-05	\\
5908.17260742188	0.000711987358998057	\\
5962.00561523438	3.21020220402917e-05	\\
5966.71600341797	0.000704092080470785	\\
5967.38891601563	0.000656274417664642	\\
5998.34289550781	2.12275234629452e-05	\\
6012.47406005859	0.000816734050515844	\\
6020.54901123047	5.25144067614016e-05	\\
6043.42803955078	0.000893140706547175	\\
6048.13842773438	4.53350567082056e-05	\\
6079.09240722656	2.14219260580754e-05	\\
6095.91522216797	0.000766808497916732	\\
6110.71929931641	5.36815131839397e-05	\\
6116.77551269531	0.000764273791690984	\\
6143.69201660156	0.000741004472198345	\\
6149.74822998047	3.51642487481725e-05	\\
6178.68347167969	0.000828806139658164	\\
6190.12298583984	6.2964837573502e-05	\\
6219.05822753906	0.000550642073933626	\\
6227.13317871094	5.55033700681161e-05	\\
6245.30181884766	4.35201059247828e-05	\\
6247.99346923828	0.000723195974145828	\\
6278.94744873047	0.00061761678073402	\\
6291.73278808594	4.01493886060199e-05	\\
6320.66802978516	2.12514677104805e-05	\\
6336.14501953125	0.000730142445553445	\\
6371.13647460938	1.72812545782072e-05	\\
6377.19268798828	0.000810791114229109	\\
6379.21142578125	0.000687483599638855	\\
6396.70715332031	2.44439080942724e-05	\\
6440.44647216797	4.87761263333343e-05	\\
6447.17559814453	0.000670915239460951	\\
6447.17559814453	0.000670915239460951	\\
6470.7275390625	3.19030723353273e-05	\\
6488.22326660156	3.2388903374743e-05	\\
6510.42938232422	0.000415743322978209	\\
6540.71044921875	4.74810176615961e-05	\\
6549.45831298828	0.00035287582235599	\\
6575.70190429688	4.57766767871919e-05	\\
6578.3935546875	0.000359618513966238	\\
6601.94549560547	0.000348964730977944	\\
6612.03918457031	6.73847710179129e-06	\\
6638.95568847656	6.57750346554632e-05	\\
6642.9931640625	0.00048871718213949	\\
6658.47015380859	9.23839626442514e-06	\\
6661.16180419922	0.000412413303642613	\\
6693.46160888672	0.000630636596691492	\\
6702.88238525391	4.21088650878984e-05	\\
6721.72393798828	0.00046207240419139	\\
6746.62170410156	2.50927482643855e-05	\\
6756.04248046875	4.37826482690074e-05	\\
6758.73413085938	0.000547489393295622	\\
6816.60461425781	1.00925813624203e-05	\\
6824.00665283203	0.000328106700533539	\\
6852.26898193359	0.00038513027153013	\\
6856.30645751953	7.66477931380692e-06	\\
6861.01684570313	0.000456134219234323	\\
6889.27917480469	8.19742656164148e-06	\\
6895.33538818359	2.40516976652129e-05	\\
6912.83111572266	0.000410905158663377	\\
6932.34558105469	2.11704920235322e-05	\\
6958.58917236328	0.000437245038986425	\\
6980.12237548828	1.67066296694105e-05	\\
6990.21606445313	0.000423966494918907	\\
6998.291015625	3.09470635972147e-05	\\
7013.09509277344	0.000479440851154226	\\
7047.41363525391	0.000425590721629268	\\
7049.43237304688	5.0328258405018e-05	\\
7068.94683837891	5.52591678477909e-05	\\
7089.80712890625	0.000577875336353003	\\
7107.30285644531	1.35739297945726e-05	\\
7127.490234375	0.00034114046233218	\\
7134.89227294922	1.33752211072538e-05	\\
7164.50042724609	0.000347808225990854	\\
7176.61285400391	1.29372202578912e-05	\\
7201.51062011719	0.000412629040940435	\\
7201.51062011719	0.000412629040940435	\\
7219.00634765625	2.01232428980168e-05	\\
7244.57702636719	8.83169510860189e-06	\\
7268.80187988281	0.000474593920583915	\\
7270.82061767578	0.000376550566924014	\\
7274.85809326172	1.46335939099871e-05	\\
7312.54119873047	2.45775497889428e-05	\\
7333.40148925781	0.000421640893240274	\\
7359.64508056641	2.86983235522241e-05	\\
7360.31799316406	0.000316593740682293	\\
7383.86993408203	2.24662220162095e-05	\\
7397.32818603516	0.000368563860419597	\\
7409.44061279297	0.000375826299141318	\\
7437.70294189453	3.22710868008879e-05	\\
7457.89031982422	2.58918361659589e-05	\\
7470.00274658203	0.000255034187469258	\\
7491.53594970703	0.000253959678306543	\\
7504.3212890625	1.96964406357576e-05	\\
7516.43371582031	2.11308238166169e-05	\\
7533.25653076172	0.00026564535302265	\\
7568.9208984375	0.000254446024468541	\\
7571.61254882813	1.21330414853765e-05	\\
7593.14575195313	1.66110516645054e-05	\\
7599.20196533203	0.000320483111651466	\\
7622.75390625	1.58850689957842e-05	\\
7632.17468261719	0.000367117125753702	\\
7659.09118652344	0.00025307040486727	\\
7681.97021484375	2.07845867006512e-05	\\
7693.40972900391	2.74413198707772e-06	\\
7708.21380615234	0.000306420606881789	\\
7724.36370849609	3.361022141379e-05	\\
7731.76574707031	0.000320555035562789	\\
7752.62603759766	3.91385256722673e-05	\\
7764.73846435547	0.000368006851639698	\\
7804.44030761719	0.000370757479478833	\\
7815.20690917969	3.11833075608556e-05	\\
7834.72137451172	0.000253490183483383	\\
7836.74011230469	1.87576543196702e-05	\\
7871.73156738281	2.61845624556441e-05	\\
7887.20855712891	0.00019104423406277	\\
7901.33972167969	8.25864059390157e-06	\\
7904.03137207031	0.000244147915000864	\\
7937.00408935547	1.43846251584721e-05	\\
7947.77069091797	0.000186523152374507	\\
7978.72467041016	3.24034488746397e-05	\\
7990.16418457031	0.000151884289600849	\\
8007.65991210938	0.000222454309071568	\\
8011.69738769531	1.06882473111334e-05	\\
8033.90350341797	0.000344495267248298	\\
8048.03466796875	1.94299502211047e-05	\\
8075.62408447266	6.26131753777384e-06	\\
8083.69903564453	0.000259755536585964	\\
8099.84893798828	0.000246860281400701	\\
8106.57806396484	2.06978715429145e-05	\\
8132.82165527344	0.000237428833294061	\\
8155.02777099609	1.48841953421158e-05	\\
8179.92553710938	2.04854192916173e-05	\\
8186.65466308594	0.000183379910750944	\\
8205.49621582031	0.000219478691004405	\\
8210.87951660156	1.35301723154009e-05	\\
8237.79602050781	0.000168869299960334	\\
8241.83349609375	1.47325245541522e-05	\\
8270.76873779297	0.000265074707681314	\\
8298.35815429688	1.15688080366688e-05	\\
8303.74145507813	0.00022793839119314	\\
8319.21844482422	1.19761515266332e-05	\\
8334.02252197266	1.08863198938896e-05	\\
8354.8828125	0.000207187943482088	\\
8375.74310302734	1.10952397736807e-05	\\
8381.79931640625	0.000248030028516343	\\
8418.13659667969	0.000209700329055526	\\
8428.90319824219	1.01004718205484e-05	\\
8436.97814941406	5.1503102657637e-06	\\
8445.72601318359	0.000168650034364668	\\
8482.06329345703	0.000157237464703389	\\
8484.08203125	1.82132908603831e-05	\\
8506.96105957031	0.000155948697860706	\\
8513.69018554688	6.99686717832081e-06	\\
8556.08367919922	0.000148912922607287	\\
8568.19610595703	1.53271912997754e-05	\\
8599.15008544922	5.2880103197633e-06	\\
8606.55212402344	0.000138693619178771	\\
8632.79571533203	0.000174701639264151	\\
8634.814453125	5.35754025704214e-06	\\
8661.05804443359	0.000162876938235165	\\
8668.46008300781	6.50250477932672e-06	\\
8678.55377197266	1.6012703479605e-05	\\
8702.77862548828	0.000180777127964908	\\
8714.89105224609	0.00040401452861627	\\
8739.11590576172	1.73316547356658e-05	\\
8753.2470703125	1.47362153826699e-05	\\
8755.26580810547	0.000218998361422971	\\
8792.27600097656	0.00017454809520279	\\
8805.06134033203	2.74378939345202e-06	\\
8819.19250488281	0.000121461518266517	\\
8845.43609619141	2.3561579038123e-06	\\
8869.66094970703	6.22564079078674e-06	\\
8872.35260009766	0.000135581155554485	\\
8888.50250244141	1.00413219073404e-05	\\
8897.92327880859	0.000168623491470174	\\
8922.14813232422	8.13723257323915e-06	\\
8945.02716064453	0.000137323491248996	\\
8975.98114013672	3.13902525907712e-07	\\
8979.345703125	0.000182418170096285	\\
8994.14978027344	3.30128033693665e-06	\\
9000.20599365234	0.000195054305142476	\\
9029.81414794922	0.000250393115486458	\\
9052.69317626953	1.54926076136e-05	\\
9056.05773925781	9.15548413121695e-06	\\
9060.09521484375	0.000194122197576997	\\
9089.03045654297	1.03205787187163e-05	\\
9113.25531005859	0.000162944891822019	\\
9123.34899902344	0.000150717179710505	\\
9142.86346435547	1.57624881993778e-05	\\
9181.89239501953	5.82417584523759e-06	\\
9187.27569580078	0.000148398712078385	\\
9195.35064697266	0.000186413770551086	\\
9208.13598632813	7.13442208294853e-06	\\
9233.03375244141	1.29043853543065e-05	\\
9237.07122802734	0.00012381878083275	\\
9278.79180908203	1.05848870842019e-05	\\
9288.21258544922	0.000125640765316649	\\
9303.68957519531	0.000146262162971955	\\
9325.89569091797	1.60457043526441e-05	\\
9346.08306884766	3.85082786595261e-06	\\
9352.13928222656	0.000181722779877004	\\
9368.28918457031	0.000226371205958144	\\
9381.07452392578	1.73723296633661e-05	\\
9412.02850341797	0.000141589937239515	\\
9428.85131835938	1.28386183326365e-05	\\
9437.59918212891	0.000176130828585375	\\
9458.45947265625	8.0189159852722e-06	\\
9467.20733642578	0.000163040142666589	\\
9475.28228759766	1.13778726167111e-05	\\
9527.76947021484	0.000158819963873382	\\
9533.82568359375	3.38950688039494e-06	\\
9533.82568359375	3.38950688039494e-06	\\
9538.53607177734	0.000141274747345306	\\
9583.62121582031	1.39828329667776e-05	\\
9592.36907958984	0.000172716504832969	\\
9605.15441894531	6.44843005891191e-06	\\
9626.01470947266	0.000131089990267477	\\
9638.12713623047	0.00014063127867984	\\
9656.29577636719	5.17034559815485e-06	\\
9679.84771728516	2.09822158400604e-06	\\
9691.96014404297	0.000194409917567377	\\
9715.51208496094	8.9662778373492e-06	\\
9737.04528808594	0.000181363193919255	\\
9743.10150146484	1.02950010038074e-05	\\
9755.21392822266	0.000189152035375893	\\
9778.09295654297	0.000156571380799903	\\
9789.53247070313	1.97840015056591e-05	\\
9833.94470214844	0.000163186643783103	\\
9840.673828125	1.14107973903745e-05	\\
9848.74877929688	6.91569615848369e-06	\\
9867.59033203125	0.000131532773801957	\\
9884.41314697266	0.00014515018725026	\\
9889.12353515625	8.18494976251677e-06	\\
9912.67547607422	4.76065925701053e-06	\\
9943.62945556641	9.22779496349759e-05	\\
9963.81683349609	0.000167853587419307	\\
9969.20013427734	8.5573338996062e-06	\\
9985.35003662109	1.85069929100034e-05	\\
10002.8457641602	0.000131253802703372	\\
10031.1080932617	1.36381694662626e-05	\\
10042.5476074219	0.000152713545704586	\\
10052.6412963867	1.1175306788287e-05	\\
10075.520324707	0.000137839440102439	\\
10088.9785766602	0.00013997698799925	\\
10093.6889648438	1.03011230042412e-05	\\
10127.3345947266	0.000119692368412318	\\
10139.4470214844	5.52285746855183e-06	\\
10151.5594482422	0.000124692092086273	\\
10160.9802246094	3.23590295398662e-06	\\
10200.0091552734	0.000107507824825785	\\
10204.0466308594	1.00983648717192e-05	\\
10225.5798339844	9.91307010248588e-06	\\
10237.6922607422	0.000133157833286847	\\
10270.6649780273	3.52160636408275e-06	\\
10271.337890625	0.000108416130821209	\\
10294.2169189453	0.000139841158839315	\\
10310.3668212891	1.46878956953288e-05	\\
10329.2083740234	0.000129407784420756	\\
10355.451965332	6.79306580991513e-06	\\
10362.8540039063	1.71628274145478e-05	\\
10374.2935180664	0.000165542055767047	\\
10399.8641967773	7.9206028116126e-06	\\
10405.9204101563	0.000126551687478084	\\
10426.7807006836	7.18266092191662e-06	\\
10446.9680786133	0.000160969081459752	\\
10465.13671875	0.000131770061479228	\\
10489.3615722656	4.21992547374961e-06	\\
10496.7636108398	1.52170998653168e-05	\\
10511.5676879883	0.000136775902957262	\\
10530.4092407227	0.000143958294561535	\\
10560.6903076172	5.36366369996266e-06	\\
10568.0923461914	0.000112710901207561	\\
10569.4381713867	2.1941421020948e-05	\\
10611.1587524414	8.46060195976019e-06	\\
10625.2899169922	0.000133462081879533	\\
10630.6732177734	0.000120270407059682	\\
10636.0565185547	1.21134255489339e-05	\\
10665.6646728516	0.000104661295810842	\\
10673.7396240234	1.08556931526385e-05	\\
10704.6936035156	0.000151865144346087	\\
10728.9184570313	7.18604284684186e-06	\\
10736.9934082031	0.000119669881870166	\\
10752.4703979492	9.3007964278308e-06	\\
10774.0036010742	0.000103660983959016	\\
10785.4431152344	5.76792262486115e-06	\\
10814.3783569336	1.5195616487862e-05	\\
10825.8178710938	0.000118857270939579	\\
10855.4260253906	0.000138248723069759	\\
10866.1926269531	9.05183097529383e-06	\\
10896.4736938477	0.000120128050646557	\\
10900.5111694336	7.69275069215741e-06	\\
10913.9694213867	0.000144437457577886	\\
10927.4276733398	4.292976525327e-06	\\
10943.5775756836	0.000135907586463776	\\
10963.0920410156	5.23895855471565e-06	\\
10981.93359375	0.000148526715148549	\\
11000.1022338867	4.64842694487046e-06	\\
11008.8500976563	6.03434694994078e-06	\\
11022.981262207	0.000108052946209171	\\
11063.3560180664	3.48115080227639e-06	\\
11073.4497070313	0.00015061582760254	\\
11079.5059204102	0.000135762624609278	\\
11104.4036865234	1.22958210390172e-05	\\
11126.6098022461	0.000102621315143327	\\
11136.0305786133	5.62224222052385e-06	\\
11150.8346557617	0.000119131906158059	\\
11163.6199951172	9.88840878673214e-06	\\
11185.8261108398	4.55077449170782e-06	\\
11187.1719360352	9.73159420061838e-05	\\
11236.9674682617	0.000105956993606549	\\
11241.0049438477	8.95102644496168e-06	\\
11257.8277587891	2.84185137682636e-06	\\
11282.0526123047	0.000112938555759068	\\
11292.1463012695	1.37268911102527e-05	\\
11311.6607666016	9.19809750521252e-05	\\
11338.5772705078	8.13008294924061e-06	\\
11350.016784668	9.34537303390595e-05	\\
11372.8958129883	1.32724868820015e-05	\\
11375.5874633789	0.000166474774855261	\\
11407.8872680664	8.6835778590649e-06	\\
11417.3080444336	0.000106021688599925	\\
11426.0559082031	0.000143827276508786	\\
11436.149597168	4.081229880882e-06	\\
11457.682800293	0.000181155562558934	\\
11486.6180419922	2.16969260935665e-06	\\
11490.6555175781	0.000134601054871859	\\
11492.6742553711	4.71710447208229e-06	\\
11533.7219238281	8.40924444759046e-05	\\
11535.7406616211	7.58961309412338e-06	\\
11559.9655151367	0.000107260030684563	\\
11574.7695922852	5.41575863790155e-06	\\
11609.0881347656	0.000113681523550367	\\
11616.4901733398	1.00787754220495e-06	\\
11637.3504638672	5.43582968128719e-06	\\
11653.5003662109	9.68868415104537e-05	\\
11675.7064819336	3.13514258387038e-06	\\
11693.8751220703	9.21395791525059e-05	\\
11707.3333740234	8.44989192820667e-06	\\
11722.1374511719	0.000110834850816302	\\
11747.7081298828	0.000115352928348934	\\
11757.12890625	5.88913642630224e-06	\\
11782.6995849609	4.96940369613718e-06	\\
11790.1016235352	0.000139972606979899	\\
11800.8682250977	1.63820224899585e-06	\\
11819.709777832	0.000139547299124524	\\
11835.8596801758	7.48230651367425e-06	\\
11862.776184082	8.85695320278789e-05	\\
11872.8698730469	1.00168222798482e-05	\\
11887.6739501953	0.000103827634353285	\\
11918.6279296875	8.84830135830105e-06	\\
11921.3195800781	8.58775153232684e-05	\\
11940.1611328125	1.87277406184414e-05	\\
11958.3297729492	0.000149458483268803	\\
11982.5546264648	0.000106233880687865	\\
11998.7045288086	5.50614905557526e-06	\\
12012.8356933594	0.000111608572837425	\\
12015.52734375	5.78392952369697e-06	\\
12050.5187988281	0.000104942071584231	\\
12068.6874389648	5.00921584985357e-06	\\
12083.4915161133	0.000139711319543875	\\
12084.1644287109	7.09538805856136e-06	\\
12127.2308349609	6.73335618578829e-06	\\
12131.2683105469	0.000118791810038024	\\
12164.241027832	0.000131181182119268	\\
12172.3159790039	8.40516534291055e-06	\\
12177.0263671875	5.17051608552561e-06	\\
12185.774230957	0.000104603975881891	\\
12210.6719970703	9.11952828962063e-05	\\
12229.5135498047	6.55739971951845e-06	\\
12246.3363647461	6.8734234859361e-06	\\
12265.8508300781	0.000116640587224083	\\
12298.8235473633	0.000121812875261423	\\
12309.5901489258	7.14442630564536e-06	\\
12327.7587890625	9.17506556526062e-05	\\
12329.1046142578	4.86240100236337e-06	\\
12349.2919921875	4.20249976104992e-06	\\
12354.6752929688	0.000131606122375709	\\
12386.9750976563	0.000125088576071962	\\
12393.0313110352	3.81665793721479e-06	\\
12421.9665527344	1.24471316416552e-05	\\
12430.0415039063	0.000119395113356439	\\
12457.6309204102	8.16945826027226e-06	\\
12465.705871582	0.000117039196485035	\\
12487.9119873047	6.56302576793912e-06	\\
12509.4451904297	9.36433397166328e-05	\\
12539.0533447266	9.20266362263283e-05	\\
12539.7262573242	3.81873868403561e-06	\\
12561.9323730469	7.32706558698795e-06	\\
12582.7926635742	0.000117095020197584	\\
12607.6904296875	3.68434201319168e-06	\\
12616.438293457	0.000115349283952772	\\
12633.9340209961	0.000100015960332034	\\
12639.3173217773	1.01944005095095e-05	\\
12668.2525634766	2.41275933759568e-06	\\
12683.7295532227	6.82644764232516e-05	\\
12706.608581543	0.000102618309230955	\\
12712.6647949219	1.27212131746425e-05	\\
12729.4876098633	0.000104349406454466	\\
12736.2167358398	6.94779136021095e-06	\\
12773.2269287109	0.00014615295622693	\\
12784.6664428711	2.7933734564848e-06	\\
12819.6578979492	4.9289887103928e-06	\\
12821.6766357422	0.00011378109625889	\\
12842.5369262695	7.87890888531281e-06	\\
12845.9014892578	0.000135845644737009	\\
12861.3784790039	0.000137885818497942	\\
12867.4346923828	6.9580737703029e-06	\\
12905.1177978516	0.000103567449787044	\\
12911.8469238281	7.93646137086238e-06	\\
12936.7446899414	0.000105262238650425	\\
12941.455078125	8.83836399922566e-06	\\
12985.8673095703	0.000103498142627093	\\
12989.9047851563	3.68864573032855e-06	\\
13008.7463378906	7.17533772233688e-06	\\
13028.2608032227	0.000102002029638538	\\
13039.0274047852	9.34338490775484e-06	\\
13045.0836181641	0.000118847827335922	\\
13092.8604125977	8.94048239785218e-06	\\
13098.9166259766	9.19263078695509e-05	\\
13109.0103149414	8.62053595781691e-05	\\
13120.4498291016	2.44287859908206e-06	\\
13152.7496337891	5.98906215806347e-06	\\
13163.5162353516	0.000111615240901711	\\
13188.4140014648	0.000105925924319947	\\
13190.4327392578	8.00348041709558e-06	\\
13207.2555541992	9.67394199819284e-05	\\
13229.4616699219	1.13368235006291e-05	\\
13240.2282714844	4.3212046693846e-06	\\
13250.9948730469	8.36082805242892e-05	\\
13279.2572021484	9.10410315295134e-05	\\
13294.0612792969	1.04367570395907e-05	\\
13310.8840942383	2.40624831411592e-06	\\
13335.7818603516	9.70250217852305e-05	\\
13343.1838989258	9.11015575768892e-05	\\
13366.0629272461	7.73199467287734e-06	\\
13394.3252563477	9.3744380700203e-05	\\
13401.0543823242	8.29938814497751e-06	\\
13421.9146728516	0.000111431984132771	\\
13434.0270996094	9.01231960268369e-07	\\
13455.5603027344	0.000123714840093095	\\
13462.2894287109	7.21173439116191e-06	\\
13499.299621582	3.58009556310801e-06	\\
13505.3558349609	9.62005639881605e-05	\\
13517.4682617188	0.000102760461111272	\\
13526.8890380859	8.11352821316102e-06	\\
13557.1701049805	7.61313348727417e-05	\\
13562.5534057617	6.82122402689058e-06	\\
13597.5448608398	2.06890716237056e-06	\\
13611.6760253906	8.50542193155979e-05	\\
13636.5737915039	7.92208313130895e-05	\\
13646.6674804688	2.97652782122763e-06	\\
13654.7424316406	7.3878059038761e-05	\\
13679.6401977539	3.98785447817972e-06	\\
13691.0797119141	7.07132526661978e-05	\\
13713.9587402344	2.4871814879474e-06	\\
13720.6878662109	8.53970267828366e-05	\\
13727.4169921875	7.42350224068258e-06	\\
13759.716796875	0.000104357891279046	\\
13768.4646606445	3.42047222247836e-06	\\
13811.5310668945	4.43777440392621e-06	\\
13815.5685424805	8.57453902308176e-05	\\
13822.9705810547	0.000109127805002083	\\
13845.1766967773	5.40809364622003e-06	\\
13860.6536865234	7.51636873646688e-05	\\
13888.916015625	2.64470230877189e-06	\\
13904.3930053711	1.7767980170745e-06	\\
13923.2345581055	8.83824430671977e-05	\\
13923.2345581055	8.83824430671977e-05	\\
13954.8614501953	7.96238246114288e-07	\\
13977.7404785156	0.000101616142042895	\\
13984.4696044922	1.04763640122466e-05	\\
13994.563293457	6.49606517504121e-05	\\
14012.7319335938	3.02463593530476e-06	\\
14057.8170776367	0.000114809441996257	\\
14059.162902832	7.66225184280692e-07	\\
14065.2191162109	7.58740309454234e-05	\\
14066.5649414063	2.07451011449068e-06	\\
14112.3229980469	3.70199122865253e-06	\\
14127.799987793	7.44448255517102e-05	\\
14129.1458129883	2.32495939489243e-06	\\
14155.3894042969	6.64662519968318e-05	\\
14170.866394043	5.05576726260857e-06	\\
14191.0537719727	9.71407064175376e-05	\\
14223.3535766602	7.39426218738238e-05	\\
14225.3723144531	3.76836000538514e-06	\\
14238.1576538086	9.44381274427289e-05	\\
14247.5784301758	7.06198759326963e-06	\\
14293.3364868164	6.14202508879582e-06	\\
14294.6823120117	7.14700680414348e-05	\\
14308.1405639648	2.3770142280505e-06	\\
14317.561340332	7.01846371851493e-05	\\
14347.8424072266	4.11944769065847e-06	\\
14364.665222168	9.17449299716406e-05	\\
14376.7776489258	5.43820403276675e-05	\\
14386.198425293	5.38015567410355e-06	\\
14429.264831543	3.45161018037283e-06	\\
14430.6106567383	6.93860873060652e-05	\\
14460.2188110352	6.19427780252651e-05	\\
14464.9291992188	6.86665391860174e-06	\\
14473.6770629883	4.6765914913161e-06	\\
14502.6123046875	6.1513398168712e-05	\\
14523.4725952148	5.02594581624245e-06	\\
14525.4913330078	5.80834129894069e-05	\\
14553.0807495117	4.86799886466377e-06	\\
14573.9410400391	8.44901400613793e-05	\\
14574.6139526367	7.14542239570341e-05	\\
14594.1284179688	2.91047998371229e-06	\\
14617.0074462891	3.63573843573191e-06	\\
14638.5406494141	4.93874967695994e-05	\\
14647.9614257813	8.42851766989161e-07	\\
14662.092590332	4.26480740093429e-05	\\
14703.8131713867	7.56309313144456e-07	\\
14707.8506469727	5.36072205958172e-05	\\
14724.6734619141	5.27457867058624e-05	\\
14740.8233642578	2.60169776410951e-06	\\
14755.6274414063	6.85247719726373e-06	\\
14772.4502563477	6.3054428031763e-05	\\
14782.5439453125	5.6639010724762e-05	\\
14799.3667602539	4.84854339762923e-06	\\
14837.0498657227	4.41124948274689e-05	\\
14847.1435546875	9.16091364153229e-07	\\
14857.2372436523	3.28424283340943e-07	\\
14878.0975341797	4.24066802590847e-05	\\
14894.2474365234	4.59947812323652e-05	\\
14903.6682128906	1.96005288999536e-06	\\
14923.8555908203	3.47976095565677e-05	\\
14924.528503418	8.47405766385015e-06	\\
14981.7260742188	6.11672165778821e-06	\\
14983.7448120117	3.5524827696434e-05	\\
15010.661315918	4.21461356737495e-05	\\
15011.3342285156	2.12580530131607e-06	\\
15042.2882080078	4.5289001092625e-06	\\
15051.0360717773	4.77806838022866e-05	\\
15072.5692749023	6.86473648315129e-06	\\
15081.3171386719	4.17914270866519e-05	\\
15095.4483032227	3.22224017454879e-05	\\
15103.5232543945	7.23906818830783e-06	\\
15142.5521850586	7.71231559491523e-06	\\
15143.2250976563	5.15232647232561e-05	\\
15168.7957763672	2.29657468490882e-06	\\
15184.9456787109	4.93528541543785e-05	\\
15192.3477172852	5.01160256961334e-05	\\
15219.2642211914	5.80204700106977e-06	\\
15244.1619873047	3.60984288895546e-06	\\
15258.9660644531	3.36745922128114e-05	\\
15261.6577148438	6.33832694681735e-06	\\
15264.3493652344	3.96221759758029e-05	\\
15301.3595581055	2.8989371786545e-06	\\
15305.3970336914	3.91044950782312e-05	\\
15332.3135375977	4.08525186358304e-05	\\
15350.4821777344	4.50027188121672e-06	\\
15381.4361572266	3.70542142458427e-05	\\
15396.9131469727	4.28735078131481e-06	\\
15401.6235351563	1.67780221479742e-06	\\
15414.4088745117	4.31926125298612e-05	\\
15436.6149902344	7.54087055931455e-06	\\
15458.1481933594	5.26414946081646e-05	\\
15474.9710083008	4.25327674278696e-05	\\
15485.7376098633	2.05433088223664e-06	\\
15505.2520751953	3.6655484930665e-06	\\
15513.3270263672	4.2415706676516e-05	\\
15539.5706176758	4.05132407933751e-06	\\
15564.4683837891	3.54290192793581e-05	\\
15588.6932373047	3.34713819665772e-05	\\
15598.1140136719	5.39204477096979e-06	\\
15610.8993530273	3.13517958500273e-05	\\
15612.2451782227	6.68189786929759e-06	\\
15641.1804199219	1.12528650834351e-05	\\
15653.9657592773	2.80061598385923e-05	\\
15699.723815918	2.55072264715306e-05	\\
15702.4154663086	1.17328678805684e-05	\\
15719.9111938477	1.02705801311168e-05	\\
15722.6028442383	2.82687202955865e-05	\\
15755.5755615234	1.22658003404059e-05	\\
15771.0525512695	2.74056227656391e-05	\\
15790.5670166016	2.94869364708126e-05	\\
15791.2399291992	6.57090257630337e-06	\\
15822.1939086914	3.06034412169219e-05	\\
15838.3438110352	6.90577145463704e-06	\\
15865.9332275391	4.10165817711328e-05	\\
15866.6061401367	3.64459457369769e-06	\\
15878.0456542969	4.26129150590736e-05	\\
15878.7185668945	3.73261033863178e-06	\\
15917.7474975586	8.12295997384331e-06	\\
15937.9348754883	2.72591082190403e-05	\\
15949.3743896484	9.12350896631223e-06	\\
15950.7202148438	2.67016405761583e-05	\\
15982.3471069336	2.33311249788272e-05	\\
15990.4220581055	1.39776490260164e-05	\\
16014.6469116211	1.45458116473838e-05	\\
16025.4135131836	2.23387182931093e-05	\\
16049.6383666992	2.19967893104923e-05	\\
16060.4049682617	1.57713837264092e-05	\\
16083.283996582	2.10409288941445e-05	\\
16084.6298217773	1.66361768600566e-05	\\
16118.2754516602	1.73785238113379e-05	\\
16121.6400146484	2.00061929943842e-05	\\
16151.921081543	1.81609782884965e-05	\\
16153.2669067383	1.92556746069691e-05	\\
16219.2123413086	1.89003447953148e-05	\\
16220.5581665039	1.85346158850022e-05	\\
16226.6143798828	1.89319893657181e-05	\\
16249.4934082031	1.84904892225815e-05	\\
16261.6058349609	1.85066001027272e-05	\\
16267.6620483398	1.88665287893504e-05	\\
16296.5972900391	1.87739040123978e-05	\\
16315.4388427734	1.85669713844079e-05	\\
16342.3553466797	1.85287484545682e-05	\\
16350.4302978516	1.87427475558558e-05	\\
16357.8323364258	1.87130315510739e-05	\\
16378.0197143555	1.85163208363287e-05	\\
16396.1883544922	1.86957936459828e-05	\\
16425.1235961914	1.85180212812529e-05	\\
16429.1610717773	1.86772034612557e-05	\\
16434.5443725586	1.84877001341917e-05	\\
16481.6482543945	1.86660746061811e-05	\\
16488.3773803711	1.84798569408016e-05	\\
16522.6959228516	1.86049703666537e-05	\\
16528.7521362305	1.84745425468917e-05	\\
16542.8833007813	1.86032907572209e-05	\\
16545.5749511719	1.84584766172688e-05	\\
16564.4165039063	1.86095743211955e-05	\\
16575.8560180664	1.84132876588297e-05	\\
16609.5016479492	1.84000942676559e-05	\\
16610.8474731445	1.85896679218455e-05	\\
16654.5867919922	1.85600852125703e-05	\\
16655.9326171875	1.83685562200613e-05	\\
16668.717956543	1.85552509999042e-05	\\
16692.2698974609	1.83500444669916e-05	\\
16709.0927124023	1.83374318664721e-05	\\
16719.8593139648	1.85222294811959e-05	\\
16738.0279541016	1.85288231226164e-05	\\
16768.9819335938	1.83299843828813e-05	\\
16789.8422241211	1.84970864601493e-05	\\
16800.6088256836	1.83260803667654e-05	\\
16808.0108642578	1.84980224278547e-05	\\
16816.0858154297	1.82935751816858e-05	\\
16857.1334838867	1.82856143318649e-05	\\
16858.479309082	1.84371830939727e-05	\\
16882.03125	1.82524475731171e-05	\\
16901.545715332	1.84290630330213e-05	\\
16926.4434814453	1.83753523564489e-05	\\
16930.4809570313	1.82286360574803e-05	\\
16964.7994995117	1.8370368180979e-05	\\
16970.8557128906	1.82303017426589e-05	\\
16990.3701782227	1.82223837687e-05	\\
16997.0993041992	1.83827300243584e-05	\\
17010.5575561523	1.83522437583627e-05	\\
17042.1844482422	1.82194876051813e-05	\\
17061.0260009766	1.81708245277349e-05	\\
17071.7926025391	1.83503239330415e-05	\\
17089.9612426758	1.8318749969625e-05	\\
17098.0361938477	1.81689348078734e-05	\\
17118.896484375	1.81425992462999e-05	\\
17127.6443481445	1.83169296988235e-05	\\
17159.944152832	1.82693491495509e-05	\\
17162.6358032227	1.81296512166022e-05	\\
17184.1690063477	1.83025180795057e-05	\\
17184.8419189453	1.81412217984335e-05	\\
17230.5999755859	1.82589552811793e-05	\\
17234.6374511719	1.81126256784484e-05	\\
17262.8997802734	1.81103394879044e-05	\\
17267.610168457	1.82156434519684e-05	\\
17298.5641479492	1.82149342446571e-05	\\
17317.4057006836	1.80880422727722e-05	\\
17332.209777832	1.80569227251249e-05	\\
17347.6867675781	1.82156888509963e-05	\\
17357.780456543	1.82036383293538e-05	\\
17381.3323974609	1.80549597070199e-05	\\
17387.3886108398	1.81961021269704e-05	\\
17406.9030761719	1.8053936799827e-05	\\
17422.380065918	1.81467393087283e-05	\\
17439.875793457	1.80243653571518e-05	\\
17474.1943359375	1.81381099939268e-05	\\
17488.3255004883	1.79740575333219e-05	\\
17499.0921020508	1.81378030896508e-05	\\
17513.8961791992	1.79884596841796e-05	\\
17529.3731689453	1.81213547734189e-05	\\
17535.4293823242	1.79940322769341e-05	\\
17568.4020996094	1.81289896902839e-05	\\
17587.9165649414	1.79467096422658e-05	\\
17601.3748168945	1.80930376206435e-05	\\
17617.5247192383	1.79499740679451e-05	\\
17642.4224853516	1.79515851696847e-05	\\
17652.5161743164	1.8063907246603e-05	\\
17674.0493774414	1.81048662762179e-05	\\
17687.5076293945	1.79012740699227e-05	\\
17725.8636474609	1.80204739559849e-05	\\
17726.5365600586	1.7915167743969e-05	\\
17733.2656860352	1.78910231963016e-05	\\
17748.0697631836	1.80302300742937e-05	\\
17774.3133544922	1.80528034803737e-05	\\
17795.8465576172	1.7863966659145e-05	\\
17811.9964599609	1.79891410757473e-05	\\
17825.4547119141	1.78507976638672e-05	\\
17851.025390625	1.78341167754984e-05	\\
17857.0816040039	1.79925070650395e-05	\\
17884.6710205078	1.80066624704306e-05	\\
17894.7647094727	1.78357113721403e-05	\\
17905.5313110352	1.79786672763173e-05	\\
17906.2042236328	1.78098458735692e-05	\\
17941.1956787109	1.79556262269745e-05	\\
17941.8685913086	1.78053818298325e-05	\\
17970.1309204102	1.79477395973012e-05	\\
17995.7015991211	1.78111645611951e-05	\\
18021.272277832	1.79540539689181e-05	\\
18026.6555786133	1.77758459430583e-05	\\
18057.6095581055	1.77253946381597e-05	\\
18058.2824707031	1.79383838135667e-05	\\
18092.6010131836	1.78858118694932e-05	\\
18095.9655761719	1.77718792771816e-05	\\
18124.9008178711	1.78823409296227e-05	\\
18135.6674194336	1.77331219376764e-05	\\
18143.0694580078	1.77105172409096e-05	\\
18153.8360595703	1.78832076560007e-05	\\
18179.4067382813	1.78707596596663e-05	\\
18196.9024658203	1.77163826128869e-05	\\
18224.4918823242	1.78528008402843e-05	\\
18226.5106201172	1.77036324436479e-05	\\
18248.0438232422	1.78227500665336e-05	\\
18277.6519775391	1.76656168361015e-05	\\
18291.1102294922	1.76556930568522e-05	\\
18311.9705200195	1.78377842954023e-05	\\
18316.6809082031	1.7804267701924e-05	\\
18339.5599365234	1.76694126625376e-05	\\
18366.4764404297	1.76391879362007e-05	\\
18368.4951782227	1.77834975925598e-05	\\
18395.4116821289	1.77883335597397e-05	\\
18414.9261474609	1.76455319267875e-05	\\
18425.6927490234	1.77699073662777e-05	\\
18429.7302246094	1.76299118705842e-05	\\
18464.0487670898	1.77601195772371e-05	\\
18479.5257568359	1.76439142311893e-05	\\
18514.5172119141	1.778217502765e-05	\\
18518.5546875	1.76121197251639e-05	\\
18519.9005126953	1.7727353759907e-05	\\
18532.6858520508	1.76110374056338e-05	\\
18562.2940063477	1.77372919355877e-05	\\
18575.7522583008	1.76043310020097e-05	\\
18610.7437133789	1.77265882423063e-05	\\
18620.8374023438	1.75427420165835e-05	\\
18620.8374023438	1.75427420165835e-05	\\
18636.3143920898	1.77302033187072e-05	\\
18671.305847168	1.7752381181475e-05	\\
18688.801574707	1.75682217932894e-05	\\
18698.2223510742	1.75590594809374e-05	\\
18717.7368164063	1.7679936662492e-05	\\
18729.8492431641	1.77040677284804e-05	\\
18731.867980957	1.75355864592939e-05	\\
18769.5510864258	1.76958767221664e-05	\\
18791.7572021484	1.74891287191068e-05	\\
18811.9445800781	1.76552921777287e-05	\\
18824.7299194336	1.74795321792249e-05	\\
18850.9735107422	1.7472317581836e-05	\\
18851.6464233398	1.76671274693715e-05	\\
18874.5254516602	1.76660411146789e-05	\\
18890.0024414063	1.74928841299172e-05	\\
18898.7503051758	1.74820287429036e-05	\\
18919.6105957031	1.7618512860599e-05	\\
18934.4146728516	1.74636203481568e-05	\\
18947.200012207	1.76172634211606e-05	\\
18967.3873901367	1.760583187864e-05	\\
18994.303894043	1.74826370432643e-05	\\
19004.3975830078	1.74656249403447e-05	\\
19013.1454467773	1.75975601557998e-05	\\
19050.8285522461	1.75852578727494e-05	\\
19062.9409790039	1.74017909414863e-05	\\
19067.6513671875	1.7432635533618e-05	\\
19088.5116577148	1.7597257339826e-05	\\
19107.3532104492	1.74191665833478e-05	\\
19129.5593261719	1.75904246414609e-05	\\
19137.6342773438	1.75507179433023e-05	\\
19167.9153442383	1.74265394819401e-05	\\
19193.4860229492	1.75449670052769e-05	\\
19201.5609741211	1.74244829644176e-05	\\
19221.0754394531	1.75542988730242e-05	\\
19227.8045654297	1.74106147052978e-05	\\
19249.3377685547	1.75322288446476e-05	\\
19255.3939819336	1.7399749746331e-05	\\
19284.3292236328	1.75479156006173e-05	\\
19288.3666992188	1.7380955784708e-05	\\
19317.9748535156	1.75371924203385e-05	\\
19319.9935913086	1.73853940985875e-05	\\
19347.5830078125	1.7503885256204e-05	\\
19361.0412597656	1.73640964034171e-05	\\
19385.2661132813	1.73707989772242e-05	\\
19396.7056274414	1.74843343700227e-05	\\
19418.9117431641	1.73665056071972e-05	\\
19419.5846557617	1.75231597399137e-05	\\
19459.9594116211	1.73413541183187e-05	\\
19476.1093139648	1.75104990557516e-05	\\
19485.530090332	1.73069368793564e-05	\\
19510.4278564453	1.7530599023809e-05	\\
19533.3068847656	1.73431393572002e-05	\\
19544.0734863281	1.75103220004775e-05	\\
19550.8026123047	1.74587390719222e-05	\\
19552.8213500977	1.73182289297858e-05	\\
19586.4669799805	1.74638058940385e-05	\\
19587.1398925781	1.73308330411219e-05	\\
19640.299987793	1.74576252801292e-05	\\
19645.6832885742	1.72984513907445e-05	\\
19668.5623168945	1.73196841215479e-05	\\
19672.5997924805	1.74471425205605e-05	\\
19689.4226074219	1.74293756420372e-05	\\
19711.6287231445	1.72696382626832e-05	\\
19722.395324707	1.74249517528589e-05	\\
19737.1994018555	1.73026072758503e-05	\\
19758.7326049805	1.7296453856077e-05	\\
19777.5741577148	1.74293265395828e-05	\\
19793.0511474609	1.74128094503109e-05	\\
19807.1823120117	1.72738036662949e-05	\\
19824.6780395508	1.72882125300701e-05	\\
19854.2861938477	1.74066525609424e-05	\\
19863.0340576172	1.72637823795921e-05	\\
19885.2401733398	1.74045836141438e-05	\\
19908.7921142578	1.74013230747342e-05	\\
19912.1566772461	1.72761758630485e-05	\\
19943.1106567383	1.72569750857852e-05	\\
19953.8772583008	1.73807723466172e-05	\\
19972.7188110352	1.72303124907247e-05	\\
19982.1395874023	1.73724109348229e-05	\\
20021.1685180664	1.73726156190448e-05	\\
20025.2059936523	1.72277887133684e-05	\\
20029.9163818359	1.73651523149849e-05	\\
20060.8703613281	1.72198348935312e-05	\\
20068.2723999023	1.71964171006567e-05	\\
20079.0390014648	1.73806898510745e-05	\\
20105.2825927734	1.73853699689491e-05	\\
20122.1054077148	1.72094083572684e-05	\\
20134.2178344727	1.73354431779333e-05	\\
20160.4614257813	1.71900738993676e-05	\\
20186.7050170898	1.73651547325211e-05	\\
20188.7237548828	1.72056299346556e-05	\\
20206.2194824219	1.71964261472048e-05	\\
20214.9673461914	1.73892031259456e-05	\\
20262.0712280273	1.73382964112045e-05	\\
20266.7816162109	1.71474292564982e-05	\\
20271.4920043945	1.73168853405149e-05	\\
20276.8753051758	1.72158678265636e-05	\\
20308.5021972656	1.73065364899254e-05	\\
20311.1938476563	1.71789767592831e-05	\\
20361.6622924805	1.73217600620354e-05	\\
20367.7185058594	1.71742073775798e-05	\\
20378.4851074219	1.73276582997895e-05	\\
20399.3453979492	1.71130035492748e-05	\\
20416.8411254883	1.73193878653527e-05	\\
20424.9160766602	1.71692295905819e-05	\\
20449.1409301758	1.72927106608983e-05	\\
20455.8700561523	1.71772048370112e-05	\\
20481.4407348633	1.73157194943049e-05	\\
20487.4969482422	1.71667083146661e-05	\\
20530.5633544922	1.71571232396166e-05	\\
20533.9279174805	1.7328307578966e-05	\\
20547.3861694336	1.71612601457935e-05	\\
20553.4423828125	1.72871673142111e-05	\\
20593.1442260742	1.72744356723556e-05	\\
20594.4900512695	1.71783895065564e-05	\\
20619.3878173828	1.73107168304111e-05	\\
20631.5002441406	1.71146960350219e-05	\\
20657.7438354492	1.71278654463409e-05	\\
20673.2208251953	1.72896372190585e-05	\\
20688.0249023438	1.7142413654557e-05	\\
20709.5581054688	1.72687337889648e-05	\\
20722.3434448242	1.72589068709061e-05	\\
20743.2037353516	1.71399809998083e-05	\\
20752.6245117188	1.72874299147965e-05	\\
20770.7931518555	1.71325668435811e-05	\\
20792.3263549805	1.73008769987982e-05	\\
20811.1679077148	1.71326628913746e-05	\\
20822.607421875	1.71142795347532e-05	\\
20844.140625	1.72737159233371e-05	\\
20856.9259643555	1.727174162829e-05	\\
20868.3654785156	1.71042531150873e-05	\\
20890.5715942383	1.71038005006561e-05	\\
20914.1235351563	1.728397259615e-05	\\
20928.254699707	1.72701311186154e-05	\\
20928.9276123047	1.70993443489701e-05	\\
20953.1524658203	1.7259886678458e-05	\\
20982.0877075195	1.7100230047484e-05	\\
21001.6021728516	1.7231811786897e-05	\\
21004.2938232422	1.71101544954728e-05	\\
21044.6685791016	1.72547755781563e-05	\\
21049.3789672852	1.70898825161226e-05	\\
21057.453918457	1.70811368892527e-05	\\
21084.3704223633	1.72426104304595e-05	\\
21109.2681884766	1.72338124444131e-05	\\
21118.0160522461	1.70750122142887e-05	\\
21147.624206543	1.72229312718866e-05	\\
21155.6991577148	1.71109893142365e-05	\\
21165.119934082	1.70866328170349e-05	\\
21191.3635253906	1.72271169179026e-05	\\
21197.4197387695	1.72108627315059e-05	\\
21199.4384765625	1.70675310268687e-05	\\
21240.4861450195	1.72315367078004e-05	\\
21241.1590576172	1.70721635671092e-05	\\
21270.0942993164	1.70754480841093e-05	\\
21274.1317749023	1.72246526582552e-05	\\
21315.1794433594	1.72157380969257e-05	\\
21329.3106079102	1.70566112533518e-05	\\
21329.9835205078	1.70815889422452e-05	\\
21358.2458496094	1.7204832868808e-05	\\
21377.0874023438	1.72449694121199e-05	\\
21395.9289550781	1.70489273356012e-05	\\
21406.6955566406	1.72207760634638e-05	\\
21421.4996337891	1.70763040193882e-05	\\
21434.2849731445	1.72337833699308e-05	\\
21446.3973999023	1.70543518757118e-05	\\
21483.4075927734	1.72193887525013e-05	\\
21487.4450683594	1.70415828339966e-05	\\
21502.9220581055	1.72209864627485e-05	\\
21517.0532226563	1.70628435682464e-05	\\
21542.6239013672	1.72151107445627e-05	\\
21561.4654541016	1.70446667703053e-05	\\
21575.5966186523	1.70388946495774e-05	\\
21587.0361328125	1.71861930488265e-05	\\
21608.5693359375	1.72047425990927e-05	\\
21619.3359375	1.70221253101357e-05	\\
21652.3086547852	1.70564543684758e-05	\\
21652.9815673828	1.72303513792458e-05	\\
21673.8418579102	1.72001378473017e-05	\\
21678.5522460938	1.70635810810952e-05	\\
21716.9082641602	1.71875454747923e-05	\\
21734.4039916992	1.70543631380563e-05	\\
21750.553894043	1.70523541988166e-05	\\
21755.9371948242	1.72162557290495e-05	\\
21781.5078735352	1.72115205814694e-05	\\
21795.6390380859	1.70471592292551e-05	\\
21821.2097167969	1.7216505676533e-05	\\
21828.6117553711	1.7048778772148e-05	\\
21860.9115600586	1.70636692882752e-05	\\
21874.3698120117	1.72203197119021e-05	\\
21883.1176757813	1.70371684171957e-05	\\
21899.9404907227	1.71799535555229e-05	\\
21931.5673828125	1.72043974662074e-05	\\
21936.9506835938	1.70689137038871e-05	\\
21974.6337890625	1.70578644422277e-05	\\
21980.0170898438	1.72338236428799e-05	\\
22004.914855957	1.70649566501487e-05	\\
22007.6065063477	1.71911309322617e-05	\\
22050	1.70305463236133e-05	\\
};
\addlegendentry{Y};

\end{axis}
\end{tikzpicture}%    	
    	\caption{The plots with $a=-0.95$}
    	\label{fig:a0.-95-plots}
\end{figure}


\section{Time domain and frequency domain}
Now the train signal has to be plotted in the time- and frequency domain.
The written Matlab-code can be found in Appendix \ref{lst:timefreqdomain.m}.
Before plotting the time domain signal we had to take into account the sample rate $Fs$.
In order to find the length of the train signal we used the formula:
\begin{equation}
t = \frac{samples}{Fs}
\end{equation}
In the script $length(y)$ is used instead of $samples$.

The plot for the time domain signal can be seen in Figure \ref{fig:timedomain} and the plot for the frequency domain signal can be seen in Figure \ref{fig:freqdomain}.


\begin{figure}[H]
	\centering
	\setlength\figureheight{6cm}
    	\setlength\figurewidth{10cm}
	% This file was created by matlab2tikz v0.4.2.
% Copyright (c) 2008--2013, Nico Schlömer <nico.schloemer@gmail.com>
% All rights reserved.
% 
% The latest updates can be retrieved from
%   http://www.mathworks.com/matlabcentral/fileexchange/22022-matlab2tikz
% where you can also make suggestions and rate matlab2tikz.
% 
% 
% 
\begin{tikzpicture}

\begin{axis}[%
width=\figurewidth,
height=\figureheight,
scale only axis,
xmin=0,
xmax=1.6,
xlabel={time [s]},
ymin=-1,
ymax=1
]
\addplot [
color=blue,
solid,
forget plot
]
table[row sep=crcr]{
0 -0.0306854029251506\\
0.000122079790744623 -0.00114711786636077\\
0.000244159581489246 -0.0106108402638371\\
0.000366239372233869 0.00344135359908231\\
0.000488319162978492 0.0140521938629194\\
0.000610398953723115 0.0140521938629194\\
0.000732478744467738 0.0243762546601663\\
0.000854558535212361 0.016346429595641\\
0.000976638325956984 0.00401491253226269\\
0.00109871811670161 -0.00172067679954115\\
0.00122079790744623 0.0117579581301979\\
0.00134287769819085 -0.0151993117292802\\
0.00146495748893548 -0.0306854029251506\\
0.0015870372796801 -0.0117579581301979\\
0.00170911707042472 -0.0255233725265271\\
0.00183119686116935 -0.0243762546601663\\
0.00195327665191397 -0.00573558933180384\\
0.00207535644265859 0.020934901061084\\
0.00219743623340321 0.0352738743905936\\
0.00231951602414784 0.0467450530542013\\
0.00244159581489246 0.0255233725265271\\
0.00256367560563708 0.0129050759965586\\
0.00268575539638171 0.00860338399770576\\
0.00280783518712633 -0.0398623458560367\\
0.00292991497787095 -0.0559219959850875\\
0.00305199476861558 -0.0375681101233152\\
0.0031740745593602 -0.0255233725265271\\
0.00329615435010482 -0.0243762546601663\\
0.00341823414084945 0.0117579581301979\\
0.00354031393159407 0.058216231717809\\
0.00366239372233869 0.058216231717809\\
0.00378447351308331 0.0490392887869229\\
0.00390655330382794 0.0220820189274448\\
0.00402863309457256 0.00688270719816461\\
0.00415071288531718 -0.0352738743905936\\
0.00427279267606181 -0.0754229997132205\\
0.00439487246680643 -0.0708345282477775\\
0.00451695225755105 -0.0352738743905936\\
0.00463903204829568 0.000573558933180384\\
0.0047611118390403 -0.00401491253226269\\
0.00488319162978492 0.0605104674505305\\
0.00500527142052954 0.0891884141095498\\
0.00512735121127417 0.0513335245196444\\
0.00524943100201879 0.00516203039862346\\
0.00537151079276341 -0.00229423573272154\\
0.00549359058350804 -0.0243762546601663\\
0.00561567037425266 -0.0628047031832521\\
0.00573775016499728 -0.058216231717809\\
0.00585982995574191 -0.0306854029251506\\
0.00598190974648653 0.0117579581301979\\
0.00610398953723115 0.00688270719816461\\
0.00622606932797577 0.0243762546601663\\
0.0063481491187204 0.0421565815887582\\
0.00647022890946502 0.0421565815887582\\
0.00659230870020964 -0.00688270719816461\\
0.00671438849095427 -0.0129050759965586\\
0.00683646828169889 0.00946372239747634\\
0.00695854807244351 -0.00516203039862346\\
0.00708062786318814 -0.0186406653283625\\
0.00720270765393276 -0.0117579581301979\\
0.00732478744467738 0.016346429595641\\
0.00744686723542201 -0.0140521938629194\\
0.00756894702616663 -0.0329796386578721\\
0.00769102681691125 -0.0375681101233152\\
0.00781310660765587 -0.007456266131345\\
0.0079351863984005 0.0140521938629194\\
0.00805726618914512 0.0255233725265271\\
0.00817934597988974 0.058216231717809\\
0.00830142577063437 0.0800114711786636\\
0.00842350556137899 0.058216231717809\\
0.00854558535212361 -0.028391167192429\\
0.00866766514286823 -0.0375681101233152\\
0.00878974493361286 -0.0800114711786636\\
0.00891182472435748 -0.112130771436765\\
0.0090339045151021 -0.0845999426441067\\
0.00915598430584673 -0.007456266131345\\
0.00927806409659135 0.0708345282477775\\
0.00940014388733597 0.102953828505879\\
0.0095222236780806 0.121307714367651\\
0.00964430346882522 0.0891884141095498\\
0.00976638325956984 0.0513335245196444\\
0.00988846305031446 -0.0536277602523659\\
0.0100105428410591 -0.121307714367651\\
0.0101326226318037 -0.112130771436765\\
0.0102547024225483 -0.0937768855749928\\
0.010376782213293 -0.0559219959850875\\
0.0104988620040376 -0.00344135359908231\\
0.0106209417947822 0.0937768855749928\\
0.0107430215855268 0.141955835962145\\
0.0108651013762715 0.0891884141095498\\
0.0109871811670161 0.0628047031832521\\
0.0111092609577607 0.007456266131345\\
0.0112313407485053 -0.0444508173214798\\
0.0113534205392499 -0.112130771436765\\
0.0114755003299946 -0.0983653570404359\\
0.0115975801207392 -0.0605104674505305\\
0.0117196599114838 -0.0266704903928879\\
0.0118417397022284 0.0352738743905936\\
0.0119638194929731 0.0559219959850875\\
0.0120858992837177 0.0937768855749928\\
0.0122079790744623 0.0628047031832521\\
0.0123300588652069 0.00344135359908231\\
0.0124521386559515 -0.0375681101233152\\
0.0125742184466962 -0.0352738743905936\\
0.0126962982374408 -0.0490392887869229\\
0.0128183780281854 -0.0306854029251506\\
0.01294045781893 0.0306854029251506\\
0.0130625376096747 0.0490392887869229\\
0.0131846174004193 0.0140521938629194\\
0.0133066971911639 -0.0174935474620017\\
0.0134287769819085 -0.0243762546601663\\
0.0135508567726532 -0.0708345282477775\\
0.0136729365633978 -0.0444508173214798\\
0.0137950163541424 -0.00286779466590192\\
0.013917096144887 0.0490392887869229\\
0.0140391759356316 0.116719242902208\\
0.0141612557263763 0.121307714367651\\
0.0142833355171209 0.0662460567823344\\
0.0144054153078655 -0.0243762546601663\\
0.0145274950986101 -0.0891884141095498\\
0.0146495748893548 -0.169486664754804\\
0.0147716546800994 -0.17866360768569\\
0.014893734470844 -0.0800114711786636\\
0.0150158142615886 0.0329796386578721\\
0.0151378940523333 0.141955835962145\\
0.0152599738430779 0.215371379409234\\
0.0153820536338225 0.17866360768569\\
0.0155041334245671 0.0754229997132205\\
0.0156262132153117 -0.0243762546601663\\
0.0157482930060564 -0.141955835962145\\
0.015870372796801 -0.206194436478348\\
0.0159924525875456 -0.169486664754804\\
0.0161145323782902 -0.0891884141095498\\
0.0162366121690349 -0.00516203039862346\\
0.0163586919597795 0.112130771436765\\
0.0164807717505241 0.197017493547462\\
0.0166028515412687 0.151132778893031\\
0.0167249313320134 0.102953828505879\\
0.016847011122758 0.0151993117292802\\
0.0169690909135026 -0.0800114711786636\\
0.0170911707042472 -0.135073128763981\\
0.0172132504949918 -0.141955835962145\\
0.0173353302857365 -0.130484657298537\\
0.0174574100764811 -0.0398623458560367\\
0.0175794898672257 0.0708345282477775\\
0.0177015696579703 0.112130771436765\\
0.017823649448715 0.135073128763981\\
0.0179457292394596 0.121307714367651\\
0.0180678090302042 0.016346429595641\\
0.0181898888209488 -0.102953828505879\\
0.0183119686116935 -0.107542299971322\\
0.0184340484024381 -0.0891884141095498\\
0.0185561281931827 -0.028391167192429\\
0.0186782079839273 0.0559219959850875\\
0.0188002877746719 0.112130771436765\\
0.0189223675654166 0.0983653570404359\\
0.0190444473561612 0.0117579581301979\\
0.0191665271469058 -0.0800114711786636\\
0.0192886069376504 -0.160309721823917\\
0.0194106867283951 -0.112130771436765\\
0.0195327665191397 -0.0306854029251506\\
0.0196548463098843 0.0708345282477775\\
0.0197769261006289 0.215371379409234\\
0.0198990058913736 0.215371379409234\\
0.0200210856821182 0.107542299971322\\
0.0201431654728628 -0.0662460567823344\\
0.0202652452636074 -0.187840550616576\\
0.020387325054352 -0.252079151132779\\
0.0205094048450967 -0.197017493547462\\
0.0206314846358413 -0.0329796386578721\\
0.0207535644265859 0.0708345282477775\\
0.0208756442173305 0.206194436478348\\
0.0209977240080752 0.197017493547462\\
0.0211198037988198 0.135073128763981\\
0.0212418835895644 0.0708345282477775\\
0.021363963380309 -0.0186406653283625\\
0.0214860431710537 -0.0983653570404359\\
0.0216081229617983 -0.187840550616576\\
0.0217302027525429 -0.141955835962145\\
0.0218522825432875 -0.17866360768569\\
0.0219743623340321 -0.102953828505879\\
0.0220964421247768 0.0708345282477775\\
0.0222185219155214 0.206194436478348\\
0.022340601706266 0.293375394321767\\
0.0224626814970106 0.233725265271007\\
0.0225847612877553 0.112130771436765\\
0.0227068410784999 -0.151132778893031\\
0.0228289208692445 -0.279609979925437\\
0.0229510006599891 -0.311729280183539\\
0.0230730804507338 -0.206194436478348\\
0.0231951602414784 0.0708345282477775\\
0.023317240032223 0.215371379409234\\
0.0234393198229676 0.270433036994551\\
0.0235613996137122 0.187840550616576\\
0.0236834794044569 0.0243762546601663\\
0.0238055591952015 -0.197017493547462\\
0.0239276389859461 -0.215371379409234\\
0.0240497187766907 -0.0800114711786636\\
0.0241717985674354 0.0117579581301979\\
0.02429387835818 0.135073128763981\\
0.0244159581489246 0.151132778893031\\
0.0245380379396692 0.0800114711786636\\
0.0246601177304139 -0.0708345282477775\\
0.0247821975211585 -0.141955835962145\\
0.0249042773119031 -0.151132778893031\\
0.0250263571026477 -0.0490392887869229\\
0.0251484368933923 0.0662460567823344\\
0.025270516684137 0.0800114711786636\\
0.0253925964748816 0.0845999426441067\\
0.0255146762656262 0.0800114711786636\\
0.0256367560563708 0.0444508173214798\\
0.0257588358471155 -0.00802982506452538\\
0.0258809156378601 0.0444508173214798\\
0.0260029954286047 0.00573558933180384\\
0.0261250752193493 -0.112130771436765\\
0.026247155010094 -0.187840550616576\\
0.0263692348008386 -0.22454832234012\\
0.0264913145915832 -0.116719242902208\\
0.0266133943823278 0.0800114711786636\\
0.0267354741730724 0.311729280183539\\
0.0268575539638171 0.366790937768856\\
0.0269796337545617 0.348437051907083\\
0.0271017135453063 0.0329796386578721\\
0.0272237933360509 -0.311729280183539\\
0.0273458731267956 -0.458560367077717\\
0.0274679529175402 -0.4034987094924\\
0.0275900327082848 -0.141955835962145\\
0.0277121124990294 0.121307714367651\\
0.027834192289774 0.440206481215945\\
0.0279562720805187 0.366790937768856\\
0.0280783518712633 0.206194436478348\\
0.0282004316620079 -0.0559219959850875\\
0.0283225114527525 -0.233725265271007\\
0.0284445912434972 -0.17866360768569\\
0.0285666710342418 -0.141955835962145\\
0.0286887508249864 -0.00860338399770576\\
0.028810830615731 0.007456266131345\\
0.0289329104064757 0.0891884141095498\\
0.0290549901972203 -0.0375681101233152\\
0.0291770699879649 -0.0536277602523659\\
0.0292991497787095 0.0891884141095498\\
0.0294212295694542 0.112130771436765\\
0.0295433093601988 0.151132778893031\\
0.0296653891509434 0.0106108402638371\\
0.029787468941688 -0.0662460567823344\\
0.0299095487324326 -0.197017493547462\\
0.0300316285231773 -0.160309721823917\\
0.0301537083139219 -0.00172067679954115\\
0.0302757881046665 0.151132778893031\\
0.0303978678954111 0.270433036994551\\
0.0305199476861558 0.0845999426441067\\
0.0306420274769004 -0.0891884141095498\\
0.030764107267645 -0.261256094063665\\
0.0308861870583896 -0.279609979925437\\
0.0310082668491342 -0.141955835962145\\
0.0311303466398789 0.169486664754804\\
0.0312524264306235 0.458560367077717\\
0.0313745062213681 0.366790937768856\\
0.0314965860121127 0.151132778893031\\
0.0316186658028574 -0.17866360768569\\
0.031740745593602 -0.458560367077717\\
0.0318628253843466 -0.458560367077717\\
0.0319849051750912 -0.206194436478348\\
0.0321069849658359 0.0937768855749928\\
0.0322290647565805 0.348437051907083\\
0.0323511445473251 0.385144823630628\\
0.0324732243380697 0.206194436478348\\
0.0325953041288143 0.0129050759965586\\
0.032717383919559 -0.107542299971322\\
0.0328394637103036 -0.151132778893031\\
0.0329615435010482 -0.130484657298537\\
0.0330836232917928 -0.0559219959850875\\
0.0332057030825375 -0.121307714367651\\
0.0333277828732821 -0.141955835962145\\
0.0334498626640267 -0.107542299971322\\
0.0335719424547713 0.0467450530542013\\
0.033694022245516 0.252079151132779\\
0.0338161020362606 0.366790937768856\\
0.0339381818270052 0.330083166045311\\
0.0340602616177498 0.00516203039862346\\
0.0341823414084944 -0.242902208201893\\
0.0343044211992391 -0.440206481215945\\
0.0344265009899837 -0.421852595354173\\
0.0345485807807283 -0.0800114711786636\\
0.0346706605714729 0.252079151132779\\
0.0347927403622176 0.440206481215945\\
0.0349148201529622 0.293375394321767\\
0.0350368999437068 0.112130771436765\\
0.0351589797344514 -0.279609979925437\\
0.0352810595251961 -0.366790937768856\\
0.0354031393159407 -0.22454832234012\\
0.0355252191066853 0.028391167192429\\
0.0356472988974299 0.348437051907083\\
0.0357693786881746 0.293375394321767\\
0.0358914584789192 0.151132778893031\\
0.0360135382696638 -0.169486664754804\\
0.0361356180604084 -0.279609979925437\\
0.036257697851153 -0.293375394321767\\
0.0363797776418977 -0.102953828505879\\
0.0365018574326423 0.17866360768569\\
0.0366239372233869 0.17866360768569\\
0.0367460170141315 0.187840550616576\\
0.0368680968048762 0.016346429595641\\
0.0369901765956208 -0.0106108402638371\\
0.0371122563863654 -0.0421565815887582\\
0.03723433617711 0.0845999426441067\\
0.0373564159678546 0.0983653570404359\\
0.0374784957585993 -0.0937768855749928\\
0.0376005755493439 -0.206194436478348\\
0.0377226553400885 -0.385144823630628\\
0.0378447351308331 -0.261256094063665\\
0.0379668149215778 0.0329796386578721\\
0.0380888947123224 0.440206481215945\\
0.038210974503067 0.568683682248351\\
0.0383330542938116 0.458560367077717\\
0.0384551340845563 0.0708345282477775\\
0.0385772138753009 -0.421852595354173\\
0.0386992936660455 -0.632922282764554\\
0.0388213734567901 -0.550329796386579\\
0.0389434532475347 -0.125896185833094\\
0.0390655330382794 0.279609979925437\\
0.039187612829024 0.531975910524806\\
0.0393096926197686 0.385144823630628\\
0.0394317724105132 0.169486664754804\\
0.0395538522012579 -0.121307714367651\\
0.0396759319920025 -0.330083166045311\\
0.0397980117827471 -0.187840550616576\\
0.0399200915734917 -0.0375681101233152\\
0.0400421713642364 0.0983653570404359\\
0.040164251154981 0.0220820189274448\\
0.0402863309457256 -0.0129050759965586\\
0.0404084107364702 -0.0983653570404359\\
0.0405304905272148 -0.0937768855749928\\
0.0406525703179595 0.0937768855749928\\
0.0407746501087041 0.17866360768569\\
0.0408967298994487 0.206194436478348\\
0.0410188096901933 -0.0129050759965586\\
0.041140889480938 -0.151132778893031\\
0.0412629692716826 -0.279609979925437\\
0.0413850490624272 -0.0891884141095498\\
0.0415071288531718 0.135073128763981\\
0.0416292086439165 0.242902208201893\\
0.0417512884346611 0.330083166045311\\
0.0418733682254057 -0.016346429595641\\
0.0419954480161503 -0.293375394321767\\
0.0421175278068949 -0.47691425293949\\
0.0422396075976396 -0.311729280183539\\
0.0423616873883842 0.0329796386578721\\
0.0424837671791288 0.421852595354173\\
0.0426058469698734 0.669630054488099\\
0.0427279267606181 0.366790937768856\\
0.0428500065513627 0.0306854029251506\\
0.0429720863421073 -0.458560367077717\\
0.0430941661328519 -0.596214511041009\\
0.0432162459235966 -0.421852595354173\\
0.0433383257143412 -0.0662460567823344\\
0.0434604055050858 0.252079151132779\\
0.0435824852958304 0.366790937768856\\
0.043704565086575 0.366790937768856\\
0.0438266448773197 0.00573558933180384\\
0.0439487246680643 -0.0490392887869229\\
0.0440708044588089 0.00114711786636077\\
0.0441928842495535 0.00516203039862346\\
0.0443149640402982 -0.0220820189274448\\
0.0444370438310428 -0.169486664754804\\
0.0445591236217874 -0.233725265271007\\
0.044681203412532 -0.366790937768856\\
0.0448032832032767 -0.107542299971322\\
0.0449253629940213 0.151132778893031\\
0.0450474427847659 0.513622024663034\\
0.0451695225755105 0.568683682248351\\
0.0452916023662551 0.197017493547462\\
0.0454136821569998 -0.17866360768569\\
0.0455357619477444 -0.495268138801262\\
0.045657841738489 -0.47691425293949\\
0.0457799215292336 -0.330083166045311\\
0.0459020013199783 0.215371379409234\\
0.0460240811107229 0.531975910524806\\
0.0461461609014675 0.366790937768856\\
0.0462682406922121 0.0628047031832521\\
0.0463903204829568 -0.293375394321767\\
0.0465124002737014 -0.366790937768856\\
0.046634480064446 -0.270433036994551\\
0.0467565598551906 0.0708345282477775\\
0.0468786396459352 0.311729280183539\\
0.0470007194366799 0.421852595354173\\
0.0471227992274245 0.187840550616576\\
0.0472448790181691 -0.242902208201893\\
0.0473669588089137 -0.311729280183539\\
0.0474890385996584 -0.206194436478348\\
0.047611118390403 -0.0845999426441067\\
0.0477331981811476 0.0444508173214798\\
0.0478552779718922 0.206194436478348\\
0.0479773577626368 0.0329796386578721\\
0.0480994375533815 -0.0937768855749928\\
0.0482215173441261 -0.0174935474620017\\
0.0483435971348707 0.151132778893031\\
0.0484656769256153 0.311729280183539\\
0.04858775671636 0.261256094063665\\
0.0487098365071046 -0.0232291367938056\\
0.0488319162978492 -0.366790937768856\\
0.0489539960885938 -0.531975910524806\\
0.0490760758793385 -0.531975910524806\\
0.0491981556700831 -0.116719242902208\\
0.0493202354608277 0.513622024663034\\
0.0494423152515723 0.743045597935188\\
0.0495643950423169 0.531975910524806\\
0.0496864748330616 0.197017493547462\\
0.0498085546238062 -0.330083166045311\\
0.0499306344145508 -0.669630054488099\\
0.0500527142052954 -0.550329796386579\\
0.0501747939960401 -0.169486664754804\\
0.0502968737867847 0.206194436478348\\
0.0504189535775293 0.366790937768856\\
0.0505410333682739 0.293375394321767\\
0.0506631131590186 0.0352738743905936\\
0.0507851929497632 -0.0444508173214798\\
0.0509072727405078 -0.107542299971322\\
0.0510293525312524 -0.0255233725265271\\
0.0511514323219971 0.135073128763981\\
0.0512735121127417 0.0891884141095498\\
0.0513955919034863 -0.169486664754804\\
0.0515176716942309 -0.242902208201893\\
0.0516397514849755 -0.17866360768569\\
0.0517618312757202 -0.0444508173214798\\
0.0518839110664648 0.206194436478348\\
0.0520059908572094 0.366790937768856\\
0.052128070647954 0.261256094063665\\
0.0522501504386987 -0.0983653570404359\\
0.0523722302294433 -0.348437051907083\\
0.0524943100201879 -0.4034987094924\\
0.0526163898109325 -0.0421565815887582\\
0.0527384696016771 0.330083166045311\\
0.0528605493924218 0.440206481215945\\
0.0529826291831664 0.348437051907083\\
0.053104708973911 0.0151993117292802\\
0.0532267887646556 -0.47691425293949\\
0.0533488685554003 -0.669630054488099\\
0.0534709483461449 -0.366790937768856\\
0.0535930281368895 0.0937768855749928\\
0.0537151079276341 0.495268138801262\\
0.0538371877183788 0.632922282764554\\
0.0539592675091234 0.458560367077717\\
0.054081347299868 0.0220820189274448\\
0.0542034270906126 -0.366790937768856\\
0.0543255068813572 -0.513622024663034\\
0.0544475866721019 -0.311729280183539\\
0.0545696664628465 -0.00946372239747634\\
0.0546917462535911 0.058216231717809\\
0.0548138260443357 0.0708345282477775\\
0.0549359058350804 0.160309721823917\\
0.055057985625825 0.0513335245196444\\
0.0551800654165696 0.058216231717809\\
0.0553021452073142 0.261256094063665\\
0.0554242249980589 0.366790937768856\\
0.0555463047888035 0.0662460567823344\\
0.0556683845795481 -0.311729280183539\\
0.0557904643702927 -0.531975910524806\\
0.0559125441610373 -0.550329796386579\\
0.056034623951782 -0.130484657298537\\
0.0561567037425266 0.293375394321767\\
0.0562787835332712 0.632922282764554\\
0.0564008633240158 0.632922282764554\\
0.0565229431147605 0.187840550616576\\
0.0566450229055051 -0.385144823630628\\
0.0567671026962497 -0.596214511041009\\
0.0568891824869943 -0.366790937768856\\
0.057011262277739 -0.0937768855749928\\
0.0571333420684836 0.348437051907083\\
0.0572554218592282 0.495268138801262\\
0.0573775016499728 0.252079151132779\\
0.0574995814407174 -0.151132778893031\\
0.0576216612314621 -0.421852595354173\\
0.0577437410222067 -0.366790937768856\\
0.0578658208129513 -0.0662460567823344\\
0.0579879006036959 0.270433036994551\\
0.0581099803944406 0.293375394321767\\
0.0582320601851852 0.293375394321767\\
0.0583541399759298 0.0662460567823344\\
0.0584762197666744 -0.261256094063665\\
0.0585982995574191 -0.233725265271007\\
0.0587203793481637 0.00114711786636077\\
0.0588424591389083 0.102953828505879\\
0.0589645389296529 -0.0421565815887582\\
0.0590866187203975 -0.102953828505879\\
0.0592086985111422 -0.252079151132779\\
0.0593307783018868 -0.187840550616576\\
0.0594528580926314 0.0891884141095498\\
0.059574937883376 0.421852595354173\\
0.0596970176741207 0.632922282764554\\
0.0598190974648653 0.4034987094924\\
0.0599411772556099 -0.160309721823917\\
0.0600632570463545 -0.669630054488099\\
0.0601853368370992 -0.706337826211643\\
0.0603074166278438 -0.513622024663034\\
0.0604294964185884 -0.0255233725265271\\
0.060551576209333 0.632922282764554\\
0.0606736560000776 0.816461141382277\\
0.0607957357908223 0.458560367077717\\
0.0609178155815669 -0.0352738743905936\\
0.0610398953723115 -0.4034987094924\\
0.0611619751630561 -0.47691425293949\\
0.0612840549538008 -0.293375394321767\\
0.0614061347445454 0.0329796386578721\\
0.06152821453529 0.151132778893031\\
0.0616502943260346 0.169486664754804\\
0.0617723741167793 -0.0174935474620017\\
0.0618944539075239 -0.22454832234012\\
0.0620165336982685 0.020934901061084\\
0.0621386134890131 0.242902208201893\\
0.0622606932797577 0.242902208201893\\
0.0623827730705024 0.151132778893031\\
0.062504852861247 -0.0266704903928879\\
0.0626269326519916 -0.293375394321767\\
0.0627490124427362 -0.458560367077717\\
0.0628710922334809 -0.135073128763981\\
0.0629931720242255 0.197017493547462\\
0.0631152518149701 0.440206481215945\\
0.0632373316057147 0.348437051907083\\
0.0633594113964594 -0.0754229997132205\\
0.063481491187204 -0.311729280183539\\
0.0636035709779486 -0.495268138801262\\
0.0637256507686932 -0.293375394321767\\
0.0638477305594378 0.169486664754804\\
0.0639698103501825 0.706337826211643\\
0.0640918901409271 0.568683682248351\\
0.0642139699316717 0.141955835962145\\
0.0643360497224163 -0.261256094063665\\
0.064458129513161 -0.669630054488099\\
0.0645802093039056 -0.596214511041009\\
0.0647022890946502 -0.279609979925437\\
0.0648243688853948 0.330083166045311\\
0.0649464486761395 0.495268138801262\\
0.0650685284668841 0.458560367077717\\
0.0651906082576287 0.151132778893031\\
0.0653126880483733 -0.187840550616576\\
0.0654347678391179 -0.0662460567823344\\
0.0655568476298626 -0.135073128763981\\
0.0656789274206072 -0.0352738743905936\\
0.0658010072113518 0\\
0.0659230870020964 -0.17866360768569\\
0.0660451667928411 -0.385144823630628\\
0.0661672465835857 -0.311729280183539\\
0.0662893263743303 0.169486664754804\\
0.0664114061650749 0.47691425293949\\
0.0665334859558196 0.706337826211643\\
0.0666555657465642 0.421852595354173\\
0.0667776455373088 -0.0375681101233152\\
0.0668997253280534 -0.531975910524806\\
0.067021805118798 -0.853168913105822\\
0.0671438849095427 -0.513622024663034\\
0.0672659647002873 0.0937768855749928\\
0.0673880444910319 0.632922282764554\\
0.0675101242817765 0.568683682248351\\
0.0676322040725212 0.348437051907083\\
0.0677542838632658 -0.0490392887869229\\
0.0678763636540104 -0.550329796386579\\
0.067998443444755 -0.458560367077717\\
0.0681205232354997 -0.0513335245196444\\
0.0682426030262443 0.348437051907083\\
0.0683646828169889 0.348437051907083\\
0.0684867626077335 0.17866360768569\\
0.0686088423984781 -0.130484657298537\\
0.0687309221892228 -0.366790937768856\\
0.0688530019799674 -0.270433036994551\\
0.068975081770712 -0.0513335245196444\\
0.0690971615614566 0.279609979925437\\
0.0692192413522013 0.270433036994551\\
0.0693413211429459 0.028391167192429\\
0.0694634009336905 -0.22454832234012\\
0.0695854807244351 -0.112130771436765\\
0.0697075605151797 0.0306854029251506\\
0.0698296403059244 0.22454832234012\\
0.069951720096669 0.385144823630628\\
0.0700737998874136 0.135073128763981\\
0.0701958796781582 -0.279609979925437\\
0.0703179594689029 -0.669630054488099\\
0.0704400392596475 -0.596214511041009\\
0.0705621190503921 -0.112130771436765\\
0.0706841988411367 0.531975910524806\\
0.0708062786318814 0.816461141382277\\
0.070928358422626 0.632922282764554\\
0.0710504382133706 0.279609979925437\\
0.0711725180041152 -0.4034987094924\\
0.0712945977948598 -0.889876684829366\\
0.0714166775856045 -0.632922282764554\\
0.0715387573763491 -0.151132778893031\\
0.0716608371670937 0.293375394321767\\
0.0717829169578383 0.458560367077717\\
0.071904996748583 0.4034987094924\\
0.0720270765393276 0.121307714367651\\
0.0721491563300722 -0.169486664754804\\
0.0722712361208168 -0.141955835962145\\
0.0723933159115615 -0.028391167192429\\
0.0725153957023061 0.261256094063665\\
0.0726374754930507 0.00458847146544307\\
0.0727595552837953 -0.293375394321767\\
0.0728816350745399 -0.330083166045311\\
0.0730037148652846 -0.270433036994551\\
0.0731257946560292 -0.0255233725265271\\
0.0732478744467738 0.348437051907083\\
0.0733699542375184 0.706337826211643\\
0.0734920340282631 0.311729280183539\\
0.0736141138190077 -0.141955835962145\\
0.0737361936097523 -0.513622024663034\\
0.0738582734004969 -0.568683682248351\\
0.0739803531912415 -0.169486664754804\\
0.0741024329819862 0.293375394321767\\
0.0742245127727308 0.568683682248351\\
0.0743465925634754 0.4034987094924\\
0.07446867235422 0.0375681101233152\\
0.0745907521449647 -0.550329796386579\\
0.0747128319357093 -0.632922282764554\\
0.0748349117264539 -0.206194436478348\\
0.0749569915171985 0.141955835962145\\
0.0750790713079432 0.513622024663034\\
0.0752011510986878 0.596214511041009\\
0.0753232308894324 0.293375394321767\\
0.075445310680177 -0.242902208201893\\
0.0755673904709216 -0.47691425293949\\
0.0756894702616663 -0.366790937768856\\
0.0758115500524109 -0.169486664754804\\
0.0759336298431555 0.116719242902208\\
0.0760557096339001 0.0983653570404359\\
0.0761777894246448 0.058216231717809\\
0.0762998692153894 0.00573558933180384\\
0.076421949006134 -0.0255233725265271\\
0.0765440287968786 0.141955835962145\\
0.0766661085876233 0.421852595354173\\
0.0767881883783679 0.421852595354173\\
0.0769102681691125 -0.0937768855749928\\
0.0770323479598571 -0.47691425293949\\
0.0771544277506018 -0.669630054488099\\
0.0772765075413464 -0.596214511041009\\
0.077398587332091 -0.0490392887869229\\
0.0775206671228356 0.568683682248351\\
0.0776427469135803 0.853168913105822\\
0.0777648267043249 0.632922282764554\\
0.0778869064950695 0.0708345282477775\\
0.0780089862858141 -0.568683682248351\\
0.0781310660765587 -0.669630054488099\\
0.0782531458673034 -0.4034987094924\\
0.078375225658048 -0.0605104674505305\\
0.0784973054487926 0.4034987094924\\
0.0786193852395372 0.458560367077717\\
0.0787414650302819 0.17866360768569\\
0.0788635448210265 -0.293375394321767\\
0.0789856246117711 -0.279609979925437\\
0.0791077044025157 -0.0983653570404359\\
0.0792297841932604 0.121307714367651\\
0.079351863984005 0.330083166045311\\
0.0794739437747496 0.187840550616576\\
0.0795960235654942 -0.00573558933180384\\
0.0797181033562388 -0.330083166045311\\
0.0798401831469835 -0.330083166045311\\
0.0799622629377281 -0.0513335245196444\\
0.0800843427284727 0.330083166045311\\
0.0802064225192173 0.348437051907083\\
0.080328502309962 -0.00516203039862346\\
0.0804505821007066 -0.187840550616576\\
0.0805726618914512 -0.495268138801262\\
0.0806947416821958 -0.330083166045311\\
0.0808168214729404 0.197017493547462\\
0.0809389012636851 0.706337826211643\\
0.0810609810544297 0.669630054488099\\
0.0811830608451743 0.233725265271007\\
0.0813051406359189 -0.366790937768856\\
0.0814272204266636 -0.853168913105822\\
0.0815493002174082 -0.743045597935188\\
0.0816713800081528 -0.311729280183539\\
0.0817934597988974 0.366790937768856\\
0.0819155395896421 0.853168913105822\\
0.0820376193803867 0.706337826211643\\
0.0821596991711313 0.151132778893031\\
0.0822817789618759 -0.242902208201893\\
0.0824038587526205 -0.421852595354173\\
0.0825259385433652 -0.421852595354173\\
0.0826480183341098 -0.107542299971322\\
0.0827700981248544 0.135073128763981\\
0.082892177915599 0.0243762546601663\\
0.0830142577063437 -0.0800114711786636\\
0.0831363374970883 -0.121307714367651\\
0.0832584172878329 -0.016346429595641\\
0.0833804970785775 0.293375394321767\\
0.0835025768693222 0.458560367077717\\
0.0836246566600668 0.366790937768856\\
0.0837467364508114 -0.016346429595641\\
0.083868816241556 -0.4034987094924\\
0.0839908960323006 -0.779753369658732\\
0.0841129758230453 -0.458560367077717\\
0.0842350556137899 0.135073128763981\\
0.0843571354045345 0.531975910524806\\
0.0844792151952791 0.632922282764554\\
0.0846012949860238 0.366790937768856\\
0.0847233747767684 -0.141955835962145\\
0.084845454567513 -0.669630054488099\\
0.0849675343582576 -0.513622024663034\\
0.0850896141490022 -0.141955835962145\\
0.0852116939397469 0.458560367077717\\
0.0853337737304915 0.632922282764554\\
0.0854558535212361 0.311729280183539\\
0.0855779333119807 -0.0800114711786636\\
0.0857000131027254 -0.513622024663034\\
0.08582209289347 -0.596214511041009\\
0.0859441726842146 -0.348437051907083\\
0.0860662524749592 0.311729280183539\\
0.0861883322657039 0.458560367077717\\
0.0863104120564485 0.293375394321767\\
0.0864324918471931 0.130484657298537\\
0.0865545716379377 -0.151132778893031\\
0.0866766514286823 -0.169486664754804\\
0.086798731219427 -0.058216231717809\\
0.0869208110101716 0.187840550616576\\
0.0870428908009162 0.102953828505879\\
0.0871649705916608 -0.160309721823917\\
0.0872870503824055 -0.495268138801262\\
0.0874091301731501 -0.513622024663034\\
0.0875312099638947 0.028391167192429\\
0.0876532897546393 0.421852595354173\\
0.087775369545384 0.779753369658732\\
0.0878974493361286 0.706337826211643\\
0.0880195291268732 0.252079151132779\\
0.0881416089176178 -0.596214511041009\\
0.0882636887083624 -0.963292228276455\\
0.0883857684991071 -0.669630054488099\\
0.0885078482898517 -0.187840550616576\\
0.0886299280805963 0.458560367077717\\
0.0887520078713409 0.706337826211643\\
0.0888740876620856 0.568683682248351\\
0.0889961674528302 0.151132778893031\\
0.0891182472435748 -0.348437051907083\\
0.0892403270343194 -0.495268138801262\\
0.0893624068250641 -0.151132778893031\\
0.0894844866158087 0.197017493547462\\
0.0896065664065533 0.112130771436765\\
0.0897286461972979 -0.00114711786636077\\
0.0898507259880425 -0.0444508173214798\\
0.0899728057787872 -0.330083166045311\\
0.0900948855695318 -0.187840550616576\\
0.0902169653602764 0.187840550616576\\
0.090339045151021 0.495268138801262\\
0.0904611249417657 0.348437051907083\\
0.0905832047325103 -0.0845999426441067\\
0.0907052845232549 -0.385144823630628\\
0.0908273643139995 -0.421852595354173\\
0.0909494441047442 -0.0800114711786636\\
0.0910715238954888 0.151132778893031\\
0.0911936036862334 0.531975910524806\\
0.091315683476978 0.513622024663034\\
0.0914377632677226 -0.0845999426441067\\
0.0915598430584673 -0.669630054488099\\
0.0916819228492119 -0.669630054488099\\
0.0918040026399565 -0.293375394321767\\
0.0919260824307011 0.169486664754804\\
0.0920481622214458 0.743045597935188\\
0.0921702420121904 0.816461141382277\\
0.092292321802935 0.385144823630628\\
0.0924144015936796 -0.270433036994551\\
0.0925364813844243 -0.779753369658732\\
0.0926585611751689 -0.669630054488099\\
0.0927806409659135 -0.206194436478348\\
0.0929027207566581 0.141955835962145\\
0.0930248005474028 0.293375394321767\\
0.0931468803381474 0.47691425293949\\
0.093268960128892 0.206194436478348\\
0.0933910399196366 -0.206194436478348\\
0.0935131197103812 -0.028391167192429\\
0.0936351995011259 0.261256094063665\\
0.0937572792918705 0.187840550616576\\
0.0938793590826151 -0.0800114711786636\\
0.0940014388733597 -0.311729280183539\\
0.0941235186641044 -0.596214511041009\\
0.094245598454849 -0.495268138801262\\
0.0943676782455936 -0.0174935474620017\\
0.0944897580363382 0.513622024663034\\
0.0946118378270829 0.926584456552911\\
0.0947339176178275 0.596214511041009\\
0.0948559974085721 -0.0937768855749928\\
0.0949780771993167 -0.596214511041009\\
0.0951001569900613 -0.706337826211643\\
0.095222236780806 -0.550329796386579\\
0.0953443165715506 0.0232291367938056\\
0.0954663963622952 0.706337826211643\\
0.0955884761530398 0.568683682248351\\
0.0957105559437845 0.169486664754804\\
0.0958326357345291 -0.348437051907083\\
0.0959547155252737 -0.495268138801262\\
0.0960767953160183 -0.293375394321767\\
0.0961988751067629 0.0628047031832521\\
0.0963209548975076 0.421852595354173\\
0.0964430346882522 0.4034987094924\\
0.0965651144789968 0.270433036994551\\
0.0966871942697414 -0.366790937768856\\
0.0968092740604861 -0.421852595354173\\
0.0969313538512307 -0.141955835962145\\
0.0970534336419753 0.0559219959850875\\
0.0971755134327199 0.169486664754804\\
0.0972975932234646 0.130484657298537\\
0.0974196730142092 0.00630914826498423\\
0.0975417528049538 -0.366790937768856\\
0.0976638325956984 -0.160309721823917\\
0.097785912386443 0.206194436478348\\
0.0979079921771877 0.531975910524806\\
0.0980300719679323 0.632922282764554\\
0.0981521517586769 0.0513335245196444\\
0.0982742315494215 -0.385144823630628\\
0.0983963113401662 -0.779753369658732\\
0.0985183911309108 -0.779753369658732\\
0.0986404709216554 -0.311729280183539\\
0.0987625507124 0.596214511041009\\
0.0988846305031447 1\\
0.0990067102938893 0.632922282764554\\
0.0991287900846339 0.261256094063665\\
0.0992508698753785 -0.348437051907083\\
0.0993729496661231 -0.669630054488099\\
0.0994950294568678 -0.596214511041009\\
0.0996171092476124 -0.151132778893031\\
0.099739189038357 0.293375394321767\\
0.0998612688291016 0.293375394321767\\
0.0999833486198463 0.141955835962145\\
0.100105428410591 -0.206194436478348\\
0.100227508201336 -0.0421565815887582\\
0.10034958799208 0.121307714367651\\
0.100471667782825 0.187840550616576\\
0.100593747573569 0.348437051907083\\
0.100715827364314 0.135073128763981\\
0.100837907155059 -0.330083166045311\\
0.100959986945803 -0.669630054488099\\
0.101082066736548 -0.293375394321767\\
0.101204146527293 0.058216231717809\\
0.101326226318037 0.458560367077717\\
0.101448306108782 0.596214511041009\\
0.101570385899526 0.233725265271007\\
0.101692465690271 -0.151132778893031\\
0.101814545481016 -0.596214511041009\\
0.10193662527176 -0.531975910524806\\
0.102058705062505 -0.0467450530542013\\
0.102180784853249 0.632922282764554\\
0.102302864643994 0.632922282764554\\
0.102424944434739 0.233725265271007\\
0.102547024225483 -0.058216231717809\\
0.102669104016228 -0.632922282764554\\
0.102791183806973 -0.779753369658732\\
0.102913263597717 -0.330083166045311\\
0.103035343388462 0.293375394321767\\
0.103157423179206 0.632922282764554\\
0.103279502969951 0.550329796386579\\
0.103401582760696 0.242902208201893\\
0.10352366255144 -0.215371379409234\\
0.103645742342185 -0.233725265271007\\
0.10376782213293 -0.293375394321767\\
0.103889901923674 -0.107542299971322\\
0.104011981714419 0.116719242902208\\
0.104134061505163 -0.0754229997132205\\
0.104256141295908 -0.348437051907083\\
0.104378221086653 -0.252079151132779\\
0.104500300877397 0.206194436478348\\
0.104622380668142 0.330083166045311\\
0.104744460458887 0.632922282764554\\
0.104866540249631 0.550329796386579\\
0.104988620040376 0.0306854029251506\\
0.10511069983112 -0.550329796386579\\
0.105232779621865 -0.926584456552911\\
0.10535485941261 -0.632922282764554\\
0.105476939203354 0.0375681101233152\\
0.105599018994099 0.632922282764554\\
0.105721098784844 0.632922282764554\\
0.105843178575588 0.596214511041009\\
0.105965258366333 0.0937768855749928\\
0.106087338157077 -0.669630054488099\\
0.106209417947822 -0.596214511041009\\
0.106331497738567 -0.141955835962145\\
0.106453577529311 0.311729280183539\\
0.106575657320056 0.293375394321767\\
0.106697737110801 0.279609979925437\\
0.106819816901545 -0.0513335245196444\\
0.10694189669229 -0.4034987094924\\
0.107063976483034 -0.330083166045311\\
0.107186056273779 -0.112130771436765\\
0.107308136064524 0.385144823630628\\
0.107430215855268 0.366790937768856\\
0.107552295646013 0.0662460567823344\\
0.107674375436758 -0.187840550616576\\
0.107796455227502 -0.151132778893031\\
0.107918535018247 -0.141955835962145\\
0.108040614808991 0.0421565815887582\\
0.108162694599736 0.440206481215945\\
0.108284774390481 0.215371379409234\\
0.108406854181225 -0.261256094063665\\
0.10852893397197 -0.531975910524806\\
0.108651013762714 -0.531975910524806\\
0.108773093553459 -0.160309721823917\\
0.108895173344204 0.440206481215945\\
0.109017253134948 0.889876684829366\\
0.109139332925693 0.669630054488099\\
0.109261412716438 0.279609979925437\\
0.109383492507182 -0.513622024663034\\
0.109505572297927 -0.963292228276455\\
0.109627652088671 -0.669630054488099\\
0.109749731879416 -0.187840550616576\\
0.109871811670161 0.311729280183539\\
0.109993891460905 0.632922282764554\\
0.11011597125165 0.632922282764554\\
0.110238051042395 0.0306854029251506\\
0.110360130833139 -0.270433036994551\\
0.110482210623884 -0.169486664754804\\
0.110604290414628 -0.0845999426441067\\
0.110726370205373 0.151132778893031\\
0.110848449996118 0.00860338399770576\\
0.110970529786862 -0.169486664754804\\
0.111092609577607 -0.330083166045311\\
0.111214689368352 -0.330083166045311\\
0.111336769159096 -0.0467450530542013\\
0.111458848949841 0.440206481215945\\
0.111580928740585 0.779753369658732\\
0.11170300853133 0.293375394321767\\
0.111825088322075 -0.135073128763981\\
0.111947168112819 -0.495268138801262\\
0.112069247903564 -0.669630054488099\\
0.112191327694309 -0.348437051907083\\
0.112313407485053 0.279609979925437\\
0.112435487275798 0.743045597935188\\
0.112557567066542 0.458560367077717\\
0.112679646857287 0.020934901061084\\
0.112801726648032 -0.550329796386579\\
0.112923806438776 -0.632922282764554\\
0.113045886229521 -0.261256094063665\\
0.113167966020266 0.121307714367651\\
0.11329004581101 0.669630054488099\\
0.113412125601755 0.706337826211643\\
0.113534205392499 0.215371379409234\\
0.113656285183244 -0.440206481215945\\
0.113778364973989 -0.550329796386579\\
0.113900444764733 -0.366790937768856\\
0.114022524555478 -0.169486664754804\\
0.114144604346223 0.197017493547462\\
0.114266684136967 0.270433036994551\\
0.114388763927712 0.187840550616576\\
0.114510843718456 -0.0800114711786636\\
0.114632923509201 -0.151132778893031\\
0.114755003299946 0.160309721823917\\
0.11487708309069 0.495268138801262\\
0.114999162881435 0.330083166045311\\
0.11512124267218 -0.22454832234012\\
0.115243322462924 -0.348437051907083\\
0.115365402253669 -0.706337826211643\\
0.115487482044413 -0.706337826211643\\
0.115609561835158 0.0197877831947233\\
0.115731641625903 0.779753369658732\\
0.115853721416647 1\\
0.115975801207392 0.513622024663034\\
0.116097880998137 -0.00802982506452538\\
0.116219960788881 -0.596214511041009\\
0.116342040579626 -0.706337826211643\\
0.11646412037037 -0.568683682248351\\
0.116586200161115 -0.058216231717809\\
0.11670827995186 0.632922282764554\\
0.116830359742604 0.531975910524806\\
0.116952439533349 0.0490392887869229\\
0.117074519324093 -0.293375394321767\\
0.117196599114838 -0.102953828505879\\
0.117318678905583 -0.160309721823917\\
0.117440758696327 0.0306854029251506\\
0.117562838487072 0.366790937768856\\
0.117684918277817 0.261256094063665\\
0.117806998068561 -0.130484657298537\\
0.117929077859306 -0.495268138801262\\
0.11805115765005 -0.311729280183539\\
0.118173237440795 0.0628047031832521\\
0.11829531723154 0.4034987094924\\
0.118417397022284 0.261256094063665\\
0.118539476813029 0.112130771436765\\
0.118661556603774 -0.0754229997132205\\
0.118783636394518 -0.632922282764554\\
0.118905716185263 -0.4034987094924\\
0.119027795976007 0.270433036994551\\
0.119149875766752 0.779753369658732\\
0.119271955557497 0.550329796386579\\
0.119394035348241 0.0891884141095498\\
0.119516115138986 -0.348437051907083\\
0.119638194929731 -0.853168913105822\\
0.119760274720475 -0.779753369658732\\
0.11988235451122 -0.242902208201893\\
0.120004434301964 0.596214511041009\\
0.120126514092709 0.963292228276455\\
0.120248593883454 0.596214511041009\\
0.120370673674198 0.0754229997132205\\
0.120492753464943 -0.330083166045311\\
0.120614833255688 -0.495268138801262\\
0.120736913046432 -0.550329796386579\\
0.120858992837177 -0.0306854029251506\\
0.120981072627921 0.242902208201893\\
0.121103152418666 0.058216231717809\\
0.121225232209411 -0.0891884141095498\\
0.121347312000155 -0.141955835962145\\
0.1214693917909 0.102953828505879\\
0.121591471581645 0.311729280183539\\
0.121713551372389 0.495268138801262\\
0.121835631163134 0.293375394321767\\
0.121957710953878 -0.0559219959850875\\
0.122079790744623 -0.550329796386579\\
0.122201870535368 -0.889876684829366\\
0.122323950326112 -0.348437051907083\\
0.122446030116857 0.261256094063665\\
0.122568109907602 0.669630054488099\\
0.122690189698346 0.632922282764554\\
0.122812269489091 0.385144823630628\\
0.122934349279835 -0.252079151132779\\
0.12305642907058 -0.743045597935188\\
0.123178508861325 -0.495268138801262\\
0.123300588652069 -0.0220820189274448\\
0.123422668442814 0.550329796386579\\
0.123544748233559 0.550329796386579\\
0.123666828024303 0.187840550616576\\
0.123788907815048 -0.22454832234012\\
0.123910987605792 -0.550329796386579\\
0.124033067396537 -0.513622024663034\\
0.124155147187282 -0.0937768855749928\\
0.124277226978026 0.495268138801262\\
0.124399306768771 0.421852595354173\\
0.124521386559515 0.233725265271007\\
0.12464346635026 -0.0106108402638371\\
0.124765546141005 -0.252079151132779\\
0.124887625931749 -0.215371379409234\\
0.125009705722494 0.0140521938629194\\
0.125131785513239 0.261256094063665\\
0.125253865303983 0.0937768855749928\\
0.125375945094728 -0.261256094063665\\
0.125498024885472 -0.568683682248351\\
0.125620104676217 -0.348437051907083\\
0.125742184466962 0.141955835962145\\
0.125864264257706 0.596214511041009\\
0.125986344048451 0.816461141382277\\
0.126108423839196 0.596214511041009\\
0.12623050362994 -0.0845999426441067\\
0.126352583420685 -0.853168913105822\\
0.126474663211429 -0.926584456552911\\
0.126596743002174 -0.550329796386579\\
0.126718822792919 0.107542299971322\\
0.126840902583663 0.669630054488099\\
0.126962982374408 0.779753369658732\\
0.127085062165153 0.440206481215945\\
0.127207141955897 -0.17866360768569\\
0.127329221746642 -0.47691425293949\\
0.127451301537386 -0.385144823630628\\
0.127573381328131 0.0559219959850875\\
0.127695461118876 0.270433036994551\\
0.12781754090962 0.0536277602523659\\
0.127939620700365 -0.028391167192429\\
0.12806170049111 -0.261256094063665\\
0.128183780281854 -0.385144823630628\\
0.128305860072599 -0.0490392887869229\\
0.128427939863343 0.47691425293949\\
0.128550019654088 0.531975910524806\\
0.128672099444833 0.17866360768569\\
0.128794179235577 -0.135073128763981\\
0.128916259026322 -0.531975910524806\\
0.129038338817067 -0.4034987094924\\
0.129160418607811 -0.0232291367938056\\
0.129282498398556 0.366790937768856\\
0.1294045781893 0.568683682248351\\
0.129526657980045 0.311729280183539\\
0.12964873777079 -0.348437051907083\\
0.129770817561534 -0.706337826211643\\
0.129892897352279 -0.47691425293949\\
0.130014977143024 -0.141955835962145\\
0.130137056933768 0.513622024663034\\
0.130259136724513 0.926584456552911\\
0.130381216515257 0.632922282764554\\
0.130503296306002 -0.0306854029251506\\
0.130625376096747 -0.632922282764554\\
0.130747455887491 -0.743045597935188\\
0.130869535678236 -0.513622024663034\\
0.130991615468981 0.020934901061084\\
0.131113695259725 0.348437051907083\\
0.13123577505047 0.458560367077717\\
0.131357854841214 0.348437051907083\\
0.131479934631959 -0.125896185833094\\
0.131602014422704 -0.17866360768569\\
0.131724094213448 0.169486664754804\\
0.131846174004193 0.215371379409234\\
0.131968253794937 0.0662460567823344\\
0.132090333585682 -0.160309721823917\\
0.132212413376427 -0.458560367077717\\
0.132334493167171 -0.632922282764554\\
0.132456572957916 -0.293375394321767\\
0.132578652748661 0.311729280183539\\
0.132700732539405 0.816461141382277\\
0.13282281233015 0.816461141382277\\
0.132944892120894 0.242902208201893\\
0.133066971911639 -0.348437051907083\\
0.133189051702384 -0.669630054488099\\
0.133311131493128 -0.706337826211643\\
0.133433211283873 -0.311729280183539\\
0.133555291074618 0.458560367077717\\
0.133677370865362 0.669630054488099\\
0.133799450656107 0.311729280183539\\
0.133921530446851 -0.0754229997132205\\
0.134043610237596 -0.4034987094924\\
0.134165690028341 -0.421852595354173\\
0.134287769819085 -0.116719242902208\\
0.13440984960983 0.270433036994551\\
0.134531929400575 0.47691425293949\\
0.134654009191319 0.385144823630628\\
0.134776088982064 -0.121307714367651\\
0.134898168772808 -0.47691425293949\\
0.135020248563553 -0.348437051907083\\
0.135142328354298 0.00946372239747634\\
0.135264408145042 0.141955835962145\\
0.135386487935787 0.206194436478348\\
0.135508567726532 0.141955835962145\\
0.135630647517276 -0.22454832234012\\
0.135752727308021 -0.330083166045311\\
0.135874807098765 -0.0197877831947233\\
0.13599688688951 0.421852595354173\\
0.136118966680255 0.632922282764554\\
0.136241046470999 0.330083166045311\\
0.136363126261744 -0.215371379409234\\
0.136485206052489 -0.669630054488099\\
0.136607285843233 -0.816461141382277\\
0.136729365633978 -0.596214511041009\\
0.136851445424722 0.206194436478348\\
0.136973525215467 1\\
0.137095605006212 0.889876684829366\\
0.137217684796956 0.385144823630628\\
0.137339764587701 -0.121307714367651\\
0.137461844378446 -0.596214511041009\\
0.13758392416919 -0.743045597935188\\
0.137706003959935 -0.366790937768856\\
0.137828083750679 0.197017493547462\\
0.137950163541424 0.293375394321767\\
0.138072243332169 0.279609979925437\\
0.138194323122913 -0.0845999426441067\\
0.138316402913658 -0.169486664754804\\
0.138438482704403 0.0559219959850875\\
0.138560562495147 0.187840550616576\\
0.138682642285892 0.293375394321767\\
0.138804722076636 0.252079151132779\\
0.138926801867381 -0.0559219959850875\\
0.139048881658126 -0.669630054488099\\
0.13917096144887 -0.550329796386579\\
0.139293041239615 -0.116719242902208\\
0.13941512103036 0.311729280183539\\
0.139537200821104 0.669630054488099\\
0.139659280611849 0.458560367077717\\
0.139781360402593 0.0398623458560367\\
0.139903440193338 -0.47691425293949\\
0.140025519984083 -0.632922282764554\\
0.140147599774827 -0.366790937768856\\
0.140269679565572 0.366790937768856\\
0.140391759356316 0.779753369658732\\
0.140513839147061 0.4034987094924\\
0.140635918937806 0.0628047031832521\\
0.14075799872855 -0.421852595354173\\
0.140880078519295 -0.816461141382277\\
0.14100215831004 -0.568683682248351\\
0.141124238100784 0.141955835962145\\
0.141246317891529 0.596214511041009\\
0.141368397682273 0.568683682248351\\
0.141490477473018 0.421852595354173\\
0.141612557263763 -0.0845999426441067\\
0.141734637054507 -0.330083166045311\\
0.141856716845252 -0.293375394321767\\
0.141978796635997 -0.151132778893031\\
0.142100876426741 0.0937768855749928\\
0.142222956217486 0.0329796386578721\\
0.14234503600823 -0.242902208201893\\
0.142467115798975 -0.348437051907083\\
0.14258919558972 -0.0306854029251506\\
0.142711275380464 0.279609979925437\\
0.142833355171209 0.531975910524806\\
0.142955434961954 0.779753369658732\\
0.143077514752698 0.293375394321767\\
0.143199594543443 -0.4034987094924\\
0.143321674334187 -0.889876684829366\\
0.143443754124932 -0.816461141382277\\
0.143565833915677 -0.311729280183539\\
0.143687913706421 0.4034987094924\\
0.143809993497166 0.816461141382277\\
0.143932073287911 0.669630054488099\\
0.144054153078655 0.348437051907083\\
0.1441762328694 -0.440206481215945\\
0.144298312660144 -0.669630054488099\\
0.144420392450889 -0.330083166045311\\
0.144542472241634 0.121307714367651\\
0.144664552032378 0.270433036994551\\
0.144786631823123 0.279609979925437\\
0.144908711613868 0.160309721823917\\
0.145030791404612 -0.385144823630628\\
0.145152871195357 -0.366790937768856\\
0.145274950986101 -0.102953828505879\\
0.145397030776846 0.270433036994551\\
0.145519110567591 0.385144823630628\\
0.145641190358335 0.215371379409234\\
0.14576327014908 -0.112130771436765\\
0.145885349939825 -0.270433036994551\\
0.146007429730569 -0.206194436478348\\
0.146129509521314 -0.102953828505879\\
0.146251589312058 0.366790937768856\\
0.146373669102803 0.440206481215945\\
0.146495748893548 -0.00860338399770576\\
0.146617828684292 -0.495268138801262\\
0.146739908475037 -0.531975910524806\\
0.146861988265782 -0.385144823630628\\
0.146984068056526 0.0891884141095498\\
0.147106147847271 0.743045597935188\\
0.147228227638015 0.889876684829366\\
0.14735030742876 0.495268138801262\\
0.147472387219505 -0.169486664754804\\
0.147594467010249 -0.853168913105822\\
0.147716546800994 -0.889876684829366\\
0.147838626591738 -0.4034987094924\\
0.147960706382483 0.0329796386578721\\
0.148082786173228 0.531975910524806\\
0.148204865963972 0.743045597935188\\
0.148326945754717 0.385144823630628\\
0.148449025545462 -0.233725265271007\\
0.148571105336206 -0.22454832234012\\
0.148693185126951 -0.0754229997132205\\
0.148815264917695 -0.00802982506452538\\
0.14893734470844 0.0444508173214798\\
0.149059424499185 -0.116719242902208\\
0.149181504289929 -0.270433036994551\\
0.149303584080674 -0.385144823630628\\
0.149425663871419 -0.17866360768569\\
0.149547743662163 0.233725265271007\\
0.149669823452908 0.816461141382277\\
0.149791903243652 0.596214511041009\\
0.149913983034397 0.00344135359908231\\
0.150036062825142 -0.293375394321767\\
0.150158142615886 -0.669630054488099\\
0.150280222406631 -0.632922282764554\\
0.150402302197376 -0.116719242902208\\
0.15052438198812 0.632922282764554\\
0.150646461778865 0.632922282764554\\
0.150768541569609 0.270433036994551\\
0.150890621360354 -0.17866360768569\\
0.151012701151099 -0.568683682248351\\
0.151134780941843 -0.495268138801262\\
0.151256860732588 -0.169486664754804\\
0.151378940523333 0.366790937768856\\
0.151501020314077 0.779753369658732\\
0.151623100104822 0.531975910524806\\
0.151745179895566 -0.169486664754804\\
0.151867259686311 -0.531975910524806\\
0.151989339477056 -0.421852595354173\\
0.1521114192678 -0.348437051907083\\
0.152233499058545 -0.0117579581301979\\
0.15235557884929 0.348437051907083\\
0.152477658640034 0.311729280183539\\
0.152599738430779 0.016346429595641\\
0.152721818221523 -0.151132778893031\\
0.152843898012268 -0.0106108402638371\\
0.152965977803013 0.311729280183539\\
0.153088057593757 0.4034987094924\\
0.153210137384502 0.0605104674505305\\
0.153332217175247 -0.252079151132779\\
0.153454296965991 -0.513622024663034\\
0.153576376756736 -0.853168913105822\\
0.15369845654748 -0.385144823630628\\
0.153820536338225 0.458560367077717\\
0.15394261612897 0.963292228276455\\
0.154064695919714 0.816461141382277\\
0.154186775710459 0.385144823630628\\
0.154308855501204 -0.22454832234012\\
0.154430935291948 -0.816461141382277\\
0.154553015082693 -0.779753369658732\\
0.154675094873437 -0.4034987094924\\
0.154797174664182 0.311729280183539\\
0.154919254454927 0.632922282764554\\
0.155041334245671 0.366790937768856\\
0.155163414036416 -0.0266704903928879\\
0.155285493827161 -0.169486664754804\\
0.155407573617905 -0.17866360768569\\
0.15552965340865 -0.197017493547462\\
0.155651733199394 0.293375394321767\\
0.155773812990139 0.366790937768856\\
0.155895892780884 -0.0306854029251506\\
0.156017972571628 -0.279609979925437\\
0.156140052362373 -0.440206481215945\\
0.156262132153117 -0.169486664754804\\
0.156384211943862 0.160309721823917\\
0.156506291734607 0.458560367077717\\
0.156628371525351 0.242902208201893\\
0.156750451316096 0.107542299971322\\
0.156872531106841 -0.348437051907083\\
0.156994610897585 -0.669630054488099\\
0.15711669068833 -0.0490392887869229\\
0.157238770479074 0.440206481215945\\
0.157360850269819 0.632922282764554\\
0.157482930060564 0.385144823630628\\
0.157605009851308 0.0329796386578721\\
0.157727089642053 -0.669630054488099\\
0.157849169432798 -0.926584456552911\\
0.157971249223542 -0.495268138801262\\
0.158093329014287 0.107542299971322\\
0.158215408805031 0.853168913105822\\
0.158337488595776 0.853168913105822\\
0.158459568386521 0.421852595354173\\
0.158581648177265 -0.116719242902208\\
0.15870372796801 -0.458560367077717\\
0.158825807758755 -0.669630054488099\\
0.158947887549499 -0.366790937768856\\
0.159069967340244 0.242902208201893\\
0.159192047130988 0.135073128763981\\
0.159314126921733 0.0352738743905936\\
0.159436206712478 -0.0421565815887582\\
0.159558286503222 -0.0444508173214798\\
0.159680366293967 0.0891884141095498\\
0.159802446084712 0.385144823630628\\
0.159924525875456 0.495268138801262\\
0.160046605666201 0.151132778893031\\
0.160168685456945 -0.270433036994551\\
0.16029076524769 -0.816461141382277\\
0.160412845038435 -0.706337826211643\\
0.160534924829179 -0.0845999426441067\\
0.160657004619924 0.385144823630628\\
0.160779084410669 0.706337826211643\\
0.160901164201413 0.706337826211643\\
0.161023243992158 0.187840550616576\\
0.161145323782902 -0.568683682248351\\
0.161267403573647 -0.706337826211643\\
0.161389483364392 -0.330083166045311\\
0.161511563155136 0.151132778893031\\
0.161633642945881 0.531975910524806\\
0.161755722736626 0.421852595354173\\
0.16187780252737 0.151132778893031\\
0.161999882318115 -0.330083166045311\\
0.162121962108859 -0.632922282764554\\
0.162244041899604 -0.385144823630628\\
0.162366121690349 0.141955835962145\\
0.162488201481093 0.513622024663034\\
0.162610281271838 0.330083166045311\\
0.162732361062582 0.279609979925437\\
0.162854440853327 -0.0559219959850875\\
0.162976520644072 -0.4034987094924\\
0.163098600434816 -0.242902208201893\\
0.163220680225561 0.107542299971322\\
0.163342760016306 0.187840550616576\\
0.16346483980705 -0.0197877831947233\\
0.163586919597795 -0.197017493547462\\
0.163708999388539 -0.421852595354173\\
0.163831079179284 -0.22454832234012\\
0.163953158970029 0.206194436478348\\
0.164075238760773 0.596214511041009\\
0.164197318551518 0.779753369658732\\
0.164319398342263 0.440206481215945\\
0.164441478133007 -0.330083166045311\\
0.164563557923752 -0.889876684829366\\
0.164685637714496 -0.853168913105822\\
0.164807717505241 -0.495268138801262\\
0.164929797295986 0.160309721823917\\
0.16505187708673 0.853168913105822\\
0.165173956877475 0.816461141382277\\
0.16529603666822 0.385144823630628\\
0.165418116458964 -0.206194436478348\\
0.165540196249709 -0.550329796386579\\
0.165662276040453 -0.4034987094924\\
0.165784355831198 -0.0800114711786636\\
0.165906435621943 0.141955835962145\\
0.166028515412687 0.135073128763981\\
0.166150595203432 0.125896185833094\\
0.166272674994177 -0.252079151132779\\
0.166394754784921 -0.385144823630628\\
0.166516834575666 0.107542299971322\\
0.16663891436641 0.495268138801262\\
0.166760994157155 0.385144823630628\\
0.1668830739479 0.121307714367651\\
0.167005153738644 -0.135073128763981\\
0.167127233529389 -0.531975910524806\\
0.167249313320134 -0.458560367077717\\
0.167371393110878 -0.0628047031832521\\
0.167493472901623 0.4034987094924\\
0.167615552692367 0.632922282764554\\
0.167737632483112 0.206194436478348\\
0.167859712273857 -0.311729280183539\\
0.167981792064601 -0.568683682248351\\
0.168103871855346 -0.47691425293949\\
0.168225951646091 -0.151132778893031\\
0.168348031436835 0.495268138801262\\
0.16847011122758 0.926584456552911\\
0.168592191018324 0.513622024663034\\
0.168714270809069 -0.0891884141095498\\
0.168836350599814 -0.596214511041009\\
0.168958430390558 -0.779753369658732\\
0.169080510181303 -0.458560367077717\\
0.169202589972048 0.00946372239747634\\
0.169324669762792 0.440206481215945\\
0.169446749553537 0.568683682248351\\
0.169568829344281 0.311729280183539\\
0.169690909135026 -0.125896185833094\\
0.169812988925771 -0.187840550616576\\
0.169935068716515 0.0490392887869229\\
0.17005714850726 0.0800114711786636\\
0.170179228298005 0.0306854029251506\\
0.170301308088749 -0.0937768855749928\\
0.170423387879494 -0.421852595354173\\
0.170545467670238 -0.568683682248351\\
0.170667547460983 -0.197017493547462\\
0.170789627251728 0.348437051907083\\
0.170911707042472 0.743045597935188\\
0.171033786833217 0.779753369658732\\
0.171155866623961 0.270433036994551\\
0.171277946414706 -0.330083166045311\\
0.171400026205451 -0.743045597935188\\
0.171522105996195 -0.816461141382277\\
0.17164418578694 -0.293375394321767\\
0.171766265577685 0.421852595354173\\
0.171888345368429 0.706337826211643\\
0.172010425159174 0.385144823630628\\
0.172132504949918 0.0891884141095498\\
0.172254584740663 -0.311729280183539\\
0.172376664531408 -0.513622024663034\\
0.172498744322152 -0.206194436478348\\
0.172620824112897 0.261256094063665\\
0.172742903903642 0.440206481215945\\
0.172864983694386 0.270433036994551\\
0.172987063485131 -0.0490392887869229\\
0.173109143275875 -0.4034987094924\\
0.17323122306662 -0.330083166045311\\
0.173353302857365 -0.121307714367651\\
0.173475382648109 0.151132778893031\\
0.173597462438854 0.348437051907083\\
0.173719542229599 0.197017493547462\\
0.173841622020343 -0.206194436478348\\
0.173963701811088 -0.366790937768856\\
0.174085781601832 0.020934901061084\\
0.174207861392577 0.279609979925437\\
0.174329941183322 0.458560367077717\\
0.174452020974066 0.330083166045311\\
0.174574100764811 -0.0662460567823344\\
0.174696180555556 -0.596214511041009\\
0.1748182603463 -0.853168913105822\\
0.174940340137045 -0.495268138801262\\
0.175062419927789 0.233725265271007\\
0.175184499718534 0.889876684829366\\
0.175306579509279 0.816461141382277\\
0.175428659300023 0.458560367077717\\
0.175550739090768 -0.0220820189274448\\
0.175672818881513 -0.632922282764554\\
0.175794898672257 -0.889876684829366\\
0.175916978463002 -0.4034987094924\\
0.176039058253746 0.17866360768569\\
0.176161138044491 0.293375394321767\\
0.176283217835236 0.293375394321767\\
0.17640529762598 0.169486664754804\\
0.176527377416725 -0.0106108402638371\\
0.17664945720747 -0.0662460567823344\\
0.176771536998214 0.058216231717809\\
0.176893616788959 0.242902208201893\\
0.177015696579703 0.242902208201893\\
0.177137776370448 -0.197017493547462\\
0.177259856161193 -0.596214511041009\\
0.177381935951937 -0.458560367077717\\
0.177504015742682 -0.107542299971322\\
0.177626095533427 0.242902208201893\\
0.177748175324171 0.596214511041009\\
0.177870255114916 0.669630054488099\\
0.17799233490566 0.116719242902208\\
0.178114414696405 -0.47691425293949\\
0.17823649448715 -0.669630054488099\\
0.178358574277894 -0.385144823630628\\
0.178480654068639 0.197017493547462\\
0.178602733859383 0.550329796386579\\
0.178724813650128 0.531975910524806\\
0.178846893440873 0.233725265271007\\
0.178968973231617 -0.330083166045311\\
0.179091053022362 -0.816461141382277\\
0.179213132813107 -0.513622024663034\\
0.179335212603851 0.0845999426441067\\
0.179457292394596 0.47691425293949\\
0.17957937218534 0.568683682248351\\
0.179701451976085 0.458560367077717\\
0.17982353176683 0.0266704903928879\\
0.179945611557574 -0.440206481215945\\
0.180067691348319 -0.458560367077717\\
0.180189771139064 -0.160309721823917\\
0.180311850929808 0.169486664754804\\
0.180433930720553 0.0490392887869229\\
0.180556010511297 -0.141955835962145\\
0.180678090302042 -0.169486664754804\\
0.180800170092787 -0.0708345282477775\\
0.180922249883531 0.112130771436765\\
0.181044329674276 0.47691425293949\\
0.181166409465021 0.779753369658732\\
0.181288489255765 0.293375394321767\\
0.18141056904651 -0.421852595354173\\
0.181532648837254 -0.853168913105822\\
0.181654728627999 -0.816461141382277\\
0.181776808418744 -0.348437051907083\\
0.181898888209488 0.293375394321767\\
0.182020968000233 0.889876684829366\\
0.182143047790978 0.816461141382277\\
0.182265127581722 0.366790937768856\\
0.182387207372467 -0.348437051907083\\
0.182509287163211 -0.706337826211643\\
0.182631366953956 -0.421852595354173\\
0.182753446744701 -0.121307714367651\\
0.182875526535445 0.261256094063665\\
0.18299760632619 0.330083166045311\\
0.183119686116935 0.22454832234012\\
0.183241765907679 -0.261256094063665\\
0.183363845698424 -0.385144823630628\\
0.183485925489168 -0.0467450530542013\\
0.183608005279913 0.187840550616576\\
0.183730085070658 0.4034987094924\\
0.183852164861402 0.187840550616576\\
0.183974244652147 0.00573558933180384\\
0.184096324442892 -0.293375394321767\\
0.184218404233636 -0.4034987094924\\
0.184340484024381 -0.151132778893031\\
0.184462563815125 0.279609979925437\\
0.18458464360587 0.513622024663034\\
0.184706723396615 0.0800114711786636\\
0.184828803187359 -0.233725265271007\\
0.184950882978104 -0.440206481215945\\
0.185072962768849 -0.513622024663034\\
0.185195042559593 -0.0605104674505305\\
0.185317122350338 0.596214511041009\\
0.185439202141082 0.889876684829366\\
0.185561281931827 0.513622024663034\\
0.185683361722572 -0.107542299971322\\
0.185805441513316 -0.706337826211643\\
0.185927521304061 -0.853168913105822\\
0.186049601094806 -0.531975910524806\\
0.18617168088555 -0.0708345282477775\\
0.186293760676295 0.632922282764554\\
0.186415840467039 0.779753369658732\\
0.186537920257784 0.366790937768856\\
0.186660000048529 -0.102953828505879\\
0.186782079839273 -0.270433036994551\\
0.186904159630018 -0.197017493547462\\
0.187026239420762 -0.160309721823917\\
0.187148319211507 0.0628047031832521\\
0.187270399002252 -0.028391167192429\\
0.187392478792996 -0.17866360768569\\
0.187514558583741 -0.366790937768856\\
0.187636638374486 -0.242902208201893\\
0.18775871816523 0.293375394321767\\
0.187880797955975 0.706337826211643\\
0.188002877746719 0.596214511041009\\
0.188124957537464 0.160309721823917\\
0.188247037328209 -0.187840550616576\\
0.188369117118953 -0.779753369658732\\
0.188491196909698 -0.816461141382277\\
0.188613276700443 -0.169486664754804\\
0.188735356491187 0.513622024663034\\
0.188857436281932 0.669630054488099\\
0.188979516072676 0.47691425293949\\
0.189101595863421 0.0174935474620017\\
0.189223675654166 -0.531975910524806\\
0.18934575544491 -0.550329796386579\\
0.189467835235655 -0.330083166045311\\
0.1895899150264 0.293375394321767\\
0.189711994817144 0.706337826211643\\
0.189834074607889 0.458560367077717\\
0.189956154398633 -0.0536277602523659\\
0.190078234189378 -0.440206481215945\\
0.190200313980123 -0.440206481215945\\
0.190322393770867 -0.421852595354173\\
0.190444473561612 0.0891884141095498\\
0.190566553352357 0.385144823630628\\
0.190688633143101 0.330083166045311\\
0.190810712933846 0.0444508173214798\\
0.19093279272459 -0.17866360768569\\
0.191054872515335 -0.0662460567823344\\
0.19117695230608 0.130484657298537\\
0.191299032096824 0.348437051907083\\
0.191421111887569 0.121307714367651\\
0.191543191678314 -0.0983653570404359\\
0.191665271469058 -0.458560367077717\\
0.191787351259803 -0.816461141382277\\
0.191909431050547 -0.385144823630628\\
0.192031510841292 0.366790937768856\\
0.192153590632037 0.853168913105822\\
0.192275670422781 0.779753369658732\\
0.192397750213526 0.495268138801262\\
0.192519830004271 -0.116719242902208\\
0.192641909795015 -0.853168913105822\\
0.19276398958576 -0.853168913105822\\
0.192886069376504 -0.495268138801262\\
0.193008149167249 0.22454832234012\\
0.193130228957994 0.568683682248351\\
0.193252308748738 0.440206481215945\\
0.193374388539483 0.215371379409234\\
0.193496468330227 -0.0983653570404359\\
0.193618548120972 -0.233725265271007\\
0.193740627911717 -0.215371379409234\\
0.193862707702461 0.206194436478348\\
0.193984787493206 0.293375394321767\\
0.194106867283951 -0.0662460567823344\\
0.194228947074695 -0.293375394321767\\
0.19435102686544 -0.366790937768856\\
0.194473106656184 -0.197017493547462\\
0.194595186446929 0.0605104674505305\\
0.194717266237674 0.47691425293949\\
0.194839346028418 0.495268138801262\\
0.194961425819163 0.141955835962145\\
0.195083505609908 -0.293375394321767\\
0.195205585400652 -0.669630054488099\\
0.195327665191397 -0.233725265271007\\
0.195449744982141 0.206194436478348\\
0.195571824772886 0.495268138801262\\
0.195693904563631 0.513622024663034\\
0.195815984354375 0.187840550616576\\
0.19593806414512 -0.4034987094924\\
0.196060143935865 -0.889876684829366\\
0.196182223726609 -0.531975910524806\\
0.196304303517354 -0.0140521938629194\\
0.196426383308098 0.596214511041009\\
0.196548463098843 0.779753369658732\\
0.196670542889588 0.531975910524806\\
0.196792622680332 0.0800114711786636\\
0.196914702471077 -0.513622024663034\\
0.197036782261822 -0.632922282764554\\
0.197158862052566 -0.440206481215945\\
0.197280941843311 0.0662460567823344\\
0.197403021634055 0.197017493547462\\
0.1975251014248 0.102953828505879\\
0.197647181215545 0.141955835962145\\
0.197769261006289 -0.0398623458560367\\
0.197891340797034 0.00344135359908231\\
0.198013420587779 0.252079151132779\\
0.198135500378523 0.47691425293949\\
0.198257580169268 0.22454832234012\\
0.198379659960012 -0.233725265271007\\
0.198501739750757 -0.632922282764554\\
0.198623819541502 -0.816461141382277\\
0.198745899332246 -0.311729280183539\\
0.198867979122991 0.206194436478348\\
0.198990058913736 0.743045597935188\\
0.19911213870448 0.926584456552911\\
0.199234218495225 0.4034987094924\\
0.199356298285969 -0.293375394321767\\
0.199478378076714 -0.743045597935188\\
0.199600457867459 -0.596214511041009\\
0.199722537658203 -0.270433036994551\\
0.199844617448948 0.330083166045311\\
0.199966697239693 0.596214511041009\\
0.200088777030437 0.348437051907083\\
0.200210856821182 -0.0754229997132205\\
0.200332936611926 -0.458560367077717\\
0.200455016402671 -0.385144823630628\\
0.200577096193416 -0.0708345282477775\\
0.20069917598416 0.348437051907083\\
0.200821255774905 0.366790937768856\\
0.20094333556565 0.242902208201893\\
0.201065415356394 0.0266704903928879\\
0.201187495147139 -0.421852595354173\\
0.201309574937883 -0.330083166045311\\
0.201431654728628 0.0186406653283625\\
0.201553734519373 0.293375394321767\\
0.201675814310117 0.125896185833094\\
0.201797894100862 -0.0800114711786636\\
0.201919973891606 -0.261256094063665\\
0.202042053682351 -0.4034987094924\\
0.202164133473096 -0.0490392887869229\\
0.20228621326384 0.4034987094924\\
0.202408293054585 0.816461141382277\\
0.20253037284533 0.550329796386579\\
0.202652452636074 -0.0536277602523659\\
0.202774532426819 -0.669630054488099\\
0.202896612217563 -0.889876684829366\\
0.203018692008308 -0.669630054488099\\
0.203140771799053 -0.107542299971322\\
0.203262851589797 0.706337826211643\\
0.203384931380542 0.926584456552911\\
0.203507011171287 0.568683682248351\\
0.203629090962031 0.007456266131345\\
0.203751170752776 -0.4034987094924\\
0.20387325054352 -0.513622024663034\\
0.203995330334265 -0.385144823630628\\
0.20411741012501 0.0266704903928879\\
0.204239489915754 0.169486664754804\\
0.204361569706499 0.107542299971322\\
0.204483649497244 -0.130484657298537\\
0.204605729287988 -0.215371379409234\\
0.204727809078733 0.0708345282477775\\
0.204849888869477 0.348437051907083\\
0.204971968660222 0.458560367077717\\
0.205094048450967 0.252079151132779\\
0.205216128241711 -0.0559219959850875\\
0.205338208032456 -0.568683682248351\\
0.205460287823201 -0.669630054488099\\
0.205582367613945 -0.261256094063665\\
0.20570444740469 0.270433036994551\\
0.205826527195434 0.669630054488099\\
0.205948606986179 0.47691425293949\\
0.206070686776924 0.112130771436765\\
0.206192766567668 -0.385144823630628\\
0.206314846358413 -0.632922282764554\\
0.206436926149158 -0.421852595354173\\
0.206559005939902 0.160309721823917\\
0.206681085730647 0.743045597935188\\
0.206803165521391 0.568683682248351\\
0.206925245312136 0.135073128763981\\
0.207047325102881 -0.261256094063665\\
0.207169404893625 -0.632922282764554\\
0.20729148468437 -0.596214511041009\\
0.207413564475115 -0.197017493547462\\
0.207535644265859 0.4034987094924\\
0.207657724056604 0.513622024663034\\
0.207779803847348 0.330083166045311\\
0.207901883638093 0.0375681101233152\\
0.208023963428838 -0.151132778893031\\
0.208146043219582 -0.0490392887869229\\
0.208268123010327 -0.0536277602523659\\
0.208390202801072 0.0891884141095498\\
0.208512282591816 -0.00946372239747634\\
0.208634362382561 -0.261256094063665\\
0.208756442173305 -0.596214511041009\\
0.20887852196405 -0.366790937768856\\
0.209000601754795 0.252079151132779\\
0.209122681545539 0.596214511041009\\
0.209244761336284 0.816461141382277\\
0.209366841127029 0.495268138801262\\
0.209488920917773 0.0232291367938056\\
0.209611000708518 -0.706337826211643\\
0.209733080499262 -1\\
0.209855160290007 -0.568683682248351\\
0.209977240080752 0.0754229997132205\\
0.210099319871496 0.669630054488099\\
0.210221399662241 0.596214511041009\\
0.210343479452985 0.421852595354173\\
0.21046555924373 -0.00573558933180384\\
0.210587639034475 -0.458560367077717\\
0.210709718825219 -0.440206481215945\\
0.210831798615964 -0.0375681101233152\\
0.210953878406709 0.348437051907083\\
0.211075958197453 0.135073128763981\\
0.211198037988198 0.0220820189274448\\
0.211320117778942 -0.187840550616576\\
0.211442197569687 -0.311729280183539\\
0.211564277360432 -0.151132778893031\\
0.211686357151176 0.141955835962145\\
0.211808436941921 0.440206481215945\\
0.211930516732666 0.242902208201893\\
0.21205259652341 -0.0800114711786636\\
0.212174676314155 -0.385144823630628\\
0.212296756104899 -0.22454832234012\\
0.212418835895644 0.0352738743905936\\
0.212540915686389 0.242902208201893\\
0.212662995477133 0.458560367077717\\
0.212785075267878 0.270433036994551\\
0.212907155058623 -0.242902208201893\\
0.213029234849367 -0.779753369658732\\
0.213151314640112 -0.596214511041009\\
0.213273394430856 -0.0891884141095498\\
0.213395474221601 0.458560367077717\\
0.213517554012346 0.779753369658732\\
0.21363963380309 0.706337826211643\\
0.213761713593835 0.311729280183539\\
0.21388379338458 -0.440206481215945\\
0.214005873175324 -0.853168913105822\\
0.214127952966069 -0.632922282764554\\
0.214250032756813 -0.102953828505879\\
0.214372112547558 0.206194436478348\\
0.214494192338303 0.348437051907083\\
0.214616272129047 0.421852595354173\\
0.214738351919792 0.151132778893031\\
0.214860431710537 -0.151132778893031\\
0.214982511501281 -0.0983653570404359\\
0.215104591292026 0.206194436478348\\
0.21522667108277 0.252079151132779\\
0.215348750873515 -0.135073128763981\\
0.21547083066426 -0.4034987094924\\
0.215592910455004 -0.47691425293949\\
0.215714990245749 -0.293375394321767\\
0.215837070036494 -0.00401491253226269\\
0.215959149827238 0.47691425293949\\
0.216081229617983 0.853168913105822\\
0.216203309408727 0.47691425293949\\
0.216325389199472 -0.233725265271007\\
0.216447468990217 -0.596214511041009\\
0.216569548780961 -0.568683682248351\\
0.216691628571706 -0.330083166045311\\
0.216813708362451 0.17866360768569\\
0.216935788153195 0.632922282764554\\
0.21705786794394 0.47691425293949\\
0.217179947734684 0.0708345282477775\\
0.217302027525429 -0.458560367077717\\
0.217424107316174 -0.596214511041009\\
0.217546187106918 -0.187840550616576\\
0.217668266897663 0.0937768855749928\\
0.217790346688407 0.421852595354173\\
0.217912426479152 0.568683682248351\\
0.218034506269897 0.330083166045311\\
0.218156586060641 -0.270433036994551\\
0.218278665851386 -0.550329796386579\\
0.218400745642131 -0.22454832234012\\
0.218522825432875 -0.0513335245196444\\
0.21864490522362 0.0845999426441067\\
0.218766985014364 0.0891884141095498\\
0.218889064805109 0.0513335245196444\\
0.219011144595854 -0.130484657298537\\
0.219133224386598 -0.102953828505879\\
0.219255304177343 0.17866360768569\\
0.219377383968088 0.513622024663034\\
0.219499463758832 0.531975910524806\\
0.219621543549577 -0.0708345282477775\\
0.219743623340321 -0.47691425293949\\
0.219865703131066 -0.706337826211643\\
0.219987782921811 -0.669630054488099\\
0.220109862712555 -0.242902208201893\\
0.2202319425033 0.568683682248351\\
0.220354022294045 1\\
0.220476102084789 0.669630054488099\\
0.220598181875534 0.206194436478348\\
0.220720261666278 -0.421852595354173\\
0.220842341457023 -0.669630054488099\\
0.220964421247768 -0.596214511041009\\
0.221086501038512 -0.215371379409234\\
0.221208580829257 0.279609979925437\\
0.221330660620002 0.458560367077717\\
0.221452740410746 0.233725265271007\\
0.221574820201491 -0.141955835962145\\
0.221696899992235 -0.125896185833094\\
0.22181897978298 -0.0708345282477775\\
0.221941059573725 0.0845999426441067\\
0.222063139364469 0.233725265271007\\
0.222185219155214 0.187840550616576\\
0.222307298945959 -0.0937768855749928\\
0.222429378736703 -0.4034987094924\\
0.222551458527448 -0.330083166045311\\
0.222673538318192 -0.0513335245196444\\
0.222795618108937 0.311729280183539\\
0.222917697899682 0.4034987094924\\
0.223039777690426 0.22454832234012\\
0.223161857481171 -0.0800114711786636\\
0.223283937271916 -0.47691425293949\\
0.22340601706266 -0.531975910524806\\
0.223528096853405 -0.00573558933180384\\
0.223650176644149 0.531975910524806\\
0.223772256434894 0.632922282764554\\
0.223894336225639 0.366790937768856\\
0.224016416016383 -0.0467450530542013\\
0.224138495807128 -0.632922282764554\\
0.224260575597872 -0.853168913105822\\
0.224382655388617 -0.495268138801262\\
0.224504735179362 0.197017493547462\\
0.224626814970106 0.706337826211643\\
0.224748894760851 0.706337826211643\\
0.224870974551596 0.366790937768856\\
0.22499305434234 -0.0306854029251506\\
0.225115134133085 -0.4034987094924\\
0.225237213923829 -0.47691425293949\\
0.225359293714574 -0.22454832234012\\
0.225481373505319 0.102953828505879\\
0.225603453296063 0.0513335245196444\\
0.225725533086808 -0.141955835962145\\
0.225847612877553 -0.0983653570404359\\
0.225969692668297 -0.016346429595641\\
0.226091772459042 0.206194436478348\\
0.226213852249786 0.458560367077717\\
0.226335932040531 0.550329796386579\\
0.226458011831276 0.187840550616576\\
0.22658009162202 -0.366790937768856\\
0.226702171412765 -0.743045597935188\\
0.22682425120351 -0.743045597935188\\
0.226946330994254 -0.125896185833094\\
0.227068410784999 0.348437051907083\\
0.227190490575743 0.706337826211643\\
0.227312570366488 0.669630054488099\\
0.227434650157233 0.215371379409234\\
0.227556729947977 -0.458560367077717\\
0.227678809738722 -0.706337826211643\\
0.227800889529467 -0.311729280183539\\
0.227922969320211 0.0800114711786636\\
0.228045049110956 0.421852595354173\\
0.2281671289017 0.385144823630628\\
0.228289208692445 0.17866360768569\\
0.22841128848319 -0.252079151132779\\
0.228533368273934 -0.513622024663034\\
0.228655448064679 -0.311729280183539\\
0.228777527855424 0.141955835962145\\
0.228899607646168 0.421852595354173\\
0.229021687436913 0.197017493547462\\
0.229143767227657 0.102953828505879\\
0.229265847018402 -0.0708345282477775\\
0.229387926809147 -0.233725265271007\\
0.229510006599891 -0.151132778893031\\
0.229632086390636 0.206194436478348\\
0.229754166181381 0.330083166045311\\
0.229876245972125 -0.0467450530542013\\
0.22999832576287 -0.366790937768856\\
0.230120405553614 -0.531975910524806\\
0.230242485344359 -0.293375394321767\\
0.230364565135104 0.141955835962145\\
0.230486644925848 0.550329796386579\\
0.230608724716593 0.853168913105822\\
0.230730804507338 0.596214511041009\\
0.230852884298082 -0.141955835962145\\
0.230974964088827 -0.816461141382277\\
0.231097043879571 -0.853168913105822\\
0.231219123670316 -0.513622024663034\\
0.231341203461061 -0.0232291367938056\\
0.231463283251805 0.596214511041009\\
0.23158536304255 0.816461141382277\\
0.231707442833295 0.458560367077717\\
0.231829522624039 -0.0845999426441067\\
0.231951602414784 -0.348437051907083\\
0.232073682205528 -0.311729280183539\\
0.232195761996273 -0.058216231717809\\
0.232317841787018 0.0536277602523659\\
0.232439921577762 -0.00946372239747634\\
0.232562001368507 -0.0375681101233152\\
0.232684081159251 -0.197017493547462\\
0.232806160949996 -0.270433036994551\\
0.232928240740741 0.0421565815887582\\
0.233050320531485 0.513622024663034\\
0.23317240032223 0.47691425293949\\
0.233294480112975 0.169486664754804\\
0.233416559903719 -0.197017493547462\\
0.233538639694464 -0.495268138801262\\
0.233660719485208 -0.47691425293949\\
0.233782799275953 -0.130484657298537\\
0.233904879066698 0.311729280183539\\
0.234026958857442 0.550329796386579\\
0.234149038648187 0.421852595354173\\
0.234271118438932 -0.141955835962145\\
0.234393198229676 -0.513622024663034\\
0.234515278020421 -0.550329796386579\\
0.234637357811165 -0.270433036994551\\
0.23475943760191 0.215371379409234\\
0.234881517392655 0.706337826211643\\
0.235003597183399 0.706337826211643\\
0.235125676974144 0.151132778893031\\
0.235247756764889 -0.366790937768856\\
0.235369836555633 -0.669630054488099\\
0.235491916346378 -0.531975910524806\\
0.235613996137122 -0.17866360768569\\
0.235736075927867 0.279609979925437\\
0.235858155718612 0.4034987094924\\
0.235980235509356 0.293375394321767\\
0.236102315300101 0.00688270719816461\\
0.236224395090846 -0.135073128763981\\
0.23634647488159 0.0800114711786636\\
0.236468554672335 0.197017493547462\\
0.236590634463079 0.151132778893031\\
0.236712714253824 -0.112130771436765\\
0.236834794044569 -0.348437051907083\\
0.236956873835313 -0.632922282764554\\
0.237078953626058 -0.47691425293949\\
0.237201033416803 0.141955835962145\\
0.237323113207547 0.669630054488099\\
0.237445192998292 0.853168913105822\\
0.237567272789036 0.513622024663034\\
0.237689352579781 -0.0754229997132205\\
0.237811432370526 -0.596214511041009\\
0.23793351216127 -0.779753369658732\\
0.238055591952015 -0.513622024663034\\
0.23817767174276 0.0605104674505305\\
0.238299751533504 0.596214511041009\\
0.238421831324249 0.495268138801262\\
0.238543911114993 0.206194436478348\\
0.238665990905738 -0.160309721823917\\
0.238788070696483 -0.348437051907083\\
0.238910150487227 -0.261256094063665\\
0.239032230277972 0.016346429595641\\
0.239154310068717 0.348437051907083\\
0.239276389859461 0.261256094063665\\
0.239398469650206 0.0243762546601663\\
0.23952054944095 -0.311729280183539\\
0.239642629231695 -0.311729280183539\\
0.23976470902244 -0.0754229997132205\\
0.239886788813184 0.130484657298537\\
0.240008868603929 0.270433036994551\\
0.240130948394674 0.187840550616576\\
0.240253028185418 -0.0375681101233152\\
0.240375107976163 -0.4034987094924\\
0.240497187766907 -0.242902208201893\\
0.240619267557652 0.160309721823917\\
0.240741347348397 0.4034987094924\\
0.240863427139141 0.47691425293949\\
0.240985506929886 0.17866360768569\\
0.24110758672063 -0.270433036994551\\
0.241229666511375 -0.706337826211643\\
0.24135174630212 -0.706337826211643\\
0.241473826092864 -0.22454832234012\\
0.241595905883609 0.440206481215945\\
0.241717985674354 0.853168913105822\\
0.241840065465098 0.632922282764554\\
0.241962145255843 0.261256094063665\\
0.242084225046587 -0.206194436478348\\
0.242206304837332 -0.706337826211643\\
0.242328384628077 -0.632922282764554\\
0.242450464418821 -0.169486664754804\\
0.242572544209566 0.187840550616576\\
0.242694624000311 0.242902208201893\\
0.242816703791055 0.187840550616576\\
0.2429387835818 0.0536277602523659\\
0.243060863372544 -0.0444508173214798\\
0.243182943163289 0.0891884141095498\\
0.243305022954034 0.215371379409234\\
0.243427102744778 0.293375394321767\\
0.243549182535523 0.0243762546601663\\
0.243671262326268 -0.385144823630628\\
0.243793342117012 -0.669630054488099\\
0.243915421907757 -0.4034987094924\\
0.244037501698501 0.0754229997132205\\
0.244159581489246 0.421852595354173\\
0.244281661279991 0.706337826211643\\
0.244403741070735 0.458560367077717\\
0.24452582086148 -0.0754229997132205\\
0.244647900652225 -0.596214511041009\\
0.244769980442969 -0.596214511041009\\
0.244892060233714 -0.252079151132779\\
0.245014140024458 0.330083166045311\\
0.245136219815203 0.596214511041009\\
0.245258299605948 0.385144823630628\\
0.245380379396692 0.0800114711786636\\
0.245502459187437 -0.421852595354173\\
0.245624538978182 -0.669630054488099\\
0.245746618768926 -0.385144823630628\\
0.245868698559671 0.22454832234012\\
0.245990778350415 0.421852595354173\\
0.24611285814116 0.458560367077717\\
0.246234937931905 0.311729280183539\\
0.246357017722649 -0.112130771436765\\
0.246479097513394 -0.330083166045311\\
0.246601177304139 -0.293375394321767\\
0.246723257094883 -0.0220820189274448\\
0.246845336885628 0.0845999426441067\\
0.246967416676372 0.00114711786636077\\
0.247089496467117 -0.187840550616576\\
0.247211576257862 -0.279609979925437\\
0.247333656048606 0.00688270719816461\\
0.247455735839351 0.261256094063665\\
0.247577815630096 0.531975910524806\\
0.24769989542084 0.632922282764554\\
0.247821975211585 0.17866360768569\\
0.247944055002329 -0.458560367077717\\
0.248066134793074 -0.853168913105822\\
0.248188214583819 -0.743045597935188\\
0.248310294374563 -0.252079151132779\\
0.248432374165308 0.421852595354173\\
0.248554453956053 0.853168913105822\\
0.248676533746797 0.669630054488099\\
0.248798613537542 0.261256094063665\\
0.248920693328286 -0.330083166045311\\
0.249042773119031 -0.632922282764554\\
0.249164852909776 -0.421852595354173\\
0.24928693270052 -0.0306854029251506\\
0.249409012491265 0.197017493547462\\
0.249531092282009 0.270433036994551\\
0.249653172072754 0.141955835962145\\
0.249775251863499 -0.215371379409234\\
0.249897331654243 -0.22454832234012\\
0.250019411444988 -0.0197877831947233\\
0.250141491235733 0.215371379409234\\
0.250263571026477 0.311729280183539\\
0.250385650817222 0.160309721823917\\
0.250507730607966 -0.130484657298537\\
0.250629810398711 -0.348437051907083\\
0.250751890189456 -0.279609979925437\\
0.2508739699802 -0.0243762546601663\\
0.250996049770945 0.330083166045311\\
0.25111812956169 0.421852595354173\\
0.251240209352434 0.0708345282477775\\
0.251362289143179 -0.279609979925437\\
0.251484368933923 -0.47691425293949\\
0.251606448724668 -0.421852595354173\\
0.251728528515413 0.0106108402638371\\
0.251850608306157 0.568683682248351\\
0.251972688096902 0.779753369658732\\
0.252094767887647 0.4034987094924\\
0.252216847678391 -0.0845999426441067\\
0.252338927469136 -0.632922282764554\\
0.25246100725988 -0.816461141382277\\
0.252583087050625 -0.47691425293949\\
0.25270516684137 0.00516203039862346\\
0.252827246632114 0.596214511041009\\
0.252949326422859 0.706337826211643\\
0.253071406213604 0.366790937768856\\
0.253193486004348 -0.0800114711786636\\
0.253315565795093 -0.206194436478348\\
0.253437645585837 -0.252079151132779\\
0.253559725376582 -0.187840550616576\\
0.253681805167327 0.0536277602523659\\
0.253803884958071 -0.028391167192429\\
0.253925964748816 -0.187840550616576\\
0.254048044539561 -0.252079151132779\\
0.254170124330305 -0.121307714367651\\
0.25429220412105 0.22454832234012\\
0.254414283911794 0.550329796386579\\
0.254536363702539 0.531975910524806\\
0.254658443493284 0.17866360768569\\
0.254780523284028 -0.206194436478348\\
0.254902603074773 -0.596214511041009\\
0.255024682865517 -0.743045597935188\\
0.255146762656262 -0.206194436478348\\
0.255268842447007 0.348437051907083\\
0.255390922237751 0.596214511041009\\
0.255513002028496 0.47691425293949\\
0.255635081819241 0.130484657298537\\
0.255757161609985 -0.330083166045311\\
0.25587924140073 -0.568683682248351\\
0.256001321191475 -0.311729280183539\\
0.256123400982219 0.0983653570404359\\
0.256245480772964 0.531975910524806\\
0.256367560563708 0.440206481215945\\
0.256489640354453 0.0891884141095498\\
0.256611720145198 -0.252079151132779\\
0.256733799935942 -0.421852595354173\\
0.256855879726687 -0.4034987094924\\
0.256977959517431 -0.112130771436765\\
0.257100039308176 0.348437051907083\\
0.257222119098921 0.293375394321767\\
0.257344198889665 0.125896185833094\\
0.25746627868041 -0.0306854029251506\\
0.257588358471155 -0.107542299971322\\
0.257710438261899 0.0266704903928879\\
0.257832518052644 0.160309721823917\\
0.257954597843388 0.17866360768569\\
0.258076677634133 -0.016346429595641\\
0.258198757424878 -0.311729280183539\\
0.258320837215622 -0.596214511041009\\
0.258442917006367 -0.495268138801262\\
0.258564996797112 0.169486664754804\\
0.258687076587856 0.632922282764554\\
0.258809156378601 0.743045597935188\\
0.258931236169345 0.632922282764554\\
0.25905331596009 0.0937768855749928\\
0.259175395750835 -0.596214511041009\\
0.259297475541579 -0.926584456552911\\
0.259419555332324 -0.596214511041009\\
0.259541635123069 -0.0628047031832521\\
0.259663714913813 0.440206481215945\\
0.259785794704558 0.550329796386579\\
0.259907874495302 0.4034987094924\\
0.260029954286047 0.151132778893031\\
0.260152034076792 -0.261256094063665\\
0.260274113867536 -0.348437051907083\\
0.260396193658281 -0.0628047031832521\\
0.260518273449026 0.206194436478348\\
0.26064035323977 0.0117579581301979\\
0.260762433030515 -0.121307714367651\\
0.260884512821259 -0.135073128763981\\
0.261006592612004 -0.215371379409234\\
0.261128672402749 -0.0845999426441067\\
0.261250752193493 0.215371379409234\\
0.261372831984238 0.47691425293949\\
0.261494911774983 0.270433036994551\\
0.261616991565727 -0.135073128763981\\
0.261739071356472 -0.47691425293949\\
0.261861151147216 -0.366790937768856\\
0.261983230937961 0.00630914826498423\\
0.262105310728706 0.252079151132779\\
0.26222739051945 0.47691425293949\\
0.262349470310195 0.4034987094924\\
0.26247155010094 -0.0800114711786636\\
0.262593629891684 -0.669630054488099\\
0.262715709682429 -0.669630054488099\\
0.262837789473173 -0.22454832234012\\
0.262959869263918 0.215371379409234\\
0.263081949054663 0.669630054488099\\
0.263204028845407 0.706337826211643\\
0.263326108636152 0.366790937768856\\
0.263448188426896 -0.187840550616576\\
0.263570268217641 -0.669630054488099\\
0.263692348008386 -0.596214511041009\\
0.26381442779913 -0.197017493547462\\
0.263936507589875 0.0891884141095498\\
0.26405858738062 0.141955835962145\\
0.264180667171364 0.270433036994551\\
0.264302746962109 0.22454832234012\\
0.264424826752854 -0.0708345282477775\\
0.264546906543598 0.0375681101233152\\
0.264668986334343 0.270433036994551\\
0.264791066125087 0.270433036994551\\
0.264913145915832 -0.102953828505879\\
0.265035225706577 -0.385144823630628\\
0.265157305497321 -0.531975910524806\\
0.265279385288066 -0.440206481215945\\
0.26540146507881 -0.0444508173214798\\
0.265523544869555 0.366790937768856\\
0.2656456246603 0.853168913105822\\
0.265767704451044 0.596214511041009\\
0.265889784241789 -0.00114711786636077\\
0.266011864032534 -0.495268138801262\\
0.266133943823278 -0.596214511041009\\
0.266256023614023 -0.421852595354173\\
0.266378103404767 -0.107542299971322\\
0.266500183195512 0.513622024663034\\
0.266622262986257 0.513622024663034\\
0.266744342777001 0.233725265271007\\
0.266866422567746 -0.22454832234012\\
0.266988502358491 -0.440206481215945\\
0.267110582149235 -0.252079151132779\\
0.26723266193998 -0.0444508173214798\\
0.267354741730724 0.293375394321767\\
0.267476821521469 0.330083166045311\\
0.267598901312214 0.293375394321767\\
0.267720981102958 -0.121307714367651\\
0.267843060893703 -0.366790937768856\\
0.267965140684448 -0.169486664754804\\
0.268087220475192 0.0266704903928879\\
0.268209300265937 0.141955835962145\\
0.268331380056681 0.016346429595641\\
0.268453459847426 -0.00344135359908231\\
0.268575539638171 -0.252079151132779\\
0.268697619428915 -0.233725265271007\\
0.26881969921966 0.102953828505879\\
0.268941779010405 0.47691425293949\\
0.269063858801149 0.596214511041009\\
0.269185938591894 0.261256094063665\\
0.269308018382638 -0.215371379409234\\
0.269430098173383 -0.669630054488099\\
0.269552177964128 -0.743045597935188\\
0.269674257754872 -0.531975910524806\\
0.269796337545617 0.187840550616576\\
0.269918417336362 0.853168913105822\\
0.270040497127106 0.743045597935188\\
0.270162576917851 0.385144823630628\\
0.270284656708595 -0.0708345282477775\\
0.27040673649934 -0.440206481215945\\
0.270528816290085 -0.596214511041009\\
0.270650896080829 -0.348437051907083\\
0.270772975871574 0.0708345282477775\\
0.270895055662318 0.233725265271007\\
0.271017135453063 0.197017493547462\\
0.271139215243808 -0.0628047031832521\\
0.271261295034552 -0.102953828505879\\
0.271383374825297 0.058216231717809\\
0.271505454616042 0.187840550616576\\
0.271627534406786 0.261256094063665\\
0.271749614197531 0.197017493547462\\
0.271871693988275 -0.0754229997132205\\
0.27199377377902 -0.495268138801262\\
0.272115853569765 -0.495268138801262\\
0.272237933360509 -0.116719242902208\\
0.272360013151254 0.293375394321767\\
0.272482092941999 0.495268138801262\\
0.272604172732743 0.385144823630628\\
0.272726252523488 0.0421565815887582\\
0.272848332314232 -0.366790937768856\\
0.272970412104977 -0.596214511041009\\
0.273092491895722 -0.330083166045311\\
0.273214571686466 0.252079151132779\\
0.273336651477211 0.669630054488099\\
0.273458731267956 0.495268138801262\\
0.2735808110587 0.141955835962145\\
0.273702890849445 -0.261256094063665\\
0.273824970640189 -0.706337826211643\\
0.273947050430934 -0.632922282764554\\
0.274069130221679 -0.112130771436765\\
0.274191210012423 0.385144823630628\\
0.274313289803168 0.531975910524806\\
0.274435369593913 0.440206481215945\\
0.274557449384657 0.135073128763981\\
0.274679529175402 -0.17866360768569\\
0.274801608966146 -0.242902208201893\\
0.274923688756891 -0.215371379409234\\
0.275045768547636 -0.0129050759965586\\
0.27516784833838 0.028391167192429\\
0.275289928129125 -0.17866360768569\\
0.27541200791987 -0.279609979925437\\
0.275534087710614 -0.151132778893031\\
0.275656167501359 0.17866360768569\\
0.275778247292103 0.385144823630628\\
0.275900327082848 0.596214511041009\\
0.276022406873593 0.440206481215945\\
0.276144486664337 -0.0983653570404359\\
0.276266566455082 -0.550329796386579\\
0.276388646245827 -0.816461141382277\\
0.276510726036571 -0.495268138801262\\
0.276632805827316 0.0329796386578721\\
0.27675488561806 0.550329796386579\\
0.276876965408805 0.632922282764554\\
0.27699904519955 0.440206481215945\\
0.277121124990294 0.0708345282477775\\
0.277243204781039 -0.495268138801262\\
0.277365284571784 -0.513622024663034\\
0.277487364362528 -0.197017493547462\\
0.277609444153273 0.160309721823917\\
0.277731523944017 0.293375394321767\\
0.277853603734762 0.252079151132779\\
0.277975683525507 0.0232291367938056\\
0.278097763316251 -0.279609979925437\\
0.278219843106996 -0.270433036994551\\
0.278341922897741 -0.130484657298537\\
0.278464002688485 0.197017493547462\\
0.27858608247923 0.330083166045311\\
0.278708162269974 0.0937768855749928\\
0.278830242060719 -0.0628047031832521\\
0.278952321851464 -0.169486664754804\\
0.279074401642208 -0.112130771436765\\
0.279196481432953 0.00458847146544307\\
0.279318561223697 0.293375394321767\\
0.279440641014442 0.252079151132779\\
0.279562720805187 -0.135073128763981\\
0.279684800595931 -0.366790937768856\\
0.279806880386676 -0.495268138801262\\
0.279928960177421 -0.233725265271007\\
0.280051039968165 0.252079151132779\\
0.28017311975891 0.669630054488099\\
0.280295199549654 0.706337826211643\\
0.280417279340399 0.348437051907083\\
0.280539359131144 -0.242902208201893\\
0.280661438921888 -0.816461141382277\\
0.280783518712633 -0.743045597935188\\
0.280905598503378 -0.330083166045311\\
0.281027678294122 0.130484657298537\\
0.281149758084867 0.550329796386579\\
0.281271837875611 0.596214511041009\\
0.281393917666356 0.330083166045311\\
0.281515997457101 -0.169486664754804\\
0.281638077247845 -0.270433036994551\\
0.28176015703859 -0.17866360768569\\
0.281882236829335 0.0117579581301979\\
0.282004316620079 0.0513335245196444\\
0.282126396410824 -0.121307714367651\\
0.282248476201568 -0.151132778893031\\
0.282370555992313 -0.261256094063665\\
0.282492635783058 -0.151132778893031\\
0.282614715573802 0.197017493547462\\
0.282736795364547 0.596214511041009\\
0.282858875155292 0.495268138801262\\
0.282980954946036 0.028391167192429\\
0.283103034736781 -0.311729280183539\\
0.283225114527525 -0.550329796386579\\
0.28334719431827 -0.440206481215945\\
0.283469274109015 -0.058216231717809\\
0.283591353899759 0.385144823630628\\
0.283713433690504 0.596214511041009\\
0.283835513481249 0.293375394321767\\
0.283957593271993 -0.151132778893031\\
0.284079673062738 -0.550329796386579\\
0.284201752853482 -0.440206481215945\\
0.284323832644227 -0.151132778893031\\
0.284445912434972 0.242902208201893\\
0.284567992225716 0.596214511041009\\
0.284690072016461 0.531975910524806\\
0.284812151807206 0.0352738743905936\\
0.28493423159795 -0.4034987094924\\
0.285056311388695 -0.458560367077717\\
0.285178391179439 -0.366790937768856\\
0.285300470970184 -0.0754229997132205\\
0.285422550760929 0.169486664754804\\
0.285544630551673 0.270433036994551\\
0.285666710342418 0.187840550616576\\
0.285788790133163 0.0106108402638371\\
0.285910869923907 -0.0421565815887582\\
0.286032949714652 0.130484657298537\\
0.286155029505396 0.279609979925437\\
0.286277109296141 0.0444508173214798\\
0.286399189086886 -0.187840550616576\\
0.28652126887763 -0.366790937768856\\
0.286643348668375 -0.550329796386579\\
0.28676542845912 -0.385144823630628\\
0.286887508249864 0.17866360768569\\
0.287009588040609 0.669630054488099\\
0.287131667831353 0.743045597935188\\
0.287253747622098 0.421852595354173\\
0.287375827412843 -0.058216231717809\\
0.287497907203587 -0.513622024663034\\
0.287619986994332 -0.743045597935188\\
0.287742066785076 -0.513622024663034\\
0.287864146575821 0.0151993117292802\\
0.287986226366566 0.47691425293949\\
0.28810830615731 0.47691425293949\\
0.288230385948055 0.215371379409234\\
0.2883524657388 -0.0266704903928879\\
0.288474545529544 -0.22454832234012\\
0.288596625320289 -0.279609979925437\\
0.288718705111033 -0.0662460567823344\\
0.288840784901778 0.242902208201893\\
0.288962864692523 0.206194436478348\\
0.289084944483267 -0.00630914826498423\\
0.289207024274012 -0.187840550616576\\
0.289329104064757 -0.242902208201893\\
0.289451183855501 -0.0662460567823344\\
0.289573263646246 0.102953828505879\\
0.28969534343699 0.233725265271007\\
0.289817423227735 0.187840550616576\\
0.28993950301848 -0.0375681101233152\\
0.290061582809224 -0.311729280183539\\
0.290183662599969 -0.261256094063665\\
0.290305742390714 0.112130771436765\\
0.290427822181458 0.348437051907083\\
0.290549901972203 0.385144823630628\\
0.290671981762947 0.22454832234012\\
0.290794061553692 -0.160309721823917\\
0.290916141344437 -0.596214511041009\\
0.291038221135181 -0.669630054488099\\
0.291160300925926 -0.261256094063665\\
0.291282380716671 0.261256094063665\\
0.291404460507415 0.743045597935188\\
0.29152654029816 0.632922282764554\\
0.291648620088904 0.330083166045311\\
0.291770699879649 -0.102953828505879\\
0.291892779670394 -0.568683682248351\\
0.292014859461138 -0.596214511041009\\
0.292136939251883 -0.293375394321767\\
0.292259019042628 0.121307714367651\\
0.292381098833372 0.160309721823917\\
0.292503178624117 0.215371379409234\\
0.292625258414861 0.141955835962145\\
0.292747338205606 0\\
0.292869417996351 0.0421565815887582\\
0.292991497787095 0.125896185833094\\
0.29311357757784 0.270433036994551\\
0.293235657368585 0.028391167192429\\
0.293357737159329 -0.252079151132779\\
0.293479816950074 -0.513622024663034\\
0.293601896740818 -0.4034987094924\\
0.293723976531563 -0.0375681101233152\\
0.293846056322308 0.252079151132779\\
0.293968136113052 0.632922282764554\\
0.294090215903797 0.513622024663034\\
0.294212295694541 0.0708345282477775\\
0.294334375485286 -0.440206481215945\\
0.294456455276031 -0.550329796386579\\
0.294578535066775 -0.348437051907083\\
0.29470061485752 0.0398623458560367\\
0.294822694648265 0.47691425293949\\
0.294944774439009 0.458560367077717\\
0.295066854229754 0.22454832234012\\
0.295188934020499 -0.233725265271007\\
0.295311013811243 -0.550329796386579\\
0.295433093601988 -0.4034987094924\\
0.295555173392732 -0.0513335245196444\\
0.295677253183477 0.293375394321767\\
0.295799332974222 0.421852595354173\\
0.295921412764966 0.4034987094924\\
0.296043492555711 0.0845999426441067\\
0.296165572346455 -0.270433036994551\\
0.2962876521372 -0.311729280183539\\
0.296409731927945 -0.160309721823917\\
0.296531811718689 0.00172067679954115\\
0.296653891509434 0.0174935474620017\\
0.296775971300179 -0.0708345282477775\\
0.296898051090923 -0.0891884141095498\\
0.297020130881668 -0.0513335245196444\\
0.297142210672412 0.0845999426441067\\
0.297264290463157 0.311729280183539\\
0.297386370253902 0.495268138801262\\
0.297508450044646 0.311729280183539\\
0.297630529835391 -0.169486664754804\\
0.297752609626136 -0.550329796386579\\
0.29787468941688 -0.632922282764554\\
0.297996769207625 -0.458560367077717\\
0.298118848998369 -0.007456266131345\\
0.298240928789114 0.531975910524806\\
0.298363008579859 0.743045597935188\\
0.298485088370603 0.47691425293949\\
0.298607168161348 0.0220820189274448\\
0.298729247952093 -0.366790937768856\\
0.298851327742837 -0.495268138801262\\
0.298973407533582 -0.330083166045311\\
0.299095487324326 -0.0845999426441067\\
0.299217567115071 0.197017493547462\\
0.299339646905816 0.293375394321767\\
0.29946172669656 0.121307714367651\\
0.299583806487305 -0.112130771436765\\
0.29970588627805 -0.135073128763981\\
0.299827966068794 0.0536277602523659\\
0.299950045859539 0.0708345282477775\\
0.300072125650283 0.160309721823917\\
0.300194205441028 0.0800114711786636\\
0.300316285231773 -0.102953828505879\\
0.300438365022517 -0.242902208201893\\
0.300560444813262 -0.187840550616576\\
0.300682524604007 0.0444508173214798\\
0.300804604394751 0.252079151132779\\
0.300926684185496 0.279609979925437\\
0.30104876397624 -0.0232291367938056\\
0.301170843766985 -0.197017493547462\\
0.30129292355773 -0.348437051907083\\
0.301415003348474 -0.293375394321767\\
0.301537083139219 0.0983653570404359\\
0.301659162929963 0.495268138801262\\
0.301781242720708 0.596214511041009\\
0.301903322511453 0.242902208201893\\
0.302025402302197 -0.169486664754804\\
0.302147482092942 -0.550329796386579\\
0.302269561883687 -0.669630054488099\\
0.302391641674431 -0.366790937768856\\
0.302513721465176 0.0559219959850875\\
0.30263580125592 0.550329796386579\\
0.302757881046665 0.632922282764554\\
0.30287996083741 0.293375394321767\\
0.303002040628154 -0.0151993117292802\\
0.303124120418899 -0.215371379409234\\
0.303246200209644 -0.261256094063665\\
0.303368280000388 -0.261256094063665\\
0.303490359791133 -0.0306854029251506\\
0.303612439581877 -0.0490392887869229\\
0.303734519372622 -0.107542299971322\\
0.303856599163367 -0.0845999426441067\\
0.303978678954111 -0.007456266131345\\
0.304100758744856 0.279609979925437\\
0.304222838535601 0.366790937768856\\
0.304344918326345 0.385144823630628\\
0.30446699811709 0.0754229997132205\\
0.304589077907834 -0.261256094063665\\
0.304711157698579 -0.531975910524806\\
0.304833237489324 -0.550329796386579\\
0.304955317280068 -0.0983653570404359\\
0.305077397070813 0.279609979925437\\
0.305199476861558 0.531975910524806\\
0.305321556652302 0.4034987094924\\
0.305443636443047 0.125896185833094\\
0.305565716233791 -0.261256094063665\\
0.305687796024536 -0.495268138801262\\
0.305809875815281 -0.270433036994551\\
0.305931955606025 0.0129050759965586\\
0.30605403539677 0.366790937768856\\
0.306176115187515 0.385144823630628\\
0.306298194978259 0.169486664754804\\
0.306420274769004 -0.125896185833094\\
0.306542354559748 -0.366790937768856\\
0.306664434350493 -0.330083166045311\\
0.306786514141238 -0.187840550616576\\
0.306908593931982 0.233725265271007\\
0.307030673722727 0.242902208201893\\
0.307152753513472 0.160309721823917\\
0.307274833304216 0.121307714367651\\
0.307396913094961 -0.0628047031832521\\
0.307518992885705 -0.0329796386578721\\
0.30764107267645 0.00286779466590192\\
0.307763152467195 0.125896185833094\\
0.307885232257939 -0.058216231717809\\
0.308007312048684 -0.252079151132779\\
0.308129391839429 -0.366790937768856\\
0.308251471630173 -0.311729280183539\\
0.308373551420918 0.0983653570404359\\
0.308495631211662 0.4034987094924\\
0.308617711002407 0.632922282764554\\
0.308739790793152 0.513622024663034\\
0.308861870583896 0.0800114711786636\\
0.308983950374641 -0.458560367077717\\
0.309106030165386 -0.743045597935188\\
0.30922810995613 -0.531975910524806\\
0.309350189746875 -0.169486664754804\\
0.309472269537619 0.311729280183539\\
0.309594349328364 0.550329796386579\\
0.309716429119109 0.440206481215945\\
0.309838508909853 0.197017493547462\\
0.309960588700598 -0.160309721823917\\
0.310082668491342 -0.311729280183539\\
0.310204748282087 -0.22454832234012\\
0.310326828072832 -0.00688270719816461\\
0.310448907863576 0.0444508173214798\\
0.310570987654321 0.000573558933180384\\
0.310693067445066 0.0197877831947233\\
0.31081514723581 -0.112130771436765\\
0.310937227026555 -0.0937768855749928\\
0.311059306817299 0.0891884141095498\\
0.311181386608044 0.270433036994551\\
0.311303466398789 0.197017493547462\\
0.311425546189533 -0.0421565815887582\\
0.311547625980278 -0.17866360768569\\
0.311669705771023 -0.270433036994551\\
0.311791785561767 -0.130484657298537\\
0.311913865352512 0.0754229997132205\\
0.312035945143256 0.311729280183539\\
0.312158024934001 0.348437051907083\\
0.312280104724746 0.058216231717809\\
0.31240218451549 -0.293375394321767\\
0.312524264306235 -0.458560367077717\\
0.31264634409698 -0.311729280183539\\
0.312768423887724 -0.0628047031832521\\
0.312890503678469 0.348437051907083\\
0.313012583469213 0.632922282764554\\
0.313134663259958 0.440206481215945\\
0.313256743050703 0.0559219959850875\\
0.313378822841447 -0.366790937768856\\
0.313500902632192 -0.550329796386579\\
0.313622982422937 -0.440206481215945\\
0.313745062213681 -0.169486664754804\\
0.313867142004426 0.151132778893031\\
0.31398922179517 0.348437051907083\\
0.314111301585915 0.385144823630628\\
0.31423338137666 0.125896185833094\\
0.314355461167404 -0.00946372239747634\\
0.314477540958149 0.00946372239747634\\
0.314599620748894 -0.0106108402638371\\
0.314721700539638 -0.0983653570404359\\
0.314843780330383 -0.169486664754804\\
0.314965860121127 -0.233725265271007\\
0.315087939911872 -0.293375394321767\\
0.315210019702617 -0.151132778893031\\
0.315332099493361 0.141955835962145\\
0.315454179284106 0.458560367077717\\
0.315576259074851 0.550329796386579\\
0.315698338865595 0.261256094063665\\
0.31582041865634 -0.135073128763981\\
0.315942498447084 -0.4034987094924\\
0.316064578237829 -0.513622024663034\\
0.316186658028574 -0.385144823630628\\
0.316308737819318 0.0891884141095498\\
0.316430817610063 0.458560367077717\\
0.316552897400808 0.421852595354173\\
0.316674977191552 0.151132778893031\\
0.316797056982297 -0.125896185833094\\
0.316919136773041 -0.311729280183539\\
0.317041216563786 -0.311729280183539\\
0.317163296354531 -0.0891884141095498\\
0.317285376145275 0.215371379409234\\
0.31740745593602 0.366790937768856\\
0.317529535726765 0.242902208201893\\
0.317651615517509 -0.0708345282477775\\
0.317773695308254 -0.233725265271007\\
0.317895775098998 -0.206194436478348\\
0.318017854889743 -0.187840550616576\\
0.318139934680488 -0.0352738743905936\\
0.318262014471232 0.141955835962145\\
0.318384094261977 0.116719242902208\\
0.318506174052721 0.0140521938629194\\
0.318628253843466 -0.0306854029251506\\
0.318750333634211 0.0845999426441067\\
0.318872413424955 0.215371379409234\\
0.3189944932157 0.215371379409234\\
0.319116573006445 0.0559219959850875\\
0.319238652797189 -0.233725265271007\\
0.319360732587934 -0.385144823630628\\
0.319482812378678 -0.513622024663034\\
0.319604892169423 -0.252079151132779\\
0.319726971960168 0.252079151132779\\
0.319849051750912 0.632922282764554\\
0.319971131541657 0.632922282764554\\
0.320093211332402 0.311729280183539\\
0.320215291123146 -0.0536277602523659\\
0.320337370913891 -0.513622024663034\\
0.320459450704635 -0.632922282764554\\
0.32058153049538 -0.385144823630628\\
0.320703610286125 0.00401491253226269\\
0.320825690076869 0.311729280183539\\
0.320947769867614 0.348437051907083\\
0.321069849658359 0.261256094063665\\
0.321191929449103 0.0754229997132205\\
0.321314009239848 -0.0754229997132205\\
0.321436089030592 -0.169486664754804\\
0.321558168821337 -0.0421565815887582\\
0.321680248612082 0.0708345282477775\\
0.321802328402826 -0.0398623458560367\\
0.321924408193571 -0.135073128763981\\
0.322046487984316 -0.151132778893031\\
0.32216856777506 -0.0662460567823344\\
0.322290647565805 0.0421565815887582\\
0.322412727356549 0.187840550616576\\
0.322534807147294 0.279609979925437\\
0.322656886938039 0.112130771436765\\
0.322778966728783 -0.121307714367651\\
0.322901046519528 -0.311729280183539\\
0.323023126310273 -0.233725265271007\\
0.323145206101017 0.0490392887869229\\
0.323267285891762 0.252079151132779\\
0.323389365682506 0.330083166045311\\
0.323511445473251 0.252079151132779\\
0.323633525263996 -0.0800114711786636\\
0.32375560505474 -0.47691425293949\\
0.323877684845485 -0.495268138801262\\
0.32399976463623 -0.187840550616576\\
0.324121844426974 0.130484657298537\\
0.324243924217719 0.47691425293949\\
0.324366004008463 0.531975910524806\\
0.324488083799208 0.330083166045311\\
0.324610163589953 -0.0375681101233152\\
0.324732243380697 -0.4034987094924\\
0.324854323171442 -0.458560367077717\\
0.324976402962186 -0.270433036994551\\
0.325098482752931 -0.0398623458560367\\
0.325220562543676 0.0662460567823344\\
0.32534264233442 0.160309721823917\\
0.325464722125165 0.206194436478348\\
0.32558680191591 0.121307714367651\\
0.325708881706654 0.0800114711786636\\
0.325830961497399 0.187840550616576\\
0.325953041288144 0.151132778893031\\
0.326075121078888 -0.0536277602523659\\
0.326197200869633 -0.311729280183539\\
0.326319280660377 -0.4034987094924\\
0.326441360451122 -0.366790937768856\\
0.326563440241867 -0.0845999426441067\\
0.326685520032611 0.270433036994551\\
0.326807599823356 0.513622024663034\\
0.3269296796141 0.531975910524806\\
0.327051759404845 0.135073128763981\\
0.32717383919559 -0.22454832234012\\
0.327295918986334 -0.458560367077717\\
0.327417998777079 -0.4034987094924\\
0.327540078567824 -0.17866360768569\\
0.327662158358568 0.169486664754804\\
0.327784238149313 0.458560367077717\\
0.327906317940057 0.270433036994551\\
0.328028397730802 0.0467450530542013\\
0.328150477521547 -0.261256094063665\\
0.328272557312291 -0.311729280183539\\
0.328394637103036 -0.160309721823917\\
0.328516716893781 0.0266704903928879\\
0.328638796684525 0.261256094063665\\
0.32876087647527 0.261256094063665\\
0.328882956266014 0.187840550616576\\
0.329005036056759 -0.121307714367651\\
0.329127115847504 -0.197017493547462\\
0.329249195638248 -0.0937768855749928\\
0.329371275428993 -0.0891884141095498\\
0.329493355219738 -0.00172067679954115\\
0.329615435010482 -0.0186406653283625\\
0.329737514801227 -0.0467450530542013\\
0.329859594591971 -0.0983653570404359\\
0.329981674382716 0.0352738743905936\\
0.330103754173461 0.233725265271007\\
0.330225833964205 0.366790937768856\\
0.33034791375495 0.348437051907083\\
0.330469993545695 -0.0352738743905936\\
0.330592073336439 -0.330083166045311\\
0.330714153127184 -0.568683682248351\\
0.330836232917928 -0.513622024663034\\
0.330958312708673 -0.151132778893031\\
0.331080392499418 0.330083166045311\\
0.331202472290162 0.669630054488099\\
0.331324552080907 0.495268138801262\\
0.331446631871652 0.242902208201893\\
0.331568711662396 -0.17866360768569\\
0.331690791453141 -0.440206481215945\\
0.331812871243885 -0.440206481215945\\
0.33193495103463 -0.252079151132779\\
0.332057030825375 0.0662460567823344\\
0.332179110616119 0.187840550616576\\
0.332301190406864 0.233725265271007\\
0.332423270197608 0.0490392887869229\\
0.332545349988353 0.016346429595641\\
0.332667429779098 0.0129050759965586\\
0.332789509569842 0.0329796386578721\\
0.332911589360587 0.0800114711786636\\
0.333033669151332 0.0329796386578721\\
0.333155748942076 -0.0845999426441067\\
0.333277828732821 -0.242902208201893\\
0.333399908523565 -0.187840550616576\\
0.33352198831431 -0.020934901061084\\
0.333644068105055 0.160309721823917\\
0.333766147895799 0.252079151132779\\
0.333888227686544 0.187840550616576\\
0.334010307477289 -0.0129050759965586\\
0.334132387268033 -0.22454832234012\\
0.334254467058778 -0.311729280183539\\
0.334376546849522 -0.151132778893031\\
0.334498626640267 0.169486664754804\\
0.334620706431012 0.385144823630628\\
0.334742786221756 0.330083166045311\\
0.334864866012501 0.0937768855749928\\
0.334986945803246 -0.17866360768569\\
0.33510902559399 -0.495268138801262\\
0.335231105384735 -0.421852595354173\\
0.335353185175479 -0.135073128763981\\
0.335475264966224 0.206194436478348\\
0.335597344756969 0.421852595354173\\
0.335719424547713 0.4034987094924\\
0.335841504338458 0.233725265271007\\
0.335963584129203 -0.0421565815887582\\
0.336085663919947 -0.233725265271007\\
0.336207743710692 -0.311729280183539\\
0.336329823501436 -0.206194436478348\\
0.336451903292181 -0.0467450530542013\\
0.336573983082926 -0.0662460567823344\\
0.33669606287367 -0.0329796386578721\\
0.336818142664415 0.0983653570404359\\
0.33694022245516 0.151132778893031\\
0.337062302245904 0.215371379409234\\
0.337184382036649 0.270433036994551\\
0.337306461827393 0.233725265271007\\
0.337428541618138 -0.0800114711786636\\
0.337550621408883 -0.330083166045311\\
0.337672701199627 -0.47691425293949\\
0.337794780990372 -0.348437051907083\\
0.337916860781117 -0.0106108402638371\\
0.338038940571861 0.270433036994551\\
0.338161020362606 0.458560367077717\\
0.33828310015335 0.421852595354173\\
0.338405179944095 0.107542299971322\\
0.33852725973484 -0.270433036994551\\
0.338649339525584 -0.366790937768856\\
0.338771419316329 -0.261256094063665\\
0.338893499107074 -0.0513335245196444\\
0.339015578897818 0.197017493547462\\
0.339137658688563 0.270433036994551\\
0.339259738479307 0.17866360768569\\
0.339381818270052 -0.0467450530542013\\
0.339503898060797 -0.252079151132779\\
0.339625977851541 -0.22454832234012\\
0.339748057642286 -0.0352738743905936\\
0.339870137433031 0.0800114711786636\\
0.339992217223775 0.141955835962145\\
0.34011429701452 0.169486664754804\\
0.340236376805264 0.116719242902208\\
0.340358456596009 -0.0421565815887582\\
0.340480536386754 -0.0983653570404359\\
0.340602616177498 -0.0306854029251506\\
0.340724695968243 0.00516203039862346\\
0.340846775758987 -0.0754229997132205\\
0.340968855549732 -0.17866360768569\\
0.341090935340477 -0.135073128763981\\
0.341213015131221 -0.0937768855749928\\
0.341335094921966 0.0800114711786636\\
0.341457174712711 0.293375394321767\\
0.341579254503455 0.421852595354173\\
0.3417013342942 0.348437051907083\\
0.341823414084944 -0.0266704903928879\\
0.341945493875689 -0.366790937768856\\
0.342067573666434 -0.531975910524806\\
0.342189653457178 -0.421852595354173\\
0.342311733247923 -0.187840550616576\\
0.342433813038668 0.252079151132779\\
0.342555892829412 0.531975910524806\\
0.342677972620157 0.47691425293949\\
0.342800052410901 0.206194436478348\\
0.342922132201646 -0.107542299971322\\
0.343044211992391 -0.293375394321767\\
0.343166291783135 -0.311729280183539\\
0.34328837157388 -0.215371379409234\\
0.343410451364625 -0.0375681101233152\\
0.343532531155369 0.112130771436765\\
0.343654610946114 0.130484657298537\\
0.343776690736858 0.0117579581301979\\
0.343898770527603 -0.00114711786636077\\
0.344020850318348 0.107542299971322\\
0.344142930109092 0.0845999426441067\\
0.344265009899837 0.0800114711786636\\
0.344387089690582 0.020934901061084\\
0.344509169481326 -0.107542299971322\\
0.344631249272071 -0.187840550616576\\
0.344753329062815 -0.197017493547462\\
0.34487540885356 -0.00688270719816461\\
0.344997488644305 0.187840550616576\\
0.345119568435049 0.261256094063665\\
0.345241648225794 0.107542299971322\\
0.345363728016539 -0.0754229997132205\\
0.345485807807283 -0.215371379409234\\
0.345607887598028 -0.311729280183539\\
0.345729967388772 -0.125896185833094\\
0.345852047179517 0.187840550616576\\
0.345974126970262 0.440206481215945\\
0.346096206761006 0.366790937768856\\
0.346218286551751 0.0845999426441067\\
0.346340366342496 -0.187840550616576\\
0.34646244613324 -0.440206481215945\\
0.346584525923985 -0.458560367077717\\
0.346706605714729 -0.215371379409234\\
0.346828685505474 0.160309721823917\\
0.346950765296219 0.4034987094924\\
0.347072845086963 0.366790937768856\\
0.347194924877708 0.233725265271007\\
0.347317004668453 0.020934901061084\\
0.347439084459197 -0.130484657298537\\
0.347561164249942 -0.233725265271007\\
0.347683244040686 -0.187840550616576\\
0.347805323831431 -0.0983653570404359\\
0.347927403622176 -0.107542299971322\\
0.34804948341292 -0.102953828505879\\
0.348171563203665 -0.0329796386578721\\
0.34829364299441 0.160309721823917\\
0.348415722785154 0.270433036994551\\
0.348537802575899 0.311729280183539\\
0.348659882366643 0.252079151132779\\
0.348781962157388 0.00172067679954115\\
0.348904041948133 -0.293375394321767\\
0.349026121738877 -0.47691425293949\\
0.349148201529622 -0.385144823630628\\
0.349270281320366 -0.0937768855749928\\
0.349392361111111 0.261256094063665\\
0.349514440901856 0.421852595354173\\
0.3496365206926 0.348437051907083\\
0.349758600483345 0.135073128763981\\
0.34988068027409 -0.17866360768569\\
0.350002760064834 -0.366790937768856\\
0.350124839855579 -0.252079151132779\\
0.350246919646323 -0.028391167192429\\
0.350368999437068 0.187840550616576\\
0.350491079227813 0.270433036994551\\
0.350613159018557 0.169486664754804\\
0.350735238809302 -0.0174935474620017\\
0.350857318600047 -0.233725265271007\\
0.350979398390791 -0.242902208201893\\
0.351101478181536 -0.116719242902208\\
0.35122355797228 0.0754229997132205\\
0.351345637763025 0.169486664754804\\
0.35146771755377 0.121307714367651\\
0.351589797344514 0.125896185833094\\
0.351711877135259 0.0129050759965586\\
0.351833956926004 -0.028391167192429\\
0.351956036716748 -0.0266704903928879\\
0.352078116507493 0.0255233725265271\\
0.352200196298237 -0.0513335245196444\\
0.352322276088982 -0.206194436478348\\
0.352444355879727 -0.215371379409234\\
0.352566435670471 -0.197017493547462\\
0.352688515461216 0.0398623458560367\\
0.352810595251961 0.293375394321767\\
0.352932675042705 0.458560367077717\\
0.35305475483345 0.4034987094924\\
0.353176834624194 0.112130771436765\\
0.353298914414939 -0.279609979925437\\
0.353420994205684 -0.568683682248351\\
0.353543073996428 -0.47691425293949\\
0.353665153787173 -0.252079151132779\\
0.353787233577918 0.125896185833094\\
0.353909313368662 0.47691425293949\\
0.354031393159407 0.47691425293949\\
0.354153472950151 0.279609979925437\\
0.354275552740896 -0.0375681101233152\\
0.354397632531641 -0.233725265271007\\
0.354519712322385 -0.311729280183539\\
0.35464179211313 -0.17866360768569\\
0.354763871903875 -0.0266704903928879\\
0.354885951694619 0.0329796386578721\\
0.355008031485364 0.102953828505879\\
0.355130111276108 0.028391167192429\\
0.355252191066853 -0.0444508173214798\\
0.355374270857598 0.0329796386578721\\
0.355496350648342 0.130484657298537\\
0.355618430439087 0.125896185833094\\
0.355740510229832 0.0117579581301979\\
0.355862590020576 -0.0708345282477775\\
0.355984669811321 -0.160309721823917\\
0.356106749602065 -0.151132778893031\\
0.35622882939281 -0.0174935474620017\\
0.356350909183555 0.116719242902208\\
0.356472988974299 0.233725265271007\\
0.356595068765044 0.160309721823917\\
0.356717148555789 -0.0891884141095498\\
0.356839228346533 -0.261256094063665\\
0.356961308137278 -0.252079151132779\\
0.357083387928022 -0.141955835962145\\
0.357205467718767 0.0754229997132205\\
0.357327547509512 0.385144823630628\\
0.357449627300256 0.440206481215945\\
0.357571707091001 0.17866360768569\\
0.357693786881745 -0.112130771436765\\
0.35781586667249 -0.348437051907083\\
0.357937946463235 -0.458560367077717\\
0.358060026253979 -0.293375394321767\\
0.358182106044724 -0.028391167192429\\
0.358304185835469 0.252079151132779\\
0.358426265626213 0.4034987094924\\
0.358548345416958 0.293375394321767\\
0.358670425207702 0.0937768855749928\\
0.358792504998447 -0.0490392887869229\\
0.358914584789192 -0.112130771436765\\
0.359036664579936 -0.206194436478348\\
0.359158744370681 -0.160309721823917\\
0.359280824161426 -0.107542299971322\\
0.35940290395217 -0.125896185833094\\
0.359524983742915 -0.0754229997132205\\
0.359647063533659 0.0605104674505305\\
0.359769143324404 0.197017493547462\\
0.359891223115149 0.293375394321767\\
0.360013302905893 0.293375394321767\\
0.360135382696638 0.0708345282477775\\
0.360257462487383 -0.151132778893031\\
0.360379542278127 -0.311729280183539\\
0.360501622068872 -0.385144823630628\\
0.360623701859616 -0.215371379409234\\
0.360745781650361 0.0983653570404359\\
0.360867861441106 0.311729280183539\\
0.36098994123185 0.311729280183539\\
0.361112021022595 0.197017493547462\\
0.36123410081334 -0.0266704903928879\\
0.361356180604084 -0.270433036994551\\
0.361478260394829 -0.270433036994551\\
0.361600340185573 -0.112130771436765\\
0.361722419976318 0.102953828505879\\
0.361844499767063 0.270433036994551\\
0.361966579557807 0.206194436478348\\
0.362088659348552 0.0536277602523659\\
0.362210739139297 -0.135073128763981\\
0.362332818930041 -0.252079151132779\\
0.362454898720786 -0.252079151132779\\
0.36257697851153 -0.0513335245196444\\
0.362699058302275 0.151132778893031\\
0.36282113809302 0.151132778893031\\
0.362943217883764 0.160309721823917\\
0.363065297674509 0.135073128763981\\
0.363187377465253 0.0306854029251506\\
0.363309457255998 -0.0398623458560367\\
0.363431537046743 -0.0605104674505305\\
0.363553616837487 -0.0628047031832521\\
0.363675696628232 -0.141955835962145\\
0.363797776418977 -0.197017493547462\\
0.363919856209721 -0.187840550616576\\
0.364041936000466 -0.0375681101233152\\
0.36416401579121 0.215371379409234\\
0.364286095581955 0.330083166045311\\
0.3644081753727 0.385144823630628\\
0.364530255163444 0.252079151132779\\
0.364652334954189 -0.0559219959850875\\
0.364774414744934 -0.4034987094924\\
0.364896494535678 -0.47691425293949\\
0.365018574326423 -0.330083166045311\\
0.365140654117167 -0.112130771436765\\
0.365262733907912 0.252079151132779\\
0.365384813698657 0.4034987094924\\
0.365506893489401 0.385144823630628\\
0.365628973280146 0.151132778893031\\
0.365751053070891 -0.121307714367651\\
0.365873132861635 -0.252079151132779\\
0.36599521265238 -0.22454832234012\\
0.366117292443124 -0.102953828505879\\
0.366239372233869 -0.0255233725265271\\
0.366361452024614 0.0800114711786636\\
0.366483531815358 0.0937768855749928\\
0.366605611606103 0.0243762546601663\\
0.366727691396848 -0.0421565815887582\\
0.366849771187592 0.00688270719816461\\
0.366971850978337 0.0605104674505305\\
0.367093930769081 0.0444508173214798\\
0.367216010559826 0.020934901061084\\
0.367338090350571 -0.0467450530542013\\
0.367460170141315 -0.0513335245196444\\
0.36758224993206 -0.0605104674505305\\
0.367704329722805 0.0117579581301979\\
0.367826409513549 0.102953828505879\\
0.367948489304294 0.125896185833094\\
0.368070569095038 0.028391167192429\\
0.368192648885783 -0.17866360768569\\
0.368314728676528 -0.17866360768569\\
0.368436808467272 -0.187840550616576\\
0.368558888258017 -0.0513335245196444\\
0.368680968048762 0.206194436478348\\
0.368803047839506 0.385144823630628\\
0.368925127630251 0.366790937768856\\
0.369047207420995 0.0662460567823344\\
0.36916928721174 -0.160309721823917\\
0.369291367002485 -0.421852595354173\\
0.369413446793229 -0.440206481215945\\
0.369535526583974 -0.233725265271007\\
0.369657606374719 0.0536277602523659\\
0.369779686165463 0.348437051907083\\
0.369901765956208 0.4034987094924\\
0.370023845746952 0.293375394321767\\
0.370145925537697 0.0605104674505305\\
0.370268005328442 -0.0708345282477775\\
0.370390085119186 -0.22454832234012\\
0.370512164909931 -0.242902208201893\\
0.370634244700676 -0.151132778893031\\
0.37075632449142 -0.0708345282477775\\
0.370878404282165 -0.0266704903928879\\
0.371000484072909 0.0232291367938056\\
0.371122563863654 0.130484657298537\\
0.371244643654399 0.169486664754804\\
0.371366723445143 0.22454832234012\\
0.371488803235888 0.135073128763981\\
0.371610883026632 -0.000573558933180384\\
0.371732962817377 -0.141955835962145\\
0.371855042608122 -0.252079151132779\\
0.371977122398866 -0.270433036994551\\
0.372099202189611 -0.121307714367651\\
0.372221281980356 0.151132778893031\\
0.3723433617711 0.22454832234012\\
0.372465441561845 0.22454832234012\\
0.372587521352589 0.121307714367651\\
0.372709601143334 -0.0754229997132205\\
0.372831680934079 -0.233725265271007\\
0.372953760724823 -0.22454832234012\\
0.373075840515568 -0.0467450530542013\\
0.373197920306313 0.125896185833094\\
0.373320000097057 0.279609979925437\\
0.373442079887802 0.197017493547462\\
0.373564159678546 0.0243762546601663\\
0.373686239469291 -0.130484657298537\\
0.373808319260036 -0.293375394321767\\
0.37393039905078 -0.270433036994551\\
0.374052478841525 -0.0754229997132205\\
0.37417455863227 0.135073128763981\\
0.374296638423014 0.197017493547462\\
0.374418718213759 0.252079151132779\\
0.374540798004503 0.215371379409234\\
0.374662877795248 0.0255233725265271\\
0.374784957585993 -0.0937768855749928\\
0.374907037376737 -0.151132778893031\\
0.375029117167482 -0.17866360768569\\
0.375151196958227 -0.160309721823917\\
0.375273276748971 -0.130484657298537\\
0.375395356539716 -0.0708345282477775\\
0.37551743633046 0.0891884141095498\\
0.375639516121205 0.233725265271007\\
0.37576159591195 0.22454832234012\\
0.375883675702694 0.242902208201893\\
0.376005755493439 0.169486664754804\\
0.376127835284184 -0.0754229997132205\\
0.376249915074928 -0.330083166045311\\
0.376371994865673 -0.348437051907083\\
0.376494074656417 -0.270433036994551\\
0.376616154447162 -0.102953828505879\\
0.376738234237907 0.17866360768569\\
0.376860314028651 0.348437051907083\\
0.376982393819396 0.348437051907083\\
0.377104473610141 0.169486664754804\\
0.377226553400885 -0.0662460567823344\\
0.37734863319163 -0.252079151132779\\
0.377470712982374 -0.242902208201893\\
0.377592792773119 -0.151132778893031\\
0.377714872563864 -0.0467450530542013\\
0.377836952354608 0.141955835962145\\
0.377959032145353 0.17866360768569\\
0.378081111936098 0.0754229997132205\\
0.378203191726842 -0.0197877831947233\\
0.378325271517587 -0.0708345282477775\\
0.378447351308331 -0.0891884141095498\\
0.378569431099076 -0.0536277602523659\\
0.378691510889821 0.0375681101233152\\
0.378813590680565 0.0845999426441067\\
0.37893567047131 0.0983653570404359\\
0.379057750262055 0.0800114711786636\\
0.379179830052799 -0.0174935474620017\\
0.379301909843544 -0.0421565815887582\\
0.379423989634288 -0.0444508173214798\\
0.379546069425033 -0.0891884141095498\\
0.379668149215778 -0.116719242902208\\
0.379790229006522 -0.0398623458560367\\
0.379912308797267 -0.0243762546601663\\
0.380034388588011 0.020934901061084\\
0.380156468378756 0.169486664754804\\
0.380278548169501 0.242902208201893\\
0.380400627960245 0.206194436478348\\
0.38052270775099 0.0845999426441067\\
0.380644787541735 -0.107542299971322\\
0.380766867332479 -0.311729280183539\\
0.380888947123224 -0.330083166045311\\
0.381011026913968 -0.242902208201893\\
0.381133106704713 -0.0559219959850875\\
0.381255186495458 0.252079151132779\\
0.381377266286202 0.4034987094924\\
0.381499346076947 0.330083166045311\\
0.381621425867692 0.17866360768569\\
0.381743505658436 -0.0559219959850875\\
0.381865585449181 -0.293375394321767\\
0.381987665239925 -0.330083166045311\\
0.38210974503067 -0.215371379409234\\
0.382231824821415 -0.0891884141095498\\
0.382353904612159 0.0800114711786636\\
0.382475984402904 0.169486664754804\\
0.382598064193649 0.17866360768569\\
0.382720143984393 0.141955835962145\\
0.382842223775138 0.0662460567823344\\
0.382964303565882 -0.0255233725265271\\
0.383086383356627 -0.0937768855749928\\
0.383208463147372 -0.0754229997132205\\
0.383330542938116 -0.0983653570404359\\
0.383452622728861 -0.0800114711786636\\
0.383574702519606 0.00630914826498423\\
0.38369678231035 0.0421565815887582\\
0.383818862101095 0.0662460567823344\\
0.383940941891839 0.0490392887869229\\
0.384063021682584 0.0628047031832521\\
0.384185101473329 -0.0352738743905936\\
0.384307181264073 -0.0662460567823344\\
0.384429261054818 -0.0708345282477775\\
0.384551340845563 -0.0375681101233152\\
0.384673420636307 0.0800114711786636\\
0.384795500427052 0.125896185833094\\
0.384917580217796 0.135073128763981\\
0.385039660008541 0.0559219959850875\\
0.385161739799286 -0.0490392887869229\\
0.38528381959003 -0.206194436478348\\
0.385405899380775 -0.242902208201893\\
0.38552797917152 -0.141955835962145\\
0.385650058962264 -0.00458847146544307\\
0.385772138753009 0.17866360768569\\
0.385894218543753 0.311729280183539\\
0.386016298334498 0.293375394321767\\
0.386138378125243 0.130484657298537\\
0.386260457915987 -0.0662460567823344\\
0.386382537706732 -0.270433036994551\\
0.386504617497477 -0.330083166045311\\
0.386626697288221 -0.233725265271007\\
0.386748777078966 -0.107542299971322\\
0.38687085686971 0.0891884141095498\\
0.386992936660455 0.233725265271007\\
0.3871150164512 0.279609979925437\\
0.387237096241944 0.187840550616576\\
0.387359176032689 0.0800114711786636\\
0.387481255823434 -0.0306854029251506\\
0.387603335614178 -0.151132778893031\\
0.387725415404923 -0.169486664754804\\
0.387847495195667 -0.187840550616576\\
0.387969574986412 -0.135073128763981\\
0.388091654777157 -0.0306854029251506\\
0.388213734567901 0.0708345282477775\\
0.388335814358646 0.141955835962145\\
0.38845789414939 0.206194436478348\\
0.388579973940135 0.169486664754804\\
0.38870205373088 0.0352738743905936\\
0.388824133521624 -0.0754229997132205\\
0.388946213312369 -0.160309721823917\\
0.389068293103114 -0.187840550616576\\
0.389190372893858 -0.112130771436765\\
0.389312452684603 0.0306854029251506\\
0.389434532475347 0.130484657298537\\
0.389556612266092 0.151132778893031\\
0.389678692056837 0.0708345282477775\\
0.389800771847581 -0.028391167192429\\
0.389922851638326 -0.130484657298537\\
0.390044931429071 -0.141955835962145\\
0.390167011219815 -0.0605104674505305\\
0.39028909101056 0.0662460567823344\\
0.390411170801304 0.17866360768569\\
0.390533250592049 0.187840550616576\\
0.390655330382794 0.0983653570404359\\
0.390777410173538 -0.0421565815887582\\
0.390899489964283 -0.160309721823917\\
0.391021569755028 -0.233725265271007\\
0.391143649545772 -0.197017493547462\\
0.391265729336517 -0.0662460567823344\\
0.391387809127261 0.0800114711786636\\
0.391509888918006 0.17866360768569\\
0.391631968708751 0.242902208201893\\
0.391754048499495 0.206194436478348\\
0.39187612829024 0.0800114711786636\\
0.391998208080985 -0.0536277602523659\\
0.392120287871729 -0.17866360768569\\
0.392242367662474 -0.233725265271007\\
0.392364447453218 -0.206194436478348\\
0.392486527243963 -0.125896185833094\\
0.392608607034708 -0.00516203039862346\\
0.392730686825452 0.135073128763981\\
0.392852766616197 0.242902208201893\\
0.392974846406942 0.233725265271007\\
0.393096926197686 0.17866360768569\\
0.393219005988431 0.0352738743905936\\
0.393341085779175 -0.141955835962145\\
0.39346316556992 -0.22454832234012\\
0.393585245360665 -0.261256094063665\\
0.393707325151409 -0.160309721823917\\
0.393829404942154 -0.020934901061084\\
0.393951484732898 0.151132778893031\\
0.394073564523643 0.215371379409234\\
0.394195644314388 0.187840550616576\\
0.394317724105132 0.121307714367651\\
0.394439803895877 -0.0421565815887582\\
0.394561883686622 -0.141955835962145\\
0.394683963477366 -0.160309721823917\\
0.394806043268111 -0.0983653570404359\\
0.394928123058856 -0.00688270719816461\\
0.3950502028496 0.0662460567823344\\
0.395172282640345 0.125896185833094\\
0.395294362431089 0.0513335245196444\\
0.395416442221834 0.000573558933180384\\
0.395538522012579 -0.0845999426441067\\
0.395660601803323 -0.112130771436765\\
0.395782681594068 -0.0708345282477775\\
0.395904761384812 0.00860338399770576\\
0.396026841175557 0.102953828505879\\
0.396148920966302 0.125896185833094\\
0.396271000757046 0.160309721823917\\
0.396393080547791 0.0398623458560367\\
0.396515160338536 -0.0628047031832521\\
0.39663724012928 -0.141955835962145\\
0.396759319920025 -0.169486664754804\\
0.396881399710769 -0.151132778893031\\
0.397003479501514 -0.0662460567823344\\
0.397125559292259 0.0605104674505305\\
0.397247639083003 0.135073128763981\\
0.397369718873748 0.197017493547462\\
0.397491798664493 0.17866360768569\\
0.397613878455237 0.121307714367651\\
0.397735958245982 0.00286779466590192\\
0.397858038036726 -0.141955835962145\\
0.397980117827471 -0.242902208201893\\
0.398102197618216 -0.233725265271007\\
0.39822427740896 -0.141955835962145\\
0.398346357199705 -0.0559219959850875\\
0.39846843699045 0.151132778893031\\
0.398590516781194 0.270433036994551\\
0.398712596571939 0.261256094063665\\
0.398834676362683 0.160309721823917\\
0.398956756153428 0.016346429595641\\
0.399078835944173 -0.135073128763981\\
0.399200915734917 -0.242902208201893\\
0.399322995525662 -0.242902208201893\\
0.399445075316407 -0.160309721823917\\
0.399567155107151 0.0329796386578721\\
0.399689234897896 0.125896185833094\\
0.39981131468864 0.187840550616576\\
0.399933394479385 0.151132778893031\\
0.40005547427013 0.0983653570404359\\
0.400177554060874 -0.0255233725265271\\
0.400299633851619 -0.135073128763981\\
0.400421713642364 -0.0845999426441067\\
0.400543793433108 -0.0708345282477775\\
0.400665873223853 -0.00573558933180384\\
0.400787953014597 0.00946372239747634\\
0.400910032805342 0.0662460567823344\\
0.401032112596087 0.058216231717809\\
0.401154192386831 -0.0117579581301979\\
0.401276272177576 -0.0490392887869229\\
0.401398351968321 -0.0800114711786636\\
0.401520431759065 -0.0140521938629194\\
0.40164251154981 -0.0140521938629194\\
0.401764591340554 0.0421565815887582\\
0.401886671131299 0.116719242902208\\
0.402008750922044 0.116719242902208\\
0.402130830712788 0.0708345282477775\\
0.402252910503533 -0.0255233725265271\\
0.402374990294277 -0.0891884141095498\\
0.402497070085022 -0.151132778893031\\
0.402619149875767 -0.169486664754804\\
0.402741229666511 -0.112130771436765\\
0.402863309457256 0.00802982506452538\\
0.402985389248001 0.130484657298537\\
0.403107469038745 0.197017493547462\\
0.40322954882949 0.22454832234012\\
0.403351628620234 0.160309721823917\\
0.403473708410979 0.00946372239747634\\
0.403595788201724 -0.151132778893031\\
0.403717867992468 -0.242902208201893\\
0.403839947783213 -0.233725265271007\\
0.403962027573958 -0.169486664754804\\
0.404084107364702 -0.0306854029251506\\
0.404206187155447 0.130484657298537\\
0.404328266946191 0.261256094063665\\
0.404450346736936 0.252079151132779\\
0.404572426527681 0.141955835962145\\
0.404694506318425 0.020934901061084\\
0.40481658610917 -0.112130771436765\\
0.404938665899915 -0.215371379409234\\
0.405060745690659 -0.233725265271007\\
0.405182825481404 -0.125896185833094\\
0.405304905272148 -0.0140521938629194\\
0.405426985062893 0.0891884141095498\\
0.405549064853638 0.169486664754804\\
0.405671144644382 0.160309721823917\\
0.405793224435127 0.0937768855749928\\
0.405915304225872 0.00688270719816461\\
0.406037384016616 -0.0983653570404359\\
0.406159463807361 -0.121307714367651\\
0.406281543598105 -0.0536277602523659\\
0.40640362338885 -0.0306854029251506\\
0.406525703179595 0.0232291367938056\\
0.406647782970339 0.0754229997132205\\
0.406769862761084 0.0628047031832521\\
0.406891942551829 -0.0151993117292802\\
0.407014022342573 -0.0421565815887582\\
0.407136102133318 -0.0628047031832521\\
0.407258181924062 -0.058216231717809\\
0.407380261714807 -0.0106108402638371\\
0.407502341505552 0.0375681101233152\\
0.407624421296296 0.107542299971322\\
0.407746501087041 0.130484657298537\\
0.407868580877786 0.0845999426441067\\
0.40799066066853 -0.0106108402638371\\
0.408112740459275 -0.0754229997132205\\
0.408234820250019 -0.151132778893031\\
0.408356900040764 -0.17866360768569\\
0.408478979831509 -0.135073128763981\\
0.408601059622253 -0.00802982506452538\\
0.408723139412998 0.125896185833094\\
0.408845219203743 0.187840550616576\\
0.408967298994487 0.242902208201893\\
0.409089378785232 0.151132778893031\\
0.409211458575976 0.0306854029251506\\
0.409333538366721 -0.112130771436765\\
0.409455618157466 -0.242902208201893\\
0.40957769794821 -0.242902208201893\\
0.409699777738955 -0.17866360768569\\
0.4098218575297 -0.0467450530542013\\
0.409943937320444 0.107542299971322\\
0.410066017111189 0.252079151132779\\
0.410188096901933 0.252079151132779\\
0.410310176692678 0.151132778893031\\
0.410432256483423 0.0421565815887582\\
0.410554336274167 -0.102953828505879\\
0.410676416064912 -0.206194436478348\\
0.410798495855656 -0.215371379409234\\
0.410920575646401 -0.141955835962145\\
0.411042655437146 -0.0151993117292802\\
0.41116473522789 0.0891884141095498\\
0.411286815018635 0.130484657298537\\
0.41140889480938 0.130484657298537\\
0.411530974600124 0.112130771436765\\
0.411653054390869 0.0186406653283625\\
0.411775134181613 -0.0754229997132205\\
0.411897213972358 -0.0891884141095498\\
0.412019293763103 -0.0559219959850875\\
0.412141373553847 -0.0421565815887582\\
0.412263453344592 -0.00344135359908231\\
0.412385533135337 0.0467450530542013\\
0.412507612926081 0.0490392887869229\\
0.412629692716826 0.0174935474620017\\
0.41275177250757 -0.0329796386578721\\
0.412873852298315 -0.0605104674505305\\
0.41299593208906 -0.0467450530542013\\
0.413118011879804 -0.0106108402638371\\
0.413240091670549 0.007456266131345\\
0.413362171461294 0.0845999426441067\\
0.413484251252038 0.141955835962145\\
0.413606331042783 0.0983653570404359\\
0.413728410833527 0.0243762546601663\\
0.413850490624272 -0.0662460567823344\\
0.413972570415017 -0.141955835962145\\
0.414094650205761 -0.197017493547462\\
0.414216729996506 -0.160309721823917\\
0.414338809787251 -0.0490392887869229\\
0.414460889577995 0.121307714367651\\
0.41458296936874 0.206194436478348\\
0.414705049159484 0.215371379409234\\
0.414827128950229 0.187840550616576\\
0.414949208740974 0.0444508173214798\\
0.415071288531718 -0.102953828505879\\
0.415193368322463 -0.233725265271007\\
0.415315448113208 -0.233725265271007\\
0.415437527903952 -0.17866360768569\\
0.415559607694697 -0.0662460567823344\\
0.415681687485441 0.0708345282477775\\
0.415803767276186 0.187840550616576\\
0.415925847066931 0.270433036994551\\
0.416047926857675 0.169486664754804\\
0.41617000664842 0.0605104674505305\\
0.416292086439165 -0.0559219959850875\\
0.416414166229909 -0.169486664754804\\
0.416536246020654 -0.206194436478348\\
0.416658325811398 -0.169486664754804\\
0.416780405602143 -0.0513335245196444\\
0.416902485392888 0.0444508173214798\\
0.417024565183632 0.135073128763981\\
0.417146644974377 0.141955835962145\\
0.417268724765122 0.0983653570404359\\
0.417390804555866 0.0628047031832521\\
0.417512884346611 -0.058216231717809\\
0.417634964137355 -0.107542299971322\\
0.4177570439281 -0.0845999426441067\\
0.417879123718845 -0.0352738743905936\\
0.418001203509589 0.00688270719816461\\
0.418123283300334 0.0467450530542013\\
0.418245363091079 0.0845999426441067\\
0.418367442881823 0.0151993117292802\\
0.418489522672568 -0.0375681101233152\\
0.418611602463312 -0.0754229997132205\\
0.418733682254057 -0.0845999426441067\\
0.418855762044802 -0.0243762546601663\\
0.418977841835546 0.0243762546601663\\
0.419099921626291 0.0937768855749928\\
0.419222001417035 0.141955835962145\\
0.41934408120778 0.125896185833094\\
0.419466160998525 0.016346429595641\\
0.419588240789269 -0.0708345282477775\\
0.419710320580014 -0.116719242902208\\
0.419832400370759 -0.197017493547462\\
0.419954480161503 -0.160309721823917\\
0.420076559952248 -0.0375681101233152\\
0.420198639742992 0.0708345282477775\\
0.420320719533737 0.169486664754804\\
0.420442799324482 0.22454832234012\\
0.420564879115226 0.197017493547462\\
0.420686958905971 0.0628047031832521\\
0.420809038696716 -0.058216231717809\\
0.42093111848746 -0.187840550616576\\
0.421053198278205 -0.233725265271007\\
0.421175278068949 -0.197017493547462\\
0.421297357859694 -0.102953828505879\\
0.421419437650439 0.0266704903928879\\
0.421541517441183 0.17866360768569\\
0.421663597231928 0.242902208201893\\
0.421785677022673 0.187840550616576\\
0.421907756813417 0.112130771436765\\
0.422029836604162 -0.0306854029251506\\
0.422151916394906 -0.151132778893031\\
0.422273996185651 -0.187840550616576\\
0.422396075976396 -0.141955835962145\\
0.42251815576714 -0.0983653570404359\\
0.422640235557885 0.00573558933180384\\
0.42276231534863 0.112130771436765\\
0.422884395139374 0.125896185833094\\
0.423006474930119 0.121307714367651\\
0.423128554720863 0.0628047031832521\\
0.423250634511608 -0.0243762546601663\\
0.423372714302353 -0.0800114711786636\\
0.423494794093097 -0.0708345282477775\\
0.423616873883842 -0.0662460567823344\\
0.423738953674587 -0.0117579581301979\\
0.423861033465331 0.0662460567823344\\
0.423983113256076 0.0536277602523659\\
0.42410519304682 0.0375681101233152\\
0.424227272837565 -0.007456266131345\\
0.42434935262831 -0.0754229997132205\\
0.424471432419054 -0.107542299971322\\
0.424593512209799 -0.0444508173214798\\
0.424715592000543 0.028391167192429\\
0.424837671791288 0.0605104674505305\\
0.424959751582033 0.135073128763981\\
0.425081831372777 0.121307714367651\\
0.425203911163522 0.0605104674505305\\
0.425325990954267 -0.0306854029251506\\
0.425448070745011 -0.116719242902208\\
0.425570150535756 -0.160309721823917\\
0.425692230326501 -0.169486664754804\\
0.425814310117245 -0.0845999426441067\\
0.42593638990799 -0.000573558933180384\\
0.426058469698734 0.141955835962145\\
0.426180549489479 0.215371379409234\\
0.426302629280224 0.197017493547462\\
0.426424709070968 0.135073128763981\\
0.426546788861713 -0.00630914826498423\\
0.426668868652457 -0.135073128763981\\
0.426790948443202 -0.242902208201893\\
0.426913028233947 -0.215371379409234\\
0.427035108024691 -0.135073128763981\\
0.427157187815436 0.00286779466590192\\
0.427279267606181 0.135073128763981\\
0.427401347396925 0.206194436478348\\
0.42752342718767 0.215371379409234\\
0.427645506978414 0.125896185833094\\
0.427767586769159 -0.00286779466590192\\
0.427889666559904 -0.121307714367651\\
0.428011746350648 -0.151132778893031\\
0.428133826141393 -0.160309721823917\\
0.428255905932138 -0.112130771436765\\
0.428377985722882 -0.0106108402638371\\
0.428500065513627 0.0800114711786636\\
0.428622145304371 0.116719242902208\\
0.428744225095116 0.107542299971322\\
0.428866304885861 0.0662460567823344\\
0.428988384676605 0.0151993117292802\\
0.42911046446735 -0.0605104674505305\\
0.429232544258095 -0.0983653570404359\\
0.429354624048839 -0.0513335245196444\\
0.429476703839584 0.000573558933180384\\
0.429598783630328 0.0352738743905936\\
0.429720863421073 0.0421565815887582\\
0.429842943211818 0.0536277602523659\\
0.429965023002562 -0.00286779466590192\\
0.430087102793307 -0.0708345282477775\\
0.430209182584052 -0.0891884141095498\\
0.430331262374796 -0.0536277602523659\\
0.430453342165541 0.00688270719816461\\
0.430575421956285 0.0513335245196444\\
0.43069750174703 0.116719242902208\\
0.430819581537775 0.125896185833094\\
0.430941661328519 0.107542299971322\\
0.431063741119264 -0.0106108402638371\\
0.431185820910009 -0.116719242902208\\
0.431307900700753 -0.135073128763981\\
0.431429980491498 -0.160309721823917\\
0.431552060282242 -0.125896185833094\\
0.431674140072987 -0.0186406653283625\\
0.431796219863732 0.125896185833094\\
0.431918299654476 0.187840550616576\\
0.432040379445221 0.17866360768569\\
0.432162459235966 0.141955835962145\\
0.43228453902671 0.0398623458560367\\
0.432406618817455 -0.0891884141095498\\
0.432528698608199 -0.197017493547462\\
0.432650778398944 -0.206194436478348\\
0.432772858189689 -0.141955835962145\\
0.432894937980433 -0.0513335245196444\\
0.433017017771178 0.0490392887869229\\
0.433139097561922 0.160309721823917\\
0.433261177352667 0.22454832234012\\
0.433383257143412 0.160309721823917\\
0.433505336934156 0.0444508173214798\\
0.433627416724901 -0.0513335245196444\\
0.433749496515646 -0.130484657298537\\
0.43387157630639 -0.187840550616576\\
0.433993656097135 -0.160309721823917\\
0.434115735887879 -0.0352738743905936\\
0.434237815678624 0.0513335245196444\\
0.434359895469369 0.116719242902208\\
0.434481975260113 0.121307714367651\\
0.434604055050858 0.0891884141095498\\
0.434726134841603 0.0151993117292802\\
0.434848214632347 -0.0490392887869229\\
0.434970294423092 -0.0845999426441067\\
0.435092374213836 -0.0708345282477775\\
0.435214454004581 0.00458847146544307\\
0.435336533795326 0.0140521938629194\\
0.43545861358607 0.0513335245196444\\
0.435580693376815 0.0536277602523659\\
0.43570277316756 0.0151993117292802\\
0.435824852958304 -0.0536277602523659\\
0.435946932749049 -0.0800114711786636\\
0.436069012539793 -0.0559219959850875\\
0.436191092330538 -0.0329796386578721\\
0.436313172121283 0.0352738743905936\\
0.436435251912027 0.0845999426441067\\
0.436557331702772 0.116719242902208\\
0.436679411493517 0.102953828505879\\
0.436801491284261 0.0375681101233152\\
0.436923571075006 -0.058216231717809\\
0.43704565086575 -0.112130771436765\\
0.437167730656495 -0.141955835962145\\
0.43728981044724 -0.151132778893031\\
0.437411890237984 -0.0708345282477775\\
0.437533970028729 0.0559219959850875\\
0.437656049819474 0.141955835962145\\
0.437778129610218 0.187840550616576\\
0.437900209400963 0.169486664754804\\
0.438022289191707 0.0891884141095498\\
0.438144368982452 -0.0329796386578721\\
0.438266448773197 -0.151132778893031\\
0.438388528563941 -0.206194436478348\\
0.438510608354686 -0.187840550616576\\
0.438632688145431 -0.0891884141095498\\
0.438754767936175 0.0140521938629194\\
0.43887684772692 0.135073128763981\\
0.438998927517664 0.197017493547462\\
0.439121007308409 0.160309721823917\\
0.439243087099154 0.0937768855749928\\
0.439365166889898 -0.007456266131345\\
0.439487246680643 -0.0937768855749928\\
0.439609326471388 -0.160309721823917\\
0.439731406262132 -0.135073128763981\\
0.439853486052877 -0.0891884141095498\\
0.439975565843621 -0.00946372239747634\\
0.440097645634366 0.0708345282477775\\
0.440219725425111 0.102953828505879\\
0.440341805215855 0.0983653570404359\\
0.4404638850066 0.0536277602523659\\
0.440585964797345 0.0220820189274448\\
0.440708044588089 -0.058216231717809\\
0.440830124378834 -0.0754229997132205\\
0.440952204169578 -0.0444508173214798\\
0.441074283960323 -0.0106108402638371\\
0.441196363751068 0.0243762546601663\\
0.441318443541812 0.0255233725265271\\
0.441440523332557 0.0375681101233152\\
0.441562603123301 -0.00286779466590192\\
0.441684682914046 -0.0352738743905936\\
0.441806762704791 -0.0628047031832521\\
0.441928842495535 -0.0467450530542013\\
0.44205092228628 -0.00860338399770576\\
0.442173002077025 0.0329796386578721\\
0.442295081867769 0.0891884141095498\\
0.442417161658514 0.112130771436765\\
0.442539241449258 0.0937768855749928\\
0.442661321240003 -0.00286779466590192\\
0.442783401030748 -0.0845999426441067\\
0.442905480821492 -0.141955835962145\\
0.443027560612237 -0.141955835962145\\
0.443149640402982 -0.107542299971322\\
0.443271720193726 -0.00946372239747634\\
0.443393799984471 0.107542299971322\\
0.443515879775215 0.169486664754804\\
0.44363795956596 0.17866360768569\\
0.443760039356705 0.112130771436765\\
0.443882119147449 0.0220820189274448\\
0.444004198938194 -0.0937768855749928\\
0.444126278728939 -0.17866360768569\\
0.444248358519683 -0.17866360768569\\
0.444370438310428 -0.125896185833094\\
0.444492518101172 -0.028391167192429\\
0.444614597891917 0.0536277602523659\\
0.444736677682662 0.141955835962145\\
0.444858757473406 0.17866360768569\\
0.444980837264151 0.141955835962145\\
0.445102917054896 0.0398623458560367\\
0.44522499684564 -0.0513335245196444\\
0.445347076636385 -0.112130771436765\\
0.445469156427129 -0.151132778893031\\
0.445591236217874 -0.121307714367651\\
0.445713316008619 -0.0559219959850875\\
0.445835395799363 0.0398623458560367\\
0.445957475590108 0.0845999426441067\\
0.446079555380853 0.102953828505879\\
0.446201635171597 0.0800114711786636\\
0.446323714962342 0.0352738743905936\\
0.446445794753086 -0.0129050759965586\\
0.446567874543831 -0.0708345282477775\\
0.446689954334576 -0.0605104674505305\\
0.44681203412532 -0.028391167192429\\
0.446934113916065 -0.00458847146544307\\
0.44705619370681 0.0151993117292802\\
0.447178273497554 0.0444508173214798\\
0.447300353288299 0.0352738743905936\\
0.447422433079043 -0.0140521938629194\\
0.447544512869788 -0.0490392887869229\\
0.447666592660533 -0.0513335245196444\\
0.447788672451277 -0.0375681101233152\\
0.447910752242022 -0.00229423573272154\\
0.448032832032767 0.058216231717809\\
0.448154911823511 0.0937768855749928\\
0.448276991614256 0.0937768855749928\\
0.448399071405 0.0467450530542013\\
0.448521151195745 -0.028391167192429\\
0.44864323098649 -0.0891884141095498\\
0.448765310777234 -0.125896185833094\\
0.448887390567979 -0.116719242902208\\
0.449009470358724 -0.0662460567823344\\
0.449131550149468 0.0220820189274448\\
0.449253629940213 0.102953828505879\\
0.449375709730957 0.151132778893031\\
0.449497789521702 0.141955835962145\\
0.449619869312447 0.0800114711786636\\
0.449741949103191 0.00802982506452538\\
0.449864028893936 -0.102953828505879\\
0.44998610868468 -0.160309721823917\\
0.450108188475425 -0.160309721823917\\
0.45023026826617 -0.102953828505879\\
0.450352348056914 -0.0197877831947233\\
0.450474427847659 0.0845999426441067\\
0.450596507638404 0.151132778893031\\
0.450718587429148 0.160309721823917\\
0.450840667219893 0.116719242902208\\
0.450962747010637 0.0140521938629194\\
0.451084826801382 -0.0708345282477775\\
0.451206906592127 -0.130484657298537\\
0.451328986382871 -0.130484657298537\\
0.451451066173616 -0.0937768855749928\\
0.451573145964361 -0.028391167192429\\
0.451695225755105 0.0467450530542013\\
0.45181730554585 0.0754229997132205\\
0.451939385336594 0.0937768855749928\\
0.452061465127339 0.0754229997132205\\
0.452183544918084 0.0306854029251506\\
0.452305624708828 -0.0243762546601663\\
0.452427704499573 -0.0605104674505305\\
0.452549784290318 -0.0536277602523659\\
0.452671864081062 -0.0329796386578721\\
0.452793943871807 0.00229423573272154\\
0.452916023662551 0.0117579581301979\\
0.453038103453296 0.0266704903928879\\
0.453160183244041 0.0306854029251506\\
0.453282263034785 -0.016346429595641\\
0.45340434282553 -0.0306854029251506\\
0.453526422616275 -0.0352738743905936\\
0.453648502407019 -0.0106108402638371\\
0.453770582197764 0.00401491253226269\\
0.453892661988508 0.0536277602523659\\
0.454014741779253 0.0708345282477775\\
0.454136821569998 0.0536277602523659\\
0.454258901360742 0.0186406653283625\\
0.454380981151487 -0.0398623458560367\\
0.454503060942232 -0.0708345282477775\\
0.454625140732976 -0.0937768855749928\\
0.454747220523721 -0.0662460567823344\\
0.454869300314465 -0.0559219959850875\\
0.45499138010521 0.0266704903928879\\
0.455113459895955 0.0983653570404359\\
0.455235539686699 0.121307714367651\\
0.455357619477444 0.116719242902208\\
0.455479699268189 0.0662460567823344\\
0.455601779058933 -0.0117579581301979\\
0.455723858849678 -0.112130771436765\\
0.455845938640422 -0.135073128763981\\
0.455968018431167 -0.130484657298537\\
0.456090098221912 -0.0754229997132205\\
0.456212178012656 0.0151993117292802\\
0.456334257803401 0.0754229997132205\\
0.456456337594146 0.135073128763981\\
0.45657841738489 0.125896185833094\\
0.456700497175635 0.0800114711786636\\
0.456822576966379 0.007456266131345\\
0.456944656757124 -0.0513335245196444\\
0.457066736547869 -0.0937768855749928\\
0.457188816338613 -0.130484657298537\\
0.457310896129358 -0.0937768855749928\\
0.457432975920102 -0.028391167192429\\
0.457555055710847 0.0352738743905936\\
0.457677135501592 0.0708345282477775\\
0.457799215292336 0.107542299971322\\
0.457921295083081 0.0891884141095498\\
0.458043374873826 0.028391167192429\\
0.45816545466457 -0.0174935474620017\\
0.458287534455315 -0.0605104674505305\\
0.458409614246059 -0.0628047031832521\\
0.458531694036804 -0.0490392887869229\\
0.458653773827549 -0.0232291367938056\\
0.458775853618293 0.00344135359908231\\
0.458897933409038 0.0329796386578721\\
0.459020013199783 0.0421565815887582\\
0.459142092990527 -0.000573558933180384\\
0.459264172781272 -0.00172067679954115\\
0.459386252572016 -0.00802982506452538\\
0.459508332362761 -0.0106108402638371\\
0.459630412153506 0\\
0.45975249194425 0.0186406653283625\\
0.459874571734995 0.0398623458560367\\
0.45999665152574 0.0306854029251506\\
0.460118731316484 0.0174935474620017\\
0.460240811107229 -0.0306854029251506\\
0.460362890897973 -0.0536277602523659\\
0.460484970688718 -0.0628047031832521\\
0.460607050479463 -0.0662460567823344\\
0.460729130270207 -0.0140521938629194\\
0.460851210060952 0.0329796386578721\\
0.460973289851697 0.0800114711786636\\
0.461095369642441 0.0845999426441067\\
0.461217449433186 0.0800114711786636\\
0.46133952922393 0.0444508173214798\\
0.461461609014675 -0.0174935474620017\\
0.46158368880542 -0.0754229997132205\\
0.461705768596164 -0.116719242902208\\
0.461827848386909 -0.0937768855749928\\
0.461949928177654 -0.0628047031832521\\
0.462072007968398 0.00401491253226269\\
0.462194087759143 0.0662460567823344\\
0.462316167549887 0.121307714367651\\
0.462438247340632 0.121307714367651\\
0.462560327131377 0.0708345282477775\\
0.462682406922121 0.0129050759965586\\
0.462804486712866 -0.0536277602523659\\
0.462926566503611 -0.0891884141095498\\
0.463048646294355 -0.116719242902208\\
0.4631707260851 -0.0891884141095498\\
0.463292805875844 -0.0329796386578721\\
0.463414885666589 0.028391167192429\\
0.463536965457334 0.0754229997132205\\
0.463659045248078 0.0845999426441067\\
0.463781125038823 0.0937768855749928\\
0.463903204829567 0.0398623458560367\\
0.464025284620312 -0.0140521938629194\\
0.464147364411057 -0.0536277602523659\\
0.464269444201801 -0.0605104674505305\\
0.464391523992546 -0.058216231717809\\
0.464513603783291 -0.0444508173214798\\
0.464635683574035 -0.00114711786636077\\
0.46475776336478 0.020934901061084\\
0.464879843155525 0.0398623458560367\\
0.465001922946269 0.0232291367938056\\
0.465124002737014 0.0220820189274448\\
0.465246082527758 0.0151993117292802\\
0.465368162318503 0.00114711786636077\\
0.465490242109248 -0.00573558933180384\\
0.465612321899992 -0.00458847146544307\\
0.465734401690737 0.016346429595641\\
0.465856481481481 -0.00344135359908231\\
0.465978561272226 -0.0106108402638371\\
0.466100641062971 -0.0186406653283625\\
0.466222720853715 -0.0174935474620017\\
0.46634480064446 -0.0352738743905936\\
0.466466880435205 -0.0306854029251506\\
0.466588960225949 0.00286779466590192\\
0.466711040016694 0.020934901061084\\
0.466833119807438 0.0513335245196444\\
0.466955199598183 0.0513335245196444\\
0.467077279388928 0.0605104674505305\\
0.467199359179672 0.0306854029251506\\
0.467321438970417 -0.0220820189274448\\
0.467443518761162 -0.0559219959850875\\
0.467565598551906 -0.0754229997132205\\
0.467687678342651 -0.0662460567823344\\
0.467809758133395 -0.058216231717809\\
0.46793183792414 -0.00946372239747634\\
0.468053917714885 0.0513335245196444\\
0.468175997505629 0.0937768855749928\\
0.468298077296374 0.0937768855749928\\
0.468420157087119 0.0754229997132205\\
0.468542236877863 0.0352738743905936\\
0.468664316668608 -0.0352738743905936\\
0.468786396459352 -0.0800114711786636\\
0.468908476250097 -0.112130771436765\\
0.469030556040842 -0.0845999426441067\\
0.469152635831586 -0.0329796386578721\\
0.469274715622331 0.0117579581301979\\
0.469396795413076 0.058216231717809\\
0.46951887520382 0.0937768855749928\\
0.469640954994565 0.0937768855749928\\
0.469763034785309 0.0421565815887582\\
0.469885114576054 0.0106108402638371\\
0.470007194366799 -0.0306854029251506\\
0.470129274157543 -0.0605104674505305\\
0.470251353948288 -0.0800114711786636\\
0.470373433739033 -0.0662460567823344\\
0.470495513529777 -0.0243762546601663\\
0.470617593320522 0.0151993117292802\\
0.470739673111266 0.0421565815887582\\
0.470861752902011 0.0467450530542013\\
0.470983832692756 0.0662460567823344\\
0.4711059124835 0.0232291367938056\\
0.471227992274245 -0.007456266131345\\
0.47135007206499 -0.0220820189274448\\
0.471472151855734 -0.0174935474620017\\
0.471594231646479 -0.0129050759965586\\
0.471716311437223 -0.0186406653283625\\
0.471838391227968 -0.007456266131345\\
0.471960471018713 -0.0117579581301979\\
0.472082550809457 0.00344135359908231\\
0.472204630600202 -0.0186406653283625\\
0.472326710390946 -0.00344135359908231\\
0.472448790181691 0.0117579581301979\\
0.472570869972436 0.0255233725265271\\
0.47269294976318 0.028391167192429\\
0.472815029553925 0.0186406653283625\\
0.47293710934467 0.0375681101233152\\
0.473059189135414 0.00946372239747634\\
0.473181268926159 -0.00946372239747634\\
0.473303348716903 -0.0421565815887582\\
0.473425428507648 -0.0444508173214798\\
0.473547508298393 -0.0490392887869229\\
0.473669588089137 -0.0467450530542013\\
0.473791667879882 -0.00401491253226269\\
0.473913747670627 0.028391167192429\\
0.474035827461371 0.0800114711786636\\
0.474157907252116 0.0662460567823344\\
0.47427998704286 0.0490392887869229\\
0.474402066833605 0.028391167192429\\
0.47452414662435 -0.0129050759965586\\
0.474646226415094 -0.0559219959850875\\
0.474768306205839 -0.0845999426441067\\
0.474890385996584 -0.0662460567823344\\
0.475012465787328 -0.0421565815887582\\
0.475134545578073 -0.00229423573272154\\
0.475256625368817 0.0329796386578721\\
0.475378705159562 0.0708345282477775\\
0.475500784950307 0.0845999426441067\\
0.475622864741051 0.058216231717809\\
0.475744944531796 0.0151993117292802\\
0.475867024322541 -0.007456266131345\\
0.475989104113285 -0.0398623458560367\\
0.47611118390403 -0.0800114711786636\\
0.476233263694774 -0.0754229997132205\\
0.476355343485519 -0.0421565815887582\\
0.476477423276264 0.00401491253226269\\
0.476599503067008 0.0220820189274448\\
0.476721582857753 0.058216231717809\\
0.476843662648498 0.0628047031832521\\
0.476965742439242 0.058216231717809\\
0.477087822229987 0.0186406653283625\\
0.477209902020731 -0.0232291367938056\\
0.477331981811476 -0.0266704903928879\\
0.477454061602221 -0.0467450530542013\\
0.477576141392965 -0.0352738743905936\\
0.47769822118371 -0.0306854029251506\\
0.477820300974455 -0.00172067679954115\\
0.477942380765199 0.0117579581301979\\
0.478064460555944 0.0197877831947233\\
0.478186540346688 0.0197877831947233\\
0.478308620137433 0.00688270719816461\\
0.478430699928178 0.0306854029251506\\
0.478552779718922 0.00946372239747634\\
0.478674859509667 0.00401491253226269\\
0.478796939300412 0.00630914826498423\\
0.478919019091156 -0.00114711786636077\\
0.479041098881901 -0.0151993117292802\\
0.479163178672645 -0.0266704903928879\\
0.47928525846339 -0.0174935474620017\\
0.479407338254135 -0.0266704903928879\\
0.479529418044879 -0.0140521938629194\\
0.479651497835624 -0.007456266131345\\
0.479773577626369 0.020934901061084\\
0.479895657417113 0.0444508173214798\\
0.480017737207858 0.0398623458560367\\
0.480139816998602 0.0306854029251506\\
0.480261896789347 0.0186406653283625\\
0.480383976580092 -0.00229423573272154\\
0.480506056370836 -0.0375681101233152\\
0.480628136161581 -0.0490392887869229\\
0.480750215952325 -0.0467450530542013\\
0.48087229574307 -0.0306854029251506\\
0.480994375533815 -0.0197877831947233\\
0.481116455324559 0.00946372239747634\\
0.481238535115304 0.0513335245196444\\
0.481360614906049 0.0559219959850875\\
0.481482694696793 0.0605104674505305\\
0.481604774487538 0.0398623458560367\\
0.481726854278282 0.0106108402638371\\
0.481848934069027 -0.0375681101233152\\
0.481971013859772 -0.0662460567823344\\
0.482093093650516 -0.0662460567823344\\
0.482215173441261 -0.0490392887869229\\
0.482337253232006 -0.0140521938629194\\
0.48245933302275 0.00946372239747634\\
0.482581412813495 0.0444508173214798\\
0.482703492604239 0.0559219959850875\\
0.482825572394984 0.0559219959850875\\
0.482947652185729 0.0306854029251506\\
0.483069731976473 0.0117579581301979\\
0.483191811767218 -0.0174935474620017\\
0.483313891557963 -0.0421565815887582\\
0.483435971348707 -0.0536277602523659\\
0.483558051139452 -0.0536277602523659\\
0.483680130930196 -0.0255233725265271\\
0.483802210720941 0.00344135359908231\\
0.483924290511686 0.0306854029251506\\
0.48404637030243 0.0352738743905936\\
0.484168450093175 0.0467450530542013\\
0.48429052988392 0.028391167192429\\
0.484412609674664 0.00860338399770576\\
0.484534689465409 -0.00344135359908231\\
0.484656769256153 -0.0220820189274448\\
0.484778849046898 -0.0220820189274448\\
0.484900928837643 -0.0306854029251506\\
0.485023008628387 -0.0220820189274448\\
0.485145088419132 -0.0140521938629194\\
0.485267168209877 0.00229423573272154\\
0.485389248000621 0.007456266131345\\
0.485511327791366 0.0117579581301979\\
0.48563340758211 0.0266704903928879\\
0.485755487372855 0.0243762546601663\\
0.4858775671636 0.0174935474620017\\
0.485999646954344 0.0117579581301979\\
0.486121726745089 0.00946372239747634\\
0.486243806535834 -0.0174935474620017\\
0.486365886326578 -0.0255233725265271\\
0.486487966117323 -0.028391167192429\\
0.486610045908067 -0.0266704903928879\\
0.486732125698812 -0.0174935474620017\\
0.486854205489557 -0.00573558933180384\\
0.486976285280301 0.0129050759965586\\
0.487098365071046 0.028391167192429\\
0.487220444861791 0.0352738743905936\\
0.487342524652535 0.0398623458560367\\
0.48746460444328 0.0255233725265271\\
0.487586684234024 0.00860338399770576\\
0.487708764024769 -0.020934901061084\\
0.487830843815514 -0.0352738743905936\\
0.487952923606258 -0.0375681101233152\\
0.488075003397003 -0.0421565815887582\\
0.488197083187747 -0.0255233725265271\\
0.488319162978492 -0.00229423573272154\\
};
\addplot [
color=blue,
solid,
forget plot
]
table[row sep=crcr]{
0.488319162978492 -0.00229423573272154\\
0.488441242769237 0.028391167192429\\
0.488563322559981 0.0421565815887582\\
0.488685402350726 0.0559219959850875\\
0.488807482141471 0.0421565815887582\\
0.488929561932215 0.0129050759965586\\
0.48905164172296 -0.00344135359908231\\
0.489173721513704 -0.0352738743905936\\
0.489295801304449 -0.0559219959850875\\
0.489417881095194 -0.0444508173214798\\
0.489539960885938 -0.028391167192429\\
0.489662040676683 -0.0151993117292802\\
0.489784120467428 0.0106108402638371\\
0.489906200258172 0.0398623458560367\\
0.490028280048917 0.0513335245196444\\
0.490150359839661 0.0398623458560367\\
0.490272439630406 0.0266704903928879\\
0.490394519421151 0.00172067679954115\\
0.490516599211895 -0.0197877831947233\\
0.49063867900264 -0.0398623458560367\\
0.490760758793385 -0.0444508173214798\\
0.490882838584129 -0.0352738743905936\\
0.491004918374874 -0.0129050759965586\\
0.491126998165618 -0.00114711786636077\\
0.491249077956363 0.0174935474620017\\
0.491371157747108 0.0329796386578721\\
0.491493237537852 0.0255233725265271\\
0.491615317328597 0.028391167192429\\
0.491737397119342 0.0151993117292802\\
0.491859476910086 0.0106108402638371\\
0.491981556700831 -0.00946372239747634\\
0.492103636491575 -0.0197877831947233\\
0.49222571628232 -0.0306854029251506\\
0.492347796073065 -0.0352738743905936\\
0.492469875863809 -0.0197877831947233\\
0.492591955654554 -0.0117579581301979\\
0.492714035445299 0.0106108402638371\\
0.492836115236043 0.0306854029251506\\
0.492958195026788 0.0306854029251506\\
0.493080274817532 0.0197877831947233\\
0.493202354608277 0.0220820189274448\\
0.493324434399022 0.0117579581301979\\
0.493446514189766 -0.00630914826498423\\
0.493568593980511 -0.0129050759965586\\
0.493690673771256 -0.020934901061084\\
0.493812753562 -0.0329796386578721\\
0.493934833352745 -0.0329796386578721\\
0.494056913143489 -0.0117579581301979\\
0.494178992934234 -0.00401491253226269\\
0.494301072724979 0.0186406653283625\\
0.494423152515723 0.0352738743905936\\
0.494545232306468 0.0352738743905936\\
0.494667312097212 0.0352738743905936\\
0.494789391887957 0.0140521938629194\\
0.494911471678702 -0.00946372239747634\\
0.495033551469446 -0.0243762546601663\\
0.495155631260191 -0.0197877831947233\\
0.495277711050936 -0.0375681101233152\\
0.49539979084168 -0.0306854029251506\\
0.495521870632425 -0.00946372239747634\\
0.49564395042317 0\\
0.495766030213914 0.00946372239747634\\
0.495888110004659 0.0306854029251506\\
0.496010189795403 0.0444508173214798\\
0.496132269586148 0.028391167192429\\
0.496254349376893 0.0186406653283625\\
0.496376429167637 -0.00946372239747634\\
0.496498508958382 -0.0197877831947233\\
0.496620588749126 -0.0352738743905936\\
0.496742668539871 -0.0444508173214798\\
0.496864748330616 -0.0266704903928879\\
0.49698682812136 -0.00573558933180384\\
0.497108907912105 0.007456266131345\\
0.49723098770285 0.0140521938629194\\
0.497353067493594 0.0375681101233152\\
0.497475147284339 0.0398623458560367\\
0.497597227075083 0.028391167192429\\
0.497719306865828 0.0117579581301979\\
0.497841386656573 0.00573558933180384\\
0.497963466447317 -0.020934901061084\\
0.498085546238062 -0.0421565815887582\\
0.498207626028807 -0.0421565815887582\\
0.498329705819551 -0.028391167192429\\
0.498451785610296 -0.00516203039862346\\
0.49857386540104 0.00401491253226269\\
0.498695945191785 0.0255233725265271\\
0.49881802498253 0.0421565815887582\\
0.498940104773274 0.0398623458560367\\
0.499062184564019 0.0174935474620017\\
0.499184264354764 -0.00229423573272154\\
0.499306344145508 -0.007456266131345\\
0.499428423936253 -0.0255233725265271\\
0.499550503726997 -0.0352738743905936\\
0.499672583517742 -0.028391167192429\\
0.499794663308487 -0.016346429595641\\
0.499916743099231 -0.00573558933180384\\
0.500038822889976 0.0117579581301979\\
0.500160902680721 0.0232291367938056\\
0.500282982471465 0.0255233725265271\\
0.50040506226221 0.028391167192429\\
0.500527142052954 0.00516203039862346\\
0.500649221843699 0.00172067679954115\\
0.500771301634444 0.00458847146544307\\
0.500893381425188 -0.00802982506452538\\
0.501015461215933 -0.0232291367938056\\
0.501137541006678 -0.028391167192429\\
0.501259620797422 -0.0117579581301979\\
0.501381700588167 -0.0174935474620017\\
0.501503780378911 -0.00688270719816461\\
0.501625860169656 0.0140521938629194\\
0.501747939960401 0.0243762546601663\\
0.501870019751145 0.0306854029251506\\
0.50199209954189 0.0255233725265271\\
0.502114179332634 0.0117579581301979\\
0.502236259123379 -0.00573558933180384\\
0.502358338914124 -0.00688270719816461\\
0.502480418704868 -0.028391167192429\\
0.502602498495613 -0.0266704903928879\\
0.502724578286358 -0.0140521938629194\\
0.502846658077102 -0.0186406653283625\\
0.502968737867847 -0.00573558933180384\\
0.503090817658592 0.00688270719816461\\
0.503212897449336 0.0186406653283625\\
0.503334977240081 0.0243762546601663\\
0.503457057030825 0.028391167192429\\
0.50357913682157 0.0255233725265271\\
0.503701216612315 0.00946372239747634\\
0.503823296403059 -0.00630914826498423\\
0.503945376193804 -0.028391167192429\\
0.504067455984549 -0.0352738743905936\\
0.504189535775293 -0.0329796386578721\\
0.504311615566038 -0.020934901061084\\
0.504433695356782 0.00401491253226269\\
0.504555775147527 0.0197877831947233\\
0.504677854938272 0.0220820189274448\\
0.504799934729016 0.0220820189274448\\
0.504922014519761 0.0329796386578721\\
0.505044094310505 0.0220820189274448\\
0.50516617410125 -0.000573558933180384\\
0.505288253891995 -0.0117579581301979\\
0.505410333682739 -0.0197877831947233\\
0.505532413473484 -0.0329796386578721\\
0.505654493264229 -0.0421565815887582\\
0.505776573054973 -0.0129050759965586\\
0.505898652845718 0.00458847146544307\\
0.506020732636462 0.00946372239747634\\
0.506142812427207 0.0243762546601663\\
0.506264892217952 0.0306854029251506\\
0.506386972008696 0.0352738743905936\\
0.506509051799441 0.007456266131345\\
0.506631131590186 -0.00229423573272154\\
0.50675321138093 -0.00630914826498423\\
0.506875291171675 -0.0197877831947233\\
0.506997370962419 -0.028391167192429\\
0.507119450753164 -0.0352738743905936\\
0.507241530543909 -0.0106108402638371\\
0.507363610334653 0.00286779466590192\\
0.507485690125398 0.0129050759965586\\
0.507607769916143 0.0197877831947233\\
0.507729849706887 0.0306854029251506\\
0.507851929497632 0.0243762546601663\\
0.507974009288376 0.0106108402638371\\
0.508096089079121 0.00630914826498423\\
0.508218168869866 -0.00802982506452538\\
0.50834024866061 -0.0117579581301979\\
0.508462328451355 -0.0375681101233152\\
0.508584408242099 -0.0352738743905936\\
0.508706488032844 -0.0117579581301979\\
0.508828567823589 -0.00172067679954115\\
0.508950647614333 0.0117579581301979\\
0.509072727405078 0.0186406653283625\\
0.509194807195823 0.0375681101233152\\
0.509316886986567 0.0151993117292802\\
0.509438966777312 0.00946372239747634\\
0.509561046568057 0.00344135359908231\\
0.509683126358801 -0.00229423573272154\\
0.509805206149546 -0.00946372239747634\\
0.50992728594029 -0.0329796386578721\\
0.510049365731035 -0.0174935474620017\\
0.51017144552178 -0.028391167192429\\
0.510293525312524 -0.00229423573272154\\
0.510415605103269 0.00114711786636077\\
0.510537684894014 0.028391167192429\\
0.510659764684758 0.0375681101233152\\
0.510781844475503 0.0129050759965586\\
0.510903924266247 0.0140521938629194\\
0.511026004056992 -0.00516203039862346\\
0.511148083847737 0.00946372239747634\\
0.511270163638481 -0.0255233725265271\\
0.511392243429226 -0.0129050759965586\\
0.51151432321997 -0.016346429595641\\
0.511636403010715 -0.0197877831947233\\
0.51175848280146 -0.00516203039862346\\
0.511880562592204 -0.0106108402638371\\
0.512002642382949 0.0306854029251506\\
0.512124722173694 0.0129050759965586\\
0.512246801964438 0.0220820189274448\\
0.512368881755183 0.0151993117292802\\
0.512490961545927 0.0117579581301979\\
0.512613041336672 0.00688270719816461\\
0.512735121127417 -0.0266704903928879\\
0.512857200918161 -0.0106108402638371\\
0.512979280708906 -0.0306854029251506\\
0.513101360499651 -0.0117579581301979\\
0.513223440290395 -0.0255233725265271\\
0.51334552008114 -0.00229423573272154\\
0.513467599871885 0.0186406653283625\\
0.513589679662629 0.0140521938629194\\
0.513711759453374 0.0352738743905936\\
0.513833839244118 0.0174935474620017\\
0.513955919034863 0.0329796386578721\\
0.514077998825608 -0.00573558933180384\\
0.514200078616352 -0.0174935474620017\\
0.514322158407097 -0.0255233725265271\\
0.514444238197841 -0.0352738743905936\\
0.514566317988586 -0.0186406653283625\\
0.514688397779331 -0.020934901061084\\
0.514810477570075 0.0174935474620017\\
0.51493255736082 0.016346429595641\\
0.515054637151565 0.0197877831947233\\
0.515176716942309 0.0151993117292802\\
0.515298796733054 0.0243762546601663\\
0.515420876523798 0.0197877831947233\\
0.515542956314543 -0.0117579581301979\\
0.515665036105288 0\\
0.515787115896032 -0.0220820189274448\\
0.515909195686777 -0.028391167192429\\
0.516031275477522 -0.0306854029251506\\
0.516153355268266 -0.0106108402638371\\
0.516275435059011 0.00516203039862346\\
0.516397514849755 0.00946372239747634\\
0.5165195946405 0.0197877831947233\\
0.516641674431245 0.0243762546601663\\
0.516763754221989 0.0352738743905936\\
0.516885834012734 0.000573558933180384\\
0.517007913803479 -0.00401491253226269\\
0.517129993594223 -0.00516203039862346\\
0.517252073384968 -0.0117579581301979\\
0.517374153175712 -0.0255233725265271\\
0.517496232966457 -0.0266704903928879\\
0.517618312757202 -0.00688270719816461\\
0.517740392547946 -0.00688270719816461\\
0.517862472338691 0.00344135359908231\\
0.517984552129436 0.0151993117292802\\
0.51810663192018 0.0329796386578721\\
0.518228711710925 0.028391167192429\\
0.518350791501669 0.0106108402638371\\
0.518472871292414 0.00573558933180384\\
0.518594951083159 0.00229423573272154\\
0.518717030873903 -0.0220820189274448\\
0.518839110664648 -0.0329796386578721\\
0.518961190455392 -0.0255233725265271\\
0.519083270246137 -0.0140521938629194\\
0.519205350036882 -0.00229423573272154\\
0.519327429827626 0.00458847146544307\\
0.519449509618371 0.0186406653283625\\
0.519571589409116 0.0329796386578721\\
0.51969366919986 0.028391167192429\\
0.519815748990605 0.00630914826498423\\
0.519937828781349 0.00516203039862346\\
0.520059908572094 0.000573558933180384\\
0.520181988362839 -0.0197877831947233\\
0.520304068153583 -0.028391167192429\\
0.520426147944328 -0.0220820189274448\\
0.520548227735073 -0.0151993117292802\\
0.520670307525817 -0.0129050759965586\\
0.520792387316562 0.00458847146544307\\
0.520914467107306 0.0197877831947233\\
0.521036546898051 0.028391167192429\\
0.521158626688796 0.0306854029251506\\
0.52128070647954 0.0117579581301979\\
0.521402786270285 0.0174935474620017\\
0.52152486606103 0\\
0.521646945851774 -0.0186406653283625\\
0.521769025642519 -0.0329796386578721\\
0.521891105433263 -0.0243762546601663\\
0.522013185224008 -0.0232291367938056\\
0.522135265014753 -0.0243762546601663\\
0.522257344805497 0.00946372239747634\\
0.522379424596242 0.0197877831947233\\
0.522501504386987 0.0375681101233152\\
0.522623584177731 0.0243762546601663\\
0.522745663968476 0.020934901061084\\
0.52286774375922 0.0140521938629194\\
0.522989823549965 -0.00286779466590192\\
0.52311190334071 -0.016346429595641\\
0.523233983131454 -0.028391167192429\\
0.523356062922199 -0.0220820189274448\\
0.523478142712944 -0.0329796386578721\\
0.523600222503688 -0.016346429595641\\
0.523722302294433 0.00688270719816461\\
0.523844382085177 0.0243762546601663\\
0.523966461875922 0.0329796386578721\\
0.524088541666667 0.028391167192429\\
0.524210621457411 0.0306854029251506\\
0.524332701248156 0.00286779466590192\\
0.524454781038901 -0.0117579581301979\\
0.524576860829645 -0.0186406653283625\\
0.52469894062039 -0.0220820189274448\\
0.524821020411134 -0.0306854029251506\\
0.524943100201879 -0.0255233725265271\\
0.525065179992624 -0.00344135359908231\\
0.525187259783368 -0.000573558933180384\\
0.525309339574113 0.0186406653283625\\
0.525431419364857 0.0243762546601663\\
0.525553499155602 0.028391167192429\\
0.525675578946347 0.0329796386578721\\
0.525797658737091 0.00946372239747634\\
0.525919738527836 -0.00172067679954115\\
0.526041818318581 -0.0151993117292802\\
0.526163898109325 -0.028391167192429\\
0.52628597790007 -0.0375681101233152\\
0.526408057690814 -0.0186406653283625\\
0.526530137481559 0.00172067679954115\\
0.526652217272304 -0.00229423573272154\\
0.526774297063048 0.0117579581301979\\
0.526896376853793 0.0220820189274448\\
0.527018456644538 0.0306854029251506\\
0.527140536435282 0.0220820189274448\\
0.527262616226027 0.00860338399770576\\
0.527384696016772 0.00458847146544307\\
0.527506775807516 -0.0129050759965586\\
0.527628855598261 -0.020934901061084\\
0.527750935389005 -0.0329796386578721\\
0.52787301517975 -0.0220820189274448\\
0.527995094970495 -0.00630914826498423\\
0.528117174761239 -0.00573558933180384\\
0.528239254551984 0.00946372239747634\\
0.528361334342728 0.0220820189274448\\
0.528483414133473 0.0306854029251506\\
0.528605493924218 0.028391167192429\\
0.528727573714962 0.0117579581301979\\
0.528849653505707 0.00946372239747634\\
0.528971733296452 -0.0106108402638371\\
0.529093813087196 -0.028391167192429\\
0.529215892877941 -0.0306854029251506\\
0.529337972668685 -0.0232291367938056\\
0.52946005245943 -0.0151993117292802\\
0.529582132250175 -0.00516203039862346\\
0.529704212040919 0.0151993117292802\\
0.529826291831664 0.0220820189274448\\
0.529948371622409 0.028391167192429\\
0.530070451413153 0.0255233725265271\\
0.530192531203898 0.0197877831947233\\
0.530314610994642 0.00946372239747634\\
0.530436690785387 -0.0106108402638371\\
0.530558770576132 -0.0306854029251506\\
0.530680850366876 -0.028391167192429\\
0.530802930157621 -0.020934901061084\\
0.530925009948366 -0.016346429595641\\
0.53104708973911 -0.00229423573272154\\
0.531169169529855 0.0186406653283625\\
0.531291249320599 0.0220820189274448\\
0.531413329111344 0.016346429595641\\
0.531535408902089 0.0243762546601663\\
0.531657488692833 0.020934901061084\\
0.531779568483578 0.00229423573272154\\
0.531901648274323 -0.0174935474620017\\
0.532023728065067 -0.0151993117292802\\
0.532145807855812 -0.0197877831947233\\
0.532267887646556 -0.028391167192429\\
0.532389967437301 -0.0186406653283625\\
0.532512047228046 -0.00802982506452538\\
0.53263412701879 0.0106108402638371\\
0.532756206809535 0.016346429595641\\
0.532878286600279 0.0243762546601663\\
0.533000366391024 0.0306854029251506\\
0.533122446181769 0.028391167192429\\
0.533244525972513 0.00688270719816461\\
0.533366605763258 -0.0232291367938056\\
0.533488685554003 -0.0140521938629194\\
0.533610765344747 -0.0197877831947233\\
0.533732845135492 -0.028391167192429\\
0.533854924926237 -0.020934901061084\\
0.533977004716981 -0.00688270719816461\\
0.534099084507726 0.0106108402638371\\
0.53422116429847 0.0129050759965586\\
0.534343244089215 0.0306854029251506\\
0.53446532387996 0.0232291367938056\\
0.534587403670704 0.028391167192429\\
0.534709483461449 0.0129050759965586\\
0.534831563252194 -0.016346429595641\\
0.534953643042938 -0.00573558933180384\\
0.535075722833683 -0.028391167192429\\
0.535197802624427 -0.0398623458560367\\
0.535319882415172 -0.0329796386578721\\
0.535441962205917 0.00229423573272154\\
0.535564041996661 0.00401491253226269\\
0.535686121787406 0.0151993117292802\\
0.53580820157815 0.0490392887869229\\
0.535930281368895 0.028391167192429\\
0.53605236115964 0.0266704903928879\\
0.536174440950384 0.00229423573272154\\
0.536296520741129 -0.00860338399770576\\
0.536418600531874 -0.0174935474620017\\
0.536540680322618 -0.0329796386578721\\
0.536662760113363 -0.0220820189274448\\
0.536784839904107 -0.0352738743905936\\
0.536906919694852 -0.00172067679954115\\
0.537028999485597 0.00458847146544307\\
0.537151079276341 0.0140521938629194\\
0.537273159067086 0.0421565815887582\\
0.537395238857831 0.0329796386578721\\
0.537517318648575 0.0329796386578721\\
0.53763939843932 -0.00401491253226269\\
0.537761478230064 -0.00401491253226269\\
0.537883558020809 -0.0329796386578721\\
0.538005637811554 -0.0306854029251506\\
0.538127717602298 -0.0243762546601663\\
0.538249797393043 -0.0243762546601663\\
0.538371877183788 0.00516203039862346\\
0.538493956974532 -0.00573558933180384\\
0.538616036765277 0.0174935474620017\\
0.538738116556021 0.0243762546601663\\
0.538860196346766 0.0352738743905936\\
0.538982276137511 0.0243762546601663\\
0.539104355928255 0.0129050759965586\\
0.539226435719 0.0117579581301979\\
0.539348515509745 -0.0232291367938056\\
0.539470595300489 -0.028391167192429\\
0.539592675091234 -0.0398623458560367\\
0.539714754881978 -0.0306854029251506\\
0.539836834672723 -0.0140521938629194\\
0.539958914463468 0.00688270719816461\\
0.540080994254212 0.028391167192429\\
0.540203074044957 0.028391167192429\\
0.540325153835702 0.0421565815887582\\
0.540447233626446 0.020934901061084\\
0.540569313417191 0.0117579581301979\\
0.540691393207935 0.00458847146544307\\
0.54081347299868 -0.0220820189274448\\
0.540935552789425 -0.0375681101233152\\
0.541057632580169 -0.0421565815887582\\
0.541179712370914 -0.028391167192429\\
0.541301792161659 -0.0140521938629194\\
0.541423871952403 0.007456266131345\\
0.541545951743148 0.0151993117292802\\
0.541668031533892 0.0352738743905936\\
0.541790111324637 0.0421565815887582\\
0.541912191115382 0.0329796386578721\\
0.542034270906126 0.0174935474620017\\
0.542156350696871 0.00401491253226269\\
0.542278430487615 -0.0197877831947233\\
0.54240051027836 -0.0513335245196444\\
0.542522590069105 -0.0352738743905936\\
0.542644669859849 -0.0329796386578721\\
0.542766749650594 -0.0151993117292802\\
0.542888829441339 0.00946372239747634\\
0.543010909232083 0.0220820189274448\\
0.543132989022828 0.0352738743905936\\
0.543255068813572 0.0375681101233152\\
0.543377148604317 0.0329796386578721\\
0.543499228395062 0\\
0.543621308185806 0.00802982506452538\\
0.543743387976551 -0.020934901061084\\
0.543865467767296 -0.0444508173214798\\
0.54398754755804 -0.0306854029251506\\
0.544109627348785 -0.0266704903928879\\
0.54423170713953 -0.0174935474620017\\
0.544353786930274 0\\
0.544475866721019 0.0329796386578721\\
0.544597946511763 0.0220820189274448\\
0.544720026302508 0.0352738743905936\\
0.544842106093253 0.0306854029251506\\
0.544964185883997 0.0106108402638371\\
0.545086265674742 -0.00114711786636077\\
0.545208345465486 -0.0186406653283625\\
0.545330425256231 -0.028391167192429\\
0.545452505046976 -0.0329796386578721\\
0.54557458483772 -0.0151993117292802\\
0.545696664628465 -0.0243762546601663\\
0.54581874441921 -0.0106108402638371\\
0.545940824209954 0.0140521938629194\\
0.546062904000699 0.020934901061084\\
0.546184983791443 0.0398623458560367\\
0.546307063582188 0.0421565815887582\\
0.546429143372933 0.0329796386578721\\
0.546551223163677 -0.00286779466590192\\
0.546673302954422 -0.016346429595641\\
0.546795382745167 -0.0329796386578721\\
0.546917462535911 -0.0421565815887582\\
0.547039542326656 -0.0306854029251506\\
0.5471616221174 -0.0243762546601663\\
0.547283701908145 0.000573558933180384\\
0.54740578169889 0.0151993117292802\\
0.547527861489634 0.028391167192429\\
0.547649941280379 0.0306854029251506\\
0.547772021071124 0.0398623458560367\\
0.547894100861868 0.0352738743905936\\
0.548016180652613 0.000573558933180384\\
0.548138260443357 -0.0129050759965586\\
0.548260340234102 -0.0306854029251506\\
0.548382420024847 -0.0490392887869229\\
0.548504499815591 -0.0398623458560367\\
0.548626579606336 -0.0266704903928879\\
0.548748659397081 0.00344135359908231\\
0.548870739187825 0.0129050759965586\\
0.54899281897857 0.0352738743905936\\
0.549114898769314 0.0467450530542013\\
0.549236978560059 0.0398623458560367\\
0.549359058350804 0.028391167192429\\
0.549481138141548 -0.00458847146544307\\
0.549603217932293 -0.0151993117292802\\
0.549725297723037 -0.0375681101233152\\
0.549847377513782 -0.0421565815887582\\
0.549969457304527 -0.0398623458560367\\
0.550091537095271 -0.016346429595641\\
0.550213616886016 -0.00114711786636077\\
0.550335696676761 0.0117579581301979\\
0.550457776467505 0.0444508173214798\\
0.55057985625825 0.0444508173214798\\
0.550701936048994 0.0421565815887582\\
0.550824015839739 0.0106108402638371\\
0.550946095630484 -0.00516203039862346\\
0.551068175421228 -0.020934901061084\\
0.551190255211973 -0.0329796386578721\\
0.551312335002718 -0.0375681101233152\\
0.551434414793462 -0.0375681101233152\\
0.551556494584207 -0.000573558933180384\\
0.551678574374951 -0.00344135359908231\\
0.551800654165696 0.020934901061084\\
0.551922733956441 0.028391167192429\\
0.552044813747185 0.0329796386578721\\
0.55216689353793 0.0306854029251506\\
0.552288973328675 0.00802982506452538\\
0.552411053119419 0.000573558933180384\\
0.552533132910164 -0.0140521938629194\\
0.552655212700908 -0.007456266131345\\
0.552777292491653 -0.0329796386578721\\
0.552899372282398 -0.0266704903928879\\
0.553021452073142 -0.020934901061084\\
0.553143531863887 -0.016346429595641\\
0.553265611654632 0.000573558933180384\\
0.553387691445376 0.0151993117292802\\
0.553509771236121 0.0490392887869229\\
0.553631851026865 0.0352738743905936\\
0.55375393081761 0.0352738743905936\\
0.553876010608355 0.0106108402638371\\
0.553998090399099 -0.00458847146544307\\
0.554120170189844 -0.020934901061084\\
0.554242249980589 -0.0444508173214798\\
0.554364329771333 -0.0421565815887582\\
0.554486409562078 -0.0444508173214798\\
0.554608489352822 -0.0117579581301979\\
0.554730569143567 -0.007456266131345\\
0.554852648934312 0.0398623458560367\\
0.554974728725056 0.0605104674505305\\
0.555096808515801 0.0513335245196444\\
0.555218888306546 0.0467450530542013\\
0.55534096809729 0.00344135359908231\\
0.555463047888035 -0.00516203039862346\\
0.555585127678779 -0.0490392887869229\\
0.555707207469524 -0.0605104674505305\\
0.555829287260269 -0.0605104674505305\\
0.555951367051013 -0.0306854029251506\\
0.556073446841758 0.00688270719816461\\
0.556195526632502 0.0117579581301979\\
0.556317606423247 0.0628047031832521\\
0.556439686213992 0.0662460567823344\\
0.556561766004736 0.0513335245196444\\
0.556683845795481 0.0151993117292802\\
0.556805925586226 -0.00573558933180384\\
0.55692800537697 -0.0266704903928879\\
0.557050085167715 -0.058216231717809\\
0.557172164958459 -0.0444508173214798\\
0.557294244749204 -0.0513335245196444\\
0.557416324539949 -0.00344135359908231\\
0.557538404330693 0.000573558933180384\\
0.557660484121438 0.020934901061084\\
0.557782563912183 0.0536277602523659\\
0.557904643702927 0.0513335245196444\\
0.558026723493672 0.0398623458560367\\
0.558148803284417 0.00458847146544307\\
0.558270883075161 0.00172067679954115\\
0.558392962865906 -0.0329796386578721\\
0.55851504265665 -0.0559219959850875\\
0.558637122447395 -0.0513335245196444\\
0.55875920223814 -0.0232291367938056\\
0.558881282028884 -0.000573558933180384\\
0.559003361819629 -0.00401491253226269\\
0.559125441610373 0.0329796386578721\\
0.559247521401118 0.0375681101233152\\
0.559369601191863 0.0306854029251506\\
0.559491680982607 0.016346429595641\\
0.559613760773352 0.0117579581301979\\
0.559735840564097 0.016346429595641\\
0.559857920354841 -0.0186406653283625\\
0.559980000145586 -0.0232291367938056\\
0.56010207993633 -0.0306854029251506\\
0.560224159727075 -0.028391167192429\\
0.56034623951782 -0.0306854029251506\\
0.560468319308564 -0.0243762546601663\\
0.560590399099309 0.00458847146544307\\
0.560712478890054 0.0306854029251506\\
0.560834558680798 0.0398623458560367\\
0.560956638471543 0.0490392887869229\\
0.561078718262287 0.0467450530542013\\
0.561200798053032 0.0232291367938056\\
0.561322877843777 -0.00229423573272154\\
0.561444957634521 -0.0444508173214798\\
0.561567037425266 -0.0513335245196444\\
0.561689117216011 -0.0628047031832521\\
0.561811197006755 -0.0536277602523659\\
0.5619332767975 -0.0243762546601663\\
0.562055356588244 0.016346429595641\\
0.562177436378989 0.0628047031832521\\
0.562299516169734 0.0662460567823344\\
0.562421595960478 0.0845999426441067\\
0.562543675751223 0.0490392887869229\\
0.562665755541968 0.00458847146544307\\
0.562787835332712 -0.0421565815887582\\
0.562909915123457 -0.0754229997132205\\
0.563031994914201 -0.0800114711786636\\
0.563154074704946 -0.0754229997132205\\
0.563276154495691 -0.0255233725265271\\
0.563398234286435 0.00946372239747634\\
0.56352031407718 0.0708345282477775\\
0.563642393867924 0.0754229997132205\\
0.563764473658669 0.0708345282477775\\
0.563886553449414 0.058216231717809\\
0.564008633240158 -0.000573558933180384\\
0.564130713030903 -0.0375681101233152\\
0.564252792821648 -0.0628047031832521\\
0.564374872612392 -0.0662460567823344\\
0.564496952403137 -0.0662460567823344\\
0.564619032193882 -0.0255233725265271\\
0.564741111984626 0.00946372239747634\\
0.564863191775371 0.028391167192429\\
0.564985271566115 0.0628047031832521\\
0.56510735135686 0.0559219959850875\\
0.565229431147605 0.0398623458560367\\
0.565351510938349 0.00172067679954115\\
0.565473590729094 -0.0151993117292802\\
0.565595670519839 -0.0306854029251506\\
0.565717750310583 -0.0352738743905936\\
0.565839830101328 -0.016346429595641\\
0.565961909892072 -0.0255233725265271\\
0.566083989682817 0.000573558933180384\\
0.566206069473562 0.00458847146544307\\
0.566328149264306 -0.00860338399770576\\
0.566450229055051 0.00802982506452538\\
0.566572308845795 0.0329796386578721\\
0.56669438863654 0.020934901061084\\
0.566816468427285 0.007456266131345\\
0.566938548218029 0.0329796386578721\\
0.567060628008774 0.0140521938629194\\
0.567182707799519 -0.00229423573272154\\
0.567304787590263 -0.0129050759965586\\
0.567426867381008 -0.028391167192429\\
0.567548947171752 -0.0421565815887582\\
0.567671026962497 -0.058216231717809\\
0.567793106753242 -0.0421565815887582\\
0.567915186543986 -0.0117579581301979\\
0.568037266334731 0.0536277602523659\\
0.568159346125476 0.0708345282477775\\
0.56828142591622 0.0754229997132205\\
0.568403505706965 0.0800114711786636\\
0.568525585497709 0.0220820189274448\\
0.568647665288454 -0.0421565815887582\\
0.568769745079199 -0.0754229997132205\\
0.568891824869943 -0.0845999426441067\\
0.569013904660688 -0.0845999426441067\\
0.569135984451433 -0.0513335245196444\\
0.569258064242177 -0.00286779466590192\\
0.569380144032922 0.0708345282477775\\
0.569502223823666 0.0983653570404359\\
0.569624303614411 0.107542299971322\\
0.569746383405156 0.0662460567823344\\
0.5698684631959 0.028391167192429\\
0.569990542986645 -0.0329796386578721\\
0.57011262277739 -0.102953828505879\\
0.570234702568134 -0.112130771436765\\
0.570356782358879 -0.0845999426441067\\
0.570478862149623 -0.0375681101233152\\
0.570600941940368 0.00860338399770576\\
0.570723021731113 0.0754229997132205\\
0.570845101521857 0.0983653570404359\\
0.570967181312602 0.0891884141095498\\
0.571089261103347 0.0444508173214798\\
0.571211340894091 -0.00946372239747634\\
0.571333420684836 -0.0421565815887582\\
0.57145550047558 -0.0662460567823344\\
0.571577580266325 -0.0662460567823344\\
0.57169966005707 -0.0490392887869229\\
0.571821739847814 -0.00114711786636077\\
0.571943819638559 0.007456266131345\\
0.572065899429304 0.028391167192429\\
0.572187979220048 0.0398623458560367\\
0.572310059010793 0.0421565815887582\\
0.572432138801537 0.0140521938629194\\
0.572554218592282 -0.0117579581301979\\
0.572676298383027 -0.00401491253226269\\
0.572798378173771 -0.0129050759965586\\
0.572920457964516 0.00516203039862346\\
0.57304253775526 0.007456266131345\\
0.573164617546005 0.00573558933180384\\
0.57328669733675 -0.0106108402638371\\
0.573408777127494 -0.0266704903928879\\
0.573530856918239 -0.0490392887869229\\
0.573652936708984 -0.0490392887869229\\
0.573775016499728 -0.0106108402638371\\
0.573897096290473 0.0232291367938056\\
0.574019176081217 0.0662460567823344\\
0.574141255871962 0.0845999426441067\\
0.574263335662707 0.0937768855749928\\
0.574385415453451 0.0421565815887582\\
0.574507495244196 -0.0232291367938056\\
0.574629575034941 -0.0891884141095498\\
0.574751654825685 -0.121307714367651\\
0.57487373461643 -0.112130771436765\\
0.574995814407175 -0.0754229997132205\\
0.575117894197919 0.0106108402638371\\
0.575239973988664 0.0937768855749928\\
0.575362053779408 0.151132778893031\\
0.575484133570153 0.141955835962145\\
0.575606213360898 0.0983653570404359\\
0.575728293151642 0.00458847146544307\\
0.575850372942387 -0.0800114711786636\\
0.575972452733131 -0.141955835962145\\
0.576094532523876 -0.151132778893031\\
0.576216612314621 -0.102953828505879\\
0.576338692105365 -0.0243762546601663\\
0.57646077189611 0.0467450530542013\\
0.576582851686855 0.112130771436765\\
0.576704931477599 0.169486664754804\\
0.576827011268344 0.116719242902208\\
0.576949091059088 0.0398623458560367\\
0.577071170849833 -0.0306854029251506\\
0.577193250640578 -0.0937768855749928\\
0.577315330431322 -0.135073128763981\\
0.577437410222067 -0.107542299971322\\
0.577559490012812 -0.0444508173214798\\
0.577681569803556 0.0197877831947233\\
0.577803649594301 0.0845999426441067\\
0.577925729385045 0.0891884141095498\\
0.57804780917579 0.0708345282477775\\
0.578169888966535 0.0243762546601663\\
0.578291968757279 -0.028391167192429\\
0.578414048548024 -0.058216231717809\\
0.578536128338769 -0.0329796386578721\\
0.578658208129513 0.00401491253226269\\
0.578780287920258 0\\
0.578902367711002 0.020934901061084\\
0.579024447501747 0.0375681101233152\\
0.579146527292492 -0.00573558933180384\\
0.579268607083236 -0.0490392887869229\\
0.579390686873981 -0.0490392887869229\\
0.579512766664726 -0.0536277602523659\\
0.57963484645547 -0.00401491253226269\\
0.579756926246215 0.0398623458560367\\
0.579879006036959 0.0754229997132205\\
0.580001085827704 0.102953828505879\\
0.580123165618449 0.0891884141095498\\
0.580245245409193 0.0174935474620017\\
0.580367325199938 -0.0662460567823344\\
0.580489404990682 -0.0891884141095498\\
0.580611484781427 -0.130484657298537\\
0.580733564572172 -0.135073128763981\\
0.580855644362916 -0.0513335245196444\\
0.580977724153661 0.0467450530542013\\
0.581099803944406 0.125896185833094\\
0.58122188373515 0.17866360768569\\
0.581343963525895 0.160309721823917\\
0.581466043316639 0.0800114711786636\\
0.581588123107384 -0.016346429595641\\
0.581710202898129 -0.121307714367651\\
0.581832282688873 -0.215371379409234\\
0.581954362479618 -0.17866360768569\\
0.582076442270363 -0.107542299971322\\
0.582198522061107 -0.0151993117292802\\
0.582320601851852 0.135073128763981\\
0.582442681642596 0.233725265271007\\
0.582564761433341 0.197017493547462\\
0.582686841224086 0.0937768855749928\\
0.58280892101483 -0.00229423573272154\\
0.582931000805575 -0.130484657298537\\
0.58305308059632 -0.17866360768569\\
0.583175160387064 -0.160309721823917\\
0.583297240177809 -0.0891884141095498\\
0.583419319968553 0.028391167192429\\
0.583541399759298 0.0891884141095498\\
0.583663479550043 0.116719242902208\\
0.583785559340787 0.107542299971322\\
0.583907639131532 0.058216231717809\\
0.584029718922277 -0.0306854029251506\\
0.584151798713021 -0.0845999426441067\\
0.584273878503766 -0.0662460567823344\\
0.58439595829451 -0.0197877831947233\\
0.584518038085255 -0.0151993117292802\\
0.584640117876 0.0375681101233152\\
0.584762197666744 0.058216231717809\\
0.584884277457489 0.0375681101233152\\
0.585006357248234 -0.0329796386578721\\
0.585128437038978 -0.0559219959850875\\
0.585250516829723 -0.058216231717809\\
0.585372596620467 -0.0559219959850875\\
0.585494676411212 0.00946372239747634\\
0.585616756201957 0.0375681101233152\\
0.585738835992701 0.116719242902208\\
0.585860915783446 0.125896185833094\\
0.585982995574191 0.0800114711786636\\
0.586105075364935 -0.016346429595641\\
0.58622715515568 -0.0754229997132205\\
0.586349234946424 -0.141955835962145\\
0.586471314737169 -0.197017493547462\\
0.586593394527914 -0.112130771436765\\
0.586715474318658 -0.00114711786636077\\
0.586837554109403 0.121307714367651\\
0.586959633900147 0.206194436478348\\
0.587081713690892 0.22454832234012\\
0.587203793481637 0.141955835962145\\
0.587325873272381 0.0117579581301979\\
0.587447953063126 -0.141955835962145\\
0.587570032853871 -0.252079151132779\\
0.587692112644615 -0.233725265271007\\
0.58781419243536 -0.135073128763981\\
0.587936272226104 -0.0306854029251506\\
0.588058352016849 0.130484657298537\\
0.588180431807594 0.261256094063665\\
0.588302511598338 0.233725265271007\\
0.588424591389083 0.107542299971322\\
0.588546671179828 0.0151993117292802\\
0.588668750970572 -0.116719242902208\\
0.588790830761317 -0.197017493547462\\
0.588912910552062 -0.17866360768569\\
0.589034990342806 -0.121307714367651\\
0.589157070133551 0.00286779466590192\\
0.589279149924295 0.102953828505879\\
0.58940122971504 0.141955835962145\\
0.589523309505785 0.125896185833094\\
0.589645389296529 0.0845999426441067\\
0.589767469087274 -0.0266704903928879\\
0.589889548878018 -0.102953828505879\\
0.590011628668763 -0.0891884141095498\\
0.590133708459508 -0.0444508173214798\\
0.590255788250252 0.000573558933180384\\
0.590377868040997 0.0375681101233152\\
0.590499947831742 0.0845999426441067\\
0.590622027622486 0.0186406653283625\\
0.590744107413231 -0.0174935474620017\\
0.590866187203975 -0.0628047031832521\\
0.59098826699472 -0.058216231717809\\
0.591110346785465 -0.0398623458560367\\
0.591232426576209 -0.0129050759965586\\
0.591354506366954 0.0398623458560367\\
0.591476586157699 0.0800114711786636\\
0.591598665948443 0.160309721823917\\
0.591720745739188 0.0662460567823344\\
0.591842825529932 0.020934901061084\\
0.591964905320677 -0.0536277602523659\\
0.592086985111422 -0.141955835962145\\
0.592209064902166 -0.197017493547462\\
0.592331144692911 -0.17866360768569\\
0.592453224483656 -0.0243762546601663\\
0.5925753042744 0.0845999426441067\\
0.592697384065145 0.233725265271007\\
0.592819463855889 0.270433036994551\\
0.592941543646634 0.215371379409234\\
0.593063623437379 0.0421565815887582\\
0.593185703228123 -0.160309721823917\\
0.593307783018868 -0.293375394321767\\
0.593429862809613 -0.270433036994551\\
0.593551942600357 -0.187840550616576\\
0.593674022391102 -0.0398623458560367\\
0.593796102181846 0.17866360768569\\
0.593918181972591 0.293375394321767\\
0.594040261763336 0.261256094063665\\
0.59416234155408 0.121307714367651\\
0.594284421344825 -0.016346429595641\\
0.594406501135569 -0.141955835962145\\
0.594528580926314 -0.215371379409234\\
0.594650660717059 -0.187840550616576\\
0.594772740507803 -0.0937768855749928\\
0.594894820298548 0.0197877831947233\\
0.595016900089293 0.0937768855749928\\
0.595138979880037 0.102953828505879\\
0.595261059670782 0.125896185833094\\
0.595383139461527 0.0891884141095498\\
0.595505219252271 0.0106108402638371\\
0.595627299043016 -0.0754229997132205\\
0.59574937883376 -0.0536277602523659\\
0.595871458624505 -0.0605104674505305\\
0.59599353841525 -0.0754229997132205\\
0.596115618205994 0.00946372239747634\\
0.596237697996739 0.0444508173214798\\
0.596359777787484 0.0754229997132205\\
0.596481857578228 0.0444508173214798\\
0.596603937368973 -0.00688270719816461\\
0.596726017159717 -0.0662460567823344\\
0.596848096950462 -0.0891884141095498\\
0.596970176741207 -0.0754229997132205\\
0.597092256531951 -0.0186406653283625\\
0.597214336322696 0.107542299971322\\
0.59733641611344 0.206194436478348\\
0.597458495904185 0.169486664754804\\
0.59758057569493 0.0398623458560367\\
0.597702655485674 -0.0754229997132205\\
0.597824735276419 -0.215371379409234\\
0.597946815067164 -0.279609979925437\\
0.598068894857908 -0.206194436478348\\
0.598190974648653 -0.0220820189274448\\
0.598313054439397 0.187840550616576\\
0.598435134230142 0.293375394321767\\
0.598557214020887 0.311729280183539\\
0.598679293811631 0.169486664754804\\
0.598801373602376 0.00172067679954115\\
0.598923453393121 -0.233725265271007\\
0.599045533183865 -0.348437051907083\\
0.59916761297461 -0.252079151132779\\
0.599289692765354 -0.107542299971322\\
0.599411772556099 0.028391167192429\\
0.599533852346844 0.17866360768569\\
0.599655932137588 0.261256094063665\\
0.599778011928333 0.187840550616576\\
0.599900091719078 0.0845999426441067\\
0.600022171509822 -0.00172067679954115\\
0.600144251300567 -0.0708345282477775\\
0.600266331091311 -0.141955835962145\\
0.600388410882056 -0.17866360768569\\
0.600510490672801 -0.160309721823917\\
0.600632570463545 -0.0754229997132205\\
0.60075465025429 0.0375681101233152\\
0.600876730045035 0.125896185833094\\
0.600998809835779 0.197017493547462\\
0.601120889626524 0.22454832234012\\
0.601242969417268 0.0513335245196444\\
0.601365049208013 -0.125896185833094\\
0.601487128998758 -0.187840550616576\\
0.601609208789502 -0.160309721823917\\
0.601731288580247 -0.0845999426441067\\
0.601853368370992 0.0662460567823344\\
0.601975448161736 0.206194436478348\\
0.602097527952481 0.187840550616576\\
0.602219607743225 0.0243762546601663\\
0.60234168753397 -0.151132778893031\\
0.602463767324715 -0.233725265271007\\
0.602585847115459 -0.141955835962145\\
0.602707926906204 -0.0421565815887582\\
0.602830006696949 0.112130771436765\\
0.602952086487693 0.293375394321767\\
0.603074166278438 0.279609979925437\\
0.603196246069182 0.0891884141095498\\
0.603318325859927 -0.141955835962145\\
0.603440405650672 -0.215371379409234\\
0.603562485441416 -0.279609979925437\\
0.603684565232161 -0.206194436478348\\
0.603806645022905 -0.0444508173214798\\
0.60392872481365 0.130484657298537\\
0.604050804604395 0.215371379409234\\
0.604172884395139 0.187840550616576\\
0.604294964185884 0.130484657298537\\
0.604417043976629 0.0800114711786636\\
0.604539123767373 0.0754229997132205\\
0.604661203558118 -0.102953828505879\\
0.604783283348862 -0.197017493547462\\
0.604905363139607 -0.22454832234012\\
0.605027442930352 -0.252079151132779\\
0.605149522721096 -0.160309721823917\\
0.605271602511841 0.0559219959850875\\
0.605393682302586 0.330083166045311\\
0.60551576209333 0.4034987094924\\
0.605637841884075 0.311729280183539\\
0.60575992167482 0.0421565815887582\\
0.605882001465564 -0.215371379409234\\
0.606004081256309 -0.348437051907083\\
0.606126161047053 -0.385144823630628\\
0.606248240837798 -0.17866360768569\\
0.606370320628543 0.141955835962145\\
0.606492400419287 0.311729280183539\\
0.606614480210032 0.242902208201893\\
0.606736560000776 0.130484657298537\\
0.606858639791521 -0.0467450530542013\\
0.606980719582266 -0.206194436478348\\
0.60710279937301 -0.17866360768569\\
0.607224879163755 -0.0398623458560367\\
0.6073469589545 0.0983653570404359\\
0.607469038745244 0.160309721823917\\
0.607591118535989 0.0891884141095498\\
0.607713198326733 -0.0559219959850875\\
0.607835278117478 -0.0800114711786636\\
0.607957357908223 -0.121307714367651\\
0.608079437698967 -0.102953828505879\\
0.608201517489712 0.020934901061084\\
0.608323597280457 0.0983653570404359\\
0.608445677071201 0.0352738743905936\\
0.608567756861946 -0.0106108402638371\\
0.60868983665269 0.0628047031832521\\
0.608811916443435 0.0662460567823344\\
0.60893399623418 0.116719242902208\\
0.609056076024924 0.102953828505879\\
0.609178155815669 -0.00229423573272154\\
0.609300235606414 -0.187840550616576\\
0.609422315397158 -0.311729280183539\\
0.609544395187903 -0.293375394321767\\
0.609666474978647 -0.121307714367651\\
0.609788554769392 0.233725265271007\\
0.609910634560137 0.421852595354173\\
0.610032714350881 0.440206481215945\\
0.610154794141626 0.270433036994551\\
0.610276873932371 -0.0559219959850875\\
0.610398953723115 -0.4034987094924\\
0.61052103351386 -0.531975910524806\\
0.610643113304604 -0.330083166045311\\
0.610765193095349 -0.0800114711786636\\
0.610887272886094 0.233725265271007\\
0.611009352676838 0.366790937768856\\
0.611131432467583 0.311729280183539\\
0.611253512258327 0.160309721823917\\
0.611375592049072 -0.0708345282477775\\
0.611497671839817 -0.151132778893031\\
0.611619751630561 -0.130484657298537\\
0.611741831421306 -0.0513335245196444\\
0.611863911212051 -0.0754229997132205\\
0.611985991002795 -0.0983653570404359\\
0.61210807079354 -0.0197877831947233\\
0.612230150584284 -0.0444508173214798\\
0.612352230375029 0.0375681101233152\\
0.612474310165774 0.169486664754804\\
0.612596389956518 0.22454832234012\\
0.612718469747263 0.102953828505879\\
0.612840549538008 -0.0708345282477775\\
0.612962629328752 -0.17866360768569\\
0.613084709119497 -0.22454832234012\\
0.613206788910242 -0.0891884141095498\\
0.613328868700986 0.0800114711786636\\
0.613450948491731 0.233725265271007\\
0.613573028282475 0.242902208201893\\
0.61369510807322 0.016346429595641\\
0.613817187863965 -0.215371379409234\\
0.613939267654709 -0.311729280183539\\
0.614061347445454 -0.22454832234012\\
0.614183427236198 -0.028391167192429\\
0.614305507026943 0.261256094063665\\
0.614427586817688 0.47691425293949\\
0.614549666608432 0.311729280183539\\
0.614671746399177 0.0490392887869229\\
0.614793826189922 -0.270433036994551\\
0.614915905980666 -0.421852595354173\\
0.615037985771411 -0.348437051907083\\
0.615160065562155 -0.169486664754804\\
0.6152821453529 0.121307714367651\\
0.615404225143645 0.270433036994551\\
0.615526304934389 0.330083166045311\\
0.615648384725134 0.135073128763981\\
0.615770464515879 0.0444508173214798\\
0.615892544306623 0.0106108402638371\\
0.616014624097368 -0.0467450530542013\\
0.616136703888112 -0.0891884141095498\\
0.616258783678857 -0.187840550616576\\
0.616380863469602 -0.17866360768569\\
0.616502943260346 -0.261256094063665\\
0.616625023051091 -0.125896185833094\\
0.616747102841836 0.125896185833094\\
0.61686918263258 0.366790937768856\\
0.616991262423325 0.458560367077717\\
0.617113342214069 0.252079151132779\\
0.617235422004814 -0.0186406653283625\\
0.617357501795559 -0.311729280183539\\
0.617479581586303 -0.47691425293949\\
0.617601661377048 -0.385144823630628\\
0.617723741167792 -0.0444508173214798\\
0.617845820958537 0.311729280183539\\
0.617967900749282 0.366790937768856\\
0.618089980540026 0.233725265271007\\
0.618212060330771 0.0329796386578721\\
0.618334140121516 -0.206194436478348\\
0.61845621991226 -0.293375394321767\\
0.618578299703005 -0.160309721823917\\
0.618700379493749 0.0937768855749928\\
0.618822459284494 0.242902208201893\\
0.618944539075239 0.22454832234012\\
0.619066618865983 0.00401491253226269\\
0.619188698656728 -0.151132778893031\\
0.619310778447473 -0.187840550616576\\
0.619432858238217 -0.187840550616576\\
0.619554938028962 -0.0421565815887582\\
0.619677017819707 0.141955835962145\\
0.619799097610451 0.125896185833094\\
0.619921177401196 0.0186406653283625\\
0.62004325719194 0.000573558933180384\\
0.620165336982685 0.0398623458560367\\
0.62028741677343 0.102953828505879\\
0.620409496564174 0.135073128763981\\
0.620531576354919 0.0891884141095498\\
0.620653656145663 -0.112130771436765\\
0.620775735936408 -0.293375394321767\\
0.620897815727153 -0.4034987094924\\
0.621019895517897 -0.270433036994551\\
0.621141975308642 0.130484657298537\\
0.621264055099387 0.421852595354173\\
0.621386134890131 0.531975910524806\\
0.621508214680876 0.421852595354173\\
0.62163029447162 0.0937768855749928\\
0.621752374262365 -0.385144823630628\\
0.62187445405311 -0.596214511041009\\
0.621996533843854 -0.440206481215945\\
0.622118613634599 -0.160309721823917\\
0.622240693425344 0.206194436478348\\
0.622362773216088 0.366790937768856\\
0.622484853006833 0.385144823630628\\
0.622606932797577 0.187840550616576\\
0.622729012588322 -0.0845999426441067\\
0.622851092379067 -0.169486664754804\\
0.622973172169811 -0.141955835962145\\
0.623095251960556 0.028391167192429\\
0.623217331751301 -0.0513335245196444\\
0.623339411542045 -0.0891884141095498\\
0.62346149133279 -0.0444508173214798\\
0.623583571123534 -0.0937768855749928\\
0.623705650914279 -0.0375681101233152\\
0.623827730705024 0.116719242902208\\
0.623949810495768 0.279609979925437\\
0.624071890286513 0.169486664754804\\
0.624193970077258 -0.0306854029251506\\
0.624316049868002 -0.197017493547462\\
0.624438129658747 -0.252079151132779\\
0.624560209449491 -0.107542299971322\\
0.624682289240236 0.0845999426441067\\
0.624804369030981 0.311729280183539\\
0.624926448821725 0.293375394321767\\
0.62504852861247 0.0559219959850875\\
0.625170608403214 -0.311729280183539\\
0.625292688193959 -0.458560367077717\\
0.625414767984704 -0.279609979925437\\
0.625536847775448 -0.028391167192429\\
0.625658927566193 0.348437051907083\\
0.625781007356938 0.550329796386579\\
0.625903087147682 0.421852595354173\\
0.626025166938427 0.0559219959850875\\
0.626147246729172 -0.348437051907083\\
0.626269326519916 -0.495268138801262\\
0.626391406310661 -0.385144823630628\\
0.626513486101405 -0.125896185833094\\
0.62663556589215 0.0937768855749928\\
0.626757645682895 0.293375394321767\\
0.626879725473639 0.279609979925437\\
0.627001805264384 0.121307714367651\\
0.627123885055129 -0.00344135359908231\\
0.627245964845873 0.0513335245196444\\
0.627368044636618 0.121307714367651\\
0.627490124427362 -0.0421565815887582\\
0.627612204218107 -0.160309721823917\\
0.627734284008852 -0.279609979925437\\
0.627856363799596 -0.348437051907083\\
0.627978443590341 -0.252079151132779\\
0.628100523381085 0.102953828505879\\
0.62822260317183 0.47691425293949\\
0.628344682962575 0.568683682248351\\
0.628466762753319 0.293375394321767\\
0.628588842544064 -0.0605104674505305\\
0.628710922334809 -0.366790937768856\\
0.628833002125553 -0.531975910524806\\
0.628955081916298 -0.348437051907083\\
0.629077161707042 0.0117579581301979\\
0.629199241497787 0.421852595354173\\
0.629321321288532 0.385144823630628\\
0.629443401079276 0.151132778893031\\
0.629565480870021 -0.116719242902208\\
0.629687560660766 -0.270433036994551\\
0.62980964045151 -0.261256094063665\\
0.629931720242255 -0.0891884141095498\\
0.630053800032999 0.233725265271007\\
0.630175879823744 0.311729280183539\\
0.630297959614489 0.187840550616576\\
0.630420039405233 -0.0662460567823344\\
0.630542119195978 -0.22454832234012\\
0.630664198986723 -0.17866360768569\\
0.630786278777467 -0.125896185833094\\
0.630908358568212 0.0117579581301979\\
0.631030438358956 0.0937768855749928\\
0.631152518149701 0.0800114711786636\\
0.631274597940446 -0.0628047031832521\\
0.63139667773119 -0.0845999426441067\\
0.631518757521935 0.141955835962145\\
0.63164083731268 0.242902208201893\\
0.631762917103424 0.330083166045311\\
0.631884996894169 0.0845999426441067\\
0.632007076684913 -0.206194436478348\\
0.632129156475658 -0.458560367077717\\
0.632251236266403 -0.596214511041009\\
0.632373316057147 -0.279609979925437\\
0.632495395847892 0.187840550616576\\
0.632617475638637 0.669630054488099\\
0.632739555429381 0.632922282764554\\
0.632861635220126 0.366790937768856\\
0.63298371501087 -0.0352738743905936\\
0.633105794801615 -0.495268138801262\\
0.63322787459236 -0.632922282764554\\
0.633349954383104 -0.385144823630628\\
0.633472034173849 0.0129050759965586\\
0.633594113964594 0.252079151132779\\
0.633716193755338 0.330083166045311\\
0.633838273546083 0.206194436478348\\
0.633960353336827 0.0243762546601663\\
0.634082433127572 -0.0800114711786636\\
0.634204512918317 -0.0708345282477775\\
0.634326592709061 0.0306854029251506\\
0.634448672499806 0.107542299971322\\
0.63457075229055 -0.0220820189274448\\
0.634692832081295 -0.187840550616576\\
0.63481491187204 -0.22454832234012\\
0.634936991662784 -0.121307714367651\\
0.635059071453529 0.0232291367938056\\
0.635181151244274 0.242902208201893\\
0.635303231035018 0.348437051907083\\
0.635425310825763 0.125896185833094\\
0.635547390616507 -0.151132778893031\\
0.635669470407252 -0.330083166045311\\
0.635791550197997 -0.270433036994551\\
0.635913629988741 0.0232291367938056\\
0.636035709779486 0.311729280183539\\
0.636157789570231 0.385144823630628\\
0.636279869360975 0.233725265271007\\
0.63640194915172 -0.0983653570404359\\
0.636524028942465 -0.47691425293949\\
0.636646108733209 -0.568683682248351\\
0.636768188523954 -0.215371379409234\\
0.636890268314698 0.169486664754804\\
0.637012348105443 0.47691425293949\\
0.637134427896188 0.596214511041009\\
0.637256507686932 0.330083166045311\\
0.637378587477677 -0.0800114711786636\\
0.637500667268421 -0.385144823630628\\
0.637622747059166 -0.421852595354173\\
0.637744826849911 -0.270433036994551\\
0.637866906640655 -0.016346429595641\\
0.6379889864314 0.0891884141095498\\
0.638111066222145 0.0891884141095498\\
0.638233146012889 0.169486664754804\\
0.638355225803634 0.0352738743905936\\
0.638477305594378 0.0891884141095498\\
0.638599385385123 0.252079151132779\\
0.638721465175868 0.270433036994551\\
0.638843544966612 -0.0329796386578721\\
0.638965624757357 -0.330083166045311\\
0.639087704548102 -0.421852595354173\\
0.639209784338846 -0.47691425293949\\
0.639331864129591 -0.125896185833094\\
0.639453943920335 0.261256094063665\\
0.63957602371108 0.568683682248351\\
0.639698103501825 0.568683682248351\\
0.639820183292569 0.187840550616576\\
0.639942263083314 -0.279609979925437\\
0.640064342874059 -0.47691425293949\\
0.640186422664803 -0.366790937768856\\
0.640308502455548 -0.187840550616576\\
0.640430582246292 0.233725265271007\\
0.640552662037037 0.440206481215945\\
0.640674741827782 0.252079151132779\\
0.640796821618526 -0.0800114711786636\\
0.640918901409271 -0.293375394321767\\
0.641040981200016 -0.270433036994551\\
0.64116306099076 -0.107542299971322\\
0.641285140781505 0.125896185833094\\
0.641407220572249 0.261256094063665\\
0.641529300362994 0.279609979925437\\
0.641651380153739 0.102953828505879\\
0.641773459944483 -0.197017493547462\\
0.641895539735228 -0.233725265271007\\
0.642017619525972 -0.0467450530542013\\
0.642139699316717 0.0197877831947233\\
0.642261779107462 -0.0106108402638371\\
0.642383858898206 -0.0329796386578721\\
0.642505938688951 -0.107542299971322\\
0.642628018479696 -0.187840550616576\\
0.64275009827044 0.020934901061084\\
0.642872178061185 0.311729280183539\\
0.64299425785193 0.495268138801262\\
0.643116337642674 0.4034987094924\\
0.643238417433419 -0.0662460567823344\\
0.643360497224163 -0.47691425293949\\
0.643482577014908 -0.669630054488099\\
0.643604656805653 -0.513622024663034\\
0.643726736596397 -0.151132778893031\\
0.643848816387142 0.495268138801262\\
0.643970896177887 0.816461141382277\\
0.644092975968631 0.495268138801262\\
0.644215055759376 0.141955835962145\\
0.64433713555012 -0.293375394321767\\
0.644459215340865 -0.458560367077717\\
0.64458129513161 -0.440206481215945\\
0.644703374922354 -0.0891884141095498\\
0.644825454713099 0.116719242902208\\
0.644947534503843 0.197017493547462\\
0.645069614294588 0.116719242902208\\
0.645191694085333 -0.121307714367651\\
0.645313773876077 0.00229423573272154\\
0.645435853666822 0.107542299971322\\
0.645557933457567 0.17866360768569\\
0.645680013248311 0.160309721823917\\
0.645802093039056 0.0708345282477775\\
0.6459241728298 -0.215371379409234\\
0.646046252620545 -0.440206481215945\\
0.64616833241129 -0.22454832234012\\
0.646290412202034 0.0662460567823344\\
0.646412491992779 0.330083166045311\\
0.646534571783524 0.4034987094924\\
0.646656651574268 0.151132778893031\\
0.646778731365013 -0.160309721823917\\
0.646900811155757 -0.421852595354173\\
0.647022890946502 -0.4034987094924\\
0.647144970737247 -0.0232291367938056\\
0.647267050527991 0.440206481215945\\
0.647389130318736 0.568683682248351\\
0.647511210109481 0.270433036994551\\
0.647633289900225 -0.0352738743905936\\
0.64775536969097 -0.458560367077717\\
0.647877449481714 -0.632922282764554\\
0.647999529272459 -0.4034987094924\\
0.648121609063204 0.0937768855749928\\
0.648243688853948 0.47691425293949\\
0.648365768644693 0.458560367077717\\
0.648487848435438 0.311729280183539\\
0.648609928226182 0.0174935474620017\\
0.648732008016927 -0.206194436478348\\
0.648854087807671 -0.233725265271007\\
0.648976167598416 -0.17866360768569\\
0.649098247389161 0.007456266131345\\
0.649220327179905 -0.0559219959850875\\
0.64934240697065 -0.169486664754804\\
0.649464486761394 -0.206194436478348\\
0.649586566552139 0.007456266131345\\
0.649708646342884 0.270433036994551\\
0.649830726133628 0.4034987094924\\
0.649952805924373 0.531975910524806\\
0.650074885715118 0.187840550616576\\
0.650196965505862 -0.242902208201893\\
0.650319045296607 -0.669630054488099\\
0.650441125087352 -0.669630054488099\\
0.650563204878096 -0.22454832234012\\
0.650685284668841 0.242902208201893\\
0.650807364459585 0.596214511041009\\
0.65092944425033 0.513622024663034\\
0.651051524041075 0.311729280183539\\
0.651173603831819 -0.22454832234012\\
0.651295683622564 -0.531975910524806\\
0.651417763413308 -0.330083166045311\\
0.651539843204053 -0.0140521938629194\\
0.651661922994798 0.252079151132779\\
0.651784002785542 0.242902208201893\\
0.651906082576287 0.187840550616576\\
0.652028162367032 -0.116719242902208\\
0.652150242157776 -0.293375394321767\\
0.652272321948521 -0.22454832234012\\
0.652394401739265 -0.0106108402638371\\
0.65251648153001 0.279609979925437\\
0.652638561320755 0.197017493547462\\
0.652760641111499 -0.0106108402638371\\
0.652882720902244 -0.0845999426441067\\
0.653004800692989 -0.0708345282477775\\
0.653126880483733 -0.0352738743905936\\
0.653248960274478 0.107542299971322\\
0.653371040065222 0.293375394321767\\
0.653493119855967 0.0306854029251506\\
0.653615199646712 -0.311729280183539\\
0.653737279437456 -0.458560367077717\\
0.653859359228201 -0.366790937768856\\
0.653981439018946 0.0628047031832521\\
0.65410351880969 0.421852595354173\\
0.654225598600435 0.706337826211643\\
0.654347678391179 0.550329796386579\\
0.654469758181924 0.102953828505879\\
0.654591837972669 -0.531975910524806\\
0.654713917763413 -0.816461141382277\\
0.654835997554158 -0.513622024663034\\
0.654958077344903 -0.169486664754804\\
0.655080157135647 0.311729280183539\\
0.655202236926392 0.632922282764554\\
0.655324316717136 0.495268138801262\\
0.655446396507881 0.107542299971322\\
0.655568476298626 -0.187840550616576\\
0.65569055608937 -0.22454832234012\\
0.655812635880115 -0.17866360768569\\
0.655934715670859 -0.00286779466590192\\
0.656056795461604 -0.00946372239747634\\
0.656178875252349 -0.112130771436765\\
0.656300955043093 -0.112130771436765\\
0.656423034833838 -0.169486664754804\\
0.656545114624583 -0.00630914826498423\\
0.656667194415327 0.311729280183539\\
0.656789274206072 0.47691425293949\\
0.656911353996817 0.215371379409234\\
0.657033433787561 -0.112130771436765\\
0.657155513578306 -0.311729280183539\\
0.65727759336905 -0.47691425293949\\
0.657399673159795 -0.242902208201893\\
0.65752175295054 0.187840550616576\\
0.657643832741284 0.440206481215945\\
0.657765912532029 0.421852595354173\\
0.657887992322774 0.0662460567823344\\
0.658010072113518 -0.330083166045311\\
0.658132151904263 -0.495268138801262\\
0.658254231695007 -0.311729280183539\\
0.658376311485752 -0.00172067679954115\\
0.658498391276497 0.421852595354173\\
0.658620471067241 0.669630054488099\\
0.658742550857986 0.293375394321767\\
0.65886463064873 -0.116719242902208\\
0.658986710439475 -0.421852595354173\\
0.65910879023022 -0.513622024663034\\
0.659230870020964 -0.311729280183539\\
0.659352949811709 0.0329796386578721\\
0.659475029602454 0.270433036994551\\
0.659597109393198 0.279609979925437\\
0.659719189183943 0.197017493547462\\
0.659841268974687 -0.0375681101233152\\
0.659963348765432 -0.00946372239747634\\
0.660085428556177 0.160309721823917\\
0.660207508346921 0.112130771436765\\
0.660329588137666 -0.028391167192429\\
0.660451667928411 -0.233725265271007\\
0.660573747719155 -0.4034987094924\\
0.6606958275099 -0.531975910524806\\
0.660817907300644 -0.0983653570404359\\
0.660939987091389 0.385144823630628\\
0.661062066882134 0.706337826211643\\
0.661184146672878 0.632922282764554\\
0.661306226463623 0.197017493547462\\
0.661428306254368 -0.311729280183539\\
0.661550386045112 -0.706337826211643\\
0.661672465835857 -0.632922282764554\\
0.661794545626601 -0.270433036994551\\
0.661916625417346 0.348437051907083\\
0.662038705208091 0.550329796386579\\
0.662160784998835 0.348437051907083\\
0.66228286478958 0.0891884141095498\\
0.662404944580325 -0.233725265271007\\
0.662527024371069 -0.348437051907083\\
0.662649104161814 -0.197017493547462\\
0.662771183952558 0.17866360768569\\
0.662893263743303 0.261256094063665\\
0.663015343534048 0.130484657298537\\
0.663137423324792 -0.0266704903928879\\
0.663259503115537 -0.252079151132779\\
0.663381582906282 -0.22454832234012\\
0.663503662697026 -0.0220820189274448\\
0.663625742487771 0.151132778893031\\
0.663747822278515 0.22454832234012\\
0.66386990206926 0.0891884141095498\\
0.663991981860005 -0.197017493547462\\
0.664114061650749 -0.279609979925437\\
0.664236141441494 -0.0490392887869229\\
0.664358221232239 0.233725265271007\\
0.664480301022983 0.366790937768856\\
0.664602380813728 0.366790937768856\\
0.664724460604472 0.0490392887869229\\
0.664846540395217 -0.458560367077717\\
0.664968620185962 -0.706337826211643\\
0.665090699976706 -0.495268138801262\\
0.665212779767451 0.0140521938629194\\
0.665334859558195 0.550329796386579\\
0.66545693934894 0.779753369658732\\
0.665579019139685 0.531975910524806\\
0.665701098930429 0.107542299971322\\
0.665823178721174 -0.421852595354173\\
0.665945258511919 -0.706337826211643\\
0.666067338302663 -0.458560367077717\\
0.666189418093408 -0.0421565815887582\\
0.666311497884152 0.17866360768569\\
0.666433577674897 0.252079151132779\\
0.666555657465642 0.252079151132779\\
0.666677737256386 0.0605104674505305\\
0.666799817047131 -0.0628047031832521\\
0.666921896837876 0.0754229997132205\\
0.66704397662862 0.197017493547462\\
0.667166056419365 0.151132778893031\\
0.66728813621011 -0.116719242902208\\
0.667410216000854 -0.421852595354173\\
0.667532295791599 -0.440206481215945\\
0.667654375582343 -0.22454832234012\\
0.667776455373088 0.121307714367651\\
0.667898535163833 0.458560367077717\\
0.668020614954577 0.632922282764554\\
0.668142694745322 0.293375394321767\\
0.668264774536066 -0.242902208201893\\
0.668386854326811 -0.550329796386579\\
0.668508934117556 -0.458560367077717\\
0.6686310139083 -0.130484657298537\\
0.668753093699045 0.270433036994551\\
0.66887517348979 0.550329796386579\\
0.668997253280534 0.348437051907083\\
0.669119333071279 -0.0140521938629194\\
0.669241412862023 -0.458560367077717\\
0.669363492652768 -0.550329796386579\\
0.669485572443513 -0.206194436478348\\
0.669607652234257 0.116719242902208\\
0.669729732025002 0.421852595354173\\
0.669851811815747 0.458560367077717\\
0.669973891606491 0.293375394321767\\
0.670095971397236 -0.141955835962145\\
0.67021805118798 -0.366790937768856\\
0.670340130978725 -0.233725265271007\\
0.67046221076947 -0.0937768855749928\\
0.670584290560214 0.0467450530542013\\
0.670706370350959 -0.0174935474620017\\
0.670828450141704 -0.0708345282477775\\
0.670950529932448 -0.112130771436765\\
0.671072609723193 -0.00172067679954115\\
0.671194689513937 0.215371379409234\\
0.671316769304682 0.495268138801262\\
0.671438849095427 0.495268138801262\\
0.671560928886171 0.0174935474620017\\
0.671683008676916 -0.458560367077717\\
0.671805088467661 -0.669630054488099\\
0.671927168258405 -0.596214511041009\\
0.67204924804915 -0.233725265271007\\
0.672171327839894 0.421852595354173\\
0.672293407630639 0.779753369658732\\
0.672415487421384 0.632922282764554\\
0.672537567212128 0.215371379409234\\
0.672659647002873 -0.311729280183539\\
0.672781726793617 -0.568683682248351\\
0.672903806584362 -0.458560367077717\\
0.673025886375107 -0.141955835962145\\
0.673147966165851 0.17866360768569\\
0.673270045956596 0.385144823630628\\
0.673392125747341 0.206194436478348\\
0.673514205538085 -0.141955835962145\\
0.67363628532883 -0.197017493547462\\
0.673758365119575 -0.0444508173214798\\
0.673880444910319 0.0513335245196444\\
0.674002524701064 0.197017493547462\\
0.674124604491808 0.17866360768569\\
0.674246684282553 -0.0490392887869229\\
0.674368764073298 -0.22454832234012\\
0.674490843864042 -0.252079151132779\\
0.674612923654787 -0.0605104674505305\\
0.674735003445532 0.22454832234012\\
0.674857083236276 0.348437051907083\\
0.674979163027021 0.0398623458560367\\
0.675101242817765 -0.151132778893031\\
0.67522332260851 -0.330083166045311\\
0.675345402399255 -0.4034987094924\\
0.675467482189999 -0.0398623458560367\\
0.675589561980744 0.47691425293949\\
0.675711641771488 0.669630054488099\\
0.675833721562233 0.385144823630628\\
0.675955801352978 -0.0421565815887582\\
0.676077881143722 -0.550329796386579\\
0.676199960934467 -0.743045597935188\\
0.676322040725212 -0.513622024663034\\
0.676444120515956 -0.0536277602523659\\
0.676566200306701 0.568683682248351\\
0.676688280097445 0.669630054488099\\
0.67681035988819 0.4034987094924\\
0.676932439678935 -0.00802982506452538\\
0.677054519469679 -0.215371379409234\\
0.677176599260424 -0.270433036994551\\
0.677298679051169 -0.293375394321767\\
0.677420758841913 -0.0306854029251506\\
0.677542838632658 0.00573558933180384\\
0.677664918423402 -0.058216231717809\\
0.677786998214147 -0.215371379409234\\
0.677909078004892 -0.0513335245196444\\
0.678031157795636 0.22454832234012\\
0.678153237586381 0.440206481215945\\
0.678275317377126 0.421852595354173\\
0.67839739716787 0.135073128763981\\
0.678519476958615 -0.141955835962145\\
0.678641556749359 -0.568683682248351\\
0.678763636540104 -0.669630054488099\\
0.678885716330849 -0.206194436478348\\
0.679007796121593 0.366790937768856\\
0.679129875912338 0.596214511041009\\
0.679251955703083 0.440206481215945\\
0.679374035493827 0.197017493547462\\
0.679496115284572 -0.330083166045311\\
0.679618195075316 -0.596214511041009\\
0.679740274866061 -0.385144823630628\\
0.679862354656806 0.0845999426441067\\
0.67998443444755 0.495268138801262\\
0.680106514238295 0.47691425293949\\
0.680228594029039 0.160309721823917\\
0.680350673819784 -0.233725265271007\\
0.680472753610529 -0.385144823630628\\
0.680594833401273 -0.47691425293949\\
0.680716913192018 -0.151132778893031\\
0.680838992982763 0.311729280183539\\
0.680961072773507 0.293375394321767\\
0.681083152564252 0.151132778893031\\
0.681205232354997 0.00516203039862346\\
0.681327312145741 -0.020934901061084\\
0.681449391936486 -0.0352738743905936\\
0.68157147172723 0.102953828505879\\
0.681693551517975 0.141955835962145\\
0.68181563130872 -0.0375681101233152\\
0.681937711099464 -0.293375394321767\\
0.682059790890209 -0.550329796386579\\
0.682181870680953 -0.385144823630628\\
0.682303950471698 0.160309721823917\\
0.682426030262443 0.596214511041009\\
0.682548110053187 0.706337826211643\\
0.682670189843932 0.568683682248351\\
0.682792269634677 0.0628047031832521\\
0.682914349425421 -0.632922282764554\\
0.683036429216166 -0.889876684829366\\
0.68315850900691 -0.550329796386579\\
0.683280588797655 -0.0375681101233152\\
0.6834026685884 0.458560367077717\\
0.683524748379144 0.596214511041009\\
0.683646828169889 0.421852595354173\\
0.683768907960634 0.107542299971322\\
0.683890987751378 -0.330083166045311\\
0.684013067542123 -0.348437051907083\\
0.684135147332867 -0.0891884141095498\\
0.684257227123612 0.169486664754804\\
0.684379306914357 0.0754229997132205\\
0.684501386705101 -0.0708345282477775\\
0.684623466495846 -0.102953828505879\\
0.684745546286591 -0.233725265271007\\
0.684867626077335 -0.112130771436765\\
0.68498970586808 0.233725265271007\\
0.685111785658824 0.458560367077717\\
0.685233865449569 0.197017493547462\\
0.685355945240314 -0.102953828505879\\
0.685478025031058 -0.385144823630628\\
0.685600104821803 -0.385144823630628\\
0.685722184612548 -0.0513335245196444\\
0.685844264403292 0.242902208201893\\
0.685966344194037 0.513622024663034\\
0.686088423984781 0.4034987094924\\
0.686210503775526 -0.0708345282477775\\
0.686332583566271 -0.632922282764554\\
0.686454663357015 -0.596214511041009\\
0.68657674314776 -0.279609979925437\\
0.686698822938504 0.17866360768569\\
0.686820902729249 0.669630054488099\\
0.686942982519994 0.706337826211643\\
0.687065062310738 0.348437051907083\\
0.687187142101483 -0.169486664754804\\
0.687309221892228 -0.632922282764554\\
0.687431301682972 -0.632922282764554\\
0.687553381473717 -0.261256094063665\\
0.687675461264462 0.102953828505879\\
0.687797541055206 0.261256094063665\\
0.687919620845951 0.366790937768856\\
0.688041700636695 0.197017493547462\\
0.68816378042744 -0.0662460567823344\\
0.688285860218185 0.0117579581301979\\
0.688407940008929 0.130484657298537\\
0.688530019799674 0.160309721823917\\
0.688652099590419 -0.0329796386578721\\
0.688774179381163 -0.311729280183539\\
0.688896259171908 -0.47691425293949\\
0.689018338962652 -0.421852595354173\\
0.689140418753397 -0.0220820189274448\\
0.689262498544142 0.348437051907083\\
0.689384578334886 0.743045597935188\\
0.689506658125631 0.596214511041009\\
0.689628737916375 0.0490392887869229\\
0.68975081770712 -0.440206481215945\\
0.689872897497865 -0.632922282764554\\
0.689994977288609 -0.495268138801262\\
0.690117057079354 -0.0937768855749928\\
0.690239136870099 0.458560367077717\\
0.690361216660843 0.550329796386579\\
0.690483296451588 0.270433036994551\\
0.690605376242332 -0.130484657298537\\
0.690727456033077 -0.421852595354173\\
0.690849535823822 -0.366790937768856\\
0.690971615614566 -0.0891884141095498\\
0.691093695405311 0.206194436478348\\
0.691215775196056 0.4034987094924\\
0.6913378549868 0.330083166045311\\
0.691459934777545 -0.0352738743905936\\
0.691582014568289 -0.348437051907083\\
0.691704094359034 -0.242902208201893\\
0.691826174149779 -0.0628047031832521\\
0.691948253940523 0.020934901061084\\
0.692070333731268 0.151132778893031\\
0.692192413522013 0.0444508173214798\\
0.692314493312757 -0.141955835962145\\
0.692436573103502 -0.17866360768569\\
0.692558652894246 0.0490392887869229\\
0.692680732684991 0.348437051907083\\
0.692802812475736 0.495268138801262\\
0.69292489226648 0.293375394321767\\
0.693046972057225 -0.22454832234012\\
0.69316905184797 -0.550329796386579\\
0.693291131638714 -0.706337826211643\\
0.693413211429459 -0.513622024663034\\
0.693535291220203 0.151132778893031\\
0.693657371010948 0.779753369658732\\
0.693779450801693 0.779753369658732\\
0.693901530592437 0.440206481215945\\
0.694023610383182 -0.0421565815887582\\
0.694145690173927 -0.568683682248351\\
0.694267769964671 -0.632922282764554\\
0.694389849755416 -0.440206481215945\\
0.69451192954616 0.0467450530542013\\
0.694634009336905 0.348437051907083\\
0.69475608912765 0.366790937768856\\
0.694878168918394 0.0467450530542013\\
0.695000248709139 -0.107542299971322\\
0.695122328499884 -0.028391167192429\\
0.695244408290628 -0.00401491253226269\\
0.695366488081373 0.187840550616576\\
0.695488567872117 0.17866360768569\\
0.695610647662862 0.000573558933180384\\
0.695732727453607 -0.366790937768856\\
0.695854807244351 -0.421852595354173\\
0.695976887035096 -0.141955835962145\\
0.69609896682584 0.187840550616576\\
0.696221046616585 0.495268138801262\\
0.69634312640733 0.385144823630628\\
0.696465206198074 0.0800114711786636\\
0.696587285988819 -0.330083166045311\\
0.696709365779564 -0.596214511041009\\
0.696831445570308 -0.330083166045311\\
0.696953525361053 0.206194436478348\\
0.697075605151797 0.632922282764554\\
0.697197684942542 0.513622024663034\\
0.697319764733287 0.17866360768569\\
0.697441844524031 -0.279609979925437\\
0.697563924314776 -0.743045597935188\\
0.697686004105521 -0.632922282764554\\
0.697808083896265 -0.169486664754804\\
0.69793016368701 0.421852595354173\\
0.698052243477755 0.632922282764554\\
0.698174323268499 0.495268138801262\\
0.698296403059244 0.17866360768569\\
0.698418482849988 -0.22454832234012\\
0.698540562640733 -0.4034987094924\\
0.698662642431478 -0.330083166045311\\
0.698784722222222 -0.0375681101233152\\
0.698906802012967 0.0662460567823344\\
0.699028881803711 -0.0800114711786636\\
0.699150961594456 -0.121307714367651\\
0.699273041385201 -0.0375681101233152\\
0.699395121175945 0.107542299971322\\
0.69951720096669 0.311729280183539\\
0.699639280757435 0.513622024663034\\
0.699761360548179 0.4034987094924\\
0.699883440338924 -0.151132778893031\\
0.700005520129668 -0.596214511041009\\
0.700127599920413 -0.743045597935188\\
0.700249679711158 -0.458560367077717\\
0.700371759501902 0.130484657298537\\
0.700493839292647 0.550329796386579\\
0.700615919083392 0.743045597935188\\
0.700737998874136 0.458560367077717\\
0.700860078664881 -0.0605104674505305\\
0.700982158455625 -0.596214511041009\\
0.70110423824637 -0.531975910524806\\
0.701226318037115 -0.169486664754804\\
0.701348397827859 0.151132778893031\\
0.701470477618604 0.440206481215945\\
0.701592557409349 0.293375394321767\\
0.701714637200093 0.0151993117292802\\
0.701836716990838 -0.330083166045311\\
0.701958796781582 -0.366790937768856\\
0.702080876572327 -0.141955835962145\\
0.702202956363072 0.215371379409234\\
0.702325036153816 0.348437051907083\\
0.702447115944561 0.0845999426441067\\
0.702569195735306 0.0754229997132205\\
0.70269127552605 -0.121307714367651\\
0.702813355316795 -0.206194436478348\\
0.702935435107539 -0.007456266131345\\
0.703057514898284 0.242902208201893\\
0.703179594689029 0.169486664754804\\
0.703301674479773 -0.17866360768569\\
0.703423754270518 -0.366790937768856\\
0.703545834061262 -0.440206481215945\\
0.703667913852007 -0.0983653570404359\\
0.703789993642752 0.330083166045311\\
0.703912073433496 0.669630054488099\\
0.704034153224241 0.706337826211643\\
0.704156233014986 0.270433036994551\\
0.70427831280573 -0.458560367077717\\
0.704400392596475 -0.889876684829366\\
0.70452247238722 -0.669630054488099\\
0.704644552177964 -0.293375394321767\\
0.704766631968709 0.293375394321767\\
0.704888711759453 0.706337826211643\\
0.705010791550198 0.669630054488099\\
0.705132871340943 0.187840550616576\\
0.705254951131687 -0.252079151132779\\
0.705377030922432 -0.366790937768856\\
0.705499110713177 -0.252079151132779\\
0.705621190503921 0.00114711786636077\\
0.705743270294666 0.0467450530542013\\
0.70586535008541 0.0174935474620017\\
0.705987429876155 -0.0754229997132205\\
0.7061095096669 -0.261256094063665\\
0.706231589457644 -0.141955835962145\\
0.706353669248389 0.279609979925437\\
0.706475749039133 0.531975910524806\\
0.706597828829878 0.311729280183539\\
0.706719908620623 0.0129050759965586\\
0.706841988411367 -0.293375394321767\\
0.706964068202112 -0.596214511041009\\
0.707086147992857 -0.4034987094924\\
0.707208227783601 0.0800114711786636\\
0.707330307574346 0.531975910524806\\
0.70745238736509 0.550329796386579\\
0.707574467155835 0.187840550616576\\
0.70769654694658 -0.293375394321767\\
0.707818626737324 -0.568683682248351\\
0.707940706528069 -0.458560367077717\\
0.708062786318814 -0.125896185833094\\
0.708184866109558 0.47691425293949\\
0.708306945900303 0.779753369658732\\
0.708429025691047 0.421852595354173\\
0.708551105481792 -0.112130771436765\\
0.708673185272537 -0.495268138801262\\
0.708795265063281 -0.596214511041009\\
0.708917344854026 -0.440206481215945\\
0.709039424644771 0.0140521938629194\\
0.709161504435515 0.366790937768856\\
0.70928358422626 0.440206481215945\\
0.709405664017004 0.242902208201893\\
0.709527743807749 -0.112130771436765\\
0.709649823598494 -0.0490392887869229\\
0.709771903389238 0.116719242902208\\
0.709893983179983 0.0845999426441067\\
0.710016062970728 -0.0513335245196444\\
0.710138142761472 -0.141955835962145\\
0.710260222552217 -0.4034987094924\\
0.710382302342961 -0.550329796386579\\
0.710504382133706 -0.17866360768569\\
0.710626461924451 0.385144823630628\\
0.710748541715195 0.779753369658732\\
0.71087062150594 0.706337826211643\\
0.710992701296684 0.242902208201893\\
0.711114781087429 -0.279609979925437\\
0.711236860878174 -0.706337826211643\\
0.711358940668918 -0.779753369658732\\
0.711481020459663 -0.385144823630628\\
0.711603100250408 0.4034987094924\\
0.711725180041152 0.669630054488099\\
0.711847259831897 0.4034987094924\\
0.711969339622642 0.135073128763981\\
0.712091419413386 -0.206194436478348\\
0.712213499204131 -0.421852595354173\\
0.712335578994875 -0.311729280183539\\
0.71245765878562 0.141955835962145\\
0.712579738576365 0.348437051907083\\
0.712701818367109 0.206194436478348\\
0.712823898157854 -0.0605104674505305\\
0.712945977948598 -0.293375394321767\\
0.713068057739343 -0.22454832234012\\
0.713190137530088 -0.0490392887869229\\
0.713312217320832 0.141955835962145\\
0.713434297111577 0.293375394321767\\
0.713556376902322 0.206194436478348\\
0.713678456693066 -0.206194436478348\\
0.713800536483811 -0.421852595354173\\
0.713922616274555 -0.0845999426441067\\
0.7140446960653 0.242902208201893\\
0.714166775856045 0.4034987094924\\
0.714288855646789 0.4034987094924\\
0.714410935437534 0.130484657298537\\
0.714533015228279 -0.440206481215945\\
0.714655095019023 -0.816461141382277\\
0.714777174809768 -0.550329796386579\\
0.714899254600512 -0.0117579581301979\\
0.715021334391257 0.669630054488099\\
0.715143414182002 0.816461141382277\\
0.715265493972746 0.568683682248351\\
0.715387573763491 0.121307714367651\\
0.715509653554236 -0.513622024663034\\
0.71563173334498 -0.816461141382277\\
0.715753813135725 -0.513622024663034\\
0.715875892926469 0.0266704903928879\\
0.715997972717214 0.242902208201893\\
0.716120052507959 0.311729280183539\\
0.716242132298703 0.270433036994551\\
0.716364212089448 0.0220820189274448\\
0.716486291880193 -0.0662460567823344\\
0.716608371670937 0.0444508173214798\\
0.716730451461682 0.187840550616576\\
0.716852531252426 0.17866360768569\\
0.716974611043171 -0.141955835962145\\
0.717096690833916 -0.440206481215945\\
0.71721877062466 -0.458560367077717\\
0.717340850415405 -0.187840550616576\\
0.717462930206149 0.151132778893031\\
0.717585009996894 0.513622024663034\\
0.717707089787639 0.743045597935188\\
0.717829169578383 0.22454832234012\\
0.717951249369128 -0.293375394321767\\
0.718073329159873 -0.632922282764554\\
0.718195408950617 -0.513622024663034\\
0.718317488741362 -0.116719242902208\\
0.718439568532107 0.385144823630628\\
0.718561648322851 0.632922282764554\\
0.718683728113596 0.330083166045311\\
0.71880580790434 -0.0186406653283625\\
0.718927887695085 -0.596214511041009\\
0.71904996748583 -0.568683682248351\\
0.719172047276574 -0.17866360768569\\
0.719294127067319 0.215371379409234\\
0.719416206858064 0.458560367077717\\
0.719538286648808 0.495268138801262\\
0.719660366439553 0.293375394321767\\
0.719782446230297 -0.311729280183539\\
0.719904526021042 -0.366790937768856\\
0.720026605811787 -0.242902208201893\\
0.720148685602531 -0.0513335245196444\\
0.720270765393276 0.0662460567823344\\
0.72039284518402 -0.0329796386578721\\
0.720514924974765 -0.0800114711786636\\
0.72063700476551 -0.107542299971322\\
0.720759084556254 0.0398623458560367\\
0.720881164346999 0.233725265271007\\
0.721003244137744 0.596214511041009\\
0.721125323928488 0.495268138801262\\
0.721247403719233 -0.135073128763981\\
0.721369483509977 -0.550329796386579\\
0.721491563300722 -0.743045597935188\\
0.721613643091467 -0.632922282764554\\
0.721735722882211 -0.0800114711786636\\
0.721857802672956 0.632922282764554\\
0.721979882463701 0.889876684829366\\
0.722101962254445 0.596214511041009\\
0.72222404204519 0.0891884141095498\\
0.722346121835935 -0.47691425293949\\
0.722468201626679 -0.632922282764554\\
0.722590281417424 -0.4034987094924\\
0.722712361208168 -0.0937768855749928\\
0.722834440998913 0.293375394321767\\
0.722956520789658 0.385144823630628\\
0.723078600580402 0.116719242902208\\
0.723200680371147 -0.187840550616576\\
0.723322760161891 -0.169486664754804\\
0.723444839952636 0.00172067679954115\\
0.723566919743381 0.121307714367651\\
0.723688999534125 0.279609979925437\\
0.72381107932487 0.0983653570404359\\
0.723933159115615 -0.151132778893031\\
0.724055238906359 -0.293375394321767\\
0.724177318697104 -0.311729280183539\\
0.724299398487848 0.0662460567823344\\
0.724421478278593 0.348437051907083\\
0.724543558069338 0.348437051907083\\
0.724665637860082 0.0255233725265271\\
0.724787717650827 -0.17866360768569\\
0.724909797441572 -0.47691425293949\\
0.725031877232316 -0.458560367077717\\
0.725153957023061 0.151132778893031\\
0.725276036813805 0.596214511041009\\
0.72539811660455 0.669630054488099\\
0.725520196395295 0.348437051907083\\
0.725642276186039 -0.169486664754804\\
0.725764355976784 -0.706337826211643\\
0.725886435767529 -0.779753369658732\\
0.726008515558273 -0.440206481215945\\
0.726130595349018 0.160309721823917\\
0.726252675139762 0.706337826211643\\
0.726374754930507 0.669630054488099\\
0.726496834721252 0.311729280183539\\
0.726618914511996 -0.121307714367651\\
0.726740994302741 -0.270433036994551\\
0.726863074093486 -0.348437051907083\\
0.72698515388423 -0.169486664754804\\
0.727107233674975 0.0536277602523659\\
0.727229313465719 -0.00401491253226269\\
0.727351393256464 -0.160309721823917\\
0.727473473047209 -0.252079151132779\\
0.727595552837953 -0.00114711786636077\\
0.727717632628698 0.293375394321767\\
0.727839712419442 0.550329796386579\\
0.727961792210187 0.47691425293949\\
0.728083872000932 0.130484657298537\\
0.728205951791676 -0.330083166045311\\
0.728328031582421 -0.816461141382277\\
0.728450111373166 -0.669630054488099\\
0.72857219116391 -0.0398623458560367\\
0.728694270954655 0.495268138801262\\
0.7288163507454 0.669630054488099\\
0.728938430536144 0.47691425293949\\
0.729060510326889 0.116719242902208\\
0.729182590117633 -0.513622024663034\\
0.729304669908378 -0.632922282764554\\
0.729426749699123 -0.279609979925437\\
0.729548829489867 0.22454832234012\\
0.729670909280612 0.550329796386579\\
0.729792989071356 0.348437051907083\\
0.729915068862101 0.0559219959850875\\
0.730037148652846 -0.348437051907083\\
0.73015922844359 -0.440206481215945\\
0.730281308234335 -0.330083166045311\\
0.73040338802508 0.0937768855749928\\
0.730525467815824 0.440206481215945\\
0.730647547606569 0.169486664754804\\
0.730769627397313 0.028391167192429\\
0.730891707188058 -0.0845999426441067\\
0.731013786978803 -0.112130771436765\\
0.731135866769547 0.028391167192429\\
0.731257946560292 0.22454832234012\\
0.731380026351037 0.242902208201893\\
0.731502106141781 -0.141955835962145\\
0.731624185932526 -0.4034987094924\\
0.73174626572327 -0.632922282764554\\
0.731868345514015 -0.293375394321767\\
0.73199042530476 0.311729280183539\\
0.732112505095504 0.669630054488099\\
0.732234584886249 0.779753369658732\\
0.732356664676994 0.47691425293949\\
0.732478744467738 -0.125896185833094\\
0.732600824258483 -0.889876684829366\\
0.732722904049227 -0.816461141382277\\
0.732844983839972 -0.421852595354173\\
0.732967063630717 0.102953828505879\\
0.733089143421461 0.596214511041009\\
0.733211223212206 0.596214511041009\\
0.733333303002951 0.385144823630628\\
0.733455382793695 -0.141955835962145\\
0.73357746258444 -0.311729280183539\\
0.733699542375184 -0.270433036994551\\
0.733821622165929 0.0444508173214798\\
0.733943701956674 0.141955835962145\\
0.734065781747418 -0.0444508173214798\\
0.734187861538163 -0.0754229997132205\\
0.734309941328907 -0.22454832234012\\
0.734432021119652 -0.242902208201893\\
0.734554100910397 0.0754229997132205\\
0.734676180701141 0.458560367077717\\
0.734798260491886 0.421852595354173\\
0.734920340282631 0.130484657298537\\
0.735042420073375 -0.233725265271007\\
0.73516449986412 -0.513622024663034\\
0.735286579654865 -0.385144823630628\\
0.735408659445609 0.0117579581301979\\
0.735530739236354 0.421852595354173\\
0.735652819027098 0.596214511041009\\
0.735774898817843 0.293375394321767\\
0.735896978608588 -0.279609979925437\\
0.736019058399332 -0.706337826211643\\
0.736141138190077 -0.513622024663034\\
0.736263217980822 -0.187840550616576\\
0.736385297771566 0.348437051907083\\
0.736507377562311 0.779753369658732\\
0.736629457353055 0.632922282764554\\
0.7367515371438 0.141955835962145\\
0.736873616934545 -0.440206481215945\\
0.736995696725289 -0.632922282764554\\
0.737117776516034 -0.513622024663034\\
0.737239856306778 -0.116719242902208\\
0.737361936097523 0.206194436478348\\
0.737484015888268 0.279609979925437\\
0.737606095679012 0.348437051907083\\
0.737728175469757 0.0490392887869229\\
0.737850255260502 -0.116719242902208\\
0.737972335051246 0.0891884141095498\\
0.738094414841991 0.279609979925437\\
0.738216494632735 0.135073128763981\\
0.73833857442348 -0.197017493547462\\
0.738460654214225 -0.4034987094924\\
0.738582734004969 -0.596214511041009\\
0.738704813795714 -0.366790937768856\\
0.738826893586459 0.160309721823917\\
0.738948973377203 0.669630054488099\\
0.739071053167948 0.816461141382277\\
0.739193132958692 0.440206481215945\\
0.739315212749437 -0.151132778893031\\
0.739437292540182 -0.632922282764554\\
0.739559372330926 -0.669630054488099\\
0.739681452121671 -0.47691425293949\\
0.739803531912416 0.135073128763981\\
0.73992561170316 0.632922282764554\\
0.740047691493905 0.47691425293949\\
0.740169771284649 0.112130771436765\\
0.740291851075394 -0.233725265271007\\
0.740413930866139 -0.366790937768856\\
0.740536010656883 -0.311729280183539\\
0.740658090447628 0.102953828505879\\
0.740780170238373 0.330083166045311\\
0.740902250029117 0.279609979925437\\
0.741024329819862 0.125896185833094\\
0.741146409610606 -0.261256094063665\\
0.741268489401351 -0.311729280183539\\
0.741390569192096 -0.121307714367651\\
0.74151264898284 0.116719242902208\\
0.741634728773585 0.141955835962145\\
0.741756808564329 0.112130771436765\\
0.741878888355074 -0.0536277602523659\\
0.742000968145819 -0.348437051907083\\
0.742123047936563 -0.160309721823917\\
0.742245127727308 0.187840550616576\\
0.742367207518053 0.513622024663034\\
0.742489287308797 0.495268138801262\\
0.742611367099542 0.116719242902208\\
0.742733446890287 -0.348437051907083\\
0.742855526681031 -0.743045597935188\\
0.742977606471776 -0.706337826211643\\
0.74309968626252 -0.293375394321767\\
0.743221766053265 0.458560367077717\\
0.74334384584401 0.926584456552911\\
0.743465925634754 0.706337826211643\\
0.743588005425499 0.270433036994551\\
0.743710085216243 -0.279609979925437\\
0.743832165006988 -0.596214511041009\\
0.743954244797733 -0.596214511041009\\
0.744076324588477 -0.233725265271007\\
0.744198404379222 0.160309721823917\\
0.744320484169967 0.270433036994551\\
0.744442563960711 0.215371379409234\\
0.744564643751456 0.0117579581301979\\
0.7446867235422 -0.0513335245196444\\
0.744808803332945 0.0467450530542013\\
0.74493088312369 0.197017493547462\\
0.745052962914434 0.261256094063665\\
0.745175042705179 0.0708345282477775\\
0.745297122495924 -0.261256094063665\\
0.745419202286668 -0.550329796386579\\
0.745541282077413 -0.4034987094924\\
0.745663361868157 -0.00401491253226269\\
0.745785441658902 0.348437051907083\\
0.745907521449647 0.596214511041009\\
0.746029601240391 0.4034987094924\\
0.746151681031136 -0.0536277602523659\\
0.746273760821881 -0.495268138801262\\
0.746395840612625 -0.568683682248351\\
0.74651792040337 -0.215371379409234\\
0.746640000194114 0.311729280183539\\
0.746762079984859 0.596214511041009\\
0.746884159775604 0.440206481215945\\
0.747006239566348 0.0605104674505305\\
0.747128319357093 -0.4034987094924\\
0.747250399147838 -0.743045597935188\\
0.747372478938582 -0.458560367077717\\
0.747494558729327 0.112130771436765\\
0.747616638520071 0.47691425293949\\
0.747738718310816 0.531975910524806\\
0.747860798101561 0.366790937768856\\
0.747982877892305 0.0174935474620017\\
0.74810495768305 -0.330083166045311\\
0.748227037473794 -0.330083166045311\\
0.748349117264539 -0.160309721823917\\
0.748471197055284 0.0708345282477775\\
0.748593276846028 0.0186406653283625\\
0.748715356636773 -0.206194436478348\\
0.748837436427518 -0.17866360768569\\
0.748959516218262 -0.0444508173214798\\
0.749081596009007 0.233725265271007\\
0.749203675799752 0.495268138801262\\
0.749325755590496 0.632922282764554\\
0.749447835381241 0.270433036994551\\
0.749569915171985 -0.385144823630628\\
0.74969199496273 -0.779753369658732\\
0.749814074753475 -0.816461141382277\\
0.749936154544219 -0.293375394321767\\
0.750058234334964 0.293375394321767\\
0.750180314125709 0.779753369658732\\
0.750302393916453 0.743045597935188\\
0.750424473707198 0.330083166045311\\
0.750546553497942 -0.311729280183539\\
0.750668633288687 -0.632922282764554\\
0.750790713079432 -0.421852595354173\\
0.750912792870176 -0.0937768855749928\\
0.751034872660921 0.270433036994551\\
0.751156952451665 0.348437051907083\\
0.75127903224241 0.206194436478348\\
0.751401112033155 -0.17866360768569\\
0.751523191823899 -0.366790937768856\\
0.751645271614644 -0.187840550616576\\
0.751767351405389 0.160309721823917\\
0.751889431196133 0.348437051907083\\
0.752011510986878 0.187840550616576\\
0.752133590777622 0.0151993117292802\\
0.752255670568367 -0.22454832234012\\
0.752377750359112 -0.293375394321767\\
0.752499830149856 -0.107542299971322\\
0.752621909940601 0.206194436478348\\
0.752743989731346 0.47691425293949\\
0.75286606952209 0.107542299971322\\
0.752988149312835 -0.270433036994551\\
0.75311022910358 -0.513622024663034\\
0.753232308894324 -0.440206481215945\\
0.753354388685069 -0.0490392887869229\\
0.753476468475813 0.458560367077717\\
0.753598548266558 0.853168913105822\\
0.753720628057303 0.596214511041009\\
0.753842707848047 0.0708345282477775\\
0.753964787638792 -0.669630054488099\\
0.754086867429536 -0.853168913105822\\
0.754208947220281 -0.568683682248351\\
0.754331027011026 -0.125896185833094\\
0.75445310680177 0.47691425293949\\
0.754575186592515 0.669630054488099\\
0.75469726638326 0.550329796386579\\
0.754819346174004 -0.0255233725265271\\
0.754941425964749 -0.233725265271007\\
0.755063505755493 -0.233725265271007\\
0.755185585546238 -0.112130771436765\\
0.755307665336983 0.00946372239747634\\
0.755429745127727 -0.0444508173214798\\
0.755551824918472 -0.121307714367651\\
0.755673904709217 -0.293375394321767\\
0.755795984499961 -0.215371379409234\\
0.755918064290706 0.0891884141095498\\
0.75604014408145 0.550329796386579\\
0.756162223872195 0.632922282764554\\
0.75628430366294 0.233725265271007\\
0.756406383453684 -0.187840550616576\\
0.756528463244429 -0.531975910524806\\
0.756650543035174 -0.669630054488099\\
0.756772622825918 -0.330083166045311\\
0.756894702616663 0.311729280183539\\
0.757016782407407 0.632922282764554\\
0.757138862198152 0.495268138801262\\
0.757260941988897 0.0983653570404359\\
0.757383021779641 -0.366790937768856\\
0.757505101570386 -0.568683682248351\\
0.757627181361131 -0.366790937768856\\
0.757749261151875 0.0662460567823344\\
0.75787134094262 0.513622024663034\\
0.757993420733364 0.632922282764554\\
0.758115500524109 0.197017493547462\\
0.758237580314854 -0.261256094063665\\
0.758359660105598 -0.458560367077717\\
0.758481739896343 -0.440206481215945\\
0.758603819687087 -0.187840550616576\\
0.758725899477832 0.17866360768569\\
0.758847979268577 0.330083166045311\\
0.758970059059321 0.17866360768569\\
0.759092138850066 0.00630914826498423\\
0.759214218640811 -0.0937768855749928\\
0.759336298431555 0.0513335245196444\\
0.7594583782223 0.293375394321767\\
0.759580458013045 0.22454832234012\\
0.759702537803789 -0.107542299971322\\
0.759824617594534 -0.330083166045311\\
0.759946697385278 -0.596214511041009\\
0.760068777176023 -0.632922282764554\\
0.760190856966768 0.0708345282477775\\
0.760312936757512 0.669630054488099\\
0.760435016548257 0.816461141382277\\
0.760557096339001 0.596214511041009\\
0.760679176129746 0.135073128763981\\
0.760801255920491 -0.550329796386579\\
0.760923335711235 -0.889876684829366\\
0.76104541550198 -0.596214511041009\\
0.761167495292725 -0.121307714367651\\
0.761289575083469 0.495268138801262\\
0.761411654874214 0.568683682248351\\
0.761533734664958 0.311729280183539\\
0.761655814455703 0.0306854029251506\\
0.761777894246448 -0.233725265271007\\
0.761899974037192 -0.311729280183539\\
0.762022053827937 -0.0937768855749928\\
0.762144133618682 0.330083166045311\\
0.762266213409426 0.125896185833094\\
0.762388293200171 -0.0891884141095498\\
0.762510372990915 -0.242902208201893\\
0.76263245278166 -0.279609979925437\\
0.762754532572405 -0.0662460567823344\\
0.762876612363149 0.206194436478348\\
0.762998692153894 0.421852595354173\\
0.763120771944639 0.252079151132779\\
0.763242851735383 -0.0490392887869229\\
0.763364931526128 -0.495268138801262\\
0.763487011316872 -0.440206481215945\\
0.763609091107617 0.0891884141095498\\
0.763731170898362 0.385144823630628\\
0.763853250689106 0.513622024663034\\
0.763975330479851 0.385144823630628\\
0.764097410270596 -0.0708345282477775\\
0.76421949006134 -0.706337826211643\\
0.764341569852085 -0.816461141382277\\
0.764463649642829 -0.330083166045311\\
0.764585729433574 0.261256094063665\\
0.764707809224319 0.779753369658732\\
0.764829889015063 0.743045597935188\\
0.764951968805808 0.385144823630628\\
0.765074048596552 -0.141955835962145\\
0.765196128387297 -0.632922282764554\\
0.765318208178042 -0.706337826211643\\
0.765440287968786 -0.270433036994551\\
0.765562367759531 0.169486664754804\\
0.765684447550276 0.135073128763981\\
0.76580652734102 0.233725265271007\\
0.765928607131765 0.141955835962145\\
0.76605068692251 -0.0628047031832521\\
0.766172766713254 0.0490392887869229\\
0.766294846503999 0.311729280183539\\
0.766416926294743 0.330083166045311\\
0.766539006085488 -0.0140521938629194\\
0.766661085876233 -0.366790937768856\\
0.766783165666977 -0.669630054488099\\
0.766905245457722 -0.513622024663034\\
0.767027325248467 -0.0605104674505305\\
0.767149405039211 0.4034987094924\\
0.767271484829956 0.816461141382277\\
0.7673935646207 0.669630054488099\\
0.767515644411445 0.0220820189274448\\
0.76763772420219 -0.531975910524806\\
0.767759803992934 -0.632922282764554\\
0.767881883783679 -0.421852595354173\\
0.768003963574423 0.00630914826498423\\
0.768126043365168 0.531975910524806\\
0.768248123155913 0.550329796386579\\
0.768370202946657 0.206194436478348\\
0.768492282737402 -0.22454832234012\\
0.768614362528147 -0.596214511041009\\
0.768736442318891 -0.348437051907083\\
0.768858522109636 0.020934901061084\\
0.76898060190038 0.330083166045311\\
0.769102681691125 0.366790937768856\\
0.76922476148187 0.348437051907083\\
0.769346841272614 0.0266704903928879\\
0.769468921063359 -0.421852595354173\\
0.769591000854104 -0.22454832234012\\
0.769713080644848 -0.0232291367938056\\
0.769835160435593 0.116719242902208\\
0.769957240226337 0.058216231717809\\
0.770079320017082 -0.141955835962145\\
0.770201399807827 -0.22454832234012\\
0.770323479598571 -0.169486664754804\\
0.770445559389316 0.135073128763981\\
0.770567639180061 0.4034987094924\\
0.770689718970805 0.743045597935188\\
0.77081179876155 0.366790937768856\\
0.770933878552294 -0.348437051907083\\
0.771055958343039 -0.743045597935188\\
0.771178038133784 -0.779753369658732\\
0.771300117924528 -0.458560367077717\\
0.771422197715273 0.112130771436765\\
0.771544277506018 0.889876684829366\\
0.771666357296762 0.853168913105822\\
0.771788437087507 0.4034987094924\\
0.771910516878251 -0.17866360768569\\
0.772032596668996 -0.596214511041009\\
0.772154676459741 -0.513622024663034\\
0.772276756250485 -0.252079151132779\\
0.77239883604123 0.107542299971322\\
0.772520915831974 0.233725265271007\\
0.772642995622719 0.293375394321767\\
0.772765075413464 -0.130484657298537\\
0.772887155204208 -0.270433036994551\\
0.773009234994953 0.0421565815887582\\
0.773131314785698 0.215371379409234\\
0.773253394576442 0.293375394321767\\
0.773375474367187 0.187840550616576\\
0.773497554157932 -0.058216231717809\\
0.773619633948676 -0.458560367077717\\
0.773741713739421 -0.4034987094924\\
0.773863793530165 -0.125896185833094\\
0.77398587332091 0.279609979925437\\
0.774107953111655 0.550329796386579\\
0.774230032902399 0.293375394321767\\
0.774352112693144 -0.112130771436765\\
0.774474192483888 -0.47691425293949\\
0.774596272274633 -0.531975910524806\\
0.774718352065378 -0.261256094063665\\
0.774840431856122 0.421852595354173\\
0.774962511646867 0.779753369658732\\
0.775084591437612 0.513622024663034\\
0.775206671228356 0.0983653570404359\\
0.775328751019101 -0.47691425293949\\
0.775450830809845 -0.816461141382277\\
0.77557291060059 -0.596214511041009\\
0.775694990391335 -0.0605104674505305\\
0.775817070182079 0.458560367077717\\
0.775939149972824 0.632922282764554\\
0.776061229763569 0.47691425293949\\
0.776183309554313 0.0106108402638371\\
0.776305389345058 -0.22454832234012\\
0.776427469135802 -0.187840550616576\\
0.776549548926547 -0.169486664754804\\
0.776671628717292 0.0174935474620017\\
0.776793708508036 -0.00946372239747634\\
0.776915788298781 -0.206194436478348\\
0.777037868089526 -0.366790937768856\\
0.77715994788027 -0.121307714367651\\
0.777282027671015 0.197017493547462\\
0.777404107461759 0.531975910524806\\
0.777526187252504 0.669630054488099\\
0.777648267043249 0.311729280183539\\
0.777770346833993 -0.17866360768569\\
0.777892426624738 -0.669630054488099\\
0.778014506415483 -0.816461141382277\\
0.778136586206227 -0.366790937768856\\
0.778258665996972 0.270433036994551\\
0.778380745787716 0.669630054488099\\
0.778502825578461 0.550329796386579\\
0.778624905369206 0.293375394321767\\
0.77874698515995 -0.197017493547462\\
0.778869064950695 -0.669630054488099\\
0.778991144741439 -0.385144823630628\\
0.779113224532184 0.007456266131345\\
0.779235304322929 0.4034987094924\\
0.779357384113673 0.421852595354173\\
0.779479463904418 0.160309721823917\\
0.779601543695163 -0.17866360768569\\
0.779723623485907 -0.440206481215945\\
0.779845703276652 -0.293375394321767\\
0.779967783067397 -0.0891884141095498\\
0.780089862858141 0.348437051907083\\
0.780211942648886 0.293375394321767\\
0.78033402243963 0.0106108402638371\\
0.780456102230375 -0.107542299971322\\
0.78057818202112 -0.121307714367651\\
0.780700261811864 0.0467450530542013\\
0.780822341602609 0.169486664754804\\
0.780944421393354 0.366790937768856\\
0.781066501184098 0.00344135359908231\\
0.781188580974843 -0.311729280183539\\
0.781310660765587 -0.632922282764554\\
0.781432740556332 -0.550329796386579\\
0.781554820347077 0.058216231717809\\
0.781676900137821 0.596214511041009\\
0.781798979928566 0.853168913105822\\
0.78192105971931 0.596214511041009\\
0.782043139510055 0.215371379409234\\
0.7821652193008 -0.596214511041009\\
0.782287299091544 -0.926584456552911\\
0.782409378882289 -0.596214511041009\\
0.782531458673034 -0.0754229997132205\\
0.782653538463778 0.348437051907083\\
0.782775618254523 0.550329796386579\\
0.782897698045267 0.495268138801262\\
0.783019777836012 0.0174935474620017\\
0.783141857626757 -0.197017493547462\\
0.783263937417501 -0.215371379409234\\
0.783386017208246 0.00860338399770576\\
0.783508096998991 0.187840550616576\\
0.783630176789735 -0.0444508173214798\\
0.78375225658048 -0.206194436478348\\
0.783874336371225 -0.279609979925437\\
0.783996416161969 -0.242902208201893\\
0.784118495952714 -0.0536277602523659\\
0.784240575743458 0.385144823630628\\
0.784362655534203 0.669630054488099\\
0.784484735324948 0.252079151132779\\
0.784606815115692 -0.151132778893031\\
0.784728894906437 -0.531975910524806\\
0.784850974697181 -0.495268138801262\\
0.784973054487926 -0.135073128763981\\
0.785095134278671 0.293375394321767\\
0.785217214069415 0.596214511041009\\
0.78533929386016 0.440206481215945\\
0.785461373650905 -0.00573558933180384\\
0.785583453441649 -0.669630054488099\\
0.785705533232394 -0.632922282764554\\
0.785827613023138 -0.22454832234012\\
0.785949692813883 0.197017493547462\\
0.786071772604628 0.568683682248351\\
0.786193852395372 0.669630054488099\\
0.786315932186117 0.348437051907083\\
0.786438011976862 -0.293375394321767\\
0.786560091767606 -0.568683682248351\\
0.786682171558351 -0.421852595354173\\
0.786804251349095 -0.0983653570404359\\
0.78692633113984 0.125896185833094\\
0.787048410930585 0.0891884141095498\\
0.787170490721329 0.160309721823917\\
0.787292570512074 0.0174935474620017\\
0.787414650302819 -0.107542299971322\\
0.787536730093563 0.116719242902208\\
0.787658809884308 0.495268138801262\\
0.787780889675052 0.421852595354173\\
0.787902969465797 -0.130484657298537\\
0.788025049256542 -0.458560367077717\\
0.788147129047286 -0.706337826211643\\
0.788269208838031 -0.596214511041009\\
0.788391288628776 -0.0937768855749928\\
0.78851336841952 0.568683682248351\\
0.788635448210265 0.963292228276455\\
0.788757528001009 0.632922282764554\\
0.788879607791754 0.0467450530542013\\
0.789001687582499 -0.596214511041009\\
0.789123767373243 -0.669630054488099\\
0.789245847163988 -0.440206481215945\\
0.789367926954733 -0.0536277602523659\\
0.789490006745477 0.458560367077717\\
0.789612086536222 0.513622024663034\\
0.789734166326966 0.151132778893031\\
0.789856246117711 -0.311729280183539\\
0.789978325908456 -0.293375394321767\\
0.7901004056992 -0.141955835962145\\
0.790222485489945 0.0708345282477775\\
0.79034456528069 0.348437051907083\\
0.790466645071434 0.22454832234012\\
0.790588724862179 0.0754229997132205\\
0.790710804652923 -0.270433036994551\\
0.790832884443668 -0.385144823630628\\
0.790954964234413 -0.0662460567823344\\
0.791077044025157 0.261256094063665\\
0.791199123815902 0.270433036994551\\
0.791321203606646 0.0197877831947233\\
0.791443283397391 -0.0845999426441067\\
0.791565363188136 -0.4034987094924\\
0.79168744297888 -0.366790937768856\\
0.791809522769625 0.116719242902208\\
0.79193160256037 0.596214511041009\\
0.792053682351114 0.669630054488099\\
0.792175762141859 0.261256094063665\\
0.792297841932603 -0.261256094063665\\
0.792419921723348 -0.743045597935188\\
0.792542001514093 -0.779753369658732\\
0.792664081304837 -0.440206481215945\\
0.792786161095582 0.242902208201893\\
0.792908240886327 0.889876684829366\\
0.793030320677071 0.779753369658732\\
0.793152400467816 0.279609979925437\\
0.793274480258561 -0.169486664754804\\
0.793396560049305 -0.421852595354173\\
0.79351863984005 -0.47691425293949\\
0.793640719630794 -0.270433036994551\\
0.793762799421539 0.0983653570404359\\
0.793884879212284 0.141955835962145\\
0.794006959003028 0.0329796386578721\\
0.794129038793773 -0.141955835962145\\
0.794251118584517 -0.0708345282477775\\
0.794373198375262 0.215371379409234\\
0.794495278166007 0.4034987094924\\
0.794617357956751 0.348437051907083\\
0.794739437747496 0.0891884141095498\\
0.794861517538241 -0.22454832234012\\
0.794983597328985 -0.743045597935188\\
0.79510567711973 -0.568683682248351\\
0.795227756910474 -0.0174935474620017\\
0.795349836701219 0.458560367077717\\
0.795471916491964 0.632922282764554\\
0.795593996282708 0.458560367077717\\
0.795716076073453 0.0375681101233152\\
0.795838155864198 -0.596214511041009\\
0.795960235654942 -0.596214511041009\\
0.796082315445687 -0.311729280183539\\
0.796204395236431 0.311729280183539\\
0.796326475027176 0.706337826211643\\
0.796448554817921 0.440206481215945\\
0.796570634608665 0.0708345282477775\\
0.79669271439941 -0.421852595354173\\
0.796814794190155 -0.596214511041009\\
0.796936873980899 -0.531975910524806\\
0.797058953771644 0.0891884141095498\\
0.797181033562388 0.47691425293949\\
0.797303113353133 0.385144823630628\\
0.797425193143878 0.270433036994551\\
0.797547272934622 -0.0662460567823344\\
0.797669352725367 -0.160309721823917\\
0.797791432516112 -0.169486664754804\\
0.797913512306856 0.0536277602523659\\
0.798035592097601 0.0983653570404359\\
0.798157671888345 -0.0662460567823344\\
0.79827975167909 -0.311729280183539\\
0.798401831469835 -0.550329796386579\\
0.798523911260579 -0.102953828505879\\
0.798645991051324 0.293375394321767\\
0.798768070842068 0.669630054488099\\
0.798890150632813 0.743045597935188\\
0.799012230423558 0.4034987094924\\
0.799134310214302 -0.293375394321767\\
0.799256390005047 -0.889876684829366\\
0.799378469795792 -0.816461141382277\\
0.799500549586536 -0.4034987094924\\
0.799622629377281 0.293375394321767\\
0.799744709168026 0.669630054488099\\
0.79986678895877 0.669630054488099\\
0.799988868749515 0.366790937768856\\
0.800110948540259 -0.215371379409234\\
0.800233028331004 -0.531975910524806\\
0.800355108121749 -0.366790937768856\\
0.800477187912493 0.0490392887869229\\
0.800599267703238 0.169486664754804\\
0.800721347493982 0.151132778893031\\
0.800843427284727 0.135073128763981\\
0.800965507075472 -0.215371379409234\\
0.801087586866216 -0.330083166045311\\
0.801209666656961 -0.0536277602523659\\
0.801331746447706 0.279609979925437\\
0.80145382623845 0.348437051907083\\
0.801575906029195 0.135073128763981\\
0.801697985819939 -0.135073128763981\\
0.801820065610684 -0.366790937768856\\
0.801942145401429 -0.22454832234012\\
0.802064225192173 -0.016346429595641\\
0.802186304982918 0.311729280183539\\
0.802308384773663 0.550329796386579\\
0.802430464564407 0.160309721823917\\
0.802552544355152 -0.385144823630628\\
0.802674624145896 -0.632922282764554\\
0.802796703936641 -0.513622024663034\\
0.802918783727386 -0.102953828505879\\
0.80304086351813 0.531975910524806\\
0.803162943308875 0.926584456552911\\
0.80328502309962 0.596214511041009\\
0.803407102890364 0.0662460567823344\\
0.803529182681109 -0.596214511041009\\
0.803651262471853 -0.853168913105822\\
0.803773342262598 -0.531975910524806\\
0.803895422053343 -0.0329796386578721\\
0.804017501844087 0.330083166045311\\
0.804139581634832 0.513622024663034\\
0.804261661425577 0.440206481215945\\
0.804383741216321 -0.0937768855749928\\
0.804505821007066 -0.160309721823917\\
0.80462790079781 -0.007456266131345\\
0.804749980588555 0.116719242902208\\
0.8048720603793 0.0891884141095498\\
0.804994140170044 -0.135073128763981\\
0.805116219960789 -0.330083166045311\\
0.805238299751533 -0.513622024663034\\
0.805360379542278 -0.22454832234012\\
0.805482459333023 0.135073128763981\\
0.805604539123767 0.669630054488099\\
0.805726618914512 0.816461141382277\\
0.805848698705257 0.233725265271007\\
0.805970778496001 -0.270433036994551\\
0.806092858286746 -0.632922282764554\\
0.80621493807749 -0.632922282764554\\
0.806337017868235 -0.348437051907083\\
0.80645909765898 0.421852595354173\\
0.806581177449724 0.706337826211643\\
0.806703257240469 0.348437051907083\\
0.806825337031214 0.0151993117292802\\
0.806947416821958 -0.440206481215945\\
0.807069496612703 -0.47691425293949\\
0.807191576403448 -0.270433036994551\\
0.807313656194192 0.187840550616576\\
0.807435735984937 0.47691425293949\\
0.807557815775681 0.513622024663034\\
0.807679895566426 0.135073128763981\\
0.807801975357171 -0.4034987094924\\
0.807924055147915 -0.348437051907083\\
0.80804613493866 -0.22454832234012\\
0.808168214729404 -0.0467450530542013\\
0.808290294520149 0.121307714367651\\
0.808412374310894 0.242902208201893\\
0.808534454101638 -0.0490392887869229\\
0.808656533892383 -0.215371379409234\\
0.808778613683128 -0.0255233725265271\\
0.808900693473872 0.233725265271007\\
0.809022773264617 0.513622024663034\\
0.809144853055361 0.330083166045311\\
0.809266932846106 -0.0754229997132205\\
0.809389012636851 -0.458560367077717\\
0.809511092427595 -0.706337826211643\\
0.80963317221834 -0.669630054488099\\
0.809755252009085 -0.0605104674505305\\
0.809877331799829 0.743045597935188\\
0.809999411590574 0.889876684829366\\
0.810121491381318 0.531975910524806\\
0.810243571172063 0.151132778893031\\
0.810365650962808 -0.440206481215945\\
0.810487730753552 -0.779753369658732\\
0.810609810544297 -0.531975910524806\\
0.810731890335042 -0.0662460567823344\\
0.810853970125786 0.330083166045311\\
0.810976049916531 0.421852595354173\\
0.811098129707275 0.187840550616576\\
0.81122020949802 -0.116719242902208\\
0.811342289288765 -0.102953828505879\\
0.811464369079509 -0.0708345282477775\\
0.811586448870254 0.0662460567823344\\
0.811708528660999 0.311729280183539\\
0.811830608451743 0.116719242902208\\
0.811952688242488 -0.233725265271007\\
0.812074768033232 -0.4034987094924\\
0.812196847823977 -0.270433036994551\\
0.812318927614722 0.00573558933180384\\
0.812441007405466 0.330083166045311\\
0.812563087196211 0.47691425293949\\
0.812685166986955 0.187840550616576\\
0.8128072467777 -0.151132778893031\\
0.812929326568445 -0.531975910524806\\
0.813051406359189 -0.47691425293949\\
0.813173486149934 0.0937768855749928\\
0.813295565940679 0.550329796386579\\
0.813417645731423 0.596214511041009\\
0.813539725522168 0.279609979925437\\
0.813661805312913 -0.0754229997132205\\
0.813783885103657 -0.743045597935188\\
0.813905964894402 -0.816461141382277\\
0.814028044685146 -0.293375394321767\\
0.814150124475891 0.293375394321767\\
0.814272204266636 0.706337826211643\\
0.81439428405738 0.632922282764554\\
0.814516363848125 0.311729280183539\\
0.81463844363887 -0.151132778893031\\
0.814760523429614 -0.440206481215945\\
0.814882603220359 -0.47691425293949\\
0.815004683011103 -0.17866360768569\\
0.815126762801848 0.160309721823917\\
0.815248842592593 -0.0117579581301979\\
0.815370922383337 -0.0983653570404359\\
0.815493002174082 -0.0536277602523659\\
0.815615081964826 0.0490392887869229\\
0.815737161755571 0.22454832234012\\
0.815859241546316 0.47691425293949\\
0.81598132133706 0.458560367077717\\
0.816103401127805 0.0513335245196444\\
0.81622548091855 -0.440206481215945\\
0.816347560709294 -0.816461141382277\\
0.816469640500039 -0.550329796386579\\
0.816591720290783 -0.0140521938629194\\
0.816713800081528 0.458560367077717\\
0.816835879872273 0.706337826211643\\
0.816957959663017 0.632922282764554\\
0.817080039453762 0.0937768855749928\\
0.817202119244507 -0.550329796386579\\
0.817324199035251 -0.632922282764554\\
0.817446278825996 -0.279609979925437\\
0.81756835861674 0.130484657298537\\
0.817690438407485 0.440206481215945\\
0.81781251819823 0.4034987094924\\
0.817934597988974 0.0708345282477775\\
0.818056677779719 -0.293375394321767\\
0.818178757570464 -0.458560367077717\\
0.818300837361208 -0.279609979925437\\
0.818422917151953 0.141955835962145\\
0.818544996942697 0.4034987094924\\
0.818667076733442 0.215371379409234\\
0.818789156524187 0.121307714367651\\
0.818911236314931 -0.0845999426441067\\
0.819033316105676 -0.252079151132779\\
0.819155395896421 -0.0937768855749928\\
0.819277475687165 0.151132778893031\\
0.81939955547791 0.206194436478348\\
0.819521635268654 -0.0891884141095498\\
0.819643715059399 -0.279609979925437\\
0.819765794850144 -0.458560367077717\\
0.819887874640888 -0.206194436478348\\
0.820009954431633 0.252079151132779\\
0.820132034222378 0.596214511041009\\
0.820254114013122 0.779753369658732\\
0.820376193803867 0.385144823630628\\
0.820498273594611 -0.279609979925437\\
0.820620353385356 -0.853168913105822\\
0.820742433176101 -0.779753369658732\\
0.820864512966845 -0.440206481215945\\
0.82098659275759 0.141955835962145\\
0.821108672548335 0.706337826211643\\
0.821230752339079 0.743045597935188\\
0.821352832129824 0.385144823630628\\
0.821474911920568 -0.17866360768569\\
0.821596991711313 -0.4034987094924\\
0.821719071502058 -0.348437051907083\\
0.821841151292802 -0.0662460567823344\\
0.821963231083547 0.0891884141095498\\
0.822085310874291 -0.0140521938629194\\
0.822207390665036 0.058216231717809\\
0.822329470455781 -0.215371379409234\\
0.822451550246525 -0.22454832234012\\
0.82257363003727 0.141955835962145\\
0.822695709828015 0.47691425293949\\
0.822817789618759 0.385144823630628\\
0.822939869409504 0.0444508173214798\\
0.823061949200248 -0.141955835962145\\
0.823184028990993 -0.550329796386579\\
0.823306108781738 -0.440206481215945\\
0.823428188572482 -0.0375681101233152\\
0.823550268363227 0.4034987094924\\
0.823672348153972 0.632922282764554\\
0.823794427944716 0.233725265271007\\
0.823916507735461 -0.215371379409234\\
0.824038587526206 -0.596214511041009\\
0.82416066731695 -0.47691425293949\\
0.824282747107695 -0.242902208201893\\
0.824404826898439 0.385144823630628\\
0.824526906689184 0.853168913105822\\
0.824648986479929 0.550329796386579\\
0.824771066270673 0.0467450530542013\\
0.824893146061418 -0.513622024663034\\
0.825015225852162 -0.632922282764554\\
0.825137305642907 -0.531975910524806\\
0.825259385433652 -0.0754229997132205\\
0.825381465224396 0.279609979925437\\
0.825503545015141 0.440206481215945\\
0.825625624805886 0.366790937768856\\
0.82574770459663 -0.0352738743905936\\
0.825869784387375 -0.116719242902208\\
0.825991864178119 0.0937768855749928\\
0.826113943968864 0.125896185833094\\
0.826236023759609 -0.0398623458560367\\
0.826358103550353 -0.141955835962145\\
0.826480183341098 -0.348437051907083\\
0.826602263131843 -0.531975910524806\\
0.826724342922587 -0.293375394321767\\
0.826846422713332 0.261256094063665\\
0.826968502504076 0.706337826211643\\
0.827090582294821 0.779753369658732\\
0.827212662085566 0.311729280183539\\
0.82733474187631 -0.215371379409234\\
0.827456821667055 -0.632922282764554\\
0.8275789014578 -0.779753369658732\\
0.827700981248544 -0.47691425293949\\
0.827823061039289 0.279609979925437\\
0.827945140830033 0.669630054488099\\
0.828067220620778 0.440206481215945\\
0.828189300411523 0.135073128763981\\
0.828311380202267 -0.187840550616576\\
0.828433459993012 -0.440206481215945\\
0.828555539783757 -0.311729280183539\\
0.828677619574501 0.0490392887869229\\
0.828799699365246 0.348437051907083\\
0.82892177915599 0.385144823630628\\
0.829043858946735 0.0708345282477775\\
0.82916593873748 -0.330083166045311\\
0.829288018528224 -0.311729280183539\\
0.829410098318969 -0.0983653570404359\\
0.829532178109713 0.0140521938629194\\
0.829654257900458 0.242902208201893\\
0.829776337691203 0.252079151132779\\
0.829898417481947 -0.0800114711786636\\
0.830020497272692 -0.330083166045311\\
0.830142577063437 -0.151132778893031\\
0.830264656854181 0.17866360768569\\
0.830386736644926 0.458560367077717\\
0.830508816435671 0.440206481215945\\
0.830630896226415 0.0329796386578721\\
0.83075297601716 -0.385144823630628\\
0.830875055807904 -0.743045597935188\\
0.830997135598649 -0.706337826211643\\
0.831119215389394 -0.0891884141095498\\
0.831241295180138 0.706337826211643\\
0.831363374970883 0.889876684829366\\
0.831485454761628 0.596214511041009\\
0.831607534552372 0.215371379409234\\
0.831729614343117 -0.440206481215945\\
0.831851694133861 -0.853168913105822\\
0.831973773924606 -0.596214511041009\\
0.832095853715351 -0.0937768855749928\\
0.832217933506095 0.293375394321767\\
0.83234001329684 0.366790937768856\\
0.832462093087584 0.293375394321767\\
0.832584172878329 -0.0106108402638371\\
0.832706252669074 -0.0306854029251506\\
0.832828332459818 -0.0398623458560367\\
0.832950412250563 0.130484657298537\\
0.833072492041308 0.242902208201893\\
0.833194571832052 0.00630914826498423\\
0.833316651622797 -0.4034987094924\\
0.833438731413541 -0.531975910524806\\
0.833560811204286 -0.206194436478348\\
0.833682890995031 0.0398623458560367\\
0.833804970785775 0.47691425293949\\
0.83392705057652 0.632922282764554\\
0.834049130367265 0.311729280183539\\
0.834171210158009 -0.215371379409234\\
0.834293289948754 -0.632922282764554\\
0.834415369739498 -0.568683682248351\\
0.834537449530243 -0.0754229997132205\\
0.834659529320988 0.47691425293949\\
0.834781609111732 0.531975910524806\\
0.834903688902477 0.385144823630628\\
0.835025768693222 0.0129050759965586\\
0.835147848483966 -0.632922282764554\\
0.835269928274711 -0.706337826211643\\
0.835392008065455 -0.233725265271007\\
0.8355140878562 0.242902208201893\\
0.835636167646945 0.513622024663034\\
0.835758247437689 0.568683682248351\\
0.835880327228434 0.311729280183539\\
0.836002407019179 -0.160309721823917\\
0.836124486809923 -0.440206481215945\\
0.836246566600668 -0.366790937768856\\
0.836368646391412 -0.0891884141095498\\
0.836490726182157 0.0983653570404359\\
0.836612805972902 -0.028391167192429\\
0.836734885763646 -0.112130771436765\\
0.836856965554391 -0.0255233725265271\\
0.836979045345135 0.0329796386578721\\
0.83710112513588 0.197017493547462\\
0.837223204926625 0.513622024663034\\
0.837345284717369 0.531975910524806\\
0.837467364508114 0.0232291367938056\\
0.837589444298859 -0.531975910524806\\
0.837711524089603 -0.779753369658732\\
0.837833603880348 -0.669630054488099\\
0.837955683671093 -0.112130771436765\\
0.838077763461837 0.47691425293949\\
0.838199843252582 0.853168913105822\\
0.838321923043326 0.706337826211643\\
0.838444002834071 0.160309721823917\\
0.838566082624816 -0.513622024663034\\
0.83868816241556 -0.669630054488099\\
0.838810242206305 -0.348437051907083\\
0.838932321997049 -0.112130771436765\\
0.839054401787794 0.293375394321767\\
0.839176481578539 0.4034987094924\\
0.839298561369283 0.197017493547462\\
0.839420641160028 -0.252079151132779\\
0.839542720950773 -0.311729280183539\\
0.839664800741517 -0.058216231717809\\
0.839786880532262 0.135073128763981\\
0.839908960323006 0.293375394321767\\
0.840031040113751 0.151132778893031\\
0.840153119904496 0.00688270719816461\\
0.84027519969524 -0.252079151132779\\
0.840397279485985 -0.330083166045311\\
0.84051935927673 -0.0662460567823344\\
0.840641439067474 0.330083166045311\\
0.840763518858219 0.366790937768856\\
0.840885598648963 -0.0266704903928879\\
0.841007678439708 -0.252079151132779\\
0.841129758230453 -0.421852595354173\\
0.841251838021197 -0.385144823630628\\
0.841373917811942 0.0662460567823344\\
0.841495997602687 0.596214511041009\\
0.841618077393431 0.816461141382277\\
0.841740157184176 0.366790937768856\\
0.84186223697492 -0.215371379409234\\
0.841984316765665 -0.743045597935188\\
0.84210639655641 -0.779753369658732\\
0.842228476347154 -0.458560367077717\\
0.842350556137899 0.0891884141095498\\
0.842472635928644 0.706337826211643\\
0.842594715719388 0.706337826211643\\
0.842716795510133 0.311729280183539\\
0.842838875300877 -0.121307714367651\\
0.842960955091622 -0.252079151132779\\
0.843083034882367 -0.270433036994551\\
0.843205114673111 -0.169486664754804\\
0.843327194463856 0.0197877831947233\\
0.8434492742546 -0.028391167192429\\
0.843571354045345 -0.0937768855749928\\
0.84369343383609 -0.270433036994551\\
0.843815513626834 -0.121307714367651\\
0.843937593417579 0.293375394321767\\
0.844059673208324 0.550329796386579\\
0.844181752999068 0.458560367077717\\
0.844303832789813 0.121307714367651\\
0.844425912580558 -0.22454832234012\\
0.844547992371302 -0.706337826211643\\
0.844670072162047 -0.669630054488099\\
0.844792151952791 -0.0845999426441067\\
0.844914231743536 0.47691425293949\\
0.845036311534281 0.632922282764554\\
0.845158391325025 0.385144823630628\\
0.84528047111577 0.0329796386578721\\
0.845402550906515 -0.458560367077717\\
0.845524630697259 -0.596214511041009\\
0.845646710488004 -0.293375394321767\\
0.845768790278748 0.270433036994551\\
0.845890870069493 0.706337826211643\\
0.846012949860238 0.4034987094924\\
0.846135029650982 0.0106108402638371\\
0.846257109441727 -0.348437051907083\\
0.846379189232471 -0.513622024663034\\
0.846501269023216 -0.440206481215945\\
0.846623348813961 -0.0375681101233152\\
0.846745428604705 0.440206481215945\\
0.84686750839545 0.348437051907083\\
0.846989588186195 0.151132778893031\\
0.847111667976939 -0.107542299971322\\
0.847233747767684 -0.0891884141095498\\
0.847355827558428 0.0197877831947233\\
0.847477907349173 0.130484657298537\\
0.847599987139918 0.135073128763981\\
0.847722066930662 -0.0754229997132205\\
0.847844146721407 -0.293375394321767\\
0.847966226512152 -0.669630054488099\\
0.848088306302896 -0.348437051907083\\
0.848210386093641 0.311729280183539\\
0.848332465884385 0.669630054488099\\
0.84845454567513 0.779753369658732\\
0.848576625465875 0.458560367077717\\
0.848698705256619 -0.0490392887869229\\
0.848820785047364 -0.743045597935188\\
0.848942864838109 -0.926584456552911\\
0.849064944628853 -0.513622024663034\\
0.849187024419598 0.169486664754804\\
0.849309104210342 0.706337826211643\\
0.849431184001087 0.513622024663034\\
0.849553263791832 0.311729280183539\\
0.849675343582576 -0.0490392887869229\\
0.849797423373321 -0.421852595354173\\
0.849919503164066 -0.440206481215945\\
0.85004158295481 0.0329796386578721\\
0.850163662745555 0.366790937768856\\
0.850285742536299 0.116719242902208\\
0.850407822327044 -0.00458847146544307\\
0.850529902117789 -0.261256094063665\\
0.850651981908533 -0.293375394321767\\
0.850774061699278 -0.0983653570404359\\
0.850896141490023 0.233725265271007\\
0.851018221280767 0.421852595354173\\
0.851140301071512 0.242902208201893\\
0.851262380862256 -0.107542299971322\\
0.851384460653001 -0.47691425293949\\
0.851506540443746 -0.252079151132779\\
0.85162862023449 0.112130771436765\\
0.851750700025235 0.293375394321767\\
0.85187277981598 0.495268138801262\\
0.851994859606724 0.261256094063665\\
0.852116939397469 -0.293375394321767\\
0.852239019188213 -0.743045597935188\\
0.852361098978958 -0.596214511041009\\
0.852483178769703 -0.102953828505879\\
0.852605258560447 0.458560367077717\\
0.852727338351192 0.853168913105822\\
0.852849418141936 0.632922282764554\\
0.852971497932681 0.270433036994551\\
0.853093577723426 -0.366790937768856\\
0.85321565751417 -0.816461141382277\\
0.853337737304915 -0.596214511041009\\
0.85345981709566 -0.135073128763981\\
0.853581896886404 0.252079151132779\\
0.853703976677149 0.270433036994551\\
0.853826056467893 0.385144823630628\\
0.853948136258638 0.107542299971322\\
0.854070216049383 -0.0891884141095498\\
0.854192295840127 0.028391167192429\\
0.854314375630872 0.141955835962145\\
0.854436455421617 0.17866360768569\\
0.854558535212361 -0.112130771436765\\
0.854680615003106 -0.348437051907083\\
0.854802694793851 -0.550329796386579\\
0.854924774584595 -0.279609979925437\\
0.85504685437534 0.058216231717809\\
0.855168934166084 0.458560367077717\\
0.855291013956829 0.816461141382277\\
0.855413093747574 0.421852595354173\\
0.855535173538318 -0.121307714367651\\
0.855657253329063 -0.550329796386579\\
0.855779333119807 -0.632922282764554\\
0.855901412910552 -0.385144823630628\\
0.856023492701297 0.160309721823917\\
0.856145572492041 0.632922282764554\\
0.856267652282786 0.458560367077717\\
0.856389732073531 0.17866360768569\\
0.856511811864275 -0.348437051907083\\
0.85663389165502 -0.550329796386579\\
0.856755971445764 -0.348437051907083\\
0.856878051236509 0.0513335245196444\\
0.857000131027254 0.366790937768856\\
0.857122210817998 0.47691425293949\\
0.857244290608743 0.385144823630628\\
0.857366370399488 -0.187840550616576\\
0.857488450190232 -0.4034987094924\\
0.857610529980977 -0.242902208201893\\
0.857732609771721 -0.0845999426441067\\
0.857854689562466 0.0398623458560367\\
0.857976769353211 0.0754229997132205\\
0.858098849143955 -0.0140521938629194\\
0.8582209289347 -0.160309721823917\\
0.858343008725445 -0.0891884141095498\\
0.858465088516189 0.187840550616576\\
0.858587168306934 0.495268138801262\\
0.858709248097678 0.596214511041009\\
0.858831327888423 0.0513335245196444\\
0.858953407679168 -0.440206481215945\\
0.859075487469912 -0.706337826211643\\
0.859197567260657 -0.779753369658732\\
0.859319647051402 -0.311729280183539\\
0.859441726842146 0.47691425293949\\
0.859563806632891 0.963292228276455\\
0.859685886423635 0.706337826211643\\
0.85980796621438 0.270433036994551\\
0.859930046005125 -0.330083166045311\\
0.860052125795869 -0.669630054488099\\
0.860174205586614 -0.531975910524806\\
0.860296285377358 -0.215371379409234\\
0.860418365168103 0.206194436478348\\
0.860540444958848 0.385144823630628\\
0.860662524749592 0.215371379409234\\
0.860784604540337 -0.151132778893031\\
0.860906684331082 -0.151132778893031\\
0.861028764121826 0.0329796386578721\\
0.861150843912571 0.141955835962145\\
0.861272923703316 0.252079151132779\\
0.86139500349406 0.121307714367651\\
0.861517083284805 -0.169486664754804\\
0.861639163075549 -0.4034987094924\\
0.861761242866294 -0.311729280183539\\
0.861883322657039 -0.0559219959850875\\
0.862005402447783 0.366790937768856\\
0.862127482238528 0.568683682248351\\
0.862249562029273 0.125896185833094\\
0.862371641820017 -0.233725265271007\\
0.862493721610762 -0.47691425293949\\
0.862615801401506 -0.495268138801262\\
0.862737881192251 -0.0329796386578721\\
0.862859960982996 0.550329796386579\\
0.86298204077374 0.743045597935188\\
0.863104120564485 0.385144823630628\\
0.863226200355229 -0.0754229997132205\\
0.863348280145974 -0.669630054488099\\
0.863470359936719 -0.779753369658732\\
0.863592439727463 -0.440206481215945\\
0.863714519518208 0.141955835962145\\
0.863836599308953 0.596214511041009\\
0.863958679099697 0.596214511041009\\
0.864080758890442 0.385144823630628\\
0.864202838681186 -0.116719242902208\\
0.864324918471931 -0.279609979925437\\
0.864446998262676 -0.242902208201893\\
0.86456907805342 -0.141955835962145\\
0.864691157844165 0.0421565815887582\\
0.86481323763491 -0.0467450530542013\\
0.864935317425654 -0.197017493547462\\
0.865057397216399 -0.293375394321767\\
0.865179477007143 -0.0398623458560367\\
0.865301556797888 0.311729280183539\\
0.865423636588633 0.568683682248351\\
0.865545716379377 0.596214511041009\\
0.865667796170122 0.187840550616576\\
0.865789875960867 -0.366790937768856\\
0.865911955751611 -0.779753369658732\\
0.866034035542356 -0.743045597935188\\
0.8661561153331 -0.197017493547462\\
0.866278195123845 0.458560367077717\\
0.86640027491459 0.669630054488099\\
0.866522354705334 0.568683682248351\\
0.866644434496079 0.197017493547462\\
0.866766514286824 -0.421852595354173\\
0.866888594077568 -0.632922282764554\\
0.867010673868313 -0.311729280183539\\
0.867132753659057 0.151132778893031\\
0.867254833449802 0.421852595354173\\
0.867376913240547 0.366790937768856\\
0.867498993031291 0.107542299971322\\
0.867621072822036 -0.293375394321767\\
0.86774315261278 -0.385144823630628\\
0.867865232403525 -0.311729280183539\\
0.86798731219427 0.0845999426441067\\
0.868109391985014 0.421852595354173\\
0.868231471775759 0.17866360768569\\
0.868353551566504 -0.0117579581301979\\
0.868475631357248 -0.135073128763981\\
0.868597711147993 -0.0937768855749928\\
0.868719790938738 -0.0352738743905936\\
0.868841870729482 0.242902208201893\\
0.868963950520227 0.330083166045311\\
0.869086030310971 -0.0559219959850875\\
0.869208110101716 -0.421852595354173\\
0.869330189892461 -0.632922282764554\\
0.869452269683205 -0.348437051907083\\
0.86957434947395 0.215371379409234\\
0.869696429264694 0.669630054488099\\
0.869818509055439 0.779753369658732\\
0.869940588846184 0.550329796386579\\
0.870062668636928 -0.0845999426441067\\
0.870184748427673 -0.853168913105822\\
0.870306828218418 -0.853168913105822\\
0.870428908009162 -0.421852595354173\\
0.870550987799907 0.0891884141095498\\
0.870673067590651 0.495268138801262\\
0.870795147381396 0.596214511041009\\
0.870917227172141 0.348437051907083\\
0.871039306962885 -0.0891884141095498\\
0.87116138675363 -0.279609979925437\\
0.871283466544375 -0.215371379409234\\
0.871405546335119 0.0754229997132205\\
0.871527626125864 0.141955835962145\\
0.871649705916608 -0.102953828505879\\
0.871771785707353 -0.206194436478348\\
0.871893865498098 -0.252079151132779\\
0.872015945288842 -0.242902208201893\\
0.872138025079587 0.0891884141095498\\
0.872260104870332 0.568683682248351\\
0.872382184661076 0.550329796386579\\
0.872504264451821 0.102953828505879\\
0.872626344242565 -0.270433036994551\\
0.87274842403331 -0.550329796386579\\
0.872870503824055 -0.47691425293949\\
0.872992583614799 -0.0306854029251506\\
0.873114663405544 0.440206481215945\\
0.873236743196289 0.568683682248351\\
0.873358822987033 0.348437051907083\\
0.873480902777778 -0.215371379409234\\
0.873602982568522 -0.669630054488099\\
0.873725062359267 -0.495268138801262\\
0.873847142150012 -0.17866360768569\\
0.873969221940756 0.279609979925437\\
0.874091301731501 0.743045597935188\\
0.874213381522245 0.632922282764554\\
0.87433546131299 0.102953828505879\\
0.874457541103735 -0.440206481215945\\
0.874579620894479 -0.550329796386579\\
0.874701700685224 -0.421852595354173\\
0.874823780475969 -0.125896185833094\\
0.874945860266713 0.197017493547462\\
0.875067940057458 0.206194436478348\\
0.875190019848203 0.252079151132779\\
0.875312099638947 -0.00229423573272154\\
0.875434179429692 -0.0891884141095498\\
0.875556259220436 0.215371379409234\\
0.875678339011181 0.330083166045311\\
0.875800418801926 0.141955835962145\\
0.87592249859267 -0.206194436478348\\
0.876044578383415 -0.4034987094924\\
0.87616665817416 -0.706337826211643\\
0.876288737964904 -0.47691425293949\\
0.876410817755649 0.17866360768569\\
0.876532897546393 0.743045597935188\\
0.876654977337138 0.926584456552911\\
0.876777057127883 0.458560367077717\\
0.876899136918627 -0.125896185833094\\
0.877021216709372 -0.669630054488099\\
0.877143296500116 -0.706337826211643\\
0.877265376290861 -0.531975910524806\\
0.877387456081606 0.116719242902208\\
0.87750953587235 0.669630054488099\\
0.877631615663095 0.47691425293949\\
0.87775369545384 0.0891884141095498\\
0.877875775244584 -0.279609979925437\\
0.877997855035329 -0.252079151132779\\
0.878119934826073 -0.261256094063665\\
0.878242014616818 0.0983653570404359\\
0.878364094407563 0.330083166045311\\
0.878486174198307 0.270433036994551\\
0.878608253989052 0.0255233725265271\\
0.878730333779797 -0.421852595354173\\
0.878852413570541 -0.261256094063665\\
0.878974493361286 0.0255233725265271\\
0.87909657315203 0.22454832234012\\
0.879218652942775 0.160309721823917\\
0.87934073273352 0.130484657298537\\
0.879462812524264 -0.121307714367651\\
0.879584892315009 -0.513622024663034\\
0.879706972105754 -0.206194436478348\\
0.879829051896498 0.293375394321767\\
0.879951131687243 0.632922282764554\\
0.880073211477987 0.513622024663034\\
0.880195291268732 0.0845999426441067\\
0.880317371059477 -0.366790937768856\\
0.880439450850221 -0.779753369658732\\
0.880561530640966 -0.779753369658732\\
0.880683610431711 -0.311729280183539\\
0.880805690222455 0.596214511041009\\
0.8809277700132 0.926584456552911\\
0.881049849803944 0.568683682248351\\
0.881171929594689 0.233725265271007\\
0.881294009385434 -0.187840550616576\\
0.881416089176178 -0.596214511041009\\
0.881538168966923 -0.632922282764554\\
0.881660248757668 -0.151132778893031\\
0.881782328548412 0.17866360768569\\
0.881904408339157 0.197017493547462\\
0.882026488129901 0.112130771436765\\
0.882148567920646 -0.0708345282477775\\
0.882270647711391 0.0536277602523659\\
0.882392727502135 0.215371379409234\\
0.88251480729288 0.197017493547462\\
0.882636887083625 0.22454832234012\\
0.882758966874369 0.0754229997132205\\
0.882881046665114 -0.366790937768856\\
0.883003126455858 -0.706337826211643\\
0.883125206246603 -0.330083166045311\\
0.883247286037348 0.125896185833094\\
0.883369365828092 0.421852595354173\\
0.883491445618837 0.596214511041009\\
0.883613525409581 0.366790937768856\\
0.883735605200326 -0.0891884141095498\\
0.883857684991071 -0.568683682248351\\
0.883979764781815 -0.568683682248351\\
0.88410184457256 -0.215371379409234\\
0.884223924363305 0.421852595354173\\
0.884346004154049 0.632922282764554\\
0.884468083944794 0.311729280183539\\
0.884590163735538 0.0628047031832521\\
0.884712243526283 -0.4034987094924\\
0.884834323317028 -0.706337826211643\\
0.884956403107772 -0.421852595354173\\
0.885078482898517 0.187840550616576\\
0.885200562689262 0.458560367077717\\
0.885322642480006 0.4034987094924\\
0.885444722270751 0.330083166045311\\
0.885566802061496 -0.0754229997132205\\
0.88568888185224 -0.279609979925437\\
0.885810961642985 -0.206194436478348\\
0.885933041433729 -0.0662460567823344\\
0.886055121224474 0.0754229997132205\\
0.886177201015219 -0.0352738743905936\\
0.886299280805963 -0.293375394321767\\
0.886421360596708 -0.293375394321767\\
0.886543440387452 0.020934901061084\\
0.886665520178197 0.270433036994551\\
0.886787599968942 0.550329796386579\\
0.886909679759686 0.706337826211643\\
0.887031759550431 0.215371379409234\\
0.887153839341176 -0.47691425293949\\
0.88727591913192 -0.816461141382277\\
0.887397998922665 -0.743045597935188\\
0.887520078713409 -0.311729280183539\\
0.887642158504154 0.421852595354173\\
0.887764238294899 0.779753369658732\\
0.887886318085643 0.632922282764554\\
0.888008397876388 0.293375394321767\\
0.888130477667133 -0.330083166045311\\
0.888252557457877 -0.632922282764554\\
0.888374637248622 -0.348437051907083\\
0.888496717039366 0.0329796386578721\\
0.888618796830111 0.22454832234012\\
0.888740876620856 0.261256094063665\\
0.8888629564116 0.130484657298537\\
0.888985036202345 -0.279609979925437\\
0.88910711599309 -0.293375394321767\\
0.889229195783834 -0.0329796386578721\\
0.889351275574579 0.233725265271007\\
0.889473355365323 0.348437051907083\\
0.889595435156068 0.151132778893031\\
0.889717514946813 -0.141955835962145\\
0.889839594737557 -0.330083166045311\\
0.889961674528302 -0.22454832234012\\
0.890083754319047 -0.0306854029251506\\
0.890205834109791 0.348437051907083\\
0.890327913900536 0.47691425293949\\
0.89044999369128 0.0444508173214798\\
0.890572073482025 -0.385144823630628\\
0.89069415327277 -0.550329796386579\\
0.890816233063514 -0.440206481215945\\
0.890938312854259 0.007456266131345\\
0.891060392645003 0.632922282764554\\
0.891182472435748 0.816461141382277\\
0.891304552226493 0.495268138801262\\
0.891426632017237 -0.0628047031832521\\
0.891548711807982 -0.669630054488099\\
0.891670791598727 -0.853168913105822\\
0.891792871389471 -0.47691425293949\\
0.891914951180216 -0.0106108402638371\\
0.892037030970961 0.458560367077717\\
0.892159110761705 0.632922282764554\\
0.89228119055245 0.366790937768856\\
0.892403270343194 -0.0375681101233152\\
0.892525350133939 -0.206194436478348\\
0.892647429924684 -0.0891884141095498\\
0.892769509715428 -0.0467450530542013\\
0.892891589506173 0.028391167192429\\
0.893013669296918 -0.0800114711786636\\
0.893135749087662 -0.279609979925437\\
0.893257828878407 -0.348437051907083\\
0.893379908669151 -0.197017493547462\\
0.893501988459896 0.197017493547462\\
0.893624068250641 0.632922282764554\\
0.893746148041385 0.706337826211643\\
0.89386822783213 0.151132778893031\\
0.893990307622874 -0.242902208201893\\
0.894112387413619 -0.632922282764554\\
0.894234467204364 -0.706337826211643\\
0.894356546995108 -0.22454832234012\\
0.894478626785853 0.366790937768856\\
0.894600706576598 0.669630054488099\\
0.894722786367342 0.440206481215945\\
0.894844866158087 0.0513335245196444\\
0.894966945948831 -0.458560367077717\\
0.895089025739576 -0.568683682248351\\
0.895211105530321 -0.270433036994551\\
0.895333185321065 0.151132778893031\\
0.89545526511181 0.568683682248351\\
0.895577344902555 0.531975910524806\\
0.895699424693299 0.141955835962145\\
0.895821504484044 -0.385144823630628\\
0.895943584274788 -0.458560367077717\\
0.896065664065533 -0.330083166045311\\
0.896187743856278 -0.0891884141095498\\
0.896309823647022 0.270433036994551\\
0.896431903437767 0.252079151132779\\
0.896553983228512 0.0937768855749928\\
0.896676063019256 -0.0891884141095498\\
0.896798142810001 -0.0559219959850875\\
0.896920222600745 0.107542299971322\\
0.89704230239149 0.330083166045311\\
0.897164382182235 0.311729280183539\\
0.897286461972979 -0.121307714367651\\
0.897408541763724 -0.440206481215945\\
0.897530621554469 -0.632922282764554\\
0.897652701345213 -0.568683682248351\\
0.897774781135958 0.0800114711786636\\
0.897896860926702 0.743045597935188\\
0.898018940717447 0.816461141382277\\
0.898141020508192 0.568683682248351\\
0.898263100298936 0.0983653570404359\\
0.898385180089681 -0.568683682248351\\
0.898507259880425 -0.853168913105822\\
0.89862933967117 -0.568683682248351\\
0.898751419461915 -0.0605104674505305\\
0.898873499252659 0.458560367077717\\
0.898995579043404 0.495268138801262\\
0.899117658834149 0.252079151132779\\
0.899239738624893 -0.0255233725265271\\
0.899361818415638 -0.169486664754804\\
0.899483898206383 -0.22454832234012\\
0.899605977997127 -0.00344135359908231\\
0.899728057787872 0.293375394321767\\
0.899850137578616 0.0983653570404359\\
0.899972217369361 -0.169486664754804\\
0.900094297160106 -0.279609979925437\\
0.90021637695085 -0.252079151132779\\
0.900338456741595 -0.0398623458560367\\
0.900460536532339 0.293375394321767\\
0.900582616323084 0.4034987094924\\
0.900704696113829 0.187840550616576\\
0.900826775904573 -0.107542299971322\\
0.900948855695318 -0.47691425293949\\
0.901070935486063 -0.4034987094924\\
0.901193015276807 0.121307714367651\\
0.901315095067552 0.458560367077717\\
0.901437174858296 0.47691425293949\\
0.901559254649041 0.293375394321767\\
0.901681334439786 -0.160309721823917\\
0.90180341423053 -0.743045597935188\\
0.901925494021275 -0.743045597935188\\
0.90204757381202 -0.242902208201893\\
0.902169653602764 0.366790937768856\\
0.902291733393509 0.743045597935188\\
0.902413813184253 0.669630054488099\\
0.902535892974998 0.293375394321767\\
0.902657972765743 -0.197017493547462\\
0.902780052556487 -0.531975910524806\\
0.902902132347232 -0.596214511041009\\
0.903024212137977 -0.169486664754804\\
0.903146291928721 0.151132778893031\\
0.903268371719466 0.0754229997132205\\
0.90339045151021 0.0891884141095498\\
0.903512531300955 0.112130771436765\\
0.9036346110917 0.0186406653283625\\
0.903756690882444 0.112130771436765\\
0.903878770673189 0.366790937768856\\
0.904000850463934 0.348437051907083\\
0.904122930254678 -0.0421565815887582\\
0.904245010045423 -0.440206481215945\\
0.904367089836167 -0.706337826211643\\
0.904489169626912 -0.440206481215945\\
0.904611249417657 -0.00114711786636077\\
0.904733329208401 0.421852595354173\\
0.904855408999146 0.779753369658732\\
0.90497748878989 0.632922282764554\\
0.905099568580635 -0.00401491253226269\\
0.90522164837138 -0.632922282764554\\
0.905343728162124 -0.568683682248351\\
0.905465807952869 -0.330083166045311\\
0.905587887743614 0.0983653570404359\\
0.905709967534358 0.47691425293949\\
0.905832047325103 0.440206481215945\\
0.905954127115848 0.160309721823917\\
0.906076206906592 -0.330083166045311\\
0.906198286697337 -0.495268138801262\\
0.906320366488081 -0.293375394321767\\
0.906442446278826 0.135073128763981\\
0.906564526069571 0.366790937768856\\
0.906686605860315 0.279609979925437\\
0.90680868565106 0.252079151132779\\
0.906930765441805 -0.107542299971322\\
0.907052845232549 -0.366790937768856\\
0.907174925023294 -0.151132778893031\\
0.907297004814038 0.0937768855749928\\
0.907419084604783 0.187840550616576\\
0.907541164395528 -0.0421565815887582\\
0.907663244186272 -0.206194436478348\\
0.907785323977017 -0.348437051907083\\
0.907907403767761 -0.187840550616576\\
0.908029483558506 0.22454832234012\\
0.908151563349251 0.550329796386579\\
0.908273643139995 0.779753369658732\\
0.90839572293074 0.293375394321767\\
0.908517802721485 -0.385144823630628\\
0.908639882512229 -0.816461141382277\\
0.908761962302974 -0.743045597935188\\
0.908884042093718 -0.440206481215945\\
0.909006121884463 0.242902208201893\\
0.909128201675208 0.853168913105822\\
0.909250281465952 0.743045597935188\\
0.909372361256697 0.330083166045311\\
0.909494441047442 -0.233725265271007\\
0.909616520838186 -0.440206481215945\\
0.909738600628931 -0.4034987094924\\
0.909860680419675 -0.151132778893031\\
0.90998276021042 0.0983653570404359\\
0.910104840001165 0.125896185833094\\
0.910226919791909 0.116719242902208\\
0.910348999582654 -0.261256094063665\\
0.910471079373399 -0.197017493547462\\
0.910593159164143 0.206194436478348\\
0.910715238954888 0.385144823630628\\
0.910837318745632 0.348437051907083\\
0.910959398536377 0.0800114711786636\\
0.911081478327122 -0.141955835962145\\
0.911203558117866 -0.568683682248351\\
0.911325637908611 -0.458560367077717\\
0.911447717699356 -0.0628047031832521\\
0.9115697974901 0.458560367077717\\
0.911691877280845 0.632922282764554\\
0.911813957071589 0.17866360768569\\
0.911936036862334 -0.17866360768569\\
0.912058116653079 -0.47691425293949\\
0.912180196443823 -0.550329796386579\\
0.912302276234568 -0.261256094063665\\
0.912424356025313 0.458560367077717\\
0.912546435816057 0.816461141382277\\
0.912668515606802 0.458560367077717\\
0.912790595397546 -0.00516203039862346\\
0.912912675188291 -0.495268138801262\\
0.913034754979036 -0.669630054488099\\
0.91315683476978 -0.531975910524806\\
0.913278914560525 -0.058216231717809\\
0.91340099435127 0.4034987094924\\
0.913523074142014 0.568683682248351\\
0.913645153932759 0.279609979925437\\
0.913767233723503 -0.130484657298537\\
0.913889313514248 -0.0754229997132205\\
0.914011393304993 0.0220820189274448\\
0.914133473095737 0.0375681101233152\\
0.914255552886482 0.0306854029251506\\
0.914377632677226 -0.112130771436765\\
0.914499712467971 -0.348437051907083\\
0.914621792258716 -0.568683682248351\\
0.91474387204946 -0.233725265271007\\
0.914865951840205 0.311729280183539\\
0.91498803163095 0.743045597935188\\
0.915110111421694 0.743045597935188\\
0.915232191212439 0.330083166045311\\
0.915354271003183 -0.197017493547462\\
0.915476350793928 -0.743045597935188\\
0.915598430584673 -0.853168913105822\\
0.915720510375417 -0.421852595354173\\
0.915842590166162 0.330083166045311\\
0.915964669956907 0.669630054488099\\
0.916086749747651 0.458560367077717\\
0.916208829538396 0.242902208201893\\
0.916330909329141 -0.169486664754804\\
0.916452989119885 -0.47691425293949\\
0.91657506891063 -0.348437051907083\\
0.916697148701374 0.0937768855749928\\
0.916819228492119 0.311729280183539\\
0.916941308282864 0.252079151132779\\
0.917063388073608 0.0129050759965586\\
0.917185467864353 -0.279609979925437\\
0.917307547655097 -0.261256094063665\\
0.917429627445842 -0.107542299971322\\
0.917551707236587 0.0754229997132205\\
0.917673787027331 0.366790937768856\\
0.917795866818076 0.242902208201893\\
0.917917946608821 -0.206194436478348\\
0.918040026399565 -0.348437051907083\\
0.91816210619031 -0.130484657298537\\
0.918284185981054 0.160309721823917\\
0.918406265771799 0.366790937768856\\
0.918528345562544 0.458560367077717\\
0.918650425353288 0.0937768855749928\\
0.918772505144033 -0.4034987094924\\
0.918894584934778 -0.743045597935188\\
0.919016664725522 -0.632922282764554\\
0.919138744516267 -0.00946372239747634\\
0.919260824307011 0.632922282764554\\
0.919382904097756 0.853168913105822\\
0.919504983888501 0.632922282764554\\
0.919627063679245 0.206194436478348\\
0.91974914346999 -0.513622024663034\\
0.919871223260735 -0.889876684829366\\
0.919993303051479 -0.550329796386579\\
0.920115382842224 -0.0490392887869229\\
0.920237462632968 0.252079151132779\\
0.920359542423713 0.348437051907083\\
0.920481622214458 0.348437051907083\\
0.920603702005202 0.058216231717809\\
0.920725781795947 -0.121307714367651\\
0.920847861586692 0.0232291367938056\\
0.920969941377436 0.151132778893031\\
0.921092021168181 0.197017493547462\\
0.921214100958925 -0.130484657298537\\
0.92133618074967 -0.4034987094924\\
0.921458260540415 -0.458560367077717\\
0.921580340331159 -0.252079151132779\\
0.921702420121904 0.0983653570404359\\
0.921824499912648 0.458560367077717\\
0.921946579703393 0.779753369658732\\
0.922068659494138 0.330083166045311\\
0.922190739284882 -0.270433036994551\\
0.922312819075627 -0.568683682248351\\
0.922434898866372 -0.531975910524806\\
0.922556978657116 -0.187840550616576\\
0.922679058447861 0.261256094063665\\
0.922801138238606 0.596214511041009\\
0.92292321802935 0.421852595354173\\
0.923045297820095 0.0266704903928879\\
0.923167377610839 -0.47691425293949\\
0.923289457401584 -0.596214511041009\\
0.923411537192329 -0.22454832234012\\
0.923533616983073 0.0983653570404359\\
0.923655696773818 0.421852595354173\\
0.923777776564563 0.550329796386579\\
0.923899856355307 0.330083166045311\\
0.924021936146052 -0.233725265271007\\
0.924144015936796 -0.421852595354173\\
0.924266095727541 -0.215371379409234\\
0.924388175518286 -0.102953828505879\\
0.92451025530903 0.0662460567823344\\
0.924632335099775 0.00860338399770576\\
0.924754414890519 -0.0513335245196444\\
0.924876494681264 -0.125896185833094\\
0.924998574472009 -0.0490392887869229\\
0.925120654262753 0.22454832234012\\
0.925242734053498 0.568683682248351\\
0.925364813844243 0.596214511041009\\
0.925486893634987 -0.0708345282477775\\
0.925608973425732 -0.513622024663034\\
0.925731053216476 -0.779753369658732\\
0.925853133007221 -0.706337826211643\\
0.925975212797966 -0.22454832234012\\
0.92609729258871 0.596214511041009\\
0.926219372379455 1\\
0.9263414521702 0.632922282764554\\
0.926463531960944 0.160309721823917\\
0.926585611751689 -0.458560367077717\\
0.926707691542433 -0.632922282764554\\
0.926829771333178 -0.458560367077717\\
0.926951851123923 -0.135073128763981\\
0.927073930914667 0.279609979925437\\
0.927196010705412 0.421852595354173\\
0.927318090496157 0.151132778893031\\
0.927440170286901 -0.233725265271007\\
0.927562250077646 -0.141955835962145\\
0.92768432986839 -0.0243762546601663\\
0.927806409659135 0.0937768855749928\\
0.92792848944988 0.270433036994551\\
0.928050569240624 0.141955835962145\\
0.928172649031369 -0.0937768855749928\\
0.928294728822114 -0.311729280183539\\
0.928416808612858 -0.293375394321767\\
0.928538888403603 -0.028391167192429\\
0.928660968194347 0.330083166045311\\
0.928783047985092 0.4034987094924\\
0.928905127775837 -0.0174935474620017\\
0.929027207566581 -0.160309721823917\\
0.929149287357326 -0.440206481215945\\
0.92927136714807 -0.47691425293949\\
0.929393446938815 0.0559219959850875\\
0.92951552672956 0.632922282764554\\
0.929637606520304 0.743045597935188\\
0.929759686311049 0.311729280183539\\
0.929881766101794 -0.125896185833094\\
0.930003845892538 -0.743045597935188\\
0.930125925683283 -0.853168913105822\\
0.930248005474028 -0.440206481215945\\
0.930370085264772 0.102953828505879\\
0.930492165055517 0.743045597935188\\
0.930614244846261 0.779753369658732\\
0.930736324637006 0.311729280183539\\
0.930858404427751 -0.135073128763981\\
0.930980484218495 -0.293375394321767\\
0.93110256400924 -0.385144823630628\\
0.931224643799984 -0.215371379409234\\
0.931346723590729 0.0983653570404359\\
0.931468803381474 0.00458847146544307\\
0.931590883172218 -0.160309721823917\\
0.931712962962963 -0.206194436478348\\
0.931835042753708 -0.0421565815887582\\
0.931957122544452 0.261256094063665\\
0.932079202335197 0.531975910524806\\
0.932201282125941 0.495268138801262\\
0.932323361916686 0.107542299971322\\
0.932445441707431 -0.270433036994551\\
0.932567521498175 -0.743045597935188\\
0.93268960128892 -0.743045597935188\\
0.932811681079665 -0.0845999426441067\\
0.932933760870409 0.458560367077717\\
0.933055840661154 0.632922282764554\\
0.933177920451898 0.513622024663034\\
0.933300000242643 0.141955835962145\\
0.933422080033388 -0.47691425293949\\
0.933544159824132 -0.632922282764554\\
0.933666239614877 -0.311729280183539\\
0.933788319405622 0.151132778893031\\
0.933910399196366 0.632922282764554\\
0.934032478987111 0.4034987094924\\
0.934154558777855 0.0559219959850875\\
0.9342766385686 -0.293375394321767\\
0.934398718359345 -0.458560367077717\\
0.934520798150089 -0.440206481215945\\
0.934642877940834 -0.0329796386578721\\
0.934764957731579 0.47691425293949\\
0.934887037522323 0.270433036994551\\
0.935009117313068 0.112130771436765\\
0.935131197103812 -0.0513335245196444\\
0.935253276894557 -0.112130771436765\\
0.935375356685302 -0.0129050759965586\\
0.935497436476046 0.151132778893031\\
0.935619516266791 0.197017493547462\\
0.935741596057535 -0.0937768855749928\\
0.93586367584828 -0.330083166045311\\
0.935985755639025 -0.669630054488099\\
0.936107835429769 -0.330083166045311\\
0.936229915220514 0.311729280183539\\
0.936351995011259 0.632922282764554\\
0.936474074802003 0.743045597935188\\
0.936596154592748 0.531975910524806\\
0.936718234383493 -0.0186406653283625\\
0.936840314174237 -0.853168913105822\\
0.936962393964982 -0.889876684829366\\
0.937084473755726 -0.458560367077717\\
0.937206553546471 0.0662460567823344\\
0.937328633337216 0.550329796386579\\
0.93745071312796 0.568683682248351\\
0.937572792918705 0.440206481215945\\
0.93769487270945 -0.028391167192429\\
0.937816952500194 -0.348437051907083\\
0.937939032290939 -0.348437051907083\\
0.938061112081683 0.0329796386578721\\
0.938183191872428 0.187840550616576\\
0.938305271663173 -0.0937768855749928\\
0.938427351453917 -0.0375681101233152\\
0.938549431244662 -0.187840550616576\\
0.938671511035406 -0.215371379409234\\
0.938793590826151 -0.020934901061084\\
0.938915670616896 0.421852595354173\\
0.93903775040764 0.440206481215945\\
0.939159830198385 0.0800114711786636\\
0.93928190998913 -0.197017493547462\\
0.939403989779874 -0.495268138801262\\
0.939526069570619 -0.279609979925437\\
0.939648149361363 0.0444508173214798\\
0.939770229152108 0.311729280183539\\
0.939892308942853 0.550329796386579\\
0.940014388733597 0.311729280183539\\
0.940136468524342 -0.261256094063665\\
0.940258548315087 -0.743045597935188\\
0.940380628105831 -0.513622024663034\\
0.940502707896576 -0.151132778893031\\
0.940624787687321 0.293375394321767\\
0.940746867478065 0.816461141382277\\
0.94086894726881 0.669630054488099\\
0.940991027059554 0.242902208201893\\
0.941113106850299 -0.421852595354173\\
0.941235186641044 -0.706337826211643\\
0.941357266431788 -0.550329796386579\\
0.941479346222533 -0.121307714367651\\
0.941601426013277 0.160309721823917\\
0.941723505804022 0.233725265271007\\
0.941845585594767 0.421852595354173\\
0.941967665385511 0.0421565815887582\\
0.942089745176256 -0.112130771436765\\
0.942211824967001 0.0628047031832521\\
0.942333904757745 0.311729280183539\\
0.94245598454849 0.169486664754804\\
0.942578064339234 -0.197017493547462\\
0.942700144129979 -0.366790937768856\\
0.942822223920724 -0.669630054488099\\
0.942944303711468 -0.348437051907083\\
0.943066383502213 0.0754229997132205\\
0.943188463292958 0.632922282764554\\
0.943310543083702 0.889876684829366\\
0.943432622874447 0.458560367077717\\
0.943554702665191 -0.151132778893031\\
0.943676782455936 -0.669630054488099\\
0.943798862246681 -0.596214511041009\\
0.943920942037425 -0.47691425293949\\
0.94404302182817 0.160309721823917\\
0.944165101618915 0.632922282764554\\
0.944287181409659 0.458560367077717\\
0.944409261200404 0.0845999426441067\\
0.944531340991148 -0.293375394321767\\
0.944653420781893 -0.421852595354173\\
0.944775500572638 -0.270433036994551\\
0.944897580363382 0.112130771436765\\
0.945019660154127 0.348437051907083\\
0.945141739944872 0.348437051907083\\
0.945263819735616 0.17866360768569\\
0.945385899526361 -0.233725265271007\\
0.945507979317105 -0.385144823630628\\
0.94563005910785 -0.0983653570404359\\
0.945752138898595 0.0352738743905936\\
0.945874218689339 0.121307714367651\\
0.945996298480084 0.102953828505879\\
0.946118378270828 -0.0628047031832521\\
0.946240458061573 -0.293375394321767\\
0.946362537852318 -0.141955835962145\\
0.946484617643062 0.22454832234012\\
0.946606697433807 0.495268138801262\\
0.946728777224552 0.568683682248351\\
0.946850857015296 0.112130771436765\\
0.946972936806041 -0.440206481215945\\
0.947095016596786 -0.743045597935188\\
0.94721709638753 -0.743045597935188\\
0.947339176178275 -0.311729280183539\\
0.947461255969019 0.531975910524806\\
0.947583335759764 0.963292228276455\\
0.947705415550509 0.632922282764554\\
0.947827495341253 0.242902208201893\\
0.947949575131998 -0.279609979925437\\
0.948071654922742 -0.632922282764554\\
0.948193734713487 -0.550329796386579\\
0.948315814504232 -0.160309721823917\\
0.948437894294976 0.206194436478348\\
0.948559974085721 0.261256094063665\\
0.948682053876466 0.151132778893031\\
0.94880413366721 -0.151132778893031\\
0.948926213457955 -0.0421565815887582\\
0.949048293248699 0.130484657298537\\
0.949170373039444 0.169486664754804\\
0.949292452830189 0.279609979925437\\
0.949414532620933 0.116719242902208\\
0.949536612411678 -0.242902208201893\\
0.949658692202423 -0.596214511041009\\
0.949780771993167 -0.330083166045311\\
0.949902851783912 0.0375681101233152\\
0.950024931574656 0.385144823630628\\
0.950147011365401 0.550329796386579\\
0.950269091156146 0.252079151132779\\
0.95039117094689 -0.0800114711786636\\
0.950513250737635 -0.513622024663034\\
0.95063533052838 -0.568683682248351\\
0.950757410319124 -0.141955835962145\\
0.950879490109869 0.531975910524806\\
0.951001569900613 0.632922282764554\\
0.951123649691358 0.330083166045311\\
0.951245729482103 0.0398623458560367\\
0.951367809272847 -0.495268138801262\\
0.951489889063592 -0.779753369658732\\
0.951611968854337 -0.47691425293949\\
0.951734048645081 0.160309721823917\\
0.951856128435826 0.531975910524806\\
0.95197820822657 0.550329796386579\\
0.952100288017315 0.330083166045311\\
0.95222236780806 -0.0605104674505305\\
0.952344447598804 -0.242902208201893\\
0.952466527389549 -0.279609979925437\\
0.952588607180293 -0.160309721823917\\
0.952710686971038 0.0800114711786636\\
0.952832766761783 0.0117579581301979\\
0.952954846552527 -0.311729280183539\\
0.953076926343272 -0.311729280183539\\
0.953199006134017 0.0329796386578721\\
0.953321085924761 0.293375394321767\\
0.953443165715506 0.531975910524806\\
0.953565245506251 0.632922282764554\\
0.953687325296995 0.279609979925437\\
0.95380940508774 -0.385144823630628\\
0.953931484878484 -0.853168913105822\\
0.954053564669229 -0.816461141382277\\
0.954175644459974 -0.22454832234012\\
0.954297724250718 0.421852595354173\\
0.954419804041463 0.706337826211643\\
0.954541883832208 0.669630054488099\\
0.954663963622952 0.311729280183539\\
0.954786043413697 -0.330083166045311\\
0.954908123204441 -0.706337826211643\\
0.955030202995186 -0.330083166045311\\
0.955152282785931 0.0605104674505305\\
0.955274362576675 0.279609979925437\\
0.95539644236742 0.293375394321767\\
0.955518522158164 0.151132778893031\\
0.955640601948909 -0.17866360768569\\
0.955762681739654 -0.366790937768856\\
0.955884761530398 -0.169486664754804\\
0.956006841321143 0.112130771436765\\
0.956128921111888 0.366790937768856\\
0.956251000902632 0.160309721823917\\
0.956373080693377 -0.0845999426441067\\
0.956495160484121 -0.151132778893031\\
0.956617240274866 -0.141955835962145\\
0.956739320065611 -0.0513335245196444\\
0.956861399856355 0.261256094063665\\
0.9569834796471 0.4034987094924\\
0.957105559437845 0.0106108402638371\\
0.957227639228589 -0.4034987094924\\
0.957349719019334 -0.596214511041009\\
0.957471798810078 -0.385144823630628\\
0.957593878600823 0.0983653570404359\\
0.957715958391568 0.596214511041009\\
0.957838038182312 0.816461141382277\\
0.957960117973057 0.596214511041009\\
0.958082197763802 -0.0129050759965586\\
0.958204277554546 -0.779753369658732\\
0.958326357345291 -0.853168913105822\\
0.958448437136035 -0.495268138801262\\
0.95857051692678 -0.0628047031832521\\
0.958692596717525 0.440206481215945\\
0.958814676508269 0.669630054488099\\
0.958936756299014 0.495268138801262\\
0.959058836089759 -0.020934901061084\\
0.959180915880503 -0.261256094063665\\
0.959302995671248 -0.187840550616576\\
0.959425075461992 -0.0536277602523659\\
0.959547155252737 0.0467450530542013\\
0.959669235043482 -0.151132778893031\\
0.959791314834226 -0.151132778893031\\
0.959913394624971 -0.197017493547462\\
0.960035474415716 -0.206194436478348\\
0.96015755420646 0.0983653570404359\\
0.960279633997205 0.550329796386579\\
0.960401713787949 0.632922282764554\\
0.960523793578694 0.0662460567823344\\
0.960645873369439 -0.242902208201893\\
0.960767953160183 -0.531975910524806\\
0.960890032950928 -0.531975910524806\\
0.961012112741673 -0.17866360768569\\
0.961134192532417 0.330083166045311\\
0.961256272323162 0.669630054488099\\
0.961378352113906 0.421852595354173\\
0.961500431904651 -0.0628047031832521\\
0.961622511695396 -0.568683682248351\\
0.96174459148614 -0.531975910524806\\
0.961866671276885 -0.270433036994551\\
0.961988751067629 0.102953828505879\\
0.962110830858374 0.632922282764554\\
0.962232910649119 0.669630054488099\\
0.962354990439863 0.197017493547462\\
0.962477070230608 -0.348437051907083\\
0.962599150021353 -0.513622024663034\\
0.962721229812097 -0.385144823630628\\
0.962843309602842 -0.187840550616576\\
0.962965389393586 0.160309721823917\\
0.963087469184331 0.242902208201893\\
0.963209548975076 0.233725265271007\\
0.96333162876582 0.00860338399770576\\
0.963453708556565 -0.116719242902208\\
0.96357578834731 0.160309721823917\\
0.963697868138054 0.348437051907083\\
0.963819947928799 0.22454832234012\\
0.963942027719543 -0.206194436478348\\
0.964064107510288 -0.385144823630628\\
0.964186187301033 -0.632922282764554\\
0.964308267091777 -0.550329796386579\\
0.964430346882522 0.058216231717809\\
0.964552426673267 0.669630054488099\\
0.964674506464011 0.926584456552911\\
0.964796586254756 0.531975910524806\\
0.9649186660455 0.0352738743905936\\
0.965040745836245 -0.568683682248351\\
0.96516282562699 -0.743045597935188\\
0.965284905417734 -0.596214511041009\\
0.965406985208479 -0.0891884141095498\\
0.965529064999224 0.531975910524806\\
0.965651144789968 0.550329796386579\\
0.965773224580713 0.233725265271007\\
0.965895304371457 -0.130484657298537\\
0.966017384162202 -0.206194436478348\\
0.966139463952947 -0.242902208201893\\
0.966261543743691 -0.0352738743905936\\
0.966383623534436 0.293375394321767\\
0.96650570332518 0.187840550616576\\
0.966627783115925 -0.000573558933180384\\
0.96674986290667 -0.311729280183539\\
0.966871942697414 -0.311729280183539\\
0.966994022488159 -0.0266704903928879\\
0.967116102278904 0.261256094063665\\
0.967238182069648 0.293375394321767\\
0.967360261860393 0.135073128763981\\
0.967482341651138 -0.0467450530542013\\
0.967604421441882 -0.458560367077717\\
0.967726501232627 -0.366790937768856\\
0.967848581023371 0.130484657298537\\
0.967970660814116 0.513622024663034\\
0.968092740604861 0.568683682248351\\
0.968214820395605 0.252079151132779\\
0.96833690018635 -0.187840550616576\\
0.968458979977095 -0.743045597935188\\
0.968581059767839 -0.743045597935188\\
0.968703139558584 -0.385144823630628\\
0.968825219349328 0.330083166045311\\
0.968947299140073 0.889876684829366\\
0.969069378930818 0.706337826211643\\
0.969191458721562 0.330083166045311\\
0.969313538512307 -0.197017493547462\\
0.969435618303051 -0.531975910524806\\
0.969557698093796 -0.632922282764554\\
0.969679777884541 -0.233725265271007\\
0.969801857675285 0.160309721823917\\
0.96992393746603 0.112130771436765\\
0.970046017256775 0.141955835962145\\
0.970168097047519 0.00286779466590192\\
0.970290176838264 0\\
0.970412256629008 0.169486664754804\\
0.970534336419753 0.311729280183539\\
0.970656416210498 0.311729280183539\\
0.970778496001242 0.0197877831947233\\
0.970900575791987 -0.330083166045311\\
0.971022655582732 -0.743045597935188\\
0.971144735373476 -0.513622024663034\\
0.971266815164221 0.0536277602523659\\
0.971388894954966 0.4034987094924\\
0.97151097474571 0.743045597935188\\
0.971633054536455 0.550329796386579\\
0.971755134327199 0.0243762546601663\\
0.971877214117944 -0.596214511041009\\
0.971999293908689 -0.669630054488099\\
0.972121373699433 -0.311729280183539\\
0.972243453490178 0.206194436478348\\
0.972365533280922 0.596214511041009\\
0.972487613071667 0.458560367077717\\
0.972609692862412 0.135073128763981\\
0.972731772653156 -0.348437051907083\\
0.972853852443901 -0.669630054488099\\
0.972975932234646 -0.421852595354173\\
0.97309801202539 0.151132778893031\\
0.973220091816135 0.440206481215945\\
0.973342171606879 0.348437051907083\\
0.973464251397624 0.293375394321767\\
0.973586331188369 0.00688270719816461\\
0.973708410979113 -0.348437051907083\\
0.973830490769858 -0.22454832234012\\
0.973952570560603 0.0174935474620017\\
0.974074650351347 0.151132778893031\\
0.974196730142092 0.00344135359908231\\
0.974318809932836 -0.293375394321767\\
0.974440889723581 -0.330083166045311\\
0.974562969514326 -0.125896185833094\\
0.97468504930507 0.252079151132779\\
0.974807129095815 0.531975910524806\\
0.97492920888656 0.779753369658732\\
0.975051288677304 0.385144823630628\\
0.975173368468049 -0.421852595354173\\
0.975295448258793 -0.853168913105822\\
0.975417528049538 -0.816461141382277\\
0.975539607840283 -0.366790937768856\\
0.975661687631027 0.242902208201893\\
0.975783767421772 0.779753369658732\\
0.975905847212517 0.779753369658732\\
0.976027927003261 0.385144823630628\\
0.976150006794006 -0.233725265271007\\
0.97627208658475 -0.596214511041009\\
0.976394166375495 -0.366790937768856\\
0.97651624616624 -0.0891884141095498\\
0.976638325956984 0.112130771436765\\
};
\addplot [
color=blue,
solid,
forget plot
]
table[row sep=crcr]{
0.976638325956984 0.112130771436765\\
0.976760405747729 0.215371379409234\\
0.976882485538473 0.197017493547462\\
0.977004565329218 -0.169486664754804\\
0.977126645119963 -0.293375394321767\\
0.977248724910707 -0.0186406653283625\\
0.977370804701452 0.261256094063665\\
0.977492884492197 0.311729280183539\\
0.977614964282941 0.151132778893031\\
0.977737044073686 -0.112130771436765\\
0.977859123864431 -0.348437051907083\\
0.977981203655175 -0.293375394321767\\
0.97810328344592 -0.102953828505879\\
0.978225363236664 0.311729280183539\\
0.978347443027409 0.495268138801262\\
0.978469522818154 0.197017493547462\\
0.978591602608898 -0.293375394321767\\
0.978713682399643 -0.495268138801262\\
0.978835762190387 -0.440206481215945\\
0.978957841981132 -0.17866360768569\\
0.979079921771877 0.47691425293949\\
0.979202001562621 0.779753369658732\\
0.979324081353366 0.596214511041009\\
0.979446161144111 0.00946372239747634\\
0.979568240934855 -0.531975910524806\\
0.9796903207256 -0.743045597935188\\
0.979812400516344 -0.568683682248351\\
0.979934480307089 -0.116719242902208\\
0.980056560097834 0.330083166045311\\
0.980178639888578 0.632922282764554\\
0.980300719679323 0.47691425293949\\
0.980422799470068 -0.00946372239747634\\
0.980544879260812 -0.169486664754804\\
0.980666959051557 -0.0983653570404359\\
0.980789038842301 -0.0708345282477775\\
0.980911118633046 -0.0421565815887582\\
0.981033198423791 -0.058216231717809\\
0.981155278214535 -0.22454832234012\\
0.98127735800528 -0.4034987094924\\
0.981399437796025 -0.206194436478348\\
0.981521517586769 0.197017493547462\\
0.981643597377514 0.596214511041009\\
0.981765677168258 0.669630054488099\\
0.981887756959003 0.293375394321767\\
0.982009836749748 -0.187840550616576\\
0.982131916540492 -0.596214511041009\\
0.982253996331237 -0.779753369658732\\
0.982376076121982 -0.4034987094924\\
0.982498155912726 0.330083166045311\\
0.982620235703471 0.632922282764554\\
0.982742315494215 0.495268138801262\\
0.98286439528496 0.141955835962145\\
0.982986475075705 -0.233725265271007\\
0.983108554866449 -0.550329796386579\\
0.983230634657194 -0.385144823630628\\
0.983352714447938 0.0628047031832521\\
0.983474794238683 0.421852595354173\\
0.983596874029428 0.495268138801262\\
0.983718953820172 0.0845999426441067\\
0.983841033610917 -0.22454832234012\\
0.983963113401662 -0.348437051907083\\
0.984085193192406 -0.270433036994551\\
0.984207272983151 -0.141955835962145\\
0.984329352773896 0.252079151132779\\
0.98445143256464 0.279609979925437\\
0.984573512355385 -0.0243762546601663\\
0.984695592146129 -0.135073128763981\\
0.984817671936874 -0.0937768855749928\\
0.984939751727619 0.169486664754804\\
0.985061831518363 0.311729280183539\\
0.985183911309108 0.311729280183539\\
0.985305991099853 -0.020934901061084\\
0.985428070890597 -0.348437051907083\\
0.985550150681342 -0.669630054488099\\
0.985672230472086 -0.669630054488099\\
0.985794310262831 0.0266704903928879\\
0.985916390053576 0.669630054488099\\
0.98603846984432 0.853168913105822\\
0.986160549635065 0.596214511041009\\
0.986282629425809 0.206194436478348\\
0.986404709216554 -0.47691425293949\\
0.986526789007299 -0.889876684829366\\
0.986648868798043 -0.632922282764554\\
0.986770948588788 -0.160309721823917\\
0.986893028379533 0.440206481215945\\
0.987015108170277 0.458560367077717\\
0.987137187961022 0.311729280183539\\
0.987259267751766 0.112130771436765\\
0.987381347542511 -0.151132778893031\\
0.987503427333256 -0.206194436478348\\
0.987625507124 -0.0845999426441067\\
0.987747586914745 0.311729280183539\\
0.98786966670549 0.0444508173214798\\
0.987991746496234 -0.206194436478348\\
0.988113826286979 -0.293375394321767\\
0.988235906077723 -0.242902208201893\\
0.988357985868468 -0.00229423573272154\\
0.988480065659213 0.242902208201893\\
0.988602145449957 0.513622024663034\\
0.988724225240702 0.252079151132779\\
0.988846305031447 -0.0708345282477775\\
0.988968384822191 -0.550329796386579\\
0.989090464612936 -0.440206481215945\\
0.98921254440368 0.0106108402638371\\
0.989334624194425 0.330083166045311\\
0.98945670398517 0.550329796386579\\
0.989578783775914 0.385144823630628\\
0.989700863566659 -0.007456266131345\\
0.989822943357404 -0.632922282764554\\
0.989945023148148 -0.743045597935188\\
0.990067102938893 -0.348437051907083\\
0.990189182729637 0.206194436478348\\
0.990311262520382 0.669630054488099\\
0.990433342311127 0.669630054488099\\
0.990555422101871 0.385144823630628\\
0.990677501892616 -0.112130771436765\\
0.990799581683361 -0.531975910524806\\
0.990921661474105 -0.596214511041009\\
0.99104374126485 -0.22454832234012\\
0.991165821055594 0.121307714367651\\
0.991287900846339 0.0513335245196444\\
0.991409980637084 0.121307714367651\\
0.991532060427828 0.125896185833094\\
0.991654140218573 -0.0151993117292802\\
0.991776220009318 0.0845999426441067\\
0.991898299800062 0.348437051907083\\
0.992020379590807 0.421852595354173\\
0.992142459381551 0.00114711786636077\\
0.992264539172296 -0.385144823630628\\
0.992386618963041 -0.706337826211643\\
0.992508698753785 -0.596214511041009\\
0.99263077854453 -0.0800114711786636\\
0.992752858335274 0.348437051907083\\
0.992874938126019 0.853168913105822\\
0.992997017916764 0.706337826211643\\
0.993119097707508 0.141955835962145\\
0.993241177498253 -0.531975910524806\\
0.993363257288998 -0.669630054488099\\
0.993485337079742 -0.4034987094924\\
0.993607416870487 -0.102953828505879\\
0.993729496661231 0.421852595354173\\
0.993851576451976 0.531975910524806\\
0.993973656242721 0.252079151132779\\
0.994095736033465 -0.233725265271007\\
0.99421781582421 -0.47691425293949\\
0.994339895614955 -0.279609979925437\\
0.994461975405699 0.0513335245196444\\
0.994584055196444 0.311729280183539\\
0.994706134987188 0.242902208201893\\
0.994828214777933 0.270433036994551\\
0.994950294568678 -0.0421565815887582\\
0.995072374359422 -0.421852595354173\\
0.995194454150167 -0.169486664754804\\
0.995316533940912 0.116719242902208\\
0.995438613731656 0.22454832234012\\
0.995560693522401 -0.0490392887869229\\
0.995682773313145 -0.125896185833094\\
0.99580485310389 -0.242902208201893\\
0.995926932894635 -0.330083166045311\\
0.996049012685379 0.0983653570404359\\
0.996171092476124 0.495268138801262\\
0.996293172266869 0.779753369658732\\
0.996415252057613 0.385144823630628\\
0.996537331848358 -0.252079151132779\\
0.996659411639102 -0.706337826211643\\
0.996781491429847 -0.779753369658732\\
0.996903571220592 -0.531975910524806\\
0.997025651011336 -0.00516203039862346\\
0.997147730802081 0.853168913105822\\
0.997269810592825 0.816461141382277\\
0.99739189038357 0.348437051907083\\
0.997513970174315 -0.0754229997132205\\
0.997636049965059 -0.385144823630628\\
0.997758129755804 -0.440206481215945\\
0.997880209546549 -0.348437051907083\\
0.998002289337293 0.028391167192429\\
0.998124369128038 0.151132778893031\\
0.998246448918783 0.151132778893031\\
0.998368528709527 -0.112130771436765\\
0.998490608500272 -0.206194436478348\\
0.998612688291016 0.187840550616576\\
0.998734768081761 0.311729280183539\\
0.998856847872506 0.242902208201893\\
0.99897892766325 0.116719242902208\\
0.999101007453995 -0.0140521938629194\\
0.99922308724474 -0.47691425293949\\
0.999345167035484 -0.596214511041009\\
0.999467246826229 -0.0444508173214798\\
0.999589326616973 0.311729280183539\\
0.999711406407718 0.531975910524806\\
0.999833486198463 0.330083166045311\\
0.999955565989207 -0.0106108402638371\\
1.00007764577995 -0.348437051907083\\
1.0001997255707 -0.596214511041009\\
1.00032180536144 -0.348437051907083\\
1.00044388515219 0.215371379409234\\
1.00056596494293 0.779753369658732\\
1.00068804473368 0.495268138801262\\
1.00081012452442 0.102953828505879\\
1.00093220431516 -0.233725265271007\\
1.00105428410591 -0.632922282764554\\
1.00117636389665 -0.669630054488099\\
1.0012984436874 -0.215371379409234\\
1.00142052347814 0.366790937768856\\
1.00154260326889 0.495268138801262\\
1.00166468305963 0.4034987094924\\
1.00178676285038 0.0983653570404359\\
1.00190884264112 -0.0937768855749928\\
1.00203092243187 -0.0800114711786636\\
1.00215300222261 -0.169486664754804\\
1.00227508201336 -0.0329796386578721\\
1.0023971618041 0.0232291367938056\\
1.00251924159484 -0.197017493547462\\
1.00264132138559 -0.47691425293949\\
1.00276340117633 -0.252079151132779\\
1.00288548096708 0.233725265271007\\
1.00300756075782 0.47691425293949\\
1.00312964054857 0.669630054488099\\
1.00325172033931 0.458560367077717\\
1.00337380013006 -0.0421565815887582\\
1.0034958799208 -0.632922282764554\\
1.00361795971155 -0.853168913105822\\
1.00374003950229 -0.568683682248351\\
1.00386211929304 0.130484657298537\\
1.00398419908378 0.632922282764554\\
1.00410627887452 0.550329796386579\\
1.00422835866527 0.4034987094924\\
1.00435043845601 -0.0306854029251506\\
1.00447251824676 -0.531975910524806\\
1.0045945980375 -0.495268138801262\\
1.00471667782825 -0.0662460567823344\\
1.00483875761899 0.270433036994551\\
1.00496083740974 0.348437051907083\\
1.00508291720048 0.197017493547462\\
1.00520499699123 -0.116719242902208\\
1.00532707678197 -0.293375394321767\\
1.00544915657272 -0.270433036994551\\
1.00557123636346 -0.0628047031832521\\
1.0056933161542 0.215371379409234\\
1.00581539594495 0.348437051907083\\
1.00593747573569 0.0151993117292802\\
1.00605955552644 -0.197017493547462\\
1.00618163531718 -0.135073128763981\\
1.00630371510793 -0.0375681101233152\\
1.00642579489867 0.17866360768569\\
1.00654787468942 0.348437051907083\\
1.00666995448016 0.242902208201893\\
1.00679203427091 -0.233725265271007\\
1.00691411406165 -0.550329796386579\\
1.0070361938524 -0.596214511041009\\
1.00715827364314 -0.233725265271007\\
1.00728035343388 0.4034987094924\\
1.00740243322463 0.743045597935188\\
1.00752451301537 0.743045597935188\\
1.00764659280612 0.330083166045311\\
1.00776867259686 -0.279609979925437\\
1.00789075238761 -0.853168913105822\\
1.00801283217835 -0.743045597935188\\
1.0081349119691 -0.279609979925437\\
1.00825699175984 0.206194436478348\\
1.00837907155059 0.513622024663034\\
1.00850115134133 0.495268138801262\\
1.00862323113208 0.206194436478348\\
1.00874531092282 -0.151132778893031\\
1.00886739071356 -0.169486664754804\\
1.00898947050431 -0.112130771436765\\
1.00911155029505 0.141955835962145\\
1.0092336300858 0.0513335245196444\\
1.00935570987654 -0.206194436478348\\
1.00947778966729 -0.293375394321767\\
1.00959986945803 -0.279609979925437\\
1.00972194924878 -0.116719242902208\\
1.00984402903952 0.293375394321767\\
1.00996610883027 0.669630054488099\\
1.01008818862101 0.385144823630628\\
1.01021026841176 -0.0467450530542013\\
1.0103323482025 -0.385144823630628\\
1.01045442799324 -0.596214511041009\\
1.01057650778399 -0.348437051907083\\
1.01069858757473 0.160309721823917\\
1.01082066736548 0.531975910524806\\
1.01094274715622 0.47691425293949\\
1.01106482694697 0.206194436478348\\
1.01118690673771 -0.385144823630628\\
1.01130898652846 -0.632922282764554\\
1.0114310663192 -0.348437051907083\\
1.01155314610995 -0.0352738743905936\\
1.01167522590069 0.421852595354173\\
1.01179730569144 0.669630054488099\\
1.01191938548218 0.458560367077717\\
1.01204146527293 -0.0800114711786636\\
1.01216354506367 -0.458560367077717\\
1.01228562485441 -0.458560367077717\\
1.01240770464516 -0.330083166045311\\
1.0125297844359 0.0467450530542013\\
1.01265186422665 0.151132778893031\\
1.01277394401739 0.135073128763981\\
1.01289602380814 0.141955835962145\\
1.01301810359888 -0.0490392887869229\\
1.01314018338963 0.0106108402638371\\
1.01326226318037 0.293375394321767\\
1.01338434297112 0.458560367077717\\
1.01350642276186 0.0117579581301979\\
1.01362850255261 -0.293375394321767\\
1.01375058234335 -0.531975910524806\\
1.01387266213409 -0.706337826211643\\
1.01399474192484 -0.293375394321767\\
1.01411682171558 0.330083166045311\\
1.01423890150633 0.816461141382277\\
1.01436098129707 0.779753369658732\\
1.01448306108782 0.366790937768856\\
1.01460514087856 -0.270433036994551\\
1.01472722066931 -0.669630054488099\\
1.01484930046005 -0.632922282764554\\
1.0149713802508 -0.4034987094924\\
1.01509346004154 0.22454832234012\\
1.01521553983229 0.596214511041009\\
1.01533761962303 0.330083166045311\\
1.01545969941377 -0.00946372239747634\\
1.01558177920452 -0.169486664754804\\
1.01570385899526 -0.252079151132779\\
1.01582593878601 -0.141955835962145\\
1.01594801857675 0.160309721823917\\
1.0160700983675 0.279609979925437\\
1.01619217815824 0.135073128763981\\
1.01631425794899 -0.0754229997132205\\
1.01643633773973 -0.330083166045311\\
1.01655841753048 -0.233725265271007\\
1.01668049732122 0.116719242902208\\
1.01680257711197 0.206194436478348\\
1.01692465690271 0.151132778893031\\
1.01704673669345 0.0800114711786636\\
1.0171688164842 -0.22454832234012\\
1.01729089627494 -0.495268138801262\\
1.01741297606569 -0.0983653570404359\\
1.01753505585643 0.348437051907083\\
1.01765713564718 0.632922282764554\\
1.01777921543792 0.47691425293949\\
1.01790129522867 0.0559219959850875\\
1.01802337501941 -0.495268138801262\\
1.01814545481016 -0.816461141382277\\
1.0182675346009 -0.669630054488099\\
1.01838961439165 -0.151132778893031\\
1.01851169418239 0.669630054488099\\
1.01863377397313 0.853168913105822\\
1.01875585376388 0.531975910524806\\
1.01887793355462 0.125896185833094\\
1.01900001334537 -0.311729280183539\\
1.01912209313611 -0.596214511041009\\
1.01924417292686 -0.495268138801262\\
1.0193662527176 -0.0129050759965586\\
1.01948833250835 0.135073128763981\\
1.01961041229909 0.112130771436765\\
1.01973249208984 0.0197877831947233\\
1.01985457188058 -0.0891884141095498\\
1.01997665167133 0.0845999426441067\\
1.02009873146207 0.270433036994551\\
1.02022081125281 0.366790937768856\\
1.02034289104356 0.242902208201893\\
1.0204649708343 -0.0754229997132205\\
1.02058705062505 -0.513622024663034\\
1.02070913041579 -0.706337826211643\\
1.02083121020654 -0.252079151132779\\
1.02095328999728 0.187840550616576\\
1.02107536978803 0.550329796386579\\
1.02119744957877 0.632922282764554\\
1.02131952936952 0.279609979925437\\
1.02144160916026 -0.233725265271007\\
1.02156368895101 -0.669630054488099\\
1.02168576874175 -0.47691425293949\\
1.02180784853249 -0.0662460567823344\\
1.02192992832324 0.513622024663034\\
1.02205200811398 0.550329796386579\\
1.02217408790473 0.252079151132779\\
1.02229616769547 -0.0375681101233152\\
1.02241824748622 -0.550329796386579\\
1.02254032727696 -0.632922282764554\\
1.02266240706771 -0.233725265271007\\
1.02278448685845 0.330083166045311\\
1.0229065666492 0.385144823630628\\
1.02302864643994 0.366790937768856\\
1.02315072623069 0.17866360768569\\
1.02327280602143 -0.135073128763981\\
1.02339488581217 -0.206194436478348\\
1.02351696560292 -0.121307714367651\\
1.02363904539366 0.058216231717809\\
1.02376112518441 0.0352738743905936\\
1.02388320497515 -0.130484657298537\\
1.0240052847659 -0.421852595354173\\
1.02412736455664 -0.270433036994551\\
1.02424944434739 0.0800114711786636\\
1.02437152413813 0.4034987094924\\
1.02449360392888 0.706337826211643\\
1.02461568371962 0.632922282764554\\
1.02473776351037 0.0536277602523659\\
1.02485984330111 -0.568683682248351\\
1.02498192309185 -0.889876684829366\\
1.0251040028826 -0.669630054488099\\
1.02522608267334 -0.121307714367651\\
1.02534816246409 0.513622024663034\\
1.02547024225483 0.743045597935188\\
1.02559232204558 0.568683682248351\\
1.02571440183632 0.197017493547462\\
1.02583648162707 -0.440206481215945\\
1.02595856141781 -0.550329796386579\\
1.02608064120856 -0.233725265271007\\
1.0262027209993 0.058216231717809\\
1.02632480079005 0.215371379409234\\
1.02644688058079 0.187840550616576\\
1.02656896037154 0.0662460567823344\\
1.02669104016228 -0.293375394321767\\
1.02681311995302 -0.22454832234012\\
1.02693519974377 0.0444508173214798\\
1.02705727953451 0.293375394321767\\
1.02717935932526 0.311729280183539\\
1.027301439116 0.00573558933180384\\
1.02742351890675 -0.206194436478348\\
1.02754559869749 -0.279609979925437\\
1.02766767848824 -0.151132778893031\\
1.02778975827898 0.0243762546601663\\
1.02791183806973 0.421852595354173\\
1.02803391786047 0.440206481215945\\
1.02815599765122 -0.116719242902208\\
1.02827807744196 -0.421852595354173\\
1.0284001572327 -0.550329796386579\\
1.02852223702345 -0.366790937768856\\
1.02864431681419 0.116719242902208\\
1.02876639660494 0.669630054488099\\
1.02888847639568 0.816461141382277\\
1.02901055618643 0.421852595354173\\
1.02913263597717 -0.112130771436765\\
1.02925471576792 -0.779753369658732\\
1.02937679555866 -0.743045597935188\\
1.02949887534941 -0.366790937768856\\
1.02962095514015 -0.007456266131345\\
1.0297430349309 0.458560367077717\\
1.02986511472164 0.596214511041009\\
1.02998719451238 0.348437051907083\\
1.03010927430313 -0.151132778893031\\
1.03023135409387 -0.0891884141095498\\
1.03035343388462 -0.0662460567823344\\
1.03047551367536 -0.0151993117292802\\
1.03059759346611 0.00630914826498423\\
1.03071967325685 -0.187840550616576\\
1.0308417530476 -0.311729280183539\\
1.03096383283834 -0.348437051907083\\
1.03108591262909 -0.0708345282477775\\
1.03120799241983 0.252079151132779\\
1.03133007221058 0.669630054488099\\
1.03145215200132 0.632922282764554\\
1.03157423179206 0.0559219959850875\\
1.03169631158281 -0.330083166045311\\
1.03181839137355 -0.632922282764554\\
1.0319404711643 -0.596214511041009\\
1.03206255095504 -0.169486664754804\\
1.03218463074579 0.47691425293949\\
1.03230671053653 0.632922282764554\\
1.03242879032728 0.348437051907083\\
1.03255087011802 -0.020934901061084\\
1.03267294990877 -0.458560367077717\\
1.03279502969951 -0.495268138801262\\
1.03291710949026 -0.22454832234012\\
1.033039189281 0.160309721823917\\
1.03316126907174 0.531975910524806\\
1.03328334886249 0.495268138801262\\
1.03340542865323 0.0232291367938056\\
1.03352750844398 -0.366790937768856\\
1.03364958823472 -0.330083166045311\\
1.03377166802547 -0.270433036994551\\
1.03389374781621 -0.0937768855749928\\
1.03401582760696 0.233725265271007\\
1.0341379073977 0.215371379409234\\
1.03425998718845 -0.00114711786636077\\
1.03438206697919 -0.112130771436765\\
1.03450414676994 -0.0197877831947233\\
1.03462622656068 0.215371379409234\\
1.03474830635142 0.385144823630628\\
1.03487038614217 0.197017493547462\\
1.03499246593291 -0.169486664754804\\
1.03511454572366 -0.385144823630628\\
1.0352366255144 -0.706337826211643\\
1.03535870530515 -0.568683682248351\\
1.03548078509589 0.206194436478348\\
1.03560286488664 0.743045597935188\\
1.03572494467738 0.816461141382277\\
1.03584702446813 0.495268138801262\\
1.03596910425887 0.0375681101233152\\
1.03609118404962 -0.550329796386579\\
1.03621326384036 -0.816461141382277\\
1.0363353436311 -0.550329796386579\\
1.03645742342185 -0.000573558933180384\\
1.03657950321259 0.47691425293949\\
1.03670158300334 0.421852595354173\\
1.03682366279408 0.206194436478348\\
1.03694574258483 0.0220820189274448\\
1.03706782237557 -0.151132778893031\\
1.03718990216632 -0.22454832234012\\
1.03731198195706 0.0329796386578721\\
1.03743406174781 0.279609979925437\\
1.03755614153855 0.0421565815887582\\
1.0376782213293 -0.215371379409234\\
1.03780030112004 -0.311729280183539\\
1.03792238091079 -0.187840550616576\\
1.03804446070153 0.0174935474620017\\
1.03816654049227 0.311729280183539\\
1.03828862028302 0.421852595354173\\
1.03841070007376 0.17866360768569\\
1.03853277986451 -0.197017493547462\\
1.03865485965525 -0.531975910524806\\
1.038776939446 -0.348437051907083\\
1.03889901923674 0.141955835962145\\
1.03902109902749 0.458560367077717\\
1.03914317881823 0.495268138801262\\
1.03926525860898 0.293375394321767\\
1.03938733839972 -0.160309721823917\\
1.03950941819047 -0.779753369658732\\
1.03963149798121 -0.706337826211643\\
1.03975357777195 -0.206194436478348\\
1.0398756575627 0.348437051907083\\
1.03999773735344 0.706337826211643\\
1.04011981714419 0.596214511041009\\
1.04024189693493 0.270433036994551\\
1.04036397672568 -0.187840550616576\\
1.04048605651642 -0.550329796386579\\
1.04060813630717 -0.550329796386579\\
1.04073021609791 -0.0891884141095498\\
1.04085229588866 0.130484657298537\\
1.0409743756794 -0.00286779466590192\\
1.04109645547015 0.0513335245196444\\
1.04121853526089 0.0937768855749928\\
1.04134061505163 -0.00344135359908231\\
1.04146269484238 0.197017493547462\\
1.04158477463312 0.440206481215945\\
1.04170685442387 0.330083166045311\\
1.04182893421461 -0.0628047031832521\\
1.04195101400536 -0.495268138801262\\
1.0420730937961 -0.743045597935188\\
1.04219517358685 -0.458560367077717\\
1.04231725337759 0.0605104674505305\\
1.04243933316834 0.47691425293949\\
1.04256141295908 0.816461141382277\\
1.04268349274983 0.596214511041009\\
1.04280557254057 -0.0662460567823344\\
1.04292765233131 -0.632922282764554\\
1.04304973212206 -0.568683682248351\\
1.0431718119128 -0.293375394321767\\
1.04329389170355 0.102953828505879\\
1.04341597149429 0.513622024663034\\
1.04353805128504 0.440206481215945\\
1.04366013107578 0.0845999426441067\\
1.04378221086653 -0.330083166045311\\
1.04390429065727 -0.47691425293949\\
1.04402637044802 -0.252079151132779\\
1.04414845023876 0.169486664754804\\
1.04427053002951 0.311729280183539\\
1.04439260982025 0.261256094063665\\
1.04451468961099 0.215371379409234\\
1.04463676940174 -0.112130771436765\\
1.04475884919248 -0.366790937768856\\
1.04488092898323 -0.0891884141095498\\
1.04500300877397 0.169486664754804\\
1.04512508856472 0.125896185833094\\
1.04524716835546 -0.0708345282477775\\
1.04536924814621 -0.242902208201893\\
1.04549132793695 -0.330083166045311\\
1.0456134077277 -0.187840550616576\\
1.04573548751844 0.242902208201893\\
1.04585756730919 0.568683682248351\\
1.04597964709993 0.743045597935188\\
1.04610172689067 0.311729280183539\\
1.04622380668142 -0.4034987094924\\
1.04634588647216 -0.779753369658732\\
1.04646796626291 -0.779753369658732\\
1.04659004605365 -0.421852595354173\\
1.0467121258444 0.187840550616576\\
1.04683420563514 0.816461141382277\\
1.04695628542589 0.779753369658732\\
1.04707836521663 0.311729280183539\\
1.04720044500738 -0.151132778893031\\
1.04732252479812 -0.495268138801262\\
1.04744460458887 -0.385144823630628\\
1.04756668437961 -0.160309721823917\\
1.04768876417035 0.0352738743905936\\
1.0478108439611 0.130484657298537\\
1.04793292375184 0.135073128763981\\
1.04805500354259 -0.141955835962145\\
1.04817708333333 -0.215371379409234\\
1.04829916312408 0.141955835962145\\
1.04842124291482 0.330083166045311\\
1.04854332270557 0.311729280183539\\
1.04866540249631 0.112130771436765\\
1.04878748228706 -0.141955835962145\\
1.0489095620778 -0.458560367077717\\
1.04903164186855 -0.440206481215945\\
1.04915372165929 -0.0662460567823344\\
1.04927580145004 0.293375394321767\\
1.04939788124078 0.596214511041009\\
1.04951996103152 0.242902208201893\\
1.04964204082227 -0.125896185833094\\
1.04976412061301 -0.440206481215945\\
1.04988620040376 -0.531975910524806\\
1.0500082801945 -0.233725265271007\\
1.05013035998525 0.348437051907083\\
1.05025243977599 0.779753369658732\\
1.05037451956674 0.47691425293949\\
1.05049659935748 0.0536277602523659\\
1.05061867914823 -0.4034987094924\\
1.05074075893897 -0.706337826211643\\
1.05086283872972 -0.550329796386579\\
1.05098491852046 -0.0983653570404359\\
1.0511069983112 0.385144823630628\\
1.05122907810195 0.531975910524806\\
1.05135115789269 0.4034987094924\\
1.05147323768344 0.016346429595641\\
1.05159531747418 -0.160309721823917\\
1.05171739726493 -0.0628047031832521\\
1.05183947705567 -0.112130771436765\\
1.05196155684642 0.000573558933180384\\
1.05208363663716 -0.058216231717809\\
1.05220571642791 -0.215371379409234\\
1.05232779621865 -0.440206481215945\\
1.0524498760094 -0.215371379409234\\
1.05257195580014 0.242902208201893\\
1.05269403559088 0.531975910524806\\
1.05281611538163 0.743045597935188\\
1.05293819517237 0.311729280183539\\
1.05306027496312 -0.116719242902208\\
1.05318235475386 -0.632922282764554\\
1.05330443454461 -0.816461141382277\\
1.05342651433535 -0.495268138801262\\
1.0535485941261 0.206194436478348\\
1.05367067391684 0.669630054488099\\
1.05379275370759 0.495268138801262\\
1.05391483349833 0.4034987094924\\
1.05403691328908 -0.160309721823917\\
1.05415899307982 -0.531975910524806\\
1.05428107287056 -0.458560367077717\\
1.05440315266131 -0.0352738743905936\\
1.05452523245205 0.348437051907083\\
1.0546473122428 0.348437051907083\\
1.05476939203354 0.197017493547462\\
1.05489147182429 -0.215371379409234\\
1.05501355161503 -0.293375394321767\\
1.05513563140578 -0.252079151132779\\
1.05525771119652 -0.0891884141095498\\
1.05537979098727 0.279609979925437\\
1.05550187077801 0.279609979925437\\
1.05562395056876 0.0232291367938056\\
1.0557460303595 -0.215371379409234\\
1.05586811015024 -0.0628047031832521\\
1.05599018994099 -0.0220820189274448\\
1.05611226973173 0.206194436478348\\
1.05623434952248 0.385144823630628\\
1.05635642931322 0.112130771436765\\
1.05647850910397 -0.261256094063665\\
1.05660058889471 -0.596214511041009\\
1.05672266868546 -0.596214511041009\\
1.0568447484762 -0.17866360768569\\
1.05696682826695 0.531975910524806\\
1.05708890805769 0.779753369658732\\
1.05721098784844 0.706337826211643\\
1.05733306763918 0.348437051907083\\
1.05745514742992 -0.4034987094924\\
1.05757722722067 -0.889876684829366\\
1.05769930701141 -0.669630054488099\\
1.05782138680216 -0.22454832234012\\
1.0579434665929 0.215371379409234\\
1.05806554638365 0.550329796386579\\
1.05818762617439 0.47691425293949\\
1.05830970596514 0.141955835962145\\
1.05843178575588 -0.121307714367651\\
1.05855386554663 -0.187840550616576\\
1.05867594533737 -0.0937768855749928\\
1.05879802512812 0.160309721823917\\
1.05892010491886 0.0375681101233152\\
1.0590421847096 -0.233725265271007\\
1.05916426450035 -0.252079151132779\\
1.05928634429109 -0.261256094063665\\
1.05940842408184 -0.0937768855749928\\
1.05953050387258 0.293375394321767\\
1.05965258366333 0.596214511041009\\
1.05977466345407 0.348437051907083\\
1.05989674324482 -0.0708345282477775\\
1.06001882303556 -0.366790937768856\\
1.06014090282631 -0.568683682248351\\
1.06026298261705 -0.261256094063665\\
1.0603850624078 0.169486664754804\\
1.06050714219854 0.495268138801262\\
1.06062922198928 0.513622024663034\\
1.06075130178003 0.160309721823917\\
1.06087338157077 -0.458560367077717\\
1.06099546136152 -0.669630054488099\\
1.06111754115226 -0.366790937768856\\
1.06123962094301 -0.00229423573272154\\
1.06136170073375 0.440206481215945\\
1.0614837805245 0.706337826211643\\
1.06160586031524 0.47691425293949\\
1.06172794010599 -0.112130771436765\\
1.06185001989673 -0.47691425293949\\
1.06197209968748 -0.495268138801262\\
1.06209417947822 -0.293375394321767\\
1.06221625926897 0.0513335245196444\\
1.06233833905971 0.135073128763981\\
1.06246041885045 0.151132778893031\\
1.0625824986412 0.125896185833094\\
1.06270457843194 -0.0398623458560367\\
1.06282665822269 0.0117579581301979\\
1.06294873801343 0.311729280183539\\
1.06307081780418 0.440206481215945\\
1.06319289759492 0.00458847146544307\\
1.06331497738567 -0.330083166045311\\
1.06343705717641 -0.531975910524806\\
1.06355913696716 -0.669630054488099\\
1.0636812167579 -0.270433036994551\\
1.06380329654865 0.348437051907083\\
1.06392537633939 0.779753369658732\\
1.06404745613013 0.779753369658732\\
1.06416953592088 0.311729280183539\\
1.06429161571162 -0.330083166045311\\
1.06441369550237 -0.669630054488099\\
1.06453577529311 -0.568683682248351\\
1.06465785508386 -0.311729280183539\\
1.0647799348746 0.252079151132779\\
1.06490201466535 0.568683682248351\\
1.06502409445609 0.330083166045311\\
1.06514617424684 -0.0513335245196444\\
1.06526825403758 -0.252079151132779\\
1.06539033382833 -0.279609979925437\\
1.06551241361907 -0.112130771436765\\
1.06563449340981 0.233725265271007\\
1.06575657320056 0.293375394321767\\
1.0658786529913 0.141955835962145\\
1.06600073278205 -0.0662460567823344\\
1.06612281257279 -0.348437051907083\\
1.06624489236354 -0.233725265271007\\
1.06636697215428 0.0845999426441067\\
1.06648905194503 0.215371379409234\\
1.06661113173577 0.141955835962145\\
1.06673321152652 0.00860338399770576\\
1.06685529131726 -0.252079151132779\\
1.06697737110801 -0.4034987094924\\
1.06709945089875 -0.0306854029251506\\
1.06722153068949 0.4034987094924\\
1.06734361048024 0.596214511041009\\
1.06746569027098 0.458560367077717\\
1.06758777006173 -0.0490392887869229\\
1.06770984985247 -0.550329796386579\\
1.06783192964322 -0.816461141382277\\
1.06795400943396 -0.596214511041009\\
1.06807608922471 -0.0662460567823344\\
1.06819816901545 0.669630054488099\\
1.0683202488062 0.853168913105822\\
1.06844232859694 0.495268138801262\\
1.06856440838769 0.141955835962145\\
1.06868648817843 -0.366790937768856\\
1.06880856796917 -0.632922282764554\\
1.06893064775992 -0.458560367077717\\
1.06905272755066 -0.0398623458560367\\
1.06917480734141 0.160309721823917\\
1.06929688713215 0.151132778893031\\
1.0694189669229 0.0983653570404359\\
1.06954104671364 -0.0559219959850875\\
1.06966312650439 0.0513335245196444\\
1.06978520629513 0.206194436478348\\
1.06990728608588 0.252079151132779\\
1.07002936587662 0.215371379409234\\
1.07015144566737 -0.0708345282477775\\
1.07027352545811 -0.440206481215945\\
1.07039560524885 -0.596214511041009\\
1.0705176850396 -0.197017493547462\\
1.07063976483034 0.17866360768569\\
1.07076184462109 0.458560367077717\\
1.07088392441183 0.596214511041009\\
1.07100600420258 0.233725265271007\\
1.07112808399332 -0.270433036994551\\
1.07125016378407 -0.596214511041009\\
1.07137224357481 -0.440206481215945\\
1.07149432336556 -0.0220820189274448\\
1.0716164031563 0.513622024663034\\
1.07173848294705 0.596214511041009\\
1.07186056273779 0.233725265271007\\
1.07198264252853 -0.0754229997132205\\
1.07210472231928 -0.596214511041009\\
1.07222680211002 -0.669630054488099\\
1.07234888190077 -0.252079151132779\\
1.07247096169151 0.293375394321767\\
1.07259304148226 0.513622024663034\\
1.072715121273 0.421852595354173\\
1.07283720106375 0.261256094063665\\
1.07295928085449 -0.187840550616576\\
1.07308136064524 -0.252079151132779\\
1.07320344043598 -0.233725265271007\\
1.07332552022673 0.0151993117292802\\
1.07344760001747 0.0628047031832521\\
1.07356967980822 -0.130484657298537\\
1.07369175959896 -0.293375394321767\\
1.0738138393897 -0.279609979925437\\
1.07393591918045 0.141955835962145\\
1.07405799897119 0.311729280183539\\
1.07418007876194 0.669630054488099\\
1.07430215855268 0.596214511041009\\
1.07442423834343 0.0628047031832521\\
1.07454631813417 -0.596214511041009\\
1.07466839792492 -0.926584456552911\\
1.07479047771566 -0.596214511041009\\
1.07491255750641 -0.107542299971322\\
1.07503463729715 0.550329796386579\\
1.0751567170879 0.743045597935188\\
1.07527879687864 0.632922282764554\\
1.07540087666938 0.112130771436765\\
1.07552295646013 -0.513622024663034\\
1.07564503625087 -0.531975910524806\\
1.07576711604162 -0.270433036994551\\
1.07588919583236 0.169486664754804\\
1.07601127562311 0.22454832234012\\
1.07613335541385 0.22454832234012\\
1.0762554352046 0.0197877831947233\\
1.07637751499534 -0.311729280183539\\
1.07649959478609 -0.261256094063665\\
1.07662167457683 -0.0375681101233152\\
1.07674375436758 0.366790937768856\\
1.07686583415832 0.311729280183539\\
1.07698791394906 0.0352738743905936\\
1.07710999373981 -0.151132778893031\\
1.07723207353055 -0.233725265271007\\
1.0773541533213 -0.17866360768569\\
1.07747623311204 -0.0220820189274448\\
1.07759831290279 0.421852595354173\\
1.07772039269353 0.348437051907083\\
1.07784247248428 -0.0891884141095498\\
1.07796455227502 -0.440206481215945\\
1.07808663206577 -0.513622024663034\\
1.07820871185651 -0.270433036994551\\
1.07833079164726 0.135073128763981\\
1.078452871438 0.669630054488099\\
1.07857495122874 0.779753369658732\\
1.07869703101949 0.4034987094924\\
1.07881911081023 -0.242902208201893\\
1.07894119060098 -0.743045597935188\\
1.07906327039172 -0.743045597935188\\
1.07918535018247 -0.348437051907083\\
1.07930742997321 0.0490392887869229\\
1.07942950976396 0.47691425293949\\
1.0795515895547 0.669630054488099\\
1.07967366934545 0.330083166045311\\
1.07979574913619 -0.0937768855749928\\
1.07991782892694 -0.233725265271007\\
1.08003990871768 -0.112130771436765\\
1.08016198850842 -0.0375681101233152\\
1.08028406829917 -0.020934901061084\\
1.08040614808991 -0.0983653570404359\\
1.08052822788066 -0.22454832234012\\
1.0806503076714 -0.279609979925437\\
1.08077238746215 -0.160309721823917\\
1.08089446725289 0.261256094063665\\
1.08101654704364 0.632922282764554\\
1.08113862683438 0.531975910524806\\
1.08126070662513 0.058216231717809\\
1.08138278641587 -0.311729280183539\\
1.08150486620662 -0.568683682248351\\
1.08162694599736 -0.596214511041009\\
1.0817490257881 -0.116719242902208\\
1.08187110557885 0.47691425293949\\
1.08199318536959 0.596214511041009\\
1.08211526516034 0.330083166045311\\
1.08223734495108 -0.116719242902208\\
1.08235942474183 -0.495268138801262\\
1.08248150453257 -0.47691425293949\\
1.08260358432332 -0.206194436478348\\
1.08272566411406 0.233725265271007\\
1.08284774390481 0.596214511041009\\
1.08296982369555 0.531975910524806\\
1.0830919034863 -0.028391167192429\\
1.08321398327704 -0.366790937768856\\
1.08333606306778 -0.421852595354173\\
1.08345814285853 -0.348437051907083\\
1.08358022264927 -0.0800114711786636\\
1.08370230244002 0.22454832234012\\
1.08382438223076 0.270433036994551\\
1.08394646202151 0.107542299971322\\
1.08406854181225 -0.028391167192429\\
1.084190621603 -0.0151993117292802\\
1.08431270139374 0.197017493547462\\
1.08443478118449 0.279609979925437\\
1.08455686097523 0.0937768855749928\\
1.08467894076598 -0.206194436478348\\
1.08480102055672 -0.421852595354173\\
1.08492310034746 -0.632922282764554\\
1.08504518013821 -0.4034987094924\\
1.08516725992895 0.270433036994551\\
1.0852893397197 0.743045597935188\\
1.08541141951044 0.779753369658732\\
1.08553349930119 0.440206481215945\\
1.08565557909193 -0.0708345282477775\\
1.08577765888268 -0.669630054488099\\
1.08589973867342 -0.779753369658732\\
1.08602181846417 -0.495268138801262\\
1.08614389825491 0.0845999426441067\\
1.08626597804566 0.568683682248351\\
1.0863880578364 0.421852595354173\\
1.08651013762714 0.22454832234012\\
1.08663221741789 -0.0754229997132205\\
1.08675429720863 -0.279609979925437\\
1.08687637699938 -0.252079151132779\\
1.08699845679012 0.0513335245196444\\
1.08712053658087 0.293375394321767\\
1.08724261637161 0.102953828505879\\
1.08736469616236 -0.0605104674505305\\
1.0874867759531 -0.293375394321767\\
1.08760885574385 -0.22454832234012\\
1.08773093553459 -0.00946372239747634\\
1.08785301532534 0.206194436478348\\
1.08797509511608 0.330083166045311\\
1.08809717490683 0.141955835962145\\
1.08821925469757 -0.151132778893031\\
1.08834133448831 -0.47691425293949\\
1.08846341427906 -0.233725265271007\\
1.0885854940698 0.206194436478348\\
1.08870757386055 0.458560367077717\\
1.08882965365129 0.495268138801262\\
1.08895173344204 0.206194436478348\\
1.08907381323278 -0.293375394321767\\
1.08919589302353 -0.816461141382277\\
1.08931797281427 -0.669630054488099\\
1.08944005260502 -0.187840550616576\\
1.08956213239576 0.495268138801262\\
1.08968421218651 0.853168913105822\\
1.08980629197725 0.596214511041009\\
1.08992837176799 0.261256094063665\\
1.09005045155874 -0.311729280183539\\
1.09017253134948 -0.669630054488099\\
1.09029461114023 -0.568683682248351\\
1.09041669093097 -0.0754229997132205\\
1.09053877072172 0.160309721823917\\
1.09066085051246 0.141955835962145\\
1.09078293030321 0.233725265271007\\
1.09090501009395 -0.0117579581301979\\
1.0910270898847 0.00172067679954115\\
1.09114916967544 0.116719242902208\\
1.09127124946619 0.279609979925437\\
1.09139332925693 0.279609979925437\\
1.09151540904767 -0.0467450530542013\\
1.09163748883842 -0.440206481215945\\
1.09175956862916 -0.669630054488099\\
1.09188164841991 -0.293375394321767\\
1.09200372821065 0.0754229997132205\\
1.0921258080014 0.47691425293949\\
1.09224788779214 0.706337826211643\\
1.09236996758289 0.458560367077717\\
1.09249204737363 -0.160309721823917\\
1.09261412716438 -0.632922282764554\\
1.09273620695512 -0.550329796386579\\
1.09285828674587 -0.215371379409234\\
1.09298036653661 0.348437051907083\\
1.09310244632735 0.531975910524806\\
1.0932245261181 0.366790937768856\\
1.09334660590884 0.0800114711786636\\
1.09346868569959 -0.440206481215945\\
1.09359076549033 -0.632922282764554\\
1.09371284528108 -0.293375394321767\\
1.09383492507182 0.252079151132779\\
1.09395700486257 0.385144823630628\\
1.09407908465331 0.385144823630628\\
1.09420116444406 0.293375394321767\\
1.0943232442348 -0.141955835962145\\
1.09444532402555 -0.330083166045311\\
1.09456740381629 -0.17866360768569\\
1.09468948360703 -0.0117579581301979\\
1.09481156339778 0.0845999426441067\\
1.09493364318852 -0.0421565815887582\\
1.09505572297927 -0.270433036994551\\
1.09517780277001 -0.233725265271007\\
1.09529988256076 0.0174935474620017\\
1.0954219623515 0.261256094063665\\
1.09554404214225 0.596214511041009\\
1.09566612193299 0.669630054488099\\
1.09578820172374 0.141955835962145\\
1.09591028151448 -0.550329796386579\\
1.09603236130523 -0.779753369658732\\
1.09615444109597 -0.706337826211643\\
1.09627652088671 -0.293375394321767\\
1.09639860067746 0.421852595354173\\
1.0965206804682 0.853168913105822\\
1.09664276025895 0.669630054488099\\
1.09676484004969 0.252079151132779\\
1.09688691984044 -0.330083166045311\\
1.09700899963118 -0.596214511041009\\
1.09713107942193 -0.330083166045311\\
1.09725315921267 -0.0536277602523659\\
1.09737523900342 0.121307714367651\\
1.09749731879416 0.270433036994551\\
1.09761939858491 0.135073128763981\\
1.09774147837565 -0.279609979925437\\
1.09786355816639 -0.22454832234012\\
1.09798563795714 0.102953828505879\\
1.09810771774788 0.261256094063665\\
1.09822979753863 0.279609979925437\\
1.09835187732937 0.116719242902208\\
1.09847395712012 -0.169486664754804\\
1.09859603691086 -0.348437051907083\\
1.09871811670161 -0.293375394321767\\
1.09884019649235 -0.0467450530542013\\
1.0989622762831 0.4034987094924\\
1.09908435607384 0.513622024663034\\
1.09920643586459 -0.00516203039862346\\
1.09932851565533 -0.311729280183539\\
1.09945059544608 -0.458560367077717\\
1.09957267523682 -0.4034987094924\\
1.09969475502756 -0.0421565815887582\\
1.09981683481831 0.596214511041009\\
1.09993891460905 0.816461141382277\\
1.1000609943998 0.366790937768856\\
1.10018307419054 -0.0800114711786636\\
1.10030515398129 -0.632922282764554\\
1.10042723377203 -0.743045597935188\\
1.10054931356278 -0.440206481215945\\
1.10067139335352 0.00946372239747634\\
1.10079347314427 0.458560367077717\\
1.10091555293501 0.632922282764554\\
1.10103763272576 0.348437051907083\\
1.1011597125165 -0.116719242902208\\
1.10128179230724 -0.135073128763981\\
1.10140387209799 -0.112130771436765\\
1.10152595188873 -0.0983653570404359\\
1.10164803167948 0.0174935474620017\\
1.10177011147022 -0.0937768855749928\\
1.10189219126097 -0.293375394321767\\
1.10201427105171 -0.348437051907083\\
1.10213635084246 -0.107542299971322\\
1.1022584306332 0.261256094063665\\
1.10238051042395 0.632922282764554\\
1.10250259021469 0.632922282764554\\
1.10262467000544 0.135073128763981\\
1.10274674979618 -0.233725265271007\\
1.10286882958692 -0.669630054488099\\
1.10299090937767 -0.743045597935188\\
1.10311298916841 -0.22454832234012\\
1.10323506895916 0.440206481215945\\
1.1033571487499 0.596214511041009\\
1.10347922854065 0.4034987094924\\
1.10360130833139 0.197017493547462\\
1.10372338812214 -0.385144823630628\\
1.10384546791288 -0.568683682248351\\
1.10396754770363 -0.293375394321767\\
1.10408962749437 0.141955835962145\\
1.10421170728512 0.47691425293949\\
1.10433378707586 0.385144823630628\\
1.1044558668666 0.0937768855749928\\
1.10457794665735 -0.293375394321767\\
1.10470002644809 -0.366790937768856\\
1.10482210623884 -0.330083166045311\\
1.10494418602958 -0.0605104674505305\\
1.10506626582033 0.385144823630628\\
1.10518834561107 0.215371379409234\\
1.10531042540182 -0.0186406653283625\\
1.10543250519256 -0.121307714367651\\
1.10555458498331 -0.0306854029251506\\
1.10567666477405 0.102953828505879\\
1.1057987445648 0.233725265271007\\
1.10592082435554 0.311729280183539\\
1.10604290414628 -0.0444508173214798\\
1.10616498393703 -0.366790937768856\\
1.10628706372777 -0.743045597935188\\
1.10640914351852 -0.513622024663034\\
1.10653122330926 0.206194436478348\\
1.10665330310001 0.632922282764554\\
1.10677538289075 0.779753369658732\\
1.1068974626815 0.550329796386579\\
1.10701954247224 0.151132778893031\\
1.10714162226299 -0.632922282764554\\
1.10726370205373 -0.889876684829366\\
1.10738578184448 -0.513622024663034\\
1.10750786163522 -0.0197877831947233\\
1.10762994142596 0.421852595354173\\
1.10775202121671 0.440206481215945\\
1.10787410100745 0.366790937768856\\
1.1079961807982 0.0174935474620017\\
1.10811826058894 -0.187840550616576\\
1.10824034037969 -0.233725265271007\\
1.10836242017043 0.0352738743905936\\
1.10848449996118 0.270433036994551\\
1.10860657975192 -0.0306854029251506\\
1.10872865954267 -0.206194436478348\\
1.10885073933341 -0.270433036994551\\
1.10897281912416 -0.206194436478348\\
1.1090948989149 -0.0490392887869229\\
1.10921697870565 0.348437051907083\\
1.10933905849639 0.513622024663034\\
1.10946113828713 0.22454832234012\\
1.10958321807788 -0.169486664754804\\
1.10970529786862 -0.513622024663034\\
1.10982737765937 -0.385144823630628\\
1.10994945745011 0.0106108402638371\\
1.11007153724086 0.366790937768856\\
1.1101936170316 0.513622024663034\\
1.11031569682235 0.366790937768856\\
1.11043777661309 -0.102953828505879\\
1.11055985640384 -0.706337826211643\\
1.11068193619458 -0.669630054488099\\
1.11080401598533 -0.233725265271007\\
1.11092609577607 0.261256094063665\\
1.11104817556681 0.669630054488099\\
1.11117025535756 0.669630054488099\\
1.1112923351483 0.330083166045311\\
1.11141441493905 -0.242902208201893\\
1.11153649472979 -0.550329796386579\\
1.11165857452054 -0.531975910524806\\
1.11178065431128 -0.197017493547462\\
1.11190273410203 0.121307714367651\\
1.11202481389277 0.121307714367651\\
1.11214689368352 0.151132778893031\\
1.11226897347426 0.141955835962145\\
1.11239105326501 -0.0708345282477775\\
1.11251313305575 0.107542299971322\\
1.11263521284649 0.4034987094924\\
1.11275729263724 0.279609979925437\\
1.11287937242798 -0.0662460567823344\\
1.11300145221873 -0.4034987094924\\
1.11312353200947 -0.632922282764554\\
1.11324561180022 -0.550329796386579\\
1.11336769159096 -0.0186406653283625\\
1.11348977138171 0.458560367077717\\
1.11361185117245 0.816461141382277\\
1.1137339309632 0.596214511041009\\
1.11385601075394 0.0197877831947233\\
1.11397809054469 -0.513622024663034\\
1.11410017033543 -0.669630054488099\\
1.11422225012617 -0.385144823630628\\
1.11434432991692 -0.0398623458560367\\
1.11446640970766 0.531975910524806\\
1.11458848949841 0.513622024663034\\
1.11471056928915 0.151132778893031\\
1.1148326490799 -0.293375394321767\\
1.11495472887064 -0.440206481215945\\
1.11507680866139 -0.279609979925437\\
1.11519888845213 0.0605104674505305\\
1.11532096824288 0.348437051907083\\
1.11544304803362 0.348437051907083\\
1.11556512782437 0.270433036994551\\
1.11568720761511 -0.206194436478348\\
1.11580928740585 -0.385144823630628\\
1.1159313671966 -0.160309721823917\\
1.11605344698734 0.102953828505879\\
1.11617552677809 0.160309721823917\\
1.11629760656883 -0.0151993117292802\\
1.11641968635958 -0.0708345282477775\\
1.11654176615032 -0.311729280183539\\
1.11666384594107 -0.206194436478348\\
1.11678592573181 0.141955835962145\\
1.11690800552256 0.531975910524806\\
1.1170300853133 0.743045597935188\\
1.11715216510405 0.206194436478348\\
1.11727424489479 -0.366790937768856\\
1.11739632468553 -0.743045597935188\\
1.11751840447628 -0.743045597935188\\
1.11764048426702 -0.440206481215945\\
1.11776256405777 0.252079151132779\\
1.11788464384851 0.926584456552911\\
1.11800672363926 0.706337826211643\\
1.11812880343 0.330083166045311\\
1.11825088322075 -0.252079151132779\\
1.11837296301149 -0.513622024663034\\
1.11849504280224 -0.458560367077717\\
1.11861712259298 -0.215371379409234\\
1.11873920238373 0.0983653570404359\\
1.11886128217447 0.215371379409234\\
1.11898336196521 0.206194436478348\\
1.11910544175596 -0.215371379409234\\
1.1192275215467 -0.135073128763981\\
1.11934960133745 0.121307714367651\\
1.11947168112819 0.242902208201893\\
1.11959376091894 0.261256094063665\\
1.11971584070968 0.125896185833094\\
1.11983792050043 -0.102953828505879\\
1.11996000029117 -0.495268138801262\\
1.12008208008192 -0.4034987094924\\
1.12020415987266 -0.0937768855749928\\
1.12032623966341 0.366790937768856\\
1.12044831945415 0.550329796386579\\
1.12057039924489 0.215371379409234\\
1.12069247903564 -0.0605104674505305\\
1.12081455882638 -0.440206481215945\\
1.12093663861713 -0.513622024663034\\
1.12105871840787 -0.270433036994551\\
1.12118079819862 0.385144823630628\\
1.12130287798936 0.743045597935188\\
1.12142495778011 0.421852595354173\\
1.12154703757085 0.0891884141095498\\
1.1216691173616 -0.421852595354173\\
1.12179119715234 -0.706337826211643\\
1.12191327694309 -0.531975910524806\\
1.12203535673383 -0.0937768855749928\\
1.12215743652457 0.421852595354173\\
1.12227951631532 0.568683682248351\\
1.12240159610606 0.366790937768856\\
1.12252367589681 0.00946372239747634\\
1.12264575568755 -0.112130771436765\\
1.1227678354783 -0.151132778893031\\
1.12288991526904 -0.197017493547462\\
1.12301199505979 0.00688270719816461\\
1.12313407485053 -0.0106108402638371\\
1.12325615464128 -0.242902208201893\\
1.12337823443202 -0.348437051907083\\
1.12350031422277 -0.121307714367651\\
1.12362239401351 0.242902208201893\\
1.12374447380426 0.495268138801262\\
1.123866553595 0.632922282764554\\
1.12398863338574 0.330083166045311\\
1.12411071317649 -0.116719242902208\\
1.12423279296723 -0.632922282764554\\
1.12435487275798 -0.853168913105822\\
1.12447695254872 -0.4034987094924\\
1.12459903233947 0.215371379409234\\
1.12472111213021 0.596214511041009\\
1.12484319192096 0.531975910524806\\
1.1249652717117 0.421852595354173\\
1.12508735150245 -0.125896185833094\\
1.12520943129319 -0.596214511041009\\
1.12533151108394 -0.421852595354173\\
1.12545359087468 -0.0513335245196444\\
1.12557567066542 0.330083166045311\\
1.12569775045617 0.330083166045311\\
1.12581983024691 0.17866360768569\\
1.12594191003766 -0.169486664754804\\
1.1260639898284 -0.311729280183539\\
1.12618606961915 -0.252079151132779\\
1.12630814940989 -0.0845999426441067\\
1.12643022920064 0.366790937768856\\
1.12655230899138 0.242902208201893\\
1.12667438878213 -0.0329796386578721\\
1.12679646857287 -0.17866360768569\\
1.12691854836362 -0.0937768855749928\\
1.12704062815436 -0.0306854029251506\\
1.1271627079451 0.169486664754804\\
1.12728478773585 0.440206481215945\\
1.12740686752659 0.0845999426441067\\
1.12752894731734 -0.293375394321767\\
1.12765102710808 -0.596214511041009\\
1.12777310689883 -0.550329796386579\\
1.12789518668957 -0.0708345282477775\\
1.12801726648032 0.440206481215945\\
1.12813934627106 0.779753369658732\\
1.12826142606181 0.669630054488099\\
1.12838350585255 0.270433036994551\\
1.1285055856433 -0.4034987094924\\
1.12862766543404 -0.853168913105822\\
1.12874974522478 -0.632922282764554\\
1.12887182501553 -0.261256094063665\\
1.12899390480627 0.187840550616576\\
1.12911598459702 0.550329796386579\\
1.12923806438776 0.495268138801262\\
1.12936014417851 0.215371379409234\\
1.12948222396925 -0.130484657298537\\
1.12960430376 -0.187840550616576\\
1.12972638355074 -0.135073128763981\\
1.12984846334149 0.107542299971322\\
1.12997054313223 -0.00172067679954115\\
1.13009262292298 -0.215371379409234\\
1.13021470271372 -0.197017493547462\\
1.13033678250446 -0.261256094063665\\
1.13045886229521 -0.0800114711786636\\
1.13058094208595 0.311729280183539\\
1.1307030218767 0.632922282764554\\
1.13082510166744 0.311729280183539\\
1.13094718145819 -0.058216231717809\\
1.13106926124893 -0.4034987094924\\
1.13119134103968 -0.568683682248351\\
1.13131342083042 -0.293375394321767\\
1.13143550062117 0.151132778893031\\
1.13155758041191 0.596214511041009\\
1.13167966020266 0.513622024663034\\
1.1318017399934 0.0983653570404359\\
1.13192381978414 -0.421852595354173\\
1.13204589957489 -0.596214511041009\\
1.13216797936563 -0.385144823630628\\
1.13229005915638 -0.0398623458560367\\
1.13241213894712 0.47691425293949\\
1.13253421873787 0.743045597935188\\
1.13265629852861 0.440206481215945\\
1.13277837831936 -0.130484657298537\\
1.1329004581101 -0.458560367077717\\
1.13302253790085 -0.495268138801262\\
1.13314461769159 -0.330083166045311\\
1.13326669748234 0.028391167192429\\
1.13338877727308 0.187840550616576\\
1.13351085706382 0.215371379409234\\
1.13363293685457 0.121307714367651\\
1.13375501664531 -0.0845999426441067\\
1.13387709643606 0.0306854029251506\\
1.1339991762268 0.311729280183539\\
1.13412125601755 0.348437051907083\\
1.13424333580829 0.0174935474620017\\
1.13436541559904 -0.293375394321767\\
1.13448749538978 -0.531975910524806\\
1.13460957518053 -0.706337826211643\\
1.13473165497127 -0.233725265271007\\
1.13485373476202 0.421852595354173\\
1.13497581455276 0.816461141382277\\
1.13509789434351 0.743045597935188\\
1.13521997413425 0.261256094063665\\
1.13534205392499 -0.366790937768856\\
1.13546413371574 -0.706337826211643\\
1.13558621350648 -0.632922282764554\\
1.13570829329723 -0.293375394321767\\
1.13583037308797 0.348437051907083\\
1.13595245287872 0.596214511041009\\
1.13607453266946 0.293375394321767\\
1.13619661246021 -0.0628047031832521\\
1.13631869225095 -0.233725265271007\\
1.1364407720417 -0.293375394321767\\
1.13656285183244 -0.17866360768569\\
1.13668493162319 0.22454832234012\\
1.13680701141393 0.311729280183539\\
1.13692909120467 0.151132778893031\\
1.13705117099542 -0.0800114711786636\\
1.13717325078616 -0.348437051907083\\
1.13729533057691 -0.169486664754804\\
1.13741741036765 0.0662460567823344\\
1.1375394901584 0.206194436478348\\
1.13766156994914 0.125896185833094\\
1.13778364973989 0.0559219959850875\\
1.13790572953063 -0.242902208201893\\
1.13802780932138 -0.458560367077717\\
1.13814988911212 0.0174935474620017\\
1.13827196890287 0.385144823630628\\
1.13839404869361 0.568683682248351\\
1.13851612848435 0.385144823630628\\
1.1386382082751 -0.0129050759965586\\
1.13876028806584 -0.531975910524806\\
1.13888236785659 -0.816461141382277\\
1.13900444764733 -0.596214511041009\\
1.13912652743808 -0.020934901061084\\
1.13924860722882 0.706337826211643\\
1.13937068701957 0.779753369658732\\
1.13949276681031 0.513622024663034\\
1.13961484660106 0.0800114711786636\\
1.1397369263918 -0.385144823630628\\
1.13985900618255 -0.632922282764554\\
1.13998108597329 -0.440206481215945\\
1.14010316576403 0.00860338399770576\\
1.14022524555478 0.151132778893031\\
1.14034732534552 0.206194436478348\\
1.14046940513627 0.0845999426441067\\
1.14059148492701 -0.0662460567823344\\
1.14071356471776 0.0352738743905936\\
1.1408356445085 0.169486664754804\\
1.14095772429925 0.270433036994551\\
1.14107980408999 0.187840550616576\\
1.14120188388074 -0.0375681101233152\\
1.14132396367148 -0.458560367077717\\
1.14144604346223 -0.568683682248351\\
1.14156812325297 -0.197017493547462\\
1.14169020304371 0.151132778893031\\
1.14181228283446 0.495268138801262\\
1.1419343626252 0.596214511041009\\
1.14205644241595 0.22454832234012\\
1.14217852220669 -0.293375394321767\\
1.14230060199744 -0.596214511041009\\
1.14242268178818 -0.421852595354173\\
1.14254476157893 0.000573558933180384\\
1.14266684136967 0.513622024663034\\
1.14278892116042 0.550329796386579\\
1.14291100095116 0.261256094063665\\
1.14303308074191 -0.102953828505879\\
1.14315516053265 -0.596214511041009\\
1.14327724032339 -0.632922282764554\\
1.14339932011414 -0.187840550616576\\
1.14352139990488 0.311729280183539\\
1.14364347969563 0.440206481215945\\
1.14376555948637 0.421852595354173\\
1.14388763927712 0.242902208201893\\
1.14400971906786 -0.187840550616576\\
1.14413179885861 -0.252079151132779\\
1.14425387864935 -0.187840550616576\\
1.1443759584401 -0.0129050759965586\\
1.14449803823084 0.0708345282477775\\
1.14462011802159 -0.125896185833094\\
1.14474219781233 -0.270433036994551\\
1.14486427760307 -0.242902208201893\\
1.14498635739382 0.102953828505879\\
1.14510843718456 0.330083166045311\\
1.14523051697531 0.632922282764554\\
1.14535259676605 0.550329796386579\\
1.1454746765568 0.0255233725265271\\
1.14559675634754 -0.513622024663034\\
1.14571883613829 -0.816461141382277\\
1.14584091592903 -0.596214511041009\\
1.14596299571978 -0.121307714367651\\
1.14608507551052 0.47691425293949\\
1.14620715530127 0.706337826211643\\
1.14632923509201 0.596214511041009\\
1.14645131488275 0.0845999426441067\\
1.1465733946735 -0.440206481215945\\
1.14669547446424 -0.495268138801262\\
1.14681755425499 -0.270433036994551\\
1.14693963404573 0.0628047031832521\\
1.14706171383648 0.22454832234012\\
1.14718379362722 0.293375394321767\\
1.14730587341797 0.028391167192429\\
1.14742795320871 -0.293375394321767\\
1.14755003299946 -0.233725265271007\\
1.1476721127902 -0.0490392887869229\\
1.14779419258095 0.270433036994551\\
1.14791627237169 0.279609979925437\\
1.14803835216243 0.121307714367651\\
1.14816043195318 -0.130484657298537\\
1.14828251174392 -0.22454832234012\\
1.14840459153467 -0.206194436478348\\
1.14852667132541 -0.0186406653283625\\
1.14864875111616 0.4034987094924\\
1.1487708309069 0.311729280183539\\
1.14889291069765 -0.058216231717809\\
1.14901499048839 -0.385144823630628\\
1.14913707027914 -0.47691425293949\\
1.14925915006988 -0.330083166045311\\
1.14938122986063 0.112130771436765\\
1.14950330965137 0.669630054488099\\
1.14962538944212 0.743045597935188\\
1.14974746923286 0.421852595354173\\
1.1498695490236 -0.22454832234012\\
1.14999162881435 -0.706337826211643\\
1.15011370860509 -0.743045597935188\\
1.15023578839584 -0.4034987094924\\
1.15035786818658 0.0559219959850875\\
1.15047994797733 0.495268138801262\\
1.15060202776807 0.706337826211643\\
1.15072410755882 0.279609979925437\\
1.15084618734956 -0.107542299971322\\
1.15096826714031 -0.187840550616576\\
1.15109034693105 -0.141955835962145\\
1.1512124267218 -0.00946372239747634\\
1.15133450651254 -0.0444508173214798\\
1.15145658630328 -0.0937768855749928\\
1.15157866609403 -0.215371379409234\\
1.15170074588477 -0.270433036994551\\
1.15182282567552 -0.151132778893031\\
1.15194490546626 0.252079151132779\\
1.15206698525701 0.669630054488099\\
1.15218906504775 0.495268138801262\\
1.1523111448385 0.0754229997132205\\
1.15243322462924 -0.293375394321767\\
1.15255530441999 -0.568683682248351\\
1.15267738421073 -0.568683682248351\\
1.15279946400148 -0.135073128763981\\
1.15292154379222 0.440206481215945\\
1.15304362358296 0.596214511041009\\
1.15316570337371 0.348437051907083\\
1.15328778316445 -0.102953828505879\\
1.1534098629552 -0.458560367077717\\
1.15353194274594 -0.440206481215945\\
1.15365402253669 -0.242902208201893\\
1.15377610232743 0.187840550616576\\
1.15389818211818 0.596214511041009\\
1.15402026190892 0.513622024663034\\
1.15414234169967 0.00688270719816461\\
1.15426442149041 -0.348437051907083\\
1.15438650128116 -0.4034987094924\\
1.1545085810719 -0.348437051907083\\
1.15463066086264 -0.0708345282477775\\
1.15475274065339 0.206194436478348\\
1.15487482044413 0.261256094063665\\
1.15499690023488 0.135073128763981\\
1.15511898002562 -0.0708345282477775\\
1.15524105981637 -0.0605104674505305\\
1.15536313960711 0.197017493547462\\
1.15548521939786 0.330083166045311\\
1.1556072991886 0.0983653570404359\\
1.15572937897935 -0.187840550616576\\
1.15585145877009 -0.366790937768856\\
1.15597353856084 -0.632922282764554\\
1.15609561835158 -0.458560367077717\\
1.15621769814232 0.187840550616576\\
1.15633977793307 0.743045597935188\\
1.15646185772381 0.779753369658732\\
1.15658393751456 0.440206481215945\\
1.1567060173053 -0.058216231717809\\
1.15682809709605 -0.550329796386579\\
1.15695017688679 -0.743045597935188\\
1.15707225667754 -0.550329796386579\\
1.15719433646828 0.0352738743905936\\
1.15731641625903 0.513622024663034\\
1.15743849604977 0.47691425293949\\
1.15756057584052 0.169486664754804\\
1.15768265563126 -0.0490392887869229\\
1.157804735422 -0.17866360768569\\
1.15792681521275 -0.252079151132779\\
1.15804889500349 0.0117579581301979\\
1.15817097479424 0.233725265271007\\
1.15829305458498 0.160309721823917\\
1.15841513437573 -0.0891884141095498\\
1.15853721416647 -0.311729280183539\\
1.15865929395722 -0.242902208201893\\
1.15878137374796 0.016346429595641\\
1.15890345353871 0.242902208201893\\
1.15902553332945 0.311729280183539\\
1.1591476131202 0.197017493547462\\
1.15926969291094 -0.0891884141095498\\
1.15939177270168 -0.495268138801262\\
1.15951385249243 -0.348437051907083\\
1.15963593228317 0.169486664754804\\
1.15975801207392 0.440206481215945\\
1.15988009186466 0.495268138801262\\
1.16000217165541 0.242902208201893\\
1.16012425144615 -0.187840550616576\\
1.1602463312369 -0.706337826211643\\
1.16036841102764 -0.706337826211643\\
1.16049049081839 -0.261256094063665\\
1.16061257060913 0.348437051907083\\
1.16073465039988 0.816461141382277\\
1.16085673019062 0.596214511041009\\
1.16097880998137 0.279609979925437\\
1.16110088977211 -0.187840550616576\\
1.16122296956285 -0.596214511041009\\
1.1613450493536 -0.568683682248351\\
1.16146712914434 -0.206194436478348\\
1.16158920893509 0.187840550616576\\
1.16171128872583 0.141955835962145\\
1.16183336851658 0.160309721823917\\
1.16195544830732 0.058216231717809\\
1.16207752809807 -0.0306854029251506\\
1.16219960788881 0.116719242902208\\
1.16232168767956 0.279609979925437\\
1.1624437674703 0.293375394321767\\
1.16256584726105 -0.016346429595641\\
1.16268792705179 -0.348437051907083\\
1.16281000684253 -0.632922282764554\\
1.16293208663328 -0.421852595354173\\
1.16305416642402 0.00516203039862346\\
1.16317624621477 0.348437051907083\\
1.16329832600551 0.706337826211643\\
1.16342040579626 0.550329796386579\\
1.163542485587 0.0266704903928879\\
1.16366456537775 -0.495268138801262\\
1.16378664516849 -0.632922282764554\\
1.16390872495924 -0.311729280183539\\
1.16403080474998 0.0845999426441067\\
1.16415288454073 0.495268138801262\\
1.16427496433147 0.458560367077717\\
1.16439704412221 0.17866360768569\\
1.16451912391296 -0.242902208201893\\
1.1646412037037 -0.568683682248351\\
1.16476328349445 -0.348437051907083\\
1.16488536328519 0.0329796386578721\\
1.16500744307594 0.366790937768856\\
1.16512952286668 0.366790937768856\\
1.16525160265743 0.311729280183539\\
1.16537368244817 -0.00229423573272154\\
1.16549576223892 -0.330083166045311\\
1.16561784202966 -0.215371379409234\\
1.16573992182041 -0.0754229997132205\\
1.16586200161115 0.0891884141095498\\
1.16598408140189 0.0117579581301979\\
1.16610616119264 -0.141955835962145\\
1.16622824098338 -0.215371379409234\\
1.16635032077413 -0.121307714367651\\
1.16647240056487 0.151132778893031\\
1.16659448035562 0.421852595354173\\
1.16671656014636 0.669630054488099\\
1.16683863993711 0.293375394321767\\
1.16696071972785 -0.293375394321767\\
1.1670827995186 -0.669630054488099\\
1.16720487930934 -0.743045597935188\\
1.16732695910009 -0.458560367077717\\
1.16744903889083 0.112130771436765\\
1.16757111868157 0.743045597935188\\
1.16769319847232 0.743045597935188\\
1.16781527826306 0.458560367077717\\
1.16793735805381 -0.0983653570404359\\
1.16805943784455 -0.513622024663034\\
1.1681815176353 -0.458560367077717\\
1.16830359742604 -0.293375394321767\\
1.16842567721679 0.0421565815887582\\
1.16854775700753 0.233725265271007\\
1.16866983679828 0.252079151132779\\
1.16879191658902 -0.0421565815887582\\
1.16891399637977 -0.22454832234012\\
1.16903607617051 -0.0151993117292802\\
1.16915815596125 0.135073128763981\\
1.169280235752 0.22454832234012\\
1.16940231554274 0.169486664754804\\
1.16952439533349 0.0266704903928879\\
1.16964647512423 -0.279609979925437\\
1.16976855491498 -0.366790937768856\\
1.16989063470572 -0.197017493547462\\
1.17001271449647 0.197017493547462\\
1.17013479428721 0.47691425293949\\
1.17025687407796 0.242902208201893\\
1.1703789538687 -0.0845999426441067\\
1.17050103365945 -0.311729280183539\\
1.17062311345019 -0.421852595354173\\
1.17074519324094 -0.348437051907083\\
1.17086727303168 0.270433036994551\\
1.17098935282242 0.706337826211643\\
1.17111143261317 0.550329796386579\\
1.17123351240391 0.160309721823917\\
1.17135559219466 -0.311729280183539\\
1.1714776719854 -0.632922282764554\\
1.17159975177615 -0.669630054488099\\
1.17172183156689 -0.261256094063665\\
1.17184391135764 0.293375394321767\\
1.17196599114838 0.632922282764554\\
1.17208807093913 0.513622024663034\\
1.17221015072987 0.130484657298537\\
1.17233223052062 -0.102953828505879\\
1.17245431031136 -0.187840550616576\\
1.1725763901021 -0.242902208201893\\
1.17269846989285 -0.130484657298537\\
1.17282054968359 0.0375681101233152\\
1.17294262947434 -0.102953828505879\\
1.17306470926508 -0.261256094063665\\
1.17318678905583 -0.187840550616576\\
1.17330886884657 0.0937768855749928\\
1.17343094863732 0.4034987094924\\
1.17355302842806 0.531975910524806\\
1.17367510821881 0.4034987094924\\
1.17379718800955 -0.00114711786636077\\
1.1739192678003 -0.4034987094924\\
1.17404134759104 -0.743045597935188\\
1.17416342738178 -0.568683682248351\\
1.17428550717253 0.0845999426441067\\
1.17440758696327 0.495268138801262\\
1.17452966675402 0.568683682248351\\
1.17465174654476 0.385144823630628\\
1.17477382633551 0.00946372239747634\\
1.17489590612625 -0.495268138801262\\
1.175017985917 -0.531975910524806\\
1.17514006570774 -0.17866360768569\\
1.17526214549849 0.270433036994551\\
1.17538422528923 0.47691425293949\\
1.17550630507998 0.293375394321767\\
1.17562838487072 -0.0329796386578721\\
1.17575046466146 -0.311729280183539\\
1.17587254445221 -0.366790937768856\\
1.17599462424295 -0.311729280183539\\
1.1761167040337 0.116719242902208\\
1.17623878382444 0.385144823630628\\
1.17636086361519 0.151132778893031\\
1.17648294340593 0.00229423573272154\\
1.17660502319668 -0.0467450530542013\\
1.17672710298742 -0.058216231717809\\
1.17684918277817 0.0490392887869229\\
1.17697126256891 0.22454832234012\\
1.17709334235966 0.151132778893031\\
1.1772154221504 -0.160309721823917\\
1.17733750194114 -0.440206481215945\\
1.17745958173189 -0.568683682248351\\
1.17758166152263 -0.22454832234012\\
1.17770374131338 0.330083166045311\\
1.17782582110412 0.669630054488099\\
1.17794790089487 0.706337826211643\\
1.17806998068561 0.440206481215945\\
1.17819206047636 -0.197017493547462\\
1.1783141402671 -0.816461141382277\\
1.17843622005785 -0.743045597935188\\
1.17855829984859 -0.4034987094924\\
1.17868037963934 0.121307714367651\\
1.17880245943008 0.531975910524806\\
1.17892453922082 0.568683682248351\\
1.17904661901157 0.293375394321767\\
1.17916869880231 -0.0708345282477775\\
1.17929077859306 -0.261256094063665\\
1.1794128583838 -0.270433036994551\\
1.17953493817455 0.0513335245196444\\
1.17965701796529 0.141955835962145\\
1.17977909775604 -0.0513335245196444\\
1.17990117754678 -0.112130771436765\\
1.18002325733753 -0.22454832234012\\
1.18014533712827 -0.206194436478348\\
1.18026741691902 0.0800114711786636\\
1.18038949670976 0.440206481215945\\
1.1805115765005 0.4034987094924\\
1.18063365629125 0.0662460567823344\\
1.18075573608199 -0.197017493547462\\
1.18087781587274 -0.458560367077717\\
1.18099989566348 -0.330083166045311\\
1.18112197545423 0.0662460567823344\\
1.18124405524497 0.348437051907083\\
1.18136613503572 0.513622024663034\\
1.18148821482646 0.270433036994551\\
1.18161029461721 -0.252079151132779\\
1.18173237440795 -0.632922282764554\\
1.1818544541987 -0.47691425293949\\
1.18197653398944 -0.141955835962145\\
1.18209861378018 0.330083166045311\\
1.18222069357093 0.743045597935188\\
1.18234277336167 0.568683682248351\\
1.18246485315242 0.121307714367651\\
1.18258693294316 -0.385144823630628\\
1.18270901273391 -0.596214511041009\\
1.18283109252465 -0.47691425293949\\
1.1829531723154 -0.0937768855749928\\
1.18307525210614 0.187840550616576\\
1.18319733189689 0.22454832234012\\
1.18331941168763 0.261256094063665\\
1.18344149147838 0.0197877831947233\\
1.18356357126912 -0.0662460567823344\\
1.18368565105986 0.141955835962145\\
1.18380773085061 0.270433036994551\\
1.18392981064135 0.141955835962145\\
1.1840518904321 -0.141955835962145\\
1.18417397022284 -0.385144823630628\\
1.18429605001359 -0.596214511041009\\
1.18441812980433 -0.4034987094924\\
1.18454020959508 0.135073128763981\\
1.18466228938582 0.596214511041009\\
1.18478436917657 0.779753369658732\\
1.18490644896731 0.458560367077717\\
1.18502852875806 -0.121307714367651\\
1.1851506085488 -0.568683682248351\\
1.18527268833955 -0.632922282764554\\
1.18539476813029 -0.421852595354173\\
1.18551684792103 0.0559219959850875\\
1.18563892771178 0.568683682248351\\
1.18576100750252 0.440206481215945\\
1.18588308729327 0.135073128763981\\
1.18600516708401 -0.22454832234012\\
1.18612724687476 -0.4034987094924\\
1.1862493266655 -0.233725265271007\\
1.18637140645625 0.0559219959850875\\
1.18649348624699 0.330083166045311\\
1.18661556603774 0.311729280183539\\
1.18673764582848 0.160309721823917\\
1.18685972561923 -0.22454832234012\\
1.18698180540997 -0.366790937768856\\
1.18710388520071 -0.0891884141095498\\
1.18722596499146 0.0891884141095498\\
1.1873480447822 0.141955835962145\\
1.18747012457295 0.0754229997132205\\
1.18759220436369 -0.0467450530542013\\
1.18771428415444 -0.311729280183539\\
1.18783636394518 -0.206194436478348\\
1.18795844373593 0.197017493547462\\
1.18808052352667 0.495268138801262\\
1.18820260331742 0.568683682248351\\
1.18832468310816 0.121307714367651\\
1.18844676289891 -0.366790937768856\\
1.18856884268965 -0.669630054488099\\
1.18869092248039 -0.743045597935188\\
1.18881300227114 -0.311729280183539\\
1.18893508206188 0.348437051907083\\
1.18905716185263 0.889876684829366\\
1.18917924164337 0.669630054488099\\
1.18930132143412 0.270433036994551\\
1.18942340122486 -0.169486664754804\\
1.18954548101561 -0.596214511041009\\
1.18966756080635 -0.550329796386579\\
1.1897896405971 -0.261256094063665\\
1.18991172038784 0.135073128763981\\
1.19003380017859 0.233725265271007\\
1.19015587996933 0.17866360768569\\
1.19027795976007 -0.0255233725265271\\
1.19040003955082 -0.0662460567823344\\
1.19052211934156 0.0983653570404359\\
1.19064419913231 0.160309721823917\\
1.19076627892305 0.270433036994551\\
1.1908883587138 0.121307714367651\\
1.19101043850454 -0.22454832234012\\
1.19113251829529 -0.513622024663034\\
1.19125459808603 -0.385144823630628\\
1.19137667787678 -0.0444508173214798\\
1.19149875766752 0.311729280183539\\
1.19162083745827 0.568683682248351\\
1.19174291724901 0.348437051907083\\
1.19186499703975 -0.0329796386578721\\
1.1919870768305 -0.440206481215945\\
1.19210915662124 -0.596214511041009\\
1.19223123641199 -0.242902208201893\\
1.19235331620273 0.348437051907083\\
1.19247539599348 0.632922282764554\\
1.19259747578422 0.440206481215945\\
1.19271955557497 0.0983653570404359\\
1.19284163536571 -0.385144823630628\\
1.19296371515646 -0.743045597935188\\
1.1930857949472 -0.495268138801262\\
1.19320787473795 0.0140521938629194\\
1.19332995452869 0.440206481215945\\
1.19345203431943 0.596214511041009\\
1.19357411411018 0.385144823630628\\
1.19369619390092 0.0243762546601663\\
1.19381827369167 -0.233725265271007\\
1.19394035348241 -0.293375394321767\\
1.19406243327316 -0.233725265271007\\
1.1941845130639 0.0375681101233152\\
1.19430659285465 0.0467450530542013\\
1.19442867264539 -0.197017493547462\\
1.19455075243614 -0.160309721823917\\
1.19467283222688 -0.0628047031832521\\
1.19479491201763 0.141955835962145\\
1.19491699180837 0.421852595354173\\
1.19503907159911 0.596214511041009\\
1.19516115138986 0.293375394321767\\
1.1952832311806 -0.270433036994551\\
1.19540531097135 -0.596214511041009\\
1.19552739076209 -0.779753369658732\\
1.19564947055284 -0.4034987094924\\
1.19577155034358 0.187840550616576\\
1.19589363013433 0.596214511041009\\
1.19601570992507 0.743045597935188\\
1.19613778971582 0.385144823630628\\
1.19625986950656 -0.135073128763981\\
1.19638194929731 -0.632922282764554\\
1.19650402908805 -0.440206481215945\\
1.1966261088788 -0.151132778893031\\
1.19674818866954 0.151132778893031\\
1.19687026846028 0.4034987094924\\
1.19699234825103 0.261256094063665\\
1.19711442804177 -0.0536277602523659\\
1.19723650783252 -0.4034987094924\\
1.19735858762326 -0.215371379409234\\
1.19748066741401 -0.020934901061084\\
1.19760274720475 0.261256094063665\\
1.1977248269955 0.252079151132779\\
1.19784690678624 0.0186406653283625\\
1.19796898657699 -0.00344135359908231\\
1.19809106636773 -0.233725265271007\\
1.19821314615848 -0.160309721823917\\
1.19833522594922 0.112130771436765\\
1.19845730573996 0.385144823630628\\
1.19857938553071 0.116719242902208\\
1.19870146532145 -0.252079151132779\\
1.1988235451122 -0.366790937768856\\
1.19894562490294 -0.495268138801262\\
1.19906770469369 -0.0983653570404359\\
1.19918978448443 0.366790937768856\\
1.19931186427518 0.743045597935188\\
1.19943394406592 0.706337826211643\\
1.19955602385667 0.17866360768569\\
1.19967810364741 -0.495268138801262\\
1.19980018343816 -0.889876684829366\\
1.1999222632289 -0.596214511041009\\
1.20004434301964 -0.293375394321767\\
1.20016642281039 0.270433036994551\\
1.20028850260113 0.743045597935188\\
1.20041058239188 0.596214511041009\\
1.20053266218262 0.141955835962145\\
1.20065474197337 -0.187840550616576\\
1.20077682176411 -0.261256094063665\\
1.20089890155486 -0.233725265271007\\
1.2010209813456 -0.0232291367938056\\
1.20114306113635 -0.0220820189274448\\
1.20126514092709 -0.102953828505879\\
1.20138722071784 -0.0754229997132205\\
1.20150930050858 -0.22454832234012\\
1.20163138029932 -0.0708345282477775\\
1.20175346009007 0.4034987094924\\
1.20187553988081 0.568683682248351\\
1.20199761967156 0.22454832234012\\
1.2021196994623 -0.058216231717809\\
1.20224177925305 -0.366790937768856\\
1.20236385904379 -0.596214511041009\\
1.20248593883454 -0.348437051907083\\
1.20260801862528 0.116719242902208\\
1.20273009841603 0.596214511041009\\
1.20285217820677 0.513622024663034\\
1.20297425799752 0.169486664754804\\
1.20309633778826 -0.366790937768856\\
1.203218417579 -0.531975910524806\\
1.20334049736975 -0.348437051907083\\
1.20346257716049 -0.112130771436765\\
1.20358465695124 0.440206481215945\\
1.20370673674198 0.706337826211643\\
1.20382881653273 0.4034987094924\\
1.20395089632347 -0.187840550616576\\
1.20407297611422 -0.440206481215945\\
1.20419505590496 -0.458560367077717\\
1.20431713569571 -0.348437051907083\\
1.20443921548645 0.0255233725265271\\
1.2045612952772 0.252079151132779\\
1.20468337506794 0.293375394321767\\
1.20480545485868 0.169486664754804\\
1.20492753464943 -0.0444508173214798\\
1.20504961444017 -0.0398623458560367\\
1.20517169423092 0.215371379409234\\
1.20529377402166 0.206194436478348\\
1.20541585381241 -0.0444508173214798\\
1.20553793360315 -0.252079151132779\\
1.2056600133939 -0.458560367077717\\
1.20578209318464 -0.596214511041009\\
1.20590417297539 -0.17866360768569\\
1.20602625276613 0.440206481215945\\
1.20614833255688 0.743045597935188\\
1.20627041234762 0.706337826211643\\
1.20639249213836 0.270433036994551\\
1.20651457192911 -0.330083166045311\\
1.20663665171985 -0.779753369658732\\
1.2067587315106 -0.706337826211643\\
1.20688081130134 -0.311729280183539\\
1.20700289109209 0.330083166045311\\
1.20712497088283 0.632922282764554\\
1.20724705067358 0.421852595354173\\
1.20736913046432 0.125896185833094\\
1.20749121025507 -0.187840550616576\\
1.20761329004581 -0.4034987094924\\
1.20773536983656 -0.252079151132779\\
1.2078574496273 0.160309721823917\\
1.20797952941804 0.252079151132779\\
1.20810160920879 0.160309721823917\\
1.20822368899953 -0.00286779466590192\\
1.20834576879028 -0.279609979925437\\
1.20846784858102 -0.22454832234012\\
1.20858992837177 0.016346429595641\\
1.20871200816251 0.141955835962145\\
1.20883408795326 0.261256094063665\\
1.208956167744 0.102953828505879\\
1.20907824753475 -0.233725265271007\\
1.20920032732549 -0.385144823630628\\
1.20932240711624 -0.0329796386578721\\
1.20944448690698 0.311729280183539\\
1.20956656669772 0.4034987094924\\
1.20968864648847 0.440206481215945\\
1.20981072627921 -0.00286779466590192\\
1.20993280606996 -0.495268138801262\\
1.2100548858607 -0.779753369658732\\
1.21017696565145 -0.596214511041009\\
1.21029904544219 0.0467450530542013\\
1.21042112523294 0.669630054488099\\
1.21054320502368 0.853168913105822\\
1.21066528481443 0.47691425293949\\
1.21078736460517 0.160309721823917\\
1.21090944439592 -0.458560367077717\\
1.21103152418666 -0.816461141382277\\
1.21115360397741 -0.495268138801262\\
1.21127568376815 -0.0197877831947233\\
1.21139776355889 0.242902208201893\\
1.21151984334964 0.261256094063665\\
1.21164192314038 0.252079151132779\\
1.21176400293113 0.0129050759965586\\
1.21188608272187 -0.0352738743905936\\
1.21200816251262 0.0708345282477775\\
1.21213024230336 0.125896185833094\\
1.21225232209411 0.215371379409234\\
1.21237440188485 -0.0708345282477775\\
1.2124964816756 -0.440206481215945\\
1.21261856146634 -0.513622024663034\\
1.21274064125709 -0.197017493547462\\
1.21286272104783 0.116719242902208\\
1.21298480083857 0.458560367077717\\
1.21310688062932 0.669630054488099\\
1.21322896042006 0.293375394321767\\
1.21335104021081 -0.233725265271007\\
1.21347312000155 -0.632922282764554\\
1.2135951997923 -0.495268138801262\\
1.21371727958304 -0.112130771436765\\
1.21383935937379 0.385144823630628\\
1.21396143916453 0.550329796386579\\
1.21408351895528 0.348437051907083\\
1.21420559874602 0.0375681101233152\\
1.21432767853677 -0.568683682248351\\
1.21444975832751 -0.632922282764554\\
1.21457183811825 -0.233725265271007\\
1.214693917909 0.233725265271007\\
1.21481599769974 0.458560367077717\\
1.21493807749049 0.47691425293949\\
1.21506015728123 0.311729280183539\\
1.21518223707198 -0.206194436478348\\
1.21530431686272 -0.366790937768856\\
1.21542639665347 -0.270433036994551\\
1.21554847644421 -0.0605104674505305\\
1.21567055623496 0.0983653570404359\\
1.2157926360257 -0.0490392887869229\\
1.21591471581645 -0.160309721823917\\
1.21603679560719 -0.0845999426441067\\
1.21615887539793 0.020934901061084\\
1.21628095518868 0.242902208201893\\
1.21640303497942 0.568683682248351\\
1.21652511477017 0.531975910524806\\
1.21664719456091 -0.0232291367938056\\
1.21676927435166 -0.550329796386579\\
1.2168913541424 -0.743045597935188\\
1.21701343393315 -0.669630054488099\\
1.21713551372389 -0.130484657298537\\
1.21725759351464 0.513622024663034\\
1.21737967330538 0.889876684829366\\
1.21750175309613 0.632922282764554\\
1.21762383288687 0.112130771436765\\
1.21774591267761 -0.4034987094924\\
1.21786799246836 -0.632922282764554\\
1.2179900722591 -0.348437051907083\\
1.21811215204985 -0.130484657298537\\
1.21823423184059 0.270433036994551\\
1.21835631163134 0.4034987094924\\
1.21847839142208 0.125896185833094\\
1.21860047121283 -0.169486664754804\\
1.21872255100357 -0.242902208201893\\
1.21884463079432 -0.0243762546601663\\
1.21896671058506 0.0891884141095498\\
1.21908879037581 0.261256094063665\\
1.21921087016655 0.141955835962145\\
1.21933294995729 -0.0662460567823344\\
1.21945502974804 -0.215371379409234\\
1.21957710953878 -0.293375394321767\\
1.21969918932953 -0.0129050759965586\\
1.21982126912027 0.293375394321767\\
1.21994334891102 0.311729280183539\\
1.22006542870176 -0.0129050759965586\\
1.22018750849251 -0.169486664754804\\
1.22030958828325 -0.385144823630628\\
1.220431668074 -0.4034987094924\\
1.22055374786474 0.0605104674505305\\
1.22067582765549 0.568683682248351\\
1.22079790744623 0.706337826211643\\
1.22091998723697 0.330083166045311\\
1.22104206702772 -0.135073128763981\\
1.22116414681846 -0.596214511041009\\
1.22128622660921 -0.779753369658732\\
1.22140830639995 -0.495268138801262\\
1.2215303861907 0.0220820189274448\\
1.22165246598144 0.632922282764554\\
1.22177454577219 0.706337826211643\\
1.22189662556293 0.366790937768856\\
1.22201870535368 -0.0306854029251506\\
1.22214078514442 -0.252079151132779\\
1.22226286493517 -0.348437051907083\\
1.22238494472591 -0.261256094063665\\
1.22250702451666 0.0220820189274448\\
1.2226291043074 0.00688270719816461\\
1.22275118409814 -0.0559219959850875\\
1.22287326388889 -0.17866360768569\\
1.22299534367963 -0.0845999426441067\\
1.22311742347038 0.233725265271007\\
1.22323950326112 0.458560367077717\\
1.22336158305187 0.4034987094924\\
1.22348366284261 0.125896185833094\\
1.22360574263336 -0.160309721823917\\
1.2237278224241 -0.632922282764554\\
1.22384990221485 -0.596214511041009\\
1.22397198200559 -0.135073128763981\\
1.22409406179634 0.366790937768856\\
1.22421614158708 0.568683682248351\\
1.22433822137782 0.4034987094924\\
1.22446030116857 0.116719242902208\\
1.22458238095931 -0.421852595354173\\
1.22470446075006 -0.513622024663034\\
1.2248265405408 -0.348437051907083\\
1.22494862033155 0.187840550616576\\
1.22507070012229 0.531975910524806\\
1.22519277991304 0.440206481215945\\
1.22531485970378 0.116719242902208\\
1.22543693949453 -0.261256094063665\\
1.22555901928527 -0.440206481215945\\
1.22568109907602 -0.495268138801262\\
1.22580317886676 -0.0754229997132205\\
1.2259252586575 0.279609979925437\\
1.22604733844825 0.348437051907083\\
1.22616941823899 0.187840550616576\\
1.22629149802974 0.0421565815887582\\
1.22641357782048 -0.0708345282477775\\
1.22653565761123 -0.0220820189274448\\
1.22665773740197 0.102953828505879\\
1.22677981719272 0.0754229997132205\\
1.22690189698346 -0.0398623458560367\\
1.22702397677421 -0.330083166045311\\
1.22714605656495 -0.513622024663034\\
1.2272681363557 -0.330083166045311\\
1.22739021614644 0.197017493547462\\
1.22751229593718 0.531975910524806\\
1.22763437572793 0.706337826211643\\
1.22775645551867 0.550329796386579\\
1.22787853530942 0.00573558933180384\\
1.22800061510016 -0.596214511041009\\
1.22812269489091 -0.889876684829366\\
1.22824477468165 -0.531975910524806\\
1.2283668544724 -0.0306854029251506\\
1.22848893426314 0.513622024663034\\
1.22861101405389 0.596214511041009\\
1.22873309384463 0.4034987094924\\
1.22885517363538 0.0467450530542013\\
1.22897725342612 -0.385144823630628\\
1.22909933321686 -0.311729280183539\\
1.22922141300761 -0.121307714367651\\
1.22934349279835 0.197017493547462\\
1.2294655725891 0.102953828505879\\
1.22958765237984 0.0140521938629194\\
1.22970973217059 -0.102953828505879\\
1.22983181196133 -0.293375394321767\\
1.22995389175208 -0.0628047031832521\\
1.23007597154282 0.151132778893031\\
1.23019805133357 0.4034987094924\\
1.23032013112431 0.17866360768569\\
1.23044221091506 -0.0891884141095498\\
1.2305642907058 -0.366790937768856\\
1.23068637049654 -0.311729280183539\\
1.23080845028729 0.0467450530542013\\
1.23093053007803 0.215371379409234\\
1.23105260986878 0.568683682248351\\
1.23117468965952 0.293375394321767\\
1.23129676945027 -0.17866360768569\\
1.23141884924101 -0.669630054488099\\
1.23154092903176 -0.632922282764554\\
1.2316630088225 -0.215371379409234\\
1.23178508861325 0.233725265271007\\
1.23190716840399 0.779753369658732\\
1.23202924819474 0.669630054488099\\
1.23215132798548 0.385144823630628\\
1.23227340777623 -0.252079151132779\\
1.23239548756697 -0.706337826211643\\
1.23251756735771 -0.632922282764554\\
1.23263964714846 -0.261256094063665\\
1.2327617269392 0.141955835962145\\
1.23288380672995 0.242902208201893\\
1.23300588652069 0.440206481215945\\
1.23312796631144 0.160309721823917\\
1.23325004610218 -0.0754229997132205\\
1.23337212589293 -0.0352738743905936\\
1.23349420568367 0.151132778893031\\
1.23361628547442 0.17866360768569\\
1.23373836526516 -0.058216231717809\\
1.23386044505591 -0.293375394321767\\
1.23398252484665 -0.495268138801262\\
1.23410460463739 -0.348437051907083\\
1.23422668442814 -0.0662460567823344\\
1.23434876421888 0.385144823630628\\
1.23447084400963 0.779753369658732\\
1.23459292380037 0.568683682248351\\
1.23471500359112 -0.0375681101233152\\
1.23483708338186 -0.513622024663034\\
1.23495916317261 -0.596214511041009\\
1.23508124296335 -0.421852595354173\\
1.2352033227541 -0.00458847146544307\\
1.23532540254484 0.495268138801262\\
1.23544748233559 0.568683682248351\\
1.23556956212633 0.233725265271007\\
1.23569164191707 -0.252079151132779\\
1.23581372170782 -0.47691425293949\\
1.23593580149856 -0.348437051907083\\
1.23605788128931 -0.0662460567823344\\
1.23617996108005 0.270433036994551\\
1.2363020408708 0.440206481215945\\
1.23642412066154 0.385144823630628\\
1.23654620045229 -0.00630914826498423\\
1.23666828024303 -0.366790937768856\\
1.23679036003378 -0.279609979925437\\
1.23691243982452 -0.121307714367651\\
1.23703451961527 -0.00458847146544307\\
1.23715659940601 0.0708345282477775\\
1.23727867919675 0.0375681101233152\\
1.2374007589875 -0.0536277602523659\\
1.23752283877824 -0.0891884141095498\\
1.23764491856899 0.0536277602523659\\
1.23776699835973 0.348437051907083\\
1.23788907815048 0.531975910524806\\
1.23801115794122 0.17866360768569\\
1.23813323773197 -0.311729280183539\\
1.23825531752271 -0.550329796386579\\
1.23837739731346 -0.706337826211643\\
1.2384994771042 -0.47691425293949\\
1.23862155689495 0.215371379409234\\
1.23874363668569 0.816461141382277\\
1.23886571647643 0.816461141382277\\
1.23898779626718 0.385144823630628\\
1.23910987605792 -0.0662460567823344\\
1.23923195584867 -0.568683682248351\\
1.23935403563941 -0.632922282764554\\
1.23947611543016 -0.4034987094924\\
1.2395981952209 -0.00458847146544307\\
1.23972027501165 0.421852595354173\\
1.23984235480239 0.366790937768856\\
1.23996443459314 0.0197877831947233\\
1.24008651438388 -0.151132778893031\\
1.24020859417463 -0.00114711786636077\\
1.24033067396537 -0.016346429595641\\
1.24045275375611 0.112130771436765\\
1.24057483354686 0.215371379409234\\
1.2406969133376 0.00802982506452538\\
1.24081899312835 -0.252079151132779\\
1.24094107291909 -0.385144823630628\\
1.24106315270984 -0.187840550616576\\
1.24118523250058 0.169486664754804\\
1.24130731229133 0.421852595354173\\
1.24142939208207 0.293375394321767\\
1.24155147187282 0.0197877831947233\\
1.24167355166356 -0.233725265271007\\
1.24179563145431 -0.531975910524806\\
1.24191771124505 -0.348437051907083\\
1.24203979103579 0.261256094063665\\
1.24216187082654 0.632922282764554\\
1.24228395061728 0.568683682248351\\
1.24240603040803 0.125896185833094\\
1.24252811019877 -0.293375394321767\\
1.24265018998952 -0.743045597935188\\
1.24277226978026 -0.632922282764554\\
1.24289434957101 -0.215371379409234\\
1.24301642936175 0.385144823630628\\
1.2431385091525 0.743045597935188\\
1.24326058894324 0.531975910524806\\
1.24338266873399 0.141955835962145\\
1.24350474852473 -0.215371379409234\\
1.24362682831547 -0.366790937768856\\
1.24374890810622 -0.385144823630628\\
1.24387098789696 -0.0662460567823344\\
1.24399306768771 0.0845999426441067\\
1.24411514747845 -0.0266704903928879\\
1.2442372272692 -0.125896185833094\\
1.24435930705994 -0.112130771436765\\
1.24448138685069 0.0937768855749928\\
1.24460346664143 0.330083166045311\\
1.24472554643218 0.531975910524806\\
1.24484762622292 0.311729280183539\\
1.24496970601367 -0.0536277602523659\\
1.24509178580441 -0.531975910524806\\
1.24521386559515 -0.779753369658732\\
1.2453359453859 -0.440206481215945\\
1.24545802517664 0.130484657298537\\
1.24558010496739 0.550329796386579\\
1.24570218475813 0.632922282764554\\
1.24582426454888 0.458560367077717\\
1.24594634433962 -0.107542299971322\\
1.24606842413037 -0.596214511041009\\
1.24619050392111 -0.513622024663034\\
1.24631258371186 -0.160309721823917\\
1.2464346635026 0.252079151132779\\
1.24655674329335 0.47691425293949\\
1.24667882308409 0.293375394321767\\
1.24680090287484 0.0129050759965586\\
1.24692298266558 -0.330083166045311\\
1.24704506245632 -0.458560367077717\\
1.24716714224707 -0.215371379409234\\
1.24728922203781 0.187840550616576\\
1.24741130182856 0.330083166045311\\
1.2475333816193 0.160309721823917\\
1.24765546141005 0.116719242902208\\
1.24777754120079 -0.0754229997132205\\
1.24789962099154 -0.151132778893031\\
1.24802170078228 -0.0243762546601663\\
1.24814378057303 0.151132778893031\\
1.24826586036377 0.107542299971322\\
1.24838794015452 -0.160309721823917\\
1.24851001994526 -0.366790937768856\\
1.248632099736 -0.458560367077717\\
1.24875417952675 -0.0490392887869229\\
1.24887625931749 0.348437051907083\\
1.24899833910824 0.632922282764554\\
1.24912041889898 0.706337826211643\\
1.24924249868973 0.261256094063665\\
1.24936457848047 -0.4034987094924\\
1.24948665827122 -0.816461141382277\\
1.24960873806196 -0.706337826211643\\
1.24973081785271 -0.348437051907083\\
1.24985289764345 0.293375394321767\\
1.2499749774342 0.669630054488099\\
1.25009705722494 0.596214511041009\\
1.25021913701568 0.293375394321767\\
1.25034121680643 -0.187840550616576\\
1.25046329659717 -0.4034987094924\\
1.25058537638792 -0.293375394321767\\
1.25070745617866 0.0243762546601663\\
1.25082953596941 0.0352738743905936\\
1.25095161576015 0.0329796386578721\\
1.2510736955509 -0.00802982506452538\\
1.25119577534164 -0.215371379409234\\
1.25131785513239 -0.169486664754804\\
1.25143993492313 0.151132778893031\\
1.25156201471388 0.47691425293949\\
1.25168409450462 0.311729280183539\\
1.25180617429536 0.0329796386578721\\
1.25192825408611 -0.261256094063665\\
1.25205033387685 -0.47691425293949\\
1.2521724136676 -0.293375394321767\\
1.25229449345834 0.0266704903928879\\
1.25241657324909 0.440206481215945\\
1.25253865303983 0.568683682248351\\
1.25266073283058 0.160309721823917\\
1.25278281262132 -0.366790937768856\\
1.25290489241207 -0.568683682248351\\
1.25302697220281 -0.421852595354173\\
1.25314905199356 -0.125896185833094\\
1.2532711317843 0.458560367077717\\
1.25339321157504 0.816461141382277\\
1.25351529136579 0.47691425293949\\
1.25363737115653 -0.0140521938629194\\
1.25375945094728 -0.495268138801262\\
1.25388153073802 -0.669630054488099\\
1.25400361052877 -0.440206481215945\\
1.25412569031951 -0.0421565815887582\\
1.25424777011026 0.261256094063665\\
1.254369849901 0.421852595354173\\
1.25449192969175 0.366790937768856\\
1.25461400948249 -0.0628047031832521\\
1.25473608927324 -0.0891884141095498\\
1.25485816906398 0.0754229997132205\\
1.25498024885472 0.160309721823917\\
1.25510232864547 -0.0467450530542013\\
1.25522440843621 -0.215371379409234\\
1.25534648822696 -0.330083166045311\\
1.2554685680177 -0.513622024663034\\
1.25559064780845 -0.17866360768569\\
1.25571272759919 0.252079151132779\\
1.25583480738994 0.706337826211643\\
1.25595688718068 0.706337826211643\\
1.25607896697143 0.242902208201893\\
1.25620104676217 -0.252079151132779\\
1.25632312655292 -0.596214511041009\\
1.25644520634366 -0.669630054488099\\
1.2565672861344 -0.421852595354173\\
1.25668936592515 0.311729280183539\\
1.25681144571589 0.632922282764554\\
1.25693352550664 0.421852595354173\\
1.25705560529738 0.135073128763981\\
1.25717768508813 -0.22454832234012\\
1.25729976487887 -0.440206481215945\\
1.25742184466962 -0.311729280183539\\
1.25754392446036 0.0662460567823344\\
1.25766600425111 0.330083166045311\\
1.25778808404185 0.366790937768856\\
1.2579101638326 0.0891884141095498\\
1.25803224362334 -0.270433036994551\\
1.25815432341409 -0.270433036994551\\
1.25827640320483 -0.141955835962145\\
1.25839848299557 -0.0306854029251506\\
1.25852056278632 0.169486664754804\\
1.25864264257706 0.197017493547462\\
1.25876472236781 -0.0329796386578721\\
1.25888680215855 -0.252079151132779\\
1.2590088819493 -0.0937768855749928\\
1.25913096174004 0.22454832234012\\
1.25925304153079 0.366790937768856\\
1.25937512132153 0.385144823630628\\
1.25949720111228 -0.00946372239747634\\
1.25961928090302 -0.4034987094924\\
1.25974136069377 -0.632922282764554\\
1.25986344048451 -0.669630054488099\\
1.25998552027525 -0.0708345282477775\\
1.260107600066 0.596214511041009\\
1.26022967985674 0.853168913105822\\
1.26035175964749 0.596214511041009\\
1.26047383943823 0.197017493547462\\
1.26059591922898 -0.348437051907083\\
1.26071799901972 -0.816461141382277\\
1.26084007881047 -0.596214511041009\\
1.26096215860121 -0.187840550616576\\
1.26108423839196 0.252079151132779\\
1.2612063181827 0.440206481215945\\
1.26132839797345 0.293375394321767\\
1.26145047776419 0.0559219959850875\\
1.26157255755493 -0.0845999426441067\\
1.26169463734568 -0.058216231717809\\
1.26181671713642 -0.00172067679954115\\
1.26193879692717 0.160309721823917\\
1.26206087671791 0.0662460567823344\\
1.26218295650866 -0.252079151132779\\
1.2623050362994 -0.311729280183539\\
1.26242711609015 -0.233725265271007\\
1.26254919588089 -0.0421565815887582\\
1.26267127567164 0.311729280183539\\
1.26279335546238 0.513622024663034\\
1.26291543525313 0.270433036994551\\
1.26303751504387 -0.116719242902208\\
1.26315959483461 -0.385144823630628\\
1.26328167462536 -0.513622024663034\\
1.2634037544161 -0.107542299971322\\
1.26352583420685 0.330083166045311\\
1.26364791399759 0.458560367077717\\
1.26376999378834 0.440206481215945\\
1.26389207357908 0.0490392887869229\\
1.26401415336983 -0.47691425293949\\
1.26413623316057 -0.743045597935188\\
1.26425831295132 -0.348437051907083\\
1.26438039274206 0.0800114711786636\\
1.26450247253281 0.495268138801262\\
1.26462455232355 0.669630054488099\\
1.26474663211429 0.421852595354173\\
1.26486871190504 -0.00229423573272154\\
1.26499079169578 -0.421852595354173\\
1.26511287148653 -0.47691425293949\\
1.26523495127727 -0.311729280183539\\
1.26535703106802 0.0605104674505305\\
1.26547911085876 0.0490392887869229\\
1.26560119064951 -0.00344135359908231\\
1.26572327044025 0.0662460567823344\\
1.265845350231 0.0117579581301979\\
1.26596743002174 0.116719242902208\\
1.26608950981249 0.366790937768856\\
1.26621158960323 0.458560367077717\\
1.26633366939397 0.116719242902208\\
1.26645574918472 -0.366790937768856\\
1.26657782897546 -0.669630054488099\\
1.26669990876621 -0.669630054488099\\
1.26682198855695 -0.215371379409234\\
1.2669440683477 0.311729280183539\\
1.26706614813844 0.743045597935188\\
1.26718822792919 0.743045597935188\\
1.26731030771993 0.279609979925437\\
1.26743238751068 -0.330083166045311\\
1.26755446730142 -0.669630054488099\\
1.26767654709217 -0.458560367077717\\
1.26779862688291 -0.17866360768569\\
1.26792070667365 0.311729280183539\\
1.2680427864644 0.47691425293949\\
1.26816486625514 0.233725265271007\\
1.26828694604589 -0.107542299971322\\
1.26840902583663 -0.348437051907083\\
1.26853110562738 -0.233725265271007\\
1.26865318541812 -0.0559219959850875\\
1.26877526520887 0.279609979925437\\
1.26889734499961 0.252079151132779\\
1.26901942479036 0.130484657298537\\
1.2691415045811 -0.0329796386578721\\
1.26926358437185 -0.293375394321767\\
1.26938566416259 -0.141955835962145\\
1.26950774395333 0.0937768855749928\\
1.26962982374408 0.252079151132779\\
1.26975190353482 0.0513335245196444\\
1.26987398332557 -0.151132778893031\\
1.26999606311631 -0.311729280183539\\
1.27011814290706 -0.385144823630628\\
1.2702402226978 0.0243762546601663\\
1.27036230248855 0.440206481215945\\
1.27048438227929 0.706337826211643\\
1.27060646207004 0.513622024663034\\
1.27072854186078 -0.0255233725265271\\
1.27085062165153 -0.596214511041009\\
1.27097270144227 -0.853168913105822\\
1.27109478123301 -0.596214511041009\\
1.27121686102376 -0.125896185833094\\
1.2713389408145 0.550329796386579\\
1.27146102060525 0.816461141382277\\
1.27158310039599 0.495268138801262\\
1.27170518018674 0.121307714367651\\
1.27182725997748 -0.311729280183539\\
1.27194933976823 -0.440206481215945\\
1.27207141955897 -0.293375394321767\\
1.27219349934972 -0.0306854029251506\\
1.27231557914046 0.0490392887869229\\
1.27243765893121 0.0232291367938056\\
1.27255973872195 -0.0375681101233152\\
1.2726818185127 -0.197017493547462\\
1.27280389830344 0.0891884141095498\\
1.27292597809418 0.348437051907083\\
1.27304805788493 0.4034987094924\\
1.27317013767567 0.206194436478348\\
1.27329221746642 -0.0891884141095498\\
1.27341429725716 -0.440206481215945\\
1.27353637704791 -0.596214511041009\\
1.27365845683865 -0.206194436478348\\
1.2737805366294 0.206194436478348\\
1.27390261642014 0.568683682248351\\
1.27402469621089 0.458560367077717\\
1.27414677600163 0.102953828505879\\
1.27426885579238 -0.348437051907083\\
1.27439093558312 -0.568683682248351\\
1.27451301537386 -0.366790937768856\\
1.27463509516461 0.016346429595641\\
1.27475717495535 0.632922282764554\\
1.2748792547461 0.596214511041009\\
1.27500133453684 0.206194436478348\\
1.27512341432759 -0.197017493547462\\
1.27524549411833 -0.513622024663034\\
1.27536757390908 -0.596214511041009\\
1.27548965369982 -0.311729280183539\\
1.27561173349057 0.242902208201893\\
1.27573381328131 0.421852595354173\\
1.27585589307206 0.366790937768856\\
1.2759779728628 0.107542299971322\\
1.27610005265354 -0.0937768855749928\\
1.27622213244429 -0.0232291367938056\\
1.27634421223503 0.0352738743905936\\
1.27646629202578 -0.007456266131345\\
1.27658837181652 -0.0106108402638371\\
1.27671045160727 -0.169486664754804\\
1.27683253139801 -0.495268138801262\\
1.27695461118876 -0.458560367077717\\
1.2770766909795 0.0708345282477775\\
1.27719877077025 0.531975910524806\\
1.27732085056099 0.706337826211643\\
1.27744293035174 0.550329796386579\\
1.27756501014248 0.130484657298537\\
1.27768708993322 -0.4034987094924\\
1.27780916972397 -0.853168913105822\\
1.27793124951471 -0.743045597935188\\
1.27805332930546 -0.116719242902208\\
1.2781754090962 0.495268138801262\\
1.27829748888695 0.596214511041009\\
1.27841956867769 0.4034987094924\\
1.27854164846844 0.17866360768569\\
1.27866372825918 -0.270433036994551\\
1.27878580804993 -0.495268138801262\\
1.27890788784067 -0.22454832234012\\
1.27902996763142 0.187840550616576\\
1.27915204742216 0.279609979925437\\
1.2792741272129 0.0845999426441067\\
1.27939620700365 -0.0329796386578721\\
1.27951828679439 -0.233725265271007\\
1.27964036658514 -0.215371379409234\\
1.27976244637588 -0.0375681101233152\\
1.27988452616663 0.206194436478348\\
1.28000660595737 0.311729280183539\\
1.28012868574812 0.0628047031832521\\
1.28025076553886 -0.233725265271007\\
1.28037284532961 -0.270433036994551\\
1.28049492512035 0.00688270719816461\\
1.2806170049111 0.169486664754804\\
1.28073908470184 0.330083166045311\\
1.28086116449258 0.366790937768856\\
1.28098324428333 -0.058216231717809\\
1.28110532407407 -0.513622024663034\\
1.28122740386482 -0.669630054488099\\
1.28134948365556 -0.348437051907083\\
1.28147156344631 0.151132778893031\\
1.28159364323705 0.632922282764554\\
1.2817157230278 0.743045597935188\\
1.28183780281854 0.458560367077717\\
1.28195988260929 0.0490392887869229\\
1.28208196240003 -0.632922282764554\\
1.28220404219078 -0.743045597935188\\
1.28232612198152 -0.421852595354173\\
1.28244820177226 0.0106108402638371\\
1.28257028156301 0.22454832234012\\
1.28269236135375 0.4034987094924\\
1.2828144411445 0.366790937768856\\
1.28293652093524 0.016346429595641\\
1.28305860072599 -0.0754229997132205\\
1.28318068051673 -0.020934901061084\\
1.28330276030748 0.141955835962145\\
1.28342484009822 0.0375681101233152\\
1.28354691988897 -0.242902208201893\\
1.28366899967971 -0.385144823630628\\
1.28379107947046 -0.293375394321767\\
1.2839131592612 -0.151132778893031\\
1.28403523905195 0.160309721823917\\
1.28415731884269 0.632922282764554\\
1.28427939863343 0.632922282764554\\
1.28440147842418 0.102953828505879\\
1.28452355821492 -0.385144823630628\\
1.28464563800567 -0.513622024663034\\
1.28476771779641 -0.47691425293949\\
1.28488979758716 -0.0662460567823344\\
1.2850118773779 0.4034987094924\\
1.28513395716865 0.568683682248351\\
1.28525603695939 0.348437051907083\\
1.28537811675014 -0.169486664754804\\
1.28550019654088 -0.531975910524806\\
1.28562227633163 -0.513622024663034\\
1.28574435612237 -0.0937768855749928\\
1.28586643591311 0.187840550616576\\
1.28598851570386 0.513622024663034\\
1.2861105954946 0.513622024663034\\
1.28623267528535 0.130484657298537\\
1.28635475507609 -0.311729280183539\\
1.28647683486684 -0.4034987094924\\
1.28659891465758 -0.215371379409234\\
1.28672099444833 -0.107542299971322\\
1.28684307423907 0.112130771436765\\
1.28696515402982 0.0708345282477775\\
1.28708723382056 -0.00229423573272154\\
1.28720931361131 -0.0628047031832521\\
1.28733139340205 -0.00516203039862346\\
1.28745347319279 0.261256094063665\\
1.28757555298354 0.495268138801262\\
1.28769763277428 0.348437051907083\\
1.28781971256503 -0.206194436478348\\
1.28794179235577 -0.495268138801262\\
1.28806387214652 -0.669630054488099\\
1.28818595193726 -0.568683682248351\\
1.28830803172801 0.0467450530542013\\
1.28843011151875 0.669630054488099\\
1.2885521913095 0.853168913105822\\
1.28867427110024 0.513622024663034\\
1.28879635089099 0.028391167192429\\
1.28891843068173 -0.531975910524806\\
1.28904051047247 -0.632922282764554\\
1.28916259026322 -0.421852595354173\\
1.28928467005396 -0.058216231717809\\
1.28940674984471 0.4034987094924\\
1.28952882963545 0.385144823630628\\
1.2896509094262 0.0708345282477775\\
1.28977298921694 -0.17866360768569\\
1.28989506900769 -0.0662460567823344\\
1.29001714879843 -0.107542299971322\\
1.29013922858918 0.0628047031832521\\
1.29026130837992 0.261256094063665\\
1.29038338817067 0.0937768855749928\\
1.29050546796141 -0.0845999426441067\\
1.29062754775215 -0.311729280183539\\
1.2907496275429 -0.215371379409234\\
1.29087170733364 0.0329796386578721\\
1.29099378712439 0.311729280183539\\
1.29111586691513 0.233725265271007\\
1.29123794670588 0.0467450530542013\\
1.29136002649662 -0.141955835962145\\
1.29148210628737 -0.47691425293949\\
1.29160418607811 -0.311729280183539\\
1.29172626586886 0.17866360768569\\
1.2918483456596 0.596214511041009\\
1.29197042545035 0.550329796386579\\
1.29209250524109 0.252079151132779\\
1.29221458503183 -0.242902208201893\\
1.29233666482258 -0.743045597935188\\
1.29245874461332 -0.706337826211643\\
1.29258082440407 -0.4034987094924\\
1.29270290419481 0.270433036994551\\
1.29282498398556 0.816461141382277\\
1.2929470637763 0.669630054488099\\
1.29306914356705 0.242902208201893\\
1.29319122335779 -0.102953828505879\\
1.29331330314854 -0.366790937768856\\
1.29343538293928 -0.495268138801262\\
1.29355746273003 -0.215371379409234\\
1.29367954252077 0.0891884141095498\\
1.29380162231152 0.0467450530542013\\
1.29392370210226 -0.007456266131345\\
1.294045781893 -0.0800114711786636\\
1.29416786168375 0.016346429595641\\
1.29428994147449 0.252079151132779\\
1.29441202126524 0.385144823630628\\
1.29453410105598 0.330083166045311\\
1.29465618084673 0.0174935474620017\\
1.29477826063747 -0.270433036994551\\
1.29490034042822 -0.706337826211643\\
1.29502242021896 -0.531975910524806\\
1.29514450000971 0.0708345282477775\\
1.29526657980045 0.4034987094924\\
1.2953886595912 0.632922282764554\\
1.29551073938194 0.440206481215945\\
1.29563281917268 0.0375681101233152\\
1.29575489896343 -0.550329796386579\\
1.29587697875417 -0.568683682248351\\
1.29599905854492 -0.279609979925437\\
1.29612113833566 0.242902208201893\\
1.29624321812641 0.596214511041009\\
1.29636529791715 0.348437051907083\\
1.2964873777079 0.102953828505879\\
1.29660945749864 -0.279609979925437\\
1.29673153728939 -0.513622024663034\\
1.29685361708013 -0.47691425293949\\
1.29697569687088 0.0662460567823344\\
1.29709777666162 0.4034987094924\\
1.29721985645236 0.293375394321767\\
1.29734193624311 0.22454832234012\\
1.29746401603385 0.0490392887869229\\
1.2975860958246 -0.160309721823917\\
1.29770817561534 -0.135073128763981\\
1.29783025540609 0.0220820189274448\\
1.29795233519683 0.0628047031832521\\
1.29807441498758 -0.0490392887869229\\
1.29819649477832 -0.293375394321767\\
1.29831857456907 -0.421852595354173\\
1.29844065435981 -0.130484657298537\\
1.29856273415056 0.279609979925437\\
1.2986848139413 0.495268138801262\\
1.29880689373204 0.669630054488099\\
1.29892897352279 0.458560367077717\\
1.29905105331353 -0.206194436478348\\
1.29917313310428 -0.779753369658732\\
1.29929521289502 -0.779753369658732\\
1.29941729268577 -0.440206481215945\\
1.29953937247651 0.107542299971322\\
1.29966145226726 0.632922282764554\\
1.299783532058 0.669630054488099\\
1.29990561184875 0.440206481215945\\
1.30002769163949 -0.0490392887869229\\
1.30014977143024 -0.513622024663034\\
1.30027185122098 -0.4034987094924\\
1.30039393101172 -0.058216231717809\\
1.30051601080247 0.112130771436765\\
1.30063809059321 0.112130771436765\\
1.30076017038396 0.141955835962145\\
1.3008822501747 -0.0891884141095498\\
1.30100432996545 -0.279609979925437\\
1.30112640975619 -0.0891884141095498\\
1.30124848954694 0.233725265271007\\
1.30137056933768 0.330083166045311\\
1.30149264912843 0.160309721823917\\
1.30161472891917 -0.0754229997132205\\
1.30173680870992 -0.311729280183539\\
1.30185888850066 -0.279609979925437\\
1.3019809682914 -0.112130771436765\\
1.30210304808215 0.197017493547462\\
1.30222512787289 0.550329796386579\\
1.30234720766364 0.270433036994551\\
1.30246928745438 -0.242902208201893\\
1.30259136724513 -0.495268138801262\\
1.30271344703587 -0.513622024663034\\
1.30283552682662 -0.22454832234012\\
1.30295760661736 0.293375394321767\\
1.30307968640811 0.779753369658732\\
1.30320176619885 0.669630054488099\\
1.3033238459896 0.22454832234012\\
1.30344592578034 -0.366790937768856\\
1.30356800557108 -0.779753369658732\\
1.30369008536183 -0.596214511041009\\
1.30381216515257 -0.22454832234012\\
1.30393424494332 0.169486664754804\\
1.30405632473406 0.495268138801262\\
1.30417840452481 0.531975910524806\\
1.30430048431555 0.112130771436765\\
1.3044225641063 -0.17866360768569\\
1.30454464389704 -0.0398623458560367\\
1.30466672368779 -0.0140521938629194\\
1.30478880347853 0.0266704903928879\\
1.30491088326928 -0.0845999426441067\\
1.30503296306002 -0.215371379409234\\
1.30515504285076 -0.311729280183539\\
1.30527712264151 -0.348437051907083\\
1.30539920243225 0.0352738743905936\\
1.305521282223 0.421852595354173\\
1.30564336201374 0.743045597935188\\
1.30576544180449 0.4034987094924\\
1.30588752159523 -0.0662460567823344\\
1.30600960138598 -0.421852595354173\\
1.30613168117672 -0.706337826211643\\
1.30625376096747 -0.495268138801262\\
1.30637584075821 0.0754229997132205\\
1.30649792054896 0.632922282764554\\
1.3066200003397 0.513622024663034\\
1.30674208013044 0.233725265271007\\
1.30686415992119 -0.160309721823917\\
1.30698623971193 -0.550329796386579\\
1.30710831950268 -0.4034987094924\\
1.30723039929342 -0.102953828505879\\
1.30735247908417 0.330083166045311\\
1.30747455887491 0.550329796386579\\
1.30759663866566 0.348437051907083\\
1.3077187184564 -0.0845999426441067\\
1.30784079824715 -0.385144823630628\\
1.30796287803789 -0.311729280183539\\
1.30808495782864 -0.242902208201893\\
1.30820703761938 0.0329796386578721\\
1.30832911741013 0.22454832234012\\
1.30845119720087 0.125896185833094\\
1.30857327699161 -0.0513335245196444\\
1.30869535678236 -0.0800114711786636\\
1.3088174365731 0.0845999426441067\\
1.30893951636385 0.293375394321767\\
1.30906159615459 0.366790937768856\\
1.30918367594534 0.0232291367938056\\
1.30930575573608 -0.252079151132779\\
1.30942783552683 -0.513622024663034\\
1.30954991531757 -0.706337826211643\\
1.30967199510832 -0.293375394321767\\
1.30979407489906 0.4034987094924\\
1.30991615468981 0.779753369658732\\
1.31003823448055 0.706337826211643\\
1.31016031427129 0.366790937768856\\
1.31028239406204 -0.206194436478348\\
1.31040447385278 -0.706337826211643\\
1.31052655364353 -0.669630054488099\\
1.31064863343427 -0.4034987094924\\
1.31077071322502 0.206194436478348\\
1.31089279301576 0.47691425293949\\
1.31101487280651 0.311729280183539\\
1.31113695259725 0.135073128763981\\
1.311259032388 -0.0845999426441067\\
1.31138111217874 -0.141955835962145\\
1.31150319196949 -0.169486664754804\\
1.31162527176023 0.169486664754804\\
1.31174735155097 0.22454832234012\\
1.31186943134172 -0.058216231717809\\
1.31199151113246 -0.22454832234012\\
1.31211359092321 -0.330083166045311\\
1.31223567071395 -0.102953828505879\\
1.3123577505047 0.121307714367651\\
1.31247983029544 0.4034987094924\\
1.31260191008619 0.293375394321767\\
1.31272398987693 0.0398623458560367\\
1.31284606966768 -0.293375394321767\\
1.31296814945842 -0.568683682248351\\
1.31309022924917 -0.151132778893031\\
1.31321230903991 0.279609979925437\\
1.31333438883065 0.568683682248351\\
1.3134564686214 0.440206481215945\\
1.31357854841214 0.141955835962145\\
1.31370062820289 -0.385144823630628\\
1.31382270799363 -0.779753369658732\\
1.31394478778438 -0.550329796386579\\
1.31406686757512 -0.125896185833094\\
1.31418894736587 0.495268138801262\\
1.31431102715661 0.706337826211643\\
1.31443310694736 0.513622024663034\\
1.3145551867381 0.160309721823917\\
1.31467726652885 -0.233725265271007\\
1.31479934631959 -0.531975910524806\\
1.31492142611033 -0.47691425293949\\
1.31504350590108 -0.0467450530542013\\
1.31516558569182 0.0398623458560367\\
1.31528766548257 0.0329796386578721\\
1.31540974527331 0.0800114711786636\\
1.31553182506406 0.0708345282477775\\
1.3156539048548 0.0754229997132205\\
1.31577598464555 0.242902208201893\\
1.31589806443629 0.366790937768856\\
1.31602014422704 0.215371379409234\\
1.31614222401778 -0.130484657298537\\
1.31626430380853 -0.550329796386579\\
1.31638638359927 -0.669630054488099\\
1.31650846339001 -0.293375394321767\\
1.31663054318076 0.141955835962145\\
1.3167526229715 0.513622024663034\\
1.31687470276225 0.706337826211643\\
1.31699678255299 0.421852595354173\\
1.31711886234374 -0.17866360768569\\
1.31724094213448 -0.596214511041009\\
1.31736302192523 -0.513622024663034\\
1.31748510171597 -0.169486664754804\\
1.31760718150672 0.187840550616576\\
1.31772926129746 0.47691425293949\\
1.31785134108821 0.385144823630628\\
1.31797342087895 0.0266704903928879\\
1.31809550066969 -0.330083166045311\\
1.31821758046044 -0.495268138801262\\
1.31833966025118 -0.17866360768569\\
1.31846174004193 0.151132778893031\\
1.31858381983267 0.311729280183539\\
1.31870589962342 0.252079151132779\\
1.31882797941416 0.206194436478348\\
1.31895005920491 -0.0536277602523659\\
1.31907213899565 -0.311729280183539\\
1.3191942187864 -0.0800114711786636\\
1.31931629857714 0.058216231717809\\
1.31943837836789 0.0708345282477775\\
1.31956045815863 -0.130484657298537\\
1.31968253794938 -0.215371379409234\\
1.31980461774012 -0.22454832234012\\
1.31992669753086 -0.0800114711786636\\
1.32004877732161 0.279609979925437\\
1.32017085711235 0.531975910524806\\
1.3202929369031 0.669630054488099\\
1.32041501669384 0.160309721823917\\
1.32053709648459 -0.440206481215945\\
1.32065917627533 -0.743045597935188\\
1.32078125606608 -0.669630054488099\\
1.32090333585682 -0.293375394321767\\
1.32102541564757 0.261256094063665\\
1.32114749543831 0.779753369658732\\
1.32126957522906 0.669630054488099\\
1.3213916550198 0.279609979925437\\
1.32151373481054 -0.197017493547462\\
1.32163581460129 -0.47691425293949\\
1.32175789439203 -0.366790937768856\\
1.32187997418278 -0.187840550616576\\
1.32200205397352 0.0800114711786636\\
1.32212413376427 0.141955835962145\\
1.32224621355501 0.135073128763981\\
1.32236829334576 -0.116719242902208\\
1.3224903731365 -0.17866360768569\\
1.32261245292725 0.116719242902208\\
1.32273453271799 0.261256094063665\\
1.32285661250874 0.261256094063665\\
1.32297869229948 0.058216231717809\\
1.32310077209022 -0.130484657298537\\
1.32322285188097 -0.348437051907083\\
1.32334493167171 -0.366790937768856\\
1.32346701146246 -0.0421565815887582\\
1.3235890912532 0.311729280183539\\
1.32371117104395 0.513622024663034\\
1.32383325083469 0.160309721823917\\
1.32395533062544 -0.197017493547462\\
1.32407741041618 -0.4034987094924\\
1.32419949020693 -0.440206481215945\\
1.32432156999767 -0.17866360768569\\
1.32444364978842 0.366790937768856\\
1.32456572957916 0.743045597935188\\
1.3246878093699 0.458560367077717\\
1.32480988916065 0.00946372239747634\\
1.32493196895139 -0.421852595354173\\
1.32505404874214 -0.632922282764554\\
1.32517612853288 -0.513622024663034\\
1.32529820832363 -0.135073128763981\\
1.32542028811437 0.348437051907083\\
1.32554236790512 0.531975910524806\\
1.32566444769586 0.4034987094924\\
1.32578652748661 0.0467450530542013\\
1.32590860727735 -0.102953828505879\\
1.3260306870681 -0.0891884141095498\\
1.32615276685884 -0.121307714367651\\
1.32627484664958 -0.116719242902208\\
1.32639692644033 -0.0559219959850875\\
1.32651900623107 -0.187840550616576\\
1.32664108602182 -0.330083166045311\\
1.32676316581256 -0.135073128763981\\
1.32688524560331 0.197017493547462\\
1.32700732539405 0.531975910524806\\
1.3271294051848 0.568683682248351\\
1.32725148497554 0.270433036994551\\
1.32737356476629 -0.160309721823917\\
1.32749564455703 -0.495268138801262\\
1.32761772434778 -0.706337826211643\\
1.32773980413852 -0.421852595354173\\
1.32786188392926 0.215371379409234\\
1.32798396372001 0.550329796386579\\
1.32810604351075 0.495268138801262\\
1.3282281233015 0.252079151132779\\
1.32835020309224 -0.0754229997132205\\
1.32847228288299 -0.458560367077717\\
1.32859436267373 -0.458560367077717\\
1.32871644246448 -0.0628047031832521\\
1.32883852225522 0.293375394321767\\
1.32896060204597 0.4034987094924\\
1.32908268183671 0.169486664754804\\
1.32920476162746 -0.102953828505879\\
1.3293268414182 -0.270433036994551\\
1.32944892120894 -0.311729280183539\\
1.32957100099969 -0.160309721823917\\
1.32969308079043 0.160309721823917\\
1.32981516058118 0.293375394321767\\
1.32993724037192 0.0467450530542013\\
1.33005932016267 -0.0662460567823344\\
1.33018139995341 -0.0255233725265271\\
1.33030347974416 0.0444508173214798\\
1.3304255595349 0.151132778893031\\
1.33054763932565 0.215371379409234\\
1.33066971911639 0.0513335245196444\\
1.33079179890714 -0.215371379409234\\
1.33091387869788 -0.513622024663034\\
1.33103595848862 -0.513622024663034\\
1.33115803827937 -0.0937768855749928\\
1.33128011807011 0.421852595354173\\
1.33140219786086 0.669630054488099\\
1.3315242776516 0.632922282764554\\
1.33164635744235 0.311729280183539\\
1.33176843723309 -0.311729280183539\\
1.33189051702384 -0.779753369658732\\
1.33201259681458 -0.706337826211643\\
1.33213467660533 -0.252079151132779\\
1.33225675639607 0.187840550616576\\
1.33237883618682 0.458560367077717\\
1.33250091597756 0.47691425293949\\
1.3326229957683 0.233725265271007\\
1.33274507555905 -0.0754229997132205\\
1.33286715534979 -0.261256094063665\\
1.33298923514054 -0.151132778893031\\
1.33311131493128 0.0800114711786636\\
1.33323339472203 0.0467450530542013\\
1.33335547451277 -0.0983653570404359\\
1.33347755430352 -0.160309721823917\\
1.33359963409426 -0.141955835962145\\
1.33372171388501 -0.151132778893031\\
1.33384379367575 0.125896185833094\\
1.3339658734665 0.440206481215945\\
1.33408795325724 0.366790937768856\\
1.33421003304799 0.00114711786636077\\
1.33433211283873 -0.293375394321767\\
1.33445419262947 -0.421852595354173\\
1.33457627242022 -0.242902208201893\\
1.33469835221096 0.0891884141095498\\
1.33482043200171 0.4034987094924\\
1.33494251179245 0.513622024663034\\
1.3350645915832 0.160309721823917\\
1.33518667137394 -0.348437051907083\\
1.33530875116469 -0.632922282764554\\
1.33543083095543 -0.421852595354173\\
1.33555291074618 -0.0983653570404359\\
1.33567499053692 0.366790937768856\\
1.33579707032767 0.706337826211643\\
1.33591915011841 0.531975910524806\\
1.33604122990915 0.112130771436765\\
1.3361633096999 -0.385144823630628\\
1.33628538949064 -0.568683682248351\\
1.33640746928139 -0.421852595354173\\
1.33652954907213 -0.112130771436765\\
1.33665162886288 0.107542299971322\\
1.33677370865362 0.233725265271007\\
1.33689578844437 0.293375394321767\\
1.33701786823511 0.0754229997132205\\
1.33713994802586 -0.016346429595641\\
1.3372620278166 0.130484657298537\\
1.33738410760735 0.197017493547462\\
1.33750618739809 0.0559219959850875\\
1.33762826718883 -0.160309721823917\\
1.33775034697958 -0.4034987094924\\
1.33787242677032 -0.495268138801262\\
1.33799450656107 -0.293375394321767\\
1.33811658635181 0.116719242902208\\
1.33823866614256 0.550329796386579\\
1.3383607459333 0.706337826211643\\
1.33848282572405 0.421852595354173\\
1.33860490551479 -0.141955835962145\\
1.33872698530554 -0.513622024663034\\
1.33884906509628 -0.596214511041009\\
1.33897114488703 -0.4034987094924\\
1.33909322467777 0.0845999426441067\\
1.33921530446851 0.513622024663034\\
1.33933738425926 0.47691425293949\\
1.33945946405 0.125896185833094\\
1.33958154384075 -0.151132778893031\\
1.33970362363149 -0.4034987094924\\
1.33982570342224 -0.279609979925437\\
1.33994778321298 -0.0352738743905936\\
1.34006986300373 0.261256094063665\\
1.34019194279447 0.366790937768856\\
1.34031402258522 0.206194436478348\\
1.34043610237596 -0.0845999426441067\\
1.34055818216671 -0.348437051907083\\
1.34068026195745 -0.130484657298537\\
1.34080234174819 -0.0375681101233152\\
1.34092442153894 0.0708345282477775\\
1.34104650132968 0.0605104674505305\\
1.34116858112043 -0.0559219959850875\\
1.34129066091117 -0.141955835962145\\
1.34141274070192 -0.141955835962145\\
1.34153482049266 0.206194436478348\\
1.34165690028341 0.385144823630628\\
1.34177898007415 0.495268138801262\\
1.3419010598649 0.107542299971322\\
1.34202313965564 -0.348437051907083\\
1.34214521944639 -0.568683682248351\\
1.34226729923713 -0.706337826211643\\
1.34238937902787 -0.311729280183539\\
1.34251145881862 0.311729280183539\\
1.34263353860936 0.853168913105822\\
1.34275561840011 0.669630054488099\\
1.34287769819085 0.330083166045311\\
1.3429997779816 -0.141955835962145\\
1.34312185777234 -0.596214511041009\\
1.34324393756309 -0.596214511041009\\
1.34336601735383 -0.348437051907083\\
1.34348809714458 0.0937768855749928\\
1.34361017693532 0.311729280183539\\
1.34373225672607 0.293375394321767\\
1.34385433651681 0.0444508173214798\\
1.34397641630755 0.00860338399770576\\
1.3440984960983 0.00946372239747634\\
1.34422057588904 -0.0243762546601663\\
1.34434265567979 0.116719242902208\\
1.34446473547053 0.141955835962145\\
1.34458681526128 -0.0845999426441067\\
1.34470889505202 -0.348437051907083\\
1.34483097484277 -0.311729280183539\\
1.34495305463351 -0.0536277602523659\\
1.34507513442426 0.197017493547462\\
1.345197214215 0.4034987094924\\
1.34531929400575 0.311729280183539\\
1.34544137379649 0.0306854029251506\\
1.34556345358723 -0.293375394321767\\
1.34568553337798 -0.513622024663034\\
1.34580761316872 -0.270433036994551\\
1.34592969295947 0.270433036994551\\
1.34605177275021 0.513622024663034\\
1.34617385254096 0.385144823630628\\
1.3462959323317 0.169486664754804\\
1.34641801212245 -0.233725265271007\\
1.34654009191319 -0.669630054488099\\
1.34666217170394 -0.596214511041009\\
1.34678425149468 -0.0937768855749928\\
1.34690633128543 0.348437051907083\\
1.34702841107617 0.596214511041009\\
1.34715049086692 0.440206481215945\\
1.34727257065766 0.17866360768569\\
1.3473946504484 -0.141955835962145\\
1.34751673023915 -0.385144823630628\\
1.34763881002989 -0.348437051907083\\
1.34776088982064 -0.0800114711786636\\
1.34788296961138 0.0800114711786636\\
1.34800504940213 -0.102953828505879\\
1.34812712919287 -0.0605104674505305\\
1.34824920898362 -0.000573558933180384\\
1.34837128877436 0.125896185833094\\
1.34849336856511 0.270433036994551\\
1.34861544835585 0.47691425293949\\
1.3487375281466 0.293375394321767\\
1.34885960793734 -0.151132778893031\\
1.34898168772808 -0.495268138801262\\
1.34910376751883 -0.743045597935188\\
1.34922584730957 -0.4034987094924\\
1.34934792710032 0.102953828505879\\
1.34947000689106 0.531975910524806\\
1.34959208668181 0.706337826211643\\
1.34971416647255 0.47691425293949\\
1.3498362462633 -0.0490392887869229\\
1.34995832605404 -0.550329796386579\\
1.35008040584479 -0.513622024663034\\
1.35020248563553 -0.252079151132779\\
1.35032456542628 0.0845999426441067\\
1.35044664521702 0.385144823630628\\
1.35056872500776 0.366790937768856\\
1.35069080479851 0.0937768855749928\\
1.35081288458925 -0.252079151132779\\
1.35093496438 -0.330083166045311\\
1.35105704417074 -0.17866360768569\\
1.35117912396149 0.112130771436765\\
1.35130120375223 0.22454832234012\\
1.35142328354298 0.135073128763981\\
1.35154536333372 0.0937768855749928\\
1.35166744312447 -0.0467450530542013\\
1.35178952291521 -0.17866360768569\\
1.35191160270596 -0.0375681101233152\\
1.3520336824967 0.22454832234012\\
1.35215576228744 0.125896185833094\\
1.35227784207819 -0.151132778893031\\
1.35239992186893 -0.311729280183539\\
1.35252200165968 -0.348437051907083\\
1.35264408145042 -0.197017493547462\\
1.35276616124117 0.242902208201893\\
1.35288824103191 0.568683682248351\\
1.35301032082266 0.706337826211643\\
1.3531324006134 0.330083166045311\\
1.35325448040415 -0.348437051907083\\
1.35337656019489 -0.743045597935188\\
1.35349863998564 -0.706337826211643\\
1.35362071977638 -0.366790937768856\\
1.35374279956712 0.0536277602523659\\
1.35386487935787 0.669630054488099\\
1.35398695914861 0.706337826211643\\
1.35410903893936 0.311729280183539\\
1.3542311187301 -0.0232291367938056\\
1.35435319852085 -0.366790937768856\\
1.35447527831159 -0.311729280183539\\
1.35459735810234 -0.22454832234012\\
1.35471943789308 -0.0398623458560367\\
1.35484151768383 0.020934901061084\\
1.35496359747457 0.0800114711786636\\
1.35508567726532 -0.0754229997132205\\
1.35520775705606 -0.135073128763981\\
1.35532983684681 0.22454832234012\\
1.35545191663755 0.311729280183539\\
1.35557399642829 0.261256094063665\\
1.35569607621904 0.0467450530542013\\
1.35581815600978 -0.151132778893031\\
1.35594023580053 -0.421852595354173\\
1.35606231559127 -0.421852595354173\\
1.35618439538202 -0.0444508173214798\\
1.35630647517276 0.293375394321767\\
1.35642855496351 0.531975910524806\\
1.35655063475425 0.261256094063665\\
1.356672714545 -0.102953828505879\\
1.35679479433574 -0.366790937768856\\
1.35691687412649 -0.47691425293949\\
1.35703895391723 -0.270433036994551\\
1.35716103370797 0.197017493547462\\
1.35728311349872 0.669630054488099\\
1.35740519328946 0.495268138801262\\
1.35752727308021 0.0891884141095498\\
1.35764935287095 -0.233725265271007\\
1.3577714326617 -0.513622024663034\\
1.35789351245244 -0.513622024663034\\
1.35801559224319 -0.233725265271007\\
1.35813767203393 0.206194436478348\\
1.35825975182468 0.385144823630628\\
1.35838183161542 0.348437051907083\\
1.35850391140617 0.116719242902208\\
1.35862599119691 -0.0559219959850875\\
1.35874807098765 0.0352738743905936\\
1.3588701507784 -0.0186406653283625\\
1.35899223056914 -0.0628047031832521\\
1.35911431035989 -0.0983653570404359\\
1.35923639015063 -0.215371379409234\\
1.35935846994138 -0.458560367077717\\
1.35948054973212 -0.311729280183539\\
1.35960262952287 0.233725265271007\\
1.35972470931361 0.513622024663034\\
1.35984678910436 0.669630054488099\\
1.3599688688951 0.440206481215945\\
1.36009094868585 -0.0559219959850875\\
1.36021302847659 -0.513622024663034\\
1.36033510826733 -0.743045597935188\\
1.36045718805808 -0.550329796386579\\
1.36057926784882 0.0232291367938056\\
1.36070134763957 0.568683682248351\\
1.36082342743031 0.531975910524806\\
1.36094550722106 0.293375394321767\\
1.3610675870118 0.0662460567823344\\
1.36118966680255 -0.385144823630628\\
1.36131174659329 -0.421852595354173\\
1.36143382638404 -0.0891884141095498\\
1.36155590617478 0.169486664754804\\
1.36167798596553 0.270433036994551\\
1.36180006575627 0.151132778893031\\
1.36192214554701 -0.0708345282477775\\
1.36204422533776 -0.261256094063665\\
1.3621663051285 -0.169486664754804\\
1.36228838491925 -0.020934901061084\\
1.36241046470999 0.169486664754804\\
1.36253254450074 0.242902208201893\\
1.36265462429148 -0.0329796386578721\\
1.36277670408223 -0.187840550616576\\
1.36289878387297 -0.135073128763981\\
1.36302086366372 0.0662460567823344\\
1.36314294345446 0.17866360768569\\
1.36326502324521 0.348437051907083\\
1.36338710303595 0.242902208201893\\
1.36350918282669 -0.22454832234012\\
1.36363126261744 -0.513622024663034\\
1.36375334240818 -0.568683682248351\\
1.36387542219893 -0.242902208201893\\
1.36399750198967 0.252079151132779\\
1.36411958178042 0.669630054488099\\
1.36424166157116 0.669630054488099\\
1.36436374136191 0.385144823630628\\
1.36448582115265 -0.121307714367651\\
1.3646079009434 -0.669630054488099\\
1.36472998073414 -0.632922282764554\\
1.36485206052489 -0.311729280183539\\
1.36497414031563 0.0117579581301979\\
1.36509622010637 0.270433036994551\\
1.36521829989712 0.440206481215945\\
1.36534037968786 0.293375394321767\\
1.36546245947861 -0.0559219959850875\\
1.36558453926935 -0.0444508173214798\\
1.3657066190601 0.00630914826498423\\
1.36582869885084 0.0662460567823344\\
1.36595077864159 0.00286779466590192\\
1.36607285843233 -0.215371379409234\\
1.36619493822308 -0.293375394321767\\
1.36631701801382 -0.293375394321767\\
1.36643909780457 -0.0708345282477775\\
1.36656117759531 0.169486664754804\\
1.36668325738605 0.568683682248351\\
1.3668053371768 0.513622024663034\\
1.36692741696754 -0.00573558933180384\\
1.36704949675829 -0.293375394321767\\
1.36717157654903 -0.495268138801262\\
1.36729365633978 -0.385144823630628\\
1.36741573613052 -0.0197877831947233\\
1.36753781592127 0.421852595354173\\
1.36765989571201 0.531975910524806\\
1.36778197550276 0.242902208201893\\
1.3679040552935 -0.169486664754804\\
1.36802613508425 -0.550329796386579\\
1.36814821487499 -0.458560367077717\\
1.36827029466573 -0.116719242902208\\
1.36839237445648 0.206194436478348\\
1.36851445424722 0.513622024663034\\
1.36863653403797 0.513622024663034\\
1.36875861382871 0.112130771436765\\
1.36888069361946 -0.293375394321767\\
1.3690027734102 -0.366790937768856\\
1.36912485320095 -0.261256094063665\\
1.36924693299169 -0.112130771436765\\
1.36936901278244 0.0375681101233152\\
1.36949109257318 0.0754229997132205\\
1.36961317236393 0.0174935474620017\\
1.36973525215467 0.0174935474620017\\
1.36985733194542 0.0444508173214798\\
1.36997941173616 0.279609979925437\\
1.3701014915269 0.421852595354173\\
1.37022357131765 0.22454832234012\\
1.37034565110839 -0.233725265271007\\
1.37046773089914 -0.458560367077717\\
1.37058981068988 -0.596214511041009\\
1.37071189048063 -0.513622024663034\\
1.37083397027137 0.112130771436765\\
1.37095605006212 0.568683682248351\\
1.37107812985286 0.743045597935188\\
1.37120020964361 0.495268138801262\\
1.37132228943435 0.0375681101233152\\
1.3714443692251 -0.458560367077717\\
1.37156644901584 -0.632922282764554\\
1.37168852880658 -0.421852595354173\\
1.37181060859733 -0.112130771436765\\
1.37193268838807 0.4034987094924\\
1.37205476817882 0.421852595354173\\
1.37217684796956 0.130484657298537\\
1.37229892776031 -0.0845999426441067\\
1.37242100755105 -0.151132778893031\\
1.3725430873418 -0.197017493547462\\
1.37266516713254 -0.0662460567823344\\
1.37278724692329 0.242902208201893\\
1.37290932671403 0.160309721823917\\
1.37303140650478 0.0106108402638371\\
1.37315348629552 -0.151132778893031\\
1.37327556608626 -0.187840550616576\\
1.37339764587701 -0.0140521938629194\\
1.37351972566775 0.141955835962145\\
1.3736418054585 0.135073128763981\\
1.37376388524924 0.020934901061084\\
1.37388596503999 -0.0845999426441067\\
1.37400804483073 -0.330083166045311\\
1.37413012462148 -0.279609979925437\\
1.37425220441222 0.215371379409234\\
1.37437428420297 0.495268138801262\\
1.37449636399371 0.47691425293949\\
1.37461844378446 0.261256094063665\\
1.3747405235752 -0.206194436478348\\
1.37486260336594 -0.632922282764554\\
1.37498468315669 -0.706337826211643\\
1.37510676294743 -0.385144823630628\\
1.37522884273818 0.151132778893031\\
1.37535092252892 0.706337826211643\\
1.37547300231967 0.669630054488099\\
1.37559508211041 0.330083166045311\\
1.37571716190116 0.0117579581301979\\
1.3758392416919 -0.366790937768856\\
1.37596132148265 -0.513622024663034\\
1.37608340127339 -0.330083166045311\\
1.37620548106414 -0.0186406653283625\\
1.37632756085488 0.0937768855749928\\
1.37644964064562 0.0891884141095498\\
1.37657172043637 0.0754229997132205\\
1.37669380022711 -0.007456266131345\\
1.37681588001786 0.169486664754804\\
1.3769379598086 0.233725265271007\\
1.37706003959935 0.206194436478348\\
1.37718211939009 0.0983653570404359\\
1.37730419918084 -0.215371379409234\\
1.37742627897158 -0.458560367077717\\
1.37754835876233 -0.47691425293949\\
1.37767043855307 -0.0352738743905936\\
1.37779251834382 0.252079151132779\\
1.37791459813456 0.47691425293949\\
1.3780366779253 0.458560367077717\\
1.37815875771605 0.0800114711786636\\
1.37828083750679 -0.311729280183539\\
1.37840291729754 -0.550329796386579\\
1.37852499708828 -0.311729280183539\\
1.37864707687903 0.0513335245196444\\
1.37876915666977 0.513622024663034\\
1.37889123646052 0.458560367077717\\
1.37901331625126 0.187840550616576\\
1.37913539604201 -0.107542299971322\\
1.37925747583275 -0.550329796386579\\
1.3793795556235 -0.513622024663034\\
1.37950163541424 -0.169486664754804\\
1.37962371520498 0.261256094063665\\
1.37974579499573 0.348437051907083\\
1.37986787478647 0.348437051907083\\
1.37998995457722 0.261256094063665\\
1.38011203436796 -0.0845999426441067\\
1.38023411415871 -0.197017493547462\\
1.38035619394945 -0.160309721823917\\
1.3804782737402 -0.0375681101233152\\
1.38060035353094 -0.0754229997132205\\
1.38072243332169 -0.17866360768569\\
1.38084451311243 -0.261256094063665\\
1.38096659290318 -0.121307714367651\\
1.38108867269392 0.187840550616576\\
1.38121075248467 0.366790937768856\\
1.38133283227541 0.550329796386579\\
1.38145491206615 0.47691425293949\\
1.3815769918569 -0.0536277602523659\\
1.38169907164764 -0.568683682248351\\
1.38182115143839 -0.706337826211643\\
1.38194323122913 -0.550329796386579\\
1.38206531101988 -0.107542299971322\\
1.38218739081062 0.4034987094924\\
1.38230947060137 0.706337826211643\\
1.38243155039211 0.550329796386579\\
1.38255363018286 0.187840550616576\\
1.3826757099736 -0.366790937768856\\
1.38279778976435 -0.550329796386579\\
1.38291986955509 -0.242902208201893\\
1.38304194934583 -0.0421565815887582\\
1.38316402913658 0.160309721823917\\
1.38328610892732 0.242902208201893\\
1.38340818871807 0.160309721823917\\
1.38353026850881 -0.160309721823917\\
1.38365234829956 -0.233725265271007\\
1.3837744280903 0.00229423573272154\\
1.38389650788105 0.125896185833094\\
1.38401858767179 0.197017493547462\\
1.38414066746254 0.0375681101233152\\
1.38426274725328 -0.116719242902208\\
1.38438482704403 -0.125896185833094\\
1.38450690683477 -0.112130771436765\\
1.38462898662551 0.0329796386578721\\
1.38475106641626 0.279609979925437\\
1.384873146207 0.330083166045311\\
1.38499522599775 -0.102953828505879\\
1.38511730578849 -0.385144823630628\\
1.38523938557924 -0.421852595354173\\
1.38536146536998 -0.293375394321767\\
1.38548354516073 0.121307714367651\\
1.38560562495147 0.550329796386579\\
1.38572770474222 0.706337826211643\\
1.38584978453296 0.4034987094924\\
1.38597186432371 -0.0891884141095498\\
1.38609394411445 -0.632922282764554\\
1.38621602390519 -0.743045597935188\\
1.38633810369594 -0.421852595354173\\
1.38646018348668 -0.0983653570404359\\
1.38658226327743 0.440206481215945\\
1.38670434306817 0.632922282764554\\
1.38682642285892 0.4034987094924\\
1.38694850264966 0.0117579581301979\\
1.38707058244041 -0.151132778893031\\
1.38719266223115 -0.151132778893031\\
1.3873147420219 -0.17866360768569\\
1.38743682181264 -0.0444508173214798\\
1.38755890160339 -0.160309721823917\\
1.38768098139413 -0.151132778893031\\
1.38780306118487 -0.151132778893031\\
1.38792514097562 -0.0937768855749928\\
1.38804722076636 0.206194436478348\\
1.38816930055711 0.495268138801262\\
1.38829138034785 0.513622024663034\\
1.3884134601386 0.0398623458560367\\
1.38853553992934 -0.206194436478348\\
1.38865761972009 -0.495268138801262\\
1.38877969951083 -0.550329796386579\\
1.38890177930158 -0.169486664754804\\
1.38902385909232 0.293375394321767\\
1.38914593888307 0.568683682248351\\
1.38926801867381 0.366790937768856\\
1.38939009846455 0.0662460567823344\\
1.3895121782553 -0.348437051907083\\
1.38963425804604 -0.495268138801262\\
1.38975633783679 -0.279609979925437\\
1.38987841762753 0.028391167192429\\
1.39000049741828 0.495268138801262\\
1.39012257720902 0.550329796386579\\
1.39024465699977 0.160309721823917\\
1.39036673679051 -0.17866360768569\\
1.39048881658126 -0.348437051907083\\
1.390610896372 -0.440206481215945\\
1.39073297616275 -0.233725265271007\\
1.39085505595349 0.141955835962145\\
1.39097713574423 0.242902208201893\\
1.39109921553498 0.169486664754804\\
1.39122129532572 0.112130771436765\\
1.39134337511647 0.0151993117292802\\
1.39146545490721 0.0708345282477775\\
1.39158753469796 0.187840550616576\\
1.3917096144887 0.0800114711786636\\
1.39183169427945 -0.0983653570404359\\
1.39195377407019 -0.311729280183539\\
1.39207585386094 -0.513622024663034\\
1.39219793365168 -0.440206481215945\\
1.39232001344243 0.0800114711786636\\
1.39244209323317 0.513622024663034\\
1.39256417302391 0.706337826211643\\
1.39268625281466 0.596214511041009\\
1.3928083326054 0.102953828505879\\
1.39293041239615 -0.440206481215945\\
1.39305249218689 -0.743045597935188\\
1.39317457197764 -0.596214511041009\\
1.39329665176838 -0.160309721823917\\
1.39341873155913 0.4034987094924\\
1.39354081134987 0.513622024663034\\
1.39366289114062 0.311729280183539\\
1.39378497093136 0.160309721823917\\
1.39390705072211 -0.206194436478348\\
1.39402913051285 -0.348437051907083\\
1.39415121030359 -0.121307714367651\\
1.39427329009434 0.169486664754804\\
1.39439536988508 0.141955835962145\\
1.39451744967583 -0.0186406653283625\\
1.39463952946657 -0.0754229997132205\\
1.39476160925732 -0.197017493547462\\
1.39488368904806 -0.116719242902208\\
1.39500576883881 0.0800114711786636\\
1.39512784862955 0.233725265271007\\
1.3952499284203 0.242902208201893\\
1.39537200821104 -0.0375681101233152\\
1.39549408800179 -0.293375394321767\\
1.39561616779253 -0.293375394321767\\
1.39573824758328 0.016346429595641\\
1.39586032737402 0.242902208201893\\
1.39598240716476 0.4034987094924\\
1.39610448695551 0.385144823630628\\
1.39622656674625 -0.0605104674505305\\
1.396348646537 -0.495268138801262\\
1.39647072632774 -0.669630054488099\\
1.39659280611849 -0.385144823630628\\
1.39671488590923 0.0937768855749928\\
1.39683696569998 0.568683682248351\\
1.39695904549072 0.706337826211643\\
1.39708112528147 0.47691425293949\\
1.39720320507221 0.0444508173214798\\
1.39732528486296 -0.458560367077717\\
1.3974473646537 -0.632922282764554\\
1.39756944444444 -0.385144823630628\\
1.39769152423519 -0.0444508173214798\\
1.39781360402593 0.0845999426441067\\
1.39793568381668 0.233725265271007\\
1.39805776360742 0.252079151132779\\
1.39817984339817 0.0605104674505305\\
1.39830192318891 0.0421565815887582\\
1.39842400297966 0.141955835962145\\
1.3985460827704 0.206194436478348\\
1.39866816256115 0.0662460567823344\\
1.39879024235189 -0.233725265271007\\
1.39891232214264 -0.458560367077717\\
1.39903440193338 -0.440206481215945\\
1.39915648172412 -0.151132778893031\\
1.39927856151487 0.160309721823917\\
1.39940064130561 0.531975910524806\\
1.39952272109636 0.669630054488099\\
1.3996448008871 0.242902208201893\\
1.39976688067785 -0.279609979925437\\
1.39988896046859 -0.513622024663034\\
1.40001104025934 -0.47691425293949\\
1.40013312005008 -0.160309721823917\\
1.40025519984083 0.279609979925437\\
1.40037727963157 0.495268138801262\\
1.40049935942232 0.348437051907083\\
1.40062143921306 -0.0151993117292802\\
1.4007435190038 -0.385144823630628\\
1.40086559879455 -0.458560367077717\\
1.40098767858529 -0.169486664754804\\
1.40110975837604 0.125896185833094\\
1.40123183816678 0.330083166045311\\
1.40135391795753 0.385144823630628\\
1.40147599774827 0.206194436478348\\
1.40159807753902 -0.130484657298537\\
1.40172015732976 -0.293375394321767\\
1.40184223712051 -0.151132778893031\\
1.40196431691125 -0.0467450530542013\\
1.402086396702 0.00946372239747634\\
1.40220847649274 -0.0800114711786636\\
1.40233055628348 -0.102953828505879\\
1.40245263607423 -0.0891884141095498\\
1.40257471586497 0.028391167192429\\
1.40269679565572 0.293375394321767\\
1.40281887544646 0.47691425293949\\
1.40294095523721 0.47691425293949\\
1.40306303502795 -0.0490392887869229\\
1.4031851148187 -0.495268138801262\\
1.40330719460944 -0.669630054488099\\
1.40342927440019 -0.596214511041009\\
1.40355135419093 -0.141955835962145\\
1.40367343398168 0.421852595354173\\
1.40379551377242 0.853168913105822\\
1.40391759356316 0.596214511041009\\
1.40403967335391 0.197017493547462\\
1.40416175314465 -0.330083166045311\\
1.4042838329354 -0.568683682248351\\
1.40440591272614 -0.385144823630628\\
1.40452799251689 -0.215371379409234\\
1.40465007230763 0.187840550616576\\
1.40477215209838 0.279609979925437\\
1.40489423188912 0.197017493547462\\
1.40501631167987 -0.130484657298537\\
1.40513839147061 -0.121307714367651\\
1.40526047126136 0.0845999426441067\\
1.4053825510521 0.0662460567823344\\
1.40550463084284 0.22454832234012\\
1.40562671063359 0.058216231717809\\
1.40574879042433 -0.125896185833094\\
1.40587087021508 -0.311729280183539\\
1.40599295000582 -0.242902208201893\\
1.40611502979657 0.0306854029251506\\
1.40623710958731 0.279609979925437\\
1.40635918937806 0.385144823630628\\
1.4064812691688 0.0536277602523659\\
1.40660334895955 -0.160309721823917\\
1.40672542875029 -0.385144823630628\\
1.40684750854104 -0.366790937768856\\
1.40696958833178 -0.058216231717809\\
1.40709166812253 0.440206481215945\\
1.40721374791327 0.632922282764554\\
1.40733582770401 0.311729280183539\\
1.40745790749476 -0.00630914826498423\\
1.4075799872855 -0.458560367077717\\
1.40770206707625 -0.632922282764554\\
1.40782414686699 -0.47691425293949\\
1.40794622665774 -0.0444508173214798\\
1.40806830644848 0.385144823630628\\
1.40819038623923 0.550329796386579\\
1.40831246602997 0.4034987094924\\
1.40843454582072 0.0845999426441067\\
1.40855662561146 -0.0891884141095498\\
1.40867870540221 -0.261256094063665\\
1.40880078519295 -0.279609979925437\\
1.40892286498369 -0.125896185833094\\
1.40904494477444 -0.00172067679954115\\
1.40916702456518 -0.135073128763981\\
1.40928910435593 -0.169486664754804\\
1.40941118414667 0.0220820189274448\\
1.40953326393742 0.187840550616576\\
1.40965534372816 0.4034987094924\\
1.40977742351891 0.421852595354173\\
1.40989950330965 0.206194436478348\\
1.4100215831004 -0.169486664754804\\
1.41014366289114 -0.513622024663034\\
1.41026574268189 -0.669630054488099\\
1.41038782247263 -0.293375394321767\\
1.41050990226337 0.252079151132779\\
1.41063198205412 0.47691425293949\\
1.41075406184486 0.495268138801262\\
1.41087614163561 0.252079151132779\\
1.41099822142635 -0.116719242902208\\
1.4111203012171 -0.531975910524806\\
1.41124238100784 -0.4034987094924\\
1.41136446079859 -0.0306854029251506\\
1.41148654058933 0.293375394321767\\
1.41160862038008 0.421852595354173\\
1.41173070017082 0.160309721823917\\
1.41185277996157 -0.0467450530542013\\
1.41197485975231 -0.330083166045311\\
1.41209693954305 -0.348437051907083\\
1.4122190193338 -0.17866360768569\\
1.41234109912454 0.22454832234012\\
1.41246317891529 0.293375394321767\\
1.41258525870603 0.0845999426441067\\
1.41270733849678 0.0444508173214798\\
1.41282941828752 -0.058216231717809\\
1.41295149807827 -0.016346429595641\\
1.41307357786901 0.028391167192429\\
1.41319565765976 0.187840550616576\\
1.4133177374505 0.0444508173214798\\
1.41343981724125 -0.242902208201893\\
1.41356189703199 -0.440206481215945\\
1.41368397682273 -0.421852595354173\\
1.41380605661348 0.0329796386578721\\
1.41392813640422 0.4034987094924\\
1.41405021619497 0.632922282764554\\
1.41417229598571 0.568683682248351\\
1.41429437577646 0.22454832234012\\
1.4144164555672 -0.4034987094924\\
1.41453853535795 -0.816461141382277\\
1.41466061514869 -0.568683682248351\\
1.41478269493944 -0.242902208201893\\
1.41490477473018 0.233725265271007\\
1.41502685452093 0.531975910524806\\
1.41514893431167 0.531975910524806\\
1.41527101410241 0.252079151132779\\
1.41539309389316 -0.169486664754804\\
1.4155151736839 -0.311729280183539\\
1.41563725347465 -0.22454832234012\\
1.41575933326539 0.0536277602523659\\
1.41588141305614 0.0490392887869229\\
1.41600349284688 -0.0490392887869229\\
1.41612557263763 -0.0106108402638371\\
1.41624765242837 -0.151132778893031\\
1.41636973221912 -0.169486664754804\\
1.41649181200986 0.107542299971322\\
1.41661389180061 0.385144823630628\\
1.41673597159135 0.279609979925437\\
1.4168580513821 -0.028391167192429\\
1.41698013117284 -0.197017493547462\\
1.41710221096358 -0.385144823630628\\
1.41722429075433 -0.197017493547462\\
1.41734637054507 0.0800114711786636\\
1.41746845033582 0.330083166045311\\
1.41759053012656 0.47691425293949\\
1.41771260991731 0.0891884141095498\\
1.41783468970805 -0.348437051907083\\
1.4179567694988 -0.531975910524806\\
1.41807884928954 -0.348437051907083\\
1.41820092908029 -0.0754229997132205\\
1.41832300887103 0.385144823630628\\
1.41844508866178 0.743045597935188\\
1.41856716845252 0.47691425293949\\
1.41868924824326 0.0186406653283625\\
1.41881132803401 -0.458560367077717\\
1.41893340782475 -0.568683682248351\\
1.4190554876155 -0.4034987094924\\
1.41917756740624 -0.0983653570404359\\
1.41929964719699 0.169486664754804\\
1.41942172698773 0.311729280183539\\
1.41954380677848 0.330083166045311\\
1.41966588656922 0.0243762546601663\\
1.41978796635997 -0.0467450530542013\\
1.41991004615071 0.141955835962145\\
1.42003212594146 0.116719242902208\\
1.4201542057322 -0.0232291367938056\\
1.42027628552294 -0.17866360768569\\
1.42039836531369 -0.330083166045311\\
1.42052044510443 -0.440206481215945\\
1.42064252489518 -0.252079151132779\\
1.42076460468592 0.187840550616576\\
1.42088668447667 0.550329796386579\\
1.42100876426741 0.669630054488099\\
1.42113084405816 0.293375394321767\\
1.4212529238489 -0.160309721823917\\
1.42137500363965 -0.458560367077717\\
1.42149708343039 -0.596214511041009\\
1.42161916322114 -0.366790937768856\\
1.42174124301188 0.125896185833094\\
1.42186332280262 0.550329796386579\\
1.42198540259337 0.385144823630628\\
1.42210748238411 0.112130771436765\\
1.42222956217486 -0.17866360768569\\
1.4223516419656 -0.385144823630628\\
1.42247372175635 -0.261256094063665\\
1.42259580154709 -0.0117579581301979\\
1.42271788133784 0.293375394321767\\
1.42283996112858 0.348437051907083\\
1.42296204091933 0.187840550616576\\
1.42308412071007 -0.141955835962145\\
1.42320620050082 -0.270433036994551\\
1.42332828029156 -0.151132778893031\\
1.4234503600823 -0.116719242902208\\
1.42357243987305 0.0490392887869229\\
1.42369451966379 0.135073128763981\\
1.42381659945454 -0.016346429595641\\
1.42393867924528 -0.141955835962145\\
1.42406075903603 -0.0421565815887582\\
1.42418283882677 0.187840550616576\\
1.42430491861752 0.330083166045311\\
1.42442699840826 0.385144823630628\\
1.42454907819901 0.0467450530542013\\
1.42467115798975 -0.330083166045311\\
1.4247932377805 -0.550329796386579\\
1.42491531757124 -0.632922282764554\\
1.42503739736198 -0.22454832234012\\
1.42515947715273 0.366790937768856\\
1.42528155694347 0.743045597935188\\
1.42540363673422 0.632922282764554\\
1.42552571652496 0.311729280183539\\
1.42564779631571 -0.151132778893031\\
1.42576987610645 -0.596214511041009\\
1.4258919558972 -0.568683682248351\\
1.42601403568794 -0.311729280183539\\
1.42613611547869 0.0800114711786636\\
1.42625819526943 0.311729280183539\\
1.42638027506018 0.311729280183539\\
1.42650235485092 0.135073128763981\\
1.42662443464166 -0.0490392887869229\\
1.42674651443241 -0.0266704903928879\\
1.42686859422315 -0.028391167192429\\
1.4269906740139 0.121307714367651\\
1.42711275380464 0.0800114711786636\\
1.42723483359539 -0.141955835962145\\
1.42735691338613 -0.252079151132779\\
1.42747899317688 -0.252079151132779\\
1.42760107296762 -0.0243762546601663\\
1.42772315275837 0.160309721823917\\
1.42784523254911 0.348437051907083\\
1.42796731233986 0.311729280183539\\
1.4280893921306 -0.0329796386578721\\
1.42821147192134 -0.293375394321767\\
1.42833355171209 -0.421852595354173\\
1.42845563150283 -0.17866360768569\\
1.42857771129358 0.22454832234012\\
1.42869979108432 0.47691425293949\\
1.42882187087507 0.421852595354173\\
1.42894395066581 0.112130771436765\\
1.42906603045656 -0.279609979925437\\
1.4291881102473 -0.632922282764554\\
1.42931019003805 -0.47691425293949\\
1.42943226982879 -0.121307714367651\\
1.42955434961954 0.293375394321767\\
1.42967642941028 0.568683682248351\\
1.42979850920102 0.47691425293949\\
1.42992058899177 0.215371379409234\\
1.43004266878251 -0.160309721823917\\
1.43016474857326 -0.311729280183539\\
1.430286828364 -0.366790937768856\\
1.43040890815475 -0.17866360768569\\
1.43053098794549 -0.0266704903928879\\
1.43065306773624 -0.0375681101233152\\
1.43077514752698 0.00573558933180384\\
1.43089722731773 0.0605104674505305\\
1.43101930710847 0.17866360768569\\
1.43114138689922 0.252079151132779\\
1.43126346668996 0.385144823630628\\
1.43138554648071 0.17866360768569\\
1.43150762627145 -0.141955835962145\\
1.43162970606219 -0.440206481215945\\
1.43175178585294 -0.596214511041009\\
1.43187386564368 -0.366790937768856\\
1.43199594543443 0.0662460567823344\\
1.43211802522517 0.440206481215945\\
1.43224010501592 0.568683682248351\\
1.43236218480666 0.440206481215945\\
1.43248426459741 -0.0306854029251506\\
1.43260634438815 -0.385144823630628\\
1.4327284241789 -0.47691425293949\\
1.43285050396964 -0.293375394321767\\
1.43297258376039 0.0708345282477775\\
1.43309466355113 0.311729280183539\\
1.43321674334187 0.366790937768856\\
1.43333882313262 0.0800114711786636\\
1.43346090292336 -0.135073128763981\\
1.43358298271411 -0.330083166045311\\
1.43370506250485 -0.233725265271007\\
1.4338271422956 0.0243762546601663\\
1.43394922208634 0.160309721823917\\
1.43407130187709 0.233725265271007\\
1.43419338166783 0.141955835962145\\
1.43431546145858 0.0220820189274448\\
1.43443754124932 -0.151132778893031\\
1.43455962104007 -0.0662460567823344\\
1.43468170083081 0.0490392887869229\\
1.43480378062155 0.0513335245196444\\
1.4349258604123 -0.0800114711786636\\
1.43504794020304 -0.233725265271007\\
1.43517001999379 -0.242902208201893\\
1.43529209978453 -0.17866360768569\\
1.43541417957528 0.215371379409234\\
1.43553625936602 0.495268138801262\\
1.43565833915677 0.596214511041009\\
1.43578041894751 0.311729280183539\\
1.43590249873826 -0.215371379409234\\
1.436024578529 -0.568683682248351\\
1.43614665831975 -0.706337826211643\\
1.43626873811049 -0.385144823630628\\
1.43639081790123 0.0352738743905936\\
1.43651289769198 0.550329796386579\\
1.43663497748272 0.669630054488099\\
1.43675705727347 0.366790937768856\\
1.43687913706421 0.0444508173214798\\
1.43700121685496 -0.311729280183539\\
1.4371232966457 -0.366790937768856\\
1.43724537643645 -0.270433036994551\\
1.43736745622719 -0.028391167192429\\
1.43748953601794 0.058216231717809\\
1.43761161580868 0.0754229997132205\\
1.43773369559943 0.0106108402638371\\
1.43785577539017 -0.0754229997132205\\
1.43797785518091 0.0628047031832521\\
1.43809993497166 0.187840550616576\\
1.4382220147624 0.215371379409234\\
1.43834409455315 0.0937768855749928\\
1.43846617434389 -0.0708345282477775\\
1.43858825413464 -0.279609979925437\\
1.43871033392538 -0.330083166045311\\
1.43883241371613 -0.0845999426441067\\
1.43895449350687 0.17866360768569\\
1.43907657329762 0.385144823630628\\
1.43919865308836 0.252079151132779\\
1.43932073287911 -0.0306854029251506\\
1.43944281266985 -0.270433036994551\\
1.43956489246059 -0.421852595354173\\
1.43968697225134 -0.279609979925437\\
1.43980905204208 0.0983653570404359\\
1.43993113183283 0.550329796386579\\
1.44005321162357 0.568683682248351\\
1.44017529141432 0.169486664754804\\
1.44029737120506 -0.141955835962145\\
1.44041945099581 -0.47691425293949\\
1.44054153078655 -0.596214511041009\\
1.4406636105773 -0.330083166045311\\
1.44078569036804 0.0845999426441067\\
1.44090777015879 0.440206481215945\\
1.44102984994953 0.421852595354173\\
1.44115192974027 0.242902208201893\\
1.44127400953102 0.016346429595641\\
1.44139608932176 -0.0708345282477775\\
1.44151816911251 -0.116719242902208\\
1.44164024890325 -0.125896185833094\\
1.441762328694 -0.0662460567823344\\
1.44188440848474 -0.169486664754804\\
1.44200648827549 -0.270433036994551\\
1.44212856806623 -0.206194436478348\\
1.44225064785698 0.0398623458560367\\
1.44237272764772 0.4034987094924\\
1.44249480743847 0.47691425293949\\
1.44261688722921 0.440206481215945\\
1.44273896701996 0.0891884141095498\\
1.4428610468107 -0.311729280183539\\
1.44298312660144 -0.596214511041009\\
1.44310520639219 -0.596214511041009\\
1.44322728618293 -0.112130771436765\\
1.44334936597368 0.279609979925437\\
1.44347144576442 0.550329796386579\\
1.44359352555517 0.4034987094924\\
1.44371560534591 0.169486664754804\\
1.44383768513666 -0.242902208201893\\
1.4439597649274 -0.458560367077717\\
1.44408184471815 -0.261256094063665\\
1.44420392450889 0.0444508173214798\\
1.44432600429964 0.330083166045311\\
1.44444808409038 0.233725265271007\\
1.44457016388112 0.130484657298537\\
1.44469224367187 -0.169486664754804\\
1.44481432346261 -0.261256094063665\\
1.44493640325336 -0.187840550616576\\
1.4450584830441 -0.00860338399770576\\
1.44518056283485 0.233725265271007\\
1.44530264262559 0.116719242902208\\
1.44542472241634 -0.0140521938629194\\
1.44554680220708 -0.135073128763981\\
1.44566888199783 0.0444508173214798\\
1.44579096178857 0.107542299971322\\
1.44591304157932 0.187840550616576\\
1.44603512137006 0.252079151132779\\
1.4461572011608 -0.0536277602523659\\
1.44627928095155 -0.366790937768856\\
1.44640136074229 -0.568683682248351\\
1.44652344053304 -0.348437051907083\\
1.44664552032378 0.0754229997132205\\
1.44676760011453 0.513622024663034\\
1.44688967990527 0.669630054488099\\
1.44701175969602 0.495268138801262\\
1.44713383948676 0.135073128763981\\
1.44725591927751 -0.47691425293949\\
1.44737799906825 -0.743045597935188\\
1.447500078859 -0.513622024663034\\
1.44762215864974 -0.141955835962145\\
1.44774423844048 0.206194436478348\\
1.44786631823123 0.440206481215945\\
1.44798839802197 0.440206481215945\\
1.44811047781272 0.187840550616576\\
1.44823255760346 -0.0891884141095498\\
1.44835463739421 -0.17866360768569\\
1.44847671718495 -0.116719242902208\\
1.4485987969757 0.0117579581301979\\
1.44872087676644 -0.0559219959850875\\
1.44884295655719 -0.17866360768569\\
1.44896503634793 -0.102953828505879\\
1.44908711613868 -0.0983653570404359\\
1.44920919592942 -0.0140521938629194\\
1.44933127572016 0.233725265271007\\
1.44945335551091 0.440206481215945\\
1.44957543530165 0.215371379409234\\
1.4496975150924 -0.151132778893031\\
1.44981959488314 -0.293375394321767\\
1.44994167467389 -0.385144823630628\\
1.45006375446463 -0.141955835962145\\
1.45018583425538 0.160309721823917\\
1.45030791404612 0.4034987094924\\
1.45042999383687 0.385144823630628\\
1.45055207362761 0.0490392887869229\\
1.45067415341836 -0.330083166045311\\
1.4507962332091 -0.531975910524806\\
1.45091831299984 -0.279609979925437\\
1.45104039279059 -0.0232291367938056\\
1.45116247258133 0.330083166045311\\
1.45128455237208 0.596214511041009\\
1.45140663216282 0.4034987094924\\
1.45152871195357 0.0117579581301979\\
1.45165079174431 -0.330083166045311\\
1.45177287153506 -0.385144823630628\\
1.4518949513258 -0.366790937768856\\
1.45201703111655 -0.112130771436765\\
1.45213911090729 0.0662460567823344\\
1.45226119069804 0.141955835962145\\
1.45238327048878 0.261256094063665\\
1.45250535027952 0.112130771436765\\
1.45262743007027 0.0983653570404359\\
1.45274950986101 0.169486664754804\\
1.45287158965176 0.215371379409234\\
1.4529936694425 -0.0891884141095498\\
1.45311574923325 -0.293375394321767\\
1.45323782902399 -0.366790937768856\\
1.45335990881474 -0.458560367077717\\
1.45348198860548 -0.187840550616576\\
1.45360406839623 0.242902208201893\\
1.45372614818697 0.596214511041009\\
1.45384822797772 0.596214511041009\\
1.45397030776846 0.293375394321767\\
1.4540923875592 -0.169486664754804\\
1.45421446734995 -0.495268138801262\\
1.45433654714069 -0.495268138801262\\
1.45445862693144 -0.348437051907083\\
1.45458070672218 0.0937768855749928\\
1.45470278651293 0.458560367077717\\
1.45482486630367 0.366790937768856\\
1.45494694609442 0.0708345282477775\\
1.45506902588516 -0.0983653570404359\\
1.45519110567591 -0.252079151132779\\
1.45531318546665 -0.206194436478348\\
1.4554352652574 0.0140521938629194\\
1.45555734504814 0.169486664754804\\
1.45567942483888 0.197017493547462\\
1.45580150462963 0.0662460567823344\\
1.45592358442037 -0.107542299971322\\
1.45604566421112 -0.197017493547462\\
1.45616774400186 -0.00573558933180384\\
1.45628982379261 0.0329796386578721\\
1.45641190358335 0.016346429595641\\
1.4565339833741 0.0375681101233152\\
1.45665606316484 -0.0845999426441067\\
1.45677814295559 -0.215371379409234\\
1.45690022274633 -0.0800114711786636\\
1.45702230253708 0.261256094063665\\
1.45714438232782 0.366790937768856\\
1.45726646211857 0.385144823630628\\
1.45738854190931 0.0605104674505305\\
1.45751062170005 -0.293375394321767\\
1.4576327014908 -0.568683682248351\\
1.45775478128154 -0.596214511041009\\
1.45787686107229 -0.215371379409234\\
1.45799894086303 0.270433036994551\\
1.45812102065378 0.706337826211643\\
1.45824310044452 0.550329796386579\\
1.45836518023527 0.270433036994551\\
1.45848726002601 -0.0754229997132205\\
1.45860933981676 -0.458560367077717\\
1.4587314196075 -0.495268138801262\\
1.45885349939825 -0.311729280183539\\
1.45897557918899 0.0536277602523659\\
1.45909765897973 0.130484657298537\\
1.45921973877048 0.215371379409234\\
1.45934181856122 0.0891884141095498\\
1.45946389835197 0.00946372239747634\\
1.45958597814271 0.116719242902208\\
1.45970805793346 0.0708345282477775\\
1.4598301377242 0.141955835962145\\
1.45995221751495 0.0174935474620017\\
1.46007429730569 -0.169486664754804\\
1.46019637709644 -0.348437051907083\\
1.46031845688718 -0.293375394321767\\
1.46044053667793 -0.00458847146544307\\
1.46056261646867 0.17866360768569\\
1.46068469625941 0.4034987094924\\
1.46080677605016 0.311729280183539\\
1.4609288558409 0.0243762546601663\\
1.46105093563165 -0.261256094063665\\
1.46117301542239 -0.4034987094924\\
1.46129509521314 -0.242902208201893\\
1.46141717500388 0.0937768855749928\\
1.46153925479463 0.421852595354173\\
1.46166133458537 0.348437051907083\\
1.46178341437612 0.17866360768569\\
1.46190549416686 -0.121307714367651\\
1.46202757395761 -0.495268138801262\\
1.46214965374835 -0.458560367077717\\
1.46227173353909 -0.187840550616576\\
1.46239381332984 0.17866360768569\\
1.46251589312058 0.4034987094924\\
1.46263797291133 0.421852595354173\\
1.46276005270207 0.242902208201893\\
1.46288213249282 -0.0329796386578721\\
1.46300421228356 -0.187840550616576\\
1.46312629207431 -0.311729280183539\\
1.46324837186505 -0.187840550616576\\
1.4633704516558 -0.0754229997132205\\
1.46349253144654 -0.112130771436765\\
1.46361461123729 -0.0662460567823344\\
1.46373669102803 0.0628047031832521\\
1.46385877081877 0.206194436478348\\
1.46398085060952 0.242902208201893\\
1.46410293040026 0.366790937768856\\
1.46422501019101 0.261256094063665\\
1.46434708998175 -0.0754229997132205\\
1.4644691697725 -0.4034987094924\\
1.46459124956324 -0.568683682248351\\
1.46471332935399 -0.421852595354173\\
1.46483540914473 -0.0662460567823344\\
1.46495748893548 0.348437051907083\\
};
\addplot [
color=blue,
solid,
forget plot
]
table[row sep=crcr]{
1.46495748893548 0.348437051907083\\
1.46507956872622 0.513622024663034\\
1.46520164851697 0.495268138801262\\
1.46532372830771 0.169486664754804\\
1.46544580809845 -0.293375394321767\\
1.4655678878892 -0.421852595354173\\
1.46568996767994 -0.311729280183539\\
1.46581204747069 -0.0891884141095498\\
1.46593412726143 0.121307714367651\\
1.46605620705218 0.293375394321767\\
1.46617828684292 0.187840550616576\\
1.46630036663367 -0.0129050759965586\\
1.46642244642441 -0.160309721823917\\
1.46654452621516 -0.206194436478348\\
1.4666666060059 -0.0266704903928879\\
1.46678868579665 0.0708345282477775\\
1.46691076558739 0.121307714367651\\
1.46703284537813 0.0662460567823344\\
1.46715492516888 0.0937768855749928\\
1.46727700495962 -0.0559219959850875\\
1.46739908475037 -0.0983653570404359\\
1.46752116454111 0.0662460567823344\\
1.46764324433186 0.0467450530542013\\
1.4677653241226 -0.0329796386578721\\
1.46788740391335 -0.206194436478348\\
1.46800948370409 -0.187840550616576\\
1.46813156349484 -0.169486664754804\\
1.46825364328558 0.0628047031832521\\
1.46837572307633 0.330083166045311\\
1.46849780286707 0.440206481215945\\
1.46861988265782 0.421852595354173\\
1.46874196244856 -0.0129050759965586\\
1.4688640422393 -0.385144823630628\\
1.46898612203005 -0.568683682248351\\
1.46910820182079 -0.47691425293949\\
1.46923028161154 -0.197017493547462\\
1.46935236140228 0.242902208201893\\
1.46947444119303 0.596214511041009\\
1.46959652098377 0.495268138801262\\
1.46971860077452 0.233725265271007\\
1.46984068056526 -0.0983653570404359\\
1.46996276035601 -0.311729280183539\\
1.47008484014675 -0.330083166045311\\
1.4702069199375 -0.242902208201893\\
1.47032899972824 -0.028391167192429\\
1.47045107951898 0.0513335245196444\\
1.47057315930973 0.102953828505879\\
1.47069523910047 -0.0174935474620017\\
1.47081731889122 0.0329796386578721\\
1.47093939868196 0.17866360768569\\
1.47106147847271 0.169486664754804\\
1.47118355826345 0.125896185833094\\
1.4713056380542 -0.0232291367938056\\
1.47142771784494 -0.135073128763981\\
1.47154979763569 -0.330083166045311\\
1.47167187742643 -0.270433036994551\\
1.47179395721718 0.0266704903928879\\
1.47191603700792 0.233725265271007\\
1.47203811679866 0.366790937768856\\
1.47216019658941 0.17866360768569\\
1.47228227638015 -0.0708345282477775\\
1.4724043561709 -0.270433036994551\\
1.47252643596164 -0.366790937768856\\
1.47264851575239 -0.169486664754804\\
1.47277059554313 0.17866360768569\\
1.47289267533388 0.47691425293949\\
1.47301475512462 0.348437051907083\\
1.47313683491537 0.112130771436765\\
1.47325891470611 -0.17866360768569\\
1.47338099449686 -0.458560367077717\\
1.4735030742876 -0.458560367077717\\
1.47362515407834 -0.206194436478348\\
1.47374723386909 0.169486664754804\\
1.47386931365983 0.348437051907083\\
1.47399139345058 0.366790937768856\\
1.47411347324132 0.215371379409234\\
1.47423555303207 0.0375681101233152\\
1.47435763282281 -0.0845999426441067\\
1.47447971261356 -0.215371379409234\\
1.4746017924043 -0.160309721823917\\
1.47472387219505 -0.135073128763981\\
1.47484595198579 -0.141955835962145\\
1.47496803177654 -0.17866360768569\\
1.47509011156728 -0.028391167192429\\
1.47521219135802 0.17866360768569\\
1.47533427114877 0.293375394321767\\
1.47545635093951 0.4034987094924\\
1.47557843073026 0.279609979925437\\
1.475700510521 0.00802982506452538\\
1.47582259031175 -0.348437051907083\\
1.47594467010249 -0.531975910524806\\
1.47606674989324 -0.421852595354173\\
1.47618882968398 -0.0708345282477775\\
1.47631090947473 0.279609979925437\\
1.47643298926547 0.421852595354173\\
1.47655506905622 0.4034987094924\\
1.47667714884696 0.141955835962145\\
1.4767992286377 -0.206194436478348\\
1.47692130842845 -0.385144823630628\\
1.47704338821919 -0.22454832234012\\
1.47716546800994 -0.0186406653283625\\
1.47728754780068 0.141955835962145\\
1.47740962759143 0.233725265271007\\
1.47753170738217 0.141955835962145\\
1.47765378717292 -0.0605104674505305\\
1.47777586696366 -0.206194436478348\\
1.47789794675441 -0.160309721823917\\
1.47802002654515 -0.0467450530542013\\
1.4781421063359 0.0983653570404359\\
1.47826418612664 0.116719242902208\\
1.47838626591739 0.0490392887869229\\
1.47850834570813 0.0628047031832521\\
1.47863042549887 0.00401491253226269\\
1.47875250528962 -0.0174935474620017\\
1.47887458508036 0.0800114711786636\\
1.47899666487111 0.112130771436765\\
1.47911874466185 -0.0845999426441067\\
1.4792408244526 -0.233725265271007\\
1.47936290424334 -0.293375394321767\\
1.47948498403409 -0.22454832234012\\
1.47960706382483 0.0398623458560367\\
1.47972914361558 0.330083166045311\\
1.47985122340632 0.513622024663034\\
1.47997330319707 0.440206481215945\\
1.48009538298781 0.112130771436765\\
1.48021746277855 -0.330083166045311\\
1.4803395425693 -0.568683682248351\\
1.48046162236004 -0.458560367077717\\
1.48058370215079 -0.252079151132779\\
1.48070578194153 0.116719242902208\\
1.48082786173228 0.458560367077717\\
1.48094994152302 0.458560367077717\\
1.48107202131377 0.261256094063665\\
1.48119410110451 -0.0352738743905936\\
1.48131618089526 -0.169486664754804\\
1.481438260686 -0.206194436478348\\
1.48156034047675 -0.151132778893031\\
1.48168242026749 -0.0983653570404359\\
1.48180450005823 -0.0559219959850875\\
1.48192657984898 0.028391167192429\\
1.48204865963972 -0.0421565815887582\\
1.48217073943047 0.00458847146544307\\
1.48229281922121 0.151132778893031\\
1.48241489901196 0.22454832234012\\
1.4825369788027 0.141955835962145\\
1.48265905859345 -0.0375681101233152\\
1.48278113838419 -0.112130771436765\\
1.48290321817494 -0.22454832234012\\
1.48302529796568 -0.197017493547462\\
1.48314737775643 -0.0375681101233152\\
1.48326945754717 0.197017493547462\\
1.48339153733791 0.293375394321767\\
1.48351361712866 0.141955835962145\\
1.4836356969194 -0.0754229997132205\\
1.48375777671015 -0.293375394321767\\
1.48387985650089 -0.270433036994551\\
1.48400193629164 -0.169486664754804\\
1.48412401608238 0.0891884141095498\\
1.48424609587313 0.4034987094924\\
1.48436817566387 0.440206481215945\\
1.48449025545462 0.187840550616576\\
1.48461233524536 -0.141955835962145\\
1.48473441503611 -0.348437051907083\\
1.48485649482685 -0.421852595354173\\
1.48497857461759 -0.293375394321767\\
1.48510065440834 -0.0151993117292802\\
1.48522273419908 0.242902208201893\\
1.48534481398983 0.366790937768856\\
1.48546689378057 0.252079151132779\\
1.48558897357132 0.0800114711786636\\
1.48571105336206 0.00630914826498423\\
1.48583313315281 -0.0398623458560367\\
1.48595521294355 -0.116719242902208\\
1.4860772927343 -0.141955835962145\\
1.48619937252504 -0.169486664754804\\
1.48632145231579 -0.22454832234012\\
1.48644353210653 -0.187840550616576\\
1.48656561189727 0.0117579581301979\\
1.48668769168802 0.293375394321767\\
1.48680977147876 0.440206481215945\\
1.48693185126951 0.385144823630628\\
1.48705393106025 0.116719242902208\\
1.487176010851 -0.215371379409234\\
1.48729809064174 -0.421852595354173\\
1.48742017043249 -0.47691425293949\\
1.48754225022323 -0.233725265271007\\
1.48766433001398 0.17866360768569\\
1.48778640980472 0.385144823630628\\
1.48790848959547 0.330083166045311\\
1.48803056938621 0.169486664754804\\
1.48815264917695 -0.0628047031832521\\
1.4882747289677 -0.293375394321767\\
1.48839680875844 -0.279609979925437\\
1.48851888854919 -0.058216231717809\\
1.48864096833993 0.151132778893031\\
1.48876304813068 0.233725265271007\\
1.48888512792142 0.151132778893031\\
1.48900720771217 -0.0197877831947233\\
1.48912928750291 -0.125896185833094\\
1.48925136729366 -0.187840550616576\\
1.4893734470844 -0.160309721823917\\
1.48949552687515 -0.00286779466590192\\
1.48961760666589 0.135073128763981\\
1.48973968645663 0.0662460567823344\\
1.48986176624738 0.0232291367938056\\
1.48998384603812 0.0845999426441067\\
1.49010592582887 0.0754229997132205\\
1.49022800561961 0.0937768855749928\\
1.49035008541036 0.0628047031832521\\
1.4904721652011 0.00516203039862346\\
1.49059424499185 -0.197017493547462\\
1.49071632478259 -0.311729280183539\\
1.49083840457334 -0.293375394321767\\
1.49096048436408 -0.125896185833094\\
1.49108256415483 0.252079151132779\\
1.49120464394557 0.4034987094924\\
1.49132672373632 0.458560367077717\\
1.49144880352706 0.311729280183539\\
1.4915708833178 -0.0490392887869229\\
1.49169296310855 -0.440206481215945\\
1.49181504289929 -0.550329796386579\\
1.49193712269004 -0.330083166045311\\
1.49205920248078 -0.107542299971322\\
1.49218128227153 0.242902208201893\\
1.49230336206227 0.4034987094924\\
1.49242544185302 0.348437051907083\\
1.49254752164376 0.151132778893031\\
1.49266960143451 -0.0983653570404359\\
1.49279168122525 -0.187840550616576\\
1.492913761016 -0.17866360768569\\
1.49303584080674 -0.0605104674505305\\
1.49315792059748 -0.0559219959850875\\
1.49328000038823 -0.0352738743905936\\
1.49340208017897 0.0232291367938056\\
1.49352415996972 -0.0490392887869229\\
1.49364623976046 0.020934901061084\\
1.49376831955121 0.116719242902208\\
1.49389039934195 0.197017493547462\\
1.4940124791327 0.0800114711786636\\
1.49413455892344 -0.0421565815887582\\
1.49425663871419 -0.125896185833094\\
1.49437871850493 -0.206194436478348\\
1.49450079829568 -0.0891884141095498\\
1.49462287808642 0.0605104674505305\\
1.49474495787716 0.22454832234012\\
1.49486703766791 0.215371379409234\\
1.49498911745865 0.0513335245196444\\
1.4951111972494 -0.215371379409234\\
1.49523327704014 -0.293375394321767\\
1.49535535683089 -0.215371379409234\\
1.49547743662163 -0.0845999426441067\\
1.49559951641238 0.242902208201893\\
1.49572159620312 0.440206481215945\\
1.49584367599387 0.366790937768856\\
1.49596575578461 0.0708345282477775\\
1.49608783557536 -0.197017493547462\\
1.4962099153661 -0.366790937768856\\
1.49633199515684 -0.4034987094924\\
1.49645407494759 -0.197017493547462\\
1.49657615473833 0.0306854029251506\\
1.49669823452908 0.293375394321767\\
1.49682031431982 0.293375394321767\\
1.49694239411057 0.233725265271007\\
1.49706447390131 0.0845999426441067\\
1.49718655369206 -0.0186406653283625\\
1.4973086334828 -0.0708345282477775\\
1.49743071327355 -0.160309721823917\\
1.49755279306429 -0.121307714367651\\
1.49767487285504 -0.17866360768569\\
1.49779695264578 -0.17866360768569\\
1.49791903243652 -0.0891884141095498\\
1.49804111222727 0.0800114711786636\\
1.49816319201801 0.293375394321767\\
1.49828527180876 0.330083166045311\\
1.4984073515995 0.270433036994551\\
1.49852943139025 0.0186406653283625\\
1.49865151118099 -0.197017493547462\\
1.49877359097174 -0.348437051907083\\
1.49889567076248 -0.348437051907083\\
1.49901775055323 -0.102953828505879\\
1.49913983034397 0.121307714367651\\
1.49926191013472 0.311729280183539\\
1.49938398992546 0.261256094063665\\
1.4995060697162 0.102953828505879\\
1.49962814950695 -0.0937768855749928\\
1.49975022929769 -0.252079151132779\\
1.49987230908844 -0.17866360768569\\
1.49999438887918 -0.0513335245196444\\
1.50011646866993 0.135073128763981\\
1.50023854846067 0.233725265271007\\
1.50036062825142 0.151132778893031\\
1.50048270804216 -0.00114711786636077\\
1.50060478783291 -0.141955835962145\\
1.50072686762365 -0.187840550616576\\
1.5008489474144 -0.187840550616576\\
1.50097102720514 -0.0628047031832521\\
1.50109310699588 0.0800114711786636\\
1.50121518678663 0.121307714367651\\
1.50133726657737 0.135073128763981\\
1.50145934636812 0.141955835962145\\
1.50158142615886 0.0937768855749928\\
1.50170350594961 0.0513335245196444\\
1.50182558574035 -0.0352738743905936\\
1.5019476655311 -0.135073128763981\\
1.50206974532184 -0.197017493547462\\
1.50219182511259 -0.22454832234012\\
1.50231390490333 -0.169486664754804\\
1.50243598469408 -0.028391167192429\\
1.50255806448482 0.215371379409234\\
1.50268014427556 0.348437051907083\\
1.50280222406631 0.348437051907083\\
1.50292430385705 0.197017493547462\\
1.5030463836478 -0.0536277602523659\\
1.50316846343854 -0.330083166045311\\
1.50329054322929 -0.440206481215945\\
1.50341262302003 -0.293375394321767\\
1.50353470281078 -0.0891884141095498\\
1.50365678260152 0.17866360768569\\
1.50377886239227 0.330083166045311\\
1.50390094218301 0.311729280183539\\
1.50402302197376 0.169486664754804\\
1.5041451017645 -0.0352738743905936\\
1.50426718155525 -0.17866360768569\\
1.50438926134599 -0.206194436478348\\
1.50451134113673 -0.107542299971322\\
1.50463342092748 -0.0605104674505305\\
1.50475550071822 0.0106108402638371\\
1.50487758050897 0.102953828505879\\
1.50499966029971 0.0708345282477775\\
1.50512174009046 0.00688270719816461\\
1.5052438198812 0.00172067679954115\\
1.50536589967195 0.020934901061084\\
1.50548797946269 -0.0140521938629194\\
1.50561005925344 -0.0243762546601663\\
1.50573213904418 -0.028391167192429\\
1.50585421883493 -0.0140521938629194\\
1.50597629862567 0.0398623458560367\\
1.50609837841641 0.0800114711786636\\
1.50622045820716 0.0352738743905936\\
1.5063425379979 0.0536277602523659\\
1.50646461778865 -0.0421565815887582\\
1.50658669757939 -0.160309721823917\\
1.50670877737014 -0.169486664754804\\
1.50683085716088 -0.102953828505879\\
1.50695293695163 0.0140521938629194\\
1.50707501674237 0.151132778893031\\
1.50719709653312 0.293375394321767\\
1.50731917632386 0.242902208201893\\
1.50744125611461 0.141955835962145\\
1.50756333590535 -0.0937768855749928\\
1.50768541569609 -0.293375394321767\\
1.50780749548684 -0.330083166045311\\
1.50792957527758 -0.270433036994551\\
1.50805165506833 -0.0845999426441067\\
1.50817373485907 0.135073128763981\\
1.50829581464982 0.366790937768856\\
1.50841789444056 0.348437051907083\\
1.50853997423131 0.197017493547462\\
1.50866205402205 0.0375681101233152\\
1.5087841338128 -0.151132778893031\\
1.50890621360354 -0.270433036994551\\
1.50902829339429 -0.279609979925437\\
1.50915037318503 -0.151132778893031\\
1.50927245297577 -0.0352738743905936\\
1.50939453276652 0.0983653570404359\\
1.50951661255726 0.17866360768569\\
1.50963869234801 0.17866360768569\\
1.50976077213875 0.151132778893031\\
1.5098828519295 0.0444508173214798\\
1.51000493172024 -0.0398623458560367\\
1.51012701151099 -0.112130771436765\\
1.51024909130173 -0.112130771436765\\
1.51037117109248 -0.135073128763981\\
1.51049325088322 -0.0845999426441067\\
1.51061533067397 0.0467450530542013\\
1.51073741046471 0.0845999426441067\\
1.51085949025545 0.107542299971322\\
1.5109815700462 0.0845999426441067\\
1.51110364983694 0.028391167192429\\
1.51122572962769 -0.0708345282477775\\
1.51134780941843 -0.116719242902208\\
1.51146988920918 -0.0754229997132205\\
1.51159196899992 -0.00286779466590192\\
1.51171404879067 0.112130771436765\\
1.51183612858141 0.151132778893031\\
1.51195820837216 0.107542299971322\\
1.5120802881629 0.00630914826498423\\
1.51220236795365 -0.121307714367651\\
1.51232444774439 -0.22454832234012\\
1.51244652753513 -0.197017493547462\\
1.51256860732588 -0.0800114711786636\\
1.51269068711662 0.0800114711786636\\
1.51281276690737 0.233725265271007\\
1.51293484669811 0.293375394321767\\
1.51305692648886 0.215371379409234\\
1.5131790062796 0.0559219959850875\\
1.51330108607035 -0.130484657298537\\
1.51342316586109 -0.279609979925437\\
1.51354524565184 -0.279609979925437\\
1.51366732544258 -0.187840550616576\\
1.51378940523333 -0.0467450530542013\\
1.51391148502407 0.112130771436765\\
1.51403356481481 0.261256094063665\\
1.51415564460556 0.270433036994551\\
1.5142777243963 0.169486664754804\\
1.51439980418705 0.0662460567823344\\
1.51452188397779 -0.0754229997132205\\
1.51464396376854 -0.187840550616576\\
1.51476604355928 -0.233725265271007\\
1.51488812335003 -0.187840550616576\\
1.51501020314077 -0.0983653570404359\\
1.51513228293152 0.0220820189274448\\
1.51525436272226 0.151132778893031\\
1.51537644251301 0.197017493547462\\
1.51549852230375 0.206194436478348\\
1.51562060209449 0.112130771436765\\
1.51574268188524 -0.0398623458560367\\
1.51586476167598 -0.151132778893031\\
1.51598684146673 -0.17866360768569\\
1.51610892125747 -0.141955835962145\\
1.51623100104822 -0.058216231717809\\
1.51635308083896 0.0800114711786636\\
1.51647516062971 0.141955835962145\\
1.51659724042045 0.116719242902208\\
1.5167193202112 0.0444508173214798\\
1.51684140000194 -0.0513335245196444\\
1.51696347979269 -0.107542299971322\\
1.51708555958343 -0.0983653570404359\\
1.51720763937417 -0.0352738743905936\\
1.51732971916492 0.0536277602523659\\
1.51745179895566 0.160309721823917\\
1.51757387874641 0.160309721823917\\
1.51769595853715 0.0444508173214798\\
1.5178180383279 -0.0513335245196444\\
1.51794011811864 -0.151132778893031\\
1.51806219790939 -0.197017493547462\\
1.51818427770013 -0.160309721823917\\
1.51830635749088 -0.0266704903928879\\
1.51842843728162 0.112130771436765\\
1.51855051707237 0.187840550616576\\
1.51867259686311 0.206194436478348\\
1.51879467665386 0.160309721823917\\
1.5189167564446 0.0662460567823344\\
1.51903883623534 -0.0983653570404359\\
1.51916091602609 -0.197017493547462\\
1.51928299581683 -0.22454832234012\\
1.51940507560758 -0.169486664754804\\
1.51952715539832 -0.102953828505879\\
1.51964923518907 0.0329796386578721\\
1.51977131497981 0.197017493547462\\
1.51989339477056 0.242902208201893\\
1.5200154745613 0.215371379409234\\
1.52013755435205 0.116719242902208\\
1.52025963414279 -0.0117579581301979\\
1.52038171393354 -0.169486664754804\\
1.52050379372428 -0.242902208201893\\
1.52062587351502 -0.233725265271007\\
1.52074795330577 -0.121307714367651\\
1.52087003309651 0.0306854029251506\\
1.52099211288726 0.141955835962145\\
1.521114192678 0.22454832234012\\
1.52123627246875 0.197017493547462\\
1.52135835225949 0.0845999426441067\\
1.52148043205024 -0.0559219959850875\\
1.52160251184098 -0.130484657298537\\
1.52172459163173 -0.160309721823917\\
1.52184667142247 -0.102953828505879\\
1.52196875121322 -0.0329796386578721\\
1.52209083100396 0.0628047031832521\\
1.5222129107947 0.121307714367651\\
1.52233499058545 0.0708345282477775\\
1.52245707037619 0.00946372239747634\\
1.52257915016694 -0.0490392887869229\\
1.52270122995768 -0.0605104674505305\\
1.52282330974843 -0.0845999426441067\\
1.52294538953917 -0.0129050759965586\\
1.52306746932992 0.0605104674505305\\
1.52318954912066 0.107542299971322\\
1.52331162891141 0.125896185833094\\
1.52343370870215 0.0559219959850875\\
1.5235557884929 -0.0186406653283625\\
1.52367786828364 -0.116719242902208\\
1.52379994807438 -0.151132778893031\\
1.52392202786513 -0.17866360768569\\
1.52404410765587 -0.0754229997132205\\
1.52416618744662 0.0708345282477775\\
1.52428826723736 0.125896185833094\\
1.52441034702811 0.215371379409234\\
1.52453242681885 0.206194436478348\\
1.5246545066096 0.102953828505879\\
1.52477658640034 -0.0490392887869229\\
1.52489866619109 -0.169486664754804\\
1.52502074598183 -0.233725265271007\\
1.52514282577258 -0.22454832234012\\
1.52526490556332 -0.102953828505879\\
1.52538698535406 0.00630914826498423\\
1.52550906514481 0.169486664754804\\
1.52563114493555 0.261256094063665\\
1.5257532247263 0.215371379409234\\
1.52587530451704 0.112130771436765\\
1.52599738430779 -0.00344135359908231\\
1.52611946409853 -0.116719242902208\\
1.52624154388928 -0.22454832234012\\
1.52636362368002 -0.197017493547462\\
1.52648570347077 -0.116719242902208\\
1.52660778326151 -0.0106108402638371\\
1.52672986305226 0.0983653570404359\\
1.526851942843 0.169486664754804\\
1.52697402263374 0.169486664754804\\
1.52709610242449 0.102953828505879\\
1.52721818221523 0.0151993117292802\\
1.52734026200598 -0.102953828505879\\
1.52746234179672 -0.112130771436765\\
1.52758442158747 -0.102953828505879\\
1.52770650137821 -0.0708345282477775\\
1.52782858116896 0.00688270719816461\\
1.5279506609597 0.0845999426441067\\
1.52807274075045 0.0983653570404359\\
1.52819482054119 0.0329796386578721\\
1.52831690033194 -0.0174935474620017\\
1.52843898012268 -0.0662460567823344\\
1.52856105991342 -0.0754229997132205\\
1.52868313970417 -0.0306854029251506\\
1.52880521949491 0.0232291367938056\\
1.52892729928566 0.102953828505879\\
1.5290493790764 0.141955835962145\\
1.52917145886715 0.0800114711786636\\
1.52929353865789 -0.020934901061084\\
1.52941561844864 -0.0891884141095498\\
1.52953769823938 -0.141955835962145\\
1.52965977803013 -0.169486664754804\\
1.52978185782087 -0.0983653570404359\\
1.52990393761162 0.0266704903928879\\
1.53002601740236 0.116719242902208\\
1.53014809719311 0.187840550616576\\
1.53027017698385 0.187840550616576\\
1.53039225677459 0.116719242902208\\
1.53051433656534 0.0151993117292802\\
1.53063641635608 -0.116719242902208\\
1.53075849614683 -0.22454832234012\\
1.53088057593757 -0.197017493547462\\
1.53100265572832 -0.141955835962145\\
1.53112473551906 -0.0375681101233152\\
1.53124681530981 0.0891884141095498\\
1.53136889510055 0.215371379409234\\
1.5314909748913 0.22454832234012\\
1.53161305468204 0.151132778893031\\
1.53173513447279 0.0536277602523659\\
1.53185721426353 -0.0800114711786636\\
1.53197929405427 -0.160309721823917\\
1.53210137384502 -0.22454832234012\\
1.53222345363576 -0.160309721823917\\
1.53234553342651 -0.0536277602523659\\
1.53246761321725 0.0628047031832521\\
1.532589693008 0.141955835962145\\
1.53271177279874 0.169486664754804\\
1.53283385258949 0.141955835962145\\
1.53295593238023 0.0306854029251506\\
1.53307801217098 -0.0662460567823344\\
1.53320009196172 -0.125896185833094\\
1.53332217175247 -0.0891884141095498\\
1.53344425154321 -0.0559219959850875\\
1.53356633133395 -0.0197877831947233\\
1.5336884111247 0.058216231717809\\
1.53381049091544 0.0754229997132205\\
1.53393257070619 0.0467450530542013\\
1.53405465049693 -0.0106108402638371\\
1.53417673028768 -0.0398623458560367\\
1.53429881007842 -0.0605104674505305\\
1.53442088986917 -0.0444508173214798\\
1.53454296965991 -0.000573558933180384\\
1.53466504945066 0.0559219959850875\\
1.5347871292414 0.116719242902208\\
1.53490920903215 0.0937768855749928\\
1.53503128882289 0.0421565815887582\\
1.53515336861363 -0.0444508173214798\\
1.53527544840438 -0.102953828505879\\
1.53539752819512 -0.160309721823917\\
1.53551960798587 -0.130484657298537\\
1.53564168777661 -0.0329796386578721\\
1.53576376756736 0.0628047031832521\\
1.5358858473581 0.151132778893031\\
1.53600792714885 0.17866360768569\\
1.53613000693959 0.169486664754804\\
1.53625208673034 0.0708345282477775\\
1.53637416652108 -0.0628047031832521\\
1.53649624631183 -0.169486664754804\\
1.53661832610257 -0.206194436478348\\
1.53674040589331 -0.17866360768569\\
1.53686248568406 -0.0891884141095498\\
1.5369845654748 0.0444508173214798\\
1.53710664526555 0.151132778893031\\
1.53722872505629 0.22454832234012\\
1.53735080484704 0.17866360768569\\
1.53747288463778 0.0845999426441067\\
1.53759496442853 -0.0421565815887582\\
1.53771704421927 -0.121307714367651\\
1.53783912401002 -0.187840550616576\\
1.53796120380076 -0.160309721823917\\
1.53808328359151 -0.0708345282477775\\
1.53820536338225 0.00630914826498423\\
1.53832744317299 0.0937768855749928\\
1.53844952296374 0.135073128763981\\
1.53857160275448 0.135073128763981\\
1.53869368254523 0.0662460567823344\\
1.53881576233597 -0.00860338399770576\\
1.53893784212672 -0.0708345282477775\\
1.53905992191746 -0.102953828505879\\
1.53918200170821 -0.0708345282477775\\
1.53930408149895 -0.0352738743905936\\
1.5394261612897 0.0255233725265271\\
1.53954824108044 0.0536277602523659\\
1.53967032087119 0.058216231717809\\
1.53979240066193 0.0186406653283625\\
1.53991448045268 -0.0266704903928879\\
1.54003656024342 -0.0352738743905936\\
1.54015864003416 -0.0662460567823344\\
1.54028071982491 -0.0174935474620017\\
1.54040279961565 0.028391167192429\\
1.5405248794064 0.0708345282477775\\
1.54064695919714 0.0937768855749928\\
1.54076903898789 0.0754229997132205\\
1.54089111877863 -0.00114711786636077\\
1.54101319856938 -0.0754229997132205\\
1.54113527836012 -0.102953828505879\\
1.54125735815087 -0.151132778893031\\
1.54137943794161 -0.0800114711786636\\
1.54150151773236 -0.000573558933180384\\
1.5416235975231 0.0983653570404359\\
1.54174567731384 0.151132778893031\\
1.54186775710459 0.160309721823917\\
1.54198983689533 0.116719242902208\\
1.54211191668608 0.00286779466590192\\
1.54223399647682 -0.0754229997132205\\
1.54235607626757 -0.169486664754804\\
1.54247815605831 -0.169486664754804\\
1.54260023584906 -0.130484657298537\\
1.5427223156398 -0.0375681101233152\\
1.54284439543055 0.0662460567823344\\
1.54296647522129 0.160309721823917\\
1.54308855501204 0.206194436478348\\
1.54321063480278 0.130484657298537\\
1.54333271459352 0.0513335245196444\\
1.54345479438427 -0.0708345282477775\\
1.54357687417501 -0.135073128763981\\
1.54369895396576 -0.169486664754804\\
1.5438210337565 -0.130484657298537\\
1.54394311354725 -0.0444508173214798\\
1.54406519333799 0.0444508173214798\\
1.54418727312874 0.116719242902208\\
1.54430935291948 0.125896185833094\\
1.54443143271023 0.112130771436765\\
1.54455351250097 0.020934901061084\\
1.54467559229172 -0.028391167192429\\
1.54479767208246 -0.0845999426441067\\
1.5449197518732 -0.0845999426441067\\
1.54504183166395 -0.0536277602523659\\
1.54516391145469 -0.0140521938629194\\
1.54528599124544 0.0421565815887582\\
1.54540807103618 0.0444508173214798\\
1.54553015082693 0.0628047031832521\\
1.54565223061767 -0.00229423573272154\\
1.54577431040842 -0.0329796386578721\\
1.54589639019916 -0.0559219959850875\\
1.54601846998991 -0.0467450530542013\\
1.54614054978065 0.00344135359908231\\
1.5462626295714 0.0329796386578721\\
1.54638470936214 0.0891884141095498\\
1.54650678915288 0.0708345282477775\\
1.54662886894363 0.0536277602523659\\
1.54675094873437 -0.020934901061084\\
1.54687302852512 -0.0845999426441067\\
1.54699510831586 -0.107542299971322\\
1.54711718810661 -0.107542299971322\\
1.54723926789735 -0.0513335245196444\\
1.5473613476881 0.00860338399770576\\
1.54748342747884 0.112130771436765\\
1.54760550726959 0.141955835962145\\
1.54772758706033 0.135073128763981\\
1.54784966685108 0.0845999426441067\\
1.54797174664182 -0.0129050759965586\\
1.54809382643256 -0.102953828505879\\
1.54821590622331 -0.160309721823917\\
1.54833798601405 -0.151132778893031\\
1.5484600658048 -0.102953828505879\\
1.54858214559554 0.0186406653283625\\
1.54870422538629 0.0891884141095498\\
1.54882630517703 0.151132778893031\\
1.54894838496778 0.169486664754804\\
1.54907046475852 0.0937768855749928\\
1.54919254454927 -0.00688270719816461\\
1.54931462434001 -0.0891884141095498\\
1.54943670413076 -0.112130771436765\\
1.5495587839215 -0.141955835962145\\
1.54968086371224 -0.0891884141095498\\
1.54980294350299 -0.0117579581301979\\
1.54992502329373 0.0559219959850875\\
1.55004710308448 0.0891884141095498\\
1.55016918287522 0.0937768855749928\\
1.55029126266597 0.0754229997132205\\
1.55041334245671 0.0306854029251506\\
1.55053542224746 -0.0232291367938056\\
1.5506575020382 -0.0937768855749928\\
1.55077958182895 -0.0662460567823344\\
1.55090166161969 -0.0375681101233152\\
1.55102374141044 -0.0129050759965586\\
1.55114582120118 0.028391167192429\\
1.55126790099192 0.0605104674505305\\
1.55138998078267 0.0513335245196444\\
1.55151206057341 -0.0129050759965586\\
1.55163414036416 -0.0306854029251506\\
1.5517562201549 -0.0467450530542013\\
1.55187829994565 -0.0232291367938056\\
1.55200037973639 -0.00344135359908231\\
1.55212245952714 0.0306854029251506\\
1.55224453931788 0.0708345282477775\\
1.55236661910863 0.058216231717809\\
1.55248869889937 0.0220820189274448\\
1.55261077869012 -0.0352738743905936\\
1.55273285848086 -0.0490392887869229\\
1.55285493827161 -0.0708345282477775\\
1.55297701806235 -0.0800114711786636\\
1.55309909785309 -0.0467450530542013\\
1.55322117764384 0.0352738743905936\\
1.55334325743458 0.0845999426441067\\
1.55346533722533 0.0891884141095498\\
1.55358741701607 0.107542299971322\\
1.55370949680682 0.058216231717809\\
1.55383157659756 -0.016346429595641\\
1.55395365638831 -0.0937768855749928\\
1.55407573617905 -0.125896185833094\\
1.5541978159698 -0.116719242902208\\
1.55431989576054 -0.0662460567823344\\
1.55444197555129 0.00172067679954115\\
1.55456405534203 0.0708345282477775\\
1.55468613513277 0.141955835962145\\
1.55480821492352 0.116719242902208\\
1.55493029471426 0.0662460567823344\\
1.55505237450501 0.00946372239747634\\
1.55517445429575 -0.0490392887869229\\
1.5552965340865 -0.107542299971322\\
1.55541861387724 -0.116719242902208\\
1.55554069366799 -0.0708345282477775\\
1.55566277345873 -0.020934901061084\\
1.55578485324948 0.0513335245196444\\
1.55590693304022 0.0754229997132205\\
1.55602901283097 0.0937768855749928\\
1.55615109262171 0.0662460567823344\\
1.55627317241245 0.00802982506452538\\
1.5563952522032 -0.0352738743905936\\
1.55651733199394 -0.0444508173214798\\
1.55663941178469 -0.0421565815887582\\
1.55676149157543 -0.0559219959850875\\
1.55688357136618 -0.00516203039862346\\
1.55700565115692 0.0129050759965586\\
1.55712773094767 0.020934901061084\\
1.55724981073841 0.0255233725265271\\
1.55737189052916 0.00458847146544307\\
1.5574939703199 -0.00286779466590192\\
1.55761605011065 -0.0186406653283625\\
1.55773812990139 -0.00688270719816461\\
1.55786020969213 -0.00860338399770576\\
1.55798228948288 0.0329796386578721\\
1.55810436927362 0.0306854029251506\\
1.55822644906437 0.0197877831947233\\
1.55834852885511 0.0243762546601663\\
1.55847060864586 -0.0129050759965586\\
1.5585926884366 -0.0444508173214798\\
1.55871476822735 -0.058216231717809\\
1.55883684801809 -0.0398623458560367\\
1.55895892780884 -0.028391167192429\\
1.55908100759958 0.0117579581301979\\
1.55920308739033 0.0605104674505305\\
1.55932516718107 0.0845999426441067\\
1.55944724697181 0.0754229997132205\\
1.55956932676256 0.0306854029251506\\
1.5596914065533 -0.0117579581301979\\
1.55981348634405 -0.058216231717809\\
1.55993556613479 -0.0891884141095498\\
1.56005764592554 -0.0891884141095498\\
1.56017972571628 -0.0559219959850875\\
1.56030180550703 0.0117579581301979\\
1.56042388529777 0.0559219959850875\\
1.56054596508852 0.0754229997132205\\
1.56066804487926 0.0845999426441067\\
1.56079012467001 0.0628047031832521\\
1.56091220446075 0.00229423573272154\\
1.56103428425149 -0.0467450530542013\\
1.56115636404224 -0.0536277602523659\\
1.56127844383298 -0.0605104674505305\\
1.56140052362373 -0.058216231717809\\
1.56152260341447 -0.0266704903928879\\
1.56164468320522 0.0106108402638371\\
1.56176676299596 0.0421565815887582\\
1.56188884278671 0.0490392887869229\\
1.56201092257745 0.0398623458560367\\
1.5621330023682 0.0375681101233152\\
1.56225508215894 0.016346429595641\\
1.56237716194969 -0.028391167192429\\
1.56249924174043 -0.0444508173214798\\
1.56262132153117 -0.0232291367938056\\
1.56274340132192 -0.0232291367938056\\
1.56286548111266 -0.00946372239747634\\
1.56298756090341 0.00458847146544307\\
1.56310964069415 0.00688270719816461\\
1.5632317204849 0.0140521938629194\\
1.56335380027564 -0.00114711786636077\\
1.56347588006639 -0.00344135359908231\\
1.56359795985713 0.00946372239747634\\
1.56372003964788 0.0266704903928879\\
1.56384211943862 0.00516203039862346\\
1.56396419922937 0.00114711786636077\\
1.56408627902011 0.0266704903928879\\
1.56420835881085 -0.0117579581301979\\
1.5643304386016 -0.0306854029251506\\
1.56445251839234 -0.020934901061084\\
1.56457459818309 -0.0197877831947233\\
1.56469667797383 -0.0197877831947233\\
1.56481875776458 -0.00401491253226269\\
1.56494083755532 0.0243762546601663\\
1.56506291734607 0.0352738743905936\\
1.56518499713681 0.0467450530542013\\
1.56530707692756 0.0129050759965586\\
1.5654291567183 0.00172067679954115\\
1.56555123650905 0.00516203039862346\\
1.56567331629979 -0.0266704903928879\\
1.56579539609054 -0.0467450530542013\\
1.56591747588128 -0.0352738743905936\\
1.56603955567202 -0.0140521938629194\\
1.56616163546277 -0.0151993117292802\\
1.56628371525351 0.016346429595641\\
1.56640579504426 0.0398623458560367\\
1.566527874835 0.0421565815887582\\
1.56664995462575 0.0306854029251506\\
1.56677203441649 0.0106108402638371\\
1.56689411420724 -0.00946372239747634\\
1.56701619399798 -0.020934901061084\\
1.56713827378873 -0.0266704903928879\\
1.56726035357947 -0.0306854029251506\\
1.56738243337022 -0.00946372239747634\\
1.56750451316096 -0.00286779466590192\\
1.5676265929517 -0.00573558933180384\\
1.56774867274245 0.00401491253226269\\
1.56787075253319 0.0243762546601663\\
1.56799283232394 0.0243762546601663\\
1.56811491211468 0.016346429595641\\
1.56823699190543 0.020934901061084\\
1.56835907169617 0.00516203039862346\\
1.56848115148692 -0.0129050759965586\\
1.56860323127766 -0.020934901061084\\
1.56872531106841 -0.0232291367938056\\
1.56884739085915 -0.0186406653283625\\
1.5689694706499 -0.00946372239747634\\
1.56909155044064 -0.00229423573272154\\
1.56921363023138 0.00229423573272154\\
1.56933571002213 0.0197877831947233\\
1.56945778981287 0.0129050759965586\\
1.56957986960362 0.0140521938629194\\
1.56970194939436 0.0255233725265271\\
1.56982402918511 0.0129050759965586\\
1.56994610897585 -0.00946372239747634\\
1.5700681887666 -0.0186406653283625\\
1.57019026855734 -0.0151993117292802\\
1.57031234834809 -0.0243762546601663\\
1.57043442813883 -0.0220820189274448\\
1.57055650792958 -0.00458847146544307\\
1.57067858772032 0.00458847146544307\\
1.57080066751106 0.0174935474620017\\
1.57092274730181 0.0186406653283625\\
1.57104482709255 0.0243762546601663\\
1.5711669068833 0.0243762546601663\\
1.57128898667404 0.00946372239747634\\
1.57141106646479 -0.0140521938629194\\
1.57153314625553 -0.020934901061084\\
1.57165522604628 -0.0129050759965586\\
1.57177730583702 -0.0232291367938056\\
1.57189938562777 -0.0151993117292802\\
1.57202146541851 -0.00344135359908231\\
1.57214354520926 0.0129050759965586\\
1.572265625 0.016346429595641\\
};
\end{axis}
\end{tikzpicture}%
	\caption{The train signal in the time domain.}
	\label{fig:timedomain}
\end{figure}



\begin{figure}[H]
	\centering
	\setlength\figureheight{6cm}
  	\setlength\figurewidth{10cm}
	% This file was created by matlab2tikz v0.4.6 running on MATLAB 7.13.
% Copyright (c) 2008--2014, Nico Schlömer <nico.schloemer@gmail.com>
% All rights reserved.
% Minimal pgfplots version: 1.3
% 
% The latest updates can be retrieved from
%   http://www.mathworks.com/matlabcentral/fileexchange/22022-matlab2tikz
% where you can also make suggestions and rate matlab2tikz.
% 
\begin{tikzpicture}

\begin{axis}[%
width=\figurewidth,
height=\figureheight,
scale only axis,
xmin=-4,
xmax=4,
ymin=0,
ymax=1000
]
\addplot [color=blue,solid,forget plot]
  table[row sep=crcr]{
-3.14159265358979	7.36765127616863	\\
-3.14159265358979	7.36765127616863	\\
-3.13964135380495	0.076640659735262	\\
-3.11915270606415	0.941932318046942	\\
-3.11281098176343	0.0710510701214799	\\
-3.11085968197859	1.13197498697061	\\
-3.1011031830544	0.0793218010754745	\\
-3.09671275853851	0.00380108186853128	\\
-3.09232233402262	0.886666688407004	\\
-3.07963888542117	1.00079331082098	\\
-3.07915106047497	0.180865362422311	\\
-3.06256501230384	1.12865832757824	\\
-3.0557354630569	0.0803265305036873	\\
-3.05378416327207	0.833454805316724	\\
-3.04597896413271	0.166401142307705	\\
-3.03817376499336	0.97082660500485	\\
-3.03085639080021	0.0999513669044146	\\
-3.01963641703739	0.0809176065351206	\\
-3.01573381746772	0.988715613233601	\\
-3.00548949359732	1.1997206794082	\\
-3.0045138437049	0.131015777209727	\\
-2.99280604499587	0.860926569085718	\\
-2.98792779553377	0.0706535820430292	\\
-2.97817129660958	0.0616518189864895	\\
-2.97524434693232	1.03080015945245	\\
-2.96304872327708	0.0865516378048978	\\
-2.95573134908394	1.28445705456723	\\
-2.94841397489079	0.0990680803129985	\\
-2.94695050005216	0.880500800763251	\\
-2.9386574759666	0.25678238422905	\\
-2.92402272758031	4.83511659697128	\\
-2.92402272758031	4.83511659697128	\\
-2.91085145403265	0.0575295178261046	\\
-2.90597320457056	2.24909378678628	\\
-2.89621670564637	0.0448076361576913	\\
-2.89133845618427	0.029099393079996	\\
-2.88889933145322	1.33255338988377	\\
-2.88060630736766	0.971297502548976	\\
-2.87719153274419	0.0838205237063527	\\
-2.8625567843579	0.0287283107431738	\\
-2.8571907099496	1.07867787015116	\\
-2.84987333575646	1.11600855177935	\\
-2.84840986091783	0.103482087702305	\\
-2.8367020622088	0.134200219609194	\\
-2.82645773833839	1.22154385598877	\\
-2.8215794888763	0.0971634314820391	\\
-2.81865253919904	1.74908001807182	\\
-2.8064569155438	0.0695426852315497	\\
-2.80206649102791	1.23932184248584	\\
-2.78499261791058	2.67459529085956	\\
-2.77962654350227	0.323227361837724	\\
-2.77523611898639	1.57633042588796	\\
-2.77035786952429	0.105123337505237	\\
-2.76450397016978	1.21385568134321	\\
-2.75865007081526	0.133677699673347	\\
-2.74060054780551	1.56852468078489	\\
-2.7401127228593	0.0779682258303014	\\
-2.72547797447301	0.0754975291914312	\\
-2.72303884974196	1.59289796322571	\\
-2.71718495038745	1.41022623351827	\\
-2.71230670092535	0.137118971325842	\\
-2.70157455210874	0.137209771740339	\\
-2.69279370307697	1.06132456446049	\\
-2.68986675339971	0.0250329568345681	\\
-2.68059807942173	1.70996972406976	\\
-2.67132940544374	0.15786506410802	\\
-2.66986593060511	1.99384391147687	\\
-2.65669465705746	1.87244359828557	\\
-2.65327988243399	0.153200650886263	\\
-2.64205990867117	0.443849925089533	\\
-2.63474253447802	3.74447522657635	\\
-2.62888863512351	1.17784950418544	\\
-2.62010778609174	22.1967016954006	\\
-2.6132782368448	5.23146242443547	\\
-2.60644868759787	123.446009212199	\\
-2.60157043813577	440.503534285936	\\
-2.59327741405021	1.80385150521185	\\
-2.59035046437295	4.87522254845647	\\
-2.58010614050255	0.488155222696539	\\
-2.56303226738521	3.23051082334747	\\
-2.562544442439	0.339595233660149	\\
-2.56156879254658	0.128555770521491	\\
-2.55571489319207	2.82565491148402	\\
-2.54351926953683	0.0890045141491992	\\
-2.5371775452361	3.05874137908025	\\
-2.52937234609675	0.388488129848639	\\
-2.51912802222635	4.2712424590207	\\
-2.51815237233393	0.861996102657129	\\
-2.50498109878627	8.11923868249531	\\
-2.49229765018482	0.905104848994251	\\
-2.49083417534619	17.9310175871543	\\
-2.48741940072273	6.31688944271806	\\
-2.47668725190611	76.4789154014914	\\
-2.47424812717507	17.078264876146	\\
-2.46205250351983	978.99854886173	\\
-2.46107685362741	462.191318657944	\\
-2.44644210524112	3.04289069200392	\\
-2.44449080545628	9.07909300259977	\\
-2.43814908115556	1.4390974546007	\\
-2.42448998266169	8.57737813394563	\\
-2.42156303298443	0.385040857063871	\\
-2.41131870911403	0.563973032268382	\\
-2.40839175943677	7.45344927585503	\\
-2.39668396072774	6.56058408303783	\\
-2.39522048588911	1.08764888041717	\\
-2.37668313793315	5.20703249981326	\\
-2.37619531298694	0.594273896201355	\\
-2.36887793879379	0.339938777149719	\\
-2.36497533922412	4.59970248526365	\\
-2.35424319040751	5.98271882898565	\\
-2.34692581621436	0.169797453594301	\\
-2.33473019255912	6.74872577241197	\\
-2.33180324288187	0.607058170578514	\\
-2.31765631944179	0.798657384218093	\\
-2.31716849449558	9.63484616887068	\\
-2.30936329535622	0.346272575028558	\\
-2.30643634567897	12.2927281657267	\\
-2.30058244632445	1.21576214499649	\\
-2.28936247256163	33.7100421049127	\\
-2.28350857320712	4.45002229264145	\\
-2.27570337406776	50.0636872998489	\\
-2.27131294955188	4.39323313007752	\\
-2.25814167600422	156.825253168614	\\
-2.25570255127317	59.0993292951885	\\
-2.24497040245656	927.970374050955	\\
-2.24399475256414	849.296872012663	\\
-2.22936000417785	10.9120243966991	\\
-2.22887217923164	24.1148474397213	\\
-2.2166765555764	2.1627080111616	\\
-2.21277395600672	14.3862274152135	\\
-2.20594440675979	1.28976907024178	\\
-2.19570008288939	0.680400012810695	\\
-2.1913096583735	8.42574283940268	\\
-2.18057750955689	1.02084396642993	\\
-2.17521143514859	11.0662610246734	\\
-2.16643058611681	0.753946801498185	\\
-2.16301581149335	12.5069383385625	\\
-2.1556984373002	6.57900839391508	\\
-2.14496628848359	0.422848744439992	\\
-2.13764891429045	7.30637267090998	\\
-2.13130718998972	0.308302652275291	\\
-2.11813591644206	0.749172040786315	\\
-2.11716026654964	5.52547643252047	\\
-2.11130636719513	6.10939136429996	\\
-2.10545246784061	0.0634192897931718	\\
-2.09911074353989	4.48316408725364	\\
-2.08691511988465	0.629412935496279	\\
-2.08008557063771	0.167611690076289	\\
-2.0795977456915	4.34753356811277	\\
-2.06788994698247	0.369418765299264	\\
-2.06057257278933	10.604352423809	\\
-2.05520649838102	5.45399894870503	\\
-2.05081607386514	0.326692833951403	\\
-2.03618132547885	3.19299903353405	\\
-2.03569350053264	0.18370593752733	\\
-2.02301005193119	2.38651191466088	\\
-2.02203440203877	0.234833382487118	\\
-2.00935095343732	0.244970570946686	\\
-1.99959445451313	3.22141956172423	\\
-1.99520402999724	2.68455974936869	\\
-1.99178925537378	0.617933316220183	\\
-1.98056928161096	0.27136459592638	\\
-1.96934930784814	2.34493722953203	\\
-1.96837365795572	2.24901664868817	\\
-1.96349540849362	0.0874524524197403	\\
-1.94934848505354	2.77511691854123	\\
-1.94349458569903	0.0784436010837776	\\
-1.93081113709758	0.244739809490196	\\
-1.92983548720516	1.74913172845618	\\
-1.9215424631196	0.216824740013943	\\
-1.91520073881887	2.95861941519642	\\
-1.90544423989468	2.10027251499752	\\
-1.90300511516363	0.0740876248727217	\\
-1.89227296634702	2.3547966467951	\\
-1.8912973164546	0.276853621172258	\\
-1.86983301882138	0.409254702880317	\\
-1.86885736892896	2.63348611782229	\\
-1.85519827043509	0.483997960846954	\\
-1.85324697065025	1.96763086422248	\\
-1.84690524634953	0.220812024833899	\\
-1.84202699688743	2.61149192409104	\\
-1.82592877366251	1.71172829718787	\\
-1.82397747387768	0.223945734582872	\\
-1.82056269925421	0.14980448897734	\\
-1.81617227473832	1.65141572987795	\\
-1.80641577581413	2.33885771285002	\\
-1.80104970140583	0.227189664369874	\\
-1.78885407775059	0.771731482470029	\\
-1.78202452850365	16.2453450977786	\\
-1.77617062914914	5.5103571302207	\\
-1.76787760506357	0.269470323568958	\\
-1.75373068162349	0.131087916530538	\\
-1.75226720678487	1.69199860410946	\\
-1.7439741826993	0.140346723536011	\\
-1.73714463345237	2.06449953793481	\\
-1.73568115861374	2.3641631991097	\\
-1.73226638399027	0.138166123177149	\\
-1.71421686098052	0.198163721646623	\\
-1.71031426141084	2.87591822824474	\\
-1.70397253711012	0.201996879723245	\\
-1.70006993754044	1.89274326592903	\\
-1.69324038829351	0.144440470625353	\\
-1.68397171431552	1.99858334144672	\\
-1.67567869022996	3.05710039603164	\\
-1.66641001625198	0.523297256620474	\\
-1.65665351732779	0.283709513114082	\\
-1.65567786743537	4.77590953663351	\\
-1.6473848433498	4.33628295948683	\\
-1.64397006872634	0.6076760416451	\\
-1.63470139474835	0.418328369053683	\\
-1.62299359603932	5.24906792702887	\\
-1.61909099646965	0.0945521033927024	\\
-1.60884667259925	6.44944113562429	\\
-1.60055364851368	0.487375237646894	\\
-1.59274844937433	6.07674320164513	\\
-1.59079714958949	9.59895148100472	\\
-1.57811370098804	1.27147971992131	\\
-1.57225980163353	13.3933086816491	\\
-1.57177197668732	0.892833154932512	\\
-1.56152765281691	0.771886860611123	\\
-1.55030767905409	14.4259527729555	\\
-1.54786855432305	0.579621352303511	\\
-1.53859988034506	16.8759561341445	\\
-1.53079468120571	4.83483021385348	\\
-1.52055035733531	86.2481185280568	\\
-1.51957470744289	88.3581064206229	\\
-1.50737908378765	0.867956589963162	\\
-1.4995738846483	17.8473043122041	\\
-1.49420781023999	1.82169679649441	\\
-1.48689043604685	9.25383707537662	\\
-1.48640261110064	0.964927386919294	\\
-1.47518263733782	7.7770934129678	\\
-1.46298701368258	0.552598974753698	\\
-1.46152353884395	0.42306875755403	\\
-1.45469398959701	4.02510423491446	\\
-1.44640096551145	0.344786635852373	\\
-1.44591314056524	3.97858124946774	\\
-1.43274186701758	0.522937532908343	\\
-1.43030274228653	3.8780843431695	\\
-1.41518016895404	0.33787401444443	\\
-1.40883844465331	4.72190466110577	\\
-1.40152107046017	0.0245081811771858	\\
-1.40005759562154	5.23065623099335	\\
-1.38688632207388	0.661197725493428	\\
-1.38054459777315	5.39185089979236	\\
-1.37127592379517	0.0748902222028399	\\
-1.36103159992477	6.54048852010046	\\
-1.35859247519372	0.127168108665201	\\
-1.3473725014309	16.7042159613774	\\
-1.34639685153848	12.9070692247449	\\
-1.33907947734534	0.992177671240348	\\
-1.32883515347494	0.351628593498956	\\
-1.31761517971212	5.24091326712106	\\
-1.30541955605688	0.43090057718749	\\
-1.30298043132583	3.88704182232504	\\
-1.30298043132583	3.88704182232504	\\
-1.28980915777817	0.0836633884903562	\\
-1.28785785799333	2.45506202248296	\\
-1.27566223433809	0.137978339952923	\\
-1.27127180982221	0.314206256696482	\\
-1.2610274859518	2.75132653348114	\\
-1.25712488638213	0.190945171413114	\\
-1.24785621240414	2.45387365018088	\\
-1.240538838211	0.230583682907331	\\
-1.2302945143406	3.39445246968729	\\
-1.22444061498608	0.395880607681412	\\
-1.22297714014746	4.01684927730566	\\
-1.21224499133084	0.303818275132535	\\
-1.20395196724528	3.53258812847748	\\
-1.1946832932673	3.95516333428437	\\
-1.18785374402036	0.439266415211917	\\
-1.18395114445069	4.60793719258353	\\
-1.17907289498859	0.672241734002513	\\
-1.16736509627956	0.448325969541571	\\
-1.16151119692505	6.03531295200997	\\
-1.14980339821602	6.46433755022158	\\
-1.14443732380771	0.199507916325144	\\
-1.13516864982973	8.18956572105221	\\
-1.13468082488352	0.544564959752818	\\
-1.128826925529	0.533595714979147	\\
-1.12394867606691	17.1058409730284	\\
-1.11419217714272	2.78575879040216	\\
-1.10248437843369	92.9156465344494	\\
-1.09272787950949	30.7062000504299	\\
-1.08687398015498	1.17988191434446	\\
-1.08345920553151	9.68391176549429	\\
-1.08102008080046	0.692281850058139	\\
-1.0678488072528	0.533849908744165	\\
-1.06736098230659	5.25065479491978	\\
-1.05467753370514	0.273772466208564	\\
-1.05272623392031	5.03642508580333	\\
-1.0371158356416	0.293807805865028	\\
-1.02833498660983	5.31179378643587	\\
-1.02394456209394	4.06567797311634	\\
-1.01613936295459	0.341844154736799	\\
-1.01028546360007	7.11710923832827	\\
-1.00004113972967	0.319008960663232	\\
-0.993699415428945	0.48477365433159	\\
-0.98784551607443	4.49430651745082	\\
-0.979064667042658	5.14256749749396	\\
-0.970283818010885	0.297076664083135	\\
-0.962966443817741	2.8785699943743	\\
-0.960039494140483	0.172308689820646	\\
-0.95174647005492	0.195994648850879	\\
-0.941014321238309	1.99787427891212	\\
-0.939063021453471	0.115472043728057	\\
-0.933696947045165	2.44143751807177	\\
-0.919550023605087	0.411694623970508	\\
-0.912720474358153	2.5568791230635	\\
-0.901988325541542	0.291101139096232	\\
-0.899549200810494	3.21866977436405	\\
-0.894670951348397	3.25978355501022	\\
-0.884914452424206	0.253176368347087	\\
-0.876133603392433	2.55015616531172	\\
-0.873206653715175	0.337283903729173	\\
-0.859059730275097	0.621684485142821	\\
-0.85613278059784	3.94394031760876	\\
-0.851254531135744	0.108476602828147	\\
-0.850278881243324	2.50142856040043	\\
-0.830278058448731	0.212183369130856	\\
-0.828814583610102	3.00837216887952	\\
-0.824424159094216	2.7600561219262	\\
-0.815155485116234	0.0924705185121859	\\
-0.808325935869299	0.0855262002725026	\\
-0.805886811138251	3.08546304002629	\\
-0.792715537590592	0.187310306557656	\\
-0.785398163397448	3.57233224715492	\\
-0.774666014580837	0.266071531322171	\\
-0.768812115226322	2.72984870821637	\\
-0.764421690710436	2.49716512808781	\\
-0.753201716947615	0.249248836389595	\\
-0.74685999264689	2.49036839282921	\\
-0.739054793507537	0.276919768160383	\\
-0.729786119529555	0.396836460613723	\\
-0.728810469637135	2.37588460793187	\\
-0.715639196089476	0.0350270304193984	\\
-0.714175721250848	3.49722891759112	\\
-0.701980097595607	0.25104728597913	\\
-0.69905314791835	3.46942117831043	\\
-0.689784473940368	0.316218039702146	\\
-0.681491449854804	3.00044343014295	\\
-0.676125375446499	0.230488320126435	\\
-0.669783651145774	3.75140124591938	\\
-0.657588027490534	0.564960778422915	\\
-0.652709778028438	3.73158905347336	\\
-0.645880228781504	0.226236822069926	\\
-0.638075029642151	2.5265684948833	\\
-0.634172430072474	0.205293269853702	\\
-0.623928106202072	2.88909231341788	\\
-0.620025506632396	3.14492322487926	\\
-0.612708132439252	0.181022886261279	\\
-0.606854233084737	0.291555053448742	\\
-0.600024683837802	4.62461552757797	\\
-0.592707309644658	0.177884461526149	\\
-0.591731659752239	4.34577054767294	\\
-0.573194311796275	3.61828928808749	\\
-0.569779537172807	0.358337189317129	\\
-0.551242189216842	0.234483874305797	\\
-0.550266539324423	3.40931038206703	\\
-0.549290889432004	4.07278540690157	\\
-0.548315239539585	0.822355904794673	\\
-0.531241366422249	0.420883235302917	\\
-0.527338766852572	5.35779737559973	\\
-0.516118793089752	0.548780369997247	\\
-0.515630968143542	6.70544962263451	\\
-0.499532744918625	1.10755376056557	\\
-0.49465449545653	8.72682866077879	\\
-0.491239720833062	1.33935278777748	\\
-0.477580622339194	12.480762251744	\\
-0.476117147500565	1.21862873644423	\\
-0.47026324814605	15.9580284007464	\\
-0.461970224060487	0.400639732094814	\\
-0.448798950512828	42.8066459059931	\\
-0.448798950512828	42.8066459059931	\\
-0.445384175889361	0.903696001543168	\\
-0.430261602556863	17.9672082568392	\\
-0.425871178040977	2.47537435966089	\\
-0.414163379331947	12.7285117069041	\\
-0.405870355246383	1.67239246397032	\\
-0.402943405569126	1.87244634571055	\\
-0.400992105784287	12.1876426177469	\\
-0.389772132021466	12.0701227079829	\\
-0.381966932882113	0.385779269582201	\\
-0.375625208581388	12.5102017064255	\\
-0.368795659334454	0.272496619768955	\\
-0.358551335464052	10.8174075212183	\\
-0.347819186647441	1.0748128582677	\\
-0.335623562992202	7.74501438887083	\\
-0.334160088153573	1.47520124579601	\\
-0.326354889014219	5.13889477399265	\\
-0.323427939336962	0.469813167814085	\\
-0.309281015896883	5.93257155638221	\\
-0.304402766434787	0.450855013190454	\\
-0.30196364170374	0.678777599798615	\\
-0.290255842994709	4.39179181818872	\\
-0.288304543209871	0.17350778136057	\\
-0.276596744500841	4.7493472273544	\\
-0.273669794823583	0.382964384977403	\\
-0.26537677073802	6.11995326044243	\\
-0.259035046437295	0.0897435393927827	\\
-0.25318114708278	4.96363450640767	\\
-0.245863772889636	0.207517802984243	\\
-0.239522048588911	4.87531623204581	\\
-0.231229024503348	0.13094123869347	\\
-0.227814249879881	3.69771811309518	\\
-0.216594276117061	3.0002868557949	\\
-0.203423002569401	0.0405543963526041	\\
-0.197081278268676	2.17282214895497	\\
-0.191715203860371	0.308061729248319	\\
-0.175129155689245	2.93028366597309	\\
-0.174641330743035	0.340852901537065	\\
-0.165372656765053	0.220424243308894	\\
-0.162445707087795	1.61855979146611	\\
-0.147323133755298	1.67069030129648	\\
-0.146835308809088	0.27733204419808	\\
-0.139517934615944	0.210574697865865	\\
-0.134639685153848	2.00702071245474	\\
-0.128297960853124	1.67786978838606	\\
-0.117077987090303	0.138437383687807	\\
-0.111224087735788	1.50573244256626	\\
-0.109272787950949	0.0888143534821085	\\
-0.0970771642957096	0.170809157418999	\\
-0.0951258645108712	1.83494618624529	\\
-0.0839058907480505	0.260485911875819	\\
-0.0834180658018409	1.73664850421743	\\
-0.0653685427920858	1.31801420790511	\\
-0.0648807178458761	0.0516464603237703	\\
-0.0507337944057977	0.204683999693026	\\
-0.0439042451588638	1.50993763848087	\\
-0.0424407703202347	0.176287389427435	\\
-0.0321964464498334	1.17013324993072	\\
-0.0170738731173357	1.58732503610507	\\
-0.0165860481711264	0.0857018338918776	\\
-0.00439042451588634	0.0367835491725901	\\
0	1.88672211069687	\\
0	1.88672211069687	\\
0.00439042451588634	0.0367835491725901	\\
0.016586048171126	0.0857018338918776	\\
0.0170738731173357	1.58732503610507	\\
0.0321964464498334	1.17013324993072	\\
0.0424407703202347	0.176287389427435	\\
0.0439042451588634	1.50993763848087	\\
0.0507337944057977	0.204683999693026	\\
0.0648807178458761	0.0516464603237703	\\
0.0653685427920858	1.31801420790511	\\
0.0834180658018409	1.73664850421743	\\
0.0839058907480501	0.260485911875819	\\
0.0951258645108708	1.83494618624529	\\
0.0970771642957096	0.170809157418999	\\
0.109272787950949	0.0888143534821085	\\
0.111224087735788	1.50573244256626	\\
0.117077987090303	0.138437383687807	\\
0.128297960853124	1.67786978838606	\\
0.134639685153848	2.00702071245474	\\
0.139517934615944	0.210574697865865	\\
0.146835308809088	0.27733204419808	\\
0.147323133755298	1.67069030129648	\\
0.162445707087795	1.61855979146611	\\
0.165372656765053	0.220424243308894	\\
0.174641330743035	0.340852901537065	\\
0.175129155689245	2.93028366597309	\\
0.191715203860371	0.308061729248319	\\
0.197081278268676	2.17282214895497	\\
0.203423002569401	0.0405543963526041	\\
0.21659427611706	3.0002868557949	\\
0.227814249879881	3.69771811309518	\\
0.231229024503348	0.13094123869347	\\
0.239522048588911	4.87531623204581	\\
0.245863772889636	0.207517802984243	\\
0.25318114708278	4.96363450640767	\\
0.259035046437295	0.0897435393927827	\\
0.26537677073802	6.11995326044243	\\
0.273669794823583	0.382964384977403	\\
0.276596744500841	4.7493472273544	\\
0.288304543209871	0.17350778136057	\\
0.290255842994709	4.39179181818872	\\
0.301963641703739	0.678777599798615	\\
0.304402766434787	0.450855013190454	\\
0.309281015896883	5.93257155638221	\\
0.323427939336962	0.469813167814085	\\
0.326354889014219	5.13889477399265	\\
0.334160088153573	1.47520124579601	\\
0.335623562992201	7.74501438887083	\\
0.347819186647441	1.0748128582677	\\
0.358551335464052	10.8174075212183	\\
0.368795659334454	0.272496619768955	\\
0.375625208581388	12.5102017064255	\\
0.381966932882113	0.385779269582201	\\
0.389772132021466	12.0701227079829	\\
0.400992105784287	12.1876426177469	\\
0.402943405569125	1.87244634571055	\\
0.405870355246383	1.67239246397032	\\
0.414163379331946	12.7285117069041	\\
0.425871178040977	2.47537435966089	\\
0.430261602556863	17.9672082568392	\\
0.44538417588936	0.903696001543168	\\
0.448798950512828	42.8066459059931	\\
0.448798950512828	42.8066459059931	\\
0.461970224060487	0.400639732094814	\\
0.47026324814605	15.9580284007464	\\
0.476117147500565	1.21862873644423	\\
0.477580622339194	12.480762251744	\\
0.491239720833062	1.33935278777748	\\
0.494654495456529	8.72682866077879	\\
0.499532744918625	1.10755376056557	\\
0.515630968143542	6.70544962263451	\\
0.516118793089752	0.548780369997247	\\
0.527338766852572	5.35779737559973	\\
0.531241366422249	0.420883235302917	\\
0.548315239539585	0.822355904794673	\\
0.549290889432004	4.07278540690157	\\
0.550266539324423	3.40931038206703	\\
0.551242189216842	0.234483874305797	\\
0.569779537172807	0.358337189317129	\\
0.573194311796274	3.61828928808749	\\
0.591731659752239	4.34577054767294	\\
0.592707309644658	0.177884461526149	\\
0.600024683837802	4.62461552757797	\\
0.606854233084736	0.291555053448742	\\
0.612708132439252	0.181022886261279	\\
0.620025506632396	3.14492322487926	\\
0.623928106202072	2.88909231341788	\\
0.634172430072474	0.205293269853702	\\
0.638075029642151	2.5265684948833	\\
0.645880228781504	0.226236822069926	\\
0.652709778028438	3.73158905347336	\\
0.657588027490534	0.564960778422915	\\
0.669783651145774	3.75140124591938	\\
0.676125375446499	0.230488320126435	\\
0.681491449854804	3.00044343014295	\\
0.689784473940367	0.316218039702146	\\
0.69905314791835	3.46942117831043	\\
0.701980097595607	0.25104728597913	\\
0.714175721250848	3.49722891759112	\\
0.715639196089476	0.0350270304193984	\\
0.728810469637135	2.37588460793187	\\
0.729786119529555	0.396836460613723	\\
0.739054793507537	0.276919768160383	\\
0.74685999264689	2.49036839282921	\\
0.753201716947615	0.249248836389595	\\
0.764421690710436	2.49716512808781	\\
0.768812115226322	2.72984870821637	\\
0.774666014580837	0.266071531322171	\\
0.785398163397448	3.57233224715492	\\
0.792715537590592	0.187310306557656	\\
0.805886811138251	3.08546304002629	\\
0.808325935869299	0.0855262002725026	\\
0.815155485116234	0.0924705185121859	\\
0.824424159094216	2.7600561219262	\\
0.828814583610102	3.00837216887952	\\
0.830278058448731	0.212183369130856	\\
0.850278881243324	2.50142856040043	\\
0.851254531135744	0.108476602828147	\\
0.856132780597839	3.94394031760876	\\
0.859059730275098	0.621684485142821	\\
0.873206653715176	0.337283903729173	\\
0.876133603392432	2.55015616531172	\\
0.884914452424206	0.253176368347087	\\
0.894670951348397	3.25978355501022	\\
0.899549200810494	3.21866977436405	\\
0.901988325541542	0.291101139096232	\\
0.912720474358153	2.5568791230635	\\
0.919550023605087	0.411694623970508	\\
0.933696947045165	2.44143751807177	\\
0.939063021453471	0.115472043728057	\\
0.94101432123831	1.99787427891212	\\
0.95174647005492	0.195994648850879	\\
0.960039494140483	0.172308689820646	\\
0.962966443817741	2.8785699943743	\\
0.970283818010884	0.297076664083135	\\
0.979064667042658	5.14256749749396	\\
0.98784551607443	4.49430651745082	\\
0.993699415428946	0.48477365433159	\\
1.00004113972967	0.319008960663232	\\
1.01028546360007	7.11710923832827	\\
1.01613936295459	0.341844154736799	\\
1.02394456209394	4.06567797311634	\\
1.02833498660983	5.31179378643587	\\
1.0371158356416	0.293807805865028	\\
1.05272623392031	5.03642508580333	\\
1.05467753370514	0.273772466208564	\\
1.06736098230659	5.25065479491978	\\
1.0678488072528	0.533849908744165	\\
1.08102008080046	0.692281850058139	\\
1.08345920553151	9.68391176549429	\\
1.08687398015498	1.17988191434446	\\
1.09272787950949	30.7062000504299	\\
1.10248437843368	92.9156465344494	\\
1.11419217714272	2.78575879040216	\\
1.12394867606691	17.1058409730284	\\
1.128826925529	0.533595714979147	\\
1.13468082488352	0.544564959752818	\\
1.13516864982973	8.18956572105221	\\
1.14443732380771	0.199507916325144	\\
1.14980339821602	6.46433755022158	\\
1.16151119692505	6.03531295200997	\\
1.16736509627956	0.448325969541571	\\
1.17907289498859	0.672241734002513	\\
1.18395114445069	4.60793719258353	\\
1.18785374402036	0.439266415211917	\\
1.1946832932673	3.95516333428437	\\
1.20395196724528	3.53258812847748	\\
1.21224499133084	0.303818275132535	\\
1.22297714014745	4.01684927730566	\\
1.22444061498608	0.395880607681412	\\
1.2302945143406	3.39445246968729	\\
1.240538838211	0.230583682907331	\\
1.24785621240414	2.45387365018088	\\
1.25712488638213	0.190945171413114	\\
1.2610274859518	2.75132653348114	\\
1.27127180982221	0.314206256696482	\\
1.27566223433809	0.137978339952923	\\
1.28785785799333	2.45506202248296	\\
1.28980915777817	0.0836633884903562	\\
1.30298043132583	3.88704182232504	\\
1.30298043132583	3.88704182232504	\\
1.30541955605688	0.43090057718749	\\
1.31761517971212	5.24091326712106	\\
1.32883515347494	0.351628593498956	\\
1.33907947734534	0.992177671240348	\\
1.34639685153848	12.9070692247449	\\
1.3473725014309	16.7042159613774	\\
1.35859247519372	0.127168108665201	\\
1.36103159992477	6.54048852010046	\\
1.37127592379517	0.0748902222028399	\\
1.38054459777315	5.39185089979236	\\
1.38688632207388	0.661197725493428	\\
1.40005759562154	5.23065623099335	\\
1.40152107046017	0.0245081811771858	\\
1.40883844465331	4.72190466110577	\\
1.41518016895404	0.33787401444443	\\
1.43030274228653	3.8780843431695	\\
1.43274186701758	0.522937532908343	\\
1.44591314056524	3.97858124946774	\\
1.44640096551145	0.344786635852373	\\
1.45469398959701	4.02510423491446	\\
1.46152353884395	0.42306875755403	\\
1.46298701368258	0.552598974753698	\\
1.47518263733782	7.7770934129678	\\
1.48640261110064	0.964927386919294	\\
1.48689043604685	9.25383707537662	\\
1.49420781023999	1.82169679649441	\\
1.4995738846483	17.8473043122041	\\
1.50737908378765	0.867956589963162	\\
1.51957470744289	88.3581064206229	\\
1.52055035733531	86.2481185280568	\\
1.53079468120571	4.83483021385348	\\
1.53859988034506	16.8759561341445	\\
1.54786855432305	0.579621352303511	\\
1.55030767905409	14.4259527729555	\\
1.56152765281691	0.771886860611123	\\
1.57177197668732	0.892833154932512	\\
1.57225980163352	13.3933086816491	\\
1.57811370098804	1.27147971992131	\\
1.59079714958949	9.59895148100472	\\
1.59274844937433	6.07674320164513	\\
1.60055364851368	0.487375237646894	\\
1.60884667259925	6.44944113562429	\\
1.61909099646965	0.0945521033927024	\\
1.62299359603932	5.24906792702887	\\
1.63470139474835	0.418328369053683	\\
1.64397006872634	0.6076760416451	\\
1.6473848433498	4.33628295948683	\\
1.65567786743537	4.77590953663351	\\
1.65665351732779	0.283709513114082	\\
1.66641001625198	0.523297256620474	\\
1.67567869022996	3.05710039603164	\\
1.68397171431552	1.99858334144672	\\
1.69324038829351	0.144440470625353	\\
1.70006993754044	1.89274326592903	\\
1.70397253711012	0.201996879723245	\\
1.71031426141084	2.87591822824474	\\
1.71421686098052	0.198163721646623	\\
1.73226638399027	0.138166123177149	\\
1.73568115861374	2.3641631991097	\\
1.73714463345237	2.06449953793481	\\
1.7439741826993	0.140346723536011	\\
1.75226720678487	1.69199860410946	\\
1.75373068162349	0.131087916530538	\\
1.76787760506357	0.269470323568958	\\
1.77617062914914	5.5103571302207	\\
1.78202452850365	16.2453450977786	\\
1.78885407775059	0.771731482470029	\\
1.80104970140582	0.227189664369874	\\
1.80641577581413	2.33885771285002	\\
1.81617227473832	1.65141572987795	\\
1.82056269925421	0.14980448897734	\\
1.82397747387768	0.223945734582872	\\
1.82592877366251	1.71172829718787	\\
1.84202699688743	2.61149192409104	\\
1.84690524634953	0.220812024833899	\\
1.85324697065025	1.96763086422248	\\
1.85519827043509	0.483997960846954	\\
1.86885736892896	2.63348611782229	\\
1.86983301882138	0.409254702880317	\\
1.8912973164546	0.276853621172258	\\
1.89227296634702	2.3547966467951	\\
1.90300511516363	0.0740876248727217	\\
1.90544423989468	2.10027251499752	\\
1.91520073881887	2.95861941519642	\\
1.9215424631196	0.216824740013943	\\
1.92983548720516	1.74913172845618	\\
1.93081113709758	0.244739809490196	\\
1.94349458569903	0.0784436010837776	\\
1.94934848505354	2.77511691854123	\\
1.96349540849362	0.0874524524197403	\\
1.96837365795572	2.24901664868817	\\
1.96934930784814	2.34493722953203	\\
1.98056928161096	0.27136459592638	\\
1.99178925537378	0.617933316220183	\\
1.99520402999724	2.68455974936869	\\
1.99959445451313	3.22141956172423	\\
2.00935095343732	0.244970570946686	\\
2.02203440203877	0.234833382487118	\\
2.02301005193119	2.38651191466088	\\
2.03569350053264	0.18370593752733	\\
2.03618132547885	3.19299903353405	\\
2.05081607386514	0.326692833951403	\\
2.05520649838102	5.45399894870503	\\
2.06057257278933	10.604352423809	\\
2.06788994698247	0.369418765299264	\\
2.0795977456915	4.34753356811277	\\
2.08008557063771	0.167611690076289	\\
2.08691511988465	0.629412935496279	\\
2.09911074353989	4.48316408725364	\\
2.10545246784061	0.0634192897931718	\\
2.11130636719513	6.10939136429996	\\
2.11716026654964	5.52547643252047	\\
2.11813591644206	0.749172040786315	\\
2.13130718998972	0.308302652275291	\\
2.13764891429045	7.30637267090998	\\
2.14496628848359	0.422848744439992	\\
2.1556984373002	6.57900839391508	\\
2.16301581149335	12.5069383385625	\\
2.16643058611681	0.753946801498185	\\
2.17521143514858	11.0662610246734	\\
2.18057750955689	1.02084396642993	\\
2.1913096583735	8.42574283940268	\\
2.19570008288939	0.680400012810695	\\
2.20594440675979	1.28976907024178	\\
2.21277395600672	14.3862274152135	\\
2.2166765555764	2.1627080111616	\\
2.22887217923164	24.1148474397213	\\
2.22936000417785	10.9120243966991	\\
2.24399475256414	849.296872012663	\\
2.24497040245656	927.970374050955	\\
2.25570255127317	59.0993292951885	\\
2.25814167600422	156.825253168614	\\
2.27131294955188	4.39323313007752	\\
2.27570337406776	50.0636872998489	\\
2.28350857320712	4.45002229264145	\\
2.28936247256163	33.7100421049127	\\
2.30058244632445	1.21576214499649	\\
2.30643634567897	12.2927281657267	\\
2.30936329535622	0.346272575028558	\\
2.31716849449558	9.63484616887068	\\
2.31765631944179	0.798657384218093	\\
2.33180324288186	0.607058170578514	\\
2.33473019255912	6.74872577241197	\\
2.34692581621436	0.169797453594301	\\
2.35424319040751	5.98271882898565	\\
2.36497533922412	4.59970248526365	\\
2.36887793879379	0.339938777149719	\\
2.37619531298694	0.594273896201355	\\
2.37668313793315	5.20703249981326	\\
2.39522048588911	1.08764888041717	\\
2.39668396072774	6.56058408303783	\\
2.40839175943677	7.45344927585503	\\
2.41131870911403	0.563973032268382	\\
2.42156303298443	0.385040857063871	\\
2.42448998266169	8.57737813394563	\\
2.43814908115556	1.4390974546007	\\
2.44449080545628	9.07909300259977	\\
2.44644210524112	3.04289069200392	\\
2.46107685362741	462.191318657944	\\
2.46205250351983	978.99854886173	\\
2.47424812717507	17.078264876146	\\
2.47668725190611	76.4789154014914	\\
2.48741940072273	6.31688944271806	\\
2.49083417534619	17.9310175871543	\\
2.49229765018482	0.905104848994251	\\
2.50498109878627	8.11923868249531	\\
2.51815237233393	0.861996102657129	\\
2.51912802222635	4.2712424590207	\\
2.52937234609675	0.388488129848639	\\
2.5371775452361	3.05874137908025	\\
2.54351926953683	0.0890045141491992	\\
2.55571489319207	2.82565491148402	\\
2.56156879254658	0.128555770521491	\\
2.562544442439	0.339595233660149	\\
2.56303226738521	3.23051082334747	\\
2.58010614050255	0.488155222696539	\\
2.59035046437295	4.87522254845647	\\
2.59327741405021	1.80385150521185	\\
2.60157043813577	440.503534285936	\\
2.60644868759787	123.446009212199	\\
2.6132782368448	5.23146242443547	\\
2.62010778609174	22.1967016954006	\\
2.62888863512351	1.17784950418544	\\
2.63474253447802	3.74447522657635	\\
2.64205990867117	0.443849925089533	\\
2.65327988243399	0.153200650886263	\\
2.65669465705746	1.87244359828557	\\
2.66986593060511	1.99384391147687	\\
2.67132940544374	0.15786506410802	\\
2.68059807942173	1.70996972406976	\\
2.68986675339971	0.0250329568345681	\\
2.69279370307697	1.06132456446049	\\
2.70157455210874	0.137209771740339	\\
2.71230670092535	0.137118971325842	\\
2.71718495038744	1.41022623351827	\\
2.72303884974196	1.59289796322571	\\
2.72547797447301	0.0754975291914312	\\
2.7401127228593	0.0779682258303014	\\
2.74060054780551	1.56852468078489	\\
2.75865007081526	0.133677699673347	\\
2.76450397016978	1.21385568134321	\\
2.77035786952429	0.105123337505237	\\
2.77523611898639	1.57633042588796	\\
2.77962654350227	0.323227361837724	\\
2.78499261791058	2.67459529085956	\\
2.80206649102792	1.23932184248584	\\
2.8064569155438	0.0695426852315497	\\
2.81865253919904	1.74908001807182	\\
2.8215794888763	0.0971634314820391	\\
2.82645773833839	1.22154385598877	\\
2.8367020622088	0.134200219609194	\\
2.84840986091783	0.103482087702305	\\
2.84987333575646	1.11600855177935	\\
2.8571907099496	1.07867787015116	\\
2.8625567843579	0.0287283107431738	\\
2.87719153274419	0.0838205237063527	\\
2.88060630736766	0.971297502548976	\\
2.88889933145322	1.33255338988377	\\
2.89133845618427	0.029099393079996	\\
2.89621670564637	0.0448076361576913	\\
2.90597320457056	2.24909378678628	\\
2.91085145403265	0.0575295178261046	\\
2.92402272758031	4.83511659697128	\\
2.92402272758031	4.83511659697128	\\
2.9386574759666	0.25678238422905	\\
2.94695050005216	0.880500800763251	\\
2.94841397489079	0.0990680803129985	\\
2.95573134908394	1.28445705456723	\\
2.96304872327708	0.0865516378048978	\\
2.97524434693232	1.03080015945245	\\
2.97817129660958	0.0616518189864895	\\
2.98792779553377	0.0706535820430292	\\
2.99280604499587	0.860926569085718	\\
3.0045138437049	0.131015777209727	\\
3.00548949359732	1.1997206794082	\\
3.01573381746772	0.988715613233601	\\
3.01963641703739	0.0809176065351206	\\
3.03085639080022	0.0999513669044146	\\
3.03817376499336	0.97082660500485	\\
3.04597896413271	0.166401142307705	\\
3.05378416327207	0.833454805316724	\\
3.0557354630569	0.0803265305036873	\\
3.06256501230384	1.12865832757824	\\
3.07915106047497	0.180865362422311	\\
3.07963888542117	1.00079331082098	\\
3.09232233402262	0.886666688407004	\\
3.09671275853851	0.00380108186853128	\\
3.1011031830544	0.0793218010754745	\\
3.11085968197859	1.13197498697061	\\
3.11281098176343	0.0710510701214799	\\
3.11915270606415	0.941932318046942	\\
3.14110482864358	0.430113284361737	\\
};
\end{axis}
\end{tikzpicture}%
	\caption{The train signal in the frequency domain.}
	\label{fig:freqdomain}
\end{figure}

\section{Time-frequency plot}
The used train signal is not stationary, because the frequency changes over time. 
So it does not make sense to record second or hours of an audio signal and then take the DFT of the complete sequence. 
Instead we splitted the signal into short segments of approximately 20 $ms$ and then we did the DFT on each segment. 
This resulted in the time-frequncy plot show in Figure \ref{fig:timefreq}.

\begin{figure}[h]
\centering
\includegraphics[scale = 0.5]{resources/labday1/timefrequentieplot.png}
\caption{Time-frequency plot.}
\label{fig:timefreq}
\end{figure}



\section{Zero padding}
Sometimes, the number of samples in a time-domain signal \textbf{x} is not very large. 
In frequency domain using the DFT, obtain the same number of samples as in time domain, and in these cases the resolution is not high.
'Zero padding' can be used to obtain a higher resolution.
Essentially, \textbf{x} (or \textbf{h}) is augmented with a lot of zeros.
After that only the DFT has to be applied to the augmented sequence. 

Firstly we use the filter in Equation \ref{eq:filter}. 
We don't apply the zero padding here (yet). 
A plot of this amplitude spectrum can be seen in Figure \ref{fig:zeropadding1}.

\begin{equation}
h = [1\quad zeros(1,5)\quad 0.9\quad zeros(1,5)\quad 0.8];
\label{eq:filter}
\end{equation}

\begin{figure}[H]
	\centering
	\setlength\figureheight{6cm}
  	\setlength\figurewidth{10cm}
	% This file was created by matlab2tikz v0.4.2.
% Copyright (c) 2008--2013, Nico Schlömer <nico.schloemer@gmail.com>
% All rights reserved.
% 
% The latest updates can be retrieved from
%   http://www.mathworks.com/matlabcentral/fileexchange/22022-matlab2tikz
% where you can also make suggestions and rate matlab2tikz.
% 
% 
% 
\begin{tikzpicture}

\begin{axis}[%
width=\figurewidth,
height=\figureheight,
scale only axis,
xmin=0,
xmax=1,
xlabel={frequency},
ymin=0,
ymax=3,
ylabel={amplitude}
]
\addplot [
color=blue,
only marks,
mark=o,
mark options={solid},
forget plot
]
table[row sep=crcr]{
0 2.7\\
0.0769230769230769 0.849045436113243\\
0.153846153846154 2.49555226711168\\
0.230769230769231 0.466566034269694\\
0.307692307692308 1.92954967807022\\
0.384615384615385 0.321656600909142\\
0.461538461538462 1.13447429946454\\
0.538461538461539 1.13447429946454\\
0.615384615384615 0.321656600909142\\
0.692307692307692 1.92954967807022\\
0.769230769230769 0.466566034269694\\
0.846153846153846 2.49555226711168\\
0.923076923076923 0.849045436113243\\
};
\end{axis}
\end{tikzpicture}%
	\caption{The amplitude spectrum with only the available samples.}
	\label{fig:zeropadding1}
\end{figure}

After plotting Figure \ref{fig:zeropadding1} we apply the 'zero padding' on the filter. 
In order to do this \textbf{h} is extended with $3\times 13 = 39$ zeros to 4 times its original length.
For the new found values we used the plot maker '+'.
In Figure \ref{fig:zeropadding2} the results of both plots are shown. 
From the figure we can conclude that we have obtained interpolation, because every 4th sample coincides with a sample from the previous plot (also see Figure \ref{fig:zeropadding1}).

\begin{figure}[H]
	\centering
	\setlength\figureheight{6cm}
  	\setlength\figurewidth{10cm}
	% This file was created by matlab2tikz v0.4.2.
% Copyright (c) 2008--2013, Nico Schlömer <nico.schloemer@gmail.com>
% All rights reserved.
% 
% The latest updates can be retrieved from
%   http://www.mathworks.com/matlabcentral/fileexchange/22022-matlab2tikz
% where you can also make suggestions and rate matlab2tikz.
% 
% 
% 
\begin{tikzpicture}

\begin{axis}[%
width=\figurewidth,
height=\figureheight,
scale only axis,
xmin=0,
xmax=1,
xlabel={frequency},
ymin=0,
ymax=3,
ylabel={amplitude}
]
\addplot [
color=blue,
only marks,
mark=o,
mark options={solid},
forget plot
]
table[row sep=crcr]{
0 2.7\\
0.0769230769230769 0.849045436113243\\
0.153846153846154 2.49555226711168\\
0.230769230769231 0.466566034269694\\
0.307692307692308 1.92954967807022\\
0.384615384615385 0.321656600909142\\
0.461538461538462 1.13447429946454\\
0.538461538461539 1.13447429946454\\
0.615384615384615 0.321656600909142\\
0.692307692307692 1.92954967807022\\
0.769230769230769 0.466566034269694\\
0.846153846153846 2.49555226711168\\
0.923076923076923 0.849045436113243\\
};
\addplot [
color=red,
solid,
forget plot
]
table[row sep=crcr]{
0 2.7\\
0.0192307692307692 2.25122933360484\\
0.0384615384615385 1.13447429946454\\
0.0576923076923077 0.205188696869487\\
0.0769230769230769 0.849045436113243\\
0.0961538461538462 0.700018622361898\\
0.115384615384615 0.321656600909142\\
0.134615384615385 1.54961370571638\\
0.153846153846154 2.49555226711168\\
0.173076923076923 2.64812785369316\\
0.192307692307692 1.92954967807022\\
0.211538461538462 0.711304609918332\\
0.230769230769231 0.466566034269694\\
0.25 0.9\\
0.269230769230769 0.466566034269694\\
0.288461538461538 0.711304609918332\\
0.307692307692308 1.92954967807022\\
0.326923076923077 2.64812785369316\\
0.346153846153846 2.49555226711168\\
0.365384615384615 1.54961370571638\\
0.384615384615385 0.321656600909142\\
0.403846153846154 0.700018622361898\\
0.423076923076923 0.849045436113243\\
0.442307692307692 0.205188696869487\\
0.461538461538462 1.13447429946454\\
0.480769230769231 2.25122933360484\\
0.5 2.7\\
0.519230769230769 2.25122933360484\\
0.538461538461538 1.13447429946454\\
0.557692307692308 0.205188696869487\\
0.576923076923077 0.849045436113243\\
0.596153846153846 0.700018622361898\\
0.615384615384615 0.321656600909142\\
0.634615384615385 1.54961370571638\\
0.653846153846154 2.49555226711168\\
0.673076923076923 2.64812785369316\\
0.692307692307692 1.92954967807022\\
0.711538461538461 0.711304609918332\\
0.730769230769231 0.466566034269694\\
0.75 0.9\\
0.769230769230769 0.466566034269694\\
0.788461538461538 0.711304609918332\\
0.807692307692308 1.92954967807022\\
0.826923076923077 2.64812785369316\\
0.846153846153846 2.49555226711168\\
0.865384615384615 1.54961370571638\\
0.884615384615385 0.321656600909142\\
0.903846153846154 0.700018622361898\\
0.923076923076923 0.849045436113243\\
0.942307692307692 0.205188696869487\\
0.961538461538461 1.13447429946454\\
0.980769230769231 2.25122933360484\\
};
\addplot [
color=blue,
only marks,
mark=+,
mark options={solid},
forget plot
]
table[row sep=crcr]{
0 2.7\\
0.0192307692307692 2.25122933360484\\
0.0384615384615385 1.13447429946454\\
0.0576923076923077 0.205188696869487\\
0.0769230769230769 0.849045436113243\\
0.0961538461538462 0.700018622361898\\
0.115384615384615 0.321656600909142\\
0.134615384615385 1.54961370571638\\
0.153846153846154 2.49555226711168\\
0.173076923076923 2.64812785369316\\
0.192307692307692 1.92954967807022\\
0.211538461538462 0.711304609918332\\
0.230769230769231 0.466566034269694\\
0.25 0.9\\
0.269230769230769 0.466566034269694\\
0.288461538461538 0.711304609918332\\
0.307692307692308 1.92954967807022\\
0.326923076923077 2.64812785369316\\
0.346153846153846 2.49555226711168\\
0.365384615384615 1.54961370571638\\
0.384615384615385 0.321656600909142\\
0.403846153846154 0.700018622361898\\
0.423076923076923 0.849045436113243\\
0.442307692307692 0.205188696869487\\
0.461538461538462 1.13447429946454\\
0.480769230769231 2.25122933360484\\
0.5 2.7\\
0.519230769230769 2.25122933360484\\
0.538461538461538 1.13447429946454\\
0.557692307692308 0.205188696869487\\
0.576923076923077 0.849045436113243\\
0.596153846153846 0.700018622361898\\
0.615384615384615 0.321656600909142\\
0.634615384615385 1.54961370571638\\
0.653846153846154 2.49555226711168\\
0.673076923076923 2.64812785369316\\
0.692307692307692 1.92954967807022\\
0.711538461538461 0.711304609918332\\
0.730769230769231 0.466566034269694\\
0.75 0.9\\
0.769230769230769 0.466566034269694\\
0.788461538461538 0.711304609918332\\
0.807692307692308 1.92954967807022\\
0.826923076923077 2.64812785369316\\
0.846153846153846 2.49555226711168\\
0.865384615384615 1.54961370571638\\
0.884615384615385 0.321656600909142\\
0.903846153846154 0.700018622361898\\
0.923076923076923 0.849045436113243\\
0.942307692307692 0.205188696869487\\
0.961538461538461 1.13447429946454\\
0.980769230769231 2.25122933360484\\
};
\end{axis}
\end{tikzpicture}%
	\caption{The amplitude spectrum.}
	\label{fig:zeropadding2}
\end{figure}


\section{The convolution property}
Here in this chapter the convolution property is demonstrated. 
This can be done by showing that Equation \ref{eq:convprop} holds.

\begin{equation}
y[n] = x[n]\star h[n] \quad \Leftrightarrow \quad Y(\omega) = X(\omega)H(\omega) 
\label{eq:convprop}
\end{equation}

To check this property we calculate $Y(\omega)$ and $X(\omega)H(\omega)$ separately and we check if both results are the same. 
We let the original signals \textbf{x} and \textbf{h} convolve with each other.
We then get the signaal $y[n]$, then, by using the DFT we get the $Y(\omega)$.\\
Now we only need to find the second part, the $X(\omega)H(\omega)$ part, respectively.
We do this by using zero padding to give both signals \textbf{x} and \textbf{h} the same length. 
If we then use DFT on our signals and use point to point multiplication, we obtain  $X(\omega)H(\omega)$.\\
Lastely we plot the absolute values in the frequencydomain. 
In Figure \ref{fig:convpro} we see two results. 
The first plot is the amplitude plot of  $X(\omega)H(\omega)$ in the frequencydomain and the second plot is the amplitude of $Y(\omega)$, also in the frequencydomain.
The plots are equal to each other, which means that convolution property holds.


\begin{figure}[h]
\centering
\includegraphics[scale = 1.0]{resources/labday1/theconvolutionproperty.png}
\caption{Time-frequency plot.}
\label{fig:convpro}
\end{figure}

\end{document} 