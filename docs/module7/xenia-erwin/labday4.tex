%!TEX program=xelatex+makeindex+bibtex
\documentclass[final]{scrreprt} %scrreprt of scrartcl
% Include all project wide packages here.
\usepackage{fullpage}
\usepackage{polyglossia}
\setmainlanguage{english}
\usepackage{csquotes}
\usepackage{graphicx}
\usepackage{epstopdf}
\usepackage{pdfpages}
\usepackage{caption}
\usepackage[list=true]{subcaption}
\usepackage{float}
\usepackage{standalone}
\usepackage{import}
\usepackage{tocloft}
\usepackage{wrapfig}
\usepackage{authblk}
\usepackage{array}
\usepackage{booktabs}
\usepackage[toc,page,title,titletoc]{appendix}
\usepackage{xunicode}
\usepackage{fontspec}
\usepackage{pgfplots}
\usepackage{SIunits}
\usepackage{units}
\pgfplotsset{compat=newest}
\pgfplotsset{plot coordinates/math parser=false}
\newlength\figureheight 
\newlength\figurewidth
\usepackage{amsmath}
\usepackage{mathtools}
\usepackage{unicode-math}
\usepackage[
    backend=bibtexu,
	texencoding=utf8,
bibencoding=utf8,
    style=ieee,
    sortlocale=en_US,
    language=auto
]{biblatex}
\usepackage{listings}
\newcommand{\includecode}[3][c]{\lstinputlisting[caption=#2, escapechar=, style=#1]{#3}}
\newcommand{\superscript}[1]{\ensuremath{^{\textrm{#1}}}}
\newcommand{\subscript}[1]{\ensuremath{_{\textrm{#1}}}}


\newcommand{\chapternumber}{\thechapter}
\renewcommand{\appendixname}{Bijlage}
\renewcommand{\appendixtocname}{Bijlagen}
\renewcommand{\appendixpagename}{Bijlagen}

\usepackage[hidelinks]{hyperref} %<--------ALTIJD ALS LAATSTE

\renewcommand{\familydefault}{\sfdefault}

\setmainfont[Ligatures=TeX]{Myriad Pro}
\setmathfont{Asana Math}
\setmonofont{Lucida Console}

\usepackage{titlesec, blindtext, color}
\definecolor{gray75}{gray}{0.75}
\newcommand{\hsp}{\hspace{20pt}}
\titleformat{\chapter}[hang]{\Huge\bfseries}{\chapternumber\hsp\textcolor{gray75}{|}\hsp}{0pt}{\Huge\bfseries}
\renewcommand{\familydefault}{\sfdefault}
\renewcommand{\arraystretch}{1.2}
\setlength\parindent{0pt}

%For code listings
\definecolor{black}{rgb}{0,0,0}
\definecolor{browntags}{rgb}{0.65,0.1,0.1}
\definecolor{bluestrings}{rgb}{0,0,1}
\definecolor{graycomments}{rgb}{0.4,0.4,0.4}
\definecolor{redkeywords}{rgb}{1,0,0}
\definecolor{bluekeywords}{rgb}{0.13,0.13,0.8}
\definecolor{greencomments}{rgb}{0,0.5,0}
\definecolor{redstrings}{rgb}{0.9,0,0}
\definecolor{purpleidentifiers}{rgb}{0.01,0,0.01}


\lstdefinestyle{csharp}{
language=[Sharp]C,
showspaces=false,
showtabs=false,
breaklines=true,
showstringspaces=false,
breakatwhitespace=true,
escapeinside={(*@}{@*)},
columns=fullflexible,
commentstyle=\color{greencomments},
keywordstyle=\color{bluekeywords}\bfseries,
stringstyle=\color{redstrings},
identifierstyle=\color{purpleidentifiers},
basicstyle=\ttfamily\small}

\lstdefinestyle{c}{
language=C,
showspaces=false,
showtabs=false,
breaklines=true,
showstringspaces=false,
breakatwhitespace=true,
escapeinside={(*@}{@*)},
columns=fullflexible,
commentstyle=\color{greencomments},
keywordstyle=\color{bluekeywords}\bfseries,
stringstyle=\color{redstrings},
identifierstyle=\color{purpleidentifiers},
}

\lstdefinestyle{matlab}{
language=Matlab,
showspaces=false,
showtabs=false,
breaklines=true,
showstringspaces=false,
breakatwhitespace=true,
escapeinside={(*@}{@*)},
columns=fullflexible,
commentstyle=\color{greencomments},
keywordstyle=\color{bluekeywords}\bfseries,
stringstyle=\color{redstrings},
identifierstyle=\color{purpleidentifiers}
}

\lstdefinestyle{vhdl}{
language=VHDL,
showspaces=false,
showtabs=false,
breaklines=true,
showstringspaces=false,
breakatwhitespace=true,
escapeinside={(*@}{@*)},
columns=fullflexible,
commentstyle=\color{greencomments},
keywordstyle=\color{bluekeywords}\bfseries,
stringstyle=\color{redstrings},
identifierstyle=\color{purpleidentifiers}
}

\lstdefinestyle{xaml}{
language=XML,
showspaces=false,
showtabs=false,
breaklines=true,
showstringspaces=false,
breakatwhitespace=true,
escapeinside={(*@}{@*)},
columns=fullflexible,
commentstyle=\color{greencomments},
keywordstyle=\color{redkeywords},
stringstyle=\color{bluestrings},
tagstyle=\color{browntags},
morestring=[b]",
  morecomment=[s]{<?}{?>},
  morekeywords={xmlns,version,typex:AsyncRecords,x:Arguments,x:Boolean,x:Byte,x:Char,x:Class,x:ClassAttributes,x:ClassModifier,x:Code,x:ConnectionId,x:Decimal,x:Double,x:FactoryMethod,x:FieldModifier,x:Int16,x:Int32,x:Int64,x:Key,x:Members,x:Name,x:Object,x:Property,x:Shared,x:Single,x:String,x:Subclass,x:SynchronousMode,x:TimeSpan,x:TypeArguments,x:Uid,x:Uri,x:XData,Grid.Column,Grid.ColumnSpan,Click,ClipToBounds,Content,DropDownOpened,FontSize,Foreground,Header,Height,HorizontalAlignment,HorizontalContentAlignment,IsCancel,IsDefault,IsEnabled,IsSelected,Margin,MinHeight,MinWidth,Padding,SnapsToDevicePixels,Target,TextWrapping,Title,VerticalAlignment,VerticalContentAlignment,Width,WindowStartupLocation,Binding,Mode,OneWay,xmlns:x}
}

%defaults
\lstset{
basicstyle=\ttfamily\small,
extendedchars=false,
numbers=left,
numberstyle=\ttfamily\tiny,
stepnumber=1,
tabsize=4,
numbersep=5pt
}
\addbibresource{../../library/bibliography.bib}

\begin{document}

\chapter{Labday 4: TDOA estimation}
\label{ch:labday4}
\section{Delta impulse TDOA}
Firstly, a setup with two microphones and a loudspeaker in 1-dimensional alignment were tested with delta impulses.
The time of arrival for delta peaks are easy to find from the sample data, hence the accuracy of the system can be measured very precise.
With two microphones aligned in a straight line before the loudspeaker, a 1-dimensional location can be estimated.
\\ \\
A sample frequency of 48 kHz was used on both the TX and RX, which would be sufficient to gain an accuracy of around 0.35 cm.
A simple Matlab script is used which finds the sample index that corresponds to the maximum amplitude in both microphone signals.
The samples indexes can be subtracted from each other and converted into a time difference using the sampling frequency.
This Matlab script is included in Appendix \ref{app:source}: tdoa-delta.m (Listing \ref{lst:tdoa-delta.m}).
\\ \\
The first microphone was placed 10 cm from the loudspeaker and the other was varied to gain a distance difference between the loudspeakers.
The arrival times of the two microphones were compared, which resulted in the data in Table \ref{tab:delta} using the estimated local speed of sound: 343.2 m/s.

\begin{table} [H]
	\centering
	\begin{tabular} { c | c }
	Distance (cm) & Error (cm) \\ \hline
	10 & 0.3 \\
	20 & 0.7 \\
	30 & 0.7 \\
	40 & 0.7 \\
	50 & 0.7 \\
	60 & 1.4 \\
	70 & 0.1 \\
	80 & 0.6 \\
	\end{tabular}
\caption{TDOA data converted into distance errors.}
\label{tab:delta}
\end{table}

The errors from Table \ref{tab:delta} are very small, but not as small as the theoretical error value of 0.35 cm.
This can be caused by several factors: position errors of the microphones, deviation from the estimated speed of sound and the fact that the channel deforms the delta impulse, making it harder to find the exact arrival time from the samples.
In overall, the results are sufficiently accurate for TDOA estimation since the target is to achieve about 1 cm accuracy.

\section{Beacon signal TDOA}


\section{A bit over the actual implementation}

To get the TDOA data out of the recorded audio we need to filter the signal and find the local maximums.
We went for an implementation with a circulant matrix, because in C\# a convolution is not that fast and because we use the Intel MKL through MathNet.Numerics the matrix multiplication is very fast.
The only downside is that the generation of the matrix uses a lot of cpu time.
We're looking to speed this up by using a different implementation of the toep function.
The resulting matrix also takes in a lot of memory.
After considering implementing the matrix multiplication on the gpu, we decided that this was not yet necessary, the CPU load when running the ASIOtest app (Appendix Section \ref{appsec:ASIOTest}) is about 50\%.
maybe later when the the ``sampling-rate'' needs to go up we'll transfer the matrix to the GPU memory once and then keep using it for multiplications. The provided PC has a NVidia Quadro so we can take full advantage of CUDA and CUDA is also implemented in MathNet.Numerics.
The matrix is generated on a worker thread so to not freeze the UI.
The actual function call is located at line 60 in file LocationSystem/Localizer.cs (Listing \ref{lst:LocationSystem-Localizer.cs}).
That beacon signal is reversed.
\\ \\
The actual filtering is done in performMeasurement() (line 93) method in the LocationSystem/Localizer.cs (Listing \ref{lst:LocationSystem-Localizer.cs}) file.
The responses matrix is made by taking the input samples of all the channels as the columns. 
This makes the filtering easy (namely one matrix multiplication).
\\ \\
After the filtering we need to do the peak detection. Peak detection also done in the performMeasurement() (line 93) method.
The actual peak detection code (line 114 to 133) is an implementation of the algorithm proposed in the reader.
So we first find the highest peak in the first channel and when we look in a smaller window of length $F_s/5$ so one period.
We may be able to make this window a little bit smaller.
After we find the relative delays we pass this info on to the actual localizer described in Chapter\ref{ch:labday5}.


\end{document} 