\documentclass[final]{scrreprt} %scrreprt of scrartcl
% Include all project wide packages here.
\usepackage{fullpage}
\usepackage{polyglossia}
\setmainlanguage{english}
\usepackage{csquotes}
\usepackage{graphicx}
\usepackage{epstopdf}
\usepackage{pdfpages}
\usepackage{caption}
\usepackage[list=true]{subcaption}
\usepackage{float}
\usepackage{standalone}
\usepackage{import}
\usepackage{tocloft}
\usepackage{wrapfig}
\usepackage{authblk}
\usepackage{array}
\usepackage{booktabs}
\usepackage[toc,page,title,titletoc]{appendix}
\usepackage{xunicode}
\usepackage{fontspec}
\usepackage{pgfplots}
\usepackage{SIunits}
\usepackage{units}
\pgfplotsset{compat=newest}
\pgfplotsset{plot coordinates/math parser=false}
\newlength\figureheight 
\newlength\figurewidth
\usepackage{amsmath}
\usepackage{mathtools}
\usepackage{unicode-math}
\usepackage[
    backend=bibtexu,
	texencoding=utf8,
bibencoding=utf8,
    style=ieee,
    sortlocale=en_US,
    language=auto
]{biblatex}
\usepackage{listings}
\newcommand{\includecode}[3][c]{\lstinputlisting[caption=#2, escapechar=, style=#1]{#3}}
\newcommand{\superscript}[1]{\ensuremath{^{\textrm{#1}}}}
\newcommand{\subscript}[1]{\ensuremath{_{\textrm{#1}}}}


\newcommand{\chapternumber}{\thechapter}
\renewcommand{\appendixname}{Bijlage}
\renewcommand{\appendixtocname}{Bijlagen}
\renewcommand{\appendixpagename}{Bijlagen}

\usepackage[hidelinks]{hyperref} %<--------ALTIJD ALS LAATSTE

\renewcommand{\familydefault}{\sfdefault}

\setmainfont[Ligatures=TeX]{Myriad Pro}
\setmathfont{Asana Math}
\setmonofont{Lucida Console}

\usepackage{titlesec, blindtext, color}
\definecolor{gray75}{gray}{0.75}
\newcommand{\hsp}{\hspace{20pt}}
\titleformat{\chapter}[hang]{\Huge\bfseries}{\chapternumber\hsp\textcolor{gray75}{|}\hsp}{0pt}{\Huge\bfseries}
\renewcommand{\familydefault}{\sfdefault}
\renewcommand{\arraystretch}{1.2}
\setlength\parindent{0pt}

%For code listings
\definecolor{black}{rgb}{0,0,0}
\definecolor{browntags}{rgb}{0.65,0.1,0.1}
\definecolor{bluestrings}{rgb}{0,0,1}
\definecolor{graycomments}{rgb}{0.4,0.4,0.4}
\definecolor{redkeywords}{rgb}{1,0,0}
\definecolor{bluekeywords}{rgb}{0.13,0.13,0.8}
\definecolor{greencomments}{rgb}{0,0.5,0}
\definecolor{redstrings}{rgb}{0.9,0,0}
\definecolor{purpleidentifiers}{rgb}{0.01,0,0.01}


\lstdefinestyle{csharp}{
language=[Sharp]C,
showspaces=false,
showtabs=false,
breaklines=true,
showstringspaces=false,
breakatwhitespace=true,
escapeinside={(*@}{@*)},
columns=fullflexible,
commentstyle=\color{greencomments},
keywordstyle=\color{bluekeywords}\bfseries,
stringstyle=\color{redstrings},
identifierstyle=\color{purpleidentifiers},
basicstyle=\ttfamily\small}

\lstdefinestyle{c}{
language=C,
showspaces=false,
showtabs=false,
breaklines=true,
showstringspaces=false,
breakatwhitespace=true,
escapeinside={(*@}{@*)},
columns=fullflexible,
commentstyle=\color{greencomments},
keywordstyle=\color{bluekeywords}\bfseries,
stringstyle=\color{redstrings},
identifierstyle=\color{purpleidentifiers},
}

\lstdefinestyle{matlab}{
language=Matlab,
showspaces=false,
showtabs=false,
breaklines=true,
showstringspaces=false,
breakatwhitespace=true,
escapeinside={(*@}{@*)},
columns=fullflexible,
commentstyle=\color{greencomments},
keywordstyle=\color{bluekeywords}\bfseries,
stringstyle=\color{redstrings},
identifierstyle=\color{purpleidentifiers}
}

\lstdefinestyle{vhdl}{
language=VHDL,
showspaces=false,
showtabs=false,
breaklines=true,
showstringspaces=false,
breakatwhitespace=true,
escapeinside={(*@}{@*)},
columns=fullflexible,
commentstyle=\color{greencomments},
keywordstyle=\color{bluekeywords}\bfseries,
stringstyle=\color{redstrings},
identifierstyle=\color{purpleidentifiers}
}

\lstdefinestyle{xaml}{
language=XML,
showspaces=false,
showtabs=false,
breaklines=true,
showstringspaces=false,
breakatwhitespace=true,
escapeinside={(*@}{@*)},
columns=fullflexible,
commentstyle=\color{greencomments},
keywordstyle=\color{redkeywords},
stringstyle=\color{bluestrings},
tagstyle=\color{browntags},
morestring=[b]",
  morecomment=[s]{<?}{?>},
  morekeywords={xmlns,version,typex:AsyncRecords,x:Arguments,x:Boolean,x:Byte,x:Char,x:Class,x:ClassAttributes,x:ClassModifier,x:Code,x:ConnectionId,x:Decimal,x:Double,x:FactoryMethod,x:FieldModifier,x:Int16,x:Int32,x:Int64,x:Key,x:Members,x:Name,x:Object,x:Property,x:Shared,x:Single,x:String,x:Subclass,x:SynchronousMode,x:TimeSpan,x:TypeArguments,x:Uid,x:Uri,x:XData,Grid.Column,Grid.ColumnSpan,Click,ClipToBounds,Content,DropDownOpened,FontSize,Foreground,Header,Height,HorizontalAlignment,HorizontalContentAlignment,IsCancel,IsDefault,IsEnabled,IsSelected,Margin,MinHeight,MinWidth,Padding,SnapsToDevicePixels,Target,TextWrapping,Title,VerticalAlignment,VerticalContentAlignment,Width,WindowStartupLocation,Binding,Mode,OneWay,xmlns:x}
}

%defaults
\lstset{
basicstyle=\ttfamily\small,
extendedchars=false,
numbers=left,
numberstyle=\ttfamily\tiny,
stepnumber=1,
tabsize=4,
numbersep=5pt
}
\addbibresource{../../library/bibliography.bib}

\begin{document}
\section{Labday 4: TDOA estimation}
\subsection{Report 1}
\subsubsection{Delta impulse TDOA}
Firstly, a setup with two microphones and a loudspeaker in 1-dimensional alignment were tested with delta impulses.
The time of arrival for delta peaks are easy to find from the sample data, hence the accuracy of the system can be measured very precise.
With two microphones aligned in a straight line before the loudspeaker, a 1-dimensional location can be estimated.
\\ \\
A sample frequency of 48 kHz was used on both the TX and RX, which would be sufficient to gain an accuracy of around 0.35 cm.
A simple Matlab script is used which finds the sample index that corresponds to the maximum amplitude in both microphone signals.
The samples indexes can be subtracted from each other and converted into a time difference using the speed of sound.
This Matlab script is included in Appendex A: tdoa-delta.m.
\\ \\
The first microphone was placed 10 cm from the loudspeaker and the other was varied to gain a distance difference between the loudspeakers.
The arrival times of the two microphones were compared, which resulted in the data in Table \ref{tab:delta} using the estimated local speed of sound: 343.2 m/s.

\begin{table} [H]
	\centering
	\begin{tabular} { c | c }
	Distance (cm) & Error (cm) \\ \hline
	10 & 0.3 \\
	20 & 0.7 \\
	30 & 0.7 \\
	40 & 0.7 \\
	50 & 0.7 \\
	60 & 1.4 \\
	70 & 0.1 \\
	80 & 0.6 \\
	\end{tabular}
\caption{TDOA data converted into distance errors.}
\label{tab:delta}
\end{table}

The errors from Table \ref{tab:delta} are very small, but not as small as the theoretical error value of 0.35 cm.
This can be caused by several factors: position errors of the microphones, deviation from the estimated speed of sound and the fact that the channel deforms the delta impulse, making it harder to find the exact arrival time from the samples.
In overall, the results are sufficiently accurate for TDOA estimation since the target is to achieve about 1 cm accuracy.

\subsubsection{Beacon signal TDOA}
After we applied of filters and the peak detection, the distances were not very consistent.
This might be due to the fact that our beacon settings are not the best possible, we will investigate this further during the system integration phase.
At the time they seemed to give the higher peaks, but when the noise level is high and the signal level low it does not filter out the noise enough.
When the signal is properly received the mean in the C\# implementation the variance is very small, around 5 to 20 samples depending on the position the car to the nearest microphone.
On a distance of half a meter we got a mean of $76$ samples, the actual number of samples should of been $70 = 0.5/343.2*48000$.
This corresponds to a distance of $\SI{54}{\centi\meter}$.
The furthest and nearly only real outlier was over 20 samples away of the mean, this measurement was discarded.
The best result was obtained to move the nearest microphone to about $\SI{5}{\centi\meter}$ of the beacon.
In figure \ref{fig:one-signal} is a typical data segment displayed and in figure \ref{fig:impulse-with-peak} in the filtered data and the peak displayed.\\
Sometimes there are several outliers, especially when the beacon is far from the microphone the algorithm struggles to find the proper peak time.
We could average over subsequent measurements but then the car would have to be stationary.
Our beacon settings for now are shown in table \ref{tab:beacon-settings}.
\begin{table} [H]
	\centering
	\begin{tabular} { c | c }
	Key & Value \\ \hline
	NBits & 32 \\
	Timer0 & 1 \\
	Timer1 & 3 \\
	Timer3 & 4 \\
	Code & e65a20e5 \\	
	\end{tabular}
\caption{Our beacon settings.}
\label{tab:beacon-settings}
\end{table}
The effects of NLOS for as good as we could measure it, made that one or more of the delays just outliers.
We take care of this in our Localizer (Chapter \ref{ch:labday5}).
For the effect of changes in the speed of sound see Chapter \ref{ch:labday5} Table \ref{tab:c-errors}

\section{Audio beacon testing}
\begin{figure}[H]
	\centering
	\setlength\figureheight{4cm}
    	\setlength\figurewidth{0.8\linewidth}
	% This file was created by matlab2tikz v0.4.6 running on MATLAB 8.3.
% Copyright (c) 2008--2014, Nico Schlömer <nico.schloemer@gmail.com>
% All rights reserved.
% Minimal pgfplots version: 1.3
% 
% The latest updates can be retrieved from
%   http://www.mathworks.com/matlabcentral/fileexchange/22022-matlab2tikz
% where you can also make suggestions and rate matlab2tikz.
% 
\begin{tikzpicture}

\begin{axis}[%
width=\figurewidth,
height=\figureheight,
scale only axis,
xmin=0,
xmax=0.2,
ymin=-0.004,
ymax=0.004
]
\addplot [color=blue,solid,forget plot]
  table[row sep=crcr]{
0	-2.26497668336378e-05	\\
0.00025	-6.65187908452936e-05	\\
0.000479166666666667	2.12192553590285e-05	\\
0.000770833333333333	9.94205620372668e-05	\\
0.00102083333333333	7.61747432989068e-05	\\
0.001375	-9.38177254283801e-05	\\
0.00158333333333333	-0.000122904792078771	\\
0.00185416666666667	7.61747432989068e-05	\\
0.00202083333333333	0.00013959409261588	\\
0.00227083333333333	-3.27825582644437e-05	\\
0.0023125	1.10864648377174e-05	\\
0.00254166666666667	-9.71555855358019e-05	\\
0.00277083333333333	-5.72204662603326e-05	\\
0.00322916666666667	5.49554897588678e-05	\\
0.00325	5.72204662603326e-05	\\
0.0036875	-0.000105142607935704	\\
0.004	-0.000151515021570958	\\
0.00416666666666667	-7.62939525884576e-05	\\
0.00420833333333333	-0.000132203116663732	\\
0.004625	0.00149023556150496	\\
0.0048125	-0.00162100815214217	\\
0.00522916666666667	0.00133764755446464	\\
0.00541666666666667	-0.00127029430586845	\\
0.0055625	-0.000382661877665669	\\
0.005625	0.00115191948134452	\\
0.00610416666666667	0.00154376053251326	\\
0.00645833333333333	-0.00157403969205916	\\
0.00666666666666667	-0.00189077877439559	\\
0.0068125	0.00162827991880476	\\
0.00695833333333333	-0.000908374902792275	\\
0.007125	0.00216221832670271	\\
0.00741666666666667	0.000897526857443154	\\
0.00785416666666667	-0.000711679516825825	\\
0.00808333333333333	-0.00127160560805351	\\
0.0083125	0.000218749075429514	\\
0.0085	0.00212097191251814	\\
0.0086875	-0.00134861480910331	\\
0.00897916666666667	-0.00113034259993583	\\
0.00920833333333333	0.00106143963057548	\\
0.00945833333333333	-0.000661611615214497	\\
0.0096875	0.0026718380395323	\\
0.00970833333333333	0.00295519875362515	\\
0.0101458333333333	-0.00243580364622176	\\
0.0102708333333333	-0.00322687672451139	\\
0.0106041666666667	0.00127398979384452	\\
0.0107291666666667	0.00313341664150357	\\
0.0109791666666667	-0.000131011023768224	\\
0.011125	0.00117194664198905	\\
0.0113958333333333	-0.00239372276701033	\\
0.0116875	-0.00192999863065779	\\
0.0118541666666667	0.000981330987997353	\\
0.0121041666666667	-0.00211000465787947	\\
0.0123541666666667	0.00203645252622664	\\
0.0124583333333333	0.00189220928587019	\\
0.012625	-0.00185847305692732	\\
0.0129791666666667	0.0015414955560118	\\
0.0133541666666667	-0.00336527870967984	\\
0.013375	-0.00245487713254988	\\
0.0138125	0.00270164036192	\\
0.0139166666666667	0.00387525605037808	\\
0.0142708333333333	-0.00237655662931502	\\
0.014375	-0.00159716629423201	\\
0.0144166666666667	0.00116038334090263	\\
0.0147708333333333	-0.000485897122416645	\\
0.0149166666666667	0.001178860780783	\\
0.0153958333333333	0.000365734129445627	\\
0.0155833333333333	-0.00174939655698836	\\
0.01575	-0.00154113792814314	\\
0.0161041666666667	0.0022739174310118	\\
0.0161875	0.00223147892393172	\\
0.0165416666666667	-0.00105559837538749	\\
0.01675	-0.00184202217496932	\\
0.0170208333333333	0.000662207661662251	\\
0.0172708333333333	-0.00176584743894637	\\
0.0174791666666667	0.00164628052152693	\\
0.017625	0.00168812298215926	\\
0.0179791666666667	-0.00136709224898368	\\
0.018125	-0.00191891216672957	\\
0.018375	0.000437259732279927	\\
0.0185416666666667	0.00146675133146346	\\
0.018875	-0.000996470567770302	\\
0.0189791666666667	-0.00136303913313895	\\
0.0191458333333333	0.00137233745772392	\\
0.019375	-0.00121617328841239	\\
0.0197083333333333	0.000443458615336567	\\
0.0200208333333333	0.00196409248746932	\\
0.0202916666666667	-0.000973582384176552	\\
0.0205625	-0.00154006504453719	\\
0.02075	-0.000377655087504536	\\
0.0208958333333333	0.00116777431685478	\\
0.0211458333333333	-0.000718236027751118	\\
0.0212291666666667	-0.000232338934438303	\\
0.0215416666666667	0.00172138237394392	\\
0.0216666666666667	0.00116503250319511	\\
0.0218958333333333	-0.00172328972257674	\\
0.0224166666666667	0.00106847297865897	\\
0.0225625	-0.000971794244833291	\\
0.0226875	-0.0013241769047454	\\
0.022875	0.00178706669248641	\\
0.0231458333333333	0.00118052971083671	\\
0.0234583333333333	-0.00171625637449324	\\
0.0235625	-0.00240409397520125	\\
0.0239375	0.00237095379270613	\\
0.0240416666666667	0.00140631210524589	\\
0.0242291666666667	-0.000908851739950478	\\
0.0245	-0.00129771244246513	\\
0.0247916666666667	0.00123381626326591	\\
0.0249375	-0.000590562878642231	\\
0.02525	0.00213539623655379	\\
0.0253541666666667	0.000282764463918284	\\
0.0255	-0.00167274498380721	\\
0.0258125	-0.00104665767867118	\\
0.0261458333333333	0.00141179573256522	\\
0.0264583333333333	0.00166952633298934	\\
0.0265833333333333	-0.000406622944865376	\\
0.0268958333333333	0.00121331226546317	\\
0.0270416666666667	-0.00137460243422538	\\
0.0273958333333333	-0.00193929695524275	\\
0.027625	0.00183379673399031	\\
0.0278333333333333	0.00306451367214322	\\
0.028125	-0.00229227566160262	\\
0.0284583333333333	0.000966429826803505	\\
0.0286041666666667	-0.00147426151670516	\\
0.0287708333333333	7.60555340093561e-05	\\
0.0290625	-0.000644564686808735	\\
0.0294583333333333	0.00201261066831648	\\
0.0295	0.000778317567892373	\\
0.0297708333333333	-0.00170350098051131	\\
0.0300208333333333	-0.00118207943160087	\\
0.0302916666666667	0.000996708986349404	\\
0.0307916666666667	0.00194585346616805	\\
0.0308541666666667	-0.00178909325040877	\\
0.0309583333333333	-0.0018086435738951	\\
0.0312916666666667	0.000113964095362462	\\
0.0314166666666667	-0.00164449238218367	\\
0.0316875	0.00180077576078475	\\
0.0318541666666667	-0.000939250108785927	\\
0.0322083333333333	0.00125670444685966	\\
0.0323125	0.00090920936781913	\\
0.0326666666666667	-0.00155878090299666	\\
0.032875	-0.00158429169096053	\\
0.0331458333333333	0.0012178422184661	\\
0.03325	0.000880718347616494	\\
0.033625	-0.00165069126524031	\\
0.0336458333333333	-0.00116300594527274	\\
0.0339166666666667	0.00184571766294539	\\
0.0341458333333333	0.00045311456779018	\\
0.0345208333333333	-0.00209486507810652	\\
0.0345625	-0.00110507023055106	\\
0.035	0.00187194370664656	\\
0.0351041666666667	0.00152528309263289	\\
0.0354375	-0.00185644649900496	\\
0.0355416666666667	-0.00112056743819267	\\
0.0357916666666667	0.00187122845090926	\\
0.0359791666666667	0.000285506277577952	\\
0.0362291666666667	-0.00194191955961287	\\
0.0364791666666667	-0.00113439571578056	\\
0.0367916666666667	0.00187492393888533	\\
0.037	0.00245380424894392	\\
0.0372916666666667	-0.000485539494547993	\\
0.0375625	-0.00228071236051619	\\
0.03775	0.000471830426249653	\\
0.0377916666666667	-0.000886440393514931	\\
0.038125	0.00190711044706404	\\
0.0384375	-0.0016872885171324	\\
0.0386666666666667	0.00113141548354179	\\
0.0388541666666667	0.000443577824626118	\\
0.0391458333333333	-0.0015649797860533	\\
0.0396041666666667	0.000470280705485493	\\
0.0396458333333333	-0.0011408330174163	\\
0.0398125	0.00143039238173515	\\
0.0401875	0.00148475193418562	\\
0.0405208333333333	-0.00138604652602226	\\
0.040625	-0.00106596958357841	\\
0.04075	0.000532746373210102	\\
0.0412291666666667	0.00170004391111434	\\
0.0414791666666667	-0.0010260344715789	\\
0.0415	-0.00136458885390311	\\
0.04175	0.00106883060652763	\\
0.042	-0.000805616495199502	\\
0.042375	0.000591158925089985	\\
0.042625	-0.00114345562178642	\\
0.0427708333333333	0.000356316595571116	\\
0.0428958333333333	-0.000973940012045205	\\
0.0432708333333333	0.00143218052107841	\\
0.0435	0.000981807825155556	\\
0.0437916666666667	-0.000650048314128071	\\
0.0439375	-0.00195896648801863	\\
0.0441041666666667	-2.36034429690335e-05	\\
0.0444166666666667	0.0010269881458953	\\
0.0446875	-0.00070023542502895	\\
0.0447916666666667	-0.000617742596659809	\\
0.0449375	0.000807523843832314	\\
0.0453541666666667	0.00134110462386161	\\
0.0455	-0.00101935875136405	\\
0.0457083333333333	-0.000966549036093056	\\
0.0460625	0.000870823976583779	\\
0.0463958333333333	-0.000286221533315256	\\
0.0464583333333333	0.000949144479818642	\\
0.0466041666666667	-0.00118672859389335	\\
0.0469791666666667	0.00087249290663749	\\
0.0470833333333333	0.000900626298971474	\\
0.0473125	-0.000820279237814248	\\
0.047625	-0.00111341488081962	\\
0.0478958333333333	0.000946879503317177	\\
0.0483125	0.0013394356938079	\\
0.048375	-0.00110924255568534	\\
0.0485833333333333	-0.00150012993253767	\\
0.0488125	0.000999450800009072	\\
0.0489375	0.00124096882063895	\\
0.0492916666666667	-0.00131797802168876	\\
0.0493958333333333	-0.0015351774636656	\\
0.0497291666666667	0.000297069578664377	\\
0.049875	-0.000565290509257466	\\
0.05	0.000949502107687294	\\
0.05025	0.000578284321818501	\\
0.050375	-0.000824213144369423	\\
0.0508541666666667	0.000742077943868935	\\
0.051	-0.00121641170699149	\\
0.0512291666666667	-0.000731110631022602	\\
0.0515625	0.00027453902293928	\\
0.05175	0.000801324960775673	\\
0.051875	-0.000364780455129221	\\
0.0523333333333333	0.000660777150187641	\\
0.052375	-0.00128293049056083	\\
0.052625	0.000525951443705708	\\
0.0528541666666667	-0.000632882176432759	\\
0.0531458333333333	-0.00110793125350028	\\
0.0534375	0.000810742494650185	\\
0.0537083333333333	0.0011262894840911	\\
0.053875	-0.000650405941996723	\\
0.0541458333333333	0.000558972416911274	\\
0.0543125	-0.00146794342435896	\\
0.0544166666666667	-0.00161790871061385	\\
0.0545625	0.00135934364516288	\\
0.0548958333333333	-0.00102818023879081	\\
0.0550833333333333	0.000795960542745888	\\
0.0555	0.0012091399403289	\\
0.0556666666666667	-0.000733971712179482	\\
0.0558333333333333	0.000379562436137348	\\
0.0561666666666667	-0.00132215034682304	\\
0.0563541666666667	-0.00136113178450614	\\
0.0566875	0.000542402325663716	\\
0.05675	-0.000359535246388987	\\
0.0568125	0.00146210205275565	\\
0.0573333333333333	0.00122606765944511	\\
0.0574791666666667	-0.00161242508329451	\\
0.0577916666666667	-0.00121247780043632	\\
0.0578333333333333	0.000335335760610178	\\
0.0581041666666667	-0.00101518642622978	\\
0.0584791666666667	0.00137686741072685	\\
0.0586458333333333	-0.000994920847006142	\\
0.0588958333333333	0.00103545200545341	\\
0.059	0.000849008676595986	\\
0.0593125	-0.000624895154032856	\\
0.0595208333333333	-0.00143599521834403	\\
0.0599166666666667	0.000396132527384907	\\
0.0599791666666667	-1.5497209460591e-05	\\
0.060375	0.00107944023329765	\\
0.0603958333333333	0.00121617328841239	\\
0.0606458333333333	-0.00130665313918144	\\
0.0609791666666667	-0.000891924020834267	\\
0.06125	0.000334978132741526	\\
0.0614583333333333	0.000742435571737587	\\
0.0616041666666667	-0.000705003796610981	\\
0.0617708333333333	0.000624537526164204	\\
0.0619375	-0.000633001385722309	\\
0.0622708333333333	0.000758409616537392	\\
0.0625416666666667	-0.000963449594564736	\\
0.0629166666666667	-0.00129854690749198	\\
0.0631458333333333	0.000560402928385884	\\
0.0635416666666667	8.48770214361139e-05	\\
0.0635833333333333	0.000944733736105263	\\
0.0636041666666667	0.000843763467855752	\\
0.06375	-0.000809788820333779	\\
0.06425	-0.000878334161825478	\\
0.0644375	0.000896692392416298	\\
0.0645833333333333	-0.000644683896098286	\\
0.0649583333333333	0.000915885088033974	\\
0.0653541666666667	0.000718236027751118	\\
0.0654166666666667	-0.000296115904347971	\\
0.065625	-0.00101757061202079	\\
0.065875	0.00127625477034599	\\
0.0659166666666667	-0.000604391156230122	\\
0.0659791666666667	0.000766039011068642	\\
0.0665208333333333	-0.000446081219706684	\\
0.0666666666666667	0.000641107617411762	\\
0.0669583333333333	0.00072002416709438	\\
0.06725	-0.000537157116923481	\\
0.067375	-0.000132083907374181	\\
0.0677291666666667	-0.00084960472304374	\\
0.06775	-0.00081610691267997	\\
0.068125	0.000816226121969521	\\
0.0682291666666667	0.000903368112631142	\\
0.0686041666666667	-0.000566363392863423	\\
0.0686875	-0.000517487584147602	\\
0.06875	0.000248193769948557	\\
0.0695208333333333	-0.000403404294047505	\\
0.0695833333333333	0.000432133732829243	\\
0.0696041666666667	0.000328660040395334	\\
0.0697291666666667	-0.000450611172709614	\\
0.070125	-0.000458121357951313	\\
0.0703958333333333	0.000454187451396137	\\
0.0705833333333333	0.000173568740137853	\\
0.0708333333333333	-0.000533342419657856	\\
0.0710208333333333	-0.000553846417460591	\\
0.0711458333333333	0.000502228795085102	\\
0.0716041666666667	0.000578999577555805	\\
0.0718958333333333	-0.000714063702616841	\\
0.0723333333333333	0.000597477017436177	\\
0.0724583333333333	0.000776052591390908	\\
0.072625	-0.000809550401754677	\\
0.0728958333333333	0.000589728413615376	\\
0.073125	-0.000896453973837197	\\
0.0733333333333333	-0.000470995961222798	\\
0.0733958333333333	0.000600934086833149	\\
0.0738541666666667	0.000506043492350727	\\
0.0740625	-0.000730156956706196	\\
0.0742083333333333	-0.000288844137685373	\\
0.0745	0.000259995489614084	\\
0.0750416666666667	0.000397205410990864	\\
0.075125	-0.000284791021840647	\\
0.0752916666666667	-0.000890493509359658	\\
0.0754375	0.000550508557353169	\\
0.0756875	-0.000458955822978169	\\
0.0759375	0.000832200166769326	\\
0.0763333333333333	0.000636816082987934	\\
0.0764583333333333	-0.000516176281962544	\\
0.0766458333333333	-0.000944018480367959	\\
0.0767916666666667	0.000436902104411274	\\
0.0770625	-0.000312209158437327	\\
0.0772291666666667	0.00093197834212333	\\
0.0777083333333333	0.000600338040385395	\\
0.0778541666666667	-0.000573396740946919	\\
0.0779583333333333	-0.00068044668296352	\\
0.0782083333333333	0.000820160028524697	\\
0.0785	-0.000791549799032509	\\
0.0786458333333333	0.000370740948710591	\\
0.0788958333333333	-0.000606060086283833	\\
0.0790625	0.000519156514201313	\\
0.0793333333333333	0.000700354634318501	\\
0.0797291666666667	-0.000398516713175923	\\
0.079875	0.000176668181666173	\\
0.0799375	-0.000923991319723427	\\
0.0801875	-0.000486493168864399	\\
0.0805833333333333	0.000723719655070454	\\
0.0806666666666667	0.000577926693949848	\\
0.0811041666666667	-0.000313043623464182	\\
0.0812083333333333	-0.000558018742594868	\\
0.0814375	0.000138640418299474	\\
0.081625	-0.000555992184672505	\\
0.0817708333333333	0.000340819387929514	\\
0.0822916666666667	-0.000624060689006001	\\
0.0824583333333333	0.000371456204447895	\\
0.082625	0.000844717142172158	\\
0.0829583333333333	-0.000640153943095356	\\
0.0831666666666667	-0.000723838864360005	\\
0.0833958333333333	0.000107049956568517	\\
0.0835833333333333	0.000970006105490029	\\
0.0838333333333333	-0.000312328367726877	\\
0.084125	-0.000808954355306923	\\
0.0841875	-2.6583675207803e-05	\\
0.0846666666666667	-0.000275969534413889	\\
0.0848125	0.000664353428874165	\\
0.0851666666666667	-0.000881195184774697	\\
0.0853958333333333	-0.000759124872274697	\\
0.0857291666666667	0.00052011018851772	\\
0.08575	0.000771165010519326	\\
0.0860208333333333	-0.000264167814748362	\\
0.0862708333333333	0.000629901944193989	\\
0.0864166666666667	-0.000723362027201802	\\
0.0868333333333333	-0.000458836613688618	\\
0.0869791666666667	0.000431299267802387	\\
0.087125	-0.000678539334330708	\\
0.0871875	0.000537157116923481	\\
0.0876875	0.000452876149211079	\\
0.0878333333333333	-0.000268816977040842	\\
0.0882708333333333	-0.000378727971110493	\\
0.0884166666666667	0.000223755865590647	\\
0.0886875	-0.000664830266032368	\\
0.0888333333333333	0.00048971181968227	\\
0.0889791666666667	-0.000362396269338205	\\
0.0893333333333333	0.000496149121318012	\\
0.0896041666666667	-0.000483751355204731	\\
0.08975	0.000403404294047505	\\
0.0898958333333333	-0.000405907689128071	\\
0.0900625	0.000510335026774555	\\
0.0905208333333333	-0.000673055707011372	\\
0.0906666666666667	0.000174045577296056	\\
0.0909166666666667	-0.00051558023551479	\\
0.0911875	0.000408649502787739	\\
0.0914791666666667	0.000684857426676899	\\
0.0916458333333333	-0.000312089949147776	\\
0.0918958333333333	0.000301003485219553	\\
0.0920625	-0.000729799328837544	\\
0.0921666666666667	-0.000551462231669575	\\
0.092625	0.000486731587443501	\\
0.0927291666666667	0.000499367772135884	\\
0.0931041666666667	-0.00016856194997672	\\
0.0933125	-0.00076878082472831	\\
0.0935625	0.000360250502126291	\\
0.0935833333333333	0.000410795269999653	\\
0.094	-0.000309586554067209	\\
0.0940625	9.97781899059191e-05	\\
0.0943541666666667	-0.000458478985819966	\\
0.0948125	0.000408291874919087	\\
0.0948541666666667	-8.8095672253985e-05	\\
0.0949375	0.000263690977590159	\\
0.0953125	-0.000813126680441201	\\
0.0954166666666667	-0.000542879162821919	\\
0.0957708333333333	0.000308871298329905	\\
0.096125	0.00035548213054426	\\
0.0962916666666667	-0.000366568623576313	\\
0.096375	-0.00052773958304897	\\
0.0967291666666667	0.000495433865580708	\\
0.0968333333333333	0.000366807042155415	\\
0.0972083333333333	-0.000470995961222798	\\
0.0973125	-0.000548124371562153	\\
0.0976875	0.00034332278301008	\\
0.0977916666666667	0.000366926251444966	\\
0.0978541666666667	-8.90493465703912e-05	\\
0.0982708333333333	0.000245928793447092	\\
0.0984375	-0.000380396901164204	\\
0.098625	-0.000295281439321116	\\
0.0987708333333333	2.68220937869046e-05	\\
0.0991041666666667	0.000176548972376622	\\
0.0992708333333333	-0.000166177764185704	\\
0.0995625	-0.000153660788782872	\\
0.0998541666666667	0.000427484570536762	\\
0.100020833333333	0.00031733515788801	\\
0.100291666666667	-0.000692963658366352	\\
0.1005	-0.000466942845378071	\\
0.100666666666667	0.000416994153056294	\\
0.101	-0.000134825721033849	\\
0.1013125	0.000417470990214497	\\
0.101416666666667	0.00027453902293928	\\
0.101854166666667	-0.000139713301905431	\\
0.101916666666667	-0.000423908291850239	\\
0.1023125	0.000180125251063146	\\
0.102375	-0.000119924559840001	\\
0.102520833333333	0.000418901501689106	\\
0.103020833333333	0.000478148518595845	\\
0.103166666666667	-0.00048661237815395	\\
0.103270833333333	-0.000610947667155415	\\
0.103625	0.000304102926747873	\\
0.103958333333333	0.000266075163381174	\\
0.104104166666667	-0.00030660632182844	\\
0.1045625	-0.000303268461721018	\\
0.104625	0.000203728704946116	\\
0.10475	-0.000178337111719884	\\
0.105083333333333	0.000260591536061838	\\
0.105479166666667	-0.000228643446462229	\\
0.105645833333333	0.000148177161463536	\\
0.105833333333333	-0.000361800222890452	\\
0.1060625	-0.000178813948878087	\\
0.1064375	0.000246763258473948	\\
0.106520833333333	0.000303983717458323	\\
0.106770833333333	-0.000285625486867502	\\
0.1069375	0.000179529204615392	\\
0.107375	-0.000325679808156565	\\
0.1078125	0.000155448928126134	\\
0.108020833333333	0.000458121357951313	\\
0.1083125	-6.5207488660235e-05	\\
0.108416666666667	0.000184655218617991	\\
0.1085625	-0.000424504338297993	\\
0.108770833333333	-0.000331759481923655	\\
0.109145833333333	0.000240206747548655	\\
0.109333333333333	0.000300645857350901	\\
0.109479166666667	-0.000277996092336252	\\
0.1096875	-0.00013804437185172	\\
0.110125	0.00021815302898176	\\
0.110354166666667	-0.000234365492360666	\\
0.110520833333333	9.8228469141759e-05	\\
0.110958333333333	-0.000341296225087717	\\
0.111020833333333	0.000167489066370763	\\
0.111208333333333	0.000197768240468577	\\
0.111270833333333	-0.000301837950246409	\\
0.1115625	-0.000162839904078282	\\
0.111708333333333	0.000319242506520823	\\
0.112	0.000123858466395177	\\
0.112145833333333	-0.000378131924662739	\\
0.112458333333333	-0.000329971342580393	\\
0.1126875	0.000247240095632151	\\
0.113	-0.000360131292836741	\\
0.113166666666667	0.000200986891286448	\\
0.113625	0.000167846694239415	\\
0.113791666666667	-0.000292181997792795	\\
0.113895833333333	-0.000578403531108052	\\
0.114270833333333	0.000121951117762364	\\
0.114541666666667	0.000367760716471821	\\
0.114708333333333	3.70740926882718e-05	\\
0.11475	0.000340104132192209	\\
0.115145833333333	-0.00043630605796352	\\
0.1154375	-0.000361800222890452	\\
0.115666666666667	0.000278353720204905	\\
0.115770833333333	0.000541806279215962	\\
0.1159375	6.43730254523689e-06	\\
0.116208333333333	0.000357508688466623	\\
0.116583333333333	-0.000375866948161274	\\
0.116604166666667	-0.00026547911693342	\\
0.1169375	-2.0384790332173e-05	\\
0.117270833333333	-8.96453930181451e-05	\\
0.117395833333333	0.000289440184133127	\\
0.117520833333333	0.000112891211756505	\\
0.11775	-0.000339627295034006	\\
0.118	3.17096746584866e-05	\\
0.1184375	-0.000241398840444162	\\
0.118854166666667	0.000195026426808909	\\
0.119041666666667	0.000322937994496897	\\
0.119333333333333	-0.0002623796754051	\\
0.119541666666667	-0.000124692931422032	\\
0.119833333333333	0.000112414374598302	\\
0.119958333333333	0.000332355528371409	\\
0.120208333333333	-0.000132083907374181	\\
0.120354166666667	3.34978140017483e-05	\\
0.120729166666667	-0.000407934247050434	\\
0.12075	-0.000308632879750803	\\
0.1210625	0.000347495108144358	\\
0.121270833333333	0.00018906596233137	\\
0.1214375	-0.000253677397267893	\\
0.121854166666667	-0.000172853484400548	\\
0.122020833333333	0.000156044974573888	\\
0.122395833333333	-0.000151991858729161	\\
0.122458333333333	0.000149011626490392	\\
0.122854166666667	0.000256657629506662	\\
0.122916666666667	-0.000118613257654943	\\
0.123208333333333	-0.000360727339284495	\\
0.123354166666667	8.89301372808404e-05	\\
0.123520833333333	-0.000221848516957834	\\
0.123770833333333	0.000229597120778635	\\
0.1240625	0.000130891814478673	\\
0.124395833333333	-0.000280499487416819	\\
0.124541666666667	0.000126957907923497	\\
0.124666666666667	-0.000267267256276682	\\
0.124979166666667	-0.000192403822438791	\\
0.12525	0.000277400045888498	\\
0.125416666666667	5.84125591558404e-05	\\
0.1255625	-0.000200390844838694	\\
0.125854166666667	7.1167953137774e-05	\\
0.126229166666667	-0.000186800985829905	\\
0.1265625	0.000224828749196604	\\
0.126729166666667	-9.72747948253527e-05	\\
0.126770833333333	0.00020432475139387	\\
0.127208333333333	-0.000126004233607091	\\
0.127645833333333	0.000264883070485666	\\
0.128	0.000177383437403478	\\
0.128125	-0.000111579909571446	\\
0.128166666666667	-0.000231742887990549	\\
0.128416666666667	9.0241439465899e-05	\\
0.128708333333333	0.000168323531397618	\\
0.128875	-0.000123977675684728	\\
0.129125	0.000109314933069982	\\
0.129395833333333	-0.00019073489238508	\\
0.1298125	-0.000147104277857579	\\
0.129958333333333	0.000152707114466466	\\
0.130041666666667	0.000205755262868479	\\
0.130416666666667	-0.00023031237651594	\\
0.1304375	-0.000252485304372385	\\
0.130791666666667	0.000197649031179026	\\
0.131125	-0.00024104121257551	\\
0.131270833333333	7.54594875616021e-05	\\
0.131479166666667	9.19103767955676e-05	\\
0.131791666666667	-0.00012433530355338	\\
0.1320625	-0.000133752837427892	\\
0.132125	0.000141739859827794	\\
0.132270833333333	-0.000115752234705724	\\
0.132645833333333	0.000221848516957834	\\
0.132979166666667	-0.00024414065410383	\\
0.133125	0.000110864653834142	\\
0.133270833333333	-0.000141382231959142	\\
0.133520833333333	0.000179171576746739	\\
0.133833333333333	0.000134468093165196	\\
0.133895833333333	-0.000139713301905431	\\
0.13425	0.00017929078603629	\\
0.1345	-0.000106096282252111	\\
0.1348125	-0.000166177764185704	\\
0.134958333333333	0.000130295768030919	\\
0.135104166666667	-0.000101685538538732	\\
0.135479166666667	0.000303149252431467	\\
0.1355	0.000267982512013987	\\
0.135645833333333	-0.000172495856531896	\\
0.136208333333333	0.000211000471608713	\\
0.136354166666667	-0.000243663816945627	\\
0.1365625	-0.000161647811182775	\\
0.136791666666667	0.000124812140711583	\\
0.136916666666667	0.000211596518056467	\\
0.13725	-0.000101089492090978	\\
0.137479166666667	-0.000251412420766428	\\
0.13775	0.00015175344015006	\\
0.137958333333333	0.000189542799489573	\\
0.138208333333333	-0.000205755262868479	\\
0.13825	0.000123500838526525	\\
0.138625	-0.000190138845937327	\\
0.139125	-0.000161290183314122	\\
0.1391875	0.000118970885523595	\\
0.139229166666667	-0.000145792975672521	\\
0.139604166666667	0.000200629263417795	\\
0.139916666666667	0.000128507628687657	\\
0.1400625	-9.32216789806262e-05	\\
0.1404375	4.80413509649225e-05	\\
0.1405625	-0.000207066565053537	\\
0.141	0.000310421019094065	\\
0.141020833333333	0.00023949149181135	\\
0.141166666666667	-5.50746990484186e-05	\\
0.141708333333333	0.000138998046168126	\\
0.141895833333333	-0.000208377867238596	\\
0.142041666666667	-0.000299930601613596	\\
0.142354166666667	0.000152349486597814	\\
0.142416666666667	1.14440936158644e-05	\\
0.142791666666667	0.000214695959584787	\\
0.142895833333333	0.000190973310964182	\\
0.1431875	-0.00012886525655631	\\
0.143520833333333	6.97374416631646e-05	\\
0.143645833333333	-0.000242352514760569	\\
0.143875	-0.000134110465296544	\\
0.1441875	0.000167250647791661	\\
0.144291666666667	0.000301122694509104	\\
0.144645833333333	-0.000233292608754709	\\
0.144708333333333	3.89814413210843e-05	\\
0.144958333333333	-0.000304698973195627	\\
0.1453125	0.000140070929774083	\\
0.145375	-6.41346050542779e-05	\\
0.145645833333333	0.000102520003565587	\\
0.146	-0.000189542799489573	\\
0.146104166666667	-0.000163674369105138	\\
0.146395833333333	4.0769580664346e-05	\\
0.146583333333333	3.98159063479397e-05	\\
0.147	-0.000182151808985509	\\
0.147020833333333	-0.000116109862574376	\\
0.147375	0.000160813346155919	\\
0.147666666666667	0.00011062623525504	\\
0.1479375	-0.000117778792628087	\\
0.1480625	8.38041378301568e-05	\\
0.148604166666667	0.000131130233057775	\\
0.148854166666667	-0.000120997443445958	\\
0.148916666666667	9.87053062999621e-05	\\
0.1494375	-3.99351156374905e-05	\\
0.14975	0.000176548972376622	\\
0.149979166666667	-0.000158905997523107	\\
0.150229166666667	0.000128150000819005	\\
0.150375	-5.29289318365045e-05	\\
0.150666666666667	0.000154733672388829	\\
0.1510625	0.000125288977869786	\\
0.151104166666667	-4.23193014285062e-05	\\
0.1513125	-0.000184655218617991	\\
0.1515625	7.25984646123834e-05	\\
0.151708333333333	-6.18696285528131e-05	\\
0.151854166666667	0.000179052367457189	\\
0.1524375	-0.000183701544301584	\\
0.152541666666667	6.46114422124811e-05	\\
0.152729166666667	0.000136613860377111	\\
0.152854166666667	-8.32080913824029e-05	\\
0.153083333333333	0.000164151206263341	\\
0.153458333333333	-0.000172615065821446	\\
0.1538125	9.60827019298449e-05	\\
0.153979166666667	0.000163197531946935	\\
0.154395833333333	-0.000126719489344396	\\
0.154854166666667	0.000168442740687169	\\
0.1549375	0.000219464331166819	\\
0.1551875	-0.000138401999720372	\\
0.1554375	0.000111579909571446	\\
0.1556875	-7.1167953137774e-05	\\
0.155958333333333	0.000148296370753087	\\
0.156229166666667	-7.10487438482232e-05	\\
0.156416666666667	-0.00015640260244254	\\
0.1565625	9.84668877208605e-05	\\
0.156708333333333	8.96453930181451e-05	\\
0.1568125	-0.000106334700831212	\\
0.157333333333333	-6.58035351079889e-05	\\
0.1575	0.000126719489344396	\\
0.157916666666667	0.000106573119410314	\\
0.1580625	-0.000136137023218907	\\
0.158270833333333	-0.000152468695887364	\\
0.158333333333333	0.000133037581690587	\\
0.15875	8.33273006719537e-05	\\
0.158791666666667	-0.000128746047266759	\\
0.159	-8.58306957525201e-05	\\
0.159270833333333	0.000184774427907541	\\
0.1595625	7.2836883191485e-05	\\
0.159916666666667	-0.000124931350001134	\\
0.159958333333333	-0.000200629263417795	\\
0.160291666666667	0.000109553351649083	\\
0.160375	-0.000101327910670079	\\
0.1606875	0.000144362464197911	\\
0.160833333333333	6.00814892095514e-05	\\
0.161270833333333	-0.000116825118311681	\\
0.161291666666667	-9.01222301763482e-05	\\
0.161729166666667	0.000113368048914708	\\
0.162020833333333	0.000159144416102208	\\
0.162083333333333	-7.0095069531817e-05	\\
0.1624375	0.00025630000163801	\\
0.1626875	-7.2836883191485e-05	\\
0.16275	6.44922329229303e-05	\\
0.163083333333333	-9.76324226940051e-05	\\
0.1634375	-9.01222301763482e-05	\\
0.1635625	9.4413771876134e-05	\\
0.163666666666667	0.000118613257654943	\\
0.164020833333333	-1.53780001710402e-05	\\
0.164229166666667	-8.60691143316217e-05	\\
0.164375	9.87053062999621e-05	\\
0.164729166666667	-7.48634411138482e-05	\\
0.164916666666667	0.000126600280054845	\\
0.165166666666667	-9.67979576671496e-05	\\
0.165333333333333	0.000178813948878087	\\
0.165458333333333	0.000133752837427892	\\
0.165708333333333	-0.000123381629236974	\\
0.165979166666667	0.000116348281153478	\\
0.166291666666667	1.19209303761636e-07	\\
0.166625	-0.00014412404561881	\\
0.1668125	5.76973034185357e-05	\\
0.1670625	-9.35793068492785e-05	\\
0.167229166666667	4.31537664553616e-05	\\
0.167541666666667	0.000118970885523595	\\
0.167708333333333	-4.54187429568265e-05	\\
0.167854166666667	9.77516265265876e-06	\\
0.167895833333333	-0.000147342696436681	\\
0.168291666666667	-0.000105023398646154	\\
0.168541666666667	8.29696728033014e-05	\\
0.168708333333333	-5.5909164075274e-05	\\
0.168958333333333	0.000131368651636876	\\
0.169166666666667	0.000133156790980138	\\
0.169520833333333	-0.000166058554896154	\\
0.169729166666667	-0.000152230277308263	\\
0.1700625	6.92606045049615e-05	\\
0.170125	0.000174760833033361	\\
0.170458333333333	1.07288371964387e-06	\\
0.170520833333333	0.00011825562978629	\\
0.170916666666667	-0.000142216696985997	\\
0.171020833333333	-0.000110030188807286	\\
0.171145833333333	-8.34465140542306e-07	\\
0.171604166666667	-9.46521904552355e-05	\\
0.171770833333333	0.000106692328699864	\\
0.172145833333333	-6.3538558606524e-05	\\
0.172291666666667	0.000124812140711583	\\
0.172479166666667	-8.45193935674615e-05	\\
0.172770833333333	2.56300008913968e-05	\\
0.1729375	-7.70092083257623e-05	\\
0.173145833333333	0.00011515618825797	\\
0.173479166666667	-0.000153064742335118	\\
0.17375	2.26497672883852e-06	\\
0.174	0.000106692328699864	\\
0.1740625	-3.38554418704007e-05	\\
0.174291666666667	4.79221416753717e-05	\\
0.174666666666667	-9.78708412731066e-05	\\
0.174895833333333	0.000127077117213048	\\
0.175416666666667	8.63075329107232e-05	\\
0.175583333333333	-0.000105381026514806	\\
0.1759375	0.000109910979517736	\\
0.176041666666667	6.41346050542779e-05	\\
0.1761875	-0.000191569357411936	\\
0.1765	-0.000105381026514806	\\
0.176854166666667	0.000178694739588536	\\
0.177	0.000176668181666173	\\
0.177333333333333	-0.000144839301356114	\\
0.1774375	-0.000122547164210118	\\
0.177895833333333	8.78572536748834e-05	\\
0.17825	-4.95910717290826e-05	\\
0.178541666666667	-0.000134229674586095	\\
0.178791666666667	5.23328853887506e-05	\\
0.178895833333333	0.000143170371302404	\\
0.1791875	-9.05990691535408e-06	\\
0.1793125	0.000108361258753575	\\
0.179479166666667	-8.90493465703912e-05	\\
0.179916666666667	-7.51018596929498e-05	\\
0.1801875	6.47306515020318e-05	\\
0.180520833333333	-0.000141978278406896	\\
0.1806875	0.000111937537440099	\\
0.181020833333333	-3.40938604495022e-05	\\
0.181145833333333	0.000108003630884923	\\
0.1815625	-7.98702312749811e-05	\\
0.181666666666667	-9.73940041149035e-05	\\
0.1818125	0.000118732466944493	\\
0.182104166666667	-2.46763265749905e-05	\\
0.182229166666667	0.000146150603541173	\\
0.1825625	-5.81741405767389e-05	\\
0.182729166666667	0.000188231497304514	\\
0.183020833333333	8.9406976258033e-06	\\
0.183166666666667	0.000180244460352696	\\
0.183666666666667	0.000147461905726232	\\
0.1838125	-4.60147930425592e-05	\\
0.183916666666667	-7.79628826421686e-05	\\
0.184083333333333	9.58442833507434e-05	\\
0.184354166666667	-8.85725094121881e-05	\\
0.1845	0.000162124648340978	\\
0.185	0.000109672560938634	\\
0.185145833333333	-8.74996258062311e-05	\\
0.185270833333333	-2.20537203858839e-05	\\
0.185604166666667	8.27312542241998e-05	\\
0.185875	-9.19103767955676e-05	\\
0.186145833333333	8.95261837285943e-05	\\
0.18625	0.000108242049464025	\\
0.186479166666667	-1.64508837769972e-05	\\
0.186666666666667	6.7353255872149e-05	\\
0.187104166666667	-0.000110030188807286	\\
0.1875625	1.37090701173292e-05	\\
0.187583333333333	5.88893963140436e-05	\\
0.187916666666667	-7.41481853765436e-05	\\
0.188020833333333	-6.31809307378717e-05	\\
0.188354166666667	4.49419057986233e-05	\\
0.1886875	5.87701870244928e-05	\\
0.188833333333333	-9.71555855358019e-05	\\
0.189041666666667	-0.000105261817225255	\\
0.189229166666667	0.000109791770228185	\\
0.189479166666667	-6.67572112433845e-06	\\
0.189625	0.000140547766932286	\\
0.189958333333333	-3.0398372473428e-05	\\
0.190104166666667	0.000128388419398107	\\
0.190333333333333	5.63860012334771e-05	\\
0.190729166666667	-0.00013041497732047	\\
0.1908125	-6.85453487676568e-05	\\
0.1911875	7.99894405645318e-05	\\
0.191333333333333	-5.4836280469317e-05	\\
0.1916875	8.23736263555475e-05	\\
0.191833333333333	-0.000117063536890782	\\
0.192104166666667	0.000121951117762364	\\
0.192354166666667	-0.000113487258204259	\\
0.192604166666667	7.76052547735162e-05	\\
0.192645833333333	5.28097225469537e-05	\\
0.192958333333333	-0.000144362464197911	\\
0.193125	-4.7683720367786e-06	\\
0.193270833333333	-0.000132322325953282	\\
0.193854166666667	2.09808367799269e-05	\\
0.194	-0.000144481673487462	\\
0.194416666666667	-0.000130653395899571	\\
0.194479166666667	2.21729296754347e-05	\\
0.194708333333333	-5.68628383916803e-05	\\
0.194875	6.74724651616998e-05	\\
0.194979166666667	0.000132441535242833	\\
0.195229166666667	-0.000106453910120763	\\
0.1954375	-7.02142788213678e-05	\\
0.1956875	6.75916744512506e-05	\\
0.195916666666667	-2.51531637331937e-05	\\
0.1960625	8.45193935674615e-05	\\
0.1964375	0.000116467490443029	\\
0.196583333333333	-8.9406976258033e-06	\\
0.196958333333333	0.000107169165858068	\\
0.197041666666667	-2.09808367799269e-05	\\
0.197375	6.48498607915826e-05	\\
0.1976875	-4.14848364016507e-05	\\
0.197708333333333	-3.40938604495022e-05	\\
0.1980625	5.8650977734942e-05	\\
0.198166666666667	5.06639553350396e-05	\\
0.198458333333333	-9.65595336310798e-06	\\
0.1989375	6.53266979497857e-05	\\
0.199041666666667	-2.38418607523272e-07	\\
0.1993125	5.49554897588678e-05	\\
0.199416666666667	-3.9696697058389e-05	\\
0.2	6.80685116094537e-05	\\
};
\end{axis}
\end{tikzpicture}%
	\caption{The audio data from microphones 1 with the beacon in the middle.}
	\label{fig:one-signal}
\end{figure}
\begin{figure}[H]
	\centering
	\setlength\figureheight{4cm}
    	\setlength\figurewidth{0.8\linewidth}
	% This file was created by matlab2tikz v0.4.6 running on MATLAB 8.3.
% Copyright (c) 2008--2014, Nico Schlömer <nico.schloemer@gmail.com>
% All rights reserved.
% Minimal pgfplots version: 1.3
% 
% The latest updates can be retrieved from
%   http://www.mathworks.com/matlabcentral/fileexchange/22022-matlab2tikz
% where you can also make suggestions and rate matlab2tikz.
% 
\begin{tikzpicture}

\begin{axis}[%
width=\figurewidth,
height=\figureheight,
scale only axis,
xmin=0,
xmax=0.249979166666667,
ymin=-0.2,
ymax=0.15
]
\addplot [color=blue,solid,forget plot]
  table[row sep=crcr]{
0	0	\\
0.000583333333333333	0	\\
0.00116666666666667	0	\\
0.00172916666666667	0	\\
0.0023125	0	\\
0.00289583333333333	0	\\
0.00345833333333333	0	\\
0.00404166666666667	0	\\
0.004625	0	\\
0.0051875	0	\\
0.00577083333333333	0	\\
0.00635416666666667	0	\\
0.00691666666666667	0	\\
0.0075	0	\\
0.00808333333333333	0	\\
0.00864583333333333	0	\\
0.00922916666666667	0	\\
0.0098125	0	\\
0.010375	0	\\
0.0109583333333333	0	\\
0.0115208333333333	0	\\
0.0121041666666667	0	\\
0.0126875	0	\\
0.01325	0	\\
0.0138333333333333	0	\\
0.0144166666666667	0	\\
0.0149791666666667	0	\\
0.0155625	0	\\
0.0161458333333333	0	\\
0.0167083333333333	0	\\
0.0172916666666667	0	\\
0.017875	0	\\
0.0184375	0	\\
0.0190208333333333	0	\\
0.0196041666666667	0	\\
0.0201666666666667	0	\\
0.02075	0	\\
0.0213125	0	\\
0.0218958333333333	0	\\
0.0224791666666667	0	\\
0.0230416666666667	0	\\
0.023625	0	\\
0.0242083333333333	0	\\
0.0247708333333333	0	\\
0.0253541666666667	0	\\
0.0259375	0	\\
0.0265	0	\\
0.0270833333333333	0	\\
0.0276666666666667	0	\\
0.0282291666666667	0	\\
0.0288125	0	\\
0.0293958333333333	0	\\
0.0299583333333333	0	\\
0.0305416666666667	0	\\
0.0311041666666667	0	\\
0.0316875	0	\\
0.0322708333333333	0	\\
0.0328333333333333	0	\\
0.0334166666666667	0	\\
0.034	0	\\
0.0345625	0	\\
0.0351458333333333	0	\\
0.0357291666666667	0	\\
0.0362916666666667	0	\\
0.036875	0	\\
0.0374583333333333	-0.000317335163572352	\\
0.0375833333333333	-0.000623226238985808	\\
0.0380208333333333	0.00117778790718148	\\
0.0381041666666667	0.00134074702100406	\\
0.0385625	-0.000573515960695659	\\
0.0386666666666667	-0.000588774767948053	\\
0.0391875	0.00107395660586462	\\
0.0393125	0.00163698215806107	\\
0.03975	-0.00295066871854033	\\
0.0403125	0.00264453917020546	\\
0.0403958333333333	0.00240755106301549	\\
0.040875	-0.00153625031634874	\\
0.0410625	-0.00283432042328968	\\
0.0414166666666667	-0.000887751719346852	\\
0.0419166666666667	0.00360143237958255	\\
0.0420625	-0.00826871485719494	\\
0.042375	-0.0144314784306516	\\
0.0426041666666667	0.0052568918824818	\\
0.0428125	0.0105196250253812	\\
0.0431666666666667	-0.0146890896676268	\\
0.0433958333333333	0.0112594378447284	\\
0.04375	-0.0185380005605111	\\
0.0441458333333333	-0.0292735134596569	\\
0.0442916666666667	0.0127607598611803	\\
0.04475	-0.0213696982282272	\\
0.0448958333333333	0.0282908709639287	\\
0.045	0.0206838870642514	\\
0.0451458333333333	-0.0248935250281193	\\
0.0457916666666667	0.0340142288472407	\\
0.0460416666666667	-0.0300470628129688	\\
0.04625	-0.0627709700045216	\\
0.0466041666666667	0.0362224621611631	\\
0.0466875	0.0441726497740547	\\
0.04675	-0.0136244315109479	\\
0.0472916666666667	0.0323625844631579	\\
0.0476458333333333	-0.0527579845308992	\\
0.048	0.0440895610375946	\\
0.0483541666666667	-0.0281336341754468	\\
0.0486458333333333	-0.042130117399438	\\
0.0488958333333333	0.0256722004432959	\\
0.0491041666666667	0.042492037096622	\\
0.0494583333333333	-0.0387342018493655	\\
0.0496666666666667	-0.0536043704041731	\\
0.0501041666666667	0.0268412860034459	\\
0.0502083333333333	0.0270457298101974	\\
0.0506458333333333	-0.0448623947835358	\\
0.05075	-0.041035061099933	\\
0.0512083333333333	0.0562874138358893	\\
0.0512916666666667	0.0544652994116177	\\
0.0515416666666667	-0.0732162082910008	\\
0.0518541666666667	-0.0233665735913746	\\
0.0521875	0.0529123608207556	\\
0.0524166666666667	0.0304209033361076	\\
0.0529375	-0.0911498186073914	\\
0.0530416666666667	-0.0996122476365144	\\
0.0533958333333333	0.128567233252397	\\
0.0535833333333333	0.0774209591618273	\\
0.0541458333333333	-0.108943712977634	\\
0.0547083333333333	0.0699590525414351	\\
0.0548958333333333	0.0861619809631975	\\
0.05525	-0.048809177316798	\\
0.0554375	-0.104976905342028	\\
0.0557916666666667	0.0698977784443287	\\
0.056	0.0943601241476699	\\
0.05625	-0.0592238970289145	\\
0.0566458333333333	-0.0675540045729122	\\
0.0570208333333333	0.0521869721014809	\\
0.0572083333333333	0.0720166049650288	\\
0.0576041666666667	-0.06395149931177	\\
0.0577083333333333	-0.0844060278259349	\\
0.0581875	0.0481638959684005	\\
0.0585416666666667	-0.0147953042142035	\\
0.0588958333333333	0.0652917703168896	\\
0.05925	-0.0784214834930026	\\
0.0593541666666667	-0.0925881979160295	\\
0.0597083333333333	0.0333380740598841	\\
0.0599583333333333	-0.0433456959362957	\\
0.0603125	0.0951804037090369	\\
0.0605	0.0171482584412388	\\
0.0608541666666667	-0.0872475004111948	\\
0.0610625	-0.0397995755442935	\\
0.0613125	0.0581820074130519	\\
0.0616875	0.00560092963678471	\\
0.0619791666666667	-0.0684460478787514	\\
0.0622083333333333	-0.0228028328911023	\\
0.0625625	0.0741877640275561	\\
0.0627916666666667	0.0325657165144548	\\
0.0630416666666667	-0.119465007191593	\\
0.0634375	-0.0281938352941324	\\
0.0637083333333333	0.0626214809572048	\\
0.0640208333333333	0.046086554909607	\\
0.0644791666666667	-0.0732083408947801	\\
0.0645625	-0.0803331230054027	\\
0.0651041666666667	0.0634524890670036	\\
0.0656041666666667	-0.043798809278087	\\
0.0656875	-0.0371681493675169	\\
0.0660208333333333	0.0172339696182462	\\
0.0665208333333333	0.0177916306022325	\\
0.0667708333333333	-0.0430692484176234	\\
0.066875	-0.0568643872338725	\\
0.0673958333333333	0.0245120552744993	\\
0.0677083333333333	0.0306780373766742	\\
0.0679791666666667	-0.0201429159651525	\\
0.0680208333333333	0.0117322220321512	\\
0.0683958333333333	-0.0521434606953335	\\
0.0685833333333333	-0.0438638975390404	\\
0.0690625	0.0570263928489112	\\
0.0691458333333333	0.047752385730746	\\
0.0697083333333333	-0.0623316838928076	\\
0.0698333333333333	-0.0928037274297822	\\
0.0702708333333333	0.0448154258276645	\\
0.0703958333333333	0.0676405507492746	\\
0.0708125	-0.0293310919132637	\\
0.0710208333333333	-0.0493294061350298	\\
0.0714375	0.00657582349185759	\\
0.0715833333333333	-0.0240206749517711	\\
0.072	0.0383292476071802	\\
0.0721041666666667	0.0472430047329908	\\
0.0724375	-0.0423888018963225	\\
0.072625	-0.0271207124687862	\\
0.0731666666666667	0.0415771058069367	\\
0.0736041666666667	-0.087781439626724	\\
0.0737916666666667	-0.033615712133269	\\
0.0741458333333333	0.0697128857659663	\\
0.0743541666666667	0.0514788697737458	\\
0.0748333333333333	-0.075045832691103	\\
0.0749375	-0.0737723199999891	\\
0.0754583333333333	0.0320174721837247	\\
0.0756666666666667	0.056305653451318	\\
0.0760208333333333	-0.0417082361764187	\\
0.0762291666666667	-0.0604206396437803	\\
0.0765833333333333	0.0259842893833593	\\
0.076875	0.0370739732154561	\\
0.0771458333333333	-0.0312676471153281	\\
0.0773958333333333	0.0261850373976813	\\
0.0777291666666667	-0.0581238342001598	\\
0.0778125	-0.0574414799275473	\\
0.0783541666666667	0.059858448038085	\\
0.0788125	-0.0772520396258187	\\
0.0789166666666667	-0.0586538384886808	\\
0.0792708333333333	0.0592302159329847	\\
0.0795833333333333	0.0547574819011061	\\
0.0798333333333333	-0.0536888903625368	\\
0.0803125	-0.074115403869655	\\
0.0805625	0.0249284535684637	\\
0.0808125	-0.041471486332739	\\
0.0810625	0.0706224527684753	\\
0.0812708333333333	0.0434391549351858	\\
0.0817291666666667	-0.0543477601027007	\\
0.0818333333333333	-0.0523318116506744	\\
0.0821875	0.0209124107855132	\\
0.0824375	-0.0207176234671351	\\
0.0829583333333333	0.0338457858499623	\\
0.0830625	0.0342388184717493	\\
0.0835208333333333	-0.0223138365074647	\\
0.0835625	-0.00697887083106252	\\
0.0840208333333333	-0.0469255505586261	\\
0.084125	-0.0355892223633418	\\
0.0844791666666667	0.0758917419680074	\\
0.0846875	0.0469873003598877	\\
0.085125	-0.050403720178565	\\
0.0853125	-0.0426818186713263	\\
0.08575	0.00166618820230724	\\
0.0858333333333333	-0.00613367665800979	\\
0.0861458333333333	0.0436031865929181	\\
0.0864791666666667	0.0289108782417316	\\
0.0868541666666667	-0.0642582256627975	\\
0.0870416666666667	-0.0374841737407223	\\
0.0875416666666667	0.0388590137481515	\\
0.0875625	0.0364465747734357	\\
0.0881041666666667	-0.023639920153073	\\
0.0882083333333333	-0.0399341631025436	\\
0.0886041666666667	0.0128855719995045	\\
0.0888541666666667	-0.025785210673348	\\
0.089125	0.00959813664076137	\\
0.0892916666666667	-0.00953006845975324	\\
0.0895625	0.0285793574191757	\\
0.0902916666666667	-0.0600113934260662	\\
0.0904375	0.00895702943762444	\\
0.0904791666666667	-0.0410748765489188	\\
0.0910208333333333	0.0411037253340965	\\
0.0915833333333333	-0.0271679191566818	\\
0.0917916666666667	-0.046905284413242	\\
0.092125	0.0110902796568553	\\
0.0921666666666667	-0.0214155933433631	\\
0.0922291666666667	0.0214145209929484	\\
0.0927916666666667	-0.0155930537584936	\\
0.0931875	0.0295913256381937	\\
0.0933125	0.0183327221632226	\\
0.0936458333333333	-0.0620879001553476	\\
0.0942291666666667	0.0557901922938981	\\
0.0944791666666667	-0.025223734298379	\\
0.0946666666666667	-0.0509234723101599	\\
0.0950208333333333	0.0231341146647992	\\
0.0952291666666667	0.0385906740134487	\\
0.0955833333333333	-0.0354143418351214	\\
0.0956666666666667	-0.0387570900998071	\\
0.0962083333333333	0.0229560164921168	\\
0.0963125	0.0257397922860036	\\
0.0967708333333333	-0.0111209166529989	\\
0.0972708333333333	-0.0316499510190624	\\
0.0973125	0.0100271714554765	\\
0.097375	-0.0328230895049728	\\
0.0978125	0.0266339813576906	\\
0.0981041666666667	0.0169281982734901	\\
0.098375	-0.0199432394656469	\\
0.0985416666666667	0.0107949985967934	\\
0.0990833333333333	-0.0453535365847983	\\
0.0991875	-0.0475791740018394	\\
0.099625	0.0435171182234626	\\
0.0997291666666667	0.0558093855252082	\\
0.100166666666667	-0.0342453754590224	\\
0.1005	-0.0458561228340386	\\
0.10075	0.0133619325526979	\\
0.1008125	-0.0107401616000971	\\
0.100979166666667	0.0177178404117058	\\
0.101479166666667	0.0145380518988532	\\
0.101895833333333	-0.0214996362424245	\\
0.1021875	-0.0365445656520933	\\
0.102520833333333	0.00745165493776767	\\
0.102583333333333	-0.00133991223765406	\\
0.103083333333333	0.0220460920427286	\\
0.103354166666667	0.0370363038030064	\\
0.1036875	-0.0460176519918605	\\
0.103895833333333	-0.0595922538177547	\\
0.10425	0.0293314488094438	\\
0.104291666666667	-0.00543594417149507	\\
0.104458333333333	0.0509982163089262	\\
0.104854166666667	0.00988638497733518	\\
0.105416666666667	-0.0267174269279167	\\
0.105916666666667	-0.0453559204975136	\\
0.105958333333333	-5.3763329560752e-05	\\
0.106	-0.0296105177887966	\\
0.106270833333333	0.0453938296186607	\\
0.1065625	0.0320851836245311	\\
0.107083333333333	-0.0693649131732172	\\
0.1071875	-0.0651252342669295	\\
0.107645833333333	0.0339785852534078	\\
0.10775	0.0411076592195059	\\
0.108104166666667	-0.0271555214894761	\\
0.108354166666667	0.0236680531270395	\\
0.108416666666667	-0.0342798271005904	\\
0.109	-0.0460780913092549	\\
0.109458333333333	0.0206191562192544	\\
0.109666666666667	0.042279486546704	\\
0.110020833333333	-0.0219841025927963	\\
0.1100625	0.00873541939245115	\\
0.1103125	-0.0517494738519417	\\
0.110604166666667	-0.0388271853469178	\\
0.111145833333333	0.0415283487991474	\\
0.11125	0.0364800732013464	\\
0.111604166666667	-0.038651947839071	\\
0.111895833333333	-0.0350428862439003	\\
0.11225	0.0211455846447279	\\
0.112354166666667	0.0154238957361486	\\
0.1128125	-0.0333572665131214	\\
0.112895833333333	-0.0273336204737689	\\
0.113145833333333	0.0293601782426549	\\
0.113895833333333	-0.0288616454810153	\\
0.114041666666667	-0.00743258088016319	\\
0.114083333333333	-0.0115582956628941	\\
0.114333333333333	0.0224078917532324	\\
0.114645833333333	0.0148953212631113	\\
0.115104166666667	-0.0481426773051226	\\
0.115208333333333	-0.0407180832864924	\\
0.1155625	0.0210620188699977	\\
0.115875	0.00238513956310271	\\
0.116291666666667	-0.0185910485425893	\\
0.116541666666667	0.00402319492957304	\\
0.1168125	-0.0275526079788051	\\
0.117020833333333	-0.0289275679883758	\\
0.117479166666667	0.00911164417129839	\\
0.117583333333333	0.0190866017208009	\\
0.118083333333333	-0.0146286503287456	\\
0.118208333333333	-0.0409652039463708	\\
0.118645833333333	0.0111584678751342	\\
0.118854166666667	0.0146125572737219	\\
0.119208333333333	-0.0185912869337699	\\
0.119416666666667	-0.0335953273756786	\\
0.1198125	0.000572919843307318	\\
0.120041666666667	0.00964427118765343	\\
0.120375	-0.0239428309741925	\\
0.120479166666667	-0.0272756847341498	\\
0.120854166666667	0.00570190005873883	\\
0.120958333333333	0.00214791321013763	\\
0.121458333333333	-0.0314854419098083	\\
0.1215625	-0.0249052076314911	\\
0.122083333333333	0.0119827998323672	\\
0.122270833333333	0.0136326567269407	\\
0.122625	-0.0303010974633935	\\
0.122729166666667	-0.0275250703222696	\\
0.123208333333333	0.00368988584443741	\\
0.123395833333333	0.00761389837236948	\\
0.123791666666667	-0.00876891708594485	\\
0.123854166666667	-0.00360679675941356	\\
0.124104166666667	-0.0257161884610468	\\
0.124416666666667	-0.0111775413370196	\\
0.124916666666667	0.00901997184530501	\\
0.125020833333333	0.00628268771299645	\\
0.125270833333333	-0.0198583628200595	\\
0.1260625	0.0131169571077407	\\
0.126104166666667	-0.0154025571055172	\\
0.126416666666667	-0.0299443039542666	\\
0.126666666666667	0.0059804922703961	\\
0.126979166666667	0.0213543200752611	\\
0.127229166666667	-0.0192685148174405	\\
0.127333333333333	-0.0224277998784146	\\
0.127395833333333	-0.00467562727840232	\\
0.1281875	-0.019767525066527	\\
0.1284375	0.00632953726449159	\\
0.128645833333333	0.0178906939911485	\\
0.129	-0.0125719322257964	\\
0.1290625	0.00193512445946453	\\
0.129395833333333	-0.0329893866718862	\\
0.129604166666667	-0.0171749612799204	\\
0.1299375	0.021538498382597	\\
0.13025	0.00141942506081705	\\
0.1305	-0.0330581704199631	\\
0.1308125	-0.0140548959719808	\\
0.1310625	0.0188406727612573	\\
0.1315	-0.0294106041645819	\\
0.131854166666667	0.00756168435896143	\\
0.131916666666667	-0.0049797302448269	\\
0.132479166666667	0.00981116391199066	\\
0.132708333333333	-0.0209223055856	\\
0.133354166666667	0.00835418791052689	\\
0.133604166666667	-0.0203270937977322	\\
0.1336875	-0.0164325258649569	\\
0.134208333333333	0.0103837263029618	\\
0.1346875	-0.0320236721460674	\\
0.134791666666667	-0.0284292733373945	\\
0.135354166666667	0.0189580938338167	\\
0.135541666666667	0.020734550673069	\\
0.135895833333333	-0.0290427242866826	\\
0.136	-0.0254210264747599	\\
0.1360625	0.00511622471776718	\\
0.1365625	0.000194311011910031	\\
0.136625	-0.0193674587475243	\\
0.137125	-0.00752413370707927	\\
0.137395833333333	0.00504064580576369	\\
0.1379375	0.00226747997294297	\\
0.1381875	-0.0160036106910866	\\
0.138375	-0.028192880924621	\\
0.138833333333333	0.00951397527683184	\\
0.139020833333333	0.0181450864819226	\\
0.139395833333333	-0.0159186146041179	\\
0.139583333333333	-0.024882319626272	\\
0.139958333333333	0.00133550180453312	\\
0.140145833333333	0.00467884591216716	\\
0.140520833333333	-0.0109521165722981	\\
0.140625	-0.0126024501628308	\\
0.140979166666667	0.0168312807493294	\\
0.141166666666667	0.0067284115775692	\\
0.141541666666667	-0.0299617088141986	\\
0.14175	-0.0197577500427997	\\
0.142208333333333	0.0140136496311243	\\
0.1423125	0.016005637187277	\\
0.142770833333333	-0.0189343712506798	\\
0.142979166666667	-0.0199639821528308	\\
0.143354166666667	-0.00334024482936002	\\
0.143666666666667	-0.0136563795030042	\\
0.144	0.0075219879213364	\\
0.1445625	-0.0155400055099335	\\
0.144666666666667	-0.0152429359230837	\\
0.145020833333333	-0.000520825701585181	\\
0.145166666666667	-0.0110270991467587	\\
0.145604166666667	0.0111035123035776	\\
0.1458125	0.00839412323261968	\\
0.1460625	-0.0167228003863329	\\
0.1463125	0.00805258847231016	\\
0.146854166666667	-0.00952780361654959	\\
0.146958333333333	-0.0106781733790058	\\
0.147395833333333	0.0198267721410161	\\
0.1475	0.0195758366354539	\\
0.147958333333333	-0.00589156219484721	\\
0.14825	-0.0176329633381442	\\
0.148604166666667	0.0143150108819441	\\
0.148645833333333	-0.0153291244151319	\\
0.149	0.0245232609887012	\\
0.149208333333333	0.0185071251019622	\\
0.1495625	-0.0225460553843959	\\
0.149770833333333	-0.0160138624497677	\\
0.15	0.00534844464891648	\\
0.150854166666667	-0.0139600054656057	\\
0.150895833333333	0.0152658242888037	\\
0.151	0.0115290895192857	\\
0.151145833333333	-0.0201873803041508	\\
0.1515625	-0.00677824098329438	\\
0.1518125	0.0109618914839302	\\
0.152104166666667	0.00563335504079987	\\
0.15225	-0.0150929705812359	\\
0.153	0.0133345144002988	\\
0.153166666666667	-0.00900936224979887	\\
0.1533125	-0.00101745117456176	\\
0.153645833333333	-0.0199304841762569	\\
0.153854166666667	-0.0130786910945062	\\
0.154291666666667	0.0181661866051286	\\
0.154395833333333	0.0165566225209091	\\
0.154958333333333	-0.0210063481828229	\\
0.1555	0.01046574253985	\\
0.156	0.00877702340591213	\\
0.156041666666667	-0.00775420761260648	\\
0.156104166666667	0.0052759654091119	\\
0.156458333333333	-0.0262402326154643	\\
0.15675	-0.0152143255954798	\\
0.157	0.00726890653101009	\\
0.157291666666667	0.00496256426026775	\\
0.157833333333333	-0.0163985509741167	\\
0.1579375	-0.0185494444924927	\\
0.158395833333333	0.0004245043520541	\\
0.158708333333333	0.00320243871618686	\\
0.158854166666667	-0.00500118787010706	\\
0.159	-0.000787139064925668	\\
0.15925	-0.00852537258435859	\\
0.159895833333333	-0.00181067012437097	\\
0.160125	-0.0103480826668942	\\
0.160208333333333	-0.014390708532801	\\
0.1606875	-0.00659453934162002	\\
0.160979166666667	0.00473737771972083	\\
0.161229166666667	-0.00746452898039252	\\
0.161291666666667	-0.00684595181229497	\\
0.161625	-0.0164545791685669	\\
0.161895833333333	-0.0135725751702012	\\
0.162333333333333	0.00639081084975146	\\
0.1624375	0.00513100682428558	\\
0.162895833333333	-0.0247069624266487	\\
0.163083333333333	-0.0176358242562884	\\
0.163541666666667	0.00718879772918513	\\
0.163645833333333	0.00612938461264889	\\
0.164	-0.00901198496748634	\\
0.164229166666667	-0.00475704721137049	\\
0.164479166666667	-0.0138760821328106	\\
0.164791666666667	-0.0105559837482474	\\
0.16525	0.0124841942081275	\\
0.165333333333333	0.00904858221927896	\\
0.165708333333333	-0.0147968547863115	\\
0.165895833333333	-0.0102683317754639	\\
0.166375	0.00565302433651027	\\
0.166479166666667	0.00271213085503064	\\
0.166916666666667	-0.00732266988671881	\\
0.16725	0.00208270567725322	\\
0.167583333333333	-0.00864410501188217	\\
0.1676875	-0.00772023293242796	\\
0.1679375	0.00305485755561108	\\
0.168479166666667	-0.00992810840588731	\\
0.16875	0.00149738798273802	\\
0.168854166666667	0.00512349659604183	\\
0.169291666666667	-0.0107734215197297	\\
0.169583333333333	-0.0126479878115333	\\
0.1699375	0.00104808815034119	\\
0.169979166666667	-0.0070791251986293	\\
0.170145833333333	0.00516653117337285	\\
0.1706875	-0.00901889895250463	\\
0.170854166666667	0.000230312324418946	\\
0.171104166666667	-0.00819552041053839	\\
0.171541666666667	0.001171112234573	\\
0.171916666666667	0.00387489843480182	\\
0.1721875	-0.0102071773046788	\\
0.172375	-0.0122143044227983	\\
0.172729166666667	0.00913596254633831	\\
0.172833333333333	0.00719022825600746	\\
0.173083333333333	-0.00531745020772689	\\
0.17375	-0.00723004427464957	\\
0.173916666666667	-0.00155425091072914	\\
0.174208333333333	0.00202250505151369	\\
0.174479166666667	-0.00820958707055297	\\
0.174583333333333	-0.00953281034531983	\\
0.175041666666667	0.00401282352468968	\\
0.175145833333333	0.00622570582878268	\\
0.1756875	-0.00452458918410059	\\
0.175916666666667	-0.00654804792804953	\\
0.176270833333333	0.00117290034455664	\\
0.176625	-0.00399518058253534	\\
0.177	0.00107026115392728	\\
0.177270833333333	-0.00613296108269878	\\
0.177479166666667	-0.00610041693397534	\\
0.177708333333333	0.00442254589106028	\\
0.178020833333333	0.00237989447168729	\\
0.1780625	-0.00572371553715811	\\
0.1785625	-0.00510656892606676	\\
0.179104166666667	0.000698208954645452	\\
0.179354166666667	-0.00456798140669434	\\
0.179604166666667	0.00278377569145505	\\
0.1798125	0.00334584750225986	\\
0.1800625	-0.00640344693374573	\\
0.180291666666667	-0.00190782567449332	\\
0.180791666666667	0.00657117434582233	\\
0.181	0.00605273312419286	\\
0.181333333333333	-0.00407779266447506	\\
0.181583333333333	0.0038309101375944	\\
0.181854166666667	-0.00598371110459084	\\
0.182145833333333	-0.00197744397291899	\\
0.182604166666667	0.00461995656905856	\\
0.1830625	-0.00523626867948224	\\
0.18325	-0.00414264252185603	\\
0.1835	-0.000808477627401771	\\
0.184145833333333	0.00193154831254105	\\
0.1843125	-0.00591671528013649	\\
0.184666666666667	0.000144004798926289	\\
0.1848125	-0.0070760258152518	\\
0.184916666666667	-0.00610399316866506	\\
0.185083333333333	0.00215816520110934	\\
0.1855625	-0.00526404445668049	\\
0.185895833333333	0.00408613730616025	\\
0.186333333333333	-0.00923693286904381	\\
0.186583333333333	0.000676155241308152	\\
0.186958333333333	0.00380742589845795	\\
0.187208333333333	-0.00220990204923055	\\
0.187375	-0.00754928684943934	\\
0.187729166666667	0.00368619007224424	\\
0.187833333333333	0.00324726141377596	\\
0.188104166666667	-0.00617909504757108	\\
0.188416666666667	-0.00202417392779353	\\
0.188875	0.00548994598216268	\\
0.188979166666667	0.00699710922322083	\\
0.189458333333333	-0.00708139020844101	\\
0.189770833333333	-0.00753271670961908	\\
0.190083333333333	0.000615596809893759	\\
0.190479166666667	0.00609564842548593	\\
0.190666666666667	0.00484383153857948	\\
0.191083333333333	-0.00904739011190259	\\
0.191291666666667	-0.00536596838742298	\\
0.191541666666667	0.00121128566468087	\\
0.1918125	-0.00362014806734123	\\
0.192145833333333	0.00534284179923361	\\
0.1924375	-0.000777244618234363	\\
0.19275	-0.0054830317815231	\\
0.1930625	-0.00399828002406366	\\
0.1934375	0.00211417701575556	\\
0.1935625	0.00161397474153091	\\
0.193708333333333	-0.00292170080240339	\\
0.194145833333333	0.0012693407243205	\\
0.194583333333333	-0.0044217114521814	\\
0.194875	-0.00300407441628181	\\
0.195125	0.00396895454710489	\\
0.195375	-0.00457453781339723	\\
0.195833333333333	0.00617086958928326	\\
0.1959375	0.00589025089126949	\\
0.196083333333333	-0.00413477460620015	\\
0.196479166666667	-0.00255191350572659	\\
0.1969375	0.00559711518587847	\\
0.197041666666667	0.00567483963743598	\\
0.1975625	-0.00159239781451959	\\
0.19775	-0.00320816072773766	\\
0.198104166666667	0.00616610121278427	\\
0.198208333333333	0.00550079406480108	\\
0.198666666666667	-0.00180339829148579	\\
0.198770833333333	-0.00120008001283622	\\
0.199020833333333	0.00539732039089813	\\
0.199333333333333	0.00554585512344374	\\
0.199666666666667	-0.00272655516067744	\\
0.199875	-0.000450372722241355	\\
0.200020833333333	0.00649356914303212	\\
0.2006875	-0.00133764756810706	\\
0.2009375	0.00710940448436759	\\
0.201041666666667	0.00589716505169235	\\
0.201375	-0.00178372882254507	\\
0.201604166666667	0.00257980852126138	\\
0.201833333333333	0.00832569699358032	\\
0.202208333333333	0.00684046828905593	\\
0.202458333333333	-0.000658989036082858	\\
0.203145833333333	-0.00107860574638607	\\
0.2033125	0.00464952047786937	\\
0.203541666666667	0.00729990093149979	\\
0.203875	-9.33408443586359e-05	\\
0.2039375	0.0047982936472124	\\
0.204083333333333	-0.00301611456700357	\\
0.204479166666667	0.000266552025607325	\\
0.204729166666667	0.00513541758823521	\\
0.205270833333333	0.000278711372970974	\\
0.205625	0.00482142027610166	\\
0.205708333333333	0.00511431754780745	\\
0.206166666666667	-0.00454318573216028	\\
0.206270833333333	-0.0032612088956796	\\
0.206604166666667	0.00237166910832798	\\
0.207	0.00308835545450847	\\
0.207354166666667	-0.00120830551395557	\\
0.207666666666667	-0.00392246286344289	\\
0.207916666666667	0.00342464483028948	\\
0.208270833333333	-0.00229477903661746	\\
0.208520833333333	0.00577938624479657	\\
0.208625	0.00685930334140039	\\
0.209	-0.00435888815711394	\\
0.209104166666667	-0.00334978139862585	\\
0.209583333333333	0.00287783176736411	\\
0.209729166666667	-0.000518679653822574	\\
0.210104166666667	0.00710964287158333	\\
0.2103125	0.00711441123252143	\\
0.210770833333333	-0.00633680888444133	\\
0.210875	-0.005643726035089	\\
0.211333333333333	0.00612151698720709	\\
0.211520833333333	0.00742006388111349	\\
0.211979166666667	-0.00311911135094078	\\
0.212	-0.00367212329399536	\\
0.212520833333333	0.0028051141489982	\\
0.212604166666667	0.00338065665177112	\\
0.212958333333333	-0.000449895889801155	\\
0.213229166666667	0.00306654011023966	\\
0.2136875	-0.00428581291643582	\\
0.213770833333333	-0.00415980869937016	\\
0.214229166666667	0.00291788612315713	\\
0.2143125	0.00329983271058154	\\
0.214645833333333	-0.00306391748517854	\\
0.214854166666667	-0.0014041664155684	\\
0.215229166666667	0.00619030066923187	\\
0.2154375	0.000149011646954023	\\
0.215875	-0.0101162207589311	\\
0.216	-0.00752306079481002	\\
0.216416666666667	0.00188505664664262	\\
0.216604166666667	0.00103843219415012	\\
0.217	-0.00470221096986734	\\
0.2171875	-0.00326216259929879	\\
0.217416666666667	0.000203371044136702	\\
0.217770833333333	-0.0032851699912726	\\
0.218125	0.00482904967080344	\\
0.218333333333333	0.00164091606657735	\\
0.218791666666667	-0.0076192626664664	\\
0.218895833333333	-0.00654673657685123	\\
0.219395833333333	0.00567996564876694	\\
0.219479166666667	0.00678765851725416	\\
0.219833333333333	-0.00312900579930897	\\
0.220083333333333	0.00347018285575018	\\
0.220583333333333	6.40153692188505e-05	\\
0.220770833333333	-0.00215542343522657	\\
0.221125	0.00392198603279326	\\
0.221375	0.000703692452702853	\\
0.2216875	0.00269615677395052	\\
0.222083333333333	0.00636160443139033	\\
0.222333333333333	0.000620722797521012	\\
0.222625	-0.000460028705418836	\\
0.222895833333333	0.00533318580642117	\\
0.2230625	0.00897753334550089	\\
0.2235	0.000110745378719912	\\
0.223625	-0.00119531168189724	\\
0.223875	0.00227165241932425	\\
0.224125	-0.00063931951729046	\\
0.224479166666667	0.00532245691925937	\\
0.224875	0.0029828551348885	\\
0.225229166666667	-0.00225758582720914	\\
0.225354166666667	-0.00421118788392505	\\
0.225625	0.000241279634821012	\\
0.225875	-0.00105929397022919	\\
0.226375	0.000642538050584562	\\
0.226833333333333	-0.00131809733396437	\\
0.226875	0.00360429323791323	\\
0.227229166666667	-0.0020291808409354	\\
0.227479166666667	0.00461232719410987	\\
0.227583333333333	0.00475132523962429	\\
0.2278125	0.00127398984929528	\\
0.2283125	0.00243186982822863	\\
0.228583333333333	0.00695383632751145	\\
0.2286875	0.00613832547902859	\\
0.229229166666667	0.00155639667633523	\\
0.229333333333333	0.00175344963740542	\\
0.229583333333333	0.00752592175697941	\\
0.229979166666667	0.00804805849182344	\\
0.230229166666667	0.000640392404676504	\\
0.230479166666667	0.00516092840962301	\\
0.230520833333333	0.00204491640980109	\\
0.230979166666667	0.00359952494943627	\\
0.231333333333333	-0.00289571322136339	\\
0.231625	-0.00300300157493893	\\
0.231875	0.00334203278799805	\\
0.232125	-0.00217103992747525	\\
0.232479166666667	0.0035543445960684	\\
0.232958333333333	0.0027034285864147	\\
0.233208333333333	-0.00644612388475707	\\
0.2333125	-0.0062490709173062	\\
0.233854166666667	0.00178229826735787	\\
0.234125	-0.0014624597649231	\\
0.234375	0.00382411522087978	\\
0.234479166666667	0.00398063702073159	\\
0.234729166666667	-0.00292182006330677	\\
0.235020833333333	-0.00212180640824045	\\
0.235270833333333	0.00349819705718346	\\
0.235625	0.00135767476763249	\\
0.236104166666667	0.000364422807251685	\\
0.236354166666667	0.00209569954583344	\\
0.236729166666667	-0.00386738826355781	\\
0.2368125	-0.00460696276411454	\\
0.237270833333333	0.00154876729473585	\\
0.237479166666667	0.00311803855851167	\\
0.237833333333333	-0.00311899222981538	\\
0.237979166666667	-0.00115096580407226	\\
0.238333333333333	-0.00414586114146687	\\
0.238520833333333	-0.00320398844039005	\\
0.239	0.000890970309527006	\\
0.239208333333333	0.000903844956425814	\\
0.239604166666667	-0.00319659745549927	\\
0.239708333333333	-0.00322687663124555	\\
0.239958333333333	0.00222611452346655	\\
0.24025	-0.000439882321529694	\\
0.240625	-0.00414037752119611	\\
0.2408125	-0.00297176873033322	\\
0.241166666666667	0.00126636042136852	\\
0.241354166666667	0.000244379052901422	\\
0.241729166666667	-0.00239431885970021	\\
0.242125	-0.00229489832736363	\\
0.242354166666667	0.00117528451892213	\\
0.242541666666667	-0.000412225763852803	\\
0.242791666666667	-0.00220847154105286	\\
0.2431875	-0.00263965158330848	\\
0.243645833333333	0.000949859709450607	\\
0.2436875	-0.00115942969787852	\\
0.244166666666667	0.00364279790369437	\\
0.24425	0.00368773980507342	\\
0.2445	0.00105869780675505	\\
0.244854166666667	0.00416922617107218	\\
0.245	0.00103914750991407	\\
0.245770833333333	0.00275123149830847	\\
0.2459375	0.0039136414350196	\\
0.246041666666667	0.00446915679572157	\\
0.246479166666667	0.00182521364055788	\\
0.246645833333333	0.00268507035470122	\\
0.247	0.000681281180348492	\\
0.247395833333333	0.00112307085385055	\\
0.247645833333333	0.00218319919440546	\\
0.24775	0.00234270124133218	\\
0.248270833333333	0.00106835377113157	\\
0.248291666666667	0.00113654149227216	\\
0.248770833333333	0.000568747595082186	\\
0.248833333333333	0.000854253871153787	\\
0.249375	7.77244690368661e-05	\\
0.249979166666667	0.000136137023218907	\\
};
\addplot [color=red,mark size=1.7pt,only marks,mark=*,mark options={solid},forget plot]
  table[row sep=crcr]{
0.0534	0.1286	\\
};
\end{axis}
\end{tikzpicture}%
	\caption{Estimated impulse response from microphones 1 with the beacon in the middle, the red dot denotes the detected peak.}
	\label{fig:impulse-with-peak}
\end{figure}
If a neighbor would use the same code the peaks would overlap and the measured TDOA would be invalid.
So the codes need to have a low cross correlation, this is most important during the filter stage in the ``receiver''.
\section{Implementation in C\#}

To get the TDOA data out of the recorded audio we need to filter the signal and find the local maximums.
We went for an implementation with a circulant matrix, because in C\# a convolution is not that fast and because we use the Intel MKL through MathNet.Numerics the matrix multiplication is very fast.
The only downside is that the generation of the matrix uses a lot of cpu time.
We're looking to speed this up by using a different implementation of the toep function.
The resulting matrix also takes in a lot of memory.
After considering implementing the matrix multiplication on the gpu, we decided that this was not yet necessary, the CPU load when running the ASIOtest app (Appendix Section \ref{appsec:ASIOTest}) is about 50\%.
Maybe later when the the ``sampling-rate'' (the repetition rate of the beacon should be matched by the frequency we do measurements) needs to go up we'll transfer the matrix to the GPU memory once and then keep using it for multiplications.
The provided PC has a NVidia Quadro so we can take full advantage of CUDA and CUDA is also implemented in MathNet.Numerics.
The matrix is generated on a worker thread so to not freeze the UI.
The actual function call is located at line 60 in file LocationSystem/Localizer.cs (Listing \ref{lst:LocationSystem-Localizer.cs}).
That beacon signal is reversed.
\\ \\
The actual filtering is done in performMeasurement() (line 93) method in the LocationSystem/Localizer.cs (Listing \ref{lst:LocationSystem-Localizer.cs}) file.
The responses matrix is made by taking the input samples of all the channels as the columns. 
This makes the filtering easy (namely one matrix multiplication).
\\ \\
After the filtering we need to do the peak detection. Peak detection also done in the performMeasurement() (line 93) method.
The actual peak detection code (line 114 to 133) is an implementation of the algorithm proposed in the reader.
So we first find the highest peak in the first channel and when we look in a smaller window of length $F_s/5$ so one period.
We may be able to make this window a little bit smaller.
After we find the relative delays we pass this info on to the actual localizer described in Chapter\ref{ch:labday5}.

\end{document}