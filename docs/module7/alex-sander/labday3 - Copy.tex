\documentclass[final]{scrreprt} %scrreprt of scrartcl
% Include all project wide packages here.
\usepackage{fullpage}
\usepackage{polyglossia}
\setmainlanguage{english}
\usepackage{csquotes}
\usepackage{graphicx}
\usepackage{epstopdf}
\usepackage{pdfpages}
\usepackage{caption}
\usepackage[list=true]{subcaption}
\usepackage{float}
\usepackage{standalone}
\usepackage{import}
\usepackage{tocloft}
\usepackage{wrapfig}
\usepackage{authblk}
\usepackage{array}
\usepackage{booktabs}
\usepackage[toc,page,title,titletoc]{appendix}
\usepackage{xunicode}
\usepackage{fontspec}
\usepackage{pgfplots}
\usepackage{SIunits}
\usepackage{units}
\pgfplotsset{compat=newest}
\pgfplotsset{plot coordinates/math parser=false}
\newlength\figureheight 
\newlength\figurewidth
\usepackage{amsmath}
\usepackage{mathtools}
\usepackage{unicode-math}
\usepackage[
    backend=bibtexu,
	texencoding=utf8,
bibencoding=utf8,
    style=ieee,
    sortlocale=en_US,
    language=auto
]{biblatex}
\usepackage{listings}
\newcommand{\includecode}[3][c]{\lstinputlisting[caption=#2, escapechar=, style=#1]{#3}}
\newcommand{\superscript}[1]{\ensuremath{^{\textrm{#1}}}}
\newcommand{\subscript}[1]{\ensuremath{_{\textrm{#1}}}}


\newcommand{\chapternumber}{\thechapter}
\renewcommand{\appendixname}{Bijlage}
\renewcommand{\appendixtocname}{Bijlagen}
\renewcommand{\appendixpagename}{Bijlagen}

\usepackage[hidelinks]{hyperref} %<--------ALTIJD ALS LAATSTE

\renewcommand{\familydefault}{\sfdefault}

\setmainfont[Ligatures=TeX]{Myriad Pro}
\setmathfont{Asana Math}
\setmonofont{Lucida Console}

\usepackage{titlesec, blindtext, color}
\definecolor{gray75}{gray}{0.75}
\newcommand{\hsp}{\hspace{20pt}}
\titleformat{\chapter}[hang]{\Huge\bfseries}{\chapternumber\hsp\textcolor{gray75}{|}\hsp}{0pt}{\Huge\bfseries}
\renewcommand{\familydefault}{\sfdefault}
\renewcommand{\arraystretch}{1.2}
\setlength\parindent{0pt}

%For code listings
\definecolor{black}{rgb}{0,0,0}
\definecolor{browntags}{rgb}{0.65,0.1,0.1}
\definecolor{bluestrings}{rgb}{0,0,1}
\definecolor{graycomments}{rgb}{0.4,0.4,0.4}
\definecolor{redkeywords}{rgb}{1,0,0}
\definecolor{bluekeywords}{rgb}{0.13,0.13,0.8}
\definecolor{greencomments}{rgb}{0,0.5,0}
\definecolor{redstrings}{rgb}{0.9,0,0}
\definecolor{purpleidentifiers}{rgb}{0.01,0,0.01}


\lstdefinestyle{csharp}{
language=[Sharp]C,
showspaces=false,
showtabs=false,
breaklines=true,
showstringspaces=false,
breakatwhitespace=true,
escapeinside={(*@}{@*)},
columns=fullflexible,
commentstyle=\color{greencomments},
keywordstyle=\color{bluekeywords}\bfseries,
stringstyle=\color{redstrings},
identifierstyle=\color{purpleidentifiers},
basicstyle=\ttfamily\small}

\lstdefinestyle{c}{
language=C,
showspaces=false,
showtabs=false,
breaklines=true,
showstringspaces=false,
breakatwhitespace=true,
escapeinside={(*@}{@*)},
columns=fullflexible,
commentstyle=\color{greencomments},
keywordstyle=\color{bluekeywords}\bfseries,
stringstyle=\color{redstrings},
identifierstyle=\color{purpleidentifiers},
}

\lstdefinestyle{matlab}{
language=Matlab,
showspaces=false,
showtabs=false,
breaklines=true,
showstringspaces=false,
breakatwhitespace=true,
escapeinside={(*@}{@*)},
columns=fullflexible,
commentstyle=\color{greencomments},
keywordstyle=\color{bluekeywords}\bfseries,
stringstyle=\color{redstrings},
identifierstyle=\color{purpleidentifiers}
}

\lstdefinestyle{vhdl}{
language=VHDL,
showspaces=false,
showtabs=false,
breaklines=true,
showstringspaces=false,
breakatwhitespace=true,
escapeinside={(*@}{@*)},
columns=fullflexible,
commentstyle=\color{greencomments},
keywordstyle=\color{bluekeywords}\bfseries,
stringstyle=\color{redstrings},
identifierstyle=\color{purpleidentifiers}
}

\lstdefinestyle{xaml}{
language=XML,
showspaces=false,
showtabs=false,
breaklines=true,
showstringspaces=false,
breakatwhitespace=true,
escapeinside={(*@}{@*)},
columns=fullflexible,
commentstyle=\color{greencomments},
keywordstyle=\color{redkeywords},
stringstyle=\color{bluestrings},
tagstyle=\color{browntags},
morestring=[b]",
  morecomment=[s]{<?}{?>},
  morekeywords={xmlns,version,typex:AsyncRecords,x:Arguments,x:Boolean,x:Byte,x:Char,x:Class,x:ClassAttributes,x:ClassModifier,x:Code,x:ConnectionId,x:Decimal,x:Double,x:FactoryMethod,x:FieldModifier,x:Int16,x:Int32,x:Int64,x:Key,x:Members,x:Name,x:Object,x:Property,x:Shared,x:Single,x:String,x:Subclass,x:SynchronousMode,x:TimeSpan,x:TypeArguments,x:Uid,x:Uri,x:XData,Grid.Column,Grid.ColumnSpan,Click,ClipToBounds,Content,DropDownOpened,FontSize,Foreground,Header,Height,HorizontalAlignment,HorizontalContentAlignment,IsCancel,IsDefault,IsEnabled,IsSelected,Margin,MinHeight,MinWidth,Padding,SnapsToDevicePixels,Target,TextWrapping,Title,VerticalAlignment,VerticalContentAlignment,Width,WindowStartupLocation,Binding,Mode,OneWay,xmlns:x}
}

%defaults
\lstset{
basicstyle=\ttfamily\small,
extendedchars=false,
numbers=left,
numberstyle=\ttfamily\tiny,
stepnumber=1,
tabsize=4,
numbersep=5pt
}
\addbibresource{../../library/bibliography.bib}

\begin{document}

\section{Labday 3: audio channel measurements}
Using the default values of Rx\_TX = 22.050 and Fs\_RX = 22.050, the inpulse response of Figure \ref{fig:response_1} was found.
The time reponse is not an ideal delta pulse since the loudspeaker, the microphone and the channel act as filters which distort the signal.
The rising edge is rather steep, but the falling edge oscillates towards the amplitude of zero.
The spectrum of this signal is shown below the time-domain impulse response.
The distortion is also visible in this spectrum since a lot of noise peaks are also present.
In the ideal case of flat spectral transfer functions, the following spectrum would occur:

\begin{equation}
	\delta(t) \xrightarrow{\mathfrak{F}} 1
\end{equation}
\\ \\
The plots of Figure \ref{fig:response_2} and \ref{fig:response_3} show the effect of varying the Fs\_RX and Fs\_TX.
When using a high Fs\_RX compared to the Fs\_TX, the spectrum will become flatter and thus more ideal as seen in Figure \ref{fig:response_2}.
Vice versa, when the Fs\_TX is high compared to the Fs\_RX, the spectrum will become less flat and thus less ideal as seen in Figure \ref{fig:response_3}.
This can be explained by looking at the transfer functions of the loudspeaker and the microphone, which are both non-linear bandpass filters causing high frequencies to be damped drastically.
\\ \\
When the Fs\_RX is relatively low compared to the Fs\_TX, aliasing might occur since the delta function contains all frequencies ideally.
Using the default setting of Fs\_RX of 22.050 Hz and varying the Fs\_TX, the spectums were analysed for aliasing.
However, no aliasing seemed to occur, even at a Fs\_TX of 176.4 kHz, which is the highest supported frequency, aliasing could not be found as seen in Figure \ref{fig:highest_freq}.
It seems that the sound card removes these aliasings from the input signal.
This information is good to know for when analysing the data from the audio beacon later on.

\begin{figure}[H]
	\centering
	\setlength\figureheight{4cm}
    	\setlength\figurewidth{0.8\linewidth}
	% This file was created by matlab2tikz v0.4.6 running on MATLAB 8.2.
% Copyright (c) 2008--2014, Nico Schlömer <nico.schloemer@gmail.com>
% All rights reserved.
% Minimal pgfplots version: 1.3
% 
% The latest updates can be retrieved from
%   http://www.mathworks.com/matlabcentral/fileexchange/22022-matlab2tikz
% where you can also make suggestions and rate matlab2tikz.
% 
\begin{tikzpicture}

\begin{axis}[%
width=\figurewidth,
height=\figureheight,
scale only axis,
xmin=0,
xmax=1,
ymin=-1,
ymax=0.5,
name=plot1
]
\addplot [color=blue,solid,forget plot]
  table[row sep=crcr]{
0	0.00213623046875	\\
4.43911750344032e-05	-0.003143310546875	\\
8.87823500688063e-05	-0.003753662109375	\\
0.000133173525103209	-0.002960205078125	\\
0.000177564700137613	-0.003448486328125	\\
0.000221955875172016	-0.002471923828125	\\
0.000266347050206419	-0.0023193359375	\\
0.000310738225240822	-0.002410888671875	\\
0.000355129400275225	-0.0020751953125	\\
0.000399520575309628	-0.001708984375	\\
0.000443911750344032	-0.001434326171875	\\
0.000488302925378435	-0.001312255859375	\\
0.000532694100412838	-0.001800537109375	\\
0.000577085275447241	-0.00225830078125	\\
0.000621476450481644	-0.002349853515625	\\
0.000665867625516047	-0.002044677734375	\\
0.000710258800550451	-0.001678466796875	\\
0.000754649975584854	-0.00213623046875	\\
0.000799041150619257	-0.002166748046875	\\
0.00084343232565366	-0.002349853515625	\\
0.000887823500688063	-0.002471923828125	\\
0.000932214675722466	-0.002227783203125	\\
0.000976605850756869	-0.002197265625	\\
0.00102099702579127	-0.00250244140625	\\
0.00106538820082568	-0.003082275390625	\\
0.00110977937586008	-0.00372314453125	\\
0.00115417055089448	-0.0037841796875	\\
0.00119856172592889	-0.003814697265625	\\
0.00124295290096329	-0.003875732421875	\\
0.00128734407599769	-0.00384521484375	\\
0.00133173525103209	-0.003631591796875	\\
0.0013761264260665	-0.003265380859375	\\
0.0014205176011009	-0.00360107421875	\\
0.0014649087761353	-0.003997802734375	\\
0.00150929995116971	-0.003448486328125	\\
0.00155369112620411	-0.002838134765625	\\
0.00159808230123851	-0.00286865234375	\\
0.00164247347627292	-0.0030517578125	\\
0.00168686465130732	-0.003143310546875	\\
0.00173125582634172	-0.0029296875	\\
0.00177564700137613	-0.00262451171875	\\
0.00182003817641053	-0.002716064453125	\\
0.00186442935144493	-0.00250244140625	\\
0.00190882052647934	-0.00250244140625	\\
0.00195321170151374	-0.002349853515625	\\
0.00199760287654814	-0.00164794921875	\\
0.00204199405158255	-0.00128173828125	\\
0.00208638522661695	-0.001068115234375	\\
0.00213077640165135	-0.00103759765625	\\
0.00217516757668575	-0.000518798828125	\\
0.00221955875172016	-0.0003662109375	\\
0.00226394992675456	-0.00048828125	\\
0.00230834110178896	-0.000274658203125	\\
0.00235273227682337	-0.0001220703125	\\
0.00239712345185777	9.1552734375e-05	\\
0.00244151462689217	-0.00018310546875	\\
0.00248590580192658	-9.1552734375e-05	\\
0.00253029697696098	0.00042724609375	\\
0.00257468815199538	0.00018310546875	\\
0.00261907932702979	3.0517578125e-05	\\
0.00266347050206419	0	\\
0.00270786167709859	3.0517578125e-05	\\
0.002752252852133	0.00030517578125	\\
0.0027966440271674	0.000457763671875	\\
0.0028410352022018	0.00042724609375	\\
0.00288542637723621	0.000274658203125	\\
0.00292981755227061	0.000274658203125	\\
0.00297420872730501	-0.00048828125	\\
0.00301859990233941	-0.000946044921875	\\
0.00306299107737382	-0.00128173828125	\\
0.00310738225240822	-0.001708984375	\\
0.00315177342744262	-0.001617431640625	\\
0.00319616460247703	-0.001739501953125	\\
0.00324055577751143	-0.0013427734375	\\
0.00328494695254583	-0.001190185546875	\\
0.00332933812758024	-0.001495361328125	\\
0.00337372930261464	-0.001220703125	\\
0.00341812047764904	-0.001312255859375	\\
0.00346251165268345	-0.001068115234375	\\
0.00350690282771785	-0.000732421875	\\
0.00355129400275225	-0.001373291015625	\\
0.00359568517778666	-0.001556396484375	\\
0.00364007635282106	-0.00164794921875	\\
0.00368446752785546	-0.001708984375	\\
0.00372885870288987	-0.001617431640625	\\
0.00377324987792427	-0.001495361328125	\\
0.00381764105295867	-0.00177001953125	\\
0.00386203222799308	-0.00164794921875	\\
0.00390642340302748	-0.0015869140625	\\
0.00395081457806188	-0.001617431640625	\\
0.00399520575309628	-0.00152587890625	\\
0.00403959692813069	-0.001312255859375	\\
0.00408398810316509	-0.0009765625	\\
0.00412837927819949	-0.00103759765625	\\
0.0041727704532339	-0.0006103515625	\\
0.0042171616282683	-0.000274658203125	\\
0.0042615528033027	-0.000396728515625	\\
0.00430594397833711	-0.0003662109375	\\
0.00435033515337151	-0.0001220703125	\\
0.00439472632840591	0.000244140625	\\
0.00443911750344032	0.000518798828125	\\
0.00448350867847472	0.0009765625	\\
0.00452789985350912	0.000762939453125	\\
0.00457229102854353	0.00048828125	\\
0.00461668220357793	0.000518798828125	\\
0.00466107337861233	0.0003662109375	\\
0.00470546455364673	0.000213623046875	\\
0.00474985572868114	0.000213623046875	\\
0.00479424690371554	-9.1552734375e-05	\\
0.00483863807874994	-6.103515625e-05	\\
0.00488302925378435	0.0001220703125	\\
0.00492742042881875	-3.0517578125e-05	\\
0.00497181160385315	-0.00054931640625	\\
0.00501620277888756	-0.000274658203125	\\
0.00506059395392196	-0.000274658203125	\\
0.00510498512895636	-0.000762939453125	\\
0.00514937630399077	-0.000579833984375	\\
0.00519376747902517	-0.00054931640625	\\
0.00523815865405957	-0.000946044921875	\\
0.00528254982909398	-0.001190185546875	\\
0.00532694100412838	-0.0013427734375	\\
0.00537133217916278	-0.00128173828125	\\
0.00541572335419719	-0.0018310546875	\\
0.00546011452923159	-0.001983642578125	\\
0.00550450570426599	-0.00201416015625	\\
0.00554889687930039	-0.001708984375	\\
0.0055932880543348	-0.00164794921875	\\
0.0056376792293692	-0.002593994140625	\\
0.0056820704044036	-0.0028076171875	\\
0.00572646157943801	-0.00250244140625	\\
0.00577085275447241	-0.00262451171875	\\
0.00581524392950681	-0.0028076171875	\\
0.00585963510454122	-0.00238037109375	\\
0.00590402627957562	-0.0023193359375	\\
0.00594841745461002	-0.002777099609375	\\
0.00599280862964443	-0.002716064453125	\\
0.00603719980467883	-0.00286865234375	\\
0.00608159097971323	-0.00274658203125	\\
0.00612598215474764	-0.002288818359375	\\
0.00617037332978204	-0.0023193359375	\\
0.00621476450481644	-0.00238037109375	\\
0.00625915567985085	-0.002410888671875	\\
0.00630354685488525	-0.001739501953125	\\
0.00634793802991965	-0.00152587890625	\\
0.00639232920495406	-0.001007080078125	\\
0.00643672037998846	-0.000701904296875	\\
0.00648111155502286	-0.0003662109375	\\
0.00652550273005726	-3.0517578125e-05	\\
0.00656989390509167	-0.0001220703125	\\
0.00661428508012607	-0.0001220703125	\\
0.00665867625516047	0.0003662109375	\\
0.00670306743019488	0.000946044921875	\\
0.00674745860522928	0.00067138671875	\\
0.00679184978026368	0.00030517578125	\\
0.00683624095529809	0.0003662109375	\\
0.00688063213033249	9.1552734375e-05	\\
0.00692502330536689	9.1552734375e-05	\\
0.0069694144804013	0.0001220703125	\\
0.0070138056554357	-3.0517578125e-05	\\
0.0070581968304701	0.000152587890625	\\
0.00710258800550451	-3.0517578125e-05	\\
0.00714697918053891	-0.0001220703125	\\
0.00719137035557331	6.103515625e-05	\\
0.00723576153060772	-0.000274658203125	\\
0.00728015270564212	-0.000396728515625	\\
0.00732454388067652	-0.000213623046875	\\
0.00736893505571092	-0.000762939453125	\\
0.00741332623074533	-0.001373291015625	\\
0.00745771740577973	-0.001068115234375	\\
0.00750210858081413	-0.001556396484375	\\
0.00754649975584854	-0.00213623046875	\\
0.00759089093088294	-0.00250244140625	\\
0.00763528210591734	-0.00244140625	\\
0.00767967328095175	-0.00201416015625	\\
0.00772406445598615	-0.00213623046875	\\
0.00776845563102055	-0.0023193359375	\\
0.00781284680605496	-0.002410888671875	\\
0.00785723798108936	-0.002227783203125	\\
0.00790162915612376	-0.001861572265625	\\
0.00794602033115816	-0.00201416015625	\\
0.00799041150619257	-0.00225830078125	\\
0.00803480268122697	-0.00189208984375	\\
0.00807919385626137	-0.001953125	\\
0.00812358503129578	-0.002197265625	\\
0.00816797620633018	-0.002105712890625	\\
0.00821236738136458	-0.002532958984375	\\
0.00825675855639899	-0.00286865234375	\\
0.00830114973143339	-0.002349853515625	\\
0.00834554090646779	-0.001983642578125	\\
0.0083899320815022	-0.0020751953125	\\
0.0084343232565366	-0.001953125	\\
0.008478714431571	-0.001434326171875	\\
0.00852310560660541	-0.001861572265625	\\
0.00856749678163981	-0.001739501953125	\\
0.00861188795667421	-0.00146484375	\\
0.00865627913170862	-0.001922607421875	\\
0.00870067030674302	-0.00189208984375	\\
0.00874506148177742	-0.002105712890625	\\
0.00878945265681183	-0.002166748046875	\\
0.00883384383184623	-0.001922607421875	\\
0.00887823500688063	-0.002197265625	\\
0.00892262618191503	-0.0023193359375	\\
0.00896701735694944	-0.0025634765625	\\
0.00901140853198384	-0.002410888671875	\\
0.00905579970701824	-0.002349853515625	\\
0.00910019088205265	-0.002655029296875	\\
0.00914458205708705	-0.002166748046875	\\
0.00918897323212145	-0.002227783203125	\\
0.00923336440715586	-0.0025634765625	\\
0.00927775558219026	-0.002655029296875	\\
0.00932214675722466	-0.002532958984375	\\
0.00936653793225907	-0.0025634765625	\\
0.00941092910729347	-0.00299072265625	\\
0.00945532028232787	-0.0029296875	\\
0.00949971145736228	-0.003326416015625	\\
0.00954410263239668	-0.003021240234375	\\
0.00958849380743108	-0.002960205078125	\\
0.00963288498246549	-0.003387451171875	\\
0.00967727615749989	-0.00323486328125	\\
0.00972166733253429	-0.00311279296875	\\
0.00976605850756869	-0.002960205078125	\\
0.0098104496826031	-0.0030517578125	\\
0.0098548408576375	-0.003265380859375	\\
0.0098992320326719	-0.00323486328125	\\
0.00994362320770631	-0.003326416015625	\\
0.00998801438274071	-0.0029296875	\\
0.0100324055577751	-0.002685546875	\\
0.0100767967328095	-0.002899169921875	\\
0.0101211879078439	-0.002471923828125	\\
0.0101655790828783	-0.002685546875	\\
0.0102099702579127	-0.002685546875	\\
0.0102543614329471	-0.002532958984375	\\
0.0102987526079815	-0.002288818359375	\\
0.0103431437830159	-0.00152587890625	\\
0.0103875349580503	-0.00146484375	\\
0.0104319261330847	-0.001983642578125	\\
0.0104763173081191	-0.001800537109375	\\
0.0105207084831535	-0.001708984375	\\
0.010565099658188	-0.001922607421875	\\
0.0106094908332224	-0.001922607421875	\\
0.0106538820082568	-0.00152587890625	\\
0.0106982731832912	-0.001617431640625	\\
0.0107426643583256	-0.0015869140625	\\
0.01078705553336	-0.00091552734375	\\
0.0108314467083944	-0.001556396484375	\\
0.0108758378834288	-0.001220703125	\\
0.0109202290584632	-0.001312255859375	\\
0.0109646202334976	-0.00189208984375	\\
0.011009011408532	-0.00164794921875	\\
0.0110534025835664	-0.0015869140625	\\
0.0110977937586008	-0.00177001953125	\\
0.0111421849336352	-0.00201416015625	\\
0.0111865761086696	-0.001617431640625	\\
0.011230967283704	-0.00146484375	\\
0.0112753584587384	-0.001739501953125	\\
0.0113197496337728	-0.001953125	\\
0.0113641408088072	-0.002105712890625	\\
0.0114085319838416	-0.00177001953125	\\
0.011452923158876	-0.001739501953125	\\
0.0114973143339104	-0.00189208984375	\\
0.0115417055089448	-0.002044677734375	\\
0.0115860966839792	-0.001800537109375	\\
0.0116304878590136	-0.001678466796875	\\
0.011674879034048	-0.00152587890625	\\
0.0117192702090824	-0.001373291015625	\\
0.0117636613841168	-0.001373291015625	\\
0.0118080525591512	-0.0010986328125	\\
0.0118524437341856	-0.000946044921875	\\
0.01189683490922	-0.00146484375	\\
0.0119412260842545	-0.00128173828125	\\
0.0119856172592889	-0.001068115234375	\\
0.0120300084343233	-0.001129150390625	\\
0.0120743996093577	-0.001007080078125	\\
0.0121187907843921	-0.001312255859375	\\
0.0121631819594265	-0.001312255859375	\\
0.0122075731344609	-0.00091552734375	\\
0.0122519643094953	-0.0008544921875	\\
0.0122963554845297	-0.000885009765625	\\
0.0123407466595641	-0.00079345703125	\\
0.0123851378345985	-0.00054931640625	\\
0.0124295290096329	-0.000335693359375	\\
0.0124739201846673	-0.000762939453125	\\
0.0125183113597017	-0.00054931640625	\\
0.0125627025347361	-0.000701904296875	\\
0.0126070937097705	-0.0009765625	\\
0.0126514848848049	-0.00103759765625	\\
0.0126958760598393	-0.000946044921875	\\
0.0127402672348737	-0.000732421875	\\
0.0127846584099081	-0.001190185546875	\\
0.0128290495849425	-0.001251220703125	\\
0.0128734407599769	-0.001800537109375	\\
0.0129178319350113	-0.001617431640625	\\
0.0129622231100457	-0.001007080078125	\\
0.0130066142850801	-0.001739501953125	\\
0.0130510054601145	-0.001953125	\\
0.0130953966351489	-0.001800537109375	\\
0.0131397878101833	-0.001708984375	\\
0.0131841789852177	-0.0018310546875	\\
0.0132285701602521	-0.002197265625	\\
0.0132729613352865	-0.002105712890625	\\
0.0133173525103209	-0.001953125	\\
0.0133617436853554	-0.002288818359375	\\
0.0134061348603898	-0.002288818359375	\\
0.0134505260354242	-0.00201416015625	\\
0.0134949172104586	-0.002227783203125	\\
0.013539308385493	-0.002471923828125	\\
0.0135836995605274	-0.00262451171875	\\
0.0136280907355618	-0.002777099609375	\\
0.0136724819105962	-0.002410888671875	\\
0.0137168730856306	-0.002105712890625	\\
0.013761264260665	-0.002044677734375	\\
0.0138056554356994	-0.002166748046875	\\
0.0138500466107338	-0.00213623046875	\\
0.0138944377857682	-0.00201416015625	\\
0.0139388289608026	-0.001922607421875	\\
0.013983220135837	-0.001983642578125	\\
0.0140276113108714	-0.002044677734375	\\
0.0140720024859058	-0.0015869140625	\\
0.0141163936609402	-0.00140380859375	\\
0.0141607848359746	-0.001708984375	\\
0.014205176011009	-0.001922607421875	\\
0.0142495671860434	-0.001953125	\\
0.0142939583610778	-0.001739501953125	\\
0.0143383495361122	-0.00164794921875	\\
0.0143827407111466	-0.00164794921875	\\
0.014427131886181	-0.001434326171875	\\
0.0144715230612154	-0.0018310546875	\\
0.0145159142362498	-0.00189208984375	\\
0.0145603054112842	-0.001739501953125	\\
0.0146046965863186	-0.001617431640625	\\
0.014649087761353	-0.00177001953125	\\
0.0146934789363874	-0.001708984375	\\
0.0147378701114218	-0.001922607421875	\\
0.0147822612864563	-0.001983642578125	\\
0.0148266524614907	-0.001373291015625	\\
0.0148710436365251	-0.001617431640625	\\
0.0149154348115595	-0.002166748046875	\\
0.0149598259865939	-0.00225830078125	\\
0.0150042171616283	-0.002197265625	\\
0.0150486083366627	-0.00238037109375	\\
0.0150929995116971	-0.002593994140625	\\
0.0151373906867315	-0.002685546875	\\
0.0151817818617659	-0.002532958984375	\\
0.0152261730368003	-0.002838134765625	\\
0.0152705642118347	-0.003082275390625	\\
0.0153149553868691	-0.00262451171875	\\
0.0153593465619035	-0.00299072265625	\\
0.0154037377369379	-0.0032958984375	\\
0.0154481289119723	-0.003082275390625	\\
0.0154925200870067	-0.0030517578125	\\
0.0155369112620411	-0.0030517578125	\\
0.0155813024370755	-0.0035400390625	\\
0.0156256936121099	-0.003631591796875	\\
0.0156700847871443	-0.003570556640625	\\
0.0157144759621787	-0.003448486328125	\\
0.0157588671372131	-0.00384521484375	\\
0.0158032583122475	-0.00408935546875	\\
0.0158476494872819	-0.00384521484375	\\
0.0158920406623163	-0.004302978515625	\\
0.0159364318373507	-0.00396728515625	\\
0.0159808230123851	-0.00360107421875	\\
0.0160252141874195	-0.003662109375	\\
0.0160696053624539	-0.00372314453125	\\
0.0161139965374883	-0.00347900390625	\\
0.0161583877125227	-0.0032958984375	\\
0.0162027788875572	-0.0035400390625	\\
0.0162471700625916	-0.003143310546875	\\
0.016291561237626	-0.003082275390625	\\
0.0163359524126604	-0.0032958984375	\\
0.0163803435876948	-0.003143310546875	\\
0.0164247347627292	-0.0029296875	\\
0.0164691259377636	-0.002960205078125	\\
0.016513517112798	-0.002685546875	\\
0.0165579082878324	-0.002655029296875	\\
0.0166022994628668	-0.002593994140625	\\
0.0166466906379012	-0.0023193359375	\\
0.0166910818129356	-0.002044677734375	\\
0.01673547298797	-0.0018310546875	\\
0.0167798641630044	-0.0018310546875	\\
0.0168242553380388	-0.0015869140625	\\
0.0168686465130732	-0.001312255859375	\\
0.0169130376881076	-0.0018310546875	\\
0.016957428863142	-0.002227783203125	\\
0.0170018200381764	-0.002044677734375	\\
0.0170462112132108	-0.001953125	\\
0.0170906023882452	-0.001953125	\\
0.0171349935632796	-0.00238037109375	\\
0.017179384738314	-0.002227783203125	\\
0.0172237759133484	-0.002105712890625	\\
0.0172681670883828	-0.002716064453125	\\
0.0173125582634172	-0.002655029296875	\\
0.0173569494384516	-0.0028076171875	\\
0.017401340613486	-0.002960205078125	\\
0.0174457317885204	-0.002716064453125	\\
0.0174901229635548	-0.002685546875	\\
0.0175345141385892	-0.0029296875	\\
0.0175789053136237	-0.00286865234375	\\
0.0176232964886581	-0.002838134765625	\\
0.0176676876636925	-0.002899169921875	\\
0.0177120788387269	-0.002960205078125	\\
0.0177564700137613	-0.00299072265625	\\
0.0178008611887957	-0.002960205078125	\\
0.0178452523638301	-0.003448486328125	\\
0.0178896435388645	-0.00323486328125	\\
0.0179340347138989	-0.003143310546875	\\
0.0179784258889333	-0.00341796875	\\
0.0180228170639677	-0.003143310546875	\\
0.0180672082390021	-0.0029296875	\\
0.0181115994140365	-0.003204345703125	\\
0.0181559905890709	-0.003265380859375	\\
0.0182003817641053	-0.0035400390625	\\
0.0182447729391397	-0.003692626953125	\\
0.0182891641141741	-0.0037841796875	\\
0.0183335552892085	-0.00360107421875	\\
0.0183779464642429	-0.003204345703125	\\
0.0184223376392773	-0.0028076171875	\\
0.0184667288143117	-0.002777099609375	\\
0.0185111199893461	-0.002777099609375	\\
0.0185555111643805	-0.002777099609375	\\
0.0185999023394149	-0.003021240234375	\\
0.0186442935144493	-0.002960205078125	\\
0.0186886846894837	-0.002685546875	\\
0.0187330758645181	-0.002777099609375	\\
0.0187774670395525	-0.0025634765625	\\
0.0188218582145869	-0.00250244140625	\\
0.0188662493896213	-0.002349853515625	\\
0.0189106405646557	-0.00201416015625	\\
0.0189550317396902	-0.001922607421875	\\
0.0189994229147246	-0.0015869140625	\\
0.019043814089759	-0.001953125	\\
0.0190882052647934	-0.001953125	\\
0.0191325964398278	-0.001922607421875	\\
0.0191769876148622	-0.002227783203125	\\
0.0192213787898966	-0.00238037109375	\\
0.019265769964931	-0.0025634765625	\\
0.0193101611399654	-0.002655029296875	\\
0.0193545523149998	-0.0028076171875	\\
0.0193989434900342	-0.002899169921875	\\
0.0194433346650686	-0.003448486328125	\\
0.019487725840103	-0.003448486328125	\\
0.0195321170151374	-0.0037841796875	\\
0.0195765081901718	-0.00408935546875	\\
0.0196208993652062	-0.003936767578125	\\
0.0196652905402406	-0.00439453125	\\
0.019709681715275	-0.004852294921875	\\
0.0197540728903094	-0.00506591796875	\\
0.0197984640653438	-0.005157470703125	\\
0.0198428552403782	-0.005096435546875	\\
0.0198872464154126	-0.00494384765625	\\
0.019931637590447	-0.0050048828125	\\
0.0199760287654814	-0.00567626953125	\\
0.0200204199405158	-0.00555419921875	\\
0.0200648111155502	-0.00531005859375	\\
0.0201092022905846	-0.00543212890625	\\
0.020153593465619	-0.0052490234375	\\
0.0201979846406534	-0.0054931640625	\\
0.0202423758156878	-0.005828857421875	\\
0.0202867669907222	-0.00592041015625	\\
0.0203311581657566	-0.0052490234375	\\
0.0203755493407911	-0.00543212890625	\\
0.0204199405158255	-0.00543212890625	\\
0.0204643316908599	-0.005615234375	\\
0.0205087228658943	-0.005279541015625	\\
0.0205531140409287	-0.0048828125	\\
0.0205975052159631	-0.00518798828125	\\
0.0206418963909975	-0.00506591796875	\\
0.0206862875660319	-0.004486083984375	\\
0.0207306787410663	-0.004638671875	\\
0.0207750699161007	-0.0048828125	\\
0.0208194610911351	-0.003753662109375	\\
0.0208638522661695	-0.00390625	\\
0.0209082434412039	-0.004119873046875	\\
0.0209526346162383	-0.003631591796875	\\
0.0209970257912727	-0.00341796875	\\
0.0210414169663071	-0.00274658203125	\\
0.0210858081413415	-0.002532958984375	\\
0.0211301993163759	-0.002410888671875	\\
0.0211745904914103	-0.002044677734375	\\
0.0212189816664447	-0.001953125	\\
0.0212633728414791	-0.0015869140625	\\
0.0213077640165135	-0.001617431640625	\\
0.0213521551915479	-0.00189208984375	\\
0.0213965463665823	-0.0018310546875	\\
0.0214409375416167	-0.001617431640625	\\
0.0214853287166511	-0.001739501953125	\\
0.0215297198916855	-0.001983642578125	\\
0.0215741110667199	-0.001983642578125	\\
0.0216185022417543	-0.0018310546875	\\
0.0216628934167887	-0.0018310546875	\\
0.0217072845918231	-0.001800537109375	\\
0.0217516757668575	-0.001983642578125	\\
0.021796066941892	-0.001953125	\\
0.0218404581169264	-0.001861572265625	\\
0.0218848492919608	-0.00201416015625	\\
0.0219292404669952	-0.001373291015625	\\
0.0219736316420296	-0.001556396484375	\\
0.022018022817064	-0.001953125	\\
0.0220624139920984	-0.00201416015625	\\
0.0221068051671328	-0.0020751953125	\\
0.0221511963421672	-0.002044677734375	\\
0.0221955875172016	-0.002197265625	\\
0.022239978692236	-0.002288818359375	\\
0.0222843698672704	-0.002532958984375	\\
0.0223287610423048	-0.002655029296875	\\
0.0223731522173392	-0.002685546875	\\
0.0224175433923736	-0.00286865234375	\\
0.022461934567408	-0.002838134765625	\\
0.0225063257424424	-0.002960205078125	\\
0.0225507169174768	-0.0029296875	\\
0.0225951080925112	-0.002716064453125	\\
0.0226394992675456	-0.002838134765625	\\
0.02268389044258	-0.00262451171875	\\
0.0227282816176144	-0.002593994140625	\\
0.0227726727926488	-0.002777099609375	\\
0.0228170639676832	-0.002777099609375	\\
0.0228614551427176	-0.002960205078125	\\
0.022905846317752	-0.002777099609375	\\
0.0229502374927864	-0.002655029296875	\\
0.0229946286678208	-0.0025634765625	\\
0.0230390198428552	-0.003082275390625	\\
0.0230834110178896	-0.003143310546875	\\
0.023127802192924	-0.003021240234375	\\
0.0231721933679584	-0.002777099609375	\\
0.0232165845429929	-0.002655029296875	\\
0.0232609757180273	-0.001953125	\\
0.0233053668930617	-0.001739501953125	\\
0.0233497580680961	-0.001708984375	\\
0.0233941492431305	-0.00164794921875	\\
0.0234385404181649	-0.001373291015625	\\
0.0234829315931993	-0.0015869140625	\\
0.0235273227682337	-0.0018310546875	\\
0.0235717139432681	-0.00146484375	\\
0.0236161051183025	-0.001190185546875	\\
0.0236604962933369	-0.001007080078125	\\
0.0237048874683713	-0.00103759765625	\\
0.0237492786434057	-0.000946044921875	\\
0.0237936698184401	-0.000885009765625	\\
0.0238380609934745	-0.000732421875	\\
0.0238824521685089	-0.000640869140625	\\
0.0239268433435433	-0.00048828125	\\
0.0239712345185777	-0.00079345703125	\\
0.0240156256936121	-0.000732421875	\\
0.0240600168686465	-0.000762939453125	\\
0.0241044080436809	-0.001190185546875	\\
0.0241487992187153	-0.0009765625	\\
0.0241931903937497	-0.001495361328125	\\
0.0242375815687841	-0.001556396484375	\\
0.0242819727438185	-0.001251220703125	\\
0.0243263639188529	-0.001220703125	\\
0.0243707550938873	-0.000946044921875	\\
0.0244151462689217	-0.001312255859375	\\
0.0244595374439561	-0.001251220703125	\\
0.0245039286189905	-0.001251220703125	\\
0.0245483197940249	-0.001220703125	\\
0.0245927109690594	-0.001220703125	\\
0.0246371021440938	-0.001617431640625	\\
0.0246814933191282	-0.00164794921875	\\
0.0247258844941626	-0.00140380859375	\\
0.024770275669197	-0.001556396484375	\\
0.0248146668442314	-0.001434326171875	\\
0.0248590580192658	-0.00152587890625	\\
0.0249034491943002	-0.001678466796875	\\
0.0249478403693346	-0.001556396484375	\\
0.024992231544369	-0.001495361328125	\\
0.0250366227194034	-0.001708984375	\\
0.0250810138944378	-0.00177001953125	\\
0.0251254050694722	-0.00146484375	\\
0.0251697962445066	-0.0018310546875	\\
0.025214187419541	-0.0020751953125	\\
0.0252585785945754	-0.00213623046875	\\
0.0253029697696098	-0.002044677734375	\\
0.0253473609446442	-0.002105712890625	\\
0.0253917521196786	-0.002044677734375	\\
0.025436143294713	-0.00201416015625	\\
0.0254805344697474	-0.0018310546875	\\
0.0255249256447818	-0.001617431640625	\\
0.0255693168198162	-0.002044677734375	\\
0.0256137079948506	-0.002044677734375	\\
0.025658099169885	-0.001861572265625	\\
0.0257024903449194	-0.001953125	\\
0.0257468815199538	-0.001617431640625	\\
0.0257912726949882	-0.00140380859375	\\
0.0258356638700226	-0.001220703125	\\
0.025880055045057	-0.00091552734375	\\
0.0259244462200914	-0.00128173828125	\\
0.0259688373951259	-0.0008544921875	\\
0.0260132285701603	-0.000457763671875	\\
0.0260576197451947	-0.000701904296875	\\
0.0261020109202291	-0.000885009765625	\\
0.0261464020952635	-0.00030517578125	\\
0.0261907932702979	-0.000274658203125	\\
0.0262351844453323	6.103515625e-05	\\
0.0262795756203667	0.00030517578125	\\
0.0263239667954011	0.000396728515625	\\
0.0263683579704355	0.0006103515625	\\
0.0264127491454699	0.000579833984375	\\
0.0264571403205043	0.000518798828125	\\
0.0265015314955387	0.000823974609375	\\
0.0265459226705731	0.000640869140625	\\
0.0265903138456075	0.000640869140625	\\
0.0266347050206419	0.0006103515625	\\
0.0266790961956763	0.0003662109375	\\
0.0267234873707107	0.00048828125	\\
0.0267678785457451	0.000335693359375	\\
0.0268122697207795	-0.000152587890625	\\
0.0268566608958139	-0.000335693359375	\\
0.0269010520708483	-0.000274658203125	\\
0.0269454432458827	-0.00030517578125	\\
0.0269898344209171	-0.000152587890625	\\
0.0270342255959515	-0.000213623046875	\\
0.0270786167709859	-0.000213623046875	\\
0.0271230079460203	-0.000244140625	\\
0.0271673991210547	-0.00030517578125	\\
0.0272117902960891	-0.000274658203125	\\
0.0272561814711235	-0.000579833984375	\\
0.0273005726461579	-0.000762939453125	\\
0.0273449638211923	-0.000946044921875	\\
0.0273893549962268	-0.001190185546875	\\
0.0274337461712612	-0.00164794921875	\\
0.0274781373462956	-0.00189208984375	\\
0.02752252852133	-0.0018310546875	\\
0.0275669196963644	-0.002044677734375	\\
0.0276113108713988	-0.002227783203125	\\
0.0276557020464332	-0.00238037109375	\\
0.0277000932214676	-0.00262451171875	\\
0.027744484396502	-0.00238037109375	\\
0.0277888755715364	-0.002105712890625	\\
0.0278332667465708	-0.001861572265625	\\
0.0278776579216052	-0.0018310546875	\\
0.0279220490966396	-0.001708984375	\\
0.027966440271674	-0.00189208984375	\\
0.0280108314467084	-0.001922607421875	\\
0.0280552226217428	-0.001220703125	\\
0.0280996137967772	-0.00128173828125	\\
0.0281440049718116	-0.001312255859375	\\
0.028188396146846	-0.001434326171875	\\
0.0282327873218804	-0.0009765625	\\
0.0282771784969148	-0.00048828125	\\
0.0283215696719492	-0.0006103515625	\\
0.0283659608469836	-0.000701904296875	\\
0.028410352022018	-0.00054931640625	\\
0.0284547431970524	-0.000274658203125	\\
0.0284991343720868	-0.0001220703125	\\
0.0285435255471212	-0.000152587890625	\\
0.0285879167221556	0.00030517578125	\\
0.02863230789719	0.000335693359375	\\
0.0286766990722244	0.00048828125	\\
0.0287210902472588	0.00079345703125	\\
0.0287654814222932	0.000885009765625	\\
0.0288098725973277	0.00091552734375	\\
0.0288542637723621	0.0009765625	\\
0.0288986549473965	0.000732421875	\\
0.0289430461224309	0.00091552734375	\\
0.0289874372974653	0.0008544921875	\\
0.0290318284724997	0.00048828125	\\
0.0290762196475341	0.000579833984375	\\
0.0291206108225685	-3.0517578125e-05	\\
0.0291650019976029	-0.000457763671875	\\
0.0292093931726373	-0.000640869140625	\\
0.0292537843476717	-0.0010986328125	\\
0.0292981755227061	-0.001251220703125	\\
0.0293425666977405	-0.00079345703125	\\
0.0293869578727749	-0.001495361328125	\\
0.0294313490478093	-0.001708984375	\\
0.0294757402228437	-0.001800537109375	\\
0.0295201313978781	-0.002166748046875	\\
0.0295645225729125	-0.00244140625	\\
0.0296089137479469	-0.003021240234375	\\
0.0296533049229813	-0.002838134765625	\\
0.0296976960980157	-0.003021240234375	\\
0.0297420872730501	-0.003143310546875	\\
0.0297864784480845	-0.002838134765625	\\
0.0298308696231189	-0.002838134765625	\\
0.0298752607981533	-0.003265380859375	\\
0.0299196519731877	-0.00347900390625	\\
0.0299640431482221	-0.003662109375	\\
0.0300084343232565	-0.0035400390625	\\
0.0300528254982909	-0.00341796875	\\
0.0300972166733253	-0.003753662109375	\\
0.0301416078483597	-0.003387451171875	\\
0.0301859990233941	-0.002960205078125	\\
0.0302303901984286	-0.003326416015625	\\
0.030274781373463	-0.00372314453125	\\
0.0303191725484974	-0.0037841796875	\\
0.0303635637235318	-0.0037841796875	\\
0.0304079548985662	-0.00360107421875	\\
0.0304523460736006	-0.00341796875	\\
0.030496737248635	-0.00286865234375	\\
0.0305411284236694	-0.0028076171875	\\
0.0305855195987038	-0.003265380859375	\\
0.0306299107737382	-0.003173828125	\\
0.0306743019487726	-0.002777099609375	\\
0.030718693123807	-0.00262451171875	\\
0.0307630842988414	-0.002227783203125	\\
0.0308074754738758	-0.001953125	\\
0.0308518666489102	-0.00213623046875	\\
0.0308962578239446	-0.002166748046875	\\
0.030940648998979	-0.001861572265625	\\
0.0309850401740134	-0.001708984375	\\
0.0310294313490478	-0.001953125	\\
0.0310738225240822	-0.00140380859375	\\
0.0311182136991166	-0.00128173828125	\\
0.031162604874151	-0.001495361328125	\\
0.0312069960491854	-0.00079345703125	\\
0.0312513872242198	-0.000640869140625	\\
0.0312957783992542	-0.0009765625	\\
0.0313401695742886	-0.000946044921875	\\
0.031384560749323	-0.000823974609375	\\
0.0314289519243574	-0.000762939453125	\\
0.0314733430993918	-0.000701904296875	\\
0.0315177342744262	-0.00067138671875	\\
0.0315621254494606	-0.0013427734375	\\
0.031606516624495	-0.00164794921875	\\
0.0316509077995295	-0.001678466796875	\\
0.0316952989745639	-0.00201416015625	\\
0.0317396901495983	-0.002166748046875	\\
0.0317840813246327	-0.002532958984375	\\
0.0318284724996671	-0.002349853515625	\\
0.0318728636747015	-0.00225830078125	\\
0.0319172548497359	-0.00213623046875	\\
0.0319616460247703	-0.002166748046875	\\
0.0320060371998047	-0.00238037109375	\\
0.0320504283748391	-0.002197265625	\\
0.0320948195498735	-0.00201416015625	\\
0.0321392107249079	-0.002105712890625	\\
0.0321836018999423	-0.0018310546875	\\
0.0322279930749767	-0.001678466796875	\\
0.0322723842500111	-0.001678466796875	\\
0.0323167754250455	-0.001220703125	\\
0.0323611666000799	-0.000823974609375	\\
0.0324055577751143	-0.000823974609375	\\
0.0324499489501487	-0.000885009765625	\\
0.0324943401251831	-0.00128173828125	\\
0.0325387313002175	-0.001007080078125	\\
0.0325831224752519	-0.000579833984375	\\
0.0326275136502863	-0.00091552734375	\\
0.0326719048253207	-0.00079345703125	\\
0.0327162960003551	-0.000518798828125	\\
0.0327606871753895	-0.000274658203125	\\
0.0328050783504239	-3.0517578125e-05	\\
0.0328494695254583	-3.0517578125e-05	\\
0.0328938607004927	-0.000213623046875	\\
0.0329382518755271	-0.000152587890625	\\
0.0329826430505615	-9.1552734375e-05	\\
0.0330270342255959	-0.000457763671875	\\
0.0330714254006304	-0.000213623046875	\\
0.0331158165756648	0.000152587890625	\\
0.0331602077506992	-0.000244140625	\\
0.0332045989257336	-0.00030517578125	\\
0.033248990100768	6.103515625e-05	\\
0.0332933812758024	0.0001220703125	\\
0.0333377724508368	0.000152587890625	\\
0.0333821636258712	0	\\
0.0334265548009056	-0.000213623046875	\\
0.03347094597594	-6.103515625e-05	\\
0.0335153371509744	9.1552734375e-05	\\
0.0335597283260088	0.00054931640625	\\
0.0336041195010432	0.00042724609375	\\
0.0336485106760776	0.00079345703125	\\
0.033692901851112	0.0010986328125	\\
0.0337372930261464	0.00115966796875	\\
0.0337816842011808	0.001220703125	\\
0.0338260753762152	0.0013427734375	\\
0.0338704665512496	0.001678466796875	\\
0.033914857726284	0.001251220703125	\\
0.0339592489013184	0.00115966796875	\\
0.0340036400763528	0.001007080078125	\\
0.0340480312513872	0.000946044921875	\\
0.0340924224264216	0.001251220703125	\\
0.034136813601456	0.000823974609375	\\
0.0341812047764904	0.000640869140625	\\
0.0342255959515248	0.000762939453125	\\
0.0342699871265592	0.0008544921875	\\
0.0343143783015936	0.000762939453125	\\
0.034358769476628	0.000579833984375	\\
0.0344031606516624	0.000885009765625	\\
0.0344475518266969	0.0006103515625	\\
0.0344919430017313	0.00067138671875	\\
0.0345363341767657	0.00091552734375	\\
0.0345807253518001	0.001068115234375	\\
0.0346251165268345	0.000885009765625	\\
0.0346695077018689	0.000762939453125	\\
0.0347138988769033	0.00079345703125	\\
0.0347582900519377	0.000579833984375	\\
0.0348026812269721	0.00042724609375	\\
0.0348470724020065	0.0003662109375	\\
0.0348914635770409	0.000518798828125	\\
0.0349358547520753	0.00054931640625	\\
0.0349802459271097	0.00067138671875	\\
0.0350246371021441	0.0006103515625	\\
0.0350690282771785	0.00067138671875	\\
0.0351134194522129	0.0008544921875	\\
0.0351578106272473	0.000457763671875	\\
0.0352022018022817	0.0008544921875	\\
0.0352465929773161	0.000946044921875	\\
0.0352909841523505	0.000579833984375	\\
0.0353353753273849	0.000518798828125	\\
0.0353797665024193	0.000396728515625	\\
0.0354241576774537	0.00054931640625	\\
0.0354685488524881	0.00018310546875	\\
0.0355129400275225	9.1552734375e-05	\\
0.0355573312025569	0.00030517578125	\\
0.0356017223775913	0.0003662109375	\\
0.0356461135526257	0.0001220703125	\\
0.0356905047276601	-6.103515625e-05	\\
0.0357348959026945	0.000579833984375	\\
0.0357792870777289	0.00042724609375	\\
0.0358236782527634	0.00048828125	\\
0.0358680694277978	0.000762939453125	\\
0.0359124606028322	0.00048828125	\\
0.0359568517778666	6.103515625e-05	\\
0.036001242952901	-0.00048828125	\\
0.0360456341279354	9.1552734375e-05	\\
0.0360900253029698	0.000274658203125	\\
0.0361344164780042	-0.000213623046875	\\
0.0361788076530386	-0.00018310546875	\\
0.036223198828073	-0.000274658203125	\\
0.0362675900031074	-0.000335693359375	\\
0.0363119811781418	3.0517578125e-05	\\
0.0363563723531762	-0.00018310546875	\\
0.0364007635282106	-0.000396728515625	\\
0.036445154703245	-0.000213623046875	\\
0.0364895458782794	-0.000335693359375	\\
0.0365339370533138	-0.000335693359375	\\
0.0365783282283482	-0.000396728515625	\\
0.0366227194033826	-0.000335693359375	\\
0.036667110578417	-0.0003662109375	\\
0.0367115017534514	-0.00030517578125	\\
0.0367558929284858	-6.103515625e-05	\\
0.0368002841035202	-0.000457763671875	\\
0.0368446752785546	-0.000457763671875	\\
0.036889066453589	-0.00042724609375	\\
0.0369334576286234	-0.000335693359375	\\
0.0369778488036578	-0.000396728515625	\\
0.0370222399786922	-0.00128173828125	\\
0.0370666311537266	-0.0009765625	\\
0.037111022328761	-0.0009765625	\\
0.0371554135037954	-0.001373291015625	\\
0.0371998046788298	-0.0013427734375	\\
0.0372441958538643	-0.00146484375	\\
0.0372885870288987	-0.001739501953125	\\
0.0373329782039331	-0.001739501953125	\\
0.0373773693789675	-0.001556396484375	\\
0.0374217605540019	-0.002227783203125	\\
0.0374661517290363	-0.00213623046875	\\
0.0375105429040707	-0.00213623046875	\\
0.0375549340791051	-0.00250244140625	\\
0.0375993252541395	-0.002105712890625	\\
0.0376437164291739	-0.00177001953125	\\
0.0376881076042083	-0.00177001953125	\\
0.0377324987792427	-0.002044677734375	\\
0.0377768899542771	-0.002044677734375	\\
0.0378212811293115	-0.001922607421875	\\
0.0378656723043459	-0.002227783203125	\\
0.0379100634793803	-0.00244140625	\\
0.0379544546544147	-0.00213623046875	\\
0.0379988458294491	-0.001861572265625	\\
0.0380432370044835	-0.00201416015625	\\
0.0380876281795179	-0.001953125	\\
0.0381320193545523	-0.0015869140625	\\
0.0381764105295867	-0.001373291015625	\\
0.0382208017046211	-0.00152587890625	\\
0.0382651928796555	-0.001251220703125	\\
0.0383095840546899	-0.000823974609375	\\
0.0383539752297243	-0.000762939453125	\\
0.0383983664047587	-0.000457763671875	\\
0.0384427575797931	-0.000213623046875	\\
0.0384871487548275	-0.000335693359375	\\
0.0385315399298619	-0.000152587890625	\\
0.0385759311048963	-0.000396728515625	\\
0.0386203222799308	-0.00054931640625	\\
0.0386647134549652	-0.00079345703125	\\
0.0387091046299996	-0.00067138671875	\\
0.038753495805034	-0.000762939453125	\\
0.0387978869800684	-0.001190185546875	\\
0.0388422781551028	-0.0013427734375	\\
0.0388866693301372	-0.001434326171875	\\
0.0389310605051716	-0.001190185546875	\\
0.038975451680206	-0.0009765625	\\
0.0390198428552404	-0.001251220703125	\\
0.0390642340302748	-0.001556396484375	\\
0.0391086252053092	-0.00152587890625	\\
0.0391530163803436	-0.00177001953125	\\
0.039197407555378	-0.00213623046875	\\
0.0392417987304124	-0.0020751953125	\\
0.0392861899054468	-0.002166748046875	\\
0.0393305810804812	-0.001922607421875	\\
0.0393749722555156	-0.002166748046875	\\
0.03941936343055	-0.002838134765625	\\
0.0394637546055844	-0.003143310546875	\\
0.0395081457806188	-0.00299072265625	\\
0.0395525369556532	-0.0028076171875	\\
0.0395969281306876	-0.002960205078125	\\
0.039641319305722	-0.002593994140625	\\
0.0396857104807564	-0.002532958984375	\\
0.0397301016557908	-0.002288818359375	\\
0.0397744928308252	-0.002197265625	\\
0.0398188840058596	-0.00250244140625	\\
0.039863275180894	-0.00244140625	\\
0.0399076663559284	-0.00250244140625	\\
0.0399520575309628	-0.002197265625	\\
0.0399964487059973	-0.002105712890625	\\
0.0400408398810317	-0.002105712890625	\\
0.0400852310560661	-0.0020751953125	\\
0.0401296222311005	-0.00189208984375	\\
0.0401740134061349	-0.00146484375	\\
0.0402184045811693	-0.001495361328125	\\
0.0402627957562037	-0.001220703125	\\
0.0403071869312381	-0.00091552734375	\\
0.0403515781062725	-0.001190185546875	\\
0.0403959692813069	-0.000885009765625	\\
0.0404403604563413	-0.000823974609375	\\
0.0404847516313757	-0.000946044921875	\\
0.0405291428064101	-0.001190185546875	\\
0.0405735339814445	-0.001251220703125	\\
0.0406179251564789	-0.000823974609375	\\
0.0406623163315133	-0.000732421875	\\
0.0407067075065477	-0.000885009765625	\\
0.0407510986815821	-0.0008544921875	\\
0.0407954898566165	-0.000885009765625	\\
0.0408398810316509	-0.00146484375	\\
0.0408842722066853	-0.00146484375	\\
0.0409286633817197	-0.001190185546875	\\
0.0409730545567541	-0.00128173828125	\\
0.0410174457317885	-0.001708984375	\\
0.0410618369068229	-0.001861572265625	\\
0.0411062280818573	-0.001800537109375	\\
0.0411506192568917	-0.0020751953125	\\
0.0411950104319261	-0.00238037109375	\\
0.0412394016069605	-0.002349853515625	\\
0.0412837927819949	-0.00244140625	\\
0.0413281839570293	-0.002471923828125	\\
0.0413725751320637	-0.002197265625	\\
0.0414169663070982	-0.002349853515625	\\
0.0414613574821326	-0.001983642578125	\\
0.041505748657167	-0.001861572265625	\\
0.0415501398322014	-0.00225830078125	\\
0.0415945310072358	-0.00201416015625	\\
0.0416389221822702	-0.00152587890625	\\
0.0416833133573046	-0.0013427734375	\\
0.041727704532339	-0.0018310546875	\\
0.0417720957073734	-0.001373291015625	\\
0.0418164868824078	-0.000823974609375	\\
0.0418608780574422	-0.000762939453125	\\
0.0419052692324766	-0.000579833984375	\\
0.041949660407511	-0.00030517578125	\\
0.0419940515825454	-0.0001220703125	\\
0.0420384427575798	0.00018310546875	\\
0.0420828339326142	0.000152587890625	\\
0.0421272251076486	9.1552734375e-05	\\
0.042171616282683	0.000640869140625	\\
0.0422160074577174	0.00091552734375	\\
0.0422603986327518	0.001007080078125	\\
0.0423047898077862	0.0008544921875	\\
0.0423491809828206	0.000732421875	\\
0.042393572157855	0.0009765625	\\
0.0424379633328894	0.0009765625	\\
0.0424823545079238	0.00115966796875	\\
0.0425267456829582	0.00152587890625	\\
0.0425711368579926	0.001739501953125	\\
0.042615528033027	0.00177001953125	\\
0.0426599192080614	0.00189208984375	\\
0.0427043103830958	0.00177001953125	\\
0.0427487015581302	0.001556396484375	\\
0.0427930927331646	0.00140380859375	\\
0.0428374839081991	0.001800537109375	\\
0.0428818750832335	0.001678466796875	\\
0.0429262662582679	0.00164794921875	\\
0.0429706574333023	0.001922607421875	\\
0.0430150486083367	0.00152587890625	\\
0.0430594397833711	0.001556396484375	\\
0.0431038309584055	0.001708984375	\\
0.0431482221334399	0.001861572265625	\\
0.0431926133084743	0.00152587890625	\\
0.0432370044835087	0.00152587890625	\\
0.0432813956585431	0.001617431640625	\\
0.0433257868335775	0.001495361328125	\\
0.0433701780086119	0.001678466796875	\\
0.0434145691836463	0.00152587890625	\\
0.0434589603586807	0.00164794921875	\\
0.0435033515337151	0.00164794921875	\\
0.0435477427087495	0.001190185546875	\\
0.0435921338837839	0.0018310546875	\\
0.0436365250588183	0.001617431640625	\\
0.0436809162338527	0.00164794921875	\\
0.0437253074088871	0.00146484375	\\
0.0437696985839215	0.001251220703125	\\
0.0438140897589559	0.00152587890625	\\
0.0438584809339903	0.00115966796875	\\
0.0439028721090247	0.001007080078125	\\
0.0439472632840591	0.00103759765625	\\
0.0439916544590935	0.001068115234375	\\
0.0440360456341279	0.001495361328125	\\
0.0440804368091623	0.00140380859375	\\
0.0441248279841967	0.00152587890625	\\
0.0441692191592311	0.00152587890625	\\
0.0442136103342655	0.001373291015625	\\
0.0442580015093	0.00189208984375	\\
0.0443023926843344	0.002044677734375	\\
0.0443467838593688	0.001708984375	\\
0.0443911750344032	0.00152587890625	\\
0.0444355662094376	0.00177001953125	\\
0.044479957384472	0.001739501953125	\\
0.0445243485595064	0.00146484375	\\
0.0445687397345408	0.001708984375	\\
0.0446131309095752	0.0018310546875	\\
0.0446575220846096	0.001617431640625	\\
0.044701913259644	0.00177001953125	\\
0.0447463044346784	0.00201416015625	\\
0.0447906956097128	0.00201416015625	\\
0.0448350867847472	0.00213623046875	\\
0.0448794779597816	0.002349853515625	\\
0.044923869134816	0.001861572265625	\\
0.0449682603098504	0.0020751953125	\\
0.0450126514848848	0.0023193359375	\\
0.0450570426599192	0.002166748046875	\\
0.0451014338349536	0.00244140625	\\
0.045145825009988	0.002471923828125	\\
0.0451902161850224	0.00201416015625	\\
0.0452346073600568	0.00201416015625	\\
0.0452789985350912	0.00244140625	\\
0.0453233897101256	0.00189208984375	\\
0.04536778088516	0.0018310546875	\\
0.0454121720601944	0.002593994140625	\\
0.0454565632352288	0.00286865234375	\\
0.0455009544102632	0.00213623046875	\\
0.0455453455852976	0.001190185546875	\\
0.045589736760332	0.001617431640625	\\
0.0456341279353664	0.0015869140625	\\
0.0456785191104008	0.00140380859375	\\
0.0457229102854353	0.00140380859375	\\
0.0457673014604697	0.001220703125	\\
0.0458116926355041	0.0009765625	\\
0.0458560838105385	0.001068115234375	\\
0.0459004749855729	0.000946044921875	\\
0.0459448661606073	0.000823974609375	\\
0.0459892573356417	0.000823974609375	\\
0.0460336485106761	0.00079345703125	\\
0.0460780396857105	0.0008544921875	\\
0.0461224308607449	0.0006103515625	\\
0.0461668220357793	0.000518798828125	\\
0.0462112132108137	0	\\
0.0462556043858481	-0.000457763671875	\\
0.0462999955608825	-0.00048828125	\\
0.0463443867359169	-0.00030517578125	\\
0.0463887779109513	-0.000335693359375	\\
0.0464331690859857	-0.00054931640625	\\
0.0464775602610201	-0.001129150390625	\\
0.0465219514360545	-0.000885009765625	\\
0.0465663426110889	-0.0008544921875	\\
0.0466107337861233	-0.0008544921875	\\
0.0466551249611577	-0.000885009765625	\\
0.0466995161361921	-0.00115966796875	\\
0.0467439073112265	-0.000579833984375	\\
0.0467882984862609	-0.00091552734375	\\
0.0468326896612953	-0.000946044921875	\\
0.0468770808363297	-0.00079345703125	\\
0.0469214720113641	-0.001251220703125	\\
0.0469658631863985	-0.001190185546875	\\
0.0470102543614329	-0.000701904296875	\\
0.0470546455364673	-0.000335693359375	\\
0.0470990367115018	-0.000396728515625	\\
0.0471434278865362	-0.00067138671875	\\
0.0471878190615706	-0.000732421875	\\
0.047232210236605	-0.0006103515625	\\
0.0472766014116394	-0.00018310546875	\\
0.0473209925866738	-0.0001220703125	\\
0.0473653837617082	-9.1552734375e-05	\\
0.0474097749367426	-0.0001220703125	\\
0.047454166111777	0.0003662109375	\\
0.0474985572868114	0.000335693359375	\\
0.0475429484618458	0	\\
0.0475873396368802	0	\\
0.0476317308119146	9.1552734375e-05	\\
0.047676121986949	-3.0517578125e-05	\\
0.0477205131619834	0.00018310546875	\\
0.0477649043370178	0.00048828125	\\
0.0478092955120522	0.00048828125	\\
0.0478536866870866	0.00048828125	\\
0.047898077862121	0.0006103515625	\\
0.0479424690371554	0.001129150390625	\\
0.0479868602121898	0.0010986328125	\\
0.0480312513872242	0.0010986328125	\\
0.0480756425622586	0.001220703125	\\
0.048120033737293	0.00103759765625	\\
0.0481644249123274	0.000762939453125	\\
0.0482088160873618	0.001007080078125	\\
0.0482532072623962	0.000823974609375	\\
0.0482975984374306	0.000823974609375	\\
0.048341989612465	0.000946044921875	\\
0.0483863807874994	0.000244140625	\\
0.0484307719625338	9.1552734375e-05	\\
0.0484751631375683	6.103515625e-05	\\
0.0485195543126027	0	\\
0.0485639454876371	-0.000213623046875	\\
0.0486083366626715	-0.000213623046875	\\
0.0486527278377059	-0.00048828125	\\
0.0486971190127403	-0.00091552734375	\\
0.0487415101877747	-0.000823974609375	\\
0.0487859013628091	-0.000518798828125	\\
0.0488302925378435	-0.00067138671875	\\
0.0488746837128779	-0.000885009765625	\\
0.0489190748879123	-0.00079345703125	\\
0.0489634660629467	-0.0009765625	\\
0.0490078572379811	-0.00140380859375	\\
0.0490522484130155	-0.00128173828125	\\
0.0490966395880499	-0.001220703125	\\
0.0491410307630843	-0.001617431640625	\\
0.0491854219381187	-0.001312255859375	\\
0.0492298131131531	-0.0010986328125	\\
0.0492742042881875	-0.0010986328125	\\
0.0493185954632219	-0.000457763671875	\\
0.0493629866382563	-0.00042724609375	\\
0.0494073778132907	-0.000885009765625	\\
0.0494517689883251	-0.000885009765625	\\
0.0494961601633595	-0.000885009765625	\\
0.0495405513383939	-0.00091552734375	\\
0.0495849425134283	-0.000701904296875	\\
0.0496293336884627	-0.000396728515625	\\
0.0496737248634971	-0.0003662109375	\\
0.0497181160385315	-9.1552734375e-05	\\
0.0497625072135659	0.00018310546875	\\
0.0498068983886003	0.000213623046875	\\
0.0498512895636348	0.0003662109375	\\
0.0498956807386692	0.0003662109375	\\
0.0499400719137036	0.00067138671875	\\
0.049984463088738	0.001007080078125	\\
0.0500288542637724	0.000946044921875	\\
0.0500732454388068	0.000946044921875	\\
0.0501176366138412	0.001190185546875	\\
0.0501620277888756	0.00146484375	\\
0.05020641896391	0.001434326171875	\\
0.0502508101389444	0.001434326171875	\\
0.0502952013139788	0.000640869140625	\\
0.0503395924890132	0.000823974609375	\\
0.0503839836640476	0.001190185546875	\\
0.050428374839082	0.000732421875	\\
0.0504727660141164	0.000274658203125	\\
0.0505171571891508	0.00030517578125	\\
0.0505615483641852	0.000152587890625	\\
0.0506059395392196	-0.00018310546875	\\
0.050650330714254	0.000274658203125	\\
0.0506947218892884	-0.0001220703125	\\
0.0507391130643228	-0.0003662109375	\\
0.0507835042393572	-0.00018310546875	\\
0.0508278954143916	-3.0517578125e-05	\\
0.050872286589426	-9.1552734375e-05	\\
0.0509166777644604	-0.000213623046875	\\
0.0509610689394948	-3.0517578125e-05	\\
0.0510054601145292	-0.000213623046875	\\
0.0510498512895636	-0.000579833984375	\\
0.051094242464598	-0.000732421875	\\
0.0511386336396324	-0.000885009765625	\\
0.0511830248146668	-0.0006103515625	\\
0.0512274159897013	-0.0006103515625	\\
0.0512718071647357	-0.0010986328125	\\
0.0513161983397701	-0.000946044921875	\\
0.0513605895148045	-0.00079345703125	\\
0.0514049806898389	-0.000762939453125	\\
0.0514493718648733	-0.0006103515625	\\
0.0514937630399077	-0.000152587890625	\\
0.0515381542149421	-0.000518798828125	\\
0.0515825453899765	-0.00054931640625	\\
0.0516269365650109	3.0517578125e-05	\\
0.0516713277400453	0.00018310546875	\\
0.0517157189150797	0.000274658203125	\\
0.0517601100901141	0.000640869140625	\\
0.0518045012651485	0.000640869140625	\\
0.0518488924401829	0.00048828125	\\
0.0518932836152173	0.00091552734375	\\
0.0519376747902517	0.0008544921875	\\
0.0519820659652861	0.00091552734375	\\
0.0520264571403205	0.001312255859375	\\
0.0520708483153549	0.001068115234375	\\
0.0521152394903893	0.00140380859375	\\
0.0521596306654237	0.001861572265625	\\
0.0522040218404581	0.001922607421875	\\
0.0522484130154925	0.0018310546875	\\
0.0522928041905269	0.002288818359375	\\
0.0523371953655613	0.002899169921875	\\
0.0523815865405957	0.00286865234375	\\
0.0524259777156301	0.00274658203125	\\
0.0524703688906645	0.00286865234375	\\
0.0525147600656989	0.0030517578125	\\
0.0525591512407333	0.002777099609375	\\
0.0526035424157677	0.0023193359375	\\
0.0526479335908022	0.002349853515625	\\
0.0526923247658366	0.002532958984375	\\
0.052736715940871	0.002685546875	\\
0.0527811071159054	0.00250244140625	\\
0.0528254982909398	0.002532958984375	\\
0.0528698894659742	0.002288818359375	\\
0.0529142806410086	0.001953125	\\
0.052958671816043	0.00201416015625	\\
0.0530030629910774	0.00177001953125	\\
0.0530474541661118	0.001739501953125	\\
0.0530918453411462	0.001983642578125	\\
0.0531362365161806	0.001953125	\\
0.053180627691215	0.001983642578125	\\
0.0532250188662494	0.001678466796875	\\
0.0532694100412838	0.0013427734375	\\
0.0533138012163182	0.001312255859375	\\
0.0533581923913526	0.00128173828125	\\
0.053402583566387	0.001617431640625	\\
0.0534469747414214	0.001312255859375	\\
0.0534913659164558	0.001068115234375	\\
0.0535357570914902	0.001312255859375	\\
0.0535801482665246	0.00067138671875	\\
0.053624539441559	0.000885009765625	\\
0.0536689306165934	0.000823974609375	\\
0.0537133217916278	0.000213623046875	\\
0.0537577129666622	0.00030517578125	\\
0.0538021041416966	0.00054931640625	\\
0.053846495316731	0.000823974609375	\\
0.0538908864917654	0.00079345703125	\\
0.0539352776667998	0.0008544921875	\\
0.0539796688418342	0.000732421875	\\
0.0540240600168686	0.001190185546875	\\
0.0540684511919031	0.00115966796875	\\
0.0541128423669375	0.00103759765625	\\
0.0541572335419719	0.001007080078125	\\
0.0542016247170063	0.0009765625	\\
0.0542460158920407	0.001220703125	\\
0.0542904070670751	0.001190185546875	\\
0.0543347982421095	0.001190185546875	\\
0.0543791894171439	0.0013427734375	\\
0.0544235805921783	0.001312255859375	\\
0.0544679717672127	0.001068115234375	\\
0.0545123629422471	0.00115966796875	\\
0.0545567541172815	0.00146484375	\\
0.0546011452923159	0.001495361328125	\\
0.0546455364673503	0.00079345703125	\\
0.0546899276423847	0.00048828125	\\
0.0547343188174191	0.000762939453125	\\
0.0547787099924535	0.0009765625	\\
0.0548231011674879	0.000732421875	\\
0.0548674923425223	0.00054931640625	\\
0.0549118835175567	0.000457763671875	\\
0.0549562746925911	0.000457763671875	\\
0.0550006658676255	0.000732421875	\\
0.0550450570426599	0.000457763671875	\\
0.0550894482176943	0.000701904296875	\\
0.0551338393927287	0.00091552734375	\\
0.0551782305677631	0.000946044921875	\\
0.0552226217427975	0.000579833984375	\\
0.0552670129178319	0.000457763671875	\\
0.0553114040928663	0.000732421875	\\
0.0553557952679007	0.0003662109375	\\
0.0554001864429351	0.000274658203125	\\
0.0554445776179695	0.000244140625	\\
0.055488968793004	-3.0517578125e-05	\\
0.0555333599680384	-6.103515625e-05	\\
0.0555777511430728	-0.0001220703125	\\
0.0556221423181072	-0.000579833984375	\\
0.0556665334931416	-0.00067138671875	\\
0.055710924668176	-0.00067138671875	\\
0.0557553158432104	-0.000457763671875	\\
0.0557997070182448	-0.000274658203125	\\
0.0558440981932792	-0.000823974609375	\\
0.0558884893683136	-0.00079345703125	\\
0.055932880543348	-0.000762939453125	\\
0.0559772717183824	-0.001129150390625	\\
0.0560216628934168	-0.0013427734375	\\
0.0560660540684512	-0.00152587890625	\\
0.0561104452434856	-0.001220703125	\\
0.05615483641852	-0.001434326171875	\\
0.0561992275935544	-0.00146484375	\\
0.0562436187685888	-0.00164794921875	\\
0.0562880099436232	-0.001922607421875	\\
0.0563324011186576	-0.00201416015625	\\
0.056376792293692	-0.0018310546875	\\
0.0564211834687264	-0.00164794921875	\\
0.0564655746437608	-0.001678466796875	\\
0.0565099658187952	-0.00140380859375	\\
0.0565543569938296	-0.001373291015625	\\
0.056598748168864	-0.0018310546875	\\
0.0566431393438984	-0.001312255859375	\\
0.0566875305189328	-0.0010986328125	\\
0.0567319216939672	-0.00128173828125	\\
0.0567763128690016	-0.00103759765625	\\
0.056820704044036	-0.00128173828125	\\
0.0568650952190704	-0.0008544921875	\\
0.0569094863941049	-0.000579833984375	\\
0.0569538775691393	-0.00103759765625	\\
0.0569982687441737	-0.000885009765625	\\
0.0570426599192081	-0.0008544921875	\\
0.0570870510942425	-0.00042724609375	\\
0.0571314422692769	0	\\
0.0571758334443113	0	\\
0.0572202246193457	0.000213623046875	\\
0.0572646157943801	-0.000213623046875	\\
0.0573090069694145	-0.000274658203125	\\
0.0573533981444489	-3.0517578125e-05	\\
0.0573977893194833	-0.000244140625	\\
0.0574421804945177	-0.000244140625	\\
0.0574865716695521	-0.000274658203125	\\
0.0575309628445865	-6.103515625e-05	\\
0.0575753540196209	9.1552734375e-05	\\
0.0576197451946553	0	\\
0.0576641363696897	-0.0001220703125	\\
0.0577085275447241	-0.00042724609375	\\
0.0577529187197585	-0.000335693359375	\\
0.0577973098947929	-0.0001220703125	\\
0.0578417010698273	9.1552734375e-05	\\
0.0578860922448617	-0.000213623046875	\\
0.0579304834198961	-0.000152587890625	\\
0.0579748745949305	3.0517578125e-05	\\
0.0580192657699649	-0.0001220703125	\\
0.0580636569449993	-9.1552734375e-05	\\
0.0581080481200337	0.0001220703125	\\
0.0581524392950681	-0.000152587890625	\\
0.0581968304701025	6.103515625e-05	\\
0.0582412216451369	0.00018310546875	\\
0.0582856128201713	9.1552734375e-05	\\
0.0583300039952058	0.00042724609375	\\
0.0583743951702402	0.000946044921875	\\
0.0584187863452746	0.00103759765625	\\
0.058463177520309	0.000396728515625	\\
0.0585075686953434	0.0003662109375	\\
0.0585519598703778	0.000823974609375	\\
0.0585963510454122	0.001007080078125	\\
0.0586407422204466	0.000701904296875	\\
0.058685133395481	0.000518798828125	\\
0.0587295245705154	0.000518798828125	\\
0.0587739157455498	0.0003662109375	\\
0.0588183069205842	0.00042724609375	\\
0.0588626980956186	0.00067138671875	\\
0.058907089270653	0.00079345703125	\\
0.0589514804456874	0.00054931640625	\\
0.0589958716207218	0.0008544921875	\\
0.0590402627957562	0.00115966796875	\\
0.0590846539707906	0.000946044921875	\\
0.059129045145825	0.0013427734375	\\
0.0591734363208594	0.001068115234375	\\
0.0592178274958938	0.00091552734375	\\
0.0592622186709282	0.001068115234375	\\
0.0593066098459626	0.000396728515625	\\
0.059351001020997	6.103515625e-05	\\
0.0593953921960314	0.0001220703125	\\
0.0594397833710658	0.000213623046875	\\
0.0594841745461002	0.000457763671875	\\
0.0595285657211346	0.0003662109375	\\
0.059572956896169	0.000152587890625	\\
0.0596173480712034	0.000274658203125	\\
0.0596617392462378	0.000823974609375	\\
0.0597061304212722	0.0008544921875	\\
0.0597505215963067	0.0006103515625	\\
0.0597949127713411	0.000701904296875	\\
0.0598393039463755	0.00042724609375	\\
0.0598836951214099	-3.0517578125e-05	\\
0.0599280862964443	-9.1552734375e-05	\\
0.0599724774714787	-0.0001220703125	\\
0.0600168686465131	-0.00018310546875	\\
0.0600612598215475	-0.0003662109375	\\
0.0601056509965819	-0.000274658203125	\\
0.0601500421716163	3.0517578125e-05	\\
0.0601944333466507	-0.000213623046875	\\
0.0602388245216851	-0.0001220703125	\\
0.0602832156967195	-0.000274658203125	\\
0.0603276068717539	-0.000274658203125	\\
0.0603719980467883	-9.1552734375e-05	\\
0.0604163892218227	-0.000213623046875	\\
0.0604607803968571	-0.00042724609375	\\
0.0605051715718915	-0.000762939453125	\\
0.0605495627469259	-0.000732421875	\\
0.0605939539219603	-0.0003662109375	\\
0.0606383450969947	-0.00067138671875	\\
0.0606827362720291	-0.0006103515625	\\
0.0607271274470635	-0.00048828125	\\
0.0607715186220979	-0.00103759765625	\\
0.0608159097971323	-0.000946044921875	\\
0.0608603009721667	-0.00115966796875	\\
0.0609046921472011	-0.000762939453125	\\
0.0609490833222355	-0.001007080078125	\\
0.0609934744972699	-0.001068115234375	\\
0.0610378656723043	-0.00054931640625	\\
0.0610822568473387	-0.000518798828125	\\
0.0611266480223732	-0.00018310546875	\\
0.0611710391974076	-0.0001220703125	\\
0.061215430372442	-0.0001220703125	\\
0.0612598215474764	-0.000244140625	\\
0.0613042127225108	-0.000152587890625	\\
0.0613486038975452	0.000152587890625	\\
0.0613929950725796	6.103515625e-05	\\
0.061437386247614	0.000335693359375	\\
0.0614817774226484	-0.000244140625	\\
0.0615261685976828	-0.000579833984375	\\
0.0615705597727172	-0.0001220703125	\\
0.0616149509477516	-6.103515625e-05	\\
0.061659342122786	-3.0517578125e-05	\\
0.0617037332978204	-0.000274658203125	\\
0.0617481244728548	-0.000762939453125	\\
0.0617925156478892	-0.000396728515625	\\
0.0618369068229236	-0.000244140625	\\
0.061881297997958	-0.000244140625	\\
0.0619256891729924	-0.00054931640625	\\
0.0619700803480268	-0.000885009765625	\\
0.0620144715230612	-0.000335693359375	\\
0.0620588626980956	-0.000640869140625	\\
0.06210325387313	-0.00079345703125	\\
0.0621476450481644	-0.00054931640625	\\
0.0621920362231988	-0.001129150390625	\\
0.0622364273982332	-0.001861572265625	\\
0.0622808185732676	-0.001953125	\\
0.062325209748302	-0.001556396484375	\\
0.0623696009233364	-0.001617431640625	\\
0.0624139920983708	-0.0020751953125	\\
0.0624583832734052	-0.002044677734375	\\
0.0625027744484396	-0.001556396484375	\\
0.0625471656234741	-0.001434326171875	\\
0.0625915567985085	-0.001495361328125	\\
0.0626359479735429	-0.001800537109375	\\
0.0626803391485773	-0.00177001953125	\\
0.0627247303236117	-0.001373291015625	\\
0.0627691214986461	-0.00103759765625	\\
0.0628135126736805	-0.001129150390625	\\
0.0628579038487149	-0.001434326171875	\\
0.0629022950237493	-0.00152587890625	\\
0.0629466861987837	-0.001495361328125	\\
0.0629910773738181	-0.001373291015625	\\
0.0630354685488525	-0.00103759765625	\\
0.0630798597238869	-0.00115966796875	\\
0.0631242508989213	-0.00140380859375	\\
0.0631686420739557	-0.001678466796875	\\
0.0632130332489901	-0.00177001953125	\\
0.0632574244240245	-0.00177001953125	\\
0.0633018155990589	-0.002044677734375	\\
0.0633462067740933	-0.00115966796875	\\
0.0633905979491277	-0.0009765625	\\
0.0634349891241621	-0.0010986328125	\\
0.0634793802991965	-0.00079345703125	\\
0.0635237714742309	-0.0008544921875	\\
0.0635681626492653	-0.000732421875	\\
0.0636125538242997	-0.0006103515625	\\
0.0636569449993341	-0.001190185546875	\\
0.0637013361743685	-0.000701904296875	\\
0.0637457273494029	-0.000885009765625	\\
0.0637901185244373	-0.0010986328125	\\
0.0638345096994717	-0.00042724609375	\\
0.0638789008745062	-0.000701904296875	\\
0.0639232920495405	-0.000457763671875	\\
0.063967683224575	-0.00018310546875	\\
0.0640120743996094	-0.000701904296875	\\
0.0640564655746438	-0.00048828125	\\
0.0641008567496782	-0.00048828125	\\
0.0641452479247126	-0.000732421875	\\
0.064189639099747	-0.00054931640625	\\
0.0642340302747814	-0.000579833984375	\\
0.0642784214498158	-0.000518798828125	\\
0.0643228126248502	-0.0003662109375	\\
0.0643672037998846	-0.0006103515625	\\
0.064411594974919	-0.000732421875	\\
0.0644559861499534	-0.00079345703125	\\
0.0645003773249878	-0.001068115234375	\\
0.0645447685000222	-0.001129150390625	\\
0.0645891596750566	-0.001007080078125	\\
0.064633550850091	-0.00079345703125	\\
0.0646779420251254	-0.001129150390625	\\
0.0647223332001598	-0.00146484375	\\
0.0647667243751942	-0.000732421875	\\
0.0648111155502286	-0.000762939453125	\\
0.064855506725263	-0.001220703125	\\
0.0648998979002974	-0.001373291015625	\\
0.0649442890753318	-0.00146484375	\\
0.0649886802503662	-0.001312255859375	\\
0.0650330714254006	-0.001220703125	\\
0.065077462600435	-0.000885009765625	\\
0.0651218537754694	-0.00128173828125	\\
0.0651662449505038	-0.001434326171875	\\
0.0652106361255382	-0.001373291015625	\\
0.0652550273005726	-0.00146484375	\\
0.0652994184756071	-0.001373291015625	\\
0.0653438096506414	-0.001220703125	\\
0.0653882008256759	-0.0009765625	\\
0.0654325920007103	-0.001007080078125	\\
0.0654769831757447	-0.000701904296875	\\
0.0655213743507791	-0.00091552734375	\\
0.0655657655258135	-0.001220703125	\\
0.0656101567008479	-0.001068115234375	\\
0.0656545478758823	-0.000762939453125	\\
0.0656989390509167	-0.0009765625	\\
0.0657433302259511	-0.001190185546875	\\
0.0657877214009855	-0.0008544921875	\\
0.0658321125760199	-0.00091552734375	\\
0.0658765037510543	-0.0010986328125	\\
0.0659208949260887	-0.000640869140625	\\
0.0659652861011231	-0.0008544921875	\\
0.0660096772761575	-0.000396728515625	\\
0.0660540684511919	-0.000244140625	\\
0.0660984596262263	-0.00067138671875	\\
0.0661428508012607	-0.000457763671875	\\
0.0661872419762951	-0.000396728515625	\\
0.0662316331513295	-0.000762939453125	\\
0.0662760243263639	-0.001220703125	\\
0.0663204155013983	-0.001129150390625	\\
0.0663648066764327	-0.00091552734375	\\
0.0664091978514671	-0.001129150390625	\\
0.0664535890265015	-0.00091552734375	\\
0.0664979802015359	-0.000701904296875	\\
0.0665423713765703	-0.001129150390625	\\
0.0665867625516047	-0.000946044921875	\\
0.0666311537266392	-0.00048828125	\\
0.0666755449016735	-0.00067138671875	\\
0.066719936076708	-0.0008544921875	\\
0.0667643272517423	-0.00067138671875	\\
0.0668087184267768	-0.00103759765625	\\
0.0668531096018112	-0.001251220703125	\\
0.0668975007768456	-0.000885009765625	\\
0.06694189195188	-0.0008544921875	\\
0.0669862831269144	-0.000640869140625	\\
0.0670306743019488	-0.000579833984375	\\
0.0670750654769832	-0.000823974609375	\\
0.0671194566520176	-0.00018310546875	\\
0.067163847827052	0.00018310546875	\\
0.0672082390020864	-0.0003662109375	\\
0.0672526301771208	-0.000701904296875	\\
0.0672970213521552	-0.00018310546875	\\
0.0673414125271896	-0.0003662109375	\\
0.067385803702224	-0.000518798828125	\\
0.0674301948772584	-0.000335693359375	\\
0.0674745860522928	-0.000579833984375	\\
0.0675189772273272	-0.000457763671875	\\
0.0675633684023616	-0.0001220703125	\\
0.067607759577396	-0.000518798828125	\\
0.0676521507524304	-0.0006103515625	\\
0.0676965419274648	-0.0003662109375	\\
0.0677409331024992	-0.000885009765625	\\
0.0677853242775336	-0.000946044921875	\\
0.067829715452568	-0.000762939453125	\\
0.0678741066276024	-0.00079345703125	\\
0.0679184978026368	-0.000579833984375	\\
0.0679628889776712	-0.000396728515625	\\
0.0680072801527056	-0.000732421875	\\
0.0680516713277401	-0.001007080078125	\\
0.0680960625027744	-0.001068115234375	\\
0.0681404536778089	-0.001190185546875	\\
0.0681848448528432	-0.001190185546875	\\
0.0682292360278777	-0.00091552734375	\\
0.0682736272029121	-0.00103759765625	\\
0.0683180183779465	-0.001007080078125	\\
0.0683624095529809	-0.000579833984375	\\
0.0684068007280153	-0.000396728515625	\\
0.0684511919030497	-0.0001220703125	\\
0.0684955830780841	3.0517578125e-05	\\
0.0685399742531185	0.00018310546875	\\
0.0685843654281529	0.0001220703125	\\
0.0686287566031873	3.0517578125e-05	\\
0.0686731477782217	0.000152587890625	\\
0.0687175389532561	0.0006103515625	\\
0.0687619301282905	0.0010986328125	\\
0.0688063213033249	0.000946044921875	\\
0.0688507124783593	0.001007080078125	\\
0.0688951036533937	0.000946044921875	\\
0.0689394948284281	0.0015869140625	\\
0.0689838860034625	0.001861572265625	\\
0.0690282771784969	0.001678466796875	\\
0.0690726683535313	0.001953125	\\
0.0691170595285657	0.00189208984375	\\
0.0691614507036001	0.00213623046875	\\
0.0692058418786345	0.002593994140625	\\
0.0692502330536689	0.00299072265625	\\
0.0692946242287033	0.002899169921875	\\
0.0693390154037377	0.002960205078125	\\
0.0693834065787721	0.002593994140625	\\
0.0694277977538065	0.002899169921875	\\
0.069472188928841	0.00299072265625	\\
0.0695165801038753	0.002593994140625	\\
0.0695609712789098	0.00286865234375	\\
0.0696053624539442	0.002471923828125	\\
0.0696497536289786	0.002349853515625	\\
0.069694144804013	0.00250244140625	\\
0.0697385359790474	0.00201416015625	\\
0.0697829271540818	0.0015869140625	\\
0.0698273183291162	0.001983642578125	\\
0.0698717095041506	0.001861572265625	\\
0.069916100679185	0.0013427734375	\\
0.0699604918542194	0.001068115234375	\\
0.0700048830292538	0.000274658203125	\\
0.0700492742042882	0.000213623046875	\\
0.0700936653793226	0.00018310546875	\\
0.070138056554357	-0.000152587890625	\\
0.0701824477293914	-0.0003662109375	\\
0.0702268389044258	-0.000640869140625	\\
0.0702712300794602	-0.0010986328125	\\
0.0703156212544946	-0.000640869140625	\\
0.070360012429529	-0.000335693359375	\\
0.0704044036045634	-0.00091552734375	\\
0.0704487947795978	-0.0009765625	\\
0.0704931859546322	-0.001007080078125	\\
0.0705375771296666	-0.00079345703125	\\
0.070581968304701	-0.000885009765625	\\
0.0706263594797354	-0.000732421875	\\
0.0706707506547698	-0.0003662109375	\\
0.0707151418298042	-0.000518798828125	\\
0.0707595330048386	-0.000457763671875	\\
0.070803924179873	-0.000396728515625	\\
0.0708483153549074	-0.000396728515625	\\
0.0708927065299419	6.103515625e-05	\\
0.0709370977049762	-0.000518798828125	\\
0.0709814888800107	-0.000335693359375	\\
0.0710258800550451	0.000762939453125	\\
0.0710702712300795	0.000885009765625	\\
0.0711146624051139	0.0009765625	\\
0.0711590535801483	0.00128173828125	\\
0.0712034447551827	0.001739501953125	\\
0.0712478359302171	0.001678466796875	\\
0.0712922271052515	0.001861572265625	\\
0.0713366182802859	0.001983642578125	\\
0.0713810094553203	0.002044677734375	\\
0.0714254006303547	0.002593994140625	\\
0.0714697918053891	0.00250244140625	\\
0.0715141829804235	0.00213623046875	\\
0.0715585741554579	0.002044677734375	\\
0.0716029653304923	0.001953125	\\
0.0716473565055267	0.001861572265625	\\
0.0716917476805611	0.001495361328125	\\
0.0717361388555955	0.001495361328125	\\
0.0717805300306299	0.00164794921875	\\
0.0718249212056643	0.001220703125	\\
0.0718693123806987	0.000701904296875	\\
0.0719137035557331	0.000579833984375	\\
0.0719580947307675	0.00048828125	\\
0.0720024859058019	9.1552734375e-05	\\
0.0720468770808363	0	\\
0.0720912682558707	-6.103515625e-05	\\
0.0721356594309051	-0.000457763671875	\\
0.0721800506059395	-0.000518798828125	\\
0.0722244417809739	-0.000946044921875	\\
0.0722688329560083	-0.000946044921875	\\
0.0723132241310428	-0.001312255859375	\\
0.0723576153060772	-0.00128173828125	\\
0.0724020064811116	-0.001220703125	\\
0.072446397656146	-0.001861572265625	\\
0.0724907888311804	-0.001617431640625	\\
0.0725351800062148	-0.001434326171875	\\
0.0725795711812492	-0.001678466796875	\\
0.0726239623562836	-0.00103759765625	\\
0.072668353531318	-0.00103759765625	\\
0.0727127447063524	-0.00103759765625	\\
0.0727571358813868	-0.000396728515625	\\
0.0728015270564212	-0.000762939453125	\\
0.0728459182314556	-0.000823974609375	\\
0.07289030940649	-0.0001220703125	\\
0.0729347005815244	-0.00030517578125	\\
0.0729790917565588	-3.0517578125e-05	\\
0.0730234829315932	0.000579833984375	\\
0.0730678741066276	0.00091552734375	\\
0.073112265281662	0.001190185546875	\\
0.0731566564566964	0.001190185546875	\\
0.0732010476317308	0.001251220703125	\\
0.0732454388067652	0.00152587890625	\\
0.0732898299817996	0.001251220703125	\\
0.073334221156834	0.00152587890625	\\
0.0733786123318684	0.001983642578125	\\
0.0734230035069028	0.001708984375	\\
0.0734673946819372	0.001617431640625	\\
0.0735117858569716	0.001953125	\\
0.073556177032006	0.001861572265625	\\
0.0736005682070404	0.001983642578125	\\
0.0736449593820748	0.001953125	\\
0.0736893505571092	0.001922607421875	\\
0.0737337417321437	0.002197265625	\\
0.0737781329071781	0.0015869140625	\\
0.0738225240822125	0.001739501953125	\\
0.0738669152572469	0.0018310546875	\\
0.0739113064322813	0.001434326171875	\\
0.0739556976073157	0.001708984375	\\
0.0740000887823501	0.001708984375	\\
0.0740444799573845	0.002044677734375	\\
0.0740888711324189	0.00213623046875	\\
0.0741332623074533	0.002197265625	\\
0.0741776534824877	0.002685546875	\\
0.0742220446575221	0.002471923828125	\\
0.0742664358325565	0.002044677734375	\\
0.0743108270075909	0.002105712890625	\\
0.0743552181826253	0.001922607421875	\\
0.0743996093576597	0.0010986328125	\\
0.0744440005326941	0.0015869140625	\\
0.0744883917077285	0.002044677734375	\\
0.0745327828827629	0.002044677734375	\\
0.0745771740577973	0.00225830078125	\\
0.0746215652328317	0.002899169921875	\\
0.0746659564078661	0.002532958984375	\\
0.0747103475829005	0.002105712890625	\\
0.0747547387579349	0.002410888671875	\\
0.0747991299329693	0.00225830078125	\\
0.0748435211080037	0.002197265625	\\
0.0748879122830381	0.002685546875	\\
0.0749323034580725	0.002655029296875	\\
0.0749766946331069	0.0025634765625	\\
0.0750210858081413	0.002593994140625	\\
0.0750654769831757	0.00244140625	\\
0.0751098681582102	0.002685546875	\\
0.0751542593332445	0.002471923828125	\\
0.075198650508279	0.0023193359375	\\
0.0752430416833134	0.002838134765625	\\
0.0752874328583478	0.00347900390625	\\
0.0753318240333822	0.003326416015625	\\
0.0753762152084166	0.00299072265625	\\
0.075420606383451	0.00286865234375	\\
0.0754649975584854	0.0023193359375	\\
0.0755093887335198	0.0020751953125	\\
0.0755537799085542	0.00189208984375	\\
0.0755981710835886	0.001800537109375	\\
0.075642562258623	0.00146484375	\\
0.0756869534336574	0.00140380859375	\\
0.0757313446086918	0.00140380859375	\\
0.0757757357837262	0.00146484375	\\
0.0758201269587606	0.001495361328125	\\
0.075864518133795	0.00152587890625	\\
0.0759089093088294	0.001678466796875	\\
0.0759533004838638	0.001373291015625	\\
0.0759976916588982	0.000885009765625	\\
0.0760420828339326	0.0006103515625	\\
0.076086474008967	0.00067138671875	\\
0.0761308651840014	0.000579833984375	\\
0.0761752563590358	0.000732421875	\\
0.0762196475340702	0.001068115234375	\\
0.0762640387091046	0.000823974609375	\\
0.076308429884139	0.00115966796875	\\
0.0763528210591734	0.00128173828125	\\
0.0763972122342078	0.000946044921875	\\
0.0764416034092422	0.000885009765625	\\
0.0764859945842766	0.00115966796875	\\
0.0765303857593111	0.001190185546875	\\
0.0765747769343454	0.000946044921875	\\
0.0766191681093799	0.001129150390625	\\
0.0766635592844143	0.001007080078125	\\
0.0767079504594487	0.001495361328125	\\
0.0767523416344831	0.001922607421875	\\
0.0767967328095175	0.001922607421875	\\
0.0768411239845519	0.00189208984375	\\
0.0768855151595863	0.002105712890625	\\
0.0769299063346207	0.00286865234375	\\
0.0769742975096551	0.002777099609375	\\
0.0770186886846895	0.00250244140625	\\
0.0770630798597239	0.00299072265625	\\
0.0771074710347583	0.002838134765625	\\
0.0771518622097927	0.002685546875	\\
0.0771962533848271	0.0030517578125	\\
0.0772406445598615	0.00286865234375	\\
0.0772850357348959	0.002899169921875	\\
0.0773294269099303	0.003143310546875	\\
0.0773738180849647	0.002960205078125	\\
0.0774182092599991	0.0025634765625	\\
0.0774626004350335	0.00262451171875	\\
0.0775069916100679	0.002716064453125	\\
0.0775513827851023	0.0023193359375	\\
0.0775957739601367	0.00213623046875	\\
0.0776401651351711	0.001800537109375	\\
0.0776845563102055	0.001739501953125	\\
0.0777289474852399	0.001556396484375	\\
0.0777733386602743	0.00140380859375	\\
0.0778177298353087	0.001556396484375	\\
0.0778621210103431	0.000640869140625	\\
0.0779065121853775	0.00079345703125	\\
0.077950903360412	0.000640869140625	\\
0.0779952945354463	-0.000152587890625	\\
0.0780396857104808	-0.00054931640625	\\
0.0780840768855152	9.1552734375e-05	\\
0.0781284680605496	0.000274658203125	\\
0.078172859235584	-0.000274658203125	\\
0.0782172504106184	-0.000579833984375	\\
0.0782616415856528	-0.000396728515625	\\
0.0783060327606872	-0.000274658203125	\\
0.0783504239357216	-3.0517578125e-05	\\
0.078394815110756	0.00030517578125	\\
0.0784392062857904	0.000274658203125	\\
0.0784835974608248	0.00048828125	\\
0.0785279886358592	0.0008544921875	\\
0.0785723798108936	0.001007080078125	\\
0.078616770985928	0.00146484375	\\
0.0786611621609624	0.00189208984375	\\
0.0787055533359968	0.0018310546875	\\
0.0787499445110312	0.00177001953125	\\
0.0787943356860656	0.002288818359375	\\
0.0788387268611	0.0028076171875	\\
0.0788831180361344	0.00341796875	\\
0.0789275092111688	0.00408935546875	\\
0.0789719003862032	0.003936767578125	\\
0.0790162915612376	0.00396728515625	\\
0.079060682736272	0.004302978515625	\\
0.0791050739113064	0.00439453125	\\
0.0791494650863408	0.0048828125	\\
0.0791938562613752	0.0054931640625	\\
0.0792382474364096	0.005859375	\\
0.0792826386114441	0.005767822265625	\\
0.0793270297864784	0.005859375	\\
0.0793714209615129	0.00579833984375	\\
0.0794158121365472	0.005859375	\\
0.0794602033115817	0.005615234375	\\
0.0795045944866161	0.00543212890625	\\
0.0795489856616505	0.00494384765625	\\
0.0795933768366849	0.004180908203125	\\
0.0796377680117193	0.004364013671875	\\
0.0796821591867537	0.003692626953125	\\
0.0797265503617881	0.003204345703125	\\
0.0797709415368225	0.00360107421875	\\
0.0798153327118569	0.00323486328125	\\
0.0798597238868913	0.0029296875	\\
0.0799041150619257	0.003143310546875	\\
0.0799485062369601	0.002777099609375	\\
0.0799928974119945	0.002197265625	\\
0.0800372885870289	0.001953125	\\
0.0800816797620633	0.001434326171875	\\
0.0801260709370977	0.001190185546875	\\
0.0801704621121321	0.001068115234375	\\
0.0802148532871665	0.000732421875	\\
0.0802592444622009	0.00042724609375	\\
0.0803036356372353	0.000244140625	\\
0.0803480268122697	0.00030517578125	\\
0.0803924179873041	0.0003662109375	\\
0.0804368091623385	-6.103515625e-05	\\
0.0804812003373729	0.000152587890625	\\
0.0805255915124073	0.000335693359375	\\
0.0805699826874417	0.000396728515625	\\
0.0806143738624761	0.000518798828125	\\
0.0806587650375105	0.000396728515625	\\
0.080703156212545	0.00091552734375	\\
0.0807475473875793	0.001007080078125	\\
0.0807919385626138	0.0009765625	\\
0.0808363297376481	0.001312255859375	\\
0.0808807209126826	0.001739501953125	\\
0.080925112087717	0.00225830078125	\\
0.0809695032627514	0.00225830078125	\\
0.0810138944377858	0.002288818359375	\\
0.0810582856128202	0.002593994140625	\\
0.0811026767878546	0.002960205078125	\\
0.081147067962889	0.002960205078125	\\
0.0811914591379234	0.002899169921875	\\
0.0812358503129578	0.002777099609375	\\
0.0812802414879922	0.00286865234375	\\
0.0813246326630266	0.003204345703125	\\
0.081369023838061	0.002960205078125	\\
0.0814134150130954	0.00341796875	\\
0.0814578061881298	0.003326416015625	\\
0.0815021973631642	0.002899169921875	\\
0.0815465885381986	0.00341796875	\\
0.081590979713233	0.003204345703125	\\
0.0816353708882674	0.00286865234375	\\
0.0816797620633018	0.00286865234375	\\
0.0817241532383362	0.002197265625	\\
0.0817685444133706	0.001861572265625	\\
0.081812935588405	0.001800537109375	\\
0.0818573267634394	0.00177001953125	\\
0.0819017179384738	0.000885009765625	\\
0.0819461091135082	0.000244140625	\\
0.0819905002885426	0.000518798828125	\\
0.082034891463577	0.0001220703125	\\
0.0820792826386114	-9.1552734375e-05	\\
0.0821236738136459	9.1552734375e-05	\\
0.0821680649886802	0.000213623046875	\\
0.0822124561637147	0	\\
0.0822568473387491	-3.0517578125e-05	\\
0.0823012385137835	-0.000335693359375	\\
0.0823456296888179	-0.00079345703125	\\
0.0823900208638523	-0.000885009765625	\\
0.0824344120388867	-0.000213623046875	\\
0.0824788032139211	0.000274658203125	\\
0.0825231943889555	-0.00030517578125	\\
0.0825675855639899	9.1552734375e-05	\\
0.0826119767390243	0.000274658203125	\\
0.0826563679140587	0.00042724609375	\\
0.0827007590890931	0.00067138671875	\\
0.0827451502641275	0.000640869140625	\\
0.0827895414391619	0.0009765625	\\
0.0828339326141963	0.00152587890625	\\
0.0828783237892307	0.00177001953125	\\
0.0829227149642651	0.001739501953125	\\
0.0829671061392995	0.001953125	\\
0.0830114973143339	0.001922607421875	\\
0.0830558884893683	0.002410888671875	\\
0.0831002796644027	0.00286865234375	\\
0.0831446708394371	0.003082275390625	\\
0.0831890620144715	0.00360107421875	\\
0.0832334531895059	0.003204345703125	\\
0.0832778443645403	0.00299072265625	\\
0.0833222355395747	0.003143310546875	\\
0.0833666267146091	0.003021240234375	\\
0.0834110178896435	0.003143310546875	\\
0.0834554090646779	0.00299072265625	\\
0.0834998002397123	0.0029296875	\\
0.0835441914147468	0.003143310546875	\\
0.0835885825897811	0.00274658203125	\\
0.0836329737648156	0.00225830078125	\\
0.08367736493985	0.0025634765625	\\
0.0837217561148844	0.002288818359375	\\
0.0837661472899188	0.00201416015625	\\
0.0838105384649532	0.00128173828125	\\
0.0838549296399876	0.00115966796875	\\
0.083899320815022	0.0015869140625	\\
0.0839437119900564	0.001251220703125	\\
0.0839881031650908	0.00115966796875	\\
0.0840324943401252	0.000885009765625	\\
0.0840768855151596	0.000640869140625	\\
0.084121276690194	0.000762939453125	\\
0.0841656678652284	0.000579833984375	\\
0.0842100590402628	0.000701904296875	\\
0.0842544502152972	0.00164794921875	\\
0.0842988413903316	0.001800537109375	\\
0.084343232565366	0.00177001953125	\\
0.0843876237404004	0.00225830078125	\\
0.0844320149154348	0.00213623046875	\\
0.0844764060904692	0.00201416015625	\\
0.0845207972655036	0.00250244140625	\\
0.084565188440538	0.002197265625	\\
0.0846095796155724	0.001983642578125	\\
0.0846539707906068	0.00238037109375	\\
0.0846983619656412	0.00311279296875	\\
0.0847427531406756	0.0032958984375	\\
0.08478714431571	0.00311279296875	\\
0.0848315354907444	0.0032958984375	\\
0.0848759266657788	0.003662109375	\\
0.0849203178408132	0.004119873046875	\\
0.0849647090158477	0.004425048828125	\\
0.0850091001908821	0.004302978515625	\\
0.0850534913659165	0.00457763671875	\\
0.0850978825409509	0.00457763671875	\\
0.0851422737159853	0.00445556640625	\\
0.0851866648910197	0.00433349609375	\\
0.0852310560660541	0.00421142578125	\\
0.0852754472410885	0.004608154296875	\\
0.0853198384161229	0.00439453125	\\
0.0853642295911573	0.004058837890625	\\
0.0854086207661917	0.00396728515625	\\
0.0854530119412261	0.00347900390625	\\
0.0854974031162605	0.003326416015625	\\
0.0855417942912949	0.003387451171875	\\
0.0855861854663293	0.00323486328125	\\
0.0856305766413637	0.00299072265625	\\
0.0856749678163981	0.00244140625	\\
0.0857193589914325	0.002166748046875	\\
0.0857637501664669	0.002227783203125	\\
0.0858081413415013	0.001922607421875	\\
0.0858525325165357	0.001617431640625	\\
0.0858969236915701	0.002105712890625	\\
0.0859413148666045	0.001739501953125	\\
0.0859857060416389	0.001068115234375	\\
0.0860300972166733	0.0009765625	\\
0.0860744883917077	0.00103759765625	\\
0.0861188795667421	0.000885009765625	\\
0.0861632707417765	0.000701904296875	\\
0.0862076619168109	0.000701904296875	\\
0.0862520530918453	0.000274658203125	\\
0.0862964442668797	0.000701904296875	\\
0.0863408354419141	0.00128173828125	\\
0.0863852266169486	0.001068115234375	\\
0.086429617791983	0.0015869140625	\\
0.0864740089670174	0.001922607421875	\\
0.0865184001420518	0.0015869140625	\\
0.0865627913170862	0.00140380859375	\\
0.0866071824921206	0.00213623046875	\\
0.086651573667155	0.002105712890625	\\
0.0866959648421894	0.00177001953125	\\
0.0867403560172238	0.00262451171875	\\
0.0867847471922582	0.002685546875	\\
0.0868291383672926	0.00250244140625	\\
0.086873529542327	0.002655029296875	\\
0.0869179207173614	0.00238037109375	\\
0.0869623118923958	0.002288818359375	\\
0.0870067030674302	0.00286865234375	\\
0.0870510942424646	0.002838134765625	\\
0.087095485417499	0.002777099609375	\\
0.0871398765925334	0.002838134765625	\\
0.0871842677675678	0.003082275390625	\\
0.0872286589426022	0.003326416015625	\\
0.0872730501176366	0.002899169921875	\\
0.087317441292671	0.002685546875	\\
0.0873618324677054	0.003021240234375	\\
0.0874062236427398	0.00311279296875	\\
0.0874506148177742	0.002899169921875	\\
0.0874950059928086	0.00274658203125	\\
0.087539397167843	0.002288818359375	\\
0.0875837883428774	0.0020751953125	\\
0.0876281795179118	0.001556396484375	\\
0.0876725706929462	0.001800537109375	\\
0.0877169618679806	0.00177001953125	\\
0.0877613530430151	0.00152587890625	\\
0.0878057442180495	0.001617431640625	\\
0.0878501353930839	0.00128173828125	\\
0.0878945265681183	0.00152587890625	\\
0.0879389177431527	0.0013427734375	\\
0.0879833089181871	0.0010986328125	\\
0.0880277000932215	0.000885009765625	\\
0.0880720912682559	0.000579833984375	\\
0.0881164824432903	0.0006103515625	\\
0.0881608736183247	0.00054931640625	\\
0.0882052647933591	0.00054931640625	\\
0.0882496559683935	0.000335693359375	\\
0.0882940471434279	0	\\
0.0883384383184623	0.0006103515625	\\
0.0883828294934967	0.00067138671875	\\
0.0884272206685311	0.00067138671875	\\
0.0884716118435655	0.00103759765625	\\
0.0885160030185999	0.001190185546875	\\
0.0885603941936343	0.00152587890625	\\
0.0886047853686687	0.001678466796875	\\
0.0886491765437031	0.001678466796875	\\
0.0886935677187375	0.002044677734375	\\
0.0887379588937719	0.00250244140625	\\
0.0887823500688063	0.002349853515625	\\
0.0888267412438407	0.0025634765625	\\
0.0888711324188751	0.002960205078125	\\
0.0889155235939095	0.00323486328125	\\
0.0889599147689439	0.00360107421875	\\
0.0890043059439783	0.003875732421875	\\
0.0890486971190127	0.00421142578125	\\
0.0890930882940471	0.003997802734375	\\
0.0891374794690815	0.004150390625	\\
0.089181870644116	0.00445556640625	\\
0.0892262618191504	0.004486083984375	\\
0.0892706529941848	0.00482177734375	\\
0.0893150441692192	0.00469970703125	\\
0.0893594353442536	0.0050048828125	\\
0.089403826519288	0.0052490234375	\\
0.0894482176943224	0.00457763671875	\\
0.0894926088693568	0.0045166015625	\\
0.0895370000443912	0.004486083984375	\\
0.0895813912194256	0.00408935546875	\\
0.08962578239446	0.003936767578125	\\
0.0896701735694944	0.003570556640625	\\
0.0897145647445288	0.00335693359375	\\
0.0897589559195632	0.003387451171875	\\
0.0898033470945976	0.00311279296875	\\
0.089847738269632	0.002899169921875	\\
0.0898921294446664	0.002899169921875	\\
0.0899365206197008	0.002777099609375	\\
0.0899809117947352	0.002655029296875	\\
0.0900253029697696	0.002593994140625	\\
0.090069694144804	0.00225830078125	\\
0.0901140853198384	0.00146484375	\\
0.0901584764948728	0.001739501953125	\\
0.0902028676699072	0.001708984375	\\
0.0902472588449416	0.001708984375	\\
0.090291650019976	0.001739501953125	\\
0.0903360411950104	0.00115966796875	\\
0.0903804323700448	0.00128173828125	\\
0.0904248235450792	0.0015869140625	\\
0.0904692147201136	0.001708984375	\\
0.0905136058951481	0.0009765625	\\
0.0905579970701824	0.001007080078125	\\
0.0906023882452169	0.0015869140625	\\
0.0906467794202512	0.0015869140625	\\
0.0906911705952857	0.001922607421875	\\
0.0907355617703201	0.00244140625	\\
0.0907799529453545	0.002532958984375	\\
0.0908243441203889	0.002105712890625	\\
0.0908687352954233	0.002227783203125	\\
0.0909131264704577	0.002593994140625	\\
0.0909575176454921	0.002899169921875	\\
0.0910019088205265	0.00341796875	\\
0.0910462999955609	0.0032958984375	\\
0.0910906911705953	0.0032958984375	\\
0.0911350823456297	0.003570556640625	\\
0.0911794735206641	0.003631591796875	\\
0.0912238646956985	0.003387451171875	\\
0.0912682558707329	0.00384521484375	\\
0.0913126470457673	0.004058837890625	\\
0.0913570382208017	0.0042724609375	\\
0.0914014293958361	0.004241943359375	\\
0.0914458205708705	0.003875732421875	\\
0.0914902117459049	0.003875732421875	\\
0.0915346029209393	0.003936767578125	\\
0.0915789940959737	0.003875732421875	\\
0.0916233852710081	0.003326416015625	\\
0.0916677764460425	0.0030517578125	\\
0.0917121676210769	0.00286865234375	\\
0.0917565587961113	0.00213623046875	\\
0.0918009499711457	0.00201416015625	\\
0.0918453411461801	0.0018310546875	\\
0.0918897323212145	0.00201416015625	\\
0.091934123496249	0.002044677734375	\\
0.0919785146712833	0.001708984375	\\
0.0920229058463178	0.00140380859375	\\
0.0920672970213521	0.001556396484375	\\
0.0921116881963866	0.001495361328125	\\
0.092156079371421	0.001861572265625	\\
0.0922004705464554	0.001800537109375	\\
0.0922448617214898	0.00146484375	\\
0.0922892528965242	0.001312255859375	\\
0.0923336440715586	0.001068115234375	\\
0.092378035246593	0.001251220703125	\\
0.0924224264216274	0.001373291015625	\\
0.0924668175966618	0.001312255859375	\\
0.0925112087716962	0.001068115234375	\\
0.0925555999467306	0.001373291015625	\\
0.092599991121765	0.001495361328125	\\
0.0926443822967994	0.0020751953125	\\
0.0926887734718338	0.00250244140625	\\
0.0927331646468682	0.002532958984375	\\
0.0927775558219026	0.002716064453125	\\
0.092821946996937	0.003204345703125	\\
0.0928663381719714	0.003265380859375	\\
0.0929107293470058	0.00360107421875	\\
0.0929551205220402	0.003662109375	\\
0.0929995116970746	0.00347900390625	\\
0.093043902872109	0.003631591796875	\\
0.0930882940471434	0.003753662109375	\\
0.0931326852221778	0.003814697265625	\\
0.0931770763972122	0.003753662109375	\\
0.0932214675722466	0.004150390625	\\
0.093265858747281	0.003936767578125	\\
0.0933102499223154	0.003570556640625	\\
0.0933546410973499	0.0035400390625	\\
0.0933990322723842	0.003143310546875	\\
0.0934434234474187	0.003021240234375	\\
0.0934878146224531	0.00262451171875	\\
0.0935322057974875	0.00238037109375	\\
0.0935765969725219	0.0023193359375	\\
0.0936209881475563	0.002044677734375	\\
0.0936653793225907	0.001220703125	\\
0.0937097704976251	0.0009765625	\\
0.0937541616726595	0.000885009765625	\\
0.0937985528476939	0.00067138671875	\\
0.0938429440227283	0.00054931640625	\\
0.0938873351977627	0.000457763671875	\\
0.0939317263727971	0.000762939453125	\\
0.0939761175478315	0.000152587890625	\\
0.0940205087228659	-9.1552734375e-05	\\
0.0940648998979003	0	\\
0.0941092910729347	-0.0001220703125	\\
0.0941536822479691	-0.000213623046875	\\
0.0941980734230035	-0.000244140625	\\
0.0942424645980379	-3.0517578125e-05	\\
0.0942868557730723	-3.0517578125e-05	\\
0.0943312469481067	0.000152587890625	\\
0.0943756381231411	0.000274658203125	\\
0.0944200292981755	0.0008544921875	\\
0.0944644204732099	0.001739501953125	\\
0.0945088116482443	0.001495361328125	\\
0.0945532028232787	0.001922607421875	\\
0.0945975939983131	0.00244140625	\\
0.0946419851733475	0.002777099609375	\\
0.0946863763483819	0.002960205078125	\\
0.0947307675234163	0.003143310546875	\\
0.0947751586984508	0.003448486328125	\\
0.0948195498734851	0.0037841796875	\\
0.0948639410485196	0.004058837890625	\\
0.094908332223554	0.003997802734375	\\
0.0949527233985884	0.003936767578125	\\
0.0949971145736228	0.004180908203125	\\
0.0950415057486572	0.0042724609375	\\
0.0950858969236916	0.003997802734375	\\
0.095130288098726	0.004058837890625	\\
0.0951746792737604	0.00396728515625	\\
0.0952190704487948	0.004302978515625	\\
0.0952634616238292	0.0040283203125	\\
0.0953078527988636	0.00390625	\\
0.095352243973898	0.004364013671875	\\
0.0953966351489324	0.00341796875	\\
0.0954410263239668	0.0028076171875	\\
0.0954854174990012	0.00311279296875	\\
0.0955298086740356	0.002716064453125	\\
0.09557419984907	0.002655029296875	\\
0.0956185910241044	0.002166748046875	\\
0.0956629821991388	0.001678466796875	\\
0.0957073733741732	0.001434326171875	\\
0.0957517645492076	0.00091552734375	\\
0.095796155724242	0.000640869140625	\\
0.0958405468992764	0.000457763671875	\\
0.0958849380743108	-0.00018310546875	\\
0.0959293292493452	-0.0001220703125	\\
0.0959737204243796	-0.000274658203125	\\
0.096018111599414	-0.000518798828125	\\
0.0960625027744484	-0.000396728515625	\\
0.0961068939494828	-0.00042724609375	\\
0.0961512851245172	-0.00030517578125	\\
0.0961956762995517	-0.000396728515625	\\
0.0962400674745861	-0.00030517578125	\\
0.0962844586496205	-0.000396728515625	\\
0.0963288498246549	-0.0006103515625	\\
0.0963732409996893	0.0001220703125	\\
0.0964176321747237	0.0001220703125	\\
0.0964620233497581	9.1552734375e-05	\\
0.0965064145247925	0	\\
0.0965508056998269	0.00048828125	\\
0.0965951968748613	0.00115966796875	\\
0.0966395880498957	0.001007080078125	\\
0.0966839792249301	0.001068115234375	\\
0.0967283703999645	0.0013427734375	\\
0.0967727615749989	0.00140380859375	\\
0.0968171527500333	0.001373291015625	\\
0.0968615439250677	0.00201416015625	\\
0.0969059351001021	0.002044677734375	\\
0.0969503262751365	0.00213623046875	\\
0.0969947174501709	0.00250244140625	\\
0.0970391086252053	0.0023193359375	\\
0.0970834998002397	0.002349853515625	\\
0.0971278909752741	0.002838134765625	\\
0.0971722821503085	0.002655029296875	\\
0.0972166733253429	0.002410888671875	\\
0.0972610645003773	0.002532958984375	\\
0.0973054556754117	0.002471923828125	\\
0.0973498468504461	0.002197265625	\\
0.0973942380254805	0.002044677734375	\\
0.0974386292005149	0.00152587890625	\\
0.0974830203755493	0.001373291015625	\\
0.0975274115505837	0.001129150390625	\\
0.0975718027256181	0.00067138671875	\\
0.0976161939006526	0.000762939453125	\\
0.097660585075687	0.0006103515625	\\
0.0977049762507214	6.103515625e-05	\\
0.0977493674257558	0.0001220703125	\\
0.0977937586007902	-0.0001220703125	\\
0.0978381497758246	-0.0003662109375	\\
0.097882540950859	-0.000152587890625	\\
0.0979269321258934	-0.00054931640625	\\
0.0979713233009278	-0.000518798828125	\\
0.0980157144759622	-0.00054931640625	\\
0.0980601056509966	-0.00054931640625	\\
0.098104496826031	-0.000579833984375	\\
0.0981488880010654	-0.00054931640625	\\
0.0981932791760998	-0.0008544921875	\\
0.0982376703511342	-0.001068115234375	\\
0.0982820615261686	-0.00054931640625	\\
0.098326452701203	-0.000732421875	\\
0.0983708438762374	-0.000946044921875	\\
0.0984152350512718	-0.000518798828125	\\
0.0984596262263062	-0.0003662109375	\\
0.0985040174013406	-0.000518798828125	\\
0.098548408576375	-0.00042724609375	\\
0.0985927997514094	-0.0001220703125	\\
0.0986371909264438	0.00054931640625	\\
0.0986815821014782	0.0009765625	\\
0.0987259732765126	0.001251220703125	\\
0.098770364451547	0.001556396484375	\\
0.0988147556265814	0.001190185546875	\\
0.0988591468016158	0.00140380859375	\\
0.0989035379766502	0.0018310546875	\\
0.0989479291516846	0.0015869140625	\\
0.098992320326719	0.001800537109375	\\
0.0990367115017535	0.002288818359375	\\
0.0990811026767879	0.00238037109375	\\
0.0991254938518223	0.0023193359375	\\
0.0991698850268567	0.00262451171875	\\
0.0992142762018911	0.002532958984375	\\
0.0992586673769255	0.00262451171875	\\
0.0993030585519599	0.003173828125	\\
0.0993474497269943	0.00311279296875	\\
0.0993918409020287	0.003082275390625	\\
0.0994362320770631	0.00323486328125	\\
0.0994806232520975	0.00299072265625	\\
0.0995250144271319	0.002685546875	\\
0.0995694056021663	0.003143310546875	\\
0.0996137967772007	0.0028076171875	\\
0.0996581879522351	0.002349853515625	\\
0.0997025791272695	0.002288818359375	\\
0.0997469703023039	0.001922607421875	\\
0.0997913614773383	0.0015869140625	\\
0.0998357526523727	0.001373291015625	\\
0.0998801438274071	0.00128173828125	\\
0.0999245350024415	0.00146484375	\\
0.0999689261774759	0.00091552734375	\\
0.10001331735251	0.000762939453125	\\
0.100057708527545	0.00091552734375	\\
0.100102099702579	0.000640869140625	\\
0.100146490877614	0.000885009765625	\\
0.100190882052648	0.00054931640625	\\
0.100235273227682	0.00042724609375	\\
0.100279664402717	0.0006103515625	\\
0.100324055577751	0.000213623046875	\\
0.100368446752786	0.00018310546875	\\
0.10041283792782	0.000732421875	\\
0.100457229102854	0.000762939453125	\\
0.100501620277889	0.000579833984375	\\
0.100546011452923	0.000396728515625	\\
0.100590402627958	0.00042724609375	\\
0.100634793802992	0.000396728515625	\\
0.100679184978026	0.00067138671875	\\
0.100723576153061	0.000946044921875	\\
0.100767967328095	0.00091552734375	\\
0.10081235850313	0.00146484375	\\
0.100856749678164	0.00177001953125	\\
0.100901140853198	0.00201416015625	\\
0.100945532028233	0.002716064453125	\\
0.100989923203267	0.00274658203125	\\
0.101034314378302	0.002685546875	\\
0.101078705553336	0.003082275390625	\\
0.10112309672837	0.00274658203125	\\
0.101167487903405	0.002685546875	\\
0.101211879078439	0.0028076171875	\\
0.101256270253474	0.002960205078125	\\
0.101300661428508	0.003021240234375	\\
0.101345052603542	0.00238037109375	\\
0.101389443778577	0.0020751953125	\\
0.101433834953611	0.0020751953125	\\
0.101478226128646	0.00201416015625	\\
0.10152261730368	0.0018310546875	\\
0.101567008478714	0.002166748046875	\\
0.101611399653749	0.002166748046875	\\
0.101655790828783	0.0018310546875	\\
0.101700182003818	0.001739501953125	\\
0.101744573178852	0.00152587890625	\\
0.101788964353886	0.00128173828125	\\
0.101833355528921	0.000762939453125	\\
0.101877746703955	0.0006103515625	\\
0.10192213787899	-0.000152587890625	\\
0.101966529054024	-0.0003662109375	\\
0.102010920229058	-0.00030517578125	\\
0.102055311404093	-0.000244140625	\\
0.102099702579127	-0.00079345703125	\\
0.102144093754162	-0.0008544921875	\\
0.102188484929196	-0.00054931640625	\\
0.10223287610423	-0.00054931640625	\\
0.102277267279265	-0.000579833984375	\\
0.102321658454299	-0.000640869140625	\\
0.102366049629334	-0.00048828125	\\
0.102410440804368	-0.000732421875	\\
0.102454831979403	-0.000701904296875	\\
0.102499223154437	-0.00054931640625	\\
0.102543614329471	-0.000640869140625	\\
0.102588005504506	-0.0006103515625	\\
0.10263239667954	-0.00048828125	\\
0.102676787854575	-0.000579833984375	\\
0.102721179029609	-0.000579833984375	\\
0.102765570204643	-0.000457763671875	\\
0.102809961379678	-0.000274658203125	\\
0.102854352554712	3.0517578125e-05	\\
0.102898743729747	0.00042724609375	\\
0.102943134904781	0.00030517578125	\\
0.102987526079815	0.000213623046875	\\
0.10303191725485	0.000732421875	\\
0.103076308429884	0.000701904296875	\\
0.103120699604919	0.000732421875	\\
0.103165090779953	0.00079345703125	\\
0.103209481954987	0.000457763671875	\\
0.103253873130022	0.000274658203125	\\
0.103298264305056	0.00030517578125	\\
0.103342655480091	0.000396728515625	\\
0.103387046655125	0.000152587890625	\\
0.103431437830159	0.0001220703125	\\
0.103475829005194	0.00018310546875	\\
0.103520220180228	3.0517578125e-05	\\
0.103564611355263	-0.000213623046875	\\
0.103609002530297	-0.00042724609375	\\
0.103653393705331	-0.000579833984375	\\
0.103697784880366	-0.00054931640625	\\
0.1037421760554	-0.00115966796875	\\
0.103786567230435	-0.0018310546875	\\
0.103830958405469	-0.001708984375	\\
0.103875349580503	-0.001678466796875	\\
0.103919740755538	-0.002044677734375	\\
0.103964131930572	-0.0018310546875	\\
0.104008523105607	-0.001495361328125	\\
0.104052914280641	-0.001708984375	\\
0.104097305455675	-0.001861572265625	\\
0.10414169663071	-0.002044677734375	\\
0.104186087805744	-0.0023193359375	\\
0.104230478980779	-0.002288818359375	\\
0.104274870155813	-0.002288818359375	\\
0.104319261330847	-0.002166748046875	\\
0.104363652505882	-0.001678466796875	\\
0.104408043680916	-0.001617431640625	\\
0.104452434855951	-0.001220703125	\\
0.104496826030985	-0.0009765625	\\
0.104541217206019	-0.000885009765625	\\
0.104585608381054	-0.000823974609375	\\
0.104629999556088	-0.000335693359375	\\
0.104674390731123	0.00042724609375	\\
0.104718781906157	0.0003662109375	\\
0.104763173081191	0.0006103515625	\\
0.104807564256226	0.000640869140625	\\
0.10485195543126	0.0003662109375	\\
0.104896346606295	0.0008544921875	\\
0.104940737781329	0.00146484375	\\
0.104985128956363	0.00152587890625	\\
0.105029520131398	0.001739501953125	\\
0.105073911306432	0.0018310546875	\\
0.105118302481467	0.001953125	\\
0.105162693656501	0.001983642578125	\\
0.105207084831535	0.001922607421875	\\
0.10525147600657	0.001983642578125	\\
0.105295867181604	0.001617431640625	\\
0.105340258356639	0.0015869140625	\\
0.105384649531673	0.00115966796875	\\
0.105429040706708	0.00054931640625	\\
0.105473431881742	0.000732421875	\\
0.105517823056776	0.0006103515625	\\
0.105562214231811	0.000396728515625	\\
0.105606605406845	0.00042724609375	\\
0.10565099658188	0.00030517578125	\\
0.105695387756914	0.000152587890625	\\
0.105739778931948	-0.000152587890625	\\
0.105784170106983	-0.000518798828125	\\
0.105828561282017	-0.001068115234375	\\
0.105872952457052	-0.000823974609375	\\
0.105917343632086	-0.00103759765625	\\
0.10596173480712	-0.001708984375	\\
0.106006125982155	-0.001251220703125	\\
0.106050517157189	-0.001220703125	\\
0.106094908332224	-0.001617431640625	\\
0.106139299507258	-0.0015869140625	\\
0.106183690682292	-0.001861572265625	\\
0.106228081857327	-0.00164794921875	\\
0.106272473032361	-0.001495361328125	\\
0.106316864207396	-0.001251220703125	\\
0.10636125538243	-0.001251220703125	\\
0.106405646557464	-0.001556396484375	\\
0.106450037732499	-0.001190185546875	\\
0.106494428907533	-0.001312255859375	\\
0.106538820082568	-0.000823974609375	\\
0.106583211257602	-0.000823974609375	\\
0.106627602432636	-0.000640869140625	\\
0.106671993607671	-0.0001220703125	\\
0.106716384782705	-0.00030517578125	\\
0.10676077595774	6.103515625e-05	\\
0.106805167132774	0.00018310546875	\\
0.106849558307808	-0.0001220703125	\\
0.106893949482843	3.0517578125e-05	\\
0.106938340657877	0.00018310546875	\\
0.106982731832912	-0.000152587890625	\\
0.107027123007946	-6.103515625e-05	\\
0.10707151418298	-0.00042724609375	\\
0.107115905358015	-0.000335693359375	\\
0.107160296533049	0.0001220703125	\\
0.107204687708084	-0.0006103515625	\\
0.107249078883118	-0.0006103515625	\\
0.107293470058152	-0.000274658203125	\\
0.107337861233187	-0.000335693359375	\\
0.107382252408221	-0.00018310546875	\\
0.107426643583256	-0.00054931640625	\\
0.10747103475829	-0.0008544921875	\\
0.107515425933324	-0.001129150390625	\\
0.107559817108359	-0.001617431640625	\\
0.107604208283393	-0.001495361328125	\\
0.107648599458428	-0.0018310546875	\\
0.107692990633462	-0.002532958984375	\\
0.107737381808496	-0.002532958984375	\\
0.107781772983531	-0.00225830078125	\\
0.107826164158565	-0.002197265625	\\
0.1078705553336	-0.002410888671875	\\
0.107914946508634	-0.002166748046875	\\
0.107959337683668	-0.001983642578125	\\
0.108003728858703	-0.002105712890625	\\
0.108048120033737	-0.002655029296875	\\
0.108092511208772	-0.00244140625	\\
0.108136902383806	-0.00213623046875	\\
0.108181293558841	-0.002166748046875	\\
0.108225684733875	-0.00201416015625	\\
0.108270075908909	-0.001861572265625	\\
0.108314467083944	-0.0018310546875	\\
0.108358858258978	-0.001861572265625	\\
0.108403249434013	-0.001617431640625	\\
0.108447640609047	-0.001739501953125	\\
0.108492031784081	-0.001617431640625	\\
0.108536422959116	-0.00115966796875	\\
0.10858081413415	-0.0010986328125	\\
0.108625205309185	-0.000640869140625	\\
0.108669596484219	-0.000213623046875	\\
0.108713987659253	0.0003662109375	\\
0.108758378834288	0.000244140625	\\
0.108802770009322	0.00018310546875	\\
0.108847161184357	0.00067138671875	\\
0.108891552359391	0.0006103515625	\\
0.108935943534425	0.00103759765625	\\
0.10898033470946	0.001129150390625	\\
0.109024725884494	0.000946044921875	\\
0.109069117059529	0.00115966796875	\\
0.109113508234563	0.00152587890625	\\
0.109157899409597	0.0015869140625	\\
0.109202290584632	0.001678466796875	\\
0.109246681759666	0.00177001953125	\\
0.109291072934701	0.0018310546875	\\
0.109335464109735	0.00189208984375	\\
0.109379855284769	0.0018310546875	\\
0.109424246459804	0.001556396484375	\\
0.109468637634838	0.001373291015625	\\
0.109513028809873	0.0008544921875	\\
0.109557419984907	0.000823974609375	\\
0.109601811159941	0.000732421875	\\
0.109646202334976	0.000457763671875	\\
0.10969059351001	0.00042724609375	\\
0.109734984685045	-0.0001220703125	\\
0.109779375860079	-0.000335693359375	\\
0.109823767035113	-0.000152587890625	\\
0.109868158210148	-0.000396728515625	\\
0.109912549385182	-0.000457763671875	\\
0.109956940560217	-0.000579833984375	\\
0.110001331735251	-0.000885009765625	\\
0.110045722910285	-0.00103759765625	\\
0.11009011408532	-0.001312255859375	\\
0.110134505260354	-0.0013427734375	\\
0.110178896435389	-0.001373291015625	\\
0.110223287610423	-0.0013427734375	\\
0.110267678785457	-0.0013427734375	\\
0.110312069960492	-0.00115966796875	\\
0.110356461135526	-0.000579833984375	\\
0.110400852310561	-0.000701904296875	\\
0.110445243485595	-0.000396728515625	\\
0.110489634660629	-3.0517578125e-05	\\
0.110534025835664	0.0001220703125	\\
0.110578417010698	0.000762939453125	\\
0.110622808185733	0.001007080078125	\\
0.110667199360767	0.0010986328125	\\
0.110711590535801	0.001312255859375	\\
0.110755981710836	0.001373291015625	\\
0.11080037288587	0.00146484375	\\
0.110844764060905	0.001556396484375	\\
0.110889155235939	0.00177001953125	\\
0.110933546410974	0.001800537109375	\\
0.110977937586008	0.00201416015625	\\
0.111022328761042	0.00189208984375	\\
0.111066719936077	0.00164794921875	\\
0.111111111111111	0.00177001953125	\\
0.111155502286146	0.001220703125	\\
0.11119989346118	0.000701904296875	\\
0.111244284636214	0.000732421875	\\
0.111288675811249	0.00079345703125	\\
0.111333066986283	0.00030517578125	\\
0.111377458161318	0.000274658203125	\\
0.111421849336352	9.1552734375e-05	\\
0.111466240511386	-0.000396728515625	\\
0.111510631686421	-0.00042724609375	\\
0.111555022861455	-0.000885009765625	\\
0.11159941403649	-0.00115966796875	\\
0.111643805211524	-0.00146484375	\\
0.111688196386558	-0.001953125	\\
0.111732587561593	-0.002105712890625	\\
0.111776978736627	-0.0020751953125	\\
0.111821369911662	-0.002471923828125	\\
0.111865761086696	-0.0030517578125	\\
0.11191015226173	-0.003143310546875	\\
0.111954543436765	-0.003204345703125	\\
0.111998934611799	-0.003387451171875	\\
0.112043325786834	-0.00347900390625	\\
0.112087716961868	-0.00390625	\\
0.112132108136902	-0.003936767578125	\\
0.112176499311937	-0.0040283203125	\\
0.112220890486971	-0.00390625	\\
0.112265281662006	-0.003326416015625	\\
0.11230967283704	-0.0035400390625	\\
0.112354064012074	-0.00335693359375	\\
0.112398455187109	-0.002960205078125	\\
0.112442846362143	-0.0025634765625	\\
0.112487237537178	-0.00250244140625	\\
0.112531628712212	-0.0020751953125	\\
0.112576019887246	-0.00164794921875	\\
0.112620411062281	-0.00146484375	\\
0.112664802237315	-0.0008544921875	\\
0.11270919341235	-0.000885009765625	\\
0.112753584587384	-0.000701904296875	\\
0.112797975762418	-0.000244140625	\\
0.112842366937453	-0.000213623046875	\\
0.112886758112487	9.1552734375e-05	\\
0.112931149287522	3.0517578125e-05	\\
0.112975540462556	0.000274658203125	\\
0.11301993163759	0.000518798828125	\\
0.113064322812625	0.0006103515625	\\
0.113108713987659	0.000732421875	\\
0.113153105162694	0.000152587890625	\\
0.113197496337728	0.000457763671875	\\
0.113241887512762	0.000732421875	\\
0.113286278687797	0.000244140625	\\
0.113330669862831	6.103515625e-05	\\
0.113375061037866	-0.000152587890625	\\
0.1134194522129	-0.00030517578125	\\
0.113463843387934	-0.00042724609375	\\
0.113508234562969	-0.000579833984375	\\
0.113552625738003	-0.000762939453125	\\
0.113597016913038	-0.00103759765625	\\
0.113641408088072	-0.001617431640625	\\
0.113685799263107	-0.001953125	\\
0.113730190438141	-0.00250244140625	\\
0.113774581613175	-0.002716064453125	\\
0.11381897278821	-0.002655029296875	\\
0.113863363963244	-0.00299072265625	\\
0.113907755138279	-0.0028076171875	\\
0.113952146313313	-0.00274658203125	\\
0.113996537488347	-0.00274658203125	\\
0.114040928663382	-0.002532958984375	\\
0.114085319838416	-0.003021240234375	\\
0.114129711013451	-0.00274658203125	\\
0.114174102188485	-0.0025634765625	\\
0.114218493363519	-0.00262451171875	\\
0.114262884538554	-0.002471923828125	\\
0.114307275713588	-0.002593994140625	\\
0.114351666888623	-0.002410888671875	\\
0.114396058063657	-0.0023193359375	\\
0.114440449238691	-0.001922607421875	\\
0.114484840413726	-0.001708984375	\\
0.11452923158876	-0.00146484375	\\
0.114573622763795	-0.00079345703125	\\
0.114618013938829	-0.00018310546875	\\
0.114662405113863	0.0001220703125	\\
0.114706796288898	0.000701904296875	\\
0.114751187463932	0.001678466796875	\\
0.114795578638967	0.00250244140625	\\
0.114839969814001	0.0029296875	\\
0.114884360989035	0.00311279296875	\\
0.11492875216407	0.003326416015625	\\
0.114973143339104	0.003662109375	\\
0.115017534514139	0.003692626953125	\\
0.115061925689173	0.003875732421875	\\
0.115106316864207	0.00384521484375	\\
0.115150708039242	0.00347900390625	\\
0.115195099214276	0.003631591796875	\\
0.115239490389311	0.00360107421875	\\
0.115283881564345	0.0032958984375	\\
0.115328272739379	0.002838134765625	\\
0.115372663914414	0.00238037109375	\\
0.115417055089448	0.002471923828125	\\
0.115461446264483	0.00262451171875	\\
0.115505837439517	0.002105712890625	\\
0.115550228614551	0.0020751953125	\\
0.115594619789586	0.00152587890625	\\
0.11563901096462	0.001220703125	\\
0.115683402139655	0.00115966796875	\\
0.115727793314689	0.0003662109375	\\
0.115772184489723	-0.000335693359375	\\
0.115816575664758	-0.000701904296875	\\
0.115860966839792	-0.001007080078125	\\
0.115905358014827	-0.00103759765625	\\
0.115949749189861	-0.00115966796875	\\
0.115994140364895	-0.0013427734375	\\
0.11603853153993	-0.00177001953125	\\
0.116082922714964	-0.002288818359375	\\
0.116127313889999	-0.002227783203125	\\
0.116171705065033	-0.00201416015625	\\
0.116216096240067	-0.001708984375	\\
0.116260487415102	-0.00213623046875	\\
0.116304878590136	-0.00201416015625	\\
0.116349269765171	-0.001251220703125	\\
0.116393660940205	-0.00115966796875	\\
0.116438052115239	-0.001129150390625	\\
0.116482443290274	-0.000396728515625	\\
0.116526834465308	-9.1552734375e-05	\\
0.116571225640343	0.00030517578125	\\
0.116615616815377	0.00042724609375	\\
0.116660007990412	0.000152587890625	\\
0.116704399165446	0.000579833984375	\\
0.11674879034048	0.001251220703125	\\
0.116793181515515	0.001190185546875	\\
0.116837572690549	0.00115966796875	\\
0.116881963865584	0.001434326171875	\\
0.116926355040618	0.001220703125	\\
0.116970746215652	0.00128173828125	\\
0.117015137390687	0.001251220703125	\\
0.117059528565721	0.00128173828125	\\
0.117103919740756	0.001312255859375	\\
0.11714831091579	0.0010986328125	\\
0.117192702090824	0.00140380859375	\\
0.117237093265859	0.001373291015625	\\
0.117281484440893	0.001068115234375	\\
0.117325875615928	0.001068115234375	\\
0.117370266790962	0.001007080078125	\\
0.117414657965996	0.001220703125	\\
0.117459049141031	0.000823974609375	\\
0.117503440316065	0.000579833984375	\\
0.1175478314911	0.000732421875	\\
0.117592222666134	0.0006103515625	\\
0.117636613841168	0.0003662109375	\\
0.117681005016203	0.000152587890625	\\
0.117725396191237	0.0003662109375	\\
0.117769787366272	0.000579833984375	\\
0.117814178541306	3.0517578125e-05	\\
0.11785856971634	0.000213623046875	\\
0.117902960891375	0.000518798828125	\\
0.117947352066409	0.000335693359375	\\
0.117991743241444	0.000213623046875	\\
0.118036134416478	0.000213623046875	\\
0.118080525591512	0.0001220703125	\\
0.118124916766547	0.000213623046875	\\
0.118169307941581	0.000823974609375	\\
0.118213699116616	0.00091552734375	\\
0.11825809029165	0.00115966796875	\\
0.118302481466684	0.001312255859375	\\
0.118346872641719	0.001678466796875	\\
0.118391263816753	0.00201416015625	\\
0.118435654991788	0.002166748046875	\\
0.118480046166822	0.00189208984375	\\
0.118524437341856	0.00189208984375	\\
0.118568828516891	0.00201416015625	\\
0.118613219691925	0.001800537109375	\\
0.11865761086696	0.002532958984375	\\
0.118702002041994	0.002532958984375	\\
0.118746393217028	0.002227783203125	\\
0.118790784392063	0.002960205078125	\\
0.118835175567097	0.003082275390625	\\
0.118879566742132	0.002960205078125	\\
0.118923957917166	0.003173828125	\\
0.1189683490922	0.002899169921875	\\
0.119012740267235	0.0028076171875	\\
0.119057131442269	0.003143310546875	\\
0.119101522617304	0.002960205078125	\\
0.119145913792338	0.003326416015625	\\
0.119190304967372	0.0035400390625	\\
0.119234696142407	0.00299072265625	\\
0.119279087317441	0.002899169921875	\\
0.119323478492476	0.0032958984375	\\
0.11936786966751	0.00323486328125	\\
0.119412260842545	0.0029296875	\\
0.119456652017579	0.002532958984375	\\
0.119501043192613	0.00262451171875	\\
0.119545434367648	0.002410888671875	\\
0.119589825542682	0.002410888671875	\\
0.119634216717717	0.00262451171875	\\
0.119678607892751	0.001953125	\\
0.119722999067785	0.001434326171875	\\
0.11976739024282	0.0010986328125	\\
0.119811781417854	0.0008544921875	\\
0.119856172592889	0.00115966796875	\\
0.119900563767923	0.00152587890625	\\
0.119944954942957	0.0008544921875	\\
0.119989346117992	0.000732421875	\\
0.120033737293026	0.000701904296875	\\
0.120078128468061	0.00042724609375	\\
0.120122519643095	0.00042724609375	\\
0.120166910818129	0.000732421875	\\
0.120211301993164	0.001190185546875	\\
0.120255693168198	0.0013427734375	\\
0.120300084343233	0.001373291015625	\\
0.120344475518267	0.00140380859375	\\
0.120388866693301	0.001434326171875	\\
0.120433257868336	0.001129150390625	\\
0.12047764904337	0.00146484375	\\
0.120522040218405	0.001434326171875	\\
0.120566431393439	0.0010986328125	\\
0.120610822568473	0.00128173828125	\\
0.120655213743508	0.001708984375	\\
0.120699604918542	0.001739501953125	\\
0.120743996093577	0.00164794921875	\\
0.120788387268611	0.001251220703125	\\
0.120832778443645	0.001190185546875	\\
0.12087716961868	0.00128173828125	\\
0.120921560793714	0.001251220703125	\\
0.120965951968749	0.00115966796875	\\
0.121010343143783	0.0013427734375	\\
0.121054734318817	0.00146484375	\\
0.121099125493852	0.00146484375	\\
0.121143516668886	0.002105712890625	\\
0.121187907843921	0.002166748046875	\\
0.121232299018955	0.001922607421875	\\
0.121276690193989	0.00177001953125	\\
0.121321081369024	0.001708984375	\\
0.121365472544058	0.001312255859375	\\
0.121409863719093	0.00103759765625	\\
0.121454254894127	0.00030517578125	\\
0.121498646069161	-3.0517578125e-05	\\
0.121543037244196	-3.0517578125e-05	\\
0.12158742841923	-0.000274658203125	\\
0.121631819594265	-0.0003662109375	\\
0.121676210769299	-0.000579833984375	\\
0.121720601944333	-0.00079345703125	\\
0.121764993119368	-0.001007080078125	\\
0.121809384294402	-0.001129150390625	\\
0.121853775469437	-0.000946044921875	\\
0.121898166644471	-0.000732421875	\\
0.121942557819505	-0.0009765625	\\
0.12198694899454	-0.001007080078125	\\
0.122031340169574	-0.001190185546875	\\
0.122075731344609	-0.00152587890625	\\
0.122120122519643	-0.001251220703125	\\
0.122164513694678	-0.00128173828125	\\
0.122208904869712	-0.00115966796875	\\
0.122253296044746	-0.001220703125	\\
0.122297687219781	-0.001495361328125	\\
0.122342078394815	-0.000946044921875	\\
0.12238646956985	-0.000579833984375	\\
0.122430860744884	-0.00054931640625	\\
0.122475251919918	-0.00054931640625	\\
0.122519643094953	-0.000457763671875	\\
0.122564034269987	9.1552734375e-05	\\
0.122608425445022	0.000579833984375	\\
0.122652816620056	0.000244140625	\\
0.12269720779509	0.000579833984375	\\
0.122741598970125	0.00103759765625	\\
0.122785990145159	0.001007080078125	\\
0.122830381320194	0.00091552734375	\\
0.122874772495228	0.000946044921875	\\
0.122919163670262	0.001190185546875	\\
0.122963554845297	0.00115966796875	\\
0.123007946020331	0.001251220703125	\\
0.123052337195366	0.0018310546875	\\
0.1230967283704	0.00140380859375	\\
0.123141119545434	0.00103759765625	\\
0.123185510720469	0.001068115234375	\\
0.123229901895503	0.000640869140625	\\
0.123274293070538	0.000579833984375	\\
0.123318684245572	0.0006103515625	\\
0.123363075420606	0.00042724609375	\\
0.123407466595641	0.00048828125	\\
0.123451857770675	0.000335693359375	\\
0.12349624894571	0.0003662109375	\\
0.123540640120744	0.00048828125	\\
0.123585031295778	0.0003662109375	\\
0.123629422470813	6.103515625e-05	\\
0.123673813645847	-9.1552734375e-05	\\
0.123718204820882	-0.0003662109375	\\
0.123762595995916	-0.000152587890625	\\
0.12380698717095	-0.000213623046875	\\
0.123851378345985	-9.1552734375e-05	\\
0.123895769521019	-3.0517578125e-05	\\
0.123940160696054	0	\\
0.123984551871088	0.0001220703125	\\
0.124028943046122	-0.00018310546875	\\
0.124073334221157	-0.0001220703125	\\
0.124117725396191	0.000213623046875	\\
0.124162116571226	0.000518798828125	\\
0.12420650774626	0.000732421875	\\
0.124250898921294	0.000946044921875	\\
0.124295290096329	0.001068115234375	\\
0.124339681271363	0.00091552734375	\\
0.124384072446398	0.001220703125	\\
0.124428463621432	0.0013427734375	\\
0.124472854796466	0.00146484375	\\
0.124517245971501	0.001495361328125	\\
0.124561637146535	0.002288818359375	\\
0.12460602832157	0.002777099609375	\\
0.124650419496604	0.002899169921875	\\
0.124694810671638	0.0030517578125	\\
0.124739201846673	0.003509521484375	\\
0.124783593021707	0.00335693359375	\\
0.124827984196742	0.003173828125	\\
0.124872375371776	0.003509521484375	\\
0.124916766546811	0.00384521484375	\\
0.124961157721845	0.00341796875	\\
0.125005548896879	0.003021240234375	\\
0.125049940071914	0.003814697265625	\\
0.125094331246948	0.003387451171875	\\
0.125138722421983	0.003173828125	\\
0.125183113597017	0.002960205078125	\\
0.125227504772051	0.002227783203125	\\
0.125271895947086	0.002288818359375	\\
0.12531628712212	0.002105712890625	\\
0.125360678297155	0.00152587890625	\\
0.125405069472189	0.001129150390625	\\
0.125449460647223	0.0010986328125	\\
0.125493851822258	0.000823974609375	\\
0.125538242997292	0.000762939453125	\\
0.125582634172327	0.0003662109375	\\
0.125627025347361	0	\\
0.125671416522395	0.000152587890625	\\
0.12571580769743	-0.000396728515625	\\
0.125760198872464	-0.00054931640625	\\
0.125804590047499	-0.000335693359375	\\
0.125848981222533	-0.00048828125	\\
0.125893372397567	-0.001251220703125	\\
0.125937763572602	-0.001129150390625	\\
0.125982154747636	-0.001007080078125	\\
0.126026545922671	-0.00079345703125	\\
0.126070937097705	-0.00079345703125	\\
0.126115328272739	-0.00067138671875	\\
0.126159719447774	-0.00048828125	\\
0.126204110622808	-0.0006103515625	\\
0.126248501797843	-3.0517578125e-05	\\
0.126292892972877	0.00018310546875	\\
0.126337284147911	3.0517578125e-05	\\
0.126381675322946	0.000213623046875	\\
0.12642606649798	0.00042724609375	\\
0.126470457673015	0.000823974609375	\\
0.126514848848049	0.000885009765625	\\
0.126559240023083	0.00115966796875	\\
0.126603631198118	0.0015869140625	\\
0.126648022373152	0.001373291015625	\\
0.126692413548187	0.001434326171875	\\
0.126736804723221	0.00146484375	\\
0.126781195898255	0.0015869140625	\\
0.12682558707329	0.00164794921875	\\
0.126869978248324	0.002044677734375	\\
0.126914369423359	0.002227783203125	\\
0.126958760598393	0.00201416015625	\\
0.127003151773427	0.001983642578125	\\
0.127047542948462	0.00201416015625	\\
0.127091934123496	0.0018310546875	\\
0.127136325298531	0.001800537109375	\\
0.127180716473565	0.001220703125	\\
0.127225107648599	0.000732421875	\\
0.127269498823634	0.00067138671875	\\
0.127313889998668	0.00042724609375	\\
0.127358281173703	0.000152587890625	\\
0.127402672348737	-0.000213623046875	\\
0.127447063523771	-0.000335693359375	\\
0.127491454698806	-0.000274658203125	\\
0.12753584587384	-6.103515625e-05	\\
0.127580237048875	0.000213623046875	\\
0.127624628223909	0.00054931640625	\\
0.127669019398943	0.000762939453125	\\
0.127713410573978	0.001007080078125	\\
0.127757801749012	0.001373291015625	\\
0.127802192924047	0.001220703125	\\
0.127846584099081	0.001007080078125	\\
0.127890975274116	0.000701904296875	\\
0.12793536644915	0.00042724609375	\\
0.127979757624184	0.000244140625	\\
0.128024148799219	0.000152587890625	\\
0.128068539974253	0.0001220703125	\\
0.128112931149288	0.000885009765625	\\
0.128157322324322	0.0010986328125	\\
0.128201713499356	0.00115966796875	\\
0.128246104674391	0.00164794921875	\\
0.128290495849425	0.001800537109375	\\
0.12833488702446	0.00189208984375	\\
0.128379278199494	0.002410888671875	\\
0.128423669374528	0.0023193359375	\\
0.128468060549563	0.002227783203125	\\
0.128512451724597	0.002349853515625	\\
0.128556842899632	0.001434326171875	\\
0.128601234074666	0.00091552734375	\\
0.1286456252497	0.000732421875	\\
0.128690016424735	0.0009765625	\\
0.128734407599769	0.001068115234375	\\
0.128778798774804	0.001007080078125	\\
0.128823189949838	0.00103759765625	\\
0.128867581124872	0.001129150390625	\\
0.128911972299907	0.001068115234375	\\
0.128956363474941	0.00140380859375	\\
0.129000754649976	0.001007080078125	\\
0.12904514582501	0.0009765625	\\
0.129089537000044	0.001617431640625	\\
0.129133928175079	0.0013427734375	\\
0.129178319350113	0.00103759765625	\\
0.129222710525148	0.00030517578125	\\
0.129267101700182	0.000274658203125	\\
0.129311492875216	0.000579833984375	\\
0.129355884050251	-6.103515625e-05	\\
0.129400275225285	-0.00030517578125	\\
0.12944466640032	-0.00042724609375	\\
0.129489057575354	-0.000732421875	\\
0.129533448750388	-0.0006103515625	\\
0.129577839925423	-0.000823974609375	\\
0.129622231100457	-0.000823974609375	\\
0.129666622275492	-0.0010986328125	\\
0.129711013450526	-0.001495361328125	\\
0.12975540462556	-0.001556396484375	\\
0.129799795800595	-0.001678466796875	\\
0.129844186975629	-0.001373291015625	\\
0.129888578150664	-0.001251220703125	\\
0.129932969325698	-0.0013427734375	\\
0.129977360500732	-0.001220703125	\\
0.130021751675767	-0.001251220703125	\\
0.130066142850801	-0.00091552734375	\\
0.130110534025836	-0.000396728515625	\\
0.13015492520087	-0.0006103515625	\\
0.130199316375904	-0.000885009765625	\\
0.130243707550939	-0.000579833984375	\\
0.130288098725973	-0.00030517578125	\\
0.130332489901008	3.0517578125e-05	\\
0.130376881076042	0.0003662109375	\\
0.130421272251077	0.000274658203125	\\
0.130465663426111	3.0517578125e-05	\\
0.130510054601145	0.00042724609375	\\
0.13055444577618	0.00054931640625	\\
0.130598836951214	0.0001220703125	\\
0.130643228126249	0.000579833984375	\\
0.130687619301283	0.00048828125	\\
0.130732010476317	0.000335693359375	\\
0.130776401651352	0.000274658203125	\\
0.130820792826386	0.00030517578125	\\
0.130865184001421	0.00042724609375	\\
0.130909575176455	3.0517578125e-05	\\
0.130953966351489	-9.1552734375e-05	\\
0.130998357526524	-0.000274658203125	\\
0.131042748701558	-0.000274658203125	\\
0.131087139876593	-0.000396728515625	\\
0.131131531051627	-0.000335693359375	\\
0.131175922226661	-0.0003662109375	\\
0.131220313401696	-0.000885009765625	\\
0.13126470457673	-0.00067138671875	\\
0.131309095751765	-0.000701904296875	\\
0.131353486926799	-0.000701904296875	\\
0.131397878101833	-0.00067138671875	\\
0.131442269276868	-0.000885009765625	\\
0.131486660451902	-0.001190185546875	\\
0.131531051626937	-0.001068115234375	\\
0.131575442801971	-0.00091552734375	\\
0.131619833977005	-0.0008544921875	\\
0.13166422515204	-0.000823974609375	\\
0.131708616327074	-0.001190185546875	\\
0.131753007502109	-0.00103759765625	\\
0.131797398677143	-0.001129150390625	\\
0.131841789852177	-0.0010986328125	\\
0.131886181027212	-0.00091552734375	\\
0.131930572202246	-0.0006103515625	\\
0.131974963377281	-0.00048828125	\\
0.132019354552315	-0.0003662109375	\\
0.132063745727349	-0.000335693359375	\\
0.132108136902384	0.000274658203125	\\
0.132152528077418	3.0517578125e-05	\\
0.132196919252453	-0.00030517578125	\\
0.132241310427487	0.000640869140625	\\
0.132285701602521	0.000518798828125	\\
0.132330092777556	0.0003662109375	\\
0.13237448395259	0.000457763671875	\\
0.132418875127625	0.0003662109375	\\
0.132463266302659	0.000396728515625	\\
0.132507657477693	0.000762939453125	\\
0.132552048652728	0.00067138671875	\\
0.132596439827762	0.001068115234375	\\
0.132640831002797	0.001129150390625	\\
0.132685222177831	0.001220703125	\\
0.132729613352865	0.001068115234375	\\
0.1327740045279	0.001373291015625	\\
0.132818395702934	0.001312255859375	\\
0.132862786877969	0.00091552734375	\\
0.132907178053003	0.00115966796875	\\
0.132951569228037	0.00091552734375	\\
0.132995960403072	0.0009765625	\\
0.133040351578106	0.000823974609375	\\
0.133084742753141	0.000732421875	\\
0.133129133928175	0.00091552734375	\\
0.133173525103209	0.001129150390625	\\
0.133217916278244	0.0008544921875	\\
0.133262307453278	0.00030517578125	\\
0.133306698628313	9.1552734375e-05	\\
0.133351089803347	0	\\
0.133395480978382	-6.103515625e-05	\\
0.133439872153416	-0.00030517578125	\\
0.13348426332845	-0.0001220703125	\\
0.133528654503485	0.00042724609375	\\
0.133573045678519	0.0001220703125	\\
0.133617436853554	-3.0517578125e-05	\\
0.133661828028588	-3.0517578125e-05	\\
0.133706219203622	-0.00042724609375	\\
0.133750610378657	-0.00048828125	\\
0.133795001553691	-0.000640869140625	\\
0.133839392728726	-0.0006103515625	\\
0.13388378390376	-0.000823974609375	\\
0.133928175078794	-0.000823974609375	\\
0.133972566253829	-0.000457763671875	\\
0.134016957428863	-0.000274658203125	\\
0.134061348603898	-9.1552734375e-05	\\
0.134105739778932	-0.000518798828125	\\
0.134150130953966	-0.00048828125	\\
0.134194522129001	3.0517578125e-05	\\
0.134238913304035	0.000457763671875	\\
0.13428330447907	0.0009765625	\\
0.134327695654104	0.001617431640625	\\
0.134372086829138	0.001495361328125	\\
0.134416478004173	0.001495361328125	\\
0.134460869179207	0.001617431640625	\\
0.134505260354242	0.00140380859375	\\
0.134549651529276	0.00164794921875	\\
0.13459404270431	0.0015869140625	\\
0.134638433879345	0.00152587890625	\\
0.134682825054379	0.002044677734375	\\
0.134727216229414	0.002197265625	\\
0.134771607404448	0.001922607421875	\\
0.134815998579482	0.002044677734375	\\
0.134860389754517	0.002349853515625	\\
0.134904780929551	0.00201416015625	\\
0.134949172104586	0.00146484375	\\
0.13499356327962	0.001220703125	\\
0.135037954454654	0.0010986328125	\\
0.135082345629689	0.00115966796875	\\
0.135126736804723	0.00115966796875	\\
0.135171127979758	0.001251220703125	\\
0.135215519154792	0.00079345703125	\\
0.135259910329826	0.00067138671875	\\
0.135304301504861	0.000640869140625	\\
0.135348692679895	0	\\
0.13539308385493	0.000213623046875	\\
0.135437475029964	6.103515625e-05	\\
0.135481866204998	-0.000213623046875	\\
0.135526257380033	0.000213623046875	\\
0.135570648555067	0.00018310546875	\\
0.135615039730102	0.00018310546875	\\
0.135659430905136	0.000518798828125	\\
0.13570382208017	0.000213623046875	\\
0.135748213255205	-6.103515625e-05	\\
0.135792604430239	0.0003662109375	\\
0.135836995605274	0.000274658203125	\\
0.135881386780308	-0.0001220703125	\\
0.135925777955342	0.000244140625	\\
0.135970169130377	0.00030517578125	\\
0.136014560305411	0.000457763671875	\\
0.136058951480446	0.0010986328125	\\
0.13610334265548	0.0010986328125	\\
0.136147733830514	0.00164794921875	\\
0.136192125005549	0.00146484375	\\
0.136236516180583	0.001495361328125	\\
0.136280907355618	0.001861572265625	\\
0.136325298530652	0.001495361328125	\\
0.136369689705687	0.001129150390625	\\
0.136414080880721	0.00115966796875	\\
0.136458472055755	0.00146484375	\\
0.13650286323079	0.00177001953125	\\
0.136547254405824	0.0018310546875	\\
0.136591645580859	0.001800537109375	\\
0.136636036755893	0.001800537109375	\\
0.136680427930927	0.001708984375	\\
0.136724819105962	0.001861572265625	\\
0.136769210280996	0.00225830078125	\\
0.136813601456031	0.0028076171875	\\
0.136857992631065	0.00274658203125	\\
0.136902383806099	0.00225830078125	\\
0.136946774981134	0.00250244140625	\\
0.136991166156168	0.002655029296875	\\
0.137035557331203	0.00213623046875	\\
0.137079948506237	0.00164794921875	\\
0.137124339681271	0.0015869140625	\\
0.137168730856306	0.001312255859375	\\
0.13721312203134	0.00140380859375	\\
0.137257513206375	0.001800537109375	\\
0.137301904381409	0.001495361328125	\\
0.137346295556443	0.001739501953125	\\
0.137390686731478	0.0018310546875	\\
0.137435077906512	0.001678466796875	\\
0.137479469081547	0.002197265625	\\
0.137523860256581	0.002105712890625	\\
0.137568251431615	0.00152587890625	\\
0.13761264260665	0.001739501953125	\\
0.137657033781684	0.001068115234375	\\
0.137701424956719	0.00018310546875	\\
0.137745816131753	0.0006103515625	\\
0.137790207306787	0.00091552734375	\\
0.137834598481822	0.000732421875	\\
0.137878989656856	0.0009765625	\\
0.137923380831891	0.00091552734375	\\
0.137967772006925	0.0008544921875	\\
0.138012163181959	0.001190185546875	\\
0.138056554356994	0.001556396484375	\\
0.138100945532028	0.001495361328125	\\
0.138145336707063	0.001434326171875	\\
0.138189727882097	0.001983642578125	\\
0.138234119057131	0.0020751953125	\\
0.138278510232166	0.00177001953125	\\
0.1383229014072	0.00152587890625	\\
0.138367292582235	0.0013427734375	\\
0.138411683757269	0.000823974609375	\\
0.138456074932303	0.001129150390625	\\
0.138500466107338	0.001495361328125	\\
0.138544857282372	0.000732421875	\\
0.138589248457407	0.000701904296875	\\
0.138633639632441	0.0009765625	\\
0.138678030807475	0.0010986328125	\\
0.13872242198251	0.00115966796875	\\
0.138766813157544	0.0015869140625	\\
0.138811204332579	0.00152587890625	\\
0.138855595507613	0.0009765625	\\
0.138899986682648	0.000885009765625	\\
0.138944377857682	0.00048828125	\\
0.138988769032716	0.000396728515625	\\
0.139033160207751	0.0003662109375	\\
0.139077551382785	0.000274658203125	\\
0.13912194255782	0.0003662109375	\\
0.139166333732854	0.000640869140625	\\
0.139210724907888	0.000732421875	\\
0.139255116082923	0.00042724609375	\\
0.139299507257957	0.000762939453125	\\
0.139343898432992	0.000885009765625	\\
0.139388289608026	0.000457763671875	\\
0.13943268078306	0.00042724609375	\\
0.139477071958095	0.00042724609375	\\
0.139521463133129	0.000640869140625	\\
0.139565854308164	0.000732421875	\\
0.139610245483198	0.000823974609375	\\
0.139654636658232	0.0010986328125	\\
0.139699027833267	0.001312255859375	\\
0.139743419008301	0.001617431640625	\\
0.139787810183336	0.00213623046875	\\
0.13983220135837	0.002655029296875	\\
0.139876592533404	0.003448486328125	\\
0.139920983708439	0.003814697265625	\\
0.139965374883473	0.0037841796875	\\
0.140009766058508	0.00390625	\\
0.140054157233542	0.003662109375	\\
0.140098548408576	0.00341796875	\\
0.140142939583611	0.003509521484375	\\
0.140187330758645	0.00384521484375	\\
0.14023172193368	0.0035400390625	\\
0.140276113108714	0.003021240234375	\\
0.140320504283748	0.00323486328125	\\
0.140364895458783	0.00360107421875	\\
0.140409286633817	0.00384521484375	\\
0.140453677808852	0.003753662109375	\\
0.140498068983886	0.003631591796875	\\
0.14054246015892	0.003265380859375	\\
0.140586851333955	0.002838134765625	\\
0.140631242508989	0.00286865234375	\\
0.140675633684024	0.002471923828125	\\
0.140720024859058	0.00238037109375	\\
0.140764416034092	0.002716064453125	\\
0.140808807209127	0.00244140625	\\
0.140853198384161	0.002166748046875	\\
0.140897589559196	0.0020751953125	\\
0.14094198073423	0.002166748046875	\\
0.140986371909264	0.002197265625	\\
0.141030763084299	0.00201416015625	\\
0.141075154259333	0.001617431640625	\\
0.141119545434368	0.001800537109375	\\
0.141163936609402	0.001953125	\\
0.141208327784436	0.001922607421875	\\
0.141252718959471	0.00164794921875	\\
0.141297110134505	0.00201416015625	\\
0.14134150130954	0.00152587890625	\\
0.141385892484574	0.00091552734375	\\
0.141430283659608	0.001556396484375	\\
0.141474674834643	0.001190185546875	\\
0.141519066009677	0.00079345703125	\\
0.141563457184712	0.00067138671875	\\
0.141607848359746	0.000823974609375	\\
0.14165223953478	0.001373291015625	\\
0.141696630709815	0.00177001953125	\\
0.141741021884849	0.001983642578125	\\
0.141785413059884	0.002166748046875	\\
0.141829804234918	0.0018310546875	\\
0.141874195409953	0.00225830078125	\\
0.141918586584987	0.002685546875	\\
0.141962977760021	0.00238037109375	\\
0.142007368935056	0.002197265625	\\
0.14205176011009	0.0020751953125	\\
0.142096151285125	0.00225830078125	\\
0.142140542460159	0.002349853515625	\\
0.142184933635193	0.002105712890625	\\
0.142229324810228	0.00213623046875	\\
0.142273715985262	0.0025634765625	\\
0.142318107160297	0.002532958984375	\\
0.142362498335331	0.002197265625	\\
0.142406889510365	0.002166748046875	\\
0.1424512806854	0.00274658203125	\\
0.142495671860434	0.00311279296875	\\
0.142540063035469	0.002838134765625	\\
0.142584454210503	0.00274658203125	\\
0.142628845385537	0.002716064453125	\\
0.142673236560572	0.002685546875	\\
0.142717627735606	0.002593994140625	\\
0.142762018910641	0.002044677734375	\\
0.142806410085675	0.001953125	\\
0.142850801260709	0.002197265625	\\
0.142895192435744	0.002227783203125	\\
0.142939583610778	0.0023193359375	\\
0.142983974785813	0.00201416015625	\\
0.143028365960847	0.002197265625	\\
0.143072757135881	0.002349853515625	\\
0.143117148310916	0.0020751953125	\\
0.14316153948595	0.001953125	\\
0.143205930660985	0.001251220703125	\\
0.143250321836019	0.0013427734375	\\
0.143294713011053	0.00152587890625	\\
0.143339104186088	0.00146484375	\\
0.143383495361122	0.001708984375	\\
0.143427886536157	0.00140380859375	\\
0.143472277711191	0.001190185546875	\\
0.143516668886225	0.001251220703125	\\
0.14356106006126	0.0015869140625	\\
0.143605451236294	0.0018310546875	\\
0.143649842411329	0.00201416015625	\\
0.143694233586363	0.002105712890625	\\
0.143738624761397	0.00213623046875	\\
0.143783015936432	0.002471923828125	\\
0.143827407111466	0.002685546875	\\
0.143871798286501	0.0025634765625	\\
0.143916189461535	0.00244140625	\\
0.143960580636569	0.002197265625	\\
0.144004971811604	0.002349853515625	\\
0.144049362986638	0.002777099609375	\\
0.144093754161673	0.0032958984375	\\
0.144138145336707	0.00323486328125	\\
0.144182536511741	0.003082275390625	\\
0.144226927686776	0.002838134765625	\\
0.14427131886181	0.0030517578125	\\
0.144315710036845	0.00311279296875	\\
0.144360101211879	0.003265380859375	\\
0.144404492386913	0.00347900390625	\\
0.144448883561948	0.0030517578125	\\
0.144493274736982	0.002899169921875	\\
0.144537665912017	0.002838134765625	\\
0.144582057087051	0.00225830078125	\\
0.144626448262086	0.0025634765625	\\
0.14467083943712	0.00286865234375	\\
0.144715230612154	0.00250244140625	\\
0.144759621787189	0.00250244140625	\\
0.144804012962223	0.0023193359375	\\
0.144848404137258	0.002197265625	\\
0.144892795312292	0.002532958984375	\\
0.144937186487326	0.00244140625	\\
0.144981577662361	0.00140380859375	\\
0.145025968837395	0.001312255859375	\\
0.14507036001243	0.00146484375	\\
0.145114751187464	0.00115966796875	\\
0.145159142362498	0.001190185546875	\\
0.145203533537533	0.0013427734375	\\
0.145247924712567	0.001129150390625	\\
0.145292315887602	0.000579833984375	\\
0.145336707062636	0.000518798828125	\\
0.14538109823767	0.000396728515625	\\
0.145425489412705	0.0003662109375	\\
0.145469880587739	0.000885009765625	\\
0.145514271762774	0.0013427734375	\\
0.145558662937808	0.001617431640625	\\
0.145603054112842	0.0015869140625	\\
0.145647445287877	0.00103759765625	\\
0.145691836462911	0.0009765625	\\
0.145736227637946	0.00079345703125	\\
0.14578061881298	0.000732421875	\\
0.145825009988014	0.001068115234375	\\
0.145869401163049	0.0010986328125	\\
0.145913792338083	0.00091552734375	\\
0.145958183513118	0.00103759765625	\\
0.146002574688152	0.001495361328125	\\
0.146046965863186	0.001312255859375	\\
0.146091357038221	0.000946044921875	\\
0.146135748213255	0.001190185546875	\\
0.14618013938829	0.00140380859375	\\
0.146224530563324	0.001922607421875	\\
0.146268921738358	0.002349853515625	\\
0.146313312913393	0.00189208984375	\\
0.146357704088427	0.001800537109375	\\
0.146402095263462	0.002288818359375	\\
0.146446486438496	0.001953125	\\
0.14649087761353	0.001861572265625	\\
0.146535268788565	0.0020751953125	\\
0.146579659963599	0.002166748046875	\\
0.146624051138634	0.00238037109375	\\
0.146668442313668	0.0023193359375	\\
0.146712833488702	0.00201416015625	\\
0.146757224663737	0.002197265625	\\
0.146801615838771	0.00213623046875	\\
0.146846007013806	0.0023193359375	\\
0.14689039818884	0.002655029296875	\\
0.146934789363874	0.00238037109375	\\
0.146979180538909	0.002288818359375	\\
0.147023571713943	0.002166748046875	\\
0.147067962888978	0.0018310546875	\\
0.147112354064012	0.001800537109375	\\
0.147156745239046	0.001708984375	\\
0.147201136414081	0.001220703125	\\
0.147245527589115	0.0013427734375	\\
0.14728991876415	0.001495361328125	\\
0.147334309939184	0.001434326171875	\\
0.147378701114219	0.00177001953125	\\
0.147423092289253	0.0018310546875	\\
0.147467483464287	0.00146484375	\\
0.147511874639322	0.001617431640625	\\
0.147556265814356	0.001739501953125	\\
0.147600656989391	0.001129150390625	\\
0.147645048164425	0.000946044921875	\\
0.147689439339459	0.001190185546875	\\
0.147733830514494	0.001251220703125	\\
0.147778221689528	0.00164794921875	\\
0.147822612864563	0.001678466796875	\\
0.147867004039597	0.001312255859375	\\
0.147911395214631	0.00128173828125	\\
0.147955786389666	0.001220703125	\\
0.1480001775647	0.0013427734375	\\
0.148044568739735	0.00115966796875	\\
0.148088959914769	0.00091552734375	\\
0.148133351089803	0.001312255859375	\\
0.148177742264838	0.00115966796875	\\
0.148222133439872	0.000885009765625	\\
0.148266524614907	0.000946044921875	\\
0.148310915789941	0.001068115234375	\\
0.148355306964975	0.0009765625	\\
0.14839969814001	0.000579833984375	\\
0.148444089315044	0.001068115234375	\\
0.148488480490079	0.001739501953125	\\
0.148532871665113	0.00164794921875	\\
0.148577262840147	0.001495361328125	\\
0.148621654015182	0.00146484375	\\
0.148666045190216	0.001922607421875	\\
0.148710436365251	0.0018310546875	\\
0.148754827540285	0.001617431640625	\\
0.148799218715319	0.001708984375	\\
0.148843609890354	0.001556396484375	\\
0.148888001065388	0.00103759765625	\\
0.148932392240423	0.001129150390625	\\
0.148976783415457	0.001312255859375	\\
0.149021174590491	0.001190185546875	\\
0.149065565765526	0.001373291015625	\\
0.14910995694056	0.0009765625	\\
0.149154348115595	0.00067138671875	\\
0.149198739290629	0.0013427734375	\\
0.149243130465663	0.001220703125	\\
0.149287521640698	0.00091552734375	\\
0.149331912815732	0.00152587890625	\\
0.149376303990767	0.001678466796875	\\
0.149420695165801	0.00140380859375	\\
0.149465086340835	0.001068115234375	\\
0.14950947751587	0.001434326171875	\\
0.149553868690904	0.001434326171875	\\
0.149598259865939	0.001220703125	\\
0.149642651040973	0.001129150390625	\\
0.149687042216007	0.000946044921875	\\
0.149731433391042	0.001190185546875	\\
0.149775824566076	0.001190185546875	\\
0.149820215741111	0.00146484375	\\
0.149864606916145	0.00189208984375	\\
0.149908998091179	0.001434326171875	\\
0.149953389266214	0.001251220703125	\\
0.149997780441248	0.001495361328125	\\
0.150042171616283	0.0013427734375	\\
0.150086562791317	0.0013427734375	\\
0.150130953966351	0.001312255859375	\\
0.150175345141386	0.000823974609375	\\
0.15021973631642	0.00115966796875	\\
0.150264127491455	0.0015869140625	\\
0.150308518666489	0.0013427734375	\\
0.150352909841524	0.000946044921875	\\
0.150397301016558	0.000732421875	\\
0.150441692191592	0.0003662109375	\\
0.150486083366627	0.00079345703125	\\
0.150530474541661	0.001007080078125	\\
0.150574865716696	0.00079345703125	\\
0.15061925689173	0.000335693359375	\\
0.150663648066764	0.00030517578125	\\
0.150708039241799	0.0006103515625	\\
0.150752430416833	0.00030517578125	\\
0.150796821591868	9.1552734375e-05	\\
0.150841212766902	-0.0003662109375	\\
0.150885603941936	-0.000762939453125	\\
0.150929995116971	-0.000579833984375	\\
0.150974386292005	-0.000152587890625	\\
0.15101877746704	-0.00054931640625	\\
0.151063168642074	-0.00067138671875	\\
0.151107559817108	-0.000457763671875	\\
0.151151950992143	-0.00067138671875	\\
0.151196342167177	-0.000335693359375	\\
0.151240733342212	-6.103515625e-05	\\
0.151285124517246	-0.00030517578125	\\
0.15132951569228	-0.00042724609375	\\
0.151373906867315	-0.000274658203125	\\
0.151418298042349	3.0517578125e-05	\\
0.151462689217384	-3.0517578125e-05	\\
0.151507080392418	0.000335693359375	\\
0.151551471567452	0.00048828125	\\
0.151595862742487	-6.103515625e-05	\\
0.151640253917521	6.103515625e-05	\\
0.151684645092556	0.000518798828125	\\
0.15172903626759	0.00048828125	\\
0.151773427442624	0.000518798828125	\\
0.151817818617659	0.00054931640625	\\
0.151862209792693	9.1552734375e-05	\\
0.151906600967728	6.103515625e-05	\\
0.151950992142762	0.00054931640625	\\
0.151995383317796	0.00042724609375	\\
0.152039774492831	0.000213623046875	\\
0.152084165667865	3.0517578125e-05	\\
0.1521285568429	0.000274658203125	\\
0.152172948017934	0.00042724609375	\\
0.152217339192968	-3.0517578125e-05	\\
0.152261730368003	-0.00018310546875	\\
0.152306121543037	0.000244140625	\\
0.152350512718072	0.000244140625	\\
0.152394903893106	0.00018310546875	\\
0.15243929506814	0.000335693359375	\\
0.152483686243175	6.103515625e-05	\\
0.152528077418209	0.00018310546875	\\
0.152572468593244	0.00067138671875	\\
0.152616859768278	0.000732421875	\\
0.152661250943312	0.000274658203125	\\
0.152705642118347	0	\\
0.152750033293381	6.103515625e-05	\\
0.152794424468416	-0.00030517578125	\\
0.15283881564345	-0.000457763671875	\\
0.152883206818485	-0.00054931640625	\\
0.152927597993519	-0.000823974609375	\\
0.152971989168553	-0.000579833984375	\\
0.153016380343588	-0.000396728515625	\\
0.153060771518622	-0.00054931640625	\\
0.153105162693657	-0.00048828125	\\
0.153149553868691	-0.000396728515625	\\
0.153193945043725	-0.0006103515625	\\
0.15323833621876	-0.000396728515625	\\
0.153282727393794	-0.000274658203125	\\
0.153327118568829	-0.000335693359375	\\
0.153371509743863	-0.000244140625	\\
0.153415900918897	-0.00042724609375	\\
0.153460292093932	-0.000396728515625	\\
0.153504683268966	-0.000762939453125	\\
0.153549074444001	-0.000823974609375	\\
0.153593465619035	-0.001007080078125	\\
0.153637856794069	-0.001068115234375	\\
0.153682247969104	-0.000732421875	\\
0.153726639144138	-0.00030517578125	\\
0.153771030319173	-0.000640869140625	\\
0.153815421494207	-0.00079345703125	\\
0.153859812669241	-0.00042724609375	\\
0.153904203844276	-0.000457763671875	\\
0.15394859501931	-0.000457763671875	\\
0.153992986194345	-0.00048828125	\\
0.154037377369379	-0.00067138671875	\\
0.154081768544413	-0.001068115234375	\\
0.154126159719448	-0.001007080078125	\\
0.154170550894482	-0.000579833984375	\\
0.154214942069517	-0.000518798828125	\\
0.154259333244551	-0.000885009765625	\\
0.154303724419585	-0.00048828125	\\
0.15434811559462	-0.00042724609375	\\
0.154392506769654	-0.000640869140625	\\
0.154436897944689	0.000274658203125	\\
0.154481289119723	0.00054931640625	\\
0.154525680294757	0.000457763671875	\\
0.154570071469792	0.00079345703125	\\
0.154614462644826	0.001129150390625	\\
0.154658853819861	0.0009765625	\\
0.154703244994895	0.001129150390625	\\
0.154747636169929	0.001190185546875	\\
0.154792027344964	0.000762939453125	\\
0.154836418519998	0.000732421875	\\
0.154880809695033	0.00067138671875	\\
0.154925200870067	0.000518798828125	\\
0.154969592045101	0.0003662109375	\\
0.155013983220136	0.000579833984375	\\
0.15505837439517	0.001007080078125	\\
0.155102765570205	0.00079345703125	\\
0.155147156745239	0.000762939453125	\\
0.155191547920273	0.000732421875	\\
0.155235939095308	0.00079345703125	\\
0.155280330270342	0.000640869140625	\\
0.155324721445377	9.1552734375e-05	\\
0.155369112620411	9.1552734375e-05	\\
0.155413503795445	0.000732421875	\\
0.15545789497048	0.00115966796875	\\
0.155502286145514	0.0010986328125	\\
0.155546677320549	0.00115966796875	\\
0.155591068495583	0.0009765625	\\
0.155635459670617	0.001068115234375	\\
0.155679850845652	0.00128173828125	\\
0.155724242020686	0.001739501953125	\\
0.155768633195721	0.001617431640625	\\
0.155813024370755	0.001800537109375	\\
0.15585741554579	0.001922607421875	\\
0.155901806720824	0.0018310546875	\\
0.155946197895858	0.002410888671875	\\
0.155990589070893	0.0020751953125	\\
0.156034980245927	0.001861572265625	\\
0.156079371420962	0.002227783203125	\\
0.156123762595996	0.00201416015625	\\
0.15616815377103	0.002349853515625	\\
0.156212544946065	0.002166748046875	\\
0.156256936121099	0.0020751953125	\\
0.156301327296134	0.0023193359375	\\
0.156345718471168	0.002471923828125	\\
0.156390109646202	0.00286865234375	\\
0.156434500821237	0.00286865234375	\\
0.156478891996271	0.002716064453125	\\
0.156523283171306	0.002716064453125	\\
0.15656767434634	0.002655029296875	\\
0.156612065521374	0.0029296875	\\
0.156656456696409	0.00323486328125	\\
0.156700847871443	0.002288818359375	\\
0.156745239046478	0.00189208984375	\\
0.156789630221512	0.002044677734375	\\
0.156834021396546	0.00201416015625	\\
0.156878412571581	0.00201416015625	\\
0.156922803746615	0.00177001953125	\\
0.15696719492165	0.00164794921875	\\
0.157011586096684	0.001617431640625	\\
0.157055977271718	0.0018310546875	\\
0.157100368446753	0.001739501953125	\\
0.157144759621787	0.001617431640625	\\
0.157189150796822	0.0009765625	\\
0.157233541971856	0.001495361328125	\\
0.15727793314689	0.00201416015625	\\
0.157322324321925	0.001495361328125	\\
0.157366715496959	0.00189208984375	\\
0.157411106671994	0.00189208984375	\\
0.157455497847028	0.001800537109375	\\
0.157499889022062	0.002166748046875	\\
0.157544280197097	0.002471923828125	\\
0.157588671372131	0.002655029296875	\\
0.157633062547166	0.002899169921875	\\
0.1576774537222	0.00299072265625	\\
0.157721844897234	0.003265380859375	\\
0.157766236072269	0.00335693359375	\\
0.157810627247303	0.003387451171875	\\
0.157855018422338	0.003082275390625	\\
0.157899409597372	0.002685546875	\\
0.157943800772406	0.003173828125	\\
0.157988191947441	0.00323486328125	\\
0.158032583122475	0.003173828125	\\
0.15807697429751	0.003204345703125	\\
0.158121365472544	0.0032958984375	\\
0.158165756647578	0.003509521484375	\\
0.158210147822613	0.003509521484375	\\
0.158254538997647	0.00323486328125	\\
0.158298930172682	0.003173828125	\\
0.158343321347716	0.003509521484375	\\
0.15838771252275	0.002899169921875	\\
0.158432103697785	0.0028076171875	\\
0.158476494872819	0.00311279296875	\\
0.158520886047854	0.00262451171875	\\
0.158565277222888	0.0023193359375	\\
0.158609668397922	0.002227783203125	\\
0.158654059572957	0.002410888671875	\\
0.158698450747991	0.001922607421875	\\
0.158742841923026	0.001983642578125	\\
0.15878723309806	0.002044677734375	\\
0.158831624273095	0.001739501953125	\\
0.158876015448129	0.001800537109375	\\
0.158920406623163	0.00140380859375	\\
0.158964797798198	0.001312255859375	\\
0.159009188973232	0.001617431640625	\\
0.159053580148267	0.00128173828125	\\
0.159097971323301	0.00115966796875	\\
0.159142362498335	0.0008544921875	\\
0.15918675367337	0.00054931640625	\\
0.159231144848404	0.00079345703125	\\
0.159275536023439	0.001190185546875	\\
0.159319927198473	0.001129150390625	\\
0.159364318373507	0.001190185546875	\\
0.159408709548542	0.001708984375	\\
0.159453100723576	0.001708984375	\\
0.159497491898611	0.0015869140625	\\
0.159541883073645	0.0018310546875	\\
0.159586274248679	0.001861572265625	\\
0.159630665423714	0.002197265625	\\
0.159675056598748	0.002410888671875	\\
0.159719447773783	0.00213623046875	\\
0.159763838948817	0.00213623046875	\\
0.159808230123851	0.002105712890625	\\
0.159852621298886	0.001708984375	\\
0.15989701247392	0.002105712890625	\\
0.159941403648955	0.002685546875	\\
0.159985794823989	0.002197265625	\\
0.160030185999023	0.002227783203125	\\
0.160074577174058	0.00201416015625	\\
0.160118968349092	0.0018310546875	\\
0.160163359524127	0.002044677734375	\\
0.160207750699161	0.00164794921875	\\
0.160252141874195	0.0015869140625	\\
0.16029653304923	0.001434326171875	\\
0.160340924224264	0.00128173828125	\\
0.160385315399299	0.0013427734375	\\
0.160429706574333	0.001556396484375	\\
0.160474097749367	0.001190185546875	\\
0.160518488924402	0.001129150390625	\\
0.160562880099436	0.0008544921875	\\
0.160607271274471	0.000701904296875	\\
0.160651662449505	0.0010986328125	\\
0.160696053624539	0.001068115234375	\\
0.160740444799574	0.001007080078125	\\
0.160784835974608	0.00079345703125	\\
0.160829227149643	0.00103759765625	\\
0.160873618324677	0.001190185546875	\\
0.160918009499711	0.001129150390625	\\
0.160962400674746	0.001190185546875	\\
0.16100679184978	0.001220703125	\\
0.161051183024815	0.0006103515625	\\
0.161095574199849	0.0008544921875	\\
0.161139965374883	0.0008544921875	\\
0.161184356549918	0.0010986328125	\\
0.161228747724952	0.001312255859375	\\
0.161273138899987	0.001129150390625	\\
0.161317530075021	0.00152587890625	\\
0.161361921250056	0.001800537109375	\\
0.16140631242509	0.001922607421875	\\
0.161450703600124	0.00225830078125	\\
0.161495094775159	0.00244140625	\\
0.161539485950193	0.002532958984375	\\
0.161583877125228	0.00225830078125	\\
0.161628268300262	0.001983642578125	\\
0.161672659475296	0.002105712890625	\\
0.161717050650331	0.002655029296875	\\
0.161761441825365	0.002777099609375	\\
0.1618058330004	0.0023193359375	\\
0.161850224175434	0.00262451171875	\\
0.161894615350468	0.002685546875	\\
0.161939006525503	0.002655029296875	\\
0.161983397700537	0.002777099609375	\\
0.162027788875572	0.00244140625	\\
0.162072180050606	0.002532958984375	\\
0.16211657122564	0.002655029296875	\\
0.162160962400675	0.0025634765625	\\
0.162205353575709	0.002593994140625	\\
0.162249744750744	0.002471923828125	\\
0.162294135925778	0.001861572265625	\\
0.162338527100812	0.001556396484375	\\
0.162382918275847	0.001190185546875	\\
0.162427309450881	0.001220703125	\\
0.162471700625916	0.0010986328125	\\
0.16251609180095	0.000701904296875	\\
0.162560482975984	0.00079345703125	\\
0.162604874151019	0.001251220703125	\\
0.162649265326053	0.0013427734375	\\
0.162693656501088	0.001190185546875	\\
0.162738047676122	0.00103759765625	\\
0.162782438851156	0.0010986328125	\\
0.162826830026191	0.001556396484375	\\
0.162871221201225	0.00146484375	\\
0.16291561237626	0.00115966796875	\\
0.162960003551294	0.001129150390625	\\
0.163004394726328	0.000579833984375	\\
0.163048785901363	0.000946044921875	\\
0.163093177076397	0.001251220703125	\\
0.163137568251432	0.00079345703125	\\
0.163181959426466	0.001373291015625	\\
0.1632263506015	0.00146484375	\\
0.163270741776535	0.0010986328125	\\
0.163315132951569	0.00146484375	\\
0.163359524126604	0.0018310546875	\\
0.163403915301638	0.001617431640625	\\
0.163448306476672	0.001983642578125	\\
0.163492697651707	0.002105712890625	\\
0.163537088826741	0.00177001953125	\\
0.163581480001776	0.00189208984375	\\
0.16362587117681	0.002044677734375	\\
0.163670262351844	0.00146484375	\\
0.163714653526879	0.001007080078125	\\
0.163759044701913	0.00201416015625	\\
0.163803435876948	0.002227783203125	\\
0.163847827051982	0.0023193359375	\\
0.163892218227016	0.002410888671875	\\
0.163936609402051	0.00225830078125	\\
0.163981000577085	0.00225830078125	\\
0.16402539175212	0.00244140625	\\
0.164069782927154	0.002532958984375	\\
0.164114174102188	0.001983642578125	\\
0.164158565277223	0.001861572265625	\\
0.164202956452257	0.002105712890625	\\
0.164247347627292	0.001861572265625	\\
0.164291738802326	0.001708984375	\\
0.164336129977361	0.00177001953125	\\
0.164380521152395	0.0015869140625	\\
0.164424912327429	0.001983642578125	\\
0.164469303502464	0.0023193359375	\\
0.164513694677498	0.00201416015625	\\
0.164558085852533	0.00201416015625	\\
0.164602477027567	0.00225830078125	\\
0.164646868202601	0.001983642578125	\\
0.164691259377636	0.0020751953125	\\
0.16473565055267	0.00250244140625	\\
0.164780041727705	0.002471923828125	\\
0.164824432902739	0.002349853515625	\\
0.164868824077773	0.0023193359375	\\
0.164913215252808	0.002197265625	\\
0.164957606427842	0.002349853515625	\\
0.165001997602877	0.00244140625	\\
0.165046388777911	0.002288818359375	\\
0.165090779952945	0.002655029296875	\\
0.16513517112798	0.002716064453125	\\
0.165179562303014	0.003204345703125	\\
0.165223953478049	0.00262451171875	\\
0.165268344653083	0.0025634765625	\\
0.165312735828117	0.002899169921875	\\
0.165357127003152	0.0025634765625	\\
0.165401518178186	0.003173828125	\\
0.165445909353221	0.003143310546875	\\
0.165490300528255	0.002838134765625	\\
0.165534691703289	0.003448486328125	\\
0.165579082878324	0.003265380859375	\\
0.165623474053358	0.003448486328125	\\
0.165667865228393	0.00372314453125	\\
0.165712256403427	0.003204345703125	\\
0.165756647578461	0.003265380859375	\\
0.165801038753496	0.00286865234375	\\
0.16584542992853	0.002899169921875	\\
0.165889821103565	0.003875732421875	\\
0.165934212278599	0.00396728515625	\\
0.165978603453633	0.003936767578125	\\
0.166022994628668	0.003814697265625	\\
0.166067385803702	0.003204345703125	\\
0.166111776978737	0.00347900390625	\\
0.166156168153771	0.003326416015625	\\
0.166200559328805	0.00299072265625	\\
0.16624495050384	0.003326416015625	\\
0.166289341678874	0.0032958984375	\\
0.166333732853909	0.003326416015625	\\
0.166378124028943	0.00341796875	\\
0.166422515203977	0.00341796875	\\
0.166466906379012	0.00335693359375	\\
0.166511297554046	0.003143310546875	\\
0.166555688729081	0.0029296875	\\
0.166600079904115	0.003082275390625	\\
0.166644471079149	0.0030517578125	\\
0.166688862254184	0.00286865234375	\\
0.166733253429218	0.003204345703125	\\
0.166777644604253	0.002777099609375	\\
0.166822035779287	0.00244140625	\\
0.166866426954321	0.002227783203125	\\
0.166910818129356	0.002044677734375	\\
0.16695520930439	0.001953125	\\
0.166999600479425	0.00213623046875	\\
0.167043991654459	0.002044677734375	\\
0.167088382829494	0.00128173828125	\\
0.167132774004528	0.001495361328125	\\
0.167177165179562	0.001739501953125	\\
0.167221556354597	0.00140380859375	\\
0.167265947529631	0.001434326171875	\\
0.167310338704666	0.001617431640625	\\
0.1673547298797	0.001708984375	\\
0.167399121054734	0.001251220703125	\\
0.167443512229769	0.001068115234375	\\
0.167487903404803	0.00128173828125	\\
0.167532294579838	0.001495361328125	\\
0.167576685754872	0.00115966796875	\\
0.167621076929906	0.00115966796875	\\
0.167665468104941	0.00140380859375	\\
0.167709859279975	0.001220703125	\\
0.16775425045501	0.000946044921875	\\
0.167798641630044	0.000640869140625	\\
0.167843032805078	0.00067138671875	\\
0.167887423980113	0.00091552734375	\\
0.167931815155147	0.001068115234375	\\
0.167976206330182	0.000885009765625	\\
0.168020597505216	0.000762939453125	\\
0.16806498868025	0.000946044921875	\\
0.168109379855285	0.000762939453125	\\
0.168153771030319	0.0008544921875	\\
0.168198162205354	0.0008544921875	\\
0.168242553380388	0.000640869140625	\\
0.168286944555422	0.00048828125	\\
0.168331335730457	0.00018310546875	\\
0.168375726905491	0.000244140625	\\
0.168420118080526	0.0001220703125	\\
0.16846450925556	0.000213623046875	\\
0.168508900430594	0.000335693359375	\\
0.168553291605629	0	\\
0.168597682780663	-0.0001220703125	\\
0.168642073955698	-0.000213623046875	\\
0.168686465130732	-0.000396728515625	\\
0.168730856305766	-0.000518798828125	\\
0.168775247480801	-0.0008544921875	\\
0.168819638655835	-0.00091552734375	\\
0.16886402983087	-0.000946044921875	\\
0.168908421005904	-0.001190185546875	\\
0.168952812180938	-0.000823974609375	\\
0.168997203355973	-0.000701904296875	\\
0.169041594531007	-0.00079345703125	\\
0.169085985706042	-0.00042724609375	\\
0.169130376881076	-0.000274658203125	\\
0.16917476805611	-0.00018310546875	\\
0.169219159231145	-0.000274658203125	\\
0.169263550406179	-0.00079345703125	\\
0.169307941581214	-0.000762939453125	\\
0.169352332756248	-0.000762939453125	\\
0.169396723931282	-0.001129150390625	\\
0.169441115106317	-0.000946044921875	\\
0.169485506281351	-0.00103759765625	\\
0.169529897456386	-0.001129150390625	\\
0.16957428863142	-0.00091552734375	\\
0.169618679806454	-0.00128173828125	\\
0.169663070981489	-0.00164794921875	\\
0.169707462156523	-0.001739501953125	\\
0.169751853331558	-0.001495361328125	\\
0.169796244506592	-0.001800537109375	\\
0.169840635681627	-0.001922607421875	\\
0.169885026856661	-0.00164794921875	\\
0.169929418031695	-0.001953125	\\
0.16997380920673	-0.001953125	\\
0.170018200381764	-0.001678466796875	\\
0.170062591556799	-0.001953125	\\
0.170106982731833	-0.001800537109375	\\
0.170151373906867	-0.001373291015625	\\
0.170195765081902	-0.0018310546875	\\
0.170240156256936	-0.002227783203125	\\
0.170284547431971	-0.00189208984375	\\
0.170328938607005	-0.00189208984375	\\
0.170373329782039	-0.002105712890625	\\
0.170417720957074	-0.002349853515625	\\
0.170462112132108	-0.002166748046875	\\
0.170506503307143	-0.00177001953125	\\
0.170550894482177	-0.0020751953125	\\
0.170595285657211	-0.002197265625	\\
0.170639676832246	-0.0015869140625	\\
0.17068406800728	-0.0009765625	\\
0.170728459182315	-0.001312255859375	\\
0.170772850357349	-0.00140380859375	\\
0.170817241532383	-0.00103759765625	\\
0.170861632707418	-0.000946044921875	\\
0.170906023882452	-0.000885009765625	\\
0.170950415057487	-0.000518798828125	\\
0.170994806232521	-0.000640869140625	\\
0.171039197407555	-0.000701904296875	\\
0.17108358858259	-0.000579833984375	\\
0.171127979757624	-0.000885009765625	\\
0.171172370932659	-0.000732421875	\\
0.171216762107693	-0.000579833984375	\\
0.171261153282727	-0.0001220703125	\\
0.171305544457762	-9.1552734375e-05	\\
0.171349935632796	-9.1552734375e-05	\\
0.171394326807831	-9.1552734375e-05	\\
0.171438717982865	-0.000457763671875	\\
0.171483109157899	-0.00042724609375	\\
0.171527500332934	-0.000396728515625	\\
0.171571891507968	-0.000732421875	\\
0.171616282683003	-0.00067138671875	\\
0.171660673858037	-0.000701904296875	\\
0.171705065033071	-0.00079345703125	\\
0.171749456208106	-0.0009765625	\\
0.17179384738314	-0.0010986328125	\\
0.171838238558175	-0.00103759765625	\\
0.171882629733209	-0.001129150390625	\\
0.171927020908243	-0.000946044921875	\\
0.171971412083278	-0.00128173828125	\\
0.172015803258312	-0.0009765625	\\
0.172060194433347	-0.00079345703125	\\
0.172104585608381	-0.001007080078125	\\
0.172148976783415	-0.001007080078125	\\
0.17219336795845	-0.001434326171875	\\
0.172237759133484	-0.001129150390625	\\
0.172282150308519	-0.000946044921875	\\
0.172326541483553	-0.000823974609375	\\
0.172370932658587	-0.000701904296875	\\
0.172415323833622	-0.001007080078125	\\
0.172459715008656	-0.000732421875	\\
0.172504106183691	-0.0001220703125	\\
0.172548497358725	-0.000244140625	\\
0.172592888533759	-0.00048828125	\\
0.172637279708794	-0.000457763671875	\\
0.172681670883828	-0.00067138671875	\\
0.172726062058863	-0.0003662109375	\\
0.172770453233897	0.000152587890625	\\
0.172814844408932	-0.00018310546875	\\
0.172859235583966	-0.0001220703125	\\
0.172903626759	-3.0517578125e-05	\\
0.172948017934035	0.00018310546875	\\
0.172992409109069	6.103515625e-05	\\
0.173036800284104	-0.000457763671875	\\
0.173081191459138	-0.000152587890625	\\
0.173125582634172	0.0001220703125	\\
0.173169973809207	0.000244140625	\\
0.173214364984241	0.000244140625	\\
0.173258756159276	3.0517578125e-05	\\
0.17330314733431	0.000213623046875	\\
0.173347538509344	0.000518798828125	\\
0.173391929684379	0.0006103515625	\\
0.173436320859413	0.000396728515625	\\
0.173480712034448	0.00042724609375	\\
0.173525103209482	0.00067138671875	\\
0.173569494384516	0.0010986328125	\\
0.173613885559551	0.00140380859375	\\
0.173658276734585	0.001190185546875	\\
0.17370266790962	0.001129150390625	\\
0.173747059084654	0.0008544921875	\\
0.173791450259688	0.000579833984375	\\
0.173835841434723	0.000579833984375	\\
0.173880232609757	0.000457763671875	\\
0.173924623784792	0.0003662109375	\\
0.173969014959826	0.000396728515625	\\
0.17401340613486	9.1552734375e-05	\\
0.174057797309895	0.00018310546875	\\
0.174102188484929	-0.0001220703125	\\
0.174146579659964	-0.00048828125	\\
0.174190970834998	-0.0001220703125	\\
0.174235362010032	-0.00067138671875	\\
0.174279753185067	-0.000518798828125	\\
0.174324144360101	-0.000274658203125	\\
0.174368535535136	-0.000762939453125	\\
0.17441292671017	-0.00054931640625	\\
0.174457317885204	-0.0003662109375	\\
0.174501709060239	-0.000518798828125	\\
0.174546100235273	-0.000701904296875	\\
0.174590491410308	-0.000244140625	\\
0.174634882585342	0	\\
0.174679273760376	-0.000396728515625	\\
0.174723664935411	-3.0517578125e-05	\\
0.174768056110445	-6.103515625e-05	\\
0.17481244728548	-0.000274658203125	\\
0.174856838460514	-0.000244140625	\\
0.174901229635548	-0.00018310546875	\\
0.174945620810583	3.0517578125e-05	\\
0.174990011985617	0.000335693359375	\\
0.175034403160652	-0.0001220703125	\\
0.175078794335686	-0.0003662109375	\\
0.17512318551072	-6.103515625e-05	\\
0.175167576685755	0	\\
0.175211967860789	-6.103515625e-05	\\
0.175256359035824	6.103515625e-05	\\
0.175300750210858	0.000244140625	\\
0.175345141385892	0.00042724609375	\\
0.175389532560927	0.000213623046875	\\
0.175433923735961	0.000213623046875	\\
0.175478314910996	0.000640869140625	\\
0.17552270608603	6.103515625e-05	\\
0.175567097261065	9.1552734375e-05	\\
0.175611488436099	3.0517578125e-05	\\
0.175655879611133	-0.000579833984375	\\
0.175700270786168	-0.000335693359375	\\
0.175744661961202	-0.000335693359375	\\
0.175789053136237	-0.00054931640625	\\
0.175833444311271	-0.000823974609375	\\
0.175877835486305	-0.00128173828125	\\
0.17592222666134	-0.001190185546875	\\
0.175966617836374	-0.001068115234375	\\
0.176011009011409	-0.001251220703125	\\
0.176055400186443	-0.001434326171875	\\
0.176099791361477	-0.0015869140625	\\
0.176144182536512	-0.001739501953125	\\
0.176188573711546	-0.001556396484375	\\
0.176232964886581	-0.00189208984375	\\
0.176277356061615	-0.002288818359375	\\
0.176321747236649	-0.001983642578125	\\
0.176366138411684	-0.002288818359375	\\
0.176410529586718	-0.0023193359375	\\
0.176454920761753	-0.001983642578125	\\
0.176499311936787	-0.00225830078125	\\
0.176543703111821	-0.002532958984375	\\
0.176588094286856	-0.0023193359375	\\
0.17663248546189	-0.002288818359375	\\
0.176676876636925	-0.002288818359375	\\
0.176721267811959	-0.002044677734375	\\
0.176765658986993	-0.002105712890625	\\
0.176810050162028	-0.002227783203125	\\
0.176854441337062	-0.002197265625	\\
0.176898832512097	-0.00189208984375	\\
0.176943223687131	-0.001983642578125	\\
0.176987614862165	-0.002105712890625	\\
0.1770320060372	-0.00152587890625	\\
0.177076397212234	-0.001556396484375	\\
0.177120788387269	-0.001220703125	\\
0.177165179562303	-0.001220703125	\\
0.177209570737337	-0.00115966796875	\\
0.177253961912372	-0.001129150390625	\\
0.177298353087406	-0.00091552734375	\\
0.177342744262441	-0.00067138671875	\\
0.177387135437475	-0.001220703125	\\
0.177431526612509	-0.001373291015625	\\
0.177475917787544	-0.00152587890625	\\
0.177520308962578	-0.00164794921875	\\
0.177564700137613	-0.00164794921875	\\
};
\addplot [color=blue,solid,forget plot]
  table[row sep=crcr]{
0.177564700137613	-0.00164794921875	\\
0.177609091312647	-0.001068115234375	\\
0.177653482487681	-0.000457763671875	\\
0.177697873662716	-0.000762939453125	\\
0.17774226483775	-0.000732421875	\\
0.177786656012785	-0.0006103515625	\\
0.177831047187819	-0.000640869140625	\\
0.177875438362853	-0.00115966796875	\\
0.177919829537888	-0.001190185546875	\\
0.177964220712922	-0.001220703125	\\
0.178008611887957	-0.00146484375	\\
0.178053003062991	-0.0009765625	\\
0.178097394238025	-0.0009765625	\\
0.17814178541306	-0.001220703125	\\
0.178186176588094	-0.001373291015625	\\
0.178230567763129	-0.001617431640625	\\
0.178274958938163	-0.001983642578125	\\
0.178319350113197	-0.00213623046875	\\
0.178363741288232	-0.002197265625	\\
0.178408132463266	-0.002197265625	\\
0.178452523638301	-0.00225830078125	\\
0.178496914813335	-0.002716064453125	\\
0.17854130598837	-0.00274658203125	\\
0.178585697163404	-0.002655029296875	\\
0.178630088338438	-0.0029296875	\\
0.178674479513473	-0.002838134765625	\\
0.178718870688507	-0.0025634765625	\\
0.178763261863542	-0.00250244140625	\\
0.178807653038576	-0.002227783203125	\\
0.17885204421361	-0.001617431640625	\\
0.178896435388645	-0.001556396484375	\\
0.178940826563679	-0.001983642578125	\\
0.178985217738714	-0.00177001953125	\\
0.179029608913748	-0.00201416015625	\\
0.179074000088782	-0.00177001953125	\\
0.179118391263817	-0.00164794921875	\\
0.179162782438851	-0.00152587890625	\\
0.179207173613886	-0.001251220703125	\\
0.17925156478892	-0.001251220703125	\\
0.179295955963954	-0.001190185546875	\\
0.179340347138989	-0.001434326171875	\\
0.179384738314023	-0.001220703125	\\
0.179429129489058	-0.0013427734375	\\
0.179473520664092	-0.00054931640625	\\
0.179517911839126	-0.000518798828125	\\
0.179562303014161	-0.001068115234375	\\
0.179606694189195	-0.0009765625	\\
0.17965108536423	-0.001068115234375	\\
0.179695476539264	-0.0009765625	\\
0.179739867714298	-0.001220703125	\\
0.179784258889333	-0.001312255859375	\\
0.179828650064367	-0.001129150390625	\\
0.179873041239402	-0.000946044921875	\\
0.179917432414436	-0.00091552734375	\\
0.17996182358947	-0.00115966796875	\\
0.180006214764505	-0.0008544921875	\\
0.180050605939539	-0.0006103515625	\\
0.180094997114574	-0.00128173828125	\\
0.180139388289608	-0.001708984375	\\
0.180183779464642	-0.00164794921875	\\
0.180228170639677	-0.00189208984375	\\
0.180272561814711	-0.002044677734375	\\
0.180316952989746	-0.001861572265625	\\
0.18036134416478	-0.002044677734375	\\
0.180405735339814	-0.001739501953125	\\
0.180450126514849	-0.00152587890625	\\
0.180494517689883	-0.00152587890625	\\
0.180538908864918	-0.001495361328125	\\
0.180583300039952	-0.0015869140625	\\
0.180627691214986	-0.001251220703125	\\
0.180672082390021	-0.001007080078125	\\
0.180716473565055	-0.001007080078125	\\
0.18076086474009	-0.001312255859375	\\
0.180805255915124	-0.0009765625	\\
0.180849647090158	-0.000701904296875	\\
0.180894038265193	-0.0010986328125	\\
0.180938429440227	-0.0008544921875	\\
0.180982820615262	-0.000732421875	\\
0.181027211790296	-0.0010986328125	\\
0.18107160296533	-0.00067138671875	\\
0.181115994140365	-0.000518798828125	\\
0.181160385315399	-0.0003662109375	\\
0.181204776490434	-0.00042724609375	\\
0.181249167665468	-0.00030517578125	\\
0.181293558840502	0.000335693359375	\\
0.181337950015537	0.000213623046875	\\
0.181382341190571	-0.00042724609375	\\
0.181426732365606	0.000335693359375	\\
0.18147112354064	0.00054931640625	\\
0.181515514715675	0.00048828125	\\
0.181559905890709	0.000762939453125	\\
0.181604297065743	0.00054931640625	\\
0.181648688240778	0.00091552734375	\\
0.181693079415812	0.00091552734375	\\
0.181737470590847	0.00054931640625	\\
0.181781861765881	0.00103759765625	\\
0.181826252940915	0.000640869140625	\\
0.18187064411595	9.1552734375e-05	\\
0.181915035290984	0.000518798828125	\\
0.181959426466019	0.000701904296875	\\
0.182003817641053	0.0003662109375	\\
0.182048208816087	0.00030517578125	\\
0.182092599991122	0.0008544921875	\\
0.182136991166156	0.0009765625	\\
0.182181382341191	0.0008544921875	\\
0.182225773516225	0.000762939453125	\\
0.182270164691259	0.000823974609375	\\
0.182314555866294	0.000701904296875	\\
0.182358947041328	0.00048828125	\\
0.182403338216363	0.000579833984375	\\
0.182447729391397	0.00054931640625	\\
0.182492120566431	0.000762939453125	\\
0.182536511741466	0.001068115234375	\\
0.1825809029165	0.001129150390625	\\
0.182625294091535	0.00067138671875	\\
0.182669685266569	0.000457763671875	\\
0.182714076441603	0.0006103515625	\\
0.182758467616638	0.00067138671875	\\
0.182802858791672	0.0003662109375	\\
0.182847249966707	0.000518798828125	\\
0.182891641141741	0.000732421875	\\
0.182936032316775	0.000885009765625	\\
0.18298042349181	0.0009765625	\\
0.183024814666844	0.000732421875	\\
0.183069205841879	0.000457763671875	\\
0.183113597016913	0.000640869140625	\\
0.183157988191947	0.00018310546875	\\
0.183202379366982	-0.000213623046875	\\
0.183246770542016	0.000396728515625	\\
0.183291161717051	0.000732421875	\\
0.183335552892085	0.000213623046875	\\
0.183379944067119	3.0517578125e-05	\\
0.183424335242154	0	\\
0.183468726417188	-9.1552734375e-05	\\
0.183513117592223	0.0001220703125	\\
0.183557508767257	0.000152587890625	\\
0.183601899942291	-0.0001220703125	\\
0.183646291117326	-0.000274658203125	\\
0.18369068229236	-0.000701904296875	\\
0.183735073467395	-0.000335693359375	\\
0.183779464642429	-0.000213623046875	\\
0.183823855817463	-0.0003662109375	\\
0.183868246992498	-0.000396728515625	\\
0.183912638167532	-0.000457763671875	\\
0.183957029342567	-0.0006103515625	\\
0.184001420517601	-0.0006103515625	\\
0.184045811692636	-0.000579833984375	\\
0.18409020286767	-0.000885009765625	\\
0.184134594042704	-0.001190185546875	\\
0.184178985217739	-0.000823974609375	\\
0.184223376392773	-0.000457763671875	\\
0.184267767567808	-0.00054931640625	\\
0.184312158742842	-0.0009765625	\\
0.184356549917876	-0.0010986328125	\\
0.184400941092911	-0.000732421875	\\
0.184445332267945	-0.001007080078125	\\
0.18448972344298	-0.001312255859375	\\
0.184534114618014	-0.001495361328125	\\
0.184578505793048	-0.001861572265625	\\
0.184622896968083	-0.001617431640625	\\
0.184667288143117	-0.001617431640625	\\
0.184711679318152	-0.001861572265625	\\
0.184756070493186	-0.001312255859375	\\
0.18480046166822	-0.001556396484375	\\
0.184844852843255	-0.0015869140625	\\
0.184889244018289	-0.001312255859375	\\
0.184933635193324	-0.001434326171875	\\
0.184978026368358	-0.001434326171875	\\
0.185022417543392	-0.00146484375	\\
0.185066808718427	-0.0013427734375	\\
0.185111199893461	-0.000885009765625	\\
0.185155591068496	-0.001312255859375	\\
0.18519998224353	-0.001220703125	\\
0.185244373418564	-0.00079345703125	\\
0.185288764593599	-0.001190185546875	\\
0.185333155768633	-0.00103759765625	\\
0.185377546943668	-0.001007080078125	\\
0.185421938118702	-0.001190185546875	\\
0.185466329293736	-0.000701904296875	\\
0.185510720468771	-0.000640869140625	\\
0.185555111643805	-0.000946044921875	\\
0.18559950281884	-0.00103759765625	\\
0.185643893993874	-0.00115966796875	\\
0.185688285168908	-0.001129150390625	\\
0.185732676343943	-0.00146484375	\\
0.185777067518977	-0.001495361328125	\\
0.185821458694012	-0.001373291015625	\\
0.185865849869046	-0.00140380859375	\\
0.18591024104408	-0.001129150390625	\\
0.185954632219115	-0.000885009765625	\\
0.185999023394149	-0.0008544921875	\\
0.186043414569184	-0.0009765625	\\
0.186087805744218	-0.0009765625	\\
0.186132196919252	-0.000946044921875	\\
0.186176588094287	-0.0009765625	\\
0.186220979269321	-0.000823974609375	\\
0.186265370444356	-0.001068115234375	\\
0.18630976161939	-0.00091552734375	\\
0.186354152794424	-0.00048828125	\\
0.186398543969459	-0.000762939453125	\\
0.186442935144493	-0.000885009765625	\\
0.186487326319528	-0.00054931640625	\\
0.186531717494562	-0.000640869140625	\\
0.186576108669596	-0.00067138671875	\\
0.186620499844631	-0.000274658203125	\\
0.186664891019665	-0.00067138671875	\\
0.1867092821947	-0.000732421875	\\
0.186753673369734	-0.00091552734375	\\
0.186798064544768	-0.000946044921875	\\
0.186842455719803	-0.000579833984375	\\
0.186886846894837	-0.0003662109375	\\
0.186931238069872	-0.000579833984375	\\
0.186975629244906	-0.000701904296875	\\
0.187020020419941	-0.000640869140625	\\
0.187064411594975	-0.000885009765625	\\
0.187108802770009	-0.00115966796875	\\
0.187153193945044	-0.000762939453125	\\
0.187197585120078	-0.000579833984375	\\
0.187241976295113	-0.00091552734375	\\
0.187286367470147	-0.00146484375	\\
0.187330758645181	-0.001251220703125	\\
0.187375149820216	-0.00128173828125	\\
0.18741954099525	-0.001251220703125	\\
0.187463932170285	-0.00091552734375	\\
0.187508323345319	-0.000885009765625	\\
0.187552714520353	-0.000732421875	\\
0.187597105695388	-0.000732421875	\\
0.187641496870422	-0.00048828125	\\
0.187685888045457	-0.000640869140625	\\
0.187730279220491	-0.001220703125	\\
0.187774670395525	-0.001312255859375	\\
0.18781906157056	-0.00146484375	\\
0.187863452745594	-0.00189208984375	\\
0.187907843920629	-0.0015869140625	\\
0.187952235095663	-0.00146484375	\\
0.187996626270697	-0.00164794921875	\\
0.188041017445732	-0.00152587890625	\\
0.188085408620766	-0.001312255859375	\\
0.188129799795801	-0.00177001953125	\\
0.188174190970835	-0.00152587890625	\\
0.188218582145869	-0.001312255859375	\\
0.188262973320904	-0.001800537109375	\\
0.188307364495938	-0.001708984375	\\
0.188351755670973	-0.001922607421875	\\
0.188396146846007	-0.002197265625	\\
0.188440538021041	-0.0020751953125	\\
0.188484929196076	-0.002044677734375	\\
0.18852932037111	-0.0023193359375	\\
0.188573711546145	-0.0020751953125	\\
0.188618102721179	-0.0018310546875	\\
0.188662493896213	-0.00146484375	\\
0.188706885071248	-0.00091552734375	\\
0.188751276246282	-0.00115966796875	\\
0.188795667421317	-0.00091552734375	\\
0.188840058596351	-0.0006103515625	\\
0.188884449771385	-0.0009765625	\\
0.18892884094642	-0.0013427734375	\\
0.188973232121454	-0.000640869140625	\\
0.189017623296489	-9.1552734375e-05	\\
0.189062014471523	-0.00042724609375	\\
0.189106405646557	-0.00067138671875	\\
0.189150796821592	-0.000274658203125	\\
0.189195187996626	-0.00067138671875	\\
0.189239579171661	-0.000213623046875	\\
0.189283970346695	-0.00018310546875	\\
0.189328361521729	-0.000457763671875	\\
0.189372752696764	-9.1552734375e-05	\\
0.189417143871798	-0.00054931640625	\\
0.189461535046833	-0.00030517578125	\\
0.189505926221867	-0.000152587890625	\\
0.189550317396902	0.000244140625	\\
0.189594708571936	0.000244140625	\\
0.18963909974697	0.00018310546875	\\
0.189683490922005	-6.103515625e-05	\\
0.189727882097039	-0.00030517578125	\\
0.189772273272074	-0.000579833984375	\\
0.189816664447108	-0.000213623046875	\\
0.189861055622142	3.0517578125e-05	\\
0.189905446797177	-0.00042724609375	\\
0.189949837972211	0.00018310546875	\\
0.189994229147246	-0.00030517578125	\\
0.19003862032228	-0.000885009765625	\\
0.190083011497314	-0.000518798828125	\\
0.190127402672349	-0.000701904296875	\\
0.190171793847383	-0.0003662109375	\\
0.190216185022418	-0.000335693359375	\\
0.190260576197452	-9.1552734375e-05	\\
0.190304967372486	-9.1552734375e-05	\\
0.190349358547521	0.0001220703125	\\
0.190393749722555	0.000213623046875	\\
0.19043814089759	-0.000274658203125	\\
0.190482532072624	-0.00018310546875	\\
0.190526923247658	-0.000274658203125	\\
0.190571314422693	0.00018310546875	\\
0.190615705597727	0	\\
0.190660096772762	-0.000213623046875	\\
0.190704487947796	9.1552734375e-05	\\
0.19074887912283	0.00048828125	\\
0.190793270297865	9.1552734375e-05	\\
0.190837661472899	-6.103515625e-05	\\
0.190882052647934	0.000244140625	\\
0.190926443822968	-6.103515625e-05	\\
0.190970834998002	-9.1552734375e-05	\\
0.191015226173037	6.103515625e-05	\\
0.191059617348071	0.00030517578125	\\
0.191104008523106	0.000274658203125	\\
0.19114839969814	0.00048828125	\\
0.191192790873174	0.000640869140625	\\
0.191237182048209	0.000579833984375	\\
0.191281573223243	0.00054931640625	\\
0.191325964398278	0.000762939453125	\\
0.191370355573312	0.00067138671875	\\
0.191414746748346	0.00018310546875	\\
0.191459137923381	0.00018310546875	\\
0.191503529098415	0.00054931640625	\\
0.19154792027345	0.000244140625	\\
0.191592311448484	0.000640869140625	\\
0.191636702623518	0.000244140625	\\
0.191681093798553	-0.000457763671875	\\
0.191725484973587	-0.000457763671875	\\
0.191769876148622	-0.00042724609375	\\
0.191814267323656	-0.00030517578125	\\
0.19185865849869	-0.0006103515625	\\
0.191903049673725	-0.000579833984375	\\
0.191947440848759	-0.000518798828125	\\
0.191991832023794	-0.001007080078125	\\
0.192036223198828	-0.000946044921875	\\
0.192080614373862	-0.00091552734375	\\
0.192125005548897	-0.0013427734375	\\
0.192169396723931	-0.00164794921875	\\
0.192213787898966	-0.00140380859375	\\
0.192258179074	-0.001739501953125	\\
0.192302570249034	-0.001739501953125	\\
0.192346961424069	-0.001922607421875	\\
0.192391352599103	-0.002227783203125	\\
0.192435743774138	-0.002471923828125	\\
0.192480134949172	-0.00250244140625	\\
0.192524526124207	-0.00201416015625	\\
0.192568917299241	-0.001861572265625	\\
0.192613308474275	-0.001739501953125	\\
0.19265769964931	-0.001708984375	\\
0.192702090824344	-0.00146484375	\\
0.192746481999379	-0.001708984375	\\
0.192790873174413	-0.00164794921875	\\
0.192835264349447	-0.00189208984375	\\
0.192879655524482	-0.0023193359375	\\
0.192924046699516	-0.0018310546875	\\
0.192968437874551	-0.001617431640625	\\
0.193012829049585	-0.001983642578125	\\
0.193057220224619	-0.001800537109375	\\
0.193101611399654	-0.0018310546875	\\
0.193146002574688	-0.001312255859375	\\
0.193190393749723	-0.0010986328125	\\
0.193234784924757	-0.001068115234375	\\
0.193279176099791	-0.00091552734375	\\
0.193323567274826	-0.00115966796875	\\
0.19336795844986	-0.00128173828125	\\
0.193412349624895	-0.001312255859375	\\
0.193456740799929	-0.00140380859375	\\
0.193501131974963	-0.0018310546875	\\
0.193545523149998	-0.002105712890625	\\
0.193589914325032	-0.002105712890625	\\
0.193634305500067	-0.00177001953125	\\
0.193678696675101	-0.001861572265625	\\
0.193723087850135	-0.001800537109375	\\
0.19376747902517	-0.00152587890625	\\
0.193811870200204	-0.001678466796875	\\
0.193856261375239	-0.001495361328125	\\
0.193900652550273	-0.0015869140625	\\
0.193945043725307	-0.001739501953125	\\
0.193989434900342	-0.001556396484375	\\
0.194033826075376	-0.00146484375	\\
0.194078217250411	-0.00164794921875	\\
0.194122608425445	-0.0018310546875	\\
0.194166999600479	-0.00152587890625	\\
0.194211390775514	-0.001617431640625	\\
0.194255781950548	-0.001953125	\\
0.194300173125583	-0.00152587890625	\\
0.194344564300617	-0.001220703125	\\
0.194388955475651	-0.001129150390625	\\
0.194433346650686	-0.0009765625	\\
0.19447773782572	-0.0010986328125	\\
0.194522129000755	-0.0013427734375	\\
0.194566520175789	-0.0010986328125	\\
0.194610911350823	-0.000579833984375	\\
0.194655302525858	-0.000579833984375	\\
0.194699693700892	-0.000823974609375	\\
0.194744084875927	-0.00079345703125	\\
0.194788476050961	-0.000762939453125	\\
0.194832867225995	-0.000518798828125	\\
0.19487725840103	-0.000335693359375	\\
0.194921649576064	-0.000335693359375	\\
0.194966040751099	-0.00048828125	\\
0.195010431926133	-0.000640869140625	\\
0.195054823101167	-0.000579833984375	\\
0.195099214276202	-0.0003662109375	\\
0.195143605451236	0.0001220703125	\\
0.195187996626271	6.103515625e-05	\\
0.195232387801305	3.0517578125e-05	\\
0.195276778976339	-6.103515625e-05	\\
0.195321170151374	-6.103515625e-05	\\
0.195365561326408	0.000152587890625	\\
0.195409952501443	9.1552734375e-05	\\
0.195454343676477	-0.00018310546875	\\
0.195498734851512	-0.000274658203125	\\
0.195543126026546	-0.0001220703125	\\
0.19558751720158	0	\\
0.195631908376615	-0.00042724609375	\\
0.195676299551649	-0.0006103515625	\\
0.195720690726684	-3.0517578125e-05	\\
0.195765081901718	-0.00048828125	\\
0.195809473076752	-0.000885009765625	\\
0.195853864251787	-0.00091552734375	\\
0.195898255426821	-0.000823974609375	\\
0.195942646601856	-0.000823974609375	\\
0.19598703777689	-0.000885009765625	\\
0.196031428951924	-0.0008544921875	\\
0.196075820126959	-0.00091552734375	\\
0.196120211301993	-0.000823974609375	\\
0.196164602477028	-0.00091552734375	\\
0.196208993652062	-0.000732421875	\\
0.196253384827096	-0.000579833984375	\\
0.196297776002131	-0.000579833984375	\\
0.196342167177165	0	\\
0.1963865583522	0.0003662109375	\\
0.196430949527234	0.000244140625	\\
0.196475340702268	0.000457763671875	\\
0.196519731877303	0.000518798828125	\\
0.196564123052337	0.0003662109375	\\
0.196608514227372	0.000518798828125	\\
0.196652905402406	0.00048828125	\\
0.19669729657744	0.000396728515625	\\
0.196741687752475	0.00054931640625	\\
0.196786078927509	0.000396728515625	\\
0.196830470102544	0.0006103515625	\\
0.196874861277578	0.000701904296875	\\
0.196919252452612	0.000396728515625	\\
0.196963643627647	0.000335693359375	\\
0.197008034802681	0.000732421875	\\
0.197052425977716	0.000823974609375	\\
0.19709681715275	0.00042724609375	\\
0.197141208327784	0.00042724609375	\\
0.197185599502819	0.000335693359375	\\
0.197229990677853	0.000518798828125	\\
0.197274381852888	0.00146484375	\\
0.197318773027922	0.00140380859375	\\
0.197363164202956	0.001007080078125	\\
0.197407555377991	0.001068115234375	\\
0.197451946553025	0.000946044921875	\\
0.19749633772806	0.00067138671875	\\
0.197540728903094	0.000701904296875	\\
0.197585120078128	0.000762939453125	\\
0.197629511253163	0.000701904296875	\\
0.197673902428197	0.00091552734375	\\
0.197718293603232	0.00091552734375	\\
0.197762684778266	0.000244140625	\\
0.1978070759533	-3.0517578125e-05	\\
0.197851467128335	0.000244140625	\\
0.197895858303369	0.000335693359375	\\
0.197940249478404	0.00054931640625	\\
0.197984640653438	0.0001220703125	\\
0.198029031828473	-3.0517578125e-05	\\
0.198073423003507	0.00054931640625	\\
0.198117814178541	0.000335693359375	\\
0.198162205353576	9.1552734375e-05	\\
0.19820659652861	0.000152587890625	\\
0.198250987703645	3.0517578125e-05	\\
0.198295378878679	0.000274658203125	\\
0.198339770053713	0.000457763671875	\\
0.198384161228748	0.000640869140625	\\
0.198428552403782	0.0010986328125	\\
0.198472943578817	0.000823974609375	\\
0.198517334753851	0.000732421875	\\
0.198561725928885	0.00091552734375	\\
0.19860611710392	0.0006103515625	\\
0.198650508278954	0.00067138671875	\\
0.198694899453989	0.0013427734375	\\
0.198739290629023	0.00128173828125	\\
0.198783681804057	0.00128173828125	\\
0.198828072979092	0.00140380859375	\\
0.198872464154126	0.0009765625	\\
0.198916855329161	0.001312255859375	\\
0.198961246504195	0.001861572265625	\\
0.199005637679229	0.002166748046875	\\
0.199050028854264	0.00213623046875	\\
0.199094420029298	0.0018310546875	\\
0.199138811204333	0.002349853515625	\\
0.199183202379367	0.0023193359375	\\
0.199227593554401	0.001861572265625	\\
0.199271984729436	0.001922607421875	\\
0.19931637590447	0.0023193359375	\\
0.199360767079505	0.0025634765625	\\
0.199405158254539	0.002593994140625	\\
0.199449549429573	0.002197265625	\\
0.199493940604608	0.001678466796875	\\
0.199538331779642	0.002105712890625	\\
0.199582722954677	0.0023193359375	\\
0.199627114129711	0.002044677734375	\\
0.199671505304745	0.001861572265625	\\
0.19971589647978	0.002044677734375	\\
0.199760287654814	0.00225830078125	\\
0.199804678829849	0.00201416015625	\\
0.199849070004883	0.001708984375	\\
0.199893461179917	0.00177001953125	\\
0.199937852354952	0.00146484375	\\
0.199982243529986	0.001617431640625	\\
0.200026634705021	0.00189208984375	\\
0.200071025880055	0.001800537109375	\\
0.200115417055089	0.00177001953125	\\
0.200159808230124	0.001220703125	\\
0.200204199405158	0.001495361328125	\\
0.200248590580193	0.001922607421875	\\
0.200292981755227	0.00164794921875	\\
0.200337372930261	0.002044677734375	\\
0.200381764105296	0.00225830078125	\\
0.20042615528033	0.002227783203125	\\
0.200470546455365	0.0023193359375	\\
0.200514937630399	0.00213623046875	\\
0.200559328805433	0.00177001953125	\\
0.200603719980468	0.0015869140625	\\
0.200648111155502	0.00152587890625	\\
0.200692502330537	0.001434326171875	\\
0.200736893505571	0.00128173828125	\\
0.200781284680605	0.001556396484375	\\
0.20082567585564	0.001434326171875	\\
0.200870067030674	0.00164794921875	\\
0.200914458205709	0.002105712890625	\\
0.200958849380743	0.001983642578125	\\
0.201003240555778	0.002197265625	\\
0.201047631730812	0.002105712890625	\\
0.201092022905846	0.001922607421875	\\
0.201136414080881	0.00189208984375	\\
0.201180805255915	0.0018310546875	\\
0.20122519643095	0.001800537109375	\\
0.201269587605984	0.0015869140625	\\
0.201313978781018	0.001708984375	\\
0.201358369956053	0.001708984375	\\
0.201402761131087	0.001556396484375	\\
0.201447152306122	0.00189208984375	\\
0.201491543481156	0.002105712890625	\\
0.20153593465619	0.001739501953125	\\
0.201580325831225	0.001861572265625	\\
0.201624717006259	0.002105712890625	\\
0.201669108181294	0.001983642578125	\\
0.201713499356328	0.001922607421875	\\
0.201757890531362	0.001983642578125	\\
0.201802281706397	0.001495361328125	\\
0.201846672881431	0.001190185546875	\\
0.201891064056466	0.001251220703125	\\
0.2019354552315	0.001251220703125	\\
0.201979846406534	0.0015869140625	\\
0.202024237581569	0.00177001953125	\\
0.202068628756603	0.00164794921875	\\
0.202113019931638	0.00152587890625	\\
0.202157411106672	0.001495361328125	\\
0.202201802281706	0.00140380859375	\\
0.202246193456741	0.00177001953125	\\
0.202290584631775	0.001953125	\\
0.20233497580681	0.00146484375	\\
0.202379366981844	0.00140380859375	\\
0.202423758156878	0.001678466796875	\\
0.202468149331913	0.00152587890625	\\
0.202512540506947	0.00152587890625	\\
0.202556931681982	0.001739501953125	\\
0.202601322857016	0.001800537109375	\\
0.20264571403205	0.001556396484375	\\
0.202690105207085	0.001708984375	\\
0.202734496382119	0.00213623046875	\\
0.202778887557154	0.001556396484375	\\
0.202823278732188	0.00189208984375	\\
0.202867669907222	0.00238037109375	\\
0.202912061082257	0.002044677734375	\\
0.202956452257291	0.002197265625	\\
0.203000843432326	0.001953125	\\
0.20304523460736	0.002197265625	\\
0.203089625782394	0.002471923828125	\\
0.203134016957429	0.001953125	\\
0.203178408132463	0.001983642578125	\\
0.203222799307498	0.002288818359375	\\
0.203267190482532	0.002288818359375	\\
0.203311581657566	0.00250244140625	\\
0.203355972832601	0.0025634765625	\\
0.203400364007635	0.002349853515625	\\
0.20344475518267	0.0023193359375	\\
0.203489146357704	0.00262451171875	\\
0.203533537532738	0.002685546875	\\
0.203577928707773	0.00274658203125	\\
0.203622319882807	0.00250244140625	\\
0.203666711057842	0.002960205078125	\\
0.203711102232876	0.00323486328125	\\
0.203755493407911	0.00286865234375	\\
0.203799884582945	0.003326416015625	\\
0.203844275757979	0.00384521484375	\\
0.203888666933014	0.00341796875	\\
0.203933058108048	0.00323486328125	\\
0.203977449283083	0.0032958984375	\\
0.204021840458117	0.002899169921875	\\
0.204066231633151	0.003082275390625	\\
0.204110622808186	0.003265380859375	\\
0.20415501398322	0.002960205078125	\\
0.204199405158255	0.00250244140625	\\
0.204243796333289	0.002655029296875	\\
0.204288187508323	0.002838134765625	\\
0.204332578683358	0.00250244140625	\\
0.204376969858392	0.002838134765625	\\
0.204421361033427	0.003021240234375	\\
0.204465752208461	0.00299072265625	\\
0.204510143383495	0.002960205078125	\\
0.20455453455853	0.00311279296875	\\
0.204598925733564	0.003570556640625	\\
0.204643316908599	0.003509521484375	\\
0.204687708083633	0.00323486328125	\\
0.204732099258667	0.002899169921875	\\
0.204776490433702	0.003265380859375	\\
0.204820881608736	0.003753662109375	\\
0.204865272783771	0.003326416015625	\\
0.204909663958805	0.003204345703125	\\
0.204954055133839	0.003265380859375	\\
0.204998446308874	0.00372314453125	\\
0.205042837483908	0.003753662109375	\\
0.205087228658943	0.00372314453125	\\
0.205131619833977	0.003692626953125	\\
0.205176011009011	0.00347900390625	\\
0.205220402184046	0.0032958984375	\\
0.20526479335908	0.00360107421875	\\
0.205309184534115	0.00372314453125	\\
0.205353575709149	0.003509521484375	\\
0.205397966884183	0.004058837890625	\\
0.205442358059218	0.003997802734375	\\
0.205486749234252	0.003692626953125	\\
0.205531140409287	0.00372314453125	\\
0.205575531584321	0.003448486328125	\\
0.205619922759355	0.003082275390625	\\
0.20566431393439	0.00299072265625	\\
0.205708705109424	0.00347900390625	\\
0.205753096284459	0.003326416015625	\\
0.205797487459493	0.00299072265625	\\
0.205841878634527	0.00341796875	\\
0.205886269809562	0.00360107421875	\\
0.205930660984596	0.003204345703125	\\
0.205975052159631	0.002960205078125	\\
0.206019443334665	0.003570556640625	\\
0.206063834509699	0.003662109375	\\
0.206108225684734	0.0035400390625	\\
0.206152616859768	0.0032958984375	\\
0.206197008034803	0.00299072265625	\\
0.206241399209837	0.003173828125	\\
0.206285790384871	0.003143310546875	\\
0.206330181559906	0.003448486328125	\\
0.20637457273494	0.003662109375	\\
0.206418963909975	0.00360107421875	\\
0.206463355085009	0.003570556640625	\\
0.206507746260044	0.003265380859375	\\
0.206552137435078	0.00335693359375	\\
0.206596528610112	0.003448486328125	\\
0.206640919785147	0.003265380859375	\\
0.206685310960181	0.003265380859375	\\
0.206729702135216	0.00341796875	\\
0.20677409331025	0.003570556640625	\\
0.206818484485284	0.00372314453125	\\
0.206862875660319	0.00372314453125	\\
0.206907266835353	0.00372314453125	\\
0.206951658010388	0.004241943359375	\\
0.206996049185422	0.004364013671875	\\
0.207040440360456	0.00408935546875	\\
0.207084831535491	0.003326416015625	\\
0.207129222710525	0.0029296875	\\
0.20717361388556	0.003265380859375	\\
0.207218005060594	0.003173828125	\\
0.207262396235628	0.002960205078125	\\
0.207306787410663	0.0029296875	\\
0.207351178585697	0.0028076171875	\\
0.207395569760732	0.00262451171875	\\
0.207439960935766	0.002685546875	\\
0.2074843521108	0.00286865234375	\\
0.207528743285835	0.002532958984375	\\
0.207573134460869	0.00244140625	\\
0.207617525635904	0.0023193359375	\\
0.207661916810938	0.002410888671875	\\
0.207706307985972	0.002716064453125	\\
0.207750699161007	0.002471923828125	\\
0.207795090336041	0.00250244140625	\\
0.207839481511076	0.002716064453125	\\
0.20788387268611	0.00201416015625	\\
0.207928263861144	0.002105712890625	\\
0.207972655036179	0.002197265625	\\
0.208017046211213	0.001800537109375	\\
0.208061437386248	0.001434326171875	\\
0.208105828561282	0.001373291015625	\\
0.208150219736316	0.001617431640625	\\
0.208194610911351	0.001800537109375	\\
0.208239002086385	0.00152587890625	\\
0.20828339326142	0.00177001953125	\\
0.208327784436454	0.001983642578125	\\
0.208372175611488	0.001708984375	\\
0.208416566786523	0.001800537109375	\\
0.208460957961557	0.00189208984375	\\
0.208505349136592	0.001861572265625	\\
0.208549740311626	0.00225830078125	\\
0.20859413148666	0.002166748046875	\\
0.208638522661695	0.00213623046875	\\
0.208682913836729	0.00250244140625	\\
0.208727305011764	0.002716064453125	\\
0.208771696186798	0.00225830078125	\\
0.208816087361832	0.00225830078125	\\
0.208860478536867	0.00262451171875	\\
0.208904869711901	0.002685546875	\\
0.208949260886936	0.002410888671875	\\
0.20899365206197	0.002410888671875	\\
0.209038043237004	0.00286865234375	\\
0.209082434412039	0.00244140625	\\
0.209126825587073	0.00213623046875	\\
0.209171216762108	0.002227783203125	\\
0.209215607937142	0.002349853515625	\\
0.209259999112176	0.002197265625	\\
0.209304390287211	0.002410888671875	\\
0.209348781462245	0.002288818359375	\\
0.20939317263728	0.001953125	\\
0.209437563812314	0.002288818359375	\\
0.209481954987349	0.002197265625	\\
0.209526346162383	0.001953125	\\
0.209570737337417	0.001556396484375	\\
0.209615128512452	0.001800537109375	\\
0.209659519687486	0.001708984375	\\
0.209703910862521	0.001556396484375	\\
0.209748302037555	0.00140380859375	\\
0.209792693212589	0.001617431640625	\\
0.209837084387624	0.001434326171875	\\
0.209881475562658	0.000823974609375	\\
0.209925866737693	0.001220703125	\\
0.209970257912727	0.0009765625	\\
0.210014649087761	0.000732421875	\\
0.210059040262796	0.001068115234375	\\
0.21010343143783	0.0008544921875	\\
0.210147822612865	0.00091552734375	\\
0.210192213787899	0.001251220703125	\\
0.210236604962933	0.001190185546875	\\
0.210280996137968	0.001068115234375	\\
0.210325387313002	0.001251220703125	\\
0.210369778488037	0.00115966796875	\\
0.210414169663071	0.000946044921875	\\
0.210458560838105	0.001129150390625	\\
0.21050295201314	0.000885009765625	\\
0.210547343188174	0.000823974609375	\\
0.210591734363209	0.0010986328125	\\
0.210636125538243	0.000885009765625	\\
0.210680516713277	0.001129150390625	\\
0.210724907888312	0.00152587890625	\\
0.210769299063346	0.00140380859375	\\
0.210813690238381	0.001953125	\\
0.210858081413415	0.0018310546875	\\
0.210902472588449	0.001190185546875	\\
0.210946863763484	0.001190185546875	\\
0.210991254938518	0.001068115234375	\\
0.211035646113553	0.000946044921875	\\
0.211080037288587	0.0009765625	\\
0.211124428463621	0.000518798828125	\\
0.211168819638656	0.000396728515625	\\
0.21121321081369	0.00054931640625	\\
0.211257601988725	0.000335693359375	\\
0.211301993163759	0.000152587890625	\\
0.211346384338793	0.000152587890625	\\
0.211390775513828	6.103515625e-05	\\
0.211435166688862	0.00018310546875	\\
0.211479557863897	0.000244140625	\\
0.211523949038931	0.0003662109375	\\
0.211568340213965	0.000152587890625	\\
0.211612731389	-0.00018310546875	\\
0.211657122564034	-0.0001220703125	\\
0.211701513739069	-0.00030517578125	\\
0.211745904914103	-0.000518798828125	\\
0.211790296089137	-0.000213623046875	\\
0.211834687264172	-0.000274658203125	\\
0.211879078439206	-0.00048828125	\\
0.211923469614241	-0.000274658203125	\\
0.211967860789275	-0.00030517578125	\\
0.21201225196431	-0.000274658203125	\\
0.212056643139344	-0.00048828125	\\
0.212101034314378	-0.0001220703125	\\
0.212145425489413	0.0003662109375	\\
0.212189816664447	-9.1552734375e-05	\\
0.212234207839482	-0.0001220703125	\\
0.212278599014516	0.00030517578125	\\
0.21232299018955	0.00054931640625	\\
0.212367381364585	0.000396728515625	\\
0.212411772539619	0.000396728515625	\\
0.212456163714654	0.00030517578125	\\
0.212500554889688	0.000640869140625	\\
0.212544946064722	0.0008544921875	\\
0.212589337239757	0.0003662109375	\\
0.212633728414791	0.000732421875	\\
0.212678119589826	0.000762939453125	\\
0.21272251076486	0.00079345703125	\\
0.212766901939894	0.000946044921875	\\
0.212811293114929	0.00067138671875	\\
0.212855684289963	0.00030517578125	\\
0.212900075464998	0.000457763671875	\\
0.212944466640032	0.001068115234375	\\
0.212988857815066	0.001007080078125	\\
0.213033248990101	0.001068115234375	\\
0.213077640165135	0.001495361328125	\\
0.21312203134017	0.000885009765625	\\
0.213166422515204	0.000732421875	\\
0.213210813690238	0.0008544921875	\\
0.213255204865273	0.001007080078125	\\
0.213299596040307	0.0010986328125	\\
0.213343987215342	0.000701904296875	\\
0.213388378390376	0.000213623046875	\\
0.21343276956541	0.000640869140625	\\
0.213477160740445	0.000701904296875	\\
0.213521551915479	0.000244140625	\\
0.213565943090514	-0.00042724609375	\\
0.213610334265548	-3.0517578125e-05	\\
0.213654725440582	0.00030517578125	\\
0.213699116615617	0.000152587890625	\\
0.213743507790651	0.00048828125	\\
0.213787898965686	0.000579833984375	\\
0.21383229014072	0.0001220703125	\\
0.213876681315754	0.0001220703125	\\
0.213921072490789	0.00048828125	\\
0.213965463665823	0.000518798828125	\\
0.214009854840858	0.0003662109375	\\
0.214054246015892	0.000457763671875	\\
0.214098637190926	0.000335693359375	\\
0.214143028365961	6.103515625e-05	\\
0.214187419540995	0.00048828125	\\
0.21423181071603	0.0006103515625	\\
0.214276201891064	0.00054931640625	\\
0.214320593066098	0.0006103515625	\\
0.214364984241133	0.000457763671875	\\
0.214409375416167	0.00091552734375	\\
0.214453766591202	0.001190185546875	\\
0.214498157766236	0.000732421875	\\
0.21454254894127	0.000762939453125	\\
0.214586940116305	0.00115966796875	\\
0.214631331291339	0.00091552734375	\\
0.214675722466374	0.00128173828125	\\
0.214720113641408	0.0010986328125	\\
0.214764504816442	0.000732421875	\\
0.214808895991477	0.0009765625	\\
0.214853287166511	0.00079345703125	\\
0.214897678341546	0.0010986328125	\\
0.21494206951658	0.001190185546875	\\
0.214986460691615	0.000732421875	\\
0.215030851866649	0.000640869140625	\\
0.215075243041683	0.001068115234375	\\
0.215119634216718	0.000732421875	\\
0.215164025391752	0.00079345703125	\\
0.215208416566787	0.000946044921875	\\
0.215252807741821	0.000762939453125	\\
0.215297198916855	0.000579833984375	\\
0.21534159009189	0.0003662109375	\\
0.215385981266924	0.00067138671875	\\
0.215430372441959	0.0003662109375	\\
0.215474763616993	0.00048828125	\\
0.215519154792027	0.00030517578125	\\
0.215563545967062	-0.000152587890625	\\
0.215607937142096	-0.000396728515625	\\
0.215652328317131	-0.00067138671875	\\
0.215696719492165	-0.000946044921875	\\
0.215741110667199	-0.001190185546875	\\
0.215785501842234	-0.000579833984375	\\
0.215829893017268	-0.00054931640625	\\
0.215874284192303	-0.00042724609375	\\
0.215918675367337	-0.000457763671875	\\
0.215963066542371	-0.000518798828125	\\
0.216007457717406	-0.000335693359375	\\
0.21605184889244	-0.000244140625	\\
0.216096240067475	6.103515625e-05	\\
0.216140631242509	-9.1552734375e-05	\\
0.216185022417543	6.103515625e-05	\\
0.216229413592578	-9.1552734375e-05	\\
0.216273804767612	-0.00042724609375	\\
0.216318195942647	-0.00048828125	\\
0.216362587117681	-0.000396728515625	\\
0.216406978292715	-0.000457763671875	\\
0.21645136946775	-9.1552734375e-05	\\
0.216495760642784	0.000518798828125	\\
0.216540151817819	0.0008544921875	\\
0.216584542992853	0.0009765625	\\
0.216628934167887	0.00048828125	\\
0.216673325342922	0.000823974609375	\\
0.216717716517956	0.0009765625	\\
0.216762107692991	0.000274658203125	\\
0.216806498868025	0.0001220703125	\\
0.216850890043059	6.103515625e-05	\\
0.216895281218094	-0.00030517578125	\\
0.216939672393128	-0.000152587890625	\\
0.216984063568163	0.000244140625	\\
0.217028454743197	-0.000274658203125	\\
0.217072845918231	-0.00079345703125	\\
0.217117237093266	-0.00042724609375	\\
0.2171616282683	-0.000579833984375	\\
0.217206019443335	-0.000885009765625	\\
0.217250410618369	-0.00103759765625	\\
0.217294801793403	-0.000823974609375	\\
0.217339192968438	-0.00079345703125	\\
0.217383584143472	-0.00140380859375	\\
0.217427975318507	-0.001708984375	\\
0.217472366493541	-0.00164794921875	\\
0.217516757668575	-0.001953125	\\
0.21756114884361	-0.002197265625	\\
0.217605540018644	-0.001800537109375	\\
0.217649931193679	-0.002227783203125	\\
0.217694322368713	-0.001983642578125	\\
0.217738713543747	-0.00225830078125	\\
0.217783104718782	-0.002471923828125	\\
0.217827495893816	-0.002227783203125	\\
0.217871887068851	-0.002349853515625	\\
0.217916278243885	-0.0020751953125	\\
0.21796066941892	-0.002166748046875	\\
0.218005060593954	-0.001739501953125	\\
0.218049451768988	-0.001495361328125	\\
0.218093842944023	-0.001312255859375	\\
0.218138234119057	-0.000640869140625	\\
0.218182625294092	-0.000762939453125	\\
0.218227016469126	-0.0009765625	\\
0.21827140764416	-0.001129150390625	\\
0.218315798819195	-0.0009765625	\\
0.218360189994229	-0.000518798828125	\\
0.218404581169264	-0.000518798828125	\\
0.218448972344298	-0.000457763671875	\\
0.218493363519332	-0.00030517578125	\\
0.218537754694367	-0.000518798828125	\\
0.218582145869401	-0.000457763671875	\\
0.218626537044436	0.0001220703125	\\
0.21867092821947	0.0003662109375	\\
0.218715319394504	6.103515625e-05	\\
0.218759710569539	6.103515625e-05	\\
0.218804101744573	3.0517578125e-05	\\
0.218848492919608	-0.000152587890625	\\
0.218892884094642	0	\\
0.218937275269676	-0.0001220703125	\\
0.218981666444711	-0.000244140625	\\
0.219026057619745	-0.000213623046875	\\
0.21907044879478	-0.0003662109375	\\
0.219114839969814	-0.000152587890625	\\
0.219159231144848	3.0517578125e-05	\\
0.219203622319883	-3.0517578125e-05	\\
0.219248013494917	0.000152587890625	\\
0.219292404669952	-9.1552734375e-05	\\
0.219336795844986	0	\\
0.21938118702002	-0.000274658203125	\\
0.219425578195055	-0.00048828125	\\
0.219469969370089	-0.000213623046875	\\
0.219514360545124	-0.000213623046875	\\
0.219558751720158	-0.0001220703125	\\
0.219603142895192	-3.0517578125e-05	\\
0.219647534070227	-0.00018310546875	\\
0.219691925245261	-9.1552734375e-05	\\
0.219736316420296	-0.000213623046875	\\
0.21978070759533	-3.0517578125e-05	\\
0.219825098770364	0.00048828125	\\
0.219869489945399	0.000274658203125	\\
0.219913881120433	-0.000152587890625	\\
0.219958272295468	-3.0517578125e-05	\\
0.220002663470502	9.1552734375e-05	\\
0.220047054645536	0.000152587890625	\\
0.220091445820571	0.000152587890625	\\
0.220135836995605	0.0003662109375	\\
0.22018022817064	0.000518798828125	\\
0.220224619345674	0.000244140625	\\
0.220269010520708	0.000396728515625	\\
0.220313401695743	0.000640869140625	\\
0.220357792870777	0.000579833984375	\\
0.220402184045812	0.00042724609375	\\
0.220446575220846	0.0008544921875	\\
0.220490966395881	0.001190185546875	\\
0.220535357570915	0.00079345703125	\\
0.220579748745949	0.00067138671875	\\
0.220624139920984	0.000885009765625	\\
0.220668531096018	0.00079345703125	\\
0.220712922271053	0.000518798828125	\\
0.220757313446087	0.000732421875	\\
0.220801704621121	0.00067138671875	\\
0.220846095796156	0.000335693359375	\\
0.22089048697119	0.00042724609375	\\
0.220934878146225	0.000732421875	\\
0.220979269321259	0.000732421875	\\
0.221023660496293	0.000396728515625	\\
0.221068051671328	0.000518798828125	\\
0.221112442846362	0.00067138671875	\\
0.221156834021397	0.000701904296875	\\
0.221201225196431	0.000946044921875	\\
0.221245616371465	0.001007080078125	\\
0.2212900075465	0.00067138671875	\\
0.221334398721534	0.000457763671875	\\
0.221378789896569	0.000335693359375	\\
0.221423181071603	0.00048828125	\\
0.221467572246637	0.000244140625	\\
0.221511963421672	-0.000213623046875	\\
0.221556354596706	0.00018310546875	\\
0.221600745771741	0	\\
0.221645136946775	-0.0001220703125	\\
0.221689528121809	3.0517578125e-05	\\
0.221733919296844	-6.103515625e-05	\\
0.221778310471878	0.000152587890625	\\
0.221822701646913	0.000701904296875	\\
0.221867092821947	0.000762939453125	\\
0.221911483996981	0.000640869140625	\\
0.221955875172016	0.000885009765625	\\
0.22200026634705	0.000823974609375	\\
0.222044657522085	0.000579833984375	\\
0.222089048697119	0.001007080078125	\\
0.222133439872153	0.001007080078125	\\
0.222177831047188	0.001220703125	\\
0.222222222222222	0.00152587890625	\\
0.222266613397257	0.001617431640625	\\
0.222311004572291	0.00164794921875	\\
0.222355395747325	0.001708984375	\\
0.22239978692236	0.0020751953125	\\
0.222444178097394	0.0023193359375	\\
0.222488569272429	0.00213623046875	\\
0.222532960447463	0.002166748046875	\\
0.222577351622497	0.002166748046875	\\
0.222621742797532	0.0020751953125	\\
0.222666133972566	0.002105712890625	\\
0.222710525147601	0.0020751953125	\\
0.222754916322635	0.00225830078125	\\
0.222799307497669	0.00189208984375	\\
0.222843698672704	0.001556396484375	\\
0.222888089847738	0.001708984375	\\
0.222932481022773	0.001739501953125	\\
0.222976872197807	0.00164794921875	\\
0.223021263372841	0.0015869140625	\\
0.223065654547876	0.0013427734375	\\
0.22311004572291	0.00146484375	\\
0.223154436897945	0.001312255859375	\\
0.223198828072979	0.00103759765625	\\
0.223243219248013	0.00091552734375	\\
0.223287610423048	0.0010986328125	\\
0.223332001598082	0.001007080078125	\\
0.223376392773117	0.0009765625	\\
0.223420783948151	0.00079345703125	\\
0.223465175123186	0.000701904296875	\\
0.22350956629822	0.000885009765625	\\
0.223553957473254	0.000762939453125	\\
0.223598348648289	0.000732421875	\\
0.223642739823323	0.000518798828125	\\
0.223687130998358	0.000579833984375	\\
0.223731522173392	0.000396728515625	\\
0.223775913348426	0.000457763671875	\\
0.223820304523461	0.00042724609375	\\
0.223864695698495	0.000335693359375	\\
0.22390908687353	0.000640869140625	\\
0.223953478048564	0.000579833984375	\\
0.223997869223598	0.000640869140625	\\
0.224042260398633	0.000457763671875	\\
0.224086651573667	0.000457763671875	\\
0.224131042748702	0.000823974609375	\\
0.224175433923736	0.000885009765625	\\
0.22421982509877	0.001068115234375	\\
0.224264216273805	0.00140380859375	\\
0.224308607448839	0.00103759765625	\\
0.224352998623874	0.00067138671875	\\
0.224397389798908	0.001129150390625	\\
0.224441780973942	0.001220703125	\\
0.224486172148977	0.000885009765625	\\
0.224530563324011	0.00091552734375	\\
0.224574954499046	0.00140380859375	\\
0.22461934567408	0.00128173828125	\\
0.224663736849114	0.000946044921875	\\
0.224708128024149	0.00140380859375	\\
0.224752519199183	0.00152587890625	\\
0.224796910374218	0.001373291015625	\\
0.224841301549252	0.0013427734375	\\
0.224885692724286	0.0010986328125	\\
0.224930083899321	0.00079345703125	\\
0.224974475074355	0.000701904296875	\\
0.22501886624939	0.000274658203125	\\
0.225063257424424	0.0001220703125	\\
0.225107648599458	-0.0001220703125	\\
0.225152039774493	-0.00048828125	\\
0.225196430949527	0.0001220703125	\\
0.225240822124562	0.00030517578125	\\
0.225285213299596	-0.000244140625	\\
0.22532960447463	-6.103515625e-05	\\
0.225373995649665	-0.000152587890625	\\
0.225418386824699	-0.000640869140625	\\
0.225462777999734	-0.000823974609375	\\
0.225507169174768	-0.00042724609375	\\
0.225551560349802	-0.000885009765625	\\
0.225595951524837	-0.000762939453125	\\
0.225640342699871	-0.000213623046875	\\
0.225684733874906	-0.0008544921875	\\
0.22572912504994	-0.000640869140625	\\
0.225773516224974	-0.00054931640625	\\
0.225817907400009	-0.000762939453125	\\
0.225862298575043	-0.0003662109375	\\
0.225906689750078	-0.0003662109375	\\
0.225951080925112	-0.000640869140625	\\
0.225995472100146	-0.0008544921875	\\
0.226039863275181	-0.000701904296875	\\
0.226084254450215	-0.0009765625	\\
0.22612864562525	-0.001495361328125	\\
0.226173036800284	-0.000732421875	\\
0.226217427975319	-0.000213623046875	\\
0.226261819150353	-0.000701904296875	\\
0.226306210325387	-0.000457763671875	\\
0.226350601500422	-0.00030517578125	\\
0.226394992675456	-0.00042724609375	\\
0.226439383850491	-0.000244140625	\\
0.226483775025525	-0.00042724609375	\\
0.226528166200559	-0.000274658203125	\\
0.226572557375594	-0.000274658203125	\\
0.226616948550628	-0.000274658203125	\\
0.226661339725663	-0.0003662109375	\\
0.226705730900697	-0.00079345703125	\\
0.226750122075731	-0.000518798828125	\\
0.226794513250766	-0.000213623046875	\\
0.2268389044258	-0.000152587890625	\\
0.226883295600835	-0.00091552734375	\\
0.226927686775869	-0.000885009765625	\\
0.226972077950903	-0.00067138671875	\\
0.227016469125938	-0.000823974609375	\\
0.227060860300972	-0.000732421875	\\
0.227105251476007	-0.000762939453125	\\
0.227149642651041	-0.00067138671875	\\
0.227194033826075	-0.000640869140625	\\
0.22723842500111	-0.001190185546875	\\
0.227282816176144	-0.001556396484375	\\
0.227327207351179	-0.0010986328125	\\
0.227371598526213	-0.001007080078125	\\
0.227415989701247	-0.001129150390625	\\
0.227460380876282	-0.001220703125	\\
0.227504772051316	-0.001129150390625	\\
0.227549163226351	-0.00140380859375	\\
0.227593554401385	-0.001708984375	\\
0.227637945576419	-0.00128173828125	\\
0.227682336751454	-0.000885009765625	\\
0.227726727926488	-0.000457763671875	\\
0.227771119101523	-0.0003662109375	\\
0.227815510276557	-0.000457763671875	\\
0.227859901451591	-0.00030517578125	\\
0.227904292626626	-0.000396728515625	\\
0.22794868380166	-0.000274658203125	\\
0.227993074976695	-0.00030517578125	\\
0.228037466151729	-0.00030517578125	\\
0.228081857326763	-0.000152587890625	\\
0.228126248501798	0.00018310546875	\\
0.228170639676832	0.000335693359375	\\
0.228215030851867	-6.103515625e-05	\\
0.228259422026901	-0.000152587890625	\\
0.228303813201935	0.000335693359375	\\
0.22834820437697	0.000732421875	\\
0.228392595552004	0.000762939453125	\\
0.228436986727039	0.0009765625	\\
0.228481377902073	0.000701904296875	\\
0.228525769077107	0.00054931640625	\\
0.228570160252142	0.000457763671875	\\
0.228614551427176	3.0517578125e-05	\\
0.228658942602211	0.000244140625	\\
0.228703333777245	0	\\
0.228747724952279	0.00018310546875	\\
0.228792116127314	0.000213623046875	\\
0.228836507302348	0.000152587890625	\\
0.228880898477383	0.000396728515625	\\
0.228925289652417	0.000396728515625	\\
0.228969680827452	9.1552734375e-05	\\
0.229014072002486	-0.00030517578125	\\
0.22905846317752	-0.000396728515625	\\
0.229102854352555	-0.00030517578125	\\
0.229147245527589	-0.000457763671875	\\
0.229191636702624	-0.000885009765625	\\
0.229236027877658	-0.000946044921875	\\
0.229280419052692	-0.000640869140625	\\
0.229324810227727	-9.1552734375e-05	\\
0.229369201402761	-0.000274658203125	\\
0.229413592577796	-0.00030517578125	\\
0.22945798375283	-0.000274658203125	\\
0.229502374927864	-0.000579833984375	\\
0.229546766102899	-0.00054931640625	\\
0.229591157277933	-0.00030517578125	\\
0.229635548452968	-3.0517578125e-05	\\
0.229679939628002	-6.103515625e-05	\\
0.229724330803036	-6.103515625e-05	\\
0.229768721978071	0.000457763671875	\\
0.229813113153105	0.00030517578125	\\
0.22985750432814	-0.000244140625	\\
0.229901895503174	6.103515625e-05	\\
0.229946286678208	0	\\
0.229990677853243	-0.000244140625	\\
0.230035069028277	0.000152587890625	\\
0.230079460203312	0.00018310546875	\\
0.230123851378346	0.000274658203125	\\
0.23016824255338	0	\\
0.230212633728415	9.1552734375e-05	\\
0.230257024903449	0.0001220703125	\\
0.230301416078484	9.1552734375e-05	\\
0.230345807253518	0	\\
0.230390198428552	-0.000213623046875	\\
0.230434589603587	0.00018310546875	\\
0.230478980778621	-3.0517578125e-05	\\
0.230523371953656	-0.00042724609375	\\
0.23056776312869	-0.000701904296875	\\
0.230612154303724	-0.000579833984375	\\
0.230656545478759	-0.0008544921875	\\
0.230700936653793	-0.001129150390625	\\
0.230745327828828	-0.001068115234375	\\
0.230789719003862	-0.001190185546875	\\
0.230834110178896	-0.001129150390625	\\
0.230878501353931	-0.00048828125	\\
0.230922892528965	-0.00079345703125	\\
0.230967283704	-0.001220703125	\\
0.231011674879034	-0.0009765625	\\
0.231056066054068	-0.0010986328125	\\
0.231100457229103	-0.001251220703125	\\
0.231144848404137	-0.001434326171875	\\
0.231189239579172	-0.001708984375	\\
0.231233630754206	-0.0018310546875	\\
0.23127802192924	-0.001953125	\\
0.231322413104275	-0.001739501953125	\\
0.231366804279309	-0.0013427734375	\\
0.231411195454344	-0.001373291015625	\\
0.231455586629378	-0.000885009765625	\\
0.231499977804412	-0.001007080078125	\\
0.231544368979447	-0.001220703125	\\
0.231588760154481	-0.001190185546875	\\
0.231633151329516	-0.001312255859375	\\
0.23167754250455	-0.0010986328125	\\
0.231721933679584	-0.00128173828125	\\
0.231766324854619	-0.0015869140625	\\
0.231810716029653	-0.0010986328125	\\
0.231855107204688	-0.000946044921875	\\
0.231899498379722	-0.00115966796875	\\
0.231943889554757	-0.000457763671875	\\
0.231988280729791	-0.000457763671875	\\
0.232032671904825	-0.000518798828125	\\
0.23207706307986	0.000152587890625	\\
0.232121454254894	-3.0517578125e-05	\\
0.232165845429929	-0.00030517578125	\\
0.232210236604963	0.000152587890625	\\
0.232254627779997	-0.00018310546875	\\
0.232299018955032	-6.103515625e-05	\\
0.232343410130066	3.0517578125e-05	\\
0.232387801305101	-0.00018310546875	\\
0.232432192480135	-0.000244140625	\\
0.232476583655169	-0.00018310546875	\\
0.232520974830204	0.000152587890625	\\
0.232565366005238	-0.000274658203125	\\
0.232609757180273	-0.000518798828125	\\
0.232654148355307	-0.00042724609375	\\
0.232698539530341	-0.0006103515625	\\
0.232742930705376	-0.000762939453125	\\
0.23278732188041	-0.00091552734375	\\
0.232831713055445	-0.0010986328125	\\
0.232876104230479	-0.0010986328125	\\
0.232920495405513	-0.00115966796875	\\
0.232964886580548	-0.001678466796875	\\
0.233009277755582	-0.00140380859375	\\
0.233053668930617	-0.0015869140625	\\
0.233098060105651	-0.002288818359375	\\
0.233142451280685	-0.002044677734375	\\
0.23318684245572	-0.001861572265625	\\
0.233231233630754	-0.001861572265625	\\
0.233275624805789	-0.0020751953125	\\
0.233320015980823	-0.00225830078125	\\
0.233364407155857	-0.00213623046875	\\
0.233408798330892	-0.00201416015625	\\
0.233453189505926	-0.001708984375	\\
0.233497580680961	-0.0020751953125	\\
0.233541971855995	-0.00189208984375	\\
0.233586363031029	-0.00140380859375	\\
0.233630754206064	-0.001434326171875	\\
0.233675145381098	-0.001556396484375	\\
0.233719536556133	-0.0015869140625	\\
0.233763927731167	-0.001220703125	\\
0.233808318906201	-0.00103759765625	\\
0.233852710081236	-0.00140380859375	\\
0.23389710125627	-0.001739501953125	\\
0.233941492431305	-0.001434326171875	\\
0.233985883606339	-0.001495361328125	\\
0.234030274781373	-0.001861572265625	\\
0.234074665956408	-0.00177001953125	\\
0.234119057131442	-0.001434326171875	\\
0.234163448306477	-0.00177001953125	\\
0.234207839481511	-0.0015869140625	\\
0.234252230656545	-0.001312255859375	\\
0.23429662183158	-0.00128173828125	\\
0.234341013006614	-0.001190185546875	\\
0.234385404181649	-0.001373291015625	\\
0.234429795356683	-0.001617431640625	\\
0.234474186531718	-0.001739501953125	\\
0.234518577706752	-0.001556396484375	\\
0.234562968881786	-0.0015869140625	\\
0.234607360056821	-0.002197265625	\\
0.234651751231855	-0.002197265625	\\
0.23469614240689	-0.002044677734375	\\
0.234740533581924	-0.00213623046875	\\
0.234784924756958	-0.002105712890625	\\
0.234829315931993	-0.002349853515625	\\
0.234873707107027	-0.00250244140625	\\
0.234918098282062	-0.002288818359375	\\
0.234962489457096	-0.00238037109375	\\
0.23500688063213	-0.002349853515625	\\
0.235051271807165	-0.002593994140625	\\
0.235095662982199	-0.002655029296875	\\
0.235140054157234	-0.00250244140625	\\
0.235184445332268	-0.00238037109375	\\
0.235228836507302	-0.002593994140625	\\
0.235273227682337	-0.0029296875	\\
0.235317618857371	-0.002899169921875	\\
0.235362010032406	-0.002471923828125	\\
0.23540640120744	-0.002410888671875	\\
0.235450792382474	-0.002349853515625	\\
0.235495183557509	-0.002166748046875	\\
0.235539574732543	-0.00244140625	\\
0.235583965907578	-0.003021240234375	\\
0.235628357082612	-0.002777099609375	\\
0.235672748257646	-0.002593994140625	\\
0.235717139432681	-0.002685546875	\\
0.235761530607715	-0.002838134765625	\\
0.23580592178275	-0.002899169921875	\\
0.235850312957784	-0.002349853515625	\\
0.235894704132818	-0.002227783203125	\\
0.235939095307853	-0.00213623046875	\\
0.235983486482887	-0.001678466796875	\\
0.236027877657922	-0.00164794921875	\\
0.236072268832956	-0.001739501953125	\\
0.23611666000799	-0.0018310546875	\\
0.236161051183025	-0.00213623046875	\\
0.236205442358059	-0.00225830078125	\\
0.236249833533094	-0.00213623046875	\\
0.236294224708128	-0.001983642578125	\\
0.236338615883162	-0.00213623046875	\\
0.236383007058197	-0.00189208984375	\\
0.236427398233231	-0.00189208984375	\\
0.236471789408266	-0.0018310546875	\\
0.2365161805833	-0.0018310546875	\\
0.236560571758334	-0.002105712890625	\\
0.236604962933369	-0.001800537109375	\\
0.236649354108403	-0.001983642578125	\\
0.236693745283438	-0.0018310546875	\\
0.236738136458472	-0.00164794921875	\\
0.236782527633506	-0.001800537109375	\\
0.236826918808541	-0.002105712890625	\\
0.236871309983575	-0.002044677734375	\\
0.23691570115861	-0.002166748046875	\\
0.236960092333644	-0.002532958984375	\\
0.237004483508678	-0.00262451171875	\\
0.237048874683713	-0.00238037109375	\\
0.237093265858747	-0.002532958984375	\\
0.237137657033782	-0.00286865234375	\\
0.237182048208816	-0.00244140625	\\
0.23722643938385	-0.00238037109375	\\
0.237270830558885	-0.00250244140625	\\
0.237315221733919	-0.002716064453125	\\
0.237359612908954	-0.002685546875	\\
0.237404004083988	-0.00225830078125	\\
0.237448395259023	-0.002410888671875	\\
0.237492786434057	-0.002593994140625	\\
0.237537177609091	-0.002593994140625	\\
0.237581568784126	-0.002166748046875	\\
0.23762595995916	-0.002349853515625	\\
0.237670351134195	-0.00250244140625	\\
0.237714742309229	-0.002105712890625	\\
0.237759133484263	-0.002349853515625	\\
0.237803524659298	-0.002288818359375	\\
0.237847915834332	-0.001953125	\\
0.237892307009367	-0.001922607421875	\\
0.237936698184401	-0.001678466796875	\\
0.237981089359435	-0.00128173828125	\\
0.23802548053447	-0.0015869140625	\\
0.238069871709504	-0.001556396484375	\\
0.238114262884539	-0.001068115234375	\\
0.238158654059573	-0.001617431640625	\\
0.238203045234607	-0.001617431640625	\\
0.238247436409642	-0.001220703125	\\
0.238291827584676	-0.000823974609375	\\
0.238336218759711	-0.001220703125	\\
0.238380609934745	-0.00140380859375	\\
0.238425001109779	-0.0009765625	\\
0.238469392284814	-0.001129150390625	\\
0.238513783459848	-0.001373291015625	\\
0.238558174634883	-0.00128173828125	\\
0.238602565809917	-0.001251220703125	\\
0.238646956984951	-0.001220703125	\\
0.238691348159986	-0.00152587890625	\\
0.23873573933502	-0.00146484375	\\
0.238780130510055	-0.00140380859375	\\
0.238824521685089	-0.0015869140625	\\
0.238868912860123	-0.001312255859375	\\
0.238913304035158	-0.00128173828125	\\
0.238957695210192	-0.00115966796875	\\
0.239002086385227	-0.00128173828125	\\
0.239046477560261	-0.00152587890625	\\
0.239090868735295	-0.001495361328125	\\
0.23913525991033	-0.001129150390625	\\
0.239179651085364	-0.001190185546875	\\
0.239224042260399	-0.0013427734375	\\
0.239268433435433	-0.001312255859375	\\
0.239312824610467	-0.0009765625	\\
0.239357215785502	-0.000885009765625	\\
0.239401606960536	-0.001220703125	\\
0.239445998135571	-0.000946044921875	\\
0.239490389310605	-0.0010986328125	\\
0.239534780485639	-0.0013427734375	\\
0.239579171660674	-0.001129150390625	\\
0.239623562835708	-0.000946044921875	\\
0.239667954010743	-0.000823974609375	\\
0.239712345185777	-0.001007080078125	\\
0.239756736360811	-0.001068115234375	\\
0.239801127535846	-0.000762939453125	\\
0.23984551871088	-0.0006103515625	\\
0.239889909885915	-0.000518798828125	\\
0.239934301060949	-0.00054931640625	\\
0.239978692235983	-0.001068115234375	\\
0.240023083411018	-0.000579833984375	\\
0.240067474586052	-0.000457763671875	\\
0.240111865761087	-0.000762939453125	\\
0.240156256936121	-0.00042724609375	\\
0.240200648111155	-0.000640869140625	\\
0.24024503928619	-0.0008544921875	\\
0.240289430461224	-0.00103759765625	\\
0.240333821636259	-0.0013427734375	\\
0.240378212811293	-0.0013427734375	\\
0.240422603986328	-0.001556396484375	\\
0.240466995161362	-0.00128173828125	\\
0.240511386336396	-0.001220703125	\\
0.240555777511431	-0.0013427734375	\\
0.240600168686465	-0.0015869140625	\\
0.2406445598615	-0.001556396484375	\\
0.240688951036534	-0.001617431640625	\\
0.240733342211568	-0.00177001953125	\\
0.240777733386603	-0.00152587890625	\\
0.240822124561637	-0.002288818359375	\\
0.240866515736672	-0.002655029296875	\\
0.240910906911706	-0.0023193359375	\\
0.24095529808674	-0.002197265625	\\
0.240999689261775	-0.002349853515625	\\
0.241044080436809	-0.00213623046875	\\
0.241088471611844	-0.001800537109375	\\
0.241132862786878	-0.001708984375	\\
0.241177253961912	-0.001861572265625	\\
0.241221645136947	-0.001953125	\\
0.241266036311981	-0.001617431640625	\\
0.241310427487016	-0.001434326171875	\\
0.24135481866205	-0.0013427734375	\\
0.241399209837084	-0.001129150390625	\\
0.241443601012119	-0.001190185546875	\\
0.241487992187153	-0.0015869140625	\\
0.241532383362188	-0.001007080078125	\\
0.241576774537222	-0.0006103515625	\\
0.241621165712256	-0.0008544921875	\\
0.241665556887291	-0.00115966796875	\\
0.241709948062325	-0.000823974609375	\\
0.24175433923736	-0.000579833984375	\\
0.241798730412394	-0.00054931640625	\\
0.241843121587428	-0.000396728515625	\\
0.241887512762463	0.000274658203125	\\
0.241931903937497	-0.00030517578125	\\
0.241976295112532	-0.000518798828125	\\
0.242020686287566	-0.000274658203125	\\
0.2420650774626	-0.00018310546875	\\
0.242109468637635	-0.000213623046875	\\
0.242153859812669	-0.000946044921875	\\
0.242198250987704	-0.001068115234375	\\
0.242242642162738	-0.000152587890625	\\
0.242287033337772	0.000335693359375	\\
0.242331424512807	0.000335693359375	\\
0.242375815687841	0.000274658203125	\\
0.242420206862876	-9.1552734375e-05	\\
0.24246459803791	-0.00067138671875	\\
0.242508989212944	-0.000762939453125	\\
0.242553380387979	-0.001129150390625	\\
0.242597771563013	-0.001220703125	\\
0.242642162738048	-0.00067138671875	\\
0.242686553913082	-0.000823974609375	\\
0.242730945088116	-0.00091552734375	\\
0.242775336263151	-0.00042724609375	\\
0.242819727438185	-0.00067138671875	\\
0.24286411861322	-0.0009765625	\\
0.242908509788254	-0.0010986328125	\\
0.242952900963289	-0.00128173828125	\\
0.242997292138323	-0.00091552734375	\\
0.243041683313357	-0.000640869140625	\\
0.243086074488392	-0.00079345703125	\\
0.243130465663426	-0.001068115234375	\\
0.243174856838461	-0.001068115234375	\\
0.243219248013495	-0.000701904296875	\\
0.243263639188529	-0.00103759765625	\\
0.243308030363564	-0.00128173828125	\\
0.243352421538598	-0.00115966796875	\\
0.243396812713633	-0.00115966796875	\\
0.243441203888667	-0.001068115234375	\\
0.243485595063701	-0.000732421875	\\
0.243529986238736	-0.000762939453125	\\
0.24357437741377	-0.0006103515625	\\
0.243618768588805	-0.000457763671875	\\
0.243663159763839	-0.001068115234375	\\
0.243707550938873	-0.0008544921875	\\
0.243751942113908	-0.00091552734375	\\
0.243796333288942	-0.00103759765625	\\
0.243840724463977	-0.000640869140625	\\
0.243885115639011	-0.0003662109375	\\
0.243929506814045	-0.000640869140625	\\
0.24397389798908	-0.000762939453125	\\
0.244018289164114	-0.0006103515625	\\
0.244062680339149	-0.0008544921875	\\
0.244107071514183	-0.000701904296875	\\
0.244151462689217	-0.001068115234375	\\
0.244195853864252	-0.000885009765625	\\
0.244240245039286	-0.00079345703125	\\
0.244284636214321	-0.000640869140625	\\
0.244329027389355	-0.000244140625	\\
0.244373418564389	-0.000823974609375	\\
0.244417809739424	-0.00091552734375	\\
0.244462200914458	-0.000701904296875	\\
0.244506592089493	-0.000762939453125	\\
0.244550983264527	-0.000457763671875	\\
0.244595374439561	-0.0008544921875	\\
0.244639765614596	-0.000640869140625	\\
0.24468415678963	-0.000518798828125	\\
0.244728547964665	-0.00115966796875	\\
0.244772939139699	-0.0009765625	\\
0.244817330314733	-0.000762939453125	\\
0.244861721489768	-0.000518798828125	\\
0.244906112664802	-0.00128173828125	\\
0.244950503839837	-0.001953125	\\
0.244994895014871	-0.0013427734375	\\
0.245039286189905	-0.0015869140625	\\
0.24508367736494	-0.001556396484375	\\
0.245128068539974	-0.00115966796875	\\
0.245172459715009	-0.001190185546875	\\
0.245216850890043	-0.001190185546875	\\
0.245261242065077	-0.00115966796875	\\
0.245305633240112	-0.001190185546875	\\
0.245350024415146	-0.001190185546875	\\
0.245394415590181	-0.001129150390625	\\
0.245438806765215	-0.001373291015625	\\
0.245483197940249	-0.00115966796875	\\
0.245527589115284	-0.0006103515625	\\
0.245571980290318	-0.000640869140625	\\
0.245616371465353	-0.00042724609375	\\
0.245660762640387	9.1552734375e-05	\\
0.245705153815421	-0.0001220703125	\\
0.245749544990456	0.00018310546875	\\
0.24579393616549	0.00054931640625	\\
0.245838327340525	0.000732421875	\\
0.245882718515559	0.000762939453125	\\
0.245927109690594	0.00018310546875	\\
0.245971500865628	0.000152587890625	\\
0.246015892040662	0.000457763671875	\\
0.246060283215697	0.0003662109375	\\
0.246104674390731	0.000335693359375	\\
0.246149065565766	0.000335693359375	\\
0.2461934567408	0.000244140625	\\
0.246237847915834	0.000457763671875	\\
0.246282239090869	0.0006103515625	\\
0.246326630265903	0.0006103515625	\\
0.246371021440938	0.000701904296875	\\
0.246415412615972	0.000579833984375	\\
0.246459803791006	0.000274658203125	\\
0.246504194966041	-0.00018310546875	\\
0.246548586141075	-0.0001220703125	\\
0.24659297731611	-6.103515625e-05	\\
0.246637368491144	-9.1552734375e-05	\\
0.246681759666178	-0.000244140625	\\
0.246726150841213	-0.00048828125	\\
0.246770542016247	-0.0006103515625	\\
0.246814933191282	-0.00067138671875	\\
0.246859324366316	-0.000274658203125	\\
0.24690371554135	-0.0001220703125	\\
0.246948106716385	-0.0006103515625	\\
0.246992497891419	-0.000762939453125	\\
0.247036889066454	-0.0006103515625	\\
0.247081280241488	-0.000823974609375	\\
0.247125671416522	-0.00054931640625	\\
0.247170062591557	-0.00030517578125	\\
0.247214453766591	-0.000457763671875	\\
0.247258844941626	-0.00103759765625	\\
0.24730323611666	-0.000640869140625	\\
0.247347627291694	-3.0517578125e-05	\\
0.247392018466729	0.000152587890625	\\
0.247436409641763	0.000213623046875	\\
0.247480800816798	0.000274658203125	\\
0.247525191991832	0.0003662109375	\\
0.247569583166866	-6.103515625e-05	\\
0.247613974341901	0.000152587890625	\\
0.247658365516935	3.0517578125e-05	\\
0.24770275669197	-0.000152587890625	\\
0.247747147867004	-3.0517578125e-05	\\
0.247791539042038	0	\\
0.247835930217073	-0.000244140625	\\
0.247880321392107	9.1552734375e-05	\\
0.247924712567142	0.0008544921875	\\
0.247969103742176	0.00079345703125	\\
0.24801349491721	0.000518798828125	\\
0.248057886092245	0.000701904296875	\\
0.248102277267279	0.000213623046875	\\
0.248146668442314	0	\\
0.248191059617348	6.103515625e-05	\\
0.248235450792382	0.000152587890625	\\
0.248279841967417	-9.1552734375e-05	\\
0.248324233142451	-6.103515625e-05	\\
0.248368624317486	0	\\
0.24841301549252	-0.000762939453125	\\
0.248457406667555	-0.000335693359375	\\
0.248501797842589	-0.000152587890625	\\
0.248546189017623	-0.000274658203125	\\
0.248590580192658	-0.00042724609375	\\
0.248634971367692	-0.00079345703125	\\
0.248679362542727	-0.001068115234375	\\
0.248723753717761	-0.001251220703125	\\
0.248768144892795	-0.0010986328125	\\
0.24881253606783	-0.00152587890625	\\
0.248856927242864	-0.00146484375	\\
0.248901318417899	-0.001220703125	\\
0.248945709592933	-0.001434326171875	\\
0.248990100767967	-0.001220703125	\\
0.249034491943002	-0.00164794921875	\\
0.249078883118036	-0.001922607421875	\\
0.249123274293071	-0.001800537109375	\\
0.249167665468105	-0.00189208984375	\\
0.249212056643139	-0.001953125	\\
0.249256447818174	-0.001617431640625	\\
0.249300838993208	-0.001678466796875	\\
0.249345230168243	-0.00213623046875	\\
0.249389621343277	-0.001953125	\\
0.249434012518311	-0.001617431640625	\\
0.249478403693346	-0.001739501953125	\\
0.24952279486838	-0.001373291015625	\\
0.249567186043415	-0.00140380859375	\\
0.249611577218449	-0.00177001953125	\\
0.249655968393483	-0.001617431640625	\\
0.249700359568518	-0.00164794921875	\\
0.249744750743552	-0.00146484375	\\
0.249789141918587	-0.0010986328125	\\
0.249833533093621	-0.001373291015625	\\
0.249877924268655	-0.00140380859375	\\
0.24992231544369	-0.00091552734375	\\
0.249966706618724	-0.000823974609375	\\
0.250011097793759	-0.00091552734375	\\
0.250055488968793	-0.001190185546875	\\
0.250099880143827	-0.0009765625	\\
0.250144271318862	-0.000823974609375	\\
0.250188662493896	-0.001007080078125	\\
0.250233053668931	-0.001129150390625	\\
0.250277444843965	-0.00152587890625	\\
0.250321836018999	-0.001312255859375	\\
0.250366227194034	-0.001220703125	\\
0.250410618369068	-0.001708984375	\\
0.250455009544103	-0.001739501953125	\\
0.250499400719137	-0.00128173828125	\\
0.250543791894171	-0.001129150390625	\\
0.250588183069206	-0.001312255859375	\\
0.25063257424424	-0.001495361328125	\\
0.250676965419275	-0.001983642578125	\\
0.250721356594309	-0.001953125	\\
0.250765747769343	-0.00189208984375	\\
0.250810138944378	-0.001678466796875	\\
0.250854530119412	-0.001373291015625	\\
0.250898921294447	-0.001556396484375	\\
0.250943312469481	-0.001556396484375	\\
0.250987703644515	-0.001556396484375	\\
0.25103209481955	-0.0018310546875	\\
0.251076485994584	-0.001678466796875	\\
0.251120877169619	-0.0018310546875	\\
0.251165268344653	-0.001953125	\\
0.251209659519687	-0.002044677734375	\\
0.251254050694722	-0.00164794921875	\\
0.251298441869756	-0.0013427734375	\\
0.251342833044791	-0.001373291015625	\\
0.251387224219825	-0.001190185546875	\\
0.251431615394859	-0.001190185546875	\\
0.251476006569894	-0.00115966796875	\\
0.251520397744928	-0.0013427734375	\\
0.251564788919963	-0.00115966796875	\\
0.251609180094997	-0.00128173828125	\\
0.251653571270032	-0.0010986328125	\\
0.251697962445066	-0.001007080078125	\\
0.2517423536201	-0.00115966796875	\\
0.251786744795135	-0.0008544921875	\\
0.251831135970169	-0.00103759765625	\\
0.251875527145204	-0.0010986328125	\\
0.251919918320238	-0.000885009765625	\\
0.251964309495272	-0.0009765625	\\
0.252008700670307	-0.000885009765625	\\
0.252053091845341	-0.000946044921875	\\
0.252097483020376	-0.001068115234375	\\
0.25214187419541	-0.0010986328125	\\
0.252186265370444	-0.00146484375	\\
0.252230656545479	-0.00164794921875	\\
0.252275047720513	-0.0018310546875	\\
0.252319438895548	-0.001953125	\\
0.252363830070582	-0.001800537109375	\\
0.252408221245616	-0.0013427734375	\\
0.252452612420651	-0.001220703125	\\
0.252497003595685	-0.00140380859375	\\
0.25254139477072	-0.001495361328125	\\
0.252585785945754	-0.0013427734375	\\
0.252630177120788	-0.0015869140625	\\
0.252674568295823	-0.001983642578125	\\
0.252718959470857	-0.00201416015625	\\
0.252763350645892	-0.001983642578125	\\
0.252807741820926	-0.00201416015625	\\
0.25285213299596	-0.001983642578125	\\
0.252896524170995	-0.001708984375	\\
0.252940915346029	-0.001617431640625	\\
0.252985306521064	-0.001556396484375	\\
0.253029697696098	-0.0020751953125	\\
0.253074088871132	-0.002349853515625	\\
0.253118480046167	-0.00201416015625	\\
0.253162871221201	-0.00201416015625	\\
0.253207262396236	-0.0020751953125	\\
0.25325165357127	-0.00177001953125	\\
0.253296044746304	-0.0020751953125	\\
0.253340435921339	-0.00250244140625	\\
0.253384827096373	-0.0020751953125	\\
0.253429218271408	-0.001922607421875	\\
0.253473609446442	-0.001861572265625	\\
0.253518000621476	-0.00152587890625	\\
0.253562391796511	-0.001678466796875	\\
0.253606782971545	-0.001434326171875	\\
0.25365117414658	-0.00103759765625	\\
0.253695565321614	-0.00128173828125	\\
0.253739956496648	-0.0015869140625	\\
0.253784347671683	-0.00152587890625	\\
0.253828738846717	-0.001739501953125	\\
0.253873130021752	-0.00177001953125	\\
0.253917521196786	-0.001800537109375	\\
0.25396191237182	-0.001800537109375	\\
0.254006303546855	-0.001678466796875	\\
0.254050694721889	-0.00164794921875	\\
0.254095085896924	-0.001556396484375	\\
0.254139477071958	-0.001556396484375	\\
0.254183868246993	-0.00189208984375	\\
0.254228259422027	-0.001800537109375	\\
0.254272650597061	-0.00152587890625	\\
0.254317041772096	-0.0015869140625	\\
0.25436143294713	-0.0018310546875	\\
0.254405824122165	-0.001983642578125	\\
0.254450215297199	-0.001678466796875	\\
0.254494606472233	-0.00201416015625	\\
0.254538997647268	-0.00225830078125	\\
0.254583388822302	-0.001922607421875	\\
0.254627779997337	-0.00225830078125	\\
0.254672171172371	-0.002105712890625	\\
0.254716562347405	-0.001922607421875	\\
0.25476095352244	-0.002166748046875	\\
0.254805344697474	-0.001556396484375	\\
0.254849735872509	-0.00128173828125	\\
0.254894127047543	-0.00115966796875	\\
0.254938518222577	-0.001220703125	\\
0.254982909397612	-0.001220703125	\\
0.255027300572646	-0.001129150390625	\\
0.255071691747681	-0.001068115234375	\\
0.255116082922715	-0.000946044921875	\\
0.255160474097749	-0.000823974609375	\\
0.255204865272784	-0.00079345703125	\\
0.255249256447818	-0.000518798828125	\\
0.255293647622853	-0.00018310546875	\\
0.255338038797887	6.103515625e-05	\\
0.255382429972921	9.1552734375e-05	\\
0.255426821147956	0.0003662109375	\\
0.25547121232299	0.000244140625	\\
0.255515603498025	0.000335693359375	\\
0.255559994673059	0.000823974609375	\\
0.255604385848093	0.00091552734375	\\
0.255648777023128	0.0009765625	\\
0.255693168198162	0.0009765625	\\
0.255737559373197	0.00103759765625	\\
0.255781950548231	0.000701904296875	\\
0.255826341723265	0.000579833984375	\\
0.2558707328983	0.000732421875	\\
0.255915124073334	0.00048828125	\\
0.255959515248369	0.0003662109375	\\
0.256003906423403	0.000213623046875	\\
0.256048297598437	0.000885009765625	\\
0.256092688773472	0.001068115234375	\\
0.256137079948506	0.000762939453125	\\
0.256181471123541	0.000762939453125	\\
0.256225862298575	0.00030517578125	\\
0.256270253473609	0.0001220703125	\\
0.256314644648644	3.0517578125e-05	\\
0.256359035823678	0	\\
0.256403426998713	-0.00048828125	\\
0.256447818173747	-0.00054931640625	\\
0.256492209348781	-0.00042724609375	\\
0.256536600523816	-0.00048828125	\\
0.25658099169885	-0.000823974609375	\\
0.256625382873885	-0.000823974609375	\\
0.256669774048919	-0.001220703125	\\
0.256714165223953	-0.0015869140625	\\
0.256758556398988	-0.001312255859375	\\
0.256802947574022	-0.001556396484375	\\
0.256847338749057	-0.001556396484375	\\
0.256891729924091	-0.001495361328125	\\
0.256936121099126	-0.001739501953125	\\
0.25698051227416	-0.001556396484375	\\
0.257024903449194	-0.001495361328125	\\
0.257069294624229	-0.001068115234375	\\
0.257113685799263	-0.001007080078125	\\
0.257158076974297	-0.00128173828125	\\
0.257202468149332	-0.001220703125	\\
0.257246859324366	-0.0009765625	\\
0.257291250499401	-0.001068115234375	\\
0.257335641674435	-0.00128173828125	\\
0.25738003284947	-0.001434326171875	\\
0.257424424024504	-0.001251220703125	\\
0.257468815199538	-0.00140380859375	\\
0.257513206374573	-0.00164794921875	\\
0.257557597549607	-0.001495361328125	\\
0.257601988724642	-0.001373291015625	\\
0.257646379899676	-0.00146484375	\\
0.25769077107471	-0.00146484375	\\
0.257735162249745	-0.00146484375	\\
0.257779553424779	-0.00146484375	\\
0.257823944599814	-0.000885009765625	\\
0.257868335774848	-0.001068115234375	\\
0.257912726949882	-0.001190185546875	\\
0.257957118124917	-0.00091552734375	\\
0.258001509299951	-0.00128173828125	\\
0.258045900474986	-0.00079345703125	\\
0.25809029165002	-0.0009765625	\\
0.258134682825054	-0.001434326171875	\\
0.258179074000089	-0.00140380859375	\\
0.258223465175123	-0.001678466796875	\\
0.258267856350158	-0.00128173828125	\\
0.258312247525192	-0.001373291015625	\\
0.258356638700226	-0.001678466796875	\\
0.258401029875261	-0.001373291015625	\\
0.258445421050295	-0.001495361328125	\\
0.25848981222533	-0.002349853515625	\\
0.258534203400364	-0.002655029296875	\\
0.258578594575398	-0.002227783203125	\\
0.258622985750433	-0.002288818359375	\\
0.258667376925467	-0.002410888671875	\\
0.258711768100502	-0.002349853515625	\\
0.258756159275536	-0.00177001953125	\\
0.25880055045057	-0.001800537109375	\\
0.258844941625605	-0.002227783203125	\\
0.258889332800639	-0.00213623046875	\\
0.258933723975674	-0.002197265625	\\
0.258978115150708	-0.0020751953125	\\
0.259022506325742	-0.002197265625	\\
0.259066897500777	-0.002349853515625	\\
0.259111288675811	-0.0020751953125	\\
0.259155679850846	-0.001739501953125	\\
0.25920007102588	-0.001922607421875	\\
0.259244462200914	-0.001739501953125	\\
0.259288853375949	-0.001434326171875	\\
0.259333244550983	-0.001556396484375	\\
0.259377635726018	-0.001953125	\\
0.259422026901052	-0.001922607421875	\\
0.259466418076086	-0.00164794921875	\\
0.259510809251121	-0.0015869140625	\\
0.259555200426155	-0.001800537109375	\\
0.25959959160119	-0.001678466796875	\\
0.259643982776224	-0.001434326171875	\\
0.259688373951258	-0.001495361328125	\\
0.259732765126293	-0.001068115234375	\\
0.259777156301327	-0.001312255859375	\\
0.259821547476362	-0.001556396484375	\\
0.259865938651396	-0.001373291015625	\\
0.259910329826431	-0.001495361328125	\\
0.259954721001465	-0.0013427734375	\\
0.259999112176499	-0.0013427734375	\\
0.260043503351534	-0.000946044921875	\\
0.260087894526568	-0.001373291015625	\\
0.260132285701603	-0.00146484375	\\
0.260176676876637	-0.001312255859375	\\
0.260221068051671	-0.00146484375	\\
0.260265459226706	-0.001495361328125	\\
0.26030985040174	-0.001708984375	\\
0.260354241576775	-0.00177001953125	\\
0.260398632751809	-0.001708984375	\\
0.260443023926843	-0.001953125	\\
0.260487415101878	-0.00201416015625	\\
0.260531806276912	-0.0015869140625	\\
0.260576197451947	-0.001495361328125	\\
0.260620588626981	-0.001190185546875	\\
0.260664979802015	-0.00128173828125	\\
0.26070937097705	-0.001312255859375	\\
0.260753762152084	-0.001708984375	\\
0.260798153327119	-0.001739501953125	\\
0.260842544502153	-0.00079345703125	\\
0.260886935677187	-0.000732421875	\\
0.260931326852222	-0.00146484375	\\
0.260975718027256	-0.001739501953125	\\
0.261020109202291	-0.001373291015625	\\
0.261064500377325	-0.00164794921875	\\
0.261108891552359	-0.001312255859375	\\
0.261153282727394	-0.00103759765625	\\
0.261197673902428	-0.001068115234375	\\
0.261242065077463	-0.001068115234375	\\
0.261286456252497	-0.000946044921875	\\
0.261330847427531	-0.0006103515625	\\
0.261375238602566	-0.000396728515625	\\
0.2614196297776	-0.0003662109375	\\
0.261464020952635	-0.00042724609375	\\
0.261508412127669	-0.000732421875	\\
0.261552803302703	-0.0003662109375	\\
0.261597194477738	-0.00030517578125	\\
0.261641585652772	-0.000457763671875	\\
0.261685976827807	-0.0001220703125	\\
0.261730368002841	3.0517578125e-05	\\
0.261774759177875	-0.000244140625	\\
0.26181915035291	-0.00054931640625	\\
0.261863541527944	3.0517578125e-05	\\
0.261907932702979	-3.0517578125e-05	\\
0.261952323878013	-6.103515625e-05	\\
0.261996715053047	3.0517578125e-05	\\
0.262041106228082	-6.103515625e-05	\\
0.262085497403116	0.000274658203125	\\
0.262129888578151	6.103515625e-05	\\
0.262174279753185	0	\\
0.262218670928219	0.000152587890625	\\
0.262263062103254	0.000244140625	\\
0.262307453278288	0.000213623046875	\\
0.262351844453323	0.00030517578125	\\
0.262396235628357	0.00042724609375	\\
0.262440626803391	0.000762939453125	\\
0.262485017978426	0.00067138671875	\\
0.26252940915346	0.000732421875	\\
0.262573800328495	0.000823974609375	\\
0.262618191503529	0.000701904296875	\\
0.262662582678564	0.000762939453125	\\
0.262706973853598	0.00030517578125	\\
0.262751365028632	0.000213623046875	\\
0.262795756203667	0.000244140625	\\
0.262840147378701	9.1552734375e-05	\\
0.262884538553736	0.00018310546875	\\
0.26292892972877	0.00054931640625	\\
0.262973320903804	0.000335693359375	\\
0.263017712078839	0.000274658203125	\\
0.263062103253873	0.00042724609375	\\
0.263106494428908	0.000213623046875	\\
0.263150885603942	0.000274658203125	\\
0.263195276778976	0.00054931640625	\\
0.263239667954011	0.000274658203125	\\
0.263284059129045	0.000213623046875	\\
0.26332845030408	0.000335693359375	\\
0.263372841479114	0.00018310546875	\\
0.263417232654148	9.1552734375e-05	\\
0.263461623829183	-3.0517578125e-05	\\
0.263506015004217	6.103515625e-05	\\
0.263550406179252	0.00042724609375	\\
0.263594797354286	0.00054931640625	\\
0.26363918852932	0.000732421875	\\
0.263683579704355	0.0009765625	\\
0.263727970879389	0.000823974609375	\\
0.263772362054424	0.001190185546875	\\
0.263816753229458	0.001007080078125	\\
0.263861144404492	0.00067138671875	\\
0.263905535579527	0.0009765625	\\
0.263949926754561	0.0013427734375	\\
0.263994317929596	0.0010986328125	\\
0.26403870910463	0.0009765625	\\
0.264083100279664	0.001373291015625	\\
0.264127491454699	0.000885009765625	\\
0.264171882629733	0.00079345703125	\\
0.264216273804768	0.0010986328125	\\
0.264260664979802	0.00091552734375	\\
0.264305056154836	0.0009765625	\\
0.264349447329871	0.000946044921875	\\
0.264393838504905	0.000732421875	\\
0.26443822967994	0.000701904296875	\\
0.264482620854974	0.000640869140625	\\
0.264527012030008	0.0006103515625	\\
0.264571403205043	0.00079345703125	\\
0.264615794380077	0.000701904296875	\\
0.264660185555112	0.000732421875	\\
0.264704576730146	0.00079345703125	\\
0.26474896790518	0.000885009765625	\\
0.264793359080215	0.001373291015625	\\
0.264837750255249	0.001251220703125	\\
0.264882141430284	0.001312255859375	\\
0.264926532605318	0.001861572265625	\\
0.264970923780352	0.001495361328125	\\
0.265015314955387	0.001800537109375	\\
0.265059706130421	0.00213623046875	\\
0.265104097305456	0.001861572265625	\\
0.26514848848049	0.001800537109375	\\
0.265192879655524	0.002197265625	\\
0.265237270830559	0.00238037109375	\\
0.265281662005593	0.0023193359375	\\
0.265326053180628	0.0025634765625	\\
0.265370444355662	0.00299072265625	\\
0.265414835530697	0.00299072265625	\\
0.265459226705731	0.00262451171875	\\
0.265503617880765	0.002532958984375	\\
0.2655480090558	0.002349853515625	\\
0.265592400230834	0.002349853515625	\\
0.265636791405868	0.002960205078125	\\
0.265681182580903	0.0028076171875	\\
0.265725573755937	0.002899169921875	\\
0.265769964930972	0.003204345703125	\\
0.265814356106006	0.0030517578125	\\
0.265858747281041	0.003021240234375	\\
0.265903138456075	0.0029296875	\\
0.265947529631109	0.00299072265625	\\
0.265991920806144	0.003021240234375	\\
0.266036311981178	0.002960205078125	\\
0.266080703156213	0.003143310546875	\\
0.266125094331247	0.003204345703125	\\
0.266169485506281	0.0029296875	\\
0.266213876681316	0.002471923828125	\\
0.26625826785635	0.002105712890625	\\
0.266302659031385	0.002288818359375	\\
0.266347050206419	0.002105712890625	\\
0.266391441381453	0.002288818359375	\\
0.266435832556488	0.002166748046875	\\
0.266480223731522	0.001739501953125	\\
0.266524614906557	0.0018310546875	\\
0.266569006081591	0.00177001953125	\\
0.266613397256625	0.001312255859375	\\
0.26665778843166	0.001129150390625	\\
0.266702179606694	0.001251220703125	\\
0.266746570781729	0.001190185546875	\\
0.266790961956763	0.00079345703125	\\
0.266835353131797	0.00079345703125	\\
0.266879744306832	0.001129150390625	\\
0.266924135481866	0.001068115234375	\\
0.266968526656901	0.000885009765625	\\
0.267012917831935	0.00079345703125	\\
0.267057309006969	0.000946044921875	\\
0.267101700182004	0.001373291015625	\\
0.267146091357038	0.00128173828125	\\
0.267190482532073	0.00146484375	\\
0.267234873707107	0.001556396484375	\\
0.267279264882141	0.001068115234375	\\
0.267323656057176	0.001220703125	\\
0.26736804723221	0.0015869140625	\\
0.267412438407245	0.00115966796875	\\
0.267456829582279	0.0015869140625	\\
0.267501220757313	0.001739501953125	\\
0.267545611932348	0.00115966796875	\\
0.267590003107382	0.001129150390625	\\
0.267634394282417	0.00152587890625	\\
0.267678785457451	0.001220703125	\\
0.267723176632485	0.0009765625	\\
0.26776756780752	0.00128173828125	\\
0.267811958982554	0.000823974609375	\\
0.267856350157589	0.000518798828125	\\
0.267900741332623	0.001068115234375	\\
0.267945132507658	0.000762939453125	\\
0.267989523682692	0.000732421875	\\
0.268033914857726	0.0010986328125	\\
0.268078306032761	0.000640869140625	\\
0.268122697207795	0.00079345703125	\\
0.268167088382829	0.000885009765625	\\
0.268211479557864	0.00054931640625	\\
0.268255870732898	0.000701904296875	\\
0.268300261907933	0.000701904296875	\\
0.268344653082967	0.0010986328125	\\
0.268389044258002	0.00079345703125	\\
0.268433435433036	0.000579833984375	\\
0.26847782660807	0.000762939453125	\\
0.268522217783105	0.000701904296875	\\
0.268566608958139	0.00067138671875	\\
0.268611000133174	0.000335693359375	\\
0.268655391308208	0.0003662109375	\\
0.268699782483242	0.000823974609375	\\
0.268744173658277	0.000823974609375	\\
0.268788564833311	0.000946044921875	\\
0.268832956008346	0.001373291015625	\\
0.26887734718338	0.001220703125	\\
0.268921738358414	0.0010986328125	\\
0.268966129533449	0.00115966796875	\\
0.269010520708483	0.001800537109375	\\
0.269054911883518	0.0015869140625	\\
0.269099303058552	0.001220703125	\\
0.269143694233586	0.001495361328125	\\
0.269188085408621	0.00164794921875	\\
0.269232476583655	0.001953125	\\
0.26927686775869	0.002105712890625	\\
0.269321258933724	0.0018310546875	\\
0.269365650108758	0.002410888671875	\\
0.269410041283793	0.00244140625	\\
0.269454432458827	0.002349853515625	\\
0.269498823633862	0.00250244140625	\\
0.269543214808896	0.002410888671875	\\
0.26958760598393	0.002655029296875	\\
0.269631997158965	0.002777099609375	\\
0.269676388333999	0.003204345703125	\\
0.269720779509034	0.0035400390625	\\
0.269765170684068	0.00323486328125	\\
0.269809561859102	0.0030517578125	\\
0.269853953034137	0.002655029296875	\\
0.269898344209171	0.002777099609375	\\
0.269942735384206	0.0030517578125	\\
0.26998712655924	0.002593994140625	\\
0.270031517734274	0.00262451171875	\\
0.270075908909309	0.00299072265625	\\
0.270120300084343	0.002532958984375	\\
0.270164691259378	0.002777099609375	\\
0.270209082434412	0.003387451171875	\\
0.270253473609446	0.003021240234375	\\
0.270297864784481	0.002532958984375	\\
0.270342255959515	0.002777099609375	\\
0.27038664713455	0.002593994140625	\\
0.270431038309584	0.002288818359375	\\
0.270475429484618	0.002410888671875	\\
0.270519820659653	0.002593994140625	\\
0.270564211834687	0.002349853515625	\\
0.270608603009722	0.00213623046875	\\
0.270652994184756	0.002349853515625	\\
0.27069738535979	0.002288818359375	\\
0.270741776534825	0.002410888671875	\\
0.270786167709859	0.00274658203125	\\
0.270830558884894	0.00299072265625	\\
0.270874950059928	0.0025634765625	\\
0.270919341234962	0.002532958984375	\\
0.270963732409997	0.002716064453125	\\
0.271008123585031	0.002716064453125	\\
0.271052514760066	0.002655029296875	\\
0.2710969059351	0.00286865234375	\\
0.271141297110135	0.00299072265625	\\
0.271185688285169	0.002593994140625	\\
0.271230079460203	0.002532958984375	\\
0.271274470635238	0.00274658203125	\\
0.271318861810272	0.002716064453125	\\
0.271363252985307	0.0028076171875	\\
0.271407644160341	0.0028076171875	\\
0.271452035335375	0.003021240234375	\\
0.27149642651041	0.00311279296875	\\
0.271540817685444	0.003326416015625	\\
0.271585208860479	0.00347900390625	\\
0.271629600035513	0.003326416015625	\\
0.271673991210547	0.00311279296875	\\
0.271718382385582	0.003204345703125	\\
0.271762773560616	0.003387451171875	\\
0.271807164735651	0.003570556640625	\\
0.271851555910685	0.003387451171875	\\
0.271895947085719	0.003326416015625	\\
0.271940338260754	0.003265380859375	\\
0.271984729435788	0.00347900390625	\\
0.272029120610823	0.003448486328125	\\
0.272073511785857	0.00347900390625	\\
0.272117902960891	0.003631591796875	\\
0.272162294135926	0.00347900390625	\\
0.27220668531096	0.003692626953125	\\
0.272251076485995	0.003570556640625	\\
0.272295467661029	0.003692626953125	\\
0.272339858836063	0.003753662109375	\\
0.272384250011098	0.003387451171875	\\
0.272428641186132	0.003662109375	\\
0.272473032361167	0.003875732421875	\\
0.272517423536201	0.003936767578125	\\
0.272561814711235	0.004150390625	\\
0.27260620588627	0.003997802734375	\\
0.272650597061304	0.004058837890625	\\
0.272694988236339	0.00396728515625	\\
0.272739379411373	0.003936767578125	\\
0.272783770586407	0.00384521484375	\\
0.272828161761442	0.0035400390625	\\
0.272872552936476	0.0035400390625	\\
0.272916944111511	0.00390625	\\
0.272961335286545	0.00396728515625	\\
0.273005726461579	0.003936767578125	\\
0.273050117636614	0.003814697265625	\\
0.273094508811648	0.003936767578125	\\
0.273138899986683	0.00421142578125	\\
0.273183291161717	0.004058837890625	\\
0.273227682336751	0.003875732421875	\\
0.273272073511786	0.003662109375	\\
0.27331646468682	0.003631591796875	\\
0.273360855861855	0.003997802734375	\\
0.273405247036889	0.00360107421875	\\
0.273449638211923	0.003509521484375	\\
0.273494029386958	0.003814697265625	\\
0.273538420561992	0.00421142578125	\\
0.273582811737027	0.00408935546875	\\
0.273627202912061	0.00360107421875	\\
0.273671594087095	0.00396728515625	\\
0.27371598526213	0.004150390625	\\
0.273760376437164	0.00390625	\\
0.273804767612199	0.00372314453125	\\
0.273849158787233	0.003570556640625	\\
0.273893549962268	0.0037841796875	\\
0.273937941137302	0.00360107421875	\\
0.273982332312336	0.003265380859375	\\
0.274026723487371	0.003387451171875	\\
0.274071114662405	0.003509521484375	\\
0.27411550583744	0.00299072265625	\\
0.274159897012474	0.003204345703125	\\
0.274204288187508	0.0030517578125	\\
0.274248679362543	0.002655029296875	\\
0.274293070537577	0.00299072265625	\\
0.274337461712612	0.0029296875	\\
0.274381852887646	0.003021240234375	\\
0.27442624406268	0.002716064453125	\\
0.274470635237715	0.002838134765625	\\
0.274515026412749	0.002899169921875	\\
0.274559417587784	0.002716064453125	\\
0.274603808762818	0.00286865234375	\\
0.274648199937852	0.00335693359375	\\
0.274692591112887	0.00311279296875	\\
0.274736982287921	0.002960205078125	\\
0.274781373462956	0.0032958984375	\\
0.27482576463799	0.00311279296875	\\
0.274870155813024	0.003204345703125	\\
0.274914546988059	0.003265380859375	\\
0.274958938163093	0.003570556640625	\\
0.275003329338128	0.00390625	\\
0.275047720513162	0.004058837890625	\\
0.275092111688196	0.003997802734375	\\
0.275136502863231	0.003662109375	\\
0.275180894038265	0.00372314453125	\\
0.2752252852133	0.003997802734375	\\
0.275269676388334	0.003814697265625	\\
0.275314067563368	0.003753662109375	\\
0.275358458738403	0.004486083984375	\\
0.275402849913437	0.00439453125	\\
0.275447241088472	0.004119873046875	\\
0.275491632263506	0.004486083984375	\\
0.27553602343854	0.00445556640625	\\
0.275580414613575	0.004425048828125	\\
0.275624805788609	0.004547119140625	\\
0.275669196963644	0.004547119140625	\\
0.275713588138678	0.004669189453125	\\
0.275757979313712	0.00482177734375	\\
0.275802370488747	0.004425048828125	\\
0.275846761663781	0.00439453125	\\
0.275891152838816	0.004608154296875	\\
0.27593554401385	0.0047607421875	\\
0.275979935188884	0.0042724609375	\\
0.276024326363919	0.003692626953125	\\
0.276068717538953	0.004058837890625	\\
0.276113108713988	0.00421142578125	\\
0.276157499889022	0.003997802734375	\\
0.276201891064056	0.0037841796875	\\
0.276246282239091	0.0040283203125	\\
0.276290673414125	0.003753662109375	\\
0.27633506458916	0.003387451171875	\\
0.276379455764194	0.00323486328125	\\
0.276423846939229	0.002899169921875	\\
0.276468238114263	0.002685546875	\\
0.276512629289297	0.002777099609375	\\
0.276557020464332	0.002716064453125	\\
0.276601411639366	0.00286865234375	\\
0.2766458028144	0.00384521484375	\\
0.276690193989435	0.003387451171875	\\
0.276734585164469	0.003021240234375	\\
0.276778976339504	0.003448486328125	\\
0.276823367514538	0.0032958984375	\\
0.276867758689573	0.00311279296875	\\
0.276912149864607	0.0030517578125	\\
0.276956541039641	0.003173828125	\\
0.277000932214676	0.003082275390625	\\
0.27704532338971	0.00262451171875	\\
0.277089714564745	0.0029296875	\\
0.277134105739779	0.00360107421875	\\
0.277178496914813	0.003204345703125	\\
0.277222888089848	0.00299072265625	\\
0.277267279264882	0.002777099609375	\\
0.277311670439917	0.00299072265625	\\
0.277356061614951	0.002960205078125	\\
0.277400452789985	0.003082275390625	\\
0.27744484396502	0.003387451171875	\\
0.277489235140054	0.003448486328125	\\
0.277533626315089	0.00347900390625	\\
0.277578017490123	0.00347900390625	\\
0.277622408665157	0.00335693359375	\\
0.277666799840192	0.003387451171875	\\
0.277711191015226	0.0032958984375	\\
0.277755582190261	0.0029296875	\\
0.277799973365295	0.002960205078125	\\
0.277844364540329	0.003173828125	\\
0.277888755715364	0.003570556640625	\\
0.277933146890398	0.00347900390625	\\
0.277977538065433	0.003204345703125	\\
0.278021929240467	0.002716064453125	\\
0.278066320415501	0.002899169921875	\\
0.278110711590536	0.002838134765625	\\
0.27815510276557	0.0029296875	\\
0.278199493940605	0.0029296875	\\
0.278243885115639	0.0028076171875	\\
0.278288276290673	0.003082275390625	\\
0.278332667465708	0.003509521484375	\\
0.278377058640742	0.003509521484375	\\
0.278421449815777	0.002960205078125	\\
0.278465840990811	0.002777099609375	\\
0.278510232165845	0.00250244140625	\\
0.27855462334088	0.002777099609375	\\
0.278599014515914	0.00299072265625	\\
0.278643405690949	0.002349853515625	\\
0.278687796865983	0.002044677734375	\\
0.278732188041017	0.00225830078125	\\
0.278776579216052	0.002227783203125	\\
0.278820970391086	0.00244140625	\\
0.278865361566121	0.00262451171875	\\
0.278909752741155	0.002838134765625	\\
0.278954143916189	0.00286865234375	\\
0.278998535091224	0.003021240234375	\\
0.279042926266258	0.003326416015625	\\
0.279087317441293	0.002960205078125	\\
0.279131708616327	0.0030517578125	\\
0.279176099791361	0.00323486328125	\\
0.279220490966396	0.003143310546875	\\
0.27926488214143	0.003570556640625	\\
0.279309273316465	0.00347900390625	\\
0.279353664491499	0.00360107421875	\\
0.279398055666533	0.003814697265625	\\
0.279442446841568	0.003753662109375	\\
0.279486838016602	0.0037841796875	\\
0.279531229191637	0.0037841796875	\\
0.279575620366671	0.0040283203125	\\
0.279620011541706	0.003997802734375	\\
0.27966440271674	0.003997802734375	\\
0.279708793891774	0.004058837890625	\\
0.279753185066809	0.004150390625	\\
0.279797576241843	0.003997802734375	\\
0.279841967416878	0.003753662109375	\\
0.279886358591912	0.003692626953125	\\
0.279930749766946	0.00372314453125	\\
0.279975140941981	0.004180908203125	\\
0.280019532117015	0.00439453125	\\
0.28006392329205	0.00439453125	\\
0.280108314467084	0.00433349609375	\\
0.280152705642118	0.004638671875	\\
0.280197096817153	0.004913330078125	\\
0.280241487992187	0.004608154296875	\\
0.280285879167222	0.004791259765625	\\
0.280330270342256	0.00457763671875	\\
0.28037466151729	0.004486083984375	\\
0.280419052692325	0.004608154296875	\\
0.280463443867359	0.004730224609375	\\
0.280507835042394	0.004638671875	\\
0.280552226217428	0.00439453125	\\
0.280596617392462	0.00469970703125	\\
0.280641008567497	0.00445556640625	\\
0.280685399742531	0.0045166015625	\\
0.280729790917566	0.004730224609375	\\
0.2807741820926	0.00433349609375	\\
0.280818573267634	0.00439453125	\\
0.280862964442669	0.00457763671875	\\
0.280907355617703	0.004302978515625	\\
0.280951746792738	0.00445556640625	\\
0.280996137967772	0.00457763671875	\\
0.281040529142806	0.0047607421875	\\
0.281084920317841	0.00469970703125	\\
0.281129311492875	0.00457763671875	\\
0.28117370266791	0.004638671875	\\
0.281218093842944	0.00445556640625	\\
0.281262485017978	0.004974365234375	\\
0.281306876193013	0.00506591796875	\\
0.281351267368047	0.004913330078125	\\
0.281395658543082	0.004638671875	\\
0.281440049718116	0.00439453125	\\
0.28148444089315	0.004730224609375	\\
0.281528832068185	0.004486083984375	\\
0.281573223243219	0.004119873046875	\\
0.281617614418254	0.004058837890625	\\
0.281662005593288	0.00421142578125	\\
0.281706396768322	0.0042724609375	\\
0.281750787943357	0.004669189453125	\\
0.281795179118391	0.004852294921875	\\
0.281839570293426	0.00469970703125	\\
0.28188396146846	0.004730224609375	\\
0.281928352643494	0.0048828125	\\
0.281972743818529	0.004608154296875	\\
0.282017134993563	0.00445556640625	\\
0.282061526168598	0.0050048828125	\\
0.282105917343632	0.004730224609375	\\
0.282150308518667	0.0045166015625	\\
0.282194699693701	0.004669189453125	\\
0.282239090868735	0.004547119140625	\\
0.28228348204377	0.004547119140625	\\
0.282327873218804	0.00433349609375	\\
0.282372264393839	0.004547119140625	\\
0.282416655568873	0.0045166015625	\\
0.282461046743907	0.00433349609375	\\
0.282505437918942	0.004547119140625	\\
0.282549829093976	0.004302978515625	\\
0.282594220269011	0.003997802734375	\\
0.282638611444045	0.00439453125	\\
0.282683002619079	0.00433349609375	\\
0.282727393794114	0.0042724609375	\\
0.282771784969148	0.0042724609375	\\
0.282816176144183	0.00433349609375	\\
0.282860567319217	0.004913330078125	\\
0.282904958494251	0.004852294921875	\\
0.282949349669286	0.004547119140625	\\
0.28299374084432	0.004425048828125	\\
0.283038132019355	0.004608154296875	\\
0.283082523194389	0.00445556640625	\\
0.283126914369423	0.00396728515625	\\
0.283171305544458	0.004425048828125	\\
0.283215696719492	0.00457763671875	\\
0.283260087894527	0.004302978515625	\\
0.283304479069561	0.004241943359375	\\
0.283348870244595	0.004180908203125	\\
0.28339326141963	0.004241943359375	\\
0.283437652594664	0.004180908203125	\\
0.283482043769699	0.00384521484375	\\
0.283526434944733	0.003936767578125	\\
0.283570826119767	0.004058837890625	\\
0.283615217294802	0.0037841796875	\\
0.283659608469836	0.003509521484375	\\
0.283703999644871	0.003387451171875	\\
0.283748390819905	0.003570556640625	\\
0.283792781994939	0.0035400390625	\\
0.283837173169974	0.002838134765625	\\
0.283881564345008	0.002960205078125	\\
0.283925955520043	0.003509521484375	\\
0.283970346695077	0.003021240234375	\\
0.284014737870111	0.00286865234375	\\
0.284059129045146	0.0028076171875	\\
0.28410352022018	0.00244140625	\\
0.284147911395215	0.00213623046875	\\
0.284192302570249	0.002227783203125	\\
0.284236693745283	0.0025634765625	\\
0.284281084920318	0.00244140625	\\
0.284325476095352	0.002777099609375	\\
0.284369867270387	0.002899169921875	\\
0.284414258445421	0.0029296875	\\
0.284458649620455	0.0032958984375	\\
0.28450304079549	0.003204345703125	\\
0.284547431970524	0.003143310546875	\\
0.284591823145559	0.003387451171875	\\
0.284636214320593	0.00311279296875	\\
0.284680605495627	0.002960205078125	\\
0.284724996670662	0.00341796875	\\
0.284769387845696	0.003326416015625	\\
0.284813779020731	0.00323486328125	\\
0.284858170195765	0.003753662109375	\\
0.2849025613708	0.004302978515625	\\
0.284946952545834	0.0040283203125	\\
0.284991343720868	0.0035400390625	\\
0.285035734895903	0.00360107421875	\\
0.285080126070937	0.003631591796875	\\
0.285124517245971	0.00341796875	\\
0.285168908421006	0.003509521484375	\\
0.28521329959604	0.0037841796875	\\
0.285257690771075	0.003814697265625	\\
0.285302081946109	0.00384521484375	\\
0.285346473121144	0.003662109375	\\
0.285390864296178	0.003509521484375	\\
0.285435255471212	0.003631591796875	\\
0.285479646646247	0.003570556640625	\\
0.285524037821281	0.0037841796875	\\
0.285568428996316	0.0037841796875	\\
0.28561282017135	0.003265380859375	\\
0.285657211346384	0.003448486328125	\\
0.285701602521419	0.0035400390625	\\
0.285745993696453	0.00335693359375	\\
0.285790384871488	0.003265380859375	\\
0.285834776046522	0.003265380859375	\\
0.285879167221556	0.003021240234375	\\
0.285923558396591	0.003021240234375	\\
0.285967949571625	0.003173828125	\\
0.28601234074666	0.003143310546875	\\
0.286056731921694	0.00323486328125	\\
0.286101123096728	0.003021240234375	\\
0.286145514271763	0.002593994140625	\\
0.286189905446797	0.002685546875	\\
0.286234296621832	0.0028076171875	\\
0.286278687796866	0.002655029296875	\\
0.2863230789719	0.002655029296875	\\
0.286367470146935	0.00274658203125	\\
0.286411861321969	0.002899169921875	\\
0.286456252497004	0.0030517578125	\\
0.286500643672038	0.00311279296875	\\
0.286545034847072	0.00341796875	\\
0.286589426022107	0.00335693359375	\\
0.286633817197141	0.00299072265625	\\
0.286678208372176	0.003265380859375	\\
0.28672259954721	0.003387451171875	\\
0.286766990722244	0.003326416015625	\\
0.286811381897279	0.00335693359375	\\
0.286855773072313	0.003143310546875	\\
0.286900164247348	0.003509521484375	\\
0.286944555422382	0.00341796875	\\
0.286988946597416	0.00274658203125	\\
0.287033337772451	0.0028076171875	\\
0.287077728947485	0.002899169921875	\\
0.28712212012252	0.003143310546875	\\
0.287166511297554	0.003387451171875	\\
0.287210902472588	0.0035400390625	\\
0.287255293647623	0.003265380859375	\\
0.287299684822657	0.003265380859375	\\
0.287344075997692	0.003448486328125	\\
0.287388467172726	0.00323486328125	\\
0.28743285834776	0.003143310546875	\\
0.287477249522795	0.00323486328125	\\
0.287521640697829	0.003509521484375	\\
0.287566031872864	0.00341796875	\\
0.287610423047898	0.003448486328125	\\
0.287654814222932	0.003448486328125	\\
0.287699205397967	0.00341796875	\\
0.287743596573001	0.003387451171875	\\
0.287787987748036	0.003173828125	\\
0.28783237892307	0.0030517578125	\\
0.287876770098104	0.00286865234375	\\
0.287921161273139	0.00286865234375	\\
0.287965552448173	0.003021240234375	\\
0.288009943623208	0.0029296875	\\
0.288054334798242	0.002838134765625	\\
0.288098725973277	0.002655029296875	\\
0.288143117148311	0.0028076171875	\\
0.288187508323345	0.002899169921875	\\
0.28823189949838	0.002655029296875	\\
0.288276290673414	0.003143310546875	\\
0.288320681848449	0.00299072265625	\\
0.288365073023483	0.003265380859375	\\
0.288409464198517	0.003509521484375	\\
0.288453855373552	0.003387451171875	\\
0.288498246548586	0.0035400390625	\\
0.288542637723621	0.003326416015625	\\
0.288587028898655	0.003265380859375	\\
0.288631420073689	0.00335693359375	\\
0.288675811248724	0.002899169921875	\\
0.288720202423758	0.002593994140625	\\
0.288764593598793	0.002532958984375	\\
0.288808984773827	0.00262451171875	\\
0.288853375948861	0.002655029296875	\\
0.288897767123896	0.002593994140625	\\
0.28894215829893	0.0028076171875	\\
0.288986549473965	0.002899169921875	\\
0.289030940648999	0.002777099609375	\\
0.289075331824033	0.002777099609375	\\
0.289119722999068	0.002960205078125	\\
0.289164114174102	0.002838134765625	\\
0.289208505349137	0.00286865234375	\\
0.289252896524171	0.002777099609375	\\
0.289297287699205	0.0028076171875	\\
0.28934167887424	0.002777099609375	\\
0.289386070049274	0.002685546875	\\
0.289430461224309	0.002532958984375	\\
0.289474852399343	0.002532958984375	\\
0.289519243574377	0.002349853515625	\\
0.289563634749412	0.00244140625	\\
0.289608025924446	0.00244140625	\\
0.289652417099481	0.00262451171875	\\
0.289696808274515	0.003204345703125	\\
0.289741199449549	0.003509521484375	\\
0.289785590624584	0.003173828125	\\
0.289829981799618	0.00311279296875	\\
0.289874372974653	0.002960205078125	\\
0.289918764149687	0.003082275390625	\\
0.289963155324721	0.00335693359375	\\
0.290007546499756	0.0029296875	\\
0.29005193767479	0.002838134765625	\\
0.290096328849825	0.002899169921875	\\
0.290140720024859	0.003265380859375	\\
0.290185111199893	0.00323486328125	\\
0.290229502374928	0.00299072265625	\\
0.290273893549962	0.003173828125	\\
0.290318284724997	0.003021240234375	\\
0.290362675900031	0.003173828125	\\
0.290407067075065	0.002960205078125	\\
0.2904514582501	0.003021240234375	\\
0.290495849425134	0.003173828125	\\
0.290540240600169	0.003204345703125	\\
0.290584631775203	0.003204345703125	\\
0.290629022950238	0.003326416015625	\\
0.290673414125272	0.003662109375	\\
0.290717805300306	0.00341796875	\\
0.290762196475341	0.0032958984375	\\
0.290806587650375	0.003662109375	\\
0.29085097882541	0.003387451171875	\\
0.290895370000444	0.002960205078125	\\
0.290939761175478	0.003204345703125	\\
0.290984152350513	0.003204345703125	\\
0.291028543525547	0.00311279296875	\\
0.291072934700582	0.00335693359375	\\
0.291117325875616	0.0032958984375	\\
0.29116171705065	0.003387451171875	\\
0.291206108225685	0.003387451171875	\\
0.291250499400719	0.00360107421875	\\
0.291294890575754	0.003692626953125	\\
0.291339281750788	0.003082275390625	\\
0.291383672925822	0.002960205078125	\\
0.291428064100857	0.003265380859375	\\
0.291472455275891	0.003173828125	\\
0.291516846450926	0.002838134765625	\\
0.29156123762596	0.0025634765625	\\
0.291605628800994	0.002471923828125	\\
0.291650019976029	0.002166748046875	\\
0.291694411151063	0.002105712890625	\\
0.291738802326098	0.00274658203125	\\
0.291783193501132	0.00238037109375	\\
0.291827584676166	0.002105712890625	\\
0.291871975851201	0.00213623046875	\\
0.291916367026235	0.00213623046875	\\
0.29196075820127	0.002227783203125	\\
0.292005149376304	0.0018310546875	\\
0.292049540551338	0.00225830078125	\\
0.292093931726373	0.0025634765625	\\
0.292138322901407	0.00225830078125	\\
0.292182714076442	0.00244140625	\\
0.292227105251476	0.00177001953125	\\
0.29227149642651	0.0018310546875	\\
0.292315887601545	0.00201416015625	\\
0.292360278776579	0.001708984375	\\
0.292404669951614	0.0015869140625	\\
0.292449061126648	0.001495361328125	\\
0.292493452301682	0.001220703125	\\
0.292537843476717	0.001129150390625	\\
0.292582234651751	0.0010986328125	\\
0.292626625826786	0.0008544921875	\\
0.29267101700182	0.001220703125	\\
0.292715408176854	0.00079345703125	\\
0.292759799351889	0.000518798828125	\\
0.292804190526923	0.00042724609375	\\
0.292848581701958	0.0003662109375	\\
0.292892972876992	0.000396728515625	\\
0.292937364052026	0.000335693359375	\\
0.292981755227061	0.000335693359375	\\
0.293026146402095	0.000244140625	\\
0.29307053757713	0.000701904296875	\\
0.293114928752164	0.0006103515625	\\
0.293159319927198	0.00048828125	\\
0.293203711102233	0.000518798828125	\\
0.293248102277267	0.000335693359375	\\
0.293292493452302	0.00048828125	\\
0.293336884627336	0.000152587890625	\\
0.293381275802371	0.000335693359375	\\
0.293425666977405	6.103515625e-05	\\
0.293470058152439	-0.000274658203125	\\
0.293514449327474	0.000244140625	\\
0.293558840502508	0.000762939453125	\\
0.293603231677542	0.000762939453125	\\
0.293647622852577	0.00018310546875	\\
0.293692014027611	0.00042724609375	\\
0.293736405202646	0.000396728515625	\\
0.29378079637768	0.000518798828125	\\
0.293825187552715	0.00042724609375	\\
0.293869578727749	0.00042724609375	\\
0.293913969902783	0.000640869140625	\\
0.293958361077818	0.000885009765625	\\
0.294002752252852	0.000732421875	\\
0.294047143427887	0.000762939453125	\\
0.294091534602921	0.000885009765625	\\
0.294135925777955	0.000732421875	\\
0.29418031695299	0.000701904296875	\\
0.294224708128024	0.0006103515625	\\
0.294269099303059	0.000823974609375	\\
0.294313490478093	0.000701904296875	\\
0.294357881653127	0.0008544921875	\\
0.294402272828162	0.001129150390625	\\
0.294446664003196	0.001373291015625	\\
0.294491055178231	0.001556396484375	\\
0.294535446353265	0.001556396484375	\\
0.294579837528299	0.001739501953125	\\
0.294624228703334	0.00177001953125	\\
0.294668619878368	0.0018310546875	\\
0.294713011053403	0.00164794921875	\\
0.294757402228437	0.001739501953125	\\
0.294801793403471	0.002044677734375	\\
0.294846184578506	0.002349853515625	\\
0.29489057575354	0.001922607421875	\\
0.294934966928575	0.002166748046875	\\
0.294979358103609	0.002777099609375	\\
0.295023749278643	0.0025634765625	\\
0.295068140453678	0.0025634765625	\\
0.295112531628712	0.002777099609375	\\
0.295156922803747	0.0023193359375	\\
0.295201313978781	0.002532958984375	\\
0.295245705153815	0.002838134765625	\\
0.29529009632885	0.002655029296875	\\
0.295334487503884	0.002410888671875	\\
0.295378878678919	0.00250244140625	\\
0.295423269853953	0.0028076171875	\\
0.295467661028987	0.0029296875	\\
0.295512052204022	0.00244140625	\\
0.295556443379056	0.00225830078125	\\
0.295600834554091	0.002410888671875	\\
0.295645225729125	0.00225830078125	\\
0.295689616904159	0.002227783203125	\\
0.295734008079194	0.00225830078125	\\
0.295778399254228	0.001953125	\\
0.295822790429263	0.00177001953125	\\
0.295867181604297	0.001678466796875	\\
0.295911572779331	0.001708984375	\\
0.295955963954366	0.00189208984375	\\
0.2960003551294	0.0018310546875	\\
0.296044746304435	0.001434326171875	\\
0.296089137479469	0.001251220703125	\\
0.296133528654503	0.001373291015625	\\
0.296177919829538	0.001373291015625	\\
0.296222311004572	0.001495361328125	\\
0.296266702179607	0.001495361328125	\\
0.296311093354641	0.001434326171875	\\
0.296355484529676	0.002166748046875	\\
0.29639987570471	0.00225830078125	\\
0.296444266879744	0.001800537109375	\\
0.296488658054779	0.001953125	\\
0.296533049229813	0.002227783203125	\\
0.296577440404848	0.002227783203125	\\
0.296621831579882	0.001922607421875	\\
0.296666222754916	0.00201416015625	\\
0.296710613929951	0.002227783203125	\\
0.296755005104985	0.001495361328125	\\
0.29679939628002	0.001983642578125	\\
0.296843787455054	0.002197265625	\\
0.296888178630088	0.001678466796875	\\
0.296932569805123	0.001800537109375	\\
0.296976960980157	0.00189208984375	\\
0.297021352155192	0.00225830078125	\\
0.297065743330226	0.002044677734375	\\
0.29711013450526	0.001800537109375	\\
0.297154525680295	0.001617431640625	\\
0.297198916855329	0.001495361328125	\\
0.297243308030364	0.00164794921875	\\
0.297287699205398	0.001922607421875	\\
0.297332090380432	0.001922607421875	\\
0.297376481555467	0.0018310546875	\\
0.297420872730501	0.001953125	\\
0.297465263905536	0.00238037109375	\\
0.29750965508057	0.0023193359375	\\
0.297554046255604	0.0020751953125	\\
0.297598437430639	0.001800537109375	\\
0.297642828605673	0.001739501953125	\\
0.297687219780708	0.002166748046875	\\
0.297731610955742	0.002044677734375	\\
0.297776002130776	0.001800537109375	\\
0.297820393305811	0.00189208984375	\\
0.297864784480845	0.001861572265625	\\
0.29790917565588	0.001800537109375	\\
0.297953566830914	0.001861572265625	\\
0.297997958005948	0.00177001953125	\\
0.298042349180983	0.00164794921875	\\
0.298086740356017	0.001373291015625	\\
0.298131131531052	0.001373291015625	\\
0.298175522706086	0.001556396484375	\\
0.29821991388112	0.001556396484375	\\
0.298264305056155	0.00128173828125	\\
0.298308696231189	0.0015869140625	\\
0.298353087406224	0.0015869140625	\\
0.298397478581258	0.001251220703125	\\
0.298441869756292	0.001129150390625	\\
0.298486260931327	0.001251220703125	\\
0.298530652106361	0.00140380859375	\\
0.298575043281396	0.001708984375	\\
0.29861943445643	0.0018310546875	\\
0.298663825631464	0.001312255859375	\\
0.298708216806499	0.00103759765625	\\
0.298752607981533	0.000732421875	\\
0.298796999156568	0.000823974609375	\\
0.298841390331602	0.001129150390625	\\
0.298885781506636	0.001007080078125	\\
0.298930172681671	0.001129150390625	\\
0.298974563856705	0.001617431640625	\\
0.29901895503174	0.001373291015625	\\
0.299063346206774	0.00128173828125	\\
0.299107737381809	0.001068115234375	\\
0.299152128556843	0.001251220703125	\\
0.299196519731877	0.001373291015625	\\
0.299240910906912	0.0010986328125	\\
0.299285302081946	0.001251220703125	\\
0.299329693256981	0.00115966796875	\\
0.299374084432015	0.001129150390625	\\
0.299418475607049	0.00140380859375	\\
0.299462866782084	0.001556396484375	\\
0.299507257957118	0.001434326171875	\\
0.299551649132153	0.001220703125	\\
0.299596040307187	0.001129150390625	\\
0.299640431482221	0.00079345703125	\\
0.299684822657256	0.000701904296875	\\
0.29972921383229	0.000885009765625	\\
0.299773605007325	0.0010986328125	\\
0.299817996182359	0.001068115234375	\\
0.299862387357393	0.001007080078125	\\
0.299906778532428	0.0009765625	\\
0.299951169707462	0.001220703125	\\
0.299995560882497	0.00115966796875	\\
0.300039952057531	0.001312255859375	\\
0.300084343232565	0.001190185546875	\\
0.3001287344076	0.001129150390625	\\
0.300173125582634	0.001251220703125	\\
0.300217516757669	0.000946044921875	\\
0.300261907932703	0.00054931640625	\\
0.300306299107737	0.00054931640625	\\
0.300350690282772	0.000396728515625	\\
0.300395081457806	0.000213623046875	\\
0.300439472632841	0.0003662109375	\\
0.300483863807875	0.000640869140625	\\
0.300528254982909	0.000579833984375	\\
0.300572646157944	0.000701904296875	\\
0.300617037332978	0.000701904296875	\\
0.300661428508013	0.000701904296875	\\
0.300705819683047	0.00091552734375	\\
0.300750210858081	0.000640869140625	\\
0.300794602033116	0.000579833984375	\\
0.30083899320815	0.000762939453125	\\
0.300883384383185	0.0010986328125	\\
0.300927775558219	0.001312255859375	\\
0.300972166733253	0.0008544921875	\\
0.301016557908288	0.00079345703125	\\
0.301060949083322	0.000823974609375	\\
0.301105340258357	0.000946044921875	\\
0.301149731433391	0.000762939453125	\\
0.301194122608425	0.000762939453125	\\
0.30123851378346	0.00128173828125	\\
0.301282904958494	0.0018310546875	\\
0.301327296133529	0.001708984375	\\
0.301371687308563	0.001495361328125	\\
0.301416078483597	0.001129150390625	\\
0.301460469658632	0.001190185546875	\\
0.301504860833666	0.00201416015625	\\
0.301549252008701	0.0018310546875	\\
0.301593643183735	0.001251220703125	\\
0.301638034358769	0.00115966796875	\\
0.301682425533804	0.001312255859375	\\
0.301726816708838	0.00115966796875	\\
0.301771207883873	0.001190185546875	\\
0.301815599058907	0.00152587890625	\\
0.301859990233942	0.001312255859375	\\
0.301904381408976	0.001068115234375	\\
0.30194877258401	0.001068115234375	\\
0.301993163759045	0.000885009765625	\\
0.302037554934079	0.00079345703125	\\
0.302081946109113	0.000762939453125	\\
0.302126337284148	0.000823974609375	\\
0.302170728459182	0.00067138671875	\\
0.302215119634217	0.001190185546875	\\
0.302259510809251	0.001495361328125	\\
0.302303901984286	0.001129150390625	\\
0.30234829315932	0.001220703125	\\
0.302392684334354	0.001068115234375	\\
0.302437075509389	0.000946044921875	\\
0.302481466684423	0.000762939453125	\\
0.302525857859458	0.000518798828125	\\
0.302570249034492	0.0006103515625	\\
0.302614640209526	0.0006103515625	\\
0.302659031384561	0.000701904296875	\\
0.302703422559595	0.00030517578125	\\
0.30274781373463	0.00067138671875	\\
0.302792204909664	0.000762939453125	\\
0.302836596084698	0.000701904296875	\\
0.302880987259733	0.0009765625	\\
0.302925378434767	0.0008544921875	\\
0.302969769609802	0.0008544921875	\\
0.303014160784836	0.00091552734375	\\
0.30305855195987	0.001129150390625	\\
0.303102943134905	0.00115966796875	\\
0.303147334309939	0.000823974609375	\\
0.303191725484974	0.000701904296875	\\
0.303236116660008	0.0009765625	\\
0.303280507835042	0.000885009765625	\\
0.303324899010077	0.000885009765625	\\
0.303369290185111	0.000518798828125	\\
0.303413681360146	0.000579833984375	\\
0.30345807253518	0.000762939453125	\\
0.303502463710214	0.000274658203125	\\
0.303546854885249	0.00018310546875	\\
0.303591246060283	-0.0001220703125	\\
0.303635637235318	-0.00042724609375	\\
0.303680028410352	9.1552734375e-05	\\
0.303724419585386	0.000152587890625	\\
0.303768810760421	-0.00030517578125	\\
0.303813201935455	-0.000152587890625	\\
0.30385759311049	0.000244140625	\\
0.303901984285524	0.00030517578125	\\
0.303946375460558	3.0517578125e-05	\\
0.303990766635593	0	\\
0.304035157810627	0	\\
0.304079548985662	0.000152587890625	\\
0.304123940160696	0.000640869140625	\\
0.30416833133573	0.000518798828125	\\
0.304212722510765	0.000457763671875	\\
0.304257113685799	0.00054931640625	\\
0.304301504860834	0.000640869140625	\\
0.304345896035868	0.000946044921875	\\
0.304390287210903	0.000946044921875	\\
0.304434678385937	0.0009765625	\\
0.304479069560971	0.00140380859375	\\
0.304523460736006	0.001800537109375	\\
0.30456785191104	0.001739501953125	\\
0.304612243086074	0.001739501953125	\\
0.304656634261109	0.00213623046875	\\
0.304701025436143	0.0020751953125	\\
0.304745416611178	0.001922607421875	\\
0.304789807786212	0.00213623046875	\\
0.304834198961247	0.001953125	\\
0.304878590136281	0.002197265625	\\
0.304922981311315	0.002227783203125	\\
0.30496737248635	0.001800537109375	\\
0.305011763661384	0.0018310546875	\\
0.305056154836419	0.0015869140625	\\
0.305100546011453	0.001708984375	\\
0.305144937186487	0.0018310546875	\\
0.305189328361522	0.001678466796875	\\
0.305233719536556	0.001708984375	\\
0.305278110711591	0.001800537109375	\\
0.305322501886625	0.0013427734375	\\
0.305366893061659	0.001495361328125	\\
0.305411284236694	0.00140380859375	\\
0.305455675411728	0.000946044921875	\\
0.305500066586763	0.000946044921875	\\
0.305544457761797	0.000946044921875	\\
0.305588848936831	0.000946044921875	\\
0.305633240111866	0.0008544921875	\\
0.3056776312869	0.000946044921875	\\
0.305722022461935	0.00091552734375	\\
0.305766413636969	0.000579833984375	\\
0.305810804812003	0.000762939453125	\\
0.305855195987038	0.000701904296875	\\
0.305899587162072	0.000762939453125	\\
0.305943978337107	0.000946044921875	\\
0.305988369512141	0.000946044921875	\\
0.306032760687175	0.000885009765625	\\
0.30607715186221	0.000457763671875	\\
0.306121543037244	0.001068115234375	\\
0.306165934212279	0.001312255859375	\\
0.306210325387313	0.00152587890625	\\
0.306254716562347	0.001800537109375	\\
0.306299107737382	0.001495361328125	\\
0.306343498912416	0.001922607421875	\\
0.306387890087451	0.002227783203125	\\
0.306432281262485	0.00201416015625	\\
0.306476672437519	0.001953125	\\
0.306521063612554	0.002166748046875	\\
0.306565454787588	0.001922607421875	\\
0.306609845962623	0.00213623046875	\\
0.306654237137657	0.001922607421875	\\
0.306698628312691	0.002105712890625	\\
0.306743019487726	0.00213623046875	\\
0.30678741066276	0.001922607421875	\\
0.306831801837795	0.00177001953125	\\
0.306876193012829	0.00146484375	\\
0.306920584187863	0.001312255859375	\\
0.306964975362898	0.00091552734375	\\
0.307009366537932	0.0009765625	\\
0.307053757712967	0.000946044921875	\\
0.307098148888001	0.000701904296875	\\
0.307142540063035	0.000244140625	\\
0.30718693123807	-0.000152587890625	\\
0.307231322413104	-0.000244140625	\\
0.307275713588139	-0.0001220703125	\\
0.307320104763173	0	\\
0.307364495938207	9.1552734375e-05	\\
0.307408887113242	9.1552734375e-05	\\
0.307453278288276	-0.000244140625	\\
0.307497669463311	-0.000335693359375	\\
0.307542060638345	-0.000518798828125	\\
0.30758645181338	-0.000732421875	\\
0.307630842988414	-0.000762939453125	\\
0.307675234163448	-0.000732421875	\\
0.307719625338483	-0.00103759765625	\\
0.307764016513517	-0.0008544921875	\\
0.307808407688552	-0.000732421875	\\
0.307852798863586	-0.000579833984375	\\
0.30789719003862	-0.000762939453125	\\
0.307941581213655	-0.000732421875	\\
0.307985972388689	-0.000274658203125	\\
0.308030363563724	-0.000518798828125	\\
0.308074754738758	-0.00048828125	\\
0.308119145913792	-0.0003662109375	\\
0.308163537088827	-0.000457763671875	\\
0.308207928263861	-0.000152587890625	\\
0.308252319438896	-3.0517578125e-05	\\
0.30829671061393	6.103515625e-05	\\
0.308341101788964	6.103515625e-05	\\
0.308385492963999	-9.1552734375e-05	\\
0.308429884139033	0.00030517578125	\\
0.308474275314068	9.1552734375e-05	\\
0.308518666489102	0.00018310546875	\\
0.308563057664136	0.000518798828125	\\
0.308607448839171	0.000244140625	\\
0.308651840014205	0.000701904296875	\\
0.30869623118924	0.00079345703125	\\
0.308740622364274	0.000274658203125	\\
0.308785013539308	3.0517578125e-05	\\
0.308829404714343	0.0001220703125	\\
0.308873795889377	0.00048828125	\\
0.308918187064412	0.000152587890625	\\
0.308962578239446	-0.00018310546875	\\
0.30900696941448	-3.0517578125e-05	\\
0.309051360589515	-0.000396728515625	\\
0.309095751764549	-0.000518798828125	\\
0.309140142939584	0.00030517578125	\\
0.309184534114618	0.000335693359375	\\
0.309228925289652	0.000152587890625	\\
0.309273316464687	0.0006103515625	\\
0.309317707639721	0.000823974609375	\\
0.309362098814756	0.00042724609375	\\
0.30940648998979	0.000396728515625	\\
0.309450881164824	9.1552734375e-05	\\
0.309495272339859	-0.000335693359375	\\
0.309539663514893	0.000213623046875	\\
0.309584054689928	-3.0517578125e-05	\\
0.309628445864962	-0.0003662109375	\\
0.309672837039996	0	\\
0.309717228215031	6.103515625e-05	\\
0.309761619390065	0.000396728515625	\\
0.3098060105651	0.00054931640625	\\
0.309850401740134	0.00018310546875	\\
0.309894792915168	0.000244140625	\\
0.309939184090203	0.0003662109375	\\
0.309983575265237	0.0003662109375	\\
0.310027966440272	0.000457763671875	\\
0.310072357615306	0.00030517578125	\\
0.31011674879034	0.00030517578125	\\
0.310161139965375	0.000335693359375	\\
0.310205531140409	0.000457763671875	\\
0.310249922315444	0.0008544921875	\\
0.310294313490478	0.000640869140625	\\
0.310338704665513	0.00054931640625	\\
0.310383095840547	0.0006103515625	\\
0.310427487015581	0.000762939453125	\\
0.310471878190616	0.001007080078125	\\
0.31051626936565	0.001068115234375	\\
0.310560660540684	0.00128173828125	\\
0.310605051715719	0.001251220703125	\\
0.310649442890753	0.0013427734375	\\
0.310693834065788	0.001220703125	\\
0.310738225240822	0.00091552734375	\\
0.310782616415857	0.000946044921875	\\
0.310827007590891	0.00091552734375	\\
0.310871398765925	0.000762939453125	\\
0.31091578994096	0.00054931640625	\\
0.310960181115994	0.000274658203125	\\
0.311004572291029	0.000396728515625	\\
0.311048963466063	0.000213623046875	\\
0.311093354641097	0.000396728515625	\\
0.311137745816132	0.00042724609375	\\
0.311182136991166	0.00018310546875	\\
0.311226528166201	0.00018310546875	\\
0.311270919341235	-0.0001220703125	\\
0.311315310516269	-6.103515625e-05	\\
0.311359701691304	-0.000396728515625	\\
0.311404092866338	-0.000152587890625	\\
0.311448484041373	-0.00030517578125	\\
0.311492875216407	-0.00042724609375	\\
0.311537266391441	-6.103515625e-05	\\
0.311581657566476	-0.000152587890625	\\
0.31162604874151	-0.0001220703125	\\
0.311670439916545	-0.000244140625	\\
0.311714831091579	-6.103515625e-05	\\
0.311759222266613	0.000213623046875	\\
0.311803613441648	9.1552734375e-05	\\
0.311848004616682	-0.000335693359375	\\
0.311892395791717	-3.0517578125e-05	\\
0.311936786966751	-9.1552734375e-05	\\
0.311981178141785	0.00018310546875	\\
0.31202556931682	0.00018310546875	\\
0.312069960491854	9.1552734375e-05	\\
0.312114351666889	0.000701904296875	\\
0.312158742841923	0.00042724609375	\\
0.312203134016957	0.000579833984375	\\
0.312247525191992	0.001190185546875	\\
0.312291916367026	0.001312255859375	\\
0.312336307542061	0.0013427734375	\\
0.312380698717095	0.00115966796875	\\
0.312425089892129	0.00103759765625	\\
0.312469481067164	0.0009765625	\\
0.312513872242198	0.00079345703125	\\
0.312558263417233	0.001220703125	\\
0.312602654592267	0.001129150390625	\\
0.312647045767301	0.0008544921875	\\
0.312691436942336	0.001190185546875	\\
0.31273582811737	0.00128173828125	\\
0.312780219292405	0.001068115234375	\\
0.312824610467439	0.001190185546875	\\
0.312869001642474	0.001373291015625	\\
0.312913392817508	0.001190185546875	\\
0.312957783992542	0.000946044921875	\\
0.313002175167577	0.000732421875	\\
0.313046566342611	0.001007080078125	\\
0.313090957517645	0.001373291015625	\\
0.31313534869268	0.001068115234375	\\
0.313179739867714	0.000518798828125	\\
0.313224131042749	0.000579833984375	\\
0.313268522217783	0.000823974609375	\\
0.313312913392818	0.000640869140625	\\
0.313357304567852	0.000396728515625	\\
0.313401695742886	0.00030517578125	\\
0.313446086917921	0.0003662109375	\\
0.313490478092955	0.00018310546875	\\
0.31353486926799	0.000274658203125	\\
0.313579260443024	0.0006103515625	\\
0.313623651618058	0.00079345703125	\\
0.313668042793093	0.0008544921875	\\
0.313712433968127	0.000213623046875	\\
0.313756825143162	0	\\
0.313801216318196	0.00048828125	\\
0.31384560749323	0.000335693359375	\\
0.313889998668265	0.00054931640625	\\
0.313934389843299	0.0009765625	\\
0.313978781018334	0.001190185546875	\\
0.314023172193368	0.001220703125	\\
0.314067563368402	0.001434326171875	\\
0.314111954543437	0.001495361328125	\\
0.314156345718471	0.00152587890625	\\
0.314200736893506	0.00115966796875	\\
0.31424512806854	0.000518798828125	\\
0.314289519243574	0.000732421875	\\
0.314333910418609	0.001434326171875	\\
0.314378301593643	0.0013427734375	\\
0.314422692768678	0.0013427734375	\\
0.314467083943712	0.001617431640625	\\
0.314511475118746	0.001434326171875	\\
0.314555866293781	0.001678466796875	\\
0.314600257468815	0.00189208984375	\\
0.31464464864385	0.001556396484375	\\
0.314689039818884	0.00152587890625	\\
0.314733430993918	0.00128173828125	\\
0.314777822168953	0.001495361328125	\\
0.314822213343987	0.001434326171875	\\
0.314866604519022	0.001190185546875	\\
0.314910995694056	0.0008544921875	\\
0.31495538686909	0.000244140625	\\
0.314999778044125	0.0003662109375	\\
0.315044169219159	0.000640869140625	\\
0.315088560394194	0.00048828125	\\
0.315132951569228	0.000335693359375	\\
0.315177342744262	0.000213623046875	\\
0.315221733919297	-0.00018310546875	\\
0.315266125094331	-0.000579833984375	\\
0.315310516269366	-0.00054931640625	\\
0.3153549074444	0.0001220703125	\\
0.315399298619434	-0.00018310546875	\\
0.315443689794469	-0.00030517578125	\\
0.315488080969503	-0.00018310546875	\\
0.315532472144538	-0.000152587890625	\\
0.315576863319572	-0.0001220703125	\\
0.315621254494606	-0.000213623046875	\\
0.315665645669641	-6.103515625e-05	\\
0.315710036844675	0.000244140625	\\
0.31575442801971	-3.0517578125e-05	\\
0.315798819194744	-0.000152587890625	\\
0.315843210369778	-0.000335693359375	\\
0.315887601544813	-0.000152587890625	\\
0.315931992719847	0.000335693359375	\\
0.315976383894882	-9.1552734375e-05	\\
0.316020775069916	0.000335693359375	\\
0.316065166244951	0.000701904296875	\\
0.316109557419985	0.000732421875	\\
0.316153948595019	0.00115966796875	\\
0.316198339770054	0.00115966796875	\\
0.316242730945088	0.00128173828125	\\
0.316287122120123	0.0013427734375	\\
0.316331513295157	0.0013427734375	\\
0.316375904470191	0.00152587890625	\\
0.316420295645226	0.001220703125	\\
0.31646468682026	0.00128173828125	\\
0.316509077995295	0.001434326171875	\\
0.316553469170329	0.001373291015625	\\
0.316597860345363	0.001373291015625	\\
0.316642251520398	0.001220703125	\\
0.316686642695432	0.00128173828125	\\
0.316731033870467	0.00103759765625	\\
0.316775425045501	0.00091552734375	\\
0.316819816220535	0.0006103515625	\\
0.31686420739557	0.00091552734375	\\
0.316908598570604	0.001068115234375	\\
0.316952989745639	0.000885009765625	\\
0.316997380920673	0.001251220703125	\\
0.317041772095707	0.0009765625	\\
0.317086163270742	0.00091552734375	\\
0.317130554445776	0.000762939453125	\\
0.317174945620811	0.0001220703125	\\
0.317219336795845	0.000152587890625	\\
0.317263727970879	0.000152587890625	\\
0.317308119145914	0.0001220703125	\\
0.317352510320948	-0.000152587890625	\\
0.317396901495983	-0.00048828125	\\
0.317441292671017	-0.000244140625	\\
0.317485683846051	-0.000457763671875	\\
0.317530075021086	-0.000732421875	\\
0.31757446619612	-0.000762939453125	\\
0.317618857371155	-0.000946044921875	\\
0.317663248546189	-0.001068115234375	\\
0.317707639721223	-0.00128173828125	\\
0.317752030896258	-0.00115966796875	\\
0.317796422071292	-0.00146484375	\\
0.317840813246327	-0.001312255859375	\\
0.317885204421361	-0.00079345703125	\\
0.317929595596395	-0.0013427734375	\\
0.31797398677143	-0.0013427734375	\\
0.318018377946464	-0.0013427734375	\\
0.318062769121499	-0.00140380859375	\\
0.318107160296533	-0.000946044921875	\\
0.318151551471567	-0.000823974609375	\\
0.318195942646602	-0.000885009765625	\\
0.318240333821636	-0.000885009765625	\\
0.318284724996671	-0.0006103515625	\\
0.318329116171705	-0.000396728515625	\\
0.318373507346739	-0.000457763671875	\\
0.318417898521774	-0.000274658203125	\\
0.318462289696808	3.0517578125e-05	\\
0.318506680871843	0.00018310546875	\\
0.318551072046877	-0.00018310546875	\\
0.318595463221911	-0.00042724609375	\\
0.318639854396946	3.0517578125e-05	\\
0.31868424557198	-3.0517578125e-05	\\
0.318728636747015	3.0517578125e-05	\\
0.318773027922049	0.00030517578125	\\
0.318817419097084	0.0006103515625	\\
0.318861810272118	0.00079345703125	\\
0.318906201447152	0.000640869140625	\\
0.318950592622187	0.000946044921875	\\
0.318994983797221	0.000946044921875	\\
0.319039374972256	0.000762939453125	\\
0.31908376614729	0.001007080078125	\\
0.319128157322324	0.00103759765625	\\
0.319172548497359	0.000946044921875	\\
0.319216939672393	0.0008544921875	\\
0.319261330847428	0.000732421875	\\
0.319305722022462	0.0009765625	\\
0.319350113197496	0.001007080078125	\\
0.319394504372531	0.00067138671875	\\
0.319438895547565	0.0009765625	\\
0.3194832867226	0.001220703125	\\
0.319527677897634	0.001251220703125	\\
0.319572069072668	0.001129150390625	\\
0.319616460247703	0.00103759765625	\\
0.319660851422737	0.0009765625	\\
0.319705242597772	0.00042724609375	\\
0.319749633772806	0.000274658203125	\\
0.31979402494784	0.000579833984375	\\
0.319838416122875	0.000335693359375	\\
0.319882807297909	0.00042724609375	\\
0.319927198472944	0.00042724609375	\\
0.319971589647978	0.00030517578125	\\
0.320015980823012	0.000579833984375	\\
0.320060371998047	0.00079345703125	\\
0.320104763173081	0.000732421875	\\
0.320149154348116	0.001220703125	\\
0.32019354552315	0.000946044921875	\\
0.320237936698184	0.000701904296875	\\
0.320282327873219	0.0010986328125	\\
0.320326719048253	0.001007080078125	\\
0.320371110223288	0.00091552734375	\\
0.320415501398322	0.001220703125	\\
0.320459892573356	0.001068115234375	\\
0.320504283748391	0.0006103515625	\\
0.320548674923425	0.00042724609375	\\
0.32059306609846	0.00067138671875	\\
0.320637457273494	0.00103759765625	\\
0.320681848448528	0.00048828125	\\
0.320726239623563	0.000213623046875	\\
0.320770630798597	0.000579833984375	\\
0.320815021973632	0.000762939453125	\\
0.320859413148666	0.000640869140625	\\
0.3209038043237	0.00067138671875	\\
0.320948195498735	0.00067138671875	\\
0.320992586673769	0.00067138671875	\\
0.321036977848804	0.0006103515625	\\
0.321081369023838	0.000396728515625	\\
0.321125760198872	9.1552734375e-05	\\
0.321170151373907	0.00018310546875	\\
0.321214542548941	0	\\
0.321258933723976	3.0517578125e-05	\\
0.32130332489901	0	\\
0.321347716074045	-0.0003662109375	\\
0.321392107249079	-0.000518798828125	\\
0.321436498424113	-0.000762939453125	\\
0.321480889599148	-0.000640869140625	\\
0.321525280774182	-0.00048828125	\\
0.321569671949216	-0.00079345703125	\\
0.321614063124251	-0.000885009765625	\\
0.321658454299285	-0.00103759765625	\\
0.32170284547432	-0.001251220703125	\\
0.321747236649354	-0.001220703125	\\
0.321791627824389	-0.000762939453125	\\
0.321836018999423	-0.0010986328125	\\
0.321880410174457	-0.001556396484375	\\
0.321924801349492	-0.0008544921875	\\
0.321969192524526	-0.001007080078125	\\
0.322013583699561	-0.001129150390625	\\
0.322057974874595	-0.001129150390625	\\
0.322102366049629	-0.001495361328125	\\
0.322146757224664	-0.001373291015625	\\
0.322191148399698	-0.001220703125	\\
0.322235539574733	-0.001251220703125	\\
0.322279930749767	-0.00091552734375	\\
0.322324321924801	-0.00103759765625	\\
0.322368713099836	-0.001373291015625	\\
0.32241310427487	-0.001251220703125	\\
0.322457495449905	-0.001373291015625	\\
0.322501886624939	-0.000823974609375	\\
0.322546277799973	-0.000732421875	\\
0.322590668975008	-0.00091552734375	\\
0.322635060150042	-0.000518798828125	\\
0.322679451325077	-0.00048828125	\\
0.322723842500111	-0.000579833984375	\\
0.322768233675145	-0.0008544921875	\\
0.32281262485018	-0.00054931640625	\\
0.322857016025214	-0.00054931640625	\\
0.322901407200249	-0.001251220703125	\\
0.322945798375283	-0.001129150390625	\\
0.322990189550317	-0.000732421875	\\
0.323034580725352	-0.00091552734375	\\
0.323078971900386	-0.000732421875	\\
0.323123363075421	-0.001068115234375	\\
0.323167754250455	-0.001373291015625	\\
0.323212145425489	-0.001373291015625	\\
0.323256536600524	-0.001251220703125	\\
0.323300927775558	-0.00146484375	\\
0.323345318950593	-0.00140380859375	\\
0.323389710125627	-0.00115966796875	\\
0.323434101300661	-0.001373291015625	\\
0.323478492475696	-0.00146484375	\\
0.32352288365073	-0.001373291015625	\\
0.323567274825765	-0.001190185546875	\\
0.323611666000799	-0.0008544921875	\\
0.323656057175833	-0.0009765625	\\
0.323700448350868	-0.001068115234375	\\
0.323744839525902	-0.0008544921875	\\
0.323789230700937	-0.001220703125	\\
0.323833621875971	-0.001007080078125	\\
0.323878013051005	-0.0006103515625	\\
0.32392240422604	-0.000579833984375	\\
0.323966795401074	-0.000274658203125	\\
0.324011186576109	-0.000701904296875	\\
0.324055577751143	-0.00067138671875	\\
0.324099968926177	-0.00054931640625	\\
0.324144360101212	-0.00067138671875	\\
0.324188751276246	-0.00067138671875	\\
0.324233142451281	-0.00030517578125	\\
0.324277533626315	-0.0008544921875	\\
0.324321924801349	-0.00042724609375	\\
0.324366315976384	-3.0517578125e-05	\\
0.324410707151418	-6.103515625e-05	\\
0.324455098326453	-0.000152587890625	\\
0.324499489501487	-0.000335693359375	\\
0.324543880676522	-0.00030517578125	\\
0.324588271851556	0.000213623046875	\\
0.32463266302659	-0.0003662109375	\\
0.324677054201625	-0.000640869140625	\\
0.324721445376659	-0.000885009765625	\\
0.324765836551694	-0.000701904296875	\\
0.324810227726728	-0.0006103515625	\\
0.324854618901762	-0.00091552734375	\\
0.324899010076797	-0.001373291015625	\\
0.324943401251831	-0.001251220703125	\\
0.324987792426866	-0.001373291015625	\\
0.3250321836019	-0.0018310546875	\\
0.325076574776934	-0.00152587890625	\\
0.325120965951969	-0.0025634765625	\\
0.325165357127003	-0.00128173828125	\\
0.325209748302038	-0.003692626953125	\\
0.325254139477072	-0.00054931640625	\\
0.325298530652106	-0.00439453125	\\
0.325342921827141	0.00042724609375	\\
0.325387313002175	-0.006591796875	\\
0.32543170417721	0.0018310546875	\\
0.325476095352244	-0.00897216796875	\\
0.325520486527278	0.00439453125	\\
0.325564877702313	-0.01190185546875	\\
0.325609268877347	0.0062255859375	\\
0.325653660052382	-0.026702880859375	\\
0.325698051227416	0.038543701171875	\\
0.32574244240245	-0.0634765625	\\
0.325786833577485	0.088104248046875	\\
0.325831224752519	-0.210479736328125	\\
0.325875615927554	-0.98052978515625	\\
0.325920007102588	-0.059234619140625	\\
0.325964398277622	0.336761474609375	\\
0.326008789452657	0.103729248046875	\\
0.326053180627691	0.0184326171875	\\
0.326097571802726	0.081329345703125	\\
0.32614196297776	0.3311767578125	\\
0.326186354152794	0.280731201171875	\\
0.326230745327829	0.310760498046875	\\
0.326275136502863	0.06005859375	\\
0.326319527677898	-0.126678466796875	\\
0.326363918852932	0.05010986328125	\\
0.326408310027966	0.125701904296875	\\
0.326452701203001	0.091949462890625	\\
0.326497092378035	0.054718017578125	\\
0.32654148355307	-0.024444580078125	\\
0.326585874728104	-0.019073486328125	\\
0.326630265903138	0.062896728515625	\\
0.326674657078173	0.022552490234375	\\
0.326719048253207	-0.08624267578125	\\
0.326763439428242	-0.061614990234375	\\
0.326807830603276	0.009521484375	\\
0.32685222177831	0.127960205078125	\\
0.326896612953345	0.079437255859375	\\
0.326941004128379	-0.053131103515625	\\
0.326985395303414	-0.137481689453125	\\
0.327029786478448	-0.1737060546875	\\
0.327074177653483	-0.1812744140625	\\
0.327118568828517	-0.1583251953125	\\
0.327162960003551	-0.116302490234375	\\
0.327207351178586	-0.0584716796875	\\
0.32725174235362	0.021026611328125	\\
0.327296133528655	0.03997802734375	\\
0.327340524703689	-0.016632080078125	\\
0.327384915878723	-0.02130126953125	\\
0.327429307053758	-0.0579833984375	\\
0.327473698228792	-0.1162109375	\\
0.327518089403827	-0.102508544921875	\\
0.327562480578861	-0.054229736328125	\\
0.327606871753895	-0.036468505859375	\\
0.32765126292893	-0.023651123046875	\\
0.327695654103964	0.036163330078125	\\
0.327740045278999	0.03680419921875	\\
0.327784436454033	0.0054931640625	\\
0.327828827629067	-0.04473876953125	\\
0.327873218804102	-0.03009033203125	\\
0.327917609979136	0.021820068359375	\\
0.327962001154171	0.056365966796875	\\
0.328006392329205	0.086822509765625	\\
0.328050783504239	0.075714111328125	\\
0.328095174679274	0.0648193359375	\\
0.328139565854308	0.04595947265625	\\
0.328183957029343	0.033721923828125	\\
0.328228348204377	0.02374267578125	\\
0.328272739379411	0.00360107421875	\\
0.328317130554446	-0.000762939453125	\\
0.32836152172948	0.01934814453125	\\
0.328405912904515	0.0440673828125	\\
0.328450304079549	0.03070068359375	\\
0.328494695254583	0.023529052734375	\\
0.328539086429618	0.026123046875	\\
0.328583477604652	0.046295166015625	\\
0.328627868779687	0.05816650390625	\\
0.328672259954721	0.027801513671875	\\
0.328716651129755	-0.01287841796875	\\
0.32876104230479	-0.02294921875	\\
0.328805433479824	-0.01654052734375	\\
0.328849824654859	-0.034210205078125	\\
0.328894215829893	-0.03924560546875	\\
0.328938607004927	-0.024200439453125	\\
0.328982998179962	-0.014404296875	\\
0.329027389354996	-0.01824951171875	\\
0.329071780530031	-0.021026611328125	\\
0.329116171705065	-0.02862548828125	\\
0.329160562880099	-0.0379638671875	\\
0.329204954055134	-0.0352783203125	\\
0.329249345230168	-0.034332275390625	\\
0.329293736405203	-0.0318603515625	\\
0.329338127580237	-0.03326416015625	\\
0.329382518755271	-0.01458740234375	\\
0.329426909930306	0.00506591796875	\\
0.32947130110534	-0.0133056640625	\\
0.329515692280375	-0.02581787109375	\\
0.329560083455409	-0.023529052734375	\\
0.329604474630443	-0.024200439453125	\\
0.329648865805478	-0.0250244140625	\\
0.329693256980512	-0.01519775390625	\\
0.329737648155547	-0.0030517578125	\\
0.329782039330581	-0.0037841796875	\\
0.329826430505616	0.008697509765625	\\
0.32987082168065	0.022857666015625	\\
0.329915212855684	0.012603759765625	\\
0.329959604030719	0.00897216796875	\\
0.330003995205753	0.016632080078125	\\
0.330048386380787	0.0260009765625	\\
0.330092777555822	0.036651611328125	\\
0.330137168730856	0.03717041015625	\\
0.330181559905891	0.029449462890625	\\
0.330225951080925	0.015045166015625	\\
0.33027034225596	0.0074462890625	\\
0.330314733430994	0.014678955078125	\\
0.330359124606028	0.022308349609375	\\
0.330403515781063	0.024078369140625	\\
0.330447906956097	0.024200439453125	\\
0.330492298131132	0.0201416015625	\\
0.330536689306166	0.012786865234375	\\
0.3305810804812	0.00738525390625	\\
0.330625471656235	0.005035400390625	\\
0.330669862831269	0.011749267578125	\\
0.330714254006304	0.015106201171875	\\
0.330758645181338	0.009033203125	\\
0.330803036356372	0.008453369140625	\\
0.330847427531407	0.002197265625	\\
0.330891818706441	-0.01116943359375	\\
0.330936209881476	-0.023956298828125	\\
0.33098060105651	-0.03363037109375	\\
0.331024992231544	-0.0267333984375	\\
0.331069383406579	-0.01702880859375	\\
0.331113774581613	-0.009674072265625	\\
0.331158165756648	0.0009765625	\\
0.331202556931682	0.00140380859375	\\
0.331246948106716	-0.012054443359375	\\
0.331291339281751	-0.02166748046875	\\
0.331335730456785	-0.026824951171875	\\
0.33138012163182	-0.030059814453125	\\
0.331424512806854	-0.025177001953125	\\
0.331468903981888	-0.01483154296875	\\
0.331513295156923	-3.0517578125e-05	\\
0.331557686331957	0.0087890625	\\
0.331602077506992	0.00604248046875	\\
0.331646468682026	-0.0057373046875	\\
0.33169085985706	-0.018035888671875	\\
0.331735251032095	-0.02178955078125	\\
0.331779642207129	-0.02227783203125	\\
0.331824033382164	-0.0146484375	\\
0.331868424557198	-0.001983642578125	\\
0.331912815732232	0.0087890625	\\
0.331957206907267	0.02020263671875	\\
0.332001598082301	0.02099609375	\\
0.332045989257336	0.01416015625	\\
0.33209038043237	0.001251220703125	\\
0.332134771607404	-0.01123046875	\\
0.332179162782439	-0.01336669921875	\\
0.332223553957473	-0.01190185546875	\\
0.332267945132508	-0.00653076171875	\\
0.332312336307542	-0.005828857421875	\\
0.332356727482576	-0.003204345703125	\\
0.332401118657611	0.0079345703125	\\
0.332445509832645	0.013336181640625	\\
0.33248990100768	0.014251708984375	\\
0.332534292182714	0.011566162109375	\\
0.332578683357748	0.0107421875	\\
0.332623074532783	0.00738525390625	\\
0.332667465707817	0.00128173828125	\\
0.332711856882852	-0.000579833984375	\\
0.332756248057886	-0.00390625	\\
0.33280063923292	-0.004852294921875	\\
0.332845030407955	-0.006317138671875	\\
0.332889421582989	-0.002960205078125	\\
0.332933812758024	0.001953125	\\
0.332978203933058	0.005859375	\\
0.333022595108093	0.00946044921875	\\
0.333066986283127	0.003662109375	\\
0.333111377458161	-0.001434326171875	\\
0.333155768633196	-0.004150390625	\\
0.33320015980823	-0.004913330078125	\\
0.333244550983265	-0.001800537109375	\\
0.333288942158299	-0.0001220703125	\\
0.333333333333333	-0.001007080078125	\\
0.333377724508368	-0.001678466796875	\\
0.333422115683402	-0.00341796875	\\
0.333466506858437	-0.0076904296875	\\
0.333510898033471	-0.008941650390625	\\
0.333555289208505	-0.010711669921875	\\
0.33359968038354	-0.014190673828125	\\
0.333644071558574	-0.01251220703125	\\
0.333688462733609	-0.006805419921875	\\
0.333732853908643	-0.0003662109375	\\
0.333777245083677	0.002716064453125	\\
0.333821636258712	0.00213623046875	\\
0.333866027433746	-0.0029296875	\\
0.333910418608781	-0.005126953125	\\
0.333954809783815	-0.003143310546875	\\
0.333999200958849	-0.00604248046875	\\
0.334043592133884	-0.006256103515625	\\
0.334087983308918	-0.00335693359375	\\
0.334132374483953	-0.0015869140625	\\
0.334176765658987	0.00146484375	\\
0.334221156834021	0.00469970703125	\\
0.334265548009056	0.007049560546875	\\
0.33430993918409	0.009521484375	\\
0.334354330359125	0.011474609375	\\
0.334398721534159	0.009979248046875	\\
0.334443112709193	0.00958251953125	\\
0.334487503884228	0.00836181640625	\\
0.334531895059262	0.004638671875	\\
0.334576286234297	0.00390625	\\
0.334620677409331	0.002960205078125	\\
0.334665068584365	0.003021240234375	\\
0.3347094597594	0.00518798828125	\\
0.334753850934434	0.002777099609375	\\
0.334798242109469	0.000762939453125	\\
0.334842633284503	0.002471923828125	\\
0.334887024459537	0.00030517578125	\\
0.334931415634572	-0.002166748046875	\\
0.334975806809606	-0.001373291015625	\\
0.335020197984641	-0.003021240234375	\\
0.335064589159675	-0.001739501953125	\\
0.335108980334709	-9.1552734375e-05	\\
0.335153371509744	-0.00750732421875	\\
0.335197762684778	-0.003997802734375	\\
0.335242153859813	-0.003173828125	\\
0.335286545034847	-0.008697509765625	\\
0.335330936209881	-0.006622314453125	\\
0.335375327384916	-0.003753662109375	\\
0.33541971855995	-0.00408935546875	\\
0.335464109734985	-0.006927490234375	\\
0.335508500910019	-0.003814697265625	\\
0.335552892085054	-0.006805419921875	\\
0.335597283260088	-0.008544921875	\\
0.335641674435122	-0.009735107421875	\\
0.335686065610157	-0.010498046875	\\
0.335730456785191	-0.008087158203125	\\
0.335774847960226	-0.00506591796875	\\
0.33581923913526	-0.003143310546875	\\
0.335863630310294	-0.00457763671875	\\
0.335908021485329	-0.004119873046875	\\
0.335952412660363	-0.00482177734375	\\
0.335996803835398	-0.004241943359375	\\
0.336041195010432	-0.000457763671875	\\
0.336085586185466	0.001678466796875	\\
0.336129977360501	0.00225830078125	\\
0.336174368535535	0.00408935546875	\\
0.33621875971057	0.00445556640625	\\
0.336263150885604	0.00567626953125	\\
0.336307542060638	0.005645751953125	\\
0.336351933235673	0.00482177734375	\\
0.336396324410707	0.004150390625	\\
0.336440715585742	0.002960205078125	\\
0.336485106760776	0.0054931640625	\\
0.33652949793581	0.00860595703125	\\
0.336573889110845	0.00714111328125	\\
0.336618280285879	0.006591796875	\\
0.336662671460914	0.00518798828125	\\
0.336707062635948	0.00238037109375	\\
0.336751453810982	0.00152587890625	\\
0.336795844986017	0.001312255859375	\\
0.336840236161051	0.00244140625	\\
0.336884627336086	0.003448486328125	\\
0.33692901851112	0.002166748046875	\\
0.336973409686154	0.001190185546875	\\
0.337017800861189	0.00140380859375	\\
0.337062192036223	-0.001373291015625	\\
0.337106583211258	-0.001373291015625	\\
0.337150974386292	6.103515625e-05	\\
0.337195365561326	0.0009765625	\\
0.337239756736361	0.001312255859375	\\
0.337284147911395	0.00091552734375	\\
0.33732853908643	0.000579833984375	\\
0.337372930261464	-0.003265380859375	\\
0.337417321436498	-0.00701904296875	\\
0.337461712611533	-0.00946044921875	\\
0.337506103786567	-0.010711669921875	\\
0.337550494961602	-0.00927734375	\\
0.337594886136636	-0.007049560546875	\\
0.33763927731167	-0.00506591796875	\\
0.337683668486705	-0.00299072265625	\\
0.337728059661739	-0.002166748046875	\\
0.337772450836774	-0.002532958984375	\\
0.337816842011808	-0.002685546875	\\
0.337861233186842	-0.00274658203125	\\
0.337905624361877	-0.003265380859375	\\
0.337950015536911	-0.00225830078125	\\
0.337994406711946	-0.0018310546875	\\
0.33803879788698	-0.00079345703125	\\
0.338083189062014	-0.0001220703125	\\
0.338127580237049	0.000579833984375	\\
0.338171971412083	0.001251220703125	\\
0.338216362587118	0.0001220703125	\\
0.338260753762152	0.001220703125	\\
0.338305144937187	0.002197265625	\\
0.338349536112221	0.0047607421875	\\
0.338393927287255	0.00677490234375	\\
0.33843831846229	0.0079345703125	\\
0.338482709637324	0.00872802734375	\\
0.338527100812358	0.008636474609375	\\
0.338571491987393	0.0087890625	\\
0.338615883162427	0.007537841796875	\\
0.338660274337462	0.006072998046875	\\
0.338704665512496	0.0047607421875	\\
0.338749056687531	0.00244140625	\\
0.338793447862565	0.0008544921875	\\
0.338837839037599	0.00115966796875	\\
0.338882230212634	0.000213623046875	\\
0.338926621387668	3.0517578125e-05	\\
0.338971012562703	0.000579833984375	\\
0.339015403737737	0.00079345703125	\\
0.339059794912771	0.0013427734375	\\
0.339104186087806	0.001068115234375	\\
0.33914857726284	0.000823974609375	\\
0.339192968437875	0.000244140625	\\
0.339237359612909	-0.00079345703125	\\
0.339281750787943	-0.002349853515625	\\
0.339326141962978	-0.00390625	\\
0.339370533138012	-0.005462646484375	\\
0.339414924313047	-0.0062255859375	\\
0.339459315488081	-0.00775146484375	\\
0.339503706663115	-0.007843017578125	\\
0.33954809783815	-0.006683349609375	\\
0.339592489013184	-0.00677490234375	\\
0.339636880188219	-0.003662109375	\\
0.339681271363253	-0.001434326171875	\\
0.339725662538287	-0.00164794921875	\\
0.339770053713322	-0.00262451171875	\\
0.339814444888356	-0.002593994140625	\\
0.339858836063391	-0.002166748046875	\\
0.339903227238425	-0.00201416015625	\\
0.339947618413459	-0.001739501953125	\\
0.339992009588494	-0.001129150390625	\\
0.340036400763528	-3.0517578125e-05	\\
0.340080791938563	0	\\
0.340125183113597	0.0015869140625	\\
0.340169574288631	0.002044677734375	\\
0.340213965463666	0.001129150390625	\\
0.3402583566387	0.001617431640625	\\
0.340302747813735	0.00140380859375	\\
0.340347138988769	0.001800537109375	\\
0.340391530163803	0.003021240234375	\\
0.340435921338838	0.003814697265625	\\
0.340480312513872	0.0035400390625	\\
0.340524703688907	0.00213623046875	\\
0.340569094863941	0.001953125	\\
0.340613486038975	0.00030517578125	\\
0.34065787721401	-0.000762939453125	\\
0.340702268389044	3.0517578125e-05	\\
0.340746659564079	0.00079345703125	\\
0.340791050739113	0	\\
0.340835441914147	0.000579833984375	\\
0.340879833089182	0.001617431640625	\\
0.340924224264216	0.000640869140625	\\
0.340968615439251	0.000335693359375	\\
0.341013006614285	-0.0013427734375	\\
0.341057397789319	-0.003021240234375	\\
0.341101788964354	-0.003631591796875	\\
0.341146180139388	-0.003570556640625	\\
0.341190571314423	-0.00372314453125	\\
0.341234962489457	-0.004547119140625	\\
0.341279353664492	-0.003509521484375	\\
0.341323744839526	-0.00341796875	\\
0.34136813601456	-0.003173828125	\\
0.341412527189595	-0.003082275390625	\\
0.341456918364629	-0.0037841796875	\\
0.341501309539664	-0.0032958984375	\\
0.341545700714698	-0.00421142578125	\\
0.341590091889732	-0.003814697265625	\\
0.341634483064767	-0.00323486328125	\\
0.341678874239801	-0.0032958984375	\\
0.341723265414836	-0.002655029296875	\\
0.34176765658987	-0.003326416015625	\\
0.341812047764904	-0.00335693359375	\\
0.341856438939939	-0.003326416015625	\\
0.341900830114973	-0.003570556640625	\\
0.341945221290008	-0.00384521484375	\\
0.341989612465042	-0.004302978515625	\\
0.342034003640076	-0.004302978515625	\\
0.342078394815111	-0.0037841796875	\\
0.342122785990145	-0.00201416015625	\\
0.34216717716518	-0.00030517578125	\\
0.342211568340214	-9.1552734375e-05	\\
0.342255959515248	0.00018310546875	\\
0.342300350690283	0.00030517578125	\\
0.342344741865317	0	\\
0.342389133040352	-0.000640869140625	\\
0.342433524215386	-0.00103759765625	\\
0.34247791539042	-0.00146484375	\\
0.342522306565455	-0.00225830078125	\\
0.342566697740489	-0.00079345703125	\\
0.342611088915524	-0.000762939453125	\\
0.342655480090558	-0.00054931640625	\\
0.342699871265592	6.103515625e-05	\\
0.342744262440627	-0.0006103515625	\\
0.342788653615661	-0.00091552734375	\\
0.342833044790696	-0.0013427734375	\\
0.34287743596573	-0.000335693359375	\\
0.342921827140764	0	\\
0.342966218315799	6.103515625e-05	\\
0.343010609490833	0.000213623046875	\\
0.343055000665868	-0.000518798828125	\\
0.343099391840902	-0.001220703125	\\
0.343143783015936	-0.001373291015625	\\
0.343188174190971	-0.001678466796875	\\
0.343232565366005	-0.00286865234375	\\
0.34327695654104	-0.00323486328125	\\
0.343321347716074	-0.003265380859375	\\
0.343365738891108	-0.0023193359375	\\
0.343410130066143	-0.002532958984375	\\
0.343454521241177	-0.002410888671875	\\
0.343498912416212	-0.000701904296875	\\
0.343543303591246	-0.0013427734375	\\
0.34358769476628	-0.00164794921875	\\
0.343632085941315	-0.001678466796875	\\
0.343676477116349	-0.00152587890625	\\
0.343720868291384	-0.001220703125	\\
0.343765259466418	-0.001434326171875	\\
0.343809650641452	-0.001800537109375	\\
0.343854041816487	-0.0010986328125	\\
0.343898432991521	-0.000579833984375	\\
0.343942824166556	-0.000732421875	\\
0.34398721534159	-0.000396728515625	\\
0.344031606516625	0.0006103515625	\\
0.344075997691659	0.000579833984375	\\
0.344120388866693	-0.000213623046875	\\
0.344164780041728	-0.0001220703125	\\
0.344209171216762	9.1552734375e-05	\\
0.344253562391797	-0.000152587890625	\\
0.344297953566831	-0.00067138671875	\\
0.344342344741865	-9.1552734375e-05	\\
0.3443867359169	-0.000518798828125	\\
0.344431127091934	3.0517578125e-05	\\
0.344475518266969	0.000701904296875	\\
0.344519909442003	-9.1552734375e-05	\\
0.344564300617037	-0.0010986328125	\\
0.344608691792072	-0.000762939453125	\\
0.344653082967106	0.0001220703125	\\
0.344697474142141	0.000335693359375	\\
0.344741865317175	0.00048828125	\\
0.344786256492209	0.00054931640625	\\
0.344830647667244	-0.000152587890625	\\
0.344875038842278	-0.001129150390625	\\
0.344919430017313	-0.0003662109375	\\
0.344963821192347	-0.0009765625	\\
0.345008212367381	-0.001983642578125	\\
0.345052603542416	-0.00177001953125	\\
0.34509699471745	-0.00262451171875	\\
0.345141385892485	-0.002777099609375	\\
0.345185777067519	-0.0018310546875	\\
0.345230168242553	-0.002105712890625	\\
0.345274559417588	-0.00189208984375	\\
0.345318950592622	-0.0009765625	\\
0.345363341767657	-0.001739501953125	\\
0.345407732942691	-0.002105712890625	\\
0.345452124117725	-0.001678466796875	\\
0.34549651529276	-0.00164794921875	\\
0.345540906467794	-0.001953125	\\
0.345585297642829	-0.002410888671875	\\
0.345629688817863	-0.003143310546875	\\
0.345674079992897	-0.004730224609375	\\
0.345718471167932	-0.0048828125	\\
0.345762862342966	-0.00482177734375	\\
0.345807253518001	-0.00445556640625	\\
0.345851644693035	-0.003753662109375	\\
0.345896035868069	-0.003662109375	\\
0.345940427043104	-0.00360107421875	\\
0.345984818218138	-0.00323486328125	\\
0.346029209393173	-0.002349853515625	\\
0.346073600568207	-0.0018310546875	\\
0.346117991743241	-0.00286865234375	\\
0.346162382918276	-0.00341796875	\\
0.34620677409331	-0.00433349609375	\\
0.346251165268345	-0.004852294921875	\\
0.346295556443379	-0.00445556640625	\\
0.346339947618413	-0.004425048828125	\\
0.346384338793448	-0.003936767578125	\\
0.346428729968482	-0.003631591796875	\\
0.346473121143517	-0.003509521484375	\\
0.346517512318551	-0.004791259765625	\\
0.346561903493585	-0.004241943359375	\\
0.34660629466862	-0.00286865234375	\\
0.346650685843654	-0.00299072265625	\\
0.346695077018689	-0.00201416015625	\\
0.346739468193723	-0.00189208984375	\\
0.346783859368758	-0.00164794921875	\\
0.346828250543792	-0.00146484375	\\
0.346872641718826	-0.001434326171875	\\
0.346917032893861	-0.00225830078125	\\
0.346961424068895	-0.00286865234375	\\
0.347005815243929	-0.00225830078125	\\
0.347050206418964	-0.00311279296875	\\
0.347094597593998	-0.002593994140625	\\
0.347138988769033	-0.001495361328125	\\
0.347183379944067	-0.00164794921875	\\
0.347227771119102	-0.001220703125	\\
0.347272162294136	-0.00091552734375	\\
0.34731655346917	-0.000396728515625	\\
0.347360944644205	0.000213623046875	\\
0.347405335819239	-0.000518798828125	\\
0.347449726994274	-0.001312255859375	\\
0.347494118169308	-0.001708984375	\\
0.347538509344342	-0.002655029296875	\\
0.347582900519377	-0.002410888671875	\\
0.347627291694411	-0.002288818359375	\\
0.347671682869446	-0.002838134765625	\\
0.34771607404448	-0.003143310546875	\\
0.347760465219514	-0.002899169921875	\\
0.347804856394549	-0.002288818359375	\\
0.347849247569583	-0.001800537109375	\\
0.347893638744618	-0.00128173828125	\\
0.347938029919652	-0.0018310546875	\\
0.347982421094686	-0.001678466796875	\\
0.348026812269721	-0.0020751953125	\\
0.348071203444755	-0.002532958984375	\\
0.34811559461979	-0.00250244140625	\\
0.348159985794824	-0.002960205078125	\\
0.348204376969858	-0.0029296875	\\
0.348248768144893	-0.002593994140625	\\
0.348293159319927	-0.002655029296875	\\
0.348337550494962	-0.003387451171875	\\
0.348381941669996	-0.003875732421875	\\
0.34842633284503	-0.003692626953125	\\
0.348470724020065	-0.003265380859375	\\
0.348515115195099	-0.002227783203125	\\
0.348559506370134	-0.002471923828125	\\
0.348603897545168	-0.00238037109375	\\
0.348648288720202	-0.00152587890625	\\
0.348692679895237	-0.00140380859375	\\
0.348737071070271	-0.001068115234375	\\
0.348781462245306	-0.00152587890625	\\
0.34882585342034	-0.0018310546875	\\
0.348870244595374	-0.0025634765625	\\
0.348914635770409	-0.002349853515625	\\
0.348959026945443	-0.00146484375	\\
0.349003418120478	-0.001800537109375	\\
0.349047809295512	-0.000946044921875	\\
0.349092200470546	-0.000640869140625	\\
0.349136591645581	-0.00042724609375	\\
0.349180982820615	-0.000274658203125	\\
0.34922537399565	-0.0008544921875	\\
0.349269765170684	-0.001495361328125	\\
0.349314156345719	-0.001495361328125	\\
0.349358547520753	-0.0010986328125	\\
0.349402938695787	-0.001983642578125	\\
0.349447329870822	-0.002166748046875	\\
0.349491721045856	-0.001708984375	\\
0.34953611222089	-0.001983642578125	\\
0.349580503395925	-0.00244140625	\\
0.349624894570959	-0.003173828125	\\
0.349669285745994	-0.003143310546875	\\
0.349713676921028	-0.0029296875	\\
0.349758068096063	-0.00238037109375	\\
0.349802459271097	-0.00250244140625	\\
0.349846850446131	-0.002685546875	\\
0.349891241621166	-0.00201416015625	\\
0.3499356327962	-0.0020751953125	\\
0.349980023971235	-0.00238037109375	\\
0.350024415146269	-0.003082275390625	\\
0.350068806321303	-0.003814697265625	\\
0.350113197496338	-0.003997802734375	\\
0.350157588671372	-0.004486083984375	\\
0.350201979846407	-0.004791259765625	\\
0.350246371021441	-0.004425048828125	\\
0.350290762196475	-0.00408935546875	\\
0.35033515337151	-0.003875732421875	\\
0.350379544546544	-0.003814697265625	\\
0.350423935721579	-0.0037841796875	\\
0.350468326896613	-0.003173828125	\\
0.350512718071647	-0.003082275390625	\\
0.350557109246682	-0.002716064453125	\\
0.350601500421716	-0.002227783203125	\\
0.350645891596751	-0.002410888671875	\\
0.350690282771785	-0.002288818359375	\\
0.350734673946819	-0.00244140625	\\
0.350779065121854	-0.002777099609375	\\
0.350823456296888	-0.00274658203125	\\
0.350867847471923	-0.00262451171875	\\
0.350912238646957	-0.00286865234375	\\
0.350956629821991	-0.002838134765625	\\
0.351001020997026	-0.00201416015625	\\
0.35104541217206	-0.001983642578125	\\
0.351089803347095	-0.001617431640625	\\
0.351134194522129	-0.001007080078125	\\
0.351178585697163	-0.00091552734375	\\
0.351222976872198	-0.0010986328125	\\
0.351267368047232	-0.001983642578125	\\
0.351311759222267	-0.001983642578125	\\
0.351356150397301	-0.00225830078125	\\
0.351400541572335	-0.00299072265625	\\
0.35144493274737	-0.00311279296875	\\
0.351489323922404	-0.003570556640625	\\
0.351533715097439	-0.003631591796875	\\
0.351578106272473	-0.00323486328125	\\
0.351622497447507	-0.002593994140625	\\
0.351666888622542	-0.002197265625	\\
0.351711279797576	-0.002044677734375	\\
0.351755670972611	-0.001922607421875	\\
0.351800062147645	-0.001922607421875	\\
0.351844453322679	-0.002410888671875	\\
0.351888844497714	-0.0028076171875	\\
0.351933235672748	-0.002716064453125	\\
0.351977626847783	-0.002899169921875	\\
0.352022018022817	-0.002685546875	\\
0.352066409197851	-0.00225830078125	\\
0.352110800372886	-0.002288818359375	\\
0.35215519154792	-0.00262451171875	\\
0.352199582722955	-0.002532958984375	\\
0.352243973897989	-0.002777099609375	\\
0.352288365073023	-0.00286865234375	\\
0.352332756248058	-0.002166748046875	\\
0.352377147423092	-0.002166748046875	\\
0.352421538598127	-0.001708984375	\\
0.352465929773161	-0.001861572265625	\\
0.352510320948196	-0.002044677734375	\\
0.35255471212323	-0.002777099609375	\\
0.352599103298264	-0.00286865234375	\\
0.352643494473299	-0.00244140625	\\
0.352687885648333	-0.00286865234375	\\
0.352732276823368	-0.002777099609375	\\
0.352776667998402	-0.002838134765625	\\
0.352821059173436	-0.002777099609375	\\
0.352865450348471	-0.002777099609375	\\
0.352909841523505	-0.00244140625	\\
0.35295423269854	-0.00177001953125	\\
0.352998623873574	-0.00115966796875	\\
0.353043015048608	-0.00164794921875	\\
0.353087406223643	-0.001251220703125	\\
0.353131797398677	-0.001007080078125	\\
0.353176188573712	-0.00103759765625	\\
0.353220579748746	-0.000518798828125	\\
0.35326497092378	-0.000518798828125	\\
0.353309362098815	-0.00103759765625	\\
0.353353753273849	-0.001220703125	\\
0.353398144448884	-0.001190185546875	\\
0.353442535623918	-0.00115966796875	\\
0.353486926798952	-0.00067138671875	\\
0.353531317973987	-9.1552734375e-05	\\
0.353575709149021	0.00048828125	\\
0.353620100324056	0.000579833984375	\\
0.35366449149909	0.001220703125	\\
0.353708882674124	0.001983642578125	\\
0.353753273849159	0.001678466796875	\\
0.353797665024193	0.000579833984375	\\
0.353842056199228	0.00067138671875	\\
0.353886447374262	6.103515625e-05	\\
0.353930838549296	-0.000335693359375	\\
0.353975229724331	0	\\
0.354019620899365	-0.00018310546875	\\
0.3540640120744	-0.0001220703125	\\
0.354108403249434	-9.1552734375e-05	\\
0.354152794424468	-9.1552734375e-05	\\
0.354197185599503	6.103515625e-05	\\
0.354241576774537	0.0001220703125	\\
0.354285967949572	0.000152587890625	\\
0.354330359124606	0.000335693359375	\\
0.35437475029964	0.000396728515625	\\
0.354419141474675	-6.103515625e-05	\\
0.354463532649709	-0.000579833984375	\\
0.354507923824744	-0.000762939453125	\\
0.354552314999778	-0.000701904296875	\\
0.354596706174812	-0.001068115234375	\\
0.354641097349847	-0.0009765625	\\
0.354685488524881	-0.00103759765625	\\
0.354729879699916	-0.001190185546875	\\
0.35477427087495	-0.00091552734375	\\
0.354818662049984	-0.001129150390625	\\
0.354863053225019	-0.000732421875	\\
0.354907444400053	-0.000823974609375	\\
0.354951835575088	-0.00103759765625	\\
0.354996226750122	-0.001007080078125	\\
0.355040617925156	-0.001007080078125	\\
0.355085009100191	-0.000885009765625	\\
0.355129400275225	-0.001129150390625	\\
};
\addplot [color=blue,solid,forget plot]
  table[row sep=crcr]{
0.355129400275225	-0.001129150390625	\\
0.35517379145026	-0.000946044921875	\\
0.355218182625294	-0.000701904296875	\\
0.355262573800329	-0.0008544921875	\\
0.355306964975363	-0.00079345703125	\\
0.355351356150397	-0.001007080078125	\\
0.355395747325432	-0.00115966796875	\\
0.355440138500466	-0.0006103515625	\\
0.355484529675501	-0.000701904296875	\\
0.355528920850535	-0.000946044921875	\\
0.355573312025569	-0.00115966796875	\\
0.355617703200604	-0.0009765625	\\
0.355662094375638	-0.000518798828125	\\
0.355706485550673	-0.0006103515625	\\
0.355750876725707	-0.000396728515625	\\
0.355795267900741	-0.000396728515625	\\
0.355839659075776	-0.0008544921875	\\
0.35588405025081	-0.000823974609375	\\
0.355928441425845	-0.000457763671875	\\
0.355972832600879	-0.000396728515625	\\
0.356017223775913	-0.00018310546875	\\
0.356061614950948	-0.00018310546875	\\
0.356106006125982	-0.000457763671875	\\
0.356150397301017	-0.00018310546875	\\
0.356194788476051	-0.00018310546875	\\
0.356239179651085	-3.0517578125e-05	\\
0.35628357082612	0.0001220703125	\\
0.356327962001154	-0.0006103515625	\\
0.356372353176189	-0.00048828125	\\
0.356416744351223	-0.0003662109375	\\
0.356461135526257	-0.000213623046875	\\
0.356505526701292	-0.000518798828125	\\
0.356549917876326	-0.000274658203125	\\
0.356594309051361	-0.00079345703125	\\
0.356638700226395	-0.001190185546875	\\
0.356683091401429	-0.001129150390625	\\
0.356727482576464	-0.00128173828125	\\
0.356771873751498	-0.00152587890625	\\
0.356816264926533	-0.001373291015625	\\
0.356860656101567	-0.0009765625	\\
0.356905047276601	-0.00042724609375	\\
0.356949438451636	6.103515625e-05	\\
0.35699382962667	6.103515625e-05	\\
0.357038220801705	0.00042724609375	\\
0.357082611976739	-0.0001220703125	\\
0.357127003151773	-0.000518798828125	\\
0.357171394326808	-0.0001220703125	\\
0.357215785501842	-0.000396728515625	\\
0.357260176676877	-0.00054931640625	\\
0.357304567851911	-0.000274658203125	\\
0.357348959026945	-9.1552734375e-05	\\
0.35739335020198	-0.000335693359375	\\
0.357437741377014	0.0001220703125	\\
0.357482132552049	0.00054931640625	\\
0.357526523727083	0.00042724609375	\\
0.357570914902117	0.000396728515625	\\
0.357615306077152	6.103515625e-05	\\
0.357659697252186	0.0003662109375	\\
0.357704088427221	0.000335693359375	\\
0.357748479602255	0.00018310546875	\\
0.35779287077729	0.000701904296875	\\
0.357837261952324	0.000518798828125	\\
0.357881653127358	0.00067138671875	\\
0.357926044302393	0.0009765625	\\
0.357970435477427	0.000640869140625	\\
0.358014826652461	0.00030517578125	\\
0.358059217827496	0.00030517578125	\\
0.35810360900253	0.00054931640625	\\
0.358148000177565	0.000823974609375	\\
0.358192391352599	0.00067138671875	\\
0.358236782527634	0.00030517578125	\\
0.358281173702668	0.00042724609375	\\
0.358325564877702	0.000885009765625	\\
0.358369956052737	0.00103759765625	\\
0.358414347227771	0.001007080078125	\\
0.358458738402806	0.000946044921875	\\
0.35850312957784	0.000457763671875	\\
0.358547520752874	0.000244140625	\\
0.358591911927909	0.000396728515625	\\
0.358636303102943	9.1552734375e-05	\\
0.358680694277978	0.000579833984375	\\
0.358725085453012	0.0003662109375	\\
0.358769476628046	-0.0003662109375	\\
0.358813867803081	-3.0517578125e-05	\\
0.358858258978115	-0.00042724609375	\\
0.35890265015315	-0.00042724609375	\\
0.358947041328184	-0.000518798828125	\\
0.358991432503218	-0.0006103515625	\\
0.359035823678253	-0.000244140625	\\
0.359080214853287	-0.00067138671875	\\
0.359124606028322	-0.0006103515625	\\
0.359168997203356	-0.0006103515625	\\
0.35921338837839	-0.00030517578125	\\
0.359257779553425	-0.000213623046875	\\
0.359302170728459	-0.000701904296875	\\
0.359346561903494	-0.00103759765625	\\
0.359390953078528	-0.001190185546875	\\
0.359435344253562	-0.001220703125	\\
0.359479735428597	-0.00140380859375	\\
0.359524126603631	-0.000579833984375	\\
0.359568517778666	-0.00091552734375	\\
0.3596129089537	-0.00140380859375	\\
0.359657300128734	-0.000396728515625	\\
0.359701691303769	-0.0013427734375	\\
0.359746082478803	-0.001495361328125	\\
0.359790473653838	-0.0010986328125	\\
0.359834864828872	-0.001190185546875	\\
0.359879256003906	-0.001922607421875	\\
0.359923647178941	-0.00177001953125	\\
0.359968038353975	-0.0010986328125	\\
0.36001242952901	-0.00177001953125	\\
0.360056820704044	-0.000946044921875	\\
0.360101211879078	-0.001251220703125	\\
0.360145603054113	-0.001495361328125	\\
0.360189994229147	-0.001678466796875	\\
0.360234385404182	-0.002777099609375	\\
0.360278776579216	-0.002685546875	\\
0.36032316775425	-0.002655029296875	\\
0.360367558929285	-0.00201416015625	\\
0.360411950104319	-0.001617431640625	\\
0.360456341279354	-0.001861572265625	\\
0.360500732454388	-0.0018310546875	\\
0.360545123629422	-0.0018310546875	\\
0.360589514804457	-0.00189208984375	\\
0.360633905979491	-0.00201416015625	\\
0.360678297154526	-0.00225830078125	\\
0.36072268832956	-0.001861572265625	\\
0.360767079504594	-0.002288818359375	\\
0.360811470679629	-0.002655029296875	\\
0.360855861854663	-0.001983642578125	\\
0.360900253029698	-0.001708984375	\\
0.360944644204732	-0.001220703125	\\
0.360989035379767	-0.002044677734375	\\
0.361033426554801	-0.00225830078125	\\
0.361077817729835	-0.001861572265625	\\
0.36112220890487	-0.0023193359375	\\
0.361166600079904	-0.002105712890625	\\
0.361210991254939	-0.001922607421875	\\
0.361255382429973	-0.001800537109375	\\
0.361299773605007	-0.001739501953125	\\
0.361344164780042	-0.00140380859375	\\
0.361388555955076	-0.00115966796875	\\
0.361432947130111	-0.00091552734375	\\
0.361477338305145	-0.001007080078125	\\
0.361521729480179	-0.001220703125	\\
0.361566120655214	-0.0009765625	\\
0.361610511830248	-0.0003662109375	\\
0.361654903005283	0	\\
0.361699294180317	-0.00054931640625	\\
0.361743685355351	-0.000823974609375	\\
0.361788076530386	-0.001007080078125	\\
0.36183246770542	-0.001190185546875	\\
0.361876858880455	-0.00091552734375	\\
0.361921250055489	-0.000885009765625	\\
0.361965641230523	-0.001251220703125	\\
0.362010032405558	-0.00128173828125	\\
0.362054423580592	-0.00103759765625	\\
0.362098814755627	-0.001068115234375	\\
0.362143205930661	-0.000946044921875	\\
0.362187597105695	-0.00128173828125	\\
0.36223198828073	-0.00164794921875	\\
0.362276379455764	-0.001556396484375	\\
0.362320770630799	-0.001739501953125	\\
0.362365161805833	-0.00177001953125	\\
0.362409552980867	-0.0020751953125	\\
0.362453944155902	-0.00238037109375	\\
0.362498335330936	-0.002532958984375	\\
0.362542726505971	-0.002471923828125	\\
0.362587117681005	-0.002532958984375	\\
0.362631508856039	-0.0028076171875	\\
0.362675900031074	-0.00262451171875	\\
0.362720291206108	-0.0028076171875	\\
0.362764682381143	-0.00274658203125	\\
0.362809073556177	-0.002716064453125	\\
0.362853464731211	-0.00250244140625	\\
0.362897855906246	-0.002349853515625	\\
0.36294224708128	-0.002655029296875	\\
0.362986638256315	-0.002685546875	\\
0.363031029431349	-0.00262451171875	\\
0.363075420606383	-0.0029296875	\\
0.363119811781418	-0.003143310546875	\\
0.363164202956452	-0.0029296875	\\
0.363208594131487	-0.002899169921875	\\
0.363252985306521	-0.00189208984375	\\
0.363297376481555	-0.00140380859375	\\
0.36334176765659	-0.00146484375	\\
0.363386158831624	-0.00128173828125	\\
0.363430550006659	-0.00146484375	\\
0.363474941181693	-0.0015869140625	\\
0.363519332356728	-0.0015869140625	\\
0.363563723531762	-0.0013427734375	\\
0.363608114706796	-0.00152587890625	\\
0.363652505881831	-0.0018310546875	\\
0.363696897056865	-0.00201416015625	\\
0.3637412882319	-0.001922607421875	\\
0.363785679406934	-0.001953125	\\
0.363830070581968	-0.00189208984375	\\
0.363874461757003	-0.00164794921875	\\
0.363918852932037	-0.001800537109375	\\
0.363963244107072	-0.001708984375	\\
0.364007635282106	-0.0015869140625	\\
0.36405202645714	-0.001922607421875	\\
0.364096417632175	-0.001800537109375	\\
0.364140808807209	-0.002227783203125	\\
0.364185199982244	-0.002532958984375	\\
0.364229591157278	-0.002166748046875	\\
0.364273982332312	-0.0023193359375	\\
0.364318373507347	-0.002685546875	\\
0.364362764682381	-0.002593994140625	\\
0.364407155857416	-0.0025634765625	\\
0.36445154703245	-0.002410888671875	\\
0.364495938207484	-0.0028076171875	\\
0.364540329382519	-0.002899169921875	\\
0.364584720557553	-0.00311279296875	\\
0.364629111732588	-0.003326416015625	\\
0.364673502907622	-0.00323486328125	\\
0.364717894082656	-0.00335693359375	\\
0.364762285257691	-0.003021240234375	\\
0.364806676432725	-0.003082275390625	\\
0.36485106760776	-0.002960205078125	\\
0.364895458782794	-0.0028076171875	\\
0.364939849957828	-0.002685546875	\\
0.364984241132863	-0.002471923828125	\\
0.365028632307897	-0.00274658203125	\\
0.365073023482932	-0.00274658203125	\\
0.365117414657966	-0.002685546875	\\
0.365161805833	-0.002685546875	\\
0.365206197008035	-0.0025634765625	\\
0.365250588183069	-0.002716064453125	\\
0.365294979358104	-0.003021240234375	\\
0.365339370533138	-0.002838134765625	\\
0.365383761708172	-0.0020751953125	\\
0.365428152883207	-0.002166748046875	\\
0.365472544058241	-0.002593994140625	\\
0.365516935233276	-0.002685546875	\\
0.36556132640831	-0.002471923828125	\\
0.365605717583344	-0.0030517578125	\\
0.365650108758379	-0.003662109375	\\
0.365694499933413	-0.003631591796875	\\
0.365738891108448	-0.003814697265625	\\
0.365783282283482	-0.004150390625	\\
0.365827673458516	-0.004150390625	\\
0.365872064633551	-0.003936767578125	\\
0.365916455808585	-0.003875732421875	\\
0.36596084698362	-0.003814697265625	\\
0.366005238158654	-0.00286865234375	\\
0.366049629333688	-0.0028076171875	\\
0.366094020508723	-0.003631591796875	\\
0.366138411683757	-0.00408935546875	\\
0.366182802858792	-0.004302978515625	\\
0.366227194033826	-0.004638671875	\\
0.366271585208861	-0.004486083984375	\\
0.366315976383895	-0.004241943359375	\\
0.366360367558929	-0.005096435546875	\\
0.366404758733964	-0.005096435546875	\\
0.366449149908998	-0.005340576171875	\\
0.366493541084032	-0.00555419921875	\\
0.366537932259067	-0.004791259765625	\\
0.366582323434101	-0.004913330078125	\\
0.366626714609136	-0.00531005859375	\\
0.36667110578417	-0.005096435546875	\\
0.366715496959205	-0.005126953125	\\
0.366759888134239	-0.005462646484375	\\
0.366804279309273	-0.0052490234375	\\
0.366848670484308	-0.00482177734375	\\
0.366893061659342	-0.005462646484375	\\
0.366937452834377	-0.00555419921875	\\
0.366981844009411	-0.005035400390625	\\
0.367026235184445	-0.005279541015625	\\
0.36707062635948	-0.00531005859375	\\
0.367115017534514	-0.005096435546875	\\
0.367159408709549	-0.00506591796875	\\
0.367203799884583	-0.004669189453125	\\
0.367248191059617	-0.00469970703125	\\
0.367292582234652	-0.0047607421875	\\
0.367336973409686	-0.00469970703125	\\
0.367381364584721	-0.004486083984375	\\
0.367425755759755	-0.004608154296875	\\
0.367470146934789	-0.004913330078125	\\
0.367514538109824	-0.004638671875	\\
0.367558929284858	-0.004241943359375	\\
0.367603320459893	-0.004180908203125	\\
0.367647711634927	-0.003814697265625	\\
0.367692102809961	-0.003448486328125	\\
0.367736493984996	-0.00396728515625	\\
0.36778088516003	-0.0037841796875	\\
0.367825276335065	-0.003692626953125	\\
0.367869667510099	-0.003692626953125	\\
0.367914058685133	-0.003326416015625	\\
0.367958449860168	-0.003448486328125	\\
0.368002841035202	-0.0040283203125	\\
0.368047232210237	-0.00347900390625	\\
0.368091623385271	-0.0037841796875	\\
0.368136014560305	-0.004241943359375	\\
0.36818040573534	-0.00396728515625	\\
0.368224796910374	-0.003936767578125	\\
0.368269188085409	-0.00372314453125	\\
0.368313579260443	-0.003753662109375	\\
0.368357970435477	-0.003753662109375	\\
0.368402361610512	-0.003509521484375	\\
0.368446752785546	-0.00384521484375	\\
0.368491143960581	-0.003814697265625	\\
0.368535535135615	-0.00311279296875	\\
0.368579926310649	-0.003265380859375	\\
0.368624317485684	-0.003265380859375	\\
0.368668708660718	-0.0032958984375	\\
0.368713099835753	-0.00335693359375	\\
0.368757491010787	-0.00311279296875	\\
0.368801882185821	-0.003082275390625	\\
0.368846273360856	-0.003204345703125	\\
0.36889066453589	-0.003265380859375	\\
0.368935055710925	-0.00341796875	\\
0.368979446885959	-0.003570556640625	\\
0.369023838060993	-0.003448486328125	\\
0.369068229236028	-0.003265380859375	\\
0.369112620411062	-0.00372314453125	\\
0.369157011586097	-0.0037841796875	\\
0.369201402761131	-0.003631591796875	\\
0.369245793936165	-0.003448486328125	\\
0.3692901851112	-0.0035400390625	\\
0.369334576286234	-0.003662109375	\\
0.369378967461269	-0.003936767578125	\\
0.369423358636303	-0.003875732421875	\\
0.369467749811338	-0.003875732421875	\\
0.369512140986372	-0.004119873046875	\\
0.369556532161406	-0.0045166015625	\\
0.369600923336441	-0.004669189453125	\\
0.369645314511475	-0.004425048828125	\\
0.36968970568651	-0.00482177734375	\\
0.369734096861544	-0.0048828125	\\
0.369778488036578	-0.00433349609375	\\
0.369822879211613	-0.00439453125	\\
0.369867270386647	-0.004547119140625	\\
0.369911661561682	-0.004638671875	\\
0.369956052736716	-0.00433349609375	\\
0.37000044391175	-0.003997802734375	\\
0.370044835086785	-0.003936767578125	\\
0.370089226261819	-0.004364013671875	\\
0.370133617436854	-0.00494384765625	\\
0.370178008611888	-0.00482177734375	\\
0.370222399786922	-0.004669189453125	\\
0.370266790961957	-0.00457763671875	\\
0.370311182136991	-0.004364013671875	\\
0.370355573312026	-0.004547119140625	\\
0.37039996448706	-0.004852294921875	\\
0.370444355662094	-0.00433349609375	\\
0.370488746837129	-0.0042724609375	\\
0.370533138012163	-0.00439453125	\\
0.370577529187198	-0.004302978515625	\\
0.370621920362232	-0.004638671875	\\
0.370666311537266	-0.004364013671875	\\
0.370710702712301	-0.0037841796875	\\
0.370755093887335	-0.003875732421875	\\
0.37079948506237	-0.003997802734375	\\
0.370843876237404	-0.00433349609375	\\
0.370888267412438	-0.004241943359375	\\
0.370932658587473	-0.00421142578125	\\
0.370977049762507	-0.003875732421875	\\
0.371021440937542	-0.003631591796875	\\
0.371065832112576	-0.003753662109375	\\
0.37111022328761	-0.00390625	\\
0.371154614462645	-0.003936767578125	\\
0.371199005637679	-0.00396728515625	\\
0.371243396812714	-0.003875732421875	\\
0.371287787987748	-0.003875732421875	\\
0.371332179162782	-0.003692626953125	\\
0.371376570337817	-0.003509521484375	\\
0.371420961512851	-0.003570556640625	\\
0.371465352687886	-0.003997802734375	\\
0.37150974386292	-0.004150390625	\\
0.371554135037954	-0.003265380859375	\\
0.371598526212989	-0.0029296875	\\
0.371642917388023	-0.00323486328125	\\
0.371687308563058	-0.003509521484375	\\
0.371731699738092	-0.003143310546875	\\
0.371776090913126	-0.003021240234375	\\
0.371820482088161	-0.003448486328125	\\
0.371864873263195	-0.003662109375	\\
0.37190926443823	-0.003875732421875	\\
0.371953655613264	-0.003692626953125	\\
0.371998046788299	-0.0035400390625	\\
0.372042437963333	-0.00396728515625	\\
0.372086829138367	-0.00372314453125	\\
0.372131220313402	-0.00347900390625	\\
0.372175611488436	-0.00335693359375	\\
0.372220002663471	-0.00335693359375	\\
0.372264393838505	-0.0035400390625	\\
0.372308785013539	-0.0037841796875	\\
0.372353176188574	-0.003662109375	\\
0.372397567363608	-0.003662109375	\\
0.372441958538643	-0.0037841796875	\\
0.372486349713677	-0.003631591796875	\\
0.372530740888711	-0.003631591796875	\\
0.372575132063746	-0.003753662109375	\\
0.37261952323878	-0.00408935546875	\\
0.372663914413815	-0.004058837890625	\\
0.372708305588849	-0.00384521484375	\\
0.372752696763883	-0.003448486328125	\\
0.372797087938918	-0.00311279296875	\\
0.372841479113952	-0.003082275390625	\\
0.372885870288987	-0.003173828125	\\
0.372930261464021	-0.002593994140625	\\
0.372974652639055	-0.002471923828125	\\
0.37301904381409	-0.002593994140625	\\
0.373063434989124	-0.002655029296875	\\
0.373107826164159	-0.002685546875	\\
0.373152217339193	-0.00238037109375	\\
0.373196608514227	-0.00274658203125	\\
0.373240999689262	-0.002777099609375	\\
0.373285390864296	-0.002716064453125	\\
0.373329782039331	-0.002349853515625	\\
0.373374173214365	-0.00152587890625	\\
0.373418564389399	-0.001800537109375	\\
0.373462955564434	-0.0018310546875	\\
0.373507346739468	-0.0015869140625	\\
0.373551737914503	-0.001739501953125	\\
0.373596129089537	-0.00152587890625	\\
0.373640520264571	-0.0020751953125	\\
0.373684911439606	-0.002349853515625	\\
0.37372930261464	-0.002349853515625	\\
0.373773693789675	-0.00250244140625	\\
0.373818084964709	-0.00262451171875	\\
0.373862476139743	-0.00262451171875	\\
0.373906867314778	-0.0025634765625	\\
0.373951258489812	-0.002593994140625	\\
0.373995649664847	-0.002685546875	\\
0.374040040839881	-0.002838134765625	\\
0.374084432014915	-0.0029296875	\\
0.37412882318995	-0.003082275390625	\\
0.374173214364984	-0.00335693359375	\\
0.374217605540019	-0.00323486328125	\\
0.374261996715053	-0.00311279296875	\\
0.374306387890087	-0.003936767578125	\\
0.374350779065122	-0.004241943359375	\\
0.374395170240156	-0.0042724609375	\\
0.374439561415191	-0.004608154296875	\\
0.374483952590225	-0.004486083984375	\\
0.374528343765259	-0.0040283203125	\\
0.374572734940294	-0.00384521484375	\\
0.374617126115328	-0.003662109375	\\
0.374661517290363	-0.003387451171875	\\
0.374705908465397	-0.00341796875	\\
0.374750299640431	-0.003326416015625	\\
0.374794690815466	-0.0035400390625	\\
0.3748390819905	-0.00341796875	\\
0.374883473165535	-0.00347900390625	\\
0.374927864340569	-0.003662109375	\\
0.374972255515603	-0.003753662109375	\\
0.375016646690638	-0.003631591796875	\\
0.375061037865672	-0.003662109375	\\
0.375105429040707	-0.003143310546875	\\
0.375149820215741	-0.0029296875	\\
0.375194211390776	-0.003143310546875	\\
0.37523860256581	-0.003204345703125	\\
0.375282993740844	-0.002655029296875	\\
0.375327384915879	-0.002685546875	\\
0.375371776090913	-0.002655029296875	\\
0.375416167265948	-0.003082275390625	\\
0.375460558440982	-0.00299072265625	\\
0.375504949616016	-0.002685546875	\\
0.375549340791051	-0.002777099609375	\\
0.375593731966085	-0.0028076171875	\\
0.37563812314112	-0.00274658203125	\\
0.375682514316154	-0.003021240234375	\\
0.375726905491188	-0.003265380859375	\\
0.375771296666223	-0.003143310546875	\\
0.375815687841257	-0.002685546875	\\
0.375860079016292	-0.00262451171875	\\
0.375904470191326	-0.002471923828125	\\
0.37594886136636	-0.002532958984375	\\
0.375993252541395	-0.002685546875	\\
0.376037643716429	-0.00262451171875	\\
0.376082034891464	-0.00250244140625	\\
0.376126426066498	-0.0029296875	\\
0.376170817241532	-0.002899169921875	\\
0.376215208416567	-0.0025634765625	\\
0.376259599591601	-0.002532958984375	\\
0.376303990766636	-0.00213623046875	\\
0.37634838194167	-0.002197265625	\\
0.376392773116704	-0.0018310546875	\\
0.376437164291739	-0.001495361328125	\\
0.376481555466773	-0.00177001953125	\\
0.376525946641808	-0.002166748046875	\\
0.376570337816842	-0.0020751953125	\\
0.376614728991876	-0.001556396484375	\\
0.376659120166911	-0.00213623046875	\\
0.376703511341945	-0.002105712890625	\\
0.37674790251698	-0.0018310546875	\\
0.376792293692014	-0.001861572265625	\\
0.376836684867048	-0.00115966796875	\\
0.376881076042083	-0.0010986328125	\\
0.376925467217117	-0.00140380859375	\\
0.376969858392152	-0.00164794921875	\\
0.377014249567186	-0.001922607421875	\\
0.37705864074222	-0.001129150390625	\\
0.377103031917255	-0.0008544921875	\\
0.377147423092289	-0.00091552734375	\\
0.377191814267324	-0.0008544921875	\\
0.377236205442358	-0.00091552734375	\\
0.377280596617392	-0.000732421875	\\
0.377324987792427	-0.00091552734375	\\
0.377369378967461	-0.000885009765625	\\
0.377413770142496	-0.000762939453125	\\
0.37745816131753	-0.000518798828125	\\
0.377502552492564	-0.000518798828125	\\
0.377546943667599	-0.000457763671875	\\
0.377591334842633	-0.000518798828125	\\
0.377635726017668	-0.00079345703125	\\
0.377680117192702	-0.0009765625	\\
0.377724508367736	-0.001068115234375	\\
0.377768899542771	-0.000946044921875	\\
0.377813290717805	-0.000701904296875	\\
0.37785768189284	-0.000396728515625	\\
0.377902073067874	-0.000335693359375	\\
0.377946464242909	-0.000762939453125	\\
0.377990855417943	-0.000518798828125	\\
0.378035246592977	-0.000640869140625	\\
0.378079637768012	-0.00128173828125	\\
0.378124028943046	-0.0009765625	\\
0.378168420118081	-0.000946044921875	\\
0.378212811293115	-0.000946044921875	\\
0.378257202468149	-0.000732421875	\\
0.378301593643184	-0.00048828125	\\
0.378345984818218	-0.00067138671875	\\
0.378390375993253	-0.000518798828125	\\
0.378434767168287	-0.000396728515625	\\
0.378479158343321	-0.000762939453125	\\
0.378523549518356	-0.0009765625	\\
0.37856794069339	-0.00140380859375	\\
0.378612331868425	-0.000946044921875	\\
0.378656723043459	-0.00054931640625	\\
0.378701114218493	-0.00067138671875	\\
0.378745505393528	-0.000274658203125	\\
0.378789896568562	-0.000213623046875	\\
0.378834287743597	-0.0009765625	\\
0.378878678918631	-0.000946044921875	\\
0.378923070093665	-0.000640869140625	\\
0.3789674612687	-0.00079345703125	\\
0.379011852443734	-0.000732421875	\\
0.379056243618769	-0.000457763671875	\\
0.379100634793803	-0.000396728515625	\\
0.379145025968837	0	\\
0.379189417143872	6.103515625e-05	\\
0.379233808318906	-0.0001220703125	\\
0.379278199493941	-0.000518798828125	\\
0.379322590668975	-0.001007080078125	\\
0.379366981844009	-0.0008544921875	\\
0.379411373019044	-0.000213623046875	\\
0.379455764194078	-0.000946044921875	\\
0.379500155369113	-0.0013427734375	\\
0.379544546544147	-0.000396728515625	\\
0.379588937719181	-0.001190185546875	\\
0.379633328894216	-0.000885009765625	\\
0.37967772006925	-0.000823974609375	\\
0.379722111244285	-0.001556396484375	\\
0.379766502419319	-0.00140380859375	\\
0.379810893594353	-0.00140380859375	\\
0.379855284769388	-0.00128173828125	\\
0.379899675944422	-0.001129150390625	\\
0.379944067119457	-0.001220703125	\\
0.379988458294491	-0.0010986328125	\\
0.380032849469525	-0.001007080078125	\\
0.38007724064456	-0.001068115234375	\\
0.380121631819594	-0.00164794921875	\\
0.380166022994629	-0.001861572265625	\\
0.380210414169663	-0.001739501953125	\\
0.380254805344697	-0.001953125	\\
0.380299196519732	-0.001922607421875	\\
0.380343587694766	-0.001922607421875	\\
0.380387978869801	-0.00201416015625	\\
0.380432370044835	-0.0023193359375	\\
0.38047676121987	-0.00244140625	\\
0.380521152394904	-0.002166748046875	\\
0.380565543569938	-0.002166748046875	\\
0.380609934744973	-0.00189208984375	\\
0.380654325920007	-0.00201416015625	\\
0.380698717095041	-0.0018310546875	\\
0.380743108270076	-0.0013427734375	\\
0.38078749944511	-0.001678466796875	\\
0.380831890620145	-0.001861572265625	\\
0.380876281795179	-0.001495361328125	\\
0.380920672970214	-0.00177001953125	\\
0.380965064145248	-0.0018310546875	\\
0.381009455320282	-0.001678466796875	\\
0.381053846495317	-0.001800537109375	\\
0.381098237670351	-0.001708984375	\\
0.381142628845386	-0.001983642578125	\\
0.38118702002042	-0.00201416015625	\\
0.381231411195454	-0.002044677734375	\\
0.381275802370489	-0.00274658203125	\\
0.381320193545523	-0.00244140625	\\
0.381364584720558	-0.002655029296875	\\
0.381408975895592	-0.002532958984375	\\
0.381453367070626	-0.00244140625	\\
0.381497758245661	-0.00299072265625	\\
0.381542149420695	-0.002716064453125	\\
0.38158654059573	-0.00238037109375	\\
0.381630931770764	-0.00244140625	\\
0.381675322945798	-0.002685546875	\\
0.381719714120833	-0.0030517578125	\\
0.381764105295867	-0.00360107421875	\\
0.381808496470902	-0.003692626953125	\\
0.381852887645936	-0.00360107421875	\\
0.38189727882097	-0.00335693359375	\\
0.381941669996005	-0.003692626953125	\\
0.381986061171039	-0.00396728515625	\\
0.382030452346074	-0.003936767578125	\\
0.382074843521108	-0.003570556640625	\\
0.382119234696142	-0.003662109375	\\
0.382163625871177	-0.00360107421875	\\
0.382208017046211	-0.003875732421875	\\
0.382252408221246	-0.003936767578125	\\
0.38229679939628	-0.003631591796875	\\
0.382341190571314	-0.003265380859375	\\
0.382385581746349	-0.002838134765625	\\
0.382429972921383	-0.00335693359375	\\
0.382474364096418	-0.003326416015625	\\
0.382518755271452	-0.00286865234375	\\
0.382563146446486	-0.003143310546875	\\
0.382607537621521	-0.003082275390625	\\
0.382651928796555	-0.002899169921875	\\
0.38269631997159	-0.002197265625	\\
0.382740711146624	-0.001983642578125	\\
0.382785102321658	-0.001922607421875	\\
0.382829493496693	-0.00177001953125	\\
0.382873884671727	-0.001922607421875	\\
0.382918275846762	-0.001708984375	\\
0.382962667021796	-0.001556396484375	\\
0.38300705819683	-0.00140380859375	\\
0.383051449371865	-0.001373291015625	\\
0.383095840546899	-0.001373291015625	\\
0.383140231721934	-0.0006103515625	\\
0.383184622896968	-0.000213623046875	\\
0.383229014072002	-0.000274658203125	\\
0.383273405247037	-0.000244140625	\\
0.383317796422071	-0.000335693359375	\\
0.383362187597106	-0.0003662109375	\\
0.38340657877214	-0.000152587890625	\\
0.383450969947174	-0.00042724609375	\\
0.383495361122209	-0.000640869140625	\\
0.383539752297243	-0.00091552734375	\\
0.383584143472278	-0.000640869140625	\\
0.383628534647312	-0.00054931640625	\\
0.383672925822347	-0.00079345703125	\\
0.383717316997381	-0.00091552734375	\\
0.383761708172415	-0.001007080078125	\\
0.38380609934745	-0.00067138671875	\\
0.383850490522484	-0.00030517578125	\\
0.383894881697519	-0.00048828125	\\
0.383939272872553	-0.000762939453125	\\
0.383983664047587	-0.000579833984375	\\
0.384028055222622	-0.00115966796875	\\
0.384072446397656	-0.0013427734375	\\
0.384116837572691	-0.00177001953125	\\
0.384161228747725	-0.002197265625	\\
0.384205619922759	-0.0023193359375	\\
0.384250011097794	-0.002166748046875	\\
0.384294402272828	-0.00189208984375	\\
0.384338793447863	-0.002227783203125	\\
0.384383184622897	-0.0020751953125	\\
0.384427575797931	-0.001434326171875	\\
0.384471966972966	-0.001861572265625	\\
0.384516358148	-0.00201416015625	\\
0.384560749323035	-0.0015869140625	\\
0.384605140498069	-0.00152587890625	\\
0.384649531673103	-0.00140380859375	\\
0.384693922848138	-0.001251220703125	\\
0.384738314023172	-0.00164794921875	\\
0.384782705198207	-0.00115966796875	\\
0.384827096373241	-0.001068115234375	\\
0.384871487548275	-0.00152587890625	\\
0.38491587872331	-0.001434326171875	\\
0.384960269898344	-0.001495361328125	\\
0.385004661073379	-0.001312255859375	\\
0.385049052248413	-0.001007080078125	\\
0.385093443423447	-0.001220703125	\\
0.385137834598482	-0.00103759765625	\\
0.385182225773516	-0.00091552734375	\\
0.385226616948551	-0.001129150390625	\\
0.385271008123585	-0.0013427734375	\\
0.385315399298619	-0.00146484375	\\
0.385359790473654	-0.001617431640625	\\
0.385404181648688	-0.002044677734375	\\
0.385448572823723	-0.0018310546875	\\
0.385492963998757	-0.002227783203125	\\
0.385537355173791	-0.002197265625	\\
0.385581746348826	-0.001861572265625	\\
0.38562613752386	-0.00201416015625	\\
0.385670528698895	-0.002197265625	\\
0.385714919873929	-0.002044677734375	\\
0.385759311048963	-0.001953125	\\
0.385803702223998	-0.002166748046875	\\
0.385848093399032	-0.002044677734375	\\
0.385892484574067	-0.00225830078125	\\
0.385936875749101	-0.002410888671875	\\
0.385981266924135	-0.002044677734375	\\
0.38602565809917	-0.00213623046875	\\
0.386070049274204	-0.002685546875	\\
0.386114440449239	-0.002593994140625	\\
0.386158831624273	-0.002532958984375	\\
0.386203222799308	-0.00274658203125	\\
0.386247613974342	-0.002471923828125	\\
0.386292005149376	-0.002197265625	\\
0.386336396324411	-0.002349853515625	\\
0.386380787499445	-0.0023193359375	\\
0.38642517867448	-0.00213623046875	\\
0.386469569849514	-0.00201416015625	\\
0.386513961024548	-0.00152587890625	\\
0.386558352199583	-0.00146484375	\\
0.386602743374617	-0.00146484375	\\
0.386647134549652	-0.001556396484375	\\
0.386691525724686	-0.001708984375	\\
0.38673591689972	-0.001617431640625	\\
0.386780308074755	-0.00140380859375	\\
0.386824699249789	-0.001220703125	\\
0.386869090424824	-0.000823974609375	\\
0.386913481599858	-0.000946044921875	\\
0.386957872774892	-0.0009765625	\\
0.387002263949927	-0.000762939453125	\\
0.387046655124961	-0.00103759765625	\\
0.387091046299996	-0.000701904296875	\\
0.38713543747503	-0.00091552734375	\\
0.387179828650064	-0.0009765625	\\
0.387224219825099	-0.001007080078125	\\
0.387268611000133	-0.00152587890625	\\
0.387313002175168	-0.0015869140625	\\
0.387357393350202	-0.0015869140625	\\
0.387401784525236	-0.001708984375	\\
0.387446175700271	-0.00164794921875	\\
0.387490566875305	-0.001495361328125	\\
0.38753495805034	-0.000946044921875	\\
0.387579349225374	-0.0010986328125	\\
0.387623740400408	-0.0010986328125	\\
0.387668131575443	-0.0009765625	\\
0.387712522750477	-0.001190185546875	\\
0.387756913925512	-0.001373291015625	\\
0.387801305100546	-0.001861572265625	\\
0.38784569627558	-0.0020751953125	\\
0.387890087450615	-0.002532958984375	\\
0.387934478625649	-0.002532958984375	\\
0.387978869800684	-0.002105712890625	\\
0.388023260975718	-0.002197265625	\\
0.388067652150752	-0.00177001953125	\\
0.388112043325787	-0.002105712890625	\\
0.388156434500821	-0.002655029296875	\\
0.388200825675856	-0.003204345703125	\\
0.38824521685089	-0.0029296875	\\
0.388289608025924	-0.00238037109375	\\
0.388333999200959	-0.002685546875	\\
0.388378390375993	-0.002532958984375	\\
0.388422781551028	-0.002716064453125	\\
0.388467172726062	-0.002716064453125	\\
0.388511563901096	-0.002471923828125	\\
0.388555955076131	-0.00250244140625	\\
0.388600346251165	-0.002655029296875	\\
0.3886447374262	-0.003021240234375	\\
0.388689128601234	-0.002685546875	\\
0.388733519776268	-0.002777099609375	\\
0.388777910951303	-0.0029296875	\\
0.388822302126337	-0.0025634765625	\\
0.388866693301372	-0.00262451171875	\\
0.388911084476406	-0.002593994140625	\\
0.388955475651441	-0.002532958984375	\\
0.388999866826475	-0.002838134765625	\\
0.389044258001509	-0.0028076171875	\\
0.389088649176544	-0.0028076171875	\\
0.389133040351578	-0.0029296875	\\
0.389177431526612	-0.00323486328125	\\
0.389221822701647	-0.003173828125	\\
0.389266213876681	-0.0032958984375	\\
0.389310605051716	-0.003387451171875	\\
0.38935499622675	-0.003448486328125	\\
0.389399387401785	-0.00360107421875	\\
0.389443778576819	-0.00384521484375	\\
0.389488169751853	-0.003692626953125	\\
0.389532560926888	-0.003173828125	\\
0.389576952101922	-0.00384521484375	\\
0.389621343276957	-0.00360107421875	\\
0.389665734451991	-0.0029296875	\\
0.389710125627025	-0.003173828125	\\
0.38975451680206	-0.00323486328125	\\
0.389798907977094	-0.003448486328125	\\
0.389843299152129	-0.00360107421875	\\
0.389887690327163	-0.003631591796875	\\
0.389932081502197	-0.00335693359375	\\
0.389976472677232	-0.00335693359375	\\
0.390020863852266	-0.003082275390625	\\
0.390065255027301	-0.002960205078125	\\
0.390109646202335	-0.003387451171875	\\
0.390154037377369	-0.0035400390625	\\
0.390198428552404	-0.003173828125	\\
0.390242819727438	-0.003173828125	\\
0.390287210902473	-0.003021240234375	\\
0.390331602077507	-0.00274658203125	\\
0.390375993252541	-0.00286865234375	\\
0.390420384427576	-0.00238037109375	\\
0.39046477560261	-0.002685546875	\\
0.390509166777645	-0.0028076171875	\\
0.390553557952679	-0.002471923828125	\\
0.390597949127713	-0.002532958984375	\\
0.390642340302748	-0.00238037109375	\\
0.390686731477782	-0.00225830078125	\\
0.390731122652817	-0.0023193359375	\\
0.390775513827851	-0.00225830078125	\\
0.390819905002885	-0.00238037109375	\\
0.39086429617792	-0.002288818359375	\\
0.390908687352954	-0.002197265625	\\
0.390953078527989	-0.002105712890625	\\
0.390997469703023	-0.001861572265625	\\
0.391041860878057	-0.002044677734375	\\
0.391086252053092	-0.0020751953125	\\
0.391130643228126	-0.00164794921875	\\
0.391175034403161	-0.00152587890625	\\
0.391219425578195	-0.00164794921875	\\
0.391263816753229	-0.002166748046875	\\
0.391308207928264	-0.00238037109375	\\
0.391352599103298	-0.002227783203125	\\
0.391396990278333	-0.00201416015625	\\
0.391441381453367	-0.002197265625	\\
0.391485772628401	-0.0020751953125	\\
0.391530163803436	-0.00238037109375	\\
0.39157455497847	-0.002471923828125	\\
0.391618946153505	-0.0023193359375	\\
0.391663337328539	-0.002349853515625	\\
0.391707728503573	-0.002471923828125	\\
0.391752119678608	-0.002899169921875	\\
0.391796510853642	-0.00286865234375	\\
0.391840902028677	-0.003021240234375	\\
0.391885293203711	-0.003204345703125	\\
0.391929684378745	-0.003265380859375	\\
0.39197407555378	-0.00299072265625	\\
0.392018466728814	-0.003143310546875	\\
0.392062857903849	-0.002777099609375	\\
0.392107249078883	-0.00225830078125	\\
0.392151640253918	-0.00225830078125	\\
0.392196031428952	-0.002716064453125	\\
0.392240422603986	-0.002410888671875	\\
0.392284813779021	-0.002166748046875	\\
0.392329204954055	-0.00225830078125	\\
0.39237359612909	-0.002105712890625	\\
0.392417987304124	-0.001617431640625	\\
0.392462378479158	-0.0015869140625	\\
0.392506769654193	-0.002044677734375	\\
0.392551160829227	-0.0015869140625	\\
0.392595552004262	-0.00140380859375	\\
0.392639943179296	-0.0013427734375	\\
0.39268433435433	-0.00146484375	\\
0.392728725529365	-0.00152587890625	\\
0.392773116704399	-0.001373291015625	\\
0.392817507879434	-0.000579833984375	\\
0.392861899054468	-0.000396728515625	\\
0.392906290229502	-0.000640869140625	\\
0.392950681404537	0	\\
0.392995072579571	-0.00018310546875	\\
0.393039463754606	-0.000213623046875	\\
0.39308385492964	0.0001220703125	\\
0.393128246104674	-0.000274658203125	\\
0.393172637279709	-0.000244140625	\\
0.393217028454743	0.00042724609375	\\
0.393261419629778	-0.00030517578125	\\
0.393305810804812	-0.001190185546875	\\
0.393350201979846	-0.00091552734375	\\
0.393394593154881	-0.0009765625	\\
0.393438984329915	-0.001220703125	\\
0.39348337550495	-0.000762939453125	\\
0.393527766679984	-0.001007080078125	\\
0.393572157855018	-0.001129150390625	\\
0.393616549030053	-0.0013427734375	\\
0.393660940205087	-0.001739501953125	\\
0.393705331380122	-0.0015869140625	\\
0.393749722555156	-0.001251220703125	\\
0.39379411373019	-0.00103759765625	\\
0.393838504905225	-0.00152587890625	\\
0.393882896080259	-0.001708984375	\\
0.393927287255294	-0.0015869140625	\\
0.393971678430328	-0.00115966796875	\\
0.394016069605362	-0.00067138671875	\\
0.394060460780397	-0.00067138671875	\\
0.394104851955431	-0.000732421875	\\
0.394149243130466	-0.0006103515625	\\
0.3941936343055	-0.00042724609375	\\
0.394238025480534	-0.00042724609375	\\
0.394282416655569	-0.000335693359375	\\
0.394326807830603	-0.0001220703125	\\
0.394371199005638	0.000213623046875	\\
0.394415590180672	0.00042724609375	\\
0.394459981355706	0.000732421875	\\
0.394504372530741	0.000823974609375	\\
0.394548763705775	0.000274658203125	\\
0.39459315488081	-6.103515625e-05	\\
0.394637546055844	0.0003662109375	\\
0.394681937230879	0.0003662109375	\\
0.394726328405913	0.00030517578125	\\
0.394770719580947	0.000244140625	\\
0.394815110755982	0.000518798828125	\\
0.394859501931016	0.001007080078125	\\
0.394903893106051	0.0010986328125	\\
0.394948284281085	0.000762939453125	\\
0.394992675456119	0.0006103515625	\\
0.395037066631154	0.000213623046875	\\
0.395081457806188	0.00042724609375	\\
0.395125848981223	0.0003662109375	\\
0.395170240156257	9.1552734375e-05	\\
0.395214631331291	0.00018310546875	\\
0.395259022506326	-0.00030517578125	\\
0.39530341368136	-0.000396728515625	\\
0.395347804856395	0.000244140625	\\
0.395392196031429	0	\\
0.395436587206463	-0.000457763671875	\\
0.395480978381498	-0.00042724609375	\\
0.395525369556532	-0.000244140625	\\
0.395569760731567	-0.0003662109375	\\
0.395614151906601	-0.0001220703125	\\
0.395658543081635	3.0517578125e-05	\\
0.39570293425667	-0.000213623046875	\\
0.395747325431704	-0.000335693359375	\\
0.395791716606739	-0.000244140625	\\
0.395836107781773	-0.000335693359375	\\
0.395880498956807	-0.00018310546875	\\
0.395924890131842	-0.0003662109375	\\
0.395969281306876	-0.000213623046875	\\
0.396013672481911	-0.000274658203125	\\
0.396058063656945	-0.00048828125	\\
0.396102454831979	-0.000335693359375	\\
0.396146846007014	3.0517578125e-05	\\
0.396191237182048	0.00030517578125	\\
0.396235628357083	-0.0003662109375	\\
0.396280019532117	-0.0001220703125	\\
0.396324410707151	-0.0001220703125	\\
0.396368801882186	9.1552734375e-05	\\
0.39641319305722	6.103515625e-05	\\
0.396457584232255	0.0001220703125	\\
0.396501975407289	0.00042724609375	\\
0.396546366582323	0.000518798828125	\\
0.396590757757358	0.000762939453125	\\
0.396635148932392	0.0006103515625	\\
0.396679540107427	0.000335693359375	\\
0.396723931282461	0.000457763671875	\\
0.396768322457495	0.000457763671875	\\
0.39681271363253	0.000518798828125	\\
0.396857104807564	0.00054931640625	\\
0.396901495982599	0.000335693359375	\\
0.396945887157633	0.000732421875	\\
0.396990278332667	0.0006103515625	\\
0.397034669507702	0.000457763671875	\\
0.397079060682736	0.0003662109375	\\
0.397123451857771	0.000396728515625	\\
0.397167843032805	0.0001220703125	\\
0.397212234207839	-0.00042724609375	\\
0.397256625382874	-0.0001220703125	\\
0.397301016557908	-3.0517578125e-05	\\
0.397345407732943	0.000457763671875	\\
0.397389798907977	0.000396728515625	\\
0.397434190083012	0.000213623046875	\\
0.397478581258046	0.000152587890625	\\
0.39752297243308	0.00018310546875	\\
0.397567363608115	-0.00018310546875	\\
0.397611754783149	-0.0001220703125	\\
0.397656145958183	0.000213623046875	\\
0.397700537133218	-0.000213623046875	\\
0.397744928308252	-0.00054931640625	\\
0.397789319483287	-0.00054931640625	\\
0.397833710658321	-0.000213623046875	\\
0.397878101833356	0.0001220703125	\\
0.39792249300839	0	\\
0.397966884183424	-9.1552734375e-05	\\
0.398011275358459	-0.00018310546875	\\
0.398055666533493	0.000152587890625	\\
0.398100057708528	0	\\
0.398144448883562	0.00018310546875	\\
0.398188840058596	0.00048828125	\\
0.398233231233631	6.103515625e-05	\\
0.398277622408665	0	\\
0.3983220135837	0.000152587890625	\\
0.398366404758734	0	\\
0.398410795933768	0.000213623046875	\\
0.398455187108803	0.000457763671875	\\
0.398499578283837	0.0003662109375	\\
0.398543969458872	0.00091552734375	\\
0.398588360633906	0.000823974609375	\\
0.39863275180894	0.001129150390625	\\
0.398677142983975	0.000579833984375	\\
0.398721534159009	0.000274658203125	\\
0.398765925334044	0.0006103515625	\\
0.398810316509078	-6.103515625e-05	\\
0.398854707684112	9.1552734375e-05	\\
0.398899098859147	0.000579833984375	\\
0.398943490034181	0.000396728515625	\\
0.398987881209216	0.00042724609375	\\
0.39903227238425	6.103515625e-05	\\
0.399076663559284	3.0517578125e-05	\\
0.399121054734319	3.0517578125e-05	\\
0.399165445909353	0.000152587890625	\\
0.399209837084388	0.000335693359375	\\
0.399254228259422	0.000335693359375	\\
0.399298619434456	0.000335693359375	\\
0.399343010609491	0.00018310546875	\\
0.399387401784525	0.000457763671875	\\
0.39943179295956	0.000579833984375	\\
0.399476184134594	0.000396728515625	\\
0.399520575309628	0.000152587890625	\\
0.399564966484663	-9.1552734375e-05	\\
0.399609357659697	-3.0517578125e-05	\\
0.399653748834732	0.00018310546875	\\
0.399698140009766	-0.000213623046875	\\
0.3997425311848	-0.000396728515625	\\
0.399786922359835	6.103515625e-05	\\
0.399831313534869	-3.0517578125e-05	\\
0.399875704709904	0.00018310546875	\\
0.399920095884938	0.00030517578125	\\
0.399964487059972	0.000152587890625	\\
0.400008878235007	-3.0517578125e-05	\\
0.400053269410041	0	\\
0.400097660585076	0.00048828125	\\
0.40014205176011	0.00030517578125	\\
0.400186442935144	0.000335693359375	\\
0.400230834110179	0.000457763671875	\\
0.400275225285213	0.0003662109375	\\
0.400319616460248	0.000701904296875	\\
0.400364007635282	0.0006103515625	\\
0.400408398810317	3.0517578125e-05	\\
0.400452789985351	-0.00018310546875	\\
0.400497181160385	-9.1552734375e-05	\\
0.40054157233542	0.000396728515625	\\
0.400585963510454	0.00079345703125	\\
0.400630354685489	0.000518798828125	\\
0.400674745860523	0.000762939453125	\\
0.400719137035557	0.00115966796875	\\
0.400763528210592	0.00079345703125	\\
0.400807919385626	0.0010986328125	\\
0.400852310560661	0.0010986328125	\\
0.400896701735695	0.00103759765625	\\
0.400941092910729	0.00091552734375	\\
0.400985484085764	0.000518798828125	\\
0.401029875260798	0.000885009765625	\\
0.401074266435833	0.00115966796875	\\
0.401118657610867	0.001129150390625	\\
0.401163048785901	0.0008544921875	\\
0.401207439960936	0.001007080078125	\\
0.40125183113597	0.00067138671875	\\
0.401296222311005	0.00018310546875	\\
0.401340613486039	0.00067138671875	\\
0.401385004661073	0.000823974609375	\\
0.401429395836108	0.000823974609375	\\
0.401473787011142	0.00091552734375	\\
0.401518178186177	0.00067138671875	\\
0.401562569361211	0.000518798828125	\\
0.401606960536245	0.00079345703125	\\
0.40165135171128	0.00067138671875	\\
0.401695742886314	0.00048828125	\\
0.401740134061349	0.0010986328125	\\
0.401784525236383	0.00115966796875	\\
0.401828916411417	0.001190185546875	\\
0.401873307586452	0.001434326171875	\\
0.401917698761486	0.0015869140625	\\
0.401962089936521	0.001556396484375	\\
0.402006481111555	0.00152587890625	\\
0.402050872286589	0.001861572265625	\\
0.402095263461624	0.001495361328125	\\
0.402139654636658	0.0013427734375	\\
0.402184045811693	0.001708984375	\\
0.402228436986727	0.001739501953125	\\
0.402272828161761	0.00177001953125	\\
0.402317219336796	0.001800537109375	\\
0.40236161051183	0.0018310546875	\\
0.402406001686865	0.00164794921875	\\
0.402450392861899	0.001495361328125	\\
0.402494784036933	0.001495361328125	\\
0.402539175211968	0.0015869140625	\\
0.402583566387002	0.002166748046875	\\
0.402627957562037	0.002349853515625	\\
0.402672348737071	0.00244140625	\\
0.402716739912105	0.00201416015625	\\
0.40276113108714	0.002197265625	\\
0.402805522262174	0.002777099609375	\\
0.402849913437209	0.002777099609375	\\
0.402894304612243	0.00262451171875	\\
0.402938695787277	0.002532958984375	\\
0.402983086962312	0.002532958984375	\\
0.403027478137346	0.002288818359375	\\
0.403071869312381	0.00225830078125	\\
0.403116260487415	0.00238037109375	\\
0.40316065166245	0.00225830078125	\\
0.403205042837484	0.002105712890625	\\
0.403249434012518	0.00213623046875	\\
0.403293825187553	0.00244140625	\\
0.403338216362587	0.00213623046875	\\
0.403382607537622	0.0020751953125	\\
0.403426998712656	0.0023193359375	\\
0.40347138988769	0.00225830078125	\\
0.403515781062725	0.001800537109375	\\
0.403560172237759	0.00152587890625	\\
0.403604563412794	0.00128173828125	\\
0.403648954587828	0.001068115234375	\\
0.403693345762862	0.00140380859375	\\
0.403737736937897	0.001312255859375	\\
0.403782128112931	0.00128173828125	\\
0.403826519287966	0.0015869140625	\\
0.403870910463	0.00177001953125	\\
0.403915301638034	0.001739501953125	\\
0.403959692813069	0.001983642578125	\\
0.404004083988103	0.001983642578125	\\
0.404048475163138	0.001800537109375	\\
0.404092866338172	0.001922607421875	\\
0.404137257513206	0.002288818359375	\\
0.404181648688241	0.002044677734375	\\
0.404226039863275	0.0020751953125	\\
0.40427043103831	0.001953125	\\
0.404314822213344	0.0015869140625	\\
0.404359213388378	0.00189208984375	\\
0.404403604563413	0.001953125	\\
0.404447995738447	0.00225830078125	\\
0.404492386913482	0.002349853515625	\\
0.404536778088516	0.0020751953125	\\
0.40458116926355	0.00238037109375	\\
0.404625560438585	0.00262451171875	\\
0.404669951613619	0.00244140625	\\
0.404714342788654	0.002105712890625	\\
0.404758733963688	0.001953125	\\
0.404803125138722	0.001983642578125	\\
0.404847516313757	0.001800537109375	\\
0.404891907488791	0.001556396484375	\\
0.404936298663826	0.001739501953125	\\
0.40498068983886	0.001708984375	\\
0.405025081013894	0.001861572265625	\\
0.405069472188929	0.00189208984375	\\
0.405113863363963	0.001800537109375	\\
0.405158254538998	0.001708984375	\\
0.405202645714032	0.001739501953125	\\
0.405247036889066	0.001708984375	\\
0.405291428064101	0.001312255859375	\\
0.405335819239135	0.001068115234375	\\
0.40538021041417	0.00115966796875	\\
0.405424601589204	0.001190185546875	\\
0.405468992764238	0.00091552734375	\\
0.405513383939273	0.000946044921875	\\
0.405557775114307	0.0009765625	\\
0.405602166289342	0.000701904296875	\\
0.405646557464376	0.0003662109375	\\
0.40569094863941	0.00048828125	\\
0.405735339814445	0.000701904296875	\\
0.405779730989479	0.00067138671875	\\
0.405824122164514	0.000885009765625	\\
0.405868513339548	0.00091552734375	\\
0.405912904514583	0.000701904296875	\\
0.405957295689617	0.00103759765625	\\
0.406001686864651	0.0010986328125	\\
0.406046078039686	0.00054931640625	\\
0.40609046921472	0.000274658203125	\\
0.406134860389754	0.000152587890625	\\
0.406179251564789	0.00048828125	\\
0.406223642739823	0.000823974609375	\\
0.406268033914858	0.000701904296875	\\
0.406312425089892	0.000518798828125	\\
0.406356816264927	0.000701904296875	\\
0.406401207439961	0.000762939453125	\\
0.406445598614995	0.0006103515625	\\
0.40648998979003	0.000762939453125	\\
0.406534380965064	0.00103759765625	\\
0.406578772140099	0.00103759765625	\\
0.406623163315133	0.000152587890625	\\
0.406667554490167	0.00030517578125	\\
0.406711945665202	0.0003662109375	\\
0.406756336840236	0.000244140625	\\
0.406800728015271	0.000396728515625	\\
0.406845119190305	0.000244140625	\\
0.406889510365339	0.000274658203125	\\
0.406933901540374	0.0001220703125	\\
0.406978292715408	0.000518798828125	\\
0.407022683890443	0.0009765625	\\
0.407067075065477	0.0006103515625	\\
0.407111466240511	0.000518798828125	\\
0.407155857415546	0.000885009765625	\\
0.40720024859058	0.00103759765625	\\
0.407244639765615	0.000946044921875	\\
0.407289030940649	0.00067138671875	\\
0.407333422115683	0.0008544921875	\\
0.407377813290718	0.00067138671875	\\
0.407422204465752	0.000732421875	\\
0.407466595640787	0.00054931640625	\\
0.407510986815821	0.00042724609375	\\
0.407555377990855	0.0010986328125	\\
0.40759976916589	0.000701904296875	\\
0.407644160340924	0.000579833984375	\\
0.407688551515959	0.000701904296875	\\
0.407732942690993	0.001129150390625	\\
0.407777333866027	0.001190185546875	\\
0.407821725041062	0.00079345703125	\\
0.407866116216096	0.00115966796875	\\
0.407910507391131	0.00079345703125	\\
0.407954898566165	0.000335693359375	\\
0.407999289741199	0.0008544921875	\\
0.408043680916234	0.00091552734375	\\
0.408088072091268	0.000762939453125	\\
0.408132463266303	0.00079345703125	\\
0.408176854441337	0.0008544921875	\\
0.408221245616371	0.0009765625	\\
0.408265636791406	0.001190185546875	\\
0.40831002796644	0.001129150390625	\\
0.408354419141475	0.001190185546875	\\
0.408398810316509	0.001190185546875	\\
0.408443201491544	0.001190185546875	\\
0.408487592666578	0.00146484375	\\
0.408531983841612	0.00128173828125	\\
0.408576375016647	0.001220703125	\\
0.408620766191681	0.0013427734375	\\
0.408665157366715	0.00115966796875	\\
0.40870954854175	0.001556396484375	\\
0.408753939716784	0.001678466796875	\\
0.408798330891819	0.001312255859375	\\
0.408842722066853	0.001312255859375	\\
0.408887113241888	0.001251220703125	\\
0.408931504416922	0.001251220703125	\\
0.408975895591956	0.00164794921875	\\
0.409020286766991	0.001922607421875	\\
0.409064677942025	0.001617431640625	\\
0.40910906911706	0.0010986328125	\\
0.409153460292094	0.0006103515625	\\
0.409197851467128	0.000579833984375	\\
0.409242242642163	0.000885009765625	\\
0.409286633817197	0.000732421875	\\
0.409331024992232	0.00054931640625	\\
0.409375416167266	0.001007080078125	\\
0.4094198073423	0.001495361328125	\\
0.409464198517335	0.001220703125	\\
0.409508589692369	0.0013427734375	\\
0.409552980867404	0.001678466796875	\\
0.409597372042438	0.00164794921875	\\
0.409641763217472	0.0018310546875	\\
0.409686154392507	0.001739501953125	\\
0.409730545567541	0.001678466796875	\\
0.409774936742576	0.001617431640625	\\
0.40981932791761	0.00177001953125	\\
0.409863719092644	0.001708984375	\\
0.409908110267679	0.001373291015625	\\
0.409952501442713	0.001373291015625	\\
0.409996892617748	0.001556396484375	\\
0.410041283792782	0.001708984375	\\
0.410085674967816	0.0013427734375	\\
0.410130066142851	0.001251220703125	\\
0.410174457317885	0.0013427734375	\\
0.41021884849292	0.00177001953125	\\
0.410263239667954	0.001800537109375	\\
0.410307630842988	0.00103759765625	\\
0.410352022018023	0.0009765625	\\
0.410396413193057	0.000885009765625	\\
0.410440804368092	0.000640869140625	\\
0.410485195543126	0.0009765625	\\
0.41052958671816	0.00048828125	\\
0.410573977893195	0.0008544921875	\\
0.410618369068229	0.0009765625	\\
0.410662760243264	0.001068115234375	\\
0.410707151418298	0.00115966796875	\\
0.410751542593332	0.001220703125	\\
0.410795933768367	0.001129150390625	\\
0.410840324943401	0.00103759765625	\\
0.410884716118436	0.001495361328125	\\
0.41092910729347	0.00103759765625	\\
0.410973498468504	0.00103759765625	\\
0.411017889643539	0.001190185546875	\\
0.411062280818573	0.001312255859375	\\
0.411106671993608	0.001739501953125	\\
0.411151063168642	0.0015869140625	\\
0.411195454343676	0.00146484375	\\
0.411239845518711	0.001251220703125	\\
0.411284236693745	0.00103759765625	\\
0.41132862786878	0.00091552734375	\\
0.411373019043814	0.000457763671875	\\
0.411417410218848	0.000823974609375	\\
0.411461801393883	0.000762939453125	\\
0.411506192568917	0.00067138671875	\\
0.411550583743952	0.001068115234375	\\
0.411594974918986	0.00115966796875	\\
0.411639366094021	0.00115966796875	\\
0.411683757269055	0.0013427734375	\\
0.411728148444089	0.001220703125	\\
0.411772539619124	0.001373291015625	\\
0.411816930794158	0.001922607421875	\\
0.411861321969193	0.0020751953125	\\
0.411905713144227	0.002044677734375	\\
0.411950104319261	0.002166748046875	\\
0.411994495494296	0.00225830078125	\\
0.41203888666933	0.002471923828125	\\
0.412083277844365	0.00225830078125	\\
0.412127669019399	0.00177001953125	\\
0.412172060194433	0.00189208984375	\\
0.412216451369468	0.00238037109375	\\
0.412260842544502	0.002655029296875	\\
0.412305233719537	0.002777099609375	\\
0.412349624894571	0.002960205078125	\\
0.412394016069605	0.002410888671875	\\
0.41243840724464	0.00244140625	\\
0.412482798419674	0.002960205078125	\\
0.412527189594709	0.0028076171875	\\
0.412571580769743	0.00286865234375	\\
0.412615971944777	0.00262451171875	\\
0.412660363119812	0.002655029296875	\\
0.412704754294846	0.002655029296875	\\
0.412749145469881	0.002105712890625	\\
0.412793536644915	0.00213623046875	\\
0.412837927819949	0.002410888671875	\\
0.412882318994984	0.002410888671875	\\
0.412926710170018	0.002166748046875	\\
0.412971101345053	0.001861572265625	\\
0.413015492520087	0.001617431640625	\\
0.413059883695121	0.001708984375	\\
0.413104274870156	0.002227783203125	\\
0.41314866604519	0.002685546875	\\
0.413193057220225	0.002288818359375	\\
0.413237448395259	0.00140380859375	\\
0.413281839570293	0.00140380859375	\\
0.413326230745328	0.001678466796875	\\
0.413370621920362	0.00140380859375	\\
0.413415013095397	0.00146484375	\\
0.413459404270431	0.001190185546875	\\
0.413503795445465	0.000823974609375	\\
0.4135481866205	0.000732421875	\\
0.413592577795534	0.000457763671875	\\
0.413636968970569	0.00054931640625	\\
0.413681360145603	0.001190185546875	\\
0.413725751320637	0.001251220703125	\\
0.413770142495672	0.001068115234375	\\
0.413814533670706	0.00164794921875	\\
0.413858924845741	0.001953125	\\
0.413903316020775	0.001617431640625	\\
0.413947707195809	0.001739501953125	\\
0.413992098370844	0.00213623046875	\\
0.414036489545878	0.001800537109375	\\
0.414080880720913	0.00146484375	\\
0.414125271895947	0.002105712890625	\\
0.414169663070981	0.002532958984375	\\
0.414214054246016	0.002197265625	\\
0.41425844542105	0.0020751953125	\\
0.414302836596085	0.002532958984375	\\
0.414347227771119	0.0025634765625	\\
0.414391618946154	0.00177001953125	\\
0.414436010121188	0.001708984375	\\
0.414480401296222	0.002349853515625	\\
0.414524792471257	0.002471923828125	\\
0.414569183646291	0.001922607421875	\\
0.414613574821326	0.00189208984375	\\
0.41465796599636	0.0018310546875	\\
0.414702357171394	0.002166748046875	\\
0.414746748346429	0.002288818359375	\\
0.414791139521463	0.00238037109375	\\
0.414835530696498	0.00244140625	\\
0.414879921871532	0.00238037109375	\\
0.414924313046566	0.00262451171875	\\
0.414968704221601	0.002716064453125	\\
0.415013095396635	0.002349853515625	\\
0.41505748657167	0.00244140625	\\
0.415101877746704	0.00274658203125	\\
0.415146268921738	0.002197265625	\\
0.415190660096773	0.002227783203125	\\
0.415235051271807	0.00213623046875	\\
0.415279442446842	0.001983642578125	\\
0.415323833621876	0.002655029296875	\\
0.41536822479691	0.00262451171875	\\
0.415412615971945	0.00250244140625	\\
0.415457007146979	0.002471923828125	\\
0.415501398322014	0.00213623046875	\\
0.415545789497048	0.00213623046875	\\
0.415590180672082	0.001861572265625	\\
0.415634571847117	0.0018310546875	\\
0.415678963022151	0.00201416015625	\\
0.415723354197186	0.002166748046875	\\
0.41576774537222	0.00238037109375	\\
0.415812136547254	0.002166748046875	\\
0.415856527722289	0.00238037109375	\\
0.415900918897323	0.002716064453125	\\
0.415945310072358	0.002716064453125	\\
0.415989701247392	0.002685546875	\\
0.416034092422426	0.002410888671875	\\
0.416078483597461	0.0020751953125	\\
0.416122874772495	0.001708984375	\\
0.41616726594753	0.00225830078125	\\
0.416211657122564	0.002288818359375	\\
0.416256048297598	0.00201416015625	\\
0.416300439472633	0.001922607421875	\\
0.416344830647667	0.001861572265625	\\
0.416389221822702	0.00201416015625	\\
0.416433612997736	0.00152587890625	\\
0.41647800417277	0.002197265625	\\
0.416522395347805	0.0020751953125	\\
0.416566786522839	0.00164794921875	\\
0.416611177697874	0.002227783203125	\\
0.416655568872908	0.002105712890625	\\
0.416699960047942	0.002166748046875	\\
0.416744351222977	0.002349853515625	\\
0.416788742398011	0.002227783203125	\\
0.416833133573046	0.0020751953125	\\
0.41687752474808	0.002105712890625	\\
0.416921915923115	0.002227783203125	\\
0.416966307098149	0.002655029296875	\\
0.417010698273183	0.00262451171875	\\
0.417055089448218	0.002471923828125	\\
0.417099480623252	0.002685546875	\\
0.417143871798286	0.00262451171875	\\
0.417188262973321	0.0028076171875	\\
0.417232654148355	0.0023193359375	\\
0.41727704532339	0.0020751953125	\\
0.417321436498424	0.002410888671875	\\
0.417365827673459	0.002410888671875	\\
0.417410218848493	0.002655029296875	\\
0.417454610023527	0.0029296875	\\
0.417499001198562	0.002899169921875	\\
0.417543392373596	0.003021240234375	\\
0.417587783548631	0.00323486328125	\\
0.417632174723665	0.00335693359375	\\
0.417676565898699	0.003326416015625	\\
0.417720957073734	0.003173828125	\\
0.417765348248768	0.003204345703125	\\
0.417809739423803	0.003265380859375	\\
0.417854130598837	0.00335693359375	\\
0.417898521773871	0.0035400390625	\\
0.417942912948906	0.0040283203125	\\
0.41798730412394	0.004058837890625	\\
0.418031695298975	0.003509521484375	\\
0.418076086474009	0.0035400390625	\\
0.418120477649043	0.003509521484375	\\
0.418164868824078	0.003692626953125	\\
0.418209259999112	0.003692626953125	\\
0.418253651174147	0.0037841796875	\\
0.418298042349181	0.0037841796875	\\
0.418342433524215	0.003875732421875	\\
0.41838682469925	0.00384521484375	\\
0.418431215874284	0.0035400390625	\\
0.418475607049319	0.003662109375	\\
0.418519998224353	0.00390625	\\
0.418564389399387	0.003875732421875	\\
0.418608780574422	0.003662109375	\\
0.418653171749456	0.0035400390625	\\
0.418697562924491	0.0037841796875	\\
0.418741954099525	0.003692626953125	\\
0.418786345274559	0.003173828125	\\
0.418830736449594	0.003387451171875	\\
0.418875127624628	0.00341796875	\\
0.418919518799663	0.0032958984375	\\
0.418963909974697	0.003509521484375	\\
0.419008301149731	0.00347900390625	\\
0.419052692324766	0.003692626953125	\\
0.4190970834998	0.00347900390625	\\
0.419141474674835	0.003173828125	\\
0.419185865849869	0.00323486328125	\\
0.419230257024903	0.003448486328125	\\
0.419274648199938	0.00335693359375	\\
0.419319039374972	0.003509521484375	\\
0.419363430550007	0.003753662109375	\\
0.419407821725041	0.003448486328125	\\
0.419452212900075	0.003662109375	\\
0.41949660407511	0.003387451171875	\\
0.419540995250144	0.003753662109375	\\
0.419585386425179	0.00408935546875	\\
0.419629777600213	0.003692626953125	\\
0.419674168775247	0.003662109375	\\
0.419718559950282	0.00347900390625	\\
0.419762951125316	0.003692626953125	\\
0.419807342300351	0.00390625	\\
0.419851733475385	0.00408935546875	\\
0.419896124650419	0.00439453125	\\
0.419940515825454	0.0042724609375	\\
0.419984907000488	0.004669189453125	\\
0.420029298175523	0.00469970703125	\\
0.420073689350557	0.004608154296875	\\
0.420118080525592	0.004547119140625	\\
0.420162471700626	0.004547119140625	\\
0.42020686287566	0.004669189453125	\\
0.420251254050695	0.004608154296875	\\
0.420295645225729	0.00457763671875	\\
0.420340036400764	0.0047607421875	\\
0.420384427575798	0.0048828125	\\
0.420428818750832	0.004913330078125	\\
0.420473209925867	0.005096435546875	\\
0.420517601100901	0.004791259765625	\\
0.420561992275936	0.004425048828125	\\
0.42060638345097	0.0047607421875	\\
0.420650774626004	0.0050048828125	\\
0.420695165801039	0.00531005859375	\\
0.420739556976073	0.004913330078125	\\
0.420783948151108	0.00482177734375	\\
0.420828339326142	0.004913330078125	\\
0.420872730501176	0.005096435546875	\\
0.420917121676211	0.00506591796875	\\
0.420961512851245	0.004791259765625	\\
0.42100590402628	0.00469970703125	\\
0.421050295201314	0.0042724609375	\\
0.421094686376348	0.004730224609375	\\
0.421139077551383	0.00421142578125	\\
0.421183468726417	0.00384521484375	\\
0.421227859901452	0.003631591796875	\\
0.421272251076486	0.00347900390625	\\
0.42131664225152	0.003448486328125	\\
0.421361033426555	0.003997802734375	\\
0.421405424601589	0.004058837890625	\\
0.421449815776624	0.00390625	\\
0.421494206951658	0.00439453125	\\
0.421538598126692	0.004669189453125	\\
0.421582989301727	0.005462646484375	\\
0.421627380476761	0.005340576171875	\\
0.421671771651796	0.0048828125	\\
0.42171616282683	0.0048828125	\\
0.421760554001864	0.0048828125	\\
0.421804945176899	0.00482177734375	\\
0.421849336351933	0.004638671875	\\
0.421893727526968	0.0045166015625	\\
0.421938118702002	0.004913330078125	\\
0.421982509877036	0.0050048828125	\\
0.422026901052071	0.0047607421875	\\
0.422071292227105	0.004974365234375	\\
0.42211568340214	0.0050048828125	\\
0.422160074577174	0.005096435546875	\\
0.422204465752208	0.0054931640625	\\
0.422248856927243	0.005096435546875	\\
0.422293248102277	0.005096435546875	\\
0.422337639277312	0.005157470703125	\\
0.422382030452346	0.004791259765625	\\
0.42242642162738	0.004608154296875	\\
0.422470812802415	0.004730224609375	\\
0.422515203977449	0.004608154296875	\\
0.422559595152484	0.00439453125	\\
0.422603986327518	0.004241943359375	\\
0.422648377502553	0.004669189453125	\\
0.422692768677587	0.004364013671875	\\
0.422737159852621	0.004364013671875	\\
0.422781551027656	0.004241943359375	\\
0.42282594220269	0.0040283203125	\\
0.422870333377725	0.003875732421875	\\
0.422914724552759	0.003082275390625	\\
0.422959115727793	0.0030517578125	\\
0.423003506902828	0.0029296875	\\
0.423047898077862	0.00286865234375	\\
0.423092289252897	0.00299072265625	\\
0.423136680427931	0.00250244140625	\\
0.423181071602965	0.0023193359375	\\
0.423225462778	0.002471923828125	\\
0.423269853953034	0.00238037109375	\\
0.423314245128069	0.0023193359375	\\
0.423358636303103	0.002166748046875	\\
0.423403027478137	0.0025634765625	\\
0.423447418653172	0.003173828125	\\
0.423491809828206	0.002685546875	\\
0.423536201003241	0.00250244140625	\\
0.423580592178275	0.00286865234375	\\
0.423624983353309	0.002685546875	\\
0.423669374528344	0.00286865234375	\\
0.423713765703378	0.00286865234375	\\
0.423758156878413	0.002899169921875	\\
0.423802548053447	0.003021240234375	\\
0.423846939228481	0.002960205078125	\\
0.423891330403516	0.003082275390625	\\
0.42393572157855	0.003082275390625	\\
0.423980112753585	0.003265380859375	\\
0.424024503928619	0.003662109375	\\
0.424068895103653	0.00372314453125	\\
0.424113286278688	0.003692626953125	\\
0.424157677453722	0.00347900390625	\\
0.424202068628757	0.003570556640625	\\
0.424246459803791	0.0040283203125	\\
0.424290850978825	0.0042724609375	\\
0.42433524215386	0.003692626953125	\\
0.424379633328894	0.003326416015625	\\
0.424424024503929	0.00335693359375	\\
0.424468415678963	0.003631591796875	\\
0.424512806853997	0.003631591796875	\\
0.424557198029032	0.003448486328125	\\
0.424601589204066	0.003570556640625	\\
0.424645980379101	0.00341796875	\\
0.424690371554135	0.003448486328125	\\
0.424734762729169	0.003936767578125	\\
0.424779153904204	0.003448486328125	\\
0.424823545079238	0.00286865234375	\\
0.424867936254273	0.0028076171875	\\
0.424912327429307	0.003143310546875	\\
0.424956718604341	0.003448486328125	\\
0.425001109779376	0.003326416015625	\\
0.42504550095441	0.003753662109375	\\
0.425089892129445	0.003753662109375	\\
0.425134283304479	0.003326416015625	\\
0.425178674479513	0.00372314453125	\\
0.425223065654548	0.003448486328125	\\
0.425267456829582	0.00323486328125	\\
0.425311848004617	0.00347900390625	\\
0.425356239179651	0.003173828125	\\
0.425400630354686	0.003265380859375	\\
0.42544502152972	0.003448486328125	\\
0.425489412704754	0.00390625	\\
0.425533803879789	0.003997802734375	\\
0.425578195054823	0.004180908203125	\\
0.425622586229857	0.003997802734375	\\
0.425666977404892	0.003875732421875	\\
0.425711368579926	0.0040283203125	\\
0.425755759754961	0.003936767578125	\\
0.425800150929995	0.00439453125	\\
0.42584454210503	0.0040283203125	\\
0.425888933280064	0.004180908203125	\\
0.425933324455098	0.004669189453125	\\
0.425977715630133	0.00457763671875	\\
0.426022106805167	0.004730224609375	\\
0.426066497980202	0.004547119140625	\\
0.426110889155236	0.004669189453125	\\
0.42615528033027	0.004608154296875	\\
0.426199671505305	0.00396728515625	\\
0.426244062680339	0.00408935546875	\\
0.426288453855374	0.004241943359375	\\
0.426332845030408	0.00433349609375	\\
0.426377236205442	0.004150390625	\\
0.426421627380477	0.00390625	\\
0.426466018555511	0.003509521484375	\\
0.426510409730546	0.00347900390625	\\
0.42655480090558	0.003143310546875	\\
0.426599192080614	0.00341796875	\\
0.426643583255649	0.003509521484375	\\
0.426687974430683	0.002838134765625	\\
0.426732365605718	0.002960205078125	\\
0.426776756780752	0.0030517578125	\\
0.426821147955786	0.002960205078125	\\
0.426865539130821	0.0029296875	\\
0.426909930305855	0.002471923828125	\\
0.42695432148089	0.002166748046875	\\
0.426998712655924	0.002227783203125	\\
0.427043103830958	0.002044677734375	\\
0.427087495005993	0.001953125	\\
0.427131886181027	0.001922607421875	\\
0.427176277356062	0.002655029296875	\\
0.427220668531096	0.002716064453125	\\
0.42726505970613	0.002227783203125	\\
0.427309450881165	0.001983642578125	\\
0.427353842056199	0.00225830078125	\\
0.427398233231234	0.002288818359375	\\
0.427442624406268	0.00244140625	\\
0.427487015581302	0.00274658203125	\\
0.427531406756337	0.00286865234375	\\
0.427575797931371	0.003173828125	\\
0.427620189106406	0.00323486328125	\\
0.42766458028144	0.003265380859375	\\
0.427708971456474	0.003265380859375	\\
0.427753362631509	0.00347900390625	\\
0.427797753806543	0.00323486328125	\\
0.427842144981578	0.003204345703125	\\
0.427886536156612	0.0035400390625	\\
0.427930927331646	0.003326416015625	\\
0.427975318506681	0.003326416015625	\\
0.428019709681715	0.003662109375	\\
0.42806410085675	0.003936767578125	\\
0.428108492031784	0.00341796875	\\
0.428152883206818	0.00341796875	\\
0.428197274381853	0.00360107421875	\\
0.428241665556887	0.003204345703125	\\
0.428286056731922	0.003021240234375	\\
0.428330447906956	0.0032958984375	\\
0.42837483908199	0.00372314453125	\\
0.428419230257025	0.003936767578125	\\
0.428463621432059	0.003814697265625	\\
0.428508012607094	0.003692626953125	\\
0.428552403782128	0.0032958984375	\\
0.428596794957163	0.00286865234375	\\
0.428641186132197	0.00286865234375	\\
0.428685577307231	0.003143310546875	\\
0.428729968482266	0.00323486328125	\\
0.4287743596573	0.0030517578125	\\
0.428818750832335	0.002685546875	\\
0.428863142007369	0.00274658203125	\\
0.428907533182403	0.002899169921875	\\
0.428951924357438	0.002655029296875	\\
0.428996315532472	0.00238037109375	\\
0.429040706707507	0.00250244140625	\\
0.429085097882541	0.00238037109375	\\
0.429129489057575	0.00244140625	\\
0.42917388023261	0.002227783203125	\\
0.429218271407644	0.001739501953125	\\
0.429262662582679	0.00225830078125	\\
0.429307053757713	0.0023193359375	\\
0.429351444932747	0.002197265625	\\
0.429395836107782	0.002716064453125	\\
0.429440227282816	0.00262451171875	\\
0.429484618457851	0.0018310546875	\\
0.429529009632885	0.001617431640625	\\
0.429573400807919	0.001800537109375	\\
0.429617791982954	0.001861572265625	\\
0.429662183157988	0.001983642578125	\\
0.429706574333023	0.002227783203125	\\
0.429750965508057	0.00244140625	\\
0.429795356683091	0.0023193359375	\\
0.429839747858126	0.00238037109375	\\
0.42988413903316	0.002105712890625	\\
0.429928530208195	0.002410888671875	\\
0.429972921383229	0.002288818359375	\\
0.430017312558263	0.002288818359375	\\
0.430061703733298	0.00225830078125	\\
0.430106094908332	0.001861572265625	\\
0.430150486083367	0.00213623046875	\\
0.430194877258401	0.0023193359375	\\
0.430239268433435	0.0020751953125	\\
0.43028365960847	0.002166748046875	\\
0.430328050783504	0.002288818359375	\\
0.430372441958539	0.001800537109375	\\
0.430416833133573	0.001617431640625	\\
0.430461224308607	0.001312255859375	\\
0.430505615483642	0.001556396484375	\\
0.430550006658676	0.001678466796875	\\
0.430594397833711	0.001373291015625	\\
0.430638789008745	0.0013427734375	\\
0.430683180183779	0.00115966796875	\\
0.430727571358814	0.00079345703125	\\
0.430771962533848	0.0009765625	\\
0.430816353708883	0.0009765625	\\
0.430860744883917	0.00091552734375	\\
0.430905136058951	0.0006103515625	\\
0.430949527233986	0.0006103515625	\\
0.43099391840902	0.000396728515625	\\
0.431038309584055	0.00042724609375	\\
0.431082700759089	0.000579833984375	\\
0.431127091934124	0.00079345703125	\\
0.431171483109158	0.00103759765625	\\
0.431215874284192	0.000946044921875	\\
0.431260265459227	0.0010986328125	\\
0.431304656634261	0.001007080078125	\\
0.431349047809296	0.000701904296875	\\
0.43139343898433	0.000701904296875	\\
0.431437830159364	0.000946044921875	\\
0.431482221334399	0.0008544921875	\\
0.431526612509433	0.001068115234375	\\
0.431571003684468	0.001556396484375	\\
0.431615394859502	0.001556396484375	\\
0.431659786034536	0.001739501953125	\\
0.431704177209571	0.00201416015625	\\
0.431748568384605	0.00225830078125	\\
0.43179295955964	0.00238037109375	\\
0.431837350734674	0.002349853515625	\\
0.431881741909708	0.002685546875	\\
0.431926133084743	0.00262451171875	\\
0.431970524259777	0.00250244140625	\\
0.432014915434812	0.002960205078125	\\
0.432059306609846	0.0029296875	\\
0.43210369778488	0.00311279296875	\\
0.432148088959915	0.003204345703125	\\
0.432192480134949	0.003082275390625	\\
0.432236871309984	0.00286865234375	\\
0.432281262485018	0.003173828125	\\
0.432325653660052	0.00323486328125	\\
0.432370044835087	0.002777099609375	\\
0.432414436010121	0.002960205078125	\\
0.432458827185156	0.002716064453125	\\
0.43250321836019	0.002716064453125	\\
0.432547609535224	0.003082275390625	\\
0.432592000710259	0.002899169921875	\\
0.432636391885293	0.002685546875	\\
0.432680783060328	0.002410888671875	\\
0.432725174235362	0.002105712890625	\\
0.432769565410396	0.002166748046875	\\
0.432813956585431	0.002105712890625	\\
0.432858347760465	0.002044677734375	\\
0.4329027389355	0.00244140625	\\
0.432947130110534	0.00238037109375	\\
0.432991521285568	0.001739501953125	\\
0.433035912460603	0.0013427734375	\\
0.433080303635637	0.001495361328125	\\
0.433124694810672	0.001495361328125	\\
0.433169085985706	0.00152587890625	\\
0.43321347716074	0.001739501953125	\\
0.433257868335775	0.00213623046875	\\
0.433302259510809	0.002349853515625	\\
0.433346650685844	0.002593994140625	\\
0.433391041860878	0.0025634765625	\\
0.433435433035912	0.0023193359375	\\
0.433479824210947	0.002655029296875	\\
0.433524215385981	0.0029296875	\\
0.433568606561016	0.0029296875	\\
0.43361299773605	0.003173828125	\\
0.433657388911084	0.00335693359375	\\
0.433701780086119	0.0037841796875	\\
0.433746171261153	0.003936767578125	\\
0.433790562436188	0.003326416015625	\\
0.433834953611222	0.003265380859375	\\
0.433879344786257	0.003265380859375	\\
0.433923735961291	0.002960205078125	\\
0.433968127136325	0.0028076171875	\\
0.43401251831136	0.0030517578125	\\
0.434056909486394	0.0029296875	\\
0.434101300661428	0.003265380859375	\\
0.434145691836463	0.003387451171875	\\
0.434190083011497	0.003082275390625	\\
0.434234474186532	0.00299072265625	\\
0.434278865361566	0.003143310546875	\\
0.434323256536601	0.003143310546875	\\
0.434367647711635	0.00311279296875	\\
0.434412038886669	0.002960205078125	\\
0.434456430061704	0.002960205078125	\\
0.434500821236738	0.002838134765625	\\
0.434545212411773	0.00311279296875	\\
0.434589603586807	0.003204345703125	\\
0.434633994761841	0.00299072265625	\\
0.434678385936876	0.003021240234375	\\
0.43472277711191	0.00286865234375	\\
0.434767168286945	0.003387451171875	\\
0.434811559461979	0.0028076171875	\\
0.434855950637013	0.002471923828125	\\
0.434900341812048	0.002593994140625	\\
0.434944732987082	0.0025634765625	\\
0.434989124162117	0.002777099609375	\\
0.435033515337151	0.002838134765625	\\
0.435077906512185	0.002838134765625	\\
0.43512229768722	0.002471923828125	\\
0.435166688862254	0.002349853515625	\\
0.435211080037289	0.0025634765625	\\
0.435255471212323	0.002593994140625	\\
0.435299862387357	0.00250244140625	\\
0.435344253562392	0.002838134765625	\\
0.435388644737426	0.002899169921875	\\
0.435433035912461	0.0030517578125	\\
0.435477427087495	0.003021240234375	\\
0.435521818262529	0.003265380859375	\\
0.435566209437564	0.00360107421875	\\
0.435610600612598	0.00347900390625	\\
0.435654991787633	0.003753662109375	\\
0.435699382962667	0.00421142578125	\\
0.435743774137701	0.003936767578125	\\
0.435788165312736	0.0032958984375	\\
0.43583255648777	0.003814697265625	\\
0.435876947662805	0.00390625	\\
0.435921338837839	0.00390625	\\
0.435965730012873	0.0040283203125	\\
0.436010121187908	0.00396728515625	\\
0.436054512362942	0.00390625	\\
0.436098903537977	0.0037841796875	\\
0.436143294713011	0.003692626953125	\\
0.436187685888045	0.003936767578125	\\
0.43623207706308	0.00390625	\\
0.436276468238114	0.003814697265625	\\
0.436320859413149	0.00360107421875	\\
0.436365250588183	0.00347900390625	\\
0.436409641763217	0.00341796875	\\
0.436454032938252	0.0030517578125	\\
0.436498424113286	0.002655029296875	\\
0.436542815288321	0.002685546875	\\
0.436587206463355	0.003143310546875	\\
0.436631597638389	0.003021240234375	\\
0.436675988813424	0.002716064453125	\\
0.436720379988458	0.0025634765625	\\
0.436764771163493	0.00262451171875	\\
0.436809162338527	0.002197265625	\\
0.436853553513561	0.0020751953125	\\
0.436897944688596	0.002288818359375	\\
0.43694233586363	0.002197265625	\\
0.436986727038665	0.002197265625	\\
0.437031118213699	0.002105712890625	\\
0.437075509388734	0.002044677734375	\\
0.437119900563768	0.00201416015625	\\
0.437164291738802	0.00177001953125	\\
0.437208682913837	0.001739501953125	\\
0.437253074088871	0.001617431640625	\\
0.437297465263906	0.001373291015625	\\
0.43734185643894	0.001495361328125	\\
0.437386247613974	0.001556396484375	\\
0.437430638789009	0.00128173828125	\\
0.437475029964043	0.000946044921875	\\
0.437519421139078	0.001068115234375	\\
0.437563812314112	0.0013427734375	\\
0.437608203489146	0.00115966796875	\\
0.437652594664181	0.001190185546875	\\
0.437696985839215	0.001739501953125	\\
0.43774137701425	0.001739501953125	\\
0.437785768189284	0.001678466796875	\\
0.437830159364318	0.001617431640625	\\
0.437874550539353	0.00146484375	\\
0.437918941714387	0.00152587890625	\\
0.437963332889422	0.001708984375	\\
0.438007724064456	0.001861572265625	\\
0.43805211523949	0.001708984375	\\
0.438096506414525	0.002197265625	\\
0.438140897589559	0.001678466796875	\\
0.438185288764594	0.00140380859375	\\
0.438229679939628	0.00164794921875	\\
0.438274071114662	0.001708984375	\\
0.438318462289697	0.001617431640625	\\
0.438362853464731	0.00177001953125	\\
0.438407244639766	0.001556396484375	\\
0.4384516358148	0.001800537109375	\\
0.438496026989834	0.00201416015625	\\
0.438540418164869	0.001953125	\\
0.438584809339903	0.001678466796875	\\
0.438629200514938	0.001739501953125	\\
0.438673591689972	0.001434326171875	\\
0.438717982865006	0.001800537109375	\\
0.438762374040041	0.00201416015625	\\
0.438806765215075	0.001434326171875	\\
0.43885115639011	0.0013427734375	\\
0.438895547565144	0.00146484375	\\
0.438939938740178	0.0009765625	\\
0.438984329915213	0.001434326171875	\\
0.439028721090247	0.001617431640625	\\
0.439073112265282	0.001129150390625	\\
0.439117503440316	0.001312255859375	\\
0.43916189461535	0.001495361328125	\\
0.439206285790385	0.001373291015625	\\
0.439250676965419	0.00103759765625	\\
0.439295068140454	0.00091552734375	\\
0.439339459315488	0.000823974609375	\\
0.439383850490522	0.001129150390625	\\
0.439428241665557	0.0009765625	\\
0.439472632840591	0.0010986328125	\\
0.439517024015626	0.00140380859375	\\
0.43956141519066	0.0010986328125	\\
0.439605806365695	0.001007080078125	\\
0.439650197540729	0.000946044921875	\\
0.439694588715763	0.000701904296875	\\
0.439738979890798	0.000579833984375	\\
0.439783371065832	0.0009765625	\\
0.439827762240867	0.001220703125	\\
0.439872153415901	0.0013427734375	\\
0.439916544590935	0.00164794921875	\\
0.43996093576597	0.001495361328125	\\
0.440005326941004	0.001556396484375	\\
0.440049718116039	0.001983642578125	\\
0.440094109291073	0.00189208984375	\\
0.440138500466107	0.001953125	\\
0.440182891641142	0.002349853515625	\\
0.440227282816176	0.0023193359375	\\
0.440271673991211	0.002044677734375	\\
0.440316065166245	0.001953125	\\
0.440360456341279	0.00152587890625	\\
0.440404847516314	0.00164794921875	\\
0.440449238691348	0.00201416015625	\\
0.440493629866383	0.001861572265625	\\
0.440538021041417	0.001861572265625	\\
0.440582412216451	0.00140380859375	\\
0.440626803391486	0.00152587890625	\\
0.44067119456652	0.0018310546875	\\
0.440715585741555	0.001556396484375	\\
0.440759976916589	0.001251220703125	\\
0.440804368091623	0.001068115234375	\\
0.440848759266658	0.001495361328125	\\
0.440893150441692	0.00146484375	\\
0.440937541616727	0.00140380859375	\\
0.440981932791761	0.001678466796875	\\
0.441026323966795	0.00146484375	\\
0.44107071514183	0.00152587890625	\\
0.441115106316864	0.00146484375	\\
0.441159497491899	0.001373291015625	\\
0.441203888666933	0.001312255859375	\\
0.441248279841967	0.0013427734375	\\
0.441292671017002	0.0015869140625	\\
0.441337062192036	0.001800537109375	\\
0.441381453367071	0.0020751953125	\\
0.441425844542105	0.001800537109375	\\
0.441470235717139	0.001953125	\\
0.441514626892174	0.00213623046875	\\
0.441559018067208	0.00250244140625	\\
0.441603409242243	0.00250244140625	\\
0.441647800417277	0.002288818359375	\\
0.441692191592311	0.0028076171875	\\
0.441736582767346	0.003143310546875	\\
0.44178097394238	0.00250244140625	\\
0.441825365117415	0.0025634765625	\\
0.441869756292449	0.00323486328125	\\
0.441914147467483	0.00311279296875	\\
0.441958538642518	0.00274658203125	\\
0.442002929817552	0.002471923828125	\\
0.442047320992587	0.002410888671875	\\
0.442091712167621	0.002532958984375	\\
0.442136103342655	0.002471923828125	\\
0.44218049451769	0.002960205078125	\\
0.442224885692724	0.002838134765625	\\
0.442269276867759	0.00250244140625	\\
0.442313668042793	0.0028076171875	\\
0.442358059217828	0.003021240234375	\\
0.442402450392862	0.002899169921875	\\
0.442446841567896	0.002838134765625	\\
0.442491232742931	0.00299072265625	\\
0.442535623917965	0.00299072265625	\\
0.442580015092999	0.002777099609375	\\
0.442624406268034	0.0023193359375	\\
0.442668797443068	0.002227783203125	\\
0.442713188618103	0.00201416015625	\\
0.442757579793137	0.002197265625	\\
0.442801970968172	0.00189208984375	\\
0.442846362143206	0.0023193359375	\\
0.44289075331824	0.002349853515625	\\
0.442935144493275	0.00189208984375	\\
0.442979535668309	0.00225830078125	\\
0.443023926843344	0.00244140625	\\
0.443068318018378	0.0025634765625	\\
0.443112709193412	0.0023193359375	\\
0.443157100368447	0.002655029296875	\\
0.443201491543481	0.002960205078125	\\
0.443245882718516	0.002593994140625	\\
0.44329027389355	0.002471923828125	\\
0.443334665068584	0.002532958984375	\\
0.443379056243619	0.00244140625	\\
0.443423447418653	0.002655029296875	\\
0.443467838593688	0.002777099609375	\\
0.443512229768722	0.002593994140625	\\
0.443556620943756	0.00286865234375	\\
0.443601012118791	0.00274658203125	\\
0.443645403293825	0.002716064453125	\\
0.44368979446886	0.00323486328125	\\
0.443734185643894	0.00341796875	\\
0.443778576818928	0.003143310546875	\\
0.443822967993963	0.0025634765625	\\
0.443867359168997	0.00286865234375	\\
0.443911750344032	0.003082275390625	\\
0.443956141519066	0.00225830078125	\\
0.4440005326941	0.001983642578125	\\
0.444044923869135	0.00177001953125	\\
0.444089315044169	0.001373291015625	\\
0.444133706219204	0.0013427734375	\\
0.444178097394238	0.001708984375	\\
0.444222488569272	0.0015869140625	\\
0.444266879744307	0.00103759765625	\\
0.444311270919341	0.00140380859375	\\
0.444355662094376	0.001495361328125	\\
0.44440005326941	0.001312255859375	\\
0.444444444444444	0.0010986328125	\\
0.444488835619479	0.000640869140625	\\
0.444533226794513	0.0001220703125	\\
0.444577617969548	0.0001220703125	\\
0.444622009144582	0.000457763671875	\\
0.444666400319616	0.000701904296875	\\
0.444710791494651	0.0009765625	\\
0.444755182669685	0.001220703125	\\
0.44479957384472	0.00115966796875	\\
0.444843965019754	0.000518798828125	\\
0.444888356194789	0.000518798828125	\\
0.444932747369823	0.000732421875	\\
0.444977138544857	0.00042724609375	\\
0.445021529719892	0.0008544921875	\\
0.445065920894926	0.00128173828125	\\
0.44511031206996	0.000946044921875	\\
0.445154703244995	0.000823974609375	\\
0.445199094420029	0.001190185546875	\\
0.445243485595064	0.001373291015625	\\
0.445287876770098	0.001068115234375	\\
0.445332267945133	0.00140380859375	\\
0.445376659120167	0.001434326171875	\\
0.445421050295201	0.001556396484375	\\
0.445465441470236	0.00189208984375	\\
0.44550983264527	0.00146484375	\\
0.445554223820305	0.001251220703125	\\
0.445598614995339	0.00152587890625	\\
0.445643006170373	0.001434326171875	\\
0.445687397345408	0.00128173828125	\\
0.445731788520442	0.001068115234375	\\
0.445776179695477	0.000823974609375	\\
0.445820570870511	0.000885009765625	\\
0.445864962045545	0.00103759765625	\\
0.44590935322058	0.000885009765625	\\
0.445953744395614	0.0009765625	\\
0.445998135570649	0.0008544921875	\\
0.446042526745683	0.00067138671875	\\
0.446086917920717	0.0003662109375	\\
0.446131309095752	0.00079345703125	\\
0.446175700270786	0.00091552734375	\\
0.446220091445821	0.0009765625	\\
0.446264482620855	0.0008544921875	\\
0.446308873795889	0.000579833984375	\\
0.446353264970924	0	\\
0.446397656145958	-0.000213623046875	\\
0.446442047320993	-0.000152587890625	\\
0.446486438496027	9.1552734375e-05	\\
0.446530829671061	0.00030517578125	\\
0.446575220846096	0.00048828125	\\
0.44661961202113	0.00042724609375	\\
0.446664003196165	0.00042724609375	\\
0.446708394371199	0.000946044921875	\\
0.446752785546233	0.000885009765625	\\
0.446797176721268	0.00054931640625	\\
0.446841567896302	0.00030517578125	\\
0.446885959071337	-0.000244140625	\\
0.446930350246371	-0.000152587890625	\\
0.446974741421405	-3.0517578125e-05	\\
0.44701913259644	0	\\
0.447063523771474	-0.000152587890625	\\
0.447107914946509	0.0003662109375	\\
0.447152306121543	0.00091552734375	\\
0.447196697296577	0.000457763671875	\\
0.447241088471612	0.000274658203125	\\
0.447285479646646	0.00067138671875	\\
0.447329870821681	0.00030517578125	\\
0.447374261996715	3.0517578125e-05	\\
0.447418653171749	0.0001220703125	\\
0.447463044346784	0.000244140625	\\
0.447507435521818	-0.000244140625	\\
0.447551826696853	9.1552734375e-05	\\
0.447596217871887	0.000457763671875	\\
0.447640609046921	-0.000396728515625	\\
0.447685000221956	-0.000396728515625	\\
0.44772939139699	-0.000244140625	\\
0.447773782572025	0.000244140625	\\
0.447818173747059	0.000518798828125	\\
0.447862564922093	0.00018310546875	\\
0.447906956097128	0.0006103515625	\\
0.447951347272162	0.0003662109375	\\
0.447995738447197	9.1552734375e-05	\\
0.448040129622231	0.000213623046875	\\
0.448084520797266	-6.103515625e-05	\\
0.4481289119723	-9.1552734375e-05	\\
0.448173303147334	-0.000396728515625	\\
0.448217694322369	-0.000396728515625	\\
0.448262085497403	-0.0001220703125	\\
0.448306476672438	-0.0001220703125	\\
0.448350867847472	0	\\
0.448395259022506	0.000244140625	\\
0.448439650197541	6.103515625e-05	\\
0.448484041372575	-0.000457763671875	\\
0.44852843254761	-0.0009765625	\\
0.448572823722644	-0.000885009765625	\\
0.448617214897678	-0.0008544921875	\\
0.448661606072713	-0.000885009765625	\\
0.448705997247747	-0.0009765625	\\
0.448750388422782	-0.0009765625	\\
0.448794779597816	-0.000701904296875	\\
0.44883917077285	-0.000823974609375	\\
0.448883561947885	-0.00103759765625	\\
0.448927953122919	-0.000701904296875	\\
0.448972344297954	-0.000335693359375	\\
0.449016735472988	-0.00030517578125	\\
0.449061126648022	-0.00067138671875	\\
0.449105517823057	-0.00054931640625	\\
0.449149908998091	-0.000518798828125	\\
0.449194300173126	-0.00042724609375	\\
0.44923869134816	-9.1552734375e-05	\\
0.449283082523194	6.103515625e-05	\\
0.449327473698229	0.000152587890625	\\
0.449371864873263	6.103515625e-05	\\
0.449416256048298	-0.000213623046875	\\
0.449460647223332	-0.000213623046875	\\
0.449505038398366	3.0517578125e-05	\\
0.449549429573401	0.00018310546875	\\
0.449593820748435	0.000579833984375	\\
0.44963821192347	0.000732421875	\\
0.449682603098504	0.000244140625	\\
0.449726994273538	0.000213623046875	\\
0.449771385448573	0.0003662109375	\\
0.449815776623607	0	\\
0.449860167798642	0.000213623046875	\\
0.449904558973676	0.000274658203125	\\
0.44994895014871	0.00054931640625	\\
0.449993341323745	0.00054931640625	\\
0.450037732498779	0.000213623046875	\\
0.450082123673814	0.000396728515625	\\
0.450126514848848	0.00030517578125	\\
0.450170906023882	0.000152587890625	\\
0.450215297198917	0.000152587890625	\\
0.450259688373951	3.0517578125e-05	\\
0.450304079548986	-0.00048828125	\\
0.45034847072402	-0.000640869140625	\\
0.450392861899054	-0.00048828125	\\
0.450437253074089	-0.000732421875	\\
0.450481644249123	-0.00091552734375	\\
0.450526035424158	-0.0008544921875	\\
0.450570426599192	-0.001220703125	\\
0.450614817774226	-0.000762939453125	\\
0.450659208949261	-0.00067138671875	\\
0.450703600124295	-0.000701904296875	\\
0.45074799129933	-0.00030517578125	\\
0.450792382474364	-0.0003662109375	\\
0.450836773649399	-0.00048828125	\\
0.450881164824433	-0.00054931640625	\\
0.450925555999467	-0.00018310546875	\\
0.450969947174502	-0.0003662109375	\\
0.451014338349536	-0.000457763671875	\\
0.45105872952457	0.000213623046875	\\
0.451103120699605	0.00079345703125	\\
0.451147511874639	0.000946044921875	\\
0.451191903049674	0.0008544921875	\\
0.451236294224708	0.000701904296875	\\
0.451280685399743	0.00079345703125	\\
0.451325076574777	0.0013427734375	\\
0.451369467749811	0.001495361328125	\\
0.451413858924846	0.00140380859375	\\
0.45145825009988	0.001495361328125	\\
0.451502641274915	0.001434326171875	\\
0.451547032449949	0.001800537109375	\\
0.451591423624983	0.001678466796875	\\
0.451635814800018	0.00103759765625	\\
0.451680205975052	0.00177001953125	\\
0.451724597150087	0.0020751953125	\\
0.451768988325121	0.001983642578125	\\
0.451813379500155	0.001983642578125	\\
0.45185777067519	0.001708984375	\\
0.451902161850224	0.00177001953125	\\
0.451946553025259	0.00164794921875	\\
0.451990944200293	0.00177001953125	\\
0.452035335375327	0.001556396484375	\\
0.452079726550362	0.00146484375	\\
0.452124117725396	0.00146484375	\\
0.452168508900431	0.00128173828125	\\
0.452212900075465	0.000885009765625	\\
0.452257291250499	0.0006103515625	\\
0.452301682425534	0.000701904296875	\\
0.452346073600568	0.000457763671875	\\
0.452390464775603	0.00067138671875	\\
0.452434855950637	0.0008544921875	\\
0.452479247125671	0.000457763671875	\\
0.452523638300706	0.00048828125	\\
0.45256802947574	0.000640869140625	\\
0.452612420650775	0.000579833984375	\\
0.452656811825809	0.000762939453125	\\
0.452701203000843	0.000732421875	\\
0.452745594175878	0.00067138671875	\\
0.452789985350912	0.000274658203125	\\
0.452834376525947	0.00018310546875	\\
0.452878767700981	0.000885009765625	\\
0.452923158876015	0.000732421875	\\
0.45296755005105	0.000762939453125	\\
0.453011941226084	0.00079345703125	\\
0.453056332401119	0.00018310546875	\\
0.453100723576153	3.0517578125e-05	\\
0.453145114751187	0.000640869140625	\\
0.453189505926222	0.00067138671875	\\
0.453233897101256	0.001251220703125	\\
0.453278288276291	0.001708984375	\\
0.453322679451325	0.00177001953125	\\
0.45336707062636	0.0018310546875	\\
0.453411461801394	0.0013427734375	\\
0.453455852976428	0.000946044921875	\\
0.453500244151463	0.0006103515625	\\
0.453544635326497	0.00079345703125	\\
0.453589026501531	0.000885009765625	\\
0.453633417676566	0.000946044921875	\\
0.4536778088516	0.00079345703125	\\
0.453722200026635	0.00079345703125	\\
0.453766591201669	0.000946044921875	\\
0.453810982376704	0.000762939453125	\\
0.453855373551738	0.000823974609375	\\
0.453899764726772	0.001068115234375	\\
0.453944155901807	0.001007080078125	\\
0.453988547076841	0.00042724609375	\\
0.454032938251876	0.000244140625	\\
0.45407732942691	0.000274658203125	\\
0.454121720601944	0.000213623046875	\\
0.454166111776979	0.000396728515625	\\
0.454210502952013	0.000213623046875	\\
0.454254894127048	-6.103515625e-05	\\
0.454299285302082	-0.000274658203125	\\
0.454343676477116	-0.00042724609375	\\
0.454388067652151	-3.0517578125e-05	\\
0.454432458827185	-0.00042724609375	\\
0.45447685000222	-0.000762939453125	\\
0.454521241177254	-0.000396728515625	\\
0.454565632352288	-0.001007080078125	\\
0.454610023527323	-0.00146484375	\\
0.454654414702357	-0.001434326171875	\\
0.454698805877392	-0.0018310546875	\\
0.454743197052426	-0.001495361328125	\\
0.45478758822746	-0.00152587890625	\\
0.454831979402495	-0.001861572265625	\\
0.454876370577529	-0.00128173828125	\\
0.454920761752564	-0.000701904296875	\\
0.454965152927598	-0.000640869140625	\\
0.455009544102632	-0.001129150390625	\\
0.455053935277667	-0.001312255859375	\\
0.455098326452701	-0.0009765625	\\
0.455142717627736	-0.000640869140625	\\
0.45518710880277	-0.00103759765625	\\
0.455231499977804	-0.0010986328125	\\
0.455275891152839	-0.000732421875	\\
0.455320282327873	-0.000701904296875	\\
0.455364673502908	-0.000579833984375	\\
0.455409064677942	-0.000518798828125	\\
0.455453455852976	-0.000244140625	\\
0.455497847028011	-0.000457763671875	\\
0.455542238203045	-0.000732421875	\\
0.45558662937808	-0.000640869140625	\\
0.455631020553114	-0.000396728515625	\\
0.455675411728148	-0.000732421875	\\
0.455719802903183	-0.0009765625	\\
0.455764194078217	-0.000579833984375	\\
0.455808585253252	-0.00054931640625	\\
0.455852976428286	-0.000762939453125	\\
0.45589736760332	-0.0009765625	\\
0.455941758778355	-0.001129150390625	\\
0.455986149953389	-0.00091552734375	\\
0.456030541128424	-0.000244140625	\\
0.456074932303458	-0.000335693359375	\\
0.456119323478492	-0.000274658203125	\\
0.456163714653527	-0.000396728515625	\\
0.456208105828561	-0.000732421875	\\
0.456252497003596	-0.000274658203125	\\
0.45629688817863	-0.000335693359375	\\
0.456341279353664	-0.00048828125	\\
0.456385670528699	-0.000152587890625	\\
0.456430061703733	-0.000213623046875	\\
0.456474452878768	-6.103515625e-05	\\
0.456518844053802	-6.103515625e-05	\\
0.456563235228837	-0.000244140625	\\
0.456607626403871	9.1552734375e-05	\\
0.456652017578905	0.000396728515625	\\
0.45669640875394	0.00042724609375	\\
0.456740799928974	0.000335693359375	\\
0.456785191104009	-6.103515625e-05	\\
0.456829582279043	-0.000274658203125	\\
0.456873973454077	6.103515625e-05	\\
0.456918364629112	9.1552734375e-05	\\
0.456962755804146	-6.103515625e-05	\\
0.457007146979181	-9.1552734375e-05	\\
0.457051538154215	3.0517578125e-05	\\
0.457095929329249	-0.000213623046875	\\
0.457140320504284	-0.000518798828125	\\
0.457184711679318	-0.00042724609375	\\
0.457229102854353	-0.00048828125	\\
0.457273494029387	-0.00042724609375	\\
0.457317885204421	-0.0006103515625	\\
0.457362276379456	-0.00067138671875	\\
0.45740666755449	-0.00079345703125	\\
0.457451058729525	-0.000518798828125	\\
0.457495449904559	-0.0003662109375	\\
0.457539841079593	-0.00079345703125	\\
0.457584232254628	-0.00054931640625	\\
0.457628623429662	-0.00067138671875	\\
0.457673014604697	-0.000579833984375	\\
0.457717405779731	-0.00018310546875	\\
0.457761796954765	-0.0003662109375	\\
0.4578061881298	-0.000946044921875	\\
0.457850579304834	-0.0006103515625	\\
0.457894970479869	-0.000518798828125	\\
0.457939361654903	-0.00067138671875	\\
0.457983752829937	-0.000579833984375	\\
0.458028144004972	-0.000823974609375	\\
0.458072535180006	-0.001220703125	\\
0.458116926355041	-0.00103759765625	\\
0.458161317530075	-0.001190185546875	\\
0.458205708705109	-0.001007080078125	\\
0.458250099880144	-0.000762939453125	\\
0.458294491055178	-0.000885009765625	\\
0.458338882230213	-0.00067138671875	\\
0.458383273405247	-0.000518798828125	\\
0.458427664580281	-0.000640869140625	\\
0.458472055755316	-0.000518798828125	\\
0.45851644693035	-0.000518798828125	\\
0.458560838105385	-0.00067138671875	\\
0.458605229280419	-0.000762939453125	\\
0.458649620455453	-0.001007080078125	\\
0.458694011630488	-0.000640869140625	\\
0.458738402805522	-0.000732421875	\\
0.458782793980557	-0.00079345703125	\\
0.458827185155591	-0.000457763671875	\\
0.458871576330625	-0.0003662109375	\\
0.45891596750566	-3.0517578125e-05	\\
0.458960358680694	-9.1552734375e-05	\\
0.459004749855729	-0.0001220703125	\\
0.459049141030763	0.00048828125	\\
0.459093532205797	0.0003662109375	\\
0.459137923380832	-0.000244140625	\\
0.459182314555866	9.1552734375e-05	\\
0.459226705730901	0.000152587890625	\\
0.459271096905935	3.0517578125e-05	\\
0.45931548808097	-6.103515625e-05	\\
0.459359879256004	-0.00018310546875	\\
0.459404270431038	0.00030517578125	\\
0.459448661606073	0.000396728515625	\\
0.459493052781107	0.000579833984375	\\
0.459537443956142	0.00030517578125	\\
0.459581835131176	0.00018310546875	\\
0.45962622630621	0.000579833984375	\\
0.459670617481245	0.000244140625	\\
0.459715008656279	0.00018310546875	\\
0.459759399831314	0.000274658203125	\\
0.459803791006348	0.000152587890625	\\
0.459848182181382	-0.00030517578125	\\
0.459892573356417	-0.000579833984375	\\
0.459936964531451	-0.0003662109375	\\
0.459981355706486	-0.0003662109375	\\
0.46002574688152	-0.00091552734375	\\
0.460070138056554	-0.00079345703125	\\
0.460114529231589	-0.000640869140625	\\
0.460158920406623	-0.00048828125	\\
0.460203311581658	-0.000457763671875	\\
0.460247702756692	-0.000579833984375	\\
0.460292093931726	-0.001068115234375	\\
0.460336485106761	-0.0013427734375	\\
0.460380876281795	-0.001190185546875	\\
0.46042526745683	-0.001373291015625	\\
0.460469658631864	-0.00103759765625	\\
0.460514049806898	-0.000701904296875	\\
0.460558440981933	-0.00067138671875	\\
0.460602832156967	-0.000762939453125	\\
0.460647223332002	-0.000762939453125	\\
0.460691614507036	-0.000640869140625	\\
0.46073600568207	-0.000579833984375	\\
0.460780396857105	-6.103515625e-05	\\
0.460824788032139	-3.0517578125e-05	\\
0.460869179207174	-0.00018310546875	\\
0.460913570382208	-0.000152587890625	\\
0.460957961557242	-0.0001220703125	\\
0.461002352732277	6.103515625e-05	\\
0.461046743907311	0.000152587890625	\\
0.461091135082346	-6.103515625e-05	\\
0.46113552625738	-0.000518798828125	\\
0.461179917432414	-0.000396728515625	\\
0.461224308607449	-0.00048828125	\\
0.461268699782483	-0.0009765625	\\
0.461313090957518	-0.00067138671875	\\
0.461357482132552	-0.0001220703125	\\
0.461401873307586	-0.000457763671875	\\
0.461446264482621	-0.000396728515625	\\
0.461490655657655	-0.0003662109375	\\
0.46153504683269	-0.000274658203125	\\
0.461579438007724	-0.0003662109375	\\
0.461623829182758	-0.00030517578125	\\
0.461668220357793	-0.0001220703125	\\
0.461712611532827	-0.000518798828125	\\
0.461757002707862	-0.00042724609375	\\
0.461801393882896	-0.0003662109375	\\
0.461845785057931	-0.000457763671875	\\
0.461890176232965	-0.000579833984375	\\
0.461934567407999	-0.000701904296875	\\
0.461978958583034	-0.00067138671875	\\
0.462023349758068	-0.000732421875	\\
0.462067740933102	-0.000457763671875	\\
0.462112132108137	-0.000457763671875	\\
0.462156523283171	-0.0003662109375	\\
0.462200914458206	-0.000457763671875	\\
0.46224530563324	-0.000518798828125	\\
0.462289696808275	-0.0006103515625	\\
0.462334087983309	-0.000640869140625	\\
0.462378479158343	-0.000732421875	\\
0.462422870333378	-0.000640869140625	\\
0.462467261508412	-0.00042724609375	\\
0.462511652683447	-0.000579833984375	\\
0.462556043858481	-0.00103759765625	\\
0.462600435033515	-0.001373291015625	\\
0.46264482620855	-0.001312255859375	\\
0.462689217383584	-0.001251220703125	\\
0.462733608558619	-0.000885009765625	\\
0.462777999733653	-0.001007080078125	\\
0.462822390908687	-0.000823974609375	\\
0.462866782083722	-0.00048828125	\\
0.462911173258756	-0.000701904296875	\\
0.462955564433791	-0.00115966796875	\\
0.462999955608825	-0.001190185546875	\\
0.463044346783859	-0.000701904296875	\\
0.463088737958894	-0.0006103515625	\\
0.463133129133928	-0.000701904296875	\\
0.463177520308963	-0.00079345703125	\\
0.463221911483997	-0.0009765625	\\
0.463266302659031	-0.000732421875	\\
0.463310693834066	-0.00067138671875	\\
0.4633550850091	-0.000701904296875	\\
0.463399476184135	-0.00067138671875	\\
0.463443867359169	-0.00054931640625	\\
0.463488258534203	-0.000823974609375	\\
0.463532649709238	-0.00103759765625	\\
0.463577040884272	-0.00067138671875	\\
0.463621432059307	-0.00067138671875	\\
0.463665823234341	-0.000732421875	\\
0.463710214409375	-0.0008544921875	\\
0.46375460558441	-0.001068115234375	\\
0.463798996759444	-0.001129150390625	\\
0.463843387934479	-0.0010986328125	\\
0.463887779109513	-0.001312255859375	\\
0.463932170284547	-0.001312255859375	\\
0.463976561459582	-0.00103759765625	\\
0.464020952634616	-0.00091552734375	\\
0.464065343809651	-0.0008544921875	\\
0.464109734984685	-0.001068115234375	\\
0.464154126159719	-0.001129150390625	\\
0.464198517334754	-0.00128173828125	\\
0.464242908509788	-0.001556396484375	\\
0.464287299684823	-0.00152587890625	\\
0.464331690859857	-0.00115966796875	\\
0.464376082034891	-0.001495361328125	\\
0.464420473209926	-0.001556396484375	\\
0.46446486438496	-0.001312255859375	\\
0.464509255559995	-0.001495361328125	\\
0.464553646735029	-0.00152587890625	\\
0.464598037910063	-0.001434326171875	\\
0.464642429085098	-0.001007080078125	\\
0.464686820260132	-0.001129150390625	\\
0.464731211435167	-0.00115966796875	\\
0.464775602610201	-0.000762939453125	\\
0.464819993785235	-0.0010986328125	\\
0.46486438496027	-0.001007080078125	\\
0.464908776135304	-0.0006103515625	\\
0.464953167310339	-0.00079345703125	\\
0.464997558485373	-0.00079345703125	\\
0.465041949660408	-0.000579833984375	\\
0.465086340835442	-0.00054931640625	\\
0.465130732010476	-0.000518798828125	\\
0.465175123185511	-0.000244140625	\\
0.465219514360545	-0.00018310546875	\\
0.46526390553558	0.000152587890625	\\
0.465308296710614	0.000152587890625	\\
0.465352687885648	-0.0001220703125	\\
0.465397079060683	0.00018310546875	\\
0.465441470235717	-3.0517578125e-05	\\
0.465485861410752	-0.000335693359375	\\
0.465530252585786	-0.000457763671875	\\
0.46557464376082	-0.000823974609375	\\
0.465619034935855	-0.000946044921875	\\
0.465663426110889	-0.0006103515625	\\
0.465707817285924	-0.000457763671875	\\
0.465752208460958	-0.000457763671875	\\
0.465796599635992	-0.000213623046875	\\
0.465840990811027	-0.000396728515625	\\
0.465885381986061	-0.001068115234375	\\
0.465929773161096	-0.001190185546875	\\
0.46597416433613	-0.000762939453125	\\
0.466018555511164	-0.000823974609375	\\
0.466062946686199	-0.0009765625	\\
0.466107337861233	-0.001220703125	\\
0.466151729036268	-0.001220703125	\\
0.466196120211302	-0.00103759765625	\\
0.466240511386336	-0.00128173828125	\\
0.466284902561371	-0.001190185546875	\\
0.466329293736405	-0.00146484375	\\
0.46637368491144	-0.001617431640625	\\
0.466418076086474	-0.001373291015625	\\
0.466462467261508	-0.001708984375	\\
0.466506858436543	-0.0018310546875	\\
0.466551249611577	-0.0015869140625	\\
0.466595640786612	-0.00140380859375	\\
0.466640031961646	-0.00128173828125	\\
0.46668442313668	-0.0010986328125	\\
0.466728814311715	-0.00079345703125	\\
0.466773205486749	-0.00042724609375	\\
0.466817596661784	-0.000213623046875	\\
0.466861987836818	0.000244140625	\\
0.466906379011852	0.000701904296875	\\
0.466950770186887	0.00042724609375	\\
0.466995161361921	0.000732421875	\\
0.467039552536956	0.0009765625	\\
0.46708394371199	0.00079345703125	\\
0.467128334887024	0.001129150390625	\\
0.467172726062059	0.001068115234375	\\
0.467217117237093	0.000885009765625	\\
0.467261508412128	0.00103759765625	\\
0.467305899587162	0.001373291015625	\\
0.467350290762196	0.0015869140625	\\
0.467394681937231	0.00164794921875	\\
0.467439073112265	0.001678466796875	\\
0.4674834642873	0.00164794921875	\\
0.467527855462334	0.00152587890625	\\
0.467572246637369	0.001251220703125	\\
0.467616637812403	0.00128173828125	\\
0.467661028987437	0.001129150390625	\\
0.467705420162472	0.00054931640625	\\
0.467749811337506	0.000518798828125	\\
0.467794202512541	0.001129150390625	\\
0.467838593687575	0.0010986328125	\\
0.467882984862609	0.000579833984375	\\
0.467927376037644	0.00054931640625	\\
0.467971767212678	0.00091552734375	\\
0.468016158387713	0.001129150390625	\\
0.468060549562747	0.0003662109375	\\
0.468104940737781	0.00067138671875	\\
0.468149331912816	0.000518798828125	\\
0.46819372308785	-9.1552734375e-05	\\
0.468238114262885	0.000244140625	\\
0.468282505437919	-0.000701904296875	\\
0.468326896612953	-0.000518798828125	\\
0.468371287787988	-0.000152587890625	\\
0.468415678963022	-0.000701904296875	\\
0.468460070138057	-0.00018310546875	\\
0.468504461313091	0.0001220703125	\\
0.468548852488125	0	\\
0.46859324366316	9.1552734375e-05	\\
0.468637634838194	-3.0517578125e-05	\\
0.468682026013229	0	\\
0.468726417188263	-0.0003662109375	\\
0.468770808363297	-0.0006103515625	\\
0.468815199538332	-0.000640869140625	\\
0.468859590713366	-0.000518798828125	\\
0.468903981888401	-0.00054931640625	\\
0.468948373063435	-0.0001220703125	\\
0.468992764238469	-0.00030517578125	\\
0.469037155413504	-0.00054931640625	\\
0.469081546588538	-0.0003662109375	\\
0.469125937763573	-0.00067138671875	\\
0.469170328938607	-0.00054931640625	\\
0.469214720113641	0	\\
0.469259111288676	-0.00018310546875	\\
0.46930350246371	-0.000640869140625	\\
0.469347893638745	-0.001068115234375	\\
0.469392284813779	-0.001739501953125	\\
0.469436675988813	-0.00177001953125	\\
0.469481067163848	-0.00140380859375	\\
0.469525458338882	-0.001708984375	\\
0.469569849513917	-0.001556396484375	\\
0.469614240688951	-0.001556396484375	\\
0.469658631863985	-0.001617431640625	\\
0.46970302303902	-0.001220703125	\\
0.469747414214054	-0.00128173828125	\\
0.469791805389089	-0.001495361328125	\\
0.469836196564123	-0.001708984375	\\
0.469880587739157	-0.001739501953125	\\
0.469924978914192	-0.001678466796875	\\
0.469969370089226	-0.001678466796875	\\
0.470013761264261	-0.001922607421875	\\
0.470058152439295	-0.001678466796875	\\
0.470102543614329	-0.001708984375	\\
0.470146934789364	-0.001983642578125	\\
0.470191325964398	-0.001678466796875	\\
0.470235717139433	-0.001678466796875	\\
0.470280108314467	-0.001739501953125	\\
0.470324499489502	-0.001617431640625	\\
0.470368890664536	-0.001617431640625	\\
0.47041328183957	-0.0015869140625	\\
0.470457673014605	-0.00128173828125	\\
0.470502064189639	-0.000732421875	\\
0.470546455364673	-0.000701904296875	\\
0.470590846539708	-0.00042724609375	\\
0.470635237714742	-0.000732421875	\\
0.470679628889777	-0.00067138671875	\\
0.470724020064811	-9.1552734375e-05	\\
0.470768411239846	-0.000640869140625	\\
0.47081280241488	-0.0001220703125	\\
0.470857193589914	0.00018310546875	\\
0.470901584764949	0.0001220703125	\\
0.470945975939983	0.000335693359375	\\
0.470990367115018	0.00030517578125	\\
0.471034758290052	0.000244140625	\\
0.471079149465086	-0.000244140625	\\
0.471123540640121	-9.1552734375e-05	\\
0.471167931815155	-6.103515625e-05	\\
0.47121232299019	-0.000457763671875	\\
0.471256714165224	-0.000701904296875	\\
0.471301105340258	-0.000518798828125	\\
0.471345496515293	-0.000701904296875	\\
0.471389887690327	-0.000823974609375	\\
0.471434278865362	-0.00030517578125	\\
0.471478670040396	-0.000579833984375	\\
0.47152306121543	-0.000762939453125	\\
0.471567452390465	-0.001434326171875	\\
0.471611843565499	-0.001708984375	\\
0.471656234740534	-0.00152587890625	\\
0.471700625915568	-0.001708984375	\\
0.471745017090602	-0.0020751953125	\\
0.471789408265637	-0.0023193359375	\\
0.471833799440671	-0.00274658203125	\\
0.471878190615706	-0.00262451171875	\\
0.47192258179074	-0.00201416015625	\\
0.471966972965774	-0.00140380859375	\\
0.472011364140809	-0.0013427734375	\\
0.472055755315843	-0.001861572265625	\\
0.472100146490878	-0.00164794921875	\\
0.472144537665912	-0.001373291015625	\\
0.472188928840946	-0.001068115234375	\\
0.472233320015981	-0.001312255859375	\\
0.472277711191015	-0.00103759765625	\\
0.47232210236605	-0.000579833984375	\\
0.472366493541084	-0.000518798828125	\\
0.472410884716118	-0.00091552734375	\\
0.472455275891153	-0.000579833984375	\\
0.472499667066187	-0.000152587890625	\\
0.472544058241222	-0.00030517578125	\\
0.472588449416256	-0.000396728515625	\\
0.47263284059129	-0.000335693359375	\\
0.472677231766325	-0.0003662109375	\\
0.472721622941359	-0.000152587890625	\\
0.472766014116394	6.103515625e-05	\\
0.472810405291428	-0.00042724609375	\\
0.472854796466462	-0.00030517578125	\\
0.472899187641497	-0.00054931640625	\\
0.472943578816531	-0.000518798828125	\\
0.472987969991566	-0.00042724609375	\\
0.4730323611666	-0.000762939453125	\\
0.473076752341634	-0.000457763671875	\\
0.473121143516669	-0.0006103515625	\\
0.473165534691703	-0.0006103515625	\\
0.473209925866738	-0.00048828125	\\
0.473254317041772	-0.001007080078125	\\
0.473298708216806	-0.001068115234375	\\
0.473343099391841	-0.00079345703125	\\
0.473387490566875	-0.000823974609375	\\
0.47343188174191	-0.001312255859375	\\
0.473476272916944	-0.001617431640625	\\
0.473520664091979	-0.001678466796875	\\
0.473565055267013	-0.0018310546875	\\
0.473609446442047	-0.001251220703125	\\
0.473653837617082	-0.00140380859375	\\
0.473698228792116	-0.001220703125	\\
0.473742619967151	-0.0009765625	\\
0.473787011142185	-0.00103759765625	\\
0.473831402317219	-0.001007080078125	\\
0.473875793492254	-0.001007080078125	\\
0.473920184667288	-0.0006103515625	\\
0.473964575842323	-0.000701904296875	\\
0.474008967017357	-0.0009765625	\\
0.474053358192391	-0.000946044921875	\\
0.474097749367426	-0.000946044921875	\\
0.47414214054246	-0.0010986328125	\\
0.474186531717495	-0.001373291015625	\\
0.474230922892529	-0.00091552734375	\\
0.474275314067563	-0.00079345703125	\\
0.474319705242598	-0.00079345703125	\\
0.474364096417632	-0.001190185546875	\\
0.474408487592667	-0.000823974609375	\\
0.474452878767701	-0.00091552734375	\\
0.474497269942735	-0.001190185546875	\\
0.47454166111777	-0.001373291015625	\\
0.474586052292804	-0.00140380859375	\\
0.474630443467839	-0.001007080078125	\\
0.474674834642873	-0.00091552734375	\\
0.474719225817907	-0.00103759765625	\\
0.474763616992942	-0.00067138671875	\\
0.474808008167976	-0.000579833984375	\\
0.474852399343011	-0.000823974609375	\\
0.474896790518045	-0.000457763671875	\\
0.474941181693079	-0.000335693359375	\\
0.474985572868114	-0.00067138671875	\\
0.475029964043148	-0.0008544921875	\\
0.475074355218183	-0.0009765625	\\
0.475118746393217	-0.000823974609375	\\
0.475163137568251	-0.00079345703125	\\
0.475207528743286	-0.000518798828125	\\
0.47525191991832	-0.0001220703125	\\
0.475296311093355	-0.000396728515625	\\
0.475340702268389	-0.00054931640625	\\
0.475385093443423	-0.000640869140625	\\
0.475429484618458	-0.0006103515625	\\
0.475473875793492	-0.000732421875	\\
0.475518266968527	-0.00079345703125	\\
0.475562658143561	-0.000640869140625	\\
0.475607049318595	-0.000518798828125	\\
0.47565144049363	-0.000396728515625	\\
0.475695831668664	-0.00018310546875	\\
0.475740222843699	0.0003662109375	\\
0.475784614018733	0.000396728515625	\\
0.475829005193767	0.00048828125	\\
0.475873396368802	0.0006103515625	\\
0.475917787543836	0.00067138671875	\\
0.475962178718871	0.00067138671875	\\
0.476006569893905	0.0008544921875	\\
0.47605096106894	0.001068115234375	\\
0.476095352243974	0.00091552734375	\\
0.476139743419008	0.000823974609375	\\
0.476184134594043	0.001129150390625	\\
0.476228525769077	0.00103759765625	\\
0.476272916944112	0.001068115234375	\\
0.476317308119146	0.001129150390625	\\
0.47636169929418	0.001190185546875	\\
0.476406090469215	0.001495361328125	\\
0.476450481644249	0.001678466796875	\\
0.476494872819284	0.001312255859375	\\
0.476539263994318	0.0010986328125	\\
0.476583655169352	0.0008544921875	\\
0.476628046344387	0.001312255859375	\\
0.476672437519421	0.0013427734375	\\
0.476716828694456	0.0008544921875	\\
0.47676121986949	0.0010986328125	\\
0.476805611044524	0.001129150390625	\\
0.476850002219559	0.001190185546875	\\
0.476894393394593	0.001373291015625	\\
0.476938784569628	0.001708984375	\\
0.476983175744662	0.001739501953125	\\
0.477027566919696	0.00164794921875	\\
0.477071958094731	0.001373291015625	\\
0.477116349269765	0.00115966796875	\\
0.4771607404448	0.00128173828125	\\
0.477205131619834	0.0009765625	\\
0.477249522794868	0.000823974609375	\\
0.477293913969903	0.000457763671875	\\
0.477338305144937	0.000335693359375	\\
0.477382696319972	0.000152587890625	\\
0.477427087495006	-0.000213623046875	\\
0.47747147867004	-0.00048828125	\\
0.477515869845075	-0.000946044921875	\\
0.477560261020109	-0.000762939453125	\\
0.477604652195144	-0.00067138671875	\\
0.477649043370178	-0.000579833984375	\\
0.477693434545212	-0.000732421875	\\
0.477737825720247	-0.00091552734375	\\
0.477782216895281	-0.00091552734375	\\
0.477826608070316	-0.0008544921875	\\
0.47787099924535	-0.00128173828125	\\
0.477915390420384	-0.001129150390625	\\
0.477959781595419	-0.000701904296875	\\
0.478004172770453	-0.000823974609375	\\
0.478048563945488	-0.0010986328125	\\
0.478092955120522	-0.001251220703125	\\
0.478137346295556	-0.00146484375	\\
0.478181737470591	-0.0013427734375	\\
0.478226128645625	-0.00079345703125	\\
0.47827051982066	-0.000762939453125	\\
0.478314910995694	-0.0013427734375	\\
0.478359302170728	-0.001007080078125	\\
0.478403693345763	-0.00042724609375	\\
0.478448084520797	-0.000762939453125	\\
0.478492475695832	-0.000579833984375	\\
0.478536866870866	-0.00030517578125	\\
0.4785812580459	-0.00042724609375	\\
0.478625649220935	0	\\
0.478670040395969	0.000274658203125	\\
0.478714431571004	0.0003662109375	\\
0.478758822746038	0.00054931640625	\\
0.478803213921073	0.000244140625	\\
0.478847605096107	0.000152587890625	\\
0.478891996271141	0.00018310546875	\\
0.478936387446176	0.0001220703125	\\
0.47898077862121	0.0001220703125	\\
0.479025169796244	-0.000274658203125	\\
0.479069560971279	-0.000213623046875	\\
0.479113952146313	-0.00018310546875	\\
0.479158343321348	-0.00054931640625	\\
0.479202734496382	-0.000518798828125	\\
0.479247125671417	-0.0003662109375	\\
0.479291516846451	-0.000579833984375	\\
0.479335908021485	-0.000762939453125	\\
0.47938029919652	-0.001068115234375	\\
0.479424690371554	-0.0010986328125	\\
0.479469081546589	-0.001190185546875	\\
0.479513472721623	-0.00177001953125	\\
0.479557863896657	-0.00177001953125	\\
0.479602255071692	-0.00189208984375	\\
0.479646646246726	-0.00189208984375	\\
0.479691037421761	-0.0018310546875	\\
0.479735428596795	-0.002105712890625	\\
0.479779819771829	-0.00189208984375	\\
0.479824210946864	-0.00238037109375	\\
0.479868602121898	-0.00213623046875	\\
0.479912993296933	-0.0013427734375	\\
0.479957384471967	-0.00189208984375	\\
0.480001775647001	-0.00262451171875	\\
0.480046166822036	-0.002685546875	\\
0.48009055799707	-0.00250244140625	\\
0.480134949172105	-0.002655029296875	\\
0.480179340347139	-0.0025634765625	\\
0.480223731522173	-0.001983642578125	\\
0.480268122697208	-0.00225830078125	\\
0.480312513872242	-0.002166748046875	\\
0.480356905047277	-0.002288818359375	\\
0.480401296222311	-0.002166748046875	\\
0.480445687397345	-0.00201416015625	\\
0.48049007857238	-0.001495361328125	\\
0.480534469747414	-0.001495361328125	\\
0.480578860922449	-0.001312255859375	\\
0.480623252097483	-0.000579833984375	\\
0.480667643272517	-0.0009765625	\\
0.480712034447552	-0.000579833984375	\\
0.480756425622586	-0.0006103515625	\\
0.480800816797621	-0.0009765625	\\
0.480845207972655	-0.00042724609375	\\
0.480889599147689	-0.0003662109375	\\
0.480933990322724	-0.00079345703125	\\
0.480978381497758	-0.000701904296875	\\
0.481022772672793	-6.103515625e-05	\\
0.481067163847827	0.00048828125	\\
0.481111555022861	0.0006103515625	\\
0.481155946197896	0.00067138671875	\\
0.48120033737293	0.000244140625	\\
0.481244728547965	0.00042724609375	\\
0.481289119722999	0.000640869140625	\\
0.481333510898033	0.0006103515625	\\
0.481377902073068	0.000946044921875	\\
0.481422293248102	0.00091552734375	\\
0.481466684423137	0.001129150390625	\\
0.481511075598171	0.001556396484375	\\
0.481555466773205	0.001312255859375	\\
0.48159985794824	0.001007080078125	\\
0.481644249123274	0.000579833984375	\\
0.481688640298309	0.000518798828125	\\
0.481733031473343	0.0006103515625	\\
0.481777422648378	0.00054931640625	\\
0.481821813823412	0.0006103515625	\\
0.481866204998446	6.103515625e-05	\\
0.481910596173481	-0.00018310546875	\\
0.481954987348515	-0.000152587890625	\\
0.48199937852355	0.0001220703125	\\
0.482043769698584	-0.000274658203125	\\
0.482088160873618	-0.00054931640625	\\
0.482132552048653	3.0517578125e-05	\\
0.482176943223687	6.103515625e-05	\\
0.482221334398722	-0.00054931640625	\\
0.482265725573756	-0.000335693359375	\\
0.48231011674879	-0.00048828125	\\
0.482354507923825	-0.000396728515625	\\
0.482398899098859	-6.103515625e-05	\\
0.482443290273894	-0.000457763671875	\\
0.482487681448928	0.0001220703125	\\
0.482532072623962	0.000457763671875	\\
0.482576463798997	0.0006103515625	\\
0.482620854974031	0.001007080078125	\\
0.482665246149066	0.000946044921875	\\
0.4827096373241	0.001434326171875	\\
0.482754028499134	0.001495361328125	\\
0.482798419674169	0.000885009765625	\\
0.482842810849203	0.0008544921875	\\
0.482887202024238	0.0013427734375	\\
0.482931593199272	0.00164794921875	\\
0.482975984374306	0.00140380859375	\\
0.483020375549341	0.00115966796875	\\
0.483064766724375	0.00091552734375	\\
0.48310915789941	0.001007080078125	\\
0.483153549074444	0.0008544921875	\\
0.483197940249478	0.001251220703125	\\
0.483242331424513	0.00128173828125	\\
0.483286722599547	0.000823974609375	\\
0.483331113774582	0.00091552734375	\\
0.483375504949616	0.00128173828125	\\
0.48341989612465	0.00103759765625	\\
0.483464287299685	0.0008544921875	\\
0.483508678474719	0.000823974609375	\\
0.483553069649754	0.000640869140625	\\
0.483597460824788	0.000579833984375	\\
0.483641851999822	0.000244140625	\\
0.483686243174857	-0.00030517578125	\\
0.483730634349891	-6.103515625e-05	\\
0.483775025524926	-3.0517578125e-05	\\
0.48381941669996	-9.1552734375e-05	\\
0.483863807874994	-0.00018310546875	\\
0.483908199050029	-0.000335693359375	\\
0.483952590225063	-0.00079345703125	\\
0.483996981400098	-0.001190185546875	\\
0.484041372575132	-0.000823974609375	\\
0.484085763750166	-0.000701904296875	\\
0.484130154925201	-0.000732421875	\\
0.484174546100235	-0.00115966796875	\\
0.48421893727527	-0.001373291015625	\\
0.484263328450304	-0.00115966796875	\\
0.484307719625338	-0.001190185546875	\\
0.484352110800373	-0.00152587890625	\\
0.484396501975407	-0.00079345703125	\\
0.484440893150442	-0.000732421875	\\
0.484485284325476	-0.000701904296875	\\
0.484529675500511	-0.000152587890625	\\
0.484574066675545	-0.0001220703125	\\
0.484618457850579	0.00030517578125	\\
0.484662849025614	0.000579833984375	\\
0.484707240200648	0.000640869140625	\\
0.484751631375683	0.000732421875	\\
0.484796022550717	0.0006103515625	\\
0.484840413725751	0.00091552734375	\\
0.484884804900786	0.001007080078125	\\
0.48492919607582	0.00152587890625	\\
0.484973587250855	0.00146484375	\\
0.485017978425889	0.001312255859375	\\
0.485062369600923	0.001373291015625	\\
0.485106760775958	0.00128173828125	\\
0.485151151950992	0.00146484375	\\
0.485195543126027	0.001708984375	\\
0.485239934301061	0.00152587890625	\\
0.485284325476095	0.0015869140625	\\
0.48532871665113	0.00164794921875	\\
0.485373107826164	0.0015869140625	\\
0.485417499001199	0.001983642578125	\\
0.485461890176233	0.002197265625	\\
0.485506281351267	0.002105712890625	\\
0.485550672526302	0.001983642578125	\\
0.485595063701336	0.002105712890625	\\
0.485639454876371	0.002410888671875	\\
0.485683846051405	0.001739501953125	\\
0.485728237226439	0.001922607421875	\\
0.485772628401474	0.002044677734375	\\
0.485817019576508	0.0018310546875	\\
0.485861410751543	0.001678466796875	\\
0.485905801926577	0.00103759765625	\\
0.485950193101611	0.00128173828125	\\
0.485994584276646	0.001312255859375	\\
0.48603897545168	0.001434326171875	\\
0.486083366626715	0.001251220703125	\\
0.486127757801749	0.0010986328125	\\
0.486172148976783	0.0008544921875	\\
0.486216540151818	0.000885009765625	\\
0.486260931326852	0.000885009765625	\\
0.486305322501887	0.000762939453125	\\
0.486349713676921	0.001007080078125	\\
0.486394104851955	0.000701904296875	\\
0.48643849602699	0.0008544921875	\\
0.486482887202024	0.000701904296875	\\
0.486527278377059	0.000244140625	\\
0.486571669552093	0.00018310546875	\\
0.486616060727127	0.00067138671875	\\
0.486660451902162	0.0008544921875	\\
0.486704843077196	0.000396728515625	\\
0.486749234252231	9.1552734375e-05	\\
0.486793625427265	3.0517578125e-05	\\
0.486838016602299	0.00048828125	\\
0.486882407777334	0.00018310546875	\\
0.486926798952368	-0.000213623046875	\\
0.486971190127403	-9.1552734375e-05	\\
0.487015581302437	-0.000335693359375	\\
0.487059972477471	-0.00030517578125	\\
0.487104363652506	-3.0517578125e-05	\\
0.48714875482754	-0.000213623046875	\\
0.487193146002575	-0.000244140625	\\
0.487237537177609	9.1552734375e-05	\\
0.487281928352644	0.000152587890625	\\
0.487326319527678	-0.00030517578125	\\
0.487370710702712	-0.00042724609375	\\
0.487415101877747	-0.0001220703125	\\
0.487459493052781	0.000335693359375	\\
0.487503884227815	0.000244140625	\\
0.48754827540285	6.103515625e-05	\\
0.487592666577884	0.00054931640625	\\
0.487637057752919	0.000335693359375	\\
0.487681448927953	0.000579833984375	\\
0.487725840102988	0.0010986328125	\\
0.487770231278022	0.000885009765625	\\
0.487814622453056	0.00054931640625	\\
0.487859013628091	0.000640869140625	\\
0.487903404803125	0.00091552734375	\\
0.48794779597816	0.0010986328125	\\
0.487992187153194	0.001068115234375	\\
0.488036578328228	0.00091552734375	\\
0.488080969503263	0.00067138671875	\\
0.488125360678297	0.000335693359375	\\
0.488169751853332	0.000152587890625	\\
0.488214143028366	0.000732421875	\\
0.4882585342034	0.001190185546875	\\
0.488302925378435	0.000885009765625	\\
0.488347316553469	0.00067138671875	\\
0.488391707728504	0.001373291015625	\\
0.488436098903538	0.001556396484375	\\
0.488480490078572	0.0009765625	\\
0.488524881253607	0.00079345703125	\\
0.488569272428641	0.001007080078125	\\
0.488613663603676	0.000762939453125	\\
0.48865805477871	0.000640869140625	\\
0.488702445953744	0.0008544921875	\\
0.488746837128779	0.00048828125	\\
0.488791228303813	0.000640869140625	\\
0.488835619478848	0.000396728515625	\\
0.488880010653882	0.0001220703125	\\
0.488924401828916	0.000518798828125	\\
0.488968793003951	0.000457763671875	\\
0.489013184178985	0.000274658203125	\\
0.48905757535402	-9.1552734375e-05	\\
0.489101966529054	-0.000244140625	\\
0.489146357704088	-3.0517578125e-05	\\
0.489190748879123	0.000274658203125	\\
0.489235140054157	0.000457763671875	\\
0.489279531229192	0.0003662109375	\\
0.489323922404226	6.103515625e-05	\\
0.48936831357926	0	\\
0.489412704754295	0.000244140625	\\
0.489457095929329	0.00054931640625	\\
0.489501487104364	0.000457763671875	\\
0.489545878279398	-3.0517578125e-05	\\
0.489590269454432	0.000213623046875	\\
0.489634660629467	-3.0517578125e-05	\\
0.489679051804501	-0.000244140625	\\
0.489723442979536	-0.0001220703125	\\
0.48976783415457	0.00018310546875	\\
0.489812225329605	0.000244140625	\\
0.489856616504639	0.000396728515625	\\
0.489901007679673	0.000732421875	\\
0.489945398854708	0.0008544921875	\\
0.489989790029742	0.00128173828125	\\
0.490034181204776	0.00115966796875	\\
0.490078572379811	0.001129150390625	\\
0.490122963554845	0.00164794921875	\\
0.49016735472988	0.001373291015625	\\
0.490211745904914	0.00164794921875	\\
0.490256137079949	0.002227783203125	\\
0.490300528254983	0.0020751953125	\\
0.490344919430017	0.002685546875	\\
0.490389310605052	0.002349853515625	\\
0.490433701780086	0.002288818359375	\\
0.490478092955121	0.00286865234375	\\
0.490522484130155	0.00299072265625	\\
0.490566875305189	0.00311279296875	\\
0.490611266480224	0.003082275390625	\\
0.490655657655258	0.00286865234375	\\
0.490700048830293	0.00311279296875	\\
0.490744440005327	0.00299072265625	\\
0.490788831180361	0.003021240234375	\\
0.490833222355396	0.00323486328125	\\
0.49087761353043	0.003326416015625	\\
0.490922004705465	0.0035400390625	\\
0.490966395880499	0.003448486328125	\\
0.491010787055533	0.00311279296875	\\
0.491055178230568	0.0028076171875	\\
0.491099569405602	0.002471923828125	\\
0.491143960580637	0.001922607421875	\\
0.491188351755671	0.0020751953125	\\
0.491232742930705	0.002227783203125	\\
0.49127713410574	0.001861572265625	\\
0.491321525280774	0.002105712890625	\\
0.491365916455809	0.001983642578125	\\
0.491410307630843	0.00164794921875	\\
0.491454698805877	0.00189208984375	\\
0.491499089980912	0.0018310546875	\\
0.491543481155946	0.00189208984375	\\
0.491587872330981	0.002166748046875	\\
0.491632263506015	0.001678466796875	\\
0.491676654681049	0.00152587890625	\\
0.491721045856084	0.0013427734375	\\
0.491765437031118	0.0010986328125	\\
0.491809828206153	0.001373291015625	\\
0.491854219381187	0.001739501953125	\\
0.491898610556221	0.0015869140625	\\
0.491943001731256	0.001800537109375	\\
0.49198739290629	0.0020751953125	\\
0.492031784081325	0.001953125	\\
0.492076175256359	0.00201416015625	\\
0.492120566431393	0.00213623046875	\\
0.492164957606428	0.002166748046875	\\
0.492209348781462	0.00213623046875	\\
0.492253739956497	0.00213623046875	\\
0.492298131131531	0.001953125	\\
0.492342522306565	0.002044677734375	\\
0.4923869134816	0.00244140625	\\
0.492431304656634	0.002471923828125	\\
0.492475695831669	0.0023193359375	\\
0.492520087006703	0.00189208984375	\\
0.492564478181737	0.001373291015625	\\
0.492608869356772	0.0018310546875	\\
0.492653260531806	0.001739501953125	\\
0.492697651706841	0.00128173828125	\\
0.492742042881875	0.001251220703125	\\
0.492786434056909	0.00146484375	\\
0.492830825231944	0.0010986328125	\\
0.492875216406978	0.0010986328125	\\
0.492919607582013	0.001129150390625	\\
0.492963998757047	0.000823974609375	\\
0.493008389932082	0.00042724609375	\\
0.493052781107116	9.1552734375e-05	\\
0.49309717228215	0.00042724609375	\\
0.493141563457185	0.00048828125	\\
0.493185954632219	0.00048828125	\\
0.493230345807254	0.0008544921875	\\
0.493274736982288	0.001251220703125	\\
0.493319128157322	0.0015869140625	\\
0.493363519332357	0.00164794921875	\\
0.493407910507391	0.0010986328125	\\
0.493452301682426	0.0008544921875	\\
0.49349669285746	0.00146484375	\\
0.493541084032494	0.001373291015625	\\
0.493585475207529	0.001129150390625	\\
0.493629866382563	0.00152587890625	\\
0.493674257557598	0.00146484375	\\
0.493718648732632	0.001617431640625	\\
0.493763039907666	0.001922607421875	\\
0.493807431082701	0.001953125	\\
0.493851822257735	0.002166748046875	\\
0.49389621343277	0.002471923828125	\\
0.493940604607804	0.00262451171875	\\
0.493984995782838	0.00225830078125	\\
0.494029386957873	0.002716064453125	\\
0.494073778132907	0.00299072265625	\\
0.494118169307942	0.002593994140625	\\
0.494162560482976	0.002899169921875	\\
0.49420695165801	0.002899169921875	\\
0.494251342833045	0.0028076171875	\\
0.494295734008079	0.002685546875	\\
0.494340125183114	0.00244140625	\\
0.494384516358148	0.00286865234375	\\
0.494428907533182	0.003021240234375	\\
0.494473298708217	0.002716064453125	\\
0.494517689883251	0.002838134765625	\\
0.494562081058286	0.002410888671875	\\
0.49460647223332	0.00262451171875	\\
0.494650863408354	0.002532958984375	\\
0.494695254583389	0.00244140625	\\
0.494739645758423	0.002105712890625	\\
0.494784036933458	0.00201416015625	\\
0.494828428108492	0.001708984375	\\
0.494872819283526	0.001312255859375	\\
0.494917210458561	0.001251220703125	\\
0.494961601633595	0.0010986328125	\\
0.49500599280863	0.001220703125	\\
0.495050383983664	0.001068115234375	\\
0.495094775158698	0.00048828125	\\
0.495139166333733	0.000579833984375	\\
0.495183557508767	0.000640869140625	\\
0.495227948683802	0.000457763671875	\\
0.495272339858836	0.000396728515625	\\
0.49531673103387	0.000152587890625	\\
0.495361122208905	0.000396728515625	\\
0.495405513383939	0.000457763671875	\\
0.495449904558974	0	\\
0.495494295734008	-0.000335693359375	\\
0.495538686909042	-3.0517578125e-05	\\
0.495583078084077	-6.103515625e-05	\\
0.495627469259111	-0.00042724609375	\\
0.495671860434146	-0.00079345703125	\\
0.49571625160918	-0.0009765625	\\
0.495760642784215	-0.000946044921875	\\
0.495805033959249	-0.000885009765625	\\
0.495849425134283	-0.001007080078125	\\
0.495893816309318	-0.00115966796875	\\
0.495938207484352	-0.001129150390625	\\
0.495982598659386	-0.00103759765625	\\
0.496026989834421	-0.00079345703125	\\
0.496071381009455	-0.0006103515625	\\
0.49611577218449	-0.001068115234375	\\
0.496160163359524	-0.000640869140625	\\
0.496204554534559	-0.000152587890625	\\
0.496248945709593	-0.00042724609375	\\
0.496293336884627	-0.000579833984375	\\
0.496337728059662	-0.000274658203125	\\
0.496382119234696	0.000152587890625	\\
0.496426510409731	-0.0001220703125	\\
0.496470901584765	-0.0003662109375	\\
0.496515292759799	-0.000335693359375	\\
0.496559683934834	-9.1552734375e-05	\\
0.496604075109868	0.00030517578125	\\
0.496648466284903	0.000946044921875	\\
0.496692857459937	0.001129150390625	\\
0.496737248634971	0.001129150390625	\\
0.496781639810006	0.001068115234375	\\
0.49682603098504	0.00079345703125	\\
0.496870422160075	0.00103759765625	\\
0.496914813335109	0.001129150390625	\\
0.496959204510143	0.000732421875	\\
0.497003595685178	0.000518798828125	\\
0.497047986860212	0.000518798828125	\\
0.497092378035247	0.00054931640625	\\
0.497136769210281	0.000762939453125	\\
0.497181160385315	0.00091552734375	\\
0.49722555156035	0.000885009765625	\\
0.497269942735384	0.00079345703125	\\
0.497314333910419	0.000946044921875	\\
0.497358725085453	0.00030517578125	\\
0.497403116260487	0.00018310546875	\\
0.497447507435522	0.000274658203125	\\
0.497491898610556	0.000152587890625	\\
0.497536289785591	0.000579833984375	\\
0.497580680960625	0.00054931640625	\\
0.497625072135659	0.00018310546875	\\
0.497669463310694	0.00018310546875	\\
0.497713854485728	-9.1552734375e-05	\\
0.497758245660763	-0.000457763671875	\\
0.497802636835797	0.000152587890625	\\
0.497847028010831	3.0517578125e-05	\\
0.497891419185866	-0.000518798828125	\\
0.4979358103609	-0.000732421875	\\
0.497980201535935	-0.000701904296875	\\
0.498024592710969	-0.00048828125	\\
0.498068983886003	-0.000640869140625	\\
0.498113375061038	-0.001190185546875	\\
0.498157766236072	-0.00128173828125	\\
0.498202157411107	-0.001129150390625	\\
0.498246548586141	-0.0006103515625	\\
0.498290939761176	-0.000335693359375	\\
0.49833533093621	-0.00091552734375	\\
0.498379722111244	-0.00103759765625	\\
0.498424113286279	-0.001129150390625	\\
0.498468504461313	-0.00091552734375	\\
0.498512895636347	-0.000885009765625	\\
0.498557286811382	-0.000640869140625	\\
0.498601677986416	-0.000213623046875	\\
0.498646069161451	-0.000640869140625	\\
0.498690460336485	-6.103515625e-05	\\
0.49873485151152	0.0006103515625	\\
0.498779242686554	0.000762939453125	\\
0.498823633861588	0.000732421875	\\
0.498868025036623	0.0003662109375	\\
0.498912416211657	0.00067138671875	\\
0.498956807386692	0.000732421875	\\
0.499001198561726	0.0006103515625	\\
0.49904558973676	0.0006103515625	\\
0.499089980911795	0.00079345703125	\\
0.499134372086829	0.000701904296875	\\
0.499178763261864	0.0006103515625	\\
0.499223154436898	0.0009765625	\\
0.499267545611932	0.000640869140625	\\
0.499311936786967	0.000701904296875	\\
0.499356327962001	0.000946044921875	\\
0.499400719137036	0.000762939453125	\\
0.49944511031207	0.00042724609375	\\
0.499489501487104	0.000274658203125	\\
0.499533892662139	0	\\
0.499578283837173	9.1552734375e-05	\\
0.499622675012208	0.000274658203125	\\
0.499667066187242	0.00030517578125	\\
0.499711457362276	-0.00030517578125	\\
0.499755848537311	-0.00048828125	\\
0.499800239712345	-0.0001220703125	\\
0.49984463088738	-0.0003662109375	\\
0.499889022062414	-0.000274658203125	\\
0.499933413237448	-0.000274658203125	\\
0.499977804412483	3.0517578125e-05	\\
0.500022195587517	-0.0003662109375	\\
0.500066586762552	-0.00018310546875	\\
0.500110977937586	-6.103515625e-05	\\
0.50015536911262	-0.000640869140625	\\
0.500199760287655	-0.000457763671875	\\
0.500244151462689	-0.000518798828125	\\
0.500288542637724	-0.001068115234375	\\
0.500332933812758	-0.001007080078125	\\
0.500377324987792	-0.000823974609375	\\
0.500421716162827	-0.0008544921875	\\
0.500466107337861	-0.00054931640625	\\
0.500510498512896	-0.000396728515625	\\
0.50055488968793	-0.00042724609375	\\
0.500599280862964	-0.00054931640625	\\
0.500643672037999	-0.000396728515625	\\
0.500688063213033	-0.000396728515625	\\
0.500732454388068	-0.0009765625	\\
0.500776845563102	-0.001007080078125	\\
0.500821236738137	-0.000762939453125	\\
0.500865627913171	-0.00042724609375	\\
0.500910019088205	-0.0003662109375	\\
0.50095441026324	-0.00048828125	\\
0.500998801438274	-0.000701904296875	\\
0.501043192613308	-0.00054931640625	\\
0.501087583788343	-0.00067138671875	\\
0.501131974963377	-0.00048828125	\\
0.501176366138412	-0.000457763671875	\\
0.501220757313446	-0.000213623046875	\\
0.50126514848848	0	\\
0.501309539663515	-0.000701904296875	\\
0.501353930838549	-0.0008544921875	\\
0.501398322013584	-0.001068115234375	\\
0.501442713188618	-0.00115966796875	\\
0.501487104363653	-0.000823974609375	\\
0.501531495538687	-0.00164794921875	\\
0.501575886713721	-0.0018310546875	\\
0.501620277888756	-0.001617431640625	\\
0.50166466906379	-0.0018310546875	\\
0.501709060238825	-0.001739501953125	\\
0.501753451413859	-0.00164794921875	\\
0.501797842588893	-0.001739501953125	\\
0.501842233763928	-0.0015869140625	\\
0.501886624938962	-0.00189208984375	\\
0.501931016113996	-0.00201416015625	\\
0.501975407289031	-0.001739501953125	\\
0.502019798464065	-0.0018310546875	\\
0.5020641896391	-0.001678466796875	\\
0.502108580814134	-0.001068115234375	\\
0.502152971989169	-0.000823974609375	\\
0.502197363164203	-0.00067138671875	\\
0.502241754339237	-0.00030517578125	\\
0.502286145514272	-0.0006103515625	\\
0.502330536689306	-0.000885009765625	\\
0.502374927864341	-0.000946044921875	\\
0.502419319039375	-0.000732421875	\\
0.502463710214409	-0.000762939453125	\\
0.502508101389444	-0.000885009765625	\\
0.502552492564478	-0.001007080078125	\\
0.502596883739513	-0.000823974609375	\\
0.502641274914547	-0.001068115234375	\\
0.502685666089581	-0.00103759765625	\\
0.502730057264616	-0.000335693359375	\\
0.50277444843965	-0.000274658203125	\\
0.502818839614685	-0.000457763671875	\\
0.502863230789719	-0.00054931640625	\\
0.502907621964753	-0.000762939453125	\\
0.502952013139788	-0.00103759765625	\\
0.502996404314822	-0.000885009765625	\\
0.503040795489857	-0.000946044921875	\\
0.503085186664891	-0.0008544921875	\\
0.503129577839925	-0.000885009765625	\\
0.50317396901496	-0.00115966796875	\\
0.503218360189994	-0.00079345703125	\\
0.503262751365029	-0.000885009765625	\\
0.503307142540063	-0.000823974609375	\\
0.503351533715097	-0.00067138671875	\\
0.503395924890132	-0.00079345703125	\\
0.503440316065166	-0.000701904296875	\\
0.503484707240201	-0.00054931640625	\\
0.503529098415235	-0.00042724609375	\\
0.503573489590269	-0.000579833984375	\\
0.503617880765304	-0.00067138671875	\\
0.503662271940338	-0.000732421875	\\
0.503706663115373	-0.000640869140625	\\
0.503751054290407	-0.000732421875	\\
0.503795445465441	-0.000518798828125	\\
0.503839836640476	-0.000762939453125	\\
0.50388422781551	-0.001007080078125	\\
0.503928618990545	-0.00042724609375	\\
0.503973010165579	-6.103515625e-05	\\
0.504017401340613	-0.000335693359375	\\
0.504061792515648	-0.000518798828125	\\
0.504106183690682	-0.000457763671875	\\
0.504150574865717	-0.00079345703125	\\
0.504194966040751	-0.000762939453125	\\
0.504239357215786	-0.00091552734375	\\
0.50428374839082	-0.001068115234375	\\
0.504328139565854	-0.0009765625	\\
0.504372530740889	-0.000885009765625	\\
0.504416921915923	-0.000640869140625	\\
0.504461313090958	-0.00067138671875	\\
0.504505704265992	-0.00128173828125	\\
0.504550095441026	-0.001556396484375	\\
0.504594486616061	-0.00164794921875	\\
0.504638877791095	-0.002166748046875	\\
0.50468326896613	-0.00244140625	\\
0.504727660141164	-0.002593994140625	\\
0.504772051316198	-0.00250244140625	\\
0.504816442491233	-0.00274658203125	\\
0.504860833666267	-0.003265380859375	\\
0.504905224841302	-0.0032958984375	\\
0.504949616016336	-0.002960205078125	\\
0.50499400719137	-0.003021240234375	\\
0.505038398366405	-0.002899169921875	\\
0.505082789541439	-0.002777099609375	\\
0.505127180716474	-0.003082275390625	\\
0.505171571891508	-0.0029296875	\\
0.505215963066542	-0.003143310546875	\\
0.505260354241577	-0.00262451171875	\\
0.505304745416611	-0.002349853515625	\\
0.505349136591646	-0.002349853515625	\\
0.50539352776668	-0.00189208984375	\\
0.505437918941714	-0.001953125	\\
0.505482310116749	-0.00146484375	\\
0.505526701291783	-0.001068115234375	\\
0.505571092466818	-0.001068115234375	\\
0.505615483641852	-0.00128173828125	\\
0.505659874816886	-0.0013427734375	\\
0.505704265991921	-0.00079345703125	\\
0.505748657166955	-0.000457763671875	\\
0.50579304834199	-0.001007080078125	\\
0.505837439517024	-0.0008544921875	\\
0.505881830692058	-0.000457763671875	\\
0.505926221867093	-0.000396728515625	\\
0.505970613042127	-0.000396728515625	\\
0.506015004217162	-0.00030517578125	\\
0.506059395392196	-0.000274658203125	\\
0.50610378656723	-0.000579833984375	\\
0.506148177742265	-0.000579833984375	\\
0.506192568917299	-0.000152587890625	\\
0.506236960092334	-0.000152587890625	\\
0.506281351267368	0	\\
0.506325742442402	-0.0001220703125	\\
0.506370133617437	0.0001220703125	\\
0.506414524792471	3.0517578125e-05	\\
0.506458915967506	-0.000274658203125	\\
0.50650330714254	6.103515625e-05	\\
0.506547698317575	3.0517578125e-05	\\
0.506592089492609	0.0001220703125	\\
0.506636480667643	3.0517578125e-05	\\
0.506680871842678	-0.00054931640625	\\
0.506725263017712	-0.00048828125	\\
0.506769654192746	-0.000213623046875	\\
0.506814045367781	-0.00030517578125	\\
0.506858436542815	-0.00079345703125	\\
0.50690282771785	-0.0010986328125	\\
0.506947218892884	-0.00079345703125	\\
0.506991610067918	-0.000823974609375	\\
0.507036001242953	-0.001220703125	\\
0.507080392417987	-0.00103759765625	\\
0.507124783593022	-0.0006103515625	\\
0.507169174768056	-0.000640869140625	\\
0.507213565943091	-0.0008544921875	\\
0.507257957118125	-0.00115966796875	\\
0.507302348293159	-0.001617431640625	\\
0.507346739468194	-0.001922607421875	\\
0.507391130643228	-0.00140380859375	\\
0.507435521818263	-0.001129150390625	\\
0.507479912993297	-0.00140380859375	\\
0.507524304168331	-0.00140380859375	\\
0.507568695343366	-0.001190185546875	\\
0.5076130865184	-0.00115966796875	\\
0.507657477693435	-0.001190185546875	\\
0.507701868868469	-0.001068115234375	\\
0.507746260043503	-0.00091552734375	\\
0.507790651218538	-0.00054931640625	\\
0.507835042393572	-9.1552734375e-05	\\
0.507879433568607	0.000213623046875	\\
0.507923824743641	6.103515625e-05	\\
0.507968215918675	-0.000335693359375	\\
0.50801260709371	-0.000518798828125	\\
0.508056998268744	-0.000335693359375	\\
0.508101389443779	-0.000152587890625	\\
0.508145780618813	-0.000244140625	\\
0.508190171793847	0.00018310546875	\\
0.508234562968882	0.000732421875	\\
0.508278954143916	0.0006103515625	\\
0.508323345318951	0.000732421875	\\
0.508367736493985	0.000823974609375	\\
0.508412127669019	0.00054931640625	\\
0.508456518844054	0.001068115234375	\\
0.508500910019088	0.000335693359375	\\
0.508545301194123	6.103515625e-05	\\
0.508589692369157	0.000152587890625	\\
0.508634083544191	-0.000396728515625	\\
0.508678474719226	-0.000274658203125	\\
0.50872286589426	3.0517578125e-05	\\
0.508767257069295	-0.000152587890625	\\
0.508811648244329	-0.00018310546875	\\
0.508856039419363	-6.103515625e-05	\\
0.508900430594398	-0.000152587890625	\\
0.508944821769432	-0.000335693359375	\\
0.508989212944467	-0.000640869140625	\\
0.509033604119501	-0.0006103515625	\\
0.509077995294535	-0.000946044921875	\\
0.50912238646957	-0.001556396484375	\\
0.509166777644604	-0.001129150390625	\\
0.509211168819639	-0.001068115234375	\\
0.509255559994673	-0.001556396484375	\\
0.509299951169708	-0.001800537109375	\\
0.509344342344742	-0.00140380859375	\\
0.509388733519776	-0.00152587890625	\\
0.509433124694811	-0.001495361328125	\\
0.509477515869845	-0.00140380859375	\\
0.509521907044879	-0.00140380859375	\\
0.509566298219914	-0.001434326171875	\\
0.509610689394948	-0.00128173828125	\\
0.509655080569983	-0.00103759765625	\\
0.509699471745017	-0.000762939453125	\\
0.509743862920051	-0.0006103515625	\\
0.509788254095086	-0.000762939453125	\\
0.50983264527012	-0.00054931640625	\\
0.509877036445155	-0.00067138671875	\\
0.509921427620189	-0.000823974609375	\\
0.509965818795224	-0.000396728515625	\\
0.510010209970258	0	\\
0.510054601145292	0.00018310546875	\\
0.510098992320327	-6.103515625e-05	\\
0.510143383495361	-0.00018310546875	\\
0.510187774670396	-0.000152587890625	\\
0.51023216584543	-3.0517578125e-05	\\
0.510276557020464	-0.000274658203125	\\
0.510320948195499	-0.00079345703125	\\
0.510365339370533	-0.00048828125	\\
0.510409730545567	-0.0006103515625	\\
0.510454121720602	-0.000244140625	\\
0.510498512895636	-0.000152587890625	\\
0.510542904070671	-0.000274658203125	\\
0.510587295245705	-0.000244140625	\\
0.51063168642074	-0.00042724609375	\\
0.510676077595774	3.0517578125e-05	\\
0.510720468770808	0	\\
0.510764859945843	-0.000579833984375	\\
0.510809251120877	-0.0006103515625	\\
0.510853642295912	-0.000640869140625	\\
0.510898033470946	-0.0006103515625	\\
0.51094242464598	-0.000701904296875	\\
0.510986815821015	-0.00103759765625	\\
0.511031206996049	-0.0010986328125	\\
0.511075598171084	-0.000823974609375	\\
0.511119989346118	-0.00079345703125	\\
0.511164380521152	-0.001007080078125	\\
0.511208771696187	-0.00091552734375	\\
0.511253162871221	-0.000823974609375	\\
0.511297554046256	-0.000762939453125	\\
0.51134194522129	-0.00103759765625	\\
0.511386336396324	-0.001251220703125	\\
0.511430727571359	-0.000396728515625	\\
0.511475118746393	-0.000244140625	\\
0.511519509921428	-0.00042724609375	\\
0.511563901096462	-0.000335693359375	\\
0.511608292271496	-0.000335693359375	\\
0.511652683446531	-0.000152587890625	\\
0.511697074621565	-0.000274658203125	\\
0.5117414657966	-0.000640869140625	\\
0.511785856971634	-0.000244140625	\\
0.511830248146668	-0.000274658203125	\\
0.511874639321703	-0.000213623046875	\\
0.511919030496737	-6.103515625e-05	\\
0.511963421671772	0.000274658203125	\\
0.512007812846806	0.0006103515625	\\
0.51205220402184	0.00048828125	\\
0.512096595196875	0.00030517578125	\\
0.512140986371909	-3.0517578125e-05	\\
0.512185377546944	-3.0517578125e-05	\\
0.512229768721978	0.000396728515625	\\
0.512274159897013	0.0008544921875	\\
0.512318551072047	0.000823974609375	\\
0.512362942247081	0.000946044921875	\\
0.512407333422116	0.0009765625	\\
0.51245172459715	0.000518798828125	\\
0.512496115772184	0.0006103515625	\\
0.512540506947219	0.000701904296875	\\
0.512584898122253	0.00054931640625	\\
0.512629289297288	0.000274658203125	\\
0.512673680472322	9.1552734375e-05	\\
0.512718071647357	0.000396728515625	\\
0.512762462822391	0.000518798828125	\\
0.512806853997425	0.00042724609375	\\
0.51285124517246	0.00048828125	\\
0.512895636347494	3.0517578125e-05	\\
0.512940027522529	-6.103515625e-05	\\
0.512984418697563	0.00054931640625	\\
0.513028809872597	-3.0517578125e-05	\\
0.513073201047632	-0.00030517578125	\\
0.513117592222666	0.000244140625	\\
0.513161983397701	-0.0001220703125	\\
0.513206374572735	-0.0003662109375	\\
0.513250765747769	-0.000457763671875	\\
0.513295156922804	-0.0003662109375	\\
0.513339548097838	-0.00054931640625	\\
0.513383939272873	-0.00091552734375	\\
0.513428330447907	-0.000823974609375	\\
0.513472721622941	-0.000762939453125	\\
0.513517112797976	-0.001068115234375	\\
0.51356150397301	-0.001251220703125	\\
0.513605895148045	-0.00115966796875	\\
0.513650286323079	-0.00103759765625	\\
0.513694677498113	-0.000946044921875	\\
0.513739068673148	-0.000946044921875	\\
0.513783459848182	-0.000732421875	\\
0.513827851023217	-0.000762939453125	\\
0.513872242198251	-0.00091552734375	\\
0.513916633373285	-0.001190185546875	\\
0.51396102454832	-0.001190185546875	\\
0.514005415723354	-0.001007080078125	\\
0.514049806898389	-0.0009765625	\\
0.514094198073423	-0.000732421875	\\
0.514138589248457	-0.001007080078125	\\
0.514182980423492	-0.000701904296875	\\
0.514227371598526	-0.000701904296875	\\
0.514271762773561	-0.001129150390625	\\
0.514316153948595	-0.000518798828125	\\
0.514360545123629	-0.000335693359375	\\
0.514404936298664	-0.000732421875	\\
0.514449327473698	-0.000732421875	\\
0.514493718648733	-0.00091552734375	\\
0.514538109823767	-0.00091552734375	\\
0.514582500998801	-0.000701904296875	\\
0.514626892173836	-0.001251220703125	\\
0.51467128334887	-0.0018310546875	\\
0.514715674523905	-0.0015869140625	\\
0.514760065698939	-0.001312255859375	\\
0.514804456873973	-0.001129150390625	\\
0.514848848049008	-0.001007080078125	\\
0.514893239224042	-0.001007080078125	\\
0.514937630399077	-0.0008544921875	\\
0.514982021574111	-0.000732421875	\\
0.515026412749146	-0.001068115234375	\\
0.51507080392418	-0.001068115234375	\\
0.515115195099214	-0.000762939453125	\\
0.515159586274249	-0.000762939453125	\\
0.515203977449283	-0.0009765625	\\
0.515248368624317	-0.00079345703125	\\
0.515292759799352	-0.00079345703125	\\
0.515337150974386	-0.000640869140625	\\
0.515381542149421	-0.000213623046875	\\
0.515425933324455	-9.1552734375e-05	\\
0.515470324499489	-0.00018310546875	\\
0.515514715674524	-0.000335693359375	\\
0.515559106849558	0.0001220703125	\\
0.515603498024593	0.00054931640625	\\
0.515647889199627	0.00054931640625	\\
0.515692280374662	0.0006103515625	\\
0.515736671549696	0.00091552734375	\\
0.51578106272473	0.00115966796875	\\
0.515825453899765	0.00146484375	\\
0.515869845074799	0.0013427734375	\\
0.515914236249834	0.0008544921875	\\
0.515958627424868	0.00115966796875	\\
0.516003018599902	0.001739501953125	\\
0.516047409774937	0.00177001953125	\\
0.516091800949971	0.0018310546875	\\
0.516136192125006	0.001922607421875	\\
0.51618058330004	0.00213623046875	\\
0.516224974475074	0.002197265625	\\
0.516269365650109	0.001953125	\\
0.516313756825143	0.00225830078125	\\
0.516358148000178	0.002471923828125	\\
0.516402539175212	0.001983642578125	\\
0.516446930350246	0.001739501953125	\\
0.516491321525281	0.001922607421875	\\
0.516535712700315	0.00128173828125	\\
0.51658010387535	0.001739501953125	\\
0.516624495050384	0.00164794921875	\\
0.516668886225418	0.001190185546875	\\
0.516713277400453	0.00103759765625	\\
0.516757668575487	0.00079345703125	\\
0.516802059750522	0.0009765625	\\
0.516846450925556	0.000762939453125	\\
0.51689084210059	0.0008544921875	\\
0.516935233275625	0.00079345703125	\\
0.516979624450659	0.000579833984375	\\
0.517024015625694	0.000518798828125	\\
0.517068406800728	-3.0517578125e-05	\\
0.517112797975762	-0.00030517578125	\\
0.517157189150797	-0.000274658203125	\\
0.517201580325831	-0.000640869140625	\\
0.517245971500866	-0.000518798828125	\\
0.5172903626759	-0.00103759765625	\\
0.517334753850934	-0.001068115234375	\\
0.517379145025969	-0.000762939453125	\\
0.517423536201003	-0.001190185546875	\\
0.517467927376038	-0.00152587890625	\\
0.517512318551072	-0.0010986328125	\\
0.517556709726106	-0.001007080078125	\\
0.517601100901141	-0.0013427734375	\\
0.517645492076175	-0.00115966796875	\\
0.51768988325121	-0.00140380859375	\\
0.517734274426244	-0.00146484375	\\
0.517778665601279	-0.0008544921875	\\
0.517823056776313	-0.00054931640625	\\
0.517867447951347	-0.000946044921875	\\
0.517911839126382	-0.001373291015625	\\
0.517956230301416	-0.00091552734375	\\
0.51800062147645	-0.00115966796875	\\
0.518045012651485	-0.001129150390625	\\
0.518089403826519	-0.00054931640625	\\
0.518133795001554	-0.000518798828125	\\
0.518178186176588	-0.0003662109375	\\
0.518222577351622	-0.000335693359375	\\
0.518266968526657	-0.000274658203125	\\
0.518311359701691	-3.0517578125e-05	\\
0.518355750876726	-3.0517578125e-05	\\
0.51840014205176	-6.103515625e-05	\\
0.518444533226795	-6.103515625e-05	\\
0.518488924401829	0.00042724609375	\\
0.518533315576863	0.000640869140625	\\
0.518577706751898	0.00067138671875	\\
0.518622097926932	0.000518798828125	\\
0.518666489101967	0.000335693359375	\\
0.518710880277001	0.00048828125	\\
0.518755271452035	0.000274658203125	\\
0.51879966262707	-0.000213623046875	\\
0.518844053802104	-0.00067138671875	\\
0.518888444977139	-0.00054931640625	\\
0.518932836152173	-0.00054931640625	\\
0.518977227327207	-0.0008544921875	\\
0.519021618502242	-0.000732421875	\\
0.519066009677276	-0.001251220703125	\\
0.519110400852311	-0.00189208984375	\\
0.519154792027345	-0.00140380859375	\\
0.519199183202379	-0.001556396484375	\\
0.519243574377414	-0.001922607421875	\\
0.519287965552448	-0.00213623046875	\\
0.519332356727483	-0.00238037109375	\\
0.519376747902517	-0.002105712890625	\\
0.519421139077551	-0.002044677734375	\\
0.519465530252586	-0.002410888671875	\\
0.51950992142762	-0.00244140625	\\
0.519554312602655	-0.0023193359375	\\
0.519598703777689	-0.0020751953125	\\
0.519643094952723	-0.002166748046875	\\
0.519687486127758	-0.002349853515625	\\
0.519731877302792	-0.001922607421875	\\
0.519776268477827	-0.001556396484375	\\
0.519820659652861	-0.001800537109375	\\
0.519865050827895	-0.002044677734375	\\
0.51990944200293	-0.001983642578125	\\
0.519953833177964	-0.002227783203125	\\
0.519998224352999	-0.00177001953125	\\
0.520042615528033	-0.001861572265625	\\
0.520087006703067	-0.001983642578125	\\
0.520131397878102	-0.001251220703125	\\
0.520175789053136	-0.0010986328125	\\
0.520220180228171	-0.001129150390625	\\
0.520264571403205	-0.000762939453125	\\
0.520308962578239	-0.0006103515625	\\
0.520353353753274	-0.0009765625	\\
0.520397744928308	-0.00079345703125	\\
0.520442136103343	-0.000946044921875	\\
0.520486527278377	-0.0008544921875	\\
0.520530918453411	-0.00091552734375	\\
0.520575309628446	-0.001068115234375	\\
0.52061970080348	-0.000732421875	\\
0.520664091978515	-0.000579833984375	\\
0.520708483153549	-0.001068115234375	\\
0.520752874328584	-0.00128173828125	\\
0.520797265503618	-0.00140380859375	\\
0.520841656678652	-0.00146484375	\\
0.520886047853687	-0.001190185546875	\\
0.520930439028721	-0.001129150390625	\\
0.520974830203755	-0.00103759765625	\\
0.52101922137879	-0.00128173828125	\\
0.521063612553824	-0.001556396484375	\\
0.521108003728859	-0.001495361328125	\\
0.521152394903893	-0.001495361328125	\\
0.521196786078928	-0.001556396484375	\\
0.521241177253962	-0.001678466796875	\\
0.521285568428996	-0.001800537109375	\\
0.521329959604031	-0.001312255859375	\\
0.521374350779065	-0.00103759765625	\\
0.5214187419541	-0.001495361328125	\\
0.521463133129134	-0.00115966796875	\\
0.521507524304168	-0.000701904296875	\\
0.521551915479203	-0.000823974609375	\\
0.521596306654237	-0.001068115234375	\\
0.521640697829272	-0.001373291015625	\\
0.521685089004306	-0.0009765625	\\
0.52172948017934	-0.000335693359375	\\
0.521773871354375	-0.0003662109375	\\
0.521818262529409	-0.000457763671875	\\
0.521862653704444	-0.000152587890625	\\
0.521907044879478	-3.0517578125e-05	\\
0.521951436054512	6.103515625e-05	\\
0.521995827229547	6.103515625e-05	\\
0.522040218404581	-6.103515625e-05	\\
0.522084609579616	9.1552734375e-05	\\
0.52212900075465	-0.000152587890625	\\
0.522173391929684	-0.0001220703125	\\
0.522217783104719	-0.000274658203125	\\
0.522262174279753	-0.0003662109375	\\
0.522306565454788	-0.000518798828125	\\
0.522350956629822	-0.000701904296875	\\
0.522395347804856	-0.001007080078125	\\
0.522439738979891	-0.00115966796875	\\
0.522484130154925	-0.00115966796875	\\
0.52252852132996	-0.001129150390625	\\
0.522572912504994	-0.000732421875	\\
0.522617303680028	-0.000885009765625	\\
0.522661694855063	-0.000885009765625	\\
0.522706086030097	-0.001220703125	\\
0.522750477205132	-0.00146484375	\\
0.522794868380166	-0.001312255859375	\\
0.5228392595552	-0.00164794921875	\\
0.522883650730235	-0.00201416015625	\\
0.522928041905269	-0.002105712890625	\\
0.522972433080304	-0.0023193359375	\\
0.523016824255338	-0.0023193359375	\\
0.523061215430372	-0.002166748046875	\\
0.523105606605407	-0.002349853515625	\\
0.523149997780441	-0.002685546875	\\
0.523194388955476	-0.00274658203125	\\
0.52323878013051	-0.00262451171875	\\
0.523283171305544	-0.002960205078125	\\
0.523327562480579	-0.0028076171875	\\
0.523371953655613	-0.00274658203125	\\
0.523416344830648	-0.00323486328125	\\
0.523460736005682	-0.003173828125	\\
0.523505127180717	-0.002838134765625	\\
0.523549518355751	-0.002838134765625	\\
0.523593909530785	-0.002685546875	\\
0.52363830070582	-0.00274658203125	\\
0.523682691880854	-0.0028076171875	\\
0.523727083055888	-0.002349853515625	\\
0.523771474230923	-0.00286865234375	\\
0.523815865405957	-0.002655029296875	\\
0.523860256580992	-0.002410888671875	\\
0.523904647756026	-0.002777099609375	\\
0.52394903893106	-0.002655029296875	\\
0.523993430106095	-0.00286865234375	\\
0.524037821281129	-0.003143310546875	\\
0.524082212456164	-0.0030517578125	\\
0.524126603631198	-0.002593994140625	\\
0.524170994806233	-0.00262451171875	\\
0.524215385981267	-0.002410888671875	\\
0.524259777156301	-0.002655029296875	\\
0.524304168331336	-0.002838134765625	\\
0.52434855950637	-0.002532958984375	\\
0.524392950681405	-0.0023193359375	\\
0.524437341856439	-0.002105712890625	\\
0.524481733031473	-0.001953125	\\
0.524526124206508	-0.00189208984375	\\
0.524570515381542	-0.00189208984375	\\
0.524614906556577	-0.0020751953125	\\
0.524659297731611	-0.0018310546875	\\
0.524703688906645	-0.00115966796875	\\
0.52474808008168	-0.00091552734375	\\
0.524792471256714	-0.001190185546875	\\
0.524836862431749	-0.00091552734375	\\
0.524881253606783	-9.1552734375e-05	\\
0.524925644781817	-0.00054931640625	\\
0.524970035956852	-0.000885009765625	\\
0.525014427131886	-0.000579833984375	\\
0.525058818306921	-0.00030517578125	\\
0.525103209481955	-0.0001220703125	\\
0.525147600656989	-3.0517578125e-05	\\
0.525191991832024	0.0001220703125	\\
0.525236383007058	0.0006103515625	\\
0.525280774182093	0.00091552734375	\\
0.525325165357127	0.000518798828125	\\
0.525369556532161	0.000457763671875	\\
0.525413947707196	0.0003662109375	\\
0.52545833888223	0.000244140625	\\
0.525502730057265	0.000335693359375	\\
0.525547121232299	0.000396728515625	\\
0.525591512407333	0.0003662109375	\\
0.525635903582368	0.000335693359375	\\
0.525680294757402	0.00048828125	\\
0.525724685932437	0.0006103515625	\\
0.525769077107471	0.000335693359375	\\
0.525813468282505	-3.0517578125e-05	\\
0.52585785945754	3.0517578125e-05	\\
0.525902250632574	0.000213623046875	\\
0.525946641807609	0.000152587890625	\\
0.525991032982643	-0.000152587890625	\\
0.526035424157677	-0.000274658203125	\\
0.526079815332712	0.000244140625	\\
0.526124206507746	0.000274658203125	\\
0.526168597682781	0.000396728515625	\\
0.526212988857815	-6.103515625e-05	\\
0.52625738003285	-0.0003662109375	\\
0.526301771207884	-0.00042724609375	\\
0.526346162382918	-0.000946044921875	\\
0.526390553557953	-0.00128173828125	\\
0.526434944732987	-0.00128173828125	\\
0.526479335908022	-0.00140380859375	\\
0.526523727083056	-0.001861572265625	\\
0.52656811825809	-0.00201416015625	\\
0.526612509433125	-0.00189208984375	\\
0.526656900608159	-0.001800537109375	\\
0.526701291783193	-0.002044677734375	\\
0.526745682958228	-0.001983642578125	\\
0.526790074133262	-0.0020751953125	\\
0.526834465308297	-0.002166748046875	\\
0.526878856483331	-0.00225830078125	\\
0.526923247658366	-0.001983642578125	\\
0.5269676388334	-0.001708984375	\\
0.527012030008434	-0.001953125	\\
0.527056421183469	-0.00152587890625	\\
0.527100812358503	-0.001220703125	\\
0.527145203533538	-0.001434326171875	\\
0.527189594708572	-0.00152587890625	\\
0.527233985883606	-0.0013427734375	\\
0.527278377058641	-0.001373291015625	\\
0.527322768233675	-0.0013427734375	\\
0.52736715940871	-0.00140380859375	\\
0.527411550583744	-0.001434326171875	\\
0.527455941758778	-0.001373291015625	\\
0.527500332933813	-0.001434326171875	\\
0.527544724108847	-0.001190185546875	\\
0.527589115283882	-0.0008544921875	\\
0.527633506458916	-0.00115966796875	\\
0.52767789763395	-0.000732421875	\\
0.527722288808985	-0.000335693359375	\\
0.527766679984019	-0.00079345703125	\\
0.527811071159054	-0.0008544921875	\\
0.527855462334088	-0.00067138671875	\\
0.527899853509122	-0.000885009765625	\\
0.527944244684157	-0.001129150390625	\\
0.527988635859191	-0.000885009765625	\\
0.528033027034226	-0.000762939453125	\\
0.52807741820926	-0.001007080078125	\\
0.528121809384294	-0.001190185546875	\\
0.528166200559329	-0.00152587890625	\\
0.528210591734363	-0.00164794921875	\\
0.528254982909398	-0.00140380859375	\\
0.528299374084432	-0.001678466796875	\\
0.528343765259466	-0.002105712890625	\\
0.528388156434501	-0.001922607421875	\\
0.528432547609535	-0.001861572265625	\\
0.52847693878457	-0.002105712890625	\\
0.528521329959604	-0.00250244140625	\\
0.528565721134638	-0.00262451171875	\\
0.528610112309673	-0.00244140625	\\
0.528654503484707	-0.002838134765625	\\
0.528698894659742	-0.0030517578125	\\
0.528743285834776	-0.003204345703125	\\
0.52878767700981	-0.00323486328125	\\
0.528832068184845	-0.0032958984375	\\
0.528876459359879	-0.003570556640625	\\
0.528920850534914	-0.003204345703125	\\
0.528965241709948	-0.003387451171875	\\
0.529009632884982	-0.003082275390625	\\
0.529054024060017	-0.00311279296875	\\
0.529098415235051	-0.00311279296875	\\
0.529142806410086	-0.002655029296875	\\
0.52918719758512	-0.002593994140625	\\
0.529231588760155	-0.002655029296875	\\
0.529275979935189	-0.001953125	\\
0.529320371110223	-0.001739501953125	\\
0.529364762285258	-0.00213623046875	\\
0.529409153460292	-0.001556396484375	\\
0.529453544635326	-0.001739501953125	\\
0.529497935810361	-0.001739501953125	\\
0.529542326985395	-0.001068115234375	\\
0.52958671816043	-0.001190185546875	\\
0.529631109335464	-0.000885009765625	\\
0.529675500510499	-0.000396728515625	\\
0.529719891685533	-0.000701904296875	\\
0.529764282860567	-0.000457763671875	\\
0.529808674035602	-0.000732421875	\\
0.529853065210636	-0.000762939453125	\\
0.529897456385671	-0.000579833984375	\\
0.529941847560705	-0.000640869140625	\\
0.529986238735739	-0.00030517578125	\\
0.530030629910774	-0.000274658203125	\\
0.530075021085808	-0.00042724609375	\\
0.530119412260843	0.0001220703125	\\
0.530163803435877	0.000152587890625	\\
0.530208194610911	0.0001220703125	\\
0.530252585785946	0.000732421875	\\
0.53029697696098	0.000335693359375	\\
0.530341368136015	-3.0517578125e-05	\\
0.530385759311049	0.0001220703125	\\
0.530430150486083	0.00030517578125	\\
0.530474541661118	0.00042724609375	\\
0.530518932836152	0.000152587890625	\\
0.530563324011187	0	\\
0.530607715186221	0.000152587890625	\\
0.530652106361255	-9.1552734375e-05	\\
0.53069649753629	-0.000213623046875	\\
0.530740888711324	-0.000274658203125	\\
0.530785279886359	-9.1552734375e-05	\\
0.530829671061393	-3.0517578125e-05	\\
0.530874062236427	-0.000335693359375	\\
0.530918453411462	-9.1552734375e-05	\\
0.530962844586496	-6.103515625e-05	\\
0.531007235761531	-0.000335693359375	\\
0.531051626936565	-0.00030517578125	\\
0.531096018111599	3.0517578125e-05	\\
0.531140409286634	0.00030517578125	\\
0.531184800461668	0.00030517578125	\\
0.531229191636703	0.00054931640625	\\
0.531273582811737	0.0013427734375	\\
0.531317973986771	0.001708984375	\\
0.531362365161806	0.001708984375	\\
0.53140675633684	0.0018310546875	\\
0.531451147511875	0.001708984375	\\
0.531495538686909	0.00177001953125	\\
0.531539929861943	0.001434326171875	\\
0.531584321036978	0.00152587890625	\\
0.531628712212012	0.002197265625	\\
0.531673103387047	0.0018310546875	\\
0.531717494562081	0.001708984375	\\
0.531761885737115	0.00201416015625	\\
0.53180627691215	0.0018310546875	\\
0.531850668087184	0.0015869140625	\\
0.531895059262219	0.001708984375	\\
0.531939450437253	0.001953125	\\
0.531983841612288	0.00201416015625	\\
0.532028232787322	0.001983642578125	\\
0.532072623962356	0.002655029296875	\\
0.532117015137391	0.0020751953125	\\
0.532161406312425	0.001861572265625	\\
0.532205797487459	0.0020751953125	\\
0.532250188662494	0.001739501953125	\\
0.532294579837528	0.001434326171875	\\
0.532338971012563	0.00140380859375	\\
0.532383362187597	0.001617431640625	\\
0.532427753362631	0.001617431640625	\\
0.532472144537666	0.00164794921875	\\
0.5325165357127	0.00152587890625	\\
0.532560926887735	0.00152587890625	\\
0.532605318062769	0.001220703125	\\
0.532649709237804	0.001373291015625	\\
0.532694100412838	0.001312255859375	\\
};
\addplot [color=blue,solid,forget plot]
  table[row sep=crcr]{
0.532694100412838	0.001312255859375	\\
0.532738491587872	0.001007080078125	\\
0.532782882762907	0.000640869140625	\\
0.532827273937941	0.00042724609375	\\
0.532871665112976	0.000823974609375	\\
0.53291605628801	0.00067138671875	\\
0.532960447463044	0.000244140625	\\
0.533004838638079	0.00018310546875	\\
0.533049229813113	0.000152587890625	\\
0.533093620988148	0	\\
0.533138012163182	-0.0003662109375	\\
0.533182403338216	-0.000244140625	\\
0.533226794513251	6.103515625e-05	\\
0.533271185688285	-3.0517578125e-05	\\
0.53331557686332	0.00018310546875	\\
0.533359968038354	0.00030517578125	\\
0.533404359213388	0.000274658203125	\\
0.533448750388423	0.000335693359375	\\
0.533493141563457	0.0003662109375	\\
0.533537532738492	0.000732421875	\\
0.533581923913526	0.000335693359375	\\
0.53362631508856	0	\\
0.533670706263595	0.000244140625	\\
0.533715097438629	0.000579833984375	\\
0.533759488613664	0.000701904296875	\\
0.533803879788698	0.00079345703125	\\
0.533848270963732	0.001220703125	\\
0.533892662138767	0.001312255859375	\\
0.533937053313801	0.001708984375	\\
0.533981444488836	0.001922607421875	\\
0.53402583566387	0.00177001953125	\\
0.534070226838904	0.001251220703125	\\
0.534114618013939	0.001556396484375	\\
0.534159009188973	0.001617431640625	\\
0.534203400364008	0.0010986328125	\\
0.534247791539042	0.001678466796875	\\
0.534292182714076	0.00189208984375	\\
0.534336573889111	0.00164794921875	\\
0.534380965064145	0.001953125	\\
0.53442535623918	0.00213623046875	\\
0.534469747414214	0.001953125	\\
0.534514138589248	0.00164794921875	\\
0.534558529764283	0.002227783203125	\\
0.534602920939317	0.0020751953125	\\
0.534647312114352	0.0015869140625	\\
0.534691703289386	0.001953125	\\
0.534736094464421	0.001861572265625	\\
0.534780485639455	0.00152587890625	\\
0.534824876814489	0.001190185546875	\\
0.534869267989524	0.00067138671875	\\
0.534913659164558	0.000946044921875	\\
0.534958050339593	0.00140380859375	\\
0.535002441514627	0.00164794921875	\\
0.535046832689661	0.00140380859375	\\
0.535091223864696	0.00140380859375	\\
0.53513561503973	0.00140380859375	\\
0.535180006214764	0.001068115234375	\\
0.535224397389799	0.00115966796875	\\
0.535268788564833	0.001190185546875	\\
0.535313179739868	0.001190185546875	\\
0.535357570914902	0.001129150390625	\\
0.535401962089937	0.00146484375	\\
0.535446353264971	0.001251220703125	\\
0.535490744440005	0.00079345703125	\\
0.53553513561504	0.00048828125	\\
0.535579526790074	0.000457763671875	\\
0.535623917965109	0.00042724609375	\\
0.535668309140143	0.00091552734375	\\
0.535712700315177	0.001068115234375	\\
0.535757091490212	0.000701904296875	\\
0.535801482665246	0.000579833984375	\\
0.535845873840281	0.000640869140625	\\
0.535890265015315	0.001129150390625	\\
0.535934656190349	0.001068115234375	\\
0.535979047365384	0.000396728515625	\\
0.536023438540418	0.000457763671875	\\
0.536067829715453	0.000457763671875	\\
0.536112220890487	0.000335693359375	\\
0.536156612065521	0.001007080078125	\\
0.536201003240556	0.000518798828125	\\
0.53624539441559	0.000152587890625	\\
0.536289785590625	0.00018310546875	\\
0.536334176765659	0.000244140625	\\
0.536378567940693	0.000640869140625	\\
0.536422959115728	0.000701904296875	\\
0.536467350290762	0.000762939453125	\\
0.536511741465797	0.00048828125	\\
0.536556132640831	0.0001220703125	\\
0.536600523815865	0.000213623046875	\\
0.5366449149909	6.103515625e-05	\\
0.536689306165934	-0.0001220703125	\\
0.536733697340969	-9.1552734375e-05	\\
0.536778088516003	-6.103515625e-05	\\
0.536822479691037	-0.0001220703125	\\
0.536866870866072	9.1552734375e-05	\\
0.536911262041106	0.000152587890625	\\
0.536955653216141	0.0001220703125	\\
0.537000044391175	-0.0001220703125	\\
0.537044435566209	3.0517578125e-05	\\
0.537088826741244	0.000213623046875	\\
0.537133217916278	0.000213623046875	\\
0.537177609091313	0.000152587890625	\\
0.537222000266347	-0.0001220703125	\\
0.537266391441381	-0.00042724609375	\\
0.537310782616416	-0.000274658203125	\\
0.53735517379145	-0.000213623046875	\\
0.537399564966485	-0.0001220703125	\\
0.537443956141519	-0.000152587890625	\\
0.537488347316553	9.1552734375e-05	\\
0.537532738491588	-3.0517578125e-05	\\
0.537577129666622	-0.0001220703125	\\
0.537621520841657	0.0003662109375	\\
0.537665912016691	6.103515625e-05	\\
0.537710303191726	-0.00018310546875	\\
0.53775469436676	-9.1552734375e-05	\\
0.537799085541794	0	\\
0.537843476716829	0.000244140625	\\
0.537887867891863	0	\\
0.537932259066897	-0.000213623046875	\\
0.537976650241932	0.000244140625	\\
0.538021041416966	0.000640869140625	\\
0.538065432592001	0.000274658203125	\\
0.538109823767035	-6.103515625e-05	\\
0.53815421494207	0.0001220703125	\\
0.538198606117104	0	\\
0.538242997292138	-3.0517578125e-05	\\
0.538287388467173	0.000152587890625	\\
0.538331779642207	0	\\
0.538376170817242	0.00030517578125	\\
0.538420561992276	0.000274658203125	\\
0.53846495316731	0.000701904296875	\\
0.538509344342345	0.00067138671875	\\
0.538553735517379	0.000396728515625	\\
0.538598126692414	0.000152587890625	\\
0.538642517867448	9.1552734375e-05	\\
0.538686909042482	0.00054931640625	\\
0.538731300217517	0.000335693359375	\\
0.538775691392551	0.000244140625	\\
0.538820082567586	0.000457763671875	\\
0.53886447374262	0.0006103515625	\\
0.538908864917654	0.0003662109375	\\
0.538953256092689	0.000244140625	\\
0.538997647267723	0.000335693359375	\\
0.539042038442758	0.00030517578125	\\
0.539086429617792	0.000244140625	\\
0.539130820792826	0.000396728515625	\\
0.539175211967861	0.00067138671875	\\
0.539219603142895	0.00067138671875	\\
0.53926399431793	0.000732421875	\\
0.539308385492964	0.001007080078125	\\
0.539352776667998	0.0010986328125	\\
0.539397167843033	0.0010986328125	\\
0.539441559018067	0.001312255859375	\\
0.539485950193102	0.001190185546875	\\
0.539530341368136	0.001312255859375	\\
0.53957473254317	0.0010986328125	\\
0.539619123718205	0.001251220703125	\\
0.539663514893239	0.0020751953125	\\
0.539707906068274	0.00177001953125	\\
0.539752297243308	0.001953125	\\
0.539796688418342	0.002410888671875	\\
0.539841079593377	0.001800537109375	\\
0.539885470768411	0.002044677734375	\\
0.539929861943446	0.00262451171875	\\
0.53997425311848	0.00238037109375	\\
0.540018644293514	0.002349853515625	\\
0.540063035468549	0.00250244140625	\\
0.540107426643583	0.00250244140625	\\
0.540151817818618	0.00262451171875	\\
0.540196208993652	0.0020751953125	\\
0.540240600168686	0.00189208984375	\\
0.540284991343721	0.002197265625	\\
0.540329382518755	0.002471923828125	\\
0.54037377369379	0.002166748046875	\\
0.540418164868824	0.001983642578125	\\
0.540462556043859	0.001983642578125	\\
0.540506947218893	0.001678466796875	\\
0.540551338393927	0.001556396484375	\\
0.540595729568962	0.00140380859375	\\
0.540640120743996	0.001129150390625	\\
0.54068451191903	0.000640869140625	\\
0.540728903094065	0.000518798828125	\\
0.540773294269099	0.0006103515625	\\
0.540817685444134	0.000274658203125	\\
0.540862076619168	0.000213623046875	\\
0.540906467794202	0.00079345703125	\\
0.540950858969237	0.000823974609375	\\
0.540995250144271	0.00067138671875	\\
0.541039641319306	0.000335693359375	\\
0.54108403249434	0.000396728515625	\\
0.541128423669375	0.0008544921875	\\
0.541172814844409	0.000762939453125	\\
0.541217206019443	0.00048828125	\\
0.541261597194478	0.000946044921875	\\
0.541305988369512	0.0008544921875	\\
0.541350379544547	0.000518798828125	\\
0.541394770719581	0.000885009765625	\\
0.541439161894615	0.00079345703125	\\
0.54148355306965	0.000579833984375	\\
0.541527944244684	0.000579833984375	\\
0.541572335419719	0.00042724609375	\\
0.541616726594753	6.103515625e-05	\\
0.541661117769787	3.0517578125e-05	\\
0.541705508944822	0.000244140625	\\
0.541749900119856	6.103515625e-05	\\
0.541794291294891	-0.000152587890625	\\
0.541838682469925	-0.000396728515625	\\
0.541883073644959	-0.000213623046875	\\
0.541927464819994	-0.00030517578125	\\
0.541971855995028	-0.000762939453125	\\
0.542016247170063	-0.000457763671875	\\
0.542060638345097	-0.000244140625	\\
0.542105029520131	-0.000274658203125	\\
0.542149420695166	-0.000213623046875	\\
0.5421938118702	-0.00018310546875	\\
0.542238203045235	-0.000244140625	\\
0.542282594220269	-0.00048828125	\\
0.542326985395303	-0.00030517578125	\\
0.542371376570338	-0.000213623046875	\\
0.542415767745372	-0.00030517578125	\\
0.542460158920407	-0.00042724609375	\\
0.542504550095441	-0.000396728515625	\\
0.542548941270475	-0.000518798828125	\\
0.54259333244551	-0.000579833984375	\\
0.542637723620544	-0.000152587890625	\\
0.542682114795579	-0.000213623046875	\\
0.542726505970613	-6.103515625e-05	\\
0.542770897145647	-6.103515625e-05	\\
0.542815288320682	-0.0003662109375	\\
0.542859679495716	3.0517578125e-05	\\
0.542904070670751	0.0001220703125	\\
0.542948461845785	-0.000213623046875	\\
0.542992853020819	0.000213623046875	\\
0.543037244195854	-6.103515625e-05	\\
0.543081635370888	-0.000518798828125	\\
0.543126026545923	0	\\
0.543170417720957	-0.0001220703125	\\
0.543214808895992	-6.103515625e-05	\\
0.543259200071026	-0.000457763671875	\\
0.54330359124606	-0.00054931640625	\\
0.543347982421095	-0.00048828125	\\
0.543392373596129	-0.00079345703125	\\
0.543436764771164	-0.0006103515625	\\
0.543481155946198	-0.00079345703125	\\
0.543525547121232	-0.001007080078125	\\
0.543569938296267	-0.0008544921875	\\
0.543614329471301	-0.000762939453125	\\
0.543658720646335	-0.000640869140625	\\
0.54370311182137	-0.00103759765625	\\
0.543747502996404	-0.00115966796875	\\
0.543791894171439	-0.000762939453125	\\
0.543836285346473	-0.001220703125	\\
0.543880676521508	-0.001495361328125	\\
0.543925067696542	-0.001312255859375	\\
0.543969458871576	-0.001129150390625	\\
0.544013850046611	-0.001495361328125	\\
0.544058241221645	-0.001190185546875	\\
0.54410263239668	-0.0013427734375	\\
0.544147023571714	-0.0009765625	\\
0.544191414746748	-0.000823974609375	\\
0.544235805921783	-0.00054931640625	\\
0.544280197096817	-0.000579833984375	\\
0.544324588271852	-0.001068115234375	\\
0.544368979446886	-0.000640869140625	\\
0.54441337062192	-0.000640869140625	\\
0.544457761796955	-0.0006103515625	\\
0.544502152971989	-0.00054931640625	\\
0.544546544147024	-0.000640869140625	\\
0.544590935322058	-0.00054931640625	\\
0.544635326497092	-0.0001220703125	\\
0.544679717672127	-0.000396728515625	\\
0.544724108847161	-0.000732421875	\\
0.544768500022196	-0.00067138671875	\\
0.54481289119723	-0.00079345703125	\\
0.544857282372264	-0.000946044921875	\\
0.544901673547299	-0.00067138671875	\\
0.544946064722333	-0.00054931640625	\\
0.544990455897368	-0.00103759765625	\\
0.545034847072402	-0.000762939453125	\\
0.545079238247436	-0.000640869140625	\\
0.545123629422471	-0.0006103515625	\\
0.545168020597505	-0.00048828125	\\
0.54521241177254	-0.000213623046875	\\
0.545256802947574	-6.103515625e-05	\\
0.545301194122608	-3.0517578125e-05	\\
0.545345585297643	-0.00054931640625	\\
0.545389976472677	-0.00054931640625	\\
0.545434367647712	9.1552734375e-05	\\
0.545478758822746	-0.000518798828125	\\
0.54552314999778	-0.000640869140625	\\
0.545567541172815	-0.000823974609375	\\
0.545611932347849	-0.001068115234375	\\
0.545656323522884	-0.000732421875	\\
0.545700714697918	-0.00079345703125	\\
0.545745105872952	-0.000946044921875	\\
0.545789497047987	-0.001129150390625	\\
0.545833888223021	-0.001251220703125	\\
0.545878279398056	-0.00152587890625	\\
0.54592267057309	-0.001434326171875	\\
0.545967061748124	-0.001007080078125	\\
0.546011452923159	-0.00091552734375	\\
0.546055844098193	-0.001007080078125	\\
0.546100235273228	-0.001068115234375	\\
0.546144626448262	-0.00079345703125	\\
0.546189017623297	-0.000762939453125	\\
0.546233408798331	-0.00115966796875	\\
0.546277799973365	-0.001129150390625	\\
0.5463221911484	-0.00091552734375	\\
0.546366582323434	-0.00091552734375	\\
0.546410973498468	-0.0008544921875	\\
0.546455364673503	-0.000732421875	\\
0.546499755848537	-0.000640869140625	\\
0.546544147023572	-0.00030517578125	\\
0.546588538198606	-0.00018310546875	\\
0.546632929373641	-0.000213623046875	\\
0.546677320548675	-0.000244140625	\\
0.546721711723709	-0.000213623046875	\\
0.546766102898744	-6.103515625e-05	\\
0.546810494073778	0.0003662109375	\\
0.546854885248813	0.0003662109375	\\
0.546899276423847	0.000335693359375	\\
0.546943667598881	0.000885009765625	\\
0.546988058773916	0.000946044921875	\\
0.54703244994895	0.001007080078125	\\
0.547076841123985	0.00103759765625	\\
0.547121232299019	0.0009765625	\\
0.547165623474053	0.0010986328125	\\
0.547210014649088	0.0010986328125	\\
0.547254405824122	0.001007080078125	\\
0.547298796999157	0.0010986328125	\\
0.547343188174191	0.001312255859375	\\
0.547387579349225	0.00115966796875	\\
0.54743197052426	0.001190185546875	\\
0.547476361699294	0.001373291015625	\\
0.547520752874329	0.001312255859375	\\
0.547565144049363	0.001373291015625	\\
0.547609535224397	0.001251220703125	\\
0.547653926399432	0.0010986328125	\\
0.547698317574466	0.00177001953125	\\
0.547742708749501	0.001556396484375	\\
0.547787099924535	0.0009765625	\\
0.547831491099569	0.001129150390625	\\
0.547875882274604	0.00103759765625	\\
0.547920273449638	0.0010986328125	\\
0.547964664624673	0.00079345703125	\\
0.548009055799707	0.000701904296875	\\
0.548053446974741	0.001129150390625	\\
0.548097838149776	0.00067138671875	\\
0.54814222932481	0.0006103515625	\\
0.548186620499845	0.000518798828125	\\
0.548231011674879	0.000701904296875	\\
0.548275402849913	0.000885009765625	\\
0.548319794024948	0.00054931640625	\\
0.548364185199982	0.00030517578125	\\
0.548408576375017	-3.0517578125e-05	\\
0.548452967550051	-0.00018310546875	\\
0.548497358725085	-0.0001220703125	\\
0.54854174990012	-0.0003662109375	\\
0.548586141075154	0.0003662109375	\\
0.548630532250189	0.001190185546875	\\
0.548674923425223	0.00079345703125	\\
0.548719314600257	0.000701904296875	\\
0.548763705775292	0.0010986328125	\\
0.548808096950326	0.000946044921875	\\
0.548852488125361	0.001220703125	\\
0.548896879300395	0.001739501953125	\\
0.54894127047543	0.001678466796875	\\
0.548985661650464	0.0018310546875	\\
0.549030052825498	0.0020751953125	\\
0.549074444000533	0.00189208984375	\\
0.549118835175567	0.001983642578125	\\
0.549163226350602	0.002410888671875	\\
0.549207617525636	0.00238037109375	\\
0.54925200870067	0.002349853515625	\\
0.549296399875705	0.00286865234375	\\
0.549340791050739	0.003173828125	\\
0.549385182225773	0.003173828125	\\
0.549429573400808	0.00335693359375	\\
0.549473964575842	0.003082275390625	\\
0.549518355750877	0.003143310546875	\\
0.549562746925911	0.003387451171875	\\
0.549607138100946	0.0032958984375	\\
0.54965152927598	0.0030517578125	\\
0.549695920451014	0.002655029296875	\\
0.549740311626049	0.0025634765625	\\
0.549784702801083	0.002532958984375	\\
0.549829093976118	0.002716064453125	\\
0.549873485151152	0.00286865234375	\\
0.549917876326186	0.002655029296875	\\
0.549962267501221	0.00250244140625	\\
0.550006658676255	0.00225830078125	\\
0.55005104985129	0.00177001953125	\\
0.550095441026324	0.001922607421875	\\
0.550139832201358	0.002044677734375	\\
0.550184223376393	0.0018310546875	\\
0.550228614551427	0.0018310546875	\\
0.550273005726462	0.001739501953125	\\
0.550317396901496	0.001617431640625	\\
0.55036178807653	0.0015869140625	\\
0.550406179251565	0.00152587890625	\\
0.550450570426599	0.0015869140625	\\
0.550494961601634	0.00140380859375	\\
0.550539352776668	0.00115966796875	\\
0.550583743951702	0.00115966796875	\\
0.550628135126737	0.00128173828125	\\
0.550672526301771	0.001007080078125	\\
0.550716917476806	0.001312255859375	\\
0.55076130865184	0.00140380859375	\\
0.550805699826874	0.0013427734375	\\
0.550850091001909	0.0013427734375	\\
0.550894482176943	0.00115966796875	\\
0.550938873351978	0.0013427734375	\\
0.550983264527012	0.0009765625	\\
0.551027655702046	0.00091552734375	\\
0.551072046877081	0.001068115234375	\\
0.551116438052115	0.001007080078125	\\
0.55116082922715	0.001678466796875	\\
0.551205220402184	0.001953125	\\
0.551249611577218	0.002105712890625	\\
0.551294002752253	0.001861572265625	\\
0.551338393927287	0.0018310546875	\\
0.551382785102322	0.002655029296875	\\
0.551427176277356	0.00262451171875	\\
0.55147156745239	0.002655029296875	\\
0.551515958627425	0.00262451171875	\\
0.551560349802459	0.002593994140625	\\
0.551604740977494	0.0025634765625	\\
0.551649132152528	0.002471923828125	\\
0.551693523327563	0.002288818359375	\\
0.551737914502597	0.001708984375	\\
0.551782305677631	0.002044677734375	\\
0.551826696852666	0.00238037109375	\\
0.5518710880277	0.001922607421875	\\
0.551915479202735	0.001983642578125	\\
0.551959870377769	0.001922607421875	\\
0.552004261552803	0.002349853515625	\\
0.552048652727838	0.002227783203125	\\
0.552093043902872	0.001800537109375	\\
0.552137435077906	0.002044677734375	\\
0.552181826252941	0.001922607421875	\\
0.552226217427975	0.001800537109375	\\
0.55227060860301	0.001861572265625	\\
0.552314999778044	0.001739501953125	\\
0.552359390953079	0.001495361328125	\\
0.552403782128113	0.00128173828125	\\
0.552448173303147	0.001129150390625	\\
0.552492564478182	0.001556396484375	\\
0.552536955653216	0.001251220703125	\\
0.552581346828251	0.000885009765625	\\
0.552625738003285	0.00140380859375	\\
0.552670129178319	0.000701904296875	\\
0.552714520353354	0.000579833984375	\\
0.552758911528388	0.001251220703125	\\
0.552803302703423	0.001434326171875	\\
0.552847693878457	0.001190185546875	\\
0.552892085053491	0.001190185546875	\\
0.552936476228526	0.001220703125	\\
0.55298086740356	0.0010986328125	\\
0.553025258578595	0.001220703125	\\
0.553069649753629	0.001007080078125	\\
0.553114040928663	0.000946044921875	\\
0.553158432103698	0.001190185546875	\\
0.553202823278732	0.000946044921875	\\
0.553247214453767	0.001617431640625	\\
0.553291605628801	0.001800537109375	\\
0.553335996803835	0.001983642578125	\\
0.55338038797887	0.00250244140625	\\
0.553424779153904	0.00189208984375	\\
0.553469170328939	0.00213623046875	\\
0.553513561503973	0.002471923828125	\\
0.553557952679007	0.002288818359375	\\
0.553602343854042	0.002166748046875	\\
0.553646735029076	0.002288818359375	\\
0.553691126204111	0.001922607421875	\\
0.553735517379145	0.002166748046875	\\
0.553779908554179	0.002227783203125	\\
0.553824299729214	0.0023193359375	\\
0.553868690904248	0.002227783203125	\\
0.553913082079283	0.00201416015625	\\
0.553957473254317	0.002197265625	\\
0.554001864429351	0.002410888671875	\\
0.554046255604386	0.00250244140625	\\
0.55409064677942	0.002593994140625	\\
0.554135037954455	0.00262451171875	\\
0.554179429129489	0.00244140625	\\
0.554223820304523	0.002471923828125	\\
0.554268211479558	0.002471923828125	\\
0.554312602654592	0.00225830078125	\\
0.554356993829627	0.001922607421875	\\
0.554401385004661	0.001708984375	\\
0.554445776179695	0.001556396484375	\\
0.55449016735473	0.00189208984375	\\
0.554534558529764	0.001922607421875	\\
0.554578949704799	0.00115966796875	\\
0.554623340879833	0.000732421875	\\
0.554667732054868	0.000885009765625	\\
0.554712123229902	0.001007080078125	\\
0.554756514404936	0.000823974609375	\\
0.554800905579971	0.001068115234375	\\
0.554845296755005	0.000762939453125	\\
0.554889687930039	0.000640869140625	\\
0.554934079105074	0.000701904296875	\\
0.554978470280108	0.0006103515625	\\
0.555022861455143	0.00091552734375	\\
0.555067252630177	0.000762939453125	\\
0.555111643805212	0.00048828125	\\
0.555156034980246	0.000579833984375	\\
0.55520042615528	0.000579833984375	\\
0.555244817330315	0.00048828125	\\
0.555289208505349	0.0003662109375	\\
0.555333599680384	0.000579833984375	\\
0.555377990855418	0.000335693359375	\\
0.555422382030452	0.000732421875	\\
0.555466773205487	0.000885009765625	\\
0.555511164380521	0.000762939453125	\\
0.555555555555556	0.000762939453125	\\
0.55559994673059	0.0009765625	\\
0.555644337905624	0.0009765625	\\
0.555688729080659	0.000396728515625	\\
0.555733120255693	0.000885009765625	\\
0.555777511430728	0.00103759765625	\\
0.555821902605762	0.000885009765625	\\
0.555866293780796	0.000823974609375	\\
0.555910684955831	0.000823974609375	\\
0.555955076130865	0.0010986328125	\\
0.5559994673059	0.00103759765625	\\
0.556043858480934	0.00140380859375	\\
0.556088249655968	0.00164794921875	\\
0.556132640831003	0.0015869140625	\\
0.556177032006037	0.001708984375	\\
0.556221423181072	0.001678466796875	\\
0.556265814356106	0.00177001953125	\\
0.55631020553114	0.001800537109375	\\
0.556354596706175	0.001739501953125	\\
0.556398987881209	0.001800537109375	\\
0.556443379056244	0.001556396484375	\\
0.556487770231278	0.00177001953125	\\
0.556532161406312	0.001983642578125	\\
0.556576552581347	0.001800537109375	\\
0.556620943756381	0.002105712890625	\\
0.556665334931416	0.002685546875	\\
0.55670972610645	0.002777099609375	\\
0.556754117281485	0.002288818359375	\\
0.556798508456519	0.00244140625	\\
0.556842899631553	0.00274658203125	\\
0.556887290806588	0.00262451171875	\\
0.556931681981622	0.002593994140625	\\
0.556976073156656	0.002227783203125	\\
0.557020464331691	0.002197265625	\\
0.557064855506725	0.002227783203125	\\
0.55710924668176	0.002288818359375	\\
0.557153637856794	0.002899169921875	\\
0.557198029031828	0.0028076171875	\\
0.557242420206863	0.00262451171875	\\
0.557286811381897	0.00262451171875	\\
0.557331202556932	0.002410888671875	\\
0.557375593731966	0.002777099609375	\\
0.557419984907001	0.0028076171875	\\
0.557464376082035	0.002105712890625	\\
0.557508767257069	0.0018310546875	\\
0.557553158432104	0.001678466796875	\\
0.557597549607138	0.001983642578125	\\
0.557641940782173	0.002197265625	\\
0.557686331957207	0.00189208984375	\\
0.557730723132241	0.002044677734375	\\
0.557775114307276	0.002105712890625	\\
0.55781950548231	0.00213623046875	\\
0.557863896657344	0.00244140625	\\
0.557908287832379	0.00225830078125	\\
0.557952679007413	0.00262451171875	\\
0.557997070182448	0.002838134765625	\\
0.558041461357482	0.002166748046875	\\
0.558085852532517	0.001800537109375	\\
0.558130243707551	0.00177001953125	\\
0.558174634882585	0.001434326171875	\\
0.55821902605762	0.00140380859375	\\
0.558263417232654	0.001251220703125	\\
0.558307808407689	0.001739501953125	\\
0.558352199582723	0.001373291015625	\\
0.558396590757757	0.00115966796875	\\
0.558440981932792	0.001678466796875	\\
0.558485373107826	0.00177001953125	\\
0.558529764282861	0.001800537109375	\\
0.558574155457895	0.001678466796875	\\
0.558618546632929	0.001861572265625	\\
0.558662937807964	0.001953125	\\
0.558707328982998	0.00189208984375	\\
0.558751720158033	0.001739501953125	\\
0.558796111333067	0.001739501953125	\\
0.558840502508101	0.002105712890625	\\
0.558884893683136	0.0023193359375	\\
0.55892928485817	0.001983642578125	\\
0.558973676033205	0.002105712890625	\\
0.559018067208239	0.0023193359375	\\
0.559062458383273	0.002655029296875	\\
0.559106849558308	0.0028076171875	\\
0.559151240733342	0.002685546875	\\
0.559195631908377	0.002532958984375	\\
0.559240023083411	0.00299072265625	\\
0.559284414258445	0.002960205078125	\\
0.55932880543348	0.002532958984375	\\
0.559373196608514	0.002716064453125	\\
0.559417587783549	0.0025634765625	\\
0.559461978958583	0.0028076171875	\\
0.559506370133617	0.003082275390625	\\
0.559550761308652	0.003570556640625	\\
0.559595152483686	0.003631591796875	\\
0.559639543658721	0.003173828125	\\
0.559683934833755	0.003173828125	\\
0.559728326008789	0.002593994140625	\\
0.559772717183824	0.002197265625	\\
0.559817108358858	0.00225830078125	\\
0.559861499533893	0.002288818359375	\\
0.559905890708927	0.002410888671875	\\
0.559950281883961	0.002288818359375	\\
0.559994673058996	0.00262451171875	\\
0.56003906423403	0.0029296875	\\
0.560083455409065	0.002349853515625	\\
0.560127846584099	0.001953125	\\
0.560172237759134	0.00213623046875	\\
0.560216628934168	0.002166748046875	\\
0.560261020109202	0.001983642578125	\\
0.560305411284237	0.002166748046875	\\
0.560349802459271	0.002166748046875	\\
0.560394193634306	0.002197265625	\\
0.56043858480934	0.002227783203125	\\
0.560482975984374	0.0025634765625	\\
0.560527367159409	0.0025634765625	\\
0.560571758334443	0.001983642578125	\\
0.560616149509477	0.00201416015625	\\
0.560660540684512	0.002593994140625	\\
0.560704931859546	0.002777099609375	\\
0.560749323034581	0.0028076171875	\\
0.560793714209615	0.0028076171875	\\
0.56083810538465	0.002777099609375	\\
0.560882496559684	0.0025634765625	\\
0.560926887734718	0.0025634765625	\\
0.560971278909753	0.002593994140625	\\
0.561015670084787	0.00244140625	\\
0.561060061259822	0.002716064453125	\\
0.561104452434856	0.002227783203125	\\
0.56114884360989	0.001983642578125	\\
0.561193234784925	0.002197265625	\\
0.561237625959959	0.0018310546875	\\
0.561282017134994	0.001708984375	\\
0.561326408310028	0.00189208984375	\\
0.561370799485062	0.0020751953125	\\
0.561415190660097	0.001800537109375	\\
0.561459581835131	0.00146484375	\\
0.561503973010166	0.001678466796875	\\
0.5615483641852	0.001739501953125	\\
0.561592755360234	0.001373291015625	\\
0.561637146535269	0.001800537109375	\\
0.561681537710303	0.001708984375	\\
0.561725928885338	0.000885009765625	\\
0.561770320060372	0.001190185546875	\\
0.561814711235406	0.00103759765625	\\
0.561859102410441	0.000518798828125	\\
0.561903493585475	0.000762939453125	\\
0.56194788476051	0.000885009765625	\\
0.561992275935544	0.000762939453125	\\
0.562036667110578	0.00042724609375	\\
0.562081058285613	0.000823974609375	\\
0.562125449460647	0.00103759765625	\\
0.562169840635682	0.00067138671875	\\
0.562214231810716	0.00067138671875	\\
0.56225862298575	0.00042724609375	\\
0.562303014160785	0.000518798828125	\\
0.562347405335819	0.001007080078125	\\
0.562391796510854	0.0010986328125	\\
0.562436187685888	0.000762939453125	\\
0.562480578860922	0.001068115234375	\\
0.562524970035957	0.00115966796875	\\
0.562569361210991	0.001617431640625	\\
0.562613752386026	0.001556396484375	\\
0.56265814356106	0.0015869140625	\\
0.562702534736094	0.001556396484375	\\
0.562746925911129	0.001495361328125	\\
0.562791317086163	0.0020751953125	\\
0.562835708261198	0.001708984375	\\
0.562880099436232	0.00152587890625	\\
0.562924490611266	0.002105712890625	\\
0.562968881786301	0.002471923828125	\\
0.563013272961335	0.0025634765625	\\
0.56305766413637	0.002471923828125	\\
0.563102055311404	0.0023193359375	\\
0.563146446486439	0.002349853515625	\\
0.563190837661473	0.002044677734375	\\
0.563235228836507	0.00238037109375	\\
0.563279620011542	0.0023193359375	\\
0.563324011186576	0.00152587890625	\\
0.563368402361611	0.001922607421875	\\
0.563412793536645	0.001922607421875	\\
0.563457184711679	0.00140380859375	\\
0.563501575886714	0.001495361328125	\\
0.563545967061748	0.001373291015625	\\
0.563590358236783	0.001556396484375	\\
0.563634749411817	0.001617431640625	\\
0.563679140586851	0.001800537109375	\\
0.563723531761886	0.001678466796875	\\
0.56376792293692	0.001434326171875	\\
0.563812314111955	0.0009765625	\\
0.563856705286989	0.000640869140625	\\
0.563901096462023	0.000762939453125	\\
0.563945487637058	0.00103759765625	\\
0.563989878812092	0.00091552734375	\\
0.564034269987127	0.001251220703125	\\
0.564078661162161	0.001220703125	\\
0.564123052337195	0.00079345703125	\\
0.56416744351223	0.0009765625	\\
0.564211834687264	0.000762939453125	\\
0.564256225862299	0.000762939453125	\\
0.564300617037333	0.001068115234375	\\
0.564345008212367	0.00048828125	\\
0.564389399387402	0.00018310546875	\\
0.564433790562436	0.00048828125	\\
0.564478181737471	0.00079345703125	\\
0.564522572912505	0.001251220703125	\\
0.564566964087539	0.00079345703125	\\
0.564611355262574	0.0010986328125	\\
0.564655746437608	0.00152587890625	\\
0.564700137612643	0.001708984375	\\
0.564744528787677	0.00189208984375	\\
0.564788919962711	0.00152587890625	\\
0.564833311137746	0.001739501953125	\\
0.56487770231278	0.001983642578125	\\
0.564922093487815	0.001983642578125	\\
0.564966484662849	0.0020751953125	\\
0.565010875837883	0.002105712890625	\\
0.565055267012918	0.0020751953125	\\
0.565099658187952	0.00201416015625	\\
0.565144049362987	0.002044677734375	\\
0.565188440538021	0.002288818359375	\\
0.565232831713055	0.00250244140625	\\
0.56527722288809	0.002899169921875	\\
0.565321614063124	0.002655029296875	\\
0.565366005238159	0.0023193359375	\\
0.565410396413193	0.002532958984375	\\
0.565454787588227	0.002471923828125	\\
0.565499178763262	0.002166748046875	\\
0.565543569938296	0.00213623046875	\\
0.565587961113331	0.00213623046875	\\
0.565632352288365	0.001678466796875	\\
0.565676743463399	0.00164794921875	\\
0.565721134638434	0.001434326171875	\\
0.565765525813468	0.0015869140625	\\
0.565809916988503	0.00177001953125	\\
0.565854308163537	0.001373291015625	\\
0.565898699338572	0.0010986328125	\\
0.565943090513606	0.00128173828125	\\
0.56598748168864	0.001312255859375	\\
0.566031872863675	0.000823974609375	\\
0.566076264038709	0.000640869140625	\\
0.566120655213744	0.000518798828125	\\
0.566165046388778	0.0003662109375	\\
0.566209437563812	0.000213623046875	\\
0.566253828738847	0.000274658203125	\\
0.566298219913881	0.000244140625	\\
0.566342611088915	0.000152587890625	\\
0.56638700226395	3.0517578125e-05	\\
0.566431393438984	0.0001220703125	\\
0.566475784614019	9.1552734375e-05	\\
0.566520175789053	0.000396728515625	\\
0.566564566964088	0.000244140625	\\
0.566608958139122	-0.0001220703125	\\
0.566653349314156	0.00030517578125	\\
0.566697740489191	0.0001220703125	\\
0.566742131664225	-0.000244140625	\\
0.56678652283926	-0.00018310546875	\\
0.566830914014294	3.0517578125e-05	\\
0.566875305189328	0.0003662109375	\\
0.566919696364363	-9.1552734375e-05	\\
0.566964087539397	0	\\
0.567008478714432	-0.000152587890625	\\
0.567052869889466	-0.000244140625	\\
0.5670972610645	-0.000335693359375	\\
0.567141652239535	-0.000457763671875	\\
0.567186043414569	3.0517578125e-05	\\
0.567230434589604	-0.000335693359375	\\
0.567274825764638	-0.000274658203125	\\
0.567319216939672	0.000457763671875	\\
0.567363608114707	0.000213623046875	\\
0.567407999289741	0.00018310546875	\\
0.567452390464776	0.00048828125	\\
0.56749678163981	0.0001220703125	\\
0.567541172814844	0.000396728515625	\\
0.567585563989879	0.000518798828125	\\
0.567629955164913	0.000244140625	\\
0.567674346339948	0.00115966796875	\\
0.567718737514982	0.001312255859375	\\
0.567763128690016	0.000701904296875	\\
0.567807519865051	0.000457763671875	\\
0.567851911040085	0.000396728515625	\\
0.56789630221512	0.000732421875	\\
0.567940693390154	0.00103759765625	\\
0.567985084565188	0.000885009765625	\\
0.568029475740223	0.00091552734375	\\
0.568073866915257	0.00091552734375	\\
0.568118258090292	0.001129150390625	\\
0.568162649265326	0.0010986328125	\\
0.56820704044036	0.000885009765625	\\
0.568251431615395	0.00079345703125	\\
0.568295822790429	0.001007080078125	\\
0.568340213965464	0.001190185546875	\\
0.568384605140498	0.00128173828125	\\
0.568428996315532	0.001495361328125	\\
0.568473387490567	0.0015869140625	\\
0.568517778665601	0.0018310546875	\\
0.568562169840636	0.001556396484375	\\
0.56860656101567	0.001495361328125	\\
0.568650952190705	0.001800537109375	\\
0.568695343365739	0.0018310546875	\\
0.568739734540773	0.00189208984375	\\
0.568784125715808	0.002044677734375	\\
0.568828516890842	0.002288818359375	\\
0.568872908065877	0.001983642578125	\\
0.568917299240911	0.00189208984375	\\
0.568961690415945	0.00225830078125	\\
0.56900608159098	0.002105712890625	\\
0.569050472766014	0.002166748046875	\\
0.569094863941048	0.0025634765625	\\
0.569139255116083	0.00244140625	\\
0.569183646291117	0.002288818359375	\\
0.569228037466152	0.002349853515625	\\
0.569272428641186	0.0028076171875	\\
0.569316819816221	0.00250244140625	\\
0.569361210991255	0.002532958984375	\\
0.569405602166289	0.002655029296875	\\
0.569449993341324	0.00244140625	\\
0.569494384516358	0.002655029296875	\\
0.569538775691393	0.0023193359375	\\
0.569583166866427	0.002044677734375	\\
0.569627558041461	0.0025634765625	\\
0.569671949216496	0.00213623046875	\\
0.56971634039153	0.0015869140625	\\
0.569760731566565	0.001739501953125	\\
0.569805122741599	0.001556396484375	\\
0.569849513916633	0.00128173828125	\\
0.569893905091668	0.001190185546875	\\
0.569938296266702	0.00140380859375	\\
0.569982687441737	0.001007080078125	\\
0.570027078616771	0.00042724609375	\\
0.570071469791805	0.000335693359375	\\
0.57011586096684	0.000457763671875	\\
0.570160252141874	0.00067138671875	\\
0.570204643316909	0.0003662109375	\\
0.570249034491943	9.1552734375e-05	\\
0.570293425666977	0.000732421875	\\
0.570337816842012	0.001220703125	\\
0.570382208017046	0.000823974609375	\\
0.570426599192081	-0.00018310546875	\\
0.570470990367115	0.000335693359375	\\
0.570515381542149	0.00042724609375	\\
0.570559772717184	0.000152587890625	\\
0.570604163892218	0.000518798828125	\\
0.570648555067253	0.000732421875	\\
0.570692946242287	0.000701904296875	\\
0.570737337417321	0.000732421875	\\
0.570781728592356	0.000701904296875	\\
0.57082611976739	0.001007080078125	\\
0.570870510942425	0.001220703125	\\
0.570914902117459	0.000946044921875	\\
0.570959293292493	0.000640869140625	\\
0.571003684467528	0.001007080078125	\\
0.571048075642562	0.000946044921875	\\
0.571092466817597	0.00067138671875	\\
0.571136857992631	0.000885009765625	\\
0.571181249167665	0.00091552734375	\\
0.5712256403427	0.00054931640625	\\
0.571270031517734	6.103515625e-05	\\
0.571314422692769	0.000213623046875	\\
0.571358813867803	0.000335693359375	\\
0.571403205042837	0.00042724609375	\\
0.571447596217872	0.0003662109375	\\
0.571491987392906	3.0517578125e-05	\\
0.571536378567941	0.000244140625	\\
0.571580769742975	0.000152587890625	\\
0.57162516091801	0	\\
0.571669552093044	0.00030517578125	\\
0.571713943268078	9.1552734375e-05	\\
0.571758334443113	-0.00030517578125	\\
0.571802725618147	-0.00030517578125	\\
0.571847116793182	-0.00042724609375	\\
0.571891507968216	-0.000213623046875	\\
0.57193589914325	-0.00048828125	\\
0.571980290318285	-0.00067138671875	\\
0.572024681493319	-0.000335693359375	\\
0.572069072668354	-0.0003662109375	\\
0.572113463843388	-0.000396728515625	\\
0.572157855018422	-0.00054931640625	\\
0.572202246193457	-0.00091552734375	\\
0.572246637368491	-0.000823974609375	\\
0.572291028543526	-0.0008544921875	\\
0.57233541971856	-0.001220703125	\\
0.572379810893594	-0.000640869140625	\\
0.572424202068629	-0.000823974609375	\\
0.572468593243663	-0.001129150390625	\\
0.572512984418698	-0.000732421875	\\
0.572557375593732	-0.0003662109375	\\
0.572601766768766	-0.0001220703125	\\
0.572646157943801	0.000244140625	\\
0.572690549118835	0.00042724609375	\\
0.57273494029387	0.00030517578125	\\
0.572779331468904	0.000579833984375	\\
0.572823722643938	0.000946044921875	\\
0.572868113818973	0.000823974609375	\\
0.572912504994007	0.0009765625	\\
0.572956896169042	0.00103759765625	\\
0.573001287344076	0.001220703125	\\
0.57304567851911	0.001556396484375	\\
0.573090069694145	0.001495361328125	\\
0.573134460869179	0.001251220703125	\\
0.573178852044214	0.001678466796875	\\
0.573223243219248	0.00164794921875	\\
0.573267634394282	0.001739501953125	\\
0.573312025569317	0.002105712890625	\\
0.573356416744351	0.00213623046875	\\
0.573400807919386	0.002349853515625	\\
0.57344519909442	0.002410888671875	\\
0.573489590269454	0.002105712890625	\\
0.573533981444489	0.0020751953125	\\
0.573578372619523	0.002410888671875	\\
0.573622763794558	0.00244140625	\\
0.573667154969592	0.002044677734375	\\
0.573711546144626	0.002197265625	\\
0.573755937319661	0.00189208984375	\\
0.573800328494695	0.00177001953125	\\
0.57384471966973	0.00201416015625	\\
0.573889110844764	0.001678466796875	\\
0.573933502019798	0.001983642578125	\\
0.573977893194833	0.0020751953125	\\
0.574022284369867	0.0015869140625	\\
0.574066675544902	0.001129150390625	\\
0.574111066719936	0.001312255859375	\\
0.57415545789497	0.0010986328125	\\
0.574199849070005	0.000823974609375	\\
0.574244240245039	0.00152587890625	\\
0.574288631420074	0.001068115234375	\\
0.574333022595108	0.000640869140625	\\
0.574377413770143	0.000640869140625	\\
0.574421804945177	0.00054931640625	\\
0.574466196120211	0.000946044921875	\\
0.574510587295246	0.000823974609375	\\
0.57455497847028	0.000213623046875	\\
0.574599369645315	0.000244140625	\\
0.574643760820349	0.00042724609375	\\
0.574688151995383	0.00030517578125	\\
0.574732543170418	0.000396728515625	\\
0.574776934345452	-9.1552734375e-05	\\
0.574821325520486	-0.000335693359375	\\
0.574865716695521	-0.000335693359375	\\
0.574910107870555	-0.000213623046875	\\
0.57495449904559	-9.1552734375e-05	\\
0.574998890220624	-9.1552734375e-05	\\
0.575043281395659	-0.000213623046875	\\
0.575087672570693	-0.000762939453125	\\
0.575132063745727	-0.00042724609375	\\
0.575176454920762	-0.000335693359375	\\
0.575220846095796	-0.000579833984375	\\
0.575265237270831	-0.000213623046875	\\
0.575309628445865	-9.1552734375e-05	\\
0.575354019620899	0	\\
0.575398410795934	0.000152587890625	\\
0.575442801970968	0.000579833984375	\\
0.575487193146003	0.00048828125	\\
0.575531584321037	6.103515625e-05	\\
0.575575975496071	0.00030517578125	\\
0.575620366671106	0.000457763671875	\\
0.57566475784614	0.0001220703125	\\
0.575709149021175	0.000274658203125	\\
0.575753540196209	0.00103759765625	\\
0.575797931371243	0.001007080078125	\\
0.575842322546278	0.001373291015625	\\
0.575886713721312	0.001220703125	\\
0.575931104896347	0.001373291015625	\\
0.575975496071381	0.001708984375	\\
0.576019887246415	0.00146484375	\\
0.57606427842145	0.001800537109375	\\
0.576108669596484	0.001800537109375	\\
0.576153060771519	0.001312255859375	\\
0.576197451946553	0.00115966796875	\\
0.576241843121587	0.001312255859375	\\
0.576286234296622	0.00140380859375	\\
0.576330625471656	0.00128173828125	\\
0.576375016646691	0.00152587890625	\\
0.576419407821725	0.00128173828125	\\
0.576463798996759	0.00128173828125	\\
0.576508190171794	0.001373291015625	\\
0.576552581346828	0.00115966796875	\\
0.576596972521863	0.00115966796875	\\
0.576641363696897	0.001190185546875	\\
0.576685754871931	0.001220703125	\\
0.576730146046966	0.001434326171875	\\
0.576774537222	0.001068115234375	\\
0.576818928397035	0.001007080078125	\\
0.576863319572069	0.0009765625	\\
0.576907710747103	0.0008544921875	\\
0.576952101922138	0.00091552734375	\\
0.576996493097172	0.0006103515625	\\
0.577040884272207	0.000335693359375	\\
0.577085275447241	0.00018310546875	\\
0.577129666622275	0.000213623046875	\\
0.57717405779731	0.00018310546875	\\
0.577218448972344	9.1552734375e-05	\\
0.577262840147379	9.1552734375e-05	\\
0.577307231322413	-9.1552734375e-05	\\
0.577351622497448	0	\\
0.577396013672482	-0.0001220703125	\\
0.577440404847516	-0.000213623046875	\\
0.577484796022551	-0.00018310546875	\\
0.577529187197585	-0.000762939453125	\\
0.57757357837262	-0.001007080078125	\\
0.577617969547654	-0.00103759765625	\\
0.577662360722688	-0.00164794921875	\\
0.577706751897723	-0.001800537109375	\\
0.577751143072757	-0.00146484375	\\
0.577795534247792	-0.00164794921875	\\
0.577839925422826	-0.0018310546875	\\
0.57788431659786	-0.001953125	\\
0.577928707772895	-0.001617431640625	\\
0.577973098947929	-0.001800537109375	\\
0.578017490122964	-0.001953125	\\
0.578061881297998	-0.001739501953125	\\
0.578106272473032	-0.0018310546875	\\
0.578150663648067	-0.001617431640625	\\
0.578195054823101	-0.0015869140625	\\
0.578239445998136	-0.00189208984375	\\
0.57828383717317	-0.0018310546875	\\
0.578328228348204	-0.00189208984375	\\
0.578372619523239	-0.0020751953125	\\
0.578417010698273	-0.002044677734375	\\
0.578461401873308	-0.00140380859375	\\
0.578505793048342	-0.00067138671875	\\
0.578550184223376	-0.00079345703125	\\
0.578594575398411	-0.0003662109375	\\
0.578638966573445	-0.00067138671875	\\
0.57868335774848	-0.000579833984375	\\
0.578727748923514	-6.103515625e-05	\\
0.578772140098548	-0.00048828125	\\
0.578816531273583	-0.0003662109375	\\
0.578860922448617	-0.00030517578125	\\
0.578905313623652	-0.000213623046875	\\
0.578949704798686	0.000335693359375	\\
0.57899409597372	0.000213623046875	\\
0.579038487148755	-0.000244140625	\\
0.579082878323789	-0.000213623046875	\\
0.579127269498824	-0.000579833984375	\\
0.579171660673858	-0.0009765625	\\
0.579216051848892	-0.000946044921875	\\
0.579260443023927	-0.001251220703125	\\
0.579304834198961	-0.001556396484375	\\
0.579349225373996	-0.001556396484375	\\
0.57939361654903	-0.00140380859375	\\
0.579438007724065	-0.00146484375	\\
0.579482398899099	-0.00177001953125	\\
0.579526790074133	-0.002288818359375	\\
0.579571181249168	-0.00262451171875	\\
0.579615572424202	-0.002777099609375	\\
0.579659963599236	-0.003326416015625	\\
0.579704354774271	-0.00341796875	\\
0.579748745949305	-0.00311279296875	\\
0.57979313712434	-0.00323486328125	\\
0.579837528299374	-0.0035400390625	\\
0.579881919474408	-0.0040283203125	\\
0.579926310649443	-0.00396728515625	\\
0.579970701824477	-0.0037841796875	\\
0.580015092999512	-0.00384521484375	\\
0.580059484174546	-0.003173828125	\\
0.580103875349581	-0.0032958984375	\\
0.580148266524615	-0.00347900390625	\\
0.580192657699649	-0.00323486328125	\\
0.580237048874684	-0.00347900390625	\\
0.580281440049718	-0.00323486328125	\\
0.580325831224753	-0.003143310546875	\\
0.580370222399787	-0.00323486328125	\\
0.580414613574821	-0.003021240234375	\\
0.580459004749856	-0.002777099609375	\\
0.58050339592489	-0.002197265625	\\
0.580547787099924	-0.00201416015625	\\
0.580592178274959	-0.002166748046875	\\
0.580636569449993	-0.00164794921875	\\
0.580680960625028	-0.00140380859375	\\
0.580725351800062	-0.00152587890625	\\
0.580769742975097	-0.001190185546875	\\
0.580814134150131	-0.001251220703125	\\
0.580858525325165	-0.00146484375	\\
0.5809029165002	-0.001251220703125	\\
0.580947307675234	-0.001129150390625	\\
0.580991698850269	-0.0009765625	\\
0.581036090025303	-0.000579833984375	\\
0.581080481200337	-0.000457763671875	\\
0.581124872375372	-0.00091552734375	\\
0.581169263550406	-0.00128173828125	\\
0.581213654725441	-0.001007080078125	\\
0.581258045900475	-0.0008544921875	\\
0.581302437075509	-0.00091552734375	\\
0.581346828250544	-0.000885009765625	\\
0.581391219425578	-0.0008544921875	\\
0.581435610600613	-0.001312255859375	\\
0.581480001775647	-0.001373291015625	\\
0.581524392950681	-0.001068115234375	\\
0.581568784125716	-0.0008544921875	\\
0.58161317530075	-0.001220703125	\\
0.581657566475785	-0.001434326171875	\\
0.581701957650819	-0.0013427734375	\\
0.581746348825853	-0.001556396484375	\\
0.581790740000888	-0.00177001953125	\\
0.581835131175922	-0.001922607421875	\\
0.581879522350957	-0.002197265625	\\
0.581923913525991	-0.00225830078125	\\
0.581968304701025	-0.0023193359375	\\
0.58201269587606	-0.002349853515625	\\
0.582057087051094	-0.00244140625	\\
0.582101478226129	-0.00274658203125	\\
0.582145869401163	-0.00299072265625	\\
0.582190260576197	-0.0028076171875	\\
0.582234651751232	-0.002960205078125	\\
0.582279042926266	-0.003204345703125	\\
0.582323434101301	-0.003265380859375	\\
0.582367825276335	-0.00323486328125	\\
0.582412216451369	-0.002716064453125	\\
0.582456607626404	-0.0029296875	\\
0.582500998801438	-0.003173828125	\\
0.582545389976473	-0.002960205078125	\\
0.582589781151507	-0.003265380859375	\\
0.582634172326541	-0.00360107421875	\\
0.582678563501576	-0.0035400390625	\\
0.58272295467661	-0.00335693359375	\\
0.582767345851645	-0.003173828125	\\
0.582811737026679	-0.003387451171875	\\
0.582856128201714	-0.003387451171875	\\
0.582900519376748	-0.00347900390625	\\
0.582944910551782	-0.003173828125	\\
0.582989301726817	-0.003143310546875	\\
0.583033692901851	-0.0032958984375	\\
0.583078084076886	-0.003265380859375	\\
0.58312247525192	-0.0030517578125	\\
0.583166866426954	-0.002685546875	\\
0.583211257601989	-0.002685546875	\\
0.583255648777023	-0.002593994140625	\\
0.583300039952057	-0.002471923828125	\\
0.583344431127092	-0.002227783203125	\\
0.583388822302126	-0.001983642578125	\\
0.583433213477161	-0.00238037109375	\\
0.583477604652195	-0.002471923828125	\\
0.58352199582723	-0.002105712890625	\\
0.583566387002264	-0.002105712890625	\\
0.583610778177298	-0.001983642578125	\\
0.583655169352333	-0.002044677734375	\\
0.583699560527367	-0.00152587890625	\\
0.583743951702402	-0.001495361328125	\\
0.583788342877436	-0.001678466796875	\\
0.58383273405247	-0.00152587890625	\\
0.583877125227505	-0.001495361328125	\\
0.583921516402539	-0.00140380859375	\\
0.583965907577574	-0.0013427734375	\\
0.584010298752608	-0.00140380859375	\\
0.584054689927642	-0.001068115234375	\\
0.584099081102677	-0.00079345703125	\\
0.584143472277711	-0.00103759765625	\\
0.584187863452746	-0.000885009765625	\\
0.58423225462778	-0.0006103515625	\\
0.584276645802814	-0.000396728515625	\\
0.584321036977849	-0.00067138671875	\\
0.584365428152883	-0.000518798828125	\\
0.584409819327918	-0.00054931640625	\\
0.584454210502952	-0.000701904296875	\\
0.584498601677986	-0.00048828125	\\
0.584542992853021	-0.000732421875	\\
0.584587384028055	-0.0008544921875	\\
0.58463177520309	-0.000732421875	\\
0.584676166378124	-0.0006103515625	\\
0.584720557553158	-0.000579833984375	\\
0.584764948728193	-0.000579833984375	\\
0.584809339903227	-0.0010986328125	\\
0.584853731078262	-0.00140380859375	\\
0.584898122253296	-0.0009765625	\\
0.58494251342833	-0.001220703125	\\
0.584986904603365	-0.001373291015625	\\
0.585031295778399	-0.001678466796875	\\
0.585075686953434	-0.0020751953125	\\
0.585120078128468	-0.002288818359375	\\
0.585164469303502	-0.00244140625	\\
0.585208860478537	-0.002532958984375	\\
0.585253251653571	-0.002685546875	\\
0.585297642828606	-0.003021240234375	\\
0.58534203400364	-0.00299072265625	\\
0.585386425178674	-0.002899169921875	\\
0.585430816353709	-0.003082275390625	\\
0.585475207528743	-0.0032958984375	\\
0.585519598703778	-0.003387451171875	\\
0.585563989878812	-0.003326416015625	\\
0.585608381053846	-0.003631591796875	\\
0.585652772228881	-0.003997802734375	\\
0.585697163403915	-0.00396728515625	\\
0.58574155457895	-0.004302978515625	\\
0.585785945753984	-0.004425048828125	\\
0.585830336929019	-0.00445556640625	\\
0.585874728104053	-0.004364013671875	\\
0.585919119279087	-0.00421142578125	\\
0.585963510454122	-0.004302978515625	\\
0.586007901629156	-0.00408935546875	\\
0.586052292804191	-0.003753662109375	\\
0.586096683979225	-0.003753662109375	\\
0.586141075154259	-0.00396728515625	\\
0.586185466329294	-0.00433349609375	\\
0.586229857504328	-0.0045166015625	\\
0.586274248679363	-0.004486083984375	\\
0.586318639854397	-0.004547119140625	\\
0.586363031029431	-0.00421142578125	\\
0.586407422204466	-0.0042724609375	\\
0.5864518133795	-0.00421142578125	\\
0.586496204554535	-0.00408935546875	\\
0.586540595729569	-0.00384521484375	\\
0.586584986904603	-0.003875732421875	\\
0.586629378079638	-0.00396728515625	\\
0.586673769254672	-0.003692626953125	\\
0.586718160429707	-0.0037841796875	\\
0.586762551604741	-0.003326416015625	\\
0.586806942779775	-0.003265380859375	\\
0.58685133395481	-0.003387451171875	\\
0.586895725129844	-0.003021240234375	\\
0.586940116304879	-0.0032958984375	\\
0.586984507479913	-0.00384521484375	\\
0.587028898654947	-0.003570556640625	\\
0.587073289829982	-0.00311279296875	\\
0.587117681005016	-0.003021240234375	\\
0.587162072180051	-0.00311279296875	\\
0.587206463355085	-0.003509521484375	\\
0.587250854530119	-0.0035400390625	\\
0.587295245705154	-0.003204345703125	\\
0.587339636880188	-0.003265380859375	\\
0.587384028055223	-0.003326416015625	\\
0.587428419230257	-0.003631591796875	\\
0.587472810405291	-0.003753662109375	\\
0.587517201580326	-0.003631591796875	\\
0.58756159275536	-0.00421142578125	\\
0.587605983930395	-0.004486083984375	\\
0.587650375105429	-0.004302978515625	\\
0.587694766280463	-0.004302978515625	\\
0.587739157455498	-0.00445556640625	\\
0.587783548630532	-0.0045166015625	\\
0.587827939805567	-0.004150390625	\\
0.587872330980601	-0.003753662109375	\\
0.587916722155636	-0.003936767578125	\\
0.58796111333067	-0.0035400390625	\\
0.588005504505704	-0.00341796875	\\
0.588049895680739	-0.0035400390625	\\
0.588094286855773	-0.003082275390625	\\
0.588138678030807	-0.0030517578125	\\
0.588183069205842	-0.00299072265625	\\
0.588227460380876	-0.00286865234375	\\
0.588271851555911	-0.002532958984375	\\
0.588316242730945	-0.00262451171875	\\
0.588360633905979	-0.00250244140625	\\
0.588405025081014	-0.00201416015625	\\
0.588449416256048	-0.002166748046875	\\
0.588493807431083	-0.002349853515625	\\
0.588538198606117	-0.002227783203125	\\
0.588582589781152	-0.002166748046875	\\
0.588626980956186	-0.00189208984375	\\
0.58867137213122	-0.00189208984375	\\
0.588715763306255	-0.0015869140625	\\
0.588760154481289	-0.00152587890625	\\
0.588804545656324	-0.00152587890625	\\
0.588848936831358	-0.001556396484375	\\
0.588893328006392	-0.001678466796875	\\
0.588937719181427	-0.0015869140625	\\
0.588982110356461	-0.001953125	\\
0.589026501531495	-0.0018310546875	\\
0.58907089270653	-0.0013427734375	\\
0.589115283881564	-0.00128173828125	\\
0.589159675056599	-0.001434326171875	\\
0.589204066231633	-0.00115966796875	\\
0.589248457406668	-0.001190185546875	\\
0.589292848581702	-0.001678466796875	\\
0.589337239756736	-0.0015869140625	\\
0.589381630931771	-0.00140380859375	\\
0.589426022106805	-0.00177001953125	\\
0.58947041328184	-0.002105712890625	\\
0.589514804456874	-0.002044677734375	\\
0.589559195631908	-0.00164794921875	\\
0.589603586806943	-0.00225830078125	\\
0.589647977981977	-0.0028076171875	\\
0.589692369157012	-0.002410888671875	\\
0.589736760332046	-0.00250244140625	\\
0.58978115150708	-0.002655029296875	\\
0.589825542682115	-0.002197265625	\\
0.589869933857149	-0.002166748046875	\\
0.589914325032184	-0.002685546875	\\
0.589958716207218	-0.00286865234375	\\
0.590003107382252	-0.0029296875	\\
0.590047498557287	-0.003204345703125	\\
0.590091889732321	-0.0032958984375	\\
0.590136280907356	-0.00347900390625	\\
0.59018067208239	-0.003265380859375	\\
0.590225063257424	-0.00335693359375	\\
0.590269454432459	-0.003448486328125	\\
0.590313845607493	-0.00335693359375	\\
0.590358236782528	-0.00341796875	\\
0.590402627957562	-0.003326416015625	\\
0.590447019132596	-0.003631591796875	\\
0.590491410307631	-0.003631591796875	\\
0.590535801482665	-0.003997802734375	\\
0.5905801926577	-0.00372314453125	\\
0.590624583832734	-0.00335693359375	\\
0.590668975007768	-0.003448486328125	\\
0.590713366182803	-0.0035400390625	\\
0.590757757357837	-0.003570556640625	\\
0.590802148532872	-0.003753662109375	\\
0.590846539707906	-0.003875732421875	\\
0.59089093088294	-0.00360107421875	\\
0.590935322057975	-0.003326416015625	\\
0.590979713233009	-0.00323486328125	\\
0.591024104408044	-0.003204345703125	\\
0.591068495583078	-0.002655029296875	\\
0.591112886758112	-0.00244140625	\\
0.591157277933147	-0.002349853515625	\\
0.591201669108181	-0.002410888671875	\\
0.591246060283216	-0.002288818359375	\\
0.59129045145825	-0.002227783203125	\\
0.591334842633285	-0.001678466796875	\\
0.591379233808319	-0.00177001953125	\\
0.591423624983353	-0.002197265625	\\
0.591468016158388	-0.001739501953125	\\
0.591512407333422	-0.001739501953125	\\
0.591556798508457	-0.001617431640625	\\
0.591601189683491	-0.001434326171875	\\
0.591645580858525	-0.001129150390625	\\
0.59168997203356	-0.0008544921875	\\
0.591734363208594	-0.000823974609375	\\
0.591778754383628	-0.00048828125	\\
0.591823145558663	-0.000244140625	\\
0.591867536733697	-0.000274658203125	\\
0.591911927908732	-0.000457763671875	\\
0.591956319083766	-0.000396728515625	\\
0.592000710258801	-0.00048828125	\\
0.592045101433835	-0.0006103515625	\\
0.592089492608869	-0.000457763671875	\\
0.592133883783904	-0.000640869140625	\\
0.592178274958938	-0.000701904296875	\\
0.592222666133973	-0.000213623046875	\\
0.592267057309007	3.0517578125e-05	\\
0.592311448484041	-0.000701904296875	\\
0.592355839659076	-0.000579833984375	\\
0.59240023083411	-6.103515625e-05	\\
0.592444622009145	-0.0003662109375	\\
0.592489013184179	-0.000335693359375	\\
0.592533404359213	-0.0006103515625	\\
0.592577795534248	-0.0003662109375	\\
0.592622186709282	-0.000457763671875	\\
0.592666577884317	-0.000823974609375	\\
0.592710969059351	-0.00103759765625	\\
0.592755360234385	-0.001129150390625	\\
0.59279975140942	-0.001129150390625	\\
0.592844142584454	-0.0009765625	\\
0.592888533759489	-0.000640869140625	\\
0.592932924934523	-0.000946044921875	\\
0.592977316109557	-0.00152587890625	\\
0.593021707284592	-0.00177001953125	\\
0.593066098459626	-0.00152587890625	\\
0.593110489634661	-0.0018310546875	\\
0.593154880809695	-0.002471923828125	\\
0.593199271984729	-0.00238037109375	\\
0.593243663159764	-0.002227783203125	\\
0.593288054334798	-0.00244140625	\\
0.593332445509833	-0.002655029296875	\\
0.593376836684867	-0.00238037109375	\\
0.593421227859901	-0.001983642578125	\\
0.593465619034936	-0.00213623046875	\\
0.59351001020997	-0.002410888671875	\\
0.593554401385005	-0.0023193359375	\\
0.593598792560039	-0.002532958984375	\\
0.593643183735074	-0.002410888671875	\\
0.593687574910108	-0.00225830078125	\\
0.593731966085142	-0.002960205078125	\\
0.593776357260177	-0.00262451171875	\\
0.593820748435211	-0.002532958984375	\\
0.593865139610245	-0.002899169921875	\\
0.59390953078528	-0.002838134765625	\\
0.593953921960314	-0.00286865234375	\\
0.593998313135349	-0.00286865234375	\\
0.594042704310383	-0.002349853515625	\\
0.594087095485417	-0.00244140625	\\
0.594131486660452	-0.00250244140625	\\
0.594175877835486	-0.00225830078125	\\
0.594220269010521	-0.002288818359375	\\
0.594264660185555	-0.002288818359375	\\
0.59430905136059	-0.00213623046875	\\
0.594353442535624	-0.00213623046875	\\
0.594397833710658	-0.0023193359375	\\
0.594442224885693	-0.00213623046875	\\
0.594486616060727	-0.0020751953125	\\
0.594531007235762	-0.001739501953125	\\
0.594575398410796	-0.001373291015625	\\
0.59461978958583	-0.000946044921875	\\
0.594664180760865	-0.000732421875	\\
0.594708571935899	-0.000701904296875	\\
0.594752963110934	-0.00054931640625	\\
0.594797354285968	-0.000701904296875	\\
0.594841745461002	-0.0009765625	\\
0.594886136636037	-0.00103759765625	\\
0.594930527811071	-0.0006103515625	\\
0.594974918986106	-0.000335693359375	\\
0.59501931016114	-0.00030517578125	\\
0.595063701336174	-0.000579833984375	\\
0.595108092511209	-0.00048828125	\\
0.595152483686243	-0.000732421875	\\
0.595196874861278	-0.000518798828125	\\
0.595241266036312	-0.000244140625	\\
0.595285657211346	-0.00067138671875	\\
0.595330048386381	-0.001007080078125	\\
0.595374439561415	-0.0010986328125	\\
0.59541883073645	-0.001068115234375	\\
0.595463221911484	-0.001129150390625	\\
0.595507613086518	-0.001434326171875	\\
0.595552004261553	-0.0015869140625	\\
0.595596395436587	-0.00189208984375	\\
0.595640786611622	-0.00225830078125	\\
0.595685177786656	-0.002105712890625	\\
0.59572956896169	-0.0018310546875	\\
0.595773960136725	-0.0020751953125	\\
0.595818351311759	-0.002655029296875	\\
0.595862742486794	-0.002288818359375	\\
0.595907133661828	-0.001922607421875	\\
0.595951524836862	-0.00213623046875	\\
0.595995916011897	-0.002410888671875	\\
0.596040307186931	-0.0025634765625	\\
0.596084698361966	-0.00238037109375	\\
0.596129089537	-0.002227783203125	\\
0.596173480712034	-0.0023193359375	\\
0.596217871887069	-0.00238037109375	\\
0.596262263062103	-0.001708984375	\\
0.596306654237138	-0.001922607421875	\\
0.596351045412172	-0.00201416015625	\\
0.596395436587207	-0.00140380859375	\\
0.596439827762241	-0.001739501953125	\\
0.596484218937275	-0.00177001953125	\\
0.59652861011231	-0.001251220703125	\\
0.596573001287344	-0.001190185546875	\\
0.596617392462378	-0.000946044921875	\\
0.596661783637413	-0.0010986328125	\\
0.596706174812447	-0.001556396484375	\\
0.596750565987482	-0.001129150390625	\\
0.596794957162516	-0.000823974609375	\\
0.59683934833755	-0.000518798828125	\\
0.596883739512585	-0.000640869140625	\\
0.596928130687619	-0.0010986328125	\\
0.596972521862654	-0.001068115234375	\\
0.597016913037688	-0.000579833984375	\\
0.597061304212723	-0.000732421875	\\
0.597105695387757	-0.0010986328125	\\
0.597150086562791	-0.00054931640625	\\
0.597194477737826	-0.000518798828125	\\
0.59723886891286	-0.000701904296875	\\
0.597283260087895	-0.001129150390625	\\
0.597327651262929	-0.001312255859375	\\
0.597372042437963	-0.001129150390625	\\
0.597416433612998	-0.000823974609375	\\
0.597460824788032	-0.00054931640625	\\
0.597505215963066	-0.000762939453125	\\
0.597549607138101	-0.000579833984375	\\
0.597593998313135	-0.000823974609375	\\
0.59763838948817	-0.00164794921875	\\
0.597682780663204	-0.001556396484375	\\
0.597727171838239	-0.00140380859375	\\
0.597771563013273	-0.001708984375	\\
0.597815954188307	-0.00201416015625	\\
0.597860345363342	-0.001861572265625	\\
0.597904736538376	-0.001495361328125	\\
0.597949127713411	-0.0013427734375	\\
0.597993518888445	-0.001678466796875	\\
0.598037910063479	-0.001678466796875	\\
0.598082301238514	-0.001800537109375	\\
0.598126692413548	-0.00152587890625	\\
0.598171083588583	-0.001434326171875	\\
0.598215474763617	-0.001922607421875	\\
0.598259865938651	-0.0018310546875	\\
0.598304257113686	-0.001617431640625	\\
0.59834864828872	-0.001434326171875	\\
0.598393039463755	-0.001434326171875	\\
0.598437430638789	-0.001434326171875	\\
0.598481821813823	-0.0018310546875	\\
0.598526212988858	-0.00189208984375	\\
0.598570604163892	-0.001617431640625	\\
0.598614995338927	-0.001800537109375	\\
0.598659386513961	-0.00201416015625	\\
0.598703777688995	-0.001739501953125	\\
0.59874816886403	-0.0010986328125	\\
0.598792560039064	-0.001190185546875	\\
0.598836951214099	-0.001190185546875	\\
0.598881342389133	-0.001007080078125	\\
0.598925733564167	-0.001129150390625	\\
0.598970124739202	-0.000762939453125	\\
0.599014515914236	-0.000946044921875	\\
0.599058907089271	-0.00067138671875	\\
0.599103298264305	-0.00018310546875	\\
0.599147689439339	-0.00048828125	\\
0.599192080614374	-0.00030517578125	\\
0.599236471789408	6.103515625e-05	\\
0.599280862964443	-0.00048828125	\\
0.599325254139477	-0.000762939453125	\\
0.599369645314511	-0.000457763671875	\\
0.599414036489546	-0.000396728515625	\\
0.59945842766458	-0.000518798828125	\\
0.599502818839615	-0.00067138671875	\\
0.599547210014649	-0.000640869140625	\\
0.599591601189683	-0.000579833984375	\\
0.599635992364718	-6.103515625e-05	\\
0.599680383539752	0.00048828125	\\
0.599724774714787	0.000274658203125	\\
0.599769165889821	0.000640869140625	\\
0.599813557064856	0.000823974609375	\\
0.59985794823989	0.000701904296875	\\
0.599902339414924	0.001007080078125	\\
0.599946730589959	0.0008544921875	\\
0.599991121764993	0.000885009765625	\\
0.600035512940028	0.000885009765625	\\
0.600079904115062	0.000732421875	\\
0.600124295290096	0.001373291015625	\\
0.600168686465131	0.001068115234375	\\
0.600213077640165	0.00079345703125	\\
0.6002574688152	0.001373291015625	\\
0.600301859990234	0.0009765625	\\
0.600346251165268	0.0010986328125	\\
0.600390642340303	0.001251220703125	\\
0.600435033515337	0.0008544921875	\\
0.600479424690372	0.001129150390625	\\
0.600523815865406	0.000823974609375	\\
0.60056820704044	0.000701904296875	\\
0.600612598215475	0.000823974609375	\\
0.600656989390509	0.000579833984375	\\
0.600701380565544	0.000335693359375	\\
0.600745771740578	0.000152587890625	\\
0.600790162915612	0.00030517578125	\\
0.600834554090647	0.000274658203125	\\
0.600878945265681	-0.0001220703125	\\
0.600923336440716	-0.000396728515625	\\
0.60096772761575	-0.00079345703125	\\
0.601012118790784	-0.00115966796875	\\
0.601056509965819	-0.001007080078125	\\
0.601100901140853	-0.001068115234375	\\
0.601145292315888	-0.001220703125	\\
0.601189683490922	-0.001617431640625	\\
0.601234074665956	-0.001953125	\\
0.601278465840991	-0.001617431640625	\\
0.601322857016025	-0.001861572265625	\\
0.60136724819106	-0.002105712890625	\\
0.601411639366094	-0.0018310546875	\\
0.601456030541128	-0.0025634765625	\\
0.601500421716163	-0.00262451171875	\\
0.601544812891197	-0.0023193359375	\\
0.601589204066232	-0.0023193359375	\\
0.601633595241266	-0.002471923828125	\\
0.6016779864163	-0.0020751953125	\\
0.601722377591335	-0.002227783203125	\\
0.601766768766369	-0.0025634765625	\\
0.601811159941404	-0.00238037109375	\\
0.601855551116438	-0.0020751953125	\\
0.601899942291472	-0.001861572265625	\\
0.601944333466507	-0.001800537109375	\\
0.601988724641541	-0.0023193359375	\\
0.602033115816576	-0.00244140625	\\
0.60207750699161	-0.002288818359375	\\
0.602121898166645	-0.00250244140625	\\
0.602166289341679	-0.00244140625	\\
0.602210680516713	-0.0023193359375	\\
0.602255071691748	-0.00250244140625	\\
0.602299462866782	-0.00225830078125	\\
0.602343854041816	-0.00213623046875	\\
0.602388245216851	-0.002349853515625	\\
0.602432636391885	-0.00213623046875	\\
0.60247702756692	-0.001983642578125	\\
0.602521418741954	-0.002166748046875	\\
0.602565809916988	-0.002105712890625	\\
0.602610201092023	-0.00189208984375	\\
0.602654592267057	-0.00201416015625	\\
0.602698983442092	-0.001678466796875	\\
0.602743374617126	-0.001739501953125	\\
0.602787765792161	-0.002166748046875	\\
0.602832156967195	-0.002197265625	\\
0.602876548142229	-0.002410888671875	\\
0.602920939317264	-0.00238037109375	\\
0.602965330492298	-0.00238037109375	\\
0.603009721667333	-0.00244140625	\\
0.603054112842367	-0.00238037109375	\\
0.603098504017401	-0.002288818359375	\\
0.603142895192436	-0.0020751953125	\\
0.60318728636747	-0.002166748046875	\\
0.603231677542505	-0.00225830078125	\\
0.603276068717539	-0.002197265625	\\
0.603320459892573	-0.001983642578125	\\
0.603364851067608	-0.00213623046875	\\
0.603409242242642	-0.002532958984375	\\
0.603453633417677	-0.002349853515625	\\
0.603498024592711	-0.002044677734375	\\
0.603542415767745	-0.001953125	\\
0.60358680694278	-0.002044677734375	\\
0.603631198117814	-0.0018310546875	\\
0.603675589292849	-0.002410888671875	\\
0.603719980467883	-0.002685546875	\\
0.603764371642917	-0.0023193359375	\\
0.603808762817952	-0.002593994140625	\\
0.603853153992986	-0.0028076171875	\\
0.603897545168021	-0.002655029296875	\\
0.603941936343055	-0.002532958984375	\\
0.603986327518089	-0.002899169921875	\\
0.604030718693124	-0.002593994140625	\\
0.604075109868158	-0.002227783203125	\\
0.604119501043193	-0.0023193359375	\\
0.604163892218227	-0.001983642578125	\\
0.604208283393261	-0.00201416015625	\\
0.604252674568296	-0.00189208984375	\\
0.60429706574333	-0.00189208984375	\\
0.604341456918365	-0.001953125	\\
0.604385848093399	-0.001983642578125	\\
0.604430239268433	-0.001556396484375	\\
0.604474630443468	-0.0009765625	\\
0.604519021618502	-0.00164794921875	\\
0.604563412793537	-0.001556396484375	\\
0.604607803968571	-0.0015869140625	\\
0.604652195143605	-0.00177001953125	\\
0.60469658631864	-0.001678466796875	\\
0.604740977493674	-0.001739501953125	\\
0.604785368668709	-0.001434326171875	\\
0.604829759843743	-0.001251220703125	\\
0.604874151018778	-0.0013427734375	\\
0.604918542193812	-0.00091552734375	\\
0.604962933368846	-0.000579833984375	\\
0.605007324543881	-0.000640869140625	\\
0.605051715718915	-0.000457763671875	\\
0.605096106893949	-0.00042724609375	\\
0.605140498068984	-0.00042724609375	\\
0.605184889244018	-0.00030517578125	\\
0.605229280419053	-0.000152587890625	\\
0.605273671594087	6.103515625e-05	\\
0.605318062769121	-3.0517578125e-05	\\
0.605362453944156	-0.000518798828125	\\
0.60540684511919	-0.00048828125	\\
0.605451236294225	-0.000396728515625	\\
0.605495627469259	0.000152587890625	\\
0.605540018644294	0.00042724609375	\\
0.605584409819328	9.1552734375e-05	\\
0.605628800994362	3.0517578125e-05	\\
0.605673192169397	0.000152587890625	\\
0.605717583344431	0.000701904296875	\\
0.605761974519466	0.00042724609375	\\
0.6058063656945	0.0003662109375	\\
0.605850756869534	0.000152587890625	\\
0.605895148044569	0.000274658203125	\\
0.605939539219603	0.000335693359375	\\
0.605983930394637	0.00018310546875	\\
0.606028321569672	0.000762939453125	\\
0.606072712744706	0.000640869140625	\\
0.606117103919741	-6.103515625e-05	\\
0.606161495094775	-0.00018310546875	\\
0.60620588626981	0.000152587890625	\\
0.606250277444844	-6.103515625e-05	\\
0.606294668619878	-0.000396728515625	\\
0.606339059794913	-0.000335693359375	\\
0.606383450969947	-0.000457763671875	\\
0.606427842144982	-0.00054931640625	\\
0.606472233320016	-0.000823974609375	\\
0.60651662449505	-0.001556396484375	\\
0.606561015670085	-0.001617431640625	\\
0.606605406845119	-0.002288818359375	\\
0.606649798020154	-0.002227783203125	\\
0.606694189195188	-0.00177001953125	\\
0.606738580370222	-0.00140380859375	\\
0.606782971545257	-0.001495361328125	\\
0.606827362720291	-0.00146484375	\\
0.606871753895326	-0.00140380859375	\\
0.60691614507036	-0.001708984375	\\
0.606960536245394	-0.0013427734375	\\
0.607004927420429	-0.001617431640625	\\
0.607049318595463	-0.00164794921875	\\
0.607093709770498	-0.001800537109375	\\
0.607138100945532	-0.001861572265625	\\
0.607182492120566	-0.001739501953125	\\
0.607226883295601	-0.001800537109375	\\
0.607271274470635	-0.0018310546875	\\
0.60731566564567	-0.00164794921875	\\
0.607360056820704	-0.0013427734375	\\
0.607404447995738	-0.0013427734375	\\
0.607448839170773	-0.001068115234375	\\
0.607493230345807	-0.0008544921875	\\
0.607537621520842	-0.000732421875	\\
0.607582012695876	-0.000732421875	\\
0.60762640387091	-0.0001220703125	\\
0.607670795045945	-3.0517578125e-05	\\
0.607715186220979	-9.1552734375e-05	\\
0.607759577396014	-9.1552734375e-05	\\
0.607803968571048	9.1552734375e-05	\\
0.607848359746083	0.00079345703125	\\
0.607892750921117	0.001007080078125	\\
0.607937142096151	0.001007080078125	\\
0.607981533271186	0.000946044921875	\\
0.60802592444622	0.001495361328125	\\
0.608070315621254	0.001861572265625	\\
0.608114706796289	0.001678466796875	\\
0.608159097971323	0.001922607421875	\\
0.608203489146358	0.0023193359375	\\
0.608247880321392	0.002655029296875	\\
0.608292271496427	0.002593994140625	\\
0.608336662671461	0.002716064453125	\\
0.608381053846495	0.002899169921875	\\
0.60842544502153	0.002716064453125	\\
0.608469836196564	0.00244140625	\\
0.608514227371599	0.00262451171875	\\
0.608558618546633	0.00286865234375	\\
0.608603009721667	0.00286865234375	\\
0.608647400896702	0.0029296875	\\
0.608691792071736	0.002899169921875	\\
0.608736183246771	0.002685546875	\\
0.608780574421805	0.00250244140625	\\
0.608824965596839	0.002410888671875	\\
0.608869356771874	0.00201416015625	\\
0.608913747946908	0.0013427734375	\\
0.608958139121943	0.001617431640625	\\
0.609002530296977	0.00128173828125	\\
0.609046921472011	0.001251220703125	\\
0.609091312647046	0.00067138671875	\\
0.60913570382208	0.000274658203125	\\
0.609180094997115	0.00042724609375	\\
0.609224486172149	6.103515625e-05	\\
0.609268877347183	-0.00030517578125	\\
0.609313268522218	-0.00054931640625	\\
0.609357659697252	-0.000732421875	\\
0.609402050872287	-0.0008544921875	\\
0.609446442047321	-0.000213623046875	\\
0.609490833222355	-0.000640869140625	\\
0.60953522439739	-0.001129150390625	\\
0.609579615572424	-0.0009765625	\\
0.609624006747459	-0.00115966796875	\\
0.609668397922493	-0.00103759765625	\\
0.609712789097527	-0.0008544921875	\\
0.609757180272562	-0.000701904296875	\\
0.609801571447596	-0.000885009765625	\\
0.609845962622631	-0.00103759765625	\\
0.609890353797665	-0.000518798828125	\\
0.609934744972699	-0.001129150390625	\\
0.609979136147734	-0.001556396484375	\\
0.610023527322768	-0.001190185546875	\\
0.610067918497803	-0.00140380859375	\\
0.610112309672837	-0.001495361328125	\\
0.610156700847871	-0.001190185546875	\\
0.610201092022906	-0.00103759765625	\\
0.61024548319794	-0.00079345703125	\\
0.610289874372975	-0.000396728515625	\\
0.610334265548009	-0.000579833984375	\\
0.610378656723043	-0.00018310546875	\\
0.610423047898078	-0.000518798828125	\\
0.610467439073112	-0.00091552734375	\\
0.610511830248147	-0.00079345703125	\\
0.610556221423181	-0.00018310546875	\\
0.610600612598216	6.103515625e-05	\\
0.61064500377325	0.000274658203125	\\
0.610689394948284	0.0006103515625	\\
0.610733786123319	0.000885009765625	\\
0.610778177298353	0.001129150390625	\\
0.610822568473387	0.000762939453125	\\
0.610866959648422	0.000732421875	\\
0.610911350823456	0.000640869140625	\\
0.610955741998491	0.000885009765625	\\
0.611000133173525	0.000946044921875	\\
0.611044524348559	0.001129150390625	\\
0.611088915523594	0.001007080078125	\\
0.611133306698628	0.00115966796875	\\
0.611177697873663	0.00140380859375	\\
0.611222089048697	0.001220703125	\\
0.611266480223732	0.001068115234375	\\
0.611310871398766	0.000946044921875	\\
0.6113552625738	0.000823974609375	\\
0.611399653748835	0.00115966796875	\\
0.611444044923869	0.0009765625	\\
0.611488436098904	0.00079345703125	\\
0.611532827273938	0.000885009765625	\\
0.611577218448972	0.000885009765625	\\
0.611621609624007	0.000823974609375	\\
0.611666000799041	0.000579833984375	\\
0.611710391974076	-0.00030517578125	\\
0.61175478314911	-9.1552734375e-05	\\
0.611799174324144	-6.103515625e-05	\\
0.611843565499179	-0.000335693359375	\\
0.611887956674213	-0.0006103515625	\\
0.611932347849248	-0.00048828125	\\
0.611976739024282	-0.000640869140625	\\
0.612021130199316	-0.000701904296875	\\
0.612065521374351	-0.000396728515625	\\
0.612109912549385	-0.000885009765625	\\
0.61215430372442	-0.001129150390625	\\
0.612198694899454	-0.0010986328125	\\
0.612243086074488	-0.00103759765625	\\
0.612287477249523	-0.001068115234375	\\
0.612331868424557	-0.001312255859375	\\
0.612376259599592	-0.000885009765625	\\
0.612420650774626	-0.000518798828125	\\
0.61246504194966	-0.00048828125	\\
0.612509433124695	-0.000701904296875	\\
0.612553824299729	-0.0010986328125	\\
0.612598215474764	-0.0009765625	\\
0.612642606649798	-0.000701904296875	\\
0.612686997824832	-0.000518798828125	\\
0.612731388999867	-0.00054931640625	\\
0.612775780174901	-0.0001220703125	\\
0.612820171349936	0.000701904296875	\\
0.61286456252497	0.000640869140625	\\
0.612908953700004	0.000762939453125	\\
0.612953344875039	0.00091552734375	\\
0.612997736050073	0.000823974609375	\\
0.613042127225108	0.000762939453125	\\
0.613086518400142	0.00115966796875	\\
0.613130909575176	0.001373291015625	\\
0.613175300750211	0.00140380859375	\\
0.613219691925245	0.00164794921875	\\
0.61326408310028	0.001617431640625	\\
0.613308474275314	0.0015869140625	\\
0.613352865450349	0.00201416015625	\\
0.613397256625383	0.001983642578125	\\
0.613441647800417	0.0018310546875	\\
0.613486038975452	0.00244140625	\\
0.613530430150486	0.00244140625	\\
0.61357482132552	0.002227783203125	\\
0.613619212500555	0.00189208984375	\\
0.613663603675589	0.00177001953125	\\
0.613707994850624	0.00201416015625	\\
0.613752386025658	0.00201416015625	\\
0.613796777200692	0.0023193359375	\\
0.613841168375727	0.002197265625	\\
0.613885559550761	0.0020751953125	\\
0.613929950725796	0.0028076171875	\\
0.61397434190083	0.002838134765625	\\
0.614018733075865	0.002777099609375	\\
0.614063124250899	0.002685546875	\\
0.614107515425933	0.00286865234375	\\
0.614151906600968	0.0030517578125	\\
0.614196297776002	0.00274658203125	\\
0.614240688951037	0.002777099609375	\\
0.614285080126071	0.00274658203125	\\
0.614329471301105	0.002197265625	\\
0.61437386247614	0.001922607421875	\\
0.614418253651174	0.002044677734375	\\
0.614462644826209	0.00225830078125	\\
0.614507036001243	0.002227783203125	\\
0.614551427176277	0.002197265625	\\
0.614595818351312	0.00213623046875	\\
0.614640209526346	0.00213623046875	\\
0.614684600701381	0.002105712890625	\\
0.614728991876415	0.001678466796875	\\
0.614773383051449	0.001220703125	\\
0.614817774226484	0.001373291015625	\\
0.614862165401518	0.001373291015625	\\
0.614906556576553	0.001373291015625	\\
0.614950947751587	0.001068115234375	\\
0.614995338926621	0.001220703125	\\
0.615039730101656	0.0010986328125	\\
0.61508412127669	0.000335693359375	\\
0.615128512451725	0.000213623046875	\\
0.615172903626759	9.1552734375e-05	\\
0.615217294801793	0.000152587890625	\\
0.615261685976828	0.000274658203125	\\
0.615306077151862	0.00048828125	\\
0.615350468326897	-0.0001220703125	\\
0.615394859501931	-6.103515625e-05	\\
0.615439250676965	0.000152587890625	\\
0.615483641852	0.000244140625	\\
0.615528033027034	0.0003662109375	\\
0.615572424202069	3.0517578125e-05	\\
0.615616815377103	6.103515625e-05	\\
0.615661206552137	0.000640869140625	\\
0.615705597727172	0.00048828125	\\
0.615749988902206	0.0003662109375	\\
0.615794380077241	0.000946044921875	\\
0.615838771252275	0.001220703125	\\
0.615883162427309	0.001251220703125	\\
0.615927553602344	0.00140380859375	\\
0.615971944777378	0.001861572265625	\\
0.616016335952413	0.001800537109375	\\
0.616060727127447	0.001495361328125	\\
0.616105118302481	0.00177001953125	\\
0.616149509477516	0.001861572265625	\\
0.61619390065255	0.001953125	\\
0.616238291827585	0.002166748046875	\\
0.616282683002619	0.0018310546875	\\
0.616327074177654	0.001739501953125	\\
0.616371465352688	0.002166748046875	\\
0.616415856527722	0.00201416015625	\\
0.616460247702757	0.001708984375	\\
0.616504638877791	0.001983642578125	\\
0.616549030052825	0.00201416015625	\\
0.61659342122786	0.001861572265625	\\
0.616637812402894	0.00225830078125	\\
0.616682203577929	0.00244140625	\\
0.616726594752963	0.002410888671875	\\
0.616770985927998	0.002593994140625	\\
0.616815377103032	0.001953125	\\
0.616859768278066	0.002105712890625	\\
0.616904159453101	0.0025634765625	\\
0.616948550628135	0.00201416015625	\\
0.61699294180317	0.001678466796875	\\
0.617037332978204	0.0015869140625	\\
0.617081724153238	0.001861572265625	\\
0.617126115328273	0.001708984375	\\
0.617170506503307	0.0015869140625	\\
0.617214897678342	0.0015869140625	\\
0.617259288853376	0.001678466796875	\\
0.61730368002841	0.00164794921875	\\
0.617348071203445	0.0018310546875	\\
0.617392462378479	0.002227783203125	\\
0.617436853553514	0.002044677734375	\\
0.617481244728548	0.002197265625	\\
0.617525635903582	0.002044677734375	\\
0.617570027078617	0.00177001953125	\\
0.617614418253651	0.002197265625	\\
0.617658809428686	0.00238037109375	\\
0.61770320060372	0.001953125	\\
0.617747591778754	0.0015869140625	\\
0.617791982953789	0.001190185546875	\\
0.617836374128823	0.001068115234375	\\
0.617880765303858	0.00115966796875	\\
0.617925156478892	0.000885009765625	\\
0.617969547653926	0.00091552734375	\\
0.618013938828961	0.000885009765625	\\
0.618058330003995	0.000518798828125	\\
0.61810272117903	0.000640869140625	\\
0.618147112354064	0.00067138671875	\\
0.618191503529098	0.0003662109375	\\
0.618235894704133	0.0003662109375	\\
0.618280285879167	0.000274658203125	\\
0.618324677054202	0.000274658203125	\\
0.618369068229236	0.00042724609375	\\
0.61841345940427	0.000274658203125	\\
0.618457850579305	0.000396728515625	\\
0.618502241754339	0.00079345703125	\\
0.618546632929374	0.00115966796875	\\
0.618591024104408	0.00115966796875	\\
0.618635415279442	0.00067138671875	\\
0.618679806454477	0.000244140625	\\
0.618724197629511	0.0003662109375	\\
0.618768588804546	0.000335693359375	\\
0.61881297997958	0.000457763671875	\\
0.618857371154614	0.000762939453125	\\
0.618901762329649	0.000762939453125	\\
0.618946153504683	0.000396728515625	\\
0.618990544679718	0.0001220703125	\\
0.619034935854752	0.000152587890625	\\
0.619079327029787	-0.000244140625	\\
0.619123718204821	-0.000152587890625	\\
0.619168109379855	0.00018310546875	\\
0.61921250055489	0.00054931640625	\\
0.619256891729924	0.000396728515625	\\
0.619301282904958	0.00054931640625	\\
0.619345674079993	0.0010986328125	\\
0.619390065255027	0.00103759765625	\\
0.619434456430062	0.00115966796875	\\
0.619478847605096	0.001129150390625	\\
0.61952323878013	0.000732421875	\\
0.619567629955165	0.000640869140625	\\
0.619612021130199	0.000732421875	\\
0.619656412305234	0.00048828125	\\
0.619700803480268	0.00067138671875	\\
0.619745194655303	0.0006103515625	\\
0.619789585830337	0.000518798828125	\\
0.619833977005371	0.000640869140625	\\
0.619878368180406	0.000244140625	\\
0.61992275935544	0.000244140625	\\
0.619967150530475	0.000152587890625	\\
0.620011541705509	-6.103515625e-05	\\
0.620055932880543	-0.0001220703125	\\
0.620100324055578	0.00018310546875	\\
0.620144715230612	0.00018310546875	\\
0.620189106405647	-0.000335693359375	\\
0.620233497580681	-0.00018310546875	\\
0.620277888755715	-0.000213623046875	\\
0.62032227993075	-0.000885009765625	\\
0.620366671105784	-0.001007080078125	\\
0.620411062280819	-0.000885009765625	\\
0.620455453455853	-0.001068115234375	\\
0.620499844630887	-0.00140380859375	\\
0.620544235805922	-0.001678466796875	\\
0.620588626980956	-0.001434326171875	\\
0.620633018155991	-0.00146484375	\\
0.620677409331025	-0.001251220703125	\\
0.620721800506059	-0.00128173828125	\\
0.620766191681094	-0.00128173828125	\\
0.620810582856128	-0.0013427734375	\\
0.620854974031163	-0.001617431640625	\\
0.620899365206197	-0.00128173828125	\\
0.620943756381231	-0.000701904296875	\\
0.620988147556266	-0.0006103515625	\\
0.6210325387313	-0.001007080078125	\\
0.621076929906335	-0.001007080078125	\\
0.621121321081369	-0.0008544921875	\\
0.621165712256403	-0.000579833984375	\\
0.621210103431438	6.103515625e-05	\\
0.621254494606472	0	\\
0.621298885781507	-0.00018310546875	\\
0.621343276956541	0.00042724609375	\\
0.621387668131575	0.000579833984375	\\
0.62143205930661	0.0006103515625	\\
0.621476450481644	0.000579833984375	\\
0.621520841656679	0.000823974609375	\\
0.621565232831713	0.00128173828125	\\
0.621609624006747	0.0013427734375	\\
0.621654015181782	0.001708984375	\\
0.621698406356816	0.00177001953125	\\
0.621742797531851	0.001495361328125	\\
0.621787188706885	0.00238037109375	\\
0.62183157988192	0.002777099609375	\\
0.621875971056954	0.0028076171875	\\
0.621920362231988	0.0028076171875	\\
0.621964753407023	0.00286865234375	\\
0.622009144582057	0.00341796875	\\
0.622053535757091	0.003387451171875	\\
0.622097926932126	0.00323486328125	\\
0.62214231810716	0.003631591796875	\\
0.622186709282195	0.003265380859375	\\
0.622231100457229	0.0028076171875	\\
0.622275491632263	0.00311279296875	\\
0.622319882807298	0.0032958984375	\\
0.622364273982332	0.002899169921875	\\
0.622408665157367	0.002471923828125	\\
0.622453056332401	0.002410888671875	\\
0.622497447507436	0.002532958984375	\\
0.62254183868247	0.002532958984375	\\
0.622586229857504	0.00238037109375	\\
0.622630621032539	0.002593994140625	\\
0.622675012207573	0.00225830078125	\\
0.622719403382608	0.00213623046875	\\
0.622763794557642	0.001983642578125	\\
0.622808185732676	0.001708984375	\\
0.622852576907711	0.0013427734375	\\
0.622896968082745	0.001678466796875	\\
0.62294135925778	0.001220703125	\\
0.622985750432814	0.001678466796875	\\
0.623030141607848	0.002227783203125	\\
0.623074532782883	0.001220703125	\\
0.623118923957917	0.001007080078125	\\
0.623163315132952	0.0009765625	\\
0.623207706307986	0.001068115234375	\\
0.62325209748302	0.00140380859375	\\
0.623296488658055	0.00128173828125	\\
0.623340879833089	0.00067138671875	\\
0.623385271008124	0.000640869140625	\\
0.623429662183158	0.00054931640625	\\
0.623474053358192	0.0006103515625	\\
0.623518444533227	0.00054931640625	\\
0.623562835708261	0.00054931640625	\\
0.623607226883296	0.00103759765625	\\
0.62365161805833	0.00115966796875	\\
0.623696009233364	0.000946044921875	\\
0.623740400408399	0.001495361328125	\\
0.623784791583433	0.00201416015625	\\
0.623829182758468	0.001556396484375	\\
0.623873573933502	0.0010986328125	\\
0.623917965108536	0.00103759765625	\\
0.623962356283571	0.000701904296875	\\
0.624006747458605	0.000885009765625	\\
0.62405113863364	0.00115966796875	\\
0.624095529808674	0.00146484375	\\
0.624139920983708	0.001312255859375	\\
0.624184312158743	0.0010986328125	\\
0.624228703333777	0.00128173828125	\\
0.624273094508812	0.001739501953125	\\
0.624317485683846	0.00201416015625	\\
0.62436187685888	0.001708984375	\\
0.624406268033915	0.0015869140625	\\
0.624450659208949	0.001678466796875	\\
0.624495050383984	0.001373291015625	\\
0.624539441559018	0.001190185546875	\\
0.624583832734052	0.00140380859375	\\
0.624628223909087	0.00140380859375	\\
0.624672615084121	0.001800537109375	\\
0.624717006259156	0.00213623046875	\\
0.62476139743419	0.00189208984375	\\
0.624805788609225	0.001739501953125	\\
0.624850179784259	0.001495361328125	\\
0.624894570959293	0.001495361328125	\\
0.624938962134328	0.001434326171875	\\
0.624983353309362	0.001312255859375	\\
0.625027744484396	0.0015869140625	\\
0.625072135659431	0.001617431640625	\\
0.625116526834465	0.00140380859375	\\
0.6251609180095	0.001190185546875	\\
0.625205309184534	0.001129150390625	\\
0.625249700359569	0.0010986328125	\\
0.625294091534603	0.00115966796875	\\
0.625338482709637	0.0009765625	\\
0.625382873884672	0.00128173828125	\\
0.625427265059706	0.001190185546875	\\
0.625471656234741	0.000946044921875	\\
0.625516047409775	0.00115966796875	\\
0.625560438584809	0.001251220703125	\\
0.625604829759844	0.001556396484375	\\
0.625649220934878	0.001251220703125	\\
0.625693612109913	0.0009765625	\\
0.625738003284947	0.000885009765625	\\
0.625782394459981	0.00054931640625	\\
0.625826785635016	9.1552734375e-05	\\
0.62587117681005	0.000213623046875	\\
0.625915567985085	-3.0517578125e-05	\\
0.625959959160119	-0.00048828125	\\
0.626004350335153	-0.000274658203125	\\
0.626048741510188	-0.00042724609375	\\
0.626093132685222	-0.000701904296875	\\
0.626137523860257	-0.0009765625	\\
0.626181915035291	-0.001068115234375	\\
0.626226306210325	-0.0006103515625	\\
0.62627069738536	-0.000396728515625	\\
0.626315088560394	-0.000823974609375	\\
0.626359479735429	-0.001220703125	\\
0.626403870910463	-0.0008544921875	\\
0.626448262085497	-0.000732421875	\\
0.626492653260532	-0.00067138671875	\\
0.626537044435566	-0.00079345703125	\\
0.626581435610601	-0.001434326171875	\\
0.626625826785635	-0.001129150390625	\\
0.626670217960669	-0.000946044921875	\\
0.626714609135704	-0.001251220703125	\\
0.626759000310738	-0.000823974609375	\\
0.626803391485773	-0.001220703125	\\
0.626847782660807	-0.001251220703125	\\
0.626892173835841	-0.00091552734375	\\
0.626936565010876	-0.001220703125	\\
0.62698095618591	-0.0009765625	\\
0.627025347360945	-0.0006103515625	\\
0.627069738535979	-0.0008544921875	\\
0.627114129711013	-0.000762939453125	\\
0.627158520886048	-0.00079345703125	\\
0.627202912061082	-0.00054931640625	\\
0.627247303236117	-0.000335693359375	\\
0.627291694411151	-0.000457763671875	\\
0.627336085586185	-0.000244140625	\\
0.62738047676122	-0.0006103515625	\\
0.627424867936254	-0.00042724609375	\\
0.627469259111289	-9.1552734375e-05	\\
0.627513650286323	-0.000244140625	\\
0.627558041461358	-0.0001220703125	\\
0.627602432636392	-9.1552734375e-05	\\
0.627646823811426	-0.0001220703125	\\
0.627691214986461	-0.000213623046875	\\
0.627735606161495	-0.000457763671875	\\
0.627779997336529	-0.000213623046875	\\
0.627824388511564	0.000213623046875	\\
0.627868779686598	3.0517578125e-05	\\
0.627913170861633	3.0517578125e-05	\\
0.627957562036667	-3.0517578125e-05	\\
0.628001953211701	-0.000274658203125	\\
0.628046344386736	0.000152587890625	\\
0.62809073556177	0.000274658203125	\\
0.628135126736805	-9.1552734375e-05	\\
0.628179517911839	-0.000396728515625	\\
0.628223909086874	-9.1552734375e-05	\\
0.628268300261908	-0.000152587890625	\\
0.628312691436942	-0.000335693359375	\\
0.628357082611977	-9.1552734375e-05	\\
0.628401473787011	-0.000244140625	\\
0.628445864962046	9.1552734375e-05	\\
0.62849025613708	0.000152587890625	\\
0.628534647312114	0	\\
0.628579038487149	-0.00018310546875	\\
0.628623429662183	-0.00042724609375	\\
0.628667820837218	-0.0006103515625	\\
0.628712212012252	-0.0009765625	\\
0.628756603187286	-0.001220703125	\\
0.628800994362321	-0.00140380859375	\\
0.628845385537355	-0.00152587890625	\\
0.62888977671239	-0.001861572265625	\\
0.628934167887424	-0.00177001953125	\\
0.628978559062458	-0.00177001953125	\\
0.629022950237493	-0.001861572265625	\\
0.629067341412527	-0.001373291015625	\\
0.629111732587562	-0.001129150390625	\\
0.629156123762596	-0.00054931640625	\\
0.62920051493763	6.103515625e-05	\\
0.629244906112665	-0.000213623046875	\\
0.629289297287699	-0.000244140625	\\
0.629333688462734	-0.000640869140625	\\
0.629378079637768	9.1552734375e-05	\\
0.629422470812802	0.000396728515625	\\
0.629466861987837	0.000732421875	\\
0.629511253162871	0.00128173828125	\\
0.629555644337906	0.001190185546875	\\
0.62960003551294	0.001312255859375	\\
0.629644426687974	0.001708984375	\\
0.629688817863009	0.001983642578125	\\
0.629733209038043	0.002044677734375	\\
0.629777600213078	0.001953125	\\
0.629821991388112	0.001983642578125	\\
0.629866382563146	0.001708984375	\\
0.629910773738181	0.001800537109375	\\
0.629955164913215	0.00201416015625	\\
0.62999955608825	0.002044677734375	\\
0.630043947263284	0.001861572265625	\\
0.630088338438318	0.001800537109375	\\
0.630132729613353	0.002288818359375	\\
0.630177120788387	0.00213623046875	\\
0.630221511963422	0.002471923828125	\\
0.630265903138456	0.002838134765625	\\
0.630310294313491	0.002838134765625	\\
0.630354685488525	0.002960205078125	\\
0.630399076663559	0.003143310546875	\\
0.630443467838594	0.002838134765625	\\
0.630487859013628	0.00262451171875	\\
0.630532250188663	0.002288818359375	\\
0.630576641363697	0.002105712890625	\\
0.630621032538731	0.002410888671875	\\
0.630665423713766	0.002716064453125	\\
0.6307098148888	0.002685546875	\\
0.630754206063834	0.0023193359375	\\
0.630798597238869	0.002105712890625	\\
0.630842988413903	0.002349853515625	\\
0.630887379588938	0.002410888671875	\\
0.630931770763972	0.002197265625	\\
0.630976161939007	0.002105712890625	\\
0.631020553114041	0.001922607421875	\\
0.631064944289075	0.002288818359375	\\
0.63110933546411	0.00311279296875	\\
0.631153726639144	0.002685546875	\\
0.631198117814179	0.002655029296875	\\
0.631242508989213	0.002685546875	\\
0.631286900164247	0.002532958984375	\\
0.631331291339282	0.00250244140625	\\
0.631375682514316	0.002166748046875	\\
0.631420073689351	0.00238037109375	\\
0.631464464864385	0.002105712890625	\\
0.631508856039419	0.00250244140625	\\
0.631553247214454	0.002471923828125	\\
0.631597638389488	0.002227783203125	\\
0.631642029564523	0.00244140625	\\
0.631686420739557	0.002288818359375	\\
0.631730811914591	0.002227783203125	\\
0.631775203089626	0.002349853515625	\\
0.63181959426466	0.00238037109375	\\
0.631863985439695	0.00225830078125	\\
0.631908376614729	0.002044677734375	\\
0.631952767789763	0.002197265625	\\
0.631997158964798	0.0023193359375	\\
0.632041550139832	0.001983642578125	\\
0.632085941314867	0.001922607421875	\\
0.632130332489901	0.0020751953125	\\
0.632174723664935	0.001617431640625	\\
0.63221911483997	0.00128173828125	\\
0.632263506015004	0.0018310546875	\\
0.632307897190039	0.002166748046875	\\
0.632352288365073	0.002593994140625	\\
0.632396679540107	0.002838134765625	\\
0.632441070715142	0.00250244140625	\\
0.632485461890176	0.002166748046875	\\
0.632529853065211	0.002288818359375	\\
0.632574244240245	0.00201416015625	\\
0.632618635415279	0.001739501953125	\\
0.632663026590314	0.001220703125	\\
0.632707417765348	0.00091552734375	\\
0.632751808940383	0.0015869140625	\\
0.632796200115417	0.00201416015625	\\
0.632840591290451	0.00189208984375	\\
0.632884982465486	0.002105712890625	\\
0.63292937364052	0.001953125	\\
0.632973764815555	0.002197265625	\\
0.633018155990589	0.00244140625	\\
0.633062547165623	0.00250244140625	\\
0.633106938340658	0.002227783203125	\\
0.633151329515692	0.001495361328125	\\
0.633195720690727	0.001190185546875	\\
0.633240111865761	0.000885009765625	\\
0.633284503040796	0.0006103515625	\\
0.63332889421583	0.0003662109375	\\
0.633373285390864	0.00042724609375	\\
0.633417676565899	0.000518798828125	\\
0.633462067740933	0.000244140625	\\
0.633506458915967	0.000457763671875	\\
0.633550850091002	0.00091552734375	\\
0.633595241266036	0.001068115234375	\\
0.633639632441071	0.00115966796875	\\
0.633684023616105	0.001312255859375	\\
0.63372841479114	0.001007080078125	\\
0.633772805966174	0.000732421875	\\
0.633817197141208	0.00103759765625	\\
0.633861588316243	0.001007080078125	\\
0.633905979491277	0.000946044921875	\\
0.633950370666312	0.00091552734375	\\
0.633994761841346	0.0006103515625	\\
0.63403915301638	0.00103759765625	\\
0.634083544191415	0.001251220703125	\\
0.634127935366449	0.00128173828125	\\
0.634172326541484	0.0008544921875	\\
0.634216717716518	0.000457763671875	\\
0.634261108891552	0.000518798828125	\\
0.634305500066587	0.00115966796875	\\
0.634349891241621	0.001495361328125	\\
0.634394282416656	0.00091552734375	\\
0.63443867359169	0.0006103515625	\\
0.634483064766724	0.001251220703125	\\
0.634527455941759	0.001129150390625	\\
0.634571847116793	0.001007080078125	\\
0.634616238291828	0.00146484375	\\
0.634660629466862	0.001190185546875	\\
0.634705020641896	0.001007080078125	\\
0.634749411816931	0.001129150390625	\\
0.634793802991965	0.001220703125	\\
0.634838194167	0.000640869140625	\\
0.634882585342034	0.000518798828125	\\
0.634926976517068	0.000457763671875	\\
0.634971367692103	6.103515625e-05	\\
0.635015758867137	0.000396728515625	\\
0.635060150042172	0.000946044921875	\\
0.635104541217206	0.0008544921875	\\
0.63514893239224	0.000396728515625	\\
0.635193323567275	0.000335693359375	\\
0.635237714742309	0.00018310546875	\\
0.635282105917344	0.000213623046875	\\
0.635326497092378	0	\\
0.635370888267412	-0.000152587890625	\\
0.635415279442447	-3.0517578125e-05	\\
0.635459670617481	0.0003662109375	\\
0.635504061792516	0.000213623046875	\\
0.63554845296755	0.0001220703125	\\
0.635592844142584	0.000640869140625	\\
0.635637235317619	0.0006103515625	\\
0.635681626492653	0.00054931640625	\\
0.635726017667688	0.00091552734375	\\
0.635770408842722	0.000762939453125	\\
0.635814800017756	0.00054931640625	\\
0.635859191192791	0.0006103515625	\\
0.635903582367825	0.0001220703125	\\
0.63594797354286	0.000274658203125	\\
0.635992364717894	0.00054931640625	\\
0.636036755892929	3.0517578125e-05	\\
0.636081147067963	0	\\
0.636125538242997	-3.0517578125e-05	\\
0.636169929418032	-0.000274658203125	\\
0.636214320593066	0.00018310546875	\\
0.6362587117681	-0.00030517578125	\\
0.636303102943135	-0.0006103515625	\\
0.636347494118169	-0.000335693359375	\\
0.636391885293204	-0.0006103515625	\\
0.636436276468238	-0.00067138671875	\\
0.636480667643272	-0.00048828125	\\
0.636525058818307	-0.00048828125	\\
0.636569449993341	-0.00042724609375	\\
0.636613841168376	-0.000579833984375	\\
0.63665823234341	-0.000274658203125	\\
0.636702623518445	6.103515625e-05	\\
0.636747014693479	0.000152587890625	\\
0.636791405868513	0.000732421875	\\
0.636835797043548	0.000518798828125	\\
0.636880188218582	0.000518798828125	\\
0.636924579393617	0.000274658203125	\\
0.636968970568651	0.000823974609375	\\
0.637013361743685	0.0009765625	\\
0.63705775291872	0.000579833984375	\\
0.637102144093754	0.00079345703125	\\
0.637146535268789	0.001129150390625	\\
0.637190926443823	0.00115966796875	\\
0.637235317618857	0.00115966796875	\\
0.637279708793892	0.00140380859375	\\
0.637324099968926	0.001312255859375	\\
0.637368491143961	0.00152587890625	\\
0.637412882318995	0.001861572265625	\\
0.637457273494029	0.0018310546875	\\
0.637501664669064	0.001739501953125	\\
0.637546055844098	0.002227783203125	\\
0.637590447019133	0.00250244140625	\\
0.637634838194167	0.002105712890625	\\
0.637679229369201	0.0020751953125	\\
0.637723620544236	0.0023193359375	\\
0.63776801171927	0.0023193359375	\\
0.637812402894305	0.002288818359375	\\
0.637856794069339	0.00213623046875	\\
0.637901185244373	0.002166748046875	\\
0.637945576419408	0.00201416015625	\\
0.637989967594442	0.0015869140625	\\
0.638034358769477	0.002044677734375	\\
0.638078749944511	0.00238037109375	\\
0.638123141119545	0.00201416015625	\\
0.63816753229458	0.00213623046875	\\
0.638211923469614	0.00201416015625	\\
0.638256314644649	0.00177001953125	\\
0.638300705819683	0.0018310546875	\\
0.638345096994717	0.001617431640625	\\
0.638389488169752	0.00115966796875	\\
0.638433879344786	0.0015869140625	\\
0.638478270519821	0.00213623046875	\\
0.638522661694855	0.0015869140625	\\
0.638567052869889	0.00091552734375	\\
0.638611444044924	0.0006103515625	\\
0.638655835219958	0.000640869140625	\\
0.638700226394993	0.000732421875	\\
0.638744617570027	0.001007080078125	\\
0.638789008745062	0.001312255859375	\\
0.638833399920096	0.001495361328125	\\
0.63887779109513	0.00164794921875	\\
0.638922182270165	0.001953125	\\
0.638966573445199	0.001953125	\\
0.639010964620234	0.001861572265625	\\
0.639055355795268	0.001983642578125	\\
0.639099746970302	0.001953125	\\
0.639144138145337	0.00250244140625	\\
0.639188529320371	0.002960205078125	\\
0.639232920495405	0.002655029296875	\\
0.63927731167044	0.00274658203125	\\
0.639321702845474	0.002655029296875	\\
0.639366094020509	0.00286865234375	\\
0.639410485195543	0.00299072265625	\\
0.639454876370578	0.00250244140625	\\
0.639499267545612	0.0025634765625	\\
0.639543658720646	0.002685546875	\\
0.639588049895681	0.0025634765625	\\
0.639632441070715	0.002471923828125	\\
0.63967683224575	0.002716064453125	\\
0.639721223420784	0.0029296875	\\
0.639765614595818	0.002777099609375	\\
0.639810005770853	0.00244140625	\\
0.639854396945887	0.00250244140625	\\
0.639898788120922	0.00286865234375	\\
0.639943179295956	0.00238037109375	\\
0.63998757047099	0.002044677734375	\\
0.640031961646025	0.001953125	\\
0.640076352821059	0.00201416015625	\\
0.640120743996094	0.001953125	\\
0.640165135171128	0.00213623046875	\\
0.640209526346162	0.00201416015625	\\
0.640253917521197	0.001556396484375	\\
0.640298308696231	0.001617431640625	\\
0.640342699871266	0.001861572265625	\\
0.6403870910463	0.002044677734375	\\
0.640431482221334	0.001495361328125	\\
0.640475873396369	0.001251220703125	\\
0.640520264571403	0.0013427734375	\\
0.640564655746438	0.00091552734375	\\
0.640609046921472	0.00103759765625	\\
0.640653438096506	0.00115966796875	\\
0.640697829271541	0.000701904296875	\\
0.640742220446575	0.0006103515625	\\
0.64078661162161	0.00067138671875	\\
0.640831002796644	0.00042724609375	\\
0.640875393971678	-3.0517578125e-05	\\
0.640919785146713	0.00018310546875	\\
0.640964176321747	0.00042724609375	\\
0.641008567496782	9.1552734375e-05	\\
0.641052958671816	9.1552734375e-05	\\
0.64109734984685	-6.103515625e-05	\\
0.641141741021885	-9.1552734375e-05	\\
0.641186132196919	0	\\
0.641230523371954	-0.000152587890625	\\
0.641274914546988	0.00018310546875	\\
0.641319305722022	-3.0517578125e-05	\\
0.641363696897057	-0.000213623046875	\\
0.641408088072091	0.00030517578125	\\
0.641452479247126	0	\\
0.64149687042216	-0.00042724609375	\\
0.641541261597194	-0.00067138671875	\\
0.641585652772229	-0.000396728515625	\\
0.641630043947263	-0.000244140625	\\
0.641674435122298	-0.000335693359375	\\
0.641718826297332	0.000335693359375	\\
0.641763217472367	0.000244140625	\\
0.641807608647401	0.000335693359375	\\
0.641851999822435	0.000579833984375	\\
0.64189639099747	0.000396728515625	\\
0.641940782172504	0.000244140625	\\
0.641985173347538	0.000762939453125	\\
0.642029564522573	0.00164794921875	\\
0.642073955697607	0.00152587890625	\\
0.642118346872642	0.001434326171875	\\
0.642162738047676	0.001434326171875	\\
0.642207129222711	0.0015869140625	\\
0.642251520397745	0.002044677734375	\\
0.642295911572779	0.001708984375	\\
0.642340302747814	0.0020751953125	\\
0.642384693922848	0.00213623046875	\\
0.642429085097883	0.00189208984375	\\
0.642473476272917	0.002349853515625	\\
0.642517867447951	0.00213623046875	\\
0.642562258622986	0.002044677734375	\\
0.64260664979802	0.002166748046875	\\
0.642651040973055	0.00201416015625	\\
0.642695432148089	0.0023193359375	\\
0.642739823323123	0.001983642578125	\\
0.642784214498158	0.001861572265625	\\
0.642828605673192	0.001922607421875	\\
0.642872996848227	0.001983642578125	\\
0.642917388023261	0.001922607421875	\\
0.642961779198295	0.001373291015625	\\
0.64300617037333	0.000885009765625	\\
0.643050561548364	0.000823974609375	\\
0.643094952723399	0.00067138671875	\\
0.643139343898433	0.001007080078125	\\
0.643183735073467	0.000946044921875	\\
0.643228126248502	0.000823974609375	\\
0.643272517423536	0.001007080078125	\\
0.643316908598571	0.000579833984375	\\
0.643361299773605	0.000244140625	\\
0.643405690948639	-6.103515625e-05	\\
0.643450082123674	-0.00018310546875	\\
0.643494473298708	0.0001220703125	\\
0.643538864473743	0.000152587890625	\\
0.643583255648777	3.0517578125e-05	\\
0.643627646823811	-0.000152587890625	\\
0.643672037998846	-0.00042724609375	\\
0.64371642917388	-0.000579833984375	\\
0.643760820348915	-0.0009765625	\\
0.643805211523949	-0.001220703125	\\
0.643849602698983	-0.0010986328125	\\
0.643893993874018	-0.001007080078125	\\
0.643938385049052	-0.000946044921875	\\
0.643982776224087	-0.00115966796875	\\
0.644027167399121	-0.001617431640625	\\
0.644071558574155	-0.001495361328125	\\
0.64411594974919	-0.001678466796875	\\
0.644160340924224	-0.001617431640625	\\
0.644204732099259	-0.001556396484375	\\
0.644249123274293	-0.001708984375	\\
0.644293514449327	-0.00189208984375	\\
0.644337905624362	-0.00238037109375	\\
0.644382296799396	-0.00250244140625	\\
0.644426687974431	-0.002166748046875	\\
0.644471079149465	-0.001953125	\\
0.6445154703245	-0.00201416015625	\\
0.644559861499534	-0.001800537109375	\\
0.644604252674568	-0.001708984375	\\
0.644648643849603	-0.001861572265625	\\
0.644693035024637	-0.001678466796875	\\
0.644737426199672	-0.001373291015625	\\
0.644781817374706	-0.001434326171875	\\
0.64482620854974	-0.00115966796875	\\
0.644870599724775	-0.0008544921875	\\
0.644914990899809	-0.0013427734375	\\
0.644959382074843	-0.001007080078125	\\
0.645003773249878	-0.001068115234375	\\
0.645048164424912	-0.001251220703125	\\
0.645092555599947	-0.00079345703125	\\
0.645136946774981	-0.000762939453125	\\
0.645181337950016	-0.00042724609375	\\
0.64522572912505	6.103515625e-05	\\
0.645270120300084	9.1552734375e-05	\\
0.645314511475119	-0.000518798828125	\\
0.645358902650153	-0.00048828125	\\
0.645403293825188	0.000457763671875	\\
0.645447685000222	3.0517578125e-05	\\
0.645492076175256	-0.000579833984375	\\
0.645536467350291	3.0517578125e-05	\\
0.645580858525325	3.0517578125e-05	\\
0.64562524970036	9.1552734375e-05	\\
0.645669640875394	0	\\
0.645714032050428	-0.000213623046875	\\
0.645758423225463	0.00018310546875	\\
0.645802814400497	9.1552734375e-05	\\
0.645847205575532	9.1552734375e-05	\\
0.645891596750566	0	\\
0.6459359879256	-0.000457763671875	\\
0.645980379100635	0.000244140625	\\
0.646024770275669	0.0003662109375	\\
0.646069161450704	-0.000244140625	\\
0.646113552625738	-0.00054931640625	\\
0.646157943800772	-0.0003662109375	\\
0.646202334975807	-0.00054931640625	\\
0.646246726150841	-0.000762939453125	\\
0.646291117325876	-0.00079345703125	\\
0.64633550850091	-0.0006103515625	\\
0.646379899675944	-0.000640869140625	\\
0.646424290850979	-0.0009765625	\\
0.646468682026013	-0.000762939453125	\\
0.646513073201048	-0.000701904296875	\\
0.646557464376082	-0.000762939453125	\\
0.646601855551116	-0.000579833984375	\\
0.646646246726151	-0.000396728515625	\\
0.646690637901185	-0.00067138671875	\\
0.64673502907622	-0.000701904296875	\\
0.646779420251254	-0.000335693359375	\\
0.646823811426288	-0.0001220703125	\\
0.646868202601323	0.000213623046875	\\
0.646912593776357	0.00042724609375	\\
0.646956984951392	0.001190185546875	\\
0.647001376126426	0.001251220703125	\\
0.64704576730146	0.00103759765625	\\
0.647090158476495	0.00146484375	\\
0.647134549651529	0.001800537109375	\\
0.647178940826564	0.00146484375	\\
0.647223332001598	0.00201416015625	\\
0.647267723176633	0.00244140625	\\
0.647312114351667	0.002471923828125	\\
0.647356505526701	0.002899169921875	\\
0.647400896701736	0.002655029296875	\\
0.64744528787677	0.0028076171875	\\
0.647489679051805	0.0029296875	\\
0.647534070226839	0.003143310546875	\\
0.647578461401873	0.003173828125	\\
0.647622852576908	0.00286865234375	\\
0.647667243751942	0.00311279296875	\\
0.647711634926976	0.003082275390625	\\
0.647756026102011	0.0029296875	\\
0.647800417277045	0.003021240234375	\\
0.64784480845208	0.0025634765625	\\
0.647889199627114	0.00262451171875	\\
0.647933590802149	0.00262451171875	\\
0.647977981977183	0.002532958984375	\\
0.648022373152217	0.00250244140625	\\
0.648066764327252	0.002410888671875	\\
0.648111155502286	0.002532958984375	\\
0.648155546677321	0.002593994140625	\\
0.648199937852355	0.00250244140625	\\
0.648244329027389	0.002197265625	\\
0.648288720202424	0.00213623046875	\\
0.648333111377458	0.001678466796875	\\
0.648377502552493	0.00091552734375	\\
0.648421893727527	0.001129150390625	\\
0.648466284902561	0.00128173828125	\\
0.648510676077596	0.0008544921875	\\
0.64855506725263	0.00091552734375	\\
0.648599458427665	0.000732421875	\\
0.648643849602699	0.000335693359375	\\
0.648688240777733	-0.0001220703125	\\
0.648732631952768	0.000152587890625	\\
0.648777023127802	0.000396728515625	\\
0.648821414302837	-9.1552734375e-05	\\
0.648865805477871	0.000335693359375	\\
0.648910196652905	-3.0517578125e-05	\\
0.64895458782794	-9.1552734375e-05	\\
0.648998979002974	-0.00018310546875	\\
0.649043370178009	-0.000396728515625	\\
0.649087761353043	-0.000457763671875	\\
0.649132152528077	-0.000823974609375	\\
0.649176543703112	-0.000885009765625	\\
0.649220934878146	-0.000762939453125	\\
0.649265326053181	-0.000640869140625	\\
0.649309717228215	-0.000823974609375	\\
0.649354108403249	-0.00115966796875	\\
0.649398499578284	-0.001312255859375	\\
0.649442890753318	-0.0010986328125	\\
0.649487281928353	-0.00115966796875	\\
0.649531673103387	-0.0013427734375	\\
0.649576064278421	-0.001373291015625	\\
0.649620455453456	-0.00128173828125	\\
0.64966484662849	-0.000946044921875	\\
0.649709237803525	-0.0010986328125	\\
0.649753628978559	-0.001129150390625	\\
0.649798020153593	-0.000335693359375	\\
0.649842411328628	9.1552734375e-05	\\
0.649886802503662	-9.1552734375e-05	\\
0.649931193678697	-0.000244140625	\\
0.649975584853731	-0.000244140625	\\
0.650019976028765	-0.00030517578125	\\
0.6500643672038	-0.000457763671875	\\
0.650108758378834	-0.00067138671875	\\
0.650153149553869	-0.000244140625	\\
0.650197540728903	-0.000244140625	\\
0.650241931903938	-0.0001220703125	\\
0.650286323078972	-9.1552734375e-05	\\
0.650330714254006	-0.00042724609375	\\
0.650375105429041	-0.00048828125	\\
0.650419496604075	-0.000152587890625	\\
0.650463887779109	-6.103515625e-05	\\
0.650508278954144	-0.000335693359375	\\
0.650552670129178	-0.00042724609375	\\
0.650597061304213	-0.000152587890625	\\
0.650641452479247	-0.000152587890625	\\
0.650685843654282	-9.1552734375e-05	\\
0.650730234829316	-0.00030517578125	\\
0.65077462600435	-0.00030517578125	\\
0.650819017179385	-3.0517578125e-05	\\
0.650863408354419	0	\\
0.650907799529454	6.103515625e-05	\\
0.650952190704488	-0.00018310546875	\\
0.650996581879522	-0.00018310546875	\\
0.651040973054557	0.0001220703125	\\
0.651085364229591	6.103515625e-05	\\
0.651129755404626	-9.1552734375e-05	\\
0.65117414657966	-6.103515625e-05	\\
0.651218537754694	-0.0001220703125	\\
0.651262928929729	6.103515625e-05	\\
0.651307320104763	-0.00030517578125	\\
0.651351711279798	-0.000732421875	\\
0.651396102454832	-0.00079345703125	\\
0.651440493629866	-0.0008544921875	\\
0.651484884804901	-0.000946044921875	\\
0.651529275979935	-0.000946044921875	\\
0.65157366715497	-0.001190185546875	\\
0.651618058330004	-0.0013427734375	\\
0.651662449505038	-0.001190185546875	\\
0.651706840680073	-0.001373291015625	\\
0.651751231855107	-0.00152587890625	\\
0.651795623030142	-0.001617431640625	\\
0.651840014205176	-0.001556396484375	\\
0.65188440538021	-0.001373291015625	\\
0.651928796555245	-0.001861572265625	\\
0.651973187730279	-0.00177001953125	\\
0.652017578905314	-0.00146484375	\\
0.652061970080348	-0.0018310546875	\\
0.652106361255382	-0.001861572265625	\\
0.652150752430417	-0.001739501953125	\\
0.652195143605451	-0.001922607421875	\\
0.652239534780486	-0.00189208984375	\\
0.65228392595552	-0.002105712890625	\\
0.652328317130554	-0.002044677734375	\\
0.652372708305589	-0.001861572265625	\\
0.652417099480623	-0.00177001953125	\\
0.652461490655658	-0.00201416015625	\\
0.652505881830692	-0.001922607421875	\\
0.652550273005726	-0.000885009765625	\\
0.652594664180761	-0.001190185546875	\\
0.652639055355795	-0.00115966796875	\\
0.65268344653083	-0.000885009765625	\\
0.652727837705864	-0.000946044921875	\\
0.652772228880898	-0.001007080078125	\\
0.652816620055933	-0.00103759765625	\\
0.652861011230967	-0.00079345703125	\\
0.652905402406002	-0.000518798828125	\\
0.652949793581036	-0.000457763671875	\\
0.652994184756071	-0.000213623046875	\\
0.653038575931105	-0.000274658203125	\\
0.653082967106139	-0.000396728515625	\\
0.653127358281174	-0.00018310546875	\\
0.653171749456208	-0.0001220703125	\\
0.653216140631243	6.103515625e-05	\\
0.653260531806277	0.00042724609375	\\
0.653304922981311	0.00030517578125	\\
0.653349314156346	-0.000274658203125	\\
0.65339370533138	-0.0001220703125	\\
0.653438096506414	-3.0517578125e-05	\\
0.653482487681449	0	\\
0.653526878856483	-0.000274658203125	\\
0.653571270031518	-0.000823974609375	\\
0.653615661206552	-0.00067138671875	\\
0.653660052381587	-0.00054931640625	\\
0.653704443556621	-0.00054931640625	\\
0.653748834731655	-0.000823974609375	\\
0.65379322590669	-0.001190185546875	\\
0.653837617081724	-0.001220703125	\\
0.653882008256759	-0.00140380859375	\\
0.653926399431793	-0.00152587890625	\\
0.653970790606827	-0.001800537109375	\\
0.654015181781862	-0.001922607421875	\\
0.654059572956896	-0.001953125	\\
0.654103964131931	-0.00177001953125	\\
0.654148355306965	-0.0015869140625	\\
0.654192746481999	-0.002044677734375	\\
0.654237137657034	-0.002044677734375	\\
0.654281528832068	-0.001739501953125	\\
0.654325920007103	-0.001678466796875	\\
0.654370311182137	-0.00189208984375	\\
0.654414702357171	-0.002197265625	\\
0.654459093532206	-0.001739501953125	\\
0.65450348470724	-0.001556396484375	\\
0.654547875882275	-0.0015869140625	\\
0.654592267057309	-0.000946044921875	\\
0.654636658232343	-0.00140380859375	\\
0.654681049407378	-0.001068115234375	\\
0.654725440582412	-0.0010986328125	\\
0.654769831757447	-0.00164794921875	\\
0.654814222932481	-0.00146484375	\\
0.654858614107515	-0.001129150390625	\\
0.65490300528255	-0.00103759765625	\\
0.654947396457584	-0.001220703125	\\
0.654991787632619	-0.00091552734375	\\
0.655036178807653	-0.000946044921875	\\
0.655080569982687	-0.0009765625	\\
0.655124961157722	-0.000457763671875	\\
0.655169352332756	3.0517578125e-05	\\
0.655213743507791	-0.00042724609375	\\
0.655258134682825	-0.000213623046875	\\
0.655302525857859	-0.000457763671875	\\
0.655346917032894	-3.0517578125e-05	\\
0.655391308207928	0.000579833984375	\\
0.655435699382963	0.0001220703125	\\
0.655480090557997	-9.1552734375e-05	\\
0.655524481733031	-0.00018310546875	\\
0.655568872908066	0.0003662109375	\\
0.6556132640831	0.00054931640625	\\
0.655657655258135	0.000457763671875	\\
0.655702046433169	0.000457763671875	\\
0.655746437608204	0.00048828125	\\
0.655790828783238	0.000946044921875	\\
0.655835219958272	0.000762939453125	\\
0.655879611133307	0.00067138671875	\\
0.655924002308341	0.000885009765625	\\
0.655968393483376	0.000823974609375	\\
0.65601278465841	0.00054931640625	\\
0.656057175833444	0.000396728515625	\\
0.656101567008479	0.0006103515625	\\
0.656145958183513	0.000640869140625	\\
0.656190349358547	0.00018310546875	\\
0.656234740533582	0.0001220703125	\\
0.656279131708616	9.1552734375e-05	\\
0.656323522883651	-0.00018310546875	\\
0.656367914058685	-0.00030517578125	\\
0.65641230523372	-0.000396728515625	\\
0.656456696408754	-0.00018310546875	\\
0.656501087583788	-0.000579833984375	\\
0.656545478758823	-0.000335693359375	\\
0.656589869933857	0.00048828125	\\
0.656634261108892	0.000213623046875	\\
0.656678652283926	-0.000518798828125	\\
0.65672304345896	-0.000335693359375	\\
0.656767434633995	0	\\
0.656811825809029	-0.000335693359375	\\
0.656856216984064	-0.000640869140625	\\
0.656900608159098	0	\\
0.656944999334132	0.00018310546875	\\
0.656989390509167	6.103515625e-05	\\
0.657033781684201	0.00079345703125	\\
0.657078172859236	0.000518798828125	\\
0.65712256403427	0.000152587890625	\\
0.657166955209304	0.0006103515625	\\
0.657211346384339	0.00067138671875	\\
0.657255737559373	0.000701904296875	\\
0.657300128734408	0.001190185546875	\\
0.657344519909442	0.00091552734375	\\
0.657388911084476	0.00079345703125	\\
0.657433302259511	0.0010986328125	\\
0.657477693434545	0.000946044921875	\\
0.65752208460958	0.000762939453125	\\
0.657566475784614	0.000732421875	\\
0.657610866959648	0.000701904296875	\\
0.657655258134683	0.0008544921875	\\
0.657699649309717	0.0006103515625	\\
0.657744040484752	3.0517578125e-05	\\
0.657788431659786	0.00018310546875	\\
0.65783282283482	-9.1552734375e-05	\\
0.657877214009855	-0.000335693359375	\\
0.657921605184889	-0.000518798828125	\\
0.657965996359924	-0.00067138671875	\\
0.658010387534958	-0.00042724609375	\\
0.658054778709992	-0.00067138671875	\\
0.658099169885027	-0.000732421875	\\
0.658143561060061	-0.0009765625	\\
0.658187952235096	-0.000885009765625	\\
0.65823234341013	-0.00079345703125	\\
0.658276734585164	-0.00140380859375	\\
0.658321125760199	-0.00115966796875	\\
0.658365516935233	-0.000732421875	\\
0.658409908110268	-0.000823974609375	\\
0.658454299285302	-0.00079345703125	\\
0.658498690460336	-0.000885009765625	\\
0.658543081635371	-0.00115966796875	\\
0.658587472810405	-0.0013427734375	\\
0.65863186398544	-0.0010986328125	\\
0.658676255160474	-0.001312255859375	\\
0.658720646335509	-0.001312255859375	\\
0.658765037510543	-0.001190185546875	\\
0.658809428685577	-0.001129150390625	\\
0.658853819860612	-0.0010986328125	\\
0.658898211035646	-0.00128173828125	\\
0.65894260221068	-0.000946044921875	\\
0.658986993385715	-0.0008544921875	\\
0.659031384560749	-0.00091552734375	\\
0.659075775735784	-0.0008544921875	\\
0.659120166910818	-0.000885009765625	\\
0.659164558085853	-0.0008544921875	\\
0.659208949260887	-0.000946044921875	\\
0.659253340435921	-0.001068115234375	\\
0.659297731610956	-0.0009765625	\\
0.65934212278599	-0.000701904296875	\\
0.659386513961025	-0.0013427734375	\\
0.659430905136059	-0.001678466796875	\\
0.659475296311093	-0.00164794921875	\\
0.659519687486128	-0.001953125	\\
0.659564078661162	-0.0018310546875	\\
0.659608469836197	-0.001739501953125	\\
0.659652861011231	-0.00164794921875	\\
0.659697252186265	-0.002044677734375	\\
0.6597416433613	-0.001983642578125	\\
0.659786034536334	-0.002044677734375	\\
0.659830425711369	-0.002105712890625	\\
0.659874816886403	-0.00177001953125	\\
0.659919208061437	-0.00238037109375	\\
0.659963599236472	-0.00244140625	\\
0.660007990411506	-0.002227783203125	\\
0.660052381586541	-0.002288818359375	\\
0.660096772761575	-0.002288818359375	\\
0.660141163936609	-0.0028076171875	\\
0.660185555111644	-0.003143310546875	\\
0.660229946286678	-0.0030517578125	\\
0.660274337461713	-0.0029296875	\\
0.660318728636747	-0.002685546875	\\
0.660363119811781	-0.0025634765625	\\
0.660407510986816	-0.00244140625	\\
0.66045190216185	-0.002593994140625	\\
0.660496293336885	-0.002410888671875	\\
0.660540684511919	-0.002288818359375	\\
0.660585075686953	-0.002685546875	\\
0.660629466861988	-0.00250244140625	\\
0.660673858037022	-0.002166748046875	\\
0.660718249212057	-0.00201416015625	\\
0.660762640387091	-0.002197265625	\\
0.660807031562126	-0.002288818359375	\\
0.66085142273716	-0.00238037109375	\\
0.660895813912194	-0.001708984375	\\
0.660940205087229	-0.001708984375	\\
0.660984596262263	-0.001861572265625	\\
0.661028987437297	-0.00152587890625	\\
0.661073378612332	-0.002044677734375	\\
0.661117769787366	-0.001739501953125	\\
0.661162160962401	-0.001556396484375	\\
0.661206552137435	-0.001617431640625	\\
0.661250943312469	-0.00140380859375	\\
0.661295334487504	-0.00189208984375	\\
0.661339725662538	-0.00177001953125	\\
0.661384116837573	-0.00164794921875	\\
0.661428508012607	-0.001617431640625	\\
0.661472899187642	-0.00115966796875	\\
0.661517290362676	-0.001495361328125	\\
0.66156168153771	-0.001556396484375	\\
0.661606072712745	-0.001556396484375	\\
0.661650463887779	-0.0013427734375	\\
0.661694855062814	-0.0009765625	\\
0.661739246237848	-0.00128173828125	\\
0.661783637412882	-0.0010986328125	\\
0.661828028587917	-0.00115966796875	\\
0.661872419762951	-0.001312255859375	\\
0.661916810937985	-0.001129150390625	\\
0.66196120211302	-0.00091552734375	\\
0.662005593288054	-0.001312255859375	\\
0.662049984463089	-0.001708984375	\\
0.662094375638123	-0.0018310546875	\\
0.662138766813158	-0.00140380859375	\\
0.662183157988192	-0.001434326171875	\\
0.662227549163226	-0.00140380859375	\\
0.662271940338261	-0.001068115234375	\\
0.662316331513295	-0.0009765625	\\
0.66236072268833	-0.000640869140625	\\
0.662405113863364	-0.0009765625	\\
0.662449505038398	-0.000885009765625	\\
0.662493896213433	-0.000946044921875	\\
0.662538287388467	-0.001007080078125	\\
0.662582678563502	-0.001312255859375	\\
0.662627069738536	-0.001312255859375	\\
0.66267146091357	-0.000885009765625	\\
0.662715852088605	-0.00103759765625	\\
0.662760243263639	-0.001129150390625	\\
0.662804634438674	-0.000396728515625	\\
0.662849025613708	-0.000244140625	\\
0.662893416788742	-0.0006103515625	\\
0.662937807963777	-0.00067138671875	\\
0.662982199138811	-0.0006103515625	\\
0.663026590313846	-0.00079345703125	\\
0.66307098148888	-0.00067138671875	\\
0.663115372663914	-0.000762939453125	\\
0.663159763838949	-0.001068115234375	\\
0.663204155013983	-0.00103759765625	\\
0.663248546189018	-0.001129150390625	\\
0.663292937364052	-0.00146484375	\\
0.663337328539086	-0.00140380859375	\\
0.663381719714121	-0.00091552734375	\\
0.663426110889155	-0.000732421875	\\
0.66347050206419	-0.000762939453125	\\
0.663514893239224	-0.000885009765625	\\
0.663559284414258	-0.001007080078125	\\
0.663603675589293	-0.00164794921875	\\
0.663648066764327	-0.001434326171875	\\
0.663692457939362	-0.00140380859375	\\
0.663736849114396	-0.0013427734375	\\
0.66378124028943	-0.000885009765625	\\
0.663825631464465	-0.0008544921875	\\
0.663870022639499	-0.000640869140625	\\
0.663914413814534	-0.000396728515625	\\
0.663958804989568	-0.0009765625	\\
0.664003196164602	-0.001190185546875	\\
0.664047587339637	-0.001220703125	\\
0.664091978514671	-0.00152587890625	\\
0.664136369689706	-0.00177001953125	\\
0.66418076086474	-0.002105712890625	\\
0.664225152039775	-0.00189208984375	\\
0.664269543214809	-0.00140380859375	\\
0.664313934389843	-0.00128173828125	\\
0.664358325564878	-0.001434326171875	\\
0.664402716739912	-0.001129150390625	\\
0.664447107914947	-0.000732421875	\\
0.664491499089981	-0.000762939453125	\\
0.664535890265015	-0.0008544921875	\\
0.66458028144005	-0.0010986328125	\\
0.664624672615084	-0.001190185546875	\\
0.664669063790118	-0.001312255859375	\\
0.664713454965153	-0.001556396484375	\\
0.664757846140187	-0.00115966796875	\\
0.664802237315222	-0.001190185546875	\\
0.664846628490256	-0.0009765625	\\
0.664891019665291	-0.000762939453125	\\
0.664935410840325	-0.001007080078125	\\
0.664979802015359	-0.00128173828125	\\
0.665024193190394	-0.001251220703125	\\
0.665068584365428	-0.001617431640625	\\
0.665112975540463	-0.00164794921875	\\
0.665157366715497	-0.0013427734375	\\
0.665201757890531	-0.001312255859375	\\
0.665246149065566	-0.00128173828125	\\
0.6652905402406	-0.00146484375	\\
0.665334931415635	-0.001220703125	\\
0.665379322590669	-0.0009765625	\\
0.665423713765703	-0.001251220703125	\\
0.665468104940738	-0.00140380859375	\\
0.665512496115772	-0.001129150390625	\\
0.665556887290807	-0.00103759765625	\\
0.665601278465841	-0.0015869140625	\\
0.665645669640875	-0.00140380859375	\\
0.66569006081591	-0.00115966796875	\\
0.665734451990944	-0.00146484375	\\
0.665778843165979	-0.00146484375	\\
0.665823234341013	-0.001068115234375	\\
0.665867625516047	-0.00115966796875	\\
0.665912016691082	-0.000732421875	\\
0.665956407866116	-0.001007080078125	\\
0.666000799041151	-0.0015869140625	\\
0.666045190216185	-0.001129150390625	\\
0.666089581391219	-0.001068115234375	\\
0.666133972566254	-0.00115966796875	\\
0.666178363741288	-0.001220703125	\\
0.666222754916323	-0.001251220703125	\\
0.666267146091357	-0.001556396484375	\\
0.666311537266391	-0.001373291015625	\\
0.666355928441426	-0.0015869140625	\\
0.66640031961646	-0.00189208984375	\\
0.666444710791495	-0.001983642578125	\\
0.666489101966529	-0.002197265625	\\
0.666533493141563	-0.00225830078125	\\
0.666577884316598	-0.00250244140625	\\
0.666622275491632	-0.00213623046875	\\
0.666666666666667	-0.001617431640625	\\
0.666711057841701	-0.00177001953125	\\
0.666755449016735	-0.002593994140625	\\
0.66679984019177	-0.002471923828125	\\
0.666844231366804	-0.002410888671875	\\
0.666888622541839	-0.002197265625	\\
0.666933013716873	-0.002227783203125	\\
0.666977404891907	-0.00238037109375	\\
0.667021796066942	-0.001922607421875	\\
0.667066187241976	-0.00201416015625	\\
0.667110578417011	-0.001983642578125	\\
0.667154969592045	-0.002044677734375	\\
0.66719936076708	-0.001617431640625	\\
0.667243751942114	-0.00128173828125	\\
0.667288143117148	-0.0013427734375	\\
0.667332534292183	-0.001190185546875	\\
0.667376925467217	-0.001007080078125	\\
0.667421316642252	-0.001068115234375	\\
0.667465707817286	-0.001220703125	\\
0.66751009899232	-0.00140380859375	\\
0.667554490167355	-0.00128173828125	\\
0.667598881342389	-0.0010986328125	\\
0.667643272517424	-0.001617431640625	\\
0.667687663692458	-0.001800537109375	\\
0.667732054867492	-0.001953125	\\
0.667776446042527	-0.00201416015625	\\
0.667820837217561	-0.00146484375	\\
0.667865228392596	-0.00146484375	\\
0.66790961956763	-0.001617431640625	\\
0.667954010742664	-0.0020751953125	\\
0.667998401917699	-0.002410888671875	\\
0.668042793092733	-0.002410888671875	\\
0.668087184267768	-0.002197265625	\\
0.668131575442802	-0.002288818359375	\\
0.668175966617836	-0.002471923828125	\\
0.668220357792871	-0.001861572265625	\\
0.668264748967905	-0.002227783203125	\\
0.66830914014294	-0.00238037109375	\\
0.668353531317974	-0.0020751953125	\\
0.668397922493008	-0.002197265625	\\
0.668442313668043	-0.002166748046875	\\
0.668486704843077	-0.002349853515625	\\
0.668531096018112	-0.001678466796875	\\
0.668575487193146	-0.0018310546875	\\
0.66861987836818	-0.002471923828125	\\
0.668664269543215	-0.002685546875	\\
0.668708660718249	-0.002655029296875	\\
0.668753051893284	-0.00244140625	\\
0.668797443068318	-0.0028076171875	\\
0.668841834243352	-0.00286865234375	\\
0.668886225418387	-0.00225830078125	\\
0.668930616593421	-0.002593994140625	\\
0.668975007768456	-0.002593994140625	\\
0.66901939894349	-0.0025634765625	\\
0.669063790118524	-0.0032958984375	\\
0.669108181293559	-0.002593994140625	\\
0.669152572468593	-0.002532958984375	\\
0.669196963643628	-0.00262451171875	\\
0.669241354818662	-0.002349853515625	\\
0.669285745993697	-0.002685546875	\\
0.669330137168731	-0.002838134765625	\\
0.669374528343765	-0.002838134765625	\\
0.6694189195188	-0.00262451171875	\\
0.669463310693834	-0.003021240234375	\\
0.669507701868868	-0.003021240234375	\\
0.669552093043903	-0.003021240234375	\\
0.669596484218937	-0.003082275390625	\\
0.669640875393972	-0.00274658203125	\\
0.669685266569006	-0.002655029296875	\\
0.66972965774404	-0.00201416015625	\\
0.669774048919075	-0.00238037109375	\\
0.669818440094109	-0.002532958984375	\\
0.669862831269144	-0.00177001953125	\\
0.669907222444178	-0.0015869140625	\\
0.669951613619213	-0.001373291015625	\\
0.669996004794247	-0.001708984375	\\
0.670040395969281	-0.001617431640625	\\
0.670084787144316	-0.001251220703125	\\
0.67012917831935	-0.0009765625	\\
0.670173569494385	-0.001251220703125	\\
0.670217960669419	-0.0009765625	\\
0.670262351844453	-0.000762939453125	\\
0.670306743019488	-0.00103759765625	\\
0.670351134194522	-0.000732421875	\\
0.670395525369556	-0.0006103515625	\\
0.670439916544591	-0.000732421875	\\
0.670484307719625	-0.00042724609375	\\
0.67052869889466	-0.00048828125	\\
0.670573090069694	-0.000732421875	\\
0.670617481244729	-0.000274658203125	\\
0.670661872419763	-0.0006103515625	\\
0.670706263594797	-0.00079345703125	\\
0.670750654769832	-0.000732421875	\\
0.670795045944866	-0.00042724609375	\\
0.670839437119901	-0.00067138671875	\\
0.670883828294935	-0.00146484375	\\
0.670928219469969	-0.001312255859375	\\
0.670972610645004	-0.001251220703125	\\
0.671017001820038	-0.00146484375	\\
0.671061392995073	-0.001556396484375	\\
0.671105784170107	-0.0018310546875	\\
0.671150175345141	-0.0020751953125	\\
0.671194566520176	-0.001922607421875	\\
0.67123895769521	-0.0020751953125	\\
0.671283348870245	-0.002105712890625	\\
0.671327740045279	-0.001922607421875	\\
0.671372131220313	-0.002044677734375	\\
0.671416522395348	-0.001800537109375	\\
0.671460913570382	-0.00189208984375	\\
0.671505304745417	-0.00225830078125	\\
0.671549695920451	-0.002197265625	\\
0.671594087095485	-0.001800537109375	\\
0.67163847827052	-0.0018310546875	\\
0.671682869445554	-0.00213623046875	\\
0.671727260620589	-0.002349853515625	\\
0.671771651795623	-0.001953125	\\
0.671816042970657	-0.001678466796875	\\
0.671860434145692	-0.001312255859375	\\
0.671904825320726	-0.0008544921875	\\
0.671949216495761	-0.000946044921875	\\
0.671993607670795	-0.001007080078125	\\
0.672037998845829	-0.00115966796875	\\
0.672082390020864	-0.000885009765625	\\
0.672126781195898	-0.001251220703125	\\
0.672171172370933	-0.001007080078125	\\
0.672215563545967	-0.000823974609375	\\
0.672259954721001	-0.000946044921875	\\
0.672304345896036	-0.0006103515625	\\
0.67234873707107	-0.0001220703125	\\
0.672393128246105	-0.00030517578125	\\
0.672437519421139	-0.0001220703125	\\
0.672481910596173	0.000396728515625	\\
0.672526301771208	-6.103515625e-05	\\
0.672570692946242	-3.0517578125e-05	\\
0.672615084121277	0.0001220703125	\\
0.672659475296311	0.00048828125	\\
0.672703866471346	0.000885009765625	\\
0.67274825764638	0.000579833984375	\\
0.672792648821414	0.000579833984375	\\
0.672837039996449	0.000579833984375	\\
0.672881431171483	0.000579833984375	\\
0.672925822346518	0.00048828125	\\
0.672970213521552	0.000244140625	\\
0.673014604696586	6.103515625e-05	\\
0.673058995871621	0.00018310546875	\\
0.673103387046655	0.000244140625	\\
0.673147778221689	-0.000244140625	\\
0.673192169396724	-0.00042724609375	\\
0.673236560571758	-0.00048828125	\\
0.673280951746793	-0.0003662109375	\\
0.673325342921827	-0.00042724609375	\\
0.673369734096862	-0.000762939453125	\\
0.673414125271896	-0.000396728515625	\\
0.67345851644693	-0.000518798828125	\\
0.673502907621965	-0.0008544921875	\\
0.673547298796999	-0.000885009765625	\\
0.673591689972034	-0.00091552734375	\\
0.673636081147068	-0.0010986328125	\\
0.673680472322102	-0.001708984375	\\
0.673724863497137	-0.001800537109375	\\
0.673769254672171	-0.001861572265625	\\
0.673813645847206	-0.001800537109375	\\
0.67385803702224	-0.00189208984375	\\
0.673902428197274	-0.00213623046875	\\
0.673946819372309	-0.0020751953125	\\
0.673991210547343	-0.002044677734375	\\
0.674035601722378	-0.001739501953125	\\
0.674079992897412	-0.001922607421875	\\
0.674124384072446	-0.0018310546875	\\
0.674168775247481	-0.001617431640625	\\
0.674213166422515	-0.001556396484375	\\
0.67425755759755	-0.00177001953125	\\
0.674301948772584	-0.002105712890625	\\
0.674346339947618	-0.00189208984375	\\
0.674390731122653	-0.001861572265625	\\
0.674435122297687	-0.00201416015625	\\
0.674479513472722	-0.00201416015625	\\
0.674523904647756	-0.00177001953125	\\
0.67456829582279	-0.001434326171875	\\
0.674612686997825	-0.00152587890625	\\
0.674657078172859	-0.00140380859375	\\
0.674701469347894	-0.001220703125	\\
0.674745860522928	-0.001434326171875	\\
0.674790251697962	-0.00140380859375	\\
0.674834642872997	-0.00146484375	\\
0.674879034048031	-0.002044677734375	\\
0.674923425223066	-0.00225830078125	\\
0.6749678163981	-0.0020751953125	\\
0.675012207573135	-0.002197265625	\\
0.675056598748169	-0.0023193359375	\\
0.675100989923203	-0.00238037109375	\\
0.675145381098238	-0.002410888671875	\\
0.675189772273272	-0.002777099609375	\\
0.675234163448306	-0.00286865234375	\\
0.675278554623341	-0.002899169921875	\\
0.675322945798375	-0.002838134765625	\\
0.67536733697341	-0.0025634765625	\\
0.675411728148444	-0.0028076171875	\\
0.675456119323478	-0.003143310546875	\\
0.675500510498513	-0.0029296875	\\
0.675544901673547	-0.00286865234375	\\
0.675589292848582	-0.002532958984375	\\
0.675633684023616	-0.0025634765625	\\
0.675678075198651	-0.002899169921875	\\
0.675722466373685	-0.002197265625	\\
0.675766857548719	-0.001739501953125	\\
0.675811248723754	-0.001739501953125	\\
0.675855639898788	-0.002166748046875	\\
0.675900031073823	-0.0020751953125	\\
0.675944422248857	-0.002197265625	\\
0.675988813423891	-0.00244140625	\\
0.676033204598926	-0.001983642578125	\\
0.67607759577396	-0.002227783203125	\\
0.676121986948995	-0.002471923828125	\\
0.676166378124029	-0.0023193359375	\\
0.676210769299063	-0.0020751953125	\\
0.676255160474098	-0.0015869140625	\\
0.676299551649132	-0.0015869140625	\\
0.676343942824167	-0.001495361328125	\\
0.676388333999201	-0.00067138671875	\\
0.676432725174235	-0.00067138671875	\\
0.67647711634927	-0.00067138671875	\\
0.676521507524304	-0.00054931640625	\\
0.676565898699339	-0.000823974609375	\\
0.676610289874373	-0.000274658203125	\\
0.676654681049407	-0.000823974609375	\\
0.676699072224442	-0.000762939453125	\\
0.676743463399476	-0.000732421875	\\
0.676787854574511	-0.001312255859375	\\
0.676832245749545	-0.001495361328125	\\
0.676876636924579	-0.001617431640625	\\
0.676921028099614	-0.001678466796875	\\
0.676965419274648	-0.002166748046875	\\
0.677009810449683	-0.002166748046875	\\
0.677054201624717	-0.001922607421875	\\
0.677098592799751	-0.001739501953125	\\
0.677142983974786	-0.001678466796875	\\
0.67718737514982	-0.001800537109375	\\
0.677231766324855	-0.001678466796875	\\
0.677276157499889	-0.00189208984375	\\
0.677320548674923	-0.00213623046875	\\
0.677364939849958	-0.002105712890625	\\
0.677409331024992	-0.002410888671875	\\
0.677453722200027	-0.002593994140625	\\
0.677498113375061	-0.0025634765625	\\
0.677542504550095	-0.00274658203125	\\
0.67758689572513	-0.002777099609375	\\
0.677631286900164	-0.002960205078125	\\
0.677675678075199	-0.002899169921875	\\
0.677720069250233	-0.002685546875	\\
0.677764460425268	-0.002899169921875	\\
0.677808851600302	-0.002716064453125	\\
0.677853242775336	-0.002777099609375	\\
0.677897633950371	-0.002777099609375	\\
0.677942025125405	-0.0029296875	\\
0.677986416300439	-0.002960205078125	\\
0.678030807475474	-0.002777099609375	\\
0.678075198650508	-0.002471923828125	\\
0.678119589825543	-0.002166748046875	\\
0.678163981000577	-0.001983642578125	\\
0.678208372175611	-0.0015869140625	\\
0.678252763350646	-0.001495361328125	\\
0.67829715452568	-0.001495361328125	\\
0.678341545700715	-0.001708984375	\\
0.678385936875749	-0.00213623046875	\\
0.678430328050784	-0.001861572265625	\\
0.678474719225818	-0.001556396484375	\\
0.678519110400852	-0.001495361328125	\\
0.678563501575887	-0.00152587890625	\\
0.678607892750921	-0.00128173828125	\\
0.678652283925956	-0.001434326171875	\\
0.67869667510099	-0.001434326171875	\\
0.678741066276024	-0.00164794921875	\\
0.678785457451059	-0.00140380859375	\\
0.678829848626093	-0.00103759765625	\\
0.678874239801127	-0.001373291015625	\\
0.678918630976162	-0.001373291015625	\\
0.678963022151196	-0.0013427734375	\\
0.679007413326231	-0.001373291015625	\\
0.679051804501265	-0.00128173828125	\\
0.6790961956763	-0.001861572265625	\\
0.679140586851334	-0.002044677734375	\\
0.679184978026368	-0.001800537109375	\\
0.679229369201403	-0.001861572265625	\\
0.679273760376437	-0.0018310546875	\\
0.679318151551472	-0.00250244140625	\\
0.679362542726506	-0.00262451171875	\\
0.67940693390154	-0.00286865234375	\\
0.679451325076575	-0.0030517578125	\\
0.679495716251609	-0.003143310546875	\\
0.679540107426644	-0.003662109375	\\
0.679584498601678	-0.00323486328125	\\
0.679628889776712	-0.002655029296875	\\
0.679673280951747	-0.002655029296875	\\
0.679717672126781	-0.00250244140625	\\
0.679762063301816	-0.00225830078125	\\
0.67980645447685	-0.00189208984375	\\
0.679850845651884	-0.001983642578125	\\
0.679895236826919	-0.001678466796875	\\
0.679939628001953	-0.00140380859375	\\
0.679984019176988	-0.001953125	\\
0.680028410352022	-0.00177001953125	\\
0.680072801527056	-0.00177001953125	\\
0.680117192702091	-0.0013427734375	\\
0.680161583877125	-0.00103759765625	\\
0.68020597505216	-0.001007080078125	\\
0.680250366227194	-0.00079345703125	\\
0.680294757402228	-0.000518798828125	\\
0.680339148577263	-0.00018310546875	\\
0.680383539752297	0	\\
0.680427930927332	0.0006103515625	\\
0.680472322102366	0.000640869140625	\\
0.6805167132774	0.000885009765625	\\
0.680561104452435	0.00140380859375	\\
0.680605495627469	0.001129150390625	\\
0.680649886802504	0.00146484375	\\
0.680694277977538	0.00115966796875	\\
0.680738669152572	0.00128173828125	\\
0.680783060327607	0.001434326171875	\\
0.680827451502641	0.000946044921875	\\
0.680871842677676	0.000701904296875	\\
0.68091623385271	0.000762939453125	\\
0.680960625027744	0.00067138671875	\\
0.681005016202779	0.00091552734375	\\
0.681049407377813	0.00091552734375	\\
0.681093798552848	0.0006103515625	\\
0.681138189727882	0.00048828125	\\
0.681182580902917	0.00018310546875	\\
0.681226972077951	0.000457763671875	\\
0.681271363252985	0.000457763671875	\\
0.68131575442802	0.000152587890625	\\
0.681360145603054	-9.1552734375e-05	\\
0.681404536778089	-0.000396728515625	\\
0.681448927953123	-0.000885009765625	\\
0.681493319128157	-0.001068115234375	\\
0.681537710303192	-0.001220703125	\\
0.681582101478226	-0.0008544921875	\\
0.681626492653261	-0.000762939453125	\\
0.681670883828295	-0.00146484375	\\
0.681715275003329	-0.001251220703125	\\
0.681759666178364	-0.00152587890625	\\
0.681804057353398	-0.001922607421875	\\
0.681848448528433	-0.001617431640625	\\
0.681892839703467	-0.00128173828125	\\
0.681937230878501	-0.00164794921875	\\
0.681981622053536	-0.001739501953125	\\
0.68202601322857	-0.00189208984375	\\
0.682070404403605	-0.001739501953125	\\
0.682114795578639	-0.001861572265625	\\
0.682159186753673	-0.002349853515625	\\
0.682203577928708	-0.001739501953125	\\
0.682247969103742	-0.002044677734375	\\
0.682292360278777	-0.002471923828125	\\
0.682336751453811	-0.001953125	\\
0.682381142628845	-0.001953125	\\
0.68242553380388	-0.001800537109375	\\
0.682469924978914	-0.001708984375	\\
0.682514316153949	-0.001983642578125	\\
0.682558707328983	-0.00177001953125	\\
0.682603098504017	-0.001190185546875	\\
0.682647489679052	-0.000640869140625	\\
0.682691880854086	-0.000518798828125	\\
0.682736272029121	-0.000274658203125	\\
0.682780663204155	0.00030517578125	\\
0.682825054379189	0.000640869140625	\\
0.682869445554224	0.000457763671875	\\
0.682913836729258	0.00042724609375	\\
0.682958227904293	0.000396728515625	\\
0.683002619079327	-9.1552734375e-05	\\
0.683047010254361	-0.00042724609375	\\
0.683091401429396	-0.000701904296875	\\
0.68313579260443	-0.001068115234375	\\
0.683180183779465	-0.0010986328125	\\
0.683224574954499	-0.00146484375	\\
0.683268966129533	-0.001373291015625	\\
0.683313357304568	-0.001190185546875	\\
0.683357748479602	-0.001251220703125	\\
0.683402139654637	-0.00115966796875	\\
0.683446530829671	-0.001129150390625	\\
0.683490922004706	-0.00128173828125	\\
0.68353531317974	-0.001251220703125	\\
0.683579704354774	-0.001129150390625	\\
0.683624095529809	-0.00146484375	\\
0.683668486704843	-0.00164794921875	\\
0.683712877879877	-0.001739501953125	\\
0.683757269054912	-0.002197265625	\\
0.683801660229946	-0.002532958984375	\\
0.683846051404981	-0.002716064453125	\\
0.683890442580015	-0.00238037109375	\\
0.683934833755049	-0.0028076171875	\\
0.683979224930084	-0.003265380859375	\\
0.684023616105118	-0.002655029296875	\\
0.684068007280153	-0.002532958984375	\\
0.684112398455187	-0.002471923828125	\\
0.684156789630222	-0.00250244140625	\\
0.684201180805256	-0.00244140625	\\
0.68424557198029	-0.001922607421875	\\
0.684289963155325	-0.001800537109375	\\
0.684334354330359	-0.001922607421875	\\
0.684378745505394	-0.001922607421875	\\
0.684423136680428	-0.001708984375	\\
0.684467527855462	-0.00177001953125	\\
0.684511919030497	-0.00177001953125	\\
0.684556310205531	-0.00164794921875	\\
0.684600701380566	-0.00177001953125	\\
0.6846450925556	-0.001708984375	\\
0.684689483730634	-0.00164794921875	\\
0.684733874905669	-0.001922607421875	\\
0.684778266080703	-0.001953125	\\
0.684822657255738	-0.001953125	\\
0.684867048430772	-0.00225830078125	\\
0.684911439605806	-0.00164794921875	\\
0.684955830780841	-0.00152587890625	\\
0.685000221955875	-0.001800537109375	\\
0.68504461313091	-0.00140380859375	\\
0.685089004305944	-0.001739501953125	\\
0.685133395480978	-0.001861572265625	\\
0.685177786656013	-0.001739501953125	\\
0.685222177831047	-0.00152587890625	\\
0.685266569006082	-0.0015869140625	\\
0.685310960181116	-0.00152587890625	\\
0.68535535135615	-0.00189208984375	\\
0.685399742531185	-0.002410888671875	\\
0.685444133706219	-0.00238037109375	\\
0.685488524881254	-0.001800537109375	\\
0.685532916056288	-0.001708984375	\\
0.685577307231322	-0.00225830078125	\\
0.685621698406357	-0.00250244140625	\\
0.685666089581391	-0.002410888671875	\\
0.685710480756426	-0.00250244140625	\\
0.68575487193146	-0.00225830078125	\\
0.685799263106494	-0.002105712890625	\\
0.685843654281529	-0.00238037109375	\\
0.685888045456563	-0.002044677734375	\\
0.685932436631598	-0.0018310546875	\\
0.685976827806632	-0.001800537109375	\\
0.686021218981666	-0.001739501953125	\\
0.686065610156701	-0.00164794921875	\\
0.686110001331735	-0.001495361328125	\\
0.68615439250677	-0.0015869140625	\\
0.686198783681804	-0.00128173828125	\\
0.686243174856839	-0.000946044921875	\\
0.686287566031873	-0.00091552734375	\\
0.686331957206907	-0.000518798828125	\\
0.686376348381942	-0.00048828125	\\
0.686420739556976	-0.00048828125	\\
0.68646513073201	-0.000885009765625	\\
0.686509521907045	-0.00054931640625	\\
0.686553913082079	-0.000396728515625	\\
0.686598304257114	-0.000518798828125	\\
0.686642695432148	-0.000518798828125	\\
0.686687086607182	-0.000762939453125	\\
0.686731477782217	-0.00079345703125	\\
0.686775868957251	-0.000640869140625	\\
0.686820260132286	-0.0003662109375	\\
0.68686465130732	-0.000396728515625	\\
0.686909042482355	-3.0517578125e-05	\\
0.686953433657389	-0.0003662109375	\\
0.686997824832423	-0.0009765625	\\
0.687042216007458	-0.000518798828125	\\
0.687086607182492	-0.00030517578125	\\
0.687130998357527	-0.000579833984375	\\
0.687175389532561	-0.00091552734375	\\
0.687219780707595	-0.00128173828125	\\
0.68726417188263	-0.001251220703125	\\
0.687308563057664	-0.001708984375	\\
0.687352954232698	-0.001556396484375	\\
0.687397345407733	-0.001922607421875	\\
0.687441736582767	-0.002410888671875	\\
0.687486127757802	-0.0023193359375	\\
0.687530518932836	-0.00238037109375	\\
0.687574910107871	-0.002349853515625	\\
0.687619301282905	-0.00238037109375	\\
0.687663692457939	-0.001983642578125	\\
0.687708083632974	-0.001800537109375	\\
0.687752474808008	-0.001495361328125	\\
0.687796865983043	-0.000885009765625	\\
0.687841257158077	-0.00018310546875	\\
0.687885648333111	-0.00048828125	\\
0.687930039508146	-0.000518798828125	\\
0.68797443068318	0.0001220703125	\\
0.688018821858215	-3.0517578125e-05	\\
0.688063213033249	-0.000213623046875	\\
0.688107604208283	-0.000213623046875	\\
0.688151995383318	-0.00048828125	\\
0.688196386558352	-6.103515625e-05	\\
0.688240777733387	0.00030517578125	\\
0.688285168908421	0.000732421875	\\
0.688329560083455	0.001312255859375	\\
0.68837395125849	0.00164794921875	\\
0.688418342433524	0.00177001953125	\\
0.688462733608559	0.00250244140625	\\
0.688507124783593	0.002716064453125	\\
0.688551515958627	0.002471923828125	\\
0.688595907133662	0.002655029296875	\\
0.688640298308696	0.002349853515625	\\
0.688684689483731	0.00238037109375	\\
0.688729080658765	0.002532958984375	\\
0.688773471833799	0.002349853515625	\\
0.688817863008834	0.002044677734375	\\
0.688862254183868	0.0015869140625	\\
0.688906645358903	0.0015869140625	\\
0.688951036533937	0.001800537109375	\\
0.688995427708971	0.001861572265625	\\
0.689039818884006	0.002044677734375	\\
0.68908421005904	0.00189208984375	\\
0.689128601234075	0.0015869140625	\\
0.689172992409109	0.00128173828125	\\
0.689217383584143	0.000823974609375	\\
0.689261774759178	0.000518798828125	\\
0.689306165934212	-3.0517578125e-05	\\
0.689350557109247	-0.0008544921875	\\
0.689394948284281	-0.00115966796875	\\
0.689439339459315	-0.00152587890625	\\
0.68948373063435	-0.00213623046875	\\
0.689528121809384	-0.00250244140625	\\
0.689572512984419	-0.00244140625	\\
0.689616904159453	-0.00274658203125	\\
0.689661295334488	-0.0028076171875	\\
0.689705686509522	-0.002593994140625	\\
0.689750077684556	-0.002777099609375	\\
0.689794468859591	-0.00250244140625	\\
0.689838860034625	-0.002288818359375	\\
0.68988325120966	-0.002532958984375	\\
0.689927642384694	-0.002593994140625	\\
0.689972033559728	-0.00262451171875	\\
0.690016424734763	-0.00286865234375	\\
0.690060815909797	-0.00250244140625	\\
0.690105207084832	-0.002471923828125	\\
0.690149598259866	-0.002288818359375	\\
0.6901939894349	-0.002288818359375	\\
0.690238380609935	-0.002471923828125	\\
0.690282771784969	-0.002349853515625	\\
0.690327162960004	-0.002105712890625	\\
0.690371554135038	-0.00164794921875	\\
0.690415945310072	-0.0010986328125	\\
0.690460336485107	-0.0006103515625	\\
0.690504727660141	-0.000518798828125	\\
0.690549118835176	-0.0001220703125	\\
0.69059351001021	0.00042724609375	\\
0.690637901185244	6.103515625e-05	\\
0.690682292360279	0.000396728515625	\\
0.690726683535313	0.000640869140625	\\
0.690771074710348	0.000946044921875	\\
0.690815465885382	0.001129150390625	\\
0.690859857060416	0.001251220703125	\\
0.690904248235451	0.00140380859375	\\
0.690948639410485	0.001068115234375	\\
0.69099303058552	0.001007080078125	\\
0.691037421760554	0.0001220703125	\\
0.691081812935588	-9.1552734375e-05	\\
0.691126204110623	-0.000244140625	\\
0.691170595285657	-0.000244140625	\\
0.691214986460692	0.0001220703125	\\
0.691259377635726	3.0517578125e-05	\\
0.69130376881076	0.0003662109375	\\
0.691348159985795	0.000518798828125	\\
0.691392551160829	0.0003662109375	\\
0.691436942335864	0.000213623046875	\\
0.691481333510898	0.000274658203125	\\
0.691525724685932	0.000244140625	\\
0.691570115860967	-0.000732421875	\\
0.691614507036001	-0.001068115234375	\\
0.691658898211036	-0.000823974609375	\\
0.69170328938607	-0.001007080078125	\\
0.691747680561104	-0.00115966796875	\\
0.691792071736139	-0.00140380859375	\\
0.691836462911173	-0.001434326171875	\\
0.691880854086208	-0.00164794921875	\\
0.691925245261242	-0.001678466796875	\\
0.691969636436277	-0.0013427734375	\\
0.692014027611311	-0.00128173828125	\\
0.692058418786345	-0.001129150390625	\\
0.69210280996138	-0.001220703125	\\
0.692147201136414	-0.000946044921875	\\
0.692191592311448	-0.000579833984375	\\
0.692235983486483	-0.00054931640625	\\
0.692280374661517	-0.000701904296875	\\
0.692324765836552	-0.000732421875	\\
0.692369157011586	-0.00042724609375	\\
0.69241354818662	-0.000244140625	\\
0.692457939361655	-0.000732421875	\\
0.692502330536689	-0.000518798828125	\\
0.692546721711724	-0.00042724609375	\\
0.692591112886758	-0.00030517578125	\\
0.692635504061793	-9.1552734375e-05	\\
0.692679895236827	-0.000244140625	\\
0.692724286411861	-0.000335693359375	\\
0.692768677586896	-0.00018310546875	\\
0.69281306876193	-0.000244140625	\\
0.692857459936965	0.000213623046875	\\
0.692901851111999	0.000244140625	\\
0.692946242287033	-6.103515625e-05	\\
0.692990633462068	0.000152587890625	\\
0.693035024637102	0.0001220703125	\\
0.693079415812137	-0.000152587890625	\\
0.693123806987171	-0.0001220703125	\\
0.693168198162205	-0.000274658203125	\\
0.69321258933724	-0.00018310546875	\\
0.693256980512274	-0.000457763671875	\\
0.693301371687309	-0.000762939453125	\\
0.693345762862343	-0.0003662109375	\\
0.693390154037377	-0.000640869140625	\\
0.693434545212412	-0.000518798828125	\\
0.693478936387446	-0.0006103515625	\\
0.693523327562481	-0.00091552734375	\\
0.693567718737515	-0.000732421875	\\
0.693612109912549	-0.001190185546875	\\
0.693656501087584	-0.001556396484375	\\
0.693700892262618	-0.00213623046875	\\
0.693745283437653	-0.00189208984375	\\
0.693789674612687	-0.001434326171875	\\
0.693834065787721	-0.00164794921875	\\
0.693878456962756	-0.00152587890625	\\
0.69392284813779	-0.001220703125	\\
0.693967239312825	-0.001312255859375	\\
0.694011630487859	-0.001220703125	\\
0.694056021662893	-0.000762939453125	\\
0.694100412837928	-0.00091552734375	\\
0.694144804012962	-0.001220703125	\\
0.694189195187997	-0.00115966796875	\\
0.694233586363031	-0.001007080078125	\\
0.694277977538065	-0.001068115234375	\\
0.6943223687131	-0.000823974609375	\\
0.694366759888134	-0.000579833984375	\\
0.694411151063169	-0.000274658203125	\\
0.694455542238203	0.00018310546875	\\
0.694499933413237	0.0001220703125	\\
0.694544324588272	0.0001220703125	\\
0.694588715763306	3.0517578125e-05	\\
0.694633106938341	-0.000396728515625	\\
0.694677498113375	-0.000244140625	\\
0.69472188928841	-0.000518798828125	\\
0.694766280463444	-0.000732421875	\\
0.694810671638478	-0.000732421875	\\
0.694855062813513	-0.000885009765625	\\
0.694899453988547	-0.00079345703125	\\
0.694943845163581	-0.001312255859375	\\
0.694988236338616	-0.00152587890625	\\
0.69503262751365	-0.001678466796875	\\
0.695077018688685	-0.00238037109375	\\
0.695121409863719	-0.002471923828125	\\
0.695165801038753	-0.002655029296875	\\
0.695210192213788	-0.00262451171875	\\
0.695254583388822	-0.002960205078125	\\
0.695298974563857	-0.003021240234375	\\
0.695343365738891	-0.003173828125	\\
0.695387756913926	-0.00347900390625	\\
0.69543214808896	-0.00341796875	\\
0.695476539263994	-0.004180908203125	\\
0.695520930439029	-0.004302978515625	\\
0.695565321614063	-0.00421142578125	\\
0.695609712789098	-0.004302978515625	\\
0.695654103964132	-0.003997802734375	\\
0.695698495139166	-0.00311279296875	\\
0.695742886314201	-0.002410888671875	\\
0.695787277489235	-0.00238037109375	\\
0.69583166866427	-0.002166748046875	\\
0.695876059839304	-0.00189208984375	\\
0.695920451014338	-0.001953125	\\
0.695964842189373	-0.002166748046875	\\
0.696009233364407	-0.00244140625	\\
0.696053624539442	-0.00213623046875	\\
0.696098015714476	-0.001708984375	\\
0.69614240688951	-0.000946044921875	\\
0.696186798064545	-0.000335693359375	\\
0.696231189239579	0.000274658203125	\\
0.696275580414614	0.001220703125	\\
0.696319971589648	0.00146484375	\\
0.696364362764682	0.00177001953125	\\
0.696408753939717	0.0025634765625	\\
0.696453145114751	0.002716064453125	\\
0.696497536289786	0.002593994140625	\\
0.69654192746482	0.00311279296875	\\
0.696586318639854	0.002716064453125	\\
0.696630709814889	0.002838134765625	\\
0.696675100989923	0.003173828125	\\
0.696719492164958	0.00244140625	\\
0.696763883339992	0.002197265625	\\
0.696808274515026	0.002471923828125	\\
0.696852665690061	0.00286865234375	\\
0.696897056865095	0.00286865234375	\\
0.69694144804013	0.00274658203125	\\
0.696985839215164	0.002655029296875	\\
0.697030230390198	0.00225830078125	\\
0.697074621565233	0.00177001953125	\\
0.697119012740267	0.00146484375	\\
0.697163403915302	0.000885009765625	\\
0.697207795090336	0.0006103515625	\\
0.69725218626537	9.1552734375e-05	\\
0.697296577440405	-0.000762939453125	\\
0.697340968615439	-0.001373291015625	\\
0.697385359790474	-0.00152587890625	\\
0.697429750965508	-0.001922607421875	\\
0.697474142140542	-0.00189208984375	\\
0.697518533315577	-0.0013427734375	\\
0.697562924490611	-0.002471923828125	\\
0.697607315665646	-0.00299072265625	\\
0.69765170684068	-0.00299072265625	\\
0.697696098015715	-0.003082275390625	\\
0.697740489190749	-0.0030517578125	\\
0.697784880365783	-0.003143310546875	\\
0.697829271540818	-0.002838134765625	\\
0.697873662715852	-0.003173828125	\\
0.697918053890886	-0.0032958984375	\\
0.697962445065921	-0.0029296875	\\
0.698006836240955	-0.0028076171875	\\
0.69805122741599	-0.003082275390625	\\
0.698095618591024	-0.00335693359375	\\
0.698140009766059	-0.003143310546875	\\
0.698184400941093	-0.003021240234375	\\
0.698228792116127	-0.003204345703125	\\
0.698273183291162	-0.003021240234375	\\
0.698317574466196	-0.00262451171875	\\
0.698361965641231	-0.0023193359375	\\
0.698406356816265	-0.00225830078125	\\
0.698450747991299	-0.0018310546875	\\
0.698495139166334	-0.001678466796875	\\
0.698539530341368	-0.001190185546875	\\
0.698583921516403	-0.00054931640625	\\
0.698628312691437	-0.000579833984375	\\
0.698672703866471	9.1552734375e-05	\\
0.698717095041506	0.0003662109375	\\
0.69876148621654	0.00054931640625	\\
0.698805877391575	0.000701904296875	\\
0.698850268566609	0.00042724609375	\\
0.698894659741643	0.000457763671875	\\
0.698939050916678	0.00018310546875	\\
0.698983442091712	-0.00030517578125	\\
0.699027833266747	-0.000732421875	\\
0.699072224441781	-0.000640869140625	\\
0.699116615616815	-0.000457763671875	\\
0.69916100679185	-0.000335693359375	\\
0.699205397966884	-0.000579833984375	\\
0.699249789141919	-0.000518798828125	\\
0.699294180316953	-0.000335693359375	\\
0.699338571491987	-0.0006103515625	\\
0.699382962667022	-0.000457763671875	\\
0.699427353842056	-0.00030517578125	\\
0.699471745017091	-0.00042724609375	\\
0.699516136192125	-0.0006103515625	\\
0.699560527367159	-0.000701904296875	\\
0.699604918542194	-0.001007080078125	\\
0.699649309717228	-0.001617431640625	\\
0.699693700892263	-0.002166748046875	\\
0.699738092067297	-0.001983642578125	\\
0.699782483242331	-0.00213623046875	\\
0.699826874417366	-0.002349853515625	\\
0.6998712655924	-0.00262451171875	\\
0.699915656767435	-0.00311279296875	\\
0.699960047942469	-0.002655029296875	\\
0.700004439117503	-0.0020751953125	\\
0.700048830292538	-0.002197265625	\\
0.700093221467572	-0.001922607421875	\\
0.700137612642607	-0.001800537109375	\\
0.700182003817641	-0.001708984375	\\
0.700226394992675	-0.0013427734375	\\
0.70027078616771	-0.001373291015625	\\
0.700315177342744	-0.001312255859375	\\
0.700359568517779	-0.001129150390625	\\
0.700403959692813	-0.0009765625	\\
0.700448350867848	-0.0003662109375	\\
0.700492742042882	-0.000396728515625	\\
0.700537133217916	-0.000213623046875	\\
0.700581524392951	6.103515625e-05	\\
0.700625915567985	0.000274658203125	\\
0.700670306743019	0.00018310546875	\\
0.700714697918054	3.0517578125e-05	\\
0.700759089093088	0.0003662109375	\\
0.700803480268123	0.000579833984375	\\
0.700847871443157	0.0001220703125	\\
0.700892262618191	0.0001220703125	\\
0.700936653793226	6.103515625e-05	\\
0.70098104496826	-0.00030517578125	\\
0.701025436143295	-0.0001220703125	\\
0.701069827318329	-0.000274658203125	\\
0.701114218493364	-0.000579833984375	\\
0.701158609668398	-0.000701904296875	\\
0.701203000843432	-0.000701904296875	\\
0.701247392018467	-0.000732421875	\\
0.701291783193501	-0.00091552734375	\\
0.701336174368536	-0.001495361328125	\\
0.70138056554357	-0.001617431640625	\\
0.701424956718604	-0.00146484375	\\
0.701469347893639	-0.0018310546875	\\
0.701513739068673	-0.002166748046875	\\
0.701558130243708	-0.002471923828125	\\
0.701602521418742	-0.002838134765625	\\
0.701646912593776	-0.00286865234375	\\
0.701691303768811	-0.00262451171875	\\
0.701735694943845	-0.003143310546875	\\
0.70178008611888	-0.0030517578125	\\
0.701824477293914	-0.002960205078125	\\
0.701868868468948	-0.003021240234375	\\
0.701913259643983	-0.0028076171875	\\
0.701957650819017	-0.0029296875	\\
0.702002041994052	-0.002197265625	\\
0.702046433169086	-0.00164794921875	\\
0.70209082434412	-0.001922607421875	\\
0.702135215519155	-0.001617431640625	\\
0.702179606694189	-0.001373291015625	\\
0.702223997869224	-0.001373291015625	\\
0.702268389044258	-0.0008544921875	\\
0.702312780219292	-0.00067138671875	\\
0.702357171394327	-0.00042724609375	\\
0.702401562569361	-0.00042724609375	\\
0.702445953744396	3.0517578125e-05	\\
0.70249034491943	0.0001220703125	\\
0.702534736094464	-0.000152587890625	\\
0.702579127269499	-3.0517578125e-05	\\
0.702623518444533	-9.1552734375e-05	\\
0.702667909619568	-0.000213623046875	\\
0.702712300794602	-0.000274658203125	\\
0.702756691969636	-0.000335693359375	\\
0.702801083144671	-0.000518798828125	\\
0.702845474319705	-0.000244140625	\\
0.70288986549474	-0.0003662109375	\\
0.702934256669774	-0.000274658203125	\\
0.702978647844808	-0.000274658203125	\\
0.703023039019843	-0.000701904296875	\\
0.703067430194877	-0.001129150390625	\\
0.703111821369912	-0.00140380859375	\\
0.703156212544946	-0.00164794921875	\\
0.703200603719981	-0.001861572265625	\\
0.703244994895015	-0.00250244140625	\\
0.703289386070049	-0.00323486328125	\\
0.703333777245084	-0.003753662109375	\\
0.703378168420118	-0.004302978515625	\\
0.703422559595152	-0.004791259765625	\\
0.703466950770187	-0.0048828125	\\
0.703511341945221	-0.00494384765625	\\
0.703555733120256	-0.004913330078125	\\
0.70360012429529	-0.00445556640625	\\
0.703644515470324	-0.00390625	\\
0.703688906645359	-0.003936767578125	\\
0.703733297820393	-0.00384521484375	\\
0.703777688995428	-0.00347900390625	\\
0.703822080170462	-0.003387451171875	\\
0.703866471345497	-0.002838134765625	\\
0.703910862520531	-0.002685546875	\\
0.703955253695565	-0.002716064453125	\\
0.7039996448706	-0.00250244140625	\\
0.704044036045634	-0.002197265625	\\
0.704088427220669	-0.002105712890625	\\
0.704132818395703	-0.001495361328125	\\
0.704177209570737	-0.00054931640625	\\
0.704221600745772	-3.0517578125e-05	\\
0.704265991920806	0.000396728515625	\\
0.704310383095841	0.0008544921875	\\
0.704354774270875	0.00164794921875	\\
0.704399165445909	0.002166748046875	\\
0.704443556620944	0.002044677734375	\\
0.704487947795978	0.002655029296875	\\
0.704532338971013	0.002960205078125	\\
0.704576730146047	0.0030517578125	\\
0.704621121321081	0.0032958984375	\\
0.704665512496116	0.0032958984375	\\
0.70470990367115	0.00341796875	\\
0.704754294846185	0.003387451171875	\\
0.704798686021219	0.00372314453125	\\
0.704843077196253	0.003814697265625	\\
0.704887468371288	0.00372314453125	\\
0.704931859546322	0.003570556640625	\\
0.704976250721357	0.003265380859375	\\
0.705020641896391	0.0029296875	\\
0.705065033071425	0.002899169921875	\\
0.70510942424646	0.00225830078125	\\
0.705153815421494	0.001495361328125	\\
0.705198206596529	0.001129150390625	\\
0.705242597771563	0.0008544921875	\\
0.705286988946597	0.000457763671875	\\
0.705331380121632	9.1552734375e-05	\\
0.705375771296666	9.1552734375e-05	\\
0.705420162471701	0	\\
0.705464553646735	-6.103515625e-05	\\
0.705508944821769	-0.000732421875	\\
0.705553335996804	-0.000823974609375	\\
0.705597727171838	-0.001007080078125	\\
0.705642118346873	-0.001556396484375	\\
0.705686509521907	-0.0018310546875	\\
0.705730900696941	-0.001708984375	\\
0.705775291871976	-0.00152587890625	\\
0.70581968304701	-0.001922607421875	\\
0.705864074222045	-0.00146484375	\\
0.705908465397079	-0.00103759765625	\\
0.705952856572113	-0.0010986328125	\\
0.705997247747148	-0.000823974609375	\\
0.706041638922182	-0.00067138671875	\\
0.706086030097217	-0.0006103515625	\\
0.706130421272251	-0.00018310546875	\\
0.706174812447286	-0.00018310546875	\\
0.70621920362232	-0.00018310546875	\\
0.706263594797354	-0.0001220703125	\\
0.706307985972389	0.000274658203125	\\
0.706352377147423	0.0003662109375	\\
0.706396768322457	0.000762939453125	\\
0.706441159497492	0.001190185546875	\\
0.706485550672526	0.000946044921875	\\
0.706529941847561	0.00146484375	\\
0.706574333022595	0.001983642578125	\\
0.70661872419763	0.002471923828125	\\
0.706663115372664	0.00286865234375	\\
0.706707506547698	0.003173828125	\\
0.706751897722733	0.003570556640625	\\
0.706796288897767	0.003692626953125	\\
0.706840680072802	0.0035400390625	\\
0.706885071247836	0.00311279296875	\\
0.70692946242287	0.002685546875	\\
0.706973853597905	0.002685546875	\\
0.707018244772939	0.0025634765625	\\
0.707062635947974	0.00225830078125	\\
0.707107027123008	0.002288818359375	\\
0.707151418298042	0.002166748046875	\\
0.707195809473077	0.0018310546875	\\
0.707240200648111	0.001373291015625	\\
0.707284591823146	0.00146484375	\\
0.70732898299818	0.001708984375	\\
0.707373374173214	0.00128173828125	\\
0.707417765348249	0.001312255859375	\\
0.707462156523283	0.001007080078125	\\
0.707506547698318	0.00079345703125	\\
0.707550938873352	0.00067138671875	\\
0.707595330048386	0.000152587890625	\\
0.707639721223421	-3.0517578125e-05	\\
0.707684112398455	-0.000213623046875	\\
0.70772850357349	-0.00042724609375	\\
0.707772894748524	-0.000732421875	\\
0.707817285923558	-0.000579833984375	\\
0.707861677098593	-0.000518798828125	\\
0.707906068273627	-0.00030517578125	\\
0.707950459448662	3.0517578125e-05	\\
0.707994850623696	0.000244140625	\\
0.70803924179873	0.000152587890625	\\
0.708083632973765	0.0003662109375	\\
0.708128024148799	0.000762939453125	\\
0.708172415323834	0.000885009765625	\\
0.708216806498868	0.001251220703125	\\
0.708261197673902	0.00152587890625	\\
0.708305588848937	0.002166748046875	\\
0.708349980023971	0.002593994140625	\\
0.708394371199006	0.002410888671875	\\
0.70843876237404	0.0028076171875	\\
0.708483153549074	0.003265380859375	\\
0.708527544724109	0.003448486328125	\\
0.708571935899143	0.00360107421875	\\
0.708616327074178	0.003875732421875	\\
0.708660718249212	0.003509521484375	\\
0.708705109424246	0.003387451171875	\\
0.708749500599281	0.00372314453125	\\
0.708793891774315	0.003570556640625	\\
0.70883828294935	0.003448486328125	\\
0.708882674124384	0.003173828125	\\
0.708927065299419	0.002716064453125	\\
0.708971456474453	0.002838134765625	\\
0.709015847649487	0.002685546875	\\
0.709060238824522	0.0020751953125	\\
0.709104629999556	0.001953125	\\
0.70914902117459	0.001861572265625	\\
0.709193412349625	0.001739501953125	\\
0.709237803524659	0.0013427734375	\\
0.709282194699694	0.00103759765625	\\
0.709326585874728	0.000213623046875	\\
0.709370977049762	-0.0003662109375	\\
0.709415368224797	0.000213623046875	\\
0.709459759399831	-6.103515625e-05	\\
0.709504150574866	0	\\
0.7095485417499	-0.0001220703125	\\
0.709592932924935	-0.00054931640625	\\
0.709637324099969	-0.000640869140625	\\
0.709681715275003	-0.001068115234375	\\
0.709726106450038	-0.001129150390625	\\
0.709770497625072	-0.00128173828125	\\
0.709814888800107	-0.0010986328125	\\
0.709859279975141	-0.000823974609375	\\
0.709903671150175	-0.000701904296875	\\
0.70994806232521	-0.00030517578125	\\
0.709992453500244	9.1552734375e-05	\\
0.710036844675279	6.103515625e-05	\\
0.710081235850313	0.000244140625	\\
0.710125627025347	0.000152587890625	\\
0.710170018200382	0.000213623046875	\\
0.710214409375416	0.000518798828125	\\
0.710258800550451	0.0009765625	\\
};
\addplot [color=blue,solid,forget plot]
  table[row sep=crcr]{
0.710258800550451	0.0009765625	\\
0.710303191725485	0.001434326171875	\\
0.710347582900519	0.001708984375	\\
0.710391974075554	0.002288818359375	\\
0.710436365250588	0.002532958984375	\\
0.710480756425623	0.002777099609375	\\
0.710525147600657	0.002838134765625	\\
0.710569538775691	0.002777099609375	\\
0.710613929950726	0.002532958984375	\\
0.71065832112576	0.00262451171875	\\
0.710702712300795	0.002655029296875	\\
0.710747103475829	0.002166748046875	\\
0.710791494650863	0.001739501953125	\\
0.710835885825898	0.001708984375	\\
0.710880277000932	0.001800537109375	\\
0.710924668175967	0.00152587890625	\\
0.710969059351001	0.0013427734375	\\
0.711013450526035	0.0010986328125	\\
0.71105784170107	0.00067138671875	\\
0.711102232876104	0.000457763671875	\\
0.711146624051139	0.000244140625	\\
0.711191015226173	-0.000579833984375	\\
0.711235406401207	-0.001220703125	\\
0.711279797576242	-0.001495361328125	\\
0.711324188751276	-0.001434326171875	\\
0.711368579926311	-0.001739501953125	\\
0.711412971101345	-0.00213623046875	\\
0.711457362276379	-0.00244140625	\\
0.711501753451414	-0.002655029296875	\\
0.711546144626448	-0.002685546875	\\
0.711590535801483	-0.002655029296875	\\
0.711634926976517	-0.00274658203125	\\
0.711679318151552	-0.002349853515625	\\
0.711723709326586	-0.00238037109375	\\
0.71176810050162	-0.002655029296875	\\
0.711812491676655	-0.00244140625	\\
0.711856882851689	-0.00238037109375	\\
0.711901274026724	-0.001861572265625	\\
0.711945665201758	-0.001220703125	\\
0.711990056376792	-0.000579833984375	\\
0.712034447551827	6.103515625e-05	\\
0.712078838726861	0.00048828125	\\
0.712123229901895	0.00079345703125	\\
0.71216762107693	0.00115966796875	\\
0.712212012251964	0.00164794921875	\\
0.712256403426999	0.00164794921875	\\
0.712300794602033	0.00152587890625	\\
0.712345185777068	0.00213623046875	\\
0.712389576952102	0.0023193359375	\\
0.712433968127136	0.002777099609375	\\
0.712478359302171	0.00341796875	\\
0.712522750477205	0.003814697265625	\\
0.71256714165224	0.004302978515625	\\
0.712611532827274	0.004638671875	\\
0.712655924002308	0.005096435546875	\\
0.712700315177343	0.0047607421875	\\
0.712744706352377	0.00433349609375	\\
0.712789097527412	0.004608154296875	\\
0.712833488702446	0.00482177734375	\\
0.71287787987748	0.00439453125	\\
0.712922271052515	0.0042724609375	\\
0.712966662227549	0.004119873046875	\\
0.713011053402584	0.003631591796875	\\
0.713055444577618	0.003326416015625	\\
0.713099835752652	0.003082275390625	\\
0.713144226927687	0.003204345703125	\\
0.713188618102721	0.00341796875	\\
0.713233009277756	0.00274658203125	\\
0.71327740045279	0.002105712890625	\\
0.713321791627824	0.001800537109375	\\
0.713366182802859	0.00103759765625	\\
0.713410573977893	0.000885009765625	\\
0.713454965152928	0.000640869140625	\\
0.713499356327962	0.000457763671875	\\
0.713543747502996	0.000823974609375	\\
0.713588138678031	0.00048828125	\\
0.713632529853065	3.0517578125e-05	\\
0.7136769210281	-6.103515625e-05	\\
0.713721312203134	-6.103515625e-05	\\
0.713765703378168	-0.000518798828125	\\
0.713810094553203	-0.00018310546875	\\
0.713854485728237	-0.0003662109375	\\
0.713898876903272	-0.000457763671875	\\
0.713943268078306	-6.103515625e-05	\\
0.71398765925334	9.1552734375e-05	\\
0.714032050428375	0.000457763671875	\\
0.714076441603409	0.000335693359375	\\
0.714120832778444	0.00067138671875	\\
0.714165223953478	0.001007080078125	\\
0.714209615128512	0.001007080078125	\\
0.714254006303547	0.001220703125	\\
0.714298397478581	0.001312255859375	\\
0.714342788653616	0.001434326171875	\\
0.71438717982865	0.002105712890625	\\
0.714431571003684	0.002716064453125	\\
0.714475962178719	0.002960205078125	\\
0.714520353353753	0.0030517578125	\\
0.714564744528788	0.002685546875	\\
0.714609135703822	0.002777099609375	\\
0.714653526878857	0.0029296875	\\
0.714697918053891	0.00299072265625	\\
0.714742309228925	0.0032958984375	\\
0.71478670040396	0.0032958984375	\\
0.714831091578994	0.003082275390625	\\
0.714875482754028	0.0029296875	\\
0.714919873929063	0.003021240234375	\\
0.714964265104097	0.002960205078125	\\
0.715008656279132	0.0025634765625	\\
0.715053047454166	0.0020751953125	\\
0.715097438629201	0.001739501953125	\\
0.715141829804235	0.0010986328125	\\
0.715186220979269	0.001068115234375	\\
0.715230612154304	0.001007080078125	\\
0.715275003329338	0.0003662109375	\\
0.715319394504373	0.000701904296875	\\
0.715363785679407	0.00079345703125	\\
0.715408176854441	0.000518798828125	\\
0.715452568029476	0.000579833984375	\\
0.71549695920451	0.0003662109375	\\
0.715541350379545	0.000274658203125	\\
0.715585741554579	0.000213623046875	\\
0.715630132729613	-0.000152587890625	\\
0.715674523904648	-0.00048828125	\\
0.715718915079682	-0.000457763671875	\\
0.715763306254717	-0.000640869140625	\\
0.715807697429751	-0.00067138671875	\\
0.715852088604785	-0.00030517578125	\\
0.71589647977982	-0.000244140625	\\
0.715940870954854	0.000152587890625	\\
0.715985262129889	0.000396728515625	\\
0.716029653304923	0.000640869140625	\\
0.716074044479957	0.00128173828125	\\
0.716118435654992	0.0015869140625	\\
0.716162826830026	0.002288818359375	\\
0.716207218005061	0.003021240234375	\\
0.716251609180095	0.0028076171875	\\
0.716296000355129	0.002532958984375	\\
0.716340391530164	0.0028076171875	\\
0.716384782705198	0.003143310546875	\\
0.716429173880233	0.002838134765625	\\
0.716473565055267	0.00299072265625	\\
0.716517956230301	0.003692626953125	\\
0.716562347405336	0.004302978515625	\\
0.71660673858037	0.0040283203125	\\
0.716651129755405	0.003631591796875	\\
0.716695520930439	0.003936767578125	\\
0.716739912105473	0.003387451171875	\\
0.716784303280508	0.00311279296875	\\
0.716828694455542	0.00323486328125	\\
0.716873085630577	0.002685546875	\\
0.716917476805611	0.0020751953125	\\
0.716961867980645	0.00152587890625	\\
0.71700625915568	0.000457763671875	\\
0.717050650330714	0.00042724609375	\\
0.717095041505749	0.00067138671875	\\
0.717139432680783	0.000274658203125	\\
0.717183823855817	6.103515625e-05	\\
0.717228215030852	-0.000244140625	\\
0.717272606205886	-0.000579833984375	\\
0.717316997380921	-0.0010986328125	\\
0.717361388555955	-0.00128173828125	\\
0.71740577973099	-0.00115966796875	\\
0.717450170906024	-0.00128173828125	\\
0.717494562081058	-0.0013427734375	\\
0.717538953256093	-0.001220703125	\\
0.717583344431127	-0.0013427734375	\\
0.717627735606161	-0.001373291015625	\\
0.717672126781196	-0.001312255859375	\\
0.71771651795623	-0.001678466796875	\\
0.717760909131265	-0.001708984375	\\
0.717805300306299	-0.0013427734375	\\
0.717849691481333	-0.001129150390625	\\
0.717894082656368	-0.00103759765625	\\
0.717938473831402	-0.00042724609375	\\
0.717982865006437	-0.000152587890625	\\
0.718027256181471	0.00042724609375	\\
0.718071647356506	0.0013427734375	\\
0.71811603853154	0.001190185546875	\\
0.718160429706574	0.001220703125	\\
0.718204820881609	0.002105712890625	\\
0.718249212056643	0.00238037109375	\\
0.718293603231678	0.00238037109375	\\
0.718337994406712	0.002838134765625	\\
0.718382385581746	0.003021240234375	\\
0.718426776756781	0.00335693359375	\\
0.718471167931815	0.00384521484375	\\
0.71851555910685	0.004241943359375	\\
0.718559950281884	0.003875732421875	\\
0.718604341456918	0.0035400390625	\\
0.718648732631953	0.00360107421875	\\
0.718693123806987	0.00341796875	\\
0.718737514982022	0.00299072265625	\\
0.718781906157056	0.0030517578125	\\
0.71882629733209	0.0035400390625	\\
0.718870688507125	0.003082275390625	\\
0.718915079682159	0.002532958984375	\\
0.718959470857194	0.00238037109375	\\
0.719003862032228	0.002105712890625	\\
0.719048253207262	0.0013427734375	\\
0.719092644382297	0.001068115234375	\\
0.719137035557331	0.000701904296875	\\
0.719181426732366	-0.0003662109375	\\
0.7192258179074	-0.0006103515625	\\
0.719270209082434	-0.000579833984375	\\
0.719314600257469	-0.00103759765625	\\
0.719358991432503	-0.00128173828125	\\
0.719403382607538	-0.00152587890625	\\
0.719447773782572	-0.002716064453125	\\
0.719492164957606	-0.00299072265625	\\
0.719536556132641	-0.003021240234375	\\
0.719580947307675	-0.003082275390625	\\
0.71962533848271	-0.003143310546875	\\
0.719669729657744	-0.00286865234375	\\
0.719714120832778	-0.002685546875	\\
0.719758512007813	-0.002532958984375	\\
0.719802903182847	-0.00262451171875	\\
0.719847294357882	-0.002716064453125	\\
0.719891685532916	-0.002716064453125	\\
0.71993607670795	-0.00311279296875	\\
0.719980467882985	-0.003173828125	\\
0.720024859058019	-0.002410888671875	\\
0.720069250233054	-0.001495361328125	\\
0.720113641408088	-0.00091552734375	\\
0.720158032583123	0.000335693359375	\\
0.720202423758157	0.001495361328125	\\
0.720246814933191	0.00189208984375	\\
0.720291206108226	0.002593994140625	\\
0.72033559728326	0.00244140625	\\
0.720379988458295	0.00225830078125	\\
0.720424379633329	0.002227783203125	\\
0.720468770808363	0.002227783203125	\\
0.720513161983398	0.002471923828125	\\
0.720557553158432	0.0028076171875	\\
0.720601944333466	0.00372314453125	\\
0.720646335508501	0.00457763671875	\\
0.720690726683535	0.004974365234375	\\
0.72073511785857	0.00531005859375	\\
0.720779509033604	0.00555419921875	\\
0.720823900208639	0.0057373046875	\\
0.720868291383673	0.0057373046875	\\
0.720912682558707	0.005035400390625	\\
0.720957073733742	0.00469970703125	\\
0.721001464908776	0.0040283203125	\\
0.721045856083811	0.0032958984375	\\
0.721090247258845	0.00347900390625	\\
0.721134638433879	0.003265380859375	\\
0.721179029608914	0.003143310546875	\\
0.721223420783948	0.003021240234375	\\
0.721267811958983	0.002777099609375	\\
0.721312203134017	0.002777099609375	\\
0.721356594309051	0.002349853515625	\\
0.721400985484086	0.0018310546875	\\
0.72144537665912	0.00152587890625	\\
0.721489767834155	0.001129150390625	\\
0.721534159009189	0.000244140625	\\
0.721578550184223	-3.0517578125e-05	\\
0.721622941359258	-0.000213623046875	\\
0.721667332534292	-0.00042724609375	\\
0.721711723709327	-0.000640869140625	\\
0.721756114884361	-0.00067138671875	\\
0.721800506059395	-0.000518798828125	\\
0.72184489723443	-0.000213623046875	\\
0.721889288409464	0.000213623046875	\\
0.721933679584499	0.00030517578125	\\
0.721978070759533	0.00048828125	\\
0.722022461934567	0.00042724609375	\\
0.722066853109602	0.00018310546875	\\
0.722111244284636	3.0517578125e-05	\\
0.722155635459671	0.000457763671875	\\
0.722200026634705	0.000213623046875	\\
0.722244417809739	-9.1552734375e-05	\\
0.722288808984774	0.000274658203125	\\
0.722333200159808	0.00042724609375	\\
0.722377591334843	0.000946044921875	\\
0.722421982509877	0.001861572265625	\\
0.722466373684911	0.002166748046875	\\
0.722510764859946	0.002410888671875	\\
0.72255515603498	0.002716064453125	\\
0.722599547210015	0.002838134765625	\\
0.722643938385049	0.00299072265625	\\
0.722688329560083	0.0028076171875	\\
0.722732720735118	0.002838134765625	\\
0.722777111910152	0.002349853515625	\\
0.722821503085187	0.002227783203125	\\
0.722865894260221	0.0025634765625	\\
0.722910285435255	0.002777099609375	\\
0.72295467661029	0.00274658203125	\\
0.722999067785324	0.00286865234375	\\
0.723043458960359	0.003265380859375	\\
0.723087850135393	0.00335693359375	\\
0.723132241310428	0.003265380859375	\\
0.723176632485462	0.0030517578125	\\
0.723221023660496	0.0029296875	\\
0.723265414835531	0.0025634765625	\\
0.723309806010565	0.00225830078125	\\
0.723354197185599	0.001800537109375	\\
0.723398588360634	0.00103759765625	\\
0.723442979535668	0.001251220703125	\\
0.723487370710703	0.00128173828125	\\
0.723531761885737	0.00146484375	\\
0.723576153060772	0.001312255859375	\\
0.723620544235806	0.001190185546875	\\
0.72366493541084	0.000823974609375	\\
0.723709326585875	0.000579833984375	\\
0.723753717760909	0.00067138671875	\\
0.723798108935944	0.00042724609375	\\
0.723842500110978	0.000396728515625	\\
0.723886891286012	0	\\
0.723931282461047	-0.00030517578125	\\
0.723975673636081	-0.00018310546875	\\
0.724020064811116	0	\\
0.72406445598615	0.00042724609375	\\
0.724108847161184	0.0008544921875	\\
0.724153238336219	0.001556396484375	\\
0.724197629511253	0.001708984375	\\
0.724242020686288	0.001953125	\\
0.724286411861322	0.003021240234375	\\
0.724330803036356	0.003021240234375	\\
0.724375194211391	0.00323486328125	\\
0.724419585386425	0.0029296875	\\
0.72446397656146	0.002349853515625	\\
0.724508367736494	0.002655029296875	\\
0.724552758911528	0.0029296875	\\
0.724597150086563	0.00323486328125	\\
0.724641541261597	0.00347900390625	\\
0.724685932436632	0.003814697265625	\\
0.724730323611666	0.0032958984375	\\
0.7247747147867	0.003814697265625	\\
0.724819105961735	0.00396728515625	\\
0.724863497136769	0.003204345703125	\\
0.724907888311804	0.00299072265625	\\
0.724952279486838	0.001953125	\\
0.724996670661872	0.001800537109375	\\
0.725041061836907	0.001739501953125	\\
0.725085453011941	0.00115966796875	\\
0.725129844186976	0.000823974609375	\\
0.72517423536201	0.000732421875	\\
0.725218626537044	0.000762939453125	\\
0.725263017712079	0.000518798828125	\\
0.725307408887113	0.000274658203125	\\
0.725351800062148	0.00018310546875	\\
0.725396191237182	-6.103515625e-05	\\
0.725440582412216	-0.000274658203125	\\
0.725484973587251	-0.000152587890625	\\
0.725529364762285	-0.000244140625	\\
0.72557375593732	-0.00054931640625	\\
0.725618147112354	-0.000396728515625	\\
0.725662538287388	-0.00030517578125	\\
0.725706929462423	-0.000213623046875	\\
0.725751320637457	3.0517578125e-05	\\
0.725795711812492	0.000518798828125	\\
0.725840102987526	0.000457763671875	\\
0.725884494162561	0.00048828125	\\
0.725928885337595	0.001190185546875	\\
0.725973276512629	0.001190185546875	\\
0.726017667687664	0.001708984375	\\
0.726062058862698	0.001739501953125	\\
0.726106450037733	0.001617431640625	\\
0.726150841212767	0.0020751953125	\\
0.726195232387801	0.002288818359375	\\
0.726239623562836	0.00286865234375	\\
0.72628401473787	0.00299072265625	\\
0.726328405912904	0.00341796875	\\
0.726372797087939	0.003875732421875	\\
0.726417188262973	0.0040283203125	\\
0.726461579438008	0.004486083984375	\\
0.726505970613042	0.004608154296875	\\
0.726550361788077	0.0047607421875	\\
0.726594752963111	0.004852294921875	\\
0.726639144138145	0.004669189453125	\\
0.72668353531318	0.003997802734375	\\
0.726727926488214	0.003631591796875	\\
0.726772317663249	0.003143310546875	\\
0.726816708838283	0.002685546875	\\
0.726861100013317	0.002410888671875	\\
0.726905491188352	0.002593994140625	\\
0.726949882363386	0.002838134765625	\\
0.726994273538421	0.00262451171875	\\
0.727038664713455	0.00225830078125	\\
0.727083055888489	0.0015869140625	\\
0.727127447063524	0.001129150390625	\\
0.727171838238558	0.000335693359375	\\
0.727216229413593	-0.00018310546875	\\
0.727260620588627	-0.00067138671875	\\
0.727305011763661	-0.00140380859375	\\
0.727349402938696	-0.001922607421875	\\
0.72739379411373	-0.002166748046875	\\
0.727438185288765	-0.0025634765625	\\
0.727482576463799	-0.002716064453125	\\
0.727526967638833	-0.0029296875	\\
0.727571358813868	-0.003509521484375	\\
0.727615749988902	-0.00341796875	\\
0.727660141163937	-0.003509521484375	\\
0.727704532338971	-0.003814697265625	\\
0.727748923514005	-0.003875732421875	\\
0.72779331468904	-0.003662109375	\\
0.727837705864074	-0.00360107421875	\\
0.727882097039109	-0.00323486328125	\\
0.727926488214143	-0.00335693359375	\\
0.727970879389177	-0.003387451171875	\\
0.728015270564212	-0.00286865234375	\\
0.728059661739246	-0.0023193359375	\\
0.728104052914281	-0.00201416015625	\\
0.728148444089315	-0.001617431640625	\\
0.728192835264349	-0.001251220703125	\\
0.728237226439384	-0.0009765625	\\
0.728281617614418	-0.000213623046875	\\
0.728326008789453	0.00048828125	\\
0.728370399964487	0.001129150390625	\\
0.728414791139521	0.00164794921875	\\
0.728459182314556	0.002471923828125	\\
0.72850357348959	0.003204345703125	\\
0.728547964664625	0.0029296875	\\
0.728592355839659	0.002960205078125	\\
0.728636747014694	0.00311279296875	\\
0.728681138189728	0.00335693359375	\\
0.728725529364762	0.00384521484375	\\
0.728769920539797	0.00408935546875	\\
0.728814311714831	0.004180908203125	\\
0.728858702889866	0.004150390625	\\
0.7289030940649	0.004180908203125	\\
0.728947485239934	0.003936767578125	\\
0.728991876414969	0.004119873046875	\\
0.729036267590003	0.0042724609375	\\
0.729080658765037	0.003997802734375	\\
0.729125049940072	0.003631591796875	\\
0.729169441115106	0.003173828125	\\
0.729213832290141	0.00299072265625	\\
0.729258223465175	0.0025634765625	\\
0.72930261464021	0.001739501953125	\\
0.729347005815244	0.00140380859375	\\
0.729391396990278	0.0013427734375	\\
0.729435788165313	0.00054931640625	\\
0.729480179340347	0.000518798828125	\\
0.729524570515382	0.000701904296875	\\
0.729568961690416	0.000152587890625	\\
0.72961335286545	0.000274658203125	\\
0.729657744040485	0.00030517578125	\\
0.729702135215519	-0.00042724609375	\\
0.729746526390554	-0.000457763671875	\\
0.729790917565588	-0.000579833984375	\\
0.729835308740622	-0.00115966796875	\\
0.729879699915657	-0.001007080078125	\\
0.729924091090691	-0.00115966796875	\\
0.729968482265726	-0.0013427734375	\\
0.73001287344076	-0.001068115234375	\\
0.730057264615794	-0.000823974609375	\\
0.730101655790829	-0.000885009765625	\\
0.730146046965863	-0.000335693359375	\\
0.730190438140898	0.000274658203125	\\
0.730234829315932	0.00042724609375	\\
0.730279220490966	0.000823974609375	\\
0.730323611666001	0.00152587890625	\\
0.730368002841035	0.00201416015625	\\
0.73041239401607	0.00201416015625	\\
0.730456785191104	0.002227783203125	\\
0.730501176366138	0.002197265625	\\
0.730545567541173	0.002166748046875	\\
0.730589958716207	0.002532958984375	\\
0.730634349891242	0.002960205078125	\\
0.730678741066276	0.003875732421875	\\
0.73072313224131	0.00390625	\\
0.730767523416345	0.00390625	\\
0.730811914591379	0.00408935546875	\\
0.730856305766414	0.004150390625	\\
0.730900696941448	0.004150390625	\\
0.730945088116482	0.0040283203125	\\
0.730989479291517	0.003692626953125	\\
0.731033870466551	0.0035400390625	\\
0.731078261641586	0.003326416015625	\\
0.73112265281662	0.0032958984375	\\
0.731167043991654	0.002899169921875	\\
0.731211435166689	0.002593994140625	\\
0.731255826341723	0.00286865234375	\\
0.731300217516758	0.002899169921875	\\
0.731344608691792	0.002899169921875	\\
0.731388999866826	0.00262451171875	\\
0.731433391041861	0.00238037109375	\\
0.731477782216895	0.0025634765625	\\
0.73152217339193	0.002197265625	\\
0.731566564566964	0.001251220703125	\\
0.731610955741999	0.0013427734375	\\
0.731655346917033	0.001129150390625	\\
0.731699738092067	0.000457763671875	\\
0.731744129267102	0.00018310546875	\\
0.731788520442136	-6.103515625e-05	\\
0.73183291161717	-0.0001220703125	\\
0.731877302792205	-0.00054931640625	\\
0.731921693967239	-0.00030517578125	\\
0.731966085142274	0	\\
0.732010476317308	-6.103515625e-05	\\
0.732054867492343	-9.1552734375e-05	\\
0.732099258667377	9.1552734375e-05	\\
0.732143649842411	0.00018310546875	\\
0.732188041017446	0.00018310546875	\\
0.73223243219248	0.0006103515625	\\
0.732276823367515	0.0008544921875	\\
0.732321214542549	0.000701904296875	\\
0.732365605717583	0.001129150390625	\\
0.732409996892618	0.0018310546875	\\
0.732454388067652	0.0018310546875	\\
0.732498779242687	0.00250244140625	\\
0.732543170417721	0.0029296875	\\
0.732587561592755	0.00286865234375	\\
0.73263195276779	0.00274658203125	\\
0.732676343942824	0.002410888671875	\\
0.732720735117859	0.00244140625	\\
0.732765126292893	0.0023193359375	\\
0.732809517467927	0.00213623046875	\\
0.732853908642962	0.001983642578125	\\
0.732898299817996	0.00213623046875	\\
0.732942690993031	0.002105712890625	\\
0.732987082168065	0.001953125	\\
0.733031473343099	0.002166748046875	\\
0.733075864518134	0.002227783203125	\\
0.733120255693168	0.001739501953125	\\
0.733164646868203	0.001220703125	\\
0.733209038043237	0.001007080078125	\\
0.733253429218271	0.00030517578125	\\
0.733297820393306	-9.1552734375e-05	\\
0.73334221156834	-0.000579833984375	\\
0.733386602743375	-0.00054931640625	\\
0.733430993918409	-0.000640869140625	\\
0.733475385093443	-0.0009765625	\\
0.733519776268478	-0.0010986328125	\\
0.733564167443512	-0.001007080078125	\\
0.733608558618547	-0.000518798828125	\\
0.733652949793581	-0.0009765625	\\
0.733697340968615	-0.000701904296875	\\
0.73374173214365	-0.0006103515625	\\
0.733786123318684	-0.001312255859375	\\
0.733830514493719	-0.001434326171875	\\
0.733874905668753	-0.00146484375	\\
0.733919296843787	-0.00146484375	\\
0.733963688018822	-0.001495361328125	\\
0.734008079193856	-0.0013427734375	\\
0.734052470368891	-0.0008544921875	\\
0.734096861543925	-0.00067138671875	\\
0.734141252718959	-0.00054931640625	\\
0.734185643893994	-0.000518798828125	\\
0.734230035069028	-0.000244140625	\\
0.734274426244063	0.000579833984375	\\
0.734318817419097	0.001007080078125	\\
0.734363208594132	0.000518798828125	\\
0.734407599769166	0.00048828125	\\
0.7344519909442	0.000823974609375	\\
0.734496382119235	0.001129150390625	\\
0.734540773294269	0.00152587890625	\\
0.734585164469304	0.0018310546875	\\
0.734629555644338	0.001861572265625	\\
0.734673946819372	0.0023193359375	\\
0.734718337994407	0.0029296875	\\
0.734762729169441	0.002899169921875	\\
0.734807120344475	0.002838134765625	\\
0.73485151151951	0.00311279296875	\\
0.734895902694544	0.0028076171875	\\
0.734940293869579	0.00262451171875	\\
0.734984685044613	0.002044677734375	\\
0.735029076219648	0.001373291015625	\\
0.735073467394682	0.0008544921875	\\
0.735117858569716	0.00054931640625	\\
0.735162249744751	-0.000152587890625	\\
0.735206640919785	-0.000213623046875	\\
0.73525103209482	0	\\
0.735295423269854	-0.000640869140625	\\
0.735339814444888	-0.001251220703125	\\
0.735384205619923	-0.001312255859375	\\
0.735428596794957	-0.001190185546875	\\
0.735472987969992	-0.0018310546875	\\
0.735517379145026	-0.00225830078125	\\
0.73556177032006	-0.0023193359375	\\
0.735606161495095	-0.003173828125	\\
0.735650552670129	-0.0037841796875	\\
0.735694943845164	-0.003814697265625	\\
0.735739335020198	-0.004058837890625	\\
0.735783726195232	-0.00408935546875	\\
0.735828117370267	-0.004364013671875	\\
0.735872508545301	-0.004425048828125	\\
0.735916899720336	-0.00433349609375	\\
0.73596129089537	-0.003875732421875	\\
0.736005682070404	-0.0035400390625	\\
0.736050073245439	-0.003814697265625	\\
0.736094464420473	-0.00372314453125	\\
0.736138855595508	-0.003387451171875	\\
0.736183246770542	-0.0029296875	\\
0.736227637945576	-0.002105712890625	\\
0.736272029120611	-0.001678466796875	\\
0.736316420295645	-0.00213623046875	\\
0.73636081147068	-0.001800537109375	\\
0.736405202645714	-0.0013427734375	\\
0.736449593820748	-0.001617431640625	\\
0.736493984995783	-0.0013427734375	\\
0.736538376170817	-0.0006103515625	\\
0.736582767345852	-0.000274658203125	\\
0.736627158520886	0.000335693359375	\\
0.73667154969592	0.0010986328125	\\
0.736715940870955	0.0018310546875	\\
0.736760332045989	0.00244140625	\\
0.736804723221024	0.002716064453125	\\
0.736849114396058	0.002655029296875	\\
0.736893505571092	0.002685546875	\\
0.736937896746127	0.0023193359375	\\
0.736982287921161	0.002227783203125	\\
0.737026679096196	0.00225830078125	\\
0.73707107027123	0.00213623046875	\\
0.737115461446265	0.002349853515625	\\
0.737159852621299	0.001953125	\\
0.737204243796333	0.00201416015625	\\
0.737248634971368	0.002471923828125	\\
0.737293026146402	0.00213623046875	\\
0.737337417321437	0.001678466796875	\\
0.737381808496471	0.00152587890625	\\
0.737426199671505	0.00115966796875	\\
0.73747059084654	0.000640869140625	\\
0.737514982021574	3.0517578125e-05	\\
0.737559373196608	-0.0003662109375	\\
0.737603764371643	-0.000946044921875	\\
0.737648155546677	-0.001251220703125	\\
0.737692546721712	-0.001373291015625	\\
0.737736937896746	-0.001129150390625	\\
0.737781329071781	-0.0013427734375	\\
0.737825720246815	-0.001739501953125	\\
0.737870111421849	-0.00128173828125	\\
0.737914502596884	-0.001220703125	\\
0.737958893771918	-0.000946044921875	\\
0.738003284946953	-0.00128173828125	\\
0.738047676121987	-0.00177001953125	\\
0.738092067297021	-0.00146484375	\\
0.738136458472056	-0.001556396484375	\\
0.73818084964709	-0.00189208984375	\\
0.738225240822125	-0.001861572265625	\\
0.738269631997159	-0.00164794921875	\\
0.738314023172193	-0.001312255859375	\\
0.738358414347228	-0.001129150390625	\\
0.738402805522262	-0.001129150390625	\\
0.738447196697297	-0.0006103515625	\\
0.738491587872331	0.000244140625	\\
0.738535979047365	0.000732421875	\\
0.7385803702224	0.00091552734375	\\
0.738624761397434	0.00091552734375	\\
0.738669152572469	0.000823974609375	\\
0.738713543747503	0.000885009765625	\\
0.738757934922537	0.00128173828125	\\
0.738802326097572	0.001373291015625	\\
0.738846717272606	0.001251220703125	\\
0.738891108447641	0.00152587890625	\\
0.738935499622675	0.001861572265625	\\
0.738979890797709	0.00262451171875	\\
0.739024281972744	0.0030517578125	\\
0.739068673147778	0.002471923828125	\\
0.739113064322813	0.002655029296875	\\
0.739157455497847	0.002777099609375	\\
0.739201846672881	0.002288818359375	\\
0.739246237847916	0.00189208984375	\\
0.73929062902295	0.001617431640625	\\
0.739335020197985	0.00146484375	\\
0.739379411373019	0.001129150390625	\\
0.739423802548053	0.001007080078125	\\
0.739468193723088	0.00103759765625	\\
0.739512584898122	0.001129150390625	\\
0.739556976073157	0.0010986328125	\\
0.739601367248191	0.0010986328125	\\
0.739645758423225	0.001220703125	\\
0.73969014959826	0.00103759765625	\\
0.739734540773294	0.00048828125	\\
0.739778931948329	-6.103515625e-05	\\
0.739823323123363	-0.000244140625	\\
0.739867714298397	-0.00042724609375	\\
0.739912105473432	-0.000762939453125	\\
0.739956496648466	-0.0010986328125	\\
0.740000887823501	-0.001708984375	\\
0.740045278998535	-0.001800537109375	\\
0.74008967017357	-0.001434326171875	\\
0.740134061348604	-0.001220703125	\\
0.740178452523638	-0.001220703125	\\
0.740222843698673	-0.00115966796875	\\
0.740267234873707	-0.001129150390625	\\
0.740311626048741	-0.001312255859375	\\
0.740356017223776	-0.00128173828125	\\
0.74040040839881	-0.001556396484375	\\
0.740444799573845	-0.001861572265625	\\
0.740489190748879	-0.001617431640625	\\
0.740533581923914	-0.00146484375	\\
0.740577973098948	-0.0010986328125	\\
0.740622364273982	-0.000457763671875	\\
0.740666755449017	9.1552734375e-05	\\
0.740711146624051	0.00054931640625	\\
0.740755537799086	0.0008544921875	\\
0.74079992897412	0.0010986328125	\\
0.740844320149154	0.0010986328125	\\
0.740888711324189	0.000579833984375	\\
0.740933102499223	0.00018310546875	\\
0.740977493674258	0.000274658203125	\\
0.741021884849292	0.000274658203125	\\
0.741066276024326	0.000274658203125	\\
0.741110667199361	0.000823974609375	\\
0.741155058374395	0.0008544921875	\\
0.74119944954943	0.00079345703125	\\
0.741243840724464	0.00146484375	\\
0.741288231899498	0.001708984375	\\
0.741332623074533	0.0008544921875	\\
0.741377014249567	0.000823974609375	\\
0.741421405424602	0.000396728515625	\\
0.741465796599636	3.0517578125e-05	\\
0.74151018777467	0.0001220703125	\\
0.741554578949705	0	\\
0.741598970124739	-0.000701904296875	\\
0.741643361299774	-0.001129150390625	\\
0.741687752474808	-0.00103759765625	\\
0.741732143649842	-0.001068115234375	\\
0.741776534824877	-0.001129150390625	\\
0.741820925999911	-0.001373291015625	\\
0.741865317174946	-0.00140380859375	\\
0.74190970834998	-0.001220703125	\\
0.741954099525014	-0.00115966796875	\\
0.741998490700049	-0.001373291015625	\\
0.742042881875083	-0.00177001953125	\\
0.742087273050118	-0.002166748046875	\\
0.742131664225152	-0.002288818359375	\\
0.742176055400187	-0.001983642578125	\\
0.742220446575221	-0.001678466796875	\\
0.742264837750255	-0.001556396484375	\\
0.74230922892529	-0.001129150390625	\\
0.742353620100324	-0.0009765625	\\
0.742398011275358	-0.000762939453125	\\
0.742442402450393	-0.000335693359375	\\
0.742486793625427	-0.000457763671875	\\
0.742531184800462	-0.000396728515625	\\
0.742575575975496	-3.0517578125e-05	\\
0.74261996715053	-0.000213623046875	\\
0.742664358325565	-0.000335693359375	\\
0.742708749500599	6.103515625e-05	\\
0.742753140675634	0.000244140625	\\
0.742797531850668	0.000762939453125	\\
0.742841923025703	0.000518798828125	\\
0.742886314200737	0.00067138671875	\\
0.742930705375771	0.000946044921875	\\
0.742975096550806	0.001007080078125	\\
0.74301948772584	0.001129150390625	\\
0.743063878900875	0.0010986328125	\\
0.743108270075909	0.000701904296875	\\
0.743152661250943	-3.0517578125e-05	\\
0.743197052425978	-0.000335693359375	\\
0.743241443601012	-0.00018310546875	\\
0.743285834776046	-0.000579833984375	\\
0.743330225951081	-0.00128173828125	\\
0.743374617126115	-0.001190185546875	\\
0.74341900830115	-0.001373291015625	\\
0.743463399476184	-0.00177001953125	\\
0.743507790651219	-0.00152587890625	\\
0.743552181826253	-0.001708984375	\\
0.743596573001287	-0.002166748046875	\\
0.743640964176322	-0.002166748046875	\\
0.743685355351356	-0.00274658203125	\\
0.743729746526391	-0.002838134765625	\\
0.743774137701425	-0.00323486328125	\\
0.743818528876459	-0.00360107421875	\\
0.743862920051494	-0.0037841796875	\\
0.743907311226528	-0.003753662109375	\\
0.743951702401563	-0.00360107421875	\\
0.743996093576597	-0.003997802734375	\\
0.744040484751631	-0.003997802734375	\\
0.744084875926666	-0.003692626953125	\\
0.7441292671017	-0.003753662109375	\\
0.744173658276735	-0.00347900390625	\\
0.744218049451769	-0.00299072265625	\\
0.744262440626803	-0.003021240234375	\\
0.744306831801838	-0.00311279296875	\\
0.744351222976872	-0.0028076171875	\\
0.744395614151907	-0.002532958984375	\\
0.744440005326941	-0.002471923828125	\\
0.744484396501975	-0.001983642578125	\\
0.74452878767701	-0.00115966796875	\\
0.744573178852044	-0.00091552734375	\\
0.744617570027079	-0.00042724609375	\\
0.744661961202113	-0.00018310546875	\\
0.744706352377147	-0.00030517578125	\\
0.744750743552182	0.0003662109375	\\
0.744795134727216	0.000518798828125	\\
0.744839525902251	0.00030517578125	\\
0.744883917077285	0.0006103515625	\\
0.744928308252319	0.00146484375	\\
0.744972699427354	0.0020751953125	\\
0.745017090602388	0.00213623046875	\\
0.745061481777423	0.0023193359375	\\
0.745105872952457	0.002716064453125	\\
0.745150264127491	0.002899169921875	\\
0.745194655302526	0.002838134765625	\\
0.74523904647756	0.0025634765625	\\
0.745283437652595	0.002288818359375	\\
0.745327828827629	0.001983642578125	\\
0.745372220002663	0.00189208984375	\\
0.745416611177698	0.001373291015625	\\
0.745461002352732	0.00115966796875	\\
0.745505393527767	0.001373291015625	\\
0.745549784702801	0.001678466796875	\\
0.745594175877836	0.0015869140625	\\
0.74563856705287	0.001373291015625	\\
0.745682958227904	0.00152587890625	\\
0.745727349402939	0.00146484375	\\
0.745771740577973	0.00128173828125	\\
0.745816131753008	0.0010986328125	\\
0.745860522928042	0.000732421875	\\
0.745904914103076	0.000152587890625	\\
0.745949305278111	-0.0003662109375	\\
0.745993696453145	-0.00048828125	\\
0.746038087628179	-0.000274658203125	\\
0.746082478803214	-0.00054931640625	\\
0.746126869978248	-0.000396728515625	\\
0.746171261153283	9.1552734375e-05	\\
0.746215652328317	-0.000213623046875	\\
0.746260043503352	0	\\
0.746304434678386	0.0006103515625	\\
0.74634882585342	0.000213623046875	\\
0.746393217028455	9.1552734375e-05	\\
0.746437608203489	6.103515625e-05	\\
0.746481999378524	0.000152587890625	\\
0.746526390553558	3.0517578125e-05	\\
0.746570781728592	3.0517578125e-05	\\
0.746615172903627	0.00048828125	\\
0.746659564078661	0.00079345703125	\\
0.746703955253696	0.001190185546875	\\
0.74674834642873	0.00164794921875	\\
0.746792737603764	0.002227783203125	\\
0.746837128778799	0.0028076171875	\\
0.746881519953833	0.002227783203125	\\
0.746925911128868	0.0020751953125	\\
0.746970302303902	0.002105712890625	\\
0.747014693478936	0.00128173828125	\\
0.747059084653971	0.000885009765625	\\
0.747103475829005	0.00103759765625	\\
0.74714786700404	0.00103759765625	\\
0.747192258179074	0.00140380859375	\\
0.747236649354108	0.001617431640625	\\
0.747281040529143	0.001708984375	\\
0.747325431704177	0.002197265625	\\
0.747369822879212	0.002044677734375	\\
0.747414214054246	0.00213623046875	\\
0.74745860522928	0.00164794921875	\\
0.747502996404315	0.001190185546875	\\
0.747547387579349	0.001068115234375	\\
0.747591778754384	0.000274658203125	\\
0.747636169929418	3.0517578125e-05	\\
0.747680561104452	-0.00030517578125	\\
0.747724952279487	-0.000579833984375	\\
0.747769343454521	-6.103515625e-05	\\
0.747813734629556	-3.0517578125e-05	\\
0.74785812580459	0.00018310546875	\\
0.747902516979624	0.0003662109375	\\
0.747946908154659	0.00030517578125	\\
0.747991299329693	0.00067138671875	\\
0.748035690504728	0.000823974609375	\\
0.748080081679762	0.000152587890625	\\
0.748124472854796	-0.0006103515625	\\
0.748168864029831	-0.001007080078125	\\
0.748213255204865	-0.00115966796875	\\
0.7482576463799	-0.00115966796875	\\
0.748302037554934	-0.001220703125	\\
0.748346428729968	-0.001312255859375	\\
0.748390819905003	-0.0010986328125	\\
0.748435211080037	-0.00048828125	\\
0.748479602255072	3.0517578125e-05	\\
0.748523993430106	0.00030517578125	\\
0.748568384605141	0.00018310546875	\\
0.748612775780175	6.103515625e-05	\\
0.748657166955209	-9.1552734375e-05	\\
0.748701558130244	0	\\
0.748745949305278	-6.103515625e-05	\\
0.748790340480313	-0.000213623046875	\\
0.748834731655347	-3.0517578125e-05	\\
0.748879122830381	0.000640869140625	\\
0.748923514005416	0.000946044921875	\\
0.74896790518045	0.00079345703125	\\
0.749012296355485	0.00152587890625	\\
0.749056687530519	0.002532958984375	\\
0.749101078705553	0.002593994140625	\\
0.749145469880588	0.002838134765625	\\
0.749189861055622	0.00250244140625	\\
0.749234252230657	0.00238037109375	\\
0.749278643405691	0.00250244140625	\\
0.749323034580725	0.001678466796875	\\
0.74936742575576	0.001678466796875	\\
0.749411816930794	0.00152587890625	\\
0.749456208105829	0.001434326171875	\\
0.749500599280863	0.00177001953125	\\
0.749544990455897	0.002044677734375	\\
0.749589381630932	0.0023193359375	\\
0.749633772805966	0.002410888671875	\\
0.749678163981001	0.002044677734375	\\
0.749722555156035	0.001434326171875	\\
0.749766946331069	0.0013427734375	\\
0.749811337506104	0.000885009765625	\\
0.749855728681138	0.000396728515625	\\
0.749900119856173	0.000732421875	\\
0.749944511031207	0.000335693359375	\\
0.749988902206241	9.1552734375e-05	\\
0.750033293381276	0.000274658203125	\\
0.75007768455631	0.000213623046875	\\
0.750122075731345	0.0001220703125	\\
0.750166466906379	-0.000335693359375	\\
0.750210858081413	-0.000213623046875	\\
0.750255249256448	3.0517578125e-05	\\
0.750299640431482	0.0001220703125	\\
0.750344031606517	-0.00030517578125	\\
0.750388422781551	-0.000762939453125	\\
0.750432813956585	-0.000701904296875	\\
0.75047720513162	-0.000885009765625	\\
0.750521596306654	-0.0009765625	\\
0.750565987481689	-0.000885009765625	\\
0.750610378656723	-0.000732421875	\\
0.750654769831758	-0.00048828125	\\
0.750699161006792	-0.000213623046875	\\
0.750743552181826	0.00042724609375	\\
0.750787943356861	0.000701904296875	\\
0.750832334531895	0.000823974609375	\\
0.750876725706929	0.000579833984375	\\
0.750921116881964	0.000457763671875	\\
0.750965508056998	0.00048828125	\\
0.751009899232033	0.000396728515625	\\
0.751054290407067	0.000518798828125	\\
0.751098681582101	0.000885009765625	\\
0.751143072757136	0.00128173828125	\\
0.75118746393217	0.001678466796875	\\
0.751231855107205	0.0020751953125	\\
0.751276246282239	0.002227783203125	\\
0.751320637457274	0.00225830078125	\\
0.751365028632308	0.00250244140625	\\
0.751409419807342	0.0015869140625	\\
0.751453810982377	0.001495361328125	\\
0.751498202157411	0.00164794921875	\\
0.751542593332446	0.001007080078125	\\
0.75158698450748	0.000701904296875	\\
0.751631375682514	0.000823974609375	\\
0.751675766857549	0.000885009765625	\\
0.751720158032583	0.000701904296875	\\
0.751764549207617	0.000518798828125	\\
0.751808940382652	0.000396728515625	\\
0.751853331557686	6.103515625e-05	\\
0.751897722732721	-0.00018310546875	\\
0.751942113907755	-0.0006103515625	\\
0.75198650508279	-0.000823974609375	\\
0.752030896257824	-0.001007080078125	\\
0.752075287432858	-0.00128173828125	\\
0.752119678607893	-0.001922607421875	\\
0.752164069782927	-0.00213623046875	\\
0.752208460957962	-0.002044677734375	\\
0.752252852132996	-0.002960205078125	\\
0.75229724330803	-0.0030517578125	\\
0.752341634483065	-0.002471923828125	\\
0.752386025658099	-0.002899169921875	\\
0.752430416833134	-0.002410888671875	\\
0.752474808008168	-0.00201416015625	\\
0.752519199183202	-0.002593994140625	\\
0.752563590358237	-0.002471923828125	\\
0.752607981533271	-0.00189208984375	\\
0.752652372708306	-0.001983642578125	\\
0.75269676388334	-0.0025634765625	\\
0.752741155058374	-0.00262451171875	\\
0.752785546233409	-0.0023193359375	\\
0.752829937408443	-0.001708984375	\\
0.752874328583478	-0.001251220703125	\\
0.752918719758512	-0.001220703125	\\
0.752963110933546	-0.000885009765625	\\
0.753007502108581	-0.000579833984375	\\
0.753051893283615	-0.000457763671875	\\
0.75309628445865	-0.000762939453125	\\
0.753140675633684	-0.00079345703125	\\
0.753185066808718	-0.00042724609375	\\
0.753229457983753	-0.0006103515625	\\
0.753273849158787	-0.000244140625	\\
0.753318240333822	0.0003662109375	\\
0.753362631508856	0.0006103515625	\\
0.75340702268389	0.001007080078125	\\
0.753451413858925	0.00140380859375	\\
0.753495805033959	0.001373291015625	\\
0.753540196208994	0.00128173828125	\\
0.753584587384028	0.000640869140625	\\
0.753628978559062	0.000152587890625	\\
0.753673369734097	-9.1552734375e-05	\\
0.753717760909131	-0.00054931640625	\\
0.753762152084166	-0.00048828125	\\
0.7538065432592	-0.000640869140625	\\
0.753850934434234	-0.000823974609375	\\
0.753895325609269	-0.0006103515625	\\
0.753939716784303	-0.000762939453125	\\
0.753984107959338	-0.000640869140625	\\
0.754028499134372	-0.00030517578125	\\
0.754072890309407	-0.000152587890625	\\
0.754117281484441	-0.000335693359375	\\
0.754161672659475	-0.000640869140625	\\
0.75420606383451	-0.000762939453125	\\
0.754250455009544	-0.000762939453125	\\
0.754294846184579	-0.0006103515625	\\
0.754339237359613	-0.0003662109375	\\
0.754383628534647	-0.00048828125	\\
0.754428019709682	-0.000244140625	\\
0.754472410884716	0.000213623046875	\\
0.75451680205975	0.001007080078125	\\
0.754561193234785	0.001708984375	\\
0.754605584409819	0.001617431640625	\\
0.754649975584854	0.00201416015625	\\
0.754694366759888	0.002044677734375	\\
0.754738757934923	0.001800537109375	\\
0.754783149109957	0.001495361328125	\\
0.754827540284991	0.001312255859375	\\
0.754871931460026	0.00177001953125	\\
0.75491632263506	0.00201416015625	\\
0.754960713810095	0.001922607421875	\\
0.755005104985129	0.002166748046875	\\
0.755049496160163	0.00244140625	\\
0.755093887335198	0.003143310546875	\\
0.755138278510232	0.003143310546875	\\
0.755182669685267	0.0028076171875	\\
0.755227060860301	0.00244140625	\\
0.755271452035335	0.001800537109375	\\
0.75531584321037	0.00164794921875	\\
0.755360234385404	0.001556396484375	\\
0.755404625560439	0.0009765625	\\
0.755449016735473	0.000335693359375	\\
0.755493407910507	-3.0517578125e-05	\\
0.755537799085542	9.1552734375e-05	\\
0.755582190260576	0.00054931640625	\\
0.755626581435611	0.000946044921875	\\
0.755670972610645	0.000762939453125	\\
0.755715363785679	0.00067138671875	\\
0.755759754960714	0.000732421875	\\
0.755804146135748	0.00030517578125	\\
0.755848537310783	0.0001220703125	\\
0.755892928485817	-9.1552734375e-05	\\
0.755937319660851	-0.00042724609375	\\
0.755981710835886	-0.000732421875	\\
0.75602610201092	-0.00140380859375	\\
0.756070493185955	-0.001861572265625	\\
0.756114884360989	-0.00177001953125	\\
0.756159275536023	-0.00128173828125	\\
0.756203666711058	-0.001129150390625	\\
0.756248057886092	-0.00054931640625	\\
0.756292449061127	3.0517578125e-05	\\
0.756336840236161	-6.103515625e-05	\\
0.756381231411196	0.00030517578125	\\
0.75642562258623	0.000152587890625	\\
0.756470013761264	-0.000152587890625	\\
0.756514404936299	-0.0006103515625	\\
0.756558796111333	-0.00054931640625	\\
0.756603187286367	-0.00054931640625	\\
0.756647578461402	-0.000213623046875	\\
0.756691969636436	-0.000274658203125	\\
0.756736360811471	-0.000457763671875	\\
0.756780751986505	-9.1552734375e-05	\\
0.756825143161539	0.000518798828125	\\
0.756869534336574	0.000946044921875	\\
0.756913925511608	0.000640869140625	\\
0.756958316686643	0.000518798828125	\\
0.757002707861677	0.000396728515625	\\
0.757047099036712	0.000152587890625	\\
0.757091490211746	-0.00018310546875	\\
0.75713588138678	-0.00048828125	\\
0.757180272561815	-0.000152587890625	\\
0.757224663736849	0.000762939453125	\\
0.757269054911884	0.000762939453125	\\
0.757313446086918	0.000823974609375	\\
0.757357837261952	0.00048828125	\\
0.757402228436987	0.00067138671875	\\
0.757446619612021	0.000946044921875	\\
0.757491010787056	0.000823974609375	\\
0.75753540196209	0.000946044921875	\\
0.757579793137124	0.000762939453125	\\
0.757624184312159	0.0003662109375	\\
0.757668575487193	6.103515625e-05	\\
0.757712966662228	9.1552734375e-05	\\
0.757757357837262	0.00042724609375	\\
0.757801749012296	0.00042724609375	\\
0.757846140187331	0.000274658203125	\\
0.757890531362365	0.000579833984375	\\
0.7579349225374	0.000640869140625	\\
0.757979313712434	0.0009765625	\\
0.758023704887468	0.00091552734375	\\
0.758068096062503	0.000732421875	\\
0.758112487237537	0.000640869140625	\\
0.758156878412572	-3.0517578125e-05	\\
0.758201269587606	-0.00054931640625	\\
0.75824566076264	-3.0517578125e-05	\\
0.758290051937675	-0.00054931640625	\\
0.758334443112709	-0.001129150390625	\\
0.758378834287744	-0.00152587890625	\\
0.758423225462778	-0.00164794921875	\\
0.758467616637812	-0.001190185546875	\\
0.758512007812847	-0.0008544921875	\\
0.758556398987881	-0.000946044921875	\\
0.758600790162916	-0.001007080078125	\\
0.75864518133795	-0.00152587890625	\\
0.758689572512984	-0.0018310546875	\\
0.758733963688019	-0.00164794921875	\\
0.758778354863053	-0.00213623046875	\\
0.758822746038088	-0.001953125	\\
0.758867137213122	-0.001800537109375	\\
0.758911528388156	-0.00225830078125	\\
0.758955919563191	-0.002044677734375	\\
0.759000310738225	-0.001556396484375	\\
0.75904470191326	-0.001220703125	\\
0.759089093088294	-0.001312255859375	\\
0.759133484263329	-0.001312255859375	\\
0.759177875438363	-0.000946044921875	\\
0.759222266613397	-0.00054931640625	\\
0.759266657788432	-0.000732421875	\\
0.759311048963466	-0.00079345703125	\\
0.7593554401385	-0.000518798828125	\\
0.759399831313535	-0.000335693359375	\\
0.759444222488569	-0.0003662109375	\\
0.759488613663604	-0.000518798828125	\\
0.759533004838638	-0.000213623046875	\\
0.759577396013672	0.0001220703125	\\
0.759621787188707	-3.0517578125e-05	\\
0.759666178363741	-3.0517578125e-05	\\
0.759710569538776	0.00018310546875	\\
0.75975496071381	0.00030517578125	\\
0.759799351888845	0.000274658203125	\\
0.759843743063879	0.00042724609375	\\
0.759888134238913	-3.0517578125e-05	\\
0.759932525413948	-0.00067138671875	\\
0.759976916588982	-0.00115966796875	\\
0.760021307764017	-0.001251220703125	\\
0.760065698939051	-0.00146484375	\\
0.760110090114085	-0.001220703125	\\
0.76015448128912	-0.000946044921875	\\
0.760198872464154	-0.001251220703125	\\
0.760243263639188	-0.001190185546875	\\
0.760287654814223	-0.001373291015625	\\
0.760332045989257	-0.001922607421875	\\
0.760376437164292	-0.0025634765625	\\
0.760420828339326	-0.00244140625	\\
0.760465219514361	-0.00262451171875	\\
0.760509610689395	-0.003082275390625	\\
0.760554001864429	-0.003387451171875	\\
0.760598393039464	-0.0032958984375	\\
0.760642784214498	-0.003143310546875	\\
0.760687175389533	-0.00299072265625	\\
0.760731566564567	-0.002471923828125	\\
0.760775957739601	-0.00244140625	\\
0.760820348914636	-0.00244140625	\\
0.76086474008967	-0.00250244140625	\\
0.760909131264705	-0.002410888671875	\\
0.760953522439739	-0.00225830078125	\\
0.760997913614773	-0.002044677734375	\\
0.761042304789808	-0.001983642578125	\\
0.761086695964842	-0.002471923828125	\\
0.761131087139877	-0.002044677734375	\\
0.761175478314911	-0.001190185546875	\\
0.761219869489945	-0.00079345703125	\\
0.76126426066498	-0.00054931640625	\\
0.761308651840014	-0.000152587890625	\\
0.761353043015049	0.000244140625	\\
0.761397434190083	0.000396728515625	\\
0.761441825365117	0.00042724609375	\\
0.761486216540152	0.0003662109375	\\
0.761530607715186	0.000579833984375	\\
0.761574998890221	0.000640869140625	\\
0.761619390065255	0.000518798828125	\\
0.761663781240289	6.103515625e-05	\\
0.761708172415324	-6.103515625e-05	\\
0.761752563590358	0.00042724609375	\\
0.761796954765393	0.0008544921875	\\
0.761841345940427	0.00115966796875	\\
0.761885737115461	0.001708984375	\\
0.761930128290496	0.001678466796875	\\
0.76197451946553	0.001739501953125	\\
0.762018910640565	0.001495361328125	\\
0.762063301815599	0.001190185546875	\\
0.762107692990633	0.00067138671875	\\
0.762152084165668	0.000335693359375	\\
0.762196475340702	0.00018310546875	\\
0.762240866515737	-0.000152587890625	\\
0.762285257690771	-0.0001220703125	\\
0.762329648865805	0.000213623046875	\\
0.76237404004084	0.000213623046875	\\
0.762418431215874	0.000152587890625	\\
0.762462822390909	0.00048828125	\\
0.762507213565943	0.000274658203125	\\
0.762551604740978	0	\\
0.762595995916012	0	\\
0.762640387091046	-0.00018310546875	\\
0.762684778266081	-0.0001220703125	\\
0.762729169441115	-9.1552734375e-05	\\
0.76277356061615	-0.000244140625	\\
0.762817951791184	0.000213623046875	\\
0.762862342966218	0.0006103515625	\\
0.762906734141253	0.00067138671875	\\
0.762951125316287	0.0009765625	\\
0.762995516491322	0.001190185546875	\\
0.763039907666356	0.00128173828125	\\
0.76308429884139	0.0018310546875	\\
0.763128690016425	0.002044677734375	\\
0.763173081191459	0.00201416015625	\\
0.763217472366494	0.00164794921875	\\
0.763261863541528	0.00164794921875	\\
0.763306254716562	0.001495361328125	\\
0.763350645891597	0.001129150390625	\\
0.763395037066631	0.001495361328125	\\
0.763439428241666	0.0015869140625	\\
0.7634838194167	0.002105712890625	\\
0.763528210591734	0.002349853515625	\\
0.763572601766769	0.002227783203125	\\
0.763616992941803	0.002197265625	\\
0.763661384116838	0.001861572265625	\\
0.763705775291872	0.00140380859375	\\
0.763750166466906	0.000823974609375	\\
0.763794557641941	0.000518798828125	\\
0.763838948816975	0.000244140625	\\
0.76388333999201	0.000244140625	\\
0.763927731167044	6.103515625e-05	\\
0.763972122342078	0.0003662109375	\\
0.764016513517113	0.00030517578125	\\
0.764060904692147	-0.0001220703125	\\
0.764105295867182	-0.000152587890625	\\
0.764149687042216	-0.000335693359375	\\
0.76419407821725	0.000152587890625	\\
0.764238469392285	-0.0001220703125	\\
0.764282860567319	-0.000823974609375	\\
0.764327251742354	-0.00103759765625	\\
0.764371642917388	-0.001495361328125	\\
0.764416034092422	-0.001434326171875	\\
0.764460425267457	-0.001495361328125	\\
0.764504816442491	-0.00140380859375	\\
0.764549207617526	-0.001220703125	\\
0.76459359879256	-0.0010986328125	\\
0.764637989967594	-0.000732421875	\\
0.764682381142629	-0.000335693359375	\\
0.764726772317663	-9.1552734375e-05	\\
0.764771163492698	-9.1552734375e-05	\\
0.764815554667732	6.103515625e-05	\\
0.764859945842767	-0.00042724609375	\\
0.764904337017801	-0.00054931640625	\\
0.764948728192835	-0.00018310546875	\\
0.76499311936787	-0.0006103515625	\\
0.765037510542904	-0.000457763671875	\\
0.765081901717938	-0.000274658203125	\\
0.765126292892973	0.000396728515625	\\
0.765170684068007	0.000823974609375	\\
0.765215075243042	0.00054931640625	\\
0.765259466418076	0.000701904296875	\\
0.76530385759311	0.00103759765625	\\
0.765348248768145	0.001373291015625	\\
0.765392639943179	0.001220703125	\\
0.765437031118214	0.0008544921875	\\
0.765481422293248	0.00103759765625	\\
0.765525813468283	0.000946044921875	\\
0.765570204643317	0.001190185546875	\\
0.765614595818351	0.001556396484375	\\
0.765658986993386	0.001495361328125	\\
0.76570337816842	0.00146484375	\\
0.765747769343455	0.00189208984375	\\
0.765792160518489	0.002197265625	\\
0.765836551693523	0.002471923828125	\\
0.765880942868558	0.002471923828125	\\
0.765925334043592	0.00213623046875	\\
0.765969725218627	0.001617431640625	\\
0.766014116393661	0.002227783203125	\\
0.766058507568695	0.00213623046875	\\
0.76610289874373	0.00177001953125	\\
0.766147289918764	0.00213623046875	\\
0.766191681093799	0.0020751953125	\\
0.766236072268833	0.002044677734375	\\
0.766280463443867	0.002197265625	\\
0.766324854618902	0.00213623046875	\\
0.766369245793936	0.001861572265625	\\
0.766413636968971	0.001373291015625	\\
0.766458028144005	0.0010986328125	\\
0.766502419319039	0.000732421875	\\
0.766546810494074	0.000396728515625	\\
0.766591201669108	9.1552734375e-05	\\
0.766635592844143	-0.000152587890625	\\
0.766679984019177	6.103515625e-05	\\
0.766724375194211	-0.00042724609375	\\
0.766768766369246	-0.00067138671875	\\
0.76681315754428	-0.000946044921875	\\
0.766857548719315	-0.001556396484375	\\
0.766901939894349	-0.00164794921875	\\
0.766946331069383	-0.001617431640625	\\
0.766990722244418	-0.00164794921875	\\
0.767035113419452	-0.002166748046875	\\
0.767079504594487	-0.001983642578125	\\
0.767123895769521	-0.00201416015625	\\
0.767168286944555	-0.002471923828125	\\
0.76721267811959	-0.002349853515625	\\
0.767257069294624	-0.001678466796875	\\
0.767301460469659	-0.001556396484375	\\
0.767345851644693	-0.0015869140625	\\
0.767390242819727	-0.001312255859375	\\
0.767434633994762	-0.0013427734375	\\
0.767479025169796	-0.00091552734375	\\
0.767523416344831	-0.000396728515625	\\
0.767567807519865	-0.000152587890625	\\
0.767612198694899	0.0003662109375	\\
0.767656589869934	0.000640869140625	\\
0.767700981044968	0.000396728515625	\\
0.767745372220003	0.0008544921875	\\
0.767789763395037	0.001129150390625	\\
0.767834154570071	0.001007080078125	\\
0.767878545745106	0.001678466796875	\\
0.76792293692014	0.001861572265625	\\
0.767967328095175	0.00189208984375	\\
0.768011719270209	0.002471923828125	\\
0.768056110445243	0.002410888671875	\\
0.768100501620278	0.00238037109375	\\
0.768144892795312	0.002593994140625	\\
0.768189283970347	0.002349853515625	\\
0.768233675145381	0.00244140625	\\
0.768278066320416	0.00250244140625	\\
0.76832245749545	0.001708984375	\\
0.768366848670484	0.001312255859375	\\
0.768411239845519	0.00140380859375	\\
0.768455631020553	0.001678466796875	\\
0.768500022195588	0.0020751953125	\\
0.768544413370622	0.00213623046875	\\
0.768588804545656	0.001922607421875	\\
0.768633195720691	0.001312255859375	\\
0.768677586895725	0.000701904296875	\\
0.768721978070759	0.000640869140625	\\
0.768766369245794	9.1552734375e-05	\\
0.768810760420828	-9.1552734375e-05	\\
0.768855151595863	-0.00048828125	\\
0.768899542770897	-0.00079345703125	\\
0.768943933945932	-0.000823974609375	\\
0.768988325120966	-0.0009765625	\\
0.769032716296	-0.000823974609375	\\
0.769077107471035	-0.00103759765625	\\
0.769121498646069	-0.000518798828125	\\
0.769165889821104	-0.00030517578125	\\
0.769210280996138	-0.000518798828125	\\
0.769254672171172	-0.000946044921875	\\
0.769299063346207	-0.0010986328125	\\
0.769343454521241	-0.00067138671875	\\
0.769387845696276	-0.001068115234375	\\
0.76943223687131	-0.001434326171875	\\
0.769476628046344	-0.00079345703125	\\
0.769521019221379	-0.000885009765625	\\
0.769565410396413	-0.0006103515625	\\
0.769609801571448	6.103515625e-05	\\
0.769654192746482	0.0006103515625	\\
0.769698583921516	0.001708984375	\\
0.769742975096551	0.001861572265625	\\
0.769787366271585	0.001953125	\\
0.76983175744662	0.002532958984375	\\
0.769876148621654	0.002410888671875	\\
0.769920539796688	0.002288818359375	\\
0.769964930971723	0.00201416015625	\\
0.770009322146757	0.0023193359375	\\
0.770053713321792	0.0025634765625	\\
0.770098104496826	0.0023193359375	\\
0.77014249567186	0.00250244140625	\\
0.770186886846895	0.00262451171875	\\
0.770231278021929	0.00274658203125	\\
0.770275669196964	0.00286865234375	\\
0.770320060371998	0.0030517578125	\\
0.770364451547032	0.002838134765625	\\
0.770408842722067	0.00244140625	\\
0.770453233897101	0.001953125	\\
0.770497625072136	0.001983642578125	\\
0.77054201624717	0.001861572265625	\\
0.770586407422204	0.001251220703125	\\
0.770630798597239	0.00091552734375	\\
0.770675189772273	0.000518798828125	\\
0.770719580947308	0.000518798828125	\\
0.770763972122342	0.000701904296875	\\
0.770808363297376	0.000701904296875	\\
0.770852754472411	0.000762939453125	\\
0.770897145647445	0.00091552734375	\\
0.77094153682248	0.000823974609375	\\
0.770985927997514	0.000244140625	\\
0.771030319172549	-0.000335693359375	\\
0.771074710347583	-0.000213623046875	\\
0.771119101522617	3.0517578125e-05	\\
0.771163492697652	0.00018310546875	\\
0.771207883872686	9.1552734375e-05	\\
0.771252275047721	0.000396728515625	\\
0.771296666222755	0.001007080078125	\\
0.771341057397789	0.000823974609375	\\
0.771385448572824	0.001220703125	\\
0.771429839747858	0.001434326171875	\\
0.771474230922893	0.001068115234375	\\
0.771518622097927	0.00128173828125	\\
0.771563013272961	0.001220703125	\\
0.771607404447996	0.001220703125	\\
0.77165179562303	0.001251220703125	\\
0.771696186798065	0.0010986328125	\\
0.771740577973099	0.00152587890625	\\
0.771784969148133	0.001800537109375	\\
0.771829360323168	0.001617431640625	\\
0.771873751498202	0.00164794921875	\\
0.771918142673237	0.001495361328125	\\
0.771962533848271	0.001800537109375	\\
0.772006925023305	0.00201416015625	\\
0.77205131619834	0.00225830078125	\\
0.772095707373374	0.002593994140625	\\
0.772140098548409	0.00213623046875	\\
0.772184489723443	0.002105712890625	\\
0.772228880898477	0.001953125	\\
0.772273272073512	0.002197265625	\\
0.772317663248546	0.00201416015625	\\
0.772362054423581	0.001739501953125	\\
0.772406445598615	0.00189208984375	\\
0.772450836773649	0.00152587890625	\\
0.772495227948684	0.001495361328125	\\
0.772539619123718	0.001434326171875	\\
0.772584010298753	0.001312255859375	\\
0.772628401473787	0.001556396484375	\\
0.772672792648821	0.0013427734375	\\
0.772717183823856	0.001129150390625	\\
0.77276157499889	0.0008544921875	\\
0.772805966173925	0.000701904296875	\\
0.772850357348959	0.000732421875	\\
0.772894748523993	0.00042724609375	\\
0.772939139699028	0.000396728515625	\\
0.772983530874062	0.00030517578125	\\
0.773027922049097	0.000396728515625	\\
0.773072313224131	0.00030517578125	\\
0.773116704399165	0.000335693359375	\\
0.7731610955742	3.0517578125e-05	\\
0.773205486749234	-0.00018310546875	\\
0.773249877924269	-0.000396728515625	\\
0.773294269099303	-0.000762939453125	\\
0.773338660274338	-0.00067138671875	\\
0.773383051449372	-0.000457763671875	\\
0.773427442624406	-0.0008544921875	\\
0.773471833799441	-0.0010986328125	\\
0.773516224974475	-0.0010986328125	\\
0.773560616149509	-0.00164794921875	\\
0.773605007324544	-0.00164794921875	\\
0.773649398499578	-0.001617431640625	\\
0.773693789674613	-0.001434326171875	\\
0.773738180849647	-0.00091552734375	\\
0.773782572024681	-0.00103759765625	\\
0.773826963199716	-0.001495361328125	\\
0.77387135437475	-0.00146484375	\\
0.773915745549785	-0.001617431640625	\\
0.773960136724819	-0.001922607421875	\\
0.774004527899854	-0.001739501953125	\\
0.774048919074888	-0.00146484375	\\
0.774093310249922	-0.001251220703125	\\
0.774137701424957	-0.000885009765625	\\
0.774182092599991	-0.00048828125	\\
0.774226483775026	-0.000274658203125	\\
0.77427087495006	9.1552734375e-05	\\
0.774315266125094	6.103515625e-05	\\
0.774359657300129	0.000213623046875	\\
0.774404048475163	0.00018310546875	\\
0.774448439650198	0.00042724609375	\\
0.774492830825232	0.00115966796875	\\
0.774537222000266	0.0009765625	\\
0.774581613175301	0.0008544921875	\\
0.774626004350335	0.00091552734375	\\
0.77467039552537	0.000762939453125	\\
0.774714786700404	0.000701904296875	\\
0.774759177875438	0.00042724609375	\\
0.774803569050473	0.00042724609375	\\
0.774847960225507	0.000701904296875	\\
0.774892351400542	0.000213623046875	\\
0.774936742575576	0.00048828125	\\
0.77498113375061	0.00054931640625	\\
0.775025524925645	-3.0517578125e-05	\\
0.775069916100679	0.0001220703125	\\
0.775114307275714	0.00030517578125	\\
0.775158698450748	-9.1552734375e-05	\\
0.775203089625782	-0.00054931640625	\\
0.775247480800817	-0.000213623046875	\\
0.775291871975851	-0.0003662109375	\\
0.775336263150886	-0.000762939453125	\\
0.77538065432592	-0.0008544921875	\\
0.775425045500954	-0.000732421875	\\
0.775469436675989	-0.00067138671875	\\
0.775513827851023	-0.000701904296875	\\
0.775558219026058	-0.000823974609375	\\
0.775602610201092	-0.00079345703125	\\
0.775647001376126	-0.00103759765625	\\
0.775691392551161	-0.00152587890625	\\
0.775735783726195	-0.001800537109375	\\
0.77578017490123	-0.001556396484375	\\
0.775824566076264	-0.001434326171875	\\
0.775868957251298	-0.00140380859375	\\
0.775913348426333	-0.000732421875	\\
0.775957739601367	-0.00054931640625	\\
0.776002130776402	-0.000762939453125	\\
0.776046521951436	-0.00030517578125	\\
0.77609091312647	-0.000152587890625	\\
0.776135304301505	-0.00018310546875	\\
0.776179695476539	-0.000335693359375	\\
0.776224086651574	-9.1552734375e-05	\\
0.776268477826608	-3.0517578125e-05	\\
0.776312869001642	-3.0517578125e-05	\\
0.776357260176677	-0.000244140625	\\
0.776401651351711	0.000152587890625	\\
0.776446042526746	0.000701904296875	\\
0.77649043370178	0.000335693359375	\\
0.776534824876814	0.000274658203125	\\
0.776579216051849	0	\\
0.776623607226883	-0.000213623046875	\\
0.776667998401918	-6.103515625e-05	\\
0.776712389576952	-0.000244140625	\\
0.776756780751987	-0.0006103515625	\\
0.776801171927021	-0.000823974609375	\\
0.776845563102055	-0.001068115234375	\\
0.77688995427709	-0.00103759765625	\\
0.776934345452124	-0.00128173828125	\\
0.776978736627159	-0.001312255859375	\\
0.777023127802193	-0.001007080078125	\\
0.777067518977227	-0.0009765625	\\
0.777111910152262	-0.00103759765625	\\
0.777156301327296	-0.00115966796875	\\
0.77720069250233	-0.001495361328125	\\
0.777245083677365	-0.0015869140625	\\
0.777289474852399	-0.00152587890625	\\
0.777333866027434	-0.001800537109375	\\
0.777378257202468	-0.001678466796875	\\
0.777422648377503	-0.001495361328125	\\
0.777467039552537	-0.001678466796875	\\
0.777511430727571	-0.001678466796875	\\
0.777555821902606	-0.001678466796875	\\
0.77760021307764	-0.001495361328125	\\
0.777644604252675	-0.00115966796875	\\
0.777688995427709	-0.0008544921875	\\
0.777733386602743	-0.00048828125	\\
0.777777777777778	-0.000885009765625	\\
0.777822168952812	-0.001129150390625	\\
0.777866560127847	-0.000946044921875	\\
0.777910951302881	-0.0009765625	\\
0.777955342477915	-0.0006103515625	\\
0.77799973365295	-0.00018310546875	\\
0.778044124827984	-0.000274658203125	\\
0.778088516003019	-0.00030517578125	\\
0.778132907178053	-3.0517578125e-05	\\
0.778177298353087	0.000396728515625	\\
0.778221689528122	0.0001220703125	\\
0.778266080703156	9.1552734375e-05	\\
0.778310471878191	0.000213623046875	\\
0.778354863053225	0.000213623046875	\\
0.778399254228259	0.00042724609375	\\
0.778443645403294	0.000396728515625	\\
0.778488036578328	3.0517578125e-05	\\
0.778532427753363	0.000274658203125	\\
0.778576818928397	0.000152587890625	\\
0.778621210103431	9.1552734375e-05	\\
0.778665601278466	0.000152587890625	\\
0.7787099924535	-0.000335693359375	\\
0.778754383628535	-0.000274658203125	\\
0.778798774803569	-0.000640869140625	\\
0.778843165978603	-0.00115966796875	\\
0.778887557153638	-0.0010986328125	\\
0.778931948328672	-0.001373291015625	\\
0.778976339503707	-0.001312255859375	\\
0.779020730678741	-0.0015869140625	\\
0.779065121853776	-0.00201416015625	\\
0.77910951302881	-0.00225830078125	\\
0.779153904203844	-0.002197265625	\\
0.779198295378879	-0.002166748046875	\\
0.779242686553913	-0.002105712890625	\\
0.779287077728947	-0.002197265625	\\
0.779331468903982	-0.002166748046875	\\
0.779375860079016	-0.001983642578125	\\
0.779420251254051	-0.00201416015625	\\
0.779464642429085	-0.00189208984375	\\
0.779509033604119	-0.001556396484375	\\
0.779553424779154	-0.001953125	\\
0.779597815954188	-0.00189208984375	\\
0.779642207129223	-0.00152587890625	\\
0.779686598304257	-0.001220703125	\\
0.779730989479292	-0.001190185546875	\\
0.779775380654326	-0.001251220703125	\\
0.77981977182936	-0.001068115234375	\\
0.779864163004395	-0.000885009765625	\\
0.779908554179429	-0.000946044921875	\\
0.779952945354464	-0.000518798828125	\\
0.779997336529498	-0.00030517578125	\\
0.780041727704532	-0.000213623046875	\\
0.780086118879567	0.000213623046875	\\
0.780130510054601	0.000457763671875	\\
0.780174901229636	0.000213623046875	\\
0.78021929240467	0.00067138671875	\\
0.780263683579704	0.0010986328125	\\
0.780308074754739	0.001068115234375	\\
0.780352465929773	0.00152587890625	\\
0.780396857104808	0.001129150390625	\\
0.780441248279842	0.001220703125	\\
0.780485639454876	0.001373291015625	\\
0.780530030629911	0.001190185546875	\\
0.780574421804945	0.001556396484375	\\
0.78061881297998	0.001708984375	\\
0.780663204155014	0.001953125	\\
0.780707595330048	0.0018310546875	\\
0.780751986505083	0.00164794921875	\\
0.780796377680117	0.00140380859375	\\
0.780840768855152	0.001373291015625	\\
0.780885160030186	0.001556396484375	\\
0.78092955120522	0.0015869140625	\\
0.780973942380255	0.001434326171875	\\
0.781018333555289	0.001617431640625	\\
0.781062724730324	0.00164794921875	\\
0.781107115905358	0.001983642578125	\\
0.781151507080392	0.00189208984375	\\
0.781195898255427	0.001922607421875	\\
0.781240289430461	0.001739501953125	\\
0.781284680605496	0.001312255859375	\\
0.78132907178053	0.001434326171875	\\
0.781373462955564	0.00115966796875	\\
0.781417854130599	0.00103759765625	\\
0.781462245305633	0.001190185546875	\\
0.781506636480668	0.00048828125	\\
0.781551027655702	0.000213623046875	\\
0.781595418830736	0.000640869140625	\\
0.781639810005771	0.000762939453125	\\
0.781684201180805	0.000396728515625	\\
0.78172859235584	0.0001220703125	\\
0.781772983530874	-0.000396728515625	\\
0.781817374705909	-0.00048828125	\\
0.781861765880943	-0.000946044921875	\\
0.781906157055977	-0.00128173828125	\\
0.781950548231012	-0.001129150390625	\\
0.781994939406046	-0.001220703125	\\
0.78203933058108	-0.0013427734375	\\
0.782083721756115	-0.001922607421875	\\
0.782128112931149	-0.0020751953125	\\
0.782172504106184	-0.0020751953125	\\
0.782216895281218	-0.001708984375	\\
0.782261286456252	-0.0013427734375	\\
0.782305677631287	-0.001129150390625	\\
0.782350068806321	-0.0008544921875	\\
0.782394459981356	-0.000701904296875	\\
0.78243885115639	-0.00042724609375	\\
0.782483242331425	3.0517578125e-05	\\
0.782527633506459	-3.0517578125e-05	\\
0.782572024681493	-3.0517578125e-05	\\
0.782616415856528	9.1552734375e-05	\\
0.782660807031562	-0.000152587890625	\\
0.782705198206597	0.00018310546875	\\
0.782749589381631	0.000579833984375	\\
0.782793980556665	0.0008544921875	\\
0.7828383717317	0.00067138671875	\\
0.782882762906734	0.000946044921875	\\
0.782927154081768	0.00146484375	\\
0.782971545256803	0.001312255859375	\\
0.783015936431837	0.001190185546875	\\
0.783060327606872	0.0018310546875	\\
0.783104718781906	0.0023193359375	\\
0.783149109956941	0.002288818359375	\\
0.783193501131975	0.00238037109375	\\
0.783237892307009	0.00164794921875	\\
0.783282283482044	0.001708984375	\\
0.783326674657078	0.001678466796875	\\
0.783371065832113	0.0013427734375	\\
0.783415457007147	0.001251220703125	\\
0.783459848182181	0.001068115234375	\\
0.783504239357216	0.00054931640625	\\
0.78354863053225	0.000213623046875	\\
0.783593021707285	0.000457763671875	\\
0.783637412882319	0.000244140625	\\
0.783681804057353	0.000640869140625	\\
0.783726195232388	0.000762939453125	\\
0.783770586407422	3.0517578125e-05	\\
0.783814977582457	-0.000152587890625	\\
0.783859368757491	6.103515625e-05	\\
0.783903759932525	6.103515625e-05	\\
0.78394815110756	-0.00030517578125	\\
0.783992542282594	-0.000457763671875	\\
0.784036933457629	-0.000457763671875	\\
0.784081324632663	-0.0003662109375	\\
0.784125715807697	0.000274658203125	\\
0.784170106982732	0.0003662109375	\\
0.784214498157766	0.00018310546875	\\
0.784258889332801	0.000396728515625	\\
0.784303280507835	0.000152587890625	\\
0.784347671682869	-0.00018310546875	\\
0.784392062857904	-6.103515625e-05	\\
0.784436454032938	-0.00048828125	\\
0.784480845207973	-0.000274658203125	\\
0.784525236383007	0.000274658203125	\\
0.784569627558041	0.000579833984375	\\
0.784614018733076	0.00079345703125	\\
0.78465840990811	0.000213623046875	\\
0.784702801083145	0.00054931640625	\\
0.784747192258179	0.001129150390625	\\
0.784791583433213	0.000732421875	\\
0.784835974608248	0.000946044921875	\\
0.784880365783282	0.00067138671875	\\
0.784924756958317	0.000640869140625	\\
0.784969148133351	0.0008544921875	\\
0.785013539308385	0.00079345703125	\\
0.78505793048342	0.001190185546875	\\
0.785102321658454	0.001007080078125	\\
0.785146712833489	0.00103759765625	\\
0.785191104008523	0.001251220703125	\\
0.785235495183558	0.001068115234375	\\
0.785279886358592	0.001556396484375	\\
0.785324277533626	0.001861572265625	\\
0.785368668708661	0.001739501953125	\\
0.785413059883695	0.00146484375	\\
0.78545745105873	0.001678466796875	\\
0.785501842233764	0.001922607421875	\\
0.785546233408798	0.001922607421875	\\
0.785590624583833	0.00164794921875	\\
0.785635015758867	0.001800537109375	\\
0.785679406933902	0.002227783203125	\\
0.785723798108936	0.002410888671875	\\
0.78576818928397	0.002471923828125	\\
0.785812580459005	0.00244140625	\\
0.785856971634039	0.002166748046875	\\
0.785901362809074	0.0020751953125	\\
0.785945753984108	0.002593994140625	\\
0.785990145159142	0.002227783203125	\\
0.786034536334177	0.0018310546875	\\
0.786078927509211	0.00189208984375	\\
0.786123318684246	0.002410888671875	\\
0.78616770985928	0.002471923828125	\\
0.786212101034314	0.00244140625	\\
0.786256492209349	0.002777099609375	\\
0.786300883384383	0.002532958984375	\\
0.786345274559418	0.002838134765625	\\
0.786389665734452	0.00311279296875	\\
0.786434056909486	0.00311279296875	\\
0.786478448084521	0.002532958984375	\\
0.786522839259555	0.002166748046875	\\
0.78656723043459	0.00201416015625	\\
0.786611621609624	0.00238037109375	\\
0.786656012784658	0.002197265625	\\
0.786700403959693	0.002227783203125	\\
0.786744795134727	0.002532958984375	\\
0.786789186309762	0.00213623046875	\\
0.786833577484796	0.002044677734375	\\
0.78687796865983	0.00238037109375	\\
0.786922359834865	0.002532958984375	\\
0.786966751009899	0.002197265625	\\
0.787011142184934	0.002166748046875	\\
0.787055533359968	0.001708984375	\\
0.787099924535002	0.0018310546875	\\
0.787144315710037	0.002044677734375	\\
0.787188706885071	0.001861572265625	\\
0.787233098060106	0.002227783203125	\\
0.78727748923514	0.00213623046875	\\
0.787321880410174	0.002044677734375	\\
0.787366271585209	0.00225830078125	\\
0.787410662760243	0.002349853515625	\\
0.787455053935278	0.002410888671875	\\
0.787499445110312	0.00201416015625	\\
0.787543836285347	0.00225830078125	\\
0.787588227460381	0.00213623046875	\\
0.787632618635415	0.001617431640625	\\
0.78767700981045	0.00164794921875	\\
0.787721400985484	0.002044677734375	\\
0.787765792160518	0.0015869140625	\\
0.787810183335553	0.00152587890625	\\
0.787854574510587	0.001678466796875	\\
0.787898965685622	0.001617431640625	\\
0.787943356860656	0.00213623046875	\\
0.78798774803569	0.00262451171875	\\
0.788032139210725	0.002685546875	\\
0.788076530385759	0.002685546875	\\
0.788120921560794	0.003143310546875	\\
0.788165312735828	0.00323486328125	\\
0.788209703910863	0.00286865234375	\\
0.788254095085897	0.002685546875	\\
0.788298486260931	0.002532958984375	\\
0.788342877435966	0.0020751953125	\\
0.788387268611	0.001983642578125	\\
0.788431659786035	0.001739501953125	\\
0.788476050961069	0.001617431640625	\\
0.788520442136103	0.001922607421875	\\
0.788564833311138	0.002197265625	\\
0.788609224486172	0.0020751953125	\\
0.788653615661207	0.00164794921875	\\
0.788698006836241	0.00177001953125	\\
0.788742398011275	0.00201416015625	\\
0.78878678918631	0.001373291015625	\\
0.788831180361344	0.00140380859375	\\
0.788875571536379	0.0018310546875	\\
0.788919962711413	0.001953125	\\
0.788964353886447	0.001953125	\\
0.789008745061482	0.00213623046875	\\
0.789053136236516	0.00225830078125	\\
0.789097527411551	0.00274658203125	\\
0.789141918586585	0.00323486328125	\\
0.789186309761619	0.003021240234375	\\
0.789230700936654	0.003021240234375	\\
0.789275092111688	0.0035400390625	\\
0.789319483286723	0.003631591796875	\\
0.789363874461757	0.00341796875	\\
0.789408265636791	0.003448486328125	\\
0.789452656811826	0.0037841796875	\\
0.78949704798686	0.00384521484375	\\
0.789541439161895	0.003936767578125	\\
0.789585830336929	0.0040283203125	\\
0.789630221511963	0.00439453125	\\
0.789674612686998	0.00482177734375	\\
0.789719003862032	0.00518798828125	\\
0.789763395037067	0.00543212890625	\\
0.789807786212101	0.005401611328125	\\
0.789852177387135	0.005279541015625	\\
0.78989656856217	0.0048828125	\\
0.789940959737204	0.004058837890625	\\
0.789985350912239	0.003662109375	\\
0.790029742087273	0.003387451171875	\\
0.790074133262307	0.0030517578125	\\
0.790118524437342	0.00299072265625	\\
0.790162915612376	0.003204345703125	\\
0.790207306787411	0.0028076171875	\\
0.790251697962445	0.002685546875	\\
0.79029608913748	0.0023193359375	\\
0.790340480312514	0.002532958984375	\\
0.790384871487548	0.00225830078125	\\
0.790429262662583	0.001678466796875	\\
0.790473653837617	0.00189208984375	\\
0.790518045012651	0.00152587890625	\\
0.790562436187686	0.001129150390625	\\
0.79060682736272	0.000762939453125	\\
0.790651218537755	0.0008544921875	\\
0.790695609712789	0.000579833984375	\\
0.790740000887823	9.1552734375e-05	\\
0.790784392062858	0.00018310546875	\\
0.790828783237892	0.00018310546875	\\
0.790873174412927	9.1552734375e-05	\\
0.790917565587961	0.0006103515625	\\
0.790961956762996	0.000823974609375	\\
0.79100634793803	0.001220703125	\\
0.791050739113064	0.001373291015625	\\
0.791095130288099	0.0009765625	\\
0.791139521463133	0.001434326171875	\\
0.791183912638168	0.001495361328125	\\
0.791228303813202	0.001556396484375	\\
0.791272694988236	0.001953125	\\
0.791317086163271	0.00189208984375	\\
0.791361477338305	0.002227783203125	\\
0.791405868513339	0.00244140625	\\
0.791450259688374	0.00238037109375	\\
0.791494650863408	0.001953125	\\
0.791539042038443	0.0020751953125	\\
0.791583433213477	0.0023193359375	\\
0.791627824388512	0.00250244140625	\\
0.791672215563546	0.00244140625	\\
0.79171660673858	0.00238037109375	\\
0.791760997913615	0.002532958984375	\\
0.791805389088649	0.0023193359375	\\
0.791849780263684	0.002410888671875	\\
0.791894171438718	0.00213623046875	\\
0.791938562613752	0.002349853515625	\\
0.791982953788787	0.002960205078125	\\
0.792027344963821	0.002471923828125	\\
0.792071736138856	0.00238037109375	\\
0.79211612731389	0.002410888671875	\\
0.792160518488924	0.00238037109375	\\
0.792204909663959	0.00201416015625	\\
0.792249300838993	0.002410888671875	\\
0.792293692014028	0.00250244140625	\\
0.792338083189062	0.002349853515625	\\
0.792382474364096	0.002410888671875	\\
0.792426865539131	0.002349853515625	\\
0.792471256714165	0.002227783203125	\\
0.7925156478892	0.0018310546875	\\
0.792560039064234	0.002197265625	\\
0.792604430239268	0.001922607421875	\\
0.792648821414303	0.001495361328125	\\
0.792693212589337	0.0010986328125	\\
0.792737603764372	0.000885009765625	\\
0.792781994939406	0.00079345703125	\\
0.79282638611444	0.000579833984375	\\
0.792870777289475	0.00054931640625	\\
0.792915168464509	0.000823974609375	\\
0.792959559639544	0.000640869140625	\\
0.793003950814578	0.000640869140625	\\
0.793048341989612	0.000732421875	\\
0.793092733164647	0.0006103515625	\\
0.793137124339681	0.000518798828125	\\
0.793181515514716	0.00048828125	\\
0.79322590668975	0.00054931640625	\\
0.793270297864785	0.00067138671875	\\
0.793314689039819	0.000885009765625	\\
0.793359080214853	0.000518798828125	\\
0.793403471389888	0.000396728515625	\\
0.793447862564922	0.000244140625	\\
0.793492253739956	0.000244140625	\\
0.793536644914991	0.000579833984375	\\
0.793581036090025	0.000457763671875	\\
0.79362542726506	0.000885009765625	\\
0.793669818440094	0.001373291015625	\\
0.793714209615129	0.0018310546875	\\
0.793758600790163	0.001953125	\\
0.793802991965197	0.002410888671875	\\
0.793847383140232	0.002471923828125	\\
0.793891774315266	0.0020751953125	\\
0.793936165490301	0.00262451171875	\\
0.793980556665335	0.003082275390625	\\
0.794024947840369	0.0029296875	\\
0.794069339015404	0.0030517578125	\\
0.794113730190438	0.003753662109375	\\
0.794158121365473	0.004486083984375	\\
0.794202512540507	0.00482177734375	\\
0.794246903715541	0.00445556640625	\\
0.794291294890576	0.0045166015625	\\
0.79433568606561	0.00482177734375	\\
0.794380077240645	0.004852294921875	\\
0.794424468415679	0.004852294921875	\\
0.794468859590713	0.004608154296875	\\
0.794513250765748	0.004486083984375	\\
0.794557641940782	0.00482177734375	\\
0.794602033115817	0.005035400390625	\\
0.794646424290851	0.0048828125	\\
0.794690815465885	0.004730224609375	\\
0.79473520664092	0.004425048828125	\\
0.794779597815954	0.005035400390625	\\
0.794823988990989	0.005035400390625	\\
0.794868380166023	0.00457763671875	\\
0.794912771341057	0.004425048828125	\\
0.794957162516092	0.003814697265625	\\
0.795001553691126	0.00335693359375	\\
0.795045944866161	0.002838134765625	\\
0.795090336041195	0.002685546875	\\
0.795134727216229	0.002410888671875	\\
0.795179118391264	0.001495361328125	\\
0.795223509566298	0.001129150390625	\\
0.795267900741333	0.00115966796875	\\
0.795312291916367	0.000946044921875	\\
0.795356683091401	0.000823974609375	\\
0.795401074266436	0.000396728515625	\\
0.79544546544147	6.103515625e-05	\\
0.795489856616505	0.000457763671875	\\
0.795534247791539	0.000518798828125	\\
0.795578638966573	0.00030517578125	\\
0.795623030141608	0.00030517578125	\\
0.795667421316642	0.00030517578125	\\
0.795711812491677	0.000152587890625	\\
0.795756203666711	0.000335693359375	\\
0.795800594841745	0.00067138671875	\\
0.79584498601678	0.000701904296875	\\
0.795889377191814	0.001007080078125	\\
0.795933768366849	0.001220703125	\\
0.795978159541883	0.0015869140625	\\
0.796022550716918	0.00177001953125	\\
0.796066941891952	0.001434326171875	\\
0.796111333066986	0.0015869140625	\\
0.796155724242021	0.001617431640625	\\
0.796200115417055	0.001983642578125	\\
0.796244506592089	0.001953125	\\
0.796288897767124	0.00213623046875	\\
0.796333288942158	0.0025634765625	\\
0.796377680117193	0.00244140625	\\
0.796422071292227	0.002349853515625	\\
0.796466462467261	0.002227783203125	\\
0.796510853642296	0.002197265625	\\
0.79655524481733	0.00244140625	\\
0.796599635992365	0.00262451171875	\\
0.796644027167399	0.002777099609375	\\
0.796688418342434	0.0028076171875	\\
0.796732809517468	0.002716064453125	\\
0.796777200692502	0.002655029296875	\\
0.796821591867537	0.00250244140625	\\
0.796865983042571	0.001953125	\\
0.796910374217606	0.00140380859375	\\
0.79695476539264	0.001190185546875	\\
0.796999156567674	0.00115966796875	\\
0.797043547742709	0.0010986328125	\\
0.797087938917743	0.000762939453125	\\
0.797132330092778	0.001068115234375	\\
0.797176721267812	0.0009765625	\\
0.797221112442846	0.0008544921875	\\
0.797265503617881	0.0010986328125	\\
0.797309894792915	0.00115966796875	\\
0.79735428596795	0.001434326171875	\\
0.797398677142984	0.001434326171875	\\
0.797443068318018	0.0015869140625	\\
0.797487459493053	0.00128173828125	\\
0.797531850668087	0.001129150390625	\\
0.797576241843122	0.00128173828125	\\
0.797620633018156	0.001312255859375	\\
0.79766502419319	0.00115966796875	\\
0.797709415368225	0.001739501953125	\\
0.797753806543259	0.00152587890625	\\
0.797798197718294	0.00103759765625	\\
0.797842588893328	0.002166748046875	\\
0.797886980068362	0.0025634765625	\\
0.797931371243397	0.002288818359375	\\
0.797975762418431	0.001983642578125	\\
0.798020153593466	0.002197265625	\\
0.7980645447685	0.00189208984375	\\
0.798108935943534	0.001800537109375	\\
0.798153327118569	0.00146484375	\\
0.798197718293603	0.00146484375	\\
0.798242109468638	0.0020751953125	\\
0.798286500643672	0.0020751953125	\\
0.798330891818706	0.002105712890625	\\
0.798375282993741	0.001983642578125	\\
0.798419674168775	0.001800537109375	\\
0.79846406534381	0.001678466796875	\\
0.798508456518844	0.0013427734375	\\
0.798552847693878	0.001495361328125	\\
0.798597238868913	0.001495361328125	\\
0.798641630043947	0.00103759765625	\\
0.798686021218982	0.00115966796875	\\
0.798730412394016	0.00103759765625	\\
0.798774803569051	0.000640869140625	\\
0.798819194744085	0.00079345703125	\\
0.798863585919119	0.00042724609375	\\
0.798907977094154	6.103515625e-05	\\
0.798952368269188	0.000823974609375	\\
0.798996759444222	0.000640869140625	\\
0.799041150619257	0	\\
0.799085541794291	0.00030517578125	\\
0.799129932969326	0.0003662109375	\\
0.79917432414436	0.00018310546875	\\
0.799218715319394	-0.0001220703125	\\
0.799263106494429	-0.000213623046875	\\
0.799307497669463	-0.000518798828125	\\
0.799351888844498	-0.000732421875	\\
0.799396280019532	-0.00048828125	\\
0.799440671194567	-0.00018310546875	\\
0.799485062369601	-0.0008544921875	\\
0.799529453544635	-0.0006103515625	\\
0.79957384471967	-0.00042724609375	\\
0.799618235894704	-0.000732421875	\\
0.799662627069739	-3.0517578125e-05	\\
0.799707018244773	0.000335693359375	\\
0.799751409419807	-0.0003662109375	\\
0.799795800594842	-0.000701904296875	\\
0.799840191769876	-0.00042724609375	\\
0.799884582944911	-0.000762939453125	\\
0.799928974119945	-0.0008544921875	\\
0.799973365294979	-0.00067138671875	\\
0.800017756470014	-0.000396728515625	\\
0.800062147645048	-0.000335693359375	\\
0.800106538820083	-0.00054931640625	\\
0.800150929995117	-0.0006103515625	\\
0.800195321170151	-0.000457763671875	\\
0.800239712345186	-0.00048828125	\\
0.80028410352022	9.1552734375e-05	\\
0.800328494695255	0.000213623046875	\\
0.800372885870289	-0.00030517578125	\\
0.800417277045323	-0.000335693359375	\\
0.800461668220358	-0.0001220703125	\\
0.800506059395392	0.00042724609375	\\
0.800550450570427	0.000335693359375	\\
0.800594841745461	0.000457763671875	\\
0.800639232920495	0.00054931640625	\\
0.80068362409553	0.000396728515625	\\
0.800728015270564	0.000885009765625	\\
0.800772406445599	0.001068115234375	\\
0.800816797620633	0.00091552734375	\\
0.800861188795667	0.00103759765625	\\
0.800905579970702	0.0008544921875	\\
0.800949971145736	0.00140380859375	\\
0.800994362320771	0.00152587890625	\\
0.801038753495805	0.001190185546875	\\
0.801083144670839	0.0010986328125	\\
0.801127535845874	0.001068115234375	\\
0.801171927020908	0.000946044921875	\\
0.801216318195943	0.000579833984375	\\
0.801260709370977	0.000457763671875	\\
0.801305100546011	-9.1552734375e-05	\\
0.801349491721046	-9.1552734375e-05	\\
0.80139388289608	0.000244140625	\\
0.801438274071115	0.000274658203125	\\
0.801482665246149	-3.0517578125e-05	\\
0.801527056421183	-0.0001220703125	\\
0.801571447596218	-0.00018310546875	\\
0.801615838771252	-0.000274658203125	\\
0.801660229946287	-0.00054931640625	\\
0.801704621121321	-0.000946044921875	\\
0.801749012296356	-0.001007080078125	\\
0.80179340347139	-0.001068115234375	\\
0.801837794646424	-0.00103759765625	\\
0.801882185821459	-0.000885009765625	\\
0.801926576996493	-0.0010986328125	\\
0.801970968171527	-0.001007080078125	\\
0.802015359346562	-0.001007080078125	\\
0.802059750521596	-0.0018310546875	\\
0.802104141696631	-0.00146484375	\\
0.802148532871665	-0.00128173828125	\\
0.8021929240467	-0.001068115234375	\\
0.802237315221734	-0.000823974609375	\\
0.802281706396768	-0.00079345703125	\\
0.802326097571803	-0.000579833984375	\\
0.802370488746837	-0.000213623046875	\\
0.802414879921872	-0.000152587890625	\\
0.802459271096906	-9.1552734375e-05	\\
0.80250366227194	3.0517578125e-05	\\
0.802548053446975	9.1552734375e-05	\\
0.802592444622009	3.0517578125e-05	\\
0.802636835797044	0.000396728515625	\\
0.802681226972078	0.000518798828125	\\
0.802725618147112	0.000823974609375	\\
0.802770009322147	0.00103759765625	\\
0.802814400497181	0.001312255859375	\\
0.802858791672216	0.0018310546875	\\
0.80290318284725	0.001312255859375	\\
0.802947574022284	0.001312255859375	\\
0.802991965197319	0.001312255859375	\\
0.803036356372353	0.001434326171875	\\
0.803080747547388	0.0013427734375	\\
0.803125138722422	0.001251220703125	\\
0.803169529897456	0.001434326171875	\\
0.803213921072491	0.00128173828125	\\
0.803258312247525	0.001434326171875	\\
0.80330270342256	0.00164794921875	\\
0.803347094597594	0.00177001953125	\\
0.803391485772628	0.001800537109375	\\
0.803435876947663	0.001434326171875	\\
0.803480268122697	0.001556396484375	\\
0.803524659297732	0.00140380859375	\\
0.803569050472766	0.001434326171875	\\
0.8036134416478	0.001434326171875	\\
0.803657832822835	0.0008544921875	\\
0.803702223997869	0.001312255859375	\\
0.803746615172904	0.0009765625	\\
0.803791006347938	0.000274658203125	\\
0.803835397522972	3.0517578125e-05	\\
0.803879788698007	-6.103515625e-05	\\
0.803924179873041	-9.1552734375e-05	\\
0.803968571048076	-0.00054931640625	\\
0.80401296222311	-0.00048828125	\\
0.804057353398144	-0.00030517578125	\\
0.804101744573179	9.1552734375e-05	\\
0.804146135748213	-9.1552734375e-05	\\
0.804190526923248	-0.000457763671875	\\
0.804234918098282	-0.000335693359375	\\
0.804279309273316	-0.00048828125	\\
0.804323700448351	-0.0008544921875	\\
0.804368091623385	-0.000885009765625	\\
0.80441248279842	-0.000457763671875	\\
0.804456873973454	-0.000213623046875	\\
0.804501265148489	9.1552734375e-05	\\
0.804545656323523	0.00042724609375	\\
0.804590047498557	0.000396728515625	\\
0.804634438673592	0.00030517578125	\\
0.804678829848626	0.000518798828125	\\
0.80472322102366	0.000457763671875	\\
0.804767612198695	0.000640869140625	\\
0.804812003373729	0.000640869140625	\\
0.804856394548764	0.000152587890625	\\
0.804900785723798	0.000274658203125	\\
0.804945176898832	0.000244140625	\\
0.804989568073867	-0.00042724609375	\\
0.805033959248901	-0.000457763671875	\\
0.805078350423936	-0.000457763671875	\\
0.80512274159897	-0.000396728515625	\\
0.805167132774005	-9.1552734375e-05	\\
0.805211523949039	0.000640869140625	\\
0.805255915124073	0.00091552734375	\\
0.805300306299108	0.000885009765625	\\
0.805344697474142	0.001190185546875	\\
0.805389088649177	0.000946044921875	\\
0.805433479824211	0.00079345703125	\\
0.805477870999245	0.00048828125	\\
0.80552226217428	0.00030517578125	\\
0.805566653349314	0.000579833984375	\\
0.805611044524349	0.000457763671875	\\
0.805655435699383	0.000396728515625	\\
0.805699826874417	0.000518798828125	\\
0.805744218049452	0.000518798828125	\\
0.805788609224486	0.00103759765625	\\
0.805833000399521	0.001068115234375	\\
0.805877391574555	0.001129150390625	\\
0.805921782749589	0.00152587890625	\\
0.805966173924624	0.00146484375	\\
0.806010565099658	0.001953125	\\
0.806054956274693	0.002197265625	\\
0.806099347449727	0.00164794921875	\\
0.806143738624761	0.001708984375	\\
0.806188129799796	0.001922607421875	\\
0.80623252097483	0.001739501953125	\\
0.806276912149865	0.001617431640625	\\
0.806321303324899	0.00146484375	\\
0.806365694499933	0.001129150390625	\\
0.806410085674968	0.0009765625	\\
0.806454476850002	0.00115966796875	\\
0.806498868025037	0.001312255859375	\\
0.806543259200071	0.001312255859375	\\
0.806587650375105	0.00103759765625	\\
0.80663204155014	0.00103759765625	\\
0.806676432725174	0.000946044921875	\\
0.806720823900209	0.00103759765625	\\
0.806765215075243	0.00103759765625	\\
0.806809606250277	0.000732421875	\\
0.806853997425312	0.0010986328125	\\
0.806898388600346	0.00128173828125	\\
0.806942779775381	0.0008544921875	\\
0.806987170950415	0.000701904296875	\\
0.807031562125449	0.00079345703125	\\
0.807075953300484	0.000457763671875	\\
0.807120344475518	0.000244140625	\\
0.807164735650553	0.00048828125	\\
0.807209126825587	0.000396728515625	\\
0.807253518000622	0.0006103515625	\\
0.807297909175656	0.0009765625	\\
0.80734230035069	0.000885009765625	\\
0.807386691525725	0.001251220703125	\\
0.807431082700759	0.001373291015625	\\
0.807475473875793	0.001129150390625	\\
0.807519865050828	0.0013427734375	\\
0.807564256225862	0.001434326171875	\\
0.807608647400897	0.0013427734375	\\
0.807653038575931	0.001312255859375	\\
0.807697429750965	0.001434326171875	\\
0.807741820926	0.00140380859375	\\
0.807786212101034	0.0013427734375	\\
0.807830603276069	0.00164794921875	\\
0.807874994451103	0.00152587890625	\\
0.807919385626138	0.001922607421875	\\
0.807963776801172	0.002166748046875	\\
0.808008167976206	0.001739501953125	\\
0.808052559151241	0.001678466796875	\\
0.808096950326275	0.0013427734375	\\
0.80814134150131	0.00103759765625	\\
0.808185732676344	0.0013427734375	\\
0.808230123851378	0.001007080078125	\\
0.808274515026413	0.000823974609375	\\
0.808318906201447	0.001251220703125	\\
0.808363297376482	0.00103759765625	\\
0.808407688551516	0.0010986328125	\\
0.80845207972655	0.000946044921875	\\
0.808496470901585	0.001190185546875	\\
0.808540862076619	0.001220703125	\\
0.808585253251654	0.001251220703125	\\
0.808629644426688	0.00115966796875	\\
0.808674035601722	0.000823974609375	\\
0.808718426776757	0.001312255859375	\\
0.808762817951791	0.001312255859375	\\
0.808807209126826	0.001556396484375	\\
0.80885160030186	0.001678466796875	\\
0.808895991476894	0.001922607421875	\\
0.808940382651929	0.002197265625	\\
0.808984773826963	0.002349853515625	\\
0.809029165001998	0.0029296875	\\
0.809073556177032	0.00311279296875	\\
0.809117947352066	0.003143310546875	\\
0.809162338527101	0.003326416015625	\\
0.809206729702135	0.002960205078125	\\
0.80925112087717	0.002960205078125	\\
0.809295512052204	0.00311279296875	\\
0.809339903227238	0.002960205078125	\\
0.809384294402273	0.002960205078125	\\
0.809428685577307	0.003143310546875	\\
0.809473076752342	0.00323486328125	\\
0.809517467927376	0.003387451171875	\\
0.80956185910241	0.003326416015625	\\
0.809606250277445	0.0035400390625	\\
0.809650641452479	0.003570556640625	\\
0.809695032627514	0.003448486328125	\\
0.809739423802548	0.003326416015625	\\
0.809783814977582	0.003082275390625	\\
0.809828206152617	0.00225830078125	\\
0.809872597327651	0.001800537109375	\\
0.809916988502686	0.00146484375	\\
0.80996137967772	0.001190185546875	\\
0.810005770852754	0.00140380859375	\\
0.810050162027789	0.001434326171875	\\
0.810094553202823	0.001312255859375	\\
0.810138944377858	0.001220703125	\\
0.810183335552892	0.00103759765625	\\
0.810227726727927	0.000762939453125	\\
0.810272117902961	0.000701904296875	\\
0.810316509077995	0.000640869140625	\\
0.81036090025303	0	\\
0.810405291428064	9.1552734375e-05	\\
0.810449682603098	0.00030517578125	\\
0.810494073778133	0.00018310546875	\\
0.810538464953167	0.0001220703125	\\
0.810582856128202	-0.000244140625	\\
0.810627247303236	-0.0003662109375	\\
0.810671638478271	6.103515625e-05	\\
0.810716029653305	0.00030517578125	\\
0.810760420828339	0.00042724609375	\\
0.810804812003374	0.0010986328125	\\
0.810849203178408	0.000946044921875	\\
0.810893594353443	0.00079345703125	\\
0.810937985528477	0.00115966796875	\\
0.810982376703511	0.00079345703125	\\
0.811026767878546	0.000946044921875	\\
0.81107115905358	0.0008544921875	\\
0.811115550228615	0.00177001953125	\\
0.811159941403649	0.00201416015625	\\
0.811204332578683	0.002349853515625	\\
0.811248723753718	0.002471923828125	\\
0.811293114928752	0.0025634765625	\\
0.811337506103787	0.00341796875	\\
0.811381897278821	0.003509521484375	\\
0.811426288453855	0.0040283203125	\\
0.81147067962889	0.0045166015625	\\
0.811515070803924	0.00439453125	\\
0.811559461978959	0.004425048828125	\\
0.811603853153993	0.00408935546875	\\
0.811648244329027	0.004364013671875	\\
0.811692635504062	0.00445556640625	\\
0.811737026679096	0.004486083984375	\\
0.811781417854131	0.00457763671875	\\
0.811825809029165	0.00439453125	\\
0.811870200204199	0.00494384765625	\\
0.811914591379234	0.00518798828125	\\
0.811958982554268	0.004791259765625	\\
0.812003373729303	0.00421142578125	\\
0.812047764904337	0.004486083984375	\\
0.812092156079371	0.004425048828125	\\
0.812136547254406	0.00390625	\\
0.81218093842944	0.00341796875	\\
0.812225329604475	0.00262451171875	\\
0.812269720779509	0.00250244140625	\\
0.812314111954543	0.00244140625	\\
0.812358503129578	0.001922607421875	\\
0.812402894304612	0.001678466796875	\\
0.812447285479647	0.001220703125	\\
0.812491676654681	0.0010986328125	\\
0.812536067829715	0.001129150390625	\\
0.81258045900475	0.000579833984375	\\
0.812624850179784	-9.1552734375e-05	\\
0.812669241354819	-0.000335693359375	\\
0.812713632529853	-0.000335693359375	\\
0.812758023704887	-0.00103759765625	\\
0.812802414879922	-0.00103759765625	\\
0.812846806054956	-0.000885009765625	\\
0.812891197229991	-0.001129150390625	\\
0.812935588405025	-0.001312255859375	\\
0.81297997958006	-0.00140380859375	\\
0.813024370755094	-0.0006103515625	\\
0.813068761930128	-0.000762939453125	\\
0.813113153105163	-0.0010986328125	\\
0.813157544280197	-0.000701904296875	\\
0.813201935455231	-0.00103759765625	\\
0.813246326630266	-0.000579833984375	\\
0.8132907178053	3.0517578125e-05	\\
0.813335108980335	-0.00030517578125	\\
0.813379500155369	0.0001220703125	\\
0.813423891330403	0.0006103515625	\\
0.813468282505438	0.00048828125	\\
0.813512673680472	0.000732421875	\\
0.813557064855507	0.001007080078125	\\
0.813601456030541	0.00091552734375	\\
0.813645847205576	0.001068115234375	\\
0.81369023838061	0.0013427734375	\\
0.813734629555644	0.002410888671875	\\
0.813779020730679	0.00299072265625	\\
0.813823411905713	0.003143310546875	\\
0.813867803080748	0.003662109375	\\
0.813912194255782	0.003509521484375	\\
0.813956585430816	0.003662109375	\\
0.814000976605851	0.0035400390625	\\
0.814045367780885	0.003326416015625	\\
0.81408975895592	0.003265380859375	\\
0.814134150130954	0.00311279296875	\\
0.814178541305988	0.003387451171875	\\
0.814222932481023	0.003204345703125	\\
0.814267323656057	0.003570556640625	\\
0.814311714831092	0.003570556640625	\\
0.814356106006126	0.003387451171875	\\
0.81440049718116	0.0040283203125	\\
0.814444888356195	0.003753662109375	\\
0.814489279531229	0.00323486328125	\\
0.814533670706264	0.00360107421875	\\
0.814578061881298	0.003448486328125	\\
0.814622453056332	0.002838134765625	\\
0.814666844231367	0.002044677734375	\\
0.814711235406401	0.001617431640625	\\
0.814755626581436	0.00128173828125	\\
0.81480001775647	0.00079345703125	\\
0.814844408931504	0.0006103515625	\\
0.814888800106539	0.0003662109375	\\
0.814933191281573	0.000244140625	\\
0.814977582456608	0.000701904296875	\\
0.815021973631642	0.001068115234375	\\
0.815066364806676	0.000823974609375	\\
0.815110755981711	0.00067138671875	\\
0.815155147156745	0.000701904296875	\\
0.81519953833178	0.00067138671875	\\
0.815243929506814	0.00048828125	\\
0.815288320681848	-0.00018310546875	\\
0.815332711856883	-0.00042724609375	\\
0.815377103031917	-9.1552734375e-05	\\
0.815421494206952	3.0517578125e-05	\\
0.815465885381986	0.000274658203125	\\
0.81551027655702	0.000701904296875	\\
0.815554667732055	0.001068115234375	\\
0.815599058907089	0.000701904296875	\\
0.815643450082124	0.00054931640625	\\
0.815687841257158	0.000823974609375	\\
0.815732232432193	0.00128173828125	\\
0.815776623607227	0.001373291015625	\\
0.815821014782261	0.00146484375	\\
0.815865405957296	0.001861572265625	\\
0.81590979713233	0.001708984375	\\
0.815954188307365	0.00128173828125	\\
0.815998579482399	0.001129150390625	\\
0.816042970657433	0.00146484375	\\
0.816087361832468	0.00177001953125	\\
0.816131753007502	0.001617431640625	\\
0.816176144182536	0.001678466796875	\\
0.816220535357571	0.00177001953125	\\
0.816264926532605	0.0015869140625	\\
0.81630931770764	0.001556396484375	\\
0.816353708882674	0.00146484375	\\
0.816398100057709	0.00128173828125	\\
0.816442491232743	0.001373291015625	\\
0.816486882407777	0.001190185546875	\\
0.816531273582812	0.001007080078125	\\
0.816575664757846	0.000640869140625	\\
0.816620055932881	0.00030517578125	\\
0.816664447107915	0.000640869140625	\\
0.816708838282949	0.0008544921875	\\
0.816753229457984	0.00054931640625	\\
0.816797620633018	0.000335693359375	\\
0.816842011808053	0.000244140625	\\
0.816886402983087	0.000244140625	\\
0.816930794158121	0.000579833984375	\\
0.816975185333156	-0.000152587890625	\\
0.81701957650819	-0.00048828125	\\
0.817063967683225	-0.000396728515625	\\
0.817108358858259	-0.00054931640625	\\
0.817152750033293	-0.000762939453125	\\
0.817197141208328	-0.000762939453125	\\
0.817241532383362	-0.0003662109375	\\
0.817285923558397	-0.00018310546875	\\
0.817330314733431	-6.103515625e-05	\\
0.817374705908465	0.00042724609375	\\
0.8174190970835	0.000579833984375	\\
0.817463488258534	0.000457763671875	\\
0.817507879433569	0.0006103515625	\\
0.817552270608603	0.001129150390625	\\
0.817596661783637	0.00140380859375	\\
0.817641052958672	0.00152587890625	\\
0.817685444133706	0.0010986328125	\\
0.817729835308741	0.0013427734375	\\
0.817774226483775	0.00177001953125	\\
0.817818617658809	0.001251220703125	\\
0.817863008833844	0.001739501953125	\\
0.817907400008878	0.001708984375	\\
0.817951791183913	0.001617431640625	\\
0.817996182358947	0.002166748046875	\\
0.818040573533981	0.002288818359375	\\
0.818084964709016	0.002105712890625	\\
0.81812935588405	0.002349853515625	\\
0.818173747059085	0.00238037109375	\\
0.818218138234119	0.00225830078125	\\
0.818262529409153	0.002410888671875	\\
0.818306920584188	0.002532958984375	\\
0.818351311759222	0.002044677734375	\\
0.818395702934257	0.00238037109375	\\
0.818440094109291	0.002166748046875	\\
0.818484485284325	0.001434326171875	\\
0.81852887645936	0.001983642578125	\\
0.818573267634394	0.0015869140625	\\
0.818617658809429	0.001007080078125	\\
0.818662049984463	0.00103759765625	\\
0.818706441159498	0.000946044921875	\\
0.818750832334532	0.0008544921875	\\
0.818795223509566	0.00042724609375	\\
0.818839614684601	0.000152587890625	\\
0.818884005859635	0.0003662109375	\\
0.818928397034669	0.000152587890625	\\
0.818972788209704	0.00030517578125	\\
0.819017179384738	9.1552734375e-05	\\
0.819061570559773	-0.000579833984375	\\
0.819105961734807	-0.00067138671875	\\
0.819150352909842	-9.1552734375e-05	\\
0.819194744084876	3.0517578125e-05	\\
0.81923913525991	-0.00018310546875	\\
0.819283526434945	-0.000396728515625	\\
0.819327917609979	-0.00018310546875	\\
0.819372308785014	0	\\
0.819416699960048	0.0001220703125	\\
0.819461091135082	0.00030517578125	\\
0.819505482310117	0.00042724609375	\\
0.819549873485151	0.00018310546875	\\
0.819594264660186	0.000396728515625	\\
0.81963865583522	0.00030517578125	\\
0.819683047010254	0.00054931640625	\\
0.819727438185289	0.000518798828125	\\
0.819771829360323	0.0008544921875	\\
0.819816220535358	0.001617431640625	\\
0.819860611710392	0.001708984375	\\
0.819905002885426	0.0018310546875	\\
0.819949394060461	0.001983642578125	\\
0.819993785235495	0.002197265625	\\
0.82003817641053	0.00250244140625	\\
0.820082567585564	0.0025634765625	\\
0.820126958760598	0.002471923828125	\\
0.820171349935633	0.00244140625	\\
0.820215741110667	0.00244140625	\\
0.820260132285702	0.00225830078125	\\
0.820304523460736	0.00201416015625	\\
0.82034891463577	0.002288818359375	\\
0.820393305810805	0.002197265625	\\
0.820437696985839	0.002288818359375	\\
0.820482088160874	0.00238037109375	\\
0.820526479335908	0.0020751953125	\\
0.820570870510942	0.001953125	\\
0.820615261685977	0.00177001953125	\\
0.820659652861011	0.001312255859375	\\
0.820704044036046	0.000885009765625	\\
0.82074843521108	0.000579833984375	\\
0.820792826386114	0.000244140625	\\
0.820837217561149	-0.00042724609375	\\
0.820881608736183	-0.000701904296875	\\
0.820925999911218	-0.0010986328125	\\
0.820970391086252	-0.0013427734375	\\
0.821014782261286	-0.001312255859375	\\
0.821059173436321	-0.000946044921875	\\
0.821103564611355	-0.0013427734375	\\
0.82114795578639	-0.001861572265625	\\
0.821192346961424	-0.00225830078125	\\
0.821236738136458	-0.002685546875	\\
0.821281129311493	-0.00335693359375	\\
0.821325520486527	-0.003448486328125	\\
0.821369911661562	-0.003814697265625	\\
0.821414302836596	-0.00439453125	\\
0.821458694011631	-0.004425048828125	\\
0.821503085186665	-0.00433349609375	\\
0.821547476361699	-0.00408935546875	\\
0.821591867536734	-0.00384521484375	\\
0.821636258711768	-0.003814697265625	\\
0.821680649886802	-0.003753662109375	\\
0.821725041061837	-0.003692626953125	\\
0.821769432236871	-0.0035400390625	\\
0.821813823411906	-0.003326416015625	\\
0.82185821458694	-0.003326416015625	\\
0.821902605761974	-0.00372314453125	\\
0.821946996937009	-0.0032958984375	\\
0.821991388112043	-0.00262451171875	\\
0.822035779287078	-0.002777099609375	\\
0.822080170462112	-0.002716064453125	\\
0.822124561637147	-0.00213623046875	\\
0.822168952812181	-0.001617431640625	\\
0.822213343987215	-0.00115966796875	\\
0.82225773516225	-0.000518798828125	\\
0.822302126337284	-0.000274658203125	\\
0.822346517512319	0.00018310546875	\\
0.822390908687353	0.0003662109375	\\
0.822435299862387	0.000335693359375	\\
0.822479691037422	0.000152587890625	\\
0.822524082212456	0.000152587890625	\\
0.822568473387491	-6.103515625e-05	\\
0.822612864562525	-3.0517578125e-05	\\
0.822657255737559	0.00030517578125	\\
0.822701646912594	0.0001220703125	\\
0.822746038087628	0.000152587890625	\\
0.822790429262663	0.00079345703125	\\
0.822834820437697	0.000732421875	\\
0.822879211612731	0.000274658203125	\\
0.822923602787766	0.00054931640625	\\
0.8229679939628	0.0009765625	\\
0.823012385137835	0.000457763671875	\\
0.823056776312869	0.00054931640625	\\
0.823101167487903	9.1552734375e-05	\\
0.823145558662938	-0.0003662109375	\\
0.823189949837972	-0.000152587890625	\\
0.823234341013007	-0.000579833984375	\\
0.823278732188041	-0.0003662109375	\\
0.823323123363075	-0.000701904296875	\\
0.82336751453811	-0.001220703125	\\
0.823411905713144	-0.000946044921875	\\
0.823456296888179	-0.000701904296875	\\
0.823500688063213	-0.0008544921875	\\
0.823545079238247	-0.00152587890625	\\
0.823589470413282	-0.00201416015625	\\
0.823633861588316	-0.001953125	\\
0.823678252763351	-0.00140380859375	\\
0.823722643938385	-0.001953125	\\
0.823767035113419	-0.002410888671875	\\
0.823811426288454	-0.00225830078125	\\
0.823855817463488	-0.002532958984375	\\
0.823900208638523	-0.00225830078125	\\
0.823944599813557	-0.002288818359375	\\
0.823988990988591	-0.002197265625	\\
0.824033382163626	-0.002349853515625	\\
0.82407777333866	-0.0025634765625	\\
0.824122164513695	-0.0025634765625	\\
0.824166555688729	-0.002349853515625	\\
0.824210946863764	-0.00213623046875	\\
0.824255338038798	-0.001556396484375	\\
0.824299729213832	-0.001434326171875	\\
0.824344120388867	-0.001556396484375	\\
0.824388511563901	-0.001312255859375	\\
0.824432902738936	-0.00091552734375	\\
0.82447729391397	-0.001220703125	\\
0.824521685089004	-0.001495361328125	\\
0.824566076264039	-0.001617431640625	\\
0.824610467439073	-0.00177001953125	\\
0.824654858614107	-0.001251220703125	\\
0.824699249789142	-0.001007080078125	\\
0.824743640964176	-0.000823974609375	\\
0.824788032139211	-0.00091552734375	\\
0.824832423314245	-0.00048828125	\\
0.82487681448928	-0.000518798828125	\\
0.824921205664314	-0.00067138671875	\\
0.824965596839348	-0.0003662109375	\\
0.825009988014383	-0.00042724609375	\\
0.825054379189417	-0.0001220703125	\\
0.825098770364452	-0.000213623046875	\\
0.825143161539486	-0.000244140625	\\
0.82518755271452	-0.00079345703125	\\
0.825231943889555	-0.00103759765625	\\
0.825276335064589	-0.00128173828125	\\
0.825320726239624	-0.0009765625	\\
0.825365117414658	-0.000762939453125	\\
0.825409508589692	-0.000946044921875	\\
0.825453899764727	-0.001007080078125	\\
0.825498290939761	-0.001190185546875	\\
0.825542682114796	-0.001068115234375	\\
0.82558707328983	-0.001251220703125	\\
0.825631464464864	-0.001495361328125	\\
0.825675855639899	-0.00164794921875	\\
0.825720246814933	-0.00152587890625	\\
0.825764637989968	-0.00140380859375	\\
0.825809029165002	-0.001373291015625	\\
0.825853420340036	-0.001373291015625	\\
0.825897811515071	-0.00079345703125	\\
0.825942202690105	-0.00103759765625	\\
0.82598659386514	-0.000579833984375	\\
0.826030985040174	6.103515625e-05	\\
0.826075376215208	-0.0001220703125	\\
0.826119767390243	0.000762939453125	\\
0.826164158565277	0.000640869140625	\\
0.826208549740312	0.000274658203125	\\
0.826252940915346	0.00030517578125	\\
0.82629733209038	0.00042724609375	\\
0.826341723265415	0.000732421875	\\
0.826386114440449	0.000823974609375	\\
0.826430505615484	0.001129150390625	\\
0.826474896790518	0.00146484375	\\
0.826519287965552	0.001739501953125	\\
0.826563679140587	0.00164794921875	\\
0.826608070315621	0.001678466796875	\\
0.826652461490656	0.001953125	\\
0.82669685266569	0.001861572265625	\\
0.826741243840724	0.001556396484375	\\
0.826785635015759	0.001129150390625	\\
0.826830026190793	0.0010986328125	\\
0.826874417365828	0.000823974609375	\\
0.826918808540862	0.0003662109375	\\
0.826963199715896	0.000823974609375	\\
0.827007590890931	0.000213623046875	\\
0.827051982065965	9.1552734375e-05	\\
0.827096373241	0.0003662109375	\\
0.827140764416034	0.000457763671875	\\
0.827185155591069	0.00048828125	\\
0.827229546766103	0.00018310546875	\\
0.827273937941137	-6.103515625e-05	\\
0.827318329116172	-6.103515625e-05	\\
0.827362720291206	0.0001220703125	\\
0.82740711146624	-9.1552734375e-05	\\
0.827451502641275	-0.000640869140625	\\
0.827495893816309	-0.000823974609375	\\
0.827540284991344	-0.000885009765625	\\
0.827584676166378	-0.001220703125	\\
0.827629067341413	-0.001373291015625	\\
0.827673458516447	-0.00152587890625	\\
0.827717849691481	-0.001495361328125	\\
0.827762240866516	-0.001251220703125	\\
0.82780663204155	-0.001495361328125	\\
0.827851023216585	-0.0013427734375	\\
0.827895414391619	-0.000640869140625	\\
0.827939805566653	-0.000701904296875	\\
0.827984196741688	-0.000946044921875	\\
0.828028587916722	-0.00115966796875	\\
0.828072979091757	-0.00103759765625	\\
0.828117370266791	-0.00079345703125	\\
0.828161761441825	-0.000732421875	\\
0.82820615261686	-0.000762939453125	\\
0.828250543791894	-0.000701904296875	\\
0.828294934966929	9.1552734375e-05	\\
0.828339326141963	0.000457763671875	\\
0.828383717316997	0.00030517578125	\\
0.828428108492032	0.00091552734375	\\
0.828472499667066	0.000823974609375	\\
0.828516890842101	0.000823974609375	\\
0.828561282017135	0.001739501953125	\\
0.828605673192169	0.00152587890625	\\
0.828650064367204	0.00146484375	\\
0.828694455542238	0.001678466796875	\\
0.828738846717273	0.001861572265625	\\
0.828783237892307	0.001922607421875	\\
0.828827629067341	0.001678466796875	\\
0.828872020242376	0.001983642578125	\\
0.82891641141741	0.001983642578125	\\
0.828960802592445	0.001861572265625	\\
0.829005193767479	0.002288818359375	\\
0.829049584942513	0.00244140625	\\
0.829093976117548	0.00213623046875	\\
0.829138367292582	0.001983642578125	\\
0.829182758467617	0.00177001953125	\\
0.829227149642651	0.001739501953125	\\
0.829271540817685	0.001922607421875	\\
0.82931593199272	0.00146484375	\\
0.829360323167754	0.001190185546875	\\
0.829404714342789	0.000762939453125	\\
0.829449105517823	0.000518798828125	\\
0.829493496692857	0.00048828125	\\
0.829537887867892	0.000213623046875	\\
0.829582279042926	6.103515625e-05	\\
0.829626670217961	-0.000213623046875	\\
0.829671061392995	-0.000640869140625	\\
0.829715452568029	-0.001251220703125	\\
0.829759843743064	-0.001678466796875	\\
0.829804234918098	-0.00213623046875	\\
0.829848626093133	-0.00262451171875	\\
0.829893017268167	-0.0025634765625	\\
0.829937408443202	-0.002532958984375	\\
0.829981799618236	-0.00311279296875	\\
0.83002619079327	-0.002532958984375	\\
0.830070581968305	-0.00225830078125	\\
0.830114973143339	-0.002349853515625	\\
0.830159364318374	-0.002227783203125	\\
0.830203755493408	-0.002288818359375	\\
0.830248146668442	-0.002044677734375	\\
0.830292537843477	-0.00189208984375	\\
0.830336929018511	-0.00164794921875	\\
0.830381320193545	-0.00128173828125	\\
0.83042571136858	-0.001495361328125	\\
0.830470102543614	-0.001312255859375	\\
0.830514493718649	-0.0008544921875	\\
0.830558884893683	-0.000732421875	\\
0.830603276068718	-0.000579833984375	\\
0.830647667243752	-0.00018310546875	\\
0.830692058418786	0.00048828125	\\
0.830736449593821	0.0008544921875	\\
0.830780840768855	0.001190185546875	\\
0.83082523194389	0.00115966796875	\\
0.830869623118924	0.00091552734375	\\
0.830914014293958	0.001220703125	\\
0.830958405468993	0.001556396484375	\\
0.831002796644027	0.002044677734375	\\
0.831047187819062	0.001922607421875	\\
0.831091578994096	0.00164794921875	\\
0.83113597016913	0.001953125	\\
0.831180361344165	0.002105712890625	\\
0.831224752519199	0.00225830078125	\\
0.831269143694234	0.00262451171875	\\
0.831313534869268	0.002593994140625	\\
0.831357926044302	0.002227783203125	\\
0.831402317219337	0.00201416015625	\\
0.831446708394371	0.001983642578125	\\
0.831491099569406	0.002044677734375	\\
0.83153549074444	0.00201416015625	\\
0.831579881919474	0.0015869140625	\\
0.831624273094509	0.00146484375	\\
0.831668664269543	0.00103759765625	\\
0.831713055444578	0.00042724609375	\\
0.831757446619612	0.00042724609375	\\
0.831801837794646	6.103515625e-05	\\
0.831846228969681	-0.000335693359375	\\
0.831890620144715	-0.0001220703125	\\
0.83193501131975	-0.000335693359375	\\
0.831979402494784	-0.000518798828125	\\
0.832023793669818	-0.000579833984375	\\
0.832068184844853	-0.0008544921875	\\
0.832112576019887	-0.001312255859375	\\
0.832156967194922	-0.001708984375	\\
0.832201358369956	-0.00201416015625	\\
0.83224574954499	-0.002166748046875	\\
0.832290140720025	-0.002197265625	\\
0.832334531895059	-0.00244140625	\\
0.832378923070094	-0.001922607421875	\\
0.832423314245128	-0.001708984375	\\
0.832467705420162	-0.00213623046875	\\
0.832512096595197	-0.0020751953125	\\
0.832556487770231	-0.001617431640625	\\
0.832600878945266	-0.001312255859375	\\
0.8326452701203	-0.0013427734375	\\
0.832689661295335	-0.000823974609375	\\
0.832734052470369	-0.000732421875	\\
0.832778443645403	-0.00054931640625	\\
0.832822834820438	-9.1552734375e-05	\\
0.832867225995472	-0.000213623046875	\\
0.832911617170507	9.1552734375e-05	\\
0.832956008345541	0.000396728515625	\\
0.833000399520575	0.0008544921875	\\
0.83304479069561	0.001007080078125	\\
0.833089181870644	0.001007080078125	\\
0.833133573045678	0.001220703125	\\
0.833177964220713	0.001739501953125	\\
0.833222355395747	0.00213623046875	\\
0.833266746570782	0.001861572265625	\\
0.833311137745816	0.001922607421875	\\
0.833355528920851	0.002166748046875	\\
0.833399920095885	0.001983642578125	\\
0.833444311270919	0.0020751953125	\\
0.833488702445954	0.002166748046875	\\
0.833533093620988	0.002227783203125	\\
0.833577484796023	0.0023193359375	\\
0.833621875971057	0.002105712890625	\\
0.833666267146091	0.002044677734375	\\
0.833710658321126	0.001953125	\\
0.83375504949616	0.00140380859375	\\
0.833799440671195	0.0013427734375	\\
0.833843831846229	0.001373291015625	\\
0.833888223021263	0.0009765625	\\
0.833932614196298	0.0006103515625	\\
0.833977005371332	0.000701904296875	\\
0.834021396546367	0.000396728515625	\\
0.834065787721401	6.103515625e-05	\\
0.834110178896435	3.0517578125e-05	\\
0.83415457007147	-0.000213623046875	\\
0.834198961246504	-0.000579833984375	\\
0.834243352421539	-0.000640869140625	\\
0.834287743596573	-0.00048828125	\\
0.834332134771607	-0.00054931640625	\\
0.834376525946642	-0.000640869140625	\\
0.834420917121676	-0.000579833984375	\\
0.834465308296711	-0.0006103515625	\\
0.834509699471745	-0.00018310546875	\\
0.834554090646779	0.000274658203125	\\
0.834598481821814	0.000518798828125	\\
0.834642872996848	0.000244140625	\\
0.834687264171883	-0.000274658203125	\\
0.834731655346917	-0.0001220703125	\\
0.834776046521951	0.000213623046875	\\
0.834820437696986	0.0003662109375	\\
0.83486482887202	0.00042724609375	\\
0.834909220047055	0.0008544921875	\\
0.834953611222089	0.000946044921875	\\
0.834998002397123	0.001190185546875	\\
0.835042393572158	0.001251220703125	\\
0.835086784747192	0.001190185546875	\\
0.835131175922227	0.001556396484375	\\
0.835175567097261	0.001861572265625	\\
0.835219958272295	0.00177001953125	\\
0.83526434944733	0.001556396484375	\\
0.835308740622364	0.001800537109375	\\
0.835353131797399	0.001953125	\\
0.835397522972433	0.0015869140625	\\
0.835441914147467	0.00128173828125	\\
0.835486305322502	0.00103759765625	\\
0.835530696497536	0.00079345703125	\\
0.835575087672571	0.000396728515625	\\
0.835619478847605	0.000518798828125	\\
0.83566387002264	0.000762939453125	\\
0.835708261197674	0.000946044921875	\\
0.835752652372708	0.000518798828125	\\
0.835797043547743	0.000457763671875	\\
0.835841434722777	9.1552734375e-05	\\
0.835885825897811	-0.000732421875	\\
0.835930217072846	-0.000640869140625	\\
0.83597460824788	-0.0008544921875	\\
0.836018999422915	-0.0010986328125	\\
0.836063390597949	-0.00128173828125	\\
0.836107781772984	-0.00091552734375	\\
0.836152172948018	-0.00091552734375	\\
0.836196564123052	-0.00128173828125	\\
0.836240955298087	-0.001190185546875	\\
0.836285346473121	-0.001129150390625	\\
0.836329737648156	-0.000946044921875	\\
0.83637412882319	-0.001129150390625	\\
0.836418519998224	-0.001129150390625	\\
0.836462911173259	-0.00054931640625	\\
0.836507302348293	-0.000213623046875	\\
0.836551693523328	-0.000274658203125	\\
0.836596084698362	-0.00048828125	\\
0.836640475873396	-0.00030517578125	\\
0.836684867048431	3.0517578125e-05	\\
0.836729258223465	-9.1552734375e-05	\\
0.8367736493985	0.0003662109375	\\
0.836818040573534	0.00079345703125	\\
0.836862431748568	0.000579833984375	\\
0.836906822923603	0.000732421875	\\
0.836951214098637	0.000701904296875	\\
0.836995605273672	0.000946044921875	\\
0.837039996448706	0.00152587890625	\\
0.83708438762374	0.0015869140625	\\
0.837128778798775	0.001373291015625	\\
0.837173169973809	0.001312255859375	\\
0.837217561148844	0.001617431640625	\\
0.837261952323878	0.001556396484375	\\
0.837306343498912	0.001373291015625	\\
0.837350734673947	0.001556396484375	\\
0.837395125848981	0.001800537109375	\\
0.837439517024016	0.001800537109375	\\
0.83748390819905	0.001495361328125	\\
0.837528299374084	0.001373291015625	\\
0.837572690549119	0.001129150390625	\\
0.837617081724153	0.0015869140625	\\
0.837661472899188	0.001495361328125	\\
0.837705864074222	0.000823974609375	\\
0.837750255249256	0.000640869140625	\\
0.837794646424291	0.000213623046875	\\
0.837839037599325	0	\\
0.83788342877436	9.1552734375e-05	\\
0.837927819949394	-0.0003662109375	\\
0.837972211124428	-0.000732421875	\\
0.838016602299463	-0.00067138671875	\\
0.838060993474497	-0.00067138671875	\\
0.838105384649532	-0.001007080078125	\\
0.838149775824566	-0.0015869140625	\\
0.8381941669996	-0.001434326171875	\\
0.838238558174635	-0.001617431640625	\\
0.838282949349669	-0.00213623046875	\\
0.838327340524704	-0.0020751953125	\\
0.838371731699738	-0.002227783203125	\\
0.838416122874773	-0.002044677734375	\\
0.838460514049807	-0.00201416015625	\\
0.838504905224841	-0.002593994140625	\\
0.838549296399876	-0.002532958984375	\\
0.83859368757491	-0.002532958984375	\\
0.838638078749945	-0.00311279296875	\\
0.838682469924979	-0.0030517578125	\\
0.838726861100013	-0.00244140625	\\
0.838771252275048	-0.00225830078125	\\
0.838815643450082	-0.002410888671875	\\
0.838860034625116	-0.002349853515625	\\
0.838904425800151	-0.0020751953125	\\
0.838948816975185	-0.001800537109375	\\
0.83899320815022	-0.001495361328125	\\
0.839037599325254	-0.000823974609375	\\
0.839081990500289	-0.0003662109375	\\
0.839126381675323	-0.000213623046875	\\
0.839170772850357	0.000274658203125	\\
0.839215164025392	0.000518798828125	\\
0.839259555200426	0.00030517578125	\\
0.839303946375461	0.000640869140625	\\
0.839348337550495	0.000396728515625	\\
0.839392728725529	9.1552734375e-05	\\
0.839437119900564	0.00079345703125	\\
0.839481511075598	0.00115966796875	\\
0.839525902250633	0.000885009765625	\\
0.839570293425667	0.0009765625	\\
0.839614684600701	0.001068115234375	\\
0.839659075775736	0.001495361328125	\\
0.83970346695077	0.001708984375	\\
0.839747858125805	0.001434326171875	\\
0.839792249300839	0.001312255859375	\\
0.839836640475873	0.00128173828125	\\
0.839881031650908	0.001251220703125	\\
0.839925422825942	0.000823974609375	\\
0.839969814000977	0.00048828125	\\
0.840014205176011	0.000213623046875	\\
0.840058596351045	-6.103515625e-05	\\
0.84010298752608	-0.00048828125	\\
0.840147378701114	-0.000762939453125	\\
0.840191769876149	-0.001129150390625	\\
0.840236161051183	-0.00177001953125	\\
0.840280552226217	-0.001708984375	\\
0.840324943401252	-0.001708984375	\\
0.840369334576286	-0.00201416015625	\\
0.840413725751321	-0.00225830078125	\\
0.840458116926355	-0.002227783203125	\\
0.840502508101389	-0.00262451171875	\\
0.840546899276424	-0.002716064453125	\\
0.840591290451458	-0.0029296875	\\
0.840635681626493	-0.00323486328125	\\
0.840680072801527	-0.003143310546875	\\
0.840724463976561	-0.0032958984375	\\
0.840768855151596	-0.003021240234375	\\
0.84081324632663	-0.00299072265625	\\
0.840857637501665	-0.0032958984375	\\
0.840902028676699	-0.003265380859375	\\
0.840946419851733	-0.00335693359375	\\
0.840990811026768	-0.003326416015625	\\
0.841035202201802	-0.00286865234375	\\
0.841079593376837	-0.002410888671875	\\
0.841123984551871	-0.002044677734375	\\
0.841168375726906	-0.002197265625	\\
0.84121276690194	-0.002655029296875	\\
0.841257158076974	-0.002349853515625	\\
0.841301549252009	-0.002288818359375	\\
0.841345940427043	-0.00225830078125	\\
0.841390331602078	-0.0020751953125	\\
0.841434722777112	-0.001495361328125	\\
0.841479113952146	-0.001190185546875	\\
0.841523505127181	-0.001617431640625	\\
0.841567896302215	-0.001434326171875	\\
0.841612287477249	-0.000640869140625	\\
0.841656678652284	-0.000885009765625	\\
0.841701069827318	-0.0013427734375	\\
0.841745461002353	-0.000823974609375	\\
0.841789852177387	-0.001007080078125	\\
0.841834243352422	-0.001312255859375	\\
0.841878634527456	-0.001312255859375	\\
0.84192302570249	-0.001068115234375	\\
0.841967416877525	-0.00115966796875	\\
0.842011808052559	-0.001556396484375	\\
0.842056199227594	-0.001678466796875	\\
0.842100590402628	-0.00189208984375	\\
0.842144981577662	-0.001678466796875	\\
0.842189372752697	-0.001678466796875	\\
0.842233763927731	-0.00177001953125	\\
0.842278155102766	-0.0020751953125	\\
0.8423225462778	-0.002197265625	\\
0.842366937452834	-0.002410888671875	\\
0.842411328627869	-0.00274658203125	\\
0.842455719802903	-0.0028076171875	\\
0.842500110977938	-0.00250244140625	\\
0.842544502152972	-0.001983642578125	\\
0.842588893328006	-0.00250244140625	\\
0.842633284503041	-0.0023193359375	\\
0.842677675678075	-0.002197265625	\\
0.84272206685311	-0.00189208984375	\\
0.842766458028144	-0.001556396484375	\\
0.842810849203178	-0.00164794921875	\\
0.842855240378213	-0.001220703125	\\
0.842899631553247	-0.00103759765625	\\
0.842944022728282	-0.00091552734375	\\
0.842988413903316	-0.00103759765625	\\
0.84303280507835	-0.00079345703125	\\
0.843077196253385	-0.000732421875	\\
0.843121587428419	-0.000457763671875	\\
0.843165978603454	-0.00054931640625	\\
0.843210369778488	-0.00079345703125	\\
0.843254760953522	-0.0003662109375	\\
0.843299152128557	-0.000518798828125	\\
0.843343543303591	-0.00030517578125	\\
0.843387934478626	-0.000152587890625	\\
0.84343232565366	-9.1552734375e-05	\\
0.843476716828694	-0.00054931640625	\\
0.843521108003729	-0.00030517578125	\\
0.843565499178763	-6.103515625e-05	\\
0.843609890353798	-0.000152587890625	\\
0.843654281528832	-0.00042724609375	\\
0.843698672703866	-0.000274658203125	\\
0.843743063878901	-0.000213623046875	\\
0.843787455053935	-0.00067138671875	\\
0.84383184622897	-0.00079345703125	\\
0.843876237404004	-0.0008544921875	\\
0.843920628579038	-0.001312255859375	\\
0.843965019754073	-0.001495361328125	\\
0.844009410929107	-0.001495361328125	\\
0.844053802104142	-0.001312255859375	\\
0.844098193279176	-0.001495361328125	\\
0.844142584454211	-0.001739501953125	\\
0.844186975629245	-0.0015869140625	\\
0.844231366804279	-0.00177001953125	\\
0.844275757979314	-0.00213623046875	\\
0.844320149154348	-0.002044677734375	\\
0.844364540329383	-0.001953125	\\
0.844408931504417	-0.00201416015625	\\
0.844453322679451	-0.001861572265625	\\
0.844497713854486	-0.00164794921875	\\
0.84454210502952	-0.001953125	\\
0.844586496204555	-0.00164794921875	\\
0.844630887379589	-0.001556396484375	\\
0.844675278554623	-0.001678466796875	\\
0.844719669729658	-0.001129150390625	\\
0.844764060904692	-0.000946044921875	\\
0.844808452079727	-0.0008544921875	\\
0.844852843254761	-0.000823974609375	\\
0.844897234429795	-0.001220703125	\\
0.84494162560483	-0.00091552734375	\\
0.844986016779864	-0.000396728515625	\\
0.845030407954899	-0.0008544921875	\\
0.845074799129933	-0.000885009765625	\\
0.845119190304967	-0.000457763671875	\\
0.845163581480002	-0.00030517578125	\\
0.845207972655036	0.000244140625	\\
0.845252363830071	0.000396728515625	\\
0.845296755005105	0.00048828125	\\
0.845341146180139	0.00054931640625	\\
0.845385537355174	0.000640869140625	\\
0.845429928530208	0.000762939453125	\\
0.845474319705243	0.000640869140625	\\
0.845518710880277	0.0006103515625	\\
0.845563102055311	0.00115966796875	\\
0.845607493230346	0.00128173828125	\\
0.84565188440538	0.001312255859375	\\
0.845696275580415	0.001220703125	\\
0.845740666755449	0.000579833984375	\\
0.845785057930483	0.000518798828125	\\
0.845829449105518	0.000946044921875	\\
0.845873840280552	0.000823974609375	\\
0.845918231455587	0.000579833984375	\\
0.845962622630621	0.000396728515625	\\
0.846007013805655	0.00030517578125	\\
0.84605140498069	0.000213623046875	\\
0.846095796155724	-6.103515625e-05	\\
0.846140187330759	-0.000457763671875	\\
0.846184578505793	-0.000640869140625	\\
0.846228969680828	-0.0008544921875	\\
0.846273360855862	-0.001220703125	\\
0.846317752030896	-0.00152587890625	\\
0.846362143205931	-0.001861572265625	\\
0.846406534380965	-0.0015869140625	\\
0.846450925555999	-0.001373291015625	\\
0.846495316731034	-0.002105712890625	\\
0.846539707906068	-0.001739501953125	\\
0.846584099081103	-0.0023193359375	\\
0.846628490256137	-0.003082275390625	\\
0.846672881431171	-0.003204345703125	\\
0.846717272606206	-0.003662109375	\\
0.84676166378124	-0.00384521484375	\\
0.846806054956275	-0.00372314453125	\\
0.846850446131309	-0.00390625	\\
0.846894837306344	-0.0040283203125	\\
0.846939228481378	-0.003753662109375	\\
0.846983619656412	-0.00408935546875	\\
0.847028010831447	-0.003936767578125	\\
0.847072402006481	-0.0040283203125	\\
0.847116793181516	-0.00390625	\\
0.84716118435655	-0.004119873046875	\\
0.847205575531584	-0.004364013671875	\\
0.847249966706619	-0.00384521484375	\\
0.847294357881653	-0.003814697265625	\\
0.847338749056687	-0.003936767578125	\\
0.847383140231722	-0.003692626953125	\\
0.847427531406756	-0.003265380859375	\\
0.847471922581791	-0.0029296875	\\
0.847516313756825	-0.00250244140625	\\
0.84756070493186	-0.0020751953125	\\
0.847605096106894	-0.00177001953125	\\
0.847649487281928	-0.001678466796875	\\
0.847693878456963	-0.001617431640625	\\
0.847738269631997	-0.00152587890625	\\
0.847782660807032	-0.001617431640625	\\
0.847827051982066	-0.001678466796875	\\
0.8478714431571	-0.0013427734375	\\
0.847915834332135	-0.00115966796875	\\
0.847960225507169	-0.00115966796875	\\
0.848004616682204	-0.00079345703125	\\
0.848049007857238	-0.000579833984375	\\
0.848093399032272	-0.00048828125	\\
0.848137790207307	-0.000213623046875	\\
0.848182181382341	0	\\
0.848226572557376	0.000152587890625	\\
0.84827096373241	-0.0001220703125	\\
0.848315354907444	-0.000213623046875	\\
0.848359746082479	-0.0001220703125	\\
0.848404137257513	-0.000396728515625	\\
0.848448528432548	-0.00048828125	\\
0.848492919607582	-0.000762939453125	\\
0.848537310782616	-0.000823974609375	\\
0.848581701957651	-0.00091552734375	\\
0.848626093132685	-0.001312255859375	\\
0.84867048430772	-0.001251220703125	\\
0.848714875482754	-0.001434326171875	\\
0.848759266657788	-0.00189208984375	\\
0.848803657832823	-0.0013427734375	\\
0.848848049007857	-0.00140380859375	\\
0.848892440182892	-0.001861572265625	\\
0.848936831357926	-0.001739501953125	\\
0.84898122253296	-0.00177001953125	\\
0.849025613707995	-0.00177001953125	\\
0.849070004883029	-0.0020751953125	\\
0.849114396058064	-0.00201416015625	\\
0.849158787233098	-0.002197265625	\\
0.849203178408132	-0.00201416015625	\\
0.849247569583167	-0.0018310546875	\\
0.849291960758201	-0.001617431640625	\\
0.849336351933236	-0.00152587890625	\\
0.84938074310827	-0.001739501953125	\\
0.849425134283304	-0.00177001953125	\\
0.849469525458339	-0.00079345703125	\\
0.849513916633373	-0.001007080078125	\\
0.849558307808408	-0.0013427734375	\\
0.849602698983442	-0.0008544921875	\\
0.849647090158477	-0.00103759765625	\\
0.849691481333511	-0.001129150390625	\\
0.849735872508545	-0.00103759765625	\\
0.84978026368358	-0.000762939453125	\\
0.849824654858614	-0.0006103515625	\\
0.849869046033649	-0.00091552734375	\\
0.849913437208683	-0.001251220703125	\\
0.849957828383717	-0.00152587890625	\\
0.850002219558752	-0.000885009765625	\\
0.850046610733786	-0.00067138671875	\\
0.85009100190882	-0.000762939453125	\\
0.850135393083855	-0.001251220703125	\\
0.850179784258889	-0.000823974609375	\\
0.850224175433924	-0.0003662109375	\\
0.850268566608958	-0.000946044921875	\\
0.850312957783993	-0.000732421875	\\
0.850357348959027	-0.000701904296875	\\
0.850401740134061	-0.000946044921875	\\
0.850446131309096	-0.00079345703125	\\
0.85049052248413	-0.000885009765625	\\
0.850534913659165	-0.001220703125	\\
0.850579304834199	-0.001068115234375	\\
0.850623696009233	-0.00115966796875	\\
0.850668087184268	-0.00128173828125	\\
0.850712478359302	-0.001251220703125	\\
0.850756869534337	-0.001373291015625	\\
0.850801260709371	-0.001617431640625	\\
0.850845651884405	-0.00177001953125	\\
0.85089004305944	-0.0015869140625	\\
0.850934434234474	-0.001251220703125	\\
0.850978825409509	-0.0010986328125	\\
0.851023216584543	-0.001129150390625	\\
0.851067607759577	-0.001190185546875	\\
0.851111998934612	-0.000946044921875	\\
0.851156390109646	-0.00042724609375	\\
0.851200781284681	-0.00054931640625	\\
0.851245172459715	-0.00048828125	\\
0.851289563634749	-0.0006103515625	\\
0.851333954809784	-0.000732421875	\\
0.851378345984818	-0.000274658203125	\\
0.851422737159853	-0.000213623046875	\\
0.851467128334887	-6.103515625e-05	\\
0.851511519509921	9.1552734375e-05	\\
0.851555910684956	0.00048828125	\\
0.85160030185999	0.000335693359375	\\
0.851644693035025	0.000335693359375	\\
0.851689084210059	0.00054931640625	\\
0.851733475385093	0.00067138671875	\\
0.851777866560128	0.000579833984375	\\
0.851822257735162	0.0006103515625	\\
0.851866648910197	0.00103759765625	\\
0.851911040085231	0.00091552734375	\\
0.851955431260265	0.00103759765625	\\
0.8519998224353	0.001007080078125	\\
0.852044213610334	0.000762939453125	\\
0.852088604785369	0.00128173828125	\\
0.852132995960403	0.0010986328125	\\
0.852177387135437	0.000946044921875	\\
0.852221778310472	0.00079345703125	\\
0.852266169485506	0.00042724609375	\\
0.852310560660541	0.000244140625	\\
0.852354951835575	0.000518798828125	\\
0.852399343010609	0.00018310546875	\\
0.852443734185644	-0.00054931640625	\\
0.852488125360678	-0.000396728515625	\\
0.852532516535713	-0.000213623046875	\\
0.852576907710747	-0.000213623046875	\\
0.852621298885782	6.103515625e-05	\\
0.852665690060816	0.000244140625	\\
0.85271008123585	0.0001220703125	\\
0.852754472410885	0.000396728515625	\\
0.852798863585919	0.000518798828125	\\
0.852843254760954	0	\\
0.852887645935988	-0.00030517578125	\\
0.852932037111022	-0.000732421875	\\
0.852976428286057	-0.0008544921875	\\
0.853020819461091	-0.000640869140625	\\
0.853065210636126	-0.00030517578125	\\
0.85310960181116	-0.000213623046875	\\
0.853153992986194	-0.000396728515625	\\
0.853198384161229	-0.000152587890625	\\
0.853242775336263	3.0517578125e-05	\\
0.853287166511298	-0.0001220703125	\\
0.853331557686332	-0.00018310546875	\\
0.853375948861366	-0.0001220703125	\\
0.853420340036401	-0.000274658203125	\\
0.853464731211435	0.00018310546875	\\
0.85350912238647	0.0003662109375	\\
0.853553513561504	0	\\
0.853597904736538	-0.000152587890625	\\
0.853642295911573	-3.0517578125e-05	\\
0.853686687086607	0.000274658203125	\\
0.853731078261642	0.0001220703125	\\
0.853775469436676	0.00030517578125	\\
0.85381986061171	0.000885009765625	\\
0.853864251786745	0.000823974609375	\\
0.853908642961779	0.0006103515625	\\
0.853953034136814	0.00030517578125	\\
0.853997425311848	0.0001220703125	\\
0.854041816486882	0.000335693359375	\\
0.854086207661917	0.00067138671875	\\
0.854130598836951	0.0009765625	\\
0.854174990011986	0.000518798828125	\\
0.85421938118702	0.00054931640625	\\
0.854263772362054	0.000335693359375	\\
0.854308163537089	0.000274658203125	\\
0.854352554712123	0.00048828125	\\
0.854396945887158	0.00030517578125	\\
0.854441337062192	-9.1552734375e-05	\\
0.854485728237226	6.103515625e-05	\\
0.854530119412261	0.00018310546875	\\
0.854574510587295	-3.0517578125e-05	\\
0.85461890176233	0	\\
0.854663292937364	6.103515625e-05	\\
0.854707684112399	0.000244140625	\\
0.854752075287433	0.0001220703125	\\
0.854796466462467	-0.000213623046875	\\
0.854840857637502	-0.00030517578125	\\
0.854885248812536	-0.00030517578125	\\
0.85492963998757	-0.00042724609375	\\
0.854974031162605	-0.0001220703125	\\
0.855018422337639	-0.00042724609375	\\
0.855062813512674	-0.000640869140625	\\
0.855107204687708	-0.001007080078125	\\
0.855151595862742	-0.00115966796875	\\
0.855195987037777	-0.00079345703125	\\
0.855240378212811	-0.001007080078125	\\
0.855284769387846	-0.001312255859375	\\
0.85532916056288	-0.001220703125	\\
0.855373551737915	-0.000732421875	\\
0.855417942912949	-0.000946044921875	\\
0.855462334087983	-0.001068115234375	\\
0.855506725263018	-0.001190185546875	\\
0.855551116438052	-0.00091552734375	\\
0.855595507613087	-0.000732421875	\\
0.855639898788121	-0.001129150390625	\\
0.855684289963155	-0.0009765625	\\
0.85572868113819	-0.001068115234375	\\
0.855773072313224	-0.00140380859375	\\
0.855817463488258	-0.0010986328125	\\
0.855861854663293	-0.000457763671875	\\
0.855906245838327	3.0517578125e-05	\\
0.855950637013362	0.000579833984375	\\
0.855995028188396	0.000640869140625	\\
0.856039419363431	0.0008544921875	\\
0.856083810538465	0.0006103515625	\\
0.856128201713499	0.00054931640625	\\
0.856172592888534	0.000701904296875	\\
0.856216984063568	0.00091552734375	\\
0.856261375238603	0.00140380859375	\\
0.856305766413637	0.00146484375	\\
0.856350157588671	0.001129150390625	\\
0.856394548763706	0.001129150390625	\\
0.85643893993874	0.001220703125	\\
0.856483331113775	0.00146484375	\\
0.856527722288809	0.001434326171875	\\
0.856572113463843	0.001953125	\\
0.856616504638878	0.001861572265625	\\
0.856660895813912	0.001922607421875	\\
0.856705286988947	0.0020751953125	\\
0.856749678163981	0.001708984375	\\
0.856794069339015	0.00189208984375	\\
0.85683846051405	0.001678466796875	\\
0.856882851689084	0.00128173828125	\\
0.856927242864119	0.00079345703125	\\
0.856971634039153	0.000396728515625	\\
0.857016025214187	0.0001220703125	\\
0.857060416389222	-6.103515625e-05	\\
0.857104807564256	0.000244140625	\\
0.857149198739291	0.00048828125	\\
0.857193589914325	0.000457763671875	\\
0.857237981089359	0.000152587890625	\\
0.857282372264394	-0.0001220703125	\\
0.857326763439428	-0.00018310546875	\\
0.857371154614463	0.000396728515625	\\
0.857415545789497	0.0008544921875	\\
0.857459936964531	0.000518798828125	\\
0.857504328139566	0.000335693359375	\\
0.8575487193146	0.000518798828125	\\
0.857593110489635	0.000518798828125	\\
0.857637501664669	0.00042724609375	\\
0.857681892839703	0.00018310546875	\\
0.857726284014738	-3.0517578125e-05	\\
0.857770675189772	0.000518798828125	\\
0.857815066364807	0.001068115234375	\\
0.857859457539841	0.00054931640625	\\
0.857903848714875	0.000732421875	\\
0.85794823988991	0.00128173828125	\\
0.857992631064944	0.000732421875	\\
0.858037022239979	0.000640869140625	\\
0.858081413415013	0.000701904296875	\\
0.858125804590048	0.000640869140625	\\
0.858170195765082	0.0010986328125	\\
0.858214586940116	0.001312255859375	\\
0.858258978115151	0.001007080078125	\\
0.858303369290185	0.00042724609375	\\
0.85834776046522	0.000579833984375	\\
0.858392151640254	0.00048828125	\\
0.858436542815288	0.000244140625	\\
0.858480933990323	0.0001220703125	\\
0.858525325165357	0.000518798828125	\\
0.858569716340391	0.00091552734375	\\
0.858614107515426	0.00103759765625	\\
0.85865849869046	0.0013427734375	\\
0.858702889865495	0.001312255859375	\\
0.858747281040529	0.00177001953125	\\
0.858791672215564	0.001800537109375	\\
0.858836063390598	0.00152587890625	\\
0.858880454565632	0.001953125	\\
0.858924845740667	0.00189208984375	\\
0.858969236915701	0.00128173828125	\\
0.859013628090736	0.001251220703125	\\
0.85905801926577	0.001617431640625	\\
0.859102410440804	0.001434326171875	\\
0.859146801615839	0.00091552734375	\\
0.859191192790873	0.001007080078125	\\
0.859235583965908	0.001373291015625	\\
0.859279975140942	0.001739501953125	\\
0.859324366315976	0.00152587890625	\\
0.859368757491011	0.001861572265625	\\
0.859413148666045	0.002410888671875	\\
0.85945753984108	0.002532958984375	\\
0.859501931016114	0.00286865234375	\\
0.859546322191148	0.002960205078125	\\
0.859590713366183	0.0028076171875	\\
0.859635104541217	0.002349853515625	\\
0.859679495716252	0.00189208984375	\\
0.859723886891286	0.002593994140625	\\
0.85976827806632	0.00286865234375	\\
0.859812669241355	0.002532958984375	\\
0.859857060416389	0.002532958984375	\\
0.859901451591424	0.0028076171875	\\
0.859945842766458	0.00299072265625	\\
0.859990233941492	0.003173828125	\\
0.860034625116527	0.003448486328125	\\
0.860079016291561	0.003143310546875	\\
0.860123407466596	0.003173828125	\\
0.86016779864163	0.00372314453125	\\
0.860212189816664	0.003570556640625	\\
0.860256580991699	0.003265380859375	\\
0.860300972166733	0.003265380859375	\\
0.860345363341768	0.003173828125	\\
0.860389754516802	0.0028076171875	\\
0.860434145691837	0.0025634765625	\\
0.860478536866871	0.00274658203125	\\
0.860522928041905	0.0023193359375	\\
0.86056731921694	0.001861572265625	\\
0.860611710391974	0.001861572265625	\\
0.860656101567008	0.001739501953125	\\
0.860700492742043	0.001129150390625	\\
0.860744883917077	0.0010986328125	\\
0.860789275092112	0.001129150390625	\\
0.860833666267146	0.000946044921875	\\
0.86087805744218	0.000823974609375	\\
0.860922448617215	0.0008544921875	\\
0.860966839792249	0.00091552734375	\\
0.861011230967284	0.0009765625	\\
0.861055622142318	0.000579833984375	\\
0.861100013317353	0.00067138671875	\\
0.861144404492387	0.000274658203125	\\
0.861188795667421	0.00030517578125	\\
0.861233186842456	0.00018310546875	\\
0.86127757801749	-0.00030517578125	\\
0.861321969192525	-0.000152587890625	\\
0.861366360367559	-0.00042724609375	\\
0.861410751542593	-0.00030517578125	\\
0.861455142717628	-9.1552734375e-05	\\
0.861499533892662	0.0001220703125	\\
0.861543925067697	0.000274658203125	\\
0.861588316242731	3.0517578125e-05	\\
0.861632707417765	-6.103515625e-05	\\
0.8616770985928	-3.0517578125e-05	\\
0.861721489767834	0.00048828125	\\
0.861765880942869	0.000579833984375	\\
0.861810272117903	0.000244140625	\\
0.861854663292937	0.000732421875	\\
0.861899054467972	0.0013427734375	\\
0.861943445643006	0.001861572265625	\\
0.861987836818041	0.00152587890625	\\
0.862032227993075	0.001373291015625	\\
0.862076619168109	0.001800537109375	\\
0.862121010343144	0.00164794921875	\\
0.862165401518178	0.001373291015625	\\
0.862209792693213	0.001495361328125	\\
0.862254183868247	0.00244140625	\\
0.862298575043281	0.002685546875	\\
0.862342966218316	0.00274658203125	\\
0.86238735739335	0.002838134765625	\\
0.862431748568385	0.002777099609375	\\
0.862476139743419	0.002777099609375	\\
0.862520530918453	0.00286865234375	\\
0.862564922093488	0.0028076171875	\\
0.862609313268522	0.002471923828125	\\
0.862653704443557	0.002655029296875	\\
0.862698095618591	0.002777099609375	\\
0.862742486793625	0.002655029296875	\\
0.86278687796866	0.002227783203125	\\
0.862831269143694	0.00238037109375	\\
0.862875660318729	0.00238037109375	\\
0.862920051493763	0.00146484375	\\
0.862964442668797	0.001129150390625	\\
0.863008833843832	0.00146484375	\\
0.863053225018866	0.001220703125	\\
0.863097616193901	0.00048828125	\\
0.863142007368935	0.000701904296875	\\
0.86318639854397	0.000732421875	\\
0.863230789719004	0.000396728515625	\\
0.863275180894038	6.103515625e-05	\\
0.863319572069073	-0.00030517578125	\\
0.863363963244107	-0.000457763671875	\\
0.863408354419141	-0.000457763671875	\\
0.863452745594176	-0.000762939453125	\\
0.86349713676921	-0.001007080078125	\\
0.863541527944245	-0.000946044921875	\\
0.863585919119279	-0.000732421875	\\
0.863630310294313	-0.000701904296875	\\
0.863674701469348	-0.000885009765625	\\
0.863719092644382	-0.001007080078125	\\
0.863763483819417	-0.000640869140625	\\
0.863807874994451	-0.000518798828125	\\
0.863852266169486	-0.001068115234375	\\
0.86389665734452	-0.000823974609375	\\
0.863941048519554	-0.00054931640625	\\
0.863985439694589	-0.00030517578125	\\
0.864029830869623	3.0517578125e-05	\\
0.864074222044658	0.000244140625	\\
0.864118613219692	0.000152587890625	\\
0.864163004394726	-0.0001220703125	\\
0.864207395569761	9.1552734375e-05	\\
0.864251786744795	-3.0517578125e-05	\\
0.864296177919829	0.000152587890625	\\
0.864340569094864	0.000946044921875	\\
0.864384960269898	0.001129150390625	\\
0.864429351444933	0.00128173828125	\\
0.864473742619967	0.001739501953125	\\
0.864518133795002	0.001800537109375	\\
0.864562524970036	0.00140380859375	\\
0.86460691614507	0.00177001953125	\\
0.864651307320105	0.001739501953125	\\
0.864695698495139	0.001953125	\\
0.864740089670174	0.0023193359375	\\
0.864784480845208	0.001983642578125	\\
0.864828872020242	0.0018310546875	\\
0.864873263195277	0.00177001953125	\\
0.864917654370311	0.00164794921875	\\
0.864962045545346	0.001983642578125	\\
0.86500643672038	0.001922607421875	\\
0.865050827895414	0.001617431640625	\\
0.865095219070449	0.00189208984375	\\
0.865139610245483	0.00177001953125	\\
0.865184001420518	0.001800537109375	\\
0.865228392595552	0.002197265625	\\
0.865272783770586	0.001922607421875	\\
0.865317174945621	0.00189208984375	\\
0.865361566120655	0.00152587890625	\\
0.86540595729569	0.001312255859375	\\
0.865450348470724	0.001129150390625	\\
0.865494739645758	0.000640869140625	\\
0.865539130820793	0.000701904296875	\\
0.865583521995827	0.000823974609375	\\
0.865627913170862	0.000640869140625	\\
0.865672304345896	0.00042724609375	\\
0.86571669552093	0.00067138671875	\\
0.865761086695965	0.00091552734375	\\
0.865805477870999	0.00067138671875	\\
0.865849869046034	0.00079345703125	\\
0.865894260221068	0.00091552734375	\\
0.865938651396102	0.001220703125	\\
0.865983042571137	0.00152587890625	\\
0.866027433746171	0.00146484375	\\
0.866071824921206	0.00152587890625	\\
0.86611621609624	0.0013427734375	\\
0.866160607271274	0.0008544921875	\\
0.866204998446309	0.00067138671875	\\
0.866249389621343	0.0009765625	\\
0.866293780796378	0.000732421875	\\
0.866338171971412	0.00115966796875	\\
0.866382563146446	0.001373291015625	\\
0.866426954321481	0.00140380859375	\\
0.866471345496515	0.00164794921875	\\
0.86651573667155	0.00164794921875	\\
0.866560127846584	0.0020751953125	\\
0.866604519021619	0.002044677734375	\\
0.866648910196653	0.00164794921875	\\
0.866693301371687	0.001678466796875	\\
0.866737692546722	0.00201416015625	\\
0.866782083721756	0.00164794921875	\\
0.866826474896791	0.00091552734375	\\
0.866870866071825	0.00091552734375	\\
0.866915257246859	0.001007080078125	\\
0.866959648421894	0.000701904296875	\\
0.867004039596928	0.00048828125	\\
0.867048430771963	0.000152587890625	\\
0.867092821946997	6.103515625e-05	\\
0.867137213122031	0.000579833984375	\\
0.867181604297066	0.000335693359375	\\
0.8672259954721	0.00030517578125	\\
0.867270386647135	0.000457763671875	\\
0.867314777822169	0.0006103515625	\\
0.867359168997203	0.000701904296875	\\
0.867403560172238	0.000640869140625	\\
0.867447951347272	0.000823974609375	\\
0.867492342522307	0.00091552734375	\\
0.867536733697341	0.000396728515625	\\
0.867581124872375	0.000640869140625	\\
0.86762551604741	0.000701904296875	\\
0.867669907222444	0.0003662109375	\\
0.867714298397479	0.00042724609375	\\
0.867758689572513	0.000457763671875	\\
0.867803080747547	0.000732421875	\\
0.867847471922582	0.0013427734375	\\
0.867891863097616	0.001556396484375	\\
0.867936254272651	0.00201416015625	\\
0.867980645447685	0.0018310546875	\\
0.868025036622719	0.00152587890625	\\
0.868069427797754	0.002105712890625	\\
0.868113818972788	0.00189208984375	\\
0.868158210147823	0.0020751953125	\\
0.868202601322857	0.002471923828125	\\
0.868246992497891	0.002044677734375	\\
0.868291383672926	0.002410888671875	\\
0.86833577484796	0.002838134765625	\\
0.868380166022995	0.002593994140625	\\
0.868424557198029	0.002685546875	\\
0.868468948373063	0.002288818359375	\\
0.868513339548098	0.002349853515625	\\
0.868557730723132	0.002716064453125	\\
0.868602121898167	0.0030517578125	\\
0.868646513073201	0.003173828125	\\
0.868690904248235	0.0029296875	\\
0.86873529542327	0.002960205078125	\\
0.868779686598304	0.0029296875	\\
0.868824077773339	0.00274658203125	\\
0.868868468948373	0.00225830078125	\\
0.868912860123408	0.00164794921875	\\
0.868957251298442	0.001617431640625	\\
0.869001642473476	0.001556396484375	\\
0.869046033648511	0.001708984375	\\
0.869090424823545	0.0015869140625	\\
0.869134815998579	0.001190185546875	\\
0.869179207173614	0.00067138671875	\\
0.869223598348648	3.0517578125e-05	\\
0.869267989523683	0.0001220703125	\\
0.869312380698717	-0.00048828125	\\
0.869356771873751	-0.00091552734375	\\
0.869401163048786	-0.0009765625	\\
0.86944555422382	-0.000885009765625	\\
0.869489945398855	-0.000701904296875	\\
0.869534336573889	-0.000946044921875	\\
0.869578727748924	-0.000946044921875	\\
0.869623118923958	-0.001129150390625	\\
0.869667510098992	-0.001190185546875	\\
0.869711901274027	-0.00128173828125	\\
0.869756292449061	-0.001220703125	\\
0.869800683624096	-0.00103759765625	\\
0.86984507479913	-0.001434326171875	\\
0.869889465974164	-0.00103759765625	\\
0.869933857149199	-0.00067138671875	\\
0.869978248324233	-0.00091552734375	\\
0.870022639499268	-0.000396728515625	\\
0.870067030674302	-0.000213623046875	\\
0.870111421849336	-0.000274658203125	\\
0.870155813024371	0.00018310546875	\\
0.870200204199405	0.000244140625	\\
0.87024459537444	0.00030517578125	\\
0.870288986549474	0.0006103515625	\\
0.870333377724508	0.000335693359375	\\
0.870377768899543	0.00054931640625	\\
0.870422160074577	0.000732421875	\\
0.870466551249612	0.000640869140625	\\
0.870510942424646	0.001220703125	\\
0.87055533359968	0.001373291015625	\\
0.870599724774715	0.0013427734375	\\
0.870644115949749	0.000946044921875	\\
0.870688507124784	0.00030517578125	\\
0.870732898299818	0.0009765625	\\
0.870777289474852	0.001312255859375	\\
0.870821680649887	0.001373291015625	\\
0.870866071824921	0.00115966796875	\\
0.870910462999956	0.000946044921875	\\
0.87095485417499	0.00103759765625	\\
0.870999245350024	0.000762939453125	\\
0.871043636525059	0.000213623046875	\\
0.871088027700093	0.000152587890625	\\
0.871132418875128	0.000335693359375	\\
0.871176810050162	-0.000335693359375	\\
0.871221201225196	-0.0003662109375	\\
0.871265592400231	-0.000244140625	\\
0.871309983575265	-0.000518798828125	\\
0.8713543747503	-0.000946044921875	\\
0.871398765925334	-0.00152587890625	\\
0.871443157100368	-0.001556396484375	\\
0.871487548275403	-0.001434326171875	\\
0.871531939450437	-0.00164794921875	\\
0.871576330625472	-0.00164794921875	\\
0.871620721800506	-0.00152587890625	\\
0.871665112975541	-0.001800537109375	\\
0.871709504150575	-0.002166748046875	\\
0.871753895325609	-0.002349853515625	\\
0.871798286500644	-0.002105712890625	\\
0.871842677675678	-0.0025634765625	\\
0.871887068850712	-0.003021240234375	\\
0.871931460025747	-0.002655029296875	\\
0.871975851200781	-0.00213623046875	\\
0.872020242375816	-0.00189208984375	\\
0.87206463355085	-0.001800537109375	\\
0.872109024725884	-0.001953125	\\
0.872153415900919	-0.001678466796875	\\
0.872197807075953	-0.00140380859375	\\
0.872242198250988	-0.00152587890625	\\
0.872286589426022	-0.001190185546875	\\
0.872330980601057	-0.0008544921875	\\
0.872375371776091	-0.000518798828125	\\
0.872419762951125	-0.00067138671875	\\
0.87246415412616	-0.000518798828125	\\
0.872508545301194	-0.00042724609375	\\
0.872552936476229	-0.00067138671875	\\
0.872597327651263	-0.000762939453125	\\
0.872641718826297	-0.0008544921875	\\
0.872686110001332	-0.000335693359375	\\
0.872730501176366	0.00018310546875	\\
0.8727748923514	0.0001220703125	\\
0.872819283526435	0.00054931640625	\\
0.872863674701469	0.001068115234375	\\
0.872908065876504	0.0006103515625	\\
0.872952457051538	0.0009765625	\\
0.872996848226573	0.000640869140625	\\
0.873041239401607	0.0003662109375	\\
0.873085630576641	0.000762939453125	\\
0.873130021751676	0.00079345703125	\\
0.87317441292671	0.000762939453125	\\
0.873218804101745	0.000885009765625	\\
0.873263195276779	0.00079345703125	\\
0.873307586451813	0.0006103515625	\\
0.873351977626848	0.001007080078125	\\
0.873396368801882	0.0009765625	\\
0.873440759976917	0.001068115234375	\\
0.873485151151951	0.0010986328125	\\
0.873529542326985	0.001007080078125	\\
0.87357393350202	0.001495361328125	\\
0.873618324677054	0.00177001953125	\\
0.873662715852089	0.0013427734375	\\
0.873707107027123	0.001678466796875	\\
0.873751498202157	0.001678466796875	\\
0.873795889377192	0.0013427734375	\\
0.873840280552226	0.001678466796875	\\
0.873884671727261	0.001373291015625	\\
0.873929062902295	0.00128173828125	\\
0.873973454077329	0.00146484375	\\
0.874017845252364	0.0008544921875	\\
0.874062236427398	0.001190185546875	\\
0.874106627602433	0.00115966796875	\\
0.874151018777467	0.00091552734375	\\
0.874195409952501	0.0010986328125	\\
0.874239801127536	0.001495361328125	\\
0.87428419230257	0.001678466796875	\\
0.874328583477605	0.001617431640625	\\
0.874372974652639	0.001617431640625	\\
0.874417365827673	0.00115966796875	\\
0.874461757002708	0.00091552734375	\\
0.874506148177742	0.0010986328125	\\
0.874550539352777	0.001068115234375	\\
0.874594930527811	0.000732421875	\\
0.874639321702846	0.000732421875	\\
0.87468371287788	0.0009765625	\\
0.874728104052914	0.0008544921875	\\
0.874772495227949	0.000701904296875	\\
0.874816886402983	0.000640869140625	\\
0.874861277578017	0.000732421875	\\
0.874905668753052	0.000946044921875	\\
0.874950059928086	0.001129150390625	\\
0.874994451103121	0.0009765625	\\
0.875038842278155	0.000579833984375	\\
0.87508323345319	0.000640869140625	\\
0.875127624628224	0.000396728515625	\\
0.875172015803258	0.001007080078125	\\
0.875216406978293	0.000762939453125	\\
0.875260798153327	0.000152587890625	\\
0.875305189328362	-6.103515625e-05	\\
0.875349580503396	-6.103515625e-05	\\
0.87539397167843	9.1552734375e-05	\\
0.875438362853465	0.000244140625	\\
0.875482754028499	-6.103515625e-05	\\
0.875527145203534	-0.000244140625	\\
0.875571536378568	-0.00030517578125	\\
0.875615927553602	-0.00042724609375	\\
0.875660318728637	-0.000518798828125	\\
0.875704709903671	-0.000213623046875	\\
0.875749101078706	0.000152587890625	\\
0.87579349225374	9.1552734375e-05	\\
0.875837883428774	-9.1552734375e-05	\\
0.875882274603809	-0.000732421875	\\
0.875926665778843	-0.00091552734375	\\
0.875971056953878	-0.00091552734375	\\
0.876015448128912	-0.001251220703125	\\
0.876059839303946	-0.001190185546875	\\
0.876104230478981	-0.0010986328125	\\
0.876148621654015	-0.001068115234375	\\
0.87619301282905	-0.000823974609375	\\
0.876237404004084	-0.000335693359375	\\
0.876281795179118	9.1552734375e-05	\\
0.876326186354153	0.0003662109375	\\
0.876370577529187	0.000335693359375	\\
0.876414968704222	-6.103515625e-05	\\
0.876459359879256	-0.000244140625	\\
0.87650375105429	3.0517578125e-05	\\
0.876548142229325	0.00048828125	\\
0.876592533404359	0.000762939453125	\\
0.876636924579394	0.0009765625	\\
0.876681315754428	0.000885009765625	\\
0.876725706929462	0.00079345703125	\\
0.876770098104497	0.00091552734375	\\
0.876814489279531	0.0013427734375	\\
0.876858880454566	0.001708984375	\\
0.8769032716296	0.0015869140625	\\
0.876947662804634	0.00146484375	\\
0.876992053979669	0.00152587890625	\\
0.877036445154703	0.00146484375	\\
0.877080836329738	0.001922607421875	\\
0.877125227504772	0.001922607421875	\\
0.877169618679806	0.001983642578125	\\
0.877214009854841	0.00177001953125	\\
0.877258401029875	0.00146484375	\\
0.87730279220491	0.001434326171875	\\
0.877347183379944	0.000701904296875	\\
0.877391574554979	0.0006103515625	\\
0.877435965730013	0.00048828125	\\
0.877480356905047	0.000244140625	\\
0.877524748080082	9.1552734375e-05	\\
0.877569139255116	-0.000396728515625	\\
0.87761353043015	-9.1552734375e-05	\\
0.877657921605185	-6.103515625e-05	\\
0.877702312780219	-0.000457763671875	\\
0.877746703955254	-0.0009765625	\\
0.877791095130288	-0.001373291015625	\\
0.877835486305322	-0.001312255859375	\\
0.877879877480357	-0.001434326171875	\\
0.877924268655391	-0.00152587890625	\\
0.877968659830426	-0.00152587890625	\\
0.87801305100546	-0.00152587890625	\\
0.878057442180495	-0.001556396484375	\\
0.878101833355529	-0.001800537109375	\\
0.878146224530563	-0.00177001953125	\\
0.878190615705598	-0.00189208984375	\\
0.878235006880632	-0.001861572265625	\\
0.878279398055667	-0.001800537109375	\\
0.878323789230701	-0.001678466796875	\\
0.878368180405735	-0.0013427734375	\\
0.87841257158077	-0.00103759765625	\\
0.878456962755804	-0.000518798828125	\\
0.878501353930839	-0.00067138671875	\\
0.878545745105873	-0.000335693359375	\\
0.878590136280907	-0.000640869140625	\\
0.878634527455942	-0.00067138671875	\\
0.878678918630976	-0.00042724609375	\\
0.878723309806011	-0.000823974609375	\\
0.878767700981045	-0.00018310546875	\\
0.878812092156079	0.0001220703125	\\
0.878856483331114	-0.000244140625	\\
0.878900874506148	-3.0517578125e-05	\\
0.878945265681183	0.000213623046875	\\
0.878989656856217	-9.1552734375e-05	\\
0.879034048031251	0.000335693359375	\\
0.879078439206286	0.000213623046875	\\
0.87912283038132	0.00018310546875	\\
0.879167221556355	-0.00018310546875	\\
0.879211612731389	9.1552734375e-05	\\
0.879256003906423	0.00030517578125	\\
0.879300395081458	0	\\
0.879344786256492	0	\\
0.879389177431527	-0.00018310546875	\\
0.879433568606561	-6.103515625e-05	\\
0.879477959781595	-0.000396728515625	\\
0.87952235095663	-0.000579833984375	\\
0.879566742131664	-0.00067138671875	\\
0.879611133306699	-0.000885009765625	\\
0.879655524481733	-0.00103759765625	\\
0.879699915656767	-0.001312255859375	\\
0.879744306831802	-0.00146484375	\\
0.879788698006836	-0.001800537109375	\\
0.879833089181871	-0.001983642578125	\\
0.879877480356905	-0.002349853515625	\\
0.879921871531939	-0.00244140625	\\
0.879966262706974	-0.002899169921875	\\
0.880010653882008	-0.003082275390625	\\
0.880055045057043	-0.0028076171875	\\
0.880099436232077	-0.002777099609375	\\
0.880143827407112	-0.00286865234375	\\
0.880188218582146	-0.0029296875	\\
0.88023260975718	-0.002685546875	\\
0.880277000932215	-0.00238037109375	\\
0.880321392107249	-0.003021240234375	\\
0.880365783282283	-0.003204345703125	\\
0.880410174457318	-0.002838134765625	\\
0.880454565632352	-0.00274658203125	\\
0.880498956807387	-0.002410888671875	\\
0.880543347982421	-0.002410888671875	\\
0.880587739157455	-0.00225830078125	\\
0.88063213033249	-0.002044677734375	\\
0.880676521507524	-0.001556396484375	\\
0.880720912682559	-0.001373291015625	\\
0.880765303857593	-0.001739501953125	\\
0.880809695032628	-0.001190185546875	\\
0.880854086207662	-0.000762939453125	\\
0.880898477382696	-0.00048828125	\\
0.880942868557731	-0.000335693359375	\\
0.880987259732765	-0.00042724609375	\\
0.8810316509078	6.103515625e-05	\\
0.881076042082834	9.1552734375e-05	\\
0.881120433257868	0.000152587890625	\\
0.881164824432903	0.00067138671875	\\
0.881209215607937	0.001007080078125	\\
0.881253606782972	0.00140380859375	\\
0.881297997958006	0.001129150390625	\\
0.88134238913304	0.00128173828125	\\
0.881386780308075	0.001708984375	\\
0.881431171483109	0.0013427734375	\\
0.881475562658144	0.001220703125	\\
0.881519953833178	0.0010986328125	\\
0.881564345008212	0.001007080078125	\\
0.881608736183247	0.0006103515625	\\
0.881653127358281	0.000762939453125	\\
0.881697518533316	0.0013427734375	\\
0.88174190970835	0.001007080078125	\\
0.881786300883384	0.00091552734375	\\
0.881830692058419	0.0009765625	\\
0.881875083233453	0.000823974609375	\\
0.881919474408488	0.00115966796875	\\
0.881963865583522	0.000762939453125	\\
0.882008256758556	0.00030517578125	\\
0.882052647933591	0.0003662109375	\\
0.882097039108625	0.000274658203125	\\
0.88214143028366	0.00018310546875	\\
0.882185821458694	0.000152587890625	\\
0.882230212633728	-6.103515625e-05	\\
0.882274603808763	-0.000213623046875	\\
0.882318994983797	-0.00042724609375	\\
0.882363386158832	-0.0003662109375	\\
0.882407777333866	-0.0008544921875	\\
0.8824521685089	-0.00091552734375	\\
0.882496559683935	-0.00048828125	\\
0.882540950858969	-0.000823974609375	\\
0.882585342034004	-0.0010986328125	\\
0.882629733209038	-0.00079345703125	\\
0.882674124384072	-0.000457763671875	\\
0.882718515559107	-0.00067138671875	\\
0.882762906734141	-0.00067138671875	\\
0.882807297909176	-0.0008544921875	\\
0.88285168908421	-0.00079345703125	\\
0.882896080259244	-0.00128173828125	\\
0.882940471434279	-0.001373291015625	\\
0.882984862609313	-0.001373291015625	\\
0.883029253784348	-0.001312255859375	\\
0.883073644959382	-0.001220703125	\\
0.883118036134417	-0.00140380859375	\\
0.883162427309451	-0.001251220703125	\\
0.883206818484485	-0.00103759765625	\\
0.88325120965952	-0.0010986328125	\\
0.883295600834554	-0.00103759765625	\\
0.883339992009588	-0.000701904296875	\\
0.883384383184623	-0.0008544921875	\\
0.883428774359657	-0.001068115234375	\\
0.883473165534692	-0.000518798828125	\\
0.883517556709726	-9.1552734375e-05	\\
0.883561947884761	-0.000396728515625	\\
0.883606339059795	-0.000885009765625	\\
0.883650730234829	-0.000640869140625	\\
0.883695121409864	-0.000335693359375	\\
0.883739512584898	-0.00054931640625	\\
0.883783903759933	-9.1552734375e-05	\\
0.883828294934967	0.00018310546875	\\
0.883872686110001	-0.000152587890625	\\
0.883917077285036	9.1552734375e-05	\\
0.88396146846007	0.000274658203125	\\
0.884005859635105	0.000213623046875	\\
0.884050250810139	0.00030517578125	\\
0.884094641985173	0.00030517578125	\\
0.884139033160208	0.00018310546875	\\
0.884183424335242	-0.0001220703125	\\
0.884227815510277	-0.000152587890625	\\
0.884272206685311	-9.1552734375e-05	\\
0.884316597860345	-0.000518798828125	\\
0.88436098903538	-0.0003662109375	\\
0.884405380210414	-0.000579833984375	\\
0.884449771385449	-0.000823974609375	\\
0.884494162560483	-0.000518798828125	\\
0.884538553735517	-0.000396728515625	\\
0.884582944910552	-0.000579833984375	\\
0.884627336085586	-0.001190185546875	\\
0.884671727260621	-0.001190185546875	\\
0.884716118435655	-0.000732421875	\\
0.884760509610689	-0.00048828125	\\
0.884804900785724	-0.000274658203125	\\
0.884849291960758	-0.000335693359375	\\
0.884893683135793	-0.00054931640625	\\
0.884938074310827	-0.000152587890625	\\
0.884982465485861	-0.000244140625	\\
0.885026856660896	-0.000640869140625	\\
0.88507124783593	-0.00054931640625	\\
0.885115639010965	-0.000244140625	\\
0.885160030185999	0.0001220703125	\\
0.885204421361033	0.000396728515625	\\
0.885248812536068	0.000152587890625	\\
0.885293203711102	0.000335693359375	\\
0.885337594886137	0.0006103515625	\\
0.885381986061171	0.00048828125	\\
0.885426377236205	0.000701904296875	\\
0.88547076841124	0.00115966796875	\\
0.885515159586274	0.0013427734375	\\
0.885559550761309	0.001129150390625	\\
0.885603941936343	0.00091552734375	\\
0.885648333111377	0.000579833984375	\\
0.885692724286412	0.000518798828125	\\
0.885737115461446	0.000244140625	\\
0.885781506636481	0.000213623046875	\\
0.885825897811515	-0.000335693359375	\\
0.88587028898655	-0.00054931640625	\\
0.885914680161584	-0.000457763671875	\\
0.885959071336618	-0.000396728515625	\\
0.886003462511653	-0.00079345703125	\\
0.886047853686687	-0.000823974609375	\\
0.886092244861721	-0.0003662109375	\\
0.886136636036756	-0.00054931640625	\\
0.88618102721179	-0.000946044921875	\\
0.886225418386825	-0.0010986328125	\\
0.886269809561859	-0.001190185546875	\\
0.886314200736893	-0.0013427734375	\\
0.886358591911928	-0.001495361328125	\\
0.886402983086962	-0.001251220703125	\\
0.886447374261997	-0.001434326171875	\\
0.886491765437031	-0.001861572265625	\\
0.886536156612066	-0.001983642578125	\\
0.8865805477871	-0.0018310546875	\\
0.886624938962134	-0.00152587890625	\\
0.886669330137169	-0.001678466796875	\\
0.886713721312203	-0.00213623046875	\\
0.886758112487238	-0.001953125	\\
0.886802503662272	-0.0013427734375	\\
0.886846894837306	-0.001373291015625	\\
0.886891286012341	-0.000946044921875	\\
0.886935677187375	-0.001007080078125	\\
0.88698006836241	-0.000946044921875	\\
0.887024459537444	-0.00042724609375	\\
0.887068850712478	-0.00042724609375	\\
0.887113241887513	-0.000732421875	\\
0.887157633062547	-0.000885009765625	\\
0.887202024237582	-0.000823974609375	\\
0.887246415412616	-0.00067138671875	\\
0.88729080658765	-0.000213623046875	\\
0.887335197762685	3.0517578125e-05	\\
0.887379588937719	-0.0001220703125	\\
0.887423980112754	-0.000396728515625	\\
0.887468371287788	-0.000396728515625	\\
0.887512762462822	-0.000335693359375	\\
0.887557153637857	-0.000152587890625	\\
0.887601544812891	0.00018310546875	\\
0.887645935987926	-0.000213623046875	\\
0.88769032716296	0.00018310546875	\\
0.887734718337994	0.000732421875	\\
0.887779109513029	0.000732421875	\\
0.887823500688063	0.0008544921875	\\
};
\addplot [color=blue,solid,forget plot]
  table[row sep=crcr]{
0.887823500688063	0.0008544921875	\\
0.887867891863098	0.000335693359375	\\
0.887912283038132	0.000518798828125	\\
0.887956674213166	0.00030517578125	\\
0.888001065388201	-3.0517578125e-05	\\
0.888045456563235	-0.0001220703125	\\
0.88808984773827	-0.00054931640625	\\
0.888134238913304	-0.00091552734375	\\
0.888178630088338	-0.000732421875	\\
0.888223021263373	-0.001129150390625	\\
0.888267412438407	-0.00146484375	\\
0.888311803613442	-0.0013427734375	\\
0.888356194788476	-0.0015869140625	\\
0.88840058596351	-0.001708984375	\\
0.888444977138545	-0.00177001953125	\\
0.888489368313579	-0.001617431640625	\\
0.888533759488614	-0.001800537109375	\\
0.888578150663648	-0.001983642578125	\\
0.888622541838683	-0.001708984375	\\
0.888666933013717	-0.001953125	\\
0.888711324188751	-0.001953125	\\
0.888755715363786	-0.002288818359375	\\
0.88880010653882	-0.002471923828125	\\
0.888844497713854	-0.0023193359375	\\
0.888888888888889	-0.002166748046875	\\
0.888933280063923	-0.002105712890625	\\
0.888977671238958	-0.002105712890625	\\
0.889022062413992	-0.001983642578125	\\
0.889066453589026	-0.002166748046875	\\
0.889110844764061	-0.001922607421875	\\
0.889155235939095	-0.001434326171875	\\
0.88919962711413	-0.001312255859375	\\
0.889244018289164	-0.00152587890625	\\
0.889288409464199	-0.001373291015625	\\
0.889332800639233	-0.001007080078125	\\
0.889377191814267	-0.000274658203125	\\
0.889421582989302	3.0517578125e-05	\\
0.889465974164336	-0.000457763671875	\\
0.889510365339371	-0.000152587890625	\\
0.889554756514405	0.000396728515625	\\
0.889599147689439	0.000732421875	\\
0.889643538864474	0.000518798828125	\\
0.889687930039508	0.00018310546875	\\
0.889732321214543	0.0003662109375	\\
0.889776712389577	0.000640869140625	\\
0.889821103564611	0.00091552734375	\\
0.889865494739646	0.000885009765625	\\
0.88990988591468	0.0010986328125	\\
0.889954277089715	0.0013427734375	\\
0.889998668264749	0.001129150390625	\\
0.890043059439783	0.001434326171875	\\
0.890087450614818	0.001617431640625	\\
0.890131841789852	0.001495361328125	\\
0.890176232964887	0.00177001953125	\\
0.890220624139921	0.001495361328125	\\
0.890265015314955	0.001312255859375	\\
0.89030940648999	0.001678466796875	\\
0.890353797665024	0.0015869140625	\\
0.890398188840059	0.00164794921875	\\
0.890442580015093	0.001373291015625	\\
0.890486971190127	0.001220703125	\\
0.890531362365162	0.0010986328125	\\
0.890575753540196	0.000946044921875	\\
0.890620144715231	0.001129150390625	\\
0.890664535890265	0.0009765625	\\
0.890708927065299	0.000762939453125	\\
0.890753318240334	0.0006103515625	\\
0.890797709415368	0.0006103515625	\\
0.890842100590403	0.000457763671875	\\
0.890886491765437	0.000244140625	\\
0.890930882940471	-0.0001220703125	\\
0.890975274115506	-0.000335693359375	\\
0.89101966529054	-0.00018310546875	\\
0.891064056465575	-0.000274658203125	\\
0.891108447640609	-0.000946044921875	\\
0.891152838815643	-0.0009765625	\\
0.891197229990678	-0.000823974609375	\\
0.891241621165712	-0.000732421875	\\
0.891286012340747	-0.00079345703125	\\
0.891330403515781	-0.000762939453125	\\
0.891374794690815	-0.000457763671875	\\
0.89141918586585	-0.00048828125	\\
0.891463577040884	-0.00018310546875	\\
0.891507968215919	-3.0517578125e-05	\\
0.891552359390953	-0.000244140625	\\
0.891596750565988	-0.000335693359375	\\
0.891641141741022	-6.103515625e-05	\\
0.891685532916056	0.00018310546875	\\
0.891729924091091	0.000335693359375	\\
0.891774315266125	0.000579833984375	\\
0.891818706441159	0.000823974609375	\\
0.891863097616194	0.00079345703125	\\
0.891907488791228	0.000579833984375	\\
0.891951879966263	0.000701904296875	\\
0.891996271141297	0.00048828125	\\
0.892040662316332	0.001007080078125	\\
0.892085053491366	0.0006103515625	\\
0.8921294446664	0.000396728515625	\\
0.892173835841435	0.000946044921875	\\
0.892218227016469	0.0013427734375	\\
0.892262618191504	0.0010986328125	\\
0.892307009366538	0.00091552734375	\\
0.892351400541572	0.0008544921875	\\
0.892395791716607	0.0006103515625	\\
0.892440182891641	0.000244140625	\\
0.892484574066676	-0.000152587890625	\\
0.89252896524171	-9.1552734375e-05	\\
0.892573356416744	-0.0006103515625	\\
0.892617747591779	-0.000396728515625	\\
0.892662138766813	-0.000823974609375	\\
0.892706529941848	-0.00103759765625	\\
0.892750921116882	-0.00054931640625	\\
0.892795312291916	-0.001373291015625	\\
0.892839703466951	-0.00115966796875	\\
0.892884094641985	-0.00115966796875	\\
0.89292848581702	-0.001251220703125	\\
0.892972876992054	-0.00103759765625	\\
0.893017268167088	-0.00128173828125	\\
0.893061659342123	-0.001007080078125	\\
0.893106050517157	-0.00091552734375	\\
0.893150441692192	-0.001220703125	\\
0.893194832867226	-0.001068115234375	\\
0.89323922404226	-0.00091552734375	\\
0.893283615217295	-0.000457763671875	\\
0.893328006392329	3.0517578125e-05	\\
0.893372397567364	0.000213623046875	\\
0.893416788742398	0.000701904296875	\\
0.893461179917432	0.000762939453125	\\
0.893505571092467	0.000579833984375	\\
0.893549962267501	0.000518798828125	\\
0.893594353442536	0.000732421875	\\
0.89363874461757	0.00115966796875	\\
0.893683135792604	0.00152587890625	\\
0.893727526967639	0.001312255859375	\\
0.893771918142673	0.00164794921875	\\
0.893816309317708	0.00164794921875	\\
0.893860700492742	0.001373291015625	\\
0.893905091667776	0.001739501953125	\\
0.893949482842811	0.001708984375	\\
0.893993874017845	0.001708984375	\\
0.89403826519288	0.001800537109375	\\
0.894082656367914	0.001495361328125	\\
0.894127047542948	0.00140380859375	\\
0.894171438717983	0.0013427734375	\\
0.894215829893017	0.000518798828125	\\
0.894260221068052	0.00018310546875	\\
0.894304612243086	0.0003662109375	\\
0.894349003418121	0.000213623046875	\\
0.894393394593155	0.000335693359375	\\
0.894437785768189	0.0003662109375	\\
0.894482176943224	0.000274658203125	\\
0.894526568118258	0.00018310546875	\\
0.894570959293292	0.00018310546875	\\
0.894615350468327	0.000213623046875	\\
0.894659741643361	6.103515625e-05	\\
0.894704132818396	-0.000579833984375	\\
0.89474852399343	-0.001251220703125	\\
0.894792915168464	-0.002044677734375	\\
0.894837306343499	-0.001953125	\\
0.894881697518533	-0.00152587890625	\\
0.894926088693568	-0.001617431640625	\\
0.894970479868602	-0.00128173828125	\\
0.895014871043637	-0.001190185546875	\\
0.895059262218671	-0.001190185546875	\\
0.895103653393705	-0.001129150390625	\\
0.89514804456874	-0.001190185546875	\\
0.895192435743774	-0.0009765625	\\
0.895236826918809	-0.000823974609375	\\
0.895281218093843	-0.00048828125	\\
0.895325609268877	-0.000274658203125	\\
0.895370000443912	-6.103515625e-05	\\
0.895414391618946	-0.000518798828125	\\
0.895458782793981	-0.000640869140625	\\
0.895503173969015	0.00018310546875	\\
0.895547565144049	0.000213623046875	\\
0.895591956319084	-9.1552734375e-05	\\
0.895636347494118	0.0003662109375	\\
0.895680738669153	0.000701904296875	\\
0.895725129844187	0.0009765625	\\
0.895769521019221	0.00091552734375	\\
0.895813912194256	0.0006103515625	\\
0.89585830336929	0.000946044921875	\\
0.895902694544325	0.001129150390625	\\
0.895947085719359	0.00128173828125	\\
0.895991476894393	0.000946044921875	\\
0.896035868069428	0.000335693359375	\\
0.896080259244462	0.00018310546875	\\
0.896124650419497	0.0006103515625	\\
0.896169041594531	0.000335693359375	\\
0.896213432769565	0.000335693359375	\\
0.8962578239446	0.000640869140625	\\
0.896302215119634	0.000579833984375	\\
0.896346606294669	0.000396728515625	\\
0.896390997469703	-0.0003662109375	\\
0.896435388644737	-0.0006103515625	\\
0.896479779819772	-0.00091552734375	\\
0.896524170994806	-0.00177001953125	\\
0.896568562169841	-0.00164794921875	\\
0.896612953344875	-0.001434326171875	\\
0.896657344519909	-0.001678466796875	\\
0.896701735694944	-0.001678466796875	\\
0.896746126869978	-0.0020751953125	\\
0.896790518045013	-0.002471923828125	\\
0.896834909220047	-0.00238037109375	\\
0.896879300395081	-0.0025634765625	\\
0.896923691570116	-0.0028076171875	\\
0.89696808274515	-0.00274658203125	\\
0.897012473920185	-0.003082275390625	\\
0.897056865095219	-0.00250244140625	\\
0.897101256270254	-0.002166748046875	\\
0.897145647445288	-0.002105712890625	\\
0.897190038620322	-0.00213623046875	\\
0.897234429795357	-0.002288818359375	\\
0.897278820970391	-0.001953125	\\
0.897323212145426	-0.001708984375	\\
0.89736760332046	-0.00128173828125	\\
0.897411994495494	-0.0008544921875	\\
0.897456385670529	-0.000640869140625	\\
0.897500776845563	-0.0003662109375	\\
0.897545168020597	-0.0001220703125	\\
0.897589559195632	0.000274658203125	\\
0.897633950370666	0.00048828125	\\
0.897678341545701	0.001007080078125	\\
0.897722732720735	0.00140380859375	\\
0.89776712389577	0.001617431640625	\\
0.897811515070804	0.002044677734375	\\
0.897855906245838	0.00201416015625	\\
0.897900297420873	0.00274658203125	\\
0.897944688595907	0.0029296875	\\
0.897989079770942	0.00250244140625	\\
0.898033470945976	0.0025634765625	\\
0.89807786212101	0.002410888671875	\\
0.898122253296045	0.00213623046875	\\
0.898166644471079	0.002044677734375	\\
0.898211035646114	0.002197265625	\\
0.898255426821148	0.002197265625	\\
0.898299817996182	0.00225830078125	\\
0.898344209171217	0.002685546875	\\
0.898388600346251	0.00262451171875	\\
0.898432991521286	0.002471923828125	\\
0.89847738269632	0.002410888671875	\\
0.898521773871354	0.00213623046875	\\
0.898566165046389	0.00177001953125	\\
0.898610556221423	0.00146484375	\\
0.898654947396458	0.0009765625	\\
0.898699338571492	0.0008544921875	\\
0.898743729746526	0.00054931640625	\\
0.898788120921561	-0.000213623046875	\\
0.898832512096595	-0.00030517578125	\\
0.89887690327163	-6.103515625e-05	\\
0.898921294446664	-0.000213623046875	\\
0.898965685621698	-0.000274658203125	\\
0.899010076796733	-0.000244140625	\\
0.899054467971767	-0.000457763671875	\\
0.899098859146802	-0.00067138671875	\\
0.899143250321836	-0.00048828125	\\
0.89918764149687	-0.000518798828125	\\
0.899232032671905	-0.000518798828125	\\
0.899276423846939	-0.000701904296875	\\
0.899320815021974	-0.001220703125	\\
0.899365206197008	-0.00152587890625	\\
0.899409597372042	-0.001800537109375	\\
0.899453988547077	-0.001708984375	\\
0.899498379722111	-0.00146484375	\\
0.899542770897146	-0.001251220703125	\\
0.89958716207218	-0.001190185546875	\\
0.899631553247214	-0.0009765625	\\
0.899675944422249	-0.000457763671875	\\
0.899720335597283	-0.000335693359375	\\
0.899764726772318	-9.1552734375e-05	\\
0.899809117947352	0.000152587890625	\\
0.899853509122386	-6.103515625e-05	\\
0.899897900297421	-0.00030517578125	\\
0.899942291472455	0	\\
0.89998668264749	0.0001220703125	\\
0.900031073822524	0.000274658203125	\\
0.900075464997559	0.000518798828125	\\
0.900119856172593	0.0006103515625	\\
0.900164247347627	0.00067138671875	\\
0.900208638522662	0.000518798828125	\\
0.900253029697696	0.0008544921875	\\
0.90029742087273	0.001129150390625	\\
0.900341812047765	0.00091552734375	\\
0.900386203222799	0.001434326171875	\\
0.900430594397834	0.001739501953125	\\
0.900474985572868	0.001190185546875	\\
0.900519376747903	0.000823974609375	\\
0.900563767922937	0.000640869140625	\\
0.900608159097971	0.00030517578125	\\
0.900652550273006	0.000244140625	\\
0.90069694144804	-3.0517578125e-05	\\
0.900741332623075	-0.000335693359375	\\
0.900785723798109	-0.00018310546875	\\
0.900830114973143	-6.103515625e-05	\\
0.900874506148178	3.0517578125e-05	\\
0.900918897323212	-0.0003662109375	\\
0.900963288498247	-0.000701904296875	\\
0.901007679673281	-0.00103759765625	\\
0.901052070848315	-0.00177001953125	\\
0.90109646202335	-0.001739501953125	\\
0.901140853198384	-0.00152587890625	\\
0.901185244373419	-0.002197265625	\\
0.901229635548453	-0.002197265625	\\
0.901274026723487	-0.001983642578125	\\
0.901318417898522	-0.001861572265625	\\
0.901362809073556	-0.00189208984375	\\
0.901407200248591	-0.001922607421875	\\
0.901451591423625	-0.001495361328125	\\
0.901495982598659	-0.001312255859375	\\
0.901540373773694	-0.001129150390625	\\
0.901584764948728	-0.000274658203125	\\
0.901629156123763	6.103515625e-05	\\
0.901673547298797	0.0001220703125	\\
0.901717938473831	0.00079345703125	\\
0.901762329648866	0.00054931640625	\\
0.9018067208239	0.00067138671875	\\
0.901851111998935	0.001373291015625	\\
0.901895503173969	0.001678466796875	\\
0.901939894349003	0.00140380859375	\\
0.901984285524038	0.001617431640625	\\
0.902028676699072	0.001953125	\\
0.902073067874107	0.00189208984375	\\
0.902117459049141	0.001922607421875	\\
0.902161850224175	0.00201416015625	\\
0.90220624139921	0.001708984375	\\
0.902250632574244	0.00164794921875	\\
0.902295023749279	0.002227783203125	\\
0.902339414924313	0.002105712890625	\\
0.902383806099347	0.001800537109375	\\
0.902428197274382	0.001800537109375	\\
0.902472588449416	0.001220703125	\\
0.902516979624451	0.00079345703125	\\
0.902561370799485	0.000335693359375	\\
0.902605761974519	0.000213623046875	\\
0.902650153149554	-6.103515625e-05	\\
0.902694544324588	-0.000457763671875	\\
0.902738935499623	-0.000762939453125	\\
0.902783326674657	-0.000823974609375	\\
0.902827717849692	-0.001068115234375	\\
0.902872109024726	-0.001495361328125	\\
0.90291650019976	-0.00164794921875	\\
0.902960891374795	-0.0018310546875	\\
0.903005282549829	-0.002227783203125	\\
0.903049673724863	-0.002227783203125	\\
0.903094064899898	-0.002166748046875	\\
0.903138456074932	-0.002410888671875	\\
0.903182847249967	-0.002410888671875	\\
0.903227238425001	-0.00244140625	\\
0.903271629600035	-0.002899169921875	\\
0.90331602077507	-0.0029296875	\\
0.903360411950104	-0.00286865234375	\\
0.903404803125139	-0.00274658203125	\\
0.903449194300173	-0.002166748046875	\\
0.903493585475208	-0.001708984375	\\
0.903537976650242	-0.001617431640625	\\
0.903582367825276	-0.00146484375	\\
0.903626759000311	-0.001251220703125	\\
0.903671150175345	-0.00128173828125	\\
0.90371554135038	-0.001068115234375	\\
0.903759932525414	-0.001251220703125	\\
0.903804323700448	-0.000732421875	\\
0.903848714875483	-0.00067138671875	\\
0.903893106050517	-0.00079345703125	\\
0.903937497225552	-0.00048828125	\\
0.903981888400586	-0.00042724609375	\\
0.90402627957562	3.0517578125e-05	\\
0.904070670750655	-0.000213623046875	\\
0.904115061925689	-0.000396728515625	\\
0.904159453100724	-0.000152587890625	\\
0.904203844275758	-0.000152587890625	\\
0.904248235450792	0.00048828125	\\
0.904292626625827	0.00091552734375	\\
0.904337017800861	0.000396728515625	\\
0.904381408975896	0.000244140625	\\
0.90442580015093	0	\\
0.904470191325964	-0.0001220703125	\\
0.904514582500999	-0.0001220703125	\\
0.904558973676033	-0.000732421875	\\
0.904603364851068	-0.000946044921875	\\
0.904647756026102	-0.0006103515625	\\
0.904692147201136	-0.000823974609375	\\
0.904736538376171	-0.00115966796875	\\
0.904780929551205	-0.0010986328125	\\
0.90482532072624	-0.00164794921875	\\
0.904869711901274	-0.00177001953125	\\
0.904914103076308	-0.0020751953125	\\
0.904958494251343	-0.002593994140625	\\
0.905002885426377	-0.0028076171875	\\
0.905047276601412	-0.00274658203125	\\
0.905091667776446	-0.003448486328125	\\
0.90513605895148	-0.003326416015625	\\
0.905180450126515	-0.00311279296875	\\
0.905224841301549	-0.003662109375	\\
0.905269232476584	-0.003387451171875	\\
0.905313623651618	-0.003204345703125	\\
0.905358014826652	-0.003173828125	\\
0.905402406001687	-0.002960205078125	\\
0.905446797176721	-0.003204345703125	\\
0.905491188351756	-0.003021240234375	\\
0.90553557952679	-0.002777099609375	\\
0.905579970701825	-0.0030517578125	\\
0.905624361876859	-0.002349853515625	\\
0.905668753051893	-0.00238037109375	\\
0.905713144226928	-0.002288818359375	\\
0.905757535401962	-0.001922607421875	\\
0.905801926576997	-0.0018310546875	\\
0.905846317752031	-0.001922607421875	\\
0.905890708927065	-0.0018310546875	\\
0.9059351001021	-0.001312255859375	\\
0.905979491277134	-0.000701904296875	\\
0.906023882452168	-0.000396728515625	\\
0.906068273627203	3.0517578125e-05	\\
0.906112664802237	0.000213623046875	\\
0.906157055977272	0.000518798828125	\\
0.906201447152306	0.0010986328125	\\
0.906245838327341	0.001190185546875	\\
0.906290229502375	0.000946044921875	\\
0.906334620677409	0.000396728515625	\\
0.906379011852444	9.1552734375e-05	\\
0.906423403027478	0.000762939453125	\\
0.906467794202513	0.000823974609375	\\
0.906512185377547	0.0009765625	\\
0.906556576552581	0.000946044921875	\\
0.906600967727616	0.00079345703125	\\
0.90664535890265	0.0010986328125	\\
0.906689750077685	0.001220703125	\\
0.906734141252719	0.00115966796875	\\
0.906778532427753	0.0008544921875	\\
0.906822923602788	0.000732421875	\\
0.906867314777822	0.000762939453125	\\
0.906911705952857	-3.0517578125e-05	\\
0.906956097127891	-0.0001220703125	\\
0.907000488302925	6.103515625e-05	\\
0.90704487947796	-0.0006103515625	\\
0.907089270652994	-0.001220703125	\\
0.907133661828029	-0.001495361328125	\\
0.907178053003063	-0.001129150390625	\\
0.907222444178097	-0.0009765625	\\
0.907266835353132	-0.0006103515625	\\
0.907311226528166	-0.000579833984375	\\
0.907355617703201	-0.00079345703125	\\
0.907400008878235	-0.0006103515625	\\
0.907444400053269	-0.000640869140625	\\
0.907488791228304	-0.00048828125	\\
0.907533182403338	-0.000457763671875	\\
0.907577573578373	-0.000732421875	\\
0.907621964753407	-0.000518798828125	\\
0.907666355928441	-0.000762939453125	\\
0.907710747103476	-0.001220703125	\\
0.90775513827851	-0.001373291015625	\\
0.907799529453545	-0.001129150390625	\\
0.907843920628579	-0.00067138671875	\\
0.907888311803613	-0.0006103515625	\\
0.907932702978648	-0.000579833984375	\\
0.907977094153682	-0.00067138671875	\\
0.908021485328717	-0.00048828125	\\
0.908065876503751	-0.000701904296875	\\
0.908110267678785	-0.00042724609375	\\
0.90815465885382	-6.103515625e-05	\\
0.908199050028854	-0.00018310546875	\\
0.908243441203889	-0.000213623046875	\\
0.908287832378923	0.0003662109375	\\
0.908332223553957	0.0006103515625	\\
0.908376614728992	6.103515625e-05	\\
0.908421005904026	0.0001220703125	\\
0.908465397079061	3.0517578125e-05	\\
0.908509788254095	0.000152587890625	\\
0.90855417942913	0.0006103515625	\\
0.908598570604164	0.000823974609375	\\
0.908642961779198	0.000823974609375	\\
0.908687352954233	0.0008544921875	\\
0.908731744129267	0.000152587890625	\\
0.908776135304301	9.1552734375e-05	\\
0.908820526479336	0.000518798828125	\\
0.90886491765437	0.000244140625	\\
0.908909308829405	-0.000244140625	\\
0.908953700004439	-0.000762939453125	\\
0.908998091179474	-0.0010986328125	\\
0.909042482354508	-0.00128173828125	\\
0.909086873529542	-0.00146484375	\\
0.909131264704577	-0.001312255859375	\\
0.909175655879611	-0.001434326171875	\\
0.909220047054646	-0.001800537109375	\\
0.90926443822968	-0.00201416015625	\\
0.909308829404714	-0.00189208984375	\\
0.909353220579749	-0.002105712890625	\\
0.909397611754783	-0.001922607421875	\\
0.909442002929818	-0.002197265625	\\
0.909486394104852	-0.003082275390625	\\
0.909530785279886	-0.00299072265625	\\
0.909575176454921	-0.003143310546875	\\
0.909619567629955	-0.003570556640625	\\
0.90966395880499	-0.00323486328125	\\
0.909708349980024	-0.003265380859375	\\
0.909752741155058	-0.00323486328125	\\
0.909797132330093	-0.002899169921875	\\
0.909841523505127	-0.002685546875	\\
0.909885914680162	-0.0023193359375	\\
0.909930305855196	-0.002105712890625	\\
0.90997469703023	-0.00164794921875	\\
0.910019088205265	-0.001007080078125	\\
0.910063479380299	-0.000579833984375	\\
0.910107870555334	-0.00091552734375	\\
0.910152261730368	-0.000762939453125	\\
0.910196652905402	-0.0009765625	\\
0.910241044080437	-0.000457763671875	\\
0.910285435255471	-0.00018310546875	\\
0.910329826430506	-0.000335693359375	\\
0.91037421760554	-0.000457763671875	\\
0.910418608780574	-0.000762939453125	\\
0.910462999955609	-0.0003662109375	\\
0.910507391130643	-0.000518798828125	\\
0.910551782305678	-0.000396728515625	\\
0.910596173480712	0.000152587890625	\\
0.910640564655746	0.000274658203125	\\
0.910684955830781	-0.000274658203125	\\
0.910729347005815	-0.0009765625	\\
0.91077373818085	-0.000732421875	\\
0.910818129355884	-0.000946044921875	\\
0.910862520530918	-0.001739501953125	\\
0.910906911705953	-0.002166748046875	\\
0.910951302880987	-0.00225830078125	\\
0.910995694056022	-0.002777099609375	\\
0.911040085231056	-0.0032958984375	\\
0.91108447640609	-0.003021240234375	\\
0.911128867581125	-0.003173828125	\\
0.911173258756159	-0.002960205078125	\\
0.911217649931194	-0.002532958984375	\\
0.911262041106228	-0.0030517578125	\\
0.911306432281263	-0.003204345703125	\\
0.911350823456297	-0.0035400390625	\\
0.911395214631331	-0.003753662109375	\\
0.911439605806366	-0.003875732421875	\\
0.9114839969814	-0.00421142578125	\\
0.911528388156435	-0.004150390625	\\
0.911572779331469	-0.0042724609375	\\
0.911617170506503	-0.003997802734375	\\
0.911661561681538	-0.003753662109375	\\
0.911705952856572	-0.003753662109375	\\
0.911750344031606	-0.003631591796875	\\
0.911794735206641	-0.003173828125	\\
0.911839126381675	-0.003265380859375	\\
0.91188351755671	-0.003143310546875	\\
0.911927908731744	-0.002410888671875	\\
0.911972299906779	-0.002593994140625	\\
0.912016691081813	-0.002655029296875	\\
0.912061082256847	-0.002532958984375	\\
0.912105473431882	-0.0025634765625	\\
0.912149864606916	-0.002197265625	\\
0.912194255781951	-0.001953125	\\
0.912238646956985	-0.001922607421875	\\
0.912283038132019	-0.00152587890625	\\
0.912327429307054	-0.0013427734375	\\
0.912371820482088	-0.001312255859375	\\
0.912416211657123	-0.00115966796875	\\
0.912460602832157	-0.001007080078125	\\
0.912504994007191	-0.001220703125	\\
0.912549385182226	-0.0018310546875	\\
0.91259377635726	-0.001708984375	\\
0.912638167532295	-0.001739501953125	\\
0.912682558707329	-0.001953125	\\
0.912726949882363	-0.0025634765625	\\
0.912771341057398	-0.00323486328125	\\
0.912815732232432	-0.002777099609375	\\
0.912860123407467	-0.002593994140625	\\
0.912904514582501	-0.003173828125	\\
0.912948905757535	-0.003143310546875	\\
0.91299329693257	-0.003265380859375	\\
0.913037688107604	-0.003631591796875	\\
0.913082079282639	-0.003570556640625	\\
0.913126470457673	-0.003814697265625	\\
0.913170861632707	-0.004364013671875	\\
0.913215252807742	-0.00494384765625	\\
0.913259643982776	-0.0048828125	\\
0.913304035157811	-0.004669189453125	\\
0.913348426332845	-0.00494384765625	\\
0.913392817507879	-0.005401611328125	\\
0.913437208682914	-0.0057373046875	\\
0.913481599857948	-0.005615234375	\\
0.913525991032983	-0.005126953125	\\
0.913570382208017	-0.00518798828125	\\
0.913614773383051	-0.005279541015625	\\
0.913659164558086	-0.0054931640625	\\
0.91370355573312	-0.00531005859375	\\
0.913747946908155	-0.00518798828125	\\
0.913792338083189	-0.005218505859375	\\
0.913836729258223	-0.005340576171875	\\
0.913881120433258	-0.005279541015625	\\
0.913925511608292	-0.004547119140625	\\
0.913969902783327	-0.0042724609375	\\
0.914014293958361	-0.004425048828125	\\
0.914058685133396	-0.004150390625	\\
0.91410307630843	-0.004150390625	\\
0.914147467483464	-0.0040283203125	\\
0.914191858658499	-0.003936767578125	\\
0.914236249833533	-0.003692626953125	\\
0.914280641008568	-0.003021240234375	\\
0.914325032183602	-0.002899169921875	\\
0.914369423358636	-0.00244140625	\\
0.914413814533671	-0.001739501953125	\\
0.914458205708705	-0.001373291015625	\\
0.914502596883739	-0.001251220703125	\\
0.914546988058774	-0.000732421875	\\
0.914591379233808	-0.0009765625	\\
0.914635770408843	-0.001220703125	\\
0.914680161583877	-0.00140380859375	\\
0.914724552758912	-0.001739501953125	\\
0.914768943933946	-0.001434326171875	\\
0.91481333510898	-0.001220703125	\\
0.914857726284015	-0.001373291015625	\\
0.914902117459049	-0.00140380859375	\\
0.914946508634084	-0.001190185546875	\\
0.914990899809118	-0.00115966796875	\\
0.915035290984152	-0.001251220703125	\\
0.915079682159187	-0.000762939453125	\\
0.915124073334221	-0.001190185546875	\\
0.915168464509256	-0.00177001953125	\\
0.91521285568429	-0.00201416015625	\\
0.915257246859324	-0.002288818359375	\\
0.915301638034359	-0.00250244140625	\\
0.915346029209393	-0.002899169921875	\\
0.915390420384428	-0.002777099609375	\\
0.915434811559462	-0.0025634765625	\\
0.915479202734496	-0.00238037109375	\\
0.915523593909531	-0.00238037109375	\\
0.915567985084565	-0.002532958984375	\\
0.9156123762596	-0.0023193359375	\\
0.915656767434634	-0.001922607421875	\\
0.915701158609668	-0.002227783203125	\\
0.915745549784703	-0.002655029296875	\\
0.915789940959737	-0.00299072265625	\\
0.915834332134772	-0.0030517578125	\\
0.915878723309806	-0.00311279296875	\\
0.91592311448484	-0.00335693359375	\\
0.915967505659875	-0.00360107421875	\\
0.916011896834909	-0.003326416015625	\\
0.916056288009944	-0.003021240234375	\\
0.916100679184978	-0.002838134765625	\\
0.916145070360012	-0.00201416015625	\\
0.916189461535047	-0.001495361328125	\\
0.916233852710081	-0.00177001953125	\\
0.916278243885116	-0.001922607421875	\\
0.91632263506015	-0.001800537109375	\\
0.916367026235184	-0.0020751953125	\\
0.916411417410219	-0.001922607421875	\\
0.916455808585253	-0.0018310546875	\\
0.916500199760288	-0.001953125	\\
0.916544590935322	-0.001312255859375	\\
0.916588982110356	-0.0013427734375	\\
0.916633373285391	-0.001556396484375	\\
0.916677764460425	-0.0010986328125	\\
0.91672215563546	-0.00042724609375	\\
0.916766546810494	-6.103515625e-05	\\
0.916810937985528	0.00042724609375	\\
0.916855329160563	0.0008544921875	\\
0.916899720335597	0.000518798828125	\\
0.916944111510632	0.000335693359375	\\
0.916988502685666	9.1552734375e-05	\\
0.917032893860701	0.000244140625	\\
0.917077285035735	0.000732421875	\\
0.917121676210769	0.00048828125	\\
0.917166067385804	0.00018310546875	\\
0.917210458560838	-0.000244140625	\\
0.917254849735872	-0.0001220703125	\\
0.917299240910907	3.0517578125e-05	\\
0.917343632085941	3.0517578125e-05	\\
0.917388023260976	-0.000213623046875	\\
0.91743241443601	-0.00054931640625	\\
0.917476805611045	-0.0001220703125	\\
0.917521196786079	-0.00048828125	\\
0.917565587961113	-0.000762939453125	\\
0.917609979136148	-0.00091552734375	\\
0.917654370311182	-0.001220703125	\\
0.917698761486217	-0.00164794921875	\\
0.917743152661251	-0.002105712890625	\\
0.917787543836285	-0.0025634765625	\\
0.91783193501132	-0.002655029296875	\\
0.917876326186354	-0.00286865234375	\\
0.917920717361389	-0.00274658203125	\\
0.917965108536423	-0.002593994140625	\\
0.918009499711457	-0.0028076171875	\\
0.918053890886492	-0.002349853515625	\\
0.918098282061526	-0.001861572265625	\\
0.918142673236561	-0.002044677734375	\\
0.918187064411595	-0.001861572265625	\\
0.918231455586629	-0.00201416015625	\\
0.918275846761664	-0.002044677734375	\\
0.918320237936698	-0.002105712890625	\\
0.918364629111733	-0.00225830078125	\\
0.918409020286767	-0.002105712890625	\\
0.918453411461801	-0.00213623046875	\\
0.918497802636836	-0.001617431640625	\\
0.91854219381187	-0.001129150390625	\\
0.918586584986905	-0.0013427734375	\\
0.918630976161939	-0.00091552734375	\\
0.918675367336973	-0.000244140625	\\
0.918719758512008	-0.000244140625	\\
0.918764149687042	-3.0517578125e-05	\\
0.918808540862077	-3.0517578125e-05	\\
0.918852932037111	-0.000518798828125	\\
0.918897323212145	-0.00079345703125	\\
0.91894171438718	-0.000518798828125	\\
0.918986105562214	-0.000396728515625	\\
0.919030496737249	-0.001190185546875	\\
0.919074887912283	-0.001373291015625	\\
0.919119279087317	-0.001617431640625	\\
0.919163670262352	-0.0015869140625	\\
0.919208061437386	-0.00201416015625	\\
0.919252452612421	-0.002349853515625	\\
0.919296843787455	-0.002410888671875	\\
0.919341234962489	-0.00238037109375	\\
0.919385626137524	-0.002197265625	\\
0.919430017312558	-0.002532958984375	\\
0.919474408487593	-0.00286865234375	\\
0.919518799662627	-0.003143310546875	\\
0.919563190837661	-0.002960205078125	\\
0.919607582012696	-0.003143310546875	\\
0.91965197318773	-0.0037841796875	\\
0.919696364362765	-0.003509521484375	\\
0.919740755537799	-0.0032958984375	\\
0.919785146712834	-0.00360107421875	\\
0.919829537887868	-0.002838134765625	\\
0.919873929062902	-0.002471923828125	\\
0.919918320237937	-0.003326416015625	\\
0.919962711412971	-0.00347900390625	\\
0.920007102588006	-0.003326416015625	\\
0.92005149376304	-0.00274658203125	\\
0.920095884938074	-0.00201416015625	\\
0.920140276113109	-0.0015869140625	\\
0.920184667288143	-0.00140380859375	\\
0.920229058463177	-0.001251220703125	\\
0.920273449638212	-0.001220703125	\\
0.920317840813246	-0.001373291015625	\\
0.920362231988281	-0.0013427734375	\\
0.920406623163315	-0.001312255859375	\\
0.92045101433835	-0.001190185546875	\\
0.920495405513384	-0.001068115234375	\\
0.920539796688418	-0.00067138671875	\\
0.920584187863453	-0.000457763671875	\\
0.920628579038487	-0.00054931640625	\\
0.920672970213522	0.0001220703125	\\
0.920717361388556	0.000213623046875	\\
0.92076175256359	-0.000457763671875	\\
0.920806143738625	-0.000640869140625	\\
0.920850534913659	-0.000457763671875	\\
0.920894926088694	-0.00079345703125	\\
0.920939317263728	-0.000579833984375	\\
0.920983708438762	-0.000701904296875	\\
0.921028099613797	-0.0010986328125	\\
0.921072490788831	-0.000823974609375	\\
0.921116881963866	-0.000946044921875	\\
0.9211612731389	-0.00091552734375	\\
0.921205664313934	-0.000732421875	\\
0.921250055488969	-0.00079345703125	\\
0.921294446664003	-0.000762939453125	\\
0.921338837839038	-0.001068115234375	\\
0.921383229014072	-0.002044677734375	\\
0.921427620189106	-0.002532958984375	\\
0.921472011364141	-0.003265380859375	\\
0.921516402539175	-0.003509521484375	\\
0.92156079371421	-0.003204345703125	\\
0.921605184889244	-0.003021240234375	\\
0.921649576064278	-0.00286865234375	\\
0.921693967239313	-0.00311279296875	\\
0.921738358414347	-0.0030517578125	\\
0.921782749589382	-0.00286865234375	\\
0.921827140764416	-0.002532958984375	\\
0.92187153193945	-0.002655029296875	\\
0.921915923114485	-0.00286865234375	\\
0.921960314289519	-0.00250244140625	\\
0.922004705464554	-0.002593994140625	\\
0.922049096639588	-0.0028076171875	\\
0.922093487814622	-0.002288818359375	\\
0.922137878989657	-0.00225830078125	\\
0.922182270164691	-0.0023193359375	\\
0.922226661339726	-0.00244140625	\\
0.92227105251476	-0.002227783203125	\\
0.922315443689794	-0.00164794921875	\\
0.922359834864829	-0.0009765625	\\
0.922404226039863	-0.000335693359375	\\
0.922448617214898	-0.000640869140625	\\
0.922493008389932	-0.0006103515625	\\
0.922537399564967	-0.0006103515625	\\
0.922581790740001	-0.00048828125	\\
0.922626181915035	-0.000335693359375	\\
0.92267057309007	-9.1552734375e-05	\\
0.922714964265104	0.000457763671875	\\
0.922759355440139	0.00103759765625	\\
0.922803746615173	0.0018310546875	\\
0.922848137790207	0.001800537109375	\\
0.922892528965242	0.001678466796875	\\
0.922936920140276	0.001617431640625	\\
0.92298131131531	0.001495361328125	\\
0.923025702490345	0.001739501953125	\\
0.923070093665379	0.00146484375	\\
0.923114484840414	0.001007080078125	\\
0.923158876015448	0.001129150390625	\\
0.923203267190483	0.0015869140625	\\
0.923247658365517	0.0015869140625	\\
0.923292049540551	0.00128173828125	\\
0.923336440715586	0.001251220703125	\\
0.92338083189062	0.00146484375	\\
0.923425223065655	0.00189208984375	\\
0.923469614240689	0.001708984375	\\
0.923514005415723	0.0010986328125	\\
0.923558396590758	0.000274658203125	\\
0.923602787765792	0.00042724609375	\\
0.923647178940827	0.000335693359375	\\
0.923691570115861	3.0517578125e-05	\\
0.923735961290895	0.00048828125	\\
0.92378035246593	0.0001220703125	\\
0.923824743640964	0.00048828125	\\
0.923869134815999	0.0010986328125	\\
0.923913525991033	0.001617431640625	\\
0.923957917166067	0.001953125	\\
0.924002308341102	0.001861572265625	\\
0.924046699516136	0.001678466796875	\\
0.924091090691171	0.001373291015625	\\
0.924135481866205	0.001220703125	\\
0.924179873041239	0.001007080078125	\\
0.924224264216274	0.000823974609375	\\
0.924268655391308	0.000823974609375	\\
0.924313046566343	0.001007080078125	\\
0.924357437741377	0.00140380859375	\\
0.924401828916411	0.001983642578125	\\
0.924446220091446	0.00225830078125	\\
0.92449061126648	0.00244140625	\\
0.924535002441515	0.002288818359375	\\
0.924579393616549	0.002105712890625	\\
0.924623784791583	0.002349853515625	\\
0.924668175966618	0.002288818359375	\\
0.924712567141652	0.002227783203125	\\
0.924756958316687	0.0028076171875	\\
0.924801349491721	0.002899169921875	\\
0.924845740666755	0.002593994140625	\\
0.92489013184179	0.00286865234375	\\
0.924934523016824	0.00323486328125	\\
0.924978914191859	0.0032958984375	\\
0.925023305366893	0.003936767578125	\\
0.925067696541927	0.004486083984375	\\
0.925112087716962	0.0042724609375	\\
0.925156478891996	0.004180908203125	\\
0.925200870067031	0.004608154296875	\\
0.925245261242065	0.004486083984375	\\
0.925289652417099	0.003997802734375	\\
0.925334043592134	0.003662109375	\\
0.925378434767168	0.003509521484375	\\
0.925422825942203	0.002960205078125	\\
0.925467217117237	0.00262451171875	\\
0.925511608292272	0.002532958984375	\\
0.925555999467306	0.002288818359375	\\
0.92560039064234	0.002410888671875	\\
0.925644781817375	0.00238037109375	\\
0.925689172992409	0.002166748046875	\\
0.925733564167443	0.00201416015625	\\
0.925777955342478	0.0015869140625	\\
0.925822346517512	0.001373291015625	\\
0.925866737692547	0.0010986328125	\\
0.925911128867581	0.0008544921875	\\
0.925955520042616	6.103515625e-05	\\
0.92599991121765	-0.000640869140625	\\
0.926044302392684	-0.001007080078125	\\
0.926088693567719	-0.00140380859375	\\
0.926133084742753	-0.001312255859375	\\
0.926177475917788	-0.001312255859375	\\
0.926221867092822	-0.000823974609375	\\
0.926266258267856	-0.00079345703125	\\
0.926310649442891	-0.00079345703125	\\
0.926355040617925	-0.000244140625	\\
0.92639943179296	0	\\
0.926443822967994	0	\\
0.926488214143028	-0.000244140625	\\
0.926532605318063	-0.00030517578125	\\
0.926576996493097	-0.0003662109375	\\
0.926621387668132	-0.00030517578125	\\
0.926665778843166	-0.000457763671875	\\
0.9267101700182	3.0517578125e-05	\\
0.926754561193235	0.000335693359375	\\
0.926798952368269	0.000640869140625	\\
0.926843343543304	0.00128173828125	\\
0.926887734718338	0.001556396484375	\\
0.926932125893372	0.00177001953125	\\
0.926976517068407	0.00189208984375	\\
0.927020908243441	0.00152587890625	\\
0.927065299418476	0.001800537109375	\\
0.92710969059351	0.0020751953125	\\
0.927154081768544	0.002166748046875	\\
0.927198472943579	0.002105712890625	\\
0.927242864118613	0.001617431640625	\\
0.927287255293648	0.00128173828125	\\
0.927331646468682	0.001007080078125	\\
0.927376037643716	0.000701904296875	\\
0.927420428818751	0.00054931640625	\\
0.927464819993785	0.000213623046875	\\
0.92750921116882	0.0001220703125	\\
0.927553602343854	-0.0001220703125	\\
0.927597993518889	-0.0006103515625	\\
0.927642384693923	-0.0006103515625	\\
0.927686775868957	-0.00079345703125	\\
0.927731167043992	-0.0013427734375	\\
0.927775558219026	-0.00164794921875	\\
0.92781994939406	-0.002166748046875	\\
0.927864340569095	-0.00225830078125	\\
0.927908731744129	-0.001953125	\\
0.927953122919164	-0.001739501953125	\\
0.927997514094198	-0.001708984375	\\
0.928041905269232	-0.001953125	\\
0.928086296444267	-0.002105712890625	\\
0.928130687619301	-0.001922607421875	\\
0.928175078794336	-0.001983642578125	\\
0.92821946996937	-0.002227783203125	\\
0.928263861144405	-0.001983642578125	\\
0.928308252319439	-0.0023193359375	\\
0.928352643494473	-0.00225830078125	\\
0.928397034669508	-0.0015869140625	\\
0.928441425844542	-0.001373291015625	\\
0.928485817019577	-0.001220703125	\\
0.928530208194611	-0.00042724609375	\\
0.928574599369645	0.0003662109375	\\
0.92861899054468	0.00018310546875	\\
0.928663381719714	0.00018310546875	\\
0.928707772894748	0.000396728515625	\\
0.928752164069783	0.0001220703125	\\
0.928796555244817	0.000579833984375	\\
0.928840946419852	0.000885009765625	\\
0.928885337594886	0.000885009765625	\\
0.928929728769921	0.001007080078125	\\
0.928974119944955	0.0009765625	\\
0.929018511119989	0.0009765625	\\
0.929062902295024	0.0008544921875	\\
0.929107293470058	0.00048828125	\\
0.929151684645093	0.000701904296875	\\
0.929196075820127	0.000213623046875	\\
0.929240466995161	-0.000213623046875	\\
0.929284858170196	-0.0003662109375	\\
0.92932924934523	-0.000823974609375	\\
0.929373640520265	-0.00128173828125	\\
0.929418031695299	-0.0015869140625	\\
0.929462422870333	-0.001708984375	\\
0.929506814045368	-0.0013427734375	\\
0.929551205220402	-0.000946044921875	\\
0.929595596395437	-0.00164794921875	\\
0.929639987570471	-0.00238037109375	\\
0.929684378745505	-0.00262451171875	\\
0.92972876992054	-0.00311279296875	\\
0.929773161095574	-0.0035400390625	\\
0.929817552270609	-0.00408935546875	\\
0.929861943445643	-0.004486083984375	\\
0.929906334620677	-0.00457763671875	\\
0.929950725795712	-0.004669189453125	\\
0.929995116970746	-0.00433349609375	\\
0.930039508145781	-0.003997802734375	\\
0.930083899320815	-0.00408935546875	\\
0.930128290495849	-0.00372314453125	\\
0.930172681670884	-0.00341796875	\\
0.930217072845918	-0.003662109375	\\
0.930261464020953	-0.003143310546875	\\
0.930305855195987	-0.0032958984375	\\
0.930350246371021	-0.00311279296875	\\
0.930394637546056	-0.002716064453125	\\
0.93043902872109	-0.002410888671875	\\
0.930483419896125	-0.001739501953125	\\
0.930527811071159	-0.00140380859375	\\
0.930572202246193	-0.001068115234375	\\
0.930616593421228	-0.000640869140625	\\
0.930660984596262	-0.0003662109375	\\
0.930705375771297	0.00042724609375	\\
0.930749766946331	0.000701904296875	\\
0.930794158121365	0.000823974609375	\\
0.9308385492964	0.00067138671875	\\
0.930882940471434	0.00067138671875	\\
0.930927331646469	0.000762939453125	\\
0.930971722821503	0.0006103515625	\\
0.931016113996538	0.000274658203125	\\
0.931060505171572	0.00030517578125	\\
0.931104896346606	0.00067138671875	\\
0.931149287521641	0.00103759765625	\\
0.931193678696675	0.001922607421875	\\
0.93123806987171	0.002227783203125	\\
0.931282461046744	0.002197265625	\\
0.931326852221778	0.00238037109375	\\
0.931371243396813	0.0020751953125	\\
0.931415634571847	0.0015869140625	\\
0.931460025746881	0.00091552734375	\\
0.931504416921916	0.000457763671875	\\
0.93154880809695	0.000244140625	\\
0.931593199271985	-9.1552734375e-05	\\
0.931637590447019	0	\\
0.931681981622054	0.000274658203125	\\
0.931726372797088	0.000640869140625	\\
0.931770763972122	0.000457763671875	\\
0.931815155147157	0.000640869140625	\\
0.931859546322191	0.000701904296875	\\
0.931903937497226	0.000518798828125	\\
0.93194832867226	0.000518798828125	\\
0.931992719847294	0.00048828125	\\
0.932037111022329	0.000396728515625	\\
0.932081502197363	9.1552734375e-05	\\
0.932125893372398	0.00048828125	\\
0.932170284547432	0.000701904296875	\\
0.932214675722466	0.000457763671875	\\
0.932259066897501	0.0010986328125	\\
0.932303458072535	0.001312255859375	\\
0.93234784924757	0.001220703125	\\
0.932392240422604	0.001068115234375	\\
0.932436631597638	0.0013427734375	\\
0.932481022772673	0.00115966796875	\\
0.932525413947707	0.001068115234375	\\
0.932569805122742	0.0008544921875	\\
0.932614196297776	0.00115966796875	\\
0.93265858747281	0.001739501953125	\\
0.932702978647845	0.00103759765625	\\
0.932747369822879	0.001068115234375	\\
0.932791760997914	0.001190185546875	\\
0.932836152172948	0.00115966796875	\\
0.932880543347982	0.0006103515625	\\
0.932924934523017	0.00042724609375	\\
0.932969325698051	0.000274658203125	\\
0.933013716873086	-0.000335693359375	\\
0.93305810804812	-0.00054931640625	\\
0.933102499223154	-0.000701904296875	\\
0.933146890398189	-0.00079345703125	\\
0.933191281573223	-0.0008544921875	\\
0.933235672748258	-0.0009765625	\\
0.933280063923292	-0.001129150390625	\\
0.933324455098326	-0.00067138671875	\\
0.933368846273361	-0.000335693359375	\\
0.933413237448395	-0.00018310546875	\\
0.93345762862343	6.103515625e-05	\\
0.933502019798464	0.0003662109375	\\
0.933546410973498	0.000213623046875	\\
0.933590802148533	-9.1552734375e-05	\\
0.933635193323567	-3.0517578125e-05	\\
0.933679584498602	-0.000396728515625	\\
0.933723975673636	-0.00042724609375	\\
0.93376836684867	-0.00030517578125	\\
0.933812758023705	-0.000518798828125	\\
0.933857149198739	-0.00042724609375	\\
0.933901540373774	-0.000152587890625	\\
0.933945931548808	-0.0001220703125	\\
0.933990322723843	9.1552734375e-05	\\
0.934034713898877	0.00030517578125	\\
0.934079105073911	0.000396728515625	\\
0.934123496248946	0.000274658203125	\\
0.93416788742398	-9.1552734375e-05	\\
0.934212278599015	-0.000396728515625	\\
0.934256669774049	-0.00091552734375	\\
0.934301060949083	-0.0010986328125	\\
0.934345452124118	-0.00128173828125	\\
0.934389843299152	-0.002197265625	\\
0.934434234474187	-0.0023193359375	\\
0.934478625649221	-0.00213623046875	\\
0.934523016824255	-0.00213623046875	\\
0.93456740799929	-0.002044677734375	\\
0.934611799174324	-0.001708984375	\\
0.934656190349359	-0.00164794921875	\\
0.934700581524393	-0.00146484375	\\
0.934744972699427	-0.00115966796875	\\
0.934789363874462	-0.00103759765625	\\
0.934833755049496	-0.001434326171875	\\
0.934878146224531	-0.001251220703125	\\
0.934922537399565	-0.001251220703125	\\
0.934966928574599	-0.00128173828125	\\
0.935011319749634	-0.000823974609375	\\
0.935055710924668	-0.00042724609375	\\
0.935100102099703	0.000274658203125	\\
0.935144493274737	0.000946044921875	\\
0.935188884449771	0.00146484375	\\
0.935233275624806	0.001708984375	\\
0.93527766679984	0.001861572265625	\\
0.935322057974875	0.00146484375	\\
0.935366449149909	0.00115966796875	\\
0.935410840324943	0.001251220703125	\\
0.935455231499978	0.0010986328125	\\
0.935499622675012	0.00091552734375	\\
0.935544013850047	0.000762939453125	\\
0.935588405025081	0.000946044921875	\\
0.935632796200115	0.0013427734375	\\
0.93567718737515	0.00091552734375	\\
0.935721578550184	0.000732421875	\\
0.935765969725219	0.00091552734375	\\
0.935810360900253	0.000457763671875	\\
0.935854752075287	-0.00042724609375	\\
0.935899143250322	-0.000823974609375	\\
0.935943534425356	-0.00146484375	\\
0.935987925600391	-0.001617431640625	\\
0.936032316775425	-0.001617431640625	\\
0.93607670795046	-0.002288818359375	\\
0.936121099125494	-0.002685546875	\\
0.936165490300528	-0.00244140625	\\
0.936209881475563	-0.002838134765625	\\
0.936254272650597	-0.0032958984375	\\
0.936298663825631	-0.002685546875	\\
0.936343055000666	-0.002410888671875	\\
0.9363874461757	-0.002777099609375	\\
0.936431837350735	-0.002593994140625	\\
0.936476228525769	-0.00262451171875	\\
0.936520619700803	-0.0028076171875	\\
0.936565010875838	-0.002685546875	\\
0.936609402050872	-0.00299072265625	\\
0.936653793225907	-0.003173828125	\\
0.936698184400941	-0.00311279296875	\\
0.936742575575976	-0.00286865234375	\\
0.93678696675101	-0.002349853515625	\\
0.936831357926044	-0.0013427734375	\\
0.936875749101079	-0.0006103515625	\\
0.936920140276113	-0.000335693359375	\\
0.936964531451148	0.000213623046875	\\
0.937008922626182	6.103515625e-05	\\
0.937053313801216	6.103515625e-05	\\
0.937097704976251	-9.1552734375e-05	\\
0.937142096151285	-0.00048828125	\\
0.937186487326319	-0.000152587890625	\\
0.937230878501354	3.0517578125e-05	\\
0.937275269676388	0.000579833984375	\\
0.937319660851423	0.001007080078125	\\
0.937364052026457	0.000823974609375	\\
0.937408443201492	0.000701904296875	\\
0.937452834376526	0.000518798828125	\\
0.93749722555156	0.00048828125	\\
0.937541616726595	0.000335693359375	\\
0.937586007901629	-0.000335693359375	\\
0.937630399076664	-0.000579833984375	\\
0.937674790251698	-0.0010986328125	\\
0.937719181426732	-0.001708984375	\\
0.937763572601767	-0.002655029296875	\\
0.937807963776801	-0.002777099609375	\\
0.937852354951836	-0.00238037109375	\\
0.93789674612687	-0.002471923828125	\\
0.937941137301904	-0.002532958984375	\\
0.937985528476939	-0.002838134765625	\\
0.938029919651973	-0.0025634765625	\\
0.938074310827008	-0.002899169921875	\\
0.938118702002042	-0.003082275390625	\\
0.938163093177076	-0.003753662109375	\\
0.938207484352111	-0.00439453125	\\
0.938251875527145	-0.0045166015625	\\
0.93829626670218	-0.00482177734375	\\
0.938340657877214	-0.004486083984375	\\
0.938385049052248	-0.00433349609375	\\
0.938429440227283	-0.004364013671875	\\
0.938473831402317	-0.003936767578125	\\
0.938518222577352	-0.00341796875	\\
0.938562613752386	-0.002960205078125	\\
0.93860700492742	-0.003082275390625	\\
0.938651396102455	-0.003021240234375	\\
0.938695787277489	-0.0028076171875	\\
0.938740178452524	-0.002716064453125	\\
0.938784569627558	-0.0023193359375	\\
0.938828960802592	-0.0018310546875	\\
0.938873351977627	-0.00189208984375	\\
0.938917743152661	-0.00140380859375	\\
0.938962134327696	-0.000885009765625	\\
0.93900652550273	-0.000274658203125	\\
0.939050916677764	-0.0003662109375	\\
0.939095307852799	-0.0003662109375	\\
0.939139699027833	0.00054931640625	\\
0.939184090202868	0.00048828125	\\
0.939228481377902	0.00067138671875	\\
0.939272872552936	0.00054931640625	\\
0.939317263727971	9.1552734375e-05	\\
0.939361654903005	9.1552734375e-05	\\
0.93940604607804	-0.000518798828125	\\
0.939450437253074	-0.00079345703125	\\
0.939494828428109	-0.0008544921875	\\
0.939539219603143	-0.0008544921875	\\
0.939583610778177	-0.00067138671875	\\
0.939628001953212	0.00030517578125	\\
0.939672393128246	0.000640869140625	\\
0.939716784303281	0.000274658203125	\\
0.939761175478315	0.00030517578125	\\
0.939805566653349	-0.000457763671875	\\
0.939849957828384	-0.001129150390625	\\
0.939894349003418	-0.00152587890625	\\
0.939938740178452	-0.001708984375	\\
0.939983131353487	-0.001495361328125	\\
0.940027522528521	-0.0015869140625	\\
0.940071913703556	-0.001373291015625	\\
0.94011630487859	-0.000579833984375	\\
0.940160696053625	-0.00048828125	\\
0.940205087228659	-0.00042724609375	\\
0.940249478403693	-0.00030517578125	\\
0.940293869578728	-0.000335693359375	\\
0.940338260753762	0.0001220703125	\\
0.940382651928797	0.000213623046875	\\
0.940427043103831	0.000152587890625	\\
0.940471434278865	0.000762939453125	\\
0.9405158254539	0.000823974609375	\\
0.940560216628934	0.001129150390625	\\
0.940604607803969	0.001617431640625	\\
0.940648998979003	0.001800537109375	\\
0.940693390154037	0.002471923828125	\\
0.940737781329072	0.002960205078125	\\
0.940782172504106	0.003082275390625	\\
0.940826563679141	0.003082275390625	\\
0.940870954854175	0.0032958984375	\\
0.940915346029209	0.00341796875	\\
0.940959737204244	0.00299072265625	\\
0.941004128379278	0.002716064453125	\\
0.941048519554313	0.002777099609375	\\
0.941092910729347	0.002105712890625	\\
0.941137301904381	0.00164794921875	\\
0.941181693079416	0.0018310546875	\\
0.94122608425445	0.001708984375	\\
0.941270475429485	0.001708984375	\\
0.941314866604519	0.00201416015625	\\
0.941359257779553	0.001678466796875	\\
0.941403648954588	0.00152587890625	\\
0.941448040129622	0.001190185546875	\\
0.941492431304657	0.000518798828125	\\
0.941536822479691	0.00030517578125	\\
0.941581213654725	0.000274658203125	\\
0.94162560482976	0.000457763671875	\\
0.941669996004794	0.00042724609375	\\
0.941714387179829	0.000335693359375	\\
0.941758778354863	0.000762939453125	\\
0.941803169529898	0.000823974609375	\\
0.941847560704932	0.001129150390625	\\
0.941891951879966	0.001495361328125	\\
0.941936343055001	0.001220703125	\\
0.941980734230035	0.001434326171875	\\
0.942025125405069	0.001617431640625	\\
0.942069516580104	0.00128173828125	\\
0.942113907755138	0.001251220703125	\\
0.942158298930173	0.001068115234375	\\
0.942202690105207	0.000579833984375	\\
0.942247081280241	0.00067138671875	\\
0.942291472455276	0.0008544921875	\\
0.94233586363031	0.001251220703125	\\
0.942380254805345	0.001708984375	\\
0.942424645980379	0.001678466796875	\\
0.942469037155414	0.0018310546875	\\
0.942513428330448	0.00213623046875	\\
0.942557819505482	0.002288818359375	\\
0.942602210680517	0.002166748046875	\\
0.942646601855551	0.00146484375	\\
0.942690993030586	0.001007080078125	\\
0.94273538420562	0.0009765625	\\
0.942779775380654	0.00054931640625	\\
0.942824166555689	0.000244140625	\\
0.942868557730723	-0.000213623046875	\\
0.942912948905758	-0.0003662109375	\\
0.942957340080792	-0.00018310546875	\\
0.943001731255826	-0.000396728515625	\\
0.943046122430861	9.1552734375e-05	\\
0.943090513605895	0.000732421875	\\
0.94313490478093	0.000701904296875	\\
0.943179295955964	0.000518798828125	\\
0.943223687130998	0	\\
0.943268078306033	0.0003662109375	\\
0.943312469481067	0.000823974609375	\\
0.943356860656102	0.000885009765625	\\
0.943401251831136	0.00067138671875	\\
0.94344564300617	0.001007080078125	\\
0.943490034181205	0.00164794921875	\\
0.943534425356239	0.001983642578125	\\
0.943578816531274	0.002410888671875	\\
0.943623207706308	0.002105712890625	\\
0.943667598881342	0.002227783203125	\\
0.943711990056377	0.002593994140625	\\
0.943756381231411	0.0025634765625	\\
0.943800772406446	0.0030517578125	\\
0.94384516358148	0.003265380859375	\\
0.943889554756514	0.003204345703125	\\
0.943933945931549	0.003082275390625	\\
0.943978337106583	0.002838134765625	\\
0.944022728281618	0.00360107421875	\\
0.944067119456652	0.00347900390625	\\
0.944111510631686	0.0029296875	\\
0.944155901806721	0.002777099609375	\\
0.944200292981755	0.002197265625	\\
0.94424468415679	0.001861572265625	\\
0.944289075331824	0.0015869140625	\\
0.944333466506858	0.001129150390625	\\
0.944377857681893	0.000701904296875	\\
0.944422248856927	0.0003662109375	\\
0.944466640031962	0.000213623046875	\\
0.944511031206996	-0.000274658203125	\\
0.944555422382031	-0.0006103515625	\\
0.944599813557065	-0.000823974609375	\\
0.944644204732099	-0.0009765625	\\
0.944688595907134	-0.0009765625	\\
0.944732987082168	-0.00079345703125	\\
0.944777378257202	-0.00054931640625	\\
0.944821769432237	-0.00103759765625	\\
0.944866160607271	-0.001190185546875	\\
0.944910551782306	-0.001190185546875	\\
0.94495494295734	-0.001556396484375	\\
0.944999334132374	-0.001617431640625	\\
0.945043725307409	-0.0018310546875	\\
0.945088116482443	-0.00177001953125	\\
0.945132507657478	-0.001129150390625	\\
0.945176898832512	-0.0006103515625	\\
0.945221290007547	-6.103515625e-05	\\
0.945265681182581	0.000640869140625	\\
0.945310072357615	0.00030517578125	\\
0.94535446353265	0.000244140625	\\
0.945398854707684	0.00054931640625	\\
0.945443245882719	0.00042724609375	\\
0.945487637057753	0.00103759765625	\\
0.945532028232787	0.001129150390625	\\
0.945576419407822	0.001068115234375	\\
0.945620810582856	0.001129150390625	\\
0.94566520175789	0.00128173828125	\\
0.945709592932925	0.0018310546875	\\
0.945753984107959	0.0018310546875	\\
0.945798375282994	0.0018310546875	\\
0.945842766458028	0.002288818359375	\\
0.945887157633063	0.00238037109375	\\
0.945931548808097	0.00201416015625	\\
0.945975939983131	0.0015869140625	\\
0.946020331158166	0.0013427734375	\\
0.9460647223332	0.001007080078125	\\
0.946109113508235	0.000579833984375	\\
0.946153504683269	-0.000152587890625	\\
0.946197895858303	-0.000762939453125	\\
0.946242287033338	-0.000457763671875	\\
0.946286678208372	-0.0003662109375	\\
0.946331069383407	-0.00018310546875	\\
0.946375460558441	3.0517578125e-05	\\
0.946419851733475	-9.1552734375e-05	\\
0.94646424290851	-0.000518798828125	\\
0.946508634083544	-0.0006103515625	\\
0.946553025258579	-0.000823974609375	\\
0.946597416433613	-0.001312255859375	\\
0.946641807608647	-0.001678466796875	\\
0.946686198783682	-0.00152587890625	\\
0.946730589958716	-0.001312255859375	\\
0.946774981133751	-0.001495361328125	\\
0.946819372308785	-0.001251220703125	\\
0.946863763483819	-0.001007080078125	\\
0.946908154658854	-0.000244140625	\\
0.946952545833888	-0.000244140625	\\
0.946996937008923	-0.000213623046875	\\
0.947041328183957	0.000213623046875	\\
0.947085719358991	-0.000152587890625	\\
0.947130110534026	-6.103515625e-05	\\
0.94717450170906	-0.00018310546875	\\
0.947218892884095	-9.1552734375e-05	\\
0.947263284059129	0.0001220703125	\\
0.947307675234163	0	\\
0.947352066409198	0.000335693359375	\\
0.947396457584232	0.000457763671875	\\
0.947440848759267	0.001007080078125	\\
0.947485239934301	0.00152587890625	\\
0.947529631109335	0.001007080078125	\\
0.94757402228437	0.00115966796875	\\
0.947618413459404	0.00189208984375	\\
0.947662804634439	0.0013427734375	\\
0.947707195809473	0.001129150390625	\\
0.947751586984507	0.00128173828125	\\
0.947795978159542	0.0006103515625	\\
0.947840369334576	0.00030517578125	\\
0.947884760509611	-9.1552734375e-05	\\
0.947929151684645	-0.000274658203125	\\
0.94797354285968	-0.0003662109375	\\
0.948017934034714	-0.000244140625	\\
0.948062325209748	-0.000244140625	\\
0.948106716384783	-0.000152587890625	\\
0.948151107559817	-3.0517578125e-05	\\
0.948195498734852	0.0001220703125	\\
0.948239889909886	-6.103515625e-05	\\
0.94828428108492	-0.0010986328125	\\
0.948328672259955	-0.00115966796875	\\
0.948373063434989	-0.0013427734375	\\
0.948417454610024	-0.001556396484375	\\
0.948461845785058	-0.0013427734375	\\
0.948506236960092	-0.0009765625	\\
0.948550628135127	-0.000335693359375	\\
0.948595019310161	0.000213623046875	\\
0.948639410485196	0.000762939453125	\\
0.94868380166023	0.001220703125	\\
0.948728192835264	0.00146484375	\\
0.948772584010299	0.001434326171875	\\
0.948816975185333	0.00164794921875	\\
0.948861366360368	0.00146484375	\\
0.948905757535402	0.001556396484375	\\
0.948950148710436	0.0015869140625	\\
0.948994539885471	0.001739501953125	\\
0.949038931060505	0.002288818359375	\\
0.94908332223554	0.002532958984375	\\
0.949127713410574	0.0030517578125	\\
0.949172104585608	0.003448486328125	\\
0.949216495760643	0.003143310546875	\\
0.949260886935677	0.003021240234375	\\
0.949305278110712	0.002777099609375	\\
0.949349669285746	0.002288818359375	\\
0.94939406046078	0.001953125	\\
0.949438451635815	0.000946044921875	\\
0.949482842810849	0.000152587890625	\\
0.949527233985884	0.000244140625	\\
0.949571625160918	-0.000335693359375	\\
0.949616016335952	-0.000152587890625	\\
0.949660407510987	9.1552734375e-05	\\
0.949704798686021	0.000244140625	\\
0.949749189861056	0.00030517578125	\\
0.94979358103609	0.0003662109375	\\
0.949837972211124	0.000518798828125	\\
0.949882363386159	3.0517578125e-05	\\
0.949926754561193	-9.1552734375e-05	\\
0.949971145736228	-0.0006103515625	\\
0.950015536911262	-0.001129150390625	\\
0.950059928086296	-0.001220703125	\\
0.950104319261331	-0.001373291015625	\\
0.950148710436365	-0.000946044921875	\\
0.9501931016114	-0.000335693359375	\\
0.950237492786434	0	\\
0.950281883961469	0.0003662109375	\\
0.950326275136503	0.000946044921875	\\
0.950370666311537	0.00146484375	\\
0.950415057486572	0.001922607421875	\\
0.950459448661606	0.001708984375	\\
0.95050383983664	0.001739501953125	\\
0.950548231011675	0.0018310546875	\\
0.950592622186709	0.001739501953125	\\
0.950637013361744	0.001617431640625	\\
0.950681404536778	0.0013427734375	\\
0.950725795711812	0.0010986328125	\\
0.950770186886847	0.001434326171875	\\
0.950814578061881	0.001983642578125	\\
0.950858969236916	0.001708984375	\\
0.95090336041195	0.001373291015625	\\
0.950947751586985	0.00115966796875	\\
0.950992142762019	0.00079345703125	\\
0.951036533937053	0.000457763671875	\\
0.951080925112088	0.000396728515625	\\
0.951125316287122	6.103515625e-05	\\
0.951169707462157	-0.00030517578125	\\
0.951214098637191	-0.000457763671875	\\
0.951258489812225	-0.000518798828125	\\
0.95130288098726	-0.0008544921875	\\
0.951347272162294	-0.001495361328125	\\
0.951391663337329	-0.0015869140625	\\
0.951436054512363	-0.001678466796875	\\
0.951480445687397	-0.001983642578125	\\
0.951524836862432	-0.00213623046875	\\
0.951569228037466	-0.00213623046875	\\
0.951613619212501	-0.002288818359375	\\
0.951658010387535	-0.0023193359375	\\
0.951702401562569	-0.002227783203125	\\
0.951746792737604	-0.002166748046875	\\
0.951791183912638	-0.002593994140625	\\
0.951835575087673	-0.0023193359375	\\
0.951879966262707	-0.0020751953125	\\
0.951924357437741	-0.0018310546875	\\
0.951968748612776	-0.001190185546875	\\
0.95201313978781	-0.001129150390625	\\
0.952057530962845	-0.00079345703125	\\
0.952101922137879	-0.00030517578125	\\
0.952146313312913	-0.00030517578125	\\
0.952190704487948	-0.00079345703125	\\
0.952235095662982	-0.0008544921875	\\
0.952279486838017	-0.00091552734375	\\
0.952323878013051	-0.000732421875	\\
0.952368269188085	-0.00042724609375	\\
0.95241266036312	-0.00054931640625	\\
0.952457051538154	-0.000885009765625	\\
0.952501442713189	-0.000579833984375	\\
0.952545833888223	-0.00042724609375	\\
0.952590225063257	-0.000579833984375	\\
0.952634616238292	-0.00030517578125	\\
0.952679007413326	-0.000885009765625	\\
0.952723398588361	-0.00146484375	\\
0.952767789763395	-0.0015869140625	\\
0.952812180938429	-0.002227783203125	\\
0.952856572113464	-0.00238037109375	\\
0.952900963288498	-0.002227783203125	\\
0.952945354463533	-0.00262451171875	\\
0.952989745638567	-0.0029296875	\\
0.953034136813602	-0.00347900390625	\\
0.953078527988636	-0.00335693359375	\\
0.95312291916367	-0.003143310546875	\\
0.953167310338705	-0.003143310546875	\\
0.953211701513739	-0.003143310546875	\\
0.953256092688773	-0.00244140625	\\
0.953300483863808	-0.002655029296875	\\
0.953344875038842	-0.003204345703125	\\
0.953389266213877	-0.00311279296875	\\
0.953433657388911	-0.00347900390625	\\
0.953478048563945	-0.003753662109375	\\
0.95352243973898	-0.00372314453125	\\
0.953566830914014	-0.0028076171875	\\
0.953611222089049	-0.002655029296875	\\
0.953655613264083	-0.00262451171875	\\
0.953700004439118	-0.00189208984375	\\
0.953744395614152	-0.001312255859375	\\
0.953788786789186	-0.000518798828125	\\
0.953833177964221	-0.000457763671875	\\
0.953877569139255	-0.000518798828125	\\
0.95392196031429	-0.000823974609375	\\
0.953966351489324	-0.000396728515625	\\
0.954010742664358	-0.000152587890625	\\
0.954055133839393	-9.1552734375e-05	\\
0.954099525014427	0.00054931640625	\\
0.954143916189461	0.000579833984375	\\
0.954188307364496	0.00091552734375	\\
0.95423269853953	0.001251220703125	\\
0.954277089714565	0.0010986328125	\\
0.954321480889599	0.000732421875	\\
0.954365872064634	0.000213623046875	\\
0.954410263239668	-9.1552734375e-05	\\
0.954454654414702	-0.000640869140625	\\
0.954499045589737	-0.0010986328125	\\
0.954543436764771	-0.00115966796875	\\
0.954587827939806	-0.00115966796875	\\
0.95463221911484	-0.001739501953125	\\
0.954676610289874	-0.001800537109375	\\
0.954721001464909	-0.001861572265625	\\
0.954765392639943	-0.001922607421875	\\
0.954809783814978	-0.001922607421875	\\
0.954854174990012	-0.00250244140625	\\
0.954898566165046	-0.00299072265625	\\
0.954942957340081	-0.003570556640625	\\
0.954987348515115	-0.0042724609375	\\
0.95503173969015	-0.0048828125	\\
0.955076130865184	-0.00506591796875	\\
0.955120522040218	-0.00494384765625	\\
0.955164913215253	-0.005035400390625	\\
0.955209304390287	-0.0052490234375	\\
0.955253695565322	-0.00506591796875	\\
0.955298086740356	-0.0048828125	\\
0.95534247791539	-0.00506591796875	\\
0.955386869090425	-0.00445556640625	\\
0.955431260265459	-0.00439453125	\\
0.955475651440494	-0.003997802734375	\\
0.955520042615528	-0.003082275390625	\\
0.955564433790562	-0.002838134765625	\\
0.955608824965597	-0.003021240234375	\\
0.955653216140631	-0.00311279296875	\\
0.955697607315666	-0.002838134765625	\\
0.9557419984907	-0.002471923828125	\\
0.955786389665734	-0.00201416015625	\\
0.955830780840769	-0.001373291015625	\\
0.955875172015803	-0.000885009765625	\\
0.955919563190838	-0.00067138671875	\\
0.955963954365872	-0.0003662109375	\\
0.956008345540906	9.1552734375e-05	\\
0.956052736715941	3.0517578125e-05	\\
0.956097127890975	0	\\
0.95614151906601	0.000274658203125	\\
0.956185910241044	0.000244140625	\\
0.956230301416078	0.000396728515625	\\
0.956274692591113	0.0001220703125	\\
0.956319083766147	-0.0008544921875	\\
0.956363474941182	-0.000732421875	\\
0.956407866116216	-0.000274658203125	\\
0.956452257291251	-0.000213623046875	\\
0.956496648466285	-0.000274658203125	\\
0.956541039641319	0.0001220703125	\\
0.956585430816354	0.000213623046875	\\
0.956629821991388	-0.000457763671875	\\
0.956674213166423	-0.0006103515625	\\
0.956718604341457	-0.00048828125	\\
0.956762995516491	-0.00103759765625	\\
0.956807386691526	-0.001312255859375	\\
0.95685177786656	-0.000823974609375	\\
0.956896169041595	-0.000518798828125	\\
0.956940560216629	-0.000823974609375	\\
0.956984951391663	-0.001007080078125	\\
0.957029342566698	-0.000732421875	\\
0.957073733741732	-0.0001220703125	\\
0.957118124916767	0.000274658203125	\\
0.957162516091801	0.00079345703125	\\
0.957206907266835	0.000885009765625	\\
0.95725129844187	0.000823974609375	\\
0.957295689616904	0.000762939453125	\\
0.957340080791939	0.00079345703125	\\
0.957384471966973	0.00103759765625	\\
0.957428863142007	0.0015869140625	\\
0.957473254317042	0.001983642578125	\\
0.957517645492076	0.00225830078125	\\
0.957562036667111	0.00250244140625	\\
0.957606427842145	0.00274658203125	\\
0.957650819017179	0.002227783203125	\\
0.957695210192214	0.001708984375	\\
0.957739601367248	0.00177001953125	\\
0.957783992542283	0.001220703125	\\
0.957828383717317	0.00103759765625	\\
0.957872774892351	0.000640869140625	\\
0.957917166067386	6.103515625e-05	\\
0.95796155724242	0.000244140625	\\
0.958005948417455	0.000335693359375	\\
0.958050339592489	-0.0001220703125	\\
0.958094730767523	6.103515625e-05	\\
0.958139121942558	0.00042724609375	\\
0.958183513117592	0.0001220703125	\\
0.958227904292627	-0.0001220703125	\\
0.958272295467661	-0.000640869140625	\\
0.958316686642695	-0.0010986328125	\\
0.95836107781773	-0.001190185546875	\\
0.958405468992764	-0.00177001953125	\\
0.958449860167799	-0.001739501953125	\\
0.958494251342833	-0.00177001953125	\\
0.958538642517867	-0.00164794921875	\\
0.958583033692902	-0.001220703125	\\
0.958627424867936	-0.00146484375	\\
0.958671816042971	-0.000946044921875	\\
0.958716207218005	-0.000885009765625	\\
0.95876059839304	-0.000762939453125	\\
0.958804989568074	0	\\
0.958849380743108	0.000396728515625	\\
0.958893771918143	0.000335693359375	\\
0.958938163093177	3.0517578125e-05	\\
0.958982554268211	0.00018310546875	\\
0.959026945443246	0.000152587890625	\\
0.95907133661828	-0.00018310546875	\\
0.959115727793315	-9.1552734375e-05	\\
0.959160118968349	-0.000396728515625	\\
0.959204510143383	-0.000335693359375	\\
0.959248901318418	0.000213623046875	\\
0.959293292493452	9.1552734375e-05	\\
0.959337683668487	-0.000244140625	\\
0.959382074843521	-0.000579833984375	\\
0.959426466018556	-0.000640869140625	\\
0.95947085719359	-0.000640869140625	\\
0.959515248368624	-0.000823974609375	\\
0.959559639543659	-0.0009765625	\\
0.959604030718693	-0.00164794921875	\\
0.959648421893728	-0.00201416015625	\\
0.959692813068762	-0.001739501953125	\\
0.959737204243796	-0.00164794921875	\\
0.959781595418831	-0.001953125	\\
0.959825986593865	-0.002197265625	\\
0.9598703777689	-0.001953125	\\
0.959914768943934	-0.00201416015625	\\
0.959959160118968	-0.001800537109375	\\
0.960003551294003	-0.001617431640625	\\
0.960047942469037	-0.00140380859375	\\
0.960092333644072	-0.0008544921875	\\
0.960136724819106	-0.0008544921875	\\
0.96018111599414	-0.001129150390625	\\
0.960225507169175	-0.00091552734375	\\
0.960269898344209	-0.000457763671875	\\
0.960314289519244	-0.00042724609375	\\
0.960358680694278	3.0517578125e-05	\\
0.960403071869312	0.000762939453125	\\
0.960447463044347	0.0003662109375	\\
0.960491854219381	0.000701904296875	\\
0.960536245394416	0.0006103515625	\\
0.96058063656945	0.001129150390625	\\
0.960625027744484	0.001373291015625	\\
0.960669418919519	0.00079345703125	\\
0.960713810094553	0.001007080078125	\\
0.960758201269588	0.001007080078125	\\
0.960802592444622	0.001129150390625	\\
0.960846983619656	0.001373291015625	\\
0.960891374794691	0.0010986328125	\\
0.960935765969725	0.0009765625	\\
0.96098015714476	0.001190185546875	\\
0.961024548319794	0.00115966796875	\\
0.961068939494828	0.000701904296875	\\
0.961113330669863	0.00030517578125	\\
0.961157721844897	0.000640869140625	\\
0.961202113019932	0.000396728515625	\\
0.961246504194966	-6.103515625e-05	\\
0.96129089537	-0.00042724609375	\\
0.961335286545035	-0.00079345703125	\\
0.961379677720069	-0.00067138671875	\\
0.961424068895104	-0.001129150390625	\\
0.961468460070138	-0.001617431640625	\\
0.961512851245173	-0.001678466796875	\\
0.961557242420207	-0.001922607421875	\\
0.961601633595241	-0.002349853515625	\\
0.961646024770276	-0.002655029296875	\\
0.96169041594531	-0.003082275390625	\\
0.961734807120344	-0.003265380859375	\\
0.961779198295379	-0.002716064453125	\\
0.961823589470413	-0.002685546875	\\
0.961867980645448	-0.002471923828125	\\
0.961912371820482	-0.00201416015625	\\
0.961956762995516	-0.002166748046875	\\
0.962001154170551	-0.00177001953125	\\
0.962045545345585	-0.00128173828125	\\
0.96208993652062	-0.001312255859375	\\
0.962134327695654	-0.000946044921875	\\
0.962178718870689	-0.000762939453125	\\
0.962223110045723	-0.000396728515625	\\
0.962267501220757	9.1552734375e-05	\\
0.962311892395792	0.000213623046875	\\
0.962356283570826	0.000274658203125	\\
0.962400674745861	6.103515625e-05	\\
0.962445065920895	-0.00018310546875	\\
0.962489457095929	0.0001220703125	\\
0.962533848270964	0.000518798828125	\\
0.962578239445998	0.001251220703125	\\
0.962622630621033	0.001129150390625	\\
0.962667021796067	0.001220703125	\\
0.962711412971101	0.0013427734375	\\
0.962755804146136	0.001068115234375	\\
0.96280019532117	0.000701904296875	\\
0.962844586496205	0.000396728515625	\\
0.962888977671239	9.1552734375e-05	\\
0.962933368846273	0.000396728515625	\\
0.962977760021308	0.00018310546875	\\
0.963022151196342	0.000244140625	\\
0.963066542371377	-3.0517578125e-05	\\
0.963110933546411	3.0517578125e-05	\\
0.963155324721445	0.00018310546875	\\
0.96319971589648	-0.000579833984375	\\
0.963244107071514	-0.000335693359375	\\
0.963288498246549	-0.000823974609375	\\
0.963332889421583	-0.00140380859375	\\
0.963377280596617	-0.001312255859375	\\
0.963421671771652	-0.00146484375	\\
0.963466062946686	-0.001495361328125	\\
0.963510454121721	-0.001800537109375	\\
0.963554845296755	-0.0020751953125	\\
0.963599236471789	-0.001861572265625	\\
0.963643627646824	-0.002227783203125	\\
0.963688018821858	-0.002197265625	\\
0.963732409996893	-0.001922607421875	\\
0.963776801171927	-0.001739501953125	\\
0.963821192346961	-0.001556396484375	\\
0.963865583521996	-0.00152587890625	\\
0.96390997469703	-0.00103759765625	\\
0.963954365872065	-0.00067138671875	\\
0.963998757047099	-0.000457763671875	\\
0.964043148222133	-0.000518798828125	\\
0.964087539397168	-0.00048828125	\\
0.964131930572202	-9.1552734375e-05	\\
0.964176321747237	0.0001220703125	\\
0.964220712922271	0.00030517578125	\\
0.964265104097305	0.000701904296875	\\
0.96430949527234	0.0009765625	\\
0.964353886447374	0.00164794921875	\\
0.964398277622409	0.001983642578125	\\
0.964442668797443	0.002197265625	\\
0.964487059972478	0.002685546875	\\
0.964531451147512	0.002716064453125	\\
0.964575842322546	0.0023193359375	\\
0.964620233497581	0.00213623046875	\\
0.964664624672615	0.00274658203125	\\
0.964709015847649	0.0023193359375	\\
0.964753407022684	0.00201416015625	\\
0.964797798197718	0.00189208984375	\\
0.964842189372753	0.001617431640625	\\
0.964886580547787	0.001617431640625	\\
0.964930971722822	0.001556396484375	\\
0.964975362897856	0.001708984375	\\
0.96501975407289	0.001739501953125	\\
0.965064145247925	0.002105712890625	\\
0.965108536422959	0.00164794921875	\\
0.965152927597994	0.00115966796875	\\
0.965197318773028	0.00146484375	\\
0.965241709948062	0.00103759765625	\\
0.965286101123097	0.000701904296875	\\
0.965330492298131	0.000885009765625	\\
0.965374883473166	0.0006103515625	\\
0.9654192746482	0.000823974609375	\\
0.965463665823234	0.00152587890625	\\
0.965508056998269	0.001983642578125	\\
0.965552448173303	0.00225830078125	\\
0.965596839348338	0.00225830078125	\\
0.965641230523372	0.002166748046875	\\
0.965685621698406	0.002044677734375	\\
0.965730012873441	0.002349853515625	\\
0.965774404048475	0.002197265625	\\
0.96581879522351	0.002288818359375	\\
0.965863186398544	0.00262451171875	\\
0.965907577573578	0.003173828125	\\
0.965951968748613	0.00341796875	\\
0.965996359923647	0.003631591796875	\\
0.966040751098682	0.0040283203125	\\
0.966085142273716	0.003631591796875	\\
0.96612953344875	0.003387451171875	\\
0.966173924623785	0.00341796875	\\
0.966218315798819	0.003326416015625	\\
0.966262706973854	0.00311279296875	\\
0.966307098148888	0.003082275390625	\\
0.966351489323922	0.0028076171875	\\
0.966395880498957	0.003082275390625	\\
0.966440271673991	0.003021240234375	\\
0.966484662849026	0.003143310546875	\\
0.96652905402406	0.00299072265625	\\
0.966573445199094	0.002655029296875	\\
0.966617836374129	0.002685546875	\\
0.966662227549163	0.002197265625	\\
0.966706618724198	0.00177001953125	\\
0.966751009899232	0.00146484375	\\
0.966795401074266	0.00128173828125	\\
0.966839792249301	0.001556396484375	\\
0.966884183424335	0.000946044921875	\\
0.96692857459937	0.000579833984375	\\
0.966972965774404	0.000640869140625	\\
0.967017356949438	0.0006103515625	\\
0.967061748124473	0.000823974609375	\\
0.967106139299507	0.00067138671875	\\
0.967150530474542	0.001190185546875	\\
0.967194921649576	0.00146484375	\\
0.967239312824611	0.00115966796875	\\
0.967283703999645	0.00140380859375	\\
0.967328095174679	0.0013427734375	\\
0.967372486349714	0.000396728515625	\\
0.967416877524748	0.0003662109375	\\
0.967461268699782	9.1552734375e-05	\\
0.967505659874817	-0.0001220703125	\\
0.967550051049851	-9.1552734375e-05	\\
0.967594442224886	-0.000152587890625	\\
0.96763883339992	0.000244140625	\\
0.967683224574954	0.000274658203125	\\
0.967727615749989	0.00048828125	\\
0.967772006925023	0.000579833984375	\\
0.967816398100058	0.000396728515625	\\
0.967860789275092	0.000640869140625	\\
0.967905180450127	0.000274658203125	\\
0.967949571625161	-0.0001220703125	\\
0.967993962800195	0.0001220703125	\\
0.96803835397523	0.000274658203125	\\
0.968082745150264	0.0001220703125	\\
0.968127136325299	-0.0001220703125	\\
0.968171527500333	-6.103515625e-05	\\
0.968215918675367	0.0001220703125	\\
0.968260309850402	0.0003662109375	\\
0.968304701025436	0.000885009765625	\\
0.968349092200471	0.000701904296875	\\
0.968393483375505	0.00067138671875	\\
0.968437874550539	0.000701904296875	\\
0.968482265725574	0.0003662109375	\\
0.968526656900608	0.0008544921875	\\
0.968571048075643	0.0020751953125	\\
0.968615439250677	0.002685546875	\\
0.968659830425711	0.00262451171875	\\
0.968704221600746	0.00244140625	\\
0.96874861277578	0.00225830078125	\\
0.968793003950815	0.002288818359375	\\
0.968837395125849	0.002105712890625	\\
0.968881786300883	0.002044677734375	\\
0.968926177475918	0.002166748046875	\\
0.968970568650952	0.00238037109375	\\
0.969014959825987	0.002960205078125	\\
0.969059351001021	0.003204345703125	\\
0.969103742176055	0.003143310546875	\\
0.96914813335109	0.002960205078125	\\
0.969192524526124	0.00225830078125	\\
0.969236915701159	0.001800537109375	\\
0.969281306876193	0.001068115234375	\\
0.969325698051227	0.000640869140625	\\
0.969370089226262	0.00079345703125	\\
0.969414480401296	0.000335693359375	\\
0.969458871576331	-0.000213623046875	\\
0.969503262751365	-0.00018310546875	\\
0.969547653926399	-0.000274658203125	\\
0.969592045101434	-0.000946044921875	\\
0.969636436276468	-0.001617431640625	\\
0.969680827451503	-0.001922607421875	\\
0.969725218626537	-0.0020751953125	\\
0.969769609801571	-0.002105712890625	\\
0.969814000976606	-0.00225830078125	\\
0.96985839215164	-0.002410888671875	\\
0.969902783326675	-0.002532958984375	\\
0.969947174501709	-0.002227783203125	\\
0.969991565676743	-0.00213623046875	\\
0.970035956851778	-0.002227783203125	\\
0.970080348026812	-0.0025634765625	\\
0.970124739201847	-0.00311279296875	\\
0.970169130376881	-0.003204345703125	\\
0.970213521551915	-0.002685546875	\\
0.97025791272695	-0.002593994140625	\\
0.970302303901984	-0.002593994140625	\\
0.970346695077019	-0.002044677734375	\\
0.970391086252053	-0.001800537109375	\\
0.970435477427087	-0.002227783203125	\\
0.970479868602122	-0.001800537109375	\\
0.970524259777156	-0.00146484375	\\
0.970568650952191	-0.00115966796875	\\
0.970613042127225	-0.000732421875	\\
0.97065743330226	-0.00048828125	\\
0.970701824477294	-0.000335693359375	\\
0.970746215652328	0.000274658203125	\\
0.970790606827363	0.000823974609375	\\
0.970834998002397	0.00079345703125	\\
0.970879389177432	0.001007080078125	\\
0.970923780352466	0.001068115234375	\\
0.9709681715275	0.00146484375	\\
0.971012562702535	0.001983642578125	\\
0.971056953877569	0.002044677734375	\\
0.971101345052604	0.001983642578125	\\
0.971145736227638	0.00213623046875	\\
0.971190127402672	0.001861572265625	\\
0.971234518577707	0.001861572265625	\\
0.971278909752741	0.0015869140625	\\
0.971323300927776	0.001220703125	\\
0.97136769210281	0.001251220703125	\\
0.971412083277844	0.001617431640625	\\
0.971456474452879	0.0018310546875	\\
0.971500865627913	0.00140380859375	\\
0.971545256802948	0.00146484375	\\
0.971589647977982	0.00091552734375	\\
0.971634039153016	0.000701904296875	\\
0.971678430328051	0.00048828125	\\
0.971722821503085	9.1552734375e-05	\\
0.97176721267812	-0.00042724609375	\\
0.971811603853154	-0.000518798828125	\\
0.971855995028188	-0.000213623046875	\\
0.971900386203223	-0.000274658203125	\\
0.971944777378257	9.1552734375e-05	\\
0.971989168553292	-0.0001220703125	\\
0.972033559728326	-0.000213623046875	\\
0.97207795090336	-0.00018310546875	\\
0.972122342078395	-0.000518798828125	\\
0.972166733253429	-0.000640869140625	\\
0.972211124428464	-0.00091552734375	\\
0.972255515603498	-0.0009765625	\\
0.972299906778532	-0.00115966796875	\\
0.972344297953567	-0.000823974609375	\\
0.972388689128601	-0.000396728515625	\\
0.972433080303636	-9.1552734375e-05	\\
0.97247747147867	0.000213623046875	\\
0.972521862653704	-9.1552734375e-05	\\
0.972566253828739	-9.1552734375e-05	\\
0.972610645003773	-3.0517578125e-05	\\
0.972655036178808	-6.103515625e-05	\\
0.972699427353842	6.103515625e-05	\\
0.972743818528876	0.00042724609375	\\
0.972788209703911	0.00048828125	\\
0.972832600878945	0.0003662109375	\\
0.97287699205398	0.000396728515625	\\
0.972921383229014	0.00067138671875	\\
0.972965774404049	0.000732421875	\\
0.973010165579083	0.000396728515625	\\
0.973054556754117	0.000885009765625	\\
0.973098947929152	0.00103759765625	\\
0.973143339104186	0.000885009765625	\\
0.97318773027922	0.000518798828125	\\
0.973232121454255	0.0003662109375	\\
0.973276512629289	0.0008544921875	\\
0.973320903804324	0.000213623046875	\\
0.973365294979358	-6.103515625e-05	\\
0.973409686154393	0.000579833984375	\\
0.973454077329427	0.0008544921875	\\
0.973498468504461	0.000457763671875	\\
0.973542859679496	0.000335693359375	\\
0.97358725085453	0.00030517578125	\\
0.973631642029565	0.0001220703125	\\
0.973676033204599	6.103515625e-05	\\
0.973720424379633	-0.0001220703125	\\
0.973764815554668	0	\\
0.973809206729702	-0.000396728515625	\\
0.973853597904737	-3.0517578125e-05	\\
0.973897989079771	0.0001220703125	\\
0.973942380254805	-0.000274658203125	\\
0.97398677142984	-3.0517578125e-05	\\
0.974031162604874	0.0001220703125	\\
0.974075553779909	0.000518798828125	\\
0.974119944954943	0.000518798828125	\\
0.974164336129977	0.000762939453125	\\
0.974208727305012	0.0006103515625	\\
0.974253118480046	0.000701904296875	\\
0.974297509655081	0.001373291015625	\\
0.974341900830115	0.001434326171875	\\
0.974386292005149	0.001708984375	\\
0.974430683180184	0.0018310546875	\\
0.974475074355218	0.001708984375	\\
0.974519465530253	0.00238037109375	\\
0.974563856705287	0.00244140625	\\
0.974608247880321	0.002105712890625	\\
0.974652639055356	0.00177001953125	\\
0.97469703023039	0.001617431640625	\\
0.974741421405425	0.0013427734375	\\
0.974785812580459	0.00103759765625	\\
0.974830203755493	0.00103759765625	\\
0.974874594930528	0.00140380859375	\\
0.974918986105562	0.001922607421875	\\
0.974963377280597	0.0018310546875	\\
0.975007768455631	0.00189208984375	\\
0.975052159630665	0.001556396484375	\\
0.9750965508057	0.001556396484375	\\
0.975140941980734	0.00164794921875	\\
0.975185333155769	0.0009765625	\\
0.975229724330803	0.0013427734375	\\
0.975274115505837	0.00115966796875	\\
0.975318506680872	0.000579833984375	\\
0.975362897855906	0.000640869140625	\\
0.975407289030941	0.000885009765625	\\
0.975451680205975	0.00115966796875	\\
0.975496071381009	0.001129150390625	\\
0.975540462556044	0.000946044921875	\\
0.975584853731078	0.001129150390625	\\
0.975629244906113	0.001068115234375	\\
0.975673636081147	0.00103759765625	\\
0.975718027256182	0.0009765625	\\
0.975762418431216	0.000335693359375	\\
0.97580680960625	-9.1552734375e-05	\\
0.975851200781285	-6.103515625e-05	\\
0.975895591956319	-0.000213623046875	\\
0.975939983131353	-0.00042724609375	\\
0.975984374306388	-0.000213623046875	\\
0.976028765481422	-6.103515625e-05	\\
0.976073156656457	-0.000335693359375	\\
0.976117547831491	-0.000457763671875	\\
0.976161939006525	-3.0517578125e-05	\\
0.97620633018156	-0.0003662109375	\\
0.976250721356594	-0.0010986328125	\\
0.976295112531629	-0.001190185546875	\\
0.976339503706663	-0.00146484375	\\
0.976383894881698	-0.00189208984375	\\
0.976428286056732	-0.00177001953125	\\
0.976472677231766	-0.00164794921875	\\
0.976517068406801	-0.00164794921875	\\
0.976561459581835	-0.0015869140625	\\
0.97660585075687	-0.00140380859375	\\
0.976650241931904	-0.001373291015625	\\
0.976694633106938	-0.001922607421875	\\
0.976739024281973	-0.001983642578125	\\
0.976783415457007	-0.001922607421875	\\
0.976827806632042	-0.00177001953125	\\
0.976872197807076	-0.00115966796875	\\
0.97691658898211	-0.000732421875	\\
0.976960980157145	-0.00054931640625	\\
0.977005371332179	-0.00042724609375	\\
0.977049762507214	6.103515625e-05	\\
0.977094153682248	0	\\
0.977138544857282	-0.00030517578125	\\
0.977182936032317	-0.000152587890625	\\
0.977227327207351	0.0003662109375	\\
0.977271718382386	0.0003662109375	\\
0.97731610955742	0.0008544921875	\\
0.977360500732454	0.001434326171875	\\
0.977404891907489	0.00152587890625	\\
0.977449283082523	0.001617431640625	\\
0.977493674257558	0.00103759765625	\\
0.977538065432592	0.00140380859375	\\
0.977582456607626	0.001251220703125	\\
0.977626847782661	0.001068115234375	\\
0.977671238957695	0.0010986328125	\\
0.97771563013273	0.00042724609375	\\
0.977760021307764	0.000518798828125	\\
0.977804412482798	0.000640869140625	\\
0.977848803657833	3.0517578125e-05	\\
0.977893194832867	-0.000457763671875	\\
0.977937586007902	-0.000701904296875	\\
0.977981977182936	-0.001373291015625	\\
0.97802636835797	-0.001800537109375	\\
0.978070759533005	-0.00177001953125	\\
0.978115150708039	-0.002227783203125	\\
0.978159541883074	-0.002655029296875	\\
0.978203933058108	-0.002471923828125	\\
0.978248324233142	-0.002471923828125	\\
0.978292715408177	-0.0030517578125	\\
0.978337106583211	-0.003509521484375	\\
0.978381497758246	-0.00408935546875	\\
0.97842588893328	-0.003875732421875	\\
0.978470280108314	-0.003997802734375	\\
0.978514671283349	-0.00494384765625	\\
0.978559062458383	-0.0047607421875	\\
0.978603453633418	-0.0047607421875	\\
0.978647844808452	-0.0050048828125	\\
0.978692235983487	-0.004638671875	\\
0.978736627158521	-0.004608154296875	\\
0.978781018333555	-0.00433349609375	\\
0.97882540950859	-0.004119873046875	\\
0.978869800683624	-0.00408935546875	\\
0.978914191858658	-0.00384521484375	\\
0.978958583033693	-0.00341796875	\\
0.979002974208727	-0.002685546875	\\
0.979047365383762	-0.002044677734375	\\
0.979091756558796	-0.001953125	\\
0.979136147733831	-0.001953125	\\
0.979180538908865	-0.00164794921875	\\
0.979224930083899	-0.001312255859375	\\
0.979269321258934	-0.00146484375	\\
0.979313712433968	-0.001007080078125	\\
0.979358103609003	-0.00091552734375	\\
0.979402494784037	-0.000274658203125	\\
0.979446885959071	0.000518798828125	\\
0.979491277134106	0.000274658203125	\\
0.97953566830914	0.0009765625	\\
0.979580059484175	0.001251220703125	\\
0.979624450659209	0.00103759765625	\\
0.979668841834243	0.001129150390625	\\
0.979713233009278	0.000885009765625	\\
0.979757624184312	0.001373291015625	\\
0.979802015359347	0.001708984375	\\
0.979846406534381	0.0013427734375	\\
0.979890797709415	0.001220703125	\\
0.97993518888445	0.001708984375	\\
0.979979580059484	0.00140380859375	\\
0.980023971234519	0.001129150390625	\\
0.980068362409553	0.000732421875	\\
0.980112753584587	0.000640869140625	\\
0.980157144759622	0.000457763671875	\\
0.980201535934656	0.000274658203125	\\
0.980245927109691	-6.103515625e-05	\\
0.980290318284725	-0.000579833984375	\\
0.980334709459759	-0.00048828125	\\
0.980379100634794	-0.000823974609375	\\
0.980423491809828	-0.000701904296875	\\
0.980467882984863	-0.000518798828125	\\
0.980512274159897	-0.000946044921875	\\
0.980556665334931	-0.001373291015625	\\
0.980601056509966	-0.001617431640625	\\
0.980645447685	-0.00164794921875	\\
0.980689838860035	-0.001861572265625	\\
0.980734230035069	-0.001800537109375	\\
0.980778621210103	-0.001800537109375	\\
0.980823012385138	-0.002044677734375	\\
0.980867403560172	-0.00164794921875	\\
0.980911794735207	-0.001800537109375	\\
0.980956185910241	-0.002166748046875	\\
0.981000577085275	-0.001708984375	\\
0.98104496826031	-0.00177001953125	\\
0.981089359435344	-0.001861572265625	\\
0.981133750610379	-0.001678466796875	\\
0.981178141785413	-0.001251220703125	\\
0.981222532960447	-0.00103759765625	\\
0.981266924135482	-0.000946044921875	\\
0.981311315310516	-0.00079345703125	\\
0.981355706485551	-0.00042724609375	\\
0.981400097660585	0.000396728515625	\\
0.98144448883562	0.00079345703125	\\
0.981488880010654	0.000823974609375	\\
0.981533271185688	0.0008544921875	\\
0.981577662360723	0.0008544921875	\\
0.981622053535757	0.000762939453125	\\
0.981666444710791	0.000946044921875	\\
0.981710835885826	0.00115966796875	\\
0.98175522706086	0.001007080078125	\\
0.981799618235895	0.000732421875	\\
0.981844009410929	0.00128173828125	\\
0.981888400585963	0.001922607421875	\\
0.981932791760998	0.001922607421875	\\
0.981977182936032	0.002288818359375	\\
0.982021574111067	0.002410888671875	\\
0.982065965286101	0.002044677734375	\\
0.982110356461136	0.001922607421875	\\
0.98215474763617	0.002044677734375	\\
0.982199138811204	0.00164794921875	\\
0.982243529986239	0.00152587890625	\\
0.982287921161273	0.001678466796875	\\
0.982332312336308	0.0015869140625	\\
0.982376703511342	0.0013427734375	\\
0.982421094686376	0.001190185546875	\\
0.982465485861411	0.0009765625	\\
0.982509877036445	0.00042724609375	\\
0.98255426821148	0.00054931640625	\\
0.982598659386514	0.000518798828125	\\
0.982643050561548	0.000152587890625	\\
0.982687441736583	0.000518798828125	\\
0.982731832911617	0.000762939453125	\\
0.982776224086652	0.0009765625	\\
0.982820615261686	0.001068115234375	\\
0.98286500643672	0.001068115234375	\\
0.982909397611755	0.00054931640625	\\
0.982953788786789	0.000579833984375	\\
0.982998179961824	0.00091552734375	\\
0.983042571136858	0.00067138671875	\\
0.983086962311892	0.001007080078125	\\
0.983131353486927	0.000457763671875	\\
0.983175744661961	0.000396728515625	\\
0.983220135836996	0.000274658203125	\\
0.98326452701203	0.00048828125	\\
0.983308918187064	0.00079345703125	\\
0.983353309362099	0.000885009765625	\\
0.983397700537133	0.001129150390625	\\
0.983442091712168	0.001434326171875	\\
0.983486482887202	0.00146484375	\\
0.983530874062236	0.001556396484375	\\
0.983575265237271	0.002044677734375	\\
0.983619656412305	0.001922607421875	\\
0.98366404758734	0.001922607421875	\\
0.983708438762374	0.00213623046875	\\
0.983752829937408	0.001678466796875	\\
0.983797221112443	0.001495361328125	\\
0.983841612287477	0.001983642578125	\\
0.983886003462512	0.002044677734375	\\
0.983930394637546	0.002105712890625	\\
0.98397478581258	0.002044677734375	\\
0.984019176987615	0.002105712890625	\\
0.984063568162649	0.002197265625	\\
0.984107959337684	0.001922607421875	\\
0.984152350512718	0.00244140625	\\
0.984196741687753	0.00189208984375	\\
0.984241132862787	0.00103759765625	\\
0.984285524037821	0.00152587890625	\\
0.984329915212856	0.00152587890625	\\
0.98437430638789	0.0010986328125	\\
0.984418697562924	0.000457763671875	\\
0.984463088737959	-0.000274658203125	\\
0.984507479912993	-0.000762939453125	\\
0.984551871088028	-0.001068115234375	\\
0.984596262263062	-0.00146484375	\\
0.984640653438096	-0.0018310546875	\\
0.984685044613131	-0.0015869140625	\\
0.984729435788165	-0.00164794921875	\\
0.9847738269632	-0.00201416015625	\\
0.984818218138234	-0.0018310546875	\\
0.984862609313269	-0.00189208984375	\\
0.984907000488303	-0.00177001953125	\\
0.984951391663337	-0.00201416015625	\\
0.984995782838372	-0.0023193359375	\\
0.985040174013406	-0.001953125	\\
0.985084565188441	-0.0015869140625	\\
0.985128956363475	-0.00164794921875	\\
0.985173347538509	-0.00213623046875	\\
0.985217738713544	-0.002044677734375	\\
0.985262129888578	-0.0013427734375	\\
0.985306521063613	-0.001556396484375	\\
0.985350912238647	-0.0013427734375	\\
0.985395303413681	-0.00091552734375	\\
0.985439694588716	-0.0009765625	\\
0.98548408576375	-6.103515625e-05	\\
0.985528476938785	0.000274658203125	\\
0.985572868113819	0.00067138671875	\\
0.985617259288853	0.001190185546875	\\
0.985661650463888	0.000823974609375	\\
0.985706041638922	0.001190185546875	\\
0.985750432813957	0.00140380859375	\\
0.985794823988991	0.002166748046875	\\
0.985839215164025	0.002716064453125	\\
0.98588360633906	0.002960205078125	\\
0.985927997514094	0.0035400390625	\\
0.985972388689129	0.003631591796875	\\
0.986016779864163	0.003265380859375	\\
0.986061171039197	0.00311279296875	\\
0.986105562214232	0.00323486328125	\\
0.986149953389266	0.003204345703125	\\
0.986194344564301	0.00299072265625	\\
0.986238735739335	0.003143310546875	\\
0.986283126914369	0.003448486328125	\\
0.986327518089404	0.002899169921875	\\
0.986371909264438	0.0025634765625	\\
0.986416300439473	0.003021240234375	\\
0.986460691614507	0.002227783203125	\\
0.986505082789541	0.001556396484375	\\
0.986549473964576	0.000885009765625	\\
0.98659386513961	0.000762939453125	\\
0.986638256314645	0.000732421875	\\
0.986682647489679	0.000244140625	\\
0.986727038664713	-0.000396728515625	\\
0.986771429839748	-0.000579833984375	\\
0.986815821014782	-0.0009765625	\\
0.986860212189817	-0.001556396484375	\\
0.986904603364851	-0.00164794921875	\\
0.986948994539885	-0.00189208984375	\\
0.98699338571492	-0.002532958984375	\\
0.987037776889954	-0.00238037109375	\\
0.987082168064989	-0.002227783203125	\\
0.987126559240023	-0.002655029296875	\\
0.987170950415058	-0.002655029296875	\\
0.987215341590092	-0.002349853515625	\\
0.987259732765126	-0.00238037109375	\\
0.987304123940161	-0.002716064453125	\\
0.987348515115195	-0.002471923828125	\\
0.987392906290229	-0.002716064453125	\\
0.987437297465264	-0.002777099609375	\\
0.987481688640298	-0.00213623046875	\\
0.987526079815333	-0.001739501953125	\\
0.987570470990367	-0.001800537109375	\\
0.987614862165402	-0.00115966796875	\\
0.987659253340436	-0.000732421875	\\
0.98770364451547	-0.00103759765625	\\
0.987748035690505	3.0517578125e-05	\\
0.987792426865539	0.000244140625	\\
0.987836818040574	0.000152587890625	\\
0.987881209215608	0.000244140625	\\
0.987925600390642	0.00054931640625	\\
0.987969991565677	0.001251220703125	\\
0.988014382740711	0.001678466796875	\\
0.988058773915746	0.00201416015625	\\
0.98810316509078	0.002105712890625	\\
0.988147556265814	0.002197265625	\\
0.988191947440849	0.00244140625	\\
0.988236338615883	0.002349853515625	\\
0.988280729790918	0.002716064453125	\\
0.988325120965952	0.00274658203125	\\
0.988369512140986	0.00262451171875	\\
0.988413903316021	0.002960205078125	\\
0.988458294491055	0.0030517578125	\\
0.98850268566609	0.0030517578125	\\
0.988547076841124	0.002655029296875	\\
0.988591468016158	0.00250244140625	\\
0.988635859191193	0.0023193359375	\\
0.988680250366227	0.001800537109375	\\
0.988724641541262	0.001708984375	\\
0.988769032716296	0.00140380859375	\\
0.98881342389133	0.00140380859375	\\
0.988857815066365	0.001678466796875	\\
0.988902206241399	0.000823974609375	\\
0.988946597416434	0.00054931640625	\\
0.988990988591468	0.0003662109375	\\
0.989035379766502	-9.1552734375e-05	\\
0.989079770941537	-0.00048828125	\\
0.989124162116571	-0.0009765625	\\
0.989168553291606	-0.00103759765625	\\
0.98921294446664	-0.0013427734375	\\
0.989257335641674	-0.00146484375	\\
0.989301726816709	-0.001495361328125	\\
0.989346117991743	-0.00128173828125	\\
0.989390509166778	-0.001556396484375	\\
0.989434900341812	-0.00189208984375	\\
0.989479291516846	-0.001800537109375	\\
0.989523682691881	-0.002288818359375	\\
0.989568073866915	-0.0020751953125	\\
0.98961246504195	-0.002410888671875	\\
0.989656856216984	-0.002349853515625	\\
0.989701247392018	-0.001922607421875	\\
0.989745638567053	-0.001953125	\\
0.989790029742087	-0.001617431640625	\\
0.989834420917122	-0.001678466796875	\\
0.989878812092156	-0.00152587890625	\\
0.989923203267191	-0.00067138671875	\\
0.989967594442225	-0.00067138671875	\\
0.990011985617259	-0.00079345703125	\\
0.990056376792294	-0.00054931640625	\\
0.990100767967328	-0.000701904296875	\\
0.990145159142362	-0.00042724609375	\\
0.990189550317397	-0.000244140625	\\
0.990233941492431	-0.000274658203125	\\
0.990278332667466	0.00042724609375	\\
0.9903227238425	0.000579833984375	\\
0.990367115017534	0.000885009765625	\\
0.990411506192569	0.00103759765625	\\
0.990455897367603	0.001068115234375	\\
0.990500288542638	0.001373291015625	\\
0.990544679717672	0.00189208984375	\\
0.990589070892707	0.001739501953125	\\
0.990633462067741	0.00128173828125	\\
0.990677853242775	0.001190185546875	\\
0.99072224441781	0.001251220703125	\\
0.990766635592844	0.001220703125	\\
0.990811026767879	0.000823974609375	\\
0.990855417942913	0.000640869140625	\\
0.990899809117947	0.000762939453125	\\
0.990944200292982	0.000244140625	\\
0.990988591468016	0.000274658203125	\\
0.991032982643051	0.00054931640625	\\
0.991077373818085	-0.000244140625	\\
0.991121764993119	-9.1552734375e-05	\\
0.991166156168154	0.000152587890625	\\
0.991210547343188	0.00030517578125	\\
0.991254938518223	0.000274658203125	\\
0.991299329693257	0.000213623046875	\\
0.991343720868291	0.0001220703125	\\
0.991388112043326	0.000152587890625	\\
0.99143250321836	0.000335693359375	\\
0.991476894393395	0.000335693359375	\\
0.991521285568429	0.000457763671875	\\
0.991565676743463	0.000152587890625	\\
0.991610067918498	-9.1552734375e-05	\\
0.991654459093532	0.0003662109375	\\
0.991698850268567	0.00048828125	\\
0.991743241443601	0.00018310546875	\\
0.991787632618635	0.00054931640625	\\
0.99183202379367	0.000579833984375	\\
0.991876414968704	0.00048828125	\\
0.991920806143739	0.00079345703125	\\
0.991965197318773	0.00091552734375	\\
0.992009588493807	0.00128173828125	\\
0.992053979668842	0.00128173828125	\\
0.992098370843876	0.001312255859375	\\
0.992142762018911	0.001312255859375	\\
0.992187153193945	0.00128173828125	\\
0.992231544368979	0.000823974609375	\\
0.992275935544014	0.0006103515625	\\
0.992320326719048	0.000946044921875	\\
0.992364717894083	0.001068115234375	\\
0.992409109069117	0.0006103515625	\\
0.992453500244151	0.000518798828125	\\
0.992497891419186	0.000762939453125	\\
0.99254228259422	0.0008544921875	\\
0.992586673769255	0.000640869140625	\\
0.992631064944289	-0.000244140625	\\
0.992675456119324	-0.000335693359375	\\
0.992719847294358	-0.00030517578125	\\
0.992764238469392	-0.00103759765625	\\
0.992808629644427	-0.001007080078125	\\
0.992853020819461	-0.000946044921875	\\
0.992897411994496	-0.000885009765625	\\
0.99294180316953	-0.001220703125	\\
0.992986194344564	-0.001922607421875	\\
0.993030585519599	-0.001861572265625	\\
0.993074976694633	-0.002166748046875	\\
0.993119367869667	-0.00274658203125	\\
0.993163759044702	-0.002685546875	\\
0.993208150219736	-0.002777099609375	\\
0.993252541394771	-0.002838134765625	\\
0.993296932569805	-0.003265380859375	\\
0.99334132374484	-0.003631591796875	\\
0.993385714919874	-0.003387451171875	\\
0.993430106094908	-0.00311279296875	\\
0.993474497269943	-0.003173828125	\\
0.993518888444977	-0.003448486328125	\\
0.993563279620012	-0.00341796875	\\
0.993607670795046	-0.002899169921875	\\
0.99365206197008	-0.003082275390625	\\
0.993696453145115	-0.003265380859375	\\
0.993740844320149	-0.00323486328125	\\
0.993785235495184	-0.003143310546875	\\
0.993829626670218	-0.003143310546875	\\
0.993874017845252	-0.002899169921875	\\
0.993918409020287	-0.0025634765625	\\
0.993962800195321	-0.0023193359375	\\
0.994007191370356	-0.002105712890625	\\
0.99405158254539	-0.00213623046875	\\
0.994095973720424	-0.001739501953125	\\
0.994140364895459	-0.001678466796875	\\
0.994184756070493	-0.00164794921875	\\
0.994229147245528	-0.0013427734375	\\
0.994273538420562	-0.0009765625	\\
0.994317929595596	-0.000823974609375	\\
0.994362320770631	-0.000518798828125	\\
0.994406711945665	-0.00030517578125	\\
0.9944511031207	-9.1552734375e-05	\\
0.994495494295734	-0.00018310546875	\\
0.994539885470768	-0.000885009765625	\\
0.994584276645803	-0.000396728515625	\\
0.994628667820837	-0.000213623046875	\\
0.994673058995872	-0.000213623046875	\\
0.994717450170906	0.000396728515625	\\
0.99476184134594	-3.0517578125e-05	\\
0.994806232520975	-0.000274658203125	\\
0.994850623696009	0.0001220703125	\\
0.994895014871044	0.000396728515625	\\
0.994939406046078	0.00018310546875	\\
0.994983797221112	-0.000274658203125	\\
0.995028188396147	-0.000579833984375	\\
0.995072579571181	-0.000335693359375	\\
0.995116970746216	-0.00103759765625	\\
0.99516136192125	-0.00140380859375	\\
0.995205753096284	-0.001190185546875	\\
0.995250144271319	-0.001861572265625	\\
0.995294535446353	-0.0023193359375	\\
0.995338926621388	-0.002899169921875	\\
0.995383317796422	-0.00311279296875	\\
0.995427708971456	-0.00311279296875	\\
0.995472100146491	-0.003173828125	\\
0.995516491321525	-0.0030517578125	\\
0.99556088249656	-0.00323486328125	\\
0.995605273671594	-0.003692626953125	\\
0.995649664846629	-0.003936767578125	\\
0.995694056021663	-0.0040283203125	\\
0.995738447196697	-0.00433349609375	\\
0.995782838371732	-0.004638671875	\\
0.995827229546766	-0.0047607421875	\\
0.9958716207218	-0.004547119140625	\\
0.995916011896835	-0.0045166015625	\\
0.995960403071869	-0.00482177734375	\\
0.996004794246904	-0.0048828125	\\
0.996049185421938	-0.00457763671875	\\
0.996093576596973	-0.004302978515625	\\
0.996137967772007	-0.004180908203125	\\
0.996182358947041	-0.0042724609375	\\
0.996226750122076	-0.00390625	\\
0.99627114129711	-0.003448486328125	\\
0.996315532472145	-0.0035400390625	\\
0.996359923647179	-0.003021240234375	\\
0.996404314822213	-0.0023193359375	\\
0.996448705997248	-0.00201416015625	\\
0.996493097172282	-0.002044677734375	\\
0.996537488347317	-0.00164794921875	\\
0.996581879522351	-0.00152587890625	\\
0.996626270697385	-0.001617431640625	\\
0.99667066187242	-0.00146484375	\\
0.996715053047454	-0.00128173828125	\\
0.996759444222489	-0.001220703125	\\
0.996803835397523	-0.000518798828125	\\
0.996848226572557	-0.0003662109375	\\
0.996892617747592	-0.000457763671875	\\
0.996937008922626	0	\\
0.996981400097661	0.000244140625	\\
0.997025791272695	0.0003662109375	\\
0.997070182447729	0.0001220703125	\\
0.997114573622764	9.1552734375e-05	\\
0.997158964797798	0.000213623046875	\\
0.997203355972833	0.00018310546875	\\
0.997247747147867	0.000213623046875	\\
0.997292138322901	0.000213623046875	\\
0.997336529497936	0.000579833984375	\\
0.99738092067297	0.00054931640625	\\
0.997425311848005	0.000518798828125	\\
0.997469703023039	0.000396728515625	\\
0.997514094198073	9.1552734375e-05	\\
0.997558485373108	0	\\
0.997602876548142	-0.00018310546875	\\
0.997647267723177	-0.000579833984375	\\
0.997691658898211	-0.000457763671875	\\
0.997736050073245	-0.000946044921875	\\
0.99778044124828	-0.001373291015625	\\
0.997824832423314	-0.0013427734375	\\
0.997869223598349	-0.001495361328125	\\
0.997913614773383	-0.001495361328125	\\
0.997958005948417	-0.00225830078125	\\
0.998002397123452	-0.00250244140625	\\
0.998046788298486	-0.002471923828125	\\
0.998091179473521	-0.003265380859375	\\
0.998135570648555	-0.003326416015625	\\
0.998179961823589	-0.002777099609375	\\
0.998224352998624	-0.003082275390625	\\
0.998268744173658	-0.003326416015625	\\
0.998313135348693	-0.003021240234375	\\
0.998357526523727	-0.00274658203125	\\
0.998401917698762	-0.002838134765625	\\
0.998446308873796	-0.00323486328125	\\
0.99849070004883	-0.003143310546875	\\
0.998535091223865	-0.003387451171875	\\
0.998579482398899	-0.003875732421875	\\
0.998623873573933	-0.003448486328125	\\
0.998668264748968	-0.00335693359375	\\
0.998712655924002	-0.003326416015625	\\
0.998757047099037	-0.002838134765625	\\
0.998801438274071	-0.002410888671875	\\
0.998845829449105	-0.00262451171875	\\
0.99889022062414	-0.002471923828125	\\
0.998934611799174	-0.00244140625	\\
0.998979002974209	-0.001983642578125	\\
0.999023394149243	-0.00146484375	\\
0.999067785324278	-0.001495361328125	\\
0.999112176499312	-0.00103759765625	\\
0.999156567674346	-0.000885009765625	\\
0.999200958849381	-0.000396728515625	\\
0.999245350024415	-0.00018310546875	\\
0.99928974119945	-0.000274658203125	\\
0.999334132374484	-0.00018310546875	\\
0.999378523549518	0.000244140625	\\
0.999422914724553	-0.000213623046875	\\
0.999467305899587	-0.00030517578125	\\
0.999511697074622	0.000274658203125	\\
0.999556088249656	-6.103515625e-05	\\
0.99960047942469	0.000335693359375	\\
0.999644870599725	0.000823974609375	\\
0.999689261774759	0.000274658203125	\\
0.999733652949794	-0.0001220703125	\\
0.999778044124828	0.000152587890625	\\
0.999822435299862	0.000152587890625	\\
0.999866826474897	-0.000152587890625	\\
0.999911217649931	-0.000274658203125	\\
0.999955608824966	-0.000457763671875	\\
1	-0.00067138671875	\\
};
\end{axis}

\begin{axis}[%
width=\figurewidth,
height=\figureheight,
scale only axis,
xmin=-15000,
xmax=15000,
xlabel={Frequency (in hertz)},
ymin=0,
ymax=0.0004,
at=(plot1.below south west),
anchor=above north west,
title={Magnitude Response}
]
\addplot [color=blue,solid,forget plot]
  table[row sep=crcr]{
-11025	2.382831139998e-06	\\
-11024.0212180398	1.57919026723033e-06	\\
-11023.0424360795	1.88549896637165e-06	\\
-11022.0636541193	1.63914987802218e-06	\\
-11021.0848721591	1.79237561779357e-06	\\
-11020.1060901989	1.74846456587537e-06	\\
-11019.1273082386	1.66564852609171e-06	\\
-11018.1485262784	1.7302617260326e-06	\\
-11017.1697443182	1.7674302516725e-06	\\
-11016.190962358	1.84915042714232e-06	\\
-11015.2121803977	1.58276010716064e-06	\\
-11014.2333984375	1.70565269229972e-06	\\
-11013.2546164773	1.53652900348306e-06	\\
-11012.275834517	1.82053735851266e-06	\\
-11011.2970525568	1.59889718473773e-06	\\
-11010.3182705966	1.91997616873525e-06	\\
-11009.3394886364	1.81310908346389e-06	\\
-11008.3607066761	1.70728039191802e-06	\\
-11007.3819247159	2.00992194388055e-06	\\
-11006.4031427557	1.96463195856825e-06	\\
-11005.4243607955	1.62906739284363e-06	\\
-11004.4455788352	1.59224759307482e-06	\\
-11003.466796875	1.98745616774755e-06	\\
-11002.4880149148	1.5376531948979e-06	\\
-11001.5092329545	1.90067934047073e-06	\\
-11000.5304509943	2.08938071400516e-06	\\
-10999.5516690341	1.41730280359168e-06	\\
-10998.5728870739	1.82914511768868e-06	\\
-10997.5941051136	1.78457242845061e-06	\\
-10996.6153231534	1.73848204203463e-06	\\
-10995.6365411932	1.89510167140853e-06	\\
-10994.657759233	1.90675974517731e-06	\\
-10993.6789772727	1.79675983101457e-06	\\
-10992.7001953125	1.96702358414496e-06	\\
-10991.7214133523	1.87637893320566e-06	\\
-10990.742631392	1.66482900257183e-06	\\
-10989.7638494318	1.51958230990639e-06	\\
-10988.7850674716	1.88320346890536e-06	\\
-10987.8062855114	1.6443897929015e-06	\\
-10986.8275035511	1.67958635247944e-06	\\
-10985.8487215909	1.9064153762883e-06	\\
-10984.8699396307	1.61255084512511e-06	\\
-10983.8911576705	1.64119002663926e-06	\\
-10982.9123757102	2.04886129870701e-06	\\
-10981.93359375	1.81165125065219e-06	\\
-10980.9548117898	1.64721170196128e-06	\\
-10979.9760298295	1.85463384936846e-06	\\
-10978.9972478693	1.61684228772293e-06	\\
-10978.0184659091	1.40575276862163e-06	\\
-10977.0396839489	1.76061502785486e-06	\\
-10976.0609019886	1.77025205770897e-06	\\
-10975.0821200284	1.54432501578006e-06	\\
-10974.1033380682	1.9913149610154e-06	\\
-10973.124556108	1.53322167915034e-06	\\
-10972.1457741477	1.84308122102719e-06	\\
-10971.1669921875	1.71081824612898e-06	\\
-10970.1882102273	2.05389619018803e-06	\\
-10969.209428267	1.65662381035829e-06	\\
-10968.2306463068	2.10344166883946e-06	\\
-10967.2518643466	2.2745196271521e-06	\\
-10966.2730823864	1.80819615464483e-06	\\
-10965.2943004261	1.85837419370972e-06	\\
-10964.3155184659	1.92203036019114e-06	\\
-10963.3367365057	1.70379263513787e-06	\\
-10962.3579545455	2.15925315707898e-06	\\
-10961.3791725852	1.90386931742604e-06	\\
-10960.400390625	2.19736420031175e-06	\\
-10959.4216086648	1.97504292365553e-06	\\
-10958.4428267045	1.98971747763407e-06	\\
-10957.4640447443	1.54185055748742e-06	\\
-10956.4852627841	1.74178279721673e-06	\\
-10955.5064808239	1.91937910915826e-06	\\
-10954.5276988636	1.56055899929925e-06	\\
-10953.5489169034	1.85947443378774e-06	\\
-10952.5701349432	1.62171576725986e-06	\\
-10951.591352983	1.89114354702634e-06	\\
-10950.6125710227	1.89135684768889e-06	\\
-10949.6337890625	2.15174094747517e-06	\\
-10948.6550071023	1.86210097592554e-06	\\
-10947.676225142	2.06586793272144e-06	\\
-10946.6974431818	1.73298175338544e-06	\\
-10945.7186612216	1.78882887772069e-06	\\
-10944.7398792614	1.81106330068532e-06	\\
-10943.7610973011	2.08424342020929e-06	\\
-10942.7823153409	1.88116161497047e-06	\\
-10941.8035333807	1.62210891973172e-06	\\
-10940.8247514205	1.84828224638361e-06	\\
-10939.8459694602	1.68739263682308e-06	\\
-10938.8671875	1.7407511499229e-06	\\
-10937.8884055398	2.24538385184149e-06	\\
-10936.9096235795	1.92985315591813e-06	\\
-10935.9308416193	2.00202815879041e-06	\\
-10934.9520596591	2.25413807224199e-06	\\
-10933.9732776989	2.03746995584854e-06	\\
-10932.9944957386	1.75858607911535e-06	\\
-10932.0157137784	2.21943239321194e-06	\\
-10931.0369318182	1.98828491217804e-06	\\
-10930.058149858	1.56183104795255e-06	\\
-10929.0793678977	2.27223671878636e-06	\\
-10928.1005859375	2.306304731284e-06	\\
-10927.1218039773	2.21190345707398e-06	\\
-10926.143022017	2.28235189331601e-06	\\
-10925.1642400568	2.24525574532931e-06	\\
-10924.1854580966	2.06597491808937e-06	\\
-10923.2066761364	1.9582038083747e-06	\\
-10922.2278941761	2.05819355780554e-06	\\
-10921.2491122159	1.63994186059649e-06	\\
-10920.2703302557	2.03215568934546e-06	\\
-10919.2915482955	2.0872314657289e-06	\\
-10918.3127663352	1.94752029087696e-06	\\
-10917.333984375	2.14134194768153e-06	\\
-10916.3552024148	2.21569594991999e-06	\\
-10915.3764204545	1.9102492482834e-06	\\
-10914.3976384943	2.05463319107212e-06	\\
-10913.4188565341	2.16257789310428e-06	\\
-10912.4400745739	1.80508393732818e-06	\\
-10911.4612926136	2.13391712652767e-06	\\
-10910.4825106534	2.16579471923e-06	\\
-10909.5037286932	1.80421995333844e-06	\\
-10908.524946733	2.30024586361995e-06	\\
-10907.5461647727	2.2260025327634e-06	\\
-10906.5673828125	1.9333702605203e-06	\\
-10905.5886008523	2.15613649214804e-06	\\
-10904.609818892	2.39036033522522e-06	\\
-10903.6310369318	2.06370820208236e-06	\\
-10902.6522549716	2.47839187682531e-06	\\
-10901.6734730114	2.4410632097992e-06	\\
-10900.6946910511	2.07464685321381e-06	\\
-10899.7159090909	2.3599574156114e-06	\\
-10898.7371271307	2.27652648488914e-06	\\
-10897.7583451705	2.20486568628228e-06	\\
-10896.7795632102	2.30778864834228e-06	\\
-10895.80078125	2.86058104815507e-06	\\
-10894.8219992898	1.98939950328963e-06	\\
-10893.8432173295	2.09969429699775e-06	\\
-10892.8644353693	2.62671425690229e-06	\\
-10891.8856534091	2.31475278799604e-06	\\
-10890.9068714489	2.28940568707195e-06	\\
-10889.9280894886	2.47225788022932e-06	\\
-10888.9493075284	2.28703781117691e-06	\\
-10887.9705255682	2.12006650087108e-06	\\
-10886.991743608	2.60322954257957e-06	\\
-10886.0129616477	2.45223880294543e-06	\\
-10885.0341796875	2.33288945654614e-06	\\
-10884.0553977273	2.27096765079753e-06	\\
-10883.076615767	2.62955766532934e-06	\\
-10882.0978338068	2.01635955448363e-06	\\
-10881.1190518466	2.10206868562304e-06	\\
-10880.1402698864	2.54399980881729e-06	\\
-10879.1614879261	2.58053374591049e-06	\\
-10878.1827059659	2.27436986390898e-06	\\
-10877.2039240057	2.54289165250006e-06	\\
-10876.2251420455	2.31544739735438e-06	\\
-10875.2463600852	2.41532464167534e-06	\\
-10874.267578125	2.11849140415536e-06	\\
-10873.2887961648	2.43623164590937e-06	\\
-10872.3100142045	2.19617013248456e-06	\\
-10871.3312322443	2.71076669164146e-06	\\
-10870.3524502841	2.58210015766809e-06	\\
-10869.3736683239	2.45068626282319e-06	\\
-10868.3948863636	2.67432805116116e-06	\\
-10867.4161044034	2.35621950099172e-06	\\
-10866.4373224432	2.60016714182786e-06	\\
-10865.458540483	2.9316325569936e-06	\\
-10864.4797585227	2.59622974818231e-06	\\
-10863.5009765625	2.4232112311446e-06	\\
-10862.5221946023	2.83226064718564e-06	\\
-10861.543412642	2.29634175467575e-06	\\
-10860.5646306818	2.17287915438652e-06	\\
-10859.5858487216	2.39389136266654e-06	\\
-10858.6070667614	2.82682117606697e-06	\\
-10857.6282848011	2.34351218863027e-06	\\
-10856.6495028409	2.80993246624519e-06	\\
-10855.6707208807	2.4702224265477e-06	\\
-10854.6919389205	2.57997828753537e-06	\\
-10853.7131569602	2.87021291431276e-06	\\
-10852.734375	2.30079381849078e-06	\\
-10851.7555930398	2.53886177261601e-06	\\
-10850.7768110795	3.00097592636873e-06	\\
-10849.7980291193	2.8492573923661e-06	\\
-10848.8192471591	2.47586743276488e-06	\\
-10847.8404651989	3.26197838045239e-06	\\
-10846.8616832386	2.8069356703971e-06	\\
-10845.8829012784	2.60233857546815e-06	\\
-10844.9041193182	2.75986258027584e-06	\\
-10843.925337358	2.96370038774889e-06	\\
-10842.9465553977	2.54096089670118e-06	\\
-10841.9677734375	3.03227367259066e-06	\\
-10840.9889914773	3.31481776704911e-06	\\
-10840.010209517	2.68735550794857e-06	\\
-10839.0314275568	3.13605996203518e-06	\\
-10838.0526455966	2.75288749734356e-06	\\
-10837.0738636364	2.87668455279798e-06	\\
-10836.0950816761	2.86739689573502e-06	\\
-10835.1162997159	2.98814653532696e-06	\\
-10834.1375177557	2.59016342201973e-06	\\
-10833.1587357955	3.12475942956101e-06	\\
-10832.1799538352	3.08566911041868e-06	\\
-10831.201171875	3.23084123217212e-06	\\
-10830.2223899148	2.89691705851272e-06	\\
-10829.2436079545	3.22636121769074e-06	\\
-10828.2648259943	2.94984703310998e-06	\\
-10827.2860440341	3.46399185623576e-06	\\
-10826.3072620739	3.04157797236386e-06	\\
-10825.3284801136	2.9428001023721e-06	\\
-10824.3496981534	3.14154719485026e-06	\\
-10823.3709161932	3.15141578059386e-06	\\
-10822.392134233	3.02389597320338e-06	\\
-10821.4133522727	3.3537720308482e-06	\\
-10820.4345703125	3.42412996976772e-06	\\
-10819.4557883523	2.97105418718509e-06	\\
-10818.477006392	2.91098376812843e-06	\\
-10817.4982244318	3.39801520262749e-06	\\
-10816.5194424716	3.27424127371647e-06	\\
-10815.5406605114	3.28803417709915e-06	\\
-10814.5618785511	3.39450345778088e-06	\\
-10813.5830965909	3.30329164323559e-06	\\
-10812.6043146307	2.9699981133835e-06	\\
-10811.6255326705	3.9835405661757e-06	\\
-10810.6467507102	3.1160407430572e-06	\\
-10809.66796875	3.39042494144775e-06	\\
-10808.6891867898	3.18042892629526e-06	\\
-10807.7104048295	3.30821342672633e-06	\\
-10806.7316228693	3.52565945707758e-06	\\
-10805.7528409091	3.37233145470102e-06	\\
-10804.7740589489	3.34335158498983e-06	\\
-10803.7952769886	3.35755419029888e-06	\\
-10802.8164950284	3.35448181181204e-06	\\
-10801.8377130682	3.69114070175614e-06	\\
-10800.858931108	3.4424283753892e-06	\\
-10799.8801491477	3.78404118225166e-06	\\
-10798.9013671875	3.74806428381135e-06	\\
-10797.9225852273	3.4864071088727e-06	\\
-10796.943803267	3.68230470733619e-06	\\
-10795.9650213068	3.49626471902129e-06	\\
-10794.9862393466	3.54346637925441e-06	\\
-10794.0074573864	3.45236789171559e-06	\\
-10793.0286754261	3.79932324126002e-06	\\
-10792.0498934659	3.42238293662287e-06	\\
-10791.0711115057	3.93676458709164e-06	\\
-10790.0923295455	4.00551462098454e-06	\\
-10789.1135475852	3.49545913668947e-06	\\
-10788.134765625	3.96337626014168e-06	\\
-10787.1559836648	3.72273462343559e-06	\\
-10786.1772017045	3.70106445629862e-06	\\
-10785.1984197443	3.83734716394664e-06	\\
-10784.2196377841	3.87113262356232e-06	\\
-10783.2408558239	3.71211752860445e-06	\\
-10782.2620738636	3.82810586254234e-06	\\
-10781.2832919034	3.67091318497332e-06	\\
-10780.3045099432	3.74938792933988e-06	\\
-10779.325727983	3.8071380806201e-06	\\
-10778.3469460227	4.14344500526704e-06	\\
-10777.3681640625	4.02359814489343e-06	\\
-10776.3893821023	3.97587950190883e-06	\\
-10775.410600142	3.9607544633629e-06	\\
-10774.4318181818	4.24805747491516e-06	\\
-10773.4530362216	4.22097200090544e-06	\\
-10772.4742542614	4.08694283616853e-06	\\
-10771.4954723011	4.21089325852965e-06	\\
-10770.5166903409	4.04244288521592e-06	\\
-10769.5379083807	4.50030470990063e-06	\\
-10768.5591264205	3.96062072773027e-06	\\
-10767.5803444602	4.2276930743583e-06	\\
-10766.6015625	4.26161771471288e-06	\\
-10765.6227805398	4.41390166170745e-06	\\
-10764.6439985795	4.33000819478834e-06	\\
-10763.6652166193	4.23029934707069e-06	\\
-10762.6864346591	4.45376380021523e-06	\\
-10761.7076526989	4.08687464518967e-06	\\
-10760.7288707386	4.22029644793975e-06	\\
-10759.7500887784	4.5084948037498e-06	\\
-10758.7713068182	4.49694506691966e-06	\\
-10757.792524858	4.60783737961488e-06	\\
-10756.8137428977	4.09332377365501e-06	\\
-10755.8349609375	4.41655221359263e-06	\\
-10754.8561789773	4.3122206982148e-06	\\
-10753.877397017	4.33438236708181e-06	\\
-10752.8986150568	4.60039242499624e-06	\\
-10751.9198330966	4.21098363120258e-06	\\
-10750.9410511364	4.39840821086577e-06	\\
-10749.9622691761	4.71018706843395e-06	\\
-10748.9834872159	4.65475129468182e-06	\\
-10748.0047052557	5.03378539634621e-06	\\
-10747.0259232955	4.76264758629347e-06	\\
-10746.0471413352	4.85188930301574e-06	\\
-10745.068359375	4.52012245149831e-06	\\
-10744.0895774148	4.81344981040234e-06	\\
-10743.1107954545	4.78842728873649e-06	\\
-10742.1320134943	5.26878456430889e-06	\\
-10741.1532315341	4.89990541841001e-06	\\
-10740.1744495739	5.10811415904822e-06	\\
-10739.1956676136	5.05705309891634e-06	\\
-10738.2168856534	4.74068906030665e-06	\\
-10737.2381036932	5.22424878528017e-06	\\
-10736.259321733	5.0660909074762e-06	\\
-10735.2805397727	5.04712202725649e-06	\\
-10734.3017578125	5.32347200852666e-06	\\
-10733.3229758523	5.23491396526659e-06	\\
-10732.344193892	4.93507788926118e-06	\\
-10731.3654119318	4.96547459168135e-06	\\
-10730.3866299716	5.49619205135845e-06	\\
-10729.4078480114	5.12579935120727e-06	\\
-10728.4290660511	5.45568204958849e-06	\\
-10727.4502840909	5.28962948120597e-06	\\
-10726.4715021307	4.91356998230282e-06	\\
-10725.4927201705	5.20803021579382e-06	\\
-10724.5139382102	5.65316187861065e-06	\\
-10723.53515625	5.30625149363576e-06	\\
-10722.5563742898	5.0100449826666e-06	\\
-10721.5775923295	4.62404216302716e-06	\\
-10720.5988103693	5.43679099265435e-06	\\
-10719.6200284091	5.09750479939941e-06	\\
-10718.6412464489	5.49682606723234e-06	\\
-10717.6624644886	5.53644972177549e-06	\\
-10716.6836825284	5.32439118386068e-06	\\
-10715.7049005682	5.99202883062202e-06	\\
-10714.726118608	5.568648760771e-06	\\
-10713.7473366477	4.92735613558923e-06	\\
-10712.7685546875	5.8651257355078e-06	\\
-10711.7897727273	5.85621304210749e-06	\\
-10710.810990767	5.35536057779727e-06	\\
-10709.8322088068	5.43027016192689e-06	\\
-10708.8534268466	5.99439033416022e-06	\\
-10707.8746448864	5.68446671141601e-06	\\
-10706.8958629261	5.9761626147126e-06	\\
-10705.9170809659	5.63533361262378e-06	\\
-10704.9382990057	6.01676777153589e-06	\\
-10703.9595170455	5.93217111586993e-06	\\
-10702.9807350852	6.15614413588227e-06	\\
-10702.001953125	5.44945228184377e-06	\\
-10701.0231711648	5.66211812244797e-06	\\
-10700.0443892045	5.84096688219434e-06	\\
-10699.0656072443	6.10499672709073e-06	\\
-10698.0868252841	5.85088020980738e-06	\\
-10697.1080433239	5.54911754114668e-06	\\
-10696.1292613636	6.1128767821633e-06	\\
-10695.1504794034	5.8715974500436e-06	\\
-10694.1716974432	6.08165967611417e-06	\\
-10693.192915483	5.80174651435698e-06	\\
-10692.2141335227	6.35600110651785e-06	\\
-10691.2353515625	6.61013812928522e-06	\\
-10690.2565696023	5.94467261290173e-06	\\
-10689.277787642	6.41053109522645e-06	\\
-10688.2990056818	6.12457684352637e-06	\\
-10687.3202237216	6.42974027336798e-06	\\
-10686.3414417614	6.18019281650538e-06	\\
-10685.3626598011	6.40327578932222e-06	\\
-10684.3838778409	6.1011994225268e-06	\\
-10683.4050958807	6.14104622648781e-06	\\
-10682.4263139205	6.43870966031705e-06	\\
-10681.4475319602	6.2251312596696e-06	\\
-10680.46875	6.23965518944774e-06	\\
-10679.4899680398	6.13084508261726e-06	\\
-10678.5111860795	6.14520370533327e-06	\\
-10677.5324041193	6.18761977031525e-06	\\
-10676.5536221591	6.45007147919105e-06	\\
-10675.5748401989	6.66859074851715e-06	\\
-10674.5960582386	6.3042832904196e-06	\\
-10673.6172762784	6.90603729437691e-06	\\
-10672.6384943182	6.51427301305962e-06	\\
-10671.659712358	6.32634082716688e-06	\\
-10670.6809303977	6.61109283645696e-06	\\
-10669.7021484375	6.89029861235799e-06	\\
-10668.7233664773	6.67100690076069e-06	\\
-10667.744584517	6.70536262118289e-06	\\
-10666.7658025568	6.6308899156188e-06	\\
-10665.7870205966	6.36829983781823e-06	\\
-10664.8082386364	6.69022309334174e-06	\\
-10663.8294566761	6.77564067150146e-06	\\
-10662.8506747159	6.91009296628048e-06	\\
-10661.8718927557	6.95118183355358e-06	\\
-10660.8931107955	6.79729977150273e-06	\\
-10659.9143288352	6.89048387227229e-06	\\
-10658.935546875	6.86780133893817e-06	\\
-10657.9567649148	6.90581715109374e-06	\\
-10656.9779829545	6.8314418520945e-06	\\
-10655.9992009943	6.7883308434886e-06	\\
-10655.0204190341	6.97176514928713e-06	\\
-10654.0416370739	6.98671768437451e-06	\\
-10653.0628551136	6.85247426012917e-06	\\
-10652.0840731534	6.89710182094812e-06	\\
-10651.1052911932	7.14719776160804e-06	\\
-10650.126509233	7.03143611989739e-06	\\
-10649.1477272727	7.12828777562948e-06	\\
-10648.1689453125	6.91730656659066e-06	\\
-10647.1901633523	7.31741732378156e-06	\\
-10646.211381392	7.53902736896785e-06	\\
-10645.2325994318	7.04673352837941e-06	\\
-10644.2538174716	7.47262396728037e-06	\\
-10643.2750355114	7.62728159316183e-06	\\
-10642.2962535511	7.224348708445e-06	\\
-10641.3174715909	7.26269463854202e-06	\\
-10640.3386896307	7.50557373143814e-06	\\
-10639.3599076705	7.59281655154384e-06	\\
-10638.3811257102	7.0958624185158e-06	\\
-10637.40234375	8.29848398770053e-06	\\
-10636.4235617898	7.85240758333473e-06	\\
-10635.4447798295	7.75721150053265e-06	\\
-10634.4659978693	7.56172811296094e-06	\\
-10633.4872159091	7.36925717928533e-06	\\
-10632.5084339489	7.30895816647084e-06	\\
-10631.5296519886	8.13867172614011e-06	\\
-10630.5508700284	7.44259460008078e-06	\\
-10629.5720880682	7.71175491481139e-06	\\
-10628.593306108	7.27929362765975e-06	\\
-10627.6145241477	7.61631913484319e-06	\\
-10626.6357421875	7.74120351576938e-06	\\
-10625.6569602273	8.06316754573778e-06	\\
-10624.678178267	8.2871272653104e-06	\\
-10623.6993963068	8.03151338588343e-06	\\
-10622.7206143466	8.17745139522824e-06	\\
-10621.7418323864	8.01761914905042e-06	\\
-10620.7630504261	7.63889707351433e-06	\\
-10619.7842684659	8.01295518370577e-06	\\
-10618.8054865057	7.72626007895359e-06	\\
-10617.8267045455	7.84527999929191e-06	\\
-10616.8479225852	8.29524920560684e-06	\\
-10615.869140625	8.24928902099547e-06	\\
-10614.8903586648	7.99097811528905e-06	\\
-10613.9115767045	8.16243687875135e-06	\\
-10612.9327947443	8.25982512221757e-06	\\
-10611.9540127841	8.13318357928412e-06	\\
-10610.9752308239	8.32193139896164e-06	\\
-10609.9964488636	8.10383931946787e-06	\\
-10609.0176669034	8.06526860470229e-06	\\
-10608.0388849432	8.18794657329525e-06	\\
-10607.060102983	8.2507796299074e-06	\\
-10606.0813210227	8.0943019682772e-06	\\
-10605.1025390625	8.2294931187198e-06	\\
-10604.1237571023	7.88864130045318e-06	\\
-10603.144975142	7.97302963846394e-06	\\
-10602.1661931818	8.58785612206035e-06	\\
-10601.1874112216	8.19947941411362e-06	\\
-10600.2086292614	8.16991114261248e-06	\\
-10599.2298473011	8.49966121881191e-06	\\
-10598.2510653409	8.60081367373513e-06	\\
-10597.2722833807	8.18056108614709e-06	\\
-10596.2935014205	8.81318279478588e-06	\\
-10595.3147194602	8.74451100879593e-06	\\
-10594.3359375	8.59101194470485e-06	\\
-10593.3571555398	8.73420376519555e-06	\\
-10592.3783735795	8.61146947854843e-06	\\
-10591.3995916193	8.35815350448082e-06	\\
-10590.4208096591	8.13184818527078e-06	\\
-10589.4420276989	8.86583914600326e-06	\\
-10588.4632457386	8.3067620214446e-06	\\
-10587.4844637784	8.94523352019843e-06	\\
-10586.5056818182	9.1413418014035e-06	\\
-10585.526899858	8.80293828433356e-06	\\
-10584.5481178977	8.44148975140561e-06	\\
-10583.5693359375	8.47032087226592e-06	\\
-10582.5905539773	8.73501777464851e-06	\\
-10581.611772017	8.64729680804831e-06	\\
-10580.6329900568	9.16546842780482e-06	\\
-10579.6542080966	8.83222627954524e-06	\\
-10578.6754261364	8.95288308399869e-06	\\
-10577.6966441761	9.02805989834768e-06	\\
-10576.7178622159	8.90980556295456e-06	\\
-10575.7390802557	8.87154389792102e-06	\\
-10574.7602982955	8.98964854385372e-06	\\
-10573.7815163352	9.25062131673879e-06	\\
-10572.802734375	9.35664157614143e-06	\\
-10571.8239524148	9.13043469610524e-06	\\
-10570.8451704545	9.30485387726841e-06	\\
-10569.8663884943	9.01572318483008e-06	\\
-10568.8876065341	8.91957746928775e-06	\\
-10567.9088245739	9.5394172963341e-06	\\
-10566.9300426136	9.69744685367984e-06	\\
-10565.9512606534	9.22069708087197e-06	\\
-10564.9724786932	8.75013397650219e-06	\\
-10563.993696733	9.39516983521783e-06	\\
-10563.0149147727	9.20206870693916e-06	\\
-10562.0361328125	9.10127819950984e-06	\\
-10561.0573508523	8.96849585079148e-06	\\
-10560.078568892	9.78193468404961e-06	\\
-10559.0997869318	9.74808429183828e-06	\\
-10558.1210049716	8.91638116745932e-06	\\
-10557.1422230114	9.66301730226409e-06	\\
-10556.1634410511	9.47202345087805e-06	\\
-10555.1846590909	9.40125544927092e-06	\\
-10554.2058771307	9.28529645630699e-06	\\
-10553.2270951705	1.01849029475985e-05	\\
-10552.2483132102	9.15917697919693e-06	\\
-10551.26953125	9.44238234107459e-06	\\
-10550.2907492898	9.97349051631921e-06	\\
-10549.3119673295	9.32450685363611e-06	\\
-10548.3331853693	9.61896480462349e-06	\\
-10547.3544034091	9.69382394206989e-06	\\
-10546.3756214489	9.52247610535405e-06	\\
-10545.3968394886	9.71286430319652e-06	\\
-10544.4180575284	9.45760664838856e-06	\\
-10543.4392755682	9.65358416082333e-06	\\
-10542.460493608	9.53753399052385e-06	\\
-10541.4817116477	1.00425273967553e-05	\\
-10540.5029296875	9.7292625510428e-06	\\
-10539.5241477273	9.80533451201369e-06	\\
-10538.545365767	9.85284848474659e-06	\\
-10537.5665838068	9.89724325760307e-06	\\
-10536.5878018466	9.78455670212192e-06	\\
-10535.6090198864	9.72956332928611e-06	\\
-10534.6302379261	1.00692431172221e-05	\\
-10533.6514559659	9.84733853903062e-06	\\
-10532.6726740057	1.0114611697105e-05	\\
-10531.6938920455	1.00491328651413e-05	\\
-10530.7151100852	9.8289503446951e-06	\\
-10529.736328125	1.07743428573982e-05	\\
-10528.7575461648	9.97615485315229e-06	\\
-10527.7787642045	9.67698253054458e-06	\\
-10526.7999822443	1.0229275879805e-05	\\
-10525.8212002841	1.02728973928074e-05	\\
-10524.8424183239	1.00693388545886e-05	\\
-10523.8636363636	1.00702583816584e-05	\\
-10522.8848544034	1.00564788144558e-05	\\
-10521.9060724432	1.01438540506297e-05	\\
-10520.927290483	1.05294979072162e-05	\\
-10519.9485085227	1.0363402931e-05	\\
-10518.9697265625	1.01706979670687e-05	\\
-10517.9909446023	1.06875464713007e-05	\\
-10517.012162642	1.01109264014372e-05	\\
-10516.0333806818	1.00192549729404e-05	\\
-10515.0545987216	1.04086780551636e-05	\\
-10514.0758167614	1.03365776185075e-05	\\
-10513.0970348011	1.05073721822273e-05	\\
-10512.1182528409	1.04294590458676e-05	\\
-10511.1394708807	1.02933987185442e-05	\\
-10510.1606889205	1.06934189415943e-05	\\
-10509.1819069602	1.04189160981753e-05	\\
-10508.203125	1.04993957921769e-05	\\
-10507.2243430398	1.05564121181387e-05	\\
-10506.2455610795	1.05575207507866e-05	\\
-10505.2667791193	1.06604321647201e-05	\\
-10504.2879971591	1.02432934778253e-05	\\
-10503.3092151989	9.8712069923943e-06	\\
-10502.3304332386	1.08718590485334e-05	\\
-10501.3516512784	1.00950986200624e-05	\\
-10500.3728693182	1.0463044123281e-05	\\
-10499.394087358	1.08681211882812e-05	\\
-10498.4153053977	1.03188603212104e-05	\\
-10497.4365234375	1.04268325709454e-05	\\
-10496.4577414773	1.05837658517425e-05	\\
-10495.478959517	1.02737518708885e-05	\\
-10494.5001775568	1.06309031299456e-05	\\
-10493.5213955966	1.08357982317388e-05	\\
-10492.5426136364	1.05110023277649e-05	\\
-10491.5638316761	1.07768751352427e-05	\\
-10490.5850497159	1.06601216670546e-05	\\
-10489.6062677557	1.03149347856523e-05	\\
-10488.6274857955	1.07429571933989e-05	\\
-10487.6487038352	1.06570273939083e-05	\\
-10486.669921875	1.09244466660592e-05	\\
-10485.6911399148	1.09531169975777e-05	\\
-10484.7123579545	1.08009186337977e-05	\\
-10483.7335759943	1.03556824305416e-05	\\
-10482.7547940341	1.06687241972075e-05	\\
-10481.7760120739	1.07345396792361e-05	\\
-10480.7972301136	1.08198284486043e-05	\\
-10479.8184481534	1.05158990872526e-05	\\
-10478.8396661932	1.08724556916262e-05	\\
-10477.860884233	1.10075997810218e-05	\\
-10476.8821022727	1.07067270108079e-05	\\
-10475.9033203125	1.07795344130803e-05	\\
-10474.9245383523	1.10122336526776e-05	\\
-10473.945756392	1.11180196824164e-05	\\
-10472.9669744318	1.11968004015631e-05	\\
-10471.9881924716	1.07617344261573e-05	\\
-10471.0094105114	1.09687712178326e-05	\\
-10470.0306285511	1.07806252081613e-05	\\
-10469.0518465909	1.0668374438693e-05	\\
-10468.0730646307	1.06205684226831e-05	\\
-10467.0942826705	1.04312307344836e-05	\\
-10466.1155007102	1.0728097020219e-05	\\
-10465.13671875	1.00903466169727e-05	\\
-10464.1579367898	1.10687526509096e-05	\\
-10463.1791548295	1.03802920147841e-05	\\
-10462.2003728693	1.08788151304888e-05	\\
-10461.2215909091	1.08495154624482e-05	\\
-10460.2428089489	1.08069910648451e-05	\\
-10459.2640269886	1.04987800689233e-05	\\
-10458.2852450284	1.0547953539419e-05	\\
-10457.3064630682	1.09430987710077e-05	\\
-10456.327681108	1.05542036138836e-05	\\
-10455.3488991477	1.10685638170762e-05	\\
-10454.3701171875	1.09000961263663e-05	\\
-10453.3913352273	1.11292855797936e-05	\\
-10452.412553267	1.04332023848141e-05	\\
-10451.4337713068	1.07939099014948e-05	\\
-10450.4549893466	1.1106468248542e-05	\\
-10449.4762073864	1.05705917867783e-05	\\
-10448.4974254261	1.15693889853149e-05	\\
-10447.5186434659	1.10128879766399e-05	\\
-10446.5398615057	1.09933867305614e-05	\\
-10445.5610795455	1.09160737924237e-05	\\
-10444.5822975852	1.12569729802973e-05	\\
-10443.603515625	1.06400892516534e-05	\\
-10442.6247336648	1.08726738496554e-05	\\
-10441.6459517045	1.08199669111165e-05	\\
-10440.6671697443	1.07594726604498e-05	\\
-10439.6883877841	1.15257791626727e-05	\\
-10438.7096058239	1.06388592427272e-05	\\
-10437.7308238636	1.06212083169201e-05	\\
-10436.7520419034	1.10136288986489e-05	\\
-10435.7732599432	1.13105966947734e-05	\\
-10434.794477983	1.09328922418489e-05	\\
-10433.8156960227	1.09525702838778e-05	\\
-10432.8369140625	1.11555431123815e-05	\\
-10431.8581321023	1.11611799156858e-05	\\
-10430.879350142	1.09016756602469e-05	\\
-10429.9005681818	1.13735363670867e-05	\\
-10428.9217862216	1.15280289057669e-05	\\
-10427.9430042614	1.13329962264817e-05	\\
-10426.9642223011	1.17961798682372e-05	\\
-10425.9854403409	1.13897689938744e-05	\\
-10425.0066583807	1.14042132731399e-05	\\
-10424.0278764205	1.1070266964448e-05	\\
-10423.0490944602	1.14137171108271e-05	\\
-10422.0703125	1.16699132312284e-05	\\
-10421.0915305398	1.15228812511831e-05	\\
-10420.1127485795	1.14953763384189e-05	\\
-10419.1339666193	1.1911765220755e-05	\\
-10418.1551846591	1.16955286124488e-05	\\
-10417.1764026989	1.122184020614e-05	\\
-10416.1976207386	1.17496182069672e-05	\\
-10415.2188387784	1.17798304515886e-05	\\
-10414.2400568182	1.13616965410974e-05	\\
-10413.261274858	1.16274535985985e-05	\\
-10412.2824928977	1.15331230957855e-05	\\
-10411.3037109375	1.17726062326906e-05	\\
-10410.3249289773	1.22154437238101e-05	\\
-10409.346147017	1.20211810776277e-05	\\
-10408.3673650568	1.15567337053941e-05	\\
-10407.3885830966	1.17679010451561e-05	\\
-10406.4098011364	1.19787768734503e-05	\\
-10405.4310191761	1.20624554659487e-05	\\
-10404.4522372159	1.19169085788704e-05	\\
-10403.4734552557	1.20273793030873e-05	\\
-10402.4946732955	1.17174104665065e-05	\\
-10401.5158913352	1.16443750957045e-05	\\
-10400.537109375	1.21561892581055e-05	\\
-10399.5583274148	1.1783384504399e-05	\\
-10398.5795454545	1.14535208334477e-05	\\
-10397.6007634943	1.18477825498338e-05	\\
-10396.6219815341	1.12560623812492e-05	\\
-10395.6431995739	1.12965783383479e-05	\\
-10394.6644176136	1.19809267445182e-05	\\
-10393.6856356534	1.18822776304452e-05	\\
-10392.7068536932	1.19323710393329e-05	\\
-10391.728071733	1.19175456037312e-05	\\
-10390.7492897727	1.20938929526106e-05	\\
-10389.7705078125	1.1559454086345e-05	\\
-10388.7917258523	1.15620717908692e-05	\\
-10387.812943892	1.24871769135931e-05	\\
-10386.8341619318	1.20897135315975e-05	\\
-10385.8553799716	1.22316152880604e-05	\\
-10384.8765980114	1.22615774110443e-05	\\
-10383.8978160511	1.12536048628771e-05	\\
-10382.9190340909	1.16140105507793e-05	\\
-10381.9402521307	1.17432163570244e-05	\\
-10380.9614701705	1.19497765623624e-05	\\
-10379.9826882102	1.1644305210715e-05	\\
-10379.00390625	1.13334970821555e-05	\\
-10378.0251242898	1.1631040370738e-05	\\
-10377.0463423295	1.17446230039272e-05	\\
-10376.0675603693	1.18111428444453e-05	\\
-10375.0887784091	1.18434186549674e-05	\\
-10374.1099964489	1.16191169769851e-05	\\
-10373.1312144886	1.18964529274419e-05	\\
-10372.1524325284	1.14761419474945e-05	\\
-10371.1736505682	1.18836577224531e-05	\\
-10370.194868608	1.21948791516964e-05	\\
-10369.2160866477	1.19891856918392e-05	\\
-10368.2373046875	1.16988446010744e-05	\\
-10367.2585227273	1.17236300152724e-05	\\
-10366.279740767	1.18797086110581e-05	\\
-10365.3009588068	1.15239863440468e-05	\\
-10364.3221768466	1.2108026933327e-05	\\
-10363.3433948864	1.12375428234562e-05	\\
-10362.3646129261	1.17816205171565e-05	\\
-10361.3858309659	1.16841960171119e-05	\\
-10360.4070490057	1.19012130065962e-05	\\
-10359.4282670455	1.15330810674234e-05	\\
-10358.4494850852	1.23347795016463e-05	\\
-10357.470703125	1.21943366481318e-05	\\
-10356.4919211648	1.13627811421646e-05	\\
-10355.5131392045	1.12903969299839e-05	\\
-10354.5343572443	1.17586949079233e-05	\\
-10353.5555752841	1.20195965486561e-05	\\
-10352.5767933239	1.25235612718482e-05	\\
-10351.5980113636	1.25338220538357e-05	\\
-10350.6192294034	1.1413514761173e-05	\\
-10349.6404474432	1.22621029771466e-05	\\
-10348.661665483	1.19450311951405e-05	\\
-10347.6828835227	1.20031780211388e-05	\\
-10346.7041015625	1.23534132083824e-05	\\
-10345.7253196023	1.21015770825148e-05	\\
-10344.746537642	1.15919416229959e-05	\\
-10343.7677556818	1.22888474598902e-05	\\
-10342.7889737216	1.18806778479926e-05	\\
-10341.8101917614	1.18972327710048e-05	\\
-10340.8314098011	1.20586676094673e-05	\\
-10339.8526278409	1.22101356416215e-05	\\
-10338.8738458807	1.17688052390373e-05	\\
-10337.8950639205	1.17687137904887e-05	\\
-10336.9162819602	1.25309215877505e-05	\\
-10335.9375	1.16336733899614e-05	\\
-10334.9587180398	1.24410013755623e-05	\\
-10333.9799360795	1.28685570376052e-05	\\
-10333.0011541193	1.27803980995204e-05	\\
-10332.0223721591	1.17940697408959e-05	\\
-10331.0435901989	1.2801044474929e-05	\\
-10330.0648082386	1.17801907681501e-05	\\
-10329.0860262784	1.25721272006879e-05	\\
-10328.1072443182	1.23508023502546e-05	\\
-10327.128462358	1.20906823430286e-05	\\
-10326.1496803977	1.16783857670286e-05	\\
-10325.1708984375	1.24733713628895e-05	\\
-10324.1921164773	1.24532649816229e-05	\\
-10323.213334517	1.28581429683001e-05	\\
-10322.2345525568	1.26783189947253e-05	\\
-10321.2557705966	1.22528097917149e-05	\\
-10320.2769886364	1.22419349889294e-05	\\
-10319.2982066761	1.22805422033886e-05	\\
-10318.3194247159	1.17175799294543e-05	\\
-10317.3406427557	1.20314611591976e-05	\\
-10316.3618607955	1.21293499206061e-05	\\
-10315.3830788352	1.22060514286334e-05	\\
-10314.404296875	1.28076568742122e-05	\\
-10313.4255149148	1.25991214857993e-05	\\
-10312.4467329545	1.25715717133037e-05	\\
-10311.4679509943	1.29711570607895e-05	\\
-10310.4891690341	1.25492950945416e-05	\\
-10309.5103870739	1.17703995827243e-05	\\
-10308.5316051136	1.25785766056848e-05	\\
-10307.5528231534	1.21113587564403e-05	\\
-10306.5740411932	1.23133812961969e-05	\\
-10305.595259233	1.22382422959256e-05	\\
-10304.6164772727	1.26934657247936e-05	\\
-10303.6376953125	1.21983554411129e-05	\\
-10302.6589133523	1.28069803271673e-05	\\
-10301.680131392	1.210893123741e-05	\\
-10300.7013494318	1.24977680372393e-05	\\
-10299.7225674716	1.21146142392856e-05	\\
-10298.7437855114	1.29180010933557e-05	\\
-10297.7650035511	1.30797988738105e-05	\\
-10296.7862215909	1.23438509274852e-05	\\
-10295.8074396307	1.24526951325648e-05	\\
-10294.8286576705	1.21507598714855e-05	\\
-10293.8498757102	1.26807038081782e-05	\\
-10292.87109375	1.31275546286303e-05	\\
-10291.8923117898	1.25211103751028e-05	\\
-10290.9135298295	1.17483521747615e-05	\\
-10289.9347478693	1.22713262846786e-05	\\
-10288.9559659091	1.25710365697721e-05	\\
-10287.9771839489	1.25104822259204e-05	\\
-10286.9984019886	1.17204773012758e-05	\\
-10286.0196200284	1.24528417381662e-05	\\
-10285.0408380682	1.27176708145899e-05	\\
-10284.062056108	1.2453482467511e-05	\\
-10283.0832741477	1.18668241089988e-05	\\
-10282.1044921875	1.20308894897384e-05	\\
-10281.1257102273	1.16683940746929e-05	\\
-10280.146928267	1.17500110211212e-05	\\
-10279.1681463068	1.23696739481273e-05	\\
-10278.1893643466	1.20843553530813e-05	\\
-10277.2105823864	1.20062267009582e-05	\\
-10276.2318004261	1.22993087658506e-05	\\
-10275.2530184659	1.15991361558853e-05	\\
-10274.2742365057	1.21501488200242e-05	\\
-10273.2954545455	1.14533399460639e-05	\\
-10272.3166725852	1.20722993033189e-05	\\
-10271.337890625	1.25299845495706e-05	\\
-10270.3591086648	1.14488714503527e-05	\\
-10269.3803267045	1.20794751970399e-05	\\
-10268.4015447443	1.26463428018956e-05	\\
-10267.4227627841	1.26097727761976e-05	\\
-10266.4439808239	1.16661019077074e-05	\\
-10265.4651988636	1.23816794538782e-05	\\
-10264.4864169034	1.19282147990076e-05	\\
-10263.5076349432	1.18157783163321e-05	\\
-10262.528852983	1.23160816074387e-05	\\
-10261.5500710227	1.15800918302727e-05	\\
-10260.5712890625	1.17766310834384e-05	\\
-10259.5925071023	1.20833075050205e-05	\\
-10258.613725142	1.2137576674415e-05	\\
-10257.6349431818	1.20064525055596e-05	\\
-10256.6561612216	1.29749505360639e-05	\\
-10255.6773792614	1.22134124301391e-05	\\
-10254.6985973011	1.21137211591291e-05	\\
-10253.7198153409	1.23541002879802e-05	\\
-10252.7410333807	1.25852483772915e-05	\\
-10251.7622514205	1.22005158395196e-05	\\
-10250.7834694602	1.26139710373703e-05	\\
-10249.8046875	1.20192354845021e-05	\\
-10248.8259055398	1.20930293558362e-05	\\
-10247.8471235795	1.22881484300588e-05	\\
-10246.8683416193	1.25248206323386e-05	\\
-10245.8895596591	1.16483255670905e-05	\\
-10244.9107776989	1.2244590486234e-05	\\
-10243.9319957386	1.29150069702955e-05	\\
-10242.9532137784	1.25352721587568e-05	\\
-10241.9744318182	1.21820821134727e-05	\\
-10240.995649858	1.26687446062693e-05	\\
-10240.0168678977	1.25153982896237e-05	\\
-10239.0380859375	1.29767195286988e-05	\\
-10238.0593039773	1.30045932283365e-05	\\
-10237.080522017	1.2336606842509e-05	\\
-10236.1017400568	1.19315411832624e-05	\\
-10235.1229580966	1.2488139549674e-05	\\
-10234.1441761364	1.25958820728064e-05	\\
-10233.1653941761	1.31029645762499e-05	\\
-10232.1866122159	1.29342550092676e-05	\\
-10231.2078302557	1.2107497344315e-05	\\
-10230.2290482955	1.24924915820283e-05	\\
-10229.2502663352	1.2630389646305e-05	\\
-10228.271484375	1.28052334667037e-05	\\
-10227.2927024148	1.28800747764212e-05	\\
-10226.3139204545	1.31858911077763e-05	\\
-10225.3351384943	1.32636589991766e-05	\\
-10224.3563565341	1.21511683521094e-05	\\
-10223.3775745739	1.30034620082936e-05	\\
-10222.3987926136	1.29722030016551e-05	\\
-10221.4200106534	1.27678172325622e-05	\\
-10220.4412286932	1.23369284847212e-05	\\
-10219.462446733	1.29979106661473e-05	\\
-10218.4836647727	1.29514484721789e-05	\\
-10217.5048828125	1.31558655770378e-05	\\
-10216.5261008523	1.28253779057611e-05	\\
-10215.547318892	1.23378340639791e-05	\\
-10214.5685369318	1.30764802482763e-05	\\
-10213.5897549716	1.32125505329836e-05	\\
-10212.6109730114	1.2756515069029e-05	\\
-10211.6321910511	1.29548311654057e-05	\\
-10210.6534090909	1.32362423036859e-05	\\
-10209.6746271307	1.28618453414262e-05	\\
-10208.6958451705	1.32516906728414e-05	\\
-10207.7170632102	1.32575489813974e-05	\\
-10206.73828125	1.2831683401002e-05	\\
-10205.7594992898	1.27901158514748e-05	\\
-10204.7807173295	1.31977795023386e-05	\\
-10203.8019353693	1.26301812881617e-05	\\
-10202.8231534091	1.28445538533715e-05	\\
-10201.8443714489	1.34167587021743e-05	\\
-10200.8655894886	1.27480453830805e-05	\\
-10199.8868075284	1.29927476627234e-05	\\
-10198.9080255682	1.31353782013768e-05	\\
-10197.929243608	1.31079837974395e-05	\\
-10196.9504616477	1.297502025241e-05	\\
-10195.9716796875	1.32246544893484e-05	\\
-10194.9928977273	1.3339668563158e-05	\\
-10194.014115767	1.39515517087361e-05	\\
-10193.0353338068	1.27063478638599e-05	\\
-10192.0565518466	1.2967614102084e-05	\\
-10191.0777698864	1.27948478407751e-05	\\
-10190.0989879261	1.27559900893912e-05	\\
-10189.1202059659	1.22873315619548e-05	\\
-10188.1414240057	1.33348678454198e-05	\\
-10187.1626420455	1.25711606349861e-05	\\
-10186.1838600852	1.29464660863429e-05	\\
-10185.205078125	1.28814091428533e-05	\\
-10184.2262961648	1.34018413163041e-05	\\
-10183.2475142045	1.2744663393249e-05	\\
-10182.2687322443	1.24324997298518e-05	\\
-10181.2899502841	1.29199366041664e-05	\\
-10180.3111683239	1.2562337114596e-05	\\
-10179.3323863636	1.28727034997573e-05	\\
-10178.3536044034	1.28733977113098e-05	\\
-10177.3748224432	1.31438717083883e-05	\\
-10176.396040483	1.25156230493888e-05	\\
-10175.4172585227	1.23230084718352e-05	\\
-10174.4384765625	1.25359291381852e-05	\\
-10173.4596946023	1.2406673427831e-05	\\
-10172.480912642	1.25654998919452e-05	\\
-10171.5021306818	1.29959402330967e-05	\\
-10170.5233487216	1.24213212712571e-05	\\
-10169.5445667614	1.27863475850986e-05	\\
-10168.5657848011	1.24402577186363e-05	\\
-10167.5870028409	1.2315190195357e-05	\\
-10166.6082208807	1.34747915570264e-05	\\
-10165.6294389205	1.23975391066212e-05	\\
-10164.6506569602	1.30032779987759e-05	\\
-10163.671875	1.27272064135534e-05	\\
-10162.6930930398	1.33434994703649e-05	\\
-10161.7143110795	1.30041558358793e-05	\\
-10160.7355291193	1.25486652036221e-05	\\
-10159.7567471591	1.32961522901601e-05	\\
-10158.7779651989	1.23940479053542e-05	\\
-10157.7991832386	1.31429876776993e-05	\\
-10156.8204012784	1.27756005457968e-05	\\
-10155.8416193182	1.25287589487115e-05	\\
-10154.862837358	1.30728181043495e-05	\\
-10153.8840553977	1.30549602340571e-05	\\
-10152.9052734375	1.3111397106613e-05	\\
-10151.9264914773	1.29774807454022e-05	\\
-10150.947709517	1.36788717639495e-05	\\
-10149.9689275568	1.25547113166547e-05	\\
-10148.9901455966	1.42773518262772e-05	\\
-10148.0113636364	1.35744045889115e-05	\\
-10147.0325816761	1.36257457890315e-05	\\
-10146.0537997159	1.35112160272148e-05	\\
-10145.0750177557	1.38365425960621e-05	\\
-10144.0962357955	1.36966502555313e-05	\\
-10143.1174538352	1.31731145469699e-05	\\
-10142.138671875	1.31504706235358e-05	\\
-10141.1598899148	1.34943904842663e-05	\\
-10140.1811079545	1.34997777931148e-05	\\
-10139.2023259943	1.3303655255281e-05	\\
-10138.2235440341	1.38774029338638e-05	\\
-10137.2447620739	1.31066791202543e-05	\\
-10136.2659801136	1.30922794522568e-05	\\
-10135.2871981534	1.40187161156002e-05	\\
-10134.3084161932	1.31036580250842e-05	\\
-10133.329634233	1.37661188442086e-05	\\
-10132.3508522727	1.39707644142084e-05	\\
-10131.3720703125	1.38452288645003e-05	\\
-10130.3932883523	1.3999836951327e-05	\\
-10129.414506392	1.36096039213493e-05	\\
-10128.4357244318	1.31244011124427e-05	\\
-10127.4569424716	1.41319999556146e-05	\\
-10126.4781605114	1.39487647788693e-05	\\
-10125.4993785511	1.41673560005336e-05	\\
-10124.5205965909	1.36529832037913e-05	\\
-10123.5418146307	1.36739280105998e-05	\\
-10122.5630326705	1.40961382023313e-05	\\
-10121.5842507102	1.35533837920204e-05	\\
-10120.60546875	1.39847739394483e-05	\\
-10119.6266867898	1.40875850641632e-05	\\
-10118.6479048295	1.37618440759962e-05	\\
-10117.6691228693	1.37063376699997e-05	\\
-10116.6903409091	1.37365163513968e-05	\\
-10115.7115589489	1.37020309955332e-05	\\
-10114.7327769886	1.37803316758823e-05	\\
-10113.7539950284	1.37687797024125e-05	\\
-10112.7752130682	1.41583950857462e-05	\\
-10111.796431108	1.42114969983083e-05	\\
-10110.8176491477	1.41447781047017e-05	\\
-10109.8388671875	1.42551928109171e-05	\\
-10108.8600852273	1.40330893229339e-05	\\
-10107.881303267	1.43434913247985e-05	\\
-10106.9025213068	1.36140860010357e-05	\\
-10105.9237393466	1.4365888401237e-05	\\
-10104.9449573864	1.37938225095544e-05	\\
-10103.9661754261	1.42555608546639e-05	\\
-10102.9873934659	1.40862080282219e-05	\\
-10102.0086115057	1.41948995822151e-05	\\
-10101.0298295455	1.41416244113017e-05	\\
-10100.0510475852	1.41523157037717e-05	\\
-10099.072265625	1.51665018761318e-05	\\
-10098.0934836648	1.3924439199207e-05	\\
-10097.1147017045	1.47803157902011e-05	\\
-10096.1359197443	1.47436135598397e-05	\\
-10095.1571377841	1.42035084255079e-05	\\
-10094.1783558239	1.47724281929201e-05	\\
-10093.1995738636	1.37942933190616e-05	\\
-10092.2207919034	1.47754251058993e-05	\\
-10091.2420099432	1.50990330727473e-05	\\
-10090.263227983	1.46353858895647e-05	\\
-10089.2844460227	1.38162530454628e-05	\\
-10088.3056640625	1.46519690482458e-05	\\
-10087.3268821023	1.4745110121222e-05	\\
-10086.348100142	1.50309023119785e-05	\\
-10085.3693181818	1.4761617157965e-05	\\
-10084.3905362216	1.46870499984778e-05	\\
-10083.4117542614	1.51802045604795e-05	\\
-10082.4329723011	1.43078992854671e-05	\\
-10081.4541903409	1.41202571093907e-05	\\
-10080.4754083807	1.41819978557545e-05	\\
-10079.4966264205	1.51860282915028e-05	\\
-10078.5178444602	1.46353485563189e-05	\\
-10077.5390625	1.38281922416433e-05	\\
-10076.5602805398	1.40237263119824e-05	\\
-10075.5814985795	1.48710909834442e-05	\\
-10074.6027166193	1.44701128662032e-05	\\
-10073.6239346591	1.41654144666296e-05	\\
-10072.6451526989	1.50704666669842e-05	\\
-10071.6663707386	1.443912807473e-05	\\
-10070.6875887784	1.48472199639859e-05	\\
-10069.7088068182	1.45081574279146e-05	\\
-10068.730024858	1.41357408190756e-05	\\
-10067.7512428977	1.39049511298868e-05	\\
-10066.7724609375	1.48105223571335e-05	\\
-10065.7936789773	1.4580307678511e-05	\\
-10064.814897017	1.47929813058554e-05	\\
-10063.8361150568	1.45696185803384e-05	\\
-10062.8573330966	1.46014591863664e-05	\\
-10061.8785511364	1.48341489991543e-05	\\
-10060.8997691761	1.54062321537744e-05	\\
-10059.9209872159	1.51025108844239e-05	\\
-10058.9422052557	1.49836553552831e-05	\\
-10057.9634232955	1.54180504138244e-05	\\
-10056.9846413352	1.45185023205995e-05	\\
-10056.005859375	1.45177501947804e-05	\\
-10055.0270774148	1.51022202911963e-05	\\
-10054.0482954545	1.48773077330158e-05	\\
-10053.0695134943	1.41449320019758e-05	\\
-10052.0907315341	1.55542518120699e-05	\\
-10051.1119495739	1.529557282465e-05	\\
-10050.1331676136	1.48215422365235e-05	\\
-10049.1543856534	1.57388242363985e-05	\\
-10048.1756036932	1.56958896998027e-05	\\
-10047.196821733	1.53489175626974e-05	\\
-10046.2180397727	1.5604467800817e-05	\\
-10045.2392578125	1.5880968185983e-05	\\
-10044.2604758523	1.61154436309655e-05	\\
-10043.281693892	1.5162651181535e-05	\\
-10042.3029119318	1.57712993521774e-05	\\
-10041.3241299716	1.55847939607273e-05	\\
-10040.3453480114	1.47897442958079e-05	\\
-10039.3665660511	1.57460289103185e-05	\\
-10038.3877840909	1.56090958758157e-05	\\
-10037.4090021307	1.59419715273299e-05	\\
-10036.4302201705	1.58744773556723e-05	\\
-10035.4514382102	1.54392214041866e-05	\\
-10034.47265625	1.57076669432745e-05	\\
-10033.4938742898	1.59731165175838e-05	\\
-10032.5150923295	1.60574221919623e-05	\\
-10031.5363103693	1.62058826943663e-05	\\
-10030.5575284091	1.65011211409544e-05	\\
-10029.5787464489	1.57680945401531e-05	\\
-10028.5999644886	1.59089822128767e-05	\\
-10027.6211825284	1.56121408905657e-05	\\
-10026.6424005682	1.65855331702925e-05	\\
-10025.663618608	1.56730656359784e-05	\\
-10024.6848366477	1.69904047896216e-05	\\
-10023.7060546875	1.51649937491545e-05	\\
-10022.7272727273	1.62227467097513e-05	\\
-10021.748490767	1.66180349956424e-05	\\
-10020.7697088068	1.60307693048183e-05	\\
-10019.7909268466	1.68719468767727e-05	\\
-10018.8121448864	1.58899680469786e-05	\\
-10017.8333629261	1.62326930294292e-05	\\
-10016.8545809659	1.66259496946608e-05	\\
-10015.8757990057	1.72258474072073e-05	\\
-10014.8970170455	1.70758891337129e-05	\\
-10013.9182350852	1.67202100871583e-05	\\
-10012.939453125	1.7056650380036e-05	\\
-10011.9606711648	1.63787400540486e-05	\\
-10010.9818892045	1.70482435051096e-05	\\
-10010.0031072443	1.74640057389989e-05	\\
-10009.0243252841	1.66430098194357e-05	\\
-10008.0455433239	1.62318866097379e-05	\\
-10007.0667613636	1.78986373920174e-05	\\
-10006.0879794034	1.7279992464874e-05	\\
-10005.1091974432	1.76496983963191e-05	\\
-10004.130415483	1.75749143938791e-05	\\
-10003.1516335227	1.68728783123379e-05	\\
-10002.1728515625	1.68614999712778e-05	\\
-10001.1940696023	1.68700017802051e-05	\\
-10000.215287642	1.86732146519278e-05	\\
-9999.23650568182	1.77136978583879e-05	\\
-9998.25772372159	1.7835131188822e-05	\\
-9997.27894176136	1.74624393176544e-05	\\
-9996.30015980114	1.69073624835266e-05	\\
-9995.32137784091	1.84094181965074e-05	\\
-9994.34259588068	1.73868962934967e-05	\\
-9993.36381392045	1.74480266120187e-05	\\
-9992.38503196023	1.81350089116091e-05	\\
-9991.40625	1.75522174559308e-05	\\
-9990.42746803977	1.79798062302407e-05	\\
-9989.44868607955	1.82536409074475e-05	\\
-9988.46990411932	1.84221586377358e-05	\\
-9987.49112215909	1.85342627670991e-05	\\
-9986.51234019886	1.87542913701307e-05	\\
-9985.53355823864	1.87029888112673e-05	\\
-9984.55477627841	1.83280187226151e-05	\\
-9983.57599431818	1.86208955624963e-05	\\
-9982.59721235795	1.79361816983109e-05	\\
-9981.61843039773	1.82091134337175e-05	\\
-9980.6396484375	1.80140143031713e-05	\\
-9979.66086647727	1.83291695555063e-05	\\
-9978.68208451705	1.8110865505652e-05	\\
-9977.70330255682	1.79391357184179e-05	\\
-9976.72452059659	1.83186985704211e-05	\\
-9975.74573863636	1.88756584278077e-05	\\
-9974.76695667614	1.85010358793467e-05	\\
-9973.78817471591	1.80846922644189e-05	\\
-9972.80939275568	1.77073134857403e-05	\\
-9971.83061079545	1.83458977587045e-05	\\
-9970.85182883523	1.88916866976256e-05	\\
-9969.873046875	1.7984220386228e-05	\\
-9968.89426491477	1.74219917091669e-05	\\
-9967.91548295455	1.83455526269232e-05	\\
-9966.93670099432	1.79831561504202e-05	\\
-9965.95791903409	1.84407316838541e-05	\\
-9964.97913707386	1.91040428447777e-05	\\
-9964.00035511364	1.85371742917336e-05	\\
-9963.02157315341	1.70633209569953e-05	\\
-9962.04279119318	1.86361232999114e-05	\\
-9961.06400923295	1.7867326010221e-05	\\
-9960.08522727273	1.84310087558304e-05	\\
-9959.1064453125	1.82724117293394e-05	\\
-9958.12766335227	1.76943893344136e-05	\\
-9957.14888139205	1.7985066726758e-05	\\
-9956.17009943182	1.85479457748753e-05	\\
-9955.19131747159	1.8601415126851e-05	\\
-9954.21253551136	1.830405154231e-05	\\
-9953.23375355114	1.89840997225576e-05	\\
-9952.25497159091	1.88035820398835e-05	\\
-9951.27618963068	1.88971389187716e-05	\\
-9950.29740767045	1.84880623248721e-05	\\
-9949.31862571023	1.93188932885633e-05	\\
-9948.33984375	1.85471379305961e-05	\\
-9947.36106178977	2.01112534781233e-05	\\
-9946.38227982955	1.84648314982678e-05	\\
-9945.40349786932	1.94551643986849e-05	\\
-9944.42471590909	1.9188340336151e-05	\\
-9943.44593394886	1.89718378301483e-05	\\
-9942.46715198864	1.91214317460171e-05	\\
-9941.48837002841	1.96108114782102e-05	\\
-9940.50958806818	1.91885357342897e-05	\\
-9939.53080610795	1.94695350520704e-05	\\
-9938.55202414773	1.88438811380399e-05	\\
-9937.5732421875	1.96376866444512e-05	\\
-9936.59446022727	2.06667200208652e-05	\\
-9935.61567826705	1.96795029893367e-05	\\
-9934.63689630682	2.03714831220094e-05	\\
-9933.65811434659	1.96502282856558e-05	\\
-9932.67933238636	1.97533472657017e-05	\\
-9931.70055042614	2.05088128284017e-05	\\
-9930.72176846591	2.01427678447659e-05	\\
-9929.74298650568	2.03178882607722e-05	\\
-9928.76420454545	2.04399404922179e-05	\\
-9927.78542258523	2.07713248912759e-05	\\
-9926.806640625	2.00013325610047e-05	\\
-9925.82785866477	2.03479998080767e-05	\\
-9924.84907670455	1.98706845384923e-05	\\
-9923.87029474432	1.98101138242942e-05	\\
-9922.89151278409	2.0122538694748e-05	\\
-9921.91273082386	1.99317080098304e-05	\\
-9920.93394886364	2.06316487904848e-05	\\
-9919.95516690341	2.06217987714791e-05	\\
-9918.97638494318	2.03850215947617e-05	\\
-9917.99760298295	2.02614563005841e-05	\\
-9917.01882102273	2.03950000588286e-05	\\
-9916.0400390625	2.04880473442627e-05	\\
-9915.06125710227	2.04447932729334e-05	\\
-9914.08247514205	2.12068736196551e-05	\\
-9913.10369318182	1.98482052443451e-05	\\
-9912.12491122159	2.02163937999668e-05	\\
-9911.14612926136	2.10226271608984e-05	\\
-9910.16734730114	2.07862022582569e-05	\\
-9909.18856534091	2.11333557389005e-05	\\
-9908.20978338068	2.081991159706e-05	\\
-9907.23100142045	2.12121090745795e-05	\\
-9906.25221946023	2.06018907872431e-05	\\
-9905.2734375	2.05812304377902e-05	\\
-9904.29465553977	2.0547937433076e-05	\\
-9903.31587357955	2.10204873979526e-05	\\
-9902.33709161932	1.99925339857827e-05	\\
-9901.35830965909	2.0909363239346e-05	\\
-9900.37952769886	2.02524771515065e-05	\\
-9899.40074573864	2.11399435592601e-05	\\
-9898.42196377841	2.09328550640051e-05	\\
-9897.44318181818	2.03870511317395e-05	\\
-9896.46439985795	2.09330469623806e-05	\\
-9895.48561789773	2.05114428924761e-05	\\
-9894.5068359375	2.12827269958017e-05	\\
-9893.52805397727	2.19093975682628e-05	\\
-9892.54927201705	2.08816224504401e-05	\\
-9891.57049005682	2.17798162451979e-05	\\
-9890.59170809659	2.0938564258895e-05	\\
-9889.61292613636	2.08639236745216e-05	\\
-9888.63414417614	2.07397857385945e-05	\\
-9887.65536221591	2.0698132641034e-05	\\
-9886.67658025568	2.06696217533632e-05	\\
-9885.69779829545	2.10320776360737e-05	\\
-9884.71901633523	2.05802841643041e-05	\\
-9883.740234375	2.11625295568762e-05	\\
-9882.76145241477	2.08866091770886e-05	\\
-9881.78267045455	2.09220355041507e-05	\\
-9880.80388849432	2.15977467664153e-05	\\
-9879.82510653409	2.06643016457008e-05	\\
-9878.84632457386	2.06783225032989e-05	\\
-9877.86754261364	2.21565744496231e-05	\\
-9876.88876065341	2.13901660934717e-05	\\
-9875.90997869318	1.99980156322153e-05	\\
-9874.93119673295	2.07671646453672e-05	\\
-9873.95241477273	2.10032036388787e-05	\\
-9872.9736328125	2.07765818777065e-05	\\
-9871.99485085227	2.09865621690497e-05	\\
-9871.01606889205	2.08212228615386e-05	\\
-9870.03728693182	2.10996451102609e-05	\\
-9869.05850497159	2.1150932705555e-05	\\
-9868.07972301136	2.06993231505165e-05	\\
-9867.10094105114	2.105044604063e-05	\\
-9866.12215909091	2.12807379480114e-05	\\
-9865.14337713068	2.20004254691251e-05	\\
-9864.16459517045	2.09479315381629e-05	\\
-9863.18581321023	2.1783580632057e-05	\\
-9862.20703125	2.1389675493206e-05	\\
-9861.22824928977	2.07222510630849e-05	\\
-9860.24946732955	2.17305780565628e-05	\\
-9859.27068536932	2.13052143688031e-05	\\
-9858.29190340909	2.06984166478386e-05	\\
-9857.31312144886	2.16644898876326e-05	\\
-9856.33433948864	2.10398497207404e-05	\\
-9855.35555752841	2.11283425467911e-05	\\
-9854.37677556818	2.13237433103418e-05	\\
-9853.39799360795	2.06780299984858e-05	\\
-9852.41921164773	2.09617333151336e-05	\\
-9851.4404296875	2.1579273113943e-05	\\
-9850.46164772727	2.12214978822028e-05	\\
-9849.48286576705	2.1122268196044e-05	\\
-9848.50408380682	2.18530930796068e-05	\\
-9847.52530184659	2.08450533944091e-05	\\
-9846.54651988636	2.16359509581608e-05	\\
-9845.56773792614	2.10002809978624e-05	\\
-9844.58895596591	2.14699359213526e-05	\\
-9843.61017400568	2.0825815162134e-05	\\
-9842.63139204545	2.22059988969335e-05	\\
-9841.65261008523	2.16542453668359e-05	\\
-9840.673828125	2.16016968794647e-05	\\
-9839.69504616477	2.16085805368869e-05	\\
-9838.71626420455	2.24323279243539e-05	\\
-9837.73748224432	2.11647604924347e-05	\\
-9836.75870028409	2.13657410467183e-05	\\
-9835.77991832386	2.23542166704655e-05	\\
-9834.80113636364	2.19695055571901e-05	\\
-9833.82235440341	2.16536171200275e-05	\\
-9832.84357244318	2.13313033660891e-05	\\
-9831.86479048295	2.19408664063375e-05	\\
-9830.88600852273	2.20184590979192e-05	\\
-9829.9072265625	2.14360840841445e-05	\\
-9828.92844460227	2.17224908902564e-05	\\
-9827.94966264205	2.13356980415258e-05	\\
-9826.97088068182	2.18230708610511e-05	\\
-9825.99209872159	2.13622003620064e-05	\\
-9825.01331676136	2.10833265297163e-05	\\
-9824.03453480114	2.24067735260409e-05	\\
-9823.05575284091	2.20803649623622e-05	\\
-9822.07697088068	2.14620097383474e-05	\\
-9821.09818892045	2.16769327752271e-05	\\
-9820.11940696023	2.22949897995038e-05	\\
-9819.140625	2.21840947558122e-05	\\
-9818.16184303977	2.15352432651309e-05	\\
-9817.18306107955	2.13993727144839e-05	\\
-9816.20427911932	2.06708542571057e-05	\\
-9815.22549715909	2.19573090659852e-05	\\
-9814.24671519886	2.19623394115493e-05	\\
-9813.26793323864	2.20670286459101e-05	\\
-9812.28915127841	2.16717415524964e-05	\\
-9811.31036931818	2.245155415937e-05	\\
-9810.33158735795	2.19152122709835e-05	\\
-9809.35280539773	2.11592208754374e-05	\\
-9808.3740234375	2.25130316055423e-05	\\
-9807.39524147727	2.26368609110678e-05	\\
-9806.41645951705	2.14890348006999e-05	\\
-9805.43767755682	2.28022199751815e-05	\\
-9804.45889559659	2.19703299649424e-05	\\
-9803.48011363636	2.12029590394768e-05	\\
-9802.50133167614	2.24210508456683e-05	\\
-9801.52254971591	2.22036933300186e-05	\\
-9800.54376775568	2.19454024631263e-05	\\
-9799.56498579545	2.21402972681199e-05	\\
-9798.58620383523	2.17640134286948e-05	\\
-9797.607421875	2.09022420111291e-05	\\
-9796.62863991477	2.21476581456184e-05	\\
-9795.64985795455	2.22456769723283e-05	\\
-9794.67107599432	2.20235828235772e-05	\\
-9793.69229403409	2.21210649681786e-05	\\
-9792.71351207386	2.11933805372347e-05	\\
-9791.73473011364	2.12200870367945e-05	\\
-9790.75594815341	2.19164973395065e-05	\\
-9789.77716619318	2.19566795357002e-05	\\
-9788.79838423295	2.07755501816432e-05	\\
-9787.81960227273	2.14217404931243e-05	\\
-9786.8408203125	2.02830630295998e-05	\\
-9785.86203835227	2.198710150365e-05	\\
-9784.88325639205	2.123347068172e-05	\\
-9783.90447443182	2.17144604011935e-05	\\
-9782.92569247159	2.1864279274403e-05	\\
-9781.94691051136	2.20783486521112e-05	\\
-9780.96812855114	2.12718536472946e-05	\\
-9779.98934659091	2.2313768447108e-05	\\
-9779.01056463068	2.19882389489211e-05	\\
-9778.03178267045	2.20120945983174e-05	\\
-9777.05300071023	2.19161999969575e-05	\\
-9776.07421875	2.16292537364567e-05	\\
-9775.09543678977	2.17706973853519e-05	\\
-9774.11665482955	2.15021869898255e-05	\\
-9773.13787286932	2.15391146016345e-05	\\
-9772.15909090909	2.16154395349505e-05	\\
-9771.18030894886	2.14757200245543e-05	\\
-9770.20152698864	2.09577625144867e-05	\\
-9769.22274502841	2.20282266342179e-05	\\
-9768.24396306818	2.08562526324299e-05	\\
-9767.26518110795	2.01324521860272e-05	\\
-9766.28639914773	2.14271214067903e-05	\\
-9765.3076171875	2.09308815397914e-05	\\
-9764.32883522727	1.99706418675624e-05	\\
-9763.35005326705	2.06034375608464e-05	\\
-9762.37127130682	2.14543168169778e-05	\\
-9761.39248934659	2.03069234397289e-05	\\
-9760.41370738636	2.11738322042269e-05	\\
-9759.43492542614	2.18272301602346e-05	\\
-9758.45614346591	1.99486827181267e-05	\\
-9757.47736150568	2.11404913618209e-05	\\
-9756.49857954545	2.10333113507035e-05	\\
-9755.51979758523	2.16466998371075e-05	\\
-9754.541015625	2.08716671807733e-05	\\
-9753.56223366477	2.10918564128951e-05	\\
-9752.58345170455	2.05779970920583e-05	\\
-9751.60466974432	2.12636098810598e-05	\\
-9750.62588778409	2.07224052654559e-05	\\
-9749.64710582386	2.14883127250177e-05	\\
-9748.66832386364	2.16567923114378e-05	\\
-9747.68954190341	2.14075348789091e-05	\\
-9746.71075994318	2.02179851663297e-05	\\
-9745.73197798295	2.04959608401467e-05	\\
-9744.75319602273	2.07360319703046e-05	\\
-9743.7744140625	2.01755675347796e-05	\\
-9742.79563210227	2.08414998501967e-05	\\
-9741.81685014205	2.06884354334319e-05	\\
-9740.83806818182	2.08170607261564e-05	\\
-9739.85928622159	2.11783012385285e-05	\\
-9738.88050426136	2.07647317876626e-05	\\
-9737.90172230114	2.08948375577344e-05	\\
-9736.92294034091	2.0138364263842e-05	\\
-9735.94415838068	2.04866008499555e-05	\\
-9734.96537642045	2.03691426470055e-05	\\
-9733.98659446023	1.98338331553409e-05	\\
-9733.0078125	2.12730408657971e-05	\\
-9732.02903053977	2.01115776124184e-05	\\
-9731.05024857955	2.03973847872182e-05	\\
-9730.07146661932	2.06004788823418e-05	\\
-9729.09268465909	2.04222263582512e-05	\\
-9728.11390269886	2.05238900596701e-05	\\
-9727.13512073864	2.06933564456643e-05	\\
-9726.15633877841	2.02307980980699e-05	\\
-9725.17755681818	2.0682344712969e-05	\\
-9724.19877485795	2.09424486028494e-05	\\
-9723.21999289773	1.98176941678755e-05	\\
-9722.2412109375	2.14492732307449e-05	\\
-9721.26242897727	2.1829484535496e-05	\\
-9720.28364701705	2.15538374212885e-05	\\
-9719.30486505682	2.13245121948983e-05	\\
-9718.32608309659	2.09898371790416e-05	\\
-9717.34730113636	2.21452575979976e-05	\\
-9716.36851917614	2.11503088632974e-05	\\
-9715.38973721591	2.16550398002587e-05	\\
-9714.41095525568	2.17612328659217e-05	\\
-9713.43217329545	2.21257038342402e-05	\\
-9712.45339133523	2.11608014660586e-05	\\
-9711.474609375	2.25500606051007e-05	\\
-9710.49582741477	2.07503947557112e-05	\\
-9709.51704545455	2.18423640664399e-05	\\
-9708.53826349432	2.18121003247414e-05	\\
-9707.55948153409	2.18622825187707e-05	\\
-9706.58069957386	2.18346079734638e-05	\\
-9705.60191761364	2.12662911289586e-05	\\
-9704.62313565341	2.15300808440693e-05	\\
-9703.64435369318	2.23811472096713e-05	\\
-9702.66557173295	2.17887193938153e-05	\\
-9701.68678977273	2.26200098050803e-05	\\
-9700.7080078125	2.18391903790374e-05	\\
-9699.72922585227	2.23078498115779e-05	\\
-9698.75044389205	2.12012004693079e-05	\\
-9697.77166193182	2.24004056912839e-05	\\
-9696.79287997159	2.3444996713208e-05	\\
-9695.81409801136	2.12898770836311e-05	\\
-9694.83531605114	2.24533847946385e-05	\\
-9693.85653409091	2.25375082233848e-05	\\
-9692.87775213068	2.21882322875458e-05	\\
-9691.89897017045	2.19457796240163e-05	\\
-9690.92018821023	2.293733991251e-05	\\
-9689.94140625	2.26285106518373e-05	\\
-9688.96262428977	2.3340446927073e-05	\\
-9687.98384232955	2.25807403372282e-05	\\
-9687.00506036932	2.20199979098003e-05	\\
-9686.02627840909	2.29696597580703e-05	\\
-9685.04749644886	2.17456507600701e-05	\\
-9684.06871448864	2.1658666602135e-05	\\
-9683.08993252841	2.27787027619299e-05	\\
-9682.11115056818	2.31275019070926e-05	\\
-9681.13236860795	2.15749722276814e-05	\\
-9680.15358664773	2.27429134393141e-05	\\
-9679.1748046875	2.23419317344065e-05	\\
-9678.19602272727	2.26564541393283e-05	\\
-9677.21724076705	2.35001933230618e-05	\\
-9676.23845880682	2.2736021003092e-05	\\
-9675.25967684659	2.22457667897566e-05	\\
-9674.28089488636	2.18452310305707e-05	\\
-9673.30211292614	2.30427145026105e-05	\\
-9672.32333096591	2.17725252705484e-05	\\
-9671.34454900568	2.2639006461564e-05	\\
-9670.36576704545	2.35442300067312e-05	\\
-9669.38698508523	2.30089564081875e-05	\\
-9668.408203125	2.29094463797361e-05	\\
-9667.42942116477	2.16974930770172e-05	\\
-9666.45063920455	2.20882744354313e-05	\\
-9665.47185724432	2.31852304879894e-05	\\
-9664.49307528409	2.22020042829986e-05	\\
-9663.51429332386	2.2955767522076e-05	\\
-9662.53551136364	2.27514192706588e-05	\\
-9661.55672940341	2.25170506768202e-05	\\
-9660.57794744318	2.23625865194727e-05	\\
-9659.59916548295	2.25196091169452e-05	\\
-9658.62038352273	2.27333445499437e-05	\\
-9657.6416015625	2.38468905342738e-05	\\
-9656.66281960227	2.23704192998431e-05	\\
-9655.68403764205	2.20932221026875e-05	\\
-9654.70525568182	2.29017470134691e-05	\\
-9653.72647372159	2.31777135765128e-05	\\
-9652.74769176136	2.37788668059092e-05	\\
-9651.76890980114	2.34663587741589e-05	\\
-9650.79012784091	2.33739515071959e-05	\\
-9649.81134588068	2.40494820183784e-05	\\
-9648.83256392045	2.38047444305233e-05	\\
-9647.85378196023	2.28384826031062e-05	\\
-9646.875	2.31825350671803e-05	\\
-9645.89621803977	2.32421336786138e-05	\\
-9644.91743607955	2.40522581733814e-05	\\
-9643.93865411932	2.44128424358483e-05	\\
-9642.95987215909	2.4155211234698e-05	\\
-9641.98109019886	2.39409141324053e-05	\\
-9641.00230823864	2.50664426781113e-05	\\
-9640.02352627841	2.44586623498548e-05	\\
-9639.04474431818	2.45145206788885e-05	\\
-9638.06596235795	2.37978682593046e-05	\\
-9637.08718039773	2.41641543324211e-05	\\
-9636.1083984375	2.42213674100603e-05	\\
-9635.12961647727	2.44049743595304e-05	\\
-9634.15083451705	2.5604726431245e-05	\\
-9633.17205255682	2.48802020567573e-05	\\
-9632.19327059659	2.51915831474951e-05	\\
-9631.21448863636	2.48619627350433e-05	\\
-9630.23570667614	2.4664926937729e-05	\\
-9629.25692471591	2.43602924525338e-05	\\
-9628.27814275568	2.56659339171717e-05	\\
-9627.29936079545	2.46285390155045e-05	\\
-9626.32057883523	2.43438967359252e-05	\\
-9625.341796875	2.67990441102938e-05	\\
-9624.36301491477	2.57782946155628e-05	\\
-9623.38423295455	2.58957145622885e-05	\\
-9622.40545099432	2.62974319294522e-05	\\
-9621.42666903409	2.61107599561171e-05	\\
-9620.44788707386	2.55752043029355e-05	\\
-9619.46910511364	2.61283679150653e-05	\\
-9618.49032315341	2.59787243416757e-05	\\
-9617.51154119318	2.60370902595686e-05	\\
-9616.53275923295	2.64806586812262e-05	\\
-9615.55397727273	2.59841725240266e-05	\\
-9614.5751953125	2.61448374095844e-05	\\
-9613.59641335227	2.57966136723578e-05	\\
-9612.61763139205	2.53885021233838e-05	\\
-9611.63884943182	2.65177615633748e-05	\\
-9610.66006747159	2.69863475111412e-05	\\
-9609.68128551136	2.65688385077221e-05	\\
-9608.70250355114	2.70153695895994e-05	\\
-9607.72372159091	2.69620384404908e-05	\\
-9606.74493963068	2.67884637984488e-05	\\
-9605.76615767045	2.72739168528981e-05	\\
-9604.78737571023	2.67809530106998e-05	\\
-9603.80859375	2.68153572104482e-05	\\
-9602.82981178977	2.70299186429872e-05	\\
-9601.85102982955	2.63181994878971e-05	\\
-9600.87224786932	2.57900299762156e-05	\\
-9599.89346590909	2.67597002412518e-05	\\
-9598.91468394886	2.80650898481257e-05	\\
-9597.93590198864	2.77622334367843e-05	\\
-9596.95712002841	2.7008902073616e-05	\\
-9595.97833806818	2.76800168552156e-05	\\
-9594.99955610795	2.68860213302746e-05	\\
-9594.02077414773	2.73547384055106e-05	\\
-9593.0419921875	2.70971987178484e-05	\\
-9592.06321022727	2.76203891827441e-05	\\
-9591.08442826705	2.85475792496331e-05	\\
-9590.10564630682	2.82151294466059e-05	\\
-9589.12686434659	2.70924726053915e-05	\\
-9588.14808238636	2.7588680897595e-05	\\
-9587.16930042614	2.73808453932347e-05	\\
-9586.19051846591	2.85414748903099e-05	\\
-9585.21173650568	2.8726137520789e-05	\\
-9584.23295454545	2.75287302227898e-05	\\
-9583.25417258523	2.91322186909025e-05	\\
-9582.275390625	2.80473453884091e-05	\\
-9581.29660866477	2.76578229623159e-05	\\
-9580.31782670455	2.93880732492547e-05	\\
-9579.33904474432	2.90200124315822e-05	\\
-9578.36026278409	2.850305088062e-05	\\
-9577.38148082386	2.87908646512759e-05	\\
-9576.40269886364	2.91602729416983e-05	\\
-9575.42391690341	2.86755044584804e-05	\\
-9574.44513494318	2.95335283386814e-05	\\
-9573.46635298295	2.94521929056118e-05	\\
-9572.48757102273	2.96395968897565e-05	\\
-9571.5087890625	2.8989860388644e-05	\\
-9570.53000710227	2.83849259323352e-05	\\
-9569.55122514205	2.83564116869008e-05	\\
-9568.57244318182	2.92813045317945e-05	\\
-9567.59366122159	2.98571988908856e-05	\\
-9566.61487926136	2.9919636051093e-05	\\
-9565.63609730114	2.98585869482984e-05	\\
-9564.65731534091	2.95227589148025e-05	\\
-9563.67853338068	2.9079437190365e-05	\\
-9562.69975142045	2.96922882090941e-05	\\
-9561.72096946023	2.95338493109683e-05	\\
-9560.7421875	3.01744049797969e-05	\\
-9559.76340553977	2.87646415420997e-05	\\
-9558.78462357955	2.97567617893303e-05	\\
-9557.80584161932	2.95584592739302e-05	\\
-9556.82705965909	2.86732507318584e-05	\\
-9555.84827769886	2.97140238929057e-05	\\
-9554.86949573864	3.04854309856461e-05	\\
-9553.89071377841	3.06022244663988e-05	\\
-9552.91193181818	3.08632271059369e-05	\\
-9551.93314985795	3.08031677047583e-05	\\
-9550.95436789773	3.05985836655738e-05	\\
-9549.9755859375	2.9131939913032e-05	\\
-9548.99680397727	3.00895869417944e-05	\\
-9548.01802201705	2.94760755252394e-05	\\
-9547.03924005682	3.05413075218606e-05	\\
-9546.06045809659	3.00268743356132e-05	\\
-9545.08167613636	3.08710547976167e-05	\\
-9544.10289417614	3.15336060676615e-05	\\
-9543.12411221591	3.05238176799582e-05	\\
-9542.14533025568	3.00020820980665e-05	\\
-9541.16654829545	3.18611586407352e-05	\\
-9540.18776633523	3.05485007507638e-05	\\
-9539.208984375	3.11707251884518e-05	\\
-9538.23020241477	3.13763722600069e-05	\\
-9537.25142045455	3.05729901570812e-05	\\
-9536.27263849432	3.14555373183294e-05	\\
-9535.29385653409	3.10229935606326e-05	\\
-9534.31507457386	3.21391017554126e-05	\\
-9533.33629261364	3.1060297641213e-05	\\
-9532.35751065341	3.28310564386087e-05	\\
-9531.37872869318	3.22814724270663e-05	\\
-9530.39994673295	3.16070913155625e-05	\\
-9529.42116477273	3.23869436289127e-05	\\
-9528.4423828125	3.24775210689885e-05	\\
-9527.46360085227	3.13038800652835e-05	\\
-9526.48481889205	3.32586045088924e-05	\\
-9525.50603693182	3.33927366623123e-05	\\
-9524.52725497159	3.30539890228842e-05	\\
-9523.54847301136	3.30772696874728e-05	\\
-9522.56969105114	3.20585192371989e-05	\\
-9521.59090909091	3.22556813970121e-05	\\
-9520.61212713068	3.17474476280818e-05	\\
-9519.63334517045	3.29112939858516e-05	\\
-9518.65456321023	3.33318828828109e-05	\\
-9517.67578125	3.28257614275894e-05	\\
-9516.69699928977	3.32732578165644e-05	\\
-9515.71821732955	3.28439574784903e-05	\\
-9514.73943536932	3.27544720520667e-05	\\
-9513.76065340909	3.30677988177827e-05	\\
-9512.78187144886	3.3171805733473e-05	\\
-9511.80308948864	3.46261581741025e-05	\\
-9510.82430752841	3.22957151201546e-05	\\
-9509.84552556818	3.3005729473499e-05	\\
-9508.86674360795	3.26073124760167e-05	\\
-9507.88796164773	3.29367506870091e-05	\\
-9506.9091796875	3.24161495504227e-05	\\
-9505.93039772727	3.28729418369315e-05	\\
-9504.95161576705	3.43936059565097e-05	\\
-9503.97283380682	3.30605283420552e-05	\\
-9502.99405184659	3.3686464551081e-05	\\
-9502.01526988636	3.42859668008874e-05	\\
-9501.03648792614	3.35004228822419e-05	\\
-9500.05770596591	3.2884234112615e-05	\\
-9499.07892400568	3.37454264165398e-05	\\
-9498.10014204545	3.37003617583596e-05	\\
-9497.12136008523	3.32550530679158e-05	\\
-9496.142578125	3.39475437301422e-05	\\
-9495.16379616477	3.36013595895133e-05	\\
-9494.18501420455	3.41782978629353e-05	\\
-9493.20623224432	3.38523735542018e-05	\\
-9492.22745028409	3.3880307226649e-05	\\
-9491.24866832386	3.43033565192953e-05	\\
-9490.26988636364	3.38923200834804e-05	\\
-9489.29110440341	3.4034750370614e-05	\\
-9488.31232244318	3.37736695991934e-05	\\
-9487.33354048295	3.44314537656103e-05	\\
-9486.35475852273	3.45446074356293e-05	\\
-9485.3759765625	3.40298696674292e-05	\\
-9484.39719460227	3.44965069075218e-05	\\
-9483.41841264205	3.32383307801131e-05	\\
-9482.43963068182	3.29106267581304e-05	\\
-9481.46084872159	3.43638165085991e-05	\\
-9480.48206676136	3.39962633540358e-05	\\
-9479.50328480114	3.39817232949254e-05	\\
-9478.52450284091	3.48467924628199e-05	\\
-9477.54572088068	3.44285528526423e-05	\\
-9476.56693892045	3.38603688178095e-05	\\
-9475.58815696023	3.38753704055473e-05	\\
-9474.609375	3.30983678297938e-05	\\
-9473.63059303977	3.39008748464437e-05	\\
-9472.65181107955	3.45675126310454e-05	\\
-9471.67302911932	3.43877637954959e-05	\\
-9470.69424715909	3.35514307175305e-05	\\
-9469.71546519886	3.39438651483083e-05	\\
-9468.73668323864	3.32424433985988e-05	\\
-9467.75790127841	3.37489325035481e-05	\\
-9466.77911931818	3.3841873172557e-05	\\
-9465.80033735795	3.32146833880872e-05	\\
-9464.82155539773	3.41894389004515e-05	\\
-9463.8427734375	3.46997163869961e-05	\\
-9462.86399147727	3.38992703519448e-05	\\
-9461.88520951705	3.36752001329255e-05	\\
-9460.90642755682	3.33106075193099e-05	\\
-9459.92764559659	3.44991340994996e-05	\\
-9458.94886363636	3.40750274274938e-05	\\
-9457.97008167614	3.45625072950006e-05	\\
-9456.99129971591	3.39285817126846e-05	\\
-9456.01251775568	3.35357776944877e-05	\\
-9455.03373579545	3.36398783802015e-05	\\
-9454.05495383523	3.34522779301222e-05	\\
-9453.076171875	3.38186629971883e-05	\\
-9452.09738991477	3.47732700341646e-05	\\
-9451.11860795455	3.55876379446473e-05	\\
-9450.13982599432	3.37184115475418e-05	\\
-9449.16104403409	3.41637313378851e-05	\\
-9448.18226207386	3.38489753797876e-05	\\
-9447.20348011364	3.38883951056862e-05	\\
-9446.22469815341	3.43356562478176e-05	\\
-9445.24591619318	3.52254910807754e-05	\\
-9444.26713423295	3.47278875157696e-05	\\
-9443.28835227273	3.35412917062488e-05	\\
-9442.3095703125	3.42300027527301e-05	\\
-9441.33078835227	3.56654963404612e-05	\\
-9440.35200639205	3.42564772003764e-05	\\
-9439.37322443182	3.51136537730406e-05	\\
-9438.39444247159	3.62052595410196e-05	\\
-9437.41566051136	3.58158939524834e-05	\\
-9436.43687855114	3.56428175554633e-05	\\
-9435.45809659091	3.55452710958615e-05	\\
-9434.47931463068	3.47748544248351e-05	\\
-9433.50053267045	3.59560005389883e-05	\\
-9432.52175071023	3.6106436944925e-05	\\
-9431.54296875	3.51653840536656e-05	\\
-9430.56418678977	3.63562013517366e-05	\\
-9429.58540482955	3.47722563098272e-05	\\
-9428.60662286932	3.54892979961261e-05	\\
-9427.62784090909	3.58665495003133e-05	\\
-9426.64905894886	3.60337015142017e-05	\\
-9425.67027698864	3.54811928313391e-05	\\
-9424.69149502841	3.54938268575176e-05	\\
-9423.71271306818	3.6086983831862e-05	\\
-9422.73393110795	3.58303553383663e-05	\\
-9421.75514914773	3.55913734089679e-05	\\
-9420.7763671875	3.66153073282726e-05	\\
-9419.79758522727	3.60486079329381e-05	\\
-9418.81880326705	3.57172925873514e-05	\\
-9417.84002130682	3.63035304060384e-05	\\
-9416.86123934659	3.68211388780207e-05	\\
-9415.88245738636	3.52482642365383e-05	\\
-9414.90367542614	3.55894191386384e-05	\\
-9413.92489346591	3.64714201285477e-05	\\
-9412.94611150568	3.56815218381808e-05	\\
-9411.96732954545	3.61759183459068e-05	\\
-9410.98854758523	3.56763700761585e-05	\\
-9410.009765625	3.55187254146415e-05	\\
-9409.03098366477	3.62907226301764e-05	\\
-9408.05220170455	3.58545136125765e-05	\\
-9407.07341974432	3.57668374284454e-05	\\
-9406.09463778409	3.52668729675666e-05	\\
-9405.11585582386	3.62317656598033e-05	\\
-9404.13707386364	3.64172195264707e-05	\\
-9403.15829190341	3.7484653259458e-05	\\
-9402.17950994318	3.61193220769257e-05	\\
-9401.20072798295	3.6417370438246e-05	\\
-9400.22194602273	3.893601507997e-05	\\
-9399.2431640625	3.62065787720768e-05	\\
-9398.26438210227	3.57211423935104e-05	\\
-9397.28560014205	3.64812042396118e-05	\\
-9396.30681818182	3.63564219901211e-05	\\
-9395.32803622159	3.5872662995129e-05	\\
-9394.34925426136	3.52995752863278e-05	\\
-9393.37047230114	3.59582558374193e-05	\\
-9392.39169034091	3.60120116698448e-05	\\
-9391.41290838068	3.60995901817172e-05	\\
-9390.43412642045	3.5919914345692e-05	\\
-9389.45534446023	3.64440911175966e-05	\\
-9388.4765625	3.75366146713686e-05	\\
-9387.49778053977	3.65884546223028e-05	\\
-9386.51899857955	3.62375325402805e-05	\\
-9385.54021661932	3.65040729887175e-05	\\
-9384.56143465909	3.56890546049108e-05	\\
-9383.58265269886	3.59885632471429e-05	\\
-9382.60387073864	3.68526124840387e-05	\\
-9381.62508877841	3.6186131790427e-05	\\
-9380.64630681818	3.56311309822734e-05	\\
-9379.66752485795	3.59202553134118e-05	\\
-9378.68874289773	3.70425812998484e-05	\\
-9377.7099609375	3.69360860891924e-05	\\
-9376.73117897727	3.66002844583346e-05	\\
-9375.75239701705	3.63670547016726e-05	\\
-9374.77361505682	3.5712922278169e-05	\\
-9373.79483309659	3.66202499801563e-05	\\
-9372.81605113636	3.62394905049411e-05	\\
-9371.83726917614	3.65116954946724e-05	\\
-9370.85848721591	3.55730086935015e-05	\\
-9369.87970525568	3.64010110908096e-05	\\
-9368.90092329545	3.59950080112856e-05	\\
-9367.92214133523	3.63404206431939e-05	\\
-9366.943359375	3.59320973886816e-05	\\
-9365.96457741477	3.58401923929102e-05	\\
-9364.98579545455	3.59609719696813e-05	\\
-9364.00701349432	3.66953585905842e-05	\\
-9363.02823153409	3.53889452083847e-05	\\
-9362.04944957386	3.60425319776555e-05	\\
-9361.07066761364	3.58213425658308e-05	\\
-9360.09188565341	3.63315806057549e-05	\\
-9359.11310369318	3.73807023610835e-05	\\
-9358.13432173295	3.59823654280354e-05	\\
-9357.15553977273	3.66185441042874e-05	\\
-9356.1767578125	3.54316549536018e-05	\\
-9355.19797585227	3.59287418576625e-05	\\
-9354.21919389205	3.55227976226823e-05	\\
-9353.24041193182	3.40367781349042e-05	\\
-9352.26162997159	3.66958047974127e-05	\\
-9351.28284801136	3.5022112171355e-05	\\
-9350.30406605114	3.56024655175826e-05	\\
-9349.32528409091	3.53341303289002e-05	\\
-9348.34650213068	3.61872754536017e-05	\\
-9347.36772017045	3.39406795790172e-05	\\
-9346.38893821023	3.65021616830492e-05	\\
-9345.41015625	3.44959139502007e-05	\\
-9344.43137428977	3.51209723082112e-05	\\
-9343.45259232955	3.5445857739618e-05	\\
-9342.47381036932	3.49994148460748e-05	\\
-9341.49502840909	3.39238315170126e-05	\\
-9340.51624644886	3.67919812703558e-05	\\
-9339.53746448864	3.38465590227521e-05	\\
-9338.55868252841	3.41890430940603e-05	\\
-9337.57990056818	3.49746352987798e-05	\\
-9336.60111860795	3.34912449436515e-05	\\
-9335.62233664773	3.37406426716092e-05	\\
-9334.6435546875	3.44810348858116e-05	\\
-9333.66477272727	3.41859575495819e-05	\\
-9332.68599076705	3.41240433506304e-05	\\
-9331.70720880682	3.4348400387026e-05	\\
-9330.72842684659	3.40467963105247e-05	\\
-9329.74964488636	3.44278954847866e-05	\\
-9328.77086292614	3.34300290354156e-05	\\
-9327.79208096591	3.36556597805996e-05	\\
-9326.81329900568	3.43504267156556e-05	\\
-9325.83451704545	3.46180740932692e-05	\\
-9324.85573508523	3.42543580247912e-05	\\
-9323.876953125	3.38652948728032e-05	\\
-9322.89817116477	3.31217452060004e-05	\\
-9321.91938920455	3.4297459787839e-05	\\
-9320.94060724432	3.34916927502703e-05	\\
-9319.96182528409	3.38729306944762e-05	\\
-9318.98304332386	3.39664823644646e-05	\\
-9318.00426136364	3.42561614014954e-05	\\
-9317.02547940341	3.41388488179017e-05	\\
-9316.04669744318	3.46816518806807e-05	\\
-9315.06791548295	3.3680858213133e-05	\\
-9314.08913352273	3.27113179936098e-05	\\
-9313.1103515625	3.44499459228798e-05	\\
-9312.13156960227	3.45351651567444e-05	\\
-9311.15278764205	3.43413406384268e-05	\\
-9310.17400568182	3.46509390093101e-05	\\
-9309.19522372159	3.52444557499686e-05	\\
-9308.21644176136	3.56261630369586e-05	\\
-9307.23765980114	3.57354883192342e-05	\\
-9306.25887784091	3.47789660508225e-05	\\
-9305.28009588068	3.50815641036182e-05	\\
-9304.30131392045	3.4520821450199e-05	\\
-9303.32253196023	3.49759918002625e-05	\\
-9302.34375	3.50270800187804e-05	\\
-9301.36496803977	3.50810754607214e-05	\\
-9300.38618607955	3.48517480105838e-05	\\
-9299.40740411932	3.46252406174751e-05	\\
-9298.42862215909	3.54020708688875e-05	\\
-9297.44984019886	3.50731016414156e-05	\\
-9296.47105823864	3.41923893810762e-05	\\
-9295.49227627841	3.66525205732995e-05	\\
-9294.51349431818	3.43697026161854e-05	\\
-9293.53471235795	3.59524352012533e-05	\\
-9292.55593039773	3.57483685100672e-05	\\
-9291.5771484375	3.57816428573453e-05	\\
-9290.59836647727	3.60261397853352e-05	\\
-9289.61958451705	3.57314710456263e-05	\\
-9288.64080255682	3.64065713918955e-05	\\
-9287.66202059659	3.64237813995626e-05	\\
-9286.68323863636	3.51057129327655e-05	\\
-9285.70445667614	3.7109931044926e-05	\\
-9284.72567471591	3.51799560478509e-05	\\
-9283.74689275568	3.65262754167595e-05	\\
-9282.76811079545	3.59766409946866e-05	\\
-9281.78932883523	3.60571947653048e-05	\\
-9280.810546875	3.57809705666453e-05	\\
-9279.83176491477	3.63991973701476e-05	\\
-9278.85298295455	3.56911368142673e-05	\\
-9277.87420099432	3.46869013679629e-05	\\
-9276.89541903409	3.66310318027689e-05	\\
-9275.91663707386	3.57191875088126e-05	\\
-9274.93785511364	3.64770313952317e-05	\\
-9273.95907315341	3.73843272898641e-05	\\
-9272.98029119318	3.63551034022478e-05	\\
-9272.00150923295	3.53858057407178e-05	\\
-9271.02272727273	3.66580684993446e-05	\\
-9270.0439453125	3.66370052953681e-05	\\
-9269.06516335227	3.47536530670358e-05	\\
-9268.08638139205	3.46180180997415e-05	\\
-9267.10759943182	3.57595004055949e-05	\\
-9266.12881747159	3.54902006503675e-05	\\
-9265.15003551136	3.47840839414334e-05	\\
-9264.17125355114	3.61727278372241e-05	\\
-9263.19247159091	3.62614171179467e-05	\\
-9262.21368963068	3.63198184202654e-05	\\
-9261.23490767045	3.67474033358703e-05	\\
-9260.25612571023	3.69927388366726e-05	\\
-9259.27734375	3.53408817109079e-05	\\
-9258.29856178977	3.64900141568118e-05	\\
-9257.31977982955	3.56456748855667e-05	\\
-9256.34099786932	3.60624243082069e-05	\\
-9255.36221590909	3.52394265026177e-05	\\
-9254.38343394886	3.52926156269365e-05	\\
-9253.40465198864	3.67725544539895e-05	\\
-9252.42587002841	3.64201086365807e-05	\\
-9251.44708806818	3.47229496143388e-05	\\
-9250.46830610795	3.64306709912757e-05	\\
-9249.48952414773	3.71336360205068e-05	\\
-9248.5107421875	3.57494377235037e-05	\\
-9247.53196022727	3.56226480295676e-05	\\
-9246.55317826705	3.53289793275923e-05	\\
-9245.57439630682	3.64920415323768e-05	\\
-9244.59561434659	3.59775539608358e-05	\\
-9243.61683238636	3.68224386289732e-05	\\
-9242.63805042614	3.47113949143618e-05	\\
-9241.65926846591	3.64705729350756e-05	\\
-9240.68048650568	3.53339108055066e-05	\\
-9239.70170454545	3.61414222430109e-05	\\
-9238.72292258523	3.61671095003148e-05	\\
-9237.744140625	3.65217023519794e-05	\\
-9236.76535866477	3.40394250284789e-05	\\
-9235.78657670455	3.5017986104338e-05	\\
-9234.80779474432	3.6251714362996e-05	\\
-9233.82901278409	3.53998923320066e-05	\\
-9232.85023082386	3.61398283373212e-05	\\
-9231.87144886364	3.59639263851941e-05	\\
-9230.89266690341	3.59991736375648e-05	\\
-9229.91388494318	3.66931741830524e-05	\\
-9228.93510298295	3.58921690693575e-05	\\
-9227.95632102273	3.66926792581906e-05	\\
-9226.9775390625	3.67863275379462e-05	\\
-9225.99875710227	3.52208701263689e-05	\\
-9225.01997514205	3.59905491533055e-05	\\
-9224.04119318182	3.57303983922541e-05	\\
-9223.06241122159	3.63851372634647e-05	\\
-9222.08362926136	3.50268787296476e-05	\\
-9221.10484730114	3.6140740209198e-05	\\
-9220.12606534091	3.44693005215285e-05	\\
-9219.14728338068	3.51206893099404e-05	\\
-9218.16850142045	3.56311100124892e-05	\\
-9217.18971946023	3.52472672798342e-05	\\
-9216.2109375	3.56058876661847e-05	\\
-9215.23215553977	3.49284900621565e-05	\\
-9214.25337357955	3.36244803009479e-05	\\
-9213.27459161932	3.40057040384704e-05	\\
-9212.29580965909	3.49410371246463e-05	\\
-9211.31702769886	3.49131132581965e-05	\\
-9210.33824573864	3.41855474075367e-05	\\
-9209.35946377841	3.49477058393988e-05	\\
-9208.38068181818	3.48906357865313e-05	\\
-9207.40189985795	3.56421578470045e-05	\\
-9206.42311789773	3.37284451603092e-05	\\
-9205.4443359375	3.51412503923968e-05	\\
-9204.46555397727	3.55849158228423e-05	\\
-9203.48677201705	3.40543067746003e-05	\\
-9202.50799005682	3.52937468507107e-05	\\
-9201.52920809659	3.47633718327078e-05	\\
-9200.55042613636	3.50940745300158e-05	\\
-9199.57164417614	3.63626287114741e-05	\\
-9198.59286221591	3.45463786948997e-05	\\
-9197.61408025568	3.46455567410831e-05	\\
-9196.63529829545	3.40822752593877e-05	\\
-9195.65651633523	3.49477733318674e-05	\\
-9194.677734375	3.45241505089951e-05	\\
-9193.69895241477	3.46335578549462e-05	\\
-9192.72017045455	3.43270929551472e-05	\\
-9191.74138849432	3.46502418327066e-05	\\
-9190.76260653409	3.42499283120423e-05	\\
-9189.78382457386	3.50368946844915e-05	\\
-9188.80504261364	3.48132793240343e-05	\\
-9187.82626065341	3.46300908431769e-05	\\
-9186.84747869318	3.53474709780979e-05	\\
-9185.86869673295	3.49260961803206e-05	\\
-9184.88991477273	3.49721910759769e-05	\\
-9183.9111328125	3.5402461483894e-05	\\
-9182.93235085227	3.45362774163495e-05	\\
-9181.95356889205	3.52210231942635e-05	\\
-9180.97478693182	3.5105143593573e-05	\\
-9179.99600497159	3.54506748177428e-05	\\
-9179.01722301136	3.423717364427e-05	\\
-9178.03844105114	3.54768177235751e-05	\\
-9177.05965909091	3.41764630494379e-05	\\
-9176.08087713068	3.40130280607052e-05	\\
-9175.10209517045	3.41662367738062e-05	\\
-9174.12331321023	3.41648528137882e-05	\\
-9173.14453125	3.46419186767275e-05	\\
-9172.16574928977	3.50923195927675e-05	\\
-9171.18696732955	3.35750538553942e-05	\\
-9170.20818536932	3.46164672128153e-05	\\
-9169.22940340909	3.40418633207188e-05	\\
-9168.25062144886	3.38014487173749e-05	\\
-9167.27183948864	3.38841065043239e-05	\\
-9166.29305752841	3.45533411414335e-05	\\
-9165.31427556818	3.42085364218583e-05	\\
-9164.33549360795	3.38071035095347e-05	\\
-9163.35671164773	3.40426609907301e-05	\\
-9162.3779296875	3.41018563152753e-05	\\
-9161.39914772727	3.27867086525779e-05	\\
-9160.42036576705	3.40422852111062e-05	\\
-9159.44158380682	3.34012222652413e-05	\\
-9158.46280184659	3.28490755805216e-05	\\
-9157.48401988636	3.2990872148098e-05	\\
-9156.50523792614	3.34361119363099e-05	\\
-9155.52645596591	3.30586479482274e-05	\\
-9154.54767400568	3.37603975505267e-05	\\
-9153.56889204545	3.44234485739852e-05	\\
-9152.59011008523	3.44183296644535e-05	\\
-9151.611328125	3.34524844195367e-05	\\
-9150.63254616477	3.25431944616375e-05	\\
-9149.65376420455	3.36774456588842e-05	\\
-9148.67498224432	3.32102601031937e-05	\\
-9147.69620028409	3.35087766577148e-05	\\
-9146.71741832386	3.30167023504655e-05	\\
-9145.73863636364	3.31237847411541e-05	\\
-9144.75985440341	3.41322027980981e-05	\\
-9143.78107244318	3.35147119380999e-05	\\
-9142.80229048295	3.33222396496833e-05	\\
-9141.82350852273	3.29519310816671e-05	\\
-9140.8447265625	3.44664870017958e-05	\\
-9139.86594460227	3.30625358302475e-05	\\
-9138.88716264205	3.47606888431084e-05	\\
-9137.90838068182	3.39987135564296e-05	\\
-9136.92959872159	3.38517281696852e-05	\\
-9135.95081676136	3.36271561709365e-05	\\
-9134.97203480114	3.39725653284972e-05	\\
-9133.99325284091	3.36514079102685e-05	\\
-9133.01447088068	3.32816428944639e-05	\\
-9132.03568892045	3.35736795688199e-05	\\
-9131.05690696023	3.20808807277238e-05	\\
-9130.078125	3.31678105725365e-05	\\
-9129.09934303977	3.32490527726595e-05	\\
-9128.12056107955	3.3775183836776e-05	\\
-9127.14177911932	3.40283363597925e-05	\\
-9126.16299715909	3.38051891387181e-05	\\
-9125.18421519886	3.31892049201639e-05	\\
-9124.20543323864	3.37462002711672e-05	\\
-9123.22665127841	3.32523043747946e-05	\\
-9122.24786931818	3.43529006874932e-05	\\
-9121.26908735795	3.41950179414039e-05	\\
-9120.29030539773	3.23843032826287e-05	\\
-9119.3115234375	3.33613151258068e-05	\\
-9118.33274147727	3.3768538380486e-05	\\
-9117.35395951705	3.32608703302146e-05	\\
-9116.37517755682	3.50402180158835e-05	\\
-9115.39639559659	3.37516681255605e-05	\\
-9114.41761363636	3.45465583720576e-05	\\
-9113.43883167614	3.39746199160423e-05	\\
-9112.46004971591	3.48716707169468e-05	\\
-9111.48126775568	3.36797398607843e-05	\\
-9110.50248579545	3.39637547307792e-05	\\
-9109.52370383523	3.4375121821859e-05	\\
-9108.544921875	3.5001626004712e-05	\\
-9107.56613991477	3.45685445684895e-05	\\
-9106.58735795455	3.38010822978656e-05	\\
-9105.60857599432	3.4376334400235e-05	\\
-9104.62979403409	3.44913095293847e-05	\\
-9103.65101207386	3.3430311865971e-05	\\
-9102.67223011364	3.39144015802135e-05	\\
-9101.69344815341	3.38863816562084e-05	\\
-9100.71466619318	3.50011563013223e-05	\\
-9099.73588423295	3.43695489836828e-05	\\
-9098.75710227273	3.58039260239584e-05	\\
-9097.7783203125	3.41727651986825e-05	\\
-9096.79953835227	3.53407069470794e-05	\\
-9095.82075639205	3.45613787645145e-05	\\
-9094.84197443182	3.37645304432466e-05	\\
-9093.86319247159	3.53987993240234e-05	\\
-9092.88441051136	3.46177979830742e-05	\\
-9091.90562855114	3.50454067015632e-05	\\
-9090.92684659091	3.42387195612221e-05	\\
-9089.94806463068	3.47405070104553e-05	\\
-9088.96928267045	3.40776519449167e-05	\\
-9087.99050071023	3.51303783568688e-05	\\
-9087.01171875	3.48195835429813e-05	\\
-9086.03293678977	3.47943879472224e-05	\\
-9085.05415482955	3.50928595040917e-05	\\
-9084.07537286932	3.44792052814419e-05	\\
-9083.09659090909	3.45403442769476e-05	\\
-9082.11780894886	3.5086575337226e-05	\\
-9081.13902698864	3.59495723288712e-05	\\
-9080.16024502841	3.36811817845514e-05	\\
-9079.18146306818	3.45714534811423e-05	\\
-9078.20268110795	3.40254202752453e-05	\\
-9077.22389914773	3.44616430667561e-05	\\
-9076.2451171875	3.41645114634206e-05	\\
-9075.26633522727	3.47356409011714e-05	\\
-9074.28755326705	3.41454763904483e-05	\\
-9073.30877130682	3.53625254079937e-05	\\
-9072.32998934659	3.5236546943637e-05	\\
-9071.35120738636	3.36824399957379e-05	\\
-9070.37242542614	3.42604068365759e-05	\\
-9069.39364346591	3.42710486126908e-05	\\
-9068.41486150568	3.32687549495815e-05	\\
-9067.43607954545	3.29589408422866e-05	\\
-9066.45729758523	3.57122024254082e-05	\\
-9065.478515625	3.45260268496504e-05	\\
-9064.49973366477	3.4632613715386e-05	\\
-9063.52095170455	3.42568656133876e-05	\\
-9062.54216974432	3.46407704151329e-05	\\
-9061.56338778409	3.45442750148482e-05	\\
-9060.58460582386	3.42540119596495e-05	\\
-9059.60582386364	3.50544179697454e-05	\\
-9058.62704190341	3.63905827663188e-05	\\
-9057.64825994318	3.52532784479758e-05	\\
-9056.66947798295	3.62377701256437e-05	\\
-9055.69069602273	3.44926101458019e-05	\\
-9054.7119140625	3.54156874282202e-05	\\
-9053.73313210227	3.57283023422924e-05	\\
-9052.75435014205	3.56441554134395e-05	\\
-9051.77556818182	3.49330974658389e-05	\\
-9050.79678622159	3.6033767539119e-05	\\
-9049.81800426136	3.51290544564333e-05	\\
-9048.83922230114	3.40196347211617e-05	\\
-9047.86044034091	3.51139358710669e-05	\\
-9046.88165838068	3.56120460452962e-05	\\
-9045.90287642045	3.42465230107729e-05	\\
-9044.92409446023	3.49709043836467e-05	\\
-9043.9453125	3.55031675574489e-05	\\
-9042.96653053977	3.49670273766692e-05	\\
-9041.98774857955	3.50301022846804e-05	\\
-9041.00896661932	3.60627598716693e-05	\\
-9040.03018465909	3.37562379745222e-05	\\
-9039.05140269886	3.41642811617126e-05	\\
-9038.07262073864	3.53317597716982e-05	\\
-9037.09383877841	3.40737474801452e-05	\\
-9036.11505681818	3.47620370194754e-05	\\
-9035.13627485795	3.4854395134876e-05	\\
-9034.15749289773	3.46433376476302e-05	\\
-9033.1787109375	3.4371975702829e-05	\\
-9032.19992897727	3.40714337888001e-05	\\
-9031.22114701705	3.50989545305179e-05	\\
-9030.24236505682	3.55468746799223e-05	\\
-9029.26358309659	3.44360302031236e-05	\\
-9028.28480113636	3.45448765365101e-05	\\
-9027.30601917614	3.4410970464839e-05	\\
-9026.32723721591	3.54526496556664e-05	\\
-9025.34845525568	3.55329774469181e-05	\\
-9024.36967329545	3.36897679838421e-05	\\
-9023.39089133523	3.42265079201813e-05	\\
-9022.412109375	3.51880761315605e-05	\\
-9021.43332741477	3.47130414553513e-05	\\
-9020.45454545455	3.5178254443932e-05	\\
-9019.47576349432	3.55011650028834e-05	\\
-9018.49698153409	3.53536796044081e-05	\\
-9017.51819957386	3.53400706052118e-05	\\
-9016.53941761364	3.54268432128288e-05	\\
-9015.56063565341	3.47679791605881e-05	\\
-9014.58185369318	3.38907060505562e-05	\\
-9013.60307173295	3.59550960798586e-05	\\
-9012.62428977273	3.6385652795681e-05	\\
-9011.6455078125	3.53776919381762e-05	\\
-9010.66672585227	3.59723602533434e-05	\\
-9009.68794389205	3.62721697931987e-05	\\
-9008.70916193182	3.42822748111604e-05	\\
-9007.73037997159	3.63517945796758e-05	\\
-9006.75159801136	3.45668168908066e-05	\\
-9005.77281605114	3.65520999495407e-05	\\
-9004.79403409091	3.48412144853566e-05	\\
-9003.81525213068	3.66022444698726e-05	\\
-9002.83647017045	3.60809790758251e-05	\\
-9001.85768821023	3.56013856220845e-05	\\
-9000.87890625	3.68288922174011e-05	\\
-8999.90012428977	3.67084134389161e-05	\\
-8998.92134232955	3.61253893460854e-05	\\
-8997.94256036932	3.6731730399503e-05	\\
-8996.96377840909	3.68928876658185e-05	\\
-8995.98499644886	3.78175340679518e-05	\\
-8995.00621448864	3.67006443281529e-05	\\
-8994.02743252841	3.65646279271205e-05	\\
-8993.04865056818	3.62597006540812e-05	\\
-8992.06986860795	3.58605775056217e-05	\\
-8991.09108664773	3.5927238744842e-05	\\
-8990.1123046875	3.70500711189054e-05	\\
-8989.13352272727	3.68658284581842e-05	\\
-8988.15474076705	3.55522596542836e-05	\\
-8987.17595880682	3.58473137730007e-05	\\
-8986.19717684659	3.77082796673688e-05	\\
-8985.21839488636	3.62407703914202e-05	\\
-8984.23961292614	3.58495090192708e-05	\\
-8983.26083096591	3.63967489747308e-05	\\
-8982.28204900568	3.68552203442735e-05	\\
-8981.30326704545	3.59609949267012e-05	\\
-8980.32448508523	3.64297364462205e-05	\\
-8979.345703125	3.80012625315918e-05	\\
-8978.36692116477	3.65936403085491e-05	\\
-8977.38813920455	3.75225608419159e-05	\\
-8976.40935724432	3.72992504750693e-05	\\
-8975.43057528409	3.73880809144668e-05	\\
-8974.45179332386	3.60288109585705e-05	\\
-8973.47301136364	3.71880083921362e-05	\\
-8972.49422940341	3.82525025495316e-05	\\
-8971.51544744318	3.77924498736563e-05	\\
-8970.53666548295	3.7961070764302e-05	\\
-8969.55788352273	3.79894818826579e-05	\\
-8968.5791015625	3.70137449314885e-05	\\
-8967.60031960227	3.7979295565608e-05	\\
-8966.62153764205	3.70564150605827e-05	\\
-8965.64275568182	3.70147185625517e-05	\\
-8964.66397372159	3.91982385309254e-05	\\
-8963.68519176136	3.6880552906319e-05	\\
-8962.70640980114	3.63941028665862e-05	\\
-8961.72762784091	3.77375603105612e-05	\\
-8960.74884588068	3.82863908297287e-05	\\
-8959.77006392045	3.76233593108461e-05	\\
-8958.79128196023	3.71724860622591e-05	\\
-8957.8125	3.65034084763043e-05	\\
-8956.83371803977	3.75580416756278e-05	\\
-8955.85493607955	3.6675471243867e-05	\\
-8954.87615411932	3.75223712530922e-05	\\
-8953.89737215909	3.67730281160418e-05	\\
-8952.91859019886	3.6351878707512e-05	\\
-8951.93980823864	3.64321095288866e-05	\\
-8950.96102627841	3.62293289116091e-05	\\
-8949.98224431818	3.61912691059822e-05	\\
-8949.00346235795	3.85679152120282e-05	\\
-8948.02468039773	3.76284243142408e-05	\\
-8947.0458984375	3.72159522587977e-05	\\
-8946.06711647727	3.88915188734338e-05	\\
-8945.08833451705	3.86441716204e-05	\\
-8944.10955255682	3.69207727164567e-05	\\
-8943.13077059659	3.78175349249725e-05	\\
-8942.15198863636	3.8155927950047e-05	\\
-8941.17320667614	3.73792439374215e-05	\\
-8940.19442471591	3.73598224961478e-05	\\
-8939.21564275568	3.75978921752074e-05	\\
-8938.23686079545	3.74075802918742e-05	\\
-8937.25807883523	3.73255694801774e-05	\\
-8936.279296875	3.71794346682288e-05	\\
-8935.30051491477	3.75238119947933e-05	\\
-8934.32173295455	3.8064193847808e-05	\\
-8933.34295099432	3.68394824936465e-05	\\
-8932.36416903409	3.78540716392438e-05	\\
-8931.38538707386	3.78099427739615e-05	\\
-8930.40660511364	3.58597108838222e-05	\\
-8929.42782315341	3.80385344549664e-05	\\
-8928.44904119318	3.73297710466731e-05	\\
-8927.47025923295	3.77480725531402e-05	\\
-8926.49147727273	3.77292245347755e-05	\\
-8925.5126953125	3.72025654565163e-05	\\
-8924.53391335227	3.77104955214297e-05	\\
-8923.55513139205	3.76129898358168e-05	\\
-8922.57634943182	3.78399225815482e-05	\\
-8921.59756747159	3.63951616360683e-05	\\
-8920.61878551136	3.68991481249312e-05	\\
-8919.64000355114	3.72697735955243e-05	\\
-8918.66122159091	3.70191220529237e-05	\\
-8917.68243963068	3.66239316795808e-05	\\
-8916.70365767045	3.73835946225487e-05	\\
-8915.72487571023	3.75395027458464e-05	\\
-8914.74609375	3.73231322523413e-05	\\
-8913.76731178977	3.82042909414011e-05	\\
-8912.78852982955	3.72872040593601e-05	\\
-8911.80974786932	3.79158949645648e-05	\\
-8910.83096590909	3.75967667874004e-05	\\
-8909.85218394886	3.72314273731331e-05	\\
-8908.87340198864	3.69493763355335e-05	\\
-8907.89462002841	3.73047430399213e-05	\\
-8906.91583806818	3.71987201052534e-05	\\
-8905.93705610795	3.69937266949746e-05	\\
-8904.95827414773	3.74859232782955e-05	\\
-8903.9794921875	3.58990914813424e-05	\\
-8903.00071022727	3.59514931981279e-05	\\
-8902.02192826705	3.78323154902626e-05	\\
-8901.04314630682	3.86329876380185e-05	\\
-8900.06436434659	3.72209429388056e-05	\\
-8899.08558238636	3.69815492118452e-05	\\
-8898.10680042614	3.77347296705947e-05	\\
-8897.12801846591	3.65802178953337e-05	\\
-8896.14923650568	3.76239461050806e-05	\\
-8895.17045454545	3.73574392900467e-05	\\
-8894.19167258523	3.54966992575295e-05	\\
-8893.212890625	3.69936175928519e-05	\\
-8892.23410866477	3.89017575336005e-05	\\
-8891.25532670455	3.59395957891627e-05	\\
-8890.27654474432	3.72700934571502e-05	\\
-8889.29776278409	3.70960857217468e-05	\\
-8888.31898082386	3.66634349118405e-05	\\
-8887.34019886364	3.60960832813572e-05	\\
-8886.36141690341	3.62180552680539e-05	\\
-8885.38263494318	3.62950515286735e-05	\\
-8884.40385298295	3.66291484223176e-05	\\
-8883.42507102273	3.76531582182619e-05	\\
-8882.4462890625	3.62354691653643e-05	\\
-8881.46750710227	3.63237776935159e-05	\\
-8880.48872514205	3.69369839904925e-05	\\
-8879.50994318182	3.64075858748088e-05	\\
-8878.53116122159	3.61863785737248e-05	\\
-8877.55237926136	3.63891339373116e-05	\\
-8876.57359730114	3.67778524299876e-05	\\
-8875.59481534091	3.60012376606707e-05	\\
-8874.61603338068	3.75390136770465e-05	\\
-8873.63725142045	3.65654816602658e-05	\\
-8872.65846946023	3.72541515819041e-05	\\
-8871.6796875	3.66018077730566e-05	\\
-8870.70090553977	3.57446269511764e-05	\\
-8869.72212357955	3.63759639513805e-05	\\
-8868.74334161932	3.66439136865607e-05	\\
-8867.76455965909	3.7331792146693e-05	\\
-8866.78577769886	3.57600782823541e-05	\\
-8865.80699573864	3.59310525739154e-05	\\
-8864.82821377841	3.77222604743304e-05	\\
-8863.84943181818	3.63103389366018e-05	\\
-8862.87064985795	3.61487037595057e-05	\\
-8861.89186789773	3.70995040833479e-05	\\
-8860.9130859375	3.67138896683207e-05	\\
-8859.93430397727	3.65179272268435e-05	\\
-8858.95552201705	3.646808539286e-05	\\
-8857.97674005682	3.68327447743709e-05	\\
-8856.99795809659	3.62637410371413e-05	\\
-8856.01917613636	3.70479929354772e-05	\\
-8855.04039417614	3.63281562957446e-05	\\
-8854.06161221591	3.68877596086084e-05	\\
-8853.08283025568	3.61864243469248e-05	\\
-8852.10404829545	3.64446236486844e-05	\\
-8851.12526633523	3.59980980645177e-05	\\
-8850.146484375	3.63024676819779e-05	\\
-8849.16770241477	3.5341775506586e-05	\\
-8848.18892045455	3.64401060399792e-05	\\
-8847.21013849432	3.63266229179869e-05	\\
-8846.23135653409	3.61944778926837e-05	\\
-8845.25257457386	3.65365822199524e-05	\\
-8844.27379261364	3.59466772185871e-05	\\
-8843.29501065341	3.65481614099835e-05	\\
-8842.31622869318	3.55511365792439e-05	\\
-8841.33744673295	3.58255486135843e-05	\\
-8840.35866477273	3.57255643308722e-05	\\
-8839.3798828125	3.61795300410054e-05	\\
-8838.40110085227	3.55315259305209e-05	\\
-8837.42231889205	3.54937966542348e-05	\\
-8836.44353693182	3.55684095878228e-05	\\
-8835.46475497159	3.61087586737306e-05	\\
-8834.48597301136	3.54605970262053e-05	\\
-8833.50719105114	3.54787416928505e-05	\\
-8832.52840909091	3.51509443018498e-05	\\
-8831.54962713068	3.5777969605631e-05	\\
-8830.57084517045	3.59778807920417e-05	\\
-8829.59206321023	3.40763950619563e-05	\\
-8828.61328125	3.56351736897138e-05	\\
-8827.63449928977	3.47784142346828e-05	\\
-8826.65571732955	3.71809206432752e-05	\\
-8825.67693536932	3.54726604782563e-05	\\
-8824.69815340909	3.53104257557782e-05	\\
-8823.71937144886	3.66851524801232e-05	\\
-8822.74058948864	3.50121931658393e-05	\\
-8821.76180752841	3.55743748532085e-05	\\
-8820.78302556818	3.62192595436701e-05	\\
-8819.80424360795	3.61798500845555e-05	\\
-8818.82546164773	3.62455686152433e-05	\\
-8817.8466796875	3.73094963503017e-05	\\
-8816.86789772727	3.67351916729529e-05	\\
-8815.88911576705	3.7725437546036e-05	\\
-8814.91033380682	3.53815783728548e-05	\\
-8813.93155184659	3.72695234600786e-05	\\
-8812.95276988636	3.72065251155942e-05	\\
-8811.97398792614	3.60802907902746e-05	\\
-8810.99520596591	3.63793725576363e-05	\\
-8810.01642400568	3.60730283338244e-05	\\
-8809.03764204545	3.82116934008635e-05	\\
-8808.05886008523	3.62636486369105e-05	\\
-8807.080078125	3.63517352696376e-05	\\
-8806.10129616477	3.70924101490305e-05	\\
-8805.12251420455	3.7978070537174e-05	\\
-8804.14373224432	3.88782923558054e-05	\\
-8803.16495028409	3.67273402127308e-05	\\
-8802.18616832386	3.73310264341527e-05	\\
-8801.20738636364	3.71529980627277e-05	\\
-8800.22860440341	3.84798580311773e-05	\\
-8799.24982244318	3.62697424843322e-05	\\
-8798.27104048295	3.66458780549617e-05	\\
-8797.29225852273	3.69906155438354e-05	\\
-8796.3134765625	3.80545687206038e-05	\\
-8795.33469460227	3.77870101867782e-05	\\
-8794.35591264205	3.76067863783156e-05	\\
-8793.37713068182	3.79234577713924e-05	\\
-8792.39834872159	3.79995024059804e-05	\\
-8791.41956676136	3.67348439049666e-05	\\
-8790.44078480114	3.83249910971849e-05	\\
-8789.46200284091	3.7636076061452e-05	\\
-8788.48322088068	3.70537375026548e-05	\\
-8787.50443892045	3.91521869462661e-05	\\
-8786.52565696023	3.85535820028436e-05	\\
-8785.546875	3.74710013836403e-05	\\
-8784.56809303977	3.84513419169742e-05	\\
-8783.58931107955	3.7426892113358e-05	\\
-8782.61052911932	3.78165400155692e-05	\\
-8781.63174715909	3.72648912188077e-05	\\
-8780.65296519886	3.77116716015715e-05	\\
-8779.67418323864	3.7204853681466e-05	\\
-8778.69540127841	3.79070311661663e-05	\\
-8777.71661931818	3.74787169832709e-05	\\
-8776.73783735795	3.72086089989254e-05	\\
-8775.75905539773	3.7933578237507e-05	\\
-8774.7802734375	3.75408463627036e-05	\\
-8773.80149147727	3.7466578969012e-05	\\
-8772.82270951705	3.7728430846141e-05	\\
-8771.84392755682	3.73030341307643e-05	\\
-8770.86514559659	3.89642610999387e-05	\\
-8769.88636363636	3.80306094615898e-05	\\
-8768.90758167614	3.78870572305404e-05	\\
-8767.92879971591	3.76334385602198e-05	\\
-8766.95001775568	3.79028746099928e-05	\\
-8765.97123579545	3.70910299105434e-05	\\
-8764.99245383523	3.88550785518546e-05	\\
-8764.013671875	3.7327264520123e-05	\\
-8763.03488991477	3.75002050856144e-05	\\
-8762.05610795455	3.75596617181611e-05	\\
-8761.07732599432	3.71639329271546e-05	\\
-8760.09854403409	3.81177645864968e-05	\\
-8759.11976207386	3.73999060712786e-05	\\
-8758.14098011364	3.76236388310171e-05	\\
-8757.16219815341	3.88057269651611e-05	\\
-8756.18341619318	3.69859750993242e-05	\\
-8755.20463423295	3.76206969245632e-05	\\
-8754.22585227273	3.91609374364723e-05	\\
-8753.2470703125	3.87330073204568e-05	\\
-8752.26828835227	3.77378184079439e-05	\\
-8751.28950639205	3.86950552090915e-05	\\
-8750.31072443182	3.79643072027765e-05	\\
-8749.33194247159	3.75117380882068e-05	\\
-8748.35316051136	3.72902474247044e-05	\\
-8747.37437855114	3.77108398642775e-05	\\
-8746.39559659091	3.98505252082795e-05	\\
-8745.41681463068	3.7335815851773e-05	\\
-8744.43803267045	3.89946155130511e-05	\\
-8743.45925071023	3.84019953392251e-05	\\
-8742.48046875	3.80557896678447e-05	\\
-8741.50168678977	3.82520373167107e-05	\\
-8740.52290482955	3.65258644771678e-05	\\
-8739.54412286932	3.8043379474255e-05	\\
-8738.56534090909	3.8384106082418e-05	\\
-8737.58655894886	3.70262418662209e-05	\\
-8736.60777698864	3.81129268260783e-05	\\
-8735.62899502841	3.67541793562844e-05	\\
-8734.65021306818	3.84831131489078e-05	\\
-8733.67143110795	3.80019944873115e-05	\\
-8732.69264914773	3.65352889396168e-05	\\
-8731.7138671875	3.84613409220162e-05	\\
-8730.73508522727	3.79276399550872e-05	\\
-8729.75630326705	3.66329048818844e-05	\\
-8728.77752130682	3.82511268457115e-05	\\
-8727.79873934659	3.95835413490313e-05	\\
-8726.81995738636	3.78807764978677e-05	\\
-8725.84117542614	3.86151681573138e-05	\\
-8724.86239346591	3.75828762421751e-05	\\
-8723.88361150568	3.72082109544103e-05	\\
-8722.90482954545	3.91836798644131e-05	\\
-8721.92604758523	3.88744007990867e-05	\\
-8720.947265625	3.80373080408551e-05	\\
-8719.96848366477	3.85388815247265e-05	\\
-8718.98970170455	3.78218672575291e-05	\\
-8718.01091974432	3.59102875796188e-05	\\
-8717.03213778409	3.769450619707e-05	\\
-8716.05335582386	3.79004430677748e-05	\\
-8715.07457386364	3.73525701099543e-05	\\
-8714.09579190341	3.83369546223697e-05	\\
-8713.11700994318	3.73443394914086e-05	\\
-8712.13822798295	3.8224289477466e-05	\\
-8711.15944602273	3.80348538209318e-05	\\
-8710.1806640625	3.87253433011269e-05	\\
-8709.20188210227	3.86220708929807e-05	\\
-8708.22310014205	3.83929045757113e-05	\\
-8707.24431818182	3.86364593852375e-05	\\
-8706.26553622159	3.93123672221103e-05	\\
-8705.28675426136	3.82151518523491e-05	\\
-8704.30797230114	3.67750205151461e-05	\\
-8703.32919034091	3.92503682557044e-05	\\
-8702.35040838068	3.87740135864606e-05	\\
-8701.37162642045	3.76990037866008e-05	\\
-8700.39284446023	3.69445081519116e-05	\\
-8699.4140625	3.63339617065954e-05	\\
-8698.43528053977	3.71912246850568e-05	\\
-8697.45649857955	3.89549575047368e-05	\\
-8696.47771661932	3.80147665528988e-05	\\
-8695.49893465909	3.75378857585581e-05	\\
-8694.52015269886	3.81726570779765e-05	\\
-8693.54137073864	3.75102105912247e-05	\\
-8692.56258877841	3.76886633714362e-05	\\
-8691.58380681818	3.85723095976431e-05	\\
-8690.60502485795	3.74232776957609e-05	\\
-8689.62624289773	3.85423161296847e-05	\\
-8688.6474609375	3.74979353677216e-05	\\
-8687.66867897727	3.91897517810754e-05	\\
-8686.68989701705	3.72730990247313e-05	\\
-8685.71111505682	3.83570999503633e-05	\\
-8684.73233309659	3.96117284579599e-05	\\
-8683.75355113636	3.77333522889656e-05	\\
-8682.77476917614	3.89009411739531e-05	\\
-8681.79598721591	3.86662525858083e-05	\\
-8680.81720525568	3.94278083314001e-05	\\
-8679.83842329545	3.7267024655242e-05	\\
-8678.85964133523	3.92583202763138e-05	\\
-8677.880859375	3.6224662196351e-05	\\
-8676.90207741477	3.9040619518091e-05	\\
-8675.92329545455	3.78831932168682e-05	\\
-8674.94451349432	3.78320344649618e-05	\\
-8673.96573153409	3.86037101414292e-05	\\
-8672.98694957386	3.91092624068336e-05	\\
-8672.00816761364	3.84713518709519e-05	\\
-8671.02938565341	3.84520637240382e-05	\\
-8670.05060369318	3.95055523784448e-05	\\
-8669.07182173295	3.75059333654507e-05	\\
-8668.09303977273	3.80480829439216e-05	\\
-8667.1142578125	3.99671200776057e-05	\\
-8666.13547585227	3.91265490267911e-05	\\
-8665.15669389205	3.96543766455711e-05	\\
-8664.17791193182	3.97923367245605e-05	\\
-8663.19912997159	3.78538328723514e-05	\\
-8662.22034801136	3.81219141197087e-05	\\
-8661.24156605114	3.97067708086742e-05	\\
-8660.26278409091	3.94987525761927e-05	\\
-8659.28400213068	3.85240494032654e-05	\\
-8658.30522017045	3.88037628777026e-05	\\
-8657.32643821023	3.92163945784418e-05	\\
-8656.34765625	3.91448917291233e-05	\\
-8655.36887428977	4.04140193715899e-05	\\
-8654.39009232955	3.83699656306727e-05	\\
-8653.41131036932	3.8423310555787e-05	\\
-8652.43252840909	4.03634564920676e-05	\\
-8651.45374644886	3.77514791426006e-05	\\
-8650.47496448864	3.91731859188693e-05	\\
-8649.49618252841	3.73403195447204e-05	\\
-8648.51740056818	3.89874936124952e-05	\\
-8647.53861860795	3.80412891010455e-05	\\
-8646.55983664773	3.86308067757982e-05	\\
-8645.5810546875	3.96175499200328e-05	\\
-8644.60227272727	3.89544762810719e-05	\\
-8643.62349076705	3.90011026829195e-05	\\
-8642.64470880682	3.85194369785724e-05	\\
-8641.66592684659	3.73110036843769e-05	\\
-8640.68714488636	3.91983072987061e-05	\\
-8639.70836292614	3.93384592341615e-05	\\
-8638.72958096591	3.88345769140795e-05	\\
-8637.75079900568	3.95643191420755e-05	\\
-8636.77201704545	3.94034812996566e-05	\\
-8635.79323508523	3.918987477788e-05	\\
-8634.814453125	3.95338590498745e-05	\\
-8633.83567116477	3.78256758327045e-05	\\
-8632.85688920455	3.86213102971783e-05	\\
-8631.87810724432	3.96145251022283e-05	\\
-8630.89932528409	3.90064474656963e-05	\\
-8629.92054332386	3.69653591910247e-05	\\
-8628.94176136364	3.81186835540192e-05	\\
-8627.96297940341	3.91571135591166e-05	\\
-8626.98419744318	3.91270608880382e-05	\\
-8626.00541548295	3.80879328499047e-05	\\
-8625.02663352273	3.82454183412913e-05	\\
-8624.0478515625	3.93207700914462e-05	\\
-8623.06906960227	3.90992580991524e-05	\\
-8622.09028764205	3.90917127152929e-05	\\
-8621.11150568182	3.86833873725736e-05	\\
-8620.13272372159	3.84584286180043e-05	\\
-8619.15394176136	3.94844204584151e-05	\\
-8618.17515980114	3.99741829559042e-05	\\
-8617.19637784091	3.81969890376196e-05	\\
-8616.21759588068	3.93770482415632e-05	\\
-8615.23881392045	3.85460137867043e-05	\\
-8614.26003196023	3.78865758341357e-05	\\
-8613.28125	3.88206647728158e-05	\\
-8612.30246803977	3.88854895774002e-05	\\
-8611.32368607955	3.82199431699917e-05	\\
-8610.34490411932	3.77412194966007e-05	\\
-8609.36612215909	3.70973799849541e-05	\\
-8608.38734019886	3.78714851720559e-05	\\
-8607.40855823864	3.79071496973473e-05	\\
-8606.42977627841	3.81093591636512e-05	\\
-8605.45099431818	3.88884993070833e-05	\\
-8604.47221235795	3.92297274511534e-05	\\
-8603.49343039773	3.87484254143095e-05	\\
-8602.5146484375	3.95017421232273e-05	\\
-8601.53586647727	3.93219804459471e-05	\\
-8600.55708451705	3.75727787536871e-05	\\
-8599.57830255682	4.00156428856293e-05	\\
-8598.59952059659	3.79228174262163e-05	\\
-8597.62073863636	3.76779043219572e-05	\\
-8596.64195667614	3.8106072651416e-05	\\
-8595.66317471591	3.88701408098952e-05	\\
-8594.68439275568	3.71000968164927e-05	\\
-8593.70561079545	3.72580775976601e-05	\\
-8592.72682883523	3.81598293654536e-05	\\
-8591.748046875	3.80583000362612e-05	\\
-8590.76926491477	3.82220539111503e-05	\\
-8589.79048295455	3.80836269783695e-05	\\
-8588.81170099432	3.81537072421812e-05	\\
-8587.83291903409	3.87709978679726e-05	\\
-8586.85413707386	3.87915380055687e-05	\\
-8585.87535511364	3.8568094099809e-05	\\
-8584.89657315341	3.82266664790739e-05	\\
-8583.91779119318	3.86689985824115e-05	\\
-8582.93900923295	3.83671774848674e-05	\\
-8581.96022727273	3.88285901480157e-05	\\
-8580.9814453125	3.81386040781442e-05	\\
-8580.00266335227	3.89236511420472e-05	\\
-8579.02388139205	3.77929476952945e-05	\\
-8578.04509943182	3.82266691595677e-05	\\
-8577.06631747159	3.63587344819381e-05	\\
-8576.08753551136	3.91853139029642e-05	\\
-8575.10875355114	3.81472873291522e-05	\\
-8574.12997159091	3.93359430594008e-05	\\
-8573.15118963068	3.80510343112486e-05	\\
-8572.17240767045	3.82695044549078e-05	\\
-8571.19362571023	3.99106344485592e-05	\\
-8570.21484375	3.93159352320304e-05	\\
-8569.23606178977	3.79038513835989e-05	\\
-8568.25727982955	3.79803894889575e-05	\\
-8567.27849786932	3.8503513715891e-05	\\
-8566.29971590909	3.71847335678986e-05	\\
-8565.32093394886	3.74204811578109e-05	\\
-8564.34215198864	3.74601859678528e-05	\\
-8563.36337002841	3.91271298760418e-05	\\
-8562.38458806818	4.06098979336072e-05	\\
-8561.40580610795	3.80096720570724e-05	\\
-8560.42702414773	3.76333153088318e-05	\\
-8559.4482421875	3.77375587799053e-05	\\
-8558.46946022727	3.88203889115584e-05	\\
-8557.49067826705	3.83518973911994e-05	\\
-8556.51189630682	3.74401212901586e-05	\\
-8555.53311434659	3.69251510738217e-05	\\
-8554.55433238636	3.76133633833956e-05	\\
-8553.57555042614	3.87179994238447e-05	\\
-8552.59676846591	3.78809110256816e-05	\\
-8551.61798650568	3.78306763036579e-05	\\
-8550.63920454545	3.91701115059143e-05	\\
-8549.66042258523	3.99866051998831e-05	\\
-8548.681640625	3.84262536395104e-05	\\
-8547.70285866477	3.76392039738072e-05	\\
-8546.72407670455	3.9033973179346e-05	\\
-8545.74529474432	3.82123152635629e-05	\\
-8544.76651278409	3.877704243017e-05	\\
-8543.78773082386	3.78213634389041e-05	\\
-8542.80894886364	3.75909198711136e-05	\\
-8541.83016690341	3.89652495746078e-05	\\
-8540.85138494318	3.9114146567218e-05	\\
-8539.87260298295	3.59904598087141e-05	\\
-8538.89382102273	3.70369891479988e-05	\\
-8537.9150390625	3.99458467131689e-05	\\
-8536.93625710227	3.84482379253392e-05	\\
-8535.95747514205	3.98533416626203e-05	\\
-8534.97869318182	3.73158939537012e-05	\\
-8533.99991122159	3.86544276351589e-05	\\
-8533.02112926136	3.7553836128989e-05	\\
-8532.04234730114	3.95841047584115e-05	\\
-8531.06356534091	3.89951187214635e-05	\\
-8530.08478338068	3.83953508207402e-05	\\
-8529.10600142045	3.80072225591841e-05	\\
-8528.12721946023	3.79371302235582e-05	\\
-8527.1484375	3.80382123780724e-05	\\
-8526.16965553977	3.77888692201089e-05	\\
-8525.19087357955	3.83004939607335e-05	\\
-8524.21209161932	3.78873270614282e-05	\\
-8523.23330965909	3.91601135107928e-05	\\
-8522.25452769886	3.85062125494724e-05	\\
-8521.27574573864	3.71563261859809e-05	\\
-8520.29696377841	4.040432352607e-05	\\
-8519.31818181818	3.70510824209711e-05	\\
-8518.33939985795	3.80415643965463e-05	\\
-8517.36061789773	3.80499790304119e-05	\\
-8516.3818359375	3.8307180431089e-05	\\
-8515.40305397727	3.79300059159943e-05	\\
-8514.42427201705	4.00744679643551e-05	\\
-8513.44549005682	3.88707313395117e-05	\\
-8512.46670809659	3.83035840745014e-05	\\
-8511.48792613636	3.81448541870658e-05	\\
-8510.50914417614	3.83062631811216e-05	\\
-8509.53036221591	3.85027983238275e-05	\\
-8508.55158025568	3.94913488602368e-05	\\
-8507.57279829545	3.87751485962879e-05	\\
-8506.59401633523	3.88553656967881e-05	\\
-8505.615234375	3.72834949354436e-05	\\
-8504.63645241477	3.92585829130856e-05	\\
-8503.65767045455	3.92594332662701e-05	\\
-8502.67888849432	3.78381879036659e-05	\\
-8501.70010653409	4.03234448284277e-05	\\
-8500.72132457386	3.97672457067638e-05	\\
-8499.74254261364	4.07824451519674e-05	\\
-8498.76376065341	3.87077979843945e-05	\\
-8497.78497869318	3.86108416630406e-05	\\
-8496.80619673295	3.79054186592187e-05	\\
-8495.82741477273	4.0805964279697e-05	\\
-8494.8486328125	4.04029105154118e-05	\\
-8493.86985085227	3.77026768455372e-05	\\
-8492.89106889205	4.03032454235005e-05	\\
-8491.91228693182	3.87343637582664e-05	\\
-8490.93350497159	3.88373487165572e-05	\\
-8489.95472301136	3.90593812246556e-05	\\
-8488.97594105114	3.9168998258167e-05	\\
-8487.99715909091	3.98162041197475e-05	\\
-8487.01837713068	4.00739962566658e-05	\\
-8486.03959517045	3.91045865155608e-05	\\
-8485.06081321023	3.92620151697518e-05	\\
-8484.08203125	3.81773897055548e-05	\\
-8483.10324928977	3.98556055073573e-05	\\
-8482.12446732955	4.04704160059059e-05	\\
-8481.14568536932	4.03919534680806e-05	\\
-8480.16690340909	3.96319866992624e-05	\\
-8479.18812144886	4.1639313244279e-05	\\
-8478.20933948864	4.05242347059161e-05	\\
-8477.23055752841	3.82211596292648e-05	\\
-8476.25177556818	3.95594245288052e-05	\\
-8475.27299360795	3.87083998941816e-05	\\
-8474.29421164773	3.90412731378141e-05	\\
-8473.3154296875	4.0109917080028e-05	\\
-8472.33664772727	3.84996202277035e-05	\\
-8471.35786576705	4.06750078964465e-05	\\
-8470.37908380682	3.88285386276361e-05	\\
-8469.40030184659	3.8952133872025e-05	\\
-8468.42151988636	3.98091734867841e-05	\\
-8467.44273792614	3.95051599103347e-05	\\
-8466.46395596591	3.99615895615469e-05	\\
-8465.48517400568	3.95011502770812e-05	\\
-8464.50639204545	3.91527416770255e-05	\\
-8463.52761008523	4.06412300910292e-05	\\
-8462.548828125	4.16058905553862e-05	\\
-8461.57004616477	3.9219520628389e-05	\\
-8460.59126420455	3.96019035630384e-05	\\
-8459.61248224432	4.11425328740971e-05	\\
-8458.63370028409	3.97297487054378e-05	\\
-8457.65491832386	3.99010041078207e-05	\\
-8456.67613636364	4.11971164002274e-05	\\
-8455.69735440341	4.0943733031393e-05	\\
-8454.71857244318	4.03740849331571e-05	\\
-8453.73979048295	4.02869637880845e-05	\\
-8452.76100852273	3.97732225601282e-05	\\
-8451.7822265625	4.11258379999597e-05	\\
-8450.80344460227	4.18304147943712e-05	\\
-8449.82466264205	4.01007368737669e-05	\\
-8448.84588068182	4.066416636492e-05	\\
-8447.86709872159	4.05510932404132e-05	\\
-8446.88831676136	3.9900252303403e-05	\\
-8445.90953480114	4.13066721821774e-05	\\
-8444.93075284091	4.05850209259103e-05	\\
-8443.95197088068	3.89900276309597e-05	\\
-8442.97318892045	3.91211714601016e-05	\\
-8441.99440696023	4.13918913202375e-05	\\
-8441.015625	4.08830134527334e-05	\\
-8440.03684303977	4.06697896177422e-05	\\
-8439.05806107955	4.1652763929496e-05	\\
-8438.07927911932	4.05197718995299e-05	\\
-8437.10049715909	4.11201330682025e-05	\\
-8436.12171519886	4.15996337907257e-05	\\
-8435.14293323864	4.04398989156423e-05	\\
-8434.16415127841	4.12258944735648e-05	\\
-8433.18536931818	4.1197856112928e-05	\\
-8432.20658735795	4.15052199038511e-05	\\
-8431.22780539773	4.06087940000078e-05	\\
-8430.2490234375	4.13031806087722e-05	\\
-8429.27024147727	4.16118004994706e-05	\\
-8428.29145951705	3.94787374911118e-05	\\
-8427.31267755682	4.12467870504279e-05	\\
-8426.33389559659	4.01812937927147e-05	\\
-8425.35511363636	4.0203896841216e-05	\\
-8424.37633167614	4.13736828612555e-05	\\
-8423.39754971591	4.19006147655642e-05	\\
-8422.41876775568	4.15730356148315e-05	\\
-8421.43998579545	4.13049942773368e-05	\\
-8420.46120383523	4.30147202165911e-05	\\
-8419.482421875	4.06434854974232e-05	\\
-8418.50363991477	4.31381548496729e-05	\\
-8417.52485795455	4.17752895582177e-05	\\
-8416.54607599432	4.14872610681894e-05	\\
-8415.56729403409	4.10777955692319e-05	\\
-8414.58851207386	4.16975911374694e-05	\\
-8413.60973011364	4.28350827382847e-05	\\
-8412.63094815341	4.17772514883785e-05	\\
-8411.65216619318	4.18252380487723e-05	\\
-8410.67338423295	4.1646367503409e-05	\\
-8409.69460227273	4.14484421286359e-05	\\
-8408.7158203125	4.3031817388212e-05	\\
-8407.73703835227	4.10953139299527e-05	\\
-8406.75825639205	4.08541329938831e-05	\\
-8405.77947443182	4.11385170507391e-05	\\
-8404.80069247159	4.02118904404613e-05	\\
-8403.82191051136	4.10323932329143e-05	\\
-8402.84312855114	4.28353553083197e-05	\\
-8401.86434659091	3.97128029628973e-05	\\
-8400.88556463068	4.03688243535642e-05	\\
-8399.90678267045	3.98310698804272e-05	\\
-8398.92800071023	4.07288356971926e-05	\\
-8397.94921875	4.04058867547083e-05	\\
-8396.97043678977	4.2229714810865e-05	\\
-8395.99165482955	4.00562287969405e-05	\\
-8395.01287286932	4.16767572708749e-05	\\
-8394.03409090909	4.05211427254158e-05	\\
-8393.05530894886	3.96975146447727e-05	\\
-8392.07652698864	4.01811649208067e-05	\\
-8391.09774502841	3.97745337673316e-05	\\
-8390.11896306818	3.9765868758523e-05	\\
-8389.14018110795	4.06129607489185e-05	\\
-8388.16139914773	4.07338437807885e-05	\\
-8387.1826171875	3.85428032626976e-05	\\
-8386.20383522727	3.89316418811013e-05	\\
-8385.22505326705	3.94890308089347e-05	\\
-8384.24627130682	4.05809525526128e-05	\\
-8383.26748934659	3.89814981323429e-05	\\
-8382.28870738636	3.88800373843201e-05	\\
-8381.30992542614	3.87670612118474e-05	\\
-8380.33114346591	3.93988724677448e-05	\\
-8379.35236150568	4.05452090972495e-05	\\
-8378.37357954545	3.87644507137233e-05	\\
-8377.39479758523	3.91795331703585e-05	\\
-8376.416015625	3.73708317079806e-05	\\
-8375.43723366477	3.91604615360004e-05	\\
-8374.45845170455	3.79843917416458e-05	\\
-8373.47966974432	3.84698846834777e-05	\\
-8372.50088778409	3.77101697456666e-05	\\
-8371.52210582386	3.83930940802898e-05	\\
-8370.54332386364	3.69332240922197e-05	\\
-8369.56454190341	3.76978776932561e-05	\\
-8368.58575994318	3.81957974292239e-05	\\
-8367.60697798295	3.84204062758142e-05	\\
-8366.62819602273	3.81039023570795e-05	\\
-8365.6494140625	3.81170662895439e-05	\\
-8364.67063210227	3.78188733506152e-05	\\
-8363.69185014205	3.85009376070507e-05	\\
-8362.71306818182	3.83819997099858e-05	\\
-8361.73428622159	3.84837198097897e-05	\\
-8360.75550426136	3.85896014729019e-05	\\
-8359.77672230114	3.83799611656182e-05	\\
-8358.79794034091	3.87613799782199e-05	\\
-8357.81915838068	3.7744128687165e-05	\\
-8356.84037642045	3.85145969424659e-05	\\
-8355.86159446023	3.78828356306076e-05	\\
-8354.8828125	3.78963693667598e-05	\\
-8353.90403053977	3.79842173553048e-05	\\
-8352.92524857955	3.78402667623119e-05	\\
-8351.94646661932	3.90529524804164e-05	\\
-8350.96768465909	3.79460180913635e-05	\\
-8349.98890269886	3.66271237165463e-05	\\
-8349.01012073864	3.7555990706342e-05	\\
-8348.03133877841	3.84888854068843e-05	\\
-8347.05255681818	3.65276155898395e-05	\\
-8346.07377485795	3.83560227238413e-05	\\
-8345.09499289773	3.76946606456098e-05	\\
-8344.1162109375	3.80163401785955e-05	\\
-8343.13742897727	3.95585496299111e-05	\\
-8342.15864701705	3.84970535283834e-05	\\
-8341.17986505682	3.69864727747998e-05	\\
-8340.20108309659	3.66366219505613e-05	\\
-8339.22230113636	3.77728219766722e-05	\\
-8338.24351917614	3.84225987167055e-05	\\
-8337.26473721591	3.67554472739613e-05	\\
-8336.28595525568	3.70952144197993e-05	\\
-8335.30717329545	3.67502160846413e-05	\\
-8334.32839133523	3.79376430797374e-05	\\
-8333.349609375	3.79594115646458e-05	\\
-8332.37082741477	3.71921856189969e-05	\\
-8331.39204545455	3.67651950944176e-05	\\
-8330.41326349432	3.79215730016954e-05	\\
-8329.43448153409	3.5207543109991e-05	\\
-8328.45569957386	3.75721387232718e-05	\\
-8327.47691761364	3.85288504125766e-05	\\
-8326.49813565341	3.62434790014895e-05	\\
-8325.51935369318	3.63577160451926e-05	\\
-8324.54057173295	3.70614715070925e-05	\\
-8323.56178977273	3.86523915069368e-05	\\
-8322.5830078125	3.64702850216157e-05	\\
-8321.60422585227	3.7372295761056e-05	\\
-8320.62544389205	3.78151403173928e-05	\\
-8319.64666193182	3.78449985666201e-05	\\
-8318.66787997159	3.68602685980674e-05	\\
-8317.68909801136	3.58480828672085e-05	\\
-8316.71031605114	3.60755005865146e-05	\\
-8315.73153409091	3.5989889373587e-05	\\
-8314.75275213068	3.53135163233306e-05	\\
-8313.77397017045	3.65406699359748e-05	\\
-8312.79518821023	3.86983518609392e-05	\\
-8311.81640625	3.60125371081305e-05	\\
-8310.83762428977	3.60796975036852e-05	\\
-8309.85884232955	3.75999081225623e-05	\\
-8308.88006036932	3.62587535379013e-05	\\
-8307.90127840909	3.71693285168029e-05	\\
-8306.92249644886	3.69842485963094e-05	\\
-8305.94371448864	3.5301450696345e-05	\\
-8304.96493252841	3.51103677709026e-05	\\
-8303.98615056818	3.55147302676569e-05	\\
-8303.00736860795	3.75309203774366e-05	\\
-8302.02858664773	3.68026723927931e-05	\\
-8301.0498046875	3.67626099663853e-05	\\
-8300.07102272727	3.78489839631852e-05	\\
-8299.09224076705	3.62918136742941e-05	\\
-8298.11345880682	3.57180990465348e-05	\\
-8297.13467684659	3.53701555228246e-05	\\
-8296.15589488636	3.45382175461711e-05	\\
-8295.17711292614	3.50463066165733e-05	\\
-8294.19833096591	3.67124833773875e-05	\\
-8293.21954900568	3.71080074948671e-05	\\
-8292.24076704545	3.57293386878006e-05	\\
-8291.26198508523	3.62566269755596e-05	\\
-8290.283203125	3.57639614275131e-05	\\
-8289.30442116477	3.69134373435027e-05	\\
-8288.32563920455	3.64957268138711e-05	\\
-8287.34685724432	3.66721293337844e-05	\\
-8286.36807528409	3.52983908549734e-05	\\
-8285.38929332386	3.56008388530685e-05	\\
-8284.41051136364	3.36680810576592e-05	\\
-8283.43172940341	3.57966750082625e-05	\\
-8282.45294744318	3.53346239331435e-05	\\
-8281.47416548295	3.48758717366232e-05	\\
-8280.49538352273	3.53653736000425e-05	\\
-8279.5166015625	3.6822834663405e-05	\\
-8278.53781960227	3.50769084461519e-05	\\
-8277.55903764205	3.47934734533585e-05	\\
-8276.58025568182	3.51464903520001e-05	\\
-8275.60147372159	3.54396820265102e-05	\\
-8274.62269176136	3.56544210941821e-05	\\
-8273.64390980114	3.56822580893229e-05	\\
-8272.66512784091	3.44603308765508e-05	\\
-8271.68634588068	3.41057331800898e-05	\\
-8270.70756392045	3.50583638284358e-05	\\
-8269.72878196023	3.59857058134351e-05	\\
-8268.75	3.54314701793702e-05	\\
-8267.77121803977	3.49460877558818e-05	\\
-8266.79243607955	3.53712923557717e-05	\\
-8265.81365411932	3.42562306303051e-05	\\
-8264.83487215909	3.67689136850442e-05	\\
-8263.85609019886	3.55248708889433e-05	\\
-8262.87730823864	3.59208978042444e-05	\\
-8261.89852627841	3.61931322317153e-05	\\
-8260.91974431818	3.35000323548345e-05	\\
-8259.94096235795	3.28448544720268e-05	\\
-8258.96218039773	3.57379735856253e-05	\\
-8257.9833984375	3.41589977564932e-05	\\
-8257.00461647727	3.45665992306085e-05	\\
-8256.02583451705	3.51276706872047e-05	\\
-8255.04705255682	3.51161229826187e-05	\\
-8254.06827059659	3.43850755174653e-05	\\
-8253.08948863636	3.47468000113596e-05	\\
-8252.11070667614	3.51061238138358e-05	\\
-8251.13192471591	3.6002979841789e-05	\\
-8250.15314275568	4.00140252684741e-05	\\
-8249.17436079545	3.35307917257502e-05	\\
-8248.19557883523	3.61712503686071e-05	\\
-8247.216796875	3.73558857038416e-05	\\
-8246.23801491477	3.58912466988522e-05	\\
-8245.25923295455	3.553593059418e-05	\\
-8244.28045099432	3.63486853688073e-05	\\
-8243.30166903409	3.57449520621201e-05	\\
-8242.32288707386	3.50111584620484e-05	\\
-8241.34410511364	3.64306981616779e-05	\\
-8240.36532315341	3.64255652689123e-05	\\
-8239.38654119318	3.56039497203189e-05	\\
-8238.40775923295	3.57829215824756e-05	\\
-8237.42897727273	3.62956779135241e-05	\\
-8236.4501953125	3.5988922293434e-05	\\
-8235.47141335227	3.54899555153065e-05	\\
-8234.49263139205	3.6650780957231e-05	\\
-8233.51384943182	3.53769288758416e-05	\\
-8232.53506747159	3.55786157186919e-05	\\
-8231.55628551136	3.66633101879338e-05	\\
-8230.57750355114	3.51968520113303e-05	\\
-8229.59872159091	3.61753481162646e-05	\\
-8228.61993963068	3.51308118818115e-05	\\
-8227.64115767045	3.602491192671e-05	\\
-8226.66237571023	3.65363999081928e-05	\\
-8225.68359375	3.51781498065508e-05	\\
-8224.70481178977	3.66103774837057e-05	\\
-8223.72602982955	3.52577637232487e-05	\\
-8222.74724786932	3.42054976538617e-05	\\
-8221.76846590909	3.58005057377411e-05	\\
-8220.78968394886	3.45180158040785e-05	\\
-8219.81090198864	3.70780891131582e-05	\\
-8218.83212002841	3.55671342622638e-05	\\
-8217.85333806818	3.48245088781424e-05	\\
-8216.87455610795	3.57881214178459e-05	\\
-8215.89577414773	3.5360871602945e-05	\\
-8214.9169921875	3.53325567764087e-05	\\
-8213.93821022727	3.66995034345582e-05	\\
-8212.95942826705	3.43864567421539e-05	\\
-8211.98064630682	3.49948742614258e-05	\\
-8211.00186434659	3.61773426861355e-05	\\
-8210.02308238636	3.31734724045094e-05	\\
-8209.04430042614	3.55003184805522e-05	\\
-8208.06551846591	3.57895033899568e-05	\\
-8207.08673650568	3.52992876730669e-05	\\
-8206.10795454545	3.5436084943665e-05	\\
-8205.12917258523	3.61400813912464e-05	\\
-8204.150390625	3.47174945474721e-05	\\
-8203.17160866477	3.49051816500613e-05	\\
-8202.19282670455	3.52585464448814e-05	\\
-8201.21404474432	3.52640878254584e-05	\\
-8200.23526278409	3.443407110482e-05	\\
-8199.25648082386	3.46552798898884e-05	\\
-8198.27769886364	3.48516832670175e-05	\\
-8197.29891690341	3.39899611123671e-05	\\
-8196.32013494318	3.59694633254599e-05	\\
-8195.34135298295	3.3445019719472e-05	\\
-8194.36257102273	3.44019193997875e-05	\\
-8193.3837890625	3.56770137478292e-05	\\
-8192.40500710227	3.41864059246603e-05	\\
-8191.42622514205	3.36044797328752e-05	\\
-8190.44744318182	3.58473112155901e-05	\\
-8189.46866122159	3.45555197950142e-05	\\
-8188.48987926136	3.32707237740572e-05	\\
-8187.51109730114	3.30828771875952e-05	\\
-8186.53231534091	3.29766656730071e-05	\\
-8185.55353338068	3.51940661988274e-05	\\
-8184.57475142045	3.48861970608363e-05	\\
-8183.59596946023	3.51463179750887e-05	\\
-8182.6171875	3.40492723188612e-05	\\
-8181.63840553977	3.41521612180942e-05	\\
-8180.65962357955	3.38852529211228e-05	\\
-8179.68084161932	3.38921905631744e-05	\\
-8178.70205965909	3.51194661015622e-05	\\
-8177.72327769886	3.46191414898837e-05	\\
-8176.74449573864	3.36876083311778e-05	\\
-8175.76571377841	3.33584406293061e-05	\\
-8174.78693181818	3.39423334744876e-05	\\
-8173.80814985795	3.43150732990225e-05	\\
-8172.82936789773	3.32521642982731e-05	\\
-8171.8505859375	3.23488107896063e-05	\\
-8170.87180397727	3.32822768901607e-05	\\
-8169.89302201705	3.34668647456774e-05	\\
-8168.91424005682	3.17986751400056e-05	\\
-8167.93545809659	3.45092047798739e-05	\\
-8166.95667613636	3.34807891629375e-05	\\
-8165.97789417614	3.28154309562358e-05	\\
-8164.99911221591	3.49319268745492e-05	\\
-8164.02033025568	3.38292720067278e-05	\\
-8163.04154829545	3.28516356721189e-05	\\
-8162.06276633523	3.3458207845567e-05	\\
-8161.083984375	3.42367229724511e-05	\\
-8160.10520241477	3.2527873489477e-05	\\
-8159.12642045455	3.24951391480697e-05	\\
-8158.14763849432	3.27738287248799e-05	\\
-8157.16885653409	3.34428013085695e-05	\\
-8156.19007457386	3.3806246958122e-05	\\
-8155.21129261364	3.31230341208227e-05	\\
-8154.23251065341	3.35936152043816e-05	\\
-8153.25372869318	3.36411974597025e-05	\\
-8152.27494673295	3.27397441056637e-05	\\
-8151.29616477273	3.17549072384865e-05	\\
-8150.3173828125	3.26447152094841e-05	\\
-8149.33860085227	3.34822903690027e-05	\\
-8148.35981889205	3.33563265845016e-05	\\
-8147.38103693182	3.36341519137854e-05	\\
-8146.40225497159	3.41680314361504e-05	\\
-8145.42347301136	3.32092630331779e-05	\\
-8144.44469105114	3.41349278847754e-05	\\
-8143.46590909091	3.21616144640236e-05	\\
-8142.48712713068	3.33476561261373e-05	\\
-8141.50834517045	3.33745033711729e-05	\\
-8140.52956321023	3.28065661037026e-05	\\
-8139.55078125	3.20111760134553e-05	\\
-8138.57199928977	3.47351597088113e-05	\\
-8137.59321732955	3.43602821147628e-05	\\
-8136.61443536932	3.17772271269069e-05	\\
-8135.63565340909	3.45867727358813e-05	\\
-8134.65687144886	3.57056195089183e-05	\\
-8133.67808948864	3.39396245244849e-05	\\
-8132.69930752841	3.388006665966e-05	\\
-8131.72052556818	3.4350002212493e-05	\\
-8130.74174360795	3.38281459434588e-05	\\
-8129.76296164773	3.51103761205347e-05	\\
-8128.7841796875	3.32613212491273e-05	\\
-8127.80539772727	3.40028639361138e-05	\\
-8126.82661576705	3.34827323136633e-05	\\
-8125.84783380682	3.38261820630461e-05	\\
-8124.86905184659	3.35899563747163e-05	\\
-8123.89026988636	3.48189833538593e-05	\\
-8122.91148792614	3.28994849799387e-05	\\
-8121.93270596591	3.36269629252197e-05	\\
-8120.95392400568	3.34745782524903e-05	\\
-8119.97514204545	3.42344837961406e-05	\\
-8118.99636008523	3.23547952984022e-05	\\
-8118.017578125	3.24672887418382e-05	\\
-8117.03879616477	3.13583017284524e-05	\\
-8116.06001420455	3.17548822083382e-05	\\
-8115.08123224432	3.43131976971859e-05	\\
-8114.10245028409	3.23458590630623e-05	\\
-8113.12366832386	3.24440673667401e-05	\\
-8112.14488636364	3.31613263993322e-05	\\
-8111.16610440341	3.18350570227718e-05	\\
-8110.18732244318	3.22998108344063e-05	\\
-8109.20854048295	3.21364461640194e-05	\\
-8108.22975852273	3.35256253662249e-05	\\
-8107.2509765625	3.21931227920503e-05	\\
-8106.27219460227	3.20439271387003e-05	\\
-8105.29341264205	3.2385128254658e-05	\\
-8104.31463068182	3.22299054173699e-05	\\
-8103.33584872159	3.28626026944726e-05	\\
-8102.35706676136	3.08313027580968e-05	\\
-8101.37828480114	3.24223207675804e-05	\\
-8100.39950284091	3.1140767441223e-05	\\
-8099.42072088068	3.1407418256466e-05	\\
-8098.44193892045	3.09119951410593e-05	\\
-8097.46315696023	3.0384135308965e-05	\\
-8096.484375	3.24610116902716e-05	\\
-8095.50559303977	3.35039469818767e-05	\\
-8094.52681107955	3.2531846197881e-05	\\
-8093.54802911932	3.24515258610372e-05	\\
-8092.56924715909	2.98022217692922e-05	\\
-8091.59046519886	3.06154371638188e-05	\\
-8090.61168323864	3.08351058083734e-05	\\
-8089.63290127841	3.11684242531459e-05	\\
-8088.65411931818	3.17429007297425e-05	\\
-8087.67533735795	3.13415109281651e-05	\\
-8086.69655539773	2.95625914422172e-05	\\
-8085.7177734375	3.17478939734785e-05	\\
-8084.73899147727	3.01992807641484e-05	\\
-8083.76020951705	2.9811577794107e-05	\\
-8082.78142755682	3.08606687844866e-05	\\
-8081.80264559659	3.1662772528104e-05	\\
-8080.82386363636	3.08494722982799e-05	\\
-8079.84508167614	3.01346185430475e-05	\\
-8078.86629971591	3.17607217463131e-05	\\
-8077.88751775568	3.19102207955314e-05	\\
-8076.90873579545	3.12001089769665e-05	\\
-8075.92995383523	3.08670207226933e-05	\\
-8074.951171875	3.01950153515133e-05	\\
-8073.97238991477	3.10360952749751e-05	\\
-8072.99360795455	3.13601051488509e-05	\\
-8072.01482599432	3.30992444764417e-05	\\
-8071.03604403409	3.11952272961128e-05	\\
-8070.05726207386	3.02082236085706e-05	\\
-8069.07848011364	3.06589481828386e-05	\\
-8068.09969815341	3.15627819225906e-05	\\
-8067.12091619318	3.03253469514084e-05	\\
-8066.14213423295	3.17596339536046e-05	\\
-8065.16335227273	3.1210903333145e-05	\\
-8064.1845703125	3.1382458511018e-05	\\
-8063.20578835227	3.10639648127738e-05	\\
-8062.22700639205	3.17601310575024e-05	\\
-8061.24822443182	3.22443504501274e-05	\\
-8060.26944247159	3.36515955546537e-05	\\
-8059.29066051136	3.13524701753458e-05	\\
-8058.31187855114	3.08129849785176e-05	\\
-8057.33309659091	3.18208336795935e-05	\\
-8056.35431463068	3.19247329383547e-05	\\
-8055.37553267045	3.17465394704408e-05	\\
-8054.39675071023	3.11296545820892e-05	\\
-8053.41796875	3.24919006327258e-05	\\
-8052.43918678977	3.18776746011988e-05	\\
-8051.46040482955	3.04053163454871e-05	\\
-8050.48162286932	3.10700891212369e-05	\\
-8049.50284090909	3.24201501083758e-05	\\
-8048.52405894886	3.34645296330038e-05	\\
-8047.54527698864	3.25380107148571e-05	\\
-8046.56649502841	3.10939472626834e-05	\\
-8045.58771306818	3.24141208683334e-05	\\
-8044.60893110795	3.06784198149232e-05	\\
-8043.63014914773	3.19753991509812e-05	\\
-8042.6513671875	3.11287440446363e-05	\\
-8041.67258522727	3.18271922262778e-05	\\
-8040.69380326705	3.24479728979067e-05	\\
-8039.71502130682	3.21024271146531e-05	\\
-8038.73623934659	3.19514954279859e-05	\\
-8037.75745738636	3.28446661397575e-05	\\
-8036.77867542614	3.24586258740192e-05	\\
-8035.79989346591	3.26716121982874e-05	\\
-8034.82111150568	3.19622369000866e-05	\\
-8033.84232954545	3.3167172335819e-05	\\
-8032.86354758523	3.28715271617532e-05	\\
-8031.884765625	3.20647057232854e-05	\\
-8030.90598366477	3.30008014105396e-05	\\
-8029.92720170455	3.31213521744607e-05	\\
-8028.94841974432	3.16646893180075e-05	\\
-8027.96963778409	3.11354891493142e-05	\\
-8026.99085582386	3.3119371447122e-05	\\
-8026.01207386364	3.16956144194341e-05	\\
-8025.03329190341	3.24278865256925e-05	\\
-8024.05450994318	3.25455973942456e-05	\\
-8023.07572798295	3.13561680642792e-05	\\
-8022.09694602273	3.44307454528973e-05	\\
-8021.1181640625	3.34869160076864e-05	\\
-8020.13938210227	3.27539216662293e-05	\\
-8019.16060014205	3.30901078281958e-05	\\
-8018.18181818182	3.28250065164994e-05	\\
-8017.20303622159	3.17025259921667e-05	\\
-8016.22425426136	3.24787580938939e-05	\\
-8015.24547230114	3.32528746442125e-05	\\
-8014.26669034091	3.24394292503158e-05	\\
-8013.28790838068	3.40231889264962e-05	\\
-8012.30912642045	3.18006584932326e-05	\\
-8011.33034446023	3.19422983802912e-05	\\
-8010.3515625	3.18412069600018e-05	\\
-8009.37278053977	3.22977354051855e-05	\\
-8008.39399857955	3.30373845495224e-05	\\
-8007.41521661932	3.29233663502772e-05	\\
-8006.43643465909	3.21337848536209e-05	\\
-8005.45765269886	3.30737606520665e-05	\\
-8004.47887073864	3.16380991621738e-05	\\
-8003.50008877841	3.13293297567125e-05	\\
-8002.52130681818	3.08000381640645e-05	\\
-8001.54252485795	3.30945193534683e-05	\\
-8000.56374289773	3.45678178188413e-05	\\
-7999.5849609375	3.83316593774093e-05	\\
-7998.60617897727	3.13940469797233e-05	\\
-7997.62739701705	3.05967517841875e-05	\\
-7996.64861505682	3.08340304945578e-05	\\
-7995.66983309659	3.25440947615856e-05	\\
-7994.69105113636	3.03150613204164e-05	\\
-7993.71226917614	3.20927262664657e-05	\\
-7992.73348721591	3.21782008377577e-05	\\
-7991.75470525568	3.08743217183603e-05	\\
-7990.77592329545	3.27426043592602e-05	\\
-7989.79714133523	3.15149898892235e-05	\\
-7988.818359375	3.13898871210879e-05	\\
-7987.83957741477	3.24579852426316e-05	\\
-7986.86079545455	3.05565733145352e-05	\\
-7985.88201349432	3.09967576953943e-05	\\
-7984.90323153409	3.19749179623479e-05	\\
-7983.92444957386	3.14858709531216e-05	\\
-7982.94566761364	3.01663087269554e-05	\\
-7981.96688565341	3.12521882045693e-05	\\
-7980.98810369318	3.1422612079107e-05	\\
-7980.00932173295	3.21330426856242e-05	\\
-7979.03053977273	3.13892953087805e-05	\\
-7978.0517578125	3.15684598576788e-05	\\
-7977.07297585227	3.13963982872312e-05	\\
-7976.09419389205	3.2662755426189e-05	\\
-7975.11541193182	3.05516776657007e-05	\\
-7974.13662997159	2.84820944678598e-05	\\
-7973.15784801136	3.14659670076826e-05	\\
-7972.17906605114	3.16868915758373e-05	\\
-7971.20028409091	2.9423666746685e-05	\\
-7970.22150213068	2.98868773311331e-05	\\
-7969.24272017045	3.17902403416144e-05	\\
-7968.26393821023	2.9937563459737e-05	\\
-7967.28515625	2.98772776247767e-05	\\
-7966.30637428977	2.91310293665638e-05	\\
-7965.32759232955	2.97376475493856e-05	\\
-7964.34881036932	2.84901269192345e-05	\\
-7963.37002840909	3.0609783414455e-05	\\
-7962.39124644886	3.07454434706971e-05	\\
-7961.41246448864	2.81651562451402e-05	\\
-7960.43368252841	3.08269907272127e-05	\\
-7959.45490056818	2.84332915701471e-05	\\
-7958.47611860795	3.03198181158811e-05	\\
-7957.49733664773	2.96289460438627e-05	\\
-7956.5185546875	2.95356765495986e-05	\\
-7955.53977272727	2.7967853472324e-05	\\
-7954.56099076705	2.88647369351713e-05	\\
-7953.58220880682	2.96094562037967e-05	\\
-7952.60342684659	3.07360200816683e-05	\\
-7951.62464488636	2.89552704245774e-05	\\
-7950.64586292614	2.82891442138914e-05	\\
-7949.66708096591	2.84493920983862e-05	\\
-7948.68829900568	2.85150206705161e-05	\\
-7947.70951704545	2.84245874810669e-05	\\
-7946.73073508523	2.94897334949418e-05	\\
-7945.751953125	3.01453285568585e-05	\\
-7944.77317116477	2.92864174594588e-05	\\
-7943.79438920455	2.95936423105226e-05	\\
-7942.81560724432	3.10295974396694e-05	\\
-7941.83682528409	2.81536852889778e-05	\\
-7940.85804332386	2.89979218030825e-05	\\
-7939.87926136364	2.9310583303834e-05	\\
-7938.90047940341	2.8964741404797e-05	\\
-7937.92169744318	2.86119318080355e-05	\\
-7936.94291548295	2.93288615555426e-05	\\
-7935.96413352273	2.68177489314724e-05	\\
-7934.9853515625	2.66342406040669e-05	\\
-7934.00656960227	2.85774043705544e-05	\\
-7933.02778764205	2.86940137591154e-05	\\
-7932.04900568182	2.83135487878528e-05	\\
-7931.07022372159	2.90093514397804e-05	\\
-7930.09144176136	2.86193191392224e-05	\\
-7929.11265980114	2.83144662408214e-05	\\
-7928.13387784091	2.88817156703215e-05	\\
-7927.15509588068	2.81135943160307e-05	\\
-7926.17631392045	2.98247346462294e-05	\\
-7925.19753196023	2.66947317112195e-05	\\
-7924.21875	2.95635660022946e-05	\\
-7923.23996803977	2.97719829698332e-05	\\
-7922.26118607955	3.01615606000153e-05	\\
-7921.28240411932	2.8125958068659e-05	\\
-7920.30362215909	2.65597197376456e-05	\\
-7919.32484019886	2.82504704524755e-05	\\
-7918.34605823864	2.87628453227302e-05	\\
-7917.36727627841	2.98570120409465e-05	\\
-7916.38849431818	2.75131615996815e-05	\\
-7915.40971235795	2.84710645135599e-05	\\
-7914.43093039773	2.74922527641733e-05	\\
-7913.4521484375	2.71925554546832e-05	\\
-7912.47336647727	2.69252370858375e-05	\\
-7911.49458451705	2.78008124554229e-05	\\
-7910.51580255682	2.88809814570741e-05	\\
-7909.53702059659	2.84317346859549e-05	\\
-7908.55823863636	2.76914241479099e-05	\\
-7907.57945667614	2.72714218956158e-05	\\
-7906.60067471591	2.84701393884432e-05	\\
-7905.62189275568	2.78309352355993e-05	\\
-7904.64311079545	2.74106541184283e-05	\\
-7903.66432883523	2.84393817487968e-05	\\
-7902.685546875	2.81285722538451e-05	\\
-7901.70676491477	2.6844737514124e-05	\\
-7900.72798295455	2.73832949831988e-05	\\
-7899.74920099432	2.82233455013402e-05	\\
-7898.77041903409	2.8073495752112e-05	\\
-7897.79163707386	2.77446233599837e-05	\\
-7896.81285511364	2.78716334248209e-05	\\
-7895.83407315341	2.80538342237721e-05	\\
-7894.85529119318	2.84441542673803e-05	\\
-7893.87650923295	2.62445049509981e-05	\\
-7892.89772727273	2.70604205567386e-05	\\
-7891.9189453125	2.79467835913453e-05	\\
-7890.94016335227	2.74497891701829e-05	\\
-7889.96138139205	2.74174903076424e-05	\\
-7888.98259943182	2.79813844416726e-05	\\
-7888.00381747159	2.79132395105657e-05	\\
-7887.02503551136	2.71602047876375e-05	\\
-7886.04625355114	2.64355808340714e-05	\\
-7885.06747159091	2.57401033181527e-05	\\
-7884.08868963068	2.74796795119031e-05	\\
-7883.10990767045	2.82846693617354e-05	\\
-7882.13112571023	2.66055685355794e-05	\\
-7881.15234375	2.62324944591112e-05	\\
-7880.17356178977	2.62606648289882e-05	\\
-7879.19477982955	2.68173961008909e-05	\\
-7878.21599786932	2.75167999760518e-05	\\
-7877.23721590909	2.76341254327034e-05	\\
-7876.25843394886	2.56549577730196e-05	\\
-7875.27965198864	2.69098349866581e-05	\\
-7874.30087002841	2.50772995949505e-05	\\
-7873.32208806818	2.6501920690168e-05	\\
-7872.34330610795	2.66843741881798e-05	\\
-7871.36452414773	2.80381786656713e-05	\\
-7870.3857421875	2.62307488573123e-05	\\
-7869.40696022727	2.61552195249998e-05	\\
-7868.42817826705	2.67185333931796e-05	\\
-7867.44939630682	2.63008079488927e-05	\\
-7866.47061434659	2.48951412108487e-05	\\
-7865.49183238636	2.7539927509212e-05	\\
-7864.51305042614	2.69814195553021e-05	\\
-7863.53426846591	2.65240654870828e-05	\\
-7862.55548650568	2.77223873245105e-05	\\
-7861.57670454545	2.66325634228488e-05	\\
-7860.59792258523	2.70853147620411e-05	\\
-7859.619140625	2.63796202755583e-05	\\
-7858.64035866477	2.76064502529444e-05	\\
-7857.66157670455	2.66755492959967e-05	\\
-7856.68279474432	2.59354880154645e-05	\\
-7855.70401278409	2.63986969407533e-05	\\
-7854.72523082386	2.6657283497042e-05	\\
-7853.74644886364	2.52656344611555e-05	\\
-7852.76766690341	2.6338649183106e-05	\\
-7851.78888494318	2.59301836995974e-05	\\
-7850.81010298295	2.62484359272163e-05	\\
-7849.83132102273	2.65775087659018e-05	\\
-7848.8525390625	2.67269584422675e-05	\\
-7847.87375710227	2.50063067688429e-05	\\
-7846.89497514205	2.58477455122455e-05	\\
-7845.91619318182	2.45484259024974e-05	\\
-7844.93741122159	2.54609578161958e-05	\\
-7843.95862926136	2.49121272712503e-05	\\
-7842.97984730114	2.39393960969016e-05	\\
-7842.00106534091	2.37070469822112e-05	\\
-7841.02228338068	2.81406756340795e-05	\\
-7840.04350142045	2.59915402825123e-05	\\
-7839.06471946023	2.75543722794988e-05	\\
-7838.0859375	2.62793654906947e-05	\\
-7837.10715553977	2.41734826032709e-05	\\
-7836.12837357955	2.50164036113439e-05	\\
-7835.14959161932	2.59742212325034e-05	\\
-7834.17080965909	2.72322777001692e-05	\\
-7833.19202769886	2.52062289385081e-05	\\
-7832.21324573864	2.63514601485968e-05	\\
-7831.23446377841	2.62970962363667e-05	\\
-7830.25568181818	2.47241120605887e-05	\\
-7829.27689985795	2.65650755246513e-05	\\
-7828.29811789773	2.64795849533817e-05	\\
-7827.3193359375	2.58683330147208e-05	\\
-7826.34055397727	2.55514169252363e-05	\\
-7825.36177201705	2.59629208725761e-05	\\
-7824.38299005682	2.59753753905585e-05	\\
-7823.40420809659	2.58501489902426e-05	\\
-7822.42542613636	2.58101244700207e-05	\\
-7821.44664417614	2.41208363777765e-05	\\
-7820.46786221591	2.47065932140872e-05	\\
-7819.48908025568	2.45145912522723e-05	\\
-7818.51029829545	2.61455371309537e-05	\\
-7817.53151633523	2.61721192446078e-05	\\
-7816.552734375	2.40880887916237e-05	\\
-7815.57395241477	2.60267245444216e-05	\\
-7814.59517045455	2.67608193984302e-05	\\
-7813.61638849432	2.64525862946495e-05	\\
-7812.63760653409	2.63015213117686e-05	\\
-7811.65882457386	2.52513266532106e-05	\\
-7810.68004261364	2.529227165881e-05	\\
-7809.70126065341	2.53813278070078e-05	\\
-7808.72247869318	2.5618151656693e-05	\\
-7807.74369673295	2.66415829737234e-05	\\
-7806.76491477273	2.50394691849714e-05	\\
-7805.7861328125	2.56846771739683e-05	\\
-7804.80735085227	2.61541588657202e-05	\\
-7803.82856889205	2.61612628108643e-05	\\
-7802.84978693182	2.64230320617799e-05	\\
-7801.87100497159	2.68033654875411e-05	\\
-7800.89222301136	2.49419208539706e-05	\\
-7799.91344105114	2.39113244961038e-05	\\
-7798.93465909091	2.6077222079106e-05	\\
-7797.95587713068	2.57038297807346e-05	\\
-7796.97709517045	2.51560558749044e-05	\\
-7795.99831321023	2.54770916948606e-05	\\
-7795.01953125	2.39426231161846e-05	\\
-7794.04074928977	2.48545911927687e-05	\\
-7793.06196732955	2.39360158589805e-05	\\
-7792.08318536932	2.38241701415316e-05	\\
-7791.10440340909	2.2857036915996e-05	\\
-7790.12562144886	2.46376462134377e-05	\\
-7789.14683948864	2.46033180022398e-05	\\
-7788.16805752841	2.28966846711989e-05	\\
-7787.18927556818	2.57996965348614e-05	\\
-7786.21049360795	2.37922766691798e-05	\\
-7785.23171164773	2.50262642723663e-05	\\
-7784.2529296875	2.41106043025948e-05	\\
-7783.27414772727	2.28083063719339e-05	\\
-7782.29536576705	2.55129264579482e-05	\\
-7781.31658380682	2.4767606780455e-05	\\
-7780.33780184659	2.49710492089385e-05	\\
-7779.35901988636	2.37761781903196e-05	\\
-7778.38023792614	2.47971106161628e-05	\\
-7777.40145596591	2.49036492168442e-05	\\
-7776.42267400568	2.36802581876426e-05	\\
-7775.44389204545	2.42012006100698e-05	\\
-7774.46511008523	2.45785159252248e-05	\\
-7773.486328125	2.3287233764093e-05	\\
-7772.50754616477	2.44632353412941e-05	\\
-7771.52876420455	2.28765194399486e-05	\\
-7770.54998224432	2.39830162988939e-05	\\
-7769.57120028409	2.26514444913927e-05	\\
-7768.59241832386	2.33171569434914e-05	\\
-7767.61363636364	2.51330382390107e-05	\\
-7766.63485440341	2.31035540071379e-05	\\
-7765.65607244318	2.39377708062252e-05	\\
-7764.67729048295	2.3584266120465e-05	\\
-7763.69850852273	2.36622675291696e-05	\\
-7762.7197265625	2.36261940364655e-05	\\
-7761.74094460227	2.35203428122973e-05	\\
-7760.76216264205	2.18676629223208e-05	\\
-7759.78338068182	2.34888828154285e-05	\\
-7758.80459872159	2.22716553705929e-05	\\
-7757.82581676136	2.18190932373553e-05	\\
-7756.84703480114	2.26450845502451e-05	\\
-7755.86825284091	2.27703446475941e-05	\\
-7754.88947088068	2.21377862485792e-05	\\
-7753.91068892045	2.24519202608654e-05	\\
-7752.93190696023	2.16987010645205e-05	\\
-7751.953125	2.10974868612582e-05	\\
-7750.97434303977	2.23389778901472e-05	\\
-7749.99556107955	2.25552897225474e-05	\\
-7749.01677911932	2.01221281598096e-05	\\
-7748.03799715909	2.21337578964336e-05	\\
-7747.05921519886	2.22963594272022e-05	\\
-7746.08043323864	2.18539811217152e-05	\\
-7745.10165127841	2.19970034042848e-05	\\
-7744.12286931818	2.15014503922505e-05	\\
-7743.14408735795	2.10601922015507e-05	\\
-7742.16530539773	2.21772813990446e-05	\\
-7741.1865234375	2.28156990387419e-05	\\
-7740.20774147727	2.27485643323505e-05	\\
-7739.22895951705	2.34608064443921e-05	\\
-7738.25017755682	2.25460662146492e-05	\\
-7737.27139559659	2.14427162411193e-05	\\
-7736.29261363636	2.11195027473133e-05	\\
-7735.31383167614	2.18827959098212e-05	\\
-7734.33504971591	2.1466893651757e-05	\\
-7733.35626775568	2.18501063125822e-05	\\
-7732.37748579545	2.11291784376357e-05	\\
-7731.39870383523	2.03111646418106e-05	\\
-7730.419921875	2.10903053349111e-05	\\
-7729.44113991477	2.20508385410041e-05	\\
-7728.46235795455	2.18773933487603e-05	\\
-7727.48357599432	2.22166990208782e-05	\\
-7726.50479403409	2.20382950303623e-05	\\
-7725.52601207386	2.07724527701714e-05	\\
-7724.54723011364	2.2144431291271e-05	\\
-7723.56844815341	2.20043455752241e-05	\\
-7722.58966619318	2.19149540166045e-05	\\
-7721.61088423295	2.27982175703092e-05	\\
-7720.63210227273	2.27737310730897e-05	\\
-7719.6533203125	2.13867079518602e-05	\\
-7718.67453835227	2.15922371932265e-05	\\
-7717.69575639205	2.36139798031779e-05	\\
-7716.71697443182	2.27739901689093e-05	\\
-7715.73819247159	2.29091807212025e-05	\\
-7714.75941051136	2.35252416821819e-05	\\
-7713.78062855114	2.22018417830876e-05	\\
-7712.80184659091	2.33327334823744e-05	\\
-7711.82306463068	2.39162663492423e-05	\\
-7710.84428267045	2.26447819916791e-05	\\
-7709.86550071023	2.22817423707769e-05	\\
-7708.88671875	2.43623563085431e-05	\\
-7707.90793678977	2.30519222048402e-05	\\
-7706.92915482955	2.33774473231597e-05	\\
-7705.95037286932	2.28262351067961e-05	\\
-7704.97159090909	2.43107618259022e-05	\\
-7703.99280894886	2.34087728132643e-05	\\
-7703.01402698864	2.26181097817845e-05	\\
-7702.03524502841	2.15066103268865e-05	\\
-7701.05646306818	2.39932413562375e-05	\\
-7700.07768110795	2.41873690619958e-05	\\
-7699.09889914773	2.25845290425303e-05	\\
-7698.1201171875	2.27045352568942e-05	\\
-7697.14133522727	2.35070405200153e-05	\\
-7696.16255326705	2.45845946839738e-05	\\
-7695.18377130682	2.30751851546075e-05	\\
-7694.20498934659	2.37632411209035e-05	\\
-7693.22620738636	2.33935111757036e-05	\\
-7692.24742542614	2.43716028998321e-05	\\
-7691.26864346591	2.35385233472885e-05	\\
-7690.28986150568	2.34081016811602e-05	\\
-7689.31107954545	2.53696507521551e-05	\\
-7688.33229758523	2.52001563968108e-05	\\
-7687.353515625	2.48219284634378e-05	\\
-7686.37473366477	2.54554367553432e-05	\\
-7685.39595170455	2.38902755786296e-05	\\
-7684.41716974432	2.3547696229491e-05	\\
-7683.43838778409	2.56623959202776e-05	\\
-7682.45960582386	2.56006939021758e-05	\\
-7681.48082386364	2.67708479601962e-05	\\
-7680.50204190341	2.43383250164093e-05	\\
-7679.52325994318	2.43000454565057e-05	\\
-7678.54447798295	2.53624237080673e-05	\\
-7677.56569602273	2.42202154236912e-05	\\
-7676.5869140625	2.44741501387521e-05	\\
-7675.60813210227	2.57848960571491e-05	\\
-7674.62935014205	2.54461613743723e-05	\\
-7673.65056818182	2.46902710611055e-05	\\
-7672.67178622159	2.41681812613186e-05	\\
-7671.69300426136	2.56686399103108e-05	\\
-7670.71422230114	2.59016878631886e-05	\\
-7669.73544034091	2.62327511527337e-05	\\
-7668.75665838068	2.50254458444438e-05	\\
-7667.77787642045	2.39730058188563e-05	\\
-7666.79909446023	2.46443476362029e-05	\\
-7665.8203125	2.4588971997558e-05	\\
-7664.84153053977	2.48573502673532e-05	\\
-7663.86274857955	2.5736508322107e-05	\\
-7662.88396661932	2.49102474824773e-05	\\
-7661.90518465909	2.4296388009127e-05	\\
-7660.92640269886	2.4881095643524e-05	\\
-7659.94762073864	2.50978434482708e-05	\\
-7658.96883877841	2.48033477087905e-05	\\
-7657.99005681818	2.42993702171783e-05	\\
-7657.01127485795	2.39752221912055e-05	\\
-7656.03249289773	2.49736327872381e-05	\\
-7655.0537109375	2.56236129219277e-05	\\
-7654.07492897727	2.56725460394709e-05	\\
-7653.09614701705	2.62797383405824e-05	\\
-7652.11736505682	2.61169698903726e-05	\\
-7651.13858309659	2.58135844372403e-05	\\
-7650.15980113636	2.4729529919025e-05	\\
-7649.18101917614	2.5297673511635e-05	\\
-7648.20223721591	2.47935268920139e-05	\\
-7647.22345525568	2.49852797547465e-05	\\
-7646.24467329545	2.57800781200075e-05	\\
-7645.26589133523	2.61822379464455e-05	\\
-7644.287109375	2.69047501143153e-05	\\
-7643.30832741477	2.62008392548853e-05	\\
-7642.32954545455	2.46159988999712e-05	\\
-7641.35076349432	2.57303235057189e-05	\\
-7640.37198153409	2.71656502617055e-05	\\
-7639.39319957386	2.64956793065216e-05	\\
-7638.41441761364	2.60962067312661e-05	\\
-7637.43563565341	2.6291181233934e-05	\\
-7636.45685369318	2.82939722989928e-05	\\
-7635.47807173295	2.78978434756992e-05	\\
-7634.49928977273	2.64820702998365e-05	\\
-7633.5205078125	2.73935197888968e-05	\\
-7632.54172585227	2.74675289290309e-05	\\
-7631.56294389205	2.62057084022157e-05	\\
-7630.58416193182	2.85277787070641e-05	\\
-7629.60537997159	2.73414516223547e-05	\\
-7628.62659801136	2.62285063729066e-05	\\
-7627.64781605114	2.75809964178506e-05	\\
-7626.66903409091	2.93189596097681e-05	\\
-7625.69025213068	2.69337812189181e-05	\\
-7624.71147017045	2.81762959084739e-05	\\
-7623.73268821023	2.74064445138699e-05	\\
-7622.75390625	2.7873846505741e-05	\\
-7621.77512428977	2.99036776030863e-05	\\
-7620.79634232955	2.99582582162792e-05	\\
-7619.81756036932	2.66806475297781e-05	\\
-7618.83877840909	2.93453352598471e-05	\\
-7617.85999644886	3.09182918815768e-05	\\
-7616.88121448864	2.87220064126334e-05	\\
-7615.90243252841	2.9931180644291e-05	\\
-7614.92365056818	3.11215678998297e-05	\\
-7613.94486860795	2.78468799630309e-05	\\
-7612.96608664773	2.95471826738534e-05	\\
-7611.9873046875	2.94098805852693e-05	\\
-7611.00852272727	2.8843342011098e-05	\\
-7610.02974076705	3.11594135292944e-05	\\
-7609.05095880682	3.03700043043171e-05	\\
-7608.07217684659	3.00698168691889e-05	\\
-7607.09339488636	3.05273280398631e-05	\\
-7606.11461292614	3.0097123629469e-05	\\
-7605.13583096591	3.07862299630792e-05	\\
-7604.15704900568	3.00593036708151e-05	\\
-7603.17826704545	3.08136597981848e-05	\\
-7602.19948508523	2.96629406181561e-05	\\
-7601.220703125	3.14917994608216e-05	\\
-7600.24192116477	3.29196521857109e-05	\\
-7599.26313920455	3.06557633313845e-05	\\
-7598.28435724432	3.20819527443726e-05	\\
-7597.30557528409	3.29925940749861e-05	\\
-7596.32679332386	3.08052560382148e-05	\\
-7595.34801136364	3.11271834837871e-05	\\
-7594.36922940341	3.29882845575937e-05	\\
-7593.39044744318	3.29716462619019e-05	\\
-7592.41166548295	3.30445942710878e-05	\\
-7591.43288352273	3.24937846442209e-05	\\
-7590.4541015625	3.2477269931372e-05	\\
-7589.47531960227	3.16200716288188e-05	\\
-7588.49653764205	3.25402901619406e-05	\\
-7587.51775568182	3.18733349731796e-05	\\
-7586.53897372159	3.37718268927027e-05	\\
-7585.56019176136	3.32357111927436e-05	\\
-7584.58140980114	3.22558086054593e-05	\\
-7583.60262784091	3.45064736433845e-05	\\
-7582.62384588068	3.47404767047597e-05	\\
-7581.64506392045	3.45313296776668e-05	\\
-7580.66628196023	3.35222788300895e-05	\\
-7579.6875	3.35333487782676e-05	\\
-7578.70871803977	3.38631108151069e-05	\\
-7577.72993607955	3.404114358737e-05	\\
-7576.75115411932	3.43741364058961e-05	\\
-7575.77237215909	3.39579856775179e-05	\\
-7574.79359019886	3.39544798820162e-05	\\
-7573.81480823864	3.30783422594636e-05	\\
-7572.83602627841	3.24769847574697e-05	\\
-7571.85724431818	3.30862456848652e-05	\\
-7570.87846235795	3.41729586894245e-05	\\
-7569.89968039773	3.22041271758989e-05	\\
-7568.9208984375	3.29528453417229e-05	\\
-7567.94211647727	3.44446361166074e-05	\\
-7566.96333451705	3.406773669841e-05	\\
-7565.98455255682	3.35149446515312e-05	\\
-7565.00577059659	3.40215940397924e-05	\\
-7564.02698863636	3.30342752230517e-05	\\
-7563.04820667614	3.15532123599282e-05	\\
-7562.06942471591	3.17402724337981e-05	\\
-7561.09064275568	3.42671941610629e-05	\\
-7560.11186079545	3.30918301723099e-05	\\
-7559.13307883523	3.16035599532905e-05	\\
-7558.154296875	3.41404680324138e-05	\\
-7557.17551491477	3.29248911123942e-05	\\
-7556.19673295455	3.30715011454927e-05	\\
-7555.21795099432	3.20198479323174e-05	\\
-7554.23916903409	3.27190881389211e-05	\\
-7553.26038707386	3.34652184558285e-05	\\
-7552.28160511364	3.39321061748175e-05	\\
-7551.30282315341	3.3171581755623e-05	\\
-7550.32404119318	3.42051650687428e-05	\\
-7549.34525923295	3.17935135297166e-05	\\
-7548.36647727273	3.21160347529568e-05	\\
-7547.3876953125	3.38812736426225e-05	\\
-7546.40891335227	3.32576454130051e-05	\\
-7545.43013139205	3.42247397563934e-05	\\
-7544.45134943182	3.32335224306737e-05	\\
-7543.47256747159	3.48549714510561e-05	\\
-7542.49378551136	3.40350841905137e-05	\\
-7541.51500355114	3.44991207390564e-05	\\
-7540.53622159091	3.3948657202388e-05	\\
-7539.55743963068	3.35907960120868e-05	\\
-7538.57865767045	3.39537508864839e-05	\\
-7537.59987571023	3.3987346387115e-05	\\
-7536.62109375	3.29205004953883e-05	\\
-7535.64231178977	3.417620264224e-05	\\
-7534.66352982955	3.2905781531791e-05	\\
-7533.68474786932	3.38108460941657e-05	\\
-7532.70596590909	3.56573715020021e-05	\\
-7531.72718394886	3.47239177755292e-05	\\
-7530.74840198864	3.41499480024821e-05	\\
-7529.76962002841	3.36438848054842e-05	\\
-7528.79083806818	3.4710969591557e-05	\\
-7527.81205610795	3.52655244429828e-05	\\
-7526.83327414773	3.39852115389937e-05	\\
-7525.8544921875	3.60380044689649e-05	\\
-7524.87571022727	3.5087988751043e-05	\\
-7523.89692826705	3.53418573832524e-05	\\
-7522.91814630682	3.59113665970188e-05	\\
-7521.93936434659	3.56943225900743e-05	\\
-7520.96058238636	3.40434203594417e-05	\\
-7519.98180042614	3.58080187971577e-05	\\
-7519.00301846591	3.70877910431902e-05	\\
-7518.02423650568	3.57025724189511e-05	\\
-7517.04545454545	3.65074444568385e-05	\\
-7516.06667258523	3.61045359840885e-05	\\
-7515.087890625	3.47163990915678e-05	\\
-7514.10910866477	3.62188698160603e-05	\\
-7513.13032670455	3.65879641281786e-05	\\
-7512.15154474432	3.59634056166011e-05	\\
-7511.17276278409	3.56233907486176e-05	\\
-7510.19398082386	3.79051763385919e-05	\\
-7509.21519886364	3.66095551931409e-05	\\
-7508.23641690341	3.65532665039516e-05	\\
-7507.25763494318	3.53553844848923e-05	\\
-7506.27885298295	3.74555471103642e-05	\\
-7505.30007102273	3.75751541264571e-05	\\
-7504.3212890625	3.70344759891272e-05	\\
-7503.34250710227	3.83317936970257e-05	\\
-7502.36372514205	3.74530410342682e-05	\\
-7501.38494318182	3.80959447135489e-05	\\
-7500.40616122159	3.88519328524149e-05	\\
-7499.42737926136	3.99691858237709e-05	\\
-7498.44859730114	3.93962466751657e-05	\\
-7497.46981534091	3.79221140849423e-05	\\
-7496.49103338068	3.74135015372419e-05	\\
-7495.51225142045	3.9506792415359e-05	\\
-7494.53346946023	4.00628459811042e-05	\\
-7493.5546875	3.87160909010492e-05	\\
-7492.57590553977	4.01659744034216e-05	\\
-7491.59712357955	3.76098699019476e-05	\\
-7490.61834161932	3.90470648140354e-05	\\
-7489.63955965909	4.18854934067069e-05	\\
-7488.66077769886	4.0089302278055e-05	\\
-7487.68199573864	4.06470161508542e-05	\\
-7486.70321377841	4.03139240490741e-05	\\
-7485.72443181818	4.06406959014064e-05	\\
-7484.74564985795	4.05874767621024e-05	\\
-7483.76686789773	4.11919931528052e-05	\\
-7482.7880859375	4.13748853261989e-05	\\
-7481.80930397727	4.12179059873949e-05	\\
-7480.83052201705	4.0769010371521e-05	\\
-7479.85174005682	4.07930870763729e-05	\\
-7478.87295809659	3.88169675580988e-05	\\
-7477.89417613636	4.00523165298444e-05	\\
-7476.91539417614	4.05010421822428e-05	\\
-7475.93661221591	3.8986523842354e-05	\\
-7474.95783025568	3.99385799478643e-05	\\
-7473.97904829545	4.00372641399358e-05	\\
-7473.00026633523	4.11020898245364e-05	\\
-7472.021484375	3.8553398209478e-05	\\
-7471.04270241477	4.11845486202494e-05	\\
-7470.06392045455	3.9535582798001e-05	\\
-7469.08513849432	4.1322528923473e-05	\\
-7468.10635653409	4.08608833311758e-05	\\
-7467.12757457386	3.95347543781955e-05	\\
-7466.14879261364	4.13329698083121e-05	\\
-7465.17001065341	4.09032500253457e-05	\\
-7464.19122869318	4.1756351836566e-05	\\
-7463.21244673295	4.25425759408604e-05	\\
-7462.23366477273	4.10489907720647e-05	\\
-7461.2548828125	3.9927529842476e-05	\\
-7460.27610085227	3.83473984630599e-05	\\
-7459.29731889205	3.94021582182297e-05	\\
-7458.31853693182	4.10797132915816e-05	\\
-7457.33975497159	4.02099769008811e-05	\\
-7456.36097301136	4.06206148604431e-05	\\
-7455.38219105114	4.08821752985309e-05	\\
-7454.40340909091	4.11758155327609e-05	\\
-7453.42462713068	4.10790462568107e-05	\\
-7452.44584517045	3.91367632412194e-05	\\
-7451.46706321023	4.02418117270393e-05	\\
-7450.48828125	4.14810032224247e-05	\\
-7449.50949928977	3.90042712086052e-05	\\
-7448.53071732955	4.10698595974231e-05	\\
-7447.55193536932	4.12547898601894e-05	\\
-7446.57315340909	4.06205045042765e-05	\\
-7445.59437144886	4.09408117378964e-05	\\
-7444.61558948864	4.0165550580428e-05	\\
-7443.63680752841	3.99927892164695e-05	\\
-7442.65802556818	4.04925644047796e-05	\\
-7441.67924360795	3.90046111213097e-05	\\
-7440.70046164773	3.98542809775814e-05	\\
-7439.7216796875	3.98967899481378e-05	\\
-7438.74289772727	4.10785595339894e-05	\\
-7437.76411576705	4.06928739719263e-05	\\
-7436.78533380682	4.03696827801323e-05	\\
-7435.80655184659	4.14455795475683e-05	\\
-7434.82776988636	4.0528267710171e-05	\\
-7433.84898792614	4.15348511734385e-05	\\
-7432.87020596591	4.14212806103689e-05	\\
-7431.89142400568	3.99300016464547e-05	\\
-7430.91264204545	3.99228192833459e-05	\\
-7429.93386008523	3.91498989192515e-05	\\
-7428.955078125	4.05095008377205e-05	\\
-7427.97629616477	4.08679105865647e-05	\\
-7426.99751420455	4.08666461514263e-05	\\
-7426.01873224432	4.15626591486912e-05	\\
-7425.03995028409	3.96368062715773e-05	\\
-7424.06116832386	4.1386123499272e-05	\\
-7423.08238636364	4.09190634204867e-05	\\
-7422.10360440341	4.11703280837559e-05	\\
-7421.12482244318	3.88464357475526e-05	\\
-7420.14604048295	4.05949110122497e-05	\\
-7419.16725852273	4.2487402015561e-05	\\
-7418.1884765625	4.18268098931872e-05	\\
-7417.20969460227	4.24137014505593e-05	\\
-7416.23091264205	4.07012692014869e-05	\\
-7415.25213068182	4.10106515103266e-05	\\
-7414.27334872159	4.06627952786345e-05	\\
-7413.29456676136	4.06602661625558e-05	\\
-7412.31578480114	4.09211571231465e-05	\\
-7411.33700284091	4.28023147324623e-05	\\
-7410.35822088068	4.14693295226285e-05	\\
-7409.37943892045	4.10897645684157e-05	\\
-7408.40065696023	4.24623225292615e-05	\\
-7407.421875	4.14184223079102e-05	\\
-7406.44309303977	4.26241753518906e-05	\\
-7405.46431107955	4.18629180052119e-05	\\
-7404.48552911932	4.323187299927e-05	\\
-7403.50674715909	4.2969973600121e-05	\\
-7402.52796519886	4.25637938924228e-05	\\
-7401.54918323864	4.16378391816814e-05	\\
-7400.57040127841	4.18383846604712e-05	\\
-7399.59161931818	4.3585870629671e-05	\\
-7398.61283735795	4.36427332139401e-05	\\
-7397.63405539773	4.28961355938758e-05	\\
-7396.6552734375	4.21180202082902e-05	\\
-7395.67649147727	4.20977205839291e-05	\\
-7394.69770951705	4.28167414339672e-05	\\
-7393.71892755682	4.26559147220825e-05	\\
-7392.74014559659	4.26016250105816e-05	\\
-7391.76136363636	4.40458356482106e-05	\\
-7390.78258167614	4.35254232789245e-05	\\
-7389.80379971591	4.32975797360377e-05	\\
-7388.82501775568	4.19713120434854e-05	\\
-7387.84623579545	4.49116458997694e-05	\\
-7386.86745383523	4.44375736648905e-05	\\
-7385.888671875	4.26485845378053e-05	\\
-7384.90988991477	4.43188913692706e-05	\\
-7383.93110795455	4.33079877832428e-05	\\
-7382.95232599432	4.30831659273398e-05	\\
-7381.97354403409	4.48565558047589e-05	\\
-7380.99476207386	4.53446029220775e-05	\\
-7380.01598011364	4.30886105987051e-05	\\
-7379.03719815341	4.44607435016759e-05	\\
-7378.05841619318	4.40404427353197e-05	\\
-7377.07963423295	4.49789169061198e-05	\\
-7376.10085227273	4.68104976165932e-05	\\
-7375.1220703125	4.52704862443782e-05	\\
-7374.14328835227	4.58122610240969e-05	\\
-7373.16450639205	4.4776064598345e-05	\\
-7372.18572443182	4.49256068684405e-05	\\
-7371.20694247159	4.36034780003209e-05	\\
-7370.22816051136	4.56732171557231e-05	\\
-7369.24937855114	4.60171312337893e-05	\\
-7368.27059659091	4.56813169585762e-05	\\
-7367.29181463068	4.61593543861118e-05	\\
-7366.31303267045	4.63198129852471e-05	\\
-7365.33425071023	4.54369881234081e-05	\\
-7364.35546875	4.59999455936889e-05	\\
-7363.37668678977	4.73169123606242e-05	\\
-7362.39790482955	4.66663070113036e-05	\\
-7361.41912286932	4.59509681388481e-05	\\
-7360.44034090909	4.71543774933524e-05	\\
-7359.46155894886	4.63564343924773e-05	\\
-7358.48277698864	4.43106686629194e-05	\\
-7357.50399502841	4.76982030767481e-05	\\
-7356.52521306818	4.83948751045211e-05	\\
-7355.54643110795	4.66227551188275e-05	\\
-7354.56764914773	4.522597450552e-05	\\
-7353.5888671875	5.03153258937737e-05	\\
-7352.61008522727	4.69829194062803e-05	\\
-7351.63130326705	4.82025962277434e-05	\\
-7350.65252130682	4.67080847055009e-05	\\
-7349.67373934659	4.65028500347447e-05	\\
-7348.69495738636	4.83823816010261e-05	\\
-7347.71617542614	4.69047611854478e-05	\\
-7346.73739346591	4.91600898650381e-05	\\
-7345.75861150568	4.77396827828751e-05	\\
-7344.77982954545	4.80151315843898e-05	\\
-7343.80104758523	4.67430306126684e-05	\\
-7342.822265625	4.57270163514252e-05	\\
-7341.84348366477	4.92446084474139e-05	\\
-7340.86470170455	4.80649897812432e-05	\\
-7339.88591974432	4.94821433800589e-05	\\
-7338.90713778409	4.75155988962342e-05	\\
-7337.92835582386	4.69463759164207e-05	\\
-7336.94957386364	4.82358656562555e-05	\\
-7335.97079190341	4.74352528335253e-05	\\
-7334.99200994318	4.62156241887825e-05	\\
-7334.01322798295	4.84052571013605e-05	\\
-7333.03444602273	4.90737090803554e-05	\\
-7332.0556640625	4.86502603643063e-05	\\
-7331.07688210227	4.65387377961048e-05	\\
-7330.09810014205	4.9088024436894e-05	\\
-7329.11931818182	4.73264058261953e-05	\\
-7328.14053622159	4.79683545477075e-05	\\
-7327.16175426136	4.90643969078182e-05	\\
-7326.18297230114	4.90003666130489e-05	\\
-7325.20419034091	4.7395113469941e-05	\\
-7324.22540838068	4.83197502526925e-05	\\
-7323.24662642045	4.76507518361777e-05	\\
-7322.26784446023	4.86964166698076e-05	\\
-7321.2890625	4.92720324617532e-05	\\
-7320.31028053977	5.03212196446345e-05	\\
-7319.33149857955	5.00384204427952e-05	\\
-7318.35271661932	4.97714376090646e-05	\\
-7317.37393465909	5.01035007050136e-05	\\
-7316.39515269886	4.79116253840607e-05	\\
-7315.41637073864	4.96840578217012e-05	\\
-7314.43758877841	4.9667880022e-05	\\
-7313.45880681818	4.90504469025224e-05	\\
-7312.48002485795	4.93427021735674e-05	\\
-7311.50124289773	4.9234655169276e-05	\\
-7310.5224609375	4.94363495763803e-05	\\
-7309.54367897727	4.8916659020368e-05	\\
-7308.56489701705	4.89684791765616e-05	\\
-7307.58611505682	4.88396253451729e-05	\\
-7306.60733309659	5.06030229490557e-05	\\
-7305.62855113636	4.9492578088391e-05	\\
-7304.64976917614	5.11165706104373e-05	\\
-7303.67098721591	5.05535165040263e-05	\\
-7302.69220525568	5.12000015204454e-05	\\
-7301.71342329545	5.0636719797977e-05	\\
-7300.73464133523	5.18553126562554e-05	\\
-7299.755859375	5.34533920225443e-05	\\
-7298.77707741477	5.14595133983993e-05	\\
-7297.79829545455	5.06267755514765e-05	\\
-7296.81951349432	5.07077135414671e-05	\\
-7295.84073153409	5.15245910849414e-05	\\
-7294.86194957386	5.07595941931658e-05	\\
-7293.88316761364	4.95226243548403e-05	\\
-7292.90438565341	5.16920173850422e-05	\\
-7291.92560369318	5.04963551813254e-05	\\
-7290.94682173295	5.12957789623352e-05	\\
-7289.96803977273	5.19993700314751e-05	\\
-7288.9892578125	5.28321645271882e-05	\\
-7288.01047585227	5.3107680281011e-05	\\
-7287.03169389205	5.26500150151687e-05	\\
-7286.05291193182	5.08499764789371e-05	\\
-7285.07412997159	5.27456390189294e-05	\\
-7284.09534801136	5.20131832006261e-05	\\
-7283.11656605114	5.24370779480023e-05	\\
-7282.13778409091	5.16444502728397e-05	\\
-7281.15900213068	5.11653980845977e-05	\\
-7280.18022017045	5.11710280698371e-05	\\
-7279.20143821023	5.31576713180841e-05	\\
-7278.22265625	5.29254238392642e-05	\\
-7277.24387428977	5.27328881989137e-05	\\
-7276.26509232955	5.22264388930807e-05	\\
-7275.28631036932	5.20949865038514e-05	\\
-7274.30752840909	5.15111146072457e-05	\\
-7273.32874644886	5.36140185066427e-05	\\
-7272.34996448864	5.4649017045884e-05	\\
-7271.37118252841	5.29281807598447e-05	\\
-7270.39240056818	5.28293623465779e-05	\\
-7269.41361860795	5.47838273621169e-05	\\
-7268.43483664773	5.24953908451885e-05	\\
-7267.4560546875	5.52040842745389e-05	\\
-7266.47727272727	5.36792966839288e-05	\\
-7265.49849076705	5.22840260019979e-05	\\
-7264.51970880682	5.41511256218913e-05	\\
-7263.54092684659	5.41182051915903e-05	\\
-7262.56214488636	5.20391294306333e-05	\\
-7261.58336292614	5.35260300623073e-05	\\
-7260.60458096591	5.35633057694594e-05	\\
-7259.62579900568	5.22367645355119e-05	\\
-7258.64701704545	5.39505761823093e-05	\\
-7257.66823508523	5.4133677566958e-05	\\
-7256.689453125	5.25800443679015e-05	\\
-7255.71067116477	5.2848242001825e-05	\\
-7254.73188920455	5.45209209590676e-05	\\
-7253.75310724432	5.42732159104192e-05	\\
-7252.77432528409	5.24785436458938e-05	\\
-7251.79554332386	5.39863821494822e-05	\\
-7250.81676136364	5.30220117592699e-05	\\
-7249.83797940341	5.80861351587496e-05	\\
-7248.85919744318	5.48165119041606e-05	\\
-7247.88041548295	5.1495938884866e-05	\\
-7246.90163352273	5.48161100159306e-05	\\
-7245.9228515625	5.23894469170547e-05	\\
-7244.94406960227	4.98369492567045e-05	\\
-7243.96528764205	5.39484506121869e-05	\\
-7242.98650568182	5.2243322474104e-05	\\
-7242.00772372159	5.32259484683037e-05	\\
-7241.02894176136	5.22638494937848e-05	\\
-7240.05015980114	5.29315045598462e-05	\\
-7239.07137784091	5.32424207533571e-05	\\
-7238.09259588068	5.40935535515678e-05	\\
-7237.11381392045	5.41978674611463e-05	\\
-7236.13503196023	5.27225484300682e-05	\\
-7235.15625	5.35698077301759e-05	\\
-7234.17746803977	5.33925046023305e-05	\\
-7233.19868607955	5.37128741693207e-05	\\
-7232.21990411932	5.44230775113938e-05	\\
-7231.24112215909	5.28442135259616e-05	\\
-7230.26234019886	5.30428616073071e-05	\\
-7229.28355823864	5.45761801638688e-05	\\
-7228.30477627841	5.40568699943253e-05	\\
-7227.32599431818	5.31968329129976e-05	\\
-7226.34721235795	5.41406197088078e-05	\\
-7225.36843039773	5.31421394151046e-05	\\
-7224.3896484375	5.22506816569222e-05	\\
-7223.41086647727	5.19288518293304e-05	\\
-7222.43208451705	5.2985341606782e-05	\\
-7221.45330255682	5.2466755951955e-05	\\
-7220.47452059659	5.2022553891515e-05	\\
-7219.49573863636	5.46587915844237e-05	\\
-7218.51695667614	5.28364668332895e-05	\\
-7217.53817471591	5.25355707418236e-05	\\
-7216.55939275568	5.33143351004426e-05	\\
-7215.58061079545	5.09733125164024e-05	\\
-7214.60182883523	5.26503424702355e-05	\\
-7213.623046875	5.26902282098739e-05	\\
-7212.64426491477	5.24869084262964e-05	\\
-7211.66548295455	5.02855069674403e-05	\\
-7210.68670099432	5.20682733909512e-05	\\
-7209.70791903409	5.00135607905717e-05	\\
-7208.72913707386	5.09858211030854e-05	\\
-7207.75035511364	5.10298606316354e-05	\\
-7206.77157315341	5.16798169347196e-05	\\
-7205.79279119318	5.20755641764498e-05	\\
-7204.81400923295	5.4235947477839e-05	\\
-7203.83522727273	5.11371562209561e-05	\\
-7202.8564453125	5.17343453682501e-05	\\
-7201.87766335227	5.24002752653479e-05	\\
-7200.89888139205	5.28007822861084e-05	\\
-7199.92009943182	5.11829810633415e-05	\\
-7198.94131747159	5.12732017835073e-05	\\
-7197.96253551136	5.16548713927735e-05	\\
-7196.98375355114	5.18342648645e-05	\\
-7196.00497159091	5.05101410433743e-05	\\
-7195.02618963068	5.00281762139664e-05	\\
-7194.04740767045	5.08524138766958e-05	\\
-7193.06862571023	5.1150452712349e-05	\\
-7192.08984375	5.15259420454453e-05	\\
-7191.11106178977	5.11427611682847e-05	\\
-7190.13227982955	5.18898496409426e-05	\\
-7189.15349786932	5.29840887502628e-05	\\
-7188.17471590909	5.1039672830596e-05	\\
-7187.19593394886	5.17908032857995e-05	\\
-7186.21715198864	5.18757646187874e-05	\\
-7185.23837002841	4.98292673159338e-05	\\
-7184.25958806818	5.06115977191539e-05	\\
-7183.28080610795	5.1350097143422e-05	\\
-7182.30202414773	5.21092836184942e-05	\\
-7181.3232421875	5.20086278374418e-05	\\
-7180.34446022727	5.14760906376447e-05	\\
-7179.36567826705	5.07365058897778e-05	\\
-7178.38689630682	5.15944428342296e-05	\\
-7177.40811434659	5.19888089716615e-05	\\
-7176.42933238636	5.06045653759113e-05	\\
-7175.45055042614	5.01852060903088e-05	\\
-7174.47176846591	5.17606667625884e-05	\\
-7173.49298650568	5.21420522182443e-05	\\
-7172.51420454545	5.04130012363281e-05	\\
-7171.53542258523	5.10754667647157e-05	\\
-7170.556640625	5.22373004199061e-05	\\
-7169.57785866477	5.17192600262202e-05	\\
-7168.59907670455	5.22874427345231e-05	\\
-7167.62029474432	5.08421799218827e-05	\\
-7166.64151278409	5.00642732293695e-05	\\
-7165.66273082386	5.22101246427715e-05	\\
-7164.68394886364	5.38481145588931e-05	\\
-7163.70516690341	5.06277256904533e-05	\\
-7162.72638494318	5.25251675402102e-05	\\
-7161.74760298295	5.08333215501067e-05	\\
-7160.76882102273	5.16498883144573e-05	\\
-7159.7900390625	5.20652318670814e-05	\\
-7158.81125710227	5.02524723006175e-05	\\
-7157.83247514205	5.14378068459133e-05	\\
-7156.85369318182	5.18533188674984e-05	\\
-7155.87491122159	5.15168484084138e-05	\\
-7154.89612926136	5.04345582028977e-05	\\
-7153.91734730114	5.17459475641268e-05	\\
-7152.93856534091	5.29177693483477e-05	\\
-7151.95978338068	5.08371474932421e-05	\\
-7150.98100142045	5.02106098598158e-05	\\
-7150.00221946023	5.07188532124191e-05	\\
-7149.0234375	5.08849386598348e-05	\\
-7148.04465553977	5.0934727132164e-05	\\
-7147.06587357955	5.07220167993386e-05	\\
-7146.08709161932	5.10359123453926e-05	\\
-7145.10830965909	5.12584305995587e-05	\\
-7144.12952769886	5.14602255910012e-05	\\
-7143.15074573864	4.92911406779112e-05	\\
-7142.17196377841	4.9181233617708e-05	\\
-7141.19318181818	5.11031033897184e-05	\\
-7140.21439985795	4.96251607940282e-05	\\
-7139.23561789773	5.02039199106817e-05	\\
-7138.2568359375	4.85816400877641e-05	\\
-7137.27805397727	5.10360924219013e-05	\\
-7136.29927201705	4.94703760243966e-05	\\
-7135.32049005682	4.93851490842493e-05	\\
-7134.34170809659	5.03135924766309e-05	\\
-7133.36292613636	5.19490814794924e-05	\\
-7132.38414417614	5.09198508871037e-05	\\
-7131.40536221591	5.10403009887699e-05	\\
-7130.42658025568	4.89705010528095e-05	\\
-7129.44779829545	4.90477449845582e-05	\\
-7128.46901633523	4.95264633800991e-05	\\
-7127.490234375	5.02173179042614e-05	\\
-7126.51145241477	5.09838837201445e-05	\\
-7125.53267045455	5.11455606354688e-05	\\
-7124.55388849432	4.98402210830697e-05	\\
-7123.57510653409	4.86877344788652e-05	\\
-7122.59632457386	4.8915371808462e-05	\\
-7121.61754261364	5.21960385814032e-05	\\
-7120.63876065341	4.87214938961726e-05	\\
-7119.65997869318	4.85467770317956e-05	\\
-7118.68119673295	4.80569463217148e-05	\\
-7117.70241477273	4.98195492183154e-05	\\
-7116.7236328125	4.99372004473201e-05	\\
-7115.74485085227	5.04674407210513e-05	\\
-7114.76606889205	4.83268402651287e-05	\\
-7113.78728693182	5.06222244625583e-05	\\
-7112.80850497159	5.0099124017603e-05	\\
-7111.82972301136	5.0376278453859e-05	\\
-7110.85094105114	4.96600952964797e-05	\\
-7109.87215909091	5.0471658864821e-05	\\
};
\addplot [color=blue,solid,forget plot]
  table[row sep=crcr]{
-7109.87215909091	5.0471658864821e-05	\\
-7108.89337713068	4.93069364782139e-05	\\
-7107.91459517045	5.00946202265651e-05	\\
-7106.93581321023	4.90458351251634e-05	\\
-7105.95703125	4.87089231005557e-05	\\
-7104.97824928977	5.10228279730826e-05	\\
-7103.99946732955	5.09092779109952e-05	\\
-7103.02068536932	4.92190319579997e-05	\\
-7102.04190340909	4.93835133331126e-05	\\
-7101.06312144886	4.94702492993626e-05	\\
-7100.08433948864	5.1871309264397e-05	\\
-7099.10555752841	5.0931959960386e-05	\\
-7098.12677556818	4.98338021136656e-05	\\
-7097.14799360795	4.88909262531996e-05	\\
-7096.16921164773	4.9276325336691e-05	\\
-7095.1904296875	5.04403567020773e-05	\\
-7094.21164772727	4.94983100431407e-05	\\
-7093.23286576705	4.98496464628285e-05	\\
-7092.25408380682	4.7613995072483e-05	\\
-7091.27530184659	4.9911882156916e-05	\\
-7090.29651988636	4.97547450828537e-05	\\
-7089.31773792614	4.91989687271483e-05	\\
-7088.33895596591	4.90801868170889e-05	\\
-7087.36017400568	5.21467851772949e-05	\\
-7086.38139204545	5.14286102788479e-05	\\
-7085.40261008523	5.07510727816474e-05	\\
-7084.423828125	5.18454790859805e-05	\\
-7083.44504616477	5.19189380467988e-05	\\
-7082.46626420455	4.9748224432681e-05	\\
-7081.48748224432	5.18780743475621e-05	\\
-7080.50870028409	5.06854367797028e-05	\\
-7079.52991832386	4.96241035388588e-05	\\
-7078.55113636364	5.05732623275892e-05	\\
-7077.57235440341	5.16216571772449e-05	\\
-7076.59357244318	5.12909438296626e-05	\\
-7075.61479048295	5.16339789861467e-05	\\
-7074.63600852273	5.04173115104926e-05	\\
-7073.6572265625	5.1823227601051e-05	\\
-7072.67844460227	5.19015626315205e-05	\\
-7071.69966264205	5.03185492919239e-05	\\
-7070.72088068182	5.2652357414015e-05	\\
-7069.74209872159	5.10901718096851e-05	\\
-7068.76331676136	5.10319160981777e-05	\\
-7067.78453480114	5.15172819849035e-05	\\
-7066.80575284091	5.13791824700696e-05	\\
-7065.82697088068	5.07546762689567e-05	\\
-7064.84818892045	5.12630456369113e-05	\\
-7063.86940696023	4.97596951192684e-05	\\
-7062.890625	5.12762928330876e-05	\\
-7061.91184303977	5.15905697144882e-05	\\
-7060.93306107955	5.21453155465276e-05	\\
-7059.95427911932	5.19188707003198e-05	\\
-7058.97549715909	4.97490049251133e-05	\\
-7057.99671519886	5.16081655206759e-05	\\
-7057.01793323864	5.17397370023653e-05	\\
-7056.03915127841	5.2451730973134e-05	\\
-7055.06036931818	5.06726868894451e-05	\\
-7054.08158735795	5.30874425163173e-05	\\
-7053.10280539773	4.9673926934855e-05	\\
-7052.1240234375	5.05034136573175e-05	\\
-7051.14524147727	5.09570100461389e-05	\\
-7050.16645951705	5.24868374796103e-05	\\
-7049.18767755682	5.12461589098138e-05	\\
-7048.20889559659	5.01608332422924e-05	\\
-7047.23011363636	4.95453998561517e-05	\\
-7046.25133167614	4.91288270871046e-05	\\
-7045.27254971591	4.91086367156414e-05	\\
-7044.29376775568	4.96472810582893e-05	\\
-7043.31498579545	5.04619880782707e-05	\\
-7042.33620383523	4.87492388128118e-05	\\
-7041.357421875	4.9863909741618e-05	\\
-7040.37863991477	4.72646251175172e-05	\\
-7039.39985795455	4.97994689856178e-05	\\
-7038.42107599432	4.98242967624048e-05	\\
-7037.44229403409	5.17172040159538e-05	\\
-7036.46351207386	4.92906166971319e-05	\\
-7035.48473011364	5.09091280129376e-05	\\
-7034.50594815341	4.77998925148656e-05	\\
-7033.52716619318	5.13805495887015e-05	\\
-7032.54838423295	4.90890155213967e-05	\\
-7031.56960227273	4.84104876524656e-05	\\
-7030.5908203125	4.99305532537688e-05	\\
-7029.61203835227	4.96508949476741e-05	\\
-7028.63325639205	4.86941428743663e-05	\\
-7027.65447443182	4.97145047073741e-05	\\
-7026.67569247159	5.05421653737066e-05	\\
-7025.69691051136	5.11207462370019e-05	\\
-7024.71812855114	5.10729381333851e-05	\\
-7023.73934659091	5.07030683745692e-05	\\
-7022.76056463068	4.99061738208626e-05	\\
-7021.78178267045	4.70515751911686e-05	\\
-7020.80300071023	5.05662257707128e-05	\\
-7019.82421875	4.9462784912416e-05	\\
-7018.84543678977	5.00944577612681e-05	\\
-7017.86665482955	5.11850870218131e-05	\\
-7016.88787286932	4.92013462473341e-05	\\
-7015.90909090909	4.88853172295921e-05	\\
-7014.93030894886	4.93517302588676e-05	\\
-7013.95152698864	4.99449541664025e-05	\\
-7012.97274502841	5.12708209294319e-05	\\
-7011.99396306818	4.94798808051692e-05	\\
-7011.01518110795	5.0963098719834e-05	\\
-7010.03639914773	4.90177152700843e-05	\\
-7009.0576171875	5.09069750794922e-05	\\
-7008.07883522727	4.93198765238483e-05	\\
-7007.10005326705	4.84656336564299e-05	\\
-7006.12127130682	5.04946564721888e-05	\\
-7005.14248934659	4.8906891444412e-05	\\
-7004.16370738636	4.94751160088441e-05	\\
-7003.18492542614	4.84061921440679e-05	\\
-7002.20614346591	4.94998320754715e-05	\\
-7001.22736150568	4.88173847002555e-05	\\
-7000.24857954545	4.81891802779455e-05	\\
-6999.26979758523	4.95933801759763e-05	\\
-6998.291015625	4.98471053261082e-05	\\
-6997.31223366477	5.29058847870144e-05	\\
-6996.33345170455	5.07623582524415e-05	\\
-6995.35466974432	4.88311690708102e-05	\\
-6994.37588778409	5.12941195125041e-05	\\
-6993.39710582386	5.04072549818672e-05	\\
-6992.41832386364	5.04233174498819e-05	\\
-6991.43954190341	5.16424406538759e-05	\\
-6990.46075994318	5.21267131300695e-05	\\
-6989.48197798295	5.21257574212969e-05	\\
-6988.50319602273	5.10537066801284e-05	\\
-6987.5244140625	5.11520659224352e-05	\\
-6986.54563210227	5.0133335703374e-05	\\
-6985.56685014205	5.35591892978161e-05	\\
-6984.58806818182	4.97184073403378e-05	\\
-6983.60928622159	5.05188986309719e-05	\\
-6982.63050426136	5.2880230159809e-05	\\
-6981.65172230114	5.21241270561998e-05	\\
-6980.67294034091	5.26206462158616e-05	\\
-6979.69415838068	5.15262585666197e-05	\\
-6978.71537642045	5.11220089328659e-05	\\
-6977.73659446023	5.20440275472897e-05	\\
-6976.7578125	5.41507476167174e-05	\\
-6975.77903053977	5.30684783668952e-05	\\
-6974.80024857955	5.34089040945777e-05	\\
-6973.82146661932	5.22320965579164e-05	\\
-6972.84268465909	5.27623949289899e-05	\\
-6971.86390269886	5.26276738276596e-05	\\
-6970.88512073864	5.25983562473774e-05	\\
-6969.90633877841	5.41167361106488e-05	\\
-6968.92755681818	5.37385804662779e-05	\\
-6967.94877485795	5.27100026684854e-05	\\
-6966.96999289773	5.38105008692629e-05	\\
-6965.9912109375	5.34513424371149e-05	\\
-6965.01242897727	5.34603490972642e-05	\\
-6964.03364701705	5.37038723643971e-05	\\
-6963.05486505682	5.33215619537023e-05	\\
-6962.07608309659	5.4012775139542e-05	\\
-6961.09730113636	5.51147250722082e-05	\\
-6960.11851917614	5.36488213942229e-05	\\
-6959.13973721591	5.25531245338651e-05	\\
-6958.16095525568	5.3069667167221e-05	\\
-6957.18217329545	5.34167344647697e-05	\\
-6956.20339133523	5.12634170715461e-05	\\
-6955.224609375	5.38416284028755e-05	\\
-6954.24582741477	5.285753471294e-05	\\
-6953.26704545455	5.29943779394305e-05	\\
-6952.28826349432	5.31746788134896e-05	\\
-6951.30948153409	5.32790212459669e-05	\\
-6950.33069957386	5.23796210093829e-05	\\
-6949.35191761364	5.25797103006936e-05	\\
-6948.37313565341	5.42782063412882e-05	\\
-6947.39435369318	5.43977115175001e-05	\\
-6946.41557173295	5.47896876317949e-05	\\
-6945.43678977273	5.41408066370599e-05	\\
-6944.4580078125	5.2913872541146e-05	\\
-6943.47922585227	5.3713363886018e-05	\\
-6942.50044389205	5.24130395376517e-05	\\
-6941.52166193182	5.26284754808971e-05	\\
-6940.54287997159	5.5195202649422e-05	\\
-6939.56409801136	5.43505826132957e-05	\\
-6938.58531605114	5.14688707683835e-05	\\
-6937.60653409091	5.3352948404427e-05	\\
-6936.62775213068	5.25912131773214e-05	\\
-6935.64897017045	5.23347120577271e-05	\\
-6934.67018821023	5.35138928762576e-05	\\
-6933.69140625	5.38243657702748e-05	\\
-6932.71262428977	5.2649916385634e-05	\\
-6931.73384232955	5.23005852753318e-05	\\
-6930.75506036932	5.26530244955834e-05	\\
-6929.77627840909	5.4192101181683e-05	\\
-6928.79749644886	5.47236113553727e-05	\\
-6927.81871448864	5.38011805662268e-05	\\
-6926.83993252841	5.29165159209081e-05	\\
-6925.86115056818	5.2550998600305e-05	\\
-6924.88236860796	5.2787065558166e-05	\\
-6923.90358664773	5.21811285382804e-05	\\
-6922.9248046875	5.47182553405516e-05	\\
-6921.94602272727	5.44518780283564e-05	\\
-6920.96724076705	5.35301774998145e-05	\\
-6919.98845880682	5.23823847757942e-05	\\
-6919.00967684659	5.42443584778208e-05	\\
-6918.03089488636	5.33735826848167e-05	\\
-6917.05211292614	5.36476201674738e-05	\\
-6916.07333096591	5.21142086491917e-05	\\
-6915.09454900568	5.48637403420827e-05	\\
-6914.11576704546	5.22178530582701e-05	\\
-6913.13698508523	5.45452688228729e-05	\\
-6912.158203125	5.51626619719653e-05	\\
-6911.17942116477	5.3455227445152e-05	\\
-6910.20063920455	5.26187185296193e-05	\\
-6909.22185724432	5.43633408066034e-05	\\
-6908.24307528409	5.34199305983523e-05	\\
-6907.26429332386	5.31173508909767e-05	\\
-6906.28551136364	5.31683239003779e-05	\\
-6905.30672940341	5.26063709955482e-05	\\
-6904.32794744318	5.19210789464829e-05	\\
-6903.34916548296	5.43214788097169e-05	\\
-6902.37038352273	5.35238983208239e-05	\\
-6901.3916015625	5.3218171256962e-05	\\
-6900.41281960227	5.17454731127319e-05	\\
-6899.43403764205	5.32595818516777e-05	\\
-6898.45525568182	5.38157144054355e-05	\\
-6897.47647372159	5.46925045454238e-05	\\
-6896.49769176136	5.49736478507208e-05	\\
-6895.51890980114	5.27708532270854e-05	\\
-6894.54012784091	5.40786555700244e-05	\\
-6893.56134588068	5.54038813622639e-05	\\
-6892.58256392046	5.4747664392776e-05	\\
-6891.60378196023	5.40364907791004e-05	\\
-6890.625	5.28722202184645e-05	\\
-6889.64621803977	5.18617236345882e-05	\\
-6888.66743607955	5.58777900458139e-05	\\
-6887.68865411932	5.3149413028665e-05	\\
-6886.70987215909	5.65198541341177e-05	\\
-6885.73109019886	5.33024533210998e-05	\\
-6884.75230823864	5.57340763516786e-05	\\
-6883.77352627841	5.44206916120312e-05	\\
-6882.79474431818	5.48070454467814e-05	\\
-6881.81596235796	5.54767550353444e-05	\\
-6880.83718039773	5.50958487753671e-05	\\
-6879.8583984375	5.633872283808e-05	\\
-6878.87961647727	5.41401553045367e-05	\\
-6877.90083451705	5.65185113654271e-05	\\
-6876.92205255682	5.40123773858571e-05	\\
-6875.94327059659	5.41209916724239e-05	\\
-6874.96448863636	5.68581880725933e-05	\\
-6873.98570667614	5.4835519647518e-05	\\
-6873.00692471591	5.52407085870841e-05	\\
-6872.02814275568	5.58419117556788e-05	\\
-6871.04936079546	5.67015226549683e-05	\\
-6870.07057883523	5.65924093113506e-05	\\
-6869.091796875	5.70862096755094e-05	\\
-6868.11301491477	5.743431475671e-05	\\
-6867.13423295455	5.58316507756557e-05	\\
-6866.15545099432	5.80932743272519e-05	\\
-6865.17666903409	5.6689402544439e-05	\\
-6864.19788707386	5.65462603905374e-05	\\
-6863.21910511364	5.79975073434141e-05	\\
-6862.24032315341	5.74173542980752e-05	\\
-6861.26154119318	5.68038562673583e-05	\\
-6860.28275923296	5.49594153683065e-05	\\
-6859.30397727273	5.80802362711424e-05	\\
-6858.3251953125	5.63646292031084e-05	\\
-6857.34641335227	5.73568219137205e-05	\\
-6856.36763139205	5.59598693731727e-05	\\
-6855.38884943182	5.86350963470198e-05	\\
-6854.41006747159	5.76345921513953e-05	\\
-6853.43128551136	5.66715287056668e-05	\\
-6852.45250355114	5.69603404551919e-05	\\
-6851.47372159091	5.85617859926601e-05	\\
-6850.49493963068	5.74666948458139e-05	\\
-6849.51615767046	5.82003673061369e-05	\\
-6848.53737571023	5.86484879881613e-05	\\
-6847.55859375	5.80358954636861e-05	\\
-6846.57981178977	6.04545591022581e-05	\\
-6845.60102982955	5.68587777414993e-05	\\
-6844.62224786932	5.90558797260436e-05	\\
-6843.64346590909	5.74016050751622e-05	\\
-6842.66468394886	5.63166898150117e-05	\\
-6841.68590198864	5.72831784965317e-05	\\
-6840.70712002841	5.89861042202451e-05	\\
-6839.72833806818	5.75312641972507e-05	\\
-6838.74955610796	5.85555921211162e-05	\\
-6837.77077414773	5.81501374944718e-05	\\
-6836.7919921875	5.76126986490421e-05	\\
-6835.81321022727	5.94183487593451e-05	\\
-6834.83442826705	5.92672950925777e-05	\\
-6833.85564630682	5.64977961407683e-05	\\
-6832.87686434659	5.94924396072088e-05	\\
-6831.89808238636	6.05097744089899e-05	\\
-6830.91930042614	5.94904354500122e-05	\\
-6829.94051846591	6.13202877725867e-05	\\
-6828.96173650568	5.95636855040651e-05	\\
-6827.98295454546	5.97837792057642e-05	\\
-6827.00417258523	5.97281722519794e-05	\\
-6826.025390625	6.02045672239998e-05	\\
-6825.04660866477	6.13133810586141e-05	\\
-6824.06782670455	6.07325178675245e-05	\\
-6823.08904474432	6.13995112146263e-05	\\
-6822.11026278409	6.14164237637914e-05	\\
-6821.13148082386	5.90276256444656e-05	\\
-6820.15269886364	6.13536426227757e-05	\\
-6819.17391690341	6.01811056274446e-05	\\
-6818.19513494318	6.09386077370516e-05	\\
-6817.21635298296	6.15624806654336e-05	\\
-6816.23757102273	6.02184953315759e-05	\\
-6815.2587890625	6.03859559608153e-05	\\
-6814.28000710227	6.27838831804155e-05	\\
-6813.30122514205	6.04911689776873e-05	\\
-6812.32244318182	6.11840088379149e-05	\\
-6811.34366122159	6.19054115816831e-05	\\
-6810.36487926136	5.92295726026382e-05	\\
-6809.38609730114	5.91239119738799e-05	\\
-6808.40731534091	6.16675323082322e-05	\\
-6807.42853338068	6.02347962666781e-05	\\
-6806.44975142046	5.99454566155081e-05	\\
-6805.47096946023	5.97660070320082e-05	\\
-6804.4921875	5.93452225691821e-05	\\
-6803.51340553977	6.10650539526294e-05	\\
-6802.53462357955	5.98133163798759e-05	\\
-6801.55584161932	6.31290204132667e-05	\\
-6800.57705965909	6.09206841364851e-05	\\
-6799.59827769886	6.04671742658958e-05	\\
-6798.61949573864	6.25550561298202e-05	\\
-6797.64071377841	6.18615943268743e-05	\\
-6796.66193181818	6.27625027175432e-05	\\
-6795.68314985796	5.98385163986803e-05	\\
-6794.70436789773	5.97022483313576e-05	\\
-6793.7255859375	6.1027002960923e-05	\\
-6792.74680397727	6.00404702633338e-05	\\
-6791.76802201705	6.00470292216285e-05	\\
-6790.78924005682	6.12630539038269e-05	\\
-6789.81045809659	6.12334432281967e-05	\\
-6788.83167613636	6.17550722620676e-05	\\
-6787.85289417614	6.25096230923484e-05	\\
-6786.87411221591	6.24694241167678e-05	\\
-6785.89533025568	5.96774276214133e-05	\\
-6784.91654829546	6.10800481645283e-05	\\
-6783.93776633523	6.11077167821823e-05	\\
-6782.958984375	6.12811210115901e-05	\\
-6781.98020241477	6.02301353716888e-05	\\
-6781.00142045455	6.27961674153378e-05	\\
-6780.02263849432	6.32021232695356e-05	\\
-6779.04385653409	6.2606479517648e-05	\\
-6778.06507457386	6.119874354652e-05	\\
-6777.08629261364	6.34961205948871e-05	\\
-6776.10751065341	6.32276319222291e-05	\\
-6775.12872869318	6.34774811932019e-05	\\
-6774.14994673296	6.30795359479196e-05	\\
-6773.17116477273	6.28939302647474e-05	\\
-6772.1923828125	6.53536386751413e-05	\\
-6771.21360085227	6.36349152568188e-05	\\
-6770.23481889205	6.37534756053213e-05	\\
-6769.25603693182	6.39040054107718e-05	\\
-6768.27725497159	6.32193990411528e-05	\\
-6767.29847301136	6.48336457387416e-05	\\
-6766.31969105114	6.326030359477e-05	\\
-6765.34090909091	6.54584919581096e-05	\\
-6764.36212713068	6.58171135376649e-05	\\
-6763.38334517046	6.24051651277236e-05	\\
-6762.40456321023	6.56412800080002e-05	\\
-6761.42578125	6.4047843228396e-05	\\
-6760.44699928977	6.40919267199578e-05	\\
-6759.46821732955	6.34634884729804e-05	\\
-6758.48943536932	6.47120843995595e-05	\\
-6757.51065340909	6.5732891870741e-05	\\
-6756.53187144886	6.50471893418544e-05	\\
-6755.55308948864	6.58396681025327e-05	\\
-6754.57430752841	6.23809853073139e-05	\\
-6753.59552556818	6.42368279222577e-05	\\
-6752.61674360796	6.45535132273245e-05	\\
-6751.63796164773	6.62136584519458e-05	\\
-6750.6591796875	6.11184082639544e-05	\\
-6749.68039772727	6.46971994523352e-05	\\
-6748.70161576705	6.64526037562545e-05	\\
-6747.72283380682	6.42535175508145e-05	\\
-6746.74405184659	6.39939045348567e-05	\\
-6745.76526988636	6.5828082699282e-05	\\
-6744.78648792614	6.31265625578433e-05	\\
-6743.80770596591	6.28354502608328e-05	\\
-6742.82892400568	6.30637068505782e-05	\\
-6741.85014204546	6.2897761664308e-05	\\
-6740.87136008523	6.43378859324154e-05	\\
-6739.892578125	6.36924460286778e-05	\\
-6738.91379616477	6.37163187711365e-05	\\
-6737.93501420455	6.38119037531556e-05	\\
-6736.95623224432	6.32672424387235e-05	\\
-6735.97745028409	6.48310684507012e-05	\\
-6734.99866832386	6.48722644862677e-05	\\
-6734.01988636364	6.23879235809926e-05	\\
-6733.04110440341	6.39434169254876e-05	\\
-6732.06232244318	6.4082022314166e-05	\\
-6731.08354048296	6.31588668890792e-05	\\
-6730.10475852273	6.2708079466906e-05	\\
-6729.1259765625	6.39798995887416e-05	\\
-6728.14719460227	6.31728364353983e-05	\\
-6727.16841264205	6.29524911584727e-05	\\
-6726.18963068182	6.17426758292082e-05	\\
-6725.21084872159	6.30659716687256e-05	\\
-6724.23206676136	6.5220846227807e-05	\\
-6723.25328480114	6.47373963490118e-05	\\
-6722.27450284091	6.18855200058217e-05	\\
-6721.29572088068	6.19763180093575e-05	\\
-6720.31693892046	6.27231911260292e-05	\\
-6719.33815696023	6.24025134204611e-05	\\
-6718.359375	6.26683671033754e-05	\\
-6717.38059303977	6.16103403200136e-05	\\
-6716.40181107955	6.29325450984537e-05	\\
-6715.42302911932	6.14863859145157e-05	\\
-6714.44424715909	6.20890096952477e-05	\\
-6713.46546519886	6.36749507166997e-05	\\
-6712.48668323864	6.38587236620751e-05	\\
-6711.50790127841	6.40175937197677e-05	\\
-6710.52911931818	6.22191771075048e-05	\\
-6709.55033735796	6.36006368444583e-05	\\
-6708.57155539773	6.36306953618692e-05	\\
-6707.5927734375	6.1120029161823e-05	\\
-6706.61399147727	6.28299274730406e-05	\\
-6705.63520951705	6.20091673380465e-05	\\
-6704.65642755682	6.36355475545587e-05	\\
-6703.67764559659	6.03106472322084e-05	\\
-6702.69886363636	6.25172575312351e-05	\\
-6701.72008167614	6.32203353713751e-05	\\
-6700.74129971591	6.14318521867244e-05	\\
-6699.76251775568	6.42548166915549e-05	\\
-6698.78373579546	6.3075403534844e-05	\\
-6697.80495383523	6.16019893479661e-05	\\
-6696.826171875	6.24174998872774e-05	\\
-6695.84738991477	6.2023221573867e-05	\\
-6694.86860795455	6.43668328120412e-05	\\
-6693.88982599432	6.34279584232607e-05	\\
-6692.91104403409	6.20678653103039e-05	\\
-6691.93226207386	6.26423253460537e-05	\\
-6690.95348011364	6.28936695187638e-05	\\
-6689.97469815341	6.31936159093831e-05	\\
-6688.99591619318	6.09102726525827e-05	\\
-6688.01713423296	6.16815058298059e-05	\\
-6687.03835227273	6.14504907714776e-05	\\
-6686.0595703125	6.15605448828716e-05	\\
-6685.08078835227	6.08848461863796e-05	\\
-6684.10200639205	6.09198397842079e-05	\\
-6683.12322443182	6.14238396552964e-05	\\
-6682.14444247159	6.24956470475049e-05	\\
-6681.16566051136	6.16718048145692e-05	\\
-6680.18687855114	6.0407261505555e-05	\\
-6679.20809659091	6.18251745935845e-05	\\
-6678.22931463068	6.21914940305904e-05	\\
-6677.25053267046	6.29229632351953e-05	\\
-6676.27175071023	6.12196461336616e-05	\\
-6675.29296875	6.051417197683e-05	\\
-6674.31418678977	6.18920021569511e-05	\\
-6673.33540482955	6.20341262539467e-05	\\
-6672.35662286932	6.32945699026812e-05	\\
-6671.37784090909	6.12798377964338e-05	\\
-6670.39905894886	5.97627653630039e-05	\\
-6669.42027698864	6.10364118878426e-05	\\
-6668.44149502841	6.22104815354327e-05	\\
-6667.46271306818	6.15712175779306e-05	\\
-6666.48393110796	6.13443247091336e-05	\\
-6665.50514914773	6.09302972761615e-05	\\
-6664.5263671875	6.15631561668784e-05	\\
-6663.54758522727	6.0241470217465e-05	\\
-6662.56880326705	6.33108900255869e-05	\\
-6661.59002130682	6.24041848952822e-05	\\
-6660.61123934659	6.10971723330159e-05	\\
-6659.63245738636	5.94851081398564e-05	\\
-6658.65367542614	6.16984314002701e-05	\\
-6657.67489346591	6.14986931027863e-05	\\
-6656.69611150568	6.19126609834887e-05	\\
-6655.71732954546	6.2329904959131e-05	\\
-6654.73854758523	6.20762923666189e-05	\\
-6653.759765625	6.24231899465047e-05	\\
-6652.78098366477	6.1245879837471e-05	\\
-6651.80220170455	6.20627748751509e-05	\\
-6650.82341974432	6.09315888255935e-05	\\
-6649.84463778409	6.048327822235e-05	\\
-6648.86585582386	6.25272001360583e-05	\\
-6647.88707386364	6.29417438688232e-05	\\
-6646.90829190341	6.31800705744496e-05	\\
-6645.92950994318	6.32144176860341e-05	\\
-6644.95072798296	6.26573171062782e-05	\\
-6643.97194602273	6.10221776470527e-05	\\
-6642.9931640625	6.18370322376139e-05	\\
-6642.01438210227	6.07934647833025e-05	\\
-6641.03560014205	6.21037089491746e-05	\\
-6640.05681818182	6.3363866435333e-05	\\
-6639.07803622159	6.33620824607047e-05	\\
-6638.09925426136	6.17844982357655e-05	\\
-6637.12047230114	6.28803095836403e-05	\\
-6636.14169034091	6.05969096333035e-05	\\
-6635.16290838068	6.08490830189532e-05	\\
-6634.18412642046	6.27648191519905e-05	\\
-6633.20534446023	6.17863350648567e-05	\\
-6632.2265625	6.28570434131942e-05	\\
-6631.24778053977	6.13209809176374e-05	\\
-6630.26899857955	6.15749818268543e-05	\\
-6629.29021661932	6.13017173983529e-05	\\
-6628.31143465909	6.14558258630771e-05	\\
-6627.33265269886	6.28253095750311e-05	\\
-6626.35387073864	6.09142556838759e-05	\\
-6625.37508877841	6.13724503190052e-05	\\
-6624.39630681818	6.15527518139132e-05	\\
-6623.41752485796	6.17006715902349e-05	\\
-6622.43874289773	6.22701209386818e-05	\\
-6621.4599609375	6.15111492087288e-05	\\
-6620.48117897727	6.0897827153172e-05	\\
-6619.50239701705	6.077796839845e-05	\\
-6618.52361505682	6.05454590036372e-05	\\
-6617.54483309659	5.9770710136547e-05	\\
-6616.56605113636	6.16469089469509e-05	\\
-6615.58726917614	6.11611870817549e-05	\\
-6614.60848721591	6.22249957700379e-05	\\
-6613.62970525568	6.19712237540353e-05	\\
-6612.65092329546	6.17348969772544e-05	\\
-6611.67214133523	6.16305791129016e-05	\\
-6610.693359375	6.05837564637488e-05	\\
-6609.71457741477	6.20398729012834e-05	\\
-6608.73579545455	6.20344447121773e-05	\\
-6607.75701349432	6.10939185784104e-05	\\
-6606.77823153409	6.03534119917533e-05	\\
-6605.79944957386	6.25279761679533e-05	\\
-6604.82066761364	6.12296787309555e-05	\\
-6603.84188565341	6.09645867211598e-05	\\
-6602.86310369318	6.13308797015104e-05	\\
-6601.88432173296	6.10413286719911e-05	\\
-6600.90553977273	6.27174593341475e-05	\\
-6599.9267578125	5.98425008089737e-05	\\
-6598.94797585227	6.29801300073714e-05	\\
-6597.96919389205	6.16169089213898e-05	\\
-6596.99041193182	6.28628735548823e-05	\\
-6596.01162997159	6.42365378164646e-05	\\
-6595.03284801136	6.09806325500563e-05	\\
-6594.05406605114	6.27349778748286e-05	\\
-6593.07528409091	6.22417974349387e-05	\\
-6592.09650213068	6.24431243589134e-05	\\
-6591.11772017046	5.96647482276562e-05	\\
-6590.13893821023	6.17188386368902e-05	\\
-6589.16015625	6.13741147315554e-05	\\
-6588.18137428977	6.13326709042723e-05	\\
-6587.20259232955	6.38826201780366e-05	\\
-6586.22381036932	6.39316781067426e-05	\\
-6585.24502840909	6.11059708543334e-05	\\
-6584.26624644886	6.30307955289988e-05	\\
-6583.28746448864	6.27476244878202e-05	\\
-6582.30868252841	6.18655360618947e-05	\\
-6581.32990056818	6.27074859851181e-05	\\
-6580.35111860796	6.36175658991961e-05	\\
-6579.37233664773	6.3115572326765e-05	\\
-6578.3935546875	6.31110375491465e-05	\\
-6577.41477272727	6.24911316534834e-05	\\
-6576.43599076705	6.12062436589187e-05	\\
-6575.45720880682	6.3560342483682e-05	\\
-6574.47842684659	6.19627367873301e-05	\\
-6573.49964488636	6.31334848139604e-05	\\
-6572.52086292614	6.42362365639158e-05	\\
-6571.54208096591	6.19696619109435e-05	\\
-6570.56329900568	6.30958276957559e-05	\\
-6569.58451704546	6.27201923911657e-05	\\
-6568.60573508523	6.3948145037949e-05	\\
-6567.626953125	6.32408893813767e-05	\\
-6566.64817116477	6.34495320017461e-05	\\
-6565.66938920455	6.42120997798879e-05	\\
-6564.69060724432	6.45433684265238e-05	\\
-6563.71182528409	6.37427703186719e-05	\\
-6562.73304332386	6.38239457911945e-05	\\
-6561.75426136364	6.34596876657299e-05	\\
-6560.77547940341	6.57119869753202e-05	\\
-6559.79669744318	6.53965656563975e-05	\\
-6558.81791548296	6.42370594775224e-05	\\
-6557.83913352273	6.44323878474394e-05	\\
-6556.8603515625	6.40873339070234e-05	\\
-6555.88156960227	6.57114233140179e-05	\\
-6554.90278764205	6.59120010047493e-05	\\
-6553.92400568182	6.5564516041436e-05	\\
-6552.94522372159	6.58403464323771e-05	\\
-6551.96644176136	6.34867819237224e-05	\\
-6550.98765980114	6.30007228157396e-05	\\
-6550.00887784091	6.54594265042243e-05	\\
-6549.03009588068	6.49095605682608e-05	\\
-6548.05131392046	6.43631155290662e-05	\\
-6547.07253196023	6.49811929388739e-05	\\
-6546.09375	6.50776042197762e-05	\\
-6545.11496803977	6.4449097945682e-05	\\
-6544.13618607955	6.3391736396131e-05	\\
-6543.15740411932	6.58876635080832e-05	\\
-6542.17862215909	6.36738427832961e-05	\\
-6541.19984019886	6.54630268844969e-05	\\
-6540.22105823864	6.37814926091821e-05	\\
-6539.24227627841	6.56778094895324e-05	\\
-6538.26349431818	6.30912024945953e-05	\\
-6537.28471235796	6.56688993765076e-05	\\
-6536.30593039773	6.60803383289231e-05	\\
-6535.3271484375	6.45065675938918e-05	\\
-6534.34836647727	6.33422804632912e-05	\\
-6533.36958451705	6.54008615654529e-05	\\
-6532.39080255682	6.3785885215866e-05	\\
-6531.41202059659	6.52537986808303e-05	\\
-6530.43323863636	6.46551882248713e-05	\\
-6529.45445667614	6.45453210968538e-05	\\
-6528.47567471591	6.5531550265634e-05	\\
-6527.49689275568	6.41292019041528e-05	\\
-6526.51811079546	6.31209451977795e-05	\\
-6525.53932883523	6.48088551921953e-05	\\
-6524.560546875	6.59324378441639e-05	\\
-6523.58176491477	6.52701827176533e-05	\\
-6522.60298295455	6.50300758879578e-05	\\
-6521.62420099432	6.54800443569294e-05	\\
-6520.64541903409	6.49604937216315e-05	\\
-6519.66663707386	6.60267899859436e-05	\\
-6518.68785511364	6.2403446628207e-05	\\
-6517.70907315341	6.36845793844993e-05	\\
-6516.73029119318	6.47755154032983e-05	\\
-6515.75150923296	6.52034197064511e-05	\\
-6514.77272727273	6.44824803122216e-05	\\
-6513.7939453125	6.28545178335858e-05	\\
-6512.81516335227	6.33543449832155e-05	\\
-6511.83638139205	6.28367334338091e-05	\\
-6510.85759943182	6.31076795693348e-05	\\
-6509.87881747159	6.40859436315844e-05	\\
-6508.90003551136	6.40221474926087e-05	\\
-6507.92125355114	6.40499883646772e-05	\\
-6506.94247159091	6.3429829582626e-05	\\
-6505.96368963068	6.29559575484223e-05	\\
-6504.98490767046	6.27683112102041e-05	\\
-6504.00612571023	6.36694405319495e-05	\\
-6503.02734375	6.16179575224146e-05	\\
-6502.04856178977	6.13088189899854e-05	\\
-6501.06977982955	6.37153410473253e-05	\\
-6500.09099786932	6.51511536762369e-05	\\
-6499.11221590909	6.45470103784469e-05	\\
-6498.13343394886	6.44801413531659e-05	\\
-6497.15465198864	6.47001566000947e-05	\\
-6496.17587002841	6.32535156927438e-05	\\
-6495.19708806818	6.46570804466066e-05	\\
-6494.21830610796	6.39916353941513e-05	\\
-6493.23952414773	6.22658567790595e-05	\\
-6492.2607421875	6.40215940990671e-05	\\
-6491.28196022727	6.32675721669714e-05	\\
-6490.30317826705	6.30851398472701e-05	\\
-6489.32439630682	6.30133139661585e-05	\\
-6488.34561434659	6.29583013460427e-05	\\
-6487.36683238636	6.3270012218767e-05	\\
-6486.38805042614	6.39519099806743e-05	\\
-6485.40926846591	6.54226172661272e-05	\\
-6484.43048650568	6.24804109835448e-05	\\
-6483.45170454546	6.41049640171126e-05	\\
-6482.47292258523	6.27392687352636e-05	\\
-6481.494140625	6.04158661505141e-05	\\
-6480.51535866477	6.42065483303028e-05	\\
-6479.53657670455	6.39659693262258e-05	\\
-6478.55779474432	6.34292004499349e-05	\\
-6477.57901278409	6.35779795767656e-05	\\
-6476.60023082386	6.35205946164217e-05	\\
-6475.62144886364	6.13287702554233e-05	\\
-6474.64266690341	6.2441728055844e-05	\\
-6473.66388494318	6.1813604517289e-05	\\
-6472.68510298296	6.17926381852786e-05	\\
-6471.70632102273	6.08221554049281e-05	\\
-6470.7275390625	6.17077300499775e-05	\\
-6469.74875710227	6.15396822224059e-05	\\
-6468.76997514205	6.33844902553763e-05	\\
-6467.79119318182	6.2496758555587e-05	\\
-6466.81241122159	6.16692108629408e-05	\\
-6465.83362926136	6.23794774381257e-05	\\
-6464.85484730114	6.1626397385932e-05	\\
-6463.87606534091	6.0171191936885e-05	\\
-6462.89728338068	5.98397938804661e-05	\\
-6461.91850142046	6.14969820486589e-05	\\
-6460.93971946023	6.1264079335212e-05	\\
-6459.9609375	6.11725145351342e-05	\\
-6458.98215553977	5.88074857860498e-05	\\
-6458.00337357955	6.07419929386814e-05	\\
-6457.02459161932	6.19648847926505e-05	\\
-6456.04580965909	5.94887574392721e-05	\\
-6455.06702769886	5.9294720437553e-05	\\
-6454.08824573864	5.85526343291197e-05	\\
-6453.10946377841	5.92820191844736e-05	\\
-6452.13068181818	6.03375259952562e-05	\\
-6451.15189985796	5.91373700873237e-05	\\
-6450.17311789773	6.12098539079458e-05	\\
-6449.1943359375	5.98064497043879e-05	\\
-6448.21555397727	6.1892430679765e-05	\\
-6447.23677201705	5.9588488482696e-05	\\
-6446.25799005682	5.85239978786601e-05	\\
-6445.27920809659	5.84111899247482e-05	\\
-6444.30042613636	5.73431504901066e-05	\\
-6443.32164417614	5.83928053232304e-05	\\
-6442.34286221591	5.90306005314959e-05	\\
-6441.36408025568	5.93166570848479e-05	\\
-6440.38529829546	5.82214777334156e-05	\\
-6439.40651633523	5.79156077228631e-05	\\
-6438.427734375	5.70989512980827e-05	\\
-6437.44895241477	5.67244581277914e-05	\\
-6436.47017045455	5.61762654010369e-05	\\
-6435.49138849432	5.78734706605368e-05	\\
-6434.51260653409	5.74517871939178e-05	\\
-6433.53382457386	5.4844465774181e-05	\\
-6432.55504261364	5.63549654736476e-05	\\
-6431.57626065341	5.83519584454986e-05	\\
-6430.59747869318	5.74633403279085e-05	\\
-6429.61869673296	5.67057414808976e-05	\\
-6428.63991477273	5.80202194642472e-05	\\
-6427.6611328125	5.50564867711579e-05	\\
-6426.68235085227	5.5368256228844e-05	\\
-6425.70356889205	5.71617795802698e-05	\\
-6424.72478693182	5.48796055361574e-05	\\
-6423.74600497159	5.57818386920545e-05	\\
-6422.76722301136	5.70813975024056e-05	\\
-6421.78844105114	5.67283733320191e-05	\\
-6420.80965909091	5.50578401214411e-05	\\
-6419.83087713068	5.33789703706016e-05	\\
-6418.85209517046	5.43066429721654e-05	\\
-6417.87331321023	5.5173820467985e-05	\\
-6416.89453125	5.48459303558714e-05	\\
-6415.91574928977	5.41617050016065e-05	\\
-6414.93696732955	5.47384611506205e-05	\\
-6413.95818536932	5.47782056236958e-05	\\
-6412.97940340909	5.6817805946723e-05	\\
-6412.00062144886	5.38513767966888e-05	\\
-6411.02183948864	5.43004770900606e-05	\\
-6410.04305752841	5.45595407654757e-05	\\
-6409.06427556818	5.18479235859951e-05	\\
-6408.08549360796	5.28565433403952e-05	\\
-6407.10671164773	5.24276720206565e-05	\\
-6406.1279296875	5.30220663381418e-05	\\
-6405.14914772727	5.35365072693376e-05	\\
-6404.17036576705	5.280833509614e-05	\\
-6403.19158380682	5.22865532689179e-05	\\
-6402.21280184659	5.19415833139552e-05	\\
-6401.23401988636	5.09725234533342e-05	\\
-6400.25523792614	5.09294724778151e-05	\\
-6399.27645596591	4.99506385047793e-05	\\
-6398.29767400568	5.12369611985935e-05	\\
-6397.31889204546	5.19543943897218e-05	\\
-6396.34011008523	5.20447532418041e-05	\\
-6395.361328125	5.02410575858985e-05	\\
-6394.38254616477	5.23789455886056e-05	\\
-6393.40376420455	5.13758779942963e-05	\\
-6392.42498224432	4.93138747303967e-05	\\
-6391.44620028409	5.34416162159027e-05	\\
-6390.46741832386	5.06297346980278e-05	\\
-6389.48863636364	4.97785433297394e-05	\\
-6388.50985440341	5.09489407523763e-05	\\
-6387.53107244318	5.22681365604714e-05	\\
-6386.55229048296	4.96109747211592e-05	\\
-6385.57350852273	5.04391027433788e-05	\\
-6384.5947265625	4.9233042854408e-05	\\
-6383.61594460227	4.9618099908894e-05	\\
-6382.63716264205	5.15811393620722e-05	\\
-6381.65838068182	4.93552857187171e-05	\\
-6380.67959872159	4.88116515157183e-05	\\
-6379.70081676136	5.07100213040314e-05	\\
-6378.72203480114	4.86185577162231e-05	\\
-6377.74325284091	4.91776205057799e-05	\\
-6376.76447088068	5.05753095367628e-05	\\
-6375.78568892046	5.12676115946077e-05	\\
-6374.80690696023	4.95356634719484e-05	\\
-6373.828125	4.9638060771381e-05	\\
-6372.84934303977	4.83386700762544e-05	\\
-6371.87056107955	4.91485440792664e-05	\\
-6370.89177911932	4.93386608583307e-05	\\
-6369.91299715909	4.89641143496853e-05	\\
-6368.93421519886	4.92064801514097e-05	\\
-6367.95543323864	5.01268863548316e-05	\\
-6366.97665127841	4.79301912696676e-05	\\
-6365.99786931818	4.92268050418309e-05	\\
-6365.01908735796	4.88772770224057e-05	\\
-6364.04030539773	4.91649357926402e-05	\\
-6363.0615234375	4.96782419842031e-05	\\
-6362.08274147727	4.84493775201274e-05	\\
-6361.10395951705	5.07335119612987e-05	\\
-6360.12517755682	4.92127270901818e-05	\\
-6359.14639559659	5.01071960457785e-05	\\
-6358.16761363636	4.91653486007029e-05	\\
-6357.18883167614	4.99190820716685e-05	\\
-6356.21004971591	4.93690018881049e-05	\\
-6355.23126775568	5.14377298500664e-05	\\
-6354.25248579546	4.9008475170637e-05	\\
-6353.27370383523	4.85308126462026e-05	\\
-6352.294921875	4.89037125727011e-05	\\
-6351.31613991477	4.86217651964005e-05	\\
-6350.33735795455	4.73917048390285e-05	\\
-6349.35857599432	4.74197665057179e-05	\\
-6348.37979403409	5.02816763393021e-05	\\
-6347.40101207386	4.92460397141231e-05	\\
-6346.42223011364	4.98500737292148e-05	\\
-6345.44344815341	4.89485023420027e-05	\\
-6344.46466619318	4.82558019714636e-05	\\
-6343.48588423296	4.78297124654601e-05	\\
-6342.50710227273	5.04550870264525e-05	\\
-6341.5283203125	4.80635748772252e-05	\\
-6340.54953835227	4.8648047775126e-05	\\
-6339.57075639205	4.82561105249063e-05	\\
-6338.59197443182	4.89528753071525e-05	\\
-6337.61319247159	4.68809233538259e-05	\\
-6336.63441051136	4.76649803421574e-05	\\
-6335.65562855114	4.86745823530078e-05	\\
-6334.67684659091	4.87375161623548e-05	\\
-6333.69806463068	4.69762912155933e-05	\\
-6332.71928267046	5.09229656561019e-05	\\
-6331.74050071023	4.86971440331423e-05	\\
-6330.76171875	4.84851157292674e-05	\\
-6329.78293678977	4.95587074648999e-05	\\
-6328.80415482955	4.71508225190399e-05	\\
-6327.82537286932	4.78159758041182e-05	\\
-6326.84659090909	4.79940360762883e-05	\\
-6325.86780894886	4.88418073064896e-05	\\
-6324.88902698864	4.75161552367628e-05	\\
-6323.91024502841	4.88174538251445e-05	\\
-6322.93146306818	4.80479681808039e-05	\\
-6321.95268110796	4.83710721861615e-05	\\
-6320.97389914773	4.85805972888262e-05	\\
-6319.9951171875	4.90849991723488e-05	\\
-6319.01633522727	4.81553063079682e-05	\\
-6318.03755326705	4.94004032015665e-05	\\
-6317.05877130682	4.73910891866886e-05	\\
-6316.07998934659	4.82816779814501e-05	\\
-6315.10120738636	4.90553113637767e-05	\\
-6314.12242542614	4.89549176243932e-05	\\
-6313.14364346591	4.77175300140462e-05	\\
-6312.16486150568	4.92949642736065e-05	\\
-6311.18607954546	4.80269501739927e-05	\\
-6310.20729758523	4.8338928848676e-05	\\
-6309.228515625	4.67392692024983e-05	\\
-6308.24973366477	4.93636664711688e-05	\\
-6307.27095170455	4.78800849492238e-05	\\
-6306.29216974432	4.82108884243413e-05	\\
-6305.31338778409	4.72784292694943e-05	\\
-6304.33460582386	4.84034531908652e-05	\\
-6303.35582386364	4.89849427020373e-05	\\
-6302.37704190341	4.83256839242708e-05	\\
-6301.39825994318	5.00374456915151e-05	\\
-6300.41947798296	5.17552922481331e-05	\\
-6299.44069602273	4.97880770210772e-05	\\
-6298.4619140625	5.11264991339072e-05	\\
-6297.48313210227	5.03182682404246e-05	\\
-6296.50435014205	5.01390279280097e-05	\\
-6295.52556818182	4.98533293829557e-05	\\
-6294.54678622159	5.00700216098081e-05	\\
-6293.56800426136	4.94535502238242e-05	\\
-6292.58922230114	4.77378106998013e-05	\\
-6291.61044034091	5.04565119763992e-05	\\
-6290.63165838068	5.19051677923104e-05	\\
-6289.65287642046	4.98432539168389e-05	\\
-6288.67409446023	5.06538961000328e-05	\\
-6287.6953125	4.98841060863994e-05	\\
-6286.71653053977	4.9598225063437e-05	\\
-6285.73774857955	5.11098026104382e-05	\\
-6284.75896661932	5.14533329939841e-05	\\
-6283.78018465909	4.93686799094773e-05	\\
-6282.80140269886	5.14454313080598e-05	\\
-6281.82262073864	5.06845947855629e-05	\\
-6280.84383877841	5.12593406914722e-05	\\
-6279.86505681818	4.92288080308909e-05	\\
-6278.88627485796	5.05690878723368e-05	\\
-6277.90749289773	5.15697542606485e-05	\\
-6276.9287109375	5.10598155566031e-05	\\
-6275.94992897727	5.17298304537335e-05	\\
-6274.97114701705	5.12133671129541e-05	\\
-6273.99236505682	5.05396941642924e-05	\\
-6273.01358309659	5.16573934871597e-05	\\
-6272.03480113636	5.0766595838035e-05	\\
-6271.05601917614	5.15734547216377e-05	\\
-6270.07723721591	5.17957774947665e-05	\\
-6269.09845525568	5.04019880707679e-05	\\
-6268.11967329546	5.33847009457553e-05	\\
-6267.14089133523	5.30347777751816e-05	\\
-6266.162109375	5.24601804203017e-05	\\
-6265.18332741477	5.10244058479618e-05	\\
-6264.20454545455	5.13680940999706e-05	\\
-6263.22576349432	5.07750032741249e-05	\\
-6262.24698153409	5.26650017514363e-05	\\
-6261.26819957386	5.34219467049386e-05	\\
-6260.28941761364	5.30516365923332e-05	\\
-6259.31063565341	5.11223944025178e-05	\\
-6258.33185369318	5.17008439750735e-05	\\
-6257.35307173296	5.06467184570909e-05	\\
-6256.37428977273	5.23555766486948e-05	\\
-6255.3955078125	5.3524242731454e-05	\\
-6254.41672585227	5.29605304130442e-05	\\
-6253.43794389205	5.32175058445114e-05	\\
-6252.45916193182	5.3139090007857e-05	\\
-6251.48037997159	5.41703993313709e-05	\\
-6250.50159801136	5.22735410718973e-05	\\
-6249.52281605114	5.38435808153684e-05	\\
-6248.54403409091	5.33981377417626e-05	\\
-6247.56525213068	5.40144003634887e-05	\\
-6246.58647017046	5.45615764905961e-05	\\
-6245.60768821023	5.50008093494956e-05	\\
-6244.62890625	5.57668692858325e-05	\\
-6243.65012428977	5.60441473086439e-05	\\
-6242.67134232955	5.69869355703343e-05	\\
-6241.69256036932	5.35392028666831e-05	\\
-6240.71377840909	5.47391453668372e-05	\\
-6239.73499644886	5.631169136277e-05	\\
-6238.75621448864	5.47659826013004e-05	\\
-6237.77743252841	5.51639005894377e-05	\\
-6236.79865056818	5.61383194029878e-05	\\
-6235.81986860796	5.61785517512806e-05	\\
-6234.84108664773	5.67769133135356e-05	\\
-6233.8623046875	5.54787416255427e-05	\\
-6232.88352272727	5.501110388196e-05	\\
-6231.90474076705	5.53533513498113e-05	\\
-6230.92595880682	5.67437799363817e-05	\\
-6229.94717684659	5.65550302924019e-05	\\
-6228.96839488636	5.66365690463968e-05	\\
-6227.98961292614	5.75954299906496e-05	\\
-6227.01083096591	5.65525265903683e-05	\\
-6226.03204900568	5.84478361848959e-05	\\
-6225.05326704546	5.57701072445536e-05	\\
-6224.07448508523	5.79428219418771e-05	\\
-6223.095703125	5.7157597778341e-05	\\
-6222.11692116477	5.67975085607852e-05	\\
-6221.13813920455	5.63877802590591e-05	\\
-6220.15935724432	5.67726640986e-05	\\
-6219.18057528409	5.83148576746098e-05	\\
-6218.20179332386	5.71475072116063e-05	\\
-6217.22301136364	5.78898193092839e-05	\\
-6216.24422940341	5.68670959842752e-05	\\
-6215.26544744318	5.77034118913135e-05	\\
-6214.28666548296	5.79826325901988e-05	\\
-6213.30788352273	5.78346697061413e-05	\\
-6212.3291015625	5.72090209720305e-05	\\
-6211.35031960227	5.93232882668334e-05	\\
-6210.37153764205	5.70290784416205e-05	\\
-6209.39275568182	5.72649498652428e-05	\\
-6208.41397372159	5.78201092993629e-05	\\
-6207.43519176136	5.79800310733106e-05	\\
-6206.45640980114	5.67917129501767e-05	\\
-6205.47762784091	5.97754511679454e-05	\\
-6204.49884588068	5.90899312262403e-05	\\
-6203.52006392046	5.81639981990553e-05	\\
-6202.54128196023	5.78694051500478e-05	\\
-6201.5625	5.94515504249685e-05	\\
-6200.58371803977	5.91400158576399e-05	\\
-6199.60493607955	5.5989119878216e-05	\\
-6198.62615411932	5.68841865707075e-05	\\
-6197.64737215909	5.69242263169051e-05	\\
-6196.66859019886	6.02561766333512e-05	\\
-6195.68980823864	5.96570479456135e-05	\\
-6194.71102627841	5.63896916965906e-05	\\
-6193.73224431818	5.59070399412661e-05	\\
-6192.75346235796	5.76220202250036e-05	\\
-6191.77468039773	5.81718603914053e-05	\\
-6190.7958984375	5.67414792098007e-05	\\
-6189.81711647727	5.82385398862302e-05	\\
-6188.83833451705	5.82025875937427e-05	\\
-6187.85955255682	5.70145131376368e-05	\\
-6186.88077059659	5.77467721153258e-05	\\
-6185.90198863636	5.78311409477675e-05	\\
-6184.92320667614	5.53747618438561e-05	\\
-6183.94442471591	5.83246466720935e-05	\\
-6182.96564275568	5.79928949172908e-05	\\
-6181.98686079546	5.84992764132729e-05	\\
-6181.00807883523	5.62089150937829e-05	\\
-6180.029296875	5.95663754082392e-05	\\
-6179.05051491477	5.87869295274843e-05	\\
-6178.07173295455	5.76755825636368e-05	\\
-6177.09295099432	5.68377357285437e-05	\\
-6176.11416903409	5.47044814437157e-05	\\
-6175.13538707386	5.72943605544067e-05	\\
-6174.15660511364	5.72984836635769e-05	\\
-6173.17782315341	5.62928953245607e-05	\\
-6172.19904119318	5.753519988136e-05	\\
-6171.22025923296	5.79470171408548e-05	\\
-6170.24147727273	5.59161015880906e-05	\\
-6169.2626953125	5.7587083248045e-05	\\
-6168.28391335227	5.6323611704291e-05	\\
-6167.30513139205	5.53998267860703e-05	\\
-6166.32634943182	5.51667854869031e-05	\\
-6165.34756747159	5.71297695996909e-05	\\
-6164.36878551136	5.76162052415429e-05	\\
-6163.39000355114	5.70264417721116e-05	\\
-6162.41122159091	5.61729972980475e-05	\\
-6161.43243963068	5.6922772883997e-05	\\
-6160.45365767046	5.64278590408159e-05	\\
-6159.47487571023	5.74269730492523e-05	\\
-6158.49609375	5.66567704298934e-05	\\
-6157.51731178977	5.58683703066842e-05	\\
-6156.53852982955	5.51571437285046e-05	\\
-6155.55974786932	5.63946867884191e-05	\\
-6154.58096590909	5.69449376740716e-05	\\
-6153.60218394886	5.50713628767112e-05	\\
-6152.62340198864	5.62276764392653e-05	\\
-6151.64462002841	5.54936874167462e-05	\\
-6150.66583806818	5.57573984049868e-05	\\
-6149.68705610796	5.59517409482929e-05	\\
-6148.70827414773	5.49676033934502e-05	\\
-6147.7294921875	5.53516828675426e-05	\\
-6146.75071022727	5.588721167902e-05	\\
-6145.77192826705	5.68851793810256e-05	\\
-6144.79314630682	5.30775469602899e-05	\\
-6143.81436434659	5.39081736959463e-05	\\
-6142.83558238636	5.56175512256403e-05	\\
-6141.85680042614	5.63891814296827e-05	\\
-6140.87801846591	5.49054976078372e-05	\\
-6139.89923650568	5.45993270985501e-05	\\
-6138.92045454546	5.63987127180871e-05	\\
-6137.94167258523	5.71038126036363e-05	\\
-6136.962890625	5.55414992470041e-05	\\
-6135.98410866477	5.50936036610464e-05	\\
-6135.00532670455	5.56326270199875e-05	\\
-6134.02654474432	5.77877796396765e-05	\\
-6133.04776278409	5.53317827496723e-05	\\
-6132.06898082386	5.5684666562008e-05	\\
-6131.09019886364	5.68691742652793e-05	\\
-6130.11141690341	5.60396098938404e-05	\\
-6129.13263494318	5.55717736595266e-05	\\
-6128.15385298296	5.622970593066e-05	\\
-6127.17507102273	5.61679747992524e-05	\\
-6126.1962890625	5.72338046171547e-05	\\
-6125.21750710227	5.7864536292163e-05	\\
-6124.23872514205	5.47507251986897e-05	\\
-6123.25994318182	5.75679111654077e-05	\\
-6122.28116122159	5.57649932413487e-05	\\
-6121.30237926136	5.47744109439061e-05	\\
-6120.32359730114	5.35304908702715e-05	\\
-6119.34481534091	5.62979108862483e-05	\\
-6118.36603338068	5.3799924968705e-05	\\
-6117.38725142046	5.38321300333664e-05	\\
-6116.40846946023	5.76233240297576e-05	\\
-6115.4296875	5.52180725449087e-05	\\
-6114.45090553977	5.45574737301039e-05	\\
-6113.47212357955	5.38343798615672e-05	\\
-6112.49334161932	5.57981026227144e-05	\\
-6111.51455965909	5.38439494264542e-05	\\
-6110.53577769886	5.6516927371672e-05	\\
-6109.55699573864	5.39935515535475e-05	\\
-6108.57821377841	5.49932676958569e-05	\\
-6107.59943181818	5.52324531548914e-05	\\
-6106.62064985796	5.4938981373725e-05	\\
-6105.64186789773	5.55663067281458e-05	\\
-6104.6630859375	5.71736970203214e-05	\\
-6103.68430397727	5.24896815268996e-05	\\
-6102.70552201705	5.45245998638557e-05	\\
-6101.72674005682	5.51315389002549e-05	\\
-6100.74795809659	5.36259378149283e-05	\\
-6099.76917613636	5.37481031223201e-05	\\
-6098.79039417614	5.37749063722546e-05	\\
-6097.81161221591	5.48623766279531e-05	\\
-6096.83283025568	5.53127644676894e-05	\\
-6095.85404829546	5.52107643852292e-05	\\
-6094.87526633523	5.31571966350921e-05	\\
-6093.896484375	5.40573823484536e-05	\\
-6092.91770241477	5.6023299985465e-05	\\
-6091.93892045455	5.46939497556591e-05	\\
-6090.96013849432	5.55566977465338e-05	\\
-6089.98135653409	5.35968955697534e-05	\\
-6089.00257457386	5.41200566198018e-05	\\
-6088.02379261364	5.50169850258233e-05	\\
-6087.04501065341	5.22289283631555e-05	\\
-6086.06622869318	5.42235864541978e-05	\\
-6085.08744673296	5.46507672478761e-05	\\
-6084.10866477273	5.47593430287814e-05	\\
-6083.1298828125	5.54714436079603e-05	\\
-6082.15110085227	5.41497603733168e-05	\\
-6081.17231889205	5.18428926753968e-05	\\
-6080.19353693182	5.45130921177011e-05	\\
-6079.21475497159	5.48576071811904e-05	\\
-6078.23597301136	5.4714500397574e-05	\\
-6077.25719105114	5.54295293343279e-05	\\
-6076.27840909091	5.33801492888289e-05	\\
-6075.29962713068	5.46640346975639e-05	\\
-6074.32084517046	5.61607547225818e-05	\\
-6073.34206321023	5.49209406125284e-05	\\
-6072.36328125	5.66426916876549e-05	\\
-6071.38449928977	5.34709178546808e-05	\\
-6070.40571732955	5.57547199155084e-05	\\
-6069.42693536932	5.47053123014576e-05	\\
-6068.44815340909	5.39519318686225e-05	\\
-6067.46937144886	5.34546507649591e-05	\\
-6066.49058948864	5.6858783250968e-05	\\
-6065.51180752841	5.43514360533867e-05	\\
-6064.53302556818	5.65844669997528e-05	\\
-6063.55424360796	5.57004007257924e-05	\\
-6062.57546164773	5.45940326524626e-05	\\
-6061.5966796875	5.5215650423887e-05	\\
-6060.61789772727	5.58947027774702e-05	\\
-6059.63911576705	5.6454912028718e-05	\\
-6058.66033380682	5.59869661015977e-05	\\
-6057.68155184659	5.53336932137791e-05	\\
-6056.70276988636	5.6048085634039e-05	\\
-6055.72398792614	5.62591058127698e-05	\\
-6054.74520596591	5.64111411417607e-05	\\
-6053.76642400568	5.69339823550778e-05	\\
-6052.78764204546	5.47406551927149e-05	\\
-6051.80886008523	5.62776258457804e-05	\\
-6050.830078125	5.5250900758408e-05	\\
-6049.85129616477	5.8150403107133e-05	\\
-6048.87251420455	5.66119733342787e-05	\\
-6047.89373224432	5.57119396590558e-05	\\
-6046.91495028409	5.73050686240248e-05	\\
-6045.93616832386	5.67976844663829e-05	\\
-6044.95738636364	5.68292581759093e-05	\\
-6043.97860440341	5.6680607631996e-05	\\
-6042.99982244318	5.70916111030895e-05	\\
-6042.02104048296	5.64725518953844e-05	\\
-6041.04225852273	5.58603042291151e-05	\\
-6040.0634765625	5.54457111307825e-05	\\
-6039.08469460227	5.65217340628233e-05	\\
-6038.10591264205	5.80739561565311e-05	\\
-6037.12713068182	5.87365622003131e-05	\\
-6036.14834872159	5.88821381479377e-05	\\
-6035.16956676136	5.75208422093864e-05	\\
-6034.19078480114	5.90821796660557e-05	\\
-6033.21200284091	5.80988404012183e-05	\\
-6032.23322088068	5.85152219493593e-05	\\
-6031.25443892046	5.81326321508436e-05	\\
-6030.27565696023	5.95849297231555e-05	\\
-6029.296875	5.75065513350399e-05	\\
-6028.31809303977	5.71513757041326e-05	\\
-6027.33931107955	5.91195914996591e-05	\\
-6026.36052911932	5.74786651519378e-05	\\
-6025.38174715909	5.85091591578471e-05	\\
-6024.40296519886	6.11182666308538e-05	\\
-6023.42418323864	5.93210738576833e-05	\\
-6022.44540127841	5.90570435753353e-05	\\
-6021.46661931818	6.02542558320374e-05	\\
-6020.48783735796	5.83391105988306e-05	\\
-6019.50905539773	5.95685429466559e-05	\\
-6018.5302734375	5.96683840752106e-05	\\
-6017.55149147727	5.82341769960358e-05	\\
-6016.57270951705	6.07685050308632e-05	\\
-6015.59392755682	5.77357974364337e-05	\\
-6014.61514559659	6.06331471530025e-05	\\
-6013.63636363636	5.93688220566664e-05	\\
-6012.65758167614	6.01299063872559e-05	\\
-6011.67879971591	5.98629380513953e-05	\\
-6010.70001775568	6.01411189636289e-05	\\
-6009.72123579546	6.03715940677951e-05	\\
-6008.74245383523	5.96338075764688e-05	\\
-6007.763671875	6.01231466088927e-05	\\
-6006.78488991477	5.95267733201565e-05	\\
-6005.80610795455	6.03375803321217e-05	\\
-6004.82732599432	5.84622728813506e-05	\\
-6003.84854403409	5.99481278691045e-05	\\
-6002.86976207386	6.05238371085807e-05	\\
-6001.89098011364	6.12840305361512e-05	\\
-6000.91219815341	5.94581169175092e-05	\\
-5999.93341619318	5.85929925606939e-05	\\
-5998.95463423296	5.75654031039531e-05	\\
-5997.97585227273	5.86823681486668e-05	\\
-5996.9970703125	5.87124758312411e-05	\\
-5996.01828835227	5.92257035952468e-05	\\
-5995.03950639205	5.95054640668586e-05	\\
-5994.06072443182	6.03536171613209e-05	\\
-5993.08194247159	6.08743850959692e-05	\\
-5992.10316051136	5.85211658982384e-05	\\
-5991.12437855114	5.86945186640803e-05	\\
-5990.14559659091	5.86825951356904e-05	\\
-5989.16681463068	5.84397410351316e-05	\\
-5988.18803267046	6.02025317601968e-05	\\
-5987.20925071023	5.87741919097929e-05	\\
-5986.23046875	6.05047515171938e-05	\\
-5985.25168678977	5.98755087308932e-05	\\
-5984.27290482955	6.10554181392819e-05	\\
-5983.29412286932	6.02236500772331e-05	\\
-5982.31534090909	5.98537864378191e-05	\\
-5981.33655894886	5.9394881004295e-05	\\
-5980.35777698864	5.62938208519332e-05	\\
-5979.37899502841	5.86523239241428e-05	\\
-5978.40021306818	5.74813771227778e-05	\\
-5977.42143110796	5.95861576780982e-05	\\
-5976.44264914773	5.82905335866653e-05	\\
-5975.4638671875	5.88055303235789e-05	\\
-5974.48508522727	5.91840701259097e-05	\\
-5973.50630326705	5.75179420336966e-05	\\
-5972.52752130682	5.78389100951427e-05	\\
-5971.54873934659	5.80203829622423e-05	\\
-5970.56995738636	5.60336536841358e-05	\\
-5969.59117542614	5.86922073950669e-05	\\
-5968.61239346591	5.85730502740915e-05	\\
-5967.63361150568	5.82176193300957e-05	\\
-5966.65482954546	5.95937712629629e-05	\\
-5965.67604758523	5.64864891716676e-05	\\
-5964.697265625	5.80021034449802e-05	\\
-5963.71848366477	5.98556746398701e-05	\\
-5962.73970170455	5.87181332398945e-05	\\
-5961.76091974432	5.87486287527098e-05	\\
-5960.78213778409	5.77181895046704e-05	\\
-5959.80335582386	5.96801787458388e-05	\\
-5958.82457386364	5.91589160665098e-05	\\
-5957.84579190341	5.82024584300571e-05	\\
-5956.86700994318	5.96103014325082e-05	\\
-5955.88822798296	5.87218493761564e-05	\\
-5954.90944602273	6.01108686825292e-05	\\
-5953.9306640625	5.89168282272376e-05	\\
-5952.95188210227	5.94573754834947e-05	\\
-5951.97310014205	5.88860224762315e-05	\\
-5950.99431818182	6.00065508715465e-05	\\
-5950.01553622159	5.96731752131525e-05	\\
-5949.03675426136	5.81459186409619e-05	\\
-5948.05797230114	5.85252566947338e-05	\\
-5947.07919034091	5.71673230529677e-05	\\
-5946.10040838068	5.89432340948475e-05	\\
-5945.12162642046	5.97090883561418e-05	\\
-5944.14284446023	5.61562319296421e-05	\\
-5943.1640625	5.74359009782383e-05	\\
-5942.18528053977	5.68394929946787e-05	\\
-5941.20649857955	5.9053641649525e-05	\\
-5940.22771661932	5.78137038907172e-05	\\
-5939.24893465909	5.97995195448916e-05	\\
-5938.27015269886	5.85296047921078e-05	\\
-5937.29137073864	5.93003500590813e-05	\\
-5936.31258877841	5.97486139478132e-05	\\
-5935.33380681818	6.02071308347579e-05	\\
-5934.35502485796	5.68934279187462e-05	\\
-5933.37624289773	5.87501233223759e-05	\\
-5932.3974609375	5.94805994978236e-05	\\
-5931.41867897727	5.86221543879334e-05	\\
-5930.43989701705	5.81706029987684e-05	\\
-5929.46111505682	5.78215606160984e-05	\\
-5928.48233309659	5.87660683052012e-05	\\
-5927.50355113636	6.04665503612385e-05	\\
-5926.52476917614	5.97542063437484e-05	\\
-5925.54598721591	5.85887722662543e-05	\\
-5924.56720525568	5.84209297988149e-05	\\
-5923.58842329546	5.77146004393412e-05	\\
-5922.60964133523	5.67782607552906e-05	\\
-5921.630859375	5.72464639365061e-05	\\
-5920.65207741477	5.71976703743129e-05	\\
-5919.67329545455	5.84372165362141e-05	\\
-5918.69451349432	5.85084538007806e-05	\\
-5917.71573153409	5.78895360364927e-05	\\
-5916.73694957386	5.770561364721e-05	\\
-5915.75816761364	5.80085375813748e-05	\\
-5914.77938565341	5.86047822646737e-05	\\
-5913.80060369318	5.90101825177734e-05	\\
-5912.82182173296	5.99945033626953e-05	\\
-5911.84303977273	6.00795293704865e-05	\\
-5910.8642578125	5.72976409698206e-05	\\
-5909.88547585227	5.65342502019014e-05	\\
-5908.90669389205	5.70966864117014e-05	\\
-5907.92791193182	5.6926596264446e-05	\\
-5906.94912997159	5.48807750361098e-05	\\
-5905.97034801136	5.69690289572768e-05	\\
-5904.99156605114	5.6827673959945e-05	\\
-5904.01278409091	5.7634289241462e-05	\\
-5903.03400213068	5.69617053444737e-05	\\
-5902.05522017046	5.62377429260517e-05	\\
-5901.07643821023	5.76102919315336e-05	\\
-5900.09765625	5.77329673705188e-05	\\
-5899.11887428977	5.59627271910652e-05	\\
-5898.14009232955	5.56379587418691e-05	\\
-5897.16131036932	5.44611550298991e-05	\\
-5896.18252840909	5.56329701254502e-05	\\
-5895.20374644886	5.47219372052183e-05	\\
-5894.22496448864	5.42632043005424e-05	\\
-5893.24618252841	5.59651154785503e-05	\\
-5892.26740056818	5.69474869061982e-05	\\
-5891.28861860796	5.39426975564004e-05	\\
-5890.30983664773	5.3455338269545e-05	\\
-5889.3310546875	5.36935514602913e-05	\\
-5888.35227272727	5.33637797160412e-05	\\
-5887.37349076705	5.31961923055543e-05	\\
-5886.39470880682	5.34583895713244e-05	\\
-5885.41592684659	5.51353226220401e-05	\\
-5884.43714488636	5.50420820880533e-05	\\
-5883.45836292614	5.26914535016508e-05	\\
-5882.47958096591	5.33719831066658e-05	\\
-5881.50079900568	5.29636782437591e-05	\\
-5880.52201704546	5.40532209560155e-05	\\
-5879.54323508523	5.46830774179892e-05	\\
-5878.564453125	5.06232316348428e-05	\\
-5877.58567116477	5.26404013231847e-05	\\
-5876.60688920455	5.26437045044948e-05	\\
-5875.62810724432	5.08314628214379e-05	\\
-5874.64932528409	5.24216519818594e-05	\\
-5873.67054332386	5.37453745488281e-05	\\
-5872.69176136364	5.11863492771192e-05	\\
-5871.71297940341	5.08565766410825e-05	\\
-5870.73419744318	5.12273071831302e-05	\\
-5869.75541548296	5.06941099620276e-05	\\
-5868.77663352273	5.24124179269137e-05	\\
-5867.7978515625	5.41814005446769e-05	\\
-5866.81906960227	5.22645035379325e-05	\\
-5865.84028764205	5.22863762889544e-05	\\
-5864.86150568182	5.00636360870354e-05	\\
-5863.88272372159	5.22898153593043e-05	\\
-5862.90394176136	5.22583022234989e-05	\\
-5861.92515980114	5.1017639646636e-05	\\
-5860.94637784091	5.02303704253396e-05	\\
-5859.96759588068	5.24461048454638e-05	\\
-5858.98881392046	5.35261016771053e-05	\\
-5858.01003196023	5.11080580520494e-05	\\
-5857.03125	5.11157437998995e-05	\\
-5856.05246803977	5.25234601751812e-05	\\
-5855.07368607955	5.2805832425403e-05	\\
-5854.09490411932	5.04390531996106e-05	\\
-5853.11612215909	4.99900616369076e-05	\\
-5852.13734019886	5.16630140791185e-05	\\
-5851.15855823864	5.10755678707407e-05	\\
-5850.17977627841	4.99075243306061e-05	\\
-5849.20099431818	5.18673499056584e-05	\\
-5848.22221235796	5.07100351213137e-05	\\
-5847.24343039773	5.13720872578597e-05	\\
-5846.2646484375	5.03278646947644e-05	\\
-5845.28586647727	5.0641033788731e-05	\\
-5844.30708451705	5.06739986617045e-05	\\
-5843.32830255682	5.06144294480151e-05	\\
-5842.34952059659	5.22160613987094e-05	\\
-5841.37073863636	5.05326558340157e-05	\\
-5840.39195667614	5.14119724649274e-05	\\
-5839.41317471591	4.9625151245247e-05	\\
-5838.43439275568	5.02571085850191e-05	\\
-5837.45561079546	5.30053015605105e-05	\\
-5836.47682883523	4.93319605440484e-05	\\
-5835.498046875	5.09547633076683e-05	\\
-5834.51926491477	5.07416076573979e-05	\\
-5833.54048295455	4.96019417600064e-05	\\
-5832.56170099432	4.9801910597789e-05	\\
-5831.58291903409	5.09072809724031e-05	\\
-5830.60413707386	4.97641698083844e-05	\\
-5829.62535511364	4.99817968391668e-05	\\
-5828.64657315341	4.97962887894551e-05	\\
-5827.66779119318	4.98194497136489e-05	\\
-5826.68900923296	4.91140510064577e-05	\\
-5825.71022727273	5.15620041678507e-05	\\
-5824.7314453125	5.0408059863681e-05	\\
-5823.75266335227	4.97251568045222e-05	\\
-5822.77388139205	4.86374093350349e-05	\\
-5821.79509943182	4.96231201033167e-05	\\
-5820.81631747159	4.95716005174327e-05	\\
-5819.83753551136	4.93623115984462e-05	\\
-5818.85875355114	5.07104048511391e-05	\\
-5817.87997159091	5.02111203368015e-05	\\
-5816.90118963068	5.05139772309526e-05	\\
-5815.92240767046	5.01999088091e-05	\\
-5814.94362571023	4.95947211867235e-05	\\
-5813.96484375	5.02235132042632e-05	\\
-5812.98606178977	4.94081015878684e-05	\\
-5812.00727982955	4.99370873518735e-05	\\
-5811.02849786932	4.99237878367598e-05	\\
-5810.04971590909	4.96933342318186e-05	\\
-5809.07093394886	4.87721490841949e-05	\\
-5808.09215198864	5.03829447208265e-05	\\
-5807.11337002841	4.80616247649043e-05	\\
-5806.13458806818	4.90674398912508e-05	\\
-5805.15580610796	5.01392696916571e-05	\\
-5804.17702414773	4.88660247981718e-05	\\
-5803.1982421875	4.95711370403547e-05	\\
-5802.21946022727	5.00137648481106e-05	\\
-5801.24067826705	5.02871870980296e-05	\\
-5800.26189630682	4.76271663474923e-05	\\
-5799.28311434659	4.98909281091283e-05	\\
-5798.30433238636	4.78305735034835e-05	\\
-5797.32555042614	4.7777522462503e-05	\\
-5796.34676846591	4.87915362011045e-05	\\
-5795.36798650568	4.98435331827063e-05	\\
-5794.38920454546	5.20424350171053e-05	\\
-5793.41042258523	4.96422440783016e-05	\\
-5792.431640625	4.8816735856732e-05	\\
-5791.45285866477	5.14732990379788e-05	\\
-5790.47407670455	4.99445718497006e-05	\\
-5789.49529474432	4.83062022902548e-05	\\
-5788.51651278409	4.90757902904676e-05	\\
-5787.53773082386	4.9096225316803e-05	\\
-5786.55894886364	4.83831378365961e-05	\\
-5785.58016690341	4.93009877977198e-05	\\
-5784.60138494318	5.02603075173937e-05	\\
-5783.62260298296	5.00175643117528e-05	\\
-5782.64382102273	4.97785603626568e-05	\\
-5781.6650390625	4.93787245305335e-05	\\
-5780.68625710227	4.87603047158454e-05	\\
-5779.70747514205	4.89209567342659e-05	\\
-5778.72869318182	4.78722558426694e-05	\\
-5777.74991122159	4.98285643252573e-05	\\
-5776.77112926136	4.82468022400899e-05	\\
-5775.79234730114	5.05354583448017e-05	\\
-5774.81356534091	5.00044748907402e-05	\\
-5773.83478338068	5.03051930406672e-05	\\
-5772.85600142046	4.77002091059149e-05	\\
-5771.87721946023	4.93268018313522e-05	\\
-5770.8984375	5.0978970327753e-05	\\
-5769.91965553977	5.05349957980867e-05	\\
-5768.94087357955	4.91908073774972e-05	\\
-5767.96209161932	5.09341277710116e-05	\\
-5766.98330965909	4.87022378236128e-05	\\
-5766.00452769886	4.76211753507392e-05	\\
-5765.02574573864	4.68688018407762e-05	\\
-5764.04696377841	4.7180378247122e-05	\\
-5763.06818181818	4.9408534112846e-05	\\
-5762.08939985796	4.79450285355443e-05	\\
-5761.11061789773	4.86082731994162e-05	\\
-5760.1318359375	4.98582437601227e-05	\\
-5759.15305397727	4.96982155270145e-05	\\
-5758.17427201705	4.69211342824051e-05	\\
-5757.19549005682	4.94698038767038e-05	\\
-5756.21670809659	5.04675159867375e-05	\\
-5755.23792613636	4.7429991326239e-05	\\
-5754.25914417614	4.82660453722059e-05	\\
-5753.28036221591	4.96405980219769e-05	\\
-5752.30158025568	4.71818424045762e-05	\\
-5751.32279829546	4.94957655728177e-05	\\
-5750.34401633523	4.91666491421451e-05	\\
-5749.365234375	4.81114750468661e-05	\\
-5748.38645241477	4.83568002375502e-05	\\
-5747.40767045455	5.08332721497225e-05	\\
-5746.42888849432	4.86091477821773e-05	\\
-5745.45010653409	4.92101898255e-05	\\
-5744.47132457386	4.99643028055478e-05	\\
-5743.49254261364	4.96188249558055e-05	\\
-5742.51376065341	5.05763279657206e-05	\\
-5741.53497869318	5.02679766782081e-05	\\
-5740.55619673296	4.78774385448513e-05	\\
-5739.57741477273	4.93027934171563e-05	\\
-5738.5986328125	4.83111300805005e-05	\\
-5737.61985085227	4.97747587237306e-05	\\
-5736.64106889205	4.90675947298407e-05	\\
-5735.66228693182	4.80869145645861e-05	\\
-5734.68350497159	4.89563022074392e-05	\\
-5733.70472301136	4.93778786092215e-05	\\
-5732.72594105114	4.91579927782711e-05	\\
-5731.74715909091	4.76710253095747e-05	\\
-5730.76837713068	4.86200018612477e-05	\\
-5729.78959517046	4.87046189580428e-05	\\
-5728.81081321023	4.85267242914736e-05	\\
-5727.83203125	4.9144712497257e-05	\\
-5726.85324928977	4.98068927097222e-05	\\
-5725.87446732955	5.04863327375189e-05	\\
-5724.89568536932	5.02811910901096e-05	\\
-5723.91690340909	4.84619905831093e-05	\\
-5722.93812144886	4.93963600684875e-05	\\
-5721.95933948864	4.8578688292294e-05	\\
-5720.98055752841	4.96972864599445e-05	\\
-5720.00177556818	4.92485420880498e-05	\\
-5719.02299360796	4.92310866082649e-05	\\
-5718.04421164773	4.81746013149588e-05	\\
-5717.0654296875	5.0633751225004e-05	\\
-5716.08664772727	5.08726115430404e-05	\\
-5715.10786576705	4.80519422049736e-05	\\
-5714.12908380682	5.0924646858758e-05	\\
-5713.15030184659	4.92967672294411e-05	\\
-5712.17151988636	5.00091886571901e-05	\\
-5711.19273792614	5.12367475925246e-05	\\
-5710.21395596591	4.83446802589968e-05	\\
-5709.23517400568	5.02133339146487e-05	\\
-5708.25639204546	4.97291723706722e-05	\\
-5707.27761008523	5.05548267939728e-05	\\
-5706.298828125	4.91701874548185e-05	\\
-5705.32004616477	4.89241519190716e-05	\\
-5704.34126420455	4.98416954242855e-05	\\
-5703.36248224432	4.99958271738374e-05	\\
-5702.38370028409	4.99983191289455e-05	\\
-5701.40491832386	4.87078005617582e-05	\\
-5700.42613636364	4.99065750945401e-05	\\
-5699.44735440341	4.95826412397917e-05	\\
-5698.46857244318	4.99785495853573e-05	\\
-5697.48979048296	5.17803992590394e-05	\\
-5696.51100852273	5.11231300353331e-05	\\
-5695.5322265625	5.15330239232729e-05	\\
-5694.55344460227	4.87105562488389e-05	\\
-5693.57466264205	5.15744128095362e-05	\\
-5692.59588068182	5.08873348004924e-05	\\
-5691.61709872159	5.17940939142298e-05	\\
-5690.63831676136	5.20088271946897e-05	\\
-5689.65953480114	4.92360458838219e-05	\\
-5688.68075284091	5.13968492057255e-05	\\
-5687.70197088068	5.1336449962094e-05	\\
-5686.72318892046	5.00091752511488e-05	\\
-5685.74440696023	5.13255103040126e-05	\\
-5684.765625	5.28837104027004e-05	\\
-5683.78684303977	5.18934536993252e-05	\\
-5682.80806107955	5.28625509212213e-05	\\
-5681.82927911932	5.20429496039942e-05	\\
-5680.85049715909	5.38226569977354e-05	\\
-5679.87171519886	5.07282069930974e-05	\\
-5678.89293323864	5.09891028333153e-05	\\
-5677.91415127841	5.20763721991311e-05	\\
-5676.93536931818	5.25310882230382e-05	\\
-5675.95658735796	5.27331935041052e-05	\\
-5674.97780539773	5.26796690917666e-05	\\
-5673.9990234375	5.05775871556287e-05	\\
-5673.02024147727	5.22740420352101e-05	\\
-5672.04145951705	5.33792693319786e-05	\\
-5671.06267755682	5.24369030204526e-05	\\
-5670.08389559659	5.39789251605422e-05	\\
-5669.10511363636	5.50325743660689e-05	\\
-5668.12633167614	5.32257330225859e-05	\\
-5667.14754971591	5.22351442531648e-05	\\
-5666.16876775568	5.23630385946406e-05	\\
-5665.18998579546	5.24449778586388e-05	\\
-5664.21120383523	5.29076263448577e-05	\\
-5663.232421875	5.3290748804022e-05	\\
-5662.25363991477	5.17902288124021e-05	\\
-5661.27485795455	5.21768752392858e-05	\\
-5660.29607599432	5.39406954040785e-05	\\
-5659.31729403409	5.6519516537919e-05	\\
-5658.33851207386	5.5929695601419e-05	\\
-5657.35973011364	5.55362645157189e-05	\\
-5656.38094815341	5.63638834727833e-05	\\
-5655.40216619318	5.52543919602603e-05	\\
-5654.42338423296	5.55630634499913e-05	\\
-5653.44460227273	5.50359078060181e-05	\\
-5652.4658203125	5.70750967499782e-05	\\
-5651.48703835227	5.68372635078954e-05	\\
-5650.50825639205	5.53482552567782e-05	\\
-5649.52947443182	5.59294338701913e-05	\\
-5648.55069247159	5.82892902685186e-05	\\
-5647.57191051136	5.68879883733036e-05	\\
-5646.59312855114	5.70514557783006e-05	\\
-5645.61434659091	5.74066638671878e-05	\\
-5644.63556463068	5.8391499149508e-05	\\
-5643.65678267046	5.65505389587017e-05	\\
-5642.67800071023	5.64538282580106e-05	\\
-5641.69921875	5.54486626288024e-05	\\
-5640.72043678977	5.95859059471719e-05	\\
-5639.74165482955	5.93067444794979e-05	\\
-5638.76287286932	5.92756215018956e-05	\\
-5637.78409090909	5.80855119239599e-05	\\
-5636.80530894886	6.17728892488082e-05	\\
-5635.82652698864	6.10289412834487e-05	\\
-5634.84774502841	5.76463216087029e-05	\\
-5633.86896306818	5.8702472895028e-05	\\
-5632.89018110796	5.87464050854249e-05	\\
-5631.91139914773	6.03019222413647e-05	\\
-5630.9326171875	5.93265764499492e-05	\\
-5629.95383522727	5.88724603982252e-05	\\
-5628.97505326705	6.09490514163469e-05	\\
-5627.99627130682	6.07795154788417e-05	\\
-5627.01748934659	6.14229509148741e-05	\\
-5626.03870738636	6.16123950471532e-05	\\
-5625.05992542614	6.1548935699375e-05	\\
-5624.08114346591	6.25749983218771e-05	\\
-5623.10236150568	6.04884372467383e-05	\\
-5622.12357954546	6.08833897343648e-05	\\
-5621.14479758523	6.12483804875895e-05	\\
-5620.166015625	6.1638762426045e-05	\\
-5619.18723366477	6.26376327910127e-05	\\
-5618.20845170455	6.27725612589199e-05	\\
-5617.22966974432	5.99863297624169e-05	\\
-5616.25088778409	6.46476432677241e-05	\\
-5615.27210582386	6.39784419167007e-05	\\
-5614.29332386364	6.33690308102483e-05	\\
-5613.31454190341	6.06748898973808e-05	\\
-5612.33575994318	6.25547668579342e-05	\\
-5611.35697798296	6.19133944699973e-05	\\
-5610.37819602273	6.25051974727973e-05	\\
-5609.3994140625	6.54955025312251e-05	\\
-5608.42063210227	6.39031055504034e-05	\\
-5607.44185014205	6.36396890008417e-05	\\
-5606.46306818182	6.17527183681077e-05	\\
-5605.48428622159	6.36252459821956e-05	\\
-5604.50550426136	6.28089641921939e-05	\\
-5603.52672230114	6.55630059395112e-05	\\
-5602.54794034091	6.33778178441062e-05	\\
-5601.56915838068	6.25068585842659e-05	\\
-5600.59037642046	6.39324075579085e-05	\\
-5599.61159446023	6.1727290725727e-05	\\
-5598.6328125	6.47087455300809e-05	\\
-5597.65403053977	6.22672240709818e-05	\\
-5596.67524857955	6.28865639370312e-05	\\
-5595.69646661932	6.50144563036687e-05	\\
-5594.71768465909	6.54723219303e-05	\\
-5593.73890269886	6.4013745427163e-05	\\
-5592.76012073864	6.27632029022033e-05	\\
-5591.78133877841	6.53687949631677e-05	\\
-5590.80255681818	6.56643968158014e-05	\\
-5589.82377485796	6.3496801577264e-05	\\
-5588.84499289773	6.38622759017077e-05	\\
-5587.8662109375	6.37689315836795e-05	\\
-5586.88742897727	6.35246841069782e-05	\\
-5585.90864701705	6.27566836806606e-05	\\
-5584.92986505682	6.50010185133791e-05	\\
-5583.95108309659	6.44325091534392e-05	\\
-5582.97230113636	6.46471087326052e-05	\\
-5581.99351917614	6.26853545813065e-05	\\
-5581.01473721591	6.56469170449638e-05	\\
-5580.03595525568	6.61818778168308e-05	\\
-5579.05717329546	6.39114660557346e-05	\\
-5578.07839133523	6.42619806662164e-05	\\
-5577.099609375	6.54196296914935e-05	\\
-5576.12082741477	6.47118043073433e-05	\\
-5575.14204545455	6.56745072635152e-05	\\
-5574.16326349432	6.37507499000255e-05	\\
-5573.18448153409	6.46939947647293e-05	\\
-5572.20569957386	6.53179276968339e-05	\\
-5571.22691761364	6.4511390397501e-05	\\
-5570.24813565341	6.42168588442644e-05	\\
-5569.26935369318	6.64300573165337e-05	\\
-5568.29057173296	6.43383152299567e-05	\\
-5567.31178977273	6.48919700841528e-05	\\
-5566.3330078125	6.50865830503824e-05	\\
-5565.35422585227	6.35877042793057e-05	\\
-5564.37544389205	6.53904358123699e-05	\\
-5563.39666193182	6.44216204690922e-05	\\
-5562.41787997159	6.31010046231562e-05	\\
-5561.43909801136	6.60504432066944e-05	\\
-5560.46031605114	6.60325221724359e-05	\\
-5559.48153409091	6.51339363818667e-05	\\
-5558.50275213068	6.60513751837022e-05	\\
-5557.52397017046	6.31905359119002e-05	\\
-5556.54518821023	6.50284173176362e-05	\\
-5555.56640625	6.47703336286412e-05	\\
-5554.58762428977	6.52189407777089e-05	\\
-5553.60884232955	6.69821983154008e-05	\\
-5552.63006036932	6.47950971747897e-05	\\
-5551.65127840909	6.4798635113798e-05	\\
-5550.67249644886	6.68849630463755e-05	\\
-5549.69371448864	6.32148403222804e-05	\\
-5548.71493252841	6.56325220489836e-05	\\
-5547.73615056818	6.60984234555894e-05	\\
-5546.75736860796	6.21984828994968e-05	\\
-5545.77858664773	6.36978259650042e-05	\\
-5544.7998046875	6.34846003669554e-05	\\
-5543.82102272727	6.63347020772887e-05	\\
-5542.84224076705	6.51266116765778e-05	\\
-5541.86345880682	6.26043790982363e-05	\\
-5540.88467684659	6.60733192978836e-05	\\
-5539.90589488636	6.44850720268169e-05	\\
-5538.92711292614	6.53973008233022e-05	\\
-5537.94833096591	6.51082186237837e-05	\\
-5536.96954900568	6.46727426201753e-05	\\
-5535.99076704546	6.5271194575428e-05	\\
-5535.01198508523	6.59616881534281e-05	\\
-5534.033203125	6.49150877114918e-05	\\
-5533.05442116477	6.45794488276926e-05	\\
-5532.07563920455	6.449343138248e-05	\\
-5531.09685724432	6.45965763866298e-05	\\
-5530.11807528409	6.53212594903629e-05	\\
-5529.13929332386	6.45795039954713e-05	\\
-5528.16051136364	6.38737603287548e-05	\\
-5527.18172940341	6.45144402723235e-05	\\
-5526.20294744318	6.55330191734436e-05	\\
-5525.22416548296	6.37633442080358e-05	\\
-5524.24538352273	6.55133771287496e-05	\\
-5523.2666015625	6.35113209086659e-05	\\
-5522.28781960227	6.36467363879904e-05	\\
-5521.30903764205	6.76887353253367e-05	\\
-5520.33025568182	6.34848041478597e-05	\\
-5519.35147372159	6.46990914534497e-05	\\
-5518.37269176136	6.65123720337991e-05	\\
-5517.39390980114	6.65778767103104e-05	\\
-5516.41512784091	6.56526163471513e-05	\\
-5515.43634588068	6.74530628539589e-05	\\
-5514.45756392046	6.67022171442779e-05	\\
-5513.47878196023	6.50606951201793e-05	\\
-5512.5	6.67882552623449e-05	\\
-5511.52121803977	6.57824020767425e-05	\\
-5510.54243607955	6.81967017977413e-05	\\
-5509.56365411932	6.48424576640756e-05	\\
-5508.58487215909	6.49017953643344e-05	\\
-5507.60609019886	6.42003466623954e-05	\\
-5506.62730823864	6.64386826645404e-05	\\
-5505.64852627841	6.57695865900359e-05	\\
-5504.66974431818	6.44961312845109e-05	\\
-5503.69096235796	6.66297293933721e-05	\\
-5502.71218039773	6.73704242280911e-05	\\
-5501.7333984375	6.58240530841034e-05	\\
-5500.75461647727	6.67891492308351e-05	\\
-5499.77583451705	6.61220843507542e-05	\\
-5498.79705255682	6.51753623760262e-05	\\
-5497.81827059659	6.55846228251005e-05	\\
-5496.83948863636	6.82176966539892e-05	\\
-5495.86070667614	6.59009355120358e-05	\\
-5494.88192471591	6.88172567857569e-05	\\
-5493.90314275568	6.60820556751947e-05	\\
-5492.92436079546	6.54954935537386e-05	\\
-5491.94557883523	6.72557028508052e-05	\\
-5490.966796875	6.74371703124316e-05	\\
-5489.98801491477	6.62997334754534e-05	\\
-5489.00923295455	6.72636221280854e-05	\\
-5488.03045099432	6.79692568239309e-05	\\
-5487.05166903409	6.64659065242782e-05	\\
-5486.07288707386	6.77549075230653e-05	\\
-5485.09410511364	6.55796860766188e-05	\\
-5484.11532315341	6.67038344986815e-05	\\
-5483.13654119318	6.56578500011268e-05	\\
-5482.15775923296	6.54984241242807e-05	\\
-5481.17897727273	6.5090280047082e-05	\\
-5480.2001953125	6.56953847558406e-05	\\
-5479.22141335227	6.58611279384982e-05	\\
-5478.24263139205	6.68111839185268e-05	\\
-5477.26384943182	6.64692002856448e-05	\\
-5476.28506747159	6.81041744581277e-05	\\
-5475.30628551136	6.53870303484485e-05	\\
-5474.32750355114	6.60869012697492e-05	\\
-5473.34872159091	6.56688506888947e-05	\\
-5472.36993963068	6.82875947768459e-05	\\
-5471.39115767046	6.78702957670816e-05	\\
-5470.41237571023	6.55699882840713e-05	\\
-5469.43359375	6.73099729376873e-05	\\
-5468.45481178977	6.60498660248598e-05	\\
-5467.47602982955	6.69961102358658e-05	\\
-5466.49724786932	6.57457365673815e-05	\\
-5465.51846590909	6.71034667645579e-05	\\
-5464.53968394886	6.70859595584766e-05	\\
-5463.56090198864	6.51552472807825e-05	\\
-5462.58212002841	6.47249478027582e-05	\\
-5461.60333806818	6.48548794999758e-05	\\
-5460.62455610796	6.59738491439544e-05	\\
-5459.64577414773	6.54770671463618e-05	\\
-5458.6669921875	6.64991591546514e-05	\\
-5457.68821022727	6.51062136000444e-05	\\
-5456.70942826705	6.52709243174514e-05	\\
-5455.73064630682	6.70715370378901e-05	\\
-5454.75186434659	6.35880761405265e-05	\\
-5453.77308238636	6.58466191820112e-05	\\
-5452.79430042614	6.66644581600943e-05	\\
-5451.81551846591	6.3155342322306e-05	\\
-5450.83673650568	6.6122431253819e-05	\\
-5449.85795454546	6.70470096055127e-05	\\
-5448.87917258523	6.45692965718088e-05	\\
-5447.900390625	6.65610338147295e-05	\\
-5446.92160866477	6.49863158202361e-05	\\
-5445.94282670455	6.48869887860093e-05	\\
-5444.96404474432	6.56047866625754e-05	\\
-5443.98526278409	6.66454515842681e-05	\\
-5443.00648082386	6.40201118106671e-05	\\
-5442.02769886364	6.49956925619341e-05	\\
-5441.04891690341	6.34234768822779e-05	\\
-5440.07013494318	6.51239046735826e-05	\\
-5439.09135298296	6.60794328427259e-05	\\
-5438.11257102273	6.70004304750837e-05	\\
-5437.1337890625	6.54707094473225e-05	\\
-5436.15500710227	6.62532107128989e-05	\\
-5435.17622514205	6.553886610868e-05	\\
-5434.19744318182	6.79359774165831e-05	\\
-5433.21866122159	6.72716022631534e-05	\\
-5432.23987926136	6.34526763846448e-05	\\
-5431.26109730114	6.51175683053625e-05	\\
-5430.28231534091	6.3896605917948e-05	\\
-5429.30353338068	6.59261196867936e-05	\\
-5428.32475142046	6.47109540381289e-05	\\
-5427.34596946023	6.48762889824415e-05	\\
-5426.3671875	6.48290808673266e-05	\\
-5425.38840553977	6.45857221144146e-05	\\
-5424.40962357955	6.55843126380105e-05	\\
-5423.43084161932	6.35357496224498e-05	\\
-5422.45205965909	6.49391194087195e-05	\\
-5421.47327769886	6.1449537542127e-05	\\
-5420.49449573864	6.52728576397475e-05	\\
-5419.51571377841	6.7219144388039e-05	\\
-5418.53693181818	6.47267768465647e-05	\\
-5417.55814985796	6.52682942239722e-05	\\
-5416.57936789773	6.47215242551991e-05	\\
-5415.6005859375	6.49238682786229e-05	\\
-5414.62180397727	6.5555878460957e-05	\\
-5413.64302201705	6.47829767175874e-05	\\
-5412.66424005682	6.67033668336073e-05	\\
-5411.68545809659	6.64693557415701e-05	\\
-5410.70667613636	6.36592601599987e-05	\\
-5409.72789417614	6.4729906547358e-05	\\
-5408.74911221591	6.43655878177779e-05	\\
-5407.77033025568	6.43968282386754e-05	\\
-5406.79154829546	6.51591388323512e-05	\\
-5405.81276633523	6.50106467768975e-05	\\
-5404.833984375	6.5713553685868e-05	\\
-5403.85520241477	6.77788813021864e-05	\\
-5402.87642045455	6.62101353368577e-05	\\
-5401.89763849432	6.49025723244217e-05	\\
-5400.91885653409	6.26774393868102e-05	\\
-5399.94007457386	6.57544996474664e-05	\\
-5398.96129261364	6.43985187904025e-05	\\
-5397.98251065341	6.46857677223755e-05	\\
-5397.00372869318	6.21654992888738e-05	\\
-5396.02494673296	6.64446249133863e-05	\\
-5395.04616477273	6.57532807878544e-05	\\
-5394.0673828125	6.42645740364008e-05	\\
-5393.08860085227	6.31973500306422e-05	\\
-5392.10981889205	6.51184161201822e-05	\\
-5391.13103693182	6.44095231533481e-05	\\
-5390.15225497159	6.15449616196622e-05	\\
-5389.17347301136	6.37446704170228e-05	\\
-5388.19469105114	6.39149037705173e-05	\\
-5387.21590909091	6.37432702666914e-05	\\
-5386.23712713068	6.19285938323024e-05	\\
-5385.25834517046	6.2203442320107e-05	\\
-5384.27956321023	6.38236947910805e-05	\\
-5383.30078125	6.33020758033735e-05	\\
-5382.32199928977	6.27763385887663e-05	\\
-5381.34321732955	6.34538368472101e-05	\\
-5380.36443536932	6.39732570620573e-05	\\
-5379.38565340909	6.11098892561845e-05	\\
-5378.40687144886	6.16075106789187e-05	\\
-5377.42808948864	6.23512340804849e-05	\\
-5376.44930752841	6.18592428767752e-05	\\
-5375.47052556818	6.15582006476032e-05	\\
-5374.49174360796	6.21038117417135e-05	\\
-5373.51296164773	6.00610492764104e-05	\\
-5372.5341796875	5.92447341875952e-05	\\
-5371.55539772727	6.17568921420625e-05	\\
-5370.57661576705	6.21253060116305e-05	\\
-5369.59783380682	5.94341261001749e-05	\\
-5368.61905184659	5.96313274614101e-05	\\
-5367.64026988636	6.07018708117278e-05	\\
-5366.66148792614	5.9685884004045e-05	\\
-5365.68270596591	6.04946958589049e-05	\\
-5364.70392400568	5.83801361325197e-05	\\
-5363.72514204546	6.0337899218688e-05	\\
-5362.74636008523	5.93012092598925e-05	\\
-5361.767578125	6.02607222318471e-05	\\
-5360.78879616477	6.02036763883172e-05	\\
-5359.81001420455	6.11458540391279e-05	\\
-5358.83123224432	6.05739372337364e-05	\\
-5357.85245028409	6.03688533645611e-05	\\
-5356.87366832386	5.93312911469419e-05	\\
-5355.89488636364	5.94551172430266e-05	\\
-5354.91610440341	5.96168847987024e-05	\\
-5353.93732244318	5.9063605097025e-05	\\
-5352.95854048296	5.90279724026033e-05	\\
-5351.97975852273	5.99204503622984e-05	\\
-5351.0009765625	5.96677260585245e-05	\\
-5350.02219460227	5.82722973400328e-05	\\
-5349.04341264205	5.96542379122619e-05	\\
-5348.06463068182	5.66453045866354e-05	\\
-5347.08584872159	5.92291198394207e-05	\\
-5346.10706676136	5.5588929654589e-05	\\
-5345.12828480114	6.006760890469e-05	\\
-5344.14950284091	5.65491336742966e-05	\\
-5343.17072088068	5.80034924675653e-05	\\
-5342.19193892046	5.67433263617911e-05	\\
-5341.21315696023	5.79600740962826e-05	\\
-5340.234375	5.8033446579256e-05	\\
-5339.25559303977	5.57440408499534e-05	\\
-5338.27681107955	5.6470769024277e-05	\\
-5337.29802911932	5.54668237995638e-05	\\
-5336.31924715909	5.60512306937357e-05	\\
-5335.34046519886	5.5522124928328e-05	\\
-5334.36168323864	5.56041922846223e-05	\\
-5333.38290127841	5.78168455584768e-05	\\
-5332.40411931818	5.77555220237832e-05	\\
-5331.42533735796	5.59171403359776e-05	\\
-5330.44655539773	5.44049629820063e-05	\\
-5329.4677734375	5.48263760925791e-05	\\
-5328.48899147727	5.51667468384621e-05	\\
-5327.51020951705	5.4452015362339e-05	\\
-5326.53142755682	5.58297249397828e-05	\\
-5325.55264559659	5.68217051022351e-05	\\
-5324.57386363636	5.67639547708366e-05	\\
-5323.59508167614	5.54473328938235e-05	\\
-5322.61629971591	5.73657255302554e-05	\\
-5321.63751775568	5.54741883680172e-05	\\
-5320.65873579546	5.72022619172266e-05	\\
-5319.67995383523	5.76150926312913e-05	\\
-5318.701171875	5.74233898674018e-05	\\
-5317.72238991477	5.75206069792549e-05	\\
-5316.74360795455	5.511528418135e-05	\\
-5315.76482599432	5.82755572229981e-05	\\
-5314.78604403409	5.70045108375046e-05	\\
-5313.80726207386	5.64875897053947e-05	\\
-5312.82848011364	5.64537048303645e-05	\\
-5311.84969815341	5.70004452072797e-05	\\
-5310.87091619318	5.48122485674648e-05	\\
-5309.89213423296	5.83067327624458e-05	\\
-5308.91335227273	5.62351279803721e-05	\\
-5307.9345703125	5.57742235946883e-05	\\
-5306.95578835227	5.68951541092358e-05	\\
-5305.97700639205	5.63052233256192e-05	\\
-5304.99822443182	5.57073335960959e-05	\\
-5304.01944247159	5.53302061172256e-05	\\
-5303.04066051136	5.63516565085448e-05	\\
-5302.06187855114	5.72422184472589e-05	\\
-5301.08309659091	5.64400920128335e-05	\\
-5300.10431463068	5.82100208353497e-05	\\
-5299.12553267046	5.72718410647961e-05	\\
-5298.14675071023	5.91948475646525e-05	\\
-5297.16796875	5.84505242312324e-05	\\
-5296.18918678977	5.57218293426929e-05	\\
-5295.21040482955	5.59564745142666e-05	\\
-5294.23162286932	5.65745186006823e-05	\\
-5293.25284090909	5.66881003796128e-05	\\
-5292.27405894886	5.60040245224148e-05	\\
-5291.29527698864	5.84210804359084e-05	\\
-5290.31649502841	5.68815691366502e-05	\\
-5289.33771306818	5.73613272948625e-05	\\
-5288.35893110796	5.86200188913897e-05	\\
-5287.38014914773	5.60937546287535e-05	\\
-5286.4013671875	5.56199244644793e-05	\\
-5285.42258522727	5.66700083214176e-05	\\
-5284.44380326705	5.58750268809662e-05	\\
-5283.46502130682	5.75793623095693e-05	\\
-5282.48623934659	5.71038154688647e-05	\\
-5281.50745738636	5.67187333367329e-05	\\
-5280.52867542614	5.65033724725301e-05	\\
-5279.54989346591	5.35339989212566e-05	\\
-5278.57111150568	5.55791917258969e-05	\\
-5277.59232954546	5.61062359142788e-05	\\
-5276.61354758523	5.47702756522107e-05	\\
-5275.634765625	5.67897772979018e-05	\\
-5274.65598366477	5.6752211302182e-05	\\
-5273.67720170455	5.38564886700901e-05	\\
-5272.69841974432	5.68416046036455e-05	\\
-5271.71963778409	5.64526952781129e-05	\\
-5270.74085582386	5.71222194101584e-05	\\
-5269.76207386364	5.56356406223936e-05	\\
-5268.78329190341	5.78903561207645e-05	\\
-5267.80450994318	5.58056202938881e-05	\\
-5266.82572798296	5.58467748396926e-05	\\
-5265.84694602273	5.63950805371271e-05	\\
-5264.8681640625	5.61489936663377e-05	\\
-5263.88938210227	5.51747889040352e-05	\\
-5262.91060014205	5.34807347875994e-05	\\
-5261.93181818182	5.7191421714728e-05	\\
-5260.95303622159	5.60149287471894e-05	\\
-5259.97425426136	5.52233948441532e-05	\\
-5258.99547230114	5.53503175651273e-05	\\
-5258.01669034091	5.51854494052433e-05	\\
-5257.03790838068	5.59062574647557e-05	\\
-5256.05912642046	5.50372935508735e-05	\\
-5255.08034446023	5.54805365782645e-05	\\
-5254.1015625	5.44023434959282e-05	\\
-5253.12278053977	5.55359592357566e-05	\\
-5252.14399857955	5.46962013380265e-05	\\
-5251.16521661932	5.6265407728073e-05	\\
-5250.18643465909	5.43879920911397e-05	\\
-5249.20765269886	5.39929662314225e-05	\\
-5248.22887073864	5.53146551495463e-05	\\
-5247.25008877841	5.3570926305289e-05	\\
-5246.27130681818	5.17965748259646e-05	\\
-5245.29252485796	5.79463381750412e-05	\\
-5244.31374289773	5.67404263145612e-05	\\
-5243.3349609375	5.57890587384969e-05	\\
-5242.35617897727	5.36225319239925e-05	\\
-5241.37739701705	5.63564205763396e-05	\\
-5240.39861505682	5.78825744274046e-05	\\
-5239.41983309659	5.77881848791128e-05	\\
-5238.44105113636	5.39987668877492e-05	\\
-5237.46226917614	5.56087229005971e-05	\\
-5236.48348721591	5.29231324167363e-05	\\
-5235.50470525568	5.56083969676716e-05	\\
-5234.52592329546	5.46500387884085e-05	\\
-5233.54714133523	5.42053438249458e-05	\\
-5232.568359375	5.47736171783217e-05	\\
-5231.58957741477	5.30322830288979e-05	\\
-5230.61079545455	5.62685152953714e-05	\\
-5229.63201349432	5.08153574704244e-05	\\
-5228.65323153409	5.64862863670232e-05	\\
-5227.67444957386	5.62484432481427e-05	\\
-5226.69566761364	5.50380020346437e-05	\\
-5225.71688565341	5.7004238263361e-05	\\
-5224.73810369318	5.39421619295391e-05	\\
-5223.75932173296	5.27658527679075e-05	\\
-5222.78053977273	5.47994424272714e-05	\\
-5221.8017578125	5.55082072409138e-05	\\
-5220.82297585227	5.3763091733193e-05	\\
-5219.84419389205	5.39333460850552e-05	\\
-5218.86541193182	5.37367668995769e-05	\\
-5217.88662997159	5.68617433392265e-05	\\
-5216.90784801136	5.32296996321157e-05	\\
-5215.92906605114	5.44201351422966e-05	\\
-5214.95028409091	5.58103660812471e-05	\\
-5213.97150213068	5.43005797467797e-05	\\
-5212.99272017046	5.43852129764274e-05	\\
-5212.01393821023	5.56854309619708e-05	\\
-5211.03515625	5.4245713184577e-05	\\
-5210.05637428977	5.40978621776216e-05	\\
-5209.07759232955	5.50819765078701e-05	\\
-5208.09881036932	5.30383990843076e-05	\\
-5207.12002840909	5.52474503421326e-05	\\
-5206.14124644886	5.38097741666343e-05	\\
-5205.16246448864	5.64989297809969e-05	\\
-5204.18368252841	5.36373106902326e-05	\\
-5203.20490056818	5.5262442742834e-05	\\
-5202.22611860796	5.61425762060647e-05	\\
-5201.24733664773	5.40742869419641e-05	\\
-5200.2685546875	5.77450288194445e-05	\\
-5199.28977272727	5.58576794140041e-05	\\
-5198.31099076705	5.25576598130091e-05	\\
-5197.33220880682	5.57762849931662e-05	\\
-5196.35342684659	5.48508727919095e-05	\\
-5195.37464488636	5.45591002103115e-05	\\
-5194.39586292614	5.91159657452906e-05	\\
-5193.41708096591	5.60549197482621e-05	\\
-5192.43829900568	5.60456785886758e-05	\\
-5191.45951704546	5.64535220778731e-05	\\
-5190.48073508523	5.78955888559521e-05	\\
-5189.501953125	5.38290590995321e-05	\\
-5188.52317116477	5.58245235676695e-05	\\
-5187.54438920455	5.68220791271422e-05	\\
-5186.56560724432	5.68483457375542e-05	\\
-5185.58682528409	5.48395850065181e-05	\\
-5184.60804332386	5.83540926818072e-05	\\
-5183.62926136364	5.49761731936049e-05	\\
-5182.65047940341	5.66830136892169e-05	\\
-5181.67169744318	5.72745646085491e-05	\\
-5180.69291548296	5.57544708276394e-05	\\
-5179.71413352273	5.62541013702681e-05	\\
-5178.7353515625	5.4491225021002e-05	\\
-5177.75656960227	5.61883525664842e-05	\\
-5176.77778764205	5.60266197671426e-05	\\
-5175.79900568182	5.67923610730291e-05	\\
-5174.82022372159	5.70672741748261e-05	\\
-5173.84144176136	5.60004101706574e-05	\\
-5172.86265980114	5.7073025071513e-05	\\
-5171.88387784091	5.66196656020797e-05	\\
-5170.90509588068	5.79831887592263e-05	\\
-5169.92631392046	5.7814304975893e-05	\\
-5168.94753196023	5.75660286571105e-05	\\
-5167.96875	5.73744769685088e-05	\\
-5166.98996803977	5.35901887192926e-05	\\
-5166.01118607955	5.51256409458343e-05	\\
-5165.03240411932	5.69492087755178e-05	\\
-5164.05362215909	5.48292322554963e-05	\\
-5163.07484019886	5.68095096532955e-05	\\
-5162.09605823864	5.66452071367769e-05	\\
-5161.11727627841	5.72567149234452e-05	\\
-5160.13849431818	5.77255900581572e-05	\\
-5159.15971235796	5.63988697359746e-05	\\
-5158.18093039773	5.58276472896482e-05	\\
-5157.2021484375	5.75463748291925e-05	\\
-5156.22336647727	5.68591242682313e-05	\\
-5155.24458451705	5.47586557396838e-05	\\
-5154.26580255682	5.74927796803874e-05	\\
-5153.28702059659	5.74160850465259e-05	\\
-5152.30823863636	5.65173814358848e-05	\\
-5151.32945667614	5.65652851931224e-05	\\
-5150.35067471591	5.80736725088884e-05	\\
-5149.37189275568	5.69020783062237e-05	\\
-5148.39311079546	5.70413226145639e-05	\\
-5147.41432883523	5.42004090598977e-05	\\
-5146.435546875	5.76053405540489e-05	\\
-5145.45676491477	5.52376690489737e-05	\\
-5144.47798295455	5.51326356297007e-05	\\
-5143.49920099432	5.52374715169696e-05	\\
-5142.52041903409	5.63030429775302e-05	\\
-5141.54163707386	5.53096766804591e-05	\\
-5140.56285511364	5.60708680083198e-05	\\
-5139.58407315341	5.48303886038496e-05	\\
-5138.60529119318	5.5736412591528e-05	\\
-5137.62650923296	5.32525762098152e-05	\\
-5136.64772727273	5.54178842810079e-05	\\
-5135.6689453125	5.38873752717798e-05	\\
-5134.69016335227	5.44651820045172e-05	\\
-5133.71138139205	5.3254531193001e-05	\\
-5132.73259943182	5.61105152026045e-05	\\
-5131.75381747159	5.48376064374848e-05	\\
-5130.77503551136	5.43874357522408e-05	\\
-5129.79625355114	5.41439921540652e-05	\\
-5128.81747159091	5.40148765835683e-05	\\
-5127.83868963068	5.28510052296075e-05	\\
-5126.85990767046	5.40418518988716e-05	\\
-5125.88112571023	5.59519870394027e-05	\\
-5124.90234375	5.4926389004012e-05	\\
-5123.92356178977	5.29567329103578e-05	\\
-5122.94477982955	5.50459771615942e-05	\\
-5121.96599786932	5.21191605578924e-05	\\
-5120.98721590909	5.27618936920513e-05	\\
-5120.00843394886	5.48947377907925e-05	\\
-5119.02965198864	5.39561699063822e-05	\\
-5118.05087002841	5.29478126263979e-05	\\
-5117.07208806818	5.4732601980603e-05	\\
-5116.09330610796	5.20539188481028e-05	\\
-5115.11452414773	5.18610283269982e-05	\\
-5114.1357421875	5.43451372508701e-05	\\
-5113.15696022727	5.31746302371443e-05	\\
-5112.17817826705	5.2947790586221e-05	\\
-5111.19939630682	5.2858481186176e-05	\\
-5110.22061434659	5.35168050007217e-05	\\
-5109.24183238636	5.12201197939274e-05	\\
-5108.26305042614	5.41762552473673e-05	\\
-5107.28426846591	5.25702970492603e-05	\\
-5106.30548650568	5.32270246623484e-05	\\
-5105.32670454546	5.19752010167626e-05	\\
-5104.34792258523	5.33361588782601e-05	\\
-5103.369140625	5.02452420128644e-05	\\
-5102.39035866477	5.21234481801583e-05	\\
-5101.41157670455	5.35433061966478e-05	\\
-5100.43279474432	5.08189708616521e-05	\\
-5099.45401278409	5.16719519218189e-05	\\
-5098.47523082386	5.15375851518701e-05	\\
-5097.49644886364	5.21571031113046e-05	\\
-5096.51766690341	5.09125634319735e-05	\\
-5095.53888494318	5.31077604520743e-05	\\
-5094.56010298296	4.99805905085762e-05	\\
-5093.58132102273	5.09895804965996e-05	\\
-5092.6025390625	5.12857227250989e-05	\\
-5091.62375710227	5.19357116853503e-05	\\
-5090.64497514205	5.24840223397393e-05	\\
-5089.66619318182	5.15606386719545e-05	\\
-5088.68741122159	5.2327019823995e-05	\\
-5087.70862926136	5.23176455177323e-05	\\
-5086.72984730114	5.2862153814549e-05	\\
-5085.75106534091	5.15435396544065e-05	\\
-5084.77228338068	5.16194140918874e-05	\\
-5083.79350142046	5.11695524258951e-05	\\
-5082.81471946023	5.33243702452783e-05	\\
-5081.8359375	5.26816774650222e-05	\\
-5080.85715553977	5.22052988823195e-05	\\
-5079.87837357955	5.1879582250807e-05	\\
-5078.89959161932	5.00298576743481e-05	\\
-5077.92080965909	5.03789208900416e-05	\\
-5076.94202769886	5.26326284318432e-05	\\
-5075.96324573864	5.26976823488365e-05	\\
-5074.98446377841	5.26593864675747e-05	\\
-5074.00568181818	5.2512710203877e-05	\\
-5073.02689985796	5.20013142524333e-05	\\
-5072.04811789773	5.28905473169194e-05	\\
-5071.0693359375	5.10865552218443e-05	\\
-5070.09055397727	5.00473087642583e-05	\\
-5069.11177201705	5.0481804794062e-05	\\
-5068.13299005682	5.38377773647703e-05	\\
-5067.15420809659	5.12125079979658e-05	\\
-5066.17542613636	5.0573064526781e-05	\\
-5065.19664417614	5.22967446688783e-05	\\
-5064.21786221591	5.18241330224103e-05	\\
-5063.23908025568	5.22532519914627e-05	\\
-5062.26029829546	5.243039151989e-05	\\
-5061.28151633523	5.11885997058637e-05	\\
-5060.302734375	5.08755070129105e-05	\\
-5059.32395241477	5.14808892279487e-05	\\
-5058.34517045455	5.05590673670925e-05	\\
-5057.36638849432	5.14122062526543e-05	\\
-5056.38760653409	5.26569966244465e-05	\\
-5055.40882457386	5.05434324854411e-05	\\
-5054.43004261364	5.16133349238047e-05	\\
-5053.45126065341	5.23513156910518e-05	\\
-5052.47247869318	5.2413118061032e-05	\\
-5051.49369673296	4.94050848948512e-05	\\
-5050.51491477273	5.17153739823126e-05	\\
-5049.5361328125	4.91754595131868e-05	\\
-5048.55735085227	4.97568143573664e-05	\\
-5047.57856889205	5.03043772749356e-05	\\
-5046.59978693182	5.14588982481077e-05	\\
-5045.62100497159	4.94144583275631e-05	\\
-5044.64222301136	5.22049606197001e-05	\\
-5043.66344105114	5.16749445234187e-05	\\
-5042.68465909091	5.38247557690047e-05	\\
-5041.70587713068	5.31843660191645e-05	\\
-5040.72709517046	5.30498636493274e-05	\\
-5039.74831321023	5.15967219676492e-05	\\
-5038.76953125	5.33671192695071e-05	\\
-5037.79074928977	5.20741559426852e-05	\\
-5036.81196732955	5.19788762886051e-05	\\
-5035.83318536932	5.24749120577504e-05	\\
-5034.85440340909	5.23959897909237e-05	\\
-5033.87562144886	5.25020260048969e-05	\\
-5032.89683948864	4.97858181768654e-05	\\
-5031.91805752841	5.30984793345022e-05	\\
-5030.93927556818	5.30971037501152e-05	\\
-5029.96049360796	5.19084044937716e-05	\\
-5028.98171164773	5.0492105309542e-05	\\
-5028.0029296875	5.22621339282401e-05	\\
-5027.02414772727	5.20350277663897e-05	\\
-5026.04536576705	5.30452542404619e-05	\\
-5025.06658380682	5.30042926402417e-05	\\
-5024.08780184659	5.27472470962938e-05	\\
-5023.10901988636	5.48791490340345e-05	\\
-5022.13023792614	5.29381346664996e-05	\\
-5021.15145596591	5.27502471443745e-05	\\
-5020.17267400568	5.41077609645378e-05	\\
-5019.19389204546	5.44112415885275e-05	\\
-5018.21511008523	5.28316973529818e-05	\\
-5017.236328125	5.398381791307e-05	\\
-5016.25754616477	5.17126663101002e-05	\\
-5015.27876420455	5.43682992495907e-05	\\
-5014.29998224432	5.12451477259633e-05	\\
-5013.32120028409	5.28758849021092e-05	\\
-5012.34241832386	5.08191625190894e-05	\\
-5011.36363636364	5.28571587456839e-05	\\
-5010.38485440341	5.38894343744934e-05	\\
-5009.40607244318	5.45992114994676e-05	\\
-5008.42729048296	5.21511792692646e-05	\\
-5007.44850852273	5.22000092845133e-05	\\
-5006.4697265625	5.45761492598304e-05	\\
-5005.49094460227	5.21680298075162e-05	\\
-5004.51216264205	5.45790327285543e-05	\\
-5003.53338068182	5.57565780324314e-05	\\
-5002.55459872159	5.35737989571313e-05	\\
-5001.57581676136	5.38200170448555e-05	\\
-5000.59703480114	5.38739727222694e-05	\\
-4999.61825284091	5.28571694044281e-05	\\
-4998.63947088068	5.31227281737213e-05	\\
-4997.66068892046	5.45913729519495e-05	\\
-4996.68190696023	5.42739732882128e-05	\\
-4995.703125	5.45919344615559e-05	\\
-4994.72434303977	5.48402388044897e-05	\\
-4993.74556107955	5.37033006586025e-05	\\
-4992.76677911932	5.59684727418981e-05	\\
-4991.78799715909	5.36724248410297e-05	\\
-4990.80921519886	5.37586183160587e-05	\\
-4989.83043323864	5.55162693873277e-05	\\
-4988.85165127841	5.34883078028261e-05	\\
-4987.87286931818	5.52969713989473e-05	\\
-4986.89408735796	5.47567652421693e-05	\\
-4985.91530539773	5.2372266167691e-05	\\
-4984.9365234375	5.4641141033836e-05	\\
-4983.95774147727	5.66998086226229e-05	\\
-4982.97895951705	5.46511381793452e-05	\\
-4982.00017755682	5.74662274498483e-05	\\
-4981.02139559659	5.57208728411443e-05	\\
-4980.04261363636	5.63568561033381e-05	\\
-4979.06383167614	5.62929706741681e-05	\\
-4978.08504971591	5.53616739401819e-05	\\
-4977.10626775568	5.65180395019717e-05	\\
-4976.12748579546	5.59026606444813e-05	\\
-4975.14870383523	5.54866788487804e-05	\\
-4974.169921875	5.49639779607741e-05	\\
-4973.19113991477	5.66063733527541e-05	\\
-4972.21235795455	5.72195519317673e-05	\\
-4971.23357599432	5.6785272602684e-05	\\
-4970.25479403409	5.64151358179009e-05	\\
-4969.27601207386	5.82696651411533e-05	\\
-4968.29723011364	5.54862923281971e-05	\\
-4967.31844815341	5.69443900946055e-05	\\
-4966.33966619318	5.75491298304258e-05	\\
-4965.36088423296	5.82660594250831e-05	\\
-4964.38210227273	5.79364579879213e-05	\\
-4963.4033203125	5.69601620386147e-05	\\
-4962.42453835227	5.80675522083232e-05	\\
-4961.44575639205	5.83244072691669e-05	\\
-4960.46697443182	5.8643426687004e-05	\\
-4959.48819247159	5.95324926130256e-05	\\
-4958.50941051136	5.75694755855366e-05	\\
-4957.53062855114	5.735118377146e-05	\\
-4956.55184659091	5.84025997672743e-05	\\
-4955.57306463068	5.89292238620093e-05	\\
-4954.59428267046	5.92996461797753e-05	\\
-4953.61550071023	5.66312679452255e-05	\\
-4952.63671875	5.76910548539351e-05	\\
-4951.65793678977	5.9749558690415e-05	\\
-4950.67915482955	5.96756660257759e-05	\\
-4949.70037286932	5.8887329000026e-05	\\
-4948.72159090909	5.81463523536322e-05	\\
-4947.74280894886	6.06348388169871e-05	\\
-4946.76402698864	5.64565985048897e-05	\\
-4945.78524502841	5.98374373675829e-05	\\
-4944.80646306818	6.14014280627875e-05	\\
-4943.82768110796	6.03333075638082e-05	\\
-4942.84889914773	6.01470427980197e-05	\\
-4941.8701171875	6.05283912882617e-05	\\
-4940.89133522727	6.30169446241715e-05	\\
-4939.91255326705	6.01384264258976e-05	\\
-4938.93377130682	6.11036452290541e-05	\\
-4937.95498934659	5.93723646814827e-05	\\
-4936.97620738636	6.15486179840616e-05	\\
-4935.99742542614	6.26727591834482e-05	\\
-4935.01864346591	6.0480200957401e-05	\\
-4934.03986150568	6.13638764777277e-05	\\
-4933.06107954546	6.25753549738469e-05	\\
-4932.08229758523	6.25280820170629e-05	\\
-4931.103515625	6.22200609595641e-05	\\
-4930.12473366477	6.4295842492984e-05	\\
-4929.14595170455	6.23988727187697e-05	\\
-4928.16716974432	6.21747028814298e-05	\\
-4927.18838778409	6.46636185205101e-05	\\
-4926.20960582386	6.18507837777486e-05	\\
-4925.23082386364	6.41735969175698e-05	\\
-4924.25204190341	6.37988674792956e-05	\\
-4923.27325994318	6.48233171810019e-05	\\
-4922.29447798296	6.3151122653567e-05	\\
-4921.31569602273	6.41648206824649e-05	\\
-4920.3369140625	6.50522332929068e-05	\\
-4919.35813210227	6.41754356164975e-05	\\
-4918.37935014205	6.52118285043724e-05	\\
-4917.40056818182	6.5163874556063e-05	\\
-4916.42178622159	6.56067489029743e-05	\\
-4915.44300426136	6.66131993236513e-05	\\
-4914.46422230114	6.59452675704849e-05	\\
-4913.48544034091	6.64576283231219e-05	\\
-4912.50665838068	6.77200813151207e-05	\\
-4911.52787642046	6.66094558020333e-05	\\
-4910.54909446023	6.73228034698817e-05	\\
-4909.5703125	6.75600299872649e-05	\\
-4908.59153053977	6.82739325967499e-05	\\
-4907.61274857955	6.54017031232243e-05	\\
-4906.63396661932	6.77163631289743e-05	\\
-4905.65518465909	6.85412729920718e-05	\\
-4904.67640269886	6.85823486271805e-05	\\
-4903.69762073864	6.82168044564504e-05	\\
-4902.71883877841	6.78209449620302e-05	\\
-4901.74005681818	6.70911139598707e-05	\\
-4900.76127485796	6.63839994173725e-05	\\
-4899.78249289773	6.8119908886641e-05	\\
-4898.8037109375	6.85481722019423e-05	\\
-4897.82492897727	6.80084132026512e-05	\\
-4896.84614701705	6.96267955711095e-05	\\
-4895.86736505682	6.88927040907199e-05	\\
-4894.88858309659	6.84079558751591e-05	\\
-4893.90980113636	7.08984510215657e-05	\\
-4892.93101917614	7.11869173046775e-05	\\
-4891.95223721591	6.93735367903514e-05	\\
-4890.97345525568	7.03348737748522e-05	\\
-4889.99467329546	7.0738054958317e-05	\\
-4889.01589133523	6.99757910940503e-05	\\
-4888.037109375	7.22833946923554e-05	\\
-4887.05832741477	7.08691971860532e-05	\\
-4886.07954545455	7.22074165654535e-05	\\
-4885.10076349432	7.02228626717176e-05	\\
-4884.12198153409	6.9946896935064e-05	\\
-4883.14319957386	7.13012467635783e-05	\\
-4882.16441761364	7.04960090925351e-05	\\
-4881.18563565341	7.09549322642953e-05	\\
-4880.20685369318	7.29460892739286e-05	\\
-4879.22807173296	7.16480927758522e-05	\\
-4878.24928977273	7.21667795273382e-05	\\
-4877.2705078125	7.22708688292184e-05	\\
-4876.29172585227	7.29303842138035e-05	\\
-4875.31294389205	7.36813410577092e-05	\\
-4874.33416193182	7.134106289327e-05	\\
-4873.35537997159	7.40675740333283e-05	\\
-4872.37659801136	7.39408140086148e-05	\\
-4871.39781605114	7.40505348084412e-05	\\
-4870.41903409091	7.34219311195087e-05	\\
-4869.44025213068	7.40828529652536e-05	\\
-4868.46147017046	7.42081818465397e-05	\\
-4867.48268821023	7.3372625117679e-05	\\
-4866.50390625	7.59574450321531e-05	\\
-4865.52512428977	7.37443030142533e-05	\\
-4864.54634232955	7.47192790427535e-05	\\
-4863.56756036932	7.55302446091109e-05	\\
-4862.58877840909	7.42942326850534e-05	\\
-4861.60999644886	7.46973235403341e-05	\\
-4860.63121448864	7.42901932870243e-05	\\
-4859.65243252841	7.39620876921719e-05	\\
-4858.67365056818	7.42270600935618e-05	\\
-4857.69486860796	7.57149694319408e-05	\\
-4856.71608664773	7.6882173752275e-05	\\
-4855.7373046875	7.4696678330585e-05	\\
-4854.75852272727	7.65176489336185e-05	\\
-4853.77974076705	7.74236577019896e-05	\\
-4852.80095880682	7.56899624887648e-05	\\
-4851.82217684659	7.57661656601663e-05	\\
-4850.84339488636	7.39567604060066e-05	\\
-4849.86461292614	7.62877021570908e-05	\\
-4848.88583096591	7.65526338749252e-05	\\
-4847.90704900568	7.58930179600704e-05	\\
-4846.92826704546	7.55816937200465e-05	\\
-4845.94948508523	7.53659557845632e-05	\\
-4844.970703125	7.48428243448198e-05	\\
-4843.99192116477	7.79692621285807e-05	\\
-4843.01313920455	7.72786804201492e-05	\\
-4842.03435724432	7.76477340322137e-05	\\
-4841.05557528409	7.66594492969558e-05	\\
-4840.07679332386	7.80114338576798e-05	\\
-4839.09801136364	7.54164774157146e-05	\\
-4838.11922940341	7.89689644109739e-05	\\
-4837.14044744318	7.64108381934833e-05	\\
-4836.16166548296	7.6090019483389e-05	\\
-4835.18288352273	7.83133446093555e-05	\\
-4834.2041015625	7.62736957601508e-05	\\
-4833.22531960227	7.77838646050743e-05	\\
-4832.24653764205	7.76507217175132e-05	\\
-4831.26775568182	7.90200629690434e-05	\\
-4830.28897372159	7.77758467501393e-05	\\
-4829.31019176136	7.77211803754975e-05	\\
-4828.33140980114	8.00056590950066e-05	\\
-4827.35262784091	7.55912930426982e-05	\\
-4826.37384588068	7.90835387584181e-05	\\
-4825.39506392046	7.50418809306346e-05	\\
-4824.41628196023	7.75037142514433e-05	\\
-4823.4375	7.63632951073809e-05	\\
-4822.45871803977	7.95172487428659e-05	\\
-4821.47993607955	7.74280691339184e-05	\\
-4820.50115411932	7.79544133256908e-05	\\
-4819.52237215909	7.64821931006062e-05	\\
-4818.54359019886	7.96407195836e-05	\\
-4817.56480823864	7.81750116438008e-05	\\
-4816.58602627841	7.82199989208351e-05	\\
-4815.60724431818	7.74982615478334e-05	\\
-4814.62846235796	7.7885472334518e-05	\\
-4813.64968039773	7.7503090129227e-05	\\
-4812.6708984375	7.83225225136226e-05	\\
-4811.69211647727	7.8281571320048e-05	\\
-4810.71333451705	7.89768721967446e-05	\\
-4809.73455255682	7.64152089646894e-05	\\
-4808.75577059659	7.99414554365344e-05	\\
-4807.77698863636	7.91985181748475e-05	\\
-4806.79820667614	7.9123201084239e-05	\\
-4805.81942471591	7.68041086272219e-05	\\
-4804.84064275568	7.75464548773558e-05	\\
-4803.86186079546	7.7974633394782e-05	\\
-4802.88307883523	7.82302342887359e-05	\\
-4801.904296875	7.91535031752606e-05	\\
-4800.92551491477	7.76414378897134e-05	\\
-4799.94673295455	7.76231707936038e-05	\\
-4798.96795099432	7.71916311237113e-05	\\
-4797.98916903409	7.95944959761195e-05	\\
-4797.01038707386	7.86005456922319e-05	\\
-4796.03160511364	7.90731712255856e-05	\\
-4795.05282315341	8.04162523369264e-05	\\
-4794.07404119318	7.91749546453672e-05	\\
-4793.09525923296	7.65671579274073e-05	\\
-4792.11647727273	7.83083332334048e-05	\\
-4791.1376953125	8.06623381388015e-05	\\
-4790.15891335227	7.99392911804836e-05	\\
-4789.18013139205	8.0556885171037e-05	\\
-4788.20134943182	7.90421429358482e-05	\\
-4787.22256747159	8.08614359295179e-05	\\
-4786.24378551136	8.08830687772725e-05	\\
-4785.26500355114	7.85850154347887e-05	\\
-4784.28622159091	7.88418402561224e-05	\\
-4783.30743963068	7.86145466929586e-05	\\
-4782.32865767046	7.8349573300153e-05	\\
-4781.34987571023	7.87088387887937e-05	\\
-4780.37109375	7.71350069475418e-05	\\
-4779.39231178977	7.88242469927032e-05	\\
-4778.41352982955	7.96767142678165e-05	\\
-4777.43474786932	8.03572456204448e-05	\\
-4776.45596590909	7.974176928191e-05	\\
-4775.47718394886	8.13797159568189e-05	\\
-4774.49840198864	7.96453935833462e-05	\\
-4773.51962002841	7.90947229608477e-05	\\
-4772.54083806818	7.79003183134277e-05	\\
-4771.56205610796	7.89552208084173e-05	\\
-4770.58327414773	8.07579082086085e-05	\\
-4769.6044921875	7.99426727448506e-05	\\
-4768.62571022727	7.91739775805434e-05	\\
-4767.64692826705	8.02813292218796e-05	\\
-4766.66814630682	7.78927469995796e-05	\\
-4765.68936434659	7.70199012143733e-05	\\
-4764.71058238636	7.87866143225407e-05	\\
-4763.73180042614	8.20541770887033e-05	\\
-4762.75301846591	7.99911108239375e-05	\\
-4761.77423650568	7.82868906488718e-05	\\
-4760.79545454546	7.71832100820945e-05	\\
-4759.81667258523	7.79285054082612e-05	\\
-4758.837890625	7.88290224397229e-05	\\
-4757.85910866477	7.98097850850931e-05	\\
-4756.88032670455	7.68344723358202e-05	\\
-4755.90154474432	7.87221657952714e-05	\\
-4754.92276278409	7.74434048287419e-05	\\
-4753.94398082386	8.12933887130455e-05	\\
-4752.96519886364	7.87842387032001e-05	\\
-4751.98641690341	7.84229271674697e-05	\\
-4751.00763494318	7.84927856449412e-05	\\
-4750.02885298296	8.0956563729041e-05	\\
-4749.05007102273	7.98399432356991e-05	\\
-4748.0712890625	7.98099192498651e-05	\\
-4747.09250710227	7.91849525056025e-05	\\
-4746.11372514205	7.92093784705723e-05	\\
-4745.13494318182	8.02876166061601e-05	\\
-4744.15616122159	7.85873015397949e-05	\\
-4743.17737926136	8.01597454364813e-05	\\
-4742.19859730114	7.98113142193742e-05	\\
-4741.21981534091	8.02184435509749e-05	\\
-4740.24103338068	8.09055283073165e-05	\\
-4739.26225142046	8.05689927126045e-05	\\
-4738.28346946023	7.9922093517473e-05	\\
-4737.3046875	7.96836825373698e-05	\\
-4736.32590553977	7.80041593428417e-05	\\
-4735.34712357955	7.84206654952167e-05	\\
-4734.36834161932	7.94032515643833e-05	\\
-4733.38955965909	7.93314686881476e-05	\\
-4732.41077769886	7.88619804572561e-05	\\
-4731.43199573864	8.12403366733659e-05	\\
-4730.45321377841	7.84147805031507e-05	\\
-4729.47443181818	7.96569079220535e-05	\\
-4728.49564985796	7.93235909575052e-05	\\
-4727.51686789773	7.96210309616041e-05	\\
-4726.5380859375	8.02738036695023e-05	\\
-4725.55930397727	8.06052717528779e-05	\\
-4724.58052201705	7.821645687245e-05	\\
-4723.60174005682	7.98920230442784e-05	\\
-4722.62295809659	7.82377047965851e-05	\\
-4721.64417613636	7.86590050259861e-05	\\
-4720.66539417614	8.05298074044786e-05	\\
-4719.68661221591	8.05273375610086e-05	\\
-4718.70783025568	7.87399412891552e-05	\\
-4717.72904829546	7.8392275826547e-05	\\
-4716.75026633523	7.83230615698836e-05	\\
-4715.771484375	7.82087227024885e-05	\\
-4714.79270241477	8.01927298616046e-05	\\
-4713.81392045455	7.81455404082567e-05	\\
-4712.83513849432	7.86880355039685e-05	\\
-4711.85635653409	8.04854162723825e-05	\\
-4710.87757457386	7.97626552489643e-05	\\
-4709.89879261364	8.01499603626197e-05	\\
-4708.92001065341	7.97504105556154e-05	\\
-4707.94122869318	8.05635024165388e-05	\\
-4706.96244673296	8.033555999168e-05	\\
-4705.98366477273	7.83004396841358e-05	\\
-4705.0048828125	7.87718789059426e-05	\\
-4704.02610085227	7.88526292876254e-05	\\
-4703.04731889205	8.08077430539797e-05	\\
-4702.06853693182	7.97100144245498e-05	\\
-4701.08975497159	8.07218037537935e-05	\\
-4700.11097301136	7.86122567974856e-05	\\
-4699.13219105114	7.82476067914691e-05	\\
-4698.15340909091	7.96572154878711e-05	\\
-4697.17462713068	7.69536434193436e-05	\\
-4696.19584517046	8.00138341694581e-05	\\
-4695.21706321023	7.90457008427625e-05	\\
-4694.23828125	7.81947407037788e-05	\\
-4693.25949928977	7.92591361498328e-05	\\
-4692.28071732955	7.89139119827396e-05	\\
-4691.30193536932	7.96287365263998e-05	\\
-4690.32315340909	7.79091327875234e-05	\\
-4689.34437144886	8.19319982214875e-05	\\
-4688.36558948864	7.84192764645548e-05	\\
-4687.38680752841	7.97568166073841e-05	\\
-4686.40802556818	8.03570769302255e-05	\\
-4685.42924360796	8.02415654687743e-05	\\
-4684.45046164773	7.91659784470102e-05	\\
-4683.4716796875	7.85516711961244e-05	\\
-4682.49289772727	7.91947892063153e-05	\\
-4681.51411576705	7.94431224906189e-05	\\
-4680.53533380682	7.80881674562189e-05	\\
-4679.55655184659	7.81711893648094e-05	\\
-4678.57776988636	7.87929869975175e-05	\\
-4677.59898792614	7.94283035148707e-05	\\
-4676.62020596591	7.74927511174548e-05	\\
-4675.64142400568	7.97625195036512e-05	\\
-4674.66264204546	7.79792882163949e-05	\\
-4673.68386008523	8.03271302345769e-05	\\
-4672.705078125	7.80849590317882e-05	\\
-4671.72629616477	7.85831621235825e-05	\\
-4670.74751420455	7.99306712059653e-05	\\
-4669.76873224432	8.11009361074733e-05	\\
-4668.78995028409	8.05609005105441e-05	\\
-4667.81116832386	8.13961668859202e-05	\\
-4666.83238636364	8.01642190932517e-05	\\
-4665.85360440341	7.86928817762852e-05	\\
-4664.87482244318	7.71510125783426e-05	\\
-4663.89604048296	7.98502886924067e-05	\\
-4662.91725852273	8.1129279623238e-05	\\
-4661.9384765625	8.21256916808492e-05	\\
-4660.95969460227	8.41588360627677e-05	\\
-4659.98091264205	8.19189045419118e-05	\\
-4659.00213068182	8.0367283859298e-05	\\
-4658.02334872159	8.28809285143065e-05	\\
-4657.04456676136	8.00998696123384e-05	\\
-4656.06578480114	8.116431596942e-05	\\
-4655.08700284091	8.2826049328088e-05	\\
-4654.10822088068	8.23808431676383e-05	\\
-4653.12943892046	8.26338339035303e-05	\\
-4652.15065696023	8.19462268802957e-05	\\
-4651.171875	8.47528083322389e-05	\\
-4650.19309303977	8.45070442532507e-05	\\
-4649.21431107955	8.25504145226047e-05	\\
-4648.23552911932	8.26745599466325e-05	\\
-4647.25674715909	8.16524841694673e-05	\\
-4646.27796519886	8.50156989696302e-05	\\
-4645.29918323864	8.48755861696468e-05	\\
-4644.32040127841	8.15482418727486e-05	\\
-4643.34161931818	8.4042584845961e-05	\\
-4642.36283735796	8.35563556057426e-05	\\
-4641.38405539773	8.5811977584048e-05	\\
-4640.4052734375	8.43835782955763e-05	\\
-4639.42649147727	8.60984027576728e-05	\\
-4638.44770951705	8.41426903503069e-05	\\
-4637.46892755682	8.28957911572556e-05	\\
-4636.49014559659	8.41684174135159e-05	\\
-4635.51136363636	8.54028162440478e-05	\\
-4634.53258167614	8.20840088023009e-05	\\
-4633.55379971591	8.49094009682539e-05	\\
-4632.57501775568	8.47310547007075e-05	\\
-4631.59623579546	8.32633211558157e-05	\\
-4630.61745383523	8.53114310734533e-05	\\
-4629.638671875	8.75675363010231e-05	\\
-4628.65988991477	8.4218517459461e-05	\\
-4627.68110795455	8.3098895960179e-05	\\
-4626.70232599432	8.51051830968791e-05	\\
-4625.72354403409	8.57627344000915e-05	\\
-4624.74476207386	8.72632729466422e-05	\\
-4623.76598011364	8.65525932714536e-05	\\
-4622.78719815341	8.72434215363202e-05	\\
-4621.80841619318	8.78471654489813e-05	\\
-4620.82963423296	8.83680632559844e-05	\\
-4619.85085227273	8.92598247825684e-05	\\
-4618.8720703125	8.70849202489646e-05	\\
-4617.89328835227	8.86556451515193e-05	\\
-4616.91450639205	8.86307165536298e-05	\\
-4615.93572443182	8.92310812430002e-05	\\
-4614.95694247159	8.87483268628357e-05	\\
-4613.97816051136	8.907727656114e-05	\\
-4612.99937855114	9.04073539784078e-05	\\
-4612.02059659091	9.13759114958552e-05	\\
-4611.04181463068	8.96073362586714e-05	\\
-4610.06303267046	9.02634219564315e-05	\\
-4609.08425071023	9.00826501803106e-05	\\
-4608.10546875	9.00460604819392e-05	\\
-4607.12668678977	9.22668444514957e-05	\\
-4606.14790482955	9.3145280899648e-05	\\
-4605.16912286932	9.20060127790896e-05	\\
-4604.19034090909	9.14542568605606e-05	\\
-4603.21155894886	9.1829999815986e-05	\\
-4602.23277698864	9.38659912874759e-05	\\
-4601.25399502841	9.10750127325239e-05	\\
-4600.27521306818	9.7673140997264e-05	\\
-4599.29643110796	9.46047966052453e-05	\\
-4598.31764914773	9.49711702632827e-05	\\
-4597.3388671875	9.46220465296327e-05	\\
-4596.36008522727	9.41749820823683e-05	\\
-4595.38130326705	9.51858070079054e-05	\\
-4594.40252130682	9.35300304551854e-05	\\
-4593.42373934659	9.76956800080182e-05	\\
-4592.44495738636	9.56636995265472e-05	\\
-4591.46617542614	9.65428061733673e-05	\\
-4590.48739346591	9.81525207963668e-05	\\
-4589.50861150568	9.58768288681301e-05	\\
-4588.52982954546	0.000100435365971322	\\
-4587.55104758523	9.67758627288254e-05	\\
-4586.572265625	9.69785132525492e-05	\\
-4585.59348366477	9.83881542824414e-05	\\
-4584.61470170455	9.97090570870323e-05	\\
-4583.63591974432	9.96901816040588e-05	\\
-4582.65713778409	9.87799746112625e-05	\\
-4581.67835582386	9.90636725224955e-05	\\
-4580.69957386364	9.95936744854312e-05	\\
-4579.72079190341	0.000101388559880982	\\
-4578.74200994318	0.000101383037102236	\\
-4577.76322798296	0.000100088613207254	\\
-4576.78444602273	0.000100373457476597	\\
-4575.8056640625	0.000102175447184772	\\
-4574.82688210227	0.000102634357841504	\\
-4573.84810014205	0.000103565613351491	\\
-4572.86931818182	0.000100831366438583	\\
-4571.89053622159	0.000103110487092495	\\
-4570.91175426136	0.000103649939110493	\\
-4569.93297230114	0.000104545074004906	\\
-4568.95419034091	0.000105484664702518	\\
-4567.97540838068	0.000105433916306429	\\
-4566.99662642046	0.000105639404165948	\\
-4566.01784446023	0.000102693871616359	\\
-4565.0390625	0.000105095313383273	\\
-4564.06028053977	0.000105440791967707	\\
-4563.08149857955	0.000106007223219336	\\
-4562.10271661932	0.000106599369576554	\\
-4561.12393465909	0.000108271128196476	\\
-4560.14515269886	0.000104397280485713	\\
-4559.16637073864	0.000107129039108578	\\
-4558.18758877841	0.000108179111606423	\\
-4557.20880681818	0.000109751605578799	\\
-4556.23002485796	0.000107136810836517	\\
-4555.25124289773	0.000109782710462064	\\
-4554.2724609375	0.00010733957184443	\\
-4553.29367897727	0.000106755994220823	\\
-4552.31489701705	0.000107857469056355	\\
-4551.33611505682	0.00010783868759807	\\
-4550.35733309659	0.000107499475193607	\\
-4549.37855113636	0.000108333969626382	\\
-4548.39976917614	0.000108140415586013	\\
-4547.42098721591	0.000111064289466998	\\
-4546.44220525568	0.000110566815730434	\\
-4545.46342329546	0.000108360918273933	\\
-4544.48464133523	0.000108015831333234	\\
-4543.505859375	0.000108959312554655	\\
-4542.52707741477	0.000110749688662788	\\
-4541.54829545455	0.00010890303128519	\\
-4540.56951349432	0.000110659850049474	\\
-4539.59073153409	0.000112521286708167	\\
-4538.61194957386	0.000111887541202859	\\
-4537.63316761364	0.000111381730233063	\\
-4536.65438565341	0.000112216561630427	\\
-4535.67560369318	0.000108200885555198	\\
-4534.69682173296	0.000112449445864662	\\
-4533.71803977273	0.000112558066362272	\\
-4532.7392578125	0.000111802086602473	\\
-4531.76047585227	0.000113762160314622	\\
-4530.78169389205	0.000114009705828046	\\
-4529.80291193182	0.000114361742159585	\\
-4528.82412997159	0.000111388688431173	\\
-4527.84534801136	0.000112657391505301	\\
-4526.86656605114	0.000110815963327824	\\
-4525.88778409091	0.000114735131236733	\\
-4524.90900213068	0.000112276209739155	\\
-4523.93022017046	0.000113690056513307	\\
-4522.95143821023	0.000113736049831925	\\
-4521.97265625	0.000113763916701683	\\
-4520.99387428977	0.000113503420429776	\\
-4520.01509232955	0.00011378869370202	\\
-4519.03631036932	0.000112725537494584	\\
-4518.05752840909	0.000115076211734674	\\
-4517.07874644886	0.000113713284365649	\\
-4516.09996448864	0.000114670726323816	\\
-4515.12118252841	0.000114995484131727	\\
-4514.14240056818	0.000115892188277425	\\
-4513.16361860796	0.000115738569750853	\\
-4512.18483664773	0.000106023306722551	\\
-4511.2060546875	0.000114050664504832	\\
-4510.22727272727	0.000114233868615286	\\
-4509.24849076705	0.000115103133167886	\\
-4508.26970880682	0.000114292463613177	\\
-4507.29092684659	0.00011576664623277	\\
-4506.31214488636	0.000114544512864584	\\
-4505.33336292614	0.000116016990298484	\\
-4504.35458096591	0.00011422402363764	\\
-4503.37579900568	0.000115582732361417	\\
-4502.39701704546	0.000116091831045409	\\
-4501.41823508523	0.000114584530665034	\\
-4500.439453125	0.000116129672762871	\\
-4499.46067116477	0.000115680302353912	\\
-4498.48188920455	0.000115277780192009	\\
-4497.50310724432	0.000114559593437624	\\
-4496.52432528409	0.0001169697764403	\\
-4495.54554332386	0.000113890272256524	\\
-4494.56676136364	0.000117013439135165	\\
-4493.58797940341	0.00011536113376381	\\
-4492.60919744318	0.000115569604957036	\\
-4491.63041548296	0.000115460291076671	\\
-4490.65163352273	0.00011367961357795	\\
-4489.6728515625	0.000116658210258158	\\
-4488.69406960227	0.000117583549011584	\\
-4487.71528764205	0.00011523880591288	\\
-4486.73650568182	0.000115411029892159	\\
-4485.75772372159	0.000117069754999531	\\
-4484.77894176136	0.000115685702794142	\\
-4483.80015980114	0.000116439820352986	\\
-4482.82137784091	0.000116529712888743	\\
-4481.84259588068	0.000118323980650882	\\
-4480.86381392046	0.000117773505048332	\\
-4479.88503196023	0.000116678028944413	\\
-4478.90625	0.000115687464700858	\\
-4477.92746803977	0.000118979898859924	\\
-4476.94868607955	0.00011547162755643	\\
-4475.96990411932	0.000115235278590141	\\
-4474.99112215909	0.00011517123984185	\\
-4474.01234019886	0.000117218212407508	\\
-4473.03355823864	0.000114565142261902	\\
-4472.05477627841	0.00011675076719947	\\
-4471.07599431818	0.000116429864742275	\\
-4470.09721235796	0.000115205248592002	\\
-4469.11843039773	0.00011589961701628	\\
-4468.1396484375	0.000116815332644877	\\
-4467.16086647727	0.000116534133238427	\\
-4466.18208451705	0.000117015651732856	\\
-4465.20330255682	0.000117117759373995	\\
-4464.22452059659	0.000116287297150036	\\
-4463.24573863636	0.000117420947115017	\\
-4462.26695667614	0.000115786345421428	\\
-4461.28817471591	0.000116211600847481	\\
-4460.30939275568	0.000114907263120915	\\
-4459.33061079546	0.000119461258593652	\\
-4458.35182883523	0.000113941469766448	\\
-4457.373046875	0.000115359912610017	\\
-4456.39426491477	0.000115775453674437	\\
-4455.41548295455	0.000116201878998926	\\
-4454.43670099432	0.000115616787528888	\\
-4453.45791903409	0.000117504760387166	\\
-4452.47913707386	0.000115739690105399	\\
-4451.50035511364	0.000114888797497631	\\
-4450.52157315341	0.000115735450623024	\\
-4449.54279119318	0.000118114671012593	\\
-4448.56400923296	0.000114931297447742	\\
-4447.58522727273	0.000114862015918878	\\
-4446.6064453125	0.000117788894104643	\\
-4445.62766335227	0.000117345163072253	\\
-4444.64888139205	0.000118924513233897	\\
-4443.67009943182	0.000118146031546844	\\
-4442.69131747159	0.000116363884691725	\\
-4441.71253551136	0.000118488496672291	\\
-4440.73375355114	0.000115763501981738	\\
-4439.75497159091	0.000115658200396103	\\
-4438.77618963068	0.000118585482806887	\\
-4437.79740767046	0.000118424319878456	\\
-4436.81862571023	0.000116029079967003	\\
-4435.83984375	0.000116896930355841	\\
-4434.86106178977	0.000117504628611855	\\
-4433.88227982955	0.000119475754037835	\\
-4432.90349786932	0.000119382251292734	\\
-4431.92471590909	0.000117979070514969	\\
-4430.94593394886	0.000115717257578117	\\
-4429.96715198864	0.000116978603341468	\\
-4428.98837002841	0.000117222229179482	\\
-4428.00958806818	0.000116266346512261	\\
-4427.03080610796	0.00011704243258302	\\
-4426.05202414773	0.000119092151817225	\\
-4425.0732421875	0.000114784144668184	\\
-4424.09446022727	0.000116308171671504	\\
-4423.11567826705	0.000118407652496082	\\
-4422.13689630682	0.000117486867331002	\\
-4421.15811434659	0.000115408785695141	\\
-4420.17933238636	0.000118379879000927	\\
-4419.20055042614	0.000116972246756642	\\
-4418.22176846591	0.000117254458986122	\\
-4417.24298650568	0.000117377244556358	\\
-4416.26420454546	0.000115954630994544	\\
-4415.28542258523	0.00011796838343482	\\
-4414.306640625	0.000115871362667795	\\
-4413.32785866477	0.000117126621682799	\\
-4412.34907670455	0.000116746577619735	\\
-4411.37029474432	0.000118517317022124	\\
-4410.39151278409	0.000117953143090697	\\
-4409.41273082386	0.000118795578472689	\\
-4408.43394886364	0.000115473194258742	\\
-4407.45516690341	0.00011553944908043	\\
-4406.47638494318	0.00011693961410382	\\
-4405.49760298296	0.000116600636560717	\\
-4404.51882102273	0.000115924746924495	\\
-4403.5400390625	0.000119803605579994	\\
-4402.56125710227	0.000117142275650159	\\
-4401.58247514205	0.000118111033171651	\\
-4400.60369318182	0.000117359318926711	\\
-4399.62491122159	0.000120203527003992	\\
-4398.64612926136	0.000116883990317412	\\
-4397.66734730114	0.000118774554421231	\\
-4396.68856534091	0.000117880583455708	\\
-4395.70978338068	0.000117796044782814	\\
-4394.73100142046	0.000115648822247088	\\
-4393.75221946023	0.000116048647821448	\\
-4392.7734375	0.000116652147784752	\\
-4391.79465553977	0.000115970278142576	\\
-4390.81587357955	0.00011737745629782	\\
-4389.83709161932	0.000115446647962266	\\
-4388.85830965909	0.000115713141453414	\\
-4387.87952769886	0.00011594549414213	\\
-4386.90074573864	0.000115579979666637	\\
-4385.92196377841	0.000117208749520153	\\
-4384.94318181818	0.000113998395736868	\\
-4383.96439985796	0.000116142271876338	\\
-4382.98561789773	0.000113168461174828	\\
-4382.0068359375	0.000115680523492215	\\
-4381.02805397727	0.000117199673584631	\\
-4380.04927201705	0.000117063247110797	\\
-4379.07049005682	0.000115604966910885	\\
-4378.09170809659	0.000114034250357541	\\
-4377.11292613636	0.000116424440214874	\\
-4376.13414417614	0.000114588958109801	\\
-4375.15536221591	0.000116227561284146	\\
-4374.17658025568	0.000113327326764371	\\
-4373.19779829546	0.000113424912589705	\\
-4372.21901633523	0.0001142097728549	\\
-4371.240234375	0.000112895057360266	\\
-4370.26145241477	0.000115214748615435	\\
-4369.28267045455	0.000114918666896762	\\
-4368.30388849432	0.000112451274189161	\\
-4367.32510653409	0.000112688245288988	\\
-4366.34632457386	0.000113872413607742	\\
-4365.36754261364	0.000113721197753494	\\
-4364.38876065341	0.000113901983597693	\\
-4363.40997869318	0.000112897606007324	\\
-4362.43119673296	0.000114940154187959	\\
-4361.45241477273	0.000113538683890425	\\
-4360.4736328125	0.000112622813331686	\\
-4359.49485085227	0.000111832883995925	\\
-4358.51606889205	0.000112088192797291	\\
-4357.53728693182	0.000114346242664493	\\
-4356.55850497159	0.000113135756733907	\\
-4355.57972301136	0.000111178555228406	\\
-4354.60094105114	0.000111602832727323	\\
-4353.62215909091	0.000111003649089038	\\
-4352.64337713068	0.000111346425287563	\\
-4351.66459517046	0.000112624620265348	\\
-4350.68581321023	0.000111618274243707	\\
-4349.70703125	0.000109405541212874	\\
-4348.72824928977	0.000108860069075982	\\
-4347.74946732955	0.000111067027558041	\\
-4346.77068536932	0.00011103653707154	\\
-4345.79190340909	0.000112198117285697	\\
-4344.81312144886	0.000111791350899336	\\
-4343.83433948864	0.000111749917743783	\\
-4342.85555752841	0.000109192809490091	\\
-4341.87677556818	0.000111308006983966	\\
-4340.89799360796	0.000112418700990077	\\
-4339.91921164773	0.000111023935306734	\\
-4338.9404296875	0.000107886493479101	\\
-4337.96164772727	0.000108307325940812	\\
-4336.98286576705	0.00010800510107298	\\
-4336.00408380682	0.000108773214356988	\\
-4335.02530184659	0.000108990649220755	\\
-4334.04651988636	0.00010888883118965	\\
-4333.06773792614	0.000110497747036147	\\
-4332.08895596591	0.000104737766949775	\\
-4331.11017400568	0.000107868001212899	\\
-4330.13139204546	0.000108675567616438	\\
-4329.15261008523	0.000108006438198913	\\
-4328.173828125	0.000106797794637051	\\
-4327.19504616477	0.000105953805881711	\\
-4326.21626420455	0.000107265577924873	\\
-4325.23748224432	0.000105975212439515	\\
-4324.25870028409	0.000105133428304788	\\
-4323.27991832386	0.000107972641268518	\\
-4322.30113636364	0.000106664419276749	\\
-4321.32235440341	0.000103722550685247	\\
-4320.34357244318	0.000107839900873619	\\
-4319.36479048296	0.000104483205704446	\\
-4318.38600852273	0.000106551668926921	\\
-4317.4072265625	0.000107756719530542	\\
-4316.42844460227	0.000106337929886803	\\
-4315.44966264205	0.000106451761840359	\\
-4314.47088068182	0.000104092396857791	\\
-4313.49209872159	0.000104155550020123	\\
-4312.51331676136	0.000107017068800774	\\
-4311.53453480114	0.000105332878270688	\\
-4310.55575284091	0.000103795972042148	\\
-4309.57697088068	0.000103804971642891	\\
-4308.59818892046	0.00010545459005973	\\
-4307.61940696023	0.000102513659422031	\\
-4306.640625	0.000106734954614597	\\
-4305.66184303977	0.000105388158474203	\\
-4304.68306107955	0.000103950477624467	\\
-4303.70427911932	0.000105499037145655	\\
-4302.72549715909	0.000105279094887237	\\
-4301.74671519886	0.000104809535682882	\\
-4300.76793323864	0.000105858879724743	\\
-4299.78915127841	0.000104384863286092	\\
-4298.81036931818	0.000104305662859382	\\
-4297.83158735796	0.000105500677178496	\\
-4296.85280539773	0.000106459186042841	\\
-4295.8740234375	0.000102630798502275	\\
-4294.89524147727	0.000105465115768709	\\
-4293.91645951705	0.000103502693146775	\\
-4292.93767755682	0.00010416398677838	\\
-4291.95889559659	0.000105673604045838	\\
-4290.98011363636	0.000103802665404053	\\
-4290.00133167614	0.000104979579704251	\\
-4289.02254971591	0.00010547705917011	\\
-4288.04376775568	0.000105002081586484	\\
-4287.06498579546	0.000102329531417306	\\
-4286.08620383523	0.00010447060059868	\\
-4285.107421875	0.000102929763906183	\\
-4284.12863991477	0.000104416019481034	\\
-4283.14985795455	0.000101253542932227	\\
-4282.17107599432	0.000102417161366814	\\
-4281.19229403409	0.000105646688372463	\\
-4280.21351207386	0.000100253555650043	\\
-4279.23473011364	0.000102557759005157	\\
-4278.25594815341	0.000102324021832206	\\
-4277.27716619318	0.000104556834813828	\\
-4276.29838423296	9.81742564384437e-05	\\
-4275.31960227273	0.000102946864006097	\\
-4274.3408203125	0.000102605069686563	\\
-4273.36203835227	0.000104329852586865	\\
-4272.38325639205	0.000103363597769147	\\
-4271.40447443182	0.000102570260184354	\\
-4270.42569247159	0.00010204286378251	\\
-4269.44691051136	0.000102125053177691	\\
-4268.46812855114	0.000102767552826459	\\
-4267.48934659091	0.000100841052973901	\\
-4266.51056463068	0.000102506273081721	\\
-4265.53178267046	0.0001011018433553	\\
-4264.55300071023	0.000102220873349984	\\
-4263.57421875	0.000100869248546103	\\
-4262.59543678977	0.000102066210421538	\\
-4261.61665482955	0.000100010821672717	\\
-4260.63787286932	0.000103140459275127	\\
-4259.65909090909	0.000101346983749047	\\
-4258.68030894886	0.0001000214820181	\\
-4257.70152698864	0.000100929268990175	\\
-4256.72274502841	0.000101318385426305	\\
-4255.74396306818	9.95837445644927e-05	\\
-4254.76518110796	0.000101993533293386	\\
-4253.78639914773	0.000100430591877105	\\
-4252.8076171875	0.000100273725203492	\\
-4251.82883522727	9.87424058329013e-05	\\
-4250.85005326705	0.000100603918527195	\\
-4249.87127130682	0.000101414867665822	\\
-4248.89248934659	0.000100251062921856	\\
-4247.91370738636	9.8940075279762e-05	\\
-4246.93492542614	0.000102276809660598	\\
-4245.95614346591	9.96894733316131e-05	\\
-4244.97736150568	9.8538922638206e-05	\\
-4243.99857954546	0.000100042648337342	\\
-4243.01979758523	9.92050911223334e-05	\\
-4242.041015625	9.75335017470263e-05	\\
-4241.06223366477	9.92935965474252e-05	\\
-4240.08345170455	9.9553993817995e-05	\\
-4239.10466974432	9.87772886363823e-05	\\
-4238.12588778409	9.70433365971493e-05	\\
-4237.14710582386	9.85960283660716e-05	\\
-4236.16832386364	9.86918286517235e-05	\\
-4235.18954190341	9.66146359332016e-05	\\
-4234.21075994318	9.96960719063801e-05	\\
-4233.23197798296	9.64985520547917e-05	\\
-4232.25319602273	9.80193405843054e-05	\\
-4231.2744140625	9.90435955034265e-05	\\
-4230.29563210227	9.99496939723589e-05	\\
-4229.31685014205	9.44618061554e-05	\\
-4228.33806818182	0.000102090218485394	\\
-4227.35928622159	9.67006789767166e-05	\\
-4226.38050426136	9.4322587166922e-05	\\
-4225.40172230114	9.73283601887682e-05	\\
-4224.42294034091	9.74948454800181e-05	\\
-4223.44415838068	9.7171858644423e-05	\\
-4222.46537642046	9.65688440736961e-05	\\
-4221.48659446023	9.76200237760917e-05	\\
-4220.5078125	9.56177111320597e-05	\\
-4219.52903053977	9.60655273974173e-05	\\
-4218.55024857955	9.62828147104835e-05	\\
-4217.57146661932	9.49580895128478e-05	\\
-4216.59268465909	9.54371317972471e-05	\\
-4215.61390269886	9.51103061200919e-05	\\
-4214.63512073864	9.6998102270207e-05	\\
-4213.65633877841	9.50445680964885e-05	\\
-4212.67755681818	9.614865654796e-05	\\
-4211.69877485796	9.58820042836269e-05	\\
-4210.71999289773	9.37216996890775e-05	\\
-4209.7412109375	9.66220962304747e-05	\\
-4208.76242897727	9.5857045205885e-05	\\
-4207.78364701705	9.44639733164188e-05	\\
-4206.80486505682	9.54686069286702e-05	\\
-4205.82608309659	9.45350966974862e-05	\\
-4204.84730113636	9.41061861794995e-05	\\
-4203.86851917614	9.51759872027832e-05	\\
-4202.88973721591	9.55495067672394e-05	\\
-4201.91095525568	9.64557604857037e-05	\\
-4200.93217329546	9.50474983927925e-05	\\
-4199.95339133523	9.13577918488113e-05	\\
-4198.974609375	9.45310423661782e-05	\\
-4197.99582741477	9.67866614943049e-05	\\
-4197.01704545455	9.44863591396585e-05	\\
-4196.03826349432	9.49073944254055e-05	\\
-4195.05948153409	9.68783218652677e-05	\\
-4194.08069957386	9.37400085294248e-05	\\
-4193.10191761364	9.16284367440943e-05	\\
-4192.12313565341	9.42454262947074e-05	\\
-4191.14435369318	9.44188386505487e-05	\\
-4190.16557173296	9.40478113060824e-05	\\
-4189.18678977273	9.16509457389509e-05	\\
-4188.2080078125	9.09454952948543e-05	\\
-4187.22922585227	9.3588608223492e-05	\\
-4186.25044389205	9.2790031044821e-05	\\
-4185.27166193182	9.18609036638891e-05	\\
-4184.29287997159	9.15340191190911e-05	\\
-4183.31409801136	9.35118215892101e-05	\\
-4182.33531605114	9.1091362489808e-05	\\
-4181.35653409091	9.15131449622112e-05	\\
-4180.37775213068	9.04902834697649e-05	\\
-4179.39897017046	9.14284764248923e-05	\\
-4178.42018821023	8.88029073679218e-05	\\
-4177.44140625	9.36299374373706e-05	\\
-4176.46262428977	9.36042083857349e-05	\\
-4175.48384232955	9.07098411440527e-05	\\
-4174.50506036932	9.54138476269107e-05	\\
-4173.52627840909	9.19555914249156e-05	\\
-4172.54749644886	9.32346912419189e-05	\\
-4171.56871448864	9.14698108932786e-05	\\
-4170.58993252841	9.18194704120757e-05	\\
-4169.61115056818	9.18549225168089e-05	\\
-4168.63236860796	9.00403951090457e-05	\\
-4167.65358664773	9.07113363689451e-05	\\
-4166.6748046875	8.99581856020288e-05	\\
-4165.69602272727	9.12374134725648e-05	\\
-4164.71724076705	8.82966758673392e-05	\\
-4163.73845880682	8.98422649766339e-05	\\
-4162.75967684659	9.04452622480571e-05	\\
-4161.78089488636	8.77602975926398e-05	\\
-4160.80211292614	8.84856323603387e-05	\\
-4159.82333096591	8.81709120046733e-05	\\
-4158.84454900568	8.92331289988592e-05	\\
-4157.86576704546	9.10981984066683e-05	\\
-4156.88698508523	8.66747851012112e-05	\\
-4155.908203125	8.68958180493727e-05	\\
-4154.92942116477	8.82066525993378e-05	\\
-4153.95063920455	8.6942526466015e-05	\\
-4152.97185724432	8.65618098601793e-05	\\
-4151.99307528409	8.810324884098e-05	\\
-4151.01429332386	8.56095794523065e-05	\\
-4150.03551136364	8.65135882814168e-05	\\
-4149.05672940341	8.44331136072075e-05	\\
-4148.07794744318	8.4943500731019e-05	\\
-4147.09916548296	8.45118588980322e-05	\\
-4146.12038352273	8.46666164074958e-05	\\
-4145.1416015625	8.76518774789179e-05	\\
-4144.16281960227	8.69138591251972e-05	\\
-4143.18403764205	8.67846805758007e-05	\\
-4142.20525568182	8.15918249677413e-05	\\
-4141.22647372159	8.63243495418266e-05	\\
-4140.24769176136	8.60859231602501e-05	\\
-4139.26890980114	8.48891069961936e-05	\\
-4138.29012784091	8.44541071199009e-05	\\
-4137.31134588068	8.41691362215713e-05	\\
-4136.33256392046	8.35321325915019e-05	\\
-4135.35378196023	8.30719968446397e-05	\\
-4134.375	8.3516376992899e-05	\\
-4133.39621803977	8.32522972659722e-05	\\
-4132.41743607955	8.37190301752474e-05	\\
-4131.43865411932	8.57646258671459e-05	\\
-4130.45987215909	8.26747482406146e-05	\\
-4129.48109019886	8.29474732447441e-05	\\
-4128.50230823864	8.32840187204088e-05	\\
-4127.52352627841	8.47513104013074e-05	\\
-4126.54474431818	7.79370240317722e-05	\\
-4125.56596235796	8.15426996080436e-05	\\
-4124.58718039773	8.23527225476368e-05	\\
-4123.6083984375	8.20022851031941e-05	\\
-4122.62961647727	8.12631101976167e-05	\\
-4121.65083451705	8.42280867061949e-05	\\
-4120.67205255682	8.17513683229553e-05	\\
-4119.69327059659	7.85244498032854e-05	\\
-4118.71448863636	8.22431702876839e-05	\\
-4117.73570667614	8.13769809514199e-05	\\
-4116.75692471591	8.16125277908602e-05	\\
-4115.77814275568	8.16500574923275e-05	\\
-4114.79936079546	8.25219929187312e-05	\\
-4113.82057883523	8.04236961430153e-05	\\
-4112.841796875	8.20927954879315e-05	\\
-4111.86301491477	8.11901039068873e-05	\\
-4110.88423295455	8.26230905766748e-05	\\
-4109.90545099432	7.80448184103952e-05	\\
-4108.92666903409	7.93557420443856e-05	\\
-4107.94788707386	8.21540460336231e-05	\\
-4106.96910511364	8.00728853713656e-05	\\
-4105.99032315341	8.00685596324788e-05	\\
-4105.01154119318	8.02057708979622e-05	\\
-4104.03275923296	7.9736129647678e-05	\\
-4103.05397727273	8.10893139587971e-05	\\
-4102.0751953125	7.94069595858111e-05	\\
-4101.09641335227	7.78330392731703e-05	\\
-4100.11763139205	7.93160566799202e-05	\\
-4099.13884943182	7.88018633853852e-05	\\
-4098.16006747159	8.04877887394644e-05	\\
-4097.18128551136	7.9028419869565e-05	\\
-4096.20250355114	8.02418160360471e-05	\\
-4095.22372159091	7.77686811287096e-05	\\
-4094.24493963068	8.07906002317808e-05	\\
-4093.26615767046	8.02240031149662e-05	\\
-4092.28737571023	7.98886916667405e-05	\\
-4091.30859375	8.04328452364007e-05	\\
-4090.32981178977	8.06377069589558e-05	\\
-4089.35102982955	8.2711673574383e-05	\\
-4088.37224786932	8.01782105495074e-05	\\
-4087.39346590909	8.10000809630277e-05	\\
-4086.41468394886	8.01646434176957e-05	\\
-4085.43590198864	8.09446974430963e-05	\\
-4084.45712002841	8.09264730384322e-05	\\
-4083.47833806818	8.18920351957306e-05	\\
-4082.49955610796	8.10128348293879e-05	\\
-4081.52077414773	8.1689005549291e-05	\\
-4080.5419921875	8.01944546788188e-05	\\
-4079.56321022727	8.08251759066008e-05	\\
-4078.58442826705	8.21025682506039e-05	\\
-4077.60564630682	8.24359204670766e-05	\\
-4076.62686434659	8.21813865271165e-05	\\
-4075.64808238636	8.01587896648959e-05	\\
-4074.66930042614	8.15389814209086e-05	\\
-4073.69051846591	8.32858077380804e-05	\\
-4072.71173650568	8.20761810450682e-05	\\
-4071.73295454546	8.20423268754095e-05	\\
-4070.75417258523	8.32661753080957e-05	\\
-4069.775390625	8.11256011601374e-05	\\
-4068.79660866477	8.53761186406746e-05	\\
-4067.81782670455	8.21467185572362e-05	\\
-4066.83904474432	8.16917550785197e-05	\\
-4065.86026278409	8.29887448230636e-05	\\
-4064.88148082386	8.32809295181237e-05	\\
-4063.90269886364	8.51887198156805e-05	\\
-4062.92391690341	8.31025899259055e-05	\\
-4061.94513494318	8.4363654238803e-05	\\
-4060.96635298296	8.3313550072138e-05	\\
-4059.98757102273	8.1729190947633e-05	\\
-4059.0087890625	8.26756213330359e-05	\\
-4058.03000710227	8.2455346225528e-05	\\
-4057.05122514205	8.54619889066205e-05	\\
-4056.07244318182	8.44702497435857e-05	\\
-4055.09366122159	8.31988846480278e-05	\\
-4054.11487926136	8.25859477041144e-05	\\
-4053.13609730114	8.43527172080354e-05	\\
-4052.15731534091	8.25659668699362e-05	\\
-4051.17853338068	8.55503590727612e-05	\\
-4050.19975142046	8.33446726209941e-05	\\
-4049.22096946023	8.36133009817181e-05	\\
-4048.2421875	8.46266607460645e-05	\\
-4047.26340553977	8.61833111903191e-05	\\
-4046.28462357955	8.45162229637547e-05	\\
-4045.30584161932	8.4143915928939e-05	\\
-4044.32705965909	8.58258421509903e-05	\\
-4043.34827769886	8.3860155605107e-05	\\
-4042.36949573864	8.30834804370537e-05	\\
-4041.39071377841	8.6597924698581e-05	\\
-4040.41193181818	8.4225645157435e-05	\\
-4039.43314985796	8.54254764886709e-05	\\
-4038.45436789773	8.35675406067719e-05	\\
-4037.4755859375	8.42141672721122e-05	\\
-4036.49680397727	8.52371732106131e-05	\\
-4035.51802201705	8.20249279055013e-05	\\
-4034.53924005682	8.60548644953469e-05	\\
-4033.56045809659	8.52034649379179e-05	\\
-4032.58167613636	8.48433847730716e-05	\\
-4031.60289417614	8.68263870387194e-05	\\
-4030.62411221591	8.2110886161721e-05	\\
-4029.64533025568	8.33339172871785e-05	\\
-4028.66654829546	8.4152462581581e-05	\\
-4027.68776633523	8.30664500094048e-05	\\
-4026.708984375	8.28879436062494e-05	\\
-4025.73020241477	8.55224898788702e-05	\\
-4024.75142045455	8.40139088013035e-05	\\
-4023.77263849432	8.3346062273819e-05	\\
-4022.79385653409	8.53707440776992e-05	\\
-4021.81507457386	8.44789899709908e-05	\\
-4020.83629261364	8.46305195430656e-05	\\
-4019.85751065341	8.25821069802212e-05	\\
-4018.87872869318	8.27385141031031e-05	\\
-4017.89994673296	8.34723993940067e-05	\\
-4016.92116477273	8.36888337275018e-05	\\
-4015.9423828125	8.10346247137425e-05	\\
-4014.96360085227	8.34170786152987e-05	\\
-4013.98481889205	8.2849595159104e-05	\\
-4013.00603693182	8.33144797523895e-05	\\
-4012.02725497159	8.46268607532077e-05	\\
-4011.04847301136	8.2076859981663e-05	\\
-4010.06969105114	8.5029113162063e-05	\\
-4009.09090909091	8.43409642144639e-05	\\
-4008.11212713068	8.23619702437981e-05	\\
-4007.13334517046	7.90606939420351e-05	\\
-4006.15456321023	8.48413860808118e-05	\\
-4005.17578125	8.4509633438105e-05	\\
-4004.19699928977	8.26627031210677e-05	\\
-4003.21821732955	8.42118505333474e-05	\\
-4002.23943536932	8.19779539327122e-05	\\
-4001.26065340909	8.11108116448971e-05	\\
-4000.28187144886	8.34597993453034e-05	\\
-3999.30308948864	8.33774162190397e-05	\\
-3998.32430752841	8.00908455477029e-05	\\
-3997.34552556818	7.87720208765103e-05	\\
-3996.36674360796	8.00931820191618e-05	\\
-3995.38796164773	8.21473097992736e-05	\\
-3994.4091796875	8.25299282754812e-05	\\
-3993.43039772727	8.31009501027916e-05	\\
-3992.45161576705	7.99796896969027e-05	\\
-3991.47283380682	8.19251135239258e-05	\\
-3990.49405184659	8.29950759173754e-05	\\
-3989.51526988636	8.10190238104072e-05	\\
-3988.53648792614	7.9441776873773e-05	\\
-3987.55770596591	8.18903973146173e-05	\\
-3986.57892400568	7.96822392197581e-05	\\
-3985.60014204546	7.93352118358595e-05	\\
-3984.62136008523	8.22685067460147e-05	\\
-3983.642578125	7.95867614727026e-05	\\
-3982.66379616477	8.06977828421364e-05	\\
-3981.68501420455	7.93269160608016e-05	\\
-3980.70623224432	8.02511771670344e-05	\\
-3979.72745028409	7.92494066678814e-05	\\
-3978.74866832386	7.85603443092131e-05	\\
-3977.76988636364	7.7572796664847e-05	\\
-3976.79110440341	8.00557142594143e-05	\\
-3975.81232244318	7.97021642927456e-05	\\
-3974.83354048296	7.70992087670022e-05	\\
-3973.85475852273	8.01796408974672e-05	\\
-3972.8759765625	8.0147824206918e-05	\\
-3971.89719460227	7.72355161710822e-05	\\
-3970.91841264205	7.69978320792884e-05	\\
-3969.93963068182	7.67086995367814e-05	\\
-3968.96084872159	7.81998276162509e-05	\\
-3967.98206676136	7.65000875449297e-05	\\
-3967.00328480114	7.61766133128395e-05	\\
-3966.02450284091	7.45264859307037e-05	\\
-3965.04572088068	7.47244277468782e-05	\\
-3964.06693892046	7.44640805179689e-05	\\
-3963.08815696023	7.46834937295551e-05	\\
-3962.109375	7.54830706489378e-05	\\
-3961.13059303977	7.56765261636045e-05	\\
-3960.15181107955	7.45075565895751e-05	\\
-3959.17302911932	7.46511783737835e-05	\\
-3958.19424715909	7.56425435419067e-05	\\
-3957.21546519886	7.55921913348682e-05	\\
-3956.23668323864	7.59372987231013e-05	\\
-3955.25790127841	7.30328117951577e-05	\\
-3954.27911931818	7.24146736118259e-05	\\
-3953.30033735796	7.43672323248137e-05	\\
-3952.32155539773	7.41652747090902e-05	\\
-3951.3427734375	7.34942867408595e-05	\\
-3950.36399147727	7.26182991808214e-05	\\
-3949.38520951705	7.37152593667239e-05	\\
-3948.40642755682	7.49401248751328e-05	\\
-3947.42764559659	7.33923217938584e-05	\\
-3946.44886363636	7.20679533931406e-05	\\
-3945.47008167614	7.20246141635164e-05	\\
-3944.49129971591	7.11475175884367e-05	\\
-3943.51251775568	7.39033111462678e-05	\\
-3942.53373579546	7.26243960660476e-05	\\
-3941.55495383523	7.15159895770461e-05	\\
-3940.576171875	6.95060026038409e-05	\\
-3939.59738991477	7.23443055072208e-05	\\
-3938.61860795455	7.17076833151739e-05	\\
-3937.63982599432	6.85511181959022e-05	\\
-3936.66104403409	7.11604221580139e-05	\\
-3935.68226207386	7.02582692974255e-05	\\
-3934.70348011364	7.11904132988697e-05	\\
-3933.72469815341	7.21365607118423e-05	\\
-3932.74591619318	7.17761399199074e-05	\\
-3931.76713423296	6.88650032439394e-05	\\
-3930.78835227273	7.00705193689106e-05	\\
-3929.8095703125	6.80184636721267e-05	\\
-3928.83078835227	6.95706102860186e-05	\\
-3927.85200639205	6.99691463609894e-05	\\
-3926.87322443182	6.95570432589401e-05	\\
-3925.89444247159	6.96697277981556e-05	\\
-3924.91566051136	6.8438557324279e-05	\\
-3923.93687855114	7.12455329284254e-05	\\
-3922.95809659091	6.89610124178354e-05	\\
-3921.97931463068	7.08768620332816e-05	\\
-3921.00053267046	6.98962800626751e-05	\\
-3920.02175071023	6.72741092147798e-05	\\
-3919.04296875	6.9111586500224e-05	\\
-3918.06418678977	6.71094468799104e-05	\\
-3917.08540482955	6.75300938000824e-05	\\
-3916.10662286932	6.81526397925892e-05	\\
-3915.12784090909	6.92434134131104e-05	\\
-3914.14905894886	6.84704563290467e-05	\\
-3913.17027698864	6.6258991004127e-05	\\
-3912.19149502841	6.49587935410592e-05	\\
-3911.21271306818	6.63760254710768e-05	\\
-3910.23393110796	6.81023501189834e-05	\\
-3909.25514914773	6.79240030828299e-05	\\
-3908.2763671875	6.63191365065156e-05	\\
-3907.29758522727	6.82464094800497e-05	\\
-3906.31880326705	6.55451071709775e-05	\\
-3905.34002130682	6.55737172892479e-05	\\
-3904.36123934659	6.42134562985696e-05	\\
-3903.38245738636	6.62506742727579e-05	\\
-3902.40367542614	6.68854976997342e-05	\\
-3901.42489346591	6.41274913744432e-05	\\
-3900.44611150568	6.47769702490694e-05	\\
-3899.46732954546	6.26380371949835e-05	\\
-3898.48854758523	6.58173727015456e-05	\\
-3897.509765625	6.47865381028853e-05	\\
-3896.53098366477	6.39766160998788e-05	\\
-3895.55220170455	6.38705540536707e-05	\\
-3894.57341974432	6.43538192906983e-05	\\
-3893.59463778409	6.41759437103009e-05	\\
-3892.61585582386	6.38409856355751e-05	\\
-3891.63707386364	6.51530544226691e-05	\\
-3890.65829190341	6.3904121615808e-05	\\
-3889.67950994318	6.32506869847685e-05	\\
-3888.70072798296	6.35015586981689e-05	\\
-3887.72194602273	6.47331809457218e-05	\\
-3886.7431640625	6.30721905880881e-05	\\
-3885.76438210227	6.30208507078287e-05	\\
-3884.78560014205	6.34891633154149e-05	\\
-3883.80681818182	6.41435827715091e-05	\\
-3882.82803622159	6.43694503572956e-05	\\
-3881.84925426136	6.26168289588097e-05	\\
-3880.87047230114	6.20686614387769e-05	\\
-3879.89169034091	6.21918675798754e-05	\\
-3878.91290838068	6.19078224637079e-05	\\
-3877.93412642046	6.21334887181955e-05	\\
-3876.95534446023	6.15262403343495e-05	\\
-3875.9765625	6.18594587691485e-05	\\
-3874.99778053977	6.15854293776165e-05	\\
-3874.01899857955	5.94317407569686e-05	\\
-3873.04021661932	6.17389888665748e-05	\\
-3872.06143465909	6.05761981514232e-05	\\
-3871.08265269886	6.00546647916228e-05	\\
-3870.10387073864	6.0709204524258e-05	\\
-3869.12508877841	5.91480537294385e-05	\\
-3868.14630681818	6.0181926058688e-05	\\
-3867.16752485796	6.01115971142898e-05	\\
-3866.18874289773	6.1873569013871e-05	\\
-3865.2099609375	6.00636018377133e-05	\\
-3864.23117897727	6.07697854008272e-05	\\
-3863.25239701705	5.91108658655806e-05	\\
-3862.27361505682	6.11641926534955e-05	\\
-3861.29483309659	5.95650626982315e-05	\\
-3860.31605113636	5.90336999084004e-05	\\
-3859.33726917614	5.65209549614877e-05	\\
-3858.35848721591	5.61760018677057e-05	\\
-3857.37970525568	5.57221511080522e-05	\\
-3856.40092329546	5.70374644900475e-05	\\
-3855.42214133523	5.69520813497048e-05	\\
-3854.443359375	5.61104713616532e-05	\\
-3853.46457741477	5.7616440309945e-05	\\
-3852.48579545455	5.64774992862805e-05	\\
-3851.50701349432	5.5618098412093e-05	\\
-3850.52823153409	5.60905562222534e-05	\\
-3849.54944957386	5.79110324936151e-05	\\
-3848.57066761364	5.59382195623055e-05	\\
-3847.59188565341	5.58655034642275e-05	\\
-3846.61310369318	5.50675953486791e-05	\\
-3845.63432173296	5.32138400264453e-05	\\
-3844.65553977273	5.32141378181602e-05	\\
-3843.6767578125	5.37696119410539e-05	\\
-3842.69797585227	5.20428654401006e-05	\\
-3841.71919389205	5.43482682540691e-05	\\
-3840.74041193182	5.3965896455367e-05	\\
-3839.76162997159	5.30761936442478e-05	\\
-3838.78284801136	5.2753757466664e-05	\\
-3837.80406605114	5.19945489281056e-05	\\
-3836.82528409091	5.10965594574372e-05	\\
-3835.84650213068	4.99802398943761e-05	\\
-3834.86772017046	4.91795773562777e-05	\\
-3833.88893821023	5.14556198762075e-05	\\
-3832.91015625	5.05203122186141e-05	\\
-3831.93137428977	4.99500713571219e-05	\\
-3830.95259232955	5.21597908265593e-05	\\
-3829.97381036932	4.89948149729324e-05	\\
-3828.99502840909	5.18077192997968e-05	\\
-3828.01624644886	5.03278504210859e-05	\\
-3827.03746448864	4.92869655815344e-05	\\
-3826.05868252841	4.85404249979958e-05	\\
-3825.07990056818	5.07738929280643e-05	\\
-3824.10111860796	5.08928290693344e-05	\\
-3823.12233664773	4.81735489981731e-05	\\
-3822.1435546875	4.98304648354514e-05	\\
-3821.16477272727	4.63273496382022e-05	\\
-3820.18599076705	5.00142192602741e-05	\\
-3819.20720880682	5.02744296257562e-05	\\
-3818.22842684659	4.92496227274518e-05	\\
-3817.24964488636	4.98404794663958e-05	\\
-3816.27086292614	5.12347928847826e-05	\\
-3815.29208096591	5.09797035712477e-05	\\
-3814.31329900568	5.10217939994528e-05	\\
-3813.33451704546	5.11259028769771e-05	\\
-3812.35573508523	5.1564624521222e-05	\\
-3811.376953125	4.98138051668705e-05	\\
-3810.39817116477	5.18138347831621e-05	\\
-3809.41938920455	5.19940144437224e-05	\\
-3808.44060724432	5.21706204933827e-05	\\
-3807.46182528409	5.10129403022596e-05	\\
-3806.48304332386	5.16882326173925e-05	\\
-3805.50426136364	5.15695789747535e-05	\\
-3804.52547940341	5.21980879530903e-05	\\
-3803.54669744318	5.11166056629124e-05	\\
-3802.56791548296	5.23707108503667e-05	\\
-3801.58913352273	5.23694637377453e-05	\\
-3800.6103515625	5.11572023735607e-05	\\
-3799.63156960227	5.27771254704745e-05	\\
-3798.65278764205	5.40274032256627e-05	\\
-3797.67400568182	5.1110794827529e-05	\\
-3796.69522372159	4.95326960328218e-05	\\
-3795.71644176136	5.22998354065478e-05	\\
-3794.73765980114	5.25850353591083e-05	\\
-3793.75887784091	5.35735361149549e-05	\\
-3792.78009588068	5.47005957597376e-05	\\
-3791.80131392046	5.12659670483311e-05	\\
-3790.82253196023	5.11609144172272e-05	\\
-3789.84375	5.39621054593514e-05	\\
-3788.86496803977	5.35458361555634e-05	\\
-3787.88618607955	5.07746527966685e-05	\\
-3786.90740411932	5.33470439593475e-05	\\
-3785.92862215909	5.16861432391699e-05	\\
-3784.94984019886	5.25579128684022e-05	\\
-3783.97105823864	5.293269709796e-05	\\
-3782.99227627841	5.40071612794723e-05	\\
-3782.01349431818	5.43261973293283e-05	\\
-3781.03471235796	5.23070010145028e-05	\\
-3780.05593039773	5.20828394672173e-05	\\
-3779.0771484375	5.26508935515129e-05	\\
-3778.09836647727	5.16833329128906e-05	\\
-3777.11958451705	5.24596824200012e-05	\\
-3776.14080255682	5.3231191857084e-05	\\
-3775.16202059659	5.18452538794321e-05	\\
-3774.18323863636	5.11751365446968e-05	\\
-3773.20445667614	5.30197116089014e-05	\\
-3772.22567471591	5.23817738681829e-05	\\
-3771.24689275568	5.09882718770439e-05	\\
-3770.26811079546	4.93481391953226e-05	\\
-3769.28932883523	4.84714691225718e-05	\\
-3768.310546875	5.21185082703813e-05	\\
-3767.33176491477	5.32514216128372e-05	\\
-3766.35298295455	5.26165098457775e-05	\\
-3765.37420099432	5.19982120179273e-05	\\
-3764.39541903409	5.17321432392243e-05	\\
-3763.41663707386	5.0308307875758e-05	\\
-3762.43785511364	5.0774661487939e-05	\\
-3761.45907315341	4.9683458792158e-05	\\
-3760.48029119318	5.06076688045789e-05	\\
-3759.50150923296	5.14150095638268e-05	\\
-3758.52272727273	5.00322453271244e-05	\\
-3757.5439453125	4.91865525859681e-05	\\
-3756.56516335227	4.97269602724454e-05	\\
-3755.58638139205	5.01325577756612e-05	\\
-3754.60759943182	5.18298024015877e-05	\\
-3753.62881747159	5.01893105953232e-05	\\
-3752.65003551136	5.12389097641488e-05	\\
-3751.67125355114	5.14130376097197e-05	\\
-3750.69247159091	4.93513958509188e-05	\\
-3749.71368963068	5.18006418974728e-05	\\
-3748.73490767046	4.7496847419387e-05	\\
-3747.75612571023	4.90141669477829e-05	\\
-3746.77734375	4.97790096068211e-05	\\
-3745.79856178977	4.85626540744485e-05	\\
-3744.81977982955	4.92385987809082e-05	\\
-3743.84099786932	4.91152459320046e-05	\\
-3742.86221590909	4.887200383708e-05	\\
-3741.88343394886	4.94982796912489e-05	\\
-3740.90465198864	4.91559670175633e-05	\\
-3739.92587002841	4.70782245423549e-05	\\
-3738.94708806818	4.7467727941902e-05	\\
-3737.96830610796	4.88766906515931e-05	\\
-3736.98952414773	4.64582887984241e-05	\\
-3736.0107421875	4.58111973264176e-05	\\
-3735.03196022727	4.69287063921319e-05	\\
-3734.05317826705	4.7073167130736e-05	\\
-3733.07439630682	4.83866236133463e-05	\\
-3732.09561434659	4.67632138288794e-05	\\
-3731.11683238636	4.4707443551977e-05	\\
-3730.13805042614	4.77671652190434e-05	\\
-3729.15926846591	4.95204928995875e-05	\\
-3728.18048650568	4.54010970824803e-05	\\
-3727.20170454546	4.49845132528553e-05	\\
-3726.22292258523	4.85194949483734e-05	\\
-3725.244140625	4.90223967290906e-05	\\
-3724.26535866477	4.54825187388606e-05	\\
-3723.28657670455	4.32990504404731e-05	\\
-3722.30779474432	4.59485192131969e-05	\\
-3721.32901278409	4.21304833367251e-05	\\
-3720.35023082386	4.43193292411948e-05	\\
-3719.37144886364	4.32483607492107e-05	\\
-3718.39266690341	4.45121999081128e-05	\\
-3717.41388494318	4.42243032390894e-05	\\
-3716.43510298296	4.32019125061052e-05	\\
-3715.45632102273	4.53058873471634e-05	\\
-3714.4775390625	4.44666719900217e-05	\\
-3713.49875710227	4.54789185357714e-05	\\
-3712.51997514205	4.46765444410173e-05	\\
-3711.54119318182	4.26819808036159e-05	\\
-3710.56241122159	4.50818252104888e-05	\\
-3709.58362926136	4.37250220548816e-05	\\
-3708.60484730114	4.35655167605668e-05	\\
-3707.62606534091	4.3187985127312e-05	\\
-3706.64728338068	4.22550882617605e-05	\\
-3705.66850142046	4.21138748553262e-05	\\
-3704.68971946023	4.36963506273122e-05	\\
-3703.7109375	4.27760586175734e-05	\\
-3702.73215553977	4.44175878228196e-05	\\
-3701.75337357955	4.22486557537034e-05	\\
-3700.77459161932	4.27661779083117e-05	\\
-3699.79580965909	4.14672707830449e-05	\\
-3698.81702769886	4.32919916253372e-05	\\
-3697.83824573864	4.26743913006083e-05	\\
-3696.85946377841	4.38765568122022e-05	\\
-3695.88068181818	4.25676168496949e-05	\\
-3694.90189985796	4.08689566064302e-05	\\
-3693.92311789773	4.38145229987531e-05	\\
-3692.9443359375	4.39475553694997e-05	\\
-3691.96555397727	3.9585894215813e-05	\\
-3690.98677201705	4.11708063019935e-05	\\
-3690.00799005682	4.12876100949182e-05	\\
-3689.02920809659	3.8999603220473e-05	\\
-3688.05042613636	4.42531684022799e-05	\\
-3687.07164417614	4.14691028283351e-05	\\
-3686.09286221591	4.03290025824336e-05	\\
-3685.11408025568	3.87622318865517e-05	\\
-3684.13529829546	4.05438536928292e-05	\\
-3683.15651633523	3.9697727116545e-05	\\
-3682.177734375	3.63981138652181e-05	\\
-3681.19895241477	4.24176305534557e-05	\\
-3680.22017045455	3.8235032198823e-05	\\
-3679.24138849432	3.80102030080849e-05	\\
-3678.26260653409	3.71334671896015e-05	\\
-3677.28382457386	3.82950488474347e-05	\\
-3676.30504261364	4.08041265941592e-05	\\
-3675.32626065341	3.93804938762789e-05	\\
-3674.34747869318	3.75808599879563e-05	\\
-3673.36869673296	3.76407057671956e-05	\\
-3672.38991477273	3.87576998229094e-05	\\
-3671.4111328125	3.82125072978215e-05	\\
-3670.43235085227	3.73742232369591e-05	\\
-3669.45356889205	4.029886859317e-05	\\
-3668.47478693182	3.7103108130636e-05	\\
-3667.49600497159	3.96790722469997e-05	\\
-3666.51722301136	3.70898453263436e-05	\\
-3665.53844105114	3.86118604622875e-05	\\
-3664.55965909091	3.72427061914332e-05	\\
-3663.58087713068	4.06749853467893e-05	\\
-3662.60209517046	3.82677036695606e-05	\\
-3661.62331321023	3.82872632057377e-05	\\
-3660.64453125	4.00079251465127e-05	\\
-3659.66574928977	3.80540068627075e-05	\\
-3658.68696732955	3.81928832516541e-05	\\
-3657.70818536932	3.78730293990579e-05	\\
-3656.72940340909	3.96381393768439e-05	\\
-3655.75062144886	3.7704009262909e-05	\\
-3654.77183948864	3.72880176919304e-05	\\
-3653.79305752841	4.20067378145406e-05	\\
-3652.81427556818	3.84352808094774e-05	\\
-3651.83549360796	3.93632103128577e-05	\\
-3650.85671164773	3.91402808492986e-05	\\
-3649.8779296875	3.94911843896974e-05	\\
-3648.89914772727	3.9329957852397e-05	\\
-3647.92036576705	3.94405417524812e-05	\\
-3646.94158380682	3.92665891053496e-05	\\
-3645.96280184659	3.86822909357547e-05	\\
-3644.98401988636	3.96208093372965e-05	\\
-3644.00523792614	3.83871463126835e-05	\\
-3643.02645596591	3.69974536490568e-05	\\
-3642.04767400568	4.04358558688976e-05	\\
-3641.06889204546	3.92862342419321e-05	\\
-3640.09011008523	4.04545040399068e-05	\\
-3639.111328125	3.80520625213307e-05	\\
-3638.13254616477	3.87553016581048e-05	\\
-3637.15376420455	3.97062495802118e-05	\\
-3636.17498224432	4.11049097636587e-05	\\
-3635.19620028409	3.86973702758838e-05	\\
-3634.21741832386	3.75880253970649e-05	\\
-3633.23863636364	3.86276755435753e-05	\\
-3632.25985440341	4.01400211836028e-05	\\
-3631.28107244318	3.83670630029043e-05	\\
-3630.30229048296	3.79559879538148e-05	\\
-3629.32350852273	3.98439163384367e-05	\\
-3628.3447265625	4.09019518803247e-05	\\
-3627.36594460227	3.95531859391507e-05	\\
-3626.38716264205	3.77594777179571e-05	\\
-3625.40838068182	3.95750376151756e-05	\\
-3624.42959872159	4.05203301194777e-05	\\
-3623.45081676136	4.16576663096037e-05	\\
-3622.47203480114	3.93868054774497e-05	\\
-3621.49325284091	4.01681645338422e-05	\\
-3620.51447088068	4.15832840442682e-05	\\
-3619.53568892046	4.13551478385204e-05	\\
-3618.55690696023	4.06010072001338e-05	\\
-3617.578125	4.12248539503652e-05	\\
-3616.59934303977	4.0352727304138e-05	\\
-3615.62056107955	4.19169523141712e-05	\\
-3614.64177911932	3.99594648417113e-05	\\
-3613.66299715909	4.14160554648722e-05	\\
-3612.68421519886	4.04076421488901e-05	\\
-3611.70543323864	4.11313961297879e-05	\\
-3610.72665127841	3.93092290090835e-05	\\
-3609.74786931818	3.94025574803305e-05	\\
-3608.76908735796	4.20721816323456e-05	\\
-3607.79030539773	4.07139339214312e-05	\\
-3606.8115234375	4.30140608957201e-05	\\
-3605.83274147727	4.02244337386981e-05	\\
-3604.85395951705	4.12371007861484e-05	\\
-3603.87517755682	3.84912127600725e-05	\\
-3602.89639559659	4.03992034045248e-05	\\
-3601.91761363636	4.17590296926622e-05	\\
-3600.93883167614	3.88014394888718e-05	\\
-3599.96004971591	4.13433596692525e-05	\\
-3598.98126775568	4.19136069800056e-05	\\
-3598.00248579546	3.80840153175029e-05	\\
-3597.02370383523	4.16337759332213e-05	\\
-3596.044921875	4.24001536232566e-05	\\
-3595.06613991477	4.29038281118927e-05	\\
-3594.08735795455	4.06246274252626e-05	\\
-3593.10857599432	4.42884438183583e-05	\\
-3592.12979403409	3.95734578877951e-05	\\
-3591.15101207386	4.2357771476413e-05	\\
-3590.17223011364	4.22028216961429e-05	\\
-3589.19344815341	4.08584235539245e-05	\\
-3588.21466619318	4.10539490120135e-05	\\
-3587.23588423296	3.95584911092386e-05	\\
-3586.25710227273	4.0344837342813e-05	\\
-3585.2783203125	4.20921744496378e-05	\\
-3584.29953835227	4.0788534804023e-05	\\
-3583.32075639205	4.15242098234815e-05	\\
-3582.34197443182	3.98755914638738e-05	\\
-3581.36319247159	3.99510767919719e-05	\\
-3580.38441051136	4.28600175555246e-05	\\
-3579.40562855114	4.05279524354438e-05	\\
-3578.42684659091	4.06609595002274e-05	\\
-3577.44806463068	3.97213493967883e-05	\\
-3576.46928267046	4.12352813269973e-05	\\
-3575.49050071023	3.95214124589203e-05	\\
-3574.51171875	3.94539073983376e-05	\\
-3573.53293678977	4.31710679439864e-05	\\
-3572.55415482955	3.90843287768188e-05	\\
-3571.57537286932	4.03774792638663e-05	\\
-3570.59659090909	3.87475610989521e-05	\\
-3569.61780894886	4.13331167239201e-05	\\
-3568.63902698864	4.1795653398356e-05	\\
-3567.66024502841	4.3119208580491e-05	\\
-3566.68146306818	4.1110527662796e-05	\\
-3565.70268110796	4.37311481980006e-05	\\
-3564.72389914773	4.21752788474431e-05	\\
-3563.7451171875	3.77732404261588e-05	\\
-3562.76633522727	4.2184076099665e-05	\\
-3561.78755326705	4.13912221348337e-05	\\
-3560.80877130682	4.23677952234134e-05	\\
-3559.82998934659	4.32870489390452e-05	\\
-3558.85120738636	4.21546859260163e-05	\\
-3557.87242542614	4.11190792942232e-05	\\
-3556.89364346591	4.1491238947762e-05	\\
-3555.91486150568	3.90516639204534e-05	\\
-3554.93607954546	4.10441343444174e-05	\\
-3553.95729758523	4.25980515992553e-05	\\
-3552.978515625	4.30237828538262e-05	\\
-3551.99973366477	4.04579925096481e-05	\\
-3551.02095170455	4.30704402560189e-05	\\
-3550.04216974432	4.25444221127675e-05	\\
-3549.06338778409	4.03020825250206e-05	\\
-3548.08460582386	4.28445114415932e-05	\\
-3547.10582386364	4.1666418489657e-05	\\
-3546.12704190341	4.20966715281125e-05	\\
-3545.14825994318	4.23392991400619e-05	\\
-3544.16947798296	3.94763687222195e-05	\\
-3543.19069602273	3.94646511383258e-05	\\
-3542.2119140625	4.05592914523559e-05	\\
-3541.23313210227	4.24958057778697e-05	\\
-3540.25435014205	4.12622489827318e-05	\\
-3539.27556818182	4.37157834078019e-05	\\
-3538.29678622159	4.00345566876998e-05	\\
-3537.31800426136	4.36881774410363e-05	\\
-3536.33922230114	4.11022822073613e-05	\\
-3535.36044034091	4.2659133700273e-05	\\
-3534.38165838068	4.24654406485526e-05	\\
-3533.40287642046	4.39817181456959e-05	\\
-3532.42409446023	4.0048637042974e-05	\\
-3531.4453125	4.02777018015511e-05	\\
-3530.46653053977	4.47500013783869e-05	\\
-3529.48774857955	4.36391154304534e-05	\\
-3528.50896661932	4.38002149016411e-05	\\
-3527.53018465909	4.45413588502033e-05	\\
-3526.55140269886	4.50644700247493e-05	\\
-3525.57262073864	4.38280729967319e-05	\\
-3524.59383877841	4.32280611816044e-05	\\
-3523.61505681818	4.59783500723744e-05	\\
-3522.63627485796	4.530211437965e-05	\\
-3521.65749289773	4.43058351137217e-05	\\
-3520.6787109375	4.43335841522289e-05	\\
-3519.69992897727	4.25116567296119e-05	\\
-3518.72114701705	4.56546353917837e-05	\\
-3517.74236505682	4.42268076303994e-05	\\
-3516.76358309659	4.3752747526609e-05	\\
-3515.78480113636	4.74694566609769e-05	\\
-3514.80601917614	4.76418819132314e-05	\\
-3513.82723721591	4.61321388130339e-05	\\
-3512.84845525568	4.44357548789921e-05	\\
-3511.86967329546	4.54175133758184e-05	\\
-3510.89089133523	4.67449365125098e-05	\\
-3509.912109375	4.4960010440694e-05	\\
-3508.93332741477	4.64241017218653e-05	\\
-3507.95454545455	4.35525152754587e-05	\\
-3506.97576349432	4.54189822779985e-05	\\
-3505.99698153409	4.59942907338422e-05	\\
-3505.01819957386	4.59535245833293e-05	\\
-3504.03941761364	4.82587481712401e-05	\\
-3503.06063565341	4.70282247729772e-05	\\
-3502.08185369318	4.7340520194908e-05	\\
-3501.10307173296	4.76102741472143e-05	\\
-3500.12428977273	4.61355738837145e-05	\\
-3499.1455078125	4.75377629140373e-05	\\
-3498.16672585227	4.7128157267359e-05	\\
-3497.18794389205	4.5380671267914e-05	\\
-3496.20916193182	4.56931362793712e-05	\\
-3495.23037997159	4.4536408992094e-05	\\
-3494.25159801136	4.66189686244779e-05	\\
-3493.27281605114	4.7644474289189e-05	\\
-3492.29403409091	4.82808743017239e-05	\\
-3491.31525213068	4.71279578651373e-05	\\
-3490.33647017046	4.6714701222714e-05	\\
-3489.35768821023	4.43526718159176e-05	\\
-3488.37890625	4.49328633374628e-05	\\
-3487.40012428977	4.60268632751805e-05	\\
-3486.42134232955	4.68238886339859e-05	\\
-3485.44256036932	4.58769732802532e-05	\\
-3484.46377840909	4.41910766922834e-05	\\
-3483.48499644886	4.54284034656416e-05	\\
-3482.50621448864	4.56798765769716e-05	\\
-3481.52743252841	4.52200522494928e-05	\\
-3480.54865056818	4.55420852420568e-05	\\
-3479.56986860796	4.64432986572641e-05	\\
-3478.59108664773	4.56987859504057e-05	\\
-3477.6123046875	4.73562573692123e-05	\\
-3476.63352272727	4.45376170399024e-05	\\
-3475.65474076705	4.71504878540732e-05	\\
-3474.67595880682	4.34084425178137e-05	\\
-3473.69717684659	4.79496328190633e-05	\\
-3472.71839488636	4.62486824292251e-05	\\
-3471.73961292614	4.64489575304372e-05	\\
-3470.76083096591	4.69067876895111e-05	\\
-3469.78204900568	4.78361053118497e-05	\\
-3468.80326704546	4.44603751714299e-05	\\
-3467.82448508523	4.59787618663981e-05	\\
-3466.845703125	4.71879545925288e-05	\\
-3465.86692116477	4.52794914158213e-05	\\
-3464.88813920455	4.44253433958194e-05	\\
-3463.90935724432	4.37605486099262e-05	\\
-3462.93057528409	4.5058783947415e-05	\\
-3461.95179332386	4.61405334035497e-05	\\
-3460.97301136364	4.81622926063327e-05	\\
-3459.99422940341	4.529291502842e-05	\\
-3459.01544744318	4.53748919538021e-05	\\
-3458.03666548296	4.68678057521889e-05	\\
-3457.05788352273	4.48309061584547e-05	\\
-3456.0791015625	4.5416971436649e-05	\\
-3455.10031960227	4.70604265002694e-05	\\
-3454.12153764205	4.8275804051729e-05	\\
-3453.14275568182	4.66251627450559e-05	\\
-3452.16397372159	4.68345308939946e-05	\\
-3451.18519176136	4.84517110695884e-05	\\
-3450.20640980114	4.52267461755825e-05	\\
-3449.22762784091	4.76520131790565e-05	\\
-3448.24884588068	4.59601421057777e-05	\\
-3447.27006392046	4.75968829060132e-05	\\
-3446.29128196023	4.6246377811968e-05	\\
-3445.3125	4.57863285341811e-05	\\
-3444.33371803977	4.71984051821329e-05	\\
-3443.35493607955	4.65419572596551e-05	\\
-3442.37615411932	4.5749487615822e-05	\\
-3441.39737215909	4.7128183090052e-05	\\
-3440.41859019886	4.72152862524229e-05	\\
-3439.43980823864	4.76016355709742e-05	\\
-3438.46102627841	4.79725580167525e-05	\\
-3437.48224431818	4.73756084224398e-05	\\
-3436.50346235796	4.90379896307467e-05	\\
-3435.52468039773	4.61066501348366e-05	\\
-3434.5458984375	4.72316603792089e-05	\\
-3433.56711647727	4.50725224839193e-05	\\
-3432.58833451705	4.57245027001536e-05	\\
-3431.60955255682	4.48791331679605e-05	\\
-3430.63077059659	4.84217260354735e-05	\\
-3429.65198863636	4.81014007744798e-05	\\
-3428.67320667614	4.98297460441504e-05	\\
-3427.69442471591	4.94116033347289e-05	\\
-3426.71564275568	4.64538850452643e-05	\\
-3425.73686079546	4.92629755003958e-05	\\
-3424.75807883523	4.86751675601306e-05	\\
-3423.779296875	5.04515750892471e-05	\\
-3422.80051491477	4.65290985403044e-05	\\
-3421.82173295455	4.77853050921342e-05	\\
-3420.84295099432	4.8010918425472e-05	\\
-3419.86416903409	4.76558015351622e-05	\\
-3418.88538707386	4.82455279286958e-05	\\
-3417.90660511364	4.7726453456157e-05	\\
-3416.92782315341	4.80129051770051e-05	\\
-3415.94904119318	4.81524689234702e-05	\\
-3414.97025923296	5.15430720042452e-05	\\
-3413.99147727273	4.85058059164649e-05	\\
-3413.0126953125	5.15985593638413e-05	\\
-3412.03391335227	5.08452831311544e-05	\\
-3411.05513139205	5.17811262439752e-05	\\
-3410.07634943182	4.99662399948792e-05	\\
-3409.09756747159	4.9126010618606e-05	\\
-3408.11878551136	4.98939960073473e-05	\\
-3407.14000355114	5.15384236164131e-05	\\
-3406.16122159091	5.05270011795487e-05	\\
-3405.18243963068	5.30999749532326e-05	\\
-3404.20365767046	5.2434799675444e-05	\\
-3403.22487571023	5.05278480891587e-05	\\
-3402.24609375	5.00246613063859e-05	\\
-3401.26731178977	5.07617774667605e-05	\\
-3400.28852982955	4.92488407543448e-05	\\
-3399.30974786932	5.09980503747598e-05	\\
-3398.33096590909	5.14688303747629e-05	\\
-3397.35218394886	5.0741247189499e-05	\\
-3396.37340198864	4.90376015462953e-05	\\
-3395.39462002841	5.28151892681635e-05	\\
-3394.41583806818	5.14897717426115e-05	\\
-3393.43705610796	5.51545374372882e-05	\\
-3392.45827414773	5.05094712008292e-05	\\
-3391.4794921875	5.11103579431298e-05	\\
-3390.50071022727	5.11080645862914e-05	\\
-3389.52192826705	5.06111633038365e-05	\\
-3388.54314630682	5.35642178395792e-05	\\
-3387.56436434659	5.21329014900205e-05	\\
-3386.58558238636	5.20588028381819e-05	\\
-3385.60680042614	5.24375673704428e-05	\\
-3384.62801846591	5.23165613237087e-05	\\
-3383.64923650568	5.26683582270135e-05	\\
-3382.67045454546	5.23993305322231e-05	\\
-3381.69167258523	5.21868014977409e-05	\\
-3380.712890625	5.13779602975566e-05	\\
-3379.73410866477	5.14353462365316e-05	\\
-3378.75532670455	5.26679009895619e-05	\\
-3377.77654474432	5.27228362354139e-05	\\
-3376.79776278409	5.30446264449575e-05	\\
-3375.81898082386	5.3603283558791e-05	\\
-3374.84019886364	5.1949050455237e-05	\\
-3373.86141690341	5.2248162238169e-05	\\
-3372.88263494318	5.26426481538416e-05	\\
-3371.90385298296	4.91500688839769e-05	\\
-3370.92507102273	5.3167283439614e-05	\\
-3369.9462890625	5.07077619093295e-05	\\
-3368.96750710227	5.23859202187424e-05	\\
-3367.98872514205	5.14624645691474e-05	\\
-3367.00994318182	5.24506406418255e-05	\\
-3366.03116122159	5.15687812357986e-05	\\
-3365.05237926136	5.19657677640007e-05	\\
-3364.07359730114	5.36437587677311e-05	\\
-3363.09481534091	5.27574930400833e-05	\\
-3362.11603338068	5.41379694862146e-05	\\
-3361.13725142046	5.42897455904587e-05	\\
-3360.15846946023	5.1252118888628e-05	\\
-3359.1796875	5.44149605162082e-05	\\
-3358.20090553977	5.22345346103172e-05	\\
-3357.22212357955	5.40959482636595e-05	\\
-3356.24334161932	5.33417963092772e-05	\\
-3355.26455965909	5.3212709080193e-05	\\
-3354.28577769886	5.34736641101127e-05	\\
-3353.30699573864	5.34567062128113e-05	\\
-3352.32821377841	5.28207948066701e-05	\\
-3351.34943181818	5.35323676875809e-05	\\
-3350.37064985796	5.24943145802123e-05	\\
-3349.39186789773	5.42714346936052e-05	\\
-3348.4130859375	5.41133951757984e-05	\\
-3347.43430397727	5.36196775185393e-05	\\
-3346.45552201705	5.55359338794254e-05	\\
-3345.47674005682	5.35615601036616e-05	\\
-3344.49795809659	5.27925929567576e-05	\\
-3343.51917613636	5.20758559267219e-05	\\
-3342.54039417614	4.94422164401982e-05	\\
-3341.56161221591	5.34845450250262e-05	\\
-3340.58283025568	4.93947332799673e-05	\\
-3339.60404829546	5.26351535118901e-05	\\
-3338.62526633523	5.29229849309779e-05	\\
-3337.646484375	5.02585369107838e-05	\\
-3336.66770241477	5.0717687787048e-05	\\
-3335.68892045455	5.28170615012429e-05	\\
-3334.71013849432	5.26283681862908e-05	\\
-3333.73135653409	5.0210786437445e-05	\\
-3332.75257457386	5.31284543422281e-05	\\
-3331.77379261364	5.32518884148425e-05	\\
-3330.79501065341	5.03918285916274e-05	\\
-3329.81622869318	5.2704484256258e-05	\\
-3328.83744673296	5.38644920529503e-05	\\
-3327.85866477273	5.0378735710641e-05	\\
-3326.8798828125	5.13921456618147e-05	\\
-3325.90110085227	5.24792302372581e-05	\\
-3324.92231889205	5.41127242550012e-05	\\
-3323.94353693182	5.27951060481202e-05	\\
-3322.96475497159	5.19827010075905e-05	\\
-3321.98597301136	5.07703567947888e-05	\\
-3321.00719105114	5.12262947582285e-05	\\
-3320.02840909091	5.28073598569126e-05	\\
-3319.04962713068	5.21814811505675e-05	\\
-3318.07084517046	5.23336042661321e-05	\\
-3317.09206321023	4.97543388852248e-05	\\
-3316.11328125	5.11895350489683e-05	\\
-3315.13449928977	5.23298323025473e-05	\\
-3314.15571732955	4.96095249702646e-05	\\
-3313.17693536932	5.20364530750661e-05	\\
-3312.19815340909	5.00576222879465e-05	\\
-3311.21937144886	5.09473266488535e-05	\\
-3310.24058948864	4.99374500501349e-05	\\
-3309.26180752841	5.03731540821242e-05	\\
-3308.28302556818	5.32287682916784e-05	\\
-3307.30424360796	4.82328215256514e-05	\\
-3306.32546164773	5.31272308029514e-05	\\
-3305.3466796875	5.34113137319996e-05	\\
-3304.36789772727	4.95550738186779e-05	\\
-3303.38911576705	5.07350262681806e-05	\\
-3302.41033380682	5.20179968955105e-05	\\
-3301.43155184659	4.90386113912221e-05	\\
-3300.45276988636	5.21185270229354e-05	\\
-3299.47398792614	4.98241997041401e-05	\\
-3298.49520596591	5.12256194603219e-05	\\
-3297.51642400568	4.90582331061618e-05	\\
-3296.53764204546	4.93552450384913e-05	\\
-3295.55886008523	5.0020632770343e-05	\\
-3294.580078125	4.94192131215545e-05	\\
-3293.60129616477	5.09844353850561e-05	\\
-3292.62251420455	4.89566459072943e-05	\\
-3291.64373224432	4.89678467037973e-05	\\
-3290.66495028409	4.7720183644832e-05	\\
-3289.68616832386	4.77819865057671e-05	\\
-3288.70738636364	4.7484522655949e-05	\\
-3287.72860440341	4.69717993610907e-05	\\
-3286.74982244318	4.91666192719022e-05	\\
-3285.77104048296	4.77668288676465e-05	\\
-3284.79225852273	4.8110326110231e-05	\\
-3283.8134765625	4.66738222867203e-05	\\
-3282.83469460227	4.51244214938203e-05	\\
-3281.85591264205	4.45108775188835e-05	\\
-3280.87713068182	4.5859622880032e-05	\\
-3279.89834872159	4.60281609307521e-05	\\
-3278.91956676136	4.47993275321321e-05	\\
-3277.94078480114	4.69224417851518e-05	\\
-3276.96200284091	4.41199206205471e-05	\\
-3275.98322088068	4.423143158138e-05	\\
-3275.00443892046	4.32649310298536e-05	\\
-3274.02565696023	4.37685754170727e-05	\\
-3273.046875	4.43150087358046e-05	\\
-3272.06809303977	4.36239975911334e-05	\\
-3271.08931107955	4.10833564129976e-05	\\
-3270.11052911932	4.20731305714153e-05	\\
-3269.13174715909	4.44398747120446e-05	\\
-3268.15296519886	4.27459137140029e-05	\\
-3267.17418323864	4.33059609665212e-05	\\
-3266.19540127841	4.47033902259115e-05	\\
-3265.21661931818	4.47608660464203e-05	\\
-3264.23783735796	4.40253909350521e-05	\\
-3263.25905539773	4.26742040401251e-05	\\
-3262.2802734375	4.10407107781447e-05	\\
-3261.30149147727	4.343683479493e-05	\\
-3260.32270951705	4.34502827503277e-05	\\
-3259.34392755682	4.13335714677769e-05	\\
-3258.36514559659	4.2735985987338e-05	\\
-3257.38636363636	4.06295051392251e-05	\\
-3256.40758167614	4.26025554055588e-05	\\
-3255.42879971591	4.02837127483814e-05	\\
-3254.45001775568	4.06831797116501e-05	\\
-3253.47123579546	4.11398692436208e-05	\\
-3252.49245383523	3.93890490046893e-05	\\
-3251.513671875	3.96568022445889e-05	\\
-3250.53488991477	3.94650178833629e-05	\\
-3249.55610795455	3.94834804434126e-05	\\
-3248.57732599432	4.06216344393081e-05	\\
-3247.59854403409	3.93543637780014e-05	\\
-3246.61976207386	3.85042765618251e-05	\\
-3245.64098011364	4.10608296911008e-05	\\
-3244.66219815341	3.83937225248458e-05	\\
-3243.68341619318	3.75545279089725e-05	\\
-3242.70463423296	3.73054947921756e-05	\\
-3241.72585227273	3.72942052248536e-05	\\
-3240.7470703125	3.78655328203314e-05	\\
-3239.76828835227	3.74503954295125e-05	\\
-3238.78950639205	3.72182389958807e-05	\\
-3237.81072443182	3.65696189750922e-05	\\
-3236.83194247159	3.38078010409376e-05	\\
-3235.85316051136	3.67247535766509e-05	\\
-3234.87437855114	3.58723037255397e-05	\\
-3233.89559659091	3.48976951795053e-05	\\
-3232.91681463068	3.58682571050846e-05	\\
-3231.93803267046	3.59993054051283e-05	\\
-3230.95925071023	3.77866015849735e-05	\\
-3229.98046875	3.52859322489456e-05	\\
-3229.00168678977	3.77075181586432e-05	\\
-3228.02290482955	3.38437513156039e-05	\\
-3227.04412286932	3.34813202355208e-05	\\
-3226.06534090909	3.2218202967311e-05	\\
-3225.08655894886	3.51154569793037e-05	\\
-3224.10777698864	3.42376285150137e-05	\\
-3223.12899502841	3.40265980547617e-05	\\
-3222.15021306818	3.23175933991142e-05	\\
-3221.17143110796	3.5083154061928e-05	\\
-3220.19264914773	3.46384052450149e-05	\\
-3219.2138671875	3.28973871032374e-05	\\
-3218.23508522727	3.15621456640487e-05	\\
-3217.25630326705	3.54034432939482e-05	\\
-3216.27752130682	3.24588702263055e-05	\\
-3215.29873934659	3.2785348452528e-05	\\
-3214.31995738636	3.27751000879112e-05	\\
-3213.34117542614	3.2511903293932e-05	\\
-3212.36239346591	3.22004467389314e-05	\\
-3211.38361150568	3.1173647998022e-05	\\
-3210.40482954546	2.9580052014934e-05	\\
-3209.42604758523	3.10480775017174e-05	\\
-3208.447265625	3.25609248324948e-05	\\
-3207.46848366477	3.25058758627663e-05	\\
-3206.48970170455	3.17825960657084e-05	\\
-3205.51091974432	3.08843163339162e-05	\\
-3204.53213778409	3.12760721725413e-05	\\
-3203.55335582386	2.77044041912007e-05	\\
-3202.57457386364	3.03784034511564e-05	\\
-3201.59579190341	3.01540231695694e-05	\\
-3200.61700994318	3.16642444542786e-05	\\
-3199.63822798296	2.85501743508355e-05	\\
-3198.65944602273	2.99728262043207e-05	\\
-3197.6806640625	3.26572054067993e-05	\\
-3196.70188210227	2.78396672826225e-05	\\
-3195.72310014205	2.78738576687453e-05	\\
-3194.74431818182	2.86065666714102e-05	\\
};
\addplot [color=blue,solid,forget plot]
  table[row sep=crcr]{
-3194.74431818182	2.86065666714102e-05	\\
-3193.76553622159	2.66317632661718e-05	\\
-3192.78675426136	2.9599488714336e-05	\\
-3191.80797230114	2.7786191432948e-05	\\
-3190.82919034091	2.97157961029983e-05	\\
-3189.85040838068	2.86897501305598e-05	\\
-3188.87162642046	2.67726131569518e-05	\\
-3187.89284446023	2.94729172879224e-05	\\
-3186.9140625	2.58586678671612e-05	\\
-3185.93528053977	2.60861806164505e-05	\\
-3184.95649857955	2.77174277273676e-05	\\
-3183.97771661932	2.68182086897995e-05	\\
-3182.99893465909	3.06900148008257e-05	\\
-3182.02015269886	2.80923078100831e-05	\\
-3181.04137073864	2.82989047842273e-05	\\
-3180.06258877841	2.79307204821153e-05	\\
-3179.08380681818	2.68698524401867e-05	\\
-3178.10502485796	2.74433952665603e-05	\\
-3177.12624289773	2.73081290323837e-05	\\
-3176.1474609375	2.63117546856178e-05	\\
-3175.16867897727	2.97388355983029e-05	\\
-3174.18989701705	2.99524237269425e-05	\\
-3173.21111505682	2.8266320250824e-05	\\
-3172.23233309659	2.81798722859114e-05	\\
-3171.25355113636	2.68349334093725e-05	\\
-3170.27476917614	2.74879992193556e-05	\\
-3169.29598721591	2.88397873394234e-05	\\
-3168.31720525568	2.76782239273165e-05	\\
-3167.33842329546	2.92394528944405e-05	\\
-3166.35964133523	2.68788330985011e-05	\\
-3165.380859375	2.82580701752575e-05	\\
-3164.40207741477	2.55535092760743e-05	\\
-3163.42329545455	2.65746526729145e-05	\\
-3162.44451349432	2.8305693003008e-05	\\
-3161.46573153409	2.86599993608604e-05	\\
-3160.48694957386	2.66415859072203e-05	\\
-3159.50816761364	2.95079768224209e-05	\\
-3158.52938565341	2.77810889532061e-05	\\
-3157.55060369318	2.92124287424555e-05	\\
-3156.57182173296	2.84317198041202e-05	\\
-3155.59303977273	2.8153470429254e-05	\\
-3154.6142578125	2.9741447019976e-05	\\
-3153.63547585227	2.65989905864601e-05	\\
-3152.65669389205	3.01938827557221e-05	\\
-3151.67791193182	2.88012837912411e-05	\\
-3150.69912997159	2.9041518428302e-05	\\
-3149.72034801136	2.92661869528001e-05	\\
-3148.74156605114	2.80003968065655e-05	\\
-3147.76278409091	2.91395624744292e-05	\\
-3146.78400213068	2.89947738557834e-05	\\
-3145.80522017046	2.73284935544949e-05	\\
-3144.82643821023	2.93165231619934e-05	\\
-3143.84765625	2.97583026787091e-05	\\
-3142.86887428977	2.89263596185191e-05	\\
-3141.89009232955	2.82268439248614e-05	\\
-3140.91131036932	2.62925424340878e-05	\\
-3139.93252840909	2.94136737476662e-05	\\
-3138.95374644886	2.75921272351883e-05	\\
-3137.97496448864	2.99643165226413e-05	\\
-3136.99618252841	2.94293802742887e-05	\\
-3136.01740056818	3.02695736760203e-05	\\
-3135.03861860796	3.05689840241159e-05	\\
-3134.05983664773	3.07573417566565e-05	\\
-3133.0810546875	2.74687447582903e-05	\\
-3132.10227272727	3.02132799249353e-05	\\
-3131.12349076705	2.85373298251967e-05	\\
-3130.14470880682	2.87024004361996e-05	\\
-3129.16592684659	2.8741437040598e-05	\\
-3128.18714488636	2.84505435891578e-05	\\
-3127.20836292614	3.06579335495828e-05	\\
-3126.22958096591	3.07206921804743e-05	\\
-3125.25079900568	3.12365029683367e-05	\\
-3124.27201704546	2.79491937688571e-05	\\
-3123.29323508523	3.10505683985319e-05	\\
-3122.314453125	3.16263689607421e-05	\\
-3121.33567116477	3.05427541582526e-05	\\
-3120.35688920455	2.76996093237341e-05	\\
-3119.37810724432	2.99060429105928e-05	\\
-3118.39932528409	2.88055992794654e-05	\\
-3117.42054332386	2.9862147254605e-05	\\
-3116.44176136364	3.04483773778095e-05	\\
-3115.46297940341	3.03375437297696e-05	\\
-3114.48419744318	3.04052698943182e-05	\\
-3113.50541548296	2.99977035057783e-05	\\
-3112.52663352273	2.95552339601399e-05	\\
-3111.5478515625	2.99643574212418e-05	\\
-3110.56906960227	3.15463968198617e-05	\\
-3109.59028764205	3.08657489676814e-05	\\
-3108.61150568182	2.96022190264134e-05	\\
-3107.63272372159	3.04636798315844e-05	\\
-3106.65394176136	3.21492664933948e-05	\\
-3105.67515980114	2.92473264956577e-05	\\
-3104.69637784091	2.87990217750103e-05	\\
-3103.71759588068	2.91098184418573e-05	\\
-3102.73881392046	3.04687158688986e-05	\\
-3101.76003196023	3.04874375511596e-05	\\
-3100.78125	3.40994632071445e-05	\\
-3099.80246803977	3.07276009972117e-05	\\
-3098.82368607955	2.98675988087206e-05	\\
-3097.84490411932	3.06722082749696e-05	\\
-3096.86612215909	2.90310263793981e-05	\\
-3095.88734019886	3.02924751422948e-05	\\
-3094.90855823864	2.91528122056927e-05	\\
-3093.92977627841	2.95536807910447e-05	\\
-3092.95099431818	2.82870417276885e-05	\\
-3091.97221235796	3.15518180678497e-05	\\
-3090.99343039773	2.97802834115898e-05	\\
-3090.0146484375	3.23267311881356e-05	\\
-3089.03586647727	2.96877355502374e-05	\\
-3088.05708451705	3.08806994790909e-05	\\
-3087.07830255682	3.14718308222679e-05	\\
-3086.09952059659	3.14015624088789e-05	\\
-3085.12073863636	3.09532204748093e-05	\\
-3084.14195667614	2.97836690859958e-05	\\
-3083.16317471591	3.25604772133214e-05	\\
-3082.18439275568	2.90378093860256e-05	\\
-3081.20561079546	3.0286656628568e-05	\\
-3080.22682883523	2.9602786759951e-05	\\
-3079.248046875	3.13011308894104e-05	\\
-3078.26926491477	3.2825173656949e-05	\\
-3077.29048295455	3.45679363794588e-05	\\
-3076.31170099432	3.28777397121566e-05	\\
-3075.33291903409	3.16619588111454e-05	\\
-3074.35413707386	3.08355616868561e-05	\\
-3073.37535511364	3.19571489885444e-05	\\
-3072.39657315341	3.10077488839121e-05	\\
-3071.41779119318	3.01827312807375e-05	\\
-3070.43900923296	3.22104530749433e-05	\\
-3069.46022727273	3.12897648569659e-05	\\
-3068.4814453125	3.23868013145833e-05	\\
-3067.50266335227	3.14852030414486e-05	\\
-3066.52388139205	3.05601263871954e-05	\\
-3065.54509943182	3.05380207320233e-05	\\
-3064.56631747159	3.12723179325694e-05	\\
-3063.58753551136	3.10246307021464e-05	\\
-3062.60875355114	2.96443355414312e-05	\\
-3061.62997159091	3.12071549396546e-05	\\
-3060.65118963068	3.13912245973827e-05	\\
-3059.67240767046	3.19275065640851e-05	\\
-3058.69362571023	3.18550692040253e-05	\\
-3057.71484375	2.96204773656025e-05	\\
-3056.73606178977	3.40390313491243e-05	\\
-3055.75727982955	2.93383609067764e-05	\\
-3054.77849786932	3.00970328667092e-05	\\
-3053.79971590909	3.16337146064558e-05	\\
-3052.82093394886	3.16643170156659e-05	\\
-3051.84215198864	3.16053091225755e-05	\\
-3050.86337002841	3.29467662170427e-05	\\
-3049.88458806818	3.05712192372608e-05	\\
-3048.90580610796	3.11281814399672e-05	\\
-3047.92702414773	3.24817758522911e-05	\\
-3046.9482421875	3.18072991651622e-05	\\
-3045.96946022727	3.20798711348323e-05	\\
-3044.99067826705	3.15406038763585e-05	\\
-3044.01189630682	3.17615886366948e-05	\\
-3043.03311434659	3.28162498593674e-05	\\
-3042.05433238636	3.33715154765386e-05	\\
-3041.07555042614	3.32456066617083e-05	\\
-3040.09676846591	3.14183720815216e-05	\\
-3039.11798650568	3.1787603223944e-05	\\
-3038.13920454546	3.33101728262072e-05	\\
-3037.16042258523	2.93203476251785e-05	\\
-3036.181640625	3.18794111308669e-05	\\
-3035.20285866477	3.30297024779143e-05	\\
-3034.22407670455	3.04238438872928e-05	\\
-3033.24529474432	3.19379936890594e-05	\\
-3032.26651278409	3.3022938190305e-05	\\
-3031.28773082386	3.03588034199235e-05	\\
-3030.30894886364	2.99292039702306e-05	\\
-3029.33016690341	3.20469058313102e-05	\\
-3028.35138494318	3.22123043659956e-05	\\
-3027.37260298296	3.30988661961465e-05	\\
-3026.39382102273	2.93765533023974e-05	\\
-3025.4150390625	3.14706514680367e-05	\\
-3024.43625710227	2.83792986509969e-05	\\
-3023.45747514205	3.14391493456881e-05	\\
-3022.47869318182	3.22750561275213e-05	\\
-3021.49991122159	3.20231792991452e-05	\\
-3020.52112926136	3.11882669159545e-05	\\
-3019.54234730114	3.05479904707452e-05	\\
-3018.56356534091	3.04105546651206e-05	\\
-3017.58478338068	3.02239266460084e-05	\\
-3016.60600142046	3.17447857007025e-05	\\
-3015.62721946023	2.80188394192639e-05	\\
-3014.6484375	3.07418696904582e-05	\\
-3013.66965553977	3.03034944737678e-05	\\
-3012.69087357955	2.89022584829522e-05	\\
-3011.71209161932	3.16536133880674e-05	\\
-3010.73330965909	3.12014505558957e-05	\\
-3009.75452769886	2.78453742802441e-05	\\
-3008.77574573864	3.10257190674746e-05	\\
-3007.79696377841	3.19956686700167e-05	\\
-3006.81818181818	3.2080975343432e-05	\\
-3005.83939985796	2.97860940149674e-05	\\
-3004.86061789773	2.92383779842303e-05	\\
-3003.8818359375	2.97719129034869e-05	\\
-3002.90305397727	3.00576803582343e-05	\\
-3001.92427201705	2.99242608409227e-05	\\
-3000.94549005682	2.96334335442084e-05	\\
-2999.96670809659	2.92826036745502e-05	\\
-2998.98792613636	3.24877650401294e-05	\\
-2998.00914417614	2.93983138079722e-05	\\
-2997.03036221591	2.88810698494846e-05	\\
-2996.05158025568	2.85265849774965e-05	\\
-2995.07279829546	2.95432845556231e-05	\\
-2994.09401633523	2.87202771264191e-05	\\
-2993.115234375	2.95178450931199e-05	\\
-2992.13645241477	2.86657947480515e-05	\\
-2991.15767045455	2.92852359444348e-05	\\
-2990.17888849432	3.080547063017e-05	\\
-2989.20010653409	2.87474810125467e-05	\\
-2988.22132457386	2.67635933122929e-05	\\
-2987.24254261364	2.76082990402083e-05	\\
-2986.26376065341	2.76325900161744e-05	\\
-2985.28497869318	2.89605009988574e-05	\\
-2984.30619673296	2.78813948237884e-05	\\
-2983.32741477273	2.86802688655019e-05	\\
-2982.3486328125	2.9991242286357e-05	\\
-2981.36985085227	2.88930184865889e-05	\\
-2980.39106889205	2.78831349962173e-05	\\
-2979.41228693182	2.68387447607717e-05	\\
-2978.43350497159	2.72888447706491e-05	\\
-2977.45472301136	3.27292562292266e-05	\\
-2976.47594105114	2.55846360303388e-05	\\
-2975.49715909091	2.73442400310812e-05	\\
-2974.51837713068	2.89299010788318e-05	\\
-2973.53959517046	2.70456211589594e-05	\\
-2972.56081321023	2.88006988396303e-05	\\
-2971.58203125	2.87717755773301e-05	\\
-2970.60324928977	2.97900517795671e-05	\\
-2969.62446732955	2.87310233823662e-05	\\
-2968.64568536932	2.84766101458026e-05	\\
-2967.66690340909	3.02536871405798e-05	\\
-2966.68812144886	2.81767056872265e-05	\\
-2965.70933948864	2.82025914795937e-05	\\
-2964.73055752841	2.61008588994387e-05	\\
-2963.75177556818	2.96256911284993e-05	\\
-2962.77299360796	2.6744586365635e-05	\\
-2961.79421164773	3.01040967651612e-05	\\
-2960.8154296875	2.65657366311122e-05	\\
-2959.83664772727	2.91500348067855e-05	\\
-2958.85786576705	2.80756965853266e-05	\\
-2957.87908380682	2.7603175936499e-05	\\
-2956.90030184659	2.66550230299437e-05	\\
-2955.92151988636	2.96743041177847e-05	\\
-2954.94273792614	2.74691873124932e-05	\\
-2953.96395596591	2.8258784313529e-05	\\
-2952.98517400568	2.66231973938438e-05	\\
-2952.00639204546	2.5476679963471e-05	\\
-2951.02761008523	2.74506620349816e-05	\\
-2950.048828125	2.83746624813777e-05	\\
-2949.07004616477	2.8337651600211e-05	\\
-2948.09126420455	2.67624321554342e-05	\\
-2947.11248224432	2.50900322084322e-05	\\
-2946.13370028409	2.6450125058077e-05	\\
-2945.15491832386	2.88918772408233e-05	\\
-2944.17613636364	2.87549288449937e-05	\\
-2943.19735440341	2.80333683125475e-05	\\
-2942.21857244318	2.65067769574803e-05	\\
-2941.23979048296	2.69663541199709e-05	\\
-2940.26100852273	3.0184520777549e-05	\\
-2939.2822265625	2.69782989093971e-05	\\
-2938.30344460227	2.71343882162781e-05	\\
-2937.32466264205	3.05277749365847e-05	\\
-2936.34588068182	2.77023129842143e-05	\\
-2935.36709872159	2.96611485897164e-05	\\
-2934.38831676136	2.86800194776205e-05	\\
-2933.40953480114	2.6693657059399e-05	\\
-2932.43075284091	2.90704741915278e-05	\\
-2931.45197088068	2.71253217776947e-05	\\
-2930.47318892046	2.86617706676169e-05	\\
-2929.49440696023	2.82677795594115e-05	\\
-2928.515625	2.84613337814769e-05	\\
-2927.53684303977	2.92157730940869e-05	\\
-2926.55806107955	2.75929777049159e-05	\\
-2925.57927911932	2.55530554545624e-05	\\
-2924.60049715909	2.7023283281937e-05	\\
-2923.62171519886	2.67367656110137e-05	\\
-2922.64293323864	2.81001269706987e-05	\\
-2921.66415127841	2.73955059160384e-05	\\
-2920.68536931818	2.74196458499763e-05	\\
-2919.70658735796	2.94350035862743e-05	\\
-2918.72780539773	2.77334242941074e-05	\\
-2917.7490234375	2.60765111086293e-05	\\
-2916.77024147727	2.78983445815806e-05	\\
-2915.79145951705	2.65997592894433e-05	\\
-2914.81267755682	2.64813790830983e-05	\\
-2913.83389559659	2.71863321874319e-05	\\
-2912.85511363636	2.88783451749334e-05	\\
-2911.87633167614	2.6375030331911e-05	\\
-2910.89754971591	2.53211797477397e-05	\\
-2909.91876775568	2.42812451271105e-05	\\
-2908.93998579546	2.65417143016612e-05	\\
-2907.96120383523	2.75713634435691e-05	\\
-2906.982421875	2.73442118407578e-05	\\
-2906.00363991477	2.81919012569464e-05	\\
-2905.02485795455	2.62807903591011e-05	\\
-2904.04607599432	2.70473433377311e-05	\\
-2903.06729403409	2.76282408477547e-05	\\
-2902.08851207386	2.40316253354833e-05	\\
-2901.10973011364	2.50745947765446e-05	\\
-2900.13094815341	2.54511056958732e-05	\\
-2899.15216619318	2.36409419196368e-05	\\
-2898.17338423296	2.88002029951845e-05	\\
-2897.19460227273	2.70484723997276e-05	\\
-2896.2158203125	2.36689693325324e-05	\\
-2895.23703835227	2.63254726909998e-05	\\
-2894.25825639205	2.33680751684435e-05	\\
-2893.27947443182	2.57002669771406e-05	\\
-2892.30069247159	2.69702018616533e-05	\\
-2891.32191051136	2.79523788757791e-05	\\
-2890.34312855114	2.77654077814304e-05	\\
-2889.36434659091	2.43650924796593e-05	\\
-2888.38556463068	2.6084455031219e-05	\\
-2887.40678267046	2.62129931638556e-05	\\
-2886.42800071023	2.42947745918932e-05	\\
-2885.44921875	2.65205634900081e-05	\\
-2884.47043678977	2.80436651621082e-05	\\
-2883.49165482955	2.61791656874858e-05	\\
-2882.51287286932	2.428805861006e-05	\\
-2881.53409090909	2.27039917041235e-05	\\
-2880.55530894886	2.33867770534241e-05	\\
-2879.57652698864	2.46690343254037e-05	\\
-2878.59774502841	2.47596495819186e-05	\\
-2877.61896306818	2.51905948401798e-05	\\
-2876.64018110796	2.51371154485832e-05	\\
-2875.66139914773	2.31310407973476e-05	\\
-2874.6826171875	2.6237922184831e-05	\\
-2873.70383522727	2.48547417894208e-05	\\
-2872.72505326705	2.39683272946166e-05	\\
-2871.74627130682	2.33730838862257e-05	\\
-2870.76748934659	2.5014334332899e-05	\\
-2869.78870738636	2.302147632696e-05	\\
-2868.80992542614	2.32672576104824e-05	\\
-2867.83114346591	2.4763023111923e-05	\\
-2866.85236150568	2.21616425185602e-05	\\
-2865.87357954546	2.46903365011015e-05	\\
-2864.89479758523	2.25842676952105e-05	\\
-2863.916015625	2.18865705426327e-05	\\
-2862.93723366477	2.30048803598309e-05	\\
-2861.95845170455	2.09041701230428e-05	\\
-2860.97966974432	2.11394946310237e-05	\\
-2860.00088778409	2.28885445691707e-05	\\
-2859.02210582386	2.16345676836806e-05	\\
-2858.04332386364	2.28551566579733e-05	\\
-2857.06454190341	2.2797144664633e-05	\\
-2856.08575994318	2.16253814110379e-05	\\
-2855.10697798296	2.1402957726015e-05	\\
-2854.12819602273	2.08797188740182e-05	\\
-2853.1494140625	1.95329159147453e-05	\\
-2852.17063210227	1.94677830472769e-05	\\
-2851.19185014205	1.97413931445058e-05	\\
-2850.21306818182	1.76968139079159e-05	\\
-2849.23428622159	1.88783596540821e-05	\\
-2848.25550426136	1.85658957232046e-05	\\
-2847.27672230114	1.89669414160988e-05	\\
-2846.29794034091	1.89337163101113e-05	\\
-2845.31915838068	1.98154295390924e-05	\\
-2844.34037642046	1.77018340496334e-05	\\
-2843.36159446023	1.7676384589318e-05	\\
-2842.3828125	1.73385645613279e-05	\\
-2841.40403053977	1.77528970905517e-05	\\
-2840.42524857955	1.65724512273348e-05	\\
-2839.44646661932	1.74085539256444e-05	\\
-2838.46768465909	1.48147427594532e-05	\\
-2837.48890269886	1.83189861409916e-05	\\
-2836.51012073864	1.5716034049034e-05	\\
-2835.53133877841	1.30637487168778e-05	\\
-2834.55255681818	1.54390691601294e-05	\\
-2833.57377485796	1.24430422638442e-05	\\
-2832.59499289773	1.3777906119137e-05	\\
-2831.6162109375	1.60377211346914e-05	\\
-2830.63742897727	1.48620347199765e-05	\\
-2829.65864701705	1.56662698628748e-05	\\
-2828.67986505682	1.44239351955038e-05	\\
-2827.70108309659	1.20881158494943e-05	\\
-2826.72230113636	1.45632494777957e-05	\\
-2825.74351917614	1.28237319297214e-05	\\
-2824.76473721591	1.33375216347001e-05	\\
-2823.78595525568	9.77849245428255e-06	\\
-2822.80717329545	1.04904004093153e-05	\\
-2821.82839133523	1.16823506557888e-05	\\
-2820.849609375	1.01426285554931e-05	\\
-2819.87082741477	1.1300531280782e-05	\\
-2818.89204545455	1.10842628578932e-05	\\
-2817.91326349432	1.00541147504547e-05	\\
-2816.93448153409	7.07156927989748e-06	\\
-2815.95569957386	7.56859546496437e-06	\\
-2814.97691761364	1.07920342891192e-05	\\
-2813.99813565341	7.39844284066512e-06	\\
-2813.01935369318	7.98687614869949e-06	\\
-2812.04057173295	8.54954203168045e-06	\\
-2811.06178977273	6.4615160626873e-06	\\
-2810.0830078125	8.52784262697893e-06	\\
-2809.10422585227	7.70853235798674e-06	\\
-2808.12544389205	7.31129018518458e-06	\\
-2807.14666193182	7.96280523780959e-06	\\
-2806.16787997159	4.13936813669873e-06	\\
-2805.18909801136	4.28032410366373e-06	\\
-2804.21031605114	6.56169913258913e-06	\\
-2803.23153409091	3.60840717315616e-06	\\
-2802.25275213068	5.70363378951656e-06	\\
-2801.27397017045	3.89155995266075e-06	\\
-2800.29518821023	3.77710956408051e-06	\\
-2799.31640625	2.84774554322067e-06	\\
-2798.33762428977	3.08252780409563e-06	\\
-2797.35884232955	2.05678962194096e-06	\\
-2796.38006036932	3.21132333327885e-06	\\
-2795.40127840909	2.48020140383517e-06	\\
-2794.42249644886	2.83907062752766e-06	\\
-2793.44371448864	1.04838176228123e-06	\\
-2792.46493252841	1.43926464463395e-06	\\
-2791.48615056818	1.40606173685635e-06	\\
-2790.50736860795	1.59043956274734e-06	\\
-2789.52858664773	2.34102893983528e-06	\\
-2788.5498046875	2.98115661286422e-06	\\
-2787.57102272727	4.44495886982964e-06	\\
-2786.59224076705	1.83167192244578e-06	\\
-2785.61345880682	6.30118668744594e-06	\\
-2784.63467684659	3.63857188388526e-06	\\
-2783.65589488636	4.26073122923893e-06	\\
-2782.67711292614	4.87683679222842e-06	\\
-2781.69833096591	5.2859208243353e-06	\\
-2780.71954900568	8.13878526672867e-06	\\
-2779.74076704545	7.11408115501689e-06	\\
-2778.76198508523	6.67416535645733e-06	\\
-2777.783203125	7.27768180232306e-06	\\
-2776.80442116477	7.60586186614173e-06	\\
-2775.82563920455	1.09778698287737e-05	\\
-2774.84685724432	1.1316226627811e-05	\\
-2773.86807528409	9.40836384365371e-06	\\
-2772.88929332386	1.3992145302665e-05	\\
-2771.91051136364	1.28460237570197e-05	\\
-2770.93172940341	1.20303696722458e-05	\\
-2769.95294744318	1.60897760863177e-05	\\
-2768.97416548295	1.45424563939311e-05	\\
-2767.99538352273	1.40312336665645e-05	\\
-2767.0166015625	1.81078819377322e-05	\\
-2766.03781960227	1.64276481641987e-05	\\
-2765.05903764205	1.75936508997786e-05	\\
-2764.08025568182	2.24567761931238e-05	\\
-2763.10147372159	2.06440654388663e-05	\\
-2762.12269176136	2.03168868362837e-05	\\
-2761.14390980114	2.19186385408391e-05	\\
-2760.16512784091	2.23891423227091e-05	\\
-2759.18634588068	2.40461567995971e-05	\\
-2758.20756392045	2.62110792739386e-05	\\
-2757.22878196023	2.76449683599753e-05	\\
-2756.25	2.70907415157002e-05	\\
-2755.27121803977	2.57699465298254e-05	\\
-2754.29243607955	2.95873219480198e-05	\\
-2753.31365411932	2.6506856703707e-05	\\
-2752.33487215909	2.94259526754603e-05	\\
-2751.35609019886	3.17154918819151e-05	\\
-2750.37730823864	3.2325004264849e-05	\\
-2749.39852627841	3.06546852559461e-05	\\
-2748.41974431818	3.65035284927056e-05	\\
-2747.44096235795	3.36509914338607e-05	\\
-2746.46218039773	3.4444412430758e-05	\\
-2745.4833984375	3.418281818305e-05	\\
-2744.50461647727	3.76622670801965e-05	\\
-2743.52583451705	3.90767338304783e-05	\\
-2742.54705255682	3.97322050937019e-05	\\
-2741.56827059659	4.23512530599431e-05	\\
-2740.58948863636	4.1154802284959e-05	\\
-2739.61070667614	4.15395741609725e-05	\\
-2738.63192471591	4.3940722815074e-05	\\
-2737.65314275568	4.25400316222213e-05	\\
-2736.67436079545	4.50538941767417e-05	\\
-2735.69557883523	4.62404835591624e-05	\\
-2734.716796875	4.57849992638468e-05	\\
-2733.73801491477	4.56572821287227e-05	\\
-2732.75923295455	4.83284481821626e-05	\\
-2731.78045099432	4.71539914893102e-05	\\
-2730.80166903409	4.89565408583174e-05	\\
-2729.82288707386	4.73270239444297e-05	\\
-2728.84410511364	4.98427669505211e-05	\\
-2727.86532315341	5.18749696840843e-05	\\
-2726.88654119318	5.36310251771399e-05	\\
-2725.90775923295	5.26680207212619e-05	\\
-2724.92897727273	5.40737630546009e-05	\\
-2723.9501953125	5.5500617705006e-05	\\
-2722.97141335227	5.30183138723711e-05	\\
-2721.99263139205	5.6325123465116e-05	\\
-2721.01384943182	5.48932615530352e-05	\\
-2720.03506747159	5.67644706569873e-05	\\
-2719.05628551136	5.93755702002545e-05	\\
-2718.07750355114	5.79996180191676e-05	\\
-2717.09872159091	5.6942003969553e-05	\\
-2716.11993963068	5.88477838862073e-05	\\
-2715.14115767045	5.99236529722567e-05	\\
-2714.16237571023	5.89940225941951e-05	\\
-2713.18359375	6.079274768826e-05	\\
-2712.20481178977	6.00515446169609e-05	\\
-2711.22602982955	6.08587087934687e-05	\\
-2710.24724786932	6.35109173119395e-05	\\
-2709.26846590909	6.06122086679644e-05	\\
-2708.28968394886	6.4402483003593e-05	\\
-2707.31090198864	6.42321266424734e-05	\\
-2706.33212002841	6.19257854704127e-05	\\
-2705.35333806818	6.29993269037094e-05	\\
-2704.37455610795	6.34128793720611e-05	\\
-2703.39577414773	6.28421201205149e-05	\\
-2702.4169921875	6.37761995959561e-05	\\
-2701.43821022727	6.36296270170227e-05	\\
-2700.45942826705	6.85098098103387e-05	\\
-2699.48064630682	6.51205169511774e-05	\\
-2698.50186434659	6.68201630305868e-05	\\
-2697.52308238636	6.40092839755057e-05	\\
-2696.54430042614	6.32477736647329e-05	\\
-2695.56551846591	6.60325104310333e-05	\\
-2694.58673650568	6.43973643110673e-05	\\
-2693.60795454545	6.61873424437701e-05	\\
-2692.62917258523	6.62715740353863e-05	\\
-2691.650390625	6.61019429901949e-05	\\
-2690.67160866477	6.7353592289133e-05	\\
-2689.69282670455	6.89161398938086e-05	\\
-2688.71404474432	6.69656269729008e-05	\\
-2687.73526278409	6.74485077248541e-05	\\
-2686.75648082386	6.65869886052797e-05	\\
-2685.77769886364	6.90122184995088e-05	\\
-2684.79891690341	6.72418323642096e-05	\\
-2683.82013494318	6.87661079583638e-05	\\
-2682.84135298295	6.87667251701375e-05	\\
-2681.86257102273	6.83943888792385e-05	\\
-2680.8837890625	6.72171860414079e-05	\\
-2679.90500710227	6.8737506888625e-05	\\
-2678.92622514205	6.99964370187317e-05	\\
-2677.94744318182	7.00056808863116e-05	\\
-2676.96866122159	6.88992554921357e-05	\\
-2675.98987926136	7.06310475536429e-05	\\
-2675.01109730114	7.00054441134455e-05	\\
-2674.03231534091	7.06451421687249e-05	\\
-2673.05353338068	7.040493753396e-05	\\
-2672.07475142045	6.95148382701782e-05	\\
-2671.09596946023	6.85770929486878e-05	\\
-2670.1171875	7.07423503117862e-05	\\
-2669.13840553977	6.86208306624719e-05	\\
-2668.15962357955	6.83106203803795e-05	\\
-2667.18084161932	6.96403436580988e-05	\\
-2666.20205965909	7.01816246674753e-05	\\
-2665.22327769886	7.08165407357922e-05	\\
-2664.24449573864	7.17251565153678e-05	\\
-2663.26571377841	6.75889785264275e-05	\\
-2662.28693181818	6.9258057786129e-05	\\
-2661.30814985795	6.94693993861179e-05	\\
-2660.32936789773	7.031931068367e-05	\\
-2659.3505859375	6.79115878975472e-05	\\
-2658.37180397727	6.95248624960888e-05	\\
-2657.39302201705	7.18111807368274e-05	\\
-2656.41424005682	6.94711020873112e-05	\\
-2655.43545809659	6.79247089281452e-05	\\
-2654.45667613636	6.96652415476408e-05	\\
-2653.47789417614	6.81985871803766e-05	\\
-2652.49911221591	6.78729773711526e-05	\\
-2651.52033025568	6.76761196954508e-05	\\
-2650.54154829545	6.82764081952595e-05	\\
-2649.56276633523	7.0316446381956e-05	\\
-2648.583984375	6.65725420987894e-05	\\
-2647.60520241477	6.94289020446712e-05	\\
-2646.62642045455	6.95011001991266e-05	\\
-2645.64763849432	6.84268698353441e-05	\\
-2644.66885653409	7.10770095611582e-05	\\
-2643.69007457386	6.7972014732494e-05	\\
-2642.71129261364	6.92987210030146e-05	\\
-2641.73251065341	6.8594330088763e-05	\\
-2640.75372869318	6.99659155405877e-05	\\
-2639.77494673295	7.05071079831196e-05	\\
-2638.79616477273	7.09138261741203e-05	\\
-2637.8173828125	7.01086983279676e-05	\\
-2636.83860085227	6.71169840757441e-05	\\
-2635.85981889205	6.75490890570875e-05	\\
-2634.88103693182	7.08047349471163e-05	\\
-2633.90225497159	6.82765516733581e-05	\\
-2632.92347301136	6.97203780461505e-05	\\
-2631.94469105114	7.26398876702801e-05	\\
-2630.96590909091	6.69114566025806e-05	\\
-2629.98712713068	6.87324777077797e-05	\\
-2629.00834517045	7.17871607189033e-05	\\
-2628.02956321023	6.76929657701618e-05	\\
-2627.05078125	6.58934687653491e-05	\\
-2626.07199928977	6.93162910176448e-05	\\
-2625.09321732955	6.85245143918182e-05	\\
-2624.11443536932	6.99929447525945e-05	\\
-2623.13565340909	7.20243590695749e-05	\\
-2622.15687144886	6.66189627958344e-05	\\
-2621.17808948864	7.03831375627185e-05	\\
-2620.19930752841	6.85803038054583e-05	\\
-2619.22052556818	7.14901015956958e-05	\\
-2618.24174360795	6.78917563798011e-05	\\
-2617.26296164773	7.0900994166576e-05	\\
-2616.2841796875	6.65447913572274e-05	\\
-2615.30539772727	6.74561707193742e-05	\\
-2614.32661576705	6.85935112202516e-05	\\
-2613.34783380682	6.83669724199402e-05	\\
-2612.36905184659	7.09747340270041e-05	\\
-2611.39026988636	6.73241264117462e-05	\\
-2610.41148792614	7.05208598446875e-05	\\
-2609.43270596591	6.78953345429653e-05	\\
-2608.45392400568	7.16945851579142e-05	\\
-2607.47514204545	7.01764097523846e-05	\\
-2606.49636008523	7.06222145096329e-05	\\
-2605.517578125	7.09423743473235e-05	\\
-2604.53879616477	7.17128784688566e-05	\\
-2603.56001420455	7.09053851750471e-05	\\
-2602.58123224432	6.97621474484221e-05	\\
-2601.60245028409	7.01889160767203e-05	\\
-2600.62366832386	7.11736235481416e-05	\\
-2599.64488636364	7.08035783972571e-05	\\
-2598.66610440341	7.24653089174098e-05	\\
-2597.68732244318	7.0537018091497e-05	\\
-2596.70854048295	7.08098063011676e-05	\\
-2595.72975852273	7.23233404145613e-05	\\
-2594.7509765625	6.96216051931405e-05	\\
-2593.77219460227	7.06615210015601e-05	\\
-2592.79341264205	7.3062585184749e-05	\\
-2591.81463068182	7.24232792714113e-05	\\
-2590.83584872159	7.33801021299497e-05	\\
-2589.85706676136	7.16464425411314e-05	\\
-2588.87828480114	7.25772274334929e-05	\\
-2587.89950284091	6.83766327566056e-05	\\
-2586.92072088068	7.3557732241883e-05	\\
-2585.94193892045	7.18847473884896e-05	\\
-2584.96315696023	7.29002071589145e-05	\\
-2583.984375	7.09241448268958e-05	\\
-2583.00559303977	7.30576703615969e-05	\\
-2582.02681107955	7.28974371351945e-05	\\
-2581.04802911932	7.2353789770297e-05	\\
-2580.06924715909	7.46709288644335e-05	\\
-2579.09046519886	7.13507785800166e-05	\\
-2578.11168323864	7.15735130779855e-05	\\
-2577.13290127841	7.4405592759981e-05	\\
-2576.15411931818	7.44646331649742e-05	\\
-2575.17533735795	7.76171279106103e-05	\\
-2574.19655539773	7.45657819662765e-05	\\
-2573.2177734375	7.60592684240498e-05	\\
-2572.23899147727	7.542620572892e-05	\\
-2571.26020951705	7.80955742472822e-05	\\
-2570.28142755682	7.92055757824273e-05	\\
-2569.30264559659	7.88753426003173e-05	\\
-2568.32386363636	7.84693018805608e-05	\\
-2567.34508167614	7.8303916179834e-05	\\
-2566.36629971591	8.13332760554973e-05	\\
-2565.38751775568	8.2311477873803e-05	\\
-2564.40873579545	8.38532439479546e-05	\\
-2563.42995383523	8.41246432816327e-05	\\
-2562.451171875	8.27774390798321e-05	\\
-2561.47238991477	8.30501199293459e-05	\\
-2560.49360795455	8.60592925947414e-05	\\
-2559.51482599432	8.48143892041425e-05	\\
-2558.53604403409	8.75275772735354e-05	\\
-2557.55726207386	8.64451706110157e-05	\\
-2556.57848011364	9.07926899735881e-05	\\
-2555.59969815341	8.73477746230265e-05	\\
-2554.62091619318	8.78292369866237e-05	\\
-2553.64213423295	9.04311676804409e-05	\\
-2552.66335227273	9.30574296423698e-05	\\
-2551.6845703125	8.91558013717916e-05	\\
-2550.70578835227	9.07942509287619e-05	\\
-2549.72700639205	9.04438273918453e-05	\\
-2548.74822443182	9.33956301768107e-05	\\
-2547.76944247159	9.24469395289939e-05	\\
-2546.79066051136	9.46952343032776e-05	\\
-2545.81187855114	9.58693923498783e-05	\\
-2544.83309659091	9.44919526480246e-05	\\
-2543.85431463068	9.52234016202394e-05	\\
-2542.87553267045	9.53709432503948e-05	\\
-2541.89675071023	9.62773691669119e-05	\\
-2540.91796875	9.97175117077393e-05	\\
-2539.93918678977	9.7268977240453e-05	\\
-2538.96040482955	0.000100224787754124	\\
-2537.98162286932	0.000100823274430604	\\
-2537.00284090909	0.000100045505577887	\\
-2536.02405894886	0.000100147650224194	\\
-2535.04527698864	0.000102218098070852	\\
-2534.06649502841	9.83682314637121e-05	\\
-2533.08771306818	0.000102662135649246	\\
-2532.10893110795	0.000102624746434306	\\
-2531.13014914773	0.000106983060102631	\\
-2530.1513671875	0.000105528148097144	\\
-2529.17258522727	0.000105786063696774	\\
-2528.19380326705	0.00010527547048505	\\
-2527.21502130682	0.00010686435341413	\\
-2526.23623934659	0.000107861468586736	\\
-2525.25745738636	0.000105766948659911	\\
-2524.27867542614	0.000107930512784064	\\
-2523.29989346591	0.000106105337585187	\\
-2522.32111150568	0.000109084841728536	\\
-2521.34232954545	0.00010728444395162	\\
-2520.36354758523	0.000110750620439491	\\
-2519.384765625	0.000107266073466245	\\
-2518.40598366477	0.000108449539436259	\\
-2517.42720170455	0.000109622014904788	\\
-2516.44841974432	0.000110348612916227	\\
-2515.46963778409	0.000110535776313188	\\
-2514.49085582386	0.00011278855010762	\\
-2513.51207386364	0.000106732093311453	\\
-2512.53329190341	0.000113800802049745	\\
-2511.55450994318	0.000112069405719592	\\
-2510.57572798295	0.000112469144134801	\\
-2509.59694602273	0.000108374090509968	\\
-2508.6181640625	0.00011124511137217	\\
-2507.63938210227	0.000113394966553725	\\
-2506.66060014205	0.000111715849311881	\\
-2505.68181818182	0.000110488413167738	\\
-2504.70303622159	0.000108168769186644	\\
-2503.72425426136	0.000110627870783499	\\
-2502.74547230114	0.000112043290459984	\\
-2501.76669034091	0.00011228234484442	\\
-2500.78790838068	0.000111325767899561	\\
-2499.80912642045	0.000116449857777445	\\
-2498.83034446023	0.000108074156692662	\\
-2497.8515625	0.000110325303174948	\\
-2496.87278053977	0.000111448362557504	\\
-2495.89399857955	0.000108555213208292	\\
-2494.91521661932	0.00011096279489066	\\
-2493.93643465909	0.000111106464644463	\\
-2492.95765269886	0.000111625436211053	\\
-2491.97887073864	0.000109678924730175	\\
-2491.00008877841	0.000109573480389458	\\
-2490.02130681818	0.000109975244647637	\\
-2489.04252485795	0.000109835300814188	\\
-2488.06374289773	0.000109152400799865	\\
-2487.0849609375	0.000109427266997669	\\
-2486.10617897727	0.000108915725112319	\\
-2485.12739701705	0.000106071571484226	\\
-2484.14861505682	0.00010600054134476	\\
-2483.16983309659	0.000108335070978324	\\
-2482.19105113636	0.00010787128324987	\\
-2481.21226917614	0.000109528179658785	\\
-2480.23348721591	0.000107164241028585	\\
-2479.25470525568	0.000108894076017163	\\
-2478.27592329545	0.000108347826435207	\\
-2477.29714133523	0.000109158696194052	\\
-2476.318359375	0.000111702556709089	\\
-2475.33957741477	0.000111269414222155	\\
-2474.36079545455	0.000109140983803943	\\
-2473.38201349432	0.00010998569407894	\\
-2472.40323153409	0.000109840115368203	\\
-2471.42444957386	0.000108813362957607	\\
-2470.44566761364	0.000108390663673153	\\
-2469.46688565341	0.000109047126185712	\\
-2468.48810369318	0.000109457650215927	\\
-2467.50932173295	0.000111251837662286	\\
-2466.53053977273	0.000109887052999262	\\
-2465.5517578125	0.000111614146070093	\\
-2464.57297585227	0.000111292277366312	\\
-2463.59419389205	0.000106993950657265	\\
-2462.61541193182	0.000107109264443939	\\
-2461.63662997159	0.000109672747098757	\\
-2460.65784801136	0.000110187690542575	\\
-2459.67906605114	0.000112271596124504	\\
-2458.70028409091	0.000109190555494576	\\
-2457.72150213068	0.000105889352156613	\\
-2456.74272017045	0.000110687715683246	\\
-2455.76393821023	0.000108996527533352	\\
-2454.78515625	0.000113117225705864	\\
-2453.80637428977	0.000110268258154739	\\
-2452.82759232955	0.000111183308271927	\\
-2451.84881036932	0.000113402473789215	\\
-2450.87002840909	0.000112355245271743	\\
-2449.89124644886	0.000111433237934226	\\
-2448.91246448864	0.000112063056804293	\\
-2447.93368252841	0.000108778933446389	\\
-2446.95490056818	0.000111313023623019	\\
-2445.97611860795	0.000112447265166107	\\
-2444.99733664773	0.000108883627096687	\\
-2444.0185546875	0.000110303430953792	\\
-2443.03977272727	0.000106394679639078	\\
-2442.06099076705	0.000109925390451604	\\
-2441.08220880682	0.000110285814969406	\\
-2440.10342684659	0.000110587306559086	\\
-2439.12464488636	0.000105703276914155	\\
-2438.14586292614	0.000109176780612148	\\
-2437.16708096591	0.00010972210841529	\\
-2436.18829900568	0.000105522306160331	\\
-2435.20951704545	0.000106031567422489	\\
-2434.23073508523	0.000112029915834983	\\
-2433.251953125	0.000106797479735007	\\
-2432.27317116477	0.000110579793225022	\\
-2431.29438920455	0.000115140109157081	\\
-2430.31560724432	0.000106330834512688	\\
-2429.33682528409	0.000109046216909584	\\
-2428.35804332386	0.000105566162787949	\\
-2427.37926136364	0.000106460478022591	\\
-2426.40047940341	0.000107425113560551	\\
-2425.42169744318	0.000106451250854862	\\
-2424.44291548295	0.000104218441566834	\\
-2423.46413352273	0.000103145135020807	\\
-2422.4853515625	0.000107658165432722	\\
-2421.50656960227	0.000106104562317601	\\
-2420.52778764205	0.000107536102508993	\\
-2419.54900568182	0.000108952042590176	\\
-2418.57022372159	0.00011057172261991	\\
-2417.59144176136	0.000105137118612674	\\
-2416.61265980114	0.000113414091245332	\\
-2415.63387784091	0.000105491533269898	\\
-2414.65509588068	0.000106740051195226	\\
-2413.67631392045	0.000102525486795906	\\
-2412.69753196023	0.000102541284049645	\\
-2411.71875	0.000100999918585507	\\
-2410.73996803977	0.000101083701348381	\\
-2409.76118607955	0.000105739228984803	\\
-2408.78240411932	0.000104297034092605	\\
-2407.80362215909	0.000100449521622986	\\
-2406.82484019886	0.000102003613869311	\\
-2405.84605823864	0.000100531242432179	\\
-2404.86727627841	9.96732772863716e-05	\\
-2403.88849431818	9.95571300629467e-05	\\
-2402.90971235795	9.99634566796276e-05	\\
-2401.93093039773	0.000102933217939029	\\
-2400.9521484375	0.000103158143074631	\\
-2399.97336647727	0.000106735514171201	\\
-2398.99458451705	0.000101471149089398	\\
-2398.01580255682	0.000100634980112166	\\
-2397.03702059659	9.93460869422118e-05	\\
-2396.05823863636	0.000102589999810173	\\
-2395.07945667614	9.7748378807384e-05	\\
-2394.10067471591	0.000105332487301941	\\
-2393.12189275568	0.000100100754278433	\\
-2392.14311079545	9.98911019438086e-05	\\
-2391.16432883523	9.99076375215384e-05	\\
-2390.185546875	9.89785155875948e-05	\\
-2389.20676491477	0.000105048256569749	\\
-2388.22798295455	9.8426124745946e-05	\\
-2387.24920099432	0.000100325370714805	\\
-2386.27041903409	0.000102160767512212	\\
-2385.29163707386	0.000102861015847961	\\
-2384.31285511364	0.000102022139337709	\\
-2383.33407315341	0.000105730558096125	\\
-2382.35529119318	0.000101895459379936	\\
-2381.37650923295	0.000100893898157745	\\
-2380.39772727273	0.000104939658735785	\\
-2379.4189453125	0.00010631995574279	\\
-2378.44016335227	0.000102023162439665	\\
-2377.46138139205	0.000106817582285908	\\
-2376.48259943182	0.00010464685922912	\\
-2375.50381747159	0.000104806877934973	\\
-2374.52503551136	0.000105203869901051	\\
-2373.54625355114	0.000107102205007709	\\
-2372.56747159091	0.000102768060035321	\\
-2371.58868963068	0.000105728391837111	\\
-2370.60990767045	0.000104245839980861	\\
-2369.63112571023	0.000107311877135536	\\
-2368.65234375	0.000109583298216105	\\
-2367.67356178977	0.00010253029888383	\\
-2366.69477982955	0.000104413573161113	\\
-2365.71599786932	0.000107081082004404	\\
-2364.73721590909	0.000103023847681962	\\
-2363.75843394886	0.000108149144230037	\\
-2362.77965198864	0.000106608948888012	\\
-2361.80087002841	0.000106829959012814	\\
-2360.82208806818	0.000109387100503383	\\
-2359.84330610795	0.000107822929272914	\\
-2358.86452414773	0.00010939269063889	\\
-2357.8857421875	0.000106692880182263	\\
-2356.90696022727	0.00010515740252328	\\
-2355.92817826705	0.000107872369891817	\\
-2354.94939630682	0.000106549043293782	\\
-2353.97061434659	0.000107317260171859	\\
-2352.99183238636	0.000106994756773183	\\
-2352.01305042614	0.00010708169030753	\\
-2351.03426846591	0.000106644117369604	\\
-2350.05548650568	0.000105513280263527	\\
-2349.07670454545	0.000107409367155235	\\
-2348.09792258523	0.000108740691113261	\\
-2347.119140625	0.000107635045313906	\\
-2346.14035866477	0.000108145810340224	\\
-2345.16157670455	0.000108814830098428	\\
-2344.18279474432	0.00010720982600181	\\
-2343.20401278409	0.000108983266443149	\\
-2342.22523082386	0.000105493547583718	\\
-2341.24644886364	0.000109387545941897	\\
-2340.26766690341	0.000107923942088508	\\
-2339.28888494318	0.000106692062747803	\\
-2338.31010298295	0.000109395830435803	\\
-2337.33132102273	0.000107525584993762	\\
-2336.3525390625	0.000109246179190769	\\
-2335.37375710227	0.000110349836129042	\\
-2334.39497514205	0.000108847056580844	\\
-2333.41619318182	0.000107554416873562	\\
-2332.43741122159	0.000108123864014529	\\
-2331.45862926136	0.000105041930962326	\\
-2330.47984730114	0.000106688545836561	\\
-2329.50106534091	0.000108764619224918	\\
-2328.52228338068	0.000106257879277678	\\
-2327.54350142045	0.000107783237375912	\\
-2326.56471946023	0.000107752127439732	\\
-2325.5859375	0.000106763846343003	\\
-2324.60715553977	0.000109511304445189	\\
-2323.62837357955	0.000108792483948526	\\
-2322.64959161932	0.000107137946234134	\\
-2321.67080965909	0.000110573855190268	\\
-2320.69202769886	0.000106155448434969	\\
-2319.71324573864	0.000107686473818289	\\
-2318.73446377841	0.000106745821025628	\\
-2317.75568181818	0.000111636571955073	\\
-2316.77689985795	0.000106817685757693	\\
-2315.79811789773	0.000110239350882063	\\
-2314.8193359375	0.00010425743920577	\\
-2313.84055397727	0.000113891384601368	\\
-2312.86177201705	0.000110102074092649	\\
-2311.88299005682	0.00010842781352708	\\
-2310.90420809659	0.000111988870441526	\\
-2309.92542613636	0.000107426271330384	\\
-2308.94664417614	0.000110221982794925	\\
-2307.96786221591	0.000110509758967145	\\
-2306.98908025568	0.000109509963644464	\\
-2306.01029829545	0.000110024684508784	\\
-2305.03151633523	0.000111701455573269	\\
-2304.052734375	0.000113226490713124	\\
-2303.07395241477	0.000112527673370207	\\
-2302.09517045455	0.000112404723768276	\\
-2301.11638849432	0.000107290665869598	\\
-2300.13760653409	0.000118022413136157	\\
-2299.15882457386	0.00011275083835516	\\
-2298.18004261364	0.000108020956096008	\\
-2297.20126065341	0.00010901466581756	\\
-2296.22247869318	0.000108953207621571	\\
-2295.24369673295	0.000109296157591262	\\
-2294.26491477273	0.000108732236702122	\\
-2293.2861328125	0.00011082563335971	\\
-2292.30735085227	0.000108617078498704	\\
-2291.32856889205	0.000106995780096543	\\
-2290.34978693182	0.000109071300720482	\\
-2289.37100497159	0.000108328919771098	\\
-2288.39222301136	0.000105246550192866	\\
-2287.41344105114	0.000106391411287857	\\
-2286.43465909091	0.000101497191372503	\\
-2285.45587713068	0.0001028166045503	\\
-2284.47709517045	0.000104744415192592	\\
-2283.49831321023	0.000105034454246864	\\
-2282.51953125	0.000102829883236027	\\
-2281.54074928977	0.0001053023311173	\\
-2280.56196732955	9.81132509638029e-05	\\
-2279.58318536932	0.000105010275831933	\\
-2278.60440340909	0.000100449937118465	\\
-2277.62562144886	0.000101510931838861	\\
-2276.64683948864	0.000100059655881502	\\
-2275.66805752841	9.71073368962367e-05	\\
-2274.68927556818	9.87781171205394e-05	\\
-2273.71049360795	9.67605030811951e-05	\\
-2272.73171164773	9.77015573594147e-05	\\
-2271.7529296875	9.63516679964919e-05	\\
-2270.77414772727	9.50865824891024e-05	\\
-2269.79536576705	9.612404396754e-05	\\
-2268.81658380682	9.44008438805665e-05	\\
-2267.83780184659	9.23386451635896e-05	\\
-2266.85901988636	9.39317575074051e-05	\\
-2265.88023792614	9.08292452364732e-05	\\
-2264.90145596591	9.12357274102222e-05	\\
-2263.92267400568	9.0684492956791e-05	\\
-2262.94389204545	9.08048078083792e-05	\\
-2261.96511008523	8.66177377597151e-05	\\
-2260.986328125	8.87068897843167e-05	\\
-2260.00754616477	8.66653632236647e-05	\\
-2259.02876420455	8.68549491820992e-05	\\
-2258.04998224432	8.61634846177e-05	\\
-2257.07120028409	8.52260851822284e-05	\\
-2256.09241832386	9.09085422791891e-05	\\
-2255.11363636364	8.06745161692588e-05	\\
-2254.13485440341	8.5652377679471e-05	\\
-2253.15607244318	8.18045326035234e-05	\\
-2252.17729048295	8.41547723632017e-05	\\
-2251.19850852273	8.02477256679081e-05	\\
-2250.2197265625	7.66786681274198e-05	\\
-2249.24094460227	7.7556815016935e-05	\\
-2248.26216264205	7.7382615117954e-05	\\
-2247.28338068182	7.86577204177471e-05	\\
-2246.30459872159	7.54473791939474e-05	\\
-2245.32581676136	7.85818593789201e-05	\\
-2244.34703480114	7.52523353675036e-05	\\
-2243.36825284091	7.5364768279774e-05	\\
-2242.38947088068	7.68849182822535e-05	\\
-2241.41068892045	7.15293302907277e-05	\\
-2240.43190696023	6.91727432777875e-05	\\
-2239.453125	7.00013732699108e-05	\\
-2238.47434303977	6.7299194510181e-05	\\
-2237.49556107955	6.70302092951047e-05	\\
-2236.51677911932	6.45802397714666e-05	\\
-2235.53799715909	6.47465590149527e-05	\\
-2234.55921519886	6.42441464160232e-05	\\
-2233.58043323864	6.17021511071913e-05	\\
-2232.60165127841	6.15607285509755e-05	\\
-2231.62286931818	6.13253614324828e-05	\\
-2230.64408735795	5.78151460220498e-05	\\
-2229.66530539773	5.78057071982437e-05	\\
-2228.6865234375	5.60172682114588e-05	\\
-2227.70774147727	5.59150306147465e-05	\\
-2226.72895951705	5.746856593939e-05	\\
-2225.75017755682	5.40544817694374e-05	\\
-2224.77139559659	5.17845531904019e-05	\\
-2223.79261363636	5.41281705716738e-05	\\
-2222.81383167614	4.61800742587375e-05	\\
-2221.83504971591	5.29090926622152e-05	\\
-2220.85626775568	4.85705823800037e-05	\\
-2219.87748579545	4.24377657469516e-05	\\
-2218.89870383523	4.63553269872725e-05	\\
-2217.919921875	4.74913432640169e-05	\\
-2216.94113991477	4.64321556690832e-05	\\
-2215.96235795455	4.76943931208463e-05	\\
-2214.98357599432	4.40143006301144e-05	\\
-2214.00479403409	4.13813552043399e-05	\\
-2213.02601207386	4.14832554788899e-05	\\
-2212.04723011364	4.2495868818441e-05	\\
-2211.06844815341	3.92493516623533e-05	\\
-2210.08966619318	4.13663109017958e-05	\\
-2209.11088423295	3.98286942495198e-05	\\
-2208.13210227273	3.46054994950262e-05	\\
-2207.1533203125	4.24551499399444e-05	\\
-2206.17453835227	3.69793661987253e-05	\\
-2205.19575639205	3.53377759474793e-05	\\
-2204.21697443182	3.70069055309267e-05	\\
-2203.23819247159	3.25564806935441e-05	\\
-2202.25941051136	3.41310648519779e-05	\\
-2201.28062855114	3.0090408829797e-05	\\
-2200.30184659091	3.4910219689192e-05	\\
-2199.32306463068	3.33196528739591e-05	\\
-2198.34428267045	3.22108959399466e-05	\\
-2197.36550071023	3.19632315526644e-05	\\
-2196.38671875	3.09993907246873e-05	\\
-2195.40793678977	3.57949126791791e-05	\\
-2194.42915482955	3.26949661704821e-05	\\
-2193.45037286932	2.97262585917547e-05	\\
-2192.47159090909	3.4889832218771e-05	\\
-2191.49280894886	3.22652373786128e-05	\\
-2190.51402698864	3.38580764302905e-05	\\
-2189.53524502841	3.4617874025096e-05	\\
-2188.55646306818	3.3529296458105e-05	\\
-2187.57768110795	3.2689825028525e-05	\\
-2186.59889914773	3.42753808154873e-05	\\
-2185.6201171875	3.35290684127887e-05	\\
-2184.64133522727	3.60938066612834e-05	\\
-2183.66255326705	3.53697921681193e-05	\\
-2182.68377130682	3.74336018542956e-05	\\
-2181.70498934659	3.42020112662883e-05	\\
-2180.72620738636	3.46300885459117e-05	\\
-2179.74742542614	3.83155498849791e-05	\\
-2178.76864346591	3.46525910775697e-05	\\
-2177.78986150568	3.61591360764395e-05	\\
-2176.81107954545	3.7852926282275e-05	\\
-2175.83229758523	3.99268909585148e-05	\\
-2174.853515625	3.76704701000296e-05	\\
-2173.87473366477	3.53203200264237e-05	\\
-2172.89595170455	3.73681345617505e-05	\\
-2171.91716974432	3.83979962097332e-05	\\
-2170.93838778409	3.79555553475125e-05	\\
-2169.95960582386	3.93611788093222e-05	\\
-2168.98082386364	3.67440843343009e-05	\\
-2168.00204190341	4.06763802439811e-05	\\
-2167.02325994318	3.78649471718898e-05	\\
-2166.04447798295	3.90674492243146e-05	\\
-2165.06569602273	3.90422353966503e-05	\\
-2164.0869140625	3.90289524736929e-05	\\
-2163.10813210227	4.09239568407509e-05	\\
-2162.12935014205	3.85425484274201e-05	\\
-2161.15056818182	4.0469862386214e-05	\\
-2160.17178622159	3.96380035368673e-05	\\
-2159.19300426136	4.15788634425616e-05	\\
-2158.21422230114	4.02761946898035e-05	\\
-2157.23544034091	4.07280029535679e-05	\\
-2156.25665838068	3.71847104534491e-05	\\
-2155.27787642045	4.20884090677233e-05	\\
-2154.29909446023	3.99942147538878e-05	\\
-2153.3203125	4.15913011637687e-05	\\
-2152.34153053977	4.28600443751128e-05	\\
-2151.36274857955	3.96117888016556e-05	\\
-2150.38396661932	4.05019444296505e-05	\\
-2149.40518465909	4.08125304185986e-05	\\
-2148.42640269886	3.55424291390207e-05	\\
-2147.44762073864	4.22297146231757e-05	\\
-2146.46883877841	3.87372782545279e-05	\\
-2145.49005681818	4.03918276257902e-05	\\
-2144.51127485795	3.91034722585739e-05	\\
-2143.53249289773	4.36637119654669e-05	\\
-2142.5537109375	4.16635182006823e-05	\\
-2141.57492897727	3.97910957582985e-05	\\
-2140.59614701705	4.00546982447072e-05	\\
-2139.61736505682	4.25049910479368e-05	\\
-2138.63858309659	4.45350715008516e-05	\\
-2137.65980113636	4.23799665794265e-05	\\
-2136.68101917614	4.1090694178331e-05	\\
-2135.70223721591	4.38730182764737e-05	\\
-2134.72345525568	4.27116091776427e-05	\\
-2133.74467329545	4.20236457952295e-05	\\
-2132.76589133523	4.33433676316529e-05	\\
-2131.787109375	4.22937146700495e-05	\\
-2130.80832741477	4.16283719003127e-05	\\
-2129.82954545455	4.20308454643921e-05	\\
-2128.85076349432	4.03681118914028e-05	\\
-2127.87198153409	4.55727386714771e-05	\\
-2126.89319957386	4.13259436366393e-05	\\
-2125.91441761364	4.23694793906412e-05	\\
-2124.93563565341	4.16785208528467e-05	\\
-2123.95685369318	3.9346936590026e-05	\\
-2122.97807173295	4.32852726458951e-05	\\
-2121.99928977273	4.03981640022247e-05	\\
-2121.0205078125	4.42557314037138e-05	\\
-2120.04172585227	4.30833820051657e-05	\\
-2119.06294389205	4.26383823640281e-05	\\
-2118.08416193182	3.94782563866214e-05	\\
-2117.10537997159	4.17114849677277e-05	\\
-2116.12659801136	4.10856676227557e-05	\\
-2115.14781605114	3.89934603659363e-05	\\
-2114.16903409091	3.94979751109026e-05	\\
-2113.19025213068	3.93822256786854e-05	\\
-2112.21147017045	4.02872942703493e-05	\\
-2111.23268821023	3.7231332782545e-05	\\
-2110.25390625	3.72092914227697e-05	\\
-2109.27512428977	3.59209257467907e-05	\\
-2108.29634232955	4.02309984623695e-05	\\
-2107.31756036932	3.46218911683196e-05	\\
-2106.33877840909	3.70255806118348e-05	\\
-2105.35999644886	3.7862500783937e-05	\\
-2104.38121448864	3.64772670598038e-05	\\
-2103.40243252841	3.45479783117417e-05	\\
-2102.42365056818	3.44312912432514e-05	\\
-2101.44486860795	3.37785930567241e-05	\\
-2100.46608664773	3.43895242248518e-05	\\
-2099.4873046875	3.11666568709141e-05	\\
-2098.50852272727	3.12130412401767e-05	\\
-2097.52974076705	3.4018735094998e-05	\\
-2096.55095880682	2.90641463938762e-05	\\
-2095.57217684659	3.28215005276637e-05	\\
-2094.59339488636	3.15664492372219e-05	\\
-2093.61461292614	2.79429888234874e-05	\\
-2092.63583096591	3.19805113925382e-05	\\
-2091.65704900568	2.38977270733898e-05	\\
-2090.67826704545	2.82073173089332e-05	\\
-2089.69948508523	2.65305493618822e-05	\\
-2088.720703125	2.64051697698505e-05	\\
-2087.74192116477	2.92504514578448e-05	\\
-2086.76313920455	2.64251589423545e-05	\\
-2085.78435724432	2.97019918543447e-05	\\
-2084.80557528409	2.7159935000022e-05	\\
-2083.82679332386	2.79442931407061e-05	\\
-2082.84801136364	2.64191929019506e-05	\\
-2081.86922940341	2.70455971917524e-05	\\
-2080.89044744318	2.56001551320162e-05	\\
-2079.91166548295	2.70565341665364e-05	\\
-2078.93288352273	2.50270229632934e-05	\\
-2077.9541015625	2.98538615960373e-05	\\
-2076.97531960227	2.61757144210691e-05	\\
-2075.99653764205	2.51152204289248e-05	\\
-2075.01775568182	2.74820262832285e-05	\\
-2074.03897372159	2.58480236665329e-05	\\
-2073.06019176136	2.87379465039631e-05	\\
-2072.08140980114	2.57808913898505e-05	\\
-2071.10262784091	2.94996732954359e-05	\\
-2070.12384588068	2.89111619738829e-05	\\
-2069.14506392045	2.97126905347779e-05	\\
-2068.16628196023	3.12126318750838e-05	\\
-2067.1875	2.91989612740849e-05	\\
-2066.20871803977	3.08079462648034e-05	\\
-2065.22993607955	3.11847886727666e-05	\\
-2064.25115411932	3.19984114506442e-05	\\
-2063.27237215909	3.16422594604723e-05	\\
-2062.29359019886	3.1447692799479e-05	\\
-2061.31480823864	3.28020855758757e-05	\\
-2060.33602627841	3.24468334306063e-05	\\
-2059.35724431818	3.55269442783698e-05	\\
-2058.37846235795	3.45103279001185e-05	\\
-2057.39968039773	3.35562309757452e-05	\\
-2056.4208984375	3.48936671842307e-05	\\
-2055.44211647727	3.62741121252172e-05	\\
-2054.46333451705	3.84945269535284e-05	\\
-2053.48455255682	3.99367122456472e-05	\\
-2052.50577059659	3.78962216847074e-05	\\
-2051.52698863636	3.73675824451777e-05	\\
-2050.54820667614	3.83862471712088e-05	\\
-2049.56942471591	4.3313984456285e-05	\\
-2048.59064275568	4.27113728942239e-05	\\
-2047.61186079545	4.09110113779197e-05	\\
-2046.63307883523	4.74245569099291e-05	\\
-2045.654296875	4.48848554654443e-05	\\
-2044.67551491477	4.55882414710021e-05	\\
-2043.69673295455	4.58841989385647e-05	\\
-2042.71795099432	4.6980860522678e-05	\\
-2041.73916903409	4.45365540883945e-05	\\
-2040.76038707386	4.72230580586486e-05	\\
-2039.78160511364	4.73176010832827e-05	\\
-2038.80282315341	5.14800442511911e-05	\\
-2037.82404119318	5.00343141549991e-05	\\
-2036.84525923295	5.12268099331004e-05	\\
-2035.86647727273	5.03455661585471e-05	\\
-2034.8876953125	5.03195038926903e-05	\\
-2033.90891335227	5.3128891009028e-05	\\
-2032.93013139205	5.10825945264359e-05	\\
-2031.95134943182	5.37294548393087e-05	\\
-2030.97256747159	5.54823604318976e-05	\\
-2029.99378551136	5.45169877535638e-05	\\
-2029.01500355114	5.62233897320401e-05	\\
-2028.03622159091	5.56085530806716e-05	\\
-2027.05743963068	5.44045288709269e-05	\\
-2026.07865767045	5.89112867295819e-05	\\
-2025.09987571023	5.5581830301891e-05	\\
-2024.12109375	5.57896261254617e-05	\\
-2023.14231178977	5.63809104560075e-05	\\
-2022.16352982955	5.52763627531626e-05	\\
-2021.18474786932	6.06435281779398e-05	\\
-2020.20596590909	6.24467012673247e-05	\\
-2019.22718394886	6.31560041671061e-05	\\
-2018.24840198864	5.98358931599768e-05	\\
-2017.26962002841	5.77429955439534e-05	\\
-2016.29083806818	5.98979528347406e-05	\\
-2015.31205610795	5.84327308015894e-05	\\
-2014.33327414773	5.77192157286447e-05	\\
-2013.3544921875	5.93179457666343e-05	\\
-2012.37571022727	6.14115363059159e-05	\\
-2011.39692826705	5.78123164883512e-05	\\
-2010.41814630682	5.86038896949792e-05	\\
-2009.43936434659	5.94733149634942e-05	\\
-2008.46058238636	5.87383979539598e-05	\\
-2007.48180042614	5.83274887504255e-05	\\
-2006.50301846591	5.92612847610127e-05	\\
-2005.52423650568	5.80721704688347e-05	\\
-2004.54545454545	5.892909277581e-05	\\
-2003.56667258523	5.63613678719514e-05	\\
-2002.587890625	6.06659369818632e-05	\\
-2001.60910866477	6.38780719635436e-05	\\
-2000.63032670455	5.60166655470374e-05	\\
-1999.65154474432	5.8436829348536e-05	\\
-1998.67276278409	5.74844807113399e-05	\\
-1997.69398082386	5.79528191532625e-05	\\
-1996.71519886364	6.14768237064682e-05	\\
-1995.73641690341	6.22065470607571e-05	\\
-1994.75763494318	6.031741808771e-05	\\
-1993.77885298295	5.89952582267569e-05	\\
-1992.80007102273	6.46468652169734e-05	\\
-1991.8212890625	6.03291787655584e-05	\\
-1990.84250710227	6.21265188151232e-05	\\
-1989.86372514205	6.21037129393651e-05	\\
-1988.88494318182	5.93953335360826e-05	\\
-1987.90616122159	6.35197998460964e-05	\\
-1986.92737926136	6.15405665022123e-05	\\
-1985.94859730114	6.10595663258426e-05	\\
-1984.96981534091	5.95205049484844e-05	\\
-1983.99103338068	6.01383831091874e-05	\\
-1983.01225142045	5.79836502542568e-05	\\
-1982.03346946023	6.14589115082668e-05	\\
-1981.0546875	5.73858399948661e-05	\\
-1980.07590553977	5.81878381909325e-05	\\
-1979.09712357955	6.18551123373764e-05	\\
-1978.11834161932	5.77084719807117e-05	\\
-1977.13955965909	5.8795736266134e-05	\\
-1976.16077769886	6.170793092055e-05	\\
-1975.18199573864	5.58373021463525e-05	\\
-1974.20321377841	6.02663506712593e-05	\\
-1973.22443181818	6.00253497169692e-05	\\
-1972.24564985795	5.71889860586288e-05	\\
-1971.26686789773	5.97980863377099e-05	\\
-1970.2880859375	5.91257533915014e-05	\\
-1969.30930397727	5.48109554562318e-05	\\
-1968.33052201705	5.93588038750038e-05	\\
-1967.35174005682	5.76527560660069e-05	\\
-1966.37295809659	5.53763655591351e-05	\\
-1965.39417613636	6.14139819152746e-05	\\
-1964.41539417614	5.58887544851984e-05	\\
-1963.43661221591	5.79313586264886e-05	\\
-1962.45783025568	6.1177645550661e-05	\\
-1961.47904829545	5.71546818593453e-05	\\
-1960.50026633523	5.55433389079216e-05	\\
-1959.521484375	5.99964569320623e-05	\\
-1958.54270241477	5.4967968077828e-05	\\
-1957.56392045455	5.47249149210386e-05	\\
-1956.58513849432	5.82733387771915e-05	\\
-1955.60635653409	5.68640049703634e-05	\\
-1954.62757457386	5.54541273636918e-05	\\
-1953.64879261364	5.79571275992206e-05	\\
-1952.67001065341	5.59453815704224e-05	\\
-1951.69122869318	5.26514304160461e-05	\\
-1950.71244673295	5.83471386939806e-05	\\
-1949.73366477273	5.48117212622323e-05	\\
-1948.7548828125	5.43578215700279e-05	\\
-1947.77610085227	5.77560461080472e-05	\\
-1946.79731889205	5.42659728123409e-05	\\
-1945.81853693182	5.56796983085735e-05	\\
-1944.83975497159	5.79292561756387e-05	\\
-1943.86097301136	5.44418895997764e-05	\\
-1942.88219105114	5.67253334639094e-05	\\
-1941.90340909091	5.46809954511213e-05	\\
-1940.92462713068	5.37864037331661e-05	\\
-1939.94584517045	5.71248193849103e-05	\\
-1938.96706321023	5.45992873981334e-05	\\
-1937.98828125	5.81701992680689e-05	\\
-1937.00949928977	5.12265146696942e-05	\\
-1936.03071732955	5.24137121833408e-05	\\
-1935.05193536932	5.37423229403369e-05	\\
-1934.07315340909	5.06441332447232e-05	\\
-1933.09437144886	5.11101985877757e-05	\\
-1932.11558948864	5.76463526148785e-05	\\
-1931.13680752841	5.31662962272553e-05	\\
-1930.15802556818	5.39559904858485e-05	\\
-1929.17924360795	5.52248874033302e-05	\\
-1928.20046164773	5.15338297467112e-05	\\
-1927.2216796875	5.5624774015676e-05	\\
-1926.24289772727	4.97783297499793e-05	\\
-1925.26411576705	4.94868446162333e-05	\\
-1924.28533380682	5.41898345657032e-05	\\
-1923.30655184659	5.40903591737497e-05	\\
-1922.32776988636	5.27476521670546e-05	\\
-1921.34898792614	5.38510005717516e-05	\\
-1920.37020596591	5.04956927259365e-05	\\
-1919.39142400568	5.21605666784246e-05	\\
-1918.41264204545	5.15122877305239e-05	\\
-1917.43386008523	5.23214379018303e-05	\\
-1916.455078125	5.37911124935694e-05	\\
-1915.47629616477	5.24308426023866e-05	\\
-1914.49751420455	5.37420599425086e-05	\\
-1913.51873224432	5.05340013375903e-05	\\
-1912.53995028409	5.70924371269171e-05	\\
-1911.56116832386	5.04879407541922e-05	\\
-1910.58238636364	5.00104112523317e-05	\\
-1909.60360440341	5.05389569145054e-05	\\
-1908.62482244318	5.01720555580318e-05	\\
-1907.64604048295	5.63773961142622e-05	\\
-1906.66725852273	5.08738424616318e-05	\\
-1905.6884765625	4.93984958507744e-05	\\
-1904.70969460227	5.19656590549529e-05	\\
-1903.73091264205	5.19706891885212e-05	\\
-1902.75213068182	5.17470728341799e-05	\\
-1901.77334872159	5.424446568604e-05	\\
-1900.79456676136	5.38345330311366e-05	\\
-1899.81578480114	5.75998435338664e-05	\\
-1898.83700284091	4.98517739390293e-05	\\
-1897.85822088068	5.44978795745926e-05	\\
-1896.87943892045	4.89851838310027e-05	\\
-1895.90065696023	5.31693383488575e-05	\\
-1894.921875	5.46472321501904e-05	\\
-1893.94309303977	5.08388873557546e-05	\\
-1892.96431107955	5.72630439463298e-05	\\
-1891.98552911932	5.36868269448146e-05	\\
-1891.00674715909	5.4249787313813e-05	\\
-1890.02796519886	5.64467378732759e-05	\\
-1889.04918323864	5.49077324743913e-05	\\
-1888.07040127841	5.87040574488291e-05	\\
-1887.09161931818	5.72134135942022e-05	\\
-1886.11283735795	5.64336364954048e-05	\\
-1885.13405539773	5.90914482902265e-05	\\
-1884.1552734375	6.03709775634363e-05	\\
-1883.17649147727	5.76704120242966e-05	\\
-1882.19770951705	5.82477797189117e-05	\\
-1881.21892755682	5.92347531614797e-05	\\
-1880.24014559659	5.94608247915881e-05	\\
-1879.26136363636	5.66973372540517e-05	\\
-1878.28258167614	6.45062145214487e-05	\\
-1877.30379971591	6.42603540653948e-05	\\
-1876.32501775568	6.28188093397921e-05	\\
-1875.34623579545	6.19739866660165e-05	\\
-1874.36745383523	5.97137345066336e-05	\\
-1873.388671875	6.38445534250981e-05	\\
-1872.40988991477	5.76830128681986e-05	\\
-1871.43110795455	6.43485860512057e-05	\\
-1870.45232599432	6.16083660151943e-05	\\
-1869.47354403409	6.1389464464702e-05	\\
-1868.49476207386	6.40867646299348e-05	\\
-1867.51598011364	5.54548621395082e-05	\\
-1866.53719815341	6.50379747396635e-05	\\
-1865.55841619318	6.50034489596605e-05	\\
-1864.57963423295	6.63667294029433e-05	\\
-1863.60085227273	6.43259787022565e-05	\\
-1862.6220703125	6.25703135983833e-05	\\
-1861.64328835227	6.36397601538391e-05	\\
-1860.66450639205	6.23848220523428e-05	\\
-1859.68572443182	6.39343614045223e-05	\\
-1858.70694247159	6.41961074570486e-05	\\
-1857.72816051136	5.92471527747254e-05	\\
-1856.74937855114	6.94564850330023e-05	\\
-1855.77059659091	6.35645110993203e-05	\\
-1854.79181463068	6.45764976632524e-05	\\
-1853.81303267045	6.80415067961354e-05	\\
-1852.83425071023	6.41335680911334e-05	\\
-1851.85546875	6.61762351726076e-05	\\
-1850.87668678977	7.11232956528646e-05	\\
-1849.89790482955	6.6797804786327e-05	\\
-1848.91912286932	6.97816185910157e-05	\\
-1847.94034090909	6.72817226141719e-05	\\
-1846.96155894886	6.5924860299285e-05	\\
-1845.98277698864	7.23119404103782e-05	\\
-1845.00399502841	6.73145703777201e-05	\\
-1844.02521306818	6.55583466905138e-05	\\
-1843.04643110795	6.95383833545663e-05	\\
-1842.06764914773	7.13849451239681e-05	\\
-1841.0888671875	7.15678656377683e-05	\\
-1840.11008522727	7.1174907103169e-05	\\
-1839.13130326705	6.79001949466957e-05	\\
-1838.15252130682	7.06568716381401e-05	\\
-1837.17373934659	7.17507457714535e-05	\\
-1836.19495738636	6.76636511642795e-05	\\
-1835.21617542614	7.0048675204497e-05	\\
-1834.23739346591	7.31331304910379e-05	\\
-1833.25861150568	7.13231762241911e-05	\\
-1832.27982954545	7.30621322756592e-05	\\
-1831.30104758523	6.88385803903191e-05	\\
-1830.322265625	7.43269713175628e-05	\\
-1829.34348366477	7.1996441207218e-05	\\
-1828.36470170455	7.20322207195167e-05	\\
-1827.38591974432	6.78796757445884e-05	\\
-1826.40713778409	6.98252475200387e-05	\\
-1825.42835582386	7.43323773675317e-05	\\
-1824.44957386364	6.99998554260749e-05	\\
-1823.47079190341	7.43980010095789e-05	\\
-1822.49200994318	7.44733973248138e-05	\\
-1821.51322798295	7.61977042633063e-05	\\
-1820.53444602273	7.50043043978092e-05	\\
-1819.5556640625	7.33997748355558e-05	\\
-1818.57688210227	7.41424820374005e-05	\\
-1817.59810014205	7.54763308490672e-05	\\
-1816.61931818182	7.62681953317253e-05	\\
-1815.64053622159	7.97067785162665e-05	\\
-1814.66175426136	7.6575473853702e-05	\\
-1813.68297230114	7.72902795663599e-05	\\
-1812.70419034091	7.83487993210411e-05	\\
-1811.72540838068	7.95839421184105e-05	\\
-1810.74662642045	7.63653049876728e-05	\\
-1809.76784446023	8.07983866442329e-05	\\
-1808.7890625	7.78418865578471e-05	\\
-1807.81028053977	8.17511660833686e-05	\\
-1806.83149857955	7.57245911078879e-05	\\
-1805.85271661932	8.07556718836621e-05	\\
-1804.87393465909	7.55831863864765e-05	\\
-1803.89515269886	7.60883659601609e-05	\\
-1802.91637073864	8.14745446460689e-05	\\
-1801.93758877841	7.92273031211842e-05	\\
-1800.95880681818	7.97684874597152e-05	\\
-1799.98002485795	8.15677270196445e-05	\\
-1799.00124289773	7.87414998582346e-05	\\
-1798.0224609375	7.58166099256122e-05	\\
-1797.04367897727	7.81387092470699e-05	\\
-1796.06489701705	8.02389902021391e-05	\\
-1795.08611505682	8.0468363221945e-05	\\
-1794.10733309659	7.92096770535643e-05	\\
-1793.12855113636	7.85709863792494e-05	\\
-1792.14976917614	7.96064683050768e-05	\\
-1791.17098721591	7.70779226921006e-05	\\
-1790.19220525568	8.20806611436575e-05	\\
-1789.21342329545	7.92718264413134e-05	\\
-1788.23464133523	7.96981907928936e-05	\\
-1787.255859375	8.18705699683388e-05	\\
-1786.27707741477	7.97061294608944e-05	\\
-1785.29829545455	8.27557438343223e-05	\\
-1784.31951349432	8.25399167620855e-05	\\
-1783.34073153409	7.75665227238093e-05	\\
-1782.36194957386	8.10939294426886e-05	\\
-1781.38316761364	7.81951116831714e-05	\\
-1780.40438565341	8.1654704933187e-05	\\
-1779.42560369318	8.23787371174378e-05	\\
-1778.44682173295	7.66456030272768e-05	\\
-1777.46803977273	8.70905410353223e-05	\\
-1776.4892578125	7.33847108132582e-05	\\
-1775.51047585227	8.60024959527438e-05	\\
-1774.53169389205	7.90614283634624e-05	\\
-1773.55291193182	8.34143688723795e-05	\\
-1772.57412997159	8.25993707873265e-05	\\
-1771.59534801136	7.85428021560494e-05	\\
-1770.61656605114	8.40054732250743e-05	\\
-1769.63778409091	8.17073753148636e-05	\\
-1768.65900213068	8.17338574850082e-05	\\
-1767.68022017045	9.14587320300602e-05	\\
-1766.70143821023	7.84732726968179e-05	\\
-1765.72265625	8.88798736038016e-05	\\
-1764.74387428977	7.75798861371638e-05	\\
-1763.76509232955	8.79226160177759e-05	\\
-1762.78631036932	8.46602753921796e-05	\\
-1761.80752840909	8.07428415714965e-05	\\
-1760.82874644886	9.24244276022106e-05	\\
-1759.84996448864	7.97400840030714e-05	\\
-1758.87118252841	8.39993373596894e-05	\\
-1757.89240056818	8.42403096863876e-05	\\
-1756.91361860795	8.64066741412109e-05	\\
-1755.93483664773	8.18412266825067e-05	\\
-1754.9560546875	8.88209698743684e-05	\\
-1753.97727272727	7.99912389307336e-05	\\
-1752.99849076705	8.74363053106705e-05	\\
-1752.01970880682	9.19596803489205e-05	\\
-1751.04092684659	7.59719019799725e-05	\\
-1750.06214488636	9.71927239222502e-05	\\
-1749.08336292614	7.60918131230214e-05	\\
-1748.10458096591	8.59964955775735e-05	\\
-1747.12579900568	9.14465622051768e-05	\\
-1746.14701704545	7.65166484098842e-05	\\
-1745.16823508523	0.000102883858309732	\\
-1744.189453125	8.49146677552517e-05	\\
-1743.21067116477	8.75163484687299e-05	\\
-1742.23188920455	9.28393785957647e-05	\\
-1741.25310724432	8.12104881456606e-05	\\
-1740.27432528409	0.000100054073688938	\\
-1739.29554332386	9.33220394416491e-05	\\
-1738.31676136364	9.25799040217914e-05	\\
-1737.33797940341	9.15449593891272e-05	\\
-1736.35919744318	9.44858471126303e-05	\\
-1735.38041548295	8.68461842111088e-05	\\
-1734.40163352273	9.43700914474884e-05	\\
-1733.4228515625	8.60128913465381e-05	\\
-1732.44406960227	9.72617621304728e-05	\\
-1731.46528764205	9.66951527054906e-05	\\
-1730.48650568182	8.71058868666556e-05	\\
-1729.50772372159	9.48866715437594e-05	\\
-1728.52894176136	9.27443548220071e-05	\\
-1727.55015980114	8.95359415933584e-05	\\
-1726.57137784091	9.8915484800154e-05	\\
-1725.59259588068	8.93659360545324e-05	\\
-1724.61381392045	9.84977588454822e-05	\\
-1723.63503196023	9.82068810073367e-05	\\
-1722.65625	9.5037935077077e-05	\\
-1721.67746803977	9.52005481178417e-05	\\
-1720.69868607955	9.54912355326844e-05	\\
-1719.71990411932	9.44514874440857e-05	\\
-1718.74112215909	0.000107481775085565	\\
-1717.76234019886	9.3764390988555e-05	\\
-1716.78355823864	0.000103678449615342	\\
-1715.80477627841	0.000100327236242762	\\
-1714.82599431818	0.00010050546486179	\\
-1713.84721235795	0.000103912976022538	\\
-1712.86843039773	9.91920314667752e-05	\\
-1711.8896484375	0.000102092334548311	\\
-1710.91086647727	9.58406411677006e-05	\\
-1709.93208451705	0.000104484194656325	\\
-1708.95330255682	9.97395948428966e-05	\\
-1707.97452059659	0.000101257933417711	\\
-1706.99573863636	0.000102650328903069	\\
-1706.01695667614	0.000104253187728651	\\
-1705.03817471591	0.000103272865818578	\\
-1704.05939275568	0.000103961915429993	\\
-1703.08061079545	0.000103493499507172	\\
-1702.10182883523	0.000104143520820736	\\
-1701.123046875	0.000107891219917535	\\
-1700.14426491477	0.00010090811218462	\\
-1699.16548295455	0.000104839833243542	\\
-1698.18670099432	0.000104085033472544	\\
-1697.20791903409	0.000108702575120128	\\
-1696.22913707386	0.000106889756281939	\\
-1695.25035511364	0.00010777470468514	\\
-1694.27157315341	0.00010884867962707	\\
-1693.29279119318	0.000108673536192188	\\
-1692.31400923295	0.000107324864107667	\\
-1691.33522727273	0.000112313798688686	\\
-1690.3564453125	0.000109128914377801	\\
-1689.37766335227	0.000114573376700618	\\
-1688.39888139205	0.00011297187143527	\\
-1687.42009943182	0.000110653136835762	\\
-1686.44131747159	0.000112486055770762	\\
-1685.46253551136	0.000106096707208828	\\
-1684.48375355114	0.000110732014601248	\\
-1683.50497159091	0.000111027815177694	\\
-1682.52618963068	0.000114014819101449	\\
-1681.54740767045	0.000117970929694056	\\
-1680.56862571023	0.000114076100339809	\\
-1679.58984375	0.000116471904882797	\\
-1678.61106178977	0.000110566656352396	\\
-1677.63227982955	0.000114761605163442	\\
-1676.65349786932	0.000119025160966206	\\
-1675.67471590909	0.000109626139405338	\\
-1674.69593394886	0.000117871834206702	\\
-1673.71715198864	0.000118389840093096	\\
-1672.73837002841	0.000112665164000839	\\
-1671.75958806818	0.000117791724757931	\\
-1670.78080610795	0.000112943204086564	\\
-1669.80202414773	0.000112870588858183	\\
-1668.8232421875	0.000114314058619039	\\
-1667.84446022727	0.000109238194423996	\\
-1666.86567826705	0.000114761210373554	\\
-1665.88689630682	0.00011203731377513	\\
-1664.90811434659	0.000111225928065881	\\
-1663.92933238636	0.000119300578606737	\\
-1662.95055042614	0.000108241869226818	\\
-1661.97176846591	0.000119635196659269	\\
-1660.99298650568	0.000109506478810979	\\
-1660.01420454545	0.000112689218072248	\\
-1659.03542258523	0.000114603531168484	\\
-1658.056640625	0.000108459261376815	\\
-1657.07785866477	0.000119475709643919	\\
-1656.09907670455	0.000109059625352328	\\
-1655.12029474432	0.000111822407395743	\\
-1654.14151278409	0.000111050101717009	\\
-1653.16273082386	0.000104486635126206	\\
-1652.18394886364	0.000113252559482829	\\
-1651.20516690341	0.000102307701315261	\\
-1650.22638494318	0.000111466854908583	\\
-1649.24760298295	0.000112046124859643	\\
-1648.26882102273	0.000109296556217961	\\
-1647.2900390625	0.000112086798207778	\\
-1646.31125710227	9.99886169842141e-05	\\
-1645.33247514205	0.000105410776788084	\\
-1644.35369318182	0.000109267580638905	\\
-1643.37491122159	0.000109843349643415	\\
-1642.39612926136	0.00010953478061216	\\
-1641.41734730114	0.000103781894405112	\\
-1640.43856534091	0.000106943198875711	\\
-1639.45978338068	0.000105143312794338	\\
-1638.48100142045	0.000106073820785327	\\
-1637.50221946023	0.000101068327774741	\\
-1636.5234375	9.79085000495474e-05	\\
-1635.54465553977	0.000106984024079666	\\
-1634.56587357955	0.000103950393951222	\\
-1633.58709161932	0.000110002562077453	\\
-1632.60830965909	0.000100662051070447	\\
-1631.62952769886	0.000102448378484961	\\
-1630.65074573864	0.000104941249404166	\\
-1629.67196377841	9.81433831343632e-05	\\
-1628.69318181818	0.000105231321808819	\\
-1627.71439985795	9.93271203979592e-05	\\
-1626.73561789773	9.57990315261617e-05	\\
-1625.7568359375	0.000103916156828444	\\
-1624.77805397727	9.38841183581585e-05	\\
-1623.79927201705	0.000100904389794152	\\
-1622.82049005682	0.000101401952098663	\\
-1621.84170809659	0.000101552571137258	\\
-1620.86292613636	9.95386072633876e-05	\\
-1619.88414417614	0.000101435242632236	\\
-1618.90536221591	9.68677193561749e-05	\\
-1617.92658025568	0.000103387223724518	\\
-1616.94779829545	9.59880608905145e-05	\\
-1615.96901633523	9.83973470665733e-05	\\
-1614.990234375	0.00010103764239282	\\
-1614.01145241477	9.79408060898293e-05	\\
-1613.03267045455	0.000100628200660164	\\
-1612.05388849432	0.000102201771109448	\\
-1611.07510653409	0.000102994530419474	\\
-1610.09632457386	0.000106096070844662	\\
-1609.11754261364	9.99740116831253e-05	\\
-1608.13876065341	0.00010590598209015	\\
-1607.15997869318	0.00010468027476166	\\
-1606.18119673295	9.8675121734299e-05	\\
-1605.20241477273	9.93360833451968e-05	\\
-1604.2236328125	0.000107355548490679	\\
-1603.24485085227	9.84872950627665e-05	\\
-1602.26606889205	0.000104947699918472	\\
-1601.28728693182	0.00010271214470255	\\
-1600.30850497159	9.71694882575627e-05	\\
-1599.32972301136	0.000108854212912692	\\
-1598.35094105114	9.91810669737604e-05	\\
-1597.37215909091	0.000110517978438457	\\
-1596.39337713068	0.000102140914293472	\\
-1595.41459517045	0.000106972911044518	\\
-1594.43581321023	0.000105519350018764	\\
-1593.45703125	0.000110575605607309	\\
-1592.47824928977	0.000109049671921997	\\
-1591.49946732955	0.000112629586856585	\\
-1590.52068536932	9.96935086774501e-05	\\
-1589.54190340909	0.000110698100692552	\\
-1588.56312144886	0.000107181106243943	\\
-1587.58433948864	0.000110997622448292	\\
-1586.60555752841	0.000107903121604873	\\
-1585.62677556818	0.000107332765405773	\\
-1584.64799360795	0.000105964531298322	\\
-1583.66921164773	0.0001138792811711	\\
-1582.6904296875	0.000111312164661998	\\
-1581.71164772727	0.000110455755326287	\\
-1580.73286576705	0.000114050803335643	\\
-1579.75408380682	0.000108058506244907	\\
-1578.77530184659	0.000113350960118912	\\
-1577.79651988636	0.000114067126785158	\\
-1576.81773792614	0.000113890143463188	\\
-1575.83895596591	0.000115962299555546	\\
-1574.86017400568	0.000117384047106676	\\
-1573.88139204545	0.000116400915588462	\\
-1572.90261008523	0.000118254899668895	\\
-1571.923828125	0.000117814715141252	\\
-1570.94504616477	0.000116320977339299	\\
-1569.96626420455	0.000116570912803038	\\
-1568.98748224432	0.000121706393174293	\\
-1568.00870028409	0.000115899494293144	\\
-1567.02991832386	0.000115996838696029	\\
-1566.05113636364	0.000116964029413109	\\
-1565.07235440341	0.00011122994969926	\\
-1564.09357244318	0.000115862730429121	\\
-1563.11479048295	0.000113377065860572	\\
-1562.13600852273	0.000119731906957325	\\
-1561.1572265625	0.000114653633476709	\\
-1560.17844460227	0.00010811848563983	\\
-1559.19966264205	0.000114775539720416	\\
-1558.22088068182	0.00011163561810863	\\
-1557.24209872159	0.000115027069386629	\\
-1556.26331676136	0.000113794103760025	\\
-1555.28453480114	0.000111949849464752	\\
-1554.30575284091	0.000114606391656174	\\
-1553.32697088068	0.000106786582885036	\\
-1552.34818892045	0.000110950202822576	\\
-1551.36940696023	0.000109454966796774	\\
-1550.390625	0.000111271277425035	\\
-1549.41184303977	0.000111531347943539	\\
-1548.43306107955	0.000102357182413687	\\
-1547.45427911932	0.000117459283334707	\\
-1546.47549715909	0.000107370026121851	\\
-1545.49671519886	0.000110353183339505	\\
-1544.51793323864	0.000114911985508392	\\
-1543.53915127841	0.000102411822258906	\\
-1542.56036931818	0.000112317291783999	\\
-1541.58158735795	0.00010172219948162	\\
-1540.60280539773	0.000117817931666958	\\
-1539.6240234375	0.000117372697552687	\\
-1538.64524147727	0.000109772914358072	\\
-1537.66645951705	0.000117120230382186	\\
-1536.68767755682	0.000101988982652503	\\
-1535.70889559659	0.000116083552177763	\\
-1534.73011363636	0.000115385195148601	\\
-1533.75133167614	0.000115646404741796	\\
-1532.77254971591	0.000116618069022677	\\
-1531.79376775568	0.000111640338755447	\\
-1530.81498579545	0.000122060710629784	\\
-1529.83620383523	0.000113238660747362	\\
-1528.857421875	0.000120144308699213	\\
-1527.87863991477	0.000118705304740787	\\
-1526.89985795455	0.000115555079501241	\\
-1525.92107599432	0.000123091130990685	\\
-1524.94229403409	0.000122285491659935	\\
-1523.96351207386	0.000122652722820418	\\
-1522.98473011364	0.000121174845726502	\\
-1522.00594815341	0.00011900136350701	\\
-1521.02716619318	0.00012221006558806	\\
-1520.04838423295	0.000119005572182875	\\
-1519.06960227273	0.000123223649185601	\\
-1518.0908203125	0.000116619255184793	\\
-1517.11203835227	0.000118477227223214	\\
-1516.13325639205	0.00011610986525931	\\
-1515.15447443182	0.000113485478296489	\\
-1514.17569247159	0.000113452432220041	\\
-1513.19691051136	0.000117556444988503	\\
-1512.21812855114	0.000110792650242792	\\
-1511.23934659091	0.000120694273501039	\\
-1510.26056463068	0.000119027472294771	\\
-1509.28178267045	0.000114729005725613	\\
-1508.30300071023	0.000115148955841686	\\
-1507.32421875	0.000108743030561848	\\
-1506.34543678977	0.000104458825813365	\\
-1505.36665482955	0.000113777424541459	\\
-1504.38787286932	0.000101739047356256	\\
-1503.40909090909	0.000112053555993615	\\
-1502.43030894886	0.000104817229582934	\\
-1501.45152698864	0.000105861076427357	\\
-1500.47274502841	9.63241460138151e-05	\\
-1499.49396306818	0.000107958945505198	\\
-1498.51518110795	0.000103800037086757	\\
-1497.53639914773	0.000101886394694159	\\
-1496.5576171875	0.000104706837403357	\\
-1495.57883522727	0.000104153128073522	\\
-1494.60005326705	0.000105308961988874	\\
-1493.62127130682	0.000102679674886555	\\
-1492.64248934659	0.000102310369421381	\\
-1491.66370738636	0.000100743582007303	\\
-1490.68492542614	0.000100202331334853	\\
-1489.70614346591	9.93101505849588e-05	\\
-1488.72736150568	0.00010248177283204	\\
-1487.74857954545	9.5103891108904e-05	\\
-1486.76979758523	9.87706724220429e-05	\\
-1485.791015625	9.55937244699063e-05	\\
-1484.81223366477	9.80783633458233e-05	\\
-1483.83345170455	9.18997125450044e-05	\\
-1482.85466974432	9.64510545316348e-05	\\
-1481.87588778409	8.87047197973062e-05	\\
-1480.89710582386	9.42237425969454e-05	\\
-1479.91832386364	9.49064958533135e-05	\\
-1478.93954190341	8.55890111567262e-05	\\
-1477.96075994318	9.08340460228802e-05	\\
-1476.98197798295	8.84558337018664e-05	\\
-1476.00319602273	8.86293870202929e-05	\\
-1475.0244140625	9.07400130256613e-05	\\
-1474.04563210227	8.5701422648172e-05	\\
-1473.06685014205	8.78382972363338e-05	\\
-1472.08806818182	8.56077173984877e-05	\\
-1471.10928622159	8.21147114010721e-05	\\
-1470.13050426136	8.31377120067996e-05	\\
-1469.15172230114	8.34726000497283e-05	\\
-1468.17294034091	8.39667410894041e-05	\\
-1467.19415838068	8.3267592345294e-05	\\
-1466.21537642045	7.84178251754897e-05	\\
-1465.23659446023	7.83400907556594e-05	\\
-1464.2578125	8.12999341417002e-05	\\
-1463.27903053977	8.15994938432249e-05	\\
-1462.30024857955	7.57847478299114e-05	\\
-1461.32146661932	7.74993512157753e-05	\\
-1460.34268465909	8.22914338647654e-05	\\
-1459.36390269886	7.5314807438369e-05	\\
-1458.38512073864	7.3782343463816e-05	\\
-1457.40633877841	7.93309650755616e-05	\\
-1456.42755681818	7.64680420061518e-05	\\
-1455.44877485795	8.08351906675585e-05	\\
-1454.46999289773	7.674545358359e-05	\\
-1453.4912109375	7.30918177857254e-05	\\
-1452.51242897727	7.05543502119789e-05	\\
-1451.53364701705	7.74240425434013e-05	\\
-1450.55486505682	7.72879172803704e-05	\\
-1449.57608309659	7.34647969891376e-05	\\
-1448.59730113636	7.54853148915668e-05	\\
-1447.61851917614	7.41026233991079e-05	\\
-1446.63973721591	7.22721303621739e-05	\\
-1445.66095525568	7.66621513414671e-05	\\
-1444.68217329545	7.39300562870543e-05	\\
-1443.70339133523	7.18716611572723e-05	\\
-1442.724609375	7.04991101651585e-05	\\
-1441.74582741477	7.18629667215879e-05	\\
-1440.76704545455	7.1946348162861e-05	\\
-1439.78826349432	7.29847985946387e-05	\\
-1438.80948153409	7.17256124430439e-05	\\
-1437.83069957386	7.44908253839465e-05	\\
-1436.85191761364	7.25279295655012e-05	\\
-1435.87313565341	6.93995650677204e-05	\\
-1434.89435369318	7.15423752921876e-05	\\
-1433.91557173295	7.10149049935035e-05	\\
-1432.93678977273	7.15677445192624e-05	\\
-1431.9580078125	7.12144548005141e-05	\\
-1430.97922585227	7.35174459530085e-05	\\
-1430.00044389205	7.44353169603084e-05	\\
-1429.02166193182	7.06756926800984e-05	\\
-1428.04287997159	7.47471909710281e-05	\\
-1427.06409801136	6.71154519665058e-05	\\
-1426.08531605114	7.48718767410704e-05	\\
-1425.10653409091	7.25608861951478e-05	\\
-1424.12775213068	6.43839766114899e-05	\\
-1423.14897017045	7.44097297764074e-05	\\
-1422.17018821023	6.82931991722567e-05	\\
-1421.19140625	7.51201137026657e-05	\\
-1420.21262428977	6.98773994352628e-05	\\
-1419.23384232955	6.59260957633283e-05	\\
-1418.25506036932	7.45793890085337e-05	\\
-1417.27627840909	6.85169641358063e-05	\\
-1416.29749644886	6.8916397523062e-05	\\
-1415.31871448864	7.20152859453534e-05	\\
-1414.33993252841	6.74308421961107e-05	\\
-1413.36115056818	6.72171692319184e-05	\\
-1412.38236860795	7.07703179164837e-05	\\
-1411.40358664773	6.32247864401085e-05	\\
-1410.4248046875	6.45814622937726e-05	\\
-1409.44602272727	6.46649229175297e-05	\\
-1408.46724076705	6.74659878993683e-05	\\
-1407.48845880682	6.52287242325247e-05	\\
-1406.50967684659	6.53732809175314e-05	\\
-1405.53089488636	7.04744553948287e-05	\\
-1404.55211292614	6.54266641013372e-05	\\
-1403.57333096591	6.61120935192129e-05	\\
-1402.59454900568	6.07741624739606e-05	\\
-1401.61576704545	6.70444427388617e-05	\\
-1400.63698508523	6.45137230574216e-05	\\
-1399.658203125	6.49188290980251e-05	\\
-1398.67942116477	6.48632503778453e-05	\\
-1397.70063920455	6.01825226882872e-05	\\
-1396.72185724432	6.41523958212766e-05	\\
-1395.74307528409	6.02381591349187e-05	\\
-1394.76429332386	6.71544160892113e-05	\\
-1393.78551136364	6.34156904377714e-05	\\
-1392.80672940341	6.48357747663121e-05	\\
-1391.82794744318	5.98801485199861e-05	\\
-1390.84916548295	7.10324427614412e-05	\\
-1389.87038352273	5.61332773058492e-05	\\
-1388.8916015625	6.61822157043821e-05	\\
-1387.91281960227	6.23490302212643e-05	\\
-1386.93403764205	6.26851258819205e-05	\\
-1385.95525568182	6.82082988786691e-05	\\
-1384.97647372159	6.59319134441482e-05	\\
-1383.99769176136	5.67303776799414e-05	\\
-1383.01890980114	6.53690147599157e-05	\\
-1382.04012784091	5.80343873148033e-05	\\
-1381.06134588068	6.23724408665018e-05	\\
-1380.08256392045	6.52115845801984e-05	\\
-1379.10378196023	5.86420841483599e-05	\\
-1378.125	6.14710623197641e-05	\\
-1377.14621803977	6.44788766094737e-05	\\
-1376.16743607955	6.32972419414464e-05	\\
-1375.18865411932	6.34573410913566e-05	\\
-1374.20987215909	6.24229441127981e-05	\\
-1373.23109019886	5.96685359117741e-05	\\
-1372.25230823864	6.08675219671737e-05	\\
-1371.27352627841	6.21175202455477e-05	\\
-1370.29474431818	6.17059034207161e-05	\\
-1369.31596235795	6.13924443997402e-05	\\
-1368.33718039773	6.35777211305345e-05	\\
-1367.3583984375	6.41477937645442e-05	\\
-1366.37961647727	6.07443873287732e-05	\\
-1365.40083451705	6.24309323478924e-05	\\
-1364.42205255682	6.50138135980226e-05	\\
-1363.44327059659	6.32832335420136e-05	\\
-1362.46448863636	6.37715667678839e-05	\\
-1361.48570667614	6.3131669129007e-05	\\
-1360.50692471591	6.57005991647192e-05	\\
-1359.52814275568	6.4504001598495e-05	\\
-1358.54936079545	6.34186167093051e-05	\\
-1357.57057883523	5.89289426300597e-05	\\
-1356.591796875	6.35433285396126e-05	\\
-1355.61301491477	7.17812400849058e-05	\\
-1354.63423295455	5.87928913441036e-05	\\
-1353.65545099432	6.08781967663705e-05	\\
-1352.67666903409	6.763412077611e-05	\\
-1351.69788707386	6.5385626961678e-05	\\
-1350.71910511364	6.83229445965428e-05	\\
-1349.74032315341	6.17185967198524e-05	\\
-1348.76154119318	6.55230297027775e-05	\\
-1347.78275923295	7.28926619557933e-05	\\
-1346.80397727273	6.35450410175194e-05	\\
-1345.8251953125	6.40907159183387e-05	\\
-1344.84641335227	6.71499933896364e-05	\\
-1343.86763139205	6.26322158483384e-05	\\
-1342.88884943182	6.56757061137347e-05	\\
-1341.91006747159	6.88432829508767e-05	\\
-1340.93128551136	6.40122342668675e-05	\\
-1339.95250355114	6.74215748581666e-05	\\
-1338.97372159091	6.50474324389933e-05	\\
-1337.99493963068	6.46381268030578e-05	\\
-1337.01615767045	6.5700581343446e-05	\\
-1336.03737571023	6.54025154504663e-05	\\
-1335.05859375	6.51186534943368e-05	\\
-1334.07981178977	6.75972664488272e-05	\\
-1333.10102982955	6.49455772123974e-05	\\
-1332.12224786932	6.87971469183353e-05	\\
-1331.14346590909	6.31217760457841e-05	\\
-1330.16468394886	6.65647324108511e-05	\\
-1329.18590198864	6.56657687493532e-05	\\
-1328.20712002841	6.08777739880321e-05	\\
-1327.22833806818	6.24967438474806e-05	\\
-1326.24955610795	6.95970307904517e-05	\\
-1325.27077414773	6.31390995989433e-05	\\
-1324.2919921875	6.42432637259983e-05	\\
-1323.31321022727	6.79422396469946e-05	\\
-1322.33442826705	6.71188562522012e-05	\\
-1321.35564630682	6.44384528732073e-05	\\
-1320.37686434659	6.01859846511576e-05	\\
-1319.39808238636	6.56448204488634e-05	\\
-1318.41930042614	5.72759576334118e-05	\\
-1317.44051846591	6.36454871264306e-05	\\
-1316.46173650568	6.69722422938509e-05	\\
-1315.48295454545	6.15817282214833e-05	\\
-1314.50417258523	6.24151465014982e-05	\\
-1313.525390625	6.74652152259395e-05	\\
-1312.54660866477	6.62038308478977e-05	\\
-1311.56782670455	6.02002465806806e-05	\\
-1310.58904474432	6.32499432507742e-05	\\
-1309.61026278409	5.35644979430262e-05	\\
-1308.63148082386	6.42582505575339e-05	\\
-1307.65269886364	6.40148289350144e-05	\\
-1306.67391690341	6.55117723026808e-05	\\
-1305.69513494318	6.70477761759113e-05	\\
-1304.71635298295	6.42196392851783e-05	\\
-1303.73757102273	6.49962024521929e-05	\\
-1302.7587890625	6.12222064349176e-05	\\
-1301.78000710227	6.65678880782066e-05	\\
-1300.80122514205	6.52829026676941e-05	\\
-1299.82244318182	5.86443141464507e-05	\\
-1298.84366122159	6.40048727717285e-05	\\
-1297.86487926136	6.28390626895765e-05	\\
-1296.88609730114	6.70580646250098e-05	\\
-1295.90731534091	6.81600043590372e-05	\\
-1294.92853338068	6.73343944143166e-05	\\
-1293.94975142045	6.35254849768846e-05	\\
-1292.97096946023	6.8396720455793e-05	\\
-1291.9921875	6.95306025040036e-05	\\
-1291.01340553977	7.16535486904712e-05	\\
-1290.03462357955	6.89775653624081e-05	\\
-1289.05584161932	6.95280960820002e-05	\\
-1288.07705965909	7.27291101853545e-05	\\
-1287.09827769886	6.48062892427927e-05	\\
-1286.11949573864	7.11973586099946e-05	\\
-1285.14071377841	6.89776680041013e-05	\\
-1284.16193181818	6.77620758216622e-05	\\
-1283.18314985795	7.12666601748941e-05	\\
-1282.20436789773	7.0368263934984e-05	\\
-1281.2255859375	7.10662787445969e-05	\\
-1280.24680397727	7.00649097221329e-05	\\
-1279.26802201705	7.48387777834824e-05	\\
-1278.28924005682	7.34494962330474e-05	\\
-1277.31045809659	7.25934020748257e-05	\\
-1276.33167613636	7.07291362782124e-05	\\
-1275.35289417614	7.02460233926155e-05	\\
-1274.37411221591	7.07902180400229e-05	\\
-1273.39533025568	7.95290555774426e-05	\\
-1272.41654829545	7.30733304248674e-05	\\
-1271.43776633523	7.79263525736223e-05	\\
-1270.458984375	7.59087736956173e-05	\\
-1269.48020241477	7.11054358029681e-05	\\
-1268.50142045455	7.39850512550219e-05	\\
-1267.52263849432	7.33644471559549e-05	\\
-1266.54385653409	7.05259194486128e-05	\\
-1265.56507457386	7.44877460264429e-05	\\
-1264.58629261364	7.12954111582764e-05	\\
-1263.60751065341	6.92731591592296e-05	\\
-1262.62872869318	6.96798888912948e-05	\\
-1261.64994673295	6.71878762406994e-05	\\
-1260.67116477273	7.13323173450988e-05	\\
-1259.6923828125	6.99544260403918e-05	\\
-1258.71360085227	6.91251984650181e-05	\\
-1257.73481889205	6.58089146408972e-05	\\
-1256.75603693182	7.16427483439094e-05	\\
-1255.77725497159	7.02180148459143e-05	\\
-1254.79847301136	6.82331933974385e-05	\\
-1253.81969105114	6.50761729170479e-05	\\
-1252.84090909091	6.7982763155239e-05	\\
-1251.86212713068	6.97427774038612e-05	\\
-1250.88334517045	6.99965319873786e-05	\\
-1249.90456321023	6.16972402616766e-05	\\
-1248.92578125	6.90608394593709e-05	\\
-1247.94699928977	7.12036514883714e-05	\\
-1246.96821732955	6.73276288538167e-05	\\
-1245.98943536932	7.51089263156498e-05	\\
-1245.01065340909	6.67856342813341e-05	\\
-1244.03187144886	6.77687728635653e-05	\\
-1243.05308948864	6.30594899704183e-05	\\
-1242.07430752841	7.02688906120023e-05	\\
-1241.09552556818	6.90950126080139e-05	\\
-1240.11674360795	6.87536174062221e-05	\\
-1239.13796164773	6.59346856665802e-05	\\
-1238.1591796875	7.11606264774428e-05	\\
-1237.18039772727	7.08841165132445e-05	\\
-1236.20161576705	6.20235793459906e-05	\\
-1235.22283380682	7.09232155169421e-05	\\
-1234.24405184659	6.78491179601326e-05	\\
-1233.26526988636	6.403917045221e-05	\\
-1232.28648792614	6.92181667372372e-05	\\
-1231.30770596591	6.86371189513373e-05	\\
-1230.32892400568	6.50105846799772e-05	\\
-1229.35014204545	7.35381712644954e-05	\\
-1228.37136008523	6.4105247234472e-05	\\
-1227.392578125	6.59644670544177e-05	\\
-1226.41379616477	7.0374636687281e-05	\\
-1225.43501420455	6.43105244344851e-05	\\
-1224.45623224432	6.9008206733843e-05	\\
-1223.47745028409	6.84994614137269e-05	\\
-1222.49866832386	6.90835995286125e-05	\\
-1221.51988636364	6.9485768995057e-05	\\
-1220.54110440341	6.62287811386821e-05	\\
-1219.56232244318	6.66629455297406e-05	\\
-1218.58354048295	7.18724242073892e-05	\\
-1217.60475852273	6.96663785375414e-05	\\
-1216.6259765625	6.743660785498e-05	\\
-1215.64719460227	6.88060642590453e-05	\\
-1214.66841264205	6.98230385825986e-05	\\
-1213.68963068182	7.17788381835568e-05	\\
-1212.71084872159	7.07670755608659e-05	\\
-1211.73206676136	6.90418612901082e-05	\\
-1210.75328480114	6.49179961781418e-05	\\
-1209.77450284091	7.13687413067307e-05	\\
-1208.79572088068	6.92003471600966e-05	\\
-1207.81693892045	7.26552849322838e-05	\\
-1206.83815696023	6.59276720148339e-05	\\
-1205.859375	7.01937821226805e-05	\\
-1204.88059303977	7.26293948595658e-05	\\
-1203.90181107955	6.97498720923992e-05	\\
-1202.92302911932	6.74912408964425e-05	\\
-1201.94424715909	6.69142673324821e-05	\\
-1200.96546519886	7.21658626090704e-05	\\
-1199.98668323864	6.70558815285499e-05	\\
-1199.00790127841	7.37119924802107e-05	\\
-1198.02911931818	6.77516790293282e-05	\\
-1197.05033735795	6.8919067304461e-05	\\
-1196.07155539773	6.63308323148036e-05	\\
-1195.0927734375	6.85546950643876e-05	\\
-1194.11399147727	6.81637765171596e-05	\\
-1193.13520951705	7.06620028182326e-05	\\
-1192.15642755682	6.62538973087948e-05	\\
-1191.17764559659	6.87614836060559e-05	\\
-1190.19886363636	7.00416094516207e-05	\\
-1189.22008167614	6.7119161720772e-05	\\
-1188.24129971591	6.92596325556606e-05	\\
-1187.26251775568	6.92054942946277e-05	\\
-1186.28373579545	6.55157060302652e-05	\\
-1185.30495383523	7.00299027540886e-05	\\
-1184.326171875	7.10864895307776e-05	\\
-1183.34738991477	6.95977789173867e-05	\\
-1182.36860795455	6.78807283673555e-05	\\
-1181.38982599432	6.64152630192257e-05	\\
-1180.41104403409	6.75927432244404e-05	\\
-1179.43226207386	6.81687648726898e-05	\\
-1178.45348011364	7.16253939712208e-05	\\
-1177.47469815341	6.62500320818051e-05	\\
-1176.49591619318	6.88170579578423e-05	\\
-1175.51713423295	6.69704858435076e-05	\\
-1174.53835227273	6.76137323825597e-05	\\
-1173.5595703125	6.46291882691736e-05	\\
-1172.58078835227	6.39684967877719e-05	\\
-1171.60200639205	6.50429943657795e-05	\\
-1170.62322443182	6.36811067833528e-05	\\
-1169.64444247159	7.12533146390782e-05	\\
-1168.66566051136	6.63080531098482e-05	\\
-1167.68687855114	6.96481963351562e-05	\\
-1166.70809659091	6.6880152443489e-05	\\
-1165.72931463068	6.78748210988951e-05	\\
-1164.75053267045	6.80613447723065e-05	\\
-1163.77175071023	6.59038967181849e-05	\\
-1162.79296875	6.4090789856072e-05	\\
-1161.81418678977	6.6443375958638e-05	\\
-1160.83540482955	6.46373839141848e-05	\\
-1159.85662286932	6.35430721064383e-05	\\
-1158.87784090909	6.73509910232222e-05	\\
-1157.89905894886	6.54865566566962e-05	\\
-1156.92027698864	6.40724400455962e-05	\\
-1155.94149502841	6.30521024523885e-05	\\
-1154.96271306818	7.08934971245045e-05	\\
-1153.98393110795	6.16271809452027e-05	\\
-1153.00514914773	6.59874625513641e-05	\\
-1152.0263671875	6.62420025765791e-05	\\
-1151.04758522727	6.3557918227914e-05	\\
-1150.06880326705	6.60563125645972e-05	\\
-1149.09002130682	6.50485290676293e-05	\\
-1148.11123934659	7.02906083323498e-05	\\
-1147.13245738636	6.10058065099975e-05	\\
-1146.15367542614	6.57038895508793e-05	\\
-1145.17489346591	6.69926196227969e-05	\\
-1144.19611150568	5.8727778823722e-05	\\
-1143.21732954545	5.97079717100843e-05	\\
-1142.23854758523	6.26008620773956e-05	\\
-1141.259765625	6.10612875911946e-05	\\
-1140.28098366477	6.65576783904134e-05	\\
-1139.30220170455	6.21094870925879e-05	\\
-1138.32341974432	6.16309151478107e-05	\\
-1137.34463778409	6.02343413634212e-05	\\
-1136.36585582386	6.57902264347514e-05	\\
-1135.38707386364	6.42938832429947e-05	\\
-1134.40829190341	6.2831106245158e-05	\\
-1133.42950994318	6.24425748709737e-05	\\
-1132.45072798295	6.20047997809859e-05	\\
-1131.47194602273	6.39524171371368e-05	\\
-1130.4931640625	6.42990667296674e-05	\\
-1129.51438210227	6.86230236004959e-05	\\
-1128.53560014205	6.35126866470001e-05	\\
-1127.55681818182	6.21992403968714e-05	\\
-1126.57803622159	6.13574944929872e-05	\\
-1125.59925426136	6.90742370775104e-05	\\
-1124.62047230114	6.25281449422499e-05	\\
-1123.64169034091	6.20472924685397e-05	\\
-1122.66290838068	6.58643115257113e-05	\\
-1121.68412642045	6.21280208557814e-05	\\
-1120.70534446023	5.8791944046492e-05	\\
-1119.7265625	6.55276129013756e-05	\\
-1118.74778053977	6.31376625036717e-05	\\
-1117.76899857955	6.25339011384832e-05	\\
-1116.79021661932	6.47323060108654e-05	\\
-1115.81143465909	6.83241025031196e-05	\\
-1114.83265269886	6.49174491741042e-05	\\
-1113.85387073864	5.93662199001657e-05	\\
-1112.87508877841	6.51726552884863e-05	\\
-1111.89630681818	6.97289024890121e-05	\\
-1110.91752485795	6.50728240116832e-05	\\
-1109.93874289773	6.65606782618762e-05	\\
-1108.9599609375	7.19630888134978e-05	\\
-1107.98117897727	6.55483973123442e-05	\\
-1107.00239701705	6.83503150882998e-05	\\
-1106.02361505682	6.13745099282085e-05	\\
-1105.04483309659	6.39515043269353e-05	\\
-1104.06605113636	7.16943613938939e-05	\\
-1103.08726917614	6.70766607636373e-05	\\
-1102.10848721591	6.45211666155676e-05	\\
-1101.12970525568	7.04606033888937e-05	\\
-1100.15092329545	5.54569060468966e-05	\\
-1099.17214133523	6.81042758556668e-05	\\
-1098.193359375	7.29457972600861e-05	\\
-1097.21457741477	6.22865370602066e-05	\\
-1096.23579545455	6.54114236499013e-05	\\
-1095.25701349432	6.469095876145e-05	\\
-1094.27823153409	6.60054524475506e-05	\\
-1093.29944957386	6.86521712645116e-05	\\
-1092.32066761364	6.36952573575155e-05	\\
-1091.34188565341	7.33910682171349e-05	\\
-1090.36310369318	7.07492129837223e-05	\\
-1089.38432173295	6.27202259645135e-05	\\
-1088.40553977273	7.14389991949601e-05	\\
-1087.4267578125	7.02435474572779e-05	\\
-1086.44797585227	6.30184330852954e-05	\\
-1085.46919389205	7.0575921769202e-05	\\
-1084.49041193182	7.15706562672213e-05	\\
-1083.51162997159	7.21164666670312e-05	\\
-1082.53284801136	6.97263145509775e-05	\\
-1081.55406605114	6.99799452737065e-05	\\
-1080.57528409091	6.49375491014594e-05	\\
-1079.59650213068	7.04631763501528e-05	\\
-1078.61772017045	7.23574639371004e-05	\\
-1077.63893821023	7.1605967765305e-05	\\
-1076.66015625	6.83022564908106e-05	\\
-1075.68137428977	6.76102403055838e-05	\\
-1074.70259232955	6.5029814658347e-05	\\
-1073.72381036932	6.94744742270887e-05	\\
-1072.74502840909	6.98395352372522e-05	\\
-1071.76624644886	6.63046484817999e-05	\\
-1070.78746448864	6.74190241438814e-05	\\
-1069.80868252841	6.8419798151557e-05	\\
-1068.82990056818	7.01981448514428e-05	\\
-1067.85111860795	6.93006689869806e-05	\\
-1066.87233664773	6.7258951455222e-05	\\
-1065.8935546875	6.78004857740086e-05	\\
-1064.91477272727	7.42923780001926e-05	\\
-1063.93599076705	6.70230074968201e-05	\\
-1062.95720880682	6.78611490176289e-05	\\
-1061.97842684659	6.69782515108321e-05	\\
-1060.99964488636	6.8318644660732e-05	\\
-1060.02086292614	6.51039166630762e-05	\\
-1059.04208096591	6.99487778588048e-05	\\
-1058.06329900568	6.81015515983378e-05	\\
-1057.08451704545	6.6287912920515e-05	\\
-1056.10573508523	7.2165677997677e-05	\\
-1055.126953125	6.82759995901256e-05	\\
-1054.14817116477	6.36564053244414e-05	\\
-1053.16938920455	6.37760771625752e-05	\\
-1052.19060724432	6.64578509485448e-05	\\
-1051.21182528409	6.84051914989086e-05	\\
-1050.23304332386	6.43138081708867e-05	\\
-1049.25426136364	6.99147988420234e-05	\\
-1048.27547940341	6.40263506627129e-05	\\
-1047.29669744318	6.25610991784982e-05	\\
-1046.31791548295	7.0574464214248e-05	\\
-1045.33913352273	6.06974090839357e-05	\\
-1044.3603515625	6.59463745351008e-05	\\
-1043.38156960227	6.27197536818028e-05	\\
-1042.40278764205	6.33086622708684e-05	\\
-1041.42400568182	6.54161114565789e-05	\\
-1040.44522372159	6.69152694117437e-05	\\
-1039.46644176136	6.15665274266747e-05	\\
-1038.48765980114	6.44176260066688e-05	\\
-1037.50887784091	6.3520017494572e-05	\\
-1036.53009588068	6.57609851696029e-05	\\
-1035.55131392045	6.543543268887e-05	\\
-1034.57253196023	6.46306102090772e-05	\\
-1033.59375	6.63229401581665e-05	\\
-1032.61496803977	6.1586369061888e-05	\\
-1031.63618607955	6.64164117165653e-05	\\
-1030.65740411932	6.39621744932414e-05	\\
-1029.67862215909	5.92518811357062e-05	\\
-1028.69984019886	6.0491822019308e-05	\\
-1027.72105823864	6.03622311063464e-05	\\
-1026.74227627841	6.44392614739832e-05	\\
-1025.76349431818	6.60382560770714e-05	\\
-1024.78471235795	6.3924395234144e-05	\\
-1023.80593039773	6.35127417476145e-05	\\
-1022.8271484375	5.76271918688749e-05	\\
-1021.84836647727	6.45616223479899e-05	\\
-1020.86958451705	6.29956592154883e-05	\\
-1019.89080255682	6.21063352469651e-05	\\
-1018.91202059659	6.41468774139692e-05	\\
-1017.93323863636	6.23617334227881e-05	\\
-1016.95445667614	6.33469584052962e-05	\\
-1015.97567471591	6.15714608554485e-05	\\
-1014.99689275568	6.27266171028978e-05	\\
-1014.01811079545	6.53223290305977e-05	\\
-1013.03932883523	6.57285753976103e-05	\\
-1012.060546875	6.0724007171956e-05	\\
-1011.08176491477	6.14870538848865e-05	\\
-1010.10298295455	6.45556576542726e-05	\\
-1009.12420099432	6.41200837023536e-05	\\
-1008.14541903409	6.57571202786521e-05	\\
-1007.16663707386	6.73678954226892e-05	\\
-1006.18785511364	6.51742193139843e-05	\\
-1005.20907315341	6.33032272064245e-05	\\
-1004.23029119318	6.98221755297969e-05	\\
-1003.25150923295	6.74748371159714e-05	\\
-1002.27272727273	6.62761593277583e-05	\\
-1001.2939453125	6.66750398017619e-05	\\
-1000.31516335227	7.21419230244386e-05	\\
-999.336381392046	5.88873817391748e-05	\\
-998.357599431818	6.74267998893388e-05	\\
-997.378817471592	6.68122088761847e-05	\\
-996.400035511364	6.54082489650428e-05	\\
-995.421253551136	6.90087575552175e-05	\\
-994.44247159091	7.15866359124481e-05	\\
-993.463689630682	7.09602529554125e-05	\\
-992.484907670454	7.04254202457983e-05	\\
-991.506125710228	6.94961192227316e-05	\\
-990.52734375	6.77358454237514e-05	\\
-989.548561789772	6.58100771461316e-05	\\
-988.569779829546	7.32654411759898e-05	\\
-987.590997869318	7.24608016395374e-05	\\
-986.612215909092	7.12718387951737e-05	\\
-985.633433948864	7.1261986326618e-05	\\
-984.654651988636	6.6690910230396e-05	\\
-983.67587002841	7.13593837789373e-05	\\
-982.697088068182	7.64256863598271e-05	\\
-981.718306107954	7.18258394549185e-05	\\
-980.739524147728	7.38616450511793e-05	\\
-979.7607421875	7.01253512385076e-05	\\
-978.781960227272	6.93497786543804e-05	\\
-977.803178267046	6.99767221308718e-05	\\
-976.824396306818	6.69962422649126e-05	\\
-975.845614346592	6.86096638695113e-05	\\
-974.866832386364	7.23302259874401e-05	\\
-973.888050426136	6.89637646425298e-05	\\
-972.90926846591	7.13946433650998e-05	\\
-971.930486505682	6.84587498587346e-05	\\
-970.951704545454	7.09703551794902e-05	\\
-969.972922585228	7.19198181348644e-05	\\
-968.994140625	6.60701615026195e-05	\\
-968.015358664772	7.25954544799269e-05	\\
-967.036576704546	6.71079839996313e-05	\\
-966.057794744318	6.57531808759982e-05	\\
-965.079012784092	7.20590631881186e-05	\\
-964.100230823864	6.99920911005958e-05	\\
-963.121448863636	6.88110445744758e-05	\\
-962.14266690341	6.40017986958014e-05	\\
-961.163884943182	7.38755961870801e-05	\\
-960.185102982954	6.47581763622779e-05	\\
-959.206321022728	7.25580265107267e-05	\\
-958.2275390625	7.06940909421247e-05	\\
-957.248757102272	6.96865627837872e-05	\\
-956.269975142046	6.90063340448371e-05	\\
-955.291193181818	6.88932246436078e-05	\\
-954.312411221592	6.68172411639442e-05	\\
-953.333629261364	6.91229145489078e-05	\\
-952.354847301136	7.0923210638446e-05	\\
-951.37606534091	7.13829090449519e-05	\\
-950.397283380682	6.60098868369011e-05	\\
-949.418501420454	7.36623104205919e-05	\\
-948.439719460228	6.75659163438335e-05	\\
-947.4609375	6.6806176691645e-05	\\
-946.482155539772	7.41664205361795e-05	\\
-945.503373579546	6.711605279668e-05	\\
-944.524591619318	6.78562805241081e-05	\\
-943.545809659092	6.98073391704652e-05	\\
-942.567027698864	6.44528592300826e-05	\\
-941.588245738636	7.07317775792983e-05	\\
-940.60946377841	6.59517083213646e-05	\\
-939.630681818182	7.13594807643563e-05	\\
-938.651899857954	6.57198265856507e-05	\\
-937.673117897728	6.22576758953016e-05	\\
-936.6943359375	6.84184959771896e-05	\\
-935.715553977272	7.01608075645626e-05	\\
-934.736772017046	6.94366560415252e-05	\\
-933.757990056818	6.74196687506869e-05	\\
-932.779208096592	6.72257216326028e-05	\\
-931.800426136364	6.52782858138942e-05	\\
-930.821644176136	6.36795174103814e-05	\\
-929.84286221591	6.84392232488573e-05	\\
-928.864080255682	6.31007600062112e-05	\\
-927.885298295454	6.43804350009059e-05	\\
-926.906516335228	6.31803267420059e-05	\\
-925.927734375	6.69182028701608e-05	\\
-924.948952414772	6.6730821701427e-05	\\
-923.970170454546	6.5687144984925e-05	\\
-922.991388494318	6.6711391720327e-05	\\
-922.012606534092	6.79165085524725e-05	\\
-921.033824573864	6.11334701273185e-05	\\
-920.055042613636	6.8867324894169e-05	\\
-919.07626065341	6.4043784128067e-05	\\
-918.097478693182	6.85245168818382e-05	\\
-917.118696732954	6.99755807772391e-05	\\
-916.139914772728	6.73004286183966e-05	\\
-915.1611328125	6.6818308942935e-05	\\
-914.182350852272	6.50593322164388e-05	\\
-913.203568892046	6.76486097134558e-05	\\
-912.224786931818	6.75603282474609e-05	\\
-911.246004971592	6.91742379110508e-05	\\
-910.267223011364	6.91631135549049e-05	\\
-909.288441051136	6.63044701029663e-05	\\
-908.30965909091	6.65337298441983e-05	\\
-907.330877130682	7.00755779040479e-05	\\
-906.352095170454	6.68380493717286e-05	\\
-905.373313210228	6.49310985948449e-05	\\
-904.39453125	6.95482441227108e-05	\\
-903.415749289772	7.00563969910263e-05	\\
-902.436967329546	6.91329671604834e-05	\\
-901.458185369318	6.82285038510304e-05	\\
-900.479403409092	7.4578508169726e-05	\\
-899.500621448864	5.79532553404562e-05	\\
-898.521839488636	6.91563549634001e-05	\\
-897.54305752841	6.75186192777917e-05	\\
-896.564275568182	6.86159942819554e-05	\\
-895.585493607954	7.46580923895372e-05	\\
-894.606711647728	7.11720503127498e-05	\\
-893.6279296875	7.15822248824777e-05	\\
-892.649147727272	7.26566355266061e-05	\\
-891.670365767046	7.05069709594241e-05	\\
-890.691583806818	7.25545652023329e-05	\\
-889.712801846592	7.30219403671658e-05	\\
-888.734019886364	7.2273929614153e-05	\\
-887.755237926136	6.74321224030886e-05	\\
-886.77645596591	6.91586713726547e-05	\\
-885.797674005682	6.77339121197433e-05	\\
-884.818892045454	7.20622322454628e-05	\\
-883.840110085228	7.24631677455416e-05	\\
-882.861328125	6.89420340873496e-05	\\
-881.882546164772	7.10024849785738e-05	\\
-880.903764204546	6.77897970499113e-05	\\
-879.924982244318	7.15279008416193e-05	\\
-878.946200284092	7.4463486338994e-05	\\
-877.967418323864	7.32247833914607e-05	\\
-876.988636363636	7.29325328538469e-05	\\
-876.00985440341	7.02364019290221e-05	\\
-875.031072443182	7.07317764794397e-05	\\
-874.052290482954	6.95424381832472e-05	\\
-873.073508522728	7.16709660537447e-05	\\
-872.0947265625	6.71625240104067e-05	\\
-871.115944602272	6.54186972146494e-05	\\
-870.137162642046	6.91044613468577e-05	\\
-869.158380681818	6.90738138326723e-05	\\
-868.179598721592	6.86089152411195e-05	\\
-867.200816761364	7.18598401762376e-05	\\
-866.222034801136	7.33991444146491e-05	\\
-865.24325284091	7.14750924077809e-05	\\
-864.264470880682	6.72118131699338e-05	\\
-863.285688920454	7.36706771120954e-05	\\
-862.306906960228	6.89964700075736e-05	\\
-861.328125	6.69111458197182e-05	\\
-860.349343039772	7.43378404555326e-05	\\
-859.370561079546	6.5427327651685e-05	\\
-858.391779119318	7.10747730779706e-05	\\
-857.412997159092	7.44140786149137e-05	\\
-856.434215198864	7.29716308869513e-05	\\
-855.455433238636	6.84531575937222e-05	\\
-854.47665127841	6.72439509879035e-05	\\
-853.497869318182	7.18860890749158e-05	\\
-852.519087357954	7.00463725102872e-05	\\
-851.540305397728	6.82253103443e-05	\\
-850.5615234375	7.22522894531805e-05	\\
-849.582741477272	6.95230589808919e-05	\\
-848.603959517046	6.77345119948678e-05	\\
-847.625177556818	6.97061763732674e-05	\\
-846.646395596592	6.58401177947199e-05	\\
-845.667613636364	7.14513945916733e-05	\\
-844.688831676136	6.88044322296852e-05	\\
-843.71004971591	6.9337729687768e-05	\\
-842.731267755682	6.83452952424399e-05	\\
-841.752485795454	6.22321049099159e-05	\\
-840.773703835228	7.10395117602711e-05	\\
-839.794921875	6.92322139879547e-05	\\
-838.816139914772	6.964198302314e-05	\\
-837.837357954546	6.64427151763759e-05	\\
-836.858575994318	6.31321264961741e-05	\\
-835.879794034092	6.6472203929381e-05	\\
-834.901012073864	6.89816431684692e-05	\\
-833.922230113636	6.89070457086136e-05	\\
-832.94344815341	7.31515584925845e-05	\\
-831.964666193182	6.92361836053778e-05	\\
-830.985884232954	6.82044123858346e-05	\\
-830.007102272728	7.20955051329959e-05	\\
-829.0283203125	7.14797623723437e-05	\\
-828.049538352272	6.74234621639639e-05	\\
-827.070756392046	6.90216557128072e-05	\\
-826.091974431818	6.76244680808312e-05	\\
-825.113192471592	6.74391678084233e-05	\\
-824.134410511364	6.44650892272847e-05	\\
-823.155628551136	7.11326291360231e-05	\\
-822.17684659091	6.67994608800322e-05	\\
-821.198064630682	7.02816296043209e-05	\\
-820.219282670454	7.04440758044989e-05	\\
-819.240500710228	6.63867860970283e-05	\\
-818.26171875	6.84448632132731e-05	\\
-817.282936789772	6.68200225976235e-05	\\
-816.304154829546	6.75553850070729e-05	\\
-815.325372869318	6.96076832810748e-05	\\
-814.346590909092	6.80669284786718e-05	\\
-813.367808948864	6.52150313915229e-05	\\
-812.389026988636	7.4217699015231e-05	\\
-811.41024502841	6.67195128083376e-05	\\
-810.431463068182	6.79455316496464e-05	\\
-809.452681107954	6.67812788989686e-05	\\
-808.473899147728	6.91402617104139e-05	\\
-807.4951171875	6.5149231587245e-05	\\
-806.516335227272	7.16493152242473e-05	\\
-805.537553267046	6.71525709984898e-05	\\
-804.558771306818	7.02246474401404e-05	\\
-803.579989346592	7.01505254032497e-05	\\
-802.601207386364	7.04495472289753e-05	\\
-801.622425426136	6.98825382165938e-05	\\
-800.64364346591	6.87613625754551e-05	\\
-799.664861505682	7.45688473942645e-05	\\
-798.686079545454	7.46394038389594e-05	\\
-797.707297585228	7.181231710891e-05	\\
-796.728515625	6.91386863614016e-05	\\
-795.749733664772	7.31588240645963e-05	\\
-794.770951704546	6.60872714982787e-05	\\
-793.792169744318	7.43747736503062e-05	\\
-792.813387784092	7.20032896739253e-05	\\
-791.834605823864	7.20270690644201e-05	\\
-790.855823863636	7.28007796877155e-05	\\
-789.87704190341	7.59447643630975e-05	\\
-788.898259943182	6.92912572849787e-05	\\
-787.919477982954	7.32264960923762e-05	\\
-786.940696022728	7.16224072526989e-05	\\
-785.9619140625	7.25328083379587e-05	\\
-784.983132102272	7.48062981885856e-05	\\
-784.004350142046	7.76269476269566e-05	\\
-783.025568181818	7.64429985774076e-05	\\
-782.046786221592	7.178894296151e-05	\\
-781.068004261364	7.71435601911231e-05	\\
-780.089222301136	7.1302303007494e-05	\\
-779.11044034091	7.41961453378057e-05	\\
-778.131658380682	7.55045072621887e-05	\\
-777.152876420454	7.89205633617282e-05	\\
-776.174094460228	7.47452288551767e-05	\\
-775.1953125	7.30966903004247e-05	\\
-774.216530539772	7.41859362649011e-05	\\
-773.237748579546	7.3745290580289e-05	\\
-772.258966619318	7.55549129833951e-05	\\
-771.280184659092	7.50842253829393e-05	\\
-770.301402698864	7.55749223302749e-05	\\
-769.322620738636	8.0061501353906e-05	\\
-768.34383877841	7.79482895023478e-05	\\
-767.365056818182	6.89831592053686e-05	\\
-766.386274857954	7.82485419364901e-05	\\
-765.407492897728	8.13377423363933e-05	\\
-764.4287109375	7.82472872371401e-05	\\
-763.449928977272	8.4630630515131e-05	\\
-762.471147017046	7.46841327412512e-05	\\
-761.492365056818	7.67313929318926e-05	\\
-760.513583096592	8.13055831393365e-05	\\
-759.534801136364	8.202813718687e-05	\\
-758.556019176136	7.58865695349508e-05	\\
-757.57723721591	7.98644060802115e-05	\\
-756.598455255682	8.30734420774827e-05	\\
-755.619673295454	7.9197780597265e-05	\\
-754.640891335228	8.0952344554762e-05	\\
-753.662109375	7.82236852589815e-05	\\
-752.683327414772	8.02293396231848e-05	\\
-751.704545454546	8.43385158386524e-05	\\
-750.725763494318	8.53392054614454e-05	\\
-749.746981534092	9.29976860607511e-05	\\
-748.768199573864	7.8156107324714e-05	\\
-747.789417613636	8.09198719933416e-05	\\
-746.81063565341	7.94356255519973e-05	\\
-745.831853693182	8.36097556127273e-05	\\
-744.853071732954	7.84300848196734e-05	\\
-743.874289772728	8.28383602633487e-05	\\
-742.8955078125	8.34212181654176e-05	\\
-741.916725852272	8.07891638955676e-05	\\
-740.937943892046	7.85914866248232e-05	\\
-739.959161931818	8.53725749290429e-05	\\
-738.980379971592	8.23926101601765e-05	\\
-738.001598011364	8.02240530490364e-05	\\
-737.022816051136	8.43489302513415e-05	\\
-736.04403409091	8.54651084632638e-05	\\
-735.065252130682	8.43392482722378e-05	\\
-734.086470170454	8.36549324605019e-05	\\
-733.107688210228	8.39402553209738e-05	\\
-732.12890625	8.53889157722478e-05	\\
-731.150124289772	8.3492543593908e-05	\\
-730.171342329546	9.0544118672495e-05	\\
-729.192560369318	8.69841712965221e-05	\\
-728.213778409092	8.59364190018999e-05	\\
-727.234996448864	8.42646777567959e-05	\\
-726.256214488636	8.54215598982911e-05	\\
-725.27743252841	9.05067997537019e-05	\\
-724.298650568182	8.82727541468191e-05	\\
-723.319868607954	8.27279020703865e-05	\\
-722.341086647728	8.75744062328281e-05	\\
-721.3623046875	8.53899228582803e-05	\\
-720.383522727272	9.44443544868839e-05	\\
-719.404740767046	8.84379334838966e-05	\\
-718.425958806818	8.75974555125849e-05	\\
-717.447176846592	8.56896409526054e-05	\\
-716.468394886364	8.86072436048944e-05	\\
-715.489612926136	8.44961447741453e-05	\\
-714.51083096591	8.98083963195808e-05	\\
-713.532049005682	8.97138664002139e-05	\\
-712.553267045454	8.79115370084078e-05	\\
-711.574485085228	7.57875103273732e-05	\\
-710.595703125	8.79325239744035e-05	\\
-709.616921164772	8.73381229391861e-05	\\
-708.638139204546	8.41308913472717e-05	\\
-707.659357244318	8.57518568846441e-05	\\
-706.680575284092	7.80205360361526e-05	\\
-705.701793323864	0.000103518248586232	\\
-704.723011363636	7.39996633338393e-05	\\
-703.74422940341	9.41326173916634e-05	\\
-702.765447443182	8.05899309830736e-05	\\
-701.786665482954	6.93430017590129e-05	\\
-700.807883522728	9.66305775622937e-05	\\
-699.8291015625	7.24287193921243e-05	\\
-698.850319602272	8.44073604696388e-05	\\
-697.871537642046	7.93580037395011e-05	\\
-696.892755681818	7.85782912592637e-05	\\
-695.913973721592	8.06194256196218e-05	\\
-694.935191761364	8.17182775440001e-05	\\
-693.956409801136	7.87722995854135e-05	\\
-692.97762784091	8.33108338675101e-05	\\
-691.998845880682	8.01003202525691e-05	\\
-691.020063920454	8.26711753647711e-05	\\
-690.041281960228	8.43636297144165e-05	\\
-689.0625	8.19097427253757e-05	\\
-688.083718039772	7.99403413849626e-05	\\
-687.104936079546	8.38939532846656e-05	\\
-686.126154119318	7.9674920121665e-05	\\
-685.147372159092	8.40439143149305e-05	\\
-684.168590198864	7.22930232364219e-05	\\
-683.189808238636	8.56139177398313e-05	\\
-682.21102627841	7.75211677766758e-05	\\
-681.232244318182	7.57697336953435e-05	\\
-680.253462357954	7.87386114641044e-05	\\
-679.274680397728	8.24100395194657e-05	\\
-678.2958984375	7.2954194394675e-05	\\
-677.317116477272	8.47361206097341e-05	\\
-676.338334517046	8.22998318074881e-05	\\
-675.359552556818	7.4574475398716e-05	\\
-674.380770596592	6.75107284220908e-05	\\
-673.401988636364	8.2829801817198e-05	\\
-672.423206676136	7.59340172380242e-05	\\
-671.44442471591	7.68031215648481e-05	\\
-670.465642755682	7.31283069781311e-05	\\
-669.486860795454	7.22492614603769e-05	\\
-668.508078835228	8.56194835796022e-05	\\
-667.529296875	8.45336390336137e-05	\\
-666.550514914772	7.955813805029e-05	\\
-665.571732954546	7.72402190491263e-05	\\
-664.592950994318	7.30940741825587e-05	\\
-663.614169034092	7.58500784976843e-05	\\
-662.635387073864	8.08121633119219e-05	\\
-661.656605113636	8.55710615459481e-05	\\
-660.67782315341	8.7270232619913e-05	\\
-659.699041193182	8.01301899623775e-05	\\
-658.720259232954	7.8494198162968e-05	\\
-657.741477272728	7.87566483541948e-05	\\
-656.7626953125	7.61206325130036e-05	\\
-655.783913352272	9.21574460244006e-05	\\
-654.805131392046	8.2720755215876e-05	\\
-653.826349431818	7.76174405816724e-05	\\
-652.847567471592	7.9130176652034e-05	\\
-651.868785511364	7.64359556852028e-05	\\
-650.890003551136	8.6395637586398e-05	\\
-649.91122159091	8.93815986952502e-05	\\
-648.932439630682	8.55936027830291e-05	\\
-647.953657670454	8.3544561211207e-05	\\
-646.974875710228	8.2463288551642e-05	\\
-645.99609375	9.0905950203028e-05	\\
-645.017311789772	8.600163052129e-05	\\
-644.038529829546	9.0827145113805e-05	\\
-643.059747869318	8.80506227721379e-05	\\
-642.080965909092	8.91848320961131e-05	\\
-641.102183948864	8.45325786667012e-05	\\
-640.123401988636	9.11130135507994e-05	\\
-639.14462002841	8.59036279191012e-05	\\
-638.165838068182	8.46285031567988e-05	\\
-637.187056107954	9.04175834114731e-05	\\
-636.208274147728	8.87492323617994e-05	\\
-635.2294921875	9.81216060456703e-05	\\
-634.250710227272	9.10194107512472e-05	\\
-633.271928267046	8.57859503723745e-05	\\
-632.293146306818	9.78820504570538e-05	\\
-631.314364346592	9.32977558787182e-05	\\
-630.335582386364	8.35431797024713e-05	\\
-629.356800426136	9.52573995732718e-05	\\
-628.37801846591	9.09352483651814e-05	\\
-627.399236505682	8.51347176659663e-05	\\
-626.420454545454	9.48410160086349e-05	\\
-625.441672585228	9.45165167243903e-05	\\
-624.462890625	9.24838004971666e-05	\\
-623.484108664772	9.27349048994567e-05	\\
-622.505326704546	9.59343636849665e-05	\\
-621.526544744318	9.02141368337834e-05	\\
-620.547762784092	8.87177913382179e-05	\\
-619.568980823864	9.70602575416961e-05	\\
-618.590198863636	9.40751011338512e-05	\\
-617.61141690341	9.31030002663672e-05	\\
-616.632634943182	0.000108134444613207	\\
-615.653852982954	9.22876872866311e-05	\\
-614.675071022728	8.9976469588324e-05	\\
-613.6962890625	9.3806105123896e-05	\\
-612.717507102272	9.57641624795003e-05	\\
-611.738725142046	0.000108063491842865	\\
-610.759943181818	9.99449970404397e-05	\\
-609.781161221592	0.000100164580484206	\\
-608.802379261364	9.85292768604479e-05	\\
-607.823597301136	9.52847470130709e-05	\\
-606.84481534091	9.77505554755719e-05	\\
-605.866033380682	0.000104089054414168	\\
-604.887251420454	0.000100312985047374	\\
-603.908469460228	0.000109833036635026	\\
-602.9296875	9.83653099663625e-05	\\
-601.950905539772	0.000113368076481642	\\
-600.972123579546	0.00011274713185725	\\
-599.993341619318	0.00012272571381101	\\
-599.014559659092	9.91880099685119e-05	\\
-598.035777698864	0.000112564544215232	\\
-597.056995738636	0.000103725041651333	\\
-596.07821377841	0.000116905545380655	\\
-595.099431818182	9.58527054311977e-05	\\
-594.120649857954	0.00011974303585481	\\
-593.141867897728	0.000102098183201944	\\
-592.1630859375	0.000112779410352825	\\
-591.184303977272	0.000137260824535062	\\
-590.205522017046	8.04233334725673e-05	\\
-589.226740056818	0.000137704543618788	\\
-588.247958096592	0.000119039987850897	\\
-587.269176136364	0.000112954368207596	\\
-586.290394176136	0.000123782512228607	\\
-585.31161221591	0.000117423555071656	\\
-584.332830255682	0.000113782528144575	\\
-583.354048295454	0.000160204894471108	\\
-582.375266335228	9.45508409445204e-05	\\
-581.396484375	0.000136140128850708	\\
-580.417702414772	0.000113432219817121	\\
-579.438920454546	0.00010781732292413	\\
-578.460138494318	0.000156485537749904	\\
-577.481356534092	0.000104534050968721	\\
-576.502574573864	0.000121012659362057	\\
-575.523792613636	0.000145535373849433	\\
-574.54501065341	0.000100988034500811	\\
-573.566228693182	0.000140287981673303	\\
-572.587446732954	0.000117052686976322	\\
-571.608664772728	0.000117634905339596	\\
-570.6298828125	0.00014847480663903	\\
-569.651100852272	0.000111912012429212	\\
-568.672318892046	0.000134808066805049	\\
-567.693536931818	0.000135873469297831	\\
-566.714754971592	0.000125313542573546	\\
-565.735973011364	0.000127397690101963	\\
-564.757191051136	0.00013422501268514	\\
-563.77840909091	0.00012603985009876	\\
-562.799627130682	0.000136573760113981	\\
-561.820845170454	0.000128944734877604	\\
-560.842063210228	0.000132486600724749	\\
-559.86328125	0.000141126189931798	\\
-558.884499289772	0.000128284496820961	\\
-557.905717329546	0.000133473262699828	\\
-556.926935369318	0.000130223719743985	\\
-555.948153409092	0.000131188390773252	\\
-554.969371448864	0.000135009222609048	\\
-553.990589488636	0.000140143871068307	\\
-553.01180752841	0.000131079777771768	\\
-552.033025568182	0.000123705364661309	\\
-551.054243607954	0.000135388479273837	\\
-550.075461647728	0.000144385723543844	\\
-549.0966796875	0.000136422474910427	\\
-548.117897727272	0.000130657059022081	\\
-547.139115767046	0.000131781483113614	\\
-546.160333806818	0.000127953431911614	\\
-545.181551846592	0.000130261293423872	\\
-544.202769886364	0.000126687719224958	\\
-543.223987926136	0.000123053693789974	\\
-542.24520596591	0.000136228456834419	\\
-541.266424005682	0.00013345269033881	\\
-540.287642045454	0.000119519568001014	\\
-539.308860085228	0.000122692976336838	\\
-538.330078125	0.000127535355463745	\\
-537.351296164772	0.000116645119994886	\\
-536.372514204546	0.000131543788455357	\\
-535.393732244318	0.000123696950747835	\\
-534.414950284092	0.000125659154055618	\\
-533.436168323864	0.000117726207200593	\\
-532.457386363636	0.000120272511178417	\\
-531.47860440341	0.000116579332758361	\\
-530.499822443182	0.000128608786659857	\\
-529.521040482954	0.000133338652972032	\\
-528.542258522728	0.00011747789952342	\\
-527.5634765625	0.000125119959674541	\\
-526.584694602272	0.000128261054453276	\\
-525.605912642046	0.000119883981852639	\\
-524.627130681818	0.000119439098227114	\\
-523.648348721592	0.000134431762128132	\\
-522.669566761364	0.000125611233660318	\\
-521.690784801136	0.000113111414700464	\\
-520.71200284091	0.000133757240700737	\\
-519.733220880682	0.000142626913608605	\\
-518.754438920454	0.000135142512351941	\\
-517.775656960228	0.000136102309816294	\\
-516.796875	0.000135147532371595	\\
-515.818093039772	0.000128633282439465	\\
-514.839311079546	0.000150862709010176	\\
-513.860529119318	0.000140037624217436	\\
-512.881747159092	0.000113947993259274	\\
-511.902965198864	0.000141254440779678	\\
-510.924183238636	0.000142622660382329	\\
-509.94540127841	0.000156085001692621	\\
-508.966619318182	0.000142728344765436	\\
-507.987837357954	0.000133807902236733	\\
-507.009055397728	0.00014663982638384	\\
-506.0302734375	0.000135873485910912	\\
-505.051491477272	0.000136668542910361	\\
-504.072709517046	0.000149039717369176	\\
-503.093927556818	0.00013639658006407	\\
-502.115145596592	0.000135636993745426	\\
-501.136363636364	0.000166684017048953	\\
-500.157581676136	0.000279591638042948	\\
-499.17879971591	0.000250678740059178	\\
-498.200017755682	0.000119945194745938	\\
-497.221235795454	9.23630665959545e-05	\\
-496.242453835228	0.000207921658848611	\\
-495.263671875	0.000157667372321216	\\
-494.284889914772	0.000191079476611039	\\
-493.306107954546	0.000129105568808526	\\
-492.327325994318	0.00012042621442941	\\
-491.348544034092	0.000247508345119192	\\
-490.369762073864	0.000144328467432106	\\
-489.390980113636	0.000177222097129791	\\
-488.41219815341	0.00017253615219624	\\
-487.433416193182	0.000157039472187588	\\
-486.454634232954	0.000243131477033972	\\
-485.475852272728	0.000125410117837352	\\
-484.4970703125	0.000146038401663673	\\
-483.518288352272	0.000233284526486954	\\
-482.539506392046	0.000134026609345064	\\
-481.560724431818	0.000182564352947147	\\
-480.581942471592	0.000196668997155521	\\
-479.603160511364	0.000145893542123153	\\
-478.624378551136	0.000221146070622862	\\
-477.64559659091	0.00013803873468763	\\
-476.666814630682	0.000158610536263311	\\
-475.688032670454	0.000166592351340871	\\
-474.709250710228	0.000173182593497764	\\
-473.73046875	0.000177670468196578	\\
-472.751686789772	0.000164622640616435	\\
-471.772904829546	0.000165320431644236	\\
-470.794122869318	0.000178083856105792	\\
-469.815340909092	0.000124127527742713	\\
-468.836558948864	0.000240440292382091	\\
-467.857776988636	0.000166984881608197	\\
-466.87899502841	0.00019670717068306	\\
-465.900213068182	0.00017768483076876	\\
-464.921431107954	0.000123990791518289	\\
-463.942649147728	0.000167416672837569	\\
-462.9638671875	0.000158499968474471	\\
-461.985085227272	0.000132746300966396	\\
-461.006303267046	0.000161184018450859	\\
-460.027521306818	0.00017339107992144	\\
-459.048739346592	0.000108256663434806	\\
-458.069957386364	0.000169053924477682	\\
-457.091175426136	0.000134905967666011	\\
-456.11239346591	0.000114274416625444	\\
-455.133611505682	0.000164584715791171	\\
-454.154829545454	0.000138406410548606	\\
-453.176047585228	0.000128094103420768	\\
-452.197265625	0.000146599682850673	\\
-451.218483664772	0.000132537564149841	\\
-450.239701704546	0.000100889746759849	\\
-449.260919744318	0.000163724778929368	\\
-448.282137784092	9.09556677000842e-05	\\
-447.303355823864	0.000160056662079332	\\
-446.324573863636	0.000124927068161261	\\
-445.34579190341	0.000112528910515303	\\
-444.367009943182	6.48029709763576e-05	\\
-443.388227982954	0.000151973465302088	\\
-442.409446022728	9.21415128614452e-05	\\
-441.4306640625	0.000142667101058632	\\
-440.451882102272	0.000130467703617277	\\
-439.473100142046	6.86944721358565e-05	\\
-438.494318181818	8.60104991171097e-05	\\
-437.515536221592	0.000115497563280457	\\
-436.536754261364	0.000105916824425247	\\
-435.557972301136	0.00011074739884441	\\
-434.57919034091	0.000109460500490995	\\
-433.600408380682	8.51846307588394e-05	\\
-432.621626420454	0.000100123720205838	\\
-431.642844460228	0.00010024042577775	\\
-430.6640625	8.41088555993166e-05	\\
-429.685280539772	0.000110136297313755	\\
-428.706498579546	0.000119815085485441	\\
-427.727716619318	9.16636039821066e-05	\\
-426.748934659092	9.45501583788745e-05	\\
-425.770152698864	9.28129454561543e-05	\\
-424.791370738636	8.00422246376203e-05	\\
-423.81258877841	9.03459694255144e-05	\\
-422.833806818182	0.000103874475302965	\\
-421.855024857954	9.08199061839862e-05	\\
-420.876242897728	8.18622582240086e-05	\\
-419.8974609375	7.48586213272427e-05	\\
-418.918678977272	8.88306247722389e-05	\\
-417.939897017046	0.000110848041752465	\\
-416.961115056818	7.85747060552218e-05	\\
-415.982333096592	0.000102553464129887	\\
-415.003551136364	8.96616184493377e-05	\\
-414.024769176136	9.4093404556972e-05	\\
-413.04598721591	5.54392300620696e-05	\\
-412.067205255682	9.96739596954701e-05	\\
-411.088423295454	0.000110034611058596	\\
-410.109641335228	4.98272544439734e-05	\\
-409.130859375	9.05613550427839e-05	\\
-408.152077414772	9.97965573876453e-05	\\
-407.173295454546	9.06533938399967e-05	\\
-406.194513494318	5.55893983165864e-05	\\
-405.215731534092	0.000113683531883658	\\
-404.236949573864	7.80246900798793e-05	\\
-403.258167613636	5.96524789662297e-05	\\
-402.27938565341	7.41878920003093e-05	\\
-401.300603693182	7.52226057041926e-05	\\
-400.321821732954	3.48370894038354e-05	\\
-399.343039772728	8.06510162888739e-05	\\
-398.3642578125	8.13149415168837e-05	\\
-397.385475852272	5.38992531185612e-05	\\
-396.406693892046	5.8502679572366e-05	\\
-395.427911931818	7.65066930805513e-05	\\
-394.449129971592	5.84497318897452e-05	\\
-393.470348011364	7.0463001244633e-05	\\
-392.491566051136	7.2597040965676e-05	\\
-391.51278409091	5.61478862987714e-05	\\
-390.534002130682	6.24281091759406e-05	\\
-389.555220170454	8.65150158146357e-05	\\
-388.576438210228	7.05914665474337e-05	\\
-387.59765625	3.15780051576232e-05	\\
-386.618874289772	7.6566883600889e-05	\\
-385.640092329546	8.39119339634297e-05	\\
-384.661310369318	4.3579206166213e-05	\\
-383.682528409092	8.91834613800139e-05	\\
-382.703746448864	5.52189348173172e-05	\\
-381.724964488636	2.93899810596454e-05	\\
-380.74618252841	8.00906720601307e-05	\\
-379.767400568182	4.42458300481866e-05	\\
-378.788618607954	5.69424395600803e-05	\\
-377.809836647728	3.71140633570963e-05	\\
-376.8310546875	6.60733292899106e-05	\\
-375.852272727272	3.68216990637627e-05	\\
-374.873490767046	5.54411385281674e-05	\\
-373.894708806818	8.17653996815524e-05	\\
-372.915926846592	3.17125081182983e-05	\\
-371.937144886364	8.57670602119705e-05	\\
-370.958362926136	3.49474389695559e-05	\\
-369.97958096591	5.05293728261086e-05	\\
-369.000799005682	4.92242711341177e-05	\\
-368.022017045454	5.19214025349082e-05	\\
-367.043235085228	7.60326946156582e-05	\\
-366.064453125	6.87933608334299e-05	\\
-365.085671164772	6.76421612176971e-05	\\
-364.106889204546	9.05133586923665e-05	\\
-363.128107244318	5.34921003043403e-05	\\
-362.149325284092	4.20248240781679e-05	\\
-361.170543323864	4.62314059848331e-05	\\
-360.191761363636	3.12018706588094e-05	\\
-359.21297940341	8.46072476340263e-05	\\
-358.234197443182	3.64432813059313e-05	\\
-357.255415482954	5.3349508478607e-05	\\
-356.276633522728	6.26623590840445e-05	\\
-355.2978515625	2.62787685969902e-05	\\
-354.319069602272	5.96934498991286e-05	\\
-353.340287642046	2.02746121612187e-05	\\
-352.361505681818	5.62009367665947e-05	\\
-351.382723721592	3.21028990996913e-05	\\
-350.403941761364	9.47331764475587e-05	\\
-349.425159801136	3.18225283988244e-06	\\
-348.44637784091	9.35772362038391e-05	\\
-347.467595880682	6.25860611747805e-05	\\
-346.488813920454	6.24349764673535e-05	\\
-345.510031960228	1.05828447723434e-05	\\
-344.53125	5.889925501635e-05	\\
-343.552468039772	5.44462563921365e-06	\\
-342.573686079546	0.000105136145796466	\\
-341.594904119318	4.8804338571435e-05	\\
-340.616122159092	3.59317222866623e-05	\\
-339.637340198864	4.48886775102556e-05	\\
-338.658558238636	3.42473392101284e-05	\\
-337.67977627841	4.05630194416877e-05	\\
-336.700994318182	4.05836520859356e-05	\\
-335.722212357954	1.20980483534506e-05	\\
-334.743430397728	1.40988845110731e-05	\\
-333.7646484375	4.04124219762785e-05	\\
-332.785866477272	1.6199126075172e-05	\\
-331.807084517046	3.03620788500954e-05	\\
-330.828302556818	3.87765803794119e-05	\\
-329.849520596592	2.7439665127071e-05	\\
-328.870738636364	3.34638588134511e-05	\\
-327.891956676136	2.28413633726523e-05	\\
-326.91317471591	2.86871248002565e-05	\\
-325.934392755682	2.92332680434826e-05	\\
-324.955610795454	4.61038399328146e-05	\\
-323.976828835228	2.36413311184072e-05	\\
-322.998046875	2.65242871642312e-05	\\
-322.019264914772	2.48797539632547e-05	\\
-321.040482954546	3.21900537526874e-05	\\
-320.061700994318	3.66201655626404e-05	\\
-319.082919034092	3.0611707245655e-05	\\
-318.104137073864	4.05897382203705e-05	\\
-317.125355113636	1.40870345549061e-05	\\
-316.14657315341	2.13844668451189e-05	\\
-315.167791193182	3.68938718725798e-05	\\
-314.189009232954	3.08614758243222e-05	\\
-313.210227272728	2.96205760632817e-05	\\
-312.2314453125	2.21300966165394e-05	\\
-311.252663352272	1.16457647720438e-05	\\
-310.273881392046	5.04573407643436e-05	\\
-309.295099431818	4.39650353866587e-05	\\
-308.316317471592	2.05573343063551e-05	\\
-307.337535511364	1.98966775354921e-05	\\
-306.358753551136	2.48263647154745e-05	\\
-305.37997159091	4.73109833948079e-05	\\
-304.401189630682	3.04567034233231e-05	\\
-303.422407670454	2.28494439338124e-05	\\
-302.443625710228	2.18066463133576e-05	\\
-301.46484375	3.72150022367026e-05	\\
-300.486061789772	1.88562756313924e-05	\\
-299.507279829546	2.00070689927915e-05	\\
-298.528497869318	1.07290580171436e-05	\\
-297.549715909092	2.82279141697488e-05	\\
-296.570933948864	1.88068230714376e-05	\\
-295.592151988636	2.50837276914217e-05	\\
-294.61337002841	2.8045587147829e-05	\\
-293.634588068182	3.35381701586884e-05	\\
-292.655806107954	3.71861765652586e-05	\\
-291.677024147728	2.81866264268961e-05	\\
-290.6982421875	6.63374912540006e-06	\\
-289.719460227272	3.11019748101432e-05	\\
-288.740678267046	2.50239457141121e-05	\\
-287.761896306818	3.13949921549237e-05	\\
-286.783114346592	1.71809081355211e-05	\\
-285.804332386364	2.83414897977376e-05	\\
-284.825550426136	3.77126058239519e-05	\\
-283.84676846591	3.52691054074574e-05	\\
-282.867986505682	1.73075982965608e-05	\\
-281.889204545454	1.66957496756971e-05	\\
-280.910422585228	2.48487325914035e-05	\\
-279.931640625	2.45817448780596e-05	\\
-278.952858664772	4.10423534990802e-05	\\
-277.974076704546	2.73158505907969e-05	\\
-276.995294744318	3.19582748219934e-05	\\
-276.016512784092	5.59524473024743e-05	\\
-275.037730823864	2.6335182856239e-05	\\
-274.058948863636	3.74022749984316e-05	\\
-273.08016690341	3.76973396506485e-05	\\
-272.101384943182	2.74512897130715e-05	\\
-271.122602982954	3.04609420439263e-05	\\
-270.143821022728	3.2703221048734e-05	\\
-269.1650390625	2.12839310229376e-05	\\
-268.186257102272	4.0261991001619e-05	\\
-267.207475142046	2.19627749909205e-05	\\
-266.228693181818	1.29147962759044e-05	\\
-265.249911221592	1.88558113124193e-05	\\
-264.271129261364	3.06718244116973e-05	\\
-263.292347301136	2.31757876572499e-05	\\
-262.31356534091	2.11668438771198e-05	\\
-261.334783380682	3.30107050067318e-05	\\
-260.356001420454	2.83082583873209e-05	\\
-259.377219460228	5.32189531685268e-06	\\
-258.3984375	3.13558016936223e-05	\\
-257.419655539772	3.72616622123681e-05	\\
-256.440873579546	3.15087694177775e-05	\\
-255.462091619318	2.8764201022962e-05	\\
-254.483309659092	1.92260308199856e-05	\\
-253.504527698864	5.29263198693489e-05	\\
-252.525745738636	4.09854960190503e-05	\\
-251.54696377841	2.16650474512315e-05	\\
-250.568181818182	2.71840572781452e-05	\\
-249.589399857954	3.86722049509518e-05	\\
-248.610617897728	3.82726549725903e-05	\\
-247.6318359375	1.29779528829385e-05	\\
-246.653053977272	3.22298074190556e-05	\\
-245.674272017046	2.53652976163585e-05	\\
-244.695490056818	3.54716641122281e-05	\\
-243.716708096592	1.40302898318878e-05	\\
-242.737926136364	3.45549062475201e-05	\\
-241.759144176136	3.34255933770816e-05	\\
-240.78036221591	6.48679995224296e-06	\\
-239.801580255682	3.35777706427941e-05	\\
-238.822798295454	7.02854219945882e-06	\\
-237.844016335228	6.89658645048605e-05	\\
-236.865234375	4.92863967536099e-05	\\
-235.886452414772	3.67442704616511e-05	\\
-234.907670454546	3.32614624778023e-06	\\
-233.928888494318	6.32453103174244e-05	\\
-232.950106534092	3.04129553797197e-05	\\
-231.971324573864	5.52684112625694e-05	\\
-230.992542613636	3.40368851969701e-05	\\
-230.01376065341	2.17275951331065e-05	\\
-229.034978693182	4.74215730242393e-06	\\
-228.056196732954	2.86249412726244e-05	\\
-227.077414772728	3.16107075810678e-05	\\
-226.0986328125	4.41425995056563e-05	\\
-225.119850852272	3.53727437563164e-05	\\
-224.141068892046	3.89300570260831e-05	\\
-223.162286931818	3.17629160717594e-05	\\
-222.183504971592	4.3328102764382e-05	\\
-221.204723011364	2.27746126402448e-05	\\
-220.225941051136	1.05967493878312e-05	\\
-219.24715909091	4.44748956136826e-05	\\
-218.268377130682	2.35899447978074e-05	\\
-217.289595170454	1.64973090312554e-05	\\
-216.310813210228	9.33475566475043e-06	\\
-215.33203125	2.31692992556046e-05	\\
-214.353249289772	4.34968319339674e-05	\\
-213.374467329546	4.32472776814306e-05	\\
-212.395685369318	2.60524097052198e-05	\\
-211.416903409092	5.42481025894837e-06	\\
-210.438121448864	3.07193690858742e-05	\\
-209.459339488636	3.75792105350347e-05	\\
-208.48055752841	2.50999982038408e-05	\\
-207.501775568182	9.68404141680292e-06	\\
-206.522993607954	1.65185490365717e-05	\\
-205.544211647728	3.62800319017328e-05	\\
-204.5654296875	5.09537566983254e-05	\\
-203.586647727272	6.25592868526863e-05	\\
-202.607865767046	1.21882043936745e-05	\\
-201.629083806818	9.49632066015683e-06	\\
-200.650301846592	6.23139698069534e-05	\\
-199.671519886364	9.37034580671494e-05	\\
-198.692737926136	1.62564810410153e-05	\\
-197.71395596591	2.26033071866479e-05	\\
-196.735174005682	2.53687729036661e-05	\\
-195.756392045454	2.16766682432244e-05	\\
-194.777610085228	3.33460041065527e-05	\\
-193.798828125	2.22264309159684e-05	\\
-192.820046164772	3.35089475090767e-05	\\
-191.841264204546	4.48369113826216e-05	\\
-190.862482244318	3.2233872205041e-05	\\
-189.883700284092	5.79732845950958e-05	\\
-188.904918323864	1.66769600863607e-05	\\
-187.926136363636	3.70321705110765e-05	\\
-186.94735440341	2.74345585992838e-05	\\
-185.968572443182	7.16919790287202e-06	\\
-184.989790482954	1.98008537729328e-05	\\
-184.011008522728	2.29213900628714e-05	\\
-183.0322265625	6.43572752379291e-06	\\
-182.053444602272	1.72350687639912e-06	\\
-181.074662642046	1.1980414107854e-05	\\
-180.095880681818	1.55398825654265e-05	\\
-179.117098721592	1.54773238968461e-05	\\
-178.138316761364	2.10657043949735e-06	\\
-177.159534801136	7.98286108276938e-06	\\
-176.18075284091	1.55551987407367e-05	\\
-175.201970880682	1.07143096910187e-05	\\
-174.223188920454	1.31919303142413e-05	\\
-173.244406960228	1.47095364027329e-05	\\
-172.265625	1.07294940931124e-05	\\
-171.286843039772	1.67269236992683e-05	\\
-170.308061079546	1.19602760688581e-05	\\
-169.329279119318	1.7005033203489e-05	\\
-168.350497159092	1.78197208293526e-05	\\
-167.371715198864	1.86006839252318e-05	\\
-166.392933238636	5.24191478222908e-06	\\
-165.41415127841	5.39074056764856e-06	\\
-164.435369318182	2.20609828180829e-05	\\
-163.456587357954	1.96687970011481e-05	\\
-162.477805397728	2.17935250776098e-05	\\
-161.4990234375	1.89922575349462e-05	\\
-160.520241477272	2.02035891609963e-05	\\
-159.541459517046	2.23099170693872e-05	\\
-158.562677556818	1.70576873384178e-05	\\
-157.583895596592	3.06268279483736e-05	\\
-156.605113636364	1.94938008440974e-05	\\
-155.626331676136	1.34491791748007e-05	\\
-154.64754971591	1.82527865668471e-05	\\
-153.668767755682	1.73799341198647e-05	\\
-152.689985795454	6.98458740695419e-06	\\
-151.711203835228	2.19285175027191e-05	\\
-150.732421875	1.4245591354e-05	\\
-149.753639914772	1.25231732493199e-05	\\
-148.774857954546	4.90766779934486e-06	\\
-147.796075994318	1.06412060113916e-05	\\
-146.817294034092	1.36118404695407e-05	\\
-145.838512073864	4.35693512428842e-06	\\
-144.859730113636	2.34473571710534e-05	\\
-143.88094815341	2.59678434165183e-05	\\
-142.902166193182	1.27115826610508e-05	\\
-141.923384232954	1.22598542140073e-05	\\
-140.944602272728	3.03240621977909e-05	\\
-139.9658203125	2.49067419801383e-05	\\
-138.987038352272	8.97978665900717e-06	\\
-138.008256392046	2.47374072087402e-05	\\
-137.029474431818	1.35767480575585e-05	\\
-136.050692471592	1.46999896775336e-05	\\
-135.071910511364	2.09507722797264e-05	\\
-134.093128551136	1.46234391687098e-05	\\
-133.11434659091	1.77800518834677e-05	\\
-132.135564630682	2.16228119829219e-05	\\
-131.156782670454	9.80857185327013e-06	\\
-130.178000710228	3.76129163982123e-06	\\
-129.19921875	1.03338194916531e-05	\\
-128.220436789772	4.68871798163384e-05	\\
-127.241654829546	1.07935891917438e-05	\\
-126.262872869318	7.02147700558056e-05	\\
-125.284090909092	3.36542671072104e-05	\\
-124.305308948864	3.26154736917817e-05	\\
-123.326526988636	4.47863022303524e-05	\\
-122.34774502841	7.31987232919203e-05	\\
-121.368963068182	9.73878227580389e-05	\\
-120.390181107954	0.000155010922028512	\\
-119.411399147728	0.000171224886951917	\\
-118.4326171875	1.54403642063348e-05	\\
-117.453835227272	0.000106469268905831	\\
-116.475053267046	7.28039667839832e-05	\\
-115.496271306818	7.59215939438787e-05	\\
-114.517489346592	4.02755088782862e-05	\\
-113.538707386364	3.94743015929727e-05	\\
-112.559925426136	4.06697527055601e-05	\\
-111.58114346591	1.21360606808603e-05	\\
-110.602361505682	4.16947202745471e-06	\\
-109.623579545454	2.83547204801079e-05	\\
-108.644797585228	1.91055141982895e-05	\\
-107.666015625	3.71529743534527e-06	\\
-106.687233664772	5.7754900010554e-05	\\
-105.708451704546	3.7847832513372e-05	\\
-104.729669744318	3.44142474430662e-05	\\
-103.750887784092	2.74800012979628e-05	\\
-102.772105823864	4.04623389570023e-05	\\
-101.793323863636	3.96927255725561e-05	\\
-100.81454190341	1.07685797998218e-05	\\
-99.835759943182	3.94696609316412e-05	\\
-98.856977982954	4.43860436115255e-05	\\
-97.8781960227279	4.13198181686382e-05	\\
-96.8994140625	1.72537829349849e-05	\\
-95.9206321022721	2.22808953296222e-05	\\
-94.941850142046	3.27040639736503e-05	\\
-93.963068181818	2.45937650899682e-05	\\
-92.9842862215919	2.28048250044688e-05	\\
-92.005504261364	7.31696362122737e-07	\\
-91.026722301136	2.08927531059405e-05	\\
-90.0479403409099	1.10713907011156e-05	\\
-89.069158380682	3.86945324885963e-05	\\
-88.090376420454	1.93969216954912e-05	\\
-87.1115944602279	1.30541720497295e-05	\\
-86.1328125	2.62866147084555e-05	\\
-85.1540305397721	2.49148733730971e-05	\\
-84.175248579546	1.46273695544779e-05	\\
-83.196466619318	1.67066130508428e-05	\\
-82.2176846590919	1.31365638307863e-05	\\
-81.238902698864	2.14944106150958e-05	\\
-80.260120738636	1.81894578566901e-05	\\
-79.2813387784099	1.16309007077348e-05	\\
-78.302556818182	2.85144893954239e-05	\\
-77.323774857954	1.26283150929728e-05	\\
-76.3449928977279	1.83634871632165e-05	\\
-75.3662109375	6.76771744422097e-06	\\
-74.3874289772721	2.51469706664721e-07	\\
-73.408647017046	1.899868129073e-05	\\
-72.429865056818	6.71890336260396e-06	\\
-71.4510830965919	1.64728773593718e-05	\\
-70.472301136364	7.15992481472832e-06	\\
-69.493519176136	1.70918509382756e-05	\\
-68.5147372159099	1.85807629046554e-05	\\
-67.535955255682	3.12851778453006e-05	\\
-66.557173295454	5.30177722664479e-06	\\
-65.5783913352279	9.22728036662847e-06	\\
-64.599609375	2.28913946808909e-05	\\
-63.6208274147721	1.31773824540722e-05	\\
-62.642045454546	3.36197353496526e-05	\\
-61.663263494318	1.57837684985379e-05	\\
-60.6844815340919	5.75585290453909e-06	\\
-59.705699573864	3.270210778886e-05	\\
-58.726917613636	2.56639253391318e-05	\\
-57.7481356534099	6.31983045212309e-06	\\
-56.769353693182	1.18044270879132e-05	\\
-55.790571732954	1.25401904978684e-05	\\
-54.8117897727279	2.70964635382923e-05	\\
-53.8330078125	9.58597187045236e-06	\\
-52.8542258522721	2.39312504183369e-05	\\
-51.875443892046	3.54070680873368e-05	\\
-50.896661931818	4.38712801747883e-05	\\
-49.9178799715919	0.000204662576379567	\\
-48.939098011364	6.32312021303513e-06	\\
-47.960316051136	7.92695378591666e-06	\\
-46.9815340909099	9.68938945978329e-06	\\
-46.002752130682	2.03777734391983e-05	\\
-45.023970170454	0.000276872174992467	\\
-44.0451882102279	6.88442172277592e-05	\\
-43.06640625	5.97712536608204e-05	\\
-42.0876242897721	2.67751436482456e-05	\\
-41.108842329546	3.82730285387189e-05	\\
-40.130060369318	2.53775167094119e-05	\\
-39.1512784090919	4.42820064911459e-05	\\
-38.172496448864	6.04892342580482e-05	\\
-37.193714488636	1.87880630177847e-05	\\
-36.2149325284099	5.43890006901381e-05	\\
-35.236150568182	3.70233934151353e-05	\\
-34.257368607954	1.89798607140582e-05	\\
-33.2785866477279	3.51871726623834e-05	\\
-32.2998046875	1.85344422130779e-05	\\
-31.3210227272721	3.57559153544255e-05	\\
-30.342240767046	3.76672345386421e-05	\\
-29.363458806818	1.64387402837853e-05	\\
-28.3846768465919	7.12151689877144e-05	\\
-27.405894886364	4.45933196338226e-05	\\
-26.427112926136	0.00017579662123765	\\
-25.4483309659099	7.68847157045286e-05	\\
-24.469549005682	0.000107994346132303	\\
-23.490767045454	3.69565125757961e-05	\\
-22.5119850852279	4.31710214122576e-05	\\
-21.533203125	5.72818370102277e-05	\\
-20.5544211647721	5.8686313043969e-05	\\
-19.575639204546	1.19193172434488e-05	\\
-18.596857244318	7.48696855980465e-05	\\
-17.6180752840919	0.000107477563981821	\\
-16.639293323864	0.000155409951333081	\\
-15.660511363636	8.98311140274758e-05	\\
-14.6817294034099	0.000258875063732758	\\
-13.702947443182	0.000321939856974678	\\
-12.724165482954	0.000154936741838522	\\
-11.7453835227279	0.000219423564261718	\\
-10.7666015625	0.000133274449817057	\\
-9.78781960227207	5.4726444541705e-05	\\
-8.80903764204595	0.000340755135618347	\\
-7.83025568181802	0.000235030017585925	\\
-6.8514737215919	0.00021812612429568	\\
-5.87269176136397	0.000369931715002386	\\
-4.89390980113603	0.000122306622619304	\\
-3.91512784090992	5.89718367792984e-05	\\
-2.93634588068198	0.000263484037222152	\\
-1.95756392045405	0.000136913383003584	\\
-0.978781960227934	8.41801673867505e-05	\\
0	8.92322171818126e-05	\\
0.978781960226115	8.41801673867505e-05	\\
1.95756392045405	0.000136913383003584	\\
2.93634588068198	0.000263484037222152	\\
3.9151278409081	5.89718367792984e-05	\\
4.89390980113603	0.000122306622619304	\\
5.87269176136397	0.000369931715002386	\\
6.85147372159008	0.00021812612429568	\\
7.83025568181802	0.000235030017585925	\\
8.80903764204413	0.000340755135618347	\\
9.78781960227207	5.4726444541705e-05	\\
10.7666015625	0.000133274449817057	\\
11.7453835227261	0.000219423564261718	\\
12.724165482954	0.000154936741838522	\\
13.702947443182	0.000321939856974678	\\
14.6817294034081	0.000258875063732758	\\
15.660511363636	8.98311140274758e-05	\\
16.639293323864	0.000155409951333081	\\
17.6180752840901	0.000107477563981821	\\
18.596857244318	7.48696855980465e-05	\\
19.5756392045441	1.19193172434488e-05	\\
20.5544211647721	5.8686313043969e-05	\\
21.533203125	5.72818370102277e-05	\\
22.5119850852261	4.31710214122576e-05	\\
23.490767045454	3.69565125757961e-05	\\
24.469549005682	0.000107994346132303	\\
25.4483309659081	7.68847157045286e-05	\\
26.427112926136	0.00017579662123765	\\
27.405894886364	4.45933196338226e-05	\\
28.3846768465901	7.12151689877144e-05	\\
29.363458806818	1.64387402837853e-05	\\
30.3422407670441	3.76672345386421e-05	\\
31.3210227272721	3.57559153544255e-05	\\
32.2998046875	1.85344422130779e-05	\\
33.2785866477261	3.51871726623834e-05	\\
34.257368607954	1.89798607140582e-05	\\
35.236150568182	3.70233934151353e-05	\\
36.2149325284081	5.43890006901381e-05	\\
37.193714488636	1.87880630177847e-05	\\
38.172496448864	6.04892342580482e-05	\\
39.1512784090901	4.42820064911459e-05	\\
40.130060369318	2.53775167094119e-05	\\
41.1088423295441	3.82730285387189e-05	\\
42.0876242897721	2.67751436482456e-05	\\
43.06640625	5.97712536608204e-05	\\
44.0451882102261	6.88442172277592e-05	\\
45.023970170454	0.000276872174992467	\\
46.002752130682	2.03777734391983e-05	\\
46.9815340909081	9.68938945978329e-06	\\
47.960316051136	7.92695378591666e-06	\\
48.939098011364	6.32312021303513e-06	\\
49.9178799715901	0.000204662576379567	\\
50.896661931818	4.38712801747883e-05	\\
51.8754438920441	3.54070680873368e-05	\\
52.8542258522721	2.39312504183369e-05	\\
53.8330078125	9.58597187045236e-06	\\
54.8117897727261	2.70964635382923e-05	\\
55.790571732954	1.25401904978684e-05	\\
56.769353693182	1.18044270879132e-05	\\
57.7481356534081	6.31983045212309e-06	\\
58.726917613636	2.56639253391318e-05	\\
59.705699573864	3.270210778886e-05	\\
60.6844815340901	5.75585290453909e-06	\\
61.663263494318	1.57837684985379e-05	\\
62.6420454545441	3.36197353496526e-05	\\
63.6208274147721	1.31773824540722e-05	\\
64.599609375	2.28913946808909e-05	\\
65.5783913352261	9.22728036662847e-06	\\
66.557173295454	5.30177722664479e-06	\\
67.535955255682	3.12851778453006e-05	\\
68.5147372159081	1.85807629046554e-05	\\
69.493519176136	1.70918509382756e-05	\\
70.472301136364	7.15992481472832e-06	\\
71.4510830965901	1.64728773593718e-05	\\
72.429865056818	6.71890336260396e-06	\\
73.4086470170441	1.899868129073e-05	\\
74.3874289772721	2.51469706664721e-07	\\
75.3662109375	6.76771744422097e-06	\\
76.3449928977261	1.83634871632165e-05	\\
77.323774857954	1.26283150929728e-05	\\
78.302556818182	2.85144893954239e-05	\\
79.2813387784081	1.16309007077348e-05	\\
80.260120738636	1.81894578566901e-05	\\
81.238902698864	2.14944106150958e-05	\\
82.2176846590901	1.31365638307863e-05	\\
83.196466619318	1.67066130508428e-05	\\
84.1752485795441	1.46273695544779e-05	\\
85.1540305397721	2.49148733730971e-05	\\
86.1328125	2.62866147084555e-05	\\
87.1115944602261	1.30541720497295e-05	\\
88.090376420454	1.93969216954912e-05	\\
89.069158380682	3.86945324885963e-05	\\
90.0479403409081	1.10713907011156e-05	\\
91.026722301136	2.08927531059405e-05	\\
92.005504261364	7.31696362122737e-07	\\
92.9842862215901	2.28048250044688e-05	\\
93.963068181818	2.45937650899682e-05	\\
94.9418501420441	3.27040639736503e-05	\\
95.9206321022721	2.22808953296222e-05	\\
96.8994140625	1.72537829349849e-05	\\
97.8781960227261	4.13198181686382e-05	\\
98.856977982954	4.43860436115255e-05	\\
99.835759943182	3.94696609316412e-05	\\
100.814541903408	1.07685797998218e-05	\\
101.793323863636	3.96927255725561e-05	\\
102.772105823864	4.04623389570023e-05	\\
103.75088778409	2.74800012979628e-05	\\
104.729669744318	3.44142474430662e-05	\\
105.708451704544	3.7847832513372e-05	\\
106.687233664772	5.7754900010554e-05	\\
107.666015625	3.71529743534527e-06	\\
108.644797585226	1.91055141982895e-05	\\
109.623579545454	2.83547204801079e-05	\\
110.602361505682	4.16947202745471e-06	\\
111.581143465908	1.21360606808603e-05	\\
112.559925426136	4.06697527055601e-05	\\
113.538707386364	3.94743015929727e-05	\\
114.51748934659	4.02755088782862e-05	\\
115.496271306818	7.59215939438787e-05	\\
116.475053267044	7.28039667839832e-05	\\
117.453835227272	0.000106469268905831	\\
118.4326171875	1.54403642063348e-05	\\
119.411399147726	0.000171224886951917	\\
120.390181107954	0.000155010922028512	\\
121.368963068182	9.73878227580389e-05	\\
122.347745028408	7.31987232919203e-05	\\
123.326526988636	4.47863022303524e-05	\\
124.305308948864	3.26154736917817e-05	\\
125.28409090909	3.36542671072104e-05	\\
126.262872869318	7.02147700558056e-05	\\
127.241654829544	1.07935891917438e-05	\\
128.220436789772	4.68871798163384e-05	\\
129.19921875	1.03338194916531e-05	\\
130.178000710226	3.76129163982123e-06	\\
131.156782670454	9.80857185327013e-06	\\
132.135564630682	2.16228119829219e-05	\\
133.114346590908	1.77800518834677e-05	\\
134.093128551136	1.46234391687098e-05	\\
135.071910511364	2.09507722797264e-05	\\
136.05069247159	1.46999896775336e-05	\\
137.029474431818	1.35767480575585e-05	\\
138.008256392044	2.47374072087402e-05	\\
138.987038352272	8.97978665900717e-06	\\
139.9658203125	2.49067419801383e-05	\\
140.944602272726	3.03240621977909e-05	\\
141.923384232954	1.22598542140073e-05	\\
142.902166193182	1.27115826610508e-05	\\
143.880948153408	2.59678434165183e-05	\\
144.859730113636	2.34473571710534e-05	\\
145.838512073864	4.35693512428842e-06	\\
146.81729403409	1.36118404695407e-05	\\
147.796075994318	1.06412060113916e-05	\\
148.774857954544	4.90766779934486e-06	\\
149.753639914772	1.25231732493199e-05	\\
150.732421875	1.4245591354e-05	\\
151.711203835226	2.19285175027191e-05	\\
152.689985795454	6.98458740695419e-06	\\
153.668767755682	1.73799341198647e-05	\\
154.647549715908	1.82527865668471e-05	\\
155.626331676136	1.34491791748007e-05	\\
156.605113636364	1.94938008440974e-05	\\
157.58389559659	3.06268279483736e-05	\\
158.562677556818	1.70576873384178e-05	\\
159.541459517044	2.23099170693872e-05	\\
160.520241477272	2.02035891609963e-05	\\
161.4990234375	1.89922575349462e-05	\\
162.477805397726	2.17935250776098e-05	\\
163.456587357954	1.96687970011481e-05	\\
164.435369318182	2.20609828180829e-05	\\
165.414151278408	5.39074056764856e-06	\\
166.392933238636	5.24191478222908e-06	\\
167.371715198864	1.86006839252318e-05	\\
168.35049715909	1.78197208293526e-05	\\
169.329279119318	1.7005033203489e-05	\\
170.308061079544	1.19602760688581e-05	\\
171.286843039772	1.67269236992683e-05	\\
172.265625	1.07294940931124e-05	\\
173.244406960226	1.47095364027329e-05	\\
174.223188920454	1.31919303142413e-05	\\
175.201970880682	1.07143096910187e-05	\\
176.180752840908	1.55551987407367e-05	\\
177.159534801136	7.98286108276938e-06	\\
178.138316761364	2.10657043949735e-06	\\
179.11709872159	1.54773238968461e-05	\\
180.095880681818	1.55398825654265e-05	\\
181.074662642044	1.1980414107854e-05	\\
182.053444602272	1.72350687639912e-06	\\
183.0322265625	6.43572752379291e-06	\\
184.011008522726	2.29213900628714e-05	\\
184.989790482954	1.98008537729328e-05	\\
185.968572443182	7.16919790287202e-06	\\
186.947354403408	2.74345585992838e-05	\\
187.926136363636	3.70321705110765e-05	\\
188.904918323864	1.66769600863607e-05	\\
189.88370028409	5.79732845950958e-05	\\
190.862482244318	3.2233872205041e-05	\\
191.841264204544	4.48369113826216e-05	\\
192.820046164772	3.35089475090767e-05	\\
193.798828125	2.22264309159684e-05	\\
194.777610085226	3.33460041065527e-05	\\
195.756392045454	2.16766682432244e-05	\\
196.735174005682	2.53687729036661e-05	\\
197.713955965908	2.26033071866479e-05	\\
198.692737926136	1.62564810410153e-05	\\
199.671519886364	9.37034580671494e-05	\\
200.65030184659	6.23139698069534e-05	\\
201.629083806818	9.49632066015683e-06	\\
202.607865767044	1.21882043936745e-05	\\
203.586647727272	6.25592868526863e-05	\\
204.5654296875	5.09537566983254e-05	\\
205.544211647726	3.62800319017328e-05	\\
206.522993607954	1.65185490365717e-05	\\
207.501775568182	9.68404141680292e-06	\\
208.480557528408	2.50999982038408e-05	\\
209.459339488636	3.75792105350347e-05	\\
210.438121448864	3.07193690858742e-05	\\
211.41690340909	5.42481025894837e-06	\\
212.395685369318	2.60524097052198e-05	\\
213.374467329544	4.32472776814306e-05	\\
214.353249289772	4.34968319339674e-05	\\
215.33203125	2.31692992556046e-05	\\
216.310813210226	9.33475566475043e-06	\\
217.289595170454	1.64973090312554e-05	\\
218.268377130682	2.35899447978074e-05	\\
219.247159090908	4.44748956136826e-05	\\
220.225941051136	1.05967493878312e-05	\\
221.204723011364	2.27746126402448e-05	\\
222.18350497159	4.3328102764382e-05	\\
223.162286931818	3.17629160717594e-05	\\
224.141068892044	3.89300570260831e-05	\\
225.119850852272	3.53727437563164e-05	\\
226.0986328125	4.41425995056563e-05	\\
227.077414772726	3.16107075810678e-05	\\
228.056196732954	2.86249412726244e-05	\\
229.034978693182	4.74215730242393e-06	\\
230.013760653408	2.17275951331065e-05	\\
230.992542613636	3.40368851969701e-05	\\
231.971324573864	5.52684112625694e-05	\\
232.95010653409	3.04129553797197e-05	\\
233.928888494318	6.32453103174244e-05	\\
234.907670454544	3.32614624778023e-06	\\
235.886452414772	3.67442704616511e-05	\\
236.865234375	4.92863967536099e-05	\\
237.844016335226	6.89658645048605e-05	\\
238.822798295454	7.02854219945882e-06	\\
239.801580255682	3.35777706427941e-05	\\
240.780362215908	6.48679995224296e-06	\\
241.759144176136	3.34255933770816e-05	\\
242.737926136364	3.45549062475201e-05	\\
243.71670809659	1.40302898318878e-05	\\
244.695490056818	3.54716641122281e-05	\\
245.674272017044	2.53652976163585e-05	\\
246.653053977272	3.22298074190556e-05	\\
247.6318359375	1.29779528829385e-05	\\
248.610617897726	3.82726549725903e-05	\\
249.589399857954	3.86722049509518e-05	\\
250.568181818182	2.71840572781452e-05	\\
251.546963778408	2.16650474512315e-05	\\
252.525745738636	4.09854960190503e-05	\\
253.504527698864	5.29263198693489e-05	\\
254.48330965909	1.92260308199856e-05	\\
255.462091619318	2.8764201022962e-05	\\
256.440873579544	3.15087694177775e-05	\\
257.419655539772	3.72616622123681e-05	\\
258.3984375	3.13558016936223e-05	\\
259.377219460226	5.32189531685268e-06	\\
260.356001420454	2.83082583873209e-05	\\
261.334783380682	3.30107050067318e-05	\\
262.313565340908	2.11668438771198e-05	\\
263.292347301136	2.31757876572499e-05	\\
264.271129261364	3.06718244116973e-05	\\
265.24991122159	1.88558113124193e-05	\\
266.228693181818	1.29147962759044e-05	\\
267.207475142044	2.19627749909205e-05	\\
268.186257102272	4.0261991001619e-05	\\
269.1650390625	2.12839310229376e-05	\\
270.143821022726	3.2703221048734e-05	\\
271.122602982954	3.04609420439263e-05	\\
272.101384943182	2.74512897130715e-05	\\
273.080166903408	3.76973396506485e-05	\\
274.058948863636	3.74022749984316e-05	\\
275.037730823864	2.6335182856239e-05	\\
276.01651278409	5.59524473024743e-05	\\
276.995294744318	3.19582748219934e-05	\\
277.974076704544	2.73158505907969e-05	\\
278.952858664772	4.10423534990802e-05	\\
279.931640625	2.45817448780596e-05	\\
280.910422585226	2.48487325914035e-05	\\
281.889204545454	1.66957496756971e-05	\\
282.867986505682	1.73075982965608e-05	\\
283.846768465908	3.52691054074574e-05	\\
284.825550426136	3.77126058239519e-05	\\
285.804332386364	2.83414897977376e-05	\\
286.78311434659	1.71809081355211e-05	\\
287.761896306818	3.13949921549237e-05	\\
288.740678267044	2.50239457141121e-05	\\
289.719460227272	3.11019748101432e-05	\\
290.6982421875	6.63374912540006e-06	\\
291.677024147726	2.81866264268961e-05	\\
292.655806107954	3.71861765652586e-05	\\
293.634588068182	3.35381701586884e-05	\\
294.613370028408	2.8045587147829e-05	\\
295.592151988636	2.50837276914217e-05	\\
296.570933948864	1.88068230714376e-05	\\
297.54971590909	2.82279141697488e-05	\\
298.528497869318	1.07290580171436e-05	\\
299.507279829544	2.00070689927915e-05	\\
300.486061789772	1.88562756313924e-05	\\
301.46484375	3.72150022367026e-05	\\
302.443625710226	2.18066463133576e-05	\\
303.422407670454	2.28494439338124e-05	\\
304.401189630682	3.04567034233231e-05	\\
305.379971590908	4.73109833948079e-05	\\
306.358753551136	2.48263647154745e-05	\\
307.337535511364	1.98966775354921e-05	\\
308.31631747159	2.05573343063551e-05	\\
309.295099431818	4.39650353866587e-05	\\
310.273881392044	5.04573407643436e-05	\\
311.252663352272	1.16457647720438e-05	\\
312.2314453125	2.21300966165394e-05	\\
313.210227272726	2.96205760632817e-05	\\
314.189009232954	3.08614758243222e-05	\\
315.167791193182	3.68938718725798e-05	\\
316.146573153408	2.13844668451189e-05	\\
317.125355113636	1.40870345549061e-05	\\
318.104137073864	4.05897382203705e-05	\\
319.08291903409	3.0611707245655e-05	\\
320.061700994318	3.66201655626404e-05	\\
321.040482954544	3.21900537526874e-05	\\
322.019264914772	2.48797539632547e-05	\\
322.998046875	2.65242871642312e-05	\\
323.976828835226	2.36413311184072e-05	\\
324.955610795454	4.61038399328146e-05	\\
325.934392755682	2.92332680434826e-05	\\
326.913174715908	2.86871248002565e-05	\\
327.891956676136	2.28413633726523e-05	\\
328.870738636364	3.34638588134511e-05	\\
329.84952059659	2.7439665127071e-05	\\
330.828302556818	3.87765803794119e-05	\\
331.807084517044	3.03620788500954e-05	\\
332.785866477272	1.6199126075172e-05	\\
333.7646484375	4.04124219762785e-05	\\
334.743430397726	1.40988845110731e-05	\\
335.722212357954	1.20980483534506e-05	\\
336.700994318182	4.05836520859356e-05	\\
337.679776278408	4.05630194416877e-05	\\
338.658558238636	3.42473392101284e-05	\\
339.637340198864	4.48886775102556e-05	\\
340.61612215909	3.59317222866623e-05	\\
341.594904119318	4.8804338571435e-05	\\
342.573686079544	0.000105136145796466	\\
343.552468039772	5.44462563921365e-06	\\
344.53125	5.889925501635e-05	\\
345.510031960226	1.05828447723434e-05	\\
346.488813920454	6.24349764673535e-05	\\
347.467595880682	6.25860611747805e-05	\\
348.446377840908	9.35772362038391e-05	\\
349.425159801136	3.18225283988244e-06	\\
350.403941761364	9.47331764475587e-05	\\
351.38272372159	3.21028990996913e-05	\\
352.361505681818	5.62009367665947e-05	\\
353.340287642044	2.02746121612187e-05	\\
354.319069602272	5.96934498991286e-05	\\
355.2978515625	2.62787685969902e-05	\\
356.276633522726	6.26623590840445e-05	\\
357.255415482954	5.3349508478607e-05	\\
358.234197443182	3.64432813059313e-05	\\
359.212979403408	8.46072476340263e-05	\\
360.191761363636	3.12018706588094e-05	\\
361.170543323864	4.62314059848331e-05	\\
362.14932528409	4.20248240781679e-05	\\
363.128107244318	5.34921003043403e-05	\\
364.106889204544	9.05133586923665e-05	\\
365.085671164772	6.76421612176971e-05	\\
366.064453125	6.87933608334299e-05	\\
367.043235085226	7.60326946156582e-05	\\
368.022017045454	5.19214025349082e-05	\\
369.000799005682	4.92242711341177e-05	\\
369.979580965908	5.05293728261086e-05	\\
370.958362926136	3.49474389695559e-05	\\
371.937144886364	8.57670602119705e-05	\\
372.91592684659	3.17125081182983e-05	\\
373.894708806818	8.17653996815524e-05	\\
374.873490767044	5.54411385281674e-05	\\
375.852272727272	3.68216990637627e-05	\\
376.8310546875	6.60733292899106e-05	\\
377.809836647726	3.71140633570963e-05	\\
378.788618607954	5.69424395600803e-05	\\
379.767400568182	4.42458300481866e-05	\\
380.746182528408	8.00906720601307e-05	\\
381.724964488636	2.93899810596454e-05	\\
382.703746448864	5.52189348173172e-05	\\
383.68252840909	8.91834613800139e-05	\\
384.661310369318	4.3579206166213e-05	\\
385.640092329544	8.39119339634297e-05	\\
386.618874289772	7.6566883600889e-05	\\
387.59765625	3.15780051576232e-05	\\
388.576438210226	7.05914665474337e-05	\\
389.555220170454	8.65150158146357e-05	\\
390.534002130682	6.24281091759406e-05	\\
391.512784090908	5.61478862987714e-05	\\
392.491566051136	7.2597040965676e-05	\\
393.470348011364	7.0463001244633e-05	\\
394.44912997159	5.84497318897452e-05	\\
395.427911931818	7.65066930805513e-05	\\
396.406693892044	5.8502679572366e-05	\\
397.385475852272	5.38992531185612e-05	\\
398.3642578125	8.13149415168837e-05	\\
399.343039772726	8.06510162888739e-05	\\
400.321821732954	3.48370894038354e-05	\\
401.300603693182	7.52226057041926e-05	\\
402.279385653408	7.41878920003093e-05	\\
403.258167613636	5.96524789662297e-05	\\
404.236949573864	7.80246900798793e-05	\\
405.21573153409	0.000113683531883658	\\
406.194513494318	5.55893983165864e-05	\\
407.173295454544	9.06533938399967e-05	\\
408.152077414772	9.97965573876453e-05	\\
409.130859375	9.05613550427839e-05	\\
410.109641335226	4.98272544439734e-05	\\
411.088423295454	0.000110034611058596	\\
412.067205255682	9.96739596954701e-05	\\
413.045987215908	5.54392300620696e-05	\\
414.024769176136	9.4093404556972e-05	\\
415.003551136364	8.96616184493377e-05	\\
415.98233309659	0.000102553464129887	\\
416.961115056818	7.85747060552218e-05	\\
417.939897017044	0.000110848041752465	\\
418.918678977272	8.88306247722389e-05	\\
419.8974609375	7.48586213272427e-05	\\
420.876242897726	8.18622582240086e-05	\\
421.855024857954	9.08199061839862e-05	\\
422.833806818182	0.000103874475302965	\\
423.812588778408	9.03459694255144e-05	\\
424.791370738636	8.00422246376203e-05	\\
425.770152698864	9.28129454561543e-05	\\
426.74893465909	9.45501583788745e-05	\\
427.727716619318	9.16636039821066e-05	\\
428.706498579544	0.000119815085485441	\\
429.685280539772	0.000110136297313755	\\
430.6640625	8.41088555993166e-05	\\
431.642844460226	0.00010024042577775	\\
432.621626420454	0.000100123720205838	\\
433.600408380682	8.51846307588394e-05	\\
434.579190340908	0.000109460500490995	\\
435.557972301136	0.00011074739884441	\\
436.536754261364	0.000105916824425247	\\
437.51553622159	0.000115497563280457	\\
438.494318181818	8.60104991171097e-05	\\
439.473100142044	6.86944721358565e-05	\\
440.451882102272	0.000130467703617277	\\
441.4306640625	0.000142667101058632	\\
442.409446022726	9.21415128614452e-05	\\
443.388227982954	0.000151973465302088	\\
444.367009943182	6.48029709763576e-05	\\
445.345791903408	0.000112528910515303	\\
446.324573863636	0.000124927068161261	\\
447.303355823864	0.000160056662079332	\\
448.28213778409	9.09556677000842e-05	\\
449.260919744318	0.000163724778929368	\\
450.239701704544	0.000100889746759849	\\
451.218483664772	0.000132537564149841	\\
452.197265625	0.000146599682850673	\\
453.176047585226	0.000128094103420768	\\
454.154829545454	0.000138406410548606	\\
455.133611505682	0.000164584715791171	\\
456.112393465908	0.000114274416625444	\\
457.091175426136	0.000134905967666011	\\
458.069957386364	0.000169053924477682	\\
459.04873934659	0.000108256663434806	\\
460.027521306818	0.00017339107992144	\\
461.006303267044	0.000161184018450859	\\
461.985085227272	0.000132746300966396	\\
462.9638671875	0.000158499968474471	\\
463.942649147726	0.000167416672837569	\\
464.921431107954	0.000123990791518289	\\
465.900213068182	0.00017768483076876	\\
466.878995028408	0.00019670717068306	\\
467.857776988636	0.000166984881608197	\\
468.836558948864	0.000240440292382091	\\
469.81534090909	0.000124127527742713	\\
470.794122869318	0.000178083856105792	\\
471.772904829544	0.000165320431644236	\\
472.751686789772	0.000164622640616435	\\
473.73046875	0.000177670468196578	\\
474.709250710226	0.000173182593497764	\\
475.688032670454	0.000166592351340871	\\
476.666814630682	0.000158610536263311	\\
477.645596590908	0.00013803873468763	\\
478.624378551136	0.000221146070622862	\\
479.603160511364	0.000145893542123153	\\
480.58194247159	0.000196668997155521	\\
481.560724431818	0.000182564352947147	\\
482.539506392044	0.000134026609345064	\\
483.518288352272	0.000233284526486954	\\
484.4970703125	0.000146038401663673	\\
485.475852272726	0.000125410117837352	\\
486.454634232954	0.000243131477033972	\\
487.433416193182	0.000157039472187588	\\
488.412198153408	0.00017253615219624	\\
489.390980113636	0.000177222097129791	\\
490.369762073864	0.000144328467432106	\\
491.34854403409	0.000247508345119192	\\
492.327325994318	0.00012042621442941	\\
493.306107954544	0.000129105568808526	\\
494.284889914772	0.000191079476611039	\\
495.263671875	0.000157667372321216	\\
496.242453835226	0.000207921658848611	\\
497.221235795454	9.23630665959545e-05	\\
498.200017755682	0.000119945194745938	\\
499.178799715908	0.000250678740059178	\\
500.157581676136	0.000279591638042948	\\
501.136363636364	0.000166684017048953	\\
502.11514559659	0.000135636993745426	\\
503.093927556818	0.00013639658006407	\\
504.072709517044	0.000149039717369176	\\
505.051491477272	0.000136668542910361	\\
506.0302734375	0.000135873485910912	\\
507.009055397726	0.00014663982638384	\\
507.987837357954	0.000133807902236733	\\
508.966619318182	0.000142728344765436	\\
509.945401278408	0.000156085001692621	\\
510.924183238636	0.000142622660382329	\\
511.902965198864	0.000141254440779678	\\
512.88174715909	0.000113947993259274	\\
513.860529119318	0.000140037624217436	\\
514.839311079544	0.000150862709010176	\\
515.818093039772	0.000128633282439465	\\
516.796875	0.000135147532371595	\\
517.775656960226	0.000136102309816294	\\
518.754438920454	0.000135142512351941	\\
519.733220880682	0.000142626913608605	\\
520.712002840908	0.000133757240700737	\\
521.690784801136	0.000113111414700464	\\
522.669566761364	0.000125611233660318	\\
523.64834872159	0.000134431762128132	\\
524.627130681818	0.000119439098227114	\\
525.605912642044	0.000119883981852639	\\
526.584694602272	0.000128261054453276	\\
527.5634765625	0.000125119959674541	\\
528.542258522726	0.00011747789952342	\\
529.521040482954	0.000133338652972032	\\
530.499822443182	0.000128608786659857	\\
531.478604403408	0.000116579332758361	\\
532.457386363636	0.000120272511178417	\\
533.436168323864	0.000117726207200593	\\
534.41495028409	0.000125659154055618	\\
535.393732244318	0.000123696950747835	\\
536.372514204544	0.000131543788455357	\\
537.351296164772	0.000116645119994886	\\
538.330078125	0.000127535355463745	\\
539.308860085226	0.000122692976336838	\\
540.287642045454	0.000119519568001014	\\
541.266424005682	0.00013345269033881	\\
542.245205965908	0.000136228456834419	\\
543.223987926136	0.000123053693789974	\\
544.202769886364	0.000126687719224958	\\
545.18155184659	0.000130261293423872	\\
546.160333806818	0.000127953431911614	\\
547.139115767044	0.000131781483113614	\\
548.117897727272	0.000130657059022081	\\
549.0966796875	0.000136422474910427	\\
550.075461647726	0.000144385723543844	\\
551.054243607954	0.000135388479273837	\\
552.033025568182	0.000123705364661309	\\
553.011807528408	0.000131079777771768	\\
553.990589488636	0.000140143871068307	\\
554.969371448864	0.000135009222609048	\\
555.94815340909	0.000131188390773252	\\
556.926935369318	0.000130223719743985	\\
557.905717329544	0.000133473262699828	\\
558.884499289772	0.000128284496820961	\\
559.86328125	0.000141126189931798	\\
560.842063210226	0.000132486600724749	\\
561.820845170454	0.000128944734877604	\\
562.799627130682	0.000136573760113981	\\
563.778409090908	0.00012603985009876	\\
564.757191051136	0.00013422501268514	\\
565.735973011364	0.000127397690101963	\\
566.71475497159	0.000125313542573546	\\
567.693536931818	0.000135873469297831	\\
568.672318892044	0.000134808066805049	\\
569.651100852272	0.000111912012429212	\\
570.6298828125	0.00014847480663903	\\
571.608664772726	0.000117634905339596	\\
572.587446732954	0.000117052686976322	\\
573.566228693182	0.000140287981673303	\\
574.545010653408	0.000100988034500811	\\
575.523792613636	0.000145535373849433	\\
576.502574573864	0.000121012659362057	\\
577.48135653409	0.000104534050968721	\\
578.460138494318	0.000156485537749904	\\
579.438920454544	0.00010781732292413	\\
580.417702414772	0.000113432219817121	\\
581.396484375	0.000136140128850708	\\
582.375266335226	9.45508409445204e-05	\\
583.354048295454	0.000160204894471108	\\
584.332830255682	0.000113782528144575	\\
585.311612215908	0.000117423555071656	\\
586.290394176136	0.000123782512228607	\\
587.269176136364	0.000112954368207596	\\
588.24795809659	0.000119039987850897	\\
589.226740056818	0.000137704543618788	\\
590.205522017044	8.04233334725673e-05	\\
591.184303977272	0.000137260824535062	\\
592.1630859375	0.000112779410352825	\\
593.141867897726	0.000102098183201944	\\
594.120649857954	0.00011974303585481	\\
595.099431818182	9.58527054311977e-05	\\
596.078213778408	0.000116905545380655	\\
597.056995738636	0.000103725041651333	\\
598.035777698864	0.000112564544215232	\\
599.01455965909	9.91880099685119e-05	\\
599.993341619318	0.00012272571381101	\\
600.972123579544	0.00011274713185725	\\
601.950905539772	0.000113368076481642	\\
602.9296875	9.83653099663625e-05	\\
603.908469460226	0.000109833036635026	\\
604.887251420454	0.000100312985047374	\\
605.866033380682	0.000104089054414168	\\
606.844815340908	9.77505554755719e-05	\\
607.823597301136	9.52847470130709e-05	\\
608.802379261364	9.85292768604479e-05	\\
609.78116122159	0.000100164580484206	\\
610.759943181818	9.99449970404397e-05	\\
611.738725142044	0.000108063491842865	\\
612.717507102272	9.57641624795003e-05	\\
613.6962890625	9.3806105123896e-05	\\
614.675071022726	8.9976469588324e-05	\\
615.653852982954	9.22876872866311e-05	\\
616.632634943182	0.000108134444613207	\\
617.611416903408	9.31030002663672e-05	\\
618.590198863636	9.40751011338512e-05	\\
619.568980823864	9.70602575416961e-05	\\
620.54776278409	8.87177913382179e-05	\\
621.526544744318	9.02141368337834e-05	\\
622.505326704544	9.59343636849665e-05	\\
623.484108664772	9.27349048994567e-05	\\
624.462890625	9.24838004971666e-05	\\
625.441672585226	9.45165167243903e-05	\\
626.420454545454	9.48410160086349e-05	\\
627.399236505682	8.51347176659663e-05	\\
628.378018465908	9.09352483651814e-05	\\
629.356800426136	9.52573995732718e-05	\\
630.335582386364	8.35431797024713e-05	\\
631.31436434659	9.32977558787182e-05	\\
632.293146306818	9.78820504570538e-05	\\
633.271928267044	8.57859503723745e-05	\\
634.250710227272	9.10194107512472e-05	\\
635.2294921875	9.81216060456703e-05	\\
636.208274147726	8.87492323617994e-05	\\
637.187056107954	9.04175834114731e-05	\\
638.165838068182	8.46285031567988e-05	\\
639.144620028408	8.59036279191012e-05	\\
640.123401988636	9.11130135507994e-05	\\
641.102183948864	8.45325786667012e-05	\\
642.08096590909	8.91848320961131e-05	\\
643.059747869318	8.80506227721379e-05	\\
644.038529829544	9.0827145113805e-05	\\
645.017311789772	8.600163052129e-05	\\
645.99609375	9.0905950203028e-05	\\
646.974875710226	8.2463288551642e-05	\\
647.953657670454	8.3544561211207e-05	\\
648.932439630682	8.55936027830291e-05	\\
649.911221590908	8.93815986952502e-05	\\
650.890003551136	8.6395637586398e-05	\\
651.868785511364	7.64359556852028e-05	\\
652.84756747159	7.9130176652034e-05	\\
653.826349431818	7.76174405816724e-05	\\
654.805131392044	8.2720755215876e-05	\\
655.783913352272	9.21574460244006e-05	\\
656.7626953125	7.61206325130036e-05	\\
657.741477272726	7.87566483541948e-05	\\
658.720259232954	7.8494198162968e-05	\\
659.699041193182	8.01301899623775e-05	\\
660.677823153408	8.7270232619913e-05	\\
661.656605113636	8.55710615459481e-05	\\
662.635387073864	8.08121633119219e-05	\\
663.61416903409	7.58500784976843e-05	\\
664.592950994318	7.30940741825587e-05	\\
665.571732954544	7.72402190491263e-05	\\
666.550514914772	7.955813805029e-05	\\
667.529296875	8.45336390336137e-05	\\
668.508078835226	8.56194835796022e-05	\\
669.486860795454	7.22492614603769e-05	\\
670.465642755682	7.31283069781311e-05	\\
671.444424715908	7.68031215648481e-05	\\
672.423206676136	7.59340172380242e-05	\\
673.401988636364	8.2829801817198e-05	\\
674.38077059659	6.75107284220908e-05	\\
675.359552556818	7.4574475398716e-05	\\
676.338334517044	8.22998318074881e-05	\\
677.317116477272	8.47361206097341e-05	\\
678.2958984375	7.2954194394675e-05	\\
679.274680397726	8.24100395194657e-05	\\
680.253462357954	7.87386114641044e-05	\\
681.232244318182	7.57697336953435e-05	\\
682.211026278408	7.75211677766758e-05	\\
683.189808238636	8.56139177398313e-05	\\
684.168590198864	7.22930232364219e-05	\\
685.14737215909	8.40439143149305e-05	\\
686.126154119318	7.9674920121665e-05	\\
687.104936079544	8.38939532846656e-05	\\
688.083718039772	7.99403413849626e-05	\\
689.0625	8.19097427253757e-05	\\
690.041281960226	8.43636297144165e-05	\\
691.020063920454	8.26711753647711e-05	\\
691.998845880682	8.01003202525691e-05	\\
692.977627840908	8.33108338675101e-05	\\
693.956409801136	7.87722995854135e-05	\\
694.935191761364	8.17182775440001e-05	\\
695.91397372159	8.06194256196218e-05	\\
696.892755681818	7.85782912592637e-05	\\
697.871537642044	7.93580037395011e-05	\\
698.850319602272	8.44073604696388e-05	\\
699.8291015625	7.24287193921243e-05	\\
700.807883522726	9.66305775622937e-05	\\
701.786665482954	6.93430017590129e-05	\\
702.765447443182	8.05899309830736e-05	\\
703.744229403408	9.41326173916634e-05	\\
704.723011363636	7.39996633338393e-05	\\
705.701793323864	0.000103518248586232	\\
706.68057528409	7.80205360361526e-05	\\
707.659357244318	8.57518568846441e-05	\\
708.638139204544	8.41308913472717e-05	\\
709.616921164772	8.73381229391861e-05	\\
710.595703125	8.79325239744035e-05	\\
711.574485085226	7.57875103273732e-05	\\
712.553267045454	8.79115370084078e-05	\\
713.532049005682	8.97138664002139e-05	\\
714.510830965908	8.98083963195808e-05	\\
715.489612926136	8.44961447741453e-05	\\
716.468394886364	8.86072436048944e-05	\\
717.44717684659	8.56896409526054e-05	\\
718.425958806818	8.75974555125849e-05	\\
719.404740767044	8.84379334838966e-05	\\
720.383522727272	9.44443544868839e-05	\\
};
\addplot [color=blue,solid,forget plot]
  table[row sep=crcr]{
720.383522727272	9.44443544868839e-05	\\
721.3623046875	8.53899228582803e-05	\\
722.341086647726	8.75744062328281e-05	\\
723.319868607954	8.27279020703865e-05	\\
724.298650568182	8.82727541468191e-05	\\
725.277432528408	9.05067997537019e-05	\\
726.256214488636	8.54215598982911e-05	\\
727.234996448864	8.42646777567959e-05	\\
728.21377840909	8.59364190018999e-05	\\
729.192560369318	8.69841712965221e-05	\\
730.171342329544	9.0544118672495e-05	\\
731.150124289772	8.3492543593908e-05	\\
732.12890625	8.53889157722478e-05	\\
733.107688210226	8.39402553209738e-05	\\
734.086470170454	8.36549324605019e-05	\\
735.065252130682	8.43392482722378e-05	\\
736.044034090908	8.54651084632638e-05	\\
737.022816051136	8.43489302513415e-05	\\
738.001598011364	8.02240530490364e-05	\\
738.98037997159	8.23926101601765e-05	\\
739.959161931818	8.53725749290429e-05	\\
740.937943892044	7.85914866248232e-05	\\
741.916725852272	8.07891638955676e-05	\\
742.8955078125	8.34212181654176e-05	\\
743.874289772726	8.28383602633487e-05	\\
744.853071732954	7.84300848196734e-05	\\
745.831853693182	8.36097556127273e-05	\\
746.810635653408	7.94356255519973e-05	\\
747.789417613636	8.09198719933416e-05	\\
748.768199573864	7.8156107324714e-05	\\
749.74698153409	9.29976860607511e-05	\\
750.725763494318	8.53392054614454e-05	\\
751.704545454544	8.43385158386524e-05	\\
752.683327414772	8.02293396231848e-05	\\
753.662109375	7.82236852589815e-05	\\
754.640891335226	8.0952344554762e-05	\\
755.619673295454	7.9197780597265e-05	\\
756.598455255682	8.30734420774827e-05	\\
757.577237215908	7.98644060802115e-05	\\
758.556019176136	7.58865695349508e-05	\\
759.534801136364	8.202813718687e-05	\\
760.51358309659	8.13055831393365e-05	\\
761.492365056818	7.67313929318926e-05	\\
762.471147017044	7.46841327412512e-05	\\
763.449928977272	8.4630630515131e-05	\\
764.4287109375	7.82472872371401e-05	\\
765.407492897726	8.13377423363933e-05	\\
766.386274857954	7.82485419364901e-05	\\
767.365056818182	6.89831592053686e-05	\\
768.343838778408	7.79482895023478e-05	\\
769.322620738636	8.0061501353906e-05	\\
770.301402698864	7.55749223302749e-05	\\
771.28018465909	7.50842253829393e-05	\\
772.258966619318	7.55549129833951e-05	\\
773.237748579544	7.3745290580289e-05	\\
774.216530539772	7.41859362649011e-05	\\
775.1953125	7.30966903004247e-05	\\
776.174094460226	7.47452288551767e-05	\\
777.152876420454	7.89205633617282e-05	\\
778.131658380682	7.55045072621887e-05	\\
779.110440340908	7.41961453378057e-05	\\
780.089222301136	7.1302303007494e-05	\\
781.068004261364	7.71435601911231e-05	\\
782.04678622159	7.178894296151e-05	\\
783.025568181818	7.64429985774076e-05	\\
784.004350142044	7.76269476269566e-05	\\
784.983132102272	7.48062981885856e-05	\\
785.9619140625	7.25328083379587e-05	\\
786.940696022726	7.16224072526989e-05	\\
787.919477982954	7.32264960923762e-05	\\
788.898259943182	6.92912572849787e-05	\\
789.877041903408	7.59447643630975e-05	\\
790.855823863636	7.28007796877155e-05	\\
791.834605823864	7.20270690644201e-05	\\
792.81338778409	7.20032896739253e-05	\\
793.792169744318	7.43747736503062e-05	\\
794.770951704544	6.60872714982787e-05	\\
795.749733664772	7.31588240645963e-05	\\
796.728515625	6.91386863614016e-05	\\
797.707297585226	7.181231710891e-05	\\
798.686079545454	7.46394038389594e-05	\\
799.664861505682	7.45688473942645e-05	\\
800.643643465908	6.87613625754551e-05	\\
801.622425426136	6.98825382165938e-05	\\
802.601207386364	7.04495472289753e-05	\\
803.57998934659	7.01505254032497e-05	\\
804.558771306818	7.02246474401404e-05	\\
805.537553267044	6.71525709984898e-05	\\
806.516335227272	7.16493152242473e-05	\\
807.4951171875	6.5149231587245e-05	\\
808.473899147726	6.91402617104139e-05	\\
809.452681107954	6.67812788989686e-05	\\
810.431463068182	6.79455316496464e-05	\\
811.410245028408	6.67195128083376e-05	\\
812.389026988636	7.4217699015231e-05	\\
813.367808948864	6.52150313915229e-05	\\
814.34659090909	6.80669284786718e-05	\\
815.325372869318	6.96076832810748e-05	\\
816.304154829544	6.75553850070729e-05	\\
817.282936789772	6.68200225976235e-05	\\
818.26171875	6.84448632132731e-05	\\
819.240500710226	6.63867860970283e-05	\\
820.219282670454	7.04440758044989e-05	\\
821.198064630682	7.02816296043209e-05	\\
822.176846590908	6.67994608800322e-05	\\
823.155628551136	7.11326291360231e-05	\\
824.134410511364	6.44650892272847e-05	\\
825.11319247159	6.74391678084233e-05	\\
826.091974431818	6.76244680808312e-05	\\
827.070756392044	6.90216557128072e-05	\\
828.049538352272	6.74234621639639e-05	\\
829.0283203125	7.14797623723437e-05	\\
830.007102272726	7.20955051329959e-05	\\
830.985884232954	6.82044123858346e-05	\\
831.964666193182	6.92361836053778e-05	\\
832.943448153408	7.31515584925845e-05	\\
833.922230113636	6.89070457086136e-05	\\
834.901012073864	6.89816431684692e-05	\\
835.87979403409	6.6472203929381e-05	\\
836.858575994318	6.31321264961741e-05	\\
837.837357954544	6.64427151763759e-05	\\
838.816139914772	6.964198302314e-05	\\
839.794921875	6.92322139879547e-05	\\
840.773703835226	7.10395117602711e-05	\\
841.752485795454	6.22321049099159e-05	\\
842.731267755682	6.83452952424399e-05	\\
843.710049715908	6.9337729687768e-05	\\
844.688831676136	6.88044322296852e-05	\\
845.667613636364	7.14513945916733e-05	\\
846.64639559659	6.58401177947199e-05	\\
847.625177556818	6.97061763732674e-05	\\
848.603959517044	6.77345119948678e-05	\\
849.582741477272	6.95230589808919e-05	\\
850.5615234375	7.22522894531805e-05	\\
851.540305397726	6.82253103443e-05	\\
852.519087357954	7.00463725102872e-05	\\
853.497869318182	7.18860890749158e-05	\\
854.476651278408	6.72439509879035e-05	\\
855.455433238636	6.84531575937222e-05	\\
856.434215198864	7.29716308869513e-05	\\
857.41299715909	7.44140786149137e-05	\\
858.391779119318	7.10747730779706e-05	\\
859.370561079544	6.5427327651685e-05	\\
860.349343039772	7.43378404555326e-05	\\
861.328125	6.69111458197182e-05	\\
862.306906960226	6.89964700075736e-05	\\
863.285688920454	7.36706771120954e-05	\\
864.264470880682	6.72118131699338e-05	\\
865.243252840908	7.14750924077809e-05	\\
866.222034801136	7.33991444146491e-05	\\
867.200816761364	7.18598401762376e-05	\\
868.17959872159	6.86089152411195e-05	\\
869.158380681818	6.90738138326723e-05	\\
870.137162642044	6.91044613468577e-05	\\
871.115944602272	6.54186972146494e-05	\\
872.0947265625	6.71625240104067e-05	\\
873.073508522726	7.16709660537447e-05	\\
874.052290482954	6.95424381832472e-05	\\
875.031072443182	7.07317764794397e-05	\\
876.009854403408	7.02364019290221e-05	\\
876.988636363636	7.29325328538469e-05	\\
877.967418323864	7.32247833914607e-05	\\
878.94620028409	7.4463486338994e-05	\\
879.924982244318	7.15279008416193e-05	\\
880.903764204544	6.77897970499113e-05	\\
881.882546164772	7.10024849785738e-05	\\
882.861328125	6.89420340873496e-05	\\
883.840110085226	7.24631677455416e-05	\\
884.818892045454	7.20622322454628e-05	\\
885.797674005682	6.77339121197433e-05	\\
886.776455965908	6.91586713726547e-05	\\
887.755237926136	6.74321224030886e-05	\\
888.734019886364	7.2273929614153e-05	\\
889.71280184659	7.30219403671658e-05	\\
890.691583806818	7.25545652023329e-05	\\
891.670365767044	7.05069709594241e-05	\\
892.649147727272	7.26566355266061e-05	\\
893.6279296875	7.15822248824777e-05	\\
894.606711647726	7.11720503127498e-05	\\
895.585493607954	7.46580923895372e-05	\\
896.564275568182	6.86159942819554e-05	\\
897.543057528408	6.75186192777917e-05	\\
898.521839488636	6.91563549634001e-05	\\
899.500621448864	5.79532553404562e-05	\\
900.47940340909	7.4578508169726e-05	\\
901.458185369318	6.82285038510304e-05	\\
902.436967329544	6.91329671604834e-05	\\
903.415749289772	7.00563969910263e-05	\\
904.39453125	6.95482441227108e-05	\\
905.373313210226	6.49310985948449e-05	\\
906.352095170454	6.68380493717286e-05	\\
907.330877130682	7.00755779040479e-05	\\
908.309659090908	6.65337298441983e-05	\\
909.288441051136	6.63044701029663e-05	\\
910.267223011364	6.91631135549049e-05	\\
911.24600497159	6.91742379110508e-05	\\
912.224786931818	6.75603282474609e-05	\\
913.203568892044	6.76486097134558e-05	\\
914.182350852272	6.50593322164388e-05	\\
915.1611328125	6.6818308942935e-05	\\
916.139914772726	6.73004286183966e-05	\\
917.118696732954	6.99755807772391e-05	\\
918.097478693182	6.85245168818382e-05	\\
919.076260653408	6.4043784128067e-05	\\
920.055042613636	6.8867324894169e-05	\\
921.033824573864	6.11334701273185e-05	\\
922.01260653409	6.79165085524725e-05	\\
922.991388494318	6.6711391720327e-05	\\
923.970170454544	6.5687144984925e-05	\\
924.948952414772	6.6730821701427e-05	\\
925.927734375	6.69182028701608e-05	\\
926.906516335226	6.31803267420059e-05	\\
927.885298295454	6.43804350009059e-05	\\
928.864080255682	6.31007600062112e-05	\\
929.842862215908	6.84392232488573e-05	\\
930.821644176136	6.36795174103814e-05	\\
931.800426136364	6.52782858138942e-05	\\
932.77920809659	6.72257216326028e-05	\\
933.757990056818	6.74196687506869e-05	\\
934.736772017044	6.94366560415252e-05	\\
935.715553977272	7.01608075645626e-05	\\
936.6943359375	6.84184959771896e-05	\\
937.673117897726	6.22576758953016e-05	\\
938.651899857954	6.57198265856507e-05	\\
939.630681818182	7.13594807643563e-05	\\
940.609463778408	6.59517083213646e-05	\\
941.588245738636	7.07317775792983e-05	\\
942.567027698864	6.44528592300826e-05	\\
943.54580965909	6.98073391704652e-05	\\
944.524591619318	6.78562805241081e-05	\\
945.503373579544	6.711605279668e-05	\\
946.482155539772	7.41664205361795e-05	\\
947.4609375	6.6806176691645e-05	\\
948.439719460226	6.75659163438335e-05	\\
949.418501420454	7.36623104205919e-05	\\
950.397283380682	6.60098868369011e-05	\\
951.376065340908	7.13829090449519e-05	\\
952.354847301136	7.0923210638446e-05	\\
953.333629261364	6.91229145489078e-05	\\
954.31241122159	6.68172411639442e-05	\\
955.291193181818	6.88932246436078e-05	\\
956.269975142044	6.90063340448371e-05	\\
957.248757102272	6.96865627837872e-05	\\
958.2275390625	7.06940909421247e-05	\\
959.206321022726	7.25580265107267e-05	\\
960.185102982954	6.47581763622779e-05	\\
961.163884943182	7.38755961870801e-05	\\
962.142666903408	6.40017986958014e-05	\\
963.121448863636	6.88110445744758e-05	\\
964.100230823864	6.99920911005958e-05	\\
965.07901278409	7.20590631881186e-05	\\
966.057794744318	6.57531808759982e-05	\\
967.036576704544	6.71079839996313e-05	\\
968.015358664772	7.25954544799269e-05	\\
968.994140625	6.60701615026195e-05	\\
969.972922585226	7.19198181348644e-05	\\
970.951704545454	7.09703551794902e-05	\\
971.930486505682	6.84587498587346e-05	\\
972.909268465908	7.13946433650998e-05	\\
973.888050426136	6.89637646425298e-05	\\
974.866832386364	7.23302259874401e-05	\\
975.84561434659	6.86096638695113e-05	\\
976.824396306818	6.69962422649126e-05	\\
977.803178267044	6.99767221308718e-05	\\
978.781960227272	6.93497786543804e-05	\\
979.7607421875	7.01253512385076e-05	\\
980.739524147726	7.38616450511793e-05	\\
981.718306107954	7.18258394549185e-05	\\
982.697088068182	7.64256863598271e-05	\\
983.675870028408	7.13593837789373e-05	\\
984.654651988636	6.6690910230396e-05	\\
985.633433948864	7.1261986326618e-05	\\
986.61221590909	7.12718387951737e-05	\\
987.590997869318	7.24608016395374e-05	\\
988.569779829544	7.32654411759898e-05	\\
989.548561789772	6.58100771461316e-05	\\
990.52734375	6.77358454237514e-05	\\
991.506125710226	6.94961192227316e-05	\\
992.484907670454	7.04254202457983e-05	\\
993.463689630682	7.09602529554125e-05	\\
994.442471590908	7.15866359124481e-05	\\
995.421253551136	6.90087575552175e-05	\\
996.400035511364	6.54082489650428e-05	\\
997.37881747159	6.68122088761847e-05	\\
998.357599431818	6.74267998893388e-05	\\
999.336381392044	5.88873817391748e-05	\\
1000.31516335227	7.21419230244386e-05	\\
1001.2939453125	6.66750398017619e-05	\\
1002.27272727273	6.62761593277583e-05	\\
1003.25150923295	6.74748371159714e-05	\\
1004.23029119318	6.98221755297969e-05	\\
1005.20907315341	6.33032272064245e-05	\\
1006.18785511364	6.51742193139843e-05	\\
1007.16663707386	6.73678954226892e-05	\\
1008.14541903409	6.57571202786521e-05	\\
1009.12420099432	6.41200837023536e-05	\\
1010.10298295454	6.45556576542726e-05	\\
1011.08176491477	6.14870538848865e-05	\\
1012.060546875	6.0724007171956e-05	\\
1013.03932883523	6.57285753976103e-05	\\
1014.01811079545	6.53223290305977e-05	\\
1014.99689275568	6.27266171028978e-05	\\
1015.97567471591	6.15714608554485e-05	\\
1016.95445667614	6.33469584052962e-05	\\
1017.93323863636	6.23617334227881e-05	\\
1018.91202059659	6.41468774139692e-05	\\
1019.89080255682	6.21063352469651e-05	\\
1020.86958451704	6.29956592154883e-05	\\
1021.84836647727	6.45616223479899e-05	\\
1022.8271484375	5.76271918688749e-05	\\
1023.80593039773	6.35127417476145e-05	\\
1024.78471235795	6.3924395234144e-05	\\
1025.76349431818	6.60382560770714e-05	\\
1026.74227627841	6.44392614739832e-05	\\
1027.72105823864	6.03622311063464e-05	\\
1028.69984019886	6.0491822019308e-05	\\
1029.67862215909	5.92518811357062e-05	\\
1030.65740411932	6.39621744932414e-05	\\
1031.63618607954	6.64164117165653e-05	\\
1032.61496803977	6.1586369061888e-05	\\
1033.59375	6.63229401581665e-05	\\
1034.57253196023	6.46306102090772e-05	\\
1035.55131392045	6.543543268887e-05	\\
1036.53009588068	6.57609851696029e-05	\\
1037.50887784091	6.3520017494572e-05	\\
1038.48765980114	6.44176260066688e-05	\\
1039.46644176136	6.15665274266747e-05	\\
1040.44522372159	6.69152694117437e-05	\\
1041.42400568182	6.54161114565789e-05	\\
1042.40278764204	6.33086622708684e-05	\\
1043.38156960227	6.27197536818028e-05	\\
1044.3603515625	6.59463745351008e-05	\\
1045.33913352273	6.06974090839357e-05	\\
1046.31791548295	7.0574464214248e-05	\\
1047.29669744318	6.25610991784982e-05	\\
1048.27547940341	6.40263506627129e-05	\\
1049.25426136364	6.99147988420234e-05	\\
1050.23304332386	6.43138081708867e-05	\\
1051.21182528409	6.84051914989086e-05	\\
1052.19060724432	6.64578509485448e-05	\\
1053.16938920454	6.37760771625752e-05	\\
1054.14817116477	6.36564053244414e-05	\\
1055.126953125	6.82759995901256e-05	\\
1056.10573508523	7.2165677997677e-05	\\
1057.08451704545	6.6287912920515e-05	\\
1058.06329900568	6.81015515983378e-05	\\
1059.04208096591	6.99487778588048e-05	\\
1060.02086292614	6.51039166630762e-05	\\
1060.99964488636	6.8318644660732e-05	\\
1061.97842684659	6.69782515108321e-05	\\
1062.95720880682	6.78611490176289e-05	\\
1063.93599076704	6.70230074968201e-05	\\
1064.91477272727	7.42923780001926e-05	\\
1065.8935546875	6.78004857740086e-05	\\
1066.87233664773	6.7258951455222e-05	\\
1067.85111860795	6.93006689869806e-05	\\
1068.82990056818	7.01981448514428e-05	\\
1069.80868252841	6.8419798151557e-05	\\
1070.78746448864	6.74190241438814e-05	\\
1071.76624644886	6.63046484817999e-05	\\
1072.74502840909	6.98395352372522e-05	\\
1073.72381036932	6.94744742270887e-05	\\
1074.70259232954	6.5029814658347e-05	\\
1075.68137428977	6.76102403055838e-05	\\
1076.66015625	6.83022564908106e-05	\\
1077.63893821023	7.1605967765305e-05	\\
1078.61772017045	7.23574639371004e-05	\\
1079.59650213068	7.04631763501528e-05	\\
1080.57528409091	6.49375491014594e-05	\\
1081.55406605114	6.99799452737065e-05	\\
1082.53284801136	6.97263145509775e-05	\\
1083.51162997159	7.21164666670312e-05	\\
1084.49041193182	7.15706562672213e-05	\\
1085.46919389204	7.0575921769202e-05	\\
1086.44797585227	6.30184330852954e-05	\\
1087.4267578125	7.02435474572779e-05	\\
1088.40553977273	7.14389991949601e-05	\\
1089.38432173295	6.27202259645135e-05	\\
1090.36310369318	7.07492129837223e-05	\\
1091.34188565341	7.33910682171349e-05	\\
1092.32066761364	6.36952573575155e-05	\\
1093.29944957386	6.86521712645116e-05	\\
1094.27823153409	6.60054524475506e-05	\\
1095.25701349432	6.469095876145e-05	\\
1096.23579545454	6.54114236499013e-05	\\
1097.21457741477	6.22865370602066e-05	\\
1098.193359375	7.29457972600861e-05	\\
1099.17214133523	6.81042758556668e-05	\\
1100.15092329545	5.54569060468966e-05	\\
1101.12970525568	7.04606033888937e-05	\\
1102.10848721591	6.45211666155676e-05	\\
1103.08726917614	6.70766607636373e-05	\\
1104.06605113636	7.16943613938939e-05	\\
1105.04483309659	6.39515043269353e-05	\\
1106.02361505682	6.13745099282085e-05	\\
1107.00239701704	6.83503150882998e-05	\\
1107.98117897727	6.55483973123442e-05	\\
1108.9599609375	7.19630888134978e-05	\\
1109.93874289773	6.65606782618762e-05	\\
1110.91752485795	6.50728240116832e-05	\\
1111.89630681818	6.97289024890121e-05	\\
1112.87508877841	6.51726552884863e-05	\\
1113.85387073864	5.93662199001657e-05	\\
1114.83265269886	6.49174491741042e-05	\\
1115.81143465909	6.83241025031196e-05	\\
1116.79021661932	6.47323060108654e-05	\\
1117.76899857954	6.25339011384832e-05	\\
1118.74778053977	6.31376625036717e-05	\\
1119.7265625	6.55276129013756e-05	\\
1120.70534446023	5.8791944046492e-05	\\
1121.68412642045	6.21280208557814e-05	\\
1122.66290838068	6.58643115257113e-05	\\
1123.64169034091	6.20472924685397e-05	\\
1124.62047230114	6.25281449422499e-05	\\
1125.59925426136	6.90742370775104e-05	\\
1126.57803622159	6.13574944929872e-05	\\
1127.55681818182	6.21992403968714e-05	\\
1128.53560014204	6.35126866470001e-05	\\
1129.51438210227	6.86230236004959e-05	\\
1130.4931640625	6.42990667296674e-05	\\
1131.47194602273	6.39524171371368e-05	\\
1132.45072798295	6.20047997809859e-05	\\
1133.42950994318	6.24425748709737e-05	\\
1134.40829190341	6.2831106245158e-05	\\
1135.38707386364	6.42938832429947e-05	\\
1136.36585582386	6.57902264347514e-05	\\
1137.34463778409	6.02343413634212e-05	\\
1138.32341974432	6.16309151478107e-05	\\
1139.30220170454	6.21094870925879e-05	\\
1140.28098366477	6.65576783904134e-05	\\
1141.259765625	6.10612875911946e-05	\\
1142.23854758523	6.26008620773956e-05	\\
1143.21732954545	5.97079717100843e-05	\\
1144.19611150568	5.8727778823722e-05	\\
1145.17489346591	6.69926196227969e-05	\\
1146.15367542614	6.57038895508793e-05	\\
1147.13245738636	6.10058065099975e-05	\\
1148.11123934659	7.02906083323498e-05	\\
1149.09002130682	6.50485290676293e-05	\\
1150.06880326704	6.60563125645972e-05	\\
1151.04758522727	6.3557918227914e-05	\\
1152.0263671875	6.62420025765791e-05	\\
1153.00514914773	6.59874625513641e-05	\\
1153.98393110795	6.16271809452027e-05	\\
1154.96271306818	7.08934971245045e-05	\\
1155.94149502841	6.30521024523885e-05	\\
1156.92027698864	6.40724400455962e-05	\\
1157.89905894886	6.54865566566962e-05	\\
1158.87784090909	6.73509910232222e-05	\\
1159.85662286932	6.35430721064383e-05	\\
1160.83540482954	6.46373839141848e-05	\\
1161.81418678977	6.6443375958638e-05	\\
1162.79296875	6.4090789856072e-05	\\
1163.77175071023	6.59038967181849e-05	\\
1164.75053267045	6.80613447723065e-05	\\
1165.72931463068	6.78748210988951e-05	\\
1166.70809659091	6.6880152443489e-05	\\
1167.68687855114	6.96481963351562e-05	\\
1168.66566051136	6.63080531098482e-05	\\
1169.64444247159	7.12533146390782e-05	\\
1170.62322443182	6.36811067833528e-05	\\
1171.60200639204	6.50429943657795e-05	\\
1172.58078835227	6.39684967877719e-05	\\
1173.5595703125	6.46291882691736e-05	\\
1174.53835227273	6.76137323825597e-05	\\
1175.51713423295	6.69704858435076e-05	\\
1176.49591619318	6.88170579578423e-05	\\
1177.47469815341	6.62500320818051e-05	\\
1178.45348011364	7.16253939712208e-05	\\
1179.43226207386	6.81687648726898e-05	\\
1180.41104403409	6.75927432244404e-05	\\
1181.38982599432	6.64152630192257e-05	\\
1182.36860795454	6.78807283673555e-05	\\
1183.34738991477	6.95977789173867e-05	\\
1184.326171875	7.10864895307776e-05	\\
1185.30495383523	7.00299027540886e-05	\\
1186.28373579545	6.55157060302652e-05	\\
1187.26251775568	6.92054942946277e-05	\\
1188.24129971591	6.92596325556606e-05	\\
1189.22008167614	6.7119161720772e-05	\\
1190.19886363636	7.00416094516207e-05	\\
1191.17764559659	6.87614836060559e-05	\\
1192.15642755682	6.62538973087948e-05	\\
1193.13520951704	7.06620028182326e-05	\\
1194.11399147727	6.81637765171596e-05	\\
1195.0927734375	6.85546950643876e-05	\\
1196.07155539773	6.63308323148036e-05	\\
1197.05033735795	6.8919067304461e-05	\\
1198.02911931818	6.77516790293282e-05	\\
1199.00790127841	7.37119924802107e-05	\\
1199.98668323864	6.70558815285499e-05	\\
1200.96546519886	7.21658626090704e-05	\\
1201.94424715909	6.69142673324821e-05	\\
1202.92302911932	6.74912408964425e-05	\\
1203.90181107954	6.97498720923992e-05	\\
1204.88059303977	7.26293948595658e-05	\\
1205.859375	7.01937821226805e-05	\\
1206.83815696023	6.59276720148339e-05	\\
1207.81693892045	7.26552849322838e-05	\\
1208.79572088068	6.92003471600966e-05	\\
1209.77450284091	7.13687413067307e-05	\\
1210.75328480114	6.49179961781418e-05	\\
1211.73206676136	6.90418612901082e-05	\\
1212.71084872159	7.07670755608659e-05	\\
1213.68963068182	7.17788381835568e-05	\\
1214.66841264204	6.98230385825986e-05	\\
1215.64719460227	6.88060642590453e-05	\\
1216.6259765625	6.743660785498e-05	\\
1217.60475852273	6.96663785375414e-05	\\
1218.58354048295	7.18724242073892e-05	\\
1219.56232244318	6.66629455297406e-05	\\
1220.54110440341	6.62287811386821e-05	\\
1221.51988636364	6.9485768995057e-05	\\
1222.49866832386	6.90835995286125e-05	\\
1223.47745028409	6.84994614137269e-05	\\
1224.45623224432	6.9008206733843e-05	\\
1225.43501420454	6.43105244344851e-05	\\
1226.41379616477	7.0374636687281e-05	\\
1227.392578125	6.59644670544177e-05	\\
1228.37136008523	6.4105247234472e-05	\\
1229.35014204545	7.35381712644954e-05	\\
1230.32892400568	6.50105846799772e-05	\\
1231.30770596591	6.86371189513373e-05	\\
1232.28648792614	6.92181667372372e-05	\\
1233.26526988636	6.403917045221e-05	\\
1234.24405184659	6.78491179601326e-05	\\
1235.22283380682	7.09232155169421e-05	\\
1236.20161576704	6.20235793459906e-05	\\
1237.18039772727	7.08841165132445e-05	\\
1238.1591796875	7.11606264774428e-05	\\
1239.13796164773	6.59346856665802e-05	\\
1240.11674360795	6.87536174062221e-05	\\
1241.09552556818	6.90950126080139e-05	\\
1242.07430752841	7.02688906120023e-05	\\
1243.05308948864	6.30594899704183e-05	\\
1244.03187144886	6.77687728635653e-05	\\
1245.01065340909	6.67856342813341e-05	\\
1245.98943536932	7.51089263156498e-05	\\
1246.96821732954	6.73276288538167e-05	\\
1247.94699928977	7.12036514883714e-05	\\
1248.92578125	6.90608394593709e-05	\\
1249.90456321023	6.16972402616766e-05	\\
1250.88334517045	6.99965319873786e-05	\\
1251.86212713068	6.97427774038612e-05	\\
1252.84090909091	6.7982763155239e-05	\\
1253.81969105114	6.50761729170479e-05	\\
1254.79847301136	6.82331933974385e-05	\\
1255.77725497159	7.02180148459143e-05	\\
1256.75603693182	7.16427483439094e-05	\\
1257.73481889204	6.58089146408972e-05	\\
1258.71360085227	6.91251984650181e-05	\\
1259.6923828125	6.99544260403918e-05	\\
1260.67116477273	7.13323173450988e-05	\\
1261.64994673295	6.71878762406994e-05	\\
1262.62872869318	6.96798888912948e-05	\\
1263.60751065341	6.92731591592296e-05	\\
1264.58629261364	7.12954111582764e-05	\\
1265.56507457386	7.44877460264429e-05	\\
1266.54385653409	7.05259194486128e-05	\\
1267.52263849432	7.33644471559549e-05	\\
1268.50142045454	7.39850512550219e-05	\\
1269.48020241477	7.11054358029681e-05	\\
1270.458984375	7.59087736956173e-05	\\
1271.43776633523	7.79263525736223e-05	\\
1272.41654829545	7.30733304248674e-05	\\
1273.39533025568	7.95290555774426e-05	\\
1274.37411221591	7.07902180400229e-05	\\
1275.35289417614	7.02460233926155e-05	\\
1276.33167613636	7.07291362782124e-05	\\
1277.31045809659	7.25934020748257e-05	\\
1278.28924005682	7.34494962330474e-05	\\
1279.26802201704	7.48387777834824e-05	\\
1280.24680397727	7.00649097221329e-05	\\
1281.2255859375	7.10662787445969e-05	\\
1282.20436789773	7.0368263934984e-05	\\
1283.18314985795	7.12666601748941e-05	\\
1284.16193181818	6.77620758216622e-05	\\
1285.14071377841	6.89776680041013e-05	\\
1286.11949573864	7.11973586099946e-05	\\
1287.09827769886	6.48062892427927e-05	\\
1288.07705965909	7.27291101853545e-05	\\
1289.05584161932	6.95280960820002e-05	\\
1290.03462357954	6.89775653624081e-05	\\
1291.01340553977	7.16535486904712e-05	\\
1291.9921875	6.95306025040036e-05	\\
1292.97096946023	6.8396720455793e-05	\\
1293.94975142045	6.35254849768846e-05	\\
1294.92853338068	6.73343944143166e-05	\\
1295.90731534091	6.81600043590372e-05	\\
1296.88609730114	6.70580646250098e-05	\\
1297.86487926136	6.28390626895765e-05	\\
1298.84366122159	6.40048727717285e-05	\\
1299.82244318182	5.86443141464507e-05	\\
1300.80122514204	6.52829026676941e-05	\\
1301.78000710227	6.65678880782066e-05	\\
1302.7587890625	6.12222064349176e-05	\\
1303.73757102273	6.49962024521929e-05	\\
1304.71635298295	6.42196392851783e-05	\\
1305.69513494318	6.70477761759113e-05	\\
1306.67391690341	6.55117723026808e-05	\\
1307.65269886364	6.40148289350144e-05	\\
1308.63148082386	6.42582505575339e-05	\\
1309.61026278409	5.35644979430262e-05	\\
1310.58904474432	6.32499432507742e-05	\\
1311.56782670454	6.02002465806806e-05	\\
1312.54660866477	6.62038308478977e-05	\\
1313.525390625	6.74652152259395e-05	\\
1314.50417258523	6.24151465014982e-05	\\
1315.48295454545	6.15817282214833e-05	\\
1316.46173650568	6.69722422938509e-05	\\
1317.44051846591	6.36454871264306e-05	\\
1318.41930042614	5.72759576334118e-05	\\
1319.39808238636	6.56448204488634e-05	\\
1320.37686434659	6.01859846511576e-05	\\
1321.35564630682	6.44384528732073e-05	\\
1322.33442826704	6.71188562522012e-05	\\
1323.31321022727	6.79422396469946e-05	\\
1324.2919921875	6.42432637259983e-05	\\
1325.27077414773	6.31390995989433e-05	\\
1326.24955610795	6.95970307904517e-05	\\
1327.22833806818	6.24967438474806e-05	\\
1328.20712002841	6.08777739880321e-05	\\
1329.18590198864	6.56657687493532e-05	\\
1330.16468394886	6.65647324108511e-05	\\
1331.14346590909	6.31217760457841e-05	\\
1332.12224786932	6.87971469183353e-05	\\
1333.10102982954	6.49455772123974e-05	\\
1334.07981178977	6.75972664488272e-05	\\
1335.05859375	6.51186534943368e-05	\\
1336.03737571023	6.54025154504663e-05	\\
1337.01615767045	6.5700581343446e-05	\\
1337.99493963068	6.46381268030578e-05	\\
1338.97372159091	6.50474324389933e-05	\\
1339.95250355114	6.74215748581666e-05	\\
1340.93128551136	6.40122342668675e-05	\\
1341.91006747159	6.88432829508767e-05	\\
1342.88884943182	6.56757061137347e-05	\\
1343.86763139204	6.26322158483384e-05	\\
1344.84641335227	6.71499933896364e-05	\\
1345.8251953125	6.40907159183387e-05	\\
1346.80397727273	6.35450410175194e-05	\\
1347.78275923295	7.28926619557933e-05	\\
1348.76154119318	6.55230297027775e-05	\\
1349.74032315341	6.17185967198524e-05	\\
1350.71910511364	6.83229445965428e-05	\\
1351.69788707386	6.5385626961678e-05	\\
1352.67666903409	6.763412077611e-05	\\
1353.65545099432	6.08781967663705e-05	\\
1354.63423295454	5.87928913441036e-05	\\
1355.61301491477	7.17812400849058e-05	\\
1356.591796875	6.35433285396126e-05	\\
1357.57057883523	5.89289426300597e-05	\\
1358.54936079545	6.34186167093051e-05	\\
1359.52814275568	6.4504001598495e-05	\\
1360.50692471591	6.57005991647192e-05	\\
1361.48570667614	6.3131669129007e-05	\\
1362.46448863636	6.37715667678839e-05	\\
1363.44327059659	6.32832335420136e-05	\\
1364.42205255682	6.50138135980226e-05	\\
1365.40083451704	6.24309323478924e-05	\\
1366.37961647727	6.07443873287732e-05	\\
1367.3583984375	6.41477937645442e-05	\\
1368.33718039773	6.35777211305345e-05	\\
1369.31596235795	6.13924443997402e-05	\\
1370.29474431818	6.17059034207161e-05	\\
1371.27352627841	6.21175202455477e-05	\\
1372.25230823864	6.08675219671737e-05	\\
1373.23109019886	5.96685359117741e-05	\\
1374.20987215909	6.24229441127981e-05	\\
1375.18865411932	6.34573410913566e-05	\\
1376.16743607954	6.32972419414464e-05	\\
1377.14621803977	6.44788766094737e-05	\\
1378.125	6.14710623197641e-05	\\
1379.10378196023	5.86420841483599e-05	\\
1380.08256392045	6.52115845801984e-05	\\
1381.06134588068	6.23724408665018e-05	\\
1382.04012784091	5.80343873148033e-05	\\
1383.01890980114	6.53690147599157e-05	\\
1383.99769176136	5.67303776799414e-05	\\
1384.97647372159	6.59319134441482e-05	\\
1385.95525568182	6.82082988786691e-05	\\
1386.93403764204	6.26851258819205e-05	\\
1387.91281960227	6.23490302212643e-05	\\
1388.8916015625	6.61822157043821e-05	\\
1389.87038352273	5.61332773058492e-05	\\
1390.84916548295	7.10324427614412e-05	\\
1391.82794744318	5.98801485199861e-05	\\
1392.80672940341	6.48357747663121e-05	\\
1393.78551136364	6.34156904377714e-05	\\
1394.76429332386	6.71544160892113e-05	\\
1395.74307528409	6.02381591349187e-05	\\
1396.72185724432	6.41523958212766e-05	\\
1397.70063920454	6.01825226882872e-05	\\
1398.67942116477	6.48632503778453e-05	\\
1399.658203125	6.49188290980251e-05	\\
1400.63698508523	6.45137230574216e-05	\\
1401.61576704545	6.70444427388617e-05	\\
1402.59454900568	6.07741624739606e-05	\\
1403.57333096591	6.61120935192129e-05	\\
1404.55211292614	6.54266641013372e-05	\\
1405.53089488636	7.04744553948287e-05	\\
1406.50967684659	6.53732809175314e-05	\\
1407.48845880682	6.52287242325247e-05	\\
1408.46724076704	6.74659878993683e-05	\\
1409.44602272727	6.46649229175297e-05	\\
1410.4248046875	6.45814622937726e-05	\\
1411.40358664773	6.32247864401085e-05	\\
1412.38236860795	7.07703179164837e-05	\\
1413.36115056818	6.72171692319184e-05	\\
1414.33993252841	6.74308421961107e-05	\\
1415.31871448864	7.20152859453534e-05	\\
1416.29749644886	6.8916397523062e-05	\\
1417.27627840909	6.85169641358063e-05	\\
1418.25506036932	7.45793890085337e-05	\\
1419.23384232954	6.59260957633283e-05	\\
1420.21262428977	6.98773994352628e-05	\\
1421.19140625	7.51201137026657e-05	\\
1422.17018821023	6.82931991722567e-05	\\
1423.14897017045	7.44097297764074e-05	\\
1424.12775213068	6.43839766114899e-05	\\
1425.10653409091	7.25608861951478e-05	\\
1426.08531605114	7.48718767410704e-05	\\
1427.06409801136	6.71154519665058e-05	\\
1428.04287997159	7.47471909710281e-05	\\
1429.02166193182	7.06756926800984e-05	\\
1430.00044389204	7.44353169603084e-05	\\
1430.97922585227	7.35174459530085e-05	\\
1431.9580078125	7.12144548005141e-05	\\
1432.93678977273	7.15677445192624e-05	\\
1433.91557173295	7.10149049935035e-05	\\
1434.89435369318	7.15423752921876e-05	\\
1435.87313565341	6.93995650677204e-05	\\
1436.85191761364	7.25279295655012e-05	\\
1437.83069957386	7.44908253839465e-05	\\
1438.80948153409	7.17256124430439e-05	\\
1439.78826349432	7.29847985946387e-05	\\
1440.76704545454	7.1946348162861e-05	\\
1441.74582741477	7.18629667215879e-05	\\
1442.724609375	7.04991101651585e-05	\\
1443.70339133523	7.18716611572723e-05	\\
1444.68217329545	7.39300562870543e-05	\\
1445.66095525568	7.66621513414671e-05	\\
1446.63973721591	7.22721303621739e-05	\\
1447.61851917614	7.41026233991079e-05	\\
1448.59730113636	7.54853148915668e-05	\\
1449.57608309659	7.34647969891376e-05	\\
1450.55486505682	7.72879172803704e-05	\\
1451.53364701704	7.74240425434013e-05	\\
1452.51242897727	7.05543502119789e-05	\\
1453.4912109375	7.30918177857254e-05	\\
1454.46999289773	7.674545358359e-05	\\
1455.44877485795	8.08351906675585e-05	\\
1456.42755681818	7.64680420061518e-05	\\
1457.40633877841	7.93309650755616e-05	\\
1458.38512073864	7.3782343463816e-05	\\
1459.36390269886	7.5314807438369e-05	\\
1460.34268465909	8.22914338647654e-05	\\
1461.32146661932	7.74993512157753e-05	\\
1462.30024857954	7.57847478299114e-05	\\
1463.27903053977	8.15994938432249e-05	\\
1464.2578125	8.12999341417002e-05	\\
1465.23659446023	7.83400907556594e-05	\\
1466.21537642045	7.84178251754897e-05	\\
1467.19415838068	8.3267592345294e-05	\\
1468.17294034091	8.39667410894041e-05	\\
1469.15172230114	8.34726000497283e-05	\\
1470.13050426136	8.31377120067996e-05	\\
1471.10928622159	8.21147114010721e-05	\\
1472.08806818182	8.56077173984877e-05	\\
1473.06685014204	8.78382972363338e-05	\\
1474.04563210227	8.5701422648172e-05	\\
1475.0244140625	9.07400130256613e-05	\\
1476.00319602273	8.86293870202929e-05	\\
1476.98197798295	8.84558337018664e-05	\\
1477.96075994318	9.08340460228802e-05	\\
1478.93954190341	8.55890111567262e-05	\\
1479.91832386364	9.49064958533135e-05	\\
1480.89710582386	9.42237425969454e-05	\\
1481.87588778409	8.87047197973062e-05	\\
1482.85466974432	9.64510545316348e-05	\\
1483.83345170454	9.18997125450044e-05	\\
1484.81223366477	9.80783633458233e-05	\\
1485.791015625	9.55937244699063e-05	\\
1486.76979758523	9.87706724220429e-05	\\
1487.74857954545	9.5103891108904e-05	\\
1488.72736150568	0.00010248177283204	\\
1489.70614346591	9.93101505849588e-05	\\
1490.68492542614	0.000100202331334853	\\
1491.66370738636	0.000100743582007303	\\
1492.64248934659	0.000102310369421381	\\
1493.62127130682	0.000102679674886555	\\
1494.60005326704	0.000105308961988874	\\
1495.57883522727	0.000104153128073522	\\
1496.5576171875	0.000104706837403357	\\
1497.53639914773	0.000101886394694159	\\
1498.51518110795	0.000103800037086757	\\
1499.49396306818	0.000107958945505198	\\
1500.47274502841	9.63241460138151e-05	\\
1501.45152698864	0.000105861076427357	\\
1502.43030894886	0.000104817229582934	\\
1503.40909090909	0.000112053555993615	\\
1504.38787286932	0.000101739047356256	\\
1505.36665482954	0.000113777424541459	\\
1506.34543678977	0.000104458825813365	\\
1507.32421875	0.000108743030561848	\\
1508.30300071023	0.000115148955841686	\\
1509.28178267045	0.000114729005725613	\\
1510.26056463068	0.000119027472294771	\\
1511.23934659091	0.000120694273501039	\\
1512.21812855114	0.000110792650242792	\\
1513.19691051136	0.000117556444988503	\\
1514.17569247159	0.000113452432220041	\\
1515.15447443182	0.000113485478296489	\\
1516.13325639204	0.00011610986525931	\\
1517.11203835227	0.000118477227223214	\\
1518.0908203125	0.000116619255184793	\\
1519.06960227273	0.000123223649185601	\\
1520.04838423295	0.000119005572182875	\\
1521.02716619318	0.00012221006558806	\\
1522.00594815341	0.00011900136350701	\\
1522.98473011364	0.000121174845726502	\\
1523.96351207386	0.000122652722820418	\\
1524.94229403409	0.000122285491659935	\\
1525.92107599432	0.000123091130990685	\\
1526.89985795454	0.000115555079501241	\\
1527.87863991477	0.000118705304740787	\\
1528.857421875	0.000120144308699213	\\
1529.83620383523	0.000113238660747362	\\
1530.81498579545	0.000122060710629784	\\
1531.79376775568	0.000111640338755447	\\
1532.77254971591	0.000116618069022677	\\
1533.75133167614	0.000115646404741796	\\
1534.73011363636	0.000115385195148601	\\
1535.70889559659	0.000116083552177763	\\
1536.68767755682	0.000101988982652503	\\
1537.66645951704	0.000117120230382186	\\
1538.64524147727	0.000109772914358072	\\
1539.6240234375	0.000117372697552687	\\
1540.60280539773	0.000117817931666958	\\
1541.58158735795	0.00010172219948162	\\
1542.56036931818	0.000112317291783999	\\
1543.53915127841	0.000102411822258906	\\
1544.51793323864	0.000114911985508392	\\
1545.49671519886	0.000110353183339505	\\
1546.47549715909	0.000107370026121851	\\
1547.45427911932	0.000117459283334707	\\
1548.43306107954	0.000102357182413687	\\
1549.41184303977	0.000111531347943539	\\
1550.390625	0.000111271277425035	\\
1551.36940696023	0.000109454966796774	\\
1552.34818892045	0.000110950202822576	\\
1553.32697088068	0.000106786582885036	\\
1554.30575284091	0.000114606391656174	\\
1555.28453480114	0.000111949849464752	\\
1556.26331676136	0.000113794103760025	\\
1557.24209872159	0.000115027069386629	\\
1558.22088068182	0.00011163561810863	\\
1559.19966264204	0.000114775539720416	\\
1560.17844460227	0.00010811848563983	\\
1561.1572265625	0.000114653633476709	\\
1562.13600852273	0.000119731906957325	\\
1563.11479048295	0.000113377065860572	\\
1564.09357244318	0.000115862730429121	\\
1565.07235440341	0.00011122994969926	\\
1566.05113636364	0.000116964029413109	\\
1567.02991832386	0.000115996838696029	\\
1568.00870028409	0.000115899494293144	\\
1568.98748224432	0.000121706393174293	\\
1569.96626420454	0.000116570912803038	\\
1570.94504616477	0.000116320977339299	\\
1571.923828125	0.000117814715141252	\\
1572.90261008523	0.000118254899668895	\\
1573.88139204545	0.000116400915588462	\\
1574.86017400568	0.000117384047106676	\\
1575.83895596591	0.000115962299555546	\\
1576.81773792614	0.000113890143463188	\\
1577.79651988636	0.000114067126785158	\\
1578.77530184659	0.000113350960118912	\\
1579.75408380682	0.000108058506244907	\\
1580.73286576704	0.000114050803335643	\\
1581.71164772727	0.000110455755326287	\\
1582.6904296875	0.000111312164661998	\\
1583.66921164773	0.0001138792811711	\\
1584.64799360795	0.000105964531298322	\\
1585.62677556818	0.000107332765405773	\\
1586.60555752841	0.000107903121604873	\\
1587.58433948864	0.000110997622448292	\\
1588.56312144886	0.000107181106243943	\\
1589.54190340909	0.000110698100692552	\\
1590.52068536932	9.96935086774501e-05	\\
1591.49946732954	0.000112629586856585	\\
1592.47824928977	0.000109049671921997	\\
1593.45703125	0.000110575605607309	\\
1594.43581321023	0.000105519350018764	\\
1595.41459517045	0.000106972911044518	\\
1596.39337713068	0.000102140914293472	\\
1597.37215909091	0.000110517978438457	\\
1598.35094105114	9.91810669737604e-05	\\
1599.32972301136	0.000108854212912692	\\
1600.30850497159	9.71694882575627e-05	\\
1601.28728693182	0.00010271214470255	\\
1602.26606889204	0.000104947699918472	\\
1603.24485085227	9.84872950627665e-05	\\
1604.2236328125	0.000107355548490679	\\
1605.20241477273	9.93360833451968e-05	\\
1606.18119673295	9.8675121734299e-05	\\
1607.15997869318	0.00010468027476166	\\
1608.13876065341	0.00010590598209015	\\
1609.11754261364	9.99740116831253e-05	\\
1610.09632457386	0.000106096070844662	\\
1611.07510653409	0.000102994530419474	\\
1612.05388849432	0.000102201771109448	\\
1613.03267045454	0.000100628200660164	\\
1614.01145241477	9.79408060898293e-05	\\
1614.990234375	0.00010103764239282	\\
1615.96901633523	9.83973470665733e-05	\\
1616.94779829545	9.59880608905145e-05	\\
1617.92658025568	0.000103387223724518	\\
1618.90536221591	9.68677193561749e-05	\\
1619.88414417614	0.000101435242632236	\\
1620.86292613636	9.95386072633876e-05	\\
1621.84170809659	0.000101552571137258	\\
1622.82049005682	0.000101401952098663	\\
1623.79927201704	0.000100904389794152	\\
1624.77805397727	9.38841183581585e-05	\\
1625.7568359375	0.000103916156828444	\\
1626.73561789773	9.57990315261617e-05	\\
1627.71439985795	9.93271203979592e-05	\\
1628.69318181818	0.000105231321808819	\\
1629.67196377841	9.81433831343632e-05	\\
1630.65074573864	0.000104941249404166	\\
1631.62952769886	0.000102448378484961	\\
1632.60830965909	0.000100662051070447	\\
1633.58709161932	0.000110002562077453	\\
1634.56587357954	0.000103950393951222	\\
1635.54465553977	0.000106984024079666	\\
1636.5234375	9.79085000495474e-05	\\
1637.50221946023	0.000101068327774741	\\
1638.48100142045	0.000106073820785327	\\
1639.45978338068	0.000105143312794338	\\
1640.43856534091	0.000106943198875711	\\
1641.41734730114	0.000103781894405112	\\
1642.39612926136	0.00010953478061216	\\
1643.37491122159	0.000109843349643415	\\
1644.35369318182	0.000109267580638905	\\
1645.33247514204	0.000105410776788084	\\
1646.31125710227	9.99886169842141e-05	\\
1647.2900390625	0.000112086798207778	\\
1648.26882102273	0.000109296556217961	\\
1649.24760298295	0.000112046124859643	\\
1650.22638494318	0.000111466854908583	\\
1651.20516690341	0.000102307701315261	\\
1652.18394886364	0.000113252559482829	\\
1653.16273082386	0.000104486635126206	\\
1654.14151278409	0.000111050101717009	\\
1655.12029474432	0.000111822407395743	\\
1656.09907670454	0.000109059625352328	\\
1657.07785866477	0.000119475709643919	\\
1658.056640625	0.000108459261376815	\\
1659.03542258523	0.000114603531168484	\\
1660.01420454545	0.000112689218072248	\\
1660.99298650568	0.000109506478810979	\\
1661.97176846591	0.000119635196659269	\\
1662.95055042614	0.000108241869226818	\\
1663.92933238636	0.000119300578606737	\\
1664.90811434659	0.000111225928065881	\\
1665.88689630682	0.00011203731377513	\\
1666.86567826704	0.000114761210373554	\\
1667.84446022727	0.000109238194423996	\\
1668.8232421875	0.000114314058619039	\\
1669.80202414773	0.000112870588858183	\\
1670.78080610795	0.000112943204086564	\\
1671.75958806818	0.000117791724757931	\\
1672.73837002841	0.000112665164000839	\\
1673.71715198864	0.000118389840093096	\\
1674.69593394886	0.000117871834206702	\\
1675.67471590909	0.000109626139405338	\\
1676.65349786932	0.000119025160966206	\\
1677.63227982954	0.000114761605163442	\\
1678.61106178977	0.000110566656352396	\\
1679.58984375	0.000116471904882797	\\
1680.56862571023	0.000114076100339809	\\
1681.54740767045	0.000117970929694056	\\
1682.52618963068	0.000114014819101449	\\
1683.50497159091	0.000111027815177694	\\
1684.48375355114	0.000110732014601248	\\
1685.46253551136	0.000106096707208828	\\
1686.44131747159	0.000112486055770762	\\
1687.42009943182	0.000110653136835762	\\
1688.39888139204	0.00011297187143527	\\
1689.37766335227	0.000114573376700618	\\
1690.3564453125	0.000109128914377801	\\
1691.33522727273	0.000112313798688686	\\
1692.31400923295	0.000107324864107667	\\
1693.29279119318	0.000108673536192188	\\
1694.27157315341	0.00010884867962707	\\
1695.25035511364	0.00010777470468514	\\
1696.22913707386	0.000106889756281939	\\
1697.20791903409	0.000108702575120128	\\
1698.18670099432	0.000104085033472544	\\
1699.16548295454	0.000104839833243542	\\
1700.14426491477	0.00010090811218462	\\
1701.123046875	0.000107891219917535	\\
1702.10182883523	0.000104143520820736	\\
1703.08061079545	0.000103493499507172	\\
1704.05939275568	0.000103961915429993	\\
1705.03817471591	0.000103272865818578	\\
1706.01695667614	0.000104253187728651	\\
1706.99573863636	0.000102650328903069	\\
1707.97452059659	0.000101257933417711	\\
1708.95330255682	9.97395948428966e-05	\\
1709.93208451704	0.000104484194656325	\\
1710.91086647727	9.58406411677006e-05	\\
1711.8896484375	0.000102092334548311	\\
1712.86843039773	9.91920314667752e-05	\\
1713.84721235795	0.000103912976022538	\\
1714.82599431818	0.00010050546486179	\\
1715.80477627841	0.000100327236242762	\\
1716.78355823864	0.000103678449615342	\\
1717.76234019886	9.3764390988555e-05	\\
1718.74112215909	0.000107481775085565	\\
1719.71990411932	9.44514874440857e-05	\\
1720.69868607954	9.54912355326844e-05	\\
1721.67746803977	9.52005481178417e-05	\\
1722.65625	9.5037935077077e-05	\\
1723.63503196023	9.82068810073367e-05	\\
1724.61381392045	9.84977588454822e-05	\\
1725.59259588068	8.93659360545324e-05	\\
1726.57137784091	9.8915484800154e-05	\\
1727.55015980114	8.95359415933584e-05	\\
1728.52894176136	9.27443548220071e-05	\\
1729.50772372159	9.48866715437594e-05	\\
1730.48650568182	8.71058868666556e-05	\\
1731.46528764204	9.66951527054906e-05	\\
1732.44406960227	9.72617621304728e-05	\\
1733.4228515625	8.60128913465381e-05	\\
1734.40163352273	9.43700914474884e-05	\\
1735.38041548295	8.68461842111088e-05	\\
1736.35919744318	9.44858471126303e-05	\\
1737.33797940341	9.15449593891272e-05	\\
1738.31676136364	9.25799040217914e-05	\\
1739.29554332386	9.33220394416491e-05	\\
1740.27432528409	0.000100054073688938	\\
1741.25310724432	8.12104881456606e-05	\\
1742.23188920454	9.28393785957647e-05	\\
1743.21067116477	8.75163484687299e-05	\\
1744.189453125	8.49146677552517e-05	\\
1745.16823508523	0.000102883858309732	\\
1746.14701704545	7.65166484098842e-05	\\
1747.12579900568	9.14465622051768e-05	\\
1748.10458096591	8.59964955775735e-05	\\
1749.08336292614	7.60918131230214e-05	\\
1750.06214488636	9.71927239222502e-05	\\
1751.04092684659	7.59719019799725e-05	\\
1752.01970880682	9.19596803489205e-05	\\
1752.99849076704	8.74363053106705e-05	\\
1753.97727272727	7.99912389307336e-05	\\
1754.9560546875	8.88209698743684e-05	\\
1755.93483664773	8.18412266825067e-05	\\
1756.91361860795	8.64066741412109e-05	\\
1757.89240056818	8.42403096863876e-05	\\
1758.87118252841	8.39993373596894e-05	\\
1759.84996448864	7.97400840030714e-05	\\
1760.82874644886	9.24244276022106e-05	\\
1761.80752840909	8.07428415714965e-05	\\
1762.78631036932	8.46602753921796e-05	\\
1763.76509232954	8.79226160177759e-05	\\
1764.74387428977	7.75798861371638e-05	\\
1765.72265625	8.88798736038016e-05	\\
1766.70143821023	7.84732726968179e-05	\\
1767.68022017045	9.14587320300602e-05	\\
1768.65900213068	8.17338574850082e-05	\\
1769.63778409091	8.17073753148636e-05	\\
1770.61656605114	8.40054732250743e-05	\\
1771.59534801136	7.85428021560494e-05	\\
1772.57412997159	8.25993707873265e-05	\\
1773.55291193182	8.34143688723795e-05	\\
1774.53169389204	7.90614283634624e-05	\\
1775.51047585227	8.60024959527438e-05	\\
1776.4892578125	7.33847108132582e-05	\\
1777.46803977273	8.70905410353223e-05	\\
1778.44682173295	7.66456030272768e-05	\\
1779.42560369318	8.23787371174378e-05	\\
1780.40438565341	8.1654704933187e-05	\\
1781.38316761364	7.81951116831714e-05	\\
1782.36194957386	8.10939294426886e-05	\\
1783.34073153409	7.75665227238093e-05	\\
1784.31951349432	8.25399167620855e-05	\\
1785.29829545454	8.27557438343223e-05	\\
1786.27707741477	7.97061294608944e-05	\\
1787.255859375	8.18705699683388e-05	\\
1788.23464133523	7.96981907928936e-05	\\
1789.21342329545	7.92718264413134e-05	\\
1790.19220525568	8.20806611436575e-05	\\
1791.17098721591	7.70779226921006e-05	\\
1792.14976917614	7.96064683050768e-05	\\
1793.12855113636	7.85709863792494e-05	\\
1794.10733309659	7.92096770535643e-05	\\
1795.08611505682	8.0468363221945e-05	\\
1796.06489701704	8.02389902021391e-05	\\
1797.04367897727	7.81387092470699e-05	\\
1798.0224609375	7.58166099256122e-05	\\
1799.00124289773	7.87414998582346e-05	\\
1799.98002485795	8.15677270196445e-05	\\
1800.95880681818	7.97684874597152e-05	\\
1801.93758877841	7.92273031211842e-05	\\
1802.91637073864	8.14745446460689e-05	\\
1803.89515269886	7.60883659601609e-05	\\
1804.87393465909	7.55831863864765e-05	\\
1805.85271661932	8.07556718836621e-05	\\
1806.83149857954	7.57245911078879e-05	\\
1807.81028053977	8.17511660833686e-05	\\
1808.7890625	7.78418865578471e-05	\\
1809.76784446023	8.07983866442329e-05	\\
1810.74662642045	7.63653049876728e-05	\\
1811.72540838068	7.95839421184105e-05	\\
1812.70419034091	7.83487993210411e-05	\\
1813.68297230114	7.72902795663599e-05	\\
1814.66175426136	7.6575473853702e-05	\\
1815.64053622159	7.97067785162665e-05	\\
1816.61931818182	7.62681953317253e-05	\\
1817.59810014204	7.54763308490672e-05	\\
1818.57688210227	7.41424820374005e-05	\\
1819.5556640625	7.33997748355558e-05	\\
1820.53444602273	7.50043043978092e-05	\\
1821.51322798295	7.61977042633063e-05	\\
1822.49200994318	7.44733973248138e-05	\\
1823.47079190341	7.43980010095789e-05	\\
1824.44957386364	6.99998554260749e-05	\\
1825.42835582386	7.43323773675317e-05	\\
1826.40713778409	6.98252475200387e-05	\\
1827.38591974432	6.78796757445884e-05	\\
1828.36470170454	7.20322207195167e-05	\\
1829.34348366477	7.1996441207218e-05	\\
1830.322265625	7.43269713175628e-05	\\
1831.30104758523	6.88385803903191e-05	\\
1832.27982954545	7.30621322756592e-05	\\
1833.25861150568	7.13231762241911e-05	\\
1834.23739346591	7.31331304910379e-05	\\
1835.21617542614	7.0048675204497e-05	\\
1836.19495738636	6.76636511642795e-05	\\
1837.17373934659	7.17507457714535e-05	\\
1838.15252130682	7.06568716381401e-05	\\
1839.13130326704	6.79001949466957e-05	\\
1840.11008522727	7.1174907103169e-05	\\
1841.0888671875	7.15678656377683e-05	\\
1842.06764914773	7.13849451239681e-05	\\
1843.04643110795	6.95383833545663e-05	\\
1844.02521306818	6.55583466905138e-05	\\
1845.00399502841	6.73145703777201e-05	\\
1845.98277698864	7.23119404103782e-05	\\
1846.96155894886	6.5924860299285e-05	\\
1847.94034090909	6.72817226141719e-05	\\
1848.91912286932	6.97816185910157e-05	\\
1849.89790482954	6.6797804786327e-05	\\
1850.87668678977	7.11232956528646e-05	\\
1851.85546875	6.61762351726076e-05	\\
1852.83425071023	6.41335680911334e-05	\\
1853.81303267045	6.80415067961354e-05	\\
1854.79181463068	6.45764976632524e-05	\\
1855.77059659091	6.35645110993203e-05	\\
1856.74937855114	6.94564850330023e-05	\\
1857.72816051136	5.92471527747254e-05	\\
1858.70694247159	6.41961074570486e-05	\\
1859.68572443182	6.39343614045223e-05	\\
1860.66450639204	6.23848220523428e-05	\\
1861.64328835227	6.36397601538391e-05	\\
1862.6220703125	6.25703135983833e-05	\\
1863.60085227273	6.43259787022565e-05	\\
1864.57963423295	6.63667294029433e-05	\\
1865.55841619318	6.50034489596605e-05	\\
1866.53719815341	6.50379747396635e-05	\\
1867.51598011364	5.54548621395082e-05	\\
1868.49476207386	6.40867646299348e-05	\\
1869.47354403409	6.1389464464702e-05	\\
1870.45232599432	6.16083660151943e-05	\\
1871.43110795454	6.43485860512057e-05	\\
1872.40988991477	5.76830128681986e-05	\\
1873.388671875	6.38445534250981e-05	\\
1874.36745383523	5.97137345066336e-05	\\
1875.34623579545	6.19739866660165e-05	\\
1876.32501775568	6.28188093397921e-05	\\
1877.30379971591	6.42603540653948e-05	\\
1878.28258167614	6.45062145214487e-05	\\
1879.26136363636	5.66973372540517e-05	\\
1880.24014559659	5.94608247915881e-05	\\
1881.21892755682	5.92347531614797e-05	\\
1882.19770951704	5.82477797189117e-05	\\
1883.17649147727	5.76704120242966e-05	\\
1884.1552734375	6.03709775634363e-05	\\
1885.13405539773	5.90914482902265e-05	\\
1886.11283735795	5.64336364954048e-05	\\
1887.09161931818	5.72134135942022e-05	\\
1888.07040127841	5.87040574488291e-05	\\
1889.04918323864	5.49077324743913e-05	\\
1890.02796519886	5.64467378732759e-05	\\
1891.00674715909	5.4249787313813e-05	\\
1891.98552911932	5.36868269448146e-05	\\
1892.96431107954	5.72630439463298e-05	\\
1893.94309303977	5.08388873557546e-05	\\
1894.921875	5.46472321501904e-05	\\
1895.90065696023	5.31693383488575e-05	\\
1896.87943892045	4.89851838310027e-05	\\
1897.85822088068	5.44978795745926e-05	\\
1898.83700284091	4.98517739390293e-05	\\
1899.81578480114	5.75998435338664e-05	\\
1900.79456676136	5.38345330311366e-05	\\
1901.77334872159	5.424446568604e-05	\\
1902.75213068182	5.17470728341799e-05	\\
1903.73091264204	5.19706891885212e-05	\\
1904.70969460227	5.19656590549529e-05	\\
1905.6884765625	4.93984958507744e-05	\\
1906.66725852273	5.08738424616318e-05	\\
1907.64604048295	5.63773961142622e-05	\\
1908.62482244318	5.01720555580318e-05	\\
1909.60360440341	5.05389569145054e-05	\\
1910.58238636364	5.00104112523317e-05	\\
1911.56116832386	5.04879407541922e-05	\\
1912.53995028409	5.70924371269171e-05	\\
1913.51873224432	5.05340013375903e-05	\\
1914.49751420454	5.37420599425086e-05	\\
1915.47629616477	5.24308426023866e-05	\\
1916.455078125	5.37911124935694e-05	\\
1917.43386008523	5.23214379018303e-05	\\
1918.41264204545	5.15122877305239e-05	\\
1919.39142400568	5.21605666784246e-05	\\
1920.37020596591	5.04956927259365e-05	\\
1921.34898792614	5.38510005717516e-05	\\
1922.32776988636	5.27476521670546e-05	\\
1923.30655184659	5.40903591737497e-05	\\
1924.28533380682	5.41898345657032e-05	\\
1925.26411576704	4.94868446162333e-05	\\
1926.24289772727	4.97783297499793e-05	\\
1927.2216796875	5.5624774015676e-05	\\
1928.20046164773	5.15338297467112e-05	\\
1929.17924360795	5.52248874033302e-05	\\
1930.15802556818	5.39559904858485e-05	\\
1931.13680752841	5.31662962272553e-05	\\
1932.11558948864	5.76463526148785e-05	\\
1933.09437144886	5.11101985877757e-05	\\
1934.07315340909	5.06441332447232e-05	\\
1935.05193536932	5.37423229403369e-05	\\
1936.03071732954	5.24137121833408e-05	\\
1937.00949928977	5.12265146696942e-05	\\
1937.98828125	5.81701992680689e-05	\\
1938.96706321023	5.45992873981334e-05	\\
1939.94584517045	5.71248193849103e-05	\\
1940.92462713068	5.37864037331661e-05	\\
1941.90340909091	5.46809954511213e-05	\\
1942.88219105114	5.67253334639094e-05	\\
1943.86097301136	5.44418895997764e-05	\\
1944.83975497159	5.79292561756387e-05	\\
1945.81853693182	5.56796983085735e-05	\\
1946.79731889204	5.42659728123409e-05	\\
1947.77610085227	5.77560461080472e-05	\\
1948.7548828125	5.43578215700279e-05	\\
1949.73366477273	5.48117212622323e-05	\\
1950.71244673295	5.83471386939806e-05	\\
1951.69122869318	5.26514304160461e-05	\\
1952.67001065341	5.59453815704224e-05	\\
1953.64879261364	5.79571275992206e-05	\\
1954.62757457386	5.54541273636918e-05	\\
1955.60635653409	5.68640049703634e-05	\\
1956.58513849432	5.82733387771915e-05	\\
1957.56392045454	5.47249149210386e-05	\\
1958.54270241477	5.4967968077828e-05	\\
1959.521484375	5.99964569320623e-05	\\
1960.50026633523	5.55433389079216e-05	\\
1961.47904829545	5.71546818593453e-05	\\
1962.45783025568	6.1177645550661e-05	\\
1963.43661221591	5.79313586264886e-05	\\
1964.41539417614	5.58887544851984e-05	\\
1965.39417613636	6.14139819152746e-05	\\
1966.37295809659	5.53763655591351e-05	\\
1967.35174005682	5.76527560660069e-05	\\
1968.33052201704	5.93588038750038e-05	\\
1969.30930397727	5.48109554562318e-05	\\
1970.2880859375	5.91257533915014e-05	\\
1971.26686789773	5.97980863377099e-05	\\
1972.24564985795	5.71889860586288e-05	\\
1973.22443181818	6.00253497169692e-05	\\
1974.20321377841	6.02663506712593e-05	\\
1975.18199573864	5.58373021463525e-05	\\
1976.16077769886	6.170793092055e-05	\\
1977.13955965909	5.8795736266134e-05	\\
1978.11834161932	5.77084719807117e-05	\\
1979.09712357954	6.18551123373764e-05	\\
1980.07590553977	5.81878381909325e-05	\\
1981.0546875	5.73858399948661e-05	\\
1982.03346946023	6.14589115082668e-05	\\
1983.01225142045	5.79836502542568e-05	\\
1983.99103338068	6.01383831091874e-05	\\
1984.96981534091	5.95205049484844e-05	\\
1985.94859730114	6.10595663258426e-05	\\
1986.92737926136	6.15405665022123e-05	\\
1987.90616122159	6.35197998460964e-05	\\
1988.88494318182	5.93953335360826e-05	\\
1989.86372514204	6.21037129393651e-05	\\
1990.84250710227	6.21265188151232e-05	\\
1991.8212890625	6.03291787655584e-05	\\
1992.80007102273	6.46468652169734e-05	\\
1993.77885298295	5.89952582267569e-05	\\
1994.75763494318	6.031741808771e-05	\\
1995.73641690341	6.22065470607571e-05	\\
1996.71519886364	6.14768237064682e-05	\\
1997.69398082386	5.79528191532625e-05	\\
1998.67276278409	5.74844807113399e-05	\\
1999.65154474432	5.8436829348536e-05	\\
2000.63032670454	5.60166655470374e-05	\\
2001.60910866477	6.38780719635436e-05	\\
2002.587890625	6.06659369818632e-05	\\
2003.56667258523	5.63613678719514e-05	\\
2004.54545454545	5.892909277581e-05	\\
2005.52423650568	5.80721704688347e-05	\\
2006.50301846591	5.92612847610127e-05	\\
2007.48180042614	5.83274887504255e-05	\\
2008.46058238636	5.87383979539598e-05	\\
2009.43936434659	5.94733149634942e-05	\\
2010.41814630682	5.86038896949792e-05	\\
2011.39692826704	5.78123164883512e-05	\\
2012.37571022727	6.14115363059159e-05	\\
2013.3544921875	5.93179457666343e-05	\\
2014.33327414773	5.77192157286447e-05	\\
2015.31205610795	5.84327308015894e-05	\\
2016.29083806818	5.98979528347406e-05	\\
2017.26962002841	5.77429955439534e-05	\\
2018.24840198864	5.98358931599768e-05	\\
2019.22718394886	6.31560041671061e-05	\\
2020.20596590909	6.24467012673247e-05	\\
2021.18474786932	6.06435281779398e-05	\\
2022.16352982954	5.52763627531626e-05	\\
2023.14231178977	5.63809104560075e-05	\\
2024.12109375	5.57896261254617e-05	\\
2025.09987571023	5.5581830301891e-05	\\
2026.07865767045	5.89112867295819e-05	\\
2027.05743963068	5.44045288709269e-05	\\
2028.03622159091	5.56085530806716e-05	\\
2029.01500355114	5.62233897320401e-05	\\
2029.99378551136	5.45169877535638e-05	\\
2030.97256747159	5.54823604318976e-05	\\
2031.95134943182	5.37294548393087e-05	\\
2032.93013139204	5.10825945264359e-05	\\
2033.90891335227	5.3128891009028e-05	\\
2034.8876953125	5.03195038926903e-05	\\
2035.86647727273	5.03455661585471e-05	\\
2036.84525923295	5.12268099331004e-05	\\
2037.82404119318	5.00343141549991e-05	\\
2038.80282315341	5.14800442511911e-05	\\
2039.78160511364	4.73176010832827e-05	\\
2040.76038707386	4.72230580586486e-05	\\
2041.73916903409	4.45365540883945e-05	\\
2042.71795099432	4.6980860522678e-05	\\
2043.69673295454	4.58841989385647e-05	\\
2044.67551491477	4.55882414710021e-05	\\
2045.654296875	4.48848554654443e-05	\\
2046.63307883523	4.74245569099291e-05	\\
2047.61186079545	4.09110113779197e-05	\\
2048.59064275568	4.27113728942239e-05	\\
2049.56942471591	4.3313984456285e-05	\\
2050.54820667614	3.83862471712088e-05	\\
2051.52698863636	3.73675824451777e-05	\\
2052.50577059659	3.78962216847074e-05	\\
2053.48455255682	3.99367122456472e-05	\\
2054.46333451704	3.84945269535284e-05	\\
2055.44211647727	3.62741121252172e-05	\\
2056.4208984375	3.48936671842307e-05	\\
2057.39968039773	3.35562309757452e-05	\\
2058.37846235795	3.45103279001185e-05	\\
2059.35724431818	3.55269442783698e-05	\\
2060.33602627841	3.24468334306063e-05	\\
2061.31480823864	3.28020855758757e-05	\\
2062.29359019886	3.1447692799479e-05	\\
2063.27237215909	3.16422594604723e-05	\\
2064.25115411932	3.19984114506442e-05	\\
2065.22993607954	3.11847886727666e-05	\\
2066.20871803977	3.08079462648034e-05	\\
2067.1875	2.91989612740849e-05	\\
2068.16628196023	3.12126318750838e-05	\\
2069.14506392045	2.97126905347779e-05	\\
2070.12384588068	2.89111619738829e-05	\\
2071.10262784091	2.94996732954359e-05	\\
2072.08140980114	2.57808913898505e-05	\\
2073.06019176136	2.87379465039631e-05	\\
2074.03897372159	2.58480236665329e-05	\\
2075.01775568182	2.74820262832285e-05	\\
2075.99653764204	2.51152204289248e-05	\\
2076.97531960227	2.61757144210691e-05	\\
2077.9541015625	2.98538615960373e-05	\\
2078.93288352273	2.50270229632934e-05	\\
2079.91166548295	2.70565341665364e-05	\\
2080.89044744318	2.56001551320162e-05	\\
2081.86922940341	2.70455971917524e-05	\\
2082.84801136364	2.64191929019506e-05	\\
2083.82679332386	2.79442931407061e-05	\\
2084.80557528409	2.7159935000022e-05	\\
2085.78435724432	2.97019918543447e-05	\\
2086.76313920454	2.64251589423545e-05	\\
2087.74192116477	2.92504514578448e-05	\\
2088.720703125	2.64051697698505e-05	\\
2089.69948508523	2.65305493618822e-05	\\
2090.67826704545	2.82073173089332e-05	\\
2091.65704900568	2.38977270733898e-05	\\
2092.63583096591	3.19805113925382e-05	\\
2093.61461292614	2.79429888234874e-05	\\
2094.59339488636	3.15664492372219e-05	\\
2095.57217684659	3.28215005276637e-05	\\
2096.55095880682	2.90641463938762e-05	\\
2097.52974076704	3.4018735094998e-05	\\
2098.50852272727	3.12130412401767e-05	\\
2099.4873046875	3.11666568709141e-05	\\
2100.46608664773	3.43895242248518e-05	\\
2101.44486860795	3.37785930567241e-05	\\
2102.42365056818	3.44312912432514e-05	\\
2103.40243252841	3.45479783117417e-05	\\
2104.38121448864	3.64772670598038e-05	\\
2105.35999644886	3.7862500783937e-05	\\
2106.33877840909	3.70255806118348e-05	\\
2107.31756036932	3.46218911683196e-05	\\
2108.29634232954	4.02309984623695e-05	\\
2109.27512428977	3.59209257467907e-05	\\
2110.25390625	3.72092914227697e-05	\\
2111.23268821023	3.7231332782545e-05	\\
2112.21147017045	4.02872942703493e-05	\\
2113.19025213068	3.93822256786854e-05	\\
2114.16903409091	3.94979751109026e-05	\\
2115.14781605114	3.89934603659363e-05	\\
2116.12659801136	4.10856676227557e-05	\\
2117.10537997159	4.17114849677277e-05	\\
2118.08416193182	3.94782563866214e-05	\\
2119.06294389204	4.26383823640281e-05	\\
2120.04172585227	4.30833820051657e-05	\\
2121.0205078125	4.42557314037138e-05	\\
2121.99928977273	4.03981640022247e-05	\\
2122.97807173295	4.32852726458951e-05	\\
2123.95685369318	3.9346936590026e-05	\\
2124.93563565341	4.16785208528467e-05	\\
2125.91441761364	4.23694793906412e-05	\\
2126.89319957386	4.13259436366393e-05	\\
2127.87198153409	4.55727386714771e-05	\\
2128.85076349432	4.03681118914028e-05	\\
2129.82954545454	4.20308454643921e-05	\\
2130.80832741477	4.16283719003127e-05	\\
2131.787109375	4.22937146700495e-05	\\
2132.76589133523	4.33433676316529e-05	\\
2133.74467329545	4.20236457952295e-05	\\
2134.72345525568	4.27116091776427e-05	\\
2135.70223721591	4.38730182764737e-05	\\
2136.68101917614	4.1090694178331e-05	\\
2137.65980113636	4.23799665794265e-05	\\
2138.63858309659	4.45350715008516e-05	\\
2139.61736505682	4.25049910479368e-05	\\
2140.59614701704	4.00546982447072e-05	\\
2141.57492897727	3.97910957582985e-05	\\
2142.5537109375	4.16635182006823e-05	\\
2143.53249289773	4.36637119654669e-05	\\
2144.51127485795	3.91034722585739e-05	\\
2145.49005681818	4.03918276257902e-05	\\
2146.46883877841	3.87372782545279e-05	\\
2147.44762073864	4.22297146231757e-05	\\
2148.42640269886	3.55424291390207e-05	\\
2149.40518465909	4.08125304185986e-05	\\
2150.38396661932	4.05019444296505e-05	\\
2151.36274857954	3.96117888016556e-05	\\
2152.34153053977	4.28600443751128e-05	\\
2153.3203125	4.15913011637687e-05	\\
2154.29909446023	3.99942147538878e-05	\\
2155.27787642045	4.20884090677233e-05	\\
2156.25665838068	3.71847104534491e-05	\\
2157.23544034091	4.07280029535679e-05	\\
2158.21422230114	4.02761946898035e-05	\\
2159.19300426136	4.15788634425616e-05	\\
2160.17178622159	3.96380035368673e-05	\\
2161.15056818182	4.0469862386214e-05	\\
2162.12935014204	3.85425484274201e-05	\\
2163.10813210227	4.09239568407509e-05	\\
2164.0869140625	3.90289524736929e-05	\\
2165.06569602273	3.90422353966503e-05	\\
2166.04447798295	3.90674492243146e-05	\\
2167.02325994318	3.78649471718898e-05	\\
2168.00204190341	4.06763802439811e-05	\\
2168.98082386364	3.67440843343009e-05	\\
2169.95960582386	3.93611788093222e-05	\\
2170.93838778409	3.79555553475125e-05	\\
2171.91716974432	3.83979962097332e-05	\\
2172.89595170454	3.73681345617505e-05	\\
2173.87473366477	3.53203200264237e-05	\\
2174.853515625	3.76704701000296e-05	\\
2175.83229758523	3.99268909585148e-05	\\
2176.81107954545	3.7852926282275e-05	\\
2177.78986150568	3.61591360764395e-05	\\
2178.76864346591	3.46525910775697e-05	\\
2179.74742542614	3.83155498849791e-05	\\
2180.72620738636	3.46300885459117e-05	\\
2181.70498934659	3.42020112662883e-05	\\
2182.68377130682	3.74336018542956e-05	\\
2183.66255326704	3.53697921681193e-05	\\
2184.64133522727	3.60938066612834e-05	\\
2185.6201171875	3.35290684127887e-05	\\
2186.59889914773	3.42753808154873e-05	\\
2187.57768110795	3.2689825028525e-05	\\
2188.55646306818	3.3529296458105e-05	\\
2189.53524502841	3.4617874025096e-05	\\
2190.51402698864	3.38580764302905e-05	\\
2191.49280894886	3.22652373786128e-05	\\
2192.47159090909	3.4889832218771e-05	\\
2193.45037286932	2.97262585917547e-05	\\
2194.42915482954	3.26949661704821e-05	\\
2195.40793678977	3.57949126791791e-05	\\
2196.38671875	3.09993907246873e-05	\\
2197.36550071023	3.19632315526644e-05	\\
2198.34428267045	3.22108959399466e-05	\\
2199.32306463068	3.33196528739591e-05	\\
2200.30184659091	3.4910219689192e-05	\\
2201.28062855114	3.0090408829797e-05	\\
2202.25941051136	3.41310648519779e-05	\\
2203.23819247159	3.25564806935441e-05	\\
2204.21697443182	3.70069055309267e-05	\\
2205.19575639204	3.53377759474793e-05	\\
2206.17453835227	3.69793661987253e-05	\\
2207.1533203125	4.24551499399444e-05	\\
2208.13210227273	3.46054994950262e-05	\\
2209.11088423295	3.98286942495198e-05	\\
2210.08966619318	4.13663109017958e-05	\\
2211.06844815341	3.92493516623533e-05	\\
2212.04723011364	4.2495868818441e-05	\\
2213.02601207386	4.14832554788899e-05	\\
2214.00479403409	4.13813552043399e-05	\\
2214.98357599432	4.40143006301144e-05	\\
2215.96235795454	4.76943931208463e-05	\\
2216.94113991477	4.64321556690832e-05	\\
2217.919921875	4.74913432640169e-05	\\
2218.89870383523	4.63553269872725e-05	\\
2219.87748579545	4.24377657469516e-05	\\
2220.85626775568	4.85705823800037e-05	\\
2221.83504971591	5.29090926622152e-05	\\
2222.81383167614	4.61800742587375e-05	\\
2223.79261363636	5.41281705716738e-05	\\
2224.77139559659	5.17845531904019e-05	\\
2225.75017755682	5.40544817694374e-05	\\
2226.72895951704	5.746856593939e-05	\\
2227.70774147727	5.59150306147465e-05	\\
2228.6865234375	5.60172682114588e-05	\\
2229.66530539773	5.78057071982437e-05	\\
2230.64408735795	5.78151460220498e-05	\\
2231.62286931818	6.13253614324828e-05	\\
2232.60165127841	6.15607285509755e-05	\\
2233.58043323864	6.17021511071913e-05	\\
2234.55921519886	6.42441464160232e-05	\\
2235.53799715909	6.47465590149527e-05	\\
2236.51677911932	6.45802397714666e-05	\\
2237.49556107954	6.70302092951047e-05	\\
2238.47434303977	6.7299194510181e-05	\\
2239.453125	7.00013732699108e-05	\\
2240.43190696023	6.91727432777875e-05	\\
2241.41068892045	7.15293302907277e-05	\\
2242.38947088068	7.68849182822535e-05	\\
2243.36825284091	7.5364768279774e-05	\\
2244.34703480114	7.52523353675036e-05	\\
2245.32581676136	7.85818593789201e-05	\\
2246.30459872159	7.54473791939474e-05	\\
2247.28338068182	7.86577204177471e-05	\\
2248.26216264204	7.7382615117954e-05	\\
2249.24094460227	7.7556815016935e-05	\\
2250.2197265625	7.66786681274198e-05	\\
2251.19850852273	8.02477256679081e-05	\\
2252.17729048295	8.41547723632017e-05	\\
2253.15607244318	8.18045326035234e-05	\\
2254.13485440341	8.5652377679471e-05	\\
2255.11363636364	8.06745161692588e-05	\\
2256.09241832386	9.09085422791891e-05	\\
2257.07120028409	8.52260851822284e-05	\\
2258.04998224432	8.61634846177e-05	\\
2259.02876420454	8.68549491820992e-05	\\
2260.00754616477	8.66653632236647e-05	\\
2260.986328125	8.87068897843167e-05	\\
2261.96511008523	8.66177377597151e-05	\\
2262.94389204545	9.08048078083792e-05	\\
2263.92267400568	9.0684492956791e-05	\\
2264.90145596591	9.12357274102222e-05	\\
2265.88023792614	9.08292452364732e-05	\\
2266.85901988636	9.39317575074051e-05	\\
2267.83780184659	9.23386451635896e-05	\\
2268.81658380682	9.44008438805665e-05	\\
2269.79536576704	9.612404396754e-05	\\
2270.77414772727	9.50865824891024e-05	\\
2271.7529296875	9.63516679964919e-05	\\
2272.73171164773	9.77015573594147e-05	\\
2273.71049360795	9.67605030811951e-05	\\
2274.68927556818	9.87781171205394e-05	\\
2275.66805752841	9.71073368962367e-05	\\
2276.64683948864	0.000100059655881502	\\
2277.62562144886	0.000101510931838861	\\
2278.60440340909	0.000100449937118465	\\
2279.58318536932	0.000105010275831933	\\
2280.56196732954	9.81132509638029e-05	\\
2281.54074928977	0.0001053023311173	\\
2282.51953125	0.000102829883236027	\\
2283.49831321023	0.000105034454246864	\\
2284.47709517045	0.000104744415192592	\\
2285.45587713068	0.0001028166045503	\\
2286.43465909091	0.000101497191372503	\\
2287.41344105114	0.000106391411287857	\\
2288.39222301136	0.000105246550192866	\\
2289.37100497159	0.000108328919771098	\\
2290.34978693182	0.000109071300720482	\\
2291.32856889204	0.000106995780096543	\\
2292.30735085227	0.000108617078498704	\\
2293.2861328125	0.00011082563335971	\\
2294.26491477273	0.000108732236702122	\\
2295.24369673295	0.000109296157591262	\\
2296.22247869318	0.000108953207621571	\\
2297.20126065341	0.00010901466581756	\\
2298.18004261364	0.000108020956096008	\\
2299.15882457386	0.00011275083835516	\\
2300.13760653409	0.000118022413136157	\\
2301.11638849432	0.000107290665869598	\\
2302.09517045454	0.000112404723768276	\\
2303.07395241477	0.000112527673370207	\\
2304.052734375	0.000113226490713124	\\
2305.03151633523	0.000111701455573269	\\
2306.01029829545	0.000110024684508784	\\
2306.98908025568	0.000109509963644464	\\
2307.96786221591	0.000110509758967145	\\
2308.94664417614	0.000110221982794925	\\
2309.92542613636	0.000107426271330384	\\
2310.90420809659	0.000111988870441526	\\
2311.88299005682	0.00010842781352708	\\
2312.86177201704	0.000110102074092649	\\
2313.84055397727	0.000113891384601368	\\
2314.8193359375	0.00010425743920577	\\
2315.79811789773	0.000110239350882063	\\
2316.77689985795	0.000106817685757693	\\
2317.75568181818	0.000111636571955073	\\
2318.73446377841	0.000106745821025628	\\
2319.71324573864	0.000107686473818289	\\
2320.69202769886	0.000106155448434969	\\
2321.67080965909	0.000110573855190268	\\
2322.64959161932	0.000107137946234134	\\
2323.62837357954	0.000108792483948526	\\
2324.60715553977	0.000109511304445189	\\
2325.5859375	0.000106763846343003	\\
2326.56471946023	0.000107752127439732	\\
2327.54350142045	0.000107783237375912	\\
2328.52228338068	0.000106257879277678	\\
2329.50106534091	0.000108764619224918	\\
2330.47984730114	0.000106688545836561	\\
2331.45862926136	0.000105041930962326	\\
2332.43741122159	0.000108123864014529	\\
2333.41619318182	0.000107554416873562	\\
2334.39497514204	0.000108847056580844	\\
2335.37375710227	0.000110349836129042	\\
2336.3525390625	0.000109246179190769	\\
2337.33132102273	0.000107525584993762	\\
2338.31010298295	0.000109395830435803	\\
2339.28888494318	0.000106692062747803	\\
2340.26766690341	0.000107923942088508	\\
2341.24644886364	0.000109387545941897	\\
2342.22523082386	0.000105493547583718	\\
2343.20401278409	0.000108983266443149	\\
2344.18279474432	0.00010720982600181	\\
2345.16157670454	0.000108814830098428	\\
2346.14035866477	0.000108145810340224	\\
2347.119140625	0.000107635045313906	\\
2348.09792258523	0.000108740691113261	\\
2349.07670454545	0.000107409367155235	\\
2350.05548650568	0.000105513280263527	\\
2351.03426846591	0.000106644117369604	\\
2352.01305042614	0.00010708169030753	\\
2352.99183238636	0.000106994756773183	\\
2353.97061434659	0.000107317260171859	\\
2354.94939630682	0.000106549043293782	\\
2355.92817826704	0.000107872369891817	\\
2356.90696022727	0.00010515740252328	\\
2357.8857421875	0.000106692880182263	\\
2358.86452414773	0.00010939269063889	\\
2359.84330610795	0.000107822929272914	\\
2360.82208806818	0.000109387100503383	\\
2361.80087002841	0.000106829959012814	\\
2362.77965198864	0.000106608948888012	\\
2363.75843394886	0.000108149144230037	\\
2364.73721590909	0.000103023847681962	\\
2365.71599786932	0.000107081082004404	\\
2366.69477982954	0.000104413573161113	\\
2367.67356178977	0.00010253029888383	\\
2368.65234375	0.000109583298216105	\\
2369.63112571023	0.000107311877135536	\\
2370.60990767045	0.000104245839980861	\\
2371.58868963068	0.000105728391837111	\\
2372.56747159091	0.000102768060035321	\\
2373.54625355114	0.000107102205007709	\\
2374.52503551136	0.000105203869901051	\\
2375.50381747159	0.000104806877934973	\\
2376.48259943182	0.00010464685922912	\\
2377.46138139204	0.000106817582285908	\\
2378.44016335227	0.000102023162439665	\\
2379.4189453125	0.00010631995574279	\\
2380.39772727273	0.000104939658735785	\\
2381.37650923295	0.000100893898157745	\\
2382.35529119318	0.000101895459379936	\\
2383.33407315341	0.000105730558096125	\\
2384.31285511364	0.000102022139337709	\\
2385.29163707386	0.000102861015847961	\\
2386.27041903409	0.000102160767512212	\\
2387.24920099432	0.000100325370714805	\\
2388.22798295454	9.8426124745946e-05	\\
2389.20676491477	0.000105048256569749	\\
2390.185546875	9.89785155875948e-05	\\
2391.16432883523	9.99076375215384e-05	\\
2392.14311079545	9.98911019438086e-05	\\
2393.12189275568	0.000100100754278433	\\
2394.10067471591	0.000105332487301941	\\
2395.07945667614	9.7748378807384e-05	\\
2396.05823863636	0.000102589999810173	\\
2397.03702059659	9.93460869422118e-05	\\
2398.01580255682	0.000100634980112166	\\
2398.99458451704	0.000101471149089398	\\
2399.97336647727	0.000106735514171201	\\
2400.9521484375	0.000103158143074631	\\
2401.93093039773	0.000102933217939029	\\
2402.90971235795	9.99634566796276e-05	\\
2403.88849431818	9.95571300629467e-05	\\
2404.86727627841	9.96732772863716e-05	\\
2405.84605823864	0.000100531242432179	\\
2406.82484019886	0.000102003613869311	\\
2407.80362215909	0.000100449521622986	\\
2408.78240411932	0.000104297034092605	\\
2409.76118607954	0.000105739228984803	\\
2410.73996803977	0.000101083701348381	\\
2411.71875	0.000100999918585507	\\
2412.69753196023	0.000102541284049645	\\
2413.67631392045	0.000102525486795906	\\
2414.65509588068	0.000106740051195226	\\
2415.63387784091	0.000105491533269898	\\
2416.61265980114	0.000113414091245332	\\
2417.59144176136	0.000105137118612674	\\
2418.57022372159	0.00011057172261991	\\
2419.54900568182	0.000108952042590176	\\
2420.52778764204	0.000107536102508993	\\
2421.50656960227	0.000106104562317601	\\
2422.4853515625	0.000107658165432722	\\
2423.46413352273	0.000103145135020807	\\
2424.44291548295	0.000104218441566834	\\
2425.42169744318	0.000106451250854862	\\
2426.40047940341	0.000107425113560551	\\
2427.37926136364	0.000106460478022591	\\
2428.35804332386	0.000105566162787949	\\
2429.33682528409	0.000109046216909584	\\
2430.31560724432	0.000106330834512688	\\
2431.29438920454	0.000115140109157081	\\
2432.27317116477	0.000110579793225022	\\
2433.251953125	0.000106797479735007	\\
2434.23073508523	0.000112029915834983	\\
2435.20951704545	0.000106031567422489	\\
2436.18829900568	0.000105522306160331	\\
2437.16708096591	0.00010972210841529	\\
2438.14586292614	0.000109176780612148	\\
2439.12464488636	0.000105703276914155	\\
2440.10342684659	0.000110587306559086	\\
2441.08220880682	0.000110285814969406	\\
2442.06099076704	0.000109925390451604	\\
2443.03977272727	0.000106394679639078	\\
2444.0185546875	0.000110303430953792	\\
2444.99733664773	0.000108883627096687	\\
2445.97611860795	0.000112447265166107	\\
2446.95490056818	0.000111313023623019	\\
2447.93368252841	0.000108778933446389	\\
2448.91246448864	0.000112063056804293	\\
2449.89124644886	0.000111433237934226	\\
2450.87002840909	0.000112355245271743	\\
2451.84881036932	0.000113402473789215	\\
2452.82759232954	0.000111183308271927	\\
2453.80637428977	0.000110268258154739	\\
2454.78515625	0.000113117225705864	\\
2455.76393821023	0.000108996527533352	\\
2456.74272017045	0.000110687715683246	\\
2457.72150213068	0.000105889352156613	\\
2458.70028409091	0.000109190555494576	\\
2459.67906605114	0.000112271596124504	\\
2460.65784801136	0.000110187690542575	\\
2461.63662997159	0.000109672747098757	\\
2462.61541193182	0.000107109264443939	\\
2463.59419389204	0.000106993950657265	\\
2464.57297585227	0.000111292277366312	\\
2465.5517578125	0.000111614146070093	\\
2466.53053977273	0.000109887052999262	\\
2467.50932173295	0.000111251837662286	\\
2468.48810369318	0.000109457650215927	\\
2469.46688565341	0.000109047126185712	\\
2470.44566761364	0.000108390663673153	\\
2471.42444957386	0.000108813362957607	\\
2472.40323153409	0.000109840115368203	\\
2473.38201349432	0.00010998569407894	\\
2474.36079545454	0.000109140983803943	\\
2475.33957741477	0.000111269414222155	\\
2476.318359375	0.000111702556709089	\\
2477.29714133523	0.000109158696194052	\\
2478.27592329545	0.000108347826435207	\\
2479.25470525568	0.000108894076017163	\\
2480.23348721591	0.000107164241028585	\\
2481.21226917614	0.000109528179658785	\\
2482.19105113636	0.00010787128324987	\\
2483.16983309659	0.000108335070978324	\\
2484.14861505682	0.00010600054134476	\\
2485.12739701704	0.000106071571484226	\\
2486.10617897727	0.000108915725112319	\\
2487.0849609375	0.000109427266997669	\\
2488.06374289773	0.000109152400799865	\\
2489.04252485795	0.000109835300814188	\\
2490.02130681818	0.000109975244647637	\\
2491.00008877841	0.000109573480389458	\\
2491.97887073864	0.000109678924730175	\\
2492.95765269886	0.000111625436211053	\\
2493.93643465909	0.000111106464644463	\\
2494.91521661932	0.00011096279489066	\\
2495.89399857954	0.000108555213208292	\\
2496.87278053977	0.000111448362557504	\\
2497.8515625	0.000110325303174948	\\
2498.83034446023	0.000108074156692662	\\
2499.80912642045	0.000116449857777445	\\
2500.78790838068	0.000111325767899561	\\
2501.76669034091	0.00011228234484442	\\
2502.74547230114	0.000112043290459984	\\
2503.72425426136	0.000110627870783499	\\
2504.70303622159	0.000108168769186644	\\
2505.68181818182	0.000110488413167738	\\
2506.66060014204	0.000111715849311881	\\
2507.63938210227	0.000113394966553725	\\
2508.6181640625	0.00011124511137217	\\
2509.59694602273	0.000108374090509968	\\
2510.57572798295	0.000112469144134801	\\
2511.55450994318	0.000112069405719592	\\
2512.53329190341	0.000113800802049745	\\
2513.51207386364	0.000106732093311453	\\
2514.49085582386	0.00011278855010762	\\
2515.46963778409	0.000110535776313188	\\
2516.44841974432	0.000110348612916227	\\
2517.42720170454	0.000109622014904788	\\
2518.40598366477	0.000108449539436259	\\
2519.384765625	0.000107266073466245	\\
2520.36354758523	0.000110750620439491	\\
2521.34232954545	0.00010728444395162	\\
2522.32111150568	0.000109084841728536	\\
2523.29989346591	0.000106105337585187	\\
2524.27867542614	0.000107930512784064	\\
2525.25745738636	0.000105766948659911	\\
2526.23623934659	0.000107861468586736	\\
2527.21502130682	0.00010686435341413	\\
2528.19380326704	0.00010527547048505	\\
2529.17258522727	0.000105786063696774	\\
2530.1513671875	0.000105528148097144	\\
2531.13014914773	0.000106983060102631	\\
2532.10893110795	0.000102624746434306	\\
2533.08771306818	0.000102662135649246	\\
2534.06649502841	9.83682314637121e-05	\\
2535.04527698864	0.000102218098070852	\\
2536.02405894886	0.000100147650224194	\\
2537.00284090909	0.000100045505577887	\\
2537.98162286932	0.000100823274430604	\\
2538.96040482954	0.000100224787754124	\\
2539.93918678977	9.7268977240453e-05	\\
2540.91796875	9.97175117077393e-05	\\
2541.89675071023	9.62773691669119e-05	\\
2542.87553267045	9.53709432503948e-05	\\
2543.85431463068	9.52234016202394e-05	\\
2544.83309659091	9.44919526480246e-05	\\
2545.81187855114	9.58693923498783e-05	\\
2546.79066051136	9.46952343032776e-05	\\
2547.76944247159	9.24469395289939e-05	\\
2548.74822443182	9.33956301768107e-05	\\
2549.72700639204	9.04438273918453e-05	\\
2550.70578835227	9.07942509287619e-05	\\
2551.6845703125	8.91558013717916e-05	\\
2552.66335227273	9.30574296423698e-05	\\
2553.64213423295	9.04311676804409e-05	\\
2554.62091619318	8.78292369866237e-05	\\
2555.59969815341	8.73477746230265e-05	\\
2556.57848011364	9.07926899735881e-05	\\
2557.55726207386	8.64451706110157e-05	\\
2558.53604403409	8.75275772735354e-05	\\
2559.51482599432	8.48143892041425e-05	\\
2560.49360795454	8.60592925947414e-05	\\
2561.47238991477	8.30501199293459e-05	\\
2562.451171875	8.27774390798321e-05	\\
2563.42995383523	8.41246432816327e-05	\\
2564.40873579545	8.38532439479546e-05	\\
2565.38751775568	8.2311477873803e-05	\\
2566.36629971591	8.13332760554973e-05	\\
2567.34508167614	7.8303916179834e-05	\\
2568.32386363636	7.84693018805608e-05	\\
2569.30264559659	7.88753426003173e-05	\\
2570.28142755682	7.92055757824273e-05	\\
2571.26020951704	7.80955742472822e-05	\\
2572.23899147727	7.542620572892e-05	\\
2573.2177734375	7.60592684240498e-05	\\
2574.19655539773	7.45657819662765e-05	\\
2575.17533735795	7.76171279106103e-05	\\
2576.15411931818	7.44646331649742e-05	\\
2577.13290127841	7.4405592759981e-05	\\
2578.11168323864	7.15735130779855e-05	\\
2579.09046519886	7.13507785800166e-05	\\
2580.06924715909	7.46709288644335e-05	\\
2581.04802911932	7.2353789770297e-05	\\
2582.02681107954	7.28974371351945e-05	\\
2583.00559303977	7.30576703615969e-05	\\
2583.984375	7.09241448268958e-05	\\
2584.96315696023	7.29002071589145e-05	\\
2585.94193892045	7.18847473884896e-05	\\
2586.92072088068	7.3557732241883e-05	\\
2587.89950284091	6.83766327566056e-05	\\
2588.87828480114	7.25772274334929e-05	\\
2589.85706676136	7.16464425411314e-05	\\
2590.83584872159	7.33801021299497e-05	\\
2591.81463068182	7.24232792714113e-05	\\
2592.79341264204	7.3062585184749e-05	\\
2593.77219460227	7.06615210015601e-05	\\
2594.7509765625	6.96216051931405e-05	\\
2595.72975852273	7.23233404145613e-05	\\
2596.70854048295	7.08098063011676e-05	\\
2597.68732244318	7.0537018091497e-05	\\
2598.66610440341	7.24653089174098e-05	\\
2599.64488636364	7.08035783972571e-05	\\
2600.62366832386	7.11736235481416e-05	\\
2601.60245028409	7.01889160767203e-05	\\
2602.58123224432	6.97621474484221e-05	\\
2603.56001420454	7.09053851750471e-05	\\
2604.53879616477	7.17128784688566e-05	\\
2605.517578125	7.09423743473235e-05	\\
2606.49636008523	7.06222145096329e-05	\\
2607.47514204545	7.01764097523846e-05	\\
2608.45392400568	7.16945851579142e-05	\\
2609.43270596591	6.78953345429653e-05	\\
2610.41148792614	7.05208598446875e-05	\\
2611.39026988636	6.73241264117462e-05	\\
2612.36905184659	7.09747340270041e-05	\\
2613.34783380682	6.83669724199402e-05	\\
2614.32661576704	6.85935112202516e-05	\\
2615.30539772727	6.74561707193742e-05	\\
2616.2841796875	6.65447913572274e-05	\\
2617.26296164773	7.0900994166576e-05	\\
2618.24174360795	6.78917563798011e-05	\\
2619.22052556818	7.14901015956958e-05	\\
2620.19930752841	6.85803038054583e-05	\\
2621.17808948864	7.03831375627185e-05	\\
2622.15687144886	6.66189627958344e-05	\\
2623.13565340909	7.20243590695749e-05	\\
2624.11443536932	6.99929447525945e-05	\\
2625.09321732954	6.85245143918182e-05	\\
2626.07199928977	6.93162910176448e-05	\\
2627.05078125	6.58934687653491e-05	\\
2628.02956321023	6.76929657701618e-05	\\
2629.00834517045	7.17871607189033e-05	\\
2629.98712713068	6.87324777077797e-05	\\
2630.96590909091	6.69114566025806e-05	\\
2631.94469105114	7.26398876702801e-05	\\
2632.92347301136	6.97203780461505e-05	\\
2633.90225497159	6.82765516733581e-05	\\
2634.88103693182	7.08047349471163e-05	\\
2635.85981889204	6.75490890570875e-05	\\
2636.83860085227	6.71169840757441e-05	\\
2637.8173828125	7.01086983279676e-05	\\
2638.79616477273	7.09138261741203e-05	\\
2639.77494673295	7.05071079831196e-05	\\
2640.75372869318	6.99659155405877e-05	\\
2641.73251065341	6.8594330088763e-05	\\
2642.71129261364	6.92987210030146e-05	\\
2643.69007457386	6.7972014732494e-05	\\
2644.66885653409	7.10770095611582e-05	\\
2645.64763849432	6.84268698353441e-05	\\
2646.62642045454	6.95011001991266e-05	\\
2647.60520241477	6.94289020446712e-05	\\
2648.583984375	6.65725420987894e-05	\\
2649.56276633523	7.0316446381956e-05	\\
2650.54154829545	6.82764081952595e-05	\\
2651.52033025568	6.76761196954508e-05	\\
2652.49911221591	6.78729773711526e-05	\\
2653.47789417614	6.81985871803766e-05	\\
2654.45667613636	6.96652415476408e-05	\\
2655.43545809659	6.79247089281452e-05	\\
2656.41424005682	6.94711020873112e-05	\\
2657.39302201704	7.18111807368274e-05	\\
2658.37180397727	6.95248624960888e-05	\\
2659.3505859375	6.79115878975472e-05	\\
2660.32936789773	7.031931068367e-05	\\
2661.30814985795	6.94693993861179e-05	\\
2662.28693181818	6.9258057786129e-05	\\
2663.26571377841	6.75889785264275e-05	\\
2664.24449573864	7.17251565153678e-05	\\
2665.22327769886	7.08165407357922e-05	\\
2666.20205965909	7.01816246674753e-05	\\
2667.18084161932	6.96403436580988e-05	\\
2668.15962357954	6.83106203803795e-05	\\
2669.13840553977	6.86208306624719e-05	\\
2670.1171875	7.07423503117862e-05	\\
2671.09596946023	6.85770929486878e-05	\\
2672.07475142045	6.95148382701782e-05	\\
2673.05353338068	7.040493753396e-05	\\
2674.03231534091	7.06451421687249e-05	\\
2675.01109730114	7.00054441134455e-05	\\
2675.98987926136	7.06310475536429e-05	\\
2676.96866122159	6.88992554921357e-05	\\
2677.94744318182	7.00056808863116e-05	\\
2678.92622514204	6.99964370187317e-05	\\
2679.90500710227	6.8737506888625e-05	\\
2680.8837890625	6.72171860414079e-05	\\
2681.86257102273	6.83943888792385e-05	\\
2682.84135298295	6.87667251701375e-05	\\
2683.82013494318	6.87661079583638e-05	\\
2684.79891690341	6.72418323642096e-05	\\
2685.77769886364	6.90122184995088e-05	\\
2686.75648082386	6.65869886052797e-05	\\
2687.73526278409	6.74485077248541e-05	\\
2688.71404474432	6.69656269729008e-05	\\
2689.69282670454	6.89161398938086e-05	\\
2690.67160866477	6.7353592289133e-05	\\
2691.650390625	6.61019429901949e-05	\\
2692.62917258523	6.62715740353863e-05	\\
2693.60795454545	6.61873424437701e-05	\\
2694.58673650568	6.43973643110673e-05	\\
2695.56551846591	6.60325104310333e-05	\\
2696.54430042614	6.32477736647329e-05	\\
2697.52308238636	6.40092839755057e-05	\\
2698.50186434659	6.68201630305868e-05	\\
2699.48064630682	6.51205169511774e-05	\\
2700.45942826704	6.85098098103387e-05	\\
2701.43821022727	6.36296270170227e-05	\\
2702.4169921875	6.37761995959561e-05	\\
2703.39577414773	6.28421201205149e-05	\\
2704.37455610795	6.34128793720611e-05	\\
2705.35333806818	6.29993269037094e-05	\\
2706.33212002841	6.19257854704127e-05	\\
2707.31090198864	6.42321266424734e-05	\\
2708.28968394886	6.4402483003593e-05	\\
2709.26846590909	6.06122086679644e-05	\\
2710.24724786932	6.35109173119395e-05	\\
2711.22602982954	6.08587087934687e-05	\\
2712.20481178977	6.00515446169609e-05	\\
2713.18359375	6.079274768826e-05	\\
2714.16237571023	5.89940225941951e-05	\\
2715.14115767045	5.99236529722567e-05	\\
2716.11993963068	5.88477838862073e-05	\\
2717.09872159091	5.6942003969553e-05	\\
2718.07750355114	5.79996180191676e-05	\\
2719.05628551136	5.93755702002545e-05	\\
2720.03506747159	5.67644706569873e-05	\\
2721.01384943182	5.48932615530352e-05	\\
2721.99263139204	5.6325123465116e-05	\\
2722.97141335227	5.30183138723711e-05	\\
2723.9501953125	5.5500617705006e-05	\\
2724.92897727273	5.40737630546009e-05	\\
2725.90775923295	5.26680207212619e-05	\\
2726.88654119318	5.36310251771399e-05	\\
2727.86532315341	5.18749696840843e-05	\\
2728.84410511364	4.98427669505211e-05	\\
2729.82288707386	4.73270239444297e-05	\\
2730.80166903409	4.89565408583174e-05	\\
2731.78045099432	4.71539914893102e-05	\\
2732.75923295454	4.83284481821626e-05	\\
2733.73801491477	4.56572821287227e-05	\\
2734.716796875	4.57849992638468e-05	\\
2735.69557883523	4.62404835591624e-05	\\
2736.67436079545	4.50538941767417e-05	\\
2737.65314275568	4.25400316222213e-05	\\
2738.63192471591	4.3940722815074e-05	\\
2739.61070667614	4.15395741609725e-05	\\
2740.58948863636	4.1154802284959e-05	\\
2741.56827059659	4.23512530599431e-05	\\
2742.54705255682	3.97322050937019e-05	\\
2743.52583451704	3.90767338304783e-05	\\
2744.50461647727	3.76622670801965e-05	\\
2745.4833984375	3.418281818305e-05	\\
2746.46218039773	3.4444412430758e-05	\\
2747.44096235795	3.36509914338607e-05	\\
2748.41974431818	3.65035284927056e-05	\\
2749.39852627841	3.06546852559461e-05	\\
2750.37730823864	3.2325004264849e-05	\\
2751.35609019886	3.17154918819151e-05	\\
2752.33487215909	2.94259526754603e-05	\\
2753.31365411932	2.6506856703707e-05	\\
2754.29243607954	2.95873219480198e-05	\\
2755.27121803977	2.57699465298254e-05	\\
2756.25	2.70907415157002e-05	\\
2757.22878196023	2.76449683599753e-05	\\
2758.20756392045	2.62110792739386e-05	\\
2759.18634588068	2.40461567995971e-05	\\
2760.16512784091	2.23891423227091e-05	\\
2761.14390980114	2.19186385408391e-05	\\
2762.12269176136	2.03168868362837e-05	\\
2763.10147372159	2.06440654388663e-05	\\
2764.08025568182	2.24567761931238e-05	\\
2765.05903764204	1.75936508997786e-05	\\
2766.03781960227	1.64276481641987e-05	\\
2767.0166015625	1.81078819377322e-05	\\
2767.99538352273	1.40312336665645e-05	\\
2768.97416548295	1.45424563939311e-05	\\
2769.95294744318	1.60897760863177e-05	\\
2770.93172940341	1.20303696722458e-05	\\
2771.91051136364	1.28460237570197e-05	\\
2772.88929332386	1.3992145302665e-05	\\
2773.86807528409	9.40836384365371e-06	\\
2774.84685724432	1.1316226627811e-05	\\
2775.82563920454	1.09778698287737e-05	\\
2776.80442116477	7.60586186614173e-06	\\
2777.783203125	7.27768180232306e-06	\\
2778.76198508523	6.67416535645733e-06	\\
2779.74076704545	7.11408115501689e-06	\\
2780.71954900568	8.13878526672867e-06	\\
2781.69833096591	5.2859208243353e-06	\\
2782.67711292614	4.87683679222842e-06	\\
2783.65589488636	4.26073122923893e-06	\\
2784.63467684659	3.63857188388526e-06	\\
2785.61345880682	6.30118668744594e-06	\\
2786.59224076704	1.83167192244578e-06	\\
2787.57102272727	4.44495886982964e-06	\\
2788.5498046875	2.98115661286422e-06	\\
2789.52858664773	2.34102893983528e-06	\\
2790.50736860795	1.59043956274734e-06	\\
2791.48615056818	1.40606173685635e-06	\\
2792.46493252841	1.43926464463395e-06	\\
2793.44371448864	1.04838176228123e-06	\\
2794.42249644886	2.83907062752766e-06	\\
2795.40127840909	2.48020140383517e-06	\\
2796.38006036932	3.21132333327885e-06	\\
2797.35884232954	2.05678962194096e-06	\\
2798.33762428977	3.08252780409563e-06	\\
2799.31640625	2.84774554322067e-06	\\
2800.29518821023	3.77710956408051e-06	\\
2801.27397017045	3.89155995266075e-06	\\
2802.25275213068	5.70363378951656e-06	\\
2803.23153409091	3.60840717315616e-06	\\
2804.21031605114	6.56169913258913e-06	\\
2805.18909801136	4.28032410366373e-06	\\
2806.16787997159	4.13936813669873e-06	\\
2807.14666193182	7.96280523780959e-06	\\
2808.12544389204	7.31129018518458e-06	\\
2809.10422585227	7.70853235798674e-06	\\
2810.0830078125	8.52784262697893e-06	\\
2811.06178977273	6.4615160626873e-06	\\
2812.04057173295	8.54954203168045e-06	\\
2813.01935369318	7.98687614869949e-06	\\
2813.99813565341	7.39844284066512e-06	\\
2814.97691761364	1.07920342891192e-05	\\
2815.95569957386	7.56859546496437e-06	\\
2816.93448153409	7.07156927989748e-06	\\
2817.91326349432	1.00541147504547e-05	\\
2818.89204545454	1.10842628578932e-05	\\
2819.87082741477	1.1300531280782e-05	\\
2820.849609375	1.01426285554931e-05	\\
2821.82839133523	1.16823506557888e-05	\\
2822.80717329545	1.04904004093153e-05	\\
2823.78595525568	9.77849245428255e-06	\\
2824.76473721591	1.33375216347001e-05	\\
2825.74351917614	1.28237319297214e-05	\\
2826.72230113636	1.45632494777957e-05	\\
2827.70108309659	1.20881158494943e-05	\\
2828.67986505682	1.44239351955038e-05	\\
2829.65864701704	1.56662698628748e-05	\\
2830.63742897727	1.48620347199765e-05	\\
2831.6162109375	1.60377211346914e-05	\\
2832.59499289773	1.3777906119137e-05	\\
2833.57377485795	1.24430422638442e-05	\\
2834.55255681818	1.54390691601294e-05	\\
2835.53133877841	1.30637487168778e-05	\\
2836.51012073864	1.5716034049034e-05	\\
2837.48890269886	1.83189861409916e-05	\\
2838.46768465909	1.48147427594532e-05	\\
2839.44646661932	1.74085539256444e-05	\\
2840.42524857955	1.65724512273348e-05	\\
2841.40403053977	1.77528970905517e-05	\\
2842.3828125	1.73385645613279e-05	\\
2843.36159446023	1.7676384589318e-05	\\
2844.34037642045	1.77018340496334e-05	\\
2845.31915838068	1.98154295390924e-05	\\
2846.29794034091	1.89337163101113e-05	\\
2847.27672230114	1.89669414160988e-05	\\
2848.25550426136	1.85658957232046e-05	\\
2849.23428622159	1.88783596540821e-05	\\
2850.21306818182	1.76968139079159e-05	\\
2851.19185014205	1.97413931445058e-05	\\
2852.17063210227	1.94677830472769e-05	\\
2853.1494140625	1.95329159147453e-05	\\
2854.12819602273	2.08797188740182e-05	\\
2855.10697798295	2.1402957726015e-05	\\
2856.08575994318	2.16253814110379e-05	\\
2857.06454190341	2.2797144664633e-05	\\
2858.04332386364	2.28551566579733e-05	\\
2859.02210582386	2.16345676836806e-05	\\
2860.00088778409	2.28885445691707e-05	\\
2860.97966974432	2.11394946310237e-05	\\
2861.95845170455	2.09041701230428e-05	\\
2862.93723366477	2.30048803598309e-05	\\
2863.916015625	2.18865705426327e-05	\\
2864.89479758523	2.25842676952105e-05	\\
2865.87357954545	2.46903365011015e-05	\\
2866.85236150568	2.21616425185602e-05	\\
2867.83114346591	2.4763023111923e-05	\\
2868.80992542614	2.32672576104824e-05	\\
2869.78870738636	2.302147632696e-05	\\
2870.76748934659	2.5014334332899e-05	\\
2871.74627130682	2.33730838862257e-05	\\
2872.72505326705	2.39683272946166e-05	\\
2873.70383522727	2.48547417894208e-05	\\
2874.6826171875	2.6237922184831e-05	\\
2875.66139914773	2.31310407973476e-05	\\
2876.64018110795	2.51371154485832e-05	\\
2877.61896306818	2.51905948401798e-05	\\
2878.59774502841	2.47596495819186e-05	\\
2879.57652698864	2.46690343254037e-05	\\
2880.55530894886	2.33867770534241e-05	\\
2881.53409090909	2.27039917041235e-05	\\
2882.51287286932	2.428805861006e-05	\\
2883.49165482955	2.61791656874858e-05	\\
2884.47043678977	2.80436651621082e-05	\\
2885.44921875	2.65205634900081e-05	\\
2886.42800071023	2.42947745918932e-05	\\
2887.40678267045	2.62129931638556e-05	\\
2888.38556463068	2.6084455031219e-05	\\
2889.36434659091	2.43650924796593e-05	\\
2890.34312855114	2.77654077814304e-05	\\
2891.32191051136	2.79523788757791e-05	\\
2892.30069247159	2.69702018616533e-05	\\
2893.27947443182	2.57002669771406e-05	\\
2894.25825639205	2.33680751684435e-05	\\
2895.23703835227	2.63254726909998e-05	\\
2896.2158203125	2.36689693325324e-05	\\
2897.19460227273	2.70484723997276e-05	\\
2898.17338423295	2.88002029951845e-05	\\
2899.15216619318	2.36409419196368e-05	\\
2900.13094815341	2.54511056958732e-05	\\
2901.10973011364	2.50745947765446e-05	\\
2902.08851207386	2.40316253354833e-05	\\
2903.06729403409	2.76282408477547e-05	\\
2904.04607599432	2.70473433377311e-05	\\
2905.02485795455	2.62807903591011e-05	\\
2906.00363991477	2.81919012569464e-05	\\
2906.982421875	2.73442118407578e-05	\\
2907.96120383523	2.75713634435691e-05	\\
2908.93998579545	2.65417143016612e-05	\\
2909.91876775568	2.42812451271105e-05	\\
2910.89754971591	2.53211797477397e-05	\\
2911.87633167614	2.6375030331911e-05	\\
2912.85511363636	2.88783451749334e-05	\\
2913.83389559659	2.71863321874319e-05	\\
2914.81267755682	2.64813790830983e-05	\\
2915.79145951705	2.65997592894433e-05	\\
2916.77024147727	2.78983445815806e-05	\\
2917.7490234375	2.60765111086293e-05	\\
2918.72780539773	2.77334242941074e-05	\\
2919.70658735795	2.94350035862743e-05	\\
2920.68536931818	2.74196458499763e-05	\\
2921.66415127841	2.73955059160384e-05	\\
2922.64293323864	2.81001269706987e-05	\\
2923.62171519886	2.67367656110137e-05	\\
2924.60049715909	2.7023283281937e-05	\\
2925.57927911932	2.55530554545624e-05	\\
2926.55806107955	2.75929777049159e-05	\\
2927.53684303977	2.92157730940869e-05	\\
2928.515625	2.84613337814769e-05	\\
2929.49440696023	2.82677795594115e-05	\\
2930.47318892045	2.86617706676169e-05	\\
2931.45197088068	2.71253217776947e-05	\\
2932.43075284091	2.90704741915278e-05	\\
2933.40953480114	2.6693657059399e-05	\\
2934.38831676136	2.86800194776205e-05	\\
2935.36709872159	2.96611485897164e-05	\\
2936.34588068182	2.77023129842143e-05	\\
2937.32466264205	3.05277749365847e-05	\\
2938.30344460227	2.71343882162781e-05	\\
2939.2822265625	2.69782989093971e-05	\\
2940.26100852273	3.0184520777549e-05	\\
2941.23979048295	2.69663541199709e-05	\\
2942.21857244318	2.65067769574803e-05	\\
2943.19735440341	2.80333683125475e-05	\\
2944.17613636364	2.87549288449937e-05	\\
2945.15491832386	2.88918772408233e-05	\\
2946.13370028409	2.6450125058077e-05	\\
2947.11248224432	2.50900322084322e-05	\\
2948.09126420455	2.67624321554342e-05	\\
2949.07004616477	2.8337651600211e-05	\\
2950.048828125	2.83746624813777e-05	\\
2951.02761008523	2.74506620349816e-05	\\
2952.00639204545	2.5476679963471e-05	\\
2952.98517400568	2.66231973938438e-05	\\
2953.96395596591	2.8258784313529e-05	\\
2954.94273792614	2.74691873124932e-05	\\
2955.92151988636	2.96743041177847e-05	\\
2956.90030184659	2.66550230299437e-05	\\
2957.87908380682	2.7603175936499e-05	\\
2958.85786576705	2.80756965853266e-05	\\
2959.83664772727	2.91500348067855e-05	\\
2960.8154296875	2.65657366311122e-05	\\
2961.79421164773	3.01040967651612e-05	\\
2962.77299360795	2.6744586365635e-05	\\
2963.75177556818	2.96256911284993e-05	\\
2964.73055752841	2.61008588994387e-05	\\
2965.70933948864	2.82025914795937e-05	\\
2966.68812144886	2.81767056872265e-05	\\
2967.66690340909	3.02536871405798e-05	\\
2968.64568536932	2.84766101458026e-05	\\
2969.62446732955	2.87310233823662e-05	\\
2970.60324928977	2.97900517795671e-05	\\
2971.58203125	2.87717755773301e-05	\\
2972.56081321023	2.88006988396303e-05	\\
2973.53959517045	2.70456211589594e-05	\\
2974.51837713068	2.89299010788318e-05	\\
2975.49715909091	2.73442400310812e-05	\\
2976.47594105114	2.55846360303388e-05	\\
2977.45472301136	3.27292562292266e-05	\\
2978.43350497159	2.72888447706491e-05	\\
2979.41228693182	2.68387447607717e-05	\\
2980.39106889205	2.78831349962173e-05	\\
2981.36985085227	2.88930184865889e-05	\\
2982.3486328125	2.9991242286357e-05	\\
2983.32741477273	2.86802688655019e-05	\\
2984.30619673295	2.78813948237884e-05	\\
2985.28497869318	2.89605009988574e-05	\\
2986.26376065341	2.76325900161744e-05	\\
2987.24254261364	2.76082990402083e-05	\\
2988.22132457386	2.67635933122929e-05	\\
2989.20010653409	2.87474810125467e-05	\\
2990.17888849432	3.080547063017e-05	\\
2991.15767045455	2.92852359444348e-05	\\
2992.13645241477	2.86657947480515e-05	\\
2993.115234375	2.95178450931199e-05	\\
2994.09401633523	2.87202771264191e-05	\\
2995.07279829545	2.95432845556231e-05	\\
2996.05158025568	2.85265849774965e-05	\\
2997.03036221591	2.88810698494846e-05	\\
2998.00914417614	2.93983138079722e-05	\\
2998.98792613636	3.24877650401294e-05	\\
2999.96670809659	2.92826036745502e-05	\\
3000.94549005682	2.96334335442084e-05	\\
3001.92427201705	2.99242608409227e-05	\\
3002.90305397727	3.00576803582343e-05	\\
3003.8818359375	2.97719129034869e-05	\\
3004.86061789773	2.92383779842303e-05	\\
3005.83939985795	2.97860940149674e-05	\\
3006.81818181818	3.2080975343432e-05	\\
3007.79696377841	3.19956686700167e-05	\\
3008.77574573864	3.10257190674746e-05	\\
3009.75452769886	2.78453742802441e-05	\\
3010.73330965909	3.12014505558957e-05	\\
3011.71209161932	3.16536133880674e-05	\\
3012.69087357955	2.89022584829522e-05	\\
3013.66965553977	3.03034944737678e-05	\\
3014.6484375	3.07418696904582e-05	\\
3015.62721946023	2.80188394192639e-05	\\
3016.60600142045	3.17447857007025e-05	\\
3017.58478338068	3.02239266460084e-05	\\
3018.56356534091	3.04105546651206e-05	\\
3019.54234730114	3.05479904707452e-05	\\
3020.52112926136	3.11882669159545e-05	\\
3021.49991122159	3.20231792991452e-05	\\
3022.47869318182	3.22750561275213e-05	\\
3023.45747514205	3.14391493456881e-05	\\
3024.43625710227	2.83792986509969e-05	\\
3025.4150390625	3.14706514680367e-05	\\
3026.39382102273	2.93765533023974e-05	\\
3027.37260298295	3.30988661961465e-05	\\
3028.35138494318	3.22123043659956e-05	\\
3029.33016690341	3.20469058313102e-05	\\
3030.30894886364	2.99292039702306e-05	\\
3031.28773082386	3.03588034199235e-05	\\
3032.26651278409	3.3022938190305e-05	\\
3033.24529474432	3.19379936890594e-05	\\
3034.22407670455	3.04238438872928e-05	\\
3035.20285866477	3.30297024779143e-05	\\
3036.181640625	3.18794111308669e-05	\\
3037.16042258523	2.93203476251785e-05	\\
3038.13920454545	3.33101728262072e-05	\\
3039.11798650568	3.1787603223944e-05	\\
3040.09676846591	3.14183720815216e-05	\\
3041.07555042614	3.32456066617083e-05	\\
3042.05433238636	3.33715154765386e-05	\\
3043.03311434659	3.28162498593674e-05	\\
3044.01189630682	3.17615886366948e-05	\\
3044.99067826705	3.15406038763585e-05	\\
3045.96946022727	3.20798711348323e-05	\\
3046.9482421875	3.18072991651622e-05	\\
3047.92702414773	3.24817758522911e-05	\\
3048.90580610795	3.11281814399672e-05	\\
3049.88458806818	3.05712192372608e-05	\\
3050.86337002841	3.29467662170427e-05	\\
3051.84215198864	3.16053091225755e-05	\\
3052.82093394886	3.16643170156659e-05	\\
3053.79971590909	3.16337146064558e-05	\\
3054.77849786932	3.00970328667092e-05	\\
3055.75727982955	2.93383609067764e-05	\\
3056.73606178977	3.40390313491243e-05	\\
3057.71484375	2.96204773656025e-05	\\
3058.69362571023	3.18550692040253e-05	\\
3059.67240767045	3.19275065640851e-05	\\
3060.65118963068	3.13912245973827e-05	\\
3061.62997159091	3.12071549396546e-05	\\
3062.60875355114	2.96443355414312e-05	\\
3063.58753551136	3.10246307021464e-05	\\
3064.56631747159	3.12723179325694e-05	\\
3065.54509943182	3.05380207320233e-05	\\
3066.52388139205	3.05601263871954e-05	\\
3067.50266335227	3.14852030414486e-05	\\
3068.4814453125	3.23868013145833e-05	\\
3069.46022727273	3.12897648569659e-05	\\
3070.43900923295	3.22104530749433e-05	\\
3071.41779119318	3.01827312807375e-05	\\
3072.39657315341	3.10077488839121e-05	\\
3073.37535511364	3.19571489885444e-05	\\
3074.35413707386	3.08355616868561e-05	\\
3075.33291903409	3.16619588111454e-05	\\
3076.31170099432	3.28777397121566e-05	\\
3077.29048295455	3.45679363794588e-05	\\
3078.26926491477	3.2825173656949e-05	\\
3079.248046875	3.13011308894104e-05	\\
3080.22682883523	2.9602786759951e-05	\\
3081.20561079545	3.0286656628568e-05	\\
3082.18439275568	2.90378093860256e-05	\\
3083.16317471591	3.25604772133214e-05	\\
3084.14195667614	2.97836690859958e-05	\\
3085.12073863636	3.09532204748093e-05	\\
3086.09952059659	3.14015624088789e-05	\\
3087.07830255682	3.14718308222679e-05	\\
3088.05708451705	3.08806994790909e-05	\\
3089.03586647727	2.96877355502374e-05	\\
3090.0146484375	3.23267311881356e-05	\\
3090.99343039773	2.97802834115898e-05	\\
3091.97221235795	3.15518180678497e-05	\\
3092.95099431818	2.82870417276885e-05	\\
3093.92977627841	2.95536807910447e-05	\\
3094.90855823864	2.91528122056927e-05	\\
3095.88734019886	3.02924751422948e-05	\\
3096.86612215909	2.90310263793981e-05	\\
3097.84490411932	3.06722082749696e-05	\\
3098.82368607955	2.98675988087206e-05	\\
3099.80246803977	3.07276009972117e-05	\\
3100.78125	3.40994632071445e-05	\\
3101.76003196023	3.04874375511596e-05	\\
3102.73881392045	3.04687158688986e-05	\\
3103.71759588068	2.91098184418573e-05	\\
3104.69637784091	2.87990217750103e-05	\\
3105.67515980114	2.92473264956577e-05	\\
3106.65394176136	3.21492664933948e-05	\\
3107.63272372159	3.04636798315844e-05	\\
3108.61150568182	2.96022190264134e-05	\\
3109.59028764205	3.08657489676814e-05	\\
3110.56906960227	3.15463968198617e-05	\\
3111.5478515625	2.99643574212418e-05	\\
3112.52663352273	2.95552339601399e-05	\\
3113.50541548295	2.99977035057783e-05	\\
3114.48419744318	3.04052698943182e-05	\\
3115.46297940341	3.03375437297696e-05	\\
3116.44176136364	3.04483773778095e-05	\\
3117.42054332386	2.9862147254605e-05	\\
3118.39932528409	2.88055992794654e-05	\\
3119.37810724432	2.99060429105928e-05	\\
3120.35688920455	2.76996093237341e-05	\\
3121.33567116477	3.05427541582526e-05	\\
3122.314453125	3.16263689607421e-05	\\
3123.29323508523	3.10505683985319e-05	\\
3124.27201704545	2.79491937688571e-05	\\
3125.25079900568	3.12365029683367e-05	\\
3126.22958096591	3.07206921804743e-05	\\
3127.20836292614	3.06579335495828e-05	\\
3128.18714488636	2.84505435891578e-05	\\
3129.16592684659	2.8741437040598e-05	\\
3130.14470880682	2.87024004361996e-05	\\
3131.12349076705	2.85373298251967e-05	\\
3132.10227272727	3.02132799249353e-05	\\
3133.0810546875	2.74687447582903e-05	\\
3134.05983664773	3.07573417566565e-05	\\
3135.03861860795	3.05689840241159e-05	\\
3136.01740056818	3.02695736760203e-05	\\
3136.99618252841	2.94293802742887e-05	\\
3137.97496448864	2.99643165226413e-05	\\
3138.95374644886	2.75921272351883e-05	\\
3139.93252840909	2.94136737476662e-05	\\
3140.91131036932	2.62925424340878e-05	\\
3141.89009232955	2.82268439248614e-05	\\
3142.86887428977	2.89263596185191e-05	\\
3143.84765625	2.97583026787091e-05	\\
3144.82643821023	2.93165231619934e-05	\\
3145.80522017045	2.73284935544949e-05	\\
3146.78400213068	2.89947738557834e-05	\\
3147.76278409091	2.91395624744292e-05	\\
3148.74156605114	2.80003968065655e-05	\\
3149.72034801136	2.92661869528001e-05	\\
3150.69912997159	2.9041518428302e-05	\\
3151.67791193182	2.88012837912411e-05	\\
3152.65669389205	3.01938827557221e-05	\\
3153.63547585227	2.65989905864601e-05	\\
3154.6142578125	2.9741447019976e-05	\\
3155.59303977273	2.8153470429254e-05	\\
3156.57182173295	2.84317198041202e-05	\\
3157.55060369318	2.92124287424555e-05	\\
3158.52938565341	2.77810889532061e-05	\\
3159.50816761364	2.95079768224209e-05	\\
3160.48694957386	2.66415859072203e-05	\\
3161.46573153409	2.86599993608604e-05	\\
3162.44451349432	2.8305693003008e-05	\\
3163.42329545455	2.65746526729145e-05	\\
3164.40207741477	2.55535092760743e-05	\\
3165.380859375	2.82580701752575e-05	\\
3166.35964133523	2.68788330985011e-05	\\
3167.33842329545	2.92394528944405e-05	\\
3168.31720525568	2.76782239273165e-05	\\
3169.29598721591	2.88397873394234e-05	\\
3170.27476917614	2.74879992193556e-05	\\
3171.25355113636	2.68349334093725e-05	\\
3172.23233309659	2.81798722859114e-05	\\
3173.21111505682	2.8266320250824e-05	\\
3174.18989701705	2.99524237269425e-05	\\
3175.16867897727	2.97388355983029e-05	\\
3176.1474609375	2.63117546856178e-05	\\
3177.12624289773	2.73081290323837e-05	\\
3178.10502485795	2.74433952665603e-05	\\
3179.08380681818	2.68698524401867e-05	\\
3180.06258877841	2.79307204821153e-05	\\
3181.04137073864	2.82989047842273e-05	\\
3182.02015269886	2.80923078100831e-05	\\
3182.99893465909	3.06900148008257e-05	\\
3183.97771661932	2.68182086897995e-05	\\
3184.95649857955	2.77174277273676e-05	\\
3185.93528053977	2.60861806164505e-05	\\
3186.9140625	2.58586678671612e-05	\\
3187.89284446023	2.94729172879224e-05	\\
3188.87162642045	2.67726131569518e-05	\\
3189.85040838068	2.86897501305598e-05	\\
3190.82919034091	2.97157961029983e-05	\\
3191.80797230114	2.7786191432948e-05	\\
3192.78675426136	2.9599488714336e-05	\\
3193.76553622159	2.66317632661718e-05	\\
3194.74431818182	2.86065666714102e-05	\\
3195.72310014205	2.78738576687453e-05	\\
3196.70188210227	2.78396672826225e-05	\\
3197.6806640625	3.26572054067993e-05	\\
3198.65944602273	2.99728262043207e-05	\\
3199.63822798295	2.85501743508355e-05	\\
3200.61700994318	3.16642444542786e-05	\\
3201.59579190341	3.01540231695694e-05	\\
3202.57457386364	3.03784034511564e-05	\\
3203.55335582386	2.77044041912007e-05	\\
3204.53213778409	3.12760721725413e-05	\\
3205.51091974432	3.08843163339162e-05	\\
3206.48970170455	3.17825960657084e-05	\\
3207.46848366477	3.25058758627663e-05	\\
3208.447265625	3.25609248324948e-05	\\
3209.42604758523	3.10480775017174e-05	\\
3210.40482954545	2.9580052014934e-05	\\
3211.38361150568	3.1173647998022e-05	\\
3212.36239346591	3.22004467389314e-05	\\
3213.34117542614	3.2511903293932e-05	\\
3214.31995738636	3.27751000879112e-05	\\
3215.29873934659	3.2785348452528e-05	\\
3216.27752130682	3.24588702263055e-05	\\
3217.25630326705	3.54034432939482e-05	\\
3218.23508522727	3.15621456640487e-05	\\
3219.2138671875	3.28973871032374e-05	\\
3220.19264914773	3.46384052450149e-05	\\
3221.17143110795	3.5083154061928e-05	\\
3222.15021306818	3.23175933991142e-05	\\
3223.12899502841	3.40265980547617e-05	\\
3224.10777698864	3.42376285150137e-05	\\
3225.08655894886	3.51154569793037e-05	\\
3226.06534090909	3.2218202967311e-05	\\
3227.04412286932	3.34813202355208e-05	\\
3228.02290482955	3.38437513156039e-05	\\
3229.00168678977	3.77075181586432e-05	\\
3229.98046875	3.52859322489456e-05	\\
3230.95925071023	3.77866015849735e-05	\\
3231.93803267045	3.59993054051283e-05	\\
3232.91681463068	3.58682571050846e-05	\\
3233.89559659091	3.48976951795053e-05	\\
3234.87437855114	3.58723037255397e-05	\\
3235.85316051136	3.67247535766509e-05	\\
3236.83194247159	3.38078010409376e-05	\\
3237.81072443182	3.65696189750922e-05	\\
3238.78950639205	3.72182389958807e-05	\\
3239.76828835227	3.74503954295125e-05	\\
3240.7470703125	3.78655328203314e-05	\\
3241.72585227273	3.72942052248536e-05	\\
3242.70463423295	3.73054947921756e-05	\\
3243.68341619318	3.75545279089725e-05	\\
3244.66219815341	3.83937225248458e-05	\\
3245.64098011364	4.10608296911008e-05	\\
3246.61976207386	3.85042765618251e-05	\\
3247.59854403409	3.93543637780014e-05	\\
3248.57732599432	4.06216344393081e-05	\\
3249.55610795455	3.94834804434126e-05	\\
3250.53488991477	3.94650178833629e-05	\\
3251.513671875	3.96568022445889e-05	\\
3252.49245383523	3.93890490046893e-05	\\
3253.47123579545	4.11398692436208e-05	\\
3254.45001775568	4.06831797116501e-05	\\
3255.42879971591	4.02837127483814e-05	\\
3256.40758167614	4.26025554055588e-05	\\
3257.38636363636	4.06295051392251e-05	\\
3258.36514559659	4.2735985987338e-05	\\
3259.34392755682	4.13335714677769e-05	\\
3260.32270951705	4.34502827503277e-05	\\
3261.30149147727	4.343683479493e-05	\\
3262.2802734375	4.10407107781447e-05	\\
3263.25905539773	4.26742040401251e-05	\\
3264.23783735795	4.40253909350521e-05	\\
3265.21661931818	4.47608660464203e-05	\\
3266.19540127841	4.47033902259115e-05	\\
3267.17418323864	4.33059609665212e-05	\\
3268.15296519886	4.27459137140029e-05	\\
3269.13174715909	4.44398747120446e-05	\\
3270.11052911932	4.20731305714153e-05	\\
3271.08931107955	4.10833564129976e-05	\\
3272.06809303977	4.36239975911334e-05	\\
3273.046875	4.43150087358046e-05	\\
3274.02565696023	4.37685754170727e-05	\\
3275.00443892045	4.32649310298536e-05	\\
3275.98322088068	4.423143158138e-05	\\
3276.96200284091	4.41199206205471e-05	\\
3277.94078480114	4.69224417851518e-05	\\
3278.91956676136	4.47993275321321e-05	\\
3279.89834872159	4.60281609307521e-05	\\
3280.87713068182	4.5859622880032e-05	\\
3281.85591264205	4.45108775188835e-05	\\
3282.83469460227	4.51244214938203e-05	\\
3283.8134765625	4.66738222867203e-05	\\
3284.79225852273	4.8110326110231e-05	\\
3285.77104048295	4.77668288676465e-05	\\
3286.74982244318	4.91666192719022e-05	\\
3287.72860440341	4.69717993610907e-05	\\
3288.70738636364	4.7484522655949e-05	\\
3289.68616832386	4.77819865057671e-05	\\
3290.66495028409	4.7720183644832e-05	\\
3291.64373224432	4.89678467037973e-05	\\
3292.62251420455	4.89566459072943e-05	\\
3293.60129616477	5.09844353850561e-05	\\
3294.580078125	4.94192131215545e-05	\\
3295.55886008523	5.0020632770343e-05	\\
3296.53764204545	4.93552450384913e-05	\\
3297.51642400568	4.90582331061618e-05	\\
3298.49520596591	5.12256194603219e-05	\\
3299.47398792614	4.98241997041401e-05	\\
3300.45276988636	5.21185270229354e-05	\\
3301.43155184659	4.90386113912221e-05	\\
3302.41033380682	5.20179968955105e-05	\\
3303.38911576705	5.07350262681806e-05	\\
3304.36789772727	4.95550738186779e-05	\\
3305.3466796875	5.34113137319996e-05	\\
3306.32546164773	5.31272308029514e-05	\\
3307.30424360795	4.82328215256514e-05	\\
3308.28302556818	5.32287682916784e-05	\\
3309.26180752841	5.03731540821242e-05	\\
3310.24058948864	4.99374500501349e-05	\\
3311.21937144886	5.09473266488535e-05	\\
3312.19815340909	5.00576222879465e-05	\\
3313.17693536932	5.20364530750661e-05	\\
3314.15571732955	4.96095249702646e-05	\\
3315.13449928977	5.23298323025473e-05	\\
3316.11328125	5.11895350489683e-05	\\
3317.09206321023	4.97543388852248e-05	\\
3318.07084517045	5.23336042661321e-05	\\
3319.04962713068	5.21814811505675e-05	\\
3320.02840909091	5.28073598569126e-05	\\
3321.00719105114	5.12262947582285e-05	\\
3321.98597301136	5.07703567947888e-05	\\
3322.96475497159	5.19827010075905e-05	\\
3323.94353693182	5.27951060481202e-05	\\
3324.92231889205	5.41127242550012e-05	\\
3325.90110085227	5.24792302372581e-05	\\
3326.8798828125	5.13921456618147e-05	\\
3327.85866477273	5.0378735710641e-05	\\
3328.83744673295	5.38644920529503e-05	\\
3329.81622869318	5.2704484256258e-05	\\
3330.79501065341	5.03918285916274e-05	\\
3331.77379261364	5.32518884148425e-05	\\
3332.75257457386	5.31284543422281e-05	\\
3333.73135653409	5.0210786437445e-05	\\
3334.71013849432	5.26283681862908e-05	\\
3335.68892045455	5.28170615012429e-05	\\
3336.66770241477	5.0717687787048e-05	\\
3337.646484375	5.02585369107838e-05	\\
3338.62526633523	5.29229849309779e-05	\\
3339.60404829545	5.26351535118901e-05	\\
3340.58283025568	4.93947332799673e-05	\\
3341.56161221591	5.34845450250262e-05	\\
3342.54039417614	4.94422164401982e-05	\\
3343.51917613636	5.20758559267219e-05	\\
3344.49795809659	5.27925929567576e-05	\\
3345.47674005682	5.35615601036616e-05	\\
3346.45552201705	5.55359338794254e-05	\\
3347.43430397727	5.36196775185393e-05	\\
3348.4130859375	5.41133951757984e-05	\\
3349.39186789773	5.42714346936052e-05	\\
3350.37064985795	5.24943145802123e-05	\\
3351.34943181818	5.35323676875809e-05	\\
3352.32821377841	5.28207948066701e-05	\\
3353.30699573864	5.34567062128113e-05	\\
3354.28577769886	5.34736641101127e-05	\\
3355.26455965909	5.3212709080193e-05	\\
3356.24334161932	5.33417963092772e-05	\\
3357.22212357955	5.40959482636595e-05	\\
3358.20090553977	5.22345346103172e-05	\\
3359.1796875	5.44149605162082e-05	\\
3360.15846946023	5.1252118888628e-05	\\
3361.13725142045	5.42897455904587e-05	\\
3362.11603338068	5.41379694862146e-05	\\
3363.09481534091	5.27574930400833e-05	\\
3364.07359730114	5.36437587677311e-05	\\
3365.05237926136	5.19657677640007e-05	\\
3366.03116122159	5.15687812357986e-05	\\
3367.00994318182	5.24506406418255e-05	\\
3367.98872514205	5.14624645691474e-05	\\
3368.96750710227	5.23859202187424e-05	\\
3369.9462890625	5.07077619093295e-05	\\
3370.92507102273	5.3167283439614e-05	\\
3371.90385298295	4.91500688839769e-05	\\
3372.88263494318	5.26426481538416e-05	\\
3373.86141690341	5.2248162238169e-05	\\
3374.84019886364	5.1949050455237e-05	\\
3375.81898082386	5.3603283558791e-05	\\
3376.79776278409	5.30446264449575e-05	\\
3377.77654474432	5.27228362354139e-05	\\
3378.75532670455	5.26679009895619e-05	\\
3379.73410866477	5.14353462365316e-05	\\
3380.712890625	5.13779602975566e-05	\\
3381.69167258523	5.21868014977409e-05	\\
3382.67045454545	5.23993305322231e-05	\\
3383.64923650568	5.26683582270135e-05	\\
3384.62801846591	5.23165613237087e-05	\\
3385.60680042614	5.24375673704428e-05	\\
3386.58558238636	5.20588028381819e-05	\\
3387.56436434659	5.21329014900205e-05	\\
3388.54314630682	5.35642178395792e-05	\\
3389.52192826705	5.06111633038365e-05	\\
3390.50071022727	5.11080645862914e-05	\\
3391.4794921875	5.11103579431298e-05	\\
3392.45827414773	5.05094712008292e-05	\\
3393.43705610795	5.51545374372882e-05	\\
3394.41583806818	5.14897717426115e-05	\\
3395.39462002841	5.28151892681635e-05	\\
3396.37340198864	4.90376015462953e-05	\\
3397.35218394886	5.0741247189499e-05	\\
3398.33096590909	5.14688303747629e-05	\\
3399.30974786932	5.09980503747598e-05	\\
3400.28852982955	4.92488407543448e-05	\\
3401.26731178977	5.07617774667605e-05	\\
3402.24609375	5.00246613063859e-05	\\
3403.22487571023	5.05278480891587e-05	\\
3404.20365767045	5.2434799675444e-05	\\
3405.18243963068	5.30999749532326e-05	\\
3406.16122159091	5.05270011795487e-05	\\
3407.14000355114	5.15384236164131e-05	\\
3408.11878551136	4.98939960073473e-05	\\
3409.09756747159	4.9126010618606e-05	\\
3410.07634943182	4.99662399948792e-05	\\
3411.05513139205	5.17811262439752e-05	\\
3412.03391335227	5.08452831311544e-05	\\
3413.0126953125	5.15985593638413e-05	\\
3413.99147727273	4.85058059164649e-05	\\
3414.97025923295	5.15430720042452e-05	\\
3415.94904119318	4.81524689234702e-05	\\
3416.92782315341	4.80129051770051e-05	\\
3417.90660511364	4.7726453456157e-05	\\
3418.88538707386	4.82455279286958e-05	\\
3419.86416903409	4.76558015351622e-05	\\
3420.84295099432	4.8010918425472e-05	\\
3421.82173295455	4.77853050921342e-05	\\
3422.80051491477	4.65290985403044e-05	\\
3423.779296875	5.04515750892471e-05	\\
3424.75807883523	4.86751675601306e-05	\\
3425.73686079545	4.92629755003958e-05	\\
3426.71564275568	4.64538850452643e-05	\\
3427.69442471591	4.94116033347289e-05	\\
3428.67320667614	4.98297460441504e-05	\\
3429.65198863636	4.81014007744798e-05	\\
3430.63077059659	4.84217260354735e-05	\\
3431.60955255682	4.48791331679605e-05	\\
3432.58833451705	4.57245027001536e-05	\\
3433.56711647727	4.50725224839193e-05	\\
3434.5458984375	4.72316603792089e-05	\\
3435.52468039773	4.61066501348366e-05	\\
3436.50346235795	4.90379896307467e-05	\\
3437.48224431818	4.73756084224398e-05	\\
3438.46102627841	4.79725580167525e-05	\\
3439.43980823864	4.76016355709742e-05	\\
3440.41859019886	4.72152862524229e-05	\\
3441.39737215909	4.7128183090052e-05	\\
3442.37615411932	4.5749487615822e-05	\\
3443.35493607955	4.65419572596551e-05	\\
3444.33371803977	4.71984051821329e-05	\\
3445.3125	4.57863285341811e-05	\\
3446.29128196023	4.6246377811968e-05	\\
3447.27006392045	4.75968829060132e-05	\\
3448.24884588068	4.59601421057777e-05	\\
3449.22762784091	4.76520131790565e-05	\\
3450.20640980114	4.52267461755825e-05	\\
3451.18519176136	4.84517110695884e-05	\\
3452.16397372159	4.68345308939946e-05	\\
3453.14275568182	4.66251627450559e-05	\\
3454.12153764205	4.8275804051729e-05	\\
3455.10031960227	4.70604265002694e-05	\\
3456.0791015625	4.5416971436649e-05	\\
3457.05788352273	4.48309061584547e-05	\\
3458.03666548295	4.68678057521889e-05	\\
3459.01544744318	4.53748919538021e-05	\\
3459.99422940341	4.529291502842e-05	\\
3460.97301136364	4.81622926063327e-05	\\
3461.95179332386	4.61405334035497e-05	\\
3462.93057528409	4.5058783947415e-05	\\
3463.90935724432	4.37605486099262e-05	\\
3464.88813920455	4.44253433958194e-05	\\
3465.86692116477	4.52794914158213e-05	\\
3466.845703125	4.71879545925288e-05	\\
3467.82448508523	4.59787618663981e-05	\\
3468.80326704545	4.44603751714299e-05	\\
3469.78204900568	4.78361053118497e-05	\\
3470.76083096591	4.69067876895111e-05	\\
3471.73961292614	4.64489575304372e-05	\\
3472.71839488636	4.62486824292251e-05	\\
3473.69717684659	4.79496328190633e-05	\\
3474.67595880682	4.34084425178137e-05	\\
3475.65474076705	4.71504878540732e-05	\\
3476.63352272727	4.45376170399024e-05	\\
3477.6123046875	4.73562573692123e-05	\\
3478.59108664773	4.56987859504057e-05	\\
3479.56986860795	4.64432986572641e-05	\\
3480.54865056818	4.55420852420568e-05	\\
3481.52743252841	4.52200522494928e-05	\\
3482.50621448864	4.56798765769716e-05	\\
3483.48499644886	4.54284034656416e-05	\\
3484.46377840909	4.41910766922834e-05	\\
3485.44256036932	4.58769732802532e-05	\\
3486.42134232955	4.68238886339859e-05	\\
3487.40012428977	4.60268632751805e-05	\\
3488.37890625	4.49328633374628e-05	\\
3489.35768821023	4.43526718159176e-05	\\
3490.33647017045	4.6714701222714e-05	\\
3491.31525213068	4.71279578651373e-05	\\
3492.29403409091	4.82808743017239e-05	\\
3493.27281605114	4.7644474289189e-05	\\
3494.25159801136	4.66189686244779e-05	\\
3495.23037997159	4.4536408992094e-05	\\
3496.20916193182	4.56931362793712e-05	\\
3497.18794389205	4.5380671267914e-05	\\
3498.16672585227	4.7128157267359e-05	\\
3499.1455078125	4.75377629140373e-05	\\
3500.12428977273	4.61355738837145e-05	\\
3501.10307173295	4.76102741472143e-05	\\
3502.08185369318	4.7340520194908e-05	\\
3503.06063565341	4.70282247729772e-05	\\
3504.03941761364	4.82587481712401e-05	\\
3505.01819957386	4.59535245833293e-05	\\
3505.99698153409	4.59942907338422e-05	\\
3506.97576349432	4.54189822779985e-05	\\
3507.95454545455	4.35525152754587e-05	\\
3508.93332741477	4.64241017218653e-05	\\
3509.912109375	4.4960010440694e-05	\\
3510.89089133523	4.67449365125098e-05	\\
3511.86967329545	4.54175133758184e-05	\\
3512.84845525568	4.44357548789921e-05	\\
3513.82723721591	4.61321388130339e-05	\\
3514.80601917614	4.76418819132314e-05	\\
3515.78480113636	4.74694566609769e-05	\\
3516.76358309659	4.3752747526609e-05	\\
3517.74236505682	4.42268076303994e-05	\\
3518.72114701705	4.56546353917837e-05	\\
3519.69992897727	4.25116567296119e-05	\\
3520.6787109375	4.43335841522289e-05	\\
3521.65749289773	4.43058351137217e-05	\\
3522.63627485795	4.530211437965e-05	\\
3523.61505681818	4.59783500723744e-05	\\
3524.59383877841	4.32280611816044e-05	\\
3525.57262073864	4.38280729967319e-05	\\
3526.55140269886	4.50644700247493e-05	\\
3527.53018465909	4.45413588502033e-05	\\
3528.50896661932	4.38002149016411e-05	\\
3529.48774857955	4.36391154304534e-05	\\
3530.46653053977	4.47500013783869e-05	\\
3531.4453125	4.02777018015511e-05	\\
3532.42409446023	4.0048637042974e-05	\\
3533.40287642045	4.39817181456959e-05	\\
3534.38165838068	4.24654406485526e-05	\\
3535.36044034091	4.2659133700273e-05	\\
3536.33922230114	4.11022822073613e-05	\\
3537.31800426136	4.36881774410363e-05	\\
3538.29678622159	4.00345566876998e-05	\\
3539.27556818182	4.37157834078019e-05	\\
3540.25435014205	4.12622489827318e-05	\\
3541.23313210227	4.24958057778697e-05	\\
3542.2119140625	4.05592914523559e-05	\\
3543.19069602273	3.94646511383258e-05	\\
3544.16947798295	3.94763687222195e-05	\\
3545.14825994318	4.23392991400619e-05	\\
3546.12704190341	4.20966715281125e-05	\\
3547.10582386364	4.1666418489657e-05	\\
3548.08460582386	4.28445114415932e-05	\\
3549.06338778409	4.03020825250206e-05	\\
3550.04216974432	4.25444221127675e-05	\\
3551.02095170455	4.30704402560189e-05	\\
3551.99973366477	4.04579925096481e-05	\\
3552.978515625	4.30237828538262e-05	\\
3553.95729758523	4.25980515992553e-05	\\
3554.93607954545	4.10441343444174e-05	\\
3555.91486150568	3.90516639204534e-05	\\
3556.89364346591	4.1491238947762e-05	\\
3557.87242542614	4.11190792942232e-05	\\
3558.85120738636	4.21546859260163e-05	\\
3559.82998934659	4.32870489390452e-05	\\
3560.80877130682	4.23677952234134e-05	\\
3561.78755326705	4.13912221348337e-05	\\
3562.76633522727	4.2184076099665e-05	\\
3563.7451171875	3.77732404261588e-05	\\
3564.72389914773	4.21752788474431e-05	\\
3565.70268110795	4.37311481980006e-05	\\
3566.68146306818	4.1110527662796e-05	\\
3567.66024502841	4.3119208580491e-05	\\
3568.63902698864	4.1795653398356e-05	\\
3569.61780894886	4.13331167239201e-05	\\
3570.59659090909	3.87475610989521e-05	\\
3571.57537286932	4.03774792638663e-05	\\
3572.55415482955	3.90843287768188e-05	\\
3573.53293678977	4.31710679439864e-05	\\
3574.51171875	3.94539073983376e-05	\\
3575.49050071023	3.95214124589203e-05	\\
3576.46928267045	4.12352813269973e-05	\\
3577.44806463068	3.97213493967883e-05	\\
3578.42684659091	4.06609595002274e-05	\\
3579.40562855114	4.05279524354438e-05	\\
3580.38441051136	4.28600175555246e-05	\\
3581.36319247159	3.99510767919719e-05	\\
3582.34197443182	3.98755914638738e-05	\\
3583.32075639205	4.15242098234815e-05	\\
3584.29953835227	4.0788534804023e-05	\\
3585.2783203125	4.20921744496378e-05	\\
3586.25710227273	4.0344837342813e-05	\\
3587.23588423295	3.95584911092386e-05	\\
3588.21466619318	4.10539490120135e-05	\\
3589.19344815341	4.08584235539245e-05	\\
3590.17223011364	4.22028216961429e-05	\\
3591.15101207386	4.2357771476413e-05	\\
3592.12979403409	3.95734578877951e-05	\\
3593.10857599432	4.42884438183583e-05	\\
3594.08735795455	4.06246274252626e-05	\\
3595.06613991477	4.29038281118927e-05	\\
3596.044921875	4.24001536232566e-05	\\
3597.02370383523	4.16337759332213e-05	\\
3598.00248579545	3.80840153175029e-05	\\
3598.98126775568	4.19136069800056e-05	\\
3599.96004971591	4.13433596692525e-05	\\
3600.93883167614	3.88014394888718e-05	\\
3601.91761363636	4.17590296926622e-05	\\
3602.89639559659	4.03992034045248e-05	\\
3603.87517755682	3.84912127600725e-05	\\
3604.85395951705	4.12371007861484e-05	\\
3605.83274147727	4.02244337386981e-05	\\
3606.8115234375	4.30140608957201e-05	\\
3607.79030539773	4.07139339214312e-05	\\
3608.76908735795	4.20721816323456e-05	\\
3609.74786931818	3.94025574803305e-05	\\
3610.72665127841	3.93092290090835e-05	\\
3611.70543323864	4.11313961297879e-05	\\
3612.68421519886	4.04076421488901e-05	\\
3613.66299715909	4.14160554648722e-05	\\
3614.64177911932	3.99594648417113e-05	\\
3615.62056107955	4.19169523141712e-05	\\
3616.59934303977	4.0352727304138e-05	\\
3617.578125	4.12248539503652e-05	\\
3618.55690696023	4.06010072001338e-05	\\
3619.53568892045	4.13551478385204e-05	\\
3620.51447088068	4.15832840442682e-05	\\
3621.49325284091	4.01681645338422e-05	\\
3622.47203480114	3.93868054774497e-05	\\
3623.45081676136	4.16576663096037e-05	\\
3624.42959872159	4.05203301194777e-05	\\
3625.40838068182	3.95750376151756e-05	\\
3626.38716264205	3.77594777179571e-05	\\
3627.36594460227	3.95531859391507e-05	\\
3628.3447265625	4.09019518803247e-05	\\
3629.32350852273	3.98439163384367e-05	\\
3630.30229048295	3.79559879538148e-05	\\
3631.28107244318	3.83670630029043e-05	\\
3632.25985440341	4.01400211836028e-05	\\
3633.23863636364	3.86276755435753e-05	\\
3634.21741832386	3.75880253970649e-05	\\
3635.19620028409	3.86973702758838e-05	\\
3636.17498224432	4.11049097636587e-05	\\
3637.15376420455	3.97062495802118e-05	\\
3638.13254616477	3.87553016581048e-05	\\
3639.111328125	3.80520625213307e-05	\\
3640.09011008523	4.04545040399068e-05	\\
3641.06889204545	3.92862342419321e-05	\\
3642.04767400568	4.04358558688976e-05	\\
3643.02645596591	3.69974536490568e-05	\\
3644.00523792614	3.83871463126835e-05	\\
3644.98401988636	3.96208093372965e-05	\\
3645.96280184659	3.86822909357547e-05	\\
3646.94158380682	3.92665891053496e-05	\\
3647.92036576705	3.94405417524812e-05	\\
3648.89914772727	3.9329957852397e-05	\\
3649.8779296875	3.94911843896974e-05	\\
3650.85671164773	3.91402808492986e-05	\\
3651.83549360795	3.93632103128577e-05	\\
3652.81427556818	3.84352808094774e-05	\\
3653.79305752841	4.20067378145406e-05	\\
3654.77183948864	3.72880176919304e-05	\\
3655.75062144886	3.7704009262909e-05	\\
3656.72940340909	3.96381393768439e-05	\\
3657.70818536932	3.78730293990579e-05	\\
3658.68696732955	3.81928832516541e-05	\\
3659.66574928977	3.80540068627075e-05	\\
3660.64453125	4.00079251465127e-05	\\
3661.62331321023	3.82872632057377e-05	\\
3662.60209517045	3.82677036695606e-05	\\
3663.58087713068	4.06749853467893e-05	\\
3664.55965909091	3.72427061914332e-05	\\
3665.53844105114	3.86118604622875e-05	\\
3666.51722301136	3.70898453263436e-05	\\
3667.49600497159	3.96790722469997e-05	\\
3668.47478693182	3.7103108130636e-05	\\
3669.45356889205	4.029886859317e-05	\\
3670.43235085227	3.73742232369591e-05	\\
3671.4111328125	3.82125072978215e-05	\\
3672.38991477273	3.87576998229094e-05	\\
3673.36869673295	3.76407057671956e-05	\\
3674.34747869318	3.75808599879563e-05	\\
3675.32626065341	3.93804938762789e-05	\\
3676.30504261364	4.08041265941592e-05	\\
3677.28382457386	3.82950488474347e-05	\\
3678.26260653409	3.71334671896015e-05	\\
3679.24138849432	3.80102030080849e-05	\\
3680.22017045455	3.8235032198823e-05	\\
3681.19895241477	4.24176305534557e-05	\\
3682.177734375	3.63981138652181e-05	\\
3683.15651633523	3.9697727116545e-05	\\
3684.13529829545	4.05438536928292e-05	\\
3685.11408025568	3.87622318865517e-05	\\
3686.09286221591	4.03290025824336e-05	\\
3687.07164417614	4.14691028283351e-05	\\
3688.05042613636	4.42531684022799e-05	\\
3689.02920809659	3.8999603220473e-05	\\
3690.00799005682	4.12876100949182e-05	\\
3690.98677201705	4.11708063019935e-05	\\
3691.96555397727	3.9585894215813e-05	\\
3692.9443359375	4.39475553694997e-05	\\
3693.92311789773	4.38145229987531e-05	\\
3694.90189985795	4.08689566064302e-05	\\
3695.88068181818	4.25676168496949e-05	\\
3696.85946377841	4.38765568122022e-05	\\
3697.83824573864	4.26743913006083e-05	\\
3698.81702769886	4.32919916253372e-05	\\
3699.79580965909	4.14672707830449e-05	\\
3700.77459161932	4.27661779083117e-05	\\
3701.75337357955	4.22486557537034e-05	\\
3702.73215553977	4.44175878228196e-05	\\
3703.7109375	4.27760586175734e-05	\\
3704.68971946023	4.36963506273122e-05	\\
3705.66850142045	4.21138748553262e-05	\\
3706.64728338068	4.22550882617605e-05	\\
3707.62606534091	4.3187985127312e-05	\\
3708.60484730114	4.35655167605668e-05	\\
3709.58362926136	4.37250220548816e-05	\\
3710.56241122159	4.50818252104888e-05	\\
3711.54119318182	4.26819808036159e-05	\\
3712.51997514205	4.46765444410173e-05	\\
3713.49875710227	4.54789185357714e-05	\\
3714.4775390625	4.44666719900217e-05	\\
3715.45632102273	4.53058873471634e-05	\\
3716.43510298295	4.32019125061052e-05	\\
3717.41388494318	4.42243032390894e-05	\\
3718.39266690341	4.45121999081128e-05	\\
3719.37144886364	4.32483607492107e-05	\\
3720.35023082386	4.43193292411948e-05	\\
3721.32901278409	4.21304833367251e-05	\\
3722.30779474432	4.59485192131969e-05	\\
3723.28657670455	4.32990504404731e-05	\\
3724.26535866477	4.54825187388606e-05	\\
3725.244140625	4.90223967290906e-05	\\
3726.22292258523	4.85194949483734e-05	\\
3727.20170454545	4.49845132528553e-05	\\
3728.18048650568	4.54010970824803e-05	\\
3729.15926846591	4.95204928995875e-05	\\
3730.13805042614	4.77671652190434e-05	\\
3731.11683238636	4.4707443551977e-05	\\
3732.09561434659	4.67632138288794e-05	\\
3733.07439630682	4.83866236133463e-05	\\
3734.05317826705	4.7073167130736e-05	\\
3735.03196022727	4.69287063921319e-05	\\
3736.0107421875	4.58111973264176e-05	\\
3736.98952414773	4.64582887984241e-05	\\
3737.96830610795	4.88766906515931e-05	\\
3738.94708806818	4.7467727941902e-05	\\
3739.92587002841	4.70782245423549e-05	\\
3740.90465198864	4.91559670175633e-05	\\
3741.88343394886	4.94982796912489e-05	\\
3742.86221590909	4.887200383708e-05	\\
3743.84099786932	4.91152459320046e-05	\\
3744.81977982955	4.92385987809082e-05	\\
3745.79856178977	4.85626540744485e-05	\\
3746.77734375	4.97790096068211e-05	\\
3747.75612571023	4.90141669477829e-05	\\
3748.73490767045	4.7496847419387e-05	\\
3749.71368963068	5.18006418974728e-05	\\
3750.69247159091	4.93513958509188e-05	\\
3751.67125355114	5.14130376097197e-05	\\
3752.65003551136	5.12389097641488e-05	\\
3753.62881747159	5.01893105953232e-05	\\
3754.60759943182	5.18298024015877e-05	\\
3755.58638139205	5.01325577756612e-05	\\
3756.56516335227	4.97269602724454e-05	\\
3757.5439453125	4.91865525859681e-05	\\
3758.52272727273	5.00322453271244e-05	\\
3759.50150923295	5.14150095638268e-05	\\
3760.48029119318	5.06076688045789e-05	\\
3761.45907315341	4.9683458792158e-05	\\
3762.43785511364	5.0774661487939e-05	\\
3763.41663707386	5.0308307875758e-05	\\
3764.39541903409	5.17321432392243e-05	\\
3765.37420099432	5.19982120179273e-05	\\
3766.35298295455	5.26165098457775e-05	\\
3767.33176491477	5.32514216128372e-05	\\
3768.310546875	5.21185082703813e-05	\\
3769.28932883523	4.84714691225718e-05	\\
3770.26811079545	4.93481391953226e-05	\\
3771.24689275568	5.09882718770439e-05	\\
3772.22567471591	5.23817738681829e-05	\\
3773.20445667614	5.30197116089014e-05	\\
3774.18323863636	5.11751365446968e-05	\\
3775.16202059659	5.18452538794321e-05	\\
3776.14080255682	5.3231191857084e-05	\\
3777.11958451705	5.24596824200012e-05	\\
3778.09836647727	5.16833329128906e-05	\\
3779.0771484375	5.26508935515129e-05	\\
3780.05593039773	5.20828394672173e-05	\\
3781.03471235795	5.23070010145028e-05	\\
3782.01349431818	5.43261973293283e-05	\\
3782.99227627841	5.40071612794723e-05	\\
3783.97105823864	5.293269709796e-05	\\
3784.94984019886	5.25579128684022e-05	\\
3785.92862215909	5.16861432391699e-05	\\
3786.90740411932	5.33470439593475e-05	\\
3787.88618607955	5.07746527966685e-05	\\
3788.86496803977	5.35458361555634e-05	\\
3789.84375	5.39621054593514e-05	\\
3790.82253196023	5.11609144172272e-05	\\
3791.80131392045	5.12659670483311e-05	\\
3792.78009588068	5.47005957597376e-05	\\
3793.75887784091	5.35735361149549e-05	\\
3794.73765980114	5.25850353591083e-05	\\
3795.71644176136	5.22998354065478e-05	\\
3796.69522372159	4.95326960328218e-05	\\
3797.67400568182	5.1110794827529e-05	\\
3798.65278764205	5.40274032256627e-05	\\
3799.63156960227	5.27771254704745e-05	\\
3800.6103515625	5.11572023735607e-05	\\
3801.58913352273	5.23694637377453e-05	\\
3802.56791548295	5.23707108503667e-05	\\
3803.54669744318	5.11166056629124e-05	\\
3804.52547940341	5.21980879530903e-05	\\
3805.50426136364	5.15695789747535e-05	\\
3806.48304332386	5.16882326173925e-05	\\
3807.46182528409	5.10129403022596e-05	\\
3808.44060724432	5.21706204933827e-05	\\
3809.41938920455	5.19940144437224e-05	\\
3810.39817116477	5.18138347831621e-05	\\
3811.376953125	4.98138051668705e-05	\\
3812.35573508523	5.1564624521222e-05	\\
3813.33451704545	5.11259028769771e-05	\\
3814.31329900568	5.10217939994528e-05	\\
3815.29208096591	5.09797035712477e-05	\\
3816.27086292614	5.12347928847826e-05	\\
3817.24964488636	4.98404794663958e-05	\\
3818.22842684659	4.92496227274518e-05	\\
3819.20720880682	5.02744296257562e-05	\\
3820.18599076705	5.00142192602741e-05	\\
3821.16477272727	4.63273496382022e-05	\\
3822.1435546875	4.98304648354514e-05	\\
3823.12233664773	4.81735489981731e-05	\\
3824.10111860795	5.08928290693344e-05	\\
3825.07990056818	5.07738929280643e-05	\\
3826.05868252841	4.85404249979958e-05	\\
3827.03746448864	4.92869655815344e-05	\\
3828.01624644886	5.03278504210859e-05	\\
3828.99502840909	5.18077192997968e-05	\\
3829.97381036932	4.89948149729324e-05	\\
3830.95259232955	5.21597908265593e-05	\\
3831.93137428977	4.99500713571219e-05	\\
3832.91015625	5.05203122186141e-05	\\
3833.88893821023	5.14556198762075e-05	\\
3834.86772017045	4.91795773562777e-05	\\
3835.84650213068	4.99802398943761e-05	\\
3836.82528409091	5.10965594574372e-05	\\
3837.80406605114	5.19945489281056e-05	\\
3838.78284801136	5.2753757466664e-05	\\
3839.76162997159	5.30761936442478e-05	\\
3840.74041193182	5.3965896455367e-05	\\
3841.71919389205	5.43482682540691e-05	\\
3842.69797585227	5.20428654401006e-05	\\
3843.6767578125	5.37696119410539e-05	\\
3844.65553977273	5.32141378181602e-05	\\
3845.63432173295	5.32138400264453e-05	\\
3846.61310369318	5.50675953486791e-05	\\
3847.59188565341	5.58655034642275e-05	\\
3848.57066761364	5.59382195623055e-05	\\
3849.54944957386	5.79110324936151e-05	\\
3850.52823153409	5.60905562222534e-05	\\
3851.50701349432	5.5618098412093e-05	\\
3852.48579545455	5.64774992862805e-05	\\
3853.46457741477	5.7616440309945e-05	\\
3854.443359375	5.61104713616532e-05	\\
3855.42214133523	5.69520813497048e-05	\\
3856.40092329545	5.70374644900475e-05	\\
3857.37970525568	5.57221511080522e-05	\\
3858.35848721591	5.61760018677057e-05	\\
3859.33726917614	5.65209549614877e-05	\\
3860.31605113636	5.90336999084004e-05	\\
3861.29483309659	5.95650626982315e-05	\\
3862.27361505682	6.11641926534955e-05	\\
3863.25239701705	5.91108658655806e-05	\\
3864.23117897727	6.07697854008272e-05	\\
3865.2099609375	6.00636018377133e-05	\\
3866.18874289773	6.1873569013871e-05	\\
3867.16752485795	6.01115971142898e-05	\\
3868.14630681818	6.0181926058688e-05	\\
3869.12508877841	5.91480537294385e-05	\\
3870.10387073864	6.0709204524258e-05	\\
3871.08265269886	6.00546647916228e-05	\\
3872.06143465909	6.05761981514232e-05	\\
3873.04021661932	6.17389888665748e-05	\\
3874.01899857955	5.94317407569686e-05	\\
3874.99778053977	6.15854293776165e-05	\\
3875.9765625	6.18594587691485e-05	\\
3876.95534446023	6.15262403343495e-05	\\
3877.93412642045	6.21334887181955e-05	\\
3878.91290838068	6.19078224637079e-05	\\
3879.89169034091	6.21918675798754e-05	\\
3880.87047230114	6.20686614387769e-05	\\
3881.84925426136	6.26168289588097e-05	\\
3882.82803622159	6.43694503572956e-05	\\
3883.80681818182	6.41435827715091e-05	\\
3884.78560014205	6.34891633154149e-05	\\
3885.76438210227	6.30208507078287e-05	\\
3886.7431640625	6.30721905880881e-05	\\
3887.72194602273	6.47331809457218e-05	\\
3888.70072798295	6.35015586981689e-05	\\
3889.67950994318	6.32506869847685e-05	\\
3890.65829190341	6.3904121615808e-05	\\
3891.63707386364	6.51530544226691e-05	\\
3892.61585582386	6.38409856355751e-05	\\
3893.59463778409	6.41759437103009e-05	\\
3894.57341974432	6.43538192906983e-05	\\
3895.55220170455	6.38705540536707e-05	\\
3896.53098366477	6.39766160998788e-05	\\
3897.509765625	6.47865381028853e-05	\\
3898.48854758523	6.58173727015456e-05	\\
3899.46732954545	6.26380371949835e-05	\\
3900.44611150568	6.47769702490694e-05	\\
3901.42489346591	6.41274913744432e-05	\\
3902.40367542614	6.68854976997342e-05	\\
3903.38245738636	6.62506742727579e-05	\\
3904.36123934659	6.42134562985696e-05	\\
3905.34002130682	6.55737172892479e-05	\\
3906.31880326705	6.55451071709775e-05	\\
3907.29758522727	6.82464094800497e-05	\\
3908.2763671875	6.63191365065156e-05	\\
3909.25514914773	6.79240030828299e-05	\\
3910.23393110795	6.81023501189834e-05	\\
3911.21271306818	6.63760254710768e-05	\\
3912.19149502841	6.49587935410592e-05	\\
3913.17027698864	6.6258991004127e-05	\\
3914.14905894886	6.84704563290467e-05	\\
3915.12784090909	6.92434134131104e-05	\\
3916.10662286932	6.81526397925892e-05	\\
3917.08540482955	6.75300938000824e-05	\\
3918.06418678977	6.71094468799104e-05	\\
3919.04296875	6.9111586500224e-05	\\
3920.02175071023	6.72741092147798e-05	\\
3921.00053267045	6.98962800626751e-05	\\
3921.97931463068	7.08768620332816e-05	\\
3922.95809659091	6.89610124178354e-05	\\
3923.93687855114	7.12455329284254e-05	\\
3924.91566051136	6.8438557324279e-05	\\
3925.89444247159	6.96697277981556e-05	\\
3926.87322443182	6.95570432589401e-05	\\
3927.85200639205	6.99691463609894e-05	\\
3928.83078835227	6.95706102860186e-05	\\
3929.8095703125	6.80184636721267e-05	\\
3930.78835227273	7.00705193689106e-05	\\
3931.76713423295	6.88650032439394e-05	\\
3932.74591619318	7.17761399199074e-05	\\
3933.72469815341	7.21365607118423e-05	\\
3934.70348011364	7.11904132988697e-05	\\
3935.68226207386	7.02582692974255e-05	\\
3936.66104403409	7.11604221580139e-05	\\
3937.63982599432	6.85511181959022e-05	\\
3938.61860795455	7.17076833151739e-05	\\
3939.59738991477	7.23443055072208e-05	\\
3940.576171875	6.95060026038409e-05	\\
3941.55495383523	7.15159895770461e-05	\\
3942.53373579545	7.26243960660476e-05	\\
3943.51251775568	7.39033111462678e-05	\\
3944.49129971591	7.11475175884367e-05	\\
3945.47008167614	7.20246141635164e-05	\\
3946.44886363636	7.20679533931406e-05	\\
3947.42764559659	7.33923217938584e-05	\\
3948.40642755682	7.49401248751328e-05	\\
3949.38520951705	7.37152593667239e-05	\\
3950.36399147727	7.26182991808214e-05	\\
3951.3427734375	7.34942867408595e-05	\\
3952.32155539773	7.41652747090902e-05	\\
3953.30033735795	7.43672323248137e-05	\\
3954.27911931818	7.24146736118259e-05	\\
3955.25790127841	7.30328117951577e-05	\\
3956.23668323864	7.59372987231013e-05	\\
3957.21546519886	7.55921913348682e-05	\\
3958.19424715909	7.56425435419067e-05	\\
3959.17302911932	7.46511783737835e-05	\\
3960.15181107955	7.45075565895751e-05	\\
3961.13059303977	7.56765261636045e-05	\\
3962.109375	7.54830706489378e-05	\\
3963.08815696023	7.46834937295551e-05	\\
3964.06693892045	7.44640805179689e-05	\\
3965.04572088068	7.47244277468782e-05	\\
3966.02450284091	7.45264859307037e-05	\\
3967.00328480114	7.61766133128395e-05	\\
3967.98206676136	7.65000875449297e-05	\\
3968.96084872159	7.81998276162509e-05	\\
3969.93963068182	7.67086995367814e-05	\\
3970.91841264205	7.69978320792884e-05	\\
3971.89719460227	7.72355161710822e-05	\\
3972.8759765625	8.0147824206918e-05	\\
3973.85475852273	8.01796408974672e-05	\\
3974.83354048295	7.70992087670022e-05	\\
3975.81232244318	7.97021642927456e-05	\\
3976.79110440341	8.00557142594143e-05	\\
3977.76988636364	7.7572796664847e-05	\\
3978.74866832386	7.85603443092131e-05	\\
3979.72745028409	7.92494066678814e-05	\\
3980.70623224432	8.02511771670344e-05	\\
3981.68501420455	7.93269160608016e-05	\\
3982.66379616477	8.06977828421364e-05	\\
3983.642578125	7.95867614727026e-05	\\
3984.62136008523	8.22685067460147e-05	\\
3985.60014204545	7.93352118358595e-05	\\
3986.57892400568	7.96822392197581e-05	\\
3987.55770596591	8.18903973146173e-05	\\
3988.53648792614	7.9441776873773e-05	\\
3989.51526988636	8.10190238104072e-05	\\
3990.49405184659	8.29950759173754e-05	\\
3991.47283380682	8.19251135239258e-05	\\
3992.45161576705	7.99796896969027e-05	\\
3993.43039772727	8.31009501027916e-05	\\
3994.4091796875	8.25299282754812e-05	\\
3995.38796164773	8.21473097992736e-05	\\
3996.36674360795	8.00931820191618e-05	\\
3997.34552556818	7.87720208765103e-05	\\
3998.32430752841	8.00908455477029e-05	\\
3999.30308948864	8.33774162190397e-05	\\
4000.28187144886	8.34597993453034e-05	\\
4001.26065340909	8.11108116448971e-05	\\
4002.23943536932	8.19779539327122e-05	\\
4003.21821732955	8.42118505333474e-05	\\
4004.19699928977	8.26627031210677e-05	\\
4005.17578125	8.4509633438105e-05	\\
4006.15456321023	8.48413860808118e-05	\\
4007.13334517045	7.90606939420351e-05	\\
4008.11212713068	8.23619702437981e-05	\\
4009.09090909091	8.43409642144639e-05	\\
4010.06969105114	8.5029113162063e-05	\\
4011.04847301136	8.2076859981663e-05	\\
4012.02725497159	8.46268607532077e-05	\\
4013.00603693182	8.33144797523895e-05	\\
4013.98481889205	8.2849595159104e-05	\\
4014.96360085227	8.34170786152987e-05	\\
4015.9423828125	8.10346247137425e-05	\\
4016.92116477273	8.36888337275018e-05	\\
4017.89994673295	8.34723993940067e-05	\\
4018.87872869318	8.27385141031031e-05	\\
4019.85751065341	8.25821069802212e-05	\\
4020.83629261364	8.46305195430656e-05	\\
4021.81507457386	8.44789899709908e-05	\\
4022.79385653409	8.53707440776992e-05	\\
4023.77263849432	8.3346062273819e-05	\\
4024.75142045455	8.40139088013035e-05	\\
4025.73020241477	8.55224898788702e-05	\\
4026.708984375	8.28879436062494e-05	\\
4027.68776633523	8.30664500094048e-05	\\
4028.66654829545	8.4152462581581e-05	\\
4029.64533025568	8.33339172871785e-05	\\
4030.62411221591	8.2110886161721e-05	\\
4031.60289417614	8.68263870387194e-05	\\
4032.58167613636	8.48433847730716e-05	\\
4033.56045809659	8.52034649379179e-05	\\
4034.53924005682	8.60548644953469e-05	\\
4035.51802201705	8.20249279055013e-05	\\
4036.49680397727	8.52371732106131e-05	\\
4037.4755859375	8.42141672721122e-05	\\
4038.45436789773	8.35675406067719e-05	\\
4039.43314985795	8.54254764886709e-05	\\
4040.41193181818	8.4225645157435e-05	\\
4041.39071377841	8.6597924698581e-05	\\
4042.36949573864	8.30834804370537e-05	\\
4043.34827769886	8.3860155605107e-05	\\
4044.32705965909	8.58258421509903e-05	\\
4045.30584161932	8.4143915928939e-05	\\
4046.28462357955	8.45162229637547e-05	\\
4047.26340553977	8.61833111903191e-05	\\
4048.2421875	8.46266607460645e-05	\\
4049.22096946023	8.36133009817181e-05	\\
4050.19975142045	8.33446726209941e-05	\\
4051.17853338068	8.55503590727612e-05	\\
4052.15731534091	8.25659668699362e-05	\\
4053.13609730114	8.43527172080354e-05	\\
4054.11487926136	8.25859477041144e-05	\\
4055.09366122159	8.31988846480278e-05	\\
4056.07244318182	8.44702497435857e-05	\\
4057.05122514205	8.54619889066205e-05	\\
4058.03000710227	8.2455346225528e-05	\\
4059.0087890625	8.26756213330359e-05	\\
4059.98757102273	8.1729190947633e-05	\\
4060.96635298295	8.3313550072138e-05	\\
4061.94513494318	8.4363654238803e-05	\\
4062.92391690341	8.31025899259055e-05	\\
4063.90269886364	8.51887198156805e-05	\\
4064.88148082386	8.32809295181237e-05	\\
4065.86026278409	8.29887448230636e-05	\\
4066.83904474432	8.16917550785197e-05	\\
4067.81782670455	8.21467185572362e-05	\\
4068.79660866477	8.53761186406746e-05	\\
4069.775390625	8.11256011601374e-05	\\
4070.75417258523	8.32661753080957e-05	\\
4071.73295454545	8.20423268754095e-05	\\
4072.71173650568	8.20761810450682e-05	\\
4073.69051846591	8.32858077380804e-05	\\
4074.66930042614	8.15389814209086e-05	\\
4075.64808238636	8.01587896648959e-05	\\
4076.62686434659	8.21813865271165e-05	\\
4077.60564630682	8.24359204670766e-05	\\
4078.58442826705	8.21025682506039e-05	\\
4079.56321022727	8.08251759066008e-05	\\
4080.5419921875	8.01944546788188e-05	\\
4081.52077414773	8.1689005549291e-05	\\
4082.49955610795	8.10128348293879e-05	\\
4083.47833806818	8.18920351957306e-05	\\
4084.45712002841	8.09264730384322e-05	\\
4085.43590198864	8.09446974430963e-05	\\
4086.41468394886	8.01646434176957e-05	\\
4087.39346590909	8.10000809630277e-05	\\
4088.37224786932	8.01782105495074e-05	\\
4089.35102982955	8.2711673574383e-05	\\
4090.32981178977	8.06377069589558e-05	\\
4091.30859375	8.04328452364007e-05	\\
4092.28737571023	7.98886916667405e-05	\\
4093.26615767045	8.02240031149662e-05	\\
4094.24493963068	8.07906002317808e-05	\\
4095.22372159091	7.77686811287096e-05	\\
4096.20250355114	8.02418160360471e-05	\\
4097.18128551136	7.9028419869565e-05	\\
4098.16006747159	8.04877887394644e-05	\\
4099.13884943182	7.88018633853852e-05	\\
4100.11763139205	7.93160566799202e-05	\\
4101.09641335227	7.78330392731703e-05	\\
4102.0751953125	7.94069595858111e-05	\\
4103.05397727273	8.10893139587971e-05	\\
4104.03275923295	7.9736129647678e-05	\\
4105.01154119318	8.02057708979622e-05	\\
4105.99032315341	8.00685596324788e-05	\\
4106.96910511364	8.00728853713656e-05	\\
4107.94788707386	8.21540460336231e-05	\\
4108.92666903409	7.93557420443856e-05	\\
4109.90545099432	7.80448184103952e-05	\\
4110.88423295455	8.26230905766748e-05	\\
4111.86301491477	8.11901039068873e-05	\\
4112.841796875	8.20927954879315e-05	\\
4113.82057883523	8.04236961430153e-05	\\
4114.79936079545	8.25219929187312e-05	\\
4115.77814275568	8.16500574923275e-05	\\
4116.75692471591	8.16125277908602e-05	\\
4117.73570667614	8.13769809514199e-05	\\
4118.71448863636	8.22431702876839e-05	\\
4119.69327059659	7.85244498032854e-05	\\
4120.67205255682	8.17513683229553e-05	\\
4121.65083451705	8.42280867061949e-05	\\
4122.62961647727	8.12631101976167e-05	\\
4123.6083984375	8.20022851031941e-05	\\
4124.58718039773	8.23527225476368e-05	\\
4125.56596235795	8.15426996080436e-05	\\
4126.54474431818	7.79370240317722e-05	\\
4127.52352627841	8.47513104013074e-05	\\
4128.50230823864	8.32840187204088e-05	\\
4129.48109019886	8.29474732447441e-05	\\
4130.45987215909	8.26747482406146e-05	\\
4131.43865411932	8.57646258671459e-05	\\
4132.41743607955	8.37190301752474e-05	\\
4133.39621803977	8.32522972659722e-05	\\
4134.375	8.3516376992899e-05	\\
4135.35378196023	8.30719968446397e-05	\\
4136.33256392045	8.35321325915019e-05	\\
4137.31134588068	8.41691362215713e-05	\\
4138.29012784091	8.44541071199009e-05	\\
4139.26890980114	8.48891069961936e-05	\\
4140.24769176136	8.60859231602501e-05	\\
4141.22647372159	8.63243495418266e-05	\\
4142.20525568182	8.15918249677413e-05	\\
4143.18403764205	8.67846805758007e-05	\\
4144.16281960227	8.69138591251972e-05	\\
4145.1416015625	8.76518774789179e-05	\\
4146.12038352273	8.46666164074958e-05	\\
4147.09916548295	8.45118588980322e-05	\\
4148.07794744318	8.4943500731019e-05	\\
4149.05672940341	8.44331136072075e-05	\\
4150.03551136364	8.65135882814168e-05	\\
4151.01429332386	8.56095794523065e-05	\\
4151.99307528409	8.810324884098e-05	\\
4152.97185724432	8.65618098601793e-05	\\
4153.95063920455	8.6942526466015e-05	\\
4154.92942116477	8.82066525993378e-05	\\
4155.908203125	8.68958180493727e-05	\\
4156.88698508523	8.66747851012112e-05	\\
4157.86576704545	9.10981984066683e-05	\\
4158.84454900568	8.92331289988592e-05	\\
4159.82333096591	8.81709120046733e-05	\\
4160.80211292614	8.84856323603387e-05	\\
4161.78089488636	8.77602975926398e-05	\\
4162.75967684659	9.04452622480571e-05	\\
4163.73845880682	8.98422649766339e-05	\\
4164.71724076705	8.82966758673392e-05	\\
4165.69602272727	9.12374134725648e-05	\\
4166.6748046875	8.99581856020288e-05	\\
4167.65358664773	9.07113363689451e-05	\\
4168.63236860795	9.00403951090457e-05	\\
4169.61115056818	9.18549225168089e-05	\\
4170.58993252841	9.18194704120757e-05	\\
4171.56871448864	9.14698108932786e-05	\\
4172.54749644886	9.32346912419189e-05	\\
4173.52627840909	9.19555914249156e-05	\\
4174.50506036932	9.54138476269107e-05	\\
4175.48384232955	9.07098411440527e-05	\\
4176.46262428977	9.36042083857349e-05	\\
4177.44140625	9.36299374373706e-05	\\
4178.42018821023	8.88029073679218e-05	\\
4179.39897017045	9.14284764248923e-05	\\
4180.37775213068	9.04902834697649e-05	\\
4181.35653409091	9.15131449622112e-05	\\
4182.33531605114	9.1091362489808e-05	\\
4183.31409801136	9.35118215892101e-05	\\
4184.29287997159	9.15340191190911e-05	\\
4185.27166193182	9.18609036638891e-05	\\
4186.25044389205	9.2790031044821e-05	\\
4187.22922585227	9.3588608223492e-05	\\
4188.2080078125	9.09454952948543e-05	\\
4189.18678977273	9.16509457389509e-05	\\
4190.16557173295	9.40478113060824e-05	\\
4191.14435369318	9.44188386505487e-05	\\
4192.12313565341	9.42454262947074e-05	\\
4193.10191761364	9.16284367440943e-05	\\
4194.08069957386	9.37400085294248e-05	\\
4195.05948153409	9.68783218652677e-05	\\
4196.03826349432	9.49073944254055e-05	\\
4197.01704545455	9.44863591396585e-05	\\
4197.99582741477	9.67866614943049e-05	\\
4198.974609375	9.45310423661782e-05	\\
4199.95339133523	9.13577918488113e-05	\\
4200.93217329545	9.50474983927925e-05	\\
4201.91095525568	9.64557604857037e-05	\\
4202.88973721591	9.55495067672394e-05	\\
4203.86851917614	9.51759872027832e-05	\\
4204.84730113636	9.41061861794995e-05	\\
4205.82608309659	9.45350966974862e-05	\\
4206.80486505682	9.54686069286702e-05	\\
4207.78364701705	9.44639733164188e-05	\\
4208.76242897727	9.5857045205885e-05	\\
4209.7412109375	9.66220962304747e-05	\\
4210.71999289773	9.37216996890775e-05	\\
4211.69877485795	9.58820042836269e-05	\\
4212.67755681818	9.614865654796e-05	\\
4213.65633877841	9.50445680964885e-05	\\
4214.63512073864	9.6998102270207e-05	\\
4215.61390269886	9.51103061200919e-05	\\
4216.59268465909	9.54371317972471e-05	\\
4217.57146661932	9.49580895128478e-05	\\
4218.55024857955	9.62828147104835e-05	\\
4219.52903053977	9.60655273974173e-05	\\
4220.5078125	9.56177111320597e-05	\\
4221.48659446023	9.76200237760917e-05	\\
4222.46537642045	9.65688440736961e-05	\\
4223.44415838068	9.7171858644423e-05	\\
4224.42294034091	9.74948454800181e-05	\\
4225.40172230114	9.73283601887682e-05	\\
4226.38050426136	9.4322587166922e-05	\\
4227.35928622159	9.67006789767166e-05	\\
4228.33806818182	0.000102090218485394	\\
4229.31685014205	9.44618061554e-05	\\
4230.29563210227	9.99496939723589e-05	\\
4231.2744140625	9.90435955034265e-05	\\
4232.25319602273	9.80193405843054e-05	\\
4233.23197798295	9.64985520547917e-05	\\
4234.21075994318	9.96960719063801e-05	\\
4235.18954190341	9.66146359332016e-05	\\
4236.16832386364	9.86918286517235e-05	\\
4237.14710582386	9.85960283660716e-05	\\
4238.12588778409	9.70433365971493e-05	\\
4239.10466974432	9.87772886363823e-05	\\
4240.08345170455	9.9553993817995e-05	\\
4241.06223366477	9.92935965474252e-05	\\
4242.041015625	9.75335017470263e-05	\\
4243.01979758523	9.92050911223334e-05	\\
4243.99857954545	0.000100042648337342	\\
4244.97736150568	9.8538922638206e-05	\\
4245.95614346591	9.96894733316131e-05	\\
4246.93492542614	0.000102276809660598	\\
4247.91370738636	9.8940075279762e-05	\\
4248.89248934659	0.000100251062921856	\\
4249.87127130682	0.000101414867665822	\\
4250.85005326705	0.000100603918527195	\\
4251.82883522727	9.87424058329013e-05	\\
4252.8076171875	0.000100273725203492	\\
4253.78639914773	0.000100430591877105	\\
4254.76518110795	0.000101993533293386	\\
4255.74396306818	9.95837445644927e-05	\\
4256.72274502841	0.000101318385426305	\\
4257.70152698864	0.000100929268990175	\\
4258.68030894886	0.0001000214820181	\\
4259.65909090909	0.000101346983749047	\\
4260.63787286932	0.000103140459275127	\\
4261.61665482955	0.000100010821672717	\\
4262.59543678977	0.000102066210421538	\\
4263.57421875	0.000100869248546103	\\
4264.55300071023	0.000102220873349984	\\
4265.53178267045	0.0001011018433553	\\
4266.51056463068	0.000102506273081721	\\
4267.48934659091	0.000100841052973901	\\
4268.46812855114	0.000102767552826459	\\
4269.44691051136	0.000102125053177691	\\
4270.42569247159	0.00010204286378251	\\
4271.40447443182	0.000102570260184354	\\
4272.38325639205	0.000103363597769147	\\
4273.36203835227	0.000104329852586865	\\
4274.3408203125	0.000102605069686563	\\
4275.31960227273	0.000102946864006097	\\
4276.29838423295	9.81742564384437e-05	\\
4277.27716619318	0.000104556834813828	\\
4278.25594815341	0.000102324021832206	\\
4279.23473011364	0.000102557759005157	\\
4280.21351207386	0.000100253555650043	\\
4281.19229403409	0.000105646688372463	\\
4282.17107599432	0.000102417161366814	\\
4283.14985795455	0.000101253542932227	\\
4284.12863991477	0.000104416019481034	\\
4285.107421875	0.000102929763906183	\\
4286.08620383523	0.00010447060059868	\\
4287.06498579545	0.000102329531417306	\\
4288.04376775568	0.000105002081586484	\\
4289.02254971591	0.00010547705917011	\\
4290.00133167614	0.000104979579704251	\\
4290.98011363636	0.000103802665404053	\\
4291.95889559659	0.000105673604045838	\\
4292.93767755682	0.00010416398677838	\\
4293.91645951705	0.000103502693146775	\\
4294.89524147727	0.000105465115768709	\\
4295.8740234375	0.000102630798502275	\\
4296.85280539773	0.000106459186042841	\\
4297.83158735795	0.000105500677178496	\\
4298.81036931818	0.000104305662859382	\\
4299.78915127841	0.000104384863286092	\\
4300.76793323864	0.000105858879724743	\\
4301.74671519886	0.000104809535682882	\\
4302.72549715909	0.000105279094887237	\\
4303.70427911932	0.000105499037145655	\\
4304.68306107955	0.000103950477624467	\\
4305.66184303977	0.000105388158474203	\\
4306.640625	0.000106734954614597	\\
4307.61940696023	0.000102513659422031	\\
4308.59818892045	0.00010545459005973	\\
4309.57697088068	0.000103804971642891	\\
4310.55575284091	0.000103795972042148	\\
4311.53453480114	0.000105332878270688	\\
4312.51331676136	0.000107017068800774	\\
4313.49209872159	0.000104155550020123	\\
4314.47088068182	0.000104092396857791	\\
4315.44966264205	0.000106451761840359	\\
4316.42844460227	0.000106337929886803	\\
4317.4072265625	0.000107756719530542	\\
4318.38600852273	0.000106551668926921	\\
4319.36479048295	0.000104483205704446	\\
4320.34357244318	0.000107839900873619	\\
4321.32235440341	0.000103722550685247	\\
4322.30113636364	0.000106664419276749	\\
4323.27991832386	0.000107972641268518	\\
4324.25870028409	0.000105133428304788	\\
4325.23748224432	0.000105975212439515	\\
4326.21626420455	0.000107265577924873	\\
4327.19504616477	0.000105953805881711	\\
4328.173828125	0.000106797794637051	\\
4329.15261008523	0.000108006438198913	\\
4330.13139204545	0.000108675567616438	\\
4331.11017400568	0.000107868001212899	\\
4332.08895596591	0.000104737766949775	\\
4333.06773792614	0.000110497747036147	\\
4334.04651988636	0.00010888883118965	\\
4335.02530184659	0.000108990649220755	\\
4336.00408380682	0.000108773214356988	\\
4336.98286576705	0.00010800510107298	\\
4337.96164772727	0.000108307325940812	\\
4338.9404296875	0.000107886493479101	\\
4339.91921164773	0.000111023935306734	\\
4340.89799360795	0.000112418700990077	\\
4341.87677556818	0.000111308006983966	\\
4342.85555752841	0.000109192809490091	\\
4343.83433948864	0.000111749917743783	\\
4344.81312144886	0.000111791350899336	\\
4345.79190340909	0.000112198117285697	\\
4346.77068536932	0.00011103653707154	\\
4347.74946732955	0.000111067027558041	\\
4348.72824928977	0.000108860069075982	\\
4349.70703125	0.000109405541212874	\\
4350.68581321023	0.000111618274243707	\\
4351.66459517045	0.000112624620265348	\\
4352.64337713068	0.000111346425287563	\\
4353.62215909091	0.000111003649089038	\\
4354.60094105114	0.000111602832727323	\\
4355.57972301136	0.000111178555228406	\\
4356.55850497159	0.000113135756733907	\\
4357.53728693182	0.000114346242664493	\\
4358.51606889205	0.000112088192797291	\\
4359.49485085227	0.000111832883995925	\\
4360.4736328125	0.000112622813331686	\\
4361.45241477273	0.000113538683890425	\\
4362.43119673295	0.000114940154187959	\\
4363.40997869318	0.000112897606007324	\\
4364.38876065341	0.000113901983597693	\\
4365.36754261364	0.000113721197753494	\\
4366.34632457386	0.000113872413607742	\\
4367.32510653409	0.000112688245288988	\\
4368.30388849432	0.000112451274189161	\\
4369.28267045455	0.000114918666896762	\\
4370.26145241477	0.000115214748615435	\\
4371.240234375	0.000112895057360266	\\
4372.21901633523	0.0001142097728549	\\
4373.19779829545	0.000113424912589705	\\
4374.17658025568	0.000113327326764371	\\
4375.15536221591	0.000116227561284146	\\
4376.13414417614	0.000114588958109801	\\
4377.11292613636	0.000116424440214874	\\
4378.09170809659	0.000114034250357541	\\
4379.07049005682	0.000115604966910885	\\
4380.04927201705	0.000117063247110797	\\
4381.02805397727	0.000117199673584631	\\
4382.0068359375	0.000115680523492215	\\
4382.98561789773	0.000113168461174828	\\
4383.96439985795	0.000116142271876338	\\
4384.94318181818	0.000113998395736868	\\
4385.92196377841	0.000117208749520153	\\
4386.90074573864	0.000115579979666637	\\
4387.87952769886	0.00011594549414213	\\
4388.85830965909	0.000115713141453414	\\
4389.83709161932	0.000115446647962266	\\
4390.81587357955	0.00011737745629782	\\
4391.79465553977	0.000115970278142576	\\
4392.7734375	0.000116652147784752	\\
4393.75221946023	0.000116048647821448	\\
4394.73100142045	0.000115648822247088	\\
4395.70978338068	0.000117796044782814	\\
4396.68856534091	0.000117880583455708	\\
4397.66734730114	0.000118774554421231	\\
4398.64612926136	0.000116883990317412	\\
4399.62491122159	0.000120203527003992	\\
4400.60369318182	0.000117359318926711	\\
4401.58247514205	0.000118111033171651	\\
4402.56125710227	0.000117142275650159	\\
4403.5400390625	0.000119803605579994	\\
4404.51882102273	0.000115924746924495	\\
4405.49760298295	0.000116600636560717	\\
4406.47638494318	0.00011693961410382	\\
4407.45516690341	0.00011553944908043	\\
4408.43394886364	0.000115473194258742	\\
4409.41273082386	0.000118795578472689	\\
4410.39151278409	0.000117953143090697	\\
4411.37029474432	0.000118517317022124	\\
4412.34907670455	0.000116746577619735	\\
4413.32785866477	0.000117126621682799	\\
4414.306640625	0.000115871362667795	\\
4415.28542258523	0.00011796838343482	\\
4416.26420454545	0.000115954630994544	\\
4417.24298650568	0.000117377244556358	\\
4418.22176846591	0.000117254458986122	\\
4419.20055042614	0.000116972246756642	\\
4420.17933238636	0.000118379879000927	\\
4421.15811434659	0.000115408785695141	\\
4422.13689630682	0.000117486867331002	\\
4423.11567826705	0.000118407652496082	\\
4424.09446022727	0.000116308171671504	\\
4425.0732421875	0.000114784144668184	\\
4426.05202414773	0.000119092151817225	\\
4427.03080610795	0.00011704243258302	\\
4428.00958806818	0.000116266346512261	\\
4428.98837002841	0.000117222229179482	\\
4429.96715198864	0.000116978603341468	\\
4430.94593394886	0.000115717257578117	\\
4431.92471590909	0.000117979070514969	\\
4432.90349786932	0.000119382251292734	\\
4433.88227982955	0.000119475754037835	\\
4434.86106178977	0.000117504628611855	\\
4435.83984375	0.000116896930355841	\\
4436.81862571023	0.000116029079967003	\\
4437.79740767045	0.000118424319878456	\\
4438.77618963068	0.000118585482806887	\\
4439.75497159091	0.000115658200396103	\\
4440.73375355114	0.000115763501981738	\\
4441.71253551136	0.000118488496672291	\\
4442.69131747159	0.000116363884691725	\\
4443.67009943182	0.000118146031546844	\\
4444.64888139205	0.000118924513233897	\\
4445.62766335227	0.000117345163072253	\\
4446.6064453125	0.000117788894104643	\\
4447.58522727273	0.000114862015918878	\\
4448.56400923295	0.000114931297447742	\\
4449.54279119318	0.000118114671012593	\\
4450.52157315341	0.000115735450623024	\\
4451.50035511364	0.000114888797497631	\\
4452.47913707386	0.000115739690105399	\\
4453.45791903409	0.000117504760387166	\\
4454.43670099432	0.000115616787528888	\\
4455.41548295455	0.000116201878998926	\\
4456.39426491477	0.000115775453674437	\\
4457.373046875	0.000115359912610017	\\
4458.35182883523	0.000113941469766448	\\
4459.33061079545	0.000119461258593652	\\
4460.30939275568	0.000114907263120915	\\
4461.28817471591	0.000116211600847481	\\
4462.26695667614	0.000115786345421428	\\
4463.24573863636	0.000117420947115017	\\
4464.22452059659	0.000116287297150036	\\
4465.20330255682	0.000117117759373995	\\
4466.18208451705	0.000117015651732856	\\
4467.16086647727	0.000116534133238427	\\
4468.1396484375	0.000116815332644877	\\
4469.11843039773	0.00011589961701628	\\
4470.09721235795	0.000115205248592002	\\
4471.07599431818	0.000116429864742275	\\
4472.05477627841	0.00011675076719947	\\
4473.03355823864	0.000114565142261902	\\
4474.01234019886	0.000117218212407508	\\
4474.99112215909	0.00011517123984185	\\
4475.96990411932	0.000115235278590141	\\
4476.94868607955	0.00011547162755643	\\
4477.92746803977	0.000118979898859924	\\
4478.90625	0.000115687464700858	\\
4479.88503196023	0.000116678028944413	\\
4480.86381392045	0.000117773505048332	\\
4481.84259588068	0.000118323980650882	\\
4482.82137784091	0.000116529712888743	\\
4483.80015980114	0.000116439820352986	\\
4484.77894176136	0.000115685702794142	\\
4485.75772372159	0.000117069754999531	\\
4486.73650568182	0.000115411029892159	\\
4487.71528764205	0.00011523880591288	\\
4488.69406960227	0.000117583549011584	\\
4489.6728515625	0.000116658210258158	\\
4490.65163352273	0.00011367961357795	\\
4491.63041548295	0.000115460291076671	\\
4492.60919744318	0.000115569604957036	\\
4493.58797940341	0.00011536113376381	\\
4494.56676136364	0.000117013439135165	\\
4495.54554332386	0.000113890272256524	\\
4496.52432528409	0.0001169697764403	\\
4497.50310724432	0.000114559593437624	\\
4498.48188920455	0.000115277780192009	\\
4499.46067116477	0.000115680302353912	\\
4500.439453125	0.000116129672762871	\\
4501.41823508523	0.000114584530665034	\\
4502.39701704545	0.000116091831045409	\\
4503.37579900568	0.000115582732361417	\\
4504.35458096591	0.00011422402363764	\\
4505.33336292614	0.000116016990298484	\\
4506.31214488636	0.000114544512864584	\\
4507.29092684659	0.00011576664623277	\\
4508.26970880682	0.000114292463613177	\\
4509.24849076705	0.000115103133167886	\\
4510.22727272727	0.000114233868615286	\\
4511.2060546875	0.000114050664504832	\\
4512.18483664773	0.000106023306722551	\\
4513.16361860795	0.000115738569750853	\\
4514.14240056818	0.000115892188277425	\\
4515.12118252841	0.000114995484131727	\\
4516.09996448864	0.000114670726323816	\\
4517.07874644886	0.000113713284365649	\\
4518.05752840909	0.000115076211734674	\\
4519.03631036932	0.000112725537494584	\\
4520.01509232955	0.00011378869370202	\\
4520.99387428977	0.000113503420429776	\\
4521.97265625	0.000113763916701683	\\
4522.95143821023	0.000113736049831925	\\
4523.93022017045	0.000113690056513307	\\
4524.90900213068	0.000112276209739155	\\
4525.88778409091	0.000114735131236733	\\
4526.86656605114	0.000110815963327824	\\
4527.84534801136	0.000112657391505301	\\
4528.82412997159	0.000111388688431173	\\
4529.80291193182	0.000114361742159585	\\
4530.78169389205	0.000114009705828046	\\
4531.76047585227	0.000113762160314622	\\
4532.7392578125	0.000111802086602473	\\
4533.71803977273	0.000112558066362272	\\
4534.69682173295	0.000112449445864662	\\
4535.67560369318	0.000108200885555198	\\
4536.65438565341	0.000112216561630427	\\
4537.63316761364	0.000111381730233063	\\
4538.61194957386	0.000111887541202859	\\
4539.59073153409	0.000112521286708167	\\
4540.56951349432	0.000110659850049474	\\
4541.54829545455	0.00010890303128519	\\
4542.52707741477	0.000110749688662788	\\
4543.505859375	0.000108959312554655	\\
4544.48464133523	0.000108015831333234	\\
4545.46342329545	0.000108360918273933	\\
4546.44220525568	0.000110566815730434	\\
4547.42098721591	0.000111064289466998	\\
4548.39976917614	0.000108140415586013	\\
4549.37855113636	0.000108333969626382	\\
4550.35733309659	0.000107499475193607	\\
4551.33611505682	0.00010783868759807	\\
4552.31489701705	0.000107857469056355	\\
4553.29367897727	0.000106755994220823	\\
4554.2724609375	0.00010733957184443	\\
4555.25124289773	0.000109782710462064	\\
4556.23002485795	0.000107136810836517	\\
4557.20880681818	0.000109751605578799	\\
4558.18758877841	0.000108179111606423	\\
4559.16637073864	0.000107129039108578	\\
4560.14515269886	0.000104397280485713	\\
4561.12393465909	0.000108271128196476	\\
4562.10271661932	0.000106599369576554	\\
4563.08149857955	0.000106007223219336	\\
4564.06028053977	0.000105440791967707	\\
4565.0390625	0.000105095313383273	\\
4566.01784446023	0.000102693871616359	\\
4566.99662642045	0.000105639404165948	\\
4567.97540838068	0.000105433916306429	\\
4568.95419034091	0.000105484664702518	\\
4569.93297230114	0.000104545074004906	\\
4570.91175426136	0.000103649939110493	\\
4571.89053622159	0.000103110487092495	\\
4572.86931818182	0.000100831366438583	\\
4573.84810014205	0.000103565613351491	\\
4574.82688210227	0.000102634357841504	\\
4575.8056640625	0.000102175447184772	\\
4576.78444602273	0.000100373457476597	\\
4577.76322798295	0.000100088613207254	\\
4578.74200994318	0.000101383037102236	\\
4579.72079190341	0.000101388559880982	\\
4580.69957386364	9.95936744854312e-05	\\
4581.67835582386	9.90636725224955e-05	\\
4582.65713778409	9.87799746112625e-05	\\
4583.63591974432	9.96901816040588e-05	\\
4584.61470170455	9.97090570870323e-05	\\
4585.59348366477	9.83881542824414e-05	\\
4586.572265625	9.69785132525492e-05	\\
4587.55104758523	9.67758627288254e-05	\\
4588.52982954545	0.000100435365971322	\\
4589.50861150568	9.58768288681301e-05	\\
4590.48739346591	9.81525207963668e-05	\\
4591.46617542614	9.65428061733673e-05	\\
4592.44495738636	9.56636995265472e-05	\\
4593.42373934659	9.76956800080182e-05	\\
4594.40252130682	9.35300304551854e-05	\\
4595.38130326705	9.51858070079054e-05	\\
4596.36008522727	9.41749820823683e-05	\\
4597.3388671875	9.46220465296327e-05	\\
4598.31764914773	9.49711702632827e-05	\\
4599.29643110795	9.46047966052453e-05	\\
4600.27521306818	9.7673140997264e-05	\\
4601.25399502841	9.10750127325239e-05	\\
4602.23277698864	9.38659912874759e-05	\\
4603.21155894886	9.1829999815986e-05	\\
4604.19034090909	9.14542568605606e-05	\\
4605.16912286932	9.20060127790896e-05	\\
4606.14790482955	9.3145280899648e-05	\\
4607.12668678977	9.22668444514957e-05	\\
4608.10546875	9.00460604819392e-05	\\
4609.08425071023	9.00826501803106e-05	\\
4610.06303267045	9.02634219564315e-05	\\
4611.04181463068	8.96073362586714e-05	\\
4612.02059659091	9.13759114958552e-05	\\
4612.99937855114	9.04073539784078e-05	\\
4613.97816051136	8.907727656114e-05	\\
4614.95694247159	8.87483268628357e-05	\\
4615.93572443182	8.92310812430002e-05	\\
4616.91450639205	8.86307165536298e-05	\\
4617.89328835227	8.86556451515193e-05	\\
4618.8720703125	8.70849202489646e-05	\\
4619.85085227273	8.92598247825684e-05	\\
4620.82963423295	8.83680632559844e-05	\\
4621.80841619318	8.78471654489813e-05	\\
4622.78719815341	8.72434215363202e-05	\\
4623.76598011364	8.65525932714536e-05	\\
4624.74476207386	8.72632729466422e-05	\\
4625.72354403409	8.57627344000915e-05	\\
4626.70232599432	8.51051830968791e-05	\\
4627.68110795455	8.3098895960179e-05	\\
4628.65988991477	8.4218517459461e-05	\\
4629.638671875	8.75675363010231e-05	\\
4630.61745383523	8.53114310734533e-05	\\
4631.59623579545	8.32633211558157e-05	\\
4632.57501775568	8.47310547007075e-05	\\
4633.55379971591	8.49094009682539e-05	\\
4634.53258167614	8.20840088023009e-05	\\
4635.51136363636	8.54028162440478e-05	\\
};
\addplot [color=blue,solid,forget plot]
  table[row sep=crcr]{
4635.51136363636	8.54028162440478e-05	\\
4636.49014559659	8.41684174135159e-05	\\
4637.46892755682	8.28957911572556e-05	\\
4638.44770951705	8.41426903503069e-05	\\
4639.42649147727	8.60984027576728e-05	\\
4640.4052734375	8.43835782955763e-05	\\
4641.38405539773	8.5811977584048e-05	\\
4642.36283735795	8.35563556057426e-05	\\
4643.34161931818	8.4042584845961e-05	\\
4644.32040127841	8.15482418727486e-05	\\
4645.29918323864	8.48755861696468e-05	\\
4646.27796519886	8.50156989696302e-05	\\
4647.25674715909	8.16524841694673e-05	\\
4648.23552911932	8.26745599466325e-05	\\
4649.21431107955	8.25504145226047e-05	\\
4650.19309303977	8.45070442532507e-05	\\
4651.171875	8.47528083322389e-05	\\
4652.15065696023	8.19462268802957e-05	\\
4653.12943892045	8.26338339035303e-05	\\
4654.10822088068	8.23808431676383e-05	\\
4655.08700284091	8.2826049328088e-05	\\
4656.06578480114	8.116431596942e-05	\\
4657.04456676136	8.00998696123384e-05	\\
4658.02334872159	8.28809285143065e-05	\\
4659.00213068182	8.0367283859298e-05	\\
4659.98091264205	8.19189045419118e-05	\\
4660.95969460227	8.41588360627677e-05	\\
4661.9384765625	8.21256916808492e-05	\\
4662.91725852273	8.1129279623238e-05	\\
4663.89604048295	7.98502886924067e-05	\\
4664.87482244318	7.71510125783426e-05	\\
4665.85360440341	7.86928817762852e-05	\\
4666.83238636364	8.01642190932517e-05	\\
4667.81116832386	8.13961668859202e-05	\\
4668.78995028409	8.05609005105441e-05	\\
4669.76873224432	8.11009361074733e-05	\\
4670.74751420455	7.99306712059653e-05	\\
4671.72629616477	7.85831621235825e-05	\\
4672.705078125	7.80849590317882e-05	\\
4673.68386008523	8.03271302345769e-05	\\
4674.66264204545	7.79792882163949e-05	\\
4675.64142400568	7.97625195036512e-05	\\
4676.62020596591	7.74927511174548e-05	\\
4677.59898792614	7.94283035148707e-05	\\
4678.57776988636	7.87929869975175e-05	\\
4679.55655184659	7.81711893648094e-05	\\
4680.53533380682	7.80881674562189e-05	\\
4681.51411576705	7.94431224906189e-05	\\
4682.49289772727	7.91947892063153e-05	\\
4683.4716796875	7.85516711961244e-05	\\
4684.45046164773	7.91659784470102e-05	\\
4685.42924360795	8.02415654687743e-05	\\
4686.40802556818	8.03570769302255e-05	\\
4687.38680752841	7.97568166073841e-05	\\
4688.36558948864	7.84192764645548e-05	\\
4689.34437144886	8.19319982214875e-05	\\
4690.32315340909	7.79091327875234e-05	\\
4691.30193536932	7.96287365263998e-05	\\
4692.28071732955	7.89139119827396e-05	\\
4693.25949928977	7.92591361498328e-05	\\
4694.23828125	7.81947407037788e-05	\\
4695.21706321023	7.90457008427625e-05	\\
4696.19584517045	8.00138341694581e-05	\\
4697.17462713068	7.69536434193436e-05	\\
4698.15340909091	7.96572154878711e-05	\\
4699.13219105114	7.82476067914691e-05	\\
4700.11097301136	7.86122567974856e-05	\\
4701.08975497159	8.07218037537935e-05	\\
4702.06853693182	7.97100144245498e-05	\\
4703.04731889205	8.08077430539797e-05	\\
4704.02610085227	7.88526292876254e-05	\\
4705.0048828125	7.87718789059426e-05	\\
4705.98366477273	7.83004396841358e-05	\\
4706.96244673295	8.033555999168e-05	\\
4707.94122869318	8.05635024165388e-05	\\
4708.92001065341	7.97504105556154e-05	\\
4709.89879261364	8.01499603626197e-05	\\
4710.87757457386	7.97626552489643e-05	\\
4711.85635653409	8.04854162723825e-05	\\
4712.83513849432	7.86880355039685e-05	\\
4713.81392045455	7.81455404082567e-05	\\
4714.79270241477	8.01927298616046e-05	\\
4715.771484375	7.82087227024885e-05	\\
4716.75026633523	7.83230615698836e-05	\\
4717.72904829545	7.8392275826547e-05	\\
4718.70783025568	7.87399412891552e-05	\\
4719.68661221591	8.05273375610086e-05	\\
4720.66539417614	8.05298074044786e-05	\\
4721.64417613636	7.86590050259861e-05	\\
4722.62295809659	7.82377047965851e-05	\\
4723.60174005682	7.98920230442784e-05	\\
4724.58052201705	7.821645687245e-05	\\
4725.55930397727	8.06052717528779e-05	\\
4726.5380859375	8.02738036695023e-05	\\
4727.51686789773	7.96210309616041e-05	\\
4728.49564985795	7.93235909575052e-05	\\
4729.47443181818	7.96569079220535e-05	\\
4730.45321377841	7.84147805031507e-05	\\
4731.43199573864	8.12403366733659e-05	\\
4732.41077769886	7.88619804572561e-05	\\
4733.38955965909	7.93314686881476e-05	\\
4734.36834161932	7.94032515643833e-05	\\
4735.34712357955	7.84206654952167e-05	\\
4736.32590553977	7.80041593428417e-05	\\
4737.3046875	7.96836825373698e-05	\\
4738.28346946023	7.9922093517473e-05	\\
4739.26225142045	8.05689927126045e-05	\\
4740.24103338068	8.09055283073165e-05	\\
4741.21981534091	8.02184435509749e-05	\\
4742.19859730114	7.98113142193742e-05	\\
4743.17737926136	8.01597454364813e-05	\\
4744.15616122159	7.85873015397949e-05	\\
4745.13494318182	8.02876166061601e-05	\\
4746.11372514205	7.92093784705723e-05	\\
4747.09250710227	7.91849525056025e-05	\\
4748.0712890625	7.98099192498651e-05	\\
4749.05007102273	7.98399432356991e-05	\\
4750.02885298295	8.0956563729041e-05	\\
4751.00763494318	7.84927856449412e-05	\\
4751.98641690341	7.84229271674697e-05	\\
4752.96519886364	7.87842387032001e-05	\\
4753.94398082386	8.12933887130455e-05	\\
4754.92276278409	7.74434048287419e-05	\\
4755.90154474432	7.87221657952714e-05	\\
4756.88032670455	7.68344723358202e-05	\\
4757.85910866477	7.98097850850931e-05	\\
4758.837890625	7.88290224397229e-05	\\
4759.81667258523	7.79285054082612e-05	\\
4760.79545454545	7.71832100820945e-05	\\
4761.77423650568	7.82868906488718e-05	\\
4762.75301846591	7.99911108239375e-05	\\
4763.73180042614	8.20541770887033e-05	\\
4764.71058238636	7.87866143225407e-05	\\
4765.68936434659	7.70199012143733e-05	\\
4766.66814630682	7.78927469995796e-05	\\
4767.64692826705	8.02813292218796e-05	\\
4768.62571022727	7.91739775805434e-05	\\
4769.6044921875	7.99426727448506e-05	\\
4770.58327414773	8.07579082086085e-05	\\
4771.56205610795	7.89552208084173e-05	\\
4772.54083806818	7.79003183134277e-05	\\
4773.51962002841	7.90947229608477e-05	\\
4774.49840198864	7.96453935833462e-05	\\
4775.47718394886	8.13797159568189e-05	\\
4776.45596590909	7.974176928191e-05	\\
4777.43474786932	8.03572456204448e-05	\\
4778.41352982955	7.96767142678165e-05	\\
4779.39231178977	7.88242469927032e-05	\\
4780.37109375	7.71350069475418e-05	\\
4781.34987571023	7.87088387887937e-05	\\
4782.32865767045	7.8349573300153e-05	\\
4783.30743963068	7.86145466929586e-05	\\
4784.28622159091	7.88418402561224e-05	\\
4785.26500355114	7.85850154347887e-05	\\
4786.24378551136	8.08830687772725e-05	\\
4787.22256747159	8.08614359295179e-05	\\
4788.20134943182	7.90421429358482e-05	\\
4789.18013139205	8.0556885171037e-05	\\
4790.15891335227	7.99392911804836e-05	\\
4791.1376953125	8.06623381388015e-05	\\
4792.11647727273	7.83083332334048e-05	\\
4793.09525923295	7.65671579274073e-05	\\
4794.07404119318	7.91749546453672e-05	\\
4795.05282315341	8.04162523369264e-05	\\
4796.03160511364	7.90731712255856e-05	\\
4797.01038707386	7.86005456922319e-05	\\
4797.98916903409	7.95944959761195e-05	\\
4798.96795099432	7.71916311237113e-05	\\
4799.94673295455	7.76231707936038e-05	\\
4800.92551491477	7.76414378897134e-05	\\
4801.904296875	7.91535031752606e-05	\\
4802.88307883523	7.82302342887359e-05	\\
4803.86186079545	7.7974633394782e-05	\\
4804.84064275568	7.75464548773558e-05	\\
4805.81942471591	7.68041086272219e-05	\\
4806.79820667614	7.9123201084239e-05	\\
4807.77698863636	7.91985181748475e-05	\\
4808.75577059659	7.99414554365344e-05	\\
4809.73455255682	7.64152089646894e-05	\\
4810.71333451705	7.89768721967446e-05	\\
4811.69211647727	7.8281571320048e-05	\\
4812.6708984375	7.83225225136226e-05	\\
4813.64968039773	7.7503090129227e-05	\\
4814.62846235795	7.7885472334518e-05	\\
4815.60724431818	7.74982615478334e-05	\\
4816.58602627841	7.82199989208351e-05	\\
4817.56480823864	7.81750116438008e-05	\\
4818.54359019886	7.96407195836e-05	\\
4819.52237215909	7.64821931006062e-05	\\
4820.50115411932	7.79544133256908e-05	\\
4821.47993607955	7.74280691339184e-05	\\
4822.45871803977	7.95172487428659e-05	\\
4823.4375	7.63632951073809e-05	\\
4824.41628196023	7.75037142514433e-05	\\
4825.39506392045	7.50418809306346e-05	\\
4826.37384588068	7.90835387584181e-05	\\
4827.35262784091	7.55912930426982e-05	\\
4828.33140980114	8.00056590950066e-05	\\
4829.31019176136	7.77211803754975e-05	\\
4830.28897372159	7.77758467501393e-05	\\
4831.26775568182	7.90200629690434e-05	\\
4832.24653764205	7.76507217175132e-05	\\
4833.22531960227	7.77838646050743e-05	\\
4834.2041015625	7.62736957601508e-05	\\
4835.18288352273	7.83133446093555e-05	\\
4836.16166548295	7.6090019483389e-05	\\
4837.14044744318	7.64108381934833e-05	\\
4838.11922940341	7.89689644109739e-05	\\
4839.09801136364	7.54164774157146e-05	\\
4840.07679332386	7.80114338576798e-05	\\
4841.05557528409	7.66594492969558e-05	\\
4842.03435724432	7.76477340322137e-05	\\
4843.01313920455	7.72786804201492e-05	\\
4843.99192116477	7.79692621285807e-05	\\
4844.970703125	7.48428243448198e-05	\\
4845.94948508523	7.53659557845632e-05	\\
4846.92826704545	7.55816937200465e-05	\\
4847.90704900568	7.58930179600704e-05	\\
4848.88583096591	7.65526338749252e-05	\\
4849.86461292614	7.62877021570908e-05	\\
4850.84339488636	7.39567604060066e-05	\\
4851.82217684659	7.57661656601663e-05	\\
4852.80095880682	7.56899624887648e-05	\\
4853.77974076705	7.74236577019896e-05	\\
4854.75852272727	7.65176489336185e-05	\\
4855.7373046875	7.4696678330585e-05	\\
4856.71608664773	7.6882173752275e-05	\\
4857.69486860795	7.57149694319408e-05	\\
4858.67365056818	7.42270600935618e-05	\\
4859.65243252841	7.39620876921719e-05	\\
4860.63121448864	7.42901932870243e-05	\\
4861.60999644886	7.46973235403341e-05	\\
4862.58877840909	7.42942326850534e-05	\\
4863.56756036932	7.55302446091109e-05	\\
4864.54634232955	7.47192790427535e-05	\\
4865.52512428977	7.37443030142533e-05	\\
4866.50390625	7.59574450321531e-05	\\
4867.48268821023	7.3372625117679e-05	\\
4868.46147017045	7.42081818465397e-05	\\
4869.44025213068	7.40828529652536e-05	\\
4870.41903409091	7.34219311195087e-05	\\
4871.39781605114	7.40505348084412e-05	\\
4872.37659801136	7.39408140086148e-05	\\
4873.35537997159	7.40675740333283e-05	\\
4874.33416193182	7.134106289327e-05	\\
4875.31294389205	7.36813410577092e-05	\\
4876.29172585227	7.29303842138035e-05	\\
4877.2705078125	7.22708688292184e-05	\\
4878.24928977273	7.21667795273382e-05	\\
4879.22807173295	7.16480927758522e-05	\\
4880.20685369318	7.29460892739286e-05	\\
4881.18563565341	7.09549322642953e-05	\\
4882.16441761364	7.04960090925351e-05	\\
4883.14319957386	7.13012467635783e-05	\\
4884.12198153409	6.9946896935064e-05	\\
4885.10076349432	7.02228626717176e-05	\\
4886.07954545455	7.22074165654535e-05	\\
4887.05832741477	7.08691971860532e-05	\\
4888.037109375	7.22833946923554e-05	\\
4889.01589133523	6.99757910940503e-05	\\
4889.99467329545	7.0738054958317e-05	\\
4890.97345525568	7.03348737748522e-05	\\
4891.95223721591	6.93735367903514e-05	\\
4892.93101917614	7.11869173046775e-05	\\
4893.90980113636	7.08984510215657e-05	\\
4894.88858309659	6.84079558751591e-05	\\
4895.86736505682	6.88927040907199e-05	\\
4896.84614701705	6.96267955711095e-05	\\
4897.82492897727	6.80084132026512e-05	\\
4898.8037109375	6.85481722019423e-05	\\
4899.78249289773	6.8119908886641e-05	\\
4900.76127485795	6.63839994173725e-05	\\
4901.74005681818	6.70911139598707e-05	\\
4902.71883877841	6.78209449620302e-05	\\
4903.69762073864	6.82168044564504e-05	\\
4904.67640269886	6.85823486271805e-05	\\
4905.65518465909	6.85412729920718e-05	\\
4906.63396661932	6.77163631289743e-05	\\
4907.61274857955	6.54017031232243e-05	\\
4908.59153053977	6.82739325967499e-05	\\
4909.5703125	6.75600299872649e-05	\\
4910.54909446023	6.73228034698817e-05	\\
4911.52787642045	6.66094558020333e-05	\\
4912.50665838068	6.77200813151207e-05	\\
4913.48544034091	6.64576283231219e-05	\\
4914.46422230114	6.59452675704849e-05	\\
4915.44300426136	6.66131993236513e-05	\\
4916.42178622159	6.56067489029743e-05	\\
4917.40056818182	6.5163874556063e-05	\\
4918.37935014205	6.52118285043724e-05	\\
4919.35813210227	6.41754356164975e-05	\\
4920.3369140625	6.50522332929068e-05	\\
4921.31569602273	6.41648206824649e-05	\\
4922.29447798295	6.3151122653567e-05	\\
4923.27325994318	6.48233171810019e-05	\\
4924.25204190341	6.37988674792956e-05	\\
4925.23082386364	6.41735969175698e-05	\\
4926.20960582386	6.18507837777486e-05	\\
4927.18838778409	6.46636185205101e-05	\\
4928.16716974432	6.21747028814298e-05	\\
4929.14595170455	6.23988727187697e-05	\\
4930.12473366477	6.4295842492984e-05	\\
4931.103515625	6.22200609595641e-05	\\
4932.08229758523	6.25280820170629e-05	\\
4933.06107954545	6.25753549738469e-05	\\
4934.03986150568	6.13638764777277e-05	\\
4935.01864346591	6.0480200957401e-05	\\
4935.99742542614	6.26727591834482e-05	\\
4936.97620738636	6.15486179840616e-05	\\
4937.95498934659	5.93723646814827e-05	\\
4938.93377130682	6.11036452290541e-05	\\
4939.91255326705	6.01384264258976e-05	\\
4940.89133522727	6.30169446241715e-05	\\
4941.8701171875	6.05283912882617e-05	\\
4942.84889914773	6.01470427980197e-05	\\
4943.82768110795	6.03333075638082e-05	\\
4944.80646306818	6.14014280627875e-05	\\
4945.78524502841	5.98374373675829e-05	\\
4946.76402698864	5.64565985048897e-05	\\
4947.74280894886	6.06348388169871e-05	\\
4948.72159090909	5.81463523536322e-05	\\
4949.70037286932	5.8887329000026e-05	\\
4950.67915482955	5.96756660257759e-05	\\
4951.65793678977	5.9749558690415e-05	\\
4952.63671875	5.76910548539351e-05	\\
4953.61550071023	5.66312679452255e-05	\\
4954.59428267045	5.92996461797753e-05	\\
4955.57306463068	5.89292238620093e-05	\\
4956.55184659091	5.84025997672743e-05	\\
4957.53062855114	5.735118377146e-05	\\
4958.50941051136	5.75694755855366e-05	\\
4959.48819247159	5.95324926130256e-05	\\
4960.46697443182	5.8643426687004e-05	\\
4961.44575639205	5.83244072691669e-05	\\
4962.42453835227	5.80675522083232e-05	\\
4963.4033203125	5.69601620386147e-05	\\
4964.38210227273	5.79364579879213e-05	\\
4965.36088423295	5.82660594250831e-05	\\
4966.33966619318	5.75491298304258e-05	\\
4967.31844815341	5.69443900946055e-05	\\
4968.29723011364	5.54862923281971e-05	\\
4969.27601207386	5.82696651411533e-05	\\
4970.25479403409	5.64151358179009e-05	\\
4971.23357599432	5.6785272602684e-05	\\
4972.21235795455	5.72195519317673e-05	\\
4973.19113991477	5.66063733527541e-05	\\
4974.169921875	5.49639779607741e-05	\\
4975.14870383523	5.54866788487804e-05	\\
4976.12748579545	5.59026606444813e-05	\\
4977.10626775568	5.65180395019717e-05	\\
4978.08504971591	5.53616739401819e-05	\\
4979.06383167614	5.62929706741681e-05	\\
4980.04261363636	5.63568561033381e-05	\\
4981.02139559659	5.57208728411443e-05	\\
4982.00017755682	5.74662274498483e-05	\\
4982.97895951705	5.46511381793452e-05	\\
4983.95774147727	5.66998086226229e-05	\\
4984.9365234375	5.4641141033836e-05	\\
4985.91530539773	5.2372266167691e-05	\\
4986.89408735795	5.47567652421693e-05	\\
4987.87286931818	5.52969713989473e-05	\\
4988.85165127841	5.34883078028261e-05	\\
4989.83043323864	5.55162693873277e-05	\\
4990.80921519886	5.37586183160587e-05	\\
4991.78799715909	5.36724248410297e-05	\\
4992.76677911932	5.59684727418981e-05	\\
4993.74556107955	5.37033006586025e-05	\\
4994.72434303977	5.48402388044897e-05	\\
4995.703125	5.45919344615559e-05	\\
4996.68190696023	5.42739732882128e-05	\\
4997.66068892045	5.45913729519495e-05	\\
4998.63947088068	5.31227281737213e-05	\\
4999.61825284091	5.28571694044281e-05	\\
5000.59703480114	5.38739727222694e-05	\\
5001.57581676136	5.38200170448555e-05	\\
5002.55459872159	5.35737989571313e-05	\\
5003.53338068182	5.57565780324314e-05	\\
5004.51216264205	5.45790327285543e-05	\\
5005.49094460227	5.21680298075162e-05	\\
5006.4697265625	5.45761492598304e-05	\\
5007.44850852273	5.22000092845133e-05	\\
5008.42729048295	5.21511792692646e-05	\\
5009.40607244318	5.45992114994676e-05	\\
5010.38485440341	5.38894343744934e-05	\\
5011.36363636364	5.28571587456839e-05	\\
5012.34241832386	5.08191625190894e-05	\\
5013.32120028409	5.28758849021092e-05	\\
5014.29998224432	5.12451477259633e-05	\\
5015.27876420455	5.43682992495907e-05	\\
5016.25754616477	5.17126663101002e-05	\\
5017.236328125	5.398381791307e-05	\\
5018.21511008523	5.28316973529818e-05	\\
5019.19389204545	5.44112415885275e-05	\\
5020.17267400568	5.41077609645378e-05	\\
5021.15145596591	5.27502471443745e-05	\\
5022.13023792614	5.29381346664996e-05	\\
5023.10901988636	5.48791490340345e-05	\\
5024.08780184659	5.27472470962938e-05	\\
5025.06658380682	5.30042926402417e-05	\\
5026.04536576705	5.30452542404619e-05	\\
5027.02414772727	5.20350277663897e-05	\\
5028.0029296875	5.22621339282401e-05	\\
5028.98171164773	5.0492105309542e-05	\\
5029.96049360795	5.19084044937716e-05	\\
5030.93927556818	5.30971037501152e-05	\\
5031.91805752841	5.30984793345022e-05	\\
5032.89683948864	4.97858181768654e-05	\\
5033.87562144886	5.25020260048969e-05	\\
5034.85440340909	5.23959897909237e-05	\\
5035.83318536932	5.24749120577504e-05	\\
5036.81196732955	5.19788762886051e-05	\\
5037.79074928977	5.20741559426852e-05	\\
5038.76953125	5.33671192695071e-05	\\
5039.74831321023	5.15967219676492e-05	\\
5040.72709517045	5.30498636493274e-05	\\
5041.70587713068	5.31843660191645e-05	\\
5042.68465909091	5.38247557690047e-05	\\
5043.66344105114	5.16749445234187e-05	\\
5044.64222301136	5.22049606197001e-05	\\
5045.62100497159	4.94144583275631e-05	\\
5046.59978693182	5.14588982481077e-05	\\
5047.57856889205	5.03043772749356e-05	\\
5048.55735085227	4.97568143573664e-05	\\
5049.5361328125	4.91754595131868e-05	\\
5050.51491477273	5.17153739823126e-05	\\
5051.49369673295	4.94050848948512e-05	\\
5052.47247869318	5.2413118061032e-05	\\
5053.45126065341	5.23513156910518e-05	\\
5054.43004261364	5.16133349238047e-05	\\
5055.40882457386	5.05434324854411e-05	\\
5056.38760653409	5.26569966244465e-05	\\
5057.36638849432	5.14122062526543e-05	\\
5058.34517045455	5.05590673670925e-05	\\
5059.32395241477	5.14808892279487e-05	\\
5060.302734375	5.08755070129105e-05	\\
5061.28151633523	5.11885997058637e-05	\\
5062.26029829545	5.243039151989e-05	\\
5063.23908025568	5.22532519914627e-05	\\
5064.21786221591	5.18241330224103e-05	\\
5065.19664417614	5.22967446688783e-05	\\
5066.17542613636	5.0573064526781e-05	\\
5067.15420809659	5.12125079979658e-05	\\
5068.13299005682	5.38377773647703e-05	\\
5069.11177201705	5.0481804794062e-05	\\
5070.09055397727	5.00473087642583e-05	\\
5071.0693359375	5.10865552218443e-05	\\
5072.04811789773	5.28905473169194e-05	\\
5073.02689985795	5.20013142524333e-05	\\
5074.00568181818	5.2512710203877e-05	\\
5074.98446377841	5.26593864675747e-05	\\
5075.96324573864	5.26976823488365e-05	\\
5076.94202769886	5.26326284318432e-05	\\
5077.92080965909	5.03789208900416e-05	\\
5078.89959161932	5.00298576743481e-05	\\
5079.87837357955	5.1879582250807e-05	\\
5080.85715553977	5.22052988823195e-05	\\
5081.8359375	5.26816774650222e-05	\\
5082.81471946023	5.33243702452783e-05	\\
5083.79350142045	5.11695524258951e-05	\\
5084.77228338068	5.16194140918874e-05	\\
5085.75106534091	5.15435396544065e-05	\\
5086.72984730114	5.2862153814549e-05	\\
5087.70862926136	5.23176455177323e-05	\\
5088.68741122159	5.2327019823995e-05	\\
5089.66619318182	5.15606386719545e-05	\\
5090.64497514205	5.24840223397393e-05	\\
5091.62375710227	5.19357116853503e-05	\\
5092.6025390625	5.12857227250989e-05	\\
5093.58132102273	5.09895804965996e-05	\\
5094.56010298295	4.99805905085762e-05	\\
5095.53888494318	5.31077604520743e-05	\\
5096.51766690341	5.09125634319735e-05	\\
5097.49644886364	5.21571031113046e-05	\\
5098.47523082386	5.15375851518701e-05	\\
5099.45401278409	5.16719519218189e-05	\\
5100.43279474432	5.08189708616521e-05	\\
5101.41157670455	5.35433061966478e-05	\\
5102.39035866477	5.21234481801583e-05	\\
5103.369140625	5.02452420128644e-05	\\
5104.34792258523	5.33361588782601e-05	\\
5105.32670454545	5.19752010167626e-05	\\
5106.30548650568	5.32270246623484e-05	\\
5107.28426846591	5.25702970492603e-05	\\
5108.26305042614	5.41762552473673e-05	\\
5109.24183238636	5.12201197939274e-05	\\
5110.22061434659	5.35168050007217e-05	\\
5111.19939630682	5.2858481186176e-05	\\
5112.17817826705	5.2947790586221e-05	\\
5113.15696022727	5.31746302371443e-05	\\
5114.1357421875	5.43451372508701e-05	\\
5115.11452414773	5.18610283269982e-05	\\
5116.09330610795	5.20539188481028e-05	\\
5117.07208806818	5.4732601980603e-05	\\
5118.05087002841	5.29478126263979e-05	\\
5119.02965198864	5.39561699063822e-05	\\
5120.00843394886	5.48947377907925e-05	\\
5120.98721590909	5.27618936920513e-05	\\
5121.96599786932	5.21191605578924e-05	\\
5122.94477982955	5.50459771615942e-05	\\
5123.92356178977	5.29567329103578e-05	\\
5124.90234375	5.4926389004012e-05	\\
5125.88112571023	5.59519870394027e-05	\\
5126.85990767045	5.40418518988716e-05	\\
5127.83868963068	5.28510052296075e-05	\\
5128.81747159091	5.40148765835683e-05	\\
5129.79625355114	5.41439921540652e-05	\\
5130.77503551136	5.43874357522408e-05	\\
5131.75381747159	5.48376064374848e-05	\\
5132.73259943182	5.61105152026045e-05	\\
5133.71138139205	5.3254531193001e-05	\\
5134.69016335227	5.44651820045172e-05	\\
5135.6689453125	5.38873752717798e-05	\\
5136.64772727273	5.54178842810079e-05	\\
5137.62650923295	5.32525762098152e-05	\\
5138.60529119318	5.5736412591528e-05	\\
5139.58407315341	5.48303886038496e-05	\\
5140.56285511364	5.60708680083198e-05	\\
5141.54163707386	5.53096766804591e-05	\\
5142.52041903409	5.63030429775302e-05	\\
5143.49920099432	5.52374715169696e-05	\\
5144.47798295455	5.51326356297007e-05	\\
5145.45676491477	5.52376690489737e-05	\\
5146.435546875	5.76053405540489e-05	\\
5147.41432883523	5.42004090598977e-05	\\
5148.39311079545	5.70413226145639e-05	\\
5149.37189275568	5.69020783062237e-05	\\
5150.35067471591	5.80736725088884e-05	\\
5151.32945667614	5.65652851931224e-05	\\
5152.30823863636	5.65173814358848e-05	\\
5153.28702059659	5.74160850465259e-05	\\
5154.26580255682	5.74927796803874e-05	\\
5155.24458451705	5.47586557396838e-05	\\
5156.22336647727	5.68591242682313e-05	\\
5157.2021484375	5.75463748291925e-05	\\
5158.18093039773	5.58276472896482e-05	\\
5159.15971235795	5.63988697359746e-05	\\
5160.13849431818	5.77255900581572e-05	\\
5161.11727627841	5.72567149234452e-05	\\
5162.09605823864	5.66452071367769e-05	\\
5163.07484019886	5.68095096532955e-05	\\
5164.05362215909	5.48292322554963e-05	\\
5165.03240411932	5.69492087755178e-05	\\
5166.01118607955	5.51256409458343e-05	\\
5166.98996803977	5.35901887192926e-05	\\
5167.96875	5.73744769685088e-05	\\
5168.94753196023	5.75660286571105e-05	\\
5169.92631392045	5.7814304975893e-05	\\
5170.90509588068	5.79831887592263e-05	\\
5171.88387784091	5.66196656020797e-05	\\
5172.86265980114	5.7073025071513e-05	\\
5173.84144176136	5.60004101706574e-05	\\
5174.82022372159	5.70672741748261e-05	\\
5175.79900568182	5.67923610730291e-05	\\
5176.77778764205	5.60266197671426e-05	\\
5177.75656960227	5.61883525664842e-05	\\
5178.7353515625	5.4491225021002e-05	\\
5179.71413352273	5.62541013702681e-05	\\
5180.69291548295	5.57544708276394e-05	\\
5181.67169744318	5.72745646085491e-05	\\
5182.65047940341	5.66830136892169e-05	\\
5183.62926136364	5.49761731936049e-05	\\
5184.60804332386	5.83540926818072e-05	\\
5185.58682528409	5.48395850065181e-05	\\
5186.56560724432	5.68483457375542e-05	\\
5187.54438920455	5.68220791271422e-05	\\
5188.52317116477	5.58245235676695e-05	\\
5189.501953125	5.38290590995321e-05	\\
5190.48073508523	5.78955888559521e-05	\\
5191.45951704545	5.64535220778731e-05	\\
5192.43829900568	5.60456785886758e-05	\\
5193.41708096591	5.60549197482621e-05	\\
5194.39586292614	5.91159657452906e-05	\\
5195.37464488636	5.45591002103115e-05	\\
5196.35342684659	5.48508727919095e-05	\\
5197.33220880682	5.57762849931662e-05	\\
5198.31099076705	5.25576598130091e-05	\\
5199.28977272727	5.58576794140041e-05	\\
5200.2685546875	5.77450288194445e-05	\\
5201.24733664773	5.40742869419641e-05	\\
5202.22611860795	5.61425762060647e-05	\\
5203.20490056818	5.5262442742834e-05	\\
5204.18368252841	5.36373106902326e-05	\\
5205.16246448864	5.64989297809969e-05	\\
5206.14124644886	5.38097741666343e-05	\\
5207.12002840909	5.52474503421326e-05	\\
5208.09881036932	5.30383990843076e-05	\\
5209.07759232955	5.50819765078701e-05	\\
5210.05637428977	5.40978621776216e-05	\\
5211.03515625	5.4245713184577e-05	\\
5212.01393821023	5.56854309619708e-05	\\
5212.99272017045	5.43852129764274e-05	\\
5213.97150213068	5.43005797467797e-05	\\
5214.95028409091	5.58103660812471e-05	\\
5215.92906605114	5.44201351422966e-05	\\
5216.90784801136	5.32296996321157e-05	\\
5217.88662997159	5.68617433392265e-05	\\
5218.86541193182	5.37367668995769e-05	\\
5219.84419389205	5.39333460850552e-05	\\
5220.82297585227	5.3763091733193e-05	\\
5221.8017578125	5.55082072409138e-05	\\
5222.78053977273	5.47994424272714e-05	\\
5223.75932173295	5.27658527679075e-05	\\
5224.73810369318	5.39421619295391e-05	\\
5225.71688565341	5.7004238263361e-05	\\
5226.69566761364	5.50380020346437e-05	\\
5227.67444957386	5.62484432481427e-05	\\
5228.65323153409	5.64862863670232e-05	\\
5229.63201349432	5.08153574704244e-05	\\
5230.61079545455	5.62685152953714e-05	\\
5231.58957741477	5.30322830288979e-05	\\
5232.568359375	5.47736171783217e-05	\\
5233.54714133523	5.42053438249458e-05	\\
5234.52592329545	5.46500387884085e-05	\\
5235.50470525568	5.56083969676716e-05	\\
5236.48348721591	5.29231324167363e-05	\\
5237.46226917614	5.56087229005971e-05	\\
5238.44105113636	5.39987668877492e-05	\\
5239.41983309659	5.77881848791128e-05	\\
5240.39861505682	5.78825744274046e-05	\\
5241.37739701705	5.63564205763396e-05	\\
5242.35617897727	5.36225319239925e-05	\\
5243.3349609375	5.57890587384969e-05	\\
5244.31374289773	5.67404263145612e-05	\\
5245.29252485795	5.79463381750412e-05	\\
5246.27130681818	5.17965748259646e-05	\\
5247.25008877841	5.3570926305289e-05	\\
5248.22887073864	5.53146551495463e-05	\\
5249.20765269886	5.39929662314225e-05	\\
5250.18643465909	5.43879920911397e-05	\\
5251.16521661932	5.6265407728073e-05	\\
5252.14399857955	5.46962013380265e-05	\\
5253.12278053977	5.55359592357566e-05	\\
5254.1015625	5.44023434959282e-05	\\
5255.08034446023	5.54805365782645e-05	\\
5256.05912642045	5.50372935508735e-05	\\
5257.03790838068	5.59062574647557e-05	\\
5258.01669034091	5.51854494052433e-05	\\
5258.99547230114	5.53503175651273e-05	\\
5259.97425426136	5.52233948441532e-05	\\
5260.95303622159	5.60149287471894e-05	\\
5261.93181818182	5.7191421714728e-05	\\
5262.91060014205	5.34807347875994e-05	\\
5263.88938210227	5.51747889040352e-05	\\
5264.8681640625	5.61489936663377e-05	\\
5265.84694602273	5.63950805371271e-05	\\
5266.82572798295	5.58467748396926e-05	\\
5267.80450994318	5.58056202938881e-05	\\
5268.78329190341	5.78903561207645e-05	\\
5269.76207386364	5.56356406223936e-05	\\
5270.74085582386	5.71222194101584e-05	\\
5271.71963778409	5.64526952781129e-05	\\
5272.69841974432	5.68416046036455e-05	\\
5273.67720170455	5.38564886700901e-05	\\
5274.65598366477	5.6752211302182e-05	\\
5275.634765625	5.67897772979018e-05	\\
5276.61354758523	5.47702756522107e-05	\\
5277.59232954545	5.61062359142788e-05	\\
5278.57111150568	5.55791917258969e-05	\\
5279.54989346591	5.35339989212566e-05	\\
5280.52867542614	5.65033724725301e-05	\\
5281.50745738636	5.67187333367329e-05	\\
5282.48623934659	5.71038154688647e-05	\\
5283.46502130682	5.75793623095693e-05	\\
5284.44380326705	5.58750268809662e-05	\\
5285.42258522727	5.66700083214176e-05	\\
5286.4013671875	5.56199244644793e-05	\\
5287.38014914773	5.60937546287535e-05	\\
5288.35893110795	5.86200188913897e-05	\\
5289.33771306818	5.73613272948625e-05	\\
5290.31649502841	5.68815691366502e-05	\\
5291.29527698864	5.84210804359084e-05	\\
5292.27405894886	5.60040245224148e-05	\\
5293.25284090909	5.66881003796128e-05	\\
5294.23162286932	5.65745186006823e-05	\\
5295.21040482955	5.59564745142666e-05	\\
5296.18918678977	5.57218293426929e-05	\\
5297.16796875	5.84505242312324e-05	\\
5298.14675071023	5.91948475646525e-05	\\
5299.12553267045	5.72718410647961e-05	\\
5300.10431463068	5.82100208353497e-05	\\
5301.08309659091	5.64400920128335e-05	\\
5302.06187855114	5.72422184472589e-05	\\
5303.04066051136	5.63516565085448e-05	\\
5304.01944247159	5.53302061172256e-05	\\
5304.99822443182	5.57073335960959e-05	\\
5305.97700639205	5.63052233256192e-05	\\
5306.95578835227	5.68951541092358e-05	\\
5307.9345703125	5.57742235946883e-05	\\
5308.91335227273	5.62351279803721e-05	\\
5309.89213423295	5.83067327624458e-05	\\
5310.87091619318	5.48122485674648e-05	\\
5311.84969815341	5.70004452072797e-05	\\
5312.82848011364	5.64537048303645e-05	\\
5313.80726207386	5.64875897053947e-05	\\
5314.78604403409	5.70045108375046e-05	\\
5315.76482599432	5.82755572229981e-05	\\
5316.74360795455	5.511528418135e-05	\\
5317.72238991477	5.75206069792549e-05	\\
5318.701171875	5.74233898674018e-05	\\
5319.67995383523	5.76150926312913e-05	\\
5320.65873579545	5.72022619172266e-05	\\
5321.63751775568	5.54741883680172e-05	\\
5322.61629971591	5.73657255302554e-05	\\
5323.59508167614	5.54473328938235e-05	\\
5324.57386363636	5.67639547708366e-05	\\
5325.55264559659	5.68217051022351e-05	\\
5326.53142755682	5.58297249397828e-05	\\
5327.51020951705	5.4452015362339e-05	\\
5328.48899147727	5.51667468384621e-05	\\
5329.4677734375	5.48263760925791e-05	\\
5330.44655539773	5.44049629820063e-05	\\
5331.42533735795	5.59171403359776e-05	\\
5332.40411931818	5.77555220237832e-05	\\
5333.38290127841	5.78168455584768e-05	\\
5334.36168323864	5.56041922846223e-05	\\
5335.34046519886	5.5522124928328e-05	\\
5336.31924715909	5.60512306937357e-05	\\
5337.29802911932	5.54668237995638e-05	\\
5338.27681107955	5.6470769024277e-05	\\
5339.25559303977	5.57440408499534e-05	\\
5340.234375	5.8033446579256e-05	\\
5341.21315696023	5.79600740962826e-05	\\
5342.19193892045	5.67433263617911e-05	\\
5343.17072088068	5.80034924675653e-05	\\
5344.14950284091	5.65491336742966e-05	\\
5345.12828480114	6.006760890469e-05	\\
5346.10706676136	5.5588929654589e-05	\\
5347.08584872159	5.92291198394207e-05	\\
5348.06463068182	5.66453045866354e-05	\\
5349.04341264205	5.96542379122619e-05	\\
5350.02219460227	5.82722973400328e-05	\\
5351.0009765625	5.96677260585245e-05	\\
5351.97975852273	5.99204503622984e-05	\\
5352.95854048295	5.90279724026033e-05	\\
5353.93732244318	5.9063605097025e-05	\\
5354.91610440341	5.96168847987024e-05	\\
5355.89488636364	5.94551172430266e-05	\\
5356.87366832386	5.93312911469419e-05	\\
5357.85245028409	6.03688533645611e-05	\\
5358.83123224432	6.05739372337364e-05	\\
5359.81001420455	6.11458540391279e-05	\\
5360.78879616477	6.02036763883172e-05	\\
5361.767578125	6.02607222318471e-05	\\
5362.74636008523	5.93012092598925e-05	\\
5363.72514204545	6.0337899218688e-05	\\
5364.70392400568	5.83801361325197e-05	\\
5365.68270596591	6.04946958589049e-05	\\
5366.66148792614	5.9685884004045e-05	\\
5367.64026988636	6.07018708117278e-05	\\
5368.61905184659	5.96313274614101e-05	\\
5369.59783380682	5.94341261001749e-05	\\
5370.57661576705	6.21253060116305e-05	\\
5371.55539772727	6.17568921420625e-05	\\
5372.5341796875	5.92447341875952e-05	\\
5373.51296164773	6.00610492764104e-05	\\
5374.49174360795	6.21038117417135e-05	\\
5375.47052556818	6.15582006476032e-05	\\
5376.44930752841	6.18592428767752e-05	\\
5377.42808948864	6.23512340804849e-05	\\
5378.40687144886	6.16075106789187e-05	\\
5379.38565340909	6.11098892561845e-05	\\
5380.36443536932	6.39732570620573e-05	\\
5381.34321732955	6.34538368472101e-05	\\
5382.32199928977	6.27763385887663e-05	\\
5383.30078125	6.33020758033735e-05	\\
5384.27956321023	6.38236947910805e-05	\\
5385.25834517045	6.2203442320107e-05	\\
5386.23712713068	6.19285938323024e-05	\\
5387.21590909091	6.37432702666914e-05	\\
5388.19469105114	6.39149037705173e-05	\\
5389.17347301136	6.37446704170228e-05	\\
5390.15225497159	6.15449616196622e-05	\\
5391.13103693182	6.44095231533481e-05	\\
5392.10981889205	6.51184161201822e-05	\\
5393.08860085227	6.31973500306422e-05	\\
5394.0673828125	6.42645740364008e-05	\\
5395.04616477273	6.57532807878544e-05	\\
5396.02494673295	6.64446249133863e-05	\\
5397.00372869318	6.21654992888738e-05	\\
5397.98251065341	6.46857677223755e-05	\\
5398.96129261364	6.43985187904025e-05	\\
5399.94007457386	6.57544996474664e-05	\\
5400.91885653409	6.26774393868102e-05	\\
5401.89763849432	6.49025723244217e-05	\\
5402.87642045455	6.62101353368577e-05	\\
5403.85520241477	6.77788813021864e-05	\\
5404.833984375	6.5713553685868e-05	\\
5405.81276633523	6.50106467768975e-05	\\
5406.79154829545	6.51591388323512e-05	\\
5407.77033025568	6.43968282386754e-05	\\
5408.74911221591	6.43655878177779e-05	\\
5409.72789417614	6.4729906547358e-05	\\
5410.70667613636	6.36592601599987e-05	\\
5411.68545809659	6.64693557415701e-05	\\
5412.66424005682	6.67033668336073e-05	\\
5413.64302201705	6.47829767175874e-05	\\
5414.62180397727	6.5555878460957e-05	\\
5415.6005859375	6.49238682786229e-05	\\
5416.57936789773	6.47215242551991e-05	\\
5417.55814985795	6.52682942239722e-05	\\
5418.53693181818	6.47267768465647e-05	\\
5419.51571377841	6.7219144388039e-05	\\
5420.49449573864	6.52728576397475e-05	\\
5421.47327769886	6.1449537542127e-05	\\
5422.45205965909	6.49391194087195e-05	\\
5423.43084161932	6.35357496224498e-05	\\
5424.40962357955	6.55843126380105e-05	\\
5425.38840553977	6.45857221144146e-05	\\
5426.3671875	6.48290808673266e-05	\\
5427.34596946023	6.48762889824415e-05	\\
5428.32475142045	6.47109540381289e-05	\\
5429.30353338068	6.59261196867936e-05	\\
5430.28231534091	6.3896605917948e-05	\\
5431.26109730114	6.51175683053625e-05	\\
5432.23987926136	6.34526763846448e-05	\\
5433.21866122159	6.72716022631534e-05	\\
5434.19744318182	6.79359774165831e-05	\\
5435.17622514205	6.553886610868e-05	\\
5436.15500710227	6.62532107128989e-05	\\
5437.1337890625	6.54707094473225e-05	\\
5438.11257102273	6.70004304750837e-05	\\
5439.09135298295	6.60794328427259e-05	\\
5440.07013494318	6.51239046735826e-05	\\
5441.04891690341	6.34234768822779e-05	\\
5442.02769886364	6.49956925619341e-05	\\
5443.00648082386	6.40201118106671e-05	\\
5443.98526278409	6.66454515842681e-05	\\
5444.96404474432	6.56047866625754e-05	\\
5445.94282670455	6.48869887860093e-05	\\
5446.92160866477	6.49863158202361e-05	\\
5447.900390625	6.65610338147295e-05	\\
5448.87917258523	6.45692965718088e-05	\\
5449.85795454545	6.70470096055127e-05	\\
5450.83673650568	6.6122431253819e-05	\\
5451.81551846591	6.3155342322306e-05	\\
5452.79430042614	6.66644581600943e-05	\\
5453.77308238636	6.58466191820112e-05	\\
5454.75186434659	6.35880761405265e-05	\\
5455.73064630682	6.70715370378901e-05	\\
5456.70942826705	6.52709243174514e-05	\\
5457.68821022727	6.51062136000444e-05	\\
5458.6669921875	6.64991591546514e-05	\\
5459.64577414773	6.54770671463618e-05	\\
5460.62455610795	6.59738491439544e-05	\\
5461.60333806818	6.48548794999758e-05	\\
5462.58212002841	6.47249478027582e-05	\\
5463.56090198864	6.51552472807825e-05	\\
5464.53968394886	6.70859595584766e-05	\\
5465.51846590909	6.71034667645579e-05	\\
5466.49724786932	6.57457365673815e-05	\\
5467.47602982955	6.69961102358658e-05	\\
5468.45481178977	6.60498660248598e-05	\\
5469.43359375	6.73099729376873e-05	\\
5470.41237571023	6.55699882840713e-05	\\
5471.39115767045	6.78702957670816e-05	\\
5472.36993963068	6.82875947768459e-05	\\
5473.34872159091	6.56688506888947e-05	\\
5474.32750355114	6.60869012697492e-05	\\
5475.30628551136	6.53870303484485e-05	\\
5476.28506747159	6.81041744581277e-05	\\
5477.26384943182	6.64692002856448e-05	\\
5478.24263139205	6.68111839185268e-05	\\
5479.22141335227	6.58611279384982e-05	\\
5480.2001953125	6.56953847558406e-05	\\
5481.17897727273	6.5090280047082e-05	\\
5482.15775923295	6.54984241242807e-05	\\
5483.13654119318	6.56578500011268e-05	\\
5484.11532315341	6.67038344986815e-05	\\
5485.09410511364	6.55796860766188e-05	\\
5486.07288707386	6.77549075230653e-05	\\
5487.05166903409	6.64659065242782e-05	\\
5488.03045099432	6.79692568239309e-05	\\
5489.00923295455	6.72636221280854e-05	\\
5489.98801491477	6.62997334754534e-05	\\
5490.966796875	6.74371703124316e-05	\\
5491.94557883523	6.72557028508052e-05	\\
5492.92436079545	6.54954935537386e-05	\\
5493.90314275568	6.60820556751947e-05	\\
5494.88192471591	6.88172567857569e-05	\\
5495.86070667614	6.59009355120358e-05	\\
5496.83948863636	6.82176966539892e-05	\\
5497.81827059659	6.55846228251005e-05	\\
5498.79705255682	6.51753623760262e-05	\\
5499.77583451705	6.61220843507542e-05	\\
5500.75461647727	6.67891492308351e-05	\\
5501.7333984375	6.58240530841034e-05	\\
5502.71218039773	6.73704242280911e-05	\\
5503.69096235795	6.66297293933721e-05	\\
5504.66974431818	6.44961312845109e-05	\\
5505.64852627841	6.57695865900359e-05	\\
5506.62730823864	6.64386826645404e-05	\\
5507.60609019886	6.42003466623954e-05	\\
5508.58487215909	6.49017953643344e-05	\\
5509.56365411932	6.48424576640756e-05	\\
5510.54243607955	6.81967017977413e-05	\\
5511.52121803977	6.57824020767425e-05	\\
5512.5	6.67882552623449e-05	\\
5513.47878196023	6.50606951201793e-05	\\
5514.45756392045	6.67022171442779e-05	\\
5515.43634588068	6.74530628539589e-05	\\
5516.41512784091	6.56526163471513e-05	\\
5517.39390980114	6.65778767103104e-05	\\
5518.37269176136	6.65123720337991e-05	\\
5519.35147372159	6.46990914534497e-05	\\
5520.33025568182	6.34848041478597e-05	\\
5521.30903764205	6.76887353253367e-05	\\
5522.28781960227	6.36467363879904e-05	\\
5523.2666015625	6.35113209086659e-05	\\
5524.24538352273	6.55133771287496e-05	\\
5525.22416548295	6.37633442080358e-05	\\
5526.20294744318	6.55330191734436e-05	\\
5527.18172940341	6.45144402723235e-05	\\
5528.16051136364	6.38737603287548e-05	\\
5529.13929332386	6.45795039954713e-05	\\
5530.11807528409	6.53212594903629e-05	\\
5531.09685724432	6.45965763866298e-05	\\
5532.07563920455	6.449343138248e-05	\\
5533.05442116477	6.45794488276926e-05	\\
5534.033203125	6.49150877114918e-05	\\
5535.01198508523	6.59616881534281e-05	\\
5535.99076704545	6.5271194575428e-05	\\
5536.96954900568	6.46727426201753e-05	\\
5537.94833096591	6.51082186237837e-05	\\
5538.92711292614	6.53973008233022e-05	\\
5539.90589488636	6.44850720268169e-05	\\
5540.88467684659	6.60733192978836e-05	\\
5541.86345880682	6.26043790982363e-05	\\
5542.84224076705	6.51266116765778e-05	\\
5543.82102272727	6.63347020772887e-05	\\
5544.7998046875	6.34846003669554e-05	\\
5545.77858664773	6.36978259650042e-05	\\
5546.75736860795	6.21984828994968e-05	\\
5547.73615056818	6.60984234555894e-05	\\
5548.71493252841	6.56325220489836e-05	\\
5549.69371448864	6.32148403222804e-05	\\
5550.67249644886	6.68849630463755e-05	\\
5551.65127840909	6.4798635113798e-05	\\
5552.63006036932	6.47950971747897e-05	\\
5553.60884232955	6.69821983154008e-05	\\
5554.58762428977	6.52189407777089e-05	\\
5555.56640625	6.47703336286412e-05	\\
5556.54518821023	6.50284173176362e-05	\\
5557.52397017045	6.31905359119002e-05	\\
5558.50275213068	6.60513751837022e-05	\\
5559.48153409091	6.51339363818667e-05	\\
5560.46031605114	6.60325221724359e-05	\\
5561.43909801136	6.60504432066944e-05	\\
5562.41787997159	6.31010046231562e-05	\\
5563.39666193182	6.44216204690922e-05	\\
5564.37544389205	6.53904358123699e-05	\\
5565.35422585227	6.35877042793057e-05	\\
5566.3330078125	6.50865830503824e-05	\\
5567.31178977273	6.48919700841528e-05	\\
5568.29057173295	6.43383152299567e-05	\\
5569.26935369318	6.64300573165337e-05	\\
5570.24813565341	6.42168588442644e-05	\\
5571.22691761364	6.4511390397501e-05	\\
5572.20569957386	6.53179276968339e-05	\\
5573.18448153409	6.46939947647293e-05	\\
5574.16326349432	6.37507499000255e-05	\\
5575.14204545455	6.56745072635152e-05	\\
5576.12082741477	6.47118043073433e-05	\\
5577.099609375	6.54196296914935e-05	\\
5578.07839133523	6.42619806662164e-05	\\
5579.05717329545	6.39114660557346e-05	\\
5580.03595525568	6.61818778168308e-05	\\
5581.01473721591	6.56469170449638e-05	\\
5581.99351917614	6.26853545813065e-05	\\
5582.97230113636	6.46471087326052e-05	\\
5583.95108309659	6.44325091534392e-05	\\
5584.92986505682	6.50010185133791e-05	\\
5585.90864701705	6.27566836806606e-05	\\
5586.88742897727	6.35246841069782e-05	\\
5587.8662109375	6.37689315836795e-05	\\
5588.84499289773	6.38622759017077e-05	\\
5589.82377485795	6.3496801577264e-05	\\
5590.80255681818	6.56643968158014e-05	\\
5591.78133877841	6.53687949631677e-05	\\
5592.76012073864	6.27632029022033e-05	\\
5593.73890269886	6.4013745427163e-05	\\
5594.71768465909	6.54723219303e-05	\\
5595.69646661932	6.50144563036687e-05	\\
5596.67524857955	6.28865639370312e-05	\\
5597.65403053977	6.22672240709818e-05	\\
5598.6328125	6.47087455300809e-05	\\
5599.61159446023	6.1727290725727e-05	\\
5600.59037642045	6.39324075579085e-05	\\
5601.56915838068	6.25068585842659e-05	\\
5602.54794034091	6.33778178441062e-05	\\
5603.52672230114	6.55630059395112e-05	\\
5604.50550426136	6.28089641921939e-05	\\
5605.48428622159	6.36252459821956e-05	\\
5606.46306818182	6.17527183681077e-05	\\
5607.44185014205	6.36396890008417e-05	\\
5608.42063210227	6.39031055504034e-05	\\
5609.3994140625	6.54955025312251e-05	\\
5610.37819602273	6.25051974727973e-05	\\
5611.35697798295	6.19133944699973e-05	\\
5612.33575994318	6.25547668579342e-05	\\
5613.31454190341	6.06748898973808e-05	\\
5614.29332386364	6.33690308102483e-05	\\
5615.27210582386	6.39784419167007e-05	\\
5616.25088778409	6.46476432677241e-05	\\
5617.22966974432	5.99863297624169e-05	\\
5618.20845170455	6.27725612589199e-05	\\
5619.18723366477	6.26376327910127e-05	\\
5620.166015625	6.1638762426045e-05	\\
5621.14479758523	6.12483804875895e-05	\\
5622.12357954545	6.08833897343648e-05	\\
5623.10236150568	6.04884372467383e-05	\\
5624.08114346591	6.25749983218771e-05	\\
5625.05992542614	6.1548935699375e-05	\\
5626.03870738636	6.16123950471532e-05	\\
5627.01748934659	6.14229509148741e-05	\\
5627.99627130682	6.07795154788417e-05	\\
5628.97505326705	6.09490514163469e-05	\\
5629.95383522727	5.88724603982252e-05	\\
5630.9326171875	5.93265764499492e-05	\\
5631.91139914773	6.03019222413647e-05	\\
5632.89018110795	5.87464050854249e-05	\\
5633.86896306818	5.8702472895028e-05	\\
5634.84774502841	5.76463216087029e-05	\\
5635.82652698864	6.10289412834487e-05	\\
5636.80530894886	6.17728892488082e-05	\\
5637.78409090909	5.80855119239599e-05	\\
5638.76287286932	5.92756215018956e-05	\\
5639.74165482955	5.93067444794979e-05	\\
5640.72043678977	5.95859059471719e-05	\\
5641.69921875	5.54486626288024e-05	\\
5642.67800071023	5.64538282580106e-05	\\
5643.65678267045	5.65505389587017e-05	\\
5644.63556463068	5.8391499149508e-05	\\
5645.61434659091	5.74066638671878e-05	\\
5646.59312855114	5.70514557783006e-05	\\
5647.57191051136	5.68879883733036e-05	\\
5648.55069247159	5.82892902685186e-05	\\
5649.52947443182	5.59294338701913e-05	\\
5650.50825639205	5.53482552567782e-05	\\
5651.48703835227	5.68372635078954e-05	\\
5652.4658203125	5.70750967499782e-05	\\
5653.44460227273	5.50359078060181e-05	\\
5654.42338423295	5.55630634499913e-05	\\
5655.40216619318	5.52543919602603e-05	\\
5656.38094815341	5.63638834727833e-05	\\
5657.35973011364	5.55362645157189e-05	\\
5658.33851207386	5.5929695601419e-05	\\
5659.31729403409	5.6519516537919e-05	\\
5660.29607599432	5.39406954040785e-05	\\
5661.27485795455	5.21768752392858e-05	\\
5662.25363991477	5.17902288124021e-05	\\
5663.232421875	5.3290748804022e-05	\\
5664.21120383523	5.29076263448577e-05	\\
5665.18998579545	5.24449778586388e-05	\\
5666.16876775568	5.23630385946406e-05	\\
5667.14754971591	5.22351442531648e-05	\\
5668.12633167614	5.32257330225859e-05	\\
5669.10511363636	5.50325743660689e-05	\\
5670.08389559659	5.39789251605422e-05	\\
5671.06267755682	5.24369030204526e-05	\\
5672.04145951705	5.33792693319786e-05	\\
5673.02024147727	5.22740420352101e-05	\\
5673.9990234375	5.05775871556287e-05	\\
5674.97780539773	5.26796690917666e-05	\\
5675.95658735795	5.27331935041052e-05	\\
5676.93536931818	5.25310882230382e-05	\\
5677.91415127841	5.20763721991311e-05	\\
5678.89293323864	5.09891028333153e-05	\\
5679.87171519886	5.07282069930974e-05	\\
5680.85049715909	5.38226569977354e-05	\\
5681.82927911932	5.20429496039942e-05	\\
5682.80806107955	5.28625509212213e-05	\\
5683.78684303977	5.18934536993252e-05	\\
5684.765625	5.28837104027004e-05	\\
5685.74440696023	5.13255103040126e-05	\\
5686.72318892045	5.00091752511488e-05	\\
5687.70197088068	5.1336449962094e-05	\\
5688.68075284091	5.13968492057255e-05	\\
5689.65953480114	4.92360458838219e-05	\\
5690.63831676136	5.20088271946897e-05	\\
5691.61709872159	5.17940939142298e-05	\\
5692.59588068182	5.08873348004924e-05	\\
5693.57466264205	5.15744128095362e-05	\\
5694.55344460227	4.87105562488389e-05	\\
5695.5322265625	5.15330239232729e-05	\\
5696.51100852273	5.11231300353331e-05	\\
5697.48979048295	5.17803992590394e-05	\\
5698.46857244318	4.99785495853573e-05	\\
5699.44735440341	4.95826412397917e-05	\\
5700.42613636364	4.99065750945401e-05	\\
5701.40491832386	4.87078005617582e-05	\\
5702.38370028409	4.99983191289455e-05	\\
5703.36248224432	4.99958271738374e-05	\\
5704.34126420455	4.98416954242855e-05	\\
5705.32004616477	4.89241519190716e-05	\\
5706.298828125	4.91701874548185e-05	\\
5707.27761008523	5.05548267939728e-05	\\
5708.25639204545	4.97291723706722e-05	\\
5709.23517400568	5.02133339146487e-05	\\
5710.21395596591	4.83446802589968e-05	\\
5711.19273792614	5.12367475925246e-05	\\
5712.17151988636	5.00091886571901e-05	\\
5713.15030184659	4.92967672294411e-05	\\
5714.12908380682	5.0924646858758e-05	\\
5715.10786576705	4.80519422049736e-05	\\
5716.08664772727	5.08726115430404e-05	\\
5717.0654296875	5.0633751225004e-05	\\
5718.04421164773	4.81746013149588e-05	\\
5719.02299360795	4.92310866082649e-05	\\
5720.00177556818	4.92485420880498e-05	\\
5720.98055752841	4.96972864599445e-05	\\
5721.95933948864	4.8578688292294e-05	\\
5722.93812144886	4.93963600684875e-05	\\
5723.91690340909	4.84619905831093e-05	\\
5724.89568536932	5.02811910901096e-05	\\
5725.87446732955	5.04863327375189e-05	\\
5726.85324928977	4.98068927097222e-05	\\
5727.83203125	4.9144712497257e-05	\\
5728.81081321023	4.85267242914736e-05	\\
5729.78959517045	4.87046189580428e-05	\\
5730.76837713068	4.86200018612477e-05	\\
5731.74715909091	4.76710253095747e-05	\\
5732.72594105114	4.91579927782711e-05	\\
5733.70472301136	4.93778786092215e-05	\\
5734.68350497159	4.89563022074392e-05	\\
5735.66228693182	4.80869145645861e-05	\\
5736.64106889205	4.90675947298407e-05	\\
5737.61985085227	4.97747587237306e-05	\\
5738.5986328125	4.83111300805005e-05	\\
5739.57741477273	4.93027934171563e-05	\\
5740.55619673295	4.78774385448513e-05	\\
5741.53497869318	5.02679766782081e-05	\\
5742.51376065341	5.05763279657206e-05	\\
5743.49254261364	4.96188249558055e-05	\\
5744.47132457386	4.99643028055478e-05	\\
5745.45010653409	4.92101898255e-05	\\
5746.42888849432	4.86091477821773e-05	\\
5747.40767045455	5.08332721497225e-05	\\
5748.38645241477	4.83568002375502e-05	\\
5749.365234375	4.81114750468661e-05	\\
5750.34401633523	4.91666491421451e-05	\\
5751.32279829545	4.94957655728177e-05	\\
5752.30158025568	4.71818424045762e-05	\\
5753.28036221591	4.96405980219769e-05	\\
5754.25914417614	4.82660453722059e-05	\\
5755.23792613636	4.7429991326239e-05	\\
5756.21670809659	5.04675159867375e-05	\\
5757.19549005682	4.94698038767038e-05	\\
5758.17427201705	4.69211342824051e-05	\\
5759.15305397727	4.96982155270145e-05	\\
5760.1318359375	4.98582437601227e-05	\\
5761.11061789773	4.86082731994162e-05	\\
5762.08939985795	4.79450285355443e-05	\\
5763.06818181818	4.9408534112846e-05	\\
5764.04696377841	4.7180378247122e-05	\\
5765.02574573864	4.68688018407762e-05	\\
5766.00452769886	4.76211753507392e-05	\\
5766.98330965909	4.87022378236128e-05	\\
5767.96209161932	5.09341277710116e-05	\\
5768.94087357955	4.91908073774972e-05	\\
5769.91965553977	5.05349957980867e-05	\\
5770.8984375	5.0978970327753e-05	\\
5771.87721946023	4.93268018313522e-05	\\
5772.85600142045	4.77002091059149e-05	\\
5773.83478338068	5.03051930406672e-05	\\
5774.81356534091	5.00044748907402e-05	\\
5775.79234730114	5.05354583448017e-05	\\
5776.77112926136	4.82468022400899e-05	\\
5777.74991122159	4.98285643252573e-05	\\
5778.72869318182	4.78722558426694e-05	\\
5779.70747514205	4.89209567342659e-05	\\
5780.68625710227	4.87603047158454e-05	\\
5781.6650390625	4.93787245305335e-05	\\
5782.64382102273	4.97785603626568e-05	\\
5783.62260298295	5.00175643117528e-05	\\
5784.60138494318	5.02603075173937e-05	\\
5785.58016690341	4.93009877977198e-05	\\
5786.55894886364	4.83831378365961e-05	\\
5787.53773082386	4.9096225316803e-05	\\
5788.51651278409	4.90757902904676e-05	\\
5789.49529474432	4.83062022902548e-05	\\
5790.47407670455	4.99445718497006e-05	\\
5791.45285866477	5.14732990379788e-05	\\
5792.431640625	4.8816735856732e-05	\\
5793.41042258523	4.96422440783016e-05	\\
5794.38920454545	5.20424350171053e-05	\\
5795.36798650568	4.98435331827063e-05	\\
5796.34676846591	4.87915362011045e-05	\\
5797.32555042614	4.7777522462503e-05	\\
5798.30433238636	4.78305735034835e-05	\\
5799.28311434659	4.98909281091283e-05	\\
5800.26189630682	4.76271663474923e-05	\\
5801.24067826705	5.02871870980296e-05	\\
5802.21946022727	5.00137648481106e-05	\\
5803.1982421875	4.95711370403547e-05	\\
5804.17702414773	4.88660247981718e-05	\\
5805.15580610795	5.01392696916571e-05	\\
5806.13458806818	4.90674398912508e-05	\\
5807.11337002841	4.80616247649043e-05	\\
5808.09215198864	5.03829447208265e-05	\\
5809.07093394886	4.87721490841949e-05	\\
5810.04971590909	4.96933342318186e-05	\\
5811.02849786932	4.99237878367598e-05	\\
5812.00727982955	4.99370873518735e-05	\\
5812.98606178977	4.94081015878684e-05	\\
5813.96484375	5.02235132042632e-05	\\
5814.94362571023	4.95947211867235e-05	\\
5815.92240767045	5.01999088091e-05	\\
5816.90118963068	5.05139772309526e-05	\\
5817.87997159091	5.02111203368015e-05	\\
5818.85875355114	5.07104048511391e-05	\\
5819.83753551136	4.93623115984462e-05	\\
5820.81631747159	4.95716005174327e-05	\\
5821.79509943182	4.96231201033167e-05	\\
5822.77388139205	4.86374093350349e-05	\\
5823.75266335227	4.97251568045222e-05	\\
5824.7314453125	5.0408059863681e-05	\\
5825.71022727273	5.15620041678507e-05	\\
5826.68900923295	4.91140510064577e-05	\\
5827.66779119318	4.98194497136489e-05	\\
5828.64657315341	4.97962887894551e-05	\\
5829.62535511364	4.99817968391668e-05	\\
5830.60413707386	4.97641698083844e-05	\\
5831.58291903409	5.09072809724031e-05	\\
5832.56170099432	4.9801910597789e-05	\\
5833.54048295455	4.96019417600064e-05	\\
5834.51926491477	5.07416076573979e-05	\\
5835.498046875	5.09547633076683e-05	\\
5836.47682883523	4.93319605440484e-05	\\
5837.45561079545	5.30053015605105e-05	\\
5838.43439275568	5.02571085850191e-05	\\
5839.41317471591	4.9625151245247e-05	\\
5840.39195667614	5.14119724649274e-05	\\
5841.37073863636	5.05326558340157e-05	\\
5842.34952059659	5.22160613987094e-05	\\
5843.32830255682	5.06144294480151e-05	\\
5844.30708451705	5.06739986617045e-05	\\
5845.28586647727	5.0641033788731e-05	\\
5846.2646484375	5.03278646947644e-05	\\
5847.24343039773	5.13720872578597e-05	\\
5848.22221235795	5.07100351213137e-05	\\
5849.20099431818	5.18673499056584e-05	\\
5850.17977627841	4.99075243306061e-05	\\
5851.15855823864	5.10755678707407e-05	\\
5852.13734019886	5.16630140791185e-05	\\
5853.11612215909	4.99900616369076e-05	\\
5854.09490411932	5.04390531996106e-05	\\
5855.07368607955	5.2805832425403e-05	\\
5856.05246803977	5.25234601751812e-05	\\
5857.03125	5.11157437998995e-05	\\
5858.01003196023	5.11080580520494e-05	\\
5858.98881392045	5.35261016771053e-05	\\
5859.96759588068	5.24461048454638e-05	\\
5860.94637784091	5.02303704253396e-05	\\
5861.92515980114	5.1017639646636e-05	\\
5862.90394176136	5.22583022234989e-05	\\
5863.88272372159	5.22898153593043e-05	\\
5864.86150568182	5.00636360870354e-05	\\
5865.84028764205	5.22863762889544e-05	\\
5866.81906960227	5.22645035379325e-05	\\
5867.7978515625	5.41814005446769e-05	\\
5868.77663352273	5.24124179269137e-05	\\
5869.75541548295	5.06941099620276e-05	\\
5870.73419744318	5.12273071831302e-05	\\
5871.71297940341	5.08565766410825e-05	\\
5872.69176136364	5.11863492771192e-05	\\
5873.67054332386	5.37453745488281e-05	\\
5874.64932528409	5.24216519818594e-05	\\
5875.62810724432	5.08314628214379e-05	\\
5876.60688920455	5.26437045044948e-05	\\
5877.58567116477	5.26404013231847e-05	\\
5878.564453125	5.06232316348428e-05	\\
5879.54323508523	5.46830774179892e-05	\\
5880.52201704545	5.40532209560155e-05	\\
5881.50079900568	5.29636782437591e-05	\\
5882.47958096591	5.33719831066658e-05	\\
5883.45836292614	5.26914535016508e-05	\\
5884.43714488636	5.50420820880533e-05	\\
5885.41592684659	5.51353226220401e-05	\\
5886.39470880682	5.34583895713244e-05	\\
5887.37349076705	5.31961923055543e-05	\\
5888.35227272727	5.33637797160412e-05	\\
5889.3310546875	5.36935514602913e-05	\\
5890.30983664773	5.3455338269545e-05	\\
5891.28861860795	5.39426975564004e-05	\\
5892.26740056818	5.69474869061982e-05	\\
5893.24618252841	5.59651154785503e-05	\\
5894.22496448864	5.42632043005424e-05	\\
5895.20374644886	5.47219372052183e-05	\\
5896.18252840909	5.56329701254502e-05	\\
5897.16131036932	5.44611550298991e-05	\\
5898.14009232955	5.56379587418691e-05	\\
5899.11887428977	5.59627271910652e-05	\\
5900.09765625	5.77329673705188e-05	\\
5901.07643821023	5.76102919315336e-05	\\
5902.05522017045	5.62377429260517e-05	\\
5903.03400213068	5.69617053444737e-05	\\
5904.01278409091	5.7634289241462e-05	\\
5904.99156605114	5.6827673959945e-05	\\
5905.97034801136	5.69690289572768e-05	\\
5906.94912997159	5.48807750361098e-05	\\
5907.92791193182	5.6926596264446e-05	\\
5908.90669389205	5.70966864117014e-05	\\
5909.88547585227	5.65342502019014e-05	\\
5910.8642578125	5.72976409698206e-05	\\
5911.84303977273	6.00795293704865e-05	\\
5912.82182173295	5.99945033626953e-05	\\
5913.80060369318	5.90101825177734e-05	\\
5914.77938565341	5.86047822646737e-05	\\
5915.75816761364	5.80085375813748e-05	\\
5916.73694957386	5.770561364721e-05	\\
5917.71573153409	5.78895360364927e-05	\\
5918.69451349432	5.85084538007806e-05	\\
5919.67329545455	5.84372165362141e-05	\\
5920.65207741477	5.71976703743129e-05	\\
5921.630859375	5.72464639365061e-05	\\
5922.60964133523	5.67782607552906e-05	\\
5923.58842329545	5.77146004393412e-05	\\
5924.56720525568	5.84209297988149e-05	\\
5925.54598721591	5.85887722662543e-05	\\
5926.52476917614	5.97542063437484e-05	\\
5927.50355113636	6.04665503612385e-05	\\
5928.48233309659	5.87660683052012e-05	\\
5929.46111505682	5.78215606160984e-05	\\
5930.43989701705	5.81706029987684e-05	\\
5931.41867897727	5.86221543879334e-05	\\
5932.3974609375	5.94805994978236e-05	\\
5933.37624289773	5.87501233223759e-05	\\
5934.35502485795	5.68934279187462e-05	\\
5935.33380681818	6.02071308347579e-05	\\
5936.31258877841	5.97486139478132e-05	\\
5937.29137073864	5.93003500590813e-05	\\
5938.27015269886	5.85296047921078e-05	\\
5939.24893465909	5.97995195448916e-05	\\
5940.22771661932	5.78137038907172e-05	\\
5941.20649857955	5.9053641649525e-05	\\
5942.18528053977	5.68394929946787e-05	\\
5943.1640625	5.74359009782383e-05	\\
5944.14284446023	5.61562319296421e-05	\\
5945.12162642045	5.97090883561418e-05	\\
5946.10040838068	5.89432340948475e-05	\\
5947.07919034091	5.71673230529677e-05	\\
5948.05797230114	5.85252566947338e-05	\\
5949.03675426136	5.81459186409619e-05	\\
5950.01553622159	5.96731752131525e-05	\\
5950.99431818182	6.00065508715465e-05	\\
5951.97310014205	5.88860224762315e-05	\\
5952.95188210227	5.94573754834947e-05	\\
5953.9306640625	5.89168282272376e-05	\\
5954.90944602273	6.01108686825292e-05	\\
5955.88822798295	5.87218493761564e-05	\\
5956.86700994318	5.96103014325082e-05	\\
5957.84579190341	5.82024584300571e-05	\\
5958.82457386364	5.91589160665098e-05	\\
5959.80335582386	5.96801787458388e-05	\\
5960.78213778409	5.77181895046704e-05	\\
5961.76091974432	5.87486287527098e-05	\\
5962.73970170455	5.87181332398945e-05	\\
5963.71848366477	5.98556746398701e-05	\\
5964.697265625	5.80021034449802e-05	\\
5965.67604758523	5.64864891716676e-05	\\
5966.65482954545	5.95937712629629e-05	\\
5967.63361150568	5.82176193300957e-05	\\
5968.61239346591	5.85730502740915e-05	\\
5969.59117542614	5.86922073950669e-05	\\
5970.56995738636	5.60336536841358e-05	\\
5971.54873934659	5.80203829622423e-05	\\
5972.52752130682	5.78389100951427e-05	\\
5973.50630326705	5.75179420336966e-05	\\
5974.48508522727	5.91840701259097e-05	\\
5975.4638671875	5.88055303235789e-05	\\
5976.44264914773	5.82905335866653e-05	\\
5977.42143110795	5.95861576780982e-05	\\
5978.40021306818	5.74813771227778e-05	\\
5979.37899502841	5.86523239241428e-05	\\
5980.35777698864	5.62938208519332e-05	\\
5981.33655894886	5.9394881004295e-05	\\
5982.31534090909	5.98537864378191e-05	\\
5983.29412286932	6.02236500772331e-05	\\
5984.27290482955	6.10554181392819e-05	\\
5985.25168678977	5.98755087308932e-05	\\
5986.23046875	6.05047515171938e-05	\\
5987.20925071023	5.87741919097929e-05	\\
5988.18803267045	6.02025317601968e-05	\\
5989.16681463068	5.84397410351316e-05	\\
5990.14559659091	5.86825951356904e-05	\\
5991.12437855114	5.86945186640803e-05	\\
5992.10316051136	5.85211658982384e-05	\\
5993.08194247159	6.08743850959692e-05	\\
5994.06072443182	6.03536171613209e-05	\\
5995.03950639205	5.95054640668586e-05	\\
5996.01828835227	5.92257035952468e-05	\\
5996.9970703125	5.87124758312411e-05	\\
5997.97585227273	5.86823681486668e-05	\\
5998.95463423295	5.75654031039531e-05	\\
5999.93341619318	5.85929925606939e-05	\\
6000.91219815341	5.94581169175092e-05	\\
6001.89098011364	6.12840305361512e-05	\\
6002.86976207386	6.05238371085807e-05	\\
6003.84854403409	5.99481278691045e-05	\\
6004.82732599432	5.84622728813506e-05	\\
6005.80610795455	6.03375803321217e-05	\\
6006.78488991477	5.95267733201565e-05	\\
6007.763671875	6.01231466088927e-05	\\
6008.74245383523	5.96338075764688e-05	\\
6009.72123579545	6.03715940677951e-05	\\
6010.70001775568	6.01411189636289e-05	\\
6011.67879971591	5.98629380513953e-05	\\
6012.65758167614	6.01299063872559e-05	\\
6013.63636363636	5.93688220566664e-05	\\
6014.61514559659	6.06331471530025e-05	\\
6015.59392755682	5.77357974364337e-05	\\
6016.57270951705	6.07685050308632e-05	\\
6017.55149147727	5.82341769960358e-05	\\
6018.5302734375	5.96683840752106e-05	\\
6019.50905539773	5.95685429466559e-05	\\
6020.48783735795	5.83391105988306e-05	\\
6021.46661931818	6.02542558320374e-05	\\
6022.44540127841	5.90570435753353e-05	\\
6023.42418323864	5.93210738576833e-05	\\
6024.40296519886	6.11182666308538e-05	\\
6025.38174715909	5.85091591578471e-05	\\
6026.36052911932	5.74786651519378e-05	\\
6027.33931107955	5.91195914996591e-05	\\
6028.31809303977	5.71513757041326e-05	\\
6029.296875	5.75065513350399e-05	\\
6030.27565696023	5.95849297231555e-05	\\
6031.25443892045	5.81326321508436e-05	\\
6032.23322088068	5.85152219493593e-05	\\
6033.21200284091	5.80988404012183e-05	\\
6034.19078480114	5.90821796660557e-05	\\
6035.16956676136	5.75208422093864e-05	\\
6036.14834872159	5.88821381479377e-05	\\
6037.12713068182	5.87365622003131e-05	\\
6038.10591264205	5.80739561565311e-05	\\
6039.08469460227	5.65217340628233e-05	\\
6040.0634765625	5.54457111307825e-05	\\
6041.04225852273	5.58603042291151e-05	\\
6042.02104048295	5.64725518953844e-05	\\
6042.99982244318	5.70916111030895e-05	\\
6043.97860440341	5.6680607631996e-05	\\
6044.95738636364	5.68292581759093e-05	\\
6045.93616832386	5.67976844663829e-05	\\
6046.91495028409	5.73050686240248e-05	\\
6047.89373224432	5.57119396590558e-05	\\
6048.87251420455	5.66119733342787e-05	\\
6049.85129616477	5.8150403107133e-05	\\
6050.830078125	5.5250900758408e-05	\\
6051.80886008523	5.62776258457804e-05	\\
6052.78764204545	5.47406551927149e-05	\\
6053.76642400568	5.69339823550778e-05	\\
6054.74520596591	5.64111411417607e-05	\\
6055.72398792614	5.62591058127698e-05	\\
6056.70276988636	5.6048085634039e-05	\\
6057.68155184659	5.53336932137791e-05	\\
6058.66033380682	5.59869661015977e-05	\\
6059.63911576705	5.6454912028718e-05	\\
6060.61789772727	5.58947027774702e-05	\\
6061.5966796875	5.5215650423887e-05	\\
6062.57546164773	5.45940326524626e-05	\\
6063.55424360795	5.57004007257924e-05	\\
6064.53302556818	5.65844669997528e-05	\\
6065.51180752841	5.43514360533867e-05	\\
6066.49058948864	5.6858783250968e-05	\\
6067.46937144886	5.34546507649591e-05	\\
6068.44815340909	5.39519318686225e-05	\\
6069.42693536932	5.47053123014576e-05	\\
6070.40571732955	5.57547199155084e-05	\\
6071.38449928977	5.34709178546808e-05	\\
6072.36328125	5.66426916876549e-05	\\
6073.34206321023	5.49209406125284e-05	\\
6074.32084517045	5.61607547225818e-05	\\
6075.29962713068	5.46640346975639e-05	\\
6076.27840909091	5.33801492888289e-05	\\
6077.25719105114	5.54295293343279e-05	\\
6078.23597301136	5.4714500397574e-05	\\
6079.21475497159	5.48576071811904e-05	\\
6080.19353693182	5.45130921177011e-05	\\
6081.17231889205	5.18428926753968e-05	\\
6082.15110085227	5.41497603733168e-05	\\
6083.1298828125	5.54714436079603e-05	\\
6084.10866477273	5.47593430287814e-05	\\
6085.08744673295	5.46507672478761e-05	\\
6086.06622869318	5.42235864541978e-05	\\
6087.04501065341	5.22289283631555e-05	\\
6088.02379261364	5.50169850258233e-05	\\
6089.00257457386	5.41200566198018e-05	\\
6089.98135653409	5.35968955697534e-05	\\
6090.96013849432	5.55566977465338e-05	\\
6091.93892045455	5.46939497556591e-05	\\
6092.91770241477	5.6023299985465e-05	\\
6093.896484375	5.40573823484536e-05	\\
6094.87526633523	5.31571966350921e-05	\\
6095.85404829545	5.52107643852292e-05	\\
6096.83283025568	5.53127644676894e-05	\\
6097.81161221591	5.48623766279531e-05	\\
6098.79039417614	5.37749063722546e-05	\\
6099.76917613636	5.37481031223201e-05	\\
6100.74795809659	5.36259378149283e-05	\\
6101.72674005682	5.51315389002549e-05	\\
6102.70552201705	5.45245998638557e-05	\\
6103.68430397727	5.24896815268996e-05	\\
6104.6630859375	5.71736970203214e-05	\\
6105.64186789773	5.55663067281458e-05	\\
6106.62064985795	5.4938981373725e-05	\\
6107.59943181818	5.52324531548914e-05	\\
6108.57821377841	5.49932676958569e-05	\\
6109.55699573864	5.39935515535475e-05	\\
6110.53577769886	5.6516927371672e-05	\\
6111.51455965909	5.38439494264542e-05	\\
6112.49334161932	5.57981026227144e-05	\\
6113.47212357955	5.38343798615672e-05	\\
6114.45090553977	5.45574737301039e-05	\\
6115.4296875	5.52180725449087e-05	\\
6116.40846946023	5.76233240297576e-05	\\
6117.38725142045	5.38321300333664e-05	\\
6118.36603338068	5.3799924968705e-05	\\
6119.34481534091	5.62979108862483e-05	\\
6120.32359730114	5.35304908702715e-05	\\
6121.30237926136	5.47744109439061e-05	\\
6122.28116122159	5.57649932413487e-05	\\
6123.25994318182	5.75679111654077e-05	\\
6124.23872514205	5.47507251986897e-05	\\
6125.21750710227	5.7864536292163e-05	\\
6126.1962890625	5.72338046171547e-05	\\
6127.17507102273	5.61679747992524e-05	\\
6128.15385298295	5.622970593066e-05	\\
6129.13263494318	5.55717736595266e-05	\\
6130.11141690341	5.60396098938404e-05	\\
6131.09019886364	5.68691742652793e-05	\\
6132.06898082386	5.5684666562008e-05	\\
6133.04776278409	5.53317827496723e-05	\\
6134.02654474432	5.77877796396765e-05	\\
6135.00532670455	5.56326270199875e-05	\\
6135.98410866477	5.50936036610464e-05	\\
6136.962890625	5.55414992470041e-05	\\
6137.94167258523	5.71038126036363e-05	\\
6138.92045454545	5.63987127180871e-05	\\
6139.89923650568	5.45993270985501e-05	\\
6140.87801846591	5.49054976078372e-05	\\
6141.85680042614	5.63891814296827e-05	\\
6142.83558238636	5.56175512256403e-05	\\
6143.81436434659	5.39081736959463e-05	\\
6144.79314630682	5.30775469602899e-05	\\
6145.77192826705	5.68851793810256e-05	\\
6146.75071022727	5.588721167902e-05	\\
6147.7294921875	5.53516828675426e-05	\\
6148.70827414773	5.49676033934502e-05	\\
6149.68705610795	5.59517409482929e-05	\\
6150.66583806818	5.57573984049868e-05	\\
6151.64462002841	5.54936874167462e-05	\\
6152.62340198864	5.62276764392653e-05	\\
6153.60218394886	5.50713628767112e-05	\\
6154.58096590909	5.69449376740716e-05	\\
6155.55974786932	5.63946867884191e-05	\\
6156.53852982955	5.51571437285046e-05	\\
6157.51731178977	5.58683703066842e-05	\\
6158.49609375	5.66567704298934e-05	\\
6159.47487571023	5.74269730492523e-05	\\
6160.45365767045	5.64278590408159e-05	\\
6161.43243963068	5.6922772883997e-05	\\
6162.41122159091	5.61729972980475e-05	\\
6163.39000355114	5.70264417721116e-05	\\
6164.36878551136	5.76162052415429e-05	\\
6165.34756747159	5.71297695996909e-05	\\
6166.32634943182	5.51667854869031e-05	\\
6167.30513139205	5.53998267860703e-05	\\
6168.28391335227	5.6323611704291e-05	\\
6169.2626953125	5.7587083248045e-05	\\
6170.24147727273	5.59161015880906e-05	\\
6171.22025923295	5.79470171408548e-05	\\
6172.19904119318	5.753519988136e-05	\\
6173.17782315341	5.62928953245607e-05	\\
6174.15660511364	5.72984836635769e-05	\\
6175.13538707386	5.72943605544067e-05	\\
6176.11416903409	5.47044814437157e-05	\\
6177.09295099432	5.68377357285437e-05	\\
6178.07173295455	5.76755825636368e-05	\\
6179.05051491477	5.87869295274843e-05	\\
6180.029296875	5.95663754082392e-05	\\
6181.00807883523	5.62089150937829e-05	\\
6181.98686079545	5.84992764132729e-05	\\
6182.96564275568	5.79928949172908e-05	\\
6183.94442471591	5.83246466720935e-05	\\
6184.92320667614	5.53747618438561e-05	\\
6185.90198863636	5.78311409477675e-05	\\
6186.88077059659	5.77467721153258e-05	\\
6187.85955255682	5.70145131376368e-05	\\
6188.83833451705	5.82025875937427e-05	\\
6189.81711647727	5.82385398862302e-05	\\
6190.7958984375	5.67414792098007e-05	\\
6191.77468039773	5.81718603914053e-05	\\
6192.75346235795	5.76220202250036e-05	\\
6193.73224431818	5.59070399412661e-05	\\
6194.71102627841	5.63896916965906e-05	\\
6195.68980823864	5.96570479456135e-05	\\
6196.66859019886	6.02561766333512e-05	\\
6197.64737215909	5.69242263169051e-05	\\
6198.62615411932	5.68841865707075e-05	\\
6199.60493607955	5.5989119878216e-05	\\
6200.58371803977	5.91400158576399e-05	\\
6201.5625	5.94515504249685e-05	\\
6202.54128196023	5.78694051500478e-05	\\
6203.52006392045	5.81639981990553e-05	\\
6204.49884588068	5.90899312262403e-05	\\
6205.47762784091	5.97754511679454e-05	\\
6206.45640980114	5.67917129501767e-05	\\
6207.43519176136	5.79800310733106e-05	\\
6208.41397372159	5.78201092993629e-05	\\
6209.39275568182	5.72649498652428e-05	\\
6210.37153764205	5.70290784416205e-05	\\
6211.35031960227	5.93232882668334e-05	\\
6212.3291015625	5.72090209720305e-05	\\
6213.30788352273	5.78346697061413e-05	\\
6214.28666548295	5.79826325901988e-05	\\
6215.26544744318	5.77034118913135e-05	\\
6216.24422940341	5.68670959842752e-05	\\
6217.22301136364	5.78898193092839e-05	\\
6218.20179332386	5.71475072116063e-05	\\
6219.18057528409	5.83148576746098e-05	\\
6220.15935724432	5.67726640986e-05	\\
6221.13813920455	5.63877802590591e-05	\\
6222.11692116477	5.67975085607852e-05	\\
6223.095703125	5.7157597778341e-05	\\
6224.07448508523	5.79428219418771e-05	\\
6225.05326704545	5.57701072445536e-05	\\
6226.03204900568	5.84478361848959e-05	\\
6227.01083096591	5.65525265903683e-05	\\
6227.98961292614	5.75954299906496e-05	\\
6228.96839488636	5.66365690463968e-05	\\
6229.94717684659	5.65550302924019e-05	\\
6230.92595880682	5.67437799363817e-05	\\
6231.90474076705	5.53533513498113e-05	\\
6232.88352272727	5.501110388196e-05	\\
6233.8623046875	5.54787416255427e-05	\\
6234.84108664773	5.67769133135356e-05	\\
6235.81986860795	5.61785517512806e-05	\\
6236.79865056818	5.61383194029878e-05	\\
6237.77743252841	5.51639005894377e-05	\\
6238.75621448864	5.47659826013004e-05	\\
6239.73499644886	5.631169136277e-05	\\
6240.71377840909	5.47391453668372e-05	\\
6241.69256036932	5.35392028666831e-05	\\
6242.67134232955	5.69869355703343e-05	\\
6243.65012428977	5.60441473086439e-05	\\
6244.62890625	5.57668692858325e-05	\\
6245.60768821023	5.50008093494956e-05	\\
6246.58647017045	5.45615764905961e-05	\\
6247.56525213068	5.40144003634887e-05	\\
6248.54403409091	5.33981377417626e-05	\\
6249.52281605114	5.38435808153684e-05	\\
6250.50159801136	5.22735410718973e-05	\\
6251.48037997159	5.41703993313709e-05	\\
6252.45916193182	5.3139090007857e-05	\\
6253.43794389205	5.32175058445114e-05	\\
6254.41672585227	5.29605304130442e-05	\\
6255.3955078125	5.3524242731454e-05	\\
6256.37428977273	5.23555766486948e-05	\\
6257.35307173295	5.06467184570909e-05	\\
6258.33185369318	5.17008439750735e-05	\\
6259.31063565341	5.11223944025178e-05	\\
6260.28941761364	5.30516365923332e-05	\\
6261.26819957386	5.34219467049386e-05	\\
6262.24698153409	5.26650017514363e-05	\\
6263.22576349432	5.07750032741249e-05	\\
6264.20454545455	5.13680940999706e-05	\\
6265.18332741477	5.10244058479618e-05	\\
6266.162109375	5.24601804203017e-05	\\
6267.14089133523	5.30347777751816e-05	\\
6268.11967329545	5.33847009457553e-05	\\
6269.09845525568	5.04019880707679e-05	\\
6270.07723721591	5.17957774947665e-05	\\
6271.05601917614	5.15734547216377e-05	\\
6272.03480113636	5.0766595838035e-05	\\
6273.01358309659	5.16573934871597e-05	\\
6273.99236505682	5.05396941642924e-05	\\
6274.97114701705	5.12133671129541e-05	\\
6275.94992897727	5.17298304537335e-05	\\
6276.9287109375	5.10598155566031e-05	\\
6277.90749289773	5.15697542606485e-05	\\
6278.88627485795	5.05690878723368e-05	\\
6279.86505681818	4.92288080308909e-05	\\
6280.84383877841	5.12593406914722e-05	\\
6281.82262073864	5.06845947855629e-05	\\
6282.80140269886	5.14454313080598e-05	\\
6283.78018465909	4.93686799094773e-05	\\
6284.75896661932	5.14533329939841e-05	\\
6285.73774857955	5.11098026104382e-05	\\
6286.71653053977	4.9598225063437e-05	\\
6287.6953125	4.98841060863994e-05	\\
6288.67409446023	5.06538961000328e-05	\\
6289.65287642045	4.98432539168389e-05	\\
6290.63165838068	5.19051677923104e-05	\\
6291.61044034091	5.04565119763992e-05	\\
6292.58922230114	4.77378106998013e-05	\\
6293.56800426136	4.94535502238242e-05	\\
6294.54678622159	5.00700216098081e-05	\\
6295.52556818182	4.98533293829557e-05	\\
6296.50435014205	5.01390279280097e-05	\\
6297.48313210227	5.03182682404246e-05	\\
6298.4619140625	5.11264991339072e-05	\\
6299.44069602273	4.97880770210772e-05	\\
6300.41947798295	5.17552922481331e-05	\\
6301.39825994318	5.00374456915151e-05	\\
6302.37704190341	4.83256839242708e-05	\\
6303.35582386364	4.89849427020373e-05	\\
6304.33460582386	4.84034531908652e-05	\\
6305.31338778409	4.72784292694943e-05	\\
6306.29216974432	4.82108884243413e-05	\\
6307.27095170455	4.78800849492238e-05	\\
6308.24973366477	4.93636664711688e-05	\\
6309.228515625	4.67392692024983e-05	\\
6310.20729758523	4.8338928848676e-05	\\
6311.18607954545	4.80269501739927e-05	\\
6312.16486150568	4.92949642736065e-05	\\
6313.14364346591	4.77175300140462e-05	\\
6314.12242542614	4.89549176243932e-05	\\
6315.10120738636	4.90553113637767e-05	\\
6316.07998934659	4.82816779814501e-05	\\
6317.05877130682	4.73910891866886e-05	\\
6318.03755326705	4.94004032015665e-05	\\
6319.01633522727	4.81553063079682e-05	\\
6319.9951171875	4.90849991723488e-05	\\
6320.97389914773	4.85805972888262e-05	\\
6321.95268110795	4.83710721861615e-05	\\
6322.93146306818	4.80479681808039e-05	\\
6323.91024502841	4.88174538251445e-05	\\
6324.88902698864	4.75161552367628e-05	\\
6325.86780894886	4.88418073064896e-05	\\
6326.84659090909	4.79940360762883e-05	\\
6327.82537286932	4.78159758041182e-05	\\
6328.80415482955	4.71508225190399e-05	\\
6329.78293678977	4.95587074648999e-05	\\
6330.76171875	4.84851157292674e-05	\\
6331.74050071023	4.86971440331423e-05	\\
6332.71928267045	5.09229656561019e-05	\\
6333.69806463068	4.69762912155933e-05	\\
6334.67684659091	4.87375161623548e-05	\\
6335.65562855114	4.86745823530078e-05	\\
6336.63441051136	4.76649803421574e-05	\\
6337.61319247159	4.68809233538259e-05	\\
6338.59197443182	4.89528753071525e-05	\\
6339.57075639205	4.82561105249063e-05	\\
6340.54953835227	4.8648047775126e-05	\\
6341.5283203125	4.80635748772252e-05	\\
6342.50710227273	5.04550870264525e-05	\\
6343.48588423295	4.78297124654601e-05	\\
6344.46466619318	4.82558019714636e-05	\\
6345.44344815341	4.89485023420027e-05	\\
6346.42223011364	4.98500737292148e-05	\\
6347.40101207386	4.92460397141231e-05	\\
6348.37979403409	5.02816763393021e-05	\\
6349.35857599432	4.74197665057179e-05	\\
6350.33735795455	4.73917048390285e-05	\\
6351.31613991477	4.86217651964005e-05	\\
6352.294921875	4.89037125727011e-05	\\
6353.27370383523	4.85308126462026e-05	\\
6354.25248579545	4.9008475170637e-05	\\
6355.23126775568	5.14377298500664e-05	\\
6356.21004971591	4.93690018881049e-05	\\
6357.18883167614	4.99190820716685e-05	\\
6358.16761363636	4.91653486007029e-05	\\
6359.14639559659	5.01071960457785e-05	\\
6360.12517755682	4.92127270901818e-05	\\
6361.10395951705	5.07335119612987e-05	\\
6362.08274147727	4.84493775201274e-05	\\
6363.0615234375	4.96782419842031e-05	\\
6364.04030539773	4.91649357926402e-05	\\
6365.01908735795	4.88772770224057e-05	\\
6365.99786931818	4.92268050418309e-05	\\
6366.97665127841	4.79301912696676e-05	\\
6367.95543323864	5.01268863548316e-05	\\
6368.93421519886	4.92064801514097e-05	\\
6369.91299715909	4.89641143496853e-05	\\
6370.89177911932	4.93386608583307e-05	\\
6371.87056107955	4.91485440792664e-05	\\
6372.84934303977	4.83386700762544e-05	\\
6373.828125	4.9638060771381e-05	\\
6374.80690696023	4.95356634719484e-05	\\
6375.78568892045	5.12676115946077e-05	\\
6376.76447088068	5.05753095367628e-05	\\
6377.74325284091	4.91776205057799e-05	\\
6378.72203480114	4.86185577162231e-05	\\
6379.70081676136	5.07100213040314e-05	\\
6380.67959872159	4.88116515157183e-05	\\
6381.65838068182	4.93552857187171e-05	\\
6382.63716264205	5.15811393620722e-05	\\
6383.61594460227	4.9618099908894e-05	\\
6384.5947265625	4.9233042854408e-05	\\
6385.57350852273	5.04391027433788e-05	\\
6386.55229048295	4.96109747211592e-05	\\
6387.53107244318	5.22681365604714e-05	\\
6388.50985440341	5.09489407523763e-05	\\
6389.48863636364	4.97785433297394e-05	\\
6390.46741832386	5.06297346980278e-05	\\
6391.44620028409	5.34416162159027e-05	\\
6392.42498224432	4.93138747303967e-05	\\
6393.40376420455	5.13758779942963e-05	\\
6394.38254616477	5.23789455886056e-05	\\
6395.361328125	5.02410575858985e-05	\\
6396.34011008523	5.20447532418041e-05	\\
6397.31889204545	5.19543943897218e-05	\\
6398.29767400568	5.12369611985935e-05	\\
6399.27645596591	4.99506385047793e-05	\\
6400.25523792614	5.09294724778151e-05	\\
6401.23401988636	5.09725234533342e-05	\\
6402.21280184659	5.19415833139552e-05	\\
6403.19158380682	5.22865532689179e-05	\\
6404.17036576705	5.280833509614e-05	\\
6405.14914772727	5.35365072693376e-05	\\
6406.1279296875	5.30220663381418e-05	\\
6407.10671164773	5.24276720206565e-05	\\
6408.08549360795	5.28565433403952e-05	\\
6409.06427556818	5.18479235859951e-05	\\
6410.04305752841	5.45595407654757e-05	\\
6411.02183948864	5.43004770900606e-05	\\
6412.00062144886	5.38513767966888e-05	\\
6412.97940340909	5.6817805946723e-05	\\
6413.95818536932	5.47782056236958e-05	\\
6414.93696732955	5.47384611506205e-05	\\
6415.91574928977	5.41617050016065e-05	\\
6416.89453125	5.48459303558714e-05	\\
6417.87331321023	5.5173820467985e-05	\\
6418.85209517045	5.43066429721654e-05	\\
6419.83087713068	5.33789703706016e-05	\\
6420.80965909091	5.50578401214411e-05	\\
6421.78844105114	5.67283733320191e-05	\\
6422.76722301136	5.70813975024056e-05	\\
6423.74600497159	5.57818386920545e-05	\\
6424.72478693182	5.48796055361574e-05	\\
6425.70356889205	5.71617795802698e-05	\\
6426.68235085227	5.5368256228844e-05	\\
6427.6611328125	5.50564867711579e-05	\\
6428.63991477273	5.80202194642472e-05	\\
6429.61869673295	5.67057414808976e-05	\\
6430.59747869318	5.74633403279085e-05	\\
6431.57626065341	5.83519584454986e-05	\\
6432.55504261364	5.63549654736476e-05	\\
6433.53382457386	5.4844465774181e-05	\\
6434.51260653409	5.74517871939178e-05	\\
6435.49138849432	5.78734706605368e-05	\\
6436.47017045455	5.61762654010369e-05	\\
6437.44895241477	5.67244581277914e-05	\\
6438.427734375	5.70989512980827e-05	\\
6439.40651633523	5.79156077228631e-05	\\
6440.38529829545	5.82214777334156e-05	\\
6441.36408025568	5.93166570848479e-05	\\
6442.34286221591	5.90306005314959e-05	\\
6443.32164417614	5.83928053232304e-05	\\
6444.30042613636	5.73431504901066e-05	\\
6445.27920809659	5.84111899247482e-05	\\
6446.25799005682	5.85239978786601e-05	\\
6447.23677201705	5.9588488482696e-05	\\
6448.21555397727	6.1892430679765e-05	\\
6449.1943359375	5.98064497043879e-05	\\
6450.17311789773	6.12098539079458e-05	\\
6451.15189985795	5.91373700873237e-05	\\
6452.13068181818	6.03375259952562e-05	\\
6453.10946377841	5.92820191844736e-05	\\
6454.08824573864	5.85526343291197e-05	\\
6455.06702769886	5.9294720437553e-05	\\
6456.04580965909	5.94887574392721e-05	\\
6457.02459161932	6.19648847926505e-05	\\
6458.00337357955	6.07419929386814e-05	\\
6458.98215553977	5.88074857860498e-05	\\
6459.9609375	6.11725145351342e-05	\\
6460.93971946023	6.1264079335212e-05	\\
6461.91850142045	6.14969820486589e-05	\\
6462.89728338068	5.98397938804661e-05	\\
6463.87606534091	6.0171191936885e-05	\\
6464.85484730114	6.1626397385932e-05	\\
6465.83362926136	6.23794774381257e-05	\\
6466.81241122159	6.16692108629408e-05	\\
6467.79119318182	6.2496758555587e-05	\\
6468.76997514205	6.33844902553763e-05	\\
6469.74875710227	6.15396822224059e-05	\\
6470.7275390625	6.17077300499775e-05	\\
6471.70632102273	6.08221554049281e-05	\\
6472.68510298295	6.17926381852786e-05	\\
6473.66388494318	6.1813604517289e-05	\\
6474.64266690341	6.2441728055844e-05	\\
6475.62144886364	6.13287702554233e-05	\\
6476.60023082386	6.35205946164217e-05	\\
6477.57901278409	6.35779795767656e-05	\\
6478.55779474432	6.34292004499349e-05	\\
6479.53657670455	6.39659693262258e-05	\\
6480.51535866477	6.42065483303028e-05	\\
6481.494140625	6.04158661505141e-05	\\
6482.47292258523	6.27392687352636e-05	\\
6483.45170454545	6.41049640171126e-05	\\
6484.43048650568	6.24804109835448e-05	\\
6485.40926846591	6.54226172661272e-05	\\
6486.38805042614	6.39519099806743e-05	\\
6487.36683238636	6.3270012218767e-05	\\
6488.34561434659	6.29583013460427e-05	\\
6489.32439630682	6.30133139661585e-05	\\
6490.30317826705	6.30851398472701e-05	\\
6491.28196022727	6.32675721669714e-05	\\
6492.2607421875	6.40215940990671e-05	\\
6493.23952414773	6.22658567790595e-05	\\
6494.21830610795	6.39916353941513e-05	\\
6495.19708806818	6.46570804466066e-05	\\
6496.17587002841	6.32535156927438e-05	\\
6497.15465198864	6.47001566000947e-05	\\
6498.13343394886	6.44801413531659e-05	\\
6499.11221590909	6.45470103784469e-05	\\
6500.09099786932	6.51511536762369e-05	\\
6501.06977982955	6.37153410473253e-05	\\
6502.04856178977	6.13088189899854e-05	\\
6503.02734375	6.16179575224146e-05	\\
6504.00612571023	6.36694405319495e-05	\\
6504.98490767045	6.27683112102041e-05	\\
6505.96368963068	6.29559575484223e-05	\\
6506.94247159091	6.3429829582626e-05	\\
6507.92125355114	6.40499883646772e-05	\\
6508.90003551136	6.40221474926087e-05	\\
6509.87881747159	6.40859436315844e-05	\\
6510.85759943182	6.31076795693348e-05	\\
6511.83638139205	6.28367334338091e-05	\\
6512.81516335227	6.33543449832155e-05	\\
6513.7939453125	6.28545178335858e-05	\\
6514.77272727273	6.44824803122216e-05	\\
6515.75150923295	6.52034197064511e-05	\\
6516.73029119318	6.47755154032983e-05	\\
6517.70907315341	6.36845793844993e-05	\\
6518.68785511364	6.2403446628207e-05	\\
6519.66663707386	6.60267899859436e-05	\\
6520.64541903409	6.49604937216315e-05	\\
6521.62420099432	6.54800443569294e-05	\\
6522.60298295455	6.50300758879578e-05	\\
6523.58176491477	6.52701827176533e-05	\\
6524.560546875	6.59324378441639e-05	\\
6525.53932883523	6.48088551921953e-05	\\
6526.51811079545	6.31209451977795e-05	\\
6527.49689275568	6.41292019041528e-05	\\
6528.47567471591	6.5531550265634e-05	\\
6529.45445667614	6.45453210968538e-05	\\
6530.43323863636	6.46551882248713e-05	\\
6531.41202059659	6.52537986808303e-05	\\
6532.39080255682	6.3785885215866e-05	\\
6533.36958451705	6.54008615654529e-05	\\
6534.34836647727	6.33422804632912e-05	\\
6535.3271484375	6.45065675938918e-05	\\
6536.30593039773	6.60803383289231e-05	\\
6537.28471235795	6.56688993765076e-05	\\
6538.26349431818	6.30912024945953e-05	\\
6539.24227627841	6.56778094895324e-05	\\
6540.22105823864	6.37814926091821e-05	\\
6541.19984019886	6.54630268844969e-05	\\
6542.17862215909	6.36738427832961e-05	\\
6543.15740411932	6.58876635080832e-05	\\
6544.13618607955	6.3391736396131e-05	\\
6545.11496803977	6.4449097945682e-05	\\
6546.09375	6.50776042197762e-05	\\
6547.07253196023	6.49811929388739e-05	\\
6548.05131392045	6.43631155290662e-05	\\
6549.03009588068	6.49095605682608e-05	\\
6550.00887784091	6.54594265042243e-05	\\
6550.98765980114	6.30007228157396e-05	\\
6551.96644176136	6.34867819237224e-05	\\
6552.94522372159	6.58403464323771e-05	\\
6553.92400568182	6.5564516041436e-05	\\
6554.90278764205	6.59120010047493e-05	\\
6555.88156960227	6.57114233140179e-05	\\
6556.8603515625	6.40873339070234e-05	\\
6557.83913352273	6.44323878474394e-05	\\
6558.81791548295	6.42370594775224e-05	\\
6559.79669744318	6.53965656563975e-05	\\
6560.77547940341	6.57119869753202e-05	\\
6561.75426136364	6.34596876657299e-05	\\
6562.73304332386	6.38239457911945e-05	\\
6563.71182528409	6.37427703186719e-05	\\
6564.69060724432	6.45433684265238e-05	\\
6565.66938920455	6.42120997798879e-05	\\
6566.64817116477	6.34495320017461e-05	\\
6567.626953125	6.32408893813767e-05	\\
6568.60573508523	6.3948145037949e-05	\\
6569.58451704545	6.27201923911657e-05	\\
6570.56329900568	6.30958276957559e-05	\\
6571.54208096591	6.19696619109435e-05	\\
6572.52086292614	6.42362365639158e-05	\\
6573.49964488636	6.31334848139604e-05	\\
6574.47842684659	6.19627367873301e-05	\\
6575.45720880682	6.3560342483682e-05	\\
6576.43599076705	6.12062436589187e-05	\\
6577.41477272727	6.24911316534834e-05	\\
6578.3935546875	6.31110375491465e-05	\\
6579.37233664773	6.3115572326765e-05	\\
6580.35111860795	6.36175658991961e-05	\\
6581.32990056818	6.27074859851181e-05	\\
6582.30868252841	6.18655360618947e-05	\\
6583.28746448864	6.27476244878202e-05	\\
6584.26624644886	6.30307955289988e-05	\\
6585.24502840909	6.11059708543334e-05	\\
6586.22381036932	6.39316781067426e-05	\\
6587.20259232955	6.38826201780366e-05	\\
6588.18137428977	6.13326709042723e-05	\\
6589.16015625	6.13741147315554e-05	\\
6590.13893821023	6.17188386368902e-05	\\
6591.11772017045	5.96647482276562e-05	\\
6592.09650213068	6.24431243589134e-05	\\
6593.07528409091	6.22417974349387e-05	\\
6594.05406605114	6.27349778748286e-05	\\
6595.03284801136	6.09806325500563e-05	\\
6596.01162997159	6.42365378164646e-05	\\
6596.99041193182	6.28628735548823e-05	\\
6597.96919389205	6.16169089213898e-05	\\
6598.94797585227	6.29801300073714e-05	\\
6599.9267578125	5.98425008089737e-05	\\
6600.90553977273	6.27174593341475e-05	\\
6601.88432173295	6.10413286719911e-05	\\
6602.86310369318	6.13308797015104e-05	\\
6603.84188565341	6.09645867211598e-05	\\
6604.82066761364	6.12296787309555e-05	\\
6605.79944957386	6.25279761679533e-05	\\
6606.77823153409	6.03534119917533e-05	\\
6607.75701349432	6.10939185784104e-05	\\
6608.73579545455	6.20344447121773e-05	\\
6609.71457741477	6.20398729012834e-05	\\
6610.693359375	6.05837564637488e-05	\\
6611.67214133523	6.16305791129016e-05	\\
6612.65092329545	6.17348969772544e-05	\\
6613.62970525568	6.19712237540353e-05	\\
6614.60848721591	6.22249957700379e-05	\\
6615.58726917614	6.11611870817549e-05	\\
6616.56605113636	6.16469089469509e-05	\\
6617.54483309659	5.9770710136547e-05	\\
6618.52361505682	6.05454590036372e-05	\\
6619.50239701705	6.077796839845e-05	\\
6620.48117897727	6.0897827153172e-05	\\
6621.4599609375	6.15111492087288e-05	\\
6622.43874289773	6.22701209386818e-05	\\
6623.41752485795	6.17006715902349e-05	\\
6624.39630681818	6.15527518139132e-05	\\
6625.37508877841	6.13724503190052e-05	\\
6626.35387073864	6.09142556838759e-05	\\
6627.33265269886	6.28253095750311e-05	\\
6628.31143465909	6.14558258630771e-05	\\
6629.29021661932	6.13017173983529e-05	\\
6630.26899857955	6.15749818268543e-05	\\
6631.24778053977	6.13209809176374e-05	\\
6632.2265625	6.28570434131942e-05	\\
6633.20534446023	6.17863350648567e-05	\\
6634.18412642045	6.27648191519905e-05	\\
6635.16290838068	6.08490830189532e-05	\\
6636.14169034091	6.05969096333035e-05	\\
6637.12047230114	6.28803095836403e-05	\\
6638.09925426136	6.17844982357655e-05	\\
6639.07803622159	6.33620824607047e-05	\\
6640.05681818182	6.3363866435333e-05	\\
6641.03560014205	6.21037089491746e-05	\\
6642.01438210227	6.07934647833025e-05	\\
6642.9931640625	6.18370322376139e-05	\\
6643.97194602273	6.10221776470527e-05	\\
6644.95072798295	6.26573171062782e-05	\\
6645.92950994318	6.32144176860341e-05	\\
6646.90829190341	6.31800705744496e-05	\\
6647.88707386364	6.29417438688232e-05	\\
6648.86585582386	6.25272001360583e-05	\\
6649.84463778409	6.048327822235e-05	\\
6650.82341974432	6.09315888255935e-05	\\
6651.80220170455	6.20627748751509e-05	\\
6652.78098366477	6.1245879837471e-05	\\
6653.759765625	6.24231899465047e-05	\\
6654.73854758523	6.20762923666189e-05	\\
6655.71732954545	6.2329904959131e-05	\\
6656.69611150568	6.19126609834887e-05	\\
6657.67489346591	6.14986931027863e-05	\\
6658.65367542614	6.16984314002701e-05	\\
6659.63245738636	5.94851081398564e-05	\\
6660.61123934659	6.10971723330159e-05	\\
6661.59002130682	6.24041848952822e-05	\\
6662.56880326705	6.33108900255869e-05	\\
6663.54758522727	6.0241470217465e-05	\\
6664.5263671875	6.15631561668784e-05	\\
6665.50514914773	6.09302972761615e-05	\\
6666.48393110795	6.13443247091336e-05	\\
6667.46271306818	6.15712175779306e-05	\\
6668.44149502841	6.22104815354327e-05	\\
6669.42027698864	6.10364118878426e-05	\\
6670.39905894886	5.97627653630039e-05	\\
6671.37784090909	6.12798377964338e-05	\\
6672.35662286932	6.32945699026812e-05	\\
6673.33540482955	6.20341262539467e-05	\\
6674.31418678977	6.18920021569511e-05	\\
6675.29296875	6.051417197683e-05	\\
6676.27175071023	6.12196461336616e-05	\\
6677.25053267045	6.29229632351953e-05	\\
6678.22931463068	6.21914940305904e-05	\\
6679.20809659091	6.18251745935845e-05	\\
6680.18687855114	6.0407261505555e-05	\\
6681.16566051136	6.16718048145692e-05	\\
6682.14444247159	6.24956470475049e-05	\\
6683.12322443182	6.14238396552964e-05	\\
6684.10200639205	6.09198397842079e-05	\\
6685.08078835227	6.08848461863796e-05	\\
6686.0595703125	6.15605448828716e-05	\\
6687.03835227273	6.14504907714776e-05	\\
6688.01713423295	6.16815058298059e-05	\\
6688.99591619318	6.09102726525827e-05	\\
6689.97469815341	6.31936159093831e-05	\\
6690.95348011364	6.28936695187638e-05	\\
6691.93226207386	6.26423253460537e-05	\\
6692.91104403409	6.20678653103039e-05	\\
6693.88982599432	6.34279584232607e-05	\\
6694.86860795455	6.43668328120412e-05	\\
6695.84738991477	6.2023221573867e-05	\\
6696.826171875	6.24174998872774e-05	\\
6697.80495383523	6.16019893479661e-05	\\
6698.78373579545	6.3075403534844e-05	\\
6699.76251775568	6.42548166915549e-05	\\
6700.74129971591	6.14318521867244e-05	\\
6701.72008167614	6.32203353713751e-05	\\
6702.69886363636	6.25172575312351e-05	\\
6703.67764559659	6.03106472322084e-05	\\
6704.65642755682	6.36355475545587e-05	\\
6705.63520951705	6.20091673380465e-05	\\
6706.61399147727	6.28299274730406e-05	\\
6707.5927734375	6.1120029161823e-05	\\
6708.57155539773	6.36306953618692e-05	\\
6709.55033735795	6.36006368444583e-05	\\
6710.52911931818	6.22191771075048e-05	\\
6711.50790127841	6.40175937197677e-05	\\
6712.48668323864	6.38587236620751e-05	\\
6713.46546519886	6.36749507166997e-05	\\
6714.44424715909	6.20890096952477e-05	\\
6715.42302911932	6.14863859145157e-05	\\
6716.40181107955	6.29325450984537e-05	\\
6717.38059303977	6.16103403200136e-05	\\
6718.359375	6.26683671033754e-05	\\
6719.33815696023	6.24025134204611e-05	\\
6720.31693892045	6.27231911260292e-05	\\
6721.29572088068	6.19763180093575e-05	\\
6722.27450284091	6.18855200058217e-05	\\
6723.25328480114	6.47373963490118e-05	\\
6724.23206676136	6.5220846227807e-05	\\
6725.21084872159	6.30659716687256e-05	\\
6726.18963068182	6.17426758292082e-05	\\
6727.16841264205	6.29524911584727e-05	\\
6728.14719460227	6.31728364353983e-05	\\
6729.1259765625	6.39798995887416e-05	\\
6730.10475852273	6.2708079466906e-05	\\
6731.08354048295	6.31588668890792e-05	\\
6732.06232244318	6.4082022314166e-05	\\
6733.04110440341	6.39434169254876e-05	\\
6734.01988636364	6.23879235809926e-05	\\
6734.99866832386	6.48722644862677e-05	\\
6735.97745028409	6.48310684507012e-05	\\
6736.95623224432	6.32672424387235e-05	\\
6737.93501420455	6.38119037531556e-05	\\
6738.91379616477	6.37163187711365e-05	\\
6739.892578125	6.36924460286778e-05	\\
6740.87136008523	6.43378859324154e-05	\\
6741.85014204545	6.2897761664308e-05	\\
6742.82892400568	6.30637068505782e-05	\\
6743.80770596591	6.28354502608328e-05	\\
6744.78648792614	6.31265625578433e-05	\\
6745.76526988636	6.5828082699282e-05	\\
6746.74405184659	6.39939045348567e-05	\\
6747.72283380682	6.42535175508145e-05	\\
6748.70161576705	6.64526037562545e-05	\\
6749.68039772727	6.46971994523352e-05	\\
6750.6591796875	6.11184082639544e-05	\\
6751.63796164773	6.62136584519458e-05	\\
6752.61674360795	6.45535132273245e-05	\\
6753.59552556818	6.42368279222577e-05	\\
6754.57430752841	6.23809853073139e-05	\\
6755.55308948864	6.58396681025327e-05	\\
6756.53187144886	6.50471893418544e-05	\\
6757.51065340909	6.5732891870741e-05	\\
6758.48943536932	6.47120843995595e-05	\\
6759.46821732955	6.34634884729804e-05	\\
6760.44699928977	6.40919267199578e-05	\\
6761.42578125	6.4047843228396e-05	\\
6762.40456321023	6.56412800080002e-05	\\
6763.38334517045	6.24051651277236e-05	\\
6764.36212713068	6.58171135376649e-05	\\
6765.34090909091	6.54584919581096e-05	\\
6766.31969105114	6.326030359477e-05	\\
6767.29847301136	6.48336457387416e-05	\\
6768.27725497159	6.32193990411528e-05	\\
6769.25603693182	6.39040054107718e-05	\\
6770.23481889205	6.37534756053213e-05	\\
6771.21360085227	6.36349152568188e-05	\\
6772.1923828125	6.53536386751413e-05	\\
6773.17116477273	6.28939302647474e-05	\\
6774.14994673295	6.30795359479196e-05	\\
6775.12872869318	6.34774811932019e-05	\\
6776.10751065341	6.32276319222291e-05	\\
6777.08629261364	6.34961205948871e-05	\\
6778.06507457386	6.119874354652e-05	\\
6779.04385653409	6.2606479517648e-05	\\
6780.02263849432	6.32021232695356e-05	\\
6781.00142045455	6.27961674153378e-05	\\
6781.98020241477	6.02301353716888e-05	\\
6782.958984375	6.12811210115901e-05	\\
6783.93776633523	6.11077167821823e-05	\\
6784.91654829545	6.10800481645283e-05	\\
6785.89533025568	5.96774276214133e-05	\\
6786.87411221591	6.24694241167678e-05	\\
6787.85289417614	6.25096230923484e-05	\\
6788.83167613636	6.17550722620676e-05	\\
6789.81045809659	6.12334432281967e-05	\\
6790.78924005682	6.12630539038269e-05	\\
6791.76802201705	6.00470292216285e-05	\\
6792.74680397727	6.00404702633338e-05	\\
6793.7255859375	6.1027002960923e-05	\\
6794.70436789773	5.97022483313576e-05	\\
6795.68314985795	5.98385163986803e-05	\\
6796.66193181818	6.27625027175432e-05	\\
6797.64071377841	6.18615943268743e-05	\\
6798.61949573864	6.25550561298202e-05	\\
6799.59827769886	6.04671742658958e-05	\\
6800.57705965909	6.09206841364851e-05	\\
6801.55584161932	6.31290204132667e-05	\\
6802.53462357955	5.98133163798759e-05	\\
6803.51340553977	6.10650539526294e-05	\\
6804.4921875	5.93452225691821e-05	\\
6805.47096946023	5.97660070320082e-05	\\
6806.44975142045	5.99454566155081e-05	\\
6807.42853338068	6.02347962666781e-05	\\
6808.40731534091	6.16675323082322e-05	\\
6809.38609730114	5.91239119738799e-05	\\
6810.36487926136	5.92295726026382e-05	\\
6811.34366122159	6.19054115816831e-05	\\
6812.32244318182	6.11840088379149e-05	\\
6813.30122514205	6.04911689776873e-05	\\
6814.28000710227	6.27838831804155e-05	\\
6815.2587890625	6.03859559608153e-05	\\
6816.23757102273	6.02184953315759e-05	\\
6817.21635298295	6.15624806654336e-05	\\
6818.19513494318	6.09386077370516e-05	\\
6819.17391690341	6.01811056274446e-05	\\
6820.15269886364	6.13536426227757e-05	\\
6821.13148082386	5.90276256444656e-05	\\
6822.11026278409	6.14164237637914e-05	\\
6823.08904474432	6.13995112146263e-05	\\
6824.06782670455	6.07325178675245e-05	\\
6825.04660866477	6.13133810586141e-05	\\
6826.025390625	6.02045672239998e-05	\\
6827.00417258523	5.97281722519794e-05	\\
6827.98295454545	5.97837792057642e-05	\\
6828.96173650568	5.95636855040651e-05	\\
6829.94051846591	6.13202877725867e-05	\\
6830.91930042614	5.94904354500122e-05	\\
6831.89808238636	6.05097744089899e-05	\\
6832.87686434659	5.94924396072088e-05	\\
6833.85564630682	5.64977961407683e-05	\\
6834.83442826705	5.92672950925777e-05	\\
6835.81321022727	5.94183487593451e-05	\\
6836.7919921875	5.76126986490421e-05	\\
6837.77077414773	5.81501374944718e-05	\\
6838.74955610795	5.85555921211162e-05	\\
6839.72833806818	5.75312641972507e-05	\\
6840.70712002841	5.89861042202451e-05	\\
6841.68590198864	5.72831784965317e-05	\\
6842.66468394886	5.63166898150117e-05	\\
6843.64346590909	5.74016050751622e-05	\\
6844.62224786932	5.90558797260436e-05	\\
6845.60102982955	5.68587777414993e-05	\\
6846.57981178977	6.04545591022581e-05	\\
6847.55859375	5.80358954636861e-05	\\
6848.53737571023	5.86484879881613e-05	\\
6849.51615767045	5.82003673061369e-05	\\
6850.49493963068	5.74666948458139e-05	\\
6851.47372159091	5.85617859926601e-05	\\
6852.45250355114	5.69603404551919e-05	\\
6853.43128551136	5.66715287056668e-05	\\
6854.41006747159	5.76345921513953e-05	\\
6855.38884943182	5.86350963470198e-05	\\
6856.36763139205	5.59598693731727e-05	\\
6857.34641335227	5.73568219137205e-05	\\
6858.3251953125	5.63646292031084e-05	\\
6859.30397727273	5.80802362711424e-05	\\
6860.28275923295	5.49594153683065e-05	\\
6861.26154119318	5.68038562673583e-05	\\
6862.24032315341	5.74173542980752e-05	\\
6863.21910511364	5.79975073434141e-05	\\
6864.19788707386	5.65462603905374e-05	\\
6865.17666903409	5.6689402544439e-05	\\
6866.15545099432	5.80932743272519e-05	\\
6867.13423295455	5.58316507756557e-05	\\
6868.11301491477	5.743431475671e-05	\\
6869.091796875	5.70862096755094e-05	\\
6870.07057883523	5.65924093113506e-05	\\
6871.04936079545	5.67015226549683e-05	\\
6872.02814275568	5.58419117556788e-05	\\
6873.00692471591	5.52407085870841e-05	\\
6873.98570667614	5.4835519647518e-05	\\
6874.96448863636	5.68581880725933e-05	\\
6875.94327059659	5.41209916724239e-05	\\
6876.92205255682	5.40123773858571e-05	\\
6877.90083451705	5.65185113654271e-05	\\
6878.87961647727	5.41401553045367e-05	\\
6879.8583984375	5.633872283808e-05	\\
6880.83718039773	5.50958487753671e-05	\\
6881.81596235795	5.54767550353444e-05	\\
6882.79474431818	5.48070454467814e-05	\\
6883.77352627841	5.44206916120312e-05	\\
6884.75230823864	5.57340763516786e-05	\\
6885.73109019886	5.33024533210998e-05	\\
6886.70987215909	5.65198541341177e-05	\\
6887.68865411932	5.3149413028665e-05	\\
6888.66743607955	5.58777900458139e-05	\\
6889.64621803977	5.18617236345882e-05	\\
6890.625	5.28722202184645e-05	\\
6891.60378196023	5.40364907791004e-05	\\
6892.58256392045	5.4747664392776e-05	\\
6893.56134588068	5.54038813622639e-05	\\
6894.54012784091	5.40786555700244e-05	\\
6895.51890980114	5.27708532270854e-05	\\
6896.49769176136	5.49736478507208e-05	\\
6897.47647372159	5.46925045454238e-05	\\
6898.45525568182	5.38157144054355e-05	\\
6899.43403764205	5.32595818516777e-05	\\
6900.41281960227	5.17454731127319e-05	\\
6901.3916015625	5.3218171256962e-05	\\
6902.37038352273	5.35238983208239e-05	\\
6903.34916548295	5.43214788097169e-05	\\
6904.32794744318	5.19210789464829e-05	\\
6905.30672940341	5.26063709955482e-05	\\
6906.28551136364	5.31683239003779e-05	\\
6907.26429332386	5.31173508909767e-05	\\
6908.24307528409	5.34199305983523e-05	\\
6909.22185724432	5.43633408066034e-05	\\
6910.20063920455	5.26187185296193e-05	\\
6911.17942116477	5.3455227445152e-05	\\
6912.158203125	5.51626619719653e-05	\\
6913.13698508523	5.45452688228729e-05	\\
6914.11576704545	5.22178530582701e-05	\\
6915.09454900568	5.48637403420827e-05	\\
6916.07333096591	5.21142086491917e-05	\\
6917.05211292614	5.36476201674738e-05	\\
6918.03089488636	5.33735826848167e-05	\\
6919.00967684659	5.42443584778208e-05	\\
6919.98845880682	5.23823847757942e-05	\\
6920.96724076705	5.35301774998145e-05	\\
6921.94602272727	5.44518780283564e-05	\\
6922.9248046875	5.47182553405516e-05	\\
6923.90358664773	5.21811285382804e-05	\\
6924.88236860795	5.2787065558166e-05	\\
6925.86115056818	5.2550998600305e-05	\\
6926.83993252841	5.29165159209081e-05	\\
6927.81871448864	5.38011805662268e-05	\\
6928.79749644886	5.47236113553727e-05	\\
6929.77627840909	5.4192101181683e-05	\\
6930.75506036932	5.26530244955834e-05	\\
6931.73384232955	5.23005852753318e-05	\\
6932.71262428977	5.2649916385634e-05	\\
6933.69140625	5.38243657702748e-05	\\
6934.67018821023	5.35138928762576e-05	\\
6935.64897017045	5.23347120577271e-05	\\
6936.62775213068	5.25912131773214e-05	\\
6937.60653409091	5.3352948404427e-05	\\
6938.58531605114	5.14688707683835e-05	\\
6939.56409801136	5.43505826132957e-05	\\
6940.54287997159	5.5195202649422e-05	\\
6941.52166193182	5.26284754808971e-05	\\
6942.50044389205	5.24130395376517e-05	\\
6943.47922585227	5.3713363886018e-05	\\
6944.4580078125	5.2913872541146e-05	\\
6945.43678977273	5.41408066370599e-05	\\
6946.41557173295	5.47896876317949e-05	\\
6947.39435369318	5.43977115175001e-05	\\
6948.37313565341	5.42782063412882e-05	\\
6949.35191761364	5.25797103006936e-05	\\
6950.33069957386	5.23796210093829e-05	\\
6951.30948153409	5.32790212459669e-05	\\
6952.28826349432	5.31746788134896e-05	\\
6953.26704545455	5.29943779394305e-05	\\
6954.24582741477	5.285753471294e-05	\\
6955.224609375	5.38416284028755e-05	\\
6956.20339133523	5.12634170715461e-05	\\
6957.18217329545	5.34167344647697e-05	\\
6958.16095525568	5.3069667167221e-05	\\
6959.13973721591	5.25531245338651e-05	\\
6960.11851917614	5.36488213942229e-05	\\
6961.09730113636	5.51147250722082e-05	\\
6962.07608309659	5.4012775139542e-05	\\
6963.05486505682	5.33215619537023e-05	\\
6964.03364701705	5.37038723643971e-05	\\
6965.01242897727	5.34603490972642e-05	\\
6965.9912109375	5.34513424371149e-05	\\
6966.96999289773	5.38105008692629e-05	\\
6967.94877485795	5.27100026684854e-05	\\
6968.92755681818	5.37385804662779e-05	\\
6969.90633877841	5.41167361106488e-05	\\
6970.88512073864	5.25983562473774e-05	\\
6971.86390269886	5.26276738276596e-05	\\
6972.84268465909	5.27623949289899e-05	\\
6973.82146661932	5.22320965579164e-05	\\
6974.80024857955	5.34089040945777e-05	\\
6975.77903053977	5.30684783668952e-05	\\
6976.7578125	5.41507476167174e-05	\\
6977.73659446023	5.20440275472897e-05	\\
6978.71537642045	5.11220089328659e-05	\\
6979.69415838068	5.15262585666197e-05	\\
6980.67294034091	5.26206462158616e-05	\\
6981.65172230114	5.21241270561998e-05	\\
6982.63050426136	5.2880230159809e-05	\\
6983.60928622159	5.05188986309719e-05	\\
6984.58806818182	4.97184073403378e-05	\\
6985.56685014205	5.35591892978161e-05	\\
6986.54563210227	5.0133335703374e-05	\\
6987.5244140625	5.11520659224352e-05	\\
6988.50319602273	5.10537066801284e-05	\\
6989.48197798295	5.21257574212969e-05	\\
6990.46075994318	5.21267131300695e-05	\\
6991.43954190341	5.16424406538759e-05	\\
6992.41832386364	5.04233174498819e-05	\\
6993.39710582386	5.04072549818672e-05	\\
6994.37588778409	5.12941195125041e-05	\\
6995.35466974432	4.88311690708102e-05	\\
6996.33345170455	5.07623582524415e-05	\\
6997.31223366477	5.29058847870144e-05	\\
6998.291015625	4.98471053261082e-05	\\
6999.26979758523	4.95933801759763e-05	\\
7000.24857954545	4.81891802779455e-05	\\
7001.22736150568	4.88173847002555e-05	\\
7002.20614346591	4.94998320754715e-05	\\
7003.18492542614	4.84061921440679e-05	\\
7004.16370738636	4.94751160088441e-05	\\
7005.14248934659	4.8906891444412e-05	\\
7006.12127130682	5.04946564721888e-05	\\
7007.10005326705	4.84656336564299e-05	\\
7008.07883522727	4.93198765238483e-05	\\
7009.0576171875	5.09069750794922e-05	\\
7010.03639914773	4.90177152700843e-05	\\
7011.01518110795	5.0963098719834e-05	\\
7011.99396306818	4.94798808051692e-05	\\
7012.97274502841	5.12708209294319e-05	\\
7013.95152698864	4.99449541664025e-05	\\
7014.93030894886	4.93517302588676e-05	\\
7015.90909090909	4.88853172295921e-05	\\
7016.88787286932	4.92013462473341e-05	\\
7017.86665482955	5.11850870218131e-05	\\
7018.84543678977	5.00944577612681e-05	\\
7019.82421875	4.9462784912416e-05	\\
7020.80300071023	5.05662257707128e-05	\\
7021.78178267045	4.70515751911686e-05	\\
7022.76056463068	4.99061738208626e-05	\\
7023.73934659091	5.07030683745692e-05	\\
7024.71812855114	5.10729381333851e-05	\\
7025.69691051136	5.11207462370019e-05	\\
7026.67569247159	5.05421653737066e-05	\\
7027.65447443182	4.97145047073741e-05	\\
7028.63325639205	4.86941428743663e-05	\\
7029.61203835227	4.96508949476741e-05	\\
7030.5908203125	4.99305532537688e-05	\\
7031.56960227273	4.84104876524656e-05	\\
7032.54838423295	4.90890155213967e-05	\\
7033.52716619318	5.13805495887015e-05	\\
7034.50594815341	4.77998925148656e-05	\\
7035.48473011364	5.09091280129376e-05	\\
7036.46351207386	4.92906166971319e-05	\\
7037.44229403409	5.17172040159538e-05	\\
7038.42107599432	4.98242967624048e-05	\\
7039.39985795455	4.97994689856178e-05	\\
7040.37863991477	4.72646251175172e-05	\\
7041.357421875	4.9863909741618e-05	\\
7042.33620383523	4.87492388128118e-05	\\
7043.31498579545	5.04619880782707e-05	\\
7044.29376775568	4.96472810582893e-05	\\
7045.27254971591	4.91086367156414e-05	\\
7046.25133167614	4.91288270871046e-05	\\
7047.23011363636	4.95453998561517e-05	\\
7048.20889559659	5.01608332422924e-05	\\
7049.18767755682	5.12461589098138e-05	\\
7050.16645951705	5.24868374796103e-05	\\
7051.14524147727	5.09570100461389e-05	\\
7052.1240234375	5.05034136573175e-05	\\
7053.10280539773	4.9673926934855e-05	\\
7054.08158735795	5.30874425163173e-05	\\
7055.06036931818	5.06726868894451e-05	\\
7056.03915127841	5.2451730973134e-05	\\
7057.01793323864	5.17397370023653e-05	\\
7057.99671519886	5.16081655206759e-05	\\
7058.97549715909	4.97490049251133e-05	\\
7059.95427911932	5.19188707003198e-05	\\
7060.93306107955	5.21453155465276e-05	\\
7061.91184303977	5.15905697144882e-05	\\
7062.890625	5.12762928330876e-05	\\
7063.86940696023	4.97596951192684e-05	\\
7064.84818892045	5.12630456369113e-05	\\
7065.82697088068	5.07546762689567e-05	\\
7066.80575284091	5.13791824700696e-05	\\
7067.78453480114	5.15172819849035e-05	\\
7068.76331676136	5.10319160981777e-05	\\
7069.74209872159	5.10901718096851e-05	\\
7070.72088068182	5.2652357414015e-05	\\
7071.69966264205	5.03185492919239e-05	\\
7072.67844460227	5.19015626315205e-05	\\
7073.6572265625	5.1823227601051e-05	\\
7074.63600852273	5.04173115104926e-05	\\
7075.61479048295	5.16339789861467e-05	\\
7076.59357244318	5.12909438296626e-05	\\
7077.57235440341	5.16216571772449e-05	\\
7078.55113636364	5.05732623275892e-05	\\
7079.52991832386	4.96241035388588e-05	\\
7080.50870028409	5.06854367797028e-05	\\
7081.48748224432	5.18780743475621e-05	\\
7082.46626420455	4.9748224432681e-05	\\
7083.44504616477	5.19189380467988e-05	\\
7084.423828125	5.18454790859805e-05	\\
7085.40261008523	5.07510727816474e-05	\\
7086.38139204545	5.14286102788479e-05	\\
7087.36017400568	5.21467851772949e-05	\\
7088.33895596591	4.90801868170889e-05	\\
7089.31773792614	4.91989687271483e-05	\\
7090.29651988636	4.97547450828537e-05	\\
7091.27530184659	4.9911882156916e-05	\\
7092.25408380682	4.7613995072483e-05	\\
7093.23286576705	4.98496464628285e-05	\\
7094.21164772727	4.94983100431407e-05	\\
7095.1904296875	5.04403567020773e-05	\\
7096.16921164773	4.9276325336691e-05	\\
7097.14799360795	4.88909262531996e-05	\\
7098.12677556818	4.98338021136656e-05	\\
7099.10555752841	5.0931959960386e-05	\\
7100.08433948864	5.1871309264397e-05	\\
7101.06312144886	4.94702492993626e-05	\\
7102.04190340909	4.93835133331126e-05	\\
7103.02068536932	4.92190319579997e-05	\\
7103.99946732955	5.09092779109952e-05	\\
7104.97824928977	5.10228279730826e-05	\\
7105.95703125	4.87089231005557e-05	\\
7106.93581321023	4.90458351251634e-05	\\
7107.91459517045	5.00946202265651e-05	\\
7108.89337713068	4.93069364782139e-05	\\
7109.87215909091	5.0471658864821e-05	\\
7110.85094105114	4.96600952964797e-05	\\
7111.82972301136	5.0376278453859e-05	\\
7112.80850497159	5.0099124017603e-05	\\
7113.78728693182	5.06222244625583e-05	\\
7114.76606889205	4.83268402651287e-05	\\
7115.74485085227	5.04674407210513e-05	\\
7116.7236328125	4.99372004473201e-05	\\
7117.70241477273	4.98195492183154e-05	\\
7118.68119673295	4.80569463217148e-05	\\
7119.65997869318	4.85467770317956e-05	\\
7120.63876065341	4.87214938961726e-05	\\
7121.61754261364	5.21960385814032e-05	\\
7122.59632457386	4.8915371808462e-05	\\
7123.57510653409	4.86877344788652e-05	\\
7124.55388849432	4.98402210830697e-05	\\
7125.53267045455	5.11455606354688e-05	\\
7126.51145241477	5.09838837201445e-05	\\
7127.490234375	5.02173179042614e-05	\\
7128.46901633523	4.95264633800991e-05	\\
7129.44779829545	4.90477449845582e-05	\\
7130.42658025568	4.89705010528095e-05	\\
7131.40536221591	5.10403009887699e-05	\\
7132.38414417614	5.09198508871037e-05	\\
7133.36292613636	5.19490814794924e-05	\\
7134.34170809659	5.03135924766309e-05	\\
7135.32049005682	4.93851490842493e-05	\\
7136.29927201705	4.94703760243966e-05	\\
7137.27805397727	5.10360924219013e-05	\\
7138.2568359375	4.85816400877641e-05	\\
7139.23561789773	5.02039199106817e-05	\\
7140.21439985795	4.96251607940282e-05	\\
7141.19318181818	5.11031033897184e-05	\\
7142.17196377841	4.9181233617708e-05	\\
7143.15074573864	4.92911406779112e-05	\\
7144.12952769886	5.14602255910012e-05	\\
7145.10830965909	5.12584305995587e-05	\\
7146.08709161932	5.10359123453926e-05	\\
7147.06587357955	5.07220167993386e-05	\\
7148.04465553977	5.0934727132164e-05	\\
7149.0234375	5.08849386598348e-05	\\
7150.00221946023	5.07188532124191e-05	\\
7150.98100142045	5.02106098598158e-05	\\
7151.95978338068	5.08371474932421e-05	\\
7152.93856534091	5.29177693483477e-05	\\
7153.91734730114	5.17459475641268e-05	\\
7154.89612926136	5.04345582028977e-05	\\
7155.87491122159	5.15168484084138e-05	\\
7156.85369318182	5.18533188674984e-05	\\
7157.83247514205	5.14378068459133e-05	\\
7158.81125710227	5.02524723006175e-05	\\
7159.7900390625	5.20652318670814e-05	\\
7160.76882102273	5.16498883144573e-05	\\
7161.74760298295	5.08333215501067e-05	\\
7162.72638494318	5.25251675402102e-05	\\
7163.70516690341	5.06277256904533e-05	\\
7164.68394886364	5.38481145588931e-05	\\
7165.66273082386	5.22101246427715e-05	\\
7166.64151278409	5.00642732293695e-05	\\
7167.62029474432	5.08421799218827e-05	\\
7168.59907670455	5.22874427345231e-05	\\
7169.57785866477	5.17192600262202e-05	\\
7170.556640625	5.22373004199061e-05	\\
7171.53542258523	5.10754667647157e-05	\\
7172.51420454545	5.04130012363281e-05	\\
7173.49298650568	5.21420522182443e-05	\\
7174.47176846591	5.17606667625884e-05	\\
7175.45055042614	5.01852060903088e-05	\\
7176.42933238636	5.06045653759113e-05	\\
7177.40811434659	5.19888089716615e-05	\\
7178.38689630682	5.15944428342296e-05	\\
7179.36567826705	5.07365058897778e-05	\\
7180.34446022727	5.14760906376447e-05	\\
7181.3232421875	5.20086278374418e-05	\\
7182.30202414773	5.21092836184942e-05	\\
7183.28080610795	5.1350097143422e-05	\\
7184.25958806818	5.06115977191539e-05	\\
7185.23837002841	4.98292673159338e-05	\\
7186.21715198864	5.18757646187874e-05	\\
7187.19593394886	5.17908032857995e-05	\\
7188.17471590909	5.1039672830596e-05	\\
7189.15349786932	5.29840887502628e-05	\\
7190.13227982955	5.18898496409426e-05	\\
7191.11106178977	5.11427611682847e-05	\\
7192.08984375	5.15259420454453e-05	\\
7193.06862571023	5.1150452712349e-05	\\
7194.04740767045	5.08524138766958e-05	\\
7195.02618963068	5.00281762139664e-05	\\
7196.00497159091	5.05101410433743e-05	\\
7196.98375355114	5.18342648645e-05	\\
7197.96253551136	5.16548713927735e-05	\\
7198.94131747159	5.12732017835073e-05	\\
7199.92009943182	5.11829810633415e-05	\\
7200.89888139205	5.28007822861084e-05	\\
7201.87766335227	5.24002752653479e-05	\\
7202.8564453125	5.17343453682501e-05	\\
7203.83522727273	5.11371562209561e-05	\\
7204.81400923295	5.4235947477839e-05	\\
7205.79279119318	5.20755641764498e-05	\\
7206.77157315341	5.16798169347196e-05	\\
7207.75035511364	5.10298606316354e-05	\\
7208.72913707386	5.09858211030854e-05	\\
7209.70791903409	5.00135607905717e-05	\\
7210.68670099432	5.20682733909512e-05	\\
7211.66548295455	5.02855069674403e-05	\\
7212.64426491477	5.24869084262964e-05	\\
7213.623046875	5.26902282098739e-05	\\
7214.60182883523	5.26503424702355e-05	\\
7215.58061079545	5.09733125164024e-05	\\
7216.55939275568	5.33143351004426e-05	\\
7217.53817471591	5.25355707418236e-05	\\
7218.51695667614	5.28364668332895e-05	\\
7219.49573863636	5.46587915844237e-05	\\
7220.47452059659	5.2022553891515e-05	\\
7221.45330255682	5.2466755951955e-05	\\
7222.43208451705	5.2985341606782e-05	\\
7223.41086647727	5.19288518293304e-05	\\
7224.3896484375	5.22506816569222e-05	\\
7225.36843039773	5.31421394151046e-05	\\
7226.34721235795	5.41406197088078e-05	\\
7227.32599431818	5.31968329129976e-05	\\
7228.30477627841	5.40568699943253e-05	\\
7229.28355823864	5.45761801638688e-05	\\
7230.26234019886	5.30428616073071e-05	\\
7231.24112215909	5.28442135259616e-05	\\
7232.21990411932	5.44230775113938e-05	\\
7233.19868607955	5.37128741693207e-05	\\
7234.17746803977	5.33925046023305e-05	\\
7235.15625	5.35698077301759e-05	\\
7236.13503196023	5.27225484300682e-05	\\
7237.11381392045	5.41978674611463e-05	\\
7238.09259588068	5.40935535515678e-05	\\
7239.07137784091	5.32424207533571e-05	\\
7240.05015980114	5.29315045598462e-05	\\
7241.02894176136	5.22638494937848e-05	\\
7242.00772372159	5.32259484683037e-05	\\
7242.98650568182	5.2243322474104e-05	\\
7243.96528764205	5.39484506121869e-05	\\
7244.94406960227	4.98369492567045e-05	\\
7245.9228515625	5.23894469170547e-05	\\
7246.90163352273	5.48161100159306e-05	\\
7247.88041548295	5.1495938884866e-05	\\
7248.85919744318	5.48165119041606e-05	\\
7249.83797940341	5.80861351587496e-05	\\
7250.81676136364	5.30220117592699e-05	\\
7251.79554332386	5.39863821494822e-05	\\
7252.77432528409	5.24785436458938e-05	\\
7253.75310724432	5.42732159104192e-05	\\
7254.73188920455	5.45209209590676e-05	\\
7255.71067116477	5.2848242001825e-05	\\
7256.689453125	5.25800443679015e-05	\\
7257.66823508523	5.4133677566958e-05	\\
7258.64701704545	5.39505761823093e-05	\\
7259.62579900568	5.22367645355119e-05	\\
7260.60458096591	5.35633057694594e-05	\\
7261.58336292614	5.35260300623073e-05	\\
7262.56214488636	5.20391294306333e-05	\\
7263.54092684659	5.41182051915903e-05	\\
7264.51970880682	5.41511256218913e-05	\\
7265.49849076705	5.22840260019979e-05	\\
7266.47727272727	5.36792966839288e-05	\\
7267.4560546875	5.52040842745389e-05	\\
7268.43483664773	5.24953908451885e-05	\\
7269.41361860795	5.47838273621169e-05	\\
7270.39240056818	5.28293623465779e-05	\\
7271.37118252841	5.29281807598447e-05	\\
7272.34996448864	5.4649017045884e-05	\\
7273.32874644886	5.36140185066427e-05	\\
7274.30752840909	5.15111146072457e-05	\\
7275.28631036932	5.20949865038514e-05	\\
7276.26509232955	5.22264388930807e-05	\\
7277.24387428977	5.27328881989137e-05	\\
7278.22265625	5.29254238392642e-05	\\
7279.20143821023	5.31576713180841e-05	\\
7280.18022017045	5.11710280698371e-05	\\
7281.15900213068	5.11653980845977e-05	\\
7282.13778409091	5.16444502728397e-05	\\
7283.11656605114	5.24370779480023e-05	\\
7284.09534801136	5.20131832006261e-05	\\
7285.07412997159	5.27456390189294e-05	\\
7286.05291193182	5.08499764789371e-05	\\
7287.03169389205	5.26500150151687e-05	\\
7288.01047585227	5.3107680281011e-05	\\
7288.9892578125	5.28321645271882e-05	\\
7289.96803977273	5.19993700314751e-05	\\
7290.94682173295	5.12957789623352e-05	\\
7291.92560369318	5.04963551813254e-05	\\
7292.90438565341	5.16920173850422e-05	\\
7293.88316761364	4.95226243548403e-05	\\
7294.86194957386	5.07595941931658e-05	\\
7295.84073153409	5.15245910849414e-05	\\
7296.81951349432	5.07077135414671e-05	\\
7297.79829545455	5.06267755514765e-05	\\
7298.77707741477	5.14595133983993e-05	\\
7299.755859375	5.34533920225443e-05	\\
7300.73464133523	5.18553126562554e-05	\\
7301.71342329545	5.0636719797977e-05	\\
7302.69220525568	5.12000015204454e-05	\\
7303.67098721591	5.05535165040263e-05	\\
7304.64976917614	5.11165706104373e-05	\\
7305.62855113636	4.9492578088391e-05	\\
7306.60733309659	5.06030229490557e-05	\\
7307.58611505682	4.88396253451729e-05	\\
7308.56489701705	4.89684791765616e-05	\\
7309.54367897727	4.8916659020368e-05	\\
7310.5224609375	4.94363495763803e-05	\\
7311.50124289773	4.9234655169276e-05	\\
7312.48002485795	4.93427021735674e-05	\\
7313.45880681818	4.90504469025224e-05	\\
7314.43758877841	4.9667880022e-05	\\
7315.41637073864	4.96840578217012e-05	\\
7316.39515269886	4.79116253840607e-05	\\
7317.37393465909	5.01035007050136e-05	\\
7318.35271661932	4.97714376090646e-05	\\
7319.33149857955	5.00384204427952e-05	\\
7320.31028053977	5.03212196446345e-05	\\
7321.2890625	4.92720324617532e-05	\\
7322.26784446023	4.86964166698076e-05	\\
7323.24662642045	4.76507518361777e-05	\\
7324.22540838068	4.83197502526925e-05	\\
7325.20419034091	4.7395113469941e-05	\\
7326.18297230114	4.90003666130489e-05	\\
7327.16175426136	4.90643969078182e-05	\\
7328.14053622159	4.79683545477075e-05	\\
7329.11931818182	4.73264058261953e-05	\\
7330.09810014205	4.9088024436894e-05	\\
7331.07688210227	4.65387377961048e-05	\\
7332.0556640625	4.86502603643063e-05	\\
7333.03444602273	4.90737090803554e-05	\\
7334.01322798295	4.84052571013605e-05	\\
7334.99200994318	4.62156241887825e-05	\\
7335.97079190341	4.74352528335253e-05	\\
7336.94957386364	4.82358656562555e-05	\\
7337.92835582386	4.69463759164207e-05	\\
7338.90713778409	4.75155988962342e-05	\\
7339.88591974432	4.94821433800589e-05	\\
7340.86470170455	4.80649897812432e-05	\\
7341.84348366477	4.92446084474139e-05	\\
7342.822265625	4.57270163514252e-05	\\
7343.80104758523	4.67430306126684e-05	\\
7344.77982954545	4.80151315843898e-05	\\
7345.75861150568	4.77396827828751e-05	\\
7346.73739346591	4.91600898650381e-05	\\
7347.71617542614	4.69047611854478e-05	\\
7348.69495738636	4.83823816010261e-05	\\
7349.67373934659	4.65028500347447e-05	\\
7350.65252130682	4.67080847055009e-05	\\
7351.63130326705	4.82025962277434e-05	\\
7352.61008522727	4.69829194062803e-05	\\
7353.5888671875	5.03153258937737e-05	\\
7354.56764914773	4.522597450552e-05	\\
7355.54643110795	4.66227551188275e-05	\\
7356.52521306818	4.83948751045211e-05	\\
7357.50399502841	4.76982030767481e-05	\\
7358.48277698864	4.43106686629194e-05	\\
7359.46155894886	4.63564343924773e-05	\\
7360.44034090909	4.71543774933524e-05	\\
7361.41912286932	4.59509681388481e-05	\\
7362.39790482955	4.66663070113036e-05	\\
7363.37668678977	4.73169123606242e-05	\\
7364.35546875	4.59999455936889e-05	\\
7365.33425071023	4.54369881234081e-05	\\
7366.31303267045	4.63198129852471e-05	\\
7367.29181463068	4.61593543861118e-05	\\
7368.27059659091	4.56813169585762e-05	\\
7369.24937855114	4.60171312337893e-05	\\
7370.22816051136	4.56732171557231e-05	\\
7371.20694247159	4.36034780003209e-05	\\
7372.18572443182	4.49256068684405e-05	\\
7373.16450639205	4.4776064598345e-05	\\
7374.14328835227	4.58122610240969e-05	\\
7375.1220703125	4.52704862443782e-05	\\
7376.10085227273	4.68104976165932e-05	\\
7377.07963423295	4.49789169061198e-05	\\
7378.05841619318	4.40404427353197e-05	\\
7379.03719815341	4.44607435016759e-05	\\
7380.01598011364	4.30886105987051e-05	\\
7380.99476207386	4.53446029220775e-05	\\
7381.97354403409	4.48565558047589e-05	\\
7382.95232599432	4.30831659273398e-05	\\
7383.93110795455	4.33079877832428e-05	\\
7384.90988991477	4.43188913692706e-05	\\
7385.888671875	4.26485845378053e-05	\\
7386.86745383523	4.44375736648905e-05	\\
7387.84623579545	4.49116458997694e-05	\\
7388.82501775568	4.19713120434854e-05	\\
7389.80379971591	4.32975797360377e-05	\\
7390.78258167614	4.35254232789245e-05	\\
7391.76136363636	4.40458356482106e-05	\\
7392.74014559659	4.26016250105816e-05	\\
7393.71892755682	4.26559147220825e-05	\\
7394.69770951705	4.28167414339672e-05	\\
7395.67649147727	4.20977205839291e-05	\\
7396.6552734375	4.21180202082902e-05	\\
7397.63405539773	4.28961355938758e-05	\\
7398.61283735795	4.36427332139401e-05	\\
7399.59161931818	4.3585870629671e-05	\\
7400.57040127841	4.18383846604712e-05	\\
7401.54918323864	4.16378391816814e-05	\\
7402.52796519886	4.25637938924228e-05	\\
7403.50674715909	4.2969973600121e-05	\\
7404.48552911932	4.323187299927e-05	\\
7405.46431107955	4.18629180052119e-05	\\
7406.44309303977	4.26241753518906e-05	\\
7407.421875	4.14184223079102e-05	\\
7408.40065696023	4.24623225292615e-05	\\
7409.37943892045	4.10897645684157e-05	\\
7410.35822088068	4.14693295226285e-05	\\
7411.33700284091	4.28023147324623e-05	\\
7412.31578480114	4.09211571231465e-05	\\
7413.29456676136	4.06602661625558e-05	\\
7414.27334872159	4.06627952786345e-05	\\
7415.25213068182	4.10106515103266e-05	\\
7416.23091264205	4.07012692014869e-05	\\
7417.20969460227	4.24137014505593e-05	\\
7418.1884765625	4.18268098931872e-05	\\
7419.16725852273	4.2487402015561e-05	\\
7420.14604048295	4.05949110122497e-05	\\
7421.12482244318	3.88464357475526e-05	\\
7422.10360440341	4.11703280837559e-05	\\
7423.08238636364	4.09190634204867e-05	\\
7424.06116832386	4.1386123499272e-05	\\
7425.03995028409	3.96368062715773e-05	\\
7426.01873224432	4.15626591486912e-05	\\
7426.99751420455	4.08666461514263e-05	\\
7427.97629616477	4.08679105865647e-05	\\
7428.955078125	4.05095008377205e-05	\\
7429.93386008523	3.91498989192515e-05	\\
7430.91264204545	3.99228192833459e-05	\\
7431.89142400568	3.99300016464547e-05	\\
7432.87020596591	4.14212806103689e-05	\\
7433.84898792614	4.15348511734385e-05	\\
7434.82776988636	4.0528267710171e-05	\\
7435.80655184659	4.14455795475683e-05	\\
7436.78533380682	4.03696827801323e-05	\\
7437.76411576705	4.06928739719263e-05	\\
7438.74289772727	4.10785595339894e-05	\\
7439.7216796875	3.98967899481378e-05	\\
7440.70046164773	3.98542809775814e-05	\\
7441.67924360795	3.90046111213097e-05	\\
7442.65802556818	4.04925644047796e-05	\\
7443.63680752841	3.99927892164695e-05	\\
7444.61558948864	4.0165550580428e-05	\\
7445.59437144886	4.09408117378964e-05	\\
7446.57315340909	4.06205045042765e-05	\\
7447.55193536932	4.12547898601894e-05	\\
7448.53071732955	4.10698595974231e-05	\\
7449.50949928977	3.90042712086052e-05	\\
7450.48828125	4.14810032224247e-05	\\
7451.46706321023	4.02418117270393e-05	\\
7452.44584517045	3.91367632412194e-05	\\
7453.42462713068	4.10790462568107e-05	\\
7454.40340909091	4.11758155327609e-05	\\
7455.38219105114	4.08821752985309e-05	\\
7456.36097301136	4.06206148604431e-05	\\
7457.33975497159	4.02099769008811e-05	\\
7458.31853693182	4.10797132915816e-05	\\
7459.29731889205	3.94021582182297e-05	\\
7460.27610085227	3.83473984630599e-05	\\
7461.2548828125	3.9927529842476e-05	\\
7462.23366477273	4.10489907720647e-05	\\
7463.21244673295	4.25425759408604e-05	\\
7464.19122869318	4.1756351836566e-05	\\
7465.17001065341	4.09032500253457e-05	\\
7466.14879261364	4.13329698083121e-05	\\
7467.12757457386	3.95347543781955e-05	\\
7468.10635653409	4.08608833311758e-05	\\
7469.08513849432	4.1322528923473e-05	\\
7470.06392045455	3.9535582798001e-05	\\
7471.04270241477	4.11845486202494e-05	\\
7472.021484375	3.8553398209478e-05	\\
7473.00026633523	4.11020898245364e-05	\\
7473.97904829545	4.00372641399358e-05	\\
7474.95783025568	3.99385799478643e-05	\\
7475.93661221591	3.8986523842354e-05	\\
7476.91539417614	4.05010421822428e-05	\\
7477.89417613636	4.00523165298444e-05	\\
7478.87295809659	3.88169675580988e-05	\\
7479.85174005682	4.07930870763729e-05	\\
7480.83052201705	4.0769010371521e-05	\\
7481.80930397727	4.12179059873949e-05	\\
7482.7880859375	4.13748853261989e-05	\\
7483.76686789773	4.11919931528052e-05	\\
7484.74564985795	4.05874767621024e-05	\\
7485.72443181818	4.06406959014064e-05	\\
7486.70321377841	4.03139240490741e-05	\\
7487.68199573864	4.06470161508542e-05	\\
7488.66077769886	4.0089302278055e-05	\\
7489.63955965909	4.18854934067069e-05	\\
7490.61834161932	3.90470648140354e-05	\\
7491.59712357955	3.76098699019476e-05	\\
7492.57590553977	4.01659744034216e-05	\\
7493.5546875	3.87160909010492e-05	\\
7494.53346946023	4.00628459811042e-05	\\
7495.51225142045	3.9506792415359e-05	\\
7496.49103338068	3.74135015372419e-05	\\
7497.46981534091	3.79221140849423e-05	\\
7498.44859730114	3.93962466751657e-05	\\
7499.42737926136	3.99691858237709e-05	\\
7500.40616122159	3.88519328524149e-05	\\
7501.38494318182	3.80959447135489e-05	\\
7502.36372514205	3.74530410342682e-05	\\
7503.34250710227	3.83317936970257e-05	\\
7504.3212890625	3.70344759891272e-05	\\
7505.30007102273	3.75751541264571e-05	\\
7506.27885298295	3.74555471103642e-05	\\
7507.25763494318	3.53553844848923e-05	\\
7508.23641690341	3.65532665039516e-05	\\
7509.21519886364	3.66095551931409e-05	\\
7510.19398082386	3.79051763385919e-05	\\
7511.17276278409	3.56233907486176e-05	\\
7512.15154474432	3.59634056166011e-05	\\
7513.13032670455	3.65879641281786e-05	\\
7514.10910866477	3.62188698160603e-05	\\
7515.087890625	3.47163990915678e-05	\\
7516.06667258523	3.61045359840885e-05	\\
7517.04545454545	3.65074444568385e-05	\\
7518.02423650568	3.57025724189511e-05	\\
7519.00301846591	3.70877910431902e-05	\\
7519.98180042614	3.58080187971577e-05	\\
7520.96058238636	3.40434203594417e-05	\\
7521.93936434659	3.56943225900743e-05	\\
7522.91814630682	3.59113665970188e-05	\\
7523.89692826705	3.53418573832524e-05	\\
7524.87571022727	3.5087988751043e-05	\\
7525.8544921875	3.60380044689649e-05	\\
7526.83327414773	3.39852115389937e-05	\\
7527.81205610795	3.52655244429828e-05	\\
7528.79083806818	3.4710969591557e-05	\\
7529.76962002841	3.36438848054842e-05	\\
7530.74840198864	3.41499480024821e-05	\\
7531.72718394886	3.47239177755292e-05	\\
7532.70596590909	3.56573715020021e-05	\\
7533.68474786932	3.38108460941657e-05	\\
7534.66352982955	3.2905781531791e-05	\\
7535.64231178977	3.417620264224e-05	\\
7536.62109375	3.29205004953883e-05	\\
7537.59987571023	3.3987346387115e-05	\\
7538.57865767045	3.39537508864839e-05	\\
7539.55743963068	3.35907960120868e-05	\\
7540.53622159091	3.3948657202388e-05	\\
7541.51500355114	3.44991207390564e-05	\\
7542.49378551136	3.40350841905137e-05	\\
7543.47256747159	3.48549714510561e-05	\\
7544.45134943182	3.32335224306737e-05	\\
7545.43013139205	3.42247397563934e-05	\\
7546.40891335227	3.32576454130051e-05	\\
7547.3876953125	3.38812736426225e-05	\\
7548.36647727273	3.21160347529568e-05	\\
7549.34525923295	3.17935135297166e-05	\\
7550.32404119318	3.42051650687428e-05	\\
7551.30282315341	3.3171581755623e-05	\\
7552.28160511364	3.39321061748175e-05	\\
7553.26038707386	3.34652184558285e-05	\\
7554.23916903409	3.27190881389211e-05	\\
7555.21795099432	3.20198479323174e-05	\\
7556.19673295455	3.30715011454927e-05	\\
7557.17551491477	3.29248911123942e-05	\\
7558.154296875	3.41404680324138e-05	\\
7559.13307883523	3.16035599532905e-05	\\
7560.11186079545	3.30918301723099e-05	\\
7561.09064275568	3.42671941610629e-05	\\
7562.06942471591	3.17402724337981e-05	\\
7563.04820667614	3.15532123599282e-05	\\
7564.02698863636	3.30342752230517e-05	\\
7565.00577059659	3.40215940397924e-05	\\
7565.98455255682	3.35149446515312e-05	\\
7566.96333451705	3.406773669841e-05	\\
7567.94211647727	3.44446361166074e-05	\\
7568.9208984375	3.29528453417229e-05	\\
7569.89968039773	3.22041271758989e-05	\\
7570.87846235795	3.41729586894245e-05	\\
7571.85724431818	3.30862456848652e-05	\\
7572.83602627841	3.24769847574697e-05	\\
7573.81480823864	3.30783422594636e-05	\\
7574.79359019886	3.39544798820162e-05	\\
7575.77237215909	3.39579856775179e-05	\\
7576.75115411932	3.43741364058961e-05	\\
7577.72993607955	3.404114358737e-05	\\
7578.70871803977	3.38631108151069e-05	\\
7579.6875	3.35333487782676e-05	\\
7580.66628196023	3.35222788300895e-05	\\
7581.64506392045	3.45313296776668e-05	\\
7582.62384588068	3.47404767047597e-05	\\
7583.60262784091	3.45064736433845e-05	\\
7584.58140980114	3.22558086054593e-05	\\
7585.56019176136	3.32357111927436e-05	\\
7586.53897372159	3.37718268927027e-05	\\
7587.51775568182	3.18733349731796e-05	\\
7588.49653764205	3.25402901619406e-05	\\
7589.47531960227	3.16200716288188e-05	\\
7590.4541015625	3.2477269931372e-05	\\
7591.43288352273	3.24937846442209e-05	\\
7592.41166548295	3.30445942710878e-05	\\
7593.39044744318	3.29716462619019e-05	\\
7594.36922940341	3.29882845575937e-05	\\
7595.34801136364	3.11271834837871e-05	\\
7596.32679332386	3.08052560382148e-05	\\
7597.30557528409	3.29925940749861e-05	\\
7598.28435724432	3.20819527443726e-05	\\
7599.26313920455	3.06557633313845e-05	\\
7600.24192116477	3.29196521857109e-05	\\
7601.220703125	3.14917994608216e-05	\\
7602.19948508523	2.96629406181561e-05	\\
7603.17826704545	3.08136597981848e-05	\\
7604.15704900568	3.00593036708151e-05	\\
7605.13583096591	3.07862299630792e-05	\\
7606.11461292614	3.0097123629469e-05	\\
7607.09339488636	3.05273280398631e-05	\\
7608.07217684659	3.00698168691889e-05	\\
7609.05095880682	3.03700043043171e-05	\\
7610.02974076705	3.11594135292944e-05	\\
7611.00852272727	2.8843342011098e-05	\\
7611.9873046875	2.94098805852693e-05	\\
7612.96608664773	2.95471826738534e-05	\\
7613.94486860795	2.78468799630309e-05	\\
7614.92365056818	3.11215678998297e-05	\\
7615.90243252841	2.9931180644291e-05	\\
7616.88121448864	2.87220064126334e-05	\\
7617.85999644886	3.09182918815768e-05	\\
7618.83877840909	2.93453352598471e-05	\\
7619.81756036932	2.66806475297781e-05	\\
7620.79634232955	2.99582582162792e-05	\\
7621.77512428977	2.99036776030863e-05	\\
7622.75390625	2.7873846505741e-05	\\
7623.73268821023	2.74064445138699e-05	\\
7624.71147017045	2.81762959084739e-05	\\
7625.69025213068	2.69337812189181e-05	\\
7626.66903409091	2.93189596097681e-05	\\
7627.64781605114	2.75809964178506e-05	\\
7628.62659801136	2.62285063729066e-05	\\
7629.60537997159	2.73414516223547e-05	\\
7630.58416193182	2.85277787070641e-05	\\
7631.56294389205	2.62057084022157e-05	\\
7632.54172585227	2.74675289290309e-05	\\
7633.5205078125	2.73935197888968e-05	\\
7634.49928977273	2.64820702998365e-05	\\
7635.47807173295	2.78978434756992e-05	\\
7636.45685369318	2.82939722989928e-05	\\
7637.43563565341	2.6291181233934e-05	\\
7638.41441761364	2.60962067312661e-05	\\
7639.39319957386	2.64956793065216e-05	\\
7640.37198153409	2.71656502617055e-05	\\
7641.35076349432	2.57303235057189e-05	\\
7642.32954545455	2.46159988999712e-05	\\
7643.30832741477	2.62008392548853e-05	\\
7644.287109375	2.69047501143153e-05	\\
7645.26589133523	2.61822379464455e-05	\\
7646.24467329545	2.57800781200075e-05	\\
7647.22345525568	2.49852797547465e-05	\\
7648.20223721591	2.47935268920139e-05	\\
7649.18101917614	2.5297673511635e-05	\\
7650.15980113636	2.4729529919025e-05	\\
7651.13858309659	2.58135844372403e-05	\\
7652.11736505682	2.61169698903726e-05	\\
7653.09614701705	2.62797383405824e-05	\\
7654.07492897727	2.56725460394709e-05	\\
7655.0537109375	2.56236129219277e-05	\\
7656.03249289773	2.49736327872381e-05	\\
7657.01127485795	2.39752221912055e-05	\\
7657.99005681818	2.42993702171783e-05	\\
7658.96883877841	2.48033477087905e-05	\\
7659.94762073864	2.50978434482708e-05	\\
7660.92640269886	2.4881095643524e-05	\\
7661.90518465909	2.4296388009127e-05	\\
7662.88396661932	2.49102474824773e-05	\\
7663.86274857955	2.5736508322107e-05	\\
7664.84153053977	2.48573502673532e-05	\\
7665.8203125	2.4588971997558e-05	\\
7666.79909446023	2.46443476362029e-05	\\
7667.77787642045	2.39730058188563e-05	\\
7668.75665838068	2.50254458444438e-05	\\
7669.73544034091	2.62327511527337e-05	\\
7670.71422230114	2.59016878631886e-05	\\
7671.69300426136	2.56686399103108e-05	\\
7672.67178622159	2.41681812613186e-05	\\
7673.65056818182	2.46902710611055e-05	\\
7674.62935014205	2.54461613743723e-05	\\
7675.60813210227	2.57848960571491e-05	\\
7676.5869140625	2.44741501387521e-05	\\
7677.56569602273	2.42202154236912e-05	\\
7678.54447798295	2.53624237080673e-05	\\
7679.52325994318	2.43000454565057e-05	\\
7680.50204190341	2.43383250164093e-05	\\
7681.48082386364	2.67708479601962e-05	\\
7682.45960582386	2.56006939021758e-05	\\
7683.43838778409	2.56623959202776e-05	\\
7684.41716974432	2.3547696229491e-05	\\
7685.39595170455	2.38902755786296e-05	\\
7686.37473366477	2.54554367553432e-05	\\
7687.353515625	2.48219284634378e-05	\\
7688.33229758523	2.52001563968108e-05	\\
7689.31107954545	2.53696507521551e-05	\\
7690.28986150568	2.34081016811602e-05	\\
7691.26864346591	2.35385233472885e-05	\\
7692.24742542614	2.43716028998321e-05	\\
7693.22620738636	2.33935111757036e-05	\\
7694.20498934659	2.37632411209035e-05	\\
7695.18377130682	2.30751851546075e-05	\\
7696.16255326705	2.45845946839738e-05	\\
7697.14133522727	2.35070405200153e-05	\\
7698.1201171875	2.27045352568942e-05	\\
7699.09889914773	2.25845290425303e-05	\\
7700.07768110795	2.41873690619958e-05	\\
7701.05646306818	2.39932413562375e-05	\\
7702.03524502841	2.15066103268865e-05	\\
7703.01402698864	2.26181097817845e-05	\\
7703.99280894886	2.34087728132643e-05	\\
7704.97159090909	2.43107618259022e-05	\\
7705.95037286932	2.28262351067961e-05	\\
7706.92915482955	2.33774473231597e-05	\\
7707.90793678977	2.30519222048402e-05	\\
7708.88671875	2.43623563085431e-05	\\
7709.86550071023	2.22817423707769e-05	\\
7710.84428267045	2.26447819916791e-05	\\
7711.82306463068	2.39162663492423e-05	\\
7712.80184659091	2.33327334823744e-05	\\
7713.78062855114	2.22018417830876e-05	\\
7714.75941051136	2.35252416821819e-05	\\
7715.73819247159	2.29091807212025e-05	\\
7716.71697443182	2.27739901689093e-05	\\
7717.69575639205	2.36139798031779e-05	\\
7718.67453835227	2.15922371932265e-05	\\
7719.6533203125	2.13867079518602e-05	\\
7720.63210227273	2.27737310730897e-05	\\
7721.61088423295	2.27982175703092e-05	\\
7722.58966619318	2.19149540166045e-05	\\
7723.56844815341	2.20043455752241e-05	\\
7724.54723011364	2.2144431291271e-05	\\
7725.52601207386	2.07724527701714e-05	\\
7726.50479403409	2.20382950303623e-05	\\
7727.48357599432	2.22166990208782e-05	\\
7728.46235795455	2.18773933487603e-05	\\
7729.44113991477	2.20508385410041e-05	\\
7730.419921875	2.10903053349111e-05	\\
7731.39870383523	2.03111646418106e-05	\\
7732.37748579545	2.11291784376357e-05	\\
7733.35626775568	2.18501063125822e-05	\\
7734.33504971591	2.1466893651757e-05	\\
7735.31383167614	2.18827959098212e-05	\\
7736.29261363636	2.11195027473133e-05	\\
7737.27139559659	2.14427162411193e-05	\\
7738.25017755682	2.25460662146492e-05	\\
7739.22895951705	2.34608064443921e-05	\\
7740.20774147727	2.27485643323505e-05	\\
7741.1865234375	2.28156990387419e-05	\\
7742.16530539773	2.21772813990446e-05	\\
7743.14408735795	2.10601922015507e-05	\\
7744.12286931818	2.15014503922505e-05	\\
7745.10165127841	2.19970034042848e-05	\\
7746.08043323864	2.18539811217152e-05	\\
7747.05921519886	2.22963594272022e-05	\\
7748.03799715909	2.21337578964336e-05	\\
7749.01677911932	2.01221281598096e-05	\\
7749.99556107955	2.25552897225474e-05	\\
7750.97434303977	2.23389778901472e-05	\\
7751.953125	2.10974868612582e-05	\\
7752.93190696023	2.16987010645205e-05	\\
7753.91068892045	2.24519202608654e-05	\\
7754.88947088068	2.21377862485792e-05	\\
7755.86825284091	2.27703446475941e-05	\\
7756.84703480114	2.26450845502451e-05	\\
7757.82581676136	2.18190932373553e-05	\\
7758.80459872159	2.22716553705929e-05	\\
7759.78338068182	2.34888828154285e-05	\\
7760.76216264205	2.18676629223208e-05	\\
7761.74094460227	2.35203428122973e-05	\\
7762.7197265625	2.36261940364655e-05	\\
7763.69850852273	2.36622675291696e-05	\\
7764.67729048295	2.3584266120465e-05	\\
7765.65607244318	2.39377708062252e-05	\\
7766.63485440341	2.31035540071379e-05	\\
7767.61363636364	2.51330382390107e-05	\\
7768.59241832386	2.33171569434914e-05	\\
7769.57120028409	2.26514444913927e-05	\\
7770.54998224432	2.39830162988939e-05	\\
7771.52876420455	2.28765194399486e-05	\\
7772.50754616477	2.44632353412941e-05	\\
7773.486328125	2.3287233764093e-05	\\
7774.46511008523	2.45785159252248e-05	\\
7775.44389204545	2.42012006100698e-05	\\
7776.42267400568	2.36802581876426e-05	\\
7777.40145596591	2.49036492168442e-05	\\
7778.38023792614	2.47971106161628e-05	\\
7779.35901988636	2.37761781903196e-05	\\
7780.33780184659	2.49710492089385e-05	\\
7781.31658380682	2.4767606780455e-05	\\
7782.29536576705	2.55129264579482e-05	\\
7783.27414772727	2.28083063719339e-05	\\
7784.2529296875	2.41106043025948e-05	\\
7785.23171164773	2.50262642723663e-05	\\
7786.21049360795	2.37922766691798e-05	\\
7787.18927556818	2.57996965348614e-05	\\
7788.16805752841	2.28966846711989e-05	\\
7789.14683948864	2.46033180022398e-05	\\
7790.12562144886	2.46376462134377e-05	\\
7791.10440340909	2.2857036915996e-05	\\
7792.08318536932	2.38241701415316e-05	\\
7793.06196732955	2.39360158589805e-05	\\
7794.04074928977	2.48545911927687e-05	\\
7795.01953125	2.39426231161846e-05	\\
7795.99831321023	2.54770916948606e-05	\\
7796.97709517045	2.51560558749044e-05	\\
7797.95587713068	2.57038297807346e-05	\\
7798.93465909091	2.6077222079106e-05	\\
7799.91344105114	2.39113244961038e-05	\\
7800.89222301136	2.49419208539706e-05	\\
7801.87100497159	2.68033654875411e-05	\\
7802.84978693182	2.64230320617799e-05	\\
7803.82856889205	2.61612628108643e-05	\\
7804.80735085227	2.61541588657202e-05	\\
7805.7861328125	2.56846771739683e-05	\\
7806.76491477273	2.50394691849714e-05	\\
7807.74369673295	2.66415829737234e-05	\\
7808.72247869318	2.5618151656693e-05	\\
7809.70126065341	2.53813278070078e-05	\\
7810.68004261364	2.529227165881e-05	\\
7811.65882457386	2.52513266532106e-05	\\
7812.63760653409	2.63015213117686e-05	\\
7813.61638849432	2.64525862946495e-05	\\
7814.59517045455	2.67608193984302e-05	\\
7815.57395241477	2.60267245444216e-05	\\
7816.552734375	2.40880887916237e-05	\\
7817.53151633523	2.61721192446078e-05	\\
7818.51029829545	2.61455371309537e-05	\\
7819.48908025568	2.45145912522723e-05	\\
7820.46786221591	2.47065932140872e-05	\\
7821.44664417614	2.41208363777765e-05	\\
7822.42542613636	2.58101244700207e-05	\\
7823.40420809659	2.58501489902426e-05	\\
7824.38299005682	2.59753753905585e-05	\\
7825.36177201705	2.59629208725761e-05	\\
7826.34055397727	2.55514169252363e-05	\\
7827.3193359375	2.58683330147208e-05	\\
7828.29811789773	2.64795849533817e-05	\\
7829.27689985795	2.65650755246513e-05	\\
7830.25568181818	2.47241120605887e-05	\\
7831.23446377841	2.62970962363667e-05	\\
7832.21324573864	2.63514601485968e-05	\\
7833.19202769886	2.52062289385081e-05	\\
7834.17080965909	2.72322777001692e-05	\\
7835.14959161932	2.59742212325034e-05	\\
7836.12837357955	2.50164036113439e-05	\\
7837.10715553977	2.41734826032709e-05	\\
7838.0859375	2.62793654906947e-05	\\
7839.06471946023	2.75543722794988e-05	\\
7840.04350142045	2.59915402825123e-05	\\
7841.02228338068	2.81406756340795e-05	\\
7842.00106534091	2.37070469822112e-05	\\
7842.97984730114	2.39393960969016e-05	\\
7843.95862926136	2.49121272712503e-05	\\
7844.93741122159	2.54609578161958e-05	\\
7845.91619318182	2.45484259024974e-05	\\
7846.89497514205	2.58477455122455e-05	\\
7847.87375710227	2.50063067688429e-05	\\
7848.8525390625	2.67269584422675e-05	\\
7849.83132102273	2.65775087659018e-05	\\
7850.81010298295	2.62484359272163e-05	\\
7851.78888494318	2.59301836995974e-05	\\
7852.76766690341	2.6338649183106e-05	\\
7853.74644886364	2.52656344611555e-05	\\
7854.72523082386	2.6657283497042e-05	\\
7855.70401278409	2.63986969407533e-05	\\
7856.68279474432	2.59354880154645e-05	\\
7857.66157670455	2.66755492959967e-05	\\
7858.64035866477	2.76064502529444e-05	\\
7859.619140625	2.63796202755583e-05	\\
7860.59792258523	2.70853147620411e-05	\\
7861.57670454545	2.66325634228488e-05	\\
7862.55548650568	2.77223873245105e-05	\\
7863.53426846591	2.65240654870828e-05	\\
7864.51305042614	2.69814195553021e-05	\\
7865.49183238636	2.7539927509212e-05	\\
7866.47061434659	2.48951412108487e-05	\\
7867.44939630682	2.63008079488927e-05	\\
7868.42817826705	2.67185333931796e-05	\\
7869.40696022727	2.61552195249998e-05	\\
7870.3857421875	2.62307488573123e-05	\\
7871.36452414773	2.80381786656713e-05	\\
7872.34330610795	2.66843741881798e-05	\\
7873.32208806818	2.6501920690168e-05	\\
7874.30087002841	2.50772995949505e-05	\\
7875.27965198864	2.69098349866581e-05	\\
7876.25843394886	2.56549577730196e-05	\\
7877.23721590909	2.76341254327034e-05	\\
7878.21599786932	2.75167999760518e-05	\\
7879.19477982955	2.68173961008909e-05	\\
7880.17356178977	2.62606648289882e-05	\\
7881.15234375	2.62324944591112e-05	\\
7882.13112571023	2.66055685355794e-05	\\
7883.10990767045	2.82846693617354e-05	\\
7884.08868963068	2.74796795119031e-05	\\
7885.06747159091	2.57401033181527e-05	\\
7886.04625355114	2.64355808340714e-05	\\
7887.02503551136	2.71602047876375e-05	\\
7888.00381747159	2.79132395105657e-05	\\
7888.98259943182	2.79813844416726e-05	\\
7889.96138139205	2.74174903076424e-05	\\
7890.94016335227	2.74497891701829e-05	\\
7891.9189453125	2.79467835913453e-05	\\
7892.89772727273	2.70604205567386e-05	\\
7893.87650923295	2.62445049509981e-05	\\
7894.85529119318	2.84441542673803e-05	\\
7895.83407315341	2.80538342237721e-05	\\
7896.81285511364	2.78716334248209e-05	\\
7897.79163707386	2.77446233599837e-05	\\
7898.77041903409	2.8073495752112e-05	\\
7899.74920099432	2.82233455013402e-05	\\
7900.72798295455	2.73832949831988e-05	\\
7901.70676491477	2.6844737514124e-05	\\
7902.685546875	2.81285722538451e-05	\\
7903.66432883523	2.84393817487968e-05	\\
7904.64311079545	2.74106541184283e-05	\\
7905.62189275568	2.78309352355993e-05	\\
7906.60067471591	2.84701393884432e-05	\\
7907.57945667614	2.72714218956158e-05	\\
7908.55823863636	2.76914241479099e-05	\\
7909.53702059659	2.84317346859549e-05	\\
7910.51580255682	2.88809814570741e-05	\\
7911.49458451705	2.78008124554229e-05	\\
7912.47336647727	2.69252370858375e-05	\\
7913.4521484375	2.71925554546832e-05	\\
7914.43093039773	2.74922527641733e-05	\\
7915.40971235795	2.84710645135599e-05	\\
7916.38849431818	2.75131615996815e-05	\\
7917.36727627841	2.98570120409465e-05	\\
7918.34605823864	2.87628453227302e-05	\\
7919.32484019886	2.82504704524755e-05	\\
7920.30362215909	2.65597197376456e-05	\\
7921.28240411932	2.8125958068659e-05	\\
7922.26118607955	3.01615606000153e-05	\\
7923.23996803977	2.97719829698332e-05	\\
7924.21875	2.95635660022946e-05	\\
7925.19753196023	2.66947317112195e-05	\\
7926.17631392045	2.98247346462294e-05	\\
7927.15509588068	2.81135943160307e-05	\\
7928.13387784091	2.88817156703215e-05	\\
7929.11265980114	2.83144662408214e-05	\\
7930.09144176136	2.86193191392224e-05	\\
7931.07022372159	2.90093514397804e-05	\\
7932.04900568182	2.83135487878528e-05	\\
7933.02778764205	2.86940137591154e-05	\\
7934.00656960227	2.85774043705544e-05	\\
7934.9853515625	2.66342406040669e-05	\\
7935.96413352273	2.68177489314724e-05	\\
7936.94291548295	2.93288615555426e-05	\\
7937.92169744318	2.86119318080355e-05	\\
7938.90047940341	2.8964741404797e-05	\\
7939.87926136364	2.9310583303834e-05	\\
7940.85804332386	2.89979218030825e-05	\\
7941.83682528409	2.81536852889778e-05	\\
7942.81560724432	3.10295974396694e-05	\\
7943.79438920455	2.95936423105226e-05	\\
7944.77317116477	2.92864174594588e-05	\\
7945.751953125	3.01453285568585e-05	\\
7946.73073508523	2.94897334949418e-05	\\
7947.70951704545	2.84245874810669e-05	\\
7948.68829900568	2.85150206705161e-05	\\
7949.66708096591	2.84493920983862e-05	\\
7950.64586292614	2.82891442138914e-05	\\
7951.62464488636	2.89552704245774e-05	\\
7952.60342684659	3.07360200816683e-05	\\
7953.58220880682	2.96094562037967e-05	\\
7954.56099076705	2.88647369351713e-05	\\
7955.53977272727	2.7967853472324e-05	\\
7956.5185546875	2.95356765495986e-05	\\
7957.49733664773	2.96289460438627e-05	\\
7958.47611860795	3.03198181158811e-05	\\
7959.45490056818	2.84332915701471e-05	\\
7960.43368252841	3.08269907272127e-05	\\
7961.41246448864	2.81651562451402e-05	\\
7962.39124644886	3.07454434706971e-05	\\
7963.37002840909	3.0609783414455e-05	\\
7964.34881036932	2.84901269192345e-05	\\
7965.32759232955	2.97376475493856e-05	\\
7966.30637428977	2.91310293665638e-05	\\
7967.28515625	2.98772776247767e-05	\\
7968.26393821023	2.9937563459737e-05	\\
7969.24272017045	3.17902403416144e-05	\\
7970.22150213068	2.98868773311331e-05	\\
7971.20028409091	2.9423666746685e-05	\\
7972.17906605114	3.16868915758373e-05	\\
7973.15784801136	3.14659670076826e-05	\\
7974.13662997159	2.84820944678598e-05	\\
7975.11541193182	3.05516776657007e-05	\\
7976.09419389205	3.2662755426189e-05	\\
7977.07297585227	3.13963982872312e-05	\\
7978.0517578125	3.15684598576788e-05	\\
7979.03053977273	3.13892953087805e-05	\\
7980.00932173295	3.21330426856242e-05	\\
7980.98810369318	3.1422612079107e-05	\\
7981.96688565341	3.12521882045693e-05	\\
7982.94566761364	3.01663087269554e-05	\\
7983.92444957386	3.14858709531216e-05	\\
7984.90323153409	3.19749179623479e-05	\\
7985.88201349432	3.09967576953943e-05	\\
7986.86079545455	3.05565733145352e-05	\\
7987.83957741477	3.24579852426316e-05	\\
7988.818359375	3.13898871210879e-05	\\
7989.79714133523	3.15149898892235e-05	\\
7990.77592329545	3.27426043592602e-05	\\
7991.75470525568	3.08743217183603e-05	\\
7992.73348721591	3.21782008377577e-05	\\
7993.71226917614	3.20927262664657e-05	\\
7994.69105113636	3.03150613204164e-05	\\
7995.66983309659	3.25440947615856e-05	\\
7996.64861505682	3.08340304945578e-05	\\
7997.62739701705	3.05967517841875e-05	\\
7998.60617897727	3.13940469797233e-05	\\
7999.5849609375	3.83316593774093e-05	\\
8000.56374289773	3.45678178188413e-05	\\
8001.54252485795	3.30945193534683e-05	\\
8002.52130681818	3.08000381640645e-05	\\
8003.50008877841	3.13293297567125e-05	\\
8004.47887073864	3.16380991621738e-05	\\
8005.45765269886	3.30737606520665e-05	\\
8006.43643465909	3.21337848536209e-05	\\
8007.41521661932	3.29233663502772e-05	\\
8008.39399857955	3.30373845495224e-05	\\
8009.37278053977	3.22977354051855e-05	\\
8010.3515625	3.18412069600018e-05	\\
8011.33034446023	3.19422983802912e-05	\\
8012.30912642045	3.18006584932326e-05	\\
8013.28790838068	3.40231889264962e-05	\\
8014.26669034091	3.24394292503158e-05	\\
8015.24547230114	3.32528746442125e-05	\\
8016.22425426136	3.24787580938939e-05	\\
8017.20303622159	3.17025259921667e-05	\\
8018.18181818182	3.28250065164994e-05	\\
8019.16060014205	3.30901078281958e-05	\\
8020.13938210227	3.27539216662293e-05	\\
8021.1181640625	3.34869160076864e-05	\\
8022.09694602273	3.44307454528973e-05	\\
8023.07572798295	3.13561680642792e-05	\\
8024.05450994318	3.25455973942456e-05	\\
8025.03329190341	3.24278865256925e-05	\\
8026.01207386364	3.16956144194341e-05	\\
8026.99085582386	3.3119371447122e-05	\\
8027.96963778409	3.11354891493142e-05	\\
8028.94841974432	3.16646893180075e-05	\\
8029.92720170455	3.31213521744607e-05	\\
8030.90598366477	3.30008014105396e-05	\\
8031.884765625	3.20647057232854e-05	\\
8032.86354758523	3.28715271617532e-05	\\
8033.84232954545	3.3167172335819e-05	\\
8034.82111150568	3.19622369000866e-05	\\
8035.79989346591	3.26716121982874e-05	\\
8036.77867542614	3.24586258740192e-05	\\
8037.75745738636	3.28446661397575e-05	\\
8038.73623934659	3.19514954279859e-05	\\
8039.71502130682	3.21024271146531e-05	\\
8040.69380326705	3.24479728979067e-05	\\
8041.67258522727	3.18271922262778e-05	\\
8042.6513671875	3.11287440446363e-05	\\
8043.63014914773	3.19753991509812e-05	\\
8044.60893110795	3.06784198149232e-05	\\
8045.58771306818	3.24141208683334e-05	\\
8046.56649502841	3.10939472626834e-05	\\
8047.54527698864	3.25380107148571e-05	\\
8048.52405894886	3.34645296330038e-05	\\
8049.50284090909	3.24201501083758e-05	\\
8050.48162286932	3.10700891212369e-05	\\
8051.46040482955	3.04053163454871e-05	\\
8052.43918678977	3.18776746011988e-05	\\
8053.41796875	3.24919006327258e-05	\\
8054.39675071023	3.11296545820892e-05	\\
8055.37553267045	3.17465394704408e-05	\\
8056.35431463068	3.19247329383547e-05	\\
8057.33309659091	3.18208336795935e-05	\\
8058.31187855114	3.08129849785176e-05	\\
8059.29066051136	3.13524701753458e-05	\\
8060.26944247159	3.36515955546537e-05	\\
8061.24822443182	3.22443504501274e-05	\\
8062.22700639205	3.17601310575024e-05	\\
8063.20578835227	3.10639648127738e-05	\\
8064.1845703125	3.1382458511018e-05	\\
8065.16335227273	3.1210903333145e-05	\\
8066.14213423295	3.17596339536046e-05	\\
8067.12091619318	3.03253469514084e-05	\\
8068.09969815341	3.15627819225906e-05	\\
8069.07848011364	3.06589481828386e-05	\\
8070.05726207386	3.02082236085706e-05	\\
8071.03604403409	3.11952272961128e-05	\\
8072.01482599432	3.30992444764417e-05	\\
8072.99360795455	3.13601051488509e-05	\\
8073.97238991477	3.10360952749751e-05	\\
8074.951171875	3.01950153515133e-05	\\
8075.92995383523	3.08670207226933e-05	\\
8076.90873579545	3.12001089769665e-05	\\
8077.88751775568	3.19102207955314e-05	\\
8078.86629971591	3.17607217463131e-05	\\
8079.84508167614	3.01346185430475e-05	\\
8080.82386363636	3.08494722982799e-05	\\
8081.80264559659	3.1662772528104e-05	\\
8082.78142755682	3.08606687844866e-05	\\
8083.76020951705	2.9811577794107e-05	\\
8084.73899147727	3.01992807641484e-05	\\
8085.7177734375	3.17478939734785e-05	\\
8086.69655539773	2.95625914422172e-05	\\
8087.67533735795	3.13415109281651e-05	\\
8088.65411931818	3.17429007297425e-05	\\
8089.63290127841	3.11684242531459e-05	\\
8090.61168323864	3.08351058083734e-05	\\
8091.59046519886	3.06154371638188e-05	\\
8092.56924715909	2.98022217692922e-05	\\
8093.54802911932	3.24515258610372e-05	\\
8094.52681107955	3.2531846197881e-05	\\
8095.50559303977	3.35039469818767e-05	\\
8096.484375	3.24610116902716e-05	\\
8097.46315696023	3.0384135308965e-05	\\
8098.44193892045	3.09119951410593e-05	\\
8099.42072088068	3.1407418256466e-05	\\
8100.39950284091	3.1140767441223e-05	\\
8101.37828480114	3.24223207675804e-05	\\
8102.35706676136	3.08313027580968e-05	\\
8103.33584872159	3.28626026944726e-05	\\
8104.31463068182	3.22299054173699e-05	\\
8105.29341264205	3.2385128254658e-05	\\
8106.27219460227	3.20439271387003e-05	\\
8107.2509765625	3.21931227920503e-05	\\
8108.22975852273	3.35256253662249e-05	\\
8109.20854048295	3.21364461640194e-05	\\
8110.18732244318	3.22998108344063e-05	\\
8111.16610440341	3.18350570227718e-05	\\
8112.14488636364	3.31613263993322e-05	\\
8113.12366832386	3.24440673667401e-05	\\
8114.10245028409	3.23458590630623e-05	\\
8115.08123224432	3.43131976971859e-05	\\
8116.06001420455	3.17548822083382e-05	\\
8117.03879616477	3.13583017284524e-05	\\
8118.017578125	3.24672887418382e-05	\\
8118.99636008523	3.23547952984022e-05	\\
8119.97514204545	3.42344837961406e-05	\\
8120.95392400568	3.34745782524903e-05	\\
8121.93270596591	3.36269629252197e-05	\\
8122.91148792614	3.28994849799387e-05	\\
8123.89026988636	3.48189833538593e-05	\\
8124.86905184659	3.35899563747163e-05	\\
8125.84783380682	3.38261820630461e-05	\\
8126.82661576705	3.34827323136633e-05	\\
8127.80539772727	3.40028639361138e-05	\\
8128.7841796875	3.32613212491273e-05	\\
8129.76296164773	3.51103761205347e-05	\\
8130.74174360795	3.38281459434588e-05	\\
8131.72052556818	3.4350002212493e-05	\\
8132.69930752841	3.388006665966e-05	\\
8133.67808948864	3.39396245244849e-05	\\
8134.65687144886	3.57056195089183e-05	\\
8135.63565340909	3.45867727358813e-05	\\
8136.61443536932	3.17772271269069e-05	\\
8137.59321732955	3.43602821147628e-05	\\
8138.57199928977	3.47351597088113e-05	\\
8139.55078125	3.20111760134553e-05	\\
8140.52956321023	3.28065661037026e-05	\\
8141.50834517045	3.33745033711729e-05	\\
8142.48712713068	3.33476561261373e-05	\\
8143.46590909091	3.21616144640236e-05	\\
8144.44469105114	3.41349278847754e-05	\\
8145.42347301136	3.32092630331779e-05	\\
8146.40225497159	3.41680314361504e-05	\\
8147.38103693182	3.36341519137854e-05	\\
8148.35981889205	3.33563265845016e-05	\\
8149.33860085227	3.34822903690027e-05	\\
8150.3173828125	3.26447152094841e-05	\\
8151.29616477273	3.17549072384865e-05	\\
8152.27494673295	3.27397441056637e-05	\\
8153.25372869318	3.36411974597025e-05	\\
8154.23251065341	3.35936152043816e-05	\\
8155.21129261364	3.31230341208227e-05	\\
8156.19007457386	3.3806246958122e-05	\\
8157.16885653409	3.34428013085695e-05	\\
8158.14763849432	3.27738287248799e-05	\\
8159.12642045455	3.24951391480697e-05	\\
8160.10520241477	3.2527873489477e-05	\\
8161.083984375	3.42367229724511e-05	\\
8162.06276633523	3.3458207845567e-05	\\
8163.04154829545	3.28516356721189e-05	\\
8164.02033025568	3.38292720067278e-05	\\
8164.99911221591	3.49319268745492e-05	\\
8165.97789417614	3.28154309562358e-05	\\
8166.95667613636	3.34807891629375e-05	\\
8167.93545809659	3.45092047798739e-05	\\
8168.91424005682	3.17986751400056e-05	\\
8169.89302201705	3.34668647456774e-05	\\
8170.87180397727	3.32822768901607e-05	\\
8171.8505859375	3.23488107896063e-05	\\
8172.82936789773	3.32521642982731e-05	\\
8173.80814985795	3.43150732990225e-05	\\
8174.78693181818	3.39423334744876e-05	\\
8175.76571377841	3.33584406293061e-05	\\
8176.74449573864	3.36876083311778e-05	\\
8177.72327769886	3.46191414898837e-05	\\
8178.70205965909	3.51194661015622e-05	\\
8179.68084161932	3.38921905631744e-05	\\
8180.65962357955	3.38852529211228e-05	\\
8181.63840553977	3.41521612180942e-05	\\
8182.6171875	3.40492723188612e-05	\\
8183.59596946023	3.51463179750887e-05	\\
8184.57475142045	3.48861970608363e-05	\\
8185.55353338068	3.51940661988274e-05	\\
8186.53231534091	3.29766656730071e-05	\\
8187.51109730114	3.30828771875952e-05	\\
8188.48987926136	3.32707237740572e-05	\\
8189.46866122159	3.45555197950142e-05	\\
8190.44744318182	3.58473112155901e-05	\\
8191.42622514205	3.36044797328752e-05	\\
8192.40500710227	3.41864059246603e-05	\\
8193.3837890625	3.56770137478292e-05	\\
8194.36257102273	3.44019193997875e-05	\\
8195.34135298295	3.3445019719472e-05	\\
8196.32013494318	3.59694633254599e-05	\\
8197.29891690341	3.39899611123671e-05	\\
8198.27769886364	3.48516832670175e-05	\\
8199.25648082386	3.46552798898884e-05	\\
8200.23526278409	3.443407110482e-05	\\
8201.21404474432	3.52640878254584e-05	\\
8202.19282670454	3.52585464448814e-05	\\
8203.17160866477	3.49051816500613e-05	\\
8204.150390625	3.47174945474721e-05	\\
8205.12917258523	3.61400813912464e-05	\\
8206.10795454545	3.5436084943665e-05	\\
8207.08673650568	3.52992876730669e-05	\\
8208.06551846591	3.57895033899568e-05	\\
8209.04430042614	3.55003184805522e-05	\\
8210.02308238636	3.31734724045094e-05	\\
8211.00186434659	3.61773426861355e-05	\\
8211.98064630682	3.49948742614258e-05	\\
8212.95942826704	3.43864567421539e-05	\\
8213.93821022727	3.66995034345582e-05	\\
8214.9169921875	3.53325567764087e-05	\\
8215.89577414773	3.5360871602945e-05	\\
8216.87455610795	3.57881214178459e-05	\\
8217.85333806818	3.48245088781424e-05	\\
8218.83212002841	3.55671342622638e-05	\\
8219.81090198864	3.70780891131582e-05	\\
8220.78968394886	3.45180158040785e-05	\\
8221.76846590909	3.58005057377411e-05	\\
8222.74724786932	3.42054976538617e-05	\\
8223.72602982954	3.52577637232487e-05	\\
8224.70481178977	3.66103774837057e-05	\\
8225.68359375	3.51781498065508e-05	\\
8226.66237571023	3.65363999081928e-05	\\
8227.64115767045	3.602491192671e-05	\\
8228.61993963068	3.51308118818115e-05	\\
8229.59872159091	3.61753481162646e-05	\\
8230.57750355114	3.51968520113303e-05	\\
8231.55628551136	3.66633101879338e-05	\\
8232.53506747159	3.55786157186919e-05	\\
8233.51384943182	3.53769288758416e-05	\\
8234.49263139204	3.6650780957231e-05	\\
8235.47141335227	3.54899555153065e-05	\\
8236.4501953125	3.5988922293434e-05	\\
8237.42897727273	3.62956779135241e-05	\\
8238.40775923295	3.57829215824756e-05	\\
8239.38654119318	3.56039497203189e-05	\\
8240.36532315341	3.64255652689123e-05	\\
8241.34410511364	3.64306981616779e-05	\\
8242.32288707386	3.50111584620484e-05	\\
8243.30166903409	3.57449520621201e-05	\\
8244.28045099432	3.63486853688073e-05	\\
8245.25923295454	3.553593059418e-05	\\
8246.23801491477	3.58912466988522e-05	\\
8247.216796875	3.73558857038416e-05	\\
8248.19557883523	3.61712503686071e-05	\\
8249.17436079545	3.35307917257502e-05	\\
8250.15314275568	4.00140252684741e-05	\\
8251.13192471591	3.6002979841789e-05	\\
8252.11070667614	3.51061238138358e-05	\\
8253.08948863636	3.47468000113596e-05	\\
8254.06827059659	3.43850755174653e-05	\\
8255.04705255682	3.51161229826187e-05	\\
8256.02583451704	3.51276706872047e-05	\\
8257.00461647727	3.45665992306085e-05	\\
8257.9833984375	3.41589977564932e-05	\\
8258.96218039773	3.57379735856253e-05	\\
8259.94096235795	3.28448544720268e-05	\\
8260.91974431818	3.35000323548345e-05	\\
8261.89852627841	3.61931322317153e-05	\\
8262.87730823864	3.59208978042444e-05	\\
8263.85609019886	3.55248708889433e-05	\\
8264.83487215909	3.67689136850442e-05	\\
8265.81365411932	3.42562306303051e-05	\\
8266.79243607954	3.53712923557717e-05	\\
8267.77121803977	3.49460877558818e-05	\\
8268.75	3.54314701793702e-05	\\
8269.72878196023	3.59857058134351e-05	\\
8270.70756392045	3.50583638284358e-05	\\
8271.68634588068	3.41057331800898e-05	\\
8272.66512784091	3.44603308765508e-05	\\
8273.64390980114	3.56822580893229e-05	\\
8274.62269176136	3.56544210941821e-05	\\
8275.60147372159	3.54396820265102e-05	\\
8276.58025568182	3.51464903520001e-05	\\
8277.55903764204	3.47934734533585e-05	\\
8278.53781960227	3.50769084461519e-05	\\
8279.5166015625	3.6822834663405e-05	\\
8280.49538352273	3.53653736000425e-05	\\
8281.47416548295	3.48758717366232e-05	\\
8282.45294744318	3.53346239331435e-05	\\
8283.43172940341	3.57966750082625e-05	\\
8284.41051136364	3.36680810576592e-05	\\
8285.38929332386	3.56008388530685e-05	\\
8286.36807528409	3.52983908549734e-05	\\
8287.34685724432	3.66721293337844e-05	\\
8288.32563920454	3.64957268138711e-05	\\
8289.30442116477	3.69134373435027e-05	\\
8290.283203125	3.57639614275131e-05	\\
8291.26198508523	3.62566269755596e-05	\\
8292.24076704545	3.57293386878006e-05	\\
8293.21954900568	3.71080074948671e-05	\\
8294.19833096591	3.67124833773875e-05	\\
8295.17711292614	3.50463066165733e-05	\\
8296.15589488636	3.45382175461711e-05	\\
8297.13467684659	3.53701555228246e-05	\\
8298.11345880682	3.57180990465348e-05	\\
8299.09224076704	3.62918136742941e-05	\\
8300.07102272727	3.78489839631852e-05	\\
8301.0498046875	3.67626099663853e-05	\\
8302.02858664773	3.68026723927931e-05	\\
8303.00736860795	3.75309203774366e-05	\\
8303.98615056818	3.55147302676569e-05	\\
8304.96493252841	3.51103677709026e-05	\\
8305.94371448864	3.5301450696345e-05	\\
8306.92249644886	3.69842485963094e-05	\\
8307.90127840909	3.71693285168029e-05	\\
8308.88006036932	3.62587535379013e-05	\\
8309.85884232954	3.75999081225623e-05	\\
8310.83762428977	3.60796975036852e-05	\\
8311.81640625	3.60125371081305e-05	\\
8312.79518821023	3.86983518609392e-05	\\
8313.77397017045	3.65406699359748e-05	\\
8314.75275213068	3.53135163233306e-05	\\
8315.73153409091	3.5989889373587e-05	\\
8316.71031605114	3.60755005865146e-05	\\
8317.68909801136	3.58480828672085e-05	\\
8318.66787997159	3.68602685980674e-05	\\
8319.64666193182	3.78449985666201e-05	\\
8320.62544389204	3.78151403173928e-05	\\
8321.60422585227	3.7372295761056e-05	\\
8322.5830078125	3.64702850216157e-05	\\
8323.56178977273	3.86523915069368e-05	\\
8324.54057173295	3.70614715070925e-05	\\
8325.51935369318	3.63577160451926e-05	\\
8326.49813565341	3.62434790014895e-05	\\
8327.47691761364	3.85288504125766e-05	\\
8328.45569957386	3.75721387232718e-05	\\
8329.43448153409	3.5207543109991e-05	\\
8330.41326349432	3.79215730016954e-05	\\
8331.39204545454	3.67651950944176e-05	\\
8332.37082741477	3.71921856189969e-05	\\
8333.349609375	3.79594115646458e-05	\\
8334.32839133523	3.79376430797374e-05	\\
8335.30717329545	3.67502160846413e-05	\\
8336.28595525568	3.70952144197993e-05	\\
8337.26473721591	3.67554472739613e-05	\\
8338.24351917614	3.84225987167055e-05	\\
8339.22230113636	3.77728219766722e-05	\\
8340.20108309659	3.66366219505613e-05	\\
8341.17986505682	3.69864727747998e-05	\\
8342.15864701704	3.84970535283834e-05	\\
8343.13742897727	3.95585496299111e-05	\\
8344.1162109375	3.80163401785955e-05	\\
8345.09499289773	3.76946606456098e-05	\\
8346.07377485795	3.83560227238413e-05	\\
8347.05255681818	3.65276155898395e-05	\\
8348.03133877841	3.84888854068843e-05	\\
8349.01012073864	3.7555990706342e-05	\\
8349.98890269886	3.66271237165463e-05	\\
8350.96768465909	3.79460180913635e-05	\\
8351.94646661932	3.90529524804164e-05	\\
8352.92524857954	3.78402667623119e-05	\\
8353.90403053977	3.79842173553048e-05	\\
8354.8828125	3.78963693667598e-05	\\
8355.86159446023	3.78828356306076e-05	\\
8356.84037642045	3.85145969424659e-05	\\
8357.81915838068	3.7744128687165e-05	\\
8358.79794034091	3.87613799782199e-05	\\
8359.77672230114	3.83799611656182e-05	\\
8360.75550426136	3.85896014729019e-05	\\
8361.73428622159	3.84837198097897e-05	\\
8362.71306818182	3.83819997099858e-05	\\
8363.69185014204	3.85009376070507e-05	\\
8364.67063210227	3.78188733506152e-05	\\
8365.6494140625	3.81170662895439e-05	\\
8366.62819602273	3.81039023570795e-05	\\
8367.60697798295	3.84204062758142e-05	\\
8368.58575994318	3.81957974292239e-05	\\
8369.56454190341	3.76978776932561e-05	\\
8370.54332386364	3.69332240922197e-05	\\
8371.52210582386	3.83930940802898e-05	\\
8372.50088778409	3.77101697456666e-05	\\
8373.47966974432	3.84698846834777e-05	\\
8374.45845170454	3.79843917416458e-05	\\
8375.43723366477	3.91604615360004e-05	\\
8376.416015625	3.73708317079806e-05	\\
8377.39479758523	3.91795331703585e-05	\\
8378.37357954545	3.87644507137233e-05	\\
8379.35236150568	4.05452090972495e-05	\\
8380.33114346591	3.93988724677448e-05	\\
8381.30992542614	3.87670612118474e-05	\\
8382.28870738636	3.88800373843201e-05	\\
8383.26748934659	3.89814981323429e-05	\\
8384.24627130682	4.05809525526128e-05	\\
8385.22505326704	3.94890308089347e-05	\\
8386.20383522727	3.89316418811013e-05	\\
8387.1826171875	3.85428032626976e-05	\\
8388.16139914773	4.07338437807885e-05	\\
8389.14018110795	4.06129607489185e-05	\\
8390.11896306818	3.9765868758523e-05	\\
8391.09774502841	3.97745337673316e-05	\\
8392.07652698864	4.01811649208067e-05	\\
8393.05530894886	3.96975146447727e-05	\\
8394.03409090909	4.05211427254158e-05	\\
8395.01287286932	4.16767572708749e-05	\\
8395.99165482954	4.00562287969405e-05	\\
8396.97043678977	4.2229714810865e-05	\\
8397.94921875	4.04058867547083e-05	\\
8398.92800071023	4.07288356971926e-05	\\
8399.90678267045	3.98310698804272e-05	\\
8400.88556463068	4.03688243535642e-05	\\
8401.86434659091	3.97128029628973e-05	\\
8402.84312855114	4.28353553083197e-05	\\
8403.82191051136	4.10323932329143e-05	\\
8404.80069247159	4.02118904404613e-05	\\
8405.77947443182	4.11385170507391e-05	\\
8406.75825639204	4.08541329938831e-05	\\
8407.73703835227	4.10953139299527e-05	\\
8408.7158203125	4.3031817388212e-05	\\
8409.69460227273	4.14484421286359e-05	\\
8410.67338423295	4.1646367503409e-05	\\
8411.65216619318	4.18252380487723e-05	\\
8412.63094815341	4.17772514883785e-05	\\
8413.60973011364	4.28350827382847e-05	\\
8414.58851207386	4.16975911374694e-05	\\
8415.56729403409	4.10777955692319e-05	\\
8416.54607599432	4.14872610681894e-05	\\
8417.52485795454	4.17752895582177e-05	\\
8418.50363991477	4.31381548496729e-05	\\
8419.482421875	4.06434854974232e-05	\\
8420.46120383523	4.30147202165911e-05	\\
8421.43998579545	4.13049942773368e-05	\\
8422.41876775568	4.15730356148315e-05	\\
8423.39754971591	4.19006147655642e-05	\\
8424.37633167614	4.13736828612555e-05	\\
8425.35511363636	4.0203896841216e-05	\\
8426.33389559659	4.01812937927147e-05	\\
8427.31267755682	4.12467870504279e-05	\\
8428.29145951704	3.94787374911118e-05	\\
8429.27024147727	4.16118004994706e-05	\\
8430.2490234375	4.13031806087722e-05	\\
8431.22780539773	4.06087940000078e-05	\\
8432.20658735795	4.15052199038511e-05	\\
8433.18536931818	4.1197856112928e-05	\\
8434.16415127841	4.12258944735648e-05	\\
8435.14293323864	4.04398989156423e-05	\\
8436.12171519886	4.15996337907257e-05	\\
8437.10049715909	4.11201330682025e-05	\\
8438.07927911932	4.05197718995299e-05	\\
8439.05806107954	4.1652763929496e-05	\\
8440.03684303977	4.06697896177422e-05	\\
8441.015625	4.08830134527334e-05	\\
8441.99440696023	4.13918913202375e-05	\\
8442.97318892045	3.91211714601016e-05	\\
8443.95197088068	3.89900276309597e-05	\\
8444.93075284091	4.05850209259103e-05	\\
8445.90953480114	4.13066721821774e-05	\\
8446.88831676136	3.9900252303403e-05	\\
8447.86709872159	4.05510932404132e-05	\\
8448.84588068182	4.066416636492e-05	\\
8449.82466264204	4.01007368737669e-05	\\
8450.80344460227	4.18304147943712e-05	\\
8451.7822265625	4.11258379999597e-05	\\
8452.76100852273	3.97732225601282e-05	\\
8453.73979048295	4.02869637880845e-05	\\
8454.71857244318	4.03740849331571e-05	\\
8455.69735440341	4.0943733031393e-05	\\
8456.67613636364	4.11971164002274e-05	\\
8457.65491832386	3.99010041078207e-05	\\
8458.63370028409	3.97297487054378e-05	\\
8459.61248224432	4.11425328740971e-05	\\
8460.59126420454	3.96019035630384e-05	\\
8461.57004616477	3.9219520628389e-05	\\
8462.548828125	4.16058905553862e-05	\\
8463.52761008523	4.06412300910292e-05	\\
8464.50639204545	3.91527416770255e-05	\\
8465.48517400568	3.95011502770812e-05	\\
8466.46395596591	3.99615895615469e-05	\\
8467.44273792614	3.95051599103347e-05	\\
8468.42151988636	3.98091734867841e-05	\\
8469.40030184659	3.8952133872025e-05	\\
8470.37908380682	3.88285386276361e-05	\\
8471.35786576704	4.06750078964465e-05	\\
8472.33664772727	3.84996202277035e-05	\\
8473.3154296875	4.0109917080028e-05	\\
8474.29421164773	3.90412731378141e-05	\\
8475.27299360795	3.87083998941816e-05	\\
8476.25177556818	3.95594245288052e-05	\\
8477.23055752841	3.82211596292648e-05	\\
8478.20933948864	4.05242347059161e-05	\\
8479.18812144886	4.1639313244279e-05	\\
8480.16690340909	3.96319866992624e-05	\\
8481.14568536932	4.03919534680806e-05	\\
8482.12446732954	4.04704160059059e-05	\\
8483.10324928977	3.98556055073573e-05	\\
8484.08203125	3.81773897055548e-05	\\
8485.06081321023	3.92620151697518e-05	\\
8486.03959517045	3.91045865155608e-05	\\
8487.01837713068	4.00739962566658e-05	\\
8487.99715909091	3.98162041197475e-05	\\
8488.97594105114	3.9168998258167e-05	\\
8489.95472301136	3.90593812246556e-05	\\
8490.93350497159	3.88373487165572e-05	\\
8491.91228693182	3.87343637582664e-05	\\
8492.89106889204	4.03032454235005e-05	\\
8493.86985085227	3.77026768455372e-05	\\
8494.8486328125	4.04029105154118e-05	\\
8495.82741477273	4.0805964279697e-05	\\
8496.80619673295	3.79054186592187e-05	\\
8497.78497869318	3.86108416630406e-05	\\
8498.76376065341	3.87077979843945e-05	\\
8499.74254261364	4.07824451519674e-05	\\
8500.72132457386	3.97672457067638e-05	\\
8501.70010653409	4.03234448284277e-05	\\
8502.67888849432	3.78381879036659e-05	\\
8503.65767045454	3.92594332662701e-05	\\
8504.63645241477	3.92585829130856e-05	\\
8505.615234375	3.72834949354436e-05	\\
8506.59401633523	3.88553656967881e-05	\\
8507.57279829545	3.87751485962879e-05	\\
8508.55158025568	3.94913488602368e-05	\\
8509.53036221591	3.85027983238275e-05	\\
8510.50914417614	3.83062631811216e-05	\\
8511.48792613636	3.81448541870658e-05	\\
8512.46670809659	3.83035840745014e-05	\\
8513.44549005682	3.88707313395117e-05	\\
8514.42427201704	4.00744679643551e-05	\\
8515.40305397727	3.79300059159943e-05	\\
8516.3818359375	3.8307180431089e-05	\\
8517.36061789773	3.80499790304119e-05	\\
8518.33939985795	3.80415643965463e-05	\\
8519.31818181818	3.70510824209711e-05	\\
8520.29696377841	4.040432352607e-05	\\
8521.27574573864	3.71563261859809e-05	\\
8522.25452769886	3.85062125494724e-05	\\
8523.23330965909	3.91601135107928e-05	\\
8524.21209161932	3.78873270614282e-05	\\
8525.19087357954	3.83004939607335e-05	\\
8526.16965553977	3.77888692201089e-05	\\
8527.1484375	3.80382123780724e-05	\\
8528.12721946023	3.79371302235582e-05	\\
8529.10600142045	3.80072225591841e-05	\\
8530.08478338068	3.83953508207402e-05	\\
8531.06356534091	3.89951187214635e-05	\\
8532.04234730114	3.95841047584115e-05	\\
8533.02112926136	3.7553836128989e-05	\\
8533.99991122159	3.86544276351589e-05	\\
8534.97869318182	3.73158939537012e-05	\\
8535.95747514204	3.98533416626203e-05	\\
8536.93625710227	3.84482379253392e-05	\\
8537.9150390625	3.99458467131689e-05	\\
8538.89382102273	3.70369891479988e-05	\\
8539.87260298295	3.59904598087141e-05	\\
8540.85138494318	3.9114146567218e-05	\\
8541.83016690341	3.89652495746078e-05	\\
8542.80894886364	3.75909198711136e-05	\\
8543.78773082386	3.78213634389041e-05	\\
8544.76651278409	3.877704243017e-05	\\
8545.74529474432	3.82123152635629e-05	\\
8546.72407670454	3.9033973179346e-05	\\
8547.70285866477	3.76392039738072e-05	\\
8548.681640625	3.84262536395104e-05	\\
8549.66042258523	3.99866051998831e-05	\\
8550.63920454545	3.91701115059143e-05	\\
};
\addplot [color=blue,solid,forget plot]
  table[row sep=crcr]{
8550.63920454545	3.91701115059143e-05	\\
8551.61798650568	3.78306763036579e-05	\\
8552.59676846591	3.78809110256816e-05	\\
8553.57555042614	3.87179994238447e-05	\\
8554.55433238636	3.76133633833956e-05	\\
8555.53311434659	3.69251510738217e-05	\\
8556.51189630682	3.74401212901586e-05	\\
8557.49067826704	3.83518973911994e-05	\\
8558.46946022727	3.88203889115584e-05	\\
8559.4482421875	3.77375587799053e-05	\\
8560.42702414773	3.76333153088318e-05	\\
8561.40580610795	3.80096720570724e-05	\\
8562.38458806818	4.06098979336072e-05	\\
8563.36337002841	3.91271298760418e-05	\\
8564.34215198864	3.74601859678528e-05	\\
8565.32093394886	3.74204811578109e-05	\\
8566.29971590909	3.71847335678986e-05	\\
8567.27849786932	3.8503513715891e-05	\\
8568.25727982954	3.79803894889575e-05	\\
8569.23606178977	3.79038513835989e-05	\\
8570.21484375	3.93159352320304e-05	\\
8571.19362571023	3.99106344485592e-05	\\
8572.17240767045	3.82695044549078e-05	\\
8573.15118963068	3.80510343112486e-05	\\
8574.12997159091	3.93359430594008e-05	\\
8575.10875355114	3.81472873291522e-05	\\
8576.08753551136	3.91853139029642e-05	\\
8577.06631747159	3.63587344819381e-05	\\
8578.04509943182	3.82266691595677e-05	\\
8579.02388139204	3.77929476952945e-05	\\
8580.00266335227	3.89236511420472e-05	\\
8580.9814453125	3.81386040781442e-05	\\
8581.96022727273	3.88285901480157e-05	\\
8582.93900923295	3.83671774848674e-05	\\
8583.91779119318	3.86689985824115e-05	\\
8584.89657315341	3.82266664790739e-05	\\
8585.87535511364	3.8568094099809e-05	\\
8586.85413707386	3.87915380055687e-05	\\
8587.83291903409	3.87709978679726e-05	\\
8588.81170099432	3.81537072421812e-05	\\
8589.79048295454	3.80836269783695e-05	\\
8590.76926491477	3.82220539111503e-05	\\
8591.748046875	3.80583000362612e-05	\\
8592.72682883523	3.81598293654536e-05	\\
8593.70561079545	3.72580775976601e-05	\\
8594.68439275568	3.71000968164927e-05	\\
8595.66317471591	3.88701408098952e-05	\\
8596.64195667614	3.8106072651416e-05	\\
8597.62073863636	3.76779043219572e-05	\\
8598.59952059659	3.79228174262163e-05	\\
8599.57830255682	4.00156428856293e-05	\\
8600.55708451704	3.75727787536871e-05	\\
8601.53586647727	3.93219804459471e-05	\\
8602.5146484375	3.95017421232273e-05	\\
8603.49343039773	3.87484254143095e-05	\\
8604.47221235795	3.92297274511534e-05	\\
8605.45099431818	3.88884993070833e-05	\\
8606.42977627841	3.81093591636512e-05	\\
8607.40855823864	3.79071496973473e-05	\\
8608.38734019886	3.78714851720559e-05	\\
8609.36612215909	3.70973799849541e-05	\\
8610.34490411932	3.77412194966007e-05	\\
8611.32368607954	3.82199431699917e-05	\\
8612.30246803977	3.88854895774002e-05	\\
8613.28125	3.88206647728158e-05	\\
8614.26003196023	3.78865758341357e-05	\\
8615.23881392045	3.85460137867043e-05	\\
8616.21759588068	3.93770482415632e-05	\\
8617.19637784091	3.81969890376196e-05	\\
8618.17515980114	3.99741829559042e-05	\\
8619.15394176136	3.94844204584151e-05	\\
8620.13272372159	3.84584286180043e-05	\\
8621.11150568182	3.86833873725736e-05	\\
8622.09028764204	3.90917127152929e-05	\\
8623.06906960227	3.90992580991524e-05	\\
8624.0478515625	3.93207700914462e-05	\\
8625.02663352273	3.82454183412913e-05	\\
8626.00541548295	3.80879328499047e-05	\\
8626.98419744318	3.91270608880382e-05	\\
8627.96297940341	3.91571135591166e-05	\\
8628.94176136364	3.81186835540192e-05	\\
8629.92054332386	3.69653591910247e-05	\\
8630.89932528409	3.90064474656963e-05	\\
8631.87810724432	3.96145251022283e-05	\\
8632.85688920454	3.86213102971783e-05	\\
8633.83567116477	3.78256758327045e-05	\\
8634.814453125	3.95338590498745e-05	\\
8635.79323508523	3.918987477788e-05	\\
8636.77201704545	3.94034812996566e-05	\\
8637.75079900568	3.95643191420755e-05	\\
8638.72958096591	3.88345769140795e-05	\\
8639.70836292614	3.93384592341615e-05	\\
8640.68714488636	3.91983072987061e-05	\\
8641.66592684659	3.73110036843769e-05	\\
8642.64470880682	3.85194369785724e-05	\\
8643.62349076704	3.90011026829195e-05	\\
8644.60227272727	3.89544762810719e-05	\\
8645.5810546875	3.96175499200328e-05	\\
8646.55983664773	3.86308067757982e-05	\\
8647.53861860795	3.80412891010455e-05	\\
8648.51740056818	3.89874936124952e-05	\\
8649.49618252841	3.73403195447204e-05	\\
8650.47496448864	3.91731859188693e-05	\\
8651.45374644886	3.77514791426006e-05	\\
8652.43252840909	4.03634564920676e-05	\\
8653.41131036932	3.8423310555787e-05	\\
8654.39009232954	3.83699656306727e-05	\\
8655.36887428977	4.04140193715899e-05	\\
8656.34765625	3.91448917291233e-05	\\
8657.32643821023	3.92163945784418e-05	\\
8658.30522017045	3.88037628777026e-05	\\
8659.28400213068	3.85240494032654e-05	\\
8660.26278409091	3.94987525761927e-05	\\
8661.24156605114	3.97067708086742e-05	\\
8662.22034801136	3.81219141197087e-05	\\
8663.19912997159	3.78538328723514e-05	\\
8664.17791193182	3.97923367245605e-05	\\
8665.15669389204	3.96543766455711e-05	\\
8666.13547585227	3.91265490267911e-05	\\
8667.1142578125	3.99671200776057e-05	\\
8668.09303977273	3.80480829439216e-05	\\
8669.07182173295	3.75059333654507e-05	\\
8670.05060369318	3.95055523784448e-05	\\
8671.02938565341	3.84520637240382e-05	\\
8672.00816761364	3.84713518709519e-05	\\
8672.98694957386	3.91092624068336e-05	\\
8673.96573153409	3.86037101414292e-05	\\
8674.94451349432	3.78320344649618e-05	\\
8675.92329545454	3.78831932168682e-05	\\
8676.90207741477	3.9040619518091e-05	\\
8677.880859375	3.6224662196351e-05	\\
8678.85964133523	3.92583202763138e-05	\\
8679.83842329545	3.7267024655242e-05	\\
8680.81720525568	3.94278083314001e-05	\\
8681.79598721591	3.86662525858083e-05	\\
8682.77476917614	3.89009411739531e-05	\\
8683.75355113636	3.77333522889656e-05	\\
8684.73233309659	3.96117284579599e-05	\\
8685.71111505682	3.83570999503633e-05	\\
8686.68989701704	3.72730990247313e-05	\\
8687.66867897727	3.91897517810754e-05	\\
8688.6474609375	3.74979353677216e-05	\\
8689.62624289773	3.85423161296847e-05	\\
8690.60502485795	3.74232776957609e-05	\\
8691.58380681818	3.85723095976431e-05	\\
8692.56258877841	3.76886633714362e-05	\\
8693.54137073864	3.75102105912247e-05	\\
8694.52015269886	3.81726570779765e-05	\\
8695.49893465909	3.75378857585581e-05	\\
8696.47771661932	3.80147665528988e-05	\\
8697.45649857954	3.89549575047368e-05	\\
8698.43528053977	3.71912246850568e-05	\\
8699.4140625	3.63339617065954e-05	\\
8700.39284446023	3.69445081519116e-05	\\
8701.37162642045	3.76990037866008e-05	\\
8702.35040838068	3.87740135864606e-05	\\
8703.32919034091	3.92503682557044e-05	\\
8704.30797230114	3.67750205151461e-05	\\
8705.28675426136	3.82151518523491e-05	\\
8706.26553622159	3.93123672221103e-05	\\
8707.24431818182	3.86364593852375e-05	\\
8708.22310014204	3.83929045757113e-05	\\
8709.20188210227	3.86220708929807e-05	\\
8710.1806640625	3.87253433011269e-05	\\
8711.15944602273	3.80348538209318e-05	\\
8712.13822798295	3.8224289477466e-05	\\
8713.11700994318	3.73443394914086e-05	\\
8714.09579190341	3.83369546223697e-05	\\
8715.07457386364	3.73525701099543e-05	\\
8716.05335582386	3.79004430677748e-05	\\
8717.03213778409	3.769450619707e-05	\\
8718.01091974432	3.59102875796188e-05	\\
8718.98970170454	3.78218672575291e-05	\\
8719.96848366477	3.85388815247265e-05	\\
8720.947265625	3.80373080408551e-05	\\
8721.92604758523	3.88744007990867e-05	\\
8722.90482954545	3.91836798644131e-05	\\
8723.88361150568	3.72082109544103e-05	\\
8724.86239346591	3.75828762421751e-05	\\
8725.84117542614	3.86151681573138e-05	\\
8726.81995738636	3.78807764978677e-05	\\
8727.79873934659	3.95835413490313e-05	\\
8728.77752130682	3.82511268457115e-05	\\
8729.75630326704	3.66329048818844e-05	\\
8730.73508522727	3.79276399550872e-05	\\
8731.7138671875	3.84613409220162e-05	\\
8732.69264914773	3.65352889396168e-05	\\
8733.67143110795	3.80019944873115e-05	\\
8734.65021306818	3.84831131489078e-05	\\
8735.62899502841	3.67541793562844e-05	\\
8736.60777698864	3.81129268260783e-05	\\
8737.58655894886	3.70262418662209e-05	\\
8738.56534090909	3.8384106082418e-05	\\
8739.54412286932	3.8043379474255e-05	\\
8740.52290482954	3.65258644771678e-05	\\
8741.50168678977	3.82520373167107e-05	\\
8742.48046875	3.80557896678447e-05	\\
8743.45925071023	3.84019953392251e-05	\\
8744.43803267045	3.89946155130511e-05	\\
8745.41681463068	3.7335815851773e-05	\\
8746.39559659091	3.98505252082795e-05	\\
8747.37437855114	3.77108398642775e-05	\\
8748.35316051136	3.72902474247044e-05	\\
8749.33194247159	3.75117380882068e-05	\\
8750.31072443182	3.79643072027765e-05	\\
8751.28950639204	3.86950552090915e-05	\\
8752.26828835227	3.77378184079439e-05	\\
8753.2470703125	3.87330073204568e-05	\\
8754.22585227273	3.91609374364723e-05	\\
8755.20463423295	3.76206969245632e-05	\\
8756.18341619318	3.69859750993242e-05	\\
8757.16219815341	3.88057269651611e-05	\\
8758.14098011364	3.76236388310171e-05	\\
8759.11976207386	3.73999060712786e-05	\\
8760.09854403409	3.81177645864968e-05	\\
8761.07732599432	3.71639329271546e-05	\\
8762.05610795454	3.75596617181611e-05	\\
8763.03488991477	3.75002050856144e-05	\\
8764.013671875	3.7327264520123e-05	\\
8764.99245383523	3.88550785518546e-05	\\
8765.97123579545	3.70910299105434e-05	\\
8766.95001775568	3.79028746099928e-05	\\
8767.92879971591	3.76334385602198e-05	\\
8768.90758167614	3.78870572305404e-05	\\
8769.88636363636	3.80306094615898e-05	\\
8770.86514559659	3.89642610999387e-05	\\
8771.84392755682	3.73030341307643e-05	\\
8772.82270951704	3.7728430846141e-05	\\
8773.80149147727	3.7466578969012e-05	\\
8774.7802734375	3.75408463627036e-05	\\
8775.75905539773	3.7933578237507e-05	\\
8776.73783735795	3.72086089989254e-05	\\
8777.71661931818	3.74787169832709e-05	\\
8778.69540127841	3.79070311661663e-05	\\
8779.67418323864	3.7204853681466e-05	\\
8780.65296519886	3.77116716015715e-05	\\
8781.63174715909	3.72648912188077e-05	\\
8782.61052911932	3.78165400155692e-05	\\
8783.58931107954	3.7426892113358e-05	\\
8784.56809303977	3.84513419169742e-05	\\
8785.546875	3.74710013836403e-05	\\
8786.52565696023	3.85535820028436e-05	\\
8787.50443892045	3.91521869462661e-05	\\
8788.48322088068	3.70537375026548e-05	\\
8789.46200284091	3.7636076061452e-05	\\
8790.44078480114	3.83249910971849e-05	\\
8791.41956676136	3.67348439049666e-05	\\
8792.39834872159	3.79995024059804e-05	\\
8793.37713068182	3.79234577713924e-05	\\
8794.35591264204	3.76067863783156e-05	\\
8795.33469460227	3.77870101867782e-05	\\
8796.3134765625	3.80545687206038e-05	\\
8797.29225852273	3.69906155438354e-05	\\
8798.27104048295	3.66458780549617e-05	\\
8799.24982244318	3.62697424843322e-05	\\
8800.22860440341	3.84798580311773e-05	\\
8801.20738636364	3.71529980627277e-05	\\
8802.18616832386	3.73310264341527e-05	\\
8803.16495028409	3.67273402127308e-05	\\
8804.14373224432	3.88782923558054e-05	\\
8805.12251420454	3.7978070537174e-05	\\
8806.10129616477	3.70924101490305e-05	\\
8807.080078125	3.63517352696376e-05	\\
8808.05886008523	3.62636486369105e-05	\\
8809.03764204545	3.82116934008635e-05	\\
8810.01642400568	3.60730283338244e-05	\\
8810.99520596591	3.63793725576363e-05	\\
8811.97398792614	3.60802907902746e-05	\\
8812.95276988636	3.72065251155942e-05	\\
8813.93155184659	3.72695234600786e-05	\\
8814.91033380682	3.53815783728548e-05	\\
8815.88911576704	3.7725437546036e-05	\\
8816.86789772727	3.67351916729529e-05	\\
8817.8466796875	3.73094963503017e-05	\\
8818.82546164773	3.62455686152433e-05	\\
8819.80424360795	3.61798500845555e-05	\\
8820.78302556818	3.62192595436701e-05	\\
8821.76180752841	3.55743748532085e-05	\\
8822.74058948864	3.50121931658393e-05	\\
8823.71937144886	3.66851524801232e-05	\\
8824.69815340909	3.53104257557782e-05	\\
8825.67693536932	3.54726604782563e-05	\\
8826.65571732954	3.71809206432752e-05	\\
8827.63449928977	3.47784142346828e-05	\\
8828.61328125	3.56351736897138e-05	\\
8829.59206321023	3.40763950619563e-05	\\
8830.57084517045	3.59778807920417e-05	\\
8831.54962713068	3.5777969605631e-05	\\
8832.52840909091	3.51509443018498e-05	\\
8833.50719105114	3.54787416928505e-05	\\
8834.48597301136	3.54605970262053e-05	\\
8835.46475497159	3.61087586737306e-05	\\
8836.44353693182	3.55684095878228e-05	\\
8837.42231889204	3.54937966542348e-05	\\
8838.40110085227	3.55315259305209e-05	\\
8839.3798828125	3.61795300410054e-05	\\
8840.35866477273	3.57255643308722e-05	\\
8841.33744673295	3.58255486135843e-05	\\
8842.31622869318	3.55511365792439e-05	\\
8843.29501065341	3.65481614099835e-05	\\
8844.27379261364	3.59466772185871e-05	\\
8845.25257457386	3.65365822199524e-05	\\
8846.23135653409	3.61944778926837e-05	\\
8847.21013849432	3.63266229179869e-05	\\
8848.18892045454	3.64401060399792e-05	\\
8849.16770241477	3.5341775506586e-05	\\
8850.146484375	3.63024676819779e-05	\\
8851.12526633523	3.59980980645177e-05	\\
8852.10404829545	3.64446236486844e-05	\\
8853.08283025568	3.61864243469248e-05	\\
8854.06161221591	3.68877596086084e-05	\\
8855.04039417614	3.63281562957446e-05	\\
8856.01917613636	3.70479929354772e-05	\\
8856.99795809659	3.62637410371413e-05	\\
8857.97674005682	3.68327447743709e-05	\\
8858.95552201704	3.646808539286e-05	\\
8859.93430397727	3.65179272268435e-05	\\
8860.9130859375	3.67138896683207e-05	\\
8861.89186789773	3.70995040833479e-05	\\
8862.87064985795	3.61487037595057e-05	\\
8863.84943181818	3.63103389366018e-05	\\
8864.82821377841	3.77222604743304e-05	\\
8865.80699573864	3.59310525739154e-05	\\
8866.78577769886	3.57600782823541e-05	\\
8867.76455965909	3.7331792146693e-05	\\
8868.74334161932	3.66439136865607e-05	\\
8869.72212357954	3.63759639513805e-05	\\
8870.70090553977	3.57446269511764e-05	\\
8871.6796875	3.66018077730566e-05	\\
8872.65846946023	3.72541515819041e-05	\\
8873.63725142045	3.65654816602658e-05	\\
8874.61603338068	3.75390136770465e-05	\\
8875.59481534091	3.60012376606707e-05	\\
8876.57359730114	3.67778524299876e-05	\\
8877.55237926136	3.63891339373116e-05	\\
8878.53116122159	3.61863785737248e-05	\\
8879.50994318182	3.64075858748088e-05	\\
8880.48872514204	3.69369839904925e-05	\\
8881.46750710227	3.63237776935159e-05	\\
8882.4462890625	3.62354691653643e-05	\\
8883.42507102273	3.76531582182619e-05	\\
8884.40385298295	3.66291484223176e-05	\\
8885.38263494318	3.62950515286735e-05	\\
8886.36141690341	3.62180552680539e-05	\\
8887.34019886364	3.60960832813572e-05	\\
8888.31898082386	3.66634349118405e-05	\\
8889.29776278409	3.70960857217468e-05	\\
8890.27654474432	3.72700934571502e-05	\\
8891.25532670454	3.59395957891627e-05	\\
8892.23410866477	3.89017575336005e-05	\\
8893.212890625	3.69936175928519e-05	\\
8894.19167258523	3.54966992575295e-05	\\
8895.17045454545	3.73574392900467e-05	\\
8896.14923650568	3.76239461050806e-05	\\
8897.12801846591	3.65802178953337e-05	\\
8898.10680042614	3.77347296705947e-05	\\
8899.08558238636	3.69815492118452e-05	\\
8900.06436434659	3.72209429388056e-05	\\
8901.04314630682	3.86329876380185e-05	\\
8902.02192826704	3.78323154902626e-05	\\
8903.00071022727	3.59514931981279e-05	\\
8903.9794921875	3.58990914813424e-05	\\
8904.95827414773	3.74859232782955e-05	\\
8905.93705610795	3.69937266949746e-05	\\
8906.91583806818	3.71987201052534e-05	\\
8907.89462002841	3.73047430399213e-05	\\
8908.87340198864	3.69493763355335e-05	\\
8909.85218394886	3.72314273731331e-05	\\
8910.83096590909	3.75967667874004e-05	\\
8911.80974786932	3.79158949645648e-05	\\
8912.78852982954	3.72872040593601e-05	\\
8913.76731178977	3.82042909414011e-05	\\
8914.74609375	3.73231322523413e-05	\\
8915.72487571023	3.75395027458464e-05	\\
8916.70365767045	3.73835946225487e-05	\\
8917.68243963068	3.66239316795808e-05	\\
8918.66122159091	3.70191220529237e-05	\\
8919.64000355114	3.72697735955243e-05	\\
8920.61878551136	3.68991481249312e-05	\\
8921.59756747159	3.63951616360683e-05	\\
8922.57634943182	3.78399225815482e-05	\\
8923.55513139204	3.76129898358168e-05	\\
8924.53391335227	3.77104955214297e-05	\\
8925.5126953125	3.72025654565163e-05	\\
8926.49147727273	3.77292245347755e-05	\\
8927.47025923295	3.77480725531402e-05	\\
8928.44904119318	3.73297710466731e-05	\\
8929.42782315341	3.80385344549664e-05	\\
8930.40660511364	3.58597108838222e-05	\\
8931.38538707386	3.78099427739615e-05	\\
8932.36416903409	3.78540716392438e-05	\\
8933.34295099432	3.68394824936465e-05	\\
8934.32173295454	3.8064193847808e-05	\\
8935.30051491477	3.75238119947933e-05	\\
8936.279296875	3.71794346682288e-05	\\
8937.25807883523	3.73255694801774e-05	\\
8938.23686079545	3.74075802918742e-05	\\
8939.21564275568	3.75978921752074e-05	\\
8940.19442471591	3.73598224961478e-05	\\
8941.17320667614	3.73792439374215e-05	\\
8942.15198863636	3.8155927950047e-05	\\
8943.13077059659	3.78175349249725e-05	\\
8944.10955255682	3.69207727164567e-05	\\
8945.08833451704	3.86441716204e-05	\\
8946.06711647727	3.88915188734338e-05	\\
8947.0458984375	3.72159522587977e-05	\\
8948.02468039773	3.76284243142408e-05	\\
8949.00346235795	3.85679152120282e-05	\\
8949.98224431818	3.61912691059822e-05	\\
8950.96102627841	3.62293289116091e-05	\\
8951.93980823864	3.64321095288866e-05	\\
8952.91859019886	3.6351878707512e-05	\\
8953.89737215909	3.67730281160418e-05	\\
8954.87615411932	3.75223712530922e-05	\\
8955.85493607954	3.6675471243867e-05	\\
8956.83371803977	3.75580416756278e-05	\\
8957.8125	3.65034084763043e-05	\\
8958.79128196023	3.71724860622591e-05	\\
8959.77006392045	3.76233593108461e-05	\\
8960.74884588068	3.82863908297287e-05	\\
8961.72762784091	3.77375603105612e-05	\\
8962.70640980114	3.63941028665862e-05	\\
8963.68519176136	3.6880552906319e-05	\\
8964.66397372159	3.91982385309254e-05	\\
8965.64275568182	3.70147185625517e-05	\\
8966.62153764204	3.70564150605827e-05	\\
8967.60031960227	3.7979295565608e-05	\\
8968.5791015625	3.70137449314885e-05	\\
8969.55788352273	3.79894818826579e-05	\\
8970.53666548295	3.7961070764302e-05	\\
8971.51544744318	3.77924498736563e-05	\\
8972.49422940341	3.82525025495316e-05	\\
8973.47301136364	3.71880083921362e-05	\\
8974.45179332386	3.60288109585705e-05	\\
8975.43057528409	3.73880809144668e-05	\\
8976.40935724432	3.72992504750693e-05	\\
8977.38813920454	3.75225608419159e-05	\\
8978.36692116477	3.65936403085491e-05	\\
8979.345703125	3.80012625315918e-05	\\
8980.32448508523	3.64297364462205e-05	\\
8981.30326704545	3.59609949267012e-05	\\
8982.28204900568	3.68552203442735e-05	\\
8983.26083096591	3.63967489747308e-05	\\
8984.23961292614	3.58495090192708e-05	\\
8985.21839488636	3.62407703914202e-05	\\
8986.19717684659	3.77082796673688e-05	\\
8987.17595880682	3.58473137730007e-05	\\
8988.15474076704	3.55522596542836e-05	\\
8989.13352272727	3.68658284581842e-05	\\
8990.1123046875	3.70500711189054e-05	\\
8991.09108664773	3.5927238744842e-05	\\
8992.06986860795	3.58605775056217e-05	\\
8993.04865056818	3.62597006540812e-05	\\
8994.02743252841	3.65646279271205e-05	\\
8995.00621448864	3.67006443281529e-05	\\
8995.98499644886	3.78175340679518e-05	\\
8996.96377840909	3.68928876658185e-05	\\
8997.94256036932	3.6731730399503e-05	\\
8998.92134232954	3.61253893460854e-05	\\
8999.90012428977	3.67084134389161e-05	\\
9000.87890625	3.68288922174011e-05	\\
9001.85768821023	3.56013856220845e-05	\\
9002.83647017045	3.60809790758251e-05	\\
9003.81525213068	3.66022444698726e-05	\\
9004.79403409091	3.48412144853566e-05	\\
9005.77281605114	3.65520999495407e-05	\\
9006.75159801136	3.45668168908066e-05	\\
9007.73037997159	3.63517945796758e-05	\\
9008.70916193182	3.42822748111604e-05	\\
9009.68794389204	3.62721697931987e-05	\\
9010.66672585227	3.59723602533434e-05	\\
9011.6455078125	3.53776919381762e-05	\\
9012.62428977273	3.6385652795681e-05	\\
9013.60307173295	3.59550960798586e-05	\\
9014.58185369318	3.38907060505562e-05	\\
9015.56063565341	3.47679791605881e-05	\\
9016.53941761364	3.54268432128288e-05	\\
9017.51819957386	3.53400706052118e-05	\\
9018.49698153409	3.53536796044081e-05	\\
9019.47576349432	3.55011650028834e-05	\\
9020.45454545454	3.5178254443932e-05	\\
9021.43332741477	3.47130414553513e-05	\\
9022.412109375	3.51880761315605e-05	\\
9023.39089133523	3.42265079201813e-05	\\
9024.36967329545	3.36897679838421e-05	\\
9025.34845525568	3.55329774469181e-05	\\
9026.32723721591	3.54526496556664e-05	\\
9027.30601917614	3.4410970464839e-05	\\
9028.28480113636	3.45448765365101e-05	\\
9029.26358309659	3.44360302031236e-05	\\
9030.24236505682	3.55468746799223e-05	\\
9031.22114701704	3.50989545305179e-05	\\
9032.19992897727	3.40714337888001e-05	\\
9033.1787109375	3.4371975702829e-05	\\
9034.15749289773	3.46433376476302e-05	\\
9035.13627485795	3.4854395134876e-05	\\
9036.11505681818	3.47620370194754e-05	\\
9037.09383877841	3.40737474801452e-05	\\
9038.07262073864	3.53317597716982e-05	\\
9039.05140269886	3.41642811617126e-05	\\
9040.03018465909	3.37562379745222e-05	\\
9041.00896661932	3.60627598716693e-05	\\
9041.98774857954	3.50301022846804e-05	\\
9042.96653053977	3.49670273766692e-05	\\
9043.9453125	3.55031675574489e-05	\\
9044.92409446023	3.49709043836467e-05	\\
9045.90287642045	3.42465230107729e-05	\\
9046.88165838068	3.56120460452962e-05	\\
9047.86044034091	3.51139358710669e-05	\\
9048.83922230114	3.40196347211617e-05	\\
9049.81800426136	3.51290544564333e-05	\\
9050.79678622159	3.6033767539119e-05	\\
9051.77556818182	3.49330974658389e-05	\\
9052.75435014204	3.56441554134395e-05	\\
9053.73313210227	3.57283023422924e-05	\\
9054.7119140625	3.54156874282202e-05	\\
9055.69069602273	3.44926101458019e-05	\\
9056.66947798295	3.62377701256437e-05	\\
9057.64825994318	3.52532784479758e-05	\\
9058.62704190341	3.63905827663188e-05	\\
9059.60582386364	3.50544179697454e-05	\\
9060.58460582386	3.42540119596495e-05	\\
9061.56338778409	3.45442750148482e-05	\\
9062.54216974432	3.46407704151329e-05	\\
9063.52095170454	3.42568656133876e-05	\\
9064.49973366477	3.4632613715386e-05	\\
9065.478515625	3.45260268496504e-05	\\
9066.45729758523	3.57122024254082e-05	\\
9067.43607954545	3.29589408422866e-05	\\
9068.41486150568	3.32687549495815e-05	\\
9069.39364346591	3.42710486126908e-05	\\
9070.37242542614	3.42604068365759e-05	\\
9071.35120738636	3.36824399957379e-05	\\
9072.32998934659	3.5236546943637e-05	\\
9073.30877130682	3.53625254079937e-05	\\
9074.28755326704	3.41454763904483e-05	\\
9075.26633522727	3.47356409011714e-05	\\
9076.2451171875	3.41645114634206e-05	\\
9077.22389914773	3.44616430667561e-05	\\
9078.20268110795	3.40254202752453e-05	\\
9079.18146306818	3.45714534811423e-05	\\
9080.16024502841	3.36811817845514e-05	\\
9081.13902698864	3.59495723288712e-05	\\
9082.11780894886	3.5086575337226e-05	\\
9083.09659090909	3.45403442769476e-05	\\
9084.07537286932	3.44792052814419e-05	\\
9085.05415482954	3.50928595040917e-05	\\
9086.03293678977	3.47943879472224e-05	\\
9087.01171875	3.48195835429813e-05	\\
9087.99050071023	3.51303783568688e-05	\\
9088.96928267045	3.40776519449167e-05	\\
9089.94806463068	3.47405070104553e-05	\\
9090.92684659091	3.42387195612221e-05	\\
9091.90562855114	3.50454067015632e-05	\\
9092.88441051136	3.46177979830742e-05	\\
9093.86319247159	3.53987993240234e-05	\\
9094.84197443182	3.37645304432466e-05	\\
9095.82075639204	3.45613787645145e-05	\\
9096.79953835227	3.53407069470794e-05	\\
9097.7783203125	3.41727651986825e-05	\\
9098.75710227273	3.58039260239584e-05	\\
9099.73588423295	3.43695489836828e-05	\\
9100.71466619318	3.50011563013223e-05	\\
9101.69344815341	3.38863816562084e-05	\\
9102.67223011364	3.39144015802135e-05	\\
9103.65101207386	3.3430311865971e-05	\\
9104.62979403409	3.44913095293847e-05	\\
9105.60857599432	3.4376334400235e-05	\\
9106.58735795454	3.38010822978656e-05	\\
9107.56613991477	3.45685445684895e-05	\\
9108.544921875	3.5001626004712e-05	\\
9109.52370383523	3.4375121821859e-05	\\
9110.50248579545	3.39637547307792e-05	\\
9111.48126775568	3.36797398607843e-05	\\
9112.46004971591	3.48716707169468e-05	\\
9113.43883167614	3.39746199160423e-05	\\
9114.41761363636	3.45465583720576e-05	\\
9115.39639559659	3.37516681255605e-05	\\
9116.37517755682	3.50402180158835e-05	\\
9117.35395951704	3.32608703302146e-05	\\
9118.33274147727	3.3768538380486e-05	\\
9119.3115234375	3.33613151258068e-05	\\
9120.29030539773	3.23843032826287e-05	\\
9121.26908735795	3.41950179414039e-05	\\
9122.24786931818	3.43529006874932e-05	\\
9123.22665127841	3.32523043747946e-05	\\
9124.20543323864	3.37462002711672e-05	\\
9125.18421519886	3.31892049201639e-05	\\
9126.16299715909	3.38051891387181e-05	\\
9127.14177911932	3.40283363597925e-05	\\
9128.12056107954	3.3775183836776e-05	\\
9129.09934303977	3.32490527726595e-05	\\
9130.078125	3.31678105725365e-05	\\
9131.05690696023	3.20808807277238e-05	\\
9132.03568892045	3.35736795688199e-05	\\
9133.01447088068	3.32816428944639e-05	\\
9133.99325284091	3.36514079102685e-05	\\
9134.97203480114	3.39725653284972e-05	\\
9135.95081676136	3.36271561709365e-05	\\
9136.92959872159	3.38517281696852e-05	\\
9137.90838068182	3.39987135564296e-05	\\
9138.88716264204	3.47606888431084e-05	\\
9139.86594460227	3.30625358302475e-05	\\
9140.8447265625	3.44664870017958e-05	\\
9141.82350852273	3.29519310816671e-05	\\
9142.80229048295	3.33222396496833e-05	\\
9143.78107244318	3.35147119380999e-05	\\
9144.75985440341	3.41322027980981e-05	\\
9145.73863636364	3.31237847411541e-05	\\
9146.71741832386	3.30167023504655e-05	\\
9147.69620028409	3.35087766577148e-05	\\
9148.67498224432	3.32102601031937e-05	\\
9149.65376420454	3.36774456588842e-05	\\
9150.63254616477	3.25431944616375e-05	\\
9151.611328125	3.34524844195367e-05	\\
9152.59011008523	3.44183296644535e-05	\\
9153.56889204545	3.44234485739852e-05	\\
9154.54767400568	3.37603975505267e-05	\\
9155.52645596591	3.30586479482274e-05	\\
9156.50523792614	3.34361119363099e-05	\\
9157.48401988636	3.2990872148098e-05	\\
9158.46280184659	3.28490755805216e-05	\\
9159.44158380682	3.34012222652413e-05	\\
9160.42036576704	3.40422852111062e-05	\\
9161.39914772727	3.27867086525779e-05	\\
9162.3779296875	3.41018563152753e-05	\\
9163.35671164773	3.40426609907301e-05	\\
9164.33549360795	3.38071035095347e-05	\\
9165.31427556818	3.42085364218583e-05	\\
9166.29305752841	3.45533411414335e-05	\\
9167.27183948864	3.38841065043239e-05	\\
9168.25062144886	3.38014487173749e-05	\\
9169.22940340909	3.40418633207188e-05	\\
9170.20818536932	3.46164672128153e-05	\\
9171.18696732954	3.35750538553942e-05	\\
9172.16574928977	3.50923195927675e-05	\\
9173.14453125	3.46419186767275e-05	\\
9174.12331321023	3.41648528137882e-05	\\
9175.10209517045	3.41662367738062e-05	\\
9176.08087713068	3.40130280607052e-05	\\
9177.05965909091	3.41764630494379e-05	\\
9178.03844105114	3.54768177235751e-05	\\
9179.01722301136	3.423717364427e-05	\\
9179.99600497159	3.54506748177428e-05	\\
9180.97478693182	3.5105143593573e-05	\\
9181.95356889204	3.52210231942635e-05	\\
9182.93235085227	3.45362774163495e-05	\\
9183.9111328125	3.5402461483894e-05	\\
9184.88991477273	3.49721910759769e-05	\\
9185.86869673295	3.49260961803206e-05	\\
9186.84747869318	3.53474709780979e-05	\\
9187.82626065341	3.46300908431769e-05	\\
9188.80504261364	3.48132793240343e-05	\\
9189.78382457386	3.50368946844915e-05	\\
9190.76260653409	3.42499283120423e-05	\\
9191.74138849432	3.46502418327066e-05	\\
9192.72017045454	3.43270929551472e-05	\\
9193.69895241477	3.46335578549462e-05	\\
9194.677734375	3.45241505089951e-05	\\
9195.65651633523	3.49477733318674e-05	\\
9196.63529829545	3.40822752593877e-05	\\
9197.61408025568	3.46455567410831e-05	\\
9198.59286221591	3.45463786948997e-05	\\
9199.57164417614	3.63626287114741e-05	\\
9200.55042613636	3.50940745300158e-05	\\
9201.52920809659	3.47633718327078e-05	\\
9202.50799005682	3.52937468507107e-05	\\
9203.48677201704	3.40543067746003e-05	\\
9204.46555397727	3.55849158228423e-05	\\
9205.4443359375	3.51412503923968e-05	\\
9206.42311789773	3.37284451603092e-05	\\
9207.40189985795	3.56421578470045e-05	\\
9208.38068181818	3.48906357865313e-05	\\
9209.35946377841	3.49477058393988e-05	\\
9210.33824573864	3.41855474075367e-05	\\
9211.31702769886	3.49131132581965e-05	\\
9212.29580965909	3.49410371246463e-05	\\
9213.27459161932	3.40057040384704e-05	\\
9214.25337357954	3.36244803009479e-05	\\
9215.23215553977	3.49284900621565e-05	\\
9216.2109375	3.56058876661847e-05	\\
9217.18971946023	3.52472672798342e-05	\\
9218.16850142045	3.56311100124892e-05	\\
9219.14728338068	3.51206893099404e-05	\\
9220.12606534091	3.44693005215285e-05	\\
9221.10484730114	3.6140740209198e-05	\\
9222.08362926136	3.50268787296476e-05	\\
9223.06241122159	3.63851372634647e-05	\\
9224.04119318182	3.57303983922541e-05	\\
9225.01997514204	3.59905491533055e-05	\\
9225.99875710227	3.52208701263689e-05	\\
9226.9775390625	3.67863275379462e-05	\\
9227.95632102273	3.66926792581906e-05	\\
9228.93510298295	3.58921690693575e-05	\\
9229.91388494318	3.66931741830524e-05	\\
9230.89266690341	3.59991736375648e-05	\\
9231.87144886364	3.59639263851941e-05	\\
9232.85023082386	3.61398283373212e-05	\\
9233.82901278409	3.53998923320066e-05	\\
9234.80779474432	3.6251714362996e-05	\\
9235.78657670454	3.5017986104338e-05	\\
9236.76535866477	3.40394250284789e-05	\\
9237.744140625	3.65217023519794e-05	\\
9238.72292258523	3.61671095003148e-05	\\
9239.70170454545	3.61414222430109e-05	\\
9240.68048650568	3.53339108055066e-05	\\
9241.65926846591	3.64705729350756e-05	\\
9242.63805042614	3.47113949143618e-05	\\
9243.61683238636	3.68224386289732e-05	\\
9244.59561434659	3.59775539608358e-05	\\
9245.57439630682	3.64920415323768e-05	\\
9246.55317826704	3.53289793275923e-05	\\
9247.53196022727	3.56226480295676e-05	\\
9248.5107421875	3.57494377235037e-05	\\
9249.48952414773	3.71336360205068e-05	\\
9250.46830610795	3.64306709912757e-05	\\
9251.44708806818	3.47229496143388e-05	\\
9252.42587002841	3.64201086365807e-05	\\
9253.40465198864	3.67725544539895e-05	\\
9254.38343394886	3.52926156269365e-05	\\
9255.36221590909	3.52394265026177e-05	\\
9256.34099786932	3.60624243082069e-05	\\
9257.31977982954	3.56456748855667e-05	\\
9258.29856178977	3.64900141568118e-05	\\
9259.27734375	3.53408817109079e-05	\\
9260.25612571023	3.69927388366726e-05	\\
9261.23490767045	3.67474033358703e-05	\\
9262.21368963068	3.63198184202654e-05	\\
9263.19247159091	3.62614171179467e-05	\\
9264.17125355114	3.61727278372241e-05	\\
9265.15003551136	3.47840839414334e-05	\\
9266.12881747159	3.54902006503675e-05	\\
9267.10759943182	3.57595004055949e-05	\\
9268.08638139204	3.46180180997415e-05	\\
9269.06516335227	3.47536530670358e-05	\\
9270.0439453125	3.66370052953681e-05	\\
9271.02272727273	3.66580684993446e-05	\\
9272.00150923295	3.53858057407178e-05	\\
9272.98029119318	3.63551034022478e-05	\\
9273.95907315341	3.73843272898641e-05	\\
9274.93785511364	3.64770313952317e-05	\\
9275.91663707386	3.57191875088126e-05	\\
9276.89541903409	3.66310318027689e-05	\\
9277.87420099432	3.46869013679629e-05	\\
9278.85298295454	3.56911368142673e-05	\\
9279.83176491477	3.63991973701476e-05	\\
9280.810546875	3.57809705666453e-05	\\
9281.78932883523	3.60571947653048e-05	\\
9282.76811079545	3.59766409946866e-05	\\
9283.74689275568	3.65262754167595e-05	\\
9284.72567471591	3.51799560478509e-05	\\
9285.70445667614	3.7109931044926e-05	\\
9286.68323863636	3.51057129327655e-05	\\
9287.66202059659	3.64237813995626e-05	\\
9288.64080255682	3.64065713918955e-05	\\
9289.61958451704	3.57314710456263e-05	\\
9290.59836647727	3.60261397853352e-05	\\
9291.5771484375	3.57816428573453e-05	\\
9292.55593039773	3.57483685100672e-05	\\
9293.53471235795	3.59524352012533e-05	\\
9294.51349431818	3.43697026161854e-05	\\
9295.49227627841	3.66525205732995e-05	\\
9296.47105823864	3.41923893810762e-05	\\
9297.44984019886	3.50731016414156e-05	\\
9298.42862215909	3.54020708688875e-05	\\
9299.40740411932	3.46252406174751e-05	\\
9300.38618607954	3.48517480105838e-05	\\
9301.36496803977	3.50810754607214e-05	\\
9302.34375	3.50270800187804e-05	\\
9303.32253196023	3.49759918002625e-05	\\
9304.30131392045	3.4520821450199e-05	\\
9305.28009588068	3.50815641036182e-05	\\
9306.25887784091	3.47789660508225e-05	\\
9307.23765980114	3.57354883192342e-05	\\
9308.21644176136	3.56261630369586e-05	\\
9309.19522372159	3.52444557499686e-05	\\
9310.17400568182	3.46509390093101e-05	\\
9311.15278764204	3.43413406384268e-05	\\
9312.13156960227	3.45351651567444e-05	\\
9313.1103515625	3.44499459228798e-05	\\
9314.08913352273	3.27113179936098e-05	\\
9315.06791548295	3.3680858213133e-05	\\
9316.04669744318	3.46816518806807e-05	\\
9317.02547940341	3.41388488179017e-05	\\
9318.00426136364	3.42561614014954e-05	\\
9318.98304332386	3.39664823644646e-05	\\
9319.96182528409	3.38729306944762e-05	\\
9320.94060724432	3.34916927502703e-05	\\
9321.91938920454	3.4297459787839e-05	\\
9322.89817116477	3.31217452060004e-05	\\
9323.876953125	3.38652948728032e-05	\\
9324.85573508523	3.42543580247912e-05	\\
9325.83451704545	3.46180740932692e-05	\\
9326.81329900568	3.43504267156556e-05	\\
9327.79208096591	3.36556597805996e-05	\\
9328.77086292614	3.34300290354156e-05	\\
9329.74964488636	3.44278954847866e-05	\\
9330.72842684659	3.40467963105247e-05	\\
9331.70720880682	3.4348400387026e-05	\\
9332.68599076704	3.41240433506304e-05	\\
9333.66477272727	3.41859575495819e-05	\\
9334.6435546875	3.44810348858116e-05	\\
9335.62233664773	3.37406426716092e-05	\\
9336.60111860795	3.34912449436515e-05	\\
9337.57990056818	3.49746352987798e-05	\\
9338.55868252841	3.41890430940603e-05	\\
9339.53746448864	3.38465590227521e-05	\\
9340.51624644886	3.67919812703558e-05	\\
9341.49502840909	3.39238315170126e-05	\\
9342.47381036932	3.49994148460748e-05	\\
9343.45259232954	3.5445857739618e-05	\\
9344.43137428977	3.51209723082112e-05	\\
9345.41015625	3.44959139502007e-05	\\
9346.38893821023	3.65021616830492e-05	\\
9347.36772017045	3.39406795790172e-05	\\
9348.34650213068	3.61872754536017e-05	\\
9349.32528409091	3.53341303289002e-05	\\
9350.30406605114	3.56024655175826e-05	\\
9351.28284801136	3.5022112171355e-05	\\
9352.26162997159	3.66958047974127e-05	\\
9353.24041193182	3.40367781349042e-05	\\
9354.21919389204	3.55227976226823e-05	\\
9355.19797585227	3.59287418576625e-05	\\
9356.1767578125	3.54316549536018e-05	\\
9357.15553977273	3.66185441042874e-05	\\
9358.13432173295	3.59823654280354e-05	\\
9359.11310369318	3.73807023610835e-05	\\
9360.09188565341	3.63315806057549e-05	\\
9361.07066761364	3.58213425658308e-05	\\
9362.04944957386	3.60425319776555e-05	\\
9363.02823153409	3.53889452083847e-05	\\
9364.00701349432	3.66953585905842e-05	\\
9364.98579545454	3.59609719696813e-05	\\
9365.96457741477	3.58401923929102e-05	\\
9366.943359375	3.59320973886816e-05	\\
9367.92214133523	3.63404206431939e-05	\\
9368.90092329545	3.59950080112856e-05	\\
9369.87970525568	3.64010110908096e-05	\\
9370.85848721591	3.55730086935015e-05	\\
9371.83726917614	3.65116954946724e-05	\\
9372.81605113636	3.62394905049411e-05	\\
9373.79483309659	3.66202499801563e-05	\\
9374.77361505682	3.5712922278169e-05	\\
9375.75239701704	3.63670547016726e-05	\\
9376.73117897727	3.66002844583346e-05	\\
9377.7099609375	3.69360860891924e-05	\\
9378.68874289773	3.70425812998484e-05	\\
9379.66752485795	3.59202553134118e-05	\\
9380.64630681818	3.56311309822734e-05	\\
9381.62508877841	3.6186131790427e-05	\\
9382.60387073864	3.68526124840387e-05	\\
9383.58265269886	3.59885632471429e-05	\\
9384.56143465909	3.56890546049108e-05	\\
9385.54021661932	3.65040729887175e-05	\\
9386.51899857954	3.62375325402805e-05	\\
9387.49778053977	3.65884546223028e-05	\\
9388.4765625	3.75366146713686e-05	\\
9389.45534446023	3.64440911175966e-05	\\
9390.43412642045	3.5919914345692e-05	\\
9391.41290838068	3.60995901817172e-05	\\
9392.39169034091	3.60120116698448e-05	\\
9393.37047230114	3.59582558374193e-05	\\
9394.34925426136	3.52995752863278e-05	\\
9395.32803622159	3.5872662995129e-05	\\
9396.30681818182	3.63564219901211e-05	\\
9397.28560014204	3.64812042396118e-05	\\
9398.26438210227	3.57211423935104e-05	\\
9399.2431640625	3.62065787720768e-05	\\
9400.22194602273	3.893601507997e-05	\\
9401.20072798295	3.6417370438246e-05	\\
9402.17950994318	3.61193220769257e-05	\\
9403.15829190341	3.7484653259458e-05	\\
9404.13707386364	3.64172195264707e-05	\\
9405.11585582386	3.62317656598033e-05	\\
9406.09463778409	3.52668729675666e-05	\\
9407.07341974432	3.57668374284454e-05	\\
9408.05220170454	3.58545136125765e-05	\\
9409.03098366477	3.62907226301764e-05	\\
9410.009765625	3.55187254146415e-05	\\
9410.98854758523	3.56763700761585e-05	\\
9411.96732954545	3.61759183459068e-05	\\
9412.94611150568	3.56815218381808e-05	\\
9413.92489346591	3.64714201285477e-05	\\
9414.90367542614	3.55894191386384e-05	\\
9415.88245738636	3.52482642365383e-05	\\
9416.86123934659	3.68211388780207e-05	\\
9417.84002130682	3.63035304060384e-05	\\
9418.81880326704	3.57172925873514e-05	\\
9419.79758522727	3.60486079329381e-05	\\
9420.7763671875	3.66153073282726e-05	\\
9421.75514914773	3.55913734089679e-05	\\
9422.73393110795	3.58303553383663e-05	\\
9423.71271306818	3.6086983831862e-05	\\
9424.69149502841	3.54938268575176e-05	\\
9425.67027698864	3.54811928313391e-05	\\
9426.64905894886	3.60337015142017e-05	\\
9427.62784090909	3.58665495003133e-05	\\
9428.60662286932	3.54892979961261e-05	\\
9429.58540482954	3.47722563098272e-05	\\
9430.56418678977	3.63562013517366e-05	\\
9431.54296875	3.51653840536656e-05	\\
9432.52175071023	3.6106436944925e-05	\\
9433.50053267045	3.59560005389883e-05	\\
9434.47931463068	3.47748544248351e-05	\\
9435.45809659091	3.55452710958615e-05	\\
9436.43687855114	3.56428175554633e-05	\\
9437.41566051136	3.58158939524834e-05	\\
9438.39444247159	3.62052595410196e-05	\\
9439.37322443182	3.51136537730406e-05	\\
9440.35200639204	3.42564772003764e-05	\\
9441.33078835227	3.56654963404612e-05	\\
9442.3095703125	3.42300027527301e-05	\\
9443.28835227273	3.35412917062488e-05	\\
9444.26713423295	3.47278875157696e-05	\\
9445.24591619318	3.52254910807754e-05	\\
9446.22469815341	3.43356562478176e-05	\\
9447.20348011364	3.38883951056862e-05	\\
9448.18226207386	3.38489753797876e-05	\\
9449.16104403409	3.41637313378851e-05	\\
9450.13982599432	3.37184115475418e-05	\\
9451.11860795454	3.55876379446473e-05	\\
9452.09738991477	3.47732700341646e-05	\\
9453.076171875	3.38186629971883e-05	\\
9454.05495383523	3.34522779301222e-05	\\
9455.03373579545	3.36398783802015e-05	\\
9456.01251775568	3.35357776944877e-05	\\
9456.99129971591	3.39285817126846e-05	\\
9457.97008167614	3.45625072950006e-05	\\
9458.94886363636	3.40750274274938e-05	\\
9459.92764559659	3.44991340994996e-05	\\
9460.90642755682	3.33106075193099e-05	\\
9461.88520951704	3.36752001329255e-05	\\
9462.86399147727	3.38992703519448e-05	\\
9463.8427734375	3.46997163869961e-05	\\
9464.82155539773	3.41894389004515e-05	\\
9465.80033735795	3.32146833880872e-05	\\
9466.77911931818	3.3841873172557e-05	\\
9467.75790127841	3.37489325035481e-05	\\
9468.73668323864	3.32424433985988e-05	\\
9469.71546519886	3.39438651483083e-05	\\
9470.69424715909	3.35514307175305e-05	\\
9471.67302911932	3.43877637954959e-05	\\
9472.65181107954	3.45675126310454e-05	\\
9473.63059303977	3.39008748464437e-05	\\
9474.609375	3.30983678297938e-05	\\
9475.58815696023	3.38753704055473e-05	\\
9476.56693892045	3.38603688178095e-05	\\
9477.54572088068	3.44285528526423e-05	\\
9478.52450284091	3.48467924628199e-05	\\
9479.50328480114	3.39817232949254e-05	\\
9480.48206676136	3.39962633540358e-05	\\
9481.46084872159	3.43638165085991e-05	\\
9482.43963068182	3.29106267581304e-05	\\
9483.41841264204	3.32383307801131e-05	\\
9484.39719460227	3.44965069075218e-05	\\
9485.3759765625	3.40298696674292e-05	\\
9486.35475852273	3.45446074356293e-05	\\
9487.33354048295	3.44314537656103e-05	\\
9488.31232244318	3.37736695991934e-05	\\
9489.29110440341	3.4034750370614e-05	\\
9490.26988636364	3.38923200834804e-05	\\
9491.24866832386	3.43033565192953e-05	\\
9492.22745028409	3.3880307226649e-05	\\
9493.20623224432	3.38523735542018e-05	\\
9494.18501420454	3.41782978629353e-05	\\
9495.16379616477	3.36013595895133e-05	\\
9496.142578125	3.39475437301422e-05	\\
9497.12136008523	3.32550530679158e-05	\\
9498.10014204545	3.37003617583596e-05	\\
9499.07892400568	3.37454264165398e-05	\\
9500.05770596591	3.2884234112615e-05	\\
9501.03648792614	3.35004228822419e-05	\\
9502.01526988636	3.42859668008874e-05	\\
9502.99405184659	3.3686464551081e-05	\\
9503.97283380682	3.30605283420552e-05	\\
9504.95161576704	3.43936059565097e-05	\\
9505.93039772727	3.28729418369315e-05	\\
9506.9091796875	3.24161495504227e-05	\\
9507.88796164773	3.29367506870091e-05	\\
9508.86674360795	3.26073124760167e-05	\\
9509.84552556818	3.3005729473499e-05	\\
9510.82430752841	3.22957151201546e-05	\\
9511.80308948864	3.46261581741025e-05	\\
9512.78187144886	3.3171805733473e-05	\\
9513.76065340909	3.30677988177827e-05	\\
9514.73943536932	3.27544720520667e-05	\\
9515.71821732954	3.28439574784903e-05	\\
9516.69699928977	3.32732578165644e-05	\\
9517.67578125	3.28257614275894e-05	\\
9518.65456321023	3.33318828828109e-05	\\
9519.63334517045	3.29112939858516e-05	\\
9520.61212713068	3.17474476280818e-05	\\
9521.59090909091	3.22556813970121e-05	\\
9522.56969105114	3.20585192371989e-05	\\
9523.54847301136	3.30772696874728e-05	\\
9524.52725497159	3.30539890228842e-05	\\
9525.50603693182	3.33927366623123e-05	\\
9526.48481889204	3.32586045088924e-05	\\
9527.46360085227	3.13038800652835e-05	\\
9528.4423828125	3.24775210689885e-05	\\
9529.42116477273	3.23869436289127e-05	\\
9530.39994673295	3.16070913155625e-05	\\
9531.37872869318	3.22814724270663e-05	\\
9532.35751065341	3.28310564386087e-05	\\
9533.33629261364	3.1060297641213e-05	\\
9534.31507457386	3.21391017554126e-05	\\
9535.29385653409	3.10229935606326e-05	\\
9536.27263849432	3.14555373183294e-05	\\
9537.25142045454	3.05729901570812e-05	\\
9538.23020241477	3.13763722600069e-05	\\
9539.208984375	3.11707251884518e-05	\\
9540.18776633523	3.05485007507638e-05	\\
9541.16654829545	3.18611586407352e-05	\\
9542.14533025568	3.00020820980665e-05	\\
9543.12411221591	3.05238176799582e-05	\\
9544.10289417614	3.15336060676615e-05	\\
9545.08167613636	3.08710547976167e-05	\\
9546.06045809659	3.00268743356132e-05	\\
9547.03924005682	3.05413075218606e-05	\\
9548.01802201704	2.94760755252394e-05	\\
9548.99680397727	3.00895869417944e-05	\\
9549.9755859375	2.9131939913032e-05	\\
9550.95436789773	3.05985836655738e-05	\\
9551.93314985795	3.08031677047583e-05	\\
9552.91193181818	3.08632271059369e-05	\\
9553.89071377841	3.06022244663988e-05	\\
9554.86949573864	3.04854309856461e-05	\\
9555.84827769886	2.97140238929057e-05	\\
9556.82705965909	2.86732507318584e-05	\\
9557.80584161932	2.95584592739302e-05	\\
9558.78462357954	2.97567617893303e-05	\\
9559.76340553977	2.87646415420997e-05	\\
9560.7421875	3.01744049797969e-05	\\
9561.72096946023	2.95338493109683e-05	\\
9562.69975142045	2.96922882090941e-05	\\
9563.67853338068	2.9079437190365e-05	\\
9564.65731534091	2.95227589148025e-05	\\
9565.63609730114	2.98585869482984e-05	\\
9566.61487926136	2.9919636051093e-05	\\
9567.59366122159	2.98571988908856e-05	\\
9568.57244318182	2.92813045317945e-05	\\
9569.55122514204	2.83564116869008e-05	\\
9570.53000710227	2.83849259323352e-05	\\
9571.5087890625	2.8989860388644e-05	\\
9572.48757102273	2.96395968897565e-05	\\
9573.46635298295	2.94521929056118e-05	\\
9574.44513494318	2.95335283386814e-05	\\
9575.42391690341	2.86755044584804e-05	\\
9576.40269886364	2.91602729416983e-05	\\
9577.38148082386	2.87908646512759e-05	\\
9578.36026278409	2.850305088062e-05	\\
9579.33904474432	2.90200124315822e-05	\\
9580.31782670454	2.93880732492547e-05	\\
9581.29660866477	2.76578229623159e-05	\\
9582.275390625	2.80473453884091e-05	\\
9583.25417258523	2.91322186909025e-05	\\
9584.23295454545	2.75287302227898e-05	\\
9585.21173650568	2.8726137520789e-05	\\
9586.19051846591	2.85414748903099e-05	\\
9587.16930042614	2.73808453932347e-05	\\
9588.14808238636	2.7588680897595e-05	\\
9589.12686434659	2.70924726053915e-05	\\
9590.10564630682	2.82151294466059e-05	\\
9591.08442826704	2.85475792496331e-05	\\
9592.06321022727	2.76203891827441e-05	\\
9593.0419921875	2.70971987178484e-05	\\
9594.02077414773	2.73547384055106e-05	\\
9594.99955610795	2.68860213302746e-05	\\
9595.97833806818	2.76800168552156e-05	\\
9596.95712002841	2.7008902073616e-05	\\
9597.93590198864	2.77622334367843e-05	\\
9598.91468394886	2.80650898481257e-05	\\
9599.89346590909	2.67597002412518e-05	\\
9600.87224786932	2.57900299762156e-05	\\
9601.85102982954	2.63181994878971e-05	\\
9602.82981178977	2.70299186429872e-05	\\
9603.80859375	2.68153572104482e-05	\\
9604.78737571023	2.67809530106998e-05	\\
9605.76615767045	2.72739168528981e-05	\\
9606.74493963068	2.67884637984488e-05	\\
9607.72372159091	2.69620384404908e-05	\\
9608.70250355114	2.70153695895994e-05	\\
9609.68128551136	2.65688385077221e-05	\\
9610.66006747159	2.69863475111412e-05	\\
9611.63884943182	2.65177615633748e-05	\\
9612.61763139204	2.53885021233838e-05	\\
9613.59641335227	2.57966136723578e-05	\\
9614.5751953125	2.61448374095844e-05	\\
9615.55397727273	2.59841725240266e-05	\\
9616.53275923295	2.64806586812262e-05	\\
9617.51154119318	2.60370902595686e-05	\\
9618.49032315341	2.59787243416757e-05	\\
9619.46910511364	2.61283679150653e-05	\\
9620.44788707386	2.55752043029355e-05	\\
9621.42666903409	2.61107599561171e-05	\\
9622.40545099432	2.62974319294522e-05	\\
9623.38423295454	2.58957145622885e-05	\\
9624.36301491477	2.57782946155628e-05	\\
9625.341796875	2.67990441102938e-05	\\
9626.32057883523	2.43438967359252e-05	\\
9627.29936079545	2.46285390155045e-05	\\
9628.27814275568	2.56659339171717e-05	\\
9629.25692471591	2.43602924525338e-05	\\
9630.23570667614	2.4664926937729e-05	\\
9631.21448863636	2.48619627350433e-05	\\
9632.19327059659	2.51915831474951e-05	\\
9633.17205255682	2.48802020567573e-05	\\
9634.15083451704	2.5604726431245e-05	\\
9635.12961647727	2.44049743595304e-05	\\
9636.1083984375	2.42213674100603e-05	\\
9637.08718039773	2.41641543324211e-05	\\
9638.06596235795	2.37978682593046e-05	\\
9639.04474431818	2.45145206788885e-05	\\
9640.02352627841	2.44586623498548e-05	\\
9641.00230823864	2.50664426781113e-05	\\
9641.98109019886	2.39409141324053e-05	\\
9642.95987215909	2.4155211234698e-05	\\
9643.93865411932	2.44128424358483e-05	\\
9644.91743607954	2.40522581733814e-05	\\
9645.89621803977	2.32421336786138e-05	\\
9646.875	2.31825350671803e-05	\\
9647.85378196023	2.28384826031062e-05	\\
9648.83256392045	2.38047444305233e-05	\\
9649.81134588068	2.40494820183784e-05	\\
9650.79012784091	2.33739515071959e-05	\\
9651.76890980114	2.34663587741589e-05	\\
9652.74769176136	2.37788668059092e-05	\\
9653.72647372159	2.31777135765128e-05	\\
9654.70525568182	2.29017470134691e-05	\\
9655.68403764204	2.20932221026875e-05	\\
9656.66281960227	2.23704192998431e-05	\\
9657.6416015625	2.38468905342738e-05	\\
9658.62038352273	2.27333445499437e-05	\\
9659.59916548295	2.25196091169452e-05	\\
9660.57794744318	2.23625865194727e-05	\\
9661.55672940341	2.25170506768202e-05	\\
9662.53551136364	2.27514192706588e-05	\\
9663.51429332386	2.2955767522076e-05	\\
9664.49307528409	2.22020042829986e-05	\\
9665.47185724432	2.31852304879894e-05	\\
9666.45063920454	2.20882744354313e-05	\\
9667.42942116477	2.16974930770172e-05	\\
9668.408203125	2.29094463797361e-05	\\
9669.38698508523	2.30089564081875e-05	\\
9670.36576704545	2.35442300067312e-05	\\
9671.34454900568	2.2639006461564e-05	\\
9672.32333096591	2.17725252705484e-05	\\
9673.30211292614	2.30427145026105e-05	\\
9674.28089488636	2.18452310305707e-05	\\
9675.25967684659	2.22457667897566e-05	\\
9676.23845880682	2.2736021003092e-05	\\
9677.21724076704	2.35001933230618e-05	\\
9678.19602272727	2.26564541393283e-05	\\
9679.1748046875	2.23419317344065e-05	\\
9680.15358664773	2.27429134393141e-05	\\
9681.13236860795	2.15749722276814e-05	\\
9682.11115056818	2.31275019070926e-05	\\
9683.08993252841	2.27787027619299e-05	\\
9684.06871448864	2.1658666602135e-05	\\
9685.04749644886	2.17456507600701e-05	\\
9686.02627840909	2.29696597580703e-05	\\
9687.00506036932	2.20199979098003e-05	\\
9687.98384232954	2.25807403372282e-05	\\
9688.96262428977	2.3340446927073e-05	\\
9689.94140625	2.26285106518373e-05	\\
9690.92018821023	2.293733991251e-05	\\
9691.89897017045	2.19457796240163e-05	\\
9692.87775213068	2.21882322875458e-05	\\
9693.85653409091	2.25375082233848e-05	\\
9694.83531605114	2.24533847946385e-05	\\
9695.81409801136	2.12898770836311e-05	\\
9696.79287997159	2.3444996713208e-05	\\
9697.77166193182	2.24004056912839e-05	\\
9698.75044389204	2.12012004693079e-05	\\
9699.72922585227	2.23078498115779e-05	\\
9700.7080078125	2.18391903790374e-05	\\
9701.68678977273	2.26200098050803e-05	\\
9702.66557173295	2.17887193938153e-05	\\
9703.64435369318	2.23811472096713e-05	\\
9704.62313565341	2.15300808440693e-05	\\
9705.60191761364	2.12662911289586e-05	\\
9706.58069957386	2.18346079734638e-05	\\
9707.55948153409	2.18622825187707e-05	\\
9708.53826349432	2.18121003247414e-05	\\
9709.51704545454	2.18423640664399e-05	\\
9710.49582741477	2.07503947557112e-05	\\
9711.474609375	2.25500606051007e-05	\\
9712.45339133523	2.11608014660586e-05	\\
9713.43217329545	2.21257038342402e-05	\\
9714.41095525568	2.17612328659217e-05	\\
9715.38973721591	2.16550398002587e-05	\\
9716.36851917614	2.11503088632974e-05	\\
9717.34730113636	2.21452575979976e-05	\\
9718.32608309659	2.09898371790416e-05	\\
9719.30486505682	2.13245121948983e-05	\\
9720.28364701704	2.15538374212885e-05	\\
9721.26242897727	2.1829484535496e-05	\\
9722.2412109375	2.14492732307449e-05	\\
9723.21999289773	1.98176941678755e-05	\\
9724.19877485795	2.09424486028494e-05	\\
9725.17755681818	2.0682344712969e-05	\\
9726.15633877841	2.02307980980699e-05	\\
9727.13512073864	2.06933564456643e-05	\\
9728.11390269886	2.05238900596701e-05	\\
9729.09268465909	2.04222263582512e-05	\\
9730.07146661932	2.06004788823418e-05	\\
9731.05024857954	2.03973847872182e-05	\\
9732.02903053977	2.01115776124184e-05	\\
9733.0078125	2.12730408657971e-05	\\
9733.98659446023	1.98338331553409e-05	\\
9734.96537642045	2.03691426470055e-05	\\
9735.94415838068	2.04866008499555e-05	\\
9736.92294034091	2.0138364263842e-05	\\
9737.90172230114	2.08948375577344e-05	\\
9738.88050426136	2.07647317876626e-05	\\
9739.85928622159	2.11783012385285e-05	\\
9740.83806818182	2.08170607261564e-05	\\
9741.81685014204	2.06884354334319e-05	\\
9742.79563210227	2.08414998501967e-05	\\
9743.7744140625	2.01755675347796e-05	\\
9744.75319602273	2.07360319703046e-05	\\
9745.73197798295	2.04959608401467e-05	\\
9746.71075994318	2.02179851663297e-05	\\
9747.68954190341	2.14075348789091e-05	\\
9748.66832386364	2.16567923114378e-05	\\
9749.64710582386	2.14883127250177e-05	\\
9750.62588778409	2.07224052654559e-05	\\
9751.60466974432	2.12636098810598e-05	\\
9752.58345170454	2.05779970920583e-05	\\
9753.56223366477	2.10918564128951e-05	\\
9754.541015625	2.08716671807733e-05	\\
9755.51979758523	2.16466998371075e-05	\\
9756.49857954545	2.10333113507035e-05	\\
9757.47736150568	2.11404913618209e-05	\\
9758.45614346591	1.99486827181267e-05	\\
9759.43492542614	2.18272301602346e-05	\\
9760.41370738636	2.11738322042269e-05	\\
9761.39248934659	2.03069234397289e-05	\\
9762.37127130682	2.14543168169778e-05	\\
9763.35005326704	2.06034375608464e-05	\\
9764.32883522727	1.99706418675624e-05	\\
9765.3076171875	2.09308815397914e-05	\\
9766.28639914773	2.14271214067903e-05	\\
9767.26518110795	2.01324521860272e-05	\\
9768.24396306818	2.08562526324299e-05	\\
9769.22274502841	2.20282266342179e-05	\\
9770.20152698864	2.09577625144867e-05	\\
9771.18030894886	2.14757200245543e-05	\\
9772.15909090909	2.16154395349505e-05	\\
9773.13787286932	2.15391146016345e-05	\\
9774.11665482954	2.15021869898255e-05	\\
9775.09543678977	2.17706973853519e-05	\\
9776.07421875	2.16292537364567e-05	\\
9777.05300071023	2.19161999969575e-05	\\
9778.03178267045	2.20120945983174e-05	\\
9779.01056463068	2.19882389489211e-05	\\
9779.98934659091	2.2313768447108e-05	\\
9780.96812855114	2.12718536472946e-05	\\
9781.94691051136	2.20783486521112e-05	\\
9782.92569247159	2.1864279274403e-05	\\
9783.90447443182	2.17144604011935e-05	\\
9784.88325639204	2.123347068172e-05	\\
9785.86203835227	2.198710150365e-05	\\
9786.8408203125	2.02830630295998e-05	\\
9787.81960227273	2.14217404931243e-05	\\
9788.79838423295	2.07755501816432e-05	\\
9789.77716619318	2.19566795357002e-05	\\
9790.75594815341	2.19164973395065e-05	\\
9791.73473011364	2.12200870367945e-05	\\
9792.71351207386	2.11933805372347e-05	\\
9793.69229403409	2.21210649681786e-05	\\
9794.67107599432	2.20235828235772e-05	\\
9795.64985795454	2.22456769723283e-05	\\
9796.62863991477	2.21476581456184e-05	\\
9797.607421875	2.09022420111291e-05	\\
9798.58620383523	2.17640134286948e-05	\\
9799.56498579545	2.21402972681199e-05	\\
9800.54376775568	2.19454024631263e-05	\\
9801.52254971591	2.22036933300186e-05	\\
9802.50133167614	2.24210508456683e-05	\\
9803.48011363636	2.12029590394768e-05	\\
9804.45889559659	2.19703299649424e-05	\\
9805.43767755682	2.28022199751815e-05	\\
9806.41645951704	2.14890348006999e-05	\\
9807.39524147727	2.26368609110678e-05	\\
9808.3740234375	2.25130316055423e-05	\\
9809.35280539773	2.11592208754374e-05	\\
9810.33158735795	2.19152122709835e-05	\\
9811.31036931818	2.245155415937e-05	\\
9812.28915127841	2.16717415524964e-05	\\
9813.26793323864	2.20670286459101e-05	\\
9814.24671519886	2.19623394115493e-05	\\
9815.22549715909	2.19573090659852e-05	\\
9816.20427911932	2.06708542571057e-05	\\
9817.18306107954	2.13993727144839e-05	\\
9818.16184303977	2.15352432651309e-05	\\
9819.140625	2.21840947558122e-05	\\
9820.11940696023	2.22949897995038e-05	\\
9821.09818892045	2.16769327752271e-05	\\
9822.07697088068	2.14620097383474e-05	\\
9823.05575284091	2.20803649623622e-05	\\
9824.03453480114	2.24067735260409e-05	\\
9825.01331676136	2.10833265297163e-05	\\
9825.99209872159	2.13622003620064e-05	\\
9826.97088068182	2.18230708610511e-05	\\
9827.94966264204	2.13356980415258e-05	\\
9828.92844460227	2.17224908902564e-05	\\
9829.9072265625	2.14360840841445e-05	\\
9830.88600852273	2.20184590979192e-05	\\
9831.86479048295	2.19408664063375e-05	\\
9832.84357244318	2.13313033660891e-05	\\
9833.82235440341	2.16536171200275e-05	\\
9834.80113636364	2.19695055571901e-05	\\
9835.77991832386	2.23542166704655e-05	\\
9836.75870028409	2.13657410467183e-05	\\
9837.73748224432	2.11647604924347e-05	\\
9838.71626420454	2.24323279243539e-05	\\
9839.69504616477	2.16085805368869e-05	\\
9840.673828125	2.16016968794647e-05	\\
9841.65261008523	2.16542453668359e-05	\\
9842.63139204545	2.22059988969335e-05	\\
9843.61017400568	2.0825815162134e-05	\\
9844.58895596591	2.14699359213526e-05	\\
9845.56773792614	2.10002809978624e-05	\\
9846.54651988636	2.16359509581608e-05	\\
9847.52530184659	2.08450533944091e-05	\\
9848.50408380682	2.18530930796068e-05	\\
9849.48286576704	2.1122268196044e-05	\\
9850.46164772727	2.12214978822028e-05	\\
9851.4404296875	2.1579273113943e-05	\\
9852.41921164773	2.09617333151336e-05	\\
9853.39799360795	2.06780299984858e-05	\\
9854.37677556818	2.13237433103418e-05	\\
9855.35555752841	2.11283425467911e-05	\\
9856.33433948864	2.10398497207404e-05	\\
9857.31312144886	2.16644898876326e-05	\\
9858.29190340909	2.06984166478386e-05	\\
9859.27068536932	2.13052143688031e-05	\\
9860.24946732954	2.17305780565628e-05	\\
9861.22824928977	2.07222510630849e-05	\\
9862.20703125	2.1389675493206e-05	\\
9863.18581321023	2.1783580632057e-05	\\
9864.16459517045	2.09479315381629e-05	\\
9865.14337713068	2.20004254691251e-05	\\
9866.12215909091	2.12807379480114e-05	\\
9867.10094105114	2.105044604063e-05	\\
9868.07972301136	2.06993231505165e-05	\\
9869.05850497159	2.1150932705555e-05	\\
9870.03728693182	2.10996451102609e-05	\\
9871.01606889204	2.08212228615386e-05	\\
9871.99485085227	2.09865621690497e-05	\\
9872.9736328125	2.07765818777065e-05	\\
9873.95241477273	2.10032036388787e-05	\\
9874.93119673295	2.07671646453672e-05	\\
9875.90997869318	1.99980156322153e-05	\\
9876.88876065341	2.13901660934717e-05	\\
9877.86754261364	2.21565744496231e-05	\\
9878.84632457386	2.06783225032989e-05	\\
9879.82510653409	2.06643016457008e-05	\\
9880.80388849432	2.15977467664153e-05	\\
9881.78267045454	2.09220355041507e-05	\\
9882.76145241477	2.08866091770886e-05	\\
9883.740234375	2.11625295568762e-05	\\
9884.71901633523	2.05802841643041e-05	\\
9885.69779829545	2.10320776360737e-05	\\
9886.67658025568	2.06696217533632e-05	\\
9887.65536221591	2.0698132641034e-05	\\
9888.63414417614	2.07397857385945e-05	\\
9889.61292613636	2.08639236745216e-05	\\
9890.59170809659	2.0938564258895e-05	\\
9891.57049005682	2.17798162451979e-05	\\
9892.54927201704	2.08816224504401e-05	\\
9893.52805397727	2.19093975682628e-05	\\
9894.5068359375	2.12827269958017e-05	\\
9895.48561789773	2.05114428924761e-05	\\
9896.46439985795	2.09330469623806e-05	\\
9897.44318181818	2.03870511317395e-05	\\
9898.42196377841	2.09328550640051e-05	\\
9899.40074573864	2.11399435592601e-05	\\
9900.37952769886	2.02524771515065e-05	\\
9901.35830965909	2.0909363239346e-05	\\
9902.33709161932	1.99925339857827e-05	\\
9903.31587357954	2.10204873979526e-05	\\
9904.29465553977	2.0547937433076e-05	\\
9905.2734375	2.05812304377902e-05	\\
9906.25221946023	2.06018907872431e-05	\\
9907.23100142045	2.12121090745795e-05	\\
9908.20978338068	2.081991159706e-05	\\
9909.18856534091	2.11333557389005e-05	\\
9910.16734730114	2.07862022582569e-05	\\
9911.14612926136	2.10226271608984e-05	\\
9912.12491122159	2.02163937999668e-05	\\
9913.10369318182	1.98482052443451e-05	\\
9914.08247514204	2.12068736196551e-05	\\
9915.06125710227	2.04447932729334e-05	\\
9916.0400390625	2.04880473442627e-05	\\
9917.01882102273	2.03950000588286e-05	\\
9917.99760298295	2.02614563005841e-05	\\
9918.97638494318	2.03850215947617e-05	\\
9919.95516690341	2.06217987714791e-05	\\
9920.93394886364	2.06316487904848e-05	\\
9921.91273082386	1.99317080098304e-05	\\
9922.89151278409	2.0122538694748e-05	\\
9923.87029474432	1.98101138242942e-05	\\
9924.84907670454	1.98706845384923e-05	\\
9925.82785866477	2.03479998080767e-05	\\
9926.806640625	2.00013325610047e-05	\\
9927.78542258523	2.07713248912759e-05	\\
9928.76420454545	2.04399404922179e-05	\\
9929.74298650568	2.03178882607722e-05	\\
9930.72176846591	2.01427678447659e-05	\\
9931.70055042614	2.05088128284017e-05	\\
9932.67933238636	1.97533472657017e-05	\\
9933.65811434659	1.96502282856558e-05	\\
9934.63689630682	2.03714831220094e-05	\\
9935.61567826704	1.96795029893367e-05	\\
9936.59446022727	2.06667200208652e-05	\\
9937.5732421875	1.96376866444512e-05	\\
9938.55202414773	1.88438811380399e-05	\\
9939.53080610795	1.94695350520704e-05	\\
9940.50958806818	1.91885357342897e-05	\\
9941.48837002841	1.96108114782102e-05	\\
9942.46715198864	1.91214317460171e-05	\\
9943.44593394886	1.89718378301483e-05	\\
9944.42471590909	1.9188340336151e-05	\\
9945.40349786932	1.94551643986849e-05	\\
9946.38227982954	1.84648314982678e-05	\\
9947.36106178977	2.01112534781233e-05	\\
9948.33984375	1.85471379305961e-05	\\
9949.31862571023	1.93188932885633e-05	\\
9950.29740767045	1.84880623248721e-05	\\
9951.27618963068	1.88971389187716e-05	\\
9952.25497159091	1.88035820398835e-05	\\
9953.23375355114	1.89840997225576e-05	\\
9954.21253551136	1.830405154231e-05	\\
9955.19131747159	1.8601415126851e-05	\\
9956.17009943182	1.85479457748753e-05	\\
9957.14888139204	1.7985066726758e-05	\\
9958.12766335227	1.76943893344136e-05	\\
9959.1064453125	1.82724117293394e-05	\\
9960.08522727273	1.84310087558304e-05	\\
9961.06400923295	1.7867326010221e-05	\\
9962.04279119318	1.86361232999114e-05	\\
9963.02157315341	1.70633209569953e-05	\\
9964.00035511364	1.85371742917336e-05	\\
9964.97913707386	1.91040428447777e-05	\\
9965.95791903409	1.84407316838541e-05	\\
9966.93670099432	1.79831561504202e-05	\\
9967.91548295454	1.83455526269232e-05	\\
9968.89426491477	1.74219917091669e-05	\\
9969.873046875	1.7984220386228e-05	\\
9970.85182883523	1.88916866976256e-05	\\
9971.83061079545	1.83458977587045e-05	\\
9972.80939275568	1.77073134857403e-05	\\
9973.78817471591	1.80846922644189e-05	\\
9974.76695667614	1.85010358793467e-05	\\
9975.74573863636	1.88756584278077e-05	\\
9976.72452059659	1.83186985704211e-05	\\
9977.70330255682	1.79391357184179e-05	\\
9978.68208451704	1.8110865505652e-05	\\
9979.66086647727	1.83291695555063e-05	\\
9980.6396484375	1.80140143031713e-05	\\
9981.61843039773	1.82091134337175e-05	\\
9982.59721235795	1.79361816983109e-05	\\
9983.57599431818	1.86208955624963e-05	\\
9984.55477627841	1.83280187226151e-05	\\
9985.53355823864	1.87029888112673e-05	\\
9986.51234019886	1.87542913701307e-05	\\
9987.49112215909	1.85342627670991e-05	\\
9988.46990411932	1.84221586377358e-05	\\
9989.44868607954	1.82536409074475e-05	\\
9990.42746803977	1.79798062302407e-05	\\
9991.40625	1.75522174559308e-05	\\
9992.38503196023	1.81350089116091e-05	\\
9993.36381392045	1.74480266120187e-05	\\
9994.34259588068	1.73868962934967e-05	\\
9995.32137784091	1.84094181965074e-05	\\
9996.30015980114	1.69073624835266e-05	\\
9997.27894176136	1.74624393176544e-05	\\
9998.25772372159	1.7835131188822e-05	\\
9999.23650568182	1.77136978583879e-05	\\
10000.215287642	1.86732146519278e-05	\\
10001.1940696023	1.68700017802051e-05	\\
10002.1728515625	1.68614999712778e-05	\\
10003.1516335227	1.68728783123379e-05	\\
10004.130415483	1.75749143938791e-05	\\
10005.1091974432	1.76496983963191e-05	\\
10006.0879794034	1.7279992464874e-05	\\
10007.0667613636	1.78986373920174e-05	\\
10008.0455433239	1.62318866097379e-05	\\
10009.0243252841	1.66430098194357e-05	\\
10010.0031072443	1.74640057389989e-05	\\
10010.9818892045	1.70482435051096e-05	\\
10011.9606711648	1.63787400540486e-05	\\
10012.939453125	1.7056650380036e-05	\\
10013.9182350852	1.67202100871583e-05	\\
10014.8970170455	1.70758891337129e-05	\\
10015.8757990057	1.72258474072073e-05	\\
10016.8545809659	1.66259496946608e-05	\\
10017.8333629261	1.62326930294292e-05	\\
10018.8121448864	1.58899680469786e-05	\\
10019.7909268466	1.68719468767727e-05	\\
10020.7697088068	1.60307693048183e-05	\\
10021.748490767	1.66180349956424e-05	\\
10022.7272727273	1.62227467097513e-05	\\
10023.7060546875	1.51649937491545e-05	\\
10024.6848366477	1.69904047896216e-05	\\
10025.663618608	1.56730656359784e-05	\\
10026.6424005682	1.65855331702925e-05	\\
10027.6211825284	1.56121408905657e-05	\\
10028.5999644886	1.59089822128767e-05	\\
10029.5787464489	1.57680945401531e-05	\\
10030.5575284091	1.65011211409544e-05	\\
10031.5363103693	1.62058826943663e-05	\\
10032.5150923295	1.60574221919623e-05	\\
10033.4938742898	1.59731165175838e-05	\\
10034.47265625	1.57076669432745e-05	\\
10035.4514382102	1.54392214041866e-05	\\
10036.4302201705	1.58744773556723e-05	\\
10037.4090021307	1.59419715273299e-05	\\
10038.3877840909	1.56090958758157e-05	\\
10039.3665660511	1.57460289103185e-05	\\
10040.3453480114	1.47897442958079e-05	\\
10041.3241299716	1.55847939607273e-05	\\
10042.3029119318	1.57712993521774e-05	\\
10043.281693892	1.5162651181535e-05	\\
10044.2604758523	1.61154436309655e-05	\\
10045.2392578125	1.5880968185983e-05	\\
10046.2180397727	1.5604467800817e-05	\\
10047.196821733	1.53489175626974e-05	\\
10048.1756036932	1.56958896998027e-05	\\
10049.1543856534	1.57388242363985e-05	\\
10050.1331676136	1.48215422365235e-05	\\
10051.1119495739	1.529557282465e-05	\\
10052.0907315341	1.55542518120699e-05	\\
10053.0695134943	1.41449320019758e-05	\\
10054.0482954545	1.48773077330158e-05	\\
10055.0270774148	1.51022202911963e-05	\\
10056.005859375	1.45177501947804e-05	\\
10056.9846413352	1.45185023205995e-05	\\
10057.9634232955	1.54180504138244e-05	\\
10058.9422052557	1.49836553552831e-05	\\
10059.9209872159	1.51025108844239e-05	\\
10060.8997691761	1.54062321537744e-05	\\
10061.8785511364	1.48341489991543e-05	\\
10062.8573330966	1.46014591863664e-05	\\
10063.8361150568	1.45696185803384e-05	\\
10064.814897017	1.47929813058554e-05	\\
10065.7936789773	1.4580307678511e-05	\\
10066.7724609375	1.48105223571335e-05	\\
10067.7512428977	1.39049511298868e-05	\\
10068.730024858	1.41357408190756e-05	\\
10069.7088068182	1.45081574279146e-05	\\
10070.6875887784	1.48472199639859e-05	\\
10071.6663707386	1.443912807473e-05	\\
10072.6451526989	1.50704666669842e-05	\\
10073.6239346591	1.41654144666296e-05	\\
10074.6027166193	1.44701128662032e-05	\\
10075.5814985795	1.48710909834442e-05	\\
10076.5602805398	1.40237263119824e-05	\\
10077.5390625	1.38281922416433e-05	\\
10078.5178444602	1.46353485563189e-05	\\
10079.4966264205	1.51860282915028e-05	\\
10080.4754083807	1.41819978557545e-05	\\
10081.4541903409	1.41202571093907e-05	\\
10082.4329723011	1.43078992854671e-05	\\
10083.4117542614	1.51802045604795e-05	\\
10084.3905362216	1.46870499984778e-05	\\
10085.3693181818	1.4761617157965e-05	\\
10086.348100142	1.50309023119785e-05	\\
10087.3268821023	1.4745110121222e-05	\\
10088.3056640625	1.46519690482458e-05	\\
10089.2844460227	1.38162530454628e-05	\\
10090.263227983	1.46353858895647e-05	\\
10091.2420099432	1.50990330727473e-05	\\
10092.2207919034	1.47754251058993e-05	\\
10093.1995738636	1.37942933190616e-05	\\
10094.1783558239	1.47724281929201e-05	\\
10095.1571377841	1.42035084255079e-05	\\
10096.1359197443	1.47436135598397e-05	\\
10097.1147017045	1.47803157902011e-05	\\
10098.0934836648	1.3924439199207e-05	\\
10099.072265625	1.51665018761318e-05	\\
10100.0510475852	1.41523157037717e-05	\\
10101.0298295455	1.41416244113017e-05	\\
10102.0086115057	1.41948995822151e-05	\\
10102.9873934659	1.40862080282219e-05	\\
10103.9661754261	1.42555608546639e-05	\\
10104.9449573864	1.37938225095544e-05	\\
10105.9237393466	1.4365888401237e-05	\\
10106.9025213068	1.36140860010357e-05	\\
10107.881303267	1.43434913247985e-05	\\
10108.8600852273	1.40330893229339e-05	\\
10109.8388671875	1.42551928109171e-05	\\
10110.8176491477	1.41447781047017e-05	\\
10111.796431108	1.42114969983083e-05	\\
10112.7752130682	1.41583950857462e-05	\\
10113.7539950284	1.37687797024125e-05	\\
10114.7327769886	1.37803316758823e-05	\\
10115.7115589489	1.37020309955332e-05	\\
10116.6903409091	1.37365163513968e-05	\\
10117.6691228693	1.37063376699997e-05	\\
10118.6479048295	1.37618440759962e-05	\\
10119.6266867898	1.40875850641632e-05	\\
10120.60546875	1.39847739394483e-05	\\
10121.5842507102	1.35533837920204e-05	\\
10122.5630326705	1.40961382023313e-05	\\
10123.5418146307	1.36739280105998e-05	\\
10124.5205965909	1.36529832037913e-05	\\
10125.4993785511	1.41673560005336e-05	\\
10126.4781605114	1.39487647788693e-05	\\
10127.4569424716	1.41319999556146e-05	\\
10128.4357244318	1.31244011124427e-05	\\
10129.414506392	1.36096039213493e-05	\\
10130.3932883523	1.3999836951327e-05	\\
10131.3720703125	1.38452288645003e-05	\\
10132.3508522727	1.39707644142084e-05	\\
10133.329634233	1.37661188442086e-05	\\
10134.3084161932	1.31036580250842e-05	\\
10135.2871981534	1.40187161156002e-05	\\
10136.2659801136	1.30922794522568e-05	\\
10137.2447620739	1.31066791202543e-05	\\
10138.2235440341	1.38774029338638e-05	\\
10139.2023259943	1.3303655255281e-05	\\
10140.1811079545	1.34997777931148e-05	\\
10141.1598899148	1.34943904842663e-05	\\
10142.138671875	1.31504706235358e-05	\\
10143.1174538352	1.31731145469699e-05	\\
10144.0962357955	1.36966502555313e-05	\\
10145.0750177557	1.38365425960621e-05	\\
10146.0537997159	1.35112160272148e-05	\\
10147.0325816761	1.36257457890315e-05	\\
10148.0113636364	1.35744045889115e-05	\\
10148.9901455966	1.42773518262772e-05	\\
10149.9689275568	1.25547113166547e-05	\\
10150.947709517	1.36788717639495e-05	\\
10151.9264914773	1.29774807454022e-05	\\
10152.9052734375	1.3111397106613e-05	\\
10153.8840553977	1.30549602340571e-05	\\
10154.862837358	1.30728181043495e-05	\\
10155.8416193182	1.25287589487115e-05	\\
10156.8204012784	1.27756005457968e-05	\\
10157.7991832386	1.31429876776993e-05	\\
10158.7779651989	1.23940479053542e-05	\\
10159.7567471591	1.32961522901601e-05	\\
10160.7355291193	1.25486652036221e-05	\\
10161.7143110795	1.30041558358793e-05	\\
10162.6930930398	1.33434994703649e-05	\\
10163.671875	1.27272064135534e-05	\\
10164.6506569602	1.30032779987759e-05	\\
10165.6294389205	1.23975391066212e-05	\\
10166.6082208807	1.34747915570264e-05	\\
10167.5870028409	1.2315190195357e-05	\\
10168.5657848011	1.24402577186363e-05	\\
10169.5445667614	1.27863475850986e-05	\\
10170.5233487216	1.24213212712571e-05	\\
10171.5021306818	1.29959402330967e-05	\\
10172.480912642	1.25654998919452e-05	\\
10173.4596946023	1.2406673427831e-05	\\
10174.4384765625	1.25359291381852e-05	\\
10175.4172585227	1.23230084718352e-05	\\
10176.396040483	1.25156230493888e-05	\\
10177.3748224432	1.31438717083883e-05	\\
10178.3536044034	1.28733977113098e-05	\\
10179.3323863636	1.28727034997573e-05	\\
10180.3111683239	1.2562337114596e-05	\\
10181.2899502841	1.29199366041664e-05	\\
10182.2687322443	1.24324997298518e-05	\\
10183.2475142045	1.2744663393249e-05	\\
10184.2262961648	1.34018413163041e-05	\\
10185.205078125	1.28814091428533e-05	\\
10186.1838600852	1.29464660863429e-05	\\
10187.1626420455	1.25711606349861e-05	\\
10188.1414240057	1.33348678454198e-05	\\
10189.1202059659	1.22873315619548e-05	\\
10190.0989879261	1.27559900893912e-05	\\
10191.0777698864	1.27948478407751e-05	\\
10192.0565518466	1.2967614102084e-05	\\
10193.0353338068	1.27063478638599e-05	\\
10194.014115767	1.39515517087361e-05	\\
10194.9928977273	1.3339668563158e-05	\\
10195.9716796875	1.32246544893484e-05	\\
10196.9504616477	1.297502025241e-05	\\
10197.929243608	1.31079837974395e-05	\\
10198.9080255682	1.31353782013768e-05	\\
10199.8868075284	1.29927476627234e-05	\\
10200.8655894886	1.27480453830805e-05	\\
10201.8443714489	1.34167587021743e-05	\\
10202.8231534091	1.28445538533715e-05	\\
10203.8019353693	1.26301812881617e-05	\\
10204.7807173295	1.31977795023386e-05	\\
10205.7594992898	1.27901158514748e-05	\\
10206.73828125	1.2831683401002e-05	\\
10207.7170632102	1.32575489813974e-05	\\
10208.6958451705	1.32516906728414e-05	\\
10209.6746271307	1.28618453414262e-05	\\
10210.6534090909	1.32362423036859e-05	\\
10211.6321910511	1.29548311654057e-05	\\
10212.6109730114	1.2756515069029e-05	\\
10213.5897549716	1.32125505329836e-05	\\
10214.5685369318	1.30764802482763e-05	\\
10215.547318892	1.23378340639791e-05	\\
10216.5261008523	1.28253779057611e-05	\\
10217.5048828125	1.31558655770378e-05	\\
10218.4836647727	1.29514484721789e-05	\\
10219.462446733	1.29979106661473e-05	\\
10220.4412286932	1.23369284847212e-05	\\
10221.4200106534	1.27678172325622e-05	\\
10222.3987926136	1.29722030016551e-05	\\
10223.3775745739	1.30034620082936e-05	\\
10224.3563565341	1.21511683521094e-05	\\
10225.3351384943	1.32636589991766e-05	\\
10226.3139204545	1.31858911077763e-05	\\
10227.2927024148	1.28800747764212e-05	\\
10228.271484375	1.28052334667037e-05	\\
10229.2502663352	1.2630389646305e-05	\\
10230.2290482955	1.24924915820283e-05	\\
10231.2078302557	1.2107497344315e-05	\\
10232.1866122159	1.29342550092676e-05	\\
10233.1653941761	1.31029645762499e-05	\\
10234.1441761364	1.25958820728064e-05	\\
10235.1229580966	1.2488139549674e-05	\\
10236.1017400568	1.19315411832624e-05	\\
10237.080522017	1.2336606842509e-05	\\
10238.0593039773	1.30045932283365e-05	\\
10239.0380859375	1.29767195286988e-05	\\
10240.0168678977	1.25153982896237e-05	\\
10240.995649858	1.26687446062693e-05	\\
10241.9744318182	1.21820821134727e-05	\\
10242.9532137784	1.25352721587568e-05	\\
10243.9319957386	1.29150069702955e-05	\\
10244.9107776989	1.2244590486234e-05	\\
10245.8895596591	1.16483255670905e-05	\\
10246.8683416193	1.25248206323386e-05	\\
10247.8471235795	1.22881484300588e-05	\\
10248.8259055398	1.20930293558362e-05	\\
10249.8046875	1.20192354845021e-05	\\
10250.7834694602	1.26139710373703e-05	\\
10251.7622514205	1.22005158395196e-05	\\
10252.7410333807	1.25852483772915e-05	\\
10253.7198153409	1.23541002879802e-05	\\
10254.6985973011	1.21137211591291e-05	\\
10255.6773792614	1.22134124301391e-05	\\
10256.6561612216	1.29749505360639e-05	\\
10257.6349431818	1.20064525055596e-05	\\
10258.613725142	1.2137576674415e-05	\\
10259.5925071023	1.20833075050205e-05	\\
10260.5712890625	1.17766310834384e-05	\\
10261.5500710227	1.15800918302727e-05	\\
10262.528852983	1.23160816074387e-05	\\
10263.5076349432	1.18157783163321e-05	\\
10264.4864169034	1.19282147990076e-05	\\
10265.4651988636	1.23816794538782e-05	\\
10266.4439808239	1.16661019077074e-05	\\
10267.4227627841	1.26097727761976e-05	\\
10268.4015447443	1.26463428018956e-05	\\
10269.3803267045	1.20794751970399e-05	\\
10270.3591086648	1.14488714503527e-05	\\
10271.337890625	1.25299845495706e-05	\\
10272.3166725852	1.20722993033189e-05	\\
10273.2954545455	1.14533399460639e-05	\\
10274.2742365057	1.21501488200242e-05	\\
10275.2530184659	1.15991361558853e-05	\\
10276.2318004261	1.22993087658506e-05	\\
10277.2105823864	1.20062267009582e-05	\\
10278.1893643466	1.20843553530813e-05	\\
10279.1681463068	1.23696739481273e-05	\\
10280.146928267	1.17500110211212e-05	\\
10281.1257102273	1.16683940746929e-05	\\
10282.1044921875	1.20308894897384e-05	\\
10283.0832741477	1.18668241089988e-05	\\
10284.062056108	1.2453482467511e-05	\\
10285.0408380682	1.27176708145899e-05	\\
10286.0196200284	1.24528417381662e-05	\\
10286.9984019886	1.17204773012758e-05	\\
10287.9771839489	1.25104822259204e-05	\\
10288.9559659091	1.25710365697721e-05	\\
10289.9347478693	1.22713262846786e-05	\\
10290.9135298295	1.17483521747615e-05	\\
10291.8923117898	1.25211103751028e-05	\\
10292.87109375	1.31275546286303e-05	\\
10293.8498757102	1.26807038081782e-05	\\
10294.8286576705	1.21507598714855e-05	\\
10295.8074396307	1.24526951325648e-05	\\
10296.7862215909	1.23438509274852e-05	\\
10297.7650035511	1.30797988738105e-05	\\
10298.7437855114	1.29180010933557e-05	\\
10299.7225674716	1.21146142392856e-05	\\
10300.7013494318	1.24977680372393e-05	\\
10301.680131392	1.210893123741e-05	\\
10302.6589133523	1.28069803271673e-05	\\
10303.6376953125	1.21983554411129e-05	\\
10304.6164772727	1.26934657247936e-05	\\
10305.595259233	1.22382422959256e-05	\\
10306.5740411932	1.23133812961969e-05	\\
10307.5528231534	1.21113587564403e-05	\\
10308.5316051136	1.25785766056848e-05	\\
10309.5103870739	1.17703995827243e-05	\\
10310.4891690341	1.25492950945416e-05	\\
10311.4679509943	1.29711570607895e-05	\\
10312.4467329545	1.25715717133037e-05	\\
10313.4255149148	1.25991214857993e-05	\\
10314.404296875	1.28076568742122e-05	\\
10315.3830788352	1.22060514286334e-05	\\
10316.3618607955	1.21293499206061e-05	\\
10317.3406427557	1.20314611591976e-05	\\
10318.3194247159	1.17175799294543e-05	\\
10319.2982066761	1.22805422033886e-05	\\
10320.2769886364	1.22419349889294e-05	\\
10321.2557705966	1.22528097917149e-05	\\
10322.2345525568	1.26783189947253e-05	\\
10323.213334517	1.28581429683001e-05	\\
10324.1921164773	1.24532649816229e-05	\\
10325.1708984375	1.24733713628895e-05	\\
10326.1496803977	1.16783857670286e-05	\\
10327.128462358	1.20906823430286e-05	\\
10328.1072443182	1.23508023502546e-05	\\
10329.0860262784	1.25721272006879e-05	\\
10330.0648082386	1.17801907681501e-05	\\
10331.0435901989	1.2801044474929e-05	\\
10332.0223721591	1.17940697408959e-05	\\
10333.0011541193	1.27803980995204e-05	\\
10333.9799360795	1.28685570376052e-05	\\
10334.9587180398	1.24410013755623e-05	\\
10335.9375	1.16336733899614e-05	\\
10336.9162819602	1.25309215877505e-05	\\
10337.8950639205	1.17687137904887e-05	\\
10338.8738458807	1.17688052390373e-05	\\
10339.8526278409	1.22101356416215e-05	\\
10340.8314098011	1.20586676094673e-05	\\
10341.8101917614	1.18972327710048e-05	\\
10342.7889737216	1.18806778479926e-05	\\
10343.7677556818	1.22888474598902e-05	\\
10344.746537642	1.15919416229959e-05	\\
10345.7253196023	1.21015770825148e-05	\\
10346.7041015625	1.23534132083824e-05	\\
10347.6828835227	1.20031780211388e-05	\\
10348.661665483	1.19450311951405e-05	\\
10349.6404474432	1.22621029771466e-05	\\
10350.6192294034	1.1413514761173e-05	\\
10351.5980113636	1.25338220538357e-05	\\
10352.5767933239	1.25235612718482e-05	\\
10353.5555752841	1.20195965486561e-05	\\
10354.5343572443	1.17586949079233e-05	\\
10355.5131392045	1.12903969299839e-05	\\
10356.4919211648	1.13627811421646e-05	\\
10357.470703125	1.21943366481318e-05	\\
10358.4494850852	1.23347795016463e-05	\\
10359.4282670455	1.15330810674234e-05	\\
10360.4070490057	1.19012130065962e-05	\\
10361.3858309659	1.16841960171119e-05	\\
10362.3646129261	1.17816205171565e-05	\\
10363.3433948864	1.12375428234562e-05	\\
10364.3221768466	1.2108026933327e-05	\\
10365.3009588068	1.15239863440468e-05	\\
10366.279740767	1.18797086110581e-05	\\
10367.2585227273	1.17236300152724e-05	\\
10368.2373046875	1.16988446010744e-05	\\
10369.2160866477	1.19891856918392e-05	\\
10370.194868608	1.21948791516964e-05	\\
10371.1736505682	1.18836577224531e-05	\\
10372.1524325284	1.14761419474945e-05	\\
10373.1312144886	1.18964529274419e-05	\\
10374.1099964489	1.16191169769851e-05	\\
10375.0887784091	1.18434186549674e-05	\\
10376.0675603693	1.18111428444453e-05	\\
10377.0463423295	1.17446230039272e-05	\\
10378.0251242898	1.1631040370738e-05	\\
10379.00390625	1.13334970821555e-05	\\
10379.9826882102	1.1644305210715e-05	\\
10380.9614701705	1.19497765623624e-05	\\
10381.9402521307	1.17432163570244e-05	\\
10382.9190340909	1.16140105507793e-05	\\
10383.8978160511	1.12536048628771e-05	\\
10384.8765980114	1.22615774110443e-05	\\
10385.8553799716	1.22316152880604e-05	\\
10386.8341619318	1.20897135315975e-05	\\
10387.812943892	1.24871769135931e-05	\\
10388.7917258523	1.15620717908692e-05	\\
10389.7705078125	1.1559454086345e-05	\\
10390.7492897727	1.20938929526106e-05	\\
10391.728071733	1.19175456037312e-05	\\
10392.7068536932	1.19323710393329e-05	\\
10393.6856356534	1.18822776304452e-05	\\
10394.6644176136	1.19809267445182e-05	\\
10395.6431995739	1.12965783383479e-05	\\
10396.6219815341	1.12560623812492e-05	\\
10397.6007634943	1.18477825498338e-05	\\
10398.5795454545	1.14535208334477e-05	\\
10399.5583274148	1.1783384504399e-05	\\
10400.537109375	1.21561892581055e-05	\\
10401.5158913352	1.16443750957045e-05	\\
10402.4946732955	1.17174104665065e-05	\\
10403.4734552557	1.20273793030873e-05	\\
10404.4522372159	1.19169085788704e-05	\\
10405.4310191761	1.20624554659487e-05	\\
10406.4098011364	1.19787768734503e-05	\\
10407.3885830966	1.17679010451561e-05	\\
10408.3673650568	1.15567337053941e-05	\\
10409.346147017	1.20211810776277e-05	\\
10410.3249289773	1.22154437238101e-05	\\
10411.3037109375	1.17726062326906e-05	\\
10412.2824928977	1.15331230957855e-05	\\
10413.261274858	1.16274535985985e-05	\\
10414.2400568182	1.13616965410974e-05	\\
10415.2188387784	1.17798304515886e-05	\\
10416.1976207386	1.17496182069672e-05	\\
10417.1764026989	1.122184020614e-05	\\
10418.1551846591	1.16955286124488e-05	\\
10419.1339666193	1.1911765220755e-05	\\
10420.1127485795	1.14953763384189e-05	\\
10421.0915305398	1.15228812511831e-05	\\
10422.0703125	1.16699132312284e-05	\\
10423.0490944602	1.14137171108271e-05	\\
10424.0278764205	1.1070266964448e-05	\\
10425.0066583807	1.14042132731399e-05	\\
10425.9854403409	1.13897689938744e-05	\\
10426.9642223011	1.17961798682372e-05	\\
10427.9430042614	1.13329962264817e-05	\\
10428.9217862216	1.15280289057669e-05	\\
10429.9005681818	1.13735363670867e-05	\\
10430.879350142	1.09016756602469e-05	\\
10431.8581321023	1.11611799156858e-05	\\
10432.8369140625	1.11555431123815e-05	\\
10433.8156960227	1.09525702838778e-05	\\
10434.794477983	1.09328922418489e-05	\\
10435.7732599432	1.13105966947734e-05	\\
10436.7520419034	1.10136288986489e-05	\\
10437.7308238636	1.06212083169201e-05	\\
10438.7096058239	1.06388592427272e-05	\\
10439.6883877841	1.15257791626727e-05	\\
10440.6671697443	1.07594726604498e-05	\\
10441.6459517045	1.08199669111165e-05	\\
10442.6247336648	1.08726738496554e-05	\\
10443.603515625	1.06400892516534e-05	\\
10444.5822975852	1.12569729802973e-05	\\
10445.5610795455	1.09160737924237e-05	\\
10446.5398615057	1.09933867305614e-05	\\
10447.5186434659	1.10128879766399e-05	\\
10448.4974254261	1.15693889853149e-05	\\
10449.4762073864	1.05705917867783e-05	\\
10450.4549893466	1.1106468248542e-05	\\
10451.4337713068	1.07939099014948e-05	\\
10452.412553267	1.04332023848141e-05	\\
10453.3913352273	1.11292855797936e-05	\\
10454.3701171875	1.09000961263663e-05	\\
10455.3488991477	1.10685638170762e-05	\\
10456.327681108	1.05542036138836e-05	\\
10457.3064630682	1.09430987710077e-05	\\
10458.2852450284	1.0547953539419e-05	\\
10459.2640269886	1.04987800689233e-05	\\
10460.2428089489	1.08069910648451e-05	\\
10461.2215909091	1.08495154624482e-05	\\
10462.2003728693	1.08788151304888e-05	\\
10463.1791548295	1.03802920147841e-05	\\
10464.1579367898	1.10687526509096e-05	\\
10465.13671875	1.00903466169727e-05	\\
10466.1155007102	1.0728097020219e-05	\\
10467.0942826705	1.04312307344836e-05	\\
10468.0730646307	1.06205684226831e-05	\\
10469.0518465909	1.0668374438693e-05	\\
10470.0306285511	1.07806252081613e-05	\\
10471.0094105114	1.09687712178326e-05	\\
10471.9881924716	1.07617344261573e-05	\\
10472.9669744318	1.11968004015631e-05	\\
10473.945756392	1.11180196824164e-05	\\
10474.9245383523	1.10122336526776e-05	\\
10475.9033203125	1.07795344130803e-05	\\
10476.8821022727	1.07067270108079e-05	\\
10477.860884233	1.10075997810218e-05	\\
10478.8396661932	1.08724556916262e-05	\\
10479.8184481534	1.05158990872526e-05	\\
10480.7972301136	1.08198284486043e-05	\\
10481.7760120739	1.07345396792361e-05	\\
10482.7547940341	1.06687241972075e-05	\\
10483.7335759943	1.03556824305416e-05	\\
10484.7123579545	1.08009186337977e-05	\\
10485.6911399148	1.09531169975777e-05	\\
10486.669921875	1.09244466660592e-05	\\
10487.6487038352	1.06570273939083e-05	\\
10488.6274857955	1.07429571933989e-05	\\
10489.6062677557	1.03149347856523e-05	\\
10490.5850497159	1.06601216670546e-05	\\
10491.5638316761	1.07768751352427e-05	\\
10492.5426136364	1.05110023277649e-05	\\
10493.5213955966	1.08357982317388e-05	\\
10494.5001775568	1.06309031299456e-05	\\
10495.478959517	1.02737518708885e-05	\\
10496.4577414773	1.05837658517425e-05	\\
10497.4365234375	1.04268325709454e-05	\\
10498.4153053977	1.03188603212104e-05	\\
10499.394087358	1.08681211882812e-05	\\
10500.3728693182	1.0463044123281e-05	\\
10501.3516512784	1.00950986200624e-05	\\
10502.3304332386	1.08718590485334e-05	\\
10503.3092151989	9.8712069923943e-06	\\
10504.2879971591	1.02432934778253e-05	\\
10505.2667791193	1.06604321647201e-05	\\
10506.2455610795	1.05575207507866e-05	\\
10507.2243430398	1.05564121181387e-05	\\
10508.203125	1.04993957921769e-05	\\
10509.1819069602	1.04189160981753e-05	\\
10510.1606889205	1.06934189415943e-05	\\
10511.1394708807	1.02933987185442e-05	\\
10512.1182528409	1.04294590458676e-05	\\
10513.0970348011	1.05073721822273e-05	\\
10514.0758167614	1.03365776185075e-05	\\
10515.0545987216	1.04086780551636e-05	\\
10516.0333806818	1.00192549729404e-05	\\
10517.012162642	1.01109264014372e-05	\\
10517.9909446023	1.06875464713007e-05	\\
10518.9697265625	1.01706979670687e-05	\\
10519.9485085227	1.0363402931e-05	\\
10520.927290483	1.05294979072162e-05	\\
10521.9060724432	1.01438540506297e-05	\\
10522.8848544034	1.00564788144558e-05	\\
10523.8636363636	1.00702583816584e-05	\\
10524.8424183239	1.00693388545886e-05	\\
10525.8212002841	1.02728973928074e-05	\\
10526.7999822443	1.0229275879805e-05	\\
10527.7787642045	9.67698253054458e-06	\\
10528.7575461648	9.97615485315229e-06	\\
10529.736328125	1.07743428573982e-05	\\
10530.7151100852	9.8289503446951e-06	\\
10531.6938920455	1.00491328651413e-05	\\
10532.6726740057	1.0114611697105e-05	\\
10533.6514559659	9.84733853903062e-06	\\
10534.6302379261	1.00692431172221e-05	\\
10535.6090198864	9.72956332928611e-06	\\
10536.5878018466	9.78455670212192e-06	\\
10537.5665838068	9.89724325760307e-06	\\
10538.545365767	9.85284848474659e-06	\\
10539.5241477273	9.80533451201369e-06	\\
10540.5029296875	9.7292625510428e-06	\\
10541.4817116477	1.00425273967553e-05	\\
10542.460493608	9.53753399052385e-06	\\
10543.4392755682	9.65358416082333e-06	\\
10544.4180575284	9.45760664838856e-06	\\
10545.3968394886	9.71286430319652e-06	\\
10546.3756214489	9.52247610535405e-06	\\
10547.3544034091	9.69382394206989e-06	\\
10548.3331853693	9.61896480462349e-06	\\
10549.3119673295	9.32450685363611e-06	\\
10550.2907492898	9.97349051631921e-06	\\
10551.26953125	9.44238234107459e-06	\\
10552.2483132102	9.15917697919693e-06	\\
10553.2270951705	1.01849029475985e-05	\\
10554.2058771307	9.28529645630699e-06	\\
10555.1846590909	9.40125544927092e-06	\\
10556.1634410511	9.47202345087805e-06	\\
10557.1422230114	9.66301730226409e-06	\\
10558.1210049716	8.91638116745932e-06	\\
10559.0997869318	9.74808429183828e-06	\\
10560.078568892	9.78193468404961e-06	\\
10561.0573508523	8.96849585079148e-06	\\
10562.0361328125	9.10127819950984e-06	\\
10563.0149147727	9.20206870693916e-06	\\
10563.993696733	9.39516983521783e-06	\\
10564.9724786932	8.75013397650219e-06	\\
10565.9512606534	9.22069708087197e-06	\\
10566.9300426136	9.69744685367984e-06	\\
10567.9088245739	9.5394172963341e-06	\\
10568.8876065341	8.91957746928775e-06	\\
10569.8663884943	9.01572318483008e-06	\\
10570.8451704545	9.30485387726841e-06	\\
10571.8239524148	9.13043469610524e-06	\\
10572.802734375	9.35664157614143e-06	\\
10573.7815163352	9.25062131673879e-06	\\
10574.7602982955	8.98964854385372e-06	\\
10575.7390802557	8.87154389792102e-06	\\
10576.7178622159	8.90980556295456e-06	\\
10577.6966441761	9.02805989834768e-06	\\
10578.6754261364	8.95288308399869e-06	\\
10579.6542080966	8.83222627954524e-06	\\
10580.6329900568	9.16546842780482e-06	\\
10581.611772017	8.64729680804831e-06	\\
10582.5905539773	8.73501777464851e-06	\\
10583.5693359375	8.47032087226592e-06	\\
10584.5481178977	8.44148975140561e-06	\\
10585.526899858	8.80293828433356e-06	\\
10586.5056818182	9.1413418014035e-06	\\
10587.4844637784	8.94523352019843e-06	\\
10588.4632457386	8.3067620214446e-06	\\
10589.4420276989	8.86583914600326e-06	\\
10590.4208096591	8.13184818527078e-06	\\
10591.3995916193	8.35815350448082e-06	\\
10592.3783735795	8.61146947854843e-06	\\
10593.3571555398	8.73420376519555e-06	\\
10594.3359375	8.59101194470485e-06	\\
10595.3147194602	8.74451100879593e-06	\\
10596.2935014205	8.81318279478588e-06	\\
10597.2722833807	8.18056108614709e-06	\\
10598.2510653409	8.60081367373513e-06	\\
10599.2298473011	8.49966121881191e-06	\\
10600.2086292614	8.16991114261248e-06	\\
10601.1874112216	8.19947941411362e-06	\\
10602.1661931818	8.58785612206035e-06	\\
10603.144975142	7.97302963846394e-06	\\
10604.1237571023	7.88864130045318e-06	\\
10605.1025390625	8.2294931187198e-06	\\
10606.0813210227	8.0943019682772e-06	\\
10607.060102983	8.2507796299074e-06	\\
10608.0388849432	8.18794657329525e-06	\\
10609.0176669034	8.06526860470229e-06	\\
10609.9964488636	8.10383931946787e-06	\\
10610.9752308239	8.32193139896164e-06	\\
10611.9540127841	8.13318357928412e-06	\\
10612.9327947443	8.25982512221757e-06	\\
10613.9115767045	8.16243687875135e-06	\\
10614.8903586648	7.99097811528905e-06	\\
10615.869140625	8.24928902099547e-06	\\
10616.8479225852	8.29524920560684e-06	\\
10617.8267045455	7.84527999929191e-06	\\
10618.8054865057	7.72626007895359e-06	\\
10619.7842684659	8.01295518370577e-06	\\
10620.7630504261	7.63889707351433e-06	\\
10621.7418323864	8.01761914905042e-06	\\
10622.7206143466	8.17745139522824e-06	\\
10623.6993963068	8.03151338588343e-06	\\
10624.678178267	8.2871272653104e-06	\\
10625.6569602273	8.06316754573778e-06	\\
10626.6357421875	7.74120351576938e-06	\\
10627.6145241477	7.61631913484319e-06	\\
10628.593306108	7.27929362765975e-06	\\
10629.5720880682	7.71175491481139e-06	\\
10630.5508700284	7.44259460008078e-06	\\
10631.5296519886	8.13867172614011e-06	\\
10632.5084339489	7.30895816647084e-06	\\
10633.4872159091	7.36925717928533e-06	\\
10634.4659978693	7.56172811296094e-06	\\
10635.4447798295	7.75721150053265e-06	\\
10636.4235617898	7.85240758333473e-06	\\
10637.40234375	8.29848398770053e-06	\\
10638.3811257102	7.0958624185158e-06	\\
10639.3599076705	7.59281655154384e-06	\\
10640.3386896307	7.50557373143814e-06	\\
10641.3174715909	7.26269463854202e-06	\\
10642.2962535511	7.224348708445e-06	\\
10643.2750355114	7.62728159316183e-06	\\
10644.2538174716	7.47262396728037e-06	\\
10645.2325994318	7.04673352837941e-06	\\
10646.211381392	7.53902736896785e-06	\\
10647.1901633523	7.31741732378156e-06	\\
10648.1689453125	6.91730656659066e-06	\\
10649.1477272727	7.12828777562948e-06	\\
10650.126509233	7.03143611989739e-06	\\
10651.1052911932	7.14719776160804e-06	\\
10652.0840731534	6.89710182094812e-06	\\
10653.0628551136	6.85247426012917e-06	\\
10654.0416370739	6.98671768437451e-06	\\
10655.0204190341	6.97176514928713e-06	\\
10655.9992009943	6.7883308434886e-06	\\
10656.9779829545	6.8314418520945e-06	\\
10657.9567649148	6.90581715109374e-06	\\
10658.935546875	6.86780133893817e-06	\\
10659.9143288352	6.89048387227229e-06	\\
10660.8931107955	6.79729977150273e-06	\\
10661.8718927557	6.95118183355358e-06	\\
10662.8506747159	6.91009296628048e-06	\\
10663.8294566761	6.77564067150146e-06	\\
10664.8082386364	6.69022309334174e-06	\\
10665.7870205966	6.36829983781823e-06	\\
10666.7658025568	6.6308899156188e-06	\\
10667.744584517	6.70536262118289e-06	\\
10668.7233664773	6.67100690076069e-06	\\
10669.7021484375	6.89029861235799e-06	\\
10670.6809303977	6.61109283645696e-06	\\
10671.659712358	6.32634082716688e-06	\\
10672.6384943182	6.51427301305962e-06	\\
10673.6172762784	6.90603729437691e-06	\\
10674.5960582386	6.3042832904196e-06	\\
10675.5748401989	6.66859074851715e-06	\\
10676.5536221591	6.45007147919105e-06	\\
10677.5324041193	6.18761977031525e-06	\\
10678.5111860795	6.14520370533327e-06	\\
10679.4899680398	6.13084508261726e-06	\\
10680.46875	6.23965518944774e-06	\\
10681.4475319602	6.2251312596696e-06	\\
10682.4263139205	6.43870966031705e-06	\\
10683.4050958807	6.14104622648781e-06	\\
10684.3838778409	6.1011994225268e-06	\\
10685.3626598011	6.40327578932222e-06	\\
10686.3414417614	6.18019281650538e-06	\\
10687.3202237216	6.42974027336798e-06	\\
10688.2990056818	6.12457684352637e-06	\\
10689.277787642	6.41053109522645e-06	\\
10690.2565696023	5.94467261290173e-06	\\
10691.2353515625	6.61013812928522e-06	\\
10692.2141335227	6.35600110651785e-06	\\
10693.192915483	5.80174651435698e-06	\\
10694.1716974432	6.08165967611417e-06	\\
10695.1504794034	5.8715974500436e-06	\\
10696.1292613636	6.1128767821633e-06	\\
10697.1080433239	5.54911754114668e-06	\\
10698.0868252841	5.85088020980738e-06	\\
10699.0656072443	6.10499672709073e-06	\\
10700.0443892045	5.84096688219434e-06	\\
10701.0231711648	5.66211812244797e-06	\\
10702.001953125	5.44945228184377e-06	\\
10702.9807350852	6.15614413588227e-06	\\
10703.9595170455	5.93217111586993e-06	\\
10704.9382990057	6.01676777153589e-06	\\
10705.9170809659	5.63533361262378e-06	\\
10706.8958629261	5.9761626147126e-06	\\
10707.8746448864	5.68446671141601e-06	\\
10708.8534268466	5.99439033416022e-06	\\
10709.8322088068	5.43027016192689e-06	\\
10710.810990767	5.35536057779727e-06	\\
10711.7897727273	5.85621304210749e-06	\\
10712.7685546875	5.8651257355078e-06	\\
10713.7473366477	4.92735613558923e-06	\\
10714.726118608	5.568648760771e-06	\\
10715.7049005682	5.99202883062202e-06	\\
10716.6836825284	5.32439118386068e-06	\\
10717.6624644886	5.53644972177549e-06	\\
10718.6412464489	5.49682606723234e-06	\\
10719.6200284091	5.09750479939941e-06	\\
10720.5988103693	5.43679099265435e-06	\\
10721.5775923295	4.62404216302716e-06	\\
10722.5563742898	5.0100449826666e-06	\\
10723.53515625	5.30625149363576e-06	\\
10724.5139382102	5.65316187861065e-06	\\
10725.4927201705	5.20803021579382e-06	\\
10726.4715021307	4.91356998230282e-06	\\
10727.4502840909	5.28962948120597e-06	\\
10728.4290660511	5.45568204958849e-06	\\
10729.4078480114	5.12579935120727e-06	\\
10730.3866299716	5.49619205135845e-06	\\
10731.3654119318	4.96547459168135e-06	\\
10732.344193892	4.93507788926118e-06	\\
10733.3229758523	5.23491396526659e-06	\\
10734.3017578125	5.32347200852666e-06	\\
10735.2805397727	5.04712202725649e-06	\\
10736.259321733	5.0660909074762e-06	\\
10737.2381036932	5.22424878528017e-06	\\
10738.2168856534	4.74068906030665e-06	\\
10739.1956676136	5.05705309891634e-06	\\
10740.1744495739	5.10811415904822e-06	\\
10741.1532315341	4.89990541841001e-06	\\
10742.1320134943	5.26878456430889e-06	\\
10743.1107954545	4.78842728873649e-06	\\
10744.0895774148	4.81344981040234e-06	\\
10745.068359375	4.52012245149831e-06	\\
10746.0471413352	4.85188930301574e-06	\\
10747.0259232955	4.76264758629347e-06	\\
10748.0047052557	5.03378539634621e-06	\\
10748.9834872159	4.65475129468182e-06	\\
10749.9622691761	4.71018706843395e-06	\\
10750.9410511364	4.39840821086577e-06	\\
10751.9198330966	4.21098363120258e-06	\\
10752.8986150568	4.60039242499624e-06	\\
10753.877397017	4.33438236708181e-06	\\
10754.8561789773	4.3122206982148e-06	\\
10755.8349609375	4.41655221359263e-06	\\
10756.8137428977	4.09332377365501e-06	\\
10757.792524858	4.60783737961488e-06	\\
10758.7713068182	4.49694506691966e-06	\\
10759.7500887784	4.5084948037498e-06	\\
10760.7288707386	4.22029644793975e-06	\\
10761.7076526989	4.08687464518967e-06	\\
10762.6864346591	4.45376380021523e-06	\\
10763.6652166193	4.23029934707069e-06	\\
10764.6439985795	4.33000819478834e-06	\\
10765.6227805398	4.41390166170745e-06	\\
10766.6015625	4.26161771471288e-06	\\
10767.5803444602	4.2276930743583e-06	\\
10768.5591264205	3.96062072773027e-06	\\
10769.5379083807	4.50030470990063e-06	\\
10770.5166903409	4.04244288521592e-06	\\
10771.4954723011	4.21089325852965e-06	\\
10772.4742542614	4.08694283616853e-06	\\
10773.4530362216	4.22097200090544e-06	\\
10774.4318181818	4.24805747491516e-06	\\
10775.410600142	3.9607544633629e-06	\\
10776.3893821023	3.97587950190883e-06	\\
10777.3681640625	4.02359814489343e-06	\\
10778.3469460227	4.14344500526704e-06	\\
10779.325727983	3.8071380806201e-06	\\
10780.3045099432	3.74938792933988e-06	\\
10781.2832919034	3.67091318497332e-06	\\
10782.2620738636	3.82810586254234e-06	\\
10783.2408558239	3.71211752860445e-06	\\
10784.2196377841	3.87113262356232e-06	\\
10785.1984197443	3.83734716394664e-06	\\
10786.1772017045	3.70106445629862e-06	\\
10787.1559836648	3.72273462343559e-06	\\
10788.134765625	3.96337626014168e-06	\\
10789.1135475852	3.49545913668947e-06	\\
10790.0923295455	4.00551462098454e-06	\\
10791.0711115057	3.93676458709164e-06	\\
10792.0498934659	3.42238293662287e-06	\\
10793.0286754261	3.79932324126002e-06	\\
10794.0074573864	3.45236789171559e-06	\\
10794.9862393466	3.54346637925441e-06	\\
10795.9650213068	3.49626471902129e-06	\\
10796.943803267	3.68230470733619e-06	\\
10797.9225852273	3.4864071088727e-06	\\
10798.9013671875	3.74806428381135e-06	\\
10799.8801491477	3.78404118225166e-06	\\
10800.858931108	3.4424283753892e-06	\\
10801.8377130682	3.69114070175614e-06	\\
10802.8164950284	3.35448181181204e-06	\\
10803.7952769886	3.35755419029888e-06	\\
10804.7740589489	3.34335158498983e-06	\\
10805.7528409091	3.37233145470102e-06	\\
10806.7316228693	3.52565945707758e-06	\\
10807.7104048295	3.30821342672633e-06	\\
10808.6891867898	3.18042892629526e-06	\\
10809.66796875	3.39042494144775e-06	\\
10810.6467507102	3.1160407430572e-06	\\
10811.6255326705	3.9835405661757e-06	\\
10812.6043146307	2.9699981133835e-06	\\
10813.5830965909	3.30329164323559e-06	\\
10814.5618785511	3.39450345778088e-06	\\
10815.5406605114	3.28803417709915e-06	\\
10816.5194424716	3.27424127371647e-06	\\
10817.4982244318	3.39801520262749e-06	\\
10818.477006392	2.91098376812843e-06	\\
10819.4557883523	2.97105418718509e-06	\\
10820.4345703125	3.42412996976772e-06	\\
10821.4133522727	3.3537720308482e-06	\\
10822.392134233	3.02389597320338e-06	\\
10823.3709161932	3.15141578059386e-06	\\
10824.3496981534	3.14154719485026e-06	\\
10825.3284801136	2.9428001023721e-06	\\
10826.3072620739	3.04157797236386e-06	\\
10827.2860440341	3.46399185623576e-06	\\
10828.2648259943	2.94984703310998e-06	\\
10829.2436079545	3.22636121769074e-06	\\
10830.2223899148	2.89691705851272e-06	\\
10831.201171875	3.23084123217212e-06	\\
10832.1799538352	3.08566911041868e-06	\\
10833.1587357955	3.12475942956101e-06	\\
10834.1375177557	2.59016342201973e-06	\\
10835.1162997159	2.98814653532696e-06	\\
10836.0950816761	2.86739689573502e-06	\\
10837.0738636364	2.87668455279798e-06	\\
10838.0526455966	2.75288749734356e-06	\\
10839.0314275568	3.13605996203518e-06	\\
10840.010209517	2.68735550794857e-06	\\
10840.9889914773	3.31481776704911e-06	\\
10841.9677734375	3.03227367259066e-06	\\
10842.9465553977	2.54096089670118e-06	\\
10843.925337358	2.96370038774889e-06	\\
10844.9041193182	2.75986258027584e-06	\\
10845.8829012784	2.60233857546815e-06	\\
10846.8616832386	2.8069356703971e-06	\\
10847.8404651989	3.26197838045239e-06	\\
10848.8192471591	2.47586743276488e-06	\\
10849.7980291193	2.8492573923661e-06	\\
10850.7768110795	3.00097592636873e-06	\\
10851.7555930398	2.53886177261601e-06	\\
10852.734375	2.30079381849078e-06	\\
10853.7131569602	2.87021291431276e-06	\\
10854.6919389205	2.57997828753537e-06	\\
10855.6707208807	2.4702224265477e-06	\\
10856.6495028409	2.80993246624519e-06	\\
10857.6282848011	2.34351218863027e-06	\\
10858.6070667614	2.82682117606697e-06	\\
10859.5858487216	2.39389136266654e-06	\\
10860.5646306818	2.17287915438652e-06	\\
10861.543412642	2.29634175467575e-06	\\
10862.5221946023	2.83226064718564e-06	\\
10863.5009765625	2.4232112311446e-06	\\
10864.4797585227	2.59622974818231e-06	\\
10865.458540483	2.9316325569936e-06	\\
10866.4373224432	2.60016714182786e-06	\\
10867.4161044034	2.35621950099172e-06	\\
10868.3948863636	2.67432805116116e-06	\\
10869.3736683239	2.45068626282319e-06	\\
10870.3524502841	2.58210015766809e-06	\\
10871.3312322443	2.71076669164146e-06	\\
10872.3100142045	2.19617013248456e-06	\\
10873.2887961648	2.43623164590937e-06	\\
10874.267578125	2.11849140415536e-06	\\
10875.2463600852	2.41532464167534e-06	\\
10876.2251420455	2.31544739735438e-06	\\
10877.2039240057	2.54289165250006e-06	\\
10878.1827059659	2.27436986390898e-06	\\
10879.1614879261	2.58053374591049e-06	\\
10880.1402698864	2.54399980881729e-06	\\
10881.1190518466	2.10206868562304e-06	\\
10882.0978338068	2.01635955448363e-06	\\
10883.076615767	2.62955766532934e-06	\\
10884.0553977273	2.27096765079753e-06	\\
10885.0341796875	2.33288945654614e-06	\\
10886.0129616477	2.45223880294543e-06	\\
10886.991743608	2.60322954257957e-06	\\
10887.9705255682	2.12006650087108e-06	\\
10888.9493075284	2.28703781117691e-06	\\
10889.9280894886	2.47225788022932e-06	\\
10890.9068714489	2.28940568707195e-06	\\
10891.8856534091	2.31475278799604e-06	\\
10892.8644353693	2.62671425690229e-06	\\
10893.8432173295	2.09969429699775e-06	\\
10894.8219992898	1.98939950328963e-06	\\
10895.80078125	2.86058104815507e-06	\\
10896.7795632102	2.30778864834228e-06	\\
10897.7583451705	2.20486568628228e-06	\\
10898.7371271307	2.27652648488914e-06	\\
10899.7159090909	2.3599574156114e-06	\\
10900.6946910511	2.07464685321381e-06	\\
10901.6734730114	2.4410632097992e-06	\\
10902.6522549716	2.47839187682531e-06	\\
10903.6310369318	2.06370820208236e-06	\\
10904.609818892	2.39036033522522e-06	\\
10905.5886008523	2.15613649214804e-06	\\
10906.5673828125	1.9333702605203e-06	\\
10907.5461647727	2.2260025327634e-06	\\
10908.524946733	2.30024586361995e-06	\\
10909.5037286932	1.80421995333844e-06	\\
10910.4825106534	2.16579471923e-06	\\
10911.4612926136	2.13391712652767e-06	\\
10912.4400745739	1.80508393732818e-06	\\
10913.4188565341	2.16257789310428e-06	\\
10914.3976384943	2.05463319107212e-06	\\
10915.3764204545	1.9102492482834e-06	\\
10916.3552024148	2.21569594991999e-06	\\
10917.333984375	2.14134194768153e-06	\\
10918.3127663352	1.94752029087696e-06	\\
10919.2915482955	2.0872314657289e-06	\\
10920.2703302557	2.03215568934546e-06	\\
10921.2491122159	1.63994186059649e-06	\\
10922.2278941761	2.05819355780554e-06	\\
10923.2066761364	1.9582038083747e-06	\\
10924.1854580966	2.06597491808937e-06	\\
10925.1642400568	2.24525574532931e-06	\\
10926.143022017	2.28235189331601e-06	\\
10927.1218039773	2.21190345707398e-06	\\
10928.1005859375	2.306304731284e-06	\\
10929.0793678977	2.27223671878636e-06	\\
10930.058149858	1.56183104795255e-06	\\
10931.0369318182	1.98828491217804e-06	\\
10932.0157137784	2.21943239321194e-06	\\
10932.9944957386	1.75858607911535e-06	\\
10933.9732776989	2.03746995584854e-06	\\
10934.9520596591	2.25413807224199e-06	\\
10935.9308416193	2.00202815879041e-06	\\
10936.9096235795	1.92985315591813e-06	\\
10937.8884055398	2.24538385184149e-06	\\
10938.8671875	1.7407511499229e-06	\\
10939.8459694602	1.68739263682308e-06	\\
10940.8247514205	1.84828224638361e-06	\\
10941.8035333807	1.62210891973172e-06	\\
10942.7823153409	1.88116161497047e-06	\\
10943.7610973011	2.08424342020929e-06	\\
10944.7398792614	1.81106330068532e-06	\\
10945.7186612216	1.78882887772069e-06	\\
10946.6974431818	1.73298175338544e-06	\\
10947.676225142	2.06586793272144e-06	\\
10948.6550071023	1.86210097592554e-06	\\
10949.6337890625	2.15174094747517e-06	\\
10950.6125710227	1.89135684768889e-06	\\
10951.591352983	1.89114354702634e-06	\\
10952.5701349432	1.62171576725986e-06	\\
10953.5489169034	1.85947443378774e-06	\\
10954.5276988636	1.56055899929925e-06	\\
10955.5064808239	1.91937910915826e-06	\\
10956.4852627841	1.74178279721673e-06	\\
10957.4640447443	1.54185055748742e-06	\\
10958.4428267045	1.98971747763407e-06	\\
10959.4216086648	1.97504292365553e-06	\\
10960.400390625	2.19736420031175e-06	\\
10961.3791725852	1.90386931742604e-06	\\
10962.3579545455	2.15925315707898e-06	\\
10963.3367365057	1.70379263513787e-06	\\
10964.3155184659	1.92203036019114e-06	\\
10965.2943004261	1.85837419370972e-06	\\
10966.2730823864	1.80819615464483e-06	\\
10967.2518643466	2.2745196271521e-06	\\
10968.2306463068	2.10344166883946e-06	\\
10969.209428267	1.65662381035829e-06	\\
10970.1882102273	2.05389619018803e-06	\\
10971.1669921875	1.71081824612898e-06	\\
10972.1457741477	1.84308122102719e-06	\\
10973.124556108	1.53322167915034e-06	\\
10974.1033380682	1.9913149610154e-06	\\
10975.0821200284	1.54432501578006e-06	\\
10976.0609019886	1.77025205770897e-06	\\
10977.0396839489	1.76061502785486e-06	\\
10978.0184659091	1.40575276862163e-06	\\
10978.9972478693	1.61684228772293e-06	\\
10979.9760298295	1.85463384936846e-06	\\
10980.9548117898	1.64721170196128e-06	\\
10981.93359375	1.81165125065219e-06	\\
10982.9123757102	2.04886129870701e-06	\\
10983.8911576705	1.64119002663926e-06	\\
10984.8699396307	1.61255084512511e-06	\\
10985.8487215909	1.9064153762883e-06	\\
10986.8275035511	1.67958635247944e-06	\\
10987.8062855114	1.6443897929015e-06	\\
10988.7850674716	1.88320346890536e-06	\\
10989.7638494318	1.51958230990639e-06	\\
10990.742631392	1.66482900257183e-06	\\
10991.7214133523	1.87637893320566e-06	\\
10992.7001953125	1.96702358414496e-06	\\
10993.6789772727	1.79675983101457e-06	\\
10994.657759233	1.90675974517731e-06	\\
10995.6365411932	1.89510167140853e-06	\\
10996.6153231534	1.73848204203463e-06	\\
10997.5941051136	1.78457242845061e-06	\\
10998.5728870739	1.82914511768868e-06	\\
10999.5516690341	1.41730280359168e-06	\\
11000.5304509943	2.08938071400516e-06	\\
11001.5092329545	1.90067934047073e-06	\\
11002.4880149148	1.5376531948979e-06	\\
11003.466796875	1.98745616774755e-06	\\
11004.4455788352	1.59224759307482e-06	\\
11005.4243607955	1.62906739284363e-06	\\
11006.4031427557	1.96463195856825e-06	\\
11007.3819247159	2.00992194388055e-06	\\
11008.3607066761	1.70728039191802e-06	\\
11009.3394886364	1.81310908346389e-06	\\
11010.3182705966	1.91997616873525e-06	\\
11011.2970525568	1.59889718473773e-06	\\
11012.275834517	1.82053735851266e-06	\\
11013.2546164773	1.53652900348306e-06	\\
11014.2333984375	1.70565269229972e-06	\\
11015.2121803977	1.58276010716064e-06	\\
11016.190962358	1.84915042714232e-06	\\
11017.1697443182	1.7674302516725e-06	\\
11018.1485262784	1.7302617260326e-06	\\
11019.1273082386	1.66564852609171e-06	\\
11020.1060901989	1.74846456587537e-06	\\
11021.0848721591	1.79237561779357e-06	\\
11022.0636541193	1.63914987802218e-06	\\
11023.0424360795	1.88549896637165e-06	\\
11024.0212180398	1.57919026723033e-06	\\
};
\end{axis}
\end{tikzpicture}%
	\caption{Impulse response at Fs\_TX = 22.050 Hz and Fs\_RX = 22.050 Hz.}
	\label{fig:response_1}
\end{figure}

\begin{figure}[H]
	\centering
	\setlength\figureheight{4cm}
    	\setlength\figurewidth{0.8\linewidth}
	% This file was created by matlab2tikz v0.4.6 running on MATLAB 8.2.
% Copyright (c) 2008--2014, Nico Schlömer <nico.schloemer@gmail.com>
% All rights reserved.
% Minimal pgfplots version: 1.3
% 
% The latest updates can be retrieved from
%   http://www.mathworks.com/matlabcentral/fileexchange/22022-matlab2tikz
% where you can also make suggestions and rate matlab2tikz.
% 
\begin{tikzpicture}

\begin{axis}[%
width=\figurewidth,
height=\figureheight,
scale only axis,
xmin=0,
xmax=1,
ymin=-1,
ymax=1,
name=plot1
]
\addplot [color=blue,solid,forget plot]
  table[row sep=crcr]{
0	0.005340576171875	\\
4.43911750344032e-05	0.0003662109375	\\
8.87823500688063e-05	-0.00067138671875	\\
0.000133173525103209	0.0003662109375	\\
0.000177564700137613	-0.000335693359375	\\
0.000221955875172016	-0.00030517578125	\\
0.000266347050206419	0.0003662109375	\\
0.000310738225240822	0.00048828125	\\
0.000355129400275225	0.000335693359375	\\
0.000399520575309628	0.000701904296875	\\
0.000443911750344032	0.00048828125	\\
0.000488302925378435	0.000946044921875	\\
0.000532694100412838	0.001220703125	\\
0.000577085275447241	0.001129150390625	\\
0.000621476450481644	0.001129150390625	\\
0.000665867625516047	0.0010986328125	\\
0.000710258800550451	0.001190185546875	\\
0.000754649975584854	0.001373291015625	\\
0.000799041150619257	0.0015869140625	\\
0.00084343232565366	0.001556396484375	\\
0.000887823500688063	0.001129150390625	\\
0.000932214675722466	0.001312255859375	\\
0.000976605850756869	0.001220703125	\\
0.00102099702579127	0.000885009765625	\\
0.00106538820082568	0.000946044921875	\\
0.00110977937586008	0.001129150390625	\\
0.00115417055089448	0.001190185546875	\\
0.00119856172592889	0.00091552734375	\\
0.00124295290096329	0.000701904296875	\\
0.00128734407599769	0.0009765625	\\
0.00133173525103209	0.001373291015625	\\
0.0013761264260665	0.000732421875	\\
0.0014205176011009	0.000152587890625	\\
0.0014649087761353	0.0003662109375	\\
0.00150929995116971	0	\\
0.00155369112620411	3.0517578125e-05	\\
0.00159808230123851	0.000518798828125	\\
0.00164247347627292	0.00079345703125	\\
0.00168686465130732	0.000885009765625	\\
0.00173125582634172	0.000701904296875	\\
0.00177564700137613	0.000762939453125	\\
0.00182003817641053	0.000885009765625	\\
0.00186442935144493	0.000885009765625	\\
0.00190882052647934	0.001312255859375	\\
0.00195321170151374	0.001434326171875	\\
0.00199760287654814	0.0010986328125	\\
0.00204199405158255	0.000823974609375	\\
0.00208638522661695	0.00091552734375	\\
0.00213077640165135	0.00091552734375	\\
0.00217516757668575	0.00091552734375	\\
0.00221955875172016	0.000946044921875	\\
0.00226394992675456	0.00091552734375	\\
0.00230834110178896	0.000762939453125	\\
0.00235273227682337	0.000274658203125	\\
0.00239712345185777	0.00054931640625	\\
0.00244151462689217	0.001007080078125	\\
0.00248590580192658	0.00079345703125	\\
0.00253029697696098	0.000579833984375	\\
0.00257468815199538	0.000518798828125	\\
0.00261907932702979	0.00030517578125	\\
0.00266347050206419	0.00018310546875	\\
0.00270786167709859	0.000335693359375	\\
0.002752252852133	0.0003662109375	\\
0.0027966440271674	-0.000152587890625	\\
0.0028410352022018	9.1552734375e-05	\\
0.00288542637723621	-0.000244140625	\\
0.00292981755227061	-0.0008544921875	\\
0.00297420872730501	-0.000457763671875	\\
0.00301859990233941	-0.000518798828125	\\
0.00306299107737382	-0.000579833984375	\\
0.00310738225240822	-0.0006103515625	\\
0.00315177342744262	-0.00115966796875	\\
0.00319616460247703	-0.0013427734375	\\
0.00324055577751143	-0.001220703125	\\
0.00328494695254583	-0.001373291015625	\\
0.00332933812758024	-0.001617431640625	\\
0.00337372930261464	-0.00152587890625	\\
0.00341812047764904	-0.001739501953125	\\
0.00346251165268345	-0.001800537109375	\\
0.00350690282771785	-0.001861572265625	\\
0.00355129400275225	-0.001708984375	\\
0.00359568517778666	-0.00152587890625	\\
0.00364007635282106	-0.00177001953125	\\
0.00368446752785546	-0.00177001953125	\\
0.00372885870288987	-0.001678466796875	\\
0.00377324987792427	-0.00201416015625	\\
0.00381764105295867	-0.002166748046875	\\
0.00386203222799308	-0.001861572265625	\\
0.00390642340302748	-0.002227783203125	\\
0.00395081457806188	-0.00225830078125	\\
0.00399520575309628	-0.002044677734375	\\
0.00403959692813069	-0.00213623046875	\\
0.00408398810316509	-0.0020751953125	\\
0.00412837927819949	-0.00213623046875	\\
0.0041727704532339	-0.0018310546875	\\
0.0042171616282683	-0.001617431640625	\\
0.0042615528033027	-0.001556396484375	\\
0.00430594397833711	-0.001617431640625	\\
0.00435033515337151	-0.00152587890625	\\
0.00439472632840591	-0.0009765625	\\
0.00443911750344032	-0.00140380859375	\\
0.00448350867847472	-0.001678466796875	\\
0.00452789985350912	-0.001434326171875	\\
0.00457229102854353	-0.0015869140625	\\
0.00461668220357793	-0.0015869140625	\\
0.00466107337861233	-0.001495361328125	\\
0.00470546455364673	-0.0015869140625	\\
0.00474985572868114	-0.00189208984375	\\
0.00479424690371554	-0.001708984375	\\
0.00483863807874994	-0.001556396484375	\\
0.00488302925378435	-0.0015869140625	\\
0.00492742042881875	-0.001617431640625	\\
0.00497181160385315	-0.001617431640625	\\
0.00501620277888756	-0.001495361328125	\\
0.00506059395392196	-0.00164794921875	\\
0.00510498512895636	-0.001861572265625	\\
0.00514937630399077	-0.001922607421875	\\
0.00519376747902517	-0.001861572265625	\\
0.00523815865405957	-0.001495361328125	\\
0.00528254982909398	-0.001739501953125	\\
0.00532694100412838	-0.0020751953125	\\
0.00537133217916278	-0.001861572265625	\\
0.00541572335419719	-0.001922607421875	\\
0.00546011452923159	-0.00189208984375	\\
0.00550450570426599	-0.001678466796875	\\
0.00554889687930039	-0.0015869140625	\\
0.0055932880543348	-0.001800537109375	\\
0.0056376792293692	-0.001861572265625	\\
0.0056820704044036	-0.001953125	\\
0.00572646157943801	-0.002532958984375	\\
0.00577085275447241	-0.002471923828125	\\
0.00581524392950681	-0.002227783203125	\\
0.00585963510454122	-0.00262451171875	\\
0.00590402627957562	-0.002899169921875	\\
0.00594841745461002	-0.00323486328125	\\
0.00599280862964443	-0.00311279296875	\\
0.00603719980467883	-0.003204345703125	\\
0.00608159097971323	-0.002960205078125	\\
0.00612598215474764	-0.00299072265625	\\
0.00617037332978204	-0.002838134765625	\\
0.00621476450481644	-0.00286865234375	\\
0.00625915567985085	-0.0030517578125	\\
0.00630354685488525	-0.00299072265625	\\
0.00634793802991965	-0.003326416015625	\\
0.00639232920495406	-0.003387451171875	\\
0.00643672037998846	-0.0028076171875	\\
0.00648111155502286	-0.0025634765625	\\
0.00652550273005726	-0.00250244140625	\\
0.00656989390509167	-0.002227783203125	\\
0.00661428508012607	-0.002197265625	\\
0.00665867625516047	-0.0025634765625	\\
0.00670306743019488	-0.002410888671875	\\
0.00674745860522928	-0.002716064453125	\\
0.00679184978026368	-0.003082275390625	\\
0.00683624095529809	-0.003021240234375	\\
0.00688063213033249	-0.002655029296875	\\
0.00692502330536689	-0.0023193359375	\\
0.0069694144804013	-0.00238037109375	\\
0.0070138056554357	-0.002777099609375	\\
0.0070581968304701	-0.00299072265625	\\
0.00710258800550451	-0.0028076171875	\\
0.00714697918053891	-0.002685546875	\\
0.00719137035557331	-0.00238037109375	\\
0.00723576153060772	-0.0018310546875	\\
0.00728015270564212	-0.002044677734375	\\
0.00732454388067652	-0.00213623046875	\\
0.00736893505571092	-0.002410888671875	\\
0.00741332623074533	-0.00225830078125	\\
0.00745771740577973	-0.002105712890625	\\
0.00750210858081413	-0.0020751953125	\\
0.00754649975584854	-0.00164794921875	\\
0.00759089093088294	-0.00164794921875	\\
0.00763528210591734	-0.001739501953125	\\
0.00767967328095175	-0.001708984375	\\
0.00772406445598615	-0.001983642578125	\\
0.00776845563102055	-0.002349853515625	\\
0.00781284680605496	-0.002227783203125	\\
0.00785723798108936	-0.00238037109375	\\
0.00790162915612376	-0.0023193359375	\\
0.00794602033115816	-0.002227783203125	\\
0.00799041150619257	-0.002227783203125	\\
0.00803480268122697	-0.002227783203125	\\
0.00807919385626137	-0.002105712890625	\\
0.00812358503129578	-0.002288818359375	\\
0.00816797620633018	-0.001953125	\\
0.00821236738136458	-0.001800537109375	\\
0.00825675855639899	-0.00213623046875	\\
0.00830114973143339	-0.00213623046875	\\
0.00834554090646779	-0.00189208984375	\\
0.0083899320815022	-0.00213623046875	\\
0.0084343232565366	-0.002349853515625	\\
0.008478714431571	-0.002044677734375	\\
0.00852310560660541	-0.00177001953125	\\
0.00856749678163981	-0.00164794921875	\\
0.00861188795667421	-0.00115966796875	\\
0.00865627913170862	-0.00079345703125	\\
0.00870067030674302	-0.000885009765625	\\
0.00874506148177742	-0.0010986328125	\\
0.00878945265681183	-0.00079345703125	\\
0.00883384383184623	-0.00042724609375	\\
0.00887823500688063	-0.000579833984375	\\
0.00892262618191503	-0.00042724609375	\\
0.00896701735694944	-0.0003662109375	\\
0.00901140853198384	-0.000457763671875	\\
0.00905579970701824	-0.00067138671875	\\
0.00910019088205265	-0.001007080078125	\\
0.00914458205708705	-0.001129150390625	\\
0.00918897323212145	-0.001007080078125	\\
0.00923336440715586	-0.00091552734375	\\
0.00927775558219026	-0.001007080078125	\\
0.00932214675722466	-0.00079345703125	\\
0.00936653793225907	-0.00103759765625	\\
0.00941092910729347	-0.000946044921875	\\
0.00945532028232787	-0.000823974609375	\\
0.00949971145736228	-0.001556396484375	\\
0.00954410263239668	-0.001556396484375	\\
0.00958849380743108	-0.0015869140625	\\
0.00963288498246549	-0.001800537109375	\\
0.00967727615749989	-0.001922607421875	\\
0.00972166733253429	-0.001678466796875	\\
0.00976605850756869	-0.00177001953125	\\
0.0098104496826031	-0.002227783203125	\\
0.0098548408576375	-0.001922607421875	\\
0.0098992320326719	-0.001953125	\\
0.00994362320770631	-0.002197265625	\\
0.00998801438274071	-0.001953125	\\
0.0100324055577751	-0.00146484375	\\
0.0100767967328095	-0.001617431640625	\\
0.0101211879078439	-0.0020751953125	\\
0.0101655790828783	-0.00213623046875	\\
0.0102099702579127	-0.0023193359375	\\
0.0102543614329471	-0.002197265625	\\
0.0102987526079815	-0.0020751953125	\\
0.0103431437830159	-0.00238037109375	\\
0.0103875349580503	-0.002410888671875	\\
0.0104319261330847	-0.0025634765625	\\
0.0104763173081191	-0.002349853515625	\\
0.0105207084831535	-0.00213623046875	\\
0.010565099658188	-0.002288818359375	\\
0.0106094908332224	-0.0020751953125	\\
0.0106538820082568	-0.001953125	\\
0.0106982731832912	-0.002044677734375	\\
0.0107426643583256	-0.001983642578125	\\
0.01078705553336	-0.0018310546875	\\
0.0108314467083944	-0.001800537109375	\\
0.0108758378834288	-0.0020751953125	\\
0.0109202290584632	-0.00201416015625	\\
0.0109646202334976	-0.0023193359375	\\
0.011009011408532	-0.001953125	\\
0.0110534025835664	-0.001678466796875	\\
0.0110977937586008	-0.0015869140625	\\
0.0111421849336352	-0.001556396484375	\\
0.0111865761086696	-0.002044677734375	\\
0.011230967283704	-0.001922607421875	\\
0.0112753584587384	-0.00189208984375	\\
0.0113197496337728	-0.001953125	\\
0.0113641408088072	-0.0018310546875	\\
0.0114085319838416	-0.002197265625	\\
0.011452923158876	-0.002197265625	\\
0.0114973143339104	-0.002349853515625	\\
0.0115417055089448	-0.002410888671875	\\
0.0115860966839792	-0.002685546875	\\
0.0116304878590136	-0.003265380859375	\\
0.011674879034048	-0.00286865234375	\\
0.0117192702090824	-0.002777099609375	\\
0.0117636613841168	-0.002655029296875	\\
0.0118080525591512	-0.0023193359375	\\
0.0118524437341856	-0.002777099609375	\\
0.01189683490922	-0.002655029296875	\\
0.0119412260842545	-0.002685546875	\\
0.0119856172592889	-0.00250244140625	\\
0.0120300084343233	-0.002655029296875	\\
0.0120743996093577	-0.002471923828125	\\
0.0121187907843921	-0.00244140625	\\
0.0121631819594265	-0.002685546875	\\
0.0122075731344609	-0.00274658203125	\\
0.0122519643094953	-0.002410888671875	\\
0.0122963554845297	-0.001800537109375	\\
0.0123407466595641	-0.00189208984375	\\
0.0123851378345985	-0.001922607421875	\\
0.0124295290096329	-0.002166748046875	\\
0.0124739201846673	-0.002471923828125	\\
0.0125183113597017	-0.00213623046875	\\
0.0125627025347361	-0.001739501953125	\\
0.0126070937097705	-0.001953125	\\
0.0126514848848049	-0.00201416015625	\\
0.0126958760598393	-0.001861572265625	\\
0.0127402672348737	-0.00225830078125	\\
0.0127846584099081	-0.002349853515625	\\
0.0128290495849425	-0.002410888671875	\\
0.0128734407599769	-0.002655029296875	\\
0.0129178319350113	-0.00262451171875	\\
0.0129622231100457	-0.002838134765625	\\
0.0130066142850801	-0.002960205078125	\\
0.0130510054601145	-0.002655029296875	\\
0.0130953966351489	-0.0029296875	\\
0.0131397878101833	-0.00311279296875	\\
0.0131841789852177	-0.0030517578125	\\
0.0132285701602521	-0.003082275390625	\\
0.0132729613352865	-0.00341796875	\\
0.0133173525103209	-0.00341796875	\\
0.0133617436853554	-0.003021240234375	\\
0.0134061348603898	-0.002899169921875	\\
0.0134505260354242	-0.003021240234375	\\
0.0134949172104586	-0.00323486328125	\\
0.013539308385493	-0.003326416015625	\\
0.0135836995605274	-0.003265380859375	\\
0.0136280907355618	-0.003326416015625	\\
0.0136724819105962	-0.00299072265625	\\
0.0137168730856306	-0.003021240234375	\\
0.013761264260665	-0.003326416015625	\\
0.0138056554356994	-0.003692626953125	\\
0.0138500466107338	-0.00347900390625	\\
0.0138944377857682	-0.0028076171875	\\
0.0139388289608026	-0.00323486328125	\\
0.013983220135837	-0.00323486328125	\\
0.0140276113108714	-0.0030517578125	\\
0.0140720024859058	-0.002777099609375	\\
0.0141163936609402	-0.0025634765625	\\
0.0141607848359746	-0.00262451171875	\\
0.014205176011009	-0.002410888671875	\\
0.0142495671860434	-0.002410888671875	\\
0.0142939583610778	-0.002349853515625	\\
0.0143383495361122	-0.002288818359375	\\
0.0143827407111466	-0.00213623046875	\\
0.014427131886181	-0.00177001953125	\\
0.0144715230612154	-0.0018310546875	\\
0.0145159142362498	-0.0020751953125	\\
0.0145603054112842	-0.001861572265625	\\
0.0146046965863186	-0.0020751953125	\\
0.014649087761353	-0.00238037109375	\\
0.0146934789363874	-0.002349853515625	\\
0.0147378701114218	-0.002471923828125	\\
0.0147822612864563	-0.002349853515625	\\
0.0148266524614907	-0.002197265625	\\
0.0148710436365251	-0.002197265625	\\
0.0149154348115595	-0.0023193359375	\\
0.0149598259865939	-0.002532958984375	\\
0.0150042171616283	-0.002349853515625	\\
0.0150486083366627	-0.00244140625	\\
0.0150929995116971	-0.00250244140625	\\
0.0151373906867315	-0.002685546875	\\
0.0151817818617659	-0.00274658203125	\\
0.0152261730368003	-0.00262451171875	\\
0.0152705642118347	-0.00238037109375	\\
0.0153149553868691	-0.00213623046875	\\
0.0153593465619035	-0.002166748046875	\\
0.0154037377369379	-0.001953125	\\
0.0154481289119723	-0.001708984375	\\
0.0154925200870067	-0.00189208984375	\\
0.0155369112620411	-0.001861572265625	\\
0.0155813024370755	-0.001678466796875	\\
0.0156256936121099	-0.00189208984375	\\
0.0156700847871443	-0.001800537109375	\\
0.0157144759621787	-0.00164794921875	\\
0.0157588671372131	-0.0013427734375	\\
0.0158032583122475	-0.000946044921875	\\
0.0158476494872819	-0.000885009765625	\\
0.0158920406623163	-0.000823974609375	\\
0.0159364318373507	-0.000762939453125	\\
0.0159808230123851	-0.000701904296875	\\
0.0160252141874195	-0.000701904296875	\\
0.0160696053624539	-0.00030517578125	\\
0.0161139965374883	0	\\
0.0161583877125227	0.000152587890625	\\
0.0162027788875572	3.0517578125e-05	\\
0.0162471700625916	0.00018310546875	\\
0.016291561237626	0.000335693359375	\\
0.0163359524126604	0.000244140625	\\
0.0163803435876948	0.000213623046875	\\
0.0164247347627292	0.000213623046875	\\
0.0164691259377636	0.00018310546875	\\
0.016513517112798	0.00042724609375	\\
0.0165579082878324	0.00030517578125	\\
0.0166022994628668	3.0517578125e-05	\\
0.0166466906379012	-0.0001220703125	\\
0.0166910818129356	-0.000335693359375	\\
0.01673547298797	-0.000396728515625	\\
0.0167798641630044	-0.00030517578125	\\
0.0168242553380388	-0.000732421875	\\
0.0168686465130732	-0.0009765625	\\
0.0169130376881076	-0.000762939453125	\\
0.016957428863142	-0.000732421875	\\
0.0170018200381764	-0.001129150390625	\\
0.0170462112132108	-0.0013427734375	\\
0.0170906023882452	-0.001129150390625	\\
0.0171349935632796	-0.001495361328125	\\
0.017179384738314	-0.0015869140625	\\
0.0172237759133484	-0.00146484375	\\
0.0172681670883828	-0.00140380859375	\\
0.0173125582634172	-0.00152587890625	\\
0.0173569494384516	-0.001861572265625	\\
0.017401340613486	-0.001800537109375	\\
0.0174457317885204	-0.001708984375	\\
0.0174901229635548	-0.002105712890625	\\
0.0175345141385892	-0.0018310546875	\\
0.0175789053136237	-0.001739501953125	\\
0.0176232964886581	-0.002197265625	\\
0.0176676876636925	-0.001861572265625	\\
0.0177120788387269	-0.001708984375	\\
0.0177564700137613	-0.001556396484375	\\
0.0178008611887957	-0.001495361328125	\\
0.0178452523638301	-0.001678466796875	\\
0.0178896435388645	-0.001678466796875	\\
0.0179340347138989	-0.001068115234375	\\
0.0179784258889333	-0.00054931640625	\\
0.0180228170639677	-0.000701904296875	\\
0.0180672082390021	-0.000457763671875	\\
0.0181115994140365	-9.1552734375e-05	\\
0.0181559905890709	-0.000152587890625	\\
0.0182003817641053	0.000274658203125	\\
0.0182447729391397	0.00079345703125	\\
0.0182891641141741	0.00042724609375	\\
0.0183335552892085	9.1552734375e-05	\\
0.0183779464642429	-3.0517578125e-05	\\
0.0184223376392773	6.103515625e-05	\\
0.0184667288143117	0.000518798828125	\\
0.0185111199893461	0.00054931640625	\\
0.0185555111643805	0.000823974609375	\\
0.0185999023394149	0.000701904296875	\\
0.0186442935144493	0.0001220703125	\\
0.0186886846894837	0.000213623046875	\\
0.0187330758645181	0.00048828125	\\
0.0187774670395525	0.0001220703125	\\
0.0188218582145869	0.000152587890625	\\
0.0188662493896213	-0.000152587890625	\\
0.0189106405646557	-0.00042724609375	\\
0.0189550317396902	-0.00091552734375	\\
0.0189994229147246	-0.0013427734375	\\
0.019043814089759	-0.001434326171875	\\
0.0190882052647934	-0.001800537109375	\\
0.0191325964398278	-0.001434326171875	\\
0.0191769876148622	-0.001373291015625	\\
0.0192213787898966	-0.00146484375	\\
0.019265769964931	-0.001251220703125	\\
0.0193101611399654	-0.000823974609375	\\
0.0193545523149998	-0.00067138671875	\\
0.0193989434900342	-0.00030517578125	\\
0.0194433346650686	0.000274658203125	\\
0.019487725840103	-0.000213623046875	\\
0.0195321170151374	-0.000640869140625	\\
0.0195765081901718	-0.000274658203125	\\
0.0196208993652062	-0.00018310546875	\\
0.0196652905402406	-9.1552734375e-05	\\
0.019709681715275	-0.000823974609375	\\
0.0197540728903094	-0.000701904296875	\\
0.0197984640653438	0.000213623046875	\\
0.0198428552403782	0.00018310546875	\\
0.0198872464154126	-0.000335693359375	\\
0.019931637590447	-0.000335693359375	\\
0.0199760287654814	-0.00030517578125	\\
0.0200204199405158	-0.000152587890625	\\
0.0200648111155502	-0.000396728515625	\\
0.0201092022905846	-0.00042724609375	\\
0.020153593465619	-0.00054931640625	\\
0.0201979846406534	-0.00079345703125	\\
0.0202423758156878	-0.000732421875	\\
0.0202867669907222	-0.000762939453125	\\
0.0203311581657566	-0.00079345703125	\\
0.0203755493407911	-0.000885009765625	\\
0.0204199405158255	-0.001251220703125	\\
0.0204643316908599	-0.000885009765625	\\
0.0205087228658943	-0.001007080078125	\\
0.0205531140409287	-0.001220703125	\\
0.0205975052159631	-0.001251220703125	\\
0.0206418963909975	-0.001220703125	\\
0.0206862875660319	-0.000885009765625	\\
0.0207306787410663	-0.000823974609375	\\
0.0207750699161007	-0.001129150390625	\\
0.0208194610911351	-0.000946044921875	\\
0.0208638522661695	-0.00115966796875	\\
0.0209082434412039	-0.001556396484375	\\
0.0209526346162383	-0.00091552734375	\\
0.0209970257912727	-0.0009765625	\\
0.0210414169663071	-0.001129150390625	\\
0.0210858081413415	-0.00128173828125	\\
0.0211301993163759	-0.001220703125	\\
0.0211745904914103	-0.001251220703125	\\
0.0212189816664447	-0.0015869140625	\\
0.0212633728414791	-0.00146484375	\\
0.0213077640165135	-0.001373291015625	\\
0.0213521551915479	-0.001190185546875	\\
0.0213965463665823	-0.00152587890625	\\
0.0214409375416167	-0.00164794921875	\\
0.0214853287166511	-0.001556396484375	\\
0.0215297198916855	-0.001678466796875	\\
0.0215741110667199	-0.001220703125	\\
0.0216185022417543	-0.0009765625	\\
0.0216628934167887	-0.001129150390625	\\
0.0217072845918231	-0.001190185546875	\\
0.0217516757668575	-0.00146484375	\\
0.021796066941892	-0.0015869140625	\\
0.0218404581169264	-0.00079345703125	\\
0.0218848492919608	-0.000701904296875	\\
0.0219292404669952	-0.0010986328125	\\
0.0219736316420296	-0.001129150390625	\\
0.022018022817064	-0.000640869140625	\\
0.0220624139920984	-0.00079345703125	\\
0.0221068051671328	-0.00128173828125	\\
0.0221511963421672	-0.001312255859375	\\
0.0221955875172016	-0.001312255859375	\\
0.022239978692236	-0.001190185546875	\\
0.0222843698672704	-0.0013427734375	\\
0.0223287610423048	-0.001190185546875	\\
0.0223731522173392	-0.00128173828125	\\
0.0224175433923736	-0.001708984375	\\
0.022461934567408	-0.00189208984375	\\
0.0225063257424424	-0.001983642578125	\\
0.0225507169174768	-0.001678466796875	\\
0.0225951080925112	-0.001373291015625	\\
0.0226394992675456	-0.001708984375	\\
0.02268389044258	-0.001739501953125	\\
0.0227282816176144	-0.001739501953125	\\
0.0227726727926488	-0.00189208984375	\\
0.0228170639676832	-0.001678466796875	\\
0.0228614551427176	-0.001861572265625	\\
0.022905846317752	-0.0018310546875	\\
0.0229502374927864	-0.001708984375	\\
0.0229946286678208	-0.0015869140625	\\
0.0230390198428552	-0.001678466796875	\\
0.0230834110178896	-0.0015869140625	\\
0.023127802192924	-0.00164794921875	\\
0.0231721933679584	-0.00189208984375	\\
0.0232165845429929	-0.001953125	\\
0.0232609757180273	-0.002410888671875	\\
0.0233053668930617	-0.00225830078125	\\
0.0233497580680961	-0.002044677734375	\\
0.0233941492431305	-0.001678466796875	\\
0.0234385404181649	-0.001556396484375	\\
0.0234829315931993	-0.00152587890625	\\
0.0235273227682337	-0.001373291015625	\\
0.0235717139432681	-0.001434326171875	\\
0.0236161051183025	-0.001739501953125	\\
0.0236604962933369	-0.001953125	\\
0.0237048874683713	-0.00201416015625	\\
0.0237492786434057	-0.001678466796875	\\
0.0237936698184401	-0.001739501953125	\\
0.0238380609934745	-0.001983642578125	\\
0.0238824521685089	-0.002044677734375	\\
0.0239268433435433	-0.00201416015625	\\
0.0239712345185777	-0.002044677734375	\\
0.0240156256936121	-0.00262451171875	\\
0.0240600168686465	-0.0028076171875	\\
0.0241044080436809	-0.002593994140625	\\
0.0241487992187153	-0.002716064453125	\\
0.0241931903937497	-0.002288818359375	\\
0.0242375815687841	-0.0023193359375	\\
0.0242819727438185	-0.00250244140625	\\
0.0243263639188529	-0.002655029296875	\\
0.0243707550938873	-0.00262451171875	\\
0.0244151462689217	-0.002655029296875	\\
0.0244595374439561	-0.0030517578125	\\
0.0245039286189905	-0.00323486328125	\\
0.0245483197940249	-0.00311279296875	\\
0.0245927109690594	-0.003265380859375	\\
0.0246371021440938	-0.003021240234375	\\
0.0246814933191282	-0.0029296875	\\
0.0247258844941626	-0.003021240234375	\\
0.024770275669197	-0.003143310546875	\\
0.0248146668442314	-0.003143310546875	\\
0.0248590580192658	-0.003387451171875	\\
0.0249034491943002	-0.0030517578125	\\
0.0249478403693346	-0.003021240234375	\\
0.024992231544369	-0.0035400390625	\\
0.0250366227194034	-0.003509521484375	\\
0.0250810138944378	-0.00360107421875	\\
0.0251254050694722	-0.00311279296875	\\
0.0251697962445066	-0.003173828125	\\
0.025214187419541	-0.003814697265625	\\
0.0252585785945754	-0.003814697265625	\\
0.0253029697696098	-0.003387451171875	\\
0.0253473609446442	-0.003021240234375	\\
0.0253917521196786	-0.002716064453125	\\
0.025436143294713	-0.002471923828125	\\
0.0254805344697474	-0.002349853515625	\\
0.0255249256447818	-0.002105712890625	\\
0.0255693168198162	-0.00177001953125	\\
0.0256137079948506	-0.00189208984375	\\
0.025658099169885	-0.00201416015625	\\
0.0257024903449194	-0.001922607421875	\\
0.0257468815199538	-0.00128173828125	\\
0.0257912726949882	-0.001068115234375	\\
0.0258356638700226	-0.000885009765625	\\
0.025880055045057	-0.00030517578125	\\
0.0259244462200914	-0.000640869140625	\\
0.0259688373951259	-9.1552734375e-05	\\
0.0260132285701603	6.103515625e-05	\\
0.0260576197451947	-0.000152587890625	\\
0.0261020109202291	0.000152587890625	\\
0.0261464020952635	-0.000244140625	\\
0.0261907932702979	-0.000396728515625	\\
0.0262351844453323	-0.0001220703125	\\
0.0262795756203667	0.0001220703125	\\
0.0263239667954011	-3.0517578125e-05	\\
0.0263683579704355	-0.00030517578125	\\
0.0264127491454699	-0.000244140625	\\
0.0264571403205043	-0.0003662109375	\\
0.0265015314955387	-0.000579833984375	\\
0.0265459226705731	-0.001220703125	\\
0.0265903138456075	-0.001220703125	\\
0.0266347050206419	-0.001220703125	\\
0.0266790961956763	-0.001708984375	\\
0.0267234873707107	-0.00146484375	\\
0.0267678785457451	-0.00164794921875	\\
0.0268122697207795	-0.002227783203125	\\
0.0268566608958139	-0.001953125	\\
0.0269010520708483	-0.00177001953125	\\
0.0269454432458827	-0.00201416015625	\\
0.0269898344209171	-0.001953125	\\
0.0270342255959515	-0.0020751953125	\\
0.0270786167709859	-0.002197265625	\\
0.0271230079460203	-0.002227783203125	\\
0.0271673991210547	-0.002105712890625	\\
0.0272117902960891	-0.001861572265625	\\
0.0272561814711235	-0.001739501953125	\\
0.0273005726461579	-0.0010986328125	\\
0.0273449638211923	-0.000701904296875	\\
0.0273893549962268	-0.00091552734375	\\
0.0274337461712612	-0.00103759765625	\\
0.0274781373462956	-0.001129150390625	\\
0.02752252852133	-0.001312255859375	\\
0.0275669196963644	-0.000946044921875	\\
0.0276113108713988	-0.00091552734375	\\
0.0276557020464332	-0.001312255859375	\\
0.0277000932214676	-0.0008544921875	\\
0.027744484396502	-0.0009765625	\\
0.0277888755715364	-0.0010986328125	\\
0.0278332667465708	-0.00091552734375	\\
0.0278776579216052	-0.001068115234375	\\
0.0279220490966396	-0.000946044921875	\\
0.027966440271674	-0.00079345703125	\\
0.0280108314467084	-0.00103759765625	\\
0.0280552226217428	-0.000640869140625	\\
0.0280996137967772	-0.000579833984375	\\
0.0281440049718116	-0.000518798828125	\\
0.028188396146846	-0.000335693359375	\\
0.0282327873218804	-0.000396728515625	\\
0.0282771784969148	-0.000579833984375	\\
0.0283215696719492	-0.000396728515625	\\
0.0283659608469836	-0.00067138671875	\\
0.028410352022018	-0.00079345703125	\\
0.0284547431970524	-0.00030517578125	\\
0.0284991343720868	-0.000579833984375	\\
0.0285435255471212	-0.000885009765625	\\
0.0285879167221556	-0.000762939453125	\\
0.02863230789719	-0.000946044921875	\\
0.0286766990722244	-0.001068115234375	\\
0.0287210902472588	-0.0010986328125	\\
0.0287654814222932	-0.001434326171875	\\
0.0288098725973277	-0.00164794921875	\\
0.0288542637723621	-0.001495361328125	\\
0.0288986549473965	-0.001617431640625	\\
0.0289430461224309	-0.002044677734375	\\
0.0289874372974653	-0.002044677734375	\\
0.0290318284724997	-0.001983642578125	\\
0.0290762196475341	-0.00238037109375	\\
0.0291206108225685	-0.00250244140625	\\
0.0291650019976029	-0.0028076171875	\\
0.0292093931726373	-0.0025634765625	\\
0.0292537843476717	-0.0025634765625	\\
0.0292981755227061	-0.00286865234375	\\
0.0293425666977405	-0.002410888671875	\\
0.0293869578727749	-0.002410888671875	\\
0.0294313490478093	-0.0020751953125	\\
0.0294757402228437	-0.001617431640625	\\
0.0295201313978781	-0.00152587890625	\\
0.0295645225729125	-0.00146484375	\\
0.0296089137479469	-0.001739501953125	\\
0.0296533049229813	-0.001678466796875	\\
0.0296976960980157	-0.001068115234375	\\
0.0297420872730501	-0.000885009765625	\\
0.0297864784480845	-0.000946044921875	\\
0.0298308696231189	-0.0008544921875	\\
0.0298752607981533	-0.000946044921875	\\
0.0299196519731877	-0.00115966796875	\\
0.0299640431482221	-0.0008544921875	\\
0.0300084343232565	-0.00091552734375	\\
0.0300528254982909	-0.000518798828125	\\
0.0300972166733253	-0.00048828125	\\
0.0301416078483597	-0.000885009765625	\\
0.0301859990233941	-0.000732421875	\\
0.0302303901984286	-0.000732421875	\\
0.030274781373463	-0.00067138671875	\\
0.0303191725484974	-0.00115966796875	\\
0.0303635637235318	-0.000518798828125	\\
0.0304079548985662	-0.00042724609375	\\
0.0304523460736006	-0.000518798828125	\\
0.030496737248635	-0.00042724609375	\\
0.0305411284236694	-0.00128173828125	\\
0.0305855195987038	-0.0010986328125	\\
0.0306299107737382	-0.000823974609375	\\
0.0306743019487726	-0.0009765625	\\
0.030718693123807	-0.001434326171875	\\
0.0307630842988414	-0.00152587890625	\\
0.0308074754738758	-0.001251220703125	\\
0.0308518666489102	-0.001190185546875	\\
0.0308962578239446	-0.001220703125	\\
0.030940648998979	-0.001495361328125	\\
0.0309850401740134	-0.001861572265625	\\
0.0310294313490478	-0.0018310546875	\\
0.0310738225240822	-0.001800537109375	\\
0.0311182136991166	-0.001708984375	\\
0.031162604874151	-0.00140380859375	\\
0.0312069960491854	-0.001678466796875	\\
0.0312513872242198	-0.001800537109375	\\
0.0312957783992542	-0.00128173828125	\\
0.0313401695742886	-0.0013427734375	\\
0.031384560749323	-0.001678466796875	\\
0.0314289519243574	-0.00164794921875	\\
0.0314733430993918	-0.001739501953125	\\
0.0315177342744262	-0.00177001953125	\\
0.0315621254494606	-0.0018310546875	\\
0.031606516624495	-0.00177001953125	\\
0.0316509077995295	-0.001861572265625	\\
0.0316952989745639	-0.001739501953125	\\
0.0317396901495983	-0.001739501953125	\\
0.0317840813246327	-0.00128173828125	\\
0.0318284724996671	-0.0008544921875	\\
0.0318728636747015	-0.000946044921875	\\
0.0319172548497359	-0.00103759765625	\\
0.0319616460247703	-0.000946044921875	\\
0.0320060371998047	-0.000732421875	\\
0.0320504283748391	-0.000946044921875	\\
0.0320948195498735	-0.00048828125	\\
0.0321392107249079	-0.000579833984375	\\
0.0321836018999423	-0.000732421875	\\
0.0322279930749767	-0.0009765625	\\
0.0322723842500111	-0.00091552734375	\\
0.0323167754250455	-0.00067138671875	\\
0.0323611666000799	-0.000457763671875	\\
0.0324055577751143	-0.000335693359375	\\
0.0324499489501487	-0.000396728515625	\\
0.0324943401251831	-0.000335693359375	\\
0.0325387313002175	-0.00067138671875	\\
0.0325831224752519	-0.00067138671875	\\
0.0326275136502863	-0.00091552734375	\\
0.0326719048253207	-0.001007080078125	\\
0.0327162960003551	-0.00103759765625	\\
0.0327606871753895	-0.001220703125	\\
0.0328050783504239	-0.00091552734375	\\
0.0328494695254583	-0.0008544921875	\\
0.0328938607004927	-0.00054931640625	\\
0.0329382518755271	-0.000732421875	\\
0.0329826430505615	-0.0006103515625	\\
0.0330270342255959	-0.00030517578125	\\
0.0330714254006304	-0.000579833984375	\\
0.0331158165756648	-0.000518798828125	\\
0.0331602077506992	-6.103515625e-05	\\
0.0332045989257336	-0.000213623046875	\\
0.033248990100768	0.00018310546875	\\
0.0332933812758024	0.00030517578125	\\
0.0333377724508368	0.000152587890625	\\
0.0333821636258712	0.000579833984375	\\
0.0334265548009056	0.000457763671875	\\
0.03347094597594	0.000396728515625	\\
0.0335153371509744	0.00042724609375	\\
0.0335597283260088	0.0001220703125	\\
0.0336041195010432	6.103515625e-05	\\
0.0336485106760776	0.00067138671875	\\
0.033692901851112	0.000640869140625	\\
0.0337372930261464	9.1552734375e-05	\\
0.0337816842011808	3.0517578125e-05	\\
0.0338260753762152	0.00091552734375	\\
0.0338704665512496	0.00079345703125	\\
0.033914857726284	0.000518798828125	\\
0.0339592489013184	0.000762939453125	\\
0.0340036400763528	0.00054931640625	\\
0.0340480312513872	0.000762939453125	\\
0.0340924224264216	0.000640869140625	\\
0.034136813601456	0.00030517578125	\\
0.0341812047764904	0.000518798828125	\\
0.0342255959515248	0.000823974609375	\\
0.0342699871265592	0.00067138671875	\\
0.0343143783015936	0.000579833984375	\\
0.034358769476628	0.000213623046875	\\
0.0344031606516624	0.000152587890625	\\
0.0344475518266969	0.00054931640625	\\
0.0344919430017313	0.000518798828125	\\
0.0345363341767657	0.0003662109375	\\
0.0345807253518001	9.1552734375e-05	\\
0.0346251165268345	0.000701904296875	\\
0.0346695077018689	0.000640869140625	\\
0.0347138988769033	0.000152587890625	\\
0.0347582900519377	0.0003662109375	\\
0.0348026812269721	0.000274658203125	\\
0.0348470724020065	0.00030517578125	\\
0.0348914635770409	0.00030517578125	\\
0.0349358547520753	0.00042724609375	\\
0.0349802459271097	-0.0001220703125	\\
0.0350246371021441	0.000152587890625	\\
0.0350690282771785	0.000579833984375	\\
0.0351134194522129	0.000274658203125	\\
0.0351578106272473	0.00054931640625	\\
0.0352022018022817	0.000518798828125	\\
0.0352465929773161	0.000274658203125	\\
0.0352909841523505	0.000823974609375	\\
0.0353353753273849	0.001190185546875	\\
0.0353797665024193	0.0009765625	\\
0.0354241576774537	0.00103759765625	\\
0.0354685488524881	0.00048828125	\\
0.0355129400275225	0.001007080078125	\\
0.0355573312025569	0.00146484375	\\
0.0356017223775913	0.001251220703125	\\
0.0356461135526257	0.001617431640625	\\
0.0356905047276601	0.00201416015625	\\
0.0357348959026945	0.0020751953125	\\
0.0357792870777289	0.0020751953125	\\
0.0358236782527634	0.00225830078125	\\
0.0358680694277978	0.0025634765625	\\
0.0359124606028322	0.002655029296875	\\
0.0359568517778666	0.002716064453125	\\
0.036001242952901	0.00225830078125	\\
0.0360456341279354	0.002288818359375	\\
0.0360900253029698	0.002593994140625	\\
0.0361344164780042	0.00250244140625	\\
0.0361788076530386	0.002655029296875	\\
0.036223198828073	0.0032958984375	\\
0.0362675900031074	0.002960205078125	\\
0.0363119811781418	0.002532958984375	\\
0.0363563723531762	0.002777099609375	\\
0.0364007635282106	0.0025634765625	\\
0.036445154703245	0.002655029296875	\\
0.0364895458782794	0.002410888671875	\\
0.0365339370533138	0.002288818359375	\\
0.0365783282283482	0.00238037109375	\\
0.0366227194033826	0.001983642578125	\\
0.036667110578417	0.001861572265625	\\
0.0367115017534514	0.00201416015625	\\
0.0367558929284858	0.001708984375	\\
0.0368002841035202	0.00146484375	\\
0.0368446752785546	0.0018310546875	\\
0.036889066453589	0.0013427734375	\\
0.0369334576286234	0.0010986328125	\\
0.0369778488036578	0.0008544921875	\\
0.0370222399786922	0.0008544921875	\\
0.0370666311537266	0.0010986328125	\\
0.037111022328761	0.0006103515625	\\
0.0371554135037954	0.00091552734375	\\
0.0371998046788298	0.00091552734375	\\
0.0372441958538643	0.00103759765625	\\
0.0372885870288987	0.00140380859375	\\
0.0373329782039331	0.00164794921875	\\
0.0373773693789675	0.00189208984375	\\
0.0374217605540019	0.001861572265625	\\
0.0374661517290363	0.001861572265625	\\
0.0375105429040707	0.00189208984375	\\
0.0375549340791051	0.002166748046875	\\
0.0375993252541395	0.00238037109375	\\
0.0376437164291739	0.002410888671875	\\
0.0376881076042083	0.002716064453125	\\
0.0377324987792427	0.002593994140625	\\
0.0377768899542771	0.002685546875	\\
0.0378212811293115	0.00274658203125	\\
0.0378656723043459	0.00299072265625	\\
0.0379100634793803	0.00311279296875	\\
0.0379544546544147	0.003021240234375	\\
0.0379988458294491	0.003387451171875	\\
0.0380432370044835	0.0028076171875	\\
0.0380876281795179	0.002655029296875	\\
0.0381320193545523	0.0025634765625	\\
0.0381764105295867	0.002532958984375	\\
0.0382208017046211	0.0029296875	\\
0.0382651928796555	0.002716064453125	\\
0.0383095840546899	0.002593994140625	\\
0.0383539752297243	0.00244140625	\\
0.0383983664047587	0.002288818359375	\\
0.0384427575797931	0.0023193359375	\\
0.0384871487548275	0.00177001953125	\\
0.0385315399298619	0.0018310546875	\\
0.0385759311048963	0.00146484375	\\
0.0386203222799308	0.001068115234375	\\
0.0386647134549652	0.001251220703125	\\
0.0387091046299996	0.0013427734375	\\
0.038753495805034	0.001068115234375	\\
0.0387978869800684	0.00067138671875	\\
0.0388422781551028	0.000518798828125	\\
0.0388866693301372	0.0008544921875	\\
0.0389310605051716	0.0008544921875	\\
0.038975451680206	0.0006103515625	\\
0.0390198428552404	0.000823974609375	\\
0.0390642340302748	0.00103759765625	\\
0.0391086252053092	0.0015869140625	\\
0.0391530163803436	0.0010986328125	\\
0.039197407555378	0.000732421875	\\
0.0392417987304124	0.000885009765625	\\
0.0392861899054468	0.001007080078125	\\
0.0393305810804812	0.001007080078125	\\
0.0393749722555156	0.001495361328125	\\
0.03941936343055	0.001617431640625	\\
0.0394637546055844	0.00103759765625	\\
0.0395081457806188	0.0013427734375	\\
0.0395525369556532	0.001708984375	\\
0.0395969281306876	0.00177001953125	\\
0.039641319305722	0.002288818359375	\\
0.0396857104807564	0.002532958984375	\\
0.0397301016557908	0.00213623046875	\\
0.0397744928308252	0.002166748046875	\\
0.0398188840058596	0.002288818359375	\\
0.039863275180894	0.00177001953125	\\
0.0399076663559284	0.001617431640625	\\
0.0399520575309628	0.0018310546875	\\
0.0399964487059973	0.001495361328125	\\
0.0400408398810317	0.001312255859375	\\
0.0400852310560661	0.001495361328125	\\
0.0401296222311005	0.000732421875	\\
0.0401740134061349	0.0008544921875	\\
0.0402184045811693	0.001373291015625	\\
0.0402627957562037	0.001251220703125	\\
0.0403071869312381	0.001312255859375	\\
0.0403515781062725	0.00115966796875	\\
0.0403959692813069	0.000885009765625	\\
0.0404403604563413	0.00079345703125	\\
0.0404847516313757	0.001007080078125	\\
0.0405291428064101	0.0008544921875	\\
0.0405735339814445	0.00048828125	\\
0.0406179251564789	0.00054931640625	\\
0.0406623163315133	0.00048828125	\\
0.0407067075065477	0.000274658203125	\\
0.0407510986815821	0.000732421875	\\
0.0407954898566165	0.000732421875	\\
0.0408398810316509	0.000701904296875	\\
0.0408842722066853	0.0009765625	\\
0.0409286633817197	0.0008544921875	\\
0.0409730545567541	0.00103759765625	\\
0.0410174457317885	0.001068115234375	\\
0.0410618369068229	0.00103759765625	\\
0.0411062280818573	0.001556396484375	\\
0.0411506192568917	0.0013427734375	\\
0.0411950104319261	0.00140380859375	\\
0.0412394016069605	0.00140380859375	\\
0.0412837927819949	0.0009765625	\\
0.0413281839570293	0.001556396484375	\\
0.0413725751320637	0.0018310546875	\\
0.0414169663070982	0.00177001953125	\\
0.0414613574821326	0.001617431640625	\\
0.041505748657167	0.0018310546875	\\
0.0415501398322014	0.00164794921875	\\
0.0415945310072358	0.001373291015625	\\
0.0416389221822702	0.00164794921875	\\
0.0416833133573046	0.001922607421875	\\
0.041727704532339	0.001922607421875	\\
0.0417720957073734	0.001922607421875	\\
0.0418164868824078	0.001800537109375	\\
0.0418608780574422	0.001556396484375	\\
0.0419052692324766	0.001434326171875	\\
0.041949660407511	0.001312255859375	\\
0.0419940515825454	0.001373291015625	\\
0.0420384427575798	0.001556396484375	\\
0.0420828339326142	0.001434326171875	\\
0.0421272251076486	0.00152587890625	\\
0.042171616282683	0.00164794921875	\\
0.0422160074577174	0.001708984375	\\
0.0422603986327518	0.001617431640625	\\
0.0423047898077862	0.001251220703125	\\
0.0423491809828206	0.001220703125	\\
0.042393572157855	0.001617431640625	\\
0.0424379633328894	0.0020751953125	\\
0.0424823545079238	0.001739501953125	\\
0.0425267456829582	0.001617431640625	\\
0.0425711368579926	0.00164794921875	\\
0.042615528033027	0.0015869140625	\\
0.0426599192080614	0.00177001953125	\\
0.0427043103830958	0.00164794921875	\\
0.0427487015581302	0.001922607421875	\\
0.0427930927331646	0.002044677734375	\\
0.0428374839081991	0.00213623046875	\\
0.0428818750832335	0.00238037109375	\\
0.0429262662582679	0.001953125	\\
0.0429706574333023	0.00189208984375	\\
0.0430150486083367	0.00244140625	\\
0.0430594397833711	0.00244140625	\\
0.0431038309584055	0.002166748046875	\\
0.0431482221334399	0.0023193359375	\\
0.0431926133084743	0.002410888671875	\\
0.0432370044835087	0.00274658203125	\\
0.0432813956585431	0.00250244140625	\\
0.0433257868335775	0.00244140625	\\
0.0433701780086119	0.002105712890625	\\
0.0434145691836463	0.002044677734375	\\
0.0434589603586807	0.00250244140625	\\
0.0435033515337151	0.002532958984375	\\
0.0435477427087495	0.0028076171875	\\
0.0435921338837839	0.002838134765625	\\
0.0436365250588183	0.002655029296875	\\
0.0436809162338527	0.002532958984375	\\
0.0437253074088871	0.0020751953125	\\
0.0437696985839215	0.00238037109375	\\
0.0438140897589559	0.002227783203125	\\
0.0438584809339903	0.001983642578125	\\
0.0439028721090247	0.001708984375	\\
0.0439472632840591	0.001617431640625	\\
0.0439916544590935	0.00201416015625	\\
0.0440360456341279	0.0015869140625	\\
0.0440804368091623	0.00115966796875	\\
0.0441248279841967	0.001373291015625	\\
0.0441692191592311	0.001373291015625	\\
0.0442136103342655	0.00115966796875	\\
0.0442580015093	0.000762939453125	\\
0.0443023926843344	0.00067138671875	\\
0.0443467838593688	0.00067138671875	\\
0.0443911750344032	0.00042724609375	\\
0.0444355662094376	0.000152587890625	\\
0.044479957384472	0.000213623046875	\\
0.0445243485595064	-9.1552734375e-05	\\
0.0445687397345408	-0.000213623046875	\\
0.0446131309095752	-0.000152587890625	\\
0.0446575220846096	-0.000701904296875	\\
0.044701913259644	-0.00054931640625	\\
0.0447463044346784	-0.000274658203125	\\
0.0447906956097128	-0.00048828125	\\
0.0448350867847472	-0.000457763671875	\\
0.0448794779597816	-0.000213623046875	\\
0.044923869134816	-9.1552734375e-05	\\
0.0449682603098504	-9.1552734375e-05	\\
0.0450126514848848	0	\\
0.0450570426599192	-3.0517578125e-05	\\
0.0451014338349536	0.000396728515625	\\
0.045145825009988	0.00042724609375	\\
0.0451902161850224	0.000396728515625	\\
0.0452346073600568	0.0009765625	\\
0.0452789985350912	0.00091552734375	\\
0.0453233897101256	0.001190185546875	\\
0.04536778088516	0.001556396484375	\\
0.0454121720601944	0.001434326171875	\\
0.0454565632352288	0.00164794921875	\\
0.0455009544102632	0.001708984375	\\
0.0455453455852976	0.001434326171875	\\
0.045589736760332	0.001556396484375	\\
0.0456341279353664	0.001678466796875	\\
0.0456785191104008	0.001373291015625	\\
0.0457229102854353	0.0015869140625	\\
0.0457673014604697	0.001922607421875	\\
0.0458116926355041	0.001922607421875	\\
0.0458560838105385	0.001983642578125	\\
0.0459004749855729	0.002105712890625	\\
0.0459448661606073	0.002288818359375	\\
0.0459892573356417	0.00201416015625	\\
0.0460336485106761	0.001708984375	\\
0.0460780396857105	0.0015869140625	\\
0.0461224308607449	0.00164794921875	\\
0.0461668220357793	0.001983642578125	\\
0.0462112132108137	0.001800537109375	\\
0.0462556043858481	0.00103759765625	\\
0.0462999955608825	0.001190185546875	\\
0.0463443867359169	0.001190185546875	\\
0.0463887779109513	0.001129150390625	\\
0.0464331690859857	0.00128173828125	\\
0.0464775602610201	0.00103759765625	\\
0.0465219514360545	0.00067138671875	\\
0.0465663426110889	0.000640869140625	\\
0.0466107337861233	0.000457763671875	\\
0.0466551249611577	0.000518798828125	\\
0.0466995161361921	0.001129150390625	\\
0.0467439073112265	0.000732421875	\\
0.0467882984862609	0.0006103515625	\\
0.0468326896612953	0.00054931640625	\\
0.0468770808363297	0.000518798828125	\\
0.0469214720113641	0.00067138671875	\\
0.0469658631863985	0.000579833984375	\\
0.0470102543614329	0.000457763671875	\\
0.0470546455364673	0.000579833984375	\\
0.0470990367115018	0.00042724609375	\\
0.0471434278865362	9.1552734375e-05	\\
0.0471878190615706	0.00042724609375	\\
0.047232210236605	-9.1552734375e-05	\\
0.0472766014116394	-0.000457763671875	\\
0.0473209925866738	-0.000152587890625	\\
0.0473653837617082	-0.000274658203125	\\
0.0474097749367426	-9.1552734375e-05	\\
0.047454166111777	0.00018310546875	\\
0.0474985572868114	0.000152587890625	\\
0.0475429484618458	9.1552734375e-05	\\
0.0475873396368802	0.00018310546875	\\
0.0476317308119146	6.103515625e-05	\\
0.047676121986949	-0.0001220703125	\\
0.0477205131619834	-0.0001220703125	\\
0.0477649043370178	-0.00030517578125	\\
0.0478092955120522	-0.000152587890625	\\
0.0478536866870866	-0.0001220703125	\\
0.047898077862121	-0.000213623046875	\\
0.0479424690371554	-9.1552734375e-05	\\
0.0479868602121898	6.103515625e-05	\\
0.0480312513872242	-0.0001220703125	\\
0.0480756425622586	-6.103515625e-05	\\
0.048120033737293	-9.1552734375e-05	\\
0.0481644249123274	-0.0001220703125	\\
0.0482088160873618	-0.000244140625	\\
0.0482532072623962	-0.000335693359375	\\
0.0482975984374306	-0.00042724609375	\\
0.048341989612465	-0.000732421875	\\
0.0483863807874994	-0.00042724609375	\\
0.0484307719625338	-0.000396728515625	\\
0.0484751631375683	-0.000335693359375	\\
0.0485195543126027	-0.00018310546875	\\
0.0485639454876371	-0.0003662109375	\\
0.0486083366626715	-0.000274658203125	\\
0.0486527278377059	-0.000244140625	\\
0.0486971190127403	-0.000274658203125	\\
0.0487415101877747	-0.00030517578125	\\
0.0487859013628091	-0.000335693359375	\\
0.0488302925378435	-0.000335693359375	\\
0.0488746837128779	-9.1552734375e-05	\\
0.0489190748879123	-0.000244140625	\\
0.0489634660629467	-0.000274658203125	\\
0.0490078572379811	-0.0001220703125	\\
0.0490522484130155	-0.00030517578125	\\
0.0490966395880499	-6.103515625e-05	\\
0.0491410307630843	-0.000213623046875	\\
0.0491854219381187	-0.000396728515625	\\
0.0492298131131531	-0.000213623046875	\\
0.0492742042881875	-9.1552734375e-05	\\
0.0493185954632219	-0.000213623046875	\\
0.0493629866382563	-6.103515625e-05	\\
0.0494073778132907	-0.00030517578125	\\
0.0494517689883251	-0.000274658203125	\\
0.0494961601633595	0.000274658203125	\\
0.0495405513383939	0.00030517578125	\\
0.0495849425134283	0.000152587890625	\\
0.0496293336884627	-0.000274658203125	\\
0.0496737248634971	-0.00054931640625	\\
0.0497181160385315	-0.000640869140625	\\
0.0497625072135659	-0.000457763671875	\\
0.0498068983886003	-0.00054931640625	\\
0.0498512895636348	-3.0517578125e-05	\\
0.0498956807386692	0.0006103515625	\\
0.0499400719137036	0.0006103515625	\\
0.049984463088738	0.001068115234375	\\
0.0500288542637724	0.001129150390625	\\
0.0500732454388068	0.0009765625	\\
0.0501176366138412	0.00146484375	\\
0.0501620277888756	0.001495361328125	\\
0.05020641896391	0.000946044921875	\\
0.0502508101389444	0.00146484375	\\
0.0502952013139788	0.00152587890625	\\
0.0503395924890132	0.00140380859375	\\
0.0503839836640476	0.001922607421875	\\
0.050428374839082	0.001983642578125	\\
0.0504727660141164	0.00225830078125	\\
0.0505171571891508	0.001953125	\\
0.0505615483641852	0.00164794921875	\\
0.0506059395392196	0.001922607421875	\\
0.050650330714254	0.002288818359375	\\
0.0506947218892884	0.002197265625	\\
0.0507391130643228	0.0023193359375	\\
0.0507835042393572	0.00225830078125	\\
0.0508278954143916	0.001953125	\\
0.050872286589426	0.002166748046875	\\
0.0509166777644604	0.002044677734375	\\
0.0509610689394948	0.00213623046875	\\
0.0510054601145292	0.00189208984375	\\
0.0510498512895636	0.00164794921875	\\
0.051094242464598	0.001800537109375	\\
0.0511386336396324	0.001739501953125	\\
0.0511830248146668	0.001983642578125	\\
0.0512274159897013	0.00213623046875	\\
0.0512718071647357	0.0018310546875	\\
0.0513161983397701	0.001800537109375	\\
0.0513605895148045	0.001953125	\\
0.0514049806898389	0.00201416015625	\\
0.0514493718648733	0.001708984375	\\
0.0514937630399077	0.001190185546875	\\
0.0515381542149421	0.001373291015625	\\
0.0515825453899765	0.001312255859375	\\
0.0516269365650109	0.0009765625	\\
0.0516713277400453	0.000701904296875	\\
0.0517157189150797	0.000244140625	\\
0.0517601100901141	0.00067138671875	\\
0.0518045012651485	0.00048828125	\\
0.0518488924401829	0.000152587890625	\\
0.0518932836152173	-9.1552734375e-05	\\
0.0519376747902517	0	\\
0.0519820659652861	0.000732421875	\\
0.0520264571403205	0.00048828125	\\
0.0520708483153549	0.00030517578125	\\
0.0521152394903893	0.00079345703125	\\
0.0521596306654237	0.00030517578125	\\
0.0522040218404581	9.1552734375e-05	\\
0.0522484130154925	0.00030517578125	\\
0.0522928041905269	9.1552734375e-05	\\
0.0523371953655613	0.00042724609375	\\
0.0523815865405957	0.000579833984375	\\
0.0524259777156301	0.000823974609375	\\
0.0524703688906645	0.0008544921875	\\
0.0525147600656989	0.000762939453125	\\
0.0525591512407333	0.001495361328125	\\
0.0526035424157677	0.00146484375	\\
0.0526479335908022	0.00146484375	\\
0.0526923247658366	0.001617431640625	\\
0.052736715940871	0.00177001953125	\\
0.0527811071159054	0.001617431640625	\\
0.0528254982909398	0.00177001953125	\\
0.0528698894659742	0.0018310546875	\\
0.0529142806410086	0.001434326171875	\\
0.052958671816043	0.001678466796875	\\
0.0530030629910774	0.0020751953125	\\
0.0530474541661118	0.002044677734375	\\
0.0530918453411462	0.002166748046875	\\
0.0531362365161806	0.002471923828125	\\
0.053180627691215	0.002777099609375	\\
0.0532250188662494	0.002685546875	\\
0.0532694100412838	0.002410888671875	\\
0.0533138012163182	0.002410888671875	\\
0.0533581923913526	0.002288818359375	\\
0.053402583566387	0.002532958984375	\\
0.0534469747414214	0.00244140625	\\
0.0534913659164558	0.002105712890625	\\
0.0535357570914902	0.001953125	\\
0.0535801482665246	0.00164794921875	\\
0.053624539441559	0.001556396484375	\\
0.0536689306165934	0.00146484375	\\
0.0537133217916278	0.001373291015625	\\
0.0537577129666622	0.00164794921875	\\
0.0538021041416966	0.00177001953125	\\
0.053846495316731	0.00201416015625	\\
0.0538908864917654	0.002349853515625	\\
0.0539352776667998	0.001953125	\\
0.0539796688418342	0.001556396484375	\\
0.0540240600168686	0.00152587890625	\\
0.0540684511919031	0.00140380859375	\\
0.0541128423669375	0.001220703125	\\
0.0541572335419719	0.001556396484375	\\
0.0542016247170063	0.00140380859375	\\
0.0542460158920407	0.001495361328125	\\
0.0542904070670751	0.001678466796875	\\
0.0543347982421095	0.00128173828125	\\
0.0543791894171439	0.001434326171875	\\
0.0544235805921783	0.001190185546875	\\
0.0544679717672127	0.001129150390625	\\
0.0545123629422471	0.001129150390625	\\
0.0545567541172815	0.001007080078125	\\
0.0546011452923159	0.001373291015625	\\
0.0546455364673503	0.001739501953125	\\
0.0546899276423847	0.001434326171875	\\
0.0547343188174191	0.0010986328125	\\
0.0547787099924535	0.00128173828125	\\
0.0548231011674879	0.0013427734375	\\
0.0548674923425223	0.00140380859375	\\
0.0549118835175567	0.001007080078125	\\
0.0549562746925911	0.0010986328125	\\
0.0550006658676255	0.00091552734375	\\
0.0550450570426599	0.00079345703125	\\
0.0550894482176943	0.00128173828125	\\
0.0551338393927287	0.001190185546875	\\
0.0551782305677631	0.001129150390625	\\
0.0552226217427975	0.00115966796875	\\
0.0552670129178319	0.000823974609375	\\
0.0553114040928663	0.000732421875	\\
0.0553557952679007	0.00091552734375	\\
0.0554001864429351	0.00079345703125	\\
0.0554445776179695	0.000579833984375	\\
0.055488968793004	0.000946044921875	\\
0.0555333599680384	0.001129150390625	\\
0.0555777511430728	0.0009765625	\\
0.0556221423181072	0.000762939453125	\\
0.0556665334931416	0.00091552734375	\\
0.055710924668176	0.0010986328125	\\
0.0557553158432104	0.000732421875	\\
0.0557997070182448	0.000762939453125	\\
0.0558440981932792	0.00103759765625	\\
0.0558884893683136	0.001678466796875	\\
0.055932880543348	0.001617431640625	\\
0.0559772717183824	0.001373291015625	\\
0.0560216628934168	0.001373291015625	\\
0.0560660540684512	0.001495361328125	\\
0.0561104452434856	0.001922607421875	\\
0.05615483641852	0.001800537109375	\\
0.0561992275935544	0.001800537109375	\\
0.0562436187685888	0.00201416015625	\\
0.0562880099436232	0.00244140625	\\
0.0563324011186576	0.00201416015625	\\
0.056376792293692	0.001708984375	\\
0.0564211834687264	0.001739501953125	\\
0.0564655746437608	0.001373291015625	\\
0.0565099658187952	0.0015869140625	\\
0.0565543569938296	0.001922607421875	\\
0.056598748168864	0.001708984375	\\
0.0566431393438984	0.0013427734375	\\
0.0566875305189328	0.00189208984375	\\
0.0567319216939672	0.001800537109375	\\
0.0567763128690016	0.001190185546875	\\
0.056820704044036	0.001220703125	\\
0.0568650952190704	0.001434326171875	\\
0.0569094863941049	0.001434326171875	\\
0.0569538775691393	0.00152587890625	\\
0.0569982687441737	0.001556396484375	\\
0.0570426599192081	0.001556396484375	\\
0.0570870510942425	0.001556396484375	\\
0.0571314422692769	0.001617431640625	\\
0.0571758334443113	0.001953125	\\
0.0572202246193457	0.00177001953125	\\
0.0572646157943801	0.00177001953125	\\
0.0573090069694145	0.0023193359375	\\
0.0573533981444489	0.0018310546875	\\
0.0573977893194833	0.001708984375	\\
0.0574421804945177	0.002197265625	\\
0.0574865716695521	0.00225830078125	\\
0.0575309628445865	0.002105712890625	\\
0.0575753540196209	0.002197265625	\\
0.0576197451946553	0.002685546875	\\
0.0576641363696897	0.003021240234375	\\
0.0577085275447241	0.002960205078125	\\
0.0577529187197585	0.002899169921875	\\
0.0577973098947929	0.003204345703125	\\
0.0578417010698273	0.0035400390625	\\
0.0578860922448617	0.003387451171875	\\
0.0579304834198961	0.003265380859375	\\
0.0579748745949305	0.003662109375	\\
0.0580192657699649	0.003753662109375	\\
0.0580636569449993	0.00384521484375	\\
0.0581080481200337	0.00360107421875	\\
0.0581524392950681	0.00360107421875	\\
0.0581968304701025	0.00396728515625	\\
0.0582412216451369	0.003875732421875	\\
0.0582856128201713	0.00390625	\\
0.0583300039952058	0.004119873046875	\\
0.0583743951702402	0.003875732421875	\\
0.0584187863452746	0.003875732421875	\\
0.058463177520309	0.00360107421875	\\
0.0585075686953434	0.003570556640625	\\
0.0585519598703778	0.0035400390625	\\
0.0585963510454122	0.0032958984375	\\
0.0586407422204466	0.003662109375	\\
0.058685133395481	0.003265380859375	\\
0.0587295245705154	0.0032958984375	\\
0.0587739157455498	0.003570556640625	\\
0.0588183069205842	0.003204345703125	\\
0.0588626980956186	0.003387451171875	\\
0.058907089270653	0.003387451171875	\\
0.0589514804456874	0.0029296875	\\
0.0589958716207218	0.0029296875	\\
0.0590402627957562	0.00274658203125	\\
0.0590846539707906	0.0023193359375	\\
0.059129045145825	0.002105712890625	\\
0.0591734363208594	0.002044677734375	\\
0.0592178274958938	0.00189208984375	\\
0.0592622186709282	0.001739501953125	\\
0.0593066098459626	0.00140380859375	\\
0.059351001020997	0.001220703125	\\
0.0593953921960314	0.0008544921875	\\
0.0594397833710658	0.000762939453125	\\
0.0594841745461002	0.000946044921875	\\
0.0595285657211346	0.001068115234375	\\
0.059572956896169	0.00146484375	\\
0.0596173480712034	0.0018310546875	\\
0.0596617392462378	0.001922607421875	\\
0.0597061304212722	0.00201416015625	\\
0.0597505215963067	0.0023193359375	\\
0.0597949127713411	0.002532958984375	\\
0.0598393039463755	0.002197265625	\\
0.0598836951214099	0.00238037109375	\\
0.0599280862964443	0.002960205078125	\\
0.0599724774714787	0.00274658203125	\\
0.0600168686465131	0.002960205078125	\\
0.0600612598215475	0.0035400390625	\\
0.0601056509965819	0.0032958984375	\\
0.0601500421716163	0.00335693359375	\\
0.0601944333466507	0.00408935546875	\\
0.0602388245216851	0.004180908203125	\\
0.0602832156967195	0.0040283203125	\\
0.0603276068717539	0.00421142578125	\\
0.0603719980467883	0.004119873046875	\\
0.0604163892218227	0.004180908203125	\\
0.0604607803968571	0.004364013671875	\\
0.0605051715718915	0.004241943359375	\\
0.0605495627469259	0.004180908203125	\\
0.0605939539219603	0.004150390625	\\
0.0606383450969947	0.0042724609375	\\
0.0606827362720291	0.004364013671875	\\
0.0607271274470635	0.00408935546875	\\
0.0607715186220979	0.004241943359375	\\
0.0608159097971323	0.004425048828125	\\
0.0608603009721667	0.003875732421875	\\
0.0609046921472011	0.003814697265625	\\
0.0609490833222355	0.00360107421875	\\
0.0609934744972699	0.003387451171875	\\
0.0610378656723043	0.00390625	\\
0.0610822568473387	0.003753662109375	\\
0.0611266480223732	0.003753662109375	\\
0.0611710391974076	0.003875732421875	\\
0.061215430372442	0.00360107421875	\\
0.0612598215474764	0.00347900390625	\\
0.0613042127225108	0.0035400390625	\\
0.0613486038975452	0.003021240234375	\\
0.0613929950725796	0.00262451171875	\\
0.061437386247614	0.002716064453125	\\
0.0614817774226484	0.002716064453125	\\
0.0615261685976828	0.0025634765625	\\
0.0615705597727172	0.002960205078125	\\
0.0616149509477516	0.00274658203125	\\
0.061659342122786	0.002410888671875	\\
0.0617037332978204	0.002593994140625	\\
0.0617481244728548	0.0025634765625	\\
0.0617925156478892	0.0025634765625	\\
0.0618369068229236	0.002532958984375	\\
0.061881297997958	0.002288818359375	\\
0.0619256891729924	0.0028076171875	\\
0.0619700803480268	0.0029296875	\\
0.0620144715230612	0.002349853515625	\\
0.0620588626980956	0.001983642578125	\\
0.06210325387313	0.0020751953125	\\
0.0621476450481644	0.002288818359375	\\
0.0621920362231988	0.00238037109375	\\
0.0622364273982332	0.002471923828125	\\
0.0622808185732676	0.002410888671875	\\
0.062325209748302	0.0025634765625	\\
0.0623696009233364	0.0020751953125	\\
0.0624139920983708	0.001495361328125	\\
0.0624583832734052	0.001861572265625	\\
0.0625027744484396	0.00189208984375	\\
0.0625471656234741	0.001556396484375	\\
0.0625915567985085	0.001617431640625	\\
0.0626359479735429	0.0018310546875	\\
0.0626803391485773	0.00140380859375	\\
0.0627247303236117	0.00115966796875	\\
0.0627691214986461	0.001220703125	\\
0.0628135126736805	0.001068115234375	\\
0.0628579038487149	0.001190185546875	\\
0.0629022950237493	0.0008544921875	\\
0.0629466861987837	0.000457763671875	\\
0.0629910773738181	0.000335693359375	\\
0.0630354685488525	0.000396728515625	\\
0.0630798597238869	0.00054931640625	\\
0.0631242508989213	0.0003662109375	\\
0.0631686420739557	0.00067138671875	\\
0.0632130332489901	0.001068115234375	\\
0.0632574244240245	0.00054931640625	\\
0.0633018155990589	0.0008544921875	\\
0.0633462067740933	0.00115966796875	\\
0.0633905979491277	0.0009765625	\\
0.0634349891241621	0.0010986328125	\\
0.0634793802991965	0.0013427734375	\\
0.0635237714742309	0.00177001953125	\\
0.0635681626492653	0.001739501953125	\\
0.0636125538242997	0.0013427734375	\\
0.0636569449993341	0.001556396484375	\\
0.0637013361743685	0.00177001953125	\\
0.0637457273494029	0.002166748046875	\\
0.0637901185244373	0.00244140625	\\
0.0638345096994717	0.002471923828125	\\
0.0638789008745062	0.002227783203125	\\
0.0639232920495405	0.001861572265625	\\
0.063967683224575	0.00238037109375	\\
0.0640120743996094	0.002593994140625	\\
0.0640564655746438	0.00213623046875	\\
0.0641008567496782	0.002166748046875	\\
0.0641452479247126	0.002105712890625	\\
0.064189639099747	0.001495361328125	\\
0.0642340302747814	0.00189208984375	\\
0.0642784214498158	0.002166748046875	\\
0.0643228126248502	0.00201416015625	\\
0.0643672037998846	0.00177001953125	\\
0.064411594974919	0.001251220703125	\\
0.0644559861499534	0.001190185546875	\\
0.0645003773249878	0.0008544921875	\\
0.0645447685000222	0.000213623046875	\\
0.0645891596750566	0.000274658203125	\\
0.064633550850091	0.000274658203125	\\
0.0646779420251254	0.000213623046875	\\
0.0647223332001598	0.00054931640625	\\
0.0647667243751942	0.0006103515625	\\
0.0648111155502286	0.00067138671875	\\
0.064855506725263	0.00054931640625	\\
0.0648998979002974	0.000396728515625	\\
0.0649442890753318	0.00091552734375	\\
0.0649886802503662	0.0009765625	\\
0.0650330714254006	0.00115966796875	\\
0.065077462600435	0.00140380859375	\\
0.0651218537754694	0.0009765625	\\
0.0651662449505038	0.001373291015625	\\
0.0652106361255382	0.00201416015625	\\
0.0652550273005726	0.00146484375	\\
0.0652994184756071	0.001220703125	\\
0.0653438096506414	0.00164794921875	\\
0.0653882008256759	0.001678466796875	\\
0.0654325920007103	0.00201416015625	\\
0.0654769831757447	0.001922607421875	\\
0.0655213743507791	0.001922607421875	\\
0.0655657655258135	0.0028076171875	\\
0.0656101567008479	0.002685546875	\\
0.0656545478758823	0.00244140625	\\
0.0656989390509167	0.002471923828125	\\
0.0657433302259511	0.00238037109375	\\
0.0657877214009855	0.00299072265625	\\
0.0658321125760199	0.00286865234375	\\
0.0658765037510543	0.00225830078125	\\
0.0659208949260887	0.002105712890625	\\
0.0659652861011231	0.002227783203125	\\
0.0660096772761575	0.002044677734375	\\
0.0660540684511919	0.0020751953125	\\
0.0660984596262263	0.001678466796875	\\
0.0661428508012607	0.001129150390625	\\
0.0661872419762951	0.0015869140625	\\
0.0662316331513295	0.001190185546875	\\
0.0662760243263639	0.0008544921875	\\
0.0663204155013983	0.00091552734375	\\
0.0663648066764327	0.0009765625	\\
0.0664091978514671	0.001068115234375	\\
0.0664535890265015	0.000701904296875	\\
0.0664979802015359	0.000518798828125	\\
0.0665423713765703	0.0006103515625	\\
0.0665867625516047	0.000396728515625	\\
0.0666311537266392	3.0517578125e-05	\\
0.0666755449016735	-0.000152587890625	\\
0.066719936076708	-0.000274658203125	\\
0.0667643272517423	0.000213623046875	\\
0.0668087184267768	0.000213623046875	\\
0.0668531096018112	6.103515625e-05	\\
0.0668975007768456	0.000152587890625	\\
0.06694189195188	9.1552734375e-05	\\
0.0669862831269144	-3.0517578125e-05	\\
0.0670306743019488	-0.000274658203125	\\
0.0670750654769832	-0.000335693359375	\\
0.0671194566520176	-0.000152587890625	\\
0.067163847827052	-3.0517578125e-05	\\
0.0672082390020864	0.000213623046875	\\
0.0672526301771208	0.000457763671875	\\
0.0672970213521552	0.000213623046875	\\
0.0673414125271896	0.00079345703125	\\
0.067385803702224	0.0013427734375	\\
0.0674301948772584	0.001068115234375	\\
0.0674745860522928	0.0013427734375	\\
0.0675189772273272	0.001495361328125	\\
0.0675633684023616	0.001617431640625	\\
0.067607759577396	0.00146484375	\\
0.0676521507524304	0.001556396484375	\\
0.0676965419274648	0.00225830078125	\\
0.0677409331024992	0.002288818359375	\\
0.0677853242775336	0.002288818359375	\\
0.067829715452568	0.003082275390625	\\
0.0678741066276024	0.002655029296875	\\
0.0679184978026368	0.002410888671875	\\
0.0679628889776712	0.002716064453125	\\
0.0680072801527056	0.0023193359375	\\
0.0680516713277401	0.002349853515625	\\
0.0680960625027744	0.00213623046875	\\
0.0681404536778089	0.001983642578125	\\
0.0681848448528432	0.001861572265625	\\
0.0682292360278777	0.001556396484375	\\
0.0682736272029121	0.00115966796875	\\
0.0683180183779465	0.000701904296875	\\
0.0683624095529809	0.000701904296875	\\
0.0684068007280153	0.0009765625	\\
0.0684511919030497	0.000640869140625	\\
0.0684955830780841	0.00048828125	\\
0.0685399742531185	0.000579833984375	\\
0.0685843654281529	0.000457763671875	\\
0.0686287566031873	0.00079345703125	\\
0.0686731477782217	0.000946044921875	\\
0.0687175389532561	0.00103759765625	\\
0.0687619301282905	0.001129150390625	\\
0.0688063213033249	0.000762939453125	\\
0.0688507124783593	0.00067138671875	\\
0.0688951036533937	0.00054931640625	\\
0.0689394948284281	0.000213623046875	\\
0.0689838860034625	0.000640869140625	\\
0.0690282771784969	0.0008544921875	\\
0.0690726683535313	0.00042724609375	\\
0.0691170595285657	0.000335693359375	\\
0.0691614507036001	0.000732421875	\\
0.0692058418786345	0.00067138671875	\\
0.0692502330536689	0.000640869140625	\\
0.0692946242287033	0.00091552734375	\\
0.0693390154037377	0.001495361328125	\\
0.0693834065787721	0.001556396484375	\\
0.0694277977538065	0.0010986328125	\\
0.069472188928841	0.00128173828125	\\
0.0695165801038753	0.0010986328125	\\
0.0695609712789098	0.00103759765625	\\
0.0696053624539442	0.00177001953125	\\
0.0696497536289786	0.001800537109375	\\
0.069694144804013	0.0013427734375	\\
0.0697385359790474	0.00128173828125	\\
0.0697829271540818	0.001373291015625	\\
0.0698273183291162	0.00140380859375	\\
0.0698717095041506	0.001007080078125	\\
0.069916100679185	0.001007080078125	\\
0.0699604918542194	0.0009765625	\\
0.0700048830292538	0.000335693359375	\\
0.0700492742042882	0.000335693359375	\\
0.0700936653793226	0.00042724609375	\\
0.070138056554357	0.0003662109375	\\
0.0701824477293914	-3.0517578125e-05	\\
0.0702268389044258	-3.0517578125e-05	\\
0.0702712300794602	0.0003662109375	\\
0.0703156212544946	-3.0517578125e-05	\\
0.070360012429529	-0.00054931640625	\\
0.0704044036045634	-0.000335693359375	\\
0.0704487947795978	-0.00048828125	\\
0.0704931859546322	-0.000640869140625	\\
0.0705375771296666	-0.00018310546875	\\
0.070581968304701	-0.000213623046875	\\
0.0706263594797354	-0.000640869140625	\\
0.0706707506547698	-0.000732421875	\\
0.0707151418298042	-0.000579833984375	\\
0.0707595330048386	-0.000579833984375	\\
0.070803924179873	-0.00103759765625	\\
0.0708483153549074	-0.00103759765625	\\
0.0708927065299419	-0.00042724609375	\\
0.0709370977049762	-0.000213623046875	\\
0.0709814888800107	-0.00054931640625	\\
0.0710258800550451	-0.000640869140625	\\
0.0710702712300795	-9.1552734375e-05	\\
0.0711146624051139	0.000213623046875	\\
0.0711590535801483	0.000579833984375	\\
0.0712034447551827	0.000732421875	\\
0.0712478359302171	0.00042724609375	\\
0.0712922271052515	0.000335693359375	\\
0.0713366182802859	0.001007080078125	\\
0.0713810094553203	0.000946044921875	\\
0.0714254006303547	0.0010986328125	\\
0.0714697918053891	0.00140380859375	\\
0.0715141829804235	0.001068115234375	\\
0.0715585741554579	0.0008544921875	\\
0.0716029653304923	0.0009765625	\\
0.0716473565055267	0.00079345703125	\\
0.0716917476805611	0.000457763671875	\\
0.0717361388555955	0.000213623046875	\\
0.0717805300306299	-3.0517578125e-05	\\
0.0718249212056643	0	\\
0.0718693123806987	-0.000274658203125	\\
0.0719137035557331	-0.000762939453125	\\
0.0719580947307675	-0.000640869140625	\\
0.0720024859058019	-0.000732421875	\\
0.0720468770808363	-0.00103759765625	\\
0.0720912682558707	-0.001556396484375	\\
0.0721356594309051	-0.00164794921875	\\
0.0721800506059395	-0.001495361328125	\\
0.0722244417809739	-0.00164794921875	\\
0.0722688329560083	-0.001556396484375	\\
0.0723132241310428	-0.001495361328125	\\
0.0723576153060772	-0.00146484375	\\
0.0724020064811116	-0.00140380859375	\\
0.072446397656146	-0.001312255859375	\\
0.0724907888311804	-0.001007080078125	\\
0.0725351800062148	-0.000885009765625	\\
0.0725795711812492	-0.0009765625	\\
0.0726239623562836	-0.000823974609375	\\
0.072668353531318	-0.000762939453125	\\
0.0727127447063524	-0.0006103515625	\\
0.0727571358813868	-0.000335693359375	\\
0.0728015270564212	-0.00042724609375	\\
0.0728459182314556	-0.00018310546875	\\
0.07289030940649	9.1552734375e-05	\\
0.0729347005815244	0.00018310546875	\\
0.0729790917565588	0.00042724609375	\\
0.0730234829315932	0.000579833984375	\\
0.0730678741066276	0.000640869140625	\\
0.073112265281662	0.000579833984375	\\
0.0731566564566964	0.000823974609375	\\
0.0732010476317308	0.00067138671875	\\
0.0732454388067652	0.0008544921875	\\
0.0732898299817996	0.001220703125	\\
0.073334221156834	0.001312255859375	\\
0.0733786123318684	0.001190185546875	\\
0.0734230035069028	0.000823974609375	\\
0.0734673946819372	0.00103759765625	\\
0.0735117858569716	0.000823974609375	\\
0.073556177032006	0.00091552734375	\\
0.0736005682070404	0.00103759765625	\\
0.0736449593820748	0.000823974609375	\\
0.0736893505571092	0.000701904296875	\\
0.0737337417321437	0.00048828125	\\
0.0737781329071781	0.000457763671875	\\
0.0738225240822125	9.1552734375e-05	\\
0.0738669152572469	-6.103515625e-05	\\
0.0739113064322813	-0.000274658203125	\\
0.0739556976073157	-0.000244140625	\\
0.0740000887823501	-0.000457763671875	\\
0.0740444799573845	-0.001007080078125	\\
0.0740888711324189	-0.001251220703125	\\
0.0741332623074533	-0.001190185546875	\\
0.0741776534824877	-0.001068115234375	\\
0.0742220446575221	-0.0010986328125	\\
0.0742664358325565	-0.0013427734375	\\
0.0743108270075909	-0.00152587890625	\\
0.0743552181826253	-0.001312255859375	\\
0.0743996093576597	-0.001190185546875	\\
0.0744440005326941	-0.00140380859375	\\
0.0744883917077285	-0.00152587890625	\\
0.0745327828827629	-0.001556396484375	\\
0.0745771740577973	-0.00152587890625	\\
0.0746215652328317	-0.001251220703125	\\
0.0746659564078661	-0.001129150390625	\\
0.0747103475829005	-0.000640869140625	\\
0.0747547387579349	-0.000518798828125	\\
0.0747991299329693	-0.000701904296875	\\
0.0748435211080037	-0.0001220703125	\\
0.0748879122830381	-3.0517578125e-05	\\
0.0749323034580725	-0.00018310546875	\\
0.0749766946331069	0.000457763671875	\\
0.0750210858081413	-0.0001220703125	\\
0.0750654769831757	-0.0003662109375	\\
0.0751098681582102	-0.000244140625	\\
0.0751542593332445	9.1552734375e-05	\\
0.075198650508279	0.000244140625	\\
0.0752430416833134	0.000457763671875	\\
0.0752874328583478	0.000946044921875	\\
0.0753318240333822	0.0008544921875	\\
0.0753762152084166	0.001007080078125	\\
0.075420606383451	0.0010986328125	\\
0.0754649975584854	0.00146484375	\\
0.0755093887335198	0.001190185546875	\\
0.0755537799085542	0.00079345703125	\\
0.0755981710835886	0.001068115234375	\\
0.075642562258623	0.00067138671875	\\
0.0756869534336574	0.00048828125	\\
0.0757313446086918	0.00067138671875	\\
0.0757757357837262	0.0006103515625	\\
0.0758201269587606	0.000335693359375	\\
0.075864518133795	-3.0517578125e-05	\\
0.0759089093088294	0.000335693359375	\\
0.0759533004838638	-0.000213623046875	\\
0.0759976916588982	-0.000823974609375	\\
0.0760420828339326	-0.000335693359375	\\
0.076086474008967	-0.00079345703125	\\
0.0761308651840014	-0.00067138671875	\\
0.0761752563590358	-0.000335693359375	\\
0.0762196475340702	-0.000518798828125	\\
0.0762640387091046	-6.103515625e-05	\\
0.076308429884139	-9.1552734375e-05	\\
0.0763528210591734	-0.00048828125	\\
0.0763972122342078	-0.000335693359375	\\
0.0764416034092422	-0.00054931640625	\\
0.0764859945842766	-0.00042724609375	\\
0.0765303857593111	-0.000152587890625	\\
0.0765747769343454	-0.00042724609375	\\
0.0766191681093799	-0.00042724609375	\\
0.0766635592844143	-0.000244140625	\\
0.0767079504594487	-9.1552734375e-05	\\
0.0767523416344831	0.0001220703125	\\
0.0767967328095175	-0.00030517578125	\\
0.0768411239845519	-0.00018310546875	\\
0.0768855151595863	0.00030517578125	\\
0.0769299063346207	0.0003662109375	\\
0.0769742975096551	0.000213623046875	\\
0.0770186886846895	0.000213623046875	\\
0.0770630798597239	0.000335693359375	\\
0.0771074710347583	0.00067138671875	\\
0.0771518622097927	0.00091552734375	\\
0.0771962533848271	0.0010986328125	\\
0.0772406445598615	0.000732421875	\\
0.0772850357348959	0.000213623046875	\\
0.0773294269099303	0.000152587890625	\\
0.0773738180849647	0.000732421875	\\
0.0774182092599991	0.001312255859375	\\
0.0774626004350335	0.001068115234375	\\
0.0775069916100679	0.0006103515625	\\
0.0775513827851023	0.000701904296875	\\
0.0775957739601367	0.000701904296875	\\
0.0776401651351711	0.000701904296875	\\
0.0776845563102055	0.000579833984375	\\
0.0777289474852399	3.0517578125e-05	\\
0.0777733386602743	-9.1552734375e-05	\\
0.0778177298353087	0.000335693359375	\\
0.0778621210103431	0.00042724609375	\\
0.0779065121853775	0.00030517578125	\\
0.077950903360412	6.103515625e-05	\\
0.0779952945354463	-3.0517578125e-05	\\
0.0780396857104808	-0.0001220703125	\\
0.0780840768855152	-0.000152587890625	\\
0.0781284680605496	-0.000274658203125	\\
0.078172859235584	-0.00067138671875	\\
0.0782172504106184	-0.0008544921875	\\
0.0782616415856528	-0.001129150390625	\\
0.0783060327606872	-0.000823974609375	\\
0.0783504239357216	-0.000701904296875	\\
0.078394815110756	-0.001220703125	\\
0.0784392062857904	-0.001617431640625	\\
0.0784835974608248	-0.001739501953125	\\
0.0785279886358592	-0.001556396484375	\\
0.0785723798108936	-0.001373291015625	\\
0.078616770985928	-0.0013427734375	\\
0.0786611621609624	-0.001312255859375	\\
0.0787055533359968	-0.001129150390625	\\
0.0787499445110312	-0.00128173828125	\\
0.0787943356860656	-0.001251220703125	\\
0.0788387268611	-0.0009765625	\\
0.0788831180361344	-0.0009765625	\\
0.0789275092111688	-0.000701904296875	\\
0.0789719003862032	-0.00042724609375	\\
0.0790162915612376	0.0001220703125	\\
0.079060682736272	-6.103515625e-05	\\
0.0791050739113064	-0.000732421875	\\
0.0791494650863408	-0.0003662109375	\\
0.0791938562613752	-0.000244140625	\\
0.0792382474364096	-0.000335693359375	\\
0.0792826386114441	-0.000335693359375	\\
0.0793270297864784	-0.000335693359375	\\
0.0793714209615129	-0.000244140625	\\
0.0794158121365472	-0.000152587890625	\\
0.0794602033115817	-0.000701904296875	\\
0.0795045944866161	-0.000762939453125	\\
0.0795489856616505	-0.00091552734375	\\
0.0795933768366849	-0.00140380859375	\\
0.0796377680117193	-0.001800537109375	\\
0.0796821591867537	-0.0015869140625	\\
0.0797265503617881	-0.001251220703125	\\
0.0797709415368225	-0.00140380859375	\\
0.0798153327118569	-0.001739501953125	\\
0.0798597238868913	-0.0013427734375	\\
0.0799041150619257	-0.001068115234375	\\
0.0799485062369601	-0.001800537109375	\\
0.0799928974119945	-0.001708984375	\\
0.0800372885870289	-0.00177001953125	\\
0.0800816797620633	-0.00225830078125	\\
0.0801260709370977	-0.00201416015625	\\
0.0801704621121321	-0.001922607421875	\\
0.0802148532871665	-0.0020751953125	\\
0.0802592444622009	-0.0018310546875	\\
0.0803036356372353	-0.001312255859375	\\
0.0803480268122697	-0.00164794921875	\\
0.0803924179873041	-0.001617431640625	\\
0.0804368091623385	-0.00140380859375	\\
0.0804812003373729	-0.001373291015625	\\
0.0805255915124073	-0.00067138671875	\\
0.0805699826874417	-0.001007080078125	\\
0.0806143738624761	-0.001220703125	\\
0.0806587650375105	-0.00067138671875	\\
0.080703156212545	-0.000274658203125	\\
0.0807475473875793	-0.000274658203125	\\
0.0807919385626138	-0.000244140625	\\
0.0808363297376481	9.1552734375e-05	\\
0.0808807209126826	0.00018310546875	\\
0.080925112087717	-0.00018310546875	\\
0.0809695032627514	-0.000335693359375	\\
0.0810138944377858	0.000213623046875	\\
0.0810582856128202	0.000396728515625	\\
0.0811026767878546	-3.0517578125e-05	\\
0.081147067962889	0.00018310546875	\\
0.0811914591379234	0.0006103515625	\\
0.0812358503129578	0.00018310546875	\\
0.0812802414879922	3.0517578125e-05	\\
0.0813246326630266	-9.1552734375e-05	\\
0.081369023838061	-0.00030517578125	\\
0.0814134150130954	-0.0001220703125	\\
0.0814578061881298	-0.000152587890625	\\
0.0815021973631642	-0.000244140625	\\
0.0815465885381986	-0.00048828125	\\
0.081590979713233	-0.000640869140625	\\
0.0816353708882674	-0.00054931640625	\\
0.0816797620633018	-0.000457763671875	\\
0.0817241532383362	-0.000579833984375	\\
0.0817685444133706	-0.0008544921875	\\
0.081812935588405	-0.000762939453125	\\
0.0818573267634394	-0.001129150390625	\\
0.0819017179384738	-0.001678466796875	\\
0.0819461091135082	-0.001556396484375	\\
0.0819905002885426	-0.001708984375	\\
0.082034891463577	-0.001800537109375	\\
0.0820792826386114	-0.0018310546875	\\
0.0821236738136459	-0.001983642578125	\\
0.0821680649886802	-0.001953125	\\
0.0822124561637147	-0.001983642578125	\\
0.0822568473387491	-0.001556396484375	\\
0.0823012385137835	-0.0015869140625	\\
0.0823456296888179	-0.001739501953125	\\
0.0823900208638523	-0.00189208984375	\\
0.0824344120388867	-0.0015869140625	\\
0.0824788032139211	-0.001556396484375	\\
0.0825231943889555	-0.00152587890625	\\
0.0825675855639899	-0.001251220703125	\\
0.0826119767390243	-0.001068115234375	\\
0.0826563679140587	-0.000640869140625	\\
0.0827007590890931	-0.0008544921875	\\
0.0827451502641275	-0.001007080078125	\\
0.0827895414391619	-0.00054931640625	\\
0.0828339326141963	3.0517578125e-05	\\
0.0828783237892307	0	\\
0.0829227149642651	-0.000274658203125	\\
0.0829671061392995	-0.000396728515625	\\
0.0830114973143339	-0.000244140625	\\
0.0830558884893683	0.00042724609375	\\
0.0831002796644027	0.00030517578125	\\
0.0831446708394371	0.00030517578125	\\
0.0831890620144715	0.000518798828125	\\
0.0832334531895059	0.00079345703125	\\
0.0832778443645403	0.000885009765625	\\
0.0833222355395747	0.000457763671875	\\
0.0833666267146091	0.00030517578125	\\
0.0834110178896435	0.000457763671875	\\
0.0834554090646779	0.00048828125	\\
0.0834998002397123	0.000152587890625	\\
0.0835441914147468	-0.000396728515625	\\
0.0835885825897811	-0.000396728515625	\\
0.0836329737648156	-0.0003662109375	\\
0.08367736493985	-0.000579833984375	\\
0.0837217561148844	-0.00054931640625	\\
0.0837661472899188	-0.000823974609375	\\
0.0838105384649532	-0.001007080078125	\\
0.0838549296399876	-0.001220703125	\\
0.083899320815022	-0.0013427734375	\\
0.0839437119900564	-0.001068115234375	\\
0.0839881031650908	-0.00103759765625	\\
0.0840324943401252	-0.0010986328125	\\
0.0840768855151596	-0.000885009765625	\\
0.084121276690194	-0.000946044921875	\\
0.0841656678652284	-0.00067138671875	\\
0.0842100590402628	-0.000946044921875	\\
0.0842544502152972	-0.00103759765625	\\
0.0842988413903316	-0.0008544921875	\\
0.084343232565366	-0.0006103515625	\\
0.0843876237404004	-0.000396728515625	\\
0.0844320149154348	-0.0003662109375	\\
0.0844764060904692	6.103515625e-05	\\
0.0845207972655036	6.103515625e-05	\\
0.084565188440538	0.0003662109375	\\
0.0846095796155724	0.0003662109375	\\
0.0846539707906068	0.000244140625	\\
0.0846983619656412	0.000457763671875	\\
0.0847427531406756	0.000396728515625	\\
0.08478714431571	0.000213623046875	\\
0.0848315354907444	0.000244140625	\\
0.0848759266657788	0.000640869140625	\\
0.0849203178408132	0.00103759765625	\\
0.0849647090158477	0.00128173828125	\\
0.0850091001908821	0.001312255859375	\\
0.0850534913659165	0.001190185546875	\\
0.0850978825409509	0.001434326171875	\\
0.0851422737159853	0.0018310546875	\\
0.0851866648910197	0.00146484375	\\
0.0852310560660541	0.00146484375	\\
0.0852754472410885	0.001617431640625	\\
0.0853198384161229	0.00128173828125	\\
0.0853642295911573	0.000885009765625	\\
0.0854086207661917	0.000762939453125	\\
0.0854530119412261	0.0008544921875	\\
0.0854974031162605	0.00054931640625	\\
0.0855417942912949	0.00018310546875	\\
0.0855861854663293	-0.000244140625	\\
0.0856305766413637	-0.000335693359375	\\
0.0856749678163981	6.103515625e-05	\\
0.0857193589914325	-0.000518798828125	\\
0.0857637501664669	-0.000579833984375	\\
0.0858081413415013	-0.000244140625	\\
0.0858525325165357	-0.000732421875	\\
0.0858969236915701	-0.0013427734375	\\
0.0859413148666045	-0.001007080078125	\\
0.0859857060416389	-0.00067138671875	\\
0.0860300972166733	-0.001220703125	\\
0.0860744883917077	-0.00103759765625	\\
0.0861188795667421	-0.000823974609375	\\
0.0861632707417765	-0.00115966796875	\\
0.0862076619168109	-0.0008544921875	\\
0.0862520530918453	-0.000946044921875	\\
0.0862964442668797	-0.001190185546875	\\
0.0863408354419141	-0.0010986328125	\\
0.0863852266169486	-0.001007080078125	\\
0.086429617791983	-0.000885009765625	\\
0.0864740089670174	-0.001068115234375	\\
0.0865184001420518	-0.00091552734375	\\
0.0865627913170862	-0.000762939453125	\\
0.0866071824921206	-0.0008544921875	\\
0.086651573667155	-0.0010986328125	\\
0.0866959648421894	-0.001068115234375	\\
0.0867403560172238	-0.001220703125	\\
0.0867847471922582	-0.00128173828125	\\
0.0868291383672926	-0.00048828125	\\
0.086873529542327	-0.000274658203125	\\
0.0869179207173614	-0.0006103515625	\\
0.0869623118923958	-0.000213623046875	\\
0.0870067030674302	-0.000335693359375	\\
0.0870510942424646	-0.0010986328125	\\
0.087095485417499	-0.001129150390625	\\
0.0871398765925334	-0.0010986328125	\\
0.0871842677675678	-0.0009765625	\\
0.0872286589426022	-0.00140380859375	\\
0.0872730501176366	-0.001556396484375	\\
0.087317441292671	-0.000946044921875	\\
0.0873618324677054	-0.0008544921875	\\
0.0874062236427398	-0.001068115234375	\\
0.0874506148177742	-0.00115966796875	\\
0.0874950059928086	-0.0010986328125	\\
0.087539397167843	-0.0010986328125	\\
0.0875837883428774	-0.001373291015625	\\
0.0876281795179118	-0.00128173828125	\\
0.0876725706929462	-0.00103759765625	\\
0.0877169618679806	-0.0008544921875	\\
0.0877613530430151	-0.001312255859375	\\
0.0878057442180495	-0.0006103515625	\\
0.0878501353930839	-0.0003662109375	\\
0.0878945265681183	-0.0006103515625	\\
0.0879389177431527	-0.000396728515625	\\
0.0879833089181871	-0.000579833984375	\\
0.0880277000932215	-0.00048828125	\\
0.0880720912682559	-0.000457763671875	\\
0.0881164824432903	-0.000396728515625	\\
0.0881608736183247	-0.0001220703125	\\
0.0882052647933591	-0.00018310546875	\\
0.0882496559683935	-0.00067138671875	\\
0.0882940471434279	-0.00103759765625	\\
0.0883384383184623	-0.00079345703125	\\
0.0883828294934967	-0.0008544921875	\\
0.0884272206685311	-0.000701904296875	\\
0.0884716118435655	-0.00091552734375	\\
0.0885160030185999	-0.000640869140625	\\
0.0885603941936343	-0.000396728515625	\\
0.0886047853686687	-0.0006103515625	\\
0.0886491765437031	-0.000762939453125	\\
0.0886935677187375	-0.000701904296875	\\
0.0887379588937719	-0.000885009765625	\\
0.0887823500688063	-0.001373291015625	\\
0.0888267412438407	-0.00115966796875	\\
0.0888711324188751	-0.00067138671875	\\
0.0889155235939095	-0.00103759765625	\\
0.0889599147689439	-0.001220703125	\\
0.0890043059439783	-0.00103759765625	\\
0.0890486971190127	-0.0009765625	\\
0.0890930882940471	-0.000946044921875	\\
0.0891374794690815	-0.00189208984375	\\
0.089181870644116	-0.002197265625	\\
0.0892262618191504	-0.00213623046875	\\
0.0892706529941848	-0.002105712890625	\\
0.0893150441692192	-0.0023193359375	\\
0.0893594353442536	-0.00146484375	\\
0.089403826519288	-0.00164794921875	\\
0.0894482176943224	-0.002227783203125	\\
0.0894926088693568	-0.00262451171875	\\
0.0895370000443912	-0.00244140625	\\
0.0895813912194256	-0.0020751953125	\\
0.08962578239446	-0.0028076171875	\\
0.0896701735694944	-0.00244140625	\\
0.0897145647445288	-0.00213623046875	\\
0.0897589559195632	-0.002349853515625	\\
0.0898033470945976	-0.002532958984375	\\
0.089847738269632	-0.002288818359375	\\
0.0898921294446664	-0.001861572265625	\\
0.0899365206197008	-0.001922607421875	\\
0.0899809117947352	-0.002197265625	\\
0.0900253029697696	-0.002655029296875	\\
0.090069694144804	-0.0023193359375	\\
0.0901140853198384	-0.002349853515625	\\
0.0901584764948728	-0.002197265625	\\
0.0902028676699072	-0.001434326171875	\\
0.0902472588449416	-0.001373291015625	\\
0.090291650019976	-0.00103759765625	\\
0.0903360411950104	-0.0009765625	\\
0.0903804323700448	-0.001068115234375	\\
0.0904248235450792	-0.000640869140625	\\
0.0904692147201136	-0.000579833984375	\\
0.0905136058951481	-0.001068115234375	\\
0.0905579970701824	-0.0009765625	\\
0.0906023882452169	-0.00079345703125	\\
0.0906467794202512	-0.0008544921875	\\
0.0906911705952857	-0.00091552734375	\\
0.0907355617703201	-0.001251220703125	\\
0.0907799529453545	-0.001373291015625	\\
0.0908243441203889	-0.00140380859375	\\
0.0908687352954233	-0.001129150390625	\\
0.0909131264704577	-0.000732421875	\\
0.0909575176454921	-0.00054931640625	\\
0.0910019088205265	-0.00067138671875	\\
0.0910462999955609	-0.00091552734375	\\
0.0910906911705953	-0.001312255859375	\\
0.0911350823456297	-0.001068115234375	\\
0.0911794735206641	-0.00115966796875	\\
0.0912238646956985	-0.00146484375	\\
0.0912682558707329	-0.0015869140625	\\
0.0913126470457673	-0.001312255859375	\\
0.0913570382208017	-0.000885009765625	\\
0.0914014293958361	-0.000732421875	\\
0.0914458205708705	-0.000885009765625	\\
0.0914902117459049	-0.000823974609375	\\
0.0915346029209393	-0.000885009765625	\\
0.0915789940959737	-0.00128173828125	\\
0.0916233852710081	-0.001434326171875	\\
0.0916677764460425	-0.001129150390625	\\
0.0917121676210769	-0.001129150390625	\\
0.0917565587961113	-0.0013427734375	\\
0.0918009499711457	-0.001312255859375	\\
0.0918453411461801	-0.00115966796875	\\
0.0918897323212145	-0.000762939453125	\\
0.091934123496249	-0.000762939453125	\\
0.0919785146712833	-0.00079345703125	\\
0.0920229058463178	-0.00042724609375	\\
0.0920672970213521	-0.000152587890625	\\
0.0921116881963866	-0.00048828125	\\
0.092156079371421	-0.001068115234375	\\
0.0922004705464554	-0.0008544921875	\\
0.0922448617214898	-0.000946044921875	\\
0.0922892528965242	-0.001251220703125	\\
0.0923336440715586	-0.001190185546875	\\
0.092378035246593	-0.000762939453125	\\
0.0924224264216274	-0.000518798828125	\\
0.0924668175966618	-0.000579833984375	\\
0.0925112087716962	-0.000152587890625	\\
0.0925555999467306	-0.000274658203125	\\
0.092599991121765	-0.00042724609375	\\
0.0926443822967994	-0.00030517578125	\\
0.0926887734718338	-0.0001220703125	\\
0.0927331646468682	-0.0001220703125	\\
0.0927775558219026	-0.000213623046875	\\
0.092821946996937	-0.000396728515625	\\
0.0928663381719714	-0.00048828125	\\
0.0929107293470058	-0.0001220703125	\\
0.0929551205220402	0.000274658203125	\\
0.0929995116970746	-0.0001220703125	\\
0.093043902872109	0.0001220703125	\\
0.0930882940471434	-0.00048828125	\\
0.0931326852221778	-0.000701904296875	\\
0.0931770763972122	-0.00042724609375	\\
0.0932214675722466	-0.00091552734375	\\
0.093265858747281	-0.0010986328125	\\
0.0933102499223154	-0.0010986328125	\\
0.0933546410973499	-0.000701904296875	\\
0.0933990322723842	-0.00115966796875	\\
0.0934434234474187	-0.001373291015625	\\
0.0934878146224531	-0.001190185546875	\\
0.0935322057974875	-0.001312255859375	\\
0.0935765969725219	-0.001129150390625	\\
0.0936209881475563	-0.001800537109375	\\
0.0936653793225907	-0.00189208984375	\\
0.0937097704976251	-0.002288818359375	\\
0.0937541616726595	-0.002838134765625	\\
0.0937985528476939	-0.002349853515625	\\
0.0938429440227283	-0.002197265625	\\
0.0938873351977627	-0.002227783203125	\\
0.0939317263727971	-0.00177001953125	\\
0.0939761175478315	-0.0013427734375	\\
0.0940205087228659	-0.001556396484375	\\
0.0940648998979003	-0.00115966796875	\\
0.0941092910729347	-0.001007080078125	\\
0.0941536822479691	-0.001922607421875	\\
0.0941980734230035	-0.002288818359375	\\
0.0942424645980379	-0.00238037109375	\\
0.0942868557730723	-0.00244140625	\\
0.0943312469481067	-0.0023193359375	\\
0.0943756381231411	-0.00189208984375	\\
0.0944200292981755	-0.00164794921875	\\
0.0944644204732099	-0.00146484375	\\
0.0945088116482443	-0.0015869140625	\\
0.0945532028232787	-0.00201416015625	\\
0.0945975939983131	-0.00177001953125	\\
0.0946419851733475	-0.0013427734375	\\
0.0946863763483819	-0.001678466796875	\\
0.0947307675234163	-0.001739501953125	\\
0.0947751586984508	-0.0018310546875	\\
0.0948195498734851	-0.001617431640625	\\
0.0948639410485196	-0.001434326171875	\\
0.094908332223554	-0.001708984375	\\
0.0949527233985884	-0.001434326171875	\\
0.0949971145736228	-0.001373291015625	\\
0.0950415057486572	-0.001861572265625	\\
0.0950858969236916	-0.001983642578125	\\
0.095130288098726	-0.0018310546875	\\
0.0951746792737604	-0.0020751953125	\\
0.0952190704487948	-0.001953125	\\
0.0952634616238292	-0.002197265625	\\
0.0953078527988636	-0.0023193359375	\\
0.095352243973898	-0.002044677734375	\\
0.0953966351489324	-0.002044677734375	\\
0.0954410263239668	-0.00213623046875	\\
0.0954854174990012	-0.001983642578125	\\
0.0955298086740356	-0.001739501953125	\\
0.09557419984907	-0.001434326171875	\\
0.0956185910241044	-0.001251220703125	\\
0.0956629821991388	-0.001434326171875	\\
0.0957073733741732	-0.001708984375	\\
0.0957517645492076	-0.001556396484375	\\
0.095796155724242	-0.001373291015625	\\
0.0958405468992764	-0.001678466796875	\\
0.0958849380743108	-0.001983642578125	\\
0.0959293292493452	-0.00152587890625	\\
0.0959737204243796	-0.001556396484375	\\
0.096018111599414	-0.001708984375	\\
0.0960625027744484	-0.001220703125	\\
0.0961068939494828	-0.00140380859375	\\
0.0961512851245172	-0.001861572265625	\\
0.0961956762995517	-0.0018310546875	\\
0.0962400674745861	-0.001739501953125	\\
0.0962844586496205	-0.001800537109375	\\
0.0963288498246549	-0.001617431640625	\\
0.0963732409996893	-0.001373291015625	\\
0.0964176321747237	-0.00140380859375	\\
0.0964620233497581	-0.001495361328125	\\
0.0965064145247925	-0.00152587890625	\\
0.0965508056998269	-0.001708984375	\\
0.0965951968748613	-0.002410888671875	\\
0.0966395880498957	-0.0028076171875	\\
0.0966839792249301	-0.002960205078125	\\
0.0967283703999645	-0.003326416015625	\\
0.0967727615749989	-0.00311279296875	\\
0.0968171527500333	-0.00311279296875	\\
0.0968615439250677	-0.0028076171875	\\
0.0969059351001021	-0.002410888671875	\\
0.0969503262751365	-0.002288818359375	\\
0.0969947174501709	-0.002166748046875	\\
0.0970391086252053	-0.002410888671875	\\
0.0970834998002397	-0.002655029296875	\\
0.0971278909752741	-0.00250244140625	\\
0.0971722821503085	-0.0025634765625	\\
0.0972166733253429	-0.00250244140625	\\
0.0972610645003773	-0.002685546875	\\
0.0973054556754117	-0.00274658203125	\\
0.0973498468504461	-0.00262451171875	\\
0.0973942380254805	-0.00274658203125	\\
0.0974386292005149	-0.002685546875	\\
0.0974830203755493	-0.00274658203125	\\
0.0975274115505837	-0.002899169921875	\\
0.0975718027256181	-0.0030517578125	\\
0.0976161939006526	-0.00274658203125	\\
0.097660585075687	-0.0028076171875	\\
0.0977049762507214	-0.002716064453125	\\
0.0977493674257558	-0.00244140625	\\
0.0977937586007902	-0.002777099609375	\\
0.0978381497758246	-0.002777099609375	\\
0.097882540950859	-0.00299072265625	\\
0.0979269321258934	-0.00286865234375	\\
0.0979713233009278	-0.002593994140625	\\
0.0980157144759622	-0.002685546875	\\
0.0980601056509966	-0.002288818359375	\\
0.098104496826031	-0.002532958984375	\\
0.0981488880010654	-0.0025634765625	\\
0.0981932791760998	-0.002655029296875	\\
0.0982376703511342	-0.002838134765625	\\
0.0982820615261686	-0.002716064453125	\\
0.098326452701203	-0.002655029296875	\\
0.0983708438762374	-0.002532958984375	\\
0.0984152350512718	-0.002288818359375	\\
0.0984596262263062	-0.002166748046875	\\
0.0985040174013406	-0.002655029296875	\\
0.098548408576375	-0.002471923828125	\\
0.0985927997514094	-0.002288818359375	\\
0.0986371909264438	-0.00213623046875	\\
0.0986815821014782	-0.0020751953125	\\
0.0987259732765126	-0.001953125	\\
0.098770364451547	-0.00140380859375	\\
0.0988147556265814	-0.00115966796875	\\
0.0988591468016158	-0.001220703125	\\
0.0989035379766502	-0.00103759765625	\\
0.0989479291516846	-0.00115966796875	\\
0.098992320326719	-0.00128173828125	\\
0.0990367115017535	-0.0013427734375	\\
0.0990811026767879	-0.001251220703125	\\
0.0991254938518223	-0.0008544921875	\\
0.0991698850268567	-0.0009765625	\\
0.0992142762018911	-0.00079345703125	\\
0.0992586673769255	-0.00048828125	\\
0.0993030585519599	0	\\
0.0993474497269943	3.0517578125e-05	\\
0.0993918409020287	-0.000244140625	\\
0.0994362320770631	-0.000579833984375	\\
0.0994806232520975	-0.00115966796875	\\
0.0995250144271319	-0.001190185546875	\\
0.0995694056021663	-0.000946044921875	\\
0.0996137967772007	-0.000946044921875	\\
0.0996581879522351	-0.0006103515625	\\
0.0997025791272695	-0.000335693359375	\\
0.0997469703023039	-0.00042724609375	\\
0.0997913614773383	-0.00042724609375	\\
0.0998357526523727	3.0517578125e-05	\\
0.0998801438274071	9.1552734375e-05	\\
0.0999245350024415	9.1552734375e-05	\\
0.0999689261774759	0.000457763671875	\\
0.10001331735251	-3.0517578125e-05	\\
0.100057708527545	0.0003662109375	\\
0.100102099702579	0.00048828125	\\
0.100146490877614	0.000579833984375	\\
0.100190882052648	0.000885009765625	\\
0.100235273227682	0.001129150390625	\\
0.100279664402717	0.001251220703125	\\
0.100324055577751	0.00152587890625	\\
0.100368446752786	0.001739501953125	\\
0.10041283792782	0.001739501953125	\\
0.100457229102854	0.00164794921875	\\
0.100501620277889	0.001708984375	\\
0.100546011452923	0.001495361328125	\\
0.100590402627958	0.0013427734375	\\
0.100634793802992	0.001068115234375	\\
0.100679184978026	0.001007080078125	\\
0.100723576153061	0.00079345703125	\\
0.100767967328095	0.00054931640625	\\
0.10081235850313	0.001068115234375	\\
0.100856749678164	0.00115966796875	\\
0.100901140853198	0.000762939453125	\\
0.100945532028233	0.000732421875	\\
0.100989923203267	0.000274658203125	\\
0.101034314378302	-0.0006103515625	\\
0.101078705553336	-0.000885009765625	\\
0.10112309672837	-0.00091552734375	\\
0.101167487903405	-0.00103759765625	\\
0.101211879078439	-0.00067138671875	\\
0.101256270253474	-0.0006103515625	\\
0.101300661428508	-0.00079345703125	\\
0.101345052603542	-0.000762939453125	\\
0.101389443778577	-0.001190185546875	\\
0.101433834953611	-0.001190185546875	\\
0.101478226128646	-0.001312255859375	\\
0.10152261730368	-0.001617431640625	\\
0.101567008478714	-0.001373291015625	\\
0.101611399653749	-0.001556396484375	\\
0.101655790828783	-0.0013427734375	\\
0.101700182003818	-0.001007080078125	\\
0.101744573178852	-0.00152587890625	\\
0.101788964353886	-0.001220703125	\\
0.101833355528921	-0.001556396484375	\\
0.101877746703955	-0.001953125	\\
0.10192213787899	-0.001800537109375	\\
0.101966529054024	-0.0018310546875	\\
0.102010920229058	-0.00177001953125	\\
0.102055311404093	-0.00189208984375	\\
0.102099702579127	-0.001922607421875	\\
0.102144093754162	-0.001922607421875	\\
0.102188484929196	-0.001373291015625	\\
0.10223287610423	-0.001556396484375	\\
0.102277267279265	-0.001556396484375	\\
0.102321658454299	-0.001312255859375	\\
0.102366049629334	-0.00152587890625	\\
0.102410440804368	-0.00146484375	\\
0.102454831979403	-0.001312255859375	\\
0.102499223154437	-0.0010986328125	\\
0.102543614329471	-0.000518798828125	\\
0.102588005504506	-0.00048828125	\\
0.10263239667954	-0.0006103515625	\\
0.102676787854575	-0.000244140625	\\
0.102721179029609	-0.000396728515625	\\
0.102765570204643	-0.000762939453125	\\
0.102809961379678	-0.0003662109375	\\
0.102854352554712	-0.00018310546875	\\
0.102898743729747	-0.0001220703125	\\
0.102943134904781	3.0517578125e-05	\\
0.102987526079815	9.1552734375e-05	\\
0.10303191725485	-0.00030517578125	\\
0.103076308429884	-0.000335693359375	\\
0.103120699604919	-0.000396728515625	\\
0.103165090779953	-0.00042724609375	\\
0.103209481954987	-0.000579833984375	\\
0.103253873130022	-0.000885009765625	\\
0.103298264305056	-0.00091552734375	\\
0.103342655480091	-0.00115966796875	\\
0.103387046655125	-0.00115966796875	\\
0.103431437830159	-0.000701904296875	\\
0.103475829005194	-0.000823974609375	\\
0.103520220180228	-0.00079345703125	\\
0.103564611355263	-0.000701904296875	\\
0.103609002530297	-0.0006103515625	\\
0.103653393705331	-0.00054931640625	\\
0.103697784880366	-0.000946044921875	\\
0.1037421760554	-0.001220703125	\\
0.103786567230435	-0.001556396484375	\\
0.103830958405469	-0.001434326171875	\\
0.103875349580503	-0.001190185546875	\\
0.103919740755538	-0.001129150390625	\\
0.103964131930572	-0.000732421875	\\
0.104008523105607	-0.000518798828125	\\
0.104052914280641	-0.000946044921875	\\
0.104097305455675	-0.0009765625	\\
0.10414169663071	-0.00079345703125	\\
0.104186087805744	-0.001220703125	\\
0.104230478980779	-0.001373291015625	\\
0.104274870155813	-0.000885009765625	\\
0.104319261330847	-0.000244140625	\\
0.104363652505882	-0.000335693359375	\\
0.104408043680916	-0.00054931640625	\\
0.104452434855951	-0.000579833984375	\\
0.104496826030985	-0.00048828125	\\
0.104541217206019	-0.00030517578125	\\
0.104585608381054	-0.000579833984375	\\
0.104629999556088	-0.000823974609375	\\
0.104674390731123	-0.000823974609375	\\
0.104718781906157	-0.000396728515625	\\
0.104763173081191	-3.0517578125e-05	\\
0.104807564256226	-0.00030517578125	\\
0.10485195543126	-0.0006103515625	\\
0.104896346606295	-0.000823974609375	\\
0.104940737781329	-0.001007080078125	\\
0.104985128956363	-0.001312255859375	\\
0.105029520131398	-0.001373291015625	\\
0.105073911306432	-0.00140380859375	\\
0.105118302481467	-0.001556396484375	\\
0.105162693656501	-0.0013427734375	\\
0.105207084831535	-0.001190185546875	\\
0.10525147600657	-0.001373291015625	\\
0.105295867181604	-0.00140380859375	\\
0.105340258356639	-0.001312255859375	\\
0.105384649531673	-0.00115966796875	\\
0.105429040706708	-0.0015869140625	\\
0.105473431881742	-0.00164794921875	\\
0.105517823056776	-0.00128173828125	\\
0.105562214231811	-0.00128173828125	\\
0.105606605406845	-0.000946044921875	\\
0.10565099658188	-0.000823974609375	\\
0.105695387756914	-0.000762939453125	\\
0.105739778931948	-0.0008544921875	\\
0.105784170106983	-0.0008544921875	\\
0.105828561282017	-0.000335693359375	\\
0.105872952457052	0.0001220703125	\\
0.105917343632086	0.00042724609375	\\
0.10596173480712	0.000701904296875	\\
0.106006125982155	0.00115966796875	\\
0.106050517157189	0.0015869140625	\\
0.106094908332224	0.00079345703125	\\
0.106139299507258	0.000640869140625	\\
0.106183690682292	0.001251220703125	\\
0.106228081857327	0.00128173828125	\\
0.106272473032361	0.001007080078125	\\
0.106316864207396	0.001251220703125	\\
0.10636125538243	0.001373291015625	\\
0.106405646557464	0.00164794921875	\\
0.106450037732499	0.001617431640625	\\
0.106494428907533	0.001312255859375	\\
0.106538820082568	0.000823974609375	\\
0.106583211257602	0.00054931640625	\\
0.106627602432636	0.000396728515625	\\
0.106671993607671	0.001068115234375	\\
0.106716384782705	0.001220703125	\\
0.10676077595774	0.001312255859375	\\
0.106805167132774	0.00189208984375	\\
0.106849558307808	0.001495361328125	\\
0.106893949482843	0.00177001953125	\\
0.106938340657877	0.001739501953125	\\
0.106982731832912	0.0010986328125	\\
0.107027123007946	0.00079345703125	\\
0.10707151418298	0.000518798828125	\\
0.107115905358015	0.001129150390625	\\
0.107160296533049	0.001708984375	\\
0.107204687708084	0.001861572265625	\\
0.107249078883118	0.00152587890625	\\
0.107293470058152	0.001495361328125	\\
0.107337861233187	0.00140380859375	\\
0.107382252408221	0.001251220703125	\\
0.107426643583256	0.00152587890625	\\
0.10747103475829	0.0010986328125	\\
0.107515425933324	0.001617431640625	\\
0.107559817108359	0.002288818359375	\\
0.107604208283393	0.00244140625	\\
0.107648599458428	0.002593994140625	\\
0.107692990633462	0.002685546875	\\
0.107737381808496	0.00262451171875	\\
0.107781772983531	0.001953125	\\
0.107826164158565	0.001861572265625	\\
0.1078705553336	0.0025634765625	\\
0.107914946508634	0.0023193359375	\\
0.107959337683668	0.002288818359375	\\
0.108003728858703	0.002655029296875	\\
0.108048120033737	0.0025634765625	\\
0.108092511208772	0.002777099609375	\\
0.108136902383806	0.002593994140625	\\
0.108181293558841	0.002655029296875	\\
0.108225684733875	0.003021240234375	\\
0.108270075908909	0.003387451171875	\\
0.108314467083944	0.002960205078125	\\
0.108358858258978	0.00341796875	\\
0.108403249434013	0.0032958984375	\\
0.108447640609047	0.002899169921875	\\
0.108492031784081	0.002899169921875	\\
0.108536422959116	0.002288818359375	\\
0.10858081413415	0.0013427734375	\\
0.108625205309185	0.0013427734375	\\
0.108669596484219	0.001007080078125	\\
0.108713987659253	0.0006103515625	\\
0.108758378834288	0.000946044921875	\\
0.108802770009322	0.001220703125	\\
0.108847161184357	0.001129150390625	\\
0.108891552359391	0.001129150390625	\\
0.108935943534425	0.00103759765625	\\
0.10898033470946	0.000762939453125	\\
0.109024725884494	0.000396728515625	\\
0.109069117059529	-9.1552734375e-05	\\
0.109113508234563	-0.000335693359375	\\
0.109157899409597	-0.00079345703125	\\
0.109202290584632	-0.00115966796875	\\
0.109246681759666	-0.0006103515625	\\
0.109291072934701	-0.0003662109375	\\
0.109335464109735	-0.000640869140625	\\
0.109379855284769	3.0517578125e-05	\\
0.109424246459804	0.000244140625	\\
0.109468637634838	0.000274658203125	\\
0.109513028809873	0.00042724609375	\\
0.109557419984907	0.000732421875	\\
0.109601811159941	0.0008544921875	\\
0.109646202334976	0.000396728515625	\\
0.10969059351001	0.0009765625	\\
0.109734984685045	0.00128173828125	\\
0.109779375860079	0.001373291015625	\\
0.109823767035113	0.00152587890625	\\
0.109868158210148	0.001007080078125	\\
0.109912549385182	0.00103759765625	\\
0.109956940560217	0.00140380859375	\\
0.110001331735251	0.001220703125	\\
0.110045722910285	0.00140380859375	\\
0.11009011408532	0.00152587890625	\\
0.110134505260354	0.001953125	\\
0.110178896435389	0.0020751953125	\\
0.110223287610423	0.001739501953125	\\
0.110267678785457	0.00140380859375	\\
0.110312069960492	0.000823974609375	\\
0.110356461135526	0.00048828125	\\
0.110400852310561	0.000732421875	\\
0.110445243485595	0.0009765625	\\
0.110489634660629	0.001129150390625	\\
0.110534025835664	0.000885009765625	\\
0.110578417010698	0.000701904296875	\\
0.110622808185733	0.0006103515625	\\
0.110667199360767	0.000244140625	\\
0.110711590535801	0.000213623046875	\\
0.110755981710836	0.000152587890625	\\
0.11080037288587	0.00018310546875	\\
0.110844764060905	0.000213623046875	\\
0.110889155235939	3.0517578125e-05	\\
0.110933546410974	0.00048828125	\\
0.110977937586008	0.000640869140625	\\
0.111022328761042	0.000732421875	\\
0.111066719936077	0.000396728515625	\\
0.111111111111111	-0.000457763671875	\\
0.111155502286146	-0.00018310546875	\\
0.11119989346118	0.000244140625	\\
0.111244284636214	-9.1552734375e-05	\\
0.111288675811249	0.0001220703125	\\
0.111333066986283	0.00048828125	\\
0.111377458161318	3.0517578125e-05	\\
0.111421849336352	0.0001220703125	\\
0.111466240511386	0.0010986328125	\\
0.111510631686421	0.00091552734375	\\
0.111555022861455	0.00091552734375	\\
0.11159941403649	0.000762939453125	\\
0.111643805211524	0.0008544921875	\\
0.111688196386558	0.000885009765625	\\
0.111732587561593	0.000823974609375	\\
0.111776978736627	0.00091552734375	\\
0.111821369911662	0.001220703125	\\
0.111865761086696	0.001190185546875	\\
0.11191015226173	0.0009765625	\\
0.111954543436765	0.001068115234375	\\
0.111998934611799	0.00091552734375	\\
0.112043325786834	0.000274658203125	\\
0.112087716961868	-0.000244140625	\\
0.112132108136902	-0.000274658203125	\\
0.112176499311937	-0.000274658203125	\\
0.112220890486971	-0.000518798828125	\\
0.112265281662006	-0.000885009765625	\\
0.11230967283704	-0.001251220703125	\\
0.112354064012074	-0.001007080078125	\\
0.112398455187109	-0.000885009765625	\\
0.112442846362143	-0.00103759765625	\\
0.112487237537178	-0.001251220703125	\\
0.112531628712212	-0.00140380859375	\\
0.112576019887246	-0.001251220703125	\\
0.112620411062281	-0.0013427734375	\\
0.112664802237315	-0.001373291015625	\\
0.11270919341235	-0.00164794921875	\\
0.112753584587384	-0.0018310546875	\\
0.112797975762418	-0.001861572265625	\\
0.112842366937453	-0.001617431640625	\\
0.112886758112487	-0.00225830078125	\\
0.112931149287522	-0.001922607421875	\\
0.112975540462556	-0.001495361328125	\\
0.11301993163759	-0.00189208984375	\\
0.113064322812625	-0.001800537109375	\\
0.113108713987659	-0.001495361328125	\\
0.113153105162694	-0.001708984375	\\
0.113197496337728	-0.0018310546875	\\
0.113241887512762	-0.00177001953125	\\
0.113286278687797	-0.001708984375	\\
0.113330669862831	-0.001312255859375	\\
0.113375061037866	-0.00079345703125	\\
0.1134194522129	-0.00067138671875	\\
0.113463843387934	-0.00030517578125	\\
0.113508234562969	0.0001220703125	\\
0.113552625738003	-9.1552734375e-05	\\
0.113597016913038	-6.103515625e-05	\\
0.113641408088072	0.00042724609375	\\
0.113685799263107	0.00048828125	\\
0.113730190438141	0.0008544921875	\\
0.113774581613175	0.001190185546875	\\
0.11381897278821	0.001190185546875	\\
0.113863363963244	0.001129150390625	\\
0.113907755138279	0.001129150390625	\\
0.113952146313313	0.001068115234375	\\
0.113996537488347	0.000823974609375	\\
0.114040928663382	0.0015869140625	\\
0.114085319838416	0.001495361328125	\\
0.114129711013451	0.001068115234375	\\
0.114174102188485	0.00091552734375	\\
0.114218493363519	0.000946044921875	\\
0.114262884538554	0.00091552734375	\\
0.114307275713588	0.001129150390625	\\
0.114351666888623	0.00140380859375	\\
0.114396058063657	0.001251220703125	\\
0.114440449238691	0.000762939453125	\\
0.114484840413726	0.000946044921875	\\
0.11452923158876	0.0010986328125	\\
0.114573622763795	0.0008544921875	\\
0.114618013938829	0.000885009765625	\\
0.114662405113863	0.000823974609375	\\
0.114706796288898	0.000885009765625	\\
0.114751187463932	0.000518798828125	\\
0.114795578638967	0.000335693359375	\\
0.114839969814001	0.000457763671875	\\
0.114884360989035	0.00079345703125	\\
0.11492875216407	0.000823974609375	\\
0.114973143339104	0.000457763671875	\\
0.115017534514139	0.000152587890625	\\
0.115061925689173	0.000274658203125	\\
0.115106316864207	0.000457763671875	\\
0.115150708039242	0.00054931640625	\\
0.115195099214276	0.00030517578125	\\
0.115239490389311	0.000335693359375	\\
0.115283881564345	0.000518798828125	\\
0.115328272739379	0.000579833984375	\\
0.115372663914414	0.000885009765625	\\
0.115417055089448	0.00103759765625	\\
0.115461446264483	0.0010986328125	\\
0.115505837439517	0.00164794921875	\\
0.115550228614551	0.0020751953125	\\
0.115594619789586	0.002777099609375	\\
0.11563901096462	0.00262451171875	\\
0.115683402139655	0.002471923828125	\\
0.115727793314689	0.002532958984375	\\
0.115772184489723	0.002197265625	\\
0.115816575664758	0.002349853515625	\\
0.115860966839792	0.0023193359375	\\
0.115905358014827	0.0018310546875	\\
0.115949749189861	0.00189208984375	\\
0.115994140364895	0.001922607421875	\\
0.11603853153993	0.001708984375	\\
0.116082922714964	0.001983642578125	\\
0.116127313889999	0.001708984375	\\
0.116171705065033	0.001617431640625	\\
0.116216096240067	0.001434326171875	\\
0.116260487415102	0.000762939453125	\\
0.116304878590136	0.000701904296875	\\
0.116349269765171	0.000274658203125	\\
0.116393660940205	0.0003662109375	\\
0.116438052115239	0.000640869140625	\\
0.116482443290274	0	\\
0.116526834465308	-0.0003662109375	\\
0.116571225640343	0.000152587890625	\\
0.116615616815377	0.000335693359375	\\
0.116660007990412	-6.103515625e-05	\\
0.116704399165446	-0.000213623046875	\\
0.11674879034048	-0.000579833984375	\\
0.116793181515515	-0.000335693359375	\\
0.116837572690549	-0.0003662109375	\\
0.116881963865584	-0.000457763671875	\\
0.116926355040618	-0.000732421875	\\
0.116970746215652	-0.0008544921875	\\
0.117015137390687	-0.0006103515625	\\
0.117059528565721	-0.000640869140625	\\
0.117103919740756	-0.0006103515625	\\
0.11714831091579	-0.00079345703125	\\
0.117192702090824	-0.000640869140625	\\
0.117237093265859	-0.000213623046875	\\
0.117281484440893	9.1552734375e-05	\\
0.117325875615928	-0.00018310546875	\\
0.117370266790962	6.103515625e-05	\\
0.117414657965996	0.0006103515625	\\
0.117459049141031	0.00054931640625	\\
0.117503440316065	0.000396728515625	\\
0.1175478314911	0.000701904296875	\\
0.117592222666134	0.000396728515625	\\
0.117636613841168	0.0001220703125	\\
0.117681005016203	0.00054931640625	\\
0.117725396191237	0.000152587890625	\\
0.117769787366272	0.0001220703125	\\
0.117814178541306	0.00030517578125	\\
0.11785856971634	0.000457763671875	\\
0.117902960891375	0.000579833984375	\\
0.117947352066409	0.000946044921875	\\
0.117991743241444	0.00048828125	\\
0.118036134416478	0.000335693359375	\\
0.118080525591512	0.00048828125	\\
0.118124916766547	0.000213623046875	\\
0.118169307941581	0.000152587890625	\\
0.118213699116616	0.000701904296875	\\
0.11825809029165	0.000762939453125	\\
0.118302481466684	0	\\
0.118346872641719	0	\\
0.118391263816753	0.000213623046875	\\
0.118435654991788	-0.00042724609375	\\
0.118480046166822	-0.00079345703125	\\
0.118524437341856	-0.000732421875	\\
0.118568828516891	-0.00067138671875	\\
0.118613219691925	-6.103515625e-05	\\
0.11865761086696	-9.1552734375e-05	\\
0.118702002041994	0	\\
0.118746393217028	0.000244140625	\\
0.118790784392063	0.0003662109375	\\
0.118835175567097	0.000640869140625	\\
0.118879566742132	0.000762939453125	\\
0.118923957917166	0.000701904296875	\\
0.1189683490922	0.00054931640625	\\
0.119012740267235	0.000732421875	\\
0.119057131442269	0.0008544921875	\\
0.119101522617304	0.000579833984375	\\
0.119145913792338	-3.0517578125e-05	\\
0.119190304967372	0.000274658203125	\\
0.119234696142407	0.001007080078125	\\
0.119279087317441	0.000762939453125	\\
0.119323478492476	0.00048828125	\\
0.11936786966751	0.000640869140625	\\
0.119412260842545	0.000732421875	\\
0.119456652017579	0.0006103515625	\\
0.119501043192613	0.000640869140625	\\
0.119545434367648	0.000335693359375	\\
0.119589825542682	9.1552734375e-05	\\
0.119634216717717	0.0003662109375	\\
0.119678607892751	0.00048828125	\\
0.119722999067785	0.000396728515625	\\
0.11976739024282	0.000244140625	\\
0.119811781417854	0.0003662109375	\\
0.119856172592889	0.00018310546875	\\
0.119900563767923	-9.1552734375e-05	\\
0.119944954942957	0.000457763671875	\\
0.119989346117992	-9.1552734375e-05	\\
0.120033737293026	-0.000457763671875	\\
0.120078128468061	-0.0003662109375	\\
0.120122519643095	-0.000457763671875	\\
0.120166910818129	-0.00054931640625	\\
0.120211301993164	-0.000457763671875	\\
0.120255693168198	-0.00042724609375	\\
0.120300084343233	-0.0003662109375	\\
0.120344475518267	-0.000579833984375	\\
0.120388866693301	-0.000885009765625	\\
0.120433257868336	-0.00067138671875	\\
0.12047764904337	-0.00042724609375	\\
0.120522040218405	-0.00067138671875	\\
0.120566431393439	-0.000701904296875	\\
0.120610822568473	-0.000274658203125	\\
0.120655213743508	-6.103515625e-05	\\
0.120699604918542	-0.00067138671875	\\
0.120743996093577	-0.00042724609375	\\
0.120788387268611	-9.1552734375e-05	\\
0.120832778443645	-0.00042724609375	\\
0.12087716961868	-0.000244140625	\\
0.120921560793714	-0.000152587890625	\\
0.120965951968749	3.0517578125e-05	\\
0.121010343143783	0.000274658203125	\\
0.121054734318817	3.0517578125e-05	\\
0.121099125493852	0.00030517578125	\\
0.121143516668886	3.0517578125e-05	\\
0.121187907843921	0.00018310546875	\\
0.121232299018955	0.000396728515625	\\
0.121276690193989	0.00054931640625	\\
0.121321081369024	0.00091552734375	\\
0.121365472544058	0.000579833984375	\\
0.121409863719093	0.000274658203125	\\
0.121454254894127	-6.103515625e-05	\\
0.121498646069161	-0.000213623046875	\\
0.121543037244196	-0.000274658203125	\\
0.12158742841923	-0.000274658203125	\\
0.121631819594265	0.000244140625	\\
0.121676210769299	0.000701904296875	\\
0.121720601944333	0.000640869140625	\\
0.121764993119368	0.00048828125	\\
0.121809384294402	0.000518798828125	\\
0.121853775469437	-6.103515625e-05	\\
0.121898166644471	-0.000244140625	\\
0.121942557819505	0	\\
0.12198694899454	-3.0517578125e-05	\\
0.122031340169574	0.00018310546875	\\
0.122075731344609	0.000396728515625	\\
0.122120122519643	0.000701904296875	\\
0.122164513694678	0.00054931640625	\\
0.122208904869712	0.000946044921875	\\
0.122253296044746	0.001068115234375	\\
0.122297687219781	0.0010986328125	\\
0.122342078394815	0.0013427734375	\\
0.12238646956985	0.0010986328125	\\
0.122430860744884	0.001312255859375	\\
0.122475251919918	0.000946044921875	\\
0.122519643094953	0.00128173828125	\\
0.122564034269987	0.002044677734375	\\
0.122608425445022	0.001373291015625	\\
0.122652816620056	0.001495361328125	\\
0.12269720779509	0.00164794921875	\\
0.122741598970125	0.001190185546875	\\
0.122785990145159	0.0013427734375	\\
0.122830381320194	0.00146484375	\\
0.122874772495228	0.0013427734375	\\
0.122919163670262	0.00115966796875	\\
0.122963554845297	0.001129150390625	\\
0.123007946020331	0.00115966796875	\\
0.123052337195366	0.00128173828125	\\
0.1230967283704	0.00079345703125	\\
0.123141119545434	0.0008544921875	\\
0.123185510720469	0.001373291015625	\\
0.123229901895503	0.001129150390625	\\
0.123274293070538	0.0018310546875	\\
0.123318684245572	0.002288818359375	\\
0.123363075420606	0.001922607421875	\\
0.123407466595641	0.002166748046875	\\
0.123451857770675	0.00164794921875	\\
0.12349624894571	0.001556396484375	\\
0.123540640120744	0.0015869140625	\\
0.123585031295778	0.001251220703125	\\
0.123629422470813	0.00146484375	\\
0.123673813645847	0.00140380859375	\\
0.123718204820882	0.0015869140625	\\
0.123762595995916	0.002349853515625	\\
0.12380698717095	0.002685546875	\\
0.123851378345985	0.00244140625	\\
0.123895769521019	0.00189208984375	\\
0.123940160696054	0.0015869140625	\\
0.123984551871088	0.001739501953125	\\
0.124028943046122	0.00201416015625	\\
0.124073334221157	0.001861572265625	\\
0.124117725396191	0.001983642578125	\\
0.124162116571226	0.001800537109375	\\
0.12420650774626	0.0013427734375	\\
0.124250898921294	0.001556396484375	\\
0.124295290096329	0.001434326171875	\\
0.124339681271363	0.0013427734375	\\
0.124384072446398	0.0013427734375	\\
0.124428463621432	0.001312255859375	\\
0.124472854796466	0.001678466796875	\\
0.124517245971501	0.001800537109375	\\
0.124561637146535	0.001922607421875	\\
0.12460602832157	0.00164794921875	\\
0.124650419496604	0.00177001953125	\\
0.124694810671638	0.001922607421875	\\
0.124739201846673	0.001617431640625	\\
0.124783593021707	0.001495361328125	\\
0.124827984196742	0.001220703125	\\
0.124872375371776	0.00091552734375	\\
0.124916766546811	0.00103759765625	\\
0.124961157721845	0.00146484375	\\
0.125005548896879	0.00152587890625	\\
0.125049940071914	0.001251220703125	\\
0.125094331246948	0.001007080078125	\\
0.125138722421983	0.0010986328125	\\
0.125183113597017	0.001190185546875	\\
0.125227504772051	0.00091552734375	\\
0.125271895947086	0.001007080078125	\\
0.12531628712212	0.0010986328125	\\
0.125360678297155	0.000946044921875	\\
0.125405069472189	0.00079345703125	\\
0.125449460647223	0.0009765625	\\
0.125493851822258	0.001220703125	\\
0.125538242997292	0.00103759765625	\\
0.125582634172327	0.001220703125	\\
0.125627025347361	0.0010986328125	\\
0.125671416522395	0.001007080078125	\\
0.12571580769743	0.00164794921875	\\
0.125760198872464	0.001800537109375	\\
0.125804590047499	0.001617431640625	\\
0.125848981222533	0.001861572265625	\\
0.125893372397567	0.001739501953125	\\
0.125937763572602	0.002105712890625	\\
0.125982154747636	0.001983642578125	\\
0.126026545922671	0.00128173828125	\\
0.126070937097705	0.001495361328125	\\
0.126115328272739	0.001678466796875	\\
0.126159719447774	0.001922607421875	\\
0.126204110622808	0.001922607421875	\\
0.126248501797843	0.00189208984375	\\
0.126292892972877	0.00201416015625	\\
0.126337284147911	0.001983642578125	\\
0.126381675322946	0.00201416015625	\\
0.12642606649798	0.002105712890625	\\
0.126470457673015	0.00189208984375	\\
0.126514848848049	0.002197265625	\\
0.126559240023083	0.002349853515625	\\
0.126603631198118	0.0020751953125	\\
0.126648022373152	0.002410888671875	\\
0.126692413548187	0.002471923828125	\\
0.126736804723221	0.002685546875	\\
0.126781195898255	0.002655029296875	\\
0.12682558707329	0.002044677734375	\\
0.126869978248324	0.001800537109375	\\
0.126914369423359	0.00164794921875	\\
0.126958760598393	0.001373291015625	\\
0.127003151773427	0.0013427734375	\\
0.127047542948462	0.00140380859375	\\
0.127091934123496	0.000762939453125	\\
0.127136325298531	0.00079345703125	\\
0.127180716473565	0.000762939453125	\\
0.127225107648599	0.000946044921875	\\
0.127269498823634	0.00103759765625	\\
0.127313889998668	0.001068115234375	\\
0.127358281173703	0.00091552734375	\\
0.127402672348737	0.000701904296875	\\
0.127447063523771	0.00067138671875	\\
0.127491454698806	0.000640869140625	\\
0.12753584587384	0.000518798828125	\\
0.127580237048875	0.0008544921875	\\
0.127624628223909	0.000946044921875	\\
0.127669019398943	0.000946044921875	\\
0.127713410573978	0.00091552734375	\\
0.127757801749012	0.000946044921875	\\
0.127802192924047	0.00103759765625	\\
0.127846584099081	0.001251220703125	\\
0.127890975274116	0.001312255859375	\\
0.12793536644915	0.001434326171875	\\
0.127979757624184	0.00164794921875	\\
0.128024148799219	0.0009765625	\\
0.128068539974253	0.001007080078125	\\
0.128112931149288	0.000732421875	\\
0.128157322324322	6.103515625e-05	\\
0.128201713499356	0.00030517578125	\\
0.128246104674391	0.00018310546875	\\
0.128290495849425	0.000335693359375	\\
0.12833488702446	0.000946044921875	\\
0.128379278199494	0.001220703125	\\
0.128423669374528	0.001220703125	\\
0.128468060549563	0.000885009765625	\\
0.128512451724597	0.00048828125	\\
0.128556842899632	0.00018310546875	\\
0.128601234074666	-0.0001220703125	\\
0.1286456252497	-0.000152587890625	\\
0.128690016424735	-0.0001220703125	\\
0.128734407599769	-0.000335693359375	\\
0.128778798774804	-6.103515625e-05	\\
0.128823189949838	6.103515625e-05	\\
0.128867581124872	-0.000335693359375	\\
0.128911972299907	-0.000335693359375	\\
0.128956363474941	-0.0006103515625	\\
0.129000754649976	-0.000640869140625	\\
0.12904514582501	-9.1552734375e-05	\\
0.129089537000044	-0.00018310546875	\\
0.129133928175079	-9.1552734375e-05	\\
0.129178319350113	0.000244140625	\\
0.129222710525148	0.00030517578125	\\
0.129267101700182	0.000640869140625	\\
0.129311492875216	0.00054931640625	\\
0.129355884050251	0.000579833984375	\\
0.129400275225285	0.00103759765625	\\
0.12944466640032	0.0006103515625	\\
0.129489057575354	0.000823974609375	\\
0.129533448750388	0.00103759765625	\\
0.129577839925423	0.000885009765625	\\
0.129622231100457	0.001739501953125	\\
0.129666622275492	0.001953125	\\
0.129711013450526	0.001617431640625	\\
0.12975540462556	0.001739501953125	\\
0.129799795800595	0.001983642578125	\\
0.129844186975629	0.0020751953125	\\
0.129888578150664	0.001556396484375	\\
0.129932969325698	0.001800537109375	\\
0.129977360500732	0.002410888671875	\\
0.130021751675767	0.002166748046875	\\
0.130066142850801	0.00225830078125	\\
0.130110534025836	0.00213623046875	\\
0.13015492520087	0.001800537109375	\\
0.130199316375904	0.001708984375	\\
0.130243707550939	0.001953125	\\
0.130288098725973	0.001678466796875	\\
0.130332489901008	0.001251220703125	\\
0.130376881076042	0.00152587890625	\\
0.130421272251077	0.001220703125	\\
0.130465663426111	0.001220703125	\\
0.130510054601145	0.00091552734375	\\
0.13055444577618	0.0006103515625	\\
0.130598836951214	0.0001220703125	\\
0.130643228126249	-0.00018310546875	\\
0.130687619301283	0.00030517578125	\\
0.130732010476317	0.000274658203125	\\
0.130776401651352	0.000244140625	\\
0.130820792826386	0.0001220703125	\\
0.130865184001421	6.103515625e-05	\\
0.130909575176455	0.00018310546875	\\
0.130953966351489	-0.0001220703125	\\
0.130998357526524	-0.000274658203125	\\
0.131042748701558	-9.1552734375e-05	\\
0.131087139876593	-0.00042724609375	\\
0.131131531051627	-0.00048828125	\\
0.131175922226661	-0.000274658203125	\\
0.131220313401696	-0.0001220703125	\\
0.13126470457673	-6.103515625e-05	\\
0.131309095751765	-0.00018310546875	\\
0.131353486926799	-0.000244140625	\\
0.131397878101833	0	\\
0.131442269276868	-3.0517578125e-05	\\
0.131486660451902	0	\\
0.131531051626937	0.00048828125	\\
0.131575442801971	0.000732421875	\\
0.131619833977005	0.0009765625	\\
0.13166422515204	0.00067138671875	\\
0.131708616327074	0.000946044921875	\\
0.131753007502109	0.000885009765625	\\
0.131797398677143	0.0008544921875	\\
0.131841789852177	0.001068115234375	\\
0.131886181027212	0.001007080078125	\\
0.131930572202246	0.00115966796875	\\
0.131974963377281	0.001617431640625	\\
0.132019354552315	0.00128173828125	\\
0.132063745727349	0.00091552734375	\\
0.132108136902384	0.001068115234375	\\
0.132152528077418	0.000885009765625	\\
0.132196919252453	0.000701904296875	\\
0.132241310427487	0.00042724609375	\\
0.132285701602521	0.0010986328125	\\
0.132330092777556	0.001068115234375	\\
0.13237448395259	0.000518798828125	\\
0.132418875127625	0.000701904296875	\\
0.132463266302659	0.000335693359375	\\
0.132507657477693	0.0001220703125	\\
0.132552048652728	0	\\
0.132596439827762	-6.103515625e-05	\\
0.132640831002797	-6.103515625e-05	\\
0.132685222177831	-0.00048828125	\\
0.132729613352865	-0.000640869140625	\\
0.1327740045279	-0.00079345703125	\\
0.132818395702934	-0.000946044921875	\\
0.132862786877969	-0.000823974609375	\\
0.132907178053003	-0.000762939453125	\\
0.132951569228037	-0.000762939453125	\\
0.132995960403072	-0.0008544921875	\\
0.133040351578106	-0.000518798828125	\\
0.133084742753141	-0.00054931640625	\\
0.133129133928175	-0.000518798828125	\\
0.133173525103209	-0.000274658203125	\\
0.133217916278244	0.0001220703125	\\
0.133262307453278	-0.000274658203125	\\
0.133306698628313	-0.000579833984375	\\
0.133351089803347	9.1552734375e-05	\\
0.133395480978382	0.000640869140625	\\
0.133439872153416	0.000457763671875	\\
0.13348426332845	0.000823974609375	\\
0.133528654503485	0.00079345703125	\\
0.133573045678519	0.00103759765625	\\
0.133617436853554	0.001708984375	\\
0.133661828028588	0.00177001953125	\\
0.133706219203622	0.00189208984375	\\
0.133750610378657	0.00152587890625	\\
0.133795001553691	0.001434326171875	\\
0.133839392728726	0.001922607421875	\\
0.13388378390376	0.00201416015625	\\
0.133928175078794	0.001983642578125	\\
0.133972566253829	0.00189208984375	\\
0.134016957428863	0.001556396484375	\\
0.134061348603898	0.0018310546875	\\
0.134105739778932	0.0018310546875	\\
0.134150130953966	0.0018310546875	\\
0.134194522129001	0.00177001953125	\\
0.134238913304035	0.001434326171875	\\
0.13428330447907	0.001800537109375	\\
0.134327695654104	0.0018310546875	\\
0.134372086829138	0.0015869140625	\\
0.134416478004173	0.0015869140625	\\
0.134460869179207	0.001373291015625	\\
0.134505260354242	0.001190185546875	\\
0.134549651529276	0.000946044921875	\\
0.13459404270431	0.000579833984375	\\
0.134638433879345	0.000579833984375	\\
0.134682825054379	0.000640869140625	\\
0.134727216229414	0.000244140625	\\
0.134771607404448	-0.000213623046875	\\
0.134815998579482	-0.0001220703125	\\
0.134860389754517	0.0001220703125	\\
0.134904780929551	-0.00030517578125	\\
0.134949172104586	-0.0003662109375	\\
0.13499356327962	-0.000274658203125	\\
0.135037954454654	-0.000274658203125	\\
0.135082345629689	-0.000244140625	\\
0.135126736804723	-0.000579833984375	\\
0.135171127979758	-0.000732421875	\\
0.135215519154792	-0.000701904296875	\\
0.135259910329826	-0.001220703125	\\
0.135304301504861	-0.00164794921875	\\
0.135348692679895	-0.001617431640625	\\
0.13539308385493	-0.001495361328125	\\
0.135437475029964	-0.00115966796875	\\
0.135481866204998	-0.00115966796875	\\
0.135526257380033	-0.00177001953125	\\
0.135570648555067	-0.001190185546875	\\
0.135615039730102	-0.001312255859375	\\
0.135659430905136	-0.001739501953125	\\
0.13570382208017	-0.00128173828125	\\
0.135748213255205	-0.001556396484375	\\
0.135792604430239	-0.001617431640625	\\
0.135836995605274	-0.001312255859375	\\
0.135881386780308	-0.00152587890625	\\
0.135925777955342	-0.001556396484375	\\
0.135970169130377	-0.001251220703125	\\
0.136014560305411	-0.001373291015625	\\
0.136058951480446	-0.00140380859375	\\
0.13610334265548	-0.001495361328125	\\
0.136147733830514	-0.001617431640625	\\
0.136192125005549	-0.001556396484375	\\
0.136236516180583	-0.00164794921875	\\
0.136280907355618	-0.001434326171875	\\
0.136325298530652	-0.001739501953125	\\
0.136369689705687	-0.002471923828125	\\
0.136414080880721	-0.002532958984375	\\
0.136458472055755	-0.00213623046875	\\
0.13650286323079	-0.0018310546875	\\
0.136547254405824	-0.002197265625	\\
0.136591645580859	-0.001922607421875	\\
0.136636036755893	-0.001922607421875	\\
0.136680427930927	-0.001861572265625	\\
0.136724819105962	-0.0015869140625	\\
0.136769210280996	-0.002166748046875	\\
0.136813601456031	-0.002655029296875	\\
0.136857992631065	-0.00244140625	\\
0.136902383806099	-0.002349853515625	\\
0.136946774981134	-0.0023193359375	\\
0.136991166156168	-0.0018310546875	\\
0.137035557331203	-0.001739501953125	\\
0.137079948506237	-0.001434326171875	\\
0.137124339681271	-0.001251220703125	\\
0.137168730856306	-0.00146484375	\\
0.13721312203134	-0.00103759765625	\\
0.137257513206375	-0.001068115234375	\\
0.137301904381409	-0.0013427734375	\\
0.137346295556443	-0.001007080078125	\\
0.137390686731478	-0.000946044921875	\\
0.137435077906512	-0.0006103515625	\\
0.137479469081547	-0.00054931640625	\\
0.137523860256581	-0.000335693359375	\\
0.137568251431615	-6.103515625e-05	\\
0.13761264260665	-0.00030517578125	\\
0.137657033781684	3.0517578125e-05	\\
0.137701424956719	-0.000244140625	\\
0.137745816131753	-0.000579833984375	\\
0.137790207306787	0	\\
0.137834598481822	0.0001220703125	\\
0.137878989656856	0.000152587890625	\\
0.137923380831891	0.000152587890625	\\
0.137967772006925	-6.103515625e-05	\\
0.138012163181959	-0.000244140625	\\
0.138056554356994	-9.1552734375e-05	\\
0.138100945532028	-3.0517578125e-05	\\
0.138145336707063	-0.000152587890625	\\
0.138189727882097	-0.00018310546875	\\
0.138234119057131	-0.000396728515625	\\
0.138278510232166	-9.1552734375e-05	\\
0.1383229014072	-0.000274658203125	\\
0.138367292582235	-0.0001220703125	\\
0.138411683757269	-0.000244140625	\\
0.138456074932303	-0.000579833984375	\\
0.138500466107338	-0.0001220703125	\\
0.138544857282372	3.0517578125e-05	\\
0.138589248457407	-0.000244140625	\\
0.138633639632441	-0.000335693359375	\\
0.138678030807475	-0.000274658203125	\\
0.13872242198251	-0.0001220703125	\\
0.138766813157544	3.0517578125e-05	\\
0.138811204332579	0.000244140625	\\
0.138855595507613	6.103515625e-05	\\
0.138899986682648	-0.000335693359375	\\
0.138944377857682	0	\\
0.138988769032716	0.000885009765625	\\
0.139033160207751	0.000274658203125	\\
0.139077551382785	-3.0517578125e-05	\\
0.13912194255782	0.000152587890625	\\
0.139166333732854	-6.103515625e-05	\\
0.139210724907888	3.0517578125e-05	\\
0.139255116082923	9.1552734375e-05	\\
0.139299507257957	0.0001220703125	\\
0.139343898432992	6.103515625e-05	\\
0.139388289608026	0.000244140625	\\
0.13943268078306	0.00018310546875	\\
0.139477071958095	0.000152587890625	\\
0.139521463133129	-6.103515625e-05	\\
0.139565854308164	-0.00042724609375	\\
0.139610245483198	-0.000396728515625	\\
0.139654636658232	-0.000946044921875	\\
0.139699027833267	-0.00079345703125	\\
0.139743419008301	-0.000762939453125	\\
0.139787810183336	-0.001129150390625	\\
0.13983220135837	-0.00079345703125	\\
0.139876592533404	-0.0008544921875	\\
0.139920983708439	-0.001495361328125	\\
0.139965374883473	-0.001678466796875	\\
0.140009766058508	-0.001617431640625	\\
0.140054157233542	-0.00177001953125	\\
0.140098548408576	-0.001739501953125	\\
0.140142939583611	-0.0015869140625	\\
0.140187330758645	-0.00115966796875	\\
0.14023172193368	-0.001190185546875	\\
0.140276113108714	-0.000946044921875	\\
0.140320504283748	-0.000823974609375	\\
0.140364895458783	-0.0008544921875	\\
0.140409286633817	-0.000579833984375	\\
0.140453677808852	-0.00042724609375	\\
0.140498068983886	-0.000579833984375	\\
0.14054246015892	-0.0008544921875	\\
0.140586851333955	-0.000640869140625	\\
0.140631242508989	-0.000274658203125	\\
0.140675633684024	9.1552734375e-05	\\
0.140720024859058	3.0517578125e-05	\\
0.140764416034092	-0.00030517578125	\\
0.140808807209127	0.000152587890625	\\
0.140853198384161	0.000152587890625	\\
0.140897589559196	0.000152587890625	\\
0.14094198073423	0.00067138671875	\\
0.140986371909264	0.000823974609375	\\
0.141030763084299	0.000762939453125	\\
0.141075154259333	0.000762939453125	\\
0.141119545434368	0.00054931640625	\\
0.141163936609402	0.000640869140625	\\
0.141208327784436	0.000579833984375	\\
0.141252718959471	0.000335693359375	\\
0.141297110134505	0.0006103515625	\\
0.14134150130954	0.00048828125	\\
0.141385892484574	0.000701904296875	\\
0.141430283659608	0.0008544921875	\\
0.141474674834643	0.000396728515625	\\
0.141519066009677	0.000579833984375	\\
0.141563457184712	0.000762939453125	\\
0.141607848359746	0.000244140625	\\
0.14165223953478	0.00030517578125	\\
0.141696630709815	0.00030517578125	\\
0.141741021884849	6.103515625e-05	\\
0.141785413059884	0.0003662109375	\\
0.141829804234918	0.00042724609375	\\
0.141874195409953	0.001007080078125	\\
0.141918586584987	0.000762939453125	\\
0.141962977760021	0.000396728515625	\\
0.142007368935056	0.000946044921875	\\
0.14205176011009	0.000885009765625	\\
0.142096151285125	0.000762939453125	\\
0.142140542460159	0.000213623046875	\\
0.142184933635193	0	\\
0.142229324810228	-0.00018310546875	\\
0.142273715985262	-0.00030517578125	\\
0.142318107160297	-0.000732421875	\\
0.142362498335331	-0.0009765625	\\
0.142406889510365	-0.000946044921875	\\
0.1424512806854	-0.0008544921875	\\
0.142495671860434	-0.0008544921875	\\
0.142540063035469	-0.001068115234375	\\
0.142584454210503	-0.001007080078125	\\
0.142628845385537	-0.001129150390625	\\
0.142673236560572	-0.000823974609375	\\
0.142717627735606	-0.001068115234375	\\
0.142762018910641	-0.000885009765625	\\
0.142806410085675	-0.00079345703125	\\
0.142850801260709	-0.001068115234375	\\
0.142895192435744	-0.0009765625	\\
0.142939583610778	-0.00164794921875	\\
0.142983974785813	-0.00140380859375	\\
0.143028365960847	-0.0013427734375	\\
0.143072757135881	-0.00177001953125	\\
0.143117148310916	-0.0010986328125	\\
0.14316153948595	-0.000518798828125	\\
0.143205930660985	-0.001007080078125	\\
0.143250321836019	-0.00128173828125	\\
0.143294713011053	-0.000946044921875	\\
0.143339104186088	-0.000823974609375	\\
0.143383495361122	-0.0008544921875	\\
0.143427886536157	-0.00091552734375	\\
0.143472277711191	-0.00103759765625	\\
0.143516668886225	-0.000518798828125	\\
0.14356106006126	-0.00042724609375	\\
0.143605451236294	-0.00042724609375	\\
0.143649842411329	-0.0001220703125	\\
0.143694233586363	-0.000152587890625	\\
0.143738624761397	-0.00067138671875	\\
0.143783015936432	-0.00103759765625	\\
0.143827407111466	-0.001129150390625	\\
0.143871798286501	-0.001556396484375	\\
0.143916189461535	-0.001373291015625	\\
0.143960580636569	-0.001220703125	\\
0.144004971811604	-0.001373291015625	\\
0.144049362986638	-0.001190185546875	\\
0.144093754161673	-0.00091552734375	\\
0.144138145336707	-0.0009765625	\\
0.144182536511741	-0.0013427734375	\\
0.144226927686776	-0.001190185546875	\\
0.14427131886181	-0.001251220703125	\\
0.144315710036845	-0.001678466796875	\\
0.144360101211879	-0.001708984375	\\
0.144404492386913	-0.001312255859375	\\
0.144448883561948	-0.00115966796875	\\
0.144493274736982	-0.00128173828125	\\
0.144537665912017	-0.001068115234375	\\
0.144582057087051	-0.001556396484375	\\
0.144626448262086	-0.001312255859375	\\
0.14467083943712	-0.00140380859375	\\
0.144715230612154	-0.001373291015625	\\
0.144759621787189	-0.000823974609375	\\
0.144804012962223	-0.0006103515625	\\
0.144848404137258	-0.000274658203125	\\
0.144892795312292	-0.00030517578125	\\
0.144937186487326	-0.0006103515625	\\
0.144981577662361	-0.000213623046875	\\
0.145025968837395	-0.0001220703125	\\
0.14507036001243	6.103515625e-05	\\
0.145114751187464	0.000274658203125	\\
0.145159142362498	0.0003662109375	\\
0.145203533537533	0.001129150390625	\\
0.145247924712567	0.000885009765625	\\
0.145292315887602	0.000518798828125	\\
0.145336707062636	0.0006103515625	\\
0.14538109823767	0.000823974609375	\\
0.145425489412705	0.000732421875	\\
0.145469880587739	0.00048828125	\\
0.145514271762774	0.000518798828125	\\
0.145558662937808	0.000579833984375	\\
0.145603054112842	0.00067138671875	\\
0.145647445287877	0.000732421875	\\
0.145691836462911	0.0008544921875	\\
0.145736227637946	0.001129150390625	\\
0.14578061881298	0.0013427734375	\\
0.145825009988014	0.001251220703125	\\
0.145869401163049	0.001129150390625	\\
0.145913792338083	0.001434326171875	\\
0.145958183513118	0.001312255859375	\\
0.146002574688152	0.00103759765625	\\
0.146046965863186	0.00103759765625	\\
0.146091357038221	0.000732421875	\\
0.146135748213255	0.000823974609375	\\
0.14618013938829	0.000762939453125	\\
0.146224530563324	0.00054931640625	\\
0.146268921738358	0.000640869140625	\\
0.146313312913393	0.00079345703125	\\
0.146357704088427	0.00048828125	\\
0.146402095263462	0.0001220703125	\\
0.146446486438496	0.000213623046875	\\
0.14649087761353	0.0001220703125	\\
0.146535268788565	0.00030517578125	\\
0.146579659963599	0.000244140625	\\
0.146624051138634	6.103515625e-05	\\
0.146668442313668	0.000244140625	\\
0.146712833488702	0	\\
0.146757224663737	-0.000579833984375	\\
0.146801615838771	-0.000518798828125	\\
0.146846007013806	-0.000396728515625	\\
0.14689039818884	-0.0006103515625	\\
0.146934789363874	-0.000152587890625	\\
0.146979180538909	-0.000244140625	\\
0.147023571713943	-0.000274658203125	\\
0.147067962888978	-0.00030517578125	\\
0.147112354064012	-0.000457763671875	\\
0.147156745239046	-0.000152587890625	\\
0.147201136414081	-3.0517578125e-05	\\
0.147245527589115	6.103515625e-05	\\
0.14728991876415	0.00042724609375	\\
0.147334309939184	-9.1552734375e-05	\\
0.147378701114219	-0.000518798828125	\\
0.147423092289253	-0.00018310546875	\\
0.147467483464287	-0.000213623046875	\\
0.147511874639322	-9.1552734375e-05	\\
0.147556265814356	0.000244140625	\\
0.147600656989391	0.00018310546875	\\
0.147645048164425	0.000457763671875	\\
0.147689439339459	0.001129150390625	\\
0.147733830514494	0.0010986328125	\\
0.147778221689528	0.001068115234375	\\
0.147822612864563	0.00103759765625	\\
0.147867004039597	0.001190185546875	\\
0.147911395214631	0.001708984375	\\
0.147955786389666	0.001800537109375	\\
0.1480001775647	0.001800537109375	\\
0.148044568739735	0.002044677734375	\\
0.148088959914769	0.001983642578125	\\
0.148133351089803	0.001739501953125	\\
0.148177742264838	0.001983642578125	\\
0.148222133439872	0.002044677734375	\\
0.148266524614907	0.00225830078125	\\
0.148310915789941	0.00213623046875	\\
0.148355306964975	0.00225830078125	\\
0.14839969814001	0.00250244140625	\\
0.148444089315044	0.002166748046875	\\
0.148488480490079	0.00213623046875	\\
0.148532871665113	0.002105712890625	\\
0.148577262840147	0.002166748046875	\\
0.148621654015182	0.00225830078125	\\
0.148666045190216	0.0020751953125	\\
0.148710436365251	0.001953125	\\
0.148754827540285	0.001312255859375	\\
0.148799218715319	0.001190185546875	\\
0.148843609890354	0.001251220703125	\\
0.148888001065388	0.0006103515625	\\
0.148932392240423	0.00067138671875	\\
0.148976783415457	0.000579833984375	\\
0.149021174590491	0.00048828125	\\
0.149065565765526	0.000518798828125	\\
0.14910995694056	0.000244140625	\\
0.149154348115595	0.000396728515625	\\
0.149198739290629	0.000640869140625	\\
0.149243130465663	0.000701904296875	\\
0.149287521640698	0.000823974609375	\\
0.149331912815732	0.00030517578125	\\
0.149376303990767	0.000579833984375	\\
0.149420695165801	0.000762939453125	\\
0.149465086340835	0.0009765625	\\
0.14950947751587	0.001495361328125	\\
0.149553868690904	0.001434326171875	\\
0.149598259865939	0.001220703125	\\
0.149642651040973	0.00146484375	\\
0.149687042216007	0.001434326171875	\\
0.149731433391042	0.00164794921875	\\
0.149775824566076	0.002227783203125	\\
0.149820215741111	0.00201416015625	\\
0.149864606916145	0.0025634765625	\\
0.149908998091179	0.00335693359375	\\
0.149953389266214	0.003265380859375	\\
0.149997780441248	0.003082275390625	\\
0.150042171616283	0.00238037109375	\\
0.150086562791317	0.002471923828125	\\
0.150130953966351	0.00244140625	\\
0.150175345141386	0.002593994140625	\\
0.15021973631642	0.002593994140625	\\
0.150264127491455	0.00213623046875	\\
0.150308518666489	0.002685546875	\\
0.150352909841524	0.00244140625	\\
0.150397301016558	0.001922607421875	\\
0.150441692191592	0.001617431640625	\\
0.150486083366627	0.001312255859375	\\
0.150530474541661	0.0008544921875	\\
0.150574865716696	0.00067138671875	\\
0.15061925689173	0.001068115234375	\\
0.150663648066764	0.00079345703125	\\
0.150708039241799	0.000885009765625	\\
0.150752430416833	0.0009765625	\\
0.150796821591868	0.00030517578125	\\
0.150841212766902	0.000274658203125	\\
0.150885603941936	0.000457763671875	\\
0.150929995116971	0.000335693359375	\\
0.150974386292005	0.000518798828125	\\
0.15101877746704	0.000335693359375	\\
0.151063168642074	-3.0517578125e-05	\\
0.151107559817108	-3.0517578125e-05	\\
0.151151950992143	-0.00018310546875	\\
0.151196342167177	-0.000640869140625	\\
0.151240733342212	-0.000579833984375	\\
0.151285124517246	-0.000244140625	\\
0.15132951569228	-0.000457763671875	\\
0.151373906867315	-0.000274658203125	\\
0.151418298042349	3.0517578125e-05	\\
0.151462689217384	0.0001220703125	\\
0.151507080392418	-0.000213623046875	\\
0.151551471567452	-0.000457763671875	\\
0.151595862742487	-0.00042724609375	\\
0.151640253917521	-0.0006103515625	\\
0.151684645092556	-0.0009765625	\\
0.15172903626759	-0.0008544921875	\\
0.151773427442624	-0.00067138671875	\\
0.151817818617659	-0.00091552734375	\\
0.151862209792693	-0.00115966796875	\\
0.151906600967728	-0.0006103515625	\\
0.151950992142762	-0.000274658203125	\\
0.151995383317796	-0.00067138671875	\\
0.152039774492831	-0.0006103515625	\\
0.152084165667865	-0.000213623046875	\\
0.1521285568429	-6.103515625e-05	\\
0.152172948017934	0.000244140625	\\
0.152217339192968	0.000335693359375	\\
0.152261730368003	-0.000152587890625	\\
0.152306121543037	-0.00030517578125	\\
0.152350512718072	0.00018310546875	\\
0.152394903893106	0	\\
0.15243929506814	0	\\
0.152483686243175	0.00042724609375	\\
0.152528077418209	0.00042724609375	\\
0.152572468593244	0.00042724609375	\\
0.152616859768278	0.000640869140625	\\
0.152661250943312	0.00067138671875	\\
0.152705642118347	0.001007080078125	\\
0.152750033293381	0.001068115234375	\\
0.152794424468416	0.001373291015625	\\
0.15283881564345	0.001434326171875	\\
0.152883206818485	0.001495361328125	\\
0.152927597993519	0.001434326171875	\\
0.152971989168553	0.001068115234375	\\
0.153016380343588	0.001312255859375	\\
0.153060771518622	0.0015869140625	\\
0.153105162693657	0.0013427734375	\\
0.153149553868691	0.00128173828125	\\
0.153193945043725	0.001556396484375	\\
0.15323833621876	0.001678466796875	\\
0.153282727393794	0.001861572265625	\\
0.153327118568829	0.001739501953125	\\
0.153371509743863	0.00152587890625	\\
0.153415900918897	0.001373291015625	\\
0.153460292093932	0.001251220703125	\\
0.153504683268966	0.0013427734375	\\
0.153549074444001	0.000823974609375	\\
0.153593465619035	0.000579833984375	\\
0.153637856794069	0.0010986328125	\\
0.153682247969104	0.001251220703125	\\
0.153726639144138	0.001373291015625	\\
0.153771030319173	0.001556396484375	\\
0.153815421494207	0.001251220703125	\\
0.153859812669241	0.0009765625	\\
0.153904203844276	0.0013427734375	\\
0.15394859501931	0.001220703125	\\
0.153992986194345	0.001190185546875	\\
0.154037377369379	0.00128173828125	\\
0.154081768544413	0.00091552734375	\\
0.154126159719448	0.000762939453125	\\
0.154170550894482	0.00042724609375	\\
0.154214942069517	0.000213623046875	\\
0.154259333244551	0.000152587890625	\\
0.154303724419585	-0.00018310546875	\\
0.15434811559462	-0.00042724609375	\\
0.154392506769654	-0.000640869140625	\\
0.154436897944689	-0.00067138671875	\\
0.154481289119723	-0.0008544921875	\\
0.154525680294757	-0.00067138671875	\\
0.154570071469792	-0.00091552734375	\\
0.154614462644826	-0.001129150390625	\\
0.154658853819861	-0.001190185546875	\\
0.154703244994895	-0.001129150390625	\\
0.154747636169929	-0.000823974609375	\\
0.154792027344964	-0.000823974609375	\\
0.154836418519998	-0.0008544921875	\\
0.154880809695033	-0.000823974609375	\\
0.154925200870067	-0.0009765625	\\
0.154969592045101	-0.000946044921875	\\
0.155013983220136	-0.00103759765625	\\
0.15505837439517	-0.001312255859375	\\
0.155102765570205	-0.00128173828125	\\
0.155147156745239	-0.001068115234375	\\
0.155191547920273	-0.001220703125	\\
0.155235939095308	-0.0013427734375	\\
0.155280330270342	-0.001434326171875	\\
0.155324721445377	-0.001708984375	\\
0.155369112620411	-0.0018310546875	\\
0.155413503795445	-0.001556396484375	\\
0.15545789497048	-0.00189208984375	\\
0.155502286145514	-0.002288818359375	\\
0.155546677320549	-0.002105712890625	\\
0.155591068495583	-0.001953125	\\
0.155635459670617	-0.0015869140625	\\
0.155679850845652	-0.001068115234375	\\
0.155724242020686	-0.00079345703125	\\
0.155768633195721	-0.000762939453125	\\
0.155813024370755	-0.000640869140625	\\
0.15585741554579	-0.001007080078125	\\
0.155901806720824	-0.001129150390625	\\
0.155946197895858	-0.0008544921875	\\
0.155990589070893	-0.000946044921875	\\
0.156034980245927	-0.000885009765625	\\
0.156079371420962	-0.000762939453125	\\
0.156123762595996	-0.00054931640625	\\
0.15616815377103	-0.000579833984375	\\
0.156212544946065	-0.00079345703125	\\
0.156256936121099	-0.000457763671875	\\
0.156301327296134	-0.000518798828125	\\
0.156345718471168	-0.000732421875	\\
0.156390109646202	-0.000396728515625	\\
0.156434500821237	-0.000885009765625	\\
0.156478891996271	-0.001312255859375	\\
0.156523283171306	-0.001495361328125	\\
0.15656767434634	-0.0015869140625	\\
0.156612065521374	-0.000946044921875	\\
0.156656456696409	-0.0008544921875	\\
0.156700847871443	-0.00067138671875	\\
0.156745239046478	-0.000732421875	\\
0.156789630221512	-0.001220703125	\\
0.156834021396546	-0.00128173828125	\\
0.156878412571581	-0.00140380859375	\\
0.156922803746615	-0.0010986328125	\\
0.15696719492165	-0.001556396484375	\\
0.157011586096684	-0.001861572265625	\\
0.157055977271718	-0.001373291015625	\\
0.157100368446753	-0.001617431640625	\\
0.157144759621787	-0.001739501953125	\\
0.157189150796822	-0.001617431640625	\\
0.157233541971856	-0.00140380859375	\\
0.15727793314689	-0.00152587890625	\\
0.157322324321925	-0.001251220703125	\\
0.157366715496959	-0.001190185546875	\\
0.157411106671994	-0.001129150390625	\\
0.157455497847028	-0.00054931640625	\\
0.157499889022062	-0.000640869140625	\\
0.157544280197097	-0.000823974609375	\\
0.157588671372131	-0.000701904296875	\\
0.157633062547166	-0.000579833984375	\\
0.1576774537222	-0.00054931640625	\\
0.157721844897234	-0.000335693359375	\\
0.157766236072269	-0.00042724609375	\\
0.157810627247303	-0.000640869140625	\\
0.157855018422338	-0.0008544921875	\\
0.157899409597372	-0.001434326171875	\\
0.157943800772406	-0.00128173828125	\\
0.157988191947441	-0.001251220703125	\\
0.158032583122475	-0.00146484375	\\
0.15807697429751	-0.001739501953125	\\
0.158121365472544	-0.001800537109375	\\
0.158165756647578	-0.002105712890625	\\
0.158210147822613	-0.002410888671875	\\
0.158254538997647	-0.002197265625	\\
0.158298930172682	-0.002105712890625	\\
0.158343321347716	-0.0023193359375	\\
0.15838771252275	-0.002227783203125	\\
0.158432103697785	-0.002593994140625	\\
0.158476494872819	-0.0029296875	\\
0.158520886047854	-0.0029296875	\\
0.158565277222888	-0.0025634765625	\\
0.158609668397922	-0.002593994140625	\\
0.158654059572957	-0.00262451171875	\\
0.158698450747991	-0.002655029296875	\\
0.158742841923026	-0.00274658203125	\\
0.15878723309806	-0.002593994140625	\\
0.158831624273095	-0.002471923828125	\\
0.158876015448129	-0.002532958984375	\\
0.158920406623163	-0.002838134765625	\\
0.158964797798198	-0.002838134765625	\\
0.159009188973232	-0.00238037109375	\\
0.159053580148267	-0.00262451171875	\\
0.159097971323301	-0.00238037109375	\\
0.159142362498335	-0.00250244140625	\\
0.15918675367337	-0.0029296875	\\
0.159231144848404	-0.002655029296875	\\
0.159275536023439	-0.0028076171875	\\
0.159319927198473	-0.002899169921875	\\
0.159364318373507	-0.002899169921875	\\
0.159408709548542	-0.002532958984375	\\
0.159453100723576	-0.00244140625	\\
0.159497491898611	-0.002288818359375	\\
0.159541883073645	-0.00238037109375	\\
0.159586274248679	-0.002838134765625	\\
0.159630665423714	-0.002777099609375	\\
0.159675056598748	-0.002838134765625	\\
0.159719447773783	-0.003143310546875	\\
0.159763838948817	-0.003021240234375	\\
0.159808230123851	-0.00238037109375	\\
0.159852621298886	-0.00225830078125	\\
0.15989701247392	-0.002105712890625	\\
0.159941403648955	-0.002410888671875	\\
0.159985794823989	-0.002288818359375	\\
0.160030185999023	-0.00213623046875	\\
0.160074577174058	-0.002349853515625	\\
0.160118968349092	-0.00244140625	\\
0.160163359524127	-0.002410888671875	\\
0.160207750699161	-0.002044677734375	\\
0.160252141874195	-0.00201416015625	\\
0.16029653304923	-0.001922607421875	\\
0.160340924224264	-0.00146484375	\\
0.160385315399299	-0.001556396484375	\\
0.160429706574333	-0.001983642578125	\\
0.160474097749367	-0.0018310546875	\\
0.160518488924402	-0.001953125	\\
0.160562880099436	-0.002166748046875	\\
0.160607271274471	-0.002044677734375	\\
0.160651662449505	-0.002288818359375	\\
0.160696053624539	-0.002349853515625	\\
0.160740444799574	-0.00213623046875	\\
0.160784835974608	-0.002288818359375	\\
0.160829227149643	-0.00238037109375	\\
0.160873618324677	-0.002716064453125	\\
0.160918009499711	-0.0028076171875	\\
0.160962400674746	-0.00244140625	\\
0.16100679184978	-0.002777099609375	\\
0.161051183024815	-0.002838134765625	\\
0.161095574199849	-0.002197265625	\\
0.161139965374883	-0.001983642578125	\\
0.161184356549918	-0.001922607421875	\\
0.161228747724952	-0.00213623046875	\\
0.161273138899987	-0.002532958984375	\\
0.161317530075021	-0.0025634765625	\\
0.161361921250056	-0.002349853515625	\\
0.16140631242509	-0.001922607421875	\\
0.161450703600124	-0.001983642578125	\\
0.161495094775159	-0.001800537109375	\\
0.161539485950193	-0.00201416015625	\\
0.161583877125228	-0.002410888671875	\\
0.161628268300262	-0.001953125	\\
0.161672659475296	-0.001953125	\\
0.161717050650331	-0.002105712890625	\\
0.161761441825365	-0.001922607421875	\\
0.1618058330004	-0.002166748046875	\\
0.161850224175434	-0.001953125	\\
0.161894615350468	-0.0020751953125	\\
0.161939006525503	-0.002227783203125	\\
0.161983397700537	-0.002532958984375	\\
0.162027788875572	-0.00323486328125	\\
0.162072180050606	-0.00262451171875	\\
0.16211657122564	-0.002685546875	\\
0.162160962400675	-0.002593994140625	\\
0.162205353575709	-0.00250244140625	\\
0.162249744750744	-0.0025634765625	\\
0.162294135925778	-0.002197265625	\\
0.162338527100812	-0.00225830078125	\\
0.162382918275847	-0.002471923828125	\\
0.162427309450881	-0.00250244140625	\\
0.162471700625916	-0.00262451171875	\\
0.16251609180095	-0.0029296875	\\
0.162560482975984	-0.003143310546875	\\
0.162604874151019	-0.002899169921875	\\
0.162649265326053	-0.00286865234375	\\
0.162693656501088	-0.00250244140625	\\
0.162738047676122	-0.00201416015625	\\
0.162782438851156	-0.00250244140625	\\
0.162826830026191	-0.002532958984375	\\
0.162871221201225	-0.0025634765625	\\
0.16291561237626	-0.002655029296875	\\
0.162960003551294	-0.00274658203125	\\
0.163004394726328	-0.0028076171875	\\
0.163048785901363	-0.00244140625	\\
0.163093177076397	-0.00250244140625	\\
0.163137568251432	-0.00250244140625	\\
0.163181959426466	-0.00250244140625	\\
0.1632263506015	-0.00213623046875	\\
0.163270741776535	-0.00213623046875	\\
0.163315132951569	-0.0023193359375	\\
0.163359524126604	-0.00213623046875	\\
0.163403915301638	-0.00140380859375	\\
0.163448306476672	-0.00152587890625	\\
0.163492697651707	-0.00128173828125	\\
0.163537088826741	-0.00091552734375	\\
0.163581480001776	-0.00054931640625	\\
0.16362587117681	-0.00054931640625	\\
0.163670262351844	-0.000457763671875	\\
0.163714653526879	-0.00048828125	\\
0.163759044701913	-0.000946044921875	\\
0.163803435876948	-0.000396728515625	\\
0.163847827051982	-0.00048828125	\\
0.163892218227016	-0.000762939453125	\\
0.163936609402051	-0.00018310546875	\\
0.163981000577085	-0.0003662109375	\\
0.16402539175212	-0.000701904296875	\\
0.164069782927154	-0.001007080078125	\\
0.164114174102188	-0.001129150390625	\\
0.164158565277223	-0.001434326171875	\\
0.164202956452257	-0.00164794921875	\\
0.164247347627292	-0.001708984375	\\
0.164291738802326	-0.00201416015625	\\
0.164336129977361	-0.001708984375	\\
0.164380521152395	-0.00164794921875	\\
0.164424912327429	-0.002044677734375	\\
0.164469303502464	-0.002197265625	\\
0.164513694677498	-0.002288818359375	\\
0.164558085852533	-0.002655029296875	\\
0.164602477027567	-0.00299072265625	\\
0.164646868202601	-0.00262451171875	\\
0.164691259377636	-0.00225830078125	\\
0.16473565055267	-0.002349853515625	\\
0.164780041727705	-0.002166748046875	\\
0.164824432902739	-0.001739501953125	\\
0.164868824077773	-0.001495361328125	\\
0.164913215252808	-0.00103759765625	\\
0.164957606427842	-0.00103759765625	\\
0.165001997602877	-0.000946044921875	\\
0.165046388777911	-0.00079345703125	\\
0.165090779952945	-0.000396728515625	\\
0.16513517112798	-0.000396728515625	\\
0.165179562303014	-0.00054931640625	\\
0.165223953478049	-0.000457763671875	\\
0.165268344653083	-0.000457763671875	\\
0.165312735828117	6.103515625e-05	\\
0.165357127003152	-9.1552734375e-05	\\
0.165401518178186	-0.000244140625	\\
0.165445909353221	-0.0003662109375	\\
0.165490300528255	-0.0006103515625	\\
0.165534691703289	-0.000335693359375	\\
0.165579082878324	0.0001220703125	\\
0.165623474053358	-0.0001220703125	\\
0.165667865228393	-0.0008544921875	\\
0.165712256403427	-0.000518798828125	\\
0.165756647578461	-0.000457763671875	\\
0.165801038753496	-0.000946044921875	\\
0.16584542992853	-0.001007080078125	\\
0.165889821103565	-0.001007080078125	\\
0.165934212278599	-0.000885009765625	\\
0.165978603453633	-0.000762939453125	\\
0.166022994628668	-0.000579833984375	\\
0.166067385803702	-0.000640869140625	\\
0.166111776978737	-0.000732421875	\\
0.166156168153771	-0.00091552734375	\\
0.166200559328805	-0.00146484375	\\
0.16624495050384	-0.00152587890625	\\
0.166289341678874	-0.001220703125	\\
0.166333732853909	-0.001007080078125	\\
0.166378124028943	-0.00067138671875	\\
0.166422515203977	-0.0010986328125	\\
0.166466906379012	-0.001556396484375	\\
0.166511297554046	-0.0010986328125	\\
0.166555688729081	-0.001312255859375	\\
0.166600079904115	-0.0015869140625	\\
0.166644471079149	-0.0010986328125	\\
0.166688862254184	-0.00140380859375	\\
0.166733253429218	-0.00164794921875	\\
0.166777644604253	-0.001190185546875	\\
0.166822035779287	-0.00115966796875	\\
0.166866426954321	-0.001373291015625	\\
0.166910818129356	-0.001068115234375	\\
0.16695520930439	-0.001068115234375	\\
0.166999600479425	-0.000823974609375	\\
0.167043991654459	-0.000823974609375	\\
0.167088382829494	-0.0006103515625	\\
0.167132774004528	-0.0003662109375	\\
0.167177165179562	-0.00030517578125	\\
0.167221556354597	-0.00018310546875	\\
0.167265947529631	-9.1552734375e-05	\\
0.167310338704666	9.1552734375e-05	\\
0.1673547298797	0.0001220703125	\\
0.167399121054734	0.000244140625	\\
0.167443512229769	0.00042724609375	\\
0.167487903404803	0.00030517578125	\\
0.167532294579838	6.103515625e-05	\\
0.167576685754872	0.000213623046875	\\
0.167621076929906	0.000244140625	\\
0.167665468104941	0.00048828125	\\
0.167709859279975	0.0006103515625	\\
0.16775425045501	0.0003662109375	\\
0.167798641630044	0.00054931640625	\\
0.167843032805078	0.00079345703125	\\
0.167887423980113	0.00054931640625	\\
0.167931815155147	0.000946044921875	\\
0.167976206330182	0.001220703125	\\
0.168020597505216	0.00079345703125	\\
0.16806498868025	0.001312255859375	\\
0.168109379855285	0.001190185546875	\\
0.168153771030319	0.0010986328125	\\
0.168198162205354	0.001251220703125	\\
0.168242553380388	0.00091552734375	\\
0.168286944555422	0.000701904296875	\\
0.168331335730457	0.000885009765625	\\
0.168375726905491	0.000946044921875	\\
0.168420118080526	0.00067138671875	\\
0.16846450925556	0.000274658203125	\\
0.168508900430594	-6.103515625e-05	\\
0.168553291605629	6.103515625e-05	\\
0.168597682780663	0.000335693359375	\\
0.168642073955698	0.000396728515625	\\
0.168686465130732	0.000640869140625	\\
0.168730856305766	0.000396728515625	\\
0.168775247480801	0.0001220703125	\\
0.168819638655835	0.000457763671875	\\
0.16886402983087	-3.0517578125e-05	\\
0.168908421005904	-6.103515625e-05	\\
0.168952812180938	0.000579833984375	\\
0.168997203355973	0.00042724609375	\\
0.169041594531007	-6.103515625e-05	\\
0.169085985706042	-3.0517578125e-05	\\
0.169130376881076	-0.0001220703125	\\
0.16917476805611	-0.00048828125	\\
0.169219159231145	-0.00054931640625	\\
0.169263550406179	-0.00054931640625	\\
0.169307941581214	-6.103515625e-05	\\
0.169352332756248	-9.1552734375e-05	\\
0.169396723931282	-0.000152587890625	\\
0.169441115106317	0.00042724609375	\\
0.169485506281351	0.000213623046875	\\
0.169529897456386	0	\\
0.16957428863142	9.1552734375e-05	\\
0.169618679806454	-0.000213623046875	\\
0.169663070981489	-0.000457763671875	\\
0.169707462156523	-0.000457763671875	\\
0.169751853331558	-0.000762939453125	\\
0.169796244506592	-0.000885009765625	\\
0.169840635681627	-0.000732421875	\\
0.169885026856661	-0.000762939453125	\\
0.169929418031695	-0.00042724609375	\\
0.16997380920673	-0.000335693359375	\\
0.170018200381764	-0.000457763671875	\\
0.170062591556799	-0.000274658203125	\\
0.170106982731833	-0.00091552734375	\\
0.170151373906867	-0.001251220703125	\\
0.170195765081902	-0.0008544921875	\\
0.170240156256936	-0.0013427734375	\\
0.170284547431971	-0.001220703125	\\
0.170328938607005	-0.000732421875	\\
0.170373329782039	-0.001312255859375	\\
0.170417720957074	-0.00140380859375	\\
0.170462112132108	-0.0010986328125	\\
0.170506503307143	-0.00067138671875	\\
0.170550894482177	-0.000396728515625	\\
0.170595285657211	-0.00054931640625	\\
0.170639676832246	-0.00042724609375	\\
0.17068406800728	-0.00030517578125	\\
0.170728459182315	-0.000274658203125	\\
0.170772850357349	-0.000152587890625	\\
0.170817241532383	-3.0517578125e-05	\\
0.170861632707418	0.000274658203125	\\
0.170906023882452	0.00030517578125	\\
0.170950415057487	0.000701904296875	\\
0.170994806232521	0.0006103515625	\\
0.171039197407555	0.0003662109375	\\
0.17108358858259	0.001251220703125	\\
0.171127979757624	0.001434326171875	\\
0.171172370932659	0.00140380859375	\\
0.171216762107693	0.00128173828125	\\
0.171261153282727	0.0013427734375	\\
0.171305544457762	0.001373291015625	\\
0.171349935632796	0.0010986328125	\\
0.171394326807831	0.00146484375	\\
0.171438717982865	0.001373291015625	\\
0.171483109157899	0.001251220703125	\\
0.171527500332934	0.00146484375	\\
0.171571891507968	0.00140380859375	\\
0.171616282683003	0.001068115234375	\\
0.171660673858037	0.000640869140625	\\
0.171705065033071	0.000457763671875	\\
0.171749456208106	3.0517578125e-05	\\
0.17179384738314	-0.000274658203125	\\
0.171838238558175	-3.0517578125e-05	\\
0.171882629733209	6.103515625e-05	\\
0.171927020908243	0.00030517578125	\\
0.171971412083278	9.1552734375e-05	\\
0.172015803258312	-3.0517578125e-05	\\
0.172060194433347	-9.1552734375e-05	\\
0.172104585608381	-0.0006103515625	\\
0.172148976783415	-0.000701904296875	\\
0.17219336795845	-0.000640869140625	\\
0.172237759133484	-0.001068115234375	\\
0.172282150308519	-0.00079345703125	\\
0.172326541483553	-0.000152587890625	\\
0.172370932658587	0.0001220703125	\\
0.172415323833622	0.000732421875	\\
0.172459715008656	0.00079345703125	\\
0.172504106183691	0.001129150390625	\\
0.172548497358725	0.0013427734375	\\
0.172592888533759	0.00152587890625	\\
0.172637279708794	0.001739501953125	\\
0.172681670883828	0.00177001953125	\\
0.172726062058863	0.002197265625	\\
0.172770453233897	0.002105712890625	\\
0.172814844408932	0.002349853515625	\\
0.172859235583966	0.003021240234375	\\
0.172903626759	0.00323486328125	\\
0.172948017934035	0.002838134765625	\\
0.172992409109069	0.00225830078125	\\
0.173036800284104	0.00262451171875	\\
0.173081191459138	0.0029296875	\\
0.173125582634172	0.00372314453125	\\
0.173169973809207	0.003448486328125	\\
0.173214364984241	0.00311279296875	\\
0.173258756159276	0.00323486328125	\\
0.17330314733431	0.002685546875	\\
0.173347538509344	0.002899169921875	\\
0.173391929684379	0.00262451171875	\\
0.173436320859413	0.00213623046875	\\
0.173480712034448	0.0020751953125	\\
0.173525103209482	0.001861572265625	\\
0.173569494384516	0.001708984375	\\
0.173613885559551	0.001922607421875	\\
0.173658276734585	0.001983642578125	\\
0.17370266790962	0.00140380859375	\\
0.173747059084654	0.001220703125	\\
0.173791450259688	0.00115966796875	\\
0.173835841434723	0.0008544921875	\\
0.173880232609757	0.0008544921875	\\
0.173924623784792	0.00091552734375	\\
0.173969014959826	0.0008544921875	\\
0.17401340613486	0.0009765625	\\
0.174057797309895	0.001190185546875	\\
0.174102188484929	0.000823974609375	\\
0.174146579659964	0.000946044921875	\\
0.174190970834998	0.001190185546875	\\
0.174235362010032	0.0008544921875	\\
0.174279753185067	0.000579833984375	\\
0.174324144360101	0.00079345703125	\\
0.174368535535136	0.000579833984375	\\
0.17441292671017	0.000335693359375	\\
0.174457317885204	0.000213623046875	\\
0.174501709060239	0.00018310546875	\\
0.174546100235273	0.000823974609375	\\
0.174590491410308	0.00091552734375	\\
0.174634882585342	0.000335693359375	\\
0.174679273760376	0.00048828125	\\
0.174723664935411	0.00079345703125	\\
0.174768056110445	0.00079345703125	\\
0.17481244728548	0.000518798828125	\\
0.174856838460514	0.00042724609375	\\
0.174901229635548	0.000885009765625	\\
0.174945620810583	0.001007080078125	\\
0.174990011985617	0.000946044921875	\\
0.175034403160652	0.0008544921875	\\
0.175078794335686	0.001007080078125	\\
0.17512318551072	0.00115966796875	\\
0.175167576685755	0.000396728515625	\\
0.175211967860789	0.000152587890625	\\
0.175256359035824	0.00054931640625	\\
0.175300750210858	0.0003662109375	\\
0.175345141385892	6.103515625e-05	\\
0.175389532560927	0.000274658203125	\\
0.175433923735961	9.1552734375e-05	\\
0.175478314910996	0.0003662109375	\\
0.17552270608603	0.000244140625	\\
0.175567097261065	3.0517578125e-05	\\
0.175611488436099	0.000396728515625	\\
0.175655879611133	-0.00018310546875	\\
0.175700270786168	-0.000457763671875	\\
0.175744661961202	-0.000152587890625	\\
0.175789053136237	-0.00054931640625	\\
0.175833444311271	-0.00067138671875	\\
0.175877835486305	-0.000152587890625	\\
0.17592222666134	0.000152587890625	\\
0.175966617836374	0.000213623046875	\\
0.176011009011409	0.00042724609375	\\
0.176055400186443	0.000396728515625	\\
0.176099791361477	0.000152587890625	\\
0.176144182536512	0.000244140625	\\
0.176188573711546	6.103515625e-05	\\
0.176232964886581	-0.00018310546875	\\
0.176277356061615	0.000213623046875	\\
0.176321747236649	0.000579833984375	\\
0.176366138411684	0.000244140625	\\
0.176410529586718	0.00048828125	\\
0.176454920761753	0.000457763671875	\\
0.176499311936787	0.000274658203125	\\
0.176543703111821	0.00048828125	\\
0.176588094286856	0.000274658203125	\\
0.17663248546189	-0.000213623046875	\\
0.176676876636925	-0.00042724609375	\\
0.176721267811959	-0.0003662109375	\\
0.176765658986993	-0.000274658203125	\\
0.176810050162028	-9.1552734375e-05	\\
0.176854441337062	-0.000274658203125	\\
0.176898832512097	-0.000335693359375	\\
0.176943223687131	-0.000335693359375	\\
0.176987614862165	-0.000762939453125	\\
0.1770320060372	-0.000823974609375	\\
0.177076397212234	-0.00091552734375	\\
0.177120788387269	-0.0006103515625	\\
0.177165179562303	-0.000823974609375	\\
0.177209570737337	-0.00103759765625	\\
0.177253961912372	-0.00128173828125	\\
0.177298353087406	-0.00128173828125	\\
0.177342744262441	-0.0013427734375	\\
0.177387135437475	-0.001800537109375	\\
0.177431526612509	-0.001983642578125	\\
0.177475917787544	-0.001953125	\\
0.177520308962578	-0.001556396484375	\\
0.177564700137613	-0.001434326171875	\\
};
\addplot [color=blue,solid,forget plot]
  table[row sep=crcr]{
0.177564700137613	-0.001434326171875	\\
0.177609091312647	-0.00103759765625	\\
0.177653482487681	-0.00091552734375	\\
0.177697873662716	-0.001373291015625	\\
0.17774226483775	-0.001068115234375	\\
0.177786656012785	-0.000732421875	\\
0.177831047187819	-0.0009765625	\\
0.177875438362853	-0.00164794921875	\\
0.177919829537888	-0.001922607421875	\\
0.177964220712922	-0.001708984375	\\
0.178008611887957	-0.001922607421875	\\
0.178053003062991	-0.0023193359375	\\
0.178097394238025	-0.001556396484375	\\
0.17814178541306	-0.00152587890625	\\
0.178186176588094	-0.00201416015625	\\
0.178230567763129	-0.001861572265625	\\
0.178274958938163	-0.00213623046875	\\
0.178319350113197	-0.0023193359375	\\
0.178363741288232	-0.00213623046875	\\
0.178408132463266	-0.00225830078125	\\
0.178452523638301	-0.001861572265625	\\
0.178496914813335	-0.001953125	\\
0.17854130598837	-0.002227783203125	\\
0.178585697163404	-0.001861572265625	\\
0.178630088338438	-0.001312255859375	\\
0.178674479513473	-0.001556396484375	\\
0.178718870688507	-0.0015869140625	\\
0.178763261863542	-0.0015869140625	\\
0.178807653038576	-0.001800537109375	\\
0.17885204421361	-0.00103759765625	\\
0.178896435388645	-0.001251220703125	\\
0.178940826563679	-0.00140380859375	\\
0.178985217738714	-0.0010986328125	\\
0.179029608913748	-0.001129150390625	\\
0.179074000088782	-0.000885009765625	\\
0.179118391263817	-0.001312255859375	\\
0.179162782438851	-0.001556396484375	\\
0.179207173613886	-0.001190185546875	\\
0.17925156478892	-0.00140380859375	\\
0.179295955963954	-0.0015869140625	\\
0.179340347138989	-0.0015869140625	\\
0.179384738314023	-0.0020751953125	\\
0.179429129489058	-0.002227783203125	\\
0.179473520664092	-0.002349853515625	\\
0.179517911839126	-0.002593994140625	\\
0.179562303014161	-0.002899169921875	\\
0.179606694189195	-0.002532958984375	\\
0.17965108536423	-0.00262451171875	\\
0.179695476539264	-0.0028076171875	\\
0.179739867714298	-0.002716064453125	\\
0.179784258889333	-0.002685546875	\\
0.179828650064367	-0.0028076171875	\\
0.179873041239402	-0.003082275390625	\\
0.179917432414436	-0.002716064453125	\\
0.17996182358947	-0.002349853515625	\\
0.180006214764505	-0.0025634765625	\\
0.180050605939539	-0.002227783203125	\\
0.180094997114574	-0.002166748046875	\\
0.180139388289608	-0.001617431640625	\\
0.180183779464642	-0.000946044921875	\\
0.180228170639677	-0.000762939453125	\\
0.180272561814711	-0.000152587890625	\\
0.180316952989746	3.0517578125e-05	\\
0.18036134416478	-9.1552734375e-05	\\
0.180405735339814	0.00030517578125	\\
0.180450126514849	0.00079345703125	\\
0.180494517689883	0.0010986328125	\\
0.180538908864918	0.00115966796875	\\
0.180583300039952	0.001556396484375	\\
0.180627691214986	0.001739501953125	\\
0.180672082390021	0.0018310546875	\\
0.180716473565055	0.001861572265625	\\
0.18076086474009	0.0015869140625	\\
0.180805255915124	0.001373291015625	\\
0.180849647090158	0.001495361328125	\\
0.180894038265193	0.001190185546875	\\
0.180938429440227	0.001251220703125	\\
0.180982820615262	0.0013427734375	\\
0.181027211790296	0.0009765625	\\
0.18107160296533	0.00079345703125	\\
0.181115994140365	0.000274658203125	\\
0.181160385315399	3.0517578125e-05	\\
0.181204776490434	6.103515625e-05	\\
0.181249167665468	-6.103515625e-05	\\
0.181293558840502	-6.103515625e-05	\\
0.181337950015537	-0.0006103515625	\\
0.181382341190571	-0.000762939453125	\\
0.181426732365606	-0.0008544921875	\\
0.18147112354064	-0.000823974609375	\\
0.181515514715675	-0.0008544921875	\\
0.181559905890709	-0.001495361328125	\\
0.181604297065743	-0.00128173828125	\\
0.181648688240778	-0.00146484375	\\
0.181693079415812	-0.001861572265625	\\
0.181737470590847	-0.00152587890625	\\
0.181781861765881	-0.001129150390625	\\
0.181826252940915	-0.00115966796875	\\
0.18187064411595	-0.001190185546875	\\
0.181915035290984	-0.001251220703125	\\
0.181959426466019	-0.001190185546875	\\
0.182003817641053	-0.001251220703125	\\
0.182048208816087	-0.00128173828125	\\
0.182092599991122	-0.001068115234375	\\
0.182136991166156	-0.001129150390625	\\
0.182181382341191	-0.000823974609375	\\
0.182225773516225	-0.000762939453125	\\
0.182270164691259	-0.001129150390625	\\
0.182314555866294	-0.00079345703125	\\
0.182358947041328	-0.000579833984375	\\
0.182403338216363	-0.000274658203125	\\
0.182447729391397	0.0001220703125	\\
0.182492120566431	-6.103515625e-05	\\
0.182536511741466	-0.0001220703125	\\
0.1825809029165	0.000244140625	\\
0.182625294091535	0.000518798828125	\\
0.182669685266569	0.0006103515625	\\
0.182714076441603	0.000762939453125	\\
0.182758467616638	0.000579833984375	\\
0.182802858791672	0.00103759765625	\\
0.182847249966707	0.001129150390625	\\
0.182891641141741	0.00103759765625	\\
0.182936032316775	0.001129150390625	\\
0.18298042349181	0.001007080078125	\\
0.183024814666844	0.0009765625	\\
0.183069205841879	0.0009765625	\\
0.183113597016913	0.0009765625	\\
0.183157988191947	0.00091552734375	\\
0.183202379366982	0.000518798828125	\\
0.183246770542016	0.000396728515625	\\
0.183291161717051	0.00048828125	\\
0.183335552892085	0.00030517578125	\\
0.183379944067119	0.00048828125	\\
0.183424335242154	0.00054931640625	\\
0.183468726417188	0.00054931640625	\\
0.183513117592223	0.0003662109375	\\
0.183557508767257	0.000457763671875	\\
0.183601899942291	0.00054931640625	\\
0.183646291117326	-3.0517578125e-05	\\
0.18369068229236	-0.000701904296875	\\
0.183735073467395	-0.0006103515625	\\
0.183779464642429	-0.0001220703125	\\
0.183823855817463	-0.000274658203125	\\
0.183868246992498	6.103515625e-05	\\
0.183912638167532	0.00048828125	\\
0.183957029342567	0.0003662109375	\\
0.184001420517601	0.000152587890625	\\
0.184045811692636	0.00018310546875	\\
0.18409020286767	0.00042724609375	\\
0.184134594042704	0.0008544921875	\\
0.184178985217739	0.000640869140625	\\
0.184223376392773	0.000335693359375	\\
0.184267767567808	0.000213623046875	\\
0.184312158742842	-6.103515625e-05	\\
0.184356549917876	6.103515625e-05	\\
0.184400941092911	0.00067138671875	\\
0.184445332267945	0.000457763671875	\\
0.18448972344298	0.000457763671875	\\
0.184534114618014	0.000396728515625	\\
0.184578505793048	0.00018310546875	\\
0.184622896968083	0.0003662109375	\\
0.184667288143117	0.0001220703125	\\
0.184711679318152	-0.000244140625	\\
0.184756070493186	-0.000152587890625	\\
0.18480046166822	-0.000457763671875	\\
0.184844852843255	-0.000640869140625	\\
0.184889244018289	-0.00091552734375	\\
0.184933635193324	-0.001129150390625	\\
0.184978026368358	-0.001007080078125	\\
0.185022417543392	-0.00103759765625	\\
0.185066808718427	-0.00091552734375	\\
0.185111199893461	-0.0008544921875	\\
0.185155591068496	-0.00140380859375	\\
0.18519998224353	-0.001495361328125	\\
0.185244373418564	-0.00164794921875	\\
0.185288764593599	-0.002593994140625	\\
0.185333155768633	-0.0023193359375	\\
0.185377546943668	-0.001708984375	\\
0.185421938118702	-0.00213623046875	\\
0.185466329293736	-0.0018310546875	\\
0.185510720468771	-0.001678466796875	\\
0.185555111643805	-0.001922607421875	\\
0.18559950281884	-0.001556396484375	\\
0.185643893993874	-0.00146484375	\\
0.185688285168908	-0.001617431640625	\\
0.185732676343943	-0.001373291015625	\\
0.185777067518977	-0.00115966796875	\\
0.185821458694012	-0.001007080078125	\\
0.185865849869046	-0.001220703125	\\
0.18591024104408	-0.0009765625	\\
0.185954632219115	-0.000213623046875	\\
0.185999023394149	-0.00042724609375	\\
0.186043414569184	-0.000213623046875	\\
0.186087805744218	-0.00030517578125	\\
0.186132196919252	-0.000732421875	\\
0.186176588094287	-0.000579833984375	\\
0.186220979269321	-0.000640869140625	\\
0.186265370444356	-0.000885009765625	\\
0.18630976161939	-0.00042724609375	\\
0.186354152794424	0	\\
0.186398543969459	-0.000274658203125	\\
0.186442935144493	-0.00048828125	\\
0.186487326319528	-0.000457763671875	\\
0.186531717494562	-0.0003662109375	\\
0.186576108669596	-0.0003662109375	\\
0.186620499844631	-3.0517578125e-05	\\
0.186664891019665	6.103515625e-05	\\
0.1867092821947	6.103515625e-05	\\
0.186753673369734	0	\\
0.186798064544768	-6.103515625e-05	\\
0.186842455719803	-3.0517578125e-05	\\
0.186886846894837	-0.0003662109375	\\
0.186931238069872	-0.000732421875	\\
0.186975629244906	-0.00054931640625	\\
0.187020020419941	-0.000701904296875	\\
0.187064411594975	-0.0006103515625	\\
0.187108802770009	-0.000457763671875	\\
0.187153193945044	-0.00042724609375	\\
0.187197585120078	-0.000396728515625	\\
0.187241976295113	-0.0003662109375	\\
0.187286367470147	-0.0003662109375	\\
0.187330758645181	-0.000244140625	\\
0.187375149820216	-0.00018310546875	\\
0.18741954099525	0	\\
0.187463932170285	-6.103515625e-05	\\
0.187508323345319	3.0517578125e-05	\\
0.187552714520353	0.000701904296875	\\
0.187597105695388	0.00030517578125	\\
0.187641496870422	-6.103515625e-05	\\
0.187685888045457	0.000701904296875	\\
0.187730279220491	0.00115966796875	\\
0.187774670395525	0.00128173828125	\\
0.18781906157056	0.001373291015625	\\
0.187863452745594	0.00152587890625	\\
0.187907843920629	0.001678466796875	\\
0.187952235095663	0.001617431640625	\\
0.187996626270697	0.002044677734375	\\
0.188041017445732	0.001922607421875	\\
0.188085408620766	0.001983642578125	\\
0.188129799795801	0.002655029296875	\\
0.188174190970835	0.002960205078125	\\
0.188218582145869	0.00311279296875	\\
0.188262973320904	0.00347900390625	\\
0.188307364495938	0.00347900390625	\\
0.188351755670973	0.003448486328125	\\
0.188396146846007	0.004119873046875	\\
0.188440538021041	0.0037841796875	\\
0.188484929196076	0.00347900390625	\\
0.18852932037111	0.003692626953125	\\
0.188573711546145	0.003204345703125	\\
0.188618102721179	0.00299072265625	\\
0.188662493896213	0.002899169921875	\\
0.188706885071248	0.002777099609375	\\
0.188751276246282	0.00244140625	\\
0.188795667421317	0.00274658203125	\\
0.188840058596351	0.0025634765625	\\
0.188884449771385	0.002197265625	\\
0.18892884094642	0.00244140625	\\
0.188973232121454	0.002655029296875	\\
0.189017623296489	0.002532958984375	\\
0.189062014471523	0.002349853515625	\\
0.189106405646557	0.0020751953125	\\
0.189150796821592	0.00225830078125	\\
0.189195187996626	0.002349853515625	\\
0.189239579171661	0.002349853515625	\\
0.189283970346695	0.002685546875	\\
0.189328361521729	0.00323486328125	\\
0.189372752696764	0.00323486328125	\\
0.189417143871798	0.0030517578125	\\
0.189461535046833	0.00335693359375	\\
0.189505926221867	0.003570556640625	\\
0.189550317396902	0.003692626953125	\\
0.189594708571936	0.0035400390625	\\
0.18963909974697	0.003387451171875	\\
0.189683490922005	0.003173828125	\\
0.189727882097039	0.0037841796875	\\
0.189772273272074	0.004364013671875	\\
0.189816664447108	0.00396728515625	\\
0.189861055622142	0.00390625	\\
0.189905446797177	0.0037841796875	\\
0.189949837972211	0.003875732421875	\\
0.189994229147246	0.004302978515625	\\
0.19003862032228	0.003753662109375	\\
0.190083011497314	0.003631591796875	\\
0.190127402672349	0.0037841796875	\\
0.190171793847383	0.003631591796875	\\
0.190216185022418	0.00347900390625	\\
0.190260576197452	0.003875732421875	\\
0.190304967372486	0.004058837890625	\\
0.190349358547521	0.003692626953125	\\
0.190393749722555	0.003814697265625	\\
0.19043814089759	0.003570556640625	\\
0.190482532072624	0.00372314453125	\\
0.190526923247658	0.0037841796875	\\
0.190571314422693	0.003570556640625	\\
0.190615705597727	0.003692626953125	\\
0.190660096772762	0.003753662109375	\\
0.190704487947796	0.003631591796875	\\
0.19074887912283	0.003570556640625	\\
0.190793270297865	0.003631591796875	\\
0.190837661472899	0.00347900390625	\\
0.190882052647934	0.003509521484375	\\
0.190926443822968	0.003875732421875	\\
0.190970834998002	0.00408935546875	\\
0.191015226173037	0.00445556640625	\\
0.191059617348071	0.003814697265625	\\
0.191104008523106	0.003631591796875	\\
0.19114839969814	0.0040283203125	\\
0.191192790873174	0.004150390625	\\
0.191237182048209	0.0037841796875	\\
0.191281573223243	0.00360107421875	\\
0.191325964398278	0.00360107421875	\\
0.191370355573312	0.00335693359375	\\
0.191414746748346	0.003173828125	\\
0.191459137923381	0.003021240234375	\\
0.191503529098415	0.002838134765625	\\
0.19154792027345	0.002899169921875	\\
0.191592311448484	0.003173828125	\\
0.191636702623518	0.002777099609375	\\
0.191681093798553	0.002471923828125	\\
0.191725484973587	0.002777099609375	\\
0.191769876148622	0.0028076171875	\\
0.191814267323656	0.0028076171875	\\
0.19185865849869	0.003021240234375	\\
0.191903049673725	0.002655029296875	\\
0.191947440848759	0.00274658203125	\\
0.191991832023794	0.003021240234375	\\
0.192036223198828	0.002655029296875	\\
0.192080614373862	0.002349853515625	\\
0.192125005548897	0.0030517578125	\\
0.192169396723931	0.003173828125	\\
0.192213787898966	0.00244140625	\\
0.192258179074	0.002349853515625	\\
0.192302570249034	0.00274658203125	\\
0.192346961424069	0.00250244140625	\\
0.192391352599103	0.00244140625	\\
0.192435743774138	0.002410888671875	\\
0.192480134949172	0.00286865234375	\\
0.192524526124207	0.0028076171875	\\
0.192568917299241	0.002471923828125	\\
0.192613308474275	0.002716064453125	\\
0.19265769964931	0.002471923828125	\\
0.192702090824344	0.002197265625	\\
0.192746481999379	0.00189208984375	\\
0.192790873174413	0.001800537109375	\\
0.192835264349447	0.001708984375	\\
0.192879655524482	0.0015869140625	\\
0.192924046699516	0.00128173828125	\\
0.192968437874551	0.001434326171875	\\
0.193012829049585	0.001220703125	\\
0.193057220224619	0.000701904296875	\\
0.193101611399654	0.0009765625	\\
0.193146002574688	0.000946044921875	\\
0.193190393749723	0.000762939453125	\\
0.193234784924757	0.000518798828125	\\
0.193279176099791	0.00042724609375	\\
0.193323567274826	0.000335693359375	\\
0.19336795844986	0.00030517578125	\\
0.193412349624895	0.000274658203125	\\
0.193456740799929	-3.0517578125e-05	\\
0.193501131974963	-9.1552734375e-05	\\
0.193545523149998	-0.00018310546875	\\
0.193589914325032	0.00018310546875	\\
0.193634305500067	0.00054931640625	\\
0.193678696675101	0.00067138671875	\\
0.193723087850135	0.000457763671875	\\
0.19376747902517	0.000274658203125	\\
0.193811870200204	0.000946044921875	\\
0.193856261375239	0.0009765625	\\
0.193900652550273	0.00091552734375	\\
0.193945043725307	0.001312255859375	\\
0.193989434900342	0.000885009765625	\\
0.194033826075376	0.0010986328125	\\
0.194078217250411	0.001495361328125	\\
0.194122608425445	0.0018310546875	\\
0.194166999600479	0.00177001953125	\\
0.194211390775514	0.00189208984375	\\
0.194255781950548	0.002197265625	\\
0.194300173125583	0.00201416015625	\\
0.194344564300617	0.001861572265625	\\
0.194388955475651	0.00164794921875	\\
0.194433346650686	0.00164794921875	\\
0.19447773782572	0.001739501953125	\\
0.194522129000755	0.001434326171875	\\
0.194566520175789	0.001068115234375	\\
0.194610911350823	0.00079345703125	\\
0.194655302525858	0.000762939453125	\\
0.194699693700892	0.0008544921875	\\
0.194744084875927	0.000701904296875	\\
0.194788476050961	0.00067138671875	\\
0.194832867225995	0.0006103515625	\\
0.19487725840103	0.0008544921875	\\
0.194921649576064	0.000762939453125	\\
0.194966040751099	0.00030517578125	\\
0.195010431926133	-6.103515625e-05	\\
0.195054823101167	6.103515625e-05	\\
0.195099214276202	-0.000244140625	\\
0.195143605451236	0.000152587890625	\\
0.195187996626271	0.000244140625	\\
0.195232387801305	0	\\
0.195276778976339	-3.0517578125e-05	\\
0.195321170151374	6.103515625e-05	\\
0.195365561326408	0.0003662109375	\\
0.195409952501443	0.000396728515625	\\
0.195454343676477	6.103515625e-05	\\
0.195498734851512	0.000244140625	\\
0.195543126026546	0.00067138671875	\\
0.19558751720158	0.000762939453125	\\
0.195631908376615	0.000885009765625	\\
0.195676299551649	0.000762939453125	\\
0.195720690726684	0.000762939453125	\\
0.195765081901718	0.001190185546875	\\
0.195809473076752	0.001556396484375	\\
0.195853864251787	0.00115966796875	\\
0.195898255426821	0.001129150390625	\\
0.195942646601856	0.001251220703125	\\
0.19598703777689	0.001129150390625	\\
0.196031428951924	0.000457763671875	\\
0.196075820126959	0.0006103515625	\\
0.196120211301993	0.000762939453125	\\
0.196164602477028	0.00054931640625	\\
0.196208993652062	0.00042724609375	\\
0.196253384827096	0.000396728515625	\\
0.196297776002131	0.00030517578125	\\
0.196342167177165	-0.0001220703125	\\
0.1963865583522	-0.0003662109375	\\
0.196430949527234	-0.000457763671875	\\
0.196475340702268	0.00018310546875	\\
0.196519731877303	0.0001220703125	\\
0.196564123052337	-0.000457763671875	\\
0.196608514227372	0.000213623046875	\\
0.196652905402406	0.000396728515625	\\
0.19669729657744	-6.103515625e-05	\\
0.196741687752475	-0.000274658203125	\\
0.196786078927509	-0.000213623046875	\\
0.196830470102544	0.0001220703125	\\
0.196874861277578	0.000244140625	\\
0.196919252452612	6.103515625e-05	\\
0.196963643627647	0.0001220703125	\\
0.197008034802681	0.000152587890625	\\
0.197052425977716	0.000335693359375	\\
0.19709681715275	0.0003662109375	\\
0.197141208327784	0.00048828125	\\
0.197185599502819	0.000823974609375	\\
0.197229990677853	0.00079345703125	\\
0.197274381852888	0.00079345703125	\\
0.197318773027922	0.00103759765625	\\
0.197363164202956	0.0010986328125	\\
0.197407555377991	0.000732421875	\\
0.197451946553025	0.0010986328125	\\
0.19749633772806	0.00103759765625	\\
0.197540728903094	0.0010986328125	\\
0.197585120078128	0.001220703125	\\
0.197629511253163	0.000946044921875	\\
0.197673902428197	0.001312255859375	\\
0.197718293603232	0.001129150390625	\\
0.197762684778266	0.000823974609375	\\
0.1978070759533	0.001678466796875	\\
0.197851467128335	0.001922607421875	\\
0.197895858303369	0.001312255859375	\\
0.197940249478404	0.00152587890625	\\
0.197984640653438	0.001739501953125	\\
0.198029031828473	0.00152587890625	\\
0.198073423003507	0.001861572265625	\\
0.198117814178541	0.00201416015625	\\
0.198162205353576	0.00189208984375	\\
0.19820659652861	0.00201416015625	\\
0.198250987703645	0.002166748046875	\\
0.198295378878679	0.002227783203125	\\
0.198339770053713	0.001739501953125	\\
0.198384161228748	0.001861572265625	\\
0.198428552403782	0.001495361328125	\\
0.198472943578817	0.001434326171875	\\
0.198517334753851	0.00189208984375	\\
0.198561725928885	0.001953125	\\
0.19860611710392	0.001983642578125	\\
0.198650508278954	0.001922607421875	\\
0.198694899453989	0.001983642578125	\\
0.198739290629023	0.002105712890625	\\
0.198783681804057	0.002105712890625	\\
0.198828072979092	0.00177001953125	\\
0.198872464154126	0.00201416015625	\\
0.198916855329161	0.00201416015625	\\
0.198961246504195	0.001220703125	\\
0.199005637679229	0.001312255859375	\\
0.199050028854264	0.001434326171875	\\
0.199094420029298	0.001312255859375	\\
0.199138811204333	0.000732421875	\\
0.199183202379367	0.000244140625	\\
0.199227593554401	0.000579833984375	\\
0.199271984729436	0.00054931640625	\\
0.19931637590447	0.000579833984375	\\
0.199360767079505	0.000396728515625	\\
0.199405158254539	0.00030517578125	\\
0.199449549429573	-6.103515625e-05	\\
0.199493940604608	-0.000152587890625	\\
0.199538331779642	3.0517578125e-05	\\
0.199582722954677	-0.0001220703125	\\
0.199627114129711	-0.000640869140625	\\
0.199671505304745	-0.000518798828125	\\
0.19971589647978	-0.0003662109375	\\
0.199760287654814	-0.00067138671875	\\
0.199804678829849	-0.000762939453125	\\
0.199849070004883	-0.00054931640625	\\
0.199893461179917	-0.000701904296875	\\
0.199937852354952	-0.000823974609375	\\
0.199982243529986	-0.000885009765625	\\
0.200026634705021	-0.00115966796875	\\
0.200071025880055	-0.000885009765625	\\
0.200115417055089	-0.001007080078125	\\
0.200159808230124	-0.00146484375	\\
0.200204199405158	-0.00140380859375	\\
0.200248590580193	-0.00146484375	\\
0.200292981755227	-0.001678466796875	\\
0.200337372930261	-0.00146484375	\\
0.200381764105296	-0.0010986328125	\\
0.20042615528033	-0.00146484375	\\
0.200470546455365	-0.002227783203125	\\
0.200514937630399	-0.002471923828125	\\
0.200559328805433	-0.0020751953125	\\
0.200603719980468	-0.001495361328125	\\
0.200648111155502	-0.001800537109375	\\
0.200692502330537	-0.00201416015625	\\
0.200736893505571	-0.001953125	\\
0.200781284680605	-0.001922607421875	\\
0.20082567585564	-0.00128173828125	\\
0.200870067030674	-0.00115966796875	\\
0.200914458205709	-0.00140380859375	\\
0.200958849380743	-0.001556396484375	\\
0.201003240555778	-0.00152587890625	\\
0.201047631730812	-0.001220703125	\\
0.201092022905846	-0.00152587890625	\\
0.201136414080881	-0.001678466796875	\\
0.201180805255915	-0.001617431640625	\\
0.20122519643095	-0.0015869140625	\\
0.201269587605984	-0.001495361328125	\\
0.201313978781018	-0.00152587890625	\\
0.201358369956053	-0.00146484375	\\
0.201402761131087	-0.0015869140625	\\
0.201447152306122	-0.00189208984375	\\
0.201491543481156	-0.001922607421875	\\
0.20153593465619	-0.00189208984375	\\
0.201580325831225	-0.001922607421875	\\
0.201624717006259	-0.002471923828125	\\
0.201669108181294	-0.002410888671875	\\
0.201713499356328	-0.002227783203125	\\
0.201757890531362	-0.002410888671875	\\
0.201802281706397	-0.0028076171875	\\
0.201846672881431	-0.0029296875	\\
0.201891064056466	-0.00286865234375	\\
0.2019354552315	-0.0029296875	\\
0.201979846406534	-0.002777099609375	\\
0.202024237581569	-0.002960205078125	\\
0.202068628756603	-0.003143310546875	\\
0.202113019931638	-0.003143310546875	\\
0.202157411106672	-0.00299072265625	\\
0.202201802281706	-0.00286865234375	\\
0.202246193456741	-0.002685546875	\\
0.202290584631775	-0.002685546875	\\
0.20233497580681	-0.0029296875	\\
0.202379366981844	-0.0030517578125	\\
0.202423758156878	-0.003082275390625	\\
0.202468149331913	-0.003204345703125	\\
0.202512540506947	-0.00347900390625	\\
0.202556931681982	-0.0029296875	\\
0.202601322857016	-0.002899169921875	\\
0.20264571403205	-0.003082275390625	\\
0.202690105207085	-0.00262451171875	\\
0.202734496382119	-0.002471923828125	\\
0.202778887557154	-0.00244140625	\\
0.202823278732188	-0.00238037109375	\\
0.202867669907222	-0.002166748046875	\\
0.202912061082257	-0.002044677734375	\\
0.202956452257291	-0.001739501953125	\\
0.203000843432326	-0.001556396484375	\\
0.20304523460736	-0.00140380859375	\\
0.203089625782394	-0.00146484375	\\
0.203134016957429	-0.001617431640625	\\
0.203178408132463	-0.001434326171875	\\
0.203222799307498	-0.0013427734375	\\
0.203267190482532	-0.001708984375	\\
0.203311581657566	-0.001556396484375	\\
0.203355972832601	-0.001068115234375	\\
0.203400364007635	-0.001129150390625	\\
0.20344475518267	-0.0008544921875	\\
0.203489146357704	-0.00048828125	\\
0.203533537532738	-0.000640869140625	\\
0.203577928707773	-0.000946044921875	\\
0.203622319882807	-0.0013427734375	\\
0.203666711057842	-0.001251220703125	\\
0.203711102232876	-0.00103759765625	\\
0.203755493407911	-0.00140380859375	\\
0.203799884582945	-0.001617431640625	\\
0.203844275757979	-0.0013427734375	\\
0.203888666933014	-0.001434326171875	\\
0.203933058108048	-0.0015869140625	\\
0.203977449283083	-0.001922607421875	\\
0.204021840458117	-0.00201416015625	\\
0.204066231633151	-0.001953125	\\
0.204110622808186	-0.00250244140625	\\
0.20415501398322	-0.00262451171875	\\
0.204199405158255	-0.002349853515625	\\
0.204243796333289	-0.002227783203125	\\
0.204288187508323	-0.00244140625	\\
0.204332578683358	-0.00250244140625	\\
0.204376969858392	-0.00225830078125	\\
0.204421361033427	-0.0023193359375	\\
0.204465752208461	-0.00244140625	\\
0.204510143383495	-0.003082275390625	\\
0.20455453455853	-0.002838134765625	\\
0.204598925733564	-0.00250244140625	\\
0.204643316908599	-0.002349853515625	\\
0.204687708083633	-0.00177001953125	\\
0.204732099258667	-0.001617431640625	\\
0.204776490433702	-0.0015869140625	\\
0.204820881608736	-0.00128173828125	\\
0.204865272783771	-0.001190185546875	\\
0.204909663958805	-0.0009765625	\\
0.204954055133839	-0.001312255859375	\\
0.204998446308874	-0.00079345703125	\\
0.205042837483908	-0.00030517578125	\\
0.205087228658943	-0.0008544921875	\\
0.205131619833977	-0.00030517578125	\\
0.205176011009011	-0.00018310546875	\\
0.205220402184046	-0.00018310546875	\\
0.20526479335908	0	\\
0.205309184534115	-0.0003662109375	\\
0.205353575709149	-0.000579833984375	\\
0.205397966884183	-0.000152587890625	\\
0.205442358059218	0.000213623046875	\\
0.205486749234252	0.0001220703125	\\
0.205531140409287	0.000213623046875	\\
0.205575531584321	9.1552734375e-05	\\
0.205619922759355	-6.103515625e-05	\\
0.20566431393439	0.000244140625	\\
0.205708705109424	0.00030517578125	\\
0.205753096284459	0.000274658203125	\\
0.205797487459493	0.00048828125	\\
0.205841878634527	0.000335693359375	\\
0.205886269809562	-6.103515625e-05	\\
0.205930660984596	-0.000213623046875	\\
0.205975052159631	-0.0003662109375	\\
0.206019443334665	-0.000335693359375	\\
0.206063834509699	0	\\
0.206108225684734	0	\\
0.206152616859768	-0.000457763671875	\\
0.206197008034803	-0.000396728515625	\\
0.206241399209837	-0.000823974609375	\\
0.206285790384871	-0.000640869140625	\\
0.206330181559906	-0.000274658203125	\\
0.20637457273494	-3.0517578125e-05	\\
0.206418963909975	-0.00018310546875	\\
0.206463355085009	-9.1552734375e-05	\\
0.206507746260044	-0.000244140625	\\
0.206552137435078	-0.000762939453125	\\
0.206596528610112	-0.0003662109375	\\
0.206640919785147	-0.00042724609375	\\
0.206685310960181	-0.000213623046875	\\
0.206729702135216	0.00018310546875	\\
0.20677409331025	9.1552734375e-05	\\
0.206818484485284	0.0001220703125	\\
0.206862875660319	0.000213623046875	\\
0.206907266835353	9.1552734375e-05	\\
0.206951658010388	9.1552734375e-05	\\
0.206996049185422	-0.000213623046875	\\
0.207040440360456	-0.000274658203125	\\
0.207084831535491	-9.1552734375e-05	\\
0.207129222710525	0.00018310546875	\\
0.20717361388556	0.000274658203125	\\
0.207218005060594	0.0006103515625	\\
0.207262396235628	0.00103759765625	\\
0.207306787410663	0.000640869140625	\\
0.207351178585697	0.000640869140625	\\
0.207395569760732	0.00115966796875	\\
0.207439960935766	0.001007080078125	\\
0.2074843521108	0.001068115234375	\\
0.207528743285835	0.0013427734375	\\
0.207573134460869	0.00140380859375	\\
0.207617525635904	0.0013427734375	\\
0.207661916810938	0.0010986328125	\\
0.207706307985972	0.00067138671875	\\
0.207750699161007	0.0006103515625	\\
0.207795090336041	0.000762939453125	\\
0.207839481511076	0.000457763671875	\\
0.20788387268611	0.00030517578125	\\
0.207928263861144	0.000244140625	\\
0.207972655036179	0.000457763671875	\\
0.208017046211213	0.000396728515625	\\
0.208061437386248	0.000457763671875	\\
0.208105828561282	0.000335693359375	\\
0.208150219736316	0.00030517578125	\\
0.208194610911351	9.1552734375e-05	\\
0.208239002086385	-0.00030517578125	\\
0.20828339326142	-0.0006103515625	\\
0.208327784436454	-0.00091552734375	\\
0.208372175611488	-0.000885009765625	\\
0.208416566786523	-0.000701904296875	\\
0.208460957961557	-0.0003662109375	\\
0.208505349136592	-0.000335693359375	\\
0.208549740311626	-0.000396728515625	\\
0.20859413148666	-0.00048828125	\\
0.208638522661695	-0.00054931640625	\\
0.208682913836729	-0.000823974609375	\\
0.208727305011764	-0.00067138671875	\\
0.208771696186798	-0.000579833984375	\\
0.208816087361832	-0.000579833984375	\\
0.208860478536867	-0.00067138671875	\\
0.208904869711901	-0.000457763671875	\\
0.208949260886936	-6.103515625e-05	\\
0.20899365206197	-0.000335693359375	\\
0.209038043237004	-0.000640869140625	\\
0.209082434412039	-0.00048828125	\\
0.209126825587073	-0.000823974609375	\\
0.209171216762108	-0.0006103515625	\\
0.209215607937142	-0.000457763671875	\\
0.209259999112176	-0.00103759765625	\\
0.209304390287211	-0.000762939453125	\\
0.209348781462245	-0.0008544921875	\\
0.20939317263728	-0.001129150390625	\\
0.209437563812314	-0.0013427734375	\\
0.209481954987349	-0.00152587890625	\\
0.209526346162383	-0.001434326171875	\\
0.209570737337417	-0.001190185546875	\\
0.209615128512452	-0.001190185546875	\\
0.209659519687486	-0.001129150390625	\\
0.209703910862521	-0.0013427734375	\\
0.209748302037555	-0.001800537109375	\\
0.209792693212589	-0.001708984375	\\
0.209837084387624	-0.001800537109375	\\
0.209881475562658	-0.001983642578125	\\
0.209925866737693	-0.00201416015625	\\
0.209970257912727	-0.002105712890625	\\
0.210014649087761	-0.00177001953125	\\
0.210059040262796	-0.0018310546875	\\
0.21010343143783	-0.0020751953125	\\
0.210147822612865	-0.002227783203125	\\
0.210192213787899	-0.00238037109375	\\
0.210236604962933	-0.0023193359375	\\
0.210280996137968	-0.0023193359375	\\
0.210325387313002	-0.002593994140625	\\
0.210369778488037	-0.002532958984375	\\
0.210414169663071	-0.002105712890625	\\
0.210458560838105	-0.002227783203125	\\
0.21050295201314	-0.001800537109375	\\
0.210547343188174	-0.001708984375	\\
0.210591734363209	-0.001556396484375	\\
0.210636125538243	-0.001434326171875	\\
0.210680516713277	-0.00164794921875	\\
0.210724907888312	-0.0015869140625	\\
0.210769299063346	-0.001312255859375	\\
0.210813690238381	-0.0015869140625	\\
0.210858081413415	-0.001251220703125	\\
0.210902472588449	-0.0008544921875	\\
0.210946863763484	-0.001007080078125	\\
0.210991254938518	-0.000823974609375	\\
0.211035646113553	-0.000701904296875	\\
0.211080037288587	-0.00054931640625	\\
0.211124428463621	-0.000701904296875	\\
0.211168819638656	-0.000946044921875	\\
0.21121321081369	-0.00091552734375	\\
0.211257601988725	-0.0008544921875	\\
0.211301993163759	-0.000762939453125	\\
0.211346384338793	-0.000579833984375	\\
0.211390775513828	-0.000732421875	\\
0.211435166688862	-0.000732421875	\\
0.211479557863897	-0.0009765625	\\
0.211523949038931	-0.00091552734375	\\
0.211568340213965	-0.000762939453125	\\
0.211612731389	-0.0009765625	\\
0.211657122564034	-0.000946044921875	\\
0.211701513739069	-0.001190185546875	\\
0.211745904914103	-0.0009765625	\\
0.211790296089137	-0.00079345703125	\\
0.211834687264172	-0.001251220703125	\\
0.211879078439206	-0.000885009765625	\\
0.211923469614241	-0.0006103515625	\\
0.211967860789275	-0.000640869140625	\\
0.21201225196431	-0.000335693359375	\\
0.212056643139344	-0.0006103515625	\\
0.212101034314378	-0.000457763671875	\\
0.212145425489413	-6.103515625e-05	\\
0.212189816664447	-0.000152587890625	\\
0.212234207839482	-0.000244140625	\\
0.212278599014516	-0.0001220703125	\\
0.21232299018955	-0.000244140625	\\
0.212367381364585	-0.000213623046875	\\
0.212411772539619	-0.000244140625	\\
0.212456163714654	-0.00018310546875	\\
0.212500554889688	0.00030517578125	\\
0.212544946064722	0.000244140625	\\
0.212589337239757	0.000213623046875	\\
0.212633728414791	0.000457763671875	\\
0.212678119589826	0.000396728515625	\\
0.21272251076486	0.000335693359375	\\
0.212766901939894	-3.0517578125e-05	\\
0.212811293114929	0.000152587890625	\\
0.212855684289963	9.1552734375e-05	\\
0.212900075464998	-0.0003662109375	\\
0.212944466640032	0.00018310546875	\\
0.212988857815066	-3.0517578125e-05	\\
0.213033248990101	-0.000640869140625	\\
0.213077640165135	-0.0006103515625	\\
0.21312203134017	-0.000152587890625	\\
0.213166422515204	0.000274658203125	\\
0.213210813690238	3.0517578125e-05	\\
0.213255204865273	-0.000213623046875	\\
0.213299596040307	-0.000274658203125	\\
0.213343987215342	9.1552734375e-05	\\
0.213388378390376	0.00030517578125	\\
0.21343276956541	0.000335693359375	\\
0.213477160740445	0.000457763671875	\\
0.213521551915479	0.000244140625	\\
0.213565943090514	-0.000152587890625	\\
0.213610334265548	0.00054931640625	\\
0.213654725440582	6.103515625e-05	\\
0.213699116615617	-0.0006103515625	\\
0.213743507790651	3.0517578125e-05	\\
0.213787898965686	-0.0003662109375	\\
0.21383229014072	-0.00018310546875	\\
0.213876681315754	0.000579833984375	\\
0.213921072490789	0.00091552734375	\\
0.213965463665823	0.0009765625	\\
0.214009854840858	0.000946044921875	\\
0.214054246015892	0.000518798828125	\\
0.214098637190926	0.0003662109375	\\
0.214143028365961	0.000762939453125	\\
0.214187419540995	0.000396728515625	\\
0.21423181071603	0.00030517578125	\\
0.214276201891064	0.00103759765625	\\
0.214320593066098	0.00103759765625	\\
0.214364984241133	0.001434326171875	\\
0.214409375416167	0.001190185546875	\\
0.214453766591202	0.00054931640625	\\
0.214498157766236	0.000762939453125	\\
0.21454254894127	0.0009765625	\\
0.214586940116305	0.000579833984375	\\
0.214631331291339	0.000396728515625	\\
0.214675722466374	0.000579833984375	\\
0.214720113641408	0.000579833984375	\\
0.214764504816442	0.000732421875	\\
0.214808895991477	0.00054931640625	\\
0.214853287166511	0.000274658203125	\\
0.214897678341546	0.00030517578125	\\
0.21494206951658	9.1552734375e-05	\\
0.214986460691615	-6.103515625e-05	\\
0.215030851866649	9.1552734375e-05	\\
0.215075243041683	3.0517578125e-05	\\
0.215119634216718	-9.1552734375e-05	\\
0.215164025391752	-6.103515625e-05	\\
0.215208416566787	-6.103515625e-05	\\
0.215252807741821	9.1552734375e-05	\\
0.215297198916855	0.0001220703125	\\
0.21534159009189	0.000518798828125	\\
0.215385981266924	0.0006103515625	\\
0.215430372441959	0.000335693359375	\\
0.215474763616993	0.00042724609375	\\
0.215519154792027	0.000244140625	\\
0.215563545967062	0.0001220703125	\\
0.215607937142096	0.000213623046875	\\
0.215652328317131	-3.0517578125e-05	\\
0.215696719492165	9.1552734375e-05	\\
0.215741110667199	6.103515625e-05	\\
0.215785501842234	6.103515625e-05	\\
0.215829893017268	0.000396728515625	\\
0.215874284192303	0.000457763671875	\\
0.215918675367337	0.0006103515625	\\
0.215963066542371	0.000518798828125	\\
0.216007457717406	0.000396728515625	\\
0.21605184889244	0.0006103515625	\\
0.216096240067475	0.00018310546875	\\
0.216140631242509	-0.0003662109375	\\
0.216185022417543	-0.000335693359375	\\
0.216229413592578	-0.000152587890625	\\
0.216273804767612	-0.000213623046875	\\
0.216318195942647	-0.00030517578125	\\
0.216362587117681	0.000152587890625	\\
0.216406978292715	0.000274658203125	\\
0.21645136946775	0.000518798828125	\\
0.216495760642784	0.00042724609375	\\
0.216540151817819	0.000244140625	\\
0.216584542992853	0.00042724609375	\\
0.216628934167887	0.000244140625	\\
0.216673325342922	0.000152587890625	\\
0.216717716517956	-0.0001220703125	\\
0.216762107692991	-0.000274658203125	\\
0.216806498868025	-0.0001220703125	\\
0.216850890043059	-0.000152587890625	\\
0.216895281218094	-0.000244140625	\\
0.216939672393128	-0.0003662109375	\\
0.216984063568163	-0.0003662109375	\\
0.217028454743197	-0.000457763671875	\\
0.217072845918231	-0.00054931640625	\\
0.217117237093266	-0.000762939453125	\\
0.2171616282683	-0.001220703125	\\
0.217206019443335	-0.001373291015625	\\
0.217250410618369	-0.00152587890625	\\
0.217294801793403	-0.001495361328125	\\
0.217339192968438	-0.0015869140625	\\
0.217383584143472	-0.001800537109375	\\
0.217427975318507	-0.00189208984375	\\
0.217472366493541	-0.001983642578125	\\
0.217516757668575	-0.0020751953125	\\
0.21756114884361	-0.002349853515625	\\
0.217605540018644	-0.002349853515625	\\
0.217649931193679	-0.00244140625	\\
0.217694322368713	-0.00238037109375	\\
0.217738713543747	-0.00244140625	\\
0.217783104718782	-0.00250244140625	\\
0.217827495893816	-0.002471923828125	\\
0.217871887068851	-0.002593994140625	\\
0.217916278243885	-0.002838134765625	\\
0.21796066941892	-0.002410888671875	\\
0.218005060593954	-0.00225830078125	\\
0.218049451768988	-0.00244140625	\\
0.218093842944023	-0.00250244140625	\\
0.218138234119057	-0.00250244140625	\\
0.218182625294092	-0.002410888671875	\\
0.218227016469126	-0.00213623046875	\\
0.21827140764416	-0.00201416015625	\\
0.218315798819195	-0.00213623046875	\\
0.218360189994229	-0.001922607421875	\\
0.218404581169264	-0.001556396484375	\\
0.218448972344298	-0.00146484375	\\
0.218493363519332	-0.0015869140625	\\
0.218537754694367	-0.001556396484375	\\
0.218582145869401	-0.000946044921875	\\
0.218626537044436	-0.00067138671875	\\
0.21867092821947	-0.00091552734375	\\
0.218715319394504	-0.0009765625	\\
0.218759710569539	-0.0013427734375	\\
0.218804101744573	-0.00115966796875	\\
0.218848492919608	-0.0009765625	\\
0.218892884094642	-0.00177001953125	\\
0.218937275269676	-0.001953125	\\
0.218981666444711	-0.001800537109375	\\
0.219026057619745	-0.00189208984375	\\
0.21907044879478	-0.0018310546875	\\
0.219114839969814	-0.002471923828125	\\
0.219159231144848	-0.003173828125	\\
0.219203622319883	-0.002899169921875	\\
0.219248013494917	-0.003326416015625	\\
0.219292404669952	-0.00335693359375	\\
0.219336795844986	-0.003570556640625	\\
0.21938118702002	-0.003936767578125	\\
0.219425578195055	-0.003814697265625	\\
0.219469969370089	-0.003997802734375	\\
0.219514360545124	-0.004119873046875	\\
0.219558751720158	-0.003936767578125	\\
0.219603142895192	-0.004302978515625	\\
0.219647534070227	-0.00445556640625	\\
0.219691925245261	-0.0042724609375	\\
0.219736316420296	-0.00433349609375	\\
0.21978070759533	-0.004180908203125	\\
0.219825098770364	-0.004302978515625	\\
0.219869489945399	-0.004486083984375	\\
0.219913881120433	-0.00445556640625	\\
0.219958272295468	-0.004241943359375	\\
0.220002663470502	-0.003936767578125	\\
0.220047054645536	-0.0040283203125	\\
0.220091445820571	-0.0037841796875	\\
0.220135836995605	-0.00323486328125	\\
0.22018022817064	-0.0032958984375	\\
0.220224619345674	-0.0029296875	\\
0.220269010520708	-0.002716064453125	\\
0.220313401695743	-0.0028076171875	\\
0.220357792870777	-0.0025634765625	\\
0.220402184045812	-0.00213623046875	\\
0.220446575220846	-0.00213623046875	\\
0.220490966395881	-0.002044677734375	\\
0.220535357570915	-0.0020751953125	\\
0.220579748745949	-0.002166748046875	\\
0.220624139920984	-0.00213623046875	\\
0.220668531096018	-0.002105712890625	\\
0.220712922271053	-0.001678466796875	\\
0.220757313446087	-0.00152587890625	\\
0.220801704621121	-0.001617431640625	\\
0.220846095796156	-0.0020751953125	\\
0.22089048697119	-0.002532958984375	\\
0.220934878146225	-0.002593994140625	\\
0.220979269321259	-0.002960205078125	\\
0.221023660496293	-0.0030517578125	\\
0.221068051671328	-0.00274658203125	\\
0.221112442846362	-0.002532958984375	\\
0.221156834021397	-0.00311279296875	\\
0.221201225196431	-0.0030517578125	\\
0.221245616371465	-0.0032958984375	\\
0.2212900075465	-0.003509521484375	\\
0.221334398721534	-0.00286865234375	\\
0.221378789896569	-0.0032958984375	\\
0.221423181071603	-0.003082275390625	\\
0.221467572246637	-0.0029296875	\\
0.221511963421672	-0.003662109375	\\
0.221556354596706	-0.00372314453125	\\
0.221600745771741	-0.003143310546875	\\
0.221645136946775	-0.0025634765625	\\
0.221689528121809	-0.00262451171875	\\
0.221733919296844	-0.002777099609375	\\
0.221778310471878	-0.0030517578125	\\
0.221822701646913	-0.002777099609375	\\
0.221867092821947	-0.002532958984375	\\
0.221911483996981	-0.002655029296875	\\
0.221955875172016	-0.002532958984375	\\
0.22200026634705	-0.002593994140625	\\
0.222044657522085	-0.00262451171875	\\
0.222089048697119	-0.002593994140625	\\
0.222133439872153	-0.0025634765625	\\
0.222177831047188	-0.002349853515625	\\
0.222222222222222	-0.0023193359375	\\
0.222266613397257	-0.001861572265625	\\
0.222311004572291	-0.001495361328125	\\
0.222355395747325	-0.00225830078125	\\
0.22239978692236	-0.001953125	\\
0.222444178097394	-0.001617431640625	\\
0.222488569272429	-0.00177001953125	\\
0.222532960447463	-0.001495361328125	\\
0.222577351622497	-0.001708984375	\\
0.222621742797532	-0.001495361328125	\\
0.222666133972566	-0.001617431640625	\\
0.222710525147601	-0.001739501953125	\\
0.222754916322635	-0.00164794921875	\\
0.222799307497669	-0.00146484375	\\
0.222843698672704	-0.001617431640625	\\
0.222888089847738	-0.002044677734375	\\
0.222932481022773	-0.001800537109375	\\
0.222976872197807	-0.001434326171875	\\
0.223021263372841	-0.00103759765625	\\
0.223065654547876	-0.001251220703125	\\
0.22311004572291	-0.001556396484375	\\
0.223154436897945	-0.00103759765625	\\
0.223198828072979	-0.0010986328125	\\
0.223243219248013	-0.001129150390625	\\
0.223287610423048	-0.001251220703125	\\
0.223332001598082	-0.00140380859375	\\
0.223376392773117	-0.001373291015625	\\
0.223420783948151	-0.00128173828125	\\
0.223465175123186	-0.001495361328125	\\
0.22350956629822	-0.0020751953125	\\
0.223553957473254	-0.001983642578125	\\
0.223598348648289	-0.001953125	\\
0.223642739823323	-0.002044677734375	\\
0.223687130998358	-0.0020751953125	\\
0.223731522173392	-0.002227783203125	\\
0.223775913348426	-0.002227783203125	\\
0.223820304523461	-0.00244140625	\\
0.223864695698495	-0.002655029296875	\\
0.22390908687353	-0.002716064453125	\\
0.223953478048564	-0.002655029296875	\\
0.223997869223598	-0.0025634765625	\\
0.224042260398633	-0.002349853515625	\\
0.224086651573667	-0.002166748046875	\\
0.224131042748702	-0.002197265625	\\
0.224175433923736	-0.002044677734375	\\
0.22421982509877	-0.002166748046875	\\
0.224264216273805	-0.001800537109375	\\
0.224308607448839	-0.001953125	\\
0.224352998623874	-0.002349853515625	\\
0.224397389798908	-0.00238037109375	\\
0.224441780973942	-0.002349853515625	\\
0.224486172148977	-0.002044677734375	\\
0.224530563324011	-0.00164794921875	\\
0.224574954499046	-0.001617431640625	\\
0.22461934567408	-0.00213623046875	\\
0.224663736849114	-0.001983642578125	\\
0.224708128024149	-0.001861572265625	\\
0.224752519199183	-0.002044677734375	\\
0.224796910374218	-0.002288818359375	\\
0.224841301549252	-0.0023193359375	\\
0.224885692724286	-0.002288818359375	\\
0.224930083899321	-0.00250244140625	\\
0.224974475074355	-0.002593994140625	\\
0.22501886624939	-0.002838134765625	\\
0.225063257424424	-0.00323486328125	\\
0.225107648599458	-0.002960205078125	\\
0.225152039774493	-0.0029296875	\\
0.225196430949527	-0.002777099609375	\\
0.225240822124562	-0.002838134765625	\\
0.225285213299596	-0.0030517578125	\\
0.22532960447463	-0.002655029296875	\\
0.225373995649665	-0.00250244140625	\\
0.225418386824699	-0.002899169921875	\\
0.225462777999734	-0.0029296875	\\
0.225507169174768	-0.00262451171875	\\
0.225551560349802	-0.0030517578125	\\
0.225595951524837	-0.002838134765625	\\
0.225640342699871	-0.00250244140625	\\
0.225684733874906	-0.00262451171875	\\
0.22572912504994	-0.002593994140625	\\
0.225773516224974	-0.002685546875	\\
0.225817907400009	-0.0023193359375	\\
0.225862298575043	-0.002838134765625	\\
0.225906689750078	-0.002899169921875	\\
0.225951080925112	-0.00311279296875	\\
0.225995472100146	-0.003387451171875	\\
0.226039863275181	-0.0032958984375	\\
0.226084254450215	-0.0028076171875	\\
0.22612864562525	-0.00238037109375	\\
0.226173036800284	-0.002410888671875	\\
0.226217427975319	-0.00201416015625	\\
0.226261819150353	-0.002197265625	\\
0.226306210325387	-0.002227783203125	\\
0.226350601500422	-0.002166748046875	\\
0.226394992675456	-0.002227783203125	\\
0.226439383850491	-0.0020751953125	\\
0.226483775025525	-0.002044677734375	\\
0.226528166200559	-0.001678466796875	\\
0.226572557375594	-0.001617431640625	\\
0.226616948550628	-0.0018310546875	\\
0.226661339725663	-0.00189208984375	\\
0.226705730900697	-0.00213623046875	\\
0.226750122075731	-0.00250244140625	\\
0.226794513250766	-0.002410888671875	\\
0.2268389044258	-0.002410888671875	\\
0.226883295600835	-0.002105712890625	\\
0.226927686775869	-0.0020751953125	\\
0.226972077950903	-0.00262451171875	\\
0.227016469125938	-0.002960205078125	\\
0.227060860300972	-0.00341796875	\\
0.227105251476007	-0.00311279296875	\\
0.227149642651041	-0.0032958984375	\\
0.227194033826075	-0.00341796875	\\
0.22723842500111	-0.003814697265625	\\
0.227282816176144	-0.00390625	\\
0.227327207351179	-0.003509521484375	\\
0.227371598526213	-0.00360107421875	\\
0.227415989701247	-0.00341796875	\\
0.227460380876282	-0.003631591796875	\\
0.227504772051316	-0.00372314453125	\\
0.227549163226351	-0.00360107421875	\\
0.227593554401385	-0.003387451171875	\\
0.227637945576419	-0.003204345703125	\\
0.227682336751454	-0.002532958984375	\\
0.227726727926488	-0.002288818359375	\\
0.227771119101523	-0.00201416015625	\\
0.227815510276557	-0.0020751953125	\\
0.227859901451591	-0.002593994140625	\\
0.227904292626626	-0.00225830078125	\\
0.22794868380166	-0.00213623046875	\\
0.227993074976695	-0.002288818359375	\\
0.228037466151729	-0.00213623046875	\\
0.228081857326763	-0.002227783203125	\\
0.228126248501798	-0.002044677734375	\\
0.228170639676832	-0.001434326171875	\\
0.228215030851867	-0.001251220703125	\\
0.228259422026901	-0.0013427734375	\\
0.228303813201935	-0.00152587890625	\\
0.22834820437697	-0.0018310546875	\\
0.228392595552004	-0.001739501953125	\\
0.228436986727039	-0.0015869140625	\\
0.228481377902073	-0.001373291015625	\\
0.228525769077107	-0.0009765625	\\
0.228570160252142	-0.0010986328125	\\
0.228614551427176	-0.001373291015625	\\
0.228658942602211	-0.0009765625	\\
0.228703333777245	-0.000640869140625	\\
0.228747724952279	-0.000518798828125	\\
0.228792116127314	-0.0008544921875	\\
0.228836507302348	-0.001007080078125	\\
0.228880898477383	-0.001220703125	\\
0.228925289652417	-0.001312255859375	\\
0.228969680827452	-0.001373291015625	\\
0.229014072002486	-0.001220703125	\\
0.22905846317752	-0.0013427734375	\\
0.229102854352555	-0.00146484375	\\
0.229147245527589	-0.0008544921875	\\
0.229191636702624	-0.00091552734375	\\
0.229236027877658	-0.000885009765625	\\
0.229280419052692	-0.000762939453125	\\
0.229324810227727	-0.0009765625	\\
0.229369201402761	-0.0006103515625	\\
0.229413592577796	-0.000701904296875	\\
0.22945798375283	-0.0008544921875	\\
0.229502374927864	-0.000823974609375	\\
0.229546766102899	-0.00103759765625	\\
0.229591157277933	-0.000885009765625	\\
0.229635548452968	-0.001007080078125	\\
0.229679939628002	-0.00115966796875	\\
0.229724330803036	-0.0008544921875	\\
0.229768721978071	-0.00048828125	\\
0.229813113153105	-0.00048828125	\\
0.22985750432814	-0.000640869140625	\\
0.229901895503174	-0.000213623046875	\\
0.229946286678208	-0.00018310546875	\\
0.229990677853243	-0.000732421875	\\
0.230035069028277	-0.000732421875	\\
0.230079460203312	-0.000579833984375	\\
0.230123851378346	-0.00067138671875	\\
0.23016824255338	-0.00042724609375	\\
0.230212633728415	-0.0001220703125	\\
0.230257024903449	-0.000518798828125	\\
0.230301416078484	-0.00054931640625	\\
0.230345807253518	-0.0001220703125	\\
0.230390198428552	-0.000152587890625	\\
0.230434589603587	-0.000274658203125	\\
0.230478980778621	-0.000152587890625	\\
0.230523371953656	-0.000579833984375	\\
0.23056776312869	-0.000518798828125	\\
0.230612154303724	-0.000396728515625	\\
0.230656545478759	-0.000152587890625	\\
0.230700936653793	-3.0517578125e-05	\\
0.230745327828828	-0.000518798828125	\\
0.230789719003862	-0.00054931640625	\\
0.230834110178896	-0.000244140625	\\
0.230878501353931	-0.000640869140625	\\
0.230922892528965	-0.000396728515625	\\
0.230967283704	-0.00018310546875	\\
0.231011674879034	-0.000762939453125	\\
0.231056066054068	-0.00091552734375	\\
0.231100457229103	-0.000457763671875	\\
0.231144848404137	-0.000457763671875	\\
0.231189239579172	-0.000701904296875	\\
0.231233630754206	-0.000762939453125	\\
0.23127802192924	-0.000946044921875	\\
0.231322413104275	-0.000946044921875	\\
0.231366804279309	-0.000579833984375	\\
0.231411195454344	-0.000244140625	\\
0.231455586629378	-0.000579833984375	\\
0.231499977804412	-0.00042724609375	\\
0.231544368979447	-0.00018310546875	\\
0.231588760154481	-0.00048828125	\\
0.231633151329516	-0.000396728515625	\\
0.23167754250455	-0.000244140625	\\
0.231721933679584	-0.00018310546875	\\
0.231766324854619	-0.000457763671875	\\
0.231810716029653	-6.103515625e-05	\\
0.231855107204688	9.1552734375e-05	\\
0.231899498379722	-0.000244140625	\\
0.231943889554757	-3.0517578125e-05	\\
0.231988280729791	-9.1552734375e-05	\\
0.232032671904825	6.103515625e-05	\\
0.23207706307986	0.000213623046875	\\
0.232121454254894	9.1552734375e-05	\\
0.232165845429929	-6.103515625e-05	\\
0.232210236604963	-0.00054931640625	\\
0.232254627779997	-0.000213623046875	\\
0.232299018955032	-0.00018310546875	\\
0.232343410130066	-0.000335693359375	\\
0.232387801305101	-0.00054931640625	\\
0.232432192480135	-0.000946044921875	\\
0.232476583655169	-0.0010986328125	\\
0.232520974830204	-0.000885009765625	\\
0.232565366005238	-0.00103759765625	\\
0.232609757180273	-0.001129150390625	\\
0.232654148355307	-0.00115966796875	\\
0.232698539530341	-0.001434326171875	\\
0.232742930705376	-0.00128173828125	\\
0.23278732188041	-0.0009765625	\\
0.232831713055445	-0.000946044921875	\\
0.232876104230479	-0.0010986328125	\\
0.232920495405513	-0.00115966796875	\\
0.232964886580548	-0.001251220703125	\\
0.233009277755582	-0.0015869140625	\\
0.233053668930617	-0.0018310546875	\\
0.233098060105651	-0.001708984375	\\
0.233142451280685	-0.001495361328125	\\
0.23318684245572	-0.001129150390625	\\
0.233231233630754	-0.000823974609375	\\
0.233275624805789	-0.00067138671875	\\
0.233320015980823	-0.000518798828125	\\
0.233364407155857	-0.0006103515625	\\
0.233408798330892	-0.000762939453125	\\
0.233453189505926	-0.00054931640625	\\
0.233497580680961	-0.000396728515625	\\
0.233541971855995	-0.000518798828125	\\
0.233586363031029	-0.000396728515625	\\
0.233630754206064	-0.00042724609375	\\
0.233675145381098	-0.00018310546875	\\
0.233719536556133	-0.000244140625	\\
0.233763927731167	-0.000640869140625	\\
0.233808318906201	-0.00048828125	\\
0.233852710081236	-0.000244140625	\\
0.23389710125627	-6.103515625e-05	\\
0.233941492431305	-9.1552734375e-05	\\
0.233985883606339	-0.000152587890625	\\
0.234030274781373	-6.103515625e-05	\\
0.234074665956408	-0.000152587890625	\\
0.234119057131442	-0.000335693359375	\\
0.234163448306477	-0.000396728515625	\\
0.234207839481511	-0.000274658203125	\\
0.234252230656545	-0.000640869140625	\\
0.23429662183158	-0.00067138671875	\\
0.234341013006614	-0.00018310546875	\\
0.234385404181649	-0.000518798828125	\\
0.234429795356683	-0.0006103515625	\\
0.234474186531718	-0.000152587890625	\\
0.234518577706752	-0.000457763671875	\\
0.234562968881786	-0.000579833984375	\\
0.234607360056821	-0.000579833984375	\\
0.234651751231855	-0.000823974609375	\\
0.23469614240689	-0.00103759765625	\\
0.234740533581924	-0.001068115234375	\\
0.234784924756958	-0.001373291015625	\\
0.234829315931993	-0.002105712890625	\\
0.234873707107027	-0.002197265625	\\
0.234918098282062	-0.001953125	\\
0.234962489457096	-0.001922607421875	\\
0.23500688063213	-0.001922607421875	\\
0.235051271807165	-0.00201416015625	\\
0.235095662982199	-0.001861572265625	\\
0.235140054157234	-0.00177001953125	\\
0.235184445332268	-0.001922607421875	\\
0.235228836507302	-0.001708984375	\\
0.235273227682337	-0.001678466796875	\\
0.235317618857371	-0.001678466796875	\\
0.235362010032406	-0.001373291015625	\\
0.23540640120744	-0.00115966796875	\\
0.235450792382474	-0.00128173828125	\\
0.235495183557509	-0.000762939453125	\\
0.235539574732543	-0.00030517578125	\\
0.235583965907578	-0.000640869140625	\\
0.235628357082612	-0.00030517578125	\\
0.235672748257646	-0.000335693359375	\\
0.235717139432681	-0.000518798828125	\\
0.235761530607715	-0.00030517578125	\\
0.23580592178275	-0.000213623046875	\\
0.235850312957784	-9.1552734375e-05	\\
0.235894704132818	3.0517578125e-05	\\
0.235939095307853	-0.000579833984375	\\
0.235983486482887	-0.000518798828125	\\
0.236027877657922	-0.000244140625	\\
0.236072268832956	-0.000213623046875	\\
0.23611666000799	0.0001220703125	\\
0.236161051183025	-0.00048828125	\\
0.236205442358059	-0.00067138671875	\\
0.236249833533094	-0.00048828125	\\
0.236294224708128	-0.00048828125	\\
0.236338615883162	-0.000762939453125	\\
0.236383007058197	-0.001007080078125	\\
0.236427398233231	-0.0010986328125	\\
0.236471789408266	-0.000732421875	\\
0.2365161805833	-0.000732421875	\\
0.236560571758334	-0.001220703125	\\
0.236604962933369	-0.0013427734375	\\
0.236649354108403	-0.001220703125	\\
0.236693745283438	-0.0008544921875	\\
0.236738136458472	-0.00030517578125	\\
0.236782527633506	-0.000885009765625	\\
0.236826918808541	-0.000762939453125	\\
0.236871309983575	-0.00030517578125	\\
0.23691570115861	-0.0008544921875	\\
0.236960092333644	-0.000823974609375	\\
0.237004483508678	-0.000579833984375	\\
0.237048874683713	-0.0006103515625	\\
0.237093265858747	-0.000244140625	\\
0.237137657033782	-0.000244140625	\\
0.237182048208816	-0.000152587890625	\\
0.23722643938385	0.0003662109375	\\
0.237270830558885	0.000152587890625	\\
0.237315221733919	-3.0517578125e-05	\\
0.237359612908954	-0.00018310546875	\\
0.237404004083988	0.0001220703125	\\
0.237448395259023	9.1552734375e-05	\\
0.237492786434057	0.000244140625	\\
0.237537177609091	0.000396728515625	\\
0.237581568784126	0.0006103515625	\\
0.23762595995916	0.00067138671875	\\
0.237670351134195	0.000579833984375	\\
0.237714742309229	0.00067138671875	\\
0.237759133484263	0.0009765625	\\
0.237803524659298	0.00140380859375	\\
0.237847915834332	0.001495361328125	\\
0.237892307009367	0.001556396484375	\\
0.237936698184401	0.001922607421875	\\
0.237981089359435	0.001434326171875	\\
0.23802548053447	0.001220703125	\\
0.238069871709504	0.00091552734375	\\
0.238114262884539	0.000701904296875	\\
0.238158654059573	0.000701904296875	\\
0.238203045234607	0.00048828125	\\
0.238247436409642	0.00042724609375	\\
0.238291827584676	-6.103515625e-05	\\
0.238336218759711	6.103515625e-05	\\
0.238380609934745	3.0517578125e-05	\\
0.238425001109779	-0.000274658203125	\\
0.238469392284814	0	\\
0.238513783459848	-9.1552734375e-05	\\
0.238558174634883	-0.00042724609375	\\
0.238602565809917	-0.000244140625	\\
0.238646956984951	-0.000701904296875	\\
0.238691348159986	-0.000732421875	\\
0.23873573933502	-0.000213623046875	\\
0.238780130510055	-0.0006103515625	\\
0.238824521685089	-0.0008544921875	\\
0.238868912860123	-0.00091552734375	\\
0.238913304035158	-0.000823974609375	\\
0.238957695210192	-0.001007080078125	\\
0.239002086385227	-0.0009765625	\\
0.239046477560261	-0.0006103515625	\\
0.239090868735295	-0.000457763671875	\\
0.23913525991033	-0.000732421875	\\
0.239179651085364	-0.00103759765625	\\
0.239224042260399	-0.0006103515625	\\
0.239268433435433	-0.000732421875	\\
0.239312824610467	-0.00091552734375	\\
0.239357215785502	-0.000579833984375	\\
0.239401606960536	-0.000701904296875	\\
0.239445998135571	-0.000701904296875	\\
0.239490389310605	-0.000518798828125	\\
0.239534780485639	-0.000732421875	\\
0.239579171660674	-0.000579833984375	\\
0.239623562835708	-0.000701904296875	\\
0.239667954010743	-0.000213623046875	\\
0.239712345185777	-0.000274658203125	\\
0.239756736360811	-0.00067138671875	\\
0.239801127535846	0	\\
0.23984551871088	0.000457763671875	\\
0.239889909885915	0.000152587890625	\\
0.239934301060949	6.103515625e-05	\\
0.239978692235983	0.00030517578125	\\
0.240023083411018	9.1552734375e-05	\\
0.240067474586052	-6.103515625e-05	\\
0.240111865761087	-6.103515625e-05	\\
0.240156256936121	-0.00054931640625	\\
0.240200648111155	-0.00067138671875	\\
0.24024503928619	-0.000885009765625	\\
0.240289430461224	-0.001373291015625	\\
0.240333821636259	-0.001220703125	\\
0.240378212811293	-0.0010986328125	\\
0.240422603986328	-0.0013427734375	\\
0.240466995161362	-0.001617431640625	\\
0.240511386336396	-0.0015869140625	\\
0.240555777511431	-0.001617431640625	\\
0.240600168686465	-0.00164794921875	\\
0.2406445598615	-0.001739501953125	\\
0.240688951036534	-0.002532958984375	\\
0.240733342211568	-0.00286865234375	\\
0.240777733386603	-0.0029296875	\\
0.240822124561637	-0.002960205078125	\\
0.240866515736672	-0.00262451171875	\\
0.240910906911706	-0.002685546875	\\
0.24095529808674	-0.0028076171875	\\
0.240999689261775	-0.00274658203125	\\
0.241044080436809	-0.00250244140625	\\
0.241088471611844	-0.002532958984375	\\
0.241132862786878	-0.002044677734375	\\
0.241177253961912	-0.001800537109375	\\
0.241221645136947	-0.001861572265625	\\
0.241266036311981	-0.002044677734375	\\
0.241310427487016	-0.00225830078125	\\
0.24135481866205	-0.001800537109375	\\
0.241399209837084	-0.00140380859375	\\
0.241443601012119	-0.001190185546875	\\
0.241487992187153	-0.001373291015625	\\
0.241532383362188	-0.00146484375	\\
0.241576774537222	-0.00115966796875	\\
0.241621165712256	-0.001068115234375	\\
0.241665556887291	-0.0013427734375	\\
0.241709948062325	-0.00103759765625	\\
0.24175433923736	-0.000640869140625	\\
0.241798730412394	-0.00067138671875	\\
0.241843121587428	-0.000640869140625	\\
0.241887512762463	-0.000701904296875	\\
0.241931903937497	-0.00054931640625	\\
0.241976295112532	-0.00067138671875	\\
0.242020686287566	-0.000518798828125	\\
0.2420650774626	-0.000579833984375	\\
0.242109468637635	-0.000946044921875	\\
0.242153859812669	-0.0013427734375	\\
0.242198250987704	-0.001556396484375	\\
0.242242642162738	-0.001129150390625	\\
0.242287033337772	-0.001007080078125	\\
0.242331424512807	-0.000518798828125	\\
0.242375815687841	-0.000823974609375	\\
0.242420206862876	-0.0006103515625	\\
0.24246459803791	-0.000518798828125	\\
0.242508989212944	-0.001190185546875	\\
0.242553380387979	-0.00091552734375	\\
0.242597771563013	-0.0006103515625	\\
0.242642162738048	-0.0010986328125	\\
0.242686553913082	-0.001251220703125	\\
0.242730945088116	-0.00079345703125	\\
0.242775336263151	-0.001007080078125	\\
0.242819727438185	-0.001251220703125	\\
0.24286411861322	-0.00091552734375	\\
0.242908509788254	-0.001068115234375	\\
0.242952900963289	-0.001220703125	\\
0.242997292138323	-0.00091552734375	\\
0.243041683313357	-0.001434326171875	\\
0.243086074488392	-0.001373291015625	\\
0.243130465663426	-0.00128173828125	\\
0.243174856838461	-0.001129150390625	\\
0.243219248013495	-0.001007080078125	\\
0.243263639188529	-0.001251220703125	\\
0.243308030363564	-0.001495361328125	\\
0.243352421538598	-0.0015869140625	\\
0.243396812713633	-0.00146484375	\\
0.243441203888667	-0.001373291015625	\\
0.243485595063701	-0.00115966796875	\\
0.243529986238736	-0.001434326171875	\\
0.24357437741377	-0.00164794921875	\\
0.243618768588805	-0.001190185546875	\\
0.243663159763839	-0.000946044921875	\\
0.243707550938873	-0.000946044921875	\\
0.243751942113908	-0.0013427734375	\\
0.243796333288942	-0.001373291015625	\\
0.243840724463977	-0.001434326171875	\\
0.243885115639011	-0.001861572265625	\\
0.243929506814045	-0.002166748046875	\\
0.24397389798908	-0.001922607421875	\\
0.244018289164114	-0.00140380859375	\\
0.244062680339149	-0.00128173828125	\\
0.244107071514183	-0.001129150390625	\\
0.244151462689217	-0.001434326171875	\\
0.244195853864252	-0.001617431640625	\\
0.244240245039286	-0.0015869140625	\\
0.244284636214321	-0.00164794921875	\\
0.244329027389355	-0.001434326171875	\\
0.244373418564389	-0.00164794921875	\\
0.244417809739424	-0.001953125	\\
0.244462200914458	-0.001922607421875	\\
0.244506592089493	-0.001708984375	\\
0.244550983264527	-0.001678466796875	\\
0.244595374439561	-0.001617431640625	\\
0.244639765614596	-0.001922607421875	\\
0.24468415678963	-0.001800537109375	\\
0.244728547964665	-0.001861572265625	\\
0.244772939139699	-0.00201416015625	\\
0.244817330314733	-0.00189208984375	\\
0.244861721489768	-0.001922607421875	\\
0.244906112664802	-0.00177001953125	\\
0.244950503839837	-0.001678466796875	\\
0.244994895014871	-0.0020751953125	\\
0.245039286189905	-0.001922607421875	\\
0.24508367736494	-0.00177001953125	\\
0.245128068539974	-0.001678466796875	\\
0.245172459715009	-0.00115966796875	\\
0.245216850890043	-0.00091552734375	\\
0.245261242065077	-0.001007080078125	\\
0.245305633240112	-0.000762939453125	\\
0.245350024415146	-0.000701904296875	\\
0.245394415590181	-0.00042724609375	\\
0.245438806765215	-0.000152587890625	\\
0.245483197940249	-6.103515625e-05	\\
0.245527589115284	0.000518798828125	\\
0.245571980290318	0.000762939453125	\\
0.245616371465353	0.00079345703125	\\
0.245660762640387	0.001068115234375	\\
0.245705153815421	0.001068115234375	\\
0.245749544990456	0.00128173828125	\\
0.24579393616549	0.00189208984375	\\
0.245838327340525	0.00164794921875	\\
0.245882718515559	0.001739501953125	\\
0.245927109690594	0.0018310546875	\\
0.245971500865628	0.00177001953125	\\
0.246015892040662	0.002532958984375	\\
0.246060283215697	0.0025634765625	\\
0.246104674390731	0.002105712890625	\\
0.246149065565766	0.001983642578125	\\
0.2461934567408	0.001800537109375	\\
0.246237847915834	0.001708984375	\\
0.246282239090869	0.002105712890625	\\
0.246326630265903	0.00201416015625	\\
0.246371021440938	0.001495361328125	\\
0.246415412615972	0.001495361328125	\\
0.246459803791006	0.00152587890625	\\
0.246504194966041	0.00115966796875	\\
0.246548586141075	0.000823974609375	\\
0.24659297731611	0.0009765625	\\
0.246637368491144	0.000701904296875	\\
0.246681759666178	0.000518798828125	\\
0.246726150841213	0.000518798828125	\\
0.246770542016247	3.0517578125e-05	\\
0.246814933191282	-0.00018310546875	\\
0.246859324366316	0	\\
0.24690371554135	-0.000152587890625	\\
0.246948106716385	-0.00018310546875	\\
0.246992497891419	-0.000396728515625	\\
0.247036889066454	-0.000579833984375	\\
0.247081280241488	-0.00048828125	\\
0.247125671416522	-0.000579833984375	\\
0.247170062591557	-0.000579833984375	\\
0.247214453766591	-0.00067138671875	\\
0.247258844941626	-0.001220703125	\\
0.24730323611666	-0.00115966796875	\\
0.247347627291694	-0.001129150390625	\\
0.247392018466729	-0.00152587890625	\\
0.247436409641763	-0.0020751953125	\\
0.247480800816798	-0.002044677734375	\\
0.247525191991832	-0.001617431640625	\\
0.247569583166866	-0.00152587890625	\\
0.247613974341901	-0.001739501953125	\\
0.247658365516935	-0.00189208984375	\\
0.24770275669197	-0.001953125	\\
0.247747147867004	-0.001556396484375	\\
0.247791539042038	-0.00164794921875	\\
0.247835930217073	-0.00213623046875	\\
0.247880321392107	-0.002349853515625	\\
0.247924712567142	-0.00225830078125	\\
0.247969103742176	-0.00250244140625	\\
0.24801349491721	-0.00250244140625	\\
0.248057886092245	-0.001739501953125	\\
0.248102277267279	-0.001922607421875	\\
0.248146668442314	-0.002349853515625	\\
0.248191059617348	-0.001617431640625	\\
0.248235450792382	-0.001556396484375	\\
0.248279841967417	-0.001922607421875	\\
0.248324233142451	-0.00177001953125	\\
0.248368624317486	-0.002044677734375	\\
0.24841301549252	-0.0023193359375	\\
0.248457406667555	-0.00201416015625	\\
0.248501797842589	-0.00213623046875	\\
0.248546189017623	-0.002197265625	\\
0.248590580192658	-0.0018310546875	\\
0.248634971367692	-0.0018310546875	\\
0.248679362542727	-0.001129150390625	\\
0.248723753717761	-0.001068115234375	\\
0.248768144892795	-0.001373291015625	\\
0.24881253606783	-0.00152587890625	\\
0.248856927242864	-0.0009765625	\\
0.248901318417899	-0.00079345703125	\\
0.248945709592933	-0.0008544921875	\\
0.248990100767967	-0.00079345703125	\\
0.249034491943002	-0.000640869140625	\\
0.249078883118036	-0.000732421875	\\
0.249123274293071	-0.000823974609375	\\
0.249167665468105	-0.0006103515625	\\
0.249212056643139	-0.000579833984375	\\
0.249256447818174	0	\\
0.249300838993208	0.000213623046875	\\
0.249345230168243	-3.0517578125e-05	\\
0.249389621343277	-0.00018310546875	\\
0.249434012518311	0.0001220703125	\\
0.249478403693346	0.0001220703125	\\
0.24952279486838	0	\\
0.249567186043415	0.0001220703125	\\
0.249611577218449	0.000457763671875	\\
0.249655968393483	0.000335693359375	\\
0.249700359568518	0.000579833984375	\\
0.249744750743552	0.000579833984375	\\
0.249789141918587	0	\\
0.249833533093621	0.000152587890625	\\
0.249877924268655	0.0001220703125	\\
0.24992231544369	0.000244140625	\\
0.249966706618724	0.000274658203125	\\
0.250011097793759	0.0001220703125	\\
0.250055488968793	-6.103515625e-05	\\
0.250099880143827	-0.0001220703125	\\
0.250144271318862	0.00018310546875	\\
0.250188662493896	0.000274658203125	\\
0.250233053668931	9.1552734375e-05	\\
0.250277444843965	0.000396728515625	\\
0.250321836018999	-3.0517578125e-05	\\
0.250366227194034	3.0517578125e-05	\\
0.250410618369068	-0.00048828125	\\
0.250455009544103	-0.000457763671875	\\
0.250499400719137	-6.103515625e-05	\\
0.250543791894171	-6.103515625e-05	\\
0.250588183069206	3.0517578125e-05	\\
0.25063257424424	-0.000335693359375	\\
0.250676965419275	-6.103515625e-05	\\
0.250721356594309	0.000213623046875	\\
0.250765747769343	0.000396728515625	\\
0.250810138944378	0.00054931640625	\\
0.250854530119412	0.000732421875	\\
0.250898921294447	0.000457763671875	\\
0.250943312469481	0.00054931640625	\\
0.250987703644515	0.000457763671875	\\
0.25103209481955	0.000335693359375	\\
0.251076485994584	0.000518798828125	\\
0.251120877169619	0.000457763671875	\\
0.251165268344653	0.000762939453125	\\
0.251209659519687	0.000885009765625	\\
0.251254050694722	0.000701904296875	\\
0.251298441869756	0.00054931640625	\\
0.251342833044791	0.00054931640625	\\
0.251387224219825	0.000579833984375	\\
0.251431615394859	0.00042724609375	\\
0.251476006569894	0.000640869140625	\\
0.251520397744928	0.00079345703125	\\
0.251564788919963	0.000732421875	\\
0.251609180094997	0.00048828125	\\
0.251653571270032	0.000457763671875	\\
0.251697962445066	0.000457763671875	\\
0.2517423536201	0.00042724609375	\\
0.251786744795135	0.000152587890625	\\
0.251831135970169	0.00018310546875	\\
0.251875527145204	0.0003662109375	\\
0.251919918320238	0.000213623046875	\\
0.251964309495272	3.0517578125e-05	\\
0.252008700670307	3.0517578125e-05	\\
0.252053091845341	0	\\
0.252097483020376	-0.000640869140625	\\
0.25214187419541	-0.000335693359375	\\
0.252186265370444	6.103515625e-05	\\
0.252230656545479	-0.0003662109375	\\
0.252275047720513	-0.00018310546875	\\
0.252319438895548	0.0001220703125	\\
0.252363830070582	0.000640869140625	\\
0.252408221245616	0.000396728515625	\\
0.252452612420651	0.0003662109375	\\
0.252497003595685	0.000152587890625	\\
0.25254139477072	0.000213623046875	\\
0.252585785945754	0.00042724609375	\\
0.252630177120788	0.00042724609375	\\
0.252674568295823	0.00042724609375	\\
0.252718959470857	0.000732421875	\\
0.252763350645892	0.00115966796875	\\
0.252807741820926	0.001373291015625	\\
0.25285213299596	0.001556396484375	\\
0.252896524170995	0.001739501953125	\\
0.252940915346029	0.0015869140625	\\
0.252985306521064	0.0020751953125	\\
0.253029697696098	0.00225830078125	\\
0.253074088871132	0.002197265625	\\
0.253118480046167	0.002166748046875	\\
0.253162871221201	0.0020751953125	\\
0.253207262396236	0.0020751953125	\\
0.25325165357127	0.002899169921875	\\
0.253296044746304	0.003173828125	\\
0.253340435921339	0.00311279296875	\\
0.253384827096373	0.003204345703125	\\
0.253429218271408	0.0029296875	\\
0.253473609446442	0.003143310546875	\\
0.253518000621476	0.00347900390625	\\
0.253562391796511	0.00372314453125	\\
0.253606782971545	0.003692626953125	\\
0.25365117414658	0.00347900390625	\\
0.253695565321614	0.003936767578125	\\
0.253739956496648	0.003570556640625	\\
0.253784347671683	0.003265380859375	\\
0.253828738846717	0.003662109375	\\
0.253873130021752	0.0030517578125	\\
0.253917521196786	0.003173828125	\\
0.25396191237182	0.003662109375	\\
0.254006303546855	0.003509521484375	\\
0.254050694721889	0.003662109375	\\
0.254095085896924	0.004180908203125	\\
0.254139477071958	0.004364013671875	\\
0.254183868246993	0.003936767578125	\\
0.254228259422027	0.00396728515625	\\
0.254272650597061	0.004364013671875	\\
0.254317041772096	0.00433349609375	\\
0.25436143294713	0.004150390625	\\
0.254405824122165	0.003814697265625	\\
0.254450215297199	0.003875732421875	\\
0.254494606472233	0.004058837890625	\\
0.254538997647268	0.0037841796875	\\
0.254583388822302	0.00384521484375	\\
0.254627779997337	0.00390625	\\
0.254672171172371	0.003509521484375	\\
0.254716562347405	0.00341796875	\\
0.25476095352244	0.00335693359375	\\
0.254805344697474	0.00323486328125	\\
0.254849735872509	0.0030517578125	\\
0.254894127047543	0.0029296875	\\
0.254938518222577	0.002960205078125	\\
0.254982909397612	0.002685546875	\\
0.255027300572646	0.002105712890625	\\
0.255071691747681	0.002044677734375	\\
0.255116082922715	0.002288818359375	\\
0.255160474097749	0.002105712890625	\\
0.255204865272784	0.001312255859375	\\
0.255249256447818	0.00115966796875	\\
0.255293647622853	0.00164794921875	\\
0.255338038797887	0.00164794921875	\\
0.255382429972921	0.00152587890625	\\
0.255426821147956	0.00146484375	\\
0.25547121232299	0.001068115234375	\\
0.255515603498025	0.00164794921875	\\
0.255559994673059	0.001861572265625	\\
0.255604385848093	0.0010986328125	\\
0.255648777023128	0.000823974609375	\\
0.255693168198162	0.001251220703125	\\
0.255737559373197	0.001220703125	\\
0.255781950548231	0.001190185546875	\\
0.255826341723265	0.00128173828125	\\
0.2558707328983	0.000885009765625	\\
0.255915124073334	0.00103759765625	\\
0.255959515248369	0.00115966796875	\\
0.256003906423403	0.000762939453125	\\
0.256048297598437	0.000885009765625	\\
0.256092688773472	0.000946044921875	\\
0.256137079948506	0.0006103515625	\\
0.256181471123541	0.0006103515625	\\
0.256225862298575	0.000274658203125	\\
0.256270253473609	0.000274658203125	\\
0.256314644648644	0.000457763671875	\\
0.256359035823678	0	\\
0.256403426998713	-0.000152587890625	\\
0.256447818173747	-0.000274658203125	\\
0.256492209348781	0.00018310546875	\\
0.256536600523816	0.000244140625	\\
0.25658099169885	6.103515625e-05	\\
0.256625382873885	6.103515625e-05	\\
0.256669774048919	0.000213623046875	\\
0.256714165223953	0.00067138671875	\\
0.256758556398988	0.0003662109375	\\
0.256802947574022	0.00042724609375	\\
0.256847338749057	0.000732421875	\\
0.256891729924091	0.0006103515625	\\
0.256936121099126	0.000701904296875	\\
0.25698051227416	0.00103759765625	\\
0.257024903449194	0.0018310546875	\\
0.257069294624229	0.002410888671875	\\
0.257113685799263	0.00213623046875	\\
0.257158076974297	0.00225830078125	\\
0.257202468149332	0.00213623046875	\\
0.257246859324366	0.0015869140625	\\
0.257291250499401	0.001861572265625	\\
0.257335641674435	0.00225830078125	\\
0.25738003284947	0.001922607421875	\\
0.257424424024504	0.00189208984375	\\
0.257468815199538	0.001983642578125	\\
0.257513206374573	0.001953125	\\
0.257557597549607	0.00201416015625	\\
0.257601988724642	0.00201416015625	\\
0.257646379899676	0.002227783203125	\\
0.25769077107471	0.001983642578125	\\
0.257735162249745	0.001953125	\\
0.257779553424779	0.002105712890625	\\
0.257823944599814	0.001983642578125	\\
0.257868335774848	0.00201416015625	\\
0.257912726949882	0.00177001953125	\\
0.257957118124917	0.001922607421875	\\
0.258001509299951	0.001861572265625	\\
0.258045900474986	0.00115966796875	\\
0.25809029165002	0.000946044921875	\\
0.258134682825054	0.001068115234375	\\
0.258179074000089	0.001007080078125	\\
0.258223465175123	0.001007080078125	\\
0.258267856350158	0.000823974609375	\\
0.258312247525192	0.000732421875	\\
0.258356638700226	0.000762939453125	\\
0.258401029875261	0.00079345703125	\\
0.258445421050295	0.0003662109375	\\
0.25848981222533	0.000640869140625	\\
0.258534203400364	0.00048828125	\\
0.258578594575398	0.00018310546875	\\
0.258622985750433	0.00030517578125	\\
0.258667376925467	0.000152587890625	\\
0.258711768100502	0.0001220703125	\\
0.258756159275536	0.000152587890625	\\
0.25880055045057	6.103515625e-05	\\
0.258844941625605	-0.000152587890625	\\
0.258889332800639	-6.103515625e-05	\\
0.258933723975674	-0.000396728515625	\\
0.258978115150708	-0.00054931640625	\\
0.259022506325742	-0.000457763671875	\\
0.259066897500777	-0.00048828125	\\
0.259111288675811	-0.00042724609375	\\
0.259155679850846	-0.0008544921875	\\
0.25920007102588	-0.00115966796875	\\
0.259244462200914	-0.00079345703125	\\
0.259288853375949	-0.00030517578125	\\
0.259333244550983	-0.00067138671875	\\
0.259377635726018	-0.000579833984375	\\
0.259422026901052	-6.103515625e-05	\\
0.259466418076086	-6.103515625e-05	\\
0.259510809251121	0.000152587890625	\\
0.259555200426155	0.0006103515625	\\
0.25959959160119	0.000518798828125	\\
0.259643982776224	0.0001220703125	\\
0.259688373951258	0.0001220703125	\\
0.259732765126293	0.000518798828125	\\
0.259777156301327	0.000579833984375	\\
0.259821547476362	0.000518798828125	\\
0.259865938651396	0.000640869140625	\\
0.259910329826431	0.00054931640625	\\
0.259954721001465	0.00079345703125	\\
0.259999112176499	0.001251220703125	\\
0.260043503351534	0.001373291015625	\\
0.260087894526568	0.001220703125	\\
0.260132285701603	0.00140380859375	\\
0.260176676876637	0.0013427734375	\\
0.260221068051671	0.0010986328125	\\
0.260265459226706	0.001495361328125	\\
0.26030985040174	0.001373291015625	\\
0.260354241576775	0.001251220703125	\\
0.260398632751809	0.00128173828125	\\
0.260443023926843	0.001312255859375	\\
0.260487415101878	0.001556396484375	\\
0.260531806276912	0.001800537109375	\\
0.260576197451947	0.001861572265625	\\
0.260620588626981	0.001739501953125	\\
0.260664979802015	0.001953125	\\
0.26070937097705	0.00213623046875	\\
0.260753762152084	0.00213623046875	\\
0.260798153327119	0.002197265625	\\
0.260842544502153	0.002105712890625	\\
0.260886935677187	0.002197265625	\\
0.260931326852222	0.002197265625	\\
0.260975718027256	0.001678466796875	\\
0.261020109202291	0.00189208984375	\\
0.261064500377325	0.002197265625	\\
0.261108891552359	0.001953125	\\
0.261153282727394	0.001861572265625	\\
0.261197673902428	0.001983642578125	\\
0.261242065077463	0.002410888671875	\\
0.261286456252497	0.0028076171875	\\
0.261330847427531	0.002685546875	\\
0.261375238602566	0.00262451171875	\\
0.2614196297776	0.002838134765625	\\
0.261464020952635	0.002960205078125	\\
0.261508412127669	0.002960205078125	\\
0.261552803302703	0.00323486328125	\\
0.261597194477738	0.003021240234375	\\
0.261641585652772	0.0029296875	\\
0.261685976827807	0.00341796875	\\
0.261730368002841	0.00408935546875	\\
0.261774759177875	0.00390625	\\
0.26181915035291	0.00341796875	\\
0.261863541527944	0.003570556640625	\\
0.261907932702979	0.00372314453125	\\
0.261952323878013	0.00341796875	\\
0.261996715053047	0.00347900390625	\\
0.262041106228082	0.00372314453125	\\
0.262085497403116	0.003570556640625	\\
0.262129888578151	0.003326416015625	\\
0.262174279753185	0.003326416015625	\\
0.262218670928219	0.003509521484375	\\
0.262263062103254	0.003265380859375	\\
0.262307453278288	0.002655029296875	\\
0.262351844453323	0.00213623046875	\\
0.262396235628357	0.002166748046875	\\
0.262440626803391	0.002227783203125	\\
0.262485017978426	0.00201416015625	\\
0.26252940915346	0.001556396484375	\\
0.262573800328495	0.001739501953125	\\
0.262618191503529	0.0018310546875	\\
0.262662582678564	0.0010986328125	\\
0.262706973853598	0.000946044921875	\\
0.262751365028632	0.0009765625	\\
0.262795756203667	0.000244140625	\\
0.262840147378701	0.000335693359375	\\
0.262884538553736	0.00048828125	\\
0.26292892972877	0.0001220703125	\\
0.262973320903804	0.000579833984375	\\
0.263017712078839	0.0008544921875	\\
0.263062103253873	0.000518798828125	\\
0.263106494428908	0.0003662109375	\\
0.263150885603942	0.00030517578125	\\
0.263195276778976	0.000823974609375	\\
0.263239667954011	0.0009765625	\\
0.263284059129045	0.00103759765625	\\
0.26332845030408	0.00146484375	\\
0.263372841479114	0.002227783203125	\\
0.263417232654148	0.00213623046875	\\
0.263461623829183	0.001739501953125	\\
0.263506015004217	0.00201416015625	\\
0.263550406179252	0.002166748046875	\\
0.263594797354286	0.00213623046875	\\
0.26363918852932	0.00213623046875	\\
0.263683579704355	0.002166748046875	\\
0.263727970879389	0.002288818359375	\\
0.263772362054424	0.0023193359375	\\
0.263816753229458	0.00189208984375	\\
0.263861144404492	0.00213623046875	\\
0.263905535579527	0.00201416015625	\\
0.263949926754561	0.00128173828125	\\
0.263994317929596	0.000946044921875	\\
0.26403870910463	0.001251220703125	\\
0.264083100279664	0.00079345703125	\\
0.264127491454699	0.00030517578125	\\
0.264171882629733	3.0517578125e-05	\\
0.264216273804768	-0.00030517578125	\\
0.264260664979802	3.0517578125e-05	\\
0.264305056154836	-0.00030517578125	\\
0.264349447329871	-0.00079345703125	\\
0.264393838504905	-0.0009765625	\\
0.26443822967994	-0.001007080078125	\\
0.264482620854974	-0.001129150390625	\\
0.264527012030008	-0.001129150390625	\\
0.264571403205043	-0.0013427734375	\\
0.264615794380077	-0.00164794921875	\\
0.264660185555112	-0.000946044921875	\\
0.264704576730146	-0.00128173828125	\\
0.26474896790518	-0.001373291015625	\\
0.264793359080215	-0.0010986328125	\\
0.264837750255249	-0.001068115234375	\\
0.264882141430284	-0.000823974609375	\\
0.264926532605318	-0.0008544921875	\\
0.264970923780352	-0.0003662109375	\\
0.265015314955387	0.0001220703125	\\
0.265059706130421	6.103515625e-05	\\
0.265104097305456	0.000457763671875	\\
0.26514848848049	0.000762939453125	\\
0.265192879655524	0.000396728515625	\\
0.265237270830559	0.00048828125	\\
0.265281662005593	0.000579833984375	\\
0.265326053180628	0.0003662109375	\\
0.265370444355662	0.000335693359375	\\
0.265414835530697	0.000732421875	\\
0.265459226705731	0.0008544921875	\\
0.265503617880765	0.000885009765625	\\
0.2655480090558	0.001251220703125	\\
0.265592400230834	0.001373291015625	\\
0.265636791405868	0.0015869140625	\\
0.265681182580903	0.00115966796875	\\
0.265725573755937	0.001190185546875	\\
0.265769964930972	0.001678466796875	\\
0.265814356106006	0.002044677734375	\\
0.265858747281041	0.002197265625	\\
0.265903138456075	0.002105712890625	\\
0.265947529631109	0.00213623046875	\\
0.265991920806144	0.00177001953125	\\
0.266036311981178	0.001434326171875	\\
0.266080703156213	0.001739501953125	\\
0.266125094331247	0.00128173828125	\\
0.266169485506281	0.0009765625	\\
0.266213876681316	0.001190185546875	\\
0.26625826785635	0.000885009765625	\\
0.266302659031385	0.00079345703125	\\
0.266347050206419	0.000640869140625	\\
0.266391441381453	0.00030517578125	\\
0.266435832556488	0.00030517578125	\\
0.266480223731522	0.000244140625	\\
0.266524614906557	0.00030517578125	\\
0.266569006081591	0.0001220703125	\\
0.266613397256625	-0.000274658203125	\\
0.26665778843166	0.000152587890625	\\
0.266702179606694	-0.000152587890625	\\
0.266746570781729	-0.000457763671875	\\
0.266790961956763	-0.000152587890625	\\
0.266835353131797	-0.000579833984375	\\
0.266879744306832	-3.0517578125e-05	\\
0.266924135481866	-0.000152587890625	\\
0.266968526656901	-0.0003662109375	\\
0.267012917831935	9.1552734375e-05	\\
0.267057309006969	-6.103515625e-05	\\
0.267101700182004	0.000152587890625	\\
0.267146091357038	9.1552734375e-05	\\
0.267190482532073	0.0003662109375	\\
0.267234873707107	0.00054931640625	\\
0.267279264882141	0.000335693359375	\\
0.267323656057176	0.0006103515625	\\
0.26736804723221	0.00103759765625	\\
0.267412438407245	0.0010986328125	\\
0.267456829582279	0.0010986328125	\\
0.267501220757313	0.00140380859375	\\
0.267545611932348	0.001312255859375	\\
0.267590003107382	0.001129150390625	\\
0.267634394282417	0.0009765625	\\
0.267678785457451	0.00128173828125	\\
0.267723176632485	0.001129150390625	\\
0.26776756780752	0.001007080078125	\\
0.267811958982554	0.000823974609375	\\
0.267856350157589	0.000762939453125	\\
0.267900741332623	0.00079345703125	\\
0.267945132507658	0.0006103515625	\\
0.267989523682692	0.000457763671875	\\
0.268033914857726	0.000457763671875	\\
0.268078306032761	0.00018310546875	\\
0.268122697207795	-3.0517578125e-05	\\
0.268167088382829	6.103515625e-05	\\
0.268211479557864	0.000152587890625	\\
0.268255870732898	0.00054931640625	\\
0.268300261907933	0.00079345703125	\\
0.268344653082967	0.001190185546875	\\
0.268389044258002	0.00140380859375	\\
0.268433435433036	0.000823974609375	\\
0.26847782660807	0.001190185546875	\\
0.268522217783105	0.001495361328125	\\
0.268566608958139	0.00103759765625	\\
0.268611000133174	0.000885009765625	\\
0.268655391308208	0.000823974609375	\\
0.268699782483242	0.00115966796875	\\
0.268744173658277	0.00152587890625	\\
0.268788564833311	0.001434326171875	\\
0.268832956008346	0.0013427734375	\\
0.26887734718338	0.00146484375	\\
0.268921738358414	0.00164794921875	\\
0.268966129533449	0.00146484375	\\
0.269010520708483	0.001617431640625	\\
0.269054911883518	0.001556396484375	\\
0.269099303058552	0.00164794921875	\\
0.269143694233586	0.001617431640625	\\
0.269188085408621	0.001861572265625	\\
0.269232476583655	0.002227783203125	\\
0.26927686775869	0.00213623046875	\\
0.269321258933724	0.001953125	\\
0.269365650108758	0.001617431640625	\\
0.269410041283793	0.001434326171875	\\
0.269454432458827	0.001495361328125	\\
0.269498823633862	0.00128173828125	\\
0.269543214808896	0.001190185546875	\\
0.26958760598393	0.0015869140625	\\
0.269631997158965	0.0010986328125	\\
0.269676388333999	0.00054931640625	\\
0.269720779509034	0.000457763671875	\\
0.269765170684068	0.000579833984375	\\
0.269809561859102	0.00054931640625	\\
0.269853953034137	0.000274658203125	\\
0.269898344209171	0.0001220703125	\\
0.269942735384206	0.000518798828125	\\
0.26998712655924	0.00042724609375	\\
0.270031517734274	0.00042724609375	\\
0.270075908909309	0.000244140625	\\
0.270120300084343	-6.103515625e-05	\\
0.270164691259378	9.1552734375e-05	\\
0.270209082434412	0.000244140625	\\
0.270253473609446	9.1552734375e-05	\\
0.270297864784481	0.00042724609375	\\
0.270342255959515	0.0006103515625	\\
0.27038664713455	0.00042724609375	\\
0.270431038309584	0.00048828125	\\
0.270475429484618	0.000396728515625	\\
0.270519820659653	0.000457763671875	\\
0.270564211834687	0.000701904296875	\\
0.270608603009722	0.0009765625	\\
0.270652994184756	0.00115966796875	\\
0.27069738535979	0.000946044921875	\\
0.270741776534825	0.0008544921875	\\
0.270786167709859	0.001190185546875	\\
0.270830558884894	0.001251220703125	\\
0.270874950059928	0.001068115234375	\\
0.270919341234962	0.001220703125	\\
0.270963732409997	0.001739501953125	\\
0.271008123585031	0.0018310546875	\\
0.271052514760066	0.00164794921875	\\
0.2710969059351	0.002044677734375	\\
0.271141297110135	0.00244140625	\\
0.271185688285169	0.00274658203125	\\
0.271230079460203	0.003326416015625	\\
0.271274470635238	0.00341796875	\\
0.271318861810272	0.002655029296875	\\
0.271363252985307	0.00262451171875	\\
0.271407644160341	0.002716064453125	\\
0.271452035335375	0.002593994140625	\\
0.27149642651041	0.002655029296875	\\
0.271540817685444	0.00250244140625	\\
0.271585208860479	0.00244140625	\\
0.271629600035513	0.00177001953125	\\
0.271673991210547	0.001495361328125	\\
0.271718382385582	0.0013427734375	\\
0.271762773560616	0.000640869140625	\\
0.271807164735651	0.00067138671875	\\
0.271851555910685	0.000274658203125	\\
0.271895947085719	-0.000152587890625	\\
0.271940338260754	-0.000396728515625	\\
0.271984729435788	-0.0008544921875	\\
0.272029120610823	-0.000457763671875	\\
0.272073511785857	3.0517578125e-05	\\
0.272117902960891	-6.103515625e-05	\\
0.272162294135926	9.1552734375e-05	\\
0.27220668531096	0.000213623046875	\\
0.272251076485995	0.00018310546875	\\
0.272295467661029	0.000579833984375	\\
0.272339858836063	0.000640869140625	\\
0.272384250011098	0.000457763671875	\\
0.272428641186132	0.000518798828125	\\
0.272473032361167	0.0008544921875	\\
0.272517423536201	0.000946044921875	\\
0.272561814711235	0.00079345703125	\\
0.27260620588627	0.001129150390625	\\
0.272650597061304	0.00103759765625	\\
0.272694988236339	0.001068115234375	\\
0.272739379411373	0.001220703125	\\
0.272783770586407	0.000946044921875	\\
0.272828161761442	0.001373291015625	\\
0.272872552936476	0.001434326171875	\\
0.272916944111511	0.001190185546875	\\
0.272961335286545	0.001220703125	\\
0.273005726461579	0.001190185546875	\\
0.273050117636614	0.001251220703125	\\
0.273094508811648	0.001708984375	\\
0.273138899986683	0.00152587890625	\\
0.273183291161717	0.001190185546875	\\
0.273227682336751	0.0010986328125	\\
0.273272073511786	0.001495361328125	\\
0.27331646468682	0.00152587890625	\\
0.273360855861855	0.00152587890625	\\
0.273405247036889	0.00164794921875	\\
0.273449638211923	0.00146484375	\\
0.273494029386958	0.001068115234375	\\
0.273538420561992	0.000762939453125	\\
0.273582811737027	0.000762939453125	\\
0.273627202912061	0.00079345703125	\\
0.273671594087095	0.000579833984375	\\
0.27371598526213	0.0003662109375	\\
0.273760376437164	0.000274658203125	\\
0.273804767612199	0.00030517578125	\\
0.273849158787233	0.0003662109375	\\
0.273893549962268	0.00042724609375	\\
0.273937941137302	0.000244140625	\\
0.273982332312336	-9.1552734375e-05	\\
0.274026723487371	-9.1552734375e-05	\\
0.274071114662405	0	\\
0.27411550583744	9.1552734375e-05	\\
0.274159897012474	-0.000152587890625	\\
0.274204288187508	-6.103515625e-05	\\
0.274248679362543	-0.00030517578125	\\
0.274293070537577	-0.000640869140625	\\
0.274337461712612	-0.000885009765625	\\
0.274381852887646	-0.000823974609375	\\
0.27442624406268	-0.000701904296875	\\
0.274470635237715	-0.000885009765625	\\
0.274515026412749	-0.000701904296875	\\
0.274559417587784	-0.000518798828125	\\
0.274603808762818	-0.0006103515625	\\
0.274648199937852	-0.000244140625	\\
0.274692591112887	-3.0517578125e-05	\\
0.274736982287921	-0.000244140625	\\
0.274781373462956	9.1552734375e-05	\\
0.27482576463799	-6.103515625e-05	\\
0.274870155813024	0.0001220703125	\\
0.274914546988059	0.00048828125	\\
0.274958938163093	0.00067138671875	\\
0.275003329338128	0.00054931640625	\\
0.275047720513162	0.000457763671875	\\
0.275092111688196	0.000762939453125	\\
0.275136502863231	0.000946044921875	\\
0.275180894038265	0.000396728515625	\\
0.2752252852133	0.000274658203125	\\
0.275269676388334	9.1552734375e-05	\\
0.275314067563368	-0.00030517578125	\\
0.275358458738403	-0.0001220703125	\\
0.275402849913437	-6.103515625e-05	\\
0.275447241088472	-0.0001220703125	\\
0.275491632263506	-0.00042724609375	\\
0.27553602343854	-0.00042724609375	\\
0.275580414613575	-9.1552734375e-05	\\
0.275624805788609	0.000335693359375	\\
0.275669196963644	0.000152587890625	\\
0.275713588138678	0.00018310546875	\\
0.275757979313712	0.000244140625	\\
0.275802370488747	0.00018310546875	\\
0.275846761663781	0.000396728515625	\\
0.275891152838816	0.000518798828125	\\
0.27593554401385	0.00054931640625	\\
0.275979935188884	0.000244140625	\\
0.276024326363919	0.00018310546875	\\
0.276068717538953	-0.000152587890625	\\
0.276113108713988	6.103515625e-05	\\
0.276157499889022	0.000396728515625	\\
0.276201891064056	0.000335693359375	\\
0.276246282239091	0.00030517578125	\\
0.276290673414125	0.000579833984375	\\
0.27633506458916	0.000701904296875	\\
0.276379455764194	0.000213623046875	\\
0.276423846939229	0.000152587890625	\\
0.276468238114263	0.000457763671875	\\
0.276512629289297	0.00048828125	\\
0.276557020464332	0.00042724609375	\\
0.276601411639366	0.00054931640625	\\
0.2766458028144	0.00103759765625	\\
0.276690193989435	0.000640869140625	\\
0.276734585164469	0.000946044921875	\\
0.276778976339504	0.001190185546875	\\
0.276823367514538	0.0009765625	\\
0.276867758689573	0.001129150390625	\\
0.276912149864607	0.001007080078125	\\
0.276956541039641	0.001129150390625	\\
0.277000932214676	0.001068115234375	\\
0.27704532338971	0.000946044921875	\\
0.277089714564745	0.001220703125	\\
0.277134105739779	0.00103759765625	\\
0.277178496914813	0.00042724609375	\\
0.277222888089848	0.000274658203125	\\
0.277267279264882	0.000335693359375	\\
0.277311670439917	3.0517578125e-05	\\
0.277356061614951	-0.000152587890625	\\
0.277400452789985	0.0001220703125	\\
0.27744484396502	0.000213623046875	\\
0.277489235140054	-9.1552734375e-05	\\
0.277533626315089	-0.0001220703125	\\
0.277578017490123	0	\\
0.277622408665157	-3.0517578125e-05	\\
0.277666799840192	-9.1552734375e-05	\\
0.277711191015226	-0.000244140625	\\
0.277755582190261	-0.00042724609375	\\
0.277799973365295	-0.0003662109375	\\
0.277844364540329	-0.000579833984375	\\
0.277888755715364	-0.0008544921875	\\
0.277933146890398	-0.0006103515625	\\
0.277977538065433	-0.000518798828125	\\
0.278021929240467	-0.001251220703125	\\
0.278066320415501	-0.00152587890625	\\
0.278110711590536	-0.00115966796875	\\
0.27815510276557	-0.001434326171875	\\
0.278199493940605	-0.001220703125	\\
0.278243885115639	-0.001007080078125	\\
0.278288276290673	-0.0008544921875	\\
0.278332667465708	-0.0008544921875	\\
0.278377058640742	-0.001373291015625	\\
0.278421449815777	-0.001007080078125	\\
0.278465840990811	-0.000518798828125	\\
0.278510232165845	-0.00067138671875	\\
0.27855462334088	-0.000762939453125	\\
0.278599014515914	-0.000335693359375	\\
0.278643405690949	-0.000457763671875	\\
0.278687796865983	-0.000274658203125	\\
0.278732188041017	0.000335693359375	\\
0.278776579216052	0.000396728515625	\\
0.278820970391086	0.000396728515625	\\
0.278865361566121	0.000701904296875	\\
0.278909752741155	0.0008544921875	\\
0.278954143916189	0.00140380859375	\\
0.278998535091224	0.001617431640625	\\
0.279042926266258	0.00140380859375	\\
0.279087317441293	0.0013427734375	\\
0.279131708616327	0.0013427734375	\\
0.279176099791361	0.0018310546875	\\
0.279220490966396	0.00146484375	\\
0.27926488214143	0.000701904296875	\\
0.279309273316465	0.001007080078125	\\
0.279353664491499	0.00091552734375	\\
0.279398055666533	0.000732421875	\\
0.279442446841568	0.00079345703125	\\
0.279486838016602	0.000396728515625	\\
0.279531229191637	3.0517578125e-05	\\
0.279575620366671	0.0001220703125	\\
0.279620011541706	-9.1552734375e-05	\\
0.27966440271674	-3.0517578125e-05	\\
0.279708793891774	3.0517578125e-05	\\
0.279753185066809	-0.000396728515625	\\
0.279797576241843	-0.00042724609375	\\
0.279841967416878	-0.00018310546875	\\
0.279886358591912	-0.000244140625	\\
0.279930749766946	0.0001220703125	\\
0.279975140941981	0.00030517578125	\\
0.280019532117015	0.00079345703125	\\
0.28006392329205	0.000823974609375	\\
0.280108314467084	-0.000244140625	\\
0.280152705642118	0.0006103515625	\\
0.280197096817153	0.001739501953125	\\
0.280241487992187	0.0010986328125	\\
0.280285879167222	0.0013427734375	\\
0.280330270342256	0.00164794921875	\\
0.28037466151729	0.001922607421875	\\
0.280419052692325	0.0025634765625	\\
0.280463443867359	0.002105712890625	\\
0.280507835042394	0.001953125	\\
0.280552226217428	0.002044677734375	\\
0.280596617392462	0.001983642578125	\\
0.280641008567497	0.001861572265625	\\
0.280685399742531	0.00189208984375	\\
0.280729790917566	0.001800537109375	\\
0.2807741820926	0.00164794921875	\\
0.280818573267634	0.001739501953125	\\
0.280862964442669	0.001678466796875	\\
0.280907355617703	0.00189208984375	\\
0.280951746792738	0.001953125	\\
0.280996137967772	0.001556396484375	\\
0.281040529142806	0.00152587890625	\\
0.281084920317841	0.0010986328125	\\
0.281129311492875	0.000885009765625	\\
0.28117370266791	0.00048828125	\\
0.281218093842944	0.000518798828125	\\
0.281262485017978	0.000457763671875	\\
0.281306876193013	-0.000213623046875	\\
0.281351267368047	-9.1552734375e-05	\\
0.281395658543082	-0.0001220703125	\\
0.281440049718116	-0.000518798828125	\\
0.28148444089315	-0.000518798828125	\\
0.281528832068185	-0.0006103515625	\\
0.281573223243219	-0.000640869140625	\\
0.281617614418254	-0.000640869140625	\\
0.281662005593288	-0.0008544921875	\\
0.281706396768322	-0.000640869140625	\\
0.281750787943357	-0.00091552734375	\\
0.281795179118391	-0.001495361328125	\\
0.281839570293426	-0.00152587890625	\\
0.28188396146846	-0.001556396484375	\\
0.281928352643494	-0.0018310546875	\\
0.281972743818529	-0.001312255859375	\\
0.282017134993563	-0.00103759765625	\\
0.282061526168598	-0.0013427734375	\\
0.282105917343632	-0.000640869140625	\\
0.282150308518667	-0.000244140625	\\
0.282194699693701	-0.000396728515625	\\
0.282239090868735	-0.000274658203125	\\
0.28228348204377	-0.000732421875	\\
0.282327873218804	-6.103515625e-05	\\
0.282372264393839	0.0003662109375	\\
0.282416655568873	-0.000152587890625	\\
0.282461046743907	9.1552734375e-05	\\
0.282505437918942	0.0006103515625	\\
0.282549829093976	0.0006103515625	\\
0.282594220269011	0.000335693359375	\\
0.282638611444045	0	\\
0.282683002619079	3.0517578125e-05	\\
0.282727393794114	0.0003662109375	\\
0.282771784969148	-3.0517578125e-05	\\
0.282816176144183	-9.1552734375e-05	\\
0.282860567319217	-3.0517578125e-05	\\
0.282904958494251	6.103515625e-05	\\
0.282949349669286	0.0003662109375	\\
0.28299374084432	3.0517578125e-05	\\
0.283038132019355	0	\\
0.283082523194389	0.000274658203125	\\
0.283126914369423	0.00030517578125	\\
0.283171305544458	0.000213623046875	\\
0.283215696719492	0.0001220703125	\\
0.283260087894527	0.00030517578125	\\
0.283304479069561	0.00018310546875	\\
0.283348870244595	0	\\
0.28339326141963	0.0001220703125	\\
0.283437652594664	0.0001220703125	\\
0.283482043769699	3.0517578125e-05	\\
0.283526434944733	0	\\
0.283570826119767	-0.00018310546875	\\
0.283615217294802	-9.1552734375e-05	\\
0.283659608469836	-0.000152587890625	\\
0.283703999644871	-0.000335693359375	\\
0.283748390819905	-6.103515625e-05	\\
0.283792781994939	0.0001220703125	\\
0.283837173169974	0.00030517578125	\\
0.283881564345008	0.000396728515625	\\
0.283925955520043	0.0003662109375	\\
0.283970346695077	0.000640869140625	\\
0.284014737870111	0.000579833984375	\\
0.284059129045146	0.0003662109375	\\
0.28410352022018	0.00042724609375	\\
0.284147911395215	0.000335693359375	\\
0.284192302570249	0.000701904296875	\\
0.284236693745283	0.000823974609375	\\
0.284281084920318	0.000732421875	\\
0.284325476095352	0.0009765625	\\
0.284369867270387	0.00079345703125	\\
0.284414258445421	0.000885009765625	\\
0.284458649620455	0.001007080078125	\\
0.28450304079549	0.000885009765625	\\
0.284547431970524	0.001220703125	\\
0.284591823145559	0.001251220703125	\\
0.284636214320593	0.001007080078125	\\
0.284680605495627	0.000640869140625	\\
0.284724996670662	0.0006103515625	\\
0.284769387845696	0.00103759765625	\\
0.284813779020731	0.001129150390625	\\
0.284858170195765	0.0003662109375	\\
0.2849025613708	0.0003662109375	\\
0.284946952545834	0.000335693359375	\\
0.284991343720868	-0.00042724609375	\\
0.285035734895903	-0.0008544921875	\\
0.285080126070937	-0.000396728515625	\\
0.285124517245971	-0.000335693359375	\\
0.285168908421006	-0.000457763671875	\\
0.28521329959604	-0.000732421875	\\
0.285257690771075	-0.000579833984375	\\
0.285302081946109	-6.103515625e-05	\\
0.285346473121144	0.00030517578125	\\
0.285390864296178	0.0001220703125	\\
0.285435255471212	-3.0517578125e-05	\\
0.285479646646247	0.0003662109375	\\
0.285524037821281	0.000213623046875	\\
0.285568428996316	0.0001220703125	\\
0.28561282017135	0	\\
0.285657211346384	-0.000274658203125	\\
0.285701602521419	3.0517578125e-05	\\
0.285745993696453	0.0001220703125	\\
0.285790384871488	0.000274658203125	\\
0.285834776046522	-0.0001220703125	\\
0.285879167221556	-3.0517578125e-05	\\
0.285923558396591	-9.1552734375e-05	\\
0.285967949571625	-0.0003662109375	\\
0.28601234074666	-0.000244140625	\\
0.286056731921694	-0.000213623046875	\\
0.286101123096728	6.103515625e-05	\\
0.286145514271763	-0.0003662109375	\\
0.286189905446797	-0.0001220703125	\\
0.286234296621832	0.000335693359375	\\
0.286278687796866	0.000396728515625	\\
0.2863230789719	0.00048828125	\\
0.286367470146935	0.00054931640625	\\
0.286411861321969	0.0003662109375	\\
0.286456252497004	0.000518798828125	\\
0.286500643672038	0.000457763671875	\\
0.286545034847072	0.000762939453125	\\
0.286589426022107	0.00103759765625	\\
0.286633817197141	0.00091552734375	\\
0.286678208372176	0.001068115234375	\\
0.28672259954721	0.001220703125	\\
0.286766990722244	0.00128173828125	\\
0.286811381897279	0.0009765625	\\
0.286855773072313	0.001251220703125	\\
0.286900164247348	0.00128173828125	\\
0.286944555422382	0.001190185546875	\\
0.286988946597416	0.001220703125	\\
0.287033337772451	0.000946044921875	\\
0.287077728947485	0.00091552734375	\\
0.28712212012252	0.0008544921875	\\
0.287166511297554	0.001129150390625	\\
0.287210902472588	0.00103759765625	\\
0.287255293647623	0.0006103515625	\\
0.287299684822657	0.000274658203125	\\
0.287344075997692	0.000274658203125	\\
0.287388467172726	-3.0517578125e-05	\\
0.28743285834776	-0.0003662109375	\\
0.287477249522795	3.0517578125e-05	\\
0.287521640697829	0.000152587890625	\\
0.287566031872864	-9.1552734375e-05	\\
0.287610423047898	0.00048828125	\\
0.287654814222932	0.000946044921875	\\
0.287699205397967	0.00067138671875	\\
0.287743596573001	0.001312255859375	\\
0.287787987748036	0.0010986328125	\\
0.28783237892307	0.000823974609375	\\
0.287876770098104	0.00146484375	\\
0.287921161273139	0.001678466796875	\\
0.287965552448173	0.00201416015625	\\
0.288009943623208	0.002105712890625	\\
0.288054334798242	0.0025634765625	\\
0.288098725973277	0.002471923828125	\\
0.288143117148311	0.002532958984375	\\
0.288187508323345	0.00238037109375	\\
0.28823189949838	0.002471923828125	\\
0.288276290673414	0.00299072265625	\\
0.288320681848449	0.0032958984375	\\
0.288365073023483	0.003662109375	\\
0.288409464198517	0.00384521484375	\\
0.288453855373552	0.003509521484375	\\
0.288498246548586	0.003326416015625	\\
0.288542637723621	0.003021240234375	\\
0.288587028898655	0.002960205078125	\\
0.288631420073689	0.0032958984375	\\
0.288675811248724	0.00299072265625	\\
0.288720202423758	0.002655029296875	\\
0.288764593598793	0.002166748046875	\\
0.288808984773827	0.00201416015625	\\
0.288853375948861	0.00189208984375	\\
0.288897767123896	0.001556396484375	\\
0.28894215829893	0.001708984375	\\
0.288986549473965	0.001312255859375	\\
0.289030940648999	0.000579833984375	\\
0.289075331824033	0.0003662109375	\\
0.289119722999068	0.000335693359375	\\
0.289164114174102	0.00054931640625	\\
0.289208505349137	0.000396728515625	\\
0.289252896524171	-0.0001220703125	\\
0.289297287699205	0.0001220703125	\\
0.28934167887424	-0.000213623046875	\\
0.289386070049274	-0.000640869140625	\\
0.289430461224309	-0.000457763671875	\\
0.289474852399343	-9.1552734375e-05	\\
0.289519243574377	-0.0001220703125	\\
0.289563634749412	-0.000732421875	\\
0.289608025924446	-0.0006103515625	\\
0.289652417099481	-0.0003662109375	\\
0.289696808274515	-0.000335693359375	\\
0.289741199449549	-0.00018310546875	\\
0.289785590624584	-0.0001220703125	\\
0.289829981799618	0.000640869140625	\\
0.289874372974653	0.00067138671875	\\
0.289918764149687	0.00115966796875	\\
0.289963155324721	0.001556396484375	\\
0.290007546499756	0.00140380859375	\\
0.29005193767479	0.001800537109375	\\
0.290096328849825	0.00225830078125	\\
0.290140720024859	0.002716064453125	\\
0.290185111199893	0.002166748046875	\\
0.290229502374928	0.00189208984375	\\
0.290273893549962	0.002532958984375	\\
0.290318284724997	0.00238037109375	\\
0.290362675900031	0.00225830078125	\\
0.290407067075065	0.00250244140625	\\
0.2904514582501	0.002197265625	\\
0.290495849425134	0.002410888671875	\\
0.290540240600169	0.00238037109375	\\
0.290584631775203	0.002655029296875	\\
0.290629022950238	0.002685546875	\\
0.290673414125272	0.002349853515625	\\
0.290717805300306	0.002532958984375	\\
0.290762196475341	0.00250244140625	\\
0.290806587650375	0.002593994140625	\\
0.29085097882541	0.002349853515625	\\
0.290895370000444	0.002197265625	\\
0.290939761175478	0.002166748046875	\\
0.290984152350513	0.001617431640625	\\
0.291028543525547	0.00164794921875	\\
0.291072934700582	0.00189208984375	\\
0.291117325875616	0.001861572265625	\\
0.29116171705065	0.001617431640625	\\
0.291206108225685	0.001373291015625	\\
0.291250499400719	0.001373291015625	\\
0.291294890575754	0.00140380859375	\\
0.291339281750788	0.00164794921875	\\
0.291383672925822	0.001678466796875	\\
0.291428064100857	0.002044677734375	\\
0.291472455275891	0.001983642578125	\\
0.291516846450926	0.001495361328125	\\
0.29156123762596	0.001312255859375	\\
0.291605628800994	0.001312255859375	\\
0.291650019976029	0.0015869140625	\\
0.291694411151063	0.001800537109375	\\
0.291738802326098	0.00201416015625	\\
0.291783193501132	0.001800537109375	\\
0.291827584676166	0.001861572265625	\\
0.291871975851201	0.002044677734375	\\
0.291916367026235	0.00225830078125	\\
0.29196075820127	0.002227783203125	\\
0.292005149376304	0.002105712890625	\\
0.292049540551338	0.00201416015625	\\
0.292093931726373	0.002227783203125	\\
0.292138322901407	0.002105712890625	\\
0.292182714076442	0.002105712890625	\\
0.292227105251476	0.00213623046875	\\
0.29227149642651	0.002532958984375	\\
0.292315887601545	0.002960205078125	\\
0.292360278776579	0.002655029296875	\\
0.292404669951614	0.00225830078125	\\
0.292449061126648	0.0018310546875	\\
0.292493452301682	0.00213623046875	\\
0.292537843476717	0.001800537109375	\\
0.292582234651751	0.001953125	\\
0.292626625826786	0.00225830078125	\\
0.29267101700182	0.001708984375	\\
0.292715408176854	0.00152587890625	\\
0.292759799351889	0.001617431640625	\\
0.292804190526923	0.001312255859375	\\
0.292848581701958	0.000946044921875	\\
0.292892972876992	0.000640869140625	\\
0.292937364052026	0.00018310546875	\\
0.292981755227061	6.103515625e-05	\\
0.293026146402095	0.000457763671875	\\
0.29307053757713	0.000732421875	\\
0.293114928752164	0.000640869140625	\\
0.293159319927198	0.000518798828125	\\
0.293203711102233	0.000396728515625	\\
0.293248102277267	0.0006103515625	\\
0.293292493452302	0.0009765625	\\
0.293336884627336	0.000640869140625	\\
0.293381275802371	0.00079345703125	\\
0.293425666977405	0.000732421875	\\
0.293470058152439	0.000885009765625	\\
0.293514449327474	0.000701904296875	\\
0.293558840502508	0.000518798828125	\\
0.293603231677542	0.000396728515625	\\
0.293647622852577	0.00042724609375	\\
0.293692014027611	0.00103759765625	\\
0.293736405202646	0.000335693359375	\\
0.29378079637768	0.00030517578125	\\
0.293825187552715	0.0006103515625	\\
0.293869578727749	0.000640869140625	\\
0.293913969902783	0.00067138671875	\\
0.293958361077818	9.1552734375e-05	\\
0.294002752252852	0.000457763671875	\\
0.294047143427887	0.000518798828125	\\
0.294091534602921	0.00079345703125	\\
0.294135925777955	0.001373291015625	\\
0.29418031695299	0.001068115234375	\\
0.294224708128024	0.0009765625	\\
0.294269099303059	0.00140380859375	\\
0.294313490478093	0.00128173828125	\\
0.294357881653127	0.001007080078125	\\
0.294402272828162	0.001251220703125	\\
0.294446664003196	0.001007080078125	\\
0.294491055178231	0.0010986328125	\\
0.294535446353265	0.001312255859375	\\
0.294579837528299	0.001251220703125	\\
0.294624228703334	0.001373291015625	\\
0.294668619878368	0.00146484375	\\
0.294713011053403	0.001312255859375	\\
0.294757402228437	0.00115966796875	\\
0.294801793403471	0.001373291015625	\\
0.294846184578506	0.001434326171875	\\
0.29489057575354	0.001434326171875	\\
0.294934966928575	0.001800537109375	\\
0.294979358103609	0.0018310546875	\\
0.295023749278643	0.001708984375	\\
0.295068140453678	0.00189208984375	\\
0.295112531628712	0.0020751953125	\\
0.295156922803747	0.001922607421875	\\
0.295201313978781	0.001678466796875	\\
0.295245705153815	0.00140380859375	\\
0.29529009632885	0.0013427734375	\\
0.295334487503884	0.0020751953125	\\
0.295378878678919	0.001739501953125	\\
0.295423269853953	0.0015869140625	\\
0.295467661028987	0.001983642578125	\\
0.295512052204022	0.001953125	\\
0.295556443379056	0.001861572265625	\\
0.295600834554091	0.0020751953125	\\
0.295645225729125	0.00244140625	\\
0.295689616904159	0.0025634765625	\\
0.295734008079194	0.002166748046875	\\
0.295778399254228	0.001983642578125	\\
0.295822790429263	0.002685546875	\\
0.295867181604297	0.003265380859375	\\
0.295911572779331	0.00286865234375	\\
0.295955963954366	0.003082275390625	\\
0.2960003551294	0.0035400390625	\\
0.296044746304435	0.00347900390625	\\
0.296089137479469	0.003692626953125	\\
0.296133528654503	0.0037841796875	\\
0.296177919829538	0.00390625	\\
0.296222311004572	0.003509521484375	\\
0.296266702179607	0.003509521484375	\\
0.296311093354641	0.00335693359375	\\
0.296355484529676	0.003143310546875	\\
0.29639987570471	0.0032958984375	\\
0.296444266879744	0.0028076171875	\\
0.296488658054779	0.002471923828125	\\
0.296533049229813	0.00244140625	\\
0.296577440404848	0.002288818359375	\\
0.296621831579882	0.001953125	\\
0.296666222754916	0.00244140625	\\
0.296710613929951	0.002410888671875	\\
0.296755005104985	0.001861572265625	\\
0.29679939628002	0.0018310546875	\\
0.296843787455054	0.001556396484375	\\
0.296888178630088	0.001129150390625	\\
0.296932569805123	0.000579833984375	\\
0.296976960980157	0.000732421875	\\
0.297021352155192	0.00048828125	\\
0.297065743330226	-0.000213623046875	\\
0.29711013450526	-0.000213623046875	\\
0.297154525680295	-0.00042724609375	\\
0.297198916855329	-0.000457763671875	\\
0.297243308030364	9.1552734375e-05	\\
0.297287699205398	-3.0517578125e-05	\\
0.297332090380432	-0.000396728515625	\\
0.297376481555467	-0.000335693359375	\\
0.297420872730501	-0.00048828125	\\
0.297465263905536	-0.00018310546875	\\
0.29750965508057	-0.00048828125	\\
0.297554046255604	-0.0006103515625	\\
0.297598437430639	-0.000701904296875	\\
0.297642828605673	-0.000762939453125	\\
0.297687219780708	-0.000152587890625	\\
0.297731610955742	-0.000152587890625	\\
0.297776002130776	-0.000274658203125	\\
0.297820393305811	-0.000274658203125	\\
0.297864784480845	-0.00048828125	\\
0.29790917565588	-0.00018310546875	\\
0.297953566830914	-0.00030517578125	\\
0.297997958005948	-0.000152587890625	\\
0.298042349180983	0.000335693359375	\\
0.298086740356017	0.00030517578125	\\
0.298131131531052	0.000335693359375	\\
0.298175522706086	0.000274658203125	\\
0.29821991388112	0.000701904296875	\\
0.298264305056155	0.00054931640625	\\
0.298308696231189	0.00030517578125	\\
0.298353087406224	0.000335693359375	\\
0.298397478581258	0.000244140625	\\
0.298441869756292	0.000732421875	\\
0.298486260931327	0.000823974609375	\\
0.298530652106361	0.001007080078125	\\
0.298575043281396	0.0010986328125	\\
0.29861943445643	0.00103759765625	\\
0.298663825631464	0.001373291015625	\\
0.298708216806499	0.001312255859375	\\
0.298752607981533	0.00115966796875	\\
0.298796999156568	0.001129150390625	\\
0.298841390331602	0.001007080078125	\\
0.298885781506636	0.0008544921875	\\
0.298930172681671	0.000885009765625	\\
0.298974563856705	0.000457763671875	\\
0.29901895503174	0.000152587890625	\\
0.299063346206774	0.000640869140625	\\
0.299107737381809	0.0008544921875	\\
0.299152128556843	0.000518798828125	\\
0.299196519731877	0.000518798828125	\\
0.299240910906912	0.000518798828125	\\
0.299285302081946	0.0001220703125	\\
0.299329693256981	-0.000213623046875	\\
0.299374084432015	3.0517578125e-05	\\
0.299418475607049	-3.0517578125e-05	\\
0.299462866782084	0.000396728515625	\\
0.299507257957118	0.000701904296875	\\
0.299551649132153	0	\\
0.299596040307187	-0.000152587890625	\\
0.299640431482221	-9.1552734375e-05	\\
0.299684822657256	6.103515625e-05	\\
0.29972921383229	0.0003662109375	\\
0.299773605007325	0.00030517578125	\\
0.299817996182359	-6.103515625e-05	\\
0.299862387357393	9.1552734375e-05	\\
0.299906778532428	0.0006103515625	\\
0.299951169707462	0.001007080078125	\\
0.299995560882497	0.000885009765625	\\
0.300039952057531	0.00048828125	\\
0.300084343232565	0.001068115234375	\\
0.3001287344076	0.00103759765625	\\
0.300173125582634	0.0010986328125	\\
0.300217516757669	0.00128173828125	\\
0.300261907932703	0.001617431640625	\\
0.300306299107737	0.001220703125	\\
0.300350690282772	0.0010986328125	\\
0.300395081457806	0.0013427734375	\\
0.300439472632841	0.00079345703125	\\
0.300483863807875	0.000946044921875	\\
0.300528254982909	0.0006103515625	\\
0.300572646157944	3.0517578125e-05	\\
0.300617037332978	0.000335693359375	\\
0.300661428508013	0.000640869140625	\\
0.300705819683047	0.0006103515625	\\
0.300750210858081	0.0006103515625	\\
0.300794602033116	0.0006103515625	\\
0.30083899320815	9.1552734375e-05	\\
0.300883384383185	0.000274658203125	\\
0.300927775558219	0.00067138671875	\\
0.300972166733253	0.000518798828125	\\
0.301016557908288	0.00054931640625	\\
0.301060949083322	0.000518798828125	\\
0.301105340258357	-0.000152587890625	\\
0.301149731433391	9.1552734375e-05	\\
0.301194122608425	6.103515625e-05	\\
0.30123851378346	-0.000244140625	\\
0.301282904958494	-0.0001220703125	\\
0.301327296133529	-0.0003662109375	\\
0.301371687308563	-0.00018310546875	\\
0.301416078483597	0.00018310546875	\\
0.301460469658632	0.000244140625	\\
0.301504860833666	0.0003662109375	\\
0.301549252008701	-0.0001220703125	\\
0.301593643183735	-0.00018310546875	\\
0.301638034358769	0.000244140625	\\
0.301682425533804	-6.103515625e-05	\\
0.301726816708838	0.0001220703125	\\
0.301771207883873	0.000274658203125	\\
0.301815599058907	0.000274658203125	\\
0.301859990233942	0.000335693359375	\\
0.301904381408976	0	\\
0.30194877258401	0.00018310546875	\\
0.301993163759045	6.103515625e-05	\\
0.302037554934079	-0.000274658203125	\\
0.302081946109113	0	\\
0.302126337284148	0.0001220703125	\\
0.302170728459182	0.00054931640625	\\
0.302215119634217	0.0006103515625	\\
0.302259510809251	0.000885009765625	\\
0.302303901984286	0.000823974609375	\\
0.30234829315932	0.00103759765625	\\
0.302392684334354	0.0008544921875	\\
0.302437075509389	0.000640869140625	\\
0.302481466684423	0.001129150390625	\\
0.302525857859458	0.001434326171875	\\
0.302570249034492	0.00140380859375	\\
0.302614640209526	0.00103759765625	\\
0.302659031384561	0.001373291015625	\\
0.302703422559595	0.00140380859375	\\
0.30274781373463	0.001190185546875	\\
0.302792204909664	0.001007080078125	\\
0.302836596084698	0.00079345703125	\\
0.302880987259733	0.0006103515625	\\
0.302925378434767	0.000457763671875	\\
0.302969769609802	0.0006103515625	\\
0.303014160784836	0.00042724609375	\\
0.30305855195987	0.0003662109375	\\
0.303102943134905	0.0006103515625	\\
0.303147334309939	0.0009765625	\\
0.303191725484974	0.00079345703125	\\
0.303236116660008	0.000457763671875	\\
0.303280507835042	0.00067138671875	\\
0.303324899010077	0.000732421875	\\
0.303369290185111	0.000732421875	\\
0.303413681360146	0.000946044921875	\\
0.30345807253518	0.00067138671875	\\
0.303502463710214	0.0006103515625	\\
0.303546854885249	0.001312255859375	\\
0.303591246060283	0.001434326171875	\\
0.303635637235318	0.001556396484375	\\
0.303680028410352	0.00152587890625	\\
0.303724419585386	0.0008544921875	\\
0.303768810760421	0.00103759765625	\\
0.303813201935455	0.001373291015625	\\
0.30385759311049	0.001129150390625	\\
0.303901984285524	0.0013427734375	\\
0.303946375460558	0.001556396484375	\\
0.303990766635593	0.001220703125	\\
0.304035157810627	0.001373291015625	\\
0.304079548985662	0.00177001953125	\\
0.304123940160696	0.00140380859375	\\
0.30416833133573	0.00152587890625	\\
0.304212722510765	0.001678466796875	\\
0.304257113685799	0.001312255859375	\\
0.304301504860834	0.000823974609375	\\
0.304345896035868	0.00048828125	\\
0.304390287210903	0.001251220703125	\\
0.304434678385937	0.001068115234375	\\
0.304479069560971	0.00054931640625	\\
0.304523460736006	0.000518798828125	\\
0.30456785191104	0.000885009765625	\\
0.304612243086074	0.0010986328125	\\
0.304656634261109	0.0008544921875	\\
0.304701025436143	0.0009765625	\\
0.304745416611178	0.001495361328125	\\
0.304789807786212	0.00164794921875	\\
0.304834198961247	0.0009765625	\\
0.304878590136281	0.001251220703125	\\
0.304922981311315	0.001312255859375	\\
0.30496737248635	0.000946044921875	\\
0.305011763661384	0.0013427734375	\\
0.305056154836419	0.00128173828125	\\
0.305100546011453	0.001495361328125	\\
0.305144937186487	0.001678466796875	\\
0.305189328361522	0.001983642578125	\\
0.305233719536556	0.002044677734375	\\
0.305278110711591	0.001678466796875	\\
0.305322501886625	0.001678466796875	\\
0.305366893061659	0.0013427734375	\\
0.305411284236694	0.001312255859375	\\
0.305455675411728	0.001251220703125	\\
0.305500066586763	0.001007080078125	\\
0.305544457761797	0.001068115234375	\\
0.305588848936831	0.001068115234375	\\
0.305633240111866	0.001129150390625	\\
0.3056776312869	0.00128173828125	\\
0.305722022461935	0.001617431640625	\\
0.305766413636969	0.001617431640625	\\
0.305810804812003	0.0013427734375	\\
0.305855195987038	0.001007080078125	\\
0.305899587162072	0.0006103515625	\\
0.305943978337107	0.000762939453125	\\
0.305988369512141	0.0008544921875	\\
0.306032760687175	0.000701904296875	\\
0.30607715186221	0.000335693359375	\\
0.306121543037244	0.000396728515625	\\
0.306165934212279	0.000152587890625	\\
0.306210325387313	0.000152587890625	\\
0.306254716562347	0.000518798828125	\\
0.306299107737382	0.000457763671875	\\
0.306343498912416	0.00048828125	\\
0.306387890087451	0.000579833984375	\\
0.306432281262485	0.000274658203125	\\
0.306476672437519	0.00042724609375	\\
0.306521063612554	0.00054931640625	\\
0.306565454787588	0.0003662109375	\\
0.306609845962623	9.1552734375e-05	\\
0.306654237137657	6.103515625e-05	\\
0.306698628312691	9.1552734375e-05	\\
0.306743019487726	0.0001220703125	\\
0.30678741066276	0.00018310546875	\\
0.306831801837795	0.000396728515625	\\
0.306876193012829	0.00030517578125	\\
0.306920584187863	0.000213623046875	\\
0.306964975362898	0.00030517578125	\\
0.307009366537932	0.00030517578125	\\
0.307053757712967	0.00018310546875	\\
0.307098148888001	3.0517578125e-05	\\
0.307142540063035	0.000152587890625	\\
0.30718693123807	0.00018310546875	\\
0.307231322413104	0.000152587890625	\\
0.307275713588139	-3.0517578125e-05	\\
0.307320104763173	0.0001220703125	\\
0.307364495938207	-9.1552734375e-05	\\
0.307408887113242	-0.0001220703125	\\
0.307453278288276	9.1552734375e-05	\\
0.307497669463311	-0.00042724609375	\\
0.307542060638345	-0.00042724609375	\\
0.30758645181338	-0.000244140625	\\
0.307630842988414	-0.000335693359375	\\
0.307675234163448	-0.000274658203125	\\
0.307719625338483	-0.000152587890625	\\
0.307764016513517	0.00018310546875	\\
0.307808407688552	-3.0517578125e-05	\\
0.307852798863586	-0.000213623046875	\\
0.30789719003862	-0.0001220703125	\\
0.307941581213655	-0.0003662109375	\\
0.307985972388689	9.1552734375e-05	\\
0.308030363563724	0.000335693359375	\\
0.308074754738758	-0.000213623046875	\\
0.308119145913792	-0.000274658203125	\\
0.308163537088827	-0.0006103515625	\\
0.308207928263861	-0.000457763671875	\\
0.308252319438896	-0.000335693359375	\\
0.30829671061393	-0.000701904296875	\\
0.308341101788964	-0.000335693359375	\\
0.308385492963999	-0.000244140625	\\
0.308429884139033	-0.000274658203125	\\
0.308474275314068	0	\\
0.308518666489102	-0.00018310546875	\\
0.308563057664136	-0.0003662109375	\\
0.308607448839171	9.1552734375e-05	\\
0.308651840014205	0.000335693359375	\\
0.30869623118924	0.000244140625	\\
0.308740622364274	0.0003662109375	\\
0.308785013539308	0.00054931640625	\\
0.308829404714343	-0.00018310546875	\\
0.308873795889377	-0.000244140625	\\
0.308918187064412	0.000274658203125	\\
0.308962578239446	0.0006103515625	\\
0.30900696941448	0.000732421875	\\
0.309051360589515	0.000640869140625	\\
0.309095751764549	0.000335693359375	\\
0.309140142939584	0.00048828125	\\
0.309184534114618	0.00054931640625	\\
0.309228925289652	6.103515625e-05	\\
0.309273316464687	-0.0001220703125	\\
0.309317707639721	-0.00018310546875	\\
0.309362098814756	-0.0008544921875	\\
0.30940648998979	-0.00048828125	\\
0.309450881164824	-0.000152587890625	\\
0.309495272339859	-0.00054931640625	\\
0.309539663514893	-0.000213623046875	\\
0.309584054689928	0.000335693359375	\\
0.309628445864962	-9.1552734375e-05	\\
0.309672837039996	-0.0001220703125	\\
0.309717228215031	0	\\
0.309761619390065	-0.0001220703125	\\
0.3098060105651	0.000274658203125	\\
0.309850401740134	0.0001220703125	\\
0.309894792915168	-0.000274658203125	\\
0.309939184090203	0.00054931640625	\\
0.309983575265237	0.001251220703125	\\
0.310027966440272	0.001220703125	\\
0.310072357615306	0.0009765625	\\
0.31011674879034	0.00115966796875	\\
0.310161139965375	0.0013427734375	\\
0.310205531140409	0.00146484375	\\
0.310249922315444	0.001495361328125	\\
0.310294313490478	0.001434326171875	\\
0.310338704665513	0.001129150390625	\\
0.310383095840547	0.00115966796875	\\
0.310427487015581	0.00152587890625	\\
0.310471878190616	0.00140380859375	\\
0.31051626936565	0.00140380859375	\\
0.310560660540684	0.001678466796875	\\
0.310605051715719	0.001068115234375	\\
0.310649442890753	0.000946044921875	\\
0.310693834065788	0.00115966796875	\\
0.310738225240822	0.0009765625	\\
0.310782616415857	0.000335693359375	\\
0.310827007590891	0.000396728515625	\\
0.310871398765925	0.0006103515625	\\
0.31091578994096	0.00042724609375	\\
0.310960181115994	0.000946044921875	\\
0.311004572291029	0.001129150390625	\\
0.311048963466063	0.000762939453125	\\
0.311093354641097	0.000762939453125	\\
0.311137745816132	0.000762939453125	\\
0.311182136991166	0.001007080078125	\\
0.311226528166201	0.00103759765625	\\
0.311270919341235	0.001068115234375	\\
0.311315310516269	0.0009765625	\\
0.311359701691304	0.000885009765625	\\
0.311404092866338	0.001190185546875	\\
0.311448484041373	0.001007080078125	\\
0.311492875216407	0.0009765625	\\
0.311537266391441	0.000885009765625	\\
0.311581657566476	0.00091552734375	\\
0.31162604874151	0.001129150390625	\\
0.311670439916545	0.00115966796875	\\
0.311714831091579	0.00091552734375	\\
0.311759222266613	0.00048828125	\\
0.311803613441648	0.000518798828125	\\
0.311848004616682	0	\\
0.311892395791717	-0.000244140625	\\
0.311936786966751	-0.000152587890625	\\
0.311981178141785	-0.00048828125	\\
0.31202556931682	-0.000579833984375	\\
0.312069960491854	-0.000335693359375	\\
0.312114351666889	-0.00054931640625	\\
0.312158742841923	-0.00067138671875	\\
0.312203134016957	-0.000579833984375	\\
0.312247525191992	-0.000335693359375	\\
0.312291916367026	-0.00048828125	\\
0.312336307542061	-0.000396728515625	\\
0.312380698717095	-0.000152587890625	\\
0.312425089892129	-0.0001220703125	\\
0.312469481067164	0.000152587890625	\\
0.312513872242198	0.000244140625	\\
0.312558263417233	0.000823974609375	\\
0.312602654592267	0.000640869140625	\\
0.312647045767301	0.000457763671875	\\
0.312691436942336	0.00103759765625	\\
0.31273582811737	0.000823974609375	\\
0.312780219292405	0.00048828125	\\
0.312824610467439	0.000701904296875	\\
0.312869001642474	0.000823974609375	\\
0.312913392817508	0.001129150390625	\\
0.312957783992542	0.00128173828125	\\
0.313002175167577	0.001190185546875	\\
0.313046566342611	0.001739501953125	\\
0.313090957517645	0.001800537109375	\\
0.31313534869268	0.001373291015625	\\
0.313179739867714	0.00128173828125	\\
0.313224131042749	0.001617431640625	\\
0.313268522217783	0.001312255859375	\\
0.313312913392818	0.001312255859375	\\
0.313357304567852	0.001678466796875	\\
0.313401695742886	0.001312255859375	\\
0.313446086917921	0.000762939453125	\\
0.313490478092955	0.0001220703125	\\
0.31353486926799	0.00018310546875	\\
0.313579260443024	0.000152587890625	\\
0.313623651618058	0.000152587890625	\\
0.313668042793093	0.000335693359375	\\
0.313712433968127	0.000274658203125	\\
0.313756825143162	-9.1552734375e-05	\\
0.313801216318196	0	\\
0.31384560749323	3.0517578125e-05	\\
0.313889998668265	-0.0001220703125	\\
0.313934389843299	-3.0517578125e-05	\\
0.313978781018334	-0.000732421875	\\
0.314023172193368	-0.00067138671875	\\
0.314067563368402	-0.000762939453125	\\
0.314111954543437	-0.001220703125	\\
0.314156345718471	-0.001251220703125	\\
0.314200736893506	-0.00146484375	\\
0.31424512806854	-0.001617431640625	\\
0.314289519243574	-0.00164794921875	\\
0.314333910418609	-0.001312255859375	\\
0.314378301593643	-0.0010986328125	\\
0.314422692768678	-0.0010986328125	\\
0.314467083943712	-0.00091552734375	\\
0.314511475118746	-0.0008544921875	\\
0.314555866293781	-0.00067138671875	\\
0.314600257468815	-0.000823974609375	\\
0.31464464864385	-0.000396728515625	\\
0.314689039818884	0.0001220703125	\\
0.314733430993918	-0.000396728515625	\\
0.314777822168953	-9.1552734375e-05	\\
0.314822213343987	0.000152587890625	\\
0.314866604519022	0.00030517578125	\\
0.314910995694056	0.0003662109375	\\
0.31495538686909	0.000762939453125	\\
0.314999778044125	0.000732421875	\\
0.315044169219159	0.000701904296875	\\
0.315088560394194	0.0006103515625	\\
0.315132951569228	0.000732421875	\\
0.315177342744262	0.0010986328125	\\
0.315221733919297	0.00091552734375	\\
0.315266125094331	0.00067138671875	\\
0.315310516269366	0.000274658203125	\\
0.3153549074444	0.00079345703125	\\
0.315399298619434	0.001068115234375	\\
0.315443689794469	0.000823974609375	\\
0.315488080969503	0.00067138671875	\\
0.315532472144538	0.00042724609375	\\
0.315576863319572	0.00048828125	\\
0.315621254494606	0.000213623046875	\\
0.315665645669641	0.000457763671875	\\
0.315710036844675	0.000457763671875	\\
0.31575442801971	-0.00018310546875	\\
0.315798819194744	-0.00030517578125	\\
0.315843210369778	-0.00018310546875	\\
0.315887601544813	-6.103515625e-05	\\
0.315931992719847	-0.000457763671875	\\
0.315976383894882	-0.0003662109375	\\
0.316020775069916	-0.000701904296875	\\
0.316065166244951	-0.000885009765625	\\
0.316109557419985	-0.000762939453125	\\
0.316153948595019	-0.0008544921875	\\
0.316198339770054	-0.000579833984375	\\
0.316242730945088	-0.000640869140625	\\
0.316287122120123	-0.0008544921875	\\
0.316331513295157	-0.000762939453125	\\
0.316375904470191	-0.000732421875	\\
0.316420295645226	-0.00079345703125	\\
0.31646468682026	-0.001068115234375	\\
0.316509077995295	-0.000946044921875	\\
0.316553469170329	-0.000885009765625	\\
0.316597860345363	-0.001251220703125	\\
0.316642251520398	-0.001007080078125	\\
0.316686642695432	-0.00079345703125	\\
0.316731033870467	-0.000946044921875	\\
0.316775425045501	-0.00128173828125	\\
0.316819816220535	-0.001190185546875	\\
0.31686420739557	-0.000946044921875	\\
0.316908598570604	-0.000946044921875	\\
0.316952989745639	-0.0009765625	\\
0.316997380920673	-0.0008544921875	\\
0.317041772095707	-0.000823974609375	\\
0.317086163270742	-0.00054931640625	\\
0.317130554445776	-0.00054931640625	\\
0.317174945620811	-0.00048828125	\\
0.317219336795845	-0.000274658203125	\\
0.317263727970879	-0.000640869140625	\\
0.317308119145914	-0.00048828125	\\
0.317352510320948	-0.00054931640625	\\
0.317396901495983	-0.0006103515625	\\
0.317441292671017	-0.000274658203125	\\
0.317485683846051	0.00018310546875	\\
0.317530075021086	0.0001220703125	\\
0.31757446619612	0.000213623046875	\\
0.317618857371155	0.000335693359375	\\
0.317663248546189	0.000579833984375	\\
0.317707639721223	0.0009765625	\\
0.317752030896258	0.000823974609375	\\
0.317796422071292	0.0008544921875	\\
0.317840813246327	0.001129150390625	\\
0.317885204421361	0.001312255859375	\\
0.317929595596395	0.00115966796875	\\
0.31797398677143	0.001434326171875	\\
0.318018377946464	0.00128173828125	\\
0.318062769121499	0.0009765625	\\
0.318107160296533	0.001312255859375	\\
0.318151551471567	0.001251220703125	\\
0.318195942646602	0.00115966796875	\\
0.318240333821636	0.001220703125	\\
0.318284724996671	0.00140380859375	\\
0.318329116171705	0.001007080078125	\\
0.318373507346739	0.00048828125	\\
0.318417898521774	0.001007080078125	\\
0.318462289696808	0.00048828125	\\
0.318506680871843	0.00030517578125	\\
0.318551072046877	0.0008544921875	\\
0.318595463221911	0.000152587890625	\\
0.318639854396946	-6.103515625e-05	\\
0.31868424557198	6.103515625e-05	\\
0.318728636747015	-0.0003662109375	\\
0.318773027922049	-0.00018310546875	\\
0.318817419097084	9.1552734375e-05	\\
0.318861810272118	9.1552734375e-05	\\
0.318906201447152	0.000274658203125	\\
0.318950592622187	0.000579833984375	\\
0.318994983797221	0.00103759765625	\\
0.319039374972256	0.000885009765625	\\
0.31908376614729	0.0009765625	\\
0.319128157322324	0.00128173828125	\\
0.319172548497359	0.001190185546875	\\
0.319216939672393	0.0013427734375	\\
0.319261330847428	0.001678466796875	\\
0.319305722022462	0.002105712890625	\\
0.319350113197496	0.00201416015625	\\
0.319394504372531	0.001800537109375	\\
0.319438895547565	0.0018310546875	\\
0.3194832867226	0.001922607421875	\\
0.319527677897634	0.0018310546875	\\
0.319572069072668	0.001800537109375	\\
0.319616460247703	0.001953125	\\
0.319660851422737	0.001953125	\\
0.319705242597772	0.002410888671875	\\
0.319749633772806	0.002227783203125	\\
0.31979402494784	0.002685546875	\\
0.319838416122875	0.002838134765625	\\
0.319882807297909	0.002410888671875	\\
0.319927198472944	0.0023193359375	\\
0.319971589647978	0.002288818359375	\\
0.320015980823012	0.00244140625	\\
0.320060371998047	0.002410888671875	\\
0.320104763173081	0.00201416015625	\\
0.320149154348116	0.00189208984375	\\
0.32019354552315	0.001983642578125	\\
0.320237936698184	0.0020751953125	\\
0.320282327873219	0.00225830078125	\\
0.320326719048253	0.002044677734375	\\
0.320371110223288	0.00189208984375	\\
0.320415501398322	0.00213623046875	\\
0.320459892573356	0.00201416015625	\\
0.320504283748391	0.00201416015625	\\
0.320548674923425	0.001983642578125	\\
0.32059306609846	0.0013427734375	\\
0.320637457273494	0.001312255859375	\\
0.320681848448528	0.0015869140625	\\
0.320726239623563	0.001190185546875	\\
0.320770630798597	0.00128173828125	\\
0.320815021973632	0.001617431640625	\\
0.320859413148666	0.001739501953125	\\
0.3209038043237	0.001983642578125	\\
0.320948195498735	0.001678466796875	\\
0.320992586673769	0.001495361328125	\\
0.321036977848804	0.001739501953125	\\
0.321081369023838	0.00177001953125	\\
0.321125760198872	0.001861572265625	\\
0.321170151373907	0.001800537109375	\\
0.321214542548941	0.001495361328125	\\
0.321258933723976	0.00189208984375	\\
0.32130332489901	0.00225830078125	\\
0.321347716074045	0.001953125	\\
0.321392107249079	0.001861572265625	\\
0.321436498424113	0.0018310546875	\\
0.321480889599148	0.00213623046875	\\
0.321525280774182	0.0023193359375	\\
0.321569671949216	0.002410888671875	\\
0.321614063124251	0.00177001953125	\\
0.321658454299285	0.00201416015625	\\
0.32170284547432	0.002288818359375	\\
0.321747236649354	0.002166748046875	\\
0.321791627824389	0.002166748046875	\\
0.321836018999423	0.00164794921875	\\
0.321880410174457	0.00201416015625	\\
0.321924801349492	0.00201416015625	\\
0.321969192524526	0.001800537109375	\\
0.322013583699561	0.001800537109375	\\
0.322057974874595	0.001434326171875	\\
0.322102366049629	0.00140380859375	\\
0.322146757224664	0.001678466796875	\\
0.322191148399698	0.00164794921875	\\
0.322235539574733	0.001953125	\\
0.322279930749767	0.001068115234375	\\
0.322324321924801	0.00079345703125	\\
0.322368713099836	0.0013427734375	\\
0.32241310427487	0.0010986328125	\\
0.322457495449905	0.001129150390625	\\
0.322501886624939	0.001373291015625	\\
0.322546277799973	0.001495361328125	\\
0.322590668975008	0.00152587890625	\\
0.322635060150042	0.001678466796875	\\
0.322679451325077	0.00140380859375	\\
0.322723842500111	0.00140380859375	\\
0.322768233675145	0.001434326171875	\\
0.32281262485018	0.000885009765625	\\
0.322857016025214	0.000946044921875	\\
0.322901407200249	0.001007080078125	\\
0.322945798375283	0.00091552734375	\\
0.322990189550317	0.000732421875	\\
0.323034580725352	0.000701904296875	\\
0.323078971900386	0.0006103515625	\\
0.323123363075421	0.001312255859375	\\
0.323167754250455	0.00146484375	\\
0.323212145425489	0.00103759765625	\\
0.323256536600524	0.001068115234375	\\
0.323300927775558	0.00079345703125	\\
0.323345318950593	0.0009765625	\\
0.323389710125627	0.001068115234375	\\
0.323434101300661	0.0010986328125	\\
0.323478492475696	0.001190185546875	\\
0.32352288365073	0.0008544921875	\\
0.323567274825765	0.001220703125	\\
0.323611666000799	0.001129150390625	\\
0.323656057175833	0.0009765625	\\
0.323700448350868	0.00103759765625	\\
0.323744839525902	0.00048828125	\\
0.323789230700937	0.000244140625	\\
0.323833621875971	0.000274658203125	\\
0.323878013051005	0.00030517578125	\\
0.32392240422604	0.0001220703125	\\
0.323966795401074	0.000213623046875	\\
0.324011186576109	0.000457763671875	\\
0.324055577751143	9.1552734375e-05	\\
0.324099968926177	0.000335693359375	\\
0.324144360101212	0.001068115234375	\\
0.324188751276246	0.001190185546875	\\
0.324233142451281	0.00091552734375	\\
0.324277533626315	0.00054931640625	\\
0.324321924801349	0.00079345703125	\\
0.324366315976384	0.00091552734375	\\
0.324410707151418	0.000732421875	\\
0.324455098326453	0.000518798828125	\\
0.324499489501487	6.103515625e-05	\\
0.324543880676522	0.0001220703125	\\
0.324588271851556	3.0517578125e-05	\\
0.32463266302659	-0.000457763671875	\\
0.324677054201625	-9.1552734375e-05	\\
0.324721445376659	-0.0003662109375	\\
0.324765836551694	-0.000762939453125	\\
0.324810227726728	-0.000396728515625	\\
0.324854618901762	-0.000701904296875	\\
0.324899010076797	-0.000701904296875	\\
0.324943401251831	-0.000762939453125	\\
0.324987792426866	-0.00030517578125	\\
0.3250321836019	-0.000701904296875	\\
0.325076574776934	-0.0013427734375	\\
0.325120965951969	-0.000579833984375	\\
0.325165357127003	-0.00054931640625	\\
0.325209748302038	-0.00018310546875	\\
0.325254139477072	0	\\
0.325298530652106	-0.000244140625	\\
0.325342921827141	-3.0517578125e-05	\\
0.325387313002175	0.00018310546875	\\
0.32543170417721	0.000396728515625	\\
0.325476095352244	0.000152587890625	\\
0.325520486527278	0.0006103515625	\\
0.325564877702313	0.00103759765625	\\
0.325609268877347	0.00030517578125	\\
0.325653660052382	0.000640869140625	\\
0.325698051227416	0.001068115234375	\\
0.32574244240245	0.000885009765625	\\
0.325786833577485	0.0009765625	\\
0.325831224752519	0.000885009765625	\\
0.325875615927554	0.000396728515625	\\
0.325920007102588	0.000640869140625	\\
0.325964398277622	0.000701904296875	\\
0.326008789452657	0.000335693359375	\\
0.326053180627691	0.00030517578125	\\
0.326097571802726	0.000213623046875	\\
0.32614196297776	0.00067138671875	\\
0.326186354152794	0.001007080078125	\\
0.326230745327829	0.000823974609375	\\
0.326275136502863	0.00079345703125	\\
0.326319527677898	0.0008544921875	\\
0.326363918852932	0.000885009765625	\\
0.326408310027966	0.000640869140625	\\
0.326452701203001	0.0009765625	\\
0.326497092378035	0.001068115234375	\\
0.32654148355307	0.00103759765625	\\
0.326585874728104	0.001068115234375	\\
0.326630265903138	0.00067138671875	\\
0.326674657078173	0.00067138671875	\\
0.326719048253207	0.00018310546875	\\
0.326763439428242	0.000244140625	\\
0.326807830603276	0.000701904296875	\\
0.32685222177831	0.000640869140625	\\
0.326896612953345	0.000701904296875	\\
0.326941004128379	0.000885009765625	\\
0.326985395303414	0.0013427734375	\\
0.327029786478448	0.001617431640625	\\
0.327074177653483	0.001556396484375	\\
0.327118568828517	0.00146484375	\\
0.327162960003551	0.00177001953125	\\
0.327207351178586	0.001800537109375	\\
0.32725174235362	0.00152587890625	\\
0.327296133528655	0.001312255859375	\\
0.327340524703689	0.0010986328125	\\
0.327384915878723	0.000701904296875	\\
0.327429307053758	0.00103759765625	\\
0.327473698228792	0.001251220703125	\\
0.327518089403827	0.001312255859375	\\
0.327562480578861	0.00164794921875	\\
0.327606871753895	0.001678466796875	\\
0.32765126292893	0.00164794921875	\\
0.327695654103964	0.001739501953125	\\
0.327740045278999	0.001556396484375	\\
0.327784436454033	0.001678466796875	\\
0.327828827629067	0.001953125	\\
0.327873218804102	0.001739501953125	\\
0.327917609979136	0.001708984375	\\
0.327962001154171	0.001922607421875	\\
0.328006392329205	0.001678466796875	\\
0.328050783504239	0.001922607421875	\\
0.328095174679274	0.002044677734375	\\
0.328139565854308	0.001617431640625	\\
0.328183957029343	0.001678466796875	\\
0.328228348204377	0.00146484375	\\
0.328272739379411	0.00152587890625	\\
0.328317130554446	0.001708984375	\\
0.32836152172948	0.001495361328125	\\
0.328405912904515	0.001556396484375	\\
0.328450304079549	0.001617431640625	\\
0.328494695254583	0.001495361328125	\\
0.328539086429618	0.0010986328125	\\
0.328583477604652	0.000946044921875	\\
0.328627868779687	0.000885009765625	\\
0.328672259954721	0.00067138671875	\\
0.328716651129755	0.0008544921875	\\
0.32876104230479	0.000946044921875	\\
0.328805433479824	0.0006103515625	\\
0.328849824654859	0.000762939453125	\\
0.328894215829893	0.00018310546875	\\
0.328938607004927	0.000244140625	\\
0.328982998179962	0.000732421875	\\
0.329027389354996	0.0006103515625	\\
0.329071780530031	0.0006103515625	\\
0.329116171705065	0.00018310546875	\\
0.329160562880099	0.000335693359375	\\
0.329204954055134	0.000396728515625	\\
0.329249345230168	0.000244140625	\\
0.329293736405203	0.000396728515625	\\
0.329338127580237	0.000946044921875	\\
0.329382518755271	0.00152587890625	\\
0.329426909930306	0.00128173828125	\\
0.32947130110534	0.001190185546875	\\
0.329515692280375	0.00152587890625	\\
0.329560083455409	0.00152587890625	\\
0.329604474630443	0.001617431640625	\\
0.329648865805478	0.00128173828125	\\
0.329693256980512	0.0008544921875	\\
0.329737648155547	0.000823974609375	\\
0.329782039330581	0.000701904296875	\\
0.329826430505616	0.000823974609375	\\
0.32987082168065	0.000701904296875	\\
0.329915212855684	0.0003662109375	\\
0.329959604030719	0.000274658203125	\\
0.330003995205753	0.000579833984375	\\
0.330048386380787	0.0009765625	\\
0.330092777555822	0.000885009765625	\\
0.330137168730856	0.000885009765625	\\
0.330181559905891	0.000732421875	\\
0.330225951080925	0.0006103515625	\\
0.33027034225596	0.000732421875	\\
0.330314733430994	0.000640869140625	\\
0.330359124606028	0.000396728515625	\\
0.330403515781063	3.0517578125e-05	\\
0.330447906956097	-0.000457763671875	\\
0.330492298131132	-0.000885009765625	\\
0.330536689306166	-0.0008544921875	\\
0.3305810804812	-0.00018310546875	\\
0.330625471656235	-0.00018310546875	\\
0.330669862831269	-0.0006103515625	\\
0.330714254006304	-0.00067138671875	\\
0.330758645181338	-0.00079345703125	\\
0.330803036356372	-0.001007080078125	\\
0.330847427531407	-0.00067138671875	\\
0.330891818706441	-0.000823974609375	\\
0.330936209881476	-0.0015869140625	\\
0.33098060105651	-0.00115966796875	\\
0.331024992231544	-0.001251220703125	\\
0.331069383406579	-0.001922607421875	\\
0.331113774581613	-0.002044677734375	\\
0.331158165756648	-0.00177001953125	\\
0.331202556931682	-0.001922607421875	\\
0.331246948106716	-0.00213623046875	\\
0.331291339281751	-0.002227783203125	\\
0.331335730456785	-0.00238037109375	\\
0.33138012163182	-0.002227783203125	\\
0.331424512806854	-0.0020751953125	\\
0.331468903981888	-0.002044677734375	\\
0.331513295156923	-0.00225830078125	\\
0.331557686331957	-0.002685546875	\\
0.331602077506992	-0.00262451171875	\\
0.331646468682026	-0.002197265625	\\
0.33169085985706	-0.00225830078125	\\
0.331735251032095	-0.0029296875	\\
0.331779642207129	-0.00286865234375	\\
0.331824033382164	-0.00225830078125	\\
0.331868424557198	-0.002532958984375	\\
0.331912815732232	-0.00250244140625	\\
0.331957206907267	-0.002960205078125	\\
0.332001598082301	-0.0032958984375	\\
0.332045989257336	-0.002532958984375	\\
0.33209038043237	-0.003082275390625	\\
0.332134771607404	-0.002593994140625	\\
0.332179162782439	-0.002288818359375	\\
0.332223553957473	-0.00299072265625	\\
0.332267945132508	-0.002593994140625	\\
0.332312336307542	-0.002960205078125	\\
0.332356727482576	-0.003204345703125	\\
0.332401118657611	-0.0035400390625	\\
0.332445509832645	-0.00396728515625	\\
0.33248990100768	-0.0040283203125	\\
0.332534292182714	-0.004241943359375	\\
0.332578683357748	-0.004150390625	\\
0.332623074532783	-0.00396728515625	\\
0.332667465707817	-0.004150390625	\\
0.332711856882852	-0.0042724609375	\\
0.332756248057886	-0.004547119140625	\\
0.33280063923292	-0.004547119140625	\\
0.332845030407955	-0.004119873046875	\\
0.332889421582989	-0.004486083984375	\\
0.332933812758024	-0.0042724609375	\\
0.332978203933058	-0.004425048828125	\\
0.333022595108093	-0.004058837890625	\\
0.333066986283127	-0.003662109375	\\
0.333111377458161	-0.003631591796875	\\
0.333155768633196	-0.00311279296875	\\
0.33320015980823	-0.00335693359375	\\
0.333244550983265	-0.00360107421875	\\
0.333288942158299	-0.00341796875	\\
0.333333333333333	-0.003204345703125	\\
0.333377724508368	-0.00335693359375	\\
0.333422115683402	-0.00335693359375	\\
0.333466506858437	-0.003173828125	\\
0.333510898033471	-0.0028076171875	\\
0.333555289208505	-0.00311279296875	\\
0.33359968038354	-0.00360107421875	\\
0.333644071558574	-0.003143310546875	\\
0.333688462733609	-0.00311279296875	\\
0.333732853908643	-0.00323486328125	\\
0.333777245083677	-0.003265380859375	\\
0.333821636258712	-0.0028076171875	\\
0.333866027433746	-0.002471923828125	\\
0.333910418608781	-0.00238037109375	\\
0.333954809783815	-0.002410888671875	\\
0.333999200958849	-0.002532958984375	\\
0.334043592133884	-0.00244140625	\\
0.334087983308918	-0.00238037109375	\\
0.334132374483953	-0.002197265625	\\
0.334176765658987	-0.00250244140625	\\
0.334221156834021	-0.00274658203125	\\
0.334265548009056	-0.001953125	\\
0.33430993918409	-0.002349853515625	\\
0.334354330359125	-0.002838134765625	\\
0.334398721534159	-0.00250244140625	\\
0.334443112709193	-0.002685546875	\\
0.334487503884228	-0.0025634765625	\\
0.334531895059262	-0.0025634765625	\\
0.334576286234297	-0.002655029296875	\\
0.334620677409331	-0.002410888671875	\\
0.334665068584365	-0.00225830078125	\\
0.3347094597594	-0.002410888671875	\\
0.334753850934434	-0.002410888671875	\\
0.334798242109469	-0.002105712890625	\\
0.334842633284503	-0.0020751953125	\\
0.334887024459537	-0.002532958984375	\\
0.334931415634572	-0.00238037109375	\\
0.334975806809606	-0.002044677734375	\\
0.335020197984641	-0.002288818359375	\\
0.335064589159675	-0.002655029296875	\\
0.335108980334709	-0.002197265625	\\
0.335153371509744	-0.002227783203125	\\
0.335197762684778	-0.002166748046875	\\
0.335242153859813	-0.001708984375	\\
0.335286545034847	-0.001953125	\\
0.335330936209881	-0.00201416015625	\\
0.335375327384916	-0.0018310546875	\\
0.33541971855995	-0.0013427734375	\\
0.335464109734985	-0.000885009765625	\\
0.335508500910019	-0.000762939453125	\\
0.335552892085054	-0.0009765625	\\
0.335597283260088	-0.001068115234375	\\
0.335641674435122	-0.00048828125	\\
0.335686065610157	-0.000457763671875	\\
0.335730456785191	-0.00054931640625	\\
0.335774847960226	-0.000457763671875	\\
0.33581923913526	-0.000152587890625	\\
0.335863630310294	9.1552734375e-05	\\
0.335908021485329	-9.1552734375e-05	\\
0.335952412660363	3.0517578125e-05	\\
0.335996803835398	-0.00030517578125	\\
0.336041195010432	-0.000396728515625	\\
0.336085586185466	-0.000244140625	\\
0.336129977360501	-0.000152587890625	\\
0.336174368535535	-0.000244140625	\\
0.33621875971057	-0.00054931640625	\\
0.336263150885604	-0.00042724609375	\\
0.336307542060638	-0.000518798828125	\\
0.336351933235673	-0.000640869140625	\\
0.336396324410707	-0.00054931640625	\\
0.336440715585742	-0.0003662109375	\\
0.336485106760776	-0.00067138671875	\\
0.33652949793581	-0.000885009765625	\\
0.336573889110845	-0.001068115234375	\\
0.336618280285879	-0.0010986328125	\\
0.336662671460914	-0.001190185546875	\\
0.336707062635948	-0.00140380859375	\\
0.336751453810982	-0.001434326171875	\\
0.336795844986017	-0.00079345703125	\\
0.336840236161051	-0.000518798828125	\\
0.336884627336086	-0.000885009765625	\\
0.33692901851112	-0.001007080078125	\\
0.336973409686154	-0.00054931640625	\\
0.337017800861189	-0.0003662109375	\\
0.337062192036223	-0.00067138671875	\\
0.337106583211258	-0.000762939453125	\\
0.337150974386292	-0.00042724609375	\\
0.337195365561326	-0.00042724609375	\\
0.337239756736361	-0.00091552734375	\\
0.337284147911395	-0.00054931640625	\\
0.33732853908643	-0.0003662109375	\\
0.337372930261464	9.1552734375e-05	\\
0.337417321436498	0.00042724609375	\\
0.337461712611533	0.000274658203125	\\
0.337506103786567	-6.103515625e-05	\\
0.337550494961602	0	\\
0.337594886136636	0.000640869140625	\\
0.33763927731167	0.001007080078125	\\
0.337683668486705	0.00079345703125	\\
0.337728059661739	0.000640869140625	\\
0.337772450836774	0.0010986328125	\\
0.337816842011808	0.000885009765625	\\
0.337861233186842	0.000701904296875	\\
0.337905624361877	0.001068115234375	\\
0.337950015536911	0.00091552734375	\\
0.337994406711946	0.0006103515625	\\
0.33803879788698	0.000762939453125	\\
0.338083189062014	0.000885009765625	\\
0.338127580237049	0.00054931640625	\\
0.338171971412083	0.00018310546875	\\
0.338216362587118	0.00030517578125	\\
0.338260753762152	0.000518798828125	\\
0.338305144937187	0.000152587890625	\\
0.338349536112221	0.00018310546875	\\
0.338393927287255	0.00018310546875	\\
0.33843831846229	0.00018310546875	\\
0.338482709637324	-6.103515625e-05	\\
0.338527100812358	-0.00054931640625	\\
0.338571491987393	-0.000335693359375	\\
0.338615883162427	-6.103515625e-05	\\
0.338660274337462	-0.000701904296875	\\
0.338704665512496	-0.000823974609375	\\
0.338749056687531	-0.00091552734375	\\
0.338793447862565	-0.00103759765625	\\
0.338837839037599	-0.0010986328125	\\
0.338882230212634	-0.000732421875	\\
0.338926621387668	-0.000823974609375	\\
0.338971012562703	-0.0009765625	\\
0.339015403737737	-0.000946044921875	\\
0.339059794912771	-0.00091552734375	\\
0.339104186087806	-0.0008544921875	\\
0.33914857726284	-0.0010986328125	\\
0.339192968437875	-0.001190185546875	\\
0.339237359612909	-0.001190185546875	\\
0.339281750787943	-0.00103759765625	\\
0.339326141962978	-0.001251220703125	\\
0.339370533138012	-0.0013427734375	\\
0.339414924313047	-0.00140380859375	\\
0.339459315488081	-0.001434326171875	\\
0.339503706663115	-0.0015869140625	\\
0.33954809783815	-0.00140380859375	\\
0.339592489013184	-0.00128173828125	\\
0.339636880188219	-0.0013427734375	\\
0.339681271363253	-0.00091552734375	\\
0.339725662538287	-0.0006103515625	\\
0.339770053713322	-0.00054931640625	\\
0.339814444888356	-0.00067138671875	\\
0.339858836063391	-0.0008544921875	\\
0.339903227238425	-0.001007080078125	\\
0.339947618413459	-0.00128173828125	\\
0.339992009588494	-0.001373291015625	\\
0.340036400763528	-0.001678466796875	\\
0.340080791938563	-0.001617431640625	\\
0.340125183113597	-0.00146484375	\\
0.340169574288631	-0.00177001953125	\\
0.340213965463666	-0.001953125	\\
0.3402583566387	-0.002105712890625	\\
0.340302747813735	-0.00201416015625	\\
0.340347138988769	-0.001251220703125	\\
0.340391530163803	-0.001373291015625	\\
0.340435921338838	-0.00146484375	\\
0.340480312513872	-0.001373291015625	\\
0.340524703688907	-0.001617431640625	\\
0.340569094863941	-0.001312255859375	\\
0.340613486038975	-0.0013427734375	\\
0.34065787721401	-0.001556396484375	\\
0.340702268389044	-0.001739501953125	\\
0.340746659564079	-0.001556396484375	\\
0.340791050739113	-0.001556396484375	\\
0.340835441914147	-0.001556396484375	\\
0.340879833089182	-0.0010986328125	\\
0.340924224264216	-0.001190185546875	\\
0.340968615439251	-0.00164794921875	\\
0.341013006614285	-0.001922607421875	\\
0.341057397789319	-0.001953125	\\
0.341101788964354	-0.001983642578125	\\
0.341146180139388	-0.0023193359375	\\
0.341190571314423	-0.002105712890625	\\
0.341234962489457	-0.001556396484375	\\
0.341279353664492	-0.001556396484375	\\
0.341323744839526	-0.00164794921875	\\
0.34136813601456	-0.001312255859375	\\
0.341412527189595	-0.001373291015625	\\
0.341456918364629	-0.00128173828125	\\
0.341501309539664	-0.000823974609375	\\
0.341545700714698	-0.00067138671875	\\
0.341590091889732	-0.000579833984375	\\
0.341634483064767	-0.00067138671875	\\
0.341678874239801	-0.001190185546875	\\
0.341723265414836	-0.00048828125	\\
0.34176765658987	-0.000213623046875	\\
0.341812047764904	-0.00030517578125	\\
0.341856438939939	-0.000274658203125	\\
0.341900830114973	-0.0008544921875	\\
0.341945221290008	-0.000946044921875	\\
0.341989612465042	-0.000701904296875	\\
0.342034003640076	-0.0003662109375	\\
0.342078394815111	-0.00018310546875	\\
0.342122785990145	-0.00030517578125	\\
0.34216717716518	6.103515625e-05	\\
0.342211568340214	0.00054931640625	\\
0.342255959515248	0.0006103515625	\\
0.342300350690283	0.0001220703125	\\
0.342344741865317	-0.00018310546875	\\
0.342389133040352	9.1552734375e-05	\\
0.342433524215386	-6.103515625e-05	\\
0.34247791539042	-0.000579833984375	\\
0.342522306565455	-0.00018310546875	\\
0.342566697740489	0	\\
0.342611088915524	0.0001220703125	\\
0.342655480090558	0.000946044921875	\\
0.342699871265592	0.00103759765625	\\
0.342744262440627	0.001068115234375	\\
0.342788653615661	0.000701904296875	\\
0.342833044790696	9.1552734375e-05	\\
0.34287743596573	-0.000244140625	\\
0.342921827140764	-0.000823974609375	\\
0.342966218315799	-0.000823974609375	\\
0.343010609490833	-0.000823974609375	\\
0.343055000665868	-0.00042724609375	\\
0.343099391840902	0.00048828125	\\
0.343143783015936	0.001312255859375	\\
0.343188174190971	0.001708984375	\\
0.343232565366005	0.00164794921875	\\
0.34327695654104	0.001220703125	\\
0.343321347716074	0.0003662109375	\\
0.343365738891108	-0.00048828125	\\
0.343410130066143	-0.001007080078125	\\
0.343454521241177	-0.001373291015625	\\
0.343498912416212	-0.0010986328125	\\
0.343543303591246	0.000213623046875	\\
0.34358769476628	0.001495361328125	\\
0.343632085941315	0.003082275390625	\\
0.343676477116349	0.003936767578125	\\
0.343720868291384	0.00390625	\\
0.343765259466418	0.003448486328125	\\
0.343809650641452	0.0025634765625	\\
0.343854041816487	0.000823974609375	\\
0.343898432991521	-6.103515625e-05	\\
0.343942824166556	-0.0006103515625	\\
0.34398721534159	-0.001312255859375	\\
0.344031606516625	6.103515625e-05	\\
0.344075997691659	0.001800537109375	\\
0.344120388866693	0.003936767578125	\\
0.344164780041728	0.006103515625	\\
0.344209171216762	0.007232666015625	\\
0.344253562391797	0.0067138671875	\\
0.344297953566831	0.00469970703125	\\
0.344342344741865	0.00213623046875	\\
0.3443867359169	-0.00067138671875	\\
0.344431127091934	-0.0028076171875	\\
0.344475518266969	-0.00323486328125	\\
0.344519909442003	-0.00244140625	\\
0.344564300617037	0.0001220703125	\\
0.344608691792072	0.00384521484375	\\
0.344653082967106	0.007476806640625	\\
0.344697474142141	0.01019287109375	\\
0.344741865317175	0.010345458984375	\\
0.344786256492209	0.0084228515625	\\
0.344830647667244	0.0047607421875	\\
0.344875038842278	-9.1552734375e-05	\\
0.344919430017313	-0.00433349609375	\\
0.344963821192347	-0.007049560546875	\\
0.345008212367381	-0.006744384765625	\\
0.345052603542416	-0.003204345703125	\\
0.34509699471745	0.00274658203125	\\
0.345141385892485	0.009124755859375	\\
0.345185777067519	0.014373779296875	\\
0.345230168242553	0.016845703125	\\
0.345274559417588	0.01483154296875	\\
0.345318950592622	0.0086669921875	\\
0.345363341767657	0.00103759765625	\\
0.345407732942691	-0.006988525390625	\\
0.345452124117725	-0.013458251953125	\\
0.34549651529276	-0.014739990234375	\\
0.345540906467794	-0.010711669921875	\\
0.345585297642829	-0.002685546875	\\
0.345629688817863	0.0078125	\\
0.345674079992897	0.017913818359375	\\
0.345718471167932	0.0238037109375	\\
0.345762862342966	0.0230712890625	\\
0.345807253518001	0.01611328125	\\
0.345851644693035	0.00482177734375	\\
0.345896035868069	-0.008056640625	\\
0.345940427043104	-0.019378662109375	\\
0.345984818218138	-0.02545166015625	\\
0.346029209393173	-0.02264404296875	\\
0.346073600568207	-0.011810302734375	\\
0.346117991743241	0.003509521484375	\\
0.346162382918276	0.020355224609375	\\
0.34620677409331	0.033477783203125	\\
0.346251165268345	0.036773681640625	\\
0.346295556443379	0.029388427734375	\\
0.346339947618413	0.0140380859375	\\
0.346384338793448	-0.00518798828125	\\
0.346428729968482	-0.02496337890625	\\
0.346473121143517	-0.0384521484375	\\
0.346517512318551	-0.039703369140625	\\
0.346561903493585	-0.02825927734375	\\
0.34660629466862	-0.00665283203125	\\
0.346650685843654	0.0194091796875	\\
0.346695077018689	0.04254150390625	\\
0.346739468193723	0.0550537109375	\\
0.346783859368758	0.0516357421875	\\
0.346828250543792	0.03314208984375	\\
0.346872641718826	0.00311279296875	\\
0.346917032893861	-0.029296875	\\
0.346961424068895	-0.05462646484375	\\
0.347005815243929	-0.06524658203125	\\
0.347050206418964	-0.055572509765625	\\
0.347094597593998	-0.027435302734375	\\
0.347138988769033	0.012359619140625	\\
0.347183379944067	0.05279541015625	\\
0.347227771119102	0.080413818359375	\\
0.347272162294136	0.085784912109375	\\
0.34731655346917	0.065826416015625	\\
0.347360944644205	0.023712158203125	\\
0.347405335819239	-0.0289306640625	\\
0.347449726994274	-0.0765380859375	\\
0.347494118169308	-0.105316162109375	\\
0.347538509344342	-0.1033935546875	\\
0.347582900519377	-0.068634033203125	\\
0.347627291694411	-0.00811767578125	\\
0.347671682869446	0.06146240234375	\\
0.34771607404448	0.119171142578125	\\
0.347760465219514	0.1461181640625	\\
0.347804856394549	0.12945556640625	\\
0.347849247569583	0.069732666015625	\\
0.347893638744618	-0.018707275390625	\\
0.347938029919652	-0.113037109375	\\
0.347982421094686	-0.18499755859375	\\
0.348026812269721	-0.208160400390625	\\
0.348071203444755	-0.166534423828125	\\
0.34811559461979	-0.064605712890625	\\
0.348159985794824	0.075439453125	\\
0.348204376969858	0.2176513671875	\\
0.348248768144893	0.314666748046875	\\
0.348293159319927	0.325653076171875	\\
0.348337550494962	0.2291259765625	\\
0.348381941669996	0.01434326171875	\\
0.34842633284503	-0.2830810546875	\\
0.348470724020065	-0.633636474609375	\\
0.348515115195099	-0.9404296875	\\
0.348559506370134	-0.99749755859375	\\
0.348603897545168	-0.99041748046875	\\
0.348648288720202	-0.964996337890625	\\
0.348692679895237	-0.71075439453125	\\
0.348737071070271	-0.27752685546875	\\
0.348781462245306	0.229461669921875	\\
0.34882585342034	0.71661376953125	\\
0.348870244595374	0.980621337890625	\\
0.348914635770409	0.994598388671875	\\
0.348959026945443	0.9842529296875	\\
0.349003418120478	0.999969482421875	\\
0.349047809295512	0.913482666015625	\\
0.349092200470546	0.61065673828125	\\
0.349136591645581	0.280487060546875	\\
0.349180982820615	0.017578125	\\
0.34922537399565	-0.128326416015625	\\
0.349269765170684	-0.13690185546875	\\
0.349314156345719	-0.03759765625	\\
0.349358547520753	0.12152099609375	\\
0.349402938695787	0.28717041015625	\\
0.349447329870822	0.40869140625	\\
0.349491721045856	0.439544677734375	\\
0.34953611222089	0.358734130859375	\\
0.349580503395925	0.17633056640625	\\
0.349624894570959	-0.071685791015625	\\
0.349669285745994	-0.326873779296875	\\
0.349713676921028	-0.541473388671875	\\
0.349758068096063	-0.684173583984375	\\
0.349802459271097	-0.73321533203125	\\
0.349846850446131	-0.685699462890625	\\
0.349891241621166	-0.56390380859375	\\
0.3499356327962	-0.4063720703125	\\
0.349980023971235	-0.25457763671875	\\
0.350024415146269	-0.14227294921875	\\
0.350068806321303	-0.086273193359375	\\
0.350113197496338	-0.09051513671875	\\
0.350157588671372	-0.14501953125	\\
0.350201979846407	-0.224822998046875	\\
0.350246371021441	-0.305084228515625	\\
0.350290762196475	-0.3643798828125	\\
0.35033515337151	-0.383270263671875	\\
0.350379544546544	-0.35101318359375	\\
0.350423935721579	-0.27313232421875	\\
0.350468326896613	-0.16461181640625	\\
0.350512718071647	-0.043792724609375	\\
0.350557109246682	0.071624755859375	\\
0.350601500421716	0.16119384765625	\\
0.350645891596751	0.21466064453125	\\
0.350690282771785	0.237945556640625	\\
0.350734673946819	0.233673095703125	\\
0.350779065121854	0.214569091796875	\\
0.350823456296888	0.1903076171875	\\
0.350867847471923	0.168304443359375	\\
0.350912238646957	0.149871826171875	\\
0.350956629821991	0.140960693359375	\\
0.351001020997026	0.142181396484375	\\
0.35104541217206	0.149200439453125	\\
0.351089803347095	0.15924072265625	\\
0.351134194522129	0.1719970703125	\\
0.351178585697163	0.18304443359375	\\
0.351222976872198	0.188446044921875	\\
0.351267368047232	0.18524169921875	\\
0.351311759222267	0.17437744140625	\\
0.351356150397301	0.1536865234375	\\
0.351400541572335	0.129058837890625	\\
0.35144493274737	0.10284423828125	\\
0.351489323922404	0.066497802734375	\\
0.351533715097439	0.0228271484375	\\
0.351578106272473	-0.021820068359375	\\
0.351622497447507	-0.067596435546875	\\
0.351666888622542	-0.107513427734375	\\
0.351711279797576	-0.13775634765625	\\
0.351755670972611	-0.156768798828125	\\
0.351800062147645	-0.160369873046875	\\
0.351844453322679	-0.150634765625	\\
0.351888844497714	-0.13214111328125	\\
0.351933235672748	-0.11236572265625	\\
0.351977626847783	-0.09991455078125	\\
0.352022018022817	-0.09442138671875	\\
0.352066409197851	-0.095367431640625	\\
0.352110800372886	-0.102935791015625	\\
0.35215519154792	-0.11572265625	\\
0.352199582722955	-0.1275634765625	\\
0.352243973897989	-0.132781982421875	\\
0.352288365073023	-0.13232421875	\\
0.352332756248058	-0.123931884765625	\\
0.352377147423092	-0.10589599609375	\\
0.352421538598127	-0.081573486328125	\\
0.352465929773161	-0.0552978515625	\\
0.352510320948196	-0.02581787109375	\\
0.35255471212323	0.003997802734375	\\
0.352599103298264	0.0311279296875	\\
0.352643494473299	0.0570068359375	\\
0.352687885648333	0.080474853515625	\\
0.352732276823368	0.0987548828125	\\
0.352776667998402	0.111236572265625	\\
0.352821059173436	0.11920166015625	\\
0.352865450348471	0.12188720703125	\\
0.352909841523505	0.11871337890625	\\
0.35295423269854	0.110870361328125	\\
0.352998623873574	0.10076904296875	\\
0.353043015048608	0.08990478515625	\\
0.353087406223643	0.080352783203125	\\
0.353131797398677	0.075469970703125	\\
0.353176188573712	0.07806396484375	\\
0.353220579748746	0.08624267578125	\\
0.35326497092378	0.098968505859375	\\
0.353309362098815	0.11083984375	\\
0.353353753273849	0.117828369140625	\\
0.353398144448884	0.116241455078125	\\
0.353442535623918	0.102325439453125	\\
0.353486926798952	0.07757568359375	\\
0.353531317973987	0.041290283203125	\\
0.353575709149021	0.00299072265625	\\
0.353620100324056	-0.032073974609375	\\
0.35366449149909	-0.0631103515625	\\
0.353708882674124	-0.0811767578125	\\
0.353753273849159	-0.089508056640625	\\
0.353797665024193	-0.089599609375	\\
0.353842056199228	-0.07958984375	\\
0.353886447374262	-0.066436767578125	\\
0.353930838549296	-0.054595947265625	\\
0.353975229724331	-0.04364013671875	\\
0.354019620899365	-0.039154052734375	\\
0.3540640120744	-0.040374755859375	\\
0.354108403249434	-0.046539306640625	\\
0.354152794424468	-0.05767822265625	\\
0.354197185599503	-0.0694580078125	\\
0.354241576774537	-0.078460693359375	\\
0.354285967949572	-0.082061767578125	\\
0.354330359124606	-0.076751708984375	\\
0.35437475029964	-0.06414794921875	\\
0.354419141474675	-0.047607421875	\\
0.354463532649709	-0.029937744140625	\\
0.354507923824744	-0.015655517578125	\\
0.354552314999778	-0.003814697265625	\\
0.354596706174812	0.0048828125	\\
0.354641097349847	0.008087158203125	\\
0.354685488524881	0.007232666015625	\\
0.354729879699916	0.003143310546875	\\
0.35477427087495	-0.00128173828125	\\
0.354818662049984	-0.004730224609375	\\
0.354863053225019	-0.0067138671875	\\
0.354907444400053	-0.007049560546875	\\
0.354951835575088	-0.00653076171875	\\
0.354996226750122	-0.00482177734375	\\
0.355040617925156	-0.001373291015625	\\
0.355085009100191	0.00628662109375	\\
0.355129400275225	0.014984130859375	\\
};
\addplot [color=blue,solid,forget plot]
  table[row sep=crcr]{
0.355129400275225	0.014984130859375	\\
0.35517379145026	0.02154541015625	\\
0.355218182625294	0.029815673828125	\\
0.355262573800329	0.03759765625	\\
0.355306964975363	0.04083251953125	\\
0.355351356150397	0.044677734375	\\
0.355395747325432	0.04339599609375	\\
0.355440138500466	0.039306640625	\\
0.355484529675501	0.0362548828125	\\
0.355528920850535	0.028228759765625	\\
0.355573312025569	0.020477294921875	\\
0.355617703200604	0.014251708984375	\\
0.355662094375638	0.009429931640625	\\
0.355706485550673	0.009674072265625	\\
0.355750876725707	0.011749267578125	\\
0.355795267900741	0.015380859375	\\
0.355839659075776	0.021240234375	\\
0.35588405025081	0.02301025390625	\\
0.355928441425845	0.021331787109375	\\
0.355972832600879	0.01605224609375	\\
0.356017223775913	0.004913330078125	\\
0.356061614950948	-0.008819580078125	\\
0.356106006125982	-0.02020263671875	\\
0.356150397301017	-0.03045654296875	\\
0.356194788476051	-0.035400390625	\\
0.356239179651085	-0.0352783203125	\\
0.35628357082612	-0.03369140625	\\
0.356327962001154	-0.0257568359375	\\
0.356372353176189	-0.01666259765625	\\
0.356416744351223	-0.010009765625	\\
0.356461135526257	-0.004669189453125	\\
0.356505526701292	-0.00469970703125	\\
0.356549917876326	-0.00946044921875	\\
0.356594309051361	-0.016571044921875	\\
0.356638700226395	-0.023895263671875	\\
0.356683091401429	-0.03167724609375	\\
0.356727482576464	-0.03839111328125	\\
0.356771873751498	-0.03961181640625	\\
0.356816264926533	-0.0345458984375	\\
0.356860656101567	-0.0233154296875	\\
0.356905047276601	-0.009674072265625	\\
0.356949438451636	0.005218505859375	\\
0.35699382962667	0.0185546875	\\
0.357038220801705	0.029449462890625	\\
0.357082611976739	0.03582763671875	\\
0.357127003151773	0.035247802734375	\\
0.357171394326808	0.03143310546875	\\
0.357215785501842	0.024139404296875	\\
0.357260176676877	0.015228271484375	\\
0.357304567851911	0.009765625	\\
0.357348959026945	0.00408935546875	\\
0.35739335020198	0.00262451171875	\\
0.357437741377014	0.0054931640625	\\
0.357482132552049	0.007598876953125	\\
0.357526523727083	0.013153076171875	\\
0.357570914902117	0.01788330078125	\\
0.357615306077152	0.0179443359375	\\
0.357659697252186	0.01593017578125	\\
0.357704088427221	0.01263427734375	\\
0.357748479602255	0.005157470703125	\\
0.35779287077729	-0.00250244140625	\\
0.357837261952324	-0.009185791015625	\\
0.357881653127358	-0.013641357421875	\\
0.357926044302393	-0.016143798828125	\\
0.357970435477427	-0.01519775390625	\\
0.358014826652461	-0.01055908203125	\\
0.358059217827496	-0.003448486328125	\\
0.35810360900253	0.00262451171875	\\
0.358148000177565	0.005859375	\\
0.358192391352599	0.00677490234375	\\
0.358236782527634	0.004058837890625	\\
0.358281173702668	-0.001068115234375	\\
0.358325564877702	-0.009796142578125	\\
0.358369956052737	-0.0194091796875	\\
0.358414347227771	-0.0286865234375	\\
0.358458738402806	-0.036102294921875	\\
0.35850312957784	-0.0391845703125	\\
0.358547520752874	-0.0396728515625	\\
0.358591911927909	-0.036529541015625	\\
0.358636303102943	-0.0294189453125	\\
0.358680694277978	-0.020843505859375	\\
0.358725085453012	-0.011566162109375	\\
0.358769476628046	-0.004669189453125	\\
0.358813867803081	-0.001190185546875	\\
0.358858258978115	-0.00042724609375	\\
0.35890265015315	-0.000701904296875	\\
0.358947041328184	-0.00384521484375	\\
0.358991432503218	-0.00701904296875	\\
0.359035823678253	-0.0084228515625	\\
0.359080214853287	-0.010223388671875	\\
0.359124606028322	-0.009490966796875	\\
0.359168997203356	-0.0072021484375	\\
0.35921338837839	-0.0040283203125	\\
0.359257779553425	0.001708984375	\\
0.359302170728459	0.00634765625	\\
0.359346561903494	0.00885009765625	\\
0.359390953078528	0.0113525390625	\\
0.359435344253562	0.01165771484375	\\
0.359479735428597	0.01141357421875	\\
0.359524126603631	0.011444091796875	\\
0.359568517778666	0.00933837890625	\\
0.3596129089537	0.009063720703125	\\
0.359657300128734	0.009979248046875	\\
0.359701691303769	0.010955810546875	\\
0.359746082478803	0.014556884765625	\\
0.359790473653838	0.0196533203125	\\
0.359834864828872	0.023590087890625	\\
0.359879256003906	0.02728271484375	\\
0.359923647178941	0.02984619140625	\\
0.359968038353975	0.028350830078125	\\
0.36001242952901	0.025634765625	\\
0.360056820704044	0.020538330078125	\\
0.360101211879078	0.013031005859375	\\
0.360145603054113	0.006256103515625	\\
0.360189994229147	-0.002197265625	\\
0.360234385404182	-0.010498046875	\\
0.360278776579216	-0.015838623046875	\\
0.36032316775425	-0.019500732421875	\\
0.360367558929285	-0.021087646484375	\\
0.360411950104319	-0.019287109375	\\
0.360456341279354	-0.015411376953125	\\
0.360500732454388	-0.010711669921875	\\
0.360545123629422	-0.006317138671875	\\
0.360589514804457	-0.0037841796875	\\
0.360633905979491	-0.002593994140625	\\
0.360678297154526	-0.0032958984375	\\
0.36072268832956	-0.007659912109375	\\
0.360767079504594	-0.01275634765625	\\
0.360811470679629	-0.01873779296875	\\
0.360855861854663	-0.02398681640625	\\
0.360900253029698	-0.025665283203125	\\
0.360944644204732	-0.02447509765625	\\
0.360989035379767	-0.020538330078125	\\
0.361033426554801	-0.014068603515625	\\
0.361077817729835	-0.005828857421875	\\
0.36112220890487	0.002349853515625	\\
0.361166600079904	0.009246826171875	\\
0.361210991254939	0.014678955078125	\\
0.361255382429973	0.016815185546875	\\
0.361299773605007	0.014373779296875	\\
0.361344164780042	0.011444091796875	\\
0.361388555955076	0.007080078125	\\
0.361432947130111	0.002166748046875	\\
0.361477338305145	0.000579833984375	\\
0.361521729480179	0.000152587890625	\\
0.361566120655214	0.001434326171875	\\
0.361610511830248	0.005767822265625	\\
0.361654903005283	0.01220703125	\\
0.361699294180317	0.017852783203125	\\
0.361743685355351	0.0224609375	\\
0.361788076530386	0.024322509765625	\\
0.36183246770542	0.024169921875	\\
0.361876858880455	0.022308349609375	\\
0.361921250055489	0.017120361328125	\\
0.361965641230523	0.01025390625	\\
0.362010032405558	0.003692626953125	\\
0.362054423580592	-0.00323486328125	\\
0.362098814755627	-0.008087158203125	\\
0.362143205930661	-0.0096435546875	\\
0.362187597105695	-0.01019287109375	\\
0.36223198828073	-0.008941650390625	\\
0.362276379455764	-0.00689697265625	\\
0.362320770630799	-0.005340576171875	\\
0.362365161805833	-0.0032958984375	\\
0.362409552980867	-0.002471923828125	\\
0.362453944155902	-0.003631591796875	\\
0.362498335330936	-0.006134033203125	\\
0.362542726505971	-0.010101318359375	\\
0.362587117681005	-0.01348876953125	\\
0.362631508856039	-0.01593017578125	\\
0.362675900031074	-0.017791748046875	\\
0.362720291206108	-0.01800537109375	\\
0.362764682381143	-0.01690673828125	\\
0.362809073556177	-0.014862060546875	\\
0.362853464731211	-0.0113525390625	\\
0.362897855906246	-0.00750732421875	\\
0.36294224708128	-0.003662109375	\\
0.362986638256315	-0.00030517578125	\\
0.363031029431349	0.00189208984375	\\
0.363075420606383	0.003814697265625	\\
0.363119811781418	0.004913330078125	\\
0.363164202956452	0.004669189453125	\\
0.363208594131487	0.00372314453125	\\
0.363252985306521	0.002716064453125	\\
0.363297376481555	0.001739501953125	\\
0.36334176765659	0.0015869140625	\\
0.363386158831624	0.002105712890625	\\
0.363430550006659	0.004058837890625	\\
0.363474941181693	0.00689697265625	\\
0.363519332356728	0.009429931640625	\\
0.363563723531762	0.012115478515625	\\
0.363608114706796	0.01434326171875	\\
0.363652505881831	0.016082763671875	\\
0.363696897056865	0.01751708984375	\\
0.3637412882319	0.017059326171875	\\
0.363785679406934	0.0152587890625	\\
0.363830070581968	0.013702392578125	\\
0.363874461757003	0.011505126953125	\\
0.363918852932037	0.00885009765625	\\
0.363963244107072	0.006683349609375	\\
0.364007635282106	0.004730224609375	\\
0.36405202645714	0.003875732421875	\\
0.364096417632175	0.003814697265625	\\
0.364140808807209	0.002899169921875	\\
0.364185199982244	0.003021240234375	\\
0.364229591157278	0.00390625	\\
0.364273982332312	0.003662109375	\\
0.364318373507347	0.003143310546875	\\
0.364362764682381	0.002044677734375	\\
0.364407155857416	0.000762939453125	\\
0.36445154703245	-0.001739501953125	\\
0.364495938207484	-0.0047607421875	\\
0.364540329382519	-0.0078125	\\
0.364584720557553	-0.010986328125	\\
0.364629111732588	-0.0133056640625	\\
0.364673502907622	-0.01458740234375	\\
0.364717894082656	-0.015045166015625	\\
0.364762285257691	-0.014129638671875	\\
0.364806676432725	-0.011993408203125	\\
0.36485106760776	-0.00933837890625	\\
0.364895458782794	-0.00677490234375	\\
0.364939849957828	-0.00457763671875	\\
0.364984241132863	-0.002410888671875	\\
0.365028632307897	-0.001068115234375	\\
0.365073023482932	-0.000762939453125	\\
0.365117414657966	-0.001190185546875	\\
0.365161805833	-0.002166748046875	\\
0.365206197008035	-0.003448486328125	\\
0.365250588183069	-0.004547119140625	\\
0.365294979358104	-0.003875732421875	\\
0.365339370533138	-0.0025634765625	\\
0.365383761708172	-0.000396728515625	\\
0.365428152883207	0.002685546875	\\
0.365472544058241	0.006011962890625	\\
0.365516935233276	0.009368896484375	\\
0.36556132640831	0.01202392578125	\\
0.365605717583344	0.013824462890625	\\
0.365650108758379	0.013580322265625	\\
0.365694499933413	0.012237548828125	\\
0.365738891108448	0.0101318359375	\\
0.365783282283482	0.00750732421875	\\
0.365827673458516	0.004608154296875	\\
0.365872064633551	0.002227783203125	\\
0.365916455808585	0.00079345703125	\\
0.36596084698362	0.00030517578125	\\
0.366005238158654	0	\\
0.366049629333688	0.000640869140625	\\
0.366094020508723	0.002227783203125	\\
0.366138411683757	0.0029296875	\\
0.366182802858792	0.00323486328125	\\
0.366227194033826	0.002960205078125	\\
0.366271585208861	0.001617431640625	\\
0.366315976383895	-0.000518798828125	\\
0.366360367558929	-0.003387451171875	\\
0.366404758733964	-0.00531005859375	\\
0.366449149908998	-0.007720947265625	\\
0.366493541084032	-0.010162353515625	\\
0.366537932259067	-0.010986328125	\\
0.366582323434101	-0.010528564453125	\\
0.366626714609136	-0.00958251953125	\\
0.36667110578417	-0.00787353515625	\\
0.366715496959205	-0.00537109375	\\
0.366759888134239	-0.0029296875	\\
0.366804279309273	-0.000885009765625	\\
0.366848670484308	0.000518798828125	\\
0.366893061659342	0.001678466796875	\\
0.366937452834377	0.00189208984375	\\
0.366981844009411	0.00146484375	\\
0.367026235184445	0.000732421875	\\
0.36707062635948	-0.000396728515625	\\
0.367115017534514	-0.001495361328125	\\
0.367159408709549	-0.001617431640625	\\
0.367203799884583	-0.000823974609375	\\
0.367248191059617	0.001007080078125	\\
0.367292582234652	0.003387451171875	\\
0.367336973409686	0.006103515625	\\
0.367381364584721	0.009246826171875	\\
0.367425755759755	0.01153564453125	\\
0.367470146934789	0.012939453125	\\
0.367514538109824	0.013580322265625	\\
0.367558929284858	0.012847900390625	\\
0.367603320459893	0.01129150390625	\\
0.367647711634927	0.00921630859375	\\
0.367692102809961	0.00701904296875	\\
0.367736493984996	0.00433349609375	\\
0.36778088516003	0.002349853515625	\\
0.367825276335065	0.001495361328125	\\
0.367869667510099	0.001739501953125	\\
0.367914058685133	0.0025634765625	\\
0.367958449860168	0.003631591796875	\\
0.368002841035202	0.005706787109375	\\
0.368047232210237	0.007049560546875	\\
0.368091623385271	0.0079345703125	\\
0.368136014560305	0.007965087890625	\\
0.36818040573534	0.00634765625	\\
0.368224796910374	0.0047607421875	\\
0.368269188085409	0.002227783203125	\\
0.368313579260443	-0.000885009765625	\\
0.368357970435477	-0.003448486328125	\\
0.368402361610512	-0.006195068359375	\\
0.368446752785546	-0.0074462890625	\\
0.368491143960581	-0.00836181640625	\\
0.368535535135615	-0.008453369140625	\\
0.368579926310649	-0.0076904296875	\\
0.368624317485684	-0.00653076171875	\\
0.368668708660718	-0.005279541015625	\\
0.368713099835753	-0.0040283203125	\\
0.368757491010787	-0.002716064453125	\\
0.368801882185821	-0.001983642578125	\\
0.368846273360856	-0.0015869140625	\\
0.36889066453589	-0.001617431640625	\\
0.368935055710925	-0.001556396484375	\\
0.368979446885959	-0.00189208984375	\\
0.369023838060993	-0.0025634765625	\\
0.369068229236028	-0.00299072265625	\\
0.369112620411062	-0.002899169921875	\\
0.369157011586097	-0.002777099609375	\\
0.369201402761131	-0.002593994140625	\\
0.369245793936165	-0.0009765625	\\
0.3692901851112	0.0006103515625	\\
0.369334576286234	0.002227783203125	\\
0.369378967461269	0.004058837890625	\\
0.369423358636303	0.005828857421875	\\
0.369467749811338	0.006927490234375	\\
0.369512140986372	0.007476806640625	\\
0.369556532161406	0.00738525390625	\\
0.369600923336441	0.0064697265625	\\
0.369645314511475	0.005889892578125	\\
0.36968970568651	0.004913330078125	\\
0.369734096861544	0.003997802734375	\\
0.369778488036578	0.003326416015625	\\
0.369822879211613	0.003204345703125	\\
0.369867270386647	0.003662109375	\\
0.369911661561682	0.004730224609375	\\
0.369956052736716	0.005950927734375	\\
0.37000044391175	0.0068359375	\\
0.370044835086785	0.0078125	\\
0.370089226261819	0.008026123046875	\\
0.370133617436854	0.007354736328125	\\
0.370178008611888	0.0062255859375	\\
0.370222399786922	0.005035400390625	\\
0.370266790961957	0.0032958984375	\\
0.370311182136991	0.001373291015625	\\
0.370355573312026	-0.00042724609375	\\
0.37039996448706	-0.001953125	\\
0.370444355662094	-0.00225830078125	\\
0.370488746837129	-0.002197265625	\\
0.370533138012163	-0.00177001953125	\\
0.370577529187198	-0.001129150390625	\\
0.370621920362232	-0.00018310546875	\\
0.370666311537266	0.000885009765625	\\
0.370710702712301	0.00140380859375	\\
0.370755093887335	0.00201416015625	\\
0.37079948506237	0.002197265625	\\
0.370843876237404	0.00128173828125	\\
0.370888267412438	0.000518798828125	\\
0.370932658587473	0.000152587890625	\\
0.370977049762507	-3.0517578125e-05	\\
0.371021440937542	-0.001251220703125	\\
0.371065832112576	-0.0020751953125	\\
0.37111022328761	-0.002349853515625	\\
0.371154614462645	-0.002044677734375	\\
0.371199005637679	-0.001708984375	\\
0.371243396812714	-0.001312255859375	\\
0.371287787987748	-0.0006103515625	\\
0.371332179162782	-0.000396728515625	\\
0.371376570337817	-0.000244140625	\\
0.371420961512851	0.000244140625	\\
0.371465352687886	0.00054931640625	\\
0.37150974386292	0.0006103515625	\\
0.371554135037954	0.000732421875	\\
0.371598526212989	0.001220703125	\\
0.371642917388023	0.001312255859375	\\
0.371687308563058	0.00115966796875	\\
0.371731699738092	0.001434326171875	\\
0.371776090913126	0.002044677734375	\\
0.371820482088161	0.0023193359375	\\
0.371864873263195	0.00201416015625	\\
0.37190926443823	0.002044677734375	\\
0.371953655613264	0.00250244140625	\\
0.371998046788299	0.002410888671875	\\
0.372042437963333	0.002105712890625	\\
0.372086829138367	0.001678466796875	\\
0.372131220313402	0.001068115234375	\\
0.372175611488436	-3.0517578125e-05	\\
0.372220002663471	-0.000640869140625	\\
0.372264393838505	-0.001434326171875	\\
0.372308785013539	-0.002410888671875	\\
0.372353176188574	-0.0029296875	\\
0.372397567363608	-0.003326416015625	\\
0.372441958538643	-0.003143310546875	\\
0.372486349713677	-0.00250244140625	\\
0.372530740888711	-0.002197265625	\\
0.372575132063746	-0.001556396484375	\\
0.37261952323878	-0.0003662109375	\\
0.372663914413815	0.000335693359375	\\
0.372708305588849	0.000579833984375	\\
0.372752696763883	0.000244140625	\\
0.372797087938918	-0.00030517578125	\\
0.372841479113952	-0.00164794921875	\\
0.372885870288987	-0.0035400390625	\\
0.372930261464021	-0.00445556640625	\\
0.372974652639055	-0.005126953125	\\
0.37301904381409	-0.005157470703125	\\
0.373063434989124	-0.004791259765625	\\
0.373107826164159	-0.0037841796875	\\
0.373152217339193	-0.002166748046875	\\
0.373196608514227	-0.000732421875	\\
0.373240999689262	0.001312255859375	\\
0.373285390864296	0.00299072265625	\\
0.373329782039331	0.002777099609375	\\
0.373374173214365	0.002471923828125	\\
0.373418564389399	0.002044677734375	\\
0.373462955564434	0.001007080078125	\\
0.373507346739468	0.00018310546875	\\
0.373551737914503	-0.000762939453125	\\
0.373596129089537	-0.001312255859375	\\
0.373640520264571	-0.001068115234375	\\
0.373684911439606	-0.000274658203125	\\
0.37372930261464	0.000762939453125	\\
0.373773693789675	0.00146484375	\\
0.373818084964709	0.002716064453125	\\
0.373862476139743	0.0040283203125	\\
0.373906867314778	0.004302978515625	\\
0.373951258489812	0.00372314453125	\\
0.373995649664847	0.003143310546875	\\
0.374040040839881	0.002197265625	\\
0.374084432014915	0.00054931640625	\\
0.37412882318995	-0.0006103515625	\\
0.374173214364984	-0.001251220703125	\\
0.374217605540019	-0.00225830078125	\\
0.374261996715053	-0.00262451171875	\\
0.374306387890087	-0.002166748046875	\\
0.374350779065122	-0.001739501953125	\\
0.374395170240156	-0.00079345703125	\\
0.374439561415191	-0.000274658203125	\\
0.374483952590225	0.000274658203125	\\
0.374528343765259	0.0009765625	\\
0.374572734940294	0.00103759765625	\\
0.374617126115328	0.000823974609375	\\
0.374661517290363	0.000762939453125	\\
0.374705908465397	0.000152587890625	\\
0.374750299640431	-0.000762939453125	\\
0.374794690815466	-0.001129150390625	\\
0.3748390819905	-0.00146484375	\\
0.374883473165535	-0.001861572265625	\\
0.374927864340569	-0.002105712890625	\\
0.374972255515603	-0.002105712890625	\\
0.375016646690638	-0.001983642578125	\\
0.375061037865672	-0.001739501953125	\\
0.375105429040707	-0.0008544921875	\\
0.375149820215741	-0.0009765625	\\
0.375194211390776	-0.001983642578125	\\
0.37523860256581	-0.0018310546875	\\
0.375282993740844	-0.002471923828125	\\
0.375327384915879	-0.00311279296875	\\
0.375371776090913	-0.003662109375	\\
0.375416167265948	-0.004425048828125	\\
0.375460558440982	-0.004364013671875	\\
0.375504949616016	-0.004241943359375	\\
0.375549340791051	-0.003814697265625	\\
0.375593731966085	-0.002044677734375	\\
0.37563812314112	-0.0009765625	\\
0.375682514316154	-0.000213623046875	\\
0.375726905491188	0.001007080078125	\\
0.375771296666223	0.00140380859375	\\
0.375815687841257	0.001495361328125	\\
0.375860079016292	0.00140380859375	\\
0.375904470191326	0.00079345703125	\\
0.37594886136636	-0.000244140625	\\
0.375993252541395	-0.001129150390625	\\
0.376037643716429	-0.001678466796875	\\
0.376082034891464	-0.001953125	\\
0.376126426066498	-0.0015869140625	\\
0.376170817241532	-0.0010986328125	\\
0.376215208416567	-0.0001220703125	\\
0.376259599591601	0.000213623046875	\\
0.376303990766636	0.0010986328125	\\
0.37634838194167	0.00189208984375	\\
0.376392773116704	0.001617431640625	\\
0.376437164291739	0.0015869140625	\\
0.376481555466773	0.000885009765625	\\
0.376525946641808	0.00018310546875	\\
0.376570337816842	-0.000518798828125	\\
0.376614728991876	-0.001495361328125	\\
0.376659120166911	-0.001739501953125	\\
0.376703511341945	-0.001922607421875	\\
0.37674790251698	-0.00225830078125	\\
0.376792293692014	-0.00238037109375	\\
0.376836684867048	-0.001922607421875	\\
0.376881076042083	-0.001373291015625	\\
0.376925467217117	-0.00140380859375	\\
0.376969858392152	-0.001251220703125	\\
0.377014249567186	-0.001220703125	\\
0.37705864074222	-0.001373291015625	\\
0.377103031917255	-0.001312255859375	\\
0.377147423092289	-0.001434326171875	\\
0.377191814267324	-0.001220703125	\\
0.377236205442358	-0.0009765625	\\
0.377280596617392	-0.0008544921875	\\
0.377324987792427	-0.000579833984375	\\
0.377369378967461	-0.000244140625	\\
0.377413770142496	0.00042724609375	\\
0.37745816131753	0.000579833984375	\\
0.377502552492564	0.000335693359375	\\
0.377546943667599	0.000579833984375	\\
0.377591334842633	0.000244140625	\\
0.377635726017668	-0.000213623046875	\\
0.377680117192702	0.000152587890625	\\
0.377724508367736	0.000579833984375	\\
0.377768899542771	0.000579833984375	\\
0.377813290717805	0.00079345703125	\\
0.37785768189284	0.00103759765625	\\
0.377902073067874	0.001129150390625	\\
0.377946464242909	0.00128173828125	\\
0.377990855417943	0.00115966796875	\\
0.378035246592977	0.001220703125	\\
0.378079637768012	0.0010986328125	\\
0.378124028943046	0.00030517578125	\\
0.378168420118081	-3.0517578125e-05	\\
0.378212811293115	-0.000518798828125	\\
0.378257202468149	-0.000823974609375	\\
0.378301593643184	-0.00067138671875	\\
0.378345984818218	-0.000640869140625	\\
0.378390375993253	-0.000762939453125	\\
0.378434767168287	-0.000823974609375	\\
0.378479158343321	-0.001007080078125	\\
0.378523549518356	-0.001190185546875	\\
0.37856794069339	-0.001983642578125	\\
0.378612331868425	-0.00299072265625	\\
0.378656723043459	-0.00372314453125	\\
0.378701114218493	-0.004608154296875	\\
0.378745505393528	-0.004547119140625	\\
0.378789896568562	-0.004425048828125	\\
0.378834287743597	-0.00421142578125	\\
0.378878678918631	-0.0035400390625	\\
0.378923070093665	-0.002777099609375	\\
0.3789674612687	-0.002288818359375	\\
0.379011852443734	-0.001983642578125	\\
0.379056243618769	-0.00152587890625	\\
0.379100634793803	-0.001861572265625	\\
0.379145025968837	-0.00238037109375	\\
0.379189417143872	-0.002960205078125	\\
0.379233808318906	-0.003753662109375	\\
0.379278199493941	-0.004241943359375	\\
0.379322590668975	-0.00457763671875	\\
0.379366981844009	-0.004608154296875	\\
0.379411373019044	-0.003997802734375	\\
0.379455764194078	-0.003326416015625	\\
0.379500155369113	-0.002716064453125	\\
0.379544546544147	-0.002044677734375	\\
0.379588937719181	-0.00152587890625	\\
0.379633328894216	-0.001220703125	\\
0.37967772006925	-0.001556396484375	\\
0.379722111244285	-0.001800537109375	\\
0.379766502419319	-0.00225830078125	\\
0.379810893594353	-0.003021240234375	\\
0.379855284769388	-0.0035400390625	\\
0.379899675944422	-0.003814697265625	\\
0.379944067119457	-0.0042724609375	\\
0.379988458294491	-0.0042724609375	\\
0.380032849469525	-0.003753662109375	\\
0.38007724064456	-0.00360107421875	\\
0.380121631819594	-0.003204345703125	\\
0.380166022994629	-0.0029296875	\\
0.380210414169663	-0.002685546875	\\
0.380254805344697	-0.00213623046875	\\
0.380299196519732	-0.002349853515625	\\
0.380343587694766	-0.002471923828125	\\
0.380387978869801	-0.00274658203125	\\
0.380432370044835	-0.00360107421875	\\
0.38047676121987	-0.00347900390625	\\
0.380521152394904	-0.00347900390625	\\
0.380565543569938	-0.00323486328125	\\
0.380609934744973	-0.002593994140625	\\
0.380654325920007	-0.002593994140625	\\
0.380698717095041	-0.001861572265625	\\
0.380743108270076	-0.001861572265625	\\
0.38078749944511	-0.001434326171875	\\
0.380831890620145	-0.00103759765625	\\
0.380876281795179	-0.001251220703125	\\
0.380920672970214	-0.000732421875	\\
0.380965064145248	-0.000946044921875	\\
0.381009455320282	-0.001007080078125	\\
0.381053846495317	-0.00079345703125	\\
0.381098237670351	-0.00103759765625	\\
0.381142628845386	-0.001068115234375	\\
0.38118702002042	-0.0010986328125	\\
0.381231411195454	-0.00164794921875	\\
0.381275802370489	-0.001129150390625	\\
0.381320193545523	-0.000885009765625	\\
0.381364584720558	-0.00128173828125	\\
0.381408975895592	-0.00146484375	\\
0.381453367070626	-0.001708984375	\\
0.381497758245661	-0.002044677734375	\\
0.381542149420695	-0.002410888671875	\\
0.38158654059573	-0.002105712890625	\\
0.381630931770764	-0.001739501953125	\\
0.381675322945798	-0.00189208984375	\\
0.381719714120833	-0.001800537109375	\\
0.381764105295867	-0.00152587890625	\\
0.381808496470902	-0.0015869140625	\\
0.381852887645936	-0.001129150390625	\\
0.38189727882097	-0.001068115234375	\\
0.381941669996005	-0.00140380859375	\\
0.381986061171039	-0.00152587890625	\\
0.382030452346074	-0.001373291015625	\\
0.382074843521108	-0.00128173828125	\\
0.382119234696142	-0.001617431640625	\\
0.382163625871177	-0.00152587890625	\\
0.382208017046211	-0.001739501953125	\\
0.382252408221246	-0.001617431640625	\\
0.38229679939628	-0.001922607421875	\\
0.382341190571314	-0.002197265625	\\
0.382385581746349	-0.00201416015625	\\
0.382429972921383	-0.001373291015625	\\
0.382474364096418	-0.00128173828125	\\
0.382518755271452	-0.001556396484375	\\
0.382563146446486	-0.001556396484375	\\
0.382607537621521	-0.00177001953125	\\
0.382651928796555	-0.00177001953125	\\
0.38269631997159	-0.002197265625	\\
0.382740711146624	-0.002471923828125	\\
0.382785102321658	-0.002960205078125	\\
0.382829493496693	-0.003173828125	\\
0.382873884671727	-0.003143310546875	\\
0.382918275846762	-0.003814697265625	\\
0.382962667021796	-0.00360107421875	\\
0.38300705819683	-0.00244140625	\\
0.383051449371865	-0.001556396484375	\\
0.383095840546899	-0.000732421875	\\
0.383140231721934	-0.000213623046875	\\
0.383184622896968	-0.00030517578125	\\
0.383229014072002	-6.103515625e-05	\\
0.383273405247037	0.0001220703125	\\
0.383317796422071	-0.00048828125	\\
0.383362187597106	-0.001922607421875	\\
0.38340657877214	-0.0023193359375	\\
0.383450969947174	-0.00238037109375	\\
0.383495361122209	-0.00286865234375	\\
0.383539752297243	-0.0020751953125	\\
0.383584143472278	-0.00103759765625	\\
0.383628534647312	-0.00030517578125	\\
0.383672925822347	0.000701904296875	\\
0.383717316997381	0.001495361328125	\\
0.383761708172415	0.0020751953125	\\
0.38380609934745	0.0020751953125	\\
0.383850490522484	0.0015869140625	\\
0.383894881697519	0.001068115234375	\\
0.383939272872553	0.000762939453125	\\
0.383983664047587	-0.000335693359375	\\
0.384028055222622	-0.000946044921875	\\
0.384072446397656	-0.001129150390625	\\
0.384116837572691	-0.00079345703125	\\
0.384161228747725	-3.0517578125e-05	\\
0.384205619922759	0.001007080078125	\\
0.384250011097794	0.0025634765625	\\
0.384294402272828	0.00341796875	\\
0.384338793447863	0.004119873046875	\\
0.384383184622897	0.0040283203125	\\
0.384427575797931	0.0032958984375	\\
0.384471966972966	0.002349853515625	\\
0.384516358148	0.001617431640625	\\
0.384560749323035	0.001220703125	\\
0.384605140498069	0.000701904296875	\\
0.384649531673103	0.000823974609375	\\
0.384693922848138	0.0010986328125	\\
0.384738314023172	0.002166748046875	\\
0.384782705198207	0.003387451171875	\\
0.384827096373241	0.00408935546875	\\
0.384871487548275	0.004791259765625	\\
0.38491587872331	0.00482177734375	\\
0.384960269898344	0.004791259765625	\\
0.385004661073379	0.003997802734375	\\
0.385049052248413	0.003204345703125	\\
0.385093443423447	0.002044677734375	\\
0.385137834598482	0.000640869140625	\\
0.385182225773516	0.000396728515625	\\
0.385226616948551	0.00048828125	\\
0.385271008123585	0.000640869140625	\\
0.385315399298619	0.00091552734375	\\
0.385359790473654	0.00140380859375	\\
0.385404181648688	0.0015869140625	\\
0.385448572823723	0.00140380859375	\\
0.385492963998757	0.001190185546875	\\
0.385537355173791	0.00030517578125	\\
0.385581746348826	-0.000579833984375	\\
0.38562613752386	-0.00146484375	\\
0.385670528698895	-0.00244140625	\\
0.385714919873929	-0.00299072265625	\\
0.385759311048963	-0.003265380859375	\\
0.385803702223998	-0.003173828125	\\
0.385848093399032	-0.002777099609375	\\
0.385892484574067	-0.0018310546875	\\
0.385936875749101	-0.0009765625	\\
0.385981266924135	-0.0006103515625	\\
0.38602565809917	-0.000274658203125	\\
0.386070049274204	-0.00030517578125	\\
0.386114440449239	0	\\
0.386158831624273	-0.000213623046875	\\
0.386203222799308	-0.00091552734375	\\
0.386247613974342	-0.00164794921875	\\
0.386292005149376	-0.001373291015625	\\
0.386336396324411	-0.000640869140625	\\
0.386380787499445	-0.000762939453125	\\
0.38642517867448	-0.0003662109375	\\
0.386469569849514	0.000244140625	\\
0.386513961024548	0.000640869140625	\\
0.386558352199583	0.00079345703125	\\
0.386602743374617	0.0009765625	\\
0.386647134549652	0.000762939453125	\\
0.386691525724686	0.00128173828125	\\
0.38673591689972	0.001617431640625	\\
0.386780308074755	0.001068115234375	\\
0.386824699249789	0.000946044921875	\\
0.386869090424824	0.000732421875	\\
0.386913481599858	0	\\
0.386957872774892	-0.000335693359375	\\
0.387002263949927	-0.000579833984375	\\
0.387046655124961	-0.0008544921875	\\
0.387091046299996	-0.000732421875	\\
0.38713543747503	-0.000885009765625	\\
0.387179828650064	-0.000823974609375	\\
0.387224219825099	-0.000762939453125	\\
0.387268611000133	-0.000640869140625	\\
0.387313002175168	-0.000396728515625	\\
0.387357393350202	-0.000244140625	\\
0.387401784525236	-0.00067138671875	\\
0.387446175700271	-0.00067138671875	\\
0.387490566875305	-0.001129150390625	\\
0.38753495805034	-0.001434326171875	\\
0.387579349225374	-0.0020751953125	\\
0.387623740400408	-0.002685546875	\\
0.387668131575443	-0.0028076171875	\\
0.387712522750477	-0.00244140625	\\
0.387756913925512	-0.002838134765625	\\
0.387801305100546	-0.0030517578125	\\
0.38784569627558	-0.0025634765625	\\
0.387890087450615	-0.002532958984375	\\
0.387934478625649	-0.0023193359375	\\
0.387978869800684	-0.001922607421875	\\
0.388023260975718	-0.001190185546875	\\
0.388067652150752	-0.000762939453125	\\
0.388112043325787	-0.00091552734375	\\
0.388156434500821	-0.001190185546875	\\
0.388200825675856	-0.001251220703125	\\
0.38824521685089	-0.00177001953125	\\
0.388289608025924	-0.001312255859375	\\
0.388333999200959	-0.001190185546875	\\
0.388378390375993	-0.0009765625	\\
0.388422781551028	-0.000213623046875	\\
0.388467172726062	0.000152587890625	\\
0.388511563901096	0.000701904296875	\\
0.388555955076131	0.00146484375	\\
0.388600346251165	0.00225830078125	\\
0.3886447374262	0.002655029296875	\\
0.388689128601234	0.003265380859375	\\
0.388733519776268	0.00347900390625	\\
0.388777910951303	0.0032958984375	\\
0.388822302126337	0.00341796875	\\
0.388866693301372	0.00341796875	\\
0.388911084476406	0.0030517578125	\\
0.388955475651441	0.002655029296875	\\
0.388999866826475	0.002410888671875	\\
0.389044258001509	0.002044677734375	\\
0.389088649176544	0.001678466796875	\\
0.389133040351578	0.001922607421875	\\
0.389177431526612	0.001983642578125	\\
0.389221822701647	0.00164794921875	\\
0.389266213876681	0.00152587890625	\\
0.389310605051716	0.001190185546875	\\
0.38935499622675	0.001373291015625	\\
0.389399387401785	0.001007080078125	\\
0.389443778576819	0.00054931640625	\\
0.389488169751853	-6.103515625e-05	\\
0.389532560926888	-0.000518798828125	\\
0.389576952101922	-0.0008544921875	\\
0.389621343276957	-0.00128173828125	\\
0.389665734451991	-0.001556396484375	\\
0.389710125627025	-0.001434326171875	\\
0.38975451680206	-0.000823974609375	\\
0.389798907977094	-0.000885009765625	\\
0.389843299152129	-0.001251220703125	\\
0.389887690327163	-0.000701904296875	\\
0.389932081502197	-0.00054931640625	\\
0.389976472677232	-0.00048828125	\\
0.390020863852266	-0.00067138671875	\\
0.390065255027301	-0.0008544921875	\\
0.390109646202335	-0.000579833984375	\\
0.390154037377369	-0.0006103515625	\\
0.390198428552404	-0.0006103515625	\\
0.390242819727438	-0.000640869140625	\\
0.390287210902473	-0.00091552734375	\\
0.390331602077507	-0.001251220703125	\\
0.390375993252541	-0.0009765625	\\
0.390420384427576	-0.000640869140625	\\
0.39046477560261	-6.103515625e-05	\\
0.390509166777645	0.0006103515625	\\
0.390553557952679	0.001617431640625	\\
0.390597949127713	0.001708984375	\\
0.390642340302748	0.001312255859375	\\
0.390686731477782	0.00140380859375	\\
0.390731122652817	0.001068115234375	\\
0.390775513827851	0.000579833984375	\\
0.390819905002885	6.103515625e-05	\\
0.39086429617792	-0.000152587890625	\\
0.390908687352954	0.0001220703125	\\
0.390953078527989	0.000274658203125	\\
0.390997469703023	0.00091552734375	\\
0.391041860878057	0.00079345703125	\\
0.391086252053092	0.0008544921875	\\
0.391130643228126	0.001190185546875	\\
0.391175034403161	0.0010986328125	\\
0.391219425578195	0.00115966796875	\\
0.391263816753229	0.0015869140625	\\
0.391308207928264	0.001953125	\\
0.391352599103298	0.001434326171875	\\
0.391396990278333	0.001068115234375	\\
0.391441381453367	0.0015869140625	\\
0.391485772628401	0.00146484375	\\
0.391530163803436	0.00146484375	\\
0.39157455497847	0.001678466796875	\\
0.391618946153505	0.001556396484375	\\
0.391663337328539	0.001190185546875	\\
0.391707728503573	0.00140380859375	\\
0.391752119678608	0.0013427734375	\\
0.391796510853642	0.001007080078125	\\
0.391840902028677	0.001007080078125	\\
0.391885293203711	0.001190185546875	\\
0.391929684378745	0.001220703125	\\
0.39197407555378	0.00079345703125	\\
0.392018466728814	0.00042724609375	\\
0.392062857903849	0.000213623046875	\\
0.392107249078883	0.00042724609375	\\
0.392151640253918	0.000701904296875	\\
0.392196031428952	0.000732421875	\\
0.392240422603986	0.000640869140625	\\
0.392284813779021	0.000244140625	\\
0.392329204954055	-0.000213623046875	\\
0.39237359612909	-6.103515625e-05	\\
0.392417987304124	0.00018310546875	\\
0.392462378479158	-0.0001220703125	\\
0.392506769654193	0.00030517578125	\\
0.392551160829227	0.000701904296875	\\
0.392595552004262	0.001312255859375	\\
0.392639943179296	0.00128173828125	\\
0.39268433435433	0.00091552734375	\\
0.392728725529365	0.001312255859375	\\
0.392773116704399	0.001617431640625	\\
0.392817507879434	0.002044677734375	\\
0.392861899054468	0.0020751953125	\\
0.392906290229502	0.002349853515625	\\
0.392950681404537	0.00299072265625	\\
0.392995072579571	0.003021240234375	\\
0.393039463754606	0.002838134765625	\\
0.39308385492964	0.003143310546875	\\
0.393128246104674	0.003662109375	\\
0.393172637279709	0.003387451171875	\\
0.393217028454743	0.00335693359375	\\
0.393261419629778	0.00372314453125	\\
0.393305810804812	0.003326416015625	\\
0.393350201979846	0.0030517578125	\\
0.393394593154881	0.002838134765625	\\
0.393438984329915	0.002593994140625	\\
0.39348337550495	0.002471923828125	\\
0.393527766679984	0.001953125	\\
0.393572157855018	0.00164794921875	\\
0.393616549030053	0.0013427734375	\\
0.393660940205087	0.000762939453125	\\
0.393705331380122	0.000885009765625	\\
0.393749722555156	0.00067138671875	\\
0.39379411373019	0.000213623046875	\\
0.393838504905225	3.0517578125e-05	\\
0.393882896080259	-0.000213623046875	\\
0.393927287255294	-0.00030517578125	\\
0.393971678430328	-0.000274658203125	\\
0.394016069605362	-0.000457763671875	\\
0.394060460780397	-0.000457763671875	\\
0.394104851955431	-0.000579833984375	\\
0.394149243130466	-0.00042724609375	\\
0.3941936343055	-0.00048828125	\\
0.394238025480534	-0.000274658203125	\\
0.394282416655569	-0.000244140625	\\
0.394326807830603	-0.000640869140625	\\
0.394371199005638	-0.0008544921875	\\
0.394415590180672	-0.0003662109375	\\
0.394459981355706	6.103515625e-05	\\
0.394504372530741	-0.000518798828125	\\
0.394548763705775	-0.000396728515625	\\
0.39459315488081	-3.0517578125e-05	\\
0.394637546055844	-3.0517578125e-05	\\
0.394681937230879	0.000152587890625	\\
0.394726328405913	-0.00054931640625	\\
0.394770719580947	-0.00042724609375	\\
0.394815110755982	0.000244140625	\\
0.394859501931016	-0.0001220703125	\\
0.394903893106051	3.0517578125e-05	\\
0.394948284281085	-6.103515625e-05	\\
0.394992675456119	-0.000579833984375	\\
0.395037066631154	-0.000396728515625	\\
0.395081457806188	-0.00018310546875	\\
0.395125848981223	-0.000518798828125	\\
0.395170240156257	-0.00067138671875	\\
0.395214631331291	-0.000396728515625	\\
0.395259022506326	-0.000244140625	\\
0.39530341368136	-0.000732421875	\\
0.395347804856395	-0.0010986328125	\\
0.395392196031429	-0.0013427734375	\\
0.395436587206463	-0.001617431640625	\\
0.395480978381498	-0.001251220703125	\\
0.395525369556532	-0.0010986328125	\\
0.395569760731567	-0.001251220703125	\\
0.395614151906601	-0.0010986328125	\\
0.395658543081635	-0.000762939453125	\\
0.39570293425667	-0.000579833984375	\\
0.395747325431704	-0.000701904296875	\\
0.395791716606739	-0.000396728515625	\\
0.395836107781773	-0.000274658203125	\\
0.395880498956807	-0.000213623046875	\\
0.395924890131842	-0.000152587890625	\\
0.395969281306876	-0.00030517578125	\\
0.396013672481911	-0.000244140625	\\
0.396058063656945	-0.000335693359375	\\
0.396102454831979	-0.000152587890625	\\
0.396146846007014	-0.000213623046875	\\
0.396191237182048	-0.0003662109375	\\
0.396235628357083	-0.00067138671875	\\
0.396280019532117	-0.000213623046875	\\
0.396324410707151	-0.000213623046875	\\
0.396368801882186	-0.00054931640625	\\
0.39641319305722	-0.000335693359375	\\
0.396457584232255	-0.000518798828125	\\
0.396501975407289	-0.00091552734375	\\
0.396546366582323	-0.00067138671875	\\
0.396590757757358	-0.0006103515625	\\
0.396635148932392	-0.000732421875	\\
0.396679540107427	-0.000701904296875	\\
0.396723931282461	-0.0001220703125	\\
0.396768322457495	0.000152587890625	\\
0.39681271363253	-0.000213623046875	\\
0.396857104807564	-0.000213623046875	\\
0.396901495982599	-0.000579833984375	\\
0.396945887157633	-0.000732421875	\\
0.396990278332667	-0.0009765625	\\
0.397034669507702	-0.001068115234375	\\
0.397079060682736	-0.001190185546875	\\
0.397123451857771	-0.00128173828125	\\
0.397167843032805	-0.00128173828125	\\
0.397212234207839	-0.00177001953125	\\
0.397256625382874	-0.001953125	\\
0.397301016557908	-0.002105712890625	\\
0.397345407732943	-0.002166748046875	\\
0.397389798907977	-0.002655029296875	\\
0.397434190083012	-0.003265380859375	\\
0.397478581258046	-0.002838134765625	\\
0.39752297243308	-0.00262451171875	\\
0.397567363608115	-0.00244140625	\\
0.397611754783149	-0.001953125	\\
0.397656145958183	-0.0018310546875	\\
0.397700537133218	-0.002166748046875	\\
0.397744928308252	-0.001922607421875	\\
0.397789319483287	-0.001556396484375	\\
0.397833710658321	-0.001739501953125	\\
0.397878101833356	-0.00164794921875	\\
0.39792249300839	-0.002197265625	\\
0.397966884183424	-0.0020751953125	\\
0.398011275358459	-0.002288818359375	\\
0.398055666533493	-0.00250244140625	\\
0.398100057708528	-0.0018310546875	\\
0.398144448883562	-0.00128173828125	\\
0.398188840058596	-0.00048828125	\\
0.398233231233631	-0.0001220703125	\\
0.398277622408665	-9.1552734375e-05	\\
0.3983220135837	0.000518798828125	\\
0.398366404758734	0.001129150390625	\\
0.398410795933768	0.0008544921875	\\
0.398455187108803	0.00067138671875	\\
0.398499578283837	0.00152587890625	\\
0.398543969458872	0.001312255859375	\\
0.398588360633906	0.001007080078125	\\
0.39863275180894	0.00140380859375	\\
0.398677142983975	0.000823974609375	\\
0.398721534159009	0.0009765625	\\
0.398765925334044	0.001312255859375	\\
0.398810316509078	0.00140380859375	\\
0.398854707684112	0.001312255859375	\\
0.398899098859147	0.000823974609375	\\
0.398943490034181	0.00103759765625	\\
0.398987881209216	0.00091552734375	\\
0.39903227238425	0.00042724609375	\\
0.399076663559284	0.00067138671875	\\
0.399121054734319	0.00079345703125	\\
0.399165445909353	0.000823974609375	\\
0.399209837084388	0.000579833984375	\\
0.399254228259422	0.0006103515625	\\
0.399298619434456	0.00030517578125	\\
0.399343010609491	3.0517578125e-05	\\
0.399387401784525	-0.00018310546875	\\
0.39943179295956	-0.000518798828125	\\
0.399476184134594	-0.000335693359375	\\
0.399520575309628	-0.00042724609375	\\
0.399564966484663	-0.000244140625	\\
0.399609357659697	-0.00030517578125	\\
0.399653748834732	-0.00048828125	\\
0.399698140009766	6.103515625e-05	\\
0.3997425311848	0.00018310546875	\\
0.399786922359835	-0.0001220703125	\\
0.399831313534869	0.000152587890625	\\
0.399875704709904	0.0001220703125	\\
0.399920095884938	-0.00030517578125	\\
0.399964487059972	-9.1552734375e-05	\\
0.400008878235007	-3.0517578125e-05	\\
0.400053269410041	-0.000213623046875	\\
0.400097660585076	-0.000152587890625	\\
0.40014205176011	0.0003662109375	\\
0.400186442935144	0.000274658203125	\\
0.400230834110179	0.00048828125	\\
0.400275225285213	0.001220703125	\\
0.400319616460248	0.001983642578125	\\
0.400364007635282	0.002044677734375	\\
0.400408398810317	0.002105712890625	\\
0.400452789985351	0.002593994140625	\\
0.400497181160385	0.00244140625	\\
0.40054157233542	0.002105712890625	\\
0.400585963510454	0.00189208984375	\\
0.400630354685489	0.001495361328125	\\
0.400674745860523	0.00091552734375	\\
0.400719137035557	0.000701904296875	\\
0.400763528210592	0.0001220703125	\\
0.400807919385626	0.000518798828125	\\
0.400852310560661	0.00115966796875	\\
0.400896701735695	0.001220703125	\\
0.400941092910729	0.00128173828125	\\
0.400985484085764	0.00115966796875	\\
0.401029875260798	0.001068115234375	\\
0.401074266435833	0.00091552734375	\\
0.401118657610867	0.000823974609375	\\
0.401163048785901	0.000701904296875	\\
0.401207439960936	9.1552734375e-05	\\
0.40125183113597	-0.000396728515625	\\
0.401296222311005	-0.000396728515625	\\
0.401340613486039	-0.00048828125	\\
0.401385004661073	-6.103515625e-05	\\
0.401429395836108	0.0006103515625	\\
0.401473787011142	0.000396728515625	\\
0.401518178186177	0.000701904296875	\\
0.401562569361211	0.001007080078125	\\
0.401606960536245	0.000457763671875	\\
0.40165135171128	0	\\
0.401695742886314	-0.00030517578125	\\
0.401740134061349	-0.001129150390625	\\
0.401784525236383	-0.001068115234375	\\
0.401828916411417	-0.001251220703125	\\
0.401873307586452	-0.000885009765625	\\
0.401917698761486	-0.0003662109375	\\
0.401962089936521	-0.000152587890625	\\
0.402006481111555	0.000732421875	\\
0.402050872286589	0.000640869140625	\\
0.402095263461624	0.00140380859375	\\
0.402139654636658	0.00201416015625	\\
0.402184045811693	0.00177001953125	\\
0.402228436986727	0.002288818359375	\\
0.402272828161761	0.002349853515625	\\
0.402317219336796	0.002227783203125	\\
0.40236161051183	0.00201416015625	\\
0.402406001686865	0.001556396484375	\\
0.402450392861899	0.0013427734375	\\
0.402494784036933	0.001068115234375	\\
0.402539175211968	0.000701904296875	\\
0.402583566387002	0.00091552734375	\\
0.402627957562037	0.001556396484375	\\
0.402672348737071	0.001617431640625	\\
0.402716739912105	0.001953125	\\
0.40276113108714	0.0023193359375	\\
0.402805522262174	0.0020751953125	\\
0.402849913437209	0.001983642578125	\\
0.402894304612243	0.00164794921875	\\
0.402938695787277	0.001190185546875	\\
0.402983086962312	3.0517578125e-05	\\
0.403027478137346	-0.0009765625	\\
0.403071869312381	-0.000518798828125	\\
0.403116260487415	-0.000762939453125	\\
0.40316065166245	-0.00115966796875	\\
0.403205042837484	-0.000823974609375	\\
0.403249434012518	-0.001007080078125	\\
0.403293825187553	-0.0008544921875	\\
0.403338216362587	-0.000457763671875	\\
0.403382607537622	-0.000213623046875	\\
0.403426998712656	-0.000152587890625	\\
0.40347138988769	-0.000335693359375	\\
0.403515781062725	-0.0003662109375	\\
0.403560172237759	-0.000823974609375	\\
0.403604563412794	-0.001861572265625	\\
0.403648954587828	-0.001617431640625	\\
0.403693345762862	-0.001251220703125	\\
0.403737736937897	-0.0015869140625	\\
0.403782128112931	-0.000946044921875	\\
0.403826519287966	-0.000335693359375	\\
0.403870910463	9.1552734375e-05	\\
0.403915301638034	0.00091552734375	\\
0.403959692813069	0.00164794921875	\\
0.404004083988103	0.002227783203125	\\
0.404048475163138	0.0025634765625	\\
0.404092866338172	0.001953125	\\
0.404137257513206	0.001953125	\\
0.404181648688241	0.001800537109375	\\
0.404226039863275	0.001739501953125	\\
0.40427043103831	0.00128173828125	\\
0.404314822213344	0.00054931640625	\\
0.404359213388378	0.000885009765625	\\
0.404403604563413	0.001251220703125	\\
0.404447995738447	0.00152587890625	\\
0.404492386913482	0.001495361328125	\\
0.404536778088516	0.00152587890625	\\
0.40458116926355	0.001678466796875	\\
0.404625560438585	0.00164794921875	\\
0.404669951613619	0.001556396484375	\\
0.404714342788654	0.001220703125	\\
0.404758733963688	0.000762939453125	\\
0.404803125138722	0.000335693359375	\\
0.404847516313757	-9.1552734375e-05	\\
0.404891907488791	-0.000396728515625	\\
0.404936298663826	-0.00079345703125	\\
0.40498068983886	-0.001373291015625	\\
0.405025081013894	-0.001373291015625	\\
0.405069472188929	-0.001495361328125	\\
0.405113863363963	-0.001312255859375	\\
0.405158254538998	-0.00091552734375	\\
0.405202645714032	-0.0013427734375	\\
0.405247036889066	-0.00164794921875	\\
0.405291428064101	-0.00177001953125	\\
0.405335819239135	-0.001922607421875	\\
0.40538021041417	-0.001922607421875	\\
0.405424601589204	-0.002105712890625	\\
0.405468992764238	-0.00189208984375	\\
0.405513383939273	-0.001434326171875	\\
0.405557775114307	-0.001190185546875	\\
0.405602166289342	-0.0009765625	\\
0.405646557464376	-0.000579833984375	\\
0.40569094863941	-0.000335693359375	\\
0.405735339814445	-0.0003662109375	\\
0.405779730989479	-0.00067138671875	\\
0.405824122164514	-0.00079345703125	\\
0.405868513339548	-0.00079345703125	\\
0.405912904514583	-0.00067138671875	\\
0.405957295689617	-0.00054931640625	\\
0.406001686864651	-0.00018310546875	\\
0.406046078039686	3.0517578125e-05	\\
0.40609046921472	-0.00018310546875	\\
0.406134860389754	-0.0001220703125	\\
0.406179251564789	-6.103515625e-05	\\
0.406223642739823	3.0517578125e-05	\\
0.406268033914858	0.000457763671875	\\
0.406312425089892	0.000518798828125	\\
0.406356816264927	0.000457763671875	\\
0.406401207439961	0.000518798828125	\\
0.406445598614995	0.000640869140625	\\
0.40648998979003	0.00067138671875	\\
0.406534380965064	0.000732421875	\\
0.406578772140099	0.00091552734375	\\
0.406623163315133	0.000518798828125	\\
0.406667554490167	0.00048828125	\\
0.406711945665202	0.000762939453125	\\
0.406756336840236	0.0006103515625	\\
0.406800728015271	0.00079345703125	\\
0.406845119190305	0.001312255859375	\\
0.406889510365339	0.000885009765625	\\
0.406933901540374	0.00048828125	\\
0.406978292715408	9.1552734375e-05	\\
0.407022683890443	0.00018310546875	\\
0.407067075065477	0.000335693359375	\\
0.407111466240511	-0.00030517578125	\\
0.407155857415546	-0.000518798828125	\\
0.40720024859058	-0.000762939453125	\\
0.407244639765615	-0.00115966796875	\\
0.407289030940649	-0.0008544921875	\\
0.407333422115683	-0.000732421875	\\
0.407377813290718	-0.001129150390625	\\
0.407422204465752	-0.00067138671875	\\
0.407466595640787	-0.00030517578125	\\
0.407510986815821	-0.000335693359375	\\
0.407555377990855	-0.0003662109375	\\
0.40759976916589	-0.0009765625	\\
0.407644160340924	-0.001129150390625	\\
0.407688551515959	-0.0009765625	\\
0.407732942690993	-0.000946044921875	\\
0.407777333866027	-0.00079345703125	\\
0.407821725041062	-0.000457763671875	\\
0.407866116216096	0.00042724609375	\\
0.407910507391131	0.0009765625	\\
0.407954898566165	0.00146484375	\\
0.407999289741199	0.001922607421875	\\
0.408043680916234	0.001800537109375	\\
0.408088072091268	0.001861572265625	\\
0.408132463266303	0.001556396484375	\\
0.408176854441337	0.001678466796875	\\
0.408221245616371	0.001617431640625	\\
0.408265636791406	0.002044677734375	\\
0.40831002796644	0.002105712890625	\\
0.408354419141475	0.001953125	\\
0.408398810316509	0.002593994140625	\\
0.408443201491544	0.0029296875	\\
0.408487592666578	0.002960205078125	\\
0.408531983841612	0.003387451171875	\\
0.408576375016647	0.003662109375	\\
0.408620766191681	0.00335693359375	\\
0.408665157366715	0.003082275390625	\\
0.40870954854175	0.002471923828125	\\
0.408753939716784	0.002227783203125	\\
0.408798330891819	0.002105712890625	\\
0.408842722066853	0.0015869140625	\\
0.408887113241888	0.001068115234375	\\
0.408931504416922	0.001220703125	\\
0.408975895591956	0.00115966796875	\\
0.409020286766991	0.00152587890625	\\
0.409064677942025	0.001739501953125	\\
0.40910906911706	0.001678466796875	\\
0.409153460292094	0.001708984375	\\
0.409197851467128	0.001251220703125	\\
0.409242242642163	0.000946044921875	\\
0.409286633817197	0.000274658203125	\\
0.409331024992232	-0.000274658203125	\\
0.409375416167266	-0.000213623046875	\\
0.4094198073423	-0.000762939453125	\\
0.409464198517335	-0.00079345703125	\\
0.409508589692369	-0.00018310546875	\\
0.409552980867404	-0.000244140625	\\
0.409597372042438	-0.00018310546875	\\
0.409641763217472	0.0001220703125	\\
0.409686154392507	0.00048828125	\\
0.409730545567541	0.0003662109375	\\
0.409774936742576	0	\\
0.40981932791761	-0.000213623046875	\\
0.409863719092644	-9.1552734375e-05	\\
0.409908110267679	-0.00048828125	\\
0.409952501442713	-0.000762939453125	\\
0.409996892617748	-0.001220703125	\\
0.410041283792782	-0.0013427734375	\\
0.410085674967816	-0.000213623046875	\\
0.410130066142851	0.0001220703125	\\
0.410174457317885	0.000152587890625	\\
0.41021884849292	0.0010986328125	\\
0.410263239667954	0.001220703125	\\
0.410307630842988	0.001312255859375	\\
0.410352022018023	0.001739501953125	\\
0.410396413193057	0.001373291015625	\\
0.410440804368092	0.00091552734375	\\
0.410485195543126	0.000579833984375	\\
0.41052958671816	0.000335693359375	\\
0.410573977893195	0.00030517578125	\\
0.410618369068229	0.000518798828125	\\
0.410662760243264	0.000274658203125	\\
0.410707151418298	0.00042724609375	\\
0.410751542593332	0.00042724609375	\\
0.410795933768367	6.103515625e-05	\\
0.410840324943401	0.0001220703125	\\
0.410884716118436	0.00018310546875	\\
0.41092910729347	0.00048828125	\\
0.410973498468504	0.000244140625	\\
0.411017889643539	-3.0517578125e-05	\\
0.411062280818573	-0.000152587890625	\\
0.411106671993608	-0.000335693359375	\\
0.411151063168642	-0.00030517578125	\\
0.411195454343676	-0.000762939453125	\\
0.411239845518711	-0.000762939453125	\\
0.411284236693745	-0.00079345703125	\\
0.41132862786878	-0.000457763671875	\\
0.411373019043814	-0.0001220703125	\\
0.411417410218848	-0.00042724609375	\\
0.411461801393883	-0.0009765625	\\
0.411506192568917	-0.0010986328125	\\
0.411550583743952	-0.000762939453125	\\
0.411594974918986	-0.000885009765625	\\
0.411639366094021	-0.0009765625	\\
0.411683757269055	-0.00054931640625	\\
0.411728148444089	-0.000152587890625	\\
0.411772539619124	-0.0001220703125	\\
0.411816930794158	0.000640869140625	\\
0.411861321969193	0.000946044921875	\\
0.411905713144227	0.000946044921875	\\
0.411950104319261	0.0013427734375	\\
0.411994495494296	0.001373291015625	\\
0.41203888666933	0.001678466796875	\\
0.412083277844365	0.001678466796875	\\
0.412127669019399	0.0010986328125	\\
0.412172060194433	0.000701904296875	\\
0.412216451369468	0.000762939453125	\\
0.412260842544502	0.00140380859375	\\
0.412305233719537	0.00177001953125	\\
0.412349624894571	0.00128173828125	\\
0.412394016069605	0.00128173828125	\\
0.41243840724464	0.000946044921875	\\
0.412482798419674	0.0008544921875	\\
0.412527189594709	0.000762939453125	\\
0.412571580769743	0.00091552734375	\\
0.412615971944777	0.000640869140625	\\
0.412660363119812	0.00030517578125	\\
0.412704754294846	-0.0001220703125	\\
0.412749145469881	-6.103515625e-05	\\
0.412793536644915	-0.000244140625	\\
0.412837927819949	-0.00079345703125	\\
0.412882318994984	-0.000762939453125	\\
0.412926710170018	-0.000762939453125	\\
0.412971101345053	-0.001007080078125	\\
0.413015492520087	-0.001251220703125	\\
0.413059883695121	-0.00091552734375	\\
0.413104274870156	-0.000823974609375	\\
0.41314866604519	-0.0008544921875	\\
0.413193057220225	-0.000762939453125	\\
0.413237448395259	-0.0009765625	\\
0.413281839570293	-0.000762939453125	\\
0.413326230745328	-0.00067138671875	\\
0.413370621920362	-0.000946044921875	\\
0.413415013095397	-0.000823974609375	\\
0.413459404270431	-0.001190185546875	\\
0.413503795445465	-0.001129150390625	\\
0.4135481866205	-0.0008544921875	\\
0.413592577795534	-0.000823974609375	\\
0.413636968970569	-0.00103759765625	\\
0.413681360145603	-0.0010986328125	\\
0.413725751320637	-0.000640869140625	\\
0.413770142495672	-0.00048828125	\\
0.413814533670706	-0.000762939453125	\\
0.413858924845741	-0.000946044921875	\\
0.413903316020775	-0.000732421875	\\
0.413947707195809	-0.00030517578125	\\
0.413992098370844	-0.00054931640625	\\
0.414036489545878	-0.0006103515625	\\
0.414080880720913	-0.000701904296875	\\
0.414125271895947	-0.00079345703125	\\
0.414169663070981	-0.000518798828125	\\
0.414214054246016	-0.0003662109375	\\
0.41425844542105	0	\\
0.414302836596085	0.0003662109375	\\
0.414347227771119	0.000152587890625	\\
0.414391618946154	-0.0001220703125	\\
0.414436010121188	-0.0001220703125	\\
0.414480401296222	-0.000701904296875	\\
0.414524792471257	-0.00091552734375	\\
0.414569183646291	-0.001190185546875	\\
0.414613574821326	-0.00103759765625	\\
0.41465796599636	-0.000732421875	\\
0.414702357171394	-0.0009765625	\\
0.414746748346429	-0.0008544921875	\\
0.414791139521463	-0.000396728515625	\\
0.414835530696498	-6.103515625e-05	\\
0.414879921871532	-0.000152587890625	\\
0.414924313046566	-0.000640869140625	\\
0.414968704221601	-0.00048828125	\\
0.415013095396635	-0.00054931640625	\\
0.41505748657167	-0.0006103515625	\\
0.415101877746704	-0.000640869140625	\\
0.415146268921738	-0.00079345703125	\\
0.415190660096773	-0.00079345703125	\\
0.415235051271807	-0.000396728515625	\\
0.415279442446842	-3.0517578125e-05	\\
0.415323833621876	-3.0517578125e-05	\\
0.41536822479691	-6.103515625e-05	\\
0.415412615971945	0.0003662109375	\\
0.415457007146979	0.00030517578125	\\
0.415501398322014	0.000152587890625	\\
0.415545789497048	0.000335693359375	\\
0.415590180672082	-3.0517578125e-05	\\
0.415634571847117	-0.000396728515625	\\
0.415678963022151	-0.00048828125	\\
0.415723354197186	-0.00018310546875	\\
0.41576774537222	-0.000274658203125	\\
0.415812136547254	-0.000518798828125	\\
0.415856527722289	0.000244140625	\\
0.415900918897323	0.0003662109375	\\
0.415945310072358	0.000244140625	\\
0.415989701247392	0.000396728515625	\\
0.416034092422426	0.000152587890625	\\
0.416078483597461	-0.000274658203125	\\
0.416122874772495	-0.00018310546875	\\
0.41616726594753	-0.000335693359375	\\
0.416211657122564	-0.0006103515625	\\
0.416256048297598	-0.0006103515625	\\
0.416300439472633	-0.000732421875	\\
0.416344830647667	-0.00103759765625	\\
0.416389221822702	-0.000762939453125	\\
0.416433612997736	-0.00103759765625	\\
0.41647800417277	-0.001129150390625	\\
0.416522395347805	-0.00103759765625	\\
0.416566786522839	-0.00115966796875	\\
0.416611177697874	-0.001068115234375	\\
0.416655568872908	-0.00115966796875	\\
0.416699960047942	-0.001373291015625	\\
0.416744351222977	-0.001739501953125	\\
0.416788742398011	-0.0020751953125	\\
0.416833133573046	-0.002105712890625	\\
0.41687752474808	-0.002197265625	\\
0.416921915923115	-0.00238037109375	\\
0.416966307098149	-0.002105712890625	\\
0.417010698273183	-0.002227783203125	\\
0.417055089448218	-0.002197265625	\\
0.417099480623252	-0.00152587890625	\\
0.417143871798286	-0.00152587890625	\\
0.417188262973321	-0.001373291015625	\\
0.417232654148355	-0.000885009765625	\\
0.41727704532339	-0.0008544921875	\\
0.417321436498424	-0.001617431640625	\\
0.417365827673459	-0.00164794921875	\\
0.417410218848493	-0.001617431640625	\\
0.417454610023527	-0.001434326171875	\\
0.417499001198562	-0.000823974609375	\\
0.417543392373596	-0.000885009765625	\\
0.417587783548631	-0.000762939453125	\\
0.417632174723665	0.0001220703125	\\
0.417676565898699	0.00054931640625	\\
0.417720957073734	0.000579833984375	\\
0.417765348248768	0.00067138671875	\\
0.417809739423803	0.000701904296875	\\
0.417854130598837	0.00067138671875	\\
0.417898521773871	0.000640869140625	\\
0.417942912948906	0.0008544921875	\\
0.41798730412394	0.000396728515625	\\
0.418031695298975	0.0003662109375	\\
0.418076086474009	0.000518798828125	\\
0.418120477649043	0.000274658203125	\\
0.418164868824078	0.000885009765625	\\
0.418209259999112	0.0008544921875	\\
0.418253651174147	0.000457763671875	\\
0.418298042349181	0.00091552734375	\\
0.418342433524215	0.000701904296875	\\
0.41838682469925	-3.0517578125e-05	\\
0.418431215874284	-0.000274658203125	\\
0.418475607049319	-0.000823974609375	\\
0.418519998224353	-0.000732421875	\\
0.418564389399387	-0.00103759765625	\\
0.418608780574422	-0.00164794921875	\\
0.418653171749456	-0.001495361328125	\\
0.418697562924491	-0.000885009765625	\\
0.418741954099525	-0.000640869140625	\\
0.418786345274559	-0.00079345703125	\\
0.418830736449594	-0.00067138671875	\\
0.418875127624628	-0.0009765625	\\
0.418919518799663	-0.001220703125	\\
0.418963909974697	-0.001739501953125	\\
0.419008301149731	-0.00164794921875	\\
0.419052692324766	-0.001861572265625	\\
0.4190970834998	-0.00213623046875	\\
0.419141474674835	-0.001617431640625	\\
0.419185865849869	-0.001373291015625	\\
0.419230257024903	-0.00115966796875	\\
0.419274648199938	-0.00079345703125	\\
0.419319039374972	-0.000518798828125	\\
0.419363430550007	-0.000701904296875	\\
0.419407821725041	-0.000457763671875	\\
0.419452212900075	-0.0003662109375	\\
0.41949660407511	-0.0006103515625	\\
0.419540995250144	-0.00018310546875	\\
0.419585386425179	0.0001220703125	\\
0.419629777600213	0	\\
0.419674168775247	0.000640869140625	\\
0.419718559950282	0.001007080078125	\\
0.419762951125316	0.00079345703125	\\
0.419807342300351	0.001007080078125	\\
0.419851733475385	0.001007080078125	\\
0.419896124650419	0.0010986328125	\\
0.419940515825454	0.00140380859375	\\
0.419984907000488	0.001708984375	\\
0.420029298175523	0.001556396484375	\\
0.420073689350557	0.001251220703125	\\
0.420118080525592	0.00103759765625	\\
0.420162471700626	0.0006103515625	\\
0.42020686287566	0.000640869140625	\\
0.420251254050695	0.00048828125	\\
0.420295645225729	0.000244140625	\\
0.420340036400764	0.00030517578125	\\
0.420384427575798	-0.0001220703125	\\
0.420428818750832	-0.00030517578125	\\
0.420473209925867	-0.0001220703125	\\
0.420517601100901	-0.000701904296875	\\
0.420561992275936	-0.0008544921875	\\
0.42060638345097	-0.001007080078125	\\
0.420650774626004	-0.000885009765625	\\
0.420695165801039	-0.000885009765625	\\
0.420739556976073	-0.000946044921875	\\
0.420783948151108	-0.000946044921875	\\
0.420828339326142	-0.000946044921875	\\
0.420872730501176	-0.001129150390625	\\
0.420917121676211	-0.001434326171875	\\
0.420961512851245	-0.00128173828125	\\
0.42100590402628	-0.001251220703125	\\
0.421050295201314	-0.001556396484375	\\
0.421094686376348	-0.00128173828125	\\
0.421139077551383	-0.00115966796875	\\
0.421183468726417	-0.0013427734375	\\
0.421227859901452	-0.000518798828125	\\
0.421272251076486	-0.00048828125	\\
0.42131664225152	-0.00067138671875	\\
0.421361033426555	-0.0006103515625	\\
0.421405424601589	-0.0006103515625	\\
0.421449815776624	-0.00042724609375	\\
0.421494206951658	-0.001068115234375	\\
0.421538598126692	-0.0006103515625	\\
0.421582989301727	0.0001220703125	\\
0.421627380476761	3.0517578125e-05	\\
0.421671771651796	0.000335693359375	\\
0.42171616282683	0.0003662109375	\\
0.421760554001864	-0.00018310546875	\\
0.421804945176899	-0.00018310546875	\\
0.421849336351933	9.1552734375e-05	\\
0.421893727526968	-0.0001220703125	\\
0.421938118702002	-0.000213623046875	\\
0.421982509877036	-0.000274658203125	\\
0.422026901052071	-0.000732421875	\\
0.422071292227105	-0.0006103515625	\\
0.42211568340214	-0.000274658203125	\\
0.422160074577174	0.0001220703125	\\
0.422204465752208	-9.1552734375e-05	\\
0.422248856927243	-0.000152587890625	\\
0.422293248102277	-0.0001220703125	\\
0.422337639277312	0.000244140625	\\
0.422382030452346	9.1552734375e-05	\\
0.42242642162738	-0.000274658203125	\\
0.422470812802415	-0.000152587890625	\\
0.422515203977449	-0.000152587890625	\\
0.422559595152484	-0.0006103515625	\\
0.422603986327518	-0.000640869140625	\\
0.422648377502553	-0.000732421875	\\
0.422692768677587	-0.0009765625	\\
0.422737159852621	-0.00115966796875	\\
0.422781551027656	-0.001434326171875	\\
0.42282594220269	-0.001220703125	\\
0.422870333377725	-0.0010986328125	\\
0.422914724552759	-0.001312255859375	\\
0.422959115727793	-0.001251220703125	\\
0.423003506902828	-0.00146484375	\\
0.423047898077862	-0.001434326171875	\\
0.423092289252897	-0.00128173828125	\\
0.423136680427931	-0.0010986328125	\\
0.423181071602965	-0.0008544921875	\\
0.423225462778	-0.0008544921875	\\
0.423269853953034	-0.000518798828125	\\
0.423314245128069	-0.000457763671875	\\
0.423358636303103	-0.000701904296875	\\
0.423403027478137	-0.0003662109375	\\
0.423447418653172	-0.00030517578125	\\
0.423491809828206	-0.000152587890625	\\
0.423536201003241	-3.0517578125e-05	\\
0.423580592178275	-0.00018310546875	\\
0.423624983353309	6.103515625e-05	\\
0.423669374528344	0.000152587890625	\\
0.423713765703378	0.000518798828125	\\
0.423758156878413	0.00048828125	\\
0.423802548053447	0.000518798828125	\\
0.423846939228481	0.00030517578125	\\
0.423891330403516	0.000213623046875	\\
0.42393572157855	0.000518798828125	\\
0.423980112753585	0.00054931640625	\\
0.424024503928619	0.000335693359375	\\
0.424068895103653	0.0003662109375	\\
0.424113286278688	0.00042724609375	\\
0.424157677453722	0.00030517578125	\\
0.424202068628757	0	\\
0.424246459803791	0.000457763671875	\\
0.424290850978825	0.000701904296875	\\
0.42433524215386	0.00042724609375	\\
0.424379633328894	0.00030517578125	\\
0.424424024503929	3.0517578125e-05	\\
0.424468415678963	0.000244140625	\\
0.424512806853997	0.0001220703125	\\
0.424557198029032	-0.0001220703125	\\
0.424601589204066	-0.000244140625	\\
0.424645980379101	-0.000579833984375	\\
0.424690371554135	-0.000335693359375	\\
0.424734762729169	-0.0003662109375	\\
0.424779153904204	-0.0003662109375	\\
0.424823545079238	0.000335693359375	\\
0.424867936254273	6.103515625e-05	\\
0.424912327429307	0.000152587890625	\\
0.424956718604341	0.000213623046875	\\
0.425001109779376	3.0517578125e-05	\\
0.42504550095441	3.0517578125e-05	\\
0.425089892129445	-0.00042724609375	\\
0.425134283304479	-0.00030517578125	\\
0.425178674479513	-0.00048828125	\\
0.425223065654548	-0.000274658203125	\\
0.425267456829582	-0.000396728515625	\\
0.425311848004617	-0.000579833984375	\\
0.425356239179651	-0.0008544921875	\\
0.425400630354686	-0.000579833984375	\\
0.42544502152972	-0.000396728515625	\\
0.425489412704754	-0.000823974609375	\\
0.425533803879789	-0.0003662109375	\\
0.425578195054823	-0.000396728515625	\\
0.425622586229857	-0.0006103515625	\\
0.425666977404892	-0.000335693359375	\\
0.425711368579926	-0.000396728515625	\\
0.425755759754961	-0.00030517578125	\\
0.425800150929995	6.103515625e-05	\\
0.42584454210503	0.000396728515625	\\
0.425888933280064	0.000152587890625	\\
0.425933324455098	0.00018310546875	\\
0.425977715630133	0.000274658203125	\\
0.426022106805167	-0.00018310546875	\\
0.426066497980202	-0.000579833984375	\\
0.426110889155236	-0.000457763671875	\\
0.42615528033027	-0.00042724609375	\\
0.426199671505305	-0.000396728515625	\\
0.426244062680339	-9.1552734375e-05	\\
0.426288453855374	-3.0517578125e-05	\\
0.426332845030408	0.000274658203125	\\
0.426377236205442	0.0006103515625	\\
0.426421627380477	0.00054931640625	\\
0.426466018555511	0.00018310546875	\\
0.426510409730546	-6.103515625e-05	\\
0.42655480090558	-0.000244140625	\\
0.426599192080614	-0.000518798828125	\\
0.426643583255649	-0.00067138671875	\\
0.426687974430683	-0.000732421875	\\
0.426732365605718	-0.000640869140625	\\
0.426776756780752	-0.000885009765625	\\
0.426821147955786	-0.000701904296875	\\
0.426865539130821	-0.000152587890625	\\
0.426909930305855	-0.0001220703125	\\
0.42695432148089	0.00018310546875	\\
0.426998712655924	0.000579833984375	\\
0.427043103830958	0.000823974609375	\\
0.427087495005993	0.000732421875	\\
0.427131886181027	0.000396728515625	\\
0.427176277356062	0.000701904296875	\\
0.427220668531096	0.0008544921875	\\
0.42726505970613	0.001007080078125	\\
0.427309450881165	0.001220703125	\\
0.427353842056199	0.0015869140625	\\
0.427398233231234	0.001678466796875	\\
0.427442624406268	0.001617431640625	\\
0.427487015581302	0.001861572265625	\\
0.427531406756337	0.00213623046875	\\
0.427575797931371	0.002166748046875	\\
0.427620189106406	0.002593994140625	\\
0.42766458028144	0.002471923828125	\\
0.427708971456474	0.002105712890625	\\
0.427753362631509	0.002227783203125	\\
0.427797753806543	0.00244140625	\\
0.427842144981578	0.002105712890625	\\
0.427886536156612	0.001739501953125	\\
0.427930927331646	0.001983642578125	\\
0.427975318506681	0.001861572265625	\\
0.428019709681715	0.002288818359375	\\
0.42806410085675	0.0023193359375	\\
0.428108492031784	0.001861572265625	\\
0.428152883206818	0.001953125	\\
0.428197274381853	0.001678466796875	\\
0.428241665556887	0.001495361328125	\\
0.428286056731922	0.001129150390625	\\
0.428330447906956	0.0006103515625	\\
0.42837483908199	0.000640869140625	\\
0.428419230257025	0.00018310546875	\\
0.428463621432059	9.1552734375e-05	\\
0.428508012607094	0.000274658203125	\\
0.428552403782128	3.0517578125e-05	\\
0.428596794957163	0.00018310546875	\\
0.428641186132197	6.103515625e-05	\\
0.428685577307231	-0.000457763671875	\\
0.428729968482266	-0.000640869140625	\\
0.4287743596573	-0.000579833984375	\\
0.428818750832335	-0.000335693359375	\\
0.428863142007369	-0.00018310546875	\\
0.428907533182403	-0.00030517578125	\\
0.428951924357438	-0.000396728515625	\\
0.428996315532472	-0.0003662109375	\\
0.429040706707507	-0.000335693359375	\\
0.429085097882541	3.0517578125e-05	\\
0.429129489057575	0.00067138671875	\\
0.42917388023261	0.0008544921875	\\
0.429218271407644	0.0008544921875	\\
0.429262662582679	0.000457763671875	\\
0.429307053757713	3.0517578125e-05	\\
0.429351444932747	-0.000213623046875	\\
0.429395836107782	-0.00054931640625	\\
0.429440227282816	-0.000732421875	\\
0.429484618457851	-0.00048828125	\\
0.429529009632885	-0.000396728515625	\\
0.429573400807919	-0.00018310546875	\\
0.429617791982954	-0.000274658203125	\\
0.429662183157988	0	\\
0.429706574333023	0.000274658203125	\\
0.429750965508057	-0.000213623046875	\\
0.429795356683091	-3.0517578125e-05	\\
0.429839747858126	0	\\
0.42988413903316	-0.00054931640625	\\
0.429928530208195	-0.000762939453125	\\
0.429972921383229	-0.00048828125	\\
0.430017312558263	-0.000152587890625	\\
0.430061703733298	-0.0001220703125	\\
0.430106094908332	-0.00067138671875	\\
0.430150486083367	-0.0006103515625	\\
0.430194877258401	-0.00054931640625	\\
0.430239268433435	9.1552734375e-05	\\
0.43028365960847	-9.1552734375e-05	\\
0.430328050783504	-0.0003662109375	\\
0.430372441958539	6.103515625e-05	\\
0.430416833133573	-0.00042724609375	\\
0.430461224308607	-0.00018310546875	\\
0.430505615483642	-0.000274658203125	\\
0.430550006658676	-0.000579833984375	\\
0.430594397833711	-0.00030517578125	\\
0.430638789008745	-0.000732421875	\\
0.430683180183779	-0.00103759765625	\\
0.430727571358814	-0.00048828125	\\
0.430771962533848	-0.0003662109375	\\
0.430816353708883	-0.0003662109375	\\
0.430860744883917	0	\\
0.430905136058951	-0.000244140625	\\
0.430949527233986	9.1552734375e-05	\\
0.43099391840902	0.000274658203125	\\
0.431038309584055	-0.00018310546875	\\
0.431082700759089	0	\\
0.431127091934124	0	\\
0.431171483109158	-0.000335693359375	\\
0.431215874284192	-0.0003662109375	\\
0.431260265459227	-9.1552734375e-05	\\
0.431304656634261	-0.000335693359375	\\
0.431349047809296	-0.00054931640625	\\
0.43139343898433	-0.000457763671875	\\
0.431437830159364	-0.000457763671875	\\
0.431482221334399	-0.000335693359375	\\
0.431526612509433	-0.0001220703125	\\
0.431571003684468	-0.000274658203125	\\
0.431615394859502	-0.0003662109375	\\
0.431659786034536	-6.103515625e-05	\\
0.431704177209571	0.0001220703125	\\
0.431748568384605	0.000518798828125	\\
0.43179295955964	0.000335693359375	\\
0.431837350734674	-0.00018310546875	\\
0.431881741909708	-9.1552734375e-05	\\
0.431926133084743	9.1552734375e-05	\\
0.431970524259777	-0.000274658203125	\\
0.432014915434812	0	\\
0.432059306609846	-0.00042724609375	\\
0.43210369778488	-0.00048828125	\\
0.432148088959915	-6.103515625e-05	\\
0.432192480134949	-0.000518798828125	\\
0.432236871309984	-0.00030517578125	\\
0.432281262485018	-0.000244140625	\\
0.432325653660052	-0.00054931640625	\\
0.432370044835087	-0.000732421875	\\
0.432414436010121	-0.0003662109375	\\
0.432458827185156	-0.00030517578125	\\
0.43250321836019	-0.00079345703125	\\
0.432547609535224	-0.000823974609375	\\
0.432592000710259	-0.000946044921875	\\
0.432636391885293	-0.0018310546875	\\
0.432680783060328	-0.001861572265625	\\
0.432725174235362	-0.00177001953125	\\
0.432769565410396	-0.001495361328125	\\
0.432813956585431	-0.001434326171875	\\
0.432858347760465	-0.001190185546875	\\
0.4329027389355	-0.00128173828125	\\
0.432947130110534	-0.001495361328125	\\
0.432991521285568	-0.00146484375	\\
0.433035912460603	-0.00164794921875	\\
0.433080303635637	-0.001922607421875	\\
0.433124694810672	-0.002105712890625	\\
0.433169085985706	-0.001708984375	\\
0.43321347716074	-0.001922607421875	\\
0.433257868335775	-0.001495361328125	\\
0.433302259510809	-0.00140380859375	\\
0.433346650685844	-0.00128173828125	\\
0.433391041860878	-0.00103759765625	\\
0.433435433035912	-0.0013427734375	\\
0.433479824210947	-0.00115966796875	\\
0.433524215385981	-0.001251220703125	\\
0.433568606561016	-0.0009765625	\\
0.43361299773605	-0.001129150390625	\\
0.433657388911084	-0.00103759765625	\\
0.433701780086119	-0.000396728515625	\\
0.433746171261153	-0.000823974609375	\\
0.433790562436188	-0.00103759765625	\\
0.433834953611222	-0.001129150390625	\\
0.433879344786257	-0.00103759765625	\\
0.433923735961291	-0.001373291015625	\\
0.433968127136325	-0.001617431640625	\\
0.43401251831136	-0.00146484375	\\
0.434056909486394	-0.001251220703125	\\
0.434101300661428	-0.0009765625	\\
0.434145691836463	-0.00103759765625	\\
0.434190083011497	-0.001129150390625	\\
0.434234474186532	-0.00128173828125	\\
0.434278865361566	-0.00091552734375	\\
0.434323256536601	-0.000732421875	\\
0.434367647711635	-0.00103759765625	\\
0.434412038886669	-0.001220703125	\\
0.434456430061704	-0.00079345703125	\\
0.434500821236738	-0.0009765625	\\
0.434545212411773	-0.0009765625	\\
0.434589603586807	-0.00140380859375	\\
0.434633994761841	-0.001251220703125	\\
0.434678385936876	-0.0010986328125	\\
0.43472277711191	-0.00128173828125	\\
0.434767168286945	-0.001251220703125	\\
0.434811559461979	-0.0013427734375	\\
0.434855950637013	-0.00128173828125	\\
0.434900341812048	-0.0010986328125	\\
0.434944732987082	-0.000885009765625	\\
0.434989124162117	-0.001068115234375	\\
0.435033515337151	-0.001434326171875	\\
0.435077906512185	-0.001708984375	\\
0.43512229768722	-0.00152587890625	\\
0.435166688862254	-0.0018310546875	\\
0.435211080037289	-0.001922607421875	\\
0.435255471212323	-0.0020751953125	\\
0.435299862387357	-0.00238037109375	\\
0.435344253562392	-0.002105712890625	\\
0.435388644737426	-0.00213623046875	\\
0.435433035912461	-0.002044677734375	\\
0.435477427087495	-0.002197265625	\\
0.435521818262529	-0.00189208984375	\\
0.435566209437564	-0.00146484375	\\
0.435610600612598	-0.001251220703125	\\
0.435654991787633	-0.000579833984375	\\
0.435699382962667	-0.001007080078125	\\
0.435743774137701	-0.001251220703125	\\
0.435788165312736	-0.0008544921875	\\
0.43583255648777	-0.000946044921875	\\
0.435876947662805	-0.0009765625	\\
0.435921338837839	-0.000274658203125	\\
0.435965730012873	-0.000274658203125	\\
0.436010121187908	-0.0003662109375	\\
0.436054512362942	-0.000335693359375	\\
0.436098903537977	-0.000640869140625	\\
0.436143294713011	-0.000457763671875	\\
0.436187685888045	-0.000640869140625	\\
0.43623207706308	-0.001220703125	\\
0.436276468238114	-0.001312255859375	\\
0.436320859413149	-0.001220703125	\\
0.436365250588183	-0.00164794921875	\\
0.436409641763217	-0.00201416015625	\\
0.436454032938252	-0.00146484375	\\
0.436498424113286	-0.001617431640625	\\
0.436542815288321	-0.001953125	\\
0.436587206463355	-0.001953125	\\
0.436631597638389	-0.002105712890625	\\
0.436675988813424	-0.00152587890625	\\
0.436720379988458	-0.0013427734375	\\
0.436764771163493	-0.001312255859375	\\
0.436809162338527	-0.001220703125	\\
0.436853553513561	-0.00164794921875	\\
0.436897944688596	-0.001556396484375	\\
0.43694233586363	-0.0010986328125	\\
0.436986727038665	-0.001129150390625	\\
0.437031118213699	-0.001434326171875	\\
0.437075509388734	-0.001312255859375	\\
0.437119900563768	-0.00146484375	\\
0.437164291738802	-0.001251220703125	\\
0.437208682913837	-0.00115966796875	\\
0.437253074088871	-0.001068115234375	\\
0.437297465263906	-0.001251220703125	\\
0.43734185643894	-0.001220703125	\\
0.437386247613974	-0.00128173828125	\\
0.437430638789009	-0.001617431640625	\\
0.437475029964043	-0.000823974609375	\\
0.437519421139078	-0.0006103515625	\\
0.437563812314112	-0.001617431640625	\\
0.437608203489146	-0.001220703125	\\
0.437652594664181	-0.0006103515625	\\
0.437696985839215	-0.00091552734375	\\
0.43774137701425	-0.000732421875	\\
0.437785768189284	-0.000732421875	\\
0.437830159364318	-0.000823974609375	\\
0.437874550539353	-0.00048828125	\\
0.437918941714387	-0.000823974609375	\\
0.437963332889422	-0.00103759765625	\\
0.438007724064456	-0.000762939453125	\\
0.43805211523949	-0.0009765625	\\
0.438096506414525	-0.00128173828125	\\
0.438140897589559	-0.001556396484375	\\
0.438185288764594	-0.00146484375	\\
0.438229679939628	-0.00103759765625	\\
0.438274071114662	-0.000946044921875	\\
0.438318462289697	-0.000701904296875	\\
0.438362853464731	-0.00054931640625	\\
0.438407244639766	-0.000946044921875	\\
0.4384516358148	-0.0010986328125	\\
0.438496026989834	-0.001190185546875	\\
0.438540418164869	-0.001495361328125	\\
0.438584809339903	-0.00146484375	\\
0.438629200514938	-0.001861572265625	\\
0.438673591689972	-0.002532958984375	\\
0.438717982865006	-0.002410888671875	\\
0.438762374040041	-0.002471923828125	\\
0.438806765215075	-0.002227783203125	\\
0.43885115639011	-0.0025634765625	\\
0.438895547565144	-0.0030517578125	\\
0.438939938740178	-0.002685546875	\\
0.438984329915213	-0.002716064453125	\\
0.439028721090247	-0.00274658203125	\\
0.439073112265282	-0.002899169921875	\\
0.439117503440316	-0.0028076171875	\\
0.43916189461535	-0.0029296875	\\
0.439206285790385	-0.002838134765625	\\
0.439250676965419	-0.00238037109375	\\
0.439295068140454	-0.0030517578125	\\
0.439339459315488	-0.003173828125	\\
0.439383850490522	-0.00286865234375	\\
0.439428241665557	-0.0028076171875	\\
0.439472632840591	-0.003082275390625	\\
0.439517024015626	-0.003509521484375	\\
0.43956141519066	-0.00274658203125	\\
0.439605806365695	-0.002410888671875	\\
0.439650197540729	-0.00238037109375	\\
0.439694588715763	-0.002716064453125	\\
0.439738979890798	-0.002777099609375	\\
0.439783371065832	-0.00286865234375	\\
0.439827762240867	-0.00262451171875	\\
0.439872153415901	-0.002288818359375	\\
0.439916544590935	-0.002288818359375	\\
0.43996093576597	-0.0020751953125	\\
0.440005326941004	-0.00274658203125	\\
0.440049718116039	-0.00262451171875	\\
0.440094109291073	-0.00250244140625	\\
0.440138500466107	-0.00274658203125	\\
0.440182891641142	-0.002685546875	\\
0.440227282816176	-0.0028076171875	\\
0.440271673991211	-0.002716064453125	\\
0.440316065166245	-0.002685546875	\\
0.440360456341279	-0.002960205078125	\\
0.440404847516314	-0.003265380859375	\\
0.440449238691348	-0.00311279296875	\\
0.440493629866383	-0.002777099609375	\\
0.440538021041417	-0.00250244140625	\\
0.440582412216451	-0.00274658203125	\\
0.440626803391486	-0.003021240234375	\\
0.44067119456652	-0.0028076171875	\\
0.440715585741555	-0.002227783203125	\\
0.440759976916589	-0.00250244140625	\\
0.440804368091623	-0.003143310546875	\\
0.440848759266658	-0.002593994140625	\\
0.440893150441692	-0.002532958984375	\\
0.440937541616727	-0.002685546875	\\
0.440981932791761	-0.002349853515625	\\
0.441026323966795	-0.00201416015625	\\
0.44107071514183	-0.002197265625	\\
0.441115106316864	-0.002532958984375	\\
0.441159497491899	-0.001922607421875	\\
0.441203888666933	-0.002044677734375	\\
0.441248279841967	-0.00177001953125	\\
0.441292671017002	-0.00152587890625	\\
0.441337062192036	-0.001251220703125	\\
0.441381453367071	-0.00048828125	\\
0.441425844542105	-0.000701904296875	\\
0.441470235717139	-0.0009765625	\\
0.441514626892174	-0.001068115234375	\\
0.441559018067208	-0.00091552734375	\\
0.441603409242243	-0.00054931640625	\\
0.441647800417277	-0.0006103515625	\\
0.441692191592311	-0.0008544921875	\\
0.441736582767346	-0.00091552734375	\\
0.44178097394238	-0.000885009765625	\\
0.441825365117415	-0.000946044921875	\\
0.441869756292449	-0.001068115234375	\\
0.441914147467483	-0.001373291015625	\\
0.441958538642518	-0.0015869140625	\\
0.442002929817552	-0.001739501953125	\\
0.442047320992587	-0.00152587890625	\\
0.442091712167621	-0.001953125	\\
0.442136103342655	-0.002166748046875	\\
0.44218049451769	-0.001708984375	\\
0.442224885692724	-0.00177001953125	\\
0.442269276867759	-0.001556396484375	\\
0.442313668042793	-0.001708984375	\\
0.442358059217828	-0.001953125	\\
0.442402450392862	-0.002197265625	\\
0.442446841567896	-0.001800537109375	\\
0.442491232742931	-0.001556396484375	\\
0.442535623917965	-0.00213623046875	\\
0.442580015092999	-0.002197265625	\\
0.442624406268034	-0.002410888671875	\\
0.442668797443068	-0.0023193359375	\\
0.442713188618103	-0.002197265625	\\
0.442757579793137	-0.0023193359375	\\
0.442801970968172	-0.0020751953125	\\
0.442846362143206	-0.00213623046875	\\
0.44289075331824	-0.0018310546875	\\
0.442935144493275	-0.002166748046875	\\
0.442979535668309	-0.0020751953125	\\
0.443023926843344	-0.00213623046875	\\
0.443068318018378	-0.002105712890625	\\
0.443112709193412	-0.001678466796875	\\
0.443157100368447	-0.001861572265625	\\
0.443201491543481	-0.001953125	\\
0.443245882718516	-0.001678466796875	\\
0.44329027389355	-0.0010986328125	\\
0.443334665068584	-0.00115966796875	\\
0.443379056243619	-0.0013427734375	\\
0.443423447418653	-0.00128173828125	\\
0.443467838593688	-0.001068115234375	\\
0.443512229768722	-0.000762939453125	\\
0.443556620943756	-0.001007080078125	\\
0.443601012118791	-0.001068115234375	\\
0.443645403293825	-0.001495361328125	\\
0.44368979446886	-0.0015869140625	\\
0.443734185643894	-0.001129150390625	\\
0.443778576818928	-0.000946044921875	\\
0.443822967993963	-0.001373291015625	\\
0.443867359168997	-0.00146484375	\\
0.443911750344032	-0.001129150390625	\\
0.443956141519066	-0.0009765625	\\
0.4440005326941	-0.0009765625	\\
0.444044923869135	-0.00140380859375	\\
0.444089315044169	-0.001190185546875	\\
0.444133706219204	-0.001373291015625	\\
0.444178097394238	-0.001861572265625	\\
0.444222488569272	-0.001922607421875	\\
0.444266879744307	-0.001739501953125	\\
0.444311270919341	-0.001983642578125	\\
0.444355662094376	-0.00213623046875	\\
0.44440005326941	-0.002105712890625	\\
0.444444444444444	-0.002166748046875	\\
0.444488835619479	-0.0018310546875	\\
0.444533226794513	-0.001678466796875	\\
0.444577617969548	-0.002044677734375	\\
0.444622009144582	-0.002105712890625	\\
0.444666400319616	-0.002471923828125	\\
0.444710791494651	-0.0025634765625	\\
0.444755182669685	-0.00238037109375	\\
0.44479957384472	-0.00250244140625	\\
0.444843965019754	-0.002471923828125	\\
0.444888356194789	-0.002716064453125	\\
0.444932747369823	-0.002685546875	\\
0.444977138544857	-0.002685546875	\\
0.445021529719892	-0.00244140625	\\
0.445065920894926	-0.002197265625	\\
0.44511031206996	-0.00238037109375	\\
0.445154703244995	-0.002197265625	\\
0.445199094420029	-0.0020751953125	\\
0.445243485595064	-0.0018310546875	\\
0.445287876770098	-0.002227783203125	\\
0.445332267945133	-0.00201416015625	\\
0.445376659120167	-0.001678466796875	\\
0.445421050295201	-0.00201416015625	\\
0.445465441470236	-0.001678466796875	\\
0.44550983264527	-0.0013427734375	\\
0.445554223820305	-0.0018310546875	\\
0.445598614995339	-0.00201416015625	\\
0.445643006170373	-0.001678466796875	\\
0.445687397345408	-0.001678466796875	\\
0.445731788520442	-0.00152587890625	\\
0.445776179695477	-0.0015869140625	\\
0.445820570870511	-0.00189208984375	\\
0.445864962045545	-0.00177001953125	\\
0.44590935322058	-0.00164794921875	\\
0.445953744395614	-0.0013427734375	\\
0.445998135570649	-0.001312255859375	\\
0.446042526745683	-0.001708984375	\\
0.446086917920717	-0.001434326171875	\\
0.446131309095752	-0.00115966796875	\\
0.446175700270786	-0.0015869140625	\\
0.446220091445821	-0.001678466796875	\\
0.446264482620855	-0.00177001953125	\\
0.446308873795889	-0.001861572265625	\\
0.446353264970924	-0.00189208984375	\\
0.446397656145958	-0.001617431640625	\\
0.446442047320993	-0.001434326171875	\\
0.446486438496027	-0.001678466796875	\\
0.446530829671061	-0.001617431640625	\\
0.446575220846096	-0.00140380859375	\\
0.44661961202113	-0.00128173828125	\\
0.446664003196165	-0.001129150390625	\\
0.446708394371199	-0.0010986328125	\\
0.446752785546233	-0.0010986328125	\\
0.446797176721268	-0.00115966796875	\\
0.446841567896302	-0.001373291015625	\\
0.446885959071337	-0.0013427734375	\\
0.446930350246371	-0.0008544921875	\\
0.446974741421405	-0.000457763671875	\\
0.44701913259644	-0.0001220703125	\\
0.447063523771474	-9.1552734375e-05	\\
0.447107914946509	-0.0001220703125	\\
0.447152306121543	6.103515625e-05	\\
0.447196697296577	0.0003662109375	\\
0.447241088471612	-0.000396728515625	\\
0.447285479646646	-0.000701904296875	\\
0.447329870821681	-0.000152587890625	\\
0.447374261996715	-0.000396728515625	\\
0.447418653171749	-0.000213623046875	\\
0.447463044346784	0.000213623046875	\\
0.447507435521818	-0.0006103515625	\\
0.447551826696853	-0.00103759765625	\\
0.447596217871887	-0.00048828125	\\
0.447640609046921	-0.0009765625	\\
0.447685000221956	-0.00146484375	\\
0.44772939139699	-0.001251220703125	\\
0.447773782572025	-0.00164794921875	\\
0.447818173747059	-0.001556396484375	\\
0.447862564922093	-0.001495361328125	\\
0.447906956097128	-0.001983642578125	\\
0.447951347272162	-0.001953125	\\
0.447995738447197	-0.0015869140625	\\
0.448040129622231	-0.001922607421875	\\
0.448084520797266	-0.00225830078125	\\
0.4481289119723	-0.002227783203125	\\
0.448173303147334	-0.0025634765625	\\
0.448217694322369	-0.00274658203125	\\
0.448262085497403	-0.002777099609375	\\
0.448306476672438	-0.00262451171875	\\
0.448350867847472	-0.002960205078125	\\
0.448395259022506	-0.003204345703125	\\
0.448439650197541	-0.002899169921875	\\
0.448484041372575	-0.002716064453125	\\
0.44852843254761	-0.002227783203125	\\
0.448572823722644	-0.002105712890625	\\
0.448617214897678	-0.001953125	\\
0.448661606072713	-0.002227783203125	\\
0.448705997247747	-0.002166748046875	\\
0.448750388422782	-0.002105712890625	\\
0.448794779597816	-0.00250244140625	\\
0.44883917077285	-0.002288818359375	\\
0.448883561947885	-0.001953125	\\
0.448927953122919	-0.001678466796875	\\
0.448972344297954	-0.001068115234375	\\
0.449016735472988	-0.000823974609375	\\
0.449061126648022	-0.000732421875	\\
0.449105517823057	-0.000213623046875	\\
0.449149908998091	-0.00018310546875	\\
0.449194300173126	-0.00042724609375	\\
0.44923869134816	-0.000274658203125	\\
0.449283082523194	-0.000213623046875	\\
0.449327473698229	-0.00048828125	\\
0.449371864873263	-6.103515625e-05	\\
0.449416256048298	-0.0001220703125	\\
0.449460647223332	-0.000244140625	\\
0.449505038398366	0.000335693359375	\\
0.449549429573401	9.1552734375e-05	\\
0.449593820748435	0.000274658203125	\\
0.44963821192347	0.000701904296875	\\
0.449682603098504	0.0003662109375	\\
0.449726994273538	0.00018310546875	\\
0.449771385448573	0.000152587890625	\\
0.449815776623607	-0.000396728515625	\\
0.449860167798642	-0.00054931640625	\\
0.449904558973676	-0.000152587890625	\\
0.44994895014871	-6.103515625e-05	\\
0.449993341323745	-0.000457763671875	\\
0.450037732498779	-0.00030517578125	\\
0.450082123673814	-3.0517578125e-05	\\
0.450126514848848	-0.0003662109375	\\
0.450170906023882	-0.000457763671875	\\
0.450215297198917	-0.000579833984375	\\
0.450259688373951	-0.000640869140625	\\
0.450304079548986	-0.00054931640625	\\
0.45034847072402	-0.000701904296875	\\
0.450392861899054	-0.000732421875	\\
0.450437253074089	-0.000946044921875	\\
0.450481644249123	-0.00067138671875	\\
0.450526035424158	-0.000579833984375	\\
0.450570426599192	-0.000823974609375	\\
0.450614817774226	-0.00079345703125	\\
0.450659208949261	-0.000732421875	\\
0.450703600124295	-0.000885009765625	\\
0.45074799129933	-0.000823974609375	\\
0.450792382474364	-0.00067138671875	\\
0.450836773649399	-0.000640869140625	\\
0.450881164824433	-0.000213623046875	\\
0.450925555999467	-3.0517578125e-05	\\
0.450969947174502	-0.000244140625	\\
0.451014338349536	-0.00018310546875	\\
0.45105872952457	0.00054931640625	\\
0.451103120699605	0.00067138671875	\\
0.451147511874639	0.00091552734375	\\
0.451191903049674	0.001220703125	\\
0.451236294224708	0.001556396484375	\\
0.451280685399743	0.0013427734375	\\
0.451325076574777	0.000823974609375	\\
0.451369467749811	0.000518798828125	\\
0.451413858924846	0.000518798828125	\\
0.45145825009988	0.0008544921875	\\
0.451502641274915	0.000732421875	\\
0.451547032449949	0.0006103515625	\\
0.451591423624983	0.0008544921875	\\
0.451635814800018	0.001007080078125	\\
0.451680205975052	0.000946044921875	\\
0.451724597150087	0.001312255859375	\\
0.451768988325121	0.001220703125	\\
0.451813379500155	0.00128173828125	\\
0.45185777067519	0.000885009765625	\\
0.451902161850224	0.0009765625	\\
0.451946553025259	0.00091552734375	\\
0.451990944200293	0.000762939453125	\\
0.452035335375327	0.000823974609375	\\
0.452079726550362	0.00054931640625	\\
0.452124117725396	0.00067138671875	\\
0.452168508900431	0.0008544921875	\\
0.452212900075465	0.00079345703125	\\
0.452257291250499	0.0008544921875	\\
0.452301682425534	0.0006103515625	\\
0.452346073600568	0.00030517578125	\\
0.452390464775603	0.000213623046875	\\
0.452434855950637	3.0517578125e-05	\\
0.452479247125671	-0.000274658203125	\\
0.452523638300706	-0.0001220703125	\\
0.45256802947574	0.000335693359375	\\
0.452612420650775	0.000274658203125	\\
0.452656811825809	0.0003662109375	\\
0.452701203000843	0.000732421875	\\
0.452745594175878	0.0006103515625	\\
0.452789985350912	0.0003662109375	\\
0.452834376525947	0.000244140625	\\
0.452878767700981	0.000152587890625	\\
0.452923158876015	-3.0517578125e-05	\\
0.45296755005105	0.00030517578125	\\
0.453011941226084	0.00048828125	\\
0.453056332401119	0.0001220703125	\\
0.453100723576153	0.000518798828125	\\
0.453145114751187	0.000335693359375	\\
0.453189505926222	0.000244140625	\\
0.453233897101256	-0.000152587890625	\\
0.453278288276291	-0.00042724609375	\\
0.453322679451325	-0.000152587890625	\\
0.45336707062636	-0.00018310546875	\\
0.453411461801394	-0.0001220703125	\\
0.453455852976428	-0.000152587890625	\\
0.453500244151463	-0.000244140625	\\
0.453544635326497	-0.000579833984375	\\
0.453589026501531	-0.000823974609375	\\
0.453633417676566	-0.000823974609375	\\
0.4536778088516	-0.00115966796875	\\
0.453722200026635	-0.001068115234375	\\
0.453766591201669	-0.00091552734375	\\
0.453810982376704	-0.001190185546875	\\
0.453855373551738	-0.001251220703125	\\
0.453899764726772	-0.00146484375	\\
0.453944155901807	-0.00128173828125	\\
0.453988547076841	-0.00115966796875	\\
0.454032938251876	-0.001373291015625	\\
0.45407732942691	-0.00140380859375	\\
0.454121720601944	-0.00128173828125	\\
0.454166111776979	-0.00091552734375	\\
0.454210502952013	-0.001007080078125	\\
0.454254894127048	-0.0010986328125	\\
0.454299285302082	-0.0010986328125	\\
0.454343676477116	-0.001068115234375	\\
0.454388067652151	-0.000762939453125	\\
0.454432458827185	-0.00079345703125	\\
0.45447685000222	-0.000732421875	\\
0.454521241177254	-0.000335693359375	\\
0.454565632352288	-0.0006103515625	\\
0.454610023527323	-0.000518798828125	\\
0.454654414702357	0.000244140625	\\
0.454698805877392	0.00042724609375	\\
0.454743197052426	0.000244140625	\\
0.45478758822746	0.000152587890625	\\
0.454831979402495	0.000732421875	\\
0.454876370577529	0.00048828125	\\
0.454920761752564	9.1552734375e-05	\\
0.454965152927598	0	\\
0.455009544102632	9.1552734375e-05	\\
0.455053935277667	0.00048828125	\\
0.455098326452701	0.000762939453125	\\
0.455142717627736	0.000701904296875	\\
0.45518710880277	0.000274658203125	\\
0.455231499977804	0.000274658203125	\\
0.455275891152839	0.0006103515625	\\
0.455320282327873	0.00115966796875	\\
0.455364673502908	0.001251220703125	\\
0.455409064677942	0.00054931640625	\\
0.455453455852976	-6.103515625e-05	\\
0.455497847028011	-0.00018310546875	\\
0.455542238203045	-0.0006103515625	\\
0.45558662937808	-0.00079345703125	\\
0.455631020553114	-0.0008544921875	\\
0.455675411728148	-0.001312255859375	\\
0.455719802903183	-0.001312255859375	\\
0.455764194078217	-0.001495361328125	\\
0.455808585253252	-0.0018310546875	\\
0.455852976428286	-0.001678466796875	\\
0.45589736760332	-0.001617431640625	\\
0.455941758778355	-0.002105712890625	\\
0.455986149953389	-0.00238037109375	\\
0.456030541128424	-0.002288818359375	\\
0.456074932303458	-0.002410888671875	\\
0.456119323478492	-0.00244140625	\\
0.456163714653527	-0.00274658203125	\\
0.456208105828561	-0.002593994140625	\\
0.456252497003596	-0.002288818359375	\\
0.45629688817863	-0.00244140625	\\
0.456341279353664	-0.002288818359375	\\
0.456385670528699	-0.002410888671875	\\
0.456430061703733	-0.002410888671875	\\
0.456474452878768	-0.001953125	\\
0.456518844053802	-0.00201416015625	\\
0.456563235228837	-0.001861572265625	\\
0.456607626403871	-0.00146484375	\\
0.456652017578905	-0.0013427734375	\\
0.45669640875394	-0.00103759765625	\\
0.456740799928974	-0.001220703125	\\
0.456785191104009	-0.001312255859375	\\
0.456829582279043	-0.0009765625	\\
0.456873973454077	-0.000762939453125	\\
0.456918364629112	-0.0001220703125	\\
0.456962755804146	0.000152587890625	\\
0.457007146979181	0.000152587890625	\\
0.457051538154215	0.00054931640625	\\
0.457095929329249	0.000518798828125	\\
0.457140320504284	0.00018310546875	\\
0.457184711679318	0.0001220703125	\\
0.457229102854353	9.1552734375e-05	\\
0.457273494029387	-0.000213623046875	\\
0.457317885204421	-0.000335693359375	\\
0.457362276379456	-0.00054931640625	\\
0.45740666755449	-0.00079345703125	\\
0.457451058729525	-0.000579833984375	\\
0.457495449904559	-0.000579833984375	\\
0.457539841079593	-0.000946044921875	\\
0.457584232254628	-0.000457763671875	\\
0.457628623429662	-0.0008544921875	\\
0.457673014604697	-0.001068115234375	\\
0.457717405779731	-0.001251220703125	\\
0.457761796954765	-0.001922607421875	\\
0.4578061881298	-0.001922607421875	\\
0.457850579304834	-0.001861572265625	\\
0.457894970479869	-0.00189208984375	\\
0.457939361654903	-0.002044677734375	\\
0.457983752829937	-0.002044677734375	\\
0.458028144004972	-0.001953125	\\
0.458072535180006	-0.001861572265625	\\
0.458116926355041	-0.001983642578125	\\
0.458161317530075	-0.002044677734375	\\
0.458205708705109	-0.001251220703125	\\
0.458250099880144	-0.0008544921875	\\
0.458294491055178	-0.001190185546875	\\
0.458338882230213	-0.001373291015625	\\
0.458383273405247	-0.001190185546875	\\
0.458427664580281	-0.001129150390625	\\
0.458472055755316	-0.00115966796875	\\
0.45851644693035	-0.001068115234375	\\
0.458560838105385	-0.00042724609375	\\
0.458605229280419	-0.0001220703125	\\
0.458649620455453	0.000274658203125	\\
0.458694011630488	0.00091552734375	\\
0.458738402805522	0.001068115234375	\\
0.458782793980557	0.0006103515625	\\
0.458827185155591	0.0010986328125	\\
0.458871576330625	0.00079345703125	\\
0.45891596750566	0.0006103515625	\\
0.458960358680694	0.0006103515625	\\
0.459004749855729	0.000396728515625	\\
0.459049141030763	0.000579833984375	\\
0.459093532205797	0.00042724609375	\\
0.459137923380832	-0.0001220703125	\\
0.459182314555866	-0.00054931640625	\\
0.459226705730901	-0.00042724609375	\\
0.459271096905935	-0.000518798828125	\\
0.45931548808097	-0.00048828125	\\
0.459359879256004	-0.00030517578125	\\
0.459404270431038	-0.000244140625	\\
0.459448661606073	-0.000152587890625	\\
0.459493052781107	-6.103515625e-05	\\
0.459537443956142	-0.00054931640625	\\
0.459581835131176	-0.001068115234375	\\
0.45962622630621	-0.00128173828125	\\
0.459670617481245	-0.00164794921875	\\
0.459715008656279	-0.00146484375	\\
0.459759399831314	-0.0015869140625	\\
0.459803791006348	-0.001373291015625	\\
0.459848182181382	-0.001373291015625	\\
0.459892573356417	-0.001678466796875	\\
0.459936964531451	-0.001617431640625	\\
0.459981355706486	-0.001800537109375	\\
0.46002574688152	-0.001922607421875	\\
0.460070138056554	-0.0015869140625	\\
0.460114529231589	-0.001617431640625	\\
0.460158920406623	-0.001983642578125	\\
0.460203311581658	-0.001434326171875	\\
0.460247702756692	-0.001556396484375	\\
0.460292093931726	-0.0013427734375	\\
0.460336485106761	-0.00091552734375	\\
0.460380876281795	-0.000457763671875	\\
0.46042526745683	-0.00048828125	\\
0.460469658631864	-0.00018310546875	\\
0.460514049806898	0.000396728515625	\\
0.460558440981933	0.00048828125	\\
0.460602832156967	0.000213623046875	\\
0.460647223332002	-0.000244140625	\\
0.460691614507036	-0.000152587890625	\\
0.46073600568207	-0.00030517578125	\\
0.460780396857105	-6.103515625e-05	\\
0.460824788032139	3.0517578125e-05	\\
0.460869179207174	-0.000213623046875	\\
0.460913570382208	6.103515625e-05	\\
0.460957961557242	0.000457763671875	\\
0.461002352732277	0.00048828125	\\
0.461046743907311	0.00048828125	\\
0.461091135082346	0.000732421875	\\
0.46113552625738	0.000213623046875	\\
0.461179917432414	-0.000213623046875	\\
0.461224308607449	-0.000213623046875	\\
0.461268699782483	-0.000762939453125	\\
0.461313090957518	-0.00030517578125	\\
0.461357482132552	-0.000274658203125	\\
0.461401873307586	-0.000732421875	\\
0.461446264482621	-0.000396728515625	\\
0.461490655657655	0.000274658203125	\\
0.46153504683269	0.000457763671875	\\
0.461579438007724	9.1552734375e-05	\\
0.461623829182758	-0.00018310546875	\\
0.461668220357793	-0.000701904296875	\\
0.461712611532827	-0.000579833984375	\\
0.461757002707862	-0.000701904296875	\\
0.461801393882896	-0.000732421875	\\
0.461845785057931	-0.000732421875	\\
0.461890176232965	-0.001220703125	\\
0.461934567407999	-0.001220703125	\\
0.461978958583034	-0.00091552734375	\\
0.462023349758068	-0.00091552734375	\\
0.462067740933102	-0.00079345703125	\\
0.462112132108137	-0.000732421875	\\
0.462156523283171	-0.00103759765625	\\
0.462200914458206	-0.000885009765625	\\
0.46224530563324	-0.000701904296875	\\
0.462289696808275	-0.000701904296875	\\
0.462334087983309	-0.0008544921875	\\
0.462378479158343	-0.001251220703125	\\
0.462422870333378	-0.00140380859375	\\
0.462467261508412	-0.0010986328125	\\
0.462511652683447	-0.001556396484375	\\
0.462556043858481	-0.001312255859375	\\
0.462600435033515	-0.0009765625	\\
0.46264482620855	-0.00115966796875	\\
0.462689217383584	-0.001129150390625	\\
0.462733608558619	-0.001312255859375	\\
0.462777999733653	-0.00152587890625	\\
0.462822390908687	-0.0018310546875	\\
0.462866782083722	-0.001373291015625	\\
0.462911173258756	-0.001861572265625	\\
0.462955564433791	-0.001708984375	\\
0.462999955608825	-0.001312255859375	\\
0.463044346783859	-0.001953125	\\
0.463088737958894	-0.001678466796875	\\
0.463133129133928	-0.001861572265625	\\
0.463177520308963	-0.002166748046875	\\
0.463221911483997	-0.002197265625	\\
0.463266302659031	-0.00213623046875	\\
0.463310693834066	-0.002838134765625	\\
0.4633550850091	-0.002960205078125	\\
0.463399476184135	-0.002532958984375	\\
0.463443867359169	-0.00250244140625	\\
0.463488258534203	-0.00262451171875	\\
0.463532649709238	-0.00286865234375	\\
0.463577040884272	-0.002655029296875	\\
0.463621432059307	-0.0025634765625	\\
0.463665823234341	-0.0023193359375	\\
0.463710214409375	-0.002105712890625	\\
0.46375460558441	-0.00262451171875	\\
0.463798996759444	-0.00225830078125	\\
0.463843387934479	-0.002288818359375	\\
0.463887779109513	-0.00250244140625	\\
0.463932170284547	-0.00262451171875	\\
0.463976561459582	-0.002899169921875	\\
0.464020952634616	-0.00250244140625	\\
0.464065343809651	-0.00244140625	\\
0.464109734984685	-0.00238037109375	\\
0.464154126159719	-0.00244140625	\\
0.464198517334754	-0.002471923828125	\\
0.464242908509788	-0.001678466796875	\\
0.464287299684823	-0.001983642578125	\\
0.464331690859857	-0.0023193359375	\\
0.464376082034891	-0.002349853515625	\\
0.464420473209926	-0.002227783203125	\\
0.46446486438496	-0.002227783203125	\\
0.464509255559995	-0.002044677734375	\\
0.464553646735029	-0.001739501953125	\\
0.464598037910063	-0.001434326171875	\\
0.464642429085098	-0.00152587890625	\\
0.464686820260132	-0.001678466796875	\\
0.464731211435167	-0.001251220703125	\\
0.464775602610201	-0.000640869140625	\\
0.464819993785235	-0.000885009765625	\\
0.46486438496027	-0.00103759765625	\\
0.464908776135304	-0.00091552734375	\\
0.464953167310339	-0.00152587890625	\\
0.464997558485373	-0.001708984375	\\
0.465041949660408	-0.00140380859375	\\
0.465086340835442	-0.001617431640625	\\
0.465130732010476	-0.001617431640625	\\
0.465175123185511	-0.001251220703125	\\
0.465219514360545	-0.001495361328125	\\
0.46526390553558	-0.001861572265625	\\
0.465308296710614	-0.002166748046875	\\
0.465352687885648	-0.002349853515625	\\
0.465397079060683	-0.002227783203125	\\
0.465441470235717	-0.002288818359375	\\
0.465485861410752	-0.00238037109375	\\
0.465530252585786	-0.002105712890625	\\
0.46557464376082	-0.001983642578125	\\
0.465619034935855	-0.002044677734375	\\
0.465663426110889	-0.001678466796875	\\
0.465707817285924	-0.001220703125	\\
0.465752208460958	-0.00140380859375	\\
0.465796599635992	-0.001068115234375	\\
0.465840990811027	-0.0009765625	\\
0.465885381986061	-0.001220703125	\\
0.465929773161096	-0.00115966796875	\\
0.46597416433613	-0.001068115234375	\\
0.466018555511164	-0.000335693359375	\\
0.466062946686199	6.103515625e-05	\\
0.466107337861233	-9.1552734375e-05	\\
0.466151729036268	-0.000335693359375	\\
0.466196120211302	-0.000213623046875	\\
0.466240511386336	0.000152587890625	\\
0.466284902561371	-0.00018310546875	\\
0.466329293736405	-0.0001220703125	\\
0.46637368491144	6.103515625e-05	\\
0.466418076086474	0.00054931640625	\\
0.466462467261508	0.00054931640625	\\
0.466506858436543	0.000396728515625	\\
0.466551249611577	0.001220703125	\\
0.466595640786612	0.001434326171875	\\
0.466640031961646	0.001220703125	\\
0.46668442313668	0.001312255859375	\\
0.466728814311715	0.00115966796875	\\
0.466773205486749	0.00091552734375	\\
0.466817596661784	0.000762939453125	\\
0.466861987836818	0.000823974609375	\\
0.466906379011852	0.000762939453125	\\
0.466950770186887	0.000701904296875	\\
0.466995161361921	0.000640869140625	\\
0.467039552536956	0.000732421875	\\
0.46708394371199	0.000946044921875	\\
0.467128334887024	0.00067138671875	\\
0.467172726062059	0.00054931640625	\\
0.467217117237093	0.0001220703125	\\
0.467261508412128	6.103515625e-05	\\
0.467305899587162	-9.1552734375e-05	\\
0.467350290762196	0.000152587890625	\\
0.467394681937231	0.000732421875	\\
0.467439073112265	0.000396728515625	\\
0.4674834642873	0.000152587890625	\\
0.467527855462334	-0.00042724609375	\\
0.467572246637369	9.1552734375e-05	\\
0.467616637812403	0.000213623046875	\\
0.467661028987437	0.000152587890625	\\
0.467705420162472	0.00048828125	\\
0.467749811337506	-0.0001220703125	\\
0.467794202512541	0.00018310546875	\\
0.467838593687575	0.0003662109375	\\
0.467882984862609	0.00030517578125	\\
0.467927376037644	0.001068115234375	\\
0.467971767212678	0.0010986328125	\\
0.468016158387713	0.000885009765625	\\
0.468060549562747	0.0009765625	\\
0.468104940737781	0.001312255859375	\\
0.468149331912816	0.0015869140625	\\
0.46819372308785	0.001373291015625	\\
0.468238114262885	0.00103759765625	\\
0.468282505437919	0.000946044921875	\\
0.468326896612953	0.001434326171875	\\
0.468371287787988	0.00164794921875	\\
0.468415678963022	0.0015869140625	\\
0.468460070138057	0.002197265625	\\
0.468504461313091	0.002471923828125	\\
0.468548852488125	0.00250244140625	\\
0.46859324366316	0.002410888671875	\\
0.468637634838194	0.001739501953125	\\
0.468682026013229	0.001739501953125	\\
0.468726417188263	0.001190185546875	\\
0.468770808363297	0.0010986328125	\\
0.468815199538332	0.001190185546875	\\
0.468859590713366	0.001190185546875	\\
0.468903981888401	0.00152587890625	\\
0.468948373063435	0.00164794921875	\\
0.468992764238469	0.00146484375	\\
0.469037155413504	0.00103759765625	\\
0.469081546588538	0.0008544921875	\\
0.469125937763573	0.0008544921875	\\
0.469170328938607	0.000823974609375	\\
0.469214720113641	0.000946044921875	\\
0.469259111288676	0.00067138671875	\\
0.46930350246371	0.000335693359375	\\
0.469347893638745	0.000244140625	\\
0.469392284813779	9.1552734375e-05	\\
0.469436675988813	0.0003662109375	\\
0.469481067163848	9.1552734375e-05	\\
0.469525458338882	0.00018310546875	\\
0.469569849513917	-3.0517578125e-05	\\
0.469614240688951	-0.0001220703125	\\
0.469658631863985	0.000244140625	\\
0.46970302303902	0.00030517578125	\\
0.469747414214054	0.00042724609375	\\
0.469791805389089	0.000579833984375	\\
0.469836196564123	0.000274658203125	\\
0.469880587739157	0.00018310546875	\\
0.469924978914192	0.0003662109375	\\
0.469969370089226	0.000244140625	\\
0.470013761264261	-0.000274658203125	\\
0.470058152439295	-0.000213623046875	\\
0.470102543614329	0.000274658203125	\\
0.470146934789364	0.0006103515625	\\
0.470191325964398	0.000701904296875	\\
0.470235717139433	0.000946044921875	\\
0.470280108314467	0.001190185546875	\\
0.470324499489502	0.0015869140625	\\
0.470368890664536	0.0015869140625	\\
0.47041328183957	0.00146484375	\\
0.470457673014605	0.001556396484375	\\
0.470502064189639	0.001251220703125	\\
0.470546455364673	0.00115966796875	\\
0.470590846539708	0.001068115234375	\\
0.470635237714742	0.0009765625	\\
0.470679628889777	0.001190185546875	\\
0.470724020064811	0.001129150390625	\\
0.470768411239846	0.001495361328125	\\
0.47081280241488	0.00115966796875	\\
0.470857193589914	0.000823974609375	\\
0.470901584764949	0.000823974609375	\\
0.470945975939983	0.000518798828125	\\
0.470990367115018	0.000640869140625	\\
0.471034758290052	0.001007080078125	\\
0.471079149465086	0.0003662109375	\\
0.471123540640121	-3.0517578125e-05	\\
0.471167931815155	-0.00030517578125	\\
0.47121232299019	-0.00030517578125	\\
0.471256714165224	0.00030517578125	\\
0.471301105340258	0.000946044921875	\\
0.471345496515293	0.00042724609375	\\
0.471389887690327	0.000152587890625	\\
0.471434278865362	0.000244140625	\\
0.471478670040396	-0.000152587890625	\\
0.47152306121543	-0.00018310546875	\\
0.471567452390465	-0.000457763671875	\\
0.471611843565499	-0.000335693359375	\\
0.471656234740534	-6.103515625e-05	\\
0.471700625915568	-0.0003662109375	\\
0.471745017090602	-0.000244140625	\\
0.471789408265637	0.0001220703125	\\
0.471833799440671	-9.1552734375e-05	\\
0.471878190615706	-0.000152587890625	\\
0.47192258179074	-0.000335693359375	\\
0.471966972965774	-0.000518798828125	\\
0.472011364140809	-0.000335693359375	\\
0.472055755315843	-0.000152587890625	\\
0.472100146490878	-3.0517578125e-05	\\
0.472144537665912	-0.000274658203125	\\
0.472188928840946	-0.000274658203125	\\
0.472233320015981	-0.00048828125	\\
0.472277711191015	-0.00048828125	\\
0.47232210236605	-0.0003662109375	\\
0.472366493541084	3.0517578125e-05	\\
0.472410884716118	-9.1552734375e-05	\\
0.472455275891153	0.000152587890625	\\
0.472499667066187	0.000732421875	\\
0.472544058241222	0.000732421875	\\
0.472588449416256	0.000457763671875	\\
0.47263284059129	0.0003662109375	\\
0.472677231766325	0.000701904296875	\\
0.472721622941359	0.00067138671875	\\
0.472766014116394	0.00091552734375	\\
0.472810405291428	0.000885009765625	\\
0.472854796466462	0.00042724609375	\\
0.472899187641497	0.000640869140625	\\
0.472943578816531	0.000732421875	\\
0.472987969991566	0.000823974609375	\\
0.4730323611666	0.000762939453125	\\
0.473076752341634	0.000762939453125	\\
0.473121143516669	0.001251220703125	\\
0.473165534691703	0.00103759765625	\\
0.473209925866738	0.0010986328125	\\
0.473254317041772	0.000946044921875	\\
0.473298708216806	0.00054931640625	\\
0.473343099391841	0.000579833984375	\\
0.473387490566875	0.00042724609375	\\
0.47343188174191	0.000396728515625	\\
0.473476272916944	0.00054931640625	\\
0.473520664091979	0.000640869140625	\\
0.473565055267013	0.000640869140625	\\
0.473609446442047	0.000762939453125	\\
0.473653837617082	0.000640869140625	\\
0.473698228792116	0.000244140625	\\
0.473742619967151	0.000274658203125	\\
0.473787011142185	0.000518798828125	\\
0.473831402317219	0.00079345703125	\\
0.473875793492254	0.000823974609375	\\
0.473920184667288	0.00067138671875	\\
0.473964575842323	0.001434326171875	\\
0.474008967017357	0.00146484375	\\
0.474053358192391	0.001190185546875	\\
0.474097749367426	0.001800537109375	\\
0.47414214054246	0.00213623046875	\\
0.474186531717495	0.002166748046875	\\
0.474230922892529	0.002227783203125	\\
0.474275314067563	0.002166748046875	\\
0.474319705242598	0.002716064453125	\\
0.474364096417632	0.002655029296875	\\
0.474408487592667	0.002288818359375	\\
0.474452878767701	0.002716064453125	\\
0.474497269942735	0.00262451171875	\\
0.47454166111777	0.002532958984375	\\
0.474586052292804	0.002532958984375	\\
0.474630443467839	0.002532958984375	\\
0.474674834642873	0.002685546875	\\
0.474719225817907	0.00311279296875	\\
0.474763616992942	0.0030517578125	\\
0.474808008167976	0.0029296875	\\
0.474852399343011	0.002960205078125	\\
0.474896790518045	0.00286865234375	\\
0.474941181693079	0.002655029296875	\\
0.474985572868114	0.002532958984375	\\
0.475029964043148	0.0025634765625	\\
0.475074355218183	0.0023193359375	\\
0.475118746393217	0.00201416015625	\\
0.475163137568251	0.002166748046875	\\
0.475207528743286	0.001953125	\\
0.47525191991832	0.001739501953125	\\
0.475296311093355	0.001922607421875	\\
0.475340702268389	0.0018310546875	\\
0.475385093443423	0.00177001953125	\\
0.475429484618458	0.00177001953125	\\
0.475473875793492	0.001922607421875	\\
0.475518266968527	0.002471923828125	\\
0.475562658143561	0.002593994140625	\\
0.475607049318595	0.00201416015625	\\
0.47565144049363	0.001678466796875	\\
0.475695831668664	0.001922607421875	\\
0.475740222843699	0.001861572265625	\\
0.475784614018733	0.00189208984375	\\
0.475829005193767	0.002471923828125	\\
0.475873396368802	0.002197265625	\\
0.475917787543836	0.00250244140625	\\
0.475962178718871	0.002716064453125	\\
0.476006569893905	0.00274658203125	\\
0.47605096106894	0.00311279296875	\\
0.476095352243974	0.00323486328125	\\
0.476139743419008	0.003204345703125	\\
0.476184134594043	0.003326416015625	\\
0.476228525769077	0.00360107421875	\\
0.476272916944112	0.003692626953125	\\
0.476317308119146	0.00372314453125	\\
0.47636169929418	0.003753662109375	\\
0.476406090469215	0.003997802734375	\\
0.476450481644249	0.0042724609375	\\
0.476494872819284	0.004638671875	\\
0.476539263994318	0.00433349609375	\\
0.476583655169352	0.004241943359375	\\
0.476628046344387	0.00408935546875	\\
0.476672437519421	0.004241943359375	\\
0.476716828694456	0.0045166015625	\\
0.47676121986949	0.00408935546875	\\
0.476805611044524	0.003662109375	\\
0.476850002219559	0.00323486328125	\\
0.476894393394593	0.0035400390625	\\
0.476938784569628	0.0037841796875	\\
0.476983175744662	0.003326416015625	\\
0.477027566919696	0.00360107421875	\\
0.477071958094731	0.003753662109375	\\
0.477116349269765	0.003143310546875	\\
0.4771607404448	0.002838134765625	\\
0.477205131619834	0.002899169921875	\\
0.477249522794868	0.003082275390625	\\
0.477293913969903	0.0023193359375	\\
0.477338305144937	0.001953125	\\
0.477382696319972	0.0018310546875	\\
0.477427087495006	0.0018310546875	\\
0.47747147867004	0.00177001953125	\\
0.477515869845075	0.001312255859375	\\
0.477560261020109	0.00140380859375	\\
0.477604652195144	0.001434326171875	\\
0.477649043370178	0.00177001953125	\\
0.477693434545212	0.001800537109375	\\
0.477737825720247	0.00189208984375	\\
0.477782216895281	0.002105712890625	\\
0.477826608070316	0.0023193359375	\\
0.47787099924535	0.002685546875	\\
0.477915390420384	0.002197265625	\\
0.477959781595419	0.00244140625	\\
0.478004172770453	0.00274658203125	\\
0.478048563945488	0.00250244140625	\\
0.478092955120522	0.002899169921875	\\
0.478137346295556	0.00286865234375	\\
0.478181737470591	0.00274658203125	\\
0.478226128645625	0.00286865234375	\\
0.47827051982066	0.003082275390625	\\
0.478314910995694	0.003326416015625	\\
0.478359302170728	0.003082275390625	\\
0.478403693345763	0.0030517578125	\\
0.478448084520797	0.002716064453125	\\
0.478492475695832	0.0028076171875	\\
0.478536866870866	0.0028076171875	\\
0.4785812580459	0.002777099609375	\\
0.478625649220935	0.00286865234375	\\
0.478670040395969	0.00250244140625	\\
0.478714431571004	0.003082275390625	\\
0.478758822746038	0.003021240234375	\\
0.478803213921073	0.002349853515625	\\
0.478847605096107	0.002044677734375	\\
0.478891996271141	0.00201416015625	\\
0.478936387446176	0.001953125	\\
0.47898077862121	0.001708984375	\\
0.479025169796244	0.001373291015625	\\
0.479069560971279	0.001007080078125	\\
0.479113952146313	0.001129150390625	\\
0.479158343321348	0.001129150390625	\\
0.479202734496382	0.0010986328125	\\
0.479247125671417	0.0003662109375	\\
0.479291516846451	0.000274658203125	\\
0.479335908021485	0.000335693359375	\\
0.47938029919652	-6.103515625e-05	\\
0.479424690371554	0	\\
0.479469081546589	3.0517578125e-05	\\
0.479513472721623	0.000335693359375	\\
0.479557863896657	0.000213623046875	\\
0.479602255071692	0.000152587890625	\\
0.479646646246726	0.0006103515625	\\
0.479691037421761	0.00042724609375	\\
0.479735428596795	0.00018310546875	\\
0.479779819771829	-9.1552734375e-05	\\
0.479824210946864	-0.000335693359375	\\
0.479868602121898	0.0001220703125	\\
0.479912993296933	0	\\
0.479957384471967	-0.000152587890625	\\
0.480001775647001	9.1552734375e-05	\\
0.480046166822036	-6.103515625e-05	\\
0.48009055799707	-0.000335693359375	\\
0.480134949172105	-0.00030517578125	\\
0.480179340347139	3.0517578125e-05	\\
0.480223731522173	6.103515625e-05	\\
0.480268122697208	3.0517578125e-05	\\
0.480312513872242	0.0006103515625	\\
0.480356905047277	0.000640869140625	\\
0.480401296222311	0.000946044921875	\\
0.480445687397345	0.001129150390625	\\
0.48049007857238	0.000885009765625	\\
0.480534469747414	0.0003662109375	\\
0.480578860922449	0.000518798828125	\\
0.480623252097483	0.00048828125	\\
0.480667643272517	-6.103515625e-05	\\
0.480712034447552	-6.103515625e-05	\\
0.480756425622586	-0.000823974609375	\\
0.480800816797621	-0.000396728515625	\\
0.480845207972655	0.0001220703125	\\
0.480889599147689	0	\\
0.480933990322724	0.000518798828125	\\
0.480978381497758	0.000640869140625	\\
0.481022772672793	0.000579833984375	\\
0.481067163847827	0.000396728515625	\\
0.481111555022861	-0.0001220703125	\\
0.481155946197896	-0.00048828125	\\
0.48120033737293	9.1552734375e-05	\\
0.481244728547965	3.0517578125e-05	\\
0.481289119722999	-0.000335693359375	\\
0.481333510898033	-9.1552734375e-05	\\
0.481377902073068	-0.0003662109375	\\
0.481422293248102	-0.000274658203125	\\
0.481466684423137	-0.000457763671875	\\
0.481511075598171	-0.000213623046875	\\
0.481555466773205	9.1552734375e-05	\\
0.48159985794824	-6.103515625e-05	\\
0.481644249123274	0.000396728515625	\\
0.481688640298309	0.00067138671875	\\
0.481733031473343	0.000457763671875	\\
0.481777422648378	0.0006103515625	\\
0.481821813823412	0.00103759765625	\\
0.481866204998446	0.000885009765625	\\
0.481910596173481	0.0008544921875	\\
0.481954987348515	0.00079345703125	\\
0.48199937852355	0.000518798828125	\\
0.482043769698584	0.000823974609375	\\
0.482088160873618	0.000732421875	\\
0.482132552048653	0.001068115234375	\\
0.482176943223687	0.001190185546875	\\
0.482221334398722	0.0009765625	\\
0.482265725573756	0.00152587890625	\\
0.48231011674879	0.00140380859375	\\
0.482354507923825	0.001220703125	\\
0.482398899098859	0.001007080078125	\\
0.482443290273894	0.00079345703125	\\
0.482487681448928	0.0006103515625	\\
0.482532072623962	0.000518798828125	\\
0.482576463798997	0.000732421875	\\
0.482620854974031	0.00091552734375	\\
0.482665246149066	0.000701904296875	\\
0.4827096373241	0.00091552734375	\\
0.482754028499134	0.000823974609375	\\
0.482798419674169	0.00067138671875	\\
0.482842810849203	0.00048828125	\\
0.482887202024238	0.00030517578125	\\
0.482931593199272	0.000457763671875	\\
0.482975984374306	-0.00030517578125	\\
0.483020375549341	-0.00079345703125	\\
0.483064766724375	-0.0003662109375	\\
0.48310915789941	-0.000762939453125	\\
0.483153549074444	-0.000732421875	\\
0.483197940249478	-0.000946044921875	\\
0.483242331424513	-0.00128173828125	\\
0.483286722599547	-0.00091552734375	\\
0.483331113774582	-0.00079345703125	\\
0.483375504949616	-0.000640869140625	\\
0.48341989612465	-0.000518798828125	\\
0.483464287299685	-0.000518798828125	\\
0.483508678474719	-0.00042724609375	\\
0.483553069649754	-9.1552734375e-05	\\
0.483597460824788	-3.0517578125e-05	\\
0.483641851999822	-0.000335693359375	\\
0.483686243174857	-0.000457763671875	\\
0.483730634349891	3.0517578125e-05	\\
0.483775025524926	-9.1552734375e-05	\\
0.48381941669996	-0.000274658203125	\\
0.483863807874994	9.1552734375e-05	\\
0.483908199050029	0.000244140625	\\
0.483952590225063	0.000518798828125	\\
0.483996981400098	0.000946044921875	\\
0.484041372575132	0.0008544921875	\\
0.484085763750166	0.001007080078125	\\
0.484130154925201	0.001068115234375	\\
0.484174546100235	0.001007080078125	\\
0.48421893727527	0.0010986328125	\\
0.484263328450304	0.001251220703125	\\
0.484307719625338	0.001861572265625	\\
0.484352110800373	0.0018310546875	\\
0.484396501975407	0.0018310546875	\\
0.484440893150442	0.00213623046875	\\
0.484485284325476	0.001739501953125	\\
0.484529675500511	0.00115966796875	\\
0.484574066675545	0.001556396484375	\\
0.484618457850579	0.00201416015625	\\
0.484662849025614	0.00140380859375	\\
0.484707240200648	0.00140380859375	\\
0.484751631375683	0.00115966796875	\\
0.484796022550717	0.000762939453125	\\
0.484840413725751	0.000579833984375	\\
0.484884804900786	0.000335693359375	\\
0.48492919607582	0.000213623046875	\\
0.484973587250855	0.00018310546875	\\
0.485017978425889	0	\\
0.485062369600923	-0.00018310546875	\\
0.485106760775958	-0.000701904296875	\\
0.485151151950992	-0.000885009765625	\\
0.485195543126027	-0.000701904296875	\\
0.485239934301061	-0.000579833984375	\\
0.485284325476095	-0.0010986328125	\\
0.48532871665113	-0.001617431640625	\\
0.485373107826164	-0.001861572265625	\\
0.485417499001199	-0.001708984375	\\
0.485461890176233	-0.0018310546875	\\
0.485506281351267	-0.0018310546875	\\
0.485550672526302	-0.001434326171875	\\
0.485595063701336	-0.00152587890625	\\
0.485639454876371	-0.00140380859375	\\
0.485683846051405	-0.00091552734375	\\
0.485728237226439	-0.000823974609375	\\
0.485772628401474	-0.00128173828125	\\
0.485817019576508	-0.0008544921875	\\
0.485861410751543	-0.000579833984375	\\
0.485905801926577	-0.000457763671875	\\
0.485950193101611	-0.0003662109375	\\
0.485994584276646	-0.00042724609375	\\
0.48603897545168	-0.000335693359375	\\
0.486083366626715	-9.1552734375e-05	\\
0.486127757801749	0.000274658203125	\\
0.486172148976783	0.000396728515625	\\
0.486216540151818	0.00048828125	\\
0.486260931326852	0.000274658203125	\\
0.486305322501887	0.0003662109375	\\
0.486349713676921	0.000732421875	\\
0.486394104851955	0.00067138671875	\\
0.48643849602699	0.000213623046875	\\
0.486482887202024	6.103515625e-05	\\
0.486527278377059	0.00018310546875	\\
0.486571669552093	6.103515625e-05	\\
0.486616060727127	-6.103515625e-05	\\
0.486660451902162	-0.000335693359375	\\
0.486704843077196	-0.000335693359375	\\
0.486749234252231	-0.0006103515625	\\
0.486793625427265	-0.001007080078125	\\
0.486838016602299	-0.00115966796875	\\
0.486882407777334	-0.001556396484375	\\
0.486926798952368	-0.001678466796875	\\
0.486971190127403	-0.001617431640625	\\
0.487015581302437	-0.002166748046875	\\
0.487059972477471	-0.00213623046875	\\
0.487104363652506	-0.00238037109375	\\
0.48714875482754	-0.002838134765625	\\
0.487193146002575	-0.002960205078125	\\
0.487237537177609	-0.0029296875	\\
0.487281928352644	-0.002593994140625	\\
0.487326319527678	-0.0025634765625	\\
0.487370710702712	-0.0025634765625	\\
0.487415101877747	-0.002716064453125	\\
0.487459493052781	-0.00250244140625	\\
0.487503884227815	-0.002166748046875	\\
0.48754827540285	-0.00213623046875	\\
0.487592666577884	-0.002166748046875	\\
0.487637057752919	-0.002197265625	\\
0.487681448927953	-0.00244140625	\\
0.487725840102988	-0.002532958984375	\\
0.487770231278022	-0.001953125	\\
0.487814622453056	-0.001800537109375	\\
0.487859013628091	-0.00225830078125	\\
0.487903404803125	-0.001922607421875	\\
0.48794779597816	-0.001678466796875	\\
0.487992187153194	-0.001922607421875	\\
0.488036578328228	-0.001953125	\\
0.488080969503263	-0.0023193359375	\\
0.488125360678297	-0.002166748046875	\\
0.488169751853332	-0.001953125	\\
0.488214143028366	-0.002288818359375	\\
0.4882585342034	-0.002197265625	\\
0.488302925378435	-0.00225830078125	\\
0.488347316553469	-0.002532958984375	\\
0.488391707728504	-0.00250244140625	\\
0.488436098903538	-0.002410888671875	\\
0.488480490078572	-0.002197265625	\\
0.488524881253607	-0.001861572265625	\\
0.488569272428641	-0.001708984375	\\
0.488613663603676	-0.00201416015625	\\
0.48865805477871	-0.00244140625	\\
0.488702445953744	-0.002532958984375	\\
0.488746837128779	-0.00244140625	\\
0.488791228303813	-0.0025634765625	\\
0.488835619478848	-0.00250244140625	\\
0.488880010653882	-0.00225830078125	\\
0.488924401828916	-0.002410888671875	\\
0.488968793003951	-0.002410888671875	\\
0.489013184178985	-0.002227783203125	\\
0.48905757535402	-0.002471923828125	\\
0.489101966529054	-0.0028076171875	\\
0.489146357704088	-0.00262451171875	\\
0.489190748879123	-0.0029296875	\\
0.489235140054157	-0.0029296875	\\
0.489279531229192	-0.00286865234375	\\
0.489323922404226	-0.002777099609375	\\
0.48936831357926	-0.00274658203125	\\
0.489412704754295	-0.00323486328125	\\
0.489457095929329	-0.00323486328125	\\
0.489501487104364	-0.00274658203125	\\
0.489545878279398	-0.00262451171875	\\
0.489590269454432	-0.00164794921875	\\
0.489634660629467	-0.0015869140625	\\
0.489679051804501	-0.001708984375	\\
0.489723442979536	-0.001434326171875	\\
0.48976783415457	-0.001434326171875	\\
0.489812225329605	-0.0015869140625	\\
0.489856616504639	-0.00128173828125	\\
0.489901007679673	-0.001434326171875	\\
0.489945398854708	-0.00164794921875	\\
0.489989790029742	-0.001434326171875	\\
0.490034181204776	-0.0008544921875	\\
0.490078572379811	-0.000518798828125	\\
0.490122963554845	-0.0006103515625	\\
0.49016735472988	-0.00067138671875	\\
0.490211745904914	-0.000823974609375	\\
0.490256137079949	-0.000732421875	\\
0.490300528254983	-0.00091552734375	\\
0.490344919430017	-0.00091552734375	\\
0.490389310605052	-0.000732421875	\\
0.490433701780086	-0.000885009765625	\\
0.490478092955121	-0.000885009765625	\\
0.490522484130155	-0.000823974609375	\\
0.490566875305189	-0.000732421875	\\
0.490611266480224	-0.000732421875	\\
0.490655657655258	-0.00079345703125	\\
0.490700048830293	-0.000518798828125	\\
0.490744440005327	-0.000518798828125	\\
0.490788831180361	-0.0009765625	\\
0.490833222355396	-0.000823974609375	\\
0.49087761353043	-0.000946044921875	\\
0.490922004705465	-0.00103759765625	\\
0.490966395880499	-0.000701904296875	\\
0.491010787055533	-0.00067138671875	\\
0.491055178230568	-0.001190185546875	\\
0.491099569405602	-0.0013427734375	\\
0.491143960580637	-0.001220703125	\\
0.491188351755671	-0.000823974609375	\\
0.491232742930705	-0.00018310546875	\\
0.49127713410574	-0.0003662109375	\\
0.491321525280774	-0.000457763671875	\\
0.491365916455809	-0.000213623046875	\\
0.491410307630843	-0.000457763671875	\\
0.491454698805877	-0.000457763671875	\\
0.491499089980912	-0.00030517578125	\\
0.491543481155946	-9.1552734375e-05	\\
0.491587872330981	6.103515625e-05	\\
0.491632263506015	0.00018310546875	\\
0.491676654681049	0.000335693359375	\\
0.491721045856084	0.00054931640625	\\
0.491765437031118	0.00079345703125	\\
0.491809828206153	0.00079345703125	\\
0.491854219381187	0.000762939453125	\\
0.491898610556221	0.00103759765625	\\
0.491943001731256	0.001007080078125	\\
0.49198739290629	0.000885009765625	\\
0.492031784081325	0.00164794921875	\\
0.492076175256359	0.001922607421875	\\
0.492120566431393	0.001739501953125	\\
0.492164957606428	0.001556396484375	\\
0.492209348781462	0.001251220703125	\\
0.492253739956497	0.0013427734375	\\
0.492298131131531	0.001800537109375	\\
0.492342522306565	0.001708984375	\\
0.4923869134816	0.001434326171875	\\
0.492431304656634	0.00140380859375	\\
0.492475695831669	0.001251220703125	\\
0.492520087006703	0.001190185546875	\\
0.492564478181737	0.0013427734375	\\
0.492608869356772	0.000823974609375	\\
0.492653260531806	0.000274658203125	\\
0.492697651706841	0.000335693359375	\\
0.492742042881875	-0.000244140625	\\
0.492786434056909	-0.000701904296875	\\
0.492830825231944	-0.00079345703125	\\
0.492875216406978	-0.00115966796875	\\
0.492919607582013	-0.0009765625	\\
0.492963998757047	-0.001220703125	\\
0.493008389932082	-0.001556396484375	\\
0.493052781107116	-0.001220703125	\\
0.49309717228215	-0.001312255859375	\\
0.493141563457185	-0.001220703125	\\
0.493185954632219	-0.001373291015625	\\
0.493230345807254	-0.0013427734375	\\
0.493274736982288	-0.0013427734375	\\
0.493319128157322	-0.0013427734375	\\
0.493363519332357	-0.00115966796875	\\
0.493407910507391	-0.000885009765625	\\
0.493452301682426	-0.001220703125	\\
0.49349669285746	-0.001434326171875	\\
0.493541084032494	-0.000732421875	\\
0.493585475207529	-0.000732421875	\\
0.493629866382563	-0.000579833984375	\\
0.493674257557598	-0.000701904296875	\\
0.493718648732632	6.103515625e-05	\\
0.493763039907666	0.000579833984375	\\
0.493807431082701	-3.0517578125e-05	\\
0.493851822257735	0.00042724609375	\\
0.49389621343277	0.000701904296875	\\
0.493940604607804	0.000640869140625	\\
0.493984995782838	0.00103759765625	\\
0.494029386957873	0.001373291015625	\\
0.494073778132907	0.0015869140625	\\
0.494118169307942	0.001495361328125	\\
0.494162560482976	0.001739501953125	\\
0.49420695165801	0.001739501953125	\\
0.494251342833045	0.00177001953125	\\
0.494295734008079	0.00238037109375	\\
0.494340125183114	0.0025634765625	\\
0.494384516358148	0.0025634765625	\\
0.494428907533182	0.002227783203125	\\
0.494473298708217	0.00201416015625	\\
0.494517689883251	0.0020751953125	\\
0.494562081058286	0.001617431640625	\\
0.49460647223332	0.001434326171875	\\
0.494650863408354	0.00128173828125	\\
0.494695254583389	0.000701904296875	\\
0.494739645758423	0.0003662109375	\\
0.494784036933458	0.00079345703125	\\
0.494828428108492	0.000885009765625	\\
0.494872819283526	0.00103759765625	\\
0.494917210458561	0.00103759765625	\\
0.494961601633595	0.000396728515625	\\
0.49500599280863	0.0003662109375	\\
0.495050383983664	-0.000335693359375	\\
0.495094775158698	-0.000579833984375	\\
0.495139166333733	-0.00048828125	\\
0.495183557508767	-0.00103759765625	\\
0.495227948683802	-0.0009765625	\\
0.495272339858836	-0.0008544921875	\\
0.49531673103387	-0.0010986328125	\\
0.495361122208905	-0.000762939453125	\\
0.495405513383939	-0.001007080078125	\\
0.495449904558974	-0.0009765625	\\
0.495494295734008	-0.000518798828125	\\
0.495538686909042	-0.000823974609375	\\
0.495583078084077	-0.0006103515625	\\
0.495627469259111	-0.000457763671875	\\
0.495671860434146	-0.00048828125	\\
0.49571625160918	-0.000244140625	\\
0.495760642784215	0	\\
0.495805033959249	0.000244140625	\\
0.495849425134283	0.00054931640625	\\
0.495893816309318	0.000396728515625	\\
0.495938207484352	0.000579833984375	\\
0.495982598659386	0.0009765625	\\
0.496026989834421	0.00103759765625	\\
0.496071381009455	0.00140380859375	\\
0.49611577218449	0.00152587890625	\\
0.496160163359524	0.001556396484375	\\
0.496204554534559	0.00201416015625	\\
0.496248945709593	0.001953125	\\
0.496293336884627	0.0013427734375	\\
0.496337728059662	0.00140380859375	\\
0.496382119234696	0.00146484375	\\
0.496426510409731	0.000946044921875	\\
0.496470901584765	0.00079345703125	\\
0.496515292759799	0.000640869140625	\\
0.496559683934834	0.00067138671875	\\
0.496604075109868	0.000640869140625	\\
0.496648466284903	0.000396728515625	\\
0.496692857459937	0.00042724609375	\\
0.496737248634971	-0.000244140625	\\
0.496781639810006	-0.00054931640625	\\
0.49682603098504	-0.000244140625	\\
0.496870422160075	-0.000518798828125	\\
0.496914813335109	-0.000946044921875	\\
0.496959204510143	-0.00103759765625	\\
0.497003595685178	-0.001007080078125	\\
0.497047986860212	-0.00067138671875	\\
0.497092378035247	-0.00048828125	\\
0.497136769210281	-0.000762939453125	\\
0.497181160385315	-0.00054931640625	\\
0.49722555156035	-0.000457763671875	\\
0.497269942735384	-0.000701904296875	\\
0.497314333910419	-0.000518798828125	\\
0.497358725085453	-0.00054931640625	\\
0.497403116260487	-0.00018310546875	\\
0.497447507435522	6.103515625e-05	\\
0.497491898610556	-6.103515625e-05	\\
0.497536289785591	0.00018310546875	\\
0.497580680960625	0.000244140625	\\
0.497625072135659	0.00042724609375	\\
0.497669463310694	0.000640869140625	\\
0.497713854485728	0.000885009765625	\\
0.497758245660763	0.001068115234375	\\
0.497802636835797	0.0010986328125	\\
0.497847028010831	0.00140380859375	\\
0.497891419185866	0.001800537109375	\\
0.4979358103609	0.001739501953125	\\
0.497980201535935	0.0013427734375	\\
0.498024592710969	0.001678466796875	\\
0.498068983886003	0.001617431640625	\\
0.498113375061038	0.001373291015625	\\
0.498157766236072	0.00128173828125	\\
0.498202157411107	0.0015869140625	\\
0.498246548586141	0.002288818359375	\\
0.498290939761176	0.00225830078125	\\
0.49833533093621	0.001953125	\\
0.498379722111244	0.00201416015625	\\
0.498424113286279	0.001953125	\\
0.498468504461313	0.0023193359375	\\
0.498512895636347	0.002593994140625	\\
0.498557286811382	0.00201416015625	\\
0.498601677986416	0.001983642578125	\\
0.498646069161451	0.001922607421875	\\
0.498690460336485	0.001129150390625	\\
0.49873485151152	0.000701904296875	\\
0.498779242686554	0.00091552734375	\\
0.498823633861588	0.00079345703125	\\
0.498868025036623	0.00048828125	\\
0.498912416211657	0.00030517578125	\\
0.498956807386692	0.00048828125	\\
0.499001198561726	0.000244140625	\\
0.49904558973676	0	\\
0.499089980911795	6.103515625e-05	\\
0.499134372086829	-0.00048828125	\\
0.499178763261864	-0.000579833984375	\\
0.499223154436898	-0.00048828125	\\
0.499267545611932	-0.00067138671875	\\
0.499311936786967	-0.00030517578125	\\
0.499356327962001	-0.000457763671875	\\
0.499400719137036	-0.000244140625	\\
0.49944511031207	0.000213623046875	\\
0.499489501487104	6.103515625e-05	\\
0.499533892662139	0.000457763671875	\\
0.499578283837173	0.00067138671875	\\
0.499622675012208	0.00091552734375	\\
0.499667066187242	0.00140380859375	\\
0.499711457362276	0.001434326171875	\\
0.499755848537311	0.00103759765625	\\
0.499800239712345	0.0010986328125	\\
0.49984463088738	0.00140380859375	\\
0.499889022062414	0.0013427734375	\\
0.499933413237448	0.001495361328125	\\
0.499977804412483	0.0020751953125	\\
0.500022195587517	0.0018310546875	\\
0.500066586762552	0.0018310546875	\\
0.500110977937586	0.00201416015625	\\
0.50015536911262	0.0023193359375	\\
0.500199760287655	0.0025634765625	\\
0.500244151462689	0.00274658203125	\\
0.500288542637724	0.0028076171875	\\
0.500332933812758	0.002838134765625	\\
0.500377324987792	0.003021240234375	\\
0.500421716162827	0.002838134765625	\\
0.500466107337861	0.002471923828125	\\
0.500510498512896	0.002593994140625	\\
0.50055488968793	0.002685546875	\\
0.500599280862964	0.002166748046875	\\
0.500643672037999	0.00177001953125	\\
0.500688063213033	0.0018310546875	\\
0.500732454388068	0.001922607421875	\\
0.500776845563102	0.0020751953125	\\
0.500821236738137	0.001739501953125	\\
0.500865627913171	0.00128173828125	\\
0.500910019088205	0.001068115234375	\\
0.50095441026324	0.001068115234375	\\
0.500998801438274	0.00103759765625	\\
0.501043192613308	0.000885009765625	\\
0.501087583788343	0.000946044921875	\\
0.501131974963377	0.000946044921875	\\
0.501176366138412	0.000762939453125	\\
0.501220757313446	0.00079345703125	\\
0.50126514848848	0.0006103515625	\\
0.501309539663515	0.00067138671875	\\
0.501353930838549	0.000732421875	\\
0.501398322013584	0.00079345703125	\\
0.501442713188618	0.000946044921875	\\
0.501487104363653	0.000701904296875	\\
0.501531495538687	0.00054931640625	\\
0.501575886713721	0.00048828125	\\
0.501620277888756	0.000396728515625	\\
0.50166466906379	0.000518798828125	\\
0.501709060238825	0.001007080078125	\\
0.501753451413859	0.000762939453125	\\
0.501797842588893	0.000762939453125	\\
0.501842233763928	0.001007080078125	\\
0.501886624938962	0.000701904296875	\\
0.501931016113996	0.0008544921875	\\
0.501975407289031	0.00067138671875	\\
0.502019798464065	0.000640869140625	\\
0.5020641896391	0.0009765625	\\
0.502108580814134	0.001251220703125	\\
0.502152971989169	0.001312255859375	\\
0.502197363164203	0.00152587890625	\\
0.502241754339237	0.001739501953125	\\
0.502286145514272	0.001434326171875	\\
0.502330536689306	0.00152587890625	\\
0.502374927864341	0.001373291015625	\\
0.502419319039375	0.00115966796875	\\
0.502463710214409	0.001068115234375	\\
0.502508101389444	0.00103759765625	\\
0.502552492564478	0.00091552734375	\\
0.502596883739513	0.0006103515625	\\
0.502641274914547	0.000579833984375	\\
0.502685666089581	0.000335693359375	\\
0.502730057264616	0.000396728515625	\\
0.50277444843965	-6.103515625e-05	\\
0.502818839614685	-0.00042724609375	\\
0.502863230789719	-0.000152587890625	\\
0.502907621964753	-0.000457763671875	\\
0.502952013139788	-0.001007080078125	\\
0.502996404314822	-0.0010986328125	\\
0.503040795489857	-0.001373291015625	\\
0.503085186664891	-0.001220703125	\\
0.503129577839925	-0.0010986328125	\\
0.50317396901496	-0.001068115234375	\\
0.503218360189994	-0.00140380859375	\\
0.503262751365029	-0.001800537109375	\\
0.503307142540063	-0.001312255859375	\\
0.503351533715097	-0.0010986328125	\\
0.503395924890132	-0.001434326171875	\\
0.503440316065166	-0.001739501953125	\\
0.503484707240201	-0.00177001953125	\\
0.503529098415235	-0.001495361328125	\\
0.503573489590269	-0.001251220703125	\\
0.503617880765304	-0.00152587890625	\\
0.503662271940338	-0.001434326171875	\\
0.503706663115373	-0.001068115234375	\\
0.503751054290407	-0.000762939453125	\\
0.503795445465441	-0.000762939453125	\\
0.503839836640476	-0.0003662109375	\\
0.50388422781551	0	\\
0.503928618990545	9.1552734375e-05	\\
0.503973010165579	0.0001220703125	\\
0.504017401340613	0.000152587890625	\\
0.504061792515648	0.000213623046875	\\
0.504106183690682	9.1552734375e-05	\\
0.504150574865717	-0.00018310546875	\\
0.504194966040751	3.0517578125e-05	\\
0.504239357215786	-6.103515625e-05	\\
0.50428374839082	0.00030517578125	\\
0.504328139565854	0.000640869140625	\\
0.504372530740889	-9.1552734375e-05	\\
0.504416921915923	0.0001220703125	\\
0.504461313090958	3.0517578125e-05	\\
0.504505704265992	-0.00048828125	\\
0.504550095441026	-0.00042724609375	\\
0.504594486616061	-0.000823974609375	\\
0.504638877791095	-0.00115966796875	\\
0.50468326896613	-0.00152587890625	\\
0.504727660141164	-0.00164794921875	\\
0.504772051316198	-0.001251220703125	\\
0.504816442491233	-0.001556396484375	\\
0.504860833666267	-0.001556396484375	\\
0.504905224841302	-0.001373291015625	\\
0.504949616016336	-0.001678466796875	\\
0.50499400719137	-0.00177001953125	\\
0.505038398366405	-0.001708984375	\\
0.505082789541439	-0.001556396484375	\\
0.505127180716474	-0.001556396484375	\\
0.505171571891508	-0.001373291015625	\\
0.505215963066542	-0.0015869140625	\\
0.505260354241577	-0.0020751953125	\\
0.505304745416611	-0.001617431640625	\\
0.505349136591646	-0.001617431640625	\\
0.50539352776668	-0.001739501953125	\\
0.505437918941714	-0.00164794921875	\\
0.505482310116749	-0.00140380859375	\\
0.505526701291783	-0.000701904296875	\\
0.505571092466818	-6.103515625e-05	\\
0.505615483641852	-0.000274658203125	\\
0.505659874816886	-0.000640869140625	\\
0.505704265991921	-0.000579833984375	\\
0.505748657166955	-0.000732421875	\\
0.50579304834199	-6.103515625e-05	\\
0.505837439517024	0.00054931640625	\\
0.505881830692058	0.000274658203125	\\
0.505926221867093	0	\\
0.505970613042127	-0.0001220703125	\\
0.506015004217162	0.000335693359375	\\
0.506059395392196	0.000732421875	\\
0.50610378656723	0.000579833984375	\\
0.506148177742265	0.000823974609375	\\
0.506192568917299	0.000701904296875	\\
0.506236960092334	0.000579833984375	\\
0.506281351267368	0.00091552734375	\\
0.506325742442402	0.000946044921875	\\
0.506370133617437	0.000640869140625	\\
0.506414524792471	0.00079345703125	\\
0.506458915967506	0.001129150390625	\\
0.50650330714254	0.000518798828125	\\
0.506547698317575	-6.103515625e-05	\\
0.506592089492609	-0.00042724609375	\\
0.506636480667643	-0.001007080078125	\\
0.506680871842678	-0.00128173828125	\\
0.506725263017712	-0.001129150390625	\\
0.506769654192746	-0.0009765625	\\
0.506814045367781	-0.001373291015625	\\
0.506858436542815	-0.000823974609375	\\
0.50690282771785	-0.000946044921875	\\
0.506947218892884	-0.0008544921875	\\
0.506991610067918	-0.00048828125	\\
0.507036001242953	-0.0008544921875	\\
0.507080392417987	-0.001068115234375	\\
0.507124783593022	-0.00146484375	\\
0.507169174768056	-0.0013427734375	\\
0.507213565943091	-0.0008544921875	\\
0.507257957118125	-0.000946044921875	\\
0.507302348293159	-0.001373291015625	\\
0.507346739468194	-0.001190185546875	\\
0.507391130643228	-0.000335693359375	\\
0.507435521818263	-0.00048828125	\\
0.507479912993297	-0.000732421875	\\
0.507524304168331	-0.000396728515625	\\
0.507568695343366	-0.000213623046875	\\
0.5076130865184	0.0001220703125	\\
0.507657477693435	6.103515625e-05	\\
0.507701868868469	0.0001220703125	\\
0.507746260043503	0.00048828125	\\
0.507790651218538	0.000732421875	\\
0.507835042393572	0.000396728515625	\\
0.507879433568607	0.00054931640625	\\
0.507923824743641	0.0008544921875	\\
0.507968215918675	0.0001220703125	\\
0.50801260709371	-0.00018310546875	\\
0.508056998268744	0.000579833984375	\\
0.508101389443779	0.001190185546875	\\
0.508145780618813	0.001190185546875	\\
0.508190171793847	0.000732421875	\\
0.508234562968882	0.00054931640625	\\
0.508278954143916	0.0008544921875	\\
0.508323345318951	0.000823974609375	\\
0.508367736493985	0.0003662109375	\\
0.508412127669019	-0.0001220703125	\\
0.508456518844054	-0.000152587890625	\\
0.508500910019088	-0.00054931640625	\\
0.508545301194123	-0.00091552734375	\\
0.508589692369157	-0.0010986328125	\\
0.508634083544191	-0.001129150390625	\\
0.508678474719226	-0.00079345703125	\\
0.50872286589426	-0.00079345703125	\\
0.508767257069295	-0.001007080078125	\\
0.508811648244329	-0.00128173828125	\\
0.508856039419363	-0.001129150390625	\\
0.508900430594398	-0.001251220703125	\\
0.508944821769432	-0.0015869140625	\\
0.508989212944467	-0.001739501953125	\\
0.509033604119501	-0.002197265625	\\
0.509077995294535	-0.0028076171875	\\
0.50912238646957	-0.0028076171875	\\
0.509166777644604	-0.002899169921875	\\
0.509211168819639	-0.00286865234375	\\
0.509255559994673	-0.002349853515625	\\
0.509299951169708	-0.002288818359375	\\
0.509344342344742	-0.001953125	\\
0.509388733519776	-0.00201416015625	\\
0.509433124694811	-0.001953125	\\
0.509477515869845	-0.001861572265625	\\
0.509521907044879	-0.00164794921875	\\
0.509566298219914	-0.00146484375	\\
0.509610689394948	-0.001922607421875	\\
0.509655080569983	-0.0018310546875	\\
0.509699471745017	-0.001373291015625	\\
0.509743862920051	-0.001129150390625	\\
0.509788254095086	-0.000640869140625	\\
0.50983264527012	-0.000762939453125	\\
0.509877036445155	-0.000762939453125	\\
0.509921427620189	-0.000274658203125	\\
0.509965818795224	0.000152587890625	\\
0.510010209970258	0.000335693359375	\\
0.510054601145292	0.000457763671875	\\
0.510098992320327	0.0006103515625	\\
0.510143383495361	0.000274658203125	\\
0.510187774670396	0.00030517578125	\\
0.51023216584543	0.00048828125	\\
0.510276557020464	-0.000152587890625	\\
0.510320948195499	6.103515625e-05	\\
0.510365339370533	0.0001220703125	\\
0.510409730545567	0	\\
0.510454121720602	-0.000152587890625	\\
0.510498512895636	-0.000335693359375	\\
0.510542904070671	-0.000213623046875	\\
0.510587295245705	-6.103515625e-05	\\
0.51063168642074	-0.0006103515625	\\
0.510676077595774	-0.0008544921875	\\
0.510720468770808	-0.00048828125	\\
0.510764859945843	-0.00042724609375	\\
0.510809251120877	-0.00091552734375	\\
0.510853642295912	-0.001373291015625	\\
0.510898033470946	-0.0018310546875	\\
0.51094242464598	-0.002105712890625	\\
0.510986815821015	-0.0020751953125	\\
0.511031206996049	-0.0020751953125	\\
0.511075598171084	-0.002166748046875	\\
0.511119989346118	-0.002471923828125	\\
0.511164380521152	-0.002593994140625	\\
0.511208771696187	-0.00262451171875	\\
0.511253162871221	-0.00286865234375	\\
0.511297554046256	-0.0032958984375	\\
0.51134194522129	-0.00341796875	\\
0.511386336396324	-0.002899169921875	\\
0.511430727571359	-0.002960205078125	\\
0.511475118746393	-0.003143310546875	\\
0.511519509921428	-0.002593994140625	\\
0.511563901096462	-0.00250244140625	\\
0.511608292271496	-0.00244140625	\\
0.511652683446531	-0.00213623046875	\\
0.511697074621565	-0.0023193359375	\\
0.5117414657966	-0.00225830078125	\\
0.511785856971634	-0.001800537109375	\\
0.511830248146668	-0.0018310546875	\\
0.511874639321703	-0.001800537109375	\\
0.511919030496737	-0.0015869140625	\\
0.511963421671772	-0.001556396484375	\\
0.512007812846806	-0.001739501953125	\\
0.51205220402184	-0.001861572265625	\\
0.512096595196875	-0.002197265625	\\
0.512140986371909	-0.002044677734375	\\
0.512185377546944	-0.002410888671875	\\
0.512229768721978	-0.002685546875	\\
0.512274159897013	-0.002532958984375	\\
0.512318551072047	-0.002960205078125	\\
0.512362942247081	-0.003143310546875	\\
0.512407333422116	-0.0029296875	\\
0.51245172459715	-0.003326416015625	\\
0.512496115772184	-0.003570556640625	\\
0.512540506947219	-0.00384521484375	\\
0.512584898122253	-0.0042724609375	\\
0.512629289297288	-0.003631591796875	\\
0.512673680472322	-0.00360107421875	\\
0.512718071647357	-0.003753662109375	\\
0.512762462822391	-0.003814697265625	\\
0.512806853997425	-0.003814697265625	\\
0.51285124517246	-0.003753662109375	\\
0.512895636347494	-0.004180908203125	\\
0.512940027522529	-0.004241943359375	\\
0.512984418697563	-0.003997802734375	\\
0.513028809872597	-0.00421142578125	\\
0.513073201047632	-0.003997802734375	\\
0.513117592222666	-0.003997802734375	\\
0.513161983397701	-0.003662109375	\\
0.513206374572735	-0.00347900390625	\\
0.513250765747769	-0.003326416015625	\\
0.513295156922804	-0.0029296875	\\
0.513339548097838	-0.0028076171875	\\
0.513383939272873	-0.002471923828125	\\
0.513428330447907	-0.0028076171875	\\
0.513472721622941	-0.002471923828125	\\
0.513517112797976	-0.0018310546875	\\
0.51356150397301	-0.0015869140625	\\
0.513605895148045	-0.001007080078125	\\
0.513650286323079	-0.00115966796875	\\
0.513694677498113	-0.001190185546875	\\
0.513739068673148	-0.00067138671875	\\
0.513783459848182	-0.000396728515625	\\
0.513827851023217	-0.000396728515625	\\
0.513872242198251	9.1552734375e-05	\\
0.513916633373285	0.000152587890625	\\
0.51396102454832	6.103515625e-05	\\
0.514005415723354	0.000244140625	\\
0.514049806898389	0.000579833984375	\\
0.514094198073423	0.00054931640625	\\
0.514138589248457	0.00042724609375	\\
0.514182980423492	0.0009765625	\\
0.514227371598526	0.000640869140625	\\
0.514271762773561	-9.1552734375e-05	\\
0.514316153948595	-0.00018310546875	\\
0.514360545123629	0.000244140625	\\
0.514404936298664	0.0001220703125	\\
0.514449327473698	-0.000762939453125	\\
0.514493718648733	-0.0008544921875	\\
0.514538109823767	-0.00067138671875	\\
0.514582500998801	-0.001373291015625	\\
0.514626892173836	-0.001708984375	\\
0.51467128334887	-0.001922607421875	\\
0.514715674523905	-0.00244140625	\\
0.514760065698939	-0.00274658203125	\\
0.514804456873973	-0.0032958984375	\\
0.514848848049008	-0.003143310546875	\\
0.514893239224042	-0.003326416015625	\\
0.514937630399077	-0.003387451171875	\\
0.514982021574111	-0.0028076171875	\\
0.515026412749146	-0.00347900390625	\\
0.51507080392418	-0.0030517578125	\\
0.515115195099214	-0.002899169921875	\\
0.515159586274249	-0.00341796875	\\
0.515203977449283	-0.003021240234375	\\
0.515248368624317	-0.00311279296875	\\
0.515292759799352	-0.003265380859375	\\
0.515337150974386	-0.003021240234375	\\
0.515381542149421	-0.00274658203125	\\
0.515425933324455	-0.0023193359375	\\
0.515470324499489	-0.001800537109375	\\
0.515514715674524	-0.0006103515625	\\
0.515559106849558	-0.000335693359375	\\
0.515603498024593	-0.000732421875	\\
0.515647889199627	-0.0003662109375	\\
0.515692280374662	-0.000152587890625	\\
0.515736671549696	0.000244140625	\\
0.51578106272473	0.000518798828125	\\
0.515825453899765	0.000823974609375	\\
0.515869845074799	0.0015869140625	\\
0.515914236249834	0.0018310546875	\\
0.515958627424868	0.00189208984375	\\
0.516003018599902	0.002197265625	\\
0.516047409774937	0.002410888671875	\\
0.516091800949971	0.002471923828125	\\
0.516136192125006	0.00238037109375	\\
0.51618058330004	0.002685546875	\\
0.516224974475074	0.002227783203125	\\
0.516269365650109	0.0020751953125	\\
0.516313756825143	0.00225830078125	\\
0.516358148000178	0.00128173828125	\\
0.516402539175212	0.0010986328125	\\
0.516446930350246	0.0013427734375	\\
0.516491321525281	0.00067138671875	\\
0.516535712700315	0.00042724609375	\\
0.51658010387535	0.000701904296875	\\
0.516624495050384	0.00054931640625	\\
0.516668886225418	0.00018310546875	\\
0.516713277400453	-9.1552734375e-05	\\
0.516757668575487	-6.103515625e-05	\\
0.516802059750522	-6.103515625e-05	\\
0.516846450925556	-0.000274658203125	\\
0.51689084210059	-0.00054931640625	\\
0.516935233275625	-0.000732421875	\\
0.516979624450659	-0.0006103515625	\\
0.517024015625694	-0.000823974609375	\\
0.517068406800728	-0.000946044921875	\\
0.517112797975762	-0.00103759765625	\\
0.517157189150797	-0.001373291015625	\\
0.517201580325831	-0.00091552734375	\\
0.517245971500866	-0.0006103515625	\\
0.5172903626759	-0.000823974609375	\\
0.517334753850934	-0.0006103515625	\\
0.517379145025969	-0.000335693359375	\\
0.517423536201003	-0.000244140625	\\
0.517467927376038	-0.000244140625	\\
0.517512318551072	-6.103515625e-05	\\
0.517556709726106	-0.00030517578125	\\
0.517601100901141	-0.000213623046875	\\
0.517645492076175	0.000244140625	\\
0.51768988325121	6.103515625e-05	\\
0.517734274426244	6.103515625e-05	\\
0.517778665601279	0.0006103515625	\\
0.517823056776313	0.0006103515625	\\
0.517867447951347	0.00054931640625	\\
0.517911839126382	0.0008544921875	\\
0.517956230301416	0.001068115234375	\\
0.51800062147645	0.001373291015625	\\
0.518045012651485	0.001068115234375	\\
0.518089403826519	0.001220703125	\\
0.518133795001554	0.0010986328125	\\
0.518178186176588	0.00103759765625	\\
0.518222577351622	0.001068115234375	\\
0.518266968526657	0.000579833984375	\\
0.518311359701691	0.000640869140625	\\
0.518355750876726	0.000762939453125	\\
0.51840014205176	0.000762939453125	\\
0.518444533226795	0.000518798828125	\\
0.518488924401829	-6.103515625e-05	\\
0.518533315576863	-0.0001220703125	\\
0.518577706751898	-0.000152587890625	\\
0.518622097926932	-0.000274658203125	\\
0.518666489101967	-0.000244140625	\\
0.518710880277001	-0.0006103515625	\\
0.518755271452035	-0.001220703125	\\
0.51879966262707	-0.00140380859375	\\
0.518844053802104	-0.00128173828125	\\
0.518888444977139	-0.00103759765625	\\
0.518932836152173	-0.00146484375	\\
0.518977227327207	-0.00177001953125	\\
0.519021618502242	-0.001739501953125	\\
0.519066009677276	-0.001983642578125	\\
0.519110400852311	-0.001983642578125	\\
0.519154792027345	-0.0020751953125	\\
0.519199183202379	-0.0018310546875	\\
0.519243574377414	-0.001220703125	\\
0.519287965552448	-0.0013427734375	\\
0.519332356727483	-0.001190185546875	\\
0.519376747902517	-0.001007080078125	\\
0.519421139077551	-0.000732421875	\\
0.519465530252586	-0.000579833984375	\\
0.51950992142762	-0.0010986328125	\\
0.519554312602655	-0.000701904296875	\\
0.519598703777689	-0.000244140625	\\
0.519643094952723	6.103515625e-05	\\
0.519687486127758	-3.0517578125e-05	\\
0.519731877302792	0.00042724609375	\\
0.519776268477827	0.0003662109375	\\
0.519820659652861	-9.1552734375e-05	\\
0.519865050827895	0.000335693359375	\\
0.51990944200293	0.000396728515625	\\
0.519953833177964	0.000152587890625	\\
0.519998224352999	0.000244140625	\\
0.520042615528033	0.000335693359375	\\
0.520087006703067	0.0006103515625	\\
0.520131397878102	0.00030517578125	\\
0.520175789053136	0.000244140625	\\
0.520220180228171	0.000518798828125	\\
0.520264571403205	-6.103515625e-05	\\
0.520308962578239	-0.000244140625	\\
0.520353353753274	0.0001220703125	\\
0.520397744928308	-0.00067138671875	\\
0.520442136103343	-0.0008544921875	\\
0.520486527278377	-0.0003662109375	\\
0.520530918453411	-0.000457763671875	\\
0.520575309628446	-0.001190185546875	\\
0.52061970080348	-0.00103759765625	\\
0.520664091978515	-0.000823974609375	\\
0.520708483153549	-0.0013427734375	\\
0.520752874328584	-0.0010986328125	\\
0.520797265503618	-0.001007080078125	\\
0.520841656678652	-0.001434326171875	\\
0.520886047853687	-0.00189208984375	\\
0.520930439028721	-0.00189208984375	\\
0.520974830203755	-0.001495361328125	\\
0.52101922137879	-0.00152587890625	\\
0.521063612553824	-0.0013427734375	\\
0.521108003728859	-0.001007080078125	\\
0.521152394903893	-0.00128173828125	\\
0.521196786078928	-0.000885009765625	\\
0.521241177253962	-0.00048828125	\\
0.521285568428996	-0.000518798828125	\\
0.521329959604031	-0.000244140625	\\
0.521374350779065	-3.0517578125e-05	\\
0.5214187419541	9.1552734375e-05	\\
0.521463133129134	0.00042724609375	\\
0.521507524304168	0.000701904296875	\\
0.521551915479203	0.000823974609375	\\
0.521596306654237	0.000762939453125	\\
0.521640697829272	0.000762939453125	\\
0.521685089004306	0.0006103515625	\\
0.52172948017934	0.000946044921875	\\
0.521773871354375	0.001251220703125	\\
0.521818262529409	0.0008544921875	\\
0.521862653704444	0.0010986328125	\\
0.521907044879478	0.001312255859375	\\
0.521951436054512	0.00103759765625	\\
0.521995827229547	0.0008544921875	\\
0.522040218404581	0.000946044921875	\\
0.522084609579616	0.000946044921875	\\
0.52212900075465	0.00115966796875	\\
0.522173391929684	0.0008544921875	\\
0.522217783104719	0.00079345703125	\\
0.522262174279753	0.000213623046875	\\
0.522306565454788	0	\\
0.522350956629822	0.000213623046875	\\
0.522395347804856	-0.00042724609375	\\
0.522439738979891	-0.00079345703125	\\
0.522484130154925	-0.00079345703125	\\
0.52252852132996	-0.001220703125	\\
0.522572912504994	-0.001007080078125	\\
0.522617303680028	-0.000762939453125	\\
0.522661694855063	-0.001312255859375	\\
0.522706086030097	-0.00146484375	\\
0.522750477205132	-0.001556396484375	\\
0.522794868380166	-0.001556396484375	\\
0.5228392595552	-0.001220703125	\\
0.522883650730235	-0.001739501953125	\\
0.522928041905269	-0.001861572265625	\\
0.522972433080304	-0.0018310546875	\\
0.523016824255338	-0.001922607421875	\\
0.523061215430372	-0.00152587890625	\\
0.523105606605407	-0.001129150390625	\\
0.523149997780441	-0.0010986328125	\\
0.523194388955476	-0.00146484375	\\
0.52323878013051	-0.0013427734375	\\
0.523283171305544	-0.00115966796875	\\
0.523327562480579	-0.0008544921875	\\
0.523371953655613	-0.000701904296875	\\
0.523416344830648	-0.000244140625	\\
0.523460736005682	0.000244140625	\\
0.523505127180717	0.000274658203125	\\
0.523549518355751	0.000396728515625	\\
0.523593909530785	0.000244140625	\\
0.52363830070582	0.00048828125	\\
0.523682691880854	0.0013427734375	\\
0.523727083055888	0.0015869140625	\\
0.523771474230923	0.001556396484375	\\
0.523815865405957	0.001922607421875	\\
0.523860256580992	0.002044677734375	\\
0.523904647756026	0.001861572265625	\\
0.52394903893106	0.0015869140625	\\
0.523993430106095	0.001434326171875	\\
0.524037821281129	0.001373291015625	\\
0.524082212456164	0.001129150390625	\\
0.524126603631198	0.00128173828125	\\
0.524170994806233	0.001251220703125	\\
0.524215385981267	0.0008544921875	\\
0.524259777156301	0.001312255859375	\\
0.524304168331336	0.001373291015625	\\
0.52434855950637	0.000823974609375	\\
0.524392950681405	0.000457763671875	\\
0.524437341856439	-0.000335693359375	\\
0.524481733031473	-0.000701904296875	\\
0.524526124206508	-0.001129150390625	\\
0.524570515381542	-0.001434326171875	\\
0.524614906556577	-0.001312255859375	\\
0.524659297731611	-0.0013427734375	\\
0.524703688906645	-0.001220703125	\\
0.52474808008168	-0.001617431640625	\\
0.524792471256714	-0.001617431640625	\\
0.524836862431749	-0.001708984375	\\
0.524881253606783	-0.0015869140625	\\
0.524925644781817	-0.001739501953125	\\
0.524970035956852	-0.00177001953125	\\
0.525014427131886	-0.001617431640625	\\
0.525058818306921	-0.001922607421875	\\
0.525103209481955	-0.00238037109375	\\
0.525147600656989	-0.002197265625	\\
0.525191991832024	-0.00189208984375	\\
0.525236383007058	-0.002105712890625	\\
0.525280774182093	-0.00213623046875	\\
0.525325165357127	-0.001739501953125	\\
0.525369556532161	-0.001617431640625	\\
0.525413947707196	-0.001129150390625	\\
0.52545833888223	-0.000579833984375	\\
0.525502730057265	-0.000274658203125	\\
0.525547121232299	-0.000518798828125	\\
0.525591512407333	-0.000244140625	\\
0.525635903582368	0.000335693359375	\\
0.525680294757402	0.0003662109375	\\
0.525724685932437	0.000640869140625	\\
0.525769077107471	0.000701904296875	\\
0.525813468282505	0.00103759765625	\\
0.52585785945754	0.000946044921875	\\
0.525902250632574	0.000457763671875	\\
0.525946641807609	-0.000152587890625	\\
0.525991032982643	0.000213623046875	\\
0.526035424157677	0.00018310546875	\\
0.526079815332712	0.00042724609375	\\
0.526124206507746	0.000823974609375	\\
0.526168597682781	0.000640869140625	\\
0.526212988857815	-0.000152587890625	\\
0.52625738003285	-0.000579833984375	\\
0.526301771207884	-0.00048828125	\\
0.526346162382918	-0.000732421875	\\
0.526390553557953	-0.00079345703125	\\
0.526434944732987	-0.001220703125	\\
0.526479335908022	-0.001495361328125	\\
0.526523727083056	-0.001556396484375	\\
0.52656811825809	-0.001800537109375	\\
0.526612509433125	-0.001617431640625	\\
0.526656900608159	-0.001678466796875	\\
0.526701291783193	-0.00177001953125	\\
0.526745682958228	-0.00225830078125	\\
0.526790074133262	-0.0025634765625	\\
0.526834465308297	-0.00213623046875	\\
0.526878856483331	-0.00250244140625	\\
0.526923247658366	-0.002716064453125	\\
0.5269676388334	-0.003173828125	\\
0.527012030008434	-0.003875732421875	\\
0.527056421183469	-0.003448486328125	\\
0.527100812358503	-0.0035400390625	\\
0.527145203533538	-0.0037841796875	\\
0.527189594708572	-0.0035400390625	\\
0.527233985883606	-0.003387451171875	\\
0.527278377058641	-0.003448486328125	\\
0.527322768233675	-0.00335693359375	\\
0.52736715940871	-0.003082275390625	\\
0.527411550583744	-0.003021240234375	\\
0.527455941758778	-0.0029296875	\\
0.527500332933813	-0.0029296875	\\
0.527544724108847	-0.002777099609375	\\
0.527589115283882	-0.002838134765625	\\
0.527633506458916	-0.002471923828125	\\
0.52767789763395	-0.0028076171875	\\
0.527722288808985	-0.003082275390625	\\
0.527766679984019	-0.00250244140625	\\
0.527811071159054	-0.002593994140625	\\
0.527855462334088	-0.0030517578125	\\
0.527899853509122	-0.002655029296875	\\
0.527944244684157	-0.002166748046875	\\
0.527988635859191	-0.00274658203125	\\
0.528033027034226	-0.002716064453125	\\
0.52807741820926	-0.002593994140625	\\
0.528121809384294	-0.003265380859375	\\
0.528166200559329	-0.003173828125	\\
0.528210591734363	-0.003082275390625	\\
0.528254982909398	-0.00372314453125	\\
0.528299374084432	-0.00372314453125	\\
0.528343765259466	-0.003692626953125	\\
0.528388156434501	-0.003814697265625	\\
0.528432547609535	-0.0037841796875	\\
0.52847693878457	-0.004119873046875	\\
0.528521329959604	-0.00421142578125	\\
0.528565721134638	-0.004364013671875	\\
0.528610112309673	-0.00396728515625	\\
0.528654503484707	-0.003997802734375	\\
0.528698894659742	-0.004241943359375	\\
0.528743285834776	-0.004486083984375	\\
0.52878767700981	-0.00457763671875	\\
0.528832068184845	-0.004547119140625	\\
0.528876459359879	-0.00433349609375	\\
0.528920850534914	-0.004302978515625	\\
0.528965241709948	-0.00408935546875	\\
0.529009632884982	-0.00372314453125	\\
0.529054024060017	-0.003631591796875	\\
0.529098415235051	-0.003936767578125	\\
0.529142806410086	-0.004180908203125	\\
0.52918719758512	-0.0040283203125	\\
0.529231588760155	-0.003875732421875	\\
0.529275979935189	-0.003814697265625	\\
0.529320371110223	-0.0037841796875	\\
0.529364762285258	-0.0035400390625	\\
0.529409153460292	-0.003173828125	\\
0.529453544635326	-0.00323486328125	\\
0.529497935810361	-0.002655029296875	\\
0.529542326985395	-0.00262451171875	\\
0.52958671816043	-0.002410888671875	\\
0.529631109335464	-0.00213623046875	\\
0.529675500510499	-0.0023193359375	\\
0.529719891685533	-0.001800537109375	\\
0.529764282860567	-0.002227783203125	\\
0.529808674035602	-0.00238037109375	\\
0.529853065210636	-0.002410888671875	\\
0.529897456385671	-0.002197265625	\\
0.529941847560705	-0.001739501953125	\\
0.529986238735739	-0.001861572265625	\\
0.530030629910774	-0.00213623046875	\\
0.530075021085808	-0.002227783203125	\\
0.530119412260843	-0.002288818359375	\\
0.530163803435877	-0.002838134765625	\\
0.530208194610911	-0.002716064453125	\\
0.530252585785946	-0.00225830078125	\\
0.53029697696098	-0.00262451171875	\\
0.530341368136015	-0.0025634765625	\\
0.530385759311049	-0.002716064453125	\\
0.530430150486083	-0.003326416015625	\\
0.530474541661118	-0.00347900390625	\\
0.530518932836152	-0.003875732421875	\\
0.530563324011187	-0.004241943359375	\\
0.530607715186221	-0.004180908203125	\\
0.530652106361255	-0.0045166015625	\\
0.53069649753629	-0.004791259765625	\\
0.530740888711324	-0.005035400390625	\\
0.530785279886359	-0.005035400390625	\\
0.530829671061393	-0.005096435546875	\\
0.530874062236427	-0.005035400390625	\\
0.530918453411462	-0.00439453125	\\
0.530962844586496	-0.004425048828125	\\
0.531007235761531	-0.004425048828125	\\
0.531051626936565	-0.004302978515625	\\
0.531096018111599	-0.00433349609375	\\
0.531140409286634	-0.004119873046875	\\
0.531184800461668	-0.00390625	\\
0.531229191636703	-0.003631591796875	\\
0.531273582811737	-0.003265380859375	\\
0.531317973986771	-0.00341796875	\\
0.531362365161806	-0.0030517578125	\\
0.53140675633684	-0.00286865234375	\\
0.531451147511875	-0.002655029296875	\\
0.531495538686909	-0.00250244140625	\\
0.531539929861943	-0.002288818359375	\\
0.531584321036978	-0.002044677734375	\\
0.531628712212012	-0.00189208984375	\\
0.531673103387047	-0.00201416015625	\\
0.531717494562081	-0.001953125	\\
0.531761885737115	-0.001800537109375	\\
0.53180627691215	-0.00146484375	\\
0.531850668087184	-0.001556396484375	\\
0.531895059262219	-0.001190185546875	\\
0.531939450437253	-0.001312255859375	\\
0.531983841612288	-0.001190185546875	\\
0.532028232787322	-0.00140380859375	\\
0.532072623962356	-0.00152587890625	\\
0.532117015137391	-0.001220703125	\\
0.532161406312425	-0.00152587890625	\\
0.532205797487459	-0.0018310546875	\\
0.532250188662494	-0.00164794921875	\\
0.532294579837528	-0.001617431640625	\\
0.532338971012563	-0.0018310546875	\\
0.532383362187597	-0.0015869140625	\\
0.532427753362631	-0.0018310546875	\\
0.532472144537666	-0.00244140625	\\
0.5325165357127	-0.002349853515625	\\
0.532560926887735	-0.00189208984375	\\
0.532605318062769	-0.0020751953125	\\
0.532649709237804	-0.002471923828125	\\
0.532694100412838	-0.002471923828125	\\
};
\addplot [color=blue,solid,forget plot]
  table[row sep=crcr]{
0.532694100412838	-0.002471923828125	\\
0.532738491587872	-0.002655029296875	\\
0.532782882762907	-0.002532958984375	\\
0.532827273937941	-0.002655029296875	\\
0.532871665112976	-0.0032958984375	\\
0.53291605628801	-0.0030517578125	\\
0.532960447463044	-0.002593994140625	\\
0.533004838638079	-0.003143310546875	\\
0.533049229813113	-0.0030517578125	\\
0.533093620988148	-0.003204345703125	\\
0.533138012163182	-0.003387451171875	\\
0.533182403338216	-0.00299072265625	\\
0.533226794513251	-0.003021240234375	\\
0.533271185688285	-0.002777099609375	\\
0.53331557686332	-0.0023193359375	\\
0.533359968038354	-0.002288818359375	\\
0.533404359213388	-0.002288818359375	\\
0.533448750388423	-0.002410888671875	\\
0.533493141563457	-0.002593994140625	\\
0.533537532738492	-0.002288818359375	\\
0.533581923913526	-0.0023193359375	\\
0.53362631508856	-0.001983642578125	\\
0.533670706263595	-0.00140380859375	\\
0.533715097438629	-0.00152587890625	\\
0.533759488613664	-0.001556396484375	\\
0.533803879788698	-0.001495361328125	\\
0.533848270963732	-0.001312255859375	\\
0.533892662138767	-0.001617431640625	\\
0.533937053313801	-0.0013427734375	\\
0.533981444488836	-0.001068115234375	\\
0.53402583566387	-0.001495361328125	\\
0.534070226838904	-0.001800537109375	\\
0.534114618013939	-0.001922607421875	\\
0.534159009188973	-0.00177001953125	\\
0.534203400364008	-0.00201416015625	\\
0.534247791539042	-0.00225830078125	\\
0.534292182714076	-0.002410888671875	\\
0.534336573889111	-0.00244140625	\\
0.534380965064145	-0.002716064453125	\\
0.53442535623918	-0.002593994140625	\\
0.534469747414214	-0.002685546875	\\
0.534514138589248	-0.002655029296875	\\
0.534558529764283	-0.002532958984375	\\
0.534602920939317	-0.003265380859375	\\
0.534647312114352	-0.003082275390625	\\
0.534691703289386	-0.002685546875	\\
0.534736094464421	-0.00299072265625	\\
0.534780485639455	-0.002838134765625	\\
0.534824876814489	-0.002777099609375	\\
0.534869267989524	-0.002838134765625	\\
0.534913659164558	-0.002655029296875	\\
0.534958050339593	-0.0025634765625	\\
0.535002441514627	-0.00286865234375	\\
0.535046832689661	-0.003082275390625	\\
0.535091223864696	-0.00274658203125	\\
0.53513561503973	-0.002532958984375	\\
0.535180006214764	-0.002349853515625	\\
0.535224397389799	-0.001953125	\\
0.535268788564833	-0.0015869140625	\\
0.535313179739868	-0.00140380859375	\\
0.535357570914902	-0.0015869140625	\\
0.535401962089937	-0.001617431640625	\\
0.535446353264971	-0.00152587890625	\\
0.535490744440005	-0.001129150390625	\\
0.53553513561504	-0.00091552734375	\\
0.535579526790074	-0.001373291015625	\\
0.535623917965109	-0.001251220703125	\\
0.535668309140143	-0.00115966796875	\\
0.535712700315177	-0.001251220703125	\\
0.535757091490212	-0.00164794921875	\\
0.535801482665246	-0.001861572265625	\\
0.535845873840281	-0.0018310546875	\\
0.535890265015315	-0.001953125	\\
0.535934656190349	-0.002227783203125	\\
0.535979047365384	-0.002685546875	\\
0.536023438540418	-0.00262451171875	\\
0.536067829715453	-0.0030517578125	\\
0.536112220890487	-0.00311279296875	\\
0.536156612065521	-0.0030517578125	\\
0.536201003240556	-0.003448486328125	\\
0.53624539441559	-0.00335693359375	\\
0.536289785590625	-0.003082275390625	\\
0.536334176765659	-0.0032958984375	\\
0.536378567940693	-0.003173828125	\\
0.536422959115728	-0.003082275390625	\\
0.536467350290762	-0.003387451171875	\\
0.536511741465797	-0.00311279296875	\\
0.536556132640831	-0.00372314453125	\\
0.536600523815865	-0.003753662109375	\\
0.5366449149909	-0.003143310546875	\\
0.536689306165934	-0.003662109375	\\
0.536733697340969	-0.003814697265625	\\
0.536778088516003	-0.00335693359375	\\
0.536822479691037	-0.003143310546875	\\
0.536866870866072	-0.003204345703125	\\
0.536911262041106	-0.00323486328125	\\
0.536955653216141	-0.003173828125	\\
0.537000044391175	-0.0028076171875	\\
0.537044435566209	-0.002777099609375	\\
0.537088826741244	-0.002838134765625	\\
0.537133217916278	-0.00286865234375	\\
0.537177609091313	-0.0029296875	\\
0.537222000266347	-0.002838134765625	\\
0.537266391441381	-0.002288818359375	\\
0.537310782616416	-0.00177001953125	\\
0.53735517379145	-0.001739501953125	\\
0.537399564966485	-0.001983642578125	\\
0.537443956141519	-0.001800537109375	\\
0.537488347316553	-0.001434326171875	\\
0.537532738491588	-0.001739501953125	\\
0.537577129666622	-0.001556396484375	\\
0.537621520841657	-0.00128173828125	\\
0.537665912016691	-0.00152587890625	\\
0.537710303191726	-0.001495361328125	\\
0.53775469436676	-0.001251220703125	\\
0.537799085541794	-0.00164794921875	\\
0.537843476716829	-0.00140380859375	\\
0.537887867891863	-0.001495361328125	\\
0.537932259066897	-0.001556396484375	\\
0.537976650241932	-0.00115966796875	\\
0.538021041416966	-0.00115966796875	\\
0.538065432592001	-0.001312255859375	\\
0.538109823767035	-0.0013427734375	\\
0.53815421494207	-0.001190185546875	\\
0.538198606117104	-0.001007080078125	\\
0.538242997292138	-0.001434326171875	\\
0.538287388467173	-0.00164794921875	\\
0.538331779642207	-0.00152587890625	\\
0.538376170817242	-0.001617431640625	\\
0.538420561992276	-0.00189208984375	\\
0.53846495316731	-0.0018310546875	\\
0.538509344342345	-0.001861572265625	\\
0.538553735517379	-0.001739501953125	\\
0.538598126692414	-0.001556396484375	\\
0.538642517867448	-0.0018310546875	\\
0.538686909042482	-0.00189208984375	\\
0.538731300217517	-0.0018310546875	\\
0.538775691392551	-0.001800537109375	\\
0.538820082567586	-0.001495361328125	\\
0.53886447374262	-0.00115966796875	\\
0.538908864917654	-0.0010986328125	\\
0.538953256092689	-0.0008544921875	\\
0.538997647267723	-0.000396728515625	\\
0.539042038442758	-9.1552734375e-05	\\
0.539086429617792	-0.000335693359375	\\
0.539130820792826	3.0517578125e-05	\\
0.539175211967861	0.00067138671875	\\
0.539219603142895	0.00054931640625	\\
0.53926399431793	0.000579833984375	\\
0.539308385492964	0.00067138671875	\\
0.539352776667998	0.0001220703125	\\
0.539397167843033	0.000457763671875	\\
0.539441559018067	0.000396728515625	\\
0.539485950193102	0.00018310546875	\\
0.539530341368136	0.000823974609375	\\
0.53957473254317	0.0009765625	\\
0.539619123718205	0.0010986328125	\\
0.539663514893239	0.00177001953125	\\
0.539707906068274	0.00201416015625	\\
0.539752297243308	0.00152587890625	\\
0.539796688418342	0.001251220703125	\\
0.539841079593377	0.001312255859375	\\
0.539885470768411	0.001190185546875	\\
0.539929861943446	0.00146484375	\\
0.53997425311848	0.00128173828125	\\
0.540018644293514	0.000885009765625	\\
0.540063035468549	0.000946044921875	\\
0.540107426643583	0.001007080078125	\\
0.540151817818618	0.00103759765625	\\
0.540196208993652	0.001312255859375	\\
0.540240600168686	0.0010986328125	\\
0.540284991343721	0.000396728515625	\\
0.540329382518755	-3.0517578125e-05	\\
0.54037377369379	-6.103515625e-05	\\
0.540418164868824	-0.0001220703125	\\
0.540462556043859	-0.000335693359375	\\
0.540506947218893	-0.0003662109375	\\
0.540551338393927	-0.000701904296875	\\
0.540595729568962	-0.000762939453125	\\
0.540640120743996	-0.0006103515625	\\
0.54068451191903	-0.000885009765625	\\
0.540728903094065	-0.00079345703125	\\
0.540773294269099	-0.00091552734375	\\
0.540817685444134	-0.000579833984375	\\
0.540862076619168	-0.00048828125	\\
0.540906467794202	-0.001068115234375	\\
0.540950858969237	-0.000946044921875	\\
0.540995250144271	-0.000732421875	\\
0.541039641319306	-0.000701904296875	\\
0.54108403249434	-0.000732421875	\\
0.541128423669375	-0.00079345703125	\\
0.541172814844409	-0.000396728515625	\\
0.541217206019443	-0.0001220703125	\\
0.541261597194478	-6.103515625e-05	\\
0.541305988369512	0.0006103515625	\\
0.541350379544547	0.000823974609375	\\
0.541394770719581	0.001068115234375	\\
0.541439161894615	0.001129150390625	\\
0.54148355306965	0.001190185546875	\\
0.541527944244684	0.001312255859375	\\
0.541572335419719	0.001251220703125	\\
0.541616726594753	0.00115966796875	\\
0.541661117769787	0.001190185546875	\\
0.541705508944822	0.00146484375	\\
0.541749900119856	0.00140380859375	\\
0.541794291294891	0.00128173828125	\\
0.541838682469925	0.001129150390625	\\
0.541883073644959	0.000762939453125	\\
0.541927464819994	0.000762939453125	\\
0.541971855995028	0.00048828125	\\
0.542016247170063	0.000762939453125	\\
0.542060638345097	0.00079345703125	\\
0.542105029520131	0.000335693359375	\\
0.542149420695166	0.000701904296875	\\
0.5421938118702	-0.000274658203125	\\
0.542238203045235	-0.000457763671875	\\
0.542282594220269	-9.1552734375e-05	\\
0.542326985395303	-0.000152587890625	\\
0.542371376570338	0.0003662109375	\\
0.542415767745372	-0.00030517578125	\\
0.542460158920407	-0.000518798828125	\\
0.542504550095441	-3.0517578125e-05	\\
0.542548941270475	-0.000640869140625	\\
0.54259333244551	-0.0009765625	\\
0.542637723620544	-0.001312255859375	\\
0.542682114795579	-0.001861572265625	\\
0.542726505970613	-0.0018310546875	\\
0.542770897145647	-0.001434326171875	\\
0.542815288320682	-0.001495361328125	\\
0.542859679495716	-0.001617431640625	\\
0.542904070670751	-0.001312255859375	\\
0.542948461845785	-0.0015869140625	\\
0.542992853020819	-0.001556396484375	\\
0.543037244195854	-0.001495361328125	\\
0.543081635370888	-0.001312255859375	\\
0.543126026545923	-0.0013427734375	\\
0.543170417720957	-0.00146484375	\\
0.543214808895992	-0.001068115234375	\\
0.543259200071026	-0.00128173828125	\\
0.54330359124606	-0.000885009765625	\\
0.543347982421095	-0.0006103515625	\\
0.543392373596129	-0.00079345703125	\\
0.543436764771164	-0.000274658203125	\\
0.543481155946198	-0.000518798828125	\\
0.543525547121232	-0.0003662109375	\\
0.543569938296267	3.0517578125e-05	\\
0.543614329471301	-0.0003662109375	\\
0.543658720646335	-0.000396728515625	\\
0.54370311182137	-0.0001220703125	\\
0.543747502996404	-0.00042724609375	\\
0.543791894171439	-0.000701904296875	\\
0.543836285346473	-0.000701904296875	\\
0.543880676521508	-0.0009765625	\\
0.543925067696542	-0.00079345703125	\\
0.543969458871576	-0.001007080078125	\\
0.544013850046611	-0.0009765625	\\
0.544058241221645	-0.00079345703125	\\
0.54410263239668	-0.000762939453125	\\
0.544147023571714	-0.00091552734375	\\
0.544191414746748	-0.001220703125	\\
0.544235805921783	-0.00146484375	\\
0.544280197096817	-0.00164794921875	\\
0.544324588271852	-0.00146484375	\\
0.544368979446886	-0.001739501953125	\\
0.54441337062192	-0.001739501953125	\\
0.544457761796955	-0.001556396484375	\\
0.544502152971989	-0.001556396484375	\\
0.544546544147024	-0.0010986328125	\\
0.544590935322058	-0.001678466796875	\\
0.544635326497092	-0.00189208984375	\\
0.544679717672127	-0.001373291015625	\\
0.544724108847161	-0.0013427734375	\\
0.544768500022196	-0.001220703125	\\
0.54481289119723	-0.001190185546875	\\
0.544857282372264	-0.00115966796875	\\
0.544901673547299	-0.000732421875	\\
0.544946064722333	-0.000457763671875	\\
0.544990455897368	-0.000152587890625	\\
0.545034847072402	-0.00030517578125	\\
0.545079238247436	6.103515625e-05	\\
0.545123629422471	0.000213623046875	\\
0.545168020597505	0.000274658203125	\\
0.54521241177254	0.000640869140625	\\
0.545256802947574	0.00030517578125	\\
0.545301194122608	0.000457763671875	\\
0.545345585297643	0.000823974609375	\\
0.545389976472677	0.00048828125	\\
0.545434367647712	0.00030517578125	\\
0.545478758822746	0.00042724609375	\\
0.54552314999778	0.00042724609375	\\
0.545567541172815	-0.000152587890625	\\
0.545611932347849	-0.00030517578125	\\
0.545656323522884	0	\\
0.545700714697918	-0.000152587890625	\\
0.545745105872952	-0.00067138671875	\\
0.545789497047987	-0.00042724609375	\\
0.545833888223021	-0.000823974609375	\\
0.545878279398056	-0.001373291015625	\\
0.54592267057309	-0.00189208984375	\\
0.545967061748124	-0.001800537109375	\\
0.546011452923159	-0.001129150390625	\\
0.546055844098193	-0.001556396484375	\\
0.546100235273228	-0.001708984375	\\
0.546144626448262	-0.001220703125	\\
0.546189017623297	-0.00201416015625	\\
0.546233408798331	-0.00225830078125	\\
0.546277799973365	-0.00189208984375	\\
0.5463221911484	-0.0018310546875	\\
0.546366582323434	-0.00177001953125	\\
0.546410973498468	-0.00189208984375	\\
0.546455364673503	-0.002044677734375	\\
0.546499755848537	-0.0020751953125	\\
0.546544147023572	-0.00201416015625	\\
0.546588538198606	-0.00177001953125	\\
0.546632929373641	-0.00177001953125	\\
0.546677320548675	-0.001495361328125	\\
0.546721711723709	-0.00189208984375	\\
0.546766102898744	-0.00238037109375	\\
0.546810494073778	-0.0020751953125	\\
0.546854885248813	-0.00177001953125	\\
0.546899276423847	-0.001312255859375	\\
0.546943667598881	-0.001678466796875	\\
0.546988058773916	-0.00189208984375	\\
0.54703244994895	-0.001678466796875	\\
0.547076841123985	-0.001220703125	\\
0.547121232299019	-0.001220703125	\\
0.547165623474053	-0.001678466796875	\\
0.547210014649088	-0.00115966796875	\\
0.547254405824122	-0.00091552734375	\\
0.547298796999157	-0.0010986328125	\\
0.547343188174191	-0.0010986328125	\\
0.547387579349225	-0.000946044921875	\\
0.54743197052426	-0.001190185546875	\\
0.547476361699294	-0.001312255859375	\\
0.547520752874329	-0.00128173828125	\\
0.547565144049363	-0.000946044921875	\\
0.547609535224397	-0.000579833984375	\\
0.547653926399432	-0.0008544921875	\\
0.547698317574466	-0.001068115234375	\\
0.547742708749501	-0.001068115234375	\\
0.547787099924535	-0.000885009765625	\\
0.547831491099569	-0.000701904296875	\\
0.547875882274604	-0.000732421875	\\
0.547920273449638	-0.0006103515625	\\
0.547964664624673	-0.00054931640625	\\
0.548009055799707	-0.0006103515625	\\
0.548053446974741	-0.00079345703125	\\
0.548097838149776	-0.000579833984375	\\
0.54814222932481	-0.000396728515625	\\
0.548186620499845	-0.00079345703125	\\
0.548231011674879	-0.000732421875	\\
0.548275402849913	-0.0010986328125	\\
0.548319794024948	-0.001312255859375	\\
0.548364185199982	-0.000885009765625	\\
0.548408576375017	-0.0009765625	\\
0.548452967550051	-0.00103759765625	\\
0.548497358725085	-0.001007080078125	\\
0.54854174990012	-0.0010986328125	\\
0.548586141075154	-0.00103759765625	\\
0.548630532250189	-0.0010986328125	\\
0.548674923425223	-0.001068115234375	\\
0.548719314600257	-0.00103759765625	\\
0.548763705775292	-0.001434326171875	\\
0.548808096950326	-0.001251220703125	\\
0.548852488125361	-0.000946044921875	\\
0.548896879300395	-0.000885009765625	\\
0.54894127047543	-0.00115966796875	\\
0.548985661650464	-0.000885009765625	\\
0.549030052825498	-0.00054931640625	\\
0.549074444000533	-0.000885009765625	\\
0.549118835175567	-0.001434326171875	\\
0.549163226350602	-0.00146484375	\\
0.549207617525636	-0.0013427734375	\\
0.54925200870067	-0.001312255859375	\\
0.549296399875705	-0.0013427734375	\\
0.549340791050739	-0.001373291015625	\\
0.549385182225773	-0.0010986328125	\\
0.549429573400808	-0.001312255859375	\\
0.549473964575842	-0.001373291015625	\\
0.549518355750877	-0.0015869140625	\\
0.549562746925911	-0.00140380859375	\\
0.549607138100946	-0.0010986328125	\\
0.54965152927598	-0.00115966796875	\\
0.549695920451014	-0.001251220703125	\\
0.549740311626049	-0.001861572265625	\\
0.549784702801083	-0.002288818359375	\\
0.549829093976118	-0.001983642578125	\\
0.549873485151152	-0.002197265625	\\
0.549917876326186	-0.00225830078125	\\
0.549962267501221	-0.00238037109375	\\
0.550006658676255	-0.002777099609375	\\
0.55005104985129	-0.00262451171875	\\
0.550095441026324	-0.003021240234375	\\
0.550139832201358	-0.00311279296875	\\
0.550184223376393	-0.002716064453125	\\
0.550228614551427	-0.0030517578125	\\
0.550273005726462	-0.003021240234375	\\
0.550317396901496	-0.0030517578125	\\
0.55036178807653	-0.003448486328125	\\
0.550406179251565	-0.00335693359375	\\
0.550450570426599	-0.0032958984375	\\
0.550494961601634	-0.004180908203125	\\
0.550539352776668	-0.004150390625	\\
0.550583743951702	-0.0035400390625	\\
0.550628135126737	-0.00390625	\\
0.550672526301771	-0.00445556640625	\\
0.550716917476806	-0.00390625	\\
0.55076130865184	-0.003631591796875	\\
0.550805699826874	-0.003936767578125	\\
0.550850091001909	-0.0040283203125	\\
0.550894482176943	-0.003387451171875	\\
0.550938873351978	-0.00341796875	\\
0.550983264527012	-0.00341796875	\\
0.551027655702046	-0.003082275390625	\\
0.551072046877081	-0.0032958984375	\\
0.551116438052115	-0.0035400390625	\\
0.55116082922715	-0.0032958984375	\\
0.551205220402184	-0.003173828125	\\
0.551249611577218	-0.0035400390625	\\
0.551294002752253	-0.00323486328125	\\
0.551338393927287	-0.003387451171875	\\
0.551382785102322	-0.003753662109375	\\
0.551427176277356	-0.00341796875	\\
0.55147156745239	-0.0030517578125	\\
0.551515958627425	-0.00323486328125	\\
0.551560349802459	-0.00341796875	\\
0.551604740977494	-0.00347900390625	\\
0.551649132152528	-0.00390625	\\
0.551693523327563	-0.004150390625	\\
0.551737914502597	-0.004241943359375	\\
0.551782305677631	-0.00421142578125	\\
0.551826696852666	-0.00445556640625	\\
0.5518710880277	-0.004638671875	\\
0.551915479202735	-0.004425048828125	\\
0.551959870377769	-0.003936767578125	\\
0.552004261552803	-0.0040283203125	\\
0.552048652727838	-0.00384521484375	\\
0.552093043902872	-0.0042724609375	\\
0.552137435077906	-0.00457763671875	\\
0.552181826252941	-0.004425048828125	\\
0.552226217427975	-0.004638671875	\\
0.55227060860301	-0.00469970703125	\\
0.552314999778044	-0.00439453125	\\
0.552359390953079	-0.0042724609375	\\
0.552403782128113	-0.004547119140625	\\
0.552448173303147	-0.00433349609375	\\
0.552492564478182	-0.00408935546875	\\
0.552536955653216	-0.0035400390625	\\
0.552581346828251	-0.0032958984375	\\
0.552625738003285	-0.00372314453125	\\
0.552670129178319	-0.003814697265625	\\
0.552714520353354	-0.003692626953125	\\
0.552758911528388	-0.00390625	\\
0.552803302703423	-0.00347900390625	\\
0.552847693878457	-0.003570556640625	\\
0.552892085053491	-0.0032958984375	\\
0.552936476228526	-0.0032958984375	\\
0.55298086740356	-0.0030517578125	\\
0.553025258578595	-0.00299072265625	\\
0.553069649753629	-0.003173828125	\\
0.553114040928663	-0.002410888671875	\\
0.553158432103698	-0.0023193359375	\\
0.553202823278732	-0.00250244140625	\\
0.553247214453767	-0.00201416015625	\\
0.553291605628801	-0.00164794921875	\\
0.553335996803835	-0.00164794921875	\\
0.55338038797887	-0.00177001953125	\\
0.553424779153904	-0.001617431640625	\\
0.553469170328939	-0.001251220703125	\\
0.553513561503973	-0.001617431640625	\\
0.553557952679007	-0.002166748046875	\\
0.553602343854042	-0.001312255859375	\\
0.553646735029076	-0.001007080078125	\\
0.553691126204111	-0.001739501953125	\\
0.553735517379145	-0.001617431640625	\\
0.553779908554179	-0.001129150390625	\\
0.553824299729214	-0.00128173828125	\\
0.553868690904248	-0.001556396484375	\\
0.553913082079283	-0.001495361328125	\\
0.553957473254317	-0.001983642578125	\\
0.554001864429351	-0.002166748046875	\\
0.554046255604386	-0.002410888671875	\\
0.55409064677942	-0.002105712890625	\\
0.554135037954455	-0.0020751953125	\\
0.554179429129489	-0.00213623046875	\\
0.554223820304523	-0.001861572265625	\\
0.554268211479558	-0.002227783203125	\\
0.554312602654592	-0.00201416015625	\\
0.554356993829627	-0.00244140625	\\
0.554401385004661	-0.00274658203125	\\
0.554445776179695	-0.00250244140625	\\
0.55449016735473	-0.00262451171875	\\
0.554534558529764	-0.00244140625	\\
0.554578949704799	-0.001739501953125	\\
0.554623340879833	-0.00177001953125	\\
0.554667732054868	-0.001861572265625	\\
0.554712123229902	-0.0015869140625	\\
0.554756514404936	-0.0013427734375	\\
0.554800905579971	-0.00103759765625	\\
0.554845296755005	-0.0013427734375	\\
0.554889687930039	-0.00140380859375	\\
0.554934079105074	-0.001068115234375	\\
0.554978470280108	-0.00128173828125	\\
0.555022861455143	-0.000885009765625	\\
0.555067252630177	-0.00054931640625	\\
0.555111643805212	-0.00079345703125	\\
0.555156034980246	-0.0006103515625	\\
0.55520042615528	-0.0008544921875	\\
0.555244817330315	-0.001007080078125	\\
0.555289208505349	-0.0009765625	\\
0.555333599680384	-0.001220703125	\\
0.555377990855418	-0.0013427734375	\\
0.555422382030452	-0.001220703125	\\
0.555466773205487	-0.001068115234375	\\
0.555511164380521	-0.001556396484375	\\
0.555555555555556	-0.001373291015625	\\
0.55559994673059	-0.001373291015625	\\
0.555644337905624	-0.001434326171875	\\
0.555688729080659	-0.000885009765625	\\
0.555733120255693	-0.00079345703125	\\
0.555777511430728	-0.00079345703125	\\
0.555821902605762	-0.001129150390625	\\
0.555866293780796	-0.00091552734375	\\
0.555910684955831	-0.00054931640625	\\
0.555955076130865	-0.000640869140625	\\
0.5559994673059	-0.000732421875	\\
0.556043858480934	-0.00091552734375	\\
0.556088249655968	-0.000579833984375	\\
0.556132640831003	-0.000640869140625	\\
0.556177032006037	-0.000640869140625	\\
0.556221423181072	-9.1552734375e-05	\\
0.556265814356106	-0.000335693359375	\\
0.55631020553114	-0.000640869140625	\\
0.556354596706175	-0.000213623046875	\\
0.556398987881209	-0.00054931640625	\\
0.556443379056244	-0.000640869140625	\\
0.556487770231278	-0.000457763671875	\\
0.556532161406312	0	\\
0.556576552581347	6.103515625e-05	\\
0.556620943756381	-0.000213623046875	\\
0.556665334931416	0.0001220703125	\\
0.55670972610645	0.000244140625	\\
0.556754117281485	0.0003662109375	\\
0.556798508456519	0.00048828125	\\
0.556842899631553	0.000274658203125	\\
0.556887290806588	0.00048828125	\\
0.556931681981622	0.000762939453125	\\
0.556976073156656	0.000885009765625	\\
0.557020464331691	0.00128173828125	\\
0.557064855506725	0.000823974609375	\\
0.55710924668176	0.000640869140625	\\
0.557153637856794	0.001495361328125	\\
0.557198029031828	0.001556396484375	\\
0.557242420206863	0.00140380859375	\\
0.557286811381897	0.001190185546875	\\
0.557331202556932	0.001220703125	\\
0.557375593731966	0.001251220703125	\\
0.557419984907001	0.000946044921875	\\
0.557464376082035	0.0008544921875	\\
0.557508767257069	0.0006103515625	\\
0.557553158432104	0.000579833984375	\\
0.557597549607138	0.00091552734375	\\
0.557641940782173	0.000640869140625	\\
0.557686331957207	0.000213623046875	\\
0.557730723132241	0.0001220703125	\\
0.557775114307276	0.000244140625	\\
0.55781950548231	0.000396728515625	\\
0.557863896657344	0.000335693359375	\\
0.557908287832379	0.00048828125	\\
0.557952679007413	0.00054931640625	\\
0.557997070182448	3.0517578125e-05	\\
0.558041461357482	0.0001220703125	\\
0.558085852532517	-0.000274658203125	\\
0.558130243707551	-0.00079345703125	\\
0.558174634882585	-0.00048828125	\\
0.55821902605762	-0.00079345703125	\\
0.558263417232654	-0.00103759765625	\\
0.558307808407689	-0.000823974609375	\\
0.558352199582723	-0.00115966796875	\\
0.558396590757757	-0.001129150390625	\\
0.558440981932792	-0.000946044921875	\\
0.558485373107826	-0.000640869140625	\\
0.558529764282861	-0.00054931640625	\\
0.558574155457895	-0.0006103515625	\\
0.558618546632929	-0.000640869140625	\\
0.558662937807964	-0.000701904296875	\\
0.558707328982998	-0.000823974609375	\\
0.558751720158033	-0.000335693359375	\\
0.558796111333067	-0.000579833984375	\\
0.558840502508101	-0.00048828125	\\
0.558884893683136	0.00030517578125	\\
0.55892928485817	0.000335693359375	\\
0.558973676033205	-0.000213623046875	\\
0.559018067208239	-3.0517578125e-05	\\
0.559062458383273	0.00042724609375	\\
0.559106849558308	0.000823974609375	\\
0.559151240733342	0.000762939453125	\\
0.559195631908377	0.000274658203125	\\
0.559240023083411	0.00042724609375	\\
0.559284414258445	0.00054931640625	\\
0.55932880543348	0.00018310546875	\\
0.559373196608514	0.000152587890625	\\
0.559417587783549	0.000457763671875	\\
0.559461978958583	0.00054931640625	\\
0.559506370133617	0.000244140625	\\
0.559550761308652	0.00042724609375	\\
0.559595152483686	0.00018310546875	\\
0.559639543658721	-6.103515625e-05	\\
0.559683934833755	0.0003662109375	\\
0.559728326008789	0.00042724609375	\\
0.559772717183824	0.000335693359375	\\
0.559817108358858	0.00091552734375	\\
0.559861499533893	0.00067138671875	\\
0.559905890708927	0.000396728515625	\\
0.559950281883961	0.000579833984375	\\
0.559994673058996	0.00042724609375	\\
0.56003906423403	-0.00048828125	\\
0.560083455409065	-0.00079345703125	\\
0.560127846584099	-0.000457763671875	\\
0.560172237759134	-0.000213623046875	\\
0.560216628934168	0.000213623046875	\\
0.560261020109202	0.000152587890625	\\
0.560305411284237	-0.000152587890625	\\
0.560349802459271	0.00042724609375	\\
0.560394193634306	0.00042724609375	\\
0.56043858480934	0.00030517578125	\\
0.560482975984374	0.000457763671875	\\
0.560527367159409	0.000335693359375	\\
0.560571758334443	0.000213623046875	\\
0.560616149509477	0.000396728515625	\\
0.560660540684512	0.000732421875	\\
0.560704931859546	0.000823974609375	\\
0.560749323034581	0.000762939453125	\\
0.560793714209615	0.00103759765625	\\
0.56083810538465	0.001220703125	\\
0.560882496559684	0.001434326171875	\\
0.560926887734718	0.001861572265625	\\
0.560971278909753	0.00177001953125	\\
0.561015670084787	0.002044677734375	\\
0.561060061259822	0.002777099609375	\\
0.561104452434856	0.002838134765625	\\
0.56114884360989	0.00262451171875	\\
0.561193234784925	0.002349853515625	\\
0.561237625959959	0.002471923828125	\\
0.561282017134994	0.0029296875	\\
0.561326408310028	0.002838134765625	\\
0.561370799485062	0.00250244140625	\\
0.561415190660097	0.002532958984375	\\
0.561459581835131	0.002685546875	\\
0.561503973010166	0.00244140625	\\
0.5615483641852	0.002227783203125	\\
0.561592755360234	0.00238037109375	\\
0.561637146535269	0.00225830078125	\\
0.561681537710303	0.0020751953125	\\
0.561725928885338	0.001953125	\\
0.561770320060372	0.001617431640625	\\
0.561814711235406	0.001434326171875	\\
0.561859102410441	0.001495361328125	\\
0.561903493585475	0.001312255859375	\\
0.56194788476051	0.00128173828125	\\
0.561992275935544	0.00115966796875	\\
0.562036667110578	0.001129150390625	\\
0.562081058285613	0.0009765625	\\
0.562125449460647	0.00079345703125	\\
0.562169840635682	0.001190185546875	\\
0.562214231810716	0.001190185546875	\\
0.56225862298575	0.001007080078125	\\
0.562303014160785	0.000701904296875	\\
0.562347405335819	0.000762939453125	\\
0.562391796510854	0.00103759765625	\\
0.562436187685888	0.000885009765625	\\
0.562480578860922	0.000518798828125	\\
0.562524970035957	0.000518798828125	\\
0.562569361210991	0.00067138671875	\\
0.562613752386026	0.00079345703125	\\
0.56265814356106	0.000732421875	\\
0.562702534736094	0.001129150390625	\\
0.562746925911129	0.001312255859375	\\
0.562791317086163	0.000823974609375	\\
0.562835708261198	0.001129150390625	\\
0.562880099436232	0.001007080078125	\\
0.562924490611266	0.00091552734375	\\
0.562968881786301	0.00128173828125	\\
0.563013272961335	0.001495361328125	\\
0.56305766413637	0.001556396484375	\\
0.563102055311404	0.001983642578125	\\
0.563146446486439	0.002532958984375	\\
0.563190837661473	0.002471923828125	\\
0.563235228836507	0.00213623046875	\\
0.563279620011542	0.0023193359375	\\
0.563324011186576	0.00213623046875	\\
0.563368402361611	0.001953125	\\
0.563412793536645	0.0020751953125	\\
0.563457184711679	0.001983642578125	\\
0.563501575886714	0.001953125	\\
0.563545967061748	0.002471923828125	\\
0.563590358236783	0.00225830078125	\\
0.563634749411817	0.001983642578125	\\
0.563679140586851	0.002410888671875	\\
0.563723531761886	0.00244140625	\\
0.56376792293692	0.00250244140625	\\
0.563812314111955	0.002532958984375	\\
0.563856705286989	0.0025634765625	\\
0.563901096462023	0.002532958984375	\\
0.563945487637058	0.002532958984375	\\
0.563989878812092	0.00238037109375	\\
0.564034269987127	0.00250244140625	\\
0.564078661162161	0.002349853515625	\\
0.564123052337195	0.00213623046875	\\
0.56416744351223	0.002532958984375	\\
0.564211834687264	0.002685546875	\\
0.564256225862299	0.002899169921875	\\
0.564300617037333	0.002685546875	\\
0.564345008212367	0.0028076171875	\\
0.564389399387402	0.002899169921875	\\
0.564433790562436	0.002349853515625	\\
0.564478181737471	0.0023193359375	\\
0.564522572912505	0.00238037109375	\\
0.564566964087539	0.00250244140625	\\
0.564611355262574	0.00225830078125	\\
0.564655746437608	0.00201416015625	\\
0.564700137612643	0.002166748046875	\\
0.564744528787677	0.002105712890625	\\
0.564788919962711	0.00225830078125	\\
0.564833311137746	0.002197265625	\\
0.56487770231278	0.002410888671875	\\
0.564922093487815	0.002288818359375	\\
0.564966484662849	0.00201416015625	\\
0.565010875837883	0.002471923828125	\\
0.565055267012918	0.00244140625	\\
0.565099658187952	0.002410888671875	\\
0.565144049362987	0.00250244140625	\\
0.565188440538021	0.002197265625	\\
0.565232831713055	0.001861572265625	\\
0.56527722288809	0.001708984375	\\
0.565321614063124	0.001312255859375	\\
0.565366005238159	0.00103759765625	\\
0.565410396413193	0.001708984375	\\
0.565454787588227	0.001495361328125	\\
0.565499178763262	0.001220703125	\\
0.565543569938296	0.00146484375	\\
0.565587961113331	0.001495361328125	\\
0.565632352288365	0.001495361328125	\\
0.565676743463399	0.001617431640625	\\
0.565721134638434	0.0010986328125	\\
0.565765525813468	0.000946044921875	\\
0.565809916988503	0.001373291015625	\\
0.565854308163537	0.001739501953125	\\
0.565898699338572	0.0013427734375	\\
0.565943090513606	0.001190185546875	\\
0.56598748168864	0.001434326171875	\\
0.566031872863675	0.0013427734375	\\
0.566076264038709	0.00103759765625	\\
0.566120655213744	0.0010986328125	\\
0.566165046388778	0.001373291015625	\\
0.566209437563812	0.001129150390625	\\
0.566253828738847	0.001312255859375	\\
0.566298219913881	0.001495361328125	\\
0.566342611088915	0.001220703125	\\
0.56638700226395	0.00146484375	\\
0.566431393438984	0.001739501953125	\\
0.566475784614019	0.001434326171875	\\
0.566520175789053	0.00146484375	\\
0.566564566964088	0.0010986328125	\\
0.566608958139122	0.001129150390625	\\
0.566653349314156	0.001312255859375	\\
0.566697740489191	0.0008544921875	\\
0.566742131664225	0.000946044921875	\\
0.56678652283926	0.00103759765625	\\
0.566830914014294	0.001190185546875	\\
0.566875305189328	0.001190185546875	\\
0.566919696364363	0.001312255859375	\\
0.566964087539397	0.00140380859375	\\
0.567008478714432	0.00091552734375	\\
0.567052869889466	0.00079345703125	\\
0.5670972610645	0.00091552734375	\\
0.567141652239535	0.0009765625	\\
0.567186043414569	0.0010986328125	\\
0.567230434589604	0.000640869140625	\\
0.567274825764638	0.000396728515625	\\
0.567319216939672	0.00115966796875	\\
0.567363608114707	0.001312255859375	\\
0.567407999289741	0.0009765625	\\
0.567452390464776	0.000885009765625	\\
0.56749678163981	0.000762939453125	\\
0.567541172814844	0.0008544921875	\\
0.567585563989879	0.0006103515625	\\
0.567629955164913	0.000335693359375	\\
0.567674346339948	0.000213623046875	\\
0.567718737514982	3.0517578125e-05	\\
0.567763128690016	0	\\
0.567807519865051	-0.00054931640625	\\
0.567851911040085	-0.00103759765625	\\
0.56789630221512	-0.00115966796875	\\
0.567940693390154	-0.00079345703125	\\
0.567985084565188	-0.000885009765625	\\
0.568029475740223	-0.000457763671875	\\
0.568073866915257	-0.00054931640625	\\
0.568118258090292	-0.001129150390625	\\
0.568162649265326	-0.00042724609375	\\
0.56820704044036	-0.000457763671875	\\
0.568251431615395	-0.00091552734375	\\
0.568295822790429	-0.0006103515625	\\
0.568340213965464	-0.001068115234375	\\
0.568384605140498	-0.001129150390625	\\
0.568428996315532	-0.000823974609375	\\
0.568473387490567	-0.000640869140625	\\
0.568517778665601	-0.0008544921875	\\
0.568562169840636	-0.00091552734375	\\
0.56860656101567	-0.0008544921875	\\
0.568650952190705	-0.0009765625	\\
0.568695343365739	-0.000518798828125	\\
0.568739734540773	-0.00042724609375	\\
0.568784125715808	-0.000152587890625	\\
0.568828516890842	0.000213623046875	\\
0.568872908065877	0	\\
0.568917299240911	0.00018310546875	\\
0.568961690415945	-3.0517578125e-05	\\
0.56900608159098	3.0517578125e-05	\\
0.569050472766014	-0.0001220703125	\\
0.569094863941048	-0.000274658203125	\\
0.569139255116083	-6.103515625e-05	\\
0.569183646291117	-0.00030517578125	\\
0.569228037466152	-0.000518798828125	\\
0.569272428641186	-0.00018310546875	\\
0.569316819816221	-6.103515625e-05	\\
0.569361210991255	-0.000213623046875	\\
0.569405602166289	-0.00018310546875	\\
0.569449993341324	3.0517578125e-05	\\
0.569494384516358	-0.000396728515625	\\
0.569538775691393	-0.0001220703125	\\
0.569583166866427	0.00018310546875	\\
0.569627558041461	-0.000244140625	\\
0.569671949216496	-3.0517578125e-05	\\
0.56971634039153	-0.00030517578125	\\
0.569760731566565	-0.000579833984375	\\
0.569805122741599	-0.000732421875	\\
0.569849513916633	-0.0008544921875	\\
0.569893905091668	-0.000640869140625	\\
0.569938296266702	-0.00054931640625	\\
0.569982687441737	-0.00030517578125	\\
0.570027078616771	-0.000640869140625	\\
0.570071469791805	-0.000885009765625	\\
0.57011586096684	-0.0008544921875	\\
0.570160252141874	-0.000518798828125	\\
0.570204643316909	-0.00067138671875	\\
0.570249034491943	-0.00079345703125	\\
0.570293425666977	-0.00067138671875	\\
0.570337816842012	-0.00048828125	\\
0.570382208017046	-0.000396728515625	\\
0.570426599192081	-0.000579833984375	\\
0.570470990367115	-0.000396728515625	\\
0.570515381542149	-6.103515625e-05	\\
0.570559772717184	-0.0001220703125	\\
0.570604163892218	-6.103515625e-05	\\
0.570648555067253	0.00030517578125	\\
0.570692946242287	0.00048828125	\\
0.570737337417321	0.00030517578125	\\
0.570781728592356	0.0001220703125	\\
0.57082611976739	0.0003662109375	\\
0.570870510942425	0.000640869140625	\\
0.570914902117459	0.0006103515625	\\
0.570959293292493	0.000213623046875	\\
0.571003684467528	0.0003662109375	\\
0.571048075642562	0.00054931640625	\\
0.571092466817597	0.000396728515625	\\
0.571136857992631	0.0003662109375	\\
0.571181249167665	0.000518798828125	\\
0.5712256403427	0.000946044921875	\\
0.571270031517734	0.00079345703125	\\
0.571314422692769	0.000762939453125	\\
0.571358813867803	0.000885009765625	\\
0.571403205042837	0.00054931640625	\\
0.571447596217872	0.00067138671875	\\
0.571491987392906	0.00067138671875	\\
0.571536378567941	0.000335693359375	\\
0.571580769742975	0.00067138671875	\\
0.57162516091801	0.00042724609375	\\
0.571669552093044	0.000244140625	\\
0.571713943268078	0.000396728515625	\\
0.571758334443113	0.000396728515625	\\
0.571802725618147	0.0010986328125	\\
0.571847116793182	0.0013427734375	\\
0.571891507968216	0.000823974609375	\\
0.57193589914325	0.00067138671875	\\
0.571980290318285	0.000885009765625	\\
0.572024681493319	0.000335693359375	\\
0.572069072668354	0.00048828125	\\
0.572113463843388	0.00079345703125	\\
0.572157855018422	0.00048828125	\\
0.572202246193457	0.000518798828125	\\
0.572246637368491	0.000823974609375	\\
0.572291028543526	0.000457763671875	\\
0.57233541971856	0.00054931640625	\\
0.572379810893594	0.000823974609375	\\
0.572424202068629	0.00103759765625	\\
0.572468593243663	0.001129150390625	\\
0.572512984418698	0.00067138671875	\\
0.572557375593732	0.000701904296875	\\
0.572601766768766	0.001068115234375	\\
0.572646157943801	0.001373291015625	\\
0.572690549118835	0.00140380859375	\\
0.57273494029387	0.0013427734375	\\
0.572779331468904	0.001220703125	\\
0.572823722643938	0.00140380859375	\\
0.572868113818973	0.001220703125	\\
0.572912504994007	0.000457763671875	\\
0.572956896169042	0.0006103515625	\\
0.573001287344076	0.001312255859375	\\
0.57304567851911	0.00115966796875	\\
0.573090069694145	0.0009765625	\\
0.573134460869179	0.0009765625	\\
0.573178852044214	0.0013427734375	\\
0.573223243219248	0.000762939453125	\\
0.573267634394282	0.000701904296875	\\
0.573312025569317	0.0008544921875	\\
0.573356416744351	0.0003662109375	\\
0.573400807919386	-0.00018310546875	\\
0.57344519909442	-0.00048828125	\\
0.573489590269454	-0.000396728515625	\\
0.573533981444489	-0.0003662109375	\\
0.573578372619523	-0.000274658203125	\\
0.573622763794558	6.103515625e-05	\\
0.573667154969592	0.0001220703125	\\
0.573711546144626	0.00018310546875	\\
0.573755937319661	0.0001220703125	\\
0.573800328494695	0.000152587890625	\\
0.57384471966973	0.0003662109375	\\
0.573889110844764	0.0001220703125	\\
0.573933502019798	3.0517578125e-05	\\
0.573977893194833	9.1552734375e-05	\\
0.574022284369867	3.0517578125e-05	\\
0.574066675544902	0.000152587890625	\\
0.574111066719936	0.00048828125	\\
0.57415545789497	0.00018310546875	\\
0.574199849070005	-0.00018310546875	\\
0.574244240245039	0.00018310546875	\\
0.574288631420074	0.000579833984375	\\
0.574333022595108	0.00067138671875	\\
0.574377413770143	0.000152587890625	\\
0.574421804945177	0.000579833984375	\\
0.574466196120211	0.00054931640625	\\
0.574510587295246	0.0006103515625	\\
0.57455497847028	0.00048828125	\\
0.574599369645315	0.00079345703125	\\
0.574643760820349	0.001251220703125	\\
0.574688151995383	0.00079345703125	\\
0.574732543170418	0.0008544921875	\\
0.574776934345452	0.0010986328125	\\
0.574821325520486	0.001251220703125	\\
0.574865716695521	0.00140380859375	\\
0.574910107870555	0.000518798828125	\\
0.57495449904559	0.000274658203125	\\
0.574998890220624	0.0003662109375	\\
0.575043281395659	0.000457763671875	\\
0.575087672570693	0.00079345703125	\\
0.575132063745727	0.00048828125	\\
0.575176454920762	0.000762939453125	\\
0.575220846095796	0.000885009765625	\\
0.575265237270831	0.000274658203125	\\
0.575309628445865	0	\\
0.575354019620899	0.00042724609375	\\
0.575398410795934	0.00030517578125	\\
0.575442801970968	6.103515625e-05	\\
0.575487193146003	0.00030517578125	\\
0.575531584321037	0.000152587890625	\\
0.575575975496071	0.000244140625	\\
0.575620366671106	0	\\
0.57566475784614	-0.000213623046875	\\
0.575709149021175	-0.00030517578125	\\
0.575753540196209	-0.000396728515625	\\
0.575797931371243	-0.0003662109375	\\
0.575842322546278	-0.00054931640625	\\
0.575886713721312	-0.000335693359375	\\
0.575931104896347	-0.000732421875	\\
0.575975496071381	-0.000701904296875	\\
0.576019887246415	-0.000274658203125	\\
0.57606427842145	-9.1552734375e-05	\\
0.576108669596484	-6.103515625e-05	\\
0.576153060771519	-9.1552734375e-05	\\
0.576197451946553	0.000213623046875	\\
0.576241843121587	0.000244140625	\\
0.576286234296622	0.00018310546875	\\
0.576330625471656	6.103515625e-05	\\
0.576375016646691	-0.000213623046875	\\
0.576419407821725	-9.1552734375e-05	\\
0.576463798996759	0.0001220703125	\\
0.576508190171794	0.000640869140625	\\
0.576552581346828	0.0003662109375	\\
0.576596972521863	-0.0001220703125	\\
0.576641363696897	0.000244140625	\\
0.576685754871931	0.000213623046875	\\
0.576730146046966	0.000244140625	\\
0.576774537222	0.000701904296875	\\
0.576818928397035	0.001129150390625	\\
0.576863319572069	0.00115966796875	\\
0.576907710747103	0.001007080078125	\\
0.576952101922138	0.00079345703125	\\
0.576996493097172	0.00091552734375	\\
0.577040884272207	0.001129150390625	\\
0.577085275447241	0.001312255859375	\\
0.577129666622275	0.001068115234375	\\
0.57717405779731	0.000823974609375	\\
0.577218448972344	0.00103759765625	\\
0.577262840147379	0.00146484375	\\
0.577307231322413	0.001220703125	\\
0.577351622497448	0.00140380859375	\\
0.577396013672482	0.001708984375	\\
0.577440404847516	0.0010986328125	\\
0.577484796022551	0.00103759765625	\\
0.577529187197585	0.0008544921875	\\
0.57757357837262	0.001007080078125	\\
0.577617969547654	0.00140380859375	\\
0.577662360722688	0.0009765625	\\
0.577706751897723	0.00128173828125	\\
0.577751143072757	0.00146484375	\\
0.577795534247792	0.0009765625	\\
0.577839925422826	0.000946044921875	\\
0.57788431659786	0.00091552734375	\\
0.577928707772895	0.0008544921875	\\
0.577973098947929	0.000946044921875	\\
0.578017490122964	0.00115966796875	\\
0.578061881297998	0.0009765625	\\
0.578106272473032	0.000885009765625	\\
0.578150663648067	0.0009765625	\\
0.578195054823101	0.000732421875	\\
0.578239445998136	0.000732421875	\\
0.57828383717317	0.000946044921875	\\
0.578328228348204	0.0009765625	\\
0.578372619523239	0.00091552734375	\\
0.578417010698273	0.001373291015625	\\
0.578461401873308	0.00140380859375	\\
0.578505793048342	0.001220703125	\\
0.578550184223376	0.00146484375	\\
0.578594575398411	0.00189208984375	\\
0.578638966573445	0.00164794921875	\\
0.57868335774848	0.001617431640625	\\
0.578727748923514	0.002410888671875	\\
0.578772140098548	0.0023193359375	\\
0.578816531273583	0.002044677734375	\\
0.578860922448617	0.002288818359375	\\
0.578905313623652	0.0025634765625	\\
0.578949704798686	0.0023193359375	\\
0.57899409597372	0.002410888671875	\\
0.579038487148755	0.00274658203125	\\
0.579082878323789	0.002960205078125	\\
0.579127269498824	0.00341796875	\\
0.579171660673858	0.003173828125	\\
0.579216051848892	0.003326416015625	\\
0.579260443023927	0.00384521484375	\\
0.579304834198961	0.003692626953125	\\
0.579349225373996	0.0035400390625	\\
0.57939361654903	0.003631591796875	\\
0.579438007724065	0.003326416015625	\\
0.579482398899099	0.00360107421875	\\
0.579526790074133	0.0037841796875	\\
0.579571181249168	0.003143310546875	\\
0.579615572424202	0.003570556640625	\\
0.579659963599236	0.00396728515625	\\
0.579704354774271	0.0037841796875	\\
0.579748745949305	0.003814697265625	\\
0.57979313712434	0.0037841796875	\\
0.579837528299374	0.003662109375	\\
0.579881919474408	0.003570556640625	\\
0.579926310649443	0.003692626953125	\\
0.579970701824477	0.003509521484375	\\
0.580015092999512	0.0035400390625	\\
0.580059484174546	0.0037841796875	\\
0.580103875349581	0.00372314453125	\\
0.580148266524615	0.003814697265625	\\
0.580192657699649	0.00372314453125	\\
0.580237048874684	0.003631591796875	\\
0.580281440049718	0.003448486328125	\\
0.580325831224753	0.003662109375	\\
0.580370222399787	0.0037841796875	\\
0.580414613574821	0.003997802734375	\\
0.580459004749856	0.0040283203125	\\
0.58050339592489	0.003753662109375	\\
0.580547787099924	0.00390625	\\
0.580592178274959	0.004364013671875	\\
0.580636569449993	0.004180908203125	\\
0.580680960625028	0.003570556640625	\\
0.580725351800062	0.003875732421875	\\
0.580769742975097	0.004119873046875	\\
0.580814134150131	0.00360107421875	\\
0.580858525325165	0.003326416015625	\\
0.5809029165002	0.00323486328125	\\
0.580947307675234	0.003173828125	\\
0.580991698850269	0.003387451171875	\\
0.581036090025303	0.003326416015625	\\
0.581080481200337	0.0030517578125	\\
0.581124872375372	0.00323486328125	\\
0.581169263550406	0.003173828125	\\
0.581213654725441	0.002532958984375	\\
0.581258045900475	0.00244140625	\\
0.581302437075509	0.00262451171875	\\
0.581346828250544	0.00238037109375	\\
0.581391219425578	0.00250244140625	\\
0.581435610600613	0.002349853515625	\\
0.581480001775647	0.002227783203125	\\
0.581524392950681	0.002532958984375	\\
0.581568784125716	0.002532958984375	\\
0.58161317530075	0.00244140625	\\
0.581657566475785	0.002777099609375	\\
0.581701957650819	0.002593994140625	\\
0.581746348825853	0.002349853515625	\\
0.581790740000888	0.0023193359375	\\
0.581835131175922	0.0020751953125	\\
0.581879522350957	0.00201416015625	\\
0.581923913525991	0.00250244140625	\\
0.581968304701025	0.002471923828125	\\
0.58201269587606	0.002532958984375	\\
0.582057087051094	0.00274658203125	\\
0.582101478226129	0.00262451171875	\\
0.582145869401163	0.002960205078125	\\
0.582190260576197	0.00274658203125	\\
0.582234651751232	0.002777099609375	\\
0.582279042926266	0.00311279296875	\\
0.582323434101301	0.002899169921875	\\
0.582367825276335	0.0030517578125	\\
0.582412216451369	0.003143310546875	\\
0.582456607626404	0.002960205078125	\\
0.582500998801438	0.0030517578125	\\
0.582545389976473	0.003082275390625	\\
0.582589781151507	0.002838134765625	\\
0.582634172326541	0.00262451171875	\\
0.582678563501576	0.002685546875	\\
0.58272295467661	0.002716064453125	\\
0.582767345851645	0.0028076171875	\\
0.582811737026679	0.00341796875	\\
0.582856128201714	0.00347900390625	\\
0.582900519376748	0.00286865234375	\\
0.582944910551782	0.003173828125	\\
0.582989301726817	0.00335693359375	\\
0.583033692901851	0.003082275390625	\\
0.583078084076886	0.00286865234375	\\
0.58312247525192	0.00262451171875	\\
0.583166866426954	0.003082275390625	\\
0.583211257601989	0.00335693359375	\\
0.583255648777023	0.003143310546875	\\
0.583300039952057	0.0029296875	\\
0.583344431127092	0.00286865234375	\\
0.583388822302126	0.00299072265625	\\
0.583433213477161	0.00286865234375	\\
0.583477604652195	0.002288818359375	\\
0.58352199582723	0.002166748046875	\\
0.583566387002264	0.002349853515625	\\
0.583610778177298	0.002838134765625	\\
0.583655169352333	0.002899169921875	\\
0.583699560527367	0.002685546875	\\
0.583743951702402	0.002471923828125	\\
0.583788342877436	0.0025634765625	\\
0.58383273405247	0.002471923828125	\\
0.583877125227505	0.002349853515625	\\
0.583921516402539	0.002105712890625	\\
0.583965907577574	0.00225830078125	\\
0.584010298752608	0.002349853515625	\\
0.584054689927642	0.001922607421875	\\
0.584099081102677	0.0020751953125	\\
0.584143472277711	0.002197265625	\\
0.584187863452746	0.0020751953125	\\
0.58423225462778	0.001983642578125	\\
0.584276645802814	0.00201416015625	\\
0.584321036977849	0.00225830078125	\\
0.584365428152883	0.002685546875	\\
0.584409819327918	0.00244140625	\\
0.584454210502952	0.002227783203125	\\
0.584498601677986	0.002227783203125	\\
0.584542992853021	0.002410888671875	\\
0.584587384028055	0.002105712890625	\\
0.58463177520309	0.002197265625	\\
0.584676166378124	0.002716064453125	\\
0.584720557553158	0.002716064453125	\\
0.584764948728193	0.002410888671875	\\
0.584809339903227	0.0023193359375	\\
0.584853731078262	0.002685546875	\\
0.584898122253296	0.0023193359375	\\
0.58494251342833	0.001953125	\\
0.584986904603365	0.001922607421875	\\
0.585031295778399	0.00213623046875	\\
0.585075686953434	0.00201416015625	\\
0.585120078128468	0.001800537109375	\\
0.585164469303502	0.002288818359375	\\
0.585208860478537	0.002471923828125	\\
0.585253251653571	0.0020751953125	\\
0.585297642828606	0.002105712890625	\\
0.58534203400364	0.002685546875	\\
0.585386425178674	0.002471923828125	\\
0.585430816353709	0.002227783203125	\\
0.585475207528743	0.00244140625	\\
0.585519598703778	0.0023193359375	\\
0.585563989878812	0.002166748046875	\\
0.585608381053846	0.002349853515625	\\
0.585652772228881	0.002227783203125	\\
0.585697163403915	0.00238037109375	\\
0.58574155457895	0.00225830078125	\\
0.585785945753984	0.001983642578125	\\
0.585830336929019	0.00201416015625	\\
0.585874728104053	0.001800537109375	\\
0.585919119279087	0.00146484375	\\
0.585963510454122	0.001800537109375	\\
0.586007901629156	0.002044677734375	\\
0.586052292804191	0.0015869140625	\\
0.586096683979225	0.00152587890625	\\
0.586141075154259	0.001495361328125	\\
0.586185466329294	0.000946044921875	\\
0.586229857504328	0.001373291015625	\\
0.586274248679363	0.00146484375	\\
0.586318639854397	0.001190185546875	\\
0.586363031029431	0.00146484375	\\
0.586407422204466	0.00128173828125	\\
0.5864518133795	0.001220703125	\\
0.586496204554535	0.001251220703125	\\
0.586540595729569	0.001556396484375	\\
0.586584986904603	0.0013427734375	\\
0.586629378079638	0.00140380859375	\\
0.586673769254672	0.00201416015625	\\
0.586718160429707	0.0015869140625	\\
0.586762551604741	0.001708984375	\\
0.586806942779775	0.001708984375	\\
0.58685133395481	0.001556396484375	\\
0.586895725129844	0.002105712890625	\\
0.586940116304879	0.001708984375	\\
0.586984507479913	0.001434326171875	\\
0.587028898654947	0.001800537109375	\\
0.587073289829982	0.002105712890625	\\
0.587117681005016	0.002197265625	\\
0.587162072180051	0.00189208984375	\\
0.587206463355085	0.001312255859375	\\
0.587250854530119	0.001373291015625	\\
0.587295245705154	0.001495361328125	\\
0.587339636880188	0.001220703125	\\
0.587384028055223	0.001434326171875	\\
0.587428419230257	0.0009765625	\\
0.587472810405291	0.001129150390625	\\
0.587517201580326	0.001495361328125	\\
0.58756159275536	0.0013427734375	\\
0.587605983930395	0.001251220703125	\\
0.587650375105429	0.001129150390625	\\
0.587694766280463	0.001495361328125	\\
0.587739157455498	0.00140380859375	\\
0.587783548630532	0.00103759765625	\\
0.587827939805567	0.001068115234375	\\
0.587872330980601	0.001190185546875	\\
0.587916722155636	0.001129150390625	\\
0.58796111333067	0.001251220703125	\\
0.588005504505704	0.00128173828125	\\
0.588049895680739	0.00091552734375	\\
0.588094286855773	0.001617431640625	\\
0.588138678030807	0.001922607421875	\\
0.588183069205842	0.0018310546875	\\
0.588227460380876	0.002288818359375	\\
0.588271851555911	0.002349853515625	\\
0.588316242730945	0.00189208984375	\\
0.588360633905979	0.001800537109375	\\
0.588405025081014	0.001983642578125	\\
0.588449416256048	0.00189208984375	\\
0.588493807431083	0.001922607421875	\\
0.588538198606117	0.002349853515625	\\
0.588582589781152	0.00225830078125	\\
0.588626980956186	0.002105712890625	\\
0.58867137213122	0.0023193359375	\\
0.588715763306255	0.002716064453125	\\
0.588760154481289	0.00299072265625	\\
0.588804545656324	0.00262451171875	\\
0.588848936831358	0.002685546875	\\
0.588893328006392	0.002655029296875	\\
0.588937719181427	0.00189208984375	\\
0.588982110356461	0.002227783203125	\\
0.589026501531495	0.002471923828125	\\
0.58907089270653	0.00225830078125	\\
0.589115283881564	0.001953125	\\
0.589159675056599	0.00177001953125	\\
0.589204066231633	0.00189208984375	\\
0.589248457406668	0.002227783203125	\\
0.589292848581702	0.0018310546875	\\
0.589337239756736	0.00140380859375	\\
0.589381630931771	0.0015869140625	\\
0.589426022106805	0.001708984375	\\
0.58947041328184	0.001556396484375	\\
0.589514804456874	0.00146484375	\\
0.589559195631908	0.001190185546875	\\
0.589603586806943	0.001312255859375	\\
0.589647977981977	0.001129150390625	\\
0.589692369157012	0.0009765625	\\
0.589736760332046	0.001373291015625	\\
0.58978115150708	0.0013427734375	\\
0.589825542682115	0.001129150390625	\\
0.589869933857149	0.00146484375	\\
0.589914325032184	0.001617431640625	\\
0.589958716207218	0.0013427734375	\\
0.590003107382252	0.001251220703125	\\
0.590047498557287	0.001220703125	\\
0.590091889732321	0.000885009765625	\\
0.590136280907356	0.000640869140625	\\
0.59018067208239	0.00079345703125	\\
0.590225063257424	0.001129150390625	\\
0.590269454432459	0.000732421875	\\
0.590313845607493	0.000640869140625	\\
0.590358236782528	0.00128173828125	\\
0.590402627957562	0.001129150390625	\\
0.590447019132596	0.000732421875	\\
0.590491410307631	0.000640869140625	\\
0.590535801482665	0.000640869140625	\\
0.5905801926577	0.00079345703125	\\
0.590624583832734	0.00079345703125	\\
0.590668975007768	0.000885009765625	\\
0.590713366182803	0.001068115234375	\\
0.590757757357837	0.001129150390625	\\
0.590802148532872	0.001190185546875	\\
0.590846539707906	0.001129150390625	\\
0.59089093088294	0.001007080078125	\\
0.590935322057975	0.001068115234375	\\
0.590979713233009	0.00103759765625	\\
0.591024104408044	0.0008544921875	\\
0.591068495583078	0.000640869140625	\\
0.591112886758112	0.000732421875	\\
0.591157277933147	0.00048828125	\\
0.591201669108181	0.000335693359375	\\
0.591246060283216	0.000244140625	\\
0.59129045145825	0.0003662109375	\\
0.591334842633285	0.000518798828125	\\
0.591379233808319	0.0006103515625	\\
0.591423624983353	0.000244140625	\\
0.591468016158388	0.000396728515625	\\
0.591512407333422	0.000274658203125	\\
0.591556798508457	0.000244140625	\\
0.591601189683491	0.00018310546875	\\
0.591645580858525	0.0001220703125	\\
0.59168997203356	9.1552734375e-05	\\
0.591734363208594	-0.000213623046875	\\
0.591778754383628	0	\\
0.591823145558663	-0.0001220703125	\\
0.591867536733697	-0.00048828125	\\
0.591911927908732	-0.000274658203125	\\
0.591956319083766	-0.00054931640625	\\
0.592000710258801	-0.000274658203125	\\
0.592045101433835	0.00018310546875	\\
0.592089492608869	-0.000244140625	\\
0.592133883783904	0.0001220703125	\\
0.592178274958938	0.000335693359375	\\
0.592222666133973	0.000152587890625	\\
0.592267057309007	0.00067138671875	\\
0.592311448484041	0.00042724609375	\\
0.592355839659076	6.103515625e-05	\\
0.59240023083411	0.00030517578125	\\
0.592444622009145	-6.103515625e-05	\\
0.592489013184179	-0.000213623046875	\\
0.592533404359213	0.0001220703125	\\
0.592577795534248	3.0517578125e-05	\\
0.592622186709282	-0.000396728515625	\\
0.592666577884317	-0.000396728515625	\\
0.592710969059351	-0.00054931640625	\\
0.592755360234385	-0.00067138671875	\\
0.59279975140942	-0.000518798828125	\\
0.592844142584454	-0.000640869140625	\\
0.592888533759489	-0.00054931640625	\\
0.592932924934523	-0.0009765625	\\
0.592977316109557	-0.00103759765625	\\
0.593021707284592	-0.000823974609375	\\
0.593066098459626	-0.001190185546875	\\
0.593110489634661	-0.001251220703125	\\
0.593154880809695	-0.001251220703125	\\
0.593199271984729	-0.00152587890625	\\
0.593243663159764	-0.001312255859375	\\
0.593288054334798	-0.001251220703125	\\
0.593332445509833	-0.001495361328125	\\
0.593376836684867	-0.00140380859375	\\
0.593421227859901	-0.000823974609375	\\
0.593465619034936	-0.001190185546875	\\
0.59351001020997	-0.001129150390625	\\
0.593554401385005	-0.00115966796875	\\
0.593598792560039	-0.000823974609375	\\
0.593643183735074	-0.000335693359375	\\
0.593687574910108	-0.0008544921875	\\
0.593731966085142	-0.000244140625	\\
0.593776357260177	-0.000244140625	\\
0.593820748435211	-0.00103759765625	\\
0.593865139610245	-0.00091552734375	\\
0.59390953078528	-0.000213623046875	\\
0.593953921960314	-0.0001220703125	\\
0.593998313135349	-0.000213623046875	\\
0.594042704310383	0	\\
0.594087095485417	-3.0517578125e-05	\\
0.594131486660452	-9.1552734375e-05	\\
0.594175877835486	0	\\
0.594220269010521	0.000274658203125	\\
0.594264660185555	0.0006103515625	\\
0.59430905136059	0.00079345703125	\\
0.594353442535624	0.0003662109375	\\
0.594397833710658	0.000335693359375	\\
0.594442224885693	0.000518798828125	\\
0.594486616060727	0.000732421875	\\
0.594531007235762	0.00091552734375	\\
0.594575398410796	0.0006103515625	\\
0.59461978958583	0.00067138671875	\\
0.594664180760865	0.000946044921875	\\
0.594708571935899	0.000946044921875	\\
0.594752963110934	0.0008544921875	\\
0.594797354285968	0.000640869140625	\\
0.594841745461002	0.00042724609375	\\
0.594886136636037	0.00054931640625	\\
0.594930527811071	0.0010986328125	\\
0.594974918986106	0.0010986328125	\\
0.59501931016114	0.000457763671875	\\
0.595063701336174	0.00079345703125	\\
0.595108092511209	0.000885009765625	\\
0.595152483686243	0.0003662109375	\\
0.595196874861278	0.00054931640625	\\
0.595241266036312	0.00054931640625	\\
0.595285657211346	0.000213623046875	\\
0.595330048386381	0.000457763671875	\\
0.595374439561415	0.000335693359375	\\
0.59541883073645	-3.0517578125e-05	\\
0.595463221911484	0.000274658203125	\\
0.595507613086518	0.000213623046875	\\
0.595552004261553	-0.000213623046875	\\
0.595596395436587	0.000152587890625	\\
0.595640786611622	-0.000152587890625	\\
0.595685177786656	-0.00030517578125	\\
0.59572956896169	0.000518798828125	\\
0.595773960136725	0.0008544921875	\\
0.595818351311759	0.001068115234375	\\
0.595862742486794	0.0008544921875	\\
0.595907133661828	0.00091552734375	\\
0.595951524836862	0.001007080078125	\\
0.595995916011897	0.000823974609375	\\
0.596040307186931	0.000732421875	\\
0.596084698361966	0.000732421875	\\
0.596129089537	0.00091552734375	\\
0.596173480712034	0.000823974609375	\\
0.596217871887069	0.0008544921875	\\
0.596262263062103	0.00115966796875	\\
0.596306654237138	0.0010986328125	\\
0.596351045412172	0.000885009765625	\\
0.596395436587207	0.001129150390625	\\
0.596439827762241	0.001190185546875	\\
0.596484218937275	0.000885009765625	\\
0.59652861011231	0.000762939453125	\\
0.596573001287344	0.0006103515625	\\
0.596617392462378	0.0006103515625	\\
0.596661783637413	0.000701904296875	\\
0.596706174812447	0.000335693359375	\\
0.596750565987482	0.00030517578125	\\
0.596794957162516	0.00030517578125	\\
0.59683934833755	0.00048828125	\\
0.596883739512585	0.000457763671875	\\
0.596928130687619	0.000244140625	\\
0.596972521862654	0.00030517578125	\\
0.597016913037688	0	\\
0.597061304212723	0.00054931640625	\\
0.597105695387757	0.000885009765625	\\
0.597150086562791	0.000396728515625	\\
0.597194477737826	0.000274658203125	\\
0.59723886891286	0.000244140625	\\
0.597283260087895	-9.1552734375e-05	\\
0.597327651262929	0	\\
0.597372042437963	9.1552734375e-05	\\
0.597416433612998	0.000640869140625	\\
0.597460824788032	0.00054931640625	\\
0.597505215963066	0.00067138671875	\\
0.597549607138101	0.000579833984375	\\
0.597593998313135	9.1552734375e-05	\\
0.59763838948817	9.1552734375e-05	\\
0.597682780663204	0.00079345703125	\\
0.597727171838239	0.000823974609375	\\
0.597771563013273	0.000396728515625	\\
0.597815954188307	0.00042724609375	\\
0.597860345363342	0.000335693359375	\\
0.597904736538376	6.103515625e-05	\\
0.597949127713411	0.000152587890625	\\
0.597993518888445	0.000823974609375	\\
0.598037910063479	0.0008544921875	\\
0.598082301238514	0.000946044921875	\\
0.598126692413548	0.0010986328125	\\
0.598171083588583	0.001251220703125	\\
0.598215474763617	0.00146484375	\\
0.598259865938651	0.001251220703125	\\
0.598304257113686	0.00128173828125	\\
0.59834864828872	0.001220703125	\\
0.598393039463755	0.00140380859375	\\
0.598437430638789	0.001495361328125	\\
0.598481821813823	0.001495361328125	\\
0.598526212988858	0.00189208984375	\\
0.598570604163892	0.0020751953125	\\
0.598614995338927	0.0020751953125	\\
0.598659386513961	0.002227783203125	\\
0.598703777688995	0.0020751953125	\\
0.59874816886403	0.00201416015625	\\
0.598792560039064	0.002166748046875	\\
0.598836951214099	0.002197265625	\\
0.598881342389133	0.002105712890625	\\
0.598925733564167	0.00238037109375	\\
0.598970124739202	0.00213623046875	\\
0.599014515914236	0.001556396484375	\\
0.599058907089271	0.001434326171875	\\
0.599103298264305	0.00164794921875	\\
0.599147689439339	0.00201416015625	\\
0.599192080614374	0.001861572265625	\\
0.599236471789408	0.00146484375	\\
0.599280862964443	0.00146484375	\\
0.599325254139477	0.00152587890625	\\
0.599369645314511	0.001617431640625	\\
0.599414036489546	0.001678466796875	\\
0.59945842766458	0.001708984375	\\
0.599502818839615	0.001434326171875	\\
0.599547210014649	0.001861572265625	\\
0.599591601189683	0.0020751953125	\\
0.599635992364718	0.002288818359375	\\
0.599680383539752	0.0020751953125	\\
0.599724774714787	0.001739501953125	\\
0.599769165889821	0.0018310546875	\\
0.599813557064856	0.001708984375	\\
0.59985794823989	0.001312255859375	\\
0.599902339414924	0.001068115234375	\\
0.599946730589959	0.00103759765625	\\
0.599991121764993	0.001068115234375	\\
0.600035512940028	0.001129150390625	\\
0.600079904115062	0.001312255859375	\\
0.600124295290096	0.00103759765625	\\
0.600168686465131	0.0010986328125	\\
0.600213077640165	0.00103759765625	\\
0.6002574688152	0.000946044921875	\\
0.600301859990234	0.00128173828125	\\
0.600346251165268	0.001068115234375	\\
0.600390642340303	0.00079345703125	\\
0.600435033515337	0.00091552734375	\\
0.600479424690372	0.000396728515625	\\
0.600523815865406	0.000152587890625	\\
0.60056820704044	0.000274658203125	\\
0.600612598215475	0.00018310546875	\\
0.600656989390509	0.00018310546875	\\
0.600701380565544	9.1552734375e-05	\\
0.600745771740578	-0.00030517578125	\\
0.600790162915612	-0.000152587890625	\\
0.600834554090647	-6.103515625e-05	\\
0.600878945265681	-0.00030517578125	\\
0.600923336440716	-0.00030517578125	\\
0.60096772761575	-9.1552734375e-05	\\
0.601012118790784	-0.000152587890625	\\
0.601056509965819	-9.1552734375e-05	\\
0.601100901140853	0.0001220703125	\\
0.601145292315888	0.000274658203125	\\
0.601189683490922	-3.0517578125e-05	\\
0.601234074665956	-9.1552734375e-05	\\
0.601278465840991	0.00018310546875	\\
0.601322857016025	0.000274658203125	\\
0.60136724819106	0.00018310546875	\\
0.601411639366094	0.00030517578125	\\
0.601456030541128	0.000732421875	\\
0.601500421716163	0.000579833984375	\\
0.601544812891197	0.0006103515625	\\
0.601589204066232	6.103515625e-05	\\
0.601633595241266	0.00030517578125	\\
0.6016779864163	0.0009765625	\\
0.601722377591335	0.000885009765625	\\
0.601766768766369	0.001312255859375	\\
0.601811159941404	0.00128173828125	\\
0.601855551116438	0.00103759765625	\\
0.601899942291472	0.001190185546875	\\
0.601944333466507	0.001312255859375	\\
0.601988724641541	0.00152587890625	\\
0.602033115816576	0.001708984375	\\
0.60207750699161	0.001953125	\\
0.602121898166645	0.001617431640625	\\
0.602166289341679	0.001708984375	\\
0.602210680516713	0.002105712890625	\\
0.602255071691748	0.001617431640625	\\
0.602299462866782	0.00152587890625	\\
0.602343854041816	0.001220703125	\\
0.602388245216851	0.00115966796875	\\
0.602432636391885	0.00103759765625	\\
0.60247702756692	0.0009765625	\\
0.602521418741954	0.001251220703125	\\
0.602565809916988	0.001007080078125	\\
0.602610201092023	0.001190185546875	\\
0.602654592267057	0.00146484375	\\
0.602698983442092	0.001190185546875	\\
0.602743374617126	0.001312255859375	\\
0.602787765792161	0.001190185546875	\\
0.602832156967195	0.00091552734375	\\
0.602876548142229	0.000885009765625	\\
0.602920939317264	0.00067138671875	\\
0.602965330492298	0.000640869140625	\\
0.603009721667333	0.0003662109375	\\
0.603054112842367	0.000244140625	\\
0.603098504017401	3.0517578125e-05	\\
0.603142895192436	0.0001220703125	\\
0.60318728636747	0.000640869140625	\\
0.603231677542505	0.0006103515625	\\
0.603276068717539	0.00054931640625	\\
0.603320459892573	0.00067138671875	\\
0.603364851067608	0.000946044921875	\\
0.603409242242642	0.00091552734375	\\
0.603453633417677	0.00091552734375	\\
0.603498024592711	0.000946044921875	\\
0.603542415767745	0.001068115234375	\\
0.60358680694278	0.00103759765625	\\
0.603631198117814	0.001129150390625	\\
0.603675589292849	0.001617431640625	\\
0.603719980467883	0.001983642578125	\\
0.603764371642917	0.00201416015625	\\
0.603808762817952	0.00213623046875	\\
0.603853153992986	0.00189208984375	\\
0.603897545168021	0.002166748046875	\\
0.603941936343055	0.002044677734375	\\
0.603986327518089	0.0013427734375	\\
0.604030718693124	0.00177001953125	\\
0.604075109868158	0.00164794921875	\\
0.604119501043193	0.0018310546875	\\
0.604163892218227	0.002532958984375	\\
0.604208283393261	0.002166748046875	\\
0.604252674568296	0.001800537109375	\\
0.60429706574333	0.0020751953125	\\
0.604341456918365	0.002227783203125	\\
0.604385848093399	0.00238037109375	\\
0.604430239268433	0.0023193359375	\\
0.604474630443468	0.00201416015625	\\
0.604519021618502	0.001617431640625	\\
0.604563412793537	0.001739501953125	\\
0.604607803968571	0.0018310546875	\\
0.604652195143605	0.0015869140625	\\
0.60469658631864	0.001434326171875	\\
0.604740977493674	0.00128173828125	\\
0.604785368668709	0.001251220703125	\\
0.604829759843743	0.00177001953125	\\
0.604874151018778	0.002410888671875	\\
0.604918542193812	0.002166748046875	\\
0.604962933368846	0.0020751953125	\\
0.605007324543881	0.002410888671875	\\
0.605051715718915	0.00201416015625	\\
0.605096106893949	0.001861572265625	\\
0.605140498068984	0.0020751953125	\\
0.605184889244018	0.001983642578125	\\
0.605229280419053	0.002044677734375	\\
0.605273671594087	0.0020751953125	\\
0.605318062769121	0.001678466796875	\\
0.605362453944156	0.002044677734375	\\
0.60540684511919	0.00250244140625	\\
0.605451236294225	0.00189208984375	\\
0.605495627469259	0.001861572265625	\\
0.605540018644294	0.00238037109375	\\
0.605584409819328	0.002288818359375	\\
0.605628800994362	0.002197265625	\\
0.605673192169397	0.0023193359375	\\
0.605717583344431	0.001983642578125	\\
0.605761974519466	0.001861572265625	\\
0.6058063656945	0.001861572265625	\\
0.605850756869534	0.0018310546875	\\
0.605895148044569	0.0018310546875	\\
0.605939539219603	0.0020751953125	\\
0.605983930394637	0.002166748046875	\\
0.606028321569672	0.002227783203125	\\
0.606072712744706	0.0018310546875	\\
0.606117103919741	0.00201416015625	\\
0.606161495094775	0.002349853515625	\\
0.60620588626981	0.00201416015625	\\
0.606250277444844	0.00213623046875	\\
0.606294668619878	0.002105712890625	\\
0.606339059794913	0.001922607421875	\\
0.606383450969947	0.001953125	\\
0.606427842144982	0.0015869140625	\\
0.606472233320016	0.00152587890625	\\
0.60651662449505	0.00140380859375	\\
0.606561015670085	0.001129150390625	\\
0.606605406845119	0.001220703125	\\
0.606649798020154	0.001190185546875	\\
0.606694189195188	0.001251220703125	\\
0.606738580370222	0.00146484375	\\
0.606782971545257	0.001190185546875	\\
0.606827362720291	0.00079345703125	\\
0.606871753895326	0.00128173828125	\\
0.60691614507036	0.001068115234375	\\
0.606960536245394	0.00067138671875	\\
0.607004927420429	0.000823974609375	\\
0.607049318595463	0.001007080078125	\\
0.607093709770498	0.000885009765625	\\
0.607138100945532	0.00091552734375	\\
0.607182492120566	0.000946044921875	\\
0.607226883295601	0.00091552734375	\\
0.607271274470635	0.001129150390625	\\
0.60731566564567	0.001190185546875	\\
0.607360056820704	0.00115966796875	\\
0.607404447995738	0.001129150390625	\\
0.607448839170773	0.001251220703125	\\
0.607493230345807	0.00115966796875	\\
0.607537621520842	0.001312255859375	\\
0.607582012695876	0.00115966796875	\\
0.60762640387091	0.001129150390625	\\
0.607670795045945	0.000885009765625	\\
0.607715186220979	0.0013427734375	\\
0.607759577396014	0.001495361328125	\\
0.607803968571048	0.0013427734375	\\
0.607848359746083	0.001983642578125	\\
0.607892750921117	0.001953125	\\
0.607937142096151	0.00164794921875	\\
0.607981533271186	0.001617431640625	\\
0.60802592444622	0.001617431640625	\\
0.608070315621254	0.001800537109375	\\
0.608114706796289	0.001434326171875	\\
0.608159097971323	0.0013427734375	\\
0.608203489146358	0.002105712890625	\\
0.608247880321392	0.001953125	\\
0.608292271496427	0.00128173828125	\\
0.608336662671461	0.0010986328125	\\
0.608381053846495	0.001068115234375	\\
0.60842544502153	0.001190185546875	\\
0.608469836196564	0.000823974609375	\\
0.608514227371599	0.00054931640625	\\
0.608558618546633	0.000152587890625	\\
0.608603009721667	-3.0517578125e-05	\\
0.608647400896702	0.000152587890625	\\
0.608691792071736	0.000152587890625	\\
0.608736183246771	0.0006103515625	\\
0.608780574421805	-0.0001220703125	\\
0.608824965596839	-0.000274658203125	\\
0.608869356771874	6.103515625e-05	\\
0.608913747946908	-0.00018310546875	\\
0.608958139121943	-0.000335693359375	\\
0.609002530296977	-0.00054931640625	\\
0.609046921472011	-0.000152587890625	\\
0.609091312647046	9.1552734375e-05	\\
0.60913570382208	-0.0003662109375	\\
0.609180094997115	-0.000579833984375	\\
0.609224486172149	-0.000244140625	\\
0.609268877347183	-0.0003662109375	\\
0.609313268522218	-0.000732421875	\\
0.609357659697252	-0.000579833984375	\\
0.609402050872287	-0.00042724609375	\\
0.609446442047321	-0.000457763671875	\\
0.609490833222355	-0.00079345703125	\\
0.60953522439739	-0.000701904296875	\\
0.609579615572424	-0.00054931640625	\\
0.609624006747459	-0.000823974609375	\\
0.609668397922493	-0.000518798828125	\\
0.609712789097527	-0.000518798828125	\\
0.609757180272562	-0.0006103515625	\\
0.609801571447596	0	\\
0.609845962622631	-0.00018310546875	\\
0.609890353797665	-0.0001220703125	\\
0.609934744972699	-0.000579833984375	\\
0.609979136147734	-0.000640869140625	\\
0.610023527322768	-0.000640869140625	\\
0.610067918497803	-0.00030517578125	\\
0.610112309672837	-0.000213623046875	\\
0.610156700847871	-0.00018310546875	\\
0.610201092022906	-0.000152587890625	\\
0.61024548319794	-0.00018310546875	\\
0.610289874372975	-0.00042724609375	\\
0.610334265548009	-0.000701904296875	\\
0.610378656723043	-0.0006103515625	\\
0.610423047898078	-0.0008544921875	\\
0.610467439073112	-0.001129150390625	\\
0.610511830248147	-0.001434326171875	\\
0.610556221423181	-0.00152587890625	\\
0.610600612598216	-0.001617431640625	\\
0.61064500377325	-0.001434326171875	\\
0.610689394948284	-0.001251220703125	\\
0.610733786123319	-0.001800537109375	\\
0.610778177298353	-0.001495361328125	\\
0.610822568473387	-0.001434326171875	\\
0.610866959648422	-0.001861572265625	\\
0.610911350823456	-0.001556396484375	\\
0.610955741998491	-0.001922607421875	\\
0.611000133173525	-0.001983642578125	\\
0.611044524348559	-0.00189208984375	\\
0.611088915523594	-0.001983642578125	\\
0.611133306698628	-0.00189208984375	\\
0.611177697873663	-0.001800537109375	\\
0.611222089048697	-0.00189208984375	\\
0.611266480223732	-0.001708984375	\\
0.611310871398766	-0.001800537109375	\\
0.6113552625738	-0.00140380859375	\\
0.611399653748835	-0.001617431640625	\\
0.611444044923869	-0.001739501953125	\\
0.611488436098904	-0.001556396484375	\\
0.611532827273938	-0.00152587890625	\\
0.611577218448972	-0.001495361328125	\\
0.611621609624007	-0.00128173828125	\\
0.611666000799041	-0.00128173828125	\\
0.611710391974076	-0.001708984375	\\
0.61175478314911	-0.001495361328125	\\
0.611799174324144	-0.00115966796875	\\
0.611843565499179	-0.0010986328125	\\
0.611887956674213	-0.000946044921875	\\
0.611932347849248	-0.0009765625	\\
0.611976739024282	-0.00115966796875	\\
0.612021130199316	-0.0013427734375	\\
0.612065521374351	-0.00140380859375	\\
0.612109912549385	-0.001434326171875	\\
0.61215430372442	-0.00091552734375	\\
0.612198694899454	-0.000946044921875	\\
0.612243086074488	-0.001312255859375	\\
0.612287477249523	-0.00128173828125	\\
0.612331868424557	-0.00103759765625	\\
0.612376259599592	-0.0009765625	\\
0.612420650774626	-0.00091552734375	\\
0.61246504194966	-0.0013427734375	\\
0.612509433124695	-0.001190185546875	\\
0.612553824299729	-0.00115966796875	\\
0.612598215474764	-0.001373291015625	\\
0.612642606649798	-0.001495361328125	\\
0.612686997824832	-0.0020751953125	\\
0.612731388999867	-0.00152587890625	\\
0.612775780174901	-0.0013427734375	\\
0.612820171349936	-0.00140380859375	\\
0.61286456252497	-0.00091552734375	\\
0.612908953700004	-0.00103759765625	\\
0.612953344875039	-0.00091552734375	\\
0.612997736050073	-0.0009765625	\\
0.613042127225108	-0.0009765625	\\
0.613086518400142	-0.000457763671875	\\
0.613130909575176	-0.00042724609375	\\
0.613175300750211	-0.000701904296875	\\
0.613219691925245	-0.00067138671875	\\
0.61326408310028	-0.000885009765625	\\
0.613308474275314	-0.000701904296875	\\
0.613352865450349	-0.000244140625	\\
0.613397256625383	-0.00018310546875	\\
0.613441647800417	-0.000274658203125	\\
0.613486038975452	-0.000457763671875	\\
0.613530430150486	-0.0001220703125	\\
0.61357482132552	0.000396728515625	\\
0.613619212500555	-0.000244140625	\\
0.613663603675589	-0.000579833984375	\\
0.613707994850624	-0.00018310546875	\\
0.613752386025658	-9.1552734375e-05	\\
0.613796777200692	0	\\
0.613841168375727	3.0517578125e-05	\\
0.613885559550761	-0.000396728515625	\\
0.613929950725796	9.1552734375e-05	\\
0.61397434190083	-3.0517578125e-05	\\
0.614018733075865	-0.000244140625	\\
0.614063124250899	-0.000274658203125	\\
0.614107515425933	-0.000579833984375	\\
0.614151906600968	-0.000244140625	\\
0.614196297776002	-0.00030517578125	\\
0.614240688951037	-0.000518798828125	\\
0.614285080126071	-0.000518798828125	\\
0.614329471301105	-0.00042724609375	\\
0.61437386247614	-0.000274658203125	\\
0.614418253651174	-0.000640869140625	\\
0.614462644826209	-0.000701904296875	\\
0.614507036001243	-0.00067138671875	\\
0.614551427176277	-0.000823974609375	\\
0.614595818351312	-0.000946044921875	\\
0.614640209526346	-0.000946044921875	\\
0.614684600701381	-0.000762939453125	\\
0.614728991876415	-0.000762939453125	\\
0.614773383051449	-0.001129150390625	\\
0.614817774226484	-0.00140380859375	\\
0.614862165401518	-0.001312255859375	\\
0.614906556576553	-0.00091552734375	\\
0.614950947751587	-0.00103759765625	\\
0.614995338926621	-0.00128173828125	\\
0.615039730101656	-0.00091552734375	\\
0.61508412127669	-0.0006103515625	\\
0.615128512451725	-0.000457763671875	\\
0.615172903626759	-0.000396728515625	\\
0.615217294801793	0	\\
0.615261685976828	-0.0003662109375	\\
0.615306077151862	-0.0006103515625	\\
0.615350468326897	-0.000274658203125	\\
0.615394859501931	-0.000213623046875	\\
0.615439250676965	-0.000396728515625	\\
0.615483641852	-9.1552734375e-05	\\
0.615528033027034	-0.0001220703125	\\
0.615572424202069	-0.000823974609375	\\
0.615616815377103	-0.0009765625	\\
0.615661206552137	-0.000823974609375	\\
0.615705597727172	-0.000457763671875	\\
0.615749988902206	-0.0006103515625	\\
0.615794380077241	-0.000762939453125	\\
0.615838771252275	-0.000335693359375	\\
0.615883162427309	-0.0003662109375	\\
0.615927553602344	-0.000579833984375	\\
0.615971944777378	-0.000885009765625	\\
0.616016335952413	-0.00103759765625	\\
0.616060727127447	-0.000885009765625	\\
0.616105118302481	-0.00091552734375	\\
0.616149509477516	-0.000762939453125	\\
0.61619390065255	-0.000885009765625	\\
0.616238291827585	-0.0008544921875	\\
0.616282683002619	-0.00067138671875	\\
0.616327074177654	-0.001129150390625	\\
0.616371465352688	-0.001251220703125	\\
0.616415856527722	-0.001373291015625	\\
0.616460247702757	-0.001678466796875	\\
0.616504638877791	-0.001251220703125	\\
0.616549030052825	-0.001068115234375	\\
0.61659342122786	-0.0015869140625	\\
0.616637812402894	-0.001617431640625	\\
0.616682203577929	-0.001617431640625	\\
0.616726594752963	-0.0020751953125	\\
0.616770985927998	-0.0023193359375	\\
0.616815377103032	-0.00177001953125	\\
0.616859768278066	-0.001434326171875	\\
0.616904159453101	-0.001068115234375	\\
0.616948550628135	-0.00146484375	\\
0.61699294180317	-0.001678466796875	\\
0.617037332978204	-0.001373291015625	\\
0.617081724153238	-0.00115966796875	\\
0.617126115328273	-0.001068115234375	\\
0.617170506503307	-0.00128173828125	\\
0.617214897678342	-0.0013427734375	\\
0.617259288853376	-0.00103759765625	\\
0.61730368002841	-0.00103759765625	\\
0.617348071203445	-0.001251220703125	\\
0.617392462378479	-0.001312255859375	\\
0.617436853553514	-0.0010986328125	\\
0.617481244728548	-0.001220703125	\\
0.617525635903582	-0.001190185546875	\\
0.617570027078617	-0.000762939453125	\\
0.617614418253651	-0.00091552734375	\\
0.617658809428686	-0.00115966796875	\\
0.61770320060372	-0.001190185546875	\\
0.617747591778754	-0.00115966796875	\\
0.617791982953789	-0.001251220703125	\\
0.617836374128823	-0.0013427734375	\\
0.617880765303858	-0.000885009765625	\\
0.617925156478892	-0.00054931640625	\\
0.617969547653926	-0.000457763671875	\\
0.618013938828961	-0.000335693359375	\\
0.618058330003995	-0.00048828125	\\
0.61810272117903	-0.000640869140625	\\
0.618147112354064	-0.000732421875	\\
0.618191503529098	-0.000640869140625	\\
0.618235894704133	-0.000335693359375	\\
0.618280285879167	-0.000396728515625	\\
0.618324677054202	-0.00054931640625	\\
0.618369068229236	-0.000579833984375	\\
0.61841345940427	-0.00091552734375	\\
0.618457850579305	-0.000823974609375	\\
0.618502241754339	-0.0009765625	\\
0.618546632929374	-0.000732421875	\\
0.618591024104408	-0.00030517578125	\\
0.618635415279442	-0.0003662109375	\\
0.618679806454477	-0.000885009765625	\\
0.618724197629511	-0.001068115234375	\\
0.618768588804546	-0.001129150390625	\\
0.61881297997958	-0.001312255859375	\\
0.618857371154614	-0.001220703125	\\
0.618901762329649	-0.0010986328125	\\
0.618946153504683	-0.001068115234375	\\
0.618990544679718	-0.00091552734375	\\
0.619034935854752	-0.000640869140625	\\
0.619079327029787	-0.000274658203125	\\
0.619123718204821	-0.000213623046875	\\
0.619168109379855	-0.00042724609375	\\
0.61921250055489	-0.000579833984375	\\
0.619256891729924	-0.00042724609375	\\
0.619301282904958	-0.0001220703125	\\
0.619345674079993	0.000274658203125	\\
0.619390065255027	-0.000335693359375	\\
0.619434456430062	-0.00079345703125	\\
0.619478847605096	-3.0517578125e-05	\\
0.61952323878013	0.0001220703125	\\
0.619567629955165	-0.000152587890625	\\
0.619612021130199	-0.0003662109375	\\
0.619656412305234	-0.000335693359375	\\
0.619700803480268	-0.00067138671875	\\
0.619745194655303	-0.000244140625	\\
0.619789585830337	-0.000152587890625	\\
0.619833977005371	-0.0001220703125	\\
0.619878368180406	3.0517578125e-05	\\
0.61992275935544	-0.000518798828125	\\
0.619967150530475	-0.000335693359375	\\
0.620011541705509	-0.000396728515625	\\
0.620055932880543	-0.0003662109375	\\
0.620100324055578	0	\\
0.620144715230612	-6.103515625e-05	\\
0.620189106405647	9.1552734375e-05	\\
0.620233497580681	9.1552734375e-05	\\
0.620277888755715	-0.00018310546875	\\
0.62032227993075	-9.1552734375e-05	\\
0.620366671105784	-0.000244140625	\\
0.620411062280819	-0.000640869140625	\\
0.620455453455853	3.0517578125e-05	\\
0.620499844630887	6.103515625e-05	\\
0.620544235805922	9.1552734375e-05	\\
0.620588626980956	0.000335693359375	\\
0.620633018155991	-0.000244140625	\\
0.620677409331025	-0.00030517578125	\\
0.620721800506059	-0.0001220703125	\\
0.620766191681094	9.1552734375e-05	\\
0.620810582856128	0.00018310546875	\\
0.620854974031163	-0.00018310546875	\\
0.620899365206197	-0.00018310546875	\\
0.620943756381231	-0.000274658203125	\\
0.620988147556266	-0.000762939453125	\\
0.6210325387313	-0.000823974609375	\\
0.621076929906335	-0.000274658203125	\\
0.621121321081369	-0.0001220703125	\\
0.621165712256403	-0.000518798828125	\\
0.621210103431438	-0.00054931640625	\\
0.621254494606472	-6.103515625e-05	\\
0.621298885781507	0.00042724609375	\\
0.621343276956541	0.00054931640625	\\
0.621387668131575	6.103515625e-05	\\
0.62143205930661	0	\\
0.621476450481644	0.000213623046875	\\
0.621520841656679	0.000732421875	\\
0.621565232831713	0.001251220703125	\\
0.621609624006747	0.000732421875	\\
0.621654015181782	0.000640869140625	\\
0.621698406356816	0.00091552734375	\\
0.621742797531851	0.0008544921875	\\
0.621787188706885	0.000946044921875	\\
0.62183157988192	0.00103759765625	\\
0.621875971056954	0.000518798828125	\\
0.621920362231988	0.000244140625	\\
0.621964753407023	0.00048828125	\\
0.622009144582057	0.00067138671875	\\
0.622053535757091	0.00030517578125	\\
0.622097926932126	0.000152587890625	\\
0.62214231810716	0.00048828125	\\
0.622186709282195	0.000396728515625	\\
0.622231100457229	0.000244140625	\\
0.622275491632263	0.00048828125	\\
0.622319882807298	0.00042724609375	\\
0.622364273982332	0.00018310546875	\\
0.622408665157367	6.103515625e-05	\\
0.622453056332401	0.000396728515625	\\
0.622497447507436	0.000457763671875	\\
0.62254183868247	9.1552734375e-05	\\
0.622586229857504	-3.0517578125e-05	\\
0.622630621032539	0.00018310546875	\\
0.622675012207573	0.0001220703125	\\
0.622719403382608	0.000152587890625	\\
0.622763794557642	0.000244140625	\\
0.622808185732676	0.0001220703125	\\
0.622852576907711	0.00042724609375	\\
0.622896968082745	0.000518798828125	\\
0.62294135925778	0.000244140625	\\
0.622985750432814	6.103515625e-05	\\
0.623030141607848	-9.1552734375e-05	\\
0.623074532782883	0.00054931640625	\\
0.623118923957917	0.00091552734375	\\
0.623163315132952	0.0006103515625	\\
0.623207706307986	0.000518798828125	\\
0.62325209748302	0.000396728515625	\\
0.623296488658055	0.001007080078125	\\
0.623340879833089	0.001556396484375	\\
0.623385271008124	0.001495361328125	\\
0.623429662183158	0.001373291015625	\\
0.623474053358192	0.001068115234375	\\
0.623518444533227	0.001129150390625	\\
0.623562835708261	0.0009765625	\\
0.623607226883296	0.000946044921875	\\
0.62365161805833	0.00146484375	\\
0.623696009233364	0.001617431640625	\\
0.623740400408399	0.001251220703125	\\
0.623784791583433	0.0010986328125	\\
0.623829182758468	0.0010986328125	\\
0.623873573933502	0.00103759765625	\\
0.623917965108536	0.0010986328125	\\
0.623962356283571	0.0008544921875	\\
0.624006747458605	0.0008544921875	\\
0.62405113863364	0.000701904296875	\\
0.624095529808674	0.0006103515625	\\
0.624139920983708	0.000152587890625	\\
0.624184312158743	-0.000274658203125	\\
0.624228703333777	0.000213623046875	\\
0.624273094508812	0.000457763671875	\\
0.624317485683846	0.00030517578125	\\
0.62436187685888	6.103515625e-05	\\
0.624406268033915	-0.00018310546875	\\
0.624450659208949	-0.00042724609375	\\
0.624495050383984	0.000152587890625	\\
0.624539441559018	9.1552734375e-05	\\
0.624583832734052	-9.1552734375e-05	\\
0.624628223909087	0.00018310546875	\\
0.624672615084121	0.000396728515625	\\
0.624717006259156	0.0003662109375	\\
0.62476139743419	-0.0001220703125	\\
0.624805788609225	3.0517578125e-05	\\
0.624850179784259	0.000335693359375	\\
0.624894570959293	-9.1552734375e-05	\\
0.624938962134328	-0.000152587890625	\\
0.624983353309362	0	\\
0.625027744484396	-0.00054931640625	\\
0.625072135659431	-0.00042724609375	\\
0.625116526834465	-9.1552734375e-05	\\
0.6251609180095	-0.0003662109375	\\
0.625205309184534	-0.000244140625	\\
0.625249700359569	0	\\
0.625294091534603	9.1552734375e-05	\\
0.625338482709637	0.000244140625	\\
0.625382873884672	0.000213623046875	\\
0.625427265059706	0.00018310546875	\\
0.625471656234741	0.000335693359375	\\
0.625516047409775	3.0517578125e-05	\\
0.625560438584809	-0.00018310546875	\\
0.625604829759844	-0.000152587890625	\\
0.625649220934878	-0.000152587890625	\\
0.625693612109913	-0.000244140625	\\
0.625738003284947	-0.00042724609375	\\
0.625782394459981	-0.000457763671875	\\
0.625826785635016	-0.000274658203125	\\
0.62587117681005	-6.103515625e-05	\\
0.625915567985085	-0.00048828125	\\
0.625959959160119	-0.000732421875	\\
0.626004350335153	-0.000640869140625	\\
0.626048741510188	-0.000762939453125	\\
0.626093132685222	-0.000823974609375	\\
0.626137523860257	-0.000885009765625	\\
0.626181915035291	-0.001190185546875	\\
0.626226306210325	-0.000701904296875	\\
0.62627069738536	-0.000701904296875	\\
0.626315088560394	-0.000518798828125	\\
0.626359479735429	-0.000244140625	\\
0.626403870910463	-0.00042724609375	\\
0.626448262085497	-0.000885009765625	\\
0.626492653260532	-0.00091552734375	\\
0.626537044435566	-0.000640869140625	\\
0.626581435610601	-0.001129150390625	\\
0.626625826785635	-0.001312255859375	\\
0.626670217960669	-0.0010986328125	\\
0.626714609135704	-0.00079345703125	\\
0.626759000310738	-0.000518798828125	\\
0.626803391485773	-0.000640869140625	\\
0.626847782660807	-0.000457763671875	\\
0.626892173835841	-0.000579833984375	\\
0.626936565010876	-0.000732421875	\\
0.62698095618591	-0.00054931640625	\\
0.627025347360945	-0.00054931640625	\\
0.627069738535979	-0.000457763671875	\\
0.627114129711013	-0.000244140625	\\
0.627158520886048	-0.000457763671875	\\
0.627202912061082	-0.00030517578125	\\
0.627247303236117	-0.00018310546875	\\
0.627291694411151	-0.00042724609375	\\
0.627336085586185	0.000457763671875	\\
0.62738047676122	9.1552734375e-05	\\
0.627424867936254	-9.1552734375e-05	\\
0.627469259111289	0.000244140625	\\
0.627513650286323	3.0517578125e-05	\\
0.627558041461358	0.000152587890625	\\
0.627602432636392	0.0001220703125	\\
0.627646823811426	0.00030517578125	\\
0.627691214986461	0.000946044921875	\\
0.627735606161495	0.001251220703125	\\
0.627779997336529	0.001312255859375	\\
0.627824388511564	0.001220703125	\\
0.627868779686598	0.00103759765625	\\
0.627913170861633	0.000946044921875	\\
0.627957562036667	0.001129150390625	\\
0.628001953211701	0.001220703125	\\
0.628046344386736	0.0010986328125	\\
0.62809073556177	0.000885009765625	\\
0.628135126736805	0.000579833984375	\\
0.628179517911839	0.000732421875	\\
0.628223909086874	0.0003662109375	\\
0.628268300261908	0.00054931640625	\\
0.628312691436942	0.00079345703125	\\
0.628357082611977	0.000885009765625	\\
0.628401473787011	0.0013427734375	\\
0.628445864962046	0.001220703125	\\
0.62849025613708	0.001129150390625	\\
0.628534647312114	0.000885009765625	\\
0.628579038487149	0.000762939453125	\\
0.628623429662183	0.0008544921875	\\
0.628667820837218	0.0010986328125	\\
0.628712212012252	0.00115966796875	\\
0.628756603187286	0.001220703125	\\
0.628800994362321	0.001007080078125	\\
0.628845385537355	0.001129150390625	\\
0.62888977671239	0.001220703125	\\
0.628934167887424	0.000823974609375	\\
0.628978559062458	0.000701904296875	\\
0.629022950237493	0.001007080078125	\\
0.629067341412527	0.00115966796875	\\
0.629111732587562	0.001373291015625	\\
0.629156123762596	0.001739501953125	\\
0.62920051493763	0.001861572265625	\\
0.629244906112665	0.001800537109375	\\
0.629289297287699	0.00146484375	\\
0.629333688462734	0.0015869140625	\\
0.629378079637768	0.002044677734375	\\
0.629422470812802	0.002105712890625	\\
0.629466861987837	0.001800537109375	\\
0.629511253162871	0.00177001953125	\\
0.629555644337906	0.001708984375	\\
0.62960003551294	0.001708984375	\\
0.629644426687974	0.00152587890625	\\
0.629688817863009	0.001708984375	\\
0.629733209038043	0.001708984375	\\
0.629777600213078	0.00177001953125	\\
0.629821991388112	0.001983642578125	\\
0.629866382563146	0.002410888671875	\\
0.629910773738181	0.00238037109375	\\
0.629955164913215	0.001678466796875	\\
0.62999955608825	0.001678466796875	\\
0.630043947263284	0.0018310546875	\\
0.630088338438318	0.0015869140625	\\
0.630132729613353	0.00152587890625	\\
0.630177120788387	0.001434326171875	\\
0.630221511963422	0.0015869140625	\\
0.630265903138456	0.0010986328125	\\
0.630310294313491	0.0010986328125	\\
0.630354685488525	0.001007080078125	\\
0.630399076663559	0.000823974609375	\\
0.630443467838594	0.001129150390625	\\
0.630487859013628	0.001068115234375	\\
0.630532250188663	0.0013427734375	\\
0.630576641363697	0.00115966796875	\\
0.630621032538731	0.000885009765625	\\
0.630665423713766	0.0008544921875	\\
0.6307098148888	0.000732421875	\\
0.630754206063834	0.000579833984375	\\
0.630798597238869	0.00030517578125	\\
0.630842988413903	0.000396728515625	\\
0.630887379588938	0.0008544921875	\\
0.630931770763972	0.00115966796875	\\
0.630976161939007	0.000885009765625	\\
0.631020553114041	0.001251220703125	\\
0.631064944289075	0.00115966796875	\\
0.63110933546411	0.00079345703125	\\
0.631153726639144	0.000396728515625	\\
0.631198117814179	0.000640869140625	\\
0.631242508989213	0.00054931640625	\\
0.631286900164247	0.000457763671875	\\
0.631331291339282	0.0006103515625	\\
0.631375682514316	0.001068115234375	\\
0.631420073689351	0.001251220703125	\\
0.631464464864385	0.00115966796875	\\
0.631508856039419	0.001251220703125	\\
0.631553247214454	0.001373291015625	\\
0.631597638389488	0.0013427734375	\\
0.631642029564523	0.001251220703125	\\
0.631686420739557	0.00115966796875	\\
0.631730811914591	0.001373291015625	\\
0.631775203089626	0.0010986328125	\\
0.63181959426466	0.000946044921875	\\
0.631863985439695	0.0008544921875	\\
0.631908376614729	0.00030517578125	\\
0.631952767789763	0.000335693359375	\\
0.631997158964798	3.0517578125e-05	\\
0.632041550139832	0.00030517578125	\\
0.632085941314867	0.0006103515625	\\
0.632130332489901	0.000396728515625	\\
0.632174723664935	6.103515625e-05	\\
0.63221911483997	-0.000213623046875	\\
0.632263506015004	-0.0003662109375	\\
0.632307897190039	-0.0003662109375	\\
0.632352288365073	-0.000274658203125	\\
0.632396679540107	-0.000335693359375	\\
0.632441070715142	-0.000335693359375	\\
0.632485461890176	-0.00030517578125	\\
0.632529853065211	9.1552734375e-05	\\
0.632574244240245	3.0517578125e-05	\\
0.632618635415279	6.103515625e-05	\\
0.632663026590314	0.000244140625	\\
0.632707417765348	0.00018310546875	\\
0.632751808940383	0.000701904296875	\\
0.632796200115417	0.000732421875	\\
0.632840591290451	0.000396728515625	\\
0.632884982465486	0.000518798828125	\\
0.63292937364052	0.000732421875	\\
0.632973764815555	0.0006103515625	\\
0.633018155990589	0.0006103515625	\\
0.633062547165623	0.000823974609375	\\
0.633106938340658	0.001007080078125	\\
0.633151329515692	0.0008544921875	\\
0.633195720690727	0.000885009765625	\\
0.633240111865761	0.001007080078125	\\
0.633284503040796	0.001251220703125	\\
0.63332889421583	0.001251220703125	\\
0.633373285390864	0.001007080078125	\\
0.633417676565899	0.001434326171875	\\
0.633462067740933	0.00146484375	\\
0.633506458915967	0.00140380859375	\\
0.633550850091002	0.00146484375	\\
0.633595241266036	0.00164794921875	\\
0.633639632441071	0.00164794921875	\\
0.633684023616105	0.001251220703125	\\
0.63372841479114	0.001373291015625	\\
0.633772805966174	0.00103759765625	\\
0.633817197141208	0.000762939453125	\\
0.633861588316243	0.000518798828125	\\
0.633905979491277	0.000244140625	\\
0.633950370666312	0.00048828125	\\
0.633994761841346	0.0008544921875	\\
0.63403915301638	0.00079345703125	\\
0.634083544191415	0.00091552734375	\\
0.634127935366449	0.000732421875	\\
0.634172326541484	0.0003662109375	\\
0.634216717716518	0.000244140625	\\
0.634261108891552	0.000823974609375	\\
0.634305500066587	0.000396728515625	\\
0.634349891241621	0.0001220703125	\\
0.634394282416656	0.00079345703125	\\
0.63443867359169	0.00018310546875	\\
0.634483064766724	0.000518798828125	\\
0.634527455941759	0.000823974609375	\\
0.634571847116793	0.000457763671875	\\
0.634616238291828	0.00091552734375	\\
0.634660629466862	0.000762939453125	\\
0.634705020641896	3.0517578125e-05	\\
0.634749411816931	0.000244140625	\\
0.634793802991965	0.0003662109375	\\
0.634838194167	-3.0517578125e-05	\\
0.634882585342034	0.0001220703125	\\
0.634926976517068	0.000396728515625	\\
0.634971367692103	0.000244140625	\\
0.635015758867137	0.0001220703125	\\
0.635060150042172	-3.0517578125e-05	\\
0.635104541217206	0	\\
0.63514893239224	-0.0001220703125	\\
0.635193323567275	0.000457763671875	\\
0.635237714742309	0.000946044921875	\\
0.635282105917344	0.001129150390625	\\
0.635326497092378	0.000885009765625	\\
0.635370888267412	0.00067138671875	\\
0.635415279442447	0.00054931640625	\\
0.635459670617481	0.0008544921875	\\
0.635504061792516	0.00054931640625	\\
0.63554845296755	0.00054931640625	\\
0.635592844142584	0.00091552734375	\\
0.635637235317619	0.000823974609375	\\
0.635681626492653	0.000701904296875	\\
0.635726017667688	0.0008544921875	\\
0.635770408842722	0.000335693359375	\\
0.635814800017756	0.000335693359375	\\
0.635859191192791	0.000701904296875	\\
0.635903582367825	0.000396728515625	\\
0.63594797354286	0.000396728515625	\\
0.635992364717894	0.000579833984375	\\
0.636036755892929	0.000518798828125	\\
0.636081147067963	0.000396728515625	\\
0.636125538242997	0.000518798828125	\\
0.636169929418032	0.000244140625	\\
0.636214320593066	0.0006103515625	\\
0.6362587117681	0.00054931640625	\\
0.636303102943135	6.103515625e-05	\\
0.636347494118169	0.000701904296875	\\
0.636391885293204	0.000885009765625	\\
0.636436276468238	0.000762939453125	\\
0.636480667643272	0.0009765625	\\
0.636525058818307	0.000823974609375	\\
0.636569449993341	0.000579833984375	\\
0.636613841168376	0.00042724609375	\\
0.63665823234341	0.000244140625	\\
0.636702623518445	0.0003662109375	\\
0.636747014693479	0.00054931640625	\\
0.636791405868513	0.000335693359375	\\
0.636835797043548	0.0006103515625	\\
0.636880188218582	0.00079345703125	\\
0.636924579393617	0.0009765625	\\
0.636968970568651	0.00079345703125	\\
0.637013361743685	0.000701904296875	\\
0.63705775291872	0.000640869140625	\\
0.637102144093754	0.00103759765625	\\
0.637146535268789	0.001708984375	\\
0.637190926443823	0.001434326171875	\\
0.637235317618857	0.00177001953125	\\
0.637279708793892	0.00164794921875	\\
0.637324099968926	0.001495361328125	\\
0.637368491143961	0.001861572265625	\\
0.637412882318995	0.001922607421875	\\
0.637457273494029	0.0020751953125	\\
0.637501664669064	0.00213623046875	\\
0.637546055844098	0.00238037109375	\\
0.637590447019133	0.002471923828125	\\
0.637634838194167	0.0020751953125	\\
0.637679229369201	0.002410888671875	\\
0.637723620544236	0.002227783203125	\\
0.63776801171927	0.002105712890625	\\
0.637812402894305	0.002593994140625	\\
0.637856794069339	0.002288818359375	\\
0.637901185244373	0.001708984375	\\
0.637945576419408	0.00201416015625	\\
0.637989967594442	0.002166748046875	\\
0.638034358769477	0.002105712890625	\\
0.638078749944511	0.002105712890625	\\
0.638123141119545	0.0020751953125	\\
0.63816753229458	0.002105712890625	\\
0.638211923469614	0.001983642578125	\\
0.638256314644649	0.001678466796875	\\
0.638300705819683	0.00146484375	\\
0.638345096994717	0.00140380859375	\\
0.638389488169752	0.001190185546875	\\
0.638433879344786	0.00103759765625	\\
0.638478270519821	0.001495361328125	\\
0.638522661694855	0.0018310546875	\\
0.638567052869889	0.001922607421875	\\
0.638611444044924	0.00152587890625	\\
0.638655835219958	0.001190185546875	\\
0.638700226394993	0.00152587890625	\\
0.638744617570027	0.002166748046875	\\
0.638789008745062	0.001708984375	\\
0.638833399920096	0.001800537109375	\\
0.63887779109513	0.001861572265625	\\
0.638922182270165	0.00146484375	\\
0.638966573445199	0.0020751953125	\\
0.639010964620234	0.002105712890625	\\
0.639055355795268	0.002044677734375	\\
0.639099746970302	0.002044677734375	\\
0.639144138145337	0.0020751953125	\\
0.639188529320371	0.00225830078125	\\
0.639232920495405	0.002288818359375	\\
0.63927731167044	0.00225830078125	\\
0.639321702845474	0.002288818359375	\\
0.639366094020509	0.00238037109375	\\
0.639410485195543	0.00238037109375	\\
0.639454876370578	0.002105712890625	\\
0.639499267545612	0.002044677734375	\\
0.639543658720646	0.00201416015625	\\
0.639588049895681	0.0020751953125	\\
0.639632441070715	0.00189208984375	\\
0.63967683224575	0.001739501953125	\\
0.639721223420784	0.00189208984375	\\
0.639765614595818	0.00164794921875	\\
0.639810005770853	0.00140380859375	\\
0.639854396945887	0.00164794921875	\\
0.639898788120922	0.00152587890625	\\
0.639943179295956	0.001373291015625	\\
0.63998757047099	0.00128173828125	\\
0.640031961646025	0.0015869140625	\\
0.640076352821059	0.001373291015625	\\
0.640120743996094	0.001007080078125	\\
0.640165135171128	0.00091552734375	\\
0.640209526346162	0.00067138671875	\\
0.640253917521197	0.00115966796875	\\
0.640298308696231	0.001220703125	\\
0.640342699871266	0.000732421875	\\
0.6403870910463	0.0009765625	\\
0.640431482221334	0.00091552734375	\\
0.640475873396369	0.001251220703125	\\
0.640520264571403	0.00128173828125	\\
0.640564655746438	0.001251220703125	\\
0.640609046921472	0.001373291015625	\\
0.640653438096506	0.001373291015625	\\
0.640697829271541	0.00115966796875	\\
0.640742220446575	0.000946044921875	\\
0.64078661162161	0.0008544921875	\\
0.640831002796644	0.00079345703125	\\
0.640875393971678	0.00103759765625	\\
0.640919785146713	0.001190185546875	\\
0.640964176321747	0.0013427734375	\\
0.641008567496782	0.001495361328125	\\
0.641052958671816	0.001739501953125	\\
0.64109734984685	0.002044677734375	\\
0.641141741021885	0.001556396484375	\\
0.641186132196919	0.0013427734375	\\
0.641230523371954	0.001617431640625	\\
0.641274914546988	0.00079345703125	\\
0.641319305722022	0.0006103515625	\\
0.641363696897057	0.00103759765625	\\
0.641408088072091	0.000732421875	\\
0.641452479247126	0.000518798828125	\\
0.64149687042216	0.00067138671875	\\
0.641541261597194	0.000732421875	\\
0.641585652772229	0.0010986328125	\\
0.641630043947263	0.00091552734375	\\
0.641674435122298	0.00067138671875	\\
0.641718826297332	0.001007080078125	\\
0.641763217472367	0.00054931640625	\\
0.641807608647401	0.00048828125	\\
0.641851999822435	0.0008544921875	\\
0.64189639099747	0.000732421875	\\
0.641940782172504	0.000640869140625	\\
0.641985173347538	0.000457763671875	\\
0.642029564522573	0.000640869140625	\\
0.642073955697607	6.103515625e-05	\\
0.642118346872642	-0.000152587890625	\\
0.642162738047676	0.0003662109375	\\
0.642207129222711	0.0003662109375	\\
0.642251520397745	0.000152587890625	\\
0.642295911572779	0.0001220703125	\\
0.642340302747814	-0.0001220703125	\\
0.642384693922848	-0.000152587890625	\\
0.642429085097883	-0.000274658203125	\\
0.642473476272917	-0.000152587890625	\\
0.642517867447951	0.00018310546875	\\
0.642562258622986	0.00042724609375	\\
0.64260664979802	0.00067138671875	\\
0.642651040973055	0.000732421875	\\
0.642695432148089	0.000823974609375	\\
0.642739823323123	0.0006103515625	\\
0.642784214498158	0.000579833984375	\\
0.642828605673192	0.001251220703125	\\
0.642872996848227	0.001373291015625	\\
0.642917388023261	0.001190185546875	\\
0.642961779198295	0.001068115234375	\\
0.64300617037333	0.001373291015625	\\
0.643050561548364	0.001556396484375	\\
0.643094952723399	0.00189208984375	\\
0.643139343898433	0.001953125	\\
0.643183735073467	0.001953125	\\
0.643228126248502	0.002471923828125	\\
0.643272517423536	0.002532958984375	\\
0.643316908598571	0.002471923828125	\\
0.643361299773605	0.002044677734375	\\
0.643405690948639	0.00177001953125	\\
0.643450082123674	0.00189208984375	\\
0.643494473298708	0.0015869140625	\\
0.643538864473743	0.001312255859375	\\
0.643583255648777	0.00140380859375	\\
0.643627646823811	0.0015869140625	\\
0.643672037998846	0.001739501953125	\\
0.64371642917388	0.001708984375	\\
0.643760820348915	0.001678466796875	\\
0.643805211523949	0.001220703125	\\
0.643849602698983	0.001129150390625	\\
0.643893993874018	0.001312255859375	\\
0.643938385049052	0.00146484375	\\
0.643982776224087	0.001312255859375	\\
0.644027167399121	0.00115966796875	\\
0.644071558574155	0.000946044921875	\\
0.64411594974919	0.000579833984375	\\
0.644160340924224	0.000732421875	\\
0.644204732099259	0.000885009765625	\\
0.644249123274293	0.0006103515625	\\
0.644293514449327	0.000762939453125	\\
0.644337905624362	0.000579833984375	\\
0.644382296799396	0.00067138671875	\\
0.644426687974431	0.00115966796875	\\
0.644471079149465	0.000701904296875	\\
0.6445154703245	0.000732421875	\\
0.644559861499534	0.000518798828125	\\
0.644604252674568	0.000335693359375	\\
0.644648643849603	0.000518798828125	\\
0.644693035024637	0.00048828125	\\
0.644737426199672	0.00140380859375	\\
0.644781817374706	0.001434326171875	\\
0.64482620854974	0.000885009765625	\\
0.644870599724775	0.0006103515625	\\
0.644914990899809	0.0008544921875	\\
0.644959382074843	0.001007080078125	\\
0.645003773249878	0.001190185546875	\\
0.645048164424912	0.00152587890625	\\
0.645092555599947	0.001556396484375	\\
0.645136946774981	0.001800537109375	\\
0.645181337950016	0.0018310546875	\\
0.64522572912505	0.00164794921875	\\
0.645270120300084	0.001312255859375	\\
0.645314511475119	0.001434326171875	\\
0.645358902650153	0.001708984375	\\
0.645403293825188	0.00189208984375	\\
0.645447685000222	0.002166748046875	\\
0.645492076175256	0.001983642578125	\\
0.645536467350291	0.00238037109375	\\
0.645580858525325	0.0023193359375	\\
0.64562524970036	0.002227783203125	\\
0.645669640875394	0.0023193359375	\\
0.645714032050428	0.002105712890625	\\
0.645758423225463	0.00262451171875	\\
0.645802814400497	0.002655029296875	\\
0.645847205575532	0.0020751953125	\\
0.645891596750566	0.002349853515625	\\
0.6459359879256	0.002349853515625	\\
0.645980379100635	0.002197265625	\\
0.646024770275669	0.00189208984375	\\
0.646069161450704	0.00213623046875	\\
0.646113552625738	0.002532958984375	\\
0.646157943800772	0.002471923828125	\\
0.646202334975807	0.00238037109375	\\
0.646246726150841	0.0020751953125	\\
0.646291117325876	0.002105712890625	\\
0.64633550850091	0.002349853515625	\\
0.646379899675944	0.002532958984375	\\
0.646424290850979	0.00274658203125	\\
0.646468682026013	0.002716064453125	\\
0.646513073201048	0.002777099609375	\\
0.646557464376082	0.002685546875	\\
0.646601855551116	0.0025634765625	\\
0.646646246726151	0.0025634765625	\\
0.646690637901185	0.002685546875	\\
0.64673502907622	0.002838134765625	\\
0.646779420251254	0.002685546875	\\
0.646823811426288	0.00311279296875	\\
0.646868202601323	0.0029296875	\\
0.646912593776357	0.002960205078125	\\
0.646956984951392	0.0029296875	\\
0.647001376126426	0.002777099609375	\\
0.64704576730146	0.002685546875	\\
0.647090158476495	0.00286865234375	\\
0.647134549651529	0.003265380859375	\\
0.647178940826564	0.002960205078125	\\
0.647223332001598	0.0028076171875	\\
0.647267723176633	0.002777099609375	\\
0.647312114351667	0.002716064453125	\\
0.647356505526701	0.003021240234375	\\
0.647400896701736	0.0029296875	\\
0.64744528787677	0.00286865234375	\\
0.647489679051805	0.003387451171875	\\
0.647534070226839	0.002777099609375	\\
0.647578461401873	0.00238037109375	\\
0.647622852576908	0.00286865234375	\\
0.647667243751942	0.002716064453125	\\
0.647711634926976	0.00213623046875	\\
0.647756026102011	0.002197265625	\\
0.647800417277045	0.0025634765625	\\
0.64784480845208	0.00238037109375	\\
0.647889199627114	0.002288818359375	\\
0.647933590802149	0.002288818359375	\\
0.647977981977183	0.002105712890625	\\
0.648022373152217	0.001983642578125	\\
0.648066764327252	0.002410888671875	\\
0.648111155502286	0.002777099609375	\\
0.648155546677321	0.00238037109375	\\
0.648199937852355	0.002532958984375	\\
0.648244329027389	0.00250244140625	\\
0.648288720202424	0.002471923828125	\\
0.648333111377458	0.0025634765625	\\
0.648377502552493	0.00262451171875	\\
0.648421893727527	0.00250244140625	\\
0.648466284902561	0.002532958984375	\\
0.648510676077596	0.00286865234375	\\
0.64855506725263	0.002471923828125	\\
0.648599458427665	0.002288818359375	\\
0.648643849602699	0.00225830078125	\\
0.648688240777733	0.002349853515625	\\
0.648732631952768	0.002410888671875	\\
0.648777023127802	0.002349853515625	\\
0.648821414302837	0.003021240234375	\\
0.648865805477871	0.00238037109375	\\
0.648910196652905	0.001922607421875	\\
0.64895458782794	0.00201416015625	\\
0.648998979002974	0.00244140625	\\
0.649043370178009	0.002777099609375	\\
0.649087761353043	0.00225830078125	\\
0.649132152528077	0.002166748046875	\\
0.649176543703112	0.002410888671875	\\
0.649220934878146	0.002410888671875	\\
0.649265326053181	0.002410888671875	\\
0.649309717228215	0.002227783203125	\\
0.649354108403249	0.00177001953125	\\
0.649398499578284	0.00152587890625	\\
0.649442890753318	0.00189208984375	\\
0.649487281928353	0.001800537109375	\\
0.649531673103387	0.001922607421875	\\
0.649576064278421	0.001800537109375	\\
0.649620455453456	0.001434326171875	\\
0.64966484662849	0.001678466796875	\\
0.649709237803525	0.0015869140625	\\
0.649753628978559	0.001312255859375	\\
0.649798020153593	0.001251220703125	\\
0.649842411328628	0.001190185546875	\\
0.649886802503662	0.00103759765625	\\
0.649931193678697	0.000823974609375	\\
0.649975584853731	0.000946044921875	\\
0.650019976028765	0.00079345703125	\\
0.6500643672038	0.000823974609375	\\
0.650108758378834	0.000762939453125	\\
0.650153149553869	0.000335693359375	\\
0.650197540728903	-6.103515625e-05	\\
0.650241931903938	6.103515625e-05	\\
0.650286323078972	6.103515625e-05	\\
0.650330714254006	-6.103515625e-05	\\
0.650375105429041	-6.103515625e-05	\\
0.650419496604075	0.000396728515625	\\
0.650463887779109	0.00048828125	\\
0.650508278954144	0.000213623046875	\\
0.650552670129178	-6.103515625e-05	\\
0.650597061304213	-0.0001220703125	\\
0.650641452479247	-0.000244140625	\\
0.650685843654282	-0.00030517578125	\\
0.650730234829316	-0.000244140625	\\
0.65077462600435	-3.0517578125e-05	\\
0.650819017179385	-0.00030517578125	\\
0.650863408354419	-0.000274658203125	\\
0.650907799529454	0.0001220703125	\\
0.650952190704488	-0.0006103515625	\\
0.650996581879522	-0.000640869140625	\\
0.651040973054557	-0.0001220703125	\\
0.651085364229591	6.103515625e-05	\\
0.651129755404626	-0.000274658203125	\\
0.65117414657966	-0.00042724609375	\\
0.651218537754694	3.0517578125e-05	\\
0.651262928929729	-0.0003662109375	\\
0.651307320104763	-0.000579833984375	\\
0.651351711279798	-0.000335693359375	\\
0.651396102454832	-0.000640869140625	\\
0.651440493629866	-0.000579833984375	\\
0.651484884804901	-0.0003662109375	\\
0.651529275979935	-0.00030517578125	\\
0.65157366715497	-0.00054931640625	\\
0.651618058330004	-0.00054931640625	\\
0.651662449505038	-0.000457763671875	\\
0.651706840680073	-0.00079345703125	\\
0.651751231855107	-0.000640869140625	\\
0.651795623030142	-0.000152587890625	\\
0.651840014205176	-0.00018310546875	\\
0.65188440538021	-0.00018310546875	\\
0.651928796555245	-0.000579833984375	\\
0.651973187730279	-0.000701904296875	\\
0.652017578905314	-0.00054931640625	\\
0.652061970080348	-0.00067138671875	\\
0.652106361255382	-0.000335693359375	\\
0.652150752430417	-0.00018310546875	\\
0.652195143605451	-0.00067138671875	\\
0.652239534780486	-0.00042724609375	\\
0.65228392595552	-6.103515625e-05	\\
0.652328317130554	-0.00018310546875	\\
0.652372708305589	-0.000396728515625	\\
0.652417099480623	-0.000396728515625	\\
0.652461490655658	-0.000701904296875	\\
0.652505881830692	-0.000244140625	\\
0.652550273005726	0.0003662109375	\\
0.652594664180761	0.000396728515625	\\
0.652639055355795	0.000701904296875	\\
0.65268344653083	0.00091552734375	\\
0.652727837705864	0.0010986328125	\\
0.652772228880898	0.00140380859375	\\
0.652816620055933	0.001129150390625	\\
0.652861011230967	0.001190185546875	\\
0.652905402406002	0.0015869140625	\\
0.652949793581036	0.001373291015625	\\
0.652994184756071	0.000701904296875	\\
0.653038575931105	0.000762939453125	\\
0.653082967106139	0.001312255859375	\\
0.653127358281174	0.0013427734375	\\
0.653171749456208	0.001129150390625	\\
0.653216140631243	0.00091552734375	\\
0.653260531806277	0.00115966796875	\\
0.653304922981311	0.0013427734375	\\
0.653349314156346	0.000946044921875	\\
0.65339370533138	0.000518798828125	\\
0.653438096506414	0.00048828125	\\
0.653482487681449	0.000457763671875	\\
0.653526878856483	0.0001220703125	\\
0.653571270031518	3.0517578125e-05	\\
0.653615661206552	0.0001220703125	\\
0.653660052381587	0.00030517578125	\\
0.653704443556621	0.000396728515625	\\
0.653748834731655	0.00018310546875	\\
0.65379322590669	-0.00018310546875	\\
0.653837617081724	9.1552734375e-05	\\
0.653882008256759	-0.000152587890625	\\
0.653926399431793	0.000213623046875	\\
0.653970790606827	0.00054931640625	\\
0.654015181781862	0.0001220703125	\\
0.654059572956896	0.000640869140625	\\
0.654103964131931	0.0003662109375	\\
0.654148355306965	-3.0517578125e-05	\\
0.654192746481999	0.000244140625	\\
0.654237137657034	-0.000152587890625	\\
0.654281528832068	-0.000457763671875	\\
0.654325920007103	-0.000518798828125	\\
0.654370311182137	0.000244140625	\\
0.654414702357171	0.00067138671875	\\
0.654459093532206	0.000701904296875	\\
0.65450348470724	0.0006103515625	\\
0.654547875882275	0.000335693359375	\\
0.654592267057309	0.000579833984375	\\
0.654636658232343	0.000823974609375	\\
0.654681049407378	0.00042724609375	\\
0.654725440582412	0.000457763671875	\\
0.654769831757447	0.00048828125	\\
0.654814222932481	0.00048828125	\\
0.654858614107515	0.00103759765625	\\
0.65490300528255	0.001068115234375	\\
0.654947396457584	0.001068115234375	\\
0.654991787632619	0.0010986328125	\\
0.655036178807653	0.001251220703125	\\
0.655080569982687	0.0015869140625	\\
0.655124961157722	0.001556396484375	\\
0.655169352332756	0.001708984375	\\
0.655213743507791	0.001617431640625	\\
0.655258134682825	0.001739501953125	\\
0.655302525857859	0.001678466796875	\\
0.655346917032894	0.001678466796875	\\
0.655391308207928	0.001495361328125	\\
0.655435699382963	0.001251220703125	\\
0.655480090557997	0.00140380859375	\\
0.655524481733031	0.001373291015625	\\
0.655568872908066	0.001373291015625	\\
0.6556132640831	0.00128173828125	\\
0.655657655258135	0.001251220703125	\\
0.655702046433169	0.001373291015625	\\
0.655746437608204	0.00128173828125	\\
0.655790828783238	0.00152587890625	\\
0.655835219958272	0.0010986328125	\\
0.655879611133307	0.000732421875	\\
0.655924002308341	0.00054931640625	\\
0.655968393483376	0.0006103515625	\\
0.65601278465841	0.00048828125	\\
0.656057175833444	0.000579833984375	\\
0.656101567008479	0.000518798828125	\\
0.656145958183513	0.00042724609375	\\
0.656190349358547	0.00091552734375	\\
0.656234740533582	0.00091552734375	\\
0.656279131708616	0.000762939453125	\\
0.656323522883651	0.000762939453125	\\
0.656367914058685	0.000946044921875	\\
0.65641230523372	0.001129150390625	\\
0.656456696408754	0.00091552734375	\\
0.656501087583788	0.000640869140625	\\
0.656545478758823	0.000579833984375	\\
0.656589869933857	0.001129150390625	\\
0.656634261108892	0.000946044921875	\\
0.656678652283926	0.000946044921875	\\
0.65672304345896	0.001190185546875	\\
0.656767434633995	0.00067138671875	\\
0.656811825809029	0.000640869140625	\\
0.656856216984064	0.000335693359375	\\
0.656900608159098	0.000335693359375	\\
0.656944999334132	0.000946044921875	\\
0.656989390509167	0.00042724609375	\\
0.657033781684201	-0.00018310546875	\\
0.657078172859236	3.0517578125e-05	\\
0.65712256403427	0.00030517578125	\\
0.657166955209304	0.0003662109375	\\
0.657211346384339	0.000335693359375	\\
0.657255737559373	3.0517578125e-05	\\
0.657300128734408	3.0517578125e-05	\\
0.657344519909442	-0.000244140625	\\
0.657388911084476	0.0001220703125	\\
0.657433302259511	-0.000244140625	\\
0.657477693434545	-0.000396728515625	\\
0.65752208460958	0.000244140625	\\
0.657566475784614	0.000244140625	\\
0.657610866959648	-3.0517578125e-05	\\
0.657655258134683	3.0517578125e-05	\\
0.657699649309717	0.0003662109375	\\
0.657744040484752	-0.000213623046875	\\
0.657788431659786	-0.000640869140625	\\
0.65783282283482	-0.0001220703125	\\
0.657877214009855	-0.0001220703125	\\
0.657921605184889	-0.00030517578125	\\
0.657965996359924	-0.00054931640625	\\
0.658010387534958	-0.0006103515625	\\
0.658054778709992	-0.000457763671875	\\
0.658099169885027	-0.00042724609375	\\
0.658143561060061	-0.000518798828125	\\
0.658187952235096	-0.000762939453125	\\
0.65823234341013	-0.000762939453125	\\
0.658276734585164	-0.000396728515625	\\
0.658321125760199	-0.00018310546875	\\
0.658365516935233	-0.000518798828125	\\
0.658409908110268	-0.000274658203125	\\
0.658454299285302	-0.000213623046875	\\
0.658498690460336	3.0517578125e-05	\\
0.658543081635371	0.0001220703125	\\
0.658587472810405	0	\\
0.65863186398544	-0.000213623046875	\\
0.658676255160474	-0.000244140625	\\
0.658720646335509	-3.0517578125e-05	\\
0.658765037510543	-6.103515625e-05	\\
0.658809428685577	3.0517578125e-05	\\
0.658853819860612	0.00067138671875	\\
0.658898211035646	0.000518798828125	\\
0.65894260221068	0.000244140625	\\
0.658986993385715	0.000335693359375	\\
0.659031384560749	0.00018310546875	\\
0.659075775735784	0.0003662109375	\\
0.659120166910818	0.000701904296875	\\
0.659164558085853	0.00067138671875	\\
0.659208949260887	3.0517578125e-05	\\
0.659253340435921	6.103515625e-05	\\
0.659297731610956	-3.0517578125e-05	\\
0.65934212278599	6.103515625e-05	\\
0.659386513961025	0.0001220703125	\\
0.659430905136059	-0.000152587890625	\\
0.659475296311093	0	\\
0.659519687486128	-0.000152587890625	\\
0.659564078661162	-0.000274658203125	\\
0.659608469836197	-0.000335693359375	\\
0.659652861011231	6.103515625e-05	\\
0.659697252186265	0.000152587890625	\\
0.6597416433613	-0.00042724609375	\\
0.659786034536334	-0.000335693359375	\\
0.659830425711369	-9.1552734375e-05	\\
0.659874816886403	-0.000396728515625	\\
0.659919208061437	-0.000579833984375	\\
0.659963599236472	-0.00048828125	\\
0.660007990411506	-0.000640869140625	\\
0.660052381586541	-0.00091552734375	\\
0.660096772761575	-0.0008544921875	\\
0.660141163936609	-0.000823974609375	\\
0.660185555111644	-0.000640869140625	\\
0.660229946286678	-0.0008544921875	\\
0.660274337461713	-0.000885009765625	\\
0.660318728636747	-0.000640869140625	\\
0.660363119811781	-0.000579833984375	\\
0.660407510986816	-0.0003662109375	\\
0.66045190216185	-0.000579833984375	\\
0.660496293336885	-0.000640869140625	\\
0.660540684511919	-0.00067138671875	\\
0.660585075686953	-0.000823974609375	\\
0.660629466861988	-0.000701904296875	\\
0.660673858037022	-0.00018310546875	\\
0.660718249212057	6.103515625e-05	\\
0.660762640387091	0	\\
0.660807031562126	-0.00030517578125	\\
0.66085142273716	-0.0001220703125	\\
0.660895813912194	6.103515625e-05	\\
0.660940205087229	0.000213623046875	\\
0.660984596262263	0.000640869140625	\\
0.661028987437297	0.000762939453125	\\
0.661073378612332	0.001007080078125	\\
0.661117769787366	0.00067138671875	\\
0.661162160962401	0.0003662109375	\\
0.661206552137435	0.00018310546875	\\
0.661250943312469	0	\\
0.661295334487504	0.00018310546875	\\
0.661339725662538	-0.00030517578125	\\
0.661384116837573	-0.00042724609375	\\
0.661428508012607	-0.00042724609375	\\
0.661472899187642	-0.000152587890625	\\
0.661517290362676	-0.00054931640625	\\
0.66156168153771	-0.0008544921875	\\
0.661606072712745	-0.000885009765625	\\
0.661650463887779	-0.001129150390625	\\
0.661694855062814	-0.00067138671875	\\
0.661739246237848	-0.000701904296875	\\
0.661783637412882	-0.00079345703125	\\
0.661828028587917	-0.000762939453125	\\
0.661872419762951	-0.000946044921875	\\
0.661916810937985	-0.000701904296875	\\
0.66196120211302	-0.0003662109375	\\
0.662005593288054	-0.000396728515625	\\
0.662049984463089	-0.0008544921875	\\
0.662094375638123	-0.000732421875	\\
0.662138766813158	-0.0001220703125	\\
0.662183157988192	-0.000274658203125	\\
0.662227549163226	-0.00030517578125	\\
0.662271940338261	-6.103515625e-05	\\
0.662316331513295	0.000457763671875	\\
0.66236072268833	0.0009765625	\\
0.662405113863364	0.000823974609375	\\
0.662449505038398	0.001068115234375	\\
0.662493896213433	0.00128173828125	\\
0.662538287388467	0.00146484375	\\
0.662582678563502	0.001495361328125	\\
0.662627069738536	0.00140380859375	\\
0.66267146091357	0.001312255859375	\\
0.662715852088605	0.0015869140625	\\
0.662760243263639	0.00189208984375	\\
0.662804634438674	0.001953125	\\
0.662849025613708	0.0018310546875	\\
0.662893416788742	0.0018310546875	\\
0.662937807963777	0.001434326171875	\\
0.662982199138811	0.0013427734375	\\
0.663026590313846	0.001922607421875	\\
0.66307098148888	0.001556396484375	\\
0.663115372663914	0.0013427734375	\\
0.663159763838949	0.00128173828125	\\
0.663204155013983	0.00128173828125	\\
0.663248546189018	0.001373291015625	\\
0.663292937364052	0.001251220703125	\\
0.663337328539086	0.001068115234375	\\
0.663381719714121	0.001007080078125	\\
0.663426110889155	0.000732421875	\\
0.66347050206419	0.00103759765625	\\
0.663514893239224	0.000762939453125	\\
0.663559284414258	0.000457763671875	\\
0.663603675589293	0.000732421875	\\
0.663648066764327	0.000823974609375	\\
0.663692457939362	0.00091552734375	\\
0.663736849114396	0.000762939453125	\\
0.66378124028943	0.0006103515625	\\
0.663825631464465	-6.103515625e-05	\\
0.663870022639499	0	\\
0.663914413814534	0.000152587890625	\\
0.663958804989568	6.103515625e-05	\\
0.664003196164602	0.0001220703125	\\
0.664047587339637	-0.0001220703125	\\
0.664091978514671	0	\\
0.664136369689706	0.000274658203125	\\
0.66418076086474	0.000518798828125	\\
0.664225152039775	0.000335693359375	\\
0.664269543214809	0.0003662109375	\\
0.664313934389843	0.00048828125	\\
0.664358325564878	0.0006103515625	\\
0.664402716739912	0.0009765625	\\
0.664447107914947	0.001068115234375	\\
0.664491499089981	0.000640869140625	\\
0.664535890265015	0.00048828125	\\
0.66458028144005	0.000335693359375	\\
0.664624672615084	-9.1552734375e-05	\\
0.664669063790118	0.000274658203125	\\
0.664713454965153	6.103515625e-05	\\
0.664757846140187	0.000213623046875	\\
0.664802237315222	0.000701904296875	\\
0.664846628490256	0.00054931640625	\\
0.664891019665291	0.000457763671875	\\
0.664935410840325	0.000518798828125	\\
0.664979802015359	0.0008544921875	\\
0.665024193190394	0.00048828125	\\
0.665068584365428	0.0008544921875	\\
0.665112975540463	0.0006103515625	\\
0.665157366715497	0.00018310546875	\\
0.665201757890531	0.000640869140625	\\
0.665246149065566	0.000640869140625	\\
0.6652905402406	0.0003662109375	\\
0.665334931415635	6.103515625e-05	\\
0.665379322590669	0.000518798828125	\\
0.665423713765703	0.00054931640625	\\
0.665468104940738	0.00042724609375	\\
0.665512496115772	0.00048828125	\\
0.665556887290807	0.000701904296875	\\
0.665601278465841	0.00079345703125	\\
0.665645669640875	0.000640869140625	\\
0.66569006081591	0.0003662109375	\\
0.665734451990944	0.00054931640625	\\
0.665778843165979	0.00048828125	\\
0.665823234341013	6.103515625e-05	\\
0.665867625516047	3.0517578125e-05	\\
0.665912016691082	0.000335693359375	\\
0.665956407866116	6.103515625e-05	\\
0.666000799041151	0.000274658203125	\\
0.666045190216185	0.000274658203125	\\
0.666089581391219	0.000274658203125	\\
0.666133972566254	0.000244140625	\\
0.666178363741288	-3.0517578125e-05	\\
0.666222754916323	0.000274658203125	\\
0.666267146091357	0.000396728515625	\\
0.666311537266391	-0.0001220703125	\\
0.666355928441426	-0.000335693359375	\\
0.66640031961646	-0.000244140625	\\
0.666444710791495	-0.000244140625	\\
0.666489101966529	-0.000396728515625	\\
0.666533493141563	-0.0006103515625	\\
0.666577884316598	-0.0008544921875	\\
0.666622275491632	-0.0006103515625	\\
0.666666666666667	-0.00054931640625	\\
0.666711057841701	-0.000579833984375	\\
0.666755449016735	-0.000640869140625	\\
0.66679984019177	-0.000518798828125	\\
0.666844231366804	-0.00030517578125	\\
0.666888622541839	-0.000152587890625	\\
0.666933013716873	0	\\
0.666977404891907	-0.0001220703125	\\
0.667021796066942	-0.000244140625	\\
0.667066187241976	-0.0001220703125	\\
0.667110578417011	-0.000732421875	\\
0.667154969592045	-0.0008544921875	\\
0.66719936076708	-0.00030517578125	\\
0.667243751942114	-0.000701904296875	\\
0.667288143117148	-0.001312255859375	\\
0.667332534292183	-0.000946044921875	\\
0.667376925467217	-0.0006103515625	\\
0.667421316642252	-0.000457763671875	\\
0.667465707817286	-0.000732421875	\\
0.66751009899232	-0.000640869140625	\\
0.667554490167355	-0.00067138671875	\\
0.667598881342389	-0.000518798828125	\\
0.667643272517424	-0.000518798828125	\\
0.667687663692458	-0.00042724609375	\\
0.667732054867492	-0.00054931640625	\\
0.667776446042527	-0.000946044921875	\\
0.667820837217561	-0.000518798828125	\\
0.667865228392596	-0.00048828125	\\
0.66790961956763	-0.000396728515625	\\
0.667954010742664	-0.0006103515625	\\
0.667998401917699	-0.0003662109375	\\
0.668042793092733	0.00018310546875	\\
0.668087184267768	-3.0517578125e-05	\\
0.668131575442802	-0.0003662109375	\\
0.668175966617836	-0.000396728515625	\\
0.668220357792871	-9.1552734375e-05	\\
0.668264748967905	-6.103515625e-05	\\
0.66830914014294	0.000213623046875	\\
0.668353531317974	0	\\
0.668397922493008	-0.000152587890625	\\
0.668442313668043	0.000213623046875	\\
0.668486704843077	-6.103515625e-05	\\
0.668531096018112	-0.00018310546875	\\
0.668575487193146	0.000213623046875	\\
0.66861987836818	0.000396728515625	\\
0.668664269543215	0.000213623046875	\\
0.668708660718249	0.000244140625	\\
0.668753051893284	0.000396728515625	\\
0.668797443068318	9.1552734375e-05	\\
0.668841834243352	9.1552734375e-05	\\
0.668886225418387	0.00042724609375	\\
0.668930616593421	0.000244140625	\\
0.668975007768456	0	\\
0.66901939894349	0.000244140625	\\
0.669063790118524	0.000152587890625	\\
0.669108181293559	0.000213623046875	\\
0.669152572468593	0.00018310546875	\\
0.669196963643628	-3.0517578125e-05	\\
0.669241354818662	-0.0001220703125	\\
0.669285745993697	-0.000244140625	\\
0.669330137168731	-6.103515625e-05	\\
0.669374528343765	-0.000213623046875	\\
0.6694189195188	-0.000518798828125	\\
0.669463310693834	-0.00030517578125	\\
0.669507701868868	-0.000396728515625	\\
0.669552093043903	-0.000457763671875	\\
0.669596484218937	0	\\
0.669640875393972	-0.00030517578125	\\
0.669685266569006	-9.1552734375e-05	\\
0.66972965774404	0.00018310546875	\\
0.669774048919075	3.0517578125e-05	\\
0.669818440094109	6.103515625e-05	\\
0.669862831269144	0.000457763671875	\\
0.669907222444178	0.000244140625	\\
0.669951613619213	0.000457763671875	\\
0.669996004794247	0.000885009765625	\\
0.670040395969281	0.000701904296875	\\
0.670084787144316	0.000518798828125	\\
0.67012917831935	0.000823974609375	\\
0.670173569494385	0.00091552734375	\\
0.670217960669419	0.0006103515625	\\
0.670262351844453	0.001220703125	\\
0.670306743019488	0.001190185546875	\\
0.670351134194522	0.000885009765625	\\
0.670395525369556	0.0013427734375	\\
0.670439916544591	0.00128173828125	\\
0.670484307719625	0.001556396484375	\\
0.67052869889466	0.001739501953125	\\
0.670573090069694	0.001617431640625	\\
0.670617481244729	0.001739501953125	\\
0.670661872419763	0.00201416015625	\\
0.670706263594797	0.0018310546875	\\
0.670750654769832	0.001800537109375	\\
0.670795045944866	0.0018310546875	\\
0.670839437119901	0.001495361328125	\\
0.670883828294935	0.0018310546875	\\
0.670928219469969	0.00177001953125	\\
0.670972610645004	0.0015869140625	\\
0.671017001820038	0.00140380859375	\\
0.671061392995073	0.00067138671875	\\
0.671105784170107	0.000640869140625	\\
0.671150175345141	0.000885009765625	\\
0.671194566520176	0.00067138671875	\\
0.67123895769521	0.0003662109375	\\
0.671283348870245	0	\\
0.671327740045279	3.0517578125e-05	\\
0.671372131220313	0.000213623046875	\\
0.671416522395348	0.00018310546875	\\
0.671460913570382	0.000457763671875	\\
0.671505304745417	0.00042724609375	\\
0.671549695920451	-0.00018310546875	\\
0.671594087095485	-0.000457763671875	\\
0.67163847827052	-0.00091552734375	\\
0.671682869445554	-0.000579833984375	\\
0.671727260620589	-0.000244140625	\\
0.671771651795623	-0.001068115234375	\\
0.671816042970657	-0.00054931640625	\\
0.671860434145692	-0.000396728515625	\\
0.671904825320726	-0.00042724609375	\\
0.671949216495761	-3.0517578125e-05	\\
0.671993607670795	-0.000244140625	\\
0.672037998845829	-0.000213623046875	\\
0.672082390020864	0.00018310546875	\\
0.672126781195898	0.00048828125	\\
0.672171172370933	0.00042724609375	\\
0.672215563545967	0.000396728515625	\\
0.672259954721001	0.0008544921875	\\
0.672304345896036	0.001007080078125	\\
0.67234873707107	0.000701904296875	\\
0.672393128246105	0.000732421875	\\
0.672437519421139	0.0008544921875	\\
0.672481910596173	0.000885009765625	\\
0.672526301771208	0.001068115234375	\\
0.672570692946242	0.000579833984375	\\
0.672615084121277	0.00048828125	\\
0.672659475296311	0.00048828125	\\
0.672703866471346	0.000885009765625	\\
0.67274825764638	0.0009765625	\\
0.672792648821414	0.00103759765625	\\
0.672837039996449	0.0015869140625	\\
0.672881431171483	0.001251220703125	\\
0.672925822346518	0.00103759765625	\\
0.672970213521552	0.00079345703125	\\
0.673014604696586	0.0003662109375	\\
0.673058995871621	0.000335693359375	\\
0.673103387046655	3.0517578125e-05	\\
0.673147778221689	-0.000396728515625	\\
0.673192169396724	-0.000762939453125	\\
0.673236560571758	0.00018310546875	\\
0.673280951746793	3.0517578125e-05	\\
0.673325342921827	0.000213623046875	\\
0.673369734096862	0.000762939453125	\\
0.673414125271896	0.00030517578125	\\
0.67345851644693	0.000518798828125	\\
0.673502907621965	0.000274658203125	\\
0.673547298796999	0.00018310546875	\\
0.673591689972034	6.103515625e-05	\\
0.673636081147068	-0.00018310546875	\\
0.673680472322102	-3.0517578125e-05	\\
0.673724863497137	-0.000457763671875	\\
0.673769254672171	-0.0003662109375	\\
0.673813645847206	-0.000244140625	\\
0.67385803702224	-0.0003662109375	\\
0.673902428197274	0	\\
0.673946819372309	-9.1552734375e-05	\\
0.673991210547343	-0.00018310546875	\\
0.674035601722378	-0.000335693359375	\\
0.674079992897412	-0.000335693359375	\\
0.674124384072446	-0.00067138671875	\\
0.674168775247481	-0.0010986328125	\\
0.674213166422515	-0.000762939453125	\\
0.67425755759755	-0.0009765625	\\
0.674301948772584	-0.000762939453125	\\
0.674346339947618	-0.000579833984375	\\
0.674390731122653	-0.00048828125	\\
0.674435122297687	-0.0003662109375	\\
0.674479513472722	-9.1552734375e-05	\\
0.674523904647756	-0.000518798828125	\\
0.67456829582279	-0.000732421875	\\
0.674612686997825	-0.00091552734375	\\
0.674657078172859	-0.000640869140625	\\
0.674701469347894	-0.000244140625	\\
0.674745860522928	-0.000518798828125	\\
0.674790251697962	-0.000823974609375	\\
0.674834642872997	-0.0008544921875	\\
0.674879034048031	-0.000640869140625	\\
0.674923425223066	-0.00067138671875	\\
0.6749678163981	-0.000701904296875	\\
0.675012207573135	-0.00054931640625	\\
0.675056598748169	-0.000274658203125	\\
0.675100989923203	-0.000396728515625	\\
0.675145381098238	-0.0003662109375	\\
0.675189772273272	-0.00079345703125	\\
0.675234163448306	-0.0009765625	\\
0.675278554623341	-0.00042724609375	\\
0.675322945798375	-0.000732421875	\\
0.67536733697341	-0.00054931640625	\\
0.675411728148444	-0.000274658203125	\\
0.675456119323478	-0.000518798828125	\\
0.675500510498513	-0.000701904296875	\\
0.675544901673547	-0.0001220703125	\\
0.675589292848582	-6.103515625e-05	\\
0.675633684023616	-0.0003662109375	\\
0.675678075198651	-0.000396728515625	\\
0.675722466373685	-0.000457763671875	\\
0.675766857548719	-0.000701904296875	\\
0.675811248723754	-0.0009765625	\\
0.675855639898788	-0.001007080078125	\\
0.675900031073823	-0.00103759765625	\\
0.675944422248857	-0.00091552734375	\\
0.675988813423891	-0.0008544921875	\\
0.676033204598926	-0.000640869140625	\\
0.67607759577396	-0.00067138671875	\\
0.676121986948995	-0.00030517578125	\\
0.676166378124029	-0.00030517578125	\\
0.676210769299063	-0.0006103515625	\\
0.676255160474098	-0.000823974609375	\\
0.676299551649132	-0.0009765625	\\
0.676343942824167	-0.0008544921875	\\
0.676388333999201	-0.001251220703125	\\
0.676432725174235	-0.001190185546875	\\
0.67647711634927	-0.000946044921875	\\
0.676521507524304	-0.000579833984375	\\
0.676565898699339	-0.000213623046875	\\
0.676610289874373	-0.000518798828125	\\
0.676654681049407	-0.0003662109375	\\
0.676699072224442	-0.000274658203125	\\
0.676743463399476	-6.103515625e-05	\\
0.676787854574511	0	\\
0.676832245749545	-0.000244140625	\\
0.676876636924579	-0.000640869140625	\\
0.676921028099614	-0.00091552734375	\\
0.676965419274648	-0.00054931640625	\\
0.677009810449683	-0.000457763671875	\\
0.677054201624717	-0.00048828125	\\
0.677098592799751	-0.0006103515625	\\
0.677142983974786	-0.000885009765625	\\
0.67718737514982	-0.000335693359375	\\
0.677231766324855	-0.00042724609375	\\
0.677276157499889	-0.000732421875	\\
0.677320548674923	-0.000244140625	\\
0.677364939849958	0.00018310546875	\\
0.677409331024992	6.103515625e-05	\\
0.677453722200027	-9.1552734375e-05	\\
0.677498113375061	0.00018310546875	\\
0.677542504550095	-3.0517578125e-05	\\
0.67758689572513	-0.000152587890625	\\
0.677631286900164	3.0517578125e-05	\\
0.677675678075199	0.0001220703125	\\
0.677720069250233	0.00030517578125	\\
0.677764460425268	0.000579833984375	\\
0.677808851600302	0.000396728515625	\\
0.677853242775336	3.0517578125e-05	\\
0.677897633950371	3.0517578125e-05	\\
0.677942025125405	0.000152587890625	\\
0.677986416300439	0.00030517578125	\\
0.678030807475474	0.000396728515625	\\
0.678075198650508	6.103515625e-05	\\
0.678119589825543	0.000274658203125	\\
0.678163981000577	0.00030517578125	\\
0.678208372175611	0.0001220703125	\\
0.678252763350646	0.000244140625	\\
0.67829715452568	0.0001220703125	\\
0.678341545700715	0	\\
0.678385936875749	0.000244140625	\\
0.678430328050784	0.000762939453125	\\
0.678474719225818	0.00042724609375	\\
0.678519110400852	0.0001220703125	\\
0.678563501575887	-0.000152587890625	\\
0.678607892750921	-0.000213623046875	\\
0.678652283925956	-0.000396728515625	\\
0.67869667510099	-0.000701904296875	\\
0.678741066276024	-0.000213623046875	\\
0.678785457451059	-0.000457763671875	\\
0.678829848626093	-0.00048828125	\\
0.678874239801127	-0.00067138671875	\\
0.678918630976162	-0.00103759765625	\\
0.678963022151196	-0.001007080078125	\\
0.679007413326231	-0.00164794921875	\\
0.679051804501265	-0.001251220703125	\\
0.6790961956763	-0.001251220703125	\\
0.679140586851334	-0.001434326171875	\\
0.679184978026368	-0.00164794921875	\\
0.679229369201403	-0.001983642578125	\\
0.679273760376437	-0.001953125	\\
0.679318151551472	-0.0023193359375	\\
0.679362542726506	-0.002532958984375	\\
0.67940693390154	-0.0020751953125	\\
0.679451325076575	-0.0023193359375	\\
0.679495716251609	-0.0023193359375	\\
0.679540107426644	-0.0020751953125	\\
0.679584498601678	-0.002593994140625	\\
0.679628889776712	-0.002410888671875	\\
0.679673280951747	-0.002227783203125	\\
0.679717672126781	-0.001983642578125	\\
0.679762063301816	-0.002197265625	\\
0.67980645447685	-0.002532958984375	\\
0.679850845651884	-0.002105712890625	\\
0.679895236826919	-0.001739501953125	\\
0.679939628001953	-0.001800537109375	\\
0.679984019176988	-0.001678466796875	\\
0.680028410352022	-0.001434326171875	\\
0.680072801527056	-0.001220703125	\\
0.680117192702091	-0.001129150390625	\\
0.680161583877125	-0.001312255859375	\\
0.68020597505216	-0.000885009765625	\\
0.680250366227194	-0.00042724609375	\\
0.680294757402228	-0.00018310546875	\\
0.680339148577263	-0.00030517578125	\\
0.680383539752297	-0.000518798828125	\\
0.680427930927332	-0.0003662109375	\\
0.680472322102366	-0.000213623046875	\\
0.6805167132774	-0.000213623046875	\\
0.680561104452435	-0.000335693359375	\\
0.680605495627469	-0.00030517578125	\\
0.680649886802504	-0.000823974609375	\\
0.680694277977538	-0.0008544921875	\\
0.680738669152572	-0.000579833984375	\\
0.680783060327607	-0.0006103515625	\\
0.680827451502641	-0.0003662109375	\\
0.680871842677676	-0.000152587890625	\\
0.68091623385271	-0.00067138671875	\\
0.680960625027744	-0.0008544921875	\\
0.681005016202779	-0.000762939453125	\\
0.681049407377813	-0.0008544921875	\\
0.681093798552848	-0.001007080078125	\\
0.681138189727882	-0.001373291015625	\\
0.681182580902917	-0.001800537109375	\\
0.681226972077951	-0.001861572265625	\\
0.681271363252985	-0.001953125	\\
0.68131575442802	-0.002197265625	\\
0.681360145603054	-0.001953125	\\
0.681404536778089	-0.002105712890625	\\
0.681448927953123	-0.002593994140625	\\
0.681493319128157	-0.0023193359375	\\
0.681537710303192	-0.00225830078125	\\
0.681582101478226	-0.00262451171875	\\
0.681626492653261	-0.002655029296875	\\
0.681670883828295	-0.002197265625	\\
0.681715275003329	-0.00244140625	\\
0.681759666178364	-0.0025634765625	\\
0.681804057353398	-0.002197265625	\\
0.681848448528433	-0.001922607421875	\\
0.681892839703467	-0.001373291015625	\\
0.681937230878501	-0.002105712890625	\\
0.681981622053536	-0.001983642578125	\\
0.68202601322857	-0.001251220703125	\\
0.682070404403605	-0.00140380859375	\\
0.682114795578639	-0.0009765625	\\
0.682159186753673	-0.0009765625	\\
0.682203577928708	-0.001190185546875	\\
0.682247969103742	-0.000946044921875	\\
0.682292360278777	-0.001190185546875	\\
0.682336751453811	-0.00128173828125	\\
0.682381142628845	-0.000762939453125	\\
0.68242553380388	-0.000701904296875	\\
0.682469924978914	-0.00103759765625	\\
0.682514316153949	-0.00103759765625	\\
0.682558707328983	-0.001220703125	\\
0.682603098504017	-0.0008544921875	\\
0.682647489679052	-0.000701904296875	\\
0.682691880854086	-0.001007080078125	\\
0.682736272029121	-0.0013427734375	\\
0.682780663204155	-0.001617431640625	\\
0.682825054379189	-0.001800537109375	\\
0.682869445554224	-0.00177001953125	\\
0.682913836729258	-0.0018310546875	\\
0.682958227904293	-0.001953125	\\
0.683002619079327	-0.002105712890625	\\
0.683047010254361	-0.001983642578125	\\
0.683091401429396	-0.001617431640625	\\
0.68313579260443	-0.00146484375	\\
0.683180183779465	-0.00164794921875	\\
0.683224574954499	-0.002044677734375	\\
0.683268966129533	-0.00213623046875	\\
0.683313357304568	-0.002227783203125	\\
0.683357748479602	-0.00213623046875	\\
0.683402139654637	-0.00225830078125	\\
0.683446530829671	-0.00225830078125	\\
0.683490922004706	-0.00213623046875	\\
0.68353531317974	-0.001861572265625	\\
0.683579704354774	-0.001708984375	\\
0.683624095529809	-0.001800537109375	\\
0.683668486704843	-0.001922607421875	\\
0.683712877879877	-0.002197265625	\\
0.683757269054912	-0.002288818359375	\\
0.683801660229946	-0.001800537109375	\\
0.683846051404981	-0.0018310546875	\\
0.683890442580015	-0.001739501953125	\\
0.683934833755049	-0.002044677734375	\\
0.683979224930084	-0.0025634765625	\\
0.684023616105118	-0.002288818359375	\\
0.684068007280153	-0.00189208984375	\\
0.684112398455187	-0.001678466796875	\\
0.684156789630222	-0.001434326171875	\\
0.684201180805256	-0.0015869140625	\\
0.68424557198029	-0.001617431640625	\\
0.684289963155325	-0.001129150390625	\\
0.684334354330359	-0.001190185546875	\\
0.684378745505394	-0.00152587890625	\\
0.684423136680428	-0.001312255859375	\\
0.684467527855462	-0.001556396484375	\\
0.684511919030497	-0.00128173828125	\\
0.684556310205531	-0.001129150390625	\\
0.684600701380566	-0.001495361328125	\\
0.6846450925556	-0.00115966796875	\\
0.684689483730634	-0.001129150390625	\\
0.684733874905669	-0.0010986328125	\\
0.684778266080703	-0.000701904296875	\\
0.684822657255738	-0.000762939453125	\\
0.684867048430772	-0.000762939453125	\\
0.684911439605806	-0.000701904296875	\\
0.684955830780841	-0.000885009765625	\\
0.685000221955875	-0.000640869140625	\\
0.68504461313091	-0.00048828125	\\
0.685089004305944	-0.00042724609375	\\
0.685133395480978	-0.000579833984375	\\
0.685177786656013	-0.00067138671875	\\
0.685222177831047	-0.00030517578125	\\
0.685266569006082	-0.00048828125	\\
0.685310960181116	-0.000244140625	\\
0.68535535135615	-0.000396728515625	\\
0.685399742531185	-0.000457763671875	\\
0.685444133706219	-6.103515625e-05	\\
0.685488524881254	-0.000457763671875	\\
0.685532916056288	-0.000946044921875	\\
0.685577307231322	-0.00091552734375	\\
0.685621698406357	-0.000579833984375	\\
0.685666089581391	-0.000457763671875	\\
0.685710480756426	-0.00079345703125	\\
0.68575487193146	-0.00048828125	\\
0.685799263106494	-0.00030517578125	\\
0.685843654281529	-0.000244140625	\\
0.685888045456563	3.0517578125e-05	\\
0.685932436631598	-3.0517578125e-05	\\
0.685976827806632	-0.0003662109375	\\
0.686021218981666	-0.00042724609375	\\
0.686065610156701	-0.000335693359375	\\
0.686110001331735	-0.0003662109375	\\
0.68615439250677	-0.00054931640625	\\
0.686198783681804	-0.000396728515625	\\
0.686243174856839	0.00018310546875	\\
0.686287566031873	-0.00018310546875	\\
0.686331957206907	-6.103515625e-05	\\
0.686376348381942	9.1552734375e-05	\\
0.686420739556976	-0.000152587890625	\\
0.68646513073201	-3.0517578125e-05	\\
0.686509521907045	-9.1552734375e-05	\\
0.686553913082079	-0.000274658203125	\\
0.686598304257114	-0.000701904296875	\\
0.686642695432148	-0.0008544921875	\\
0.686687086607182	-0.000823974609375	\\
0.686731477782217	-0.001068115234375	\\
0.686775868957251	-0.001129150390625	\\
0.686820260132286	-0.001251220703125	\\
0.68686465130732	-0.0010986328125	\\
0.686909042482355	-0.0015869140625	\\
0.686953433657389	-0.001617431640625	\\
0.686997824832423	-0.00115966796875	\\
0.687042216007458	-0.001495361328125	\\
0.687086607182492	-0.001190185546875	\\
0.687130998357527	-0.0009765625	\\
0.687175389532561	-0.001495361328125	\\
0.687219780707595	-0.001617431640625	\\
0.68726417188263	-0.001312255859375	\\
0.687308563057664	-0.00091552734375	\\
0.687352954232698	-0.000762939453125	\\
0.687397345407733	-0.000732421875	\\
0.687441736582767	-0.0006103515625	\\
0.687486127757802	0.000274658203125	\\
0.687530518932836	0.000152587890625	\\
0.687574910107871	3.0517578125e-05	\\
0.687619301282905	0.0001220703125	\\
0.687663692457939	-0.0001220703125	\\
0.687708083632974	-0.000244140625	\\
0.687752474808008	0.00030517578125	\\
0.687796865983043	0.000579833984375	\\
0.687841257158077	0.00042724609375	\\
0.687885648333111	0.00067138671875	\\
0.687930039508146	0.001312255859375	\\
0.68797443068318	0.0010986328125	\\
0.688018821858215	0.000518798828125	\\
0.688063213033249	0.000946044921875	\\
0.688107604208283	0.0010986328125	\\
0.688151995383318	0.001007080078125	\\
0.688196386558352	0.001373291015625	\\
0.688240777733387	0.00146484375	\\
0.688285168908421	0.001312255859375	\\
0.688329560083455	0.00128173828125	\\
0.68837395125849	0.0013427734375	\\
0.688418342433524	0.001251220703125	\\
0.688462733608559	0.001220703125	\\
0.688507124783593	0.00091552734375	\\
0.688551515958627	0.000701904296875	\\
0.688595907133662	0.00091552734375	\\
0.688640298308696	0.00115966796875	\\
0.688684689483731	0.0010986328125	\\
0.688729080658765	0.000946044921875	\\
0.688773471833799	0.001068115234375	\\
0.688817863008834	0.000946044921875	\\
0.688862254183868	0.000732421875	\\
0.688906645358903	0.000701904296875	\\
0.688951036533937	0.00067138671875	\\
0.688995427708971	0.000579833984375	\\
0.689039818884006	0.0006103515625	\\
0.68908421005904	0.000274658203125	\\
0.689128601234075	-0.000244140625	\\
0.689172992409109	6.103515625e-05	\\
0.689217383584143	0.000152587890625	\\
0.689261774759178	9.1552734375e-05	\\
0.689306165934212	0.000640869140625	\\
0.689350557109247	0.00054931640625	\\
0.689394948284281	0.000640869140625	\\
0.689439339459315	0.00091552734375	\\
0.68948373063435	0.00115966796875	\\
0.689528121809384	0.000640869140625	\\
0.689572512984419	0.00048828125	\\
0.689616904159453	0.000823974609375	\\
0.689661295334488	0.00103759765625	\\
0.689705686509522	0.001220703125	\\
0.689750077684556	0.001068115234375	\\
0.689794468859591	0.00115966796875	\\
0.689838860034625	0.000946044921875	\\
0.68988325120966	0.00091552734375	\\
0.689927642384694	0.000823974609375	\\
0.689972033559728	0.00079345703125	\\
0.690016424734763	0.001068115234375	\\
0.690060815909797	0.00115966796875	\\
0.690105207084832	0.00146484375	\\
0.690149598259866	0.00152587890625	\\
0.6901939894349	0.001922607421875	\\
0.690238380609935	0.0018310546875	\\
0.690282771784969	0.0018310546875	\\
0.690327162960004	0.001678466796875	\\
0.690371554135038	0.00189208984375	\\
0.690415945310072	0.00152587890625	\\
0.690460336485107	0.000946044921875	\\
0.690504727660141	0.0013427734375	\\
0.690549118835176	0.0013427734375	\\
0.69059351001021	0.0010986328125	\\
0.690637901185244	0.00103759765625	\\
0.690682292360279	0.000946044921875	\\
0.690726683535313	0.00115966796875	\\
0.690771074710348	0.0006103515625	\\
0.690815465885382	6.103515625e-05	\\
0.690859857060416	-3.0517578125e-05	\\
0.690904248235451	-3.0517578125e-05	\\
0.690948639410485	-0.000244140625	\\
0.69099303058552	-0.00018310546875	\\
0.691037421760554	-0.000274658203125	\\
0.691081812935588	-0.00054931640625	\\
0.691126204110623	-0.000152587890625	\\
0.691170595285657	0.000213623046875	\\
0.691214986460692	0.0003662109375	\\
0.691259377635726	0.000213623046875	\\
0.69130376881076	-6.103515625e-05	\\
0.691348159985795	-0.000152587890625	\\
0.691392551160829	6.103515625e-05	\\
0.691436942335864	0.000244140625	\\
0.691481333510898	0.001007080078125	\\
0.691525724685932	0.001312255859375	\\
0.691570115860967	0.00103759765625	\\
0.691614507036001	0.0010986328125	\\
0.691658898211036	0.001983642578125	\\
0.69170328938607	0.002410888671875	\\
0.691747680561104	0.002288818359375	\\
0.691792071736139	0.002349853515625	\\
0.691836462911173	0.0023193359375	\\
0.691880854086208	0.0028076171875	\\
0.691925245261242	0.00311279296875	\\
0.691969636436277	0.003448486328125	\\
0.692014027611311	0.00335693359375	\\
0.692058418786345	0.00335693359375	\\
0.69210280996138	0.00372314453125	\\
0.692147201136414	0.00323486328125	\\
0.692191592311448	0.003082275390625	\\
0.692235983486483	0.003692626953125	\\
0.692280374661517	0.003631591796875	\\
0.692324765836552	0.00396728515625	\\
0.692369157011586	0.003997802734375	\\
0.69241354818662	0.004119873046875	\\
0.692457939361655	0.003997802734375	\\
0.692502330536689	0.003173828125	\\
0.692546721711724	0.0032958984375	\\
0.692591112886758	0.00372314453125	\\
0.692635504061793	0.00360107421875	\\
0.692679895236827	0.0032958984375	\\
0.692724286411861	0.003082275390625	\\
0.692768677586896	0.0029296875	\\
0.69281306876193	0.003173828125	\\
0.692857459936965	0.0032958984375	\\
0.692901851111999	0.003509521484375	\\
0.692946242287033	0.003631591796875	\\
0.692990633462068	0.00360107421875	\\
0.693035024637102	0.003448486328125	\\
0.693079415812137	0.003326416015625	\\
0.693123806987171	0.003143310546875	\\
0.693168198162205	0.002685546875	\\
0.69321258933724	0.0029296875	\\
0.693256980512274	0.0030517578125	\\
0.693301371687309	0.003021240234375	\\
0.693345762862343	0.0025634765625	\\
0.693390154037377	0.002349853515625	\\
0.693434545212412	0.002899169921875	\\
0.693478936387446	0.002777099609375	\\
0.693523327562481	0.0029296875	\\
0.693567718737515	0.0030517578125	\\
0.693612109912549	0.0023193359375	\\
0.693656501087584	0.002288818359375	\\
0.693700892262618	0.002227783203125	\\
0.693745283437653	0.00177001953125	\\
0.693789674612687	0.00213623046875	\\
0.693834065787721	0.00213623046875	\\
0.693878456962756	0.002166748046875	\\
0.69392284813779	0.002593994140625	\\
0.693967239312825	0.002685546875	\\
0.694011630487859	0.002899169921875	\\
0.694056021662893	0.002593994140625	\\
0.694100412837928	0.002532958984375	\\
0.694144804012962	0.00274658203125	\\
0.694189195187997	0.002655029296875	\\
0.694233586363031	0.00286865234375	\\
0.694277977538065	0.002685546875	\\
0.6943223687131	0.002105712890625	\\
0.694366759888134	0.002349853515625	\\
0.694411151063169	0.002288818359375	\\
0.694455542238203	0.002410888671875	\\
0.694499933413237	0.0025634765625	\\
0.694544324588272	0.002777099609375	\\
0.694588715763306	0.002593994140625	\\
0.694633106938341	0.00262451171875	\\
0.694677498113375	0.002777099609375	\\
0.69472188928841	0.002655029296875	\\
0.694766280463444	0.00244140625	\\
0.694810671638478	0.002960205078125	\\
0.694855062813513	0.00341796875	\\
0.694899453988547	0.0029296875	\\
0.694943845163581	0.003021240234375	\\
0.694988236338616	0.003204345703125	\\
0.69503262751365	0.003143310546875	\\
0.695077018688685	0.00286865234375	\\
0.695121409863719	0.003021240234375	\\
0.695165801038753	0.003082275390625	\\
0.695210192213788	0.003326416015625	\\
0.695254583388822	0.00341796875	\\
0.695298974563857	0.00341796875	\\
0.695343365738891	0.003570556640625	\\
0.695387756913926	0.003265380859375	\\
0.69543214808896	0.0032958984375	\\
0.695476539263994	0.003143310546875	\\
0.695520930439029	0.003204345703125	\\
0.695565321614063	0.00286865234375	\\
0.695609712789098	0.00274658203125	\\
0.695654103964132	0.002471923828125	\\
0.695698495139166	0.00250244140625	\\
0.695742886314201	0.00286865234375	\\
0.695787277489235	0.002777099609375	\\
0.69583166866427	0.002593994140625	\\
0.695876059839304	0.0028076171875	\\
0.695920451014338	0.003173828125	\\
0.695964842189373	0.0028076171875	\\
0.696009233364407	0.002777099609375	\\
0.696053624539442	0.002899169921875	\\
0.696098015714476	0.0025634765625	\\
0.69614240688951	0.002288818359375	\\
0.696186798064545	0.00250244140625	\\
0.696231189239579	0.0018310546875	\\
0.696275580414614	0.001983642578125	\\
0.696319971589648	0.002349853515625	\\
0.696364362764682	0.001983642578125	\\
0.696408753939717	0.0020751953125	\\
0.696453145114751	0.001708984375	\\
0.696497536289786	0.002197265625	\\
0.69654192746482	0.002105712890625	\\
0.696586318639854	0.001861572265625	\\
0.696630709814889	0.0020751953125	\\
0.696675100989923	0.001617431640625	\\
0.696719492164958	0.00189208984375	\\
0.696763883339992	0.0018310546875	\\
0.696808274515026	0.00164794921875	\\
0.696852665690061	0.001708984375	\\
0.696897056865095	0.00177001953125	\\
0.69694144804013	0.001861572265625	\\
0.696985839215164	0.0018310546875	\\
0.697030230390198	0.0018310546875	\\
0.697074621565233	0.001556396484375	\\
0.697119012740267	0.0018310546875	\\
0.697163403915302	0.00225830078125	\\
0.697207795090336	0.002288818359375	\\
0.69725218626537	0.001953125	\\
0.697296577440405	0.002197265625	\\
0.697340968615439	0.00225830078125	\\
0.697385359790474	0.002105712890625	\\
0.697429750965508	0.00274658203125	\\
0.697474142140542	0.00299072265625	\\
0.697518533315577	0.0029296875	\\
0.697562924490611	0.0029296875	\\
0.697607315665646	0.003173828125	\\
0.69765170684068	0.002655029296875	\\
0.697696098015715	0.002593994140625	\\
0.697740489190749	0.0028076171875	\\
0.697784880365783	0.002960205078125	\\
0.697829271540818	0.0029296875	\\
0.697873662715852	0.003204345703125	\\
0.697918053890886	0.003265380859375	\\
0.697962445065921	0.0032958984375	\\
0.698006836240955	0.003204345703125	\\
0.69805122741599	0.002838134765625	\\
0.698095618591024	0.002471923828125	\\
0.698140009766059	0.002471923828125	\\
0.698184400941093	0.002349853515625	\\
0.698228792116127	0.002410888671875	\\
0.698273183291162	0.002227783203125	\\
0.698317574466196	0.001953125	\\
0.698361965641231	0.0018310546875	\\
0.698406356816265	0.00177001953125	\\
0.698450747991299	0.002349853515625	\\
0.698495139166334	0.002227783203125	\\
0.698539530341368	0.0015869140625	\\
0.698583921516403	0.0015869140625	\\
0.698628312691437	0.001220703125	\\
0.698672703866471	0.0008544921875	\\
0.698717095041506	0.001312255859375	\\
0.69876148621654	0.001251220703125	\\
0.698805877391575	0.00103759765625	\\
0.698850268566609	0.00048828125	\\
0.698894659741643	0.000396728515625	\\
0.698939050916678	0.000457763671875	\\
0.698983442091712	0.000732421875	\\
0.699027833266747	0.000518798828125	\\
0.699072224441781	0.000152587890625	\\
0.699116615616815	0.000732421875	\\
0.69916100679185	0.000885009765625	\\
0.699205397966884	0.000640869140625	\\
0.699249789141919	0.000762939453125	\\
0.699294180316953	0.0008544921875	\\
0.699338571491987	0.000823974609375	\\
0.699382962667022	0.00054931640625	\\
0.699427353842056	0.000762939453125	\\
0.699471745017091	0.001312255859375	\\
0.699516136192125	0.001556396484375	\\
0.699560527367159	0.0013427734375	\\
0.699604918542194	0.000732421875	\\
0.699649309717228	0.00128173828125	\\
0.699693700892263	0.00177001953125	\\
0.699738092067297	0.001434326171875	\\
0.699782483242331	0.00164794921875	\\
0.699826874417366	0.0018310546875	\\
0.6998712655924	0.001800537109375	\\
0.699915656767435	0.0018310546875	\\
0.699960047942469	0.001739501953125	\\
0.700004439117503	0.001983642578125	\\
0.700048830292538	0.002197265625	\\
0.700093221467572	0.00250244140625	\\
0.700137612642607	0.002288818359375	\\
0.700182003817641	0.002349853515625	\\
0.700226394992675	0.002166748046875	\\
0.70027078616771	0.00189208984375	\\
0.700315177342744	0.001983642578125	\\
0.700359568517779	0.001708984375	\\
0.700403959692813	0.002044677734375	\\
0.700448350867848	0.00250244140625	\\
0.700492742042882	0.00201416015625	\\
0.700537133217916	0.001739501953125	\\
0.700581524392951	0.001861572265625	\\
0.700625915567985	0.001922607421875	\\
0.700670306743019	0.00213623046875	\\
0.700714697918054	0.002288818359375	\\
0.700759089093088	0.002227783203125	\\
0.700803480268123	0.001556396484375	\\
0.700847871443157	0.001312255859375	\\
0.700892262618191	0.001312255859375	\\
0.700936653793226	0.00091552734375	\\
0.70098104496826	0.00115966796875	\\
0.701025436143295	0.001068115234375	\\
0.701069827318329	0.0006103515625	\\
0.701114218493364	0.0009765625	\\
0.701158609668398	0.00067138671875	\\
0.701203000843432	0.00054931640625	\\
0.701247392018467	0.00128173828125	\\
0.701291783193501	0.001373291015625	\\
0.701336174368536	0.001312255859375	\\
0.70138056554357	0.001495361328125	\\
0.701424956718604	0.00115966796875	\\
0.701469347893639	0.0010986328125	\\
0.701513739068673	0.001495361328125	\\
0.701558130243708	0.001434326171875	\\
0.701602521418742	0.001251220703125	\\
0.701646912593776	0.001251220703125	\\
0.701691303768811	0.0013427734375	\\
0.701735694943845	0.001312255859375	\\
0.70178008611888	0.0013427734375	\\
0.701824477293914	0.001708984375	\\
0.701868868468948	0.00201416015625	\\
0.701913259643983	0.00238037109375	\\
0.701957650819017	0.00238037109375	\\
0.702002041994052	0.002288818359375	\\
0.702046433169086	0.0023193359375	\\
0.70209082434412	0.002166748046875	\\
0.702135215519155	0.002349853515625	\\
0.702179606694189	0.00250244140625	\\
0.702223997869224	0.001861572265625	\\
0.702268389044258	0.001922607421875	\\
0.702312780219292	0.0025634765625	\\
0.702357171394327	0.00274658203125	\\
0.702401562569361	0.002838134765625	\\
0.702445953744396	0.002655029296875	\\
0.70249034491943	0.002532958984375	\\
0.702534736094464	0.00262451171875	\\
0.702579127269499	0.002685546875	\\
0.702623518444533	0.0023193359375	\\
0.702667909619568	0.00262451171875	\\
0.702712300794602	0.0028076171875	\\
0.702756691969636	0.00238037109375	\\
0.702801083144671	0.002532958984375	\\
0.702845474319705	0.00250244140625	\\
0.70288986549474	0.00244140625	\\
0.702934256669774	0.00286865234375	\\
0.702978647844808	0.002716064453125	\\
0.703023039019843	0.002593994140625	\\
0.703067430194877	0.002532958984375	\\
0.703111821369912	0.002349853515625	\\
0.703156212544946	0.0023193359375	\\
0.703200603719981	0.002197265625	\\
0.703244994895015	0.002197265625	\\
0.703289386070049	0.0023193359375	\\
0.703333777245084	0.00250244140625	\\
0.703378168420118	0.002532958984375	\\
0.703422559595152	0.001861572265625	\\
0.703466950770187	0.001800537109375	\\
0.703511341945221	0.002105712890625	\\
0.703555733120256	0.002044677734375	\\
0.70360012429529	0.002044677734375	\\
0.703644515470324	0.001678466796875	\\
0.703688906645359	0.001251220703125	\\
0.703733297820393	0.00152587890625	\\
0.703777688995428	0.001312255859375	\\
0.703822080170462	0.001373291015625	\\
0.703866471345497	0.00152587890625	\\
0.703910862520531	0.001129150390625	\\
0.703955253695565	0.0008544921875	\\
0.7039996448706	0.000885009765625	\\
0.704044036045634	0.000946044921875	\\
0.704088427220669	0.00048828125	\\
0.704132818395703	0.00048828125	\\
0.704177209570737	0.0006103515625	\\
0.704221600745772	0.00067138671875	\\
0.704265991920806	0.0006103515625	\\
0.704310383095841	0.00042724609375	\\
0.704354774270875	0.000823974609375	\\
0.704399165445909	0.001007080078125	\\
0.704443556620944	0.000732421875	\\
0.704487947795978	0.00067138671875	\\
0.704532338971013	0.001007080078125	\\
0.704576730146047	0.0009765625	\\
0.704621121321081	0.0008544921875	\\
0.704665512496116	0.00103759765625	\\
0.70470990367115	0.001068115234375	\\
0.704754294846185	0.00103759765625	\\
0.704798686021219	0.00140380859375	\\
0.704843077196253	0.001251220703125	\\
0.704887468371288	0.0010986328125	\\
0.704931859546322	0.001373291015625	\\
0.704976250721357	0.00146484375	\\
0.705020641896391	0.0010986328125	\\
0.705065033071425	0.001190185546875	\\
0.70510942424646	0.00115966796875	\\
0.705153815421494	0.000946044921875	\\
0.705198206596529	0.001190185546875	\\
0.705242597771563	0.001068115234375	\\
0.705286988946597	0.001190185546875	\\
0.705331380121632	0.001129150390625	\\
0.705375771296666	0.000701904296875	\\
0.705420162471701	0.000518798828125	\\
0.705464553646735	0.00067138671875	\\
0.705508944821769	0.00115966796875	\\
0.705553335996804	0.00115966796875	\\
0.705597727171838	0.000762939453125	\\
0.705642118346873	0.00103759765625	\\
0.705686509521907	0.001495361328125	\\
0.705730900696941	0.00140380859375	\\
0.705775291871976	0.00152587890625	\\
0.70581968304701	0.001312255859375	\\
0.705864074222045	0.001220703125	\\
0.705908465397079	0.001068115234375	\\
0.705952856572113	0.00079345703125	\\
0.705997247747148	0.00091552734375	\\
0.706041638922182	0.001251220703125	\\
0.706086030097217	0.0010986328125	\\
0.706130421272251	0.001312255859375	\\
0.706174812447286	0.001495361328125	\\
0.70621920362232	0.00140380859375	\\
0.706263594797354	0.00091552734375	\\
0.706307985972389	0.00103759765625	\\
0.706352377147423	0.00128173828125	\\
0.706396768322457	0.00091552734375	\\
0.706441159497492	0.000946044921875	\\
0.706485550672526	0.000518798828125	\\
0.706529941847561	0.000823974609375	\\
0.706574333022595	0.00140380859375	\\
0.70661872419763	0.00115966796875	\\
0.706663115372664	0.001190185546875	\\
0.706707506547698	0.001495361328125	\\
0.706751897722733	0.001434326171875	\\
0.706796288897767	0.00152587890625	\\
0.706840680072802	0.001434326171875	\\
0.706885071247836	0.001373291015625	\\
0.70692946242287	0.001739501953125	\\
0.706973853597905	0.001495361328125	\\
0.707018244772939	0.001495361328125	\\
0.707062635947974	0.00164794921875	\\
0.707107027123008	0.001373291015625	\\
0.707151418298042	0.001617431640625	\\
0.707195809473077	0.001708984375	\\
0.707240200648111	0.00146484375	\\
0.707284591823146	0.001495361328125	\\
0.70732898299818	0.0015869140625	\\
0.707373374173214	0.00189208984375	\\
0.707417765348249	0.002227783203125	\\
0.707462156523283	0.002197265625	\\
0.707506547698318	0.0020751953125	\\
0.707550938873352	0.00238037109375	\\
0.707595330048386	0.002288818359375	\\
0.707639721223421	0.002197265625	\\
0.707684112398455	0.00238037109375	\\
0.70772850357349	0.002532958984375	\\
0.707772894748524	0.002349853515625	\\
0.707817285923558	0.00225830078125	\\
0.707861677098593	0.00225830078125	\\
0.707906068273627	0.00201416015625	\\
0.707950459448662	0.001708984375	\\
0.707994850623696	0.00177001953125	\\
0.70803924179873	0.001861572265625	\\
0.708083632973765	0.00189208984375	\\
0.708128024148799	0.002197265625	\\
0.708172415323834	0.0023193359375	\\
0.708216806498868	0.002288818359375	\\
0.708261197673902	0.0018310546875	\\
0.708305588848937	0.0020751953125	\\
0.708349980023971	0.002593994140625	\\
0.708394371199006	0.002227783203125	\\
0.70843876237404	0.002044677734375	\\
0.708483153549074	0.002105712890625	\\
0.708527544724109	0.002105712890625	\\
0.708571935899143	0.0013427734375	\\
0.708616327074178	0.001129150390625	\\
0.708660718249212	0.00152587890625	\\
0.708705109424246	0.0015869140625	\\
0.708749500599281	0.001251220703125	\\
0.708793891774315	0.001434326171875	\\
0.70883828294935	0.001220703125	\\
0.708882674124384	0.0010986328125	\\
0.708927065299419	0.00146484375	\\
0.708971456474453	0.001373291015625	\\
0.709015847649487	0.00152587890625	\\
0.709060238824522	0.001312255859375	\\
0.709104629999556	0.0013427734375	\\
0.70914902117459	0.0015869140625	\\
0.709193412349625	0.00146484375	\\
0.709237803524659	0.001495361328125	\\
0.709282194699694	0.001495361328125	\\
0.709326585874728	0.001953125	\\
0.709370977049762	0.001800537109375	\\
0.709415368224797	0.001617431640625	\\
0.709459759399831	0.001678466796875	\\
0.709504150574866	0.001953125	\\
0.7095485417499	0.002349853515625	\\
0.709592932924935	0.0020751953125	\\
0.709637324099969	0.002166748046875	\\
0.709681715275003	0.0020751953125	\\
0.709726106450038	0.0023193359375	\\
0.709770497625072	0.00238037109375	\\
0.709814888800107	0.00225830078125	\\
0.709859279975141	0.002471923828125	\\
0.709903671150175	0.002532958984375	\\
0.70994806232521	0.002777099609375	\\
0.709992453500244	0.002838134765625	\\
0.710036844675279	0.002593994140625	\\
0.710081235850313	0.00238037109375	\\
0.710125627025347	0.002685546875	\\
0.710170018200382	0.002716064453125	\\
0.710214409375416	0.002960205078125	\\
0.710258800550451	0.002838134765625	\\
};
\addplot [color=blue,solid,forget plot]
  table[row sep=crcr]{
0.710258800550451	0.002838134765625	\\
0.710303191725485	0.002777099609375	\\
0.710347582900519	0.003082275390625	\\
0.710391974075554	0.0029296875	\\
0.710436365250588	0.002410888671875	\\
0.710480756425623	0.002471923828125	\\
0.710525147600657	0.002685546875	\\
0.710569538775691	0.002532958984375	\\
0.710613929950726	0.002777099609375	\\
0.71065832112576	0.002777099609375	\\
0.710702712300795	0.00274658203125	\\
0.710747103475829	0.00299072265625	\\
0.710791494650863	0.003387451171875	\\
0.710835885825898	0.003326416015625	\\
0.710880277000932	0.0035400390625	\\
0.710924668175967	0.0035400390625	\\
0.710969059351001	0.003631591796875	\\
0.711013450526035	0.003662109375	\\
0.71105784170107	0.00360107421875	\\
0.711102232876104	0.003692626953125	\\
0.711146624051139	0.00390625	\\
0.711191015226173	0.00421142578125	\\
0.711235406401207	0.00433349609375	\\
0.711279797576242	0.004669189453125	\\
0.711324188751276	0.004913330078125	\\
0.711368579926311	0.004638671875	\\
0.711412971101345	0.0047607421875	\\
0.711457362276379	0.004608154296875	\\
0.711501753451414	0.00433349609375	\\
0.711546144626448	0.00433349609375	\\
0.711590535801483	0.00469970703125	\\
0.711634926976517	0.0050048828125	\\
0.711679318151552	0.005218505859375	\\
0.711723709326586	0.00537109375	\\
0.71176810050162	0.00543212890625	\\
0.711812491676655	0.005645751953125	\\
0.711856882851689	0.005645751953125	\\
0.711901274026724	0.00537109375	\\
0.711945665201758	0.005401611328125	\\
0.711990056376792	0.005340576171875	\\
0.712034447551827	0.00518798828125	\\
0.712078838726861	0.005523681640625	\\
0.712123229901895	0.005401611328125	\\
0.71216762107693	0.00518798828125	\\
0.712212012251964	0.005706787109375	\\
0.712256403426999	0.00506591796875	\\
0.712300794602033	0.00439453125	\\
0.712345185777068	0.004730224609375	\\
0.712389576952102	0.005218505859375	\\
0.712433968127136	0.0050048828125	\\
0.712478359302171	0.004547119140625	\\
0.712522750477205	0.0047607421875	\\
0.71256714165224	0.004302978515625	\\
0.712611532827274	0.003692626953125	\\
0.712655924002308	0.00360107421875	\\
0.712700315177343	0.003509521484375	\\
0.712744706352377	0.003875732421875	\\
0.712789097527412	0.003875732421875	\\
0.712833488702446	0.003997802734375	\\
0.71287787987748	0.003936767578125	\\
0.712922271052515	0.003814697265625	\\
0.712966662227549	0.00408935546875	\\
0.713011053402584	0.003936767578125	\\
0.713055444577618	0.003936767578125	\\
0.713099835752652	0.003814697265625	\\
0.713144226927687	0.00360107421875	\\
0.713188618102721	0.003631591796875	\\
0.713233009277756	0.0037841796875	\\
0.71327740045279	0.003875732421875	\\
0.713321791627824	0.003448486328125	\\
0.713366182802859	0.00323486328125	\\
0.713410573977893	0.003631591796875	\\
0.713454965152928	0.003448486328125	\\
0.713499356327962	0.003570556640625	\\
0.713543747502996	0.003631591796875	\\
0.713588138678031	0.00372314453125	\\
0.713632529853065	0.003631591796875	\\
0.7136769210281	0.003509521484375	\\
0.713721312203134	0.003662109375	\\
0.713765703378168	0.003509521484375	\\
0.713810094553203	0.003631591796875	\\
0.713854485728237	0.003448486328125	\\
0.713898876903272	0.003082275390625	\\
0.713943268078306	0.003143310546875	\\
0.71398765925334	0.003662109375	\\
0.714032050428375	0.003631591796875	\\
0.714076441603409	0.00360107421875	\\
0.714120832778444	0.00347900390625	\\
0.714165223953478	0.0037841796875	\\
0.714209615128512	0.00360107421875	\\
0.714254006303547	0.003997802734375	\\
0.714298397478581	0.004119873046875	\\
0.714342788653616	0.00341796875	\\
0.71438717982865	0.003509521484375	\\
0.714431571003684	0.003448486328125	\\
0.714475962178719	0.0030517578125	\\
0.714520353353753	0.003631591796875	\\
0.714564744528788	0.004119873046875	\\
0.714609135703822	0.003875732421875	\\
0.714653526878857	0.00421142578125	\\
0.714697918053891	0.00439453125	\\
0.714742309228925	0.004669189453125	\\
0.71478670040396	0.004791259765625	\\
0.714831091578994	0.00439453125	\\
0.714875482754028	0.004119873046875	\\
0.714919873929063	0.00433349609375	\\
0.714964265104097	0.004669189453125	\\
0.715008656279132	0.004669189453125	\\
0.715053047454166	0.00469970703125	\\
0.715097438629201	0.00482177734375	\\
0.715141829804235	0.00457763671875	\\
0.715186220979269	0.0045166015625	\\
0.715230612154304	0.004241943359375	\\
0.715275003329338	0.003875732421875	\\
0.715319394504373	0.004241943359375	\\
0.715363785679407	0.00433349609375	\\
0.715408176854441	0.004425048828125	\\
0.715452568029476	0.00457763671875	\\
0.71549695920451	0.00421142578125	\\
0.715541350379545	0.0040283203125	\\
0.715585741554579	0.004150390625	\\
0.715630132729613	0.003936767578125	\\
0.715674523904648	0.003997802734375	\\
0.715718915079682	0.00396728515625	\\
0.715763306254717	0.004150390625	\\
0.715807697429751	0.004425048828125	\\
0.715852088604785	0.003875732421875	\\
0.71589647977982	0.003448486328125	\\
0.715940870954854	0.003631591796875	\\
0.715985262129889	0.00347900390625	\\
0.716029653304923	0.003082275390625	\\
0.716074044479957	0.00323486328125	\\
0.716118435654992	0.003143310546875	\\
0.716162826830026	0.0032958984375	\\
0.716207218005061	0.0030517578125	\\
0.716251609180095	0.002838134765625	\\
0.716296000355129	0.00274658203125	\\
0.716340391530164	0.002532958984375	\\
0.716384782705198	0.002471923828125	\\
0.716429173880233	0.002044677734375	\\
0.716473565055267	0.001617431640625	\\
0.716517956230301	0.00177001953125	\\
0.716562347405336	0.00164794921875	\\
0.71660673858037	0.00146484375	\\
0.716651129755405	0.001129150390625	\\
0.716695520930439	0.00140380859375	\\
0.716739912105473	0.001434326171875	\\
0.716784303280508	0.0015869140625	\\
0.716828694455542	0.001617431640625	\\
0.716873085630577	0.00146484375	\\
0.716917476805611	0.001556396484375	\\
0.716961867980645	0.0018310546875	\\
0.71700625915568	0.001800537109375	\\
0.717050650330714	0.001617431640625	\\
0.717095041505749	0.00189208984375	\\
0.717139432680783	0.00189208984375	\\
0.717183823855817	0.00213623046875	\\
0.717228215030852	0.002044677734375	\\
0.717272606205886	0.001983642578125	\\
0.717316997380921	0.002716064453125	\\
0.717361388555955	0.00244140625	\\
0.71740577973099	0.002349853515625	\\
0.717450170906024	0.00250244140625	\\
0.717494562081058	0.002532958984375	\\
0.717538953256093	0.00262451171875	\\
0.717583344431127	0.002655029296875	\\
0.717627735606161	0.002899169921875	\\
0.717672126781196	0.002777099609375	\\
0.71771651795623	0.002471923828125	\\
0.717760909131265	0.002685546875	\\
0.717805300306299	0.0028076171875	\\
0.717849691481333	0.002288818359375	\\
0.717894082656368	0.002410888671875	\\
0.717938473831402	0.00225830078125	\\
0.717982865006437	0.002197265625	\\
0.718027256181471	0.002288818359375	\\
0.718071647356506	0.00225830078125	\\
0.71811603853154	0.00225830078125	\\
0.718160429706574	0.00225830078125	\\
0.718204820881609	0.00213623046875	\\
0.718249212056643	0.001678466796875	\\
0.718293603231678	0.001434326171875	\\
0.718337994406712	0.00152587890625	\\
0.718382385581746	0.00146484375	\\
0.718426776756781	0.001708984375	\\
0.718471167931815	0.0015869140625	\\
0.71851555910685	0.001251220703125	\\
0.718559950281884	0.00152587890625	\\
0.718604341456918	0.001434326171875	\\
0.718648732631953	0.001434326171875	\\
0.718693123806987	0.0015869140625	\\
0.718737514982022	0.001495361328125	\\
0.718781906157056	0.001617431640625	\\
0.71882629733209	0.0015869140625	\\
0.718870688507125	0.001556396484375	\\
0.718915079682159	0.001678466796875	\\
0.718959470857194	0.001678466796875	\\
0.719003862032228	0.001739501953125	\\
0.719048253207262	0.001312255859375	\\
0.719092644382297	0.001556396484375	\\
0.719137035557331	0.002166748046875	\\
0.719181426732366	0.0020751953125	\\
0.7192258179074	0.0020751953125	\\
0.719270209082434	0.002349853515625	\\
0.719314600257469	0.002349853515625	\\
0.719358991432503	0.0025634765625	\\
0.719403382607538	0.002685546875	\\
0.719447773782572	0.002349853515625	\\
0.719492164957606	0.002166748046875	\\
0.719536556132641	0.002593994140625	\\
0.719580947307675	0.002899169921875	\\
0.71962533848271	0.002716064453125	\\
0.719669729657744	0.002655029296875	\\
0.719714120832778	0.00262451171875	\\
0.719758512007813	0.00262451171875	\\
0.719802903182847	0.002410888671875	\\
0.719847294357882	0.002227783203125	\\
0.719891685532916	0.002410888671875	\\
0.71993607670795	0.002197265625	\\
0.719980467882985	0.002105712890625	\\
0.720024859058019	0.002105712890625	\\
0.720069250233054	0.002227783203125	\\
0.720113641408088	0.00213623046875	\\
0.720158032583123	0.00164794921875	\\
0.720202423758157	0.0013427734375	\\
0.720246814933191	0.001129150390625	\\
0.720291206108226	0.001007080078125	\\
0.72033559728326	0.001312255859375	\\
0.720379988458295	0.00146484375	\\
0.720424379633329	0.00164794921875	\\
0.720468770808363	0.00115966796875	\\
0.720513161983398	0.00091552734375	\\
0.720557553158432	0.001220703125	\\
0.720601944333466	0.000732421875	\\
0.720646335508501	0.000518798828125	\\
0.720690726683535	0.000579833984375	\\
0.72073511785857	0.0006103515625	\\
0.720779509033604	0.0008544921875	\\
0.720823900208639	0.001068115234375	\\
0.720868291383673	0.00115966796875	\\
0.720912682558707	0.00103759765625	\\
0.720957073733742	0.00079345703125	\\
0.721001464908776	0.0008544921875	\\
0.721045856083811	0.000335693359375	\\
0.721090247258845	0.000701904296875	\\
0.721134638433879	0.00115966796875	\\
0.721179029608914	0.00048828125	\\
0.721223420783948	0.000579833984375	\\
0.721267811958983	0.00067138671875	\\
0.721312203134017	0.000244140625	\\
0.721356594309051	0.000823974609375	\\
0.721400985484086	0.000823974609375	\\
0.72144537665912	0.000396728515625	\\
0.721489767834155	0.000518798828125	\\
0.721534159009189	0.000701904296875	\\
0.721578550184223	0.000640869140625	\\
0.721622941359258	0.000518798828125	\\
0.721667332534292	0.00018310546875	\\
0.721711723709327	0.000274658203125	\\
0.721756114884361	0.000396728515625	\\
0.721800506059395	0.000213623046875	\\
0.72184489723443	0.00018310546875	\\
0.721889288409464	-3.0517578125e-05	\\
0.721933679584499	-0.00018310546875	\\
0.721978070759533	-9.1552734375e-05	\\
0.722022461934567	-3.0517578125e-05	\\
0.722066853109602	-0.000274658203125	\\
0.722111244284636	-0.00079345703125	\\
0.722155635459671	-0.000152587890625	\\
0.722200026634705	-0.0001220703125	\\
0.722244417809739	-0.000274658203125	\\
0.722288808984774	0.000152587890625	\\
0.722333200159808	-6.103515625e-05	\\
0.722377591334843	0.000244140625	\\
0.722421982509877	0	\\
0.722466373684911	-0.00042724609375	\\
0.722510764859946	-0.000274658203125	\\
0.72255515603498	-0.00042724609375	\\
0.722599547210015	-0.000274658203125	\\
0.722643938385049	-0.000244140625	\\
0.722688329560083	-0.000457763671875	\\
0.722732720735118	-0.00018310546875	\\
0.722777111910152	0.00048828125	\\
0.722821503085187	0.000518798828125	\\
0.722865894260221	-9.1552734375e-05	\\
0.722910285435255	0.000152587890625	\\
0.72295467661029	3.0517578125e-05	\\
0.722999067785324	0.000457763671875	\\
0.723043458960359	0.000732421875	\\
0.723087850135393	0.000701904296875	\\
0.723132241310428	0.000579833984375	\\
0.723176632485462	0.000579833984375	\\
0.723221023660496	0.00067138671875	\\
0.723265414835531	0.000518798828125	\\
0.723309806010565	0.000457763671875	\\
0.723354197185599	0.000335693359375	\\
0.723398588360634	0.000213623046875	\\
0.723442979535668	0	\\
0.723487370710703	0.00048828125	\\
0.723531761885737	0.0003662109375	\\
0.723576153060772	0.000335693359375	\\
0.723620544235806	0.000457763671875	\\
0.72366493541084	-0.00018310546875	\\
0.723709326585875	-0.0006103515625	\\
0.723753717760909	-0.000762939453125	\\
0.723798108935944	-0.000579833984375	\\
0.723842500110978	-0.000274658203125	\\
0.723886891286012	-0.000579833984375	\\
0.723931282461047	-0.000823974609375	\\
0.723975673636081	-0.000762939453125	\\
0.724020064811116	-0.000946044921875	\\
0.72406445598615	-0.001251220703125	\\
0.724108847161184	-0.001220703125	\\
0.724153238336219	-0.001678466796875	\\
0.724197629511253	-0.001617431640625	\\
0.724242020686288	-0.00164794921875	\\
0.724286411861322	-0.0025634765625	\\
0.724330803036356	-0.00244140625	\\
0.724375194211391	-0.002288818359375	\\
0.724419585386425	-0.00250244140625	\\
0.72446397656146	-0.002349853515625	\\
0.724508367736494	-0.002716064453125	\\
0.724552758911528	-0.002716064453125	\\
0.724597150086563	-0.00238037109375	\\
0.724641541261597	-0.0018310546875	\\
0.724685932436632	-0.001678466796875	\\
0.724730323611666	-0.002227783203125	\\
0.7247747147867	-0.002105712890625	\\
0.724819105961735	-0.001983642578125	\\
0.724863497136769	-0.002227783203125	\\
0.724907888311804	-0.002777099609375	\\
0.724952279486838	-0.00262451171875	\\
0.724996670661872	-0.00262451171875	\\
0.725041061836907	-0.0028076171875	\\
0.725085453011941	-0.002960205078125	\\
0.725129844186976	-0.002532958984375	\\
0.72517423536201	-0.00213623046875	\\
0.725218626537044	-0.0023193359375	\\
0.725263017712079	-0.00189208984375	\\
0.725307408887113	-0.001983642578125	\\
0.725351800062148	-0.002105712890625	\\
0.725396191237182	-0.00164794921875	\\
0.725440582412216	-0.001678466796875	\\
0.725484973587251	-0.002044677734375	\\
0.725529364762285	-0.001922607421875	\\
0.72557375593732	-0.002227783203125	\\
0.725618147112354	-0.00225830078125	\\
0.725662538287388	-0.002044677734375	\\
0.725706929462423	-0.002777099609375	\\
0.725751320637457	-0.002838134765625	\\
0.725795711812492	-0.002471923828125	\\
0.725840102987526	-0.00274658203125	\\
0.725884494162561	-0.002227783203125	\\
0.725928885337595	-0.002471923828125	\\
0.725973276512629	-0.002777099609375	\\
0.726017667687664	-0.00250244140625	\\
0.726062058862698	-0.0025634765625	\\
0.726106450037733	-0.002166748046875	\\
0.726150841212767	-0.002227783203125	\\
0.726195232387801	-0.002227783203125	\\
0.726239623562836	-0.002044677734375	\\
0.72628401473787	-0.002197265625	\\
0.726328405912904	-0.002227783203125	\\
0.726372797087939	-0.00225830078125	\\
0.726417188262973	-0.0023193359375	\\
0.726461579438008	-0.0020751953125	\\
0.726505970613042	-0.00177001953125	\\
0.726550361788077	-0.00177001953125	\\
0.726594752963111	-0.00201416015625	\\
0.726639144138145	-0.001922607421875	\\
0.72668353531318	-0.00177001953125	\\
0.726727926488214	-0.001678466796875	\\
0.726772317663249	-0.001251220703125	\\
0.726816708838283	-0.00115966796875	\\
0.726861100013317	-0.00115966796875	\\
0.726905491188352	-0.00091552734375	\\
0.726949882363386	-0.0006103515625	\\
0.726994273538421	-0.000579833984375	\\
0.727038664713455	-0.000640869140625	\\
0.727083055888489	-0.000274658203125	\\
0.727127447063524	-0.000244140625	\\
0.727171838238558	-0.000152587890625	\\
0.727216229413593	0.000457763671875	\\
0.727260620588627	0.00018310546875	\\
0.727305011763661	9.1552734375e-05	\\
0.727349402938696	0.0003662109375	\\
0.72739379411373	0.00018310546875	\\
0.727438185288765	6.103515625e-05	\\
0.727482576463799	6.103515625e-05	\\
0.727526967638833	9.1552734375e-05	\\
0.727571358813868	0.000244140625	\\
0.727615749988902	0.000518798828125	\\
0.727660141163937	0.00067138671875	\\
0.727704532338971	0.000274658203125	\\
0.727748923514005	-0.000152587890625	\\
0.72779331468904	-0.000213623046875	\\
0.727837705864074	-0.00030517578125	\\
0.727882097039109	0.000274658203125	\\
0.727926488214143	3.0517578125e-05	\\
0.727970879389177	-0.000274658203125	\\
0.728015270564212	0.0001220703125	\\
0.728059661739246	-0.00042724609375	\\
0.728104052914281	-0.000457763671875	\\
0.728148444089315	-0.0003662109375	\\
0.728192835264349	-0.000762939453125	\\
0.728237226439384	-0.00079345703125	\\
0.728281617614418	-0.00103759765625	\\
0.728326008789453	-0.000762939453125	\\
0.728370399964487	-0.00079345703125	\\
0.728414791139521	-0.000823974609375	\\
0.728459182314556	-0.0009765625	\\
0.72850357348959	-0.001129150390625	\\
0.728547964664625	-0.001220703125	\\
0.728592355839659	-0.00128173828125	\\
0.728636747014694	-0.001220703125	\\
0.728681138189728	-0.001373291015625	\\
0.728725529364762	-0.001312255859375	\\
0.728769920539797	-0.00115966796875	\\
0.728814311714831	-0.00140380859375	\\
0.728858702889866	-0.00079345703125	\\
0.7289030940649	-0.000762939453125	\\
0.728947485239934	-0.00115966796875	\\
0.728991876414969	-0.00067138671875	\\
0.729036267590003	-0.00115966796875	\\
0.729080658765037	-0.001129150390625	\\
0.729125049940072	-0.000579833984375	\\
0.729169441115106	-0.00048828125	\\
0.729213832290141	-0.000457763671875	\\
0.729258223465175	-0.000244140625	\\
0.72930261464021	-0.000152587890625	\\
0.729347005815244	-0.000274658203125	\\
0.729391396990278	-0.000396728515625	\\
0.729435788165313	-0.000396728515625	\\
0.729480179340347	-0.00054931640625	\\
0.729524570515382	-0.000640869140625	\\
0.729568961690416	-0.0003662109375	\\
0.72961335286545	-0.000396728515625	\\
0.729657744040485	-0.0003662109375	\\
0.729702135215519	-0.000579833984375	\\
0.729746526390554	-0.00079345703125	\\
0.729790917565588	-0.00079345703125	\\
0.729835308740622	-0.000732421875	\\
0.729879699915657	-0.0003662109375	\\
0.729924091090691	-0.000518798828125	\\
0.729968482265726	-0.000823974609375	\\
0.73001287344076	-0.000946044921875	\\
0.730057264615794	-0.001190185546875	\\
0.730101655790829	-0.001373291015625	\\
0.730146046965863	-0.0013427734375	\\
0.730190438140898	-0.00115966796875	\\
0.730234829315932	-0.00152587890625	\\
0.730279220490966	-0.00146484375	\\
0.730323611666001	-0.001312255859375	\\
0.730368002841035	-0.0009765625	\\
0.73041239401607	-0.001007080078125	\\
0.730456785191104	-0.000823974609375	\\
0.730501176366138	-0.00042724609375	\\
0.730545567541173	-0.00067138671875	\\
0.730589958716207	-0.000762939453125	\\
0.730634349891242	-0.001220703125	\\
0.730678741066276	-0.001129150390625	\\
0.73072313224131	-0.0006103515625	\\
0.730767523416345	-0.00054931640625	\\
0.730811914591379	-0.0008544921875	\\
0.730856305766414	-0.00030517578125	\\
0.730900696941448	-0.00030517578125	\\
0.730945088116482	-0.000701904296875	\\
0.730989479291517	-0.000335693359375	\\
0.731033870466551	-0.00042724609375	\\
0.731078261641586	-0.000579833984375	\\
0.73112265281662	-0.000457763671875	\\
0.731167043991654	-0.000244140625	\\
0.731211435166689	-0.0001220703125	\\
0.731255826341723	6.103515625e-05	\\
0.731300217516758	-0.00018310546875	\\
0.731344608691792	-9.1552734375e-05	\\
0.731388999866826	0.00018310546875	\\
0.731433391041861	-0.000213623046875	\\
0.731477782216895	-0.000244140625	\\
0.73152217339193	0.000244140625	\\
0.731566564566964	0.000518798828125	\\
0.731610955741999	0.000213623046875	\\
0.731655346917033	-3.0517578125e-05	\\
0.731699738092067	-0.000274658203125	\\
0.731744129267102	-0.000335693359375	\\
0.731788520442136	-0.000274658203125	\\
0.73183291161717	-0.000579833984375	\\
0.731877302792205	-0.000640869140625	\\
0.731921693967239	-0.0006103515625	\\
0.731966085142274	-0.00042724609375	\\
0.732010476317308	-0.000152587890625	\\
0.732054867492343	-0.000457763671875	\\
0.732099258667377	-0.00042724609375	\\
0.732143649842411	-0.000213623046875	\\
0.732188041017446	6.103515625e-05	\\
0.73223243219248	0.000152587890625	\\
0.732276823367515	9.1552734375e-05	\\
0.732321214542549	0.000335693359375	\\
0.732365605717583	0.00030517578125	\\
0.732409996892618	0	\\
0.732454388067652	0.000213623046875	\\
0.732498779242687	0.000152587890625	\\
0.732543170417721	3.0517578125e-05	\\
0.732587561592755	0.000213623046875	\\
0.73263195276779	0.00042724609375	\\
0.732676343942824	0.000152587890625	\\
0.732720735117859	0	\\
0.732765126292893	9.1552734375e-05	\\
0.732809517467927	0.00042724609375	\\
0.732853908642962	0.000213623046875	\\
0.732898299817996	6.103515625e-05	\\
0.732942690993031	-0.00030517578125	\\
0.732987082168065	-3.0517578125e-05	\\
0.733031473343099	3.0517578125e-05	\\
0.733075864518134	-6.103515625e-05	\\
0.733120255693168	6.103515625e-05	\\
0.733164646868203	-6.103515625e-05	\\
0.733209038043237	-0.000152587890625	\\
0.733253429218271	-0.00018310546875	\\
0.733297820393306	0.000152587890625	\\
0.73334221156834	-0.0003662109375	\\
0.733386602743375	-0.00030517578125	\\
0.733430993918409	-0.000152587890625	\\
0.733475385093443	-0.0006103515625	\\
0.733519776268478	-0.00067138671875	\\
0.733564167443512	-0.000732421875	\\
0.733608558618547	-0.0003662109375	\\
0.733652949793581	-0.000274658203125	\\
0.733697340968615	-0.00042724609375	\\
0.73374173214365	-0.00054931640625	\\
0.733786123318684	-0.000579833984375	\\
0.733830514493719	-0.0003662109375	\\
0.733874905668753	-0.000335693359375	\\
0.733919296843787	-0.000152587890625	\\
0.733963688018822	0.000244140625	\\
0.734008079193856	-0.000152587890625	\\
0.734052470368891	3.0517578125e-05	\\
0.734096861543925	6.103515625e-05	\\
0.734141252718959	3.0517578125e-05	\\
0.734185643893994	0.000213623046875	\\
0.734230035069028	0.0001220703125	\\
0.734274426244063	0.0003662109375	\\
0.734318817419097	0.00030517578125	\\
0.734363208594132	0.000335693359375	\\
0.734407599769166	0.00067138671875	\\
0.7344519909442	0.00054931640625	\\
0.734496382119235	0.0008544921875	\\
0.734540773294269	0.000885009765625	\\
0.734585164469304	0.001220703125	\\
0.734629555644338	0.001434326171875	\\
0.734673946819372	0.00128173828125	\\
0.734718337994407	0.00146484375	\\
0.734762729169441	0.00103759765625	\\
0.734807120344475	0.001434326171875	\\
0.73485151151951	0.001678466796875	\\
0.734895902694544	0.00177001953125	\\
0.734940293869579	0.001983642578125	\\
0.734984685044613	0.00189208984375	\\
0.735029076219648	0.00213623046875	\\
0.735073467394682	0.002716064453125	\\
0.735117858569716	0.002960205078125	\\
0.735162249744751	0.0029296875	\\
0.735206640919785	0.003143310546875	\\
0.73525103209482	0.002899169921875	\\
0.735295423269854	0.00274658203125	\\
0.735339814444888	0.002532958984375	\\
0.735384205619923	0.0023193359375	\\
0.735428596794957	0.002288818359375	\\
0.735472987969992	0.002777099609375	\\
0.735517379145026	0.00299072265625	\\
0.73556177032006	0.0029296875	\\
0.735606161495095	0.0028076171875	\\
0.735650552670129	0.003082275390625	\\
0.735694943845164	0.002899169921875	\\
0.735739335020198	0.002471923828125	\\
0.735783726195232	0.0029296875	\\
0.735828117370267	0.003021240234375	\\
0.735872508545301	0.002777099609375	\\
0.735916899720336	0.00286865234375	\\
0.73596129089537	0.00244140625	\\
0.736005682070404	0.0023193359375	\\
0.736050073245439	0.002410888671875	\\
0.736094464420473	0.00225830078125	\\
0.736138855595508	0.001983642578125	\\
0.736183246770542	0.001953125	\\
0.736227637945576	0.00201416015625	\\
0.736272029120611	0.001617431640625	\\
0.736316420295645	0.00152587890625	\\
0.73636081147068	0.00128173828125	\\
0.736405202645714	0.001556396484375	\\
0.736449593820748	0.00164794921875	\\
0.736493984995783	0.001220703125	\\
0.736538376170817	0.001373291015625	\\
0.736582767345852	0.001617431640625	\\
0.736627158520886	0.00164794921875	\\
0.73667154969592	0.001617431640625	\\
0.736715940870955	0.001617431640625	\\
0.736760332045989	0.001922607421875	\\
0.736804723221024	0.00238037109375	\\
0.736849114396058	0.002105712890625	\\
0.736893505571092	0.002105712890625	\\
0.736937896746127	0.00250244140625	\\
0.736982287921161	0.002532958984375	\\
0.737026679096196	0.00323486328125	\\
0.73707107027123	0.003509521484375	\\
0.737115461446265	0.003265380859375	\\
0.737159852621299	0.003265380859375	\\
0.737204243796333	0.003509521484375	\\
0.737248634971368	0.003448486328125	\\
0.737293026146402	0.00372314453125	\\
0.737337417321437	0.004425048828125	\\
0.737381808496471	0.004425048828125	\\
0.737426199671505	0.004150390625	\\
0.73747059084654	0.004638671875	\\
0.737514982021574	0.00457763671875	\\
0.737559373196608	0.00445556640625	\\
0.737603764371643	0.004486083984375	\\
0.737648155546677	0.0045166015625	\\
0.737692546721712	0.0042724609375	\\
0.737736937896746	0.004302978515625	\\
0.737781329071781	0.00408935546875	\\
0.737825720246815	0.003662109375	\\
0.737870111421849	0.003570556640625	\\
0.737914502596884	0.003509521484375	\\
0.737958893771918	0.003082275390625	\\
0.738003284946953	0.002960205078125	\\
0.738047676121987	0.003173828125	\\
0.738092067297021	0.0030517578125	\\
0.738136458472056	0.00311279296875	\\
0.73818084964709	0.0029296875	\\
0.738225240822125	0.00238037109375	\\
0.738269631997159	0.002166748046875	\\
0.738314023172193	0.00152587890625	\\
0.738358414347228	0.001190185546875	\\
0.738402805522262	0.001312255859375	\\
0.738447196697297	0.000762939453125	\\
0.738491587872331	0.000762939453125	\\
0.738535979047365	0.00067138671875	\\
0.7385803702224	0.000457763671875	\\
0.738624761397434	0.000518798828125	\\
0.738669152572469	9.1552734375e-05	\\
0.738713543747503	0.000244140625	\\
0.738757934922537	0.00018310546875	\\
0.738802326097572	3.0517578125e-05	\\
0.738846717272606	0.0001220703125	\\
0.738891108447641	-0.000152587890625	\\
0.738935499622675	-0.000213623046875	\\
0.738979890797709	-0.000152587890625	\\
0.739024281972744	-0.000579833984375	\\
0.739068673147778	-0.000885009765625	\\
0.739113064322813	-0.000732421875	\\
0.739157455497847	-0.0006103515625	\\
0.739201846672881	-0.000579833984375	\\
0.739246237847916	-0.000518798828125	\\
0.73929062902295	-0.000396728515625	\\
0.739335020197985	-0.0006103515625	\\
0.739379411373019	-0.000946044921875	\\
0.739423802548053	-0.0003662109375	\\
0.739468193723088	-0.0006103515625	\\
0.739512584898122	-0.00128173828125	\\
0.739556976073157	-0.00091552734375	\\
0.739601367248191	-0.000762939453125	\\
0.739645758423225	-0.00067138671875	\\
0.73969014959826	-0.000946044921875	\\
0.739734540773294	-0.001220703125	\\
0.739778931948329	-0.0008544921875	\\
0.739823323123363	-0.00103759765625	\\
0.739867714298397	-0.001220703125	\\
0.739912105473432	-0.00146484375	\\
0.739956496648466	-0.001800537109375	\\
0.740000887823501	-0.001678466796875	\\
0.740045278998535	-0.001678466796875	\\
0.74008967017357	-0.001068115234375	\\
0.740134061348604	-0.0008544921875	\\
0.740178452523638	-0.001434326171875	\\
0.740222843698673	-0.001251220703125	\\
0.740267234873707	-0.000946044921875	\\
0.740311626048741	-0.00079345703125	\\
0.740356017223776	-0.0006103515625	\\
0.74040040839881	-0.000396728515625	\\
0.740444799573845	-0.000762939453125	\\
0.740489190748879	-0.001007080078125	\\
0.740533581923914	-0.000762939453125	\\
0.740577973098948	-0.000732421875	\\
0.740622364273982	-0.000579833984375	\\
0.740666755449017	-0.000518798828125	\\
0.740711146624051	-0.000885009765625	\\
0.740755537799086	-0.000701904296875	\\
0.74079992897412	-0.000579833984375	\\
0.740844320149154	-0.000457763671875	\\
0.740888711324189	-0.0001220703125	\\
0.740933102499223	3.0517578125e-05	\\
0.740977493674258	9.1552734375e-05	\\
0.741021884849292	-0.000396728515625	\\
0.741066276024326	-0.000152587890625	\\
0.741110667199361	-0.00042724609375	\\
0.741155058374395	-0.000732421875	\\
0.74119944954943	-0.000457763671875	\\
0.741243840724464	-0.0008544921875	\\
0.741288231899498	-0.000579833984375	\\
0.741332623074533	-0.000823974609375	\\
0.741377014249567	-0.001220703125	\\
0.741421405424602	-0.000823974609375	\\
0.741465796599636	-0.00079345703125	\\
0.74151018777467	-0.000946044921875	\\
0.741554578949705	-0.00146484375	\\
0.741598970124739	-0.001678466796875	\\
0.741643361299774	-0.001129150390625	\\
0.741687752474808	-0.001495361328125	\\
0.741732143649842	-0.00164794921875	\\
0.741776534824877	-0.0015869140625	\\
0.741820925999911	-0.001556396484375	\\
0.741865317174946	-0.001708984375	\\
0.74190970834998	-0.00189208984375	\\
0.741954099525014	-0.001434326171875	\\
0.741998490700049	-0.0015869140625	\\
0.742042881875083	-0.0010986328125	\\
0.742087273050118	-0.00048828125	\\
0.742131664225152	-0.001068115234375	\\
0.742176055400187	-0.00140380859375	\\
0.742220446575221	-0.001129150390625	\\
0.742264837750255	-0.001190185546875	\\
0.74230922892529	-0.00177001953125	\\
0.742353620100324	-0.002044677734375	\\
0.742398011275358	-0.00146484375	\\
0.742442402450393	-0.00152587890625	\\
0.742486793625427	-0.002105712890625	\\
0.742531184800462	-0.00140380859375	\\
0.742575575975496	-0.00128173828125	\\
0.74261996715053	-0.00177001953125	\\
0.742664358325565	-0.001983642578125	\\
0.742708749500599	-0.001678466796875	\\
0.742753140675634	-0.001708984375	\\
0.742797531850668	-0.002044677734375	\\
0.742841923025703	-0.001708984375	\\
0.742886314200737	-0.001922607421875	\\
0.742930705375771	-0.001678466796875	\\
0.742975096550806	-0.0013427734375	\\
0.74301948772584	-0.001861572265625	\\
0.743063878900875	-0.00177001953125	\\
0.743108270075909	-0.001617431640625	\\
0.743152661250943	-0.00152587890625	\\
0.743197052425978	-0.001556396484375	\\
0.743241443601012	-0.001373291015625	\\
0.743285834776046	-0.001617431640625	\\
0.743330225951081	-0.002349853515625	\\
0.743374617126115	-0.002105712890625	\\
0.74341900830115	-0.001800537109375	\\
0.743463399476184	-0.00164794921875	\\
0.743507790651219	-0.001373291015625	\\
0.743552181826253	-0.001708984375	\\
0.743596573001287	-0.001800537109375	\\
0.743640964176322	-0.001678466796875	\\
0.743685355351356	-0.0020751953125	\\
0.743729746526391	-0.00189208984375	\\
0.743774137701425	-0.001708984375	\\
0.743818528876459	-0.0015869140625	\\
0.743862920051494	-0.001861572265625	\\
0.743907311226528	-0.001983642578125	\\
0.743951702401563	-0.001983642578125	\\
0.743996093576597	-0.00244140625	\\
0.744040484751631	-0.002227783203125	\\
0.744084875926666	-0.002044677734375	\\
0.7441292671017	-0.00213623046875	\\
0.744173658276735	-0.00225830078125	\\
0.744218049451769	-0.002532958984375	\\
0.744262440626803	-0.001953125	\\
0.744306831801838	-0.001800537109375	\\
0.744351222976872	-0.00250244140625	\\
0.744395614151907	-0.001953125	\\
0.744440005326941	-0.001861572265625	\\
0.744484396501975	-0.002349853515625	\\
0.74452878767701	-0.00201416015625	\\
0.744573178852044	-0.002410888671875	\\
0.744617570027079	-0.00201416015625	\\
0.744661961202113	-0.001708984375	\\
0.744706352377147	-0.001556396484375	\\
0.744750743552182	-0.00152587890625	\\
0.744795134727216	-0.001007080078125	\\
0.744839525902251	-0.00103759765625	\\
0.744883917077285	-0.0015869140625	\\
0.744928308252319	-0.0013427734375	\\
0.744972699427354	-0.001373291015625	\\
0.745017090602388	-0.00146484375	\\
0.745061481777423	-0.001220703125	\\
0.745105872952457	-0.001220703125	\\
0.745150264127491	-0.001495361328125	\\
0.745194655302526	-0.00128173828125	\\
0.74523904647756	-0.00115966796875	\\
0.745283437652595	-0.0010986328125	\\
0.745327828827629	-0.00091552734375	\\
0.745372220002663	-0.000457763671875	\\
0.745416611177698	-0.00042724609375	\\
0.745461002352732	-0.00042724609375	\\
0.745505393527767	0.0003662109375	\\
0.745549784702801	-6.103515625e-05	\\
0.745594175877836	-0.000762939453125	\\
0.74563856705287	-0.000457763671875	\\
0.745682958227904	-0.000457763671875	\\
0.745727349402939	-0.00067138671875	\\
0.745771740577973	-0.00048828125	\\
0.745816131753008	-0.00067138671875	\\
0.745860522928042	-0.00054931640625	\\
0.745904914103076	-0.0003662109375	\\
0.745949305278111	-0.00054931640625	\\
0.745993696453145	-0.001312255859375	\\
0.746038087628179	-0.00140380859375	\\
0.746082478803214	-0.000762939453125	\\
0.746126869978248	-0.001251220703125	\\
0.746171261153283	-0.001556396484375	\\
0.746215652328317	-0.001495361328125	\\
0.746260043503352	-0.00152587890625	\\
0.746304434678386	-0.001678466796875	\\
0.74634882585342	-0.001953125	\\
0.746393217028455	-0.002044677734375	\\
0.746437608203489	-0.001708984375	\\
0.746481999378524	-0.001922607421875	\\
0.746526390553558	-0.00244140625	\\
0.746570781728592	-0.002471923828125	\\
0.746615172903627	-0.002655029296875	\\
0.746659564078661	-0.002777099609375	\\
0.746703955253696	-0.00274658203125	\\
0.74674834642873	-0.002685546875	\\
0.746792737603764	-0.002838134765625	\\
0.746837128778799	-0.0030517578125	\\
0.746881519953833	-0.0029296875	\\
0.746925911128868	-0.002716064453125	\\
0.746970302303902	-0.0030517578125	\\
0.747014693478936	-0.002777099609375	\\
0.747059084653971	-0.002716064453125	\\
0.747103475829005	-0.002655029296875	\\
0.74714786700404	-0.002960205078125	\\
0.747192258179074	-0.00341796875	\\
0.747236649354108	-0.002960205078125	\\
0.747281040529143	-0.003387451171875	\\
0.747325431704177	-0.003387451171875	\\
0.747369822879212	-0.0029296875	\\
0.747414214054246	-0.003143310546875	\\
0.74745860522928	-0.003387451171875	\\
0.747502996404315	-0.003082275390625	\\
0.747547387579349	-0.002960205078125	\\
0.747591778754384	-0.00286865234375	\\
0.747636169929418	-0.002655029296875	\\
0.747680561104452	-0.00299072265625	\\
0.747724952279487	-0.003265380859375	\\
0.747769343454521	-0.003387451171875	\\
0.747813734629556	-0.003265380859375	\\
0.74785812580459	-0.00335693359375	\\
0.747902516979624	-0.003387451171875	\\
0.747946908154659	-0.003448486328125	\\
0.747991299329693	-0.00311279296875	\\
0.748035690504728	-0.0030517578125	\\
0.748080081679762	-0.0029296875	\\
0.748124472854796	-0.003082275390625	\\
0.748168864029831	-0.003082275390625	\\
0.748213255204865	-0.0029296875	\\
0.7482576463799	-0.00323486328125	\\
0.748302037554934	-0.003265380859375	\\
0.748346428729968	-0.003326416015625	\\
0.748390819905003	-0.003265380859375	\\
0.748435211080037	-0.003448486328125	\\
0.748479602255072	-0.00384521484375	\\
0.748523993430106	-0.003570556640625	\\
0.748568384605141	-0.0037841796875	\\
0.748612775780175	-0.003662109375	\\
0.748657166955209	-0.003326416015625	\\
0.748701558130244	-0.0032958984375	\\
0.748745949305278	-0.00311279296875	\\
0.748790340480313	-0.003173828125	\\
0.748834731655347	-0.003143310546875	\\
0.748879122830381	-0.0029296875	\\
0.748923514005416	-0.002685546875	\\
0.74896790518045	-0.002716064453125	\\
0.749012296355485	-0.003082275390625	\\
0.749056687530519	-0.002716064453125	\\
0.749101078705553	-0.002960205078125	\\
0.749145469880588	-0.003387451171875	\\
0.749189861055622	-0.0035400390625	\\
0.749234252230657	-0.003448486328125	\\
0.749278643405691	-0.003570556640625	\\
0.749323034580725	-0.003387451171875	\\
0.74936742575576	-0.003143310546875	\\
0.749411816930794	-0.00238037109375	\\
0.749456208105829	-0.002655029296875	\\
0.749500599280863	-0.0030517578125	\\
0.749544990455897	-0.002838134765625	\\
0.749589381630932	-0.002899169921875	\\
0.749633772805966	-0.002838134765625	\\
0.749678163981001	-0.00323486328125	\\
0.749722555156035	-0.00311279296875	\\
0.749766946331069	-0.002349853515625	\\
0.749811337506104	-0.002899169921875	\\
0.749855728681138	-0.003387451171875	\\
0.749900119856173	-0.0030517578125	\\
0.749944511031207	-0.00335693359375	\\
0.749988902206241	-0.003204345703125	\\
0.750033293381276	-0.00311279296875	\\
0.75007768455631	-0.00299072265625	\\
0.750122075731345	-0.003448486328125	\\
0.750166466906379	-0.003387451171875	\\
0.750210858081413	-0.002899169921875	\\
0.750255249256448	-0.002838134765625	\\
0.750299640431482	-0.00262451171875	\\
0.750344031606517	-0.00250244140625	\\
0.750388422781551	-0.00250244140625	\\
0.750432813956585	-0.002471923828125	\\
0.75047720513162	-0.002349853515625	\\
0.750521596306654	-0.0020751953125	\\
0.750565987481689	-0.00189208984375	\\
0.750610378656723	-0.0013427734375	\\
0.750654769831758	-0.001861572265625	\\
0.750699161006792	-0.001983642578125	\\
0.750743552181826	-0.0015869140625	\\
0.750787943356861	-0.001953125	\\
0.750832334531895	-0.001434326171875	\\
0.750876725706929	-0.00115966796875	\\
0.750921116881964	-0.001678466796875	\\
0.750965508056998	-0.001861572265625	\\
0.751009899232033	-0.001495361328125	\\
0.751054290407067	-0.000946044921875	\\
0.751098681582101	-0.0009765625	\\
0.751143072757136	-0.001312255859375	\\
0.75118746393217	-0.001220703125	\\
0.751231855107205	-0.000823974609375	\\
0.751276246282239	-0.001129150390625	\\
0.751320637457274	-0.0009765625	\\
0.751365028632308	-0.001251220703125	\\
0.751409419807342	-0.001434326171875	\\
0.751453810982377	-0.001312255859375	\\
0.751498202157411	-0.001190185546875	\\
0.751542593332446	-0.00079345703125	\\
0.75158698450748	-0.00103759765625	\\
0.751631375682514	-0.0013427734375	\\
0.751675766857549	-0.00115966796875	\\
0.751720158032583	-0.00067138671875	\\
0.751764549207617	-0.0008544921875	\\
0.751808940382652	-0.00067138671875	\\
0.751853331557686	-0.00054931640625	\\
0.751897722732721	-0.001312255859375	\\
0.751942113907755	-0.000579833984375	\\
0.75198650508279	-0.00054931640625	\\
0.752030896257824	-0.000732421875	\\
0.752075287432858	-0.000885009765625	\\
0.752119678607893	-0.00128173828125	\\
0.752164069782927	-0.00115966796875	\\
0.752208460957962	-0.001190185546875	\\
0.752252852132996	-0.001190185546875	\\
0.75229724330803	-0.001251220703125	\\
0.752341634483065	-0.000823974609375	\\
0.752386025658099	-0.00067138671875	\\
0.752430416833134	-0.00079345703125	\\
0.752474808008168	-0.000823974609375	\\
0.752519199183202	-0.000701904296875	\\
0.752563590358237	-0.000732421875	\\
0.752607981533271	-0.001007080078125	\\
0.752652372708306	-0.000701904296875	\\
0.75269676388334	-0.00042724609375	\\
0.752741155058374	-0.000579833984375	\\
0.752785546233409	-0.000335693359375	\\
0.752829937408443	-0.00042724609375	\\
0.752874328583478	-0.00018310546875	\\
0.752918719758512	0.000396728515625	\\
0.752963110933546	-6.103515625e-05	\\
0.753007502108581	-6.103515625e-05	\\
0.753051893283615	0	\\
0.75309628445865	-0.0001220703125	\\
0.753140675633684	0.0001220703125	\\
0.753185066808718	0.00030517578125	\\
0.753229457983753	0.000213623046875	\\
0.753273849158787	0.00018310546875	\\
0.753318240333822	0.000396728515625	\\
0.753362631508856	0.000732421875	\\
0.75340702268389	0.000762939453125	\\
0.753451413858925	0.00048828125	\\
0.753495805033959	0.000335693359375	\\
0.753540196208994	0.000274658203125	\\
0.753584587384028	0.000823974609375	\\
0.753628978559062	0.0009765625	\\
0.753673369734097	0.00048828125	\\
0.753717760909131	0.0006103515625	\\
0.753762152084166	0.00079345703125	\\
0.7538065432592	0.001007080078125	\\
0.753850934434234	0.000885009765625	\\
0.753895325609269	0.00140380859375	\\
0.753939716784303	0.001556396484375	\\
0.753984107959338	0.00128173828125	\\
0.754028499134372	0.001312255859375	\\
0.754072890309407	0.00128173828125	\\
0.754117281484441	0.001220703125	\\
0.754161672659475	0.001312255859375	\\
0.75420606383451	0.001556396484375	\\
0.754250455009544	0.001739501953125	\\
0.754294846184579	0.001434326171875	\\
0.754339237359613	0.001220703125	\\
0.754383628534647	0.00140380859375	\\
0.754428019709682	0.001617431640625	\\
0.754472410884716	0.001617431640625	\\
0.75451680205975	0.0015869140625	\\
0.754561193234785	0.001495361328125	\\
0.754605584409819	0.001434326171875	\\
0.754649975584854	0.001312255859375	\\
0.754694366759888	0.001220703125	\\
0.754738757934923	0.001251220703125	\\
0.754783149109957	0.000823974609375	\\
0.754827540284991	0.000396728515625	\\
0.754871931460026	0.000640869140625	\\
0.75491632263506	0.001251220703125	\\
0.754960713810095	0.00115966796875	\\
0.755005104985129	0.0008544921875	\\
0.755049496160163	0.000762939453125	\\
0.755093887335198	0.000701904296875	\\
0.755138278510232	0.000762939453125	\\
0.755182669685267	0.0006103515625	\\
0.755227060860301	0.000244140625	\\
0.755271452035335	0.000701904296875	\\
0.75531584321037	0.00054931640625	\\
0.755360234385404	0.00030517578125	\\
0.755404625560439	0.000518798828125	\\
0.755449016735473	0.0003662109375	\\
0.755493407910507	0.000518798828125	\\
0.755537799085542	0.00048828125	\\
0.755582190260576	0.00018310546875	\\
0.755626581435611	0.000274658203125	\\
0.755670972610645	9.1552734375e-05	\\
0.755715363785679	3.0517578125e-05	\\
0.755759754960714	-0.000274658203125	\\
0.755804146135748	-0.000244140625	\\
0.755848537310783	0.000274658203125	\\
0.755892928485817	0.000579833984375	\\
0.755937319660851	0.000335693359375	\\
0.755981710835886	3.0517578125e-05	\\
0.75602610201092	-0.0001220703125	\\
0.756070493185955	-0.000518798828125	\\
0.756114884360989	-0.00018310546875	\\
0.756159275536023	-0.000274658203125	\\
0.756203666711058	-0.00042724609375	\\
0.756248057886092	-0.00018310546875	\\
0.756292449061127	-0.000396728515625	\\
0.756336840236161	-0.00048828125	\\
0.756381231411196	-9.1552734375e-05	\\
0.75642562258623	-0.000457763671875	\\
0.756470013761264	-0.000885009765625	\\
0.756514404936299	-0.000885009765625	\\
0.756558796111333	-0.001007080078125	\\
0.756603187286367	-0.000885009765625	\\
0.756647578461402	-0.000885009765625	\\
0.756691969636436	-0.00048828125	\\
0.756736360811471	-0.000213623046875	\\
0.756780751986505	-0.00067138671875	\\
0.756825143161539	-0.000732421875	\\
0.756869534336574	-0.000579833984375	\\
0.756913925511608	-0.0008544921875	\\
0.756958316686643	-0.0008544921875	\\
0.757002707861677	-0.0006103515625	\\
0.757047099036712	-0.000457763671875	\\
0.757091490211746	-0.00048828125	\\
0.75713588138678	-0.000732421875	\\
0.757180272561815	-0.000396728515625	\\
0.757224663736849	-0.000762939453125	\\
0.757269054911884	-0.001495361328125	\\
0.757313446086918	-0.0010986328125	\\
0.757357837261952	-0.000885009765625	\\
0.757402228436987	-0.001373291015625	\\
0.757446619612021	-0.00140380859375	\\
0.757491010787056	-0.0006103515625	\\
0.75753540196209	-0.0008544921875	\\
0.757579793137124	-0.001251220703125	\\
0.757624184312159	-0.001220703125	\\
0.757668575487193	-0.001190185546875	\\
0.757712966662228	-0.00103759765625	\\
0.757757357837262	-0.000823974609375	\\
0.757801749012296	-0.000701904296875	\\
0.757846140187331	-0.00054931640625	\\
0.757890531362365	-0.000213623046875	\\
0.7579349225374	-0.000396728515625	\\
0.757979313712434	-0.00030517578125	\\
0.758023704887468	-0.000244140625	\\
0.758068096062503	-0.0003662109375	\\
0.758112487237537	0.000152587890625	\\
0.758156878412572	0.000152587890625	\\
0.758201269587606	-3.0517578125e-05	\\
0.75824566076264	-0.000244140625	\\
0.758290051937675	-0.000152587890625	\\
0.758334443112709	0.000335693359375	\\
0.758378834287744	0.000518798828125	\\
0.758423225462778	0.0003662109375	\\
0.758467616637812	0.000396728515625	\\
0.758512007812847	0.000701904296875	\\
0.758556398987881	0.00042724609375	\\
0.758600790162916	0.00067138671875	\\
0.75864518133795	0.000732421875	\\
0.758689572512984	6.103515625e-05	\\
0.758733963688019	0.000152587890625	\\
0.758778354863053	0.00054931640625	\\
0.758822746038088	0.0003662109375	\\
0.758867137213122	0.00030517578125	\\
0.758911528388156	0.00042724609375	\\
0.758955919563191	0.00054931640625	\\
0.759000310738225	0.000579833984375	\\
0.75904470191326	0.000213623046875	\\
0.759089093088294	6.103515625e-05	\\
0.759133484263329	0.0001220703125	\\
0.759177875438363	0.000457763671875	\\
0.759222266613397	0.00079345703125	\\
0.759266657788432	0.00042724609375	\\
0.759311048963466	0	\\
0.7593554401385	9.1552734375e-05	\\
0.759399831313535	0.00018310546875	\\
0.759444222488569	0.000244140625	\\
0.759488613663604	0.000244140625	\\
0.759533004838638	0.000213623046875	\\
0.759577396013672	9.1552734375e-05	\\
0.759621787188707	0.000152587890625	\\
0.759666178363741	0.000274658203125	\\
0.759710569538776	0.000244140625	\\
0.75975496071381	0.000335693359375	\\
0.759799351888845	0.0003662109375	\\
0.759843743063879	0.00042724609375	\\
0.759888134238913	0.000579833984375	\\
0.759932525413948	0.000885009765625	\\
0.759976916588982	0.0008544921875	\\
0.760021307764017	0.001007080078125	\\
0.760065698939051	0.00103759765625	\\
0.760110090114085	0.000701904296875	\\
0.76015448128912	0.0008544921875	\\
0.760198872464154	0.000518798828125	\\
0.760243263639188	0.00042724609375	\\
0.760287654814223	0.001007080078125	\\
0.760332045989257	0.001068115234375	\\
0.760376437164292	0.001068115234375	\\
0.760420828339326	0.001251220703125	\\
0.760465219514361	0.00115966796875	\\
0.760509610689395	0.001373291015625	\\
0.760554001864429	0.001312255859375	\\
0.760598393039464	0.001220703125	\\
0.760642784214498	0.00128173828125	\\
0.760687175389533	0.001220703125	\\
0.760731566564567	0.00103759765625	\\
0.760775957739601	0.00103759765625	\\
0.760820348914636	0.001068115234375	\\
0.76086474008967	0.0010986328125	\\
0.760909131264705	0.001129150390625	\\
0.760953522439739	0.000762939453125	\\
0.760997913614773	0.000946044921875	\\
0.761042304789808	0.00079345703125	\\
0.761086695964842	0.000732421875	\\
0.761131087139877	0.000518798828125	\\
0.761175478314911	0.0006103515625	\\
0.761219869489945	0.000213623046875	\\
0.76126426066498	-0.000274658203125	\\
0.761308651840014	0	\\
0.761353043015049	0.00030517578125	\\
0.761397434190083	0.000244140625	\\
0.761441825365117	0.00054931640625	\\
0.761486216540152	0.00054931640625	\\
0.761530607715186	0.00048828125	\\
0.761574998890221	0.00067138671875	\\
0.761619390065255	0.000946044921875	\\
0.761663781240289	0.001007080078125	\\
0.761708172415324	0.0009765625	\\
0.761752563590358	0.0008544921875	\\
0.761796954765393	0.000701904296875	\\
0.761841345940427	0.00054931640625	\\
0.761885737115461	0.000732421875	\\
0.761930128290496	0.000579833984375	\\
0.76197451946553	0.0006103515625	\\
0.762018910640565	0.001068115234375	\\
0.762063301815599	0.00115966796875	\\
0.762107692990633	0.001495361328125	\\
0.762152084165668	0.00146484375	\\
0.762196475340702	0.0010986328125	\\
0.762240866515737	0.00091552734375	\\
0.762285257690771	0.001129150390625	\\
0.762329648865805	0.0009765625	\\
0.76237404004084	0.00091552734375	\\
0.762418431215874	0.00054931640625	\\
0.762462822390909	0.00048828125	\\
0.762507213565943	0.000823974609375	\\
0.762551604740978	0.0006103515625	\\
0.762595995916012	0.00067138671875	\\
0.762640387091046	0.00048828125	\\
0.762684778266081	0.000274658203125	\\
0.762729169441115	9.1552734375e-05	\\
0.76277356061615	3.0517578125e-05	\\
0.762817951791184	0.0003662109375	\\
0.762862342966218	-0.00030517578125	\\
0.762906734141253	-0.000396728515625	\\
0.762951125316287	0.0001220703125	\\
0.762995516491322	6.103515625e-05	\\
0.763039907666356	-0.000457763671875	\\
0.76308429884139	-0.000457763671875	\\
0.763128690016425	0	\\
0.763173081191459	-3.0517578125e-05	\\
0.763217472366494	-6.103515625e-05	\\
0.763261863541528	0.0001220703125	\\
0.763306254716562	0.00018310546875	\\
0.763350645891597	0	\\
0.763395037066631	6.103515625e-05	\\
0.763439428241666	-6.103515625e-05	\\
0.7634838194167	0.00018310546875	\\
0.763528210591734	-6.103515625e-05	\\
0.763572601766769	-0.000152587890625	\\
0.763616992941803	-6.103515625e-05	\\
0.763661384116838	-0.000518798828125	\\
0.763705775291872	-0.000152587890625	\\
0.763750166466906	-0.000274658203125	\\
0.763794557641941	-0.0006103515625	\\
0.763838948816975	-0.000885009765625	\\
0.76388333999201	-0.001129150390625	\\
0.763927731167044	-0.00091552734375	\\
0.763972122342078	-0.000946044921875	\\
0.764016513517113	-0.00128173828125	\\
0.764060904692147	-0.001190185546875	\\
0.764105295867182	-0.001129150390625	\\
0.764149687042216	-0.0010986328125	\\
0.76419407821725	-0.00091552734375	\\
0.764238469392285	-0.000701904296875	\\
0.764282860567319	-0.00079345703125	\\
0.764327251742354	-0.00091552734375	\\
0.764371642917388	-0.001129150390625	\\
0.764416034092422	-0.0013427734375	\\
0.764460425267457	-0.001800537109375	\\
0.764504816442491	-0.001983642578125	\\
0.764549207617526	-0.001800537109375	\\
0.76459359879256	-0.001953125	\\
0.764637989967594	-0.00164794921875	\\
0.764682381142629	-0.001739501953125	\\
0.764726772317663	-0.001953125	\\
0.764771163492698	-0.001861572265625	\\
0.764815554667732	-0.001800537109375	\\
0.764859945842767	-0.0015869140625	\\
0.764904337017801	-0.00164794921875	\\
0.764948728192835	-0.0018310546875	\\
0.76499311936787	-0.002197265625	\\
0.765037510542904	-0.0025634765625	\\
0.765081901717938	-0.002227783203125	\\
0.765126292892973	-0.002471923828125	\\
0.765170684068007	-0.002288818359375	\\
0.765215075243042	-0.00238037109375	\\
0.765259466418076	-0.002227783203125	\\
0.76530385759311	-0.002288818359375	\\
0.765348248768145	-0.002532958984375	\\
0.765392639943179	-0.002593994140625	\\
0.765437031118214	-0.0025634765625	\\
0.765481422293248	-0.002349853515625	\\
0.765525813468283	-0.0023193359375	\\
0.765570204643317	-0.0025634765625	\\
0.765614595818351	-0.002685546875	\\
0.765658986993386	-0.002471923828125	\\
0.76570337816842	-0.002288818359375	\\
0.765747769343455	-0.002166748046875	\\
0.765792160518489	-0.002044677734375	\\
0.765836551693523	-0.0020751953125	\\
0.765880942868558	-0.002166748046875	\\
0.765925334043592	-0.001922607421875	\\
0.765969725218627	-0.002105712890625	\\
0.766014116393661	-0.0020751953125	\\
0.766058507568695	-0.00201416015625	\\
0.76610289874373	-0.0023193359375	\\
0.766147289918764	-0.002166748046875	\\
0.766191681093799	-0.00146484375	\\
0.766236072268833	-0.001556396484375	\\
0.766280463443867	-0.002227783203125	\\
0.766324854618902	-0.001678466796875	\\
0.766369245793936	-0.0015869140625	\\
0.766413636968971	-0.0015869140625	\\
0.766458028144005	-0.00164794921875	\\
0.766502419319039	-0.00164794921875	\\
0.766546810494074	-0.0013427734375	\\
0.766591201669108	-0.00152587890625	\\
0.766635592844143	-0.001739501953125	\\
0.766679984019177	-0.001800537109375	\\
0.766724375194211	-0.00140380859375	\\
0.766768766369246	-0.001617431640625	\\
0.76681315754428	-0.0018310546875	\\
0.766857548719315	-0.001800537109375	\\
0.766901939894349	-0.001922607421875	\\
0.766946331069383	-0.00177001953125	\\
0.766990722244418	-0.001678466796875	\\
0.767035113419452	-0.002288818359375	\\
0.767079504594487	-0.00286865234375	\\
0.767123895769521	-0.002471923828125	\\
0.767168286944555	-0.001861572265625	\\
0.76721267811959	-0.001708984375	\\
0.767257069294624	-0.001708984375	\\
0.767301460469659	-0.00189208984375	\\
0.767345851644693	-0.001678466796875	\\
0.767390242819727	-0.00146484375	\\
0.767434633994762	-0.001220703125	\\
0.767479025169796	-0.001495361328125	\\
0.767523416344831	-0.001556396484375	\\
0.767567807519865	-0.001434326171875	\\
0.767612198694899	-0.001434326171875	\\
0.767656589869934	-0.00115966796875	\\
0.767700981044968	-0.001312255859375	\\
0.767745372220003	-0.00140380859375	\\
0.767789763395037	-0.001251220703125	\\
0.767834154570071	-0.000885009765625	\\
0.767878545745106	-0.00079345703125	\\
0.76792293692014	-0.00103759765625	\\
0.767967328095175	-0.0006103515625	\\
0.768011719270209	-0.000274658203125	\\
0.768056110445243	-0.000152587890625	\\
0.768100501620278	-9.1552734375e-05	\\
0.768144892795312	-9.1552734375e-05	\\
0.768189283970347	-0.00054931640625	\\
0.768233675145381	-0.00054931640625	\\
0.768278066320416	-0.000244140625	\\
0.76832245749545	-6.103515625e-05	\\
0.768366848670484	-0.000244140625	\\
0.768411239845519	-0.00042724609375	\\
0.768455631020553	-0.000518798828125	\\
0.768500022195588	-0.000457763671875	\\
0.768544413370622	-0.000396728515625	\\
0.768588804545656	-0.000152587890625	\\
0.768633195720691	-0.000335693359375	\\
0.768677586895725	-0.00042724609375	\\
0.768721978070759	-0.00054931640625	\\
0.768766369245794	-0.001007080078125	\\
0.768810760420828	-0.0009765625	\\
0.768855151595863	-0.0009765625	\\
0.768899542770897	-0.001251220703125	\\
0.768943933945932	-0.001312255859375	\\
0.768988325120966	-0.00091552734375	\\
0.769032716296	-0.00091552734375	\\
0.769077107471035	-0.001190185546875	\\
0.769121498646069	-0.00146484375	\\
0.769165889821104	-0.0013427734375	\\
0.769210280996138	-0.001007080078125	\\
0.769254672171172	-0.001434326171875	\\
0.769299063346207	-0.001495361328125	\\
0.769343454521241	-0.001190185546875	\\
0.769387845696276	-0.00177001953125	\\
0.76943223687131	-0.001678466796875	\\
0.769476628046344	-0.00128173828125	\\
0.769521019221379	-0.00115966796875	\\
0.769565410396413	-0.0009765625	\\
0.769609801571448	-0.0010986328125	\\
0.769654192746482	-0.001129150390625	\\
0.769698583921516	-0.000885009765625	\\
0.769742975096551	-0.00048828125	\\
0.769787366271585	-0.0001220703125	\\
0.76983175744662	0	\\
0.769876148621654	-0.0001220703125	\\
0.769920539796688	0.000152587890625	\\
0.769964930971723	0.00054931640625	\\
0.770009322146757	0.000701904296875	\\
0.770053713321792	0.000823974609375	\\
0.770098104496826	0.00091552734375	\\
0.77014249567186	0.001129150390625	\\
0.770186886846895	0.001739501953125	\\
0.770231278021929	0.00177001953125	\\
0.770275669196964	0.001434326171875	\\
0.770320060371998	0.001617431640625	\\
0.770364451547032	0.001495361328125	\\
0.770408842722067	0.00152587890625	\\
0.770453233897101	0.001678466796875	\\
0.770497625072136	0.0010986328125	\\
0.77054201624717	0.0010986328125	\\
0.770586407422204	0.00103759765625	\\
0.770630798597239	0.00115966796875	\\
0.770675189772273	0.000732421875	\\
0.770719580947308	0.00018310546875	\\
0.770763972122342	0.000640869140625	\\
0.770808363297376	0.0008544921875	\\
0.770852754472411	0.0003662109375	\\
0.770897145647445	0	\\
0.77094153682248	0.00030517578125	\\
0.770985927997514	-0.000244140625	\\
0.771030319172549	-0.00042724609375	\\
0.771074710347583	-0.0003662109375	\\
0.771119101522617	-0.0008544921875	\\
0.771163492697652	-0.000640869140625	\\
0.771207883872686	-0.00067138671875	\\
0.771252275047721	-0.000823974609375	\\
0.771296666222755	-0.00103759765625	\\
0.771341057397789	-0.001190185546875	\\
0.771385448572824	-0.00146484375	\\
0.771429839747858	-0.001190185546875	\\
0.771474230922893	-0.001220703125	\\
0.771518622097927	-0.001129150390625	\\
0.771563013272961	-0.001251220703125	\\
0.771607404447996	-0.001495361328125	\\
0.77165179562303	-0.00115966796875	\\
0.771696186798065	-0.0010986328125	\\
0.771740577973099	-0.00115966796875	\\
0.771784969148133	-0.00079345703125	\\
0.771829360323168	-0.000732421875	\\
0.771873751498202	-0.0006103515625	\\
0.771918142673237	6.103515625e-05	\\
0.771962533848271	0.00018310546875	\\
0.772006925023305	0.000213623046875	\\
0.77205131619834	0.00048828125	\\
0.772095707373374	0.00067138671875	\\
0.772140098548409	0.00067138671875	\\
0.772184489723443	0.0006103515625	\\
0.772228880898477	0.001007080078125	\\
0.772273272073512	0.0009765625	\\
0.772317663248546	0.000640869140625	\\
0.772362054423581	0.0009765625	\\
0.772406445598615	0.00067138671875	\\
0.772450836773649	0.000640869140625	\\
0.772495227948684	0.00091552734375	\\
0.772539619123718	0.000457763671875	\\
0.772584010298753	0.00067138671875	\\
0.772628401473787	0.00067138671875	\\
0.772672792648821	0.0006103515625	\\
0.772717183823856	0.0008544921875	\\
0.77276157499889	0.000579833984375	\\
0.772805966173925	0.0006103515625	\\
0.772850357348959	0.000732421875	\\
0.772894748523993	0.000213623046875	\\
0.772939139699028	0.000213623046875	\\
0.772983530874062	0.00030517578125	\\
0.773027922049097	0.00030517578125	\\
0.773072313224131	0.00030517578125	\\
0.773116704399165	0.0003662109375	\\
0.7731610955742	0.000152587890625	\\
0.773205486749234	-0.0001220703125	\\
0.773249877924269	-0.00018310546875	\\
0.773294269099303	-0.000152587890625	\\
0.773338660274338	-0.000152587890625	\\
0.773383051449372	-0.00048828125	\\
0.773427442624406	-0.0006103515625	\\
0.773471833799441	-0.0006103515625	\\
0.773516224974475	-0.000579833984375	\\
0.773560616149509	-0.000518798828125	\\
0.773605007324544	-0.0006103515625	\\
0.773649398499578	-0.0006103515625	\\
0.773693789674613	-0.00030517578125	\\
0.773738180849647	3.0517578125e-05	\\
0.773782572024681	3.0517578125e-05	\\
0.773826963199716	-0.000244140625	\\
0.77387135437475	-3.0517578125e-05	\\
0.773915745549785	0.00054931640625	\\
0.773960136724819	0.000152587890625	\\
0.774004527899854	0.00067138671875	\\
0.774048919074888	0.00128173828125	\\
0.774093310249922	0.001068115234375	\\
0.774137701424957	0.000946044921875	\\
0.774182092599991	0.00054931640625	\\
0.774226483775026	0.00079345703125	\\
0.77427087495006	0.000885009765625	\\
0.774315266125094	0.001190185546875	\\
0.774359657300129	0.0009765625	\\
0.774404048475163	0.0006103515625	\\
0.774448439650198	0.001190185546875	\\
0.774492830825232	0.001556396484375	\\
0.774537222000266	0.0013427734375	\\
0.774581613175301	0.001617431640625	\\
0.774626004350335	0.00152587890625	\\
0.77467039552537	0.00091552734375	\\
0.774714786700404	0.001434326171875	\\
0.774759177875438	0.001251220703125	\\
0.774803569050473	0.0010986328125	\\
0.774847960225507	0.00128173828125	\\
0.774892351400542	0.001007080078125	\\
0.774936742575576	0.001312255859375	\\
0.77498113375061	0.001708984375	\\
0.775025524925645	0.00152587890625	\\
0.775069916100679	0.001434326171875	\\
0.775114307275714	0.00140380859375	\\
0.775158698450748	0.00146484375	\\
0.775203089625782	0.001678466796875	\\
0.775247480800817	0.00177001953125	\\
0.775291871975851	0.00177001953125	\\
0.775336263150886	0.001678466796875	\\
0.77538065432592	0.001556396484375	\\
0.775425045500954	0.00164794921875	\\
0.775469436675989	0.0018310546875	\\
0.775513827851023	0.00225830078125	\\
0.775558219026058	0.0020751953125	\\
0.775602610201092	0.0020751953125	\\
0.775647001376126	0.00238037109375	\\
0.775691392551161	0.0023193359375	\\
0.775735783726195	0.00274658203125	\\
0.77578017490123	0.003021240234375	\\
0.775824566076264	0.003082275390625	\\
0.775868957251298	0.003082275390625	\\
0.775913348426333	0.003021240234375	\\
0.775957739601367	0.003448486328125	\\
0.776002130776402	0.003265380859375	\\
0.776046521951436	0.002960205078125	\\
0.77609091312647	0.003387451171875	\\
0.776135304301505	0.003509521484375	\\
0.776179695476539	0.00323486328125	\\
0.776224086651574	0.003448486328125	\\
0.776268477826608	0.0035400390625	\\
0.776312869001642	0.00360107421875	\\
0.776357260176677	0.00360107421875	\\
0.776401651351711	0.003082275390625	\\
0.776446042526746	0.00341796875	\\
0.77649043370178	0.003509521484375	\\
0.776534824876814	0.0035400390625	\\
0.776579216051849	0.00347900390625	\\
0.776623607226883	0.003204345703125	\\
0.776667998401918	0.00335693359375	\\
0.776712389576952	0.003021240234375	\\
0.776756780751987	0.002777099609375	\\
0.776801171927021	0.002960205078125	\\
0.776845563102055	0.003143310546875	\\
0.77688995427709	0.0025634765625	\\
0.776934345452124	0.00225830078125	\\
0.776978736627159	0.00213623046875	\\
0.777023127802193	0.001556396484375	\\
0.777067518977227	0.00164794921875	\\
0.777111910152262	0.001220703125	\\
0.777156301327296	0.00103759765625	\\
0.77720069250233	0.001373291015625	\\
0.777245083677365	0.001312255859375	\\
0.777289474852399	0.001434326171875	\\
0.777333866027434	0.00152587890625	\\
0.777378257202468	0.00146484375	\\
0.777422648377503	0.001861572265625	\\
0.777467039552537	0.001922607421875	\\
0.777511430727571	0.001861572265625	\\
0.777555821902606	0.001739501953125	\\
0.77760021307764	0.001678466796875	\\
0.777644604252675	0.001678466796875	\\
0.777688995427709	0.001983642578125	\\
0.777733386602743	0.0025634765625	\\
0.777777777777778	0.00225830078125	\\
0.777822168952812	0.002410888671875	\\
0.777866560127847	0.00244140625	\\
0.777910951302881	0.002197265625	\\
0.777955342477915	0.002349853515625	\\
0.77799973365295	0.002655029296875	\\
0.778044124827984	0.002349853515625	\\
0.778088516003019	0.0025634765625	\\
0.778132907178053	0.00323486328125	\\
0.778177298353087	0.003021240234375	\\
0.778221689528122	0.002899169921875	\\
0.778266080703156	0.00335693359375	\\
0.778310471878191	0.003448486328125	\\
0.778354863053225	0.003448486328125	\\
0.778399254228259	0.003448486328125	\\
0.778443645403294	0.003448486328125	\\
0.778488036578328	0.003753662109375	\\
0.778532427753363	0.003692626953125	\\
0.778576818928397	0.00360107421875	\\
0.778621210103431	0.003326416015625	\\
0.778665601278466	0.003021240234375	\\
0.7787099924535	0.00262451171875	\\
0.778754383628535	0.00262451171875	\\
0.778798774803569	0.002655029296875	\\
0.778843165978603	0.0028076171875	\\
0.778887557153638	0.002716064453125	\\
0.778931948328672	0.002288818359375	\\
0.778976339503707	0.001983642578125	\\
0.779020730678741	0.002044677734375	\\
0.779065121853776	0.001739501953125	\\
0.77910951302881	0.001495361328125	\\
0.779153904203844	0.001678466796875	\\
0.779198295378879	0.00146484375	\\
0.779242686553913	0.0013427734375	\\
0.779287077728947	0.00146484375	\\
0.779331468903982	0.00140380859375	\\
0.779375860079016	0.001617431640625	\\
0.779420251254051	0.00128173828125	\\
0.779464642429085	0.00079345703125	\\
0.779509033604119	0.001220703125	\\
0.779553424779154	0.00103759765625	\\
0.779597815954188	0.000885009765625	\\
0.779642207129223	0.001251220703125	\\
0.779686598304257	0.001129150390625	\\
0.779730989479292	0.001251220703125	\\
0.779775380654326	0.00164794921875	\\
0.77981977182936	0.001922607421875	\\
0.779864163004395	0.001983642578125	\\
0.779908554179429	0.0015869140625	\\
0.779952945354464	0.001953125	\\
0.779997336529498	0.002471923828125	\\
0.780041727704532	0.002288818359375	\\
0.780086118879567	0.002197265625	\\
0.780130510054601	0.0023193359375	\\
0.780174901229636	0.002349853515625	\\
0.78021929240467	0.002349853515625	\\
0.780263683579704	0.002532958984375	\\
0.780308074754739	0.002685546875	\\
0.780352465929773	0.003082275390625	\\
0.780396857104808	0.0030517578125	\\
0.780441248279842	0.002716064453125	\\
0.780485639454876	0.0028076171875	\\
0.780530030629911	0.00250244140625	\\
0.780574421804945	0.002532958984375	\\
0.78061881297998	0.002532958984375	\\
0.780663204155014	0.002044677734375	\\
0.780707595330048	0.001983642578125	\\
0.780751986505083	0.001708984375	\\
0.780796377680117	0.000946044921875	\\
0.780840768855152	0.001068115234375	\\
0.780885160030186	0.00115966796875	\\
0.78092955120522	0.001434326171875	\\
0.780973942380255	0.001556396484375	\\
0.781018333555289	0.001678466796875	\\
0.781062724730324	0.002166748046875	\\
0.781107115905358	0.00140380859375	\\
0.781151507080392	0.001739501953125	\\
0.781195898255427	0.0020751953125	\\
0.781240289430461	0.001922607421875	\\
0.781284680605496	0.0018310546875	\\
0.78132907178053	0.001434326171875	\\
0.781373462955564	0.001495361328125	\\
0.781417854130599	0.0015869140625	\\
0.781462245305633	0.001220703125	\\
0.781506636480668	0.001068115234375	\\
0.781551027655702	0.001739501953125	\\
0.781595418830736	0.001739501953125	\\
0.781639810005771	0.00146484375	\\
0.781684201180805	0.00164794921875	\\
0.78172859235584	0.001708984375	\\
0.781772983530874	0.001556396484375	\\
0.781817374705909	0.001190185546875	\\
0.781861765880943	0.0006103515625	\\
0.781906157055977	0.000762939453125	\\
0.781950548231012	0.0008544921875	\\
0.781994939406046	0.000946044921875	\\
0.78203933058108	0.001007080078125	\\
0.782083721756115	0.000579833984375	\\
0.782128112931149	0.00054931640625	\\
0.782172504106184	0.000762939453125	\\
0.782216895281218	0.00103759765625	\\
0.782261286456252	0.000946044921875	\\
0.782305677631287	0.0010986328125	\\
0.782350068806321	0.000640869140625	\\
0.782394459981356	0.000701904296875	\\
0.78243885115639	0.00128173828125	\\
0.782483242331425	0.0009765625	\\
0.782527633506459	0.0006103515625	\\
0.782572024681493	0.000274658203125	\\
0.782616415856528	0.0001220703125	\\
0.782660807031562	0.000244140625	\\
0.782705198206597	0.00048828125	\\
0.782749589381631	0.000457763671875	\\
0.782793980556665	0.000457763671875	\\
0.7828383717317	0.000518798828125	\\
0.782882762906734	0.000579833984375	\\
0.782927154081768	0.000396728515625	\\
0.782971545256803	0.000335693359375	\\
0.783015936431837	0.000244140625	\\
0.783060327606872	0.000579833984375	\\
0.783104718781906	0.000579833984375	\\
0.783149109956941	0.0006103515625	\\
0.783193501131975	0.000946044921875	\\
0.783237892307009	0.000457763671875	\\
0.783282283482044	0.000579833984375	\\
0.783326674657078	0.0010986328125	\\
0.783371065832113	0.001312255859375	\\
0.783415457007147	0.001190185546875	\\
0.783459848182181	0.00128173828125	\\
0.783504239357216	0.00115966796875	\\
0.78354863053225	0.001251220703125	\\
0.783593021707285	0.0018310546875	\\
0.783637412882319	0.001617431640625	\\
0.783681804057353	0.00164794921875	\\
0.783726195232388	0.001800537109375	\\
0.783770586407422	0.001922607421875	\\
0.783814977582457	0.001922607421875	\\
0.783859368757491	0.001953125	\\
0.783903759932525	0.00201416015625	\\
0.78394815110756	0.001708984375	\\
0.783992542282594	0.001678466796875	\\
0.784036933457629	0.00164794921875	\\
0.784081324632663	0.001861572265625	\\
0.784125715807697	0.00201416015625	\\
0.784170106982732	0.002349853515625	\\
0.784214498157766	0.002197265625	\\
0.784258889332801	0.001800537109375	\\
0.784303280507835	0.001495361328125	\\
0.784347671682869	0.001312255859375	\\
0.784392062857904	0.00164794921875	\\
0.784436454032938	0.001373291015625	\\
0.784480845207973	0.001220703125	\\
0.784525236383007	0.00140380859375	\\
0.784569627558041	0.000885009765625	\\
0.784614018733076	0.000946044921875	\\
0.78465840990811	0.00091552734375	\\
0.784702801083145	0.0003662109375	\\
0.784747192258179	0.000335693359375	\\
0.784791583433213	0.000457763671875	\\
0.784835974608248	0.000518798828125	\\
0.784880365783282	0.00048828125	\\
0.784924756958317	0.000152587890625	\\
0.784969148133351	6.103515625e-05	\\
0.785013539308385	9.1552734375e-05	\\
0.78505793048342	0	\\
0.785102321658454	0.0003662109375	\\
0.785146712833489	0.00048828125	\\
0.785191104008523	0	\\
0.785235495183558	0.000335693359375	\\
0.785279886358592	0.000213623046875	\\
0.785324277533626	-3.0517578125e-05	\\
0.785368668708661	0.000244140625	\\
0.785413059883695	0.00054931640625	\\
0.78545745105873	0.000396728515625	\\
0.785501842233764	9.1552734375e-05	\\
0.785546233408798	0.000579833984375	\\
0.785590624583833	0.000640869140625	\\
0.785635015758867	0.00048828125	\\
0.785679406933902	0.0009765625	\\
0.785723798108936	0.001129150390625	\\
0.78576818928397	0.00152587890625	\\
0.785812580459005	0.00152587890625	\\
0.785856971634039	0.001007080078125	\\
0.785901362809074	0.0009765625	\\
0.785945753984108	0.001251220703125	\\
0.785990145159142	0.00146484375	\\
0.786034536334177	0.001312255859375	\\
0.786078927509211	0.001495361328125	\\
0.786123318684246	0.001922607421875	\\
0.78616770985928	0.001678466796875	\\
0.786212101034314	0.0015869140625	\\
0.786256492209349	0.001678466796875	\\
0.786300883384383	0.001678466796875	\\
0.786345274559418	0.00140380859375	\\
0.786389665734452	0.0013427734375	\\
0.786434056909486	0.001617431640625	\\
0.786478448084521	0.001312255859375	\\
0.786522839259555	0.001312255859375	\\
0.78656723043459	0.00128173828125	\\
0.786611621609624	0.0008544921875	\\
0.786656012784658	0.000518798828125	\\
0.786700403959693	0.00030517578125	\\
0.786744795134727	0.000213623046875	\\
0.786789186309762	9.1552734375e-05	\\
0.786833577484796	-9.1552734375e-05	\\
0.78687796865983	-0.00042724609375	\\
0.786922359834865	-0.00018310546875	\\
0.786966751009899	-0.00030517578125	\\
0.787011142184934	-0.0006103515625	\\
0.787055533359968	6.103515625e-05	\\
0.787099924535002	-0.000213623046875	\\
0.787144315710037	-0.000640869140625	\\
0.787188706885071	-0.001007080078125	\\
0.787233098060106	-0.00128173828125	\\
0.78727748923514	-0.00115966796875	\\
0.787321880410174	-0.001373291015625	\\
0.787366271585209	-0.00177001953125	\\
0.787410662760243	-0.001556396484375	\\
0.787455053935278	-0.001800537109375	\\
0.787499445110312	-0.001708984375	\\
0.787543836285347	-0.00146484375	\\
0.787588227460381	-0.001678466796875	\\
0.787632618635415	-0.0015869140625	\\
0.78767700981045	-0.00189208984375	\\
0.787721400985484	-0.001678466796875	\\
0.787765792160518	-0.0013427734375	\\
0.787810183335553	-0.00140380859375	\\
0.787854574510587	-0.001312255859375	\\
0.787898965685622	-0.00146484375	\\
0.787943356860656	-0.00128173828125	\\
0.78798774803569	-0.0010986328125	\\
0.788032139210725	-0.00103759765625	\\
0.788076530385759	-0.000701904296875	\\
0.788120921560794	-0.0003662109375	\\
0.788165312735828	-0.000518798828125	\\
0.788209703910863	-0.00091552734375	\\
0.788254095085897	-0.00091552734375	\\
0.788298486260931	-0.0008544921875	\\
0.788342877435966	-0.00091552734375	\\
0.788387268611	-0.000762939453125	\\
0.788431659786035	-0.00103759765625	\\
0.788476050961069	-0.0009765625	\\
0.788520442136103	-0.001068115234375	\\
0.788564833311138	-0.001129150390625	\\
0.788609224486172	-0.001220703125	\\
0.788653615661207	-0.001708984375	\\
0.788698006836241	-0.001556396484375	\\
0.788742398011275	-0.001434326171875	\\
0.78878678918631	-0.0013427734375	\\
0.788831180361344	-0.00164794921875	\\
0.788875571536379	-0.002197265625	\\
0.788919962711413	-0.002471923828125	\\
0.788964353886447	-0.002288818359375	\\
0.789008745061482	-0.0020751953125	\\
0.789053136236516	-0.002349853515625	\\
0.789097527411551	-0.002044677734375	\\
0.789141918586585	-0.001922607421875	\\
0.789186309761619	-0.00225830078125	\\
0.789230700936654	-0.001983642578125	\\
0.789275092111688	-0.002227783203125	\\
0.789319483286723	-0.00213623046875	\\
0.789363874461757	-0.001800537109375	\\
0.789408265636791	-0.0015869140625	\\
0.789452656811826	-0.001251220703125	\\
0.78949704798686	-0.001068115234375	\\
0.789541439161895	-0.00128173828125	\\
0.789585830336929	-0.0009765625	\\
0.789630221511963	-0.000823974609375	\\
0.789674612686998	-0.001220703125	\\
0.789719003862032	-0.00115966796875	\\
0.789763395037067	-0.000732421875	\\
0.789807786212101	-0.000274658203125	\\
0.789852177387135	-6.103515625e-05	\\
0.78989656856217	0.00018310546875	\\
0.789940959737204	0.000701904296875	\\
0.789985350912239	0.000335693359375	\\
0.790029742087273	0.000579833984375	\\
0.790074133262307	0.0008544921875	\\
0.790118524437342	0.000823974609375	\\
0.790162915612376	0.00067138671875	\\
0.790207306787411	0.000396728515625	\\
0.790251697962445	0.000732421875	\\
0.79029608913748	0.000640869140625	\\
0.790340480312514	0.000396728515625	\\
0.790384871487548	0.000335693359375	\\
0.790429262662583	6.103515625e-05	\\
0.790473653837617	-0.0001220703125	\\
0.790518045012651	0	\\
0.790562436187686	-0.000213623046875	\\
0.79060682736272	-0.0006103515625	\\
0.790651218537755	-0.000396728515625	\\
0.790695609712789	-6.103515625e-05	\\
0.790740000887823	-0.000579833984375	\\
0.790784392062858	-0.0003662109375	\\
0.790828783237892	0.00018310546875	\\
0.790873174412927	-0.00018310546875	\\
0.790917565587961	-0.0006103515625	\\
0.790961956762996	-0.000579833984375	\\
0.79100634793803	-0.000732421875	\\
0.791050739113064	-0.00030517578125	\\
0.791095130288099	0.000396728515625	\\
0.791139521463133	0.000335693359375	\\
0.791183912638168	3.0517578125e-05	\\
0.791228303813202	0.000152587890625	\\
0.791272694988236	0.0003662109375	\\
0.791317086163271	0.000274658203125	\\
0.791361477338305	9.1552734375e-05	\\
0.791405868513339	-0.000213623046875	\\
0.791450259688374	-3.0517578125e-05	\\
0.791494650863408	0.00079345703125	\\
0.791539042038443	0.000640869140625	\\
0.791583433213477	0.00054931640625	\\
0.791627824388512	0.0006103515625	\\
0.791672215563546	0.00091552734375	\\
0.79171660673858	0.0006103515625	\\
0.791760997913615	0.000762939453125	\\
0.791805389088649	0.000946044921875	\\
0.791849780263684	0.000762939453125	\\
0.791894171438718	0.001251220703125	\\
0.791938562613752	0.001312255859375	\\
0.791982953788787	0.00128173828125	\\
0.792027344963821	0.001617431640625	\\
0.792071736138856	0.00177001953125	\\
0.79211612731389	0.00146484375	\\
0.792160518488924	0.001495361328125	\\
0.792204909663959	0.001495361328125	\\
0.792249300838993	0.001251220703125	\\
0.792293692014028	0.00140380859375	\\
0.792338083189062	0.002044677734375	\\
0.792382474364096	0.002410888671875	\\
0.792426865539131	0.00201416015625	\\
0.792471256714165	0.0018310546875	\\
0.7925156478892	0.0018310546875	\\
0.792560039064234	0.00164794921875	\\
0.792604430239268	0.001617431640625	\\
0.792648821414303	0.00128173828125	\\
0.792693212589337	0.001190185546875	\\
0.792737603764372	0.00128173828125	\\
0.792781994939406	0.001434326171875	\\
0.79282638611444	0.001556396484375	\\
0.792870777289475	0.001068115234375	\\
0.792915168464509	0.00091552734375	\\
0.792959559639544	0.0009765625	\\
0.793003950814578	0.001007080078125	\\
0.793048341989612	0.00091552734375	\\
0.793092733164647	0.000885009765625	\\
0.793137124339681	0.000732421875	\\
0.793181515514716	0.000885009765625	\\
0.79322590668975	0.0010986328125	\\
0.793270297864785	0.000823974609375	\\
0.793314689039819	0.0006103515625	\\
0.793359080214853	0.00079345703125	\\
0.793403471389888	0.001434326171875	\\
0.793447862564922	0.001373291015625	\\
0.793492253739956	0.00091552734375	\\
0.793536644914991	0.001312255859375	\\
0.793581036090025	0.001953125	\\
0.79362542726506	0.002227783203125	\\
0.793669818440094	0.002410888671875	\\
0.793714209615129	0.002288818359375	\\
0.793758600790163	0.001861572265625	\\
0.793802991965197	0.00244140625	\\
0.793847383140232	0.003021240234375	\\
0.793891774315266	0.00299072265625	\\
0.793936165490301	0.00299072265625	\\
0.793980556665335	0.002655029296875	\\
0.794024947840369	0.002716064453125	\\
0.794069339015404	0.0025634765625	\\
0.794113730190438	0.00244140625	\\
0.794158121365473	0.002471923828125	\\
0.794202512540507	0.00238037109375	\\
0.794246903715541	0.00250244140625	\\
0.794291294890576	0.0025634765625	\\
0.79433568606561	0.00244140625	\\
0.794380077240645	0.0029296875	\\
0.794424468415679	0.00335693359375	\\
0.794468859590713	0.002777099609375	\\
0.794513250765748	0.0032958984375	\\
0.794557641940782	0.00299072265625	\\
0.794602033115817	0.00213623046875	\\
0.794646424290851	0.002777099609375	\\
0.794690815465885	0.0025634765625	\\
0.79473520664092	0.002471923828125	\\
0.794779597815954	0.002655029296875	\\
0.794823988990989	0.00213623046875	\\
0.794868380166023	0.00213623046875	\\
0.794912771341057	0.001861572265625	\\
0.794957162516092	0.002166748046875	\\
0.795001553691126	0.001983642578125	\\
0.795045944866161	0.00164794921875	\\
0.795090336041195	0.001434326171875	\\
0.795134727216229	0.001129150390625	\\
0.795179118391264	0.001251220703125	\\
0.795223509566298	0.000946044921875	\\
0.795267900741333	0.000885009765625	\\
0.795312291916367	0.001068115234375	\\
0.795356683091401	0.001007080078125	\\
0.795401074266436	0.001312255859375	\\
0.79544546544147	0.000640869140625	\\
0.795489856616505	0.000823974609375	\\
0.795534247791539	0.001007080078125	\\
0.795578638966573	0.00091552734375	\\
0.795623030141608	0.000946044921875	\\
0.795667421316642	0.000732421875	\\
0.795711812491677	0.000640869140625	\\
0.795756203666711	0.00079345703125	\\
0.795800594841745	0.0010986328125	\\
0.79584498601678	0.000946044921875	\\
0.795889377191814	0.000946044921875	\\
0.795933768366849	0.00115966796875	\\
0.795978159541883	0.001373291015625	\\
0.796022550716918	0.00103759765625	\\
0.796066941891952	0.001434326171875	\\
0.796111333066986	0.001312255859375	\\
0.796155724242021	0.001251220703125	\\
0.796200115417055	0.00146484375	\\
0.796244506592089	0.00146484375	\\
0.796288897767124	0.001495361328125	\\
0.796333288942158	0.00140380859375	\\
0.796377680117193	0.00140380859375	\\
0.796422071292227	0.001312255859375	\\
0.796466462467261	0.001373291015625	\\
0.796510853642296	0.00140380859375	\\
0.79655524481733	0.001190185546875	\\
0.796599635992365	0.001251220703125	\\
0.796644027167399	0.0010986328125	\\
0.796688418342434	0.00103759765625	\\
0.796732809517468	0.001190185546875	\\
0.796777200692502	0.001434326171875	\\
0.796821591867537	0.001312255859375	\\
0.796865983042571	0.001129150390625	\\
0.796910374217606	0.001220703125	\\
0.79695476539264	0.001068115234375	\\
0.796999156567674	0.0010986328125	\\
0.797043547742709	0.000762939453125	\\
0.797087938917743	0.000885009765625	\\
0.797132330092778	0.0010986328125	\\
0.797176721267812	0.001220703125	\\
0.797221112442846	0.00140380859375	\\
0.797265503617881	0.00091552734375	\\
0.797309894792915	0.001251220703125	\\
0.79735428596795	0.001220703125	\\
0.797398677142984	0.00079345703125	\\
0.797443068318018	0.00079345703125	\\
0.797487459493053	0.000823974609375	\\
0.797531850668087	0.000885009765625	\\
0.797576241843122	0.00128173828125	\\
0.797620633018156	0.0009765625	\\
0.79766502419319	0.001434326171875	\\
0.797709415368225	0.00164794921875	\\
0.797753806543259	0.001251220703125	\\
0.797798197718294	0.001190185546875	\\
0.797842588893328	0.001708984375	\\
0.797886980068362	0.002044677734375	\\
0.797931371243397	0.00152587890625	\\
0.797975762418431	0.00140380859375	\\
0.798020153593466	0.001495361328125	\\
0.7980645447685	0.00146484375	\\
0.798108935943534	0.001312255859375	\\
0.798153327118569	0.0015869140625	\\
0.798197718293603	0.001617431640625	\\
0.798242109468638	0.00140380859375	\\
0.798286500643672	0.00189208984375	\\
0.798330891818706	0.002105712890625	\\
0.798375282993741	0.00189208984375	\\
0.798419674168775	0.00164794921875	\\
0.79846406534381	0.00140380859375	\\
0.798508456518844	0.00128173828125	\\
0.798552847693878	0.001373291015625	\\
0.798597238868913	0.001220703125	\\
0.798641630043947	0.000732421875	\\
0.798686021218982	0.000885009765625	\\
0.798730412394016	0.00103759765625	\\
0.798774803569051	0.000823974609375	\\
0.798819194744085	0.00091552734375	\\
0.798863585919119	0.001251220703125	\\
0.798907977094154	0.0009765625	\\
0.798952368269188	0.00091552734375	\\
0.798996759444222	0.0010986328125	\\
0.799041150619257	0.00103759765625	\\
0.799085541794291	0.001068115234375	\\
0.799129932969326	0.00103759765625	\\
0.79917432414436	0.000823974609375	\\
0.799218715319394	0.000518798828125	\\
0.799263106494429	0.000457763671875	\\
0.799307497669463	0.00054931640625	\\
0.799351888844498	0.00067138671875	\\
0.799396280019532	0.000640869140625	\\
0.799440671194567	0.000946044921875	\\
0.799485062369601	0.001312255859375	\\
0.799529453544635	0.00164794921875	\\
0.79957384471967	0.00213623046875	\\
0.799618235894704	0.00225830078125	\\
0.799662627069739	0.00238037109375	\\
0.799707018244773	0.0023193359375	\\
0.799751409419807	0.002105712890625	\\
0.799795800594842	0.002532958984375	\\
0.799840191769876	0.002471923828125	\\
0.799884582944911	0.002410888671875	\\
0.799928974119945	0.00311279296875	\\
0.799973365294979	0.00311279296875	\\
0.800017756470014	0.00250244140625	\\
0.800062147645048	0.002532958984375	\\
0.800106538820083	0.002532958984375	\\
0.800150929995117	0.0023193359375	\\
0.800195321170151	0.002716064453125	\\
0.800239712345186	0.002227783203125	\\
0.80028410352022	0.001678466796875	\\
0.800328494695255	0.001739501953125	\\
0.800372885870289	0.00213623046875	\\
0.800417277045323	0.0020751953125	\\
0.800461668220358	0.00189208984375	\\
0.800506059395392	0.0018310546875	\\
0.800550450570427	0.0018310546875	\\
0.800594841745461	0.001739501953125	\\
0.800639232920495	0.001434326171875	\\
0.80068362409553	0.001373291015625	\\
0.800728015270564	0.0010986328125	\\
0.800772406445599	0.0009765625	\\
0.800816797620633	0.00140380859375	\\
0.800861188795667	0.001678466796875	\\
0.800905579970702	0.001312255859375	\\
0.800949971145736	0.0013427734375	\\
0.800994362320771	0.001129150390625	\\
0.801038753495805	0.000823974609375	\\
0.801083144670839	0.001220703125	\\
0.801127535845874	0.0013427734375	\\
0.801171927020908	0.0010986328125	\\
0.801216318195943	0.001434326171875	\\
0.801260709370977	0.001220703125	\\
0.801305100546011	0.001190185546875	\\
0.801349491721046	0.00177001953125	\\
0.80139388289608	0.0018310546875	\\
0.801438274071115	0.00201416015625	\\
0.801482665246149	0.001861572265625	\\
0.801527056421183	0.002044677734375	\\
0.801571447596218	0.00225830078125	\\
0.801615838771252	0.001739501953125	\\
0.801660229946287	0.001678466796875	\\
0.801704621121321	0.0015869140625	\\
0.801749012296356	0.000946044921875	\\
0.80179340347139	0.00103759765625	\\
0.801837794646424	0.001495361328125	\\
0.801882185821459	0.001739501953125	\\
0.801926576996493	0.001953125	\\
0.801970968171527	0.00213623046875	\\
0.802015359346562	0.002471923828125	\\
0.802059750521596	0.002685546875	\\
0.802104141696631	0.0023193359375	\\
0.802148532871665	0.002227783203125	\\
0.8021929240467	0.002288818359375	\\
0.802237315221734	0.00238037109375	\\
0.802281706396768	0.002197265625	\\
0.802326097571803	0.001922607421875	\\
0.802370488746837	0.002166748046875	\\
0.802414879921872	0.00189208984375	\\
0.802459271096906	0.001678466796875	\\
0.80250366227194	0.001983642578125	\\
0.802548053446975	0.002197265625	\\
0.802592444622009	0.001861572265625	\\
0.802636835797044	0.001922607421875	\\
0.802681226972078	0.0018310546875	\\
0.802725618147112	0.001617431640625	\\
0.802770009322147	0.001129150390625	\\
0.802814400497181	0.001068115234375	\\
0.802858791672216	0.001251220703125	\\
0.80290318284725	0.000946044921875	\\
0.802947574022284	0.001007080078125	\\
0.802991965197319	0.000885009765625	\\
0.803036356372353	0.000457763671875	\\
0.803080747547388	0.00048828125	\\
0.803125138722422	0.000518798828125	\\
0.803169529897456	0.000335693359375	\\
0.803213921072491	0.000396728515625	\\
0.803258312247525	6.103515625e-05	\\
0.80330270342256	-0.000152587890625	\\
0.803347094597594	0.0001220703125	\\
0.803391485772628	-9.1552734375e-05	\\
0.803435876947663	-0.00042724609375	\\
0.803480268122697	-0.000152587890625	\\
0.803524659297732	-0.000244140625	\\
0.803569050472766	-0.00048828125	\\
0.8036134416478	-0.000213623046875	\\
0.803657832822835	-0.000274658203125	\\
0.803702223997869	-0.000701904296875	\\
0.803746615172904	-0.00048828125	\\
0.803791006347938	-0.00030517578125	\\
0.803835397522972	-0.000396728515625	\\
0.803879788698007	0.000213623046875	\\
0.803924179873041	0.000457763671875	\\
0.803968571048076	0.00030517578125	\\
0.80401296222311	0.00030517578125	\\
0.804057353398144	-6.103515625e-05	\\
0.804101744573179	0.000152587890625	\\
0.804146135748213	0.00018310546875	\\
0.804190526923248	0.000244140625	\\
0.804234918098282	0.00048828125	\\
0.804279309273316	0.000518798828125	\\
0.804323700448351	0.000335693359375	\\
0.804368091623385	0.000518798828125	\\
0.80441248279842	0.000457763671875	\\
0.804456873973454	0.000244140625	\\
0.804501265148489	-6.103515625e-05	\\
0.804545656323523	3.0517578125e-05	\\
0.804590047498557	9.1552734375e-05	\\
0.804634438673592	-6.103515625e-05	\\
0.804678829848626	0.000152587890625	\\
0.80472322102366	-3.0517578125e-05	\\
0.804767612198695	-0.000244140625	\\
0.804812003373729	0.0001220703125	\\
0.804856394548764	6.103515625e-05	\\
0.804900785723798	3.0517578125e-05	\\
0.804945176898832	0.0001220703125	\\
0.804989568073867	-0.000335693359375	\\
0.805033959248901	-0.000396728515625	\\
0.805078350423936	-0.000518798828125	\\
0.80512274159897	-0.00030517578125	\\
0.805167132774005	0.0001220703125	\\
0.805211523949039	-0.000579833984375	\\
0.805255915124073	-0.000518798828125	\\
0.805300306299108	-0.00054931640625	\\
0.805344697474142	-0.0008544921875	\\
0.805389088649177	-0.000762939453125	\\
0.805433479824211	-0.00079345703125	\\
0.805477870999245	-0.00067138671875	\\
0.80552226217428	-0.00128173828125	\\
0.805566653349314	-0.001434326171875	\\
0.805611044524349	-0.001129150390625	\\
0.805655435699383	-0.001312255859375	\\
0.805699826874417	-0.001434326171875	\\
0.805744218049452	-0.0010986328125	\\
0.805788609224486	-0.00091552734375	\\
0.805833000399521	-0.001190185546875	\\
0.805877391574555	-0.001129150390625	\\
0.805921782749589	-0.000762939453125	\\
0.805966173924624	-0.001007080078125	\\
0.806010565099658	-0.001129150390625	\\
0.806054956274693	-0.001373291015625	\\
0.806099347449727	-0.001495361328125	\\
0.806143738624761	-0.001495361328125	\\
0.806188129799796	-0.00146484375	\\
0.80623252097483	-0.001739501953125	\\
0.806276912149865	-0.00213623046875	\\
0.806321303324899	-0.001922607421875	\\
0.806365694499933	-0.001617431640625	\\
0.806410085674968	-0.00164794921875	\\
0.806454476850002	-0.001861572265625	\\
0.806498868025037	-0.0015869140625	\\
0.806543259200071	-0.00140380859375	\\
0.806587650375105	-0.001617431640625	\\
0.80663204155014	-0.001678466796875	\\
0.806676432725174	-0.00177001953125	\\
0.806720823900209	-0.00164794921875	\\
0.806765215075243	-0.001708984375	\\
0.806809606250277	-0.002197265625	\\
0.806853997425312	-0.002685546875	\\
0.806898388600346	-0.002777099609375	\\
0.806942779775381	-0.002471923828125	\\
0.806987170950415	-0.0020751953125	\\
0.807031562125449	-0.0023193359375	\\
0.807075953300484	-0.002593994140625	\\
0.807120344475518	-0.00244140625	\\
0.807164735650553	-0.002227783203125	\\
0.807209126825587	-0.0020751953125	\\
0.807253518000622	-0.002105712890625	\\
0.807297909175656	-0.002288818359375	\\
0.80734230035069	-0.002288818359375	\\
0.807386691525725	-0.001922607421875	\\
0.807431082700759	-0.001983642578125	\\
0.807475473875793	-0.0020751953125	\\
0.807519865050828	-0.0018310546875	\\
0.807564256225862	-0.001708984375	\\
0.807608647400897	-0.001739501953125	\\
0.807653038575931	-0.001678466796875	\\
0.807697429750965	-0.0013427734375	\\
0.807741820926	-0.00103759765625	\\
0.807786212101034	-0.000823974609375	\\
0.807830603276069	-0.000946044921875	\\
0.807874994451103	-0.0013427734375	\\
0.807919385626138	-0.00152587890625	\\
0.807963776801172	-0.001312255859375	\\
0.808008167976206	-0.001220703125	\\
0.808052559151241	-0.001373291015625	\\
0.808096950326275	-0.001708984375	\\
0.80814134150131	-0.0015869140625	\\
0.808185732676344	-0.00128173828125	\\
0.808230123851378	-0.001190185546875	\\
0.808274515026413	-0.001495361328125	\\
0.808318906201447	-0.002471923828125	\\
0.808363297376482	-0.00201416015625	\\
0.808407688551516	-0.00225830078125	\\
0.80845207972655	-0.0029296875	\\
0.808496470901585	-0.0025634765625	\\
0.808540862076619	-0.002685546875	\\
0.808585253251654	-0.00250244140625	\\
0.808629644426688	-0.002471923828125	\\
0.808674035601722	-0.003021240234375	\\
0.808718426776757	-0.002838134765625	\\
0.808762817951791	-0.002410888671875	\\
0.808807209126826	-0.002410888671875	\\
0.80885160030186	-0.00250244140625	\\
0.808895991476894	-0.002227783203125	\\
0.808940382651929	-0.002593994140625	\\
0.808984773826963	-0.003021240234375	\\
0.809029165001998	-0.00299072265625	\\
0.809073556177032	-0.0028076171875	\\
0.809117947352066	-0.002960205078125	\\
0.809162338527101	-0.00274658203125	\\
0.809206729702135	-0.002471923828125	\\
0.80925112087717	-0.002532958984375	\\
0.809295512052204	-0.0023193359375	\\
0.809339903227238	-0.001953125	\\
0.809384294402273	-0.001922607421875	\\
0.809428685577307	-0.00225830078125	\\
0.809473076752342	-0.001953125	\\
0.809517467927376	-0.001312255859375	\\
0.80956185910241	-0.001556396484375	\\
0.809606250277445	-0.00164794921875	\\
0.809650641452479	-0.001251220703125	\\
0.809695032627514	-0.001434326171875	\\
0.809739423802548	-0.001556396484375	\\
0.809783814977582	-0.0008544921875	\\
0.809828206152617	-0.001190185546875	\\
0.809872597327651	-0.001312255859375	\\
0.809916988502686	-0.0006103515625	\\
0.80996137967772	-0.000518798828125	\\
0.810005770852754	-0.0006103515625	\\
0.810050162027789	-0.000518798828125	\\
0.810094553202823	-0.00042724609375	\\
0.810138944377858	-0.000640869140625	\\
0.810183335552892	-0.00067138671875	\\
0.810227726727927	-0.00067138671875	\\
0.810272117902961	-0.000885009765625	\\
0.810316509077995	-0.000762939453125	\\
0.81036090025303	-0.0006103515625	\\
0.810405291428064	-0.00048828125	\\
0.810449682603098	-0.00067138671875	\\
0.810494073778133	-0.00091552734375	\\
0.810538464953167	-0.0008544921875	\\
0.810582856128202	-0.00079345703125	\\
0.810627247303236	-0.000823974609375	\\
0.810671638478271	-0.000762939453125	\\
0.810716029653305	-0.0001220703125	\\
0.810760420828339	-0.000213623046875	\\
0.810804812003374	-0.00018310546875	\\
0.810849203178408	9.1552734375e-05	\\
0.810893594353443	-0.000579833984375	\\
0.810937985528477	-0.001007080078125	\\
0.810982376703511	-0.00115966796875	\\
0.811026767878546	-0.001373291015625	\\
0.81107115905358	-0.00115966796875	\\
0.811115550228615	-0.001556396484375	\\
0.811159941403649	-0.00177001953125	\\
0.811204332578683	-0.001220703125	\\
0.811248723753718	-0.0013427734375	\\
0.811293114928752	-0.00128173828125	\\
0.811337506103787	-0.00164794921875	\\
0.811381897278821	-0.001953125	\\
0.811426288453855	-0.001678466796875	\\
0.81147067962889	-0.001708984375	\\
0.811515070803924	-0.001708984375	\\
0.811559461978959	-0.0018310546875	\\
0.811603853153993	-0.0015869140625	\\
0.811648244329027	-0.001708984375	\\
0.811692635504062	-0.0023193359375	\\
0.811737026679096	-0.00238037109375	\\
0.811781417854131	-0.002471923828125	\\
0.811825809029165	-0.0028076171875	\\
0.811870200204199	-0.00238037109375	\\
0.811914591379234	-0.001861572265625	\\
0.811958982554268	-0.002410888671875	\\
0.812003373729303	-0.002471923828125	\\
0.812047764904337	-0.001861572265625	\\
0.812092156079371	-0.002288818359375	\\
0.812136547254406	-0.002593994140625	\\
0.81218093842944	-0.00238037109375	\\
0.812225329604475	-0.00238037109375	\\
0.812269720779509	-0.002227783203125	\\
0.812314111954543	-0.001739501953125	\\
0.812358503129578	-0.002044677734375	\\
0.812402894304612	-0.001922607421875	\\
0.812447285479647	-0.001556396484375	\\
0.812491676654681	-0.002044677734375	\\
0.812536067829715	-0.0013427734375	\\
0.81258045900475	-0.00140380859375	\\
0.812624850179784	-0.001739501953125	\\
0.812669241354819	-0.001617431640625	\\
0.812713632529853	-0.001617431640625	\\
0.812758023704887	-0.00146484375	\\
0.812802414879922	-0.00177001953125	\\
0.812846806054956	-0.001800537109375	\\
0.812891197229991	-0.0018310546875	\\
0.812935588405025	-0.001373291015625	\\
0.81297997958006	-0.001220703125	\\
0.813024370755094	-0.001373291015625	\\
0.813068761930128	-0.001617431640625	\\
0.813113153105163	-0.001800537109375	\\
0.813157544280197	-0.001983642578125	\\
0.813201935455231	-0.0020751953125	\\
0.813246326630266	-0.00152587890625	\\
0.8132907178053	-0.00146484375	\\
0.813335108980335	-0.00189208984375	\\
0.813379500155369	-0.002197265625	\\
0.813423891330403	-0.002410888671875	\\
0.813468282505438	-0.00225830078125	\\
0.813512673680472	-0.002227783203125	\\
0.813557064855507	-0.002593994140625	\\
0.813601456030541	-0.002197265625	\\
0.813645847205576	-0.0020751953125	\\
0.81369023838061	-0.0018310546875	\\
0.813734629555644	-0.00140380859375	\\
0.813779020730679	-0.002227783203125	\\
0.813823411905713	-0.002410888671875	\\
0.813867803080748	-0.00189208984375	\\
0.813912194255782	-0.001983642578125	\\
0.813956585430816	-0.00201416015625	\\
0.814000976605851	-0.0015869140625	\\
0.814045367780885	-0.001800537109375	\\
0.81408975895592	-0.001983642578125	\\
0.814134150130954	-0.001739501953125	\\
0.814178541305988	-0.002044677734375	\\
0.814222932481023	-0.002288818359375	\\
0.814267323656057	-0.002288818359375	\\
0.814311714831092	-0.002349853515625	\\
0.814356106006126	-0.002471923828125	\\
0.81440049718116	-0.002410888671875	\\
0.814444888356195	-0.00262451171875	\\
0.814489279531229	-0.00262451171875	\\
0.814533670706264	-0.00250244140625	\\
0.814578061881298	-0.001983642578125	\\
0.814622453056332	-0.001922607421875	\\
0.814666844231367	-0.00201416015625	\\
0.814711235406401	-0.0020751953125	\\
0.814755626581436	-0.002166748046875	\\
0.81480001775647	-0.00201416015625	\\
0.814844408931504	-0.002044677734375	\\
0.814888800106539	-0.002044677734375	\\
0.814933191281573	-0.00201416015625	\\
0.814977582456608	-0.002044677734375	\\
0.815021973631642	-0.00213623046875	\\
0.815066364806676	-0.00225830078125	\\
0.815110755981711	-0.002105712890625	\\
0.815155147156745	-0.002197265625	\\
0.81519953833178	-0.002044677734375	\\
0.815243929506814	-0.00201416015625	\\
0.815288320681848	-0.0020751953125	\\
0.815332711856883	-0.0018310546875	\\
0.815377103031917	-0.00164794921875	\\
0.815421494206952	-0.0015869140625	\\
0.815465885381986	-0.00152587890625	\\
0.81551027655702	-0.001190185546875	\\
0.815554667732055	-0.00115966796875	\\
0.815599058907089	-0.00164794921875	\\
0.815643450082124	-0.001708984375	\\
0.815687841257158	-0.00140380859375	\\
0.815732232432193	-0.000701904296875	\\
0.815776623607227	-0.000762939453125	\\
0.815821014782261	-0.000946044921875	\\
0.815865405957296	-0.000823974609375	\\
0.81590979713233	-0.000823974609375	\\
0.815954188307365	-0.000518798828125	\\
0.815998579482399	-0.00079345703125	\\
0.816042970657433	-0.00048828125	\\
0.816087361832468	-0.000946044921875	\\
0.816131753007502	-0.001129150390625	\\
0.816176144182536	-0.000885009765625	\\
0.816220535357571	-0.0009765625	\\
0.816264926532605	-0.00054931640625	\\
0.81630931770764	-0.0006103515625	\\
0.816353708882674	-0.001007080078125	\\
0.816398100057709	-0.001251220703125	\\
0.816442491232743	-0.0013427734375	\\
0.816486882407777	-0.00103759765625	\\
0.816531273582812	-0.0010986328125	\\
0.816575664757846	-0.00128173828125	\\
0.816620055932881	-0.00152587890625	\\
0.816664447107915	-0.00128173828125	\\
0.816708838282949	-0.000823974609375	\\
0.816753229457984	-0.00079345703125	\\
0.816797620633018	-0.00054931640625	\\
0.816842011808053	-0.0009765625	\\
0.816886402983087	-0.001434326171875	\\
0.816930794158121	-0.0009765625	\\
0.816975185333156	-0.001434326171875	\\
0.81701957650819	-0.001617431640625	\\
0.817063967683225	-0.001190185546875	\\
0.817108358858259	-0.000946044921875	\\
0.817152750033293	-0.00115966796875	\\
0.817197141208328	-0.00115966796875	\\
0.817241532383362	-0.000640869140625	\\
0.817285923558397	-0.000640869140625	\\
0.817330314733431	-0.000701904296875	\\
0.817374705908465	-0.000579833984375	\\
0.8174190970835	-0.000823974609375	\\
0.817463488258534	-0.001007080078125	\\
0.817507879433569	-0.00091552734375	\\
0.817552270608603	-0.00079345703125	\\
0.817596661783637	-0.000885009765625	\\
0.817641052958672	-0.000579833984375	\\
0.817685444133706	-0.000640869140625	\\
0.817729835308741	-0.00042724609375	\\
0.817774226483775	-6.103515625e-05	\\
0.817818617658809	-0.00018310546875	\\
0.817863008833844	-6.103515625e-05	\\
0.817907400008878	-0.00042724609375	\\
0.817951791183913	-0.000457763671875	\\
0.817996182358947	-0.000152587890625	\\
0.818040573533981	0	\\
0.818084964709016	-9.1552734375e-05	\\
0.81812935588405	-0.00030517578125	\\
0.818173747059085	-0.00054931640625	\\
0.818218138234119	-0.000274658203125	\\
0.818262529409153	-0.000579833984375	\\
0.818306920584188	-0.000885009765625	\\
0.818351311759222	-0.00079345703125	\\
0.818395702934257	-0.00103759765625	\\
0.818440094109291	-0.00103759765625	\\
0.818484485284325	-0.00091552734375	\\
0.81852887645936	-0.00103759765625	\\
0.818573267634394	-0.001007080078125	\\
0.818617658809429	-0.001251220703125	\\
0.818662049984463	-0.001251220703125	\\
0.818706441159498	-0.0013427734375	\\
0.818750832334532	-0.001220703125	\\
0.818795223509566	-0.000946044921875	\\
0.818839614684601	-0.0008544921875	\\
0.818884005859635	-0.00054931640625	\\
0.818928397034669	-0.000579833984375	\\
0.818972788209704	-0.00067138671875	\\
0.819017179384738	-0.0006103515625	\\
0.819061570559773	-0.000335693359375	\\
0.819105961734807	-0.000274658203125	\\
0.819150352909842	-0.000457763671875	\\
0.819194744084876	-0.000640869140625	\\
0.81923913525991	-0.00048828125	\\
0.819283526434945	-0.000244140625	\\
0.819327917609979	-0.000274658203125	\\
0.819372308785014	-0.00018310546875	\\
0.819416699960048	0.0001220703125	\\
0.819461091135082	6.103515625e-05	\\
0.819505482310117	0.000152587890625	\\
0.819549873485151	6.103515625e-05	\\
0.819594264660186	0.00018310546875	\\
0.81963865583522	3.0517578125e-05	\\
0.819683047010254	-9.1552734375e-05	\\
0.819727438185289	-0.000244140625	\\
0.819771829360323	-0.0001220703125	\\
0.819816220535358	-0.00018310546875	\\
0.819860611710392	-0.000396728515625	\\
0.819905002885426	-9.1552734375e-05	\\
0.819949394060461	-0.00048828125	\\
0.819993785235495	-0.000457763671875	\\
0.82003817641053	0.000152587890625	\\
0.820082567585564	0.000335693359375	\\
0.820126958760598	0.0003662109375	\\
0.820171349935633	0.000274658203125	\\
0.820215741110667	9.1552734375e-05	\\
0.820260132285702	0.00018310546875	\\
0.820304523460736	0.000213623046875	\\
0.82034891463577	3.0517578125e-05	\\
0.820393305810805	-0.000457763671875	\\
0.820437696985839	-0.000640869140625	\\
0.820482088160874	-0.000579833984375	\\
0.820526479335908	-0.00054931640625	\\
0.820570870510942	-0.0001220703125	\\
0.820615261685977	-3.0517578125e-05	\\
0.820659652861011	-0.00018310546875	\\
0.820704044036046	-0.000152587890625	\\
0.82074843521108	-0.000457763671875	\\
0.820792826386114	-0.000640869140625	\\
0.820837217561149	-0.00042724609375	\\
0.820881608736183	-0.000396728515625	\\
0.820925999911218	-0.00018310546875	\\
0.820970391086252	-0.000396728515625	\\
0.821014782261286	-0.000457763671875	\\
0.821059173436321	-0.000335693359375	\\
0.821103564611355	-0.00054931640625	\\
0.82114795578639	-0.000457763671875	\\
0.821192346961424	-0.000457763671875	\\
0.821236738136458	-0.00048828125	\\
0.821281129311493	-0.00054931640625	\\
0.821325520486527	-0.000335693359375	\\
0.821369911661562	-0.000579833984375	\\
0.821414302836596	-0.000885009765625	\\
0.821458694011631	-0.00091552734375	\\
0.821503085186665	-0.0008544921875	\\
0.821547476361699	-0.000762939453125	\\
0.821591867536734	-0.001068115234375	\\
0.821636258711768	-0.001495361328125	\\
0.821680649886802	-0.0010986328125	\\
0.821725041061837	-0.001312255859375	\\
0.821769432236871	-0.001373291015625	\\
0.821813823411906	-0.00115966796875	\\
0.82185821458694	-0.001007080078125	\\
0.821902605761974	-0.00115966796875	\\
0.821946996937009	-0.001495361328125	\\
0.821991388112043	-0.000946044921875	\\
0.822035779287078	-0.00128173828125	\\
0.822080170462112	-0.00152587890625	\\
0.822124561637147	-0.001434326171875	\\
0.822168952812181	-0.001251220703125	\\
0.822213343987215	-0.00103759765625	\\
0.82225773516225	-0.001373291015625	\\
0.822302126337284	-0.00152587890625	\\
0.822346517512319	-0.001312255859375	\\
0.822390908687353	-0.000885009765625	\\
0.822435299862387	-0.000701904296875	\\
0.822479691037422	-0.0013427734375	\\
0.822524082212456	-0.0013427734375	\\
0.822568473387491	-0.001434326171875	\\
0.822612864562525	-0.001373291015625	\\
0.822657255737559	-0.001220703125	\\
0.822701646912594	-0.001556396484375	\\
0.822746038087628	-0.001556396484375	\\
0.822790429262663	-0.001373291015625	\\
0.822834820437697	-0.001220703125	\\
0.822879211612731	-0.001190185546875	\\
0.822923602787766	-0.001129150390625	\\
0.8229679939628	-0.001068115234375	\\
0.823012385137835	-0.001495361328125	\\
0.823056776312869	-0.00128173828125	\\
0.823101167487903	-0.001373291015625	\\
0.823145558662938	-0.001373291015625	\\
0.823189949837972	-0.001068115234375	\\
0.823234341013007	-0.000701904296875	\\
0.823278732188041	-0.000762939453125	\\
0.823323123363075	-0.000762939453125	\\
0.82336751453811	-0.00042724609375	\\
0.823411905713144	-0.0006103515625	\\
0.823456296888179	-0.000823974609375	\\
0.823500688063213	-0.000823974609375	\\
0.823545079238247	-0.000274658203125	\\
0.823589470413282	-0.0008544921875	\\
0.823633861588316	-0.00091552734375	\\
0.823678252763351	-0.000396728515625	\\
0.823722643938385	-0.00079345703125	\\
0.823767035113419	-0.000946044921875	\\
0.823811426288454	-0.001129150390625	\\
0.823855817463488	-0.001220703125	\\
0.823900208638523	-0.001190185546875	\\
0.823944599813557	-0.000946044921875	\\
0.823988990988591	-0.0006103515625	\\
0.824033382163626	-0.000732421875	\\
0.82407777333866	-0.00048828125	\\
0.824122164513695	-0.00054931640625	\\
0.824166555688729	-0.0010986328125	\\
0.824210946863764	-0.00140380859375	\\
0.824255338038798	-0.00140380859375	\\
0.824299729213832	-0.001434326171875	\\
0.824344120388867	-0.0018310546875	\\
0.824388511563901	-0.00177001953125	\\
0.824432902738936	-0.001678466796875	\\
0.82447729391397	-0.00152587890625	\\
0.824521685089004	-0.00146484375	\\
0.824566076264039	-0.0015869140625	\\
0.824610467439073	-0.001373291015625	\\
0.824654858614107	-0.00140380859375	\\
0.824699249789142	-0.0015869140625	\\
0.824743640964176	-0.001220703125	\\
0.824788032139211	-0.001312255859375	\\
0.824832423314245	-0.001434326171875	\\
0.82487681448928	-0.00140380859375	\\
0.824921205664314	-0.00140380859375	\\
0.824965596839348	-0.0010986328125	\\
0.825009988014383	-0.0008544921875	\\
0.825054379189417	-0.000732421875	\\
0.825098770364452	-0.00079345703125	\\
0.825143161539486	-0.00079345703125	\\
0.82518755271452	-0.001007080078125	\\
0.825231943889555	-0.00140380859375	\\
0.825276335064589	-0.00103759765625	\\
0.825320726239624	-0.000762939453125	\\
0.825365117414658	-0.000946044921875	\\
0.825409508589692	-0.00091552734375	\\
0.825453899764727	-0.001007080078125	\\
0.825498290939761	-0.00103759765625	\\
0.825542682114796	-0.001129150390625	\\
0.82558707328983	-0.001007080078125	\\
0.825631464464864	-0.000885009765625	\\
0.825675855639899	-0.0009765625	\\
0.825720246814933	-0.001007080078125	\\
0.825764637989968	-0.00103759765625	\\
0.825809029165002	-0.000762939453125	\\
0.825853420340036	-0.00048828125	\\
0.825897811515071	-0.000579833984375	\\
0.825942202690105	-0.000579833984375	\\
0.82598659386514	-0.00067138671875	\\
0.826030985040174	-0.000701904296875	\\
0.826075376215208	-0.000274658203125	\\
0.826119767390243	-0.000640869140625	\\
0.826164158565277	-0.0009765625	\\
0.826208549740312	-0.0010986328125	\\
0.826252940915346	-0.00128173828125	\\
0.82629733209038	-0.001251220703125	\\
0.826341723265415	-0.001220703125	\\
0.826386114440449	-0.00140380859375	\\
0.826430505615484	-0.001495361328125	\\
0.826474896790518	-0.001495361328125	\\
0.826519287965552	-0.00152587890625	\\
0.826563679140587	-0.001800537109375	\\
0.826608070315621	-0.00152587890625	\\
0.826652461490656	-0.001251220703125	\\
0.82669685266569	-0.001373291015625	\\
0.826741243840724	-0.0015869140625	\\
0.826785635015759	-0.001861572265625	\\
0.826830026190793	-0.001373291015625	\\
0.826874417365828	-0.001434326171875	\\
0.826918808540862	-0.00213623046875	\\
0.826963199715896	-0.001922607421875	\\
0.827007590890931	-0.001739501953125	\\
0.827051982065965	-0.0018310546875	\\
0.827096373241	-0.002166748046875	\\
0.827140764416034	-0.00189208984375	\\
0.827185155591069	-0.0013427734375	\\
0.827229546766103	-0.00152587890625	\\
0.827273937941137	-0.00164794921875	\\
0.827318329116172	-0.00177001953125	\\
0.827362720291206	-0.001617431640625	\\
0.82740711146624	-0.00152587890625	\\
0.827451502641275	-0.001495361328125	\\
0.827495893816309	-0.00140380859375	\\
0.827540284991344	-0.001373291015625	\\
0.827584676166378	-0.00177001953125	\\
0.827629067341413	-0.001190185546875	\\
0.827673458516447	-0.001373291015625	\\
0.827717849691481	-0.001708984375	\\
0.827762240866516	-0.00189208984375	\\
0.82780663204155	-0.002197265625	\\
0.827851023216585	-0.00225830078125	\\
0.827895414391619	-0.002227783203125	\\
0.827939805566653	-0.001922607421875	\\
0.827984196741688	-0.002288818359375	\\
0.828028587916722	-0.002716064453125	\\
0.828072979091757	-0.002716064453125	\\
0.828117370266791	-0.003265380859375	\\
0.828161761441825	-0.003082275390625	\\
0.82820615261686	-0.0028076171875	\\
0.828250543791894	-0.0030517578125	\\
0.828294934966929	-0.00299072265625	\\
0.828339326141963	-0.0028076171875	\\
0.828383717316997	-0.00299072265625	\\
0.828428108492032	-0.003082275390625	\\
0.828472499667066	-0.0029296875	\\
0.828516890842101	-0.00250244140625	\\
0.828561282017135	-0.002471923828125	\\
0.828605673192169	-0.00262451171875	\\
0.828650064367204	-0.0028076171875	\\
0.828694455542238	-0.002532958984375	\\
0.828738846717273	-0.00262451171875	\\
0.828783237892307	-0.002960205078125	\\
0.828827629067341	-0.0025634765625	\\
0.828872020242376	-0.00238037109375	\\
0.82891641141741	-0.002288818359375	\\
0.828960802592445	-0.002471923828125	\\
0.829005193767479	-0.00244140625	\\
0.829049584942513	-0.00238037109375	\\
0.829093976117548	-0.002166748046875	\\
0.829138367292582	-0.00201416015625	\\
0.829182758467617	-0.001983642578125	\\
0.829227149642651	-0.0015869140625	\\
0.829271540817685	-0.001312255859375	\\
0.82931593199272	-0.00189208984375	\\
0.829360323167754	-0.00201416015625	\\
0.829404714342789	-0.002044677734375	\\
0.829449105517823	-0.00189208984375	\\
0.829493496692857	-0.001708984375	\\
0.829537887867892	-0.0020751953125	\\
0.829582279042926	-0.002166748046875	\\
0.829626670217961	-0.00244140625	\\
0.829671061392995	-0.00250244140625	\\
0.829715452568029	-0.002716064453125	\\
0.829759843743064	-0.00250244140625	\\
0.829804234918098	-0.0023193359375	\\
0.829848626093133	-0.002593994140625	\\
0.829893017268167	-0.002532958984375	\\
0.829937408443202	-0.002227783203125	\\
0.829981799618236	-0.00244140625	\\
0.83002619079327	-0.0028076171875	\\
0.830070581968305	-0.00335693359375	\\
0.830114973143339	-0.00347900390625	\\
0.830159364318374	-0.00341796875	\\
0.830203755493408	-0.003265380859375	\\
0.830248146668442	-0.003509521484375	\\
0.830292537843477	-0.003692626953125	\\
0.830336929018511	-0.0035400390625	\\
0.830381320193545	-0.0032958984375	\\
0.83042571136858	-0.003448486328125	\\
0.830470102543614	-0.00360107421875	\\
0.830514493718649	-0.00347900390625	\\
0.830558884893683	-0.00384521484375	\\
0.830603276068718	-0.00335693359375	\\
0.830647667243752	-0.00286865234375	\\
0.830692058418786	-0.003326416015625	\\
0.830736449593821	-0.003509521484375	\\
0.830780840768855	-0.00311279296875	\\
0.83082523194389	-0.003082275390625	\\
0.830869623118924	-0.00323486328125	\\
0.830914014293958	-0.003326416015625	\\
0.830958405468993	-0.003021240234375	\\
0.831002796644027	-0.002593994140625	\\
0.831047187819062	-0.00262451171875	\\
0.831091578994096	-0.00250244140625	\\
0.83113597016913	-0.002288818359375	\\
0.831180361344165	-0.00244140625	\\
0.831224752519199	-0.002288818359375	\\
0.831269143694234	-0.00244140625	\\
0.831313534869268	-0.002716064453125	\\
0.831357926044302	-0.0025634765625	\\
0.831402317219337	-0.002716064453125	\\
0.831446708394371	-0.002899169921875	\\
0.831491099569406	-0.002471923828125	\\
0.83153549074444	-0.00250244140625	\\
0.831579881919474	-0.0028076171875	\\
0.831624273094509	-0.002410888671875	\\
0.831668664269543	-0.002471923828125	\\
0.831713055444578	-0.00250244140625	\\
0.831757446619612	-0.00244140625	\\
0.831801837794646	-0.00262451171875	\\
0.831846228969681	-0.00262451171875	\\
0.831890620144715	-0.0025634765625	\\
0.83193501131975	-0.00274658203125	\\
0.831979402494784	-0.002532958984375	\\
0.832023793669818	-0.002838134765625	\\
0.832068184844853	-0.00311279296875	\\
0.832112576019887	-0.003082275390625	\\
0.832156967194922	-0.003265380859375	\\
0.832201358369956	-0.003082275390625	\\
0.83224574954499	-0.002716064453125	\\
0.832290140720025	-0.00286865234375	\\
0.832334531895059	-0.002532958984375	\\
0.832378923070094	-0.002655029296875	\\
0.832423314245128	-0.002288818359375	\\
0.832467705420162	-0.002349853515625	\\
0.832512096595197	-0.002655029296875	\\
0.832556487770231	-0.00244140625	\\
0.832600878945266	-0.00274658203125	\\
0.8326452701203	-0.0025634765625	\\
0.832689661295335	-0.00238037109375	\\
0.832734052470369	-0.003082275390625	\\
0.832778443645403	-0.00323486328125	\\
0.832822834820438	-0.002593994140625	\\
0.832867225995472	-0.00238037109375	\\
0.832911617170507	-0.00201416015625	\\
0.832956008345541	-0.001983642578125	\\
0.833000399520575	-0.00201416015625	\\
0.83304479069561	-0.001678466796875	\\
0.833089181870644	-0.00146484375	\\
0.833133573045678	-0.00177001953125	\\
0.833177964220713	-0.001739501953125	\\
0.833222355395747	-0.00164794921875	\\
0.833266746570782	-0.001739501953125	\\
0.833311137745816	-0.00164794921875	\\
0.833355528920851	-0.001800537109375	\\
0.833399920095885	-0.00152587890625	\\
0.833444311270919	-0.00152587890625	\\
0.833488702445954	-0.001312255859375	\\
0.833533093620988	-0.00128173828125	\\
0.833577484796023	-0.001190185546875	\\
0.833621875971057	-0.000823974609375	\\
0.833666267146091	-0.0009765625	\\
0.833710658321126	-0.000946044921875	\\
0.83375504949616	-0.0015869140625	\\
0.833799440671195	-0.001678466796875	\\
0.833843831846229	-0.001739501953125	\\
0.833888223021263	-0.00213623046875	\\
0.833932614196298	-0.00164794921875	\\
0.833977005371332	-0.0009765625	\\
0.834021396546367	-0.001220703125	\\
0.834065787721401	-0.001190185546875	\\
0.834110178896435	-0.0010986328125	\\
0.83415457007147	-0.0015869140625	\\
0.834198961246504	-0.001373291015625	\\
0.834243352421539	-0.001129150390625	\\
0.834287743596573	-0.00164794921875	\\
0.834332134771607	-0.00213623046875	\\
0.834376525946642	-0.00177001953125	\\
0.834420917121676	-0.001708984375	\\
0.834465308296711	-0.00140380859375	\\
0.834509699471745	-0.001251220703125	\\
0.834554090646779	-0.001739501953125	\\
0.834598481821814	-0.001861572265625	\\
0.834642872996848	-0.001312255859375	\\
0.834687264171883	-0.001373291015625	\\
0.834731655346917	-0.001800537109375	\\
0.834776046521951	-0.00189208984375	\\
0.834820437696986	-0.00146484375	\\
0.83486482887202	-0.001190185546875	\\
0.834909220047055	-0.0018310546875	\\
0.834953611222089	-0.001373291015625	\\
0.834998002397123	-0.000885009765625	\\
0.835042393572158	-0.001190185546875	\\
0.835086784747192	-0.0008544921875	\\
0.835131175922227	-0.00054931640625	\\
0.835175567097261	-0.0006103515625	\\
0.835219958272295	-0.000457763671875	\\
0.83526434944733	-0.000396728515625	\\
0.835308740622364	-0.0001220703125	\\
0.835353131797399	-0.000274658203125	\\
0.835397522972433	-3.0517578125e-05	\\
0.835441914147467	-0.000213623046875	\\
0.835486305322502	-0.000579833984375	\\
0.835530696497536	-0.000518798828125	\\
0.835575087672571	-0.00091552734375	\\
0.835619478847605	-0.00115966796875	\\
0.83566387002264	-0.000732421875	\\
0.835708261197674	-0.00079345703125	\\
0.835752652372708	-0.001068115234375	\\
0.835797043547743	-0.000396728515625	\\
0.835841434722777	-0.00079345703125	\\
0.835885825897811	-0.0009765625	\\
0.835930217072846	-0.000579833984375	\\
0.83597460824788	-0.00128173828125	\\
0.836018999422915	-0.001373291015625	\\
0.836063390597949	-0.00164794921875	\\
0.836107781772984	-0.001678466796875	\\
0.836152172948018	-0.001495361328125	\\
0.836196564123052	-0.00140380859375	\\
0.836240955298087	-0.001373291015625	\\
0.836285346473121	-0.001708984375	\\
0.836329737648156	-0.001739501953125	\\
0.83637412882319	-0.001617431640625	\\
0.836418519998224	-0.002197265625	\\
0.836462911173259	-0.001983642578125	\\
0.836507302348293	-0.0015869140625	\\
0.836551693523328	-0.001983642578125	\\
0.836596084698362	-0.00225830078125	\\
0.836640475873396	-0.002227783203125	\\
0.836684867048431	-0.002532958984375	\\
0.836729258223465	-0.002197265625	\\
0.8367736493985	-0.00189208984375	\\
0.836818040573534	-0.002105712890625	\\
0.836862431748568	-0.001800537109375	\\
0.836906822923603	-0.001953125	\\
0.836951214098637	-0.002044677734375	\\
0.836995605273672	-0.001739501953125	\\
0.837039996448706	-0.0020751953125	\\
0.83708438762374	-0.002044677734375	\\
0.837128778798775	-0.00177001953125	\\
0.837173169973809	-0.00164794921875	\\
0.837217561148844	-0.00164794921875	\\
0.837261952323878	-0.0013427734375	\\
0.837306343498912	-0.00177001953125	\\
0.837350734673947	-0.00177001953125	\\
0.837395125848981	-0.00128173828125	\\
0.837439517024016	-0.001129150390625	\\
0.83748390819905	-0.0008544921875	\\
0.837528299374084	-0.001129150390625	\\
0.837572690549119	-0.0015869140625	\\
0.837617081724153	-0.001556396484375	\\
0.837661472899188	-0.001800537109375	\\
0.837705864074222	-0.00152587890625	\\
0.837750255249256	-0.001556396484375	\\
0.837794646424291	-0.00164794921875	\\
0.837839037599325	-0.0018310546875	\\
0.83788342877436	-0.0023193359375	\\
0.837927819949394	-0.002197265625	\\
0.837972211124428	-0.0018310546875	\\
0.838016602299463	-0.001922607421875	\\
0.838060993474497	-0.002166748046875	\\
0.838105384649532	-0.00238037109375	\\
0.838149775824566	-0.0020751953125	\\
0.8381941669996	-0.002166748046875	\\
0.838238558174635	-0.001861572265625	\\
0.838282949349669	-0.0015869140625	\\
0.838327340524704	-0.001739501953125	\\
0.838371731699738	-0.001739501953125	\\
0.838416122874773	-0.00115966796875	\\
0.838460514049807	-0.000701904296875	\\
0.838504905224841	-0.001190185546875	\\
0.838549296399876	-0.001068115234375	\\
0.83859368757491	-0.000579833984375	\\
0.838638078749945	-0.000762939453125	\\
0.838682469924979	-0.000946044921875	\\
0.838726861100013	-0.001007080078125	\\
0.838771252275048	-0.001007080078125	\\
0.838815643450082	-0.00067138671875	\\
0.838860034625116	-0.001007080078125	\\
0.838904425800151	-0.000335693359375	\\
0.838948816975185	-0.0001220703125	\\
0.83899320815022	-0.00067138671875	\\
0.839037599325254	-0.0006103515625	\\
0.839081990500289	-0.000213623046875	\\
0.839126381675323	-0.00018310546875	\\
0.839170772850357	-0.0001220703125	\\
0.839215164025392	-3.0517578125e-05	\\
0.839259555200426	-0.000457763671875	\\
0.839303946375461	-0.00030517578125	\\
0.839348337550495	-0.000396728515625	\\
0.839392728725529	-0.000762939453125	\\
0.839437119900564	-0.00048828125	\\
0.839481511075598	-0.0008544921875	\\
0.839525902250633	-0.000946044921875	\\
0.839570293425667	-0.000640869140625	\\
0.839614684600701	-0.0003662109375	\\
0.839659075775736	-0.000579833984375	\\
0.83970346695077	-0.000732421875	\\
0.839747858125805	-0.000640869140625	\\
0.839792249300839	-0.000640869140625	\\
0.839836640475873	-0.00048828125	\\
0.839881031650908	-0.000244140625	\\
0.839925422825942	-0.00030517578125	\\
0.839969814000977	-0.000885009765625	\\
0.840014205176011	-0.00091552734375	\\
0.840058596351045	-0.00103759765625	\\
0.84010298752608	-0.0010986328125	\\
0.840147378701114	-0.000823974609375	\\
0.840191769876149	-0.0006103515625	\\
0.840236161051183	-0.000579833984375	\\
0.840280552226217	-0.000274658203125	\\
0.840324943401252	-0.000213623046875	\\
0.840369334576286	-0.00018310546875	\\
0.840413725751321	-3.0517578125e-05	\\
0.840458116926355	-0.000335693359375	\\
0.840502508101389	-0.000396728515625	\\
0.840546899276424	-0.000762939453125	\\
0.840591290451458	-0.0009765625	\\
0.840635681626493	-0.00067138671875	\\
0.840680072801527	-0.000335693359375	\\
0.840724463976561	-0.000213623046875	\\
0.840768855151596	-0.000885009765625	\\
0.84081324632663	-0.000457763671875	\\
0.840857637501665	0	\\
0.840902028676699	-3.0517578125e-05	\\
0.840946419851733	6.103515625e-05	\\
0.840990811026768	-6.103515625e-05	\\
0.841035202201802	0.00018310546875	\\
0.841079593376837	0.00018310546875	\\
0.841123984551871	6.103515625e-05	\\
0.841168375726906	9.1552734375e-05	\\
0.84121276690194	9.1552734375e-05	\\
0.841257158076974	-0.0001220703125	\\
0.841301549252009	9.1552734375e-05	\\
0.841345940427043	0.00030517578125	\\
0.841390331602078	0.0001220703125	\\
0.841434722777112	0.000244140625	\\
0.841479113952146	0.0001220703125	\\
0.841523505127181	-0.00018310546875	\\
0.841567896302215	-0.000274658203125	\\
0.841612287477249	6.103515625e-05	\\
0.841656678652284	0.000274658203125	\\
0.841701069827318	0.000518798828125	\\
0.841745461002353	0.000213623046875	\\
0.841789852177387	0.000213623046875	\\
0.841834243352422	0.00067138671875	\\
0.841878634527456	0.000274658203125	\\
0.84192302570249	-9.1552734375e-05	\\
0.841967416877525	0.000244140625	\\
0.842011808052559	0.000274658203125	\\
0.842056199227594	0.000213623046875	\\
0.842100590402628	0.000244140625	\\
0.842144981577662	0.000213623046875	\\
0.842189372752697	0.000701904296875	\\
0.842233763927731	0.000640869140625	\\
0.842278155102766	0.0006103515625	\\
0.8423225462778	0.000579833984375	\\
0.842366937452834	0.0001220703125	\\
0.842411328627869	0.00054931640625	\\
0.842455719802903	0.000244140625	\\
0.842500110977938	0.000335693359375	\\
0.842544502152972	0.00030517578125	\\
0.842588893328006	0.000274658203125	\\
0.842633284503041	0.000701904296875	\\
0.842677675678075	0.000579833984375	\\
0.84272206685311	0.000579833984375	\\
0.842766458028144	0.00091552734375	\\
0.842810849203178	0.000946044921875	\\
0.842855240378213	0.000640869140625	\\
0.842899631553247	0.000885009765625	\\
0.842944022728282	0.001007080078125	\\
0.842988413903316	0.001007080078125	\\
0.84303280507835	0.001190185546875	\\
0.843077196253385	0.001129150390625	\\
0.843121587428419	0.00152587890625	\\
0.843165978603454	0.001190185546875	\\
0.843210369778488	0.000701904296875	\\
0.843254760953522	0.000823974609375	\\
0.843299152128557	0.0006103515625	\\
0.843343543303591	0.000732421875	\\
0.843387934478626	0.00079345703125	\\
0.84343232565366	0.000518798828125	\\
0.843476716828694	0.000732421875	\\
0.843521108003729	0.00103759765625	\\
0.843565499178763	0.000732421875	\\
0.843609890353798	0.001129150390625	\\
0.843654281528832	0.00115966796875	\\
0.843698672703866	0.0006103515625	\\
0.843743063878901	0.000640869140625	\\
0.843787455053935	0.000762939453125	\\
0.84383184622897	0.000518798828125	\\
0.843876237404004	0.000579833984375	\\
0.843920628579038	0.000396728515625	\\
0.843965019754073	0.000396728515625	\\
0.844009410929107	0.000244140625	\\
0.844053802104142	-0.000213623046875	\\
0.844098193279176	-0.000335693359375	\\
0.844142584454211	0.000213623046875	\\
0.844186975629245	0.0001220703125	\\
0.844231366804279	-0.00018310546875	\\
0.844275757979314	0.0006103515625	\\
0.844320149154348	0.000640869140625	\\
0.844364540329383	0.00030517578125	\\
0.844408931504417	0.000274658203125	\\
0.844453322679451	0.000152587890625	\\
0.844497713854486	0.000335693359375	\\
0.84454210502952	0.000396728515625	\\
0.844586496204555	-3.0517578125e-05	\\
0.844630887379589	0.000213623046875	\\
0.844675278554623	0.00048828125	\\
0.844719669729658	0.000457763671875	\\
0.844764060904692	0.00018310546875	\\
0.844808452079727	0.00018310546875	\\
0.844852843254761	0.00018310546875	\\
0.844897234429795	-0.00018310546875	\\
0.84494162560483	-0.00054931640625	\\
0.844986016779864	-0.00030517578125	\\
0.845030407954899	-0.00048828125	\\
0.845074799129933	-0.00067138671875	\\
0.845119190304967	-0.000244140625	\\
0.845163581480002	-6.103515625e-05	\\
0.845207972655036	0.000244140625	\\
0.845252363830071	-0.000152587890625	\\
0.845296755005105	-0.00048828125	\\
0.845341146180139	-0.00018310546875	\\
0.845385537355174	-0.000244140625	\\
0.845429928530208	-0.000213623046875	\\
0.845474319705243	-0.000213623046875	\\
0.845518710880277	-3.0517578125e-05	\\
0.845563102055311	-0.0003662109375	\\
0.845607493230346	-0.000762939453125	\\
0.84565188440538	-0.000823974609375	\\
0.845696275580415	-0.001007080078125	\\
0.845740666755449	-0.0009765625	\\
0.845785057930483	-0.000518798828125	\\
0.845829449105518	-0.000885009765625	\\
0.845873840280552	-0.001190185546875	\\
0.845918231455587	-0.001190185546875	\\
0.845962622630621	-0.001434326171875	\\
0.846007013805655	-0.001068115234375	\\
0.84605140498069	-0.00091552734375	\\
0.846095796155724	-0.001068115234375	\\
0.846140187330759	-0.000762939453125	\\
0.846184578505793	-0.000732421875	\\
0.846228969680828	-0.001007080078125	\\
0.846273360855862	-0.000732421875	\\
0.846317752030896	-0.000823974609375	\\
0.846362143205931	-0.001007080078125	\\
0.846406534380965	-0.00067138671875	\\
0.846450925555999	-0.00054931640625	\\
0.846495316731034	-0.0010986328125	\\
0.846539707906068	-0.000640869140625	\\
0.846584099081103	-0.000152587890625	\\
0.846628490256137	6.103515625e-05	\\
0.846672881431171	9.1552734375e-05	\\
0.846717272606206	-0.000274658203125	\\
0.84676166378124	-6.103515625e-05	\\
0.846806054956275	0.0001220703125	\\
0.846850446131309	0.000244140625	\\
0.846894837306344	0.0003662109375	\\
0.846939228481378	0.000457763671875	\\
0.846983619656412	0.0003662109375	\\
0.847028010831447	-6.103515625e-05	\\
0.847072402006481	-0.000244140625	\\
0.847116793181516	-0.00018310546875	\\
0.84716118435655	-0.000518798828125	\\
0.847205575531584	-0.000518798828125	\\
0.847249966706619	0.0001220703125	\\
0.847294357881653	0.0006103515625	\\
0.847338749056687	-0.000213623046875	\\
0.847383140231722	-0.000274658203125	\\
0.847427531406756	6.103515625e-05	\\
0.847471922581791	0.00042724609375	\\
0.847516313756825	0.000213623046875	\\
0.84756070493186	0.000457763671875	\\
0.847605096106894	0.00054931640625	\\
0.847649487281928	0.00054931640625	\\
0.847693878456963	0.000701904296875	\\
0.847738269631997	0.00030517578125	\\
0.847782660807032	0.000457763671875	\\
0.847827051982066	0.0003662109375	\\
0.8478714431571	0.0003662109375	\\
0.847915834332135	0.00042724609375	\\
0.847960225507169	0.000457763671875	\\
0.848004616682204	0.00030517578125	\\
0.848049007857238	0.000274658203125	\\
0.848093399032272	0.000396728515625	\\
0.848137790207307	0.000762939453125	\\
0.848182181382341	0.0009765625	\\
0.848226572557376	0.001129150390625	\\
0.84827096373241	0.001068115234375	\\
0.848315354907444	0.000946044921875	\\
0.848359746082479	0.0006103515625	\\
0.848404137257513	0.000732421875	\\
0.848448528432548	0.00048828125	\\
0.848492919607582	0.00030517578125	\\
0.848537310782616	0.0003662109375	\\
0.848581701957651	0.0006103515625	\\
0.848626093132685	0.001007080078125	\\
0.84867048430772	0.00103759765625	\\
0.848714875482754	0.0008544921875	\\
0.848759266657788	0.000244140625	\\
0.848803657832823	0.0003662109375	\\
0.848848049007857	0.00042724609375	\\
0.848892440182892	0.00067138671875	\\
0.848936831357926	0.00067138671875	\\
0.84898122253296	0.00054931640625	\\
0.849025613707995	0.001068115234375	\\
0.849070004883029	0.001251220703125	\\
0.849114396058064	0.000946044921875	\\
0.849158787233098	0.000640869140625	\\
0.849203178408132	0.0006103515625	\\
0.849247569583167	0.00042724609375	\\
0.849291960758201	0.000335693359375	\\
0.849336351933236	0.000213623046875	\\
0.84938074310827	0.000396728515625	\\
0.849425134283304	0.0010986328125	\\
0.849469525458339	0.000732421875	\\
0.849513916633373	0.00048828125	\\
0.849558307808408	0.00079345703125	\\
0.849602698983442	0.00079345703125	\\
0.849647090158477	0.0006103515625	\\
0.849691481333511	0.000640869140625	\\
0.849735872508545	0.000213623046875	\\
0.84978026368358	-0.000213623046875	\\
0.849824654858614	0.000152587890625	\\
0.849869046033649	9.1552734375e-05	\\
0.849913437208683	-0.000213623046875	\\
0.849957828383717	-3.0517578125e-05	\\
0.850002219558752	-0.000152587890625	\\
0.850046610733786	-0.0003662109375	\\
0.85009100190882	-9.1552734375e-05	\\
0.850135393083855	0	\\
0.850179784258889	0.000213623046875	\\
0.850224175433924	0.000396728515625	\\
0.850268566608958	0.00042724609375	\\
0.850312957783993	0.00030517578125	\\
0.850357348959027	0.00042724609375	\\
0.850401740134061	0.000823974609375	\\
0.850446131309096	0.00030517578125	\\
0.85049052248413	-0.0003662109375	\\
0.850534913659165	-0.00018310546875	\\
0.850579304834199	0.0001220703125	\\
0.850623696009233	-0.000396728515625	\\
0.850668087184268	-0.00030517578125	\\
0.850712478359302	-6.103515625e-05	\\
0.850756869534337	-3.0517578125e-05	\\
0.850801260709371	0.0001220703125	\\
0.850845651884405	0.00018310546875	\\
0.85089004305944	0.0001220703125	\\
0.850934434234474	0.000274658203125	\\
0.850978825409509	0.00067138671875	\\
0.851023216584543	0.0006103515625	\\
0.851067607759577	0.000885009765625	\\
0.851111998934612	0.000579833984375	\\
0.851156390109646	0.000732421875	\\
0.851200781284681	0.0006103515625	\\
0.851245172459715	0.000335693359375	\\
0.851289563634749	0.00103759765625	\\
0.851333954809784	0.0008544921875	\\
0.851378345984818	3.0517578125e-05	\\
0.851422737159853	0.000274658203125	\\
0.851467128334887	0.000396728515625	\\
0.851511519509921	0.000274658203125	\\
0.851555910684956	0.000518798828125	\\
0.85160030185999	0.0001220703125	\\
0.851644693035025	0.0003662109375	\\
0.851689084210059	0.000335693359375	\\
0.851733475385093	0.0001220703125	\\
0.851777866560128	3.0517578125e-05	\\
0.851822257735162	-9.1552734375e-05	\\
0.851866648910197	-6.103515625e-05	\\
0.851911040085231	-0.000152587890625	\\
0.851955431260265	-0.000274658203125	\\
0.8519998224353	-0.00030517578125	\\
0.852044213610334	-0.000640869140625	\\
0.852088604785369	-0.000518798828125	\\
0.852132995960403	-9.1552734375e-05	\\
0.852177387135437	-0.00091552734375	\\
0.852221778310472	-0.0009765625	\\
0.852266169485506	-0.001190185546875	\\
0.852310560660541	-0.00128173828125	\\
0.852354951835575	-0.000885009765625	\\
0.852399343010609	-0.00128173828125	\\
0.852443734185644	-0.001129150390625	\\
0.852488125360678	-0.00079345703125	\\
0.852532516535713	-0.000885009765625	\\
0.852576907710747	-0.000518798828125	\\
0.852621298885782	-0.0003662109375	\\
0.852665690060816	-0.00067138671875	\\
0.85271008123585	-0.000244140625	\\
0.852754472410885	-0.000274658203125	\\
0.852798863585919	-0.00079345703125	\\
0.852843254760954	-0.000885009765625	\\
0.852887645935988	-0.000457763671875	\\
0.852932037111022	-0.00067138671875	\\
0.852976428286057	-0.0009765625	\\
0.853020819461091	-0.00067138671875	\\
0.853065210636126	-0.000823974609375	\\
0.85310960181116	-0.0009765625	\\
0.853153992986194	-0.0003662109375	\\
0.853198384161229	-0.00054931640625	\\
0.853242775336263	-0.0008544921875	\\
0.853287166511298	-0.00079345703125	\\
0.853331557686332	-0.001007080078125	\\
0.853375948861366	-0.000457763671875	\\
0.853420340036401	-0.00018310546875	\\
0.853464731211435	-0.00067138671875	\\
0.85350912238647	-0.001007080078125	\\
0.853553513561504	-0.000885009765625	\\
0.853597904736538	-0.00103759765625	\\
0.853642295911573	-0.0013427734375	\\
0.853686687086607	-0.001556396484375	\\
0.853731078261642	-0.001678466796875	\\
0.853775469436676	-0.001495361328125	\\
0.85381986061171	-0.00140380859375	\\
0.853864251786745	-0.00146484375	\\
0.853908642961779	-0.001220703125	\\
0.853953034136814	-0.001007080078125	\\
0.853997425311848	-0.001220703125	\\
0.854041816486882	-0.00146484375	\\
0.854086207661917	-0.00140380859375	\\
0.854130598836951	-0.00164794921875	\\
0.854174990011986	-0.00140380859375	\\
0.85421938118702	-0.001251220703125	\\
0.854263772362054	-0.00128173828125	\\
0.854308163537089	-0.001068115234375	\\
0.854352554712123	-0.001068115234375	\\
0.854396945887158	-0.000823974609375	\\
0.854441337062192	-0.00091552734375	\\
0.854485728237226	-0.001068115234375	\\
0.854530119412261	-0.000518798828125	\\
0.854574510587295	-0.000396728515625	\\
0.85461890176233	-0.0009765625	\\
0.854663292937364	-0.001190185546875	\\
0.854707684112399	-0.0008544921875	\\
0.854752075287433	-0.001373291015625	\\
0.854796466462467	-0.001556396484375	\\
0.854840857637502	-0.00091552734375	\\
0.854885248812536	-0.000885009765625	\\
0.85492963998757	-0.00079345703125	\\
0.854974031162605	-0.000762939453125	\\
0.855018422337639	-0.001251220703125	\\
0.855062813512674	-0.001251220703125	\\
0.855107204687708	-0.0010986328125	\\
0.855151595862742	-0.001190185546875	\\
0.855195987037777	-0.00091552734375	\\
0.855240378212811	-0.00103759765625	\\
0.855284769387846	-0.0010986328125	\\
0.85532916056288	-0.0010986328125	\\
0.855373551737915	-0.000946044921875	\\
0.855417942912949	-0.00091552734375	\\
0.855462334087983	-0.000885009765625	\\
0.855506725263018	-0.00067138671875	\\
0.855551116438052	-0.000762939453125	\\
0.855595507613087	-0.001129150390625	\\
0.855639898788121	-0.001068115234375	\\
0.855684289963155	-0.001007080078125	\\
0.85572868113819	-0.0010986328125	\\
0.855773072313224	-0.000885009765625	\\
0.855817463488258	-0.0010986328125	\\
0.855861854663293	-0.00103759765625	\\
0.855906245838327	-0.0008544921875	\\
0.855950637013362	-0.0009765625	\\
0.855995028188396	-0.0008544921875	\\
0.856039419363431	-0.000823974609375	\\
0.856083810538465	-0.000152587890625	\\
0.856128201713499	0.00018310546875	\\
0.856172592888534	0.00042724609375	\\
0.856216984063568	0.00042724609375	\\
0.856261375238603	0.000213623046875	\\
0.856305766413637	0.000396728515625	\\
0.856350157588671	0.0003662109375	\\
0.856394548763706	0.000579833984375	\\
0.85643893993874	0.0008544921875	\\
0.856483331113775	0.00054931640625	\\
0.856527722288809	0.00054931640625	\\
0.856572113463843	0.001129150390625	\\
0.856616504638878	0.00079345703125	\\
0.856660895813912	0.001251220703125	\\
0.856705286988947	0.00128173828125	\\
0.856749678163981	0.000701904296875	\\
0.856794069339015	0.0013427734375	\\
0.85683846051405	0.00146484375	\\
0.856882851689084	0.001129150390625	\\
0.856927242864119	0.00115966796875	\\
0.856971634039153	0.00103759765625	\\
0.857016025214187	0.0008544921875	\\
0.857060416389222	0.001007080078125	\\
0.857104807564256	0.00091552734375	\\
0.857149198739291	0.00054931640625	\\
0.857193589914325	0.000457763671875	\\
0.857237981089359	0.00042724609375	\\
0.857282372264394	0.00054931640625	\\
0.857326763439428	0.00030517578125	\\
0.857371154614463	0.000213623046875	\\
0.857415545789497	0.0006103515625	\\
0.857459936964531	0.000396728515625	\\
0.857504328139566	0.000396728515625	\\
0.8575487193146	0.001007080078125	\\
0.857593110489635	0.000335693359375	\\
0.857637501664669	0.00018310546875	\\
0.857681892839703	0.00030517578125	\\
0.857726284014738	-0.000213623046875	\\
0.857770675189772	6.103515625e-05	\\
0.857815066364807	0.000152587890625	\\
0.857859457539841	0.000244140625	\\
0.857903848714875	0.000213623046875	\\
0.85794823988991	0.00048828125	\\
0.857992631064944	0.00042724609375	\\
0.858037022239979	-0.000152587890625	\\
0.858081413415013	-0.000457763671875	\\
0.858125804590048	-0.000274658203125	\\
0.858170195765082	-0.0001220703125	\\
0.858214586940116	-0.000152587890625	\\
0.858258978115151	0	\\
0.858303369290185	0.0001220703125	\\
0.85834776046522	0.00042724609375	\\
0.858392151640254	0.00042724609375	\\
0.858436542815288	0.00054931640625	\\
0.858480933990323	0.000823974609375	\\
0.858525325165357	0.0009765625	\\
0.858569716340391	0.00048828125	\\
0.858614107515426	0.000396728515625	\\
0.85865849869046	0.00079345703125	\\
0.858702889865495	0.0009765625	\\
0.858747281040529	0.000823974609375	\\
0.858791672215564	0.0008544921875	\\
0.858836063390598	0.000457763671875	\\
0.858880454565632	0.00030517578125	\\
0.858924845740667	0.000213623046875	\\
0.858969236915701	9.1552734375e-05	\\
0.859013628090736	-6.103515625e-05	\\
0.85905801926577	-0.00067138671875	\\
0.859102410440804	-0.000640869140625	\\
0.859146801615839	-0.00048828125	\\
0.859191192790873	-0.000396728515625	\\
0.859235583965908	-0.00042724609375	\\
0.859279975140942	-0.000396728515625	\\
0.859324366315976	-6.103515625e-05	\\
0.859368757491011	-9.1552734375e-05	\\
0.859413148666045	-0.00030517578125	\\
0.85945753984108	-0.000213623046875	\\
0.859501931016114	-0.00030517578125	\\
0.859546322191148	-3.0517578125e-05	\\
0.859590713366183	0.000213623046875	\\
0.859635104541217	-9.1552734375e-05	\\
0.859679495716252	0	\\
0.859723886891286	0.000335693359375	\\
0.85976827806632	3.0517578125e-05	\\
0.859812669241355	0.00030517578125	\\
0.859857060416389	9.1552734375e-05	\\
0.859901451591424	-0.000244140625	\\
0.859945842766458	-0.000335693359375	\\
0.859990233941492	-0.000335693359375	\\
0.860034625116527	-0.000152587890625	\\
0.860079016291561	-0.000274658203125	\\
0.860123407466596	-0.000396728515625	\\
0.86016779864163	-0.000213623046875	\\
0.860212189816664	-0.000335693359375	\\
0.860256580991699	-0.000732421875	\\
0.860300972166733	-0.00140380859375	\\
0.860345363341768	-0.001373291015625	\\
0.860389754516802	-0.001190185546875	\\
0.860434145691837	-0.00128173828125	\\
0.860478536866871	-0.0013427734375	\\
0.860522928041905	-0.001617431640625	\\
0.86056731921694	-0.001495361328125	\\
0.860611710391974	-0.0015869140625	\\
0.860656101567008	-0.001678466796875	\\
0.860700492742043	-0.001556396484375	\\
0.860744883917077	-0.001953125	\\
0.860789275092112	-0.002349853515625	\\
0.860833666267146	-0.00225830078125	\\
0.86087805744218	-0.002288818359375	\\
0.860922448617215	-0.002288818359375	\\
0.860966839792249	-0.00201416015625	\\
0.861011230967284	-0.001953125	\\
0.861055622142318	-0.001678466796875	\\
0.861100013317353	-0.00189208984375	\\
0.861144404492387	-0.001953125	\\
0.861188795667421	-0.001861572265625	\\
0.861233186842456	-0.002166748046875	\\
0.86127757801749	-0.00225830078125	\\
0.861321969192525	-0.001922607421875	\\
0.861366360367559	-0.00177001953125	\\
0.861410751542593	-0.002105712890625	\\
0.861455142717628	-0.001861572265625	\\
0.861499533892662	-0.00152587890625	\\
0.861543925067697	-0.0013427734375	\\
0.861588316242731	-0.001373291015625	\\
0.861632707417765	-0.001556396484375	\\
0.8616770985928	-0.00128173828125	\\
0.861721489767834	-0.001220703125	\\
0.861765880942869	-0.001129150390625	\\
0.861810272117903	-0.001190185546875	\\
0.861854663292937	-0.001434326171875	\\
0.861899054467972	-0.00146484375	\\
0.861943445643006	-0.001251220703125	\\
0.861987836818041	-0.00115966796875	\\
0.862032227993075	-0.001495361328125	\\
0.862076619168109	-0.00128173828125	\\
0.862121010343144	-0.00103759765625	\\
0.862165401518178	-0.000701904296875	\\
0.862209792693213	-0.00079345703125	\\
0.862254183868247	-0.0013427734375	\\
0.862298575043281	-0.001220703125	\\
0.862342966218316	-0.000823974609375	\\
0.86238735739335	-0.000732421875	\\
0.862431748568385	-0.000732421875	\\
0.862476139743419	-0.000823974609375	\\
0.862520530918453	-0.0015869140625	\\
0.862564922093488	-0.001129150390625	\\
0.862609313268522	-0.0010986328125	\\
0.862653704443557	-0.001617431640625	\\
0.862698095618591	-0.0013427734375	\\
0.862742486793625	-0.001220703125	\\
0.86278687796866	-0.001068115234375	\\
0.862831269143694	-0.001007080078125	\\
0.862875660318729	-0.000823974609375	\\
0.862920051493763	-0.001251220703125	\\
0.862964442668797	-0.00177001953125	\\
0.863008833843832	-0.001739501953125	\\
0.863053225018866	-0.00164794921875	\\
0.863097616193901	-0.00201416015625	\\
0.863142007368935	-0.0020751953125	\\
0.86318639854397	-0.001953125	\\
0.863230789719004	-0.002532958984375	\\
0.863275180894038	-0.0020751953125	\\
0.863319572069073	-0.001434326171875	\\
0.863363963244107	-0.001678466796875	\\
0.863408354419141	-0.001190185546875	\\
0.863452745594176	-0.001617431640625	\\
0.86349713676921	-0.0015869140625	\\
0.863541527944245	-0.001251220703125	\\
0.863585919119279	-0.00146484375	\\
0.863630310294313	-0.001708984375	\\
0.863674701469348	-0.0015869140625	\\
0.863719092644382	-0.001708984375	\\
0.863763483819417	-0.00164794921875	\\
0.863807874994451	-0.00164794921875	\\
0.863852266169486	-0.0023193359375	\\
0.86389665734452	-0.002410888671875	\\
0.863941048519554	-0.001922607421875	\\
0.863985439694589	-0.002410888671875	\\
0.864029830869623	-0.002288818359375	\\
0.864074222044658	-0.002044677734375	\\
0.864118613219692	-0.00177001953125	\\
0.864163004394726	-0.001556396484375	\\
0.864207395569761	-0.001373291015625	\\
0.864251786744795	-0.0010986328125	\\
0.864296177919829	-0.00091552734375	\\
0.864340569094864	-0.000732421875	\\
0.864384960269898	-0.000732421875	\\
0.864429351444933	-0.0006103515625	\\
0.864473742619967	-0.0003662109375	\\
0.864518133795002	-0.000396728515625	\\
0.864562524970036	-0.000701904296875	\\
0.86460691614507	-0.000885009765625	\\
0.864651307320105	-0.000823974609375	\\
0.864695698495139	-0.0008544921875	\\
0.864740089670174	-0.00091552734375	\\
0.864784480845208	-0.00115966796875	\\
0.864828872020242	-0.001556396484375	\\
0.864873263195277	-0.001739501953125	\\
0.864917654370311	-0.001708984375	\\
0.864962045545346	-0.001434326171875	\\
0.86500643672038	-0.00115966796875	\\
0.865050827895414	-0.000823974609375	\\
0.865095219070449	-0.000579833984375	\\
0.865139610245483	-0.00048828125	\\
0.865184001420518	0	\\
0.865228392595552	0.00030517578125	\\
0.865272783770586	0.000701904296875	\\
0.865317174945621	0.000457763671875	\\
0.865361566120655	0.0008544921875	\\
0.86540595729569	0.001251220703125	\\
0.865450348470724	0.0009765625	\\
0.865494739645758	0.000640869140625	\\
0.865539130820793	0.000640869140625	\\
0.865583521995827	0.000701904296875	\\
0.865627913170862	0.000732421875	\\
0.865672304345896	0.00054931640625	\\
0.86571669552093	0.00018310546875	\\
0.865761086695965	-3.0517578125e-05	\\
0.865805477870999	0.00018310546875	\\
0.865849869046034	0.000396728515625	\\
0.865894260221068	0.000274658203125	\\
0.865938651396102	0.000335693359375	\\
0.865983042571137	0.00067138671875	\\
0.866027433746171	0.00067138671875	\\
0.866071824921206	0.00030517578125	\\
0.86611621609624	0.000732421875	\\
0.866160607271274	0.0008544921875	\\
0.866204998446309	0.00079345703125	\\
0.866249389621343	0.0009765625	\\
0.866293780796378	0.0006103515625	\\
0.866338171971412	0.00079345703125	\\
0.866382563146446	0.00115966796875	\\
0.866426954321481	0.000518798828125	\\
0.866471345496515	0.000244140625	\\
0.86651573667155	6.103515625e-05	\\
0.866560127846584	0	\\
0.866604519021619	-0.0001220703125	\\
0.866648910196653	-0.000152587890625	\\
0.866693301371687	-0.000244140625	\\
0.866737692546722	-0.000274658203125	\\
0.866782083721756	-0.000213623046875	\\
0.866826474896791	-0.000732421875	\\
0.866870866071825	-0.0008544921875	\\
0.866915257246859	-0.001220703125	\\
0.866959648421894	-0.001129150390625	\\
0.867004039596928	-0.0013427734375	\\
0.867048430771963	-0.001708984375	\\
0.867092821946997	-0.00091552734375	\\
0.867137213122031	-0.00042724609375	\\
0.867181604297066	-0.000640869140625	\\
0.8672259954721	-0.000213623046875	\\
0.867270386647135	-0.00048828125	\\
0.867314777822169	-0.000640869140625	\\
0.867359168997203	-6.103515625e-05	\\
0.867403560172238	-0.000274658203125	\\
0.867447951347272	-0.000640869140625	\\
0.867492342522307	-0.0006103515625	\\
0.867536733697341	-0.000823974609375	\\
0.867581124872375	-0.00115966796875	\\
0.86762551604741	-0.001129150390625	\\
0.867669907222444	-0.0008544921875	\\
0.867714298397479	-0.00103759765625	\\
0.867758689572513	-0.00115966796875	\\
0.867803080747547	-0.000579833984375	\\
0.867847471922582	-0.0001220703125	\\
0.867891863097616	0.00048828125	\\
0.867936254272651	0.000946044921875	\\
0.867980645447685	0.0006103515625	\\
0.868025036622719	0.0003662109375	\\
0.868069427797754	0.000396728515625	\\
0.868113818972788	0.000274658203125	\\
0.868158210147823	0.0009765625	\\
0.868202601322857	0.001190185546875	\\
0.868246992497891	0.001434326171875	\\
0.868291383672926	0.00146484375	\\
0.86833577484796	0.00115966796875	\\
0.868380166022995	0.000701904296875	\\
0.868424557198029	0.000335693359375	\\
0.868468948373063	-0.000518798828125	\\
0.868513339548098	-0.000640869140625	\\
0.868557730723132	-0.000457763671875	\\
0.868602121898167	-0.00067138671875	\\
0.868646513073201	-0.000518798828125	\\
0.868690904248235	-0.000457763671875	\\
0.86873529542327	-0.000396728515625	\\
0.868779686598304	-0.000396728515625	\\
0.868824077773339	-0.000885009765625	\\
0.868868468948373	-0.000518798828125	\\
0.868912860123408	-0.000701904296875	\\
0.868957251298442	-0.000335693359375	\\
0.869001642473476	-0.0006103515625	\\
0.869046033648511	-0.0009765625	\\
0.869090424823545	-0.000701904296875	\\
0.869134815998579	-0.0010986328125	\\
0.869179207173614	-0.001556396484375	\\
0.869223598348648	-0.001708984375	\\
0.869267989523683	-0.00177001953125	\\
0.869312380698717	-0.001922607421875	\\
0.869356771873751	-0.00177001953125	\\
0.869401163048786	-0.0015869140625	\\
0.86944555422382	-0.00115966796875	\\
0.869489945398855	-0.001068115234375	\\
0.869534336573889	-0.000946044921875	\\
0.869578727748924	-0.0008544921875	\\
0.869623118923958	-0.000946044921875	\\
0.869667510098992	-0.001068115234375	\\
0.869711901274027	-0.00103759765625	\\
0.869756292449061	-0.001007080078125	\\
0.869800683624096	-0.001007080078125	\\
0.86984507479913	-0.001068115234375	\\
0.869889465974164	-0.001190185546875	\\
0.869933857149199	-0.0015869140625	\\
0.869978248324233	-0.00140380859375	\\
0.870022639499268	-0.00115966796875	\\
0.870067030674302	-0.000885009765625	\\
0.870111421849336	-0.000823974609375	\\
0.870155813024371	-0.0003662109375	\\
0.870200204199405	-0.0003662109375	\\
0.87024459537444	-0.00018310546875	\\
0.870288986549474	0.00018310546875	\\
0.870333377724508	-3.0517578125e-05	\\
0.870377768899543	0.0001220703125	\\
0.870422160074577	-0.00018310546875	\\
0.870466551249612	9.1552734375e-05	\\
0.870510942424646	0.0001220703125	\\
0.87055533359968	-0.000518798828125	\\
0.870599724774715	-0.000732421875	\\
0.870644115949749	-0.001007080078125	\\
0.870688507124784	-0.000823974609375	\\
0.870732898299818	-0.001190185546875	\\
0.870777289474852	-0.001434326171875	\\
0.870821680649887	-0.000885009765625	\\
0.870866071824921	-0.0009765625	\\
0.870910462999956	-0.00103759765625	\\
0.87095485417499	-0.000732421875	\\
0.870999245350024	-0.0003662109375	\\
0.871043636525059	-3.0517578125e-05	\\
0.871088027700093	-0.000396728515625	\\
0.871132418875128	-0.000701904296875	\\
0.871176810050162	-0.00067138671875	\\
0.871221201225196	-0.000762939453125	\\
0.871265592400231	-0.001129150390625	\\
0.871309983575265	-0.001190185546875	\\
0.8713543747503	-0.001129150390625	\\
0.871398765925334	-0.0015869140625	\\
0.871443157100368	-0.00213623046875	\\
0.871487548275403	-0.002410888671875	\\
0.871531939450437	-0.002349853515625	\\
0.871576330625472	-0.00244140625	\\
0.871620721800506	-0.002288818359375	\\
0.871665112975541	-0.001983642578125	\\
0.871709504150575	-0.00244140625	\\
0.871753895325609	-0.0023193359375	\\
0.871798286500644	-0.002044677734375	\\
0.871842677675678	-0.001983642578125	\\
0.871887068850712	-0.001678466796875	\\
0.871931460025747	-0.00128173828125	\\
0.871975851200781	-0.001495361328125	\\
0.872020242375816	-0.001312255859375	\\
0.87206463355085	-0.0013427734375	\\
0.872109024725884	-0.001678466796875	\\
0.872153415900919	-0.0018310546875	\\
0.872197807075953	-0.002166748046875	\\
0.872242198250988	-0.00250244140625	\\
0.872286589426022	-0.002410888671875	\\
0.872330980601057	-0.00250244140625	\\
0.872375371776091	-0.002777099609375	\\
0.872419762951125	-0.002685546875	\\
0.87246415412616	-0.00299072265625	\\
0.872508545301194	-0.002777099609375	\\
0.872552936476229	-0.002532958984375	\\
0.872597327651263	-0.00244140625	\\
0.872641718826297	-0.0020751953125	\\
0.872686110001332	-0.00177001953125	\\
0.872730501176366	-0.001434326171875	\\
0.8727748923514	-0.001617431640625	\\
0.872819283526435	-0.00128173828125	\\
0.872863674701469	-0.001190185546875	\\
0.872908065876504	-0.00079345703125	\\
0.872952457051538	-0.000457763671875	\\
0.872996848226573	-0.000518798828125	\\
0.873041239401607	-0.000396728515625	\\
0.873085630576641	-0.000823974609375	\\
0.873130021751676	-0.0009765625	\\
0.87317441292671	-0.00128173828125	\\
0.873218804101745	-0.00177001953125	\\
0.873263195276779	-0.001678466796875	\\
0.873307586451813	-0.00152587890625	\\
0.873351977626848	-0.001312255859375	\\
0.873396368801882	-0.001190185546875	\\
0.873440759976917	-0.0010986328125	\\
0.873485151151951	-0.0009765625	\\
0.873529542326985	-0.000732421875	\\
0.87357393350202	-0.000732421875	\\
0.873618324677054	-0.00079345703125	\\
0.873662715852089	-0.000946044921875	\\
0.873707107027123	-0.00115966796875	\\
0.873751498202157	-0.0006103515625	\\
0.873795889377192	-0.00042724609375	\\
0.873840280552226	-0.0003662109375	\\
0.873884671727261	-9.1552734375e-05	\\
0.873929062902295	-0.00030517578125	\\
0.873973454077329	0	\\
0.874017845252364	-0.0003662109375	\\
0.874062236427398	-0.0006103515625	\\
0.874106627602433	-0.000152587890625	\\
0.874151018777467	-0.000213623046875	\\
0.874195409952501	-0.00042724609375	\\
0.874239801127536	-0.00054931640625	\\
0.87428419230257	-0.00054931640625	\\
0.874328583477605	-0.00030517578125	\\
0.874372974652639	-0.000396728515625	\\
0.874417365827673	-0.0003662109375	\\
0.874461757002708	-0.00048828125	\\
0.874506148177742	-0.000274658203125	\\
0.874550539352777	-0.0001220703125	\\
0.874594930527811	-0.00018310546875	\\
0.874639321702846	-0.000701904296875	\\
0.87468371287788	-0.001129150390625	\\
0.874728104052914	-0.00067138671875	\\
0.874772495227949	-0.000579833984375	\\
0.874816886402983	-0.00054931640625	\\
0.874861277578017	-0.000396728515625	\\
0.874905668753052	-0.00030517578125	\\
0.874950059928086	-0.000152587890625	\\
0.874994451103121	-6.103515625e-05	\\
0.875038842278155	-0.00054931640625	\\
0.87508323345319	-0.00048828125	\\
0.875127624628224	-0.000701904296875	\\
0.875172015803258	-0.0008544921875	\\
0.875216406978293	-0.000457763671875	\\
0.875260798153327	-0.000579833984375	\\
0.875305189328362	-0.000762939453125	\\
0.875349580503396	-0.00146484375	\\
0.87539397167843	-0.002105712890625	\\
0.875438362853465	-0.001800537109375	\\
0.875482754028499	-0.00201416015625	\\
0.875527145203534	-0.002044677734375	\\
0.875571536378568	-0.001922607421875	\\
0.875615927553602	-0.001983642578125	\\
0.875660318728637	-0.002349853515625	\\
0.875704709903671	-0.001983642578125	\\
0.875749101078706	-0.001800537109375	\\
0.87579349225374	-0.0018310546875	\\
0.875837883428774	-0.001861572265625	\\
0.875882274603809	-0.002105712890625	\\
0.875926665778843	-0.00201416015625	\\
0.875971056953878	-0.002105712890625	\\
0.876015448128912	-0.0018310546875	\\
0.876059839303946	-0.002044677734375	\\
0.876104230478981	-0.0025634765625	\\
0.876148621654015	-0.002685546875	\\
0.87619301282905	-0.002777099609375	\\
0.876237404004084	-0.00274658203125	\\
0.876281795179118	-0.002471923828125	\\
0.876326186354153	-0.0023193359375	\\
0.876370577529187	-0.002593994140625	\\
0.876414968704222	-0.002471923828125	\\
0.876459359879256	-0.00189208984375	\\
0.87650375105429	-0.00213623046875	\\
0.876548142229325	-0.001800537109375	\\
0.876592533404359	-0.001953125	\\
0.876636924579394	-0.002532958984375	\\
0.876681315754428	-0.002166748046875	\\
0.876725706929462	-0.002227783203125	\\
0.876770098104497	-0.0023193359375	\\
0.876814489279531	-0.002471923828125	\\
0.876858880454566	-0.002593994140625	\\
0.8769032716296	-0.0028076171875	\\
0.876947662804634	-0.00323486328125	\\
0.876992053979669	-0.002899169921875	\\
0.877036445154703	-0.0028076171875	\\
0.877080836329738	-0.0029296875	\\
0.877125227504772	-0.002838134765625	\\
0.877169618679806	-0.002593994140625	\\
0.877214009854841	-0.002838134765625	\\
0.877258401029875	-0.0028076171875	\\
0.87730279220491	-0.002960205078125	\\
0.877347183379944	-0.00360107421875	\\
0.877391574554979	-0.003448486328125	\\
0.877435965730013	-0.003692626953125	\\
0.877480356905047	-0.004180908203125	\\
0.877524748080082	-0.00396728515625	\\
0.877569139255116	-0.0042724609375	\\
0.87761353043015	-0.004364013671875	\\
0.877657921605185	-0.00433349609375	\\
0.877702312780219	-0.004119873046875	\\
0.877746703955254	-0.003662109375	\\
0.877791095130288	-0.003936767578125	\\
0.877835486305322	-0.00408935546875	\\
0.877879877480357	-0.00360107421875	\\
0.877924268655391	-0.003692626953125	\\
0.877968659830426	-0.00390625	\\
0.87801305100546	-0.003265380859375	\\
0.878057442180495	-0.00311279296875	\\
0.878101833355529	-0.003082275390625	\\
0.878146224530563	-0.002593994140625	\\
0.878190615705598	-0.0025634765625	\\
0.878235006880632	-0.002410888671875	\\
0.878279398055667	-0.002532958984375	\\
0.878323789230701	-0.002838134765625	\\
0.878368180405735	-0.002685546875	\\
0.87841257158077	-0.00274658203125	\\
0.878456962755804	-0.0025634765625	\\
0.878501353930839	-0.003021240234375	\\
0.878545745105873	-0.003662109375	\\
0.878590136280907	-0.00396728515625	\\
0.878634527455942	-0.003814697265625	\\
0.878678918630976	-0.004150390625	\\
0.878723309806011	-0.00408935546875	\\
0.878767700981045	-0.003875732421875	\\
0.878812092156079	-0.004150390625	\\
0.878856483331114	-0.004119873046875	\\
0.878900874506148	-0.004180908203125	\\
0.878945265681183	-0.004119873046875	\\
0.878989656856217	-0.0035400390625	\\
0.879034048031251	-0.003662109375	\\
0.879078439206286	-0.003387451171875	\\
0.87912283038132	-0.0030517578125	\\
0.879167221556355	-0.0035400390625	\\
0.879211612731389	-0.003662109375	\\
0.879256003906423	-0.003509521484375	\\
0.879300395081458	-0.003875732421875	\\
0.879344786256492	-0.00372314453125	\\
0.879389177431527	-0.00335693359375	\\
0.879433568606561	-0.003387451171875	\\
0.879477959781595	-0.00341796875	\\
0.87952235095663	-0.00384521484375	\\
0.879566742131664	-0.004119873046875	\\
0.879611133306699	-0.003936767578125	\\
0.879655524481733	-0.00421142578125	\\
0.879699915656767	-0.00469970703125	\\
0.879744306831802	-0.004608154296875	\\
0.879788698006836	-0.004364013671875	\\
0.879833089181871	-0.004669189453125	\\
0.879877480356905	-0.00469970703125	\\
0.879921871531939	-0.004791259765625	\\
0.879966262706974	-0.004425048828125	\\
0.880010653882008	-0.0040283203125	\\
0.880055045057043	-0.003662109375	\\
0.880099436232077	-0.003326416015625	\\
0.880143827407112	-0.00347900390625	\\
0.880188218582146	-0.003692626953125	\\
0.88023260975718	-0.003448486328125	\\
0.880277000932215	-0.003021240234375	\\
0.880321392107249	-0.003204345703125	\\
0.880365783282283	-0.00396728515625	\\
0.880410174457318	-0.003753662109375	\\
0.880454565632352	-0.003662109375	\\
0.880498956807387	-0.00390625	\\
0.880543347982421	-0.004119873046875	\\
0.880587739157455	-0.0045166015625	\\
0.88063213033249	-0.004364013671875	\\
0.880676521507524	-0.00433349609375	\\
0.880720912682559	-0.004425048828125	\\
0.880765303857593	-0.00408935546875	\\
0.880809695032628	-0.004180908203125	\\
0.880854086207662	-0.00408935546875	\\
0.880898477382696	-0.00372314453125	\\
0.880942868557731	-0.0040283203125	\\
0.880987259732765	-0.0035400390625	\\
0.8810316509078	-0.003326416015625	\\
0.881076042082834	-0.003173828125	\\
0.881120433257868	-0.0028076171875	\\
0.881164824432903	-0.002532958984375	\\
0.881209215607937	-0.002685546875	\\
0.881253606782972	-0.003204345703125	\\
0.881297997958006	-0.003204345703125	\\
0.88134238913304	-0.00347900390625	\\
0.881386780308075	-0.003753662109375	\\
0.881431171483109	-0.003265380859375	\\
0.881475562658144	-0.002838134765625	\\
0.881519953833178	-0.00299072265625	\\
0.881564345008212	-0.003204345703125	\\
0.881608736183247	-0.003021240234375	\\
0.881653127358281	-0.003143310546875	\\
0.881697518533316	-0.003570556640625	\\
0.88174190970835	-0.003509521484375	\\
0.881786300883384	-0.002838134765625	\\
0.881830692058419	-0.002593994140625	\\
0.881875083233453	-0.0028076171875	\\
0.881919474408488	-0.002655029296875	\\
0.881963865583522	-0.002532958984375	\\
0.882008256758556	-0.002349853515625	\\
0.882052647933591	-0.00244140625	\\
0.882097039108625	-0.00274658203125	\\
0.88214143028366	-0.00262451171875	\\
0.882185821458694	-0.002410888671875	\\
0.882230212633728	-0.002532958984375	\\
0.882274603808763	-0.00213623046875	\\
0.882318994983797	-0.0020751953125	\\
0.882363386158832	-0.0018310546875	\\
0.882407777333866	-0.001861572265625	\\
0.8824521685089	-0.002349853515625	\\
0.882496559683935	-0.0023193359375	\\
0.882540950858969	-0.00244140625	\\
0.882585342034004	-0.002685546875	\\
0.882629733209038	-0.002288818359375	\\
0.882674124384072	-0.00225830078125	\\
0.882718515559107	-0.00213623046875	\\
0.882762906734141	-0.00177001953125	\\
0.882807297909176	-0.001861572265625	\\
0.88285168908421	-0.00177001953125	\\
0.882896080259244	-0.001678466796875	\\
0.882940471434279	-0.001678466796875	\\
0.882984862609313	-0.0013427734375	\\
0.883029253784348	-0.001220703125	\\
0.883073644959382	-0.001800537109375	\\
0.883118036134417	-0.001922607421875	\\
0.883162427309451	-0.001739501953125	\\
0.883206818484485	-0.00213623046875	\\
0.88325120965952	-0.002410888671875	\\
0.883295600834554	-0.0023193359375	\\
0.883339992009588	-0.002197265625	\\
0.883384383184623	-0.002227783203125	\\
0.883428774359657	-0.002410888671875	\\
0.883473165534692	-0.001800537109375	\\
0.883517556709726	-0.00152587890625	\\
0.883561947884761	-0.00140380859375	\\
0.883606339059795	-0.001556396484375	\\
0.883650730234829	-0.00152587890625	\\
0.883695121409864	-0.00146484375	\\
0.883739512584898	-0.0018310546875	\\
0.883783903759933	-0.001800537109375	\\
0.883828294934967	-0.001556396484375	\\
0.883872686110001	-0.001220703125	\\
0.883917077285036	-0.00146484375	\\
0.88396146846007	-0.001800537109375	\\
0.884005859635105	-0.00177001953125	\\
0.884050250810139	-0.002227783203125	\\
0.884094641985173	-0.0025634765625	\\
0.884139033160208	-0.002532958984375	\\
0.884183424335242	-0.00250244140625	\\
0.884227815510277	-0.002288818359375	\\
0.884272206685311	-0.0020751953125	\\
0.884316597860345	-0.0015869140625	\\
0.88436098903538	-0.001251220703125	\\
0.884405380210414	-0.001922607421875	\\
0.884449771385449	-0.00250244140625	\\
0.884494162560483	-0.002685546875	\\
0.884538553735517	-0.00286865234375	\\
0.884582944910552	-0.00299072265625	\\
0.884627336085586	-0.00299072265625	\\
0.884671727260621	-0.003265380859375	\\
0.884716118435655	-0.0037841796875	\\
0.884760509610689	-0.003570556640625	\\
0.884804900785724	-0.003509521484375	\\
0.884849291960758	-0.003753662109375	\\
0.884893683135793	-0.0029296875	\\
0.884938074310827	-0.0025634765625	\\
0.884982465485861	-0.0025634765625	\\
0.885026856660896	-0.00244140625	\\
0.88507124783593	-0.002716064453125	\\
0.885115639010965	-0.002777099609375	\\
0.885160030185999	-0.00244140625	\\
0.885204421361033	-0.00244140625	\\
0.885248812536068	-0.00286865234375	\\
0.885293203711102	-0.0029296875	\\
0.885337594886137	-0.00299072265625	\\
0.885381986061171	-0.002838134765625	\\
0.885426377236205	-0.0029296875	\\
0.88547076841124	-0.003021240234375	\\
0.885515159586274	-0.0029296875	\\
0.885559550761309	-0.002777099609375	\\
0.885603941936343	-0.003173828125	\\
0.885648333111377	-0.00335693359375	\\
0.885692724286412	-0.002685546875	\\
0.885737115461446	-0.00225830078125	\\
0.885781506636481	-0.002105712890625	\\
0.885825897811515	-0.00225830078125	\\
0.88587028898655	-0.001983642578125	\\
0.885914680161584	-0.00189208984375	\\
0.885959071336618	-0.002227783203125	\\
0.886003462511653	-0.0018310546875	\\
0.886047853686687	-0.00201416015625	\\
0.886092244861721	-0.002288818359375	\\
0.886136636036756	-0.002288818359375	\\
0.88618102721179	-0.0023193359375	\\
0.886225418386825	-0.002288818359375	\\
0.886269809561859	-0.002197265625	\\
0.886314200736893	-0.002716064453125	\\
0.886358591911928	-0.003265380859375	\\
0.886402983086962	-0.0032958984375	\\
0.886447374261997	-0.00311279296875	\\
0.886491765437031	-0.003143310546875	\\
0.886536156612066	-0.002777099609375	\\
0.8865805477871	-0.002899169921875	\\
0.886624938962134	-0.003173828125	\\
0.886669330137169	-0.003082275390625	\\
0.886713721312203	-0.002655029296875	\\
0.886758112487238	-0.002197265625	\\
0.886802503662272	-0.0025634765625	\\
0.886846894837306	-0.00244140625	\\
0.886891286012341	-0.002349853515625	\\
0.886935677187375	-0.0020751953125	\\
0.88698006836241	-0.002288818359375	\\
0.887024459537444	-0.00274658203125	\\
0.887068850712478	-0.002288818359375	\\
0.887113241887513	-0.002532958984375	\\
0.887157633062547	-0.003173828125	\\
0.887202024237582	-0.002593994140625	\\
0.887246415412616	-0.002197265625	\\
0.88729080658765	-0.002288818359375	\\
0.887335197762685	-0.002044677734375	\\
0.887379588937719	-0.00201416015625	\\
0.887423980112754	-0.002197265625	\\
0.887468371287788	-0.00213623046875	\\
0.887512762462822	-0.0020751953125	\\
0.887557153637857	-0.001800537109375	\\
0.887601544812891	-0.0015869140625	\\
0.887645935987926	-0.00146484375	\\
0.88769032716296	-0.0010986328125	\\
0.887734718337994	-0.0008544921875	\\
0.887779109513029	-0.00067138671875	\\
0.887823500688063	-0.000762939453125	\\
};
\addplot [color=blue,solid,forget plot]
  table[row sep=crcr]{
0.887823500688063	-0.000762939453125	\\
0.887867891863098	-0.00067138671875	\\
0.887912283038132	-0.00048828125	\\
0.887956674213166	-0.0003662109375	\\
0.888001065388201	-0.000457763671875	\\
0.888045456563235	-0.0008544921875	\\
0.88808984773827	-0.0013427734375	\\
0.888134238913304	-0.00140380859375	\\
0.888178630088338	-0.00140380859375	\\
0.888223021263373	-0.00103759765625	\\
0.888267412438407	-0.00091552734375	\\
0.888311803613442	-0.001129150390625	\\
0.888356194788476	-0.001068115234375	\\
0.88840058596351	-0.001312255859375	\\
0.888444977138545	-0.001312255859375	\\
0.888489368313579	-0.001190185546875	\\
0.888533759488614	-0.0018310546875	\\
0.888578150663648	-0.001617431640625	\\
0.888622541838683	-0.001373291015625	\\
0.888666933013717	-0.001556396484375	\\
0.888711324188751	-0.001678466796875	\\
0.888755715363786	-0.001739501953125	\\
0.88880010653882	-0.00152587890625	\\
0.888844497713854	-0.00140380859375	\\
0.888888888888889	-0.0013427734375	\\
0.888933280063923	-0.00146484375	\\
0.888977671238958	-0.00128173828125	\\
0.889022062413992	-0.00140380859375	\\
0.889066453589026	-0.001617431640625	\\
0.889110844764061	-0.00146484375	\\
0.889155235939095	-0.0013427734375	\\
0.88919962711413	-0.001251220703125	\\
0.889244018289164	-0.0009765625	\\
0.889288409464199	-0.000885009765625	\\
0.889332800639233	-0.00091552734375	\\
0.889377191814267	-0.000732421875	\\
0.889421582989302	-0.000762939453125	\\
0.889465974164336	-0.00079345703125	\\
0.889510365339371	-0.000946044921875	\\
0.889554756514405	-0.000396728515625	\\
0.889599147689439	-0.000701904296875	\\
0.889643538864474	-0.00048828125	\\
0.889687930039508	-0.000396728515625	\\
0.889732321214543	-0.00054931640625	\\
0.889776712389577	-0.00067138671875	\\
0.889821103564611	-0.00103759765625	\\
0.889865494739646	-0.000823974609375	\\
0.88990988591468	-0.000518798828125	\\
0.889954277089715	-0.000885009765625	\\
0.889998668264749	-0.000823974609375	\\
0.890043059439783	-0.000579833984375	\\
0.890087450614818	-0.000457763671875	\\
0.890131841789852	-0.00042724609375	\\
0.890176232964887	-0.00091552734375	\\
0.890220624139921	-0.0009765625	\\
0.890265015314955	-0.000518798828125	\\
0.89030940648999	-0.000762939453125	\\
0.890353797665024	-0.0008544921875	\\
0.890398188840059	-0.000732421875	\\
0.890442580015093	-0.000823974609375	\\
0.890486971190127	-0.000762939453125	\\
0.890531362365162	-0.000762939453125	\\
0.890575753540196	-0.00054931640625	\\
0.890620144715231	-0.000762939453125	\\
0.890664535890265	-0.001220703125	\\
0.890708927065299	-0.000823974609375	\\
0.890753318240334	-0.001068115234375	\\
0.890797709415368	-0.001800537109375	\\
0.890842100590403	-0.001708984375	\\
0.890886491765437	-0.00189208984375	\\
0.890930882940471	-0.001953125	\\
0.890975274115506	-0.001708984375	\\
0.89101966529054	-0.001556396484375	\\
0.891064056465575	-0.00115966796875	\\
0.891108447640609	-0.00115966796875	\\
0.891152838815643	-0.0008544921875	\\
0.891197229990678	-0.001129150390625	\\
0.891241621165712	-0.00128173828125	\\
0.891286012340747	-0.00115966796875	\\
0.891330403515781	-0.001007080078125	\\
0.891374794690815	-0.001068115234375	\\
0.89141918586585	-0.001220703125	\\
0.891463577040884	-0.00128173828125	\\
0.891507968215919	-0.00103759765625	\\
0.891552359390953	-0.000823974609375	\\
0.891596750565988	-0.000732421875	\\
0.891641141741022	-0.00042724609375	\\
0.891685532916056	-0.00079345703125	\\
0.891729924091091	-0.001129150390625	\\
0.891774315266125	-0.00103759765625	\\
0.891818706441159	-0.00079345703125	\\
0.891863097616194	-0.000823974609375	\\
0.891907488791228	-0.000732421875	\\
0.891951879966263	-0.00079345703125	\\
0.891996271141297	-0.0008544921875	\\
0.892040662316332	-0.000457763671875	\\
0.892085053491366	-0.0001220703125	\\
0.8921294446664	-0.00067138671875	\\
0.892173835841435	-0.0009765625	\\
0.892218227016469	-0.00079345703125	\\
0.892262618191504	-0.001190185546875	\\
0.892307009366538	-0.000823974609375	\\
0.892351400541572	-0.00030517578125	\\
0.892395791716607	-0.00030517578125	\\
0.892440182891641	-0.000579833984375	\\
0.892484574066676	-0.0001220703125	\\
0.89252896524171	-3.0517578125e-05	\\
0.892573356416744	-0.00042724609375	\\
0.892617747591779	-0.000518798828125	\\
0.892662138766813	-0.000579833984375	\\
0.892706529941848	-0.000152587890625	\\
0.892750921116882	-0.000732421875	\\
0.892795312291916	-0.001373291015625	\\
0.892839703466951	-0.00146484375	\\
0.892884094641985	-0.001861572265625	\\
0.89292848581702	-0.00146484375	\\
0.892972876992054	-0.001373291015625	\\
0.893017268167088	-0.00146484375	\\
0.893061659342123	-0.001220703125	\\
0.893106050517157	-0.0009765625	\\
0.893150441692192	-0.0009765625	\\
0.893194832867226	-0.0008544921875	\\
0.89323922404226	-0.000457763671875	\\
0.893283615217295	-0.00048828125	\\
0.893328006392329	-0.000579833984375	\\
0.893372397567364	-0.000762939453125	\\
0.893416788742398	-0.0003662109375	\\
0.893461179917432	-0.00054931640625	\\
0.893505571092467	-0.000640869140625	\\
0.893549962267501	-9.1552734375e-05	\\
0.893594353442536	-0.00048828125	\\
0.89363874461757	-0.000946044921875	\\
0.893683135792604	-0.0008544921875	\\
0.893727526967639	-0.000762939453125	\\
0.893771918142673	-0.000701904296875	\\
0.893816309317708	-0.000640869140625	\\
0.893860700492742	-0.0006103515625	\\
0.893905091667776	-0.000762939453125	\\
0.893949482842811	-0.0006103515625	\\
0.893993874017845	-0.00067138671875	\\
0.89403826519288	-0.000579833984375	\\
0.894082656367914	-0.000274658203125	\\
0.894127047542948	-3.0517578125e-05	\\
0.894171438717983	-0.00018310546875	\\
0.894215829893017	-0.000518798828125	\\
0.894260221068052	-0.00054931640625	\\
0.894304612243086	-0.000762939453125	\\
0.894349003418121	-0.00067138671875	\\
0.894393394593155	-0.000640869140625	\\
0.894437785768189	-0.001068115234375	\\
0.894482176943224	-0.001373291015625	\\
0.894526568118258	-0.001068115234375	\\
0.894570959293292	-0.00146484375	\\
0.894615350468327	-0.001617431640625	\\
0.894659741643361	-0.00177001953125	\\
0.894704132818396	-0.00244140625	\\
0.89474852399343	-0.001861572265625	\\
0.894792915168464	-0.002227783203125	\\
0.894837306343499	-0.00286865234375	\\
0.894881697518533	-0.003021240234375	\\
0.894926088693568	-0.003021240234375	\\
0.894970479868602	-0.00311279296875	\\
0.895014871043637	-0.00286865234375	\\
0.895059262218671	-0.002838134765625	\\
0.895103653393705	-0.003143310546875	\\
0.89514804456874	-0.003265380859375	\\
0.895192435743774	-0.003143310546875	\\
0.895236826918809	-0.003204345703125	\\
0.895281218093843	-0.003509521484375	\\
0.895325609268877	-0.0029296875	\\
0.895370000443912	-0.00274658203125	\\
0.895414391618946	-0.003326416015625	\\
0.895458782793981	-0.00323486328125	\\
0.895503173969015	-0.00335693359375	\\
0.895547565144049	-0.00341796875	\\
0.895591956319084	-0.003631591796875	\\
0.895636347494118	-0.003997802734375	\\
0.895680738669153	-0.004119873046875	\\
0.895725129844187	-0.00390625	\\
0.895769521019221	-0.00396728515625	\\
0.895813912194256	-0.003692626953125	\\
0.89585830336929	-0.003570556640625	\\
0.895902694544325	-0.003448486328125	\\
0.895947085719359	-0.003143310546875	\\
0.895991476894393	-0.003448486328125	\\
0.896035868069428	-0.00311279296875	\\
0.896080259244462	-0.002960205078125	\\
0.896124650419497	-0.002838134765625	\\
0.896169041594531	-0.0025634765625	\\
0.896213432769565	-0.002227783203125	\\
0.8962578239446	-0.0020751953125	\\
0.896302215119634	-0.002471923828125	\\
0.896346606294669	-0.00250244140625	\\
0.896390997469703	-0.002655029296875	\\
0.896435388644737	-0.00299072265625	\\
0.896479779819772	-0.003143310546875	\\
0.896524170994806	-0.002899169921875	\\
0.896568562169841	-0.003021240234375	\\
0.896612953344875	-0.00347900390625	\\
0.896657344519909	-0.002960205078125	\\
0.896701735694944	-0.00311279296875	\\
0.896746126869978	-0.00335693359375	\\
0.896790518045013	-0.003143310546875	\\
0.896834909220047	-0.003143310546875	\\
0.896879300395081	-0.00286865234375	\\
0.896923691570116	-0.002838134765625	\\
0.89696808274515	-0.0029296875	\\
0.897012473920185	-0.00311279296875	\\
0.897056865095219	-0.00311279296875	\\
0.897101256270254	-0.003021240234375	\\
0.897145647445288	-0.003173828125	\\
0.897190038620322	-0.003387451171875	\\
0.897234429795357	-0.00311279296875	\\
0.897278820970391	-0.0030517578125	\\
0.897323212145426	-0.003387451171875	\\
0.89736760332046	-0.003204345703125	\\
0.897411994495494	-0.002777099609375	\\
0.897456385670529	-0.002777099609375	\\
0.897500776845563	-0.002960205078125	\\
0.897545168020597	-0.00238037109375	\\
0.897589559195632	-0.002166748046875	\\
0.897633950370666	-0.0025634765625	\\
0.897678341545701	-0.002227783203125	\\
0.897722732720735	-0.002349853515625	\\
0.89776712389577	-0.002471923828125	\\
0.897811515070804	-0.002471923828125	\\
0.897855906245838	-0.002288818359375	\\
0.897900297420873	-0.001983642578125	\\
0.897944688595907	-0.002349853515625	\\
0.897989079770942	-0.002777099609375	\\
0.898033470945976	-0.002777099609375	\\
0.89807786212101	-0.00244140625	\\
0.898122253296045	-0.001983642578125	\\
0.898166644471079	-0.001739501953125	\\
0.898211035646114	-0.0020751953125	\\
0.898255426821148	-0.001983642578125	\\
0.898299817996182	-0.002349853515625	\\
0.898344209171217	-0.002716064453125	\\
0.898388600346251	-0.00286865234375	\\
0.898432991521286	-0.00323486328125	\\
0.89847738269632	-0.003265380859375	\\
0.898521773871354	-0.003631591796875	\\
0.898566165046389	-0.003448486328125	\\
0.898610556221423	-0.0029296875	\\
0.898654947396458	-0.00299072265625	\\
0.898699338571492	-0.002685546875	\\
0.898743729746526	-0.002532958984375	\\
0.898788120921561	-0.002471923828125	\\
0.898832512096595	-0.00244140625	\\
0.89887690327163	-0.002471923828125	\\
0.898921294446664	-0.0023193359375	\\
0.898965685621698	-0.0028076171875	\\
0.899010076796733	-0.00299072265625	\\
0.899054467971767	-0.0030517578125	\\
0.899098859146802	-0.003173828125	\\
0.899143250321836	-0.00311279296875	\\
0.89918764149687	-0.00274658203125	\\
0.899232032671905	-0.002899169921875	\\
0.899276423846939	-0.00347900390625	\\
0.899320815021974	-0.003631591796875	\\
0.899365206197008	-0.0037841796875	\\
0.899409597372042	-0.003814697265625	\\
0.899453988547077	-0.003753662109375	\\
0.899498379722111	-0.003662109375	\\
0.899542770897146	-0.00335693359375	\\
0.89958716207218	-0.003326416015625	\\
0.899631553247214	-0.00372314453125	\\
0.899675944422249	-0.00335693359375	\\
0.899720335597283	-0.003173828125	\\
0.899764726772318	-0.003143310546875	\\
0.899809117947352	-0.0032958984375	\\
0.899853509122386	-0.003387451171875	\\
0.899897900297421	-0.0032958984375	\\
0.899942291472455	-0.003173828125	\\
0.89998668264749	-0.003173828125	\\
0.900031073822524	-0.003173828125	\\
0.900075464997559	-0.003173828125	\\
0.900119856172593	-0.003082275390625	\\
0.900164247347627	-0.0028076171875	\\
0.900208638522662	-0.003082275390625	\\
0.900253029697696	-0.003082275390625	\\
0.90029742087273	-0.0025634765625	\\
0.900341812047765	-0.002593994140625	\\
0.900386203222799	-0.00286865234375	\\
0.900430594397834	-0.002685546875	\\
0.900474985572868	-0.00299072265625	\\
0.900519376747903	-0.002960205078125	\\
0.900563767922937	-0.002838134765625	\\
0.900608159097971	-0.002471923828125	\\
0.900652550273006	-0.002166748046875	\\
0.90069694144804	-0.00225830078125	\\
0.900741332623075	-0.001983642578125	\\
0.900785723798109	-0.002044677734375	\\
0.900830114973143	-0.001556396484375	\\
0.900874506148178	-0.0010986328125	\\
0.900918897323212	-0.001251220703125	\\
0.900963288498247	-0.0018310546875	\\
0.901007679673281	-0.001922607421875	\\
0.901052070848315	-0.001800537109375	\\
0.90109646202335	-0.002471923828125	\\
0.901140853198384	-0.002899169921875	\\
0.901185244373419	-0.003265380859375	\\
0.901229635548453	-0.003173828125	\\
0.901274026723487	-0.0032958984375	\\
0.901318417898522	-0.00341796875	\\
0.901362809073556	-0.003326416015625	\\
0.901407200248591	-0.003143310546875	\\
0.901451591423625	-0.00311279296875	\\
0.901495982598659	-0.003509521484375	\\
0.901540373773694	-0.003082275390625	\\
0.901584764948728	-0.00274658203125	\\
0.901629156123763	-0.0030517578125	\\
0.901673547298797	-0.002899169921875	\\
0.901717938473831	-0.002349853515625	\\
0.901762329648866	-0.0020751953125	\\
0.9018067208239	-0.002227783203125	\\
0.901851111998935	-0.002532958984375	\\
0.901895503173969	-0.00250244140625	\\
0.901939894349003	-0.002227783203125	\\
0.901984285524038	-0.002960205078125	\\
0.902028676699072	-0.003204345703125	\\
0.902073067874107	-0.002838134765625	\\
0.902117459049141	-0.003082275390625	\\
0.902161850224175	-0.003173828125	\\
0.90220624139921	-0.00323486328125	\\
0.902250632574244	-0.0028076171875	\\
0.902295023749279	-0.002899169921875	\\
0.902339414924313	-0.002655029296875	\\
0.902383806099347	-0.00225830078125	\\
0.902428197274382	-0.00225830078125	\\
0.902472588449416	-0.002044677734375	\\
0.902516979624451	-0.002288818359375	\\
0.902561370799485	-0.002044677734375	\\
0.902605761974519	-0.00213623046875	\\
0.902650153149554	-0.002349853515625	\\
0.902694544324588	-0.00250244140625	\\
0.902738935499623	-0.002716064453125	\\
0.902783326674657	-0.0028076171875	\\
0.902827717849692	-0.002716064453125	\\
0.902872109024726	-0.002410888671875	\\
0.90291650019976	-0.00238037109375	\\
0.902960891374795	-0.002593994140625	\\
0.903005282549829	-0.002532958984375	\\
0.903049673724863	-0.00238037109375	\\
0.903094064899898	-0.002197265625	\\
0.903138456074932	-0.002044677734375	\\
0.903182847249967	-0.0020751953125	\\
0.903227238425001	-0.002197265625	\\
0.903271629600035	-0.00164794921875	\\
0.90331602077507	-0.001800537109375	\\
0.903360411950104	-0.00225830078125	\\
0.903404803125139	-0.0023193359375	\\
0.903449194300173	-0.002410888671875	\\
0.903493585475208	-0.002685546875	\\
0.903537976650242	-0.002838134765625	\\
0.903582367825276	-0.00274658203125	\\
0.903626759000311	-0.002532958984375	\\
0.903671150175345	-0.002532958984375	\\
0.90371554135038	-0.00244140625	\\
0.903759932525414	-0.002532958984375	\\
0.903804323700448	-0.002593994140625	\\
0.903848714875483	-0.002471923828125	\\
0.903893106050517	-0.002410888671875	\\
0.903937497225552	-0.00286865234375	\\
0.903981888400586	-0.0025634765625	\\
0.90402627957562	-0.002227783203125	\\
0.904070670750655	-0.0023193359375	\\
0.904115061925689	-0.002655029296875	\\
0.904159453100724	-0.002471923828125	\\
0.904203844275758	-0.002777099609375	\\
0.904248235450792	-0.0030517578125	\\
0.904292626625827	-0.002777099609375	\\
0.904337017800861	-0.002777099609375	\\
0.904381408975896	-0.002716064453125	\\
0.90442580015093	-0.00274658203125	\\
0.904470191325964	-0.0028076171875	\\
0.904514582500999	-0.00238037109375	\\
0.904558973676033	-0.0018310546875	\\
0.904603364851068	-0.00189208984375	\\
0.904647756026102	-0.0015869140625	\\
0.904692147201136	-0.001312255859375	\\
0.904736538376171	-0.00177001953125	\\
0.904780929551205	-0.001800537109375	\\
0.90482532072624	-0.0020751953125	\\
0.904869711901274	-0.002044677734375	\\
0.904914103076308	-0.0020751953125	\\
0.904958494251343	-0.00213623046875	\\
0.905002885426377	-0.0023193359375	\\
0.905047276601412	-0.002532958984375	\\
0.905091667776446	-0.002288818359375	\\
0.90513605895148	-0.001922607421875	\\
0.905180450126515	-0.00189208984375	\\
0.905224841301549	-0.001953125	\\
0.905269232476584	-0.001678466796875	\\
0.905313623651618	-0.00140380859375	\\
0.905358014826652	-0.001556396484375	\\
0.905402406001687	-0.00164794921875	\\
0.905446797176721	-0.001617431640625	\\
0.905491188351756	-0.001129150390625	\\
0.90553557952679	-0.000732421875	\\
0.905579970701825	-0.000762939453125	\\
0.905624361876859	-0.00067138671875	\\
0.905668753051893	-0.000274658203125	\\
0.905713144226928	0.0001220703125	\\
0.905757535401962	0.000335693359375	\\
0.905801926576997	0.000213623046875	\\
0.905846317752031	0.000579833984375	\\
0.905890708927065	0.000396728515625	\\
0.9059351001021	-0.000213623046875	\\
0.905979491277134	-0.0003662109375	\\
0.906023882452168	-0.00030517578125	\\
0.906068273627203	-0.0003662109375	\\
0.906112664802237	-0.00067138671875	\\
0.906157055977272	-0.00103759765625	\\
0.906201447152306	-0.00103759765625	\\
0.906245838327341	-0.001251220703125	\\
0.906290229502375	-0.00115966796875	\\
0.906334620677409	-0.00091552734375	\\
0.906379011852444	-0.00103759765625	\\
0.906423403027478	-0.001373291015625	\\
0.906467794202513	-0.001190185546875	\\
0.906512185377547	-0.000640869140625	\\
0.906556576552581	-0.000396728515625	\\
0.906600967727616	-0.00030517578125	\\
0.90664535890265	-0.0006103515625	\\
0.906689750077685	-0.001007080078125	\\
0.906734141252719	-0.000732421875	\\
0.906778532427753	-0.00079345703125	\\
0.906822923602788	-0.001495361328125	\\
0.906867314777822	-0.001373291015625	\\
0.906911705952857	-0.001495361328125	\\
0.906956097127891	-0.001312255859375	\\
0.907000488302925	-0.00091552734375	\\
0.90704487947796	-0.0010986328125	\\
0.907089270652994	-0.00079345703125	\\
0.907133661828029	-0.000640869140625	\\
0.907178053003063	-0.00042724609375	\\
0.907222444178097	-0.00018310546875	\\
0.907266835353132	-6.103515625e-05	\\
0.907311226528166	0.000152587890625	\\
0.907355617703201	0.000244140625	\\
0.907400008878235	0.000640869140625	\\
0.907444400053269	0.00067138671875	\\
0.907488791228304	0.00067138671875	\\
0.907533182403338	0.0008544921875	\\
0.907577573578373	0.000640869140625	\\
0.907621964753407	0.00054931640625	\\
0.907666355928441	0.000640869140625	\\
0.907710747103476	0.000579833984375	\\
0.90775513827851	0.00030517578125	\\
0.907799529453545	0.000152587890625	\\
0.907843920628579	0.00067138671875	\\
0.907888311803613	0.0006103515625	\\
0.907932702978648	9.1552734375e-05	\\
0.907977094153682	0.00018310546875	\\
0.908021485328717	0.000335693359375	\\
0.908065876503751	0.000274658203125	\\
0.908110267678785	0.000335693359375	\\
0.90815465885382	0.000762939453125	\\
0.908199050028854	0.00054931640625	\\
0.908243441203889	0.000274658203125	\\
0.908287832378923	0.00030517578125	\\
0.908332223553957	0.000579833984375	\\
0.908376614728992	0.001007080078125	\\
0.908421005904026	0.00091552734375	\\
0.908465397079061	0.000946044921875	\\
0.908509788254095	0.001220703125	\\
0.90855417942913	0.0013427734375	\\
0.908598570604164	0.00115966796875	\\
0.908642961779198	0.001312255859375	\\
0.908687352954233	0.001556396484375	\\
0.908731744129267	0.001495361328125	\\
0.908776135304301	0.001190185546875	\\
0.908820526479336	0.00115966796875	\\
0.90886491765437	0.001129150390625	\\
0.908909308829405	0.001129150390625	\\
0.908953700004439	0.000762939453125	\\
0.908998091179474	0.00079345703125	\\
0.909042482354508	0.000396728515625	\\
0.909086873529542	0.00048828125	\\
0.909131264704577	0.001007080078125	\\
0.909175655879611	0.001129150390625	\\
0.909220047054646	0.000823974609375	\\
0.90926443822968	0.000579833984375	\\
0.909308829404714	0.001190185546875	\\
0.909353220579749	0.0009765625	\\
0.909397611754783	0.0008544921875	\\
0.909442002929818	0.000823974609375	\\
0.909486394104852	0.00079345703125	\\
0.909530785279886	0.0009765625	\\
0.909575176454921	0.00079345703125	\\
0.909619567629955	0.00091552734375	\\
0.90966395880499	0.00103759765625	\\
0.909708349980024	0.000823974609375	\\
0.909752741155058	0.0013427734375	\\
0.909797132330093	0.001220703125	\\
0.909841523505127	0.0010986328125	\\
0.909885914680162	0.00067138671875	\\
0.909930305855196	0.000732421875	\\
0.90997469703023	0.0013427734375	\\
0.910019088205265	0.001190185546875	\\
0.910063479380299	0.00103759765625	\\
0.910107870555334	0.001190185546875	\\
0.910152261730368	0.00128173828125	\\
0.910196652905402	0.001373291015625	\\
0.910241044080437	0.0018310546875	\\
0.910285435255471	0.001922607421875	\\
0.910329826430506	0.001800537109375	\\
0.91037421760554	0.001739501953125	\\
0.910418608780574	0.00213623046875	\\
0.910462999955609	0.0020751953125	\\
0.910507391130643	0.001220703125	\\
0.910551782305678	0.0013427734375	\\
0.910596173480712	0.00128173828125	\\
0.910640564655746	0.001129150390625	\\
0.910684955830781	0.001251220703125	\\
0.910729347005815	0.000640869140625	\\
0.91077373818085	0.000274658203125	\\
0.910818129355884	0.000885009765625	\\
0.910862520530918	0.001129150390625	\\
0.910906911705953	0.001220703125	\\
0.910951302880987	0.001708984375	\\
0.910995694056022	0.001617431640625	\\
0.911040085231056	0.001373291015625	\\
0.91108447640609	0.001800537109375	\\
0.911128867581125	0.00177001953125	\\
0.911173258756159	0.00146484375	\\
0.911217649931194	0.00140380859375	\\
0.911262041106228	0.00152587890625	\\
0.911306432281263	0.00146484375	\\
0.911350823456297	0.0013427734375	\\
0.911395214631331	0.001251220703125	\\
0.911439605806366	0.0013427734375	\\
0.9114839969814	0.001556396484375	\\
0.911528388156435	0.0013427734375	\\
0.911572779331469	0.001678466796875	\\
0.911617170506503	0.001556396484375	\\
0.911661561681538	0.001617431640625	\\
0.911705952856572	0.001861572265625	\\
0.911750344031606	0.00177001953125	\\
0.911794735206641	0.00177001953125	\\
0.911839126381675	0.00164794921875	\\
0.91188351755671	0.001678466796875	\\
0.911927908731744	0.001434326171875	\\
0.911972299906779	0.001251220703125	\\
0.912016691081813	0.00146484375	\\
0.912061082256847	0.001556396484375	\\
0.912105473431882	0.001129150390625	\\
0.912149864606916	0.00115966796875	\\
0.912194255781951	0.001007080078125	\\
0.912238646956985	0.000701904296875	\\
0.912283038132019	0.000732421875	\\
0.912327429307054	0.0006103515625	\\
0.912371820482088	0.000518798828125	\\
0.912416211657123	9.1552734375e-05	\\
0.912460602832157	3.0517578125e-05	\\
0.912504994007191	0.0006103515625	\\
0.912549385182226	0.000579833984375	\\
0.91259377635726	0.000396728515625	\\
0.912638167532295	0.000335693359375	\\
0.912682558707329	0.000152587890625	\\
0.912726949882363	0.0006103515625	\\
0.912771341057398	0.000640869140625	\\
0.912815732232432	0.000579833984375	\\
0.912860123407467	0.00115966796875	\\
0.912904514582501	0.000762939453125	\\
0.912948905757535	0.0006103515625	\\
0.91299329693257	0.00115966796875	\\
0.913037688107604	0.000823974609375	\\
0.913082079282639	0.0010986328125	\\
0.913126470457673	0.001434326171875	\\
0.913170861632707	0.001373291015625	\\
0.913215252807742	0.001708984375	\\
0.913259643982776	0.00146484375	\\
0.913304035157811	0.00140380859375	\\
0.913348426332845	0.0013427734375	\\
0.913392817507879	0.001312255859375	\\
0.913437208682914	0.0010986328125	\\
0.913481599857948	0.000946044921875	\\
0.913525991032983	0.001220703125	\\
0.913570382208017	0.001708984375	\\
0.913614773383051	0.001922607421875	\\
0.913659164558086	0.002197265625	\\
0.91370355573312	0.002227783203125	\\
0.913747946908155	0.00213623046875	\\
0.913792338083189	0.002685546875	\\
0.913836729258223	0.002899169921875	\\
0.913881120433258	0.0029296875	\\
0.913925511608292	0.003021240234375	\\
0.913969902783327	0.0025634765625	\\
0.914014293958361	0.0025634765625	\\
0.914058685133396	0.002349853515625	\\
0.91410307630843	0.0023193359375	\\
0.914147467483464	0.002227783203125	\\
0.914191858658499	0.002288818359375	\\
0.914236249833533	0.002349853515625	\\
0.914280641008568	0.00201416015625	\\
0.914325032183602	0.0020751953125	\\
0.914369423358636	0.002166748046875	\\
0.914413814533671	0.002197265625	\\
0.914458205708705	0.002105712890625	\\
0.914502596883739	0.002044677734375	\\
0.914546988058774	0.001983642578125	\\
0.914591379233808	0.0020751953125	\\
0.914635770408843	0.002655029296875	\\
0.914680161583877	0.002716064453125	\\
0.914724552758912	0.002655029296875	\\
0.914768943933946	0.00274658203125	\\
0.91481333510898	0.002838134765625	\\
0.914857726284015	0.00299072265625	\\
0.914902117459049	0.002655029296875	\\
0.914946508634084	0.002044677734375	\\
0.914990899809118	0.002288818359375	\\
0.915035290984152	0.001922607421875	\\
0.915079682159187	0.002197265625	\\
0.915124073334221	0.002288818359375	\\
0.915168464509256	0.001953125	\\
0.91521285568429	0.001861572265625	\\
0.915257246859324	0.001678466796875	\\
0.915301638034359	0.001678466796875	\\
0.915346029209393	0.0020751953125	\\
0.915390420384428	0.002227783203125	\\
0.915434811559462	0.001800537109375	\\
0.915479202734496	0.00164794921875	\\
0.915523593909531	0.0018310546875	\\
0.915567985084565	0.001983642578125	\\
0.9156123762596	0.002166748046875	\\
0.915656767434634	0.0020751953125	\\
0.915701158609668	0.001983642578125	\\
0.915745549784703	0.002197265625	\\
0.915789940959737	0.002166748046875	\\
0.915834332134772	0.002349853515625	\\
0.915878723309806	0.00250244140625	\\
0.91592311448484	0.00244140625	\\
0.915967505659875	0.00250244140625	\\
0.916011896834909	0.001800537109375	\\
0.916056288009944	0.001678466796875	\\
0.916100679184978	0.00213623046875	\\
0.916145070360012	0.00177001953125	\\
0.916189461535047	0.00140380859375	\\
0.916233852710081	0.0013427734375	\\
0.916278243885116	0.0013427734375	\\
0.91632263506015	0.001007080078125	\\
0.916367026235184	0.00048828125	\\
0.916411417410219	0.000701904296875	\\
0.916455808585253	0.0008544921875	\\
0.916500199760288	0.000579833984375	\\
0.916544590935322	0.000396728515625	\\
0.916588982110356	0.0008544921875	\\
0.916633373285391	0.00103759765625	\\
0.916677764460425	0.0006103515625	\\
0.91672215563546	0.00091552734375	\\
0.916766546810494	0.001251220703125	\\
0.916810937985528	0.001251220703125	\\
0.916855329160563	0.001495361328125	\\
0.916899720335597	0.00128173828125	\\
0.916944111510632	0.0010986328125	\\
0.916988502685666	0.00115966796875	\\
0.917032893860701	0.001220703125	\\
0.917077285035735	0.000885009765625	\\
0.917121676210769	0.00115966796875	\\
0.917166067385804	0.00128173828125	\\
0.917210458560838	0.001190185546875	\\
0.917254849735872	0.0013427734375	\\
0.917299240910907	0.0010986328125	\\
0.917343632085941	0.001129150390625	\\
0.917388023260976	0.001556396484375	\\
0.91743241443601	0.001708984375	\\
0.917476805611045	0.001953125	\\
0.917521196786079	0.00152587890625	\\
0.917565587961113	0.001312255859375	\\
0.917609979136148	0.00152587890625	\\
0.917654370311182	0.00152587890625	\\
0.917698761486217	0.001068115234375	\\
0.917743152661251	0.0008544921875	\\
0.917787543836285	0.000579833984375	\\
0.91783193501132	0.00042724609375	\\
0.917876326186354	0.00067138671875	\\
0.917920717361389	0.0008544921875	\\
0.917965108536423	0.00115966796875	\\
0.918009499711457	0.00115966796875	\\
0.918053890886492	0.000732421875	\\
0.918098282061526	0.0009765625	\\
0.918142673236561	0.001373291015625	\\
0.918187064411595	0.00067138671875	\\
0.918231455586629	0.001068115234375	\\
0.918275846761664	0.0009765625	\\
0.918320237936698	0.00091552734375	\\
0.918364629111733	0.00091552734375	\\
0.918409020286767	0.000701904296875	\\
0.918453411461801	0.000762939453125	\\
0.918497802636836	0.000457763671875	\\
0.91854219381187	0.00030517578125	\\
0.918586584986905	0.00048828125	\\
0.918630976161939	0.00091552734375	\\
0.918675367336973	0.000946044921875	\\
0.918719758512008	0.000885009765625	\\
0.918764149687042	0.000823974609375	\\
0.918808540862077	0.001068115234375	\\
0.918852932037111	0.001068115234375	\\
0.918897323212145	0.000732421875	\\
0.91894171438718	0.000396728515625	\\
0.918986105562214	0.00054931640625	\\
0.919030496737249	0.00067138671875	\\
0.919074887912283	0.000701904296875	\\
0.919119279087317	0.0006103515625	\\
0.919163670262352	0.00115966796875	\\
0.919208061437386	0.0015869140625	\\
0.919252452612421	0.001220703125	\\
0.919296843787455	0.001068115234375	\\
0.919341234962489	0.00115966796875	\\
0.919385626137524	0.00128173828125	\\
0.919430017312558	0.001129150390625	\\
0.919474408487593	0.00115966796875	\\
0.919518799662627	0.001861572265625	\\
0.919563190837661	0.001983642578125	\\
0.919607582012696	0.001708984375	\\
0.91965197318773	0.001495361328125	\\
0.919696364362765	0.0009765625	\\
0.919740755537799	0.00115966796875	\\
0.919785146712834	0.001251220703125	\\
0.919829537887868	0.00054931640625	\\
0.919873929062902	0.0006103515625	\\
0.919918320237937	0.000396728515625	\\
0.919962711412971	-6.103515625e-05	\\
0.920007102588006	-0.000152587890625	\\
0.92005149376304	3.0517578125e-05	\\
0.920095884938074	-0.000396728515625	\\
0.920140276113109	-0.001068115234375	\\
0.920184667288143	-0.000579833984375	\\
0.920229058463177	-0.00030517578125	\\
0.920273449638212	-0.000457763671875	\\
0.920317840813246	-0.0006103515625	\\
0.920362231988281	-0.000732421875	\\
0.920406623163315	-0.0008544921875	\\
0.92045101433835	-0.00042724609375	\\
0.920495405513384	-0.000579833984375	\\
0.920539796688418	-0.000946044921875	\\
0.920584187863453	-0.000579833984375	\\
0.920628579038487	-0.00079345703125	\\
0.920672970213522	-0.000396728515625	\\
0.920717361388556	-0.000518798828125	\\
0.92076175256359	-0.00054931640625	\\
0.920806143738625	-0.000213623046875	\\
0.920850534913659	-0.0006103515625	\\
0.920894926088694	-0.000732421875	\\
0.920939317263728	-0.00103759765625	\\
0.920983708438762	-0.001434326171875	\\
0.921028099613797	-0.001434326171875	\\
0.921072490788831	-0.001068115234375	\\
0.921116881963866	-0.000396728515625	\\
0.9211612731389	-0.000335693359375	\\
0.921205664313934	-0.0003662109375	\\
0.921250055488969	-0.000244140625	\\
0.921294446664003	9.1552734375e-05	\\
0.921338837839038	0	\\
0.921383229014072	-0.00030517578125	\\
0.921427620189106	-9.1552734375e-05	\\
0.921472011364141	-0.00018310546875	\\
0.921516402539175	-0.00018310546875	\\
0.92156079371421	0.00018310546875	\\
0.921605184889244	-0.0001220703125	\\
0.921649576064278	0.0001220703125	\\
0.921693967239313	0.0003662109375	\\
0.921738358414347	0.000213623046875	\\
0.921782749589382	0.00048828125	\\
0.921827140764416	0.00067138671875	\\
0.92187153193945	0.00091552734375	\\
0.921915923114485	0.000579833984375	\\
0.921960314289519	-3.0517578125e-05	\\
0.922004705464554	0.00018310546875	\\
0.922049096639588	0.000701904296875	\\
0.922093487814622	-3.0517578125e-05	\\
0.922137878989657	-0.00054931640625	\\
0.922182270164691	-0.000274658203125	\\
0.922226661339726	-0.0006103515625	\\
0.92227105251476	-0.000579833984375	\\
0.922315443689794	-0.000518798828125	\\
0.922359834864829	-0.000640869140625	\\
0.922404226039863	-0.000457763671875	\\
0.922448617214898	-0.000518798828125	\\
0.922493008389932	-0.000396728515625	\\
0.922537399564967	-0.000213623046875	\\
0.922581790740001	-0.0003662109375	\\
0.922626181915035	-0.00048828125	\\
0.92267057309007	-0.00048828125	\\
0.922714964265104	6.103515625e-05	\\
0.922759355440139	-0.0001220703125	\\
0.922803746615173	-0.00018310546875	\\
0.922848137790207	-0.00030517578125	\\
0.922892528965242	-0.000701904296875	\\
0.922936920140276	-0.000518798828125	\\
0.92298131131531	-0.0006103515625	\\
0.923025702490345	-0.000396728515625	\\
0.923070093665379	-0.000335693359375	\\
0.923114484840414	-0.000152587890625	\\
0.923158876015448	0.000335693359375	\\
0.923203267190483	0.000213623046875	\\
0.923247658365517	-0.000244140625	\\
0.923292049540551	-0.0003662109375	\\
0.923336440715586	-0.000244140625	\\
0.92338083189062	0.000152587890625	\\
0.923425223065655	0.000335693359375	\\
0.923469614240689	0.00054931640625	\\
0.923514005415723	0.00048828125	\\
0.923558396590758	0.000213623046875	\\
0.923602787765792	0.00042724609375	\\
0.923647178940827	0.0003662109375	\\
0.923691570115861	0.00054931640625	\\
0.923735961290895	0.000640869140625	\\
0.92378035246593	0.00048828125	\\
0.923824743640964	0.000762939453125	\\
0.923869134815999	0.000579833984375	\\
0.923913525991033	0.000335693359375	\\
0.923957917166067	0.00067138671875	\\
0.924002308341102	0.000640869140625	\\
0.924046699516136	0.00048828125	\\
0.924091090691171	0.000640869140625	\\
0.924135481866205	0.000885009765625	\\
0.924179873041239	0.0006103515625	\\
0.924224264216274	0.000274658203125	\\
0.924268655391308	6.103515625e-05	\\
0.924313046566343	0.00030517578125	\\
0.924357437741377	0.00079345703125	\\
0.924401828916411	0.00054931640625	\\
0.924446220091446	0.00018310546875	\\
0.92449061126648	0.0006103515625	\\
0.924535002441515	0.0008544921875	\\
0.924579393616549	0.000762939453125	\\
0.924623784791583	0.001220703125	\\
0.924668175966618	0.001129150390625	\\
0.924712567141652	0.00067138671875	\\
0.924756958316687	0.00103759765625	\\
0.924801349491721	0.000640869140625	\\
0.924845740666755	0.0006103515625	\\
0.92489013184179	0.000885009765625	\\
0.924934523016824	0.000640869140625	\\
0.924978914191859	0.000518798828125	\\
0.925023305366893	0.00079345703125	\\
0.925067696541927	0.00091552734375	\\
0.925112087716962	0.000885009765625	\\
0.925156478891996	0.000885009765625	\\
0.925200870067031	0.00079345703125	\\
0.925245261242065	0.0010986328125	\\
0.925289652417099	0.00115966796875	\\
0.925334043592134	0.0008544921875	\\
0.925378434767168	0.0008544921875	\\
0.925422825942203	0.0008544921875	\\
0.925467217117237	0.000732421875	\\
0.925511608292272	0.000701904296875	\\
0.925555999467306	0.00067138671875	\\
0.92560039064234	0.000701904296875	\\
0.925644781817375	0.000701904296875	\\
0.925689172992409	0.000823974609375	\\
0.925733564167443	0.000457763671875	\\
0.925777955342478	-0.000457763671875	\\
0.925822346517512	-0.0003662109375	\\
0.925866737692547	-0.00048828125	\\
0.925911128867581	-0.000457763671875	\\
0.925955520042616	-0.000244140625	\\
0.92599991121765	-0.0001220703125	\\
0.926044302392684	-0.000335693359375	\\
0.926088693567719	-0.000732421875	\\
0.926133084742753	-0.0009765625	\\
0.926177475917788	-0.001007080078125	\\
0.926221867092822	-0.001373291015625	\\
0.926266258267856	-0.001434326171875	\\
0.926310649442891	-0.001373291015625	\\
0.926355040617925	-0.00115966796875	\\
0.92639943179296	-0.0006103515625	\\
0.926443822967994	-0.00048828125	\\
0.926488214143028	-0.000457763671875	\\
0.926532605318063	-0.00018310546875	\\
0.926576996493097	0.000244140625	\\
0.926621387668132	0.00048828125	\\
0.926665778843166	0.000823974609375	\\
0.9267101700182	0.00067138671875	\\
0.926754561193235	0.00115966796875	\\
0.926798952368269	0.001617431640625	\\
0.926843343543304	0.001373291015625	\\
0.926887734718338	0.0015869140625	\\
0.926932125893372	0.001800537109375	\\
0.926976517068407	0.001556396484375	\\
0.927020908243441	0.0015869140625	\\
0.927065299418476	0.001983642578125	\\
0.92710969059351	0.001953125	\\
0.927154081768544	0.0018310546875	\\
0.927198472943579	0.001983642578125	\\
0.927242864118613	0.002349853515625	\\
0.927287255293648	0.00286865234375	\\
0.927331646468682	0.002593994140625	\\
0.927376037643716	0.003021240234375	\\
0.927420428818751	0.00323486328125	\\
0.927464819993785	0.00262451171875	\\
0.92750921116882	0.00213623046875	\\
0.927553602343854	0.002044677734375	\\
0.927597993518889	0.0018310546875	\\
0.927642384693923	0.001800537109375	\\
0.927686775868957	0.0023193359375	\\
0.927731167043992	0.002685546875	\\
0.927775558219026	0.00286865234375	\\
0.92781994939406	0.0023193359375	\\
0.927864340569095	0.0018310546875	\\
0.927908731744129	0.00213623046875	\\
0.927953122919164	0.001708984375	\\
0.927997514094198	0.001495361328125	\\
0.928041905269232	0.00152587890625	\\
0.928086296444267	0.001373291015625	\\
0.928130687619301	0.00146484375	\\
0.928175078794336	0.001373291015625	\\
0.92821946996937	0.001678466796875	\\
0.928263861144405	0.001678466796875	\\
0.928308252319439	0.001739501953125	\\
0.928352643494473	0.001922607421875	\\
0.928397034669508	0.002105712890625	\\
0.928441425844542	0.002166748046875	\\
0.928485817019577	0.002410888671875	\\
0.928530208194611	0.0023193359375	\\
0.928574599369645	0.00201416015625	\\
0.92861899054468	0.00213623046875	\\
0.928663381719714	0.001953125	\\
0.928707772894748	0.00189208984375	\\
0.928752164069783	0.00152587890625	\\
0.928796555244817	0.002044677734375	\\
0.928840946419852	0.001708984375	\\
0.928885337594886	0.001312255859375	\\
0.928929728769921	0.001434326171875	\\
0.928974119944955	0.001556396484375	\\
0.929018511119989	0.0018310546875	\\
0.929062902295024	0.00189208984375	\\
0.929107293470058	0.002044677734375	\\
0.929151684645093	0.002105712890625	\\
0.929196075820127	0.001983642578125	\\
0.929240466995161	0.002288818359375	\\
0.929284858170196	0.002288818359375	\\
0.92932924934523	0.001983642578125	\\
0.929373640520265	0.0023193359375	\\
0.929418031695299	0.002349853515625	\\
0.929462422870333	0.002349853515625	\\
0.929506814045368	0.002227783203125	\\
0.929551205220402	0.00201416015625	\\
0.929595596395437	0.0020751953125	\\
0.929639987570471	0.00238037109375	\\
0.929684378745505	0.00201416015625	\\
0.92972876992054	0.001953125	\\
0.929773161095574	0.001739501953125	\\
0.929817552270609	0.001495361328125	\\
0.929861943445643	0.001617431640625	\\
0.929906334620677	0.001373291015625	\\
0.929950725795712	0.001434326171875	\\
0.929995116970746	0.0018310546875	\\
0.930039508145781	0.001251220703125	\\
0.930083899320815	0.001495361328125	\\
0.930128290495849	0.0018310546875	\\
0.930172681670884	0.001739501953125	\\
0.930217072845918	0.002166748046875	\\
0.930261464020953	0.00225830078125	\\
0.930305855195987	0.001922607421875	\\
0.930350246371021	0.001739501953125	\\
0.930394637546056	0.001556396484375	\\
0.93043902872109	0.00140380859375	\\
0.930483419896125	0.00128173828125	\\
0.930527811071159	0.001007080078125	\\
0.930572202246193	0.000946044921875	\\
0.930616593421228	0.0013427734375	\\
0.930660984596262	0.00128173828125	\\
0.930705375771297	0.000640869140625	\\
0.930749766946331	0.000213623046875	\\
0.930794158121365	0.000701904296875	\\
0.9308385492964	0.00140380859375	\\
0.930882940471434	0.001434326171875	\\
0.930927331646469	0.0013427734375	\\
0.930971722821503	0.001678466796875	\\
0.931016113996538	0.00146484375	\\
0.931060505171572	0.00128173828125	\\
0.931104896346606	0.001556396484375	\\
0.931149287521641	0.001007080078125	\\
0.931193678696675	0.00115966796875	\\
0.93123806987171	0.001373291015625	\\
0.931282461046744	0.001617431640625	\\
0.931326852221778	0.002410888671875	\\
0.931371243396813	0.00238037109375	\\
0.931415634571847	0.0025634765625	\\
0.931460025746881	0.00274658203125	\\
0.931504416921916	0.0020751953125	\\
0.93154880809695	0.0023193359375	\\
0.931593199271985	0.00262451171875	\\
0.931637590447019	0.0023193359375	\\
0.931681981622054	0.002197265625	\\
0.931726372797088	0.00213623046875	\\
0.931770763972122	0.002471923828125	\\
0.931815155147157	0.00274658203125	\\
0.931859546322191	0.002655029296875	\\
0.931903937497226	0.00238037109375	\\
0.93194832867226	0.00262451171875	\\
0.931992719847294	0.003173828125	\\
0.932037111022329	0.003082275390625	\\
0.932081502197363	0.002960205078125	\\
0.932125893372398	0.0035400390625	\\
0.932170284547432	0.003814697265625	\\
0.932214675722466	0.003875732421875	\\
0.932259066897501	0.00433349609375	\\
0.932303458072535	0.00457763671875	\\
0.93234784924757	0.0045166015625	\\
0.932392240422604	0.004547119140625	\\
0.932436631597638	0.00469970703125	\\
0.932481022772673	0.004852294921875	\\
0.932525413947707	0.00457763671875	\\
0.932569805122742	0.0042724609375	\\
0.932614196297776	0.004150390625	\\
0.93265858747281	0.004058837890625	\\
0.932702978647845	0.004058837890625	\\
0.932747369822879	0.003875732421875	\\
0.932791760997914	0.003875732421875	\\
0.932836152172948	0.003692626953125	\\
0.932880543347982	0.003570556640625	\\
0.932924934523017	0.003875732421875	\\
0.932969325698051	0.00384521484375	\\
0.933013716873086	0.003875732421875	\\
0.93305810804812	0.00360107421875	\\
0.933102499223154	0.00347900390625	\\
0.933146890398189	0.003570556640625	\\
0.933191281573223	0.00360107421875	\\
0.933235672748258	0.00360107421875	\\
0.933280063923292	0.003448486328125	\\
0.933324455098326	0.0032958984375	\\
0.933368846273361	0.002960205078125	\\
0.933413237448395	0.002838134765625	\\
0.93345762862343	0.00262451171875	\\
0.933502019798464	0.0025634765625	\\
0.933546410973498	0.002685546875	\\
0.933590802148533	0.002593994140625	\\
0.933635193323567	0.002593994140625	\\
0.933679584498602	0.002655029296875	\\
0.933723975673636	0.002166748046875	\\
0.93376836684867	0.001983642578125	\\
0.933812758023705	0.001861572265625	\\
0.933857149198739	0.001739501953125	\\
0.933901540373774	0.00177001953125	\\
0.933945931548808	0.001739501953125	\\
0.933990322723843	0.0018310546875	\\
0.934034713898877	0.00164794921875	\\
0.934079105073911	0.0015869140625	\\
0.934123496248946	0.00164794921875	\\
0.93416788742398	0.001495361328125	\\
0.934212278599015	0.001739501953125	\\
0.934256669774049	0.002410888671875	\\
0.934301060949083	0.002227783203125	\\
0.934345452124118	0.002197265625	\\
0.934389843299152	0.00238037109375	\\
0.934434234474187	0.00274658203125	\\
0.934478625649221	0.002655029296875	\\
0.934523016824255	0.00238037109375	\\
0.93456740799929	0.00213623046875	\\
0.934611799174324	0.00225830078125	\\
0.934656190349359	0.00244140625	\\
0.934700581524393	0.002044677734375	\\
0.934744972699427	0.00213623046875	\\
0.934789363874462	0.00201416015625	\\
0.934833755049496	0.00213623046875	\\
0.934878146224531	0.0023193359375	\\
0.934922537399565	0.002197265625	\\
0.934966928574599	0.002655029296875	\\
0.935011319749634	0.002471923828125	\\
0.935055710924668	0.002227783203125	\\
0.935100102099703	0.002349853515625	\\
0.935144493274737	0.0025634765625	\\
0.935188884449771	0.00262451171875	\\
0.935233275624806	0.002838134765625	\\
0.93527766679984	0.002838134765625	\\
0.935322057974875	0.0028076171875	\\
0.935366449149909	0.002471923828125	\\
0.935410840324943	0.002349853515625	\\
0.935455231499978	0.00225830078125	\\
0.935499622675012	0.00225830078125	\\
0.935544013850047	0.001861572265625	\\
0.935588405025081	0.001708984375	\\
0.935632796200115	0.002044677734375	\\
0.93567718737515	0.001434326171875	\\
0.935721578550184	0.001068115234375	\\
0.935765969725219	0.00115966796875	\\
0.935810360900253	0.000946044921875	\\
0.935854752075287	0.001190185546875	\\
0.935899143250322	0.001190185546875	\\
0.935943534425356	0.000579833984375	\\
0.935987925600391	0.000732421875	\\
0.936032316775425	0.001068115234375	\\
0.93607670795046	0.000823974609375	\\
0.936121099125494	0.000640869140625	\\
0.936165490300528	0.000579833984375	\\
0.936209881475563	0.000518798828125	\\
0.936254272650597	0.00048828125	\\
0.936298663825631	0.000640869140625	\\
0.936343055000666	0.0006103515625	\\
0.9363874461757	0.00048828125	\\
0.936431837350735	0.00091552734375	\\
0.936476228525769	0.0006103515625	\\
0.936520619700803	0.000579833984375	\\
0.936565010875838	0.00146484375	\\
0.936609402050872	0.0015869140625	\\
0.936653793225907	0.001312255859375	\\
0.936698184400941	0.001434326171875	\\
0.936742575575976	0.00146484375	\\
0.93678696675101	0.001678466796875	\\
0.936831357926044	0.001983642578125	\\
0.936875749101079	0.002105712890625	\\
0.936920140276113	0.002288818359375	\\
0.936964531451148	0.002197265625	\\
0.937008922626182	0.001708984375	\\
0.937053313801216	0.0015869140625	\\
0.937097704976251	0.00128173828125	\\
0.937142096151285	0.001190185546875	\\
0.937186487326319	0.00152587890625	\\
0.937230878501354	0.00140380859375	\\
0.937275269676388	0.00152587890625	\\
0.937319660851423	0.001220703125	\\
0.937364052026457	0.0009765625	\\
0.937408443201492	0.00128173828125	\\
0.937452834376526	0.001251220703125	\\
0.93749722555156	0.000885009765625	\\
0.937541616726595	0.00091552734375	\\
0.937586007901629	0.00091552734375	\\
0.937630399076664	0.000946044921875	\\
0.937674790251698	0.001251220703125	\\
0.937719181426732	0.001190185546875	\\
0.937763572601767	0.000518798828125	\\
0.937807963776801	0.0001220703125	\\
0.937852354951836	0.000244140625	\\
0.93789674612687	3.0517578125e-05	\\
0.937941137301904	0.000244140625	\\
0.937985528476939	0.0003662109375	\\
0.938029919651973	0.000457763671875	\\
0.938074310827008	0.00018310546875	\\
0.938118702002042	-0.00018310546875	\\
0.938163093177076	0	\\
0.938207484352111	-0.000152587890625	\\
0.938251875527145	-0.000579833984375	\\
0.93829626670218	-9.1552734375e-05	\\
0.938340657877214	0.000457763671875	\\
0.938385049052248	0.000579833984375	\\
0.938429440227283	0.0009765625	\\
0.938473831402317	0.000823974609375	\\
0.938518222577352	0.000213623046875	\\
0.938562613752386	0.000213623046875	\\
0.93860700492742	0.00030517578125	\\
0.938651396102455	0.0001220703125	\\
0.938695787277489	0.000335693359375	\\
0.938740178452524	0.000640869140625	\\
0.938784569627558	9.1552734375e-05	\\
0.938828960802592	0.00054931640625	\\
0.938873351977627	0.00091552734375	\\
0.938917743152661	0.0008544921875	\\
0.938962134327696	0.001190185546875	\\
0.93900652550273	0.0010986328125	\\
0.939050916677764	0.001007080078125	\\
0.939095307852799	0.00146484375	\\
0.939139699027833	0.00152587890625	\\
0.939184090202868	0.000885009765625	\\
0.939228481377902	0.000946044921875	\\
0.939272872552936	0.001495361328125	\\
0.939317263727971	0.001129150390625	\\
0.939361654903005	0.001434326171875	\\
0.93940604607804	0.00140380859375	\\
0.939450437253074	0.000732421875	\\
0.939494828428109	0.000885009765625	\\
0.939539219603143	0.0008544921875	\\
0.939583610778177	0.000823974609375	\\
0.939628001953212	0.000823974609375	\\
0.939672393128246	0.001190185546875	\\
0.939716784303281	0.001129150390625	\\
0.939761175478315	0.0010986328125	\\
0.939805566653349	0.0013427734375	\\
0.939849957828384	0.00146484375	\\
0.939894349003418	0.001495361328125	\\
0.939938740178452	0.001190185546875	\\
0.939983131353487	0.00103759765625	\\
0.940027522528521	0.001312255859375	\\
0.940071913703556	0.000946044921875	\\
0.94011630487859	0.00067138671875	\\
0.940160696053625	0.00079345703125	\\
0.940205087228659	0.000885009765625	\\
0.940249478403693	0.00128173828125	\\
0.940293869578728	0.00152587890625	\\
0.940338260753762	0.00128173828125	\\
0.940382651928797	0.001220703125	\\
0.940427043103831	0.00128173828125	\\
0.940471434278865	0.00128173828125	\\
0.9405158254539	0.001068115234375	\\
0.940560216628934	0.001251220703125	\\
0.940604607803969	0.00146484375	\\
0.940648998979003	0.001556396484375	\\
0.940693390154037	0.00164794921875	\\
0.940737781329072	0.00201416015625	\\
0.940782172504106	0.001861572265625	\\
0.940826563679141	0.00164794921875	\\
0.940870954854175	0.00201416015625	\\
0.940915346029209	0.002197265625	\\
0.940959737204244	0.002227783203125	\\
0.941004128379278	0.00286865234375	\\
0.941048519554313	0.0032958984375	\\
0.941092910729347	0.002838134765625	\\
0.941137301904381	0.002685546875	\\
0.941181693079416	0.002655029296875	\\
0.94122608425445	0.00274658203125	\\
0.941270475429485	0.00244140625	\\
0.941314866604519	0.002349853515625	\\
0.941359257779553	0.002960205078125	\\
0.941403648954588	0.0028076171875	\\
0.941448040129622	0.002716064453125	\\
0.941492431304657	0.00262451171875	\\
0.941536822479691	0.002685546875	\\
0.941581213654725	0.0023193359375	\\
0.94162560482976	0.001953125	\\
0.941669996004794	0.00225830078125	\\
0.941714387179829	0.00262451171875	\\
0.941758778354863	0.00244140625	\\
0.941803169529898	0.001861572265625	\\
0.941847560704932	0.001617431640625	\\
0.941891951879966	0.001678466796875	\\
0.941936343055001	0.00152587890625	\\
0.941980734230035	0.001617431640625	\\
0.942025125405069	0.00152587890625	\\
0.942069516580104	0.001251220703125	\\
0.942113907755138	0.001007080078125	\\
0.942158298930173	0.001251220703125	\\
0.942202690105207	0.000701904296875	\\
0.942247081280241	0.000701904296875	\\
0.942291472455276	0.000946044921875	\\
0.94233586363031	0.00067138671875	\\
0.942380254805345	0.0006103515625	\\
0.942424645980379	0.000274658203125	\\
0.942469037155414	0.0001220703125	\\
0.942513428330448	0.000335693359375	\\
0.942557819505482	0.000732421875	\\
0.942602210680517	0.000762939453125	\\
0.942646601855551	0.00067138671875	\\
0.942690993030586	0.000823974609375	\\
0.94273538420562	0.0009765625	\\
0.942779775380654	0.0013427734375	\\
0.942824166555689	0.00115966796875	\\
0.942868557730723	0.000762939453125	\\
0.942912948905758	0.000762939453125	\\
0.942957340080792	0.0006103515625	\\
0.943001731255826	0.001007080078125	\\
0.943046122430861	0.0009765625	\\
0.943090513605895	0.000885009765625	\\
0.94313490478093	0.000579833984375	\\
0.943179295955964	0.000335693359375	\\
0.943223687130998	0.000396728515625	\\
0.943268078306033	0.00054931640625	\\
0.943312469481067	0.00030517578125	\\
0.943356860656102	-0.0003662109375	\\
0.943401251831136	-0.00018310546875	\\
0.94344564300617	-0.00042724609375	\\
0.943490034181205	-0.000640869140625	\\
0.943534425356239	-0.000823974609375	\\
0.943578816531274	-0.00152587890625	\\
0.943623207706308	-0.00152587890625	\\
0.943667598881342	-0.001495361328125	\\
0.943711990056377	-0.00164794921875	\\
0.943756381231411	-0.00140380859375	\\
0.943800772406446	-0.001556396484375	\\
0.94384516358148	-0.001922607421875	\\
0.943889554756514	-0.001861572265625	\\
0.943933945931549	-0.002227783203125	\\
0.943978337106583	-0.001800537109375	\\
0.944022728281618	-0.00140380859375	\\
0.944067119456652	-0.001739501953125	\\
0.944111510631686	-0.00152587890625	\\
0.944155901806721	-0.00164794921875	\\
0.944200292981755	-0.00189208984375	\\
0.94424468415679	-0.00164794921875	\\
0.944289075331824	-0.001495361328125	\\
0.944333466506858	-0.001556396484375	\\
0.944377857681893	-0.001434326171875	\\
0.944422248856927	-0.001617431640625	\\
0.944466640031962	-0.00164794921875	\\
0.944511031206996	-0.001373291015625	\\
0.944555422382031	-0.001861572265625	\\
0.944599813557065	-0.001922607421875	\\
0.944644204732099	-0.001556396484375	\\
0.944688595907134	-0.001708984375	\\
0.944732987082168	-0.001739501953125	\\
0.944777378257202	-0.001556396484375	\\
0.944821769432237	-0.001708984375	\\
0.944866160607271	-0.001708984375	\\
0.944910551782306	-0.0018310546875	\\
0.94495494295734	-0.001617431640625	\\
0.944999334132374	-0.001708984375	\\
0.945043725307409	-0.001495361328125	\\
0.945088116482443	-0.00115966796875	\\
0.945132507657478	-0.001007080078125	\\
0.945176898832512	-0.00079345703125	\\
0.945221290007547	-0.00067138671875	\\
0.945265681182581	-0.000823974609375	\\
0.945310072357615	-0.0008544921875	\\
0.94535446353265	-0.00048828125	\\
0.945398854707684	-0.00079345703125	\\
0.945443245882719	-0.000762939453125	\\
0.945487637057753	-0.000274658203125	\\
0.945532028232787	-0.000823974609375	\\
0.945576419407822	-0.00115966796875	\\
0.945620810582856	-0.00091552734375	\\
0.94566520175789	-0.00048828125	\\
0.945709592932925	-0.00067138671875	\\
0.945753984107959	-0.0009765625	\\
0.945798375282994	-0.0009765625	\\
0.945842766458028	-0.000579833984375	\\
0.945887157633063	-0.0001220703125	\\
0.945931548808097	-0.00018310546875	\\
0.945975939983131	-0.000274658203125	\\
0.946020331158166	0.000244140625	\\
0.9460647223332	0.000518798828125	\\
0.946109113508235	0.00030517578125	\\
0.946153504683269	0.000152587890625	\\
0.946197895858303	-0.000152587890625	\\
0.946242287033338	-0.000274658203125	\\
0.946286678208372	0.000244140625	\\
0.946331069383407	0.0003662109375	\\
0.946375460558441	0.000518798828125	\\
0.946419851733475	0.00091552734375	\\
0.94646424290851	0.000885009765625	\\
0.946508634083544	0.001373291015625	\\
0.946553025258579	0.00152587890625	\\
0.946597416433613	0.00152587890625	\\
0.946641807608647	0.0018310546875	\\
0.946686198783682	0.001739501953125	\\
0.946730589958716	0.00140380859375	\\
0.946774981133751	0.00128173828125	\\
0.946819372308785	0.001373291015625	\\
0.946863763483819	0.001556396484375	\\
0.946908154658854	0.0013427734375	\\
0.946952545833888	0.00140380859375	\\
0.946996937008923	0.0018310546875	\\
0.947041328183957	0.001556396484375	\\
0.947085719358991	0.001708984375	\\
0.947130110534026	0.001800537109375	\\
0.94717450170906	0.0013427734375	\\
0.947218892884095	0.001220703125	\\
0.947263284059129	0.001129150390625	\\
0.947307675234163	0.0010986328125	\\
0.947352066409198	0.001251220703125	\\
0.947396457584232	0.0008544921875	\\
0.947440848759267	0.000762939453125	\\
0.947485239934301	0.000640869140625	\\
0.947529631109335	0.0009765625	\\
0.94757402228437	0.001129150390625	\\
0.947618413459404	0.00079345703125	\\
0.947662804634439	0.001312255859375	\\
0.947707195809473	0.001251220703125	\\
0.947751586984507	0.000946044921875	\\
0.947795978159542	0.001220703125	\\
0.947840369334576	0.00128173828125	\\
0.947884760509611	0.0010986328125	\\
0.947929151684645	0.00091552734375	\\
0.94797354285968	0.000885009765625	\\
0.948017934034714	0.001434326171875	\\
0.948062325209748	0.0018310546875	\\
0.948106716384783	0.001800537109375	\\
0.948151107559817	0.00225830078125	\\
0.948195498734852	0.00201416015625	\\
0.948239889909886	0.001922607421875	\\
0.94828428108492	0.002044677734375	\\
0.948328672259955	0.002349853515625	\\
0.948373063434989	0.00262451171875	\\
0.948417454610024	0.0023193359375	\\
0.948461845785058	0.002655029296875	\\
0.948506236960092	0.00299072265625	\\
0.948550628135127	0.00311279296875	\\
0.948595019310161	0.0029296875	\\
0.948639410485196	0.00244140625	\\
0.94868380166023	0.00244140625	\\
0.948728192835264	0.002471923828125	\\
0.948772584010299	0.00262451171875	\\
0.948816975185333	0.002532958984375	\\
0.948861366360368	0.002349853515625	\\
0.948905757535402	0.00250244140625	\\
0.948950148710436	0.002410888671875	\\
0.948994539885471	0.002105712890625	\\
0.949038931060505	0.0020751953125	\\
0.94908332223554	0.002197265625	\\
0.949127713410574	0.00244140625	\\
0.949172104585608	0.0023193359375	\\
0.949216495760643	0.001861572265625	\\
0.949260886935677	0.002044677734375	\\
0.949305278110712	0.001861572265625	\\
0.949349669285746	0.001800537109375	\\
0.94939406046078	0.001953125	\\
0.949438451635815	0.002105712890625	\\
0.949482842810849	0.00189208984375	\\
0.949527233985884	0.001953125	\\
0.949571625160918	0.0020751953125	\\
0.949616016335952	0.001617431640625	\\
0.949660407510987	0.0020751953125	\\
0.949704798686021	0.00164794921875	\\
0.949749189861056	0.0013427734375	\\
0.94979358103609	0.00152587890625	\\
0.949837972211124	0.00164794921875	\\
0.949882363386159	0.00164794921875	\\
0.949926754561193	0.001861572265625	\\
0.949971145736228	0.0018310546875	\\
0.950015536911262	0.0015869140625	\\
0.950059928086296	0.0018310546875	\\
0.950104319261331	0.002227783203125	\\
0.950148710436365	0.001678466796875	\\
0.9501931016114	0.00152587890625	\\
0.950237492786434	0.002044677734375	\\
0.950281883961469	0.002197265625	\\
0.950326275136503	0.00201416015625	\\
0.950370666311537	0.002288818359375	\\
0.950415057486572	0.00250244140625	\\
0.950459448661606	0.00244140625	\\
0.95050383983664	0.002593994140625	\\
0.950548231011675	0.002105712890625	\\
0.950592622186709	0.001800537109375	\\
0.950637013361744	0.002655029296875	\\
0.950681404536778	0.002655029296875	\\
0.950725795711812	0.0020751953125	\\
0.950770186886847	0.00213623046875	\\
0.950814578061881	0.002197265625	\\
0.950858969236916	0.002593994140625	\\
0.95090336041195	0.002838134765625	\\
0.950947751586985	0.00225830078125	\\
0.950992142762019	0.001922607421875	\\
0.951036533937053	0.002349853515625	\\
0.951080925112088	0.002105712890625	\\
0.951125316287122	0.00189208984375	\\
0.951169707462157	0.001708984375	\\
0.951214098637191	0.00164794921875	\\
0.951258489812225	0.001983642578125	\\
0.95130288098726	0.0015869140625	\\
0.951347272162294	0.0010986328125	\\
0.951391663337329	0.00091552734375	\\
0.951436054512363	0.00103759765625	\\
0.951480445687397	0.0009765625	\\
0.951524836862432	0.001068115234375	\\
0.951569228037466	0.000732421875	\\
0.951613619212501	0.000335693359375	\\
0.951658010387535	0.000518798828125	\\
0.951702401562569	0.000701904296875	\\
0.951746792737604	0.0006103515625	\\
0.951791183912638	0.000457763671875	\\
0.951835575087673	0.000274658203125	\\
0.951879966262707	-0.000213623046875	\\
0.951924357437741	0	\\
0.951968748612776	0.00018310546875	\\
0.95201313978781	0.000579833984375	\\
0.952057530962845	0.000823974609375	\\
0.952101922137879	0.000732421875	\\
0.952146313312913	0.0010986328125	\\
0.952190704487948	0.00115966796875	\\
0.952235095662982	0.001068115234375	\\
0.952279486838017	0.000885009765625	\\
0.952323878013051	0.000732421875	\\
0.952368269188085	0.0009765625	\\
0.95241266036312	0.00091552734375	\\
0.952457051538154	0.00079345703125	\\
0.952501442713189	0.000701904296875	\\
0.952545833888223	0.00048828125	\\
0.952590225063257	0.00030517578125	\\
0.952634616238292	0.000396728515625	\\
0.952679007413326	0.000518798828125	\\
0.952723398588361	0.00054931640625	\\
0.952767789763395	0.000640869140625	\\
0.952812180938429	0.000640869140625	\\
0.952856572113464	0.000244140625	\\
0.952900963288498	0.000518798828125	\\
0.952945354463533	0.00018310546875	\\
0.952989745638567	-0.000396728515625	\\
0.953034136813602	-0.000335693359375	\\
0.953078527988636	-0.000335693359375	\\
0.95312291916367	-3.0517578125e-05	\\
0.953167310338705	9.1552734375e-05	\\
0.953211701513739	-0.00048828125	\\
0.953256092688773	-0.000335693359375	\\
0.953300483863808	-0.000457763671875	\\
0.953344875038842	-0.00067138671875	\\
0.953389266213877	-0.00048828125	\\
0.953433657388911	-0.000885009765625	\\
0.953478048563945	-0.000579833984375	\\
0.95352243973898	-0.00128173828125	\\
0.953566830914014	-0.0013427734375	\\
0.953611222089049	-0.000823974609375	\\
0.953655613264083	-0.001129150390625	\\
0.953700004439118	-0.000762939453125	\\
0.953744395614152	-0.000213623046875	\\
0.953788786789186	0.000335693359375	\\
0.953833177964221	0.000213623046875	\\
0.953877569139255	0.000213623046875	\\
0.95392196031429	0.00048828125	\\
0.953966351489324	0.000274658203125	\\
0.954010742664358	0.000518798828125	\\
0.954055133839393	0.000640869140625	\\
0.954099525014427	0.0008544921875	\\
0.954143916189461	0.0009765625	\\
0.954188307364496	0.00091552734375	\\
0.95423269853953	0.000732421875	\\
0.954277089714565	0.0008544921875	\\
0.954321480889599	0.00146484375	\\
0.954365872064634	0.001739501953125	\\
0.954410263239668	0.001312255859375	\\
0.954454654414702	0.001708984375	\\
0.954499045589737	0.00164794921875	\\
0.954543436764771	0.001708984375	\\
0.954587827939806	0.0020751953125	\\
0.95463221911484	0.00213623046875	\\
0.954676610289874	0.002105712890625	\\
0.954721001464909	0.00274658203125	\\
0.954765392639943	0.00274658203125	\\
0.954809783814978	0.002593994140625	\\
0.954854174990012	0.0029296875	\\
0.954898566165046	0.0030517578125	\\
0.954942957340081	0.002899169921875	\\
0.954987348515115	0.00238037109375	\\
0.95503173969015	0.00250244140625	\\
0.955076130865184	0.002685546875	\\
0.955120522040218	0.002716064453125	\\
0.955164913215253	0.002655029296875	\\
0.955209304390287	0.00262451171875	\\
0.955253695565322	0.00274658203125	\\
0.955298086740356	0.00262451171875	\\
0.95534247791539	0.002349853515625	\\
0.955386869090425	0.0029296875	\\
0.955431260265459	0.0028076171875	\\
0.955475651440494	0.002593994140625	\\
0.955520042615528	0.002899169921875	\\
0.955564433790562	0.00274658203125	\\
0.955608824965597	0.00244140625	\\
0.955653216140631	0.00201416015625	\\
0.955697607315666	0.001678466796875	\\
0.9557419984907	0.001617431640625	\\
0.955786389665734	0.001800537109375	\\
0.955830780840769	0.001800537109375	\\
0.955875172015803	0.001312255859375	\\
0.955919563190838	0.001312255859375	\\
0.955963954365872	0.001617431640625	\\
0.956008345540906	0.001220703125	\\
0.956052736715941	0.00189208984375	\\
0.956097127890975	0.002227783203125	\\
0.95614151906601	0.002349853515625	\\
0.956185910241044	0.00250244140625	\\
0.956230301416078	0.002105712890625	\\
0.956274692591113	0.002197265625	\\
0.956319083766147	0.002532958984375	\\
0.956363474941182	0.0020751953125	\\
0.956407866116216	0.002105712890625	\\
0.956452257291251	0.002197265625	\\
0.956496648466285	0.0023193359375	\\
0.956541039641319	0.0025634765625	\\
0.956585430816354	0.002166748046875	\\
0.956629821991388	0.00213623046875	\\
0.956674213166423	0.002105712890625	\\
0.956718604341457	0.002410888671875	\\
0.956762995516491	0.002471923828125	\\
0.956807386691526	0.00225830078125	\\
0.95685177786656	0.00225830078125	\\
0.956896169041595	0.00238037109375	\\
0.956940560216629	0.0018310546875	\\
0.956984951391663	0.00189208984375	\\
0.957029342566698	0.001983642578125	\\
0.957073733741732	0.002166748046875	\\
0.957118124916767	0.00238037109375	\\
0.957162516091801	0.0023193359375	\\
0.957206907266835	0.002227783203125	\\
0.95725129844187	0.002410888671875	\\
0.957295689616904	0.002655029296875	\\
0.957340080791939	0.002166748046875	\\
0.957384471966973	0.001983642578125	\\
0.957428863142007	0.001922607421875	\\
0.957473254317042	0.001617431640625	\\
0.957517645492076	0.001739501953125	\\
0.957562036667111	0.001495361328125	\\
0.957606427842145	0.00079345703125	\\
0.957650819017179	0.00048828125	\\
0.957695210192214	0.00042724609375	\\
0.957739601367248	0.0008544921875	\\
0.957783992542283	0.0010986328125	\\
0.957828383717317	0.000946044921875	\\
0.957872774892351	0.000701904296875	\\
0.957917166067386	0.000579833984375	\\
0.95796155724242	0.00048828125	\\
0.958005948417455	0.000762939453125	\\
0.958050339592489	0.000732421875	\\
0.958094730767523	9.1552734375e-05	\\
0.958139121942558	0.00042724609375	\\
0.958183513117592	0.000885009765625	\\
0.958227904292627	0.000823974609375	\\
0.958272295467661	0.00079345703125	\\
0.958316686642695	0.00054931640625	\\
0.95836107781773	0.00048828125	\\
0.958405468992764	0.001068115234375	\\
0.958449860167799	0.0008544921875	\\
0.958494251342833	0.00042724609375	\\
0.958538642517867	0.0009765625	\\
0.958583033692902	0.00030517578125	\\
0.958627424867936	3.0517578125e-05	\\
0.958671816042971	0.000396728515625	\\
0.958716207218005	0.000885009765625	\\
0.95876059839304	0.001312255859375	\\
0.958804989568074	0.001007080078125	\\
0.958849380743108	0.000946044921875	\\
0.958893771918143	0.001220703125	\\
0.958938163093177	0.001220703125	\\
0.958982554268211	0.001220703125	\\
0.959026945443246	0.001129150390625	\\
0.95907133661828	0.0008544921875	\\
0.959115727793315	0.000579833984375	\\
0.959160118968349	0.00067138671875	\\
0.959204510143383	0.00048828125	\\
0.959248901318418	0.000396728515625	\\
0.959293292493452	0.000396728515625	\\
0.959337683668487	0.0001220703125	\\
0.959382074843521	0.000152587890625	\\
0.959426466018556	0.000335693359375	\\
0.95947085719359	0.000213623046875	\\
0.959515248368624	0.000701904296875	\\
0.959559639543659	0.001007080078125	\\
0.959604030718693	0.00079345703125	\\
0.959648421893728	0.0006103515625	\\
0.959692813068762	0.000732421875	\\
0.959737204243796	0.00030517578125	\\
0.959781595418831	0.000396728515625	\\
0.959825986593865	0.000579833984375	\\
0.9598703777689	-0.0001220703125	\\
0.959914768943934	-0.0003662109375	\\
0.959959160118968	-0.00048828125	\\
0.960003551294003	-0.000274658203125	\\
0.960047942469037	-0.000701904296875	\\
0.960092333644072	-0.001251220703125	\\
0.960136724819106	-0.000946044921875	\\
0.96018111599414	-0.000946044921875	\\
0.960225507169175	-0.001068115234375	\\
0.960269898344209	-0.000946044921875	\\
0.960314289519244	-0.00067138671875	\\
0.960358680694278	-0.00115966796875	\\
0.960403071869312	-0.001922607421875	\\
0.960447463044347	-0.00213623046875	\\
0.960491854219381	-0.0023193359375	\\
0.960536245394416	-0.002288818359375	\\
0.96058063656945	-0.002166748046875	\\
0.960625027744484	-0.00244140625	\\
0.960669418919519	-0.002227783203125	\\
0.960713810094553	-0.002166748046875	\\
0.960758201269588	-0.002227783203125	\\
0.960802592444622	-0.00262451171875	\\
0.960846983619656	-0.00274658203125	\\
0.960891374794691	-0.00274658203125	\\
0.960935765969725	-0.002960205078125	\\
0.96098015714476	-0.002655029296875	\\
0.961024548319794	-0.00238037109375	\\
0.961068939494828	-0.002471923828125	\\
0.961113330669863	-0.002685546875	\\
0.961157721844897	-0.002777099609375	\\
0.961202113019932	-0.002655029296875	\\
0.961246504194966	-0.002593994140625	\\
0.96129089537	-0.00262451171875	\\
0.961335286545035	-0.003204345703125	\\
0.961379677720069	-0.003448486328125	\\
0.961424068895104	-0.003082275390625	\\
0.961468460070138	-0.00262451171875	\\
0.961512851245173	-0.00244140625	\\
0.961557242420207	-0.001953125	\\
0.961601633595241	-0.002044677734375	\\
0.961646024770276	-0.002471923828125	\\
0.96169041594531	-0.002288818359375	\\
0.961734807120344	-0.002471923828125	\\
0.961779198295379	-0.002532958984375	\\
0.961823589470413	-0.002838134765625	\\
0.961867980645448	-0.00286865234375	\\
0.961912371820482	-0.002716064453125	\\
0.961956762995516	-0.00262451171875	\\
0.962001154170551	-0.00262451171875	\\
0.962045545345585	-0.002960205078125	\\
0.96208993652062	-0.00250244140625	\\
0.962134327695654	-0.00238037109375	\\
0.962178718870689	-0.002532958984375	\\
0.962223110045723	-0.0023193359375	\\
0.962267501220757	-0.0018310546875	\\
0.962311892395792	-0.001800537109375	\\
0.962356283570826	-0.001983642578125	\\
0.962400674745861	-0.001983642578125	\\
0.962445065920895	-0.00244140625	\\
0.962489457095929	-0.002471923828125	\\
0.962533848270964	-0.002471923828125	\\
0.962578239445998	-0.00244140625	\\
0.962622630621033	-0.00244140625	\\
0.962667021796067	-0.00213623046875	\\
0.962711412971101	-0.00189208984375	\\
0.962755804146136	-0.00140380859375	\\
0.96280019532117	-0.001129150390625	\\
0.962844586496205	-0.001190185546875	\\
0.962888977671239	-0.00128173828125	\\
0.962933368846273	-0.001556396484375	\\
0.962977760021308	-0.001617431640625	\\
0.963022151196342	-0.00140380859375	\\
0.963066542371377	-0.001007080078125	\\
0.963110933546411	-0.001190185546875	\\
0.963155324721445	-0.001007080078125	\\
0.96319971589648	-0.000762939453125	\\
0.963244107071514	-0.000579833984375	\\
0.963288498246549	-0.000274658203125	\\
0.963332889421583	-0.00042724609375	\\
0.963377280596617	-0.000274658203125	\\
0.963421671771652	-0.00030517578125	\\
0.963466062946686	-0.000335693359375	\\
0.963510454121721	-0.000274658203125	\\
0.963554845296755	-0.000152587890625	\\
0.963599236471789	-0.000152587890625	\\
0.963643627646824	-0.000518798828125	\\
0.963688018821858	-6.103515625e-05	\\
0.963732409996893	0.000335693359375	\\
0.963776801171927	0.000732421875	\\
0.963821192346961	0.001373291015625	\\
0.963865583521996	0.001434326171875	\\
0.96390997469703	0.00140380859375	\\
0.963954365872065	0.00091552734375	\\
0.963998757047099	0.00115966796875	\\
0.964043148222133	0.001434326171875	\\
0.964087539397168	0.00103759765625	\\
0.964131930572202	0.001312255859375	\\
0.964176321747237	0.00164794921875	\\
0.964220712922271	0.00140380859375	\\
0.964265104097305	0.00164794921875	\\
0.96430949527234	0.0018310546875	\\
0.964353886447374	0.001495361328125	\\
0.964398277622409	0.001556396484375	\\
0.964442668797443	0.001708984375	\\
0.964487059972478	0.001861572265625	\\
0.964531451147512	0.001922607421875	\\
0.964575842322546	0.00140380859375	\\
0.964620233497581	0.000732421875	\\
0.964664624672615	0.000457763671875	\\
0.964709015847649	0.00048828125	\\
0.964753407022684	0.000701904296875	\\
0.964797798197718	0.000823974609375	\\
0.964842189372753	0.000762939453125	\\
0.964886580547787	0.000823974609375	\\
0.964930971722822	0.0006103515625	\\
0.964975362897856	0.0010986328125	\\
0.96501975407289	0.001129150390625	\\
0.965064145247925	0.000640869140625	\\
0.965108536422959	0.00091552734375	\\
0.965152927597994	0.000274658203125	\\
0.965197318773028	-0.000457763671875	\\
0.965241709948062	3.0517578125e-05	\\
0.965286101123097	0.000152587890625	\\
0.965330492298131	0	\\
0.965374883473166	-0.000213623046875	\\
0.9654192746482	-6.103515625e-05	\\
0.965463665823234	-3.0517578125e-05	\\
0.965508056998269	3.0517578125e-05	\\
0.965552448173303	0.00018310546875	\\
0.965596839348338	-0.000244140625	\\
0.965641230523372	-0.0003662109375	\\
0.965685621698406	-0.000274658203125	\\
0.965730012873441	-0.000213623046875	\\
0.965774404048475	-6.103515625e-05	\\
0.96581879522351	0.00018310546875	\\
0.965863186398544	0.00054931640625	\\
0.965907577573578	0.00054931640625	\\
0.965951968748613	0.00079345703125	\\
0.965996359923647	0.0015869140625	\\
0.966040751098682	0.00201416015625	\\
0.966085142273716	0.00213623046875	\\
0.96612953344875	0.00177001953125	\\
0.966173924623785	0.00201416015625	\\
0.966218315798819	0.002197265625	\\
0.966262706973854	0.001739501953125	\\
0.966307098148888	0.002410888671875	\\
0.966351489323922	0.002105712890625	\\
0.966395880498957	0.001251220703125	\\
0.966440271673991	0.001708984375	\\
0.966484662849026	0.00152587890625	\\
0.96652905402406	0.00164794921875	\\
0.966573445199094	0.001983642578125	\\
0.966617836374129	0.00164794921875	\\
0.966662227549163	0.001617431640625	\\
0.966706618724198	0.00152587890625	\\
0.966751009899232	0.00103759765625	\\
0.966795401074266	0.00079345703125	\\
0.966839792249301	0.00091552734375	\\
0.966884183424335	0.00091552734375	\\
0.96692857459937	0.000885009765625	\\
0.966972965774404	0.000762939453125	\\
0.967017356949438	0.001129150390625	\\
0.967061748124473	0.00115966796875	\\
0.967106139299507	0.001129150390625	\\
0.967150530474542	0.00164794921875	\\
0.967194921649576	0.001251220703125	\\
0.967239312824611	0.001129150390625	\\
0.967283703999645	0.00140380859375	\\
0.967328095174679	0.00079345703125	\\
0.967372486349714	0.000244140625	\\
0.967416877524748	0.000335693359375	\\
0.967461268699782	6.103515625e-05	\\
0.967505659874817	-9.1552734375e-05	\\
0.967550051049851	0.000335693359375	\\
0.967594442224886	0.00067138671875	\\
0.96763883339992	0.0008544921875	\\
0.967683224574954	0.000885009765625	\\
0.967727615749989	0.00091552734375	\\
0.967772006925023	0.000762939453125	\\
0.967816398100058	0.000701904296875	\\
0.967860789275092	0.000152587890625	\\
0.967905180450127	0.0001220703125	\\
0.967949571625161	0.000579833984375	\\
0.967993962800195	0.000457763671875	\\
0.96803835397523	0.0003662109375	\\
0.968082745150264	0.0006103515625	\\
0.968127136325299	0.00054931640625	\\
0.968171527500333	0.0010986328125	\\
0.968215918675367	0.00128173828125	\\
0.968260309850402	0.001312255859375	\\
0.968304701025436	0.001190185546875	\\
0.968349092200471	0.0008544921875	\\
0.968393483375505	0.000335693359375	\\
0.968437874550539	0.000213623046875	\\
0.968482265725574	0.0006103515625	\\
0.968526656900608	0.000518798828125	\\
0.968571048075643	0	\\
0.968615439250677	0.00030517578125	\\
0.968659830425711	0.000640869140625	\\
0.968704221600746	0.0003662109375	\\
0.96874861277578	0.0003662109375	\\
0.968793003950815	0.000274658203125	\\
0.968837395125849	-0.0001220703125	\\
0.968881786300883	-0.000244140625	\\
0.968926177475918	-0.0003662109375	\\
0.968970568650952	-0.000885009765625	\\
0.969014959825987	-0.001007080078125	\\
0.969059351001021	-0.00115966796875	\\
0.969103742176055	-0.00103759765625	\\
0.96914813335109	-0.00067138671875	\\
0.969192524526124	-0.000762939453125	\\
0.969236915701159	-0.000885009765625	\\
0.969281306876193	-0.001007080078125	\\
0.969325698051227	-0.001190185546875	\\
0.969370089226262	-0.001373291015625	\\
0.969414480401296	-0.001678466796875	\\
0.969458871576331	-0.001495361328125	\\
0.969503262751365	-0.001678466796875	\\
0.969547653926399	-0.001556396484375	\\
0.969592045101434	-0.00140380859375	\\
0.969636436276468	-0.001495361328125	\\
0.969680827451503	-0.00128173828125	\\
0.969725218626537	-0.001007080078125	\\
0.969769609801571	-0.00091552734375	\\
0.969814000976606	-0.0008544921875	\\
0.96985839215164	-0.00091552734375	\\
0.969902783326675	-0.000885009765625	\\
0.969947174501709	-0.0008544921875	\\
0.969991565676743	-0.0010986328125	\\
0.970035956851778	-0.00128173828125	\\
0.970080348026812	-0.001312255859375	\\
0.970124739201847	-0.001220703125	\\
0.970169130376881	-0.0006103515625	\\
0.970213521551915	-0.00048828125	\\
0.97025791272695	-0.000518798828125	\\
0.970302303901984	-0.000518798828125	\\
0.970346695077019	-0.000518798828125	\\
0.970391086252053	0.00054931640625	\\
0.970435477427087	0.00048828125	\\
0.970479868602122	0.000152587890625	\\
0.970524259777156	0.000457763671875	\\
0.970568650952191	0.000335693359375	\\
0.970613042127225	3.0517578125e-05	\\
0.97065743330226	-0.0001220703125	\\
0.970701824477294	0.0001220703125	\\
0.970746215652328	0	\\
0.970790606827363	-0.000152587890625	\\
0.970834998002397	0	\\
0.970879389177432	0.000457763671875	\\
0.970923780352466	0.00018310546875	\\
0.9709681715275	0.00030517578125	\\
0.971012562702535	0.000152587890625	\\
0.971056953877569	0.0001220703125	\\
0.971101345052604	0	\\
0.971145736227638	-0.000762939453125	\\
0.971190127402672	-0.0009765625	\\
0.971234518577707	-0.000732421875	\\
0.971278909752741	-0.0003662109375	\\
0.971323300927776	-0.000244140625	\\
0.97136769210281	-0.000457763671875	\\
0.971412083277844	-0.0003662109375	\\
0.971456474452879	0	\\
0.971500865627913	0.0001220703125	\\
0.971545256802948	9.1552734375e-05	\\
0.971589647977982	-0.00048828125	\\
0.971634039153016	-0.000518798828125	\\
0.971678430328051	-0.000396728515625	\\
0.971722821503085	-0.000152587890625	\\
0.97176721267812	-6.103515625e-05	\\
0.971811603853154	-0.000518798828125	\\
0.971855995028188	-3.0517578125e-05	\\
0.971900386203223	0.000274658203125	\\
0.971944777378257	0.0003662109375	\\
0.971989168553292	0.00115966796875	\\
0.972033559728326	0.001190185546875	\\
0.97207795090336	0.001007080078125	\\
0.972122342078395	0.0009765625	\\
0.972166733253429	0.00091552734375	\\
0.972211124428464	0.000885009765625	\\
0.972255515603498	0.00152587890625	\\
0.972299906778532	0.002166748046875	\\
0.972344297953567	0.001953125	\\
0.972388689128601	0.002105712890625	\\
0.972433080303636	0.002716064453125	\\
0.97247747147867	0.003082275390625	\\
0.972521862653704	0.003143310546875	\\
0.972566253828739	0.003143310546875	\\
0.972610645003773	0.00286865234375	\\
0.972655036178808	0.002471923828125	\\
0.972699427353842	0.00262451171875	\\
0.972743818528876	0.002532958984375	\\
0.972788209703911	0.002532958984375	\\
0.972832600878945	0.002838134765625	\\
0.97287699205398	0.002777099609375	\\
0.972921383229014	0.0028076171875	\\
0.972965774404049	0.00311279296875	\\
0.973010165579083	0.002655029296875	\\
0.973054556754117	0.002410888671875	\\
0.973098947929152	0.002838134765625	\\
0.973143339104186	0.003021240234375	\\
0.97318773027922	0.002716064453125	\\
0.973232121454255	0.002349853515625	\\
0.973276512629289	0.002044677734375	\\
0.973320903804324	0.001953125	\\
0.973365294979358	0.00225830078125	\\
0.973409686154393	0.00244140625	\\
0.973454077329427	0.002471923828125	\\
0.973498468504461	0.002349853515625	\\
0.973542859679496	0.002166748046875	\\
0.97358725085453	0.00244140625	\\
0.973631642029565	0.002532958984375	\\
0.973676033204599	0.001922607421875	\\
0.973720424379633	0.001953125	\\
0.973764815554668	0.001953125	\\
0.973809206729702	0.001373291015625	\\
0.973853597904737	0.00140380859375	\\
0.973897989079771	0.001678466796875	\\
0.973942380254805	0.001434326171875	\\
0.97398677142984	0.0018310546875	\\
0.974031162604874	0.0023193359375	\\
0.974075553779909	0.002044677734375	\\
0.974119944954943	0.002410888671875	\\
0.974164336129977	0.002288818359375	\\
0.974208727305012	0.00238037109375	\\
0.974253118480046	0.001739501953125	\\
0.974297509655081	0.00128173828125	\\
0.974341900830115	0.001495361328125	\\
0.974386292005149	0.001556396484375	\\
0.974430683180184	0.00146484375	\\
0.974475074355218	0.001434326171875	\\
0.974519465530253	0.00140380859375	\\
0.974563856705287	0.00177001953125	\\
0.974608247880321	0.002197265625	\\
0.974652639055356	0.001922607421875	\\
0.97469703023039	0.001922607421875	\\
0.974741421405425	0.002105712890625	\\
0.974785812580459	0.00152587890625	\\
0.974830203755493	0.0013427734375	\\
0.974874594930528	0.00140380859375	\\
0.974918986105562	0.001312255859375	\\
0.974963377280597	0.00128173828125	\\
0.975007768455631	0.001373291015625	\\
0.975052159630665	0.001434326171875	\\
0.9750965508057	0.00152587890625	\\
0.975140941980734	0.001556396484375	\\
0.975185333155769	0.001617431640625	\\
0.975229724330803	0.001861572265625	\\
0.975274115505837	0.00177001953125	\\
0.975318506680872	0.0010986328125	\\
0.975362897855906	0.000946044921875	\\
0.975407289030941	0.00146484375	\\
0.975451680205975	0.001800537109375	\\
0.975496071381009	0.001434326171875	\\
0.975540462556044	0.00146484375	\\
0.975584853731078	0.001556396484375	\\
0.975629244906113	0.00189208984375	\\
0.975673636081147	0.002227783203125	\\
0.975718027256182	0.0020751953125	\\
0.975762418431216	0.002410888671875	\\
0.97580680960625	0.002349853515625	\\
0.975851200781285	0.002227783203125	\\
0.975895591956319	0.0025634765625	\\
0.975939983131353	0.002593994140625	\\
0.975984374306388	0.0023193359375	\\
0.976028765481422	0.002227783203125	\\
0.976073156656457	0.002044677734375	\\
0.976117547831491	0.002227783203125	\\
0.976161939006525	0.001953125	\\
0.97620633018156	0.00177001953125	\\
0.976250721356594	0.001861572265625	\\
0.976295112531629	0.001708984375	\\
0.976339503706663	0.00177001953125	\\
0.976383894881698	0.001617431640625	\\
0.976428286056732	0.001373291015625	\\
0.976472677231766	0.001434326171875	\\
0.976517068406801	0.00091552734375	\\
0.976561459581835	0.000701904296875	\\
0.97660585075687	0.0010986328125	\\
0.976650241931904	0.00054931640625	\\
0.976694633106938	0.00048828125	\\
0.976739024281973	0.000762939453125	\\
0.976783415457007	0.0008544921875	\\
0.976827806632042	0.000885009765625	\\
0.976872197807076	0.000946044921875	\\
0.97691658898211	0.000823974609375	\\
0.976960980157145	0.000732421875	\\
0.977005371332179	0.000335693359375	\\
0.977049762507214	-0.000335693359375	\\
0.977094153682248	0.000152587890625	\\
0.977138544857282	0.000244140625	\\
0.977182936032317	9.1552734375e-05	\\
0.977227327207351	0.000823974609375	\\
0.977271718382386	0.000579833984375	\\
0.97731610955742	0.0003662109375	\\
0.977360500732454	0.000885009765625	\\
0.977404891907489	0.000518798828125	\\
0.977449283082523	0.000457763671875	\\
0.977493674257558	0.000274658203125	\\
0.977538065432592	0.0003662109375	\\
0.977582456607626	0.000518798828125	\\
0.977626847782661	0	\\
0.977671238957695	0.00048828125	\\
0.97771563013273	0.00018310546875	\\
0.977760021307764	0.00042724609375	\\
0.977804412482798	0.000274658203125	\\
0.977848803657833	0.000152587890625	\\
0.977893194832867	0.000396728515625	\\
0.977937586007902	0.0001220703125	\\
0.977981977182936	0.000213623046875	\\
0.97802636835797	-3.0517578125e-05	\\
0.978070759533005	-0.000457763671875	\\
0.978115150708039	-0.000732421875	\\
0.978159541883074	-0.000732421875	\\
0.978203933058108	-0.000244140625	\\
0.978248324233142	-0.000518798828125	\\
0.978292715408177	-0.00054931640625	\\
0.978337106583211	-0.00079345703125	\\
0.978381497758246	-0.00103759765625	\\
0.97842588893328	-0.001007080078125	\\
0.978470280108314	-0.001068115234375	\\
0.978514671283349	-0.00115966796875	\\
0.978559062458383	-0.001708984375	\\
0.978603453633418	-0.00189208984375	\\
0.978647844808452	-0.001739501953125	\\
0.978692235983487	-0.00146484375	\\
0.978736627158521	-0.00103759765625	\\
0.978781018333555	-0.001312255859375	\\
0.97882540950859	-0.001129150390625	\\
0.978869800683624	-0.001495361328125	\\
0.978914191858658	-0.002044677734375	\\
0.978958583033693	-0.001312255859375	\\
0.979002974208727	-0.00140380859375	\\
0.979047365383762	-0.001678466796875	\\
0.979091756558796	-0.001434326171875	\\
0.979136147733831	-0.001373291015625	\\
0.979180538908865	-0.000885009765625	\\
0.979224930083899	-0.0009765625	\\
0.979269321258934	-0.000732421875	\\
0.979313712433968	-0.000640869140625	\\
0.979358103609003	-0.001007080078125	\\
0.979402494784037	-0.00067138671875	\\
0.979446885959071	-0.0006103515625	\\
0.979491277134106	-0.00067138671875	\\
0.97953566830914	-0.000579833984375	\\
0.979580059484175	-0.0009765625	\\
0.979624450659209	-0.0010986328125	\\
0.979668841834243	-0.000823974609375	\\
0.979713233009278	-0.000457763671875	\\
0.979757624184312	-0.000335693359375	\\
0.979802015359347	-0.000701904296875	\\
0.979846406534381	-0.000518798828125	\\
0.979890797709415	-0.000396728515625	\\
0.97993518888445	-0.00030517578125	\\
0.979979580059484	-0.00030517578125	\\
0.980023971234519	0	\\
0.980068362409553	6.103515625e-05	\\
0.980112753584587	-9.1552734375e-05	\\
0.980157144759622	-0.000152587890625	\\
0.980201535934656	3.0517578125e-05	\\
0.980245927109691	0.000244140625	\\
0.980290318284725	0.00067138671875	\\
0.980334709459759	0.00042724609375	\\
0.980379100634794	0.000244140625	\\
0.980423491809828	0.000518798828125	\\
0.980467882984863	0.000457763671875	\\
0.980512274159897	0.000335693359375	\\
0.980556665334931	-0.000335693359375	\\
0.980601056509966	-0.00054931640625	\\
0.980645447685	6.103515625e-05	\\
0.980689838860035	0.000152587890625	\\
0.980734230035069	0.0003662109375	\\
0.980778621210103	0.000457763671875	\\
0.980823012385138	0.000457763671875	\\
0.980867403560172	0.00048828125	\\
0.980911794735207	0.000640869140625	\\
0.980956185910241	0.00079345703125	\\
0.981000577085275	0.000640869140625	\\
0.98104496826031	0.000213623046875	\\
0.981089359435344	3.0517578125e-05	\\
0.981133750610379	0	\\
0.981178141785413	0	\\
0.981222532960447	0.0006103515625	\\
0.981266924135482	0.0010986328125	\\
0.981311315310516	0.00115966796875	\\
0.981355706485551	0.0015869140625	\\
0.981400097660585	0.002105712890625	\\
0.98144448883562	0.00213623046875	\\
0.981488880010654	0.002166748046875	\\
0.981533271185688	0.002349853515625	\\
0.981577662360723	0.001953125	\\
0.981622053535757	0.00189208984375	\\
0.981666444710791	0.00201416015625	\\
0.981710835885826	0.002044677734375	\\
0.98175522706086	0.002532958984375	\\
0.981799618235895	0.002685546875	\\
0.981844009410929	0.002838134765625	\\
0.981888400585963	0.002471923828125	\\
0.981932791760998	0.002410888671875	\\
0.981977182936032	0.0029296875	\\
0.982021574111067	0.00274658203125	\\
0.982065965286101	0.002960205078125	\\
0.982110356461136	0.0023193359375	\\
0.98215474763617	0.002105712890625	\\
0.982199138811204	0.00274658203125	\\
0.982243529986239	0.00244140625	\\
0.982287921161273	0.001739501953125	\\
0.982332312336308	0.001220703125	\\
0.982376703511342	0.00152587890625	\\
0.982421094686376	0.001983642578125	\\
0.982465485861411	0.002349853515625	\\
0.982509877036445	0.002288818359375	\\
0.98255426821148	0.001800537109375	\\
0.982598659386514	0.00189208984375	\\
0.982643050561548	0.00177001953125	\\
0.982687441736583	0.001556396484375	\\
0.982731832911617	0.001220703125	\\
0.982776224086652	0.000640869140625	\\
0.982820615261686	0.00018310546875	\\
0.98286500643672	0	\\
0.982909397611755	-6.103515625e-05	\\
0.982953788786789	0.00018310546875	\\
0.982998179961824	3.0517578125e-05	\\
0.983042571136858	-0.00018310546875	\\
0.983086962311892	-0.00018310546875	\\
0.983131353486927	0.000213623046875	\\
0.983175744661961	0.000457763671875	\\
0.983220135836996	-0.000213623046875	\\
0.98326452701203	-0.000518798828125	\\
0.983308918187064	-0.00079345703125	\\
0.983353309362099	-0.000946044921875	\\
0.983397700537133	-0.00067138671875	\\
0.983442091712168	-0.000579833984375	\\
0.983486482887202	-0.000457763671875	\\
0.983530874062236	-0.00042724609375	\\
0.983575265237271	-0.0001220703125	\\
0.983619656412305	9.1552734375e-05	\\
0.98366404758734	0	\\
0.983708438762374	6.103515625e-05	\\
0.983752829937408	6.103515625e-05	\\
0.983797221112443	9.1552734375e-05	\\
0.983841612287477	-9.1552734375e-05	\\
0.983886003462512	6.103515625e-05	\\
0.983930394637546	0.000335693359375	\\
0.98397478581258	0.000213623046875	\\
0.984019176987615	0.00030517578125	\\
0.984063568162649	0.000274658203125	\\
0.984107959337684	0.000335693359375	\\
0.984152350512718	0.000701904296875	\\
0.984196741687753	0.00067138671875	\\
0.984241132862787	0.000640869140625	\\
0.984285524037821	0.000396728515625	\\
0.984329915212856	0.000579833984375	\\
0.98437430638789	0.001068115234375	\\
0.984418697562924	0.0009765625	\\
0.984463088737959	0.001068115234375	\\
0.984507479912993	0.000396728515625	\\
0.984551871088028	9.1552734375e-05	\\
0.984596262263062	0.000579833984375	\\
0.984640653438096	0.000335693359375	\\
0.984685044613131	0.00018310546875	\\
0.984729435788165	-0.000335693359375	\\
0.9847738269632	-0.000274658203125	\\
0.984818218138234	-0.00042724609375	\\
0.984862609313269	0.000213623046875	\\
0.984907000488303	0.00054931640625	\\
0.984951391663337	0.0003662109375	\\
0.984995782838372	0.000396728515625	\\
0.985040174013406	0.0001220703125	\\
0.985084565188441	-6.103515625e-05	\\
0.985128956363475	-0.00042724609375	\\
0.985173347538509	-0.00048828125	\\
0.985217738713544	-0.000335693359375	\\
0.985262129888578	-0.00030517578125	\\
0.985306521063613	-0.00018310546875	\\
0.985350912238647	-0.000396728515625	\\
0.985395303413681	-0.00030517578125	\\
0.985439694588716	9.1552734375e-05	\\
0.98548408576375	0.000732421875	\\
0.985528476938785	0.000946044921875	\\
0.985572868113819	0.000396728515625	\\
0.985617259288853	0.000213623046875	\\
0.985661650463888	-3.0517578125e-05	\\
0.985706041638922	-6.103515625e-05	\\
0.985750432813957	-0.000244140625	\\
0.985794823988991	-0.0001220703125	\\
0.985839215164025	0.000152587890625	\\
0.98588360633906	-3.0517578125e-05	\\
0.985927997514094	0.000244140625	\\
0.985972388689129	0.000274658203125	\\
0.986016779864163	0.00054931640625	\\
0.986061171039197	0.000732421875	\\
0.986105562214232	0.000518798828125	\\
0.986149953389266	0.0006103515625	\\
0.986194344564301	0	\\
0.986238735739335	-3.0517578125e-05	\\
0.986283126914369	6.103515625e-05	\\
0.986327518089404	-0.000457763671875	\\
0.986371909264438	-0.000396728515625	\\
0.986416300439473	-0.00030517578125	\\
0.986460691614507	-0.000640869140625	\\
0.986505082789541	-0.000885009765625	\\
0.986549473964576	-0.000701904296875	\\
0.98659386513961	-0.00030517578125	\\
0.986638256314645	9.1552734375e-05	\\
0.986682647489679	0.00018310546875	\\
0.986727038664713	0.000701904296875	\\
0.986771429839748	0.000396728515625	\\
0.986815821014782	0.000518798828125	\\
0.986860212189817	0.000885009765625	\\
0.986904603364851	0.000579833984375	\\
0.986948994539885	0.0001220703125	\\
0.98699338571492	9.1552734375e-05	\\
0.987037776889954	6.103515625e-05	\\
0.987082168064989	9.1552734375e-05	\\
0.987126559240023	0.000244140625	\\
0.987170950415058	0.00042724609375	\\
0.987215341590092	9.1552734375e-05	\\
0.987259732765126	0	\\
0.987304123940161	0.0003662109375	\\
0.987348515115195	0.0006103515625	\\
0.987392906290229	0.00042724609375	\\
0.987437297465264	0.00042724609375	\\
0.987481688640298	0.000640869140625	\\
0.987526079815333	0.00042724609375	\\
0.987570470990367	-0.00030517578125	\\
0.987614862165402	-0.000152587890625	\\
0.987659253340436	0	\\
0.98770364451547	-0.0001220703125	\\
0.987748035690505	0.00030517578125	\\
0.987792426865539	0.0006103515625	\\
0.987836818040574	0.001129150390625	\\
0.987881209215608	0.0006103515625	\\
0.987925600390642	0.000274658203125	\\
0.987969991565677	0.000213623046875	\\
0.988014382740711	0.00042724609375	\\
0.988058773915746	0.000732421875	\\
0.98810316509078	0.000274658203125	\\
0.988147556265814	0.00030517578125	\\
0.988191947440849	0.00091552734375	\\
0.988236338615883	0.001251220703125	\\
0.988280729790918	0.001190185546875	\\
0.988325120965952	0.00146484375	\\
0.988369512140986	0.00164794921875	\\
0.988413903316021	0.00152587890625	\\
0.988458294491055	0.00146484375	\\
0.98850268566609	0.00128173828125	\\
0.988547076841124	0.0008544921875	\\
0.988591468016158	0.000762939453125	\\
0.988635859191193	0.0009765625	\\
0.988680250366227	0.0009765625	\\
0.988724641541262	0.001190185546875	\\
0.988769032716296	0.001373291015625	\\
0.98881342389133	0.00152587890625	\\
0.988857815066365	0.00177001953125	\\
0.988902206241399	0.001983642578125	\\
0.988946597416434	0.001739501953125	\\
0.988990988591468	0.001495361328125	\\
0.989035379766502	0.001678466796875	\\
0.989079770941537	0.0020751953125	\\
0.989124162116571	0.00164794921875	\\
0.989168553291606	0.001708984375	\\
0.98921294446664	0.0020751953125	\\
0.989257335641674	0.001922607421875	\\
0.989301726816709	0.0018310546875	\\
0.989346117991743	0.001983642578125	\\
0.989390509166778	0.0018310546875	\\
0.989434900341812	0.001434326171875	\\
0.989479291516846	0.0008544921875	\\
0.989523682691881	0.000762939453125	\\
0.989568073866915	0.00079345703125	\\
0.98961246504195	0.00067138671875	\\
0.989656856216984	0.00079345703125	\\
0.989701247392018	0.00054931640625	\\
0.989745638567053	0.001007080078125	\\
0.989790029742087	0.001434326171875	\\
0.989834420917122	0.001678466796875	\\
0.989878812092156	0.001312255859375	\\
0.989923203267191	0.001220703125	\\
0.989967594442225	0.001434326171875	\\
0.990011985617259	0.00103759765625	\\
0.990056376792294	0.001129150390625	\\
0.990100767967328	0.001007080078125	\\
0.990145159142362	0.001495361328125	\\
0.990189550317397	0.001434326171875	\\
0.990233941492431	0.001739501953125	\\
0.990278332667466	0.001861572265625	\\
0.9903227238425	0.0018310546875	\\
0.990367115017534	0.002166748046875	\\
0.990411506192569	0.00213623046875	\\
0.990455897367603	0.001983642578125	\\
0.990500288542638	0.002288818359375	\\
0.990544679717672	0.00262451171875	\\
0.990589070892707	0.00238037109375	\\
0.990633462067741	0.00274658203125	\\
0.990677853242775	0.0029296875	\\
0.99072224441781	0.00274658203125	\\
0.990766635592844	0.002716064453125	\\
0.990811026767879	0.0028076171875	\\
0.990855417942913	0.0025634765625	\\
0.990899809117947	0.002532958984375	\\
0.990944200292982	0.002685546875	\\
0.990988591468016	0.00238037109375	\\
0.991032982643051	0.00286865234375	\\
0.991077373818085	0.0035400390625	\\
0.991121764993119	0.003570556640625	\\
0.991166156168154	0.00311279296875	\\
0.991210547343188	0.003326416015625	\\
0.991254938518223	0.003448486328125	\\
0.991299329693257	0.00341796875	\\
0.991343720868291	0.00347900390625	\\
0.991388112043326	0.00372314453125	\\
0.99143250321836	0.003509521484375	\\
0.991476894393395	0.00323486328125	\\
0.991521285568429	0.00311279296875	\\
0.991565676743463	0.00323486328125	\\
0.991610067918498	0.003143310546875	\\
0.991654459093532	0.002960205078125	\\
0.991698850268567	0.003509521484375	\\
0.991743241443601	0.003387451171875	\\
0.991787632618635	0.003387451171875	\\
0.99183202379367	0.00311279296875	\\
0.991876414968704	0.0029296875	\\
0.991920806143739	0.002960205078125	\\
0.991965197318773	0.002471923828125	\\
0.992009588493807	0.002288818359375	\\
0.992053979668842	0.002197265625	\\
0.992098370843876	0.00244140625	\\
0.992142762018911	0.002471923828125	\\
0.992187153193945	0.002410888671875	\\
0.992231544368979	0.0023193359375	\\
0.992275935544014	0.002288818359375	\\
0.992320326719048	0.001983642578125	\\
0.992364717894083	0.001861572265625	\\
0.992409109069117	0.002349853515625	\\
0.992453500244151	0.002166748046875	\\
0.992497891419186	0.002197265625	\\
0.99254228259422	0.002105712890625	\\
0.992586673769255	0.00140380859375	\\
0.992631064944289	0.001953125	\\
0.992675456119324	0.00250244140625	\\
0.992719847294358	0.00189208984375	\\
0.992764238469392	0.002044677734375	\\
0.992808629644427	0.00189208984375	\\
0.992853020819461	0.001434326171875	\\
0.992897411994496	0.001617431640625	\\
0.99294180316953	0.00189208984375	\\
0.992986194344564	0.001953125	\\
0.993030585519599	0.002197265625	\\
0.993074976694633	0.002197265625	\\
0.993119367869667	0.001739501953125	\\
0.993163759044702	0.00189208984375	\\
0.993208150219736	0.001953125	\\
0.993252541394771	0.001708984375	\\
0.993296932569805	0.002105712890625	\\
0.99334132374484	0.001861572265625	\\
0.993385714919874	0.001678466796875	\\
0.993430106094908	0.002166748046875	\\
0.993474497269943	0.002105712890625	\\
0.993518888444977	0.00201416015625	\\
0.993563279620012	0.00244140625	\\
0.993607670795046	0.002716064453125	\\
0.99365206197008	0.002227783203125	\\
0.993696453145115	0.002044677734375	\\
0.993740844320149	0.00225830078125	\\
0.993785235495184	0.00238037109375	\\
0.993829626670218	0.0018310546875	\\
0.993874017845252	0.001922607421875	\\
0.993918409020287	0.001800537109375	\\
0.993962800195321	0.00177001953125	\\
0.994007191370356	0.001739501953125	\\
0.99405158254539	0.00201416015625	\\
0.994095973720424	0.00244140625	\\
0.994140364895459	0.002044677734375	\\
0.994184756070493	0.002410888671875	\\
0.994229147245528	0.00244140625	\\
0.994273538420562	0.0025634765625	\\
0.994317929595596	0.002716064453125	\\
0.994362320770631	0.002288818359375	\\
0.994406711945665	0.0018310546875	\\
0.9944511031207	0.001861572265625	\\
0.994495494295734	0.001739501953125	\\
0.994539885470768	0.0013427734375	\\
0.994584276645803	0.00152587890625	\\
0.994628667820837	0.001678466796875	\\
0.994673058995872	0.001861572265625	\\
0.994717450170906	0.001617431640625	\\
0.99476184134594	0.00164794921875	\\
0.994806232520975	0.001556396484375	\\
0.994850623696009	0.001007080078125	\\
0.994895014871044	0.00115966796875	\\
0.994939406046078	0.00128173828125	\\
0.994983797221112	0.00177001953125	\\
0.995028188396147	0.001739501953125	\\
0.995072579571181	0.001617431640625	\\
0.995116970746216	0.001617431640625	\\
0.99516136192125	0.00146484375	\\
0.995205753096284	0.001861572265625	\\
0.995250144271319	0.00213623046875	\\
0.995294535446353	0.002410888671875	\\
0.995338926621388	0.0029296875	\\
0.995383317796422	0.00299072265625	\\
0.995427708971456	0.003143310546875	\\
0.995472100146491	0.003326416015625	\\
0.995516491321525	0.00311279296875	\\
0.99556088249656	0.003509521484375	\\
0.995605273671594	0.0035400390625	\\
0.995649664846629	0.003875732421875	\\
0.995694056021663	0.0045166015625	\\
0.995738447196697	0.003997802734375	\\
0.995782838371732	0.00390625	\\
0.995827229546766	0.004058837890625	\\
0.9958716207218	0.003448486328125	\\
0.995916011896835	0.003631591796875	\\
0.995960403071869	0.003570556640625	\\
0.996004794246904	0.00311279296875	\\
0.996049185421938	0.00286865234375	\\
0.996093576596973	0.00299072265625	\\
0.996137967772007	0.003692626953125	\\
0.996182358947041	0.003662109375	\\
0.996226750122076	0.003662109375	\\
0.99627114129711	0.00347900390625	\\
0.996315532472145	0.0032958984375	\\
0.996359923647179	0.003326416015625	\\
0.996404314822213	0.003173828125	\\
0.996448705997248	0.00286865234375	\\
0.996493097172282	0.002410888671875	\\
0.996537488347317	0.002288818359375	\\
0.996581879522351	0.002227783203125	\\
0.996626270697385	0.0023193359375	\\
0.99667066187242	0.002471923828125	\\
0.996715053047454	0.0023193359375	\\
0.996759444222489	0.00238037109375	\\
0.996803835397523	0.002532958984375	\\
0.996848226572557	0.002593994140625	\\
0.996892617747592	0.0025634765625	\\
0.996937008922626	0.002593994140625	\\
0.996981400097661	0.00250244140625	\\
0.997025791272695	0.0025634765625	\\
0.997070182447729	0.00244140625	\\
0.997114573622764	0.002410888671875	\\
0.997158964797798	0.00250244140625	\\
0.997203355972833	0.00250244140625	\\
0.997247747147867	0.002593994140625	\\
0.997292138322901	0.00244140625	\\
0.997336529497936	0.0028076171875	\\
0.99738092067297	0.002777099609375	\\
0.997425311848005	0.002197265625	\\
0.997469703023039	0.002685546875	\\
0.997514094198073	0.003204345703125	\\
0.997558485373108	0.0032958984375	\\
0.997602876548142	0.003997802734375	\\
0.997647267723177	0.0040283203125	\\
0.997691658898211	0.00360107421875	\\
0.997736050073245	0.003875732421875	\\
0.99778044124828	0.0040283203125	\\
0.997824832423314	0.00421142578125	\\
0.997869223598349	0.004608154296875	\\
0.997913614773383	0.00457763671875	\\
0.997958005948417	0.004486083984375	\\
0.998002397123452	0.0047607421875	\\
0.998046788298486	0.005126953125	\\
0.998091179473521	0.00494384765625	\\
0.998135570648555	0.00494384765625	\\
0.998179961823589	0.00518798828125	\\
0.998224352998624	0.004638671875	\\
0.998268744173658	0.004852294921875	\\
0.998313135348693	0.00543212890625	\\
0.998357526523727	0.004974365234375	\\
0.998401917698762	0.004852294921875	\\
0.998446308873796	0.004364013671875	\\
0.99849070004883	0.00469970703125	\\
0.998535091223865	0.004913330078125	\\
0.998579482398899	0.004302978515625	\\
0.998623873573933	0.004241943359375	\\
0.998668264748968	0.00421142578125	\\
0.998712655924002	0.00408935546875	\\
0.998757047099037	0.0037841796875	\\
0.998801438274071	0.00347900390625	\\
0.998845829449105	0.003143310546875	\\
0.99889022062414	0.00311279296875	\\
0.998934611799174	0.0029296875	\\
0.998979002974209	0.002838134765625	\\
0.999023394149243	0.003082275390625	\\
0.999067785324278	0.0028076171875	\\
0.999112176499312	0.00323486328125	\\
0.999156567674346	0.003387451171875	\\
0.999200958849381	0.0025634765625	\\
0.999245350024415	0.002532958984375	\\
0.99928974119945	0.002655029296875	\\
0.999334132374484	0.00213623046875	\\
0.999378523549518	0.001678466796875	\\
0.999422914724553	0.001739501953125	\\
0.999467305899587	0.002044677734375	\\
0.999511697074622	0.0023193359375	\\
0.999556088249656	0.00250244140625	\\
0.99960047942469	0.002593994140625	\\
0.999644870599725	0.002838134765625	\\
0.999689261774759	0.00244140625	\\
0.999733652949794	0.002197265625	\\
0.999778044124828	0.0023193359375	\\
0.999822435299862	0.002166748046875	\\
0.999866826474897	0.002288818359375	\\
0.999911217649931	0.001800537109375	\\
0.999955608824966	0.002044677734375	\\
1	0.0025634765625	\\
};
\end{axis}

\begin{axis}[%
width=\figurewidth,
height=\figureheight,
scale only axis,
xmin=-15000,
xmax=15000,
xlabel={Frequency (in hertz)},
ymin=0,
ymax=0.001,
at=(plot1.below south west),
anchor=above north west,
title={Magnitude Response}
]
\addplot [color=blue,solid,forget plot]
  table[row sep=crcr]{
-11025	3.38662754405629e-08	\\
-11024.0212180398	5.30959412617509e-07	\\
-11023.0424360795	5.10337298220626e-07	\\
-11022.0636541193	5.50907265193604e-07	\\
-11021.0848721591	6.6011513147811e-07	\\
-11020.1060901989	4.28458922583225e-07	\\
-11019.1273082386	7.17533568927806e-07	\\
-11018.1485262784	7.57313195523294e-07	\\
-11017.1697443182	6.8621791813706e-07	\\
-11016.190962358	3.81838592842965e-07	\\
-11015.2121803977	9.91893348557525e-07	\\
-11014.2333984375	3.42338821312317e-07	\\
-11013.2546164773	2.6198148557011e-07	\\
-11012.275834517	6.75651850090316e-07	\\
-11011.2970525568	6.56794225468298e-07	\\
-11010.3182705966	9.85929460001768e-07	\\
-11009.3394886364	3.31157533927104e-07	\\
-11008.3607066761	4.01454245613196e-07	\\
-11007.3819247159	4.01912947284972e-07	\\
-11006.4031427557	5.62219683936411e-07	\\
-11005.4243607955	6.3371327302889e-07	\\
-11004.4455788352	4.22328497432368e-07	\\
-11003.466796875	9.29672298585725e-07	\\
-11002.4880149148	2.90230669240145e-07	\\
-11001.5092329545	4.41515246060294e-07	\\
-11000.5304509943	6.98576254335477e-07	\\
-10999.5516690341	4.45109737944723e-07	\\
-10998.5728870739	2.76759769743835e-07	\\
-10997.5941051136	5.20227982800681e-07	\\
-10996.6153231534	2.67402629858814e-07	\\
-10995.6365411932	6.19461014753136e-07	\\
-10994.657759233	3.24079661045065e-07	\\
-10993.6789772727	6.26512081115878e-07	\\
-10992.7001953125	8.69916449273368e-07	\\
-10991.7214133523	9.08856669005153e-07	\\
-10990.742631392	4.91477176994509e-07	\\
-10989.7638494318	5.24477945064403e-07	\\
-10988.7850674716	5.70179462943642e-07	\\
-10987.8062855114	7.84796934365969e-07	\\
-10986.8275035511	9.40166585106247e-07	\\
-10985.8487215909	3.54840054170456e-07	\\
-10984.8699396307	6.30514722080956e-07	\\
-10983.8911576705	8.44866672975862e-07	\\
-10982.9123757102	5.77290528450632e-07	\\
-10981.93359375	2.36304189229501e-07	\\
-10980.9548117898	3.36985232072936e-07	\\
-10979.9760298295	4.86824334808672e-07	\\
-10978.9972478693	6.24712440753149e-07	\\
-10978.0184659091	5.5411003513513e-07	\\
-10977.0396839489	6.85975822721285e-07	\\
-10976.0609019886	4.8664568252691e-07	\\
-10975.0821200284	8.79022962721898e-07	\\
-10974.1033380682	5.8087407174701e-07	\\
-10973.124556108	5.23754700393762e-07	\\
-10972.1457741477	4.41948214412574e-07	\\
-10971.1669921875	6.13264380976273e-07	\\
-10970.1882102273	7.65966906081159e-07	\\
-10969.209428267	5.42340080135558e-07	\\
-10968.2306463068	5.11876627730306e-07	\\
-10967.2518643466	7.73821973113947e-07	\\
-10966.2730823864	2.11916579525814e-07	\\
-10965.2943004261	5.14473515938612e-07	\\
-10964.3155184659	8.18638623542357e-07	\\
-10963.3367365057	5.1657578057323e-07	\\
-10962.3579545455	7.45255734575346e-07	\\
-10961.3791725852	8.34753436857455e-07	\\
-10960.400390625	8.24635045792196e-07	\\
-10959.4216086648	9.02542396905141e-07	\\
-10958.4428267045	3.66552065076097e-07	\\
-10957.4640447443	4.50652362521072e-07	\\
-10956.4852627841	8.37886263904176e-07	\\
-10955.5064808239	6.76671128449939e-07	\\
-10954.5276988636	5.90347423199851e-07	\\
-10953.5489169034	7.95695279576832e-07	\\
-10952.5701349432	6.84363260892501e-07	\\
-10951.591352983	5.12917927327227e-07	\\
-10950.6125710227	7.82266955984564e-07	\\
-10949.6337890625	4.98372140683218e-07	\\
-10948.6550071023	5.79428938278661e-07	\\
-10947.676225142	9.65620917433811e-07	\\
-10946.6974431818	5.18714713149959e-07	\\
-10945.7186612216	6.20117002874425e-07	\\
-10944.7398792614	1.34630451782182e-06	\\
-10943.7610973011	1.8757160037276e-07	\\
-10942.7823153409	7.10044209467946e-07	\\
-10941.8035333807	9.90361901995299e-07	\\
-10940.8247514205	6.77967993423971e-07	\\
-10939.8459694602	1.16483709908782e-06	\\
-10938.8671875	5.01900690314978e-07	\\
-10937.8884055398	7.81603784225626e-07	\\
-10936.9096235795	8.77806987783109e-07	\\
-10935.9308416193	7.62240904634588e-07	\\
-10934.9520596591	9.95970647008428e-07	\\
-10933.9732776989	5.33447742975312e-07	\\
-10932.9944957386	5.63410282131737e-07	\\
-10932.0157137784	7.83739523268714e-07	\\
-10931.0369318182	6.14189074188719e-07	\\
-10930.058149858	6.12671770913124e-07	\\
-10929.0793678977	8.3191988802879e-07	\\
-10928.1005859375	3.08010546676463e-07	\\
-10927.1218039773	8.40033686720347e-07	\\
-10926.143022017	5.84809908942327e-07	\\
-10925.1642400568	7.136025924705e-07	\\
-10924.1854580966	4.82335893301549e-07	\\
-10923.2066761364	6.50189240837747e-07	\\
-10922.2278941761	5.5617019941668e-07	\\
-10921.2491122159	5.74453449090531e-07	\\
-10920.2703302557	9.71178968898342e-07	\\
-10919.2915482955	7.23267567155897e-07	\\
-10918.3127663352	5.6502654055575e-07	\\
-10917.333984375	4.42069446563874e-07	\\
-10916.3552024148	6.88386827450016e-07	\\
-10915.3764204545	6.73292357463926e-07	\\
-10914.3976384943	8.34005539781444e-07	\\
-10913.4188565341	9.54166199072391e-07	\\
-10912.4400745739	5.93732398534381e-07	\\
-10911.4612926136	9.7079133691358e-07	\\
-10910.4825106534	7.85813522141179e-07	\\
-10909.5037286932	6.11313961949503e-07	\\
-10908.524946733	1.04860099206969e-06	\\
-10907.5461647727	6.11166942642351e-07	\\
-10906.5673828125	4.12077675006109e-07	\\
-10905.5886008523	4.94411068541262e-07	\\
-10904.609818892	5.00107642583979e-07	\\
-10903.6310369318	6.13151045357296e-07	\\
-10902.6522549716	7.72738062376132e-07	\\
-10901.6734730114	5.40993248066768e-07	\\
-10900.6946910511	6.98024467903576e-07	\\
-10899.7159090909	5.56504895820497e-07	\\
-10898.7371271307	8.79266071726998e-07	\\
-10897.7583451705	5.75094674539913e-07	\\
-10896.7795632102	9.74020550319666e-07	\\
-10895.80078125	6.04795289327291e-07	\\
-10894.8219992898	9.28224776697126e-07	\\
-10893.8432173295	8.84796051061147e-07	\\
-10892.8644353693	6.67625211319515e-07	\\
-10891.8856534091	6.45249431753882e-07	\\
-10890.9068714489	3.15675630723434e-07	\\
-10889.9280894886	4.83710156002149e-07	\\
-10888.9493075284	1.1449962067618e-06	\\
-10887.9705255682	3.34675979243743e-07	\\
-10886.991743608	5.24569189994584e-07	\\
-10886.0129616477	1.03227184208389e-06	\\
-10885.0341796875	7.80029607991756e-07	\\
-10884.0553977273	5.96978916425206e-07	\\
-10883.076615767	7.75632472687738e-07	\\
-10882.0978338068	6.65469414498454e-07	\\
-10881.1190518466	1.01597102810192e-06	\\
-10880.1402698864	7.10768654680752e-07	\\
-10879.1614879261	7.54301498784416e-07	\\
-10878.1827059659	1.04789348443228e-06	\\
-10877.2039240057	8.6219441713004e-07	\\
-10876.2251420455	6.97737087051715e-07	\\
-10875.2463600852	5.83447758567659e-07	\\
-10874.267578125	1.0746633144105e-06	\\
-10873.2887961648	9.02434324185518e-07	\\
-10872.3100142045	1.20434775744645e-06	\\
-10871.3312322443	6.83779551939578e-07	\\
-10870.3524502841	9.25737245707202e-07	\\
-10869.3736683239	3.07625846824499e-07	\\
-10868.3948863636	4.57794746396119e-07	\\
-10867.4161044034	7.03148208619759e-07	\\
-10866.4373224432	7.52616126902326e-07	\\
-10865.458540483	6.6917033496074e-07	\\
-10864.4797585227	8.61352952248241e-07	\\
-10863.5009765625	5.42034368795125e-07	\\
-10862.5221946023	9.78748728674991e-07	\\
-10861.543412642	6.82744183702614e-07	\\
-10860.5646306818	7.8816276234212e-07	\\
-10859.5858487216	4.14982119352487e-07	\\
-10858.6070667614	1.04136081214544e-06	\\
-10857.6282848011	7.28965913243878e-07	\\
-10856.6495028409	9.08087968543735e-07	\\
-10855.6707208807	9.57787194635344e-07	\\
-10854.6919389205	6.81610790757814e-07	\\
-10853.7131569602	7.64831085392417e-07	\\
-10852.734375	8.07866674918663e-07	\\
-10851.7555930398	1.18862774727776e-06	\\
-10850.7768110795	7.77520668048528e-07	\\
-10849.7980291193	7.6362210288186e-07	\\
-10848.8192471591	7.29668679196515e-07	\\
-10847.8404651989	8.11345561115581e-07	\\
-10846.8616832386	5.84661676594614e-07	\\
-10845.8829012784	5.38277121138537e-07	\\
-10844.9041193182	8.37366930109959e-07	\\
-10843.925337358	1.29378890742223e-06	\\
-10842.9465553977	5.33542506866333e-07	\\
-10841.9677734375	8.39924187342018e-07	\\
-10840.9889914773	9.66534720249135e-07	\\
-10840.010209517	1.00311529968671e-06	\\
-10839.0314275568	1.12462693047474e-06	\\
-10838.0526455966	5.1962433538367e-07	\\
-10837.0738636364	8.74805255483205e-07	\\
-10836.0950816761	8.69408217702968e-07	\\
-10835.1162997159	6.43856315769804e-07	\\
-10834.1375177557	1.30037812644772e-06	\\
-10833.1587357955	6.56654765943114e-07	\\
-10832.1799538352	7.50256905904708e-07	\\
-10831.201171875	1.0102836998063e-06	\\
-10830.2223899148	1.14863269873427e-06	\\
-10829.2436079545	5.45457516811204e-07	\\
-10828.2648259943	8.72972670234353e-07	\\
-10827.2860440341	7.77792514166894e-07	\\
-10826.3072620739	5.25117844214487e-07	\\
-10825.3284801136	9.14852668316345e-07	\\
-10824.3496981534	9.14159374308403e-07	\\
-10823.3709161932	7.78130579389848e-07	\\
-10822.392134233	8.18846617424669e-07	\\
-10821.4133522727	8.83573166573188e-07	\\
-10820.4345703125	7.14762124818643e-07	\\
-10819.4557883523	7.23185486714909e-07	\\
-10818.477006392	9.6143918340445e-07	\\
-10817.4982244318	8.13400820080715e-07	\\
-10816.5194424716	7.36388750865701e-07	\\
-10815.5406605114	1.01097722707375e-06	\\
-10814.5618785511	1.04632859303039e-06	\\
-10813.5830965909	1.33700141751051e-06	\\
-10812.6043146307	7.92109779252718e-07	\\
-10811.6255326705	6.54781615256993e-07	\\
-10810.6467507102	8.84146323731053e-07	\\
-10809.66796875	5.54750063032532e-07	\\
-10808.6891867898	1.12425402729904e-06	\\
-10807.7104048295	9.30485714164565e-07	\\
-10806.7316228693	6.00523728449677e-07	\\
-10805.7528409091	8.825561153143e-07	\\
-10804.7740589489	1.31913514668045e-06	\\
-10803.7952769886	6.03833451493279e-07	\\
-10802.8164950284	1.11354400266057e-06	\\
-10801.8377130682	1.14714839257254e-06	\\
-10800.858931108	5.14565524479341e-07	\\
-10799.8801491477	1.08692498453014e-06	\\
-10798.9013671875	1.26933598138648e-06	\\
-10797.9225852273	8.35296934625328e-07	\\
-10796.943803267	8.02523872989245e-07	\\
-10795.9650213068	8.06684882292214e-07	\\
-10794.9862393466	7.07772163894269e-07	\\
-10794.0074573864	1.01134114217704e-06	\\
-10793.0286754261	7.57476181195963e-07	\\
-10792.0498934659	7.07655615680922e-07	\\
-10791.0711115057	8.96473604575744e-07	\\
-10790.0923295455	5.43051366128641e-07	\\
-10789.1135475852	5.90181402721571e-07	\\
-10788.134765625	1.07456009118104e-06	\\
-10787.1559836648	8.54530350230112e-07	\\
-10786.1772017045	5.77356820323827e-07	\\
-10785.1984197443	1.03061483464529e-06	\\
-10784.2196377841	8.09724738745168e-07	\\
-10783.2408558239	8.36506521018159e-07	\\
-10782.2620738636	9.68898313441805e-07	\\
-10781.2832919034	5.44522979713383e-07	\\
-10780.3045099432	1.22582955132064e-06	\\
-10779.325727983	9.23657262390026e-07	\\
-10778.3469460227	8.98952511418278e-07	\\
-10777.3681640625	8.10103337934255e-07	\\
-10776.3893821023	7.74468086398063e-07	\\
-10775.410600142	8.31354731682029e-07	\\
-10774.4318181818	1.19158137404652e-06	\\
-10773.4530362216	9.16509594128108e-07	\\
-10772.4742542614	6.37785862078685e-07	\\
-10771.4954723011	1.12148324744413e-06	\\
-10770.5166903409	7.2142312408998e-07	\\
-10769.5379083807	4.709715291446e-07	\\
-10768.5591264205	9.05750985451857e-07	\\
-10767.5803444602	6.80883342381876e-07	\\
-10766.6015625	1.2167560092127e-06	\\
-10765.6227805398	1.16049206420452e-06	\\
-10764.6439985795	1.19010378951285e-06	\\
-10763.6652166193	9.9947624968555e-07	\\
-10762.6864346591	9.63737822168003e-07	\\
-10761.7076526989	7.17190980794043e-07	\\
-10760.7288707386	1.42366049393144e-06	\\
-10759.7500887784	8.76914366698789e-07	\\
-10758.7713068182	8.65894976415727e-07	\\
-10757.792524858	1.15263519191395e-06	\\
-10756.8137428977	1.01325082918025e-06	\\
-10755.8349609375	4.65411424210861e-07	\\
-10754.8561789773	1.20977176459295e-06	\\
-10753.877397017	1.30617087552714e-06	\\
-10752.8986150568	7.57880413938248e-07	\\
-10751.9198330966	1.01017865102175e-06	\\
-10750.9410511364	9.04196254831282e-07	\\
-10749.9622691761	1.37236572005246e-06	\\
-10748.9834872159	1.10816268212752e-06	\\
-10748.0047052557	7.3997410551051e-07	\\
-10747.0259232955	8.84394731261531e-07	\\
-10746.0471413352	1.06170340098428e-06	\\
-10745.068359375	6.76336927508534e-07	\\
-10744.0895774148	9.95719570311e-07	\\
-10743.1107954545	8.61462845099727e-07	\\
-10742.1320134943	1.38306434156696e-06	\\
-10741.1532315341	9.45292099083923e-07	\\
-10740.1744495739	1.6930262037781e-06	\\
-10739.1956676136	9.14207650926907e-07	\\
-10738.2168856534	9.03953511656978e-07	\\
-10737.2381036932	1.04123438022035e-06	\\
-10736.259321733	1.23624261945343e-06	\\
-10735.2805397727	9.33845839505711e-07	\\
-10734.3017578125	9.26998063563484e-07	\\
-10733.3229758523	1.04618094269504e-06	\\
-10732.344193892	1.09120153545394e-06	\\
-10731.3654119318	1.55325675150019e-06	\\
-10730.3866299716	6.99607345391125e-07	\\
-10729.4078480114	1.17017060701732e-06	\\
-10728.4290660511	7.50340792460194e-07	\\
-10727.4502840909	1.34387807354341e-06	\\
-10726.4715021307	8.48109122152825e-07	\\
-10725.4927201705	1.1480640749309e-06	\\
-10724.5139382102	9.06345483654522e-07	\\
-10723.53515625	1.31238244300328e-06	\\
-10722.5563742898	9.67476640915045e-07	\\
-10721.5775923295	1.35773878731774e-06	\\
-10720.5988103693	1.35707056161204e-06	\\
-10719.6200284091	6.95075847596755e-07	\\
-10718.6412464489	9.79897653635245e-07	\\
-10717.6624644886	1.23776265400079e-06	\\
-10716.6836825284	1.19898167001622e-06	\\
-10715.7049005682	1.42661635425057e-06	\\
-10714.726118608	9.22554544431068e-07	\\
-10713.7473366477	8.91558662790105e-07	\\
-10712.7685546875	1.71050563173421e-06	\\
-10711.7897727273	9.87711701996288e-07	\\
-10710.810990767	1.09678779038173e-06	\\
-10709.8322088068	1.55858915784376e-06	\\
-10708.8534268466	7.84966039156301e-07	\\
-10707.8746448864	1.10628316403492e-06	\\
-10706.8958629261	1.49532724654364e-06	\\
-10705.9170809659	1.00801082897222e-06	\\
-10704.9382990057	1.5355813332062e-06	\\
-10703.9595170455	1.21409413171094e-06	\\
-10702.9807350852	8.05934928605465e-07	\\
-10702.001953125	1.00204901564185e-06	\\
-10701.0231711648	1.06435355867879e-06	\\
-10700.0443892045	1.28981769560089e-06	\\
-10699.0656072443	1.47982096722143e-06	\\
-10698.0868252841	1.31872772539123e-06	\\
-10697.1080433239	1.06715304944772e-06	\\
-10696.1292613636	1.27667724575606e-06	\\
-10695.1504794034	1.11163942058653e-06	\\
-10694.1716974432	1.36007836085971e-06	\\
-10693.192915483	1.18171794931844e-06	\\
-10692.2141335227	1.09989014540179e-06	\\
-10691.2353515625	1.05654462082832e-06	\\
-10690.2565696023	1.28451444217955e-06	\\
-10689.277787642	9.8749644774865e-07	\\
-10688.2990056818	9.61596515841535e-07	\\
-10687.3202237216	1.34255282217387e-06	\\
-10686.3414417614	1.14978216564901e-06	\\
-10685.3626598011	1.11523537644563e-06	\\
-10684.3838778409	1.23327378579098e-06	\\
-10683.4050958807	1.43690470680728e-06	\\
-10682.4263139205	9.14727396223715e-07	\\
-10681.4475319602	1.22174047458248e-06	\\
-10680.46875	7.09693041570424e-07	\\
-10679.4899680398	1.40641455173812e-06	\\
-10678.5111860795	1.32940818444415e-06	\\
-10677.5324041193	7.79445952617765e-07	\\
-10676.5536221591	1.13604914166743e-06	\\
-10675.5748401989	1.15199717534385e-06	\\
-10674.5960582386	8.9964962012894e-07	\\
-10673.6172762784	9.07972701734127e-07	\\
-10672.6384943182	7.82433983010385e-07	\\
-10671.659712358	1.12385021868787e-06	\\
-10670.6809303977	1.43406267202839e-06	\\
-10669.7021484375	1.00935951723266e-06	\\
-10668.7233664773	1.26114731161669e-06	\\
-10667.744584517	1.28290109498472e-06	\\
-10666.7658025568	1.17136631768708e-06	\\
-10665.7870205966	5.20103806841207e-07	\\
-10664.8082386364	8.87205843093113e-07	\\
-10663.8294566761	7.3457988474699e-07	\\
-10662.8506747159	1.0456602518983e-06	\\
-10661.8718927557	1.36286907441947e-06	\\
-10660.8931107955	1.10459848261392e-06	\\
-10659.9143288352	1.34086982362349e-06	\\
-10658.935546875	1.28313135465562e-06	\\
-10657.9567649148	7.41052020890727e-07	\\
-10656.9779829545	1.18280287375864e-06	\\
-10655.9992009943	1.01518673720785e-06	\\
-10655.0204190341	1.19555162516852e-06	\\
-10654.0416370739	1.75689678485848e-06	\\
-10653.0628551136	1.17226001421102e-06	\\
-10652.0840731534	8.45431746704895e-07	\\
-10651.1052911932	1.24893784020052e-06	\\
-10650.126509233	1.23624085348074e-06	\\
-10649.1477272727	1.30015865384325e-06	\\
-10648.1689453125	1.5920580318005e-06	\\
-10647.1901633523	1.03955651612615e-06	\\
-10646.211381392	8.77123793627223e-07	\\
-10645.2325994318	1.35248146145722e-06	\\
-10644.2538174716	8.59318732104206e-07	\\
-10643.2750355114	1.06981373563705e-06	\\
-10642.2962535511	1.67213195542445e-06	\\
-10641.3174715909	9.86484173507697e-07	\\
-10640.3386896307	1.04622196836616e-06	\\
-10639.3599076705	1.14107522092087e-06	\\
-10638.3811257102	8.36237558913689e-07	\\
-10637.40234375	1.23538656112414e-06	\\
-10636.4235617898	1.32489087895761e-06	\\
-10635.4447798295	8.14251273460531e-07	\\
-10634.4659978693	1.54085612074183e-06	\\
-10633.4872159091	1.25263022450026e-06	\\
-10632.5084339489	6.99553924543877e-07	\\
-10631.5296519886	1.07991146387811e-06	\\
-10630.5508700284	1.53538726701595e-06	\\
-10629.5720880682	1.67469576781069e-06	\\
-10628.593306108	1.45713644199684e-06	\\
-10627.6145241477	1.04221264918413e-06	\\
-10626.6357421875	1.68134617638534e-06	\\
-10625.6569602273	1.20816306826399e-06	\\
-10624.678178267	9.43210451877194e-07	\\
-10623.6993963068	1.33340711213548e-06	\\
-10622.7206143466	1.36163414050286e-06	\\
-10621.7418323864	1.25034821278496e-06	\\
-10620.7630504261	1.28588254536182e-06	\\
-10619.7842684659	9.07786753637668e-07	\\
-10618.8054865057	9.35223869244079e-07	\\
-10617.8267045455	1.50593615115216e-06	\\
-10616.8479225852	1.00211251461252e-06	\\
-10615.869140625	1.01458926753571e-06	\\
-10614.8903586648	1.16909419859495e-06	\\
-10613.9115767045	1.03359244011182e-06	\\
-10612.9327947443	6.48918097487538e-07	\\
-10611.9540127841	1.19019361958979e-06	\\
-10610.9752308239	1.43484799159533e-06	\\
-10609.9964488636	1.2321937057779e-06	\\
-10609.0176669034	1.30375284687674e-06	\\
-10608.0388849432	1.4207735010112e-06	\\
-10607.060102983	9.38943116781527e-07	\\
-10606.0813210227	1.18525222584072e-06	\\
-10605.1025390625	1.19728765756499e-06	\\
-10604.1237571023	1.41015636597106e-06	\\
-10603.144975142	1.45258351453447e-06	\\
-10602.1661931818	9.71471333358039e-07	\\
-10601.1874112216	9.42015647653366e-07	\\
-10600.2086292614	9.0960797894706e-07	\\
-10599.2298473011	7.22872342255598e-07	\\
-10598.2510653409	1.38840037427003e-06	\\
-10597.2722833807	1.11889012980423e-06	\\
-10596.2935014205	1.15971917930522e-06	\\
-10595.3147194602	5.06055280343126e-07	\\
-10594.3359375	1.05433240953476e-06	\\
-10593.3571555398	1.03864946303928e-06	\\
-10592.3783735795	1.33814281760763e-06	\\
-10591.3995916193	1.08601548181474e-06	\\
-10590.4208096591	8.66751438917859e-07	\\
-10589.4420276989	1.5198388422777e-06	\\
-10588.4632457386	1.08784952150934e-06	\\
-10587.4844637784	1.02468905725001e-06	\\
-10586.5056818182	1.4409532423403e-06	\\
-10585.526899858	1.00864842735429e-06	\\
-10584.5481178977	1.21460284084855e-06	\\
-10583.5693359375	9.6584340281318e-07	\\
-10582.5905539773	1.46897048513464e-06	\\
-10581.611772017	8.79022689900505e-07	\\
-10580.6329900568	1.14242350253068e-06	\\
-10579.6542080966	8.33021319766494e-07	\\
-10578.6754261364	1.01931578274814e-06	\\
-10577.6966441761	1.11237015129579e-06	\\
-10576.7178622159	1.14224905518916e-06	\\
-10575.7390802557	1.24668460541156e-06	\\
-10574.7602982955	1.05505937170591e-06	\\
-10573.7815163352	1.03079870322302e-06	\\
-10572.802734375	1.2614524481055e-06	\\
-10571.8239524148	9.48941978961943e-07	\\
-10570.8451704545	1.06851844650108e-06	\\
-10569.8663884943	1.27097869368886e-06	\\
-10568.8876065341	1.11393481280282e-06	\\
-10567.9088245739	6.36357619213095e-07	\\
-10566.9300426136	1.47793030308563e-06	\\
-10565.9512606534	8.83189165452641e-07	\\
-10564.9724786932	9.9493496023969e-07	\\
-10563.993696733	1.24636640471882e-06	\\
-10563.0149147727	8.32403974675008e-07	\\
-10562.0361328125	1.01882648557313e-06	\\
-10561.0573508523	1.11359809668269e-06	\\
-10560.078568892	3.19491901482362e-07	\\
-10559.0997869318	1.08778466867816e-06	\\
-10558.1210049716	1.2196593150056e-06	\\
-10557.1422230114	6.63690803561799e-07	\\
-10556.1634410511	8.02858016913831e-07	\\
-10555.1846590909	1.20399993292218e-06	\\
-10554.2058771307	1.51244637337325e-06	\\
-10553.2270951705	1.30042167704746e-06	\\
-10552.2483132102	8.59542286877475e-07	\\
-10551.26953125	9.90467894703602e-07	\\
-10550.2907492898	1.4190347646133e-06	\\
-10549.3119673295	1.3181131682949e-06	\\
-10548.3331853693	6.85364491248679e-07	\\
-10547.3544034091	1.325423066756e-06	\\
-10546.3756214489	1.20757201758105e-06	\\
-10545.3968394886	1.63657043404615e-06	\\
-10544.4180575284	1.00144025474843e-06	\\
-10543.4392755682	1.29172119591371e-06	\\
-10542.460493608	1.30750145996755e-06	\\
-10541.4817116477	1.2978343001519e-06	\\
-10540.5029296875	1.03803668348951e-06	\\
-10539.5241477273	1.18048997628343e-06	\\
-10538.545365767	1.59447544796527e-06	\\
-10537.5665838068	1.12326784982441e-06	\\
-10536.5878018466	1.03900207662713e-06	\\
-10535.6090198864	1.40761106310146e-06	\\
-10534.6302379261	1.38845348117079e-06	\\
-10533.6514559659	1.13942264211354e-06	\\
-10532.6726740057	1.23645360918493e-06	\\
-10531.6938920455	9.54030039648895e-07	\\
-10530.7151100852	9.85517702671571e-07	\\
-10529.736328125	1.5174703063069e-06	\\
-10528.7575461648	1.17115333904821e-06	\\
-10527.7787642045	1.12265505962915e-06	\\
-10526.7999822443	1.08772312752647e-06	\\
-10525.8212002841	1.18070580432723e-06	\\
-10524.8424183239	1.36213584860658e-06	\\
-10523.8636363636	1.39807002894066e-06	\\
-10522.8848544034	1.08271460800809e-06	\\
-10521.9060724432	1.79284104179365e-06	\\
-10520.927290483	1.48340546667891e-06	\\
-10519.9485085227	1.35478700863775e-06	\\
-10518.9697265625	1.51019186079285e-06	\\
-10517.9909446023	1.12308909391826e-06	\\
-10517.012162642	1.19490705443689e-06	\\
-10516.0333806818	1.27533792766481e-06	\\
-10515.0545987216	1.17371795018452e-06	\\
-10514.0758167614	1.64327031580275e-06	\\
-10513.0970348011	1.24021408227995e-06	\\
-10512.1182528409	1.57208844883174e-06	\\
-10511.1394708807	1.24045707667511e-06	\\
-10510.1606889205	1.41873846339184e-06	\\
-10509.1819069602	9.13698381152309e-07	\\
-10508.203125	1.61764952375548e-06	\\
-10507.2243430398	1.66376771936251e-06	\\
-10506.2455610795	1.25792763526968e-06	\\
-10505.2667791193	2.13140801492921e-06	\\
-10504.2879971591	1.19908843903561e-06	\\
-10503.3092151989	8.38583001546729e-07	\\
-10502.3304332386	1.31737214506058e-06	\\
-10501.3516512784	1.05327042260265e-06	\\
-10500.3728693182	1.52011115617783e-06	\\
-10499.394087358	1.22920177995743e-06	\\
-10498.4153053977	1.45041049429501e-06	\\
-10497.4365234375	1.39628807018193e-06	\\
-10496.4577414773	1.61141144179997e-06	\\
-10495.478959517	1.45204561747332e-06	\\
-10494.5001775568	1.61754857288057e-06	\\
-10493.5213955966	1.43735929816636e-06	\\
-10492.5426136364	1.40861284887983e-06	\\
-10491.5638316761	1.13429677804318e-06	\\
-10490.5850497159	9.20817304765547e-07	\\
-10489.6062677557	9.19746112979523e-07	\\
-10488.6274857955	1.18218826232927e-06	\\
-10487.6487038352	1.54114189439723e-06	\\
-10486.669921875	1.23560625549172e-06	\\
-10485.6911399148	1.7390620258724e-06	\\
-10484.7123579545	1.7052427555296e-06	\\
-10483.7335759943	9.60452861740211e-07	\\
-10482.7547940341	1.19023733015013e-06	\\
-10481.7760120739	1.38872678302744e-06	\\
-10480.7972301136	1.53547887826848e-06	\\
-10479.8184481534	1.23565216676767e-06	\\
-10478.8396661932	1.06261607533897e-06	\\
-10477.860884233	1.82153627247388e-06	\\
-10476.8821022727	1.84177176023498e-06	\\
-10475.9033203125	1.60360021914298e-06	\\
-10474.9245383523	1.09436576676186e-06	\\
-10473.945756392	9.83599016104181e-07	\\
-10472.9669744318	1.17094627019572e-06	\\
-10471.9881924716	1.43013686292969e-06	\\
-10471.0094105114	9.01421005339458e-07	\\
-10470.0306285511	7.08481560633555e-07	\\
-10469.0518465909	1.49117087196611e-06	\\
-10468.0730646307	1.20694919924027e-06	\\
-10467.0942826705	1.5354358909569e-06	\\
-10466.1155007102	1.29635873384947e-06	\\
-10465.13671875	1.10452562933725e-06	\\
-10464.1579367898	1.06367604452015e-06	\\
-10463.1791548295	1.46553760129083e-06	\\
-10462.2003728693	1.29264651977366e-06	\\
-10461.2215909091	1.15462338571469e-06	\\
-10460.2428089489	1.08095944031303e-06	\\
-10459.2640269886	1.37533748270484e-06	\\
-10458.2852450284	1.21935265374016e-06	\\
-10457.3064630682	1.41238501821363e-06	\\
-10456.327681108	7.53630568451334e-07	\\
-10455.3488991477	1.05243209860579e-06	\\
-10454.3701171875	9.54865512172458e-07	\\
-10453.3913352273	9.77161990244998e-07	\\
-10452.412553267	1.37515315369505e-06	\\
-10451.4337713068	9.59531977685145e-07	\\
-10450.4549893466	1.0406574721418e-06	\\
-10449.4762073864	1.56614045394108e-06	\\
-10448.4974254261	1.32845433584008e-06	\\
-10447.5186434659	1.0843537971401e-06	\\
-10446.5398615057	1.45300856780802e-06	\\
-10445.5610795455	9.27841131111081e-07	\\
-10444.5822975852	1.19853157652929e-06	\\
-10443.603515625	1.8436276431368e-06	\\
-10442.6247336648	8.37304113216928e-07	\\
-10441.6459517045	9.16606104518415e-07	\\
-10440.6671697443	1.20664400369035e-06	\\
-10439.6883877841	1.27969564819966e-06	\\
-10438.7096058239	9.35755282892836e-07	\\
-10437.7308238636	1.56308787301999e-06	\\
-10436.7520419034	9.41307297513778e-07	\\
-10435.7732599432	1.34784546314927e-06	\\
-10434.794477983	9.33883325268369e-07	\\
-10433.8156960227	1.07722698394549e-06	\\
-10432.8369140625	1.13933352360231e-06	\\
-10431.8581321023	1.67231141467689e-06	\\
-10430.879350142	9.74240153746448e-07	\\
-10429.9005681818	8.61729833898584e-07	\\
-10428.9217862216	1.61813098631531e-06	\\
-10427.9430042614	1.66384967953221e-06	\\
-10426.9642223011	1.7303787320276e-06	\\
-10425.9854403409	1.19748543234866e-06	\\
-10425.0066583807	1.55454317935813e-06	\\
-10424.0278764205	1.42875867368428e-06	\\
-10423.0490944602	1.20357996959744e-06	\\
-10422.0703125	1.65282849958927e-06	\\
-10421.0915305398	1.55245187053658e-06	\\
-10420.1127485795	1.35525579784875e-06	\\
-10419.1339666193	1.30581485794451e-06	\\
-10418.1551846591	1.83228968553934e-06	\\
-10417.1764026989	1.41456189295732e-06	\\
-10416.1976207386	1.37951563431885e-06	\\
-10415.2188387784	1.96342887083718e-06	\\
-10414.2400568182	1.19145406561808e-06	\\
-10413.261274858	1.33197080257195e-06	\\
-10412.2824928977	1.7430474364242e-06	\\
-10411.3037109375	1.23963171304834e-06	\\
-10410.3249289773	1.37702829765246e-06	\\
-10409.346147017	1.58306665744298e-06	\\
-10408.3673650568	7.61395272325053e-07	\\
-10407.3885830966	2.13200555078696e-06	\\
-10406.4098011364	1.5946104694119e-06	\\
-10405.4310191761	1.64071369952916e-06	\\
-10404.4522372159	1.31635621877249e-06	\\
-10403.4734552557	1.45372379722159e-06	\\
-10402.4946732955	1.46897077331624e-06	\\
-10401.5158913352	1.47760503400137e-06	\\
-10400.537109375	2.20273848335318e-06	\\
-10399.5583274148	1.83044769593654e-06	\\
-10398.5795454545	2.03563192195858e-06	\\
-10397.6007634943	1.37163290848173e-06	\\
-10396.6219815341	1.66401634033086e-06	\\
-10395.6431995739	1.54884634785666e-06	\\
-10394.6644176136	1.76237929415876e-06	\\
-10393.6856356534	1.54822173451921e-06	\\
-10392.7068536932	1.53064003819727e-06	\\
-10391.728071733	1.14246068357966e-06	\\
-10390.7492897727	1.5037537729569e-06	\\
-10389.7705078125	1.6588296985341e-06	\\
-10388.7917258523	1.45629577999644e-06	\\
-10387.812943892	1.80187027398858e-06	\\
-10386.8341619318	1.56500550124786e-06	\\
-10385.8553799716	1.25988700520168e-06	\\
-10384.8765980114	1.51662016987692e-06	\\
-10383.8978160511	1.08666587256066e-06	\\
-10382.9190340909	1.72480917049025e-06	\\
-10381.9402521307	1.9948136669159e-06	\\
-10380.9614701705	1.79169937953518e-06	\\
-10379.9826882102	1.73547778366712e-06	\\
-10379.00390625	1.10832192796323e-06	\\
-10378.0251242898	1.18414972797409e-06	\\
-10377.0463423295	1.43001723899398e-06	\\
-10376.0675603693	1.61113381621054e-06	\\
-10375.0887784091	1.77572888543724e-06	\\
-10374.1099964489	1.31056129371033e-06	\\
-10373.1312144886	1.99551012119044e-06	\\
-10372.1524325284	1.49511297451431e-06	\\
-10371.1736505682	1.29905591644365e-06	\\
-10370.194868608	1.72103930452819e-06	\\
-10369.2160866477	1.24716749232911e-06	\\
-10368.2373046875	1.62416635119106e-06	\\
-10367.2585227273	1.31427144298718e-06	\\
-10366.279740767	1.28144695914991e-06	\\
-10365.3009588068	1.79367164681148e-06	\\
-10364.3221768466	1.34099411010357e-06	\\
-10363.3433948864	1.00018501633363e-06	\\
-10362.3646129261	1.45393261070938e-06	\\
-10361.3858309659	1.25393511816135e-06	\\
-10360.4070490057	1.31641795946221e-06	\\
-10359.4282670455	9.17315704634804e-07	\\
-10358.4494850852	1.86117271617838e-06	\\
-10357.470703125	1.38795134996618e-06	\\
-10356.4919211648	1.64110427844474e-06	\\
-10355.5131392045	1.28745802296968e-06	\\
-10354.5343572443	1.14234240652059e-06	\\
-10353.5555752841	1.2954936499962e-06	\\
-10352.5767933239	1.21373783525054e-06	\\
-10351.5980113636	1.17602274468545e-06	\\
-10350.6192294034	1.68763793967536e-06	\\
-10349.6404474432	1.66751678918013e-06	\\
-10348.661665483	1.46637056278732e-06	\\
-10347.6828835227	2.09589961244168e-06	\\
-10346.7041015625	1.46589522686462e-06	\\
-10345.7253196023	1.53457136863804e-06	\\
-10344.746537642	2.04602785677107e-06	\\
-10343.7677556818	1.65481430151932e-06	\\
-10342.7889737216	1.63268609596052e-06	\\
-10341.8101917614	1.52281230610365e-06	\\
-10340.8314098011	1.54553960837218e-06	\\
-10339.8526278409	1.80959813004777e-06	\\
-10338.8738458807	1.45670003800833e-06	\\
-10337.8950639205	1.30556018860012e-06	\\
-10336.9162819602	1.70947921785809e-06	\\
-10335.9375	1.62240964531603e-06	\\
-10334.9587180398	1.94715142637092e-06	\\
-10333.9799360795	2.28336860601104e-06	\\
-10333.0011541193	1.37038934520702e-06	\\
-10332.0223721591	1.26675635213519e-06	\\
-10331.0435901989	1.76989225753911e-06	\\
-10330.0648082386	1.39169554008796e-06	\\
-10329.0860262784	1.64203294978426e-06	\\
-10328.1072443182	1.93511727565013e-06	\\
-10327.128462358	1.57746611093448e-06	\\
-10326.1496803977	1.32176680191323e-06	\\
-10325.1708984375	1.48058066084758e-06	\\
-10324.1921164773	1.8663253310955e-06	\\
-10323.213334517	1.74656379452618e-06	\\
-10322.2345525568	1.40746932830217e-06	\\
-10321.2557705966	1.44438971672969e-06	\\
-10320.2769886364	1.82116850412722e-06	\\
-10319.2982066761	1.89019040892719e-06	\\
-10318.3194247159	1.89004772735831e-06	\\
-10317.3406427557	1.893017436584e-06	\\
-10316.3618607955	1.40800341290863e-06	\\
-10315.3830788352	1.39475085896265e-06	\\
-10314.404296875	1.69292959161427e-06	\\
-10313.4255149148	1.1906518299569e-06	\\
-10312.4467329545	1.13498288989404e-06	\\
-10311.4679509943	1.75598981472755e-06	\\
-10310.4891690341	1.11600925468009e-06	\\
-10309.5103870739	2.26690959496047e-06	\\
-10308.5316051136	1.60156610100995e-06	\\
-10307.5528231534	1.35933569042787e-06	\\
-10306.5740411932	1.26339969897287e-06	\\
-10305.595259233	1.55142568297982e-06	\\
-10304.6164772727	2.04670114336344e-06	\\
-10303.6376953125	1.68640956516544e-06	\\
-10302.6589133523	1.62350716045943e-06	\\
-10301.680131392	1.26281555540549e-06	\\
-10300.7013494318	1.93632188525465e-06	\\
-10299.7225674716	1.79438797536231e-06	\\
-10298.7437855114	1.40289627081351e-06	\\
-10297.7650035511	2.33012968412464e-06	\\
-10296.7862215909	1.63463317177082e-06	\\
-10295.8074396307	1.84722519097224e-06	\\
-10294.8286576705	1.39563233707183e-06	\\
-10293.8498757102	1.49756554028129e-06	\\
-10292.87109375	1.76451205681251e-06	\\
-10291.8923117898	1.97818177161982e-06	\\
-10290.9135298295	1.22318020868388e-06	\\
-10289.9347478693	1.07409731364523e-06	\\
-10288.9559659091	1.87960481621246e-06	\\
-10287.9771839489	1.08269718153188e-06	\\
-10286.9984019886	1.40662073214753e-06	\\
-10286.0196200284	1.62571758649883e-06	\\
-10285.0408380682	1.40189626690661e-06	\\
-10284.062056108	1.73538916027072e-06	\\
-10283.0832741477	1.66042059613721e-06	\\
-10282.1044921875	1.56839599810674e-06	\\
-10281.1257102273	1.51922213789902e-06	\\
-10280.146928267	2.1862714652184e-06	\\
-10279.1681463068	1.31336792229135e-06	\\
-10278.1893643466	1.86833061776086e-06	\\
-10277.2105823864	2.02095787318588e-06	\\
-10276.2318004261	1.85434785696593e-06	\\
-10275.2530184659	1.94394119076225e-06	\\
-10274.2742365057	1.42135351734559e-06	\\
-10273.2954545455	1.1591108373823e-06	\\
-10272.3166725852	1.40011299330409e-06	\\
-10271.337890625	1.95352107092281e-06	\\
-10270.3591086648	1.65090194585472e-06	\\
-10269.3803267045	2.11549716492843e-06	\\
-10268.4015447443	1.63817785710535e-06	\\
-10267.4227627841	1.71542478156377e-06	\\
-10266.4439808239	1.93285837924309e-06	\\
-10265.4651988636	1.70657752246686e-06	\\
-10264.4864169034	1.78055584810209e-06	\\
-10263.5076349432	1.35438257375095e-06	\\
-10262.528852983	1.13116630020535e-06	\\
-10261.5500710227	2.19322632069145e-06	\\
-10260.5712890625	2.01481782465392e-06	\\
-10259.5925071023	1.5545466701202e-06	\\
-10258.613725142	2.03269692098901e-06	\\
-10257.6349431818	1.33377051272189e-06	\\
-10256.6561612216	1.19492106510154e-06	\\
-10255.6773792614	1.62888171549363e-06	\\
-10254.6985973011	1.567877323037e-06	\\
-10253.7198153409	1.75783377233206e-06	\\
-10252.7410333807	1.59378980967664e-06	\\
-10251.7622514205	1.33957900653683e-06	\\
-10250.7834694602	1.84342959127273e-06	\\
-10249.8046875	2.26050912699477e-06	\\
-10248.8259055398	1.80275292313791e-06	\\
-10247.8471235795	1.21232736041134e-06	\\
-10246.8683416193	1.4464428640108e-06	\\
-10245.8895596591	1.90723421943378e-06	\\
-10244.9107776989	1.31650491749884e-06	\\
-10243.9319957386	1.99494429272584e-06	\\
-10242.9532137784	1.56542832094154e-06	\\
-10241.9744318182	1.37952787670656e-06	\\
-10240.995649858	2.57444032651588e-06	\\
-10240.0168678977	1.14570874692485e-06	\\
-10239.0380859375	1.31235355154404e-06	\\
-10238.0593039773	2.24378103457532e-06	\\
-10237.080522017	1.29235445347601e-06	\\
-10236.1017400568	1.86774336565074e-06	\\
-10235.1229580966	1.40218633850984e-06	\\
-10234.1441761364	1.27399423406514e-06	\\
-10233.1653941761	1.59544248488088e-06	\\
-10232.1866122159	1.80484755675179e-06	\\
-10231.2078302557	1.18909660781325e-06	\\
-10230.2290482955	2.21783557720396e-06	\\
-10229.2502663352	1.70926861629057e-06	\\
-10228.271484375	2.03872802200006e-06	\\
-10227.2927024148	1.83582065156745e-06	\\
-10226.3139204545	1.92547047369663e-06	\\
-10225.3351384943	1.51537489292913e-06	\\
-10224.3563565341	1.59439979861285e-06	\\
-10223.3775745739	1.83707728516575e-06	\\
-10222.3987926136	1.2196489588541e-06	\\
-10221.4200106534	1.6358704260609e-06	\\
-10220.4412286932	1.02925609870426e-06	\\
-10219.462446733	1.7006967381243e-06	\\
-10218.4836647727	1.43756075509535e-06	\\
-10217.5048828125	1.44350176866637e-06	\\
-10216.5261008523	1.79250427969321e-06	\\
-10215.547318892	2.0221932907106e-06	\\
-10214.5685369318	1.60289989790558e-06	\\
-10213.5897549716	1.62702058559893e-06	\\
-10212.6109730114	1.58779898331451e-06	\\
-10211.6321910511	1.32843405096004e-06	\\
-10210.6534090909	1.76170440830488e-06	\\
-10209.6746271307	1.47801289480038e-06	\\
-10208.6958451705	1.07224885950937e-06	\\
-10207.7170632102	2.15085819578797e-06	\\
-10206.73828125	1.90576808670429e-06	\\
-10205.7594992898	7.42643307871499e-07	\\
-10204.7807173295	1.55396668571957e-06	\\
-10203.8019353693	1.61206330414218e-06	\\
-10202.8231534091	1.18885338598283e-06	\\
-10201.8443714489	1.98759578984544e-06	\\
-10200.8655894886	1.51349976927675e-06	\\
-10199.8868075284	1.89928072603351e-06	\\
-10198.9080255682	2.21201810593555e-06	\\
-10197.929243608	2.33692834893597e-06	\\
-10196.9504616477	2.24459842158997e-06	\\
-10195.9716796875	2.0244702156786e-06	\\
-10194.9928977273	1.69378613223192e-06	\\
-10194.014115767	1.59030172813498e-06	\\
-10193.0353338068	1.68967132619387e-06	\\
-10192.0565518466	1.30876125148961e-06	\\
-10191.0777698864	2.54934471669789e-06	\\
-10190.0989879261	1.54204219208685e-06	\\
-10189.1202059659	2.1149492142902e-06	\\
-10188.1414240057	1.75228766860849e-06	\\
-10187.1626420455	1.79509051698951e-06	\\
-10186.1838600852	1.40273960102498e-06	\\
-10185.205078125	1.95504837247752e-06	\\
-10184.2262961648	1.7790406574406e-06	\\
-10183.2475142045	1.5606635158403e-06	\\
-10182.2687322443	2.44668170258942e-06	\\
-10181.2899502841	1.22664058181305e-06	\\
-10180.3111683239	1.54050287901814e-06	\\
-10179.3323863636	2.04936764744207e-06	\\
-10178.3536044034	1.29224211208519e-06	\\
-10177.3748224432	1.97695285829321e-06	\\
-10176.396040483	1.43061661613205e-06	\\
-10175.4172585227	1.79796458466486e-06	\\
-10174.4384765625	2.47464091675153e-06	\\
-10173.4596946023	2.15699248226619e-06	\\
-10172.480912642	2.03152445744683e-06	\\
-10171.5021306818	2.04678758702732e-06	\\
-10170.5233487216	2.3528369299135e-06	\\
-10169.5445667614	1.76955494291631e-06	\\
-10168.5657848011	2.22458790559701e-06	\\
-10167.5870028409	1.5744166338468e-06	\\
-10166.6082208807	2.00303899097837e-06	\\
-10165.6294389205	1.88732274996058e-06	\\
-10164.6506569602	1.40250140163793e-06	\\
-10163.671875	1.63871809056413e-06	\\
-10162.6930930398	2.07647065693739e-06	\\
-10161.7143110795	1.73005706759515e-06	\\
-10160.7355291193	1.96093566906338e-06	\\
-10159.7567471591	1.83192532109902e-06	\\
-10158.7779651989	1.26477260251454e-06	\\
-10157.7991832386	1.38458127710446e-06	\\
-10156.8204012784	1.54115296679911e-06	\\
-10155.8416193182	1.75863408504621e-06	\\
-10154.862837358	2.28585171392625e-06	\\
-10153.8840553977	1.74979059643192e-06	\\
-10152.9052734375	1.92282304297415e-06	\\
-10151.9264914773	1.94104022767843e-06	\\
-10150.947709517	2.01900204294618e-06	\\
-10149.9689275568	1.90033313831964e-06	\\
-10148.9901455966	1.64471118796887e-06	\\
-10148.0113636364	1.79150967777806e-06	\\
-10147.0325816761	1.94155642555418e-06	\\
-10146.0537997159	1.56413134039225e-06	\\
-10145.0750177557	2.00633539567598e-06	\\
-10144.0962357955	1.8591776830951e-06	\\
-10143.1174538352	1.80024504880951e-06	\\
-10142.138671875	1.74783025052219e-06	\\
-10141.1598899148	1.99834180526714e-06	\\
-10140.1811079545	1.97518576308396e-06	\\
-10139.2023259943	1.98736538094721e-06	\\
-10138.2235440341	1.85785968316452e-06	\\
-10137.2447620739	1.55688645248827e-06	\\
-10136.2659801136	2.03691897869917e-06	\\
-10135.2871981534	1.65071019995429e-06	\\
-10134.3084161932	1.89907918615481e-06	\\
-10133.329634233	1.62433269247766e-06	\\
-10132.3508522727	2.23953977244552e-06	\\
-10131.3720703125	1.48854727598078e-06	\\
-10130.3932883523	1.89083999720376e-06	\\
-10129.414506392	1.84165856557938e-06	\\
-10128.4357244318	1.62975141904103e-06	\\
-10127.4569424716	1.41162451802147e-06	\\
-10126.4781605114	1.50993518853895e-06	\\
-10125.4993785511	1.77704372829531e-06	\\
-10124.5205965909	1.27594200823243e-06	\\
-10123.5418146307	2.02170029365266e-06	\\
-10122.5630326705	1.89182283737386e-06	\\
-10121.5842507102	1.32168584896414e-06	\\
-10120.60546875	1.70862135927718e-06	\\
-10119.6266867898	1.95889276387388e-06	\\
-10118.6479048295	1.29632945950639e-06	\\
-10117.6691228693	1.76099936791243e-06	\\
-10116.6903409091	1.87672081055581e-06	\\
-10115.7115589489	1.71216627188856e-06	\\
-10114.7327769886	2.08782071990064e-06	\\
-10113.7539950284	2.16214017201349e-06	\\
-10112.7752130682	1.72765109014322e-06	\\
-10111.796431108	2.56525672693557e-06	\\
-10110.8176491477	2.0721237989898e-06	\\
-10109.8388671875	1.62137758652208e-06	\\
-10108.8600852273	1.67690038591826e-06	\\
-10107.881303267	1.65773352295257e-06	\\
-10106.9025213068	1.28141891807604e-06	\\
-10105.9237393466	2.25862931363355e-06	\\
-10104.9449573864	1.25366362973538e-06	\\
-10103.9661754261	1.20842968852438e-06	\\
-10102.9873934659	1.49488321268458e-06	\\
-10102.0086115057	1.35534191510208e-06	\\
-10101.0298295455	1.42501953912516e-06	\\
-10100.0510475852	1.33054384753502e-06	\\
-10099.072265625	1.81219976080244e-06	\\
-10098.0934836648	1.84157027968977e-06	\\
-10097.1147017045	1.54871082897008e-06	\\
-10096.1359197443	1.83804923999216e-06	\\
-10095.1571377841	1.69807519971341e-06	\\
-10094.1783558239	2.01704962773827e-06	\\
-10093.1995738636	1.6669750229126e-06	\\
-10092.2207919034	2.25315987886969e-06	\\
-10091.2420099432	1.48374535219234e-06	\\
-10090.263227983	2.36771722838426e-06	\\
-10089.2844460227	1.80139803740392e-06	\\
-10088.3056640625	2.32248819469435e-06	\\
-10087.3268821023	2.33822995975794e-06	\\
-10086.348100142	2.5452652863046e-06	\\
-10085.3693181818	1.88157332876931e-06	\\
-10084.3905362216	2.68727627394215e-06	\\
-10083.4117542614	1.14262500151097e-06	\\
-10082.4329723011	1.76727373983116e-06	\\
-10081.4541903409	1.96847057959137e-06	\\
-10080.4754083807	2.09591385692122e-06	\\
-10079.4966264205	1.52517799362045e-06	\\
-10078.5178444602	2.07423059926138e-06	\\
-10077.5390625	2.1488929459564e-06	\\
-10076.5602805398	1.76997023844188e-06	\\
-10075.5814985795	2.29318830755292e-06	\\
-10074.6027166193	1.87593724469966e-06	\\
-10073.6239346591	1.87695855929041e-06	\\
-10072.6451526989	2.1618340655165e-06	\\
-10071.6663707386	1.6711269303751e-06	\\
-10070.6875887784	2.13344781825145e-06	\\
-10069.7088068182	8.99943650093156e-07	\\
-10068.730024858	1.30704150456415e-06	\\
-10067.7512428977	1.33170708897234e-06	\\
-10066.7724609375	1.41231171907538e-06	\\
-10065.7936789773	1.04356705058007e-06	\\
-10064.814897017	2.10744195599665e-06	\\
-10063.8361150568	1.03259413255376e-06	\\
-10062.8573330966	1.40361996086677e-06	\\
-10061.8785511364	1.3656758332259e-06	\\
-10060.8997691761	1.810100227229e-06	\\
-10059.9209872159	1.69134322313535e-06	\\
-10058.9422052557	1.66781668942868e-06	\\
-10057.9634232955	1.7639160272708e-06	\\
-10056.9846413352	2.07161424143985e-06	\\
-10056.005859375	1.84071915677085e-06	\\
-10055.0270774148	1.10378957328089e-06	\\
-10054.0482954545	2.36559526880912e-06	\\
-10053.0695134943	1.84999890201152e-06	\\
-10052.0907315341	1.60429956976115e-06	\\
-10051.1119495739	1.04896829403143e-06	\\
-10050.1331676136	1.14920825199044e-06	\\
-10049.1543856534	1.39572620340714e-06	\\
-10048.1756036932	1.59935833153507e-06	\\
-10047.196821733	1.14925344464945e-06	\\
-10046.2180397727	2.09845351934725e-06	\\
-10045.2392578125	2.27362423571298e-06	\\
-10044.2604758523	1.5392244577712e-06	\\
-10043.281693892	1.43897869214689e-06	\\
-10042.3029119318	1.79005901559147e-06	\\
-10041.3241299716	2.2003531590873e-06	\\
-10040.3453480114	1.4674507616313e-06	\\
-10039.3665660511	2.32635901640185e-06	\\
-10038.3877840909	1.02937252499369e-06	\\
-10037.4090021307	1.52368850771256e-06	\\
-10036.4302201705	1.08686893910592e-06	\\
-10035.4514382102	1.030454899718e-06	\\
-10034.47265625	1.49411589283414e-06	\\
-10033.4938742898	1.5940266103133e-06	\\
-10032.5150923295	1.11679015527284e-06	\\
-10031.5363103693	1.47652352527321e-06	\\
-10030.5575284091	2.09006556450372e-06	\\
-10029.5787464489	1.48515465698924e-06	\\
-10028.5999644886	1.28553237227517e-06	\\
-10027.6211825284	1.54885784340586e-06	\\
-10026.6424005682	1.1749654631812e-06	\\
-10025.663618608	1.83970134058036e-06	\\
-10024.6848366477	1.59905458741629e-06	\\
-10023.7060546875	1.22190432160141e-06	\\
-10022.7272727273	1.4343296130075e-06	\\
-10021.748490767	1.53102391651306e-06	\\
-10020.7697088068	1.21393691706256e-06	\\
-10019.7909268466	1.98558166855525e-06	\\
-10018.8121448864	1.7392549578923e-06	\\
-10017.8333629261	1.63450183684815e-06	\\
-10016.8545809659	1.93346195085642e-06	\\
-10015.8757990057	1.99379715227111e-06	\\
-10014.8970170455	1.48383986926398e-06	\\
-10013.9182350852	1.41996773194045e-06	\\
-10012.939453125	1.15081577899614e-06	\\
-10011.9606711648	1.702483157836e-06	\\
-10010.9818892045	1.37294009503404e-06	\\
-10010.0031072443	1.39089275312273e-06	\\
-10009.0243252841	1.35028389032687e-06	\\
-10008.0455433239	1.76149276159488e-06	\\
-10007.0667613636	1.32537117358333e-06	\\
-10006.0879794034	6.8417186764418e-07	\\
-10005.1091974432	1.26164773205354e-06	\\
-10004.130415483	1.4607076999907e-06	\\
-10003.1516335227	1.69938646059155e-06	\\
-10002.1728515625	1.63406633965975e-06	\\
-10001.1940696023	1.41237672271975e-06	\\
-10000.215287642	2.4572876060175e-06	\\
-9999.23650568182	1.57202707577607e-06	\\
-9998.25772372159	1.45280271927434e-06	\\
-9997.27894176136	1.40055164319302e-06	\\
-9996.30015980114	1.3976098947686e-06	\\
-9995.32137784091	1.43770369084974e-06	\\
-9994.34259588068	1.32495571587247e-06	\\
-9993.36381392045	1.47773641055453e-06	\\
-9992.38503196023	1.87563886015108e-06	\\
-9991.40625	1.09719177934531e-06	\\
-9990.42746803977	1.26647684287726e-06	\\
-9989.44868607955	1.37752299454918e-06	\\
-9988.46990411932	1.39442263018597e-06	\\
-9987.49112215909	6.43325968613317e-07	\\
-9986.51234019886	1.3865683918285e-06	\\
-9985.53355823864	1.79287434910466e-06	\\
-9984.55477627841	1.5407614195102e-06	\\
-9983.57599431818	1.74868194248061e-06	\\
-9982.59721235795	2.06294231607413e-06	\\
-9981.61843039773	1.05286365795666e-06	\\
-9980.6396484375	2.04812674952116e-06	\\
-9979.66086647727	1.33824661184758e-06	\\
-9978.68208451705	1.81134494826327e-06	\\
-9977.70330255682	1.80781896110332e-06	\\
-9976.72452059659	1.72230612336254e-06	\\
-9975.74573863636	1.54764270004178e-06	\\
-9974.76695667614	1.72342207040912e-06	\\
-9973.78817471591	9.27860536225131e-07	\\
-9972.80939275568	1.78998370291424e-06	\\
-9971.83061079545	1.27399499561867e-06	\\
-9970.85182883523	1.36013998817574e-06	\\
-9969.873046875	1.77281718091066e-06	\\
-9968.89426491477	1.53966382628731e-06	\\
-9967.91548295455	1.66301716874401e-06	\\
-9966.93670099432	2.18540027211576e-06	\\
-9965.95791903409	1.05880108796178e-06	\\
-9964.97913707386	8.56451329439525e-07	\\
-9964.00035511364	1.91119317478071e-06	\\
-9963.02157315341	1.42842219412778e-06	\\
-9962.04279119318	2.35412117926493e-06	\\
-9961.06400923295	1.81389153347209e-06	\\
-9960.08522727273	1.8620932243118e-06	\\
-9959.1064453125	1.74930424866717e-06	\\
-9958.12766335227	1.8844853130007e-06	\\
-9957.14888139205	1.26527928866907e-06	\\
-9956.17009943182	7.51814638072277e-07	\\
-9955.19131747159	1.81582109759585e-06	\\
-9954.21253551136	1.20678484537295e-06	\\
-9953.23375355114	1.37751186300261e-06	\\
-9952.25497159091	2.09992736121899e-06	\\
-9951.27618963068	1.89680555410271e-06	\\
-9950.29740767045	1.91248891815801e-06	\\
-9949.31862571023	2.25826198599252e-06	\\
-9948.33984375	1.78470453359896e-06	\\
-9947.36106178977	1.93151095982084e-06	\\
-9946.38227982955	1.98651975218473e-06	\\
-9945.40349786932	1.35536575022742e-06	\\
-9944.42471590909	1.41479694422185e-06	\\
-9943.44593394886	1.57224551521421e-06	\\
-9942.46715198864	1.6962762223578e-06	\\
-9941.48837002841	1.8784830158598e-06	\\
-9940.50958806818	2.26202987429551e-06	\\
-9939.53080610795	1.14004986776608e-06	\\
-9938.55202414773	1.53515418758475e-06	\\
-9937.5732421875	1.34991605461249e-06	\\
-9936.59446022727	2.70551575190347e-06	\\
-9935.61567826705	2.14149181398831e-06	\\
-9934.63689630682	1.99201068678137e-06	\\
-9933.65811434659	1.86038042335357e-06	\\
-9932.67933238636	2.53030175682976e-06	\\
-9931.70055042614	2.28142243869288e-06	\\
-9930.72176846591	1.89522618830531e-06	\\
-9929.74298650568	2.15317299674227e-06	\\
-9928.76420454545	2.23172366036153e-06	\\
-9927.78542258523	1.49225266112219e-06	\\
-9926.806640625	1.55596868247202e-06	\\
-9925.82785866477	1.37447444488919e-06	\\
-9924.84907670455	2.7428053560269e-06	\\
-9923.87029474432	1.6592892762739e-06	\\
-9922.89151278409	1.28950220607963e-06	\\
-9921.91273082386	1.9364533273872e-06	\\
-9920.93394886364	1.71948772846501e-06	\\
-9919.95516690341	2.25042407233974e-06	\\
-9918.97638494318	1.788908929738e-06	\\
-9917.99760298295	1.41541985232463e-06	\\
-9917.01882102273	1.86561656186681e-06	\\
-9916.0400390625	1.79659662846958e-06	\\
-9915.06125710227	1.14867434636725e-06	\\
-9914.08247514205	1.86290509039858e-06	\\
-9913.10369318182	1.49754283770706e-06	\\
-9912.12491122159	1.20049132055839e-06	\\
-9911.14612926136	1.32857109926153e-06	\\
-9910.16734730114	2.02940884020239e-06	\\
-9909.18856534091	1.51183658451111e-06	\\
-9908.20978338068	9.23387292746899e-07	\\
-9907.23100142045	2.26555487602633e-06	\\
-9906.25221946023	1.78258992517571e-06	\\
-9905.2734375	1.27866049866981e-06	\\
-9904.29465553977	2.57000767548497e-06	\\
-9903.31587357955	2.38194276376568e-06	\\
-9902.33709161932	1.651545650273e-06	\\
-9901.35830965909	1.10197515879817e-06	\\
-9900.37952769886	2.11261135382173e-06	\\
-9899.40074573864	2.14522992317272e-06	\\
-9898.42196377841	1.84307002625973e-06	\\
-9897.44318181818	1.56467942445449e-06	\\
-9896.46439985795	1.89292365591278e-06	\\
-9895.48561789773	1.5964876991993e-06	\\
-9894.5068359375	1.07896961402449e-06	\\
-9893.52805397727	1.69384921337365e-06	\\
-9892.54927201705	1.57406101954497e-06	\\
-9891.57049005682	1.0551124893495e-06	\\
-9890.59170809659	1.72115602083781e-06	\\
-9889.61292613636	1.81156089517556e-06	\\
-9888.63414417614	1.4146166860776e-06	\\
-9887.65536221591	2.05555360845719e-06	\\
-9886.67658025568	1.56516650880298e-06	\\
-9885.69779829545	1.0130835567441e-06	\\
-9884.71901633523	1.66569225542018e-06	\\
-9883.740234375	1.42994265384119e-06	\\
-9882.76145241477	1.11139450680212e-06	\\
-9881.78267045455	1.80407247168487e-06	\\
-9880.80388849432	1.70214190913209e-06	\\
-9879.82510653409	1.47329966262208e-06	\\
-9878.84632457386	2.04065461622126e-06	\\
-9877.86754261364	1.34985113666536e-06	\\
-9876.88876065341	1.47083651348394e-06	\\
-9875.90997869318	8.5223095308202e-07	\\
-9874.93119673295	9.82012113060812e-07	\\
-9873.95241477273	1.49646746957777e-06	\\
-9872.9736328125	1.64753622731443e-06	\\
-9871.99485085227	1.43252281248932e-06	\\
-9871.01606889205	2.21792619215352e-06	\\
-9870.03728693182	1.73299283553783e-06	\\
-9869.05850497159	1.55610768211346e-06	\\
-9868.07972301136	2.49159025939222e-06	\\
-9867.10094105114	1.06475838271299e-06	\\
-9866.12215909091	1.6206460766485e-06	\\
-9865.14337713068	1.82860810658841e-06	\\
-9864.16459517045	4.35820251348085e-07	\\
-9863.18581321023	9.70156541278394e-07	\\
-9862.20703125	1.85746579849873e-06	\\
-9861.22824928977	5.33792560908589e-07	\\
-9860.24946732955	1.34723849734352e-06	\\
-9859.27068536932	1.31131912968756e-06	\\
-9858.29190340909	9.74279342871117e-07	\\
-9857.31312144886	1.33377475618598e-06	\\
-9856.33433948864	1.66800182824856e-06	\\
-9855.35555752841	9.3230553125719e-07	\\
-9854.37677556818	1.8782268558928e-06	\\
-9853.39799360795	1.88921124462277e-06	\\
-9852.41921164773	5.59199268543706e-07	\\
-9851.4404296875	2.11725425552421e-06	\\
-9850.46164772727	1.62136717328977e-06	\\
-9849.48286576705	1.32900996589974e-06	\\
-9848.50408380682	2.39863222313885e-06	\\
-9847.52530184659	9.40649382322585e-07	\\
-9846.54651988636	1.84725821520408e-06	\\
-9845.56773792614	1.78975163609882e-06	\\
-9844.58895596591	1.83410862229285e-06	\\
-9843.61017400568	9.51623517419882e-07	\\
-9842.63139204545	1.57445919843914e-06	\\
-9841.65261008523	1.77028055770314e-06	\\
-9840.673828125	1.73276449966221e-06	\\
-9839.69504616477	1.20584085424133e-06	\\
-9838.71626420455	1.65520753683063e-06	\\
-9837.73748224432	2.01002257380222e-06	\\
-9836.75870028409	1.07596969103616e-06	\\
-9835.77991832386	1.6099924452849e-06	\\
-9834.80113636364	2.10516966065206e-06	\\
-9833.82235440341	1.32739445477457e-06	\\
-9832.84357244318	1.38667137732996e-06	\\
-9831.86479048295	1.65196914520551e-06	\\
-9830.88600852273	1.68502911248876e-06	\\
-9829.9072265625	1.87758437220119e-06	\\
-9828.92844460227	1.69558599178319e-06	\\
-9827.94966264205	1.76202544953166e-06	\\
-9826.97088068182	9.7182107609655e-07	\\
-9825.99209872159	1.56616368738718e-06	\\
-9825.01331676136	1.03051434577327e-06	\\
-9824.03453480114	1.187280663546e-06	\\
-9823.05575284091	1.68104637623053e-06	\\
-9822.07697088068	1.42595188991058e-06	\\
-9821.09818892045	1.7978175916512e-06	\\
-9820.11940696023	1.19035297069524e-06	\\
-9819.140625	1.65404351296318e-06	\\
-9818.16184303977	1.91073683434515e-06	\\
-9817.18306107955	8.11262874233179e-07	\\
-9816.20427911932	1.55402141249153e-06	\\
-9815.22549715909	1.44121489293333e-06	\\
-9814.24671519886	5.51776896602492e-07	\\
-9813.26793323864	1.340912247364e-06	\\
-9812.28915127841	1.16880787963046e-06	\\
-9811.31036931818	2.18426128755967e-06	\\
-9810.33158735795	1.12726451381219e-06	\\
-9809.35280539773	2.29650507105461e-06	\\
-9808.3740234375	1.2309412156927e-06	\\
-9807.39524147727	1.00673365729159e-06	\\
-9806.41645951705	1.47297234581018e-06	\\
-9805.43767755682	1.08895614189472e-06	\\
-9804.45889559659	1.04108437483607e-06	\\
-9803.48011363636	1.97633057337287e-06	\\
-9802.50133167614	1.86761587294018e-06	\\
-9801.52254971591	1.33962599108963e-06	\\
-9800.54376775568	1.85577684486983e-06	\\
-9799.56498579545	1.61545897041491e-06	\\
-9798.58620383523	1.65059684041485e-06	\\
-9797.607421875	1.32095796444311e-06	\\
-9796.62863991477	1.2458993325956e-06	\\
-9795.64985795455	2.12614176216716e-06	\\
-9794.67107599432	1.38947500909854e-06	\\
-9793.69229403409	9.65657936752557e-07	\\
-9792.71351207386	1.83348428227011e-06	\\
-9791.73473011364	2.7729336609533e-06	\\
-9790.75594815341	1.27019415189918e-06	\\
-9789.77716619318	2.53907540915865e-06	\\
-9788.79838423295	1.35502913479637e-06	\\
-9787.81960227273	7.57171981812985e-07	\\
-9786.8408203125	1.65464733132815e-06	\\
-9785.86203835227	1.62170267374395e-06	\\
-9784.88325639205	2.14702620769018e-06	\\
-9783.90447443182	1.44175938843908e-06	\\
-9782.92569247159	5.75479635068282e-07	\\
-9781.94691051136	1.09795212432575e-06	\\
-9780.96812855114	1.64416489604543e-06	\\
-9779.98934659091	1.11050926639466e-06	\\
-9779.01056463068	1.29778004370245e-06	\\
-9778.03178267045	1.34407129316664e-06	\\
-9777.05300071023	2.08582047475435e-06	\\
-9776.07421875	1.71245676379958e-06	\\
-9775.09543678977	2.47645881065887e-06	\\
-9774.11665482955	1.17261471120273e-06	\\
-9773.13787286932	4.85671726073883e-07	\\
-9772.15909090909	1.51939624995448e-06	\\
-9771.18030894886	7.11246180454093e-07	\\
-9770.20152698864	2.22522039855496e-06	\\
-9769.22274502841	2.05441102305968e-06	\\
-9768.24396306818	1.1011655001397e-06	\\
-9767.26518110795	1.96796571344117e-06	\\
-9766.28639914773	1.45984913168075e-06	\\
-9765.3076171875	2.08773945743463e-06	\\
-9764.32883522727	9.13775295353676e-07	\\
-9763.35005326705	1.53587376416227e-06	\\
-9762.37127130682	1.11169803950079e-06	\\
-9761.39248934659	2.40912475104534e-06	\\
-9760.41370738636	8.54369464953844e-07	\\
-9759.43492542614	1.47648368306146e-06	\\
-9758.45614346591	1.8349609954564e-06	\\
-9757.47736150568	1.20166743335836e-06	\\
-9756.49857954545	1.90229845884032e-06	\\
-9755.51979758523	1.79856299215516e-06	\\
-9754.541015625	1.20884635265388e-06	\\
-9753.56223366477	1.86362917106975e-06	\\
-9752.58345170455	2.13674884825246e-06	\\
-9751.60466974432	1.34817123286002e-06	\\
-9750.62588778409	1.35168111296198e-06	\\
-9749.64710582386	1.34237884282743e-06	\\
-9748.66832386364	8.45108936510256e-07	\\
-9747.68954190341	1.34562010584543e-06	\\
-9746.71075994318	2.24582561288636e-06	\\
-9745.73197798295	8.02108996414532e-07	\\
-9744.75319602273	8.93580543650336e-07	\\
-9743.7744140625	1.82048386478487e-06	\\
-9742.79563210227	1.48003983275487e-06	\\
-9741.81685014205	1.58810770279629e-06	\\
-9740.83806818182	1.36684061374693e-06	\\
-9739.85928622159	1.67917256554233e-06	\\
-9738.88050426136	1.95229359719937e-06	\\
-9737.90172230114	1.38896125380037e-06	\\
-9736.92294034091	1.62647736543715e-06	\\
-9735.94415838068	8.88815693268383e-07	\\
-9734.96537642045	1.5404660725427e-06	\\
-9733.98659446023	1.53816347463942e-06	\\
-9733.0078125	2.08105441108376e-06	\\
-9732.02903053977	1.23841329512432e-06	\\
-9731.05024857955	1.77293599304373e-06	\\
-9730.07146661932	2.27908462363731e-06	\\
-9729.09268465909	1.40364950894192e-06	\\
-9728.11390269886	1.43300478954693e-06	\\
-9727.13512073864	1.5799858098041e-06	\\
-9726.15633877841	1.72420766478434e-06	\\
-9725.17755681818	1.6237405060779e-06	\\
-9724.19877485795	1.26585677879369e-06	\\
-9723.21999289773	1.35269680594346e-06	\\
-9722.2412109375	1.65861223953161e-06	\\
-9721.26242897727	1.81122560572319e-06	\\
-9720.28364701705	9.70977202755725e-07	\\
-9719.30486505682	1.16885144431099e-06	\\
-9718.32608309659	1.79001270236022e-06	\\
-9717.34730113636	8.98611063772713e-07	\\
-9716.36851917614	1.24811299933887e-06	\\
-9715.38973721591	1.59032503941331e-06	\\
-9714.41095525568	1.021317873416e-06	\\
-9713.43217329545	2.52582366924622e-06	\\
-9712.45339133523	1.16825675873503e-06	\\
-9711.474609375	7.14354584093275e-07	\\
-9710.49582741477	2.30176722446918e-06	\\
-9709.51704545455	1.90790202341455e-06	\\
-9708.53826349432	1.77910019630234e-06	\\
-9707.55948153409	1.38132869640979e-06	\\
-9706.58069957386	1.920846838143e-06	\\
-9705.60191761364	1.94288439188408e-06	\\
-9704.62313565341	2.40855999241598e-06	\\
-9703.64435369318	1.3971547445267e-06	\\
-9702.66557173295	1.35398716968495e-06	\\
-9701.68678977273	2.06829219223033e-06	\\
-9700.7080078125	1.58476914860583e-06	\\
-9699.72922585227	1.74973213326393e-06	\\
-9698.75044389205	1.14689424311111e-06	\\
-9697.77166193182	1.21734967949325e-06	\\
-9696.79287997159	1.65772631728657e-06	\\
-9695.81409801136	1.63342731615419e-06	\\
-9694.83531605114	8.50416588515309e-07	\\
-9693.85653409091	1.61274666593281e-06	\\
-9692.87775213068	9.04002295923602e-07	\\
-9691.89897017045	1.87200433990697e-06	\\
-9690.92018821023	1.08642904088027e-06	\\
-9689.94140625	1.10454965600828e-06	\\
-9688.96262428977	1.19424340933482e-06	\\
-9687.98384232955	1.16250424163057e-06	\\
-9687.00506036932	1.50436804052755e-06	\\
-9686.02627840909	2.13802317842111e-06	\\
-9685.04749644886	8.28469280250547e-07	\\
-9684.06871448864	8.87695273608508e-07	\\
-9683.08993252841	9.75713194744584e-07	\\
-9682.11115056818	1.12725108849928e-06	\\
-9681.13236860795	2.0556320961744e-06	\\
-9680.15358664773	1.2861245927256e-06	\\
-9679.1748046875	2.12339989769078e-06	\\
-9678.19602272727	1.30130821231406e-06	\\
-9677.21724076705	8.16951030154323e-07	\\
-9676.23845880682	2.76405702551088e-07	\\
-9675.25967684659	1.4535915095941e-06	\\
-9674.28089488636	1.38415012166851e-07	\\
-9673.30211292614	4.33882706583638e-07	\\
-9672.32333096591	1.80176349522415e-06	\\
-9671.34454900568	1.53849844205465e-06	\\
-9670.36576704545	1.43845033700891e-06	\\
-9669.38698508523	2.12192376445461e-06	\\
-9668.408203125	1.31937489224826e-06	\\
-9667.42942116477	8.30038479769984e-07	\\
-9666.45063920455	2.22848405140867e-06	\\
-9665.47185724432	8.20910592027106e-07	\\
-9664.49307528409	8.91044391274929e-07	\\
-9663.51429332386	1.62994090594114e-06	\\
-9662.53551136364	1.51883429635231e-06	\\
-9661.55672940341	8.00247020255558e-07	\\
-9660.57794744318	1.1898044673138e-06	\\
-9659.59916548295	1.71747645051753e-06	\\
-9658.62038352273	5.57620654040434e-07	\\
-9657.6416015625	1.40181687460179e-06	\\
-9656.66281960227	1.16469178813039e-06	\\
-9655.68403764205	1.77396084739306e-06	\\
-9654.70525568182	1.30553862713622e-06	\\
-9653.72647372159	1.41255693919786e-06	\\
-9652.74769176136	2.24990557558777e-06	\\
-9651.76890980114	1.37355250262671e-06	\\
-9650.79012784091	2.35194994636249e-06	\\
-9649.81134588068	1.93262612268533e-06	\\
-9648.83256392045	2.17545555868138e-06	\\
-9647.85378196023	1.70708638302484e-06	\\
-9646.875	2.24927442922318e-06	\\
-9645.89621803977	1.09165163785271e-06	\\
-9644.91743607955	6.11526523135712e-07	\\
-9643.93865411932	1.54867701937238e-06	\\
-9642.95987215909	1.93251044430958e-06	\\
-9641.98109019886	1.37087804071297e-06	\\
-9641.00230823864	2.52470416365474e-06	\\
-9640.02352627841	2.09802279336388e-06	\\
-9639.04474431818	1.0563715525059e-06	\\
-9638.06596235795	2.138518083784e-06	\\
-9637.08718039773	1.86353728963111e-06	\\
-9636.1083984375	9.21034207017037e-07	\\
-9635.12961647727	1.44512106366203e-06	\\
-9634.15083451705	9.85159732706283e-07	\\
-9633.17205255682	1.24858780913705e-06	\\
-9632.19327059659	1.66179207014459e-06	\\
-9631.21448863636	5.50511530068687e-07	\\
-9630.23570667614	2.21701735106338e-07	\\
-9629.25692471591	1.45835283887651e-06	\\
-9628.27814275568	9.00495950882308e-07	\\
-9627.29936079545	1.37126368272168e-06	\\
-9626.32057883523	2.0290150908879e-06	\\
-9625.341796875	1.35833092177807e-06	\\
-9624.36301491477	5.40513325490104e-07	\\
-9623.38423295455	1.86276153072211e-06	\\
-9622.40545099432	1.34133305426725e-06	\\
-9621.42666903409	1.31236948568782e-06	\\
-9620.44788707386	1.21824813275503e-06	\\
-9619.46910511364	2.02349042999547e-06	\\
-9618.49032315341	2.25749943966176e-06	\\
-9617.51154119318	3.88484499058578e-07	\\
-9616.53275923295	1.83199588591984e-06	\\
-9615.55397727273	1.62478009723808e-06	\\
-9614.5751953125	2.26470374640113e-06	\\
-9613.59641335227	1.15653985608989e-06	\\
-9612.61763139205	1.35841398827186e-06	\\
-9611.63884943182	1.03556522475811e-06	\\
-9610.66006747159	1.52372770327425e-06	\\
-9609.68128551136	1.38838137997117e-06	\\
-9608.70250355114	1.41838248095949e-06	\\
-9607.72372159091	7.27645680199084e-07	\\
-9606.74493963068	1.12295601592369e-06	\\
-9605.76615767045	5.65011131795598e-07	\\
-9604.78737571023	1.54893543570149e-06	\\
-9603.80859375	8.20388254354482e-07	\\
-9602.82981178977	1.18033892756355e-06	\\
-9601.85102982955	6.59220508236174e-07	\\
-9600.87224786932	8.89667977458859e-07	\\
-9599.89346590909	1.51452594609946e-06	\\
-9598.91468394886	2.17605840231328e-06	\\
-9597.93590198864	2.0599712049728e-06	\\
-9596.95712002841	1.04627702110055e-06	\\
-9595.97833806818	2.12946013064506e-06	\\
-9594.99955610795	1.1401257658441e-06	\\
-9594.02077414773	2.67629986767757e-07	\\
-9593.0419921875	1.46041512619939e-06	\\
-9592.06321022727	1.38975008796861e-06	\\
-9591.08442826705	5.16292625775367e-07	\\
-9590.10564630682	1.25067965840717e-06	\\
-9589.12686434659	1.18310009871786e-06	\\
-9588.14808238636	1.09143293240168e-06	\\
-9587.16930042614	8.99642256211695e-07	\\
-9586.19051846591	1.89785677155627e-06	\\
-9585.21173650568	1.67014536435367e-06	\\
-9584.23295454545	1.16581057813755e-06	\\
-9583.25417258523	2.51546770318008e-07	\\
-9582.275390625	6.55070899904047e-07	\\
-9581.29660866477	5.97448194913892e-07	\\
-9580.31782670455	9.23935947124265e-08	\\
-9579.33904474432	1.31744997250482e-06	\\
-9578.36026278409	1.66876671913501e-06	\\
-9577.38148082386	1.89888560295986e-06	\\
-9576.40269886364	6.12773599420247e-07	\\
-9575.42391690341	1.47105212386152e-06	\\
-9574.44513494318	1.89547991735777e-06	\\
-9573.46635298295	1.54184046765412e-06	\\
-9572.48757102273	3.78850230922092e-07	\\
-9571.5087890625	1.45495605947522e-07	\\
-9570.53000710227	1.86695984901689e-06	\\
-9569.55122514205	1.02779420227023e-06	\\
-9568.57244318182	9.41685117324801e-07	\\
-9567.59366122159	1.34104717048946e-06	\\
-9566.61487926136	1.2144688716167e-06	\\
-9565.63609730114	6.45977486242379e-07	\\
-9564.65731534091	2.27307901671776e-06	\\
-9563.67853338068	1.17188901144031e-06	\\
-9562.69975142045	1.0093751215643e-06	\\
-9561.72096946023	1.36318890992215e-06	\\
-9560.7421875	7.8058396313261e-07	\\
-9559.76340553977	1.14080204979291e-06	\\
-9558.78462357955	1.16752895690326e-06	\\
-9557.80584161932	1.01901491012334e-06	\\
-9556.82705965909	9.19854414759198e-07	\\
-9555.84827769886	1.44351333639976e-06	\\
-9554.86949573864	1.85949690827582e-06	\\
-9553.89071377841	1.64768820469012e-06	\\
-9552.91193181818	1.98054186327817e-06	\\
-9551.93314985795	1.41339568614202e-06	\\
-9550.95436789773	1.10039117122893e-06	\\
-9549.9755859375	1.03398452122625e-06	\\
-9548.99680397727	7.4680872721568e-07	\\
-9548.01802201705	1.8863856101682e-06	\\
-9547.03924005682	1.77950359350857e-06	\\
-9546.06045809659	1.49106338270203e-06	\\
-9545.08167613636	9.39628391041931e-07	\\
-9544.10289417614	4.49318390111886e-07	\\
-9543.12411221591	1.0916825353483e-06	\\
-9542.14533025568	1.41062907651406e-06	\\
-9541.16654829545	7.10915375419117e-07	\\
-9540.18776633523	1.58102222310369e-06	\\
-9539.208984375	1.03063184888319e-06	\\
-9538.23020241477	1.05896105967678e-06	\\
-9537.25142045455	1.56442937965666e-06	\\
-9536.27263849432	1.66356302598516e-06	\\
-9535.29385653409	1.18283508309512e-06	\\
-9534.31507457386	9.75739476199934e-07	\\
-9533.33629261364	1.58483368333141e-06	\\
-9532.35751065341	1.51123195872057e-06	\\
-9531.37872869318	2.07368810227091e-06	\\
-9530.39994673295	1.11103747209238e-06	\\
-9529.42116477273	1.12200124210212e-06	\\
-9528.4423828125	2.58621483001598e-06	\\
-9527.46360085227	1.40132790375885e-06	\\
-9526.48481889205	5.31636061254308e-07	\\
-9525.50603693182	1.484253307511e-06	\\
-9524.52725497159	1.96623517379157e-06	\\
-9523.54847301136	7.38239643578063e-07	\\
-9522.56969105114	1.51204151647305e-06	\\
-9521.59090909091	3.88828652091792e-07	\\
-9520.61212713068	1.49365878841452e-06	\\
-9519.63334517045	1.36872059013914e-06	\\
-9518.65456321023	8.58748381081656e-07	\\
-9517.67578125	2.57160312847532e-07	\\
-9516.69699928977	1.10471156384117e-06	\\
-9515.71821732955	9.02158931544055e-07	\\
-9514.73943536932	1.010507380137e-06	\\
-9513.76065340909	1.57648984749869e-06	\\
-9512.78187144886	7.75180306081029e-07	\\
-9511.80308948864	8.37200356907423e-07	\\
-9510.82430752841	5.27416127663875e-07	\\
-9509.84552556818	1.50348263476709e-06	\\
-9508.86674360795	1.32419691592014e-06	\\
-9507.88796164773	1.53307614785915e-06	\\
-9506.9091796875	1.60758759921891e-06	\\
-9505.93039772727	9.08991274386124e-07	\\
-9504.95161576705	8.75229732697439e-07	\\
-9503.97283380682	1.17693490938752e-06	\\
-9502.99405184659	1.19991165390747e-06	\\
-9502.01526988636	8.6169315242637e-07	\\
-9501.03648792614	1.05757522159016e-06	\\
-9500.05770596591	8.29177996762416e-07	\\
-9499.07892400568	3.60072484794225e-07	\\
-9498.10014204545	9.69127560834668e-07	\\
-9497.12136008523	1.11138191291733e-06	\\
-9496.142578125	1.01716297952709e-06	\\
-9495.16379616477	1.17220154110032e-06	\\
-9494.18501420455	6.81467978549313e-07	\\
-9493.20623224432	4.17428710083882e-07	\\
-9492.22745028409	9.59583925142852e-07	\\
-9491.24866832386	1.14776211352107e-06	\\
-9490.26988636364	1.09357549099589e-06	\\
-9489.29110440341	9.0714727200529e-07	\\
-9488.31232244318	1.70106658058823e-07	\\
-9487.33354048295	1.12730733001443e-06	\\
-9486.35475852273	1.46700568350115e-06	\\
-9485.3759765625	5.67069843070656e-07	\\
-9484.39719460227	7.44882154410025e-07	\\
-9483.41841264205	1.50572377141598e-06	\\
-9482.43963068182	6.97758043549694e-07	\\
-9481.46084872159	9.0192589747548e-07	\\
-9480.48206676136	7.28693270472058e-07	\\
-9479.50328480114	8.48670901573024e-07	\\
-9478.52450284091	3.39325080815367e-07	\\
-9477.54572088068	7.99842919313981e-07	\\
-9476.56693892045	1.103490049112e-06	\\
-9475.58815696023	2.13117241210455e-06	\\
-9474.609375	1.72739307398069e-06	\\
-9473.63059303977	9.47482563772298e-07	\\
-9472.65181107955	8.16262875656045e-08	\\
-9471.67302911932	5.33996067512617e-07	\\
-9470.69424715909	7.29366607567173e-07	\\
-9469.71546519886	8.68481735362943e-07	\\
-9468.73668323864	8.81214535251027e-07	\\
-9467.75790127841	1.05076945125438e-06	\\
-9466.77911931818	1.80146975010839e-06	\\
-9465.80033735795	1.70909438342722e-06	\\
-9464.82155539773	1.25542238892949e-06	\\
-9463.8427734375	1.30529472630458e-06	\\
-9462.86399147727	8.64221491777155e-07	\\
-9461.88520951705	6.39014315555371e-07	\\
-9460.90642755682	6.97531663029686e-07	\\
-9459.92764559659	1.27727315038041e-06	\\
-9458.94886363636	9.30592861699854e-07	\\
-9457.97008167614	9.11547941027198e-07	\\
-9456.99129971591	7.64373077434674e-07	\\
-9456.01251775568	7.00131462425708e-07	\\
-9455.03373579545	4.12944721021235e-08	\\
-9454.05495383523	7.93218224056602e-07	\\
-9453.076171875	6.52919617555847e-07	\\
-9452.09738991477	1.40198472217627e-06	\\
-9451.11860795455	3.974141558128e-07	\\
-9450.13982599432	1.27818607090822e-06	\\
-9449.16104403409	3.10505491537297e-07	\\
-9448.18226207386	4.38627912510905e-07	\\
-9447.20348011364	5.47628242473973e-07	\\
-9446.22469815341	1.04605258623526e-06	\\
-9445.24591619318	7.72545923899016e-07	\\
-9444.26713423295	9.64493472466269e-07	\\
-9443.28835227273	8.75417954291667e-07	\\
-9442.3095703125	4.18464355764074e-07	\\
-9441.33078835227	5.29462854278188e-07	\\
-9440.35200639205	7.49079543799861e-07	\\
-9439.37322443182	2.45456815432894e-07	\\
-9438.39444247159	1.97370610498713e-06	\\
-9437.41566051136	4.53379048245453e-07	\\
-9436.43687855114	8.18960574830875e-07	\\
-9435.45809659091	2.59618319541005e-07	\\
-9434.47931463068	9.23222104944489e-07	\\
-9433.50053267045	1.55604660661723e-06	\\
-9432.52175071023	5.78071886870576e-07	\\
-9431.54296875	4.48410817000834e-07	\\
-9430.56418678977	5.13317689486358e-07	\\
-9429.58540482955	4.75701555724256e-07	\\
-9428.60662286932	5.68247383234004e-07	\\
-9427.62784090909	1.00602115703271e-06	\\
-9426.64905894886	7.69859605166571e-07	\\
-9425.67027698864	8.9664969052057e-07	\\
-9424.69149502841	7.92925825442631e-08	\\
-9423.71271306818	6.24367740089579e-07	\\
-9422.73393110795	1.25194847274705e-06	\\
-9421.75514914773	9.91226915437376e-07	\\
-9420.7763671875	1.19872032675867e-06	\\
-9419.79758522727	3.89503260591257e-07	\\
-9418.81880326705	6.5742108425336e-07	\\
-9417.84002130682	1.26628860421959e-06	\\
-9416.86123934659	1.13571773316186e-06	\\
-9415.88245738636	3.33638816284809e-07	\\
-9414.90367542614	1.01590222788588e-06	\\
-9413.92489346591	6.22075340567346e-07	\\
-9412.94611150568	4.17358723844697e-07	\\
-9411.96732954545	4.49589113884894e-07	\\
-9410.98854758523	5.55344168301636e-07	\\
-9410.009765625	1.11175445582646e-06	\\
-9409.03098366477	1.04183944009444e-06	\\
-9408.05220170455	8.96573410638228e-07	\\
-9407.07341974432	3.64716344758893e-07	\\
-9406.09463778409	3.77520650545523e-07	\\
-9405.11585582386	1.14535636697698e-06	\\
-9404.13707386364	9.35971935272287e-07	\\
-9403.15829190341	8.02223165678658e-07	\\
-9402.17950994318	1.73938252613073e-06	\\
-9401.20072798295	5.40631720806198e-07	\\
-9400.22194602273	1.31001991104272e-06	\\
-9399.2431640625	3.38867081352699e-07	\\
-9398.26438210227	7.21201809527722e-07	\\
-9397.28560014205	1.0090860793513e-06	\\
-9396.30681818182	1.50353122560658e-06	\\
-9395.32803622159	5.09085949520379e-07	\\
-9394.34925426136	6.56297709865317e-07	\\
-9393.37047230114	6.87601949893929e-07	\\
-9392.39169034091	5.32587325538808e-07	\\
-9391.41290838068	4.7345973308218e-07	\\
-9390.43412642045	6.45904358170057e-07	\\
-9389.45534446023	6.65202091255661e-07	\\
-9388.4765625	6.03508678343688e-07	\\
-9387.49778053977	8.00678561029293e-07	\\
-9386.51899857955	9.34695743956055e-07	\\
-9385.54021661932	1.82661164470884e-06	\\
-9384.56143465909	5.51600926926797e-07	\\
-9383.58265269886	5.55757559896576e-07	\\
-9382.60387073864	5.99849677095614e-07	\\
-9381.62508877841	1.7508616091789e-06	\\
-9380.64630681818	7.9193936230922e-07	\\
-9379.66752485795	9.2423735249465e-07	\\
-9378.68874289773	6.51668534203339e-07	\\
-9377.7099609375	9.92941748949503e-07	\\
-9376.73117897727	3.47755993014192e-07	\\
-9375.75239701705	1.48517974411769e-07	\\
-9374.77361505682	1.10907819888185e-06	\\
-9373.79483309659	1.11134283817704e-06	\\
-9372.81605113636	7.46970000346113e-07	\\
-9371.83726917614	4.54841616743854e-07	\\
-9370.85848721591	6.33622590246485e-07	\\
-9369.87970525568	6.9585148571009e-07	\\
-9368.90092329545	1.08537384032427e-06	\\
-9367.92214133523	5.390843994041e-07	\\
-9366.943359375	2.4462077381245e-07	\\
-9365.96457741477	1.11620400384662e-06	\\
-9364.98579545455	1.63647064011503e-07	\\
-9364.00701349432	1.16655751405632e-06	\\
-9363.02823153409	1.14267535678151e-06	\\
-9362.04944957386	9.14036550649078e-07	\\
-9361.07066761364	3.92507291385461e-07	\\
-9360.09188565341	1.04903400494758e-06	\\
-9359.11310369318	4.09332916712071e-07	\\
-9358.13432173295	7.40723044203781e-07	\\
-9357.15553977273	7.37872568176912e-07	\\
-9356.1767578125	9.53940846975855e-07	\\
-9355.19797585227	1.69718893553471e-06	\\
-9354.21919389205	1.21318367319133e-06	\\
-9353.24041193182	1.38209181899039e-06	\\
-9352.26162997159	8.13083767590717e-07	\\
-9351.28284801136	1.27501438413465e-06	\\
-9350.30406605114	7.68458306208263e-07	\\
-9349.32528409091	1.04456342507218e-06	\\
-9348.34650213068	7.26379417440243e-07	\\
-9347.36772017045	3.45603257521548e-07	\\
-9346.38893821023	8.19349842504997e-07	\\
-9345.41015625	1.10516564263216e-06	\\
-9344.43137428977	1.04353433169235e-06	\\
-9343.45259232955	4.91272254513205e-07	\\
-9342.47381036932	1.45625110830639e-06	\\
-9341.49502840909	6.62113731953244e-07	\\
-9340.51624644886	4.44479674295842e-07	\\
-9339.53746448864	5.58568420411389e-07	\\
-9338.55868252841	4.88620972262281e-07	\\
-9337.57990056818	9.83027018347073e-07	\\
-9336.60111860795	4.10371134314238e-07	\\
-9335.62233664773	3.76158168583449e-07	\\
-9334.6435546875	2.77816181959462e-07	\\
-9333.66477272727	5.20843620926003e-07	\\
-9332.68599076705	6.40744035743505e-07	\\
-9331.70720880682	2.00636582506908e-08	\\
-9330.72842684659	4.18592695312603e-07	\\
-9329.74964488636	5.2421564898287e-07	\\
-9328.77086292614	6.6422065190669e-07	\\
-9327.79208096591	3.37182656335801e-07	\\
-9326.81329900568	5.64620871767836e-07	\\
-9325.83451704545	4.37495099465317e-07	\\
-9324.85573508523	1.73556723467256e-06	\\
-9323.876953125	1.4789121558704e-06	\\
-9322.89817116477	2.34929523930397e-06	\\
-9321.91938920455	5.24893946755281e-07	\\
-9320.94060724432	1.28134293121225e-07	\\
-9319.96182528409	1.99424911009697e-06	\\
-9318.98304332386	9.82289769723645e-07	\\
-9318.00426136364	2.43828340737185e-07	\\
-9317.02547940341	9.74906469458958e-07	\\
-9316.04669744318	2.73335330360526e-07	\\
-9315.06791548295	3.86789672489791e-07	\\
-9314.08913352273	7.40656576530351e-07	\\
-9313.1103515625	9.64898720633358e-07	\\
-9312.13156960227	1.51583399961555e-06	\\
-9311.15278764205	9.42101074117874e-07	\\
-9310.17400568182	9.54730774410535e-07	\\
-9309.19522372159	1.40224572604593e-06	\\
-9308.21644176136	1.51035991934998e-06	\\
-9307.23765980114	4.61042995984844e-07	\\
-9306.25887784091	9.9255640633435e-07	\\
-9305.28009588068	3.4956491231497e-07	\\
-9304.30131392045	4.38198288216869e-07	\\
-9303.32253196023	1.00063146781546e-06	\\
-9302.34375	6.36809716302977e-07	\\
-9301.36496803977	8.39955239315413e-07	\\
-9300.38618607955	1.2857790950678e-06	\\
-9299.40740411932	9.53186881491565e-07	\\
-9298.42862215909	4.36465734160233e-07	\\
-9297.44984019886	9.62883710166045e-07	\\
-9296.47105823864	3.8144553336652e-07	\\
-9295.49227627841	4.74448756150737e-07	\\
-9294.51349431818	8.25054553719788e-07	\\
-9293.53471235795	1.02589874322834e-06	\\
-9292.55593039773	8.54427364242348e-07	\\
-9291.5771484375	4.98411394686139e-07	\\
-9290.59836647727	8.54822445595341e-07	\\
-9289.61958451705	5.53653850715051e-07	\\
-9288.64080255682	8.36909783382818e-07	\\
-9287.66202059659	1.19407496571074e-06	\\
-9286.68323863636	5.01489034938593e-07	\\
-9285.70445667614	1.15685850191773e-06	\\
-9284.72567471591	7.88166087558649e-07	\\
-9283.74689275568	1.68209758058059e-07	\\
-9282.76811079545	6.66196723562332e-07	\\
-9281.78932883523	2.01929972624301e-07	\\
-9280.810546875	6.32105705942981e-07	\\
-9279.83176491477	3.54416641356528e-07	\\
-9278.85298295455	3.09773361100886e-07	\\
-9277.87420099432	2.47403821075025e-07	\\
-9276.89541903409	1.26723626576509e-06	\\
-9275.91663707386	1.63569504008343e-06	\\
-9274.93785511364	1.36110421812059e-06	\\
-9273.95907315341	3.47795175177363e-07	\\
-9272.98029119318	5.33230249863173e-07	\\
-9272.00150923295	4.73968618783679e-07	\\
-9271.02272727273	4.33758774449021e-07	\\
-9270.0439453125	3.39992781905984e-07	\\
-9269.06516335227	5.44617579284376e-07	\\
-9268.08638139205	4.41109231637537e-07	\\
-9267.10759943182	1.0610842977859e-06	\\
-9266.12881747159	4.96052575075334e-07	\\
-9265.15003551136	7.09983458011427e-07	\\
-9264.17125355114	8.46843922950447e-07	\\
-9263.19247159091	3.39399975835859e-07	\\
-9262.21368963068	4.25690878040225e-07	\\
-9261.23490767045	3.28174528498026e-07	\\
-9260.25612571023	1.44824509616788e-06	\\
-9259.27734375	1.82742075650962e-06	\\
-9258.29856178977	7.02952957039027e-07	\\
-9257.31977982955	6.33650222513339e-07	\\
-9256.34099786932	6.22810337677897e-07	\\
-9255.36221590909	5.06915259300238e-07	\\
-9254.38343394886	7.87676841807818e-08	\\
-9253.40465198864	1.68631537934412e-06	\\
-9252.42587002841	6.49589326258631e-07	\\
-9251.44708806818	5.81232072851467e-07	\\
-9250.46830610795	8.44009335354947e-07	\\
-9249.48952414773	3.19994538246584e-07	\\
-9248.5107421875	2.6822396521005e-07	\\
-9247.53196022727	3.62018831386706e-07	\\
-9246.55317826705	1.37684862756222e-06	\\
-9245.57439630682	1.33814774525271e-06	\\
-9244.59561434659	1.31013404965679e-06	\\
-9243.61683238636	1.55761484717326e-07	\\
-9242.63805042614	5.04255061415902e-07	\\
-9241.65926846591	8.34694559109992e-07	\\
-9240.68048650568	6.77721876323977e-07	\\
-9239.70170454545	4.12400583664712e-07	\\
-9238.72292258523	4.09138640001689e-07	\\
-9237.744140625	9.66912241549525e-07	\\
-9236.76535866477	9.29335555554594e-07	\\
-9235.78657670455	1.33901489715682e-06	\\
-9234.80779474432	1.11955987654573e-06	\\
-9233.82901278409	7.14930054792279e-07	\\
-9232.85023082386	1.13766570950342e-06	\\
-9231.87144886364	1.75964492463003e-06	\\
-9230.89266690341	9.46939982387482e-07	\\
-9229.91388494318	6.17755722759856e-07	\\
-9228.93510298295	4.97227887400041e-07	\\
-9227.95632102273	1.1326955866698e-06	\\
-9226.9775390625	1.33096635798897e-06	\\
-9225.99875710227	1.06114769728407e-06	\\
-9225.01997514205	1.32847331333223e-06	\\
-9224.04119318182	1.01748198416591e-06	\\
-9223.06241122159	7.81266307551434e-07	\\
-9222.08362926136	7.04102413560418e-07	\\
-9221.10484730114	3.87407801800029e-07	\\
-9220.12606534091	1.33248585621518e-06	\\
-9219.14728338068	8.59907058166526e-07	\\
-9218.16850142045	4.4026389430923e-07	\\
-9217.18971946023	1.03119933508935e-06	\\
-9216.2109375	1.55678086422852e-06	\\
-9215.23215553977	2.06568815282174e-06	\\
-9214.25337357955	4.68704838841551e-07	\\
-9213.27459161932	1.61611736714171e-06	\\
-9212.29580965909	1.77229636906379e-06	\\
-9211.31702769886	8.56009714877834e-07	\\
-9210.33824573864	1.09829390248981e-06	\\
-9209.35946377841	8.77759041168074e-07	\\
-9208.38068181818	2.84989986283952e-07	\\
-9207.40189985795	1.89632873697649e-07	\\
-9206.42311789773	4.69592329652699e-07	\\
-9205.4443359375	8.97059554828987e-07	\\
-9204.46555397727	5.90973171744822e-07	\\
-9203.48677201705	6.87058023720824e-07	\\
-9202.50799005682	1.46726283888877e-06	\\
-9201.52920809659	1.52183622462631e-06	\\
-9200.55042613636	7.67970713843733e-07	\\
-9199.57164417614	1.53725388241587e-06	\\
-9198.59286221591	6.97189693197062e-07	\\
-9197.61408025568	6.93063610095174e-07	\\
-9196.63529829545	1.05704460785399e-06	\\
-9195.65651633523	7.85335210379425e-07	\\
-9194.677734375	4.69892454538563e-07	\\
-9193.69895241477	1.13509834536489e-06	\\
-9192.72017045455	1.18411429524331e-06	\\
-9191.74138849432	1.43104671915632e-06	\\
-9190.76260653409	1.81673247471318e-06	\\
-9189.78382457386	1.39872163106183e-06	\\
-9188.80504261364	7.94737113267989e-07	\\
-9187.82626065341	3.58988663535729e-07	\\
-9186.84747869318	6.48908519866124e-08	\\
-9185.86869673295	1.03000766973058e-06	\\
-9184.88991477273	8.02022215231596e-07	\\
-9183.9111328125	7.52051940114893e-07	\\
-9182.93235085227	9.26420618548275e-07	\\
-9181.95356889205	1.06675367147427e-06	\\
-9180.97478693182	1.29254565984508e-06	\\
-9179.99600497159	8.76374990263791e-07	\\
-9179.01722301136	1.73219334194377e-07	\\
-9178.03844105114	1.5615739964684e-06	\\
-9177.05965909091	1.84529836173547e-06	\\
-9176.08087713068	4.70555041240439e-07	\\
-9175.10209517045	7.50201779533432e-07	\\
-9174.12331321023	7.05842475450723e-07	\\
-9173.14453125	7.26261712421764e-07	\\
-9172.16574928977	3.02687288366555e-07	\\
-9171.18696732955	2.59810583017099e-07	\\
-9170.20818536932	1.21395355457095e-06	\\
-9169.22940340909	2.02068966884092e-06	\\
-9168.25062144886	1.77042471511417e-06	\\
-9167.27183948864	1.54527694259899e-07	\\
-9166.29305752841	3.28271395303942e-07	\\
-9165.31427556818	2.11679701172265e-06	\\
-9164.33549360795	4.16893118041511e-07	\\
-9163.35671164773	1.025701905352e-06	\\
-9162.3779296875	1.06998625877004e-06	\\
-9161.39914772727	1.03290902245733e-07	\\
-9160.42036576705	6.16194501321901e-07	\\
-9159.44158380682	2.1176261818624e-07	\\
-9158.46280184659	1.43986924054267e-06	\\
-9157.48401988636	7.46816698823038e-07	\\
-9156.50523792614	1.0875646943965e-06	\\
-9155.52645596591	1.78012161184405e-07	\\
-9154.54767400568	1.54076784319953e-06	\\
-9153.56889204545	9.52002705352945e-07	\\
-9152.59011008523	9.55987646079004e-07	\\
-9151.611328125	1.03003406438844e-06	\\
-9150.63254616477	9.04194936637502e-07	\\
-9149.65376420455	7.718706914153e-07	\\
-9148.67498224432	6.60209047344211e-07	\\
-9147.69620028409	1.51187109142664e-06	\\
-9146.71741832386	3.91398886126464e-07	\\
-9145.73863636364	9.23492727821801e-07	\\
-9144.75985440341	6.35993571098178e-07	\\
-9143.78107244318	8.71556641034821e-07	\\
-9142.80229048295	1.14792248995753e-06	\\
-9141.82350852273	1.50993694968181e-06	\\
-9140.8447265625	1.76925865908812e-06	\\
-9139.86594460227	1.09911746441098e-06	\\
-9138.88716264205	2.54855800753059e-07	\\
-9137.90838068182	3.68283007085571e-07	\\
-9136.92959872159	1.4216825485453e-06	\\
-9135.95081676136	1.0175068914972e-06	\\
-9134.97203480114	4.73053321392716e-07	\\
-9133.99325284091	8.26713542400407e-07	\\
-9133.01447088068	1.92042662713406e-06	\\
-9132.03568892045	7.71533871589721e-07	\\
-9131.05690696023	1.70069146802068e-06	\\
-9130.078125	1.50860989681033e-06	\\
-9129.09934303977	1.40512858063301e-06	\\
-9128.12056107955	1.47334241002783e-06	\\
-9127.14177911932	8.97941670624931e-07	\\
-9126.16299715909	1.22916616616696e-06	\\
-9125.18421519886	9.70868533327948e-07	\\
-9124.20543323864	6.34246252828328e-07	\\
-9123.22665127841	1.09555983375927e-06	\\
-9122.24786931818	1.65217275860146e-06	\\
-9121.26908735795	2.056344523441e-06	\\
-9120.29030539773	6.36445037643075e-07	\\
-9119.3115234375	1.22269927719994e-06	\\
-9118.33274147727	7.66077220505724e-07	\\
-9117.35395951705	1.24571608899988e-06	\\
-9116.37517755682	8.35480915727792e-07	\\
-9115.39639559659	1.71016537491295e-06	\\
-9114.41761363636	1.60027394115551e-06	\\
-9113.43883167614	1.10790930494926e-06	\\
-9112.46004971591	1.29270223933388e-06	\\
-9111.48126775568	1.60305135826334e-06	\\
-9110.50248579545	7.47667805028555e-07	\\
-9109.52370383523	5.08870544153114e-07	\\
-9108.544921875	1.03032716655128e-06	\\
-9107.56613991477	8.27159731264097e-07	\\
-9106.58735795455	7.68665142031946e-07	\\
-9105.60857599432	1.53484376644953e-06	\\
-9104.62979403409	5.92559959913776e-07	\\
-9103.65101207386	5.64621262125653e-07	\\
-9102.67223011364	1.26064707734842e-06	\\
-9101.69344815341	7.96859670629279e-07	\\
-9100.71466619318	1.96769416009767e-06	\\
-9099.73588423295	9.70368707482217e-07	\\
-9098.75710227273	6.45278193920873e-07	\\
-9097.7783203125	9.41081101950708e-07	\\
-9096.79953835227	5.58770168482422e-07	\\
-9095.82075639205	1.71882205420706e-06	\\
-9094.84197443182	1.25693316135585e-06	\\
-9093.86319247159	1.47854697488012e-06	\\
-9092.88441051136	1.88385075535555e-06	\\
-9091.90562855114	6.31772791088688e-07	\\
-9090.92684659091	7.99889808206917e-07	\\
-9089.94806463068	1.86108342654507e-06	\\
-9088.96928267045	1.49041416725719e-06	\\
-9087.99050071023	1.18305873161161e-06	\\
-9087.01171875	1.01236506721086e-06	\\
-9086.03293678977	2.26634818022439e-07	\\
-9085.05415482955	1.38629138082078e-06	\\
-9084.07537286932	2.94519992623528e-07	\\
-9083.09659090909	2.04784155611674e-06	\\
-9082.11780894886	9.78975013841631e-07	\\
-9081.13902698864	1.21517037700969e-06	\\
-9080.16024502841	1.00886202859803e-06	\\
-9079.18146306818	9.50470847728033e-07	\\
-9078.20268110795	2.16334267885535e-06	\\
-9077.22389914773	1.9193342741854e-06	\\
-9076.2451171875	1.26395626464931e-06	\\
-9075.26633522727	1.48474215260783e-06	\\
-9074.28755326705	1.38703412023419e-06	\\
-9073.30877130682	1.65675852859406e-06	\\
-9072.32998934659	1.34331155789774e-06	\\
-9071.35120738636	1.8858517048757e-06	\\
-9070.37242542614	1.04888868131937e-06	\\
-9069.39364346591	1.38877984645899e-06	\\
-9068.41486150568	1.73949409820695e-06	\\
-9067.43607954545	1.30923070594408e-06	\\
-9066.45729758523	1.35984679810746e-06	\\
-9065.478515625	1.18763251092793e-06	\\
-9064.49973366477	8.56531186160044e-07	\\
-9063.52095170455	1.27668776486516e-06	\\
-9062.54216974432	1.09844151704934e-06	\\
-9061.56338778409	1.56159960417566e-06	\\
-9060.58460582386	2.05624428034158e-06	\\
-9059.60582386364	9.57214294656849e-07	\\
-9058.62704190341	1.1706776955583e-06	\\
-9057.64825994318	1.8223052140211e-06	\\
-9056.66947798295	7.14300858662457e-07	\\
-9055.69069602273	2.39046312805843e-06	\\
-9054.7119140625	2.05885238982237e-06	\\
-9053.73313210227	8.22080902937566e-07	\\
-9052.75435014205	1.15572806825149e-06	\\
-9051.77556818182	1.03074843494387e-06	\\
-9050.79678622159	5.58908752625064e-07	\\
-9049.81800426136	1.24002202359597e-06	\\
-9048.83922230114	1.32426371441466e-06	\\
-9047.86044034091	1.4582996458828e-06	\\
-9046.88165838068	1.75876490579964e-06	\\
-9045.90287642045	3.2596439361205e-07	\\
-9044.92409446023	2.18623769162331e-06	\\
-9043.9453125	8.00024416522668e-07	\\
-9042.96653053977	1.31522588574906e-06	\\
-9041.98774857955	1.30785529673424e-06	\\
-9041.00896661932	3.29455953214679e-07	\\
-9040.03018465909	1.49644273627409e-06	\\
-9039.05140269886	1.73374528104319e-06	\\
-9038.07262073864	5.65213260446919e-07	\\
-9037.09383877841	6.76422129328237e-07	\\
-9036.11505681818	7.6085698367954e-07	\\
-9035.13627485795	5.56706195959081e-07	\\
-9034.15749289773	1.36611496543957e-06	\\
-9033.1787109375	4.04316945297541e-07	\\
-9032.19992897727	8.94147112374634e-07	\\
-9031.22114701705	2.01604689256199e-06	\\
-9030.24236505682	1.94225183884979e-06	\\
-9029.26358309659	1.14116091419598e-06	\\
-9028.28480113636	6.0508190335382e-07	\\
-9027.30601917614	2.56160016036411e-06	\\
-9026.32723721591	1.09915478214168e-06	\\
-9025.34845525568	3.79344531601801e-07	\\
-9024.36967329545	6.83016914923635e-07	\\
-9023.39089133523	1.89902449906691e-06	\\
-9022.412109375	1.41690501227685e-06	\\
-9021.43332741477	1.45619395138481e-06	\\
-9020.45454545455	1.01019643922589e-06	\\
-9019.47576349432	2.79439040214657e-06	\\
-9018.49698153409	1.38964862306233e-06	\\
-9017.51819957386	1.23759362954841e-06	\\
-9016.53941761364	1.84229959474683e-06	\\
-9015.56063565341	2.22317482906632e-06	\\
-9014.58185369318	1.80187461485671e-06	\\
-9013.60307173295	8.0849057497255e-07	\\
-9012.62428977273	9.30850130537497e-07	\\
-9011.6455078125	4.8272073074455e-07	\\
-9010.66672585227	1.62260315934751e-06	\\
-9009.68794389205	8.54753117364289e-07	\\
-9008.70916193182	1.60400071805361e-06	\\
-9007.73037997159	1.481899255764e-06	\\
-9006.75159801136	1.91222895513168e-06	\\
-9005.77281605114	2.04653991626266e-06	\\
-9004.79403409091	1.70923884235473e-06	\\
-9003.81525213068	2.12822916444477e-06	\\
-9002.83647017045	1.37771389978565e-06	\\
-9001.85768821023	1.78383354428274e-06	\\
-9000.87890625	1.99950261867047e-06	\\
-8999.90012428977	3.61968388493234e-06	\\
-8998.92134232955	2.85455261884322e-06	\\
-8997.94256036932	1.56418905517721e-06	\\
-8996.96377840909	2.78364774944937e-06	\\
-8995.98499644886	8.18896036525753e-07	\\
-8995.00621448864	1.43733451668579e-06	\\
-8994.02743252841	2.73159848629123e-06	\\
-8993.04865056818	2.1521630400874e-06	\\
-8992.06986860795	1.77453587662718e-06	\\
-8991.09108664773	1.89384102680694e-06	\\
-8990.1123046875	1.16339089257023e-06	\\
-8989.13352272727	2.72461293228184e-06	\\
-8988.15474076705	1.11860539044014e-06	\\
-8987.17595880682	1.32565793120889e-06	\\
-8986.19717684659	2.09918304212716e-06	\\
-8985.21839488636	9.76319019853391e-07	\\
-8984.23961292614	1.23698104198297e-06	\\
-8983.26083096591	2.91020252177098e-06	\\
-8982.28204900568	2.56254194375569e-06	\\
-8981.30326704545	2.29608917160771e-06	\\
-8980.32448508523	2.07125907042997e-06	\\
-8979.345703125	3.35885785694376e-06	\\
-8978.36692116477	3.16416591594629e-06	\\
-8977.38813920455	2.41286205367693e-06	\\
-8976.40935724432	2.04155216763524e-06	\\
-8975.43057528409	2.11095970164975e-06	\\
-8974.45179332386	2.26506965658796e-06	\\
-8973.47301136364	1.00693880876899e-06	\\
-8972.49422940341	1.56802996186119e-06	\\
-8971.51544744318	2.0888605556908e-06	\\
-8970.53666548295	1.75095600211994e-06	\\
-8969.55788352273	2.33833874754307e-06	\\
-8968.5791015625	2.3573055846851e-06	\\
-8967.60031960227	9.13399819865814e-07	\\
-8966.62153764205	3.05063833723406e-06	\\
-8965.64275568182	2.01044060959087e-06	\\
-8964.66397372159	1.59759791510215e-06	\\
-8963.68519176136	3.01446701016822e-06	\\
-8962.70640980114	3.93022071591879e-06	\\
-8961.72762784091	2.48187312675573e-06	\\
-8960.74884588068	2.69216188113444e-06	\\
-8959.77006392045	2.91918302277644e-06	\\
-8958.79128196023	1.87878262887084e-06	\\
-8957.8125	2.07615353562826e-06	\\
-8956.83371803977	1.55782977082778e-06	\\
-8955.85493607955	2.85535950753474e-06	\\
-8954.87615411932	3.13437351260756e-06	\\
-8953.89737215909	1.92423193612808e-06	\\
-8952.91859019886	2.09583772902445e-06	\\
-8951.93980823864	2.22316539972519e-06	\\
-8950.96102627841	7.75221675606493e-07	\\
-8949.98224431818	3.58791358251981e-06	\\
-8949.00346235795	1.26435953746364e-06	\\
-8948.02468039773	1.21383426058361e-06	\\
-8947.0458984375	2.09435087646751e-06	\\
-8946.06711647727	1.07448036458306e-06	\\
-8945.08833451705	1.87274930001004e-06	\\
-8944.10955255682	1.04017040773862e-06	\\
-8943.13077059659	2.49736304923193e-06	\\
-8942.15198863636	2.31391044260718e-06	\\
-8941.17320667614	1.39521643206244e-07	\\
-8940.19442471591	1.75442478245662e-06	\\
-8939.21564275568	4.08700227161784e-07	\\
-8938.23686079545	1.84550289917676e-06	\\
-8937.25807883523	1.91721577533782e-06	\\
-8936.279296875	9.84178691777908e-07	\\
-8935.30051491477	2.0760470129334e-06	\\
-8934.32173295455	8.19202710472283e-07	\\
-8933.34295099432	1.95070027131913e-06	\\
-8932.36416903409	3.07601740862584e-06	\\
-8931.38538707386	1.77167644615773e-06	\\
-8930.40660511364	1.34550543150866e-07	\\
-8929.42782315341	3.20425864002611e-06	\\
-8928.44904119318	1.50227283697752e-06	\\
-8927.47025923295	2.63831419353374e-06	\\
-8926.49147727273	2.91306993462244e-06	\\
-8925.5126953125	2.05034060881722e-06	\\
-8924.53391335227	2.0954939838871e-06	\\
-8923.55513139205	2.44700856030521e-06	\\
-8922.57634943182	1.55049866397578e-06	\\
-8921.59756747159	1.71495047707442e-06	\\
-8920.61878551136	3.30924004763649e-06	\\
-8919.64000355114	1.4007751698396e-06	\\
-8918.66122159091	1.52139011141659e-06	\\
-8917.68243963068	2.91810261814323e-06	\\
-8916.70365767045	6.8384180537982e-07	\\
-8915.72487571023	4.05229245263935e-06	\\
-8914.74609375	1.72732026662955e-06	\\
-8913.76731178977	1.42638345937911e-06	\\
-8912.78852982955	1.55776233463834e-06	\\
-8911.80974786932	1.53733341069004e-06	\\
-8910.83096590909	2.7479275886799e-06	\\
-8909.85218394886	1.78314144587565e-06	\\
-8908.87340198864	1.33952308103243e-06	\\
-8907.89462002841	1.14398463232444e-06	\\
-8906.91583806818	9.37543095801631e-07	\\
-8905.93705610795	1.0908334036817e-06	\\
-8904.95827414773	2.16946081066233e-06	\\
-8903.9794921875	1.13499318708122e-06	\\
-8903.00071022727	1.7136433811069e-06	\\
-8902.02192826705	1.38063353678966e-06	\\
-8901.04314630682	2.77104291395959e-06	\\
-8900.06436434659	4.05642169178121e-06	\\
-8899.08558238636	2.12225794350329e-06	\\
-8898.10680042614	1.89103123225885e-06	\\
-8897.12801846591	3.07797461823195e-06	\\
-8896.14923650568	7.38288918596365e-07	\\
-8895.17045454545	2.72320502321358e-06	\\
-8894.19167258523	2.38533444780504e-06	\\
-8893.212890625	2.23577295876895e-06	\\
-8892.23410866477	1.8148573074628e-06	\\
-8891.25532670455	2.52951630199236e-06	\\
-8890.27654474432	3.2582332449953e-06	\\
-8889.29776278409	1.88277924056618e-06	\\
-8888.31898082386	2.23726363230009e-06	\\
-8887.34019886364	1.8591169763829e-06	\\
-8886.36141690341	1.44266093099942e-06	\\
-8885.38263494318	2.27833189337566e-06	\\
-8884.40385298295	3.76450810234757e-06	\\
-8883.42507102273	2.0979731256236e-06	\\
-8882.4462890625	2.74195500513601e-06	\\
-8881.46750710227	2.79452874505e-06	\\
-8880.48872514205	3.66410058455237e-06	\\
-8879.50994318182	1.90295376001105e-06	\\
-8878.53116122159	3.22892283191207e-06	\\
-8877.55237926136	1.46029740614205e-06	\\
-8876.57359730114	2.15232118092061e-06	\\
-8875.59481534091	3.06015623245974e-06	\\
-8874.61603338068	1.81962810426555e-06	\\
-8873.63725142045	2.1591640945514e-06	\\
-8872.65846946023	2.15511681406188e-06	\\
-8871.6796875	2.07135303936208e-06	\\
-8870.70090553977	2.34826671596026e-06	\\
-8869.72212357955	2.47187085756481e-06	\\
-8868.74334161932	1.97575277531478e-06	\\
-8867.76455965909	2.7543907648003e-06	\\
-8866.78577769886	1.61911808679879e-06	\\
-8865.80699573864	1.61339433000953e-06	\\
-8864.82821377841	2.25679636095849e-06	\\
-8863.84943181818	2.60050149504162e-06	\\
-8862.87064985795	2.36117900226651e-06	\\
-8861.89186789773	1.97014436205896e-06	\\
-8860.9130859375	1.26427731486776e-06	\\
-8859.93430397727	3.1107851255329e-06	\\
-8858.95552201705	2.66573679078238e-06	\\
-8857.97674005682	1.04551999625788e-06	\\
-8856.99795809659	1.85650534128506e-06	\\
-8856.01917613636	1.97679201948274e-06	\\
-8855.04039417614	1.43514372983442e-06	\\
-8854.06161221591	1.99882167748441e-06	\\
-8853.08283025568	2.09453530723238e-06	\\
-8852.10404829545	1.94114443301059e-06	\\
-8851.12526633523	2.23120538379971e-06	\\
-8850.146484375	2.70636119805073e-06	\\
-8849.16770241477	1.96113228378063e-06	\\
-8848.18892045455	1.34511154181103e-06	\\
-8847.21013849432	2.32928793083313e-06	\\
-8846.23135653409	2.65572333003283e-06	\\
-8845.25257457386	2.69362576746287e-06	\\
-8844.27379261364	1.74897874868165e-06	\\
-8843.29501065341	4.04814291264881e-06	\\
-8842.31622869318	3.0276025193212e-06	\\
-8841.33744673295	2.31878346397848e-06	\\
-8840.35866477273	1.18852732585131e-06	\\
-8839.3798828125	1.2141623832896e-06	\\
-8838.40110085227	2.37700484272566e-06	\\
-8837.42231889205	1.39530932175276e-06	\\
-8836.44353693182	2.34555967163408e-06	\\
-8835.46475497159	3.67918441857658e-06	\\
-8834.48597301136	3.23346063142948e-06	\\
-8833.50719105114	3.14222462246649e-06	\\
-8832.52840909091	1.8720994999813e-06	\\
-8831.54962713068	2.44108651036182e-06	\\
-8830.57084517045	3.80054260699873e-06	\\
-8829.59206321023	2.14652422842305e-06	\\
-8828.61328125	3.59573454556529e-06	\\
-8827.63449928977	2.95034232706139e-06	\\
-8826.65571732955	1.00286572000818e-06	\\
-8825.67693536932	2.34525191165373e-06	\\
-8824.69815340909	2.72907135903645e-06	\\
-8823.71937144886	2.72108091751827e-06	\\
-8822.74058948864	2.29113055180151e-06	\\
-8821.76180752841	1.36989152137488e-06	\\
-8820.78302556818	2.9013719632796e-06	\\
-8819.80424360795	1.67209272789278e-06	\\
-8818.82546164773	2.82101720132502e-06	\\
-8817.8466796875	4.33077808187654e-06	\\
-8816.86789772727	1.96828472858691e-06	\\
-8815.88911576705	2.28296659059787e-06	\\
-8814.91033380682	2.71296018208186e-06	\\
-8813.93155184659	2.05847655423689e-06	\\
-8812.95276988636	2.4615192657209e-06	\\
-8811.97398792614	2.98577239704341e-06	\\
-8810.99520596591	4.12412287592922e-06	\\
-8810.01642400568	3.28595630479861e-06	\\
-8809.03764204545	2.74853874560878e-06	\\
-8808.05886008523	3.06903134217629e-06	\\
-8807.080078125	2.62095864964104e-06	\\
-8806.10129616477	3.75181182757445e-06	\\
-8805.12251420455	2.6808117963073e-06	\\
-8804.14373224432	2.45265058796585e-06	\\
-8803.16495028409	2.16996278161085e-06	\\
-8802.18616832386	2.69452927517449e-06	\\
-8801.20738636364	3.04815161246522e-06	\\
-8800.22860440341	3.34101137139337e-06	\\
-8799.24982244318	3.03717722307558e-06	\\
-8798.27104048295	2.40587049241343e-06	\\
-8797.29225852273	1.48948858265697e-06	\\
-8796.3134765625	2.60601268486776e-06	\\
-8795.33469460227	2.42460568038357e-06	\\
-8794.35591264205	2.83731814644813e-06	\\
-8793.37713068182	2.10340118894317e-06	\\
-8792.39834872159	2.07115346116652e-06	\\
-8791.41956676136	2.53377693158204e-06	\\
-8790.44078480114	1.18237758274433e-06	\\
-8789.46200284091	2.37020560930862e-06	\\
-8788.48322088068	3.26391116000665e-06	\\
-8787.50443892045	1.73447474892679e-06	\\
-8786.52565696023	2.43942778709396e-06	\\
-8785.546875	2.96574671783003e-06	\\
-8784.56809303977	2.286617556116e-06	\\
-8783.58931107955	2.25200342196874e-06	\\
-8782.61052911932	2.16114928248351e-06	\\
-8781.63174715909	2.9660119547303e-06	\\
-8780.65296519886	2.68212964856617e-06	\\
-8779.67418323864	1.43009428381599e-06	\\
-8778.69540127841	2.66455906180417e-06	\\
-8777.71661931818	1.45374354678058e-06	\\
-8776.73783735795	3.45519194863692e-06	\\
-8775.75905539773	2.99917504169805e-06	\\
-8774.7802734375	3.1224020170324e-06	\\
-8773.80149147727	2.96546876780736e-06	\\
-8772.82270951705	2.62803562737369e-06	\\
-8771.84392755682	3.08089169908627e-06	\\
-8770.86514559659	3.72461117536113e-06	\\
-8769.88636363636	3.22278451358341e-06	\\
-8768.90758167614	2.98530169243893e-06	\\
-8767.92879971591	2.58415208596026e-06	\\
-8766.95001775568	2.6214185056095e-06	\\
-8765.97123579545	2.59373299222188e-06	\\
-8764.99245383523	2.69362128586215e-06	\\
-8764.013671875	3.32551083495277e-06	\\
-8763.03488991477	1.84163331569974e-06	\\
-8762.05610795455	2.00684143503431e-06	\\
-8761.07732599432	4.3989538300432e-06	\\
-8760.09854403409	3.93965352450115e-06	\\
-8759.11976207386	2.34560177800176e-06	\\
-8758.14098011364	3.13873809450203e-06	\\
-8757.16219815341	2.26474464883409e-06	\\
-8756.18341619318	2.37655507679856e-06	\\
-8755.20463423295	2.04285467215576e-06	\\
-8754.22585227273	3.13652799161212e-06	\\
-8753.2470703125	2.44739622058649e-06	\\
-8752.26828835227	2.36068564566267e-06	\\
-8751.28950639205	2.6278163600248e-06	\\
-8750.31072443182	4.39855941203281e-07	\\
-8749.33194247159	3.50494856775953e-06	\\
-8748.35316051136	1.98020848552783e-06	\\
-8747.37437855114	3.65223851306969e-06	\\
-8746.39559659091	2.01984985118422e-06	\\
-8745.41681463068	1.90735935131734e-06	\\
-8744.43803267045	3.03480942785112e-06	\\
-8743.45925071023	2.33715795696452e-06	\\
-8742.48046875	2.55724205425525e-06	\\
-8741.50168678977	2.9344563602255e-06	\\
-8740.52290482955	2.48839854393014e-06	\\
-8739.54412286932	2.08129408383322e-06	\\
-8738.56534090909	2.11606409128068e-06	\\
-8737.58655894886	1.32781061859003e-06	\\
-8736.60777698864	3.1462545120627e-06	\\
-8735.62899502841	3.56118891121807e-06	\\
-8734.65021306818	2.78443761779356e-06	\\
-8733.67143110795	1.31390533577988e-06	\\
-8732.69264914773	2.39959693813504e-06	\\
-8731.7138671875	3.18728174127853e-06	\\
-8730.73508522727	2.40260361345862e-06	\\
-8729.75630326705	3.70245468932601e-06	\\
-8728.77752130682	2.78572634403114e-06	\\
-8727.79873934659	2.47437953741367e-06	\\
-8726.81995738636	1.99766350450912e-06	\\
-8725.84117542614	3.40586594823726e-06	\\
-8724.86239346591	3.42031001044256e-06	\\
-8723.88361150568	2.01357228227019e-06	\\
-8722.90482954545	2.28590877211032e-06	\\
-8721.92604758523	2.87750930836962e-06	\\
-8720.947265625	2.31227438015978e-06	\\
-8719.96848366477	3.14639964341097e-06	\\
-8718.98970170455	3.24733743132851e-06	\\
-8718.01091974432	1.13990905436083e-06	\\
-8717.03213778409	4.05582528343673e-06	\\
-8716.05335582386	1.6100101286333e-06	\\
-8715.07457386364	2.73121666263706e-06	\\
-8714.09579190341	2.84432873513897e-06	\\
-8713.11700994318	2.67504420011627e-06	\\
-8712.13822798295	3.39740230448375e-06	\\
-8711.15944602273	2.62494080873288e-06	\\
-8710.1806640625	3.00986830639377e-06	\\
-8709.20188210227	2.42224864029916e-06	\\
-8708.22310014205	3.06009588353424e-06	\\
-8707.24431818182	2.99166379439728e-06	\\
-8706.26553622159	1.86678606969537e-06	\\
-8705.28675426136	1.89889214472334e-06	\\
-8704.30797230114	1.21976262367136e-06	\\
-8703.32919034091	1.67805675585822e-06	\\
-8702.35040838068	3.60565079855287e-06	\\
-8701.37162642045	2.8645033645048e-06	\\
-8700.39284446023	3.68944423738639e-06	\\
-8699.4140625	2.26193080195523e-06	\\
-8698.43528053977	3.07550529445477e-06	\\
-8697.45649857955	2.01214244822957e-06	\\
-8696.47771661932	2.39486608798948e-06	\\
-8695.49893465909	2.45347336882115e-06	\\
-8694.52015269886	2.31254269898072e-06	\\
-8693.54137073864	2.77894580645474e-06	\\
-8692.56258877841	1.74697971271583e-06	\\
-8691.58380681818	2.97490703133594e-06	\\
-8690.60502485795	3.17971392170209e-06	\\
-8689.62624289773	2.34395811705805e-06	\\
-8688.6474609375	1.24106515651506e-06	\\
-8687.66867897727	3.48049175995858e-06	\\
-8686.68989701705	3.50610465611466e-06	\\
-8685.71111505682	2.63669627400646e-06	\\
-8684.73233309659	3.18576894487291e-06	\\
-8683.75355113636	3.07419665041785e-06	\\
-8682.77476917614	2.05613823687218e-06	\\
-8681.79598721591	2.49679043481926e-06	\\
-8680.81720525568	2.58360711790541e-06	\\
-8679.83842329545	2.4966678792265e-06	\\
-8678.85964133523	2.24359944090606e-06	\\
-8677.880859375	2.67398215817645e-06	\\
-8676.90207741477	3.83144371511104e-06	\\
-8675.92329545455	2.73822387076835e-06	\\
-8674.94451349432	3.12972792072986e-06	\\
-8673.96573153409	3.27384449011975e-06	\\
-8672.98694957386	2.07247586291306e-06	\\
-8672.00816761364	2.81452160124163e-06	\\
-8671.02938565341	2.44976935583511e-06	\\
-8670.05060369318	3.15772910575126e-06	\\
-8669.07182173295	2.57991036398732e-06	\\
-8668.09303977273	2.55066683273874e-06	\\
-8667.1142578125	2.26660714232643e-06	\\
-8666.13547585227	2.89365191241982e-06	\\
-8665.15669389205	2.93173085757296e-06	\\
-8664.17791193182	3.09977042294155e-06	\\
-8663.19912997159	2.56029766218544e-06	\\
-8662.22034801136	2.24672937750524e-06	\\
-8661.24156605114	1.62020492069873e-06	\\
-8660.26278409091	2.48335243557367e-06	\\
-8659.28400213068	2.59552482185387e-06	\\
-8658.30522017045	8.22159070470364e-07	\\
-8657.32643821023	3.11889039798168e-06	\\
-8656.34765625	2.02914690698014e-06	\\
-8655.36887428977	2.13002433709377e-06	\\
-8654.39009232955	3.13756403820467e-06	\\
-8653.41131036932	2.99439896452848e-06	\\
-8652.43252840909	4.42669663721081e-06	\\
-8651.45374644886	2.16292573074131e-06	\\
-8650.47496448864	2.8373054945434e-06	\\
-8649.49618252841	2.3877563412989e-06	\\
-8648.51740056818	4.21578232273827e-06	\\
-8647.53861860795	3.45747288509824e-06	\\
-8646.55983664773	2.86277442015884e-06	\\
-8645.5810546875	3.09769627229872e-06	\\
-8644.60227272727	2.42900961985154e-06	\\
-8643.62349076705	1.5773430957681e-06	\\
-8642.64470880682	2.86052149540327e-06	\\
-8641.66592684659	2.86321402328804e-06	\\
-8640.68714488636	2.36108018195528e-06	\\
-8639.70836292614	2.37400192414492e-06	\\
-8638.72958096591	3.82970624660038e-06	\\
-8637.75079900568	2.05377631664665e-06	\\
-8636.77201704545	3.90484157194444e-06	\\
-8635.79323508523	2.80664704261669e-06	\\
-8634.814453125	2.73585283645849e-06	\\
-8633.83567116477	2.66889707303301e-06	\\
-8632.85688920455	3.9927999358026e-06	\\
-8631.87810724432	2.49639934384753e-06	\\
-8630.89932528409	3.56248707536917e-06	\\
-8629.92054332386	3.41176225367732e-06	\\
-8628.94176136364	2.73681013967799e-06	\\
-8627.96297940341	1.47939043899555e-06	\\
-8626.98419744318	1.80195375432261e-06	\\
-8626.00541548295	2.28430798330806e-06	\\
-8625.02663352273	3.59684753022792e-06	\\
-8624.0478515625	1.66665273984968e-06	\\
-8623.06906960227	4.11049093180834e-06	\\
-8622.09028764205	3.28602307978567e-06	\\
-8621.11150568182	1.77021669807264e-06	\\
-8620.13272372159	3.43741570532331e-06	\\
-8619.15394176136	2.64267136584899e-06	\\
-8618.17515980114	3.74823460838237e-06	\\
-8617.19637784091	2.09593072058824e-06	\\
-8616.21759588068	3.12585818398557e-06	\\
-8615.23881392045	3.47167396240222e-06	\\
-8614.26003196023	1.27259265427105e-06	\\
-8613.28125	3.65392424499478e-06	\\
-8612.30246803977	2.95750755549884e-06	\\
-8611.32368607955	3.31667110131612e-06	\\
-8610.34490411932	2.93107215490785e-06	\\
-8609.36612215909	2.79902019719089e-06	\\
-8608.38734019886	2.32381052313553e-06	\\
-8607.40855823864	2.18837883667202e-06	\\
-8606.42977627841	2.55742423050093e-06	\\
-8605.45099431818	3.29326977094098e-06	\\
-8604.47221235795	2.4216297989274e-06	\\
-8603.49343039773	3.94708399453755e-06	\\
-8602.5146484375	2.87897318638937e-06	\\
-8601.53586647727	3.04513262713238e-06	\\
-8600.55708451705	2.93489239577708e-06	\\
-8599.57830255682	3.31211644809859e-06	\\
-8598.59952059659	1.7071913807e-06	\\
-8597.62073863636	2.76266343310021e-06	\\
-8596.64195667614	2.79802184845313e-06	\\
-8595.66317471591	4.96045631420238e-06	\\
-8594.68439275568	3.27812104240706e-06	\\
-8593.70561079545	2.28214917617834e-06	\\
-8592.72682883523	3.70161626284122e-06	\\
-8591.748046875	3.9861739700561e-06	\\
-8590.76926491477	3.00239140043313e-06	\\
-8589.79048295455	3.39689025096771e-06	\\
-8588.81170099432	3.28889218886515e-06	\\
-8587.83291903409	2.64461383619663e-06	\\
-8586.85413707386	2.95938859284197e-06	\\
-8585.87535511364	3.10225523919898e-06	\\
-8584.89657315341	3.52119880959946e-06	\\
-8583.91779119318	2.76262726876622e-06	\\
-8582.93900923295	3.66666993308931e-06	\\
-8581.96022727273	2.33762875242731e-06	\\
-8580.9814453125	4.05611994452719e-06	\\
-8580.00266335227	4.23792432624831e-06	\\
-8579.02388139205	3.00365611816968e-06	\\
-8578.04509943182	3.98304050465736e-06	\\
-8577.06631747159	3.0659997555447e-06	\\
-8576.08753551136	2.76738650945508e-06	\\
-8575.10875355114	3.55167503806865e-06	\\
-8574.12997159091	3.08465812175217e-06	\\
-8573.15118963068	2.94200497530058e-06	\\
-8572.17240767045	3.48348654613146e-06	\\
-8571.19362571023	4.08818677889495e-06	\\
-8570.21484375	3.42680261854997e-06	\\
-8569.23606178977	2.74717405772727e-06	\\
-8568.25727982955	2.90126282665625e-06	\\
-8567.27849786932	3.14230050872903e-06	\\
-8566.29971590909	4.25807784951737e-06	\\
-8565.32093394886	4.30182932395546e-06	\\
-8564.34215198864	3.41191333346786e-06	\\
-8563.36337002841	3.93955537125441e-06	\\
-8562.38458806818	3.23393859578716e-06	\\
-8561.40580610795	3.32684500255721e-06	\\
-8560.42702414773	4.2432045242635e-06	\\
-8559.4482421875	3.38941677230627e-06	\\
-8558.46946022727	3.82852590967594e-06	\\
-8557.49067826705	2.42885446582145e-06	\\
-8556.51189630682	4.08818989605861e-06	\\
-8555.53311434659	3.68933804179216e-06	\\
-8554.55433238636	5.1461016006667e-06	\\
-8553.57555042614	1.90245573644042e-06	\\
-8552.59676846591	3.38063765383325e-06	\\
-8551.61798650568	5.06146357010971e-06	\\
-8550.63920454545	2.98421491873844e-06	\\
-8549.66042258523	2.20972397395324e-06	\\
-8548.681640625	1.66207146206088e-06	\\
-8547.70285866477	3.20166927893619e-06	\\
-8546.72407670455	2.85545611044929e-06	\\
-8545.74529474432	2.18450356195563e-06	\\
-8544.76651278409	4.94666809187827e-06	\\
-8543.78773082386	3.27533463420181e-06	\\
-8542.80894886364	4.07349492140676e-06	\\
-8541.83016690341	5.07420841171186e-06	\\
-8540.85138494318	3.4691984788449e-06	\\
-8539.87260298295	3.74904546375019e-06	\\
-8538.89382102273	3.25406828972341e-06	\\
-8537.9150390625	3.09072286123444e-06	\\
-8536.93625710227	4.55548591539163e-06	\\
-8535.95747514205	3.51818172265178e-06	\\
-8534.97869318182	2.98645990742925e-06	\\
-8533.99991122159	3.98353114443226e-06	\\
-8533.02112926136	4.71614451995033e-06	\\
-8532.04234730114	4.0751911518794e-06	\\
-8531.06356534091	3.9219255706459e-06	\\
-8530.08478338068	5.12660960705058e-06	\\
-8529.10600142045	4.37121932788782e-06	\\
-8528.12721946023	4.509813997395e-06	\\
-8527.1484375	3.52760337331994e-06	\\
-8526.16965553977	3.00221415041378e-06	\\
-8525.19087357955	2.58749086069944e-06	\\
-8524.21209161932	4.82480802693169e-06	\\
-8523.23330965909	2.82692331814064e-06	\\
-8522.25452769886	4.77484109951697e-06	\\
-8521.27574573864	3.55435068739517e-06	\\
-8520.29696377841	3.97205765531748e-06	\\
-8519.31818181818	3.82606876687092e-06	\\
-8518.33939985795	4.23233200717902e-06	\\
-8517.36061789773	4.38715793850193e-06	\\
-8516.3818359375	4.76215499634798e-06	\\
-8515.40305397727	4.42974249972951e-06	\\
-8514.42427201705	5.21202408471458e-06	\\
-8513.44549005682	2.54751807541059e-06	\\
-8512.46670809659	3.91651643919912e-06	\\
-8511.48792613636	3.95727952118881e-06	\\
-8510.50914417614	5.46796599696347e-06	\\
-8509.53036221591	3.0194846355406e-06	\\
-8508.55158025568	4.55539890505747e-06	\\
-8507.57279829545	4.24986817002066e-06	\\
-8506.59401633523	4.34690548892798e-06	\\
-8505.615234375	3.14029270998502e-06	\\
-8504.63645241477	4.89633047591287e-06	\\
-8503.65767045455	4.18255072356627e-06	\\
-8502.67888849432	4.34668536513956e-06	\\
-8501.70010653409	3.18483569504692e-06	\\
-8500.72132457386	4.09435391526558e-06	\\
-8499.74254261364	4.29328533728939e-06	\\
-8498.76376065341	2.65170424238405e-06	\\
-8497.78497869318	3.49153953547497e-06	\\
-8496.80619673295	3.40632205709469e-06	\\
-8495.82741477273	3.86363013583024e-06	\\
-8494.8486328125	4.33780823568058e-06	\\
-8493.86985085227	4.75739230000583e-06	\\
-8492.89106889205	2.55700181159241e-06	\\
-8491.91228693182	3.05591195670661e-06	\\
-8490.93350497159	3.00464729658534e-06	\\
-8489.95472301136	3.30572146040899e-06	\\
-8488.97594105114	2.53938622527093e-06	\\
-8487.99715909091	2.92088261826921e-06	\\
-8487.01837713068	4.00718024139016e-06	\\
-8486.03959517045	3.70148945637796e-06	\\
-8485.06081321023	3.68871070140109e-06	\\
-8484.08203125	2.94719220236785e-06	\\
-8483.10324928977	4.16290571655629e-06	\\
-8482.12446732955	2.82133440901103e-06	\\
-8481.14568536932	4.10993729452668e-06	\\
-8480.16690340909	3.28379835800295e-06	\\
-8479.18812144886	2.95331306574765e-06	\\
-8478.20933948864	3.50609710740318e-06	\\
-8477.23055752841	2.71719986809223e-06	\\
-8476.25177556818	3.40028002571498e-06	\\
-8475.27299360795	4.00106014975804e-06	\\
-8474.29421164773	2.41445450982072e-06	\\
-8473.3154296875	4.73748471302194e-06	\\
-8472.33664772727	4.41179131715913e-06	\\
-8471.35786576705	4.28102313399466e-06	\\
-8470.37908380682	3.00070582597243e-06	\\
-8469.40030184659	3.19178491362545e-06	\\
-8468.42151988636	4.4195275694735e-06	\\
-8467.44273792614	3.28608711549321e-06	\\
-8466.46395596591	2.49163749225215e-06	\\
-8465.48517400568	3.90014374553533e-06	\\
-8464.50639204545	3.4549909915899e-06	\\
-8463.52761008523	2.53301171621846e-06	\\
-8462.548828125	3.85306373472417e-06	\\
-8461.57004616477	4.3109511018563e-06	\\
-8460.59126420455	3.0223218682101e-06	\\
-8459.61248224432	3.0721812643043e-06	\\
-8458.63370028409	2.18217934398865e-06	\\
-8457.65491832386	3.08173445725528e-06	\\
-8456.67613636364	2.2732895318113e-06	\\
-8455.69735440341	2.19221235941894e-06	\\
-8454.71857244318	3.27428091813218e-06	\\
-8453.73979048295	2.51943557070209e-06	\\
-8452.76100852273	2.92633503725563e-06	\\
-8451.7822265625	4.04621617910954e-06	\\
-8450.80344460227	2.06515608856371e-06	\\
-8449.82466264205	3.74853157980745e-06	\\
-8448.84588068182	4.6354189321211e-06	\\
-8447.86709872159	3.64334949257105e-06	\\
-8446.88831676136	4.15708393189235e-06	\\
-8445.90953480114	2.05941461376731e-06	\\
-8444.93075284091	3.22237520820792e-06	\\
-8443.95197088068	3.99071284092812e-06	\\
-8442.97318892045	3.55788401926458e-06	\\
-8441.99440696023	4.167657904767e-06	\\
-8441.015625	3.33285135861253e-06	\\
-8440.03684303977	3.62400961827291e-06	\\
-8439.05806107955	2.67667815131344e-06	\\
-8438.07927911932	3.60446269594013e-06	\\
-8437.10049715909	5.24587348613625e-06	\\
-8436.12171519886	4.75027130366023e-06	\\
-8435.14293323864	2.74383837333887e-06	\\
-8434.16415127841	3.33724152046893e-06	\\
-8433.18536931818	3.32023050515714e-06	\\
-8432.20658735795	3.22129074396842e-06	\\
-8431.22780539773	2.16634633242642e-06	\\
-8430.2490234375	2.67818309598508e-06	\\
-8429.27024147727	3.60571147228688e-06	\\
-8428.29145951705	4.61456000463167e-06	\\
-8427.31267755682	2.92097022482183e-06	\\
-8426.33389559659	2.7069144701764e-06	\\
-8425.35511363636	2.80623936739656e-06	\\
-8424.37633167614	4.03085583592749e-06	\\
-8423.39754971591	4.07926009214457e-06	\\
-8422.41876775568	2.38844510328113e-06	\\
-8421.43998579545	3.59238523554745e-06	\\
-8420.46120383523	4.18372707195538e-06	\\
-8419.482421875	3.97500516933704e-06	\\
-8418.50363991477	3.84606744030403e-06	\\
-8417.52485795455	2.68094446460607e-06	\\
-8416.54607599432	1.79449653251047e-06	\\
-8415.56729403409	3.85281369869656e-06	\\
-8414.58851207386	4.02415665955858e-06	\\
-8413.60973011364	3.91932701955495e-06	\\
-8412.63094815341	3.83495592592134e-06	\\
-8411.65216619318	3.06530163967231e-06	\\
-8410.67338423295	3.46297628790418e-06	\\
-8409.69460227273	5.14246252410844e-06	\\
-8408.7158203125	3.73038514774661e-06	\\
-8407.73703835227	5.13564348102845e-06	\\
-8406.75825639205	4.55567528943941e-06	\\
-8405.77947443182	2.96274188737983e-06	\\
-8404.80069247159	3.38868299515085e-06	\\
-8403.82191051136	4.37752055508318e-06	\\
-8402.84312855114	3.32914699455622e-06	\\
-8401.86434659091	3.95962459837941e-06	\\
-8400.88556463068	3.85987937691385e-06	\\
-8399.90678267045	5.50941762384108e-06	\\
-8398.92800071023	4.12955819150867e-06	\\
-8397.94921875	4.23657357762497e-06	\\
-8396.97043678977	2.50333035720446e-06	\\
-8395.99165482955	4.45639850498184e-06	\\
-8395.01287286932	3.53732602722362e-06	\\
-8394.03409090909	3.48204781299924e-06	\\
-8393.05530894886	3.30777327959902e-06	\\
-8392.07652698864	4.11905888044e-06	\\
-8391.09774502841	3.22924417518071e-06	\\
-8390.11896306818	3.96348340051255e-06	\\
-8389.14018110795	3.43705230420214e-06	\\
-8388.16139914773	3.83351878075508e-06	\\
-8387.1826171875	3.16458083885723e-06	\\
-8386.20383522727	3.7156132539024e-06	\\
-8385.22505326705	4.96872116478368e-06	\\
-8384.24627130682	2.28338505882279e-06	\\
-8383.26748934659	3.75914298724034e-06	\\
-8382.28870738636	4.15252468788738e-06	\\
-8381.30992542614	3.33360315829038e-06	\\
-8380.33114346591	2.18115659826215e-06	\\
-8379.35236150568	4.02199102004621e-06	\\
-8378.37357954545	4.03270990920717e-06	\\
-8377.39479758523	4.4008709709293e-06	\\
-8376.416015625	4.15782700868712e-06	\\
-8375.43723366477	3.49086510343283e-06	\\
-8374.45845170455	3.69821317200647e-06	\\
-8373.47966974432	4.9555934817398e-06	\\
-8372.50088778409	4.7599485553678e-06	\\
-8371.52210582386	2.71531232091999e-06	\\
-8370.54332386364	3.74281212892694e-06	\\
-8369.56454190341	3.87818909233984e-06	\\
-8368.58575994318	4.85267005839422e-06	\\
-8367.60697798295	4.42741677105964e-06	\\
-8366.62819602273	4.10708370755097e-06	\\
-8365.6494140625	2.6996666364539e-06	\\
-8364.67063210227	5.17028795368443e-06	\\
-8363.69185014205	4.24474142752469e-06	\\
-8362.71306818182	3.68102762804321e-06	\\
-8361.73428622159	2.31539728966751e-06	\\
-8360.75550426136	3.15169743582535e-06	\\
-8359.77672230114	2.6979625145736e-06	\\
-8358.79794034091	4.09270263297388e-06	\\
-8357.81915838068	2.8134758324076e-06	\\
-8356.84037642045	3.51006936440622e-06	\\
-8355.86159446023	3.70302499306216e-06	\\
-8354.8828125	4.0458963739989e-06	\\
-8353.90403053977	5.75080628631169e-06	\\
-8352.92524857955	4.22844193040556e-06	\\
-8351.94646661932	4.18550099695824e-06	\\
-8350.96768465909	4.52249897361583e-06	\\
-8349.98890269886	4.78992672196626e-06	\\
-8349.01012073864	4.37123144745623e-06	\\
-8348.03133877841	2.72787713193264e-06	\\
-8347.05255681818	3.89596060796767e-06	\\
-8346.07377485795	2.80134149958677e-06	\\
-8345.09499289773	5.13151311098645e-06	\\
-8344.1162109375	2.25345131709488e-06	\\
-8343.13742897727	4.3086171549939e-06	\\
-8342.15864701705	4.57385555434247e-06	\\
-8341.17986505682	5.72077272624378e-06	\\
-8340.20108309659	5.01856787153624e-06	\\
-8339.22230113636	3.63120241477459e-06	\\
-8338.24351917614	3.36124160414768e-06	\\
-8337.26473721591	3.96906578971154e-06	\\
-8336.28595525568	4.4353795054887e-06	\\
-8335.30717329545	4.47899233027712e-06	\\
-8334.32839133523	3.63614298684012e-06	\\
-8333.349609375	3.87904143927563e-06	\\
-8332.37082741477	3.91840773945176e-06	\\
-8331.39204545455	4.27735069895042e-06	\\
-8330.41326349432	4.33640291617984e-06	\\
-8329.43448153409	4.25556228228178e-06	\\
-8328.45569957386	4.99693153705762e-06	\\
-8327.47691761364	2.78628690471846e-06	\\
-8326.49813565341	3.39833758915808e-06	\\
-8325.51935369318	3.92432170175196e-06	\\
-8324.54057173295	3.55392861217828e-06	\\
-8323.56178977273	3.43419103984659e-06	\\
-8322.5830078125	4.20032470506118e-06	\\
-8321.60422585227	4.87235209094209e-06	\\
-8320.62544389205	1.78246677589139e-06	\\
-8319.64666193182	3.62921663088758e-06	\\
-8318.66787997159	3.83605382230752e-06	\\
-8317.68909801136	5.46227739703373e-06	\\
-8316.71031605114	2.07161112027053e-06	\\
-8315.73153409091	3.77856767488408e-06	\\
-8314.75275213068	5.159708403271e-06	\\
-8313.77397017045	3.42414993241669e-06	\\
-8312.79518821023	3.00705304256692e-06	\\
-8311.81640625	4.42405879001193e-06	\\
-8310.83762428977	3.6929399937579e-06	\\
-8309.85884232955	2.77693137478642e-06	\\
-8308.88006036932	2.0773052319464e-06	\\
-8307.90127840909	3.71129941887126e-06	\\
-8306.92249644886	3.44059688750784e-06	\\
-8305.94371448864	3.27702048831936e-06	\\
-8304.96493252841	3.53393825227458e-06	\\
-8303.98615056818	4.45079677440787e-06	\\
-8303.00736860795	3.7828635357981e-06	\\
-8302.02858664773	3.34280771820485e-06	\\
-8301.0498046875	5.00887698627318e-06	\\
-8300.07102272727	3.31784423058742e-06	\\
-8299.09224076705	3.12944184614309e-06	\\
-8298.11345880682	3.50374705850787e-06	\\
-8297.13467684659	3.39175323089224e-06	\\
-8296.15589488636	3.70789471477184e-06	\\
-8295.17711292614	3.05610029741626e-06	\\
-8294.19833096591	2.38525038160611e-06	\\
-8293.21954900568	3.7142428876889e-06	\\
-8292.24076704545	3.75559809870672e-06	\\
-8291.26198508523	4.38028152210444e-06	\\
-8290.283203125	4.06622761520832e-06	\\
-8289.30442116477	2.68826067696052e-06	\\
-8288.32563920455	3.21859907686732e-06	\\
-8287.34685724432	4.50319886487694e-06	\\
-8286.36807528409	2.8545564396732e-06	\\
-8285.38929332386	2.97048336444056e-06	\\
-8284.41051136364	3.73212784014146e-06	\\
-8283.43172940341	4.4825937642362e-06	\\
-8282.45294744318	2.24749542352369e-06	\\
-8281.47416548295	4.54360232545101e-06	\\
-8280.49538352273	3.7306728169535e-06	\\
-8279.5166015625	4.90071420116844e-06	\\
-8278.53781960227	4.83241088552398e-06	\\
-8277.55903764205	3.37698502175783e-06	\\
-8276.58025568182	3.91077635439863e-06	\\
-8275.60147372159	3.93477129592653e-06	\\
-8274.62269176136	3.42214490973501e-06	\\
-8273.64390980114	4.42446730095757e-06	\\
-8272.66512784091	3.8741078118355e-06	\\
-8271.68634588068	2.59382236076687e-06	\\
-8270.70756392045	2.18234396759037e-06	\\
-8269.72878196023	3.98295003574514e-06	\\
-8268.75	3.65938337503455e-06	\\
-8267.77121803977	3.47412006639646e-06	\\
-8266.79243607955	3.38100995526953e-06	\\
-8265.81365411932	3.34386389519262e-06	\\
-8264.83487215909	2.68883070016863e-06	\\
-8263.85609019886	3.34125399043517e-06	\\
-8262.87730823864	2.53219133894125e-06	\\
-8261.89852627841	4.36329687450477e-06	\\
-8260.91974431818	2.66166726842328e-06	\\
-8259.94096235795	3.96671305375378e-06	\\
-8258.96218039773	3.45730389660909e-06	\\
-8257.9833984375	4.61064678728154e-06	\\
-8257.00461647727	4.68273116855704e-06	\\
-8256.02583451705	2.28585074597904e-06	\\
-8255.04705255682	4.60284551985926e-06	\\
-8254.06827059659	4.0243374696911e-06	\\
-8253.08948863636	2.66844277519147e-06	\\
-8252.11070667614	4.46979008228949e-06	\\
-8251.13192471591	4.79884481447568e-06	\\
-8250.15314275568	1.44810896206441e-06	\\
-8249.17436079545	2.45047547978565e-06	\\
-8248.19557883523	2.72385387087176e-06	\\
-8247.216796875	3.08190632597718e-06	\\
-8246.23801491477	2.91174343403663e-06	\\
-8245.25923295455	4.70492957312932e-06	\\
-8244.28045099432	4.10830751536057e-06	\\
-8243.30166903409	3.0830841436991e-06	\\
-8242.32288707386	3.35841779872415e-06	\\
-8241.34410511364	4.16909225441466e-06	\\
-8240.36532315341	3.1437996598493e-06	\\
-8239.38654119318	4.24481379577595e-06	\\
-8238.40775923295	3.74053160856989e-06	\\
-8237.42897727273	2.97333497500662e-06	\\
-8236.4501953125	3.11080297632373e-06	\\
-8235.47141335227	2.63351763875217e-06	\\
-8234.49263139205	6.03714486525444e-06	\\
-8233.51384943182	3.06516538720431e-06	\\
-8232.53506747159	2.89991451136236e-06	\\
-8231.55628551136	3.66471736858576e-06	\\
-8230.57750355114	4.80461072737684e-06	\\
-8229.59872159091	3.97710570208932e-06	\\
-8228.61993963068	4.62486807739592e-06	\\
-8227.64115767045	4.2620020741042e-06	\\
-8226.66237571023	4.12846960540697e-06	\\
-8225.68359375	4.1670709284116e-06	\\
-8224.70481178977	4.4326445030219e-06	\\
-8223.72602982955	5.23614413140266e-06	\\
-8222.74724786932	4.16851133436525e-06	\\
-8221.76846590909	4.94850446486526e-06	\\
-8220.78968394886	3.95181047535169e-06	\\
-8219.81090198864	3.77086686424597e-06	\\
-8218.83212002841	4.37040324658822e-06	\\
-8217.85333806818	4.58504476595839e-06	\\
-8216.87455610795	5.24944279038826e-06	\\
-8215.89577414773	4.12331185012413e-06	\\
-8214.9169921875	4.68511687076236e-06	\\
-8213.93821022727	5.0145194154808e-06	\\
-8212.95942826705	3.3210031785172e-06	\\
-8211.98064630682	4.09122431671904e-06	\\
-8211.00186434659	5.12120382203815e-06	\\
-8210.02308238636	3.71999690419774e-06	\\
-8209.04430042614	4.04892894114637e-06	\\
-8208.06551846591	4.20594809793508e-06	\\
-8207.08673650568	5.34374142014117e-06	\\
-8206.10795454545	4.72534938401061e-06	\\
-8205.12917258523	4.20036467115612e-06	\\
-8204.150390625	2.13057339747495e-06	\\
-8203.17160866477	3.42256208441359e-06	\\
-8202.19282670455	5.16216927257434e-06	\\
-8201.21404474432	4.55807003376236e-06	\\
-8200.23526278409	5.47829737998736e-06	\\
-8199.25648082386	3.99088124799845e-06	\\
-8198.27769886364	4.87200027312989e-06	\\
-8197.29891690341	4.73764689294427e-06	\\
-8196.32013494318	3.9493133203462e-06	\\
-8195.34135298295	3.91364691425224e-06	\\
-8194.36257102273	3.64126489131634e-06	\\
-8193.3837890625	5.36744090518927e-06	\\
-8192.40500710227	3.9923671664337e-06	\\
-8191.42622514205	3.0699575478541e-06	\\
-8190.44744318182	5.08874300248422e-06	\\
-8189.46866122159	4.00440005064646e-06	\\
-8188.48987926136	4.20786884875947e-06	\\
-8187.51109730114	4.06412848901452e-06	\\
-8186.53231534091	4.38117847901353e-06	\\
-8185.55353338068	4.67511037122568e-06	\\
-8184.57475142045	3.56333660228709e-06	\\
-8183.59596946023	3.02495949643541e-06	\\
-8182.6171875	5.31565783557162e-06	\\
-8181.63840553977	4.1909045646948e-06	\\
-8180.65962357955	4.63841444638597e-06	\\
-8179.68084161932	4.11568346194605e-06	\\
-8178.70205965909	3.37374518709655e-06	\\
-8177.72327769886	5.51687825000007e-06	\\
-8176.74449573864	4.71609218496146e-06	\\
-8175.76571377841	4.30291502928423e-06	\\
-8174.78693181818	4.40982060775676e-06	\\
-8173.80814985795	4.53828799644435e-06	\\
-8172.82936789773	4.26386116477686e-06	\\
-8171.8505859375	6.118269058491e-06	\\
-8170.87180397727	3.59150349721012e-06	\\
-8169.89302201705	4.01508888143297e-06	\\
-8168.91424005682	3.95431380793079e-06	\\
-8167.93545809659	5.21642589459192e-06	\\
-8166.95667613636	4.30535128041952e-06	\\
-8165.97789417614	3.51286321818986e-06	\\
-8164.99911221591	3.35568928212556e-06	\\
-8164.02033025568	4.59609033914816e-06	\\
-8163.04154829545	5.29633353876296e-06	\\
-8162.06276633523	3.83073024001608e-06	\\
-8161.083984375	4.1994643118067e-06	\\
-8160.10520241477	5.68304052966202e-06	\\
-8159.12642045455	3.68006237873378e-06	\\
-8158.14763849432	3.10317244257895e-06	\\
-8157.16885653409	2.50223017523717e-06	\\
-8156.19007457386	6.04053946393796e-06	\\
-8155.21129261364	5.66137681378543e-06	\\
-8154.23251065341	7.15556613928182e-06	\\
-8153.25372869318	3.35168028844656e-06	\\
-8152.27494673295	4.00731977527929e-06	\\
-8151.29616477273	3.04796799339356e-06	\\
-8150.3173828125	4.551939481779e-06	\\
-8149.33860085227	3.99358080300891e-06	\\
-8148.35981889205	4.93734248994016e-06	\\
-8147.38103693182	2.72863227980132e-06	\\
-8146.40225497159	3.98427655642898e-06	\\
-8145.42347301136	4.05106088345693e-06	\\
-8144.44469105114	3.80171884922394e-06	\\
-8143.46590909091	3.15770775634447e-06	\\
-8142.48712713068	4.05508888282349e-06	\\
-8141.50834517045	4.95749134587162e-06	\\
-8140.52956321023	4.90279037917904e-06	\\
-8139.55078125	3.05544146715157e-06	\\
-8138.57199928977	3.93610797872437e-06	\\
-8137.59321732955	3.65002998558843e-06	\\
-8136.61443536932	3.32060156021312e-06	\\
-8135.63565340909	2.06993158620054e-06	\\
-8134.65687144886	4.28180622570597e-06	\\
-8133.67808948864	2.9256057156691e-06	\\
-8132.69930752841	4.26554101180591e-06	\\
-8131.72052556818	4.43007612673432e-06	\\
-8130.74174360795	3.55470958133233e-06	\\
-8129.76296164773	4.49954665366131e-06	\\
-8128.7841796875	4.55424274749987e-06	\\
-8127.80539772727	3.56789256857298e-06	\\
-8126.82661576705	3.47084219267295e-06	\\
-8125.84783380682	4.04985957142157e-06	\\
-8124.86905184659	2.11269274639469e-06	\\
-8123.89026988636	2.31541126323269e-06	\\
-8122.91148792614	1.95606479194772e-06	\\
-8121.93270596591	2.12715196080639e-06	\\
-8120.95392400568	3.86380343944485e-06	\\
-8119.97514204545	3.28496995589459e-06	\\
-8118.99636008523	1.9981601911624e-06	\\
-8118.017578125	2.91827364758733e-06	\\
-8117.03879616477	2.80033049212405e-06	\\
-8116.06001420455	4.56468002852782e-06	\\
-8115.08123224432	3.65831295706154e-06	\\
-8114.10245028409	4.94123768592745e-06	\\
-8113.12366832386	3.78252257721143e-06	\\
-8112.14488636364	5.47075308283436e-06	\\
-8111.16610440341	3.61809230501143e-06	\\
-8110.18732244318	4.61152077680146e-06	\\
-8109.20854048295	4.62189074947038e-06	\\
-8108.22975852273	3.91199153274808e-06	\\
-8107.2509765625	3.53734145518259e-06	\\
-8106.27219460227	2.97434601929001e-06	\\
-8105.29341264205	1.58903639765489e-06	\\
-8104.31463068182	4.59249608166484e-06	\\
-8103.33584872159	4.32177277983625e-06	\\
-8102.35706676136	4.53256569387676e-06	\\
-8101.37828480114	5.34561805457993e-06	\\
-8100.39950284091	3.59757220760891e-06	\\
-8099.42072088068	2.35616267727136e-06	\\
-8098.44193892045	3.55900693823748e-06	\\
-8097.46315696023	2.68311725433526e-06	\\
-8096.484375	4.54354706494647e-06	\\
-8095.50559303977	4.32693736370179e-06	\\
-8094.52681107955	3.11647785043075e-06	\\
-8093.54802911932	2.57117860253193e-06	\\
-8092.56924715909	3.65500212990581e-06	\\
-8091.59046519886	1.2169811648049e-06	\\
-8090.61168323864	3.39968552321193e-06	\\
-8089.63290127841	3.09119719494041e-06	\\
-8088.65411931818	5.48866176398775e-06	\\
-8087.67533735795	2.47031396772035e-06	\\
-8086.69655539773	2.94023931046392e-06	\\
-8085.7177734375	3.41175403429316e-06	\\
-8084.73899147727	4.68895536225944e-06	\\
-8083.76020951705	4.86227019578342e-06	\\
-8082.78142755682	4.47579990660129e-06	\\
-8081.80264559659	3.99908885211802e-06	\\
-8080.82386363636	5.38569408524191e-06	\\
-8079.84508167614	3.72673140906615e-06	\\
-8078.86629971591	4.28512116253197e-06	\\
-8077.88751775568	4.45109630089424e-06	\\
-8076.90873579545	3.48418482071997e-06	\\
-8075.92995383523	4.87037274741344e-06	\\
-8074.951171875	3.12672629885649e-06	\\
-8073.97238991477	2.53615077419106e-06	\\
-8072.99360795455	4.21782919536291e-06	\\
-8072.01482599432	5.17127529149464e-06	\\
-8071.03604403409	3.97962588853393e-06	\\
-8070.05726207386	5.33454281790793e-06	\\
-8069.07848011364	3.33472684662752e-06	\\
-8068.09969815341	5.41784542570842e-06	\\
-8067.12091619318	3.41235154963275e-06	\\
-8066.14213423295	4.43998216789173e-06	\\
-8065.16335227273	4.3371790006387e-06	\\
-8064.1845703125	3.62799653881025e-06	\\
-8063.20578835227	4.0360142308523e-06	\\
-8062.22700639205	4.05799331151515e-06	\\
-8061.24822443182	4.737876964188e-06	\\
-8060.26944247159	3.49785089099633e-06	\\
-8059.29066051136	4.12588677897186e-06	\\
-8058.31187855114	2.63693519284635e-06	\\
-8057.33309659091	4.0947234939761e-06	\\
-8056.35431463068	4.39494156489764e-06	\\
-8055.37553267045	5.01848226886935e-06	\\
-8054.39675071023	3.46726561831839e-06	\\
-8053.41796875	3.47518171718091e-06	\\
-8052.43918678977	2.37977928653006e-06	\\
-8051.46040482955	2.92431516485942e-06	\\
-8050.48162286932	2.47204710794422e-06	\\
-8049.50284090909	3.31060849966303e-06	\\
-8048.52405894886	4.11152369179559e-06	\\
-8047.54527698864	3.46838948413013e-06	\\
-8046.56649502841	3.38144435085433e-06	\\
-8045.58771306818	3.31267899886258e-06	\\
-8044.60893110795	5.31436629749645e-06	\\
-8043.63014914773	3.62015081469506e-06	\\
-8042.6513671875	2.66836457621912e-06	\\
-8041.67258522727	4.24008848078004e-06	\\
-8040.69380326705	5.16023572139366e-06	\\
-8039.71502130682	1.83172487242726e-06	\\
-8038.73623934659	3.98179792586316e-06	\\
-8037.75745738636	3.77046479036915e-06	\\
-8036.77867542614	3.8842607456487e-06	\\
-8035.79989346591	2.71342547886521e-06	\\
-8034.82111150568	2.67820757963718e-06	\\
-8033.84232954545	2.53319660567421e-06	\\
-8032.86354758523	2.77731548897425e-06	\\
-8031.884765625	4.35585284440794e-06	\\
-8030.90598366477	3.00151777797763e-06	\\
-8029.92720170455	3.31814598757878e-06	\\
-8028.94841974432	5.57586196730591e-06	\\
-8027.96963778409	3.32266222162037e-06	\\
-8026.99085582386	3.13862455826724e-06	\\
-8026.01207386364	2.48634180626028e-06	\\
-8025.03329190341	4.00976615898969e-06	\\
-8024.05450994318	4.63298650397658e-06	\\
-8023.07572798295	2.26994458801907e-06	\\
-8022.09694602273	2.31000285380168e-06	\\
-8021.1181640625	2.32258382408176e-06	\\
-8020.13938210227	4.1427798111212e-06	\\
-8019.16060014205	2.1193750663052e-06	\\
-8018.18181818182	1.99822406918664e-06	\\
-8017.20303622159	3.71679217118054e-06	\\
-8016.22425426136	3.29576548179396e-06	\\
-8015.24547230114	3.28148356521127e-06	\\
-8014.26669034091	3.47179938842379e-06	\\
-8013.28790838068	5.01148420162452e-07	\\
-8012.30912642045	2.99651347399327e-06	\\
-8011.33034446023	2.74824629297583e-06	\\
-8010.3515625	3.6421803227421e-06	\\
-8009.37278053977	2.8421961941267e-06	\\
-8008.39399857955	2.83914457312905e-06	\\
-8007.41521661932	2.61774098903825e-06	\\
-8006.43643465909	3.45189065731284e-06	\\
-8005.45765269886	2.90772930253132e-06	\\
-8004.47887073864	2.68807800332974e-06	\\
-8003.50008877841	2.79461876416412e-06	\\
-8002.52130681818	2.79370767648692e-06	\\
-8001.54252485795	2.96897702909853e-06	\\
-8000.56374289773	7.6656306452505e-06	\\
-7999.5849609375	8.93074460766869e-06	\\
-7998.60617897727	3.51130687700943e-06	\\
-7997.62739701705	9.36595472954119e-07	\\
-7996.64861505682	3.28452277579537e-06	\\
-7995.66983309659	4.9056321771598e-06	\\
-7994.69105113636	2.36190814512752e-06	\\
-7993.71226917614	2.83368761816081e-06	\\
-7992.73348721591	3.2881576092079e-06	\\
-7991.75470525568	1.62013376460382e-06	\\
-7990.77592329545	2.29911428519417e-06	\\
-7989.79714133523	3.04753515907302e-06	\\
-7988.818359375	4.89439975573399e-06	\\
-7987.83957741477	2.62485076377334e-06	\\
-7986.86079545455	3.76231671906241e-06	\\
-7985.88201349432	3.55400955602508e-06	\\
-7984.90323153409	4.03385837989218e-06	\\
-7983.92444957386	3.84469496522077e-06	\\
-7982.94566761364	4.29583589003192e-06	\\
-7981.96688565341	5.32998241164676e-06	\\
-7980.98810369318	5.24114556742944e-06	\\
-7980.00932173295	3.90932214001025e-06	\\
-7979.03053977273	2.85993725539786e-06	\\
-7978.0517578125	3.63623912354511e-06	\\
-7977.07297585227	4.31960602637649e-06	\\
-7976.09419389205	3.86527045271258e-06	\\
-7975.11541193182	2.92412075139964e-06	\\
-7974.13662997159	3.87346182653758e-06	\\
-7973.15784801136	2.8713896167185e-06	\\
-7972.17906605114	4.39148698028076e-06	\\
-7971.20028409091	4.5260899812153e-06	\\
-7970.22150213068	5.7461723022978e-06	\\
-7969.24272017045	4.98931705154118e-06	\\
-7968.26393821023	6.05922671271326e-06	\\
-7967.28515625	3.62124553838588e-06	\\
-7966.30637428977	5.56851302261156e-06	\\
-7965.32759232955	4.07699634148717e-06	\\
-7964.34881036932	5.25356004110589e-06	\\
-7963.37002840909	5.71062640167147e-06	\\
-7962.39124644886	5.53428180174318e-06	\\
-7961.41246448864	4.8752560059011e-06	\\
-7960.43368252841	5.29019053053121e-06	\\
-7959.45490056818	5.94695153095406e-06	\\
-7958.47611860795	5.97861498079418e-06	\\
-7957.49733664773	5.97894527455636e-06	\\
-7956.5185546875	7.58321507548155e-06	\\
-7955.53977272727	4.65446863276287e-06	\\
-7954.56099076705	6.04741935179723e-06	\\
-7953.58220880682	4.50208135373816e-06	\\
-7952.60342684659	5.2056011253162e-06	\\
-7951.62464488636	5.18347723160764e-06	\\
-7950.64586292614	4.31719600537898e-06	\\
-7949.66708096591	4.60056786094266e-06	\\
-7948.68829900568	3.79396494416971e-06	\\
-7947.70951704545	5.86341465025652e-06	\\
-7946.73073508523	5.74875796495384e-06	\\
-7945.751953125	4.86711435227581e-06	\\
-7944.77317116477	5.20323473914406e-06	\\
-7943.79438920455	4.1962232950004e-06	\\
-7942.81560724432	4.47000891104082e-06	\\
-7941.83682528409	4.9562631930333e-06	\\
-7940.85804332386	4.82359004478958e-06	\\
-7939.87926136364	4.81250542816228e-06	\\
-7938.90047940341	4.68189517278082e-06	\\
-7937.92169744318	5.30606034973306e-06	\\
-7936.94291548295	5.6882188897147e-06	\\
-7935.96413352273	4.21584423939073e-06	\\
-7934.9853515625	6.89839868215745e-06	\\
-7934.00656960227	3.45351047591823e-06	\\
-7933.02778764205	3.65979753347456e-06	\\
-7932.04900568182	5.74678207241272e-06	\\
-7931.07022372159	3.47199103849089e-06	\\
-7930.09144176136	5.32811262722658e-06	\\
-7929.11265980114	3.47410193061751e-06	\\
-7928.13387784091	5.6444276499872e-06	\\
-7927.15509588068	4.68190532859842e-06	\\
-7926.17631392045	3.69221667566231e-06	\\
-7925.19753196023	3.80938148357651e-06	\\
-7924.21875	5.19482194964716e-06	\\
-7923.23996803977	3.09769965760844e-06	\\
-7922.26118607955	3.90418381623217e-06	\\
-7921.28240411932	5.03786992992522e-06	\\
-7920.30362215909	3.76206748918844e-06	\\
-7919.32484019886	3.30674558688336e-06	\\
-7918.34605823864	2.72962842979384e-06	\\
-7917.36727627841	3.84574583332815e-06	\\
-7916.38849431818	3.33801525067724e-06	\\
-7915.40971235795	1.68794902395497e-06	\\
-7914.43093039773	3.7759416224427e-06	\\
-7913.4521484375	4.17047611007577e-06	\\
-7912.47336647727	4.09749015083066e-06	\\
-7911.49458451705	2.67458148705469e-06	\\
-7910.51580255682	3.02838559010587e-06	\\
-7909.53702059659	2.58838772370549e-06	\\
-7908.55823863636	3.31361912942897e-06	\\
-7907.57945667614	3.46528629083978e-06	\\
-7906.60067471591	2.96490114178381e-06	\\
-7905.62189275568	4.15375744028159e-06	\\
-7904.64311079545	3.55650989672247e-06	\\
-7903.66432883523	2.4114961782452e-06	\\
-7902.685546875	2.22676361909668e-06	\\
-7901.70676491477	3.45423245692985e-06	\\
-7900.72798295455	3.7570033439445e-06	\\
-7899.74920099432	1.98737999908575e-06	\\
-7898.77041903409	3.63253078530786e-06	\\
-7897.79163707386	3.41982050472028e-06	\\
-7896.81285511364	3.33441878075491e-06	\\
-7895.83407315341	3.08298890042333e-06	\\
-7894.85529119318	2.69273403491728e-06	\\
-7893.87650923295	4.40311548087774e-06	\\
-7892.89772727273	2.16627975804845e-06	\\
-7891.9189453125	2.13054613005661e-06	\\
-7890.94016335227	4.43892323140954e-06	\\
-7889.96138139205	4.31400509555247e-06	\\
-7888.98259943182	3.72939120372255e-06	\\
-7888.00381747159	2.2237178413394e-06	\\
-7887.02503551136	3.67656809250265e-06	\\
-7886.04625355114	3.58129482751424e-06	\\
-7885.06747159091	4.5615459249884e-06	\\
-7884.08868963068	4.76079929855204e-06	\\
-7883.10990767045	4.02296427776396e-06	\\
-7882.13112571023	5.24689366884665e-06	\\
-7881.15234375	4.37356788247811e-06	\\
-7880.17356178977	3.14022323890567e-06	\\
-7879.19477982955	3.6027884504659e-06	\\
-7878.21599786932	3.42170242140251e-06	\\
-7877.23721590909	4.98376181473731e-06	\\
-7876.25843394886	4.74346232858677e-06	\\
-7875.27965198864	3.58124296217825e-06	\\
-7874.30087002841	3.82007827386475e-06	\\
-7873.32208806818	5.18064807638913e-06	\\
-7872.34330610795	2.6381147051253e-06	\\
-7871.36452414773	2.30835058821093e-06	\\
-7870.3857421875	5.04416648641942e-06	\\
-7869.40696022727	4.23048319252244e-06	\\
-7868.42817826705	5.68634063832196e-06	\\
-7867.44939630682	4.82442026495339e-06	\\
-7866.47061434659	3.45680966403192e-06	\\
-7865.49183238636	2.8657677212984e-06	\\
-7864.51305042614	4.26041710821128e-06	\\
-7863.53426846591	4.21716771807619e-06	\\
-7862.55548650568	5.29594626830844e-06	\\
-7861.57670454545	4.67808482227467e-06	\\
-7860.59792258523	3.39315624063424e-06	\\
-7859.619140625	3.93839701373357e-06	\\
-7858.64035866477	5.73985382432565e-06	\\
-7857.66157670455	5.55866431346123e-06	\\
-7856.68279474432	4.43917052510581e-06	\\
-7855.70401278409	4.05669250484929e-06	\\
-7854.72523082386	3.00438357601906e-06	\\
-7853.74644886364	4.76191538705938e-06	\\
-7852.76766690341	2.70627707573686e-06	\\
-7851.78888494318	3.35701019221998e-06	\\
-7850.81010298295	5.37264180561708e-06	\\
-7849.83132102273	4.04349579586436e-06	\\
-7848.8525390625	4.56548341684755e-06	\\
-7847.87375710227	4.81121396393302e-06	\\
-7846.89497514205	4.58286646330611e-06	\\
-7845.91619318182	4.33567264396821e-06	\\
-7844.93741122159	3.63450352479181e-06	\\
-7843.95862926136	3.26944912298455e-06	\\
-7842.97984730114	2.66226442075913e-06	\\
-7842.00106534091	3.4452403896727e-06	\\
-7841.02228338068	3.83744345290518e-06	\\
-7840.04350142045	3.80462733948569e-06	\\
-7839.06471946023	4.17228088863576e-06	\\
-7838.0859375	3.45489915527688e-06	\\
-7837.10715553977	5.35470227180458e-06	\\
-7836.12837357955	3.34110687847005e-06	\\
-7835.14959161932	2.69123978248195e-06	\\
-7834.17080965909	3.32428321093399e-06	\\
-7833.19202769886	5.05402345523039e-06	\\
-7832.21324573864	2.60059023043102e-06	\\
-7831.23446377841	6.01714400250215e-06	\\
-7830.25568181818	5.46550422187125e-06	\\
-7829.27689985795	4.90525686658367e-06	\\
-7828.29811789773	3.65911162150355e-06	\\
-7827.3193359375	3.29946946313407e-06	\\
-7826.34055397727	5.09778069092448e-06	\\
-7825.36177201705	3.76984263592982e-06	\\
-7824.38299005682	4.02097847888771e-06	\\
-7823.40420809659	5.215024576354e-06	\\
-7822.42542613636	3.81724133568985e-06	\\
-7821.44664417614	5.43581427079734e-06	\\
-7820.46786221591	3.88438070475803e-06	\\
-7819.48908025568	4.67007304760345e-06	\\
-7818.51029829545	4.60459162370192e-06	\\
-7817.53151633523	4.60716237754002e-06	\\
-7816.552734375	5.02516756713952e-06	\\
-7815.57395241477	4.75136512487336e-06	\\
-7814.59517045455	4.28893822995476e-06	\\
-7813.61638849432	4.82427296066673e-06	\\
-7812.63760653409	4.29622775479527e-06	\\
-7811.65882457386	6.03627315050138e-06	\\
-7810.68004261364	1.05662354843601e-06	\\
-7809.70126065341	4.47432751198009e-06	\\
-7808.72247869318	5.31230418221074e-06	\\
-7807.74369673295	4.16479948913211e-06	\\
-7806.76491477273	4.32038683162304e-06	\\
-7805.7861328125	5.44079256392773e-06	\\
-7804.80735085227	3.95198885871925e-06	\\
-7803.82856889205	7.30207909570171e-06	\\
-7802.84978693182	4.94946146148587e-06	\\
-7801.87100497159	5.2615137244291e-06	\\
-7800.89222301136	5.16791597304632e-06	\\
-7799.91344105114	5.60383885396528e-06	\\
-7798.93465909091	5.01111726225561e-06	\\
-7797.95587713068	4.52745336627556e-06	\\
-7796.97709517045	3.44766832771558e-06	\\
-7795.99831321023	6.68220235450437e-06	\\
-7795.01953125	3.38433659547656e-06	\\
-7794.04074928977	4.13831220620899e-06	\\
-7793.06196732955	4.26027806624917e-06	\\
-7792.08318536932	5.01656048659397e-06	\\
-7791.10440340909	4.78853931123531e-06	\\
-7790.12562144886	4.72922637812534e-06	\\
-7789.14683948864	6.15030287699029e-06	\\
-7788.16805752841	4.68381464895215e-06	\\
-7787.18927556818	4.22816546546588e-06	\\
-7786.21049360795	6.31488479139916e-06	\\
-7785.23171164773	5.72160298677731e-06	\\
-7784.2529296875	3.16123612714862e-06	\\
-7783.27414772727	4.25147801781855e-06	\\
-7782.29536576705	5.29553235207457e-06	\\
-7781.31658380682	4.09657110988685e-06	\\
-7780.33780184659	5.33070870472397e-06	\\
-7779.35901988636	5.25854656844097e-06	\\
-7778.38023792614	5.49656103399419e-06	\\
-7777.40145596591	4.99571749574305e-06	\\
-7776.42267400568	6.03065366493948e-06	\\
-7775.44389204545	4.64748760244117e-06	\\
-7774.46511008523	4.25571874052913e-06	\\
-7773.486328125	4.08940857631113e-06	\\
-7772.50754616477	5.92871536771216e-06	\\
-7771.52876420455	4.01541983840201e-06	\\
-7770.54998224432	4.3517920142964e-06	\\
-7769.57120028409	4.76552388859543e-06	\\
-7768.59241832386	4.22996054833292e-06	\\
-7767.61363636364	5.48242148983841e-06	\\
-7766.63485440341	3.340634070762e-06	\\
-7765.65607244318	4.73231363258992e-06	\\
-7764.67729048295	4.33254538856977e-06	\\
-7763.69850852273	4.2908025167518e-06	\\
-7762.7197265625	3.48879632089484e-06	\\
-7761.74094460227	3.41588146490222e-06	\\
-7760.76216264205	4.34195690034981e-06	\\
-7759.78338068182	4.36005036054809e-06	\\
-7758.80459872159	4.17085415385711e-06	\\
-7757.82581676136	3.06119674875592e-06	\\
-7756.84703480114	3.84201648427402e-06	\\
-7755.86825284091	4.98698700940848e-06	\\
-7754.88947088068	4.56117961782315e-06	\\
-7753.91068892045	3.24904514898141e-06	\\
-7752.93190696023	5.20256728779281e-06	\\
-7751.953125	5.11586571123422e-06	\\
-7750.97434303977	4.24186058942224e-06	\\
-7749.99556107955	3.83199644835725e-06	\\
-7749.01677911932	4.29920168701701e-06	\\
-7748.03799715909	4.48674439766442e-06	\\
-7747.05921519886	4.40448886946479e-06	\\
-7746.08043323864	3.10971572066128e-06	\\
-7745.10165127841	3.12903631441394e-06	\\
-7744.12286931818	4.18663628391368e-06	\\
-7743.14408735795	5.37452121639737e-06	\\
-7742.16530539773	4.18982431513199e-06	\\
-7741.1865234375	3.83910962561585e-06	\\
-7740.20774147727	1.95407818578341e-06	\\
-7739.22895951705	4.31776830469163e-06	\\
-7738.25017755682	4.90618001004555e-06	\\
-7737.27139559659	2.47054544936932e-06	\\
-7736.29261363636	2.22093269973854e-06	\\
-7735.31383167614	4.32130862961888e-06	\\
-7734.33504971591	2.94722066671542e-06	\\
-7733.35626775568	4.73348105978403e-06	\\
-7732.37748579545	4.87386689838634e-06	\\
-7731.39870383523	4.41758134190466e-06	\\
-7730.419921875	3.86560248724129e-06	\\
-7729.44113991477	4.12729278171093e-06	\\
-7728.46235795455	1.85608655273888e-06	\\
-7727.48357599432	3.64380065293939e-06	\\
-7726.50479403409	4.36759961762938e-06	\\
-7725.52601207386	1.82972445443292e-06	\\
-7724.54723011364	3.95879703107426e-06	\\
-7723.56844815341	3.87251547214441e-06	\\
-7722.58966619318	3.34591909470378e-06	\\
-7721.61088423295	3.82844600385207e-06	\\
-7720.63210227273	3.55249809493949e-06	\\
-7719.6533203125	4.64098388946889e-06	\\
-7718.67453835227	2.17203319423422e-06	\\
-7717.69575639205	3.28512642460656e-06	\\
-7716.71697443182	3.51065333079166e-06	\\
-7715.73819247159	4.68486497142283e-06	\\
-7714.75941051136	3.62581613470794e-06	\\
-7713.78062855114	3.0625214286451e-06	\\
-7712.80184659091	2.55244976078485e-06	\\
-7711.82306463068	3.54165479877292e-06	\\
-7710.84428267045	4.67956304707366e-06	\\
-7709.86550071023	4.33667858424485e-06	\\
-7708.88671875	3.09109159917926e-06	\\
-7707.90793678977	4.20689719788065e-06	\\
-7706.92915482955	2.95736774474916e-06	\\
-7705.95037286932	4.0488013303604e-06	\\
-7704.97159090909	2.98780881281544e-06	\\
-7703.99280894886	2.94925393800252e-06	\\
-7703.01402698864	4.19861099051873e-06	\\
-7702.03524502841	3.95900197486213e-06	\\
-7701.05646306818	3.33587846532945e-06	\\
-7700.07768110795	3.90248259033142e-06	\\
-7699.09889914773	4.76394164988522e-06	\\
-7698.1201171875	4.09926944899665e-06	\\
-7697.14133522727	3.96039221511818e-06	\\
-7696.16255326705	4.64199081300388e-06	\\
-7695.18377130682	3.38122594260053e-06	\\
-7694.20498934659	4.76079014148617e-06	\\
-7693.22620738636	5.12850908014277e-06	\\
-7692.24742542614	4.09359689925947e-06	\\
-7691.26864346591	4.20631812017111e-06	\\
-7690.28986150568	3.12652210215256e-06	\\
-7689.31107954545	4.45764977305961e-06	\\
-7688.33229758523	4.28292976312151e-06	\\
-7687.353515625	3.93110414571915e-06	\\
-7686.37473366477	3.74238914890005e-06	\\
-7685.39595170455	3.87465502019885e-06	\\
-7684.41716974432	3.90078892768004e-06	\\
-7683.43838778409	5.43337737793354e-06	\\
-7682.45960582386	4.65200465296081e-06	\\
-7681.48082386364	4.51297928333044e-06	\\
-7680.50204190341	4.16520828496909e-06	\\
-7679.52325994318	3.76061645585216e-06	\\
-7678.54447798295	2.7340793909629e-06	\\
-7677.56569602273	4.81614797464205e-06	\\
-7676.5869140625	4.07065295723604e-06	\\
-7675.60813210227	3.61409084716237e-06	\\
-7674.62935014205	3.99977465602232e-06	\\
-7673.65056818182	4.55422597174519e-06	\\
-7672.67178622159	4.95470070926906e-06	\\
-7671.69300426136	4.91426154338271e-06	\\
-7670.71422230114	4.27529894237362e-06	\\
-7669.73544034091	2.10107482764759e-06	\\
-7668.75665838068	4.72210833159117e-06	\\
-7667.77787642045	4.24744934960366e-06	\\
-7666.79909446023	3.20133858984376e-06	\\
-7665.8203125	3.85621761547107e-06	\\
-7664.84153053977	3.69681556443122e-06	\\
-7663.86274857955	4.78149509066182e-06	\\
-7662.88396661932	2.653884403519e-06	\\
-7661.90518465909	3.86708475906068e-06	\\
-7660.92640269886	4.90447495583144e-06	\\
-7659.94762073864	4.08155683028503e-06	\\
-7658.96883877841	1.85442369018306e-06	\\
-7657.99005681818	4.22191408084168e-06	\\
-7657.01127485795	4.47754003783897e-06	\\
-7656.03249289773	4.5850474256527e-06	\\
-7655.0537109375	4.42849162943661e-06	\\
-7654.07492897727	1.62777026192603e-06	\\
-7653.09614701705	5.58706200820448e-06	\\
-7652.11736505682	4.46567907122458e-06	\\
-7651.13858309659	5.15685895543199e-06	\\
-7650.15980113636	5.34532852713528e-06	\\
-7649.18101917614	4.81642287414884e-06	\\
-7648.20223721591	3.79753435933981e-06	\\
-7647.22345525568	3.58830053875123e-06	\\
-7646.24467329545	4.74788576829582e-06	\\
-7645.26589133523	3.52417757031737e-06	\\
-7644.287109375	2.84948814230443e-06	\\
-7643.30832741477	5.35507849625666e-06	\\
-7642.32954545455	3.6901464657307e-06	\\
-7641.35076349432	5.62411805369633e-06	\\
-7640.37198153409	4.10011638203443e-06	\\
-7639.39319957386	4.6764788397044e-06	\\
-7638.41441761364	4.61307410953056e-06	\\
-7637.43563565341	2.42813646683603e-06	\\
-7636.45685369318	4.5403922554975e-06	\\
-7635.47807173295	4.006786417681e-06	\\
-7634.49928977273	4.56186603792686e-06	\\
-7633.5205078125	2.45509981747207e-06	\\
-7632.54172585227	4.18981309727331e-06	\\
-7631.56294389205	5.0777017296725e-06	\\
-7630.58416193182	3.13579845275262e-06	\\
-7629.60537997159	5.21435470953109e-06	\\
-7628.62659801136	3.89992442850543e-06	\\
-7627.64781605114	4.92922571483509e-06	\\
-7626.66903409091	5.33775078500456e-06	\\
-7625.69025213068	5.44979654787558e-06	\\
-7624.71147017045	5.47611962933178e-06	\\
-7623.73268821023	6.62740790107727e-06	\\
-7622.75390625	3.42360299898322e-06	\\
-7621.77512428977	3.6227120069341e-06	\\
-7620.79634232955	2.46233482841621e-06	\\
-7619.81756036932	3.59290603120155e-06	\\
-7618.83877840909	3.99407895649451e-06	\\
-7617.85999644886	4.02426121597567e-06	\\
-7616.88121448864	3.94981227693274e-06	\\
-7615.90243252841	5.56574013213049e-06	\\
-7614.92365056818	6.41313204681477e-06	\\
-7613.94486860795	3.67246612244696e-06	\\
-7612.96608664773	5.76705411508653e-06	\\
-7611.9873046875	5.14655931937152e-06	\\
-7611.00852272727	3.4303957864252e-06	\\
-7610.02974076705	4.06166698253436e-06	\\
-7609.05095880682	4.84853802115915e-06	\\
-7608.07217684659	3.82572633451429e-06	\\
-7607.09339488636	5.40787266788987e-06	\\
-7606.11461292614	4.71230342354168e-06	\\
-7605.13583096591	3.92022497715903e-06	\\
-7604.15704900568	4.17354926565721e-06	\\
-7603.17826704545	4.18881037018395e-06	\\
-7602.19948508523	3.74784057920321e-06	\\
-7601.220703125	5.89352219803403e-06	\\
-7600.24192116477	4.27135765210192e-06	\\
-7599.26313920455	4.63510344992137e-06	\\
-7598.28435724432	4.49087887697452e-06	\\
-7597.30557528409	2.37427714411579e-06	\\
-7596.32679332386	4.92337388321958e-06	\\
-7595.34801136364	5.91710282389855e-06	\\
-7594.36922940341	4.78464747404864e-06	\\
-7593.39044744318	3.87004394508581e-06	\\
-7592.41166548295	5.50764738586513e-06	\\
-7591.43288352273	3.9855012501075e-06	\\
-7590.4541015625	4.53206265567111e-06	\\
-7589.47531960227	3.66176774506017e-06	\\
-7588.49653764205	3.7520689306491e-06	\\
-7587.51775568182	3.74990394916391e-06	\\
-7586.53897372159	5.00796710321266e-06	\\
-7585.56019176136	3.97733676116789e-06	\\
-7584.58140980114	4.66180699097234e-06	\\
-7583.60262784091	3.73214738875939e-06	\\
-7582.62384588068	4.58896372991298e-06	\\
-7581.64506392045	4.31599982339838e-06	\\
-7580.66628196023	5.19652705067176e-06	\\
-7579.6875	6.07974693784516e-06	\\
-7578.70871803977	2.65425148217887e-06	\\
-7577.72993607955	3.75967845040228e-06	\\
-7576.75115411932	5.65359790407393e-06	\\
-7575.77237215909	3.06551024123267e-06	\\
-7574.79359019886	4.48000878755261e-06	\\
-7573.81480823864	5.60508242442801e-06	\\
-7572.83602627841	3.8719040703187e-06	\\
-7571.85724431818	4.45077451533475e-06	\\
-7570.87846235795	5.12725392454989e-06	\\
-7569.89968039773	6.00360354399783e-06	\\
-7568.9208984375	1.70194309848029e-06	\\
-7567.94211647727	3.5701466867e-06	\\
-7566.96333451705	4.65594915972451e-06	\\
-7565.98455255682	2.7828822126662e-06	\\
-7565.00577059659	5.73968874228248e-06	\\
-7564.02698863636	4.94718032053381e-06	\\
-7563.04820667614	3.1225217725807e-06	\\
-7562.06942471591	5.11513910649809e-06	\\
-7561.09064275568	3.93573168652844e-06	\\
-7560.11186079545	2.50263250958826e-06	\\
-7559.13307883523	2.48124243402737e-06	\\
-7558.154296875	4.18233147446881e-06	\\
-7557.17551491477	2.76900873159877e-06	\\
-7556.19673295455	4.2219316832547e-06	\\
-7555.21795099432	5.60487290659301e-06	\\
-7554.23916903409	5.50154933041115e-06	\\
-7553.26038707386	3.80483634281238e-06	\\
-7552.28160511364	3.33939440990148e-06	\\
-7551.30282315341	4.72949325606651e-06	\\
-7550.32404119318	4.29817976431332e-06	\\
-7549.34525923295	2.55097938407784e-06	\\
-7548.36647727273	4.14128069456093e-06	\\
-7547.3876953125	4.08433042306726e-06	\\
-7546.40891335227	4.36514333777945e-06	\\
-7545.43013139205	3.48066677186207e-06	\\
-7544.45134943182	4.49201347873191e-06	\\
-7543.47256747159	3.63069587958068e-06	\\
-7542.49378551136	4.44765176472957e-06	\\
-7541.51500355114	4.85411512719485e-06	\\
-7540.53622159091	5.81106002718266e-06	\\
-7539.55743963068	4.13849856737795e-06	\\
-7538.57865767045	4.18498271610446e-06	\\
-7537.59987571023	4.54316982089048e-06	\\
-7536.62109375	4.75454579507797e-06	\\
-7535.64231178977	4.51874871154152e-06	\\
-7534.66352982955	4.44895194032244e-06	\\
-7533.68474786932	6.68784551646466e-06	\\
-7532.70596590909	5.57250773459329e-06	\\
-7531.72718394886	5.52215496141287e-06	\\
-7530.74840198864	4.58700038903706e-06	\\
-7529.76962002841	4.40182230915195e-06	\\
-7528.79083806818	4.22082893767708e-06	\\
-7527.81205610795	5.92016705180006e-06	\\
-7526.83327414773	5.61124368942171e-06	\\
-7525.8544921875	4.62467006793725e-06	\\
-7524.87571022727	5.01578295534371e-06	\\
-7523.89692826705	3.75258689180738e-06	\\
-7522.91814630682	5.46985458655243e-06	\\
-7521.93936434659	5.21088325766533e-06	\\
-7520.96058238636	5.33854233142122e-06	\\
-7519.98180042614	6.13432373972132e-06	\\
-7519.00301846591	5.43009235810254e-06	\\
-7518.02423650568	4.22610143300588e-06	\\
-7517.04545454545	4.46280736816697e-06	\\
-7516.06667258523	4.65336632076806e-06	\\
-7515.087890625	6.26985183100388e-06	\\
-7514.10910866477	5.09758191158809e-06	\\
-7513.13032670455	5.5078996742285e-06	\\
-7512.15154474432	4.75851665710801e-06	\\
-7511.17276278409	4.63557761045767e-06	\\
-7510.19398082386	4.96296010632828e-06	\\
-7509.21519886364	5.86940830232645e-06	\\
-7508.23641690341	4.2580367154282e-06	\\
-7507.25763494318	5.1692563935846e-06	\\
-7506.27885298295	4.5128871899865e-06	\\
-7505.30007102273	6.63834019971496e-06	\\
-7504.3212890625	4.32524613302753e-06	\\
-7503.34250710227	4.69982511451809e-06	\\
-7502.36372514205	3.05014191476273e-06	\\
-7501.38494318182	4.20956771407161e-06	\\
-7500.40616122159	6.37017825905959e-06	\\
-7499.42737926136	7.84357946747074e-06	\\
-7498.44859730114	5.79082206642018e-06	\\
-7497.46981534091	6.06597537715129e-06	\\
-7496.49103338068	5.52882612795589e-06	\\
-7495.51225142045	3.96782682636193e-06	\\
-7494.53346946023	4.31643715422768e-06	\\
-7493.5546875	6.03389221024544e-06	\\
-7492.57590553977	3.59133088572032e-06	\\
-7491.59712357955	4.07262639513861e-06	\\
-7490.61834161932	4.31161958945985e-06	\\
-7489.63955965909	3.01939800233069e-06	\\
-7488.66077769886	3.82141619577163e-06	\\
-7487.68199573864	4.35003716038716e-06	\\
-7486.70321377841	3.32860649137197e-06	\\
-7485.72443181818	4.53457735323544e-06	\\
-7484.74564985795	3.55222883631918e-06	\\
-7483.76686789773	3.78309470585254e-06	\\
-7482.7880859375	3.91602254365692e-06	\\
-7481.80930397727	3.62698730066872e-06	\\
-7480.83052201705	4.98241793309592e-06	\\
-7479.85174005682	4.59569171376656e-06	\\
-7478.87295809659	2.83356114553346e-06	\\
-7477.89417613636	2.63910397925408e-06	\\
-7476.91539417614	4.33527575337374e-06	\\
-7475.93661221591	4.02164660333881e-06	\\
-7474.95783025568	4.92522709553059e-06	\\
-7473.97904829545	3.08163089470188e-06	\\
-7473.00026633523	3.64812872042375e-06	\\
-7472.021484375	4.14663681480613e-06	\\
-7471.04270241477	3.61124724209352e-06	\\
-7470.06392045455	4.71379229883799e-06	\\
-7469.08513849432	3.98570656752896e-06	\\
-7468.10635653409	4.03171322755911e-06	\\
-7467.12757457386	4.92270625918633e-06	\\
-7466.14879261364	5.40363653642592e-06	\\
-7465.17001065341	4.39733318804492e-06	\\
-7464.19122869318	1.9883279240002e-06	\\
-7463.21244673295	5.17103477258006e-06	\\
-7462.23366477273	3.56903433638363e-06	\\
-7461.2548828125	4.27568691935e-06	\\
-7460.27610085227	3.17668808002343e-06	\\
-7459.29731889205	3.07386192788218e-06	\\
-7458.31853693182	3.54086846506586e-06	\\
-7457.33975497159	6.06808225205653e-06	\\
-7456.36097301136	4.4092718834956e-06	\\
-7455.38219105114	5.32253441603504e-06	\\
-7454.40340909091	4.23565793416046e-06	\\
-7453.42462713068	6.03065127013799e-06	\\
-7452.44584517045	5.32626703398889e-06	\\
-7451.46706321023	5.36580021213891e-06	\\
-7450.48828125	2.33704168944561e-06	\\
-7449.50949928977	6.8777854758642e-06	\\
-7448.53071732955	3.81359323343171e-06	\\
-7447.55193536932	6.23683740365732e-06	\\
-7446.57315340909	5.92785522511475e-06	\\
-7445.59437144886	7.83956973457039e-06	\\
-7444.61558948864	3.87403447504924e-06	\\
-7443.63680752841	4.52587857424582e-06	\\
-7442.65802556818	5.18133368194925e-06	\\
-7441.67924360795	4.62018039085558e-06	\\
-7440.70046164773	5.39426412212003e-06	\\
-7439.7216796875	3.94991508187455e-06	\\
-7438.74289772727	4.33433821883973e-06	\\
-7437.76411576705	5.57216090692897e-06	\\
-7436.78533380682	4.27351488872869e-06	\\
-7435.80655184659	5.6806372251235e-06	\\
-7434.82776988636	5.24917909419145e-06	\\
-7433.84898792614	4.55286667398749e-06	\\
-7432.87020596591	5.52678068559061e-06	\\
-7431.89142400568	6.42144790067013e-06	\\
-7430.91264204545	5.48260241264376e-06	\\
-7429.93386008523	7.3771411616295e-06	\\
-7428.955078125	3.95095627943618e-06	\\
-7427.97629616477	5.98273482242881e-06	\\
-7426.99751420455	5.58826091693333e-06	\\
-7426.01873224432	5.71957847888348e-06	\\
-7425.03995028409	6.37982177866927e-06	\\
-7424.06116832386	6.72292317818973e-06	\\
-7423.08238636364	4.99128006675008e-06	\\
-7422.10360440341	5.71578879732527e-06	\\
-7421.12482244318	5.17201211469855e-06	\\
-7420.14604048295	4.126986380686e-06	\\
-7419.16725852273	6.35807364942749e-06	\\
-7418.1884765625	6.36301084061331e-06	\\
-7417.20969460227	6.55954827066786e-06	\\
-7416.23091264205	4.70483858937576e-06	\\
-7415.25213068182	6.20523452022968e-06	\\
-7414.27334872159	4.21790938294673e-06	\\
-7413.29456676136	6.6216255841148e-06	\\
-7412.31578480114	5.61550343410083e-06	\\
-7411.33700284091	2.90278188898411e-06	\\
-7410.35822088068	6.16133896477355e-06	\\
-7409.37943892045	4.68829422373808e-06	\\
-7408.40065696023	5.41174804651997e-06	\\
-7407.421875	5.43545041997864e-06	\\
-7406.44309303977	3.36659760262065e-06	\\
-7405.46431107955	5.04685945558179e-06	\\
-7404.48552911932	6.27548004172095e-06	\\
-7403.50674715909	2.62168543003197e-06	\\
-7402.52796519886	4.66271984936084e-06	\\
-7401.54918323864	5.75256410081528e-06	\\
-7400.57040127841	5.70531235421055e-06	\\
-7399.59161931818	4.55267633569967e-06	\\
-7398.61283735795	4.20280094473796e-06	\\
-7397.63405539773	6.92971044259062e-06	\\
-7396.6552734375	6.55776022743622e-06	\\
-7395.67649147727	5.6349573243772e-06	\\
-7394.69770951705	6.65271652507997e-06	\\
-7393.71892755682	4.77704923642584e-06	\\
-7392.74014559659	3.64376626670725e-06	\\
-7391.76136363636	6.49572685809801e-06	\\
-7390.78258167614	6.44060523333739e-06	\\
-7389.80379971591	5.69470898964145e-06	\\
-7388.82501775568	6.37441486510835e-06	\\
-7387.84623579545	6.70297292936198e-06	\\
-7386.86745383523	3.69214133286593e-06	\\
-7385.888671875	5.98806930515918e-06	\\
-7384.90988991477	4.80435939880114e-06	\\
-7383.93110795455	4.43234492850008e-06	\\
-7382.95232599432	6.3230757015239e-06	\\
-7381.97354403409	4.62352974081509e-06	\\
-7380.99476207386	4.62149698849273e-06	\\
-7380.01598011364	4.47455363338233e-06	\\
-7379.03719815341	4.12580538239628e-06	\\
-7378.05841619318	3.50192850411342e-06	\\
-7377.07963423295	5.92706766272616e-06	\\
-7376.10085227273	6.11754641454362e-06	\\
-7375.1220703125	5.66125343766712e-06	\\
-7374.14328835227	3.53074214823303e-06	\\
-7373.16450639205	4.2498745971454e-06	\\
-7372.18572443182	3.40771502331951e-06	\\
-7371.20694247159	6.59263739813131e-06	\\
-7370.22816051136	5.82045405367903e-06	\\
-7369.24937855114	7.28944575541746e-06	\\
-7368.27059659091	3.77402135397824e-06	\\
-7367.29181463068	5.47098540516169e-06	\\
-7366.31303267045	5.13952327171812e-06	\\
-7365.33425071023	3.75737222685654e-06	\\
-7364.35546875	5.85498453247868e-06	\\
-7363.37668678977	7.83766998504582e-06	\\
-7362.39790482955	3.75900932752029e-06	\\
-7361.41912286932	4.16266025856394e-06	\\
-7360.44034090909	4.57223407750292e-06	\\
-7359.46155894886	3.04246569646537e-06	\\
-7358.48277698864	4.39287509297493e-06	\\
-7357.50399502841	4.30644035293221e-06	\\
-7356.52521306818	5.49096904570087e-06	\\
-7355.54643110795	3.81126311520601e-06	\\
-7354.56764914773	6.03582332259435e-06	\\
-7353.5888671875	7.09958585065447e-06	\\
-7352.61008522727	6.27049524797653e-06	\\
-7351.63130326705	6.18657358749274e-06	\\
-7350.65252130682	6.01850881682164e-06	\\
-7349.67373934659	5.08706229065823e-06	\\
-7348.69495738636	4.99510464890396e-06	\\
-7347.71617542614	4.76664400145056e-06	\\
-7346.73739346591	4.20253629876795e-06	\\
-7345.75861150568	5.60045947911637e-06	\\
-7344.77982954545	4.60197945084053e-06	\\
-7343.80104758523	4.28842486001295e-06	\\
-7342.822265625	6.19034327857331e-06	\\
-7341.84348366477	6.48712397562185e-06	\\
-7340.86470170455	5.08261708251055e-06	\\
-7339.88591974432	4.7857867548349e-06	\\
-7338.90713778409	5.60269344669455e-06	\\
-7337.92835582386	3.86963500057275e-06	\\
-7336.94957386364	4.06591964934346e-06	\\
-7335.97079190341	4.97055406634593e-06	\\
-7334.99200994318	5.49551013708553e-06	\\
-7334.01322798295	4.54898597443027e-06	\\
-7333.03444602273	6.20124130241944e-06	\\
-7332.0556640625	5.09782188618115e-06	\\
-7331.07688210227	5.82811211448095e-06	\\
-7330.09810014205	6.81160196326671e-06	\\
-7329.11931818182	5.73063726081646e-06	\\
-7328.14053622159	3.67011802284032e-06	\\
-7327.16175426136	4.11307042332488e-06	\\
-7326.18297230114	6.11432584455409e-06	\\
-7325.20419034091	5.22171980104129e-06	\\
-7324.22540838068	5.30073041313878e-06	\\
-7323.24662642045	6.47664576304779e-06	\\
-7322.26784446023	6.78738919096852e-06	\\
-7321.2890625	5.21990516212425e-06	\\
-7320.31028053977	4.13797475812765e-06	\\
-7319.33149857955	3.18845843102239e-06	\\
-7318.35271661932	5.91932897876754e-06	\\
-7317.37393465909	5.88362883326509e-06	\\
-7316.39515269886	5.63103412880109e-06	\\
-7315.41637073864	5.77113170149396e-06	\\
-7314.43758877841	5.06046391725548e-06	\\
-7313.45880681818	4.83877063129812e-06	\\
-7312.48002485795	4.5354501185797e-06	\\
-7311.50124289773	6.3856660969621e-06	\\
-7310.5224609375	4.88257498783354e-06	\\
-7309.54367897727	3.28520442838928e-06	\\
-7308.56489701705	6.50550940970217e-06	\\
-7307.58611505682	5.81579563841884e-06	\\
-7306.60733309659	7.20560793479691e-06	\\
-7305.62855113636	4.92618056942603e-06	\\
-7304.64976917614	4.68340994711553e-06	\\
-7303.67098721591	5.57007003100621e-06	\\
-7302.69220525568	6.22949927544613e-06	\\
-7301.71342329545	6.90456651377604e-06	\\
-7300.73464133523	5.04356469764049e-06	\\
-7299.755859375	8.49609671558767e-06	\\
-7298.77707741477	6.53756474411316e-06	\\
-7297.79829545455	7.01450720084982e-06	\\
-7296.81951349432	4.79934633118096e-06	\\
-7295.84073153409	3.84988465060479e-06	\\
-7294.86194957386	8.08672907454759e-06	\\
-7293.88316761364	4.97687869165066e-06	\\
-7292.90438565341	5.18029364766841e-06	\\
-7291.92560369318	5.90960939020471e-06	\\
-7290.94682173295	5.19876324772159e-06	\\
-7289.96803977273	7.04207992874677e-06	\\
-7288.9892578125	5.90567241850227e-06	\\
-7288.01047585227	5.19449161762797e-06	\\
-7287.03169389205	6.54718028022432e-06	\\
-7286.05291193182	5.65274390480751e-06	\\
-7285.07412997159	5.97189554869655e-06	\\
-7284.09534801136	6.92310088667202e-06	\\
-7283.11656605114	5.84525388599957e-06	\\
-7282.13778409091	6.61111805910105e-06	\\
-7281.15900213068	6.04782243438953e-06	\\
-7280.18022017045	6.29129775074323e-06	\\
-7279.20143821023	5.28994317737088e-06	\\
-7278.22265625	7.75683085661251e-06	\\
-7277.24387428977	6.08451386080312e-06	\\
-7276.26509232955	5.24364300361165e-06	\\
-7275.28631036932	5.23825693155155e-06	\\
-7274.30752840909	7.69410203386553e-06	\\
-7273.32874644886	6.44028276979984e-06	\\
-7272.34996448864	7.89233400325049e-06	\\
-7271.37118252841	4.0653290986326e-06	\\
-7270.39240056818	5.99122649754034e-06	\\
-7269.41361860795	5.65970485058254e-06	\\
-7268.43483664773	6.31351610604241e-06	\\
-7267.4560546875	5.88707651721158e-06	\\
-7266.47727272727	5.93826576243472e-06	\\
-7265.49849076705	5.55078551710015e-06	\\
-7264.51970880682	7.59737054797787e-06	\\
-7263.54092684659	4.87126287621297e-06	\\
-7262.56214488636	5.36777772080135e-06	\\
-7261.58336292614	8.22337358894503e-06	\\
-7260.60458096591	6.25304054444838e-06	\\
-7259.62579900568	6.59645577891387e-06	\\
-7258.64701704545	7.29558725188643e-06	\\
-7257.66823508523	6.4836127661686e-06	\\
-7256.689453125	6.67660232302533e-06	\\
-7255.71067116477	5.36214195609306e-06	\\
-7254.73188920455	5.28821327355609e-06	\\
-7253.75310724432	6.37938320564234e-06	\\
-7252.77432528409	8.86314796228757e-06	\\
-7251.79554332386	4.83951943638927e-06	\\
-7250.81676136364	4.45210568483488e-06	\\
-7249.83797940341	3.90134954226128e-06	\\
-7248.85919744318	6.00747094544544e-06	\\
-7247.88041548295	6.78595053231277e-06	\\
-7246.90163352273	6.88781159856256e-06	\\
-7245.9228515625	6.44923201546063e-06	\\
-7244.94406960227	6.52818103537212e-06	\\
-7243.96528764205	7.46762313125893e-06	\\
-7242.98650568182	7.43972270964175e-06	\\
-7242.00772372159	5.20333674930356e-06	\\
-7241.02894176136	5.76505436238659e-06	\\
-7240.05015980114	6.94801747759374e-06	\\
-7239.07137784091	5.3729366445403e-06	\\
-7238.09259588068	7.87920873387898e-06	\\
-7237.11381392045	6.71949917802931e-06	\\
-7236.13503196023	7.12003526979881e-06	\\
-7235.15625	6.29830834450768e-06	\\
-7234.17746803977	3.33577170446266e-06	\\
-7233.19868607955	8.16578409368431e-06	\\
-7232.21990411932	6.27480475475747e-06	\\
-7231.24112215909	5.78791793756544e-06	\\
-7230.26234019886	7.854569517969e-06	\\
-7229.28355823864	4.22612778975569e-06	\\
-7228.30477627841	6.41056863149036e-06	\\
-7227.32599431818	7.8287058389997e-06	\\
-7226.34721235795	7.75192002061109e-06	\\
-7225.36843039773	7.01158503074477e-06	\\
-7224.3896484375	4.52825393625191e-06	\\
-7223.41086647727	7.66004394122578e-06	\\
-7222.43208451705	5.67203444822749e-06	\\
-7221.45330255682	5.83931083165451e-06	\\
-7220.47452059659	5.53289083092619e-06	\\
-7219.49573863636	6.00364557122267e-06	\\
-7218.51695667614	6.64700051891178e-06	\\
-7217.53817471591	6.04126342541703e-06	\\
-7216.55939275568	6.45783605652371e-06	\\
-7215.58061079545	7.4554910806769e-06	\\
-7214.60182883523	7.15949685390977e-06	\\
-7213.623046875	5.6901688363231e-06	\\
-7212.64426491477	7.40424955192906e-06	\\
-7211.66548295455	6.72206188613146e-06	\\
-7210.68670099432	5.81618181275752e-06	\\
-7209.70791903409	6.34747031062573e-06	\\
-7208.72913707386	6.95292260704011e-06	\\
-7207.75035511364	7.79517052533959e-06	\\
-7206.77157315341	5.82047959350547e-06	\\
-7205.79279119318	7.38179560118472e-06	\\
-7204.81400923295	7.82610948429572e-06	\\
-7203.83522727273	7.12276890463046e-06	\\
-7202.8564453125	7.44475342166574e-06	\\
-7201.87766335227	7.97298583655694e-06	\\
-7200.89888139205	6.79965051638e-06	\\
-7199.92009943182	9.24680614621822e-06	\\
-7198.94131747159	8.13458770625187e-06	\\
-7197.96253551136	6.19471759690052e-06	\\
-7196.98375355114	6.66674505983161e-06	\\
-7196.00497159091	7.79704756933837e-06	\\
-7195.02618963068	7.41678570623166e-06	\\
-7194.04740767045	8.67138707455109e-06	\\
-7193.06862571023	7.38775251478698e-06	\\
-7192.08984375	6.68679622827646e-06	\\
-7191.11106178977	6.91174330259568e-06	\\
-7190.13227982955	6.50674960095784e-06	\\
-7189.15349786932	8.17511408248838e-06	\\
-7188.17471590909	7.34798700097976e-06	\\
-7187.19593394886	7.03197449267639e-06	\\
-7186.21715198864	7.2880023229128e-06	\\
-7185.23837002841	7.81221746365987e-06	\\
-7184.25958806818	7.81249625639423e-06	\\
-7183.28080610795	6.1719514933079e-06	\\
-7182.30202414773	7.04777377014604e-06	\\
-7181.3232421875	6.18431350402293e-06	\\
-7180.34446022727	5.44764372649933e-06	\\
-7179.36567826705	7.01767543620656e-06	\\
-7178.38689630682	8.45243179096142e-06	\\
-7177.40811434659	8.80843180526465e-06	\\
-7176.42933238636	7.88591648747968e-06	\\
-7175.45055042614	7.05490683209751e-06	\\
-7174.47176846591	7.25719045321533e-06	\\
-7173.49298650568	7.00993475660987e-06	\\
-7172.51420454545	7.04346038885424e-06	\\
-7171.53542258523	8.67074961294718e-06	\\
-7170.556640625	7.11407844363775e-06	\\
-7169.57785866477	7.12648803974074e-06	\\
-7168.59907670455	7.75841253788997e-06	\\
-7167.62029474432	8.64227004670672e-06	\\
-7166.64151278409	8.90974234978848e-06	\\
-7165.66273082386	7.47254179461801e-06	\\
-7164.68394886364	6.29433254772897e-06	\\
-7163.70516690341	9.14313752330234e-06	\\
-7162.72638494318	5.8369136368988e-06	\\
-7161.74760298295	7.09731050924802e-06	\\
-7160.76882102273	6.93859777556614e-06	\\
-7159.7900390625	5.91061060308913e-06	\\
-7158.81125710227	6.18422137420537e-06	\\
-7157.83247514205	7.89113410913996e-06	\\
-7156.85369318182	5.83366395503756e-06	\\
-7155.87491122159	6.36368741139652e-06	\\
-7154.89612926136	8.49693481973793e-06	\\
-7153.91734730114	7.15464667012756e-06	\\
-7152.93856534091	4.32513843623343e-06	\\
-7151.95978338068	5.95194029451851e-06	\\
-7150.98100142045	6.85116427219099e-06	\\
-7150.00221946023	7.14635142567354e-06	\\
-7149.0234375	7.64926882342665e-06	\\
-7148.04465553977	6.63476282763165e-06	\\
-7147.06587357955	6.28779183036578e-06	\\
-7146.08709161932	7.50718741017012e-06	\\
-7145.10830965909	7.5304164531287e-06	\\
-7144.12952769886	6.59339382048789e-06	\\
-7143.15074573864	6.02231257063746e-06	\\
-7142.17196377841	4.64300110113398e-06	\\
-7141.19318181818	7.27433471736708e-06	\\
-7140.21439985795	5.52155355895944e-06	\\
-7139.23561789773	7.42243583336104e-06	\\
-7138.2568359375	6.89533337210247e-06	\\
-7137.27805397727	7.85553201392156e-06	\\
-7136.29927201705	8.44721352546233e-06	\\
-7135.32049005682	7.83946718919924e-06	\\
-7134.34170809659	6.34212407245676e-06	\\
-7133.36292613636	6.93200039046396e-06	\\
-7132.38414417614	7.83551053568814e-06	\\
-7131.40536221591	6.47878261953394e-06	\\
-7130.42658025568	8.15700079191428e-06	\\
-7129.44779829545	7.70839866421907e-06	\\
-7128.46901633523	7.41538567388927e-06	\\
-7127.490234375	7.74914816712916e-06	\\
-7126.51145241477	7.56855503300748e-06	\\
-7125.53267045455	6.21928652954467e-06	\\
-7124.55388849432	7.28247245617063e-06	\\
-7123.57510653409	7.21929135702035e-06	\\
-7122.59632457386	5.6504466301135e-06	\\
-7121.61754261364	6.45531390502483e-06	\\
-7120.63876065341	5.9592760723336e-06	\\
-7119.65997869318	6.58507167394266e-06	\\
-7118.68119673295	9.1984243085036e-06	\\
-7117.70241477273	8.18352491359644e-06	\\
-7116.7236328125	6.82304740189407e-06	\\
-7115.74485085227	6.54730538337779e-06	\\
-7114.76606889205	6.19047827160818e-06	\\
-7113.78728693182	6.05406338560555e-06	\\
-7112.80850497159	7.64502854441088e-06	\\
-7111.82972301136	6.7195250744904e-06	\\
-7110.85094105114	5.53470080144032e-06	\\
-7109.87215909091	7.35924855132916e-06	\\
};
\addplot [color=blue,solid,forget plot]
  table[row sep=crcr]{
-7109.87215909091	7.35924855132916e-06	\\
-7108.89337713068	8.25129761552966e-06	\\
-7107.91459517045	7.33495237912972e-06	\\
-7106.93581321023	7.29650812167973e-06	\\
-7105.95703125	4.23045525030101e-06	\\
-7104.97824928977	6.92540660605786e-06	\\
-7103.99946732955	8.18434822351605e-06	\\
-7103.02068536932	6.85485049777401e-06	\\
-7102.04190340909	6.02326800687085e-06	\\
-7101.06312144886	6.75571418529659e-06	\\
-7100.08433948864	1.06072978387776e-05	\\
-7099.10555752841	7.12051467560178e-06	\\
-7098.12677556818	8.44456323026646e-06	\\
-7097.14799360795	5.39229078480982e-06	\\
-7096.16921164773	7.43874071960211e-06	\\
-7095.1904296875	6.82064111924879e-06	\\
-7094.21164772727	7.255072570526e-06	\\
-7093.23286576705	6.90132557031644e-06	\\
-7092.25408380682	7.04320520345721e-06	\\
-7091.27530184659	5.30042095354619e-06	\\
-7090.29651988636	6.53529197226998e-06	\\
-7089.31773792614	8.42722832204058e-06	\\
-7088.33895596591	7.17010744103538e-06	\\
-7087.36017400568	6.66483952597213e-06	\\
-7086.38139204545	6.79340955235815e-06	\\
-7085.40261008523	7.2454034237409e-06	\\
-7084.423828125	8.50460717088626e-06	\\
-7083.44504616477	9.09688148279616e-06	\\
-7082.46626420455	8.86480579213838e-06	\\
-7081.48748224432	7.34426885208812e-06	\\
-7080.50870028409	7.12013179951134e-06	\\
-7079.52991832386	6.86778083116649e-06	\\
-7078.55113636364	9.6157427519997e-06	\\
-7077.57235440341	8.82951068363649e-06	\\
-7076.59357244318	7.06720866045392e-06	\\
-7075.61479048295	8.23200434922432e-06	\\
-7074.63600852273	8.0381226764478e-06	\\
-7073.6572265625	7.81019664843505e-06	\\
-7072.67844460227	6.09788685597578e-06	\\
-7071.69966264205	6.16817558238271e-06	\\
-7070.72088068182	8.54827420746701e-06	\\
-7069.74209872159	8.60451664407381e-06	\\
-7068.76331676136	7.32919445873412e-06	\\
-7067.78453480114	7.65630503719013e-06	\\
-7066.80575284091	7.25830531489625e-06	\\
-7065.82697088068	7.46581583281739e-06	\\
-7064.84818892045	6.07381548000757e-06	\\
-7063.86940696023	5.20365264696553e-06	\\
-7062.890625	8.87485961764247e-06	\\
-7061.91184303977	7.83189091081941e-06	\\
-7060.93306107955	8.72059985890183e-06	\\
-7059.95427911932	8.98349981078117e-06	\\
-7058.97549715909	1.03421246938909e-05	\\
-7057.99671519886	1.02657956956678e-05	\\
-7057.01793323864	8.09776152664792e-06	\\
-7056.03915127841	9.21016668132798e-06	\\
-7055.06036931818	8.23578046941164e-06	\\
-7054.08158735795	6.69765307245801e-06	\\
-7053.10280539773	9.32986717022498e-06	\\
-7052.1240234375	8.49483442705152e-06	\\
-7051.14524147727	7.48788955150821e-06	\\
-7050.16645951705	9.77230723587304e-06	\\
-7049.18767755682	7.12711149844373e-06	\\
-7048.20889559659	7.95654977458009e-06	\\
-7047.23011363636	9.62605041039973e-06	\\
-7046.25133167614	8.85728441553021e-06	\\
-7045.27254971591	8.23169489675528e-06	\\
-7044.29376775568	7.28914975917798e-06	\\
-7043.31498579545	6.85317376648453e-06	\\
-7042.33620383523	6.42530339311496e-06	\\
-7041.357421875	8.29434870289765e-06	\\
-7040.37863991477	7.94639180864503e-06	\\
-7039.39985795455	9.11314765243033e-06	\\
-7038.42107599432	7.77048797198546e-06	\\
-7037.44229403409	8.59479523774439e-06	\\
-7036.46351207386	8.24111703676664e-06	\\
-7035.48473011364	5.14814316730946e-06	\\
-7034.50594815341	8.21348257870197e-06	\\
-7033.52716619318	7.85167938186288e-06	\\
-7032.54838423295	8.34377044444961e-06	\\
-7031.56960227273	9.09038903764478e-06	\\
-7030.5908203125	6.69810853064447e-06	\\
-7029.61203835227	7.47336950654812e-06	\\
-7028.63325639205	7.64816255502259e-06	\\
-7027.65447443182	8.21892572797298e-06	\\
-7026.67569247159	7.38876467298259e-06	\\
-7025.69691051136	8.66931014608045e-06	\\
-7024.71812855114	8.82112059277644e-06	\\
-7023.73934659091	6.87282233332515e-06	\\
-7022.76056463068	8.04564673270359e-06	\\
-7021.78178267045	8.37496285335695e-06	\\
-7020.80300071023	7.43891932936118e-06	\\
-7019.82421875	8.78920000468643e-06	\\
-7018.84543678977	6.64364429853281e-06	\\
-7017.86665482955	6.75984623661704e-06	\\
-7016.88787286932	8.87290043550987e-06	\\
-7015.90909090909	9.06522000291515e-06	\\
-7014.93030894886	8.17194475874678e-06	\\
-7013.95152698864	6.76973294410322e-06	\\
-7012.97274502841	7.76399948890574e-06	\\
-7011.99396306818	7.30341144034841e-06	\\
-7011.01518110795	7.7834259684121e-06	\\
-7010.03639914773	7.28816541244749e-06	\\
-7009.0576171875	7.23404720537692e-06	\\
-7008.07883522727	9.56792534345002e-06	\\
-7007.10005326705	8.45486729541963e-06	\\
-7006.12127130682	7.4167853276853e-06	\\
-7005.14248934659	8.11451249283504e-06	\\
-7004.16370738636	7.92708132751626e-06	\\
-7003.18492542614	9.32535449536314e-06	\\
-7002.20614346591	9.40479335931138e-06	\\
-7001.22736150568	9.82844048473676e-06	\\
-7000.24857954545	8.87079832640022e-06	\\
-6999.26979758523	8.57633627225182e-06	\\
-6998.291015625	7.76634498674089e-06	\\
-6997.31223366477	8.93632573742766e-06	\\
-6996.33345170455	7.36009896695807e-06	\\
-6995.35466974432	7.55413798790209e-06	\\
-6994.37588778409	7.46161236239331e-06	\\
-6993.39710582386	7.17076585925017e-06	\\
-6992.41832386364	7.75235499387994e-06	\\
-6991.43954190341	9.0873400585358e-06	\\
-6990.46075994318	6.55092946677137e-06	\\
-6989.48197798295	8.21347894498842e-06	\\
-6988.50319602273	8.12982988163961e-06	\\
-6987.5244140625	6.46705596761305e-06	\\
-6986.54563210227	8.27395137749764e-06	\\
-6985.56685014205	7.44995894994569e-06	\\
-6984.58806818182	7.99808508894917e-06	\\
-6983.60928622159	7.18212710333978e-06	\\
-6982.63050426136	9.23356865819495e-06	\\
-6981.65172230114	8.49717071867544e-06	\\
-6980.67294034091	9.92741257887067e-06	\\
-6979.69415838068	9.12519921721257e-06	\\
-6978.71537642045	6.24783804648671e-06	\\
-6977.73659446023	1.0022419169085e-05	\\
-6976.7578125	7.58593811037188e-06	\\
-6975.77903053977	8.90300079694572e-06	\\
-6974.80024857955	8.61757862407703e-06	\\
-6973.82146661932	9.72315530217442e-06	\\
-6972.84268465909	7.99626937498038e-06	\\
-6971.86390269886	8.59255720728475e-06	\\
-6970.88512073864	9.08497834666059e-06	\\
-6969.90633877841	1.09456602069691e-05	\\
-6968.92755681818	9.80112814729426e-06	\\
-6967.94877485795	9.28874842310321e-06	\\
-6966.96999289773	8.71577777647897e-06	\\
-6965.9912109375	9.11959004046721e-06	\\
-6965.01242897727	8.81046407487184e-06	\\
-6964.03364701705	8.97298813342166e-06	\\
-6963.05486505682	8.32895115752831e-06	\\
-6962.07608309659	9.07136724411174e-06	\\
-6961.09730113636	9.00168047107274e-06	\\
-6960.11851917614	7.09746588892635e-06	\\
-6959.13973721591	8.28177102330892e-06	\\
-6958.16095525568	9.23092128304258e-06	\\
-6957.18217329545	8.47951655294338e-06	\\
-6956.20339133523	7.36163813116578e-06	\\
-6955.224609375	8.68099944844862e-06	\\
-6954.24582741477	8.4028772467489e-06	\\
-6953.26704545455	8.99308173190414e-06	\\
-6952.28826349432	1.0297084435329e-05	\\
-6951.30948153409	8.80755690808172e-06	\\
-6950.33069957386	7.46388039895635e-06	\\
-6949.35191761364	9.26575585199599e-06	\\
-6948.37313565341	8.7281312304348e-06	\\
-6947.39435369318	8.78612783424607e-06	\\
-6946.41557173295	9.70222537775898e-06	\\
-6945.43678977273	7.51601602234304e-06	\\
-6944.4580078125	9.31166230298962e-06	\\
-6943.47922585227	9.71922348726519e-06	\\
-6942.50044389205	7.51987118591076e-06	\\
-6941.52166193182	8.53711149533544e-06	\\
-6940.54287997159	8.19274714860983e-06	\\
-6939.56409801136	8.8607462248829e-06	\\
-6938.58531605114	9.32364116767374e-06	\\
-6937.60653409091	8.46480561456296e-06	\\
-6936.62775213068	8.49327366611013e-06	\\
-6935.64897017045	7.67420462893607e-06	\\
-6934.67018821023	8.63070844514547e-06	\\
-6933.69140625	8.99933742317623e-06	\\
-6932.71262428977	7.49436976698643e-06	\\
-6931.73384232955	1.01834203940139e-05	\\
-6930.75506036932	8.04675614558912e-06	\\
-6929.77627840909	8.72167971146345e-06	\\
-6928.79749644886	6.98581029093473e-06	\\
-6927.81871448864	9.49753940410925e-06	\\
-6926.83993252841	9.69290798845811e-06	\\
-6925.86115056818	9.22904649092858e-06	\\
-6924.88236860796	8.97028175867884e-06	\\
-6923.90358664773	1.06824369348199e-05	\\
-6922.9248046875	8.99317318594291e-06	\\
-6921.94602272727	1.01860711371339e-05	\\
-6920.96724076705	1.02003517451901e-05	\\
-6919.98845880682	9.68600902913366e-06	\\
-6919.00967684659	1.11397530729292e-05	\\
-6918.03089488636	8.82141556750206e-06	\\
-6917.05211292614	9.00203344235214e-06	\\
-6916.07333096591	8.37653669560228e-06	\\
-6915.09454900568	8.02298186007617e-06	\\
-6914.11576704546	7.77361763537613e-06	\\
-6913.13698508523	1.11699417711737e-05	\\
-6912.158203125	8.01146520020537e-06	\\
-6911.17942116477	8.69338778282981e-06	\\
-6910.20063920455	9.5006140856838e-06	\\
-6909.22185724432	8.31462803022397e-06	\\
-6908.24307528409	1.01458861258545e-05	\\
-6907.26429332386	1.03298588080146e-05	\\
-6906.28551136364	1.07472164462807e-05	\\
-6905.30672940341	8.37679704188218e-06	\\
-6904.32794744318	1.02198987322786e-05	\\
-6903.34916548296	1.0101199172237e-05	\\
-6902.37038352273	1.0921560324706e-05	\\
-6901.3916015625	1.05532708762451e-05	\\
-6900.41281960227	8.44041352181143e-06	\\
-6899.43403764205	1.05392318192771e-05	\\
-6898.45525568182	8.65060477833337e-06	\\
-6897.47647372159	1.00829493006774e-05	\\
-6896.49769176136	9.23292796466293e-06	\\
-6895.51890980114	8.96183075003254e-06	\\
-6894.54012784091	9.44278689112294e-06	\\
-6893.56134588068	9.24168869775202e-06	\\
-6892.58256392046	8.8151228879249e-06	\\
-6891.60378196023	8.72428033607437e-06	\\
-6890.625	9.29258582281622e-06	\\
-6889.64621803977	8.24362109628806e-06	\\
-6888.66743607955	1.01932547874952e-05	\\
-6887.68865411932	9.88551107455513e-06	\\
-6886.70987215909	7.5852183758129e-06	\\
-6885.73109019886	7.22908774253638e-06	\\
-6884.75230823864	1.08553407060873e-05	\\
-6883.77352627841	9.09262026337794e-06	\\
-6882.79474431818	1.01089741796325e-05	\\
-6881.81596235796	1.06566127534164e-05	\\
-6880.83718039773	9.92067207963429e-06	\\
-6879.8583984375	1.07661642962295e-05	\\
-6878.87961647727	1.0178425432694e-05	\\
-6877.90083451705	7.36001068195214e-06	\\
-6876.92205255682	1.08643785561331e-05	\\
-6875.94327059659	7.87896865487354e-06	\\
-6874.96448863636	9.07718815642694e-06	\\
-6873.98570667614	1.03866166119185e-05	\\
-6873.00692471591	1.00053798228432e-05	\\
-6872.02814275568	9.28901291908604e-06	\\
-6871.04936079546	1.01331049912772e-05	\\
-6870.07057883523	1.05697648808878e-05	\\
-6869.091796875	8.8412768792666e-06	\\
-6868.11301491477	8.92419247900583e-06	\\
-6867.13423295455	7.19327585647972e-06	\\
-6866.15545099432	1.02464982656099e-05	\\
-6865.17666903409	1.00747478712198e-05	\\
-6864.19788707386	7.64340853197388e-06	\\
-6863.21910511364	9.22862602151231e-06	\\
-6862.24032315341	8.26134282434608e-06	\\
-6861.26154119318	8.47464887955442e-06	\\
-6860.28275923296	9.64251138279133e-06	\\
-6859.30397727273	8.42474129676289e-06	\\
-6858.3251953125	8.72808842958726e-06	\\
-6857.34641335227	9.21358098440631e-06	\\
-6856.36763139205	9.18244832943715e-06	\\
-6855.38884943182	1.09698946287161e-05	\\
-6854.41006747159	9.97934050798025e-06	\\
-6853.43128551136	9.8022175590626e-06	\\
-6852.45250355114	9.5121037967609e-06	\\
-6851.47372159091	8.87282988064946e-06	\\
-6850.49493963068	1.17560562579236e-05	\\
-6849.51615767046	8.49904101631058e-06	\\
-6848.53737571023	1.09267461738759e-05	\\
-6847.55859375	7.58495106702352e-06	\\
-6846.57981178977	1.01235245044894e-05	\\
-6845.60102982955	9.8616684430908e-06	\\
-6844.62224786932	8.8630983721806e-06	\\
-6843.64346590909	9.8195008136711e-06	\\
-6842.66468394886	1.13884509066025e-05	\\
-6841.68590198864	9.23607878188403e-06	\\
-6840.70712002841	1.02120225435867e-05	\\
-6839.72833806818	9.07800174921569e-06	\\
-6838.74955610796	1.02492339568958e-05	\\
-6837.77077414773	1.00588753232353e-05	\\
-6836.7919921875	9.8179034630573e-06	\\
-6835.81321022727	7.69464368632543e-06	\\
-6834.83442826705	7.84073537880776e-06	\\
-6833.85564630682	7.7142257705185e-06	\\
-6832.87686434659	9.39040977152764e-06	\\
-6831.89808238636	8.59527235257778e-06	\\
-6830.91930042614	9.7597348904039e-06	\\
-6829.94051846591	8.48917922823713e-06	\\
-6828.96173650568	9.2388431827356e-06	\\
-6827.98295454546	1.08683805510451e-05	\\
-6827.00417258523	8.93949578892115e-06	\\
-6826.025390625	9.24018210409258e-06	\\
-6825.04660866477	9.37135434907956e-06	\\
-6824.06782670455	9.82682772225835e-06	\\
-6823.08904474432	9.17061578893449e-06	\\
-6822.11026278409	1.08096628972136e-05	\\
-6821.13148082386	8.69418525795146e-06	\\
-6820.15269886364	1.07205139215519e-05	\\
-6819.17391690341	8.46961780899977e-06	\\
-6818.19513494318	1.12971415562093e-05	\\
-6817.21635298296	9.23871928320269e-06	\\
-6816.23757102273	8.71267112554366e-06	\\
-6815.2587890625	1.08783359221289e-05	\\
-6814.28000710227	8.19952437752894e-06	\\
-6813.30122514205	9.91513903866378e-06	\\
-6812.32244318182	1.08530704275437e-05	\\
-6811.34366122159	9.85721579418695e-06	\\
-6810.36487926136	9.48308600842679e-06	\\
-6809.38609730114	8.88505229857698e-06	\\
-6808.40731534091	1.06278746380527e-05	\\
-6807.42853338068	1.0930826141547e-05	\\
-6806.44975142046	9.56501110293176e-06	\\
-6805.47096946023	9.71238062138811e-06	\\
-6804.4921875	8.49140584265952e-06	\\
-6803.51340553977	1.01970783803324e-05	\\
-6802.53462357955	9.26662113384368e-06	\\
-6801.55584161932	1.02493151348023e-05	\\
-6800.57705965909	8.62319598249204e-06	\\
-6799.59827769886	8.19321730433703e-06	\\
-6798.61949573864	1.04824722560887e-05	\\
-6797.64071377841	1.00269202664532e-05	\\
-6796.66193181818	7.96310725862634e-06	\\
-6795.68314985796	1.10260304820986e-05	\\
-6794.70436789773	1.11319950755916e-05	\\
-6793.7255859375	8.82297063706853e-06	\\
-6792.74680397727	8.0380034535093e-06	\\
-6791.76802201705	9.07231432033201e-06	\\
-6790.78924005682	9.0924790574116e-06	\\
-6789.81045809659	9.27959618279247e-06	\\
-6788.83167613636	9.77933796695317e-06	\\
-6787.85289417614	9.49965129669802e-06	\\
-6786.87411221591	8.38339914611323e-06	\\
-6785.89533025568	8.41019672119945e-06	\\
-6784.91654829546	1.02333762171423e-05	\\
-6783.93776633523	1.18209902087948e-05	\\
-6782.958984375	1.01186796112958e-05	\\
-6781.98020241477	9.97748018869789e-06	\\
-6781.00142045455	9.10558867240921e-06	\\
-6780.02263849432	8.9962281413815e-06	\\
-6779.04385653409	1.02738059827678e-05	\\
-6778.06507457386	1.08803189715766e-05	\\
-6777.08629261364	9.08246075228492e-06	\\
-6776.10751065341	9.00339156253311e-06	\\
-6775.12872869318	1.0971859247878e-05	\\
-6774.14994673296	9.3891147961578e-06	\\
-6773.17116477273	8.52905742643099e-06	\\
-6772.1923828125	9.36726744366475e-06	\\
-6771.21360085227	8.25833830520883e-06	\\
-6770.23481889205	1.09437689708816e-05	\\
-6769.25603693182	9.51969350357448e-06	\\
-6768.27725497159	9.44449578580569e-06	\\
-6767.29847301136	1.02130802398973e-05	\\
-6766.31969105114	6.9119674760777e-06	\\
-6765.34090909091	9.50063840155734e-06	\\
-6764.36212713068	1.15301300690708e-05	\\
-6763.38334517046	9.34507806214105e-06	\\
-6762.40456321023	8.08273913100568e-06	\\
-6761.42578125	1.02014475584547e-05	\\
-6760.44699928977	9.38969751345229e-06	\\
-6759.46821732955	1.0343254598511e-05	\\
-6758.48943536932	1.08649321435988e-05	\\
-6757.51065340909	1.18942493654369e-05	\\
-6756.53187144886	9.58345410279519e-06	\\
-6755.55308948864	8.6731573914278e-06	\\
-6754.57430752841	9.08714144086891e-06	\\
-6753.59552556818	1.0347976960432e-05	\\
-6752.61674360796	9.941311157474e-06	\\
-6751.63796164773	1.07538302806613e-05	\\
-6750.6591796875	1.07829332089805e-05	\\
-6749.68039772727	9.71146783098741e-06	\\
-6748.70161576705	9.79331479679848e-06	\\
-6747.72283380682	9.53451726382814e-06	\\
-6746.74405184659	1.11154040242839e-05	\\
-6745.76526988636	9.26685459998644e-06	\\
-6744.78648792614	1.05554194082495e-05	\\
-6743.80770596591	8.96705891369385e-06	\\
-6742.82892400568	1.05105865708806e-05	\\
-6741.85014204546	1.0192661872495e-05	\\
-6740.87136008523	9.14435830841633e-06	\\
-6739.892578125	9.21129665805899e-06	\\
-6738.91379616477	1.15406850042343e-05	\\
-6737.93501420455	9.90563138543158e-06	\\
-6736.95623224432	8.40038004738296e-06	\\
-6735.97745028409	1.05571260424312e-05	\\
-6734.99866832386	1.02531065723532e-05	\\
-6734.01988636364	9.99354687659644e-06	\\
-6733.04110440341	1.00352127856001e-05	\\
-6732.06232244318	8.14711263237529e-06	\\
-6731.08354048296	9.844390143243e-06	\\
-6730.10475852273	9.14309946364242e-06	\\
-6729.1259765625	1.05244656327399e-05	\\
-6728.14719460227	8.19441863084917e-06	\\
-6727.16841264205	8.63054137796399e-06	\\
-6726.18963068182	8.27478340964148e-06	\\
-6725.21084872159	1.05451528258964e-05	\\
-6724.23206676136	1.01174516949605e-05	\\
-6723.25328480114	1.10576057639201e-05	\\
-6722.27450284091	8.19985594381252e-06	\\
-6721.29572088068	8.55308233320014e-06	\\
-6720.31693892046	1.0168260145603e-05	\\
-6719.33815696023	8.85582691090681e-06	\\
-6718.359375	8.74424330293729e-06	\\
-6717.38059303977	7.9703068367325e-06	\\
-6716.40181107955	1.02892895226897e-05	\\
-6715.42302911932	1.04085196345868e-05	\\
-6714.44424715909	9.95128156277939e-06	\\
-6713.46546519886	9.67279968320117e-06	\\
-6712.48668323864	1.022776192331e-05	\\
-6711.50790127841	8.84536268512212e-06	\\
-6710.52911931818	9.31288469399555e-06	\\
-6709.55033735796	6.62260684501762e-06	\\
-6708.57155539773	7.7583020773419e-06	\\
-6707.5927734375	1.07469294339182e-05	\\
-6706.61399147727	7.23829287117992e-06	\\
-6705.63520951705	9.19263456176653e-06	\\
-6704.65642755682	9.73534244344964e-06	\\
-6703.67764559659	8.84408653328434e-06	\\
-6702.69886363636	1.06393850627745e-05	\\
-6701.72008167614	8.96440655895696e-06	\\
-6700.74129971591	8.60624661006163e-06	\\
-6699.76251775568	1.21163091993361e-05	\\
-6698.78373579546	8.46474993398969e-06	\\
-6697.80495383523	9.27014967727539e-06	\\
-6696.826171875	1.00039730136193e-05	\\
-6695.84738991477	9.33976100473258e-06	\\
-6694.86860795455	9.00617632575546e-06	\\
-6693.88982599432	9.67401570525247e-06	\\
-6692.91104403409	8.05217950242716e-06	\\
-6691.93226207386	9.74108603975881e-06	\\
-6690.95348011364	1.00199482511863e-05	\\
-6689.97469815341	8.61455054644416e-06	\\
-6688.99591619318	7.71556677881636e-06	\\
-6688.01713423296	9.68724490832347e-06	\\
-6687.03835227273	8.80043484878581e-06	\\
-6686.0595703125	1.01248872429164e-05	\\
-6685.08078835227	9.92320701925564e-06	\\
-6684.10200639205	1.08266180691017e-05	\\
-6683.12322443182	1.06617665700213e-05	\\
-6682.14444247159	9.42082571464723e-06	\\
-6681.16566051136	1.09209438822737e-05	\\
-6680.18687855114	1.02926774394069e-05	\\
-6679.20809659091	9.84501202371369e-06	\\
-6678.22931463068	7.86576795797916e-06	\\
-6677.25053267046	9.64834674865656e-06	\\
-6676.27175071023	9.47011436766862e-06	\\
-6675.29296875	8.86724721817021e-06	\\
-6674.31418678977	9.21056148804963e-06	\\
-6673.33540482955	1.05765875119328e-05	\\
-6672.35662286932	1.08298923457039e-05	\\
-6671.37784090909	8.87387499643085e-06	\\
-6670.39905894886	8.77744372239721e-06	\\
-6669.42027698864	1.08049696147971e-05	\\
-6668.44149502841	9.64372685253848e-06	\\
-6667.46271306818	9.08159882221243e-06	\\
-6666.48393110796	1.06059240963756e-05	\\
-6665.50514914773	1.02196326754741e-05	\\
-6664.5263671875	9.7976230554374e-06	\\
-6663.54758522727	1.18994167259939e-05	\\
-6662.56880326705	1.14529424254817e-05	\\
-6661.59002130682	8.66935572123222e-06	\\
-6660.61123934659	9.65606561081815e-06	\\
-6659.63245738636	9.83175456975615e-06	\\
-6658.65367542614	1.00252074172898e-05	\\
-6657.67489346591	8.5727286472194e-06	\\
-6656.69611150568	9.19827459074944e-06	\\
-6655.71732954546	1.06728234273659e-05	\\
-6654.73854758523	8.68541953752701e-06	\\
-6653.759765625	9.60526077631119e-06	\\
-6652.78098366477	9.72409421540679e-06	\\
-6651.80220170455	9.67238951448365e-06	\\
-6650.82341974432	1.08807119816402e-05	\\
-6649.84463778409	1.04127244251406e-05	\\
-6648.86585582386	1.20984745553112e-05	\\
-6647.88707386364	1.04964072112591e-05	\\
-6646.90829190341	1.07740816878393e-05	\\
-6645.92950994318	8.60062464244184e-06	\\
-6644.95072798296	8.64485335244325e-06	\\
-6643.97194602273	1.16515868954087e-05	\\
-6642.9931640625	1.0058390450274e-05	\\
-6642.01438210227	1.06752092349147e-05	\\
-6641.03560014205	9.98072396816646e-06	\\
-6640.05681818182	1.27306406447397e-05	\\
-6639.07803622159	1.00402627546131e-05	\\
-6638.09925426136	9.28744249367836e-06	\\
-6637.12047230114	8.1581427239316e-06	\\
-6636.14169034091	9.75329740721772e-06	\\
-6635.16290838068	9.52771720788153e-06	\\
-6634.18412642046	9.60429299290296e-06	\\
-6633.20534446023	1.1836212396245e-05	\\
-6632.2265625	9.05844241060396e-06	\\
-6631.24778053977	1.08893570434591e-05	\\
-6630.26899857955	1.03570336641043e-05	\\
-6629.29021661932	9.92925741539242e-06	\\
-6628.31143465909	1.02091313348063e-05	\\
-6627.33265269886	1.05969357562381e-05	\\
-6626.35387073864	1.20772847097035e-05	\\
-6625.37508877841	9.34333135136956e-06	\\
-6624.39630681818	9.52528074586188e-06	\\
-6623.41752485796	1.25044331314231e-05	\\
-6622.43874289773	9.17289939484285e-06	\\
-6621.4599609375	1.11375438490071e-05	\\
-6620.48117897727	9.54606543879888e-06	\\
-6619.50239701705	9.08316740351739e-06	\\
-6618.52361505682	1.1207639256795e-05	\\
-6617.54483309659	8.99891315948358e-06	\\
-6616.56605113636	1.03941924130596e-05	\\
-6615.58726917614	9.27248998433246e-06	\\
-6614.60848721591	1.06546371025948e-05	\\
-6613.62970525568	9.54904041576369e-06	\\
-6612.65092329546	1.15220685588074e-05	\\
-6611.67214133523	1.08941114755877e-05	\\
-6610.693359375	6.90492414648985e-06	\\
-6609.71457741477	9.48606154589629e-06	\\
-6608.73579545455	9.93688615949263e-06	\\
-6607.75701349432	1.11341085331958e-05	\\
-6606.77823153409	8.93482846373729e-06	\\
-6605.79944957386	8.7366949956304e-06	\\
-6604.82066761364	1.00237771192905e-05	\\
-6603.84188565341	1.08632751314513e-05	\\
-6602.86310369318	9.38712387552058e-06	\\
-6601.88432173296	8.75672183853774e-06	\\
-6600.90553977273	8.41378362216357e-06	\\
-6599.9267578125	1.00760309810401e-05	\\
-6598.94797585227	9.15027650715671e-06	\\
-6597.96919389205	8.75147613599474e-06	\\
-6596.99041193182	1.0576206122809e-05	\\
-6596.01162997159	9.28340932447007e-06	\\
-6595.03284801136	1.07372915474606e-05	\\
-6594.05406605114	1.02178774682879e-05	\\
-6593.07528409091	8.43793565854272e-06	\\
-6592.09650213068	7.38493060661225e-06	\\
-6591.11772017046	8.73267360294533e-06	\\
-6590.13893821023	1.05639611177426e-05	\\
-6589.16015625	1.10941063283308e-05	\\
-6588.18137428977	8.99229795891344e-06	\\
-6587.20259232955	9.56986929775039e-06	\\
-6586.22381036932	1.18566657621575e-05	\\
-6585.24502840909	9.65575594762394e-06	\\
-6584.26624644886	9.18952453066364e-06	\\
-6583.28746448864	1.0765571777113e-05	\\
-6582.30868252841	9.94787152525188e-06	\\
-6581.32990056818	8.70164303787301e-06	\\
-6580.35111860796	7.92428007595391e-06	\\
-6579.37233664773	1.08341030692809e-05	\\
-6578.3935546875	8.84417112084272e-06	\\
-6577.41477272727	8.02567015943106e-06	\\
-6576.43599076705	8.76890355584427e-06	\\
-6575.45720880682	1.04919234549829e-05	\\
-6574.47842684659	7.81226306356776e-06	\\
-6573.49964488636	1.106727212745e-05	\\
-6572.52086292614	9.51603012614309e-06	\\
-6571.54208096591	1.03071535964401e-05	\\
-6570.56329900568	1.02290749121807e-05	\\
-6569.58451704546	9.26665566629073e-06	\\
-6568.60573508523	1.01770223083324e-05	\\
-6567.626953125	8.634013258845e-06	\\
-6566.64817116477	1.05991582712665e-05	\\
-6565.66938920455	8.93997424233124e-06	\\
-6564.69060724432	1.0488853747695e-05	\\
-6563.71182528409	9.61630340399219e-06	\\
-6562.73304332386	8.82972655355662e-06	\\
-6561.75426136364	9.83438193673061e-06	\\
-6560.77547940341	8.55259793768197e-06	\\
-6559.79669744318	9.7365503823691e-06	\\
-6558.81791548296	9.66337687359546e-06	\\
-6557.83913352273	9.01987143565354e-06	\\
-6556.8603515625	8.68714517666511e-06	\\
-6555.88156960227	1.05763315532051e-05	\\
-6554.90278764205	7.44802143329878e-06	\\
-6553.92400568182	1.04074604493555e-05	\\
-6552.94522372159	8.72237743272725e-06	\\
-6551.96644176136	1.06751280668334e-05	\\
-6550.98765980114	9.23842694083774e-06	\\
-6550.00887784091	7.58577170346584e-06	\\
-6549.03009588068	9.8310190037104e-06	\\
-6548.05131392046	1.12496930490478e-05	\\
-6547.07253196023	1.10926602228029e-05	\\
-6546.09375	1.10430507667687e-05	\\
-6545.11496803977	1.01817049361448e-05	\\
-6544.13618607955	9.06494465459859e-06	\\
-6543.15740411932	1.13241700984208e-05	\\
-6542.17862215909	9.86521250431792e-06	\\
-6541.19984019886	1.04750754081504e-05	\\
-6540.22105823864	1.07876535371499e-05	\\
-6539.24227627841	7.81767072353379e-06	\\
-6538.26349431818	7.38001420918392e-06	\\
-6537.28471235796	9.41891328164913e-06	\\
-6536.30593039773	1.01791721627068e-05	\\
-6535.3271484375	7.89064688613315e-06	\\
-6534.34836647727	1.133558540592e-05	\\
-6533.36958451705	7.95670537267756e-06	\\
-6532.39080255682	8.93633499775863e-06	\\
-6531.41202059659	1.12904151692859e-05	\\
-6530.43323863636	1.14650041631869e-05	\\
-6529.45445667614	1.01228862631768e-05	\\
-6528.47567471591	9.44885201567444e-06	\\
-6527.49689275568	9.66270184151972e-06	\\
-6526.51811079546	9.70549608039152e-06	\\
-6525.53932883523	9.73162791284576e-06	\\
-6524.560546875	1.00581571585313e-05	\\
-6523.58176491477	1.18104899996697e-05	\\
-6522.60298295455	1.12119407352382e-05	\\
-6521.62420099432	9.01004002499872e-06	\\
-6520.64541903409	8.93439914246969e-06	\\
-6519.66663707386	1.12375262077202e-05	\\
-6518.68785511364	9.90340280475377e-06	\\
-6517.70907315341	9.86795325434928e-06	\\
-6516.73029119318	1.07818569268166e-05	\\
-6515.75150923296	9.44798274054369e-06	\\
-6514.77272727273	1.07111086933885e-05	\\
-6513.7939453125	1.0334509285984e-05	\\
-6512.81516335227	9.01387944382202e-06	\\
-6511.83638139205	9.99619598446862e-06	\\
-6510.85759943182	1.17121020222684e-05	\\
-6509.87881747159	9.14118784187417e-06	\\
-6508.90003551136	9.66809101584527e-06	\\
-6507.92125355114	9.83958342355111e-06	\\
-6506.94247159091	1.13109979610064e-05	\\
-6505.96368963068	1.17775319373826e-05	\\
-6504.98490767046	9.57943848666833e-06	\\
-6504.00612571023	1.1316834332946e-05	\\
-6503.02734375	9.65822916349868e-06	\\
-6502.04856178977	1.17226644439546e-05	\\
-6501.06977982955	1.08870604717252e-05	\\
-6500.09099786932	1.20796016866131e-05	\\
-6499.11221590909	7.4911607910725e-06	\\
-6498.13343394886	1.07807040620109e-05	\\
-6497.15465198864	9.34925874582167e-06	\\
-6496.17587002841	1.10363965495903e-05	\\
-6495.19708806818	9.47648310337573e-06	\\
-6494.21830610796	9.74301875415291e-06	\\
-6493.23952414773	9.8535398994028e-06	\\
-6492.2607421875	8.24436870057312e-06	\\
-6491.28196022727	1.17822227126401e-05	\\
-6490.30317826705	1.24918838058156e-05	\\
-6489.32439630682	1.22384437379514e-05	\\
-6488.34561434659	1.05394210661115e-05	\\
-6487.36683238636	1.19576146491379e-05	\\
-6486.38805042614	1.02391207002672e-05	\\
-6485.40926846591	1.03545270004786e-05	\\
-6484.43048650568	1.19329798504392e-05	\\
-6483.45170454546	1.10263855977816e-05	\\
-6482.47292258523	1.00960771586755e-05	\\
-6481.494140625	1.01753969536674e-05	\\
-6480.51535866477	8.9438861799159e-06	\\
-6479.53657670455	9.45186700868745e-06	\\
-6478.55779474432	1.01330036690682e-05	\\
-6477.57901278409	1.13511224481748e-05	\\
-6476.60023082386	9.42113760022019e-06	\\
-6475.62144886364	1.04818480864405e-05	\\
-6474.64266690341	9.70730017300926e-06	\\
-6473.66388494318	8.90538094774641e-06	\\
-6472.68510298296	1.13068465797059e-05	\\
-6471.70632102273	9.24163473579731e-06	\\
-6470.7275390625	9.90320221422706e-06	\\
-6469.74875710227	9.32035903295983e-06	\\
-6468.76997514205	1.20543151116436e-05	\\
-6467.79119318182	1.04966479786814e-05	\\
-6466.81241122159	1.15989223735319e-05	\\
-6465.83362926136	1.04901039391886e-05	\\
-6464.85484730114	1.07212185369158e-05	\\
-6463.87606534091	1.01771258535724e-05	\\
-6462.89728338068	1.00761165015946e-05	\\
-6461.91850142046	1.22686588448391e-05	\\
-6460.93971946023	1.00179748185771e-05	\\
-6459.9609375	1.15868621348894e-05	\\
-6458.98215553977	1.08134598669562e-05	\\
-6458.00337357955	1.12632712437586e-05	\\
-6457.02459161932	9.14709037466739e-06	\\
-6456.04580965909	1.13312599025356e-05	\\
-6455.06702769886	1.07398887919549e-05	\\
-6454.08824573864	1.046576218379e-05	\\
-6453.10946377841	1.18429843258854e-05	\\
-6452.13068181818	8.66432090683722e-06	\\
-6451.15189985796	1.01256820259379e-05	\\
-6450.17311789773	1.12861762048676e-05	\\
-6449.1943359375	1.09776557661116e-05	\\
-6448.21555397727	1.09851840058765e-05	\\
-6447.23677201705	1.10722961558416e-05	\\
-6446.25799005682	9.59424526023042e-06	\\
-6445.27920809659	1.04980092875332e-05	\\
-6444.30042613636	1.08712521585489e-05	\\
-6443.32164417614	1.00686000437155e-05	\\
-6442.34286221591	1.04101695215663e-05	\\
-6441.36408025568	1.03606584779197e-05	\\
-6440.38529829546	1.14185555899461e-05	\\
-6439.40651633523	1.12261856923342e-05	\\
-6438.427734375	1.08493256085155e-05	\\
-6437.44895241477	9.02062069698073e-06	\\
-6436.47017045455	1.23852183168548e-05	\\
-6435.49138849432	1.12098129659966e-05	\\
-6434.51260653409	1.11788447573032e-05	\\
-6433.53382457386	9.79273672859635e-06	\\
-6432.55504261364	1.2123684222804e-05	\\
-6431.57626065341	1.15186506373441e-05	\\
-6430.59747869318	1.01953697236045e-05	\\
-6429.61869673296	1.19449577497703e-05	\\
-6428.63991477273	1.05211626833765e-05	\\
-6427.6611328125	1.15053144883815e-05	\\
-6426.68235085227	1.18314987709472e-05	\\
-6425.70356889205	1.27758884419201e-05	\\
-6424.72478693182	1.25986141353416e-05	\\
-6423.74600497159	1.08909874595043e-05	\\
-6422.76722301136	1.04530155885658e-05	\\
-6421.78844105114	1.14583334085348e-05	\\
-6420.80965909091	1.21031707620299e-05	\\
-6419.83087713068	1.1941534095934e-05	\\
-6418.85209517046	1.22559211324314e-05	\\
-6417.87331321023	1.19096384069506e-05	\\
-6416.89453125	1.00887121725971e-05	\\
-6415.91574928977	1.27726863896253e-05	\\
-6414.93696732955	1.21104968893371e-05	\\
-6413.95818536932	1.01103832167017e-05	\\
-6412.97940340909	1.04050879619263e-05	\\
-6412.00062144886	1.19749553169733e-05	\\
-6411.02183948864	1.2614314159232e-05	\\
-6410.04305752841	1.22126335587435e-05	\\
-6409.06427556818	1.16875753316358e-05	\\
-6408.08549360796	1.16765336638379e-05	\\
-6407.10671164773	1.08040279955292e-05	\\
-6406.1279296875	8.90161742598433e-06	\\
-6405.14914772727	1.25073574535437e-05	\\
-6404.17036576705	9.48202359727489e-06	\\
-6403.19158380682	9.80762252363353e-06	\\
-6402.21280184659	9.25970014384601e-06	\\
-6401.23401988636	1.02959016641804e-05	\\
-6400.25523792614	1.11726759322529e-05	\\
-6399.27645596591	1.01047109514741e-05	\\
-6398.29767400568	9.67683573947223e-06	\\
-6397.31889204546	1.20179967843523e-05	\\
-6396.34011008523	1.02124437363741e-05	\\
-6395.361328125	1.05192146064512e-05	\\
-6394.38254616477	1.03487625451309e-05	\\
-6393.40376420455	1.01935121259305e-05	\\
-6392.42498224432	1.09801730395312e-05	\\
-6391.44620028409	1.27462712774085e-05	\\
-6390.46741832386	9.72244927334872e-06	\\
-6389.48863636364	8.40285890150948e-06	\\
-6388.50985440341	9.96533253836317e-06	\\
-6387.53107244318	9.17551999660984e-06	\\
-6386.55229048296	9.91069272636556e-06	\\
-6385.57350852273	9.53443679044535e-06	\\
-6384.5947265625	1.09418921985359e-05	\\
-6383.61594460227	1.16989739256013e-05	\\
-6382.63716264205	1.18637363183611e-05	\\
-6381.65838068182	1.15888929264046e-05	\\
-6380.67959872159	1.04869825683514e-05	\\
-6379.70081676136	1.13803925980183e-05	\\
-6378.72203480114	9.21284674631958e-06	\\
-6377.74325284091	9.88566341182709e-06	\\
-6376.76447088068	1.18691801467672e-05	\\
-6375.78568892046	1.1154514530894e-05	\\
-6374.80690696023	1.01806257556481e-05	\\
-6373.828125	1.00479274217585e-05	\\
-6372.84934303977	9.39509403756763e-06	\\
-6371.87056107955	1.23022325612574e-05	\\
-6370.89177911932	1.04723558423604e-05	\\
-6369.91299715909	1.0729825554414e-05	\\
-6368.93421519886	1.07868396988604e-05	\\
-6367.95543323864	1.0400762244411e-05	\\
-6366.97665127841	9.571815364797e-06	\\
-6365.99786931818	1.03228737119079e-05	\\
-6365.01908735796	1.11027539050293e-05	\\
-6364.04030539773	7.35600634833834e-06	\\
-6363.0615234375	1.11368420221821e-05	\\
-6362.08274147727	1.10808497522679e-05	\\
-6361.10395951705	9.21627415629105e-06	\\
-6360.12517755682	1.0944934330783e-05	\\
-6359.14639559659	9.31206456643084e-06	\\
-6358.16761363636	1.14327078407132e-05	\\
-6357.18883167614	9.92764441745972e-06	\\
-6356.21004971591	1.00256795540012e-05	\\
-6355.23126775568	1.20806812889193e-05	\\
-6354.25248579546	1.10229281013754e-05	\\
-6353.27370383523	1.04803029031896e-05	\\
-6352.294921875	1.10889686105496e-05	\\
-6351.31613991477	7.72014978821977e-06	\\
-6350.33735795455	9.90259510422571e-06	\\
-6349.35857599432	9.76003299960721e-06	\\
-6348.37979403409	8.90195122540262e-06	\\
-6347.40101207386	1.03630009155192e-05	\\
-6346.42223011364	1.03345251667387e-05	\\
-6345.44344815341	1.17748767067942e-05	\\
-6344.46466619318	1.10783743888029e-05	\\
-6343.48588423296	1.21233717461419e-05	\\
-6342.50710227273	1.06210110638758e-05	\\
-6341.5283203125	9.31192395821949e-06	\\
-6340.54953835227	1.09336612031665e-05	\\
-6339.57075639205	1.09303473966715e-05	\\
-6338.59197443182	9.89592737703139e-06	\\
-6337.61319247159	1.1638491470695e-05	\\
-6336.63441051136	1.24540089612248e-05	\\
-6335.65562855114	9.63714178323274e-06	\\
-6334.67684659091	1.01130935610541e-05	\\
-6333.69806463068	7.89983563896885e-06	\\
-6332.71928267046	1.06308091872447e-05	\\
-6331.74050071023	1.10322337207436e-05	\\
-6330.76171875	1.06469982436533e-05	\\
-6329.78293678977	9.59020511206869e-06	\\
-6328.80415482955	9.03761949807712e-06	\\
-6327.82537286932	1.08455164601242e-05	\\
-6326.84659090909	9.0539739937883e-06	\\
-6325.86780894886	1.06152092137177e-05	\\
-6324.88902698864	8.96919245474136e-06	\\
-6323.91024502841	1.0661523639163e-05	\\
-6322.93146306818	1.06873349975052e-05	\\
-6321.95268110796	9.72944023426055e-06	\\
-6320.97389914773	8.76769153868357e-06	\\
-6319.9951171875	9.22518127622863e-06	\\
-6319.01633522727	8.94975841187904e-06	\\
-6318.03755326705	1.04112237950359e-05	\\
-6317.05877130682	1.12865948637781e-05	\\
-6316.07998934659	1.01996537122665e-05	\\
-6315.10120738636	9.23270889774404e-06	\\
-6314.12242542614	1.16680713924164e-05	\\
-6313.14364346591	9.85875448362906e-06	\\
-6312.16486150568	1.15368174472672e-05	\\
-6311.18607954546	1.09705139166511e-05	\\
-6310.20729758523	1.11044134634413e-05	\\
-6309.228515625	1.09166938236719e-05	\\
-6308.24973366477	8.88608363458445e-06	\\
-6307.27095170455	1.11999280037123e-05	\\
-6306.29216974432	8.4231377215323e-06	\\
-6305.31338778409	9.42779465584762e-06	\\
-6304.33460582386	9.78469332231141e-06	\\
-6303.35582386364	9.99575346254727e-06	\\
-6302.37704190341	1.04329978283385e-05	\\
-6301.39825994318	1.0517351175524e-05	\\
-6300.41947798296	1.19811097207315e-05	\\
-6299.44069602273	1.07496949823172e-05	\\
-6298.4619140625	9.94070327421994e-06	\\
-6297.48313210227	9.82329062946656e-06	\\
-6296.50435014205	1.08698401013016e-05	\\
-6295.52556818182	1.00417096847338e-05	\\
-6294.54678622159	1.03996974019626e-05	\\
-6293.56800426136	9.57393258784269e-06	\\
-6292.58922230114	1.01089999249936e-05	\\
-6291.61044034091	1.04552793423252e-05	\\
-6290.63165838068	1.05909462157904e-05	\\
-6289.65287642046	8.04483785534612e-06	\\
-6288.67409446023	7.81192077400789e-06	\\
-6287.6953125	9.85462493194414e-06	\\
-6286.71653053977	8.94752981332253e-06	\\
-6285.73774857955	9.63222318715081e-06	\\
-6284.75896661932	8.62927718124355e-06	\\
-6283.78018465909	8.97120854192342e-06	\\
-6282.80140269886	9.20575902330128e-06	\\
-6281.82262073864	1.00857529975078e-05	\\
-6280.84383877841	1.12057740706608e-05	\\
-6279.86505681818	9.8596378567294e-06	\\
-6278.88627485796	1.05226483343694e-05	\\
-6277.90749289773	1.0672116751715e-05	\\
-6276.9287109375	8.4738873532606e-06	\\
-6275.94992897727	9.71719991533497e-06	\\
-6274.97114701705	9.12639993985013e-06	\\
-6273.99236505682	8.9698811392069e-06	\\
-6273.01358309659	9.80560848304246e-06	\\
-6272.03480113636	7.93499179672075e-06	\\
-6271.05601917614	1.11490098981613e-05	\\
-6270.07723721591	8.00189118843089e-06	\\
-6269.09845525568	8.11736048009991e-06	\\
-6268.11967329546	8.97455115413802e-06	\\
-6267.14089133523	1.14354305162818e-05	\\
-6266.162109375	8.93505657387418e-06	\\
-6265.18332741477	9.83564529581551e-06	\\
-6264.20454545455	9.70968298234748e-06	\\
-6263.22576349432	9.26613239579253e-06	\\
-6262.24698153409	9.25759508685716e-06	\\
-6261.26819957386	1.08059238349355e-05	\\
-6260.28941761364	8.60990367484358e-06	\\
-6259.31063565341	9.75014446535197e-06	\\
-6258.33185369318	1.03590367352936e-05	\\
-6257.35307173296	9.72531184012167e-06	\\
-6256.37428977273	8.77003834024329e-06	\\
-6255.3955078125	8.75962115842567e-06	\\
-6254.41672585227	9.85920595192773e-06	\\
-6253.43794389205	1.09512803095052e-05	\\
-6252.45916193182	9.4115654588967e-06	\\
-6251.48037997159	9.70292700299525e-06	\\
-6250.50159801136	1.2675343054917e-05	\\
-6249.52281605114	8.89158958724129e-06	\\
-6248.54403409091	9.07405982517984e-06	\\
-6247.56525213068	8.86125825365139e-06	\\
-6246.58647017046	9.68907027416338e-06	\\
-6245.60768821023	9.12876952392022e-06	\\
-6244.62890625	8.22344134553775e-06	\\
-6243.65012428977	9.32942538935103e-06	\\
-6242.67134232955	9.59461399568428e-06	\\
-6241.69256036932	7.93594127656541e-06	\\
-6240.71377840909	7.19738211789916e-06	\\
-6239.73499644886	9.63165946268592e-06	\\
-6238.75621448864	8.69295703293211e-06	\\
-6237.77743252841	8.05283146619266e-06	\\
-6236.79865056818	7.93330423588441e-06	\\
-6235.81986860796	9.96886441491547e-06	\\
-6234.84108664773	9.30582522988441e-06	\\
-6233.8623046875	8.18961727633379e-06	\\
-6232.88352272727	8.01258439985777e-06	\\
-6231.90474076705	9.74804228080189e-06	\\
-6230.92595880682	9.85477790864282e-06	\\
-6229.94717684659	8.84449107097012e-06	\\
-6228.96839488636	8.606365827652e-06	\\
-6227.98961292614	9.68082700659218e-06	\\
-6227.01083096591	6.49290417342215e-06	\\
-6226.03204900568	8.94424930875237e-06	\\
-6225.05326704546	8.40279473458372e-06	\\
-6224.07448508523	6.72939721906557e-06	\\
-6223.095703125	1.01166879531668e-05	\\
-6222.11692116477	9.47056085869885e-06	\\
-6221.13813920455	8.12475194016142e-06	\\
-6220.15935724432	7.02817354063346e-06	\\
-6219.18057528409	8.36672330427057e-06	\\
-6218.20179332386	8.17248378937337e-06	\\
-6217.22301136364	8.50089848398617e-06	\\
-6216.24422940341	6.75801893481847e-06	\\
-6215.26544744318	7.73601564976628e-06	\\
-6214.28666548296	9.6681982539178e-06	\\
-6213.30788352273	8.0876573254261e-06	\\
-6212.3291015625	7.74096715572328e-06	\\
-6211.35031960227	9.44325180838626e-06	\\
-6210.37153764205	8.38831758670776e-06	\\
-6209.39275568182	9.08912304230002e-06	\\
-6208.41397372159	8.51381161493954e-06	\\
-6207.43519176136	8.42953256640058e-06	\\
-6206.45640980114	8.11609988473631e-06	\\
-6205.47762784091	9.53002219262659e-06	\\
-6204.49884588068	9.94512298238006e-06	\\
-6203.52006392046	9.66978861034682e-06	\\
-6202.54128196023	8.8312408658881e-06	\\
-6201.5625	9.55432822294066e-06	\\
-6200.58371803977	1.09821845096599e-05	\\
-6199.60493607955	8.51683985622671e-06	\\
-6198.62615411932	9.31974713517938e-06	\\
-6197.64737215909	9.06548434048706e-06	\\
-6196.66859019886	1.08978493950525e-05	\\
-6195.68980823864	9.31598745628798e-06	\\
-6194.71102627841	1.02692192797677e-05	\\
-6193.73224431818	1.08749658284328e-05	\\
-6192.75346235796	1.01722664015718e-05	\\
-6191.77468039773	8.943667368999e-06	\\
-6190.7958984375	9.04456023274418e-06	\\
-6189.81711647727	1.04852165433096e-05	\\
-6188.83833451705	8.77563974921167e-06	\\
-6187.85955255682	8.72853584128419e-06	\\
-6186.88077059659	1.03354481542657e-05	\\
-6185.90198863636	9.51574258753799e-06	\\
-6184.92320667614	8.70266900327966e-06	\\
-6183.94442471591	1.02013633511637e-05	\\
-6182.96564275568	1.11526809685661e-05	\\
-6181.98686079546	8.62498224553618e-06	\\
-6181.00807883523	9.80575429025174e-06	\\
-6180.029296875	1.11838654010654e-05	\\
-6179.05051491477	1.0150569673807e-05	\\
-6178.07173295455	1.02087711065576e-05	\\
-6177.09295099432	1.01972457679535e-05	\\
-6176.11416903409	8.91740780882304e-06	\\
-6175.13538707386	9.22202280919412e-06	\\
-6174.15660511364	8.73406865347042e-06	\\
-6173.17782315341	8.88857039950822e-06	\\
-6172.19904119318	9.57227680815324e-06	\\
-6171.22025923296	9.76280351295801e-06	\\
-6170.24147727273	9.7269127356407e-06	\\
-6169.2626953125	7.2511978972976e-06	\\
-6168.28391335227	9.09932778475869e-06	\\
-6167.30513139205	1.11186033855131e-05	\\
-6166.32634943182	7.75058864481988e-06	\\
-6165.34756747159	9.66500130772939e-06	\\
-6164.36878551136	8.32824913911014e-06	\\
-6163.39000355114	9.3611980164646e-06	\\
-6162.41122159091	9.595258984468e-06	\\
-6161.43243963068	8.86721033733329e-06	\\
-6160.45365767046	9.93081512898914e-06	\\
-6159.47487571023	8.21456800611828e-06	\\
-6158.49609375	8.93698669753542e-06	\\
-6157.51731178977	1.0707484302498e-05	\\
-6156.53852982955	1.01286294701591e-05	\\
-6155.55974786932	7.48722661667455e-06	\\
-6154.58096590909	8.30631728406354e-06	\\
-6153.60218394886	8.57967101315341e-06	\\
-6152.62340198864	9.09166967480471e-06	\\
-6151.64462002841	1.03241214222022e-05	\\
-6150.66583806818	9.74013650292401e-06	\\
-6149.68705610796	8.83072722793556e-06	\\
-6148.70827414773	8.98814404795571e-06	\\
-6147.7294921875	1.10587129034939e-05	\\
-6146.75071022727	7.51722030388523e-06	\\
-6145.77192826705	9.57277520887882e-06	\\
-6144.79314630682	8.61639589953168e-06	\\
-6143.81436434659	9.73392598622237e-06	\\
-6142.83558238636	8.35238403248184e-06	\\
-6141.85680042614	9.91672350848772e-06	\\
-6140.87801846591	9.5457772717763e-06	\\
-6139.89923650568	8.53779672614675e-06	\\
-6138.92045454546	9.77805702948509e-06	\\
-6137.94167258523	9.30960854097489e-06	\\
-6136.962890625	1.00048462337851e-05	\\
-6135.98410866477	9.40726486682945e-06	\\
-6135.00532670455	6.27726838957153e-06	\\
-6134.02654474432	8.60576919901801e-06	\\
-6133.04776278409	8.89592648059635e-06	\\
-6132.06898082386	7.53012975900916e-06	\\
-6131.09019886364	7.91221582278533e-06	\\
-6130.11141690341	7.50212288881511e-06	\\
-6129.13263494318	8.83829180711423e-06	\\
-6128.15385298296	1.08637126510099e-05	\\
-6127.17507102273	9.25158265981474e-06	\\
-6126.1962890625	8.29501771108177e-06	\\
-6125.21750710227	7.93005300011885e-06	\\
-6124.23872514205	9.78742794125969e-06	\\
-6123.25994318182	1.1231206389025e-05	\\
-6122.28116122159	8.33713176983699e-06	\\
-6121.30237926136	7.61520276520857e-06	\\
-6120.32359730114	8.46688870683861e-06	\\
-6119.34481534091	8.34348790460137e-06	\\
-6118.36603338068	8.51255891668588e-06	\\
-6117.38725142046	8.60752868705689e-06	\\
-6116.40846946023	9.2729926526876e-06	\\
-6115.4296875	9.71562543551046e-06	\\
-6114.45090553977	9.11180992084696e-06	\\
-6113.47212357955	1.03365072678794e-05	\\
-6112.49334161932	7.09327489243407e-06	\\
-6111.51455965909	6.91649000321268e-06	\\
-6110.53577769886	8.83464455383785e-06	\\
-6109.55699573864	9.76954164557626e-06	\\
-6108.57821377841	8.39523383463471e-06	\\
-6107.59943181818	8.5402287678086e-06	\\
-6106.62064985796	7.24080559604678e-06	\\
-6105.64186789773	9.12184136505484e-06	\\
-6104.6630859375	7.55092538362535e-06	\\
-6103.68430397727	8.7535351534808e-06	\\
-6102.70552201705	9.13839316902373e-06	\\
-6101.72674005682	8.72931150067244e-06	\\
-6100.74795809659	7.84224097017145e-06	\\
-6099.76917613636	7.18792029534069e-06	\\
-6098.79039417614	9.10595787609455e-06	\\
-6097.81161221591	8.78773133542202e-06	\\
-6096.83283025568	9.23481214905585e-06	\\
-6095.85404829546	8.48179302927712e-06	\\
-6094.87526633523	9.19013145145352e-06	\\
-6093.896484375	1.11341838998879e-05	\\
-6092.91770241477	6.29002562599006e-06	\\
-6091.93892045455	7.94299755122165e-06	\\
-6090.96013849432	8.57507824212356e-06	\\
-6089.98135653409	8.05298862705954e-06	\\
-6089.00257457386	8.68573028193673e-06	\\
-6088.02379261364	9.14874361246359e-06	\\
-6087.04501065341	7.30039966135235e-06	\\
-6086.06622869318	9.47184666472641e-06	\\
-6085.08744673296	9.6176847894911e-06	\\
-6084.10866477273	8.51711257463503e-06	\\
-6083.1298828125	8.09169664443315e-06	\\
-6082.15110085227	8.53095214081756e-06	\\
-6081.17231889205	7.03225173441951e-06	\\
-6080.19353693182	9.8994518075644e-06	\\
-6079.21475497159	7.36818164705704e-06	\\
-6078.23597301136	8.27209160204658e-06	\\
-6077.25719105114	1.05001509807898e-05	\\
-6076.27840909091	9.90912862622051e-06	\\
-6075.29962713068	7.26126248265234e-06	\\
-6074.32084517046	1.02634538186948e-05	\\
-6073.34206321023	8.02205546219304e-06	\\
-6072.36328125	8.3484116397007e-06	\\
-6071.38449928977	7.78131984716835e-06	\\
-6070.40571732955	7.85904220978372e-06	\\
-6069.42693536932	7.38199659573616e-06	\\
-6068.44815340909	8.01799022383062e-06	\\
-6067.46937144886	1.00528880296318e-05	\\
-6066.49058948864	6.82744269541218e-06	\\
-6065.51180752841	7.35684191092854e-06	\\
-6064.53302556818	8.9154926502413e-06	\\
-6063.55424360796	8.76791957158896e-06	\\
-6062.57546164773	9.94063173326335e-06	\\
-6061.5966796875	5.86441665273443e-06	\\
-6060.61789772727	8.53633636665026e-06	\\
-6059.63911576705	6.01197520969291e-06	\\
-6058.66033380682	8.25872758995492e-06	\\
-6057.68155184659	8.1358384095064e-06	\\
-6056.70276988636	9.2613507268856e-06	\\
-6055.72398792614	5.89081179091988e-06	\\
-6054.74520596591	7.14682567702485e-06	\\
-6053.76642400568	8.83243090827698e-06	\\
-6052.78764204546	8.02072763041593e-06	\\
-6051.80886008523	6.68206957084512e-06	\\
-6050.830078125	8.54312497806115e-06	\\
-6049.85129616477	7.18799566698952e-06	\\
-6048.87251420455	6.55018126655101e-06	\\
-6047.89373224432	7.29999619255187e-06	\\
-6046.91495028409	8.24387413745198e-06	\\
-6045.93616832386	7.42661975125231e-06	\\
-6044.95738636364	7.02450303554431e-06	\\
-6043.97860440341	7.57249465041052e-06	\\
-6042.99982244318	8.82395078125808e-06	\\
-6042.02104048296	8.19186582498038e-06	\\
-6041.04225852273	6.03762379582153e-06	\\
-6040.0634765625	7.19548399802713e-06	\\
-6039.08469460227	7.64677736935493e-06	\\
-6038.10591264205	6.9811338829406e-06	\\
-6037.12713068182	7.27738450730722e-06	\\
-6036.14834872159	6.73767866486019e-06	\\
-6035.16956676136	7.61129389340539e-06	\\
-6034.19078480114	8.20768824127824e-06	\\
-6033.21200284091	9.02396872766899e-06	\\
-6032.23322088068	1.11808161523198e-05	\\
-6031.25443892046	8.84079435659564e-06	\\
-6030.27565696023	8.03652301054324e-06	\\
-6029.296875	7.84617668110957e-06	\\
-6028.31809303977	7.71440518945805e-06	\\
-6027.33931107955	7.28882875958495e-06	\\
-6026.36052911932	6.8700670084246e-06	\\
-6025.38174715909	6.66167876026283e-06	\\
-6024.40296519886	7.21508424742276e-06	\\
-6023.42418323864	8.02436039250032e-06	\\
-6022.44540127841	8.89948296611478e-06	\\
-6021.46661931818	5.59923415520393e-06	\\
-6020.48783735796	7.01629912133545e-06	\\
-6019.50905539773	7.30507547759708e-06	\\
-6018.5302734375	6.57603250632656e-06	\\
-6017.55149147727	7.99066070270861e-06	\\
-6016.57270951705	7.02548993485912e-06	\\
-6015.59392755682	7.66542191162578e-06	\\
-6014.61514559659	9.94864144261751e-06	\\
-6013.63636363636	8.09550546167061e-06	\\
-6012.65758167614	7.06578098825764e-06	\\
-6011.67879971591	9.01613618781101e-06	\\
-6010.70001775568	5.96635488055524e-06	\\
-6009.72123579546	8.26319771048071e-06	\\
-6008.74245383523	6.62642235165664e-06	\\
-6007.763671875	5.90063782604622e-06	\\
-6006.78488991477	8.11273049558184e-06	\\
-6005.80610795455	5.9358365258453e-06	\\
-6004.82732599432	7.43362008450094e-06	\\
-6003.84854403409	5.71510241234958e-06	\\
-6002.86976207386	5.84191274584365e-06	\\
-6001.89098011364	7.61024368223472e-06	\\
-6000.91219815341	6.91729112291435e-06	\\
-5999.93341619318	8.78133676863711e-06	\\
-5998.95463423296	8.39024542815233e-06	\\
-5997.97585227273	7.16480318480784e-06	\\
-5996.9970703125	6.2921659786577e-06	\\
-5996.01828835227	7.73759690713587e-06	\\
-5995.03950639205	7.27825726523442e-06	\\
-5994.06072443182	6.64539405177482e-06	\\
-5993.08194247159	6.07038293193139e-06	\\
-5992.10316051136	6.59137508486034e-06	\\
-5991.12437855114	7.01937231601031e-06	\\
-5990.14559659091	7.72515186001033e-06	\\
-5989.16681463068	7.32906073620174e-06	\\
-5988.18803267046	7.84954847815106e-06	\\
-5987.20925071023	8.16728788477781e-06	\\
-5986.23046875	9.62429996283366e-06	\\
-5985.25168678977	6.81650003500882e-06	\\
-5984.27290482955	7.4714499863021e-06	\\
-5983.29412286932	7.46368375193769e-06	\\
-5982.31534090909	7.5882037050164e-06	\\
-5981.33655894886	8.13557142465036e-06	\\
-5980.35777698864	9.60898622412178e-06	\\
-5979.37899502841	7.94752497174615e-06	\\
-5978.40021306818	9.24394896199432e-06	\\
-5977.42143110796	6.9300325090459e-06	\\
-5976.44264914773	6.80228666423215e-06	\\
-5975.4638671875	7.9973959547869e-06	\\
-5974.48508522727	9.37558209587707e-06	\\
-5973.50630326705	6.44967699435845e-06	\\
-5972.52752130682	6.47066358568284e-06	\\
-5971.54873934659	6.17288903673027e-06	\\
-5970.56995738636	7.69623178459432e-06	\\
-5969.59117542614	5.96052304471275e-06	\\
-5968.61239346591	5.87741090143695e-06	\\
-5967.63361150568	6.25492217767648e-06	\\
-5966.65482954546	6.6234737859965e-06	\\
-5965.67604758523	4.47081332651697e-06	\\
-5964.697265625	7.20854919141658e-06	\\
-5963.71848366477	7.70337736148254e-06	\\
-5962.73970170455	6.64024142916108e-06	\\
-5961.76091974432	7.22704545652029e-06	\\
-5960.78213778409	5.68824997226216e-06	\\
-5959.80335582386	6.33653029352594e-06	\\
-5958.82457386364	7.74046834655501e-06	\\
-5957.84579190341	7.39712833572816e-06	\\
-5956.86700994318	6.32764596021663e-06	\\
-5955.88822798296	7.93663210764396e-06	\\
-5954.90944602273	5.66660244578609e-06	\\
-5953.9306640625	5.22484361901542e-06	\\
-5952.95188210227	8.22029693773625e-06	\\
-5951.97310014205	7.02768517745254e-06	\\
-5950.99431818182	5.02063045809234e-06	\\
-5950.01553622159	6.73139995249984e-06	\\
-5949.03675426136	6.39070194795395e-06	\\
-5948.05797230114	4.71166419712261e-06	\\
-5947.07919034091	6.90892780271575e-06	\\
-5946.10040838068	7.0535102100882e-06	\\
-5945.12162642046	7.2054391823562e-06	\\
-5944.14284446023	6.17860588617163e-06	\\
-5943.1640625	5.76360126764859e-06	\\
-5942.18528053977	5.31680355731665e-06	\\
-5941.20649857955	7.07706160569937e-06	\\
-5940.22771661932	6.35008597148829e-06	\\
-5939.24893465909	6.57238793773554e-06	\\
-5938.27015269886	5.50681162975805e-06	\\
-5937.29137073864	5.65602685838488e-06	\\
-5936.31258877841	6.14432512229681e-06	\\
-5935.33380681818	4.05033797841072e-06	\\
-5934.35502485796	6.0270560383344e-06	\\
-5933.37624289773	7.6856272236373e-06	\\
-5932.3974609375	7.86582555321916e-06	\\
-5931.41867897727	6.49020231168335e-06	\\
-5930.43989701705	6.93747483227633e-06	\\
-5929.46111505682	6.9531435574339e-06	\\
-5928.48233309659	5.73939775405363e-06	\\
-5927.50355113636	9.10924566776317e-06	\\
-5926.52476917614	5.70908734727141e-06	\\
-5925.54598721591	6.073515114487e-06	\\
-5924.56720525568	6.28065003398384e-06	\\
-5923.58842329546	5.32191538628521e-06	\\
-5922.60964133523	6.89011183415885e-06	\\
-5921.630859375	6.04730292431723e-06	\\
-5920.65207741477	7.57073762964095e-06	\\
-5919.67329545455	8.96856984607112e-06	\\
-5918.69451349432	6.80044667356274e-06	\\
-5917.71573153409	6.08546200078414e-06	\\
-5916.73694957386	6.04867479027933e-06	\\
-5915.75816761364	5.5436924203495e-06	\\
-5914.77938565341	8.02737897778049e-06	\\
-5913.80060369318	8.24054783941324e-06	\\
-5912.82182173296	5.73490490071116e-06	\\
-5911.84303977273	5.83745169927617e-06	\\
-5910.8642578125	5.46713695809314e-06	\\
-5909.88547585227	8.33454856895248e-06	\\
-5908.90669389205	7.40082526986058e-06	\\
-5907.92791193182	7.02612298933825e-06	\\
-5906.94912997159	7.5893413480159e-06	\\
-5905.97034801136	7.23238008553926e-06	\\
-5904.99156605114	7.05331521531778e-06	\\
-5904.01278409091	8.13580865821771e-06	\\
-5903.03400213068	6.20420949867122e-06	\\
-5902.05522017046	8.60606357405511e-06	\\
-5901.07643821023	6.91699126345071e-06	\\
-5900.09765625	9.70438510819767e-06	\\
-5899.11887428977	8.26349475562775e-06	\\
-5898.14009232955	7.5268200956796e-06	\\
-5897.16131036932	6.94320079762952e-06	\\
-5896.18252840909	7.43848470080983e-06	\\
-5895.20374644886	6.18780849058078e-06	\\
-5894.22496448864	8.3679136474457e-06	\\
-5893.24618252841	5.3743870061269e-06	\\
-5892.26740056818	8.5579375413396e-06	\\
-5891.28861860796	6.88077769619037e-06	\\
-5890.30983664773	7.62738651911333e-06	\\
-5889.3310546875	5.61576789100996e-06	\\
-5888.35227272727	5.96961247140024e-06	\\
-5887.37349076705	7.65281030362646e-06	\\
-5886.39470880682	4.83061695281141e-06	\\
-5885.41592684659	7.25046702211988e-06	\\
-5884.43714488636	5.40806153776107e-06	\\
-5883.45836292614	5.85745303565445e-06	\\
-5882.47958096591	5.38640526321397e-06	\\
-5881.50079900568	5.94993870839976e-06	\\
-5880.52201704546	6.62358186614338e-06	\\
-5879.54323508523	8.1041968058256e-06	\\
-5878.564453125	5.18392082011942e-06	\\
-5877.58567116477	4.74146609188101e-06	\\
-5876.60688920455	6.2005567430874e-06	\\
-5875.62810724432	6.47792887268535e-06	\\
-5874.64932528409	5.73157623914241e-06	\\
-5873.67054332386	6.8251357699835e-06	\\
-5872.69176136364	7.34575796902162e-06	\\
-5871.71297940341	6.2112343056049e-06	\\
-5870.73419744318	5.67720974253866e-06	\\
-5869.75541548296	7.15516404400904e-06	\\
-5868.77663352273	7.0893873309972e-06	\\
-5867.7978515625	6.43263348857862e-06	\\
-5866.81906960227	5.57124231582357e-06	\\
-5865.84028764205	8.01038131070208e-06	\\
-5864.86150568182	8.10867160216877e-06	\\
-5863.88272372159	7.39977506812186e-06	\\
-5862.90394176136	6.99539404642205e-06	\\
-5861.92515980114	7.85994851780402e-06	\\
-5860.94637784091	6.0252630357275e-06	\\
-5859.96759588068	7.9844782789552e-06	\\
-5858.98881392046	5.53326818075454e-06	\\
-5858.01003196023	7.14337379865044e-06	\\
-5857.03125	6.20659546383579e-06	\\
-5856.05246803977	4.49173965516723e-06	\\
-5855.07368607955	5.87328163170446e-06	\\
-5854.09490411932	6.63143335255843e-06	\\
-5853.11612215909	7.13362702528391e-06	\\
-5852.13734019886	7.76495720145176e-06	\\
-5851.15855823864	5.45163521295674e-06	\\
-5850.17977627841	5.35917985793463e-06	\\
-5849.20099431818	6.3005980561593e-06	\\
-5848.22221235796	6.59503616392806e-06	\\
-5847.24343039773	6.73104474191351e-06	\\
-5846.2646484375	6.62039539136864e-06	\\
-5845.28586647727	6.56463094596566e-06	\\
-5844.30708451705	6.28086716499657e-06	\\
-5843.32830255682	7.0294769914667e-06	\\
-5842.34952059659	4.98230532468022e-06	\\
-5841.37073863636	4.11123407382184e-06	\\
-5840.39195667614	7.40271493651596e-06	\\
-5839.41317471591	6.83984374050556e-06	\\
-5838.43439275568	8.64280659051145e-06	\\
-5837.45561079546	5.52293145789177e-06	\\
-5836.47682883523	4.39713732168751e-06	\\
-5835.498046875	6.4251104434133e-06	\\
-5834.51926491477	5.46250535241086e-06	\\
-5833.54048295455	3.00525644711927e-06	\\
-5832.56170099432	5.32663806910987e-06	\\
-5831.58291903409	5.99033432855354e-06	\\
-5830.60413707386	3.47234632602314e-06	\\
-5829.62535511364	5.46646803086568e-06	\\
-5828.64657315341	6.50943860345674e-06	\\
-5827.66779119318	7.05341733866305e-06	\\
-5826.68900923296	5.47818890890684e-06	\\
-5825.71022727273	4.39862689122851e-06	\\
-5824.7314453125	5.65969576364161e-06	\\
-5823.75266335227	7.70929128448733e-06	\\
-5822.77388139205	4.78462766709398e-06	\\
-5821.79509943182	4.10609179529559e-06	\\
-5820.81631747159	6.50584099030892e-06	\\
-5819.83753551136	5.96344302998424e-06	\\
-5818.85875355114	6.09981689426314e-06	\\
-5817.87997159091	4.30576097222548e-06	\\
-5816.90118963068	5.27460218256339e-06	\\
-5815.92240767046	5.35370320521001e-06	\\
-5814.94362571023	5.30930002707807e-06	\\
-5813.96484375	5.51974407391441e-06	\\
-5812.98606178977	7.50189175765712e-06	\\
-5812.00727982955	8.1378539219231e-06	\\
-5811.02849786932	4.17000050415817e-06	\\
-5810.04971590909	6.68822429688192e-06	\\
-5809.07093394886	5.01709511621846e-06	\\
-5808.09215198864	5.85473282528339e-06	\\
-5807.11337002841	7.0462344606459e-06	\\
-5806.13458806818	4.44073555699521e-06	\\
-5805.15580610796	6.00990513483603e-06	\\
-5804.17702414773	6.08399516857077e-06	\\
-5803.1982421875	5.52045044374246e-06	\\
-5802.21946022727	4.43950129407114e-06	\\
-5801.24067826705	5.5703614250066e-06	\\
-5800.26189630682	7.44385229677776e-06	\\
-5799.28311434659	5.43268885976532e-06	\\
-5798.30433238636	6.17152930158459e-06	\\
-5797.32555042614	3.53701902049728e-06	\\
-5796.34676846591	9.07622113018184e-06	\\
-5795.36798650568	4.07984330491375e-06	\\
-5794.38920454546	4.60840614496541e-06	\\
-5793.41042258523	5.11473612369279e-06	\\
-5792.431640625	5.19199018568388e-06	\\
-5791.45285866477	4.75236773939548e-06	\\
-5790.47407670455	3.00885606805617e-06	\\
-5789.49529474432	5.54557989673096e-06	\\
-5788.51651278409	4.1033032289168e-06	\\
-5787.53773082386	3.55152946419158e-06	\\
-5786.55894886364	3.24766140326912e-06	\\
-5785.58016690341	7.00462621275288e-06	\\
-5784.60138494318	5.52941465289892e-06	\\
-5783.62260298296	3.66777488144297e-06	\\
-5782.64382102273	5.7244812927116e-06	\\
-5781.6650390625	6.17298810463645e-06	\\
-5780.68625710227	3.34921918319382e-06	\\
-5779.70747514205	5.91556128972426e-06	\\
-5778.72869318182	6.38963815475968e-06	\\
-5777.74991122159	4.19602735762775e-06	\\
-5776.77112926136	4.62955539743308e-06	\\
-5775.79234730114	7.05056542881873e-06	\\
-5774.81356534091	5.09604791110146e-06	\\
-5773.83478338068	5.8263628685887e-06	\\
-5772.85600142046	5.22319296197987e-06	\\
-5771.87721946023	3.28722562783976e-06	\\
-5770.8984375	6.33852826776542e-06	\\
-5769.91965553977	2.97129392370794e-06	\\
-5768.94087357955	4.68065702743549e-06	\\
-5767.96209161932	4.69367113315428e-06	\\
-5766.98330965909	5.38645904635496e-06	\\
-5766.00452769886	4.35167994062655e-06	\\
-5765.02574573864	5.47772608614066e-06	\\
-5764.04696377841	4.39036270382512e-06	\\
-5763.06818181818	5.5148333850542e-06	\\
-5762.08939985796	6.15409083016731e-06	\\
-5761.11061789773	3.31451858030038e-06	\\
-5760.1318359375	5.39722873996538e-06	\\
-5759.15305397727	5.23378352938129e-06	\\
-5758.17427201705	4.07919649298169e-06	\\
-5757.19549005682	3.94120504677216e-06	\\
-5756.21670809659	6.84312774691063e-06	\\
-5755.23792613636	5.84269239095972e-06	\\
-5754.25914417614	4.73806181517037e-06	\\
-5753.28036221591	2.69985316105076e-06	\\
-5752.30158025568	5.51323412443657e-06	\\
-5751.32279829546	3.57755780114107e-06	\\
-5750.34401633523	3.44911443090692e-06	\\
-5749.365234375	4.23832730720087e-06	\\
-5748.38645241477	4.12898943431546e-06	\\
-5747.40767045455	4.1267110476503e-06	\\
-5746.42888849432	3.39279334996302e-06	\\
-5745.45010653409	4.24968422454317e-06	\\
-5744.47132457386	4.81586910408351e-06	\\
-5743.49254261364	4.32069056073315e-06	\\
-5742.51376065341	2.92695000981175e-06	\\
-5741.53497869318	5.63755669765929e-06	\\
-5740.55619673296	5.66359129102891e-06	\\
-5739.57741477273	3.77581330940419e-06	\\
-5738.5986328125	2.80458612422658e-06	\\
-5737.61985085227	4.90165696718367e-06	\\
-5736.64106889205	4.40211784199102e-06	\\
-5735.66228693182	3.81797059697275e-06	\\
-5734.68350497159	3.81141031054895e-06	\\
-5733.70472301136	2.69680026393806e-06	\\
-5732.72594105114	4.13215673990551e-06	\\
-5731.74715909091	3.15229724941845e-06	\\
-5730.76837713068	4.58249757145236e-06	\\
-5729.78959517046	3.49213091636749e-06	\\
-5728.81081321023	3.62389701408104e-06	\\
-5727.83203125	4.34444678893001e-06	\\
-5726.85324928977	5.30722791316222e-06	\\
-5725.87446732955	5.10177691158592e-06	\\
-5724.89568536932	1.24861283318404e-06	\\
-5723.91690340909	3.18177188510659e-06	\\
-5722.93812144886	3.99009868420121e-06	\\
-5721.95933948864	3.73825673870317e-06	\\
-5720.98055752841	2.75806298980645e-06	\\
-5720.00177556818	4.50147730488051e-06	\\
-5719.02299360796	3.36492554327865e-06	\\
-5718.04421164773	3.47488720093756e-06	\\
-5717.0654296875	3.66856863305158e-06	\\
-5716.08664772727	3.75150820964911e-06	\\
-5715.10786576705	1.88789511933992e-06	\\
-5714.12908380682	4.16286148829105e-06	\\
-5713.15030184659	1.39756901001076e-06	\\
-5712.17151988636	2.59549451129393e-06	\\
-5711.19273792614	2.31336931568266e-06	\\
-5710.21395596591	3.6167107967681e-06	\\
-5709.23517400568	1.35710878946157e-06	\\
-5708.25639204546	2.60391460121648e-06	\\
-5707.27761008523	2.22654087022812e-06	\\
-5706.298828125	2.1740315360634e-06	\\
-5705.32004616477	5.31233568694704e-07	\\
-5704.34126420455	3.21275875236496e-06	\\
-5703.36248224432	4.99512009306568e-06	\\
-5702.38370028409	2.45342285762778e-06	\\
-5701.40491832386	5.00018976506416e-06	\\
-5700.42613636364	3.23526475361748e-06	\\
-5699.44735440341	2.81551884172352e-06	\\
-5698.46857244318	2.32337283666842e-06	\\
-5697.48979048296	2.49374024315065e-06	\\
-5696.51100852273	3.43067230715313e-06	\\
-5695.5322265625	2.87244863801995e-06	\\
-5694.55344460227	1.6871370181616e-06	\\
-5693.57466264205	1.55473768334289e-06	\\
-5692.59588068182	1.51753119117785e-06	\\
-5691.61709872159	2.69882308124388e-06	\\
-5690.63831676136	2.42983134696921e-06	\\
-5689.65953480114	4.41401830269714e-06	\\
-5688.68075284091	3.36316648601853e-06	\\
-5687.70197088068	1.95797652583886e-06	\\
-5686.72318892046	2.70374276607321e-06	\\
-5685.74440696023	7.71944773034193e-07	\\
-5684.765625	2.2566948721409e-06	\\
-5683.78684303977	3.38036438428574e-06	\\
-5682.80806107955	2.55673766799839e-06	\\
-5681.82927911932	3.59435918221634e-06	\\
-5680.85049715909	4.89436084249702e-06	\\
-5679.87171519886	2.51064672552932e-06	\\
-5678.89293323864	2.45324946061198e-06	\\
-5677.91415127841	2.98733022667537e-06	\\
-5676.93536931818	3.05815296727589e-06	\\
-5675.95658735796	1.92801628205255e-06	\\
-5674.97780539773	2.22801019519187e-06	\\
-5673.9990234375	2.40347771439867e-06	\\
-5673.02024147727	3.19062719525392e-06	\\
-5672.04145951705	2.67893639823999e-06	\\
-5671.06267755682	2.45067699082365e-06	\\
-5670.08389559659	3.24381185265142e-06	\\
-5669.10511363636	2.95840451396965e-06	\\
-5668.12633167614	3.01122529544294e-06	\\
-5667.14754971591	3.35590651364209e-06	\\
-5666.16876775568	4.00948816596416e-06	\\
-5665.18998579546	2.75243558772374e-06	\\
-5664.21120383523	2.78536634946e-06	\\
-5663.232421875	2.91896846640479e-06	\\
-5662.25363991477	4.27944557430937e-06	\\
-5661.27485795455	3.33446512203468e-06	\\
-5660.29607599432	3.8598559437409e-06	\\
-5659.31729403409	1.1204966582178e-06	\\
-5658.33851207386	4.10394283103811e-06	\\
-5657.35973011364	2.14071708976162e-06	\\
-5656.38094815341	2.90942201871212e-06	\\
-5655.40216619318	2.82752735074356e-06	\\
-5654.42338423296	3.27457872844562e-06	\\
-5653.44460227273	2.55228807523807e-06	\\
-5652.4658203125	4.14704813562134e-06	\\
-5651.48703835227	8.65520992137649e-07	\\
-5650.50825639205	3.05690366003323e-06	\\
-5649.52947443182	2.40458402707604e-06	\\
-5648.55069247159	2.48068598412947e-06	\\
-5647.57191051136	2.80504669889547e-06	\\
-5646.59312855114	2.99060329307127e-06	\\
-5645.61434659091	2.35060859948032e-06	\\
-5644.63556463068	3.41142915564446e-06	\\
-5643.65678267046	2.6546289350296e-06	\\
-5642.67800071023	3.76340601570487e-06	\\
-5641.69921875	2.7596736747187e-06	\\
-5640.72043678977	3.54358322503483e-06	\\
-5639.74165482955	5.6089455472304e-06	\\
-5638.76287286932	2.65796971364613e-06	\\
-5637.78409090909	2.58812922476142e-06	\\
-5636.80530894886	3.72048805047492e-06	\\
-5635.82652698864	4.46735225854482e-06	\\
-5634.84774502841	2.59308227489426e-06	\\
-5633.86896306818	3.48554904785775e-06	\\
-5632.89018110796	3.11306109111305e-06	\\
-5631.91139914773	4.27892990780979e-06	\\
-5630.9326171875	4.57425050330195e-06	\\
-5629.95383522727	4.62042026250683e-06	\\
-5628.97505326705	2.31554879634993e-06	\\
-5627.99627130682	3.5259111888451e-06	\\
-5627.01748934659	1.85082764419644e-06	\\
-5626.03870738636	2.46836346463536e-06	\\
-5625.05992542614	2.94881916889357e-06	\\
-5624.08114346591	3.42388741736621e-06	\\
-5623.10236150568	2.72915062757932e-06	\\
-5622.12357954546	2.59522662367627e-06	\\
-5621.14479758523	4.33236121280905e-06	\\
-5620.166015625	3.56214291763677e-06	\\
-5619.18723366477	5.83462168635838e-06	\\
-5618.20845170455	3.68586603599502e-06	\\
-5617.22966974432	1.94937361504381e-06	\\
-5616.25088778409	2.44739801943348e-06	\\
-5615.27210582386	1.03689677015613e-06	\\
-5614.29332386364	2.95248712019765e-06	\\
-5613.31454190341	2.28374009697683e-06	\\
-5612.33575994318	2.59943367010581e-06	\\
-5611.35697798296	1.72697080838684e-06	\\
-5610.37819602273	2.93930401277507e-06	\\
-5609.3994140625	2.64776034737558e-06	\\
-5608.42063210227	4.64958618954838e-06	\\
-5607.44185014205	7.21048644362967e-07	\\
-5606.46306818182	3.72841523131879e-06	\\
-5605.48428622159	2.83214087268586e-06	\\
-5604.50550426136	2.71029655308374e-06	\\
-5603.52672230114	1.72453714113792e-06	\\
-5602.54794034091	1.17853549550202e-06	\\
-5601.56915838068	2.34429313877198e-06	\\
-5600.59037642046	1.89620147350251e-06	\\
-5599.61159446023	2.14615058898293e-06	\\
-5598.6328125	2.40673385492808e-06	\\
-5597.65403053977	8.17209115154463e-07	\\
-5596.67524857955	2.04737181678572e-06	\\
-5595.69646661932	3.3395622015146e-06	\\
-5594.71768465909	3.50591487095135e-06	\\
-5593.73890269886	4.13439686708873e-06	\\
-5592.76012073864	3.98184676653276e-06	\\
-5591.78133877841	1.66413753389027e-06	\\
-5590.80255681818	1.79720278998695e-06	\\
-5589.82377485796	2.09579030600454e-06	\\
-5588.84499289773	1.93516583630776e-06	\\
-5587.8662109375	1.93242425568271e-06	\\
-5586.88742897727	3.31354859278215e-06	\\
-5585.90864701705	1.94529435004197e-06	\\
-5584.92986505682	1.27070454063261e-06	\\
-5583.95108309659	1.03982953163674e-06	\\
-5582.97230113636	2.18748273236117e-06	\\
-5581.99351917614	1.04671350994675e-06	\\
-5581.01473721591	1.9955361955916e-06	\\
-5580.03595525568	1.3488612542671e-06	\\
-5579.05717329546	2.56325724185649e-06	\\
-5578.07839133523	1.48850289800425e-06	\\
-5577.099609375	2.89690156287906e-06	\\
-5576.12082741477	2.50092046554426e-06	\\
-5575.14204545455	2.42834614390103e-06	\\
-5574.16326349432	1.91880568076104e-06	\\
-5573.18448153409	1.31553464819913e-06	\\
-5572.20569957386	3.21953803585953e-06	\\
-5571.22691761364	1.91067861254133e-06	\\
-5570.24813565341	3.82653838026807e-07	\\
-5569.26935369318	3.00941156161105e-06	\\
-5568.29057173296	1.58466560341614e-06	\\
-5567.31178977273	1.56119061716774e-06	\\
-5566.3330078125	6.42827157596374e-07	\\
-5565.35422585227	2.77138455037893e-06	\\
-5564.37544389205	1.77122234722883e-06	\\
-5563.39666193182	3.56923759703695e-06	\\
-5562.41787997159	2.26312475220962e-06	\\
-5561.43909801136	2.11392700314969e-06	\\
-5560.46031605114	1.08679305711696e-06	\\
-5559.48153409091	2.25358104648117e-06	\\
-5558.50275213068	1.737455776883e-06	\\
-5557.52397017046	2.61819644647415e-06	\\
-5556.54518821023	1.8581203867649e-06	\\
-5555.56640625	3.28841160865367e-06	\\
-5554.58762428977	4.36531545173569e-06	\\
-5553.60884232955	1.62527201444339e-06	\\
-5552.63006036932	1.69617049856307e-06	\\
-5551.65127840909	1.39547104342369e-06	\\
-5550.67249644886	2.01929331868661e-06	\\
-5549.69371448864	4.4626462730168e-06	\\
-5548.71493252841	2.84994537666657e-06	\\
-5547.73615056818	1.92805024952396e-06	\\
-5546.75736860796	3.34625942645954e-06	\\
-5545.77858664773	1.29428456009264e-06	\\
-5544.7998046875	2.72585138848447e-06	\\
-5543.82102272727	1.55573109047875e-06	\\
-5542.84224076705	1.81151579571597e-06	\\
-5541.86345880682	3.02663910022844e-06	\\
-5540.88467684659	1.32555962287282e-06	\\
-5539.90589488636	9.9956641359876e-07	\\
-5538.92711292614	2.36925064354264e-06	\\
-5537.94833096591	2.54988169469145e-06	\\
-5536.96954900568	2.28854549944614e-06	\\
-5535.99076704546	2.96055810203108e-06	\\
-5535.01198508523	1.85720305442416e-06	\\
-5534.033203125	9.85784324732954e-07	\\
-5533.05442116477	2.48880731352997e-06	\\
-5532.07563920455	2.8363923362149e-06	\\
-5531.09685724432	1.89566885540811e-06	\\
-5530.11807528409	4.12168035907684e-06	\\
-5529.13929332386	2.28654660140905e-06	\\
-5528.16051136364	1.36647464194486e-06	\\
-5527.18172940341	1.83625398930382e-06	\\
-5526.20294744318	1.43539572244978e-06	\\
-5525.22416548296	9.51265978290034e-07	\\
-5524.24538352273	2.4203643250241e-06	\\
-5523.2666015625	2.3499572543406e-06	\\
-5522.28781960227	3.87557919313462e-06	\\
-5521.30903764205	2.63357439019759e-06	\\
-5520.33025568182	3.04025154860771e-06	\\
-5519.35147372159	4.40538441606794e-06	\\
-5518.37269176136	1.47401990240473e-06	\\
-5517.39390980114	2.15928471225174e-06	\\
-5516.41512784091	3.43889172519947e-06	\\
-5515.43634588068	2.55087686986875e-06	\\
-5514.45756392046	3.75267268212606e-06	\\
-5513.47878196023	9.98909627085533e-07	\\
-5512.5	1.63834666168623e-06	\\
-5511.52121803977	1.9159204970526e-06	\\
-5510.54243607955	3.97227853660492e-06	\\
-5509.56365411932	2.17808526143806e-06	\\
-5508.58487215909	2.47171491616947e-06	\\
-5507.60609019886	4.11176363350335e-06	\\
-5506.62730823864	2.56204844126422e-06	\\
-5505.64852627841	2.3475277170237e-06	\\
-5504.66974431818	2.26583800692182e-06	\\
-5503.69096235796	1.74584230749751e-06	\\
-5502.71218039773	1.53204575114561e-06	\\
-5501.7333984375	2.63096675466936e-06	\\
-5500.75461647727	1.44188838741465e-06	\\
-5499.77583451705	1.99872182790458e-06	\\
-5498.79705255682	1.80447008280795e-06	\\
-5497.81827059659	3.07043140593241e-06	\\
-5496.83948863636	3.90083729839067e-06	\\
-5495.86070667614	2.91909570791727e-06	\\
-5494.88192471591	2.15143217428906e-06	\\
-5493.90314275568	2.28084364543513e-06	\\
-5492.92436079546	2.86424021006539e-06	\\
-5491.94557883523	2.50459614046506e-06	\\
-5490.966796875	2.73050841282613e-06	\\
-5489.98801491477	2.72544083105568e-06	\\
-5489.00923295455	1.3068766440264e-06	\\
-5488.03045099432	3.32777669582084e-06	\\
-5487.05166903409	3.36292615674927e-06	\\
-5486.07288707386	3.71763126664881e-06	\\
-5485.09410511364	4.58286034767414e-06	\\
-5484.11532315341	1.88851401498737e-06	\\
-5483.13654119318	2.14650786800395e-06	\\
-5482.15775923296	3.96969332828661e-06	\\
-5481.17897727273	2.62221713826437e-06	\\
-5480.2001953125	2.61754973986274e-06	\\
-5479.22141335227	3.87969302625123e-06	\\
-5478.24263139205	3.14702925220268e-06	\\
-5477.26384943182	2.31400212786446e-06	\\
-5476.28506747159	5.41104995251887e-06	\\
-5475.30628551136	4.29930672369003e-06	\\
-5474.32750355114	4.80045365556606e-06	\\
-5473.34872159091	3.65757031496851e-06	\\
-5472.36993963068	2.24424310066123e-06	\\
-5471.39115767046	2.05040301635333e-06	\\
-5470.41237571023	4.61641881209402e-06	\\
-5469.43359375	3.50033042633927e-06	\\
-5468.45481178977	1.98546131998449e-06	\\
-5467.47602982955	3.04359967938048e-06	\\
-5466.49724786932	3.6102258472408e-06	\\
-5465.51846590909	3.46730108071842e-06	\\
-5464.53968394886	4.21753472410353e-06	\\
-5463.56090198864	3.3137210004509e-06	\\
-5462.58212002841	4.97090260783623e-06	\\
-5461.60333806818	1.93533598033263e-06	\\
-5460.62455610796	3.36496789550804e-06	\\
-5459.64577414773	4.17081231938498e-06	\\
-5458.6669921875	2.11945108440207e-06	\\
-5457.68821022727	4.63285479051515e-06	\\
-5456.70942826705	3.67510090423449e-06	\\
-5455.73064630682	3.66735456418892e-06	\\
-5454.75186434659	2.94404130602052e-06	\\
-5453.77308238636	4.01315577137422e-06	\\
-5452.79430042614	4.60981965580343e-06	\\
-5451.81551846591	3.58356244730089e-06	\\
-5450.83673650568	2.94827223611196e-06	\\
-5449.85795454546	3.48953294251815e-06	\\
-5448.87917258523	7.35518687347091e-06	\\
-5447.900390625	3.49319879730547e-06	\\
-5446.92160866477	3.19392670199188e-06	\\
-5445.94282670455	4.58526820037636e-06	\\
-5444.96404474432	3.75452590910588e-06	\\
-5443.98526278409	5.58487000495601e-06	\\
-5443.00648082386	3.9663498038807e-06	\\
-5442.02769886364	5.20289902603444e-06	\\
-5441.04891690341	3.15920497732422e-06	\\
-5440.07013494318	3.11681025295727e-06	\\
-5439.09135298296	5.48062888979438e-06	\\
-5438.11257102273	3.98613333337245e-06	\\
-5437.1337890625	4.2661817863537e-06	\\
-5436.15500710227	5.52711025930034e-06	\\
-5435.17622514205	5.92372231458171e-06	\\
-5434.19744318182	2.73016947506973e-06	\\
-5433.21866122159	5.14144746212968e-06	\\
-5432.23987926136	5.02806012648195e-06	\\
-5431.26109730114	5.60446785975293e-06	\\
-5430.28231534091	4.30672826991119e-06	\\
-5429.30353338068	4.04795761758847e-06	\\
-5428.32475142046	3.19699352200059e-06	\\
-5427.34596946023	3.06057540198154e-06	\\
-5426.3671875	4.73534193303709e-06	\\
-5425.38840553977	3.43112418022194e-06	\\
-5424.40962357955	2.62666249698531e-06	\\
-5423.43084161932	4.01390538501982e-06	\\
-5422.45205965909	3.58553872876674e-06	\\
-5421.47327769886	4.96164581897878e-06	\\
-5420.49449573864	5.12439361927181e-06	\\
-5419.51571377841	4.59816628356286e-06	\\
-5418.53693181818	2.77730985921359e-06	\\
-5417.55814985796	3.59527486643709e-06	\\
-5416.57936789773	5.10742296250142e-06	\\
-5415.6005859375	3.03091571885825e-06	\\
-5414.62180397727	4.37834599897247e-06	\\
-5413.64302201705	5.23140722533141e-06	\\
-5412.66424005682	3.39923608318156e-06	\\
-5411.68545809659	3.73946165092316e-06	\\
-5410.70667613636	4.64894790165254e-06	\\
-5409.72789417614	5.4624704488963e-06	\\
-5408.74911221591	3.8674712129874e-06	\\
-5407.77033025568	4.55606434211635e-06	\\
-5406.79154829546	5.39962587370106e-06	\\
-5405.81276633523	4.58058429977663e-06	\\
-5404.833984375	3.2845330367596e-06	\\
-5403.85520241477	6.73500312999681e-06	\\
-5402.87642045455	4.52940335327135e-06	\\
-5401.89763849432	5.77215999642903e-06	\\
-5400.91885653409	4.7790540368055e-06	\\
-5399.94007457386	2.90031280022907e-06	\\
-5398.96129261364	5.64914925187575e-06	\\
-5397.98251065341	5.46731549425516e-06	\\
-5397.00372869318	5.03211490314501e-06	\\
-5396.02494673296	6.5806676604055e-06	\\
-5395.04616477273	6.37899022588532e-06	\\
-5394.0673828125	4.55001076621074e-06	\\
-5393.08860085227	4.00994495457421e-06	\\
-5392.10981889205	6.12788146133799e-06	\\
-5391.13103693182	5.17344232571752e-06	\\
-5390.15225497159	6.26589832473426e-06	\\
-5389.17347301136	5.62065065497103e-06	\\
-5388.19469105114	4.0627960718839e-06	\\
-5387.21590909091	5.56744443191845e-06	\\
-5386.23712713068	5.01916967009867e-06	\\
-5385.25834517046	5.18554179816559e-06	\\
-5384.27956321023	6.78158026196859e-06	\\
-5383.30078125	4.75656788903514e-06	\\
-5382.32199928977	4.78580726223221e-06	\\
-5381.34321732955	6.81682875598599e-06	\\
-5380.36443536932	5.16391467458329e-06	\\
-5379.38565340909	5.74308387970533e-06	\\
-5378.40687144886	6.18499592433606e-06	\\
-5377.42808948864	4.78515288893764e-06	\\
-5376.44930752841	6.51770629097137e-06	\\
-5375.47052556818	5.00656629703358e-06	\\
-5374.49174360796	5.66585759318276e-06	\\
-5373.51296164773	5.6735085431436e-06	\\
-5372.5341796875	4.14542618124596e-06	\\
-5371.55539772727	7.51032629897029e-06	\\
-5370.57661576705	6.54101948231723e-06	\\
-5369.59783380682	5.93367167568714e-06	\\
-5368.61905184659	3.99740145876865e-06	\\
-5367.64026988636	5.07154628033617e-06	\\
-5366.66148792614	6.00743078945007e-06	\\
-5365.68270596591	5.96139616157358e-06	\\
-5364.70392400568	6.61859295652934e-06	\\
-5363.72514204546	5.39456817200177e-06	\\
-5362.74636008523	5.99827914901776e-06	\\
-5361.767578125	5.23294769093151e-06	\\
-5360.78879616477	5.75911431932695e-06	\\
-5359.81001420455	4.57319373482484e-06	\\
-5358.83123224432	6.79640723645879e-06	\\
-5357.85245028409	5.02056144951631e-06	\\
-5356.87366832386	3.96743940527714e-06	\\
-5355.89488636364	4.90766580550344e-06	\\
-5354.91610440341	3.10834656864339e-06	\\
-5353.93732244318	5.55821316989416e-06	\\
-5352.95854048296	8.22220305659903e-06	\\
-5351.97975852273	5.54787711825419e-06	\\
-5351.0009765625	6.18342334840672e-06	\\
-5350.02219460227	4.95997654613046e-06	\\
-5349.04341264205	8.68264536346362e-06	\\
-5348.06463068182	5.62084329909844e-06	\\
-5347.08584872159	5.72184289747337e-06	\\
-5346.10706676136	5.32244617641599e-06	\\
-5345.12828480114	5.01060606755444e-06	\\
-5344.14950284091	6.69566534636342e-06	\\
-5343.17072088068	4.43744629913431e-06	\\
-5342.19193892046	6.99017042286526e-06	\\
-5341.21315696023	6.33402208319315e-06	\\
-5340.234375	8.28199732998112e-06	\\
-5339.25559303977	6.30530873730269e-06	\\
-5338.27681107955	5.51832506320012e-06	\\
-5337.29802911932	7.56322115061055e-06	\\
-5336.31924715909	7.02618856071983e-06	\\
-5335.34046519886	5.74263700454656e-06	\\
-5334.36168323864	8.10302600409124e-06	\\
-5333.38290127841	8.0557234968821e-06	\\
-5332.40411931818	4.99500889058363e-06	\\
-5331.42533735796	7.14530208353253e-06	\\
-5330.44655539773	5.59757616985623e-06	\\
-5329.4677734375	6.29278780488368e-06	\\
-5328.48899147727	6.21298353553175e-06	\\
-5327.51020951705	5.28076468548383e-06	\\
-5326.53142755682	9.33993613889446e-06	\\
-5325.55264559659	7.72697405764298e-06	\\
-5324.57386363636	6.0801300725747e-06	\\
-5323.59508167614	5.5059769069516e-06	\\
-5322.61629971591	7.70840090816852e-06	\\
-5321.63751775568	8.40893889786133e-06	\\
-5320.65873579546	6.27378404631214e-06	\\
-5319.67995383523	6.71214839521846e-06	\\
-5318.701171875	5.07545279642176e-06	\\
-5317.72238991477	7.11955001449307e-06	\\
-5316.74360795455	8.33576139919797e-06	\\
-5315.76482599432	7.47417861794247e-06	\\
-5314.78604403409	7.31111146595509e-06	\\
-5313.80726207386	7.24818731030994e-06	\\
-5312.82848011364	4.79604089785092e-06	\\
-5311.84969815341	6.93569585685143e-06	\\
-5310.87091619318	6.36016113875627e-06	\\
-5309.89213423296	6.02734696670679e-06	\\
-5308.91335227273	6.38877101513248e-06	\\
-5307.9345703125	7.4598346544008e-06	\\
-5306.95578835227	8.71001349253767e-06	\\
-5305.97700639205	5.20835494418126e-06	\\
-5304.99822443182	7.19602442156513e-06	\\
-5304.01944247159	7.88702302981969e-06	\\
-5303.04066051136	8.79755904480906e-06	\\
-5302.06187855114	9.48253101283424e-06	\\
-5301.08309659091	8.18310726453467e-06	\\
-5300.10431463068	6.37142975277194e-06	\\
-5299.12553267046	6.96435097530212e-06	\\
-5298.14675071023	7.8141632639685e-06	\\
-5297.16796875	6.3713062123929e-06	\\
-5296.18918678977	7.06184352019165e-06	\\
-5295.21040482955	7.7233911747051e-06	\\
-5294.23162286932	5.6611157498365e-06	\\
-5293.25284090909	6.09931949140241e-06	\\
-5292.27405894886	8.48974115894688e-06	\\
-5291.29527698864	7.6050250409256e-06	\\
-5290.31649502841	8.11725756226479e-06	\\
-5289.33771306818	1.03061093361066e-05	\\
-5288.35893110796	8.1422746245765e-06	\\
-5287.38014914773	8.24841712726116e-06	\\
-5286.4013671875	5.87053459300198e-06	\\
-5285.42258522727	6.52334930179393e-06	\\
-5284.44380326705	7.51058982346251e-06	\\
-5283.46502130682	7.41930296790914e-06	\\
-5282.48623934659	7.87947997982137e-06	\\
-5281.50745738636	5.38231469700063e-06	\\
-5280.52867542614	8.9270775445994e-06	\\
-5279.54989346591	7.06220135534147e-06	\\
-5278.57111150568	1.01312312349022e-05	\\
-5277.59232954546	6.5356503392288e-06	\\
-5276.61354758523	6.68106082676831e-06	\\
-5275.634765625	7.63301530388815e-06	\\
-5274.65598366477	9.84469485595679e-06	\\
-5273.67720170455	5.74961320827577e-06	\\
-5272.69841974432	8.11594989694408e-06	\\
-5271.71963778409	6.95278746584165e-06	\\
-5270.74085582386	8.33328182716991e-06	\\
-5269.76207386364	8.71109936213589e-06	\\
-5268.78329190341	9.14170147927029e-06	\\
-5267.80450994318	8.44761452907982e-06	\\
-5266.82572798296	9.52631972229368e-06	\\
-5265.84694602273	9.46730650116492e-06	\\
-5264.8681640625	8.51868658169387e-06	\\
-5263.88938210227	7.57653967756023e-06	\\
-5262.91060014205	7.35499449713878e-06	\\
-5261.93181818182	8.31947119182719e-06	\\
-5260.95303622159	8.24847357848963e-06	\\
-5259.97425426136	7.67749993728488e-06	\\
-5258.99547230114	7.70806530591943e-06	\\
-5258.01669034091	9.12959258423361e-06	\\
-5257.03790838068	8.83124181947338e-06	\\
-5256.05912642046	9.77580752993282e-06	\\
-5255.08034446023	7.45947433868711e-06	\\
-5254.1015625	8.38676633721425e-06	\\
-5253.12278053977	9.54094187709251e-06	\\
-5252.14399857955	9.43323961642247e-06	\\
-5251.16521661932	5.61374323166661e-06	\\
-5250.18643465909	1.01462052408556e-05	\\
-5249.20765269886	7.58224764829982e-06	\\
-5248.22887073864	8.64247602364868e-06	\\
-5247.25008877841	9.29629711504888e-06	\\
-5246.27130681818	8.8308376931459e-06	\\
-5245.29252485796	1.01584037883701e-05	\\
-5244.31374289773	9.44168408197482e-06	\\
-5243.3349609375	7.99688681109769e-06	\\
-5242.35617897727	8.49499412920708e-06	\\
-5241.37739701705	1.03772821074155e-05	\\
-5240.39861505682	8.35993005202834e-06	\\
-5239.41983309659	7.7840538851796e-06	\\
-5238.44105113636	6.35773824912379e-06	\\
-5237.46226917614	9.14085480665667e-06	\\
-5236.48348721591	1.06137920169473e-05	\\
-5235.50470525568	8.47193127094265e-06	\\
-5234.52592329546	7.25223137428109e-06	\\
-5233.54714133523	1.0451868143951e-05	\\
-5232.568359375	8.69764753806636e-06	\\
-5231.58957741477	7.497153209043e-06	\\
-5230.61079545455	8.93194035569029e-06	\\
-5229.63201349432	9.59128494678791e-06	\\
-5228.65323153409	7.50545411255459e-06	\\
-5227.67444957386	1.03791556782597e-05	\\
-5226.69566761364	9.08782170345361e-06	\\
-5225.71688565341	7.96289431080868e-06	\\
-5224.73810369318	7.83028882921204e-06	\\
-5223.75932173296	7.37573015148506e-06	\\
-5222.78053977273	7.72528658827423e-06	\\
-5221.8017578125	9.0410553831789e-06	\\
-5220.82297585227	6.81242847588497e-06	\\
-5219.84419389205	1.07918622662269e-05	\\
-5218.86541193182	9.33370564864592e-06	\\
-5217.88662997159	8.40565819239667e-06	\\
-5216.90784801136	7.72798162597143e-06	\\
-5215.92906605114	8.91865390508975e-06	\\
-5214.95028409091	7.95540242621574e-06	\\
-5213.97150213068	8.92015378202372e-06	\\
-5212.99272017046	8.68299163659968e-06	\\
-5212.01393821023	1.00401118626219e-05	\\
-5211.03515625	1.26506698012263e-05	\\
-5210.05637428977	1.06979131802917e-05	\\
-5209.07759232955	8.4067015229925e-06	\\
-5208.09881036932	9.37319445901511e-06	\\
-5207.12002840909	8.0043443234627e-06	\\
-5206.14124644886	9.74173551972345e-06	\\
-5205.16246448864	1.00113451833129e-05	\\
-5204.18368252841	8.98891007542519e-06	\\
-5203.20490056818	8.16962905565869e-06	\\
-5202.22611860796	8.30546331317969e-06	\\
-5201.24733664773	9.34265267710763e-06	\\
-5200.2685546875	8.50294347426778e-06	\\
-5199.28977272727	1.083715198097e-05	\\
-5198.31099076705	9.11398981192027e-06	\\
-5197.33220880682	8.94170775384942e-06	\\
-5196.35342684659	9.55446111427073e-06	\\
-5195.37464488636	9.62928742924164e-06	\\
-5194.39586292614	8.26242295124987e-06	\\
-5193.41708096591	8.55045404932202e-06	\\
-5192.43829900568	1.02368502262597e-05	\\
-5191.45951704546	1.16301911745168e-05	\\
-5190.48073508523	1.08421615295379e-05	\\
-5189.501953125	8.91745227423467e-06	\\
-5188.52317116477	9.64461268062731e-06	\\
-5187.54438920455	9.8232085013634e-06	\\
-5186.56560724432	9.38230431662936e-06	\\
-5185.58682528409	9.96574400170652e-06	\\
-5184.60804332386	9.7126649985473e-06	\\
-5183.62926136364	1.00244928143893e-05	\\
-5182.65047940341	1.00957548120783e-05	\\
-5181.67169744318	9.66316270766932e-06	\\
-5180.69291548296	1.00880083844038e-05	\\
-5179.71413352273	1.16292951727276e-05	\\
-5178.7353515625	9.87859910617639e-06	\\
-5177.75656960227	1.01690673670092e-05	\\
-5176.77778764205	1.00474912585655e-05	\\
-5175.79900568182	9.80731471286379e-06	\\
-5174.82022372159	9.87868205917725e-06	\\
-5173.84144176136	1.17495791204833e-05	\\
-5172.86265980114	9.49504906691263e-06	\\
-5171.88387784091	9.28242289318479e-06	\\
-5170.90509588068	7.55654782443623e-06	\\
-5169.92631392046	8.82602737998598e-06	\\
-5168.94753196023	9.90278235227101e-06	\\
-5167.96875	1.06642987679196e-05	\\
-5166.98996803977	1.20291156300775e-05	\\
-5166.01118607955	1.14248382779741e-05	\\
-5165.03240411932	9.86041571311596e-06	\\
-5164.05362215909	1.08272195713392e-05	\\
-5163.07484019886	8.8970066730408e-06	\\
-5162.09605823864	1.06775902387809e-05	\\
-5161.11727627841	9.5060054410902e-06	\\
-5160.13849431818	1.14948617898034e-05	\\
-5159.15971235796	1.08465477756061e-05	\\
-5158.18093039773	1.0341058102373e-05	\\
-5157.2021484375	1.03362178547837e-05	\\
-5156.22336647727	1.04841424970098e-05	\\
-5155.24458451705	1.06700763536672e-05	\\
-5154.26580255682	1.10558800440652e-05	\\
-5153.28702059659	8.98176781824093e-06	\\
-5152.30823863636	1.31106224679887e-05	\\
-5151.32945667614	8.14673463578032e-06	\\
-5150.35067471591	1.090577903594e-05	\\
-5149.37189275568	1.11259620698797e-05	\\
-5148.39311079546	9.97586578082539e-06	\\
-5147.41432883523	8.44556317327079e-06	\\
-5146.435546875	1.22386869632293e-05	\\
-5145.45676491477	8.39812274939686e-06	\\
-5144.47798295455	1.14511303207811e-05	\\
-5143.49920099432	1.23593116494116e-05	\\
-5142.52041903409	1.19706574691733e-05	\\
-5141.54163707386	1.10764013229568e-05	\\
-5140.56285511364	1.1940443731773e-05	\\
-5139.58407315341	1.094858599789e-05	\\
-5138.60529119318	1.22633574327934e-05	\\
-5137.62650923296	1.18969483390149e-05	\\
-5136.64772727273	1.04988798707441e-05	\\
-5135.6689453125	1.01299711702282e-05	\\
-5134.69016335227	1.18616350368732e-05	\\
-5133.71138139205	1.05916962570627e-05	\\
-5132.73259943182	1.29401774285617e-05	\\
-5131.75381747159	1.12975384743272e-05	\\
-5130.77503551136	1.07616593258064e-05	\\
-5129.79625355114	1.14948620038817e-05	\\
-5128.81747159091	1.0355731402568e-05	\\
-5127.83868963068	1.0745211973741e-05	\\
-5126.85990767046	1.04842715731993e-05	\\
-5125.88112571023	1.09455762466489e-05	\\
-5124.90234375	1.08195064182716e-05	\\
-5123.92356178977	1.21495221331822e-05	\\
-5122.94477982955	1.27391261815541e-05	\\
-5121.96599786932	1.29479692715903e-05	\\
-5120.98721590909	1.37398473711899e-05	\\
-5120.00843394886	1.08831209439848e-05	\\
-5119.02965198864	1.16900243739636e-05	\\
-5118.05087002841	1.23988483666045e-05	\\
-5117.07208806818	1.17538125214182e-05	\\
-5116.09330610796	1.04683157182517e-05	\\
-5115.11452414773	1.24045713285598e-05	\\
-5114.1357421875	1.16623455872092e-05	\\
-5113.15696022727	1.14166554382953e-05	\\
-5112.17817826705	1.22238617930523e-05	\\
-5111.19939630682	1.17525984104873e-05	\\
-5110.22061434659	1.21637552406209e-05	\\
-5109.24183238636	1.1916340332677e-05	\\
-5108.26305042614	1.07070333196695e-05	\\
-5107.28426846591	1.22227311265977e-05	\\
-5106.30548650568	1.08592064014383e-05	\\
-5105.32670454546	1.1951687791438e-05	\\
-5104.34792258523	1.13326883920386e-05	\\
-5103.369140625	1.04003765593207e-05	\\
-5102.39035866477	1.18307131714877e-05	\\
-5101.41157670455	1.15388689413153e-05	\\
-5100.43279474432	1.20191320784505e-05	\\
-5099.45401278409	1.09841657090204e-05	\\
-5098.47523082386	1.19365285506424e-05	\\
-5097.49644886364	1.0438161262533e-05	\\
-5096.51766690341	1.2040315596342e-05	\\
-5095.53888494318	1.32962710860665e-05	\\
-5094.56010298296	1.0899645268167e-05	\\
-5093.58132102273	1.2339557416585e-05	\\
-5092.6025390625	1.34142951558377e-05	\\
-5091.62375710227	1.24810882999824e-05	\\
-5090.64497514205	1.32973633270412e-05	\\
-5089.66619318182	1.21311322396846e-05	\\
-5088.68741122159	1.28214663562429e-05	\\
-5087.70862926136	1.170866272073e-05	\\
-5086.72984730114	1.26702890101183e-05	\\
-5085.75106534091	1.35266732170701e-05	\\
-5084.77228338068	1.17535981817518e-05	\\
-5083.79350142046	1.2145038753289e-05	\\
-5082.81471946023	1.21563724680173e-05	\\
-5081.8359375	1.23887408614183e-05	\\
-5080.85715553977	1.29898828987879e-05	\\
-5079.87837357955	1.18415797187407e-05	\\
-5078.89959161932	1.32895054785241e-05	\\
-5077.92080965909	1.18261966941396e-05	\\
-5076.94202769886	1.05172512682507e-05	\\
-5075.96324573864	1.3958005711732e-05	\\
-5074.98446377841	1.1074877664363e-05	\\
-5074.00568181818	1.42011700591108e-05	\\
-5073.02689985796	1.32943355424184e-05	\\
-5072.04811789773	1.19612807650232e-05	\\
-5071.0693359375	1.19570959035584e-05	\\
-5070.09055397727	1.32017392511447e-05	\\
-5069.11177201705	1.15244585816157e-05	\\
-5068.13299005682	1.27451052450687e-05	\\
-5067.15420809659	1.3173312989528e-05	\\
-5066.17542613636	1.42494593642031e-05	\\
-5065.19664417614	1.3362626667033e-05	\\
-5064.21786221591	1.55467942640596e-05	\\
-5063.23908025568	1.31922456489795e-05	\\
-5062.26029829546	1.18783872813711e-05	\\
-5061.28151633523	1.3983089460575e-05	\\
-5060.302734375	1.3350197564548e-05	\\
-5059.32395241477	1.46149386460475e-05	\\
-5058.34517045455	1.20465644962356e-05	\\
-5057.36638849432	1.22068127244601e-05	\\
-5056.38760653409	1.50726316937125e-05	\\
-5055.40882457386	1.35404614271219e-05	\\
-5054.43004261364	1.31244579648498e-05	\\
-5053.45126065341	1.40016986954069e-05	\\
-5052.47247869318	1.19647986467653e-05	\\
-5051.49369673296	1.18074045463803e-05	\\
-5050.51491477273	1.46202498644477e-05	\\
-5049.5361328125	1.16345525344174e-05	\\
-5048.55735085227	1.04469068061881e-05	\\
-5047.57856889205	1.28803300965049e-05	\\
-5046.59978693182	1.34112226107101e-05	\\
-5045.62100497159	1.41410771812582e-05	\\
-5044.64222301136	1.46234692918679e-05	\\
-5043.66344105114	1.53942896661112e-05	\\
-5042.68465909091	1.33206850583128e-05	\\
-5041.70587713068	1.29210348313316e-05	\\
-5040.72709517046	1.2531619476005e-05	\\
-5039.74831321023	1.32361306286663e-05	\\
-5038.76953125	1.38391329824032e-05	\\
-5037.79074928977	1.5186814593088e-05	\\
-5036.81196732955	1.5612407442259e-05	\\
-5035.83318536932	1.47184718255408e-05	\\
-5034.85440340909	1.50375961555609e-05	\\
-5033.87562144886	1.13427766886002e-05	\\
-5032.89683948864	1.3766682231916e-05	\\
-5031.91805752841	1.39317750713881e-05	\\
-5030.93927556818	1.23633258999866e-05	\\
-5029.96049360796	1.07694321869024e-05	\\
-5028.98171164773	1.54413018028926e-05	\\
-5028.0029296875	1.28431257858492e-05	\\
-5027.02414772727	1.34548510816308e-05	\\
-5026.04536576705	1.29074330652568e-05	\\
-5025.06658380682	1.38921382535061e-05	\\
-5024.08780184659	1.26674860583442e-05	\\
-5023.10901988636	1.32015478306994e-05	\\
-5022.13023792614	1.32442185569365e-05	\\
-5021.15145596591	1.43123206224825e-05	\\
-5020.17267400568	1.3813422143605e-05	\\
-5019.19389204546	1.4512914036516e-05	\\
-5018.21511008523	1.2545343904138e-05	\\
-5017.236328125	1.55170301286565e-05	\\
-5016.25754616477	1.36418356608059e-05	\\
-5015.27876420455	1.2864749429143e-05	\\
-5014.29998224432	1.28971369586751e-05	\\
-5013.32120028409	1.50772740014307e-05	\\
-5012.34241832386	1.39207941958743e-05	\\
-5011.36363636364	1.46575322760948e-05	\\
-5010.38485440341	1.44368754165396e-05	\\
-5009.40607244318	1.30619572060789e-05	\\
-5008.42729048296	1.37020891964775e-05	\\
-5007.44850852273	1.35070772217017e-05	\\
-5006.4697265625	1.53854957615744e-05	\\
-5005.49094460227	1.46965845512475e-05	\\
-5004.51216264205	1.41216581022507e-05	\\
-5003.53338068182	1.48193207936811e-05	\\
-5002.55459872159	1.52253194762034e-05	\\
-5001.57581676136	1.49563998426017e-05	\\
-5000.59703480114	1.25594044505984e-05	\\
-4999.61825284091	1.42698968137538e-05	\\
-4998.63947088068	1.45727123630716e-05	\\
-4997.66068892046	1.37261524688853e-05	\\
-4996.68190696023	1.47557735920408e-05	\\
-4995.703125	1.46044499687616e-05	\\
-4994.72434303977	1.29965776283105e-05	\\
-4993.74556107955	1.48621341865424e-05	\\
-4992.76677911932	1.60788082580717e-05	\\
-4991.78799715909	1.57579932108106e-05	\\
-4990.80921519886	1.47329011192349e-05	\\
-4989.83043323864	1.42315974786696e-05	\\
-4988.85165127841	1.13216152848398e-05	\\
-4987.87286931818	1.40979564391709e-05	\\
-4986.89408735796	1.55832646959035e-05	\\
-4985.91530539773	1.42919117747532e-05	\\
-4984.9365234375	1.41354480869629e-05	\\
-4983.95774147727	1.5636520630716e-05	\\
-4982.97895951705	1.45321145718836e-05	\\
-4982.00017755682	1.45680520179693e-05	\\
-4981.02139559659	1.43129848744843e-05	\\
-4980.04261363636	1.74975010618703e-05	\\
-4979.06383167614	1.37853004668816e-05	\\
-4978.08504971591	1.52226523569622e-05	\\
-4977.10626775568	1.6463586348437e-05	\\
-4976.12748579546	1.52395167292643e-05	\\
-4975.14870383523	1.64106138044906e-05	\\
-4974.169921875	1.37568251923925e-05	\\
-4973.19113991477	1.37971074724687e-05	\\
-4972.21235795455	1.55069075644611e-05	\\
-4971.23357599432	1.7304408001125e-05	\\
-4970.25479403409	1.39768680854514e-05	\\
-4969.27601207386	1.43768463795073e-05	\\
-4968.29723011364	1.44965345583009e-05	\\
-4967.31844815341	1.54925806678082e-05	\\
-4966.33966619318	1.65122108875955e-05	\\
-4965.36088423296	1.56459047514577e-05	\\
-4964.38210227273	1.60092149565778e-05	\\
-4963.4033203125	1.4299505469702e-05	\\
-4962.42453835227	1.68818791261352e-05	\\
-4961.44575639205	1.49010055575935e-05	\\
-4960.46697443182	1.58494600858405e-05	\\
-4959.48819247159	1.52903799934745e-05	\\
-4958.50941051136	1.58271229107658e-05	\\
-4957.53062855114	1.56255238654886e-05	\\
-4956.55184659091	1.51229706070278e-05	\\
-4955.57306463068	1.55339469431808e-05	\\
-4954.59428267046	1.60173990888204e-05	\\
-4953.61550071023	1.6726100206074e-05	\\
-4952.63671875	1.72462944661927e-05	\\
-4951.65793678977	1.39003031111797e-05	\\
-4950.67915482955	1.5929648763658e-05	\\
-4949.70037286932	1.53879359239252e-05	\\
-4948.72159090909	1.56561534382226e-05	\\
-4947.74280894886	1.67265469040876e-05	\\
-4946.76402698864	1.55282583300416e-05	\\
-4945.78524502841	1.75061813757617e-05	\\
-4944.80646306818	1.4298938239469e-05	\\
-4943.82768110796	1.41231057863203e-05	\\
-4942.84889914773	1.62713308674954e-05	\\
-4941.8701171875	1.72071034663971e-05	\\
-4940.89133522727	1.63787468790245e-05	\\
-4939.91255326705	1.54268172263567e-05	\\
-4938.93377130682	1.59188674757494e-05	\\
-4937.95498934659	1.5783504213248e-05	\\
-4936.97620738636	1.56942450840825e-05	\\
-4935.99742542614	1.78397026696931e-05	\\
-4935.01864346591	1.75963037402781e-05	\\
-4934.03986150568	1.46937650665324e-05	\\
-4933.06107954546	1.63883782284827e-05	\\
-4932.08229758523	1.75672797511568e-05	\\
-4931.103515625	1.72853032588585e-05	\\
-4930.12473366477	1.39586571139512e-05	\\
-4929.14595170455	1.7403027123715e-05	\\
-4928.16716974432	1.69499200708182e-05	\\
-4927.18838778409	1.36494390830952e-05	\\
-4926.20960582386	1.364147954605e-05	\\
-4925.23082386364	1.54529161247138e-05	\\
-4924.25204190341	1.5758044474239e-05	\\
-4923.27325994318	1.61770275674424e-05	\\
-4922.29447798296	1.61225849073919e-05	\\
-4921.31569602273	1.77424939295402e-05	\\
-4920.3369140625	1.75899280783252e-05	\\
-4919.35813210227	1.65395818802921e-05	\\
-4918.37935014205	1.80402331511489e-05	\\
-4917.40056818182	1.71653569311946e-05	\\
-4916.42178622159	1.73401021537326e-05	\\
-4915.44300426136	1.52149358660077e-05	\\
-4914.46422230114	1.66597465484779e-05	\\
-4913.48544034091	1.57483187546445e-05	\\
-4912.50665838068	1.46890927109404e-05	\\
-4911.52787642046	1.59318939658553e-05	\\
-4910.54909446023	1.66622220397207e-05	\\
-4909.5703125	1.64977833362471e-05	\\
-4908.59153053977	1.50606340003156e-05	\\
-4907.61274857955	1.84750012610085e-05	\\
-4906.63396661932	1.57232900493356e-05	\\
-4905.65518465909	1.68692758141966e-05	\\
-4904.67640269886	1.54879731255346e-05	\\
-4903.69762073864	1.77522617630349e-05	\\
-4902.71883877841	1.70810845545941e-05	\\
-4901.74005681818	1.73888595023739e-05	\\
-4900.76127485796	1.65032944189537e-05	\\
-4899.78249289773	1.85920077643208e-05	\\
-4898.8037109375	1.81821578530808e-05	\\
-4897.82492897727	1.6090147864985e-05	\\
-4896.84614701705	1.83705699425739e-05	\\
-4895.86736505682	1.52427232092729e-05	\\
-4894.88858309659	1.65451072326359e-05	\\
-4893.90980113636	1.69867229480863e-05	\\
-4892.93101917614	1.74953086684252e-05	\\
-4891.95223721591	1.76603486363206e-05	\\
-4890.97345525568	1.71283220624368e-05	\\
-4889.99467329546	1.9760945207341e-05	\\
-4889.01589133523	1.89118114526873e-05	\\
-4888.037109375	1.71879523235259e-05	\\
-4887.05832741477	1.93035821217663e-05	\\
-4886.07954545455	2.0001869321531e-05	\\
-4885.10076349432	1.88223369295706e-05	\\
-4884.12198153409	1.93841255288325e-05	\\
-4883.14319957386	1.6689665988861e-05	\\
-4882.16441761364	1.94973625366227e-05	\\
-4881.18563565341	2.0619465712161e-05	\\
-4880.20685369318	1.83058814013946e-05	\\
-4879.22807173296	1.91924634818596e-05	\\
-4878.24928977273	1.74376599152288e-05	\\
-4877.2705078125	1.88983066716961e-05	\\
-4876.29172585227	1.93023707886227e-05	\\
-4875.31294389205	1.94578984301067e-05	\\
-4874.33416193182	1.70735078067142e-05	\\
-4873.35537997159	1.9313395288316e-05	\\
-4872.37659801136	1.79897944958303e-05	\\
-4871.39781605114	1.84372188014044e-05	\\
-4870.41903409091	1.96100462909191e-05	\\
-4869.44025213068	1.8141774128237e-05	\\
-4868.46147017046	1.93969727054338e-05	\\
-4867.48268821023	1.92047199289914e-05	\\
-4866.50390625	1.90700297907524e-05	\\
-4865.52512428977	1.88638608028199e-05	\\
-4864.54634232955	1.92723578405879e-05	\\
-4863.56756036932	1.88695756527491e-05	\\
-4862.58877840909	1.85755095022445e-05	\\
-4861.60999644886	1.82089393223765e-05	\\
-4860.63121448864	1.98317202384471e-05	\\
-4859.65243252841	1.96365887112293e-05	\\
-4858.67365056818	1.95206716984508e-05	\\
-4857.69486860796	2.05141453695047e-05	\\
-4856.71608664773	1.98006503430818e-05	\\
-4855.7373046875	1.82467967111204e-05	\\
-4854.75852272727	1.93461039106264e-05	\\
-4853.77974076705	2.00333806251267e-05	\\
-4852.80095880682	1.83292171770646e-05	\\
-4851.82217684659	1.94573945912405e-05	\\
-4850.84339488636	1.95250375201066e-05	\\
-4849.86461292614	1.89568243250859e-05	\\
-4848.88583096591	2.15101960108289e-05	\\
-4847.90704900568	2.26445250542874e-05	\\
-4846.92826704546	2.01197740884707e-05	\\
-4845.94948508523	2.00215624297193e-05	\\
-4844.970703125	2.13574272591831e-05	\\
-4843.99192116477	1.91999758093064e-05	\\
-4843.01313920455	2.0903611477947e-05	\\
-4842.03435724432	2.08383207307465e-05	\\
-4841.05557528409	2.19868695742328e-05	\\
-4840.07679332386	2.04833138497479e-05	\\
-4839.09801136364	2.03537436366129e-05	\\
-4838.11922940341	1.95141759199961e-05	\\
-4837.14044744318	2.07397801571515e-05	\\
-4836.16166548296	2.0148143634241e-05	\\
-4835.18288352273	2.22075459235524e-05	\\
-4834.2041015625	1.91229100799716e-05	\\
-4833.22531960227	1.92865362986879e-05	\\
-4832.24653764205	2.07512965397224e-05	\\
-4831.26775568182	2.02125984855054e-05	\\
-4830.28897372159	1.84203612871886e-05	\\
-4829.31019176136	1.98727863946462e-05	\\
-4828.33140980114	2.13750595499101e-05	\\
-4827.35262784091	1.91293697815364e-05	\\
-4826.37384588068	2.13126891824438e-05	\\
-4825.39506392046	1.9719893076554e-05	\\
-4824.41628196023	1.88521069282704e-05	\\
-4823.4375	2.05662742299835e-05	\\
-4822.45871803977	1.85180461384951e-05	\\
-4821.47993607955	1.93605953617648e-05	\\
-4820.50115411932	2.06192257918496e-05	\\
-4819.52237215909	2.06729176377349e-05	\\
-4818.54359019886	2.03391044924687e-05	\\
-4817.56480823864	2.21821395818628e-05	\\
-4816.58602627841	1.80724118683929e-05	\\
-4815.60724431818	1.78102962942985e-05	\\
-4814.62846235796	1.89109430653268e-05	\\
-4813.64968039773	1.97956733408894e-05	\\
-4812.6708984375	1.9153758513585e-05	\\
-4811.69211647727	1.95647110569905e-05	\\
-4810.71333451705	2.16572193365178e-05	\\
-4809.73455255682	1.89325833345783e-05	\\
-4808.75577059659	2.03852696976586e-05	\\
-4807.77698863636	2.16214120269657e-05	\\
-4806.79820667614	2.1669918267842e-05	\\
-4805.81942471591	2.15486283133539e-05	\\
-4804.84064275568	2.18633768829945e-05	\\
-4803.86186079546	2.03061228598077e-05	\\
-4802.88307883523	2.15230610746463e-05	\\
-4801.904296875	2.2223335843652e-05	\\
-4800.92551491477	2.0777762797389e-05	\\
-4799.94673295455	2.09842214992362e-05	\\
-4798.96795099432	1.97233898608012e-05	\\
-4797.98916903409	2.00037295301326e-05	\\
-4797.01038707386	1.99671474248562e-05	\\
-4796.03160511364	2.18712636615266e-05	\\
-4795.05282315341	2.09403515314977e-05	\\
-4794.07404119318	2.14383400681894e-05	\\
-4793.09525923296	1.99495591720587e-05	\\
-4792.11647727273	2.21238886419044e-05	\\
-4791.1376953125	1.84576340117297e-05	\\
-4790.15891335227	2.13078545034211e-05	\\
-4789.18013139205	2.04263269422406e-05	\\
-4788.20134943182	1.90596393154593e-05	\\
-4787.22256747159	2.08330242346759e-05	\\
-4786.24378551136	2.11855807693084e-05	\\
-4785.26500355114	2.10126365272023e-05	\\
-4784.28622159091	2.20761886736583e-05	\\
-4783.30743963068	2.05708180410383e-05	\\
-4782.32865767046	2.16892178813489e-05	\\
-4781.34987571023	2.14704341098814e-05	\\
-4780.37109375	2.15200549995047e-05	\\
-4779.39231178977	2.11852570254735e-05	\\
-4778.41352982955	1.96827958910457e-05	\\
-4777.43474786932	2.22718422559832e-05	\\
-4776.45596590909	2.24605028842021e-05	\\
-4775.47718394886	2.2138574637177e-05	\\
-4774.49840198864	2.12482658461466e-05	\\
-4773.51962002841	2.14842617633293e-05	\\
-4772.54083806818	2.27014193497636e-05	\\
-4771.56205610796	1.83128520449957e-05	\\
-4770.58327414773	2.14166420087582e-05	\\
-4769.6044921875	2.09771702802687e-05	\\
-4768.62571022727	2.34877445406811e-05	\\
-4767.64692826705	2.09986597654534e-05	\\
-4766.66814630682	2.35541825197053e-05	\\
-4765.68936434659	2.12330734367231e-05	\\
-4764.71058238636	2.26339032877066e-05	\\
-4763.73180042614	2.10042445575313e-05	\\
-4762.75301846591	2.28046719648014e-05	\\
-4761.77423650568	2.28858515489981e-05	\\
-4760.79545454546	2.24248025376796e-05	\\
-4759.81667258523	2.04217376058972e-05	\\
-4758.837890625	2.12912785679423e-05	\\
-4757.85910866477	2.28341420971678e-05	\\
-4756.88032670455	2.2128578588028e-05	\\
-4755.90154474432	2.17977105659914e-05	\\
-4754.92276278409	2.19048581302078e-05	\\
-4753.94398082386	2.28756655110585e-05	\\
-4752.96519886364	1.96673934302712e-05	\\
-4751.98641690341	2.05797427163411e-05	\\
-4751.00763494318	2.19781936989413e-05	\\
-4750.02885298296	1.93132443545283e-05	\\
-4749.05007102273	2.20966516699317e-05	\\
-4748.0712890625	2.18032683049709e-05	\\
-4747.09250710227	2.10176132780935e-05	\\
-4746.11372514205	2.20798832683951e-05	\\
-4745.13494318182	2.36295841343066e-05	\\
-4744.15616122159	2.35685701055096e-05	\\
-4743.17737926136	2.30614192631526e-05	\\
-4742.19859730114	2.23790990613333e-05	\\
-4741.21981534091	2.2753006879401e-05	\\
-4740.24103338068	2.43966625128907e-05	\\
-4739.26225142046	2.32879747075583e-05	\\
-4738.28346946023	2.29906228857652e-05	\\
-4737.3046875	2.21920589237759e-05	\\
-4736.32590553977	2.32544434812342e-05	\\
-4735.34712357955	2.13875252717645e-05	\\
-4734.36834161932	2.31698377683119e-05	\\
-4733.38955965909	2.3416194341535e-05	\\
-4732.41077769886	2.37335894464128e-05	\\
-4731.43199573864	2.41204949446178e-05	\\
-4730.45321377841	2.26165483416366e-05	\\
-4729.47443181818	2.33680511525247e-05	\\
-4728.49564985796	2.09074823471222e-05	\\
-4727.51686789773	2.28613649931711e-05	\\
-4726.5380859375	2.22831703527214e-05	\\
-4725.55930397727	2.35915532328164e-05	\\
-4724.58052201705	2.43578910709571e-05	\\
-4723.60174005682	2.12359554128103e-05	\\
-4722.62295809659	2.2320396131542e-05	\\
-4721.64417613636	2.47961910940312e-05	\\
-4720.66539417614	2.55217967006813e-05	\\
-4719.68661221591	2.40641135237026e-05	\\
-4718.70783025568	2.3856258921605e-05	\\
-4717.72904829546	2.19408020311296e-05	\\
-4716.75026633523	2.17498983599566e-05	\\
-4715.771484375	2.51555206924496e-05	\\
-4714.79270241477	2.26202483366372e-05	\\
-4713.81392045455	2.1506517189889e-05	\\
-4712.83513849432	2.26480233883852e-05	\\
-4711.85635653409	2.14052195976432e-05	\\
-4710.87757457386	2.06211502645387e-05	\\
-4709.89879261364	2.14431212194826e-05	\\
-4708.92001065341	2.25403717099026e-05	\\
-4707.94122869318	2.34493233756991e-05	\\
-4706.96244673296	2.20567361827978e-05	\\
-4705.98366477273	2.22489467398192e-05	\\
-4705.0048828125	2.11806360510715e-05	\\
-4704.02610085227	2.19972791599914e-05	\\
-4703.04731889205	2.28235894643043e-05	\\
-4702.06853693182	2.27219932093501e-05	\\
-4701.08975497159	2.22140470399798e-05	\\
-4700.11097301136	2.22796550165981e-05	\\
-4699.13219105114	2.29278489311647e-05	\\
-4698.15340909091	2.27775116184228e-05	\\
-4697.17462713068	2.14032613473103e-05	\\
-4696.19584517046	2.11483925571861e-05	\\
-4695.21706321023	2.54930787524755e-05	\\
-4694.23828125	2.18978421554699e-05	\\
-4693.25949928977	2.28514486968552e-05	\\
-4692.28071732955	2.45536177457322e-05	\\
-4691.30193536932	2.22939708650307e-05	\\
-4690.32315340909	2.19463745091205e-05	\\
-4689.34437144886	2.45194971571154e-05	\\
-4688.36558948864	2.38579643428839e-05	\\
-4687.38680752841	2.41354285770709e-05	\\
-4686.40802556818	2.27931384984485e-05	\\
-4685.42924360796	2.3111138270439e-05	\\
-4684.45046164773	2.29753947876983e-05	\\
-4683.4716796875	2.49668091739029e-05	\\
-4682.49289772727	2.27865320613628e-05	\\
-4681.51411576705	2.44057082041566e-05	\\
-4680.53533380682	2.37369537604607e-05	\\
-4679.55655184659	2.43881275095511e-05	\\
-4678.57776988636	2.35569720633168e-05	\\
-4677.59898792614	2.40343109505817e-05	\\
-4676.62020596591	2.51982195602165e-05	\\
-4675.64142400568	2.1896903199618e-05	\\
-4674.66264204546	2.27787987371462e-05	\\
-4673.68386008523	2.25205896097181e-05	\\
-4672.705078125	2.26316421168743e-05	\\
-4671.72629616477	2.37899946720449e-05	\\
-4670.74751420455	2.3886581668041e-05	\\
-4669.76873224432	2.52736450022048e-05	\\
-4668.78995028409	2.38240625895316e-05	\\
-4667.81116832386	2.36770960457788e-05	\\
-4666.83238636364	2.27803910807378e-05	\\
-4665.85360440341	2.25632549155568e-05	\\
-4664.87482244318	2.47832205416121e-05	\\
-4663.89604048296	2.40420324739042e-05	\\
-4662.91725852273	2.50794206167244e-05	\\
-4661.9384765625	2.35767835930172e-05	\\
-4660.95969460227	2.43502236788825e-05	\\
-4659.98091264205	2.36737235455205e-05	\\
-4659.00213068182	2.28307368941201e-05	\\
-4658.02334872159	2.48043149400997e-05	\\
-4657.04456676136	2.22915541145934e-05	\\
-4656.06578480114	2.45590677228465e-05	\\
-4655.08700284091	2.43651714108412e-05	\\
-4654.10822088068	2.56229281120267e-05	\\
-4653.12943892046	2.31036089019556e-05	\\
-4652.15065696023	2.5304682023329e-05	\\
-4651.171875	2.44408756125047e-05	\\
-4650.19309303977	2.53317240644854e-05	\\
-4649.21431107955	2.45962864402631e-05	\\
-4648.23552911932	2.39786218807514e-05	\\
-4647.25674715909	2.22337904031382e-05	\\
-4646.27796519886	2.37531125986942e-05	\\
-4645.29918323864	2.39728807060142e-05	\\
-4644.32040127841	2.44938354854258e-05	\\
-4643.34161931818	2.3662661449575e-05	\\
-4642.36283735796	2.30565177327314e-05	\\
-4641.38405539773	2.51271166871197e-05	\\
-4640.4052734375	2.59806038422348e-05	\\
-4639.42649147727	2.25611795152719e-05	\\
-4638.44770951705	2.31707148160428e-05	\\
-4637.46892755682	2.41558897612031e-05	\\
-4636.49014559659	2.68209128170558e-05	\\
-4635.51136363636	2.54234225744709e-05	\\
-4634.53258167614	2.67454799933587e-05	\\
-4633.55379971591	2.37619202353967e-05	\\
-4632.57501775568	2.55098458253845e-05	\\
-4631.59623579546	2.52021857128584e-05	\\
-4630.61745383523	2.38398732874729e-05	\\
-4629.638671875	2.84340458906968e-05	\\
-4628.65988991477	2.4447058410653e-05	\\
-4627.68110795455	2.57219312841589e-05	\\
-4626.70232599432	2.50691735780676e-05	\\
-4625.72354403409	2.57702388229485e-05	\\
-4624.74476207386	2.40631499093656e-05	\\
-4623.76598011364	2.64965611352141e-05	\\
-4622.78719815341	2.37211259471855e-05	\\
-4621.80841619318	2.52969692338833e-05	\\
-4620.82963423296	2.76585044484731e-05	\\
-4619.85085227273	2.20709611859989e-05	\\
-4618.8720703125	2.72066309089637e-05	\\
-4617.89328835227	2.58563646286546e-05	\\
-4616.91450639205	2.67598820356943e-05	\\
-4615.93572443182	2.29529012543564e-05	\\
-4614.95694247159	2.46804219459984e-05	\\
-4613.97816051136	2.44550320893877e-05	\\
-4612.99937855114	2.4676187743644e-05	\\
-4612.02059659091	2.59881598711274e-05	\\
-4611.04181463068	2.63694004566117e-05	\\
-4610.06303267046	2.44628552511855e-05	\\
-4609.08425071023	2.51156822767434e-05	\\
-4608.10546875	2.64290633377096e-05	\\
-4607.12668678977	2.36471556486488e-05	\\
-4606.14790482955	2.65192006650996e-05	\\
-4605.16912286932	2.48903436352957e-05	\\
-4604.19034090909	2.60173568363865e-05	\\
-4603.21155894886	2.43718667000606e-05	\\
-4602.23277698864	2.24045386872905e-05	\\
-4601.25399502841	2.57438046629373e-05	\\
-4600.27521306818	2.4995722786168e-05	\\
-4599.29643110796	2.72024224775505e-05	\\
-4598.31764914773	2.41871078980882e-05	\\
-4597.3388671875	2.59631418915926e-05	\\
-4596.36008522727	2.58050572589769e-05	\\
-4595.38130326705	2.44933559983431e-05	\\
-4594.40252130682	2.55007983808658e-05	\\
-4593.42373934659	2.54012328660213e-05	\\
-4592.44495738636	2.50871243253782e-05	\\
-4591.46617542614	2.57277591255334e-05	\\
-4590.48739346591	2.65582976571853e-05	\\
-4589.50861150568	2.77123459058749e-05	\\
-4588.52982954546	2.27952736352494e-05	\\
-4587.55104758523	2.58451157187862e-05	\\
-4586.572265625	2.33778521809666e-05	\\
-4585.59348366477	2.70879359963872e-05	\\
-4584.61470170455	2.41878697602132e-05	\\
-4583.63591974432	2.50980312197006e-05	\\
-4582.65713778409	2.57436645162635e-05	\\
-4581.67835582386	2.58056161178786e-05	\\
-4580.69957386364	2.62235260641615e-05	\\
-4579.72079190341	2.61564110341234e-05	\\
-4578.74200994318	2.61109328832031e-05	\\
-4577.76322798296	2.5969550754694e-05	\\
-4576.78444602273	2.68601268935168e-05	\\
-4575.8056640625	2.36637306860579e-05	\\
-4574.82688210227	2.54845066092273e-05	\\
-4573.84810014205	2.51040201586521e-05	\\
-4572.86931818182	2.4192423774666e-05	\\
-4571.89053622159	2.69706514730061e-05	\\
-4570.91175426136	2.60714930975269e-05	\\
-4569.93297230114	2.67926137435145e-05	\\
-4568.95419034091	2.61810028177076e-05	\\
-4567.97540838068	2.87372985077184e-05	\\
-4566.99662642046	2.70024167994467e-05	\\
-4566.01784446023	2.78250997461323e-05	\\
-4565.0390625	2.81908809583993e-05	\\
-4564.06028053977	2.55982180084293e-05	\\
-4563.08149857955	2.64332686954062e-05	\\
-4562.10271661932	2.72191638787437e-05	\\
-4561.12393465909	2.78226048455989e-05	\\
-4560.14515269886	2.6323307558689e-05	\\
-4559.16637073864	2.5949860231447e-05	\\
-4558.18758877841	2.65970824764954e-05	\\
-4557.20880681818	2.88439645497163e-05	\\
-4556.23002485796	2.53387841408716e-05	\\
-4555.25124289773	2.95417209487889e-05	\\
-4554.2724609375	2.55364126795981e-05	\\
-4553.29367897727	2.58354791952665e-05	\\
-4552.31489701705	2.58273895765833e-05	\\
-4551.33611505682	2.89984668371343e-05	\\
-4550.35733309659	2.76846309279693e-05	\\
-4549.37855113636	2.77017225231844e-05	\\
-4548.39976917614	2.95442355705399e-05	\\
-4547.42098721591	2.84961557835167e-05	\\
-4546.44220525568	2.63234998864682e-05	\\
-4545.46342329546	2.82645443137656e-05	\\
-4544.48464133523	2.73703224913139e-05	\\
-4543.505859375	2.86705406584118e-05	\\
-4542.52707741477	2.61634941728213e-05	\\
-4541.54829545455	2.82223717230512e-05	\\
-4540.56951349432	2.89788029052335e-05	\\
-4539.59073153409	2.85430970352585e-05	\\
-4538.61194957386	2.86127605326872e-05	\\
-4537.63316761364	2.80008783016049e-05	\\
-4536.65438565341	2.91961265946138e-05	\\
-4535.67560369318	2.89283886130174e-05	\\
-4534.69682173296	2.90092300478233e-05	\\
-4533.71803977273	2.97427329486731e-05	\\
-4532.7392578125	2.78333528787037e-05	\\
-4531.76047585227	2.75068754621257e-05	\\
-4530.78169389205	2.51992179863756e-05	\\
-4529.80291193182	2.64089130507338e-05	\\
-4528.82412997159	2.78039073241268e-05	\\
-4527.84534801136	3.02882194002734e-05	\\
-4526.86656605114	2.88139762277135e-05	\\
-4525.88778409091	2.82436476104098e-05	\\
-4524.90900213068	2.79545409334384e-05	\\
-4523.93022017046	3.08821860861023e-05	\\
-4522.95143821023	3.21891521504185e-05	\\
-4521.97265625	2.82218672314643e-05	\\
-4520.99387428977	2.74858008545113e-05	\\
-4520.01509232955	3.08918842158216e-05	\\
-4519.03631036932	3.00298032053489e-05	\\
-4518.05752840909	3.03071828042427e-05	\\
-4517.07874644886	3.00535492431541e-05	\\
-4516.09996448864	2.83981246277137e-05	\\
-4515.12118252841	3.06091026992776e-05	\\
-4514.14240056818	2.89727785245329e-05	\\
-4513.16361860796	2.80823020989079e-05	\\
-4512.18483664773	2.9619562771163e-05	\\
-4511.2060546875	3.05927442339334e-05	\\
-4510.22727272727	3.0380567432697e-05	\\
-4509.24849076705	2.73246495160566e-05	\\
-4508.26970880682	2.87925371915558e-05	\\
-4507.29092684659	3.01650089687926e-05	\\
-4506.31214488636	3.10752082401659e-05	\\
-4505.33336292614	3.05653702573234e-05	\\
-4504.35458096591	3.07233698188008e-05	\\
-4503.37579900568	3.20994904688071e-05	\\
-4502.39701704546	2.9387723529721e-05	\\
-4501.41823508523	2.95958863038523e-05	\\
-4500.439453125	2.71307179739496e-05	\\
-4499.46067116477	2.40305952218599e-05	\\
-4498.48188920455	2.85871043686736e-05	\\
-4497.50310724432	3.05620424367457e-05	\\
-4496.52432528409	2.99472315410325e-05	\\
-4495.54554332386	3.0923878093453e-05	\\
-4494.56676136364	2.91369827655263e-05	\\
-4493.58797940341	2.99000449230308e-05	\\
-4492.60919744318	2.95803637161445e-05	\\
-4491.63041548296	3.04057753901568e-05	\\
-4490.65163352273	3.17549740002031e-05	\\
-4489.6728515625	2.98471412852203e-05	\\
-4488.69406960227	2.8713809569376e-05	\\
-4487.71528764205	3.18104268737079e-05	\\
-4486.73650568182	3.25085066777566e-05	\\
-4485.75772372159	3.07787351301286e-05	\\
-4484.77894176136	3.18085536907407e-05	\\
-4483.80015980114	2.99617711793581e-05	\\
-4482.82137784091	3.21745113858326e-05	\\
-4481.84259588068	2.82966743421704e-05	\\
-4480.86381392046	3.06616263640707e-05	\\
-4479.88503196023	2.73308625642006e-05	\\
-4478.90625	3.0303682846756e-05	\\
-4477.92746803977	2.99615634348374e-05	\\
-4476.94868607955	3.19804204524326e-05	\\
-4475.96990411932	3.04919255083236e-05	\\
-4474.99112215909	3.06439944656548e-05	\\
-4474.01234019886	3.02553940195165e-05	\\
-4473.03355823864	3.16614812139366e-05	\\
-4472.05477627841	3.19073231160632e-05	\\
-4471.07599431818	2.98973458256812e-05	\\
-4470.09721235796	3.08243899429313e-05	\\
-4469.11843039773	3.0805201359234e-05	\\
-4468.1396484375	2.81960198601701e-05	\\
-4467.16086647727	3.24378898614638e-05	\\
-4466.18208451705	2.87458967972225e-05	\\
-4465.20330255682	3.11679311709099e-05	\\
-4464.22452059659	3.33366369336921e-05	\\
-4463.24573863636	3.12086485031754e-05	\\
-4462.26695667614	3.088859144162e-05	\\
-4461.28817471591	2.91829550833507e-05	\\
-4460.30939275568	2.98673259594534e-05	\\
-4459.33061079546	3.12028754994883e-05	\\
-4458.35182883523	3.07762872114968e-05	\\
-4457.373046875	3.24321061439939e-05	\\
-4456.39426491477	3.107191840732e-05	\\
-4455.41548295455	3.13757681557317e-05	\\
-4454.43670099432	3.09006517886862e-05	\\
-4453.45791903409	3.16423314606101e-05	\\
-4452.47913707386	3.02642002902806e-05	\\
-4451.50035511364	3.27003191969045e-05	\\
-4450.52157315341	3.25976997878961e-05	\\
-4449.54279119318	3.10129727708621e-05	\\
-4448.56400923296	3.11271399097038e-05	\\
-4447.58522727273	3.28821586381087e-05	\\
-4446.6064453125	2.95245920775919e-05	\\
-4445.62766335227	3.17763362540555e-05	\\
-4444.64888139205	3.22438901863614e-05	\\
-4443.67009943182	2.98736898969717e-05	\\
-4442.69131747159	3.15659309732391e-05	\\
-4441.71253551136	2.88993223595701e-05	\\
-4440.73375355114	3.13489181462075e-05	\\
-4439.75497159091	3.43143680660188e-05	\\
-4438.77618963068	3.34531208311042e-05	\\
-4437.79740767046	3.11750266038337e-05	\\
-4436.81862571023	3.07157356379005e-05	\\
-4435.83984375	3.15193583344512e-05	\\
-4434.86106178977	3.18761828571679e-05	\\
-4433.88227982955	3.16950369904701e-05	\\
-4432.90349786932	3.10498350247196e-05	\\
-4431.92471590909	2.91303404342308e-05	\\
-4430.94593394886	3.19418801004453e-05	\\
-4429.96715198864	3.34783069681707e-05	\\
-4428.98837002841	2.96528271984983e-05	\\
-4428.00958806818	3.05334349403089e-05	\\
-4427.03080610796	3.20088396189123e-05	\\
-4426.05202414773	3.18253775789422e-05	\\
-4425.0732421875	3.13784244418026e-05	\\
-4424.09446022727	3.10444938847399e-05	\\
-4423.11567826705	2.97161508890962e-05	\\
-4422.13689630682	3.24274868930923e-05	\\
-4421.15811434659	3.37601442473793e-05	\\
-4420.17933238636	3.23105039292325e-05	\\
-4419.20055042614	3.21794836570613e-05	\\
-4418.22176846591	3.49789680169152e-05	\\
-4417.24298650568	3.15631115994833e-05	\\
-4416.26420454546	3.25972480530001e-05	\\
-4415.28542258523	3.17661592056329e-05	\\
-4414.306640625	3.16094815433976e-05	\\
-4413.32785866477	3.09348635932056e-05	\\
-4412.34907670455	3.18218534882456e-05	\\
-4411.37029474432	3.16817613567857e-05	\\
-4410.39151278409	3.33599534003039e-05	\\
-4409.41273082386	3.22696980931661e-05	\\
-4408.43394886364	3.32828146376574e-05	\\
-4407.45516690341	3.36365601772657e-05	\\
-4406.47638494318	3.25563118650913e-05	\\
-4405.49760298296	3.29402620836374e-05	\\
-4404.51882102273	3.25133760591098e-05	\\
-4403.5400390625	3.19532108996657e-05	\\
-4402.56125710227	3.45483814121225e-05	\\
-4401.58247514205	3.06349208469233e-05	\\
-4400.60369318182	3.44393011281105e-05	\\
-4399.62491122159	3.77077914794347e-05	\\
-4398.64612926136	3.40375922654546e-05	\\
-4397.66734730114	3.07271589821421e-05	\\
-4396.68856534091	3.28817508970625e-05	\\
-4395.70978338068	3.26457824946744e-05	\\
-4394.73100142046	3.30644942303286e-05	\\
-4393.75221946023	3.32142719871943e-05	\\
-4392.7734375	3.35861871612885e-05	\\
-4391.79465553977	3.15899059553573e-05	\\
-4390.81587357955	3.14814627856665e-05	\\
-4389.83709161932	3.27940903996108e-05	\\
-4388.85830965909	3.5119119933016e-05	\\
-4387.87952769886	3.3263078068046e-05	\\
-4386.90074573864	3.22647561048205e-05	\\
-4385.92196377841	3.45567903605276e-05	\\
-4384.94318181818	3.51537005101777e-05	\\
-4383.96439985796	3.41867236226714e-05	\\
-4382.98561789773	3.3715397258635e-05	\\
-4382.0068359375	3.47443051974035e-05	\\
-4381.02805397727	3.23364932207339e-05	\\
-4380.04927201705	3.23680595220179e-05	\\
-4379.07049005682	3.3085191193992e-05	\\
-4378.09170809659	3.29485619831788e-05	\\
-4377.11292613636	3.33327450707784e-05	\\
-4376.13414417614	3.46422030738745e-05	\\
-4375.15536221591	3.60472488414904e-05	\\
-4374.17658025568	3.39610784860632e-05	\\
-4373.19779829546	3.039166877913e-05	\\
-4372.21901633523	3.40165261048553e-05	\\
-4371.240234375	3.36483948766154e-05	\\
-4370.26145241477	3.29980288952585e-05	\\
-4369.28267045455	3.13816164136304e-05	\\
-4368.30388849432	3.30879978657537e-05	\\
-4367.32510653409	3.53659788542259e-05	\\
-4366.34632457386	3.26795439080212e-05	\\
-4365.36754261364	3.41804527802665e-05	\\
-4364.38876065341	3.55296001944269e-05	\\
-4363.40997869318	3.43652230355167e-05	\\
-4362.43119673296	3.3144636773713e-05	\\
-4361.45241477273	3.18000224222258e-05	\\
-4360.4736328125	3.37869731088637e-05	\\
-4359.49485085227	3.1343399757224e-05	\\
-4358.51606889205	3.29746254763586e-05	\\
-4357.53728693182	3.30770680008922e-05	\\
-4356.55850497159	3.31677436397521e-05	\\
-4355.57972301136	3.24024112955316e-05	\\
-4354.60094105114	3.17121582719665e-05	\\
-4353.62215909091	3.4310604377607e-05	\\
-4352.64337713068	3.4736141355635e-05	\\
-4351.66459517046	3.29993736886286e-05	\\
-4350.68581321023	3.41565058917763e-05	\\
-4349.70703125	3.3646441448743e-05	\\
-4348.72824928977	3.22114164959297e-05	\\
-4347.74946732955	3.28595476332183e-05	\\
-4346.77068536932	3.24520251594755e-05	\\
-4345.79190340909	3.60122246777346e-05	\\
-4344.81312144886	3.21361992266896e-05	\\
-4343.83433948864	3.38541887997715e-05	\\
-4342.85555752841	3.3693110432813e-05	\\
-4341.87677556818	3.2398389553779e-05	\\
-4340.89799360796	3.37252372505971e-05	\\
-4339.91921164773	3.44835409704568e-05	\\
-4338.9404296875	3.55744500108954e-05	\\
-4337.96164772727	3.42041076458536e-05	\\
-4336.98286576705	3.24919518535941e-05	\\
-4336.00408380682	3.26333972429853e-05	\\
-4335.02530184659	3.2831036761376e-05	\\
-4334.04651988636	3.3163156771119e-05	\\
-4333.06773792614	3.47448392949664e-05	\\
-4332.08895596591	3.45671579776165e-05	\\
-4331.11017400568	3.49060616356307e-05	\\
-4330.13139204546	3.51683817623937e-05	\\
-4329.15261008523	3.13955258042597e-05	\\
-4328.173828125	3.27746541568323e-05	\\
-4327.19504616477	3.35339922924162e-05	\\
-4326.21626420455	3.34186128272383e-05	\\
-4325.23748224432	3.29282279252888e-05	\\
-4324.25870028409	3.38567226049046e-05	\\
-4323.27991832386	3.31109918843393e-05	\\
-4322.30113636364	3.09802896823191e-05	\\
-4321.32235440341	3.30727843404395e-05	\\
-4320.34357244318	3.55967777931456e-05	\\
-4319.36479048296	3.15130716812106e-05	\\
-4318.38600852273	3.31176758754215e-05	\\
-4317.4072265625	3.45191302806288e-05	\\
-4316.42844460227	3.37010878640504e-05	\\
-4315.44966264205	3.39582277082636e-05	\\
-4314.47088068182	3.29007348687667e-05	\\
-4313.49209872159	3.37302773549183e-05	\\
-4312.51331676136	3.09598572988931e-05	\\
-4311.53453480114	3.3610172727763e-05	\\
-4310.55575284091	3.2914796713293e-05	\\
-4309.57697088068	3.27691587888153e-05	\\
-4308.59818892046	3.31655068924751e-05	\\
-4307.61940696023	3.31255913016112e-05	\\
-4306.640625	3.40104815847196e-05	\\
-4305.66184303977	3.02715250247169e-05	\\
-4304.68306107955	3.33865021991275e-05	\\
-4303.70427911932	2.95036617060367e-05	\\
-4302.72549715909	3.38218826317062e-05	\\
-4301.74671519886	3.49728960458587e-05	\\
-4300.76793323864	3.20348454908577e-05	\\
-4299.78915127841	3.19977371211573e-05	\\
-4298.81036931818	3.29693954873795e-05	\\
-4297.83158735796	3.1505549262636e-05	\\
-4296.85280539773	3.11253392543165e-05	\\
-4295.8740234375	3.25795573925723e-05	\\
-4294.89524147727	3.32553445958266e-05	\\
-4293.91645951705	3.15561942691237e-05	\\
-4292.93767755682	3.50561262187247e-05	\\
-4291.95889559659	3.19755363866417e-05	\\
-4290.98011363636	3.46568301457193e-05	\\
-4290.00133167614	3.26609078562654e-05	\\
-4289.02254971591	3.38902272035202e-05	\\
-4288.04376775568	3.33411343237926e-05	\\
-4287.06498579546	3.10674488853641e-05	\\
-4286.08620383523	3.30279938185171e-05	\\
-4285.107421875	3.43084778935217e-05	\\
-4284.12863991477	3.4071088780477e-05	\\
-4283.14985795455	3.3245074608916e-05	\\
-4282.17107599432	3.38644667619269e-05	\\
-4281.19229403409	3.26477496312398e-05	\\
-4280.21351207386	3.22155248971064e-05	\\
-4279.23473011364	3.2475442616394e-05	\\
-4278.25594815341	3.37049833162142e-05	\\
-4277.27716619318	3.43966807448773e-05	\\
-4276.29838423296	3.33249536906589e-05	\\
-4275.31960227273	3.20571136703242e-05	\\
-4274.3408203125	3.39876739468582e-05	\\
-4273.36203835227	3.39529043082362e-05	\\
-4272.38325639205	3.53681810960236e-05	\\
-4271.40447443182	3.42511917270158e-05	\\
-4270.42569247159	3.60247333318052e-05	\\
-4269.44691051136	3.35515463832023e-05	\\
-4268.46812855114	3.35404909587839e-05	\\
-4267.48934659091	3.34420736639207e-05	\\
-4266.51056463068	3.19993472021276e-05	\\
-4265.53178267046	3.34727284566445e-05	\\
-4264.55300071023	3.34997167783048e-05	\\
-4263.57421875	3.66369032763261e-05	\\
-4262.59543678977	3.38032771084011e-05	\\
-4261.61665482955	3.38336831663603e-05	\\
-4260.63787286932	3.28750731287026e-05	\\
-4259.65909090909	3.37606934183855e-05	\\
-4258.68030894886	3.25900911993443e-05	\\
-4257.70152698864	3.3591271775393e-05	\\
-4256.72274502841	3.39543856034529e-05	\\
-4255.74396306818	3.36285048966199e-05	\\
-4254.76518110796	3.50812604216961e-05	\\
-4253.78639914773	3.59109225079307e-05	\\
-4252.8076171875	3.26834829443497e-05	\\
-4251.82883522727	3.77325290342555e-05	\\
-4250.85005326705	3.55686129334944e-05	\\
-4249.87127130682	3.27390162611121e-05	\\
-4248.89248934659	3.51718725032374e-05	\\
-4247.91370738636	3.43215824906029e-05	\\
-4246.93492542614	3.52535626105578e-05	\\
-4245.95614346591	3.41023331378509e-05	\\
-4244.97736150568	3.74843126065444e-05	\\
-4243.99857954546	3.28092779058372e-05	\\
-4243.01979758523	3.43349420682958e-05	\\
-4242.041015625	3.71352495442231e-05	\\
-4241.06223366477	3.49998680013558e-05	\\
-4240.08345170455	3.69136053168414e-05	\\
-4239.10466974432	3.4108589121832e-05	\\
-4238.12588778409	3.27415847052579e-05	\\
-4237.14710582386	3.46241591555198e-05	\\
-4236.16832386364	3.40889690200271e-05	\\
-4235.18954190341	3.59294729876349e-05	\\
-4234.21075994318	3.59870653512437e-05	\\
-4233.23197798296	3.39911979547738e-05	\\
-4232.25319602273	3.42771939604588e-05	\\
-4231.2744140625	3.50594987144478e-05	\\
-4230.29563210227	3.34552395522904e-05	\\
-4229.31685014205	3.50118415754208e-05	\\
-4228.33806818182	3.54458789160536e-05	\\
-4227.35928622159	3.62630895020346e-05	\\
-4226.38050426136	3.62964066143296e-05	\\
-4225.40172230114	3.53963758028709e-05	\\
-4224.42294034091	3.51851185331409e-05	\\
-4223.44415838068	3.44872718270468e-05	\\
-4222.46537642046	3.41837108203593e-05	\\
-4221.48659446023	3.52562263382225e-05	\\
-4220.5078125	3.39324729559305e-05	\\
-4219.52903053977	3.60924084426355e-05	\\
-4218.55024857955	3.50433171377788e-05	\\
-4217.57146661932	3.543819948865e-05	\\
-4216.59268465909	3.26006851913613e-05	\\
-4215.61390269886	3.5657608851366e-05	\\
-4214.63512073864	3.44160525109592e-05	\\
-4213.65633877841	3.43581917950668e-05	\\
-4212.67755681818	3.55566679093175e-05	\\
-4211.69877485796	3.48884969256846e-05	\\
-4210.71999289773	3.46395836088176e-05	\\
-4209.7412109375	3.56667203345744e-05	\\
-4208.76242897727	3.50850321386463e-05	\\
-4207.78364701705	3.43500537866271e-05	\\
-4206.80486505682	3.48137554037518e-05	\\
-4205.82608309659	3.50037753117028e-05	\\
-4204.84730113636	3.5909327346635e-05	\\
-4203.86851917614	3.30131643563118e-05	\\
-4202.88973721591	3.49342950374709e-05	\\
-4201.91095525568	3.64933589572659e-05	\\
-4200.93217329546	3.33279768938304e-05	\\
-4199.95339133523	3.78481300435509e-05	\\
-4198.974609375	3.41125872879517e-05	\\
-4197.99582741477	3.25032243975922e-05	\\
-4197.01704545455	3.59092162941435e-05	\\
-4196.03826349432	3.4496088972745e-05	\\
-4195.05948153409	3.45068432832485e-05	\\
-4194.08069957386	3.63532061176319e-05	\\
-4193.10191761364	3.62145324415735e-05	\\
-4192.12313565341	3.58002338544737e-05	\\
-4191.14435369318	3.35823367909168e-05	\\
-4190.16557173296	3.72227933624499e-05	\\
-4189.18678977273	3.48413394545512e-05	\\
-4188.2080078125	3.38574036050232e-05	\\
-4187.22922585227	3.48375545176276e-05	\\
-4186.25044389205	3.41388512379439e-05	\\
-4185.27166193182	3.57162799908957e-05	\\
-4184.29287997159	3.49267387059841e-05	\\
-4183.31409801136	3.5246778204859e-05	\\
-4182.33531605114	3.41503831175877e-05	\\
-4181.35653409091	3.43577088529307e-05	\\
-4180.37775213068	3.62952631012799e-05	\\
-4179.39897017046	3.53780203582233e-05	\\
-4178.42018821023	3.71139160817103e-05	\\
-4177.44140625	3.50688366877986e-05	\\
-4176.46262428977	3.55165315034729e-05	\\
-4175.48384232955	3.57870219985711e-05	\\
-4174.50506036932	3.35310854406711e-05	\\
-4173.52627840909	3.4303352643419e-05	\\
-4172.54749644886	3.65747766513374e-05	\\
-4171.56871448864	3.73292384411186e-05	\\
-4170.58993252841	3.54657605420613e-05	\\
-4169.61115056818	3.39968988521338e-05	\\
-4168.63236860796	3.61743316819106e-05	\\
-4167.65358664773	3.4763244697539e-05	\\
-4166.6748046875	3.62014204368647e-05	\\
-4165.69602272727	3.62242931257356e-05	\\
-4164.71724076705	3.4487381823746e-05	\\
-4163.73845880682	3.59801047122007e-05	\\
-4162.75967684659	3.67474323566451e-05	\\
-4161.78089488636	3.57900076315507e-05	\\
-4160.80211292614	3.6179242196422e-05	\\
-4159.82333096591	3.54889860173582e-05	\\
-4158.84454900568	3.50795954176108e-05	\\
-4157.86576704546	3.41977108061168e-05	\\
-4156.88698508523	3.6942822102535e-05	\\
-4155.908203125	3.59738625200867e-05	\\
-4154.92942116477	3.59883918710668e-05	\\
-4153.95063920455	3.49337855044759e-05	\\
-4152.97185724432	3.40251127108951e-05	\\
-4151.99307528409	3.70459518907956e-05	\\
-4151.01429332386	3.48987898022493e-05	\\
-4150.03551136364	3.48261996895657e-05	\\
-4149.05672940341	3.68270244121592e-05	\\
-4148.07794744318	3.6913535919954e-05	\\
-4147.09916548296	3.46750276797281e-05	\\
-4146.12038352273	3.42793532093592e-05	\\
-4145.1416015625	3.35425233925085e-05	\\
-4144.16281960227	3.67702129195119e-05	\\
-4143.18403764205	3.48252512684627e-05	\\
-4142.20525568182	3.44845363184194e-05	\\
-4141.22647372159	3.55814363372518e-05	\\
-4140.24769176136	3.63540038127318e-05	\\
-4139.26890980114	3.54373061392991e-05	\\
-4138.29012784091	3.45703837525705e-05	\\
-4137.31134588068	3.35568693366068e-05	\\
-4136.33256392046	3.51562063046372e-05	\\
-4135.35378196023	3.7722068735364e-05	\\
-4134.375	3.72745916628171e-05	\\
-4133.39621803977	3.79667396525732e-05	\\
-4132.41743607955	3.68899102738879e-05	\\
-4131.43865411932	3.43922463733566e-05	\\
-4130.45987215909	3.40882198472998e-05	\\
-4129.48109019886	3.45452653014847e-05	\\
-4128.50230823864	3.5977004483185e-05	\\
-4127.52352627841	3.66437879676719e-05	\\
-4126.54474431818	3.39556898594376e-05	\\
-4125.56596235796	3.54509924999383e-05	\\
-4124.58718039773	3.77122387981471e-05	\\
-4123.6083984375	3.57467228036602e-05	\\
-4122.62961647727	3.65289809329828e-05	\\
-4121.65083451705	3.63412024862809e-05	\\
-4120.67205255682	3.53322015877229e-05	\\
-4119.69327059659	3.55772625912992e-05	\\
-4118.71448863636	3.63372770944149e-05	\\
-4117.73570667614	3.7212361580019e-05	\\
-4116.75692471591	3.82322239391853e-05	\\
-4115.77814275568	3.757358271393e-05	\\
-4114.79936079546	3.70073764449792e-05	\\
-4113.82057883523	3.56160323637305e-05	\\
-4112.841796875	3.52984634656083e-05	\\
-4111.86301491477	3.76468344073036e-05	\\
-4110.88423295455	3.67362478706957e-05	\\
-4109.90545099432	3.56385496948522e-05	\\
-4108.92666903409	3.6157024983327e-05	\\
-4107.94788707386	3.6459962136519e-05	\\
-4106.96910511364	3.73433452516661e-05	\\
-4105.99032315341	3.58611262720385e-05	\\
-4105.01154119318	3.62098359662692e-05	\\
-4104.03275923296	3.60697037215655e-05	\\
-4103.05397727273	3.48427097890542e-05	\\
-4102.0751953125	3.74941886511018e-05	\\
-4101.09641335227	3.6752266511966e-05	\\
-4100.11763139205	3.82329599543454e-05	\\
-4099.13884943182	3.64592142282641e-05	\\
-4098.16006747159	3.73524493331842e-05	\\
-4097.18128551136	3.65418592598548e-05	\\
-4096.20250355114	3.49857498225985e-05	\\
-4095.22372159091	3.50816568012223e-05	\\
-4094.24493963068	3.5372889562914e-05	\\
-4093.26615767046	3.42812403845336e-05	\\
-4092.28737571023	3.80554508465652e-05	\\
-4091.30859375	3.63006470647251e-05	\\
-4090.32981178977	3.67067884328859e-05	\\
-4089.35102982955	3.47290552605005e-05	\\
-4088.37224786932	3.80160458458103e-05	\\
-4087.39346590909	3.40122787647654e-05	\\
-4086.41468394886	3.6645730920396e-05	\\
-4085.43590198864	3.84745024223629e-05	\\
-4084.45712002841	3.58884588831868e-05	\\
-4083.47833806818	3.6082166942625e-05	\\
-4082.49955610796	3.66041067990781e-05	\\
-4081.52077414773	3.62071161693976e-05	\\
-4080.5419921875	3.618593680337e-05	\\
-4079.56321022727	3.59056101628065e-05	\\
-4078.58442826705	3.57553941698285e-05	\\
-4077.60564630682	3.38868489712816e-05	\\
-4076.62686434659	3.45992052961129e-05	\\
-4075.64808238636	3.47341414367514e-05	\\
-4074.66930042614	3.55058117831463e-05	\\
-4073.69051846591	3.50506604088687e-05	\\
-4072.71173650568	3.21537913156015e-05	\\
-4071.73295454546	3.76949777599975e-05	\\
-4070.75417258523	3.66178058993481e-05	\\
-4069.775390625	3.53785391396951e-05	\\
-4068.79660866477	3.44068070378356e-05	\\
-4067.81782670455	3.71672883969791e-05	\\
-4066.83904474432	3.60415840557341e-05	\\
-4065.86026278409	3.41842783815104e-05	\\
-4064.88148082386	3.38731333969737e-05	\\
-4063.90269886364	3.45917441796581e-05	\\
-4062.92391690341	3.68213394853825e-05	\\
-4061.94513494318	3.61989637326438e-05	\\
-4060.96635298296	3.44153800958853e-05	\\
-4059.98757102273	3.7091811046542e-05	\\
-4059.0087890625	3.60602558944427e-05	\\
-4058.03000710227	3.43088946897392e-05	\\
-4057.05122514205	3.54037638895146e-05	\\
-4056.07244318182	3.58863740306284e-05	\\
-4055.09366122159	3.51309241505528e-05	\\
-4054.11487926136	3.83721392482364e-05	\\
-4053.13609730114	3.59541964066141e-05	\\
-4052.15731534091	3.83431251207475e-05	\\
-4051.17853338068	3.50174121688824e-05	\\
-4050.19975142046	3.77713456635099e-05	\\
-4049.22096946023	4.02117536575056e-05	\\
-4048.2421875	3.53667199973092e-05	\\
-4047.26340553977	3.65715555639513e-05	\\
-4046.28462357955	3.74604829240342e-05	\\
-4045.30584161932	3.54969739446893e-05	\\
-4044.32705965909	3.66514420064691e-05	\\
-4043.34827769886	3.40893128254211e-05	\\
-4042.36949573864	3.76712590423786e-05	\\
-4041.39071377841	3.70041903454519e-05	\\
-4040.41193181818	3.74046476636693e-05	\\
-4039.43314985796	3.86526704815869e-05	\\
-4038.45436789773	3.65063335340068e-05	\\
-4037.4755859375	3.62435361438542e-05	\\
-4036.49680397727	3.8160035757688e-05	\\
-4035.51802201705	3.78976083717408e-05	\\
-4034.53924005682	3.92315862972102e-05	\\
-4033.56045809659	3.71508487060412e-05	\\
-4032.58167613636	3.80619729095213e-05	\\
-4031.60289417614	3.77277147772901e-05	\\
-4030.62411221591	3.52646675444819e-05	\\
-4029.64533025568	3.62603607123902e-05	\\
-4028.66654829546	3.82575658883096e-05	\\
-4027.68776633523	3.74469711431148e-05	\\
-4026.708984375	3.72129280103316e-05	\\
-4025.73020241477	3.81418847740943e-05	\\
-4024.75142045455	3.81887689229935e-05	\\
-4023.77263849432	3.73487572711687e-05	\\
-4022.79385653409	3.96241815529477e-05	\\
-4021.81507457386	3.72525850982735e-05	\\
-4020.83629261364	3.81915468691392e-05	\\
-4019.85751065341	4.02939412507671e-05	\\
-4018.87872869318	3.48588748604846e-05	\\
-4017.89994673296	3.98179715989584e-05	\\
-4016.92116477273	3.64259890450457e-05	\\
-4015.9423828125	3.67204753347031e-05	\\
-4014.96360085227	3.64770828663404e-05	\\
-4013.98481889205	3.59477773842605e-05	\\
-4013.00603693182	3.75050354793152e-05	\\
-4012.02725497159	3.94257390949646e-05	\\
-4011.04847301136	3.73885692493373e-05	\\
-4010.06969105114	3.94152247201988e-05	\\
-4009.09090909091	3.76334108070311e-05	\\
-4008.11212713068	3.70967910238885e-05	\\
-4007.13334517046	3.86212126389689e-05	\\
-4006.15456321023	3.82646128896548e-05	\\
-4005.17578125	3.68724579973586e-05	\\
-4004.19699928977	3.55507878043494e-05	\\
-4003.21821732955	3.64812160763042e-05	\\
-4002.23943536932	3.74122771871902e-05	\\
-4001.26065340909	3.7822830307267e-05	\\
-4000.28187144886	3.61988174186019e-05	\\
-3999.30308948864	3.61047071777833e-05	\\
-3998.32430752841	3.42479026834783e-05	\\
-3997.34552556818	3.3755015264789e-05	\\
-3996.36674360796	3.61309776020054e-05	\\
-3995.38796164773	3.26100809271271e-05	\\
-3994.4091796875	3.52674144387914e-05	\\
-3993.43039772727	3.64670677469413e-05	\\
-3992.45161576705	3.3733188053385e-05	\\
-3991.47283380682	3.56283424090712e-05	\\
-3990.49405184659	3.51340601253645e-05	\\
-3989.51526988636	3.43628777243854e-05	\\
-3988.53648792614	3.81093487794065e-05	\\
-3987.55770596591	3.65371145554785e-05	\\
-3986.57892400568	3.46211413873597e-05	\\
-3985.60014204546	3.4015268096308e-05	\\
-3984.62136008523	3.53209500601333e-05	\\
-3983.642578125	3.45046293576232e-05	\\
-3982.66379616477	3.45894380721352e-05	\\
-3981.68501420455	3.45808383246945e-05	\\
-3980.70623224432	3.66276951335477e-05	\\
-3979.72745028409	3.61101372834493e-05	\\
-3978.74866832386	3.62655860162791e-05	\\
-3977.76988636364	3.55797204488501e-05	\\
-3976.79110440341	3.70555577596657e-05	\\
-3975.81232244318	3.57414941881344e-05	\\
-3974.83354048296	3.54446137935795e-05	\\
-3973.85475852273	3.34934967623495e-05	\\
-3972.8759765625	3.58930182939237e-05	\\
-3971.89719460227	3.40996022254497e-05	\\
-3970.91841264205	3.47850158246021e-05	\\
-3969.93963068182	3.64158845118955e-05	\\
-3968.96084872159	3.4925648515238e-05	\\
-3967.98206676136	3.55356714692688e-05	\\
-3967.00328480114	3.60929402572878e-05	\\
-3966.02450284091	3.50500990424076e-05	\\
-3965.04572088068	3.44462163116122e-05	\\
-3964.06693892046	3.43676295268025e-05	\\
-3963.08815696023	3.44779233516333e-05	\\
-3962.109375	3.68642153756685e-05	\\
-3961.13059303977	3.23820815514722e-05	\\
-3960.15181107955	3.42756092908166e-05	\\
-3959.17302911932	3.63007697419998e-05	\\
-3958.19424715909	3.53564094046429e-05	\\
-3957.21546519886	3.25512695073677e-05	\\
-3956.23668323864	3.4431036558494e-05	\\
-3955.25790127841	3.39917592564911e-05	\\
-3954.27911931818	3.42915720352514e-05	\\
-3953.30033735796	3.56110052254414e-05	\\
-3952.32155539773	3.49420791970733e-05	\\
-3951.3427734375	3.31126138985359e-05	\\
-3950.36399147727	3.70237775621783e-05	\\
-3949.38520951705	3.51581575410083e-05	\\
-3948.40642755682	3.59094438527609e-05	\\
-3947.42764559659	3.55504239821652e-05	\\
-3946.44886363636	3.30464076391637e-05	\\
-3945.47008167614	3.50548234988493e-05	\\
-3944.49129971591	3.50141089583796e-05	\\
-3943.51251775568	3.38238062844589e-05	\\
-3942.53373579546	3.51306349414964e-05	\\
-3941.55495383523	3.41479596469523e-05	\\
-3940.576171875	3.45464147958486e-05	\\
-3939.59738991477	3.46993200134869e-05	\\
-3938.61860795455	3.1821860317257e-05	\\
-3937.63982599432	3.47163703474606e-05	\\
-3936.66104403409	3.56678482175735e-05	\\
-3935.68226207386	3.49032078157838e-05	\\
-3934.70348011364	3.53353484328903e-05	\\
-3933.72469815341	3.40966472585558e-05	\\
-3932.74591619318	3.70590698760782e-05	\\
-3931.76713423296	3.50640074004105e-05	\\
-3930.78835227273	3.21899972003675e-05	\\
-3929.8095703125	3.48460917230373e-05	\\
-3928.83078835227	3.49935556860161e-05	\\
-3927.85200639205	3.50821390012227e-05	\\
-3926.87322443182	3.589608609391e-05	\\
-3925.89444247159	3.33551535627561e-05	\\
-3924.91566051136	3.45995788144136e-05	\\
-3923.93687855114	3.45094792833328e-05	\\
-3922.95809659091	3.39896739971099e-05	\\
-3921.97931463068	3.63611011921996e-05	\\
-3921.00053267046	3.2498770887404e-05	\\
-3920.02175071023	3.38775422287033e-05	\\
-3919.04296875	3.33794634327198e-05	\\
-3918.06418678977	3.30022412596526e-05	\\
-3917.08540482955	3.49915699462495e-05	\\
-3916.10662286932	3.40734738276426e-05	\\
-3915.12784090909	3.36475413947058e-05	\\
-3914.14905894886	3.39018322260506e-05	\\
-3913.17027698864	3.29409971792924e-05	\\
-3912.19149502841	3.23624147213908e-05	\\
-3911.21271306818	3.28855533651332e-05	\\
-3910.23393110796	3.38288453729439e-05	\\
-3909.25514914773	3.45882164272461e-05	\\
-3908.2763671875	3.47957203012843e-05	\\
-3907.29758522727	3.28181497597068e-05	\\
-3906.31880326705	3.4970792888101e-05	\\
-3905.34002130682	3.27408479305961e-05	\\
-3904.36123934659	3.61191210152482e-05	\\
-3903.38245738636	3.34513273951119e-05	\\
-3902.40367542614	3.49883370908951e-05	\\
-3901.42489346591	3.45681788787931e-05	\\
-3900.44611150568	3.29176161435983e-05	\\
-3899.46732954546	3.3875261675765e-05	\\
-3898.48854758523	3.09559350448832e-05	\\
-3897.509765625	3.65664921683877e-05	\\
-3896.53098366477	3.24299594437356e-05	\\
-3895.55220170455	3.36325008888062e-05	\\
-3894.57341974432	3.5550733700977e-05	\\
-3893.59463778409	3.39171454489499e-05	\\
-3892.61585582386	3.41060291093809e-05	\\
-3891.63707386364	3.35796465544187e-05	\\
-3890.65829190341	3.45766022907388e-05	\\
-3889.67950994318	3.39279082332605e-05	\\
-3888.70072798296	3.64780049274707e-05	\\
-3887.72194602273	3.31605613939348e-05	\\
-3886.7431640625	3.37921829887331e-05	\\
-3885.76438210227	3.60228061889081e-05	\\
-3884.78560014205	3.37937166844476e-05	\\
-3883.80681818182	3.59181228467903e-05	\\
-3882.82803622159	3.45049589785192e-05	\\
-3881.84925426136	3.59306725165451e-05	\\
-3880.87047230114	3.45621404677614e-05	\\
-3879.89169034091	3.67396309475544e-05	\\
-3878.91290838068	3.49779571268744e-05	\\
-3877.93412642046	3.3785517920329e-05	\\
-3876.95534446023	3.47170944544659e-05	\\
-3875.9765625	3.63175167531845e-05	\\
-3874.99778053977	3.19778912258022e-05	\\
-3874.01899857955	3.57650067895132e-05	\\
-3873.04021661932	3.36554564953934e-05	\\
-3872.06143465909	3.41399674209634e-05	\\
-3871.08265269886	3.61236678291814e-05	\\
-3870.10387073864	3.42402705961978e-05	\\
-3869.12508877841	3.63366460396324e-05	\\
-3868.14630681818	3.49991690239692e-05	\\
-3867.16752485796	3.5063319311749e-05	\\
-3866.18874289773	3.55093177365271e-05	\\
-3865.2099609375	3.56240795593074e-05	\\
-3864.23117897727	3.61270834222681e-05	\\
-3863.25239701705	3.66933566853952e-05	\\
-3862.27361505682	3.40159769763397e-05	\\
-3861.29483309659	3.43056058744416e-05	\\
-3860.31605113636	3.41734800870737e-05	\\
-3859.33726917614	3.4897298707544e-05	\\
-3858.35848721591	3.34211645492332e-05	\\
-3857.37970525568	3.36355210355846e-05	\\
-3856.40092329546	3.5409877262567e-05	\\
-3855.42214133523	3.38520825004761e-05	\\
-3854.443359375	3.66487950281673e-05	\\
-3853.46457741477	3.69558200244854e-05	\\
-3852.48579545455	3.6195517475286e-05	\\
-3851.50701349432	3.47049896801156e-05	\\
-3850.52823153409	3.44870246482536e-05	\\
-3849.54944957386	3.40929329331364e-05	\\
-3848.57066761364	3.48761024647139e-05	\\
-3847.59188565341	3.36322624223605e-05	\\
-3846.61310369318	3.51938774690666e-05	\\
-3845.63432173296	3.24789722771272e-05	\\
-3844.65553977273	3.59426796984599e-05	\\
-3843.6767578125	3.39030391548479e-05	\\
-3842.69797585227	3.1413361356587e-05	\\
-3841.71919389205	3.47171750205115e-05	\\
-3840.74041193182	3.4710467472957e-05	\\
-3839.76162997159	3.58059661456161e-05	\\
-3838.78284801136	3.52690309573263e-05	\\
-3837.80406605114	3.48376234112261e-05	\\
-3836.82528409091	3.579990463777e-05	\\
-3835.84650213068	3.46319549338358e-05	\\
-3834.86772017046	3.50750373676628e-05	\\
-3833.88893821023	3.35975668124914e-05	\\
-3832.91015625	3.50402600158726e-05	\\
-3831.93137428977	3.52038372838127e-05	\\
-3830.95259232955	3.41746635748252e-05	\\
-3829.97381036932	3.46313870782849e-05	\\
-3828.99502840909	3.4523177602341e-05	\\
-3828.01624644886	3.42560644397617e-05	\\
-3827.03746448864	3.29727092803532e-05	\\
-3826.05868252841	3.42890340472684e-05	\\
-3825.07990056818	3.5182186592878e-05	\\
-3824.10111860796	3.33332388751142e-05	\\
-3823.12233664773	3.35530414355867e-05	\\
-3822.1435546875	3.59098841992396e-05	\\
-3821.16477272727	3.41442961699499e-05	\\
-3820.18599076705	3.46609729757336e-05	\\
-3819.20720880682	3.45107224300231e-05	\\
-3818.22842684659	3.38196297800478e-05	\\
-3817.24964488636	3.24286990639166e-05	\\
-3816.27086292614	3.42351183756759e-05	\\
-3815.29208096591	3.61430217978034e-05	\\
-3814.31329900568	3.46154854534424e-05	\\
-3813.33451704546	3.64194895505583e-05	\\
-3812.35573508523	3.33758711153951e-05	\\
-3811.376953125	3.78488601589647e-05	\\
-3810.39817116477	3.70979090217577e-05	\\
-3809.41938920455	3.35680873393449e-05	\\
-3808.44060724432	3.53802978681807e-05	\\
-3807.46182528409	3.55992671389905e-05	\\
-3806.48304332386	3.46060297639791e-05	\\
-3805.50426136364	3.41029071675993e-05	\\
-3804.52547940341	3.47605048205215e-05	\\
-3803.54669744318	3.47174296245274e-05	\\
-3802.56791548296	3.68119546026037e-05	\\
-3801.58913352273	3.35906156629553e-05	\\
-3800.6103515625	3.3810247932687e-05	\\
-3799.63156960227	3.29233737939458e-05	\\
-3798.65278764205	3.50540237631073e-05	\\
-3797.67400568182	3.50772093339205e-05	\\
-3796.69522372159	3.34566335265856e-05	\\
-3795.71644176136	3.52226944440856e-05	\\
-3794.73765980114	3.20786504041153e-05	\\
-3793.75887784091	3.35213467697501e-05	\\
-3792.78009588068	3.26709945779431e-05	\\
-3791.80131392046	3.43765917060562e-05	\\
-3790.82253196023	3.42783871509125e-05	\\
-3789.84375	3.53466427604851e-05	\\
-3788.86496803977	3.39062411212685e-05	\\
-3787.88618607955	3.31567141629754e-05	\\
-3786.90740411932	3.41171564980935e-05	\\
-3785.92862215909	3.30052979306134e-05	\\
-3784.94984019886	3.45514687827176e-05	\\
-3783.97105823864	3.36935814194797e-05	\\
-3782.99227627841	3.18987827315325e-05	\\
-3782.01349431818	3.58165864881071e-05	\\
-3781.03471235796	3.25987604973748e-05	\\
-3780.05593039773	3.52887505271564e-05	\\
-3779.0771484375	3.42907983234344e-05	\\
-3778.09836647727	3.21013552301206e-05	\\
-3777.11958451705	3.52425128612192e-05	\\
-3776.14080255682	3.37694716162487e-05	\\
-3775.16202059659	3.33498160899791e-05	\\
-3774.18323863636	3.41831646940696e-05	\\
-3773.20445667614	3.51085249119357e-05	\\
-3772.22567471591	3.23039120692714e-05	\\
-3771.24689275568	3.47472850681208e-05	\\
-3770.26811079546	3.22951032287199e-05	\\
-3769.28932883523	3.18842114658934e-05	\\
-3768.310546875	3.13038639547399e-05	\\
-3767.33176491477	3.30681299849948e-05	\\
-3766.35298295455	3.44250712245809e-05	\\
-3765.37420099432	3.64203358531089e-05	\\
-3764.39541903409	3.46603465347041e-05	\\
-3763.41663707386	3.38453904026001e-05	\\
-3762.43785511364	3.30184966122052e-05	\\
-3761.45907315341	3.51780459852282e-05	\\
-3760.48029119318	3.34279070671933e-05	\\
-3759.50150923296	3.37081442909216e-05	\\
-3758.52272727273	3.19021819468666e-05	\\
-3757.5439453125	3.53483006867727e-05	\\
-3756.56516335227	3.35078808296422e-05	\\
-3755.58638139205	3.33621799302531e-05	\\
-3754.60759943182	3.38628816054026e-05	\\
-3753.62881747159	3.56006596383645e-05	\\
-3752.65003551136	3.38685325553621e-05	\\
-3751.67125355114	3.36063041209425e-05	\\
-3750.69247159091	3.65863490454899e-05	\\
-3749.71368963068	3.48018618938163e-05	\\
-3748.73490767046	3.4779031430303e-05	\\
-3747.75612571023	3.35756035959122e-05	\\
-3746.77734375	3.57871743065431e-05	\\
-3745.79856178977	3.13623196988427e-05	\\
-3744.81977982955	3.28889648392428e-05	\\
-3743.84099786932	3.33487247785625e-05	\\
-3742.86221590909	3.49371834635965e-05	\\
-3741.88343394886	3.23277714665068e-05	\\
-3740.90465198864	3.41807769145943e-05	\\
-3739.92587002841	3.52331408699113e-05	\\
-3738.94708806818	3.67259283902181e-05	\\
-3737.96830610796	3.45420970799827e-05	\\
-3736.98952414773	3.09862363022146e-05	\\
-3736.0107421875	3.02140565021914e-05	\\
-3735.03196022727	3.38596818609816e-05	\\
-3734.05317826705	3.56766675457496e-05	\\
-3733.07439630682	3.54622553118109e-05	\\
-3732.09561434659	3.58184349001529e-05	\\
-3731.11683238636	3.30412965244633e-05	\\
-3730.13805042614	3.37752159990279e-05	\\
-3729.15926846591	3.41451183035796e-05	\\
-3728.18048650568	3.63241110435629e-05	\\
-3727.20170454546	3.39018124324312e-05	\\
-3726.22292258523	3.47165128376876e-05	\\
-3725.244140625	3.31573535204026e-05	\\
-3724.26535866477	3.61984097071618e-05	\\
-3723.28657670455	3.57291992771072e-05	\\
-3722.30779474432	3.5039950485825e-05	\\
-3721.32901278409	3.43478181050854e-05	\\
-3720.35023082386	3.33900964584739e-05	\\
-3719.37144886364	3.37014165712643e-05	\\
-3718.39266690341	3.5429270992899e-05	\\
-3717.41388494318	3.48805135610882e-05	\\
-3716.43510298296	3.48220681097986e-05	\\
-3715.45632102273	3.4149029353443e-05	\\
-3714.4775390625	3.17636299589119e-05	\\
-3713.49875710227	3.12029011325774e-05	\\
-3712.51997514205	3.40701850037271e-05	\\
-3711.54119318182	3.44560816007999e-05	\\
-3710.56241122159	3.3366673026959e-05	\\
-3709.58362926136	3.51441930524116e-05	\\
-3708.60484730114	3.31886035159483e-05	\\
-3707.62606534091	3.35085144955235e-05	\\
-3706.64728338068	3.51636359986649e-05	\\
-3705.66850142046	3.27797536456758e-05	\\
-3704.68971946023	3.2117692466093e-05	\\
-3703.7109375	3.48206386875579e-05	\\
-3702.73215553977	3.26156606431249e-05	\\
-3701.75337357955	3.37639331778968e-05	\\
-3700.77459161932	3.31523255270545e-05	\\
-3699.79580965909	3.35476949298982e-05	\\
-3698.81702769886	3.40525962297576e-05	\\
-3697.83824573864	3.4178002992879e-05	\\
-3696.85946377841	3.36067197046502e-05	\\
-3695.88068181818	3.27760149966634e-05	\\
-3694.90189985796	3.15692205242659e-05	\\
-3693.92311789773	3.49877657589392e-05	\\
-3692.9443359375	3.38114079779243e-05	\\
-3691.96555397727	3.25834953048737e-05	\\
-3690.98677201705	3.13429105047883e-05	\\
-3690.00799005682	3.38924229439764e-05	\\
-3689.02920809659	3.22503251175927e-05	\\
-3688.05042613636	3.46209089015922e-05	\\
-3687.07164417614	3.22199506189462e-05	\\
-3686.09286221591	3.2295227955755e-05	\\
-3685.11408025568	3.56781794750067e-05	\\
-3684.13529829546	3.35272261586882e-05	\\
-3683.15651633523	3.46406107447251e-05	\\
-3682.177734375	3.23696827709506e-05	\\
-3681.19895241477	3.39005464750859e-05	\\
-3680.22017045455	3.39368292998494e-05	\\
-3679.24138849432	3.20130848957507e-05	\\
-3678.26260653409	3.1342820803433e-05	\\
-3677.28382457386	3.44369394424223e-05	\\
-3676.30504261364	3.03903167422164e-05	\\
-3675.32626065341	3.17471768364824e-05	\\
-3674.34747869318	3.19243199273218e-05	\\
-3673.36869673296	3.25107907242625e-05	\\
-3672.38991477273	3.3998199197297e-05	\\
-3671.4111328125	3.27791334688967e-05	\\
-3670.43235085227	3.31967555523319e-05	\\
-3669.45356889205	3.19704804372848e-05	\\
-3668.47478693182	3.2267782608248e-05	\\
-3667.49600497159	3.39260652449786e-05	\\
-3666.51722301136	3.38747969067948e-05	\\
-3665.53844105114	3.38107539018395e-05	\\
-3664.55965909091	3.20975539390964e-05	\\
-3663.58087713068	3.22252609532275e-05	\\
-3662.60209517046	3.44664478847757e-05	\\
-3661.62331321023	3.35938157981017e-05	\\
-3660.64453125	3.29633057470358e-05	\\
-3659.66574928977	3.23324058592303e-05	\\
-3658.68696732955	3.34886617586564e-05	\\
-3657.70818536932	3.30093953716917e-05	\\
-3656.72940340909	3.3334795946028e-05	\\
-3655.75062144886	3.2971260633405e-05	\\
-3654.77183948864	3.13515898835369e-05	\\
-3653.79305752841	3.31423010189029e-05	\\
-3652.81427556818	3.2322007203779e-05	\\
-3651.83549360796	3.22770925860172e-05	\\
-3650.85671164773	3.04749733603536e-05	\\
-3649.8779296875	3.32589993362644e-05	\\
-3648.89914772727	3.08338951105587e-05	\\
-3647.92036576705	3.23036311925852e-05	\\
-3646.94158380682	3.26382832497266e-05	\\
-3645.96280184659	3.15281543948543e-05	\\
-3644.98401988636	3.34046948311941e-05	\\
-3644.00523792614	3.26774132002456e-05	\\
-3643.02645596591	3.15195385204521e-05	\\
-3642.04767400568	3.40946251711456e-05	\\
-3641.06889204546	3.19424424589621e-05	\\
-3640.09011008523	3.04379795143538e-05	\\
-3639.111328125	3.30645977297408e-05	\\
-3638.13254616477	3.24941110218292e-05	\\
-3637.15376420455	3.20290087086978e-05	\\
-3636.17498224432	3.15371223053435e-05	\\
-3635.19620028409	3.19377710101344e-05	\\
-3634.21741832386	3.02593227810306e-05	\\
-3633.23863636364	3.16274262429433e-05	\\
-3632.25985440341	2.92292917996694e-05	\\
-3631.28107244318	3.40004949335849e-05	\\
-3630.30229048296	3.09705232699547e-05	\\
-3629.32350852273	3.00508093334312e-05	\\
-3628.3447265625	3.34148267327348e-05	\\
-3627.36594460227	3.13700783674886e-05	\\
-3626.38716264205	2.97879155253996e-05	\\
-3625.40838068182	3.13201868780371e-05	\\
-3624.42959872159	2.83842467747765e-05	\\
-3623.45081676136	2.87984984811277e-05	\\
-3622.47203480114	3.28978418325733e-05	\\
-3621.49325284091	3.28334030632037e-05	\\
-3620.51447088068	3.36379662310895e-05	\\
-3619.53568892046	3.19600046472147e-05	\\
-3618.55690696023	3.29663555608415e-05	\\
-3617.578125	3.29264954796146e-05	\\
-3616.59934303977	3.01703560310205e-05	\\
-3615.62056107955	3.30833632411294e-05	\\
-3614.64177911932	3.10190935437458e-05	\\
-3613.66299715909	3.29335282474841e-05	\\
-3612.68421519886	3.17878797322483e-05	\\
-3611.70543323864	3.2667440297437e-05	\\
-3610.72665127841	3.18703645557273e-05	\\
-3609.74786931818	3.24591524506042e-05	\\
-3608.76908735796	3.19918316590457e-05	\\
-3607.79030539773	3.21593124469321e-05	\\
-3606.8115234375	3.39627682438148e-05	\\
-3605.83274147727	3.34684496429579e-05	\\
-3604.85395951705	3.01944370834397e-05	\\
-3603.87517755682	2.98140904384344e-05	\\
-3602.89639559659	3.17597192659738e-05	\\
-3601.91761363636	3.24979098264235e-05	\\
-3600.93883167614	2.95838749974778e-05	\\
-3599.96004971591	3.00123082307002e-05	\\
-3598.98126775568	2.87895924364745e-05	\\
-3598.00248579546	3.07882269862151e-05	\\
-3597.02370383523	3.19081182448765e-05	\\
-3596.044921875	3.04398743820313e-05	\\
-3595.06613991477	3.22906401913084e-05	\\
-3594.08735795455	3.41379613359892e-05	\\
-3593.10857599432	3.22003889158316e-05	\\
-3592.12979403409	3.09791214745536e-05	\\
-3591.15101207386	2.96927029734491e-05	\\
-3590.17223011364	2.94605842727865e-05	\\
-3589.19344815341	3.31053527804444e-05	\\
-3588.21466619318	3.07249713527641e-05	\\
-3587.23588423296	2.98618785778966e-05	\\
-3586.25710227273	3.1084365872997e-05	\\
-3585.2783203125	3.19314329780967e-05	\\
-3584.29953835227	2.79449896717459e-05	\\
-3583.32075639205	3.12023100410688e-05	\\
-3582.34197443182	3.14416048982283e-05	\\
-3581.36319247159	2.97984127170211e-05	\\
-3580.38441051136	3.10439870392684e-05	\\
-3579.40562855114	2.84829016000359e-05	\\
-3578.42684659091	3.24661241120109e-05	\\
-3577.44806463068	3.18593919151794e-05	\\
-3576.46928267046	3.02062708570289e-05	\\
-3575.49050071023	2.96221377696519e-05	\\
-3574.51171875	3.20125287974332e-05	\\
-3573.53293678977	2.98083959318896e-05	\\
-3572.55415482955	3.12434058548091e-05	\\
-3571.57537286932	3.05082396989042e-05	\\
-3570.59659090909	3.03595648223029e-05	\\
-3569.61780894886	3.04147526312299e-05	\\
-3568.63902698864	3.14520311361096e-05	\\
-3567.66024502841	3.02926356921493e-05	\\
-3566.68146306818	3.10732441686364e-05	\\
-3565.70268110796	3.12663145965369e-05	\\
-3564.72389914773	3.39250858379539e-05	\\
-3563.7451171875	3.12632969713372e-05	\\
-3562.76633522727	3.06237454245946e-05	\\
-3561.78755326705	3.0331367806634e-05	\\
-3560.80877130682	2.89823574601633e-05	\\
-3559.82998934659	3.07890242496202e-05	\\
-3558.85120738636	3.20212379927383e-05	\\
-3557.87242542614	2.94122040622427e-05	\\
-3556.89364346591	3.06460181549766e-05	\\
-3555.91486150568	3.02807983658074e-05	\\
-3554.93607954546	3.14248056457464e-05	\\
-3553.95729758523	3.12663779331294e-05	\\
-3552.978515625	3.16999949906511e-05	\\
-3551.99973366477	3.03822909697404e-05	\\
-3551.02095170455	3.11935341865922e-05	\\
-3550.04216974432	3.14651972745766e-05	\\
-3549.06338778409	3.12419197697021e-05	\\
-3548.08460582386	2.90242470529341e-05	\\
-3547.10582386364	3.44614967852862e-05	\\
-3546.12704190341	3.23876061579202e-05	\\
-3545.14825994318	3.31583232697772e-05	\\
-3544.16947798296	3.14207902872781e-05	\\
-3543.19069602273	2.92190936240504e-05	\\
-3542.2119140625	3.16571120288689e-05	\\
-3541.23313210227	3.04605071060934e-05	\\
-3540.25435014205	3.03966535728934e-05	\\
-3539.27556818182	3.03891824539604e-05	\\
-3538.29678622159	3.00430631920768e-05	\\
-3537.31800426136	2.88085794192258e-05	\\
-3536.33922230114	2.83377644978893e-05	\\
-3535.36044034091	3.18186607796173e-05	\\
-3534.38165838068	3.00844730983516e-05	\\
-3533.40287642046	2.992009046984e-05	\\
-3532.42409446023	3.28356182541766e-05	\\
-3531.4453125	3.09865048497648e-05	\\
-3530.46653053977	3.06453649403005e-05	\\
-3529.48774857955	3.14907997934837e-05	\\
-3528.50896661932	2.95394246643574e-05	\\
-3527.53018465909	3.19709983870649e-05	\\
-3526.55140269886	2.91818283251705e-05	\\
-3525.57262073864	3.00431075673617e-05	\\
-3524.59383877841	3.04937536486636e-05	\\
-3523.61505681818	3.07638207523804e-05	\\
-3522.63627485796	2.94756994334023e-05	\\
-3521.65749289773	2.9202319346388e-05	\\
-3520.6787109375	3.08282939106443e-05	\\
-3519.69992897727	3.06883613538112e-05	\\
-3518.72114701705	3.07414377989754e-05	\\
-3517.74236505682	2.97260085710111e-05	\\
-3516.76358309659	3.01083623339323e-05	\\
-3515.78480113636	3.22465074573014e-05	\\
-3514.80601917614	2.92355478880245e-05	\\
-3513.82723721591	2.92060596044579e-05	\\
-3512.84845525568	3.23775898814687e-05	\\
-3511.86967329546	2.92067441773715e-05	\\
-3510.89089133523	3.22602548970116e-05	\\
-3509.912109375	3.31058093255974e-05	\\
-3508.93332741477	2.90604387465258e-05	\\
-3507.95454545455	2.96821646806882e-05	\\
-3506.97576349432	3.14224653006568e-05	\\
-3505.99698153409	2.73704523495504e-05	\\
-3505.01819957386	3.08937703145004e-05	\\
-3504.03941761364	2.98105976030497e-05	\\
-3503.06063565341	2.76302764144062e-05	\\
-3502.08185369318	3.13928120082581e-05	\\
-3501.10307173296	2.81706713885898e-05	\\
-3500.12428977273	2.82613841643857e-05	\\
-3499.1455078125	3.25063538651123e-05	\\
-3498.16672585227	3.07748988194794e-05	\\
-3497.18794389205	2.91073039574252e-05	\\
-3496.20916193182	3.12886740179192e-05	\\
-3495.23037997159	2.93072644214791e-05	\\
-3494.25159801136	3.03298943733362e-05	\\
-3493.27281605114	3.34578926955502e-05	\\
-3492.29403409091	3.06325294157232e-05	\\
-3491.31525213068	3.08543540298525e-05	\\
-3490.33647017046	3.1678042402865e-05	\\
-3489.35768821023	2.95651686289953e-05	\\
-3488.37890625	2.9087697575943e-05	\\
-3487.40012428977	2.76713670810792e-05	\\
-3486.42134232955	2.83775088924405e-05	\\
-3485.44256036932	3.10581652237643e-05	\\
-3484.46377840909	3.03934925006381e-05	\\
-3483.48499644886	3.02496319835946e-05	\\
-3482.50621448864	3.37805294416533e-05	\\
-3481.52743252841	3.05057904372478e-05	\\
-3480.54865056818	2.7291056918047e-05	\\
-3479.56986860796	3.06435804829661e-05	\\
-3478.59108664773	2.9910291023369e-05	\\
-3477.6123046875	2.92533016101257e-05	\\
-3476.63352272727	3.28573360152577e-05	\\
-3475.65474076705	2.93155655444493e-05	\\
-3474.67595880682	2.85583975466979e-05	\\
-3473.69717684659	2.95384082173207e-05	\\
-3472.71839488636	3.03273724049403e-05	\\
-3471.73961292614	2.8451460339749e-05	\\
-3470.76083096591	2.94103259758411e-05	\\
-3469.78204900568	3.09279536846232e-05	\\
-3468.80326704546	2.72249455375721e-05	\\
-3467.82448508523	2.98849349075253e-05	\\
-3466.845703125	2.95334053648717e-05	\\
-3465.86692116477	3.01939736417525e-05	\\
-3464.88813920455	3.00533331815292e-05	\\
-3463.90935724432	3.09328583847883e-05	\\
-3462.93057528409	2.89806899892971e-05	\\
-3461.95179332386	2.93960834638029e-05	\\
-3460.97301136364	2.70871202265569e-05	\\
-3459.99422940341	3.05863990441047e-05	\\
-3459.01544744318	3.08021972234352e-05	\\
-3458.03666548296	3.16895965393413e-05	\\
-3457.05788352273	3.21067667810808e-05	\\
-3456.0791015625	2.92809332261032e-05	\\
-3455.10031960227	2.90104288546103e-05	\\
-3454.12153764205	2.86541931890928e-05	\\
-3453.14275568182	3.0666492192735e-05	\\
-3452.16397372159	2.91349338959771e-05	\\
-3451.18519176136	2.91315804017189e-05	\\
-3450.20640980114	2.96466661724788e-05	\\
-3449.22762784091	2.90034414741593e-05	\\
-3448.24884588068	3.01695276197779e-05	\\
-3447.27006392046	2.96690710690059e-05	\\
-3446.29128196023	3.0573414075731e-05	\\
-3445.3125	2.76909823776539e-05	\\
-3444.33371803977	3.05326694212446e-05	\\
-3443.35493607955	2.93460279402054e-05	\\
-3442.37615411932	2.76145541277159e-05	\\
-3441.39737215909	3.09386017685597e-05	\\
-3440.41859019886	2.82991781157109e-05	\\
-3439.43980823864	2.98418819375177e-05	\\
-3438.46102627841	2.79569160746253e-05	\\
-3437.48224431818	3.06547811815036e-05	\\
-3436.50346235796	2.82631950627109e-05	\\
-3435.52468039773	2.95592366969078e-05	\\
-3434.5458984375	3.01382611696e-05	\\
-3433.56711647727	2.71140469273827e-05	\\
-3432.58833451705	2.82794325242648e-05	\\
-3431.60955255682	2.70197167780712e-05	\\
-3430.63077059659	2.72963584069296e-05	\\
-3429.65198863636	2.99886677749099e-05	\\
-3428.67320667614	2.97094416900903e-05	\\
-3427.69442471591	2.79405541938724e-05	\\
-3426.71564275568	2.97672591800029e-05	\\
-3425.73686079546	2.90505805575198e-05	\\
-3424.75807883523	2.78103511369646e-05	\\
-3423.779296875	2.85091182021127e-05	\\
-3422.80051491477	2.81631317370148e-05	\\
-3421.82173295455	2.92309103352138e-05	\\
-3420.84295099432	2.74281311007597e-05	\\
-3419.86416903409	2.53106849632627e-05	\\
-3418.88538707386	2.74713581534988e-05	\\
-3417.90660511364	2.67074764874426e-05	\\
-3416.92782315341	2.83045603726673e-05	\\
-3415.94904119318	2.83596483900598e-05	\\
-3414.97025923296	2.56085620650217e-05	\\
-3413.99147727273	2.63458484941931e-05	\\
-3413.0126953125	2.86307676104218e-05	\\
-3412.03391335227	2.69338653294862e-05	\\
-3411.05513139205	2.94020695938216e-05	\\
-3410.07634943182	2.67130806249633e-05	\\
-3409.09756747159	2.82703525499011e-05	\\
-3408.11878551136	2.78107084066653e-05	\\
-3407.14000355114	2.78688780474313e-05	\\
-3406.16122159091	2.66500929150618e-05	\\
-3405.18243963068	2.8835187753696e-05	\\
-3404.20365767046	2.71028650906147e-05	\\
-3403.22487571023	2.61909457799396e-05	\\
-3402.24609375	2.90455206802608e-05	\\
-3401.26731178977	2.75179000529531e-05	\\
-3400.28852982955	2.84467579346675e-05	\\
-3399.30974786932	2.76105275727557e-05	\\
-3398.33096590909	2.58999674395906e-05	\\
-3397.35218394886	3.07461659741481e-05	\\
-3396.37340198864	2.79387425055287e-05	\\
-3395.39462002841	2.64686230588728e-05	\\
-3394.41583806818	2.92718491223774e-05	\\
-3393.43705610796	2.98181702569018e-05	\\
-3392.45827414773	2.78373754871366e-05	\\
-3391.4794921875	2.77350804954452e-05	\\
-3390.50071022727	2.74580256388975e-05	\\
-3389.52192826705	2.89314707443917e-05	\\
-3388.54314630682	2.76854566738929e-05	\\
-3387.56436434659	2.72732510915949e-05	\\
-3386.58558238636	2.9680293348556e-05	\\
-3385.60680042614	2.92379015692261e-05	\\
-3384.62801846591	2.88259025200584e-05	\\
-3383.64923650568	2.67157402234892e-05	\\
-3382.67045454546	2.97309978829973e-05	\\
-3381.69167258523	2.9912042788596e-05	\\
-3380.712890625	2.89303282627187e-05	\\
-3379.73410866477	2.98464163966345e-05	\\
-3378.75532670455	2.78180725416448e-05	\\
-3377.77654474432	2.74951900260209e-05	\\
-3376.79776278409	2.50727803579365e-05	\\
-3375.81898082386	2.84276312409201e-05	\\
-3374.84019886364	3.03185761540765e-05	\\
-3373.86141690341	2.91215960273473e-05	\\
-3372.88263494318	3.04781956986885e-05	\\
-3371.90385298296	3.03781618642647e-05	\\
-3370.92507102273	2.83544440973174e-05	\\
-3369.9462890625	2.93964730303613e-05	\\
-3368.96750710227	3.031068131872e-05	\\
-3367.98872514205	2.64425119139487e-05	\\
-3367.00994318182	3.00702578401813e-05	\\
-3366.03116122159	2.93035573216402e-05	\\
-3365.05237926136	2.9305819855798e-05	\\
-3364.07359730114	2.90884483126578e-05	\\
-3363.09481534091	3.0475155152169e-05	\\
-3362.11603338068	3.03862793675497e-05	\\
-3361.13725142046	2.86067422145057e-05	\\
-3360.15846946023	2.85271076076003e-05	\\
-3359.1796875	2.90406253064213e-05	\\
-3358.20090553977	2.85492540317915e-05	\\
-3357.22212357955	2.77421547183218e-05	\\
-3356.24334161932	2.84787367432297e-05	\\
-3355.26455965909	2.97464344088181e-05	\\
-3354.28577769886	3.2637135074205e-05	\\
-3353.30699573864	2.92562950725974e-05	\\
-3352.32821377841	2.84609129211972e-05	\\
-3351.34943181818	2.71347740496179e-05	\\
-3350.37064985796	2.89900055914948e-05	\\
-3349.39186789773	2.81603245210007e-05	\\
-3348.4130859375	2.97540502226791e-05	\\
-3347.43430397727	2.92407598209287e-05	\\
-3346.45552201705	2.72946552410378e-05	\\
-3345.47674005682	2.91834414076727e-05	\\
-3344.49795809659	3.01260993282634e-05	\\
-3343.51917613636	2.811014719006e-05	\\
-3342.54039417614	2.86589640415587e-05	\\
-3341.56161221591	2.78303652837759e-05	\\
-3340.58283025568	3.0076957597942e-05	\\
-3339.60404829546	2.7964794435749e-05	\\
-3338.62526633523	2.68851236921197e-05	\\
-3337.646484375	2.61556496547848e-05	\\
-3336.66770241477	2.88054067678642e-05	\\
-3335.68892045455	2.91421501434458e-05	\\
-3334.71013849432	2.97761349163899e-05	\\
-3333.73135653409	2.80437953156626e-05	\\
-3332.75257457386	2.67700503873022e-05	\\
-3331.77379261364	2.67584581424003e-05	\\
-3330.79501065341	2.93243485699211e-05	\\
-3329.81622869318	2.87194741675009e-05	\\
-3328.83744673296	2.60357721801597e-05	\\
-3327.85866477273	2.82012973789189e-05	\\
-3326.8798828125	2.74880904239222e-05	\\
-3325.90110085227	2.65108941055002e-05	\\
-3324.92231889205	2.83168973033446e-05	\\
-3323.94353693182	2.67613390413015e-05	\\
-3322.96475497159	2.66234454956375e-05	\\
-3321.98597301136	2.60212013540934e-05	\\
-3321.00719105114	2.66628054536684e-05	\\
-3320.02840909091	2.66580878110526e-05	\\
-3319.04962713068	3.13807056041631e-05	\\
-3318.07084517046	2.84334348465271e-05	\\
-3317.09206321023	2.83022086771202e-05	\\
-3316.11328125	2.86674036346137e-05	\\
-3315.13449928977	2.90254121466765e-05	\\
-3314.15571732955	2.44051397702824e-05	\\
-3313.17693536932	2.89955145242825e-05	\\
-3312.19815340909	2.49543556527529e-05	\\
-3311.21937144886	2.6965220427585e-05	\\
-3310.24058948864	2.78872912071328e-05	\\
-3309.26180752841	2.71538266201274e-05	\\
-3308.28302556818	2.66345589168079e-05	\\
-3307.30424360796	2.70770326242243e-05	\\
-3306.32546164773	2.62947689234708e-05	\\
-3305.3466796875	2.85611319471449e-05	\\
-3304.36789772727	2.56665746121191e-05	\\
-3303.38911576705	2.50637680583406e-05	\\
-3302.41033380682	2.4821386860985e-05	\\
-3301.43155184659	2.65384525932237e-05	\\
-3300.45276988636	2.70114159232004e-05	\\
-3299.47398792614	2.50695224004521e-05	\\
-3298.49520596591	2.46625415911885e-05	\\
-3297.51642400568	2.51415879132147e-05	\\
-3296.53764204546	2.60329433845952e-05	\\
-3295.55886008523	2.58239319948518e-05	\\
-3294.580078125	2.38454840117552e-05	\\
-3293.60129616477	2.84380033123638e-05	\\
-3292.62251420455	2.74949038977979e-05	\\
-3291.64373224432	2.70334719813144e-05	\\
-3290.66495028409	2.79162330259097e-05	\\
-3289.68616832386	2.79842356716843e-05	\\
-3288.70738636364	2.72265201574907e-05	\\
-3287.72860440341	2.71817508195805e-05	\\
-3286.74982244318	2.64425930587565e-05	\\
-3285.77104048296	2.83892176897133e-05	\\
-3284.79225852273	2.6331113991714e-05	\\
-3283.8134765625	2.6310930441163e-05	\\
-3282.83469460227	2.59541233298859e-05	\\
-3281.85591264205	2.48897228268446e-05	\\
-3280.87713068182	2.57791151264827e-05	\\
-3279.89834872159	2.67299881625761e-05	\\
-3278.91956676136	2.77066537505407e-05	\\
-3277.94078480114	2.60868872900086e-05	\\
-3276.96200284091	2.62047572655381e-05	\\
-3275.98322088068	2.6965148843671e-05	\\
-3275.00443892046	2.769768537884e-05	\\
-3274.02565696023	2.63409178172924e-05	\\
-3273.046875	2.67274104995821e-05	\\
-3272.06809303977	2.83751268357361e-05	\\
-3271.08931107955	2.78071926465021e-05	\\
-3270.11052911932	2.82100866308229e-05	\\
-3269.13174715909	2.72675934747914e-05	\\
-3268.15296519886	2.69257644654389e-05	\\
-3267.17418323864	2.80354723581697e-05	\\
-3266.19540127841	2.53241710049852e-05	\\
-3265.21661931818	2.74997031875501e-05	\\
-3264.23783735796	2.65023510196097e-05	\\
-3263.25905539773	2.53231581998652e-05	\\
-3262.2802734375	2.49544091317482e-05	\\
-3261.30149147727	2.6522931295981e-05	\\
-3260.32270951705	2.47170641603112e-05	\\
-3259.34392755682	2.57012079127383e-05	\\
-3258.36514559659	2.4251760578561e-05	\\
-3257.38636363636	2.91493391655013e-05	\\
-3256.40758167614	2.62313753271878e-05	\\
-3255.42879971591	2.70404172598839e-05	\\
-3254.45001775568	2.6561402776179e-05	\\
-3253.47123579546	2.41783907635198e-05	\\
-3252.49245383523	2.66851695320554e-05	\\
-3251.513671875	2.74636439380422e-05	\\
-3250.53488991477	2.65573835609638e-05	\\
-3249.55610795455	2.62596658599943e-05	\\
-3248.57732599432	2.62918901433421e-05	\\
-3247.59854403409	2.54467613433442e-05	\\
-3246.61976207386	2.53833779757922e-05	\\
-3245.64098011364	2.63746924213561e-05	\\
-3244.66219815341	2.62657249000196e-05	\\
-3243.68341619318	2.59468636618605e-05	\\
-3242.70463423296	2.5545003235616e-05	\\
-3241.72585227273	2.59474621552655e-05	\\
-3240.7470703125	2.61838866555709e-05	\\
-3239.76828835227	2.76760998988635e-05	\\
-3238.78950639205	2.48923081236455e-05	\\
-3237.81072443182	2.31296691443272e-05	\\
-3236.83194247159	2.59414059098524e-05	\\
-3235.85316051136	2.77445115148932e-05	\\
-3234.87437855114	2.46695804191386e-05	\\
-3233.89559659091	2.48599115451628e-05	\\
-3232.91681463068	2.43610630310536e-05	\\
-3231.93803267046	2.4233898829734e-05	\\
-3230.95925071023	2.42233017652336e-05	\\
-3229.98046875	2.78565672777605e-05	\\
-3229.00168678977	2.68952154845005e-05	\\
-3228.02290482955	2.76872291088727e-05	\\
-3227.04412286932	2.47860394640727e-05	\\
-3226.06534090909	2.54701044400599e-05	\\
-3225.08655894886	2.65061247139658e-05	\\
-3224.10777698864	2.55562766685636e-05	\\
-3223.12899502841	2.34117351212464e-05	\\
-3222.15021306818	2.77004165156562e-05	\\
-3221.17143110796	2.56951767102979e-05	\\
-3220.19264914773	2.3747742792618e-05	\\
-3219.2138671875	2.51172240821672e-05	\\
-3218.23508522727	2.49407118964842e-05	\\
-3217.25630326705	2.6286131858124e-05	\\
-3216.27752130682	2.43048140067802e-05	\\
-3215.29873934659	2.47182137408663e-05	\\
-3214.31995738636	2.4537509238848e-05	\\
-3213.34117542614	2.36321807190867e-05	\\
-3212.36239346591	2.43735769258516e-05	\\
-3211.38361150568	2.50479418128356e-05	\\
-3210.40482954546	2.59931175959529e-05	\\
-3209.42604758523	2.31119668052e-05	\\
-3208.447265625	2.36060925436506e-05	\\
-3207.46848366477	2.44749853283153e-05	\\
-3206.48970170455	2.51201092827802e-05	\\
-3205.51091974432	2.42612857674201e-05	\\
-3204.53213778409	2.64754296410716e-05	\\
-3203.55335582386	2.44242134858995e-05	\\
-3202.57457386364	2.55837356743133e-05	\\
-3201.59579190341	2.57137393936695e-05	\\
-3200.61700994318	2.56220156505222e-05	\\
-3199.63822798296	2.23867375309626e-05	\\
-3198.65944602273	2.55128847987403e-05	\\
-3197.6806640625	2.42778834097404e-05	\\
-3196.70188210227	2.50571311080104e-05	\\
-3195.72310014205	2.35790159553827e-05	\\
-3194.74431818182	2.44930639519883e-05	\\
};
\addplot [color=blue,solid,forget plot]
  table[row sep=crcr]{
-3194.74431818182	2.44930639519883e-05	\\
-3193.76553622159	2.49937024752897e-05	\\
-3192.78675426136	2.51314487840439e-05	\\
-3191.80797230114	2.44495826328206e-05	\\
-3190.82919034091	2.48267881813477e-05	\\
-3189.85040838068	2.48304760101527e-05	\\
-3188.87162642046	2.47369978546905e-05	\\
-3187.89284446023	2.59381721346717e-05	\\
-3186.9140625	2.49221062653559e-05	\\
-3185.93528053977	2.42721848345294e-05	\\
-3184.95649857955	2.40518081311757e-05	\\
-3183.97771661932	2.19189311602256e-05	\\
-3182.99893465909	2.21257348595129e-05	\\
-3182.02015269886	2.49359786475228e-05	\\
-3181.04137073864	2.50404722172278e-05	\\
-3180.06258877841	2.34042712629621e-05	\\
-3179.08380681818	2.51002708127759e-05	\\
-3178.10502485796	2.33611459122473e-05	\\
-3177.12624289773	2.72368468820841e-05	\\
-3176.1474609375	2.5274794726214e-05	\\
-3175.16867897727	2.2430162473246e-05	\\
-3174.18989701705	2.46503212315939e-05	\\
-3173.21111505682	2.69373616325791e-05	\\
-3172.23233309659	2.32336653650422e-05	\\
-3171.25355113636	2.55390089969731e-05	\\
-3170.27476917614	2.41570399716415e-05	\\
-3169.29598721591	2.69048977884425e-05	\\
-3168.31720525568	2.53420455509605e-05	\\
-3167.33842329546	2.19470735699161e-05	\\
-3166.35964133523	2.57976361790301e-05	\\
-3165.380859375	2.54841313913096e-05	\\
-3164.40207741477	2.20474169235945e-05	\\
-3163.42329545455	2.34339125807107e-05	\\
-3162.44451349432	2.40332789916624e-05	\\
-3161.46573153409	2.24466630392026e-05	\\
-3160.48694957386	2.40851281878694e-05	\\
-3159.50816761364	2.50691403794329e-05	\\
-3158.52938565341	2.19508498928436e-05	\\
-3157.55060369318	2.249879103797e-05	\\
-3156.57182173296	2.54041332643774e-05	\\
-3155.59303977273	2.38194371879052e-05	\\
-3154.6142578125	2.42267078680982e-05	\\
-3153.63547585227	2.26711211342244e-05	\\
-3152.65669389205	2.13581762475268e-05	\\
-3151.67791193182	2.32472281727439e-05	\\
-3150.69912997159	2.34627759194839e-05	\\
-3149.72034801136	2.23173400953447e-05	\\
-3148.74156605114	2.2298210867019e-05	\\
-3147.76278409091	2.40490603391454e-05	\\
-3146.78400213068	2.29562567293633e-05	\\
-3145.80522017046	2.29014592092463e-05	\\
-3144.82643821023	2.29617568296803e-05	\\
-3143.84765625	2.03952110721527e-05	\\
-3142.86887428977	2.20869222505923e-05	\\
-3141.89009232955	2.28260234722611e-05	\\
-3140.91131036932	2.29046916910961e-05	\\
-3139.93252840909	2.32630414849858e-05	\\
-3138.95374644886	2.31894850163205e-05	\\
-3137.97496448864	2.30625967106554e-05	\\
-3136.99618252841	2.20920533225822e-05	\\
-3136.01740056818	2.32498874583937e-05	\\
-3135.03861860796	2.22096137905034e-05	\\
-3134.05983664773	2.25133244990203e-05	\\
-3133.0810546875	2.14110994672225e-05	\\
-3132.10227272727	2.10121206014034e-05	\\
-3131.12349076705	2.30805711342974e-05	\\
-3130.14470880682	2.1343357681667e-05	\\
-3129.16592684659	2.38182911232122e-05	\\
-3128.18714488636	2.19349612655691e-05	\\
-3127.20836292614	2.25511195890831e-05	\\
-3126.22958096591	2.22004202541228e-05	\\
-3125.25079900568	1.97494959122096e-05	\\
-3124.27201704546	2.13237892383118e-05	\\
-3123.29323508523	2.20122749819884e-05	\\
-3122.314453125	1.95094269531777e-05	\\
-3121.33567116477	2.11735925578466e-05	\\
-3120.35688920455	2.23903040784307e-05	\\
-3119.37810724432	2.27162791933153e-05	\\
-3118.39932528409	2.18280894895399e-05	\\
-3117.42054332386	2.16243456248481e-05	\\
-3116.44176136364	2.15717382875973e-05	\\
-3115.46297940341	2.09263326077827e-05	\\
-3114.48419744318	2.2416695861739e-05	\\
-3113.50541548296	2.00869656892136e-05	\\
-3112.52663352273	2.25419325706004e-05	\\
-3111.5478515625	2.12974534477075e-05	\\
-3110.56906960227	1.70971943267541e-05	\\
-3109.59028764205	2.15166169418515e-05	\\
-3108.61150568182	2.05452237864934e-05	\\
-3107.63272372159	1.90380733567631e-05	\\
-3106.65394176136	1.96904876814061e-05	\\
-3105.67515980114	2.24471139553551e-05	\\
-3104.69637784091	1.93271405982667e-05	\\
-3103.71759588068	2.22001554930271e-05	\\
-3102.73881392046	2.17135944402231e-05	\\
-3101.76003196023	1.86363668517638e-05	\\
-3100.78125	1.77233720264678e-05	\\
-3099.80246803977	1.81242854653395e-05	\\
-3098.82368607955	2.01581587820279e-05	\\
-3097.84490411932	2.11318114688053e-05	\\
-3096.86612215909	1.90186896629363e-05	\\
-3095.88734019886	1.95915202684005e-05	\\
-3094.90855823864	2.060581697804e-05	\\
-3093.92977627841	2.04984259777177e-05	\\
-3092.95099431818	1.94393728271848e-05	\\
-3091.97221235796	1.9596399355261e-05	\\
-3090.99343039773	2.15720296238791e-05	\\
-3090.0146484375	1.99839220290391e-05	\\
-3089.03586647727	2.18671408540984e-05	\\
-3088.05708451705	1.92691763169621e-05	\\
-3087.07830255682	2.18801323816361e-05	\\
-3086.09952059659	1.71350244177662e-05	\\
-3085.12073863636	1.94352182942994e-05	\\
-3084.14195667614	2.02846031766605e-05	\\
-3083.16317471591	1.72709120200915e-05	\\
-3082.18439275568	1.89636204200004e-05	\\
-3081.20561079546	2.08932331680671e-05	\\
-3080.22682883523	1.90624418855525e-05	\\
-3079.248046875	2.01525930455698e-05	\\
-3078.26926491477	2.15754660138862e-05	\\
-3077.29048295455	1.88122450655951e-05	\\
-3076.31170099432	2.06387819434248e-05	\\
-3075.33291903409	2.08305415143197e-05	\\
-3074.35413707386	1.84071153183618e-05	\\
-3073.37535511364	1.96921080834277e-05	\\
-3072.39657315341	1.89518649136127e-05	\\
-3071.41779119318	1.93861169414814e-05	\\
-3070.43900923296	1.80320075222958e-05	\\
-3069.46022727273	1.93333647838383e-05	\\
-3068.4814453125	2.1141504531014e-05	\\
-3067.50266335227	2.04915038624815e-05	\\
-3066.52388139205	2.11249099560844e-05	\\
-3065.54509943182	2.08811566482654e-05	\\
-3064.56631747159	1.97183922570874e-05	\\
-3063.58753551136	1.6147943486279e-05	\\
-3062.60875355114	1.93764831106611e-05	\\
-3061.62997159091	1.89461356794533e-05	\\
-3060.65118963068	1.99922640321775e-05	\\
-3059.67240767046	1.87999396268018e-05	\\
-3058.69362571023	1.93060387713118e-05	\\
-3057.71484375	1.9978012328033e-05	\\
-3056.73606178977	1.80606572566157e-05	\\
-3055.75727982955	1.74669514287305e-05	\\
-3054.77849786932	1.7786664145648e-05	\\
-3053.79971590909	1.75381436247568e-05	\\
-3052.82093394886	1.85724117360982e-05	\\
-3051.84215198864	1.95117166118709e-05	\\
-3050.86337002841	1.95397533877935e-05	\\
-3049.88458806818	1.66329499170867e-05	\\
-3048.90580610796	1.8427382622255e-05	\\
-3047.92702414773	1.97581941597322e-05	\\
-3046.9482421875	1.89046197541266e-05	\\
-3045.96946022727	1.83836695163041e-05	\\
-3044.99067826705	1.71949175073433e-05	\\
-3044.01189630682	1.69538426726525e-05	\\
-3043.03311434659	1.8959905538742e-05	\\
-3042.05433238636	1.87492151403952e-05	\\
-3041.07555042614	1.9047753405261e-05	\\
-3040.09676846591	1.9688984485723e-05	\\
-3039.11798650568	1.67200245906987e-05	\\
-3038.13920454546	1.88873120666583e-05	\\
-3037.16042258523	2.01581340271959e-05	\\
-3036.181640625	1.77333019426263e-05	\\
-3035.20285866477	1.6647069470847e-05	\\
-3034.22407670455	1.93383308352324e-05	\\
-3033.24529474432	1.62074255736376e-05	\\
-3032.26651278409	1.99897037709428e-05	\\
-3031.28773082386	1.82718790491818e-05	\\
-3030.30894886364	1.89453602345133e-05	\\
-3029.33016690341	1.7114380974091e-05	\\
-3028.35138494318	1.77890002342489e-05	\\
-3027.37260298296	1.88326166570252e-05	\\
-3026.39382102273	2.03162166519096e-05	\\
-3025.4150390625	1.8356087238543e-05	\\
-3024.43625710227	1.46250048071723e-05	\\
-3023.45747514205	1.73288565377181e-05	\\
-3022.47869318182	1.74732013067107e-05	\\
-3021.49991122159	1.81789783573614e-05	\\
-3020.52112926136	1.96233245117423e-05	\\
-3019.54234730114	1.92935999361123e-05	\\
-3018.56356534091	1.79940300533929e-05	\\
-3017.58478338068	2.01788878224487e-05	\\
-3016.60600142046	1.67215627711379e-05	\\
-3015.62721946023	1.842045498498e-05	\\
-3014.6484375	1.943807944961e-05	\\
-3013.66965553977	1.57946786694552e-05	\\
-3012.69087357955	1.82067398914517e-05	\\
-3011.71209161932	1.8606415803894e-05	\\
-3010.73330965909	1.82434211703607e-05	\\
-3009.75452769886	1.82732541202478e-05	\\
-3008.77574573864	1.8712909445827e-05	\\
-3007.79696377841	1.81815748565331e-05	\\
-3006.81818181818	1.8228400824243e-05	\\
-3005.83939985796	1.87303046192519e-05	\\
-3004.86061789773	1.83858843264072e-05	\\
-3003.8818359375	1.66467080603905e-05	\\
-3002.90305397727	2.1454517574439e-05	\\
-3001.92427201705	1.75409893661511e-05	\\
-3000.94549005682	1.93047367160637e-05	\\
-2999.96670809659	1.60896345296361e-05	\\
-2998.98792613636	1.82578621691894e-05	\\
-2998.00914417614	2.00555678671201e-05	\\
-2997.03036221591	1.72969395014995e-05	\\
-2996.05158025568	1.95819766448208e-05	\\
-2995.07279829546	1.59828201523878e-05	\\
-2994.09401633523	1.7490253211493e-05	\\
-2993.115234375	1.74789024481298e-05	\\
-2992.13645241477	1.98184260829429e-05	\\
-2991.15767045455	1.83137777194535e-05	\\
-2990.17888849432	1.7306870317261e-05	\\
-2989.20010653409	1.76516833668867e-05	\\
-2988.22132457386	2.00281359233279e-05	\\
-2987.24254261364	1.6881977205576e-05	\\
-2986.26376065341	1.975767776712e-05	\\
-2985.28497869318	1.64529411867136e-05	\\
-2984.30619673296	1.78989046786787e-05	\\
-2983.32741477273	1.72234871197945e-05	\\
-2982.3486328125	1.98128108970674e-05	\\
-2981.36985085227	1.73568143613023e-05	\\
-2980.39106889205	1.6298133056546e-05	\\
-2979.41228693182	1.6651985486597e-05	\\
-2978.43350497159	1.89630259084631e-05	\\
-2977.45472301136	1.831277337377e-05	\\
-2976.47594105114	1.86393376840052e-05	\\
-2975.49715909091	1.70214198237477e-05	\\
-2974.51837713068	1.84581047094636e-05	\\
-2973.53959517046	1.67999240448346e-05	\\
-2972.56081321023	1.85993738547852e-05	\\
-2971.58203125	1.97983610962164e-05	\\
-2970.60324928977	1.81668929314531e-05	\\
-2969.62446732955	1.64377193613238e-05	\\
-2968.64568536932	1.91080857693209e-05	\\
-2967.66690340909	1.61239788294538e-05	\\
-2966.68812144886	1.75660990481137e-05	\\
-2965.70933948864	1.96525622362759e-05	\\
-2964.73055752841	1.32223736267166e-05	\\
-2963.75177556818	1.66698298962428e-05	\\
-2962.77299360796	1.67461895991208e-05	\\
-2961.79421164773	1.81687644315901e-05	\\
-2960.8154296875	1.84135855437673e-05	\\
-2959.83664772727	1.566005444551e-05	\\
-2958.85786576705	1.82748266228578e-05	\\
-2957.87908380682	1.79517762890306e-05	\\
-2956.90030184659	1.8363295823188e-05	\\
-2955.92151988636	1.76845237095588e-05	\\
-2954.94273792614	2.07745573173736e-05	\\
-2953.96395596591	2.13023242278591e-05	\\
-2952.98517400568	1.82208988742954e-05	\\
-2952.00639204546	2.02288946001371e-05	\\
-2951.02761008523	1.63440223447943e-05	\\
-2950.048828125	1.99011209981104e-05	\\
-2949.07004616477	1.96631986981876e-05	\\
-2948.09126420455	1.79654480984308e-05	\\
-2947.11248224432	1.84328677827259e-05	\\
-2946.13370028409	1.99224344479094e-05	\\
-2945.15491832386	1.81049432115915e-05	\\
-2944.17613636364	1.8867956946944e-05	\\
-2943.19735440341	1.99546355353341e-05	\\
-2942.21857244318	1.70159830231526e-05	\\
-2941.23979048296	1.79221195108957e-05	\\
-2940.26100852273	1.89466849913821e-05	\\
-2939.2822265625	1.90473977083559e-05	\\
-2938.30344460227	1.83675930492844e-05	\\
-2937.32466264205	1.65022636725186e-05	\\
-2936.34588068182	1.83791959290153e-05	\\
-2935.36709872159	1.83801460590794e-05	\\
-2934.38831676136	1.79995369406683e-05	\\
-2933.40953480114	2.07082665697271e-05	\\
-2932.43075284091	1.7421470024258e-05	\\
-2931.45197088068	1.99338173646117e-05	\\
-2930.47318892046	2.01976212093897e-05	\\
-2929.49440696023	1.98183419878125e-05	\\
-2928.515625	1.89340629466283e-05	\\
-2927.53684303977	1.86737894545519e-05	\\
-2926.55806107955	1.89409187900189e-05	\\
-2925.57927911932	1.80700861373462e-05	\\
-2924.60049715909	1.7502967582197e-05	\\
-2923.62171519886	1.87239991538971e-05	\\
-2922.64293323864	1.54420014504817e-05	\\
-2921.66415127841	1.76707807879969e-05	\\
-2920.68536931818	2.02820077465299e-05	\\
-2919.70658735796	1.69579591913075e-05	\\
-2918.72780539773	2.0354057012636e-05	\\
-2917.7490234375	1.8483231861992e-05	\\
-2916.77024147727	1.92135282174097e-05	\\
-2915.79145951705	1.79980304964091e-05	\\
-2914.81267755682	1.71170588116863e-05	\\
-2913.83389559659	2.07555200616372e-05	\\
-2912.85511363636	1.99004364241069e-05	\\
-2911.87633167614	2.18171454279241e-05	\\
-2910.89754971591	1.73875432259079e-05	\\
-2909.91876775568	2.0999820592396e-05	\\
-2908.93998579546	1.90681718926136e-05	\\
-2907.96120383523	1.61110448114415e-05	\\
-2906.982421875	2.00230305434867e-05	\\
-2906.00363991477	1.78214048040843e-05	\\
-2905.02485795455	1.96848248721722e-05	\\
-2904.04607599432	2.12127833193788e-05	\\
-2903.06729403409	1.74951356638755e-05	\\
-2902.08851207386	1.89379116057392e-05	\\
-2901.10973011364	2.05670134871905e-05	\\
-2900.13094815341	1.74462410930161e-05	\\
-2899.15216619318	1.9746363958316e-05	\\
-2898.17338423296	1.83372444835143e-05	\\
-2897.19460227273	1.93205829984685e-05	\\
-2896.2158203125	1.87576562520079e-05	\\
-2895.23703835227	1.94204066863415e-05	\\
-2894.25825639205	2.04859155960764e-05	\\
-2893.27947443182	1.59309210952779e-05	\\
-2892.30069247159	1.85912188554911e-05	\\
-2891.32191051136	1.92135761564015e-05	\\
-2890.34312855114	1.8977041956412e-05	\\
-2889.36434659091	1.85296942215542e-05	\\
-2888.38556463068	1.872282673268e-05	\\
-2887.40678267046	1.90789213419691e-05	\\
-2886.42800071023	1.87994093139989e-05	\\
-2885.44921875	1.65369903961317e-05	\\
-2884.47043678977	2.05376352896774e-05	\\
-2883.49165482955	1.84659511427551e-05	\\
-2882.51287286932	1.91669866469091e-05	\\
-2881.53409090909	1.99301657355375e-05	\\
-2880.55530894886	2.27325064805072e-05	\\
-2879.57652698864	1.73363176731688e-05	\\
-2878.59774502841	1.79998811222487e-05	\\
-2877.61896306818	1.87849777520466e-05	\\
-2876.64018110796	1.79443350638556e-05	\\
-2875.66139914773	2.05122662637587e-05	\\
-2874.6826171875	2.10698480914893e-05	\\
-2873.70383522727	2.15109766086765e-05	\\
-2872.72505326705	1.48393688155959e-05	\\
-2871.74627130682	1.85422340265653e-05	\\
-2870.76748934659	1.90285065001494e-05	\\
-2869.78870738636	2.0008583493363e-05	\\
-2868.80992542614	1.89591975393546e-05	\\
-2867.83114346591	1.88234193522348e-05	\\
-2866.85236150568	2.02451466166985e-05	\\
-2865.87357954546	2.11857986243474e-05	\\
-2864.89479758523	1.91519470275468e-05	\\
-2863.916015625	1.91060266839785e-05	\\
-2862.93723366477	2.15490446412264e-05	\\
-2861.95845170455	1.89768543925169e-05	\\
-2860.97966974432	2.0161307779564e-05	\\
-2860.00088778409	2.02116009387723e-05	\\
-2859.02210582386	2.12055219392017e-05	\\
-2858.04332386364	1.84676853704035e-05	\\
-2857.06454190341	2.03454639327565e-05	\\
-2856.08575994318	1.81034539952858e-05	\\
-2855.10697798296	1.95312356986149e-05	\\
-2854.12819602273	2.10156543334026e-05	\\
-2853.1494140625	1.85096307081488e-05	\\
-2852.17063210227	1.83893473628002e-05	\\
-2851.19185014205	2.05711783568582e-05	\\
-2850.21306818182	1.92116747857969e-05	\\
-2849.23428622159	1.94971886658953e-05	\\
-2848.25550426136	1.95250975009233e-05	\\
-2847.27672230114	2.05256159829947e-05	\\
-2846.29794034091	1.97558614590263e-05	\\
-2845.31915838068	1.86088956850067e-05	\\
-2844.34037642046	1.94368078495917e-05	\\
-2843.36159446023	1.82953898693164e-05	\\
-2842.3828125	2.0701679787986e-05	\\
-2841.40403053977	2.13530718576188e-05	\\
-2840.42524857955	2.12745816326391e-05	\\
-2839.44646661932	1.90178230916797e-05	\\
-2838.46768465909	1.73486992933872e-05	\\
-2837.48890269886	2.15143889811613e-05	\\
-2836.51012073864	1.99371524492691e-05	\\
-2835.53133877841	2.17613302827372e-05	\\
-2834.55255681818	2.2390212848128e-05	\\
-2833.57377485796	2.12204026888541e-05	\\
-2832.59499289773	1.91194736382327e-05	\\
-2831.6162109375	2.43553745359511e-05	\\
-2830.63742897727	1.85298904198995e-05	\\
-2829.65864701705	2.11191440427505e-05	\\
-2828.67986505682	2.03038150422579e-05	\\
-2827.70108309659	2.30040920266733e-05	\\
-2826.72230113636	1.84207584584354e-05	\\
-2825.74351917614	1.92965808596299e-05	\\
-2824.76473721591	1.80144132697005e-05	\\
-2823.78595525568	2.1366441073681e-05	\\
-2822.80717329545	1.93306356870358e-05	\\
-2821.82839133523	2.45921342073449e-05	\\
-2820.849609375	2.02027807156024e-05	\\
-2819.87082741477	2.14555373613167e-05	\\
-2818.89204545455	2.04980874256181e-05	\\
-2817.91326349432	2.25876817926739e-05	\\
-2816.93448153409	2.32455252075634e-05	\\
-2815.95569957386	2.1054547248773e-05	\\
-2814.97691761364	2.32804583698245e-05	\\
-2813.99813565341	2.3283801428466e-05	\\
-2813.01935369318	2.01070874466811e-05	\\
-2812.04057173295	2.09600746360778e-05	\\
-2811.06178977273	2.27772283870946e-05	\\
-2810.0830078125	2.02465036511171e-05	\\
-2809.10422585227	2.09292223502584e-05	\\
-2808.12544389205	2.19040890138052e-05	\\
-2807.14666193182	2.15441011107391e-05	\\
-2806.16787997159	2.24231130960884e-05	\\
-2805.18909801136	2.17473093238762e-05	\\
-2804.21031605114	2.08664167822329e-05	\\
-2803.23153409091	2.25439246232273e-05	\\
-2802.25275213068	2.27983110825656e-05	\\
-2801.27397017045	2.15714992871492e-05	\\
-2800.29518821023	2.25202719630697e-05	\\
-2799.31640625	2.15239071807475e-05	\\
-2798.33762428977	2.22226203016062e-05	\\
-2797.35884232955	2.36793144002128e-05	\\
-2796.38006036932	2.27199241168056e-05	\\
-2795.40127840909	2.05274700424997e-05	\\
-2794.42249644886	2.55097511077365e-05	\\
-2793.44371448864	2.23166100948465e-05	\\
-2792.46493252841	2.40885525130122e-05	\\
-2791.48615056818	2.1307469951159e-05	\\
-2790.50736860795	2.25578319670255e-05	\\
-2789.52858664773	2.52306640408662e-05	\\
-2788.5498046875	2.34439517463401e-05	\\
-2787.57102272727	2.54852953420921e-05	\\
-2786.59224076705	2.40878391884499e-05	\\
-2785.61345880682	2.39939122428335e-05	\\
-2784.63467684659	2.43930463385165e-05	\\
-2783.65589488636	2.28659075278871e-05	\\
-2782.67711292614	2.20979010428199e-05	\\
-2781.69833096591	2.2186235222722e-05	\\
-2780.71954900568	2.5287665057269e-05	\\
-2779.74076704545	2.3319102447362e-05	\\
-2778.76198508523	2.35105742424524e-05	\\
-2777.783203125	2.63141703308884e-05	\\
-2776.80442116477	2.29125708461976e-05	\\
-2775.82563920455	2.41091493582746e-05	\\
-2774.84685724432	2.26355399163891e-05	\\
-2773.86807528409	2.43314498514879e-05	\\
-2772.88929332386	2.15581486715954e-05	\\
-2771.91051136364	2.66508453890658e-05	\\
-2770.93172940341	2.87432226194282e-05	\\
-2769.95294744318	2.33984969530706e-05	\\
-2768.97416548295	2.40521964813243e-05	\\
-2767.99538352273	2.57001830202924e-05	\\
-2767.0166015625	2.58954475655435e-05	\\
-2766.03781960227	2.45811179659459e-05	\\
-2765.05903764205	2.47011815392471e-05	\\
-2764.08025568182	2.6533384248695e-05	\\
-2763.10147372159	2.62909245998858e-05	\\
-2762.12269176136	2.53010803187141e-05	\\
-2761.14390980114	2.79422297195596e-05	\\
-2760.16512784091	2.63895706419908e-05	\\
-2759.18634588068	2.75318592023154e-05	\\
-2758.20756392045	2.83120450198848e-05	\\
-2757.22878196023	2.59772006073186e-05	\\
-2756.25	2.44409613957638e-05	\\
-2755.27121803977	2.70411420766344e-05	\\
-2754.29243607955	2.79024157128599e-05	\\
-2753.31365411932	2.6234149330077e-05	\\
-2752.33487215909	2.64744578563594e-05	\\
-2751.35609019886	2.62310700456499e-05	\\
-2750.37730823864	2.66661399606421e-05	\\
-2749.39852627841	2.7100545085349e-05	\\
-2748.41974431818	2.76455896414105e-05	\\
-2747.44096235795	2.77659949740954e-05	\\
-2746.46218039773	2.53014246873049e-05	\\
-2745.4833984375	2.82868796619988e-05	\\
-2744.50461647727	2.7217786070398e-05	\\
-2743.52583451705	2.87159088372172e-05	\\
-2742.54705255682	2.70514821200995e-05	\\
-2741.56827059659	2.80979273215702e-05	\\
-2740.58948863636	2.83146981332733e-05	\\
-2739.61070667614	2.85392420005007e-05	\\
-2738.63192471591	3.03056741384919e-05	\\
-2737.65314275568	2.96540790564767e-05	\\
-2736.67436079545	3.05416521954355e-05	\\
-2735.69557883523	3.09909421809157e-05	\\
-2734.716796875	2.60735692610234e-05	\\
-2733.73801491477	2.7885387031144e-05	\\
-2732.75923295455	2.83362048152514e-05	\\
-2731.78045099432	3.1744387593972e-05	\\
-2730.80166903409	3.16779821234013e-05	\\
-2729.82288707386	2.84114620647156e-05	\\
-2728.84410511364	3.00548694360573e-05	\\
-2727.86532315341	2.89546732828627e-05	\\
-2726.88654119318	2.77635188000487e-05	\\
-2725.90775923295	3.08460494245915e-05	\\
-2724.92897727273	2.96153464653543e-05	\\
-2723.9501953125	2.83533954918557e-05	\\
-2722.97141335227	3.16491464141443e-05	\\
-2721.99263139205	2.93940929610234e-05	\\
-2721.01384943182	2.53488631389033e-05	\\
-2720.03506747159	2.86784001024809e-05	\\
-2719.05628551136	2.82098953450447e-05	\\
-2718.07750355114	2.80372080376854e-05	\\
-2717.09872159091	2.73452651108748e-05	\\
-2716.11993963068	2.9087570820793e-05	\\
-2715.14115767045	3.0388035211364e-05	\\
-2714.16237571023	2.71034073498786e-05	\\
-2713.18359375	2.81249244504799e-05	\\
-2712.20481178977	2.89967102653706e-05	\\
-2711.22602982955	2.83284934324881e-05	\\
-2710.24724786932	2.5223654624928e-05	\\
-2709.26846590909	2.71772158485433e-05	\\
-2708.28968394886	3.00540995062243e-05	\\
-2707.31090198864	2.72832594135825e-05	\\
-2706.33212002841	2.72671740354263e-05	\\
-2705.35333806818	2.85141345429866e-05	\\
-2704.37455610795	2.99031398907807e-05	\\
-2703.39577414773	2.79624981052944e-05	\\
-2702.4169921875	2.85073593017633e-05	\\
-2701.43821022727	2.91851830224988e-05	\\
-2700.45942826705	2.30774068471414e-05	\\
-2699.48064630682	2.92699138079144e-05	\\
-2698.50186434659	2.68642769109073e-05	\\
-2697.52308238636	2.82352297984222e-05	\\
-2696.54430042614	2.71159383297595e-05	\\
-2695.56551846591	2.6609337651442e-05	\\
-2694.58673650568	2.96258652136909e-05	\\
-2693.60795454545	2.7047659229525e-05	\\
-2692.62917258523	2.79286966105448e-05	\\
-2691.650390625	2.75122586799911e-05	\\
-2690.67160866477	2.54456435407681e-05	\\
-2689.69282670455	2.77597399466493e-05	\\
-2688.71404474432	2.63031221743365e-05	\\
-2687.73526278409	2.94023261456112e-05	\\
-2686.75648082386	2.78397152593009e-05	\\
-2685.77769886364	2.52510795265835e-05	\\
-2684.79891690341	2.59978760279909e-05	\\
-2683.82013494318	2.38861099784432e-05	\\
-2682.84135298295	2.57950899306832e-05	\\
-2681.86257102273	2.44965698445296e-05	\\
-2680.8837890625	2.42640207245632e-05	\\
-2679.90500710227	2.5076468591681e-05	\\
-2678.92622514205	2.45238040212456e-05	\\
-2677.94744318182	2.65423070300594e-05	\\
-2676.96866122159	2.48474884484333e-05	\\
-2675.98987926136	2.42507265231138e-05	\\
-2675.01109730114	2.57010559517779e-05	\\
-2674.03231534091	2.57071884399892e-05	\\
-2673.05353338068	2.40935772453399e-05	\\
-2672.07475142045	2.77915749850453e-05	\\
-2671.09596946023	2.62622191067065e-05	\\
-2670.1171875	2.52214365305995e-05	\\
-2669.13840553977	2.56661101377593e-05	\\
-2668.15962357955	2.40977051611461e-05	\\
-2667.18084161932	2.62621607632892e-05	\\
-2666.20205965909	2.50189957892324e-05	\\
-2665.22327769886	2.4710439206489e-05	\\
-2664.24449573864	2.58459738819332e-05	\\
-2663.26571377841	2.54678866256394e-05	\\
-2662.28693181818	2.42702073626954e-05	\\
-2661.30814985795	2.24700937734973e-05	\\
-2660.32936789773	2.65030240812193e-05	\\
-2659.3505859375	2.29861636256438e-05	\\
-2658.37180397727	2.72337957584663e-05	\\
-2657.39302201705	2.30587625648383e-05	\\
-2656.41424005682	2.35844356775057e-05	\\
-2655.43545809659	2.0803269913584e-05	\\
-2654.45667613636	2.68695605803838e-05	\\
-2653.47789417614	2.23652669276292e-05	\\
-2652.49911221591	2.51453299174458e-05	\\
-2651.52033025568	2.29074429184303e-05	\\
-2650.54154829545	2.65120076803271e-05	\\
-2649.56276633523	2.37436867752002e-05	\\
-2648.583984375	2.54296745778406e-05	\\
-2647.60520241477	2.53218939470525e-05	\\
-2646.62642045455	2.54938788229841e-05	\\
-2645.64763849432	2.68213399469634e-05	\\
-2644.66885653409	2.4761523244046e-05	\\
-2643.69007457386	2.20030661751509e-05	\\
-2642.71129261364	2.64912419753963e-05	\\
-2641.73251065341	2.37163430876328e-05	\\
-2640.75372869318	2.49523186446873e-05	\\
-2639.77494673295	2.48125660344154e-05	\\
-2638.79616477273	2.5375890055062e-05	\\
-2637.8173828125	2.6974045770843e-05	\\
-2636.83860085227	2.35584730611921e-05	\\
-2635.85981889205	2.50105749220682e-05	\\
-2634.88103693182	2.48164660327306e-05	\\
-2633.90225497159	2.65819447561779e-05	\\
-2632.92347301136	2.40791869178586e-05	\\
-2631.94469105114	2.60547220887187e-05	\\
-2630.96590909091	2.14264827366176e-05	\\
-2629.98712713068	2.54212585467141e-05	\\
-2629.00834517045	2.7905781632233e-05	\\
-2628.02956321023	2.57248169859696e-05	\\
-2627.05078125	2.50059690602025e-05	\\
-2626.07199928977	2.66916347408925e-05	\\
-2625.09321732955	2.61047026233646e-05	\\
-2624.11443536932	2.69921722220364e-05	\\
-2623.13565340909	2.26527855554288e-05	\\
-2622.15687144886	2.13724047626409e-05	\\
-2621.17808948864	2.6908940403992e-05	\\
-2620.19930752841	2.39535526995534e-05	\\
-2619.22052556818	2.42857121140061e-05	\\
-2618.24174360795	2.50831368303207e-05	\\
-2617.26296164773	2.46729211956196e-05	\\
-2616.2841796875	2.41084442052029e-05	\\
-2615.30539772727	2.51024031279861e-05	\\
-2614.32661576705	2.72610450055072e-05	\\
-2613.34783380682	2.48277507769937e-05	\\
-2612.36905184659	2.56555588988557e-05	\\
-2611.39026988636	2.54651776489215e-05	\\
-2610.41148792614	2.66641398131324e-05	\\
-2609.43270596591	3.01411483233989e-05	\\
-2608.45392400568	2.62801023720113e-05	\\
-2607.47514204545	2.44263631530813e-05	\\
-2606.49636008523	2.48943429555044e-05	\\
-2605.517578125	2.55941748282823e-05	\\
-2604.53879616477	2.46496529346256e-05	\\
-2603.56001420455	2.56850644275e-05	\\
-2602.58123224432	2.50115803662932e-05	\\
-2601.60245028409	2.56124644369014e-05	\\
-2600.62366832386	2.66409035956135e-05	\\
-2599.64488636364	2.65596503289559e-05	\\
-2598.66610440341	2.78886326441706e-05	\\
-2597.68732244318	2.78408737892308e-05	\\
-2596.70854048295	2.63521761610519e-05	\\
-2595.72975852273	2.72035480414977e-05	\\
-2594.7509765625	2.69223299527807e-05	\\
-2593.77219460227	2.63391381957757e-05	\\
-2592.79341264205	2.69453063458697e-05	\\
-2591.81463068182	2.59331307770943e-05	\\
-2590.83584872159	2.7883753103648e-05	\\
-2589.85706676136	2.56977687627729e-05	\\
-2588.87828480114	2.83320260442978e-05	\\
-2587.89950284091	2.67580631542588e-05	\\
-2586.92072088068	2.86988992786332e-05	\\
-2585.94193892045	2.79524992485564e-05	\\
-2584.96315696023	2.35520662048938e-05	\\
-2583.984375	2.49856525756688e-05	\\
-2583.00559303977	2.81671247751105e-05	\\
-2582.02681107955	2.66363844600349e-05	\\
-2581.04802911932	2.60877153854574e-05	\\
-2580.06924715909	2.75219989407548e-05	\\
-2579.09046519886	2.85327655354963e-05	\\
-2578.11168323864	2.73817413552102e-05	\\
-2577.13290127841	2.72057601847118e-05	\\
-2576.15411931818	2.82700505178506e-05	\\
-2575.17533735795	2.72387237893865e-05	\\
-2574.19655539773	2.37943442843658e-05	\\
-2573.2177734375	2.90901005067119e-05	\\
-2572.23899147727	2.66156942641854e-05	\\
-2571.26020951705	2.88452267189581e-05	\\
-2570.28142755682	3.12685689389126e-05	\\
-2569.30264559659	2.94117871703021e-05	\\
-2568.32386363636	2.83039325499956e-05	\\
-2567.34508167614	2.91325218785564e-05	\\
-2566.36629971591	3.01681070264308e-05	\\
-2565.38751775568	3.02295164923102e-05	\\
-2564.40873579545	2.93332923154521e-05	\\
-2563.42995383523	2.87177050904552e-05	\\
-2562.451171875	3.0039418791313e-05	\\
-2561.47238991477	2.87726786816992e-05	\\
-2560.49360795455	3.08257071272492e-05	\\
-2559.51482599432	2.8322613303047e-05	\\
-2558.53604403409	2.62978953311492e-05	\\
-2557.55726207386	2.64893879430028e-05	\\
-2556.57848011364	2.89564069804444e-05	\\
-2555.59969815341	2.87493245356511e-05	\\
-2554.62091619318	3.09211540282683e-05	\\
-2553.64213423295	2.88646232868546e-05	\\
-2552.66335227273	2.90077312858853e-05	\\
-2551.6845703125	2.76143099173774e-05	\\
-2550.70578835227	2.80046085955354e-05	\\
-2549.72700639205	2.93398461407816e-05	\\
-2548.74822443182	3.35596994224686e-05	\\
-2547.76944247159	3.0818190097289e-05	\\
-2546.79066051136	2.91132090181823e-05	\\
-2545.81187855114	3.1597323829362e-05	\\
-2544.83309659091	2.97172629307902e-05	\\
-2543.85431463068	3.08902440950758e-05	\\
-2542.87553267045	2.94390396379311e-05	\\
-2541.89675071023	2.85197733676731e-05	\\
-2540.91796875	3.09447090582703e-05	\\
-2539.93918678977	2.95671650355854e-05	\\
-2538.96040482955	2.83553437700712e-05	\\
-2537.98162286932	2.84354108439284e-05	\\
-2537.00284090909	2.87018993413095e-05	\\
-2536.02405894886	2.83684177504166e-05	\\
-2535.04527698864	3.00547029504747e-05	\\
-2534.06649502841	2.96612335906605e-05	\\
-2533.08771306818	3.1524677979947e-05	\\
-2532.10893110795	2.80853097331465e-05	\\
-2531.13014914773	3.19266139992086e-05	\\
-2530.1513671875	2.92729710104672e-05	\\
-2529.17258522727	2.88353928248409e-05	\\
-2528.19380326705	3.26985334525992e-05	\\
-2527.21502130682	2.58226770913235e-05	\\
-2526.23623934659	3.28513070170368e-05	\\
-2525.25745738636	3.28019486279945e-05	\\
-2524.27867542614	3.21959835635737e-05	\\
-2523.29989346591	2.92069325582674e-05	\\
-2522.32111150568	3.0511116286155e-05	\\
-2521.34232954545	2.74634277990169e-05	\\
-2520.36354758523	3.57262949684138e-05	\\
-2519.384765625	3.02208849576764e-05	\\
-2518.40598366477	3.0114329194009e-05	\\
-2517.42720170455	2.80088566865963e-05	\\
-2516.44841974432	3.23631395136899e-05	\\
-2515.46963778409	3.10446579561967e-05	\\
-2514.49085582386	3.15722962128562e-05	\\
-2513.51207386364	3.15594616645272e-05	\\
-2512.53329190341	3.04897794321421e-05	\\
-2511.55450994318	3.3029449360611e-05	\\
-2510.57572798295	3.10668366998133e-05	\\
-2509.59694602273	3.2657804815578e-05	\\
-2508.6181640625	2.90229100006585e-05	\\
-2507.63938210227	3.16446573529461e-05	\\
-2506.66060014205	3.16857179522241e-05	\\
-2505.68181818182	3.1454177355769e-05	\\
-2504.70303622159	2.83358814709232e-05	\\
-2503.72425426136	3.04501156498962e-05	\\
-2502.74547230114	2.93911321709623e-05	\\
-2501.76669034091	3.3526008295231e-05	\\
-2500.78790838068	3.34462571563857e-05	\\
-2499.80912642045	3.70255184880236e-05	\\
-2498.83034446023	3.21816800963609e-05	\\
-2497.8515625	3.32466150013672e-05	\\
-2496.87278053977	3.2497615889678e-05	\\
-2495.89399857955	3.02513686479453e-05	\\
-2494.91521661932	3.10310035098945e-05	\\
-2493.93643465909	2.92087504696752e-05	\\
-2492.95765269886	3.35081113552793e-05	\\
-2491.97887073864	3.33445067704515e-05	\\
-2491.00008877841	3.05779777584642e-05	\\
-2490.02130681818	3.01541364459647e-05	\\
-2489.04252485795	3.00249173067783e-05	\\
-2488.06374289773	3.18175368538256e-05	\\
-2487.0849609375	3.10421822443782e-05	\\
-2486.10617897727	2.90418878282497e-05	\\
-2485.12739701705	3.13616737505557e-05	\\
-2484.14861505682	3.49701113141453e-05	\\
-2483.16983309659	3.5169536008127e-05	\\
-2482.19105113636	2.965404719979e-05	\\
-2481.21226917614	3.06428835685974e-05	\\
-2480.23348721591	3.27047091157365e-05	\\
-2479.25470525568	3.43075754731708e-05	\\
-2478.27592329545	2.94598971663245e-05	\\
-2477.29714133523	3.23518811035126e-05	\\
-2476.318359375	3.57851456276376e-05	\\
-2475.33957741477	3.52053329243497e-05	\\
-2474.36079545455	3.06179937593344e-05	\\
-2473.38201349432	2.80304646021835e-05	\\
-2472.40323153409	3.04443221261863e-05	\\
-2471.42444957386	3.12665035667582e-05	\\
-2470.44566761364	3.25039181295669e-05	\\
-2469.46688565341	3.21434810703436e-05	\\
-2468.48810369318	3.23455860194759e-05	\\
-2467.50932173295	3.40331952200639e-05	\\
-2466.53053977273	3.30056615339298e-05	\\
-2465.5517578125	3.23581739970601e-05	\\
-2464.57297585227	3.63693182669536e-05	\\
-2463.59419389205	3.30539441823603e-05	\\
-2462.61541193182	3.03961402537896e-05	\\
-2461.63662997159	3.32672515155086e-05	\\
-2460.65784801136	3.24522496227256e-05	\\
-2459.67906605114	3.59870550062031e-05	\\
-2458.70028409091	3.40144001580881e-05	\\
-2457.72150213068	3.4372636350279e-05	\\
-2456.74272017045	3.52597492903651e-05	\\
-2455.76393821023	3.22604685942023e-05	\\
-2454.78515625	3.31637199462611e-05	\\
-2453.80637428977	3.08258600130511e-05	\\
-2452.82759232955	3.55518745883662e-05	\\
-2451.84881036932	3.21738425589679e-05	\\
-2450.87002840909	3.19468600051208e-05	\\
-2449.89124644886	3.26251403428665e-05	\\
-2448.91246448864	3.65323109203318e-05	\\
-2447.93368252841	3.30843156285024e-05	\\
-2446.95490056818	3.69496660903378e-05	\\
-2445.97611860795	3.38588582069362e-05	\\
-2444.99733664773	3.49722517339871e-05	\\
-2444.0185546875	3.56673024580862e-05	\\
-2443.03977272727	3.54508621614431e-05	\\
-2442.06099076705	3.49018302145757e-05	\\
-2441.08220880682	3.62041515430332e-05	\\
-2440.10342684659	3.56411627079402e-05	\\
-2439.12464488636	3.18104397331512e-05	\\
-2438.14586292614	3.50232757188302e-05	\\
-2437.16708096591	3.47163296590247e-05	\\
-2436.18829900568	3.53625960113351e-05	\\
-2435.20951704545	3.12423437252828e-05	\\
-2434.23073508523	3.33420184118105e-05	\\
-2433.251953125	3.73388803603807e-05	\\
-2432.27317116477	3.67793689034032e-05	\\
-2431.29438920455	3.59710736952565e-05	\\
-2430.31560724432	3.62235799546294e-05	\\
-2429.33682528409	3.8180782659714e-05	\\
-2428.35804332386	3.73605952680397e-05	\\
-2427.37926136364	3.55649544731469e-05	\\
-2426.40047940341	3.19360460312752e-05	\\
-2425.42169744318	3.38947656095298e-05	\\
-2424.44291548295	3.64566380361821e-05	\\
-2423.46413352273	3.3295725565957e-05	\\
-2422.4853515625	3.26751799521101e-05	\\
-2421.50656960227	3.46752727280727e-05	\\
-2420.52778764205	3.82119905104643e-05	\\
-2419.54900568182	3.52750667194167e-05	\\
-2418.57022372159	3.279573821521e-05	\\
-2417.59144176136	3.24615024463098e-05	\\
-2416.61265980114	3.31106123955289e-05	\\
-2415.63387784091	3.67359526795406e-05	\\
-2414.65509588068	3.44539746569922e-05	\\
-2413.67631392045	3.61485832076398e-05	\\
-2412.69753196023	3.32592003686008e-05	\\
-2411.71875	3.36300284303144e-05	\\
-2410.73996803977	3.88080289883775e-05	\\
-2409.76118607955	3.52347678040514e-05	\\
-2408.78240411932	3.49145301270616e-05	\\
-2407.80362215909	3.62802292958243e-05	\\
-2406.82484019886	3.28785935294017e-05	\\
-2405.84605823864	3.53742570064402e-05	\\
-2404.86727627841	3.66841723732158e-05	\\
-2403.88849431818	3.76285591227474e-05	\\
-2402.90971235795	3.54233631855874e-05	\\
-2401.93093039773	3.56192800602889e-05	\\
-2400.9521484375	3.46195748203281e-05	\\
-2399.97336647727	3.69771558832445e-05	\\
-2398.99458451705	3.40365084736552e-05	\\
-2398.01580255682	3.33428507451887e-05	\\
-2397.03702059659	3.38998857107721e-05	\\
-2396.05823863636	3.54817723210152e-05	\\
-2395.07945667614	3.42795961316019e-05	\\
-2394.10067471591	3.838414839212e-05	\\
-2393.12189275568	3.49818714203217e-05	\\
-2392.14311079545	3.53002681348643e-05	\\
-2391.16432883523	3.50947063057155e-05	\\
-2390.185546875	3.61949294518884e-05	\\
-2389.20676491477	3.75226874889335e-05	\\
-2388.22798295455	3.8087538783604e-05	\\
-2387.24920099432	4.00466556461509e-05	\\
-2386.27041903409	4.09318803911507e-05	\\
-2385.29163707386	3.79133914296867e-05	\\
-2384.31285511364	3.56180074752992e-05	\\
-2383.33407315341	3.80599055738151e-05	\\
-2382.35529119318	3.6939644177041e-05	\\
-2381.37650923295	3.92773651454703e-05	\\
-2380.39772727273	3.84879955632507e-05	\\
-2379.4189453125	3.70398602234748e-05	\\
-2378.44016335227	3.67652261601878e-05	\\
-2377.46138139205	3.77389222426152e-05	\\
-2376.48259943182	3.66212363192233e-05	\\
-2375.50381747159	3.96315815586641e-05	\\
-2374.52503551136	4.0420173069877e-05	\\
-2373.54625355114	3.76368295527051e-05	\\
-2372.56747159091	3.45508204723533e-05	\\
-2371.58868963068	4.15768830907451e-05	\\
-2370.60990767045	3.66037126304705e-05	\\
-2369.63112571023	3.73289982170374e-05	\\
-2368.65234375	4.0124190003563e-05	\\
-2367.67356178977	4.08008515748499e-05	\\
-2366.69477982955	3.86886460333673e-05	\\
-2365.71599786932	3.75395171879081e-05	\\
-2364.73721590909	3.85288066413634e-05	\\
-2363.75843394886	4.34313353672442e-05	\\
-2362.77965198864	4.4403376851365e-05	\\
-2361.80087002841	3.66405446804211e-05	\\
-2360.82208806818	3.77851486848144e-05	\\
-2359.84330610795	4.45516875262124e-05	\\
-2358.86452414773	4.38022291890418e-05	\\
-2357.8857421875	4.21725290450767e-05	\\
-2356.90696022727	3.71045419791081e-05	\\
-2355.92817826705	4.07339659568597e-05	\\
-2354.94939630682	4.08459217195228e-05	\\
-2353.97061434659	4.08035012525228e-05	\\
-2352.99183238636	4.14611541294044e-05	\\
-2352.01305042614	4.04478312581714e-05	\\
-2351.03426846591	4.1376876638517e-05	\\
-2350.05548650568	3.74025628015646e-05	\\
-2349.07670454545	4.03650112457786e-05	\\
-2348.09792258523	3.91434321573833e-05	\\
-2347.119140625	4.16737975676661e-05	\\
-2346.14035866477	3.98391743599582e-05	\\
-2345.16157670455	3.84147447605877e-05	\\
-2344.18279474432	3.82556287392553e-05	\\
-2343.20401278409	4.13470721524656e-05	\\
-2342.22523082386	3.77189842685547e-05	\\
-2341.24644886364	4.13725378488857e-05	\\
-2340.26766690341	4.42854635505754e-05	\\
-2339.28888494318	4.10943182273212e-05	\\
-2338.31010298295	4.59658385949722e-05	\\
-2337.33132102273	4.23784877580783e-05	\\
-2336.3525390625	4.16419087288869e-05	\\
-2335.37375710227	3.8646293981072e-05	\\
-2334.39497514205	4.01394637609116e-05	\\
-2333.41619318182	4.44758178260021e-05	\\
-2332.43741122159	4.02412454227842e-05	\\
-2331.45862926136	4.22780311704674e-05	\\
-2330.47984730114	4.01196478328427e-05	\\
-2329.50106534091	4.33077929632842e-05	\\
-2328.52228338068	3.89542333528138e-05	\\
-2327.54350142045	4.33571309489464e-05	\\
-2326.56471946023	4.21311391527949e-05	\\
-2325.5859375	3.97093451805517e-05	\\
-2324.60715553977	4.41680443825302e-05	\\
-2323.62837357955	4.47509746756291e-05	\\
-2322.64959161932	4.20899060834961e-05	\\
-2321.67080965909	3.91104483214459e-05	\\
-2320.69202769886	4.2121360553302e-05	\\
-2319.71324573864	4.42268032492523e-05	\\
-2318.73446377841	4.42466890942763e-05	\\
-2317.75568181818	4.1975483705692e-05	\\
-2316.77689985795	4.30074101769248e-05	\\
-2315.79811789773	4.36963948750771e-05	\\
-2314.8193359375	4.2491914898498e-05	\\
-2313.84055397727	3.98763231866166e-05	\\
-2312.86177201705	4.09859326731342e-05	\\
-2311.88299005682	3.93189577036531e-05	\\
-2310.90420809659	4.61703708740261e-05	\\
-2309.92542613636	4.54235200812785e-05	\\
-2308.94664417614	4.2455336936605e-05	\\
-2307.96786221591	4.2201192299901e-05	\\
-2306.98908025568	4.42281480352729e-05	\\
-2306.01029829545	4.47480935248064e-05	\\
-2305.03151633523	4.2637975224156e-05	\\
-2304.052734375	4.09711944903239e-05	\\
-2303.07395241477	4.42277238119126e-05	\\
-2302.09517045455	4.38810145743407e-05	\\
-2301.11638849432	4.23737093210821e-05	\\
-2300.13760653409	4.41097516562877e-05	\\
-2299.15882457386	4.24559671621872e-05	\\
-2298.18004261364	4.43550342872027e-05	\\
-2297.20126065341	4.03146944572552e-05	\\
-2296.22247869318	4.44133782569275e-05	\\
-2295.24369673295	4.3321344184931e-05	\\
-2294.26491477273	4.34261755103306e-05	\\
-2293.2861328125	4.13831550452349e-05	\\
-2292.30735085227	4.04183261518104e-05	\\
-2291.32856889205	3.99844364082718e-05	\\
-2290.34978693182	4.48721494385149e-05	\\
-2289.37100497159	4.32356951361937e-05	\\
-2288.39222301136	4.24949100924144e-05	\\
-2287.41344105114	4.25818933294535e-05	\\
-2286.43465909091	4.01070336677292e-05	\\
-2285.45587713068	4.13253313977231e-05	\\
-2284.47709517045	4.22280537460472e-05	\\
-2283.49831321023	4.22172451116417e-05	\\
-2282.51953125	4.0382651471049e-05	\\
-2281.54074928977	4.34482400146214e-05	\\
-2280.56196732955	4.10683695332873e-05	\\
-2279.58318536932	3.96321755026458e-05	\\
-2278.60440340909	4.36770276691326e-05	\\
-2277.62562144886	4.30905452456153e-05	\\
-2276.64683948864	4.29536895723665e-05	\\
-2275.66805752841	4.09830505075139e-05	\\
-2274.68927556818	4.0594243530569e-05	\\
-2273.71049360795	3.98451934365557e-05	\\
-2272.73171164773	4.22507421517922e-05	\\
-2271.7529296875	4.14068725887413e-05	\\
-2270.77414772727	4.13586244398889e-05	\\
-2269.79536576705	3.74148895067546e-05	\\
-2268.81658380682	3.9834458407795e-05	\\
-2267.83780184659	3.99687669177714e-05	\\
-2266.85901988636	4.27593507030579e-05	\\
-2265.88023792614	4.05501685889252e-05	\\
-2264.90145596591	4.22876935358686e-05	\\
-2263.92267400568	3.86936956303814e-05	\\
-2262.94389204545	4.17127003896498e-05	\\
-2261.96511008523	3.95393655811713e-05	\\
-2260.986328125	3.83359504396517e-05	\\
-2260.00754616477	4.06844079850499e-05	\\
-2259.02876420455	3.99458449444316e-05	\\
-2258.04998224432	4.04069085141025e-05	\\
-2257.07120028409	4.1431596680544e-05	\\
-2256.09241832386	4.00390749818585e-05	\\
-2255.11363636364	3.53526841131004e-05	\\
-2254.13485440341	3.87875475424754e-05	\\
-2253.15607244318	3.80618261604687e-05	\\
-2252.17729048295	3.8249390014748e-05	\\
-2251.19850852273	3.82713051110705e-05	\\
-2250.2197265625	4.03246394299573e-05	\\
-2249.24094460227	3.98819432976716e-05	\\
-2248.26216264205	3.91366144945425e-05	\\
-2247.28338068182	3.90208876215411e-05	\\
-2246.30459872159	3.39984838923706e-05	\\
-2245.32581676136	3.74181277467878e-05	\\
-2244.34703480114	3.92012805823409e-05	\\
-2243.36825284091	3.99491634262686e-05	\\
-2242.38947088068	3.75117307263353e-05	\\
-2241.41068892045	3.61895706048588e-05	\\
-2240.43190696023	3.67817511125022e-05	\\
-2239.453125	3.78116139094702e-05	\\
-2238.47434303977	3.81254721084168e-05	\\
-2237.49556107955	3.93855082157725e-05	\\
-2236.51677911932	3.60388659770186e-05	\\
-2235.53799715909	3.8962072961599e-05	\\
-2234.55921519886	3.69419975919397e-05	\\
-2233.58043323864	3.67463222964136e-05	\\
-2232.60165127841	3.6290336199729e-05	\\
-2231.62286931818	3.7689162478252e-05	\\
-2230.64408735795	3.73590363781108e-05	\\
-2229.66530539773	4.00650512721277e-05	\\
-2228.6865234375	3.75402419545454e-05	\\
-2227.70774147727	4.00526433062529e-05	\\
-2226.72895951705	3.62093560461636e-05	\\
-2225.75017755682	3.81472473978971e-05	\\
-2224.77139559659	3.7289716237892e-05	\\
-2223.79261363636	3.91929636706829e-05	\\
-2222.81383167614	3.54114069807791e-05	\\
-2221.83504971591	3.71664724269401e-05	\\
-2220.85626775568	3.98631115683516e-05	\\
-2219.87748579545	3.88680418643614e-05	\\
-2218.89870383523	3.74278517220395e-05	\\
-2217.919921875	3.68923257181398e-05	\\
-2216.94113991477	3.62348412276371e-05	\\
-2215.96235795455	3.88670249235125e-05	\\
-2214.98357599432	3.68726230512169e-05	\\
-2214.00479403409	4.17075035255965e-05	\\
-2213.02601207386	3.46023167414842e-05	\\
-2212.04723011364	4.14523366106673e-05	\\
-2211.06844815341	3.80276562529982e-05	\\
-2210.08966619318	3.63922641826674e-05	\\
-2209.11088423295	4.07272663966033e-05	\\
-2208.13210227273	3.8681364744335e-05	\\
-2207.1533203125	3.68821363359198e-05	\\
-2206.17453835227	4.16545004755364e-05	\\
-2205.19575639205	3.90047465555597e-05	\\
-2204.21697443182	3.95080501084159e-05	\\
-2203.23819247159	3.71904880867608e-05	\\
-2202.25941051136	3.88617362345523e-05	\\
-2201.28062855114	3.71113503704911e-05	\\
-2200.30184659091	3.7984710470771e-05	\\
-2199.32306463068	3.80403295714478e-05	\\
-2198.34428267045	3.96849904562913e-05	\\
-2197.36550071023	3.7823827173766e-05	\\
-2196.38671875	3.75203881680602e-05	\\
-2195.40793678977	4.11160833395993e-05	\\
-2194.42915482955	3.99091066889672e-05	\\
-2193.45037286932	3.78344651601823e-05	\\
-2192.47159090909	4.20404323205583e-05	\\
-2191.49280894886	3.89307387839001e-05	\\
-2190.51402698864	3.84806189437007e-05	\\
-2189.53524502841	4.01366847624898e-05	\\
-2188.55646306818	4.05236415850614e-05	\\
-2187.57768110795	3.94074527626326e-05	\\
-2186.59889914773	3.75217477226414e-05	\\
-2185.6201171875	3.98513966160201e-05	\\
-2184.64133522727	3.84897528354467e-05	\\
-2183.66255326705	4.18860398846851e-05	\\
-2182.68377130682	3.96080062440808e-05	\\
-2181.70498934659	4.23627015421186e-05	\\
-2180.72620738636	3.81155112185085e-05	\\
-2179.74742542614	3.90603077078313e-05	\\
-2178.76864346591	3.79354930330172e-05	\\
-2177.78986150568	4.01047329936582e-05	\\
-2176.81107954545	4.10701852277502e-05	\\
-2175.83229758523	4.16622316709541e-05	\\
-2174.853515625	4.13471743533938e-05	\\
-2173.87473366477	4.14838445743737e-05	\\
-2172.89595170455	3.84187820517969e-05	\\
-2171.91716974432	4.45715138547893e-05	\\
-2170.93838778409	4.12322678575292e-05	\\
-2169.95960582386	4.22805841274327e-05	\\
-2168.98082386364	4.10585583487203e-05	\\
-2168.00204190341	4.01725954077288e-05	\\
-2167.02325994318	4.13769372864272e-05	\\
-2166.04447798295	4.30000633502915e-05	\\
-2165.06569602273	3.97043965640063e-05	\\
-2164.0869140625	4.54076410626733e-05	\\
-2163.10813210227	4.10934923261606e-05	\\
-2162.12935014205	4.12353736097467e-05	\\
-2161.15056818182	4.05390869249973e-05	\\
-2160.17178622159	4.05443873076942e-05	\\
-2159.19300426136	3.90601153131864e-05	\\
-2158.21422230114	4.37193618713594e-05	\\
-2157.23544034091	3.9968923232183e-05	\\
-2156.25665838068	4.38569125564729e-05	\\
-2155.27787642045	4.28693605914596e-05	\\
-2154.29909446023	4.11824856901105e-05	\\
-2153.3203125	4.45114491332108e-05	\\
-2152.34153053977	4.54302484459923e-05	\\
-2151.36274857955	4.4322649151619e-05	\\
-2150.38396661932	4.27614430721946e-05	\\
-2149.40518465909	4.45683200664184e-05	\\
-2148.42640269886	4.26676530103327e-05	\\
-2147.44762073864	4.63558271751998e-05	\\
-2146.46883877841	4.57738804952212e-05	\\
-2145.49005681818	4.2267639712788e-05	\\
-2144.51127485795	4.27050803971275e-05	\\
-2143.53249289773	4.4422299323682e-05	\\
-2142.5537109375	4.2067788403555e-05	\\
-2141.57492897727	4.40325115562613e-05	\\
-2140.59614701705	4.3892593629549e-05	\\
-2139.61736505682	4.15040820223293e-05	\\
-2138.63858309659	4.13880402559019e-05	\\
-2137.65980113636	4.68719534415898e-05	\\
-2136.68101917614	4.57296551872686e-05	\\
-2135.70223721591	4.56244248763674e-05	\\
-2134.72345525568	4.49752648078924e-05	\\
-2133.74467329545	4.40467631993037e-05	\\
-2132.76589133523	4.75433288985883e-05	\\
-2131.787109375	4.66013443936659e-05	\\
-2130.80832741477	4.5652333627324e-05	\\
-2129.82954545455	4.83607688883001e-05	\\
-2128.85076349432	4.35345946775957e-05	\\
-2127.87198153409	4.59384075409383e-05	\\
-2126.89319957386	4.55616785039408e-05	\\
-2125.91441761364	4.38975346700251e-05	\\
-2124.93563565341	4.25338382736516e-05	\\
-2123.95685369318	4.28589359187242e-05	\\
-2122.97807173295	4.55274508369932e-05	\\
-2121.99928977273	4.48908796973114e-05	\\
-2121.0205078125	4.1735686456902e-05	\\
-2120.04172585227	4.25541003427858e-05	\\
-2119.06294389205	4.63893333250215e-05	\\
-2118.08416193182	4.64981894966271e-05	\\
-2117.10537997159	4.63088540136544e-05	\\
-2116.12659801136	5.03354282666185e-05	\\
-2115.14781605114	4.46295438999884e-05	\\
-2114.16903409091	4.67745030629035e-05	\\
-2113.19025213068	4.38607093863149e-05	\\
-2112.21147017045	4.30811739293004e-05	\\
-2111.23268821023	4.4346877909237e-05	\\
-2110.25390625	4.53296453553866e-05	\\
-2109.27512428977	4.73054937042176e-05	\\
-2108.29634232955	4.56285711777607e-05	\\
-2107.31756036932	4.62902550078992e-05	\\
-2106.33877840909	4.29105488460104e-05	\\
-2105.35999644886	4.78220772972167e-05	\\
-2104.38121448864	4.57930998798147e-05	\\
-2103.40243252841	4.45158021459453e-05	\\
-2102.42365056818	4.33184803225706e-05	\\
-2101.44486860795	4.6941631787116e-05	\\
-2100.46608664773	4.81048251581092e-05	\\
-2099.4873046875	4.65587860867088e-05	\\
-2098.50852272727	4.47716393020923e-05	\\
-2097.52974076705	4.48147985680148e-05	\\
-2096.55095880682	4.59110628149754e-05	\\
-2095.57217684659	4.46255644363266e-05	\\
-2094.59339488636	4.75116767665358e-05	\\
-2093.61461292614	4.41960986236894e-05	\\
-2092.63583096591	4.37311675914129e-05	\\
-2091.65704900568	4.53743715903457e-05	\\
-2090.67826704545	4.42760113965598e-05	\\
-2089.69948508523	4.34587932530007e-05	\\
-2088.720703125	4.52583450993066e-05	\\
-2087.74192116477	4.10267968876772e-05	\\
-2086.76313920455	4.03728148109159e-05	\\
-2085.78435724432	4.49716228477513e-05	\\
-2084.80557528409	4.16824565607019e-05	\\
-2083.82679332386	4.74935669810165e-05	\\
-2082.84801136364	4.58941147901089e-05	\\
-2081.86922940341	4.10161885137469e-05	\\
-2080.89044744318	4.56607863441921e-05	\\
-2079.91166548295	4.63227589351866e-05	\\
-2078.93288352273	4.12443924696675e-05	\\
-2077.9541015625	4.25125149163331e-05	\\
-2076.97531960227	4.58105020952118e-05	\\
-2075.99653764205	4.3123034597401e-05	\\
-2075.01775568182	4.61082223539162e-05	\\
-2074.03897372159	4.22949480477163e-05	\\
-2073.06019176136	4.1673528750919e-05	\\
-2072.08140980114	4.58461771770281e-05	\\
-2071.10262784091	4.17110749268099e-05	\\
-2070.12384588068	4.23941771838257e-05	\\
-2069.14506392045	4.24272832627179e-05	\\
-2068.16628196023	4.25164487427647e-05	\\
-2067.1875	4.09062310137809e-05	\\
-2066.20871803977	4.31783231171223e-05	\\
-2065.22993607955	3.97605451323997e-05	\\
-2064.25115411932	4.07350421569936e-05	\\
-2063.27237215909	3.85999862659472e-05	\\
-2062.29359019886	3.89521685734964e-05	\\
-2061.31480823864	4.17656665330426e-05	\\
-2060.33602627841	3.80205272902788e-05	\\
-2059.35724431818	4.07406615114303e-05	\\
-2058.37846235795	3.90133104629514e-05	\\
-2057.39968039773	4.11027115188376e-05	\\
-2056.4208984375	3.89526013481517e-05	\\
-2055.44211647727	3.88784203533842e-05	\\
-2054.46333451705	3.4667675695049e-05	\\
-2053.48455255682	4.09930403647184e-05	\\
-2052.50577059659	3.68764500282288e-05	\\
-2051.52698863636	3.84921931681034e-05	\\
-2050.54820667614	3.46586762708203e-05	\\
-2049.56942471591	3.85180133547127e-05	\\
-2048.59064275568	3.80041458486795e-05	\\
-2047.61186079545	3.47347486177089e-05	\\
-2046.63307883523	3.29324486648671e-05	\\
-2045.654296875	3.28373359565008e-05	\\
-2044.67551491477	3.58606871092792e-05	\\
-2043.69673295455	3.17596813723438e-05	\\
-2042.71795099432	3.49141559558745e-05	\\
-2041.73916903409	3.02832213151942e-05	\\
-2040.76038707386	3.68466187544548e-05	\\
-2039.78160511364	3.32938281089078e-05	\\
-2038.80282315341	3.05935869623907e-05	\\
-2037.82404119318	3.06283963577057e-05	\\
-2036.84525923295	2.99429137307778e-05	\\
-2035.86647727273	2.90551789960893e-05	\\
-2034.8876953125	3.64530949104441e-05	\\
-2033.90891335227	3.04661005407207e-05	\\
-2032.93013139205	2.57847013443228e-05	\\
-2031.95134943182	2.96266855210577e-05	\\
-2030.97256747159	2.42211680380562e-05	\\
-2029.99378551136	2.78612940028551e-05	\\
-2029.01500355114	2.7584492911238e-05	\\
-2028.03622159091	2.61658209822513e-05	\\
-2027.05743963068	2.54016927122705e-05	\\
-2026.07865767045	2.44220440493527e-05	\\
-2025.09987571023	2.08035248376758e-05	\\
-2024.12109375	2.08495933712796e-05	\\
-2023.14231178977	1.87831118490923e-05	\\
-2022.16352982955	1.81084581151927e-05	\\
-2021.18474786932	1.74099471008649e-05	\\
-2020.20596590909	1.44539032586076e-05	\\
-2019.22718394886	1.11581703323986e-05	\\
-2018.24840198864	1.31689124197014e-05	\\
-2017.26962002841	9.0506861901178e-06	\\
-2016.29083806818	8.75094765722526e-06	\\
-2015.31205610795	7.64169863180601e-06	\\
-2014.33327414773	5.14888247973786e-06	\\
-2013.3544921875	6.99630063720067e-06	\\
-2012.37571022727	5.19508502038622e-06	\\
-2011.39692826705	6.54294044037294e-06	\\
-2010.41814630682	5.27184423898744e-06	\\
-2009.43936434659	3.69930079920909e-06	\\
-2008.46058238636	6.21654873652613e-06	\\
-2007.48180042614	1.06917134519788e-05	\\
-2006.50301846591	1.04233276930869e-05	\\
-2005.52423650568	1.04682360444182e-05	\\
-2004.54545454545	1.5149431559193e-05	\\
-2003.56667258523	1.92123428537741e-05	\\
-2002.587890625	1.74403176096298e-05	\\
-2001.60910866477	2.08781186344482e-05	\\
-2000.63032670455	2.28870761016013e-05	\\
-1999.65154474432	2.51487938910366e-05	\\
-1998.67276278409	1.93103467854148e-05	\\
-1997.69398082386	2.7218674997871e-05	\\
-1996.71519886364	3.16190677014573e-05	\\
-1995.73641690341	3.28769036011884e-05	\\
-1994.75763494318	3.56223183065848e-05	\\
-1993.77885298295	3.94164049345877e-05	\\
-1992.80007102273	3.95916257067998e-05	\\
-1991.8212890625	4.50011965976149e-05	\\
-1990.84250710227	4.49296320281333e-05	\\
-1989.86372514205	5.0618936660624e-05	\\
-1988.88494318182	5.20939489206574e-05	\\
-1987.90616122159	5.59279585238717e-05	\\
-1986.92737926136	5.61304999468018e-05	\\
-1985.94859730114	5.82044529263225e-05	\\
-1984.96981534091	5.9376780490924e-05	\\
-1983.99103338068	6.21920002921307e-05	\\
-1983.01225142045	6.16215567979818e-05	\\
-1982.03346946023	6.92599182604714e-05	\\
-1981.0546875	7.09813781356162e-05	\\
-1980.07590553977	6.82879577809948e-05	\\
-1979.09712357955	7.46147110634756e-05	\\
-1978.11834161932	7.44569288698151e-05	\\
-1977.13955965909	7.99260118231738e-05	\\
-1976.16077769886	8.31049073776718e-05	\\
-1975.18199573864	8.35816387238715e-05	\\
-1974.20321377841	8.95726625715138e-05	\\
-1973.22443181818	9.17902208695251e-05	\\
-1972.24564985795	9.64103136522388e-05	\\
-1971.26686789773	9.65719298173626e-05	\\
-1970.2880859375	0.000102707714971143	\\
-1969.30930397727	0.000103960456583585	\\
-1968.33052201705	0.000101394860802496	\\
-1967.35174005682	0.000104749427491777	\\
-1966.37295809659	0.000111594054320631	\\
-1965.39417613636	0.000113948164038818	\\
-1964.41539417614	0.000116298507083717	\\
-1963.43661221591	0.000121476945689561	\\
-1962.45783025568	0.000123266529137007	\\
-1961.47904829545	0.000126816029300493	\\
-1960.50026633523	0.000126370986558272	\\
-1959.521484375	0.000130495278260394	\\
-1958.54270241477	0.000133074015490224	\\
-1957.56392045455	0.000139524754395204	\\
-1956.58513849432	0.000138465005990341	\\
-1955.60635653409	0.000140289515700449	\\
-1954.62757457386	0.00014951423127497	\\
-1953.64879261364	0.000149807528791539	\\
-1952.67001065341	0.000150462591931463	\\
-1951.69122869318	0.000154819790239631	\\
-1950.71244673295	0.000158574175692318	\\
-1949.73366477273	0.000160870259741115	\\
-1948.7548828125	0.00016172453506016	\\
-1947.77610085227	0.000167541979167074	\\
-1946.79731889205	0.000173830536461948	\\
-1945.81853693182	0.000177606371041097	\\
-1944.83975497159	0.000176096586661828	\\
-1943.86097301136	0.000179527896643211	\\
-1942.88219105114	0.000187200627134809	\\
-1941.90340909091	0.000186554775010758	\\
-1940.92462713068	0.000187652897909247	\\
-1939.94584517045	0.00019126575430914	\\
-1938.96706321023	0.000198564524330994	\\
-1937.98828125	0.000197092340801801	\\
-1937.00949928977	0.000205003039295772	\\
-1936.03071732955	0.00021103362822989	\\
-1935.05193536932	0.000210347113172151	\\
-1934.07315340909	0.000211726323749391	\\
-1933.09437144886	0.000218268608602501	\\
-1932.11558948864	0.000223140331764476	\\
-1931.13680752841	0.000226044908372449	\\
-1930.15802556818	0.000226379565177529	\\
-1929.17924360795	0.00022893712186055	\\
-1928.20046164773	0.00022866512489373	\\
-1927.2216796875	0.000233197409547999	\\
-1926.24289772727	0.000242298464739942	\\
-1925.26411576705	0.000241384616933135	\\
-1924.28533380682	0.000243908400878493	\\
-1923.30655184659	0.000245162522425086	\\
-1922.32776988636	0.000250589115891391	\\
-1921.34898792614	0.000252873660602047	\\
-1920.37020596591	0.000255808400656811	\\
-1919.39142400568	0.000254988728139077	\\
-1918.41264204545	0.000258229463197138	\\
-1917.43386008523	0.000261691362212589	\\
-1916.455078125	0.000267960117447649	\\
-1915.47629616477	0.000269330629987006	\\
-1914.49751420455	0.000269445938196846	\\
-1913.51873224432	0.000272695998949203	\\
-1912.53995028409	0.000277075835111636	\\
-1911.56116832386	0.000274332180770315	\\
-1910.58238636364	0.000273043711696756	\\
-1909.60360440341	0.00028325579999696	\\
-1908.62482244318	0.000287609416164273	\\
-1907.64604048295	0.000284546235926162	\\
-1906.66725852273	0.000289419772890747	\\
-1905.6884765625	0.000294470288424788	\\
-1904.70969460227	0.000301063973755327	\\
-1903.73091264205	0.000298025806026721	\\
-1902.75213068182	0.000301024406395621	\\
-1901.77334872159	0.000303634997553136	\\
-1900.79456676136	0.000305596683324454	\\
-1899.81578480114	0.000304159092734534	\\
-1898.83700284091	0.000308504097550723	\\
-1897.85822088068	0.000312227477501777	\\
-1896.87943892045	0.000313577924975102	\\
-1895.90065696023	0.000314688795776001	\\
-1894.921875	0.000313751826310608	\\
-1893.94309303977	0.000320908846678231	\\
-1892.96431107955	0.000319735333007497	\\
-1891.98552911932	0.000322349063752288	\\
-1891.00674715909	0.000320854302388855	\\
-1890.02796519886	0.000325274602997532	\\
-1889.04918323864	0.000322813185277696	\\
-1888.07040127841	0.000329783311385037	\\
-1887.09161931818	0.000330018087709634	\\
-1886.11283735795	0.000334284341766601	\\
-1885.13405539773	0.000339024471225563	\\
-1884.1552734375	0.000336225659959507	\\
-1883.17649147727	0.000340067610318941	\\
-1882.19770951705	0.000338947513118915	\\
-1881.21892755682	0.000338494361048044	\\
-1880.24014559659	0.000340602578716757	\\
-1879.26136363636	0.000341314277115351	\\
-1878.28258167614	0.000345612503018002	\\
-1877.30379971591	0.000346315568668048	\\
-1876.32501775568	0.000347575988117473	\\
-1875.34623579545	0.000347688657107931	\\
-1874.36745383523	0.000344075130188303	\\
-1873.388671875	0.0003442287491678	\\
-1872.40988991477	0.000344006105781949	\\
-1871.43110795455	0.000341911340018586	\\
-1870.45232599432	0.000348223385161989	\\
-1869.47354403409	0.00034870164693069	\\
-1868.49476207386	0.000349606772943809	\\
-1867.51598011364	0.000351044143424854	\\
-1866.53719815341	0.00035343243952732	\\
-1865.55841619318	0.00035301169368135	\\
-1864.57963423295	0.000357499801576864	\\
-1863.60085227273	0.000360301170690631	\\
-1862.6220703125	0.000357797154995798	\\
-1861.64328835227	0.000359850424992643	\\
-1860.66450639205	0.000357579353731034	\\
-1859.68572443182	0.000366900044367508	\\
-1858.70694247159	0.000370107443104972	\\
-1857.72816051136	0.000370468906823479	\\
-1856.74937855114	0.000372739354102266	\\
-1855.77059659091	0.000377202800153288	\\
-1854.79181463068	0.000380719091086814	\\
-1853.81303267045	0.000383957495528657	\\
-1852.83425071023	0.000382179654030131	\\
-1851.85546875	0.000386544248980247	\\
-1850.87668678977	0.000392101266044215	\\
-1849.89790482955	0.000392246721016914	\\
-1848.91912286932	0.000393662983915964	\\
-1847.94034090909	0.000396668358453259	\\
-1846.96155894886	0.000397909957438221	\\
-1845.98277698864	0.000404196449701007	\\
-1845.00399502841	0.000401560248792356	\\
-1844.02521306818	0.00040216453492324	\\
-1843.04643110795	0.000409629893664215	\\
-1842.06764914773	0.000407950620843069	\\
-1841.0888671875	0.00041431856290398	\\
-1840.11008522727	0.000416729734923823	\\
-1839.13130326705	0.000415910757225383	\\
-1838.15252130682	0.000419202070784375	\\
-1837.17373934659	0.000417527898492894	\\
-1836.19495738636	0.000425435750483026	\\
-1835.21617542614	0.000430448992304758	\\
-1834.23739346591	0.000426478903227998	\\
-1833.25861150568	0.000433161709552025	\\
-1832.27982954545	0.000436077361945966	\\
-1831.30104758523	0.000439708387441764	\\
-1830.322265625	0.000444181096280731	\\
-1829.34348366477	0.000441056467907933	\\
-1828.36470170455	0.000447831179922502	\\
-1827.38591974432	0.000454066242198313	\\
-1826.40713778409	0.000454942975153657	\\
-1825.42835582386	0.000460575810896779	\\
-1824.44957386364	0.0004640002059863	\\
-1823.47079190341	0.000464644687556032	\\
-1822.49200994318	0.000469027452096033	\\
-1821.51322798295	0.000473795079276656	\\
-1820.53444602273	0.000472122035665012	\\
-1819.5556640625	0.000479167341818438	\\
-1818.57688210227	0.000478694650573023	\\
-1817.59810014205	0.000485498249126857	\\
-1816.61931818182	0.000485985652186629	\\
-1815.64053622159	0.000489517981087009	\\
-1814.66175426136	0.000487738847149405	\\
-1813.68297230114	0.000480323604392023	\\
-1812.70419034091	0.000491705471638404	\\
-1811.72540838068	0.000491329896419524	\\
-1810.74662642045	0.000490158962967859	\\
-1809.76784446023	0.000499172905687967	\\
-1808.7890625	0.000494422133704232	\\
-1807.81028053977	0.000501744272391781	\\
-1806.83149857955	0.000505135646870643	\\
-1805.85271661932	0.000502850855873495	\\
-1804.87393465909	0.000510133603211201	\\
-1803.89515269886	0.000509858426594418	\\
-1802.91637073864	0.00051062812487803	\\
-1801.93758877841	0.000510050666944359	\\
-1800.95880681818	0.000512338674166331	\\
-1799.98002485795	0.000518828551895239	\\
-1799.00124289773	0.000519192220190028	\\
-1798.0224609375	0.000519293431808208	\\
-1797.04367897727	0.000523970655381201	\\
-1796.06489701705	0.000521344956511302	\\
-1795.08611505682	0.000526685956046625	\\
-1794.10733309659	0.000528594953384224	\\
-1793.12855113636	0.000529268542353878	\\
-1792.14976917614	0.000533760866512239	\\
-1791.17098721591	0.000533831459621175	\\
-1790.19220525568	0.000533704170281513	\\
-1789.21342329545	0.000536511602300891	\\
-1788.23464133523	0.00053624562733977	\\
-1787.255859375	0.000536702538208114	\\
-1786.27707741477	0.000537703175066844	\\
-1785.29829545455	0.000534377955803003	\\
-1784.31951349432	0.000538502870895218	\\
-1783.34073153409	0.000539336533955844	\\
-1782.36194957386	0.000541308615903312	\\
-1781.38316761364	0.000542663437709115	\\
-1780.40438565341	0.000544240376600658	\\
-1779.42560369318	0.000547257196208022	\\
-1778.44682173295	0.000546535500839788	\\
-1777.46803977273	0.000541336882474835	\\
-1776.4892578125	0.000546793752271042	\\
-1775.51047585227	0.000551443183641368	\\
-1774.53169389205	0.00055519694044645	\\
-1773.55291193182	0.000554635593885558	\\
-1772.57412997159	0.000554505844370661	\\
-1771.59534801136	0.000560244108158295	\\
-1770.61656605114	0.000557448695808287	\\
-1769.63778409091	0.000560215041375951	\\
-1768.65900213068	0.000561728304816505	\\
-1767.68022017045	0.000564354667369644	\\
-1766.70143821023	0.000567147783813101	\\
-1765.72265625	0.000567713038890776	\\
-1764.74387428977	0.000565724932130722	\\
-1763.76509232955	0.000571559561250173	\\
-1762.78631036932	0.000569535664944741	\\
-1761.80752840909	0.000570614371929593	\\
-1760.82874644886	0.000572885681225869	\\
-1759.84996448864	0.000571604402323157	\\
-1758.87118252841	0.000572178421750014	\\
-1757.89240056818	0.000573890658908948	\\
-1756.91361860795	0.000571905290344365	\\
-1755.93483664773	0.000579498022373472	\\
-1754.9560546875	0.000578121946444224	\\
-1753.97727272727	0.00057656583735483	\\
-1752.99849076705	0.000578661502705437	\\
-1752.01970880682	0.00057758615773402	\\
-1751.04092684659	0.000580306054044476	\\
-1750.06214488636	0.000584184756913481	\\
-1749.08336292614	0.000586090367081916	\\
-1748.10458096591	0.000579182736896883	\\
-1747.12579900568	0.000583275210953882	\\
-1746.14701704545	0.000581799600748843	\\
-1745.16823508523	0.000587262367555125	\\
-1744.189453125	0.000584518510584918	\\
-1743.21067116477	0.000583808724764426	\\
-1742.23188920455	0.000583580063395316	\\
-1741.25310724432	0.000588220355971448	\\
-1740.27432528409	0.000585714337729151	\\
-1739.29554332386	0.000589584866381618	\\
-1738.31676136364	0.000585494256857855	\\
-1737.33797940341	0.000585433725277173	\\
-1736.35919744318	0.000585629492428035	\\
-1735.38041548295	0.000582339225765997	\\
-1734.40163352273	0.000583619299712424	\\
-1733.4228515625	0.000577186635814033	\\
-1732.44406960227	0.000583130660074624	\\
-1731.46528764205	0.000584959254554408	\\
-1730.48650568182	0.000581453327752165	\\
-1729.50772372159	0.000584394148118052	\\
-1728.52894176136	0.000589892531720699	\\
-1727.55015980114	0.000588596888522002	\\
-1726.57137784091	0.000591070044050455	\\
-1725.59259588068	0.000596616737645682	\\
-1724.61381392045	0.000593592914512002	\\
-1723.63503196023	0.000596288274597268	\\
-1722.65625	0.000597802662356271	\\
-1721.67746803977	0.000600722205930348	\\
-1720.69868607955	0.000605358227677281	\\
-1719.71990411932	0.000605656002148306	\\
-1718.74112215909	0.000609132620463476	\\
-1717.76234019886	0.000609891036392875	\\
-1716.78355823864	0.000606676404966251	\\
-1715.80477627841	0.000610998913805304	\\
-1714.82599431818	0.000606734218127657	\\
-1713.84721235795	0.000609031777770082	\\
-1712.86843039773	0.000616199770722226	\\
-1711.8896484375	0.00061368829419967	\\
-1710.91086647727	0.00062152831372475	\\
-1709.93208451705	0.000627794713942724	\\
-1708.95330255682	0.000625219241319493	\\
-1707.97452059659	0.000626773663234063	\\
-1706.99573863636	0.000626456343226703	\\
-1706.01695667614	0.000620586587997614	\\
-1705.03817471591	0.000630033669124645	\\
-1704.05939275568	0.00062832014504044	\\
-1703.08061079545	0.000630365237636544	\\
-1702.10182883523	0.000634761518194081	\\
-1701.123046875	0.000633970540520204	\\
-1700.14426491477	0.000628195430951056	\\
-1699.16548295455	0.000629136438254021	\\
-1698.18670099432	0.000633074806981033	\\
-1697.20791903409	0.00063876463257986	\\
-1696.22913707386	0.000649241967363488	\\
-1695.25035511364	0.000648838452731498	\\
-1694.27157315341	0.000653575982854397	\\
-1693.29279119318	0.000660341529495992	\\
-1692.31400923295	0.000661230376903143	\\
-1691.33522727273	0.000662898847903121	\\
-1690.3564453125	0.000665214182054544	\\
-1689.37766335227	0.000670018898996225	\\
-1688.39888139205	0.000671450323614509	\\
-1687.42009943182	0.000670624973523453	\\
-1686.44131747159	0.000670950659886681	\\
-1685.46253551136	0.000675379981387754	\\
-1684.48375355114	0.000674778095655927	\\
-1683.50497159091	0.000672871354840892	\\
-1682.52618963068	0.000674427190699324	\\
-1681.54740767045	0.000671352350207489	\\
-1680.56862571023	0.000675119637235731	\\
-1679.58984375	0.000664690392621821	\\
-1678.61106178977	0.000674631776492834	\\
-1677.63227982955	0.000672283187243664	\\
-1676.65349786932	0.000671779136179109	\\
-1675.67471590909	0.000671580335013745	\\
-1674.69593394886	0.000675474543780318	\\
-1673.71715198864	0.000665495300683976	\\
-1672.73837002841	0.000669973799858834	\\
-1671.75958806818	0.000667636190033173	\\
-1670.78080610795	0.000660667723806901	\\
-1669.80202414773	0.000665524551330595	\\
-1668.8232421875	0.000660104874495851	\\
-1667.84446022727	0.000658286601883906	\\
-1666.86567826705	0.00065923332215048	\\
-1665.88689630682	0.000654407415598499	\\
-1664.90811434659	0.000654023670519504	\\
-1663.92933238636	0.000650939492417663	\\
-1662.95055042614	0.000649706239915703	\\
-1661.97176846591	0.000656259729351336	\\
-1660.99298650568	0.00064632085366451	\\
-1660.01420454545	0.000644351981883953	\\
-1659.03542258523	0.000643660462320559	\\
-1658.056640625	0.000643887459646833	\\
-1657.07785866477	0.000642534393031592	\\
-1656.09907670455	0.000641270181782412	\\
-1655.12029474432	0.000642229712869736	\\
-1654.14151278409	0.000634129787563707	\\
-1653.16273082386	0.000634684050045935	\\
-1652.18394886364	0.000632967555494033	\\
-1651.20516690341	0.000628783747352898	\\
-1650.22638494318	0.000619454820667979	\\
-1649.24760298295	0.000617930835170663	\\
-1648.26882102273	0.00061776498084666	\\
-1647.2900390625	0.000616467449849702	\\
-1646.31125710227	0.000610244210470536	\\
-1645.33247514205	0.000604288100065146	\\
-1644.35369318182	0.000603806229763558	\\
-1643.37491122159	0.000600975318189418	\\
-1642.39612926136	0.000595361013476345	\\
-1641.41734730114	0.000595301682229377	\\
-1640.43856534091	0.000593054844912966	\\
-1639.45978338068	0.000592175685702277	\\
-1638.48100142045	0.000586649503442429	\\
-1637.50221946023	0.000589139880844365	\\
-1636.5234375	0.000587495928191551	\\
-1635.54465553977	0.000587919852181399	\\
-1634.56587357955	0.000585542791089123	\\
-1633.58709161932	0.000579749866701612	\\
-1632.60830965909	0.000583036797521097	\\
-1631.62952769886	0.000578688613339141	\\
-1630.65074573864	0.000574434087330951	\\
-1629.67196377841	0.000575856472165377	\\
-1628.69318181818	0.000574297619659546	\\
-1627.71439985795	0.000574599410970509	\\
-1626.73561789773	0.00056991478899999	\\
-1625.7568359375	0.000561954466412596	\\
-1624.77805397727	0.000569746923896686	\\
-1623.79927201705	0.000571351149176994	\\
-1622.82049005682	0.000565755392706546	\\
-1621.84170809659	0.000572201145112737	\\
-1620.86292613636	0.00056880928634668	\\
-1619.88414417614	0.000574204913283541	\\
-1618.90536221591	0.000570760894328513	\\
-1617.92658025568	0.000570301098838708	\\
-1616.94779829545	0.000574240383470377	\\
-1615.96901633523	0.000578740259984784	\\
-1614.990234375	0.000580022990810944	\\
-1614.01145241477	0.000577914986739419	\\
-1613.03267045455	0.000582493767394898	\\
-1612.05388849432	0.000582391015003052	\\
-1611.07510653409	0.000591583276387177	\\
-1610.09632457386	0.000593483856157959	\\
-1609.11754261364	0.000595230488052676	\\
-1608.13876065341	0.000597481769905494	\\
-1607.15997869318	0.000605157914480917	\\
-1606.18119673295	0.00060479434776571	\\
-1605.20241477273	0.000610845129452451	\\
-1604.2236328125	0.000613981446364822	\\
-1603.24485085227	0.000622458634190472	\\
-1602.26606889205	0.000625029784824313	\\
-1601.28728693182	0.000629879351225097	\\
-1600.30850497159	0.000629027788571618	\\
-1599.32972301136	0.000634115739374592	\\
-1598.35094105114	0.000638181403265997	\\
-1597.37215909091	0.00064510224739289	\\
-1596.39337713068	0.00064380105685615	\\
-1595.41459517045	0.000647298967104256	\\
-1594.43581321023	0.000645284639201299	\\
-1593.45703125	0.000650152050066192	\\
-1592.47824928977	0.000648957605882257	\\
-1591.49946732955	0.000653789883449657	\\
-1590.52068536932	0.000652244294815227	\\
-1589.54190340909	0.000657640859738976	\\
-1588.56312144886	0.000655418775286853	\\
-1587.58433948864	0.00065359912325178	\\
-1586.60555752841	0.000662259838916439	\\
-1585.62677556818	0.000669347962599351	\\
-1584.64799360795	0.000665267498460164	\\
-1583.66921164773	0.000669075839604476	\\
-1582.6904296875	0.000672832703937879	\\
-1581.71164772727	0.000674922425762462	\\
-1580.73286576705	0.000679748699423136	\\
-1579.75408380682	0.000679984304589047	\\
-1578.77530184659	0.000686575743159273	\\
-1577.79651988636	0.000678781806779802	\\
-1576.81773792614	0.000684392747463241	\\
-1575.83895596591	0.000681949664862463	\\
-1574.86017400568	0.00068180706194907	\\
-1573.88139204545	0.000683492014163985	\\
-1572.90261008523	0.000682209007831244	\\
-1571.923828125	0.000682580644568324	\\
-1570.94504616477	0.000682021793576858	\\
-1569.96626420455	0.000677128011843145	\\
-1568.98748224432	0.000675371600389721	\\
-1568.00870028409	0.000673773105714961	\\
-1567.02991832386	0.000672679205434066	\\
-1566.05113636364	0.000670183119017725	\\
-1565.07235440341	0.000668849250413043	\\
-1564.09357244318	0.000663828284080712	\\
-1563.11479048295	0.000663262298123515	\\
-1562.13600852273	0.000666020895643107	\\
-1561.1572265625	0.000658028727985594	\\
-1560.17844460227	0.00065297675979225	\\
-1559.19966264205	0.000653991420825161	\\
-1558.22088068182	0.000644158625338134	\\
-1557.24209872159	0.00064684590213547	\\
-1556.26331676136	0.000640768683475767	\\
-1555.28453480114	0.000638942829602197	\\
-1554.30575284091	0.000640132683091975	\\
-1553.32697088068	0.000635496541103804	\\
-1552.34818892045	0.000626973413559476	\\
-1551.36940696023	0.000629024034691461	\\
-1550.390625	0.000627299262275015	\\
-1549.41184303977	0.000625414593101839	\\
-1548.43306107955	0.000627772290041343	\\
-1547.45427911932	0.000623557020474331	\\
-1546.47549715909	0.000627177327864932	\\
-1545.49671519886	0.000623505216000636	\\
-1544.51793323864	0.000628643602727507	\\
-1543.53915127841	0.00062660805142211	\\
-1542.56036931818	0.000627589231385993	\\
-1541.58158735795	0.000624973935894413	\\
-1540.60280539773	0.00062728122828287	\\
-1539.6240234375	0.000629639975728645	\\
-1538.64524147727	0.000625000507300999	\\
-1537.66645951705	0.000628066909771238	\\
-1536.68767755682	0.000630286143777146	\\
-1535.70889559659	0.000631957241118865	\\
-1534.73011363636	0.000629143436286444	\\
-1533.75133167614	0.000636367335564592	\\
-1532.77254971591	0.000638075176367731	\\
-1531.79376775568	0.000640604960710366	\\
-1530.81498579545	0.000643454213062822	\\
-1529.83620383523	0.000644479328705218	\\
-1528.857421875	0.000650192947136624	\\
-1527.87863991477	0.000647152908226443	\\
-1526.89985795455	0.000657643254155682	\\
-1525.92107599432	0.00065402924212659	\\
-1524.94229403409	0.000652967096833202	\\
-1523.96351207386	0.000651190100963596	\\
-1522.98473011364	0.000656484454674625	\\
-1522.00594815341	0.000650688065982599	\\
-1521.02716619318	0.000644012565971856	\\
-1520.04838423295	0.000647732131889774	\\
-1519.06960227273	0.000637242207005659	\\
-1518.0908203125	0.0006400972492654	\\
-1517.11203835227	0.000637385642899642	\\
-1516.13325639205	0.000630483591471177	\\
-1515.15447443182	0.000623634134726635	\\
-1514.17569247159	0.000621478156677358	\\
-1513.19691051136	0.000621620505040699	\\
-1512.21812855114	0.000609680973350296	\\
-1511.23934659091	0.000612903772343956	\\
-1510.26056463068	0.000609929364309395	\\
-1509.28178267045	0.000600842514046326	\\
-1508.30300071023	0.00059920255092149	\\
-1507.32421875	0.000591461423354261	\\
-1506.34543678977	0.000587499344712148	\\
-1505.36665482955	0.000586153200706658	\\
-1504.38787286932	0.000580449871822854	\\
-1503.40909090909	0.000577637054009965	\\
-1502.43030894886	0.00057416862239937	\\
-1501.45152698864	0.000571049421973608	\\
-1500.47274502841	0.000567119368605163	\\
-1499.49396306818	0.000563506571140241	\\
-1498.51518110795	0.000554919839946777	\\
-1497.53639914773	0.000555352125316557	\\
-1496.5576171875	0.000547667076451431	\\
-1495.57883522727	0.000544679909756037	\\
-1494.60005326705	0.000540120141443895	\\
-1493.62127130682	0.000539573355455154	\\
-1492.64248934659	0.000534465835361288	\\
-1491.66370738636	0.000535264692864405	\\
-1490.68492542614	0.000525954690760402	\\
-1489.70614346591	0.000532242531918237	\\
-1488.72736150568	0.000525050985582149	\\
-1487.74857954545	0.000525171717764583	\\
-1486.76979758523	0.000516540178553688	\\
-1485.791015625	0.000518803358404013	\\
-1484.81223366477	0.000509577989997884	\\
-1483.83345170455	0.000511539693418598	\\
-1482.85466974432	0.000508325445054018	\\
-1481.87588778409	0.000501823666646573	\\
-1480.89710582386	0.000506403749406407	\\
-1479.91832386364	0.000503788641238143	\\
-1478.93954190341	0.000497286784811959	\\
-1477.96075994318	0.000490854497015948	\\
-1476.98197798295	0.000491766616082791	\\
-1476.00319602273	0.000489662431824225	\\
-1475.0244140625	0.000484503231653959	\\
-1474.04563210227	0.000485804381778565	\\
-1473.06685014205	0.000482345599171597	\\
-1472.08806818182	0.000474519592124671	\\
-1471.10928622159	0.000473153162937933	\\
-1470.13050426136	0.000466147594295103	\\
-1469.15172230114	0.000463528421871612	\\
-1468.17294034091	0.000466434309177002	\\
-1467.19415838068	0.000462030483906254	\\
-1466.21537642045	0.000458258679850596	\\
-1465.23659446023	0.000454472332173789	\\
-1464.2578125	0.000450528025199402	\\
-1463.27903053977	0.000446499754224226	\\
-1462.30024857955	0.0004456828151563	\\
-1461.32146661932	0.000438402025323104	\\
-1460.34268465909	0.000440989980948655	\\
-1459.36390269886	0.000430896906808554	\\
-1458.38512073864	0.000433266813941701	\\
-1457.40633877841	0.00042723615606915	\\
-1456.42755681818	0.000429219808524528	\\
-1455.44877485795	0.000423033190491329	\\
-1454.46999289773	0.000417125746659943	\\
-1453.4912109375	0.00041895782843127	\\
-1452.51242897727	0.000418177970714619	\\
-1451.53364701705	0.000414729987474357	\\
-1450.55486505682	0.00040703583729055	\\
-1449.57608309659	0.000406140866392021	\\
-1448.59730113636	0.000401504279515867	\\
-1447.61851917614	0.00040345002560008	\\
-1446.63973721591	0.000396162116693471	\\
-1445.66095525568	0.000393422846313272	\\
-1444.68217329545	0.000390683423014431	\\
-1443.70339133523	0.000391528020535591	\\
-1442.724609375	0.0003844141029607	\\
-1441.74582741477	0.000383872672241426	\\
-1440.76704545455	0.000386230907976723	\\
-1439.78826349432	0.000377953874451972	\\
-1438.80948153409	0.000376366563728858	\\
-1437.83069957386	0.000377760762636969	\\
-1436.85191761364	0.000368403433008487	\\
-1435.87313565341	0.000375362027366137	\\
-1434.89435369318	0.00037158514563368	\\
-1433.91557173295	0.000367457198367911	\\
-1432.93678977273	0.00037397295339973	\\
-1431.9580078125	0.000366296238706924	\\
-1430.97922585227	0.000369946944162816	\\
-1430.00044389205	0.000367427091380924	\\
-1429.02166193182	0.00036763249408382	\\
-1428.04287997159	0.00036612551541637	\\
-1427.06409801136	0.000364034766591774	\\
-1426.08531605114	0.000358547694278342	\\
-1425.10653409091	0.000356004742243205	\\
-1424.12775213068	0.000356780956602596	\\
-1423.14897017045	0.000355142448372006	\\
-1422.17018821023	0.000352383564468268	\\
-1421.19140625	0.000348828826433975	\\
-1420.21262428977	0.000345467286316257	\\
-1419.23384232955	0.000342439363622643	\\
-1418.25506036932	0.000345293607941831	\\
-1417.27627840909	0.000339738155065009	\\
-1416.29749644886	0.000340207137485812	\\
-1415.31871448864	0.000337759505638317	\\
-1414.33993252841	0.00033651592324309	\\
-1413.36115056818	0.000334141807484871	\\
-1412.38236860795	0.000327833374240234	\\
-1411.40358664773	0.000333033903120249	\\
-1410.4248046875	0.000323857698807975	\\
-1409.44602272727	0.000325240318061154	\\
-1408.46724076705	0.000324466231605776	\\
-1407.48845880682	0.000330494606284882	\\
-1406.50967684659	0.000321970170885423	\\
-1405.53089488636	0.000321721113901135	\\
-1404.55211292614	0.000321378145612414	\\
-1403.57333096591	0.000317417354204143	\\
-1402.59454900568	0.000321771525704547	\\
-1401.61576704545	0.000320448561055529	\\
-1400.63698508523	0.000321327094296582	\\
-1399.658203125	0.000319401497168963	\\
-1398.67942116477	0.000316962866726154	\\
-1397.70063920455	0.000311091129102713	\\
-1396.72185724432	0.000316488620567731	\\
-1395.74307528409	0.00031290051051572	\\
-1394.76429332386	0.000309064534840423	\\
-1393.78551136364	0.000306801486934409	\\
-1392.80672940341	0.000308521445304498	\\
-1391.82794744318	0.000308071934876239	\\
-1390.84916548295	0.000310714022514924	\\
-1389.87038352273	0.000311529807201982	\\
-1388.8916015625	0.000307473824713359	\\
-1387.91281960227	0.000316138863325759	\\
-1386.93403764205	0.000310161688531593	\\
-1385.95525568182	0.000313378255978623	\\
-1384.97647372159	0.000308988480525104	\\
-1383.99769176136	0.000313313476666835	\\
-1383.01890980114	0.000315904009566405	\\
-1382.04012784091	0.000315749223210676	\\
-1381.06134588068	0.000310283667046371	\\
-1380.08256392045	0.000315826025752121	\\
-1379.10378196023	0.000318902465757387	\\
-1378.125	0.000317559138700551	\\
-1377.14621803977	0.00031643546688707	\\
-1376.16743607955	0.000318609238452216	\\
-1375.18865411932	0.000319438693473834	\\
-1374.20987215909	0.000321998968031843	\\
-1373.23109019886	0.000316785057683362	\\
-1372.25230823864	0.000322783509656643	\\
-1371.27352627841	0.000316711167953629	\\
-1370.29474431818	0.000326906900883031	\\
-1369.31596235795	0.000317382945212469	\\
-1368.33718039773	0.000325494362011385	\\
-1367.3583984375	0.000323862577703979	\\
-1366.37961647727	0.000323317001829902	\\
-1365.40083451705	0.000325396865472035	\\
-1364.42205255682	0.000322098700574279	\\
-1363.44327059659	0.000325537447306009	\\
-1362.46448863636	0.000323996625834431	\\
-1361.48570667614	0.000322004313465323	\\
-1360.50692471591	0.00032438424350655	\\
-1359.52814275568	0.000322278556767916	\\
-1358.54936079545	0.000320369872198001	\\
-1357.57057883523	0.00032415576238197	\\
-1356.591796875	0.00032376364495934	\\
-1355.61301491477	0.000321991240087777	\\
-1354.63423295455	0.000328729884986168	\\
-1353.65545099432	0.000325658840761659	\\
-1352.67666903409	0.000326822240282933	\\
-1351.69788707386	0.000326082195512349	\\
-1350.71910511364	0.000327303479077768	\\
-1349.74032315341	0.000327891270959323	\\
-1348.76154119318	0.000332359744749378	\\
-1347.78275923295	0.000333545311803946	\\
-1346.80397727273	0.000329080314179537	\\
-1345.8251953125	0.000331272329743951	\\
-1344.84641335227	0.000332379550092162	\\
-1343.86763139205	0.000333843367148132	\\
-1342.88884943182	0.000335368096740331	\\
-1341.91006747159	0.000330694521041026	\\
-1340.93128551136	0.000335988960085066	\\
-1339.95250355114	0.000331792442241525	\\
-1338.97372159091	0.000332804293505161	\\
-1337.99493963068	0.000334427812980036	\\
-1337.01615767045	0.000334685358163893	\\
-1336.03737571023	0.000332529624965626	\\
-1335.05859375	0.000334217723735148	\\
-1334.07981178977	0.000334530003013581	\\
-1333.10102982955	0.000336727000081561	\\
-1332.12224786932	0.000334228883459863	\\
-1331.14346590909	0.000336475627112993	\\
-1330.16468394886	0.000336492067712299	\\
-1329.18590198864	0.000338117558776997	\\
-1328.20712002841	0.000333008297713279	\\
-1327.22833806818	0.000340797791787026	\\
-1326.24955610795	0.000336859285669586	\\
-1325.27077414773	0.000336733962350813	\\
-1324.2919921875	0.000336622149792072	\\
-1323.31321022727	0.000340078739145278	\\
-1322.33442826705	0.000337239492994245	\\
-1321.35564630682	0.000340391845891586	\\
-1320.37686434659	0.000337452622184823	\\
-1319.39808238636	0.000335740833047007	\\
-1318.41930042614	0.000333876518912244	\\
-1317.44051846591	0.00033843260771956	\\
-1316.46173650568	0.000332968772081657	\\
-1315.48295454545	0.000339243117613303	\\
-1314.50417258523	0.000332892477505677	\\
-1313.525390625	0.000342486453268794	\\
-1312.54660866477	0.000333792228881977	\\
-1311.56782670455	0.000334972503111293	\\
-1310.58904474432	0.00034102573703228	\\
-1309.61026278409	0.000338861437208805	\\
-1308.63148082386	0.000336366559242183	\\
-1307.65269886364	0.000334047816164701	\\
-1306.67391690341	0.000335919146127847	\\
-1305.69513494318	0.000333853360337313	\\
-1304.71635298295	0.000335488429099215	\\
-1303.73757102273	0.000332001208629352	\\
-1302.7587890625	0.000334342786813072	\\
-1301.78000710227	0.000333902171468764	\\
-1300.80122514205	0.000333612821579971	\\
-1299.82244318182	0.00033767515527372	\\
-1298.84366122159	0.000334091140667236	\\
-1297.86487926136	0.000333030231629166	\\
-1296.88609730114	0.000334576231445864	\\
-1295.90731534091	0.000336998287647506	\\
-1294.92853338068	0.000337683544620549	\\
-1293.94975142045	0.000335378148434344	\\
-1292.97096946023	0.000340279727241216	\\
-1291.9921875	0.000342357862620552	\\
-1291.01340553977	0.00033796259101491	\\
-1290.03462357955	0.000340162926187272	\\
-1289.05584161932	0.000342358621408591	\\
-1288.07705965909	0.000339010265298016	\\
-1287.09827769886	0.000344203415941358	\\
-1286.11949573864	0.000338751197218641	\\
-1285.14071377841	0.000339937612706636	\\
-1284.16193181818	0.000342043792537556	\\
-1283.18314985795	0.000339070203735209	\\
-1282.20436789773	0.000342924026321429	\\
-1281.2255859375	0.000340488432954019	\\
-1280.24680397727	0.000342356586481721	\\
-1279.26802201705	0.000339769586407272	\\
-1278.28924005682	0.000346522128532485	\\
-1277.31045809659	0.000337739465164644	\\
-1276.33167613636	0.000341143681752154	\\
-1275.35289417614	0.00033889141585354	\\
-1274.37411221591	0.000339394085152136	\\
-1273.39533025568	0.000336147404358038	\\
-1272.41654829545	0.000343475264404884	\\
-1271.43776633523	0.000337163801724772	\\
-1270.458984375	0.000335945432212293	\\
-1269.48020241477	0.00033311315479769	\\
-1268.50142045455	0.000337591542295163	\\
-1267.52263849432	0.000338183187403906	\\
-1266.54385653409	0.000338068752072103	\\
-1265.56507457386	0.00033579028091985	\\
-1264.58629261364	0.000332435593192866	\\
-1263.60751065341	0.000336573014144204	\\
-1262.62872869318	0.000331848272701294	\\
-1261.64994673295	0.000329583749628609	\\
-1260.67116477273	0.000333456875740678	\\
-1259.6923828125	0.000331706105665107	\\
-1258.71360085227	0.000329589123418887	\\
-1257.73481889205	0.000328816108679864	\\
-1256.75603693182	0.000325912374574185	\\
-1255.77725497159	0.000326917679488957	\\
-1254.79847301136	0.000320072612466069	\\
-1253.81969105114	0.000331155971938633	\\
-1252.84090909091	0.000323207821926161	\\
-1251.86212713068	0.000326969467363185	\\
-1250.88334517045	0.000324465580484415	\\
-1249.90456321023	0.000331670573787541	\\
-1248.92578125	0.000325032004090524	\\
-1247.94699928977	0.000325699477668964	\\
-1246.96821732955	0.000323391120676555	\\
-1245.98943536932	0.00032326502813457	\\
-1245.01065340909	0.00032616657352527	\\
-1244.03187144886	0.000322199951690105	\\
-1243.05308948864	0.000318939799054219	\\
-1242.07430752841	0.000318285705800156	\\
-1241.09552556818	0.000317700493974103	\\
-1240.11674360795	0.000314741911964352	\\
-1239.13796164773	0.000316249487284574	\\
-1238.1591796875	0.000318132109760372	\\
-1237.18039772727	0.000315753245640239	\\
-1236.20161576705	0.000314428942487371	\\
-1235.22283380682	0.000315890050654776	\\
-1234.24405184659	0.000312062590051239	\\
-1233.26526988636	0.000310131569609677	\\
-1232.28648792614	0.000308053781662238	\\
-1231.30770596591	0.000310998962406483	\\
-1230.32892400568	0.000311180756781511	\\
-1229.35014204545	0.000314665825257915	\\
-1228.37136008523	0.000310082764551248	\\
-1227.392578125	0.000310604324713085	\\
-1226.41379616477	0.000312010973644589	\\
-1225.43501420455	0.000311949052641266	\\
-1224.45623224432	0.000312726442698749	\\
-1223.47745028409	0.000310328268969754	\\
-1222.49866832386	0.000307629840249586	\\
-1221.51988636364	0.000311648377283656	\\
-1220.54110440341	0.000311287041132441	\\
-1219.56232244318	0.000309416124538545	\\
-1218.58354048295	0.000311461866154362	\\
-1217.60475852273	0.000308741789971928	\\
-1216.6259765625	0.000315915507547871	\\
-1215.64719460227	0.000311906006437204	\\
-1214.66841264205	0.000321309595103979	\\
-1213.68963068182	0.000317345331519105	\\
-1212.71084872159	0.000316589405690889	\\
-1211.73206676136	0.000315669901295566	\\
-1210.75328480114	0.000314857653983953	\\
-1209.77450284091	0.000315087882781493	\\
-1208.79572088068	0.000317606409688798	\\
-1207.81693892045	0.000316106018217273	\\
-1206.83815696023	0.000322072350703403	\\
-1205.859375	0.000320510890012287	\\
-1204.88059303977	0.00032190966575564	\\
-1203.90181107955	0.000319547027720477	\\
-1202.92302911932	0.000320016532001943	\\
-1201.94424715909	0.000322296596576484	\\
-1200.96546519886	0.000324929698708201	\\
-1199.98668323864	0.000322812007848167	\\
-1199.00790127841	0.000323034197156028	\\
-1198.02911931818	0.000317664502258613	\\
-1197.05033735795	0.000324877784185634	\\
-1196.07155539773	0.000323446244487045	\\
-1195.0927734375	0.000322409520734261	\\
-1194.11399147727	0.000323559680025753	\\
-1193.13520951705	0.00031791987931001	\\
-1192.15642755682	0.000325555010354946	\\
-1191.17764559659	0.000319776919649472	\\
-1190.19886363636	0.000324797194762476	\\
-1189.22008167614	0.000321637370469783	\\
-1188.24129971591	0.000325726090328243	\\
-1187.26251775568	0.000321755780278128	\\
-1186.28373579545	0.000322563204391796	\\
-1185.30495383523	0.000325101685602799	\\
-1184.326171875	0.000325950125210307	\\
-1183.34738991477	0.000327004033192437	\\
-1182.36860795455	0.000328127830298061	\\
-1181.38982599432	0.000327794735446469	\\
-1180.41104403409	0.000327733037616276	\\
-1179.43226207386	0.000332135439497714	\\
-1178.45348011364	0.000326291475454755	\\
-1177.47469815341	0.000332781191058152	\\
-1176.49591619318	0.000328416444238671	\\
-1175.51713423295	0.000333527691819626	\\
-1174.53835227273	0.000327691056700838	\\
-1173.5595703125	0.000329320143067609	\\
-1172.58078835227	0.000324898673068755	\\
-1171.60200639205	0.000330582810999229	\\
-1170.62322443182	0.000327710437510211	\\
-1169.64444247159	0.000331544367650338	\\
-1168.66566051136	0.00032338971237054	\\
-1167.68687855114	0.000329508094836459	\\
-1166.70809659091	0.000324460549205842	\\
-1165.72931463068	0.000325645403010817	\\
-1164.75053267045	0.00032184586972051	\\
-1163.77175071023	0.000327940023383576	\\
-1162.79296875	0.000319862265223796	\\
-1161.81418678977	0.000328157475413454	\\
-1160.83540482955	0.000317739421441598	\\
-1159.85662286932	0.000323188449814278	\\
-1158.87784090909	0.000319761494483206	\\
-1157.89905894886	0.000319592296532484	\\
-1156.92027698864	0.000315642316172	\\
-1155.94149502841	0.000312732893174119	\\
-1154.96271306818	0.000311045884183165	\\
-1153.98393110795	0.000314519885079509	\\
-1153.00514914773	0.000310945150400733	\\
-1152.0263671875	0.000309213366542687	\\
-1151.04758522727	0.000308270106806825	\\
-1150.06880326705	0.000308361633237842	\\
-1149.09002130682	0.000313324909805466	\\
-1148.11123934659	0.000308945424614496	\\
-1147.13245738636	0.000306973838613318	\\
-1146.15367542614	0.000308411104978462	\\
-1145.17489346591	0.000302596211954312	\\
-1144.19611150568	0.000305830545789182	\\
-1143.21732954545	0.00030350244198999	\\
-1142.23854758523	0.000300935542099772	\\
-1141.259765625	0.000306027864828746	\\
-1140.28098366477	0.000301952133298309	\\
-1139.30220170455	0.000302198453423243	\\
-1138.32341974432	0.000300680127224709	\\
-1137.34463778409	0.000302268858697532	\\
-1136.36585582386	0.000294580768878807	\\
-1135.38707386364	0.000304067153654346	\\
-1134.40829190341	0.000295903853565367	\\
-1133.42950994318	0.000300305120511905	\\
-1132.45072798295	0.000298378870882057	\\
-1131.47194602273	0.000304259541829962	\\
-1130.4931640625	0.000297721685115026	\\
-1129.51438210227	0.000300069237098543	\\
-1128.53560014205	0.000302951448589926	\\
-1127.55681818182	0.000302596141833831	\\
-1126.57803622159	0.000303229779179572	\\
-1125.59925426136	0.000301850777640668	\\
-1124.62047230114	0.000310290329291305	\\
-1123.64169034091	0.00030658349306847	\\
-1122.66290838068	0.000306728935894694	\\
-1121.68412642045	0.000305539504157251	\\
-1120.70534446023	0.000311397464488931	\\
-1119.7265625	0.000309990721293119	\\
-1118.74778053977	0.000307568139246085	\\
-1117.76899857955	0.000310698648903044	\\
-1116.79021661932	0.00030963294961069	\\
-1115.81143465909	0.000306948235095124	\\
-1114.83265269886	0.000312481836194363	\\
-1113.85387073864	0.000315487479092061	\\
-1112.87508877841	0.000316823157775872	\\
-1111.89630681818	0.000314979438348804	\\
-1110.91752485795	0.000318331590286661	\\
-1109.93874289773	0.000319230855725263	\\
-1108.9599609375	0.000319344854158387	\\
-1107.98117897727	0.000321392849752901	\\
-1107.00239701705	0.000323444751967145	\\
-1106.02361505682	0.000327744487878215	\\
-1105.04483309659	0.000325814049161476	\\
-1104.06605113636	0.000324169039902274	\\
-1103.08726917614	0.000323774071679368	\\
-1102.10848721591	0.000327346434975903	\\
-1101.12970525568	0.000323829990970886	\\
-1100.15092329545	0.000324559709655045	\\
-1099.17214133523	0.000325279403777811	\\
-1098.193359375	0.000331666672242086	\\
-1097.21457741477	0.000329711760725751	\\
-1096.23579545455	0.00033141542246469	\\
-1095.25701349432	0.000327938762872249	\\
-1094.27823153409	0.000330163411653881	\\
-1093.29944957386	0.000330016472048421	\\
-1092.32066761364	0.000332380216426153	\\
-1091.34188565341	0.000336371693569426	\\
-1090.36310369318	0.000332327443832419	\\
-1089.38432173295	0.000337563391578227	\\
-1088.40553977273	0.000335521340014428	\\
-1087.4267578125	0.000339556850618136	\\
-1086.44797585227	0.000344454655458133	\\
-1085.46919389205	0.000343617972023498	\\
-1084.49041193182	0.000344818071147576	\\
-1083.51162997159	0.000343530595547597	\\
-1082.53284801136	0.000343748749271828	\\
-1081.55406605114	0.000341382499382802	\\
-1080.57528409091	0.000344362416688414	\\
-1079.59650213068	0.000346770442366444	\\
-1078.61772017045	0.000343140337617786	\\
-1077.63893821023	0.000343421888030803	\\
-1076.66015625	0.000343269522370626	\\
-1075.68137428977	0.000345505481084489	\\
-1074.70259232955	0.000341773785360043	\\
-1073.72381036932	0.000346218503354227	\\
-1072.74502840909	0.000350823521526987	\\
-1071.76624644886	0.000340897456941581	\\
-1070.78746448864	0.000345627180893296	\\
-1069.80868252841	0.000343221189450719	\\
-1068.82990056818	0.000349913307487546	\\
-1067.85111860795	0.000339246383202782	\\
-1066.87233664773	0.000340035250769789	\\
-1065.8935546875	0.000344285909831319	\\
-1064.91477272727	0.000342683846421267	\\
-1063.93599076705	0.00033567603797191	\\
-1062.95720880682	0.000336831682048821	\\
-1061.97842684659	0.000334421939216228	\\
-1060.99964488636	0.000332066378881608	\\
-1060.02086292614	0.000331639329160326	\\
-1059.04208096591	0.000334513642653533	\\
-1058.06329900568	0.000326483315134597	\\
-1057.08451704545	0.000329022889883417	\\
-1056.10573508523	0.000322450006419818	\\
-1055.126953125	0.000325175979357225	\\
-1054.14817116477	0.000321731745827782	\\
-1053.16938920455	0.000320501812608458	\\
-1052.19060724432	0.000323726360766443	\\
-1051.21182528409	0.000312658149454483	\\
-1050.23304332386	0.000322434299760839	\\
-1049.25426136364	0.000310982247318774	\\
-1048.27547940341	0.000325440366361696	\\
-1047.29669744318	0.000308612182460409	\\
-1046.31791548295	0.000321487864687623	\\
-1045.33913352273	0.000309694779775224	\\
-1044.3603515625	0.000313764163413286	\\
-1043.38156960227	0.000310923567610688	\\
-1042.40278764205	0.000308478575860409	\\
-1041.42400568182	0.000307399747060624	\\
-1040.44522372159	0.000311416619943814	\\
-1039.46644176136	0.000310426650186991	\\
-1038.48765980114	0.000308727240151344	\\
-1037.50887784091	0.000310793060678246	\\
-1036.53009588068	0.000307183610895743	\\
-1035.55131392045	0.000309857348976224	\\
-1034.57253196023	0.000310298796745755	\\
-1033.59375	0.000307558294074412	\\
-1032.61496803977	0.000308452757224011	\\
-1031.63618607955	0.00030725686645874	\\
-1030.65740411932	0.00030618721345376	\\
-1029.67862215909	0.000305841761694413	\\
-1028.69984019886	0.000307757267571716	\\
-1027.72105823864	0.000304029102189603	\\
-1026.74227627841	0.000305372122395876	\\
-1025.76349431818	0.000304633927537138	\\
-1024.78471235795	0.000302268809738377	\\
-1023.80593039773	0.000309151674767156	\\
-1022.8271484375	0.00030486464521271	\\
-1021.84836647727	0.000306618083489201	\\
-1020.86958451705	0.000298191923176652	\\
-1019.89080255682	0.000304759092328819	\\
-1018.91202059659	0.000304193414528074	\\
-1017.93323863636	0.00030563549154706	\\
-1016.95445667614	0.000304177891191969	\\
-1015.97567471591	0.000307754420279217	\\
-1014.99689275568	0.000303404039956939	\\
-1014.01811079545	0.000305421916664902	\\
-1013.03932883523	0.000302979223477831	\\
-1012.060546875	0.000311254636777323	\\
-1011.08176491477	0.000302302547983654	\\
-1010.10298295455	0.000306884437893201	\\
-1009.12420099432	0.000310472522491393	\\
-1008.14541903409	0.000313558081437493	\\
-1007.16663707386	0.000312985172464529	\\
-1006.18785511364	0.00031434789356399	\\
-1005.20907315341	0.000315684025185178	\\
-1004.23029119318	0.000313840867752007	\\
-1003.25150923295	0.000319476635391106	\\
-1002.27272727273	0.000318244101375238	\\
-1001.2939453125	0.000320460948902998	\\
-1000.31516335227	0.000315389728667589	\\
-999.336381392046	0.000317755001666636	\\
-998.357599431818	0.000329174930174571	\\
-997.378817471592	0.000327669274144496	\\
-996.400035511364	0.000324578240220267	\\
-995.421253551136	0.000329150116436786	\\
-994.44247159091	0.000324963126601179	\\
-993.463689630682	0.000332182273015529	\\
-992.484907670454	0.000326041786259574	\\
-991.506125710228	0.000339939849149441	\\
-990.52734375	0.000326739710024749	\\
-989.548561789772	0.000342024666099609	\\
-988.569779829546	0.00033358747765949	\\
-987.590997869318	0.000334167044685475	\\
-986.612215909092	0.000336531914137821	\\
-985.633433948864	0.000336884528642772	\\
-984.654651988636	0.000336953197677078	\\
-983.67587002841	0.000340402448952197	\\
-982.697088068182	0.000341419229513254	\\
-981.718306107954	0.000339020194803796	\\
-980.739524147728	0.000338791612613386	\\
-979.7607421875	0.000336403315347887	\\
-978.781960227272	0.000340710044344939	\\
-977.803178267046	0.000340439902273129	\\
-976.824396306818	0.000342106075430266	\\
-975.845614346592	0.000335330846314888	\\
-974.866832386364	0.000340257568219634	\\
-973.888050426136	0.000341299600485243	\\
-972.90926846591	0.000338742719793713	\\
-971.930486505682	0.000335084972996105	\\
-970.951704545454	0.000336762942928399	\\
-969.972922585228	0.000335469917605967	\\
-968.994140625	0.000340954664437745	\\
-968.015358664772	0.000337687553578744	\\
-967.036576704546	0.000338942428875229	\\
-966.057794744318	0.00033921476819718	\\
-965.079012784092	0.000334661914373888	\\
-964.100230823864	0.000340844196237101	\\
-963.121448863636	0.000335343192217271	\\
-962.14266690341	0.000338489714673545	\\
-961.163884943182	0.000331939153835542	\\
-960.185102982954	0.000345650792561675	\\
-959.206321022728	0.000337449984216187	\\
-958.2275390625	0.000342404259265784	\\
-957.248757102272	0.000337532864960897	\\
-956.269975142046	0.00034410397862707	\\
-955.291193181818	0.000341264810414463	\\
-954.312411221592	0.000338350941720834	\\
-953.333629261364	0.000341536650198537	\\
-952.354847301136	0.000345289180894863	\\
-951.37606534091	0.000345103523738443	\\
-950.397283380682	0.00033940808788847	\\
-949.418501420454	0.000346557535714772	\\
-948.439719460228	0.000342205720098678	\\
-947.4609375	0.000343869454075142	\\
-946.482155539772	0.000342810555561573	\\
-945.503373579546	0.000346166249608629	\\
-944.524591619318	0.000344452227546228	\\
-943.545809659092	0.000344091317804076	\\
-942.567027698864	0.0003421843139349	\\
-941.588245738636	0.000347049602899243	\\
-940.60946377841	0.000343129723802597	\\
-939.630681818182	0.000342021059579273	\\
-938.651899857954	0.000344497627316962	\\
-937.673117897728	0.000348118459721736	\\
-936.6943359375	0.000344863001216241	\\
-935.715553977272	0.000347065297624397	\\
-934.736772017046	0.000342839187385158	\\
-933.757990056818	0.00034357718771522	\\
-932.779208096592	0.000350135679438222	\\
-931.800426136364	0.00034604897678682	\\
-930.821644176136	0.000351728597500608	\\
-929.84286221591	0.000348355824937781	\\
-928.864080255682	0.00034534336740959	\\
-927.885298295454	0.000349696514085033	\\
-926.906516335228	0.000350420334413884	\\
-925.927734375	0.000348312120521282	\\
-924.948952414772	0.000352492386394158	\\
-923.970170454546	0.000352643430864872	\\
-922.991388494318	0.000357071946354517	\\
-922.012606534092	0.000352613258387925	\\
-921.033824573864	0.000355692533269434	\\
-920.055042613636	0.000357322852762003	\\
-919.07626065341	0.000359187143549216	\\
-918.097478693182	0.000359778833008919	\\
-917.118696732954	0.000354183583700345	\\
-916.139914772728	0.000359785115175761	\\
-915.1611328125	0.000355830032927462	\\
-914.182350852272	0.00036513150778242	\\
-913.203568892046	0.000362373925758919	\\
-912.224786931818	0.00036128992342536	\\
-911.246004971592	0.000363361601479946	\\
-910.267223011364	0.000364303697958534	\\
-909.288441051136	0.000362646672808588	\\
-908.30965909091	0.000365646140837556	\\
-907.330877130682	0.000356997248868916	\\
-906.352095170454	0.000363672922697632	\\
-905.373313210228	0.000358932554700845	\\
-904.39453125	0.00036128442024565	\\
-903.415749289772	0.000355075409089564	\\
-902.436967329546	0.000369017880204785	\\
-901.458185369318	0.000359552122564175	\\
-900.479403409092	0.000357221791391717	\\
-899.500621448864	0.00036115166181936	\\
-898.521839488636	0.000357966516596942	\\
-897.54305752841	0.000368429504088805	\\
-896.564275568182	0.000362209631017436	\\
-895.585493607954	0.000361543746626763	\\
-894.606711647728	0.000368335425500639	\\
-893.6279296875	0.000362189630660084	\\
-892.649147727272	0.000357653769150886	\\
-891.670365767046	0.000363537525143111	\\
-890.691583806818	0.000358047595569464	\\
-889.712801846592	0.000361169358879778	\\
-888.734019886364	0.000359933791956504	\\
-887.755237926136	0.000361837726585593	\\
-886.77645596591	0.000358571240622669	\\
-885.797674005682	0.000363154978405518	\\
-884.818892045454	0.000361230281328663	\\
-883.840110085228	0.00036025324581571	\\
-882.861328125	0.00036394092731962	\\
-881.882546164772	0.000366404096134241	\\
-880.903764204546	0.000366424904036563	\\
-879.924982244318	0.000365735772442759	\\
-878.946200284092	0.000365240693119853	\\
-877.967418323864	0.000361641759683916	\\
-876.988636363636	0.000363271539323824	\\
-876.00985440341	0.000355409756257656	\\
-875.031072443182	0.000352768514256841	\\
-874.052290482954	0.00035585660789403	\\
-873.073508522728	0.000357048970669856	\\
-872.0947265625	0.000362412190498973	\\
-871.115944602272	0.000361025233637074	\\
-870.137162642046	0.000363927347870117	\\
-869.158380681818	0.00035864917267131	\\
-868.179598721592	0.000358289019544122	\\
-867.200816761364	0.000358502821206479	\\
-866.222034801136	0.000362613826789129	\\
-865.24325284091	0.000362382458597178	\\
-864.264470880682	0.00035804793469584	\\
-863.285688920454	0.000359642523312827	\\
-862.306906960228	0.000360868955001507	\\
-861.328125	0.000357512669781369	\\
-860.349343039772	0.000360736020839248	\\
-859.370561079546	0.000357498743688474	\\
-858.391779119318	0.000357298047952243	\\
-857.412997159092	0.000356340858112936	\\
-856.434215198864	0.000359632554650799	\\
-855.455433238636	0.000358453718508122	\\
-854.47665127841	0.000362386892230666	\\
-853.497869318182	0.000357852037607357	\\
-852.519087357954	0.000354131832806729	\\
-851.540305397728	0.000358589961484198	\\
-850.5615234375	0.000350441551051811	\\
-849.582741477272	0.000358078594971218	\\
-848.603959517046	0.000353609559578694	\\
-847.625177556818	0.000357949823914134	\\
-846.646395596592	0.000358171397919013	\\
-845.667613636364	0.000363368058403659	\\
-844.688831676136	0.00035597203777332	\\
-843.71004971591	0.000358559294609069	\\
-842.731267755682	0.000358921087984156	\\
-841.752485795454	0.000355216970878505	\\
-840.773703835228	0.000361142948902388	\\
-839.794921875	0.000360110644123754	\\
-838.816139914772	0.000359377747321737	\\
-837.837357954546	0.0003605584284887	\\
-836.858575994318	0.000360433429057549	\\
-835.879794034092	0.000359302206166514	\\
-834.901012073864	0.000360677427059107	\\
-833.922230113636	0.000364486639724269	\\
-832.94344815341	0.000354317241813029	\\
-831.964666193182	0.000360749189636149	\\
-830.985884232954	0.000358061717848881	\\
-830.007102272728	0.000357570939102836	\\
-829.0283203125	0.000363884032357837	\\
-828.049538352272	0.000363915326419756	\\
-827.070756392046	0.000357494001937058	\\
-826.091974431818	0.000359036038092382	\\
-825.113192471592	0.000359270379610236	\\
-824.134410511364	0.000359885883391077	\\
-823.155628551136	0.000357585068134866	\\
-822.17684659091	0.000363105310008366	\\
-821.198064630682	0.000358498729752457	\\
-820.219282670454	0.000359891031455629	\\
-819.240500710228	0.000359393577059056	\\
-818.26171875	0.00036294444764364	\\
-817.282936789772	0.000362520035902964	\\
-816.304154829546	0.000360711966742722	\\
-815.325372869318	0.000362465570274008	\\
-814.346590909092	0.000366458613622801	\\
-813.367808948864	0.000361967547633767	\\
-812.389026988636	0.00036896034213402	\\
-811.41024502841	0.000363172008422343	\\
-810.431463068182	0.000363886927536274	\\
-809.452681107954	0.000365045482237448	\\
-808.473899147728	0.000372048268117796	\\
-807.4951171875	0.000361995561360341	\\
-806.516335227272	0.000368200789183821	\\
-805.537553267046	0.000368965188891843	\\
-804.558771306818	0.000369521256212023	\\
-803.579989346592	0.000370827821035067	\\
-802.601207386364	0.000374168320389532	\\
-801.622425426136	0.000372619649136604	\\
-800.64364346591	0.000369046114368141	\\
-799.664861505682	0.000387281215133889	\\
-798.686079545454	0.000376847147073916	\\
-797.707297585228	0.000379275320498739	\\
-796.728515625	0.000376695228202946	\\
-795.749733664772	0.000371718948673588	\\
-794.770951704546	0.000378035676615638	\\
-793.792169744318	0.000381066586259766	\\
-792.813387784092	0.000386111745754866	\\
-791.834605823864	0.000379335487472973	\\
-790.855823863636	0.000379728589904276	\\
-789.87704190341	0.000389070806949999	\\
-788.898259943182	0.000381714504752601	\\
-787.919477982954	0.000385049154359919	\\
-786.940696022728	0.000387289012811762	\\
-785.9619140625	0.000388657419616567	\\
-784.983132102272	0.000387636804229557	\\
-784.004350142046	0.000390719173649634	\\
-783.025568181818	0.000391105292603901	\\
-782.046786221592	0.000393275907185405	\\
-781.068004261364	0.000395797164131119	\\
-780.089222301136	0.000394718402449922	\\
-779.11044034091	0.000407530912336004	\\
-778.131658380682	0.000400861349911803	\\
-777.152876420454	0.000406268482148997	\\
-776.174094460228	0.000406868259593208	\\
-775.1953125	0.000408459631273616	\\
-774.216530539772	0.000393269332048179	\\
-773.237748579546	0.000403252267977562	\\
-772.258966619318	0.000413651850923412	\\
-771.280184659092	0.000403273979977472	\\
-770.301402698864	0.000402010609837437	\\
-769.322620738636	0.000409535530369498	\\
-768.34383877841	0.000412565349310782	\\
-767.365056818182	0.000413763609277144	\\
-766.386274857954	0.000422960109710369	\\
-765.407492897728	0.000421480867282226	\\
-764.4287109375	0.000415533668146299	\\
-763.449928977272	0.00041243364231404	\\
-762.471147017046	0.000412610582644388	\\
-761.492365056818	0.000425622864252744	\\
-760.513583096592	0.000426899910489845	\\
-759.534801136364	0.000423547249296871	\\
-758.556019176136	0.000421278867155107	\\
-757.57723721591	0.000430784877124188	\\
-756.598455255682	0.000432380758048506	\\
-755.619673295454	0.000430191055626157	\\
-754.640891335228	0.000429239608949873	\\
-753.662109375	0.000432783659748129	\\
-752.683327414772	0.000429957769786631	\\
-751.704545454546	0.000433115069028559	\\
-750.725763494318	0.000426715345030693	\\
-749.746981534092	0.000420402976582785	\\
-748.768199573864	0.000428853655175524	\\
-747.789417613636	0.000428186545411864	\\
-746.81063565341	0.000431905134506291	\\
-745.831853693182	0.000431657364204076	\\
-744.853071732954	0.000441086665548647	\\
-743.874289772728	0.00043606824289142	\\
-742.8955078125	0.000437421062485129	\\
-741.916725852272	0.000446483269732432	\\
-740.937943892046	0.000436271324705758	\\
-739.959161931818	0.000441315374807742	\\
-738.980379971592	0.000435953562955529	\\
-738.001598011364	0.000439291632525042	\\
-737.022816051136	0.000445264797474352	\\
-736.04403409091	0.000447324913742122	\\
-735.065252130682	0.000446615408059968	\\
-734.086470170454	0.000447318287787492	\\
-733.107688210228	0.000448991286958392	\\
-732.12890625	0.000449868868655442	\\
-731.150124289772	0.000449132468744214	\\
-730.171342329546	0.000446802002737392	\\
-729.192560369318	0.000443161422162841	\\
-728.213778409092	0.000450263783534024	\\
-727.234996448864	0.000450000139170865	\\
-726.256214488636	0.00044669054560905	\\
-725.27743252841	0.000448192673718274	\\
-724.298650568182	0.000446540659082589	\\
-723.319868607954	0.000440820638349307	\\
-722.341086647728	0.000450848506192438	\\
-721.3623046875	0.000445041366019106	\\
-720.383522727272	0.000453578643273814	\\
-719.404740767046	0.000450451511562155	\\
-718.425958806818	0.000447319421147237	\\
-717.447176846592	0.000447162651191069	\\
-716.468394886364	0.000443324171851333	\\
-715.489612926136	0.000447213938065882	\\
-714.51083096591	0.00044732711205269	\\
-713.532049005682	0.000453675229419475	\\
-712.553267045454	0.000450130472557303	\\
-711.574485085228	0.000451178826983599	\\
-710.595703125	0.000446427140239383	\\
-709.616921164772	0.000442545749526556	\\
-708.638139204546	0.000454707131603396	\\
-707.659357244318	0.000444879908812344	\\
-706.680575284092	0.000435749821181758	\\
-705.701793323864	0.000430575797100328	\\
-704.723011363636	0.000431033763340727	\\
-703.74422940341	0.000430921432946656	\\
-702.765447443182	0.000430795357056578	\\
-701.786665482954	0.000422078740779385	\\
-700.807883522728	0.000423634071590212	\\
-699.8291015625	0.000440586154873072	\\
-698.850319602272	0.000424639950118866	\\
-697.871537642046	0.000424403624166772	\\
-696.892755681818	0.000420409762747579	\\
-695.913973721592	0.000427094880936272	\\
-694.935191761364	0.000420948066929771	\\
-693.956409801136	0.0004243002287261	\\
-692.97762784091	0.000426474873332083	\\
-691.998845880682	0.000421031112606315	\\
-691.020063920454	0.000421333838773364	\\
-690.041281960228	0.000416243450974186	\\
-689.0625	0.000421444777831555	\\
-688.083718039772	0.000421628579385172	\\
-687.104936079546	0.000416845461976184	\\
-686.126154119318	0.000419442930637181	\\
-685.147372159092	0.000416326888810402	\\
-684.168590198864	0.000420254930989076	\\
-683.189808238636	0.000412389269512549	\\
-682.21102627841	0.000417797207898671	\\
-681.232244318182	0.000417358403350417	\\
-680.253462357954	0.000416527744461969	\\
-679.274680397728	0.00041155985312757	\\
-678.2958984375	0.000409681546648084	\\
-677.317116477272	0.000413307613983418	\\
-676.338334517046	0.0004141613713572	\\
-675.359552556818	0.000412785054056189	\\
-674.380770596592	0.000413394129879789	\\
-673.401988636364	0.000412741475047683	\\
-672.423206676136	0.000414294509921178	\\
-671.44442471591	0.000419474800137572	\\
-670.465642755682	0.000413055453589533	\\
-669.486860795454	0.000419003776997431	\\
-668.508078835228	0.000420502234381295	\\
-667.529296875	0.000415237587773298	\\
-666.550514914772	0.000417742711770502	\\
-665.571732954546	0.000414967785112846	\\
-664.592950994318	0.000424308109381539	\\
-663.614169034092	0.000426920989905532	\\
-662.635387073864	0.000430302694161085	\\
-661.656605113636	0.000430449470419103	\\
-660.67782315341	0.000432544972028996	\\
-659.699041193182	0.000432573362528639	\\
-658.720259232954	0.000433413502229515	\\
-657.741477272728	0.000438053426235723	\\
-656.7626953125	0.000437472915386572	\\
-655.783913352272	0.000441286223375637	\\
-654.805131392046	0.000443922265129739	\\
-653.826349431818	0.000446145870458671	\\
-652.847567471592	0.000445653187307438	\\
-651.868785511364	0.000452678993219936	\\
-650.890003551136	0.000460800289773057	\\
-649.91122159091	0.000456548345815864	\\
-648.932439630682	0.000458760493401183	\\
-647.953657670454	0.000460083297659286	\\
-646.974875710228	0.000452481752912528	\\
-645.99609375	0.000461390675049161	\\
-645.017311789772	0.000471312122250115	\\
-644.038529829546	0.000475735938382578	\\
-643.059747869318	0.000468217475916418	\\
-642.080965909092	0.000470477491361538	\\
-641.102183948864	0.000460309455749447	\\
-640.123401988636	0.00046255489839695	\\
-639.14462002841	0.00047387751304705	\\
-638.165838068182	0.000493103334859888	\\
-637.187056107954	0.000473684122474706	\\
-636.208274147728	0.000462847832447862	\\
-635.2294921875	0.000469412109943308	\\
-634.250710227272	0.000475670945615061	\\
-633.271928267046	0.000483264128608659	\\
-632.293146306818	0.000476959531923803	\\
-631.314364346592	0.000484456237589804	\\
-630.335582386364	0.000489281039095445	\\
-629.356800426136	0.000495844683813666	\\
-628.37801846591	0.000498860426666975	\\
-627.399236505682	0.000501267034049476	\\
-626.420454545454	0.000501235578696279	\\
-625.441672585228	0.000501083632926134	\\
-624.462890625	0.00051017168785292	\\
-623.484108664772	0.000509976464658221	\\
-622.505326704546	0.000502525885871015	\\
-621.526544744318	0.000513422165584343	\\
-620.547762784092	0.000505139966723556	\\
-619.568980823864	0.000507848772368735	\\
-618.590198863636	0.000513283763292571	\\
-617.61141690341	0.000516653531862756	\\
-616.632634943182	0.00051473569922847	\\
-615.653852982954	0.000519006277160791	\\
-614.675071022728	0.00052213403382999	\\
-613.6962890625	0.000523790700241881	\\
-612.717507102272	0.00051945497008044	\\
-611.738725142046	0.000517157684343363	\\
-610.759943181818	0.000523126340929024	\\
-609.781161221592	0.000532100015206733	\\
-608.802379261364	0.000534689144215091	\\
-607.823597301136	0.000537705580239551	\\
-606.84481534091	0.000537668064922131	\\
-605.866033380682	0.000544232081105145	\\
-604.887251420454	0.000540662202576679	\\
-603.908469460228	0.000537803567041007	\\
-602.9296875	0.000536301424684235	\\
-601.950905539772	0.000540520206551256	\\
-600.972123579546	0.000546751190066371	\\
-599.993341619318	0.000590534533314193	\\
-599.014559659092	0.000557832224416055	\\
-598.035777698864	0.000558935182200191	\\
-597.056995738636	0.000555765865501316	\\
-596.07821377841	0.000559853363247804	\\
-595.099431818182	0.000567484767310676	\\
-594.120649857954	0.000566720771749107	\\
-593.141867897728	0.000572517422265572	\\
-592.1630859375	0.000567664887802017	\\
-591.184303977272	0.000564567717641641	\\
-590.205522017046	0.000566318585436051	\\
-589.226740056818	0.000571304660558879	\\
-588.247958096592	0.000583047405864624	\\
-587.269176136364	0.000585506140687121	\\
-586.290394176136	0.000578088770500676	\\
-585.31161221591	0.000585530630987809	\\
-584.332830255682	0.000594845166613946	\\
-583.354048295454	0.000583531227181468	\\
-582.375266335228	0.000596467143256938	\\
-581.396484375	0.00059759071720384	\\
-580.417702414772	0.000596052294797448	\\
-579.438920454546	0.000604305315744237	\\
-578.460138494318	0.000604774610153619	\\
-577.481356534092	0.000603007082488287	\\
-576.502574573864	0.000613890403304167	\\
-575.523792613636	0.000614262072175963	\\
-574.54501065341	0.000618933044569433	\\
-573.566228693182	0.000621164384711626	\\
-572.587446732954	0.000622856716715079	\\
-571.608664772728	0.00062281226403368	\\
-570.6298828125	0.000630052823000363	\\
-569.651100852272	0.000634198289231003	\\
-568.672318892046	0.00062172959036516	\\
-567.693536931818	0.000627151593125559	\\
-566.714754971592	0.000626760220824279	\\
-565.735973011364	0.000628867898823465	\\
-564.757191051136	0.000635341615052699	\\
-563.77840909091	0.000645811123298568	\\
-562.799627130682	0.000646057641271127	\\
-561.820845170454	0.000648854491965655	\\
-560.842063210228	0.000646722784563072	\\
-559.86328125	0.000660621037014245	\\
-558.884499289772	0.000660059123061524	\\
-557.905717329546	0.000652361672308592	\\
-556.926935369318	0.000659307586361189	\\
-555.948153409092	0.000663332279227068	\\
-554.969371448864	0.000666989987909872	\\
-553.990589488636	0.000669907802275683	\\
-553.01180752841	0.00068444646703689	\\
-552.033025568182	0.000689040721003372	\\
-551.054243607954	0.000684305450639966	\\
-550.075461647728	0.000694096784599783	\\
-549.0966796875	0.000698379007380058	\\
-548.117897727272	0.000687488063658621	\\
-547.139115767046	0.000689350616145491	\\
-546.160333806818	0.000694843549207874	\\
-545.181551846592	0.000709445013244889	\\
-544.202769886364	0.000707524953814304	\\
-543.223987926136	0.000698495957323456	\\
-542.24520596591	0.000697918192219131	\\
-541.266424005682	0.000708753078801522	\\
-540.287642045454	0.000724236866939608	\\
-539.308860085228	0.000710593724055228	\\
-538.330078125	0.000710369996094321	\\
-537.351296164772	0.000710520886519686	\\
-536.372514204546	0.000718725456789808	\\
-535.393732244318	0.000719606500336875	\\
-534.414950284092	0.000704869252906275	\\
-533.436168323864	0.000700655974227001	\\
-532.457386363636	0.000722026891568053	\\
-531.47860440341	0.000706909683516415	\\
-530.499822443182	0.000700429872158579	\\
-529.521040482954	0.000717300504270478	\\
-528.542258522728	0.000742617450763622	\\
-527.5634765625	0.000725763019290151	\\
-526.584694602272	0.000714957363442081	\\
-525.605912642046	0.000714137998776813	\\
-524.627130681818	0.000701374525713751	\\
-523.648348721592	0.000695087516456081	\\
-522.669566761364	0.000702392742791508	\\
-521.690784801136	0.000695399285732952	\\
-520.71200284091	0.000672364541360367	\\
-519.733220880682	0.000686833109358082	\\
-518.754438920454	0.000724673022529109	\\
-517.775656960228	0.000739895682661187	\\
-516.796875	0.000715049394487831	\\
-515.818093039772	0.000731333015319539	\\
-514.839311079546	0.000728185161438411	\\
-513.860529119318	0.00068377869236241	\\
-512.881747159092	0.000693932098697732	\\
-511.902965198864	0.000723877771929658	\\
-510.924183238636	0.000719259559665077	\\
-509.94540127841	0.00067911188208221	\\
-508.966619318182	0.000691719932714647	\\
-507.987837357954	0.000683246590785125	\\
-507.009055397728	0.000668246526756626	\\
-506.0302734375	0.000680850077971221	\\
-505.051491477272	0.000676188901817867	\\
-504.072709517046	0.000697623489735392	\\
-503.093927556818	0.000692535396733239	\\
-502.115145596592	0.000660251368986687	\\
-501.136363636364	0.000679466337917638	\\
-500.157581676136	0.000713859789714559	\\
-499.17879971591	0.000797494171721587	\\
-498.200017755682	0.000709464427306083	\\
-497.221235795454	0.000729950606836928	\\
-496.242453835228	0.000702453054526458	\\
-495.263671875	0.000684346320816259	\\
-494.284889914772	0.000732544160379324	\\
-493.306107954546	0.000764562552473391	\\
-492.327325994318	0.000761031795224785	\\
-491.348544034092	0.000748363453545651	\\
-490.369762073864	0.000733805666264713	\\
-489.390980113636	0.000696150685761498	\\
-488.41219815341	0.000749156463280417	\\
-487.433416193182	0.000759208483684986	\\
-486.454634232954	0.000777289261126711	\\
-485.475852272728	0.00077342734506969	\\
-484.4970703125	0.000757834792122798	\\
-483.518288352272	0.000777147881358657	\\
-482.539506392046	0.00077368546522479	\\
-481.560724431818	0.000804416709481232	\\
-480.581942471592	0.000779816816055381	\\
-479.603160511364	0.000785145458883419	\\
-478.624378551136	0.000798514908405753	\\
-477.64559659091	0.000780170785821425	\\
-476.666814630682	0.000797589020116901	\\
-475.688032670454	0.000800611815102283	\\
-474.709250710228	0.000805164307057959	\\
-473.73046875	0.000803694943841879	\\
-472.751686789772	0.000809567765039488	\\
-471.772904829546	0.000812196035775532	\\
-470.794122869318	0.000817323525201492	\\
-469.815340909092	0.000812406681586047	\\
-468.836558948864	0.000817938304929245	\\
-467.857776988636	0.000805402297444207	\\
-466.87899502841	0.000805353748001744	\\
-465.900213068182	0.000801693880140547	\\
-464.921431107954	0.000806799076931883	\\
-463.942649147728	0.000794000049546115	\\
-462.9638671875	0.00078474778609925	\\
-461.985085227272	0.000778173494398689	\\
-461.006303267046	0.000767377674245471	\\
-460.027521306818	0.000770472445900856	\\
-459.048739346592	0.000754560427916563	\\
-458.069957386364	0.000765812669460104	\\
-457.091175426136	0.000744418788944396	\\
-456.11239346591	0.000729368771000686	\\
-455.133611505682	0.000705585320714134	\\
-454.154829545454	0.000702872788570825	\\
-453.176047585228	0.00068524839865789	\\
-452.197265625	0.000698392741506693	\\
-451.218483664772	0.000687163948103886	\\
-450.239701704546	0.00069267040744902	\\
-449.260919744318	0.00066995684149833	\\
-448.282137784092	0.000667259749741115	\\
-447.303355823864	0.000646542396875525	\\
-446.324573863636	0.000647902294599424	\\
-445.34579190341	0.000644845883952817	\\
-444.367009943182	0.000636157427156203	\\
-443.388227982954	0.000645742921811967	\\
-442.409446022728	0.000644823412556112	\\
-441.4306640625	0.000632398666720328	\\
-440.451882102272	0.000610276392575713	\\
-439.473100142046	0.000616335879227731	\\
-438.494318181818	0.000607600579138135	\\
-437.515536221592	0.000622009571865365	\\
-436.536754261364	0.000586780046324272	\\
-435.557972301136	0.000574023868962524	\\
-434.57919034091	0.000577993891373059	\\
-433.600408380682	0.000545371592927881	\\
-432.621626420454	0.000569593205282882	\\
-431.642844460228	0.000534920556026856	\\
-430.6640625	0.00052926939770762	\\
-429.685280539772	0.000552018876296973	\\
-428.706498579546	0.00051005286636233	\\
-427.727716619318	0.000524055978654457	\\
-426.748934659092	0.000511442871488304	\\
-425.770152698864	0.000502617226102546	\\
-424.791370738636	0.00051689766979051	\\
-423.81258877841	0.000505129527003524	\\
-422.833806818182	0.0004745582175976	\\
-421.855024857954	0.000476848720551874	\\
-420.876242897728	0.000471603655293965	\\
-419.8974609375	0.000471991828523001	\\
-418.918678977272	0.000470626738192827	\\
-417.939897017046	0.000465667366235931	\\
-416.961115056818	0.000443359324537912	\\
-415.982333096592	0.000459895264768692	\\
-415.003551136364	0.000458650930587253	\\
-414.024769176136	0.000446524245611072	\\
-413.04598721591	0.000443291241002881	\\
-412.067205255682	0.000423507550467567	\\
-411.088423295454	0.000427761652158531	\\
-410.109641335228	0.000429532967679815	\\
-409.130859375	0.000414416404866264	\\
-408.152077414772	0.00039909352480343	\\
-407.173295454546	0.000401458900400445	\\
-406.194513494318	0.000391869647461938	\\
-405.215731534092	0.000381432478103281	\\
-404.236949573864	0.000385651933637837	\\
-403.258167613636	0.000359404637877428	\\
-402.27938565341	0.000349025177789184	\\
-401.300603693182	0.000371740131441439	\\
-400.321821732954	0.000339334182671549	\\
-399.343039772728	0.000335347996175879	\\
-398.3642578125	0.000344442780972123	\\
-397.385475852272	0.000350141422802597	\\
-396.406693892046	0.000316705559785357	\\
-395.427911931818	0.000330558896339723	\\
-394.449129971592	0.000325848175494846	\\
-393.470348011364	0.000289256707137838	\\
-392.491566051136	0.000318779028616718	\\
-391.51278409091	0.000334920372135362	\\
-390.534002130682	0.000352156176045803	\\
-389.555220170454	0.000332440644803491	\\
-388.576438210228	0.000279906697178805	\\
-387.59765625	0.000304218299892425	\\
-386.618874289772	0.000292484253018635	\\
-385.640092329546	0.00029191439696336	\\
-384.661310369318	0.000305980698287177	\\
-383.682528409092	0.000279571672990918	\\
-382.703746448864	0.00027513919736743	\\
-381.724964488636	0.0002656651758667	\\
-380.74618252841	0.000268986745958727	\\
-379.767400568182	0.00027426484649614	\\
-378.788618607954	0.000260975650765576	\\
-377.809836647728	0.000264579912873251	\\
-376.8310546875	0.000240622300179319	\\
-375.852272727272	0.000252783803561825	\\
-374.873490767046	0.000262586539365935	\\
-373.894708806818	0.000260004856180482	\\
-372.915926846592	0.00023464774569226	\\
-371.937144886364	0.000232031757199944	\\
-370.958362926136	0.00023814883348655	\\
-369.97958096591	0.00023014733631303	\\
-369.000799005682	0.000257951585568001	\\
-368.022017045454	0.000232527736688923	\\
-367.043235085228	0.000256877236615585	\\
-366.064453125	0.000191286322518438	\\
-365.085671164772	0.00020217533927124	\\
-364.106889204546	0.000212276515971821	\\
-363.128107244318	0.000199440704844725	\\
-362.149325284092	0.000241394262748553	\\
-361.170543323864	0.000180003766968209	\\
-360.191761363636	0.000216251533996184	\\
-359.21297940341	0.000189516871216615	\\
-358.234197443182	0.000205966162358668	\\
-357.255415482954	0.000194558882992859	\\
-356.276633522728	0.000192888490848245	\\
-355.2978515625	0.000206059567807046	\\
-354.319069602272	0.000187085656585501	\\
-353.340287642046	0.000179622401422498	\\
-352.361505681818	0.000189634416046771	\\
-351.382723721592	0.000178744545938276	\\
-350.403941761364	0.000190685092886255	\\
-349.425159801136	0.000160781714999568	\\
-348.44637784091	0.000156084811481472	\\
-347.467595880682	0.000155205487037473	\\
-346.488813920454	0.000151986474602767	\\
-345.510031960228	0.000141923680957474	\\
-344.53125	0.000162470359700211	\\
-343.552468039772	0.000144401240070487	\\
-342.573686079546	0.000157421793840945	\\
-341.594904119318	0.000159011083969119	\\
-340.616122159092	0.000140654433804388	\\
-339.637340198864	0.000155367875877346	\\
-338.658558238636	0.000146375153494178	\\
-337.67977627841	0.000140137252570915	\\
-336.700994318182	0.000144382103474679	\\
-335.722212357954	0.000141363055328715	\\
-334.743430397728	0.000138744271223921	\\
-333.7646484375	0.000146147088818722	\\
-332.785866477272	0.000140964320318043	\\
-331.807084517046	0.000148421718565616	\\
-330.828302556818	0.000149288751551462	\\
-329.849520596592	0.000152344113430702	\\
-328.870738636364	0.000129070372937275	\\
-327.891956676136	0.000136217507316548	\\
-326.91317471591	0.000125500457921222	\\
-325.934392755682	0.000148970759481006	\\
-324.955610795454	0.000134317620036379	\\
-323.976828835228	0.000132943657289625	\\
-322.998046875	0.000128254748638131	\\
-322.019264914772	0.000140317525486932	\\
-321.040482954546	0.000131461649134001	\\
-320.061700994318	0.000140997575847143	\\
-319.082919034092	0.000124686075364928	\\
-318.104137073864	0.000129632585488753	\\
-317.125355113636	0.000102023389726029	\\
-316.14657315341	0.000124681708506034	\\
-315.167791193182	0.000130448074788304	\\
-314.189009232954	0.000132351722692438	\\
-313.210227272728	0.000139298308127154	\\
-312.2314453125	0.000127216323400669	\\
-311.252663352272	0.000114204442062945	\\
-310.273881392046	0.000130387237310371	\\
-309.295099431818	0.000145241044469826	\\
-308.316317471592	0.000127460749140974	\\
-307.337535511364	0.00012484068797014	\\
-306.358753551136	0.000105382335124353	\\
-305.37997159091	0.0001154487261005	\\
-304.401189630682	0.000113357555890417	\\
-303.422407670454	0.000108557527600684	\\
-302.443625710228	0.000115438531979831	\\
-301.46484375	0.000121007830810054	\\
-300.486061789772	9.32103077072246e-05	\\
-299.507279829546	0.000121073753900516	\\
-298.528497869318	0.000104477857937821	\\
-297.549715909092	0.000112393154739985	\\
-296.570933948864	0.000139097959360939	\\
-295.592151988636	0.000120531702406794	\\
-294.61337002841	0.000107713731767905	\\
-293.634588068182	0.000111939876795084	\\
-292.655806107954	0.000102885498277775	\\
-291.677024147728	0.00010324419095264	\\
-290.6982421875	0.000112101747931578	\\
-289.719460227272	9.36904011339184e-05	\\
-288.740678267046	0.000114747593989161	\\
-287.761896306818	0.000111617423194784	\\
-286.783114346592	0.000111847467106474	\\
-285.804332386364	0.00012541420882391	\\
-284.825550426136	0.000119471637299126	\\
-283.84676846591	0.000101860068486777	\\
-282.867986505682	0.000100200357575776	\\
-281.889204545454	0.000132053441491043	\\
-280.910422585228	8.91782424840796e-05	\\
-279.931640625	0.000131972555498486	\\
-278.952858664772	0.000153930636838786	\\
-277.974076704546	0.000120580111619444	\\
-276.995294744318	0.000142120828139295	\\
-276.016512784092	0.000131434565230875	\\
-275.037730823864	0.000133869640527961	\\
-274.058948863636	0.000150495565782251	\\
-273.08016690341	0.000140144102142511	\\
-272.101384943182	0.00014014655427457	\\
-271.122602982954	0.00014506391009367	\\
-270.143821022728	0.000137558613669841	\\
-269.1650390625	0.000139574863712037	\\
-268.186257102272	0.000154007596513996	\\
-267.207475142046	0.00014506586670119	\\
-266.228693181818	0.000135391092794402	\\
-265.249911221592	0.000124737441284668	\\
-264.271129261364	0.000124199756810285	\\
-263.292347301136	0.000108999445747105	\\
-262.31356534091	0.000115973918817741	\\
-261.334783380682	0.000146311977145849	\\
-260.356001420454	0.000137941378455446	\\
-259.377219460228	0.000164455778747134	\\
-258.3984375	0.000135438976545667	\\
-257.419655539772	9.2474560797087e-05	\\
-256.440873579546	0.000147486781188806	\\
-255.462091619318	0.000143316045460146	\\
-254.483309659092	9.52125096705028e-05	\\
-253.504527698864	7.12827770146786e-05	\\
-252.525745738636	8.53519052464383e-05	\\
-251.54696377841	9.59765968715111e-05	\\
-250.568181818182	0.0001046771871974	\\
-249.589399857954	0.000119722792880223	\\
-248.610617897728	0.000123692926095414	\\
-247.6318359375	0.000124733190642853	\\
-246.653053977272	0.000108036411785152	\\
-245.674272017046	0.000117081870787211	\\
-244.695490056818	9.41673382304855e-05	\\
-243.716708096592	0.00010617396208945	\\
-242.737926136364	0.000102030362603424	\\
-241.759144176136	0.000112628430072434	\\
-240.78036221591	0.000112227607850338	\\
-239.801580255682	0.000119497523742638	\\
-238.822798295454	0.000106703162141826	\\
-237.844016335228	9.71508082601154e-05	\\
-236.865234375	0.00010243152711372	\\
-235.886452414772	7.3789344456671e-05	\\
-234.907670454546	8.92165736626509e-05	\\
-233.928888494318	7.94281283165522e-05	\\
-232.950106534092	7.68781472288303e-05	\\
-231.971324573864	9.46775600811704e-05	\\
-230.992542613636	9.21657062869118e-05	\\
-230.01376065341	8.83037706581796e-05	\\
-229.034978693182	0.000127458348233068	\\
-228.056196732954	7.46624707937194e-05	\\
-227.077414772728	9.17342812851607e-05	\\
-226.0986328125	8.51848544814497e-05	\\
-225.119850852272	4.3482106367201e-05	\\
-224.141068892046	0.000100870826067776	\\
-223.162286931818	6.50678348397031e-05	\\
-222.183504971592	5.88661042319523e-05	\\
-221.204723011364	0.00010836948603356	\\
-220.225941051136	3.652263636493e-05	\\
-219.24715909091	0.000114547323102079	\\
-218.268377130682	5.35834347141231e-05	\\
-217.289595170454	8.04386021934509e-05	\\
-216.310813210228	7.81834205698619e-05	\\
-215.33203125	6.26912129237901e-05	\\
-214.353249289772	7.58930094869525e-05	\\
-213.374467329546	5.6097015501629e-05	\\
-212.395685369318	4.8674789242024e-05	\\
-211.416903409092	7.66575413913886e-05	\\
-210.438121448864	7.29262452003661e-05	\\
-209.459339488636	8.90041410748698e-05	\\
-208.48055752841	6.19012943348839e-05	\\
-207.501775568182	6.46610976228644e-05	\\
-206.522993607954	7.40802685802187e-05	\\
-205.544211647728	7.4273412343995e-05	\\
-204.5654296875	6.39402895422875e-05	\\
-203.586647727272	5.29036424167975e-05	\\
-202.607865767046	5.66297539811899e-05	\\
-201.629083806818	6.21583697558732e-05	\\
-200.650301846592	3.22328911970667e-05	\\
-199.671519886364	5.65611246263901e-05	\\
-198.692737926136	7.72790708798447e-05	\\
-197.71395596591	6.27249535963587e-05	\\
-196.735174005682	4.46355321484176e-05	\\
-195.756392045454	5.99730172452839e-05	\\
-194.777610085228	5.21265526499063e-05	\\
-193.798828125	4.52666493219493e-05	\\
-192.820046164772	5.19287005388992e-05	\\
-191.841264204546	5.99750912351376e-05	\\
-190.862482244318	5.70441187997368e-05	\\
-189.883700284092	7.08810250168312e-05	\\
-188.904918323864	4.50912749855283e-05	\\
-187.926136363636	6.28687630270985e-05	\\
-186.94735440341	4.92922797808303e-05	\\
-185.968572443182	5.11136379928811e-05	\\
-184.989790482954	5.9791782410357e-05	\\
-184.011008522728	4.94305180392767e-05	\\
-183.0322265625	4.9935523649234e-05	\\
-182.053444602272	4.44926258224506e-05	\\
-181.074662642046	4.44675360788865e-05	\\
-180.095880681818	4.03660744882651e-05	\\
-179.117098721592	5.40904287896924e-05	\\
-178.138316761364	4.27072795798431e-05	\\
-177.159534801136	4.02411308455519e-05	\\
-176.18075284091	3.90519339085165e-05	\\
-175.201970880682	3.69679004182215e-05	\\
-174.223188920454	5.9894990172472e-05	\\
-173.244406960228	5.72735376552048e-05	\\
-172.265625	4.087290182584e-05	\\
-171.286843039772	4.85614270558126e-05	\\
-170.308061079546	4.43616282358763e-05	\\
-169.329279119318	4.4786646305488e-05	\\
-168.350497159092	5.12604263502584e-05	\\
-167.371715198864	2.55551907977567e-05	\\
-166.392933238636	4.49302632034853e-05	\\
-165.41415127841	4.49489830536527e-05	\\
-164.435369318182	4.08230325598458e-05	\\
-163.456587357954	4.35778547305167e-05	\\
-162.477805397728	5.844990854475e-05	\\
-161.4990234375	4.581376814368e-05	\\
-160.520241477272	5.64257715629308e-05	\\
-159.541459517046	6.06849262287523e-05	\\
-158.562677556818	4.1087455485509e-05	\\
-157.583895596592	4.14242047244858e-05	\\
-156.605113636364	5.65106666839185e-05	\\
-155.626331676136	4.79854380333861e-05	\\
-154.64754971591	3.59731755659223e-05	\\
-153.668767755682	4.4029364414074e-05	\\
-152.689985795454	3.62777344680658e-05	\\
-151.711203835228	6.19480748261362e-05	\\
-150.732421875	4.03487381355498e-05	\\
-149.753639914772	3.98362064667109e-05	\\
-148.774857954546	3.07562975270311e-05	\\
-147.796075994318	5.32004132751071e-05	\\
-146.817294034092	4.40918966445944e-05	\\
-145.838512073864	3.93162433763985e-05	\\
-144.859730113636	3.70817950812149e-05	\\
-143.88094815341	3.14966404332974e-05	\\
-142.902166193182	6.50188619673545e-05	\\
-141.923384232954	3.68466564741278e-05	\\
-140.944602272728	3.30379497068845e-05	\\
-139.9658203125	4.74894817671965e-05	\\
-138.987038352272	4.44216221242509e-05	\\
-138.008256392046	5.73004735087139e-05	\\
-137.029474431818	3.98125006908516e-05	\\
-136.050692471592	6.48360308401022e-05	\\
-135.071910511364	4.85349860611299e-05	\\
-134.093128551136	2.76234461711898e-05	\\
-133.11434659091	3.74660862030409e-05	\\
-132.135564630682	5.71187341227589e-05	\\
-131.156782670454	4.0651598051865e-05	\\
-130.178000710228	5.4859452036036e-05	\\
-129.19921875	7.67810925227458e-05	\\
-128.220436789772	7.68658320780353e-05	\\
-127.241654829546	0.00011708246911103	\\
-126.262872869318	0.00014664322461089	\\
-125.284090909092	6.31477824791924e-05	\\
-124.305308948864	0.00010037840724653	\\
-123.326526988636	3.24176903231895e-05	\\
-122.34774502841	1.35518681139792e-05	\\
-121.368963068182	7.66249930153113e-05	\\
-120.390181107954	0.000270111766554618	\\
-119.411399147728	0.000211895245880749	\\
-118.4326171875	0.000112876050868535	\\
-117.453835227272	2.07345905255025e-05	\\
-116.475053267046	0.000105283776373111	\\
-115.496271306818	5.46988031016115e-05	\\
-114.517489346592	5.19123798083867e-05	\\
-113.538707386364	0.000107251645738367	\\
-112.559925426136	8.47189401289803e-05	\\
-111.58114346591	3.40129050734472e-05	\\
-110.602361505682	8.06217612429431e-05	\\
-109.623579545454	2.14049123867658e-05	\\
-108.644797585228	8.11869504195618e-05	\\
-107.666015625	0.00010443984742404	\\
-106.687233664772	7.7056737821241e-05	\\
-105.708451704546	8.17684639076552e-05	\\
-104.729669744318	7.81846332929912e-05	\\
-103.750887784092	6.66686048051278e-05	\\
-102.772105823864	7.18372577073273e-05	\\
-101.793323863636	4.39147117698867e-05	\\
-100.81454190341	5.8707135371223e-05	\\
-99.835759943182	5.53761828684255e-05	\\
-98.856977982954	4.99402017640746e-05	\\
-97.8781960227279	5.42935181480564e-05	\\
-96.8994140625	4.93795964169997e-05	\\
-95.9206321022721	8.16487398876892e-05	\\
-94.941850142046	2.99574187595776e-05	\\
-93.963068181818	4.93062904082564e-05	\\
-92.9842862215919	6.31631933004181e-05	\\
-92.005504261364	3.77586024591734e-05	\\
-91.026722301136	5.29451100549901e-05	\\
-90.0479403409099	7.19545621314077e-05	\\
-89.069158380682	3.47297899531105e-05	\\
-88.090376420454	4.28473551882929e-05	\\
-87.1115944602279	6.53134590730104e-05	\\
-86.1328125	3.43197484375053e-05	\\
-85.1540305397721	5.98141102462286e-05	\\
-84.175248579546	4.75552145266502e-05	\\
-83.196466619318	4.71241475857144e-05	\\
-82.2176846590919	7.19517894025579e-05	\\
-81.238902698864	4.72522960554445e-05	\\
-80.260120738636	4.75978572585258e-05	\\
-79.2813387784099	4.88133696208483e-05	\\
-78.302556818182	5.75935852738357e-05	\\
-77.323774857954	6.85888519301253e-05	\\
-76.3449928977279	6.98563191066642e-05	\\
-75.3662109375	4.15409583894086e-05	\\
-74.3874289772721	3.91871928142215e-05	\\
-73.408647017046	4.91027594271808e-05	\\
-72.429865056818	4.34767635161036e-05	\\
-71.4510830965919	6.6630635827021e-05	\\
-70.472301136364	6.9431201310176e-05	\\
-69.493519176136	4.93044933720795e-05	\\
-68.5147372159099	4.6271254959876e-05	\\
-67.535955255682	5.8766150114106e-05	\\
-66.557173295454	6.35566750079082e-05	\\
-65.5783913352279	4.76205006933066e-05	\\
-64.599609375	6.57656238342638e-05	\\
-63.6208274147721	6.16166356692444e-05	\\
-62.642045454546	7.99502006373799e-05	\\
-61.663263494318	4.71306325053048e-05	\\
-60.6844815340919	6.44478208288466e-05	\\
-59.705699573864	5.13375584353183e-05	\\
-58.726917613636	8.72753611805253e-05	\\
-57.7481356534099	9.05194217057188e-05	\\
-56.769353693182	6.10877720375193e-05	\\
-55.790571732954	3.42092747881036e-05	\\
-54.8117897727279	7.27967430788993e-05	\\
-53.8330078125	4.99069043617518e-05	\\
-52.8542258522721	7.09935088348252e-05	\\
-51.875443892046	8.35601950957063e-05	\\
-50.896661931818	3.75461810685269e-05	\\
-49.9178799715919	0.00018685234466754	\\
-48.939098011364	0.000103160831592034	\\
-47.960316051136	4.85438978550724e-05	\\
-46.9815340909099	8.48030575363725e-05	\\
-46.002752130682	6.53716070357524e-05	\\
-45.023970170454	0.000454326997562471	\\
-44.0451882102279	7.61075136925077e-05	\\
-43.06640625	4.54359464523047e-05	\\
-42.0876242897721	0.000105825835244653	\\
-41.108842329546	4.05121794009293e-05	\\
-40.130060369318	2.71020259794975e-05	\\
-39.1512784090919	9.68823190267101e-05	\\
-38.172496448864	1.86124582410343e-05	\\
-37.193714488636	8.31116226659717e-05	\\
-36.2149325284099	9.21128545248799e-05	\\
-35.236150568182	6.12237956226424e-05	\\
-34.257368607954	7.67964011745328e-05	\\
-33.2785866477279	2.78531328939208e-05	\\
-32.2998046875	8.23377300713245e-05	\\
-31.3210227272721	4.22288433205007e-05	\\
-30.342240767046	6.8385724114675e-06	\\
-29.363458806818	8.69909700582968e-05	\\
-28.3846768465919	0.000110314953698436	\\
-27.405894886364	5.87357930658566e-05	\\
-26.427112926136	0.000140905701222012	\\
-25.4483309659099	5.20745833372659e-05	\\
-24.469549005682	3.65562333047206e-05	\\
-23.490767045454	8.45117611529979e-05	\\
-22.5119850852279	4.83262204205793e-05	\\
-21.533203125	4.68374243604681e-05	\\
-20.5544211647721	0.000107654689356597	\\
-19.575639204546	2.38154863343682e-05	\\
-18.596857244318	8.71090715480435e-05	\\
-17.6180752840919	0.000123833121343423	\\
-16.639293323864	6.88975340229461e-05	\\
-15.660511363636	9.03183220365503e-05	\\
-14.6817294034099	0.000381431105427128	\\
-13.702947443182	0.00026300603308024	\\
-12.724165482954	0.000113050132284684	\\
-11.7453835227279	0.000108473827749725	\\
-10.7666015625	0.000151964042457809	\\
-9.78781960227207	0.000174555788341801	\\
-8.80903764204595	0.000257759769948209	\\
-7.83025568181802	0.00014653095384962	\\
-6.8514737215919	5.85750347138625e-05	\\
-5.87269176136397	0.000238021235436448	\\
-4.89390980113603	2.44449410715138e-05	\\
-3.91512784090992	0.000117247573651667	\\
-2.93634588068198	0.000232019421121568	\\
-1.95756392045405	0.000195704162326967	\\
-0.978781960227934	8.26273199719674e-05	\\
0	3.86332923715765e-05	\\
0.978781960226115	8.26273199719674e-05	\\
1.95756392045405	0.000195704162326967	\\
2.93634588068198	0.000232019421121568	\\
3.9151278409081	0.000117247573651667	\\
4.89390980113603	2.44449410715138e-05	\\
5.87269176136397	0.000238021235436448	\\
6.85147372159008	5.85750347138625e-05	\\
7.83025568181802	0.00014653095384962	\\
8.80903764204413	0.000257759769948209	\\
9.78781960227207	0.000174555788341801	\\
10.7666015625	0.000151964042457809	\\
11.7453835227261	0.000108473827749725	\\
12.724165482954	0.000113050132284684	\\
13.702947443182	0.00026300603308024	\\
14.6817294034081	0.000381431105427128	\\
15.660511363636	9.03183220365503e-05	\\
16.639293323864	6.88975340229461e-05	\\
17.6180752840901	0.000123833121343423	\\
18.596857244318	8.71090715480435e-05	\\
19.5756392045441	2.38154863343682e-05	\\
20.5544211647721	0.000107654689356597	\\
21.533203125	4.68374243604681e-05	\\
22.5119850852261	4.83262204205793e-05	\\
23.490767045454	8.45117611529979e-05	\\
24.469549005682	3.65562333047206e-05	\\
25.4483309659081	5.20745833372659e-05	\\
26.427112926136	0.000140905701222012	\\
27.405894886364	5.87357930658566e-05	\\
28.3846768465901	0.000110314953698436	\\
29.363458806818	8.69909700582968e-05	\\
30.3422407670441	6.8385724114675e-06	\\
31.3210227272721	4.22288433205007e-05	\\
32.2998046875	8.23377300713245e-05	\\
33.2785866477261	2.78531328939208e-05	\\
34.257368607954	7.67964011745328e-05	\\
35.236150568182	6.12237956226424e-05	\\
36.2149325284081	9.21128545248799e-05	\\
37.193714488636	8.31116226659717e-05	\\
38.172496448864	1.86124582410343e-05	\\
39.1512784090901	9.68823190267101e-05	\\
40.130060369318	2.71020259794975e-05	\\
41.1088423295441	4.05121794009293e-05	\\
42.0876242897721	0.000105825835244653	\\
43.06640625	4.54359464523047e-05	\\
44.0451882102261	7.61075136925077e-05	\\
45.023970170454	0.000454326997562471	\\
46.002752130682	6.53716070357524e-05	\\
46.9815340909081	8.48030575363725e-05	\\
47.960316051136	4.85438978550724e-05	\\
48.939098011364	0.000103160831592034	\\
49.9178799715901	0.00018685234466754	\\
50.896661931818	3.75461810685269e-05	\\
51.8754438920441	8.35601950957063e-05	\\
52.8542258522721	7.09935088348252e-05	\\
53.8330078125	4.99069043617518e-05	\\
54.8117897727261	7.27967430788993e-05	\\
55.790571732954	3.42092747881036e-05	\\
56.769353693182	6.10877720375193e-05	\\
57.7481356534081	9.05194217057188e-05	\\
58.726917613636	8.72753611805253e-05	\\
59.705699573864	5.13375584353183e-05	\\
60.6844815340901	6.44478208288466e-05	\\
61.663263494318	4.71306325053048e-05	\\
62.6420454545441	7.99502006373799e-05	\\
63.6208274147721	6.16166356692444e-05	\\
64.599609375	6.57656238342638e-05	\\
65.5783913352261	4.76205006933066e-05	\\
66.557173295454	6.35566750079082e-05	\\
67.535955255682	5.8766150114106e-05	\\
68.5147372159081	4.6271254959876e-05	\\
69.493519176136	4.93044933720795e-05	\\
70.472301136364	6.9431201310176e-05	\\
71.4510830965901	6.6630635827021e-05	\\
72.429865056818	4.34767635161036e-05	\\
73.4086470170441	4.91027594271808e-05	\\
74.3874289772721	3.91871928142215e-05	\\
75.3662109375	4.15409583894086e-05	\\
76.3449928977261	6.98563191066642e-05	\\
77.323774857954	6.85888519301253e-05	\\
78.302556818182	5.75935852738357e-05	\\
79.2813387784081	4.88133696208483e-05	\\
80.260120738636	4.75978572585258e-05	\\
81.238902698864	4.72522960554445e-05	\\
82.2176846590901	7.19517894025579e-05	\\
83.196466619318	4.71241475857144e-05	\\
84.1752485795441	4.75552145266502e-05	\\
85.1540305397721	5.98141102462286e-05	\\
86.1328125	3.43197484375053e-05	\\
87.1115944602261	6.53134590730104e-05	\\
88.090376420454	4.28473551882929e-05	\\
89.069158380682	3.47297899531105e-05	\\
90.0479403409081	7.19545621314077e-05	\\
91.026722301136	5.29451100549901e-05	\\
92.005504261364	3.77586024591734e-05	\\
92.9842862215901	6.31631933004181e-05	\\
93.963068181818	4.93062904082564e-05	\\
94.9418501420441	2.99574187595776e-05	\\
95.9206321022721	8.16487398876892e-05	\\
96.8994140625	4.93795964169997e-05	\\
97.8781960227261	5.42935181480564e-05	\\
98.856977982954	4.99402017640746e-05	\\
99.835759943182	5.53761828684255e-05	\\
100.814541903408	5.8707135371223e-05	\\
101.793323863636	4.39147117698867e-05	\\
102.772105823864	7.18372577073273e-05	\\
103.75088778409	6.66686048051278e-05	\\
104.729669744318	7.81846332929912e-05	\\
105.708451704544	8.17684639076552e-05	\\
106.687233664772	7.7056737821241e-05	\\
107.666015625	0.00010443984742404	\\
108.644797585226	8.11869504195618e-05	\\
109.623579545454	2.14049123867658e-05	\\
110.602361505682	8.06217612429431e-05	\\
111.581143465908	3.40129050734472e-05	\\
112.559925426136	8.47189401289803e-05	\\
113.538707386364	0.000107251645738367	\\
114.51748934659	5.19123798083867e-05	\\
115.496271306818	5.46988031016115e-05	\\
116.475053267044	0.000105283776373111	\\
117.453835227272	2.07345905255025e-05	\\
118.4326171875	0.000112876050868535	\\
119.411399147726	0.000211895245880749	\\
120.390181107954	0.000270111766554618	\\
121.368963068182	7.66249930153113e-05	\\
122.347745028408	1.35518681139792e-05	\\
123.326526988636	3.24176903231895e-05	\\
124.305308948864	0.00010037840724653	\\
125.28409090909	6.31477824791924e-05	\\
126.262872869318	0.00014664322461089	\\
127.241654829544	0.00011708246911103	\\
128.220436789772	7.68658320780353e-05	\\
129.19921875	7.67810925227458e-05	\\
130.178000710226	5.4859452036036e-05	\\
131.156782670454	4.0651598051865e-05	\\
132.135564630682	5.71187341227589e-05	\\
133.114346590908	3.74660862030409e-05	\\
134.093128551136	2.76234461711898e-05	\\
135.071910511364	4.85349860611299e-05	\\
136.05069247159	6.48360308401022e-05	\\
137.029474431818	3.98125006908516e-05	\\
138.008256392044	5.73004735087139e-05	\\
138.987038352272	4.44216221242509e-05	\\
139.9658203125	4.74894817671965e-05	\\
140.944602272726	3.30379497068845e-05	\\
141.923384232954	3.68466564741278e-05	\\
142.902166193182	6.50188619673545e-05	\\
143.880948153408	3.14966404332974e-05	\\
144.859730113636	3.70817950812149e-05	\\
145.838512073864	3.93162433763985e-05	\\
146.81729403409	4.40918966445944e-05	\\
147.796075994318	5.32004132751071e-05	\\
148.774857954544	3.07562975270311e-05	\\
149.753639914772	3.98362064667109e-05	\\
150.732421875	4.03487381355498e-05	\\
151.711203835226	6.19480748261362e-05	\\
152.689985795454	3.62777344680658e-05	\\
153.668767755682	4.4029364414074e-05	\\
154.647549715908	3.59731755659223e-05	\\
155.626331676136	4.79854380333861e-05	\\
156.605113636364	5.65106666839185e-05	\\
157.58389559659	4.14242047244858e-05	\\
158.562677556818	4.1087455485509e-05	\\
159.541459517044	6.06849262287523e-05	\\
160.520241477272	5.64257715629308e-05	\\
161.4990234375	4.581376814368e-05	\\
162.477805397726	5.844990854475e-05	\\
163.456587357954	4.35778547305167e-05	\\
164.435369318182	4.08230325598458e-05	\\
165.414151278408	4.49489830536527e-05	\\
166.392933238636	4.49302632034853e-05	\\
167.371715198864	2.55551907977567e-05	\\
168.35049715909	5.12604263502584e-05	\\
169.329279119318	4.4786646305488e-05	\\
170.308061079544	4.43616282358763e-05	\\
171.286843039772	4.85614270558126e-05	\\
172.265625	4.087290182584e-05	\\
173.244406960226	5.72735376552048e-05	\\
174.223188920454	5.9894990172472e-05	\\
175.201970880682	3.69679004182215e-05	\\
176.180752840908	3.90519339085165e-05	\\
177.159534801136	4.02411308455519e-05	\\
178.138316761364	4.27072795798431e-05	\\
179.11709872159	5.40904287896924e-05	\\
180.095880681818	4.03660744882651e-05	\\
181.074662642044	4.44675360788865e-05	\\
182.053444602272	4.44926258224506e-05	\\
183.0322265625	4.9935523649234e-05	\\
184.011008522726	4.94305180392767e-05	\\
184.989790482954	5.9791782410357e-05	\\
185.968572443182	5.11136379928811e-05	\\
186.947354403408	4.92922797808303e-05	\\
187.926136363636	6.28687630270985e-05	\\
188.904918323864	4.50912749855283e-05	\\
189.88370028409	7.08810250168312e-05	\\
190.862482244318	5.70441187997368e-05	\\
191.841264204544	5.99750912351376e-05	\\
192.820046164772	5.19287005388992e-05	\\
193.798828125	4.52666493219493e-05	\\
194.777610085226	5.21265526499063e-05	\\
195.756392045454	5.99730172452839e-05	\\
196.735174005682	4.46355321484176e-05	\\
197.713955965908	6.27249535963587e-05	\\
198.692737926136	7.72790708798447e-05	\\
199.671519886364	5.65611246263901e-05	\\
200.65030184659	3.22328911970667e-05	\\
201.629083806818	6.21583697558732e-05	\\
202.607865767044	5.66297539811899e-05	\\
203.586647727272	5.29036424167975e-05	\\
204.5654296875	6.39402895422875e-05	\\
205.544211647726	7.4273412343995e-05	\\
206.522993607954	7.40802685802187e-05	\\
207.501775568182	6.46610976228644e-05	\\
208.480557528408	6.19012943348839e-05	\\
209.459339488636	8.90041410748698e-05	\\
210.438121448864	7.29262452003661e-05	\\
211.41690340909	7.66575413913886e-05	\\
212.395685369318	4.8674789242024e-05	\\
213.374467329544	5.6097015501629e-05	\\
214.353249289772	7.58930094869525e-05	\\
215.33203125	6.26912129237901e-05	\\
216.310813210226	7.81834205698619e-05	\\
217.289595170454	8.04386021934509e-05	\\
218.268377130682	5.35834347141231e-05	\\
219.247159090908	0.000114547323102079	\\
220.225941051136	3.652263636493e-05	\\
221.204723011364	0.00010836948603356	\\
222.18350497159	5.88661042319523e-05	\\
223.162286931818	6.50678348397031e-05	\\
224.141068892044	0.000100870826067776	\\
225.119850852272	4.3482106367201e-05	\\
226.0986328125	8.51848544814497e-05	\\
227.077414772726	9.17342812851607e-05	\\
228.056196732954	7.46624707937194e-05	\\
229.034978693182	0.000127458348233068	\\
230.013760653408	8.83037706581796e-05	\\
230.992542613636	9.21657062869118e-05	\\
231.971324573864	9.46775600811704e-05	\\
232.95010653409	7.68781472288303e-05	\\
233.928888494318	7.94281283165522e-05	\\
234.907670454544	8.92165736626509e-05	\\
235.886452414772	7.3789344456671e-05	\\
236.865234375	0.00010243152711372	\\
237.844016335226	9.71508082601154e-05	\\
238.822798295454	0.000106703162141826	\\
239.801580255682	0.000119497523742638	\\
240.780362215908	0.000112227607850338	\\
241.759144176136	0.000112628430072434	\\
242.737926136364	0.000102030362603424	\\
243.71670809659	0.00010617396208945	\\
244.695490056818	9.41673382304855e-05	\\
245.674272017044	0.000117081870787211	\\
246.653053977272	0.000108036411785152	\\
247.6318359375	0.000124733190642853	\\
248.610617897726	0.000123692926095414	\\
249.589399857954	0.000119722792880223	\\
250.568181818182	0.0001046771871974	\\
251.546963778408	9.59765968715111e-05	\\
252.525745738636	8.53519052464383e-05	\\
253.504527698864	7.12827770146786e-05	\\
254.48330965909	9.52125096705028e-05	\\
255.462091619318	0.000143316045460146	\\
256.440873579544	0.000147486781188806	\\
257.419655539772	9.2474560797087e-05	\\
258.3984375	0.000135438976545667	\\
259.377219460226	0.000164455778747134	\\
260.356001420454	0.000137941378455446	\\
261.334783380682	0.000146311977145849	\\
262.313565340908	0.000115973918817741	\\
263.292347301136	0.000108999445747105	\\
264.271129261364	0.000124199756810285	\\
265.24991122159	0.000124737441284668	\\
266.228693181818	0.000135391092794402	\\
267.207475142044	0.00014506586670119	\\
268.186257102272	0.000154007596513996	\\
269.1650390625	0.000139574863712037	\\
270.143821022726	0.000137558613669841	\\
271.122602982954	0.00014506391009367	\\
272.101384943182	0.00014014655427457	\\
273.080166903408	0.000140144102142511	\\
274.058948863636	0.000150495565782251	\\
275.037730823864	0.000133869640527961	\\
276.01651278409	0.000131434565230875	\\
276.995294744318	0.000142120828139295	\\
277.974076704544	0.000120580111619444	\\
278.952858664772	0.000153930636838786	\\
279.931640625	0.000131972555498486	\\
280.910422585226	8.91782424840796e-05	\\
281.889204545454	0.000132053441491043	\\
282.867986505682	0.000100200357575776	\\
283.846768465908	0.000101860068486777	\\
284.825550426136	0.000119471637299126	\\
285.804332386364	0.00012541420882391	\\
286.78311434659	0.000111847467106474	\\
287.761896306818	0.000111617423194784	\\
288.740678267044	0.000114747593989161	\\
289.719460227272	9.36904011339184e-05	\\
290.6982421875	0.000112101747931578	\\
291.677024147726	0.00010324419095264	\\
292.655806107954	0.000102885498277775	\\
293.634588068182	0.000111939876795084	\\
294.613370028408	0.000107713731767905	\\
295.592151988636	0.000120531702406794	\\
296.570933948864	0.000139097959360939	\\
297.54971590909	0.000112393154739985	\\
298.528497869318	0.000104477857937821	\\
299.507279829544	0.000121073753900516	\\
300.486061789772	9.32103077072246e-05	\\
301.46484375	0.000121007830810054	\\
302.443625710226	0.000115438531979831	\\
303.422407670454	0.000108557527600684	\\
304.401189630682	0.000113357555890417	\\
305.379971590908	0.0001154487261005	\\
306.358753551136	0.000105382335124353	\\
307.337535511364	0.00012484068797014	\\
308.31631747159	0.000127460749140974	\\
309.295099431818	0.000145241044469826	\\
310.273881392044	0.000130387237310371	\\
311.252663352272	0.000114204442062945	\\
312.2314453125	0.000127216323400669	\\
313.210227272726	0.000139298308127154	\\
314.189009232954	0.000132351722692438	\\
315.167791193182	0.000130448074788304	\\
316.146573153408	0.000124681708506034	\\
317.125355113636	0.000102023389726029	\\
318.104137073864	0.000129632585488753	\\
319.08291903409	0.000124686075364928	\\
320.061700994318	0.000140997575847143	\\
321.040482954544	0.000131461649134001	\\
322.019264914772	0.000140317525486932	\\
322.998046875	0.000128254748638131	\\
323.976828835226	0.000132943657289625	\\
324.955610795454	0.000134317620036379	\\
325.934392755682	0.000148970759481006	\\
326.913174715908	0.000125500457921222	\\
327.891956676136	0.000136217507316548	\\
328.870738636364	0.000129070372937275	\\
329.84952059659	0.000152344113430702	\\
330.828302556818	0.000149288751551462	\\
331.807084517044	0.000148421718565616	\\
332.785866477272	0.000140964320318043	\\
333.7646484375	0.000146147088818722	\\
334.743430397726	0.000138744271223921	\\
335.722212357954	0.000141363055328715	\\
336.700994318182	0.000144382103474679	\\
337.679776278408	0.000140137252570915	\\
338.658558238636	0.000146375153494178	\\
339.637340198864	0.000155367875877346	\\
340.61612215909	0.000140654433804388	\\
341.594904119318	0.000159011083969119	\\
342.573686079544	0.000157421793840945	\\
343.552468039772	0.000144401240070487	\\
344.53125	0.000162470359700211	\\
345.510031960226	0.000141923680957474	\\
346.488813920454	0.000151986474602767	\\
347.467595880682	0.000155205487037473	\\
348.446377840908	0.000156084811481472	\\
349.425159801136	0.000160781714999568	\\
350.403941761364	0.000190685092886255	\\
351.38272372159	0.000178744545938276	\\
352.361505681818	0.000189634416046771	\\
353.340287642044	0.000179622401422498	\\
354.319069602272	0.000187085656585501	\\
355.2978515625	0.000206059567807046	\\
356.276633522726	0.000192888490848245	\\
357.255415482954	0.000194558882992859	\\
358.234197443182	0.000205966162358668	\\
359.212979403408	0.000189516871216615	\\
360.191761363636	0.000216251533996184	\\
361.170543323864	0.000180003766968209	\\
362.14932528409	0.000241394262748553	\\
363.128107244318	0.000199440704844725	\\
364.106889204544	0.000212276515971821	\\
365.085671164772	0.00020217533927124	\\
366.064453125	0.000191286322518438	\\
367.043235085226	0.000256877236615585	\\
368.022017045454	0.000232527736688923	\\
369.000799005682	0.000257951585568001	\\
369.979580965908	0.00023014733631303	\\
370.958362926136	0.00023814883348655	\\
371.937144886364	0.000232031757199944	\\
372.91592684659	0.00023464774569226	\\
373.894708806818	0.000260004856180482	\\
374.873490767044	0.000262586539365935	\\
375.852272727272	0.000252783803561825	\\
376.8310546875	0.000240622300179319	\\
377.809836647726	0.000264579912873251	\\
378.788618607954	0.000260975650765576	\\
379.767400568182	0.00027426484649614	\\
380.746182528408	0.000268986745958727	\\
381.724964488636	0.0002656651758667	\\
382.703746448864	0.00027513919736743	\\
383.68252840909	0.000279571672990918	\\
384.661310369318	0.000305980698287177	\\
385.640092329544	0.00029191439696336	\\
386.618874289772	0.000292484253018635	\\
387.59765625	0.000304218299892425	\\
388.576438210226	0.000279906697178805	\\
389.555220170454	0.000332440644803491	\\
390.534002130682	0.000352156176045803	\\
391.512784090908	0.000334920372135362	\\
392.491566051136	0.000318779028616718	\\
393.470348011364	0.000289256707137838	\\
394.44912997159	0.000325848175494846	\\
395.427911931818	0.000330558896339723	\\
396.406693892044	0.000316705559785357	\\
397.385475852272	0.000350141422802597	\\
398.3642578125	0.000344442780972123	\\
399.343039772726	0.000335347996175879	\\
400.321821732954	0.000339334182671549	\\
401.300603693182	0.000371740131441439	\\
402.279385653408	0.000349025177789184	\\
403.258167613636	0.000359404637877428	\\
404.236949573864	0.000385651933637837	\\
405.21573153409	0.000381432478103281	\\
406.194513494318	0.000391869647461938	\\
407.173295454544	0.000401458900400445	\\
408.152077414772	0.00039909352480343	\\
409.130859375	0.000414416404866264	\\
410.109641335226	0.000429532967679815	\\
411.088423295454	0.000427761652158531	\\
412.067205255682	0.000423507550467567	\\
413.045987215908	0.000443291241002881	\\
414.024769176136	0.000446524245611072	\\
415.003551136364	0.000458650930587253	\\
415.98233309659	0.000459895264768692	\\
416.961115056818	0.000443359324537912	\\
417.939897017044	0.000465667366235931	\\
418.918678977272	0.000470626738192827	\\
419.8974609375	0.000471991828523001	\\
420.876242897726	0.000471603655293965	\\
421.855024857954	0.000476848720551874	\\
422.833806818182	0.0004745582175976	\\
423.812588778408	0.000505129527003524	\\
424.791370738636	0.00051689766979051	\\
425.770152698864	0.000502617226102546	\\
426.74893465909	0.000511442871488304	\\
427.727716619318	0.000524055978654457	\\
428.706498579544	0.00051005286636233	\\
429.685280539772	0.000552018876296973	\\
430.6640625	0.00052926939770762	\\
431.642844460226	0.000534920556026856	\\
432.621626420454	0.000569593205282882	\\
433.600408380682	0.000545371592927881	\\
434.579190340908	0.000577993891373059	\\
435.557972301136	0.000574023868962524	\\
436.536754261364	0.000586780046324272	\\
437.51553622159	0.000622009571865365	\\
438.494318181818	0.000607600579138135	\\
439.473100142044	0.000616335879227731	\\
440.451882102272	0.000610276392575713	\\
441.4306640625	0.000632398666720328	\\
442.409446022726	0.000644823412556112	\\
443.388227982954	0.000645742921811967	\\
444.367009943182	0.000636157427156203	\\
445.345791903408	0.000644845883952817	\\
446.324573863636	0.000647902294599424	\\
447.303355823864	0.000646542396875525	\\
448.28213778409	0.000667259749741115	\\
449.260919744318	0.00066995684149833	\\
450.239701704544	0.00069267040744902	\\
451.218483664772	0.000687163948103886	\\
452.197265625	0.000698392741506693	\\
453.176047585226	0.00068524839865789	\\
454.154829545454	0.000702872788570825	\\
455.133611505682	0.000705585320714134	\\
456.112393465908	0.000729368771000686	\\
457.091175426136	0.000744418788944396	\\
458.069957386364	0.000765812669460104	\\
459.04873934659	0.000754560427916563	\\
460.027521306818	0.000770472445900856	\\
461.006303267044	0.000767377674245471	\\
461.985085227272	0.000778173494398689	\\
462.9638671875	0.00078474778609925	\\
463.942649147726	0.000794000049546115	\\
464.921431107954	0.000806799076931883	\\
465.900213068182	0.000801693880140547	\\
466.878995028408	0.000805353748001744	\\
467.857776988636	0.000805402297444207	\\
468.836558948864	0.000817938304929245	\\
469.81534090909	0.000812406681586047	\\
470.794122869318	0.000817323525201492	\\
471.772904829544	0.000812196035775532	\\
472.751686789772	0.000809567765039488	\\
473.73046875	0.000803694943841879	\\
474.709250710226	0.000805164307057959	\\
475.688032670454	0.000800611815102283	\\
476.666814630682	0.000797589020116901	\\
477.645596590908	0.000780170785821425	\\
478.624378551136	0.000798514908405753	\\
479.603160511364	0.000785145458883419	\\
480.58194247159	0.000779816816055381	\\
481.560724431818	0.000804416709481232	\\
482.539506392044	0.00077368546522479	\\
483.518288352272	0.000777147881358657	\\
484.4970703125	0.000757834792122798	\\
485.475852272726	0.00077342734506969	\\
486.454634232954	0.000777289261126711	\\
487.433416193182	0.000759208483684986	\\
488.412198153408	0.000749156463280417	\\
489.390980113636	0.000696150685761498	\\
490.369762073864	0.000733805666264713	\\
491.34854403409	0.000748363453545651	\\
492.327325994318	0.000761031795224785	\\
493.306107954544	0.000764562552473391	\\
494.284889914772	0.000732544160379324	\\
495.263671875	0.000684346320816259	\\
496.242453835226	0.000702453054526458	\\
497.221235795454	0.000729950606836928	\\
498.200017755682	0.000709464427306083	\\
499.178799715908	0.000797494171721587	\\
500.157581676136	0.000713859789714559	\\
501.136363636364	0.000679466337917638	\\
502.11514559659	0.000660251368986687	\\
503.093927556818	0.000692535396733239	\\
504.072709517044	0.000697623489735392	\\
505.051491477272	0.000676188901817867	\\
506.0302734375	0.000680850077971221	\\
507.009055397726	0.000668246526756626	\\
507.987837357954	0.000683246590785125	\\
508.966619318182	0.000691719932714647	\\
509.945401278408	0.00067911188208221	\\
510.924183238636	0.000719259559665077	\\
511.902965198864	0.000723877771929658	\\
512.88174715909	0.000693932098697732	\\
513.860529119318	0.00068377869236241	\\
514.839311079544	0.000728185161438411	\\
515.818093039772	0.000731333015319539	\\
516.796875	0.000715049394487831	\\
517.775656960226	0.000739895682661187	\\
518.754438920454	0.000724673022529109	\\
519.733220880682	0.000686833109358082	\\
520.712002840908	0.000672364541360367	\\
521.690784801136	0.000695399285732952	\\
522.669566761364	0.000702392742791508	\\
523.64834872159	0.000695087516456081	\\
524.627130681818	0.000701374525713751	\\
525.605912642044	0.000714137998776813	\\
526.584694602272	0.000714957363442081	\\
527.5634765625	0.000725763019290151	\\
528.542258522726	0.000742617450763622	\\
529.521040482954	0.000717300504270478	\\
530.499822443182	0.000700429872158579	\\
531.478604403408	0.000706909683516415	\\
532.457386363636	0.000722026891568053	\\
533.436168323864	0.000700655974227001	\\
534.41495028409	0.000704869252906275	\\
535.393732244318	0.000719606500336875	\\
536.372514204544	0.000718725456789808	\\
537.351296164772	0.000710520886519686	\\
538.330078125	0.000710369996094321	\\
539.308860085226	0.000710593724055228	\\
540.287642045454	0.000724236866939608	\\
541.266424005682	0.000708753078801522	\\
542.245205965908	0.000697918192219131	\\
543.223987926136	0.000698495957323456	\\
544.202769886364	0.000707524953814304	\\
545.18155184659	0.000709445013244889	\\
546.160333806818	0.000694843549207874	\\
547.139115767044	0.000689350616145491	\\
548.117897727272	0.000687488063658621	\\
549.0966796875	0.000698379007380058	\\
550.075461647726	0.000694096784599783	\\
551.054243607954	0.000684305450639966	\\
552.033025568182	0.000689040721003372	\\
553.011807528408	0.00068444646703689	\\
553.990589488636	0.000669907802275683	\\
554.969371448864	0.000666989987909872	\\
555.94815340909	0.000663332279227068	\\
556.926935369318	0.000659307586361189	\\
557.905717329544	0.000652361672308592	\\
558.884499289772	0.000660059123061524	\\
559.86328125	0.000660621037014245	\\
560.842063210226	0.000646722784563072	\\
561.820845170454	0.000648854491965655	\\
562.799627130682	0.000646057641271127	\\
563.778409090908	0.000645811123298568	\\
564.757191051136	0.000635341615052699	\\
565.735973011364	0.000628867898823465	\\
566.71475497159	0.000626760220824279	\\
567.693536931818	0.000627151593125559	\\
568.672318892044	0.00062172959036516	\\
569.651100852272	0.000634198289231003	\\
570.6298828125	0.000630052823000363	\\
571.608664772726	0.00062281226403368	\\
572.587446732954	0.000622856716715079	\\
573.566228693182	0.000621164384711626	\\
574.545010653408	0.000618933044569433	\\
575.523792613636	0.000614262072175963	\\
576.502574573864	0.000613890403304167	\\
577.48135653409	0.000603007082488287	\\
578.460138494318	0.000604774610153619	\\
579.438920454544	0.000604305315744237	\\
580.417702414772	0.000596052294797448	\\
581.396484375	0.00059759071720384	\\
582.375266335226	0.000596467143256938	\\
583.354048295454	0.000583531227181468	\\
584.332830255682	0.000594845166613946	\\
585.311612215908	0.000585530630987809	\\
586.290394176136	0.000578088770500676	\\
587.269176136364	0.000585506140687121	\\
588.24795809659	0.000583047405864624	\\
589.226740056818	0.000571304660558879	\\
590.205522017044	0.000566318585436051	\\
591.184303977272	0.000564567717641641	\\
592.1630859375	0.000567664887802017	\\
593.141867897726	0.000572517422265572	\\
594.120649857954	0.000566720771749107	\\
595.099431818182	0.000567484767310676	\\
596.078213778408	0.000559853363247804	\\
597.056995738636	0.000555765865501316	\\
598.035777698864	0.000558935182200191	\\
599.01455965909	0.000557832224416055	\\
599.993341619318	0.000590534533314193	\\
600.972123579544	0.000546751190066371	\\
601.950905539772	0.000540520206551256	\\
602.9296875	0.000536301424684235	\\
603.908469460226	0.000537803567041007	\\
604.887251420454	0.000540662202576679	\\
605.866033380682	0.000544232081105145	\\
606.844815340908	0.000537668064922131	\\
607.823597301136	0.000537705580239551	\\
608.802379261364	0.000534689144215091	\\
609.78116122159	0.000532100015206733	\\
610.759943181818	0.000523126340929024	\\
611.738725142044	0.000517157684343363	\\
612.717507102272	0.00051945497008044	\\
613.6962890625	0.000523790700241881	\\
614.675071022726	0.00052213403382999	\\
615.653852982954	0.000519006277160791	\\
616.632634943182	0.00051473569922847	\\
617.611416903408	0.000516653531862756	\\
618.590198863636	0.000513283763292571	\\
619.568980823864	0.000507848772368735	\\
620.54776278409	0.000505139966723556	\\
621.526544744318	0.000513422165584343	\\
622.505326704544	0.000502525885871015	\\
623.484108664772	0.000509976464658221	\\
624.462890625	0.00051017168785292	\\
625.441672585226	0.000501083632926134	\\
626.420454545454	0.000501235578696279	\\
627.399236505682	0.000501267034049476	\\
628.378018465908	0.000498860426666975	\\
629.356800426136	0.000495844683813666	\\
630.335582386364	0.000489281039095445	\\
631.31436434659	0.000484456237589804	\\
632.293146306818	0.000476959531923803	\\
633.271928267044	0.000483264128608659	\\
634.250710227272	0.000475670945615061	\\
635.2294921875	0.000469412109943308	\\
636.208274147726	0.000462847832447862	\\
637.187056107954	0.000473684122474706	\\
638.165838068182	0.000493103334859888	\\
639.144620028408	0.00047387751304705	\\
640.123401988636	0.00046255489839695	\\
641.102183948864	0.000460309455749447	\\
642.08096590909	0.000470477491361538	\\
643.059747869318	0.000468217475916418	\\
644.038529829544	0.000475735938382578	\\
645.017311789772	0.000471312122250115	\\
645.99609375	0.000461390675049161	\\
646.974875710226	0.000452481752912528	\\
647.953657670454	0.000460083297659286	\\
648.932439630682	0.000458760493401183	\\
649.911221590908	0.000456548345815864	\\
650.890003551136	0.000460800289773057	\\
651.868785511364	0.000452678993219936	\\
652.84756747159	0.000445653187307438	\\
653.826349431818	0.000446145870458671	\\
654.805131392044	0.000443922265129739	\\
655.783913352272	0.000441286223375637	\\
656.7626953125	0.000437472915386572	\\
657.741477272726	0.000438053426235723	\\
658.720259232954	0.000433413502229515	\\
659.699041193182	0.000432573362528639	\\
660.677823153408	0.000432544972028996	\\
661.656605113636	0.000430449470419103	\\
662.635387073864	0.000430302694161085	\\
663.61416903409	0.000426920989905532	\\
664.592950994318	0.000424308109381539	\\
665.571732954544	0.000414967785112846	\\
666.550514914772	0.000417742711770502	\\
667.529296875	0.000415237587773298	\\
668.508078835226	0.000420502234381295	\\
669.486860795454	0.000419003776997431	\\
670.465642755682	0.000413055453589533	\\
671.444424715908	0.000419474800137572	\\
672.423206676136	0.000414294509921178	\\
673.401988636364	0.000412741475047683	\\
674.38077059659	0.000413394129879789	\\
675.359552556818	0.000412785054056189	\\
676.338334517044	0.0004141613713572	\\
677.317116477272	0.000413307613983418	\\
678.2958984375	0.000409681546648084	\\
679.274680397726	0.00041155985312757	\\
680.253462357954	0.000416527744461969	\\
681.232244318182	0.000417358403350417	\\
682.211026278408	0.000417797207898671	\\
683.189808238636	0.000412389269512549	\\
684.168590198864	0.000420254930989076	\\
685.14737215909	0.000416326888810402	\\
686.126154119318	0.000419442930637181	\\
687.104936079544	0.000416845461976184	\\
688.083718039772	0.000421628579385172	\\
689.0625	0.000421444777831555	\\
690.041281960226	0.000416243450974186	\\
691.020063920454	0.000421333838773364	\\
691.998845880682	0.000421031112606315	\\
692.977627840908	0.000426474873332083	\\
693.956409801136	0.0004243002287261	\\
694.935191761364	0.000420948066929771	\\
695.91397372159	0.000427094880936272	\\
696.892755681818	0.000420409762747579	\\
697.871537642044	0.000424403624166772	\\
698.850319602272	0.000424639950118866	\\
699.8291015625	0.000440586154873072	\\
700.807883522726	0.000423634071590212	\\
701.786665482954	0.000422078740779385	\\
702.765447443182	0.000430795357056578	\\
703.744229403408	0.000430921432946656	\\
704.723011363636	0.000431033763340727	\\
705.701793323864	0.000430575797100328	\\
706.68057528409	0.000435749821181758	\\
707.659357244318	0.000444879908812344	\\
708.638139204544	0.000454707131603396	\\
709.616921164772	0.000442545749526556	\\
710.595703125	0.000446427140239383	\\
711.574485085226	0.000451178826983599	\\
712.553267045454	0.000450130472557303	\\
713.532049005682	0.000453675229419475	\\
714.510830965908	0.00044732711205269	\\
715.489612926136	0.000447213938065882	\\
716.468394886364	0.000443324171851333	\\
717.44717684659	0.000447162651191069	\\
718.425958806818	0.000447319421147237	\\
719.404740767044	0.000450451511562155	\\
720.383522727272	0.000453578643273814	\\
};
\addplot [color=blue,solid,forget plot]
  table[row sep=crcr]{
720.383522727272	0.000453578643273814	\\
721.3623046875	0.000445041366019106	\\
722.341086647726	0.000450848506192438	\\
723.319868607954	0.000440820638349307	\\
724.298650568182	0.000446540659082589	\\
725.277432528408	0.000448192673718274	\\
726.256214488636	0.00044669054560905	\\
727.234996448864	0.000450000139170865	\\
728.21377840909	0.000450263783534024	\\
729.192560369318	0.000443161422162841	\\
730.171342329544	0.000446802002737392	\\
731.150124289772	0.000449132468744214	\\
732.12890625	0.000449868868655442	\\
733.107688210226	0.000448991286958392	\\
734.086470170454	0.000447318287787492	\\
735.065252130682	0.000446615408059968	\\
736.044034090908	0.000447324913742122	\\
737.022816051136	0.000445264797474352	\\
738.001598011364	0.000439291632525042	\\
738.98037997159	0.000435953562955529	\\
739.959161931818	0.000441315374807742	\\
740.937943892044	0.000436271324705758	\\
741.916725852272	0.000446483269732432	\\
742.8955078125	0.000437421062485129	\\
743.874289772726	0.00043606824289142	\\
744.853071732954	0.000441086665548647	\\
745.831853693182	0.000431657364204076	\\
746.810635653408	0.000431905134506291	\\
747.789417613636	0.000428186545411864	\\
748.768199573864	0.000428853655175524	\\
749.74698153409	0.000420402976582785	\\
750.725763494318	0.000426715345030693	\\
751.704545454544	0.000433115069028559	\\
752.683327414772	0.000429957769786631	\\
753.662109375	0.000432783659748129	\\
754.640891335226	0.000429239608949873	\\
755.619673295454	0.000430191055626157	\\
756.598455255682	0.000432380758048506	\\
757.577237215908	0.000430784877124188	\\
758.556019176136	0.000421278867155107	\\
759.534801136364	0.000423547249296871	\\
760.51358309659	0.000426899910489845	\\
761.492365056818	0.000425622864252744	\\
762.471147017044	0.000412610582644388	\\
763.449928977272	0.00041243364231404	\\
764.4287109375	0.000415533668146299	\\
765.407492897726	0.000421480867282226	\\
766.386274857954	0.000422960109710369	\\
767.365056818182	0.000413763609277144	\\
768.343838778408	0.000412565349310782	\\
769.322620738636	0.000409535530369498	\\
770.301402698864	0.000402010609837437	\\
771.28018465909	0.000403273979977472	\\
772.258966619318	0.000413651850923412	\\
773.237748579544	0.000403252267977562	\\
774.216530539772	0.000393269332048179	\\
775.1953125	0.000408459631273616	\\
776.174094460226	0.000406868259593208	\\
777.152876420454	0.000406268482148997	\\
778.131658380682	0.000400861349911803	\\
779.110440340908	0.000407530912336004	\\
780.089222301136	0.000394718402449922	\\
781.068004261364	0.000395797164131119	\\
782.04678622159	0.000393275907185405	\\
783.025568181818	0.000391105292603901	\\
784.004350142044	0.000390719173649634	\\
784.983132102272	0.000387636804229557	\\
785.9619140625	0.000388657419616567	\\
786.940696022726	0.000387289012811762	\\
787.919477982954	0.000385049154359919	\\
788.898259943182	0.000381714504752601	\\
789.877041903408	0.000389070806949999	\\
790.855823863636	0.000379728589904276	\\
791.834605823864	0.000379335487472973	\\
792.81338778409	0.000386111745754866	\\
793.792169744318	0.000381066586259766	\\
794.770951704544	0.000378035676615638	\\
795.749733664772	0.000371718948673588	\\
796.728515625	0.000376695228202946	\\
797.707297585226	0.000379275320498739	\\
798.686079545454	0.000376847147073916	\\
799.664861505682	0.000387281215133889	\\
800.643643465908	0.000369046114368141	\\
801.622425426136	0.000372619649136604	\\
802.601207386364	0.000374168320389532	\\
803.57998934659	0.000370827821035067	\\
804.558771306818	0.000369521256212023	\\
805.537553267044	0.000368965188891843	\\
806.516335227272	0.000368200789183821	\\
807.4951171875	0.000361995561360341	\\
808.473899147726	0.000372048268117796	\\
809.452681107954	0.000365045482237448	\\
810.431463068182	0.000363886927536274	\\
811.410245028408	0.000363172008422343	\\
812.389026988636	0.00036896034213402	\\
813.367808948864	0.000361967547633767	\\
814.34659090909	0.000366458613622801	\\
815.325372869318	0.000362465570274008	\\
816.304154829544	0.000360711966742722	\\
817.282936789772	0.000362520035902964	\\
818.26171875	0.00036294444764364	\\
819.240500710226	0.000359393577059056	\\
820.219282670454	0.000359891031455629	\\
821.198064630682	0.000358498729752457	\\
822.176846590908	0.000363105310008366	\\
823.155628551136	0.000357585068134866	\\
824.134410511364	0.000359885883391077	\\
825.11319247159	0.000359270379610236	\\
826.091974431818	0.000359036038092382	\\
827.070756392044	0.000357494001937058	\\
828.049538352272	0.000363915326419756	\\
829.0283203125	0.000363884032357837	\\
830.007102272726	0.000357570939102836	\\
830.985884232954	0.000358061717848881	\\
831.964666193182	0.000360749189636149	\\
832.943448153408	0.000354317241813029	\\
833.922230113636	0.000364486639724269	\\
834.901012073864	0.000360677427059107	\\
835.87979403409	0.000359302206166514	\\
836.858575994318	0.000360433429057549	\\
837.837357954544	0.0003605584284887	\\
838.816139914772	0.000359377747321737	\\
839.794921875	0.000360110644123754	\\
840.773703835226	0.000361142948902388	\\
841.752485795454	0.000355216970878505	\\
842.731267755682	0.000358921087984156	\\
843.710049715908	0.000358559294609069	\\
844.688831676136	0.00035597203777332	\\
845.667613636364	0.000363368058403659	\\
846.64639559659	0.000358171397919013	\\
847.625177556818	0.000357949823914134	\\
848.603959517044	0.000353609559578694	\\
849.582741477272	0.000358078594971218	\\
850.5615234375	0.000350441551051811	\\
851.540305397726	0.000358589961484198	\\
852.519087357954	0.000354131832806729	\\
853.497869318182	0.000357852037607357	\\
854.476651278408	0.000362386892230666	\\
855.455433238636	0.000358453718508122	\\
856.434215198864	0.000359632554650799	\\
857.41299715909	0.000356340858112936	\\
858.391779119318	0.000357298047952243	\\
859.370561079544	0.000357498743688474	\\
860.349343039772	0.000360736020839248	\\
861.328125	0.000357512669781369	\\
862.306906960226	0.000360868955001507	\\
863.285688920454	0.000359642523312827	\\
864.264470880682	0.00035804793469584	\\
865.243252840908	0.000362382458597178	\\
866.222034801136	0.000362613826789129	\\
867.200816761364	0.000358502821206479	\\
868.17959872159	0.000358289019544122	\\
869.158380681818	0.00035864917267131	\\
870.137162642044	0.000363927347870117	\\
871.115944602272	0.000361025233637074	\\
872.0947265625	0.000362412190498973	\\
873.073508522726	0.000357048970669856	\\
874.052290482954	0.00035585660789403	\\
875.031072443182	0.000352768514256841	\\
876.009854403408	0.000355409756257656	\\
876.988636363636	0.000363271539323824	\\
877.967418323864	0.000361641759683916	\\
878.94620028409	0.000365240693119853	\\
879.924982244318	0.000365735772442759	\\
880.903764204544	0.000366424904036563	\\
881.882546164772	0.000366404096134241	\\
882.861328125	0.00036394092731962	\\
883.840110085226	0.00036025324581571	\\
884.818892045454	0.000361230281328663	\\
885.797674005682	0.000363154978405518	\\
886.776455965908	0.000358571240622669	\\
887.755237926136	0.000361837726585593	\\
888.734019886364	0.000359933791956504	\\
889.71280184659	0.000361169358879778	\\
890.691583806818	0.000358047595569464	\\
891.670365767044	0.000363537525143111	\\
892.649147727272	0.000357653769150886	\\
893.6279296875	0.000362189630660084	\\
894.606711647726	0.000368335425500639	\\
895.585493607954	0.000361543746626763	\\
896.564275568182	0.000362209631017436	\\
897.543057528408	0.000368429504088805	\\
898.521839488636	0.000357966516596942	\\
899.500621448864	0.00036115166181936	\\
900.47940340909	0.000357221791391717	\\
901.458185369318	0.000359552122564175	\\
902.436967329544	0.000369017880204785	\\
903.415749289772	0.000355075409089564	\\
904.39453125	0.00036128442024565	\\
905.373313210226	0.000358932554700845	\\
906.352095170454	0.000363672922697632	\\
907.330877130682	0.000356997248868916	\\
908.309659090908	0.000365646140837556	\\
909.288441051136	0.000362646672808588	\\
910.267223011364	0.000364303697958534	\\
911.24600497159	0.000363361601479946	\\
912.224786931818	0.00036128992342536	\\
913.203568892044	0.000362373925758919	\\
914.182350852272	0.00036513150778242	\\
915.1611328125	0.000355830032927462	\\
916.139914772726	0.000359785115175761	\\
917.118696732954	0.000354183583700345	\\
918.097478693182	0.000359778833008919	\\
919.076260653408	0.000359187143549216	\\
920.055042613636	0.000357322852762003	\\
921.033824573864	0.000355692533269434	\\
922.01260653409	0.000352613258387925	\\
922.991388494318	0.000357071946354517	\\
923.970170454544	0.000352643430864872	\\
924.948952414772	0.000352492386394158	\\
925.927734375	0.000348312120521282	\\
926.906516335226	0.000350420334413884	\\
927.885298295454	0.000349696514085033	\\
928.864080255682	0.00034534336740959	\\
929.842862215908	0.000348355824937781	\\
930.821644176136	0.000351728597500608	\\
931.800426136364	0.00034604897678682	\\
932.77920809659	0.000350135679438222	\\
933.757990056818	0.00034357718771522	\\
934.736772017044	0.000342839187385158	\\
935.715553977272	0.000347065297624397	\\
936.6943359375	0.000344863001216241	\\
937.673117897726	0.000348118459721736	\\
938.651899857954	0.000344497627316962	\\
939.630681818182	0.000342021059579273	\\
940.609463778408	0.000343129723802597	\\
941.588245738636	0.000347049602899243	\\
942.567027698864	0.0003421843139349	\\
943.54580965909	0.000344091317804076	\\
944.524591619318	0.000344452227546228	\\
945.503373579544	0.000346166249608629	\\
946.482155539772	0.000342810555561573	\\
947.4609375	0.000343869454075142	\\
948.439719460226	0.000342205720098678	\\
949.418501420454	0.000346557535714772	\\
950.397283380682	0.00033940808788847	\\
951.376065340908	0.000345103523738443	\\
952.354847301136	0.000345289180894863	\\
953.333629261364	0.000341536650198537	\\
954.31241122159	0.000338350941720834	\\
955.291193181818	0.000341264810414463	\\
956.269975142044	0.00034410397862707	\\
957.248757102272	0.000337532864960897	\\
958.2275390625	0.000342404259265784	\\
959.206321022726	0.000337449984216187	\\
960.185102982954	0.000345650792561675	\\
961.163884943182	0.000331939153835542	\\
962.142666903408	0.000338489714673545	\\
963.121448863636	0.000335343192217271	\\
964.100230823864	0.000340844196237101	\\
965.07901278409	0.000334661914373888	\\
966.057794744318	0.00033921476819718	\\
967.036576704544	0.000338942428875229	\\
968.015358664772	0.000337687553578744	\\
968.994140625	0.000340954664437745	\\
969.972922585226	0.000335469917605967	\\
970.951704545454	0.000336762942928399	\\
971.930486505682	0.000335084972996105	\\
972.909268465908	0.000338742719793713	\\
973.888050426136	0.000341299600485243	\\
974.866832386364	0.000340257568219634	\\
975.84561434659	0.000335330846314888	\\
976.824396306818	0.000342106075430266	\\
977.803178267044	0.000340439902273129	\\
978.781960227272	0.000340710044344939	\\
979.7607421875	0.000336403315347887	\\
980.739524147726	0.000338791612613386	\\
981.718306107954	0.000339020194803796	\\
982.697088068182	0.000341419229513254	\\
983.675870028408	0.000340402448952197	\\
984.654651988636	0.000336953197677078	\\
985.633433948864	0.000336884528642772	\\
986.61221590909	0.000336531914137821	\\
987.590997869318	0.000334167044685475	\\
988.569779829544	0.00033358747765949	\\
989.548561789772	0.000342024666099609	\\
990.52734375	0.000326739710024749	\\
991.506125710226	0.000339939849149441	\\
992.484907670454	0.000326041786259574	\\
993.463689630682	0.000332182273015529	\\
994.442471590908	0.000324963126601179	\\
995.421253551136	0.000329150116436786	\\
996.400035511364	0.000324578240220267	\\
997.37881747159	0.000327669274144496	\\
998.357599431818	0.000329174930174571	\\
999.336381392044	0.000317755001666636	\\
1000.31516335227	0.000315389728667589	\\
1001.2939453125	0.000320460948902998	\\
1002.27272727273	0.000318244101375238	\\
1003.25150923295	0.000319476635391106	\\
1004.23029119318	0.000313840867752007	\\
1005.20907315341	0.000315684025185178	\\
1006.18785511364	0.00031434789356399	\\
1007.16663707386	0.000312985172464529	\\
1008.14541903409	0.000313558081437493	\\
1009.12420099432	0.000310472522491393	\\
1010.10298295454	0.000306884437893201	\\
1011.08176491477	0.000302302547983654	\\
1012.060546875	0.000311254636777323	\\
1013.03932883523	0.000302979223477831	\\
1014.01811079545	0.000305421916664902	\\
1014.99689275568	0.000303404039956939	\\
1015.97567471591	0.000307754420279217	\\
1016.95445667614	0.000304177891191969	\\
1017.93323863636	0.00030563549154706	\\
1018.91202059659	0.000304193414528074	\\
1019.89080255682	0.000304759092328819	\\
1020.86958451704	0.000298191923176652	\\
1021.84836647727	0.000306618083489201	\\
1022.8271484375	0.00030486464521271	\\
1023.80593039773	0.000309151674767156	\\
1024.78471235795	0.000302268809738377	\\
1025.76349431818	0.000304633927537138	\\
1026.74227627841	0.000305372122395876	\\
1027.72105823864	0.000304029102189603	\\
1028.69984019886	0.000307757267571716	\\
1029.67862215909	0.000305841761694413	\\
1030.65740411932	0.00030618721345376	\\
1031.63618607954	0.00030725686645874	\\
1032.61496803977	0.000308452757224011	\\
1033.59375	0.000307558294074412	\\
1034.57253196023	0.000310298796745755	\\
1035.55131392045	0.000309857348976224	\\
1036.53009588068	0.000307183610895743	\\
1037.50887784091	0.000310793060678246	\\
1038.48765980114	0.000308727240151344	\\
1039.46644176136	0.000310426650186991	\\
1040.44522372159	0.000311416619943814	\\
1041.42400568182	0.000307399747060624	\\
1042.40278764204	0.000308478575860409	\\
1043.38156960227	0.000310923567610688	\\
1044.3603515625	0.000313764163413286	\\
1045.33913352273	0.000309694779775224	\\
1046.31791548295	0.000321487864687623	\\
1047.29669744318	0.000308612182460409	\\
1048.27547940341	0.000325440366361696	\\
1049.25426136364	0.000310982247318774	\\
1050.23304332386	0.000322434299760839	\\
1051.21182528409	0.000312658149454483	\\
1052.19060724432	0.000323726360766443	\\
1053.16938920454	0.000320501812608458	\\
1054.14817116477	0.000321731745827782	\\
1055.126953125	0.000325175979357225	\\
1056.10573508523	0.000322450006419818	\\
1057.08451704545	0.000329022889883417	\\
1058.06329900568	0.000326483315134597	\\
1059.04208096591	0.000334513642653533	\\
1060.02086292614	0.000331639329160326	\\
1060.99964488636	0.000332066378881608	\\
1061.97842684659	0.000334421939216228	\\
1062.95720880682	0.000336831682048821	\\
1063.93599076704	0.00033567603797191	\\
1064.91477272727	0.000342683846421267	\\
1065.8935546875	0.000344285909831319	\\
1066.87233664773	0.000340035250769789	\\
1067.85111860795	0.000339246383202782	\\
1068.82990056818	0.000349913307487546	\\
1069.80868252841	0.000343221189450719	\\
1070.78746448864	0.000345627180893296	\\
1071.76624644886	0.000340897456941581	\\
1072.74502840909	0.000350823521526987	\\
1073.72381036932	0.000346218503354227	\\
1074.70259232954	0.000341773785360043	\\
1075.68137428977	0.000345505481084489	\\
1076.66015625	0.000343269522370626	\\
1077.63893821023	0.000343421888030803	\\
1078.61772017045	0.000343140337617786	\\
1079.59650213068	0.000346770442366444	\\
1080.57528409091	0.000344362416688414	\\
1081.55406605114	0.000341382499382802	\\
1082.53284801136	0.000343748749271828	\\
1083.51162997159	0.000343530595547597	\\
1084.49041193182	0.000344818071147576	\\
1085.46919389204	0.000343617972023498	\\
1086.44797585227	0.000344454655458133	\\
1087.4267578125	0.000339556850618136	\\
1088.40553977273	0.000335521340014428	\\
1089.38432173295	0.000337563391578227	\\
1090.36310369318	0.000332327443832419	\\
1091.34188565341	0.000336371693569426	\\
1092.32066761364	0.000332380216426153	\\
1093.29944957386	0.000330016472048421	\\
1094.27823153409	0.000330163411653881	\\
1095.25701349432	0.000327938762872249	\\
1096.23579545454	0.00033141542246469	\\
1097.21457741477	0.000329711760725751	\\
1098.193359375	0.000331666672242086	\\
1099.17214133523	0.000325279403777811	\\
1100.15092329545	0.000324559709655045	\\
1101.12970525568	0.000323829990970886	\\
1102.10848721591	0.000327346434975903	\\
1103.08726917614	0.000323774071679368	\\
1104.06605113636	0.000324169039902274	\\
1105.04483309659	0.000325814049161476	\\
1106.02361505682	0.000327744487878215	\\
1107.00239701704	0.000323444751967145	\\
1107.98117897727	0.000321392849752901	\\
1108.9599609375	0.000319344854158387	\\
1109.93874289773	0.000319230855725263	\\
1110.91752485795	0.000318331590286661	\\
1111.89630681818	0.000314979438348804	\\
1112.87508877841	0.000316823157775872	\\
1113.85387073864	0.000315487479092061	\\
1114.83265269886	0.000312481836194363	\\
1115.81143465909	0.000306948235095124	\\
1116.79021661932	0.00030963294961069	\\
1117.76899857954	0.000310698648903044	\\
1118.74778053977	0.000307568139246085	\\
1119.7265625	0.000309990721293119	\\
1120.70534446023	0.000311397464488931	\\
1121.68412642045	0.000305539504157251	\\
1122.66290838068	0.000306728935894694	\\
1123.64169034091	0.00030658349306847	\\
1124.62047230114	0.000310290329291305	\\
1125.59925426136	0.000301850777640668	\\
1126.57803622159	0.000303229779179572	\\
1127.55681818182	0.000302596141833831	\\
1128.53560014204	0.000302951448589926	\\
1129.51438210227	0.000300069237098543	\\
1130.4931640625	0.000297721685115026	\\
1131.47194602273	0.000304259541829962	\\
1132.45072798295	0.000298378870882057	\\
1133.42950994318	0.000300305120511905	\\
1134.40829190341	0.000295903853565367	\\
1135.38707386364	0.000304067153654346	\\
1136.36585582386	0.000294580768878807	\\
1137.34463778409	0.000302268858697532	\\
1138.32341974432	0.000300680127224709	\\
1139.30220170454	0.000302198453423243	\\
1140.28098366477	0.000301952133298309	\\
1141.259765625	0.000306027864828746	\\
1142.23854758523	0.000300935542099772	\\
1143.21732954545	0.00030350244198999	\\
1144.19611150568	0.000305830545789182	\\
1145.17489346591	0.000302596211954312	\\
1146.15367542614	0.000308411104978462	\\
1147.13245738636	0.000306973838613318	\\
1148.11123934659	0.000308945424614496	\\
1149.09002130682	0.000313324909805466	\\
1150.06880326704	0.000308361633237842	\\
1151.04758522727	0.000308270106806825	\\
1152.0263671875	0.000309213366542687	\\
1153.00514914773	0.000310945150400733	\\
1153.98393110795	0.000314519885079509	\\
1154.96271306818	0.000311045884183165	\\
1155.94149502841	0.000312732893174119	\\
1156.92027698864	0.000315642316172	\\
1157.89905894886	0.000319592296532484	\\
1158.87784090909	0.000319761494483206	\\
1159.85662286932	0.000323188449814278	\\
1160.83540482954	0.000317739421441598	\\
1161.81418678977	0.000328157475413454	\\
1162.79296875	0.000319862265223796	\\
1163.77175071023	0.000327940023383576	\\
1164.75053267045	0.00032184586972051	\\
1165.72931463068	0.000325645403010817	\\
1166.70809659091	0.000324460549205842	\\
1167.68687855114	0.000329508094836459	\\
1168.66566051136	0.00032338971237054	\\
1169.64444247159	0.000331544367650338	\\
1170.62322443182	0.000327710437510211	\\
1171.60200639204	0.000330582810999229	\\
1172.58078835227	0.000324898673068755	\\
1173.5595703125	0.000329320143067609	\\
1174.53835227273	0.000327691056700838	\\
1175.51713423295	0.000333527691819626	\\
1176.49591619318	0.000328416444238671	\\
1177.47469815341	0.000332781191058152	\\
1178.45348011364	0.000326291475454755	\\
1179.43226207386	0.000332135439497714	\\
1180.41104403409	0.000327733037616276	\\
1181.38982599432	0.000327794735446469	\\
1182.36860795454	0.000328127830298061	\\
1183.34738991477	0.000327004033192437	\\
1184.326171875	0.000325950125210307	\\
1185.30495383523	0.000325101685602799	\\
1186.28373579545	0.000322563204391796	\\
1187.26251775568	0.000321755780278128	\\
1188.24129971591	0.000325726090328243	\\
1189.22008167614	0.000321637370469783	\\
1190.19886363636	0.000324797194762476	\\
1191.17764559659	0.000319776919649472	\\
1192.15642755682	0.000325555010354946	\\
1193.13520951704	0.00031791987931001	\\
1194.11399147727	0.000323559680025753	\\
1195.0927734375	0.000322409520734261	\\
1196.07155539773	0.000323446244487045	\\
1197.05033735795	0.000324877784185634	\\
1198.02911931818	0.000317664502258613	\\
1199.00790127841	0.000323034197156028	\\
1199.98668323864	0.000322812007848167	\\
1200.96546519886	0.000324929698708201	\\
1201.94424715909	0.000322296596576484	\\
1202.92302911932	0.000320016532001943	\\
1203.90181107954	0.000319547027720477	\\
1204.88059303977	0.00032190966575564	\\
1205.859375	0.000320510890012287	\\
1206.83815696023	0.000322072350703403	\\
1207.81693892045	0.000316106018217273	\\
1208.79572088068	0.000317606409688798	\\
1209.77450284091	0.000315087882781493	\\
1210.75328480114	0.000314857653983953	\\
1211.73206676136	0.000315669901295566	\\
1212.71084872159	0.000316589405690889	\\
1213.68963068182	0.000317345331519105	\\
1214.66841264204	0.000321309595103979	\\
1215.64719460227	0.000311906006437204	\\
1216.6259765625	0.000315915507547871	\\
1217.60475852273	0.000308741789971928	\\
1218.58354048295	0.000311461866154362	\\
1219.56232244318	0.000309416124538545	\\
1220.54110440341	0.000311287041132441	\\
1221.51988636364	0.000311648377283656	\\
1222.49866832386	0.000307629840249586	\\
1223.47745028409	0.000310328268969754	\\
1224.45623224432	0.000312726442698749	\\
1225.43501420454	0.000311949052641266	\\
1226.41379616477	0.000312010973644589	\\
1227.392578125	0.000310604324713085	\\
1228.37136008523	0.000310082764551248	\\
1229.35014204545	0.000314665825257915	\\
1230.32892400568	0.000311180756781511	\\
1231.30770596591	0.000310998962406483	\\
1232.28648792614	0.000308053781662238	\\
1233.26526988636	0.000310131569609677	\\
1234.24405184659	0.000312062590051239	\\
1235.22283380682	0.000315890050654776	\\
1236.20161576704	0.000314428942487371	\\
1237.18039772727	0.000315753245640239	\\
1238.1591796875	0.000318132109760372	\\
1239.13796164773	0.000316249487284574	\\
1240.11674360795	0.000314741911964352	\\
1241.09552556818	0.000317700493974103	\\
1242.07430752841	0.000318285705800156	\\
1243.05308948864	0.000318939799054219	\\
1244.03187144886	0.000322199951690105	\\
1245.01065340909	0.00032616657352527	\\
1245.98943536932	0.00032326502813457	\\
1246.96821732954	0.000323391120676555	\\
1247.94699928977	0.000325699477668964	\\
1248.92578125	0.000325032004090524	\\
1249.90456321023	0.000331670573787541	\\
1250.88334517045	0.000324465580484415	\\
1251.86212713068	0.000326969467363185	\\
1252.84090909091	0.000323207821926161	\\
1253.81969105114	0.000331155971938633	\\
1254.79847301136	0.000320072612466069	\\
1255.77725497159	0.000326917679488957	\\
1256.75603693182	0.000325912374574185	\\
1257.73481889204	0.000328816108679864	\\
1258.71360085227	0.000329589123418887	\\
1259.6923828125	0.000331706105665107	\\
1260.67116477273	0.000333456875740678	\\
1261.64994673295	0.000329583749628609	\\
1262.62872869318	0.000331848272701294	\\
1263.60751065341	0.000336573014144204	\\
1264.58629261364	0.000332435593192866	\\
1265.56507457386	0.00033579028091985	\\
1266.54385653409	0.000338068752072103	\\
1267.52263849432	0.000338183187403906	\\
1268.50142045454	0.000337591542295163	\\
1269.48020241477	0.00033311315479769	\\
1270.458984375	0.000335945432212293	\\
1271.43776633523	0.000337163801724772	\\
1272.41654829545	0.000343475264404884	\\
1273.39533025568	0.000336147404358038	\\
1274.37411221591	0.000339394085152136	\\
1275.35289417614	0.00033889141585354	\\
1276.33167613636	0.000341143681752154	\\
1277.31045809659	0.000337739465164644	\\
1278.28924005682	0.000346522128532485	\\
1279.26802201704	0.000339769586407272	\\
1280.24680397727	0.000342356586481721	\\
1281.2255859375	0.000340488432954019	\\
1282.20436789773	0.000342924026321429	\\
1283.18314985795	0.000339070203735209	\\
1284.16193181818	0.000342043792537556	\\
1285.14071377841	0.000339937612706636	\\
1286.11949573864	0.000338751197218641	\\
1287.09827769886	0.000344203415941358	\\
1288.07705965909	0.000339010265298016	\\
1289.05584161932	0.000342358621408591	\\
1290.03462357954	0.000340162926187272	\\
1291.01340553977	0.00033796259101491	\\
1291.9921875	0.000342357862620552	\\
1292.97096946023	0.000340279727241216	\\
1293.94975142045	0.000335378148434344	\\
1294.92853338068	0.000337683544620549	\\
1295.90731534091	0.000336998287647506	\\
1296.88609730114	0.000334576231445864	\\
1297.86487926136	0.000333030231629166	\\
1298.84366122159	0.000334091140667236	\\
1299.82244318182	0.00033767515527372	\\
1300.80122514204	0.000333612821579971	\\
1301.78000710227	0.000333902171468764	\\
1302.7587890625	0.000334342786813072	\\
1303.73757102273	0.000332001208629352	\\
1304.71635298295	0.000335488429099215	\\
1305.69513494318	0.000333853360337313	\\
1306.67391690341	0.000335919146127847	\\
1307.65269886364	0.000334047816164701	\\
1308.63148082386	0.000336366559242183	\\
1309.61026278409	0.000338861437208805	\\
1310.58904474432	0.00034102573703228	\\
1311.56782670454	0.000334972503111293	\\
1312.54660866477	0.000333792228881977	\\
1313.525390625	0.000342486453268794	\\
1314.50417258523	0.000332892477505677	\\
1315.48295454545	0.000339243117613303	\\
1316.46173650568	0.000332968772081657	\\
1317.44051846591	0.00033843260771956	\\
1318.41930042614	0.000333876518912244	\\
1319.39808238636	0.000335740833047007	\\
1320.37686434659	0.000337452622184823	\\
1321.35564630682	0.000340391845891586	\\
1322.33442826704	0.000337239492994245	\\
1323.31321022727	0.000340078739145278	\\
1324.2919921875	0.000336622149792072	\\
1325.27077414773	0.000336733962350813	\\
1326.24955610795	0.000336859285669586	\\
1327.22833806818	0.000340797791787026	\\
1328.20712002841	0.000333008297713279	\\
1329.18590198864	0.000338117558776997	\\
1330.16468394886	0.000336492067712299	\\
1331.14346590909	0.000336475627112993	\\
1332.12224786932	0.000334228883459863	\\
1333.10102982954	0.000336727000081561	\\
1334.07981178977	0.000334530003013581	\\
1335.05859375	0.000334217723735148	\\
1336.03737571023	0.000332529624965626	\\
1337.01615767045	0.000334685358163893	\\
1337.99493963068	0.000334427812980036	\\
1338.97372159091	0.000332804293505161	\\
1339.95250355114	0.000331792442241525	\\
1340.93128551136	0.000335988960085066	\\
1341.91006747159	0.000330694521041026	\\
1342.88884943182	0.000335368096740331	\\
1343.86763139204	0.000333843367148132	\\
1344.84641335227	0.000332379550092162	\\
1345.8251953125	0.000331272329743951	\\
1346.80397727273	0.000329080314179537	\\
1347.78275923295	0.000333545311803946	\\
1348.76154119318	0.000332359744749378	\\
1349.74032315341	0.000327891270959323	\\
1350.71910511364	0.000327303479077768	\\
1351.69788707386	0.000326082195512349	\\
1352.67666903409	0.000326822240282933	\\
1353.65545099432	0.000325658840761659	\\
1354.63423295454	0.000328729884986168	\\
1355.61301491477	0.000321991240087777	\\
1356.591796875	0.00032376364495934	\\
1357.57057883523	0.00032415576238197	\\
1358.54936079545	0.000320369872198001	\\
1359.52814275568	0.000322278556767916	\\
1360.50692471591	0.00032438424350655	\\
1361.48570667614	0.000322004313465323	\\
1362.46448863636	0.000323996625834431	\\
1363.44327059659	0.000325537447306009	\\
1364.42205255682	0.000322098700574279	\\
1365.40083451704	0.000325396865472035	\\
1366.37961647727	0.000323317001829902	\\
1367.3583984375	0.000323862577703979	\\
1368.33718039773	0.000325494362011385	\\
1369.31596235795	0.000317382945212469	\\
1370.29474431818	0.000326906900883031	\\
1371.27352627841	0.000316711167953629	\\
1372.25230823864	0.000322783509656643	\\
1373.23109019886	0.000316785057683362	\\
1374.20987215909	0.000321998968031843	\\
1375.18865411932	0.000319438693473834	\\
1376.16743607954	0.000318609238452216	\\
1377.14621803977	0.00031643546688707	\\
1378.125	0.000317559138700551	\\
1379.10378196023	0.000318902465757387	\\
1380.08256392045	0.000315826025752121	\\
1381.06134588068	0.000310283667046371	\\
1382.04012784091	0.000315749223210676	\\
1383.01890980114	0.000315904009566405	\\
1383.99769176136	0.000313313476666835	\\
1384.97647372159	0.000308988480525104	\\
1385.95525568182	0.000313378255978623	\\
1386.93403764204	0.000310161688531593	\\
1387.91281960227	0.000316138863325759	\\
1388.8916015625	0.000307473824713359	\\
1389.87038352273	0.000311529807201982	\\
1390.84916548295	0.000310714022514924	\\
1391.82794744318	0.000308071934876239	\\
1392.80672940341	0.000308521445304498	\\
1393.78551136364	0.000306801486934409	\\
1394.76429332386	0.000309064534840423	\\
1395.74307528409	0.00031290051051572	\\
1396.72185724432	0.000316488620567731	\\
1397.70063920454	0.000311091129102713	\\
1398.67942116477	0.000316962866726154	\\
1399.658203125	0.000319401497168963	\\
1400.63698508523	0.000321327094296582	\\
1401.61576704545	0.000320448561055529	\\
1402.59454900568	0.000321771525704547	\\
1403.57333096591	0.000317417354204143	\\
1404.55211292614	0.000321378145612414	\\
1405.53089488636	0.000321721113901135	\\
1406.50967684659	0.000321970170885423	\\
1407.48845880682	0.000330494606284882	\\
1408.46724076704	0.000324466231605776	\\
1409.44602272727	0.000325240318061154	\\
1410.4248046875	0.000323857698807975	\\
1411.40358664773	0.000333033903120249	\\
1412.38236860795	0.000327833374240234	\\
1413.36115056818	0.000334141807484871	\\
1414.33993252841	0.00033651592324309	\\
1415.31871448864	0.000337759505638317	\\
1416.29749644886	0.000340207137485812	\\
1417.27627840909	0.000339738155065009	\\
1418.25506036932	0.000345293607941831	\\
1419.23384232954	0.000342439363622643	\\
1420.21262428977	0.000345467286316257	\\
1421.19140625	0.000348828826433975	\\
1422.17018821023	0.000352383564468268	\\
1423.14897017045	0.000355142448372006	\\
1424.12775213068	0.000356780956602596	\\
1425.10653409091	0.000356004742243205	\\
1426.08531605114	0.000358547694278342	\\
1427.06409801136	0.000364034766591774	\\
1428.04287997159	0.00036612551541637	\\
1429.02166193182	0.00036763249408382	\\
1430.00044389204	0.000367427091380924	\\
1430.97922585227	0.000369946944162816	\\
1431.9580078125	0.000366296238706924	\\
1432.93678977273	0.00037397295339973	\\
1433.91557173295	0.000367457198367911	\\
1434.89435369318	0.00037158514563368	\\
1435.87313565341	0.000375362027366137	\\
1436.85191761364	0.000368403433008487	\\
1437.83069957386	0.000377760762636969	\\
1438.80948153409	0.000376366563728858	\\
1439.78826349432	0.000377953874451972	\\
1440.76704545454	0.000386230907976723	\\
1441.74582741477	0.000383872672241426	\\
1442.724609375	0.0003844141029607	\\
1443.70339133523	0.000391528020535591	\\
1444.68217329545	0.000390683423014431	\\
1445.66095525568	0.000393422846313272	\\
1446.63973721591	0.000396162116693471	\\
1447.61851917614	0.00040345002560008	\\
1448.59730113636	0.000401504279515867	\\
1449.57608309659	0.000406140866392021	\\
1450.55486505682	0.00040703583729055	\\
1451.53364701704	0.000414729987474357	\\
1452.51242897727	0.000418177970714619	\\
1453.4912109375	0.00041895782843127	\\
1454.46999289773	0.000417125746659943	\\
1455.44877485795	0.000423033190491329	\\
1456.42755681818	0.000429219808524528	\\
1457.40633877841	0.00042723615606915	\\
1458.38512073864	0.000433266813941701	\\
1459.36390269886	0.000430896906808554	\\
1460.34268465909	0.000440989980948655	\\
1461.32146661932	0.000438402025323104	\\
1462.30024857954	0.0004456828151563	\\
1463.27903053977	0.000446499754224226	\\
1464.2578125	0.000450528025199402	\\
1465.23659446023	0.000454472332173789	\\
1466.21537642045	0.000458258679850596	\\
1467.19415838068	0.000462030483906254	\\
1468.17294034091	0.000466434309177002	\\
1469.15172230114	0.000463528421871612	\\
1470.13050426136	0.000466147594295103	\\
1471.10928622159	0.000473153162937933	\\
1472.08806818182	0.000474519592124671	\\
1473.06685014204	0.000482345599171597	\\
1474.04563210227	0.000485804381778565	\\
1475.0244140625	0.000484503231653959	\\
1476.00319602273	0.000489662431824225	\\
1476.98197798295	0.000491766616082791	\\
1477.96075994318	0.000490854497015948	\\
1478.93954190341	0.000497286784811959	\\
1479.91832386364	0.000503788641238143	\\
1480.89710582386	0.000506403749406407	\\
1481.87588778409	0.000501823666646573	\\
1482.85466974432	0.000508325445054018	\\
1483.83345170454	0.000511539693418598	\\
1484.81223366477	0.000509577989997884	\\
1485.791015625	0.000518803358404013	\\
1486.76979758523	0.000516540178553688	\\
1487.74857954545	0.000525171717764583	\\
1488.72736150568	0.000525050985582149	\\
1489.70614346591	0.000532242531918237	\\
1490.68492542614	0.000525954690760402	\\
1491.66370738636	0.000535264692864405	\\
1492.64248934659	0.000534465835361288	\\
1493.62127130682	0.000539573355455154	\\
1494.60005326704	0.000540120141443895	\\
1495.57883522727	0.000544679909756037	\\
1496.5576171875	0.000547667076451431	\\
1497.53639914773	0.000555352125316557	\\
1498.51518110795	0.000554919839946777	\\
1499.49396306818	0.000563506571140241	\\
1500.47274502841	0.000567119368605163	\\
1501.45152698864	0.000571049421973608	\\
1502.43030894886	0.00057416862239937	\\
1503.40909090909	0.000577637054009965	\\
1504.38787286932	0.000580449871822854	\\
1505.36665482954	0.000586153200706658	\\
1506.34543678977	0.000587499344712148	\\
1507.32421875	0.000591461423354261	\\
1508.30300071023	0.00059920255092149	\\
1509.28178267045	0.000600842514046326	\\
1510.26056463068	0.000609929364309395	\\
1511.23934659091	0.000612903772343956	\\
1512.21812855114	0.000609680973350296	\\
1513.19691051136	0.000621620505040699	\\
1514.17569247159	0.000621478156677358	\\
1515.15447443182	0.000623634134726635	\\
1516.13325639204	0.000630483591471177	\\
1517.11203835227	0.000637385642899642	\\
1518.0908203125	0.0006400972492654	\\
1519.06960227273	0.000637242207005659	\\
1520.04838423295	0.000647732131889774	\\
1521.02716619318	0.000644012565971856	\\
1522.00594815341	0.000650688065982599	\\
1522.98473011364	0.000656484454674625	\\
1523.96351207386	0.000651190100963596	\\
1524.94229403409	0.000652967096833202	\\
1525.92107599432	0.00065402924212659	\\
1526.89985795454	0.000657643254155682	\\
1527.87863991477	0.000647152908226443	\\
1528.857421875	0.000650192947136624	\\
1529.83620383523	0.000644479328705218	\\
1530.81498579545	0.000643454213062822	\\
1531.79376775568	0.000640604960710366	\\
1532.77254971591	0.000638075176367731	\\
1533.75133167614	0.000636367335564592	\\
1534.73011363636	0.000629143436286444	\\
1535.70889559659	0.000631957241118865	\\
1536.68767755682	0.000630286143777146	\\
1537.66645951704	0.000628066909771238	\\
1538.64524147727	0.000625000507300999	\\
1539.6240234375	0.000629639975728645	\\
1540.60280539773	0.00062728122828287	\\
1541.58158735795	0.000624973935894413	\\
1542.56036931818	0.000627589231385993	\\
1543.53915127841	0.00062660805142211	\\
1544.51793323864	0.000628643602727507	\\
1545.49671519886	0.000623505216000636	\\
1546.47549715909	0.000627177327864932	\\
1547.45427911932	0.000623557020474331	\\
1548.43306107954	0.000627772290041343	\\
1549.41184303977	0.000625414593101839	\\
1550.390625	0.000627299262275015	\\
1551.36940696023	0.000629024034691461	\\
1552.34818892045	0.000626973413559476	\\
1553.32697088068	0.000635496541103804	\\
1554.30575284091	0.000640132683091975	\\
1555.28453480114	0.000638942829602197	\\
1556.26331676136	0.000640768683475767	\\
1557.24209872159	0.00064684590213547	\\
1558.22088068182	0.000644158625338134	\\
1559.19966264204	0.000653991420825161	\\
1560.17844460227	0.00065297675979225	\\
1561.1572265625	0.000658028727985594	\\
1562.13600852273	0.000666020895643107	\\
1563.11479048295	0.000663262298123515	\\
1564.09357244318	0.000663828284080712	\\
1565.07235440341	0.000668849250413043	\\
1566.05113636364	0.000670183119017725	\\
1567.02991832386	0.000672679205434066	\\
1568.00870028409	0.000673773105714961	\\
1568.98748224432	0.000675371600389721	\\
1569.96626420454	0.000677128011843145	\\
1570.94504616477	0.000682021793576858	\\
1571.923828125	0.000682580644568324	\\
1572.90261008523	0.000682209007831244	\\
1573.88139204545	0.000683492014163985	\\
1574.86017400568	0.00068180706194907	\\
1575.83895596591	0.000681949664862463	\\
1576.81773792614	0.000684392747463241	\\
1577.79651988636	0.000678781806779802	\\
1578.77530184659	0.000686575743159273	\\
1579.75408380682	0.000679984304589047	\\
1580.73286576704	0.000679748699423136	\\
1581.71164772727	0.000674922425762462	\\
1582.6904296875	0.000672832703937879	\\
1583.66921164773	0.000669075839604476	\\
1584.64799360795	0.000665267498460164	\\
1585.62677556818	0.000669347962599351	\\
1586.60555752841	0.000662259838916439	\\
1587.58433948864	0.00065359912325178	\\
1588.56312144886	0.000655418775286853	\\
1589.54190340909	0.000657640859738976	\\
1590.52068536932	0.000652244294815227	\\
1591.49946732954	0.000653789883449657	\\
1592.47824928977	0.000648957605882257	\\
1593.45703125	0.000650152050066192	\\
1594.43581321023	0.000645284639201299	\\
1595.41459517045	0.000647298967104256	\\
1596.39337713068	0.00064380105685615	\\
1597.37215909091	0.00064510224739289	\\
1598.35094105114	0.000638181403265997	\\
1599.32972301136	0.000634115739374592	\\
1600.30850497159	0.000629027788571618	\\
1601.28728693182	0.000629879351225097	\\
1602.26606889204	0.000625029784824313	\\
1603.24485085227	0.000622458634190472	\\
1604.2236328125	0.000613981446364822	\\
1605.20241477273	0.000610845129452451	\\
1606.18119673295	0.00060479434776571	\\
1607.15997869318	0.000605157914480917	\\
1608.13876065341	0.000597481769905494	\\
1609.11754261364	0.000595230488052676	\\
1610.09632457386	0.000593483856157959	\\
1611.07510653409	0.000591583276387177	\\
1612.05388849432	0.000582391015003052	\\
1613.03267045454	0.000582493767394898	\\
1614.01145241477	0.000577914986739419	\\
1614.990234375	0.000580022990810944	\\
1615.96901633523	0.000578740259984784	\\
1616.94779829545	0.000574240383470377	\\
1617.92658025568	0.000570301098838708	\\
1618.90536221591	0.000570760894328513	\\
1619.88414417614	0.000574204913283541	\\
1620.86292613636	0.00056880928634668	\\
1621.84170809659	0.000572201145112737	\\
1622.82049005682	0.000565755392706546	\\
1623.79927201704	0.000571351149176994	\\
1624.77805397727	0.000569746923896686	\\
1625.7568359375	0.000561954466412596	\\
1626.73561789773	0.00056991478899999	\\
1627.71439985795	0.000574599410970509	\\
1628.69318181818	0.000574297619659546	\\
1629.67196377841	0.000575856472165377	\\
1630.65074573864	0.000574434087330951	\\
1631.62952769886	0.000578688613339141	\\
1632.60830965909	0.000583036797521097	\\
1633.58709161932	0.000579749866701612	\\
1634.56587357954	0.000585542791089123	\\
1635.54465553977	0.000587919852181399	\\
1636.5234375	0.000587495928191551	\\
1637.50221946023	0.000589139880844365	\\
1638.48100142045	0.000586649503442429	\\
1639.45978338068	0.000592175685702277	\\
1640.43856534091	0.000593054844912966	\\
1641.41734730114	0.000595301682229377	\\
1642.39612926136	0.000595361013476345	\\
1643.37491122159	0.000600975318189418	\\
1644.35369318182	0.000603806229763558	\\
1645.33247514204	0.000604288100065146	\\
1646.31125710227	0.000610244210470536	\\
1647.2900390625	0.000616467449849702	\\
1648.26882102273	0.00061776498084666	\\
1649.24760298295	0.000617930835170663	\\
1650.22638494318	0.000619454820667979	\\
1651.20516690341	0.000628783747352898	\\
1652.18394886364	0.000632967555494033	\\
1653.16273082386	0.000634684050045935	\\
1654.14151278409	0.000634129787563707	\\
1655.12029474432	0.000642229712869736	\\
1656.09907670454	0.000641270181782412	\\
1657.07785866477	0.000642534393031592	\\
1658.056640625	0.000643887459646833	\\
1659.03542258523	0.000643660462320559	\\
1660.01420454545	0.000644351981883953	\\
1660.99298650568	0.00064632085366451	\\
1661.97176846591	0.000656259729351336	\\
1662.95055042614	0.000649706239915703	\\
1663.92933238636	0.000650939492417663	\\
1664.90811434659	0.000654023670519504	\\
1665.88689630682	0.000654407415598499	\\
1666.86567826704	0.00065923332215048	\\
1667.84446022727	0.000658286601883906	\\
1668.8232421875	0.000660104874495851	\\
1669.80202414773	0.000665524551330595	\\
1670.78080610795	0.000660667723806901	\\
1671.75958806818	0.000667636190033173	\\
1672.73837002841	0.000669973799858834	\\
1673.71715198864	0.000665495300683976	\\
1674.69593394886	0.000675474543780318	\\
1675.67471590909	0.000671580335013745	\\
1676.65349786932	0.000671779136179109	\\
1677.63227982954	0.000672283187243664	\\
1678.61106178977	0.000674631776492834	\\
1679.58984375	0.000664690392621821	\\
1680.56862571023	0.000675119637235731	\\
1681.54740767045	0.000671352350207489	\\
1682.52618963068	0.000674427190699324	\\
1683.50497159091	0.000672871354840892	\\
1684.48375355114	0.000674778095655927	\\
1685.46253551136	0.000675379981387754	\\
1686.44131747159	0.000670950659886681	\\
1687.42009943182	0.000670624973523453	\\
1688.39888139204	0.000671450323614509	\\
1689.37766335227	0.000670018898996225	\\
1690.3564453125	0.000665214182054544	\\
1691.33522727273	0.000662898847903121	\\
1692.31400923295	0.000661230376903143	\\
1693.29279119318	0.000660341529495992	\\
1694.27157315341	0.000653575982854397	\\
1695.25035511364	0.000648838452731498	\\
1696.22913707386	0.000649241967363488	\\
1697.20791903409	0.00063876463257986	\\
1698.18670099432	0.000633074806981033	\\
1699.16548295454	0.000629136438254021	\\
1700.14426491477	0.000628195430951056	\\
1701.123046875	0.000633970540520204	\\
1702.10182883523	0.000634761518194081	\\
1703.08061079545	0.000630365237636544	\\
1704.05939275568	0.00062832014504044	\\
1705.03817471591	0.000630033669124645	\\
1706.01695667614	0.000620586587997614	\\
1706.99573863636	0.000626456343226703	\\
1707.97452059659	0.000626773663234063	\\
1708.95330255682	0.000625219241319493	\\
1709.93208451704	0.000627794713942724	\\
1710.91086647727	0.00062152831372475	\\
1711.8896484375	0.00061368829419967	\\
1712.86843039773	0.000616199770722226	\\
1713.84721235795	0.000609031777770082	\\
1714.82599431818	0.000606734218127657	\\
1715.80477627841	0.000610998913805304	\\
1716.78355823864	0.000606676404966251	\\
1717.76234019886	0.000609891036392875	\\
1718.74112215909	0.000609132620463476	\\
1719.71990411932	0.000605656002148306	\\
1720.69868607954	0.000605358227677281	\\
1721.67746803977	0.000600722205930348	\\
1722.65625	0.000597802662356271	\\
1723.63503196023	0.000596288274597268	\\
1724.61381392045	0.000593592914512002	\\
1725.59259588068	0.000596616737645682	\\
1726.57137784091	0.000591070044050455	\\
1727.55015980114	0.000588596888522002	\\
1728.52894176136	0.000589892531720699	\\
1729.50772372159	0.000584394148118052	\\
1730.48650568182	0.000581453327752165	\\
1731.46528764204	0.000584959254554408	\\
1732.44406960227	0.000583130660074624	\\
1733.4228515625	0.000577186635814033	\\
1734.40163352273	0.000583619299712424	\\
1735.38041548295	0.000582339225765997	\\
1736.35919744318	0.000585629492428035	\\
1737.33797940341	0.000585433725277173	\\
1738.31676136364	0.000585494256857855	\\
1739.29554332386	0.000589584866381618	\\
1740.27432528409	0.000585714337729151	\\
1741.25310724432	0.000588220355971448	\\
1742.23188920454	0.000583580063395316	\\
1743.21067116477	0.000583808724764426	\\
1744.189453125	0.000584518510584918	\\
1745.16823508523	0.000587262367555125	\\
1746.14701704545	0.000581799600748843	\\
1747.12579900568	0.000583275210953882	\\
1748.10458096591	0.000579182736896883	\\
1749.08336292614	0.000586090367081916	\\
1750.06214488636	0.000584184756913481	\\
1751.04092684659	0.000580306054044476	\\
1752.01970880682	0.00057758615773402	\\
1752.99849076704	0.000578661502705437	\\
1753.97727272727	0.00057656583735483	\\
1754.9560546875	0.000578121946444224	\\
1755.93483664773	0.000579498022373472	\\
1756.91361860795	0.000571905290344365	\\
1757.89240056818	0.000573890658908948	\\
1758.87118252841	0.000572178421750014	\\
1759.84996448864	0.000571604402323157	\\
1760.82874644886	0.000572885681225869	\\
1761.80752840909	0.000570614371929593	\\
1762.78631036932	0.000569535664944741	\\
1763.76509232954	0.000571559561250173	\\
1764.74387428977	0.000565724932130722	\\
1765.72265625	0.000567713038890776	\\
1766.70143821023	0.000567147783813101	\\
1767.68022017045	0.000564354667369644	\\
1768.65900213068	0.000561728304816505	\\
1769.63778409091	0.000560215041375951	\\
1770.61656605114	0.000557448695808287	\\
1771.59534801136	0.000560244108158295	\\
1772.57412997159	0.000554505844370661	\\
1773.55291193182	0.000554635593885558	\\
1774.53169389204	0.00055519694044645	\\
1775.51047585227	0.000551443183641368	\\
1776.4892578125	0.000546793752271042	\\
1777.46803977273	0.000541336882474835	\\
1778.44682173295	0.000546535500839788	\\
1779.42560369318	0.000547257196208022	\\
1780.40438565341	0.000544240376600658	\\
1781.38316761364	0.000542663437709115	\\
1782.36194957386	0.000541308615903312	\\
1783.34073153409	0.000539336533955844	\\
1784.31951349432	0.000538502870895218	\\
1785.29829545454	0.000534377955803003	\\
1786.27707741477	0.000537703175066844	\\
1787.255859375	0.000536702538208114	\\
1788.23464133523	0.00053624562733977	\\
1789.21342329545	0.000536511602300891	\\
1790.19220525568	0.000533704170281513	\\
1791.17098721591	0.000533831459621175	\\
1792.14976917614	0.000533760866512239	\\
1793.12855113636	0.000529268542353878	\\
1794.10733309659	0.000528594953384224	\\
1795.08611505682	0.000526685956046625	\\
1796.06489701704	0.000521344956511302	\\
1797.04367897727	0.000523970655381201	\\
1798.0224609375	0.000519293431808208	\\
1799.00124289773	0.000519192220190028	\\
1799.98002485795	0.000518828551895239	\\
1800.95880681818	0.000512338674166331	\\
1801.93758877841	0.000510050666944359	\\
1802.91637073864	0.00051062812487803	\\
1803.89515269886	0.000509858426594418	\\
1804.87393465909	0.000510133603211201	\\
1805.85271661932	0.000502850855873495	\\
1806.83149857954	0.000505135646870643	\\
1807.81028053977	0.000501744272391781	\\
1808.7890625	0.000494422133704232	\\
1809.76784446023	0.000499172905687967	\\
1810.74662642045	0.000490158962967859	\\
1811.72540838068	0.000491329896419524	\\
1812.70419034091	0.000491705471638404	\\
1813.68297230114	0.000480323604392023	\\
1814.66175426136	0.000487738847149405	\\
1815.64053622159	0.000489517981087009	\\
1816.61931818182	0.000485985652186629	\\
1817.59810014204	0.000485498249126857	\\
1818.57688210227	0.000478694650573023	\\
1819.5556640625	0.000479167341818438	\\
1820.53444602273	0.000472122035665012	\\
1821.51322798295	0.000473795079276656	\\
1822.49200994318	0.000469027452096033	\\
1823.47079190341	0.000464644687556032	\\
1824.44957386364	0.0004640002059863	\\
1825.42835582386	0.000460575810896779	\\
1826.40713778409	0.000454942975153657	\\
1827.38591974432	0.000454066242198313	\\
1828.36470170454	0.000447831179922502	\\
1829.34348366477	0.000441056467907933	\\
1830.322265625	0.000444181096280731	\\
1831.30104758523	0.000439708387441764	\\
1832.27982954545	0.000436077361945966	\\
1833.25861150568	0.000433161709552025	\\
1834.23739346591	0.000426478903227998	\\
1835.21617542614	0.000430448992304758	\\
1836.19495738636	0.000425435750483026	\\
1837.17373934659	0.000417527898492894	\\
1838.15252130682	0.000419202070784375	\\
1839.13130326704	0.000415910757225383	\\
1840.11008522727	0.000416729734923823	\\
1841.0888671875	0.00041431856290398	\\
1842.06764914773	0.000407950620843069	\\
1843.04643110795	0.000409629893664215	\\
1844.02521306818	0.00040216453492324	\\
1845.00399502841	0.000401560248792356	\\
1845.98277698864	0.000404196449701007	\\
1846.96155894886	0.000397909957438221	\\
1847.94034090909	0.000396668358453259	\\
1848.91912286932	0.000393662983915964	\\
1849.89790482954	0.000392246721016914	\\
1850.87668678977	0.000392101266044215	\\
1851.85546875	0.000386544248980247	\\
1852.83425071023	0.000382179654030131	\\
1853.81303267045	0.000383957495528657	\\
1854.79181463068	0.000380719091086814	\\
1855.77059659091	0.000377202800153288	\\
1856.74937855114	0.000372739354102266	\\
1857.72816051136	0.000370468906823479	\\
1858.70694247159	0.000370107443104972	\\
1859.68572443182	0.000366900044367508	\\
1860.66450639204	0.000357579353731034	\\
1861.64328835227	0.000359850424992643	\\
1862.6220703125	0.000357797154995798	\\
1863.60085227273	0.000360301170690631	\\
1864.57963423295	0.000357499801576864	\\
1865.55841619318	0.00035301169368135	\\
1866.53719815341	0.00035343243952732	\\
1867.51598011364	0.000351044143424854	\\
1868.49476207386	0.000349606772943809	\\
1869.47354403409	0.00034870164693069	\\
1870.45232599432	0.000348223385161989	\\
1871.43110795454	0.000341911340018586	\\
1872.40988991477	0.000344006105781949	\\
1873.388671875	0.0003442287491678	\\
1874.36745383523	0.000344075130188303	\\
1875.34623579545	0.000347688657107931	\\
1876.32501775568	0.000347575988117473	\\
1877.30379971591	0.000346315568668048	\\
1878.28258167614	0.000345612503018002	\\
1879.26136363636	0.000341314277115351	\\
1880.24014559659	0.000340602578716757	\\
1881.21892755682	0.000338494361048044	\\
1882.19770951704	0.000338947513118915	\\
1883.17649147727	0.000340067610318941	\\
1884.1552734375	0.000336225659959507	\\
1885.13405539773	0.000339024471225563	\\
1886.11283735795	0.000334284341766601	\\
1887.09161931818	0.000330018087709634	\\
1888.07040127841	0.000329783311385037	\\
1889.04918323864	0.000322813185277696	\\
1890.02796519886	0.000325274602997532	\\
1891.00674715909	0.000320854302388855	\\
1891.98552911932	0.000322349063752288	\\
1892.96431107954	0.000319735333007497	\\
1893.94309303977	0.000320908846678231	\\
1894.921875	0.000313751826310608	\\
1895.90065696023	0.000314688795776001	\\
1896.87943892045	0.000313577924975102	\\
1897.85822088068	0.000312227477501777	\\
1898.83700284091	0.000308504097550723	\\
1899.81578480114	0.000304159092734534	\\
1900.79456676136	0.000305596683324454	\\
1901.77334872159	0.000303634997553136	\\
1902.75213068182	0.000301024406395621	\\
1903.73091264204	0.000298025806026721	\\
1904.70969460227	0.000301063973755327	\\
1905.6884765625	0.000294470288424788	\\
1906.66725852273	0.000289419772890747	\\
1907.64604048295	0.000284546235926162	\\
1908.62482244318	0.000287609416164273	\\
1909.60360440341	0.00028325579999696	\\
1910.58238636364	0.000273043711696756	\\
1911.56116832386	0.000274332180770315	\\
1912.53995028409	0.000277075835111636	\\
1913.51873224432	0.000272695998949203	\\
1914.49751420454	0.000269445938196846	\\
1915.47629616477	0.000269330629987006	\\
1916.455078125	0.000267960117447649	\\
1917.43386008523	0.000261691362212589	\\
1918.41264204545	0.000258229463197138	\\
1919.39142400568	0.000254988728139077	\\
1920.37020596591	0.000255808400656811	\\
1921.34898792614	0.000252873660602047	\\
1922.32776988636	0.000250589115891391	\\
1923.30655184659	0.000245162522425086	\\
1924.28533380682	0.000243908400878493	\\
1925.26411576704	0.000241384616933135	\\
1926.24289772727	0.000242298464739942	\\
1927.2216796875	0.000233197409547999	\\
1928.20046164773	0.00022866512489373	\\
1929.17924360795	0.00022893712186055	\\
1930.15802556818	0.000226379565177529	\\
1931.13680752841	0.000226044908372449	\\
1932.11558948864	0.000223140331764476	\\
1933.09437144886	0.000218268608602501	\\
1934.07315340909	0.000211726323749391	\\
1935.05193536932	0.000210347113172151	\\
1936.03071732954	0.00021103362822989	\\
1937.00949928977	0.000205003039295772	\\
1937.98828125	0.000197092340801801	\\
1938.96706321023	0.000198564524330994	\\
1939.94584517045	0.00019126575430914	\\
1940.92462713068	0.000187652897909247	\\
1941.90340909091	0.000186554775010758	\\
1942.88219105114	0.000187200627134809	\\
1943.86097301136	0.000179527896643211	\\
1944.83975497159	0.000176096586661828	\\
1945.81853693182	0.000177606371041097	\\
1946.79731889204	0.000173830536461948	\\
1947.77610085227	0.000167541979167074	\\
1948.7548828125	0.00016172453506016	\\
1949.73366477273	0.000160870259741115	\\
1950.71244673295	0.000158574175692318	\\
1951.69122869318	0.000154819790239631	\\
1952.67001065341	0.000150462591931463	\\
1953.64879261364	0.000149807528791539	\\
1954.62757457386	0.00014951423127497	\\
1955.60635653409	0.000140289515700449	\\
1956.58513849432	0.000138465005990341	\\
1957.56392045454	0.000139524754395204	\\
1958.54270241477	0.000133074015490224	\\
1959.521484375	0.000130495278260394	\\
1960.50026633523	0.000126370986558272	\\
1961.47904829545	0.000126816029300493	\\
1962.45783025568	0.000123266529137007	\\
1963.43661221591	0.000121476945689561	\\
1964.41539417614	0.000116298507083717	\\
1965.39417613636	0.000113948164038818	\\
1966.37295809659	0.000111594054320631	\\
1967.35174005682	0.000104749427491777	\\
1968.33052201704	0.000101394860802496	\\
1969.30930397727	0.000103960456583585	\\
1970.2880859375	0.000102707714971143	\\
1971.26686789773	9.65719298173626e-05	\\
1972.24564985795	9.64103136522388e-05	\\
1973.22443181818	9.17902208695251e-05	\\
1974.20321377841	8.95726625715138e-05	\\
1975.18199573864	8.35816387238715e-05	\\
1976.16077769886	8.31049073776718e-05	\\
1977.13955965909	7.99260118231738e-05	\\
1978.11834161932	7.44569288698151e-05	\\
1979.09712357954	7.46147110634756e-05	\\
1980.07590553977	6.82879577809948e-05	\\
1981.0546875	7.09813781356162e-05	\\
1982.03346946023	6.92599182604714e-05	\\
1983.01225142045	6.16215567979818e-05	\\
1983.99103338068	6.21920002921307e-05	\\
1984.96981534091	5.9376780490924e-05	\\
1985.94859730114	5.82044529263225e-05	\\
1986.92737926136	5.61304999468018e-05	\\
1987.90616122159	5.59279585238717e-05	\\
1988.88494318182	5.20939489206574e-05	\\
1989.86372514204	5.0618936660624e-05	\\
1990.84250710227	4.49296320281333e-05	\\
1991.8212890625	4.50011965976149e-05	\\
1992.80007102273	3.95916257067998e-05	\\
1993.77885298295	3.94164049345877e-05	\\
1994.75763494318	3.56223183065848e-05	\\
1995.73641690341	3.28769036011884e-05	\\
1996.71519886364	3.16190677014573e-05	\\
1997.69398082386	2.7218674997871e-05	\\
1998.67276278409	1.93103467854148e-05	\\
1999.65154474432	2.51487938910366e-05	\\
2000.63032670454	2.28870761016013e-05	\\
2001.60910866477	2.08781186344482e-05	\\
2002.587890625	1.74403176096298e-05	\\
2003.56667258523	1.92123428537741e-05	\\
2004.54545454545	1.5149431559193e-05	\\
2005.52423650568	1.04682360444182e-05	\\
2006.50301846591	1.04233276930869e-05	\\
2007.48180042614	1.06917134519788e-05	\\
2008.46058238636	6.21654873652613e-06	\\
2009.43936434659	3.69930079920909e-06	\\
2010.41814630682	5.27184423898744e-06	\\
2011.39692826704	6.54294044037294e-06	\\
2012.37571022727	5.19508502038622e-06	\\
2013.3544921875	6.99630063720067e-06	\\
2014.33327414773	5.14888247973786e-06	\\
2015.31205610795	7.64169863180601e-06	\\
2016.29083806818	8.75094765722526e-06	\\
2017.26962002841	9.0506861901178e-06	\\
2018.24840198864	1.31689124197014e-05	\\
2019.22718394886	1.11581703323986e-05	\\
2020.20596590909	1.44539032586076e-05	\\
2021.18474786932	1.74099471008649e-05	\\
2022.16352982954	1.81084581151927e-05	\\
2023.14231178977	1.87831118490923e-05	\\
2024.12109375	2.08495933712796e-05	\\
2025.09987571023	2.08035248376758e-05	\\
2026.07865767045	2.44220440493527e-05	\\
2027.05743963068	2.54016927122705e-05	\\
2028.03622159091	2.61658209822513e-05	\\
2029.01500355114	2.7584492911238e-05	\\
2029.99378551136	2.78612940028551e-05	\\
2030.97256747159	2.42211680380562e-05	\\
2031.95134943182	2.96266855210577e-05	\\
2032.93013139204	2.57847013443228e-05	\\
2033.90891335227	3.04661005407207e-05	\\
2034.8876953125	3.64530949104441e-05	\\
2035.86647727273	2.90551789960893e-05	\\
2036.84525923295	2.99429137307778e-05	\\
2037.82404119318	3.06283963577057e-05	\\
2038.80282315341	3.05935869623907e-05	\\
2039.78160511364	3.32938281089078e-05	\\
2040.76038707386	3.68466187544548e-05	\\
2041.73916903409	3.02832213151942e-05	\\
2042.71795099432	3.49141559558745e-05	\\
2043.69673295454	3.17596813723438e-05	\\
2044.67551491477	3.58606871092792e-05	\\
2045.654296875	3.28373359565008e-05	\\
2046.63307883523	3.29324486648671e-05	\\
2047.61186079545	3.47347486177089e-05	\\
2048.59064275568	3.80041458486795e-05	\\
2049.56942471591	3.85180133547127e-05	\\
2050.54820667614	3.46586762708203e-05	\\
2051.52698863636	3.84921931681034e-05	\\
2052.50577059659	3.68764500282288e-05	\\
2053.48455255682	4.09930403647184e-05	\\
2054.46333451704	3.4667675695049e-05	\\
2055.44211647727	3.88784203533842e-05	\\
2056.4208984375	3.89526013481517e-05	\\
2057.39968039773	4.11027115188376e-05	\\
2058.37846235795	3.90133104629514e-05	\\
2059.35724431818	4.07406615114303e-05	\\
2060.33602627841	3.80205272902788e-05	\\
2061.31480823864	4.17656665330426e-05	\\
2062.29359019886	3.89521685734964e-05	\\
2063.27237215909	3.85999862659472e-05	\\
2064.25115411932	4.07350421569936e-05	\\
2065.22993607954	3.97605451323997e-05	\\
2066.20871803977	4.31783231171223e-05	\\
2067.1875	4.09062310137809e-05	\\
2068.16628196023	4.25164487427647e-05	\\
2069.14506392045	4.24272832627179e-05	\\
2070.12384588068	4.23941771838257e-05	\\
2071.10262784091	4.17110749268099e-05	\\
2072.08140980114	4.58461771770281e-05	\\
2073.06019176136	4.1673528750919e-05	\\
2074.03897372159	4.22949480477163e-05	\\
2075.01775568182	4.61082223539162e-05	\\
2075.99653764204	4.3123034597401e-05	\\
2076.97531960227	4.58105020952118e-05	\\
2077.9541015625	4.25125149163331e-05	\\
2078.93288352273	4.12443924696675e-05	\\
2079.91166548295	4.63227589351866e-05	\\
2080.89044744318	4.56607863441921e-05	\\
2081.86922940341	4.10161885137469e-05	\\
2082.84801136364	4.58941147901089e-05	\\
2083.82679332386	4.74935669810165e-05	\\
2084.80557528409	4.16824565607019e-05	\\
2085.78435724432	4.49716228477513e-05	\\
2086.76313920454	4.03728148109159e-05	\\
2087.74192116477	4.10267968876772e-05	\\
2088.720703125	4.52583450993066e-05	\\
2089.69948508523	4.34587932530007e-05	\\
2090.67826704545	4.42760113965598e-05	\\
2091.65704900568	4.53743715903457e-05	\\
2092.63583096591	4.37311675914129e-05	\\
2093.61461292614	4.41960986236894e-05	\\
2094.59339488636	4.75116767665358e-05	\\
2095.57217684659	4.46255644363266e-05	\\
2096.55095880682	4.59110628149754e-05	\\
2097.52974076704	4.48147985680148e-05	\\
2098.50852272727	4.47716393020923e-05	\\
2099.4873046875	4.65587860867088e-05	\\
2100.46608664773	4.81048251581092e-05	\\
2101.44486860795	4.6941631787116e-05	\\
2102.42365056818	4.33184803225706e-05	\\
2103.40243252841	4.45158021459453e-05	\\
2104.38121448864	4.57930998798147e-05	\\
2105.35999644886	4.78220772972167e-05	\\
2106.33877840909	4.29105488460104e-05	\\
2107.31756036932	4.62902550078992e-05	\\
2108.29634232954	4.56285711777607e-05	\\
2109.27512428977	4.73054937042176e-05	\\
2110.25390625	4.53296453553866e-05	\\
2111.23268821023	4.4346877909237e-05	\\
2112.21147017045	4.30811739293004e-05	\\
2113.19025213068	4.38607093863149e-05	\\
2114.16903409091	4.67745030629035e-05	\\
2115.14781605114	4.46295438999884e-05	\\
2116.12659801136	5.03354282666185e-05	\\
2117.10537997159	4.63088540136544e-05	\\
2118.08416193182	4.64981894966271e-05	\\
2119.06294389204	4.63893333250215e-05	\\
2120.04172585227	4.25541003427858e-05	\\
2121.0205078125	4.1735686456902e-05	\\
2121.99928977273	4.48908796973114e-05	\\
2122.97807173295	4.55274508369932e-05	\\
2123.95685369318	4.28589359187242e-05	\\
2124.93563565341	4.25338382736516e-05	\\
2125.91441761364	4.38975346700251e-05	\\
2126.89319957386	4.55616785039408e-05	\\
2127.87198153409	4.59384075409383e-05	\\
2128.85076349432	4.35345946775957e-05	\\
2129.82954545454	4.83607688883001e-05	\\
2130.80832741477	4.5652333627324e-05	\\
2131.787109375	4.66013443936659e-05	\\
2132.76589133523	4.75433288985883e-05	\\
2133.74467329545	4.40467631993037e-05	\\
2134.72345525568	4.49752648078924e-05	\\
2135.70223721591	4.56244248763674e-05	\\
2136.68101917614	4.57296551872686e-05	\\
2137.65980113636	4.68719534415898e-05	\\
2138.63858309659	4.13880402559019e-05	\\
2139.61736505682	4.15040820223293e-05	\\
2140.59614701704	4.3892593629549e-05	\\
2141.57492897727	4.40325115562613e-05	\\
2142.5537109375	4.2067788403555e-05	\\
2143.53249289773	4.4422299323682e-05	\\
2144.51127485795	4.27050803971275e-05	\\
2145.49005681818	4.2267639712788e-05	\\
2146.46883877841	4.57738804952212e-05	\\
2147.44762073864	4.63558271751998e-05	\\
2148.42640269886	4.26676530103327e-05	\\
2149.40518465909	4.45683200664184e-05	\\
2150.38396661932	4.27614430721946e-05	\\
2151.36274857954	4.4322649151619e-05	\\
2152.34153053977	4.54302484459923e-05	\\
2153.3203125	4.45114491332108e-05	\\
2154.29909446023	4.11824856901105e-05	\\
2155.27787642045	4.28693605914596e-05	\\
2156.25665838068	4.38569125564729e-05	\\
2157.23544034091	3.9968923232183e-05	\\
2158.21422230114	4.37193618713594e-05	\\
2159.19300426136	3.90601153131864e-05	\\
2160.17178622159	4.05443873076942e-05	\\
2161.15056818182	4.05390869249973e-05	\\
2162.12935014204	4.12353736097467e-05	\\
2163.10813210227	4.10934923261606e-05	\\
2164.0869140625	4.54076410626733e-05	\\
2165.06569602273	3.97043965640063e-05	\\
2166.04447798295	4.30000633502915e-05	\\
2167.02325994318	4.13769372864272e-05	\\
2168.00204190341	4.01725954077288e-05	\\
2168.98082386364	4.10585583487203e-05	\\
2169.95960582386	4.22805841274327e-05	\\
2170.93838778409	4.12322678575292e-05	\\
2171.91716974432	4.45715138547893e-05	\\
2172.89595170454	3.84187820517969e-05	\\
2173.87473366477	4.14838445743737e-05	\\
2174.853515625	4.13471743533938e-05	\\
2175.83229758523	4.16622316709541e-05	\\
2176.81107954545	4.10701852277502e-05	\\
2177.78986150568	4.01047329936582e-05	\\
2178.76864346591	3.79354930330172e-05	\\
2179.74742542614	3.90603077078313e-05	\\
2180.72620738636	3.81155112185085e-05	\\
2181.70498934659	4.23627015421186e-05	\\
2182.68377130682	3.96080062440808e-05	\\
2183.66255326704	4.18860398846851e-05	\\
2184.64133522727	3.84897528354467e-05	\\
2185.6201171875	3.98513966160201e-05	\\
2186.59889914773	3.75217477226414e-05	\\
2187.57768110795	3.94074527626326e-05	\\
2188.55646306818	4.05236415850614e-05	\\
2189.53524502841	4.01366847624898e-05	\\
2190.51402698864	3.84806189437007e-05	\\
2191.49280894886	3.89307387839001e-05	\\
2192.47159090909	4.20404323205583e-05	\\
2193.45037286932	3.78344651601823e-05	\\
2194.42915482954	3.99091066889672e-05	\\
2195.40793678977	4.11160833395993e-05	\\
2196.38671875	3.75203881680602e-05	\\
2197.36550071023	3.7823827173766e-05	\\
2198.34428267045	3.96849904562913e-05	\\
2199.32306463068	3.80403295714478e-05	\\
2200.30184659091	3.7984710470771e-05	\\
2201.28062855114	3.71113503704911e-05	\\
2202.25941051136	3.88617362345523e-05	\\
2203.23819247159	3.71904880867608e-05	\\
2204.21697443182	3.95080501084159e-05	\\
2205.19575639204	3.90047465555597e-05	\\
2206.17453835227	4.16545004755364e-05	\\
2207.1533203125	3.68821363359198e-05	\\
2208.13210227273	3.8681364744335e-05	\\
2209.11088423295	4.07272663966033e-05	\\
2210.08966619318	3.63922641826674e-05	\\
2211.06844815341	3.80276562529982e-05	\\
2212.04723011364	4.14523366106673e-05	\\
2213.02601207386	3.46023167414842e-05	\\
2214.00479403409	4.17075035255965e-05	\\
2214.98357599432	3.68726230512169e-05	\\
2215.96235795454	3.88670249235125e-05	\\
2216.94113991477	3.62348412276371e-05	\\
2217.919921875	3.68923257181398e-05	\\
2218.89870383523	3.74278517220395e-05	\\
2219.87748579545	3.88680418643614e-05	\\
2220.85626775568	3.98631115683516e-05	\\
2221.83504971591	3.71664724269401e-05	\\
2222.81383167614	3.54114069807791e-05	\\
2223.79261363636	3.91929636706829e-05	\\
2224.77139559659	3.7289716237892e-05	\\
2225.75017755682	3.81472473978971e-05	\\
2226.72895951704	3.62093560461636e-05	\\
2227.70774147727	4.00526433062529e-05	\\
2228.6865234375	3.75402419545454e-05	\\
2229.66530539773	4.00650512721277e-05	\\
2230.64408735795	3.73590363781108e-05	\\
2231.62286931818	3.7689162478252e-05	\\
2232.60165127841	3.6290336199729e-05	\\
2233.58043323864	3.67463222964136e-05	\\
2234.55921519886	3.69419975919397e-05	\\
2235.53799715909	3.8962072961599e-05	\\
2236.51677911932	3.60388659770186e-05	\\
2237.49556107954	3.93855082157725e-05	\\
2238.47434303977	3.81254721084168e-05	\\
2239.453125	3.78116139094702e-05	\\
2240.43190696023	3.67817511125022e-05	\\
2241.41068892045	3.61895706048588e-05	\\
2242.38947088068	3.75117307263353e-05	\\
2243.36825284091	3.99491634262686e-05	\\
2244.34703480114	3.92012805823409e-05	\\
2245.32581676136	3.74181277467878e-05	\\
2246.30459872159	3.39984838923706e-05	\\
2247.28338068182	3.90208876215411e-05	\\
2248.26216264204	3.91366144945425e-05	\\
2249.24094460227	3.98819432976716e-05	\\
2250.2197265625	4.03246394299573e-05	\\
2251.19850852273	3.82713051110705e-05	\\
2252.17729048295	3.8249390014748e-05	\\
2253.15607244318	3.80618261604687e-05	\\
2254.13485440341	3.87875475424754e-05	\\
2255.11363636364	3.53526841131004e-05	\\
2256.09241832386	4.00390749818585e-05	\\
2257.07120028409	4.1431596680544e-05	\\
2258.04998224432	4.04069085141025e-05	\\
2259.02876420454	3.99458449444316e-05	\\
2260.00754616477	4.06844079850499e-05	\\
2260.986328125	3.83359504396517e-05	\\
2261.96511008523	3.95393655811713e-05	\\
2262.94389204545	4.17127003896498e-05	\\
2263.92267400568	3.86936956303814e-05	\\
2264.90145596591	4.22876935358686e-05	\\
2265.88023792614	4.05501685889252e-05	\\
2266.85901988636	4.27593507030579e-05	\\
2267.83780184659	3.99687669177714e-05	\\
2268.81658380682	3.9834458407795e-05	\\
2269.79536576704	3.74148895067546e-05	\\
2270.77414772727	4.13586244398889e-05	\\
2271.7529296875	4.14068725887413e-05	\\
2272.73171164773	4.22507421517922e-05	\\
2273.71049360795	3.98451934365557e-05	\\
2274.68927556818	4.0594243530569e-05	\\
2275.66805752841	4.09830505075139e-05	\\
2276.64683948864	4.29536895723665e-05	\\
2277.62562144886	4.30905452456153e-05	\\
2278.60440340909	4.36770276691326e-05	\\
2279.58318536932	3.96321755026458e-05	\\
2280.56196732954	4.10683695332873e-05	\\
2281.54074928977	4.34482400146214e-05	\\
2282.51953125	4.0382651471049e-05	\\
2283.49831321023	4.22172451116417e-05	\\
2284.47709517045	4.22280537460472e-05	\\
2285.45587713068	4.13253313977231e-05	\\
2286.43465909091	4.01070336677292e-05	\\
2287.41344105114	4.25818933294535e-05	\\
2288.39222301136	4.24949100924144e-05	\\
2289.37100497159	4.32356951361937e-05	\\
2290.34978693182	4.48721494385149e-05	\\
2291.32856889204	3.99844364082718e-05	\\
2292.30735085227	4.04183261518104e-05	\\
2293.2861328125	4.13831550452349e-05	\\
2294.26491477273	4.34261755103306e-05	\\
2295.24369673295	4.3321344184931e-05	\\
2296.22247869318	4.44133782569275e-05	\\
2297.20126065341	4.03146944572552e-05	\\
2298.18004261364	4.43550342872027e-05	\\
2299.15882457386	4.24559671621872e-05	\\
2300.13760653409	4.41097516562877e-05	\\
2301.11638849432	4.23737093210821e-05	\\
2302.09517045454	4.38810145743407e-05	\\
2303.07395241477	4.42277238119126e-05	\\
2304.052734375	4.09711944903239e-05	\\
2305.03151633523	4.2637975224156e-05	\\
2306.01029829545	4.47480935248064e-05	\\
2306.98908025568	4.42281480352729e-05	\\
2307.96786221591	4.2201192299901e-05	\\
2308.94664417614	4.2455336936605e-05	\\
2309.92542613636	4.54235200812785e-05	\\
2310.90420809659	4.61703708740261e-05	\\
2311.88299005682	3.93189577036531e-05	\\
2312.86177201704	4.09859326731342e-05	\\
2313.84055397727	3.98763231866166e-05	\\
2314.8193359375	4.2491914898498e-05	\\
2315.79811789773	4.36963948750771e-05	\\
2316.77689985795	4.30074101769248e-05	\\
2317.75568181818	4.1975483705692e-05	\\
2318.73446377841	4.42466890942763e-05	\\
2319.71324573864	4.42268032492523e-05	\\
2320.69202769886	4.2121360553302e-05	\\
2321.67080965909	3.91104483214459e-05	\\
2322.64959161932	4.20899060834961e-05	\\
2323.62837357954	4.47509746756291e-05	\\
2324.60715553977	4.41680443825302e-05	\\
2325.5859375	3.97093451805517e-05	\\
2326.56471946023	4.21311391527949e-05	\\
2327.54350142045	4.33571309489464e-05	\\
2328.52228338068	3.89542333528138e-05	\\
2329.50106534091	4.33077929632842e-05	\\
2330.47984730114	4.01196478328427e-05	\\
2331.45862926136	4.22780311704674e-05	\\
2332.43741122159	4.02412454227842e-05	\\
2333.41619318182	4.44758178260021e-05	\\
2334.39497514204	4.01394637609116e-05	\\
2335.37375710227	3.8646293981072e-05	\\
2336.3525390625	4.16419087288869e-05	\\
2337.33132102273	4.23784877580783e-05	\\
2338.31010298295	4.59658385949722e-05	\\
2339.28888494318	4.10943182273212e-05	\\
2340.26766690341	4.42854635505754e-05	\\
2341.24644886364	4.13725378488857e-05	\\
2342.22523082386	3.77189842685547e-05	\\
2343.20401278409	4.13470721524656e-05	\\
2344.18279474432	3.82556287392553e-05	\\
2345.16157670454	3.84147447605877e-05	\\
2346.14035866477	3.98391743599582e-05	\\
2347.119140625	4.16737975676661e-05	\\
2348.09792258523	3.91434321573833e-05	\\
2349.07670454545	4.03650112457786e-05	\\
2350.05548650568	3.74025628015646e-05	\\
2351.03426846591	4.1376876638517e-05	\\
2352.01305042614	4.04478312581714e-05	\\
2352.99183238636	4.14611541294044e-05	\\
2353.97061434659	4.08035012525228e-05	\\
2354.94939630682	4.08459217195228e-05	\\
2355.92817826704	4.07339659568597e-05	\\
2356.90696022727	3.71045419791081e-05	\\
2357.8857421875	4.21725290450767e-05	\\
2358.86452414773	4.38022291890418e-05	\\
2359.84330610795	4.45516875262124e-05	\\
2360.82208806818	3.77851486848144e-05	\\
2361.80087002841	3.66405446804211e-05	\\
2362.77965198864	4.4403376851365e-05	\\
2363.75843394886	4.34313353672442e-05	\\
2364.73721590909	3.85288066413634e-05	\\
2365.71599786932	3.75395171879081e-05	\\
2366.69477982954	3.86886460333673e-05	\\
2367.67356178977	4.08008515748499e-05	\\
2368.65234375	4.0124190003563e-05	\\
2369.63112571023	3.73289982170374e-05	\\
2370.60990767045	3.66037126304705e-05	\\
2371.58868963068	4.15768830907451e-05	\\
2372.56747159091	3.45508204723533e-05	\\
2373.54625355114	3.76368295527051e-05	\\
2374.52503551136	4.0420173069877e-05	\\
2375.50381747159	3.96315815586641e-05	\\
2376.48259943182	3.66212363192233e-05	\\
2377.46138139204	3.77389222426152e-05	\\
2378.44016335227	3.67652261601878e-05	\\
2379.4189453125	3.70398602234748e-05	\\
2380.39772727273	3.84879955632507e-05	\\
2381.37650923295	3.92773651454703e-05	\\
2382.35529119318	3.6939644177041e-05	\\
2383.33407315341	3.80599055738151e-05	\\
2384.31285511364	3.56180074752992e-05	\\
2385.29163707386	3.79133914296867e-05	\\
2386.27041903409	4.09318803911507e-05	\\
2387.24920099432	4.00466556461509e-05	\\
2388.22798295454	3.8087538783604e-05	\\
2389.20676491477	3.75226874889335e-05	\\
2390.185546875	3.61949294518884e-05	\\
2391.16432883523	3.50947063057155e-05	\\
2392.14311079545	3.53002681348643e-05	\\
2393.12189275568	3.49818714203217e-05	\\
2394.10067471591	3.838414839212e-05	\\
2395.07945667614	3.42795961316019e-05	\\
2396.05823863636	3.54817723210152e-05	\\
2397.03702059659	3.38998857107721e-05	\\
2398.01580255682	3.33428507451887e-05	\\
2398.99458451704	3.40365084736552e-05	\\
2399.97336647727	3.69771558832445e-05	\\
2400.9521484375	3.46195748203281e-05	\\
2401.93093039773	3.56192800602889e-05	\\
2402.90971235795	3.54233631855874e-05	\\
2403.88849431818	3.76285591227474e-05	\\
2404.86727627841	3.66841723732158e-05	\\
2405.84605823864	3.53742570064402e-05	\\
2406.82484019886	3.28785935294017e-05	\\
2407.80362215909	3.62802292958243e-05	\\
2408.78240411932	3.49145301270616e-05	\\
2409.76118607954	3.52347678040514e-05	\\
2410.73996803977	3.88080289883775e-05	\\
2411.71875	3.36300284303144e-05	\\
2412.69753196023	3.32592003686008e-05	\\
2413.67631392045	3.61485832076398e-05	\\
2414.65509588068	3.44539746569922e-05	\\
2415.63387784091	3.67359526795406e-05	\\
2416.61265980114	3.31106123955289e-05	\\
2417.59144176136	3.24615024463098e-05	\\
2418.57022372159	3.279573821521e-05	\\
2419.54900568182	3.52750667194167e-05	\\
2420.52778764204	3.82119905104643e-05	\\
2421.50656960227	3.46752727280727e-05	\\
2422.4853515625	3.26751799521101e-05	\\
2423.46413352273	3.3295725565957e-05	\\
2424.44291548295	3.64566380361821e-05	\\
2425.42169744318	3.38947656095298e-05	\\
2426.40047940341	3.19360460312752e-05	\\
2427.37926136364	3.55649544731469e-05	\\
2428.35804332386	3.73605952680397e-05	\\
2429.33682528409	3.8180782659714e-05	\\
2430.31560724432	3.62235799546294e-05	\\
2431.29438920454	3.59710736952565e-05	\\
2432.27317116477	3.67793689034032e-05	\\
2433.251953125	3.73388803603807e-05	\\
2434.23073508523	3.33420184118105e-05	\\
2435.20951704545	3.12423437252828e-05	\\
2436.18829900568	3.53625960113351e-05	\\
2437.16708096591	3.47163296590247e-05	\\
2438.14586292614	3.50232757188302e-05	\\
2439.12464488636	3.18104397331512e-05	\\
2440.10342684659	3.56411627079402e-05	\\
2441.08220880682	3.62041515430332e-05	\\
2442.06099076704	3.49018302145757e-05	\\
2443.03977272727	3.54508621614431e-05	\\
2444.0185546875	3.56673024580862e-05	\\
2444.99733664773	3.49722517339871e-05	\\
2445.97611860795	3.38588582069362e-05	\\
2446.95490056818	3.69496660903378e-05	\\
2447.93368252841	3.30843156285024e-05	\\
2448.91246448864	3.65323109203318e-05	\\
2449.89124644886	3.26251403428665e-05	\\
2450.87002840909	3.19468600051208e-05	\\
2451.84881036932	3.21738425589679e-05	\\
2452.82759232954	3.55518745883662e-05	\\
2453.80637428977	3.08258600130511e-05	\\
2454.78515625	3.31637199462611e-05	\\
2455.76393821023	3.22604685942023e-05	\\
2456.74272017045	3.52597492903651e-05	\\
2457.72150213068	3.4372636350279e-05	\\
2458.70028409091	3.40144001580881e-05	\\
2459.67906605114	3.59870550062031e-05	\\
2460.65784801136	3.24522496227256e-05	\\
2461.63662997159	3.32672515155086e-05	\\
2462.61541193182	3.03961402537896e-05	\\
2463.59419389204	3.30539441823603e-05	\\
2464.57297585227	3.63693182669536e-05	\\
2465.5517578125	3.23581739970601e-05	\\
2466.53053977273	3.30056615339298e-05	\\
2467.50932173295	3.40331952200639e-05	\\
2468.48810369318	3.23455860194759e-05	\\
2469.46688565341	3.21434810703436e-05	\\
2470.44566761364	3.25039181295669e-05	\\
2471.42444957386	3.12665035667582e-05	\\
2472.40323153409	3.04443221261863e-05	\\
2473.38201349432	2.80304646021835e-05	\\
2474.36079545454	3.06179937593344e-05	\\
2475.33957741477	3.52053329243497e-05	\\
2476.318359375	3.57851456276376e-05	\\
2477.29714133523	3.23518811035126e-05	\\
2478.27592329545	2.94598971663245e-05	\\
2479.25470525568	3.43075754731708e-05	\\
2480.23348721591	3.27047091157365e-05	\\
2481.21226917614	3.06428835685974e-05	\\
2482.19105113636	2.965404719979e-05	\\
2483.16983309659	3.5169536008127e-05	\\
2484.14861505682	3.49701113141453e-05	\\
2485.12739701704	3.13616737505557e-05	\\
2486.10617897727	2.90418878282497e-05	\\
2487.0849609375	3.10421822443782e-05	\\
2488.06374289773	3.18175368538256e-05	\\
2489.04252485795	3.00249173067783e-05	\\
2490.02130681818	3.01541364459647e-05	\\
2491.00008877841	3.05779777584642e-05	\\
2491.97887073864	3.33445067704515e-05	\\
2492.95765269886	3.35081113552793e-05	\\
2493.93643465909	2.92087504696752e-05	\\
2494.91521661932	3.10310035098945e-05	\\
2495.89399857954	3.02513686479453e-05	\\
2496.87278053977	3.2497615889678e-05	\\
2497.8515625	3.32466150013672e-05	\\
2498.83034446023	3.21816800963609e-05	\\
2499.80912642045	3.70255184880236e-05	\\
2500.78790838068	3.34462571563857e-05	\\
2501.76669034091	3.3526008295231e-05	\\
2502.74547230114	2.93911321709623e-05	\\
2503.72425426136	3.04501156498962e-05	\\
2504.70303622159	2.83358814709232e-05	\\
2505.68181818182	3.1454177355769e-05	\\
2506.66060014204	3.16857179522241e-05	\\
2507.63938210227	3.16446573529461e-05	\\
2508.6181640625	2.90229100006585e-05	\\
2509.59694602273	3.2657804815578e-05	\\
2510.57572798295	3.10668366998133e-05	\\
2511.55450994318	3.3029449360611e-05	\\
2512.53329190341	3.04897794321421e-05	\\
2513.51207386364	3.15594616645272e-05	\\
2514.49085582386	3.15722962128562e-05	\\
2515.46963778409	3.10446579561967e-05	\\
2516.44841974432	3.23631395136899e-05	\\
2517.42720170454	2.80088566865963e-05	\\
2518.40598366477	3.0114329194009e-05	\\
2519.384765625	3.02208849576764e-05	\\
2520.36354758523	3.57262949684138e-05	\\
2521.34232954545	2.74634277990169e-05	\\
2522.32111150568	3.0511116286155e-05	\\
2523.29989346591	2.92069325582674e-05	\\
2524.27867542614	3.21959835635737e-05	\\
2525.25745738636	3.28019486279945e-05	\\
2526.23623934659	3.28513070170368e-05	\\
2527.21502130682	2.58226770913235e-05	\\
2528.19380326704	3.26985334525992e-05	\\
2529.17258522727	2.88353928248409e-05	\\
2530.1513671875	2.92729710104672e-05	\\
2531.13014914773	3.19266139992086e-05	\\
2532.10893110795	2.80853097331465e-05	\\
2533.08771306818	3.1524677979947e-05	\\
2534.06649502841	2.96612335906605e-05	\\
2535.04527698864	3.00547029504747e-05	\\
2536.02405894886	2.83684177504166e-05	\\
2537.00284090909	2.87018993413095e-05	\\
2537.98162286932	2.84354108439284e-05	\\
2538.96040482954	2.83553437700712e-05	\\
2539.93918678977	2.95671650355854e-05	\\
2540.91796875	3.09447090582703e-05	\\
2541.89675071023	2.85197733676731e-05	\\
2542.87553267045	2.94390396379311e-05	\\
2543.85431463068	3.08902440950758e-05	\\
2544.83309659091	2.97172629307902e-05	\\
2545.81187855114	3.1597323829362e-05	\\
2546.79066051136	2.91132090181823e-05	\\
2547.76944247159	3.0818190097289e-05	\\
2548.74822443182	3.35596994224686e-05	\\
2549.72700639204	2.93398461407816e-05	\\
2550.70578835227	2.80046085955354e-05	\\
2551.6845703125	2.76143099173774e-05	\\
2552.66335227273	2.90077312858853e-05	\\
2553.64213423295	2.88646232868546e-05	\\
2554.62091619318	3.09211540282683e-05	\\
2555.59969815341	2.87493245356511e-05	\\
2556.57848011364	2.89564069804444e-05	\\
2557.55726207386	2.64893879430028e-05	\\
2558.53604403409	2.62978953311492e-05	\\
2559.51482599432	2.8322613303047e-05	\\
2560.49360795454	3.08257071272492e-05	\\
2561.47238991477	2.87726786816992e-05	\\
2562.451171875	3.0039418791313e-05	\\
2563.42995383523	2.87177050904552e-05	\\
2564.40873579545	2.93332923154521e-05	\\
2565.38751775568	3.02295164923102e-05	\\
2566.36629971591	3.01681070264308e-05	\\
2567.34508167614	2.91325218785564e-05	\\
2568.32386363636	2.83039325499956e-05	\\
2569.30264559659	2.94117871703021e-05	\\
2570.28142755682	3.12685689389126e-05	\\
2571.26020951704	2.88452267189581e-05	\\
2572.23899147727	2.66156942641854e-05	\\
2573.2177734375	2.90901005067119e-05	\\
2574.19655539773	2.37943442843658e-05	\\
2575.17533735795	2.72387237893865e-05	\\
2576.15411931818	2.82700505178506e-05	\\
2577.13290127841	2.72057601847118e-05	\\
2578.11168323864	2.73817413552102e-05	\\
2579.09046519886	2.85327655354963e-05	\\
2580.06924715909	2.75219989407548e-05	\\
2581.04802911932	2.60877153854574e-05	\\
2582.02681107954	2.66363844600349e-05	\\
2583.00559303977	2.81671247751105e-05	\\
2583.984375	2.49856525756688e-05	\\
2584.96315696023	2.35520662048938e-05	\\
2585.94193892045	2.79524992485564e-05	\\
2586.92072088068	2.86988992786332e-05	\\
2587.89950284091	2.67580631542588e-05	\\
2588.87828480114	2.83320260442978e-05	\\
2589.85706676136	2.56977687627729e-05	\\
2590.83584872159	2.7883753103648e-05	\\
2591.81463068182	2.59331307770943e-05	\\
2592.79341264204	2.69453063458697e-05	\\
2593.77219460227	2.63391381957757e-05	\\
2594.7509765625	2.69223299527807e-05	\\
2595.72975852273	2.72035480414977e-05	\\
2596.70854048295	2.63521761610519e-05	\\
2597.68732244318	2.78408737892308e-05	\\
2598.66610440341	2.78886326441706e-05	\\
2599.64488636364	2.65596503289559e-05	\\
2600.62366832386	2.66409035956135e-05	\\
2601.60245028409	2.56124644369014e-05	\\
2602.58123224432	2.50115803662932e-05	\\
2603.56001420454	2.56850644275e-05	\\
2604.53879616477	2.46496529346256e-05	\\
2605.517578125	2.55941748282823e-05	\\
2606.49636008523	2.48943429555044e-05	\\
2607.47514204545	2.44263631530813e-05	\\
2608.45392400568	2.62801023720113e-05	\\
2609.43270596591	3.01411483233989e-05	\\
2610.41148792614	2.66641398131324e-05	\\
2611.39026988636	2.54651776489215e-05	\\
2612.36905184659	2.56555588988557e-05	\\
2613.34783380682	2.48277507769937e-05	\\
2614.32661576704	2.72610450055072e-05	\\
2615.30539772727	2.51024031279861e-05	\\
2616.2841796875	2.41084442052029e-05	\\
2617.26296164773	2.46729211956196e-05	\\
2618.24174360795	2.50831368303207e-05	\\
2619.22052556818	2.42857121140061e-05	\\
2620.19930752841	2.39535526995534e-05	\\
2621.17808948864	2.6908940403992e-05	\\
2622.15687144886	2.13724047626409e-05	\\
2623.13565340909	2.26527855554288e-05	\\
2624.11443536932	2.69921722220364e-05	\\
2625.09321732954	2.61047026233646e-05	\\
2626.07199928977	2.66916347408925e-05	\\
2627.05078125	2.50059690602025e-05	\\
2628.02956321023	2.57248169859696e-05	\\
2629.00834517045	2.7905781632233e-05	\\
2629.98712713068	2.54212585467141e-05	\\
2630.96590909091	2.14264827366176e-05	\\
2631.94469105114	2.60547220887187e-05	\\
2632.92347301136	2.40791869178586e-05	\\
2633.90225497159	2.65819447561779e-05	\\
2634.88103693182	2.48164660327306e-05	\\
2635.85981889204	2.50105749220682e-05	\\
2636.83860085227	2.35584730611921e-05	\\
2637.8173828125	2.6974045770843e-05	\\
2638.79616477273	2.5375890055062e-05	\\
2639.77494673295	2.48125660344154e-05	\\
2640.75372869318	2.49523186446873e-05	\\
2641.73251065341	2.37163430876328e-05	\\
2642.71129261364	2.64912419753963e-05	\\
2643.69007457386	2.20030661751509e-05	\\
2644.66885653409	2.4761523244046e-05	\\
2645.64763849432	2.68213399469634e-05	\\
2646.62642045454	2.54938788229841e-05	\\
2647.60520241477	2.53218939470525e-05	\\
2648.583984375	2.54296745778406e-05	\\
2649.56276633523	2.37436867752002e-05	\\
2650.54154829545	2.65120076803271e-05	\\
2651.52033025568	2.29074429184303e-05	\\
2652.49911221591	2.51453299174458e-05	\\
2653.47789417614	2.23652669276292e-05	\\
2654.45667613636	2.68695605803838e-05	\\
2655.43545809659	2.0803269913584e-05	\\
2656.41424005682	2.35844356775057e-05	\\
2657.39302201704	2.30587625648383e-05	\\
2658.37180397727	2.72337957584663e-05	\\
2659.3505859375	2.29861636256438e-05	\\
2660.32936789773	2.65030240812193e-05	\\
2661.30814985795	2.24700937734973e-05	\\
2662.28693181818	2.42702073626954e-05	\\
2663.26571377841	2.54678866256394e-05	\\
2664.24449573864	2.58459738819332e-05	\\
2665.22327769886	2.4710439206489e-05	\\
2666.20205965909	2.50189957892324e-05	\\
2667.18084161932	2.62621607632892e-05	\\
2668.15962357954	2.40977051611461e-05	\\
2669.13840553977	2.56661101377593e-05	\\
2670.1171875	2.52214365305995e-05	\\
2671.09596946023	2.62622191067065e-05	\\
2672.07475142045	2.77915749850453e-05	\\
2673.05353338068	2.40935772453399e-05	\\
2674.03231534091	2.57071884399892e-05	\\
2675.01109730114	2.57010559517779e-05	\\
2675.98987926136	2.42507265231138e-05	\\
2676.96866122159	2.48474884484333e-05	\\
2677.94744318182	2.65423070300594e-05	\\
2678.92622514204	2.45238040212456e-05	\\
2679.90500710227	2.5076468591681e-05	\\
2680.8837890625	2.42640207245632e-05	\\
2681.86257102273	2.44965698445296e-05	\\
2682.84135298295	2.57950899306832e-05	\\
2683.82013494318	2.38861099784432e-05	\\
2684.79891690341	2.59978760279909e-05	\\
2685.77769886364	2.52510795265835e-05	\\
2686.75648082386	2.78397152593009e-05	\\
2687.73526278409	2.94023261456112e-05	\\
2688.71404474432	2.63031221743365e-05	\\
2689.69282670454	2.77597399466493e-05	\\
2690.67160866477	2.54456435407681e-05	\\
2691.650390625	2.75122586799911e-05	\\
2692.62917258523	2.79286966105448e-05	\\
2693.60795454545	2.7047659229525e-05	\\
2694.58673650568	2.96258652136909e-05	\\
2695.56551846591	2.6609337651442e-05	\\
2696.54430042614	2.71159383297595e-05	\\
2697.52308238636	2.82352297984222e-05	\\
2698.50186434659	2.68642769109073e-05	\\
2699.48064630682	2.92699138079144e-05	\\
2700.45942826704	2.30774068471414e-05	\\
2701.43821022727	2.91851830224988e-05	\\
2702.4169921875	2.85073593017633e-05	\\
2703.39577414773	2.79624981052944e-05	\\
2704.37455610795	2.99031398907807e-05	\\
2705.35333806818	2.85141345429866e-05	\\
2706.33212002841	2.72671740354263e-05	\\
2707.31090198864	2.72832594135825e-05	\\
2708.28968394886	3.00540995062243e-05	\\
2709.26846590909	2.71772158485433e-05	\\
2710.24724786932	2.5223654624928e-05	\\
2711.22602982954	2.83284934324881e-05	\\
2712.20481178977	2.89967102653706e-05	\\
2713.18359375	2.81249244504799e-05	\\
2714.16237571023	2.71034073498786e-05	\\
2715.14115767045	3.0388035211364e-05	\\
2716.11993963068	2.9087570820793e-05	\\
2717.09872159091	2.73452651108748e-05	\\
2718.07750355114	2.80372080376854e-05	\\
2719.05628551136	2.82098953450447e-05	\\
2720.03506747159	2.86784001024809e-05	\\
2721.01384943182	2.53488631389033e-05	\\
2721.99263139204	2.93940929610234e-05	\\
2722.97141335227	3.16491464141443e-05	\\
2723.9501953125	2.83533954918557e-05	\\
2724.92897727273	2.96153464653543e-05	\\
2725.90775923295	3.08460494245915e-05	\\
2726.88654119318	2.77635188000487e-05	\\
2727.86532315341	2.89546732828627e-05	\\
2728.84410511364	3.00548694360573e-05	\\
2729.82288707386	2.84114620647156e-05	\\
2730.80166903409	3.16779821234013e-05	\\
2731.78045099432	3.1744387593972e-05	\\
2732.75923295454	2.83362048152514e-05	\\
2733.73801491477	2.7885387031144e-05	\\
2734.716796875	2.60735692610234e-05	\\
2735.69557883523	3.09909421809157e-05	\\
2736.67436079545	3.05416521954355e-05	\\
2737.65314275568	2.96540790564767e-05	\\
2738.63192471591	3.03056741384919e-05	\\
2739.61070667614	2.85392420005007e-05	\\
2740.58948863636	2.83146981332733e-05	\\
2741.56827059659	2.80979273215702e-05	\\
2742.54705255682	2.70514821200995e-05	\\
2743.52583451704	2.87159088372172e-05	\\
2744.50461647727	2.7217786070398e-05	\\
2745.4833984375	2.82868796619988e-05	\\
2746.46218039773	2.53014246873049e-05	\\
2747.44096235795	2.77659949740954e-05	\\
2748.41974431818	2.76455896414105e-05	\\
2749.39852627841	2.7100545085349e-05	\\
2750.37730823864	2.66661399606421e-05	\\
2751.35609019886	2.62310700456499e-05	\\
2752.33487215909	2.64744578563594e-05	\\
2753.31365411932	2.6234149330077e-05	\\
2754.29243607954	2.79024157128599e-05	\\
2755.27121803977	2.70411420766344e-05	\\
2756.25	2.44409613957638e-05	\\
2757.22878196023	2.59772006073186e-05	\\
2758.20756392045	2.83120450198848e-05	\\
2759.18634588068	2.75318592023154e-05	\\
2760.16512784091	2.63895706419908e-05	\\
2761.14390980114	2.79422297195596e-05	\\
2762.12269176136	2.53010803187141e-05	\\
2763.10147372159	2.62909245998858e-05	\\
2764.08025568182	2.6533384248695e-05	\\
2765.05903764204	2.47011815392471e-05	\\
2766.03781960227	2.45811179659459e-05	\\
2767.0166015625	2.58954475655435e-05	\\
2767.99538352273	2.57001830202924e-05	\\
2768.97416548295	2.40521964813243e-05	\\
2769.95294744318	2.33984969530706e-05	\\
2770.93172940341	2.87432226194282e-05	\\
2771.91051136364	2.66508453890658e-05	\\
2772.88929332386	2.15581486715954e-05	\\
2773.86807528409	2.43314498514879e-05	\\
2774.84685724432	2.26355399163891e-05	\\
2775.82563920454	2.41091493582746e-05	\\
2776.80442116477	2.29125708461976e-05	\\
2777.783203125	2.63141703308884e-05	\\
2778.76198508523	2.35105742424524e-05	\\
2779.74076704545	2.3319102447362e-05	\\
2780.71954900568	2.5287665057269e-05	\\
2781.69833096591	2.2186235222722e-05	\\
2782.67711292614	2.20979010428199e-05	\\
2783.65589488636	2.28659075278871e-05	\\
2784.63467684659	2.43930463385165e-05	\\
2785.61345880682	2.39939122428335e-05	\\
2786.59224076704	2.40878391884499e-05	\\
2787.57102272727	2.54852953420921e-05	\\
2788.5498046875	2.34439517463401e-05	\\
2789.52858664773	2.52306640408662e-05	\\
2790.50736860795	2.25578319670255e-05	\\
2791.48615056818	2.1307469951159e-05	\\
2792.46493252841	2.40885525130122e-05	\\
2793.44371448864	2.23166100948465e-05	\\
2794.42249644886	2.55097511077365e-05	\\
2795.40127840909	2.05274700424997e-05	\\
2796.38006036932	2.27199241168056e-05	\\
2797.35884232954	2.36793144002128e-05	\\
2798.33762428977	2.22226203016062e-05	\\
2799.31640625	2.15239071807475e-05	\\
2800.29518821023	2.25202719630697e-05	\\
2801.27397017045	2.15714992871492e-05	\\
2802.25275213068	2.27983110825656e-05	\\
2803.23153409091	2.25439246232273e-05	\\
2804.21031605114	2.08664167822329e-05	\\
2805.18909801136	2.17473093238762e-05	\\
2806.16787997159	2.24231130960884e-05	\\
2807.14666193182	2.15441011107391e-05	\\
2808.12544389204	2.19040890138052e-05	\\
2809.10422585227	2.09292223502584e-05	\\
2810.0830078125	2.02465036511171e-05	\\
2811.06178977273	2.27772283870946e-05	\\
2812.04057173295	2.09600746360778e-05	\\
2813.01935369318	2.01070874466811e-05	\\
2813.99813565341	2.3283801428466e-05	\\
2814.97691761364	2.32804583698245e-05	\\
2815.95569957386	2.1054547248773e-05	\\
2816.93448153409	2.32455252075634e-05	\\
2817.91326349432	2.25876817926739e-05	\\
2818.89204545454	2.04980874256181e-05	\\
2819.87082741477	2.14555373613167e-05	\\
2820.849609375	2.02027807156024e-05	\\
2821.82839133523	2.45921342073449e-05	\\
2822.80717329545	1.93306356870358e-05	\\
2823.78595525568	2.1366441073681e-05	\\
2824.76473721591	1.80144132697005e-05	\\
2825.74351917614	1.92965808596299e-05	\\
2826.72230113636	1.84207584584354e-05	\\
2827.70108309659	2.30040920266733e-05	\\
2828.67986505682	2.03038150422579e-05	\\
2829.65864701704	2.11191440427505e-05	\\
2830.63742897727	1.85298904198995e-05	\\
2831.6162109375	2.43553745359511e-05	\\
2832.59499289773	1.91194736382327e-05	\\
2833.57377485795	2.12204026888541e-05	\\
2834.55255681818	2.2390212848128e-05	\\
2835.53133877841	2.17613302827372e-05	\\
2836.51012073864	1.99371524492691e-05	\\
2837.48890269886	2.15143889811613e-05	\\
2838.46768465909	1.73486992933872e-05	\\
2839.44646661932	1.90178230916797e-05	\\
2840.42524857955	2.12745816326391e-05	\\
2841.40403053977	2.13530718576188e-05	\\
2842.3828125	2.0701679787986e-05	\\
2843.36159446023	1.82953898693164e-05	\\
2844.34037642045	1.94368078495917e-05	\\
2845.31915838068	1.86088956850067e-05	\\
2846.29794034091	1.97558614590263e-05	\\
2847.27672230114	2.05256159829947e-05	\\
2848.25550426136	1.95250975009233e-05	\\
2849.23428622159	1.94971886658953e-05	\\
2850.21306818182	1.92116747857969e-05	\\
2851.19185014205	2.05711783568582e-05	\\
2852.17063210227	1.83893473628002e-05	\\
2853.1494140625	1.85096307081488e-05	\\
2854.12819602273	2.10156543334026e-05	\\
2855.10697798295	1.95312356986149e-05	\\
2856.08575994318	1.81034539952858e-05	\\
2857.06454190341	2.03454639327565e-05	\\
2858.04332386364	1.84676853704035e-05	\\
2859.02210582386	2.12055219392017e-05	\\
2860.00088778409	2.02116009387723e-05	\\
2860.97966974432	2.0161307779564e-05	\\
2861.95845170455	1.89768543925169e-05	\\
2862.93723366477	2.15490446412264e-05	\\
2863.916015625	1.91060266839785e-05	\\
2864.89479758523	1.91519470275468e-05	\\
2865.87357954545	2.11857986243474e-05	\\
2866.85236150568	2.02451466166985e-05	\\
2867.83114346591	1.88234193522348e-05	\\
2868.80992542614	1.89591975393546e-05	\\
2869.78870738636	2.0008583493363e-05	\\
2870.76748934659	1.90285065001494e-05	\\
2871.74627130682	1.85422340265653e-05	\\
2872.72505326705	1.48393688155959e-05	\\
2873.70383522727	2.15109766086765e-05	\\
2874.6826171875	2.10698480914893e-05	\\
2875.66139914773	2.05122662637587e-05	\\
2876.64018110795	1.79443350638556e-05	\\
2877.61896306818	1.87849777520466e-05	\\
2878.59774502841	1.79998811222487e-05	\\
2879.57652698864	1.73363176731688e-05	\\
2880.55530894886	2.27325064805072e-05	\\
2881.53409090909	1.99301657355375e-05	\\
2882.51287286932	1.91669866469091e-05	\\
2883.49165482955	1.84659511427551e-05	\\
2884.47043678977	2.05376352896774e-05	\\
2885.44921875	1.65369903961317e-05	\\
2886.42800071023	1.87994093139989e-05	\\
2887.40678267045	1.90789213419691e-05	\\
2888.38556463068	1.872282673268e-05	\\
2889.36434659091	1.85296942215542e-05	\\
2890.34312855114	1.8977041956412e-05	\\
2891.32191051136	1.92135761564015e-05	\\
2892.30069247159	1.85912188554911e-05	\\
2893.27947443182	1.59309210952779e-05	\\
2894.25825639205	2.04859155960764e-05	\\
2895.23703835227	1.94204066863415e-05	\\
2896.2158203125	1.87576562520079e-05	\\
2897.19460227273	1.93205829984685e-05	\\
2898.17338423295	1.83372444835143e-05	\\
2899.15216619318	1.9746363958316e-05	\\
2900.13094815341	1.74462410930161e-05	\\
2901.10973011364	2.05670134871905e-05	\\
2902.08851207386	1.89379116057392e-05	\\
2903.06729403409	1.74951356638755e-05	\\
2904.04607599432	2.12127833193788e-05	\\
2905.02485795455	1.96848248721722e-05	\\
2906.00363991477	1.78214048040843e-05	\\
2906.982421875	2.00230305434867e-05	\\
2907.96120383523	1.61110448114415e-05	\\
2908.93998579545	1.90681718926136e-05	\\
2909.91876775568	2.0999820592396e-05	\\
2910.89754971591	1.73875432259079e-05	\\
2911.87633167614	2.18171454279241e-05	\\
2912.85511363636	1.99004364241069e-05	\\
2913.83389559659	2.07555200616372e-05	\\
2914.81267755682	1.71170588116863e-05	\\
2915.79145951705	1.79980304964091e-05	\\
2916.77024147727	1.92135282174097e-05	\\
2917.7490234375	1.8483231861992e-05	\\
2918.72780539773	2.0354057012636e-05	\\
2919.70658735795	1.69579591913075e-05	\\
2920.68536931818	2.02820077465299e-05	\\
2921.66415127841	1.76707807879969e-05	\\
2922.64293323864	1.54420014504817e-05	\\
2923.62171519886	1.87239991538971e-05	\\
2924.60049715909	1.7502967582197e-05	\\
2925.57927911932	1.80700861373462e-05	\\
2926.55806107955	1.89409187900189e-05	\\
2927.53684303977	1.86737894545519e-05	\\
2928.515625	1.89340629466283e-05	\\
2929.49440696023	1.98183419878125e-05	\\
2930.47318892045	2.01976212093897e-05	\\
2931.45197088068	1.99338173646117e-05	\\
2932.43075284091	1.7421470024258e-05	\\
2933.40953480114	2.07082665697271e-05	\\
2934.38831676136	1.79995369406683e-05	\\
2935.36709872159	1.83801460590794e-05	\\
2936.34588068182	1.83791959290153e-05	\\
2937.32466264205	1.65022636725186e-05	\\
2938.30344460227	1.83675930492844e-05	\\
2939.2822265625	1.90473977083559e-05	\\
2940.26100852273	1.89466849913821e-05	\\
2941.23979048295	1.79221195108957e-05	\\
2942.21857244318	1.70159830231526e-05	\\
2943.19735440341	1.99546355353341e-05	\\
2944.17613636364	1.8867956946944e-05	\\
2945.15491832386	1.81049432115915e-05	\\
2946.13370028409	1.99224344479094e-05	\\
2947.11248224432	1.84328677827259e-05	\\
2948.09126420455	1.79654480984308e-05	\\
2949.07004616477	1.96631986981876e-05	\\
2950.048828125	1.99011209981104e-05	\\
2951.02761008523	1.63440223447943e-05	\\
2952.00639204545	2.02288946001371e-05	\\
2952.98517400568	1.82208988742954e-05	\\
2953.96395596591	2.13023242278591e-05	\\
2954.94273792614	2.07745573173736e-05	\\
2955.92151988636	1.76845237095588e-05	\\
2956.90030184659	1.8363295823188e-05	\\
2957.87908380682	1.79517762890306e-05	\\
2958.85786576705	1.82748266228578e-05	\\
2959.83664772727	1.566005444551e-05	\\
2960.8154296875	1.84135855437673e-05	\\
2961.79421164773	1.81687644315901e-05	\\
2962.77299360795	1.67461895991208e-05	\\
2963.75177556818	1.66698298962428e-05	\\
2964.73055752841	1.32223736267166e-05	\\
2965.70933948864	1.96525622362759e-05	\\
2966.68812144886	1.75660990481137e-05	\\
2967.66690340909	1.61239788294538e-05	\\
2968.64568536932	1.91080857693209e-05	\\
2969.62446732955	1.64377193613238e-05	\\
2970.60324928977	1.81668929314531e-05	\\
2971.58203125	1.97983610962164e-05	\\
2972.56081321023	1.85993738547852e-05	\\
2973.53959517045	1.67999240448346e-05	\\
2974.51837713068	1.84581047094636e-05	\\
2975.49715909091	1.70214198237477e-05	\\
2976.47594105114	1.86393376840052e-05	\\
2977.45472301136	1.831277337377e-05	\\
2978.43350497159	1.89630259084631e-05	\\
2979.41228693182	1.6651985486597e-05	\\
2980.39106889205	1.6298133056546e-05	\\
2981.36985085227	1.73568143613023e-05	\\
2982.3486328125	1.98128108970674e-05	\\
2983.32741477273	1.72234871197945e-05	\\
2984.30619673295	1.78989046786787e-05	\\
2985.28497869318	1.64529411867136e-05	\\
2986.26376065341	1.975767776712e-05	\\
2987.24254261364	1.6881977205576e-05	\\
2988.22132457386	2.00281359233279e-05	\\
2989.20010653409	1.76516833668867e-05	\\
2990.17888849432	1.7306870317261e-05	\\
2991.15767045455	1.83137777194535e-05	\\
2992.13645241477	1.98184260829429e-05	\\
2993.115234375	1.74789024481298e-05	\\
2994.09401633523	1.7490253211493e-05	\\
2995.07279829545	1.59828201523878e-05	\\
2996.05158025568	1.95819766448208e-05	\\
2997.03036221591	1.72969395014995e-05	\\
2998.00914417614	2.00555678671201e-05	\\
2998.98792613636	1.82578621691894e-05	\\
2999.96670809659	1.60896345296361e-05	\\
3000.94549005682	1.93047367160637e-05	\\
3001.92427201705	1.75409893661511e-05	\\
3002.90305397727	2.1454517574439e-05	\\
3003.8818359375	1.66467080603905e-05	\\
3004.86061789773	1.83858843264072e-05	\\
3005.83939985795	1.87303046192519e-05	\\
3006.81818181818	1.8228400824243e-05	\\
3007.79696377841	1.81815748565331e-05	\\
3008.77574573864	1.8712909445827e-05	\\
3009.75452769886	1.82732541202478e-05	\\
3010.73330965909	1.82434211703607e-05	\\
3011.71209161932	1.8606415803894e-05	\\
3012.69087357955	1.82067398914517e-05	\\
3013.66965553977	1.57946786694552e-05	\\
3014.6484375	1.943807944961e-05	\\
3015.62721946023	1.842045498498e-05	\\
3016.60600142045	1.67215627711379e-05	\\
3017.58478338068	2.01788878224487e-05	\\
3018.56356534091	1.79940300533929e-05	\\
3019.54234730114	1.92935999361123e-05	\\
3020.52112926136	1.96233245117423e-05	\\
3021.49991122159	1.81789783573614e-05	\\
3022.47869318182	1.74732013067107e-05	\\
3023.45747514205	1.73288565377181e-05	\\
3024.43625710227	1.46250048071723e-05	\\
3025.4150390625	1.8356087238543e-05	\\
3026.39382102273	2.03162166519096e-05	\\
3027.37260298295	1.88326166570252e-05	\\
3028.35138494318	1.77890002342489e-05	\\
3029.33016690341	1.7114380974091e-05	\\
3030.30894886364	1.89453602345133e-05	\\
3031.28773082386	1.82718790491818e-05	\\
3032.26651278409	1.99897037709428e-05	\\
3033.24529474432	1.62074255736376e-05	\\
3034.22407670455	1.93383308352324e-05	\\
3035.20285866477	1.6647069470847e-05	\\
3036.181640625	1.77333019426263e-05	\\
3037.16042258523	2.01581340271959e-05	\\
3038.13920454545	1.88873120666583e-05	\\
3039.11798650568	1.67200245906987e-05	\\
3040.09676846591	1.9688984485723e-05	\\
3041.07555042614	1.9047753405261e-05	\\
3042.05433238636	1.87492151403952e-05	\\
3043.03311434659	1.8959905538742e-05	\\
3044.01189630682	1.69538426726525e-05	\\
3044.99067826705	1.71949175073433e-05	\\
3045.96946022727	1.83836695163041e-05	\\
3046.9482421875	1.89046197541266e-05	\\
3047.92702414773	1.97581941597322e-05	\\
3048.90580610795	1.8427382622255e-05	\\
3049.88458806818	1.66329499170867e-05	\\
3050.86337002841	1.95397533877935e-05	\\
3051.84215198864	1.95117166118709e-05	\\
3052.82093394886	1.85724117360982e-05	\\
3053.79971590909	1.75381436247568e-05	\\
3054.77849786932	1.7786664145648e-05	\\
3055.75727982955	1.74669514287305e-05	\\
3056.73606178977	1.80606572566157e-05	\\
3057.71484375	1.9978012328033e-05	\\
3058.69362571023	1.93060387713118e-05	\\
3059.67240767045	1.87999396268018e-05	\\
3060.65118963068	1.99922640321775e-05	\\
3061.62997159091	1.89461356794533e-05	\\
3062.60875355114	1.93764831106611e-05	\\
3063.58753551136	1.6147943486279e-05	\\
3064.56631747159	1.97183922570874e-05	\\
3065.54509943182	2.08811566482654e-05	\\
3066.52388139205	2.11249099560844e-05	\\
3067.50266335227	2.04915038624815e-05	\\
3068.4814453125	2.1141504531014e-05	\\
3069.46022727273	1.93333647838383e-05	\\
3070.43900923295	1.80320075222958e-05	\\
3071.41779119318	1.93861169414814e-05	\\
3072.39657315341	1.89518649136127e-05	\\
3073.37535511364	1.96921080834277e-05	\\
3074.35413707386	1.84071153183618e-05	\\
3075.33291903409	2.08305415143197e-05	\\
3076.31170099432	2.06387819434248e-05	\\
3077.29048295455	1.88122450655951e-05	\\
3078.26926491477	2.15754660138862e-05	\\
3079.248046875	2.01525930455698e-05	\\
3080.22682883523	1.90624418855525e-05	\\
3081.20561079545	2.08932331680671e-05	\\
3082.18439275568	1.89636204200004e-05	\\
3083.16317471591	1.72709120200915e-05	\\
3084.14195667614	2.02846031766605e-05	\\
3085.12073863636	1.94352182942994e-05	\\
3086.09952059659	1.71350244177662e-05	\\
3087.07830255682	2.18801323816361e-05	\\
3088.05708451705	1.92691763169621e-05	\\
3089.03586647727	2.18671408540984e-05	\\
3090.0146484375	1.99839220290391e-05	\\
3090.99343039773	2.15720296238791e-05	\\
3091.97221235795	1.9596399355261e-05	\\
3092.95099431818	1.94393728271848e-05	\\
3093.92977627841	2.04984259777177e-05	\\
3094.90855823864	2.060581697804e-05	\\
3095.88734019886	1.95915202684005e-05	\\
3096.86612215909	1.90186896629363e-05	\\
3097.84490411932	2.11318114688053e-05	\\
3098.82368607955	2.01581587820279e-05	\\
3099.80246803977	1.81242854653395e-05	\\
3100.78125	1.77233720264678e-05	\\
3101.76003196023	1.86363668517638e-05	\\
3102.73881392045	2.17135944402231e-05	\\
3103.71759588068	2.22001554930271e-05	\\
3104.69637784091	1.93271405982667e-05	\\
3105.67515980114	2.24471139553551e-05	\\
3106.65394176136	1.96904876814061e-05	\\
3107.63272372159	1.90380733567631e-05	\\
3108.61150568182	2.05452237864934e-05	\\
3109.59028764205	2.15166169418515e-05	\\
3110.56906960227	1.70971943267541e-05	\\
3111.5478515625	2.12974534477075e-05	\\
3112.52663352273	2.25419325706004e-05	\\
3113.50541548295	2.00869656892136e-05	\\
3114.48419744318	2.2416695861739e-05	\\
3115.46297940341	2.09263326077827e-05	\\
3116.44176136364	2.15717382875973e-05	\\
3117.42054332386	2.16243456248481e-05	\\
3118.39932528409	2.18280894895399e-05	\\
3119.37810724432	2.27162791933153e-05	\\
3120.35688920455	2.23903040784307e-05	\\
3121.33567116477	2.11735925578466e-05	\\
3122.314453125	1.95094269531777e-05	\\
3123.29323508523	2.20122749819884e-05	\\
3124.27201704545	2.13237892383118e-05	\\
3125.25079900568	1.97494959122096e-05	\\
3126.22958096591	2.22004202541228e-05	\\
3127.20836292614	2.25511195890831e-05	\\
3128.18714488636	2.19349612655691e-05	\\
3129.16592684659	2.38182911232122e-05	\\
3130.14470880682	2.1343357681667e-05	\\
3131.12349076705	2.30805711342974e-05	\\
3132.10227272727	2.10121206014034e-05	\\
3133.0810546875	2.14110994672225e-05	\\
3134.05983664773	2.25133244990203e-05	\\
3135.03861860795	2.22096137905034e-05	\\
3136.01740056818	2.32498874583937e-05	\\
3136.99618252841	2.20920533225822e-05	\\
3137.97496448864	2.30625967106554e-05	\\
3138.95374644886	2.31894850163205e-05	\\
3139.93252840909	2.32630414849858e-05	\\
3140.91131036932	2.29046916910961e-05	\\
3141.89009232955	2.28260234722611e-05	\\
3142.86887428977	2.20869222505923e-05	\\
3143.84765625	2.03952110721527e-05	\\
3144.82643821023	2.29617568296803e-05	\\
3145.80522017045	2.29014592092463e-05	\\
3146.78400213068	2.29562567293633e-05	\\
3147.76278409091	2.40490603391454e-05	\\
3148.74156605114	2.2298210867019e-05	\\
3149.72034801136	2.23173400953447e-05	\\
3150.69912997159	2.34627759194839e-05	\\
3151.67791193182	2.32472281727439e-05	\\
3152.65669389205	2.13581762475268e-05	\\
3153.63547585227	2.26711211342244e-05	\\
3154.6142578125	2.42267078680982e-05	\\
3155.59303977273	2.38194371879052e-05	\\
3156.57182173295	2.54041332643774e-05	\\
3157.55060369318	2.249879103797e-05	\\
3158.52938565341	2.19508498928436e-05	\\
3159.50816761364	2.50691403794329e-05	\\
3160.48694957386	2.40851281878694e-05	\\
3161.46573153409	2.24466630392026e-05	\\
3162.44451349432	2.40332789916624e-05	\\
3163.42329545455	2.34339125807107e-05	\\
3164.40207741477	2.20474169235945e-05	\\
3165.380859375	2.54841313913096e-05	\\
3166.35964133523	2.57976361790301e-05	\\
3167.33842329545	2.19470735699161e-05	\\
3168.31720525568	2.53420455509605e-05	\\
3169.29598721591	2.69048977884425e-05	\\
3170.27476917614	2.41570399716415e-05	\\
3171.25355113636	2.55390089969731e-05	\\
3172.23233309659	2.32336653650422e-05	\\
3173.21111505682	2.69373616325791e-05	\\
3174.18989701705	2.46503212315939e-05	\\
3175.16867897727	2.2430162473246e-05	\\
3176.1474609375	2.5274794726214e-05	\\
3177.12624289773	2.72368468820841e-05	\\
3178.10502485795	2.33611459122473e-05	\\
3179.08380681818	2.51002708127759e-05	\\
3180.06258877841	2.34042712629621e-05	\\
3181.04137073864	2.50404722172278e-05	\\
3182.02015269886	2.49359786475228e-05	\\
3182.99893465909	2.21257348595129e-05	\\
3183.97771661932	2.19189311602256e-05	\\
3184.95649857955	2.40518081311757e-05	\\
3185.93528053977	2.42721848345294e-05	\\
3186.9140625	2.49221062653559e-05	\\
3187.89284446023	2.59381721346717e-05	\\
3188.87162642045	2.47369978546905e-05	\\
3189.85040838068	2.48304760101527e-05	\\
3190.82919034091	2.48267881813477e-05	\\
3191.80797230114	2.44495826328206e-05	\\
3192.78675426136	2.51314487840439e-05	\\
3193.76553622159	2.49937024752897e-05	\\
3194.74431818182	2.44930639519883e-05	\\
3195.72310014205	2.35790159553827e-05	\\
3196.70188210227	2.50571311080104e-05	\\
3197.6806640625	2.42778834097404e-05	\\
3198.65944602273	2.55128847987403e-05	\\
3199.63822798295	2.23867375309626e-05	\\
3200.61700994318	2.56220156505222e-05	\\
3201.59579190341	2.57137393936695e-05	\\
3202.57457386364	2.55837356743133e-05	\\
3203.55335582386	2.44242134858995e-05	\\
3204.53213778409	2.64754296410716e-05	\\
3205.51091974432	2.42612857674201e-05	\\
3206.48970170455	2.51201092827802e-05	\\
3207.46848366477	2.44749853283153e-05	\\
3208.447265625	2.36060925436506e-05	\\
3209.42604758523	2.31119668052e-05	\\
3210.40482954545	2.59931175959529e-05	\\
3211.38361150568	2.50479418128356e-05	\\
3212.36239346591	2.43735769258516e-05	\\
3213.34117542614	2.36321807190867e-05	\\
3214.31995738636	2.4537509238848e-05	\\
3215.29873934659	2.47182137408663e-05	\\
3216.27752130682	2.43048140067802e-05	\\
3217.25630326705	2.6286131858124e-05	\\
3218.23508522727	2.49407118964842e-05	\\
3219.2138671875	2.51172240821672e-05	\\
3220.19264914773	2.3747742792618e-05	\\
3221.17143110795	2.56951767102979e-05	\\
3222.15021306818	2.77004165156562e-05	\\
3223.12899502841	2.34117351212464e-05	\\
3224.10777698864	2.55562766685636e-05	\\
3225.08655894886	2.65061247139658e-05	\\
3226.06534090909	2.54701044400599e-05	\\
3227.04412286932	2.47860394640727e-05	\\
3228.02290482955	2.76872291088727e-05	\\
3229.00168678977	2.68952154845005e-05	\\
3229.98046875	2.78565672777605e-05	\\
3230.95925071023	2.42233017652336e-05	\\
3231.93803267045	2.4233898829734e-05	\\
3232.91681463068	2.43610630310536e-05	\\
3233.89559659091	2.48599115451628e-05	\\
3234.87437855114	2.46695804191386e-05	\\
3235.85316051136	2.77445115148932e-05	\\
3236.83194247159	2.59414059098524e-05	\\
3237.81072443182	2.31296691443272e-05	\\
3238.78950639205	2.48923081236455e-05	\\
3239.76828835227	2.76760998988635e-05	\\
3240.7470703125	2.61838866555709e-05	\\
3241.72585227273	2.59474621552655e-05	\\
3242.70463423295	2.5545003235616e-05	\\
3243.68341619318	2.59468636618605e-05	\\
3244.66219815341	2.62657249000196e-05	\\
3245.64098011364	2.63746924213561e-05	\\
3246.61976207386	2.53833779757922e-05	\\
3247.59854403409	2.54467613433442e-05	\\
3248.57732599432	2.62918901433421e-05	\\
3249.55610795455	2.62596658599943e-05	\\
3250.53488991477	2.65573835609638e-05	\\
3251.513671875	2.74636439380422e-05	\\
3252.49245383523	2.66851695320554e-05	\\
3253.47123579545	2.41783907635198e-05	\\
3254.45001775568	2.6561402776179e-05	\\
3255.42879971591	2.70404172598839e-05	\\
3256.40758167614	2.62313753271878e-05	\\
3257.38636363636	2.91493391655013e-05	\\
3258.36514559659	2.4251760578561e-05	\\
3259.34392755682	2.57012079127383e-05	\\
3260.32270951705	2.47170641603112e-05	\\
3261.30149147727	2.6522931295981e-05	\\
3262.2802734375	2.49544091317482e-05	\\
3263.25905539773	2.53231581998652e-05	\\
3264.23783735795	2.65023510196097e-05	\\
3265.21661931818	2.74997031875501e-05	\\
3266.19540127841	2.53241710049852e-05	\\
3267.17418323864	2.80354723581697e-05	\\
3268.15296519886	2.69257644654389e-05	\\
3269.13174715909	2.72675934747914e-05	\\
3270.11052911932	2.82100866308229e-05	\\
3271.08931107955	2.78071926465021e-05	\\
3272.06809303977	2.83751268357361e-05	\\
3273.046875	2.67274104995821e-05	\\
3274.02565696023	2.63409178172924e-05	\\
3275.00443892045	2.769768537884e-05	\\
3275.98322088068	2.6965148843671e-05	\\
3276.96200284091	2.62047572655381e-05	\\
3277.94078480114	2.60868872900086e-05	\\
3278.91956676136	2.77066537505407e-05	\\
3279.89834872159	2.67299881625761e-05	\\
3280.87713068182	2.57791151264827e-05	\\
3281.85591264205	2.48897228268446e-05	\\
3282.83469460227	2.59541233298859e-05	\\
3283.8134765625	2.6310930441163e-05	\\
3284.79225852273	2.6331113991714e-05	\\
3285.77104048295	2.83892176897133e-05	\\
3286.74982244318	2.64425930587565e-05	\\
3287.72860440341	2.71817508195805e-05	\\
3288.70738636364	2.72265201574907e-05	\\
3289.68616832386	2.79842356716843e-05	\\
3290.66495028409	2.79162330259097e-05	\\
3291.64373224432	2.70334719813144e-05	\\
3292.62251420455	2.74949038977979e-05	\\
3293.60129616477	2.84380033123638e-05	\\
3294.580078125	2.38454840117552e-05	\\
3295.55886008523	2.58239319948518e-05	\\
3296.53764204545	2.60329433845952e-05	\\
3297.51642400568	2.51415879132147e-05	\\
3298.49520596591	2.46625415911885e-05	\\
3299.47398792614	2.50695224004521e-05	\\
3300.45276988636	2.70114159232004e-05	\\
3301.43155184659	2.65384525932237e-05	\\
3302.41033380682	2.4821386860985e-05	\\
3303.38911576705	2.50637680583406e-05	\\
3304.36789772727	2.56665746121191e-05	\\
3305.3466796875	2.85611319471449e-05	\\
3306.32546164773	2.62947689234708e-05	\\
3307.30424360795	2.70770326242243e-05	\\
3308.28302556818	2.66345589168079e-05	\\
3309.26180752841	2.71538266201274e-05	\\
3310.24058948864	2.78872912071328e-05	\\
3311.21937144886	2.6965220427585e-05	\\
3312.19815340909	2.49543556527529e-05	\\
3313.17693536932	2.89955145242825e-05	\\
3314.15571732955	2.44051397702824e-05	\\
3315.13449928977	2.90254121466765e-05	\\
3316.11328125	2.86674036346137e-05	\\
3317.09206321023	2.83022086771202e-05	\\
3318.07084517045	2.84334348465271e-05	\\
3319.04962713068	3.13807056041631e-05	\\
3320.02840909091	2.66580878110526e-05	\\
3321.00719105114	2.66628054536684e-05	\\
3321.98597301136	2.60212013540934e-05	\\
3322.96475497159	2.66234454956375e-05	\\
3323.94353693182	2.67613390413015e-05	\\
3324.92231889205	2.83168973033446e-05	\\
3325.90110085227	2.65108941055002e-05	\\
3326.8798828125	2.74880904239222e-05	\\
3327.85866477273	2.82012973789189e-05	\\
3328.83744673295	2.60357721801597e-05	\\
3329.81622869318	2.87194741675009e-05	\\
3330.79501065341	2.93243485699211e-05	\\
3331.77379261364	2.67584581424003e-05	\\
3332.75257457386	2.67700503873022e-05	\\
3333.73135653409	2.80437953156626e-05	\\
3334.71013849432	2.97761349163899e-05	\\
3335.68892045455	2.91421501434458e-05	\\
3336.66770241477	2.88054067678642e-05	\\
3337.646484375	2.61556496547848e-05	\\
3338.62526633523	2.68851236921197e-05	\\
3339.60404829545	2.7964794435749e-05	\\
3340.58283025568	3.0076957597942e-05	\\
3341.56161221591	2.78303652837759e-05	\\
3342.54039417614	2.86589640415587e-05	\\
3343.51917613636	2.811014719006e-05	\\
3344.49795809659	3.01260993282634e-05	\\
3345.47674005682	2.91834414076727e-05	\\
3346.45552201705	2.72946552410378e-05	\\
3347.43430397727	2.92407598209287e-05	\\
3348.4130859375	2.97540502226791e-05	\\
3349.39186789773	2.81603245210007e-05	\\
3350.37064985795	2.89900055914948e-05	\\
3351.34943181818	2.71347740496179e-05	\\
3352.32821377841	2.84609129211972e-05	\\
3353.30699573864	2.92562950725974e-05	\\
3354.28577769886	3.2637135074205e-05	\\
3355.26455965909	2.97464344088181e-05	\\
3356.24334161932	2.84787367432297e-05	\\
3357.22212357955	2.77421547183218e-05	\\
3358.20090553977	2.85492540317915e-05	\\
3359.1796875	2.90406253064213e-05	\\
3360.15846946023	2.85271076076003e-05	\\
3361.13725142045	2.86067422145057e-05	\\
3362.11603338068	3.03862793675497e-05	\\
3363.09481534091	3.0475155152169e-05	\\
3364.07359730114	2.90884483126578e-05	\\
3365.05237926136	2.9305819855798e-05	\\
3366.03116122159	2.93035573216402e-05	\\
3367.00994318182	3.00702578401813e-05	\\
3367.98872514205	2.64425119139487e-05	\\
3368.96750710227	3.031068131872e-05	\\
3369.9462890625	2.93964730303613e-05	\\
3370.92507102273	2.83544440973174e-05	\\
3371.90385298295	3.03781618642647e-05	\\
3372.88263494318	3.04781956986885e-05	\\
3373.86141690341	2.91215960273473e-05	\\
3374.84019886364	3.03185761540765e-05	\\
3375.81898082386	2.84276312409201e-05	\\
3376.79776278409	2.50727803579365e-05	\\
3377.77654474432	2.74951900260209e-05	\\
3378.75532670455	2.78180725416448e-05	\\
3379.73410866477	2.98464163966345e-05	\\
3380.712890625	2.89303282627187e-05	\\
3381.69167258523	2.9912042788596e-05	\\
3382.67045454545	2.97309978829973e-05	\\
3383.64923650568	2.67157402234892e-05	\\
3384.62801846591	2.88259025200584e-05	\\
3385.60680042614	2.92379015692261e-05	\\
3386.58558238636	2.9680293348556e-05	\\
3387.56436434659	2.72732510915949e-05	\\
3388.54314630682	2.76854566738929e-05	\\
3389.52192826705	2.89314707443917e-05	\\
3390.50071022727	2.74580256388975e-05	\\
3391.4794921875	2.77350804954452e-05	\\
3392.45827414773	2.78373754871366e-05	\\
3393.43705610795	2.98181702569018e-05	\\
3394.41583806818	2.92718491223774e-05	\\
3395.39462002841	2.64686230588728e-05	\\
3396.37340198864	2.79387425055287e-05	\\
3397.35218394886	3.07461659741481e-05	\\
3398.33096590909	2.58999674395906e-05	\\
3399.30974786932	2.76105275727557e-05	\\
3400.28852982955	2.84467579346675e-05	\\
3401.26731178977	2.75179000529531e-05	\\
3402.24609375	2.90455206802608e-05	\\
3403.22487571023	2.61909457799396e-05	\\
3404.20365767045	2.71028650906147e-05	\\
3405.18243963068	2.8835187753696e-05	\\
3406.16122159091	2.66500929150618e-05	\\
3407.14000355114	2.78688780474313e-05	\\
3408.11878551136	2.78107084066653e-05	\\
3409.09756747159	2.82703525499011e-05	\\
3410.07634943182	2.67130806249633e-05	\\
3411.05513139205	2.94020695938216e-05	\\
3412.03391335227	2.69338653294862e-05	\\
3413.0126953125	2.86307676104218e-05	\\
3413.99147727273	2.63458484941931e-05	\\
3414.97025923295	2.56085620650217e-05	\\
3415.94904119318	2.83596483900598e-05	\\
3416.92782315341	2.83045603726673e-05	\\
3417.90660511364	2.67074764874426e-05	\\
3418.88538707386	2.74713581534988e-05	\\
3419.86416903409	2.53106849632627e-05	\\
3420.84295099432	2.74281311007597e-05	\\
3421.82173295455	2.92309103352138e-05	\\
3422.80051491477	2.81631317370148e-05	\\
3423.779296875	2.85091182021127e-05	\\
3424.75807883523	2.78103511369646e-05	\\
3425.73686079545	2.90505805575198e-05	\\
3426.71564275568	2.97672591800029e-05	\\
3427.69442471591	2.79405541938724e-05	\\
3428.67320667614	2.97094416900903e-05	\\
3429.65198863636	2.99886677749099e-05	\\
3430.63077059659	2.72963584069296e-05	\\
3431.60955255682	2.70197167780712e-05	\\
3432.58833451705	2.82794325242648e-05	\\
3433.56711647727	2.71140469273827e-05	\\
3434.5458984375	3.01382611696e-05	\\
3435.52468039773	2.95592366969078e-05	\\
3436.50346235795	2.82631950627109e-05	\\
3437.48224431818	3.06547811815036e-05	\\
3438.46102627841	2.79569160746253e-05	\\
3439.43980823864	2.98418819375177e-05	\\
3440.41859019886	2.82991781157109e-05	\\
3441.39737215909	3.09386017685597e-05	\\
3442.37615411932	2.76145541277159e-05	\\
3443.35493607955	2.93460279402054e-05	\\
3444.33371803977	3.05326694212446e-05	\\
3445.3125	2.76909823776539e-05	\\
3446.29128196023	3.0573414075731e-05	\\
3447.27006392045	2.96690710690059e-05	\\
3448.24884588068	3.01695276197779e-05	\\
3449.22762784091	2.90034414741593e-05	\\
3450.20640980114	2.96466661724788e-05	\\
3451.18519176136	2.91315804017189e-05	\\
3452.16397372159	2.91349338959771e-05	\\
3453.14275568182	3.0666492192735e-05	\\
3454.12153764205	2.86541931890928e-05	\\
3455.10031960227	2.90104288546103e-05	\\
3456.0791015625	2.92809332261032e-05	\\
3457.05788352273	3.21067667810808e-05	\\
3458.03666548295	3.16895965393413e-05	\\
3459.01544744318	3.08021972234352e-05	\\
3459.99422940341	3.05863990441047e-05	\\
3460.97301136364	2.70871202265569e-05	\\
3461.95179332386	2.93960834638029e-05	\\
3462.93057528409	2.89806899892971e-05	\\
3463.90935724432	3.09328583847883e-05	\\
3464.88813920455	3.00533331815292e-05	\\
3465.86692116477	3.01939736417525e-05	\\
3466.845703125	2.95334053648717e-05	\\
3467.82448508523	2.98849349075253e-05	\\
3468.80326704545	2.72249455375721e-05	\\
3469.78204900568	3.09279536846232e-05	\\
3470.76083096591	2.94103259758411e-05	\\
3471.73961292614	2.8451460339749e-05	\\
3472.71839488636	3.03273724049403e-05	\\
3473.69717684659	2.95384082173207e-05	\\
3474.67595880682	2.85583975466979e-05	\\
3475.65474076705	2.93155655444493e-05	\\
3476.63352272727	3.28573360152577e-05	\\
3477.6123046875	2.92533016101257e-05	\\
3478.59108664773	2.9910291023369e-05	\\
3479.56986860795	3.06435804829661e-05	\\
3480.54865056818	2.7291056918047e-05	\\
3481.52743252841	3.05057904372478e-05	\\
3482.50621448864	3.37805294416533e-05	\\
3483.48499644886	3.02496319835946e-05	\\
3484.46377840909	3.03934925006381e-05	\\
3485.44256036932	3.10581652237643e-05	\\
3486.42134232955	2.83775088924405e-05	\\
3487.40012428977	2.76713670810792e-05	\\
3488.37890625	2.9087697575943e-05	\\
3489.35768821023	2.95651686289953e-05	\\
3490.33647017045	3.1678042402865e-05	\\
3491.31525213068	3.08543540298525e-05	\\
3492.29403409091	3.06325294157232e-05	\\
3493.27281605114	3.34578926955502e-05	\\
3494.25159801136	3.03298943733362e-05	\\
3495.23037997159	2.93072644214791e-05	\\
3496.20916193182	3.12886740179192e-05	\\
3497.18794389205	2.91073039574252e-05	\\
3498.16672585227	3.07748988194794e-05	\\
3499.1455078125	3.25063538651123e-05	\\
3500.12428977273	2.82613841643857e-05	\\
3501.10307173295	2.81706713885898e-05	\\
3502.08185369318	3.13928120082581e-05	\\
3503.06063565341	2.76302764144062e-05	\\
3504.03941761364	2.98105976030497e-05	\\
3505.01819957386	3.08937703145004e-05	\\
3505.99698153409	2.73704523495504e-05	\\
3506.97576349432	3.14224653006568e-05	\\
3507.95454545455	2.96821646806882e-05	\\
3508.93332741477	2.90604387465258e-05	\\
3509.912109375	3.31058093255974e-05	\\
3510.89089133523	3.22602548970116e-05	\\
3511.86967329545	2.92067441773715e-05	\\
3512.84845525568	3.23775898814687e-05	\\
3513.82723721591	2.92060596044579e-05	\\
3514.80601917614	2.92355478880245e-05	\\
3515.78480113636	3.22465074573014e-05	\\
3516.76358309659	3.01083623339323e-05	\\
3517.74236505682	2.97260085710111e-05	\\
3518.72114701705	3.07414377989754e-05	\\
3519.69992897727	3.06883613538112e-05	\\
3520.6787109375	3.08282939106443e-05	\\
3521.65749289773	2.9202319346388e-05	\\
3522.63627485795	2.94756994334023e-05	\\
3523.61505681818	3.07638207523804e-05	\\
3524.59383877841	3.04937536486636e-05	\\
3525.57262073864	3.00431075673617e-05	\\
3526.55140269886	2.91818283251705e-05	\\
3527.53018465909	3.19709983870649e-05	\\
3528.50896661932	2.95394246643574e-05	\\
3529.48774857955	3.14907997934837e-05	\\
3530.46653053977	3.06453649403005e-05	\\
3531.4453125	3.09865048497648e-05	\\
3532.42409446023	3.28356182541766e-05	\\
3533.40287642045	2.992009046984e-05	\\
3534.38165838068	3.00844730983516e-05	\\
3535.36044034091	3.18186607796173e-05	\\
3536.33922230114	2.83377644978893e-05	\\
3537.31800426136	2.88085794192258e-05	\\
3538.29678622159	3.00430631920768e-05	\\
3539.27556818182	3.03891824539604e-05	\\
3540.25435014205	3.03966535728934e-05	\\
3541.23313210227	3.04605071060934e-05	\\
3542.2119140625	3.16571120288689e-05	\\
3543.19069602273	2.92190936240504e-05	\\
3544.16947798295	3.14207902872781e-05	\\
3545.14825994318	3.31583232697772e-05	\\
3546.12704190341	3.23876061579202e-05	\\
3547.10582386364	3.44614967852862e-05	\\
3548.08460582386	2.90242470529341e-05	\\
3549.06338778409	3.12419197697021e-05	\\
3550.04216974432	3.14651972745766e-05	\\
3551.02095170455	3.11935341865922e-05	\\
3551.99973366477	3.03822909697404e-05	\\
3552.978515625	3.16999949906511e-05	\\
3553.95729758523	3.12663779331294e-05	\\
3554.93607954545	3.14248056457464e-05	\\
3555.91486150568	3.02807983658074e-05	\\
3556.89364346591	3.06460181549766e-05	\\
3557.87242542614	2.94122040622427e-05	\\
3558.85120738636	3.20212379927383e-05	\\
3559.82998934659	3.07890242496202e-05	\\
3560.80877130682	2.89823574601633e-05	\\
3561.78755326705	3.0331367806634e-05	\\
3562.76633522727	3.06237454245946e-05	\\
3563.7451171875	3.12632969713372e-05	\\
3564.72389914773	3.39250858379539e-05	\\
3565.70268110795	3.12663145965369e-05	\\
3566.68146306818	3.10732441686364e-05	\\
3567.66024502841	3.02926356921493e-05	\\
3568.63902698864	3.14520311361096e-05	\\
3569.61780894886	3.04147526312299e-05	\\
3570.59659090909	3.03595648223029e-05	\\
3571.57537286932	3.05082396989042e-05	\\
3572.55415482955	3.12434058548091e-05	\\
3573.53293678977	2.98083959318896e-05	\\
3574.51171875	3.20125287974332e-05	\\
3575.49050071023	2.96221377696519e-05	\\
3576.46928267045	3.02062708570289e-05	\\
3577.44806463068	3.18593919151794e-05	\\
3578.42684659091	3.24661241120109e-05	\\
3579.40562855114	2.84829016000359e-05	\\
3580.38441051136	3.10439870392684e-05	\\
3581.36319247159	2.97984127170211e-05	\\
3582.34197443182	3.14416048982283e-05	\\
3583.32075639205	3.12023100410688e-05	\\
3584.29953835227	2.79449896717459e-05	\\
3585.2783203125	3.19314329780967e-05	\\
3586.25710227273	3.1084365872997e-05	\\
3587.23588423295	2.98618785778966e-05	\\
3588.21466619318	3.07249713527641e-05	\\
3589.19344815341	3.31053527804444e-05	\\
3590.17223011364	2.94605842727865e-05	\\
3591.15101207386	2.96927029734491e-05	\\
3592.12979403409	3.09791214745536e-05	\\
3593.10857599432	3.22003889158316e-05	\\
3594.08735795455	3.41379613359892e-05	\\
3595.06613991477	3.22906401913084e-05	\\
3596.044921875	3.04398743820313e-05	\\
3597.02370383523	3.19081182448765e-05	\\
3598.00248579545	3.07882269862151e-05	\\
3598.98126775568	2.87895924364745e-05	\\
3599.96004971591	3.00123082307002e-05	\\
3600.93883167614	2.95838749974778e-05	\\
3601.91761363636	3.24979098264235e-05	\\
3602.89639559659	3.17597192659738e-05	\\
3603.87517755682	2.98140904384344e-05	\\
3604.85395951705	3.01944370834397e-05	\\
3605.83274147727	3.34684496429579e-05	\\
3606.8115234375	3.39627682438148e-05	\\
3607.79030539773	3.21593124469321e-05	\\
3608.76908735795	3.19918316590457e-05	\\
3609.74786931818	3.24591524506042e-05	\\
3610.72665127841	3.18703645557273e-05	\\
3611.70543323864	3.2667440297437e-05	\\
3612.68421519886	3.17878797322483e-05	\\
3613.66299715909	3.29335282474841e-05	\\
3614.64177911932	3.10190935437458e-05	\\
3615.62056107955	3.30833632411294e-05	\\
3616.59934303977	3.01703560310205e-05	\\
3617.578125	3.29264954796146e-05	\\
3618.55690696023	3.29663555608415e-05	\\
3619.53568892045	3.19600046472147e-05	\\
3620.51447088068	3.36379662310895e-05	\\
3621.49325284091	3.28334030632037e-05	\\
3622.47203480114	3.28978418325733e-05	\\
3623.45081676136	2.87984984811277e-05	\\
3624.42959872159	2.83842467747765e-05	\\
3625.40838068182	3.13201868780371e-05	\\
3626.38716264205	2.97879155253996e-05	\\
3627.36594460227	3.13700783674886e-05	\\
3628.3447265625	3.34148267327348e-05	\\
3629.32350852273	3.00508093334312e-05	\\
3630.30229048295	3.09705232699547e-05	\\
3631.28107244318	3.40004949335849e-05	\\
3632.25985440341	2.92292917996694e-05	\\
3633.23863636364	3.16274262429433e-05	\\
3634.21741832386	3.02593227810306e-05	\\
3635.19620028409	3.19377710101344e-05	\\
3636.17498224432	3.15371223053435e-05	\\
3637.15376420455	3.20290087086978e-05	\\
3638.13254616477	3.24941110218292e-05	\\
3639.111328125	3.30645977297408e-05	\\
3640.09011008523	3.04379795143538e-05	\\
3641.06889204545	3.19424424589621e-05	\\
3642.04767400568	3.40946251711456e-05	\\
3643.02645596591	3.15195385204521e-05	\\
3644.00523792614	3.26774132002456e-05	\\
3644.98401988636	3.34046948311941e-05	\\
3645.96280184659	3.15281543948543e-05	\\
3646.94158380682	3.26382832497266e-05	\\
3647.92036576705	3.23036311925852e-05	\\
3648.89914772727	3.08338951105587e-05	\\
3649.8779296875	3.32589993362644e-05	\\
3650.85671164773	3.04749733603536e-05	\\
3651.83549360795	3.22770925860172e-05	\\
3652.81427556818	3.2322007203779e-05	\\
3653.79305752841	3.31423010189029e-05	\\
3654.77183948864	3.13515898835369e-05	\\
3655.75062144886	3.2971260633405e-05	\\
3656.72940340909	3.3334795946028e-05	\\
3657.70818536932	3.30093953716917e-05	\\
3658.68696732955	3.34886617586564e-05	\\
3659.66574928977	3.23324058592303e-05	\\
3660.64453125	3.29633057470358e-05	\\
3661.62331321023	3.35938157981017e-05	\\
3662.60209517045	3.44664478847757e-05	\\
3663.58087713068	3.22252609532275e-05	\\
3664.55965909091	3.20975539390964e-05	\\
3665.53844105114	3.38107539018395e-05	\\
3666.51722301136	3.38747969067948e-05	\\
3667.49600497159	3.39260652449786e-05	\\
3668.47478693182	3.2267782608248e-05	\\
3669.45356889205	3.19704804372848e-05	\\
3670.43235085227	3.31967555523319e-05	\\
3671.4111328125	3.27791334688967e-05	\\
3672.38991477273	3.3998199197297e-05	\\
3673.36869673295	3.25107907242625e-05	\\
3674.34747869318	3.19243199273218e-05	\\
3675.32626065341	3.17471768364824e-05	\\
3676.30504261364	3.03903167422164e-05	\\
3677.28382457386	3.44369394424223e-05	\\
3678.26260653409	3.1342820803433e-05	\\
3679.24138849432	3.20130848957507e-05	\\
3680.22017045455	3.39368292998494e-05	\\
3681.19895241477	3.39005464750859e-05	\\
3682.177734375	3.23696827709506e-05	\\
3683.15651633523	3.46406107447251e-05	\\
3684.13529829545	3.35272261586882e-05	\\
3685.11408025568	3.56781794750067e-05	\\
3686.09286221591	3.2295227955755e-05	\\
3687.07164417614	3.22199506189462e-05	\\
3688.05042613636	3.46209089015922e-05	\\
3689.02920809659	3.22503251175927e-05	\\
3690.00799005682	3.38924229439764e-05	\\
3690.98677201705	3.13429105047883e-05	\\
3691.96555397727	3.25834953048737e-05	\\
3692.9443359375	3.38114079779243e-05	\\
3693.92311789773	3.49877657589392e-05	\\
3694.90189985795	3.15692205242659e-05	\\
3695.88068181818	3.27760149966634e-05	\\
3696.85946377841	3.36067197046502e-05	\\
3697.83824573864	3.4178002992879e-05	\\
3698.81702769886	3.40525962297576e-05	\\
3699.79580965909	3.35476949298982e-05	\\
3700.77459161932	3.31523255270545e-05	\\
3701.75337357955	3.37639331778968e-05	\\
3702.73215553977	3.26156606431249e-05	\\
3703.7109375	3.48206386875579e-05	\\
3704.68971946023	3.2117692466093e-05	\\
3705.66850142045	3.27797536456758e-05	\\
3706.64728338068	3.51636359986649e-05	\\
3707.62606534091	3.35085144955235e-05	\\
3708.60484730114	3.31886035159483e-05	\\
3709.58362926136	3.51441930524116e-05	\\
3710.56241122159	3.3366673026959e-05	\\
3711.54119318182	3.44560816007999e-05	\\
3712.51997514205	3.40701850037271e-05	\\
3713.49875710227	3.12029011325774e-05	\\
3714.4775390625	3.17636299589119e-05	\\
3715.45632102273	3.4149029353443e-05	\\
3716.43510298295	3.48220681097986e-05	\\
3717.41388494318	3.48805135610882e-05	\\
3718.39266690341	3.5429270992899e-05	\\
3719.37144886364	3.37014165712643e-05	\\
3720.35023082386	3.33900964584739e-05	\\
3721.32901278409	3.43478181050854e-05	\\
3722.30779474432	3.5039950485825e-05	\\
3723.28657670455	3.57291992771072e-05	\\
3724.26535866477	3.61984097071618e-05	\\
3725.244140625	3.31573535204026e-05	\\
3726.22292258523	3.47165128376876e-05	\\
3727.20170454545	3.39018124324312e-05	\\
3728.18048650568	3.63241110435629e-05	\\
3729.15926846591	3.41451183035796e-05	\\
3730.13805042614	3.37752159990279e-05	\\
3731.11683238636	3.30412965244633e-05	\\
3732.09561434659	3.58184349001529e-05	\\
3733.07439630682	3.54622553118109e-05	\\
3734.05317826705	3.56766675457496e-05	\\
3735.03196022727	3.38596818609816e-05	\\
3736.0107421875	3.02140565021914e-05	\\
3736.98952414773	3.09862363022146e-05	\\
3737.96830610795	3.45420970799827e-05	\\
3738.94708806818	3.67259283902181e-05	\\
3739.92587002841	3.52331408699113e-05	\\
3740.90465198864	3.41807769145943e-05	\\
3741.88343394886	3.23277714665068e-05	\\
3742.86221590909	3.49371834635965e-05	\\
3743.84099786932	3.33487247785625e-05	\\
3744.81977982955	3.28889648392428e-05	\\
3745.79856178977	3.13623196988427e-05	\\
3746.77734375	3.57871743065431e-05	\\
3747.75612571023	3.35756035959122e-05	\\
3748.73490767045	3.4779031430303e-05	\\
3749.71368963068	3.48018618938163e-05	\\
3750.69247159091	3.65863490454899e-05	\\
3751.67125355114	3.36063041209425e-05	\\
3752.65003551136	3.38685325553621e-05	\\
3753.62881747159	3.56006596383645e-05	\\
3754.60759943182	3.38628816054026e-05	\\
3755.58638139205	3.33621799302531e-05	\\
3756.56516335227	3.35078808296422e-05	\\
3757.5439453125	3.53483006867727e-05	\\
3758.52272727273	3.19021819468666e-05	\\
3759.50150923295	3.37081442909216e-05	\\
3760.48029119318	3.34279070671933e-05	\\
3761.45907315341	3.51780459852282e-05	\\
3762.43785511364	3.30184966122052e-05	\\
3763.41663707386	3.38453904026001e-05	\\
3764.39541903409	3.46603465347041e-05	\\
3765.37420099432	3.64203358531089e-05	\\
3766.35298295455	3.44250712245809e-05	\\
3767.33176491477	3.30681299849948e-05	\\
3768.310546875	3.13038639547399e-05	\\
3769.28932883523	3.18842114658934e-05	\\
3770.26811079545	3.22951032287199e-05	\\
3771.24689275568	3.47472850681208e-05	\\
3772.22567471591	3.23039120692714e-05	\\
3773.20445667614	3.51085249119357e-05	\\
3774.18323863636	3.41831646940696e-05	\\
3775.16202059659	3.33498160899791e-05	\\
3776.14080255682	3.37694716162487e-05	\\
3777.11958451705	3.52425128612192e-05	\\
3778.09836647727	3.21013552301206e-05	\\
3779.0771484375	3.42907983234344e-05	\\
3780.05593039773	3.52887505271564e-05	\\
3781.03471235795	3.25987604973748e-05	\\
3782.01349431818	3.58165864881071e-05	\\
3782.99227627841	3.18987827315325e-05	\\
3783.97105823864	3.36935814194797e-05	\\
3784.94984019886	3.45514687827176e-05	\\
3785.92862215909	3.30052979306134e-05	\\
3786.90740411932	3.41171564980935e-05	\\
3787.88618607955	3.31567141629754e-05	\\
3788.86496803977	3.39062411212685e-05	\\
3789.84375	3.53466427604851e-05	\\
3790.82253196023	3.42783871509125e-05	\\
3791.80131392045	3.43765917060562e-05	\\
3792.78009588068	3.26709945779431e-05	\\
3793.75887784091	3.35213467697501e-05	\\
3794.73765980114	3.20786504041153e-05	\\
3795.71644176136	3.52226944440856e-05	\\
3796.69522372159	3.34566335265856e-05	\\
3797.67400568182	3.50772093339205e-05	\\
3798.65278764205	3.50540237631073e-05	\\
3799.63156960227	3.29233737939458e-05	\\
3800.6103515625	3.3810247932687e-05	\\
3801.58913352273	3.35906156629553e-05	\\
3802.56791548295	3.68119546026037e-05	\\
3803.54669744318	3.47174296245274e-05	\\
3804.52547940341	3.47605048205215e-05	\\
3805.50426136364	3.41029071675993e-05	\\
3806.48304332386	3.46060297639791e-05	\\
3807.46182528409	3.55992671389905e-05	\\
3808.44060724432	3.53802978681807e-05	\\
3809.41938920455	3.35680873393449e-05	\\
3810.39817116477	3.70979090217577e-05	\\
3811.376953125	3.78488601589647e-05	\\
3812.35573508523	3.33758711153951e-05	\\
3813.33451704545	3.64194895505583e-05	\\
3814.31329900568	3.46154854534424e-05	\\
3815.29208096591	3.61430217978034e-05	\\
3816.27086292614	3.42351183756759e-05	\\
3817.24964488636	3.24286990639166e-05	\\
3818.22842684659	3.38196297800478e-05	\\
3819.20720880682	3.45107224300231e-05	\\
3820.18599076705	3.46609729757336e-05	\\
3821.16477272727	3.41442961699499e-05	\\
3822.1435546875	3.59098841992396e-05	\\
3823.12233664773	3.35530414355867e-05	\\
3824.10111860795	3.33332388751142e-05	\\
3825.07990056818	3.5182186592878e-05	\\
3826.05868252841	3.42890340472684e-05	\\
3827.03746448864	3.29727092803532e-05	\\
3828.01624644886	3.42560644397617e-05	\\
3828.99502840909	3.4523177602341e-05	\\
3829.97381036932	3.46313870782849e-05	\\
3830.95259232955	3.41746635748252e-05	\\
3831.93137428977	3.52038372838127e-05	\\
3832.91015625	3.50402600158726e-05	\\
3833.88893821023	3.35975668124914e-05	\\
3834.86772017045	3.50750373676628e-05	\\
3835.84650213068	3.46319549338358e-05	\\
3836.82528409091	3.579990463777e-05	\\
3837.80406605114	3.48376234112261e-05	\\
3838.78284801136	3.52690309573263e-05	\\
3839.76162997159	3.58059661456161e-05	\\
3840.74041193182	3.4710467472957e-05	\\
3841.71919389205	3.47171750205115e-05	\\
3842.69797585227	3.1413361356587e-05	\\
3843.6767578125	3.39030391548479e-05	\\
3844.65553977273	3.59426796984599e-05	\\
3845.63432173295	3.24789722771272e-05	\\
3846.61310369318	3.51938774690666e-05	\\
3847.59188565341	3.36322624223605e-05	\\
3848.57066761364	3.48761024647139e-05	\\
3849.54944957386	3.40929329331364e-05	\\
3850.52823153409	3.44870246482536e-05	\\
3851.50701349432	3.47049896801156e-05	\\
3852.48579545455	3.6195517475286e-05	\\
3853.46457741477	3.69558200244854e-05	\\
3854.443359375	3.66487950281673e-05	\\
3855.42214133523	3.38520825004761e-05	\\
3856.40092329545	3.5409877262567e-05	\\
3857.37970525568	3.36355210355846e-05	\\
3858.35848721591	3.34211645492332e-05	\\
3859.33726917614	3.4897298707544e-05	\\
3860.31605113636	3.41734800870737e-05	\\
3861.29483309659	3.43056058744416e-05	\\
3862.27361505682	3.40159769763397e-05	\\
3863.25239701705	3.66933566853952e-05	\\
3864.23117897727	3.61270834222681e-05	\\
3865.2099609375	3.56240795593074e-05	\\
3866.18874289773	3.55093177365271e-05	\\
3867.16752485795	3.5063319311749e-05	\\
3868.14630681818	3.49991690239692e-05	\\
3869.12508877841	3.63366460396324e-05	\\
3870.10387073864	3.42402705961978e-05	\\
3871.08265269886	3.61236678291814e-05	\\
3872.06143465909	3.41399674209634e-05	\\
3873.04021661932	3.36554564953934e-05	\\
3874.01899857955	3.57650067895132e-05	\\
3874.99778053977	3.19778912258022e-05	\\
3875.9765625	3.63175167531845e-05	\\
3876.95534446023	3.47170944544659e-05	\\
3877.93412642045	3.3785517920329e-05	\\
3878.91290838068	3.49779571268744e-05	\\
3879.89169034091	3.67396309475544e-05	\\
3880.87047230114	3.45621404677614e-05	\\
3881.84925426136	3.59306725165451e-05	\\
3882.82803622159	3.45049589785192e-05	\\
3883.80681818182	3.59181228467903e-05	\\
3884.78560014205	3.37937166844476e-05	\\
3885.76438210227	3.60228061889081e-05	\\
3886.7431640625	3.37921829887331e-05	\\
3887.72194602273	3.31605613939348e-05	\\
3888.70072798295	3.64780049274707e-05	\\
3889.67950994318	3.39279082332605e-05	\\
3890.65829190341	3.45766022907388e-05	\\
3891.63707386364	3.35796465544187e-05	\\
3892.61585582386	3.41060291093809e-05	\\
3893.59463778409	3.39171454489499e-05	\\
3894.57341974432	3.5550733700977e-05	\\
3895.55220170455	3.36325008888062e-05	\\
3896.53098366477	3.24299594437356e-05	\\
3897.509765625	3.65664921683877e-05	\\
3898.48854758523	3.09559350448832e-05	\\
3899.46732954545	3.3875261675765e-05	\\
3900.44611150568	3.29176161435983e-05	\\
3901.42489346591	3.45681788787931e-05	\\
3902.40367542614	3.49883370908951e-05	\\
3903.38245738636	3.34513273951119e-05	\\
3904.36123934659	3.61191210152482e-05	\\
3905.34002130682	3.27408479305961e-05	\\
3906.31880326705	3.4970792888101e-05	\\
3907.29758522727	3.28181497597068e-05	\\
3908.2763671875	3.47957203012843e-05	\\
3909.25514914773	3.45882164272461e-05	\\
3910.23393110795	3.38288453729439e-05	\\
3911.21271306818	3.28855533651332e-05	\\
3912.19149502841	3.23624147213908e-05	\\
3913.17027698864	3.29409971792924e-05	\\
3914.14905894886	3.39018322260506e-05	\\
3915.12784090909	3.36475413947058e-05	\\
3916.10662286932	3.40734738276426e-05	\\
3917.08540482955	3.49915699462495e-05	\\
3918.06418678977	3.30022412596526e-05	\\
3919.04296875	3.33794634327198e-05	\\
3920.02175071023	3.38775422287033e-05	\\
3921.00053267045	3.2498770887404e-05	\\
3921.97931463068	3.63611011921996e-05	\\
3922.95809659091	3.39896739971099e-05	\\
3923.93687855114	3.45094792833328e-05	\\
3924.91566051136	3.45995788144136e-05	\\
3925.89444247159	3.33551535627561e-05	\\
3926.87322443182	3.589608609391e-05	\\
3927.85200639205	3.50821390012227e-05	\\
3928.83078835227	3.49935556860161e-05	\\
3929.8095703125	3.48460917230373e-05	\\
3930.78835227273	3.21899972003675e-05	\\
3931.76713423295	3.50640074004105e-05	\\
3932.74591619318	3.70590698760782e-05	\\
3933.72469815341	3.40966472585558e-05	\\
3934.70348011364	3.53353484328903e-05	\\
3935.68226207386	3.49032078157838e-05	\\
3936.66104403409	3.56678482175735e-05	\\
3937.63982599432	3.47163703474606e-05	\\
3938.61860795455	3.1821860317257e-05	\\
3939.59738991477	3.46993200134869e-05	\\
3940.576171875	3.45464147958486e-05	\\
3941.55495383523	3.41479596469523e-05	\\
3942.53373579545	3.51306349414964e-05	\\
3943.51251775568	3.38238062844589e-05	\\
3944.49129971591	3.50141089583796e-05	\\
3945.47008167614	3.50548234988493e-05	\\
3946.44886363636	3.30464076391637e-05	\\
3947.42764559659	3.55504239821652e-05	\\
3948.40642755682	3.59094438527609e-05	\\
3949.38520951705	3.51581575410083e-05	\\
3950.36399147727	3.70237775621783e-05	\\
3951.3427734375	3.31126138985359e-05	\\
3952.32155539773	3.49420791970733e-05	\\
3953.30033735795	3.56110052254414e-05	\\
3954.27911931818	3.42915720352514e-05	\\
3955.25790127841	3.39917592564911e-05	\\
3956.23668323864	3.4431036558494e-05	\\
3957.21546519886	3.25512695073677e-05	\\
3958.19424715909	3.53564094046429e-05	\\
3959.17302911932	3.63007697419998e-05	\\
3960.15181107955	3.42756092908166e-05	\\
3961.13059303977	3.23820815514722e-05	\\
3962.109375	3.68642153756685e-05	\\
3963.08815696023	3.44779233516333e-05	\\
3964.06693892045	3.43676295268025e-05	\\
3965.04572088068	3.44462163116122e-05	\\
3966.02450284091	3.50500990424076e-05	\\
3967.00328480114	3.60929402572878e-05	\\
3967.98206676136	3.55356714692688e-05	\\
3968.96084872159	3.4925648515238e-05	\\
3969.93963068182	3.64158845118955e-05	\\
3970.91841264205	3.47850158246021e-05	\\
3971.89719460227	3.40996022254497e-05	\\
3972.8759765625	3.58930182939237e-05	\\
3973.85475852273	3.34934967623495e-05	\\
3974.83354048295	3.54446137935795e-05	\\
3975.81232244318	3.57414941881344e-05	\\
3976.79110440341	3.70555577596657e-05	\\
3977.76988636364	3.55797204488501e-05	\\
3978.74866832386	3.62655860162791e-05	\\
3979.72745028409	3.61101372834493e-05	\\
3980.70623224432	3.66276951335477e-05	\\
3981.68501420455	3.45808383246945e-05	\\
3982.66379616477	3.45894380721352e-05	\\
3983.642578125	3.45046293576232e-05	\\
3984.62136008523	3.53209500601333e-05	\\
3985.60014204545	3.4015268096308e-05	\\
3986.57892400568	3.46211413873597e-05	\\
3987.55770596591	3.65371145554785e-05	\\
3988.53648792614	3.81093487794065e-05	\\
3989.51526988636	3.43628777243854e-05	\\
3990.49405184659	3.51340601253645e-05	\\
3991.47283380682	3.56283424090712e-05	\\
3992.45161576705	3.3733188053385e-05	\\
3993.43039772727	3.64670677469413e-05	\\
3994.4091796875	3.52674144387914e-05	\\
3995.38796164773	3.26100809271271e-05	\\
3996.36674360795	3.61309776020054e-05	\\
3997.34552556818	3.3755015264789e-05	\\
3998.32430752841	3.42479026834783e-05	\\
3999.30308948864	3.61047071777833e-05	\\
4000.28187144886	3.61988174186019e-05	\\
4001.26065340909	3.7822830307267e-05	\\
4002.23943536932	3.74122771871902e-05	\\
4003.21821732955	3.64812160763042e-05	\\
4004.19699928977	3.55507878043494e-05	\\
4005.17578125	3.68724579973586e-05	\\
4006.15456321023	3.82646128896548e-05	\\
4007.13334517045	3.86212126389689e-05	\\
4008.11212713068	3.70967910238885e-05	\\
4009.09090909091	3.76334108070311e-05	\\
4010.06969105114	3.94152247201988e-05	\\
4011.04847301136	3.73885692493373e-05	\\
4012.02725497159	3.94257390949646e-05	\\
4013.00603693182	3.75050354793152e-05	\\
4013.98481889205	3.59477773842605e-05	\\
4014.96360085227	3.64770828663404e-05	\\
4015.9423828125	3.67204753347031e-05	\\
4016.92116477273	3.64259890450457e-05	\\
4017.89994673295	3.98179715989584e-05	\\
4018.87872869318	3.48588748604846e-05	\\
4019.85751065341	4.02939412507671e-05	\\
4020.83629261364	3.81915468691392e-05	\\
4021.81507457386	3.72525850982735e-05	\\
4022.79385653409	3.96241815529477e-05	\\
4023.77263849432	3.73487572711687e-05	\\
4024.75142045455	3.81887689229935e-05	\\
4025.73020241477	3.81418847740943e-05	\\
4026.708984375	3.72129280103316e-05	\\
4027.68776633523	3.74469711431148e-05	\\
4028.66654829545	3.82575658883096e-05	\\
4029.64533025568	3.62603607123902e-05	\\
4030.62411221591	3.52646675444819e-05	\\
4031.60289417614	3.77277147772901e-05	\\
4032.58167613636	3.80619729095213e-05	\\
4033.56045809659	3.71508487060412e-05	\\
4034.53924005682	3.92315862972102e-05	\\
4035.51802201705	3.78976083717408e-05	\\
4036.49680397727	3.8160035757688e-05	\\
4037.4755859375	3.62435361438542e-05	\\
4038.45436789773	3.65063335340068e-05	\\
4039.43314985795	3.86526704815869e-05	\\
4040.41193181818	3.74046476636693e-05	\\
4041.39071377841	3.70041903454519e-05	\\
4042.36949573864	3.76712590423786e-05	\\
4043.34827769886	3.40893128254211e-05	\\
4044.32705965909	3.66514420064691e-05	\\
4045.30584161932	3.54969739446893e-05	\\
4046.28462357955	3.74604829240342e-05	\\
4047.26340553977	3.65715555639513e-05	\\
4048.2421875	3.53667199973092e-05	\\
4049.22096946023	4.02117536575056e-05	\\
4050.19975142045	3.77713456635099e-05	\\
4051.17853338068	3.50174121688824e-05	\\
4052.15731534091	3.83431251207475e-05	\\
4053.13609730114	3.59541964066141e-05	\\
4054.11487926136	3.83721392482364e-05	\\
4055.09366122159	3.51309241505528e-05	\\
4056.07244318182	3.58863740306284e-05	\\
4057.05122514205	3.54037638895146e-05	\\
4058.03000710227	3.43088946897392e-05	\\
4059.0087890625	3.60602558944427e-05	\\
4059.98757102273	3.7091811046542e-05	\\
4060.96635298295	3.44153800958853e-05	\\
4061.94513494318	3.61989637326438e-05	\\
4062.92391690341	3.68213394853825e-05	\\
4063.90269886364	3.45917441796581e-05	\\
4064.88148082386	3.38731333969737e-05	\\
4065.86026278409	3.41842783815104e-05	\\
4066.83904474432	3.60415840557341e-05	\\
4067.81782670455	3.71672883969791e-05	\\
4068.79660866477	3.44068070378356e-05	\\
4069.775390625	3.53785391396951e-05	\\
4070.75417258523	3.66178058993481e-05	\\
4071.73295454545	3.76949777599975e-05	\\
4072.71173650568	3.21537913156015e-05	\\
4073.69051846591	3.50506604088687e-05	\\
4074.66930042614	3.55058117831463e-05	\\
4075.64808238636	3.47341414367514e-05	\\
4076.62686434659	3.45992052961129e-05	\\
4077.60564630682	3.38868489712816e-05	\\
4078.58442826705	3.57553941698285e-05	\\
4079.56321022727	3.59056101628065e-05	\\
4080.5419921875	3.618593680337e-05	\\
4081.52077414773	3.62071161693976e-05	\\
4082.49955610795	3.66041067990781e-05	\\
4083.47833806818	3.6082166942625e-05	\\
4084.45712002841	3.58884588831868e-05	\\
4085.43590198864	3.84745024223629e-05	\\
4086.41468394886	3.6645730920396e-05	\\
4087.39346590909	3.40122787647654e-05	\\
4088.37224786932	3.80160458458103e-05	\\
4089.35102982955	3.47290552605005e-05	\\
4090.32981178977	3.67067884328859e-05	\\
4091.30859375	3.63006470647251e-05	\\
4092.28737571023	3.80554508465652e-05	\\
4093.26615767045	3.42812403845336e-05	\\
4094.24493963068	3.5372889562914e-05	\\
4095.22372159091	3.50816568012223e-05	\\
4096.20250355114	3.49857498225985e-05	\\
4097.18128551136	3.65418592598548e-05	\\
4098.16006747159	3.73524493331842e-05	\\
4099.13884943182	3.64592142282641e-05	\\
4100.11763139205	3.82329599543454e-05	\\
4101.09641335227	3.6752266511966e-05	\\
4102.0751953125	3.74941886511018e-05	\\
4103.05397727273	3.48427097890542e-05	\\
4104.03275923295	3.60697037215655e-05	\\
4105.01154119318	3.62098359662692e-05	\\
4105.99032315341	3.58611262720385e-05	\\
4106.96910511364	3.73433452516661e-05	\\
4107.94788707386	3.6459962136519e-05	\\
4108.92666903409	3.6157024983327e-05	\\
4109.90545099432	3.56385496948522e-05	\\
4110.88423295455	3.67362478706957e-05	\\
4111.86301491477	3.76468344073036e-05	\\
4112.841796875	3.52984634656083e-05	\\
4113.82057883523	3.56160323637305e-05	\\
4114.79936079545	3.70073764449792e-05	\\
4115.77814275568	3.757358271393e-05	\\
4116.75692471591	3.82322239391853e-05	\\
4117.73570667614	3.7212361580019e-05	\\
4118.71448863636	3.63372770944149e-05	\\
4119.69327059659	3.55772625912992e-05	\\
4120.67205255682	3.53322015877229e-05	\\
4121.65083451705	3.63412024862809e-05	\\
4122.62961647727	3.65289809329828e-05	\\
4123.6083984375	3.57467228036602e-05	\\
4124.58718039773	3.77122387981471e-05	\\
4125.56596235795	3.54509924999383e-05	\\
4126.54474431818	3.39556898594376e-05	\\
4127.52352627841	3.66437879676719e-05	\\
4128.50230823864	3.5977004483185e-05	\\
4129.48109019886	3.45452653014847e-05	\\
4130.45987215909	3.40882198472998e-05	\\
4131.43865411932	3.43922463733566e-05	\\
4132.41743607955	3.68899102738879e-05	\\
4133.39621803977	3.79667396525732e-05	\\
4134.375	3.72745916628171e-05	\\
4135.35378196023	3.7722068735364e-05	\\
4136.33256392045	3.51562063046372e-05	\\
4137.31134588068	3.35568693366068e-05	\\
4138.29012784091	3.45703837525705e-05	\\
4139.26890980114	3.54373061392991e-05	\\
4140.24769176136	3.63540038127318e-05	\\
4141.22647372159	3.55814363372518e-05	\\
4142.20525568182	3.44845363184194e-05	\\
4143.18403764205	3.48252512684627e-05	\\
4144.16281960227	3.67702129195119e-05	\\
4145.1416015625	3.35425233925085e-05	\\
4146.12038352273	3.42793532093592e-05	\\
4147.09916548295	3.46750276797281e-05	\\
4148.07794744318	3.6913535919954e-05	\\
4149.05672940341	3.68270244121592e-05	\\
4150.03551136364	3.48261996895657e-05	\\
4151.01429332386	3.48987898022493e-05	\\
4151.99307528409	3.70459518907956e-05	\\
4152.97185724432	3.40251127108951e-05	\\
4153.95063920455	3.49337855044759e-05	\\
4154.92942116477	3.59883918710668e-05	\\
4155.908203125	3.59738625200867e-05	\\
4156.88698508523	3.6942822102535e-05	\\
4157.86576704545	3.41977108061168e-05	\\
4158.84454900568	3.50795954176108e-05	\\
4159.82333096591	3.54889860173582e-05	\\
4160.80211292614	3.6179242196422e-05	\\
4161.78089488636	3.57900076315507e-05	\\
4162.75967684659	3.67474323566451e-05	\\
4163.73845880682	3.59801047122007e-05	\\
4164.71724076705	3.4487381823746e-05	\\
4165.69602272727	3.62242931257356e-05	\\
4166.6748046875	3.62014204368647e-05	\\
4167.65358664773	3.4763244697539e-05	\\
4168.63236860795	3.61743316819106e-05	\\
4169.61115056818	3.39968988521338e-05	\\
4170.58993252841	3.54657605420613e-05	\\
4171.56871448864	3.73292384411186e-05	\\
4172.54749644886	3.65747766513374e-05	\\
4173.52627840909	3.4303352643419e-05	\\
4174.50506036932	3.35310854406711e-05	\\
4175.48384232955	3.57870219985711e-05	\\
4176.46262428977	3.55165315034729e-05	\\
4177.44140625	3.50688366877986e-05	\\
4178.42018821023	3.71139160817103e-05	\\
4179.39897017045	3.53780203582233e-05	\\
4180.37775213068	3.62952631012799e-05	\\
4181.35653409091	3.43577088529307e-05	\\
4182.33531605114	3.41503831175877e-05	\\
4183.31409801136	3.5246778204859e-05	\\
4184.29287997159	3.49267387059841e-05	\\
4185.27166193182	3.57162799908957e-05	\\
4186.25044389205	3.41388512379439e-05	\\
4187.22922585227	3.48375545176276e-05	\\
4188.2080078125	3.38574036050232e-05	\\
4189.18678977273	3.48413394545512e-05	\\
4190.16557173295	3.72227933624499e-05	\\
4191.14435369318	3.35823367909168e-05	\\
4192.12313565341	3.58002338544737e-05	\\
4193.10191761364	3.62145324415735e-05	\\
4194.08069957386	3.63532061176319e-05	\\
4195.05948153409	3.45068432832485e-05	\\
4196.03826349432	3.4496088972745e-05	\\
4197.01704545455	3.59092162941435e-05	\\
4197.99582741477	3.25032243975922e-05	\\
4198.974609375	3.41125872879517e-05	\\
4199.95339133523	3.78481300435509e-05	\\
4200.93217329545	3.33279768938304e-05	\\
4201.91095525568	3.64933589572659e-05	\\
4202.88973721591	3.49342950374709e-05	\\
4203.86851917614	3.30131643563118e-05	\\
4204.84730113636	3.5909327346635e-05	\\
4205.82608309659	3.50037753117028e-05	\\
4206.80486505682	3.48137554037518e-05	\\
4207.78364701705	3.43500537866271e-05	\\
4208.76242897727	3.50850321386463e-05	\\
4209.7412109375	3.56667203345744e-05	\\
4210.71999289773	3.46395836088176e-05	\\
4211.69877485795	3.48884969256846e-05	\\
4212.67755681818	3.55566679093175e-05	\\
4213.65633877841	3.43581917950668e-05	\\
4214.63512073864	3.44160525109592e-05	\\
4215.61390269886	3.5657608851366e-05	\\
4216.59268465909	3.26006851913613e-05	\\
4217.57146661932	3.543819948865e-05	\\
4218.55024857955	3.50433171377788e-05	\\
4219.52903053977	3.60924084426355e-05	\\
4220.5078125	3.39324729559305e-05	\\
4221.48659446023	3.52562263382225e-05	\\
4222.46537642045	3.41837108203593e-05	\\
4223.44415838068	3.44872718270468e-05	\\
4224.42294034091	3.51851185331409e-05	\\
4225.40172230114	3.53963758028709e-05	\\
4226.38050426136	3.62964066143296e-05	\\
4227.35928622159	3.62630895020346e-05	\\
4228.33806818182	3.54458789160536e-05	\\
4229.31685014205	3.50118415754208e-05	\\
4230.29563210227	3.34552395522904e-05	\\
4231.2744140625	3.50594987144478e-05	\\
4232.25319602273	3.42771939604588e-05	\\
4233.23197798295	3.39911979547738e-05	\\
4234.21075994318	3.59870653512437e-05	\\
4235.18954190341	3.59294729876349e-05	\\
4236.16832386364	3.40889690200271e-05	\\
4237.14710582386	3.46241591555198e-05	\\
4238.12588778409	3.27415847052579e-05	\\
4239.10466974432	3.4108589121832e-05	\\
4240.08345170455	3.69136053168414e-05	\\
4241.06223366477	3.49998680013558e-05	\\
4242.041015625	3.71352495442231e-05	\\
4243.01979758523	3.43349420682958e-05	\\
4243.99857954545	3.28092779058372e-05	\\
4244.97736150568	3.74843126065444e-05	\\
4245.95614346591	3.41023331378509e-05	\\
4246.93492542614	3.52535626105578e-05	\\
4247.91370738636	3.43215824906029e-05	\\
4248.89248934659	3.51718725032374e-05	\\
4249.87127130682	3.27390162611121e-05	\\
4250.85005326705	3.55686129334944e-05	\\
4251.82883522727	3.77325290342555e-05	\\
4252.8076171875	3.26834829443497e-05	\\
4253.78639914773	3.59109225079307e-05	\\
4254.76518110795	3.50812604216961e-05	\\
4255.74396306818	3.36285048966199e-05	\\
4256.72274502841	3.39543856034529e-05	\\
4257.70152698864	3.3591271775393e-05	\\
4258.68030894886	3.25900911993443e-05	\\
4259.65909090909	3.37606934183855e-05	\\
4260.63787286932	3.28750731287026e-05	\\
4261.61665482955	3.38336831663603e-05	\\
4262.59543678977	3.38032771084011e-05	\\
4263.57421875	3.66369032763261e-05	\\
4264.55300071023	3.34997167783048e-05	\\
4265.53178267045	3.34727284566445e-05	\\
4266.51056463068	3.19993472021276e-05	\\
4267.48934659091	3.34420736639207e-05	\\
4268.46812855114	3.35404909587839e-05	\\
4269.44691051136	3.35515463832023e-05	\\
4270.42569247159	3.60247333318052e-05	\\
4271.40447443182	3.42511917270158e-05	\\
4272.38325639205	3.53681810960236e-05	\\
4273.36203835227	3.39529043082362e-05	\\
4274.3408203125	3.39876739468582e-05	\\
4275.31960227273	3.20571136703242e-05	\\
4276.29838423295	3.33249536906589e-05	\\
4277.27716619318	3.43966807448773e-05	\\
4278.25594815341	3.37049833162142e-05	\\
4279.23473011364	3.2475442616394e-05	\\
4280.21351207386	3.22155248971064e-05	\\
4281.19229403409	3.26477496312398e-05	\\
4282.17107599432	3.38644667619269e-05	\\
4283.14985795455	3.3245074608916e-05	\\
4284.12863991477	3.4071088780477e-05	\\
4285.107421875	3.43084778935217e-05	\\
4286.08620383523	3.30279938185171e-05	\\
4287.06498579545	3.10674488853641e-05	\\
4288.04376775568	3.33411343237926e-05	\\
4289.02254971591	3.38902272035202e-05	\\
4290.00133167614	3.26609078562654e-05	\\
4290.98011363636	3.46568301457193e-05	\\
4291.95889559659	3.19755363866417e-05	\\
4292.93767755682	3.50561262187247e-05	\\
4293.91645951705	3.15561942691237e-05	\\
4294.89524147727	3.32553445958266e-05	\\
4295.8740234375	3.25795573925723e-05	\\
4296.85280539773	3.11253392543165e-05	\\
4297.83158735795	3.1505549262636e-05	\\
4298.81036931818	3.29693954873795e-05	\\
4299.78915127841	3.19977371211573e-05	\\
4300.76793323864	3.20348454908577e-05	\\
4301.74671519886	3.49728960458587e-05	\\
4302.72549715909	3.38218826317062e-05	\\
4303.70427911932	2.95036617060367e-05	\\
4304.68306107955	3.33865021991275e-05	\\
4305.66184303977	3.02715250247169e-05	\\
4306.640625	3.40104815847196e-05	\\
4307.61940696023	3.31255913016112e-05	\\
4308.59818892045	3.31655068924751e-05	\\
4309.57697088068	3.27691587888153e-05	\\
4310.55575284091	3.2914796713293e-05	\\
4311.53453480114	3.3610172727763e-05	\\
4312.51331676136	3.09598572988931e-05	\\
4313.49209872159	3.37302773549183e-05	\\
4314.47088068182	3.29007348687667e-05	\\
4315.44966264205	3.39582277082636e-05	\\
4316.42844460227	3.37010878640504e-05	\\
4317.4072265625	3.45191302806288e-05	\\
4318.38600852273	3.31176758754215e-05	\\
4319.36479048295	3.15130716812106e-05	\\
4320.34357244318	3.55967777931456e-05	\\
4321.32235440341	3.30727843404395e-05	\\
4322.30113636364	3.09802896823191e-05	\\
4323.27991832386	3.31109918843393e-05	\\
4324.25870028409	3.38567226049046e-05	\\
4325.23748224432	3.29282279252888e-05	\\
4326.21626420455	3.34186128272383e-05	\\
4327.19504616477	3.35339922924162e-05	\\
4328.173828125	3.27746541568323e-05	\\
4329.15261008523	3.13955258042597e-05	\\
4330.13139204545	3.51683817623937e-05	\\
4331.11017400568	3.49060616356307e-05	\\
4332.08895596591	3.45671579776165e-05	\\
4333.06773792614	3.47448392949664e-05	\\
4334.04651988636	3.3163156771119e-05	\\
4335.02530184659	3.2831036761376e-05	\\
4336.00408380682	3.26333972429853e-05	\\
4336.98286576705	3.24919518535941e-05	\\
4337.96164772727	3.42041076458536e-05	\\
4338.9404296875	3.55744500108954e-05	\\
4339.91921164773	3.44835409704568e-05	\\
4340.89799360795	3.37252372505971e-05	\\
4341.87677556818	3.2398389553779e-05	\\
4342.85555752841	3.3693110432813e-05	\\
4343.83433948864	3.38541887997715e-05	\\
4344.81312144886	3.21361992266896e-05	\\
4345.79190340909	3.60122246777346e-05	\\
4346.77068536932	3.24520251594755e-05	\\
4347.74946732955	3.28595476332183e-05	\\
4348.72824928977	3.22114164959297e-05	\\
4349.70703125	3.3646441448743e-05	\\
4350.68581321023	3.41565058917763e-05	\\
4351.66459517045	3.29993736886286e-05	\\
4352.64337713068	3.4736141355635e-05	\\
4353.62215909091	3.4310604377607e-05	\\
4354.60094105114	3.17121582719665e-05	\\
4355.57972301136	3.24024112955316e-05	\\
4356.55850497159	3.31677436397521e-05	\\
4357.53728693182	3.30770680008922e-05	\\
4358.51606889205	3.29746254763586e-05	\\
4359.49485085227	3.1343399757224e-05	\\
4360.4736328125	3.37869731088637e-05	\\
4361.45241477273	3.18000224222258e-05	\\
4362.43119673295	3.3144636773713e-05	\\
4363.40997869318	3.43652230355167e-05	\\
4364.38876065341	3.55296001944269e-05	\\
4365.36754261364	3.41804527802665e-05	\\
4366.34632457386	3.26795439080212e-05	\\
4367.32510653409	3.53659788542259e-05	\\
4368.30388849432	3.30879978657537e-05	\\
4369.28267045455	3.13816164136304e-05	\\
4370.26145241477	3.29980288952585e-05	\\
4371.240234375	3.36483948766154e-05	\\
4372.21901633523	3.40165261048553e-05	\\
4373.19779829545	3.039166877913e-05	\\
4374.17658025568	3.39610784860632e-05	\\
4375.15536221591	3.60472488414904e-05	\\
4376.13414417614	3.46422030738745e-05	\\
4377.11292613636	3.33327450707784e-05	\\
4378.09170809659	3.29485619831788e-05	\\
4379.07049005682	3.3085191193992e-05	\\
4380.04927201705	3.23680595220179e-05	\\
4381.02805397727	3.23364932207339e-05	\\
4382.0068359375	3.47443051974035e-05	\\
4382.98561789773	3.3715397258635e-05	\\
4383.96439985795	3.41867236226714e-05	\\
4384.94318181818	3.51537005101777e-05	\\
4385.92196377841	3.45567903605276e-05	\\
4386.90074573864	3.22647561048205e-05	\\
4387.87952769886	3.3263078068046e-05	\\
4388.85830965909	3.5119119933016e-05	\\
4389.83709161932	3.27940903996108e-05	\\
4390.81587357955	3.14814627856665e-05	\\
4391.79465553977	3.15899059553573e-05	\\
4392.7734375	3.35861871612885e-05	\\
4393.75221946023	3.32142719871943e-05	\\
4394.73100142045	3.30644942303286e-05	\\
4395.70978338068	3.26457824946744e-05	\\
4396.68856534091	3.28817508970625e-05	\\
4397.66734730114	3.07271589821421e-05	\\
4398.64612926136	3.40375922654546e-05	\\
4399.62491122159	3.77077914794347e-05	\\
4400.60369318182	3.44393011281105e-05	\\
4401.58247514205	3.06349208469233e-05	\\
4402.56125710227	3.45483814121225e-05	\\
4403.5400390625	3.19532108996657e-05	\\
4404.51882102273	3.25133760591098e-05	\\
4405.49760298295	3.29402620836374e-05	\\
4406.47638494318	3.25563118650913e-05	\\
4407.45516690341	3.36365601772657e-05	\\
4408.43394886364	3.32828146376574e-05	\\
4409.41273082386	3.22696980931661e-05	\\
4410.39151278409	3.33599534003039e-05	\\
4411.37029474432	3.16817613567857e-05	\\
4412.34907670455	3.18218534882456e-05	\\
4413.32785866477	3.09348635932056e-05	\\
4414.306640625	3.16094815433976e-05	\\
4415.28542258523	3.17661592056329e-05	\\
4416.26420454545	3.25972480530001e-05	\\
4417.24298650568	3.15631115994833e-05	\\
4418.22176846591	3.49789680169152e-05	\\
4419.20055042614	3.21794836570613e-05	\\
4420.17933238636	3.23105039292325e-05	\\
4421.15811434659	3.37601442473793e-05	\\
4422.13689630682	3.24274868930923e-05	\\
4423.11567826705	2.97161508890962e-05	\\
4424.09446022727	3.10444938847399e-05	\\
4425.0732421875	3.13784244418026e-05	\\
4426.05202414773	3.18253775789422e-05	\\
4427.03080610795	3.20088396189123e-05	\\
4428.00958806818	3.05334349403089e-05	\\
4428.98837002841	2.96528271984983e-05	\\
4429.96715198864	3.34783069681707e-05	\\
4430.94593394886	3.19418801004453e-05	\\
4431.92471590909	2.91303404342308e-05	\\
4432.90349786932	3.10498350247196e-05	\\
4433.88227982955	3.16950369904701e-05	\\
4434.86106178977	3.18761828571679e-05	\\
4435.83984375	3.15193583344512e-05	\\
4436.81862571023	3.07157356379005e-05	\\
4437.79740767045	3.11750266038337e-05	\\
4438.77618963068	3.34531208311042e-05	\\
4439.75497159091	3.43143680660188e-05	\\
4440.73375355114	3.13489181462075e-05	\\
4441.71253551136	2.88993223595701e-05	\\
4442.69131747159	3.15659309732391e-05	\\
4443.67009943182	2.98736898969717e-05	\\
4444.64888139205	3.22438901863614e-05	\\
4445.62766335227	3.17763362540555e-05	\\
4446.6064453125	2.95245920775919e-05	\\
4447.58522727273	3.28821586381087e-05	\\
4448.56400923295	3.11271399097038e-05	\\
4449.54279119318	3.10129727708621e-05	\\
4450.52157315341	3.25976997878961e-05	\\
4451.50035511364	3.27003191969045e-05	\\
4452.47913707386	3.02642002902806e-05	\\
4453.45791903409	3.16423314606101e-05	\\
4454.43670099432	3.09006517886862e-05	\\
4455.41548295455	3.13757681557317e-05	\\
4456.39426491477	3.107191840732e-05	\\
4457.373046875	3.24321061439939e-05	\\
4458.35182883523	3.07762872114968e-05	\\
4459.33061079545	3.12028754994883e-05	\\
4460.30939275568	2.98673259594534e-05	\\
4461.28817471591	2.91829550833507e-05	\\
4462.26695667614	3.088859144162e-05	\\
4463.24573863636	3.12086485031754e-05	\\
4464.22452059659	3.33366369336921e-05	\\
4465.20330255682	3.11679311709099e-05	\\
4466.18208451705	2.87458967972225e-05	\\
4467.16086647727	3.24378898614638e-05	\\
4468.1396484375	2.81960198601701e-05	\\
4469.11843039773	3.0805201359234e-05	\\
4470.09721235795	3.08243899429313e-05	\\
4471.07599431818	2.98973458256812e-05	\\
4472.05477627841	3.19073231160632e-05	\\
4473.03355823864	3.16614812139366e-05	\\
4474.01234019886	3.02553940195165e-05	\\
4474.99112215909	3.06439944656548e-05	\\
4475.96990411932	3.04919255083236e-05	\\
4476.94868607955	3.19804204524326e-05	\\
4477.92746803977	2.99615634348374e-05	\\
4478.90625	3.0303682846756e-05	\\
4479.88503196023	2.73308625642006e-05	\\
4480.86381392045	3.06616263640707e-05	\\
4481.84259588068	2.82966743421704e-05	\\
4482.82137784091	3.21745113858326e-05	\\
4483.80015980114	2.99617711793581e-05	\\
4484.77894176136	3.18085536907407e-05	\\
4485.75772372159	3.07787351301286e-05	\\
4486.73650568182	3.25085066777566e-05	\\
4487.71528764205	3.18104268737079e-05	\\
4488.69406960227	2.8713809569376e-05	\\
4489.6728515625	2.98471412852203e-05	\\
4490.65163352273	3.17549740002031e-05	\\
4491.63041548295	3.04057753901568e-05	\\
4492.60919744318	2.95803637161445e-05	\\
4493.58797940341	2.99000449230308e-05	\\
4494.56676136364	2.91369827655263e-05	\\
4495.54554332386	3.0923878093453e-05	\\
4496.52432528409	2.99472315410325e-05	\\
4497.50310724432	3.05620424367457e-05	\\
4498.48188920455	2.85871043686736e-05	\\
4499.46067116477	2.40305952218599e-05	\\
4500.439453125	2.71307179739496e-05	\\
4501.41823508523	2.95958863038523e-05	\\
4502.39701704545	2.9387723529721e-05	\\
4503.37579900568	3.20994904688071e-05	\\
4504.35458096591	3.07233698188008e-05	\\
4505.33336292614	3.05653702573234e-05	\\
4506.31214488636	3.10752082401659e-05	\\
4507.29092684659	3.01650089687926e-05	\\
4508.26970880682	2.87925371915558e-05	\\
4509.24849076705	2.73246495160566e-05	\\
4510.22727272727	3.0380567432697e-05	\\
4511.2060546875	3.05927442339334e-05	\\
4512.18483664773	2.9619562771163e-05	\\
4513.16361860795	2.80823020989079e-05	\\
4514.14240056818	2.89727785245329e-05	\\
4515.12118252841	3.06091026992776e-05	\\
4516.09996448864	2.83981246277137e-05	\\
4517.07874644886	3.00535492431541e-05	\\
4518.05752840909	3.03071828042427e-05	\\
4519.03631036932	3.00298032053489e-05	\\
4520.01509232955	3.08918842158216e-05	\\
4520.99387428977	2.74858008545113e-05	\\
4521.97265625	2.82218672314643e-05	\\
4522.95143821023	3.21891521504185e-05	\\
4523.93022017045	3.08821860861023e-05	\\
4524.90900213068	2.79545409334384e-05	\\
4525.88778409091	2.82436476104098e-05	\\
4526.86656605114	2.88139762277135e-05	\\
4527.84534801136	3.02882194002734e-05	\\
4528.82412997159	2.78039073241268e-05	\\
4529.80291193182	2.64089130507338e-05	\\
4530.78169389205	2.51992179863756e-05	\\
4531.76047585227	2.75068754621257e-05	\\
4532.7392578125	2.78333528787037e-05	\\
4533.71803977273	2.97427329486731e-05	\\
4534.69682173295	2.90092300478233e-05	\\
4535.67560369318	2.89283886130174e-05	\\
4536.65438565341	2.91961265946138e-05	\\
4537.63316761364	2.80008783016049e-05	\\
4538.61194957386	2.86127605326872e-05	\\
4539.59073153409	2.85430970352585e-05	\\
4540.56951349432	2.89788029052335e-05	\\
4541.54829545455	2.82223717230512e-05	\\
4542.52707741477	2.61634941728213e-05	\\
4543.505859375	2.86705406584118e-05	\\
4544.48464133523	2.73703224913139e-05	\\
4545.46342329545	2.82645443137656e-05	\\
4546.44220525568	2.63234998864682e-05	\\
4547.42098721591	2.84961557835167e-05	\\
4548.39976917614	2.95442355705399e-05	\\
4549.37855113636	2.77017225231844e-05	\\
4550.35733309659	2.76846309279693e-05	\\
4551.33611505682	2.89984668371343e-05	\\
4552.31489701705	2.58273895765833e-05	\\
4553.29367897727	2.58354791952665e-05	\\
4554.2724609375	2.55364126795981e-05	\\
4555.25124289773	2.95417209487889e-05	\\
4556.23002485795	2.53387841408716e-05	\\
4557.20880681818	2.88439645497163e-05	\\
4558.18758877841	2.65970824764954e-05	\\
4559.16637073864	2.5949860231447e-05	\\
4560.14515269886	2.6323307558689e-05	\\
4561.12393465909	2.78226048455989e-05	\\
4562.10271661932	2.72191638787437e-05	\\
4563.08149857955	2.64332686954062e-05	\\
4564.06028053977	2.55982180084293e-05	\\
4565.0390625	2.81908809583993e-05	\\
4566.01784446023	2.78250997461323e-05	\\
4566.99662642045	2.70024167994467e-05	\\
4567.97540838068	2.87372985077184e-05	\\
4568.95419034091	2.61810028177076e-05	\\
4569.93297230114	2.67926137435145e-05	\\
4570.91175426136	2.60714930975269e-05	\\
4571.89053622159	2.69706514730061e-05	\\
4572.86931818182	2.4192423774666e-05	\\
4573.84810014205	2.51040201586521e-05	\\
4574.82688210227	2.54845066092273e-05	\\
4575.8056640625	2.36637306860579e-05	\\
4576.78444602273	2.68601268935168e-05	\\
4577.76322798295	2.5969550754694e-05	\\
4578.74200994318	2.61109328832031e-05	\\
4579.72079190341	2.61564110341234e-05	\\
4580.69957386364	2.62235260641615e-05	\\
4581.67835582386	2.58056161178786e-05	\\
4582.65713778409	2.57436645162635e-05	\\
4583.63591974432	2.50980312197006e-05	\\
4584.61470170455	2.41878697602132e-05	\\
4585.59348366477	2.70879359963872e-05	\\
4586.572265625	2.33778521809666e-05	\\
4587.55104758523	2.58451157187862e-05	\\
4588.52982954545	2.27952736352494e-05	\\
4589.50861150568	2.77123459058749e-05	\\
4590.48739346591	2.65582976571853e-05	\\
4591.46617542614	2.57277591255334e-05	\\
4592.44495738636	2.50871243253782e-05	\\
4593.42373934659	2.54012328660213e-05	\\
4594.40252130682	2.55007983808658e-05	\\
4595.38130326705	2.44933559983431e-05	\\
4596.36008522727	2.58050572589769e-05	\\
4597.3388671875	2.59631418915926e-05	\\
4598.31764914773	2.41871078980882e-05	\\
4599.29643110795	2.72024224775505e-05	\\
4600.27521306818	2.4995722786168e-05	\\
4601.25399502841	2.57438046629373e-05	\\
4602.23277698864	2.24045386872905e-05	\\
4603.21155894886	2.43718667000606e-05	\\
4604.19034090909	2.60173568363865e-05	\\
4605.16912286932	2.48903436352957e-05	\\
4606.14790482955	2.65192006650996e-05	\\
4607.12668678977	2.36471556486488e-05	\\
4608.10546875	2.64290633377096e-05	\\
4609.08425071023	2.51156822767434e-05	\\
4610.06303267045	2.44628552511855e-05	\\
4611.04181463068	2.63694004566117e-05	\\
4612.02059659091	2.59881598711274e-05	\\
4612.99937855114	2.4676187743644e-05	\\
4613.97816051136	2.44550320893877e-05	\\
4614.95694247159	2.46804219459984e-05	\\
4615.93572443182	2.29529012543564e-05	\\
4616.91450639205	2.67598820356943e-05	\\
4617.89328835227	2.58563646286546e-05	\\
4618.8720703125	2.72066309089637e-05	\\
4619.85085227273	2.20709611859989e-05	\\
4620.82963423295	2.76585044484731e-05	\\
4621.80841619318	2.52969692338833e-05	\\
4622.78719815341	2.37211259471855e-05	\\
4623.76598011364	2.64965611352141e-05	\\
4624.74476207386	2.40631499093656e-05	\\
4625.72354403409	2.57702388229485e-05	\\
4626.70232599432	2.50691735780676e-05	\\
4627.68110795455	2.57219312841589e-05	\\
4628.65988991477	2.4447058410653e-05	\\
4629.638671875	2.84340458906968e-05	\\
4630.61745383523	2.38398732874729e-05	\\
4631.59623579545	2.52021857128584e-05	\\
4632.57501775568	2.55098458253845e-05	\\
4633.55379971591	2.37619202353967e-05	\\
4634.53258167614	2.67454799933587e-05	\\
4635.51136363636	2.54234225744709e-05	\\
};
\addplot [color=blue,solid,forget plot]
  table[row sep=crcr]{
4635.51136363636	2.54234225744709e-05	\\
4636.49014559659	2.68209128170558e-05	\\
4637.46892755682	2.41558897612031e-05	\\
4638.44770951705	2.31707148160428e-05	\\
4639.42649147727	2.25611795152719e-05	\\
4640.4052734375	2.59806038422348e-05	\\
4641.38405539773	2.51271166871197e-05	\\
4642.36283735795	2.30565177327314e-05	\\
4643.34161931818	2.3662661449575e-05	\\
4644.32040127841	2.44938354854258e-05	\\
4645.29918323864	2.39728807060142e-05	\\
4646.27796519886	2.37531125986942e-05	\\
4647.25674715909	2.22337904031382e-05	\\
4648.23552911932	2.39786218807514e-05	\\
4649.21431107955	2.45962864402631e-05	\\
4650.19309303977	2.53317240644854e-05	\\
4651.171875	2.44408756125047e-05	\\
4652.15065696023	2.5304682023329e-05	\\
4653.12943892045	2.31036089019556e-05	\\
4654.10822088068	2.56229281120267e-05	\\
4655.08700284091	2.43651714108412e-05	\\
4656.06578480114	2.45590677228465e-05	\\
4657.04456676136	2.22915541145934e-05	\\
4658.02334872159	2.48043149400997e-05	\\
4659.00213068182	2.28307368941201e-05	\\
4659.98091264205	2.36737235455205e-05	\\
4660.95969460227	2.43502236788825e-05	\\
4661.9384765625	2.35767835930172e-05	\\
4662.91725852273	2.50794206167244e-05	\\
4663.89604048295	2.40420324739042e-05	\\
4664.87482244318	2.47832205416121e-05	\\
4665.85360440341	2.25632549155568e-05	\\
4666.83238636364	2.27803910807378e-05	\\
4667.81116832386	2.36770960457788e-05	\\
4668.78995028409	2.38240625895316e-05	\\
4669.76873224432	2.52736450022048e-05	\\
4670.74751420455	2.3886581668041e-05	\\
4671.72629616477	2.37899946720449e-05	\\
4672.705078125	2.26316421168743e-05	\\
4673.68386008523	2.25205896097181e-05	\\
4674.66264204545	2.27787987371462e-05	\\
4675.64142400568	2.1896903199618e-05	\\
4676.62020596591	2.51982195602165e-05	\\
4677.59898792614	2.40343109505817e-05	\\
4678.57776988636	2.35569720633168e-05	\\
4679.55655184659	2.43881275095511e-05	\\
4680.53533380682	2.37369537604607e-05	\\
4681.51411576705	2.44057082041566e-05	\\
4682.49289772727	2.27865320613628e-05	\\
4683.4716796875	2.49668091739029e-05	\\
4684.45046164773	2.29753947876983e-05	\\
4685.42924360795	2.3111138270439e-05	\\
4686.40802556818	2.27931384984485e-05	\\
4687.38680752841	2.41354285770709e-05	\\
4688.36558948864	2.38579643428839e-05	\\
4689.34437144886	2.45194971571154e-05	\\
4690.32315340909	2.19463745091205e-05	\\
4691.30193536932	2.22939708650307e-05	\\
4692.28071732955	2.45536177457322e-05	\\
4693.25949928977	2.28514486968552e-05	\\
4694.23828125	2.18978421554699e-05	\\
4695.21706321023	2.54930787524755e-05	\\
4696.19584517045	2.11483925571861e-05	\\
4697.17462713068	2.14032613473103e-05	\\
4698.15340909091	2.27775116184228e-05	\\
4699.13219105114	2.29278489311647e-05	\\
4700.11097301136	2.22796550165981e-05	\\
4701.08975497159	2.22140470399798e-05	\\
4702.06853693182	2.27219932093501e-05	\\
4703.04731889205	2.28235894643043e-05	\\
4704.02610085227	2.19972791599914e-05	\\
4705.0048828125	2.11806360510715e-05	\\
4705.98366477273	2.22489467398192e-05	\\
4706.96244673295	2.20567361827978e-05	\\
4707.94122869318	2.34493233756991e-05	\\
4708.92001065341	2.25403717099026e-05	\\
4709.89879261364	2.14431212194826e-05	\\
4710.87757457386	2.06211502645387e-05	\\
4711.85635653409	2.14052195976432e-05	\\
4712.83513849432	2.26480233883852e-05	\\
4713.81392045455	2.1506517189889e-05	\\
4714.79270241477	2.26202483366372e-05	\\
4715.771484375	2.51555206924496e-05	\\
4716.75026633523	2.17498983599566e-05	\\
4717.72904829545	2.19408020311296e-05	\\
4718.70783025568	2.3856258921605e-05	\\
4719.68661221591	2.40641135237026e-05	\\
4720.66539417614	2.55217967006813e-05	\\
4721.64417613636	2.47961910940312e-05	\\
4722.62295809659	2.2320396131542e-05	\\
4723.60174005682	2.12359554128103e-05	\\
4724.58052201705	2.43578910709571e-05	\\
4725.55930397727	2.35915532328164e-05	\\
4726.5380859375	2.22831703527214e-05	\\
4727.51686789773	2.28613649931711e-05	\\
4728.49564985795	2.09074823471222e-05	\\
4729.47443181818	2.33680511525247e-05	\\
4730.45321377841	2.26165483416366e-05	\\
4731.43199573864	2.41204949446178e-05	\\
4732.41077769886	2.37335894464128e-05	\\
4733.38955965909	2.3416194341535e-05	\\
4734.36834161932	2.31698377683119e-05	\\
4735.34712357955	2.13875252717645e-05	\\
4736.32590553977	2.32544434812342e-05	\\
4737.3046875	2.21920589237759e-05	\\
4738.28346946023	2.29906228857652e-05	\\
4739.26225142045	2.32879747075583e-05	\\
4740.24103338068	2.43966625128907e-05	\\
4741.21981534091	2.2753006879401e-05	\\
4742.19859730114	2.23790990613333e-05	\\
4743.17737926136	2.30614192631526e-05	\\
4744.15616122159	2.35685701055096e-05	\\
4745.13494318182	2.36295841343066e-05	\\
4746.11372514205	2.20798832683951e-05	\\
4747.09250710227	2.10176132780935e-05	\\
4748.0712890625	2.18032683049709e-05	\\
4749.05007102273	2.20966516699317e-05	\\
4750.02885298295	1.93132443545283e-05	\\
4751.00763494318	2.19781936989413e-05	\\
4751.98641690341	2.05797427163411e-05	\\
4752.96519886364	1.96673934302712e-05	\\
4753.94398082386	2.28756655110585e-05	\\
4754.92276278409	2.19048581302078e-05	\\
4755.90154474432	2.17977105659914e-05	\\
4756.88032670455	2.2128578588028e-05	\\
4757.85910866477	2.28341420971678e-05	\\
4758.837890625	2.12912785679423e-05	\\
4759.81667258523	2.04217376058972e-05	\\
4760.79545454545	2.24248025376796e-05	\\
4761.77423650568	2.28858515489981e-05	\\
4762.75301846591	2.28046719648014e-05	\\
4763.73180042614	2.10042445575313e-05	\\
4764.71058238636	2.26339032877066e-05	\\
4765.68936434659	2.12330734367231e-05	\\
4766.66814630682	2.35541825197053e-05	\\
4767.64692826705	2.09986597654534e-05	\\
4768.62571022727	2.34877445406811e-05	\\
4769.6044921875	2.09771702802687e-05	\\
4770.58327414773	2.14166420087582e-05	\\
4771.56205610795	1.83128520449957e-05	\\
4772.54083806818	2.27014193497636e-05	\\
4773.51962002841	2.14842617633293e-05	\\
4774.49840198864	2.12482658461466e-05	\\
4775.47718394886	2.2138574637177e-05	\\
4776.45596590909	2.24605028842021e-05	\\
4777.43474786932	2.22718422559832e-05	\\
4778.41352982955	1.96827958910457e-05	\\
4779.39231178977	2.11852570254735e-05	\\
4780.37109375	2.15200549995047e-05	\\
4781.34987571023	2.14704341098814e-05	\\
4782.32865767045	2.16892178813489e-05	\\
4783.30743963068	2.05708180410383e-05	\\
4784.28622159091	2.20761886736583e-05	\\
4785.26500355114	2.10126365272023e-05	\\
4786.24378551136	2.11855807693084e-05	\\
4787.22256747159	2.08330242346759e-05	\\
4788.20134943182	1.90596393154593e-05	\\
4789.18013139205	2.04263269422406e-05	\\
4790.15891335227	2.13078545034211e-05	\\
4791.1376953125	1.84576340117297e-05	\\
4792.11647727273	2.21238886419044e-05	\\
4793.09525923295	1.99495591720587e-05	\\
4794.07404119318	2.14383400681894e-05	\\
4795.05282315341	2.09403515314977e-05	\\
4796.03160511364	2.18712636615266e-05	\\
4797.01038707386	1.99671474248562e-05	\\
4797.98916903409	2.00037295301326e-05	\\
4798.96795099432	1.97233898608012e-05	\\
4799.94673295455	2.09842214992362e-05	\\
4800.92551491477	2.0777762797389e-05	\\
4801.904296875	2.2223335843652e-05	\\
4802.88307883523	2.15230610746463e-05	\\
4803.86186079545	2.03061228598077e-05	\\
4804.84064275568	2.18633768829945e-05	\\
4805.81942471591	2.15486283133539e-05	\\
4806.79820667614	2.1669918267842e-05	\\
4807.77698863636	2.16214120269657e-05	\\
4808.75577059659	2.03852696976586e-05	\\
4809.73455255682	1.89325833345783e-05	\\
4810.71333451705	2.16572193365178e-05	\\
4811.69211647727	1.95647110569905e-05	\\
4812.6708984375	1.9153758513585e-05	\\
4813.64968039773	1.97956733408894e-05	\\
4814.62846235795	1.89109430653268e-05	\\
4815.60724431818	1.78102962942985e-05	\\
4816.58602627841	1.80724118683929e-05	\\
4817.56480823864	2.21821395818628e-05	\\
4818.54359019886	2.03391044924687e-05	\\
4819.52237215909	2.06729176377349e-05	\\
4820.50115411932	2.06192257918496e-05	\\
4821.47993607955	1.93605953617648e-05	\\
4822.45871803977	1.85180461384951e-05	\\
4823.4375	2.05662742299835e-05	\\
4824.41628196023	1.88521069282704e-05	\\
4825.39506392045	1.9719893076554e-05	\\
4826.37384588068	2.13126891824438e-05	\\
4827.35262784091	1.91293697815364e-05	\\
4828.33140980114	2.13750595499101e-05	\\
4829.31019176136	1.98727863946462e-05	\\
4830.28897372159	1.84203612871886e-05	\\
4831.26775568182	2.02125984855054e-05	\\
4832.24653764205	2.07512965397224e-05	\\
4833.22531960227	1.92865362986879e-05	\\
4834.2041015625	1.91229100799716e-05	\\
4835.18288352273	2.22075459235524e-05	\\
4836.16166548295	2.0148143634241e-05	\\
4837.14044744318	2.07397801571515e-05	\\
4838.11922940341	1.95141759199961e-05	\\
4839.09801136364	2.03537436366129e-05	\\
4840.07679332386	2.04833138497479e-05	\\
4841.05557528409	2.19868695742328e-05	\\
4842.03435724432	2.08383207307465e-05	\\
4843.01313920455	2.0903611477947e-05	\\
4843.99192116477	1.91999758093064e-05	\\
4844.970703125	2.13574272591831e-05	\\
4845.94948508523	2.00215624297193e-05	\\
4846.92826704545	2.01197740884707e-05	\\
4847.90704900568	2.26445250542874e-05	\\
4848.88583096591	2.15101960108289e-05	\\
4849.86461292614	1.89568243250859e-05	\\
4850.84339488636	1.95250375201066e-05	\\
4851.82217684659	1.94573945912405e-05	\\
4852.80095880682	1.83292171770646e-05	\\
4853.77974076705	2.00333806251267e-05	\\
4854.75852272727	1.93461039106264e-05	\\
4855.7373046875	1.82467967111204e-05	\\
4856.71608664773	1.98006503430818e-05	\\
4857.69486860795	2.05141453695047e-05	\\
4858.67365056818	1.95206716984508e-05	\\
4859.65243252841	1.96365887112293e-05	\\
4860.63121448864	1.98317202384471e-05	\\
4861.60999644886	1.82089393223765e-05	\\
4862.58877840909	1.85755095022445e-05	\\
4863.56756036932	1.88695756527491e-05	\\
4864.54634232955	1.92723578405879e-05	\\
4865.52512428977	1.88638608028199e-05	\\
4866.50390625	1.90700297907524e-05	\\
4867.48268821023	1.92047199289914e-05	\\
4868.46147017045	1.93969727054338e-05	\\
4869.44025213068	1.8141774128237e-05	\\
4870.41903409091	1.96100462909191e-05	\\
4871.39781605114	1.84372188014044e-05	\\
4872.37659801136	1.79897944958303e-05	\\
4873.35537997159	1.9313395288316e-05	\\
4874.33416193182	1.70735078067142e-05	\\
4875.31294389205	1.94578984301067e-05	\\
4876.29172585227	1.93023707886227e-05	\\
4877.2705078125	1.88983066716961e-05	\\
4878.24928977273	1.74376599152288e-05	\\
4879.22807173295	1.91924634818596e-05	\\
4880.20685369318	1.83058814013946e-05	\\
4881.18563565341	2.0619465712161e-05	\\
4882.16441761364	1.94973625366227e-05	\\
4883.14319957386	1.6689665988861e-05	\\
4884.12198153409	1.93841255288325e-05	\\
4885.10076349432	1.88223369295706e-05	\\
4886.07954545455	2.0001869321531e-05	\\
4887.05832741477	1.93035821217663e-05	\\
4888.037109375	1.71879523235259e-05	\\
4889.01589133523	1.89118114526873e-05	\\
4889.99467329545	1.9760945207341e-05	\\
4890.97345525568	1.71283220624368e-05	\\
4891.95223721591	1.76603486363206e-05	\\
4892.93101917614	1.74953086684252e-05	\\
4893.90980113636	1.69867229480863e-05	\\
4894.88858309659	1.65451072326359e-05	\\
4895.86736505682	1.52427232092729e-05	\\
4896.84614701705	1.83705699425739e-05	\\
4897.82492897727	1.6090147864985e-05	\\
4898.8037109375	1.81821578530808e-05	\\
4899.78249289773	1.85920077643208e-05	\\
4900.76127485795	1.65032944189537e-05	\\
4901.74005681818	1.73888595023739e-05	\\
4902.71883877841	1.70810845545941e-05	\\
4903.69762073864	1.77522617630349e-05	\\
4904.67640269886	1.54879731255346e-05	\\
4905.65518465909	1.68692758141966e-05	\\
4906.63396661932	1.57232900493356e-05	\\
4907.61274857955	1.84750012610085e-05	\\
4908.59153053977	1.50606340003156e-05	\\
4909.5703125	1.64977833362471e-05	\\
4910.54909446023	1.66622220397207e-05	\\
4911.52787642045	1.59318939658553e-05	\\
4912.50665838068	1.46890927109404e-05	\\
4913.48544034091	1.57483187546445e-05	\\
4914.46422230114	1.66597465484779e-05	\\
4915.44300426136	1.52149358660077e-05	\\
4916.42178622159	1.73401021537326e-05	\\
4917.40056818182	1.71653569311946e-05	\\
4918.37935014205	1.80402331511489e-05	\\
4919.35813210227	1.65395818802921e-05	\\
4920.3369140625	1.75899280783252e-05	\\
4921.31569602273	1.77424939295402e-05	\\
4922.29447798295	1.61225849073919e-05	\\
4923.27325994318	1.61770275674424e-05	\\
4924.25204190341	1.5758044474239e-05	\\
4925.23082386364	1.54529161247138e-05	\\
4926.20960582386	1.364147954605e-05	\\
4927.18838778409	1.36494390830952e-05	\\
4928.16716974432	1.69499200708182e-05	\\
4929.14595170455	1.7403027123715e-05	\\
4930.12473366477	1.39586571139512e-05	\\
4931.103515625	1.72853032588585e-05	\\
4932.08229758523	1.75672797511568e-05	\\
4933.06107954545	1.63883782284827e-05	\\
4934.03986150568	1.46937650665324e-05	\\
4935.01864346591	1.75963037402781e-05	\\
4935.99742542614	1.78397026696931e-05	\\
4936.97620738636	1.56942450840825e-05	\\
4937.95498934659	1.5783504213248e-05	\\
4938.93377130682	1.59188674757494e-05	\\
4939.91255326705	1.54268172263567e-05	\\
4940.89133522727	1.63787468790245e-05	\\
4941.8701171875	1.72071034663971e-05	\\
4942.84889914773	1.62713308674954e-05	\\
4943.82768110795	1.41231057863203e-05	\\
4944.80646306818	1.4298938239469e-05	\\
4945.78524502841	1.75061813757617e-05	\\
4946.76402698864	1.55282583300416e-05	\\
4947.74280894886	1.67265469040876e-05	\\
4948.72159090909	1.56561534382226e-05	\\
4949.70037286932	1.53879359239252e-05	\\
4950.67915482955	1.5929648763658e-05	\\
4951.65793678977	1.39003031111797e-05	\\
4952.63671875	1.72462944661927e-05	\\
4953.61550071023	1.6726100206074e-05	\\
4954.59428267045	1.60173990888204e-05	\\
4955.57306463068	1.55339469431808e-05	\\
4956.55184659091	1.51229706070278e-05	\\
4957.53062855114	1.56255238654886e-05	\\
4958.50941051136	1.58271229107658e-05	\\
4959.48819247159	1.52903799934745e-05	\\
4960.46697443182	1.58494600858405e-05	\\
4961.44575639205	1.49010055575935e-05	\\
4962.42453835227	1.68818791261352e-05	\\
4963.4033203125	1.4299505469702e-05	\\
4964.38210227273	1.60092149565778e-05	\\
4965.36088423295	1.56459047514577e-05	\\
4966.33966619318	1.65122108875955e-05	\\
4967.31844815341	1.54925806678082e-05	\\
4968.29723011364	1.44965345583009e-05	\\
4969.27601207386	1.43768463795073e-05	\\
4970.25479403409	1.39768680854514e-05	\\
4971.23357599432	1.7304408001125e-05	\\
4972.21235795455	1.55069075644611e-05	\\
4973.19113991477	1.37971074724687e-05	\\
4974.169921875	1.37568251923925e-05	\\
4975.14870383523	1.64106138044906e-05	\\
4976.12748579545	1.52395167292643e-05	\\
4977.10626775568	1.6463586348437e-05	\\
4978.08504971591	1.52226523569622e-05	\\
4979.06383167614	1.37853004668816e-05	\\
4980.04261363636	1.74975010618703e-05	\\
4981.02139559659	1.43129848744843e-05	\\
4982.00017755682	1.45680520179693e-05	\\
4982.97895951705	1.45321145718836e-05	\\
4983.95774147727	1.5636520630716e-05	\\
4984.9365234375	1.41354480869629e-05	\\
4985.91530539773	1.42919117747532e-05	\\
4986.89408735795	1.55832646959035e-05	\\
4987.87286931818	1.40979564391709e-05	\\
4988.85165127841	1.13216152848398e-05	\\
4989.83043323864	1.42315974786696e-05	\\
4990.80921519886	1.47329011192349e-05	\\
4991.78799715909	1.57579932108106e-05	\\
4992.76677911932	1.60788082580717e-05	\\
4993.74556107955	1.48621341865424e-05	\\
4994.72434303977	1.29965776283105e-05	\\
4995.703125	1.46044499687616e-05	\\
4996.68190696023	1.47557735920408e-05	\\
4997.66068892045	1.37261524688853e-05	\\
4998.63947088068	1.45727123630716e-05	\\
4999.61825284091	1.42698968137538e-05	\\
5000.59703480114	1.25594044505984e-05	\\
5001.57581676136	1.49563998426017e-05	\\
5002.55459872159	1.52253194762034e-05	\\
5003.53338068182	1.48193207936811e-05	\\
5004.51216264205	1.41216581022507e-05	\\
5005.49094460227	1.46965845512475e-05	\\
5006.4697265625	1.53854957615744e-05	\\
5007.44850852273	1.35070772217017e-05	\\
5008.42729048295	1.37020891964775e-05	\\
5009.40607244318	1.30619572060789e-05	\\
5010.38485440341	1.44368754165396e-05	\\
5011.36363636364	1.46575322760948e-05	\\
5012.34241832386	1.39207941958743e-05	\\
5013.32120028409	1.50772740014307e-05	\\
5014.29998224432	1.28971369586751e-05	\\
5015.27876420455	1.2864749429143e-05	\\
5016.25754616477	1.36418356608059e-05	\\
5017.236328125	1.55170301286565e-05	\\
5018.21511008523	1.2545343904138e-05	\\
5019.19389204545	1.4512914036516e-05	\\
5020.17267400568	1.3813422143605e-05	\\
5021.15145596591	1.43123206224825e-05	\\
5022.13023792614	1.32442185569365e-05	\\
5023.10901988636	1.32015478306994e-05	\\
5024.08780184659	1.26674860583442e-05	\\
5025.06658380682	1.38921382535061e-05	\\
5026.04536576705	1.29074330652568e-05	\\
5027.02414772727	1.34548510816308e-05	\\
5028.0029296875	1.28431257858492e-05	\\
5028.98171164773	1.54413018028926e-05	\\
5029.96049360795	1.07694321869024e-05	\\
5030.93927556818	1.23633258999866e-05	\\
5031.91805752841	1.39317750713881e-05	\\
5032.89683948864	1.3766682231916e-05	\\
5033.87562144886	1.13427766886002e-05	\\
5034.85440340909	1.50375961555609e-05	\\
5035.83318536932	1.47184718255408e-05	\\
5036.81196732955	1.5612407442259e-05	\\
5037.79074928977	1.5186814593088e-05	\\
5038.76953125	1.38391329824032e-05	\\
5039.74831321023	1.32361306286663e-05	\\
5040.72709517045	1.2531619476005e-05	\\
5041.70587713068	1.29210348313316e-05	\\
5042.68465909091	1.33206850583128e-05	\\
5043.66344105114	1.53942896661112e-05	\\
5044.64222301136	1.46234692918679e-05	\\
5045.62100497159	1.41410771812582e-05	\\
5046.59978693182	1.34112226107101e-05	\\
5047.57856889205	1.28803300965049e-05	\\
5048.55735085227	1.04469068061881e-05	\\
5049.5361328125	1.16345525344174e-05	\\
5050.51491477273	1.46202498644477e-05	\\
5051.49369673295	1.18074045463803e-05	\\
5052.47247869318	1.19647986467653e-05	\\
5053.45126065341	1.40016986954069e-05	\\
5054.43004261364	1.31244579648498e-05	\\
5055.40882457386	1.35404614271219e-05	\\
5056.38760653409	1.50726316937125e-05	\\
5057.36638849432	1.22068127244601e-05	\\
5058.34517045455	1.20465644962356e-05	\\
5059.32395241477	1.46149386460475e-05	\\
5060.302734375	1.3350197564548e-05	\\
5061.28151633523	1.3983089460575e-05	\\
5062.26029829545	1.18783872813711e-05	\\
5063.23908025568	1.31922456489795e-05	\\
5064.21786221591	1.55467942640596e-05	\\
5065.19664417614	1.3362626667033e-05	\\
5066.17542613636	1.42494593642031e-05	\\
5067.15420809659	1.3173312989528e-05	\\
5068.13299005682	1.27451052450687e-05	\\
5069.11177201705	1.15244585816157e-05	\\
5070.09055397727	1.32017392511447e-05	\\
5071.0693359375	1.19570959035584e-05	\\
5072.04811789773	1.19612807650232e-05	\\
5073.02689985795	1.32943355424184e-05	\\
5074.00568181818	1.42011700591108e-05	\\
5074.98446377841	1.1074877664363e-05	\\
5075.96324573864	1.3958005711732e-05	\\
5076.94202769886	1.05172512682507e-05	\\
5077.92080965909	1.18261966941396e-05	\\
5078.89959161932	1.32895054785241e-05	\\
5079.87837357955	1.18415797187407e-05	\\
5080.85715553977	1.29898828987879e-05	\\
5081.8359375	1.23887408614183e-05	\\
5082.81471946023	1.21563724680173e-05	\\
5083.79350142045	1.2145038753289e-05	\\
5084.77228338068	1.17535981817518e-05	\\
5085.75106534091	1.35266732170701e-05	\\
5086.72984730114	1.26702890101183e-05	\\
5087.70862926136	1.170866272073e-05	\\
5088.68741122159	1.28214663562429e-05	\\
5089.66619318182	1.21311322396846e-05	\\
5090.64497514205	1.32973633270412e-05	\\
5091.62375710227	1.24810882999824e-05	\\
5092.6025390625	1.34142951558377e-05	\\
5093.58132102273	1.2339557416585e-05	\\
5094.56010298295	1.0899645268167e-05	\\
5095.53888494318	1.32962710860665e-05	\\
5096.51766690341	1.2040315596342e-05	\\
5097.49644886364	1.0438161262533e-05	\\
5098.47523082386	1.19365285506424e-05	\\
5099.45401278409	1.09841657090204e-05	\\
5100.43279474432	1.20191320784505e-05	\\
5101.41157670455	1.15388689413153e-05	\\
5102.39035866477	1.18307131714877e-05	\\
5103.369140625	1.04003765593207e-05	\\
5104.34792258523	1.13326883920386e-05	\\
5105.32670454545	1.1951687791438e-05	\\
5106.30548650568	1.08592064014383e-05	\\
5107.28426846591	1.22227311265977e-05	\\
5108.26305042614	1.07070333196695e-05	\\
5109.24183238636	1.1916340332677e-05	\\
5110.22061434659	1.21637552406209e-05	\\
5111.19939630682	1.17525984104873e-05	\\
5112.17817826705	1.22238617930523e-05	\\
5113.15696022727	1.14166554382953e-05	\\
5114.1357421875	1.16623455872092e-05	\\
5115.11452414773	1.24045713285598e-05	\\
5116.09330610795	1.04683157182517e-05	\\
5117.07208806818	1.17538125214182e-05	\\
5118.05087002841	1.23988483666045e-05	\\
5119.02965198864	1.16900243739636e-05	\\
5120.00843394886	1.08831209439848e-05	\\
5120.98721590909	1.37398473711899e-05	\\
5121.96599786932	1.29479692715903e-05	\\
5122.94477982955	1.27391261815541e-05	\\
5123.92356178977	1.21495221331822e-05	\\
5124.90234375	1.08195064182716e-05	\\
5125.88112571023	1.09455762466489e-05	\\
5126.85990767045	1.04842715731993e-05	\\
5127.83868963068	1.0745211973741e-05	\\
5128.81747159091	1.0355731402568e-05	\\
5129.79625355114	1.14948620038817e-05	\\
5130.77503551136	1.07616593258064e-05	\\
5131.75381747159	1.12975384743272e-05	\\
5132.73259943182	1.29401774285617e-05	\\
5133.71138139205	1.05916962570627e-05	\\
5134.69016335227	1.18616350368732e-05	\\
5135.6689453125	1.01299711702282e-05	\\
5136.64772727273	1.04988798707441e-05	\\
5137.62650923295	1.18969483390149e-05	\\
5138.60529119318	1.22633574327934e-05	\\
5139.58407315341	1.094858599789e-05	\\
5140.56285511364	1.1940443731773e-05	\\
5141.54163707386	1.10764013229568e-05	\\
5142.52041903409	1.19706574691733e-05	\\
5143.49920099432	1.23593116494116e-05	\\
5144.47798295455	1.14511303207811e-05	\\
5145.45676491477	8.39812274939686e-06	\\
5146.435546875	1.22386869632293e-05	\\
5147.41432883523	8.44556317327079e-06	\\
5148.39311079545	9.97586578082539e-06	\\
5149.37189275568	1.11259620698797e-05	\\
5150.35067471591	1.090577903594e-05	\\
5151.32945667614	8.14673463578032e-06	\\
5152.30823863636	1.31106224679887e-05	\\
5153.28702059659	8.98176781824093e-06	\\
5154.26580255682	1.10558800440652e-05	\\
5155.24458451705	1.06700763536672e-05	\\
5156.22336647727	1.04841424970098e-05	\\
5157.2021484375	1.03362178547837e-05	\\
5158.18093039773	1.0341058102373e-05	\\
5159.15971235795	1.08465477756061e-05	\\
5160.13849431818	1.14948617898034e-05	\\
5161.11727627841	9.5060054410902e-06	\\
5162.09605823864	1.06775902387809e-05	\\
5163.07484019886	8.8970066730408e-06	\\
5164.05362215909	1.08272195713392e-05	\\
5165.03240411932	9.86041571311596e-06	\\
5166.01118607955	1.14248382779741e-05	\\
5166.98996803977	1.20291156300775e-05	\\
5167.96875	1.06642987679196e-05	\\
5168.94753196023	9.90278235227101e-06	\\
5169.92631392045	8.82602737998598e-06	\\
5170.90509588068	7.55654782443623e-06	\\
5171.88387784091	9.28242289318479e-06	\\
5172.86265980114	9.49504906691263e-06	\\
5173.84144176136	1.17495791204833e-05	\\
5174.82022372159	9.87868205917725e-06	\\
5175.79900568182	9.80731471286379e-06	\\
5176.77778764205	1.00474912585655e-05	\\
5177.75656960227	1.01690673670092e-05	\\
5178.7353515625	9.87859910617639e-06	\\
5179.71413352273	1.16292951727276e-05	\\
5180.69291548295	1.00880083844038e-05	\\
5181.67169744318	9.66316270766932e-06	\\
5182.65047940341	1.00957548120783e-05	\\
5183.62926136364	1.00244928143893e-05	\\
5184.60804332386	9.7126649985473e-06	\\
5185.58682528409	9.96574400170652e-06	\\
5186.56560724432	9.38230431662936e-06	\\
5187.54438920455	9.8232085013634e-06	\\
5188.52317116477	9.64461268062731e-06	\\
5189.501953125	8.91745227423467e-06	\\
5190.48073508523	1.08421615295379e-05	\\
5191.45951704545	1.16301911745168e-05	\\
5192.43829900568	1.02368502262597e-05	\\
5193.41708096591	8.55045404932202e-06	\\
5194.39586292614	8.26242295124987e-06	\\
5195.37464488636	9.62928742924164e-06	\\
5196.35342684659	9.55446111427073e-06	\\
5197.33220880682	8.94170775384942e-06	\\
5198.31099076705	9.11398981192027e-06	\\
5199.28977272727	1.083715198097e-05	\\
5200.2685546875	8.50294347426778e-06	\\
5201.24733664773	9.34265267710763e-06	\\
5202.22611860795	8.30546331317969e-06	\\
5203.20490056818	8.16962905565869e-06	\\
5204.18368252841	8.98891007542519e-06	\\
5205.16246448864	1.00113451833129e-05	\\
5206.14124644886	9.74173551972345e-06	\\
5207.12002840909	8.0043443234627e-06	\\
5208.09881036932	9.37319445901511e-06	\\
5209.07759232955	8.4067015229925e-06	\\
5210.05637428977	1.06979131802917e-05	\\
5211.03515625	1.26506698012263e-05	\\
5212.01393821023	1.00401118626219e-05	\\
5212.99272017045	8.68299163659968e-06	\\
5213.97150213068	8.92015378202372e-06	\\
5214.95028409091	7.95540242621574e-06	\\
5215.92906605114	8.91865390508975e-06	\\
5216.90784801136	7.72798162597143e-06	\\
5217.88662997159	8.40565819239667e-06	\\
5218.86541193182	9.33370564864592e-06	\\
5219.84419389205	1.07918622662269e-05	\\
5220.82297585227	6.81242847588497e-06	\\
5221.8017578125	9.0410553831789e-06	\\
5222.78053977273	7.72528658827423e-06	\\
5223.75932173295	7.37573015148506e-06	\\
5224.73810369318	7.83028882921204e-06	\\
5225.71688565341	7.96289431080868e-06	\\
5226.69566761364	9.08782170345361e-06	\\
5227.67444957386	1.03791556782597e-05	\\
5228.65323153409	7.50545411255459e-06	\\
5229.63201349432	9.59128494678791e-06	\\
5230.61079545455	8.93194035569029e-06	\\
5231.58957741477	7.497153209043e-06	\\
5232.568359375	8.69764753806636e-06	\\
5233.54714133523	1.0451868143951e-05	\\
5234.52592329545	7.25223137428109e-06	\\
5235.50470525568	8.47193127094265e-06	\\
5236.48348721591	1.06137920169473e-05	\\
5237.46226917614	9.14085480665667e-06	\\
5238.44105113636	6.35773824912379e-06	\\
5239.41983309659	7.7840538851796e-06	\\
5240.39861505682	8.35993005202834e-06	\\
5241.37739701705	1.03772821074155e-05	\\
5242.35617897727	8.49499412920708e-06	\\
5243.3349609375	7.99688681109769e-06	\\
5244.31374289773	9.44168408197482e-06	\\
5245.29252485795	1.01584037883701e-05	\\
5246.27130681818	8.8308376931459e-06	\\
5247.25008877841	9.29629711504888e-06	\\
5248.22887073864	8.64247602364868e-06	\\
5249.20765269886	7.58224764829982e-06	\\
5250.18643465909	1.01462052408556e-05	\\
5251.16521661932	5.61374323166661e-06	\\
5252.14399857955	9.43323961642247e-06	\\
5253.12278053977	9.54094187709251e-06	\\
5254.1015625	8.38676633721425e-06	\\
5255.08034446023	7.45947433868711e-06	\\
5256.05912642045	9.77580752993282e-06	\\
5257.03790838068	8.83124181947338e-06	\\
5258.01669034091	9.12959258423361e-06	\\
5258.99547230114	7.70806530591943e-06	\\
5259.97425426136	7.67749993728488e-06	\\
5260.95303622159	8.24847357848963e-06	\\
5261.93181818182	8.31947119182719e-06	\\
5262.91060014205	7.35499449713878e-06	\\
5263.88938210227	7.57653967756023e-06	\\
5264.8681640625	8.51868658169387e-06	\\
5265.84694602273	9.46730650116492e-06	\\
5266.82572798295	9.52631972229368e-06	\\
5267.80450994318	8.44761452907982e-06	\\
5268.78329190341	9.14170147927029e-06	\\
5269.76207386364	8.71109936213589e-06	\\
5270.74085582386	8.33328182716991e-06	\\
5271.71963778409	6.95278746584165e-06	\\
5272.69841974432	8.11594989694408e-06	\\
5273.67720170455	5.74961320827577e-06	\\
5274.65598366477	9.84469485595679e-06	\\
5275.634765625	7.63301530388815e-06	\\
5276.61354758523	6.68106082676831e-06	\\
5277.59232954545	6.5356503392288e-06	\\
5278.57111150568	1.01312312349022e-05	\\
5279.54989346591	7.06220135534147e-06	\\
5280.52867542614	8.9270775445994e-06	\\
5281.50745738636	5.38231469700063e-06	\\
5282.48623934659	7.87947997982137e-06	\\
5283.46502130682	7.41930296790914e-06	\\
5284.44380326705	7.51058982346251e-06	\\
5285.42258522727	6.52334930179393e-06	\\
5286.4013671875	5.87053459300198e-06	\\
5287.38014914773	8.24841712726116e-06	\\
5288.35893110795	8.1422746245765e-06	\\
5289.33771306818	1.03061093361066e-05	\\
5290.31649502841	8.11725756226479e-06	\\
5291.29527698864	7.6050250409256e-06	\\
5292.27405894886	8.48974115894688e-06	\\
5293.25284090909	6.09931949140241e-06	\\
5294.23162286932	5.6611157498365e-06	\\
5295.21040482955	7.7233911747051e-06	\\
5296.18918678977	7.06184352019165e-06	\\
5297.16796875	6.3713062123929e-06	\\
5298.14675071023	7.8141632639685e-06	\\
5299.12553267045	6.96435097530212e-06	\\
5300.10431463068	6.37142975277194e-06	\\
5301.08309659091	8.18310726453467e-06	\\
5302.06187855114	9.48253101283424e-06	\\
5303.04066051136	8.79755904480906e-06	\\
5304.01944247159	7.88702302981969e-06	\\
5304.99822443182	7.19602442156513e-06	\\
5305.97700639205	5.20835494418126e-06	\\
5306.95578835227	8.71001349253767e-06	\\
5307.9345703125	7.4598346544008e-06	\\
5308.91335227273	6.38877101513248e-06	\\
5309.89213423295	6.02734696670679e-06	\\
5310.87091619318	6.36016113875627e-06	\\
5311.84969815341	6.93569585685143e-06	\\
5312.82848011364	4.79604089785092e-06	\\
5313.80726207386	7.24818731030994e-06	\\
5314.78604403409	7.31111146595509e-06	\\
5315.76482599432	7.47417861794247e-06	\\
5316.74360795455	8.33576139919797e-06	\\
5317.72238991477	7.11955001449307e-06	\\
5318.701171875	5.07545279642176e-06	\\
5319.67995383523	6.71214839521846e-06	\\
5320.65873579545	6.27378404631214e-06	\\
5321.63751775568	8.40893889786133e-06	\\
5322.61629971591	7.70840090816852e-06	\\
5323.59508167614	5.5059769069516e-06	\\
5324.57386363636	6.0801300725747e-06	\\
5325.55264559659	7.72697405764298e-06	\\
5326.53142755682	9.33993613889446e-06	\\
5327.51020951705	5.28076468548383e-06	\\
5328.48899147727	6.21298353553175e-06	\\
5329.4677734375	6.29278780488368e-06	\\
5330.44655539773	5.59757616985623e-06	\\
5331.42533735795	7.14530208353253e-06	\\
5332.40411931818	4.99500889058363e-06	\\
5333.38290127841	8.0557234968821e-06	\\
5334.36168323864	8.10302600409124e-06	\\
5335.34046519886	5.74263700454656e-06	\\
5336.31924715909	7.02618856071983e-06	\\
5337.29802911932	7.56322115061055e-06	\\
5338.27681107955	5.51832506320012e-06	\\
5339.25559303977	6.30530873730269e-06	\\
5340.234375	8.28199732998112e-06	\\
5341.21315696023	6.33402208319315e-06	\\
5342.19193892045	6.99017042286526e-06	\\
5343.17072088068	4.43744629913431e-06	\\
5344.14950284091	6.69566534636342e-06	\\
5345.12828480114	5.01060606755444e-06	\\
5346.10706676136	5.32244617641599e-06	\\
5347.08584872159	5.72184289747337e-06	\\
5348.06463068182	5.62084329909844e-06	\\
5349.04341264205	8.68264536346362e-06	\\
5350.02219460227	4.95997654613046e-06	\\
5351.0009765625	6.18342334840672e-06	\\
5351.97975852273	5.54787711825419e-06	\\
5352.95854048295	8.22220305659903e-06	\\
5353.93732244318	5.55821316989416e-06	\\
5354.91610440341	3.10834656864339e-06	\\
5355.89488636364	4.90766580550344e-06	\\
5356.87366832386	3.96743940527714e-06	\\
5357.85245028409	5.02056144951631e-06	\\
5358.83123224432	6.79640723645879e-06	\\
5359.81001420455	4.57319373482484e-06	\\
5360.78879616477	5.75911431932695e-06	\\
5361.767578125	5.23294769093151e-06	\\
5362.74636008523	5.99827914901776e-06	\\
5363.72514204545	5.39456817200177e-06	\\
5364.70392400568	6.61859295652934e-06	\\
5365.68270596591	5.96139616157358e-06	\\
5366.66148792614	6.00743078945007e-06	\\
5367.64026988636	5.07154628033617e-06	\\
5368.61905184659	3.99740145876865e-06	\\
5369.59783380682	5.93367167568714e-06	\\
5370.57661576705	6.54101948231723e-06	\\
5371.55539772727	7.51032629897029e-06	\\
5372.5341796875	4.14542618124596e-06	\\
5373.51296164773	5.6735085431436e-06	\\
5374.49174360795	5.66585759318276e-06	\\
5375.47052556818	5.00656629703358e-06	\\
5376.44930752841	6.51770629097137e-06	\\
5377.42808948864	4.78515288893764e-06	\\
5378.40687144886	6.18499592433606e-06	\\
5379.38565340909	5.74308387970533e-06	\\
5380.36443536932	5.16391467458329e-06	\\
5381.34321732955	6.81682875598599e-06	\\
5382.32199928977	4.78580726223221e-06	\\
5383.30078125	4.75656788903514e-06	\\
5384.27956321023	6.78158026196859e-06	\\
5385.25834517045	5.18554179816559e-06	\\
5386.23712713068	5.01916967009867e-06	\\
5387.21590909091	5.56744443191845e-06	\\
5388.19469105114	4.0627960718839e-06	\\
5389.17347301136	5.62065065497103e-06	\\
5390.15225497159	6.26589832473426e-06	\\
5391.13103693182	5.17344232571752e-06	\\
5392.10981889205	6.12788146133799e-06	\\
5393.08860085227	4.00994495457421e-06	\\
5394.0673828125	4.55001076621074e-06	\\
5395.04616477273	6.37899022588532e-06	\\
5396.02494673295	6.5806676604055e-06	\\
5397.00372869318	5.03211490314501e-06	\\
5397.98251065341	5.46731549425516e-06	\\
5398.96129261364	5.64914925187575e-06	\\
5399.94007457386	2.90031280022907e-06	\\
5400.91885653409	4.7790540368055e-06	\\
5401.89763849432	5.77215999642903e-06	\\
5402.87642045455	4.52940335327135e-06	\\
5403.85520241477	6.73500312999681e-06	\\
5404.833984375	3.2845330367596e-06	\\
5405.81276633523	4.58058429977663e-06	\\
5406.79154829545	5.39962587370106e-06	\\
5407.77033025568	4.55606434211635e-06	\\
5408.74911221591	3.8674712129874e-06	\\
5409.72789417614	5.4624704488963e-06	\\
5410.70667613636	4.64894790165254e-06	\\
5411.68545809659	3.73946165092316e-06	\\
5412.66424005682	3.39923608318156e-06	\\
5413.64302201705	5.23140722533141e-06	\\
5414.62180397727	4.37834599897247e-06	\\
5415.6005859375	3.03091571885825e-06	\\
5416.57936789773	5.10742296250142e-06	\\
5417.55814985795	3.59527486643709e-06	\\
5418.53693181818	2.77730985921359e-06	\\
5419.51571377841	4.59816628356286e-06	\\
5420.49449573864	5.12439361927181e-06	\\
5421.47327769886	4.96164581897878e-06	\\
5422.45205965909	3.58553872876674e-06	\\
5423.43084161932	4.01390538501982e-06	\\
5424.40962357955	2.62666249698531e-06	\\
5425.38840553977	3.43112418022194e-06	\\
5426.3671875	4.73534193303709e-06	\\
5427.34596946023	3.06057540198154e-06	\\
5428.32475142045	3.19699352200059e-06	\\
5429.30353338068	4.04795761758847e-06	\\
5430.28231534091	4.30672826991119e-06	\\
5431.26109730114	5.60446785975293e-06	\\
5432.23987926136	5.02806012648195e-06	\\
5433.21866122159	5.14144746212968e-06	\\
5434.19744318182	2.73016947506973e-06	\\
5435.17622514205	5.92372231458171e-06	\\
5436.15500710227	5.52711025930034e-06	\\
5437.1337890625	4.2661817863537e-06	\\
5438.11257102273	3.98613333337245e-06	\\
5439.09135298295	5.48062888979438e-06	\\
5440.07013494318	3.11681025295727e-06	\\
5441.04891690341	3.15920497732422e-06	\\
5442.02769886364	5.20289902603444e-06	\\
5443.00648082386	3.9663498038807e-06	\\
5443.98526278409	5.58487000495601e-06	\\
5444.96404474432	3.75452590910588e-06	\\
5445.94282670455	4.58526820037636e-06	\\
5446.92160866477	3.19392670199188e-06	\\
5447.900390625	3.49319879730547e-06	\\
5448.87917258523	7.35518687347091e-06	\\
5449.85795454545	3.48953294251815e-06	\\
5450.83673650568	2.94827223611196e-06	\\
5451.81551846591	3.58356244730089e-06	\\
5452.79430042614	4.60981965580343e-06	\\
5453.77308238636	4.01315577137422e-06	\\
5454.75186434659	2.94404130602052e-06	\\
5455.73064630682	3.66735456418892e-06	\\
5456.70942826705	3.67510090423449e-06	\\
5457.68821022727	4.63285479051515e-06	\\
5458.6669921875	2.11945108440207e-06	\\
5459.64577414773	4.17081231938498e-06	\\
5460.62455610795	3.36496789550804e-06	\\
5461.60333806818	1.93533598033263e-06	\\
5462.58212002841	4.97090260783623e-06	\\
5463.56090198864	3.3137210004509e-06	\\
5464.53968394886	4.21753472410353e-06	\\
5465.51846590909	3.46730108071842e-06	\\
5466.49724786932	3.6102258472408e-06	\\
5467.47602982955	3.04359967938048e-06	\\
5468.45481178977	1.98546131998449e-06	\\
5469.43359375	3.50033042633927e-06	\\
5470.41237571023	4.61641881209402e-06	\\
5471.39115767045	2.05040301635333e-06	\\
5472.36993963068	2.24424310066123e-06	\\
5473.34872159091	3.65757031496851e-06	\\
5474.32750355114	4.80045365556606e-06	\\
5475.30628551136	4.29930672369003e-06	\\
5476.28506747159	5.41104995251887e-06	\\
5477.26384943182	2.31400212786446e-06	\\
5478.24263139205	3.14702925220268e-06	\\
5479.22141335227	3.87969302625123e-06	\\
5480.2001953125	2.61754973986274e-06	\\
5481.17897727273	2.62221713826437e-06	\\
5482.15775923295	3.96969332828661e-06	\\
5483.13654119318	2.14650786800395e-06	\\
5484.11532315341	1.88851401498737e-06	\\
5485.09410511364	4.58286034767414e-06	\\
5486.07288707386	3.71763126664881e-06	\\
5487.05166903409	3.36292615674927e-06	\\
5488.03045099432	3.32777669582084e-06	\\
5489.00923295455	1.3068766440264e-06	\\
5489.98801491477	2.72544083105568e-06	\\
5490.966796875	2.73050841282613e-06	\\
5491.94557883523	2.50459614046506e-06	\\
5492.92436079545	2.86424021006539e-06	\\
5493.90314275568	2.28084364543513e-06	\\
5494.88192471591	2.15143217428906e-06	\\
5495.86070667614	2.91909570791727e-06	\\
5496.83948863636	3.90083729839067e-06	\\
5497.81827059659	3.07043140593241e-06	\\
5498.79705255682	1.80447008280795e-06	\\
5499.77583451705	1.99872182790458e-06	\\
5500.75461647727	1.44188838741465e-06	\\
5501.7333984375	2.63096675466936e-06	\\
5502.71218039773	1.53204575114561e-06	\\
5503.69096235795	1.74584230749751e-06	\\
5504.66974431818	2.26583800692182e-06	\\
5505.64852627841	2.3475277170237e-06	\\
5506.62730823864	2.56204844126422e-06	\\
5507.60609019886	4.11176363350335e-06	\\
5508.58487215909	2.47171491616947e-06	\\
5509.56365411932	2.17808526143806e-06	\\
5510.54243607955	3.97227853660492e-06	\\
5511.52121803977	1.9159204970526e-06	\\
5512.5	1.63834666168623e-06	\\
5513.47878196023	9.98909627085533e-07	\\
5514.45756392045	3.75267268212606e-06	\\
5515.43634588068	2.55087686986875e-06	\\
5516.41512784091	3.43889172519947e-06	\\
5517.39390980114	2.15928471225174e-06	\\
5518.37269176136	1.47401990240473e-06	\\
5519.35147372159	4.40538441606794e-06	\\
5520.33025568182	3.04025154860771e-06	\\
5521.30903764205	2.63357439019759e-06	\\
5522.28781960227	3.87557919313462e-06	\\
5523.2666015625	2.3499572543406e-06	\\
5524.24538352273	2.4203643250241e-06	\\
5525.22416548295	9.51265978290034e-07	\\
5526.20294744318	1.43539572244978e-06	\\
5527.18172940341	1.83625398930382e-06	\\
5528.16051136364	1.36647464194486e-06	\\
5529.13929332386	2.28654660140905e-06	\\
5530.11807528409	4.12168035907684e-06	\\
5531.09685724432	1.89566885540811e-06	\\
5532.07563920455	2.8363923362149e-06	\\
5533.05442116477	2.48880731352997e-06	\\
5534.033203125	9.85784324732954e-07	\\
5535.01198508523	1.85720305442416e-06	\\
5535.99076704545	2.96055810203108e-06	\\
5536.96954900568	2.28854549944614e-06	\\
5537.94833096591	2.54988169469145e-06	\\
5538.92711292614	2.36925064354264e-06	\\
5539.90589488636	9.9956641359876e-07	\\
5540.88467684659	1.32555962287282e-06	\\
5541.86345880682	3.02663910022844e-06	\\
5542.84224076705	1.81151579571597e-06	\\
5543.82102272727	1.55573109047875e-06	\\
5544.7998046875	2.72585138848447e-06	\\
5545.77858664773	1.29428456009264e-06	\\
5546.75736860795	3.34625942645954e-06	\\
5547.73615056818	1.92805024952396e-06	\\
5548.71493252841	2.84994537666657e-06	\\
5549.69371448864	4.4626462730168e-06	\\
5550.67249644886	2.01929331868661e-06	\\
5551.65127840909	1.39547104342369e-06	\\
5552.63006036932	1.69617049856307e-06	\\
5553.60884232955	1.62527201444339e-06	\\
5554.58762428977	4.36531545173569e-06	\\
5555.56640625	3.28841160865367e-06	\\
5556.54518821023	1.8581203867649e-06	\\
5557.52397017045	2.61819644647415e-06	\\
5558.50275213068	1.737455776883e-06	\\
5559.48153409091	2.25358104648117e-06	\\
5560.46031605114	1.08679305711696e-06	\\
5561.43909801136	2.11392700314969e-06	\\
5562.41787997159	2.26312475220962e-06	\\
5563.39666193182	3.56923759703695e-06	\\
5564.37544389205	1.77122234722883e-06	\\
5565.35422585227	2.77138455037893e-06	\\
5566.3330078125	6.42827157596374e-07	\\
5567.31178977273	1.56119061716774e-06	\\
5568.29057173295	1.58466560341614e-06	\\
5569.26935369318	3.00941156161105e-06	\\
5570.24813565341	3.82653838026807e-07	\\
5571.22691761364	1.91067861254133e-06	\\
5572.20569957386	3.21953803585953e-06	\\
5573.18448153409	1.31553464819913e-06	\\
5574.16326349432	1.91880568076104e-06	\\
5575.14204545455	2.42834614390103e-06	\\
5576.12082741477	2.50092046554426e-06	\\
5577.099609375	2.89690156287906e-06	\\
5578.07839133523	1.48850289800425e-06	\\
5579.05717329545	2.56325724185649e-06	\\
5580.03595525568	1.3488612542671e-06	\\
5581.01473721591	1.9955361955916e-06	\\
5581.99351917614	1.04671350994675e-06	\\
5582.97230113636	2.18748273236117e-06	\\
5583.95108309659	1.03982953163674e-06	\\
5584.92986505682	1.27070454063261e-06	\\
5585.90864701705	1.94529435004197e-06	\\
5586.88742897727	3.31354859278215e-06	\\
5587.8662109375	1.93242425568271e-06	\\
5588.84499289773	1.93516583630776e-06	\\
5589.82377485795	2.09579030600454e-06	\\
5590.80255681818	1.79720278998695e-06	\\
5591.78133877841	1.66413753389027e-06	\\
5592.76012073864	3.98184676653276e-06	\\
5593.73890269886	4.13439686708873e-06	\\
5594.71768465909	3.50591487095135e-06	\\
5595.69646661932	3.3395622015146e-06	\\
5596.67524857955	2.04737181678572e-06	\\
5597.65403053977	8.17209115154463e-07	\\
5598.6328125	2.40673385492808e-06	\\
5599.61159446023	2.14615058898293e-06	\\
5600.59037642045	1.89620147350251e-06	\\
5601.56915838068	2.34429313877198e-06	\\
5602.54794034091	1.17853549550202e-06	\\
5603.52672230114	1.72453714113792e-06	\\
5604.50550426136	2.71029655308374e-06	\\
5605.48428622159	2.83214087268586e-06	\\
5606.46306818182	3.72841523131879e-06	\\
5607.44185014205	7.21048644362967e-07	\\
5608.42063210227	4.64958618954838e-06	\\
5609.3994140625	2.64776034737558e-06	\\
5610.37819602273	2.93930401277507e-06	\\
5611.35697798295	1.72697080838684e-06	\\
5612.33575994318	2.59943367010581e-06	\\
5613.31454190341	2.28374009697683e-06	\\
5614.29332386364	2.95248712019765e-06	\\
5615.27210582386	1.03689677015613e-06	\\
5616.25088778409	2.44739801943348e-06	\\
5617.22966974432	1.94937361504381e-06	\\
5618.20845170455	3.68586603599502e-06	\\
5619.18723366477	5.83462168635838e-06	\\
5620.166015625	3.56214291763677e-06	\\
5621.14479758523	4.33236121280905e-06	\\
5622.12357954545	2.59522662367627e-06	\\
5623.10236150568	2.72915062757932e-06	\\
5624.08114346591	3.42388741736621e-06	\\
5625.05992542614	2.94881916889357e-06	\\
5626.03870738636	2.46836346463536e-06	\\
5627.01748934659	1.85082764419644e-06	\\
5627.99627130682	3.5259111888451e-06	\\
5628.97505326705	2.31554879634993e-06	\\
5629.95383522727	4.62042026250683e-06	\\
5630.9326171875	4.57425050330195e-06	\\
5631.91139914773	4.27892990780979e-06	\\
5632.89018110795	3.11306109111305e-06	\\
5633.86896306818	3.48554904785775e-06	\\
5634.84774502841	2.59308227489426e-06	\\
5635.82652698864	4.46735225854482e-06	\\
5636.80530894886	3.72048805047492e-06	\\
5637.78409090909	2.58812922476142e-06	\\
5638.76287286932	2.65796971364613e-06	\\
5639.74165482955	5.6089455472304e-06	\\
5640.72043678977	3.54358322503483e-06	\\
5641.69921875	2.7596736747187e-06	\\
5642.67800071023	3.76340601570487e-06	\\
5643.65678267045	2.6546289350296e-06	\\
5644.63556463068	3.41142915564446e-06	\\
5645.61434659091	2.35060859948032e-06	\\
5646.59312855114	2.99060329307127e-06	\\
5647.57191051136	2.80504669889547e-06	\\
5648.55069247159	2.48068598412947e-06	\\
5649.52947443182	2.40458402707604e-06	\\
5650.50825639205	3.05690366003323e-06	\\
5651.48703835227	8.65520992137649e-07	\\
5652.4658203125	4.14704813562134e-06	\\
5653.44460227273	2.55228807523807e-06	\\
5654.42338423295	3.27457872844562e-06	\\
5655.40216619318	2.82752735074356e-06	\\
5656.38094815341	2.90942201871212e-06	\\
5657.35973011364	2.14071708976162e-06	\\
5658.33851207386	4.10394283103811e-06	\\
5659.31729403409	1.1204966582178e-06	\\
5660.29607599432	3.8598559437409e-06	\\
5661.27485795455	3.33446512203468e-06	\\
5662.25363991477	4.27944557430937e-06	\\
5663.232421875	2.91896846640479e-06	\\
5664.21120383523	2.78536634946e-06	\\
5665.18998579545	2.75243558772374e-06	\\
5666.16876775568	4.00948816596416e-06	\\
5667.14754971591	3.35590651364209e-06	\\
5668.12633167614	3.01122529544294e-06	\\
5669.10511363636	2.95840451396965e-06	\\
5670.08389559659	3.24381185265142e-06	\\
5671.06267755682	2.45067699082365e-06	\\
5672.04145951705	2.67893639823999e-06	\\
5673.02024147727	3.19062719525392e-06	\\
5673.9990234375	2.40347771439867e-06	\\
5674.97780539773	2.22801019519187e-06	\\
5675.95658735795	1.92801628205255e-06	\\
5676.93536931818	3.05815296727589e-06	\\
5677.91415127841	2.98733022667537e-06	\\
5678.89293323864	2.45324946061198e-06	\\
5679.87171519886	2.51064672552932e-06	\\
5680.85049715909	4.89436084249702e-06	\\
5681.82927911932	3.59435918221634e-06	\\
5682.80806107955	2.55673766799839e-06	\\
5683.78684303977	3.38036438428574e-06	\\
5684.765625	2.2566948721409e-06	\\
5685.74440696023	7.71944773034193e-07	\\
5686.72318892045	2.70374276607321e-06	\\
5687.70197088068	1.95797652583886e-06	\\
5688.68075284091	3.36316648601853e-06	\\
5689.65953480114	4.41401830269714e-06	\\
5690.63831676136	2.42983134696921e-06	\\
5691.61709872159	2.69882308124388e-06	\\
5692.59588068182	1.51753119117785e-06	\\
5693.57466264205	1.55473768334289e-06	\\
5694.55344460227	1.6871370181616e-06	\\
5695.5322265625	2.87244863801995e-06	\\
5696.51100852273	3.43067230715313e-06	\\
5697.48979048295	2.49374024315065e-06	\\
5698.46857244318	2.32337283666842e-06	\\
5699.44735440341	2.81551884172352e-06	\\
5700.42613636364	3.23526475361748e-06	\\
5701.40491832386	5.00018976506416e-06	\\
5702.38370028409	2.45342285762778e-06	\\
5703.36248224432	4.99512009306568e-06	\\
5704.34126420455	3.21275875236496e-06	\\
5705.32004616477	5.31233568694704e-07	\\
5706.298828125	2.1740315360634e-06	\\
5707.27761008523	2.22654087022812e-06	\\
5708.25639204545	2.60391460121648e-06	\\
5709.23517400568	1.35710878946157e-06	\\
5710.21395596591	3.6167107967681e-06	\\
5711.19273792614	2.31336931568266e-06	\\
5712.17151988636	2.59549451129393e-06	\\
5713.15030184659	1.39756901001076e-06	\\
5714.12908380682	4.16286148829105e-06	\\
5715.10786576705	1.88789511933992e-06	\\
5716.08664772727	3.75150820964911e-06	\\
5717.0654296875	3.66856863305158e-06	\\
5718.04421164773	3.47488720093756e-06	\\
5719.02299360795	3.36492554327865e-06	\\
5720.00177556818	4.50147730488051e-06	\\
5720.98055752841	2.75806298980645e-06	\\
5721.95933948864	3.73825673870317e-06	\\
5722.93812144886	3.99009868420121e-06	\\
5723.91690340909	3.18177188510659e-06	\\
5724.89568536932	1.24861283318404e-06	\\
5725.87446732955	5.10177691158592e-06	\\
5726.85324928977	5.30722791316222e-06	\\
5727.83203125	4.34444678893001e-06	\\
5728.81081321023	3.62389701408104e-06	\\
5729.78959517045	3.49213091636749e-06	\\
5730.76837713068	4.58249757145236e-06	\\
5731.74715909091	3.15229724941845e-06	\\
5732.72594105114	4.13215673990551e-06	\\
5733.70472301136	2.69680026393806e-06	\\
5734.68350497159	3.81141031054895e-06	\\
5735.66228693182	3.81797059697275e-06	\\
5736.64106889205	4.40211784199102e-06	\\
5737.61985085227	4.90165696718367e-06	\\
5738.5986328125	2.80458612422658e-06	\\
5739.57741477273	3.77581330940419e-06	\\
5740.55619673295	5.66359129102891e-06	\\
5741.53497869318	5.63755669765929e-06	\\
5742.51376065341	2.92695000981175e-06	\\
5743.49254261364	4.32069056073315e-06	\\
5744.47132457386	4.81586910408351e-06	\\
5745.45010653409	4.24968422454317e-06	\\
5746.42888849432	3.39279334996302e-06	\\
5747.40767045455	4.1267110476503e-06	\\
5748.38645241477	4.12898943431546e-06	\\
5749.365234375	4.23832730720087e-06	\\
5750.34401633523	3.44911443090692e-06	\\
5751.32279829545	3.57755780114107e-06	\\
5752.30158025568	5.51323412443657e-06	\\
5753.28036221591	2.69985316105076e-06	\\
5754.25914417614	4.73806181517037e-06	\\
5755.23792613636	5.84269239095972e-06	\\
5756.21670809659	6.84312774691063e-06	\\
5757.19549005682	3.94120504677216e-06	\\
5758.17427201705	4.07919649298169e-06	\\
5759.15305397727	5.23378352938129e-06	\\
5760.1318359375	5.39722873996538e-06	\\
5761.11061789773	3.31451858030038e-06	\\
5762.08939985795	6.15409083016731e-06	\\
5763.06818181818	5.5148333850542e-06	\\
5764.04696377841	4.39036270382512e-06	\\
5765.02574573864	5.47772608614066e-06	\\
5766.00452769886	4.35167994062655e-06	\\
5766.98330965909	5.38645904635496e-06	\\
5767.96209161932	4.69367113315428e-06	\\
5768.94087357955	4.68065702743549e-06	\\
5769.91965553977	2.97129392370794e-06	\\
5770.8984375	6.33852826776542e-06	\\
5771.87721946023	3.28722562783976e-06	\\
5772.85600142045	5.22319296197987e-06	\\
5773.83478338068	5.8263628685887e-06	\\
5774.81356534091	5.09604791110146e-06	\\
5775.79234730114	7.05056542881873e-06	\\
5776.77112926136	4.62955539743308e-06	\\
5777.74991122159	4.19602735762775e-06	\\
5778.72869318182	6.38963815475968e-06	\\
5779.70747514205	5.91556128972426e-06	\\
5780.68625710227	3.34921918319382e-06	\\
5781.6650390625	6.17298810463645e-06	\\
5782.64382102273	5.7244812927116e-06	\\
5783.62260298295	3.66777488144297e-06	\\
5784.60138494318	5.52941465289892e-06	\\
5785.58016690341	7.00462621275288e-06	\\
5786.55894886364	3.24766140326912e-06	\\
5787.53773082386	3.55152946419158e-06	\\
5788.51651278409	4.1033032289168e-06	\\
5789.49529474432	5.54557989673096e-06	\\
5790.47407670455	3.00885606805617e-06	\\
5791.45285866477	4.75236773939548e-06	\\
5792.431640625	5.19199018568388e-06	\\
5793.41042258523	5.11473612369279e-06	\\
5794.38920454545	4.60840614496541e-06	\\
5795.36798650568	4.07984330491375e-06	\\
5796.34676846591	9.07622113018184e-06	\\
5797.32555042614	3.53701902049728e-06	\\
5798.30433238636	6.17152930158459e-06	\\
5799.28311434659	5.43268885976532e-06	\\
5800.26189630682	7.44385229677776e-06	\\
5801.24067826705	5.5703614250066e-06	\\
5802.21946022727	4.43950129407114e-06	\\
5803.1982421875	5.52045044374246e-06	\\
5804.17702414773	6.08399516857077e-06	\\
5805.15580610795	6.00990513483603e-06	\\
5806.13458806818	4.44073555699521e-06	\\
5807.11337002841	7.0462344606459e-06	\\
5808.09215198864	5.85473282528339e-06	\\
5809.07093394886	5.01709511621846e-06	\\
5810.04971590909	6.68822429688192e-06	\\
5811.02849786932	4.17000050415817e-06	\\
5812.00727982955	8.1378539219231e-06	\\
5812.98606178977	7.50189175765712e-06	\\
5813.96484375	5.51974407391441e-06	\\
5814.94362571023	5.30930002707807e-06	\\
5815.92240767045	5.35370320521001e-06	\\
5816.90118963068	5.27460218256339e-06	\\
5817.87997159091	4.30576097222548e-06	\\
5818.85875355114	6.09981689426314e-06	\\
5819.83753551136	5.96344302998424e-06	\\
5820.81631747159	6.50584099030892e-06	\\
5821.79509943182	4.10609179529559e-06	\\
5822.77388139205	4.78462766709398e-06	\\
5823.75266335227	7.70929128448733e-06	\\
5824.7314453125	5.65969576364161e-06	\\
5825.71022727273	4.39862689122851e-06	\\
5826.68900923295	5.47818890890684e-06	\\
5827.66779119318	7.05341733866305e-06	\\
5828.64657315341	6.50943860345674e-06	\\
5829.62535511364	5.46646803086568e-06	\\
5830.60413707386	3.47234632602314e-06	\\
5831.58291903409	5.99033432855354e-06	\\
5832.56170099432	5.32663806910987e-06	\\
5833.54048295455	3.00525644711927e-06	\\
5834.51926491477	5.46250535241086e-06	\\
5835.498046875	6.4251104434133e-06	\\
5836.47682883523	4.39713732168751e-06	\\
5837.45561079545	5.52293145789177e-06	\\
5838.43439275568	8.64280659051145e-06	\\
5839.41317471591	6.83984374050556e-06	\\
5840.39195667614	7.40271493651596e-06	\\
5841.37073863636	4.11123407382184e-06	\\
5842.34952059659	4.98230532468022e-06	\\
5843.32830255682	7.0294769914667e-06	\\
5844.30708451705	6.28086716499657e-06	\\
5845.28586647727	6.56463094596566e-06	\\
5846.2646484375	6.62039539136864e-06	\\
5847.24343039773	6.73104474191351e-06	\\
5848.22221235795	6.59503616392806e-06	\\
5849.20099431818	6.3005980561593e-06	\\
5850.17977627841	5.35917985793463e-06	\\
5851.15855823864	5.45163521295674e-06	\\
5852.13734019886	7.76495720145176e-06	\\
5853.11612215909	7.13362702528391e-06	\\
5854.09490411932	6.63143335255843e-06	\\
5855.07368607955	5.87328163170446e-06	\\
5856.05246803977	4.49173965516723e-06	\\
5857.03125	6.20659546383579e-06	\\
5858.01003196023	7.14337379865044e-06	\\
5858.98881392045	5.53326818075454e-06	\\
5859.96759588068	7.9844782789552e-06	\\
5860.94637784091	6.0252630357275e-06	\\
5861.92515980114	7.85994851780402e-06	\\
5862.90394176136	6.99539404642205e-06	\\
5863.88272372159	7.39977506812186e-06	\\
5864.86150568182	8.10867160216877e-06	\\
5865.84028764205	8.01038131070208e-06	\\
5866.81906960227	5.57124231582357e-06	\\
5867.7978515625	6.43263348857862e-06	\\
5868.77663352273	7.0893873309972e-06	\\
5869.75541548295	7.15516404400904e-06	\\
5870.73419744318	5.67720974253866e-06	\\
5871.71297940341	6.2112343056049e-06	\\
5872.69176136364	7.34575796902162e-06	\\
5873.67054332386	6.8251357699835e-06	\\
5874.64932528409	5.73157623914241e-06	\\
5875.62810724432	6.47792887268535e-06	\\
5876.60688920455	6.2005567430874e-06	\\
5877.58567116477	4.74146609188101e-06	\\
5878.564453125	5.18392082011942e-06	\\
5879.54323508523	8.1041968058256e-06	\\
5880.52201704545	6.62358186614338e-06	\\
5881.50079900568	5.94993870839976e-06	\\
5882.47958096591	5.38640526321397e-06	\\
5883.45836292614	5.85745303565445e-06	\\
5884.43714488636	5.40806153776107e-06	\\
5885.41592684659	7.25046702211988e-06	\\
5886.39470880682	4.83061695281141e-06	\\
5887.37349076705	7.65281030362646e-06	\\
5888.35227272727	5.96961247140024e-06	\\
5889.3310546875	5.61576789100996e-06	\\
5890.30983664773	7.62738651911333e-06	\\
5891.28861860795	6.88077769619037e-06	\\
5892.26740056818	8.5579375413396e-06	\\
5893.24618252841	5.3743870061269e-06	\\
5894.22496448864	8.3679136474457e-06	\\
5895.20374644886	6.18780849058078e-06	\\
5896.18252840909	7.43848470080983e-06	\\
5897.16131036932	6.94320079762952e-06	\\
5898.14009232955	7.5268200956796e-06	\\
5899.11887428977	8.26349475562775e-06	\\
5900.09765625	9.70438510819767e-06	\\
5901.07643821023	6.91699126345071e-06	\\
5902.05522017045	8.60606357405511e-06	\\
5903.03400213068	6.20420949867122e-06	\\
5904.01278409091	8.13580865821771e-06	\\
5904.99156605114	7.05331521531778e-06	\\
5905.97034801136	7.23238008553926e-06	\\
5906.94912997159	7.5893413480159e-06	\\
5907.92791193182	7.02612298933825e-06	\\
5908.90669389205	7.40082526986058e-06	\\
5909.88547585227	8.33454856895248e-06	\\
5910.8642578125	5.46713695809314e-06	\\
5911.84303977273	5.83745169927617e-06	\\
5912.82182173295	5.73490490071116e-06	\\
5913.80060369318	8.24054783941324e-06	\\
5914.77938565341	8.02737897778049e-06	\\
5915.75816761364	5.5436924203495e-06	\\
5916.73694957386	6.04867479027933e-06	\\
5917.71573153409	6.08546200078414e-06	\\
5918.69451349432	6.80044667356274e-06	\\
5919.67329545455	8.96856984607112e-06	\\
5920.65207741477	7.57073762964095e-06	\\
5921.630859375	6.04730292431723e-06	\\
5922.60964133523	6.89011183415885e-06	\\
5923.58842329545	5.32191538628521e-06	\\
5924.56720525568	6.28065003398384e-06	\\
5925.54598721591	6.073515114487e-06	\\
5926.52476917614	5.70908734727141e-06	\\
5927.50355113636	9.10924566776317e-06	\\
5928.48233309659	5.73939775405363e-06	\\
5929.46111505682	6.9531435574339e-06	\\
5930.43989701705	6.93747483227633e-06	\\
5931.41867897727	6.49020231168335e-06	\\
5932.3974609375	7.86582555321916e-06	\\
5933.37624289773	7.6856272236373e-06	\\
5934.35502485795	6.0270560383344e-06	\\
5935.33380681818	4.05033797841072e-06	\\
5936.31258877841	6.14432512229681e-06	\\
5937.29137073864	5.65602685838488e-06	\\
5938.27015269886	5.50681162975805e-06	\\
5939.24893465909	6.57238793773554e-06	\\
5940.22771661932	6.35008597148829e-06	\\
5941.20649857955	7.07706160569937e-06	\\
5942.18528053977	5.31680355731665e-06	\\
5943.1640625	5.76360126764859e-06	\\
5944.14284446023	6.17860588617163e-06	\\
5945.12162642045	7.2054391823562e-06	\\
5946.10040838068	7.0535102100882e-06	\\
5947.07919034091	6.90892780271575e-06	\\
5948.05797230114	4.71166419712261e-06	\\
5949.03675426136	6.39070194795395e-06	\\
5950.01553622159	6.73139995249984e-06	\\
5950.99431818182	5.02063045809234e-06	\\
5951.97310014205	7.02768517745254e-06	\\
5952.95188210227	8.22029693773625e-06	\\
5953.9306640625	5.22484361901542e-06	\\
5954.90944602273	5.66660244578609e-06	\\
5955.88822798295	7.93663210764396e-06	\\
5956.86700994318	6.32764596021663e-06	\\
5957.84579190341	7.39712833572816e-06	\\
5958.82457386364	7.74046834655501e-06	\\
5959.80335582386	6.33653029352594e-06	\\
5960.78213778409	5.68824997226216e-06	\\
5961.76091974432	7.22704545652029e-06	\\
5962.73970170455	6.64024142916108e-06	\\
5963.71848366477	7.70337736148254e-06	\\
5964.697265625	7.20854919141658e-06	\\
5965.67604758523	4.47081332651697e-06	\\
5966.65482954545	6.6234737859965e-06	\\
5967.63361150568	6.25492217767648e-06	\\
5968.61239346591	5.87741090143695e-06	\\
5969.59117542614	5.96052304471275e-06	\\
5970.56995738636	7.69623178459432e-06	\\
5971.54873934659	6.17288903673027e-06	\\
5972.52752130682	6.47066358568284e-06	\\
5973.50630326705	6.44967699435845e-06	\\
5974.48508522727	9.37558209587707e-06	\\
5975.4638671875	7.9973959547869e-06	\\
5976.44264914773	6.80228666423215e-06	\\
5977.42143110795	6.9300325090459e-06	\\
5978.40021306818	9.24394896199432e-06	\\
5979.37899502841	7.94752497174615e-06	\\
5980.35777698864	9.60898622412178e-06	\\
5981.33655894886	8.13557142465036e-06	\\
5982.31534090909	7.5882037050164e-06	\\
5983.29412286932	7.46368375193769e-06	\\
5984.27290482955	7.4714499863021e-06	\\
5985.25168678977	6.81650003500882e-06	\\
5986.23046875	9.62429996283366e-06	\\
5987.20925071023	8.16728788477781e-06	\\
5988.18803267045	7.84954847815106e-06	\\
5989.16681463068	7.32906073620174e-06	\\
5990.14559659091	7.72515186001033e-06	\\
5991.12437855114	7.01937231601031e-06	\\
5992.10316051136	6.59137508486034e-06	\\
5993.08194247159	6.07038293193139e-06	\\
5994.06072443182	6.64539405177482e-06	\\
5995.03950639205	7.27825726523442e-06	\\
5996.01828835227	7.73759690713587e-06	\\
5996.9970703125	6.2921659786577e-06	\\
5997.97585227273	7.16480318480784e-06	\\
5998.95463423295	8.39024542815233e-06	\\
5999.93341619318	8.78133676863711e-06	\\
6000.91219815341	6.91729112291435e-06	\\
6001.89098011364	7.61024368223472e-06	\\
6002.86976207386	5.84191274584365e-06	\\
6003.84854403409	5.71510241234958e-06	\\
6004.82732599432	7.43362008450094e-06	\\
6005.80610795455	5.9358365258453e-06	\\
6006.78488991477	8.11273049558184e-06	\\
6007.763671875	5.90063782604622e-06	\\
6008.74245383523	6.62642235165664e-06	\\
6009.72123579545	8.26319771048071e-06	\\
6010.70001775568	5.96635488055524e-06	\\
6011.67879971591	9.01613618781101e-06	\\
6012.65758167614	7.06578098825764e-06	\\
6013.63636363636	8.09550546167061e-06	\\
6014.61514559659	9.94864144261751e-06	\\
6015.59392755682	7.66542191162578e-06	\\
6016.57270951705	7.02548993485912e-06	\\
6017.55149147727	7.99066070270861e-06	\\
6018.5302734375	6.57603250632656e-06	\\
6019.50905539773	7.30507547759708e-06	\\
6020.48783735795	7.01629912133545e-06	\\
6021.46661931818	5.59923415520393e-06	\\
6022.44540127841	8.89948296611478e-06	\\
6023.42418323864	8.02436039250032e-06	\\
6024.40296519886	7.21508424742276e-06	\\
6025.38174715909	6.66167876026283e-06	\\
6026.36052911932	6.8700670084246e-06	\\
6027.33931107955	7.28882875958495e-06	\\
6028.31809303977	7.71440518945805e-06	\\
6029.296875	7.84617668110957e-06	\\
6030.27565696023	8.03652301054324e-06	\\
6031.25443892045	8.84079435659564e-06	\\
6032.23322088068	1.11808161523198e-05	\\
6033.21200284091	9.02396872766899e-06	\\
6034.19078480114	8.20768824127824e-06	\\
6035.16956676136	7.61129389340539e-06	\\
6036.14834872159	6.73767866486019e-06	\\
6037.12713068182	7.27738450730722e-06	\\
6038.10591264205	6.9811338829406e-06	\\
6039.08469460227	7.64677736935493e-06	\\
6040.0634765625	7.19548399802713e-06	\\
6041.04225852273	6.03762379582153e-06	\\
6042.02104048295	8.19186582498038e-06	\\
6042.99982244318	8.82395078125808e-06	\\
6043.97860440341	7.57249465041052e-06	\\
6044.95738636364	7.02450303554431e-06	\\
6045.93616832386	7.42661975125231e-06	\\
6046.91495028409	8.24387413745198e-06	\\
6047.89373224432	7.29999619255187e-06	\\
6048.87251420455	6.55018126655101e-06	\\
6049.85129616477	7.18799566698952e-06	\\
6050.830078125	8.54312497806115e-06	\\
6051.80886008523	6.68206957084512e-06	\\
6052.78764204545	8.02072763041593e-06	\\
6053.76642400568	8.83243090827698e-06	\\
6054.74520596591	7.14682567702485e-06	\\
6055.72398792614	5.89081179091988e-06	\\
6056.70276988636	9.2613507268856e-06	\\
6057.68155184659	8.1358384095064e-06	\\
6058.66033380682	8.25872758995492e-06	\\
6059.63911576705	6.01197520969291e-06	\\
6060.61789772727	8.53633636665026e-06	\\
6061.5966796875	5.86441665273443e-06	\\
6062.57546164773	9.94063173326335e-06	\\
6063.55424360795	8.76791957158896e-06	\\
6064.53302556818	8.9154926502413e-06	\\
6065.51180752841	7.35684191092854e-06	\\
6066.49058948864	6.82744269541218e-06	\\
6067.46937144886	1.00528880296318e-05	\\
6068.44815340909	8.01799022383062e-06	\\
6069.42693536932	7.38199659573616e-06	\\
6070.40571732955	7.85904220978372e-06	\\
6071.38449928977	7.78131984716835e-06	\\
6072.36328125	8.3484116397007e-06	\\
6073.34206321023	8.02205546219304e-06	\\
6074.32084517045	1.02634538186948e-05	\\
6075.29962713068	7.26126248265234e-06	\\
6076.27840909091	9.90912862622051e-06	\\
6077.25719105114	1.05001509807898e-05	\\
6078.23597301136	8.27209160204658e-06	\\
6079.21475497159	7.36818164705704e-06	\\
6080.19353693182	9.8994518075644e-06	\\
6081.17231889205	7.03225173441951e-06	\\
6082.15110085227	8.53095214081756e-06	\\
6083.1298828125	8.09169664443315e-06	\\
6084.10866477273	8.51711257463503e-06	\\
6085.08744673295	9.6176847894911e-06	\\
6086.06622869318	9.47184666472641e-06	\\
6087.04501065341	7.30039966135235e-06	\\
6088.02379261364	9.14874361246359e-06	\\
6089.00257457386	8.68573028193673e-06	\\
6089.98135653409	8.05298862705954e-06	\\
6090.96013849432	8.57507824212356e-06	\\
6091.93892045455	7.94299755122165e-06	\\
6092.91770241477	6.29002562599006e-06	\\
6093.896484375	1.11341838998879e-05	\\
6094.87526633523	9.19013145145352e-06	\\
6095.85404829545	8.48179302927712e-06	\\
6096.83283025568	9.23481214905585e-06	\\
6097.81161221591	8.78773133542202e-06	\\
6098.79039417614	9.10595787609455e-06	\\
6099.76917613636	7.18792029534069e-06	\\
6100.74795809659	7.84224097017145e-06	\\
6101.72674005682	8.72931150067244e-06	\\
6102.70552201705	9.13839316902373e-06	\\
6103.68430397727	8.7535351534808e-06	\\
6104.6630859375	7.55092538362535e-06	\\
6105.64186789773	9.12184136505484e-06	\\
6106.62064985795	7.24080559604678e-06	\\
6107.59943181818	8.5402287678086e-06	\\
6108.57821377841	8.39523383463471e-06	\\
6109.55699573864	9.76954164557626e-06	\\
6110.53577769886	8.83464455383785e-06	\\
6111.51455965909	6.91649000321268e-06	\\
6112.49334161932	7.09327489243407e-06	\\
6113.47212357955	1.03365072678794e-05	\\
6114.45090553977	9.11180992084696e-06	\\
6115.4296875	9.71562543551046e-06	\\
6116.40846946023	9.2729926526876e-06	\\
6117.38725142045	8.60752868705689e-06	\\
6118.36603338068	8.51255891668588e-06	\\
6119.34481534091	8.34348790460137e-06	\\
6120.32359730114	8.46688870683861e-06	\\
6121.30237926136	7.61520276520857e-06	\\
6122.28116122159	8.33713176983699e-06	\\
6123.25994318182	1.1231206389025e-05	\\
6124.23872514205	9.78742794125969e-06	\\
6125.21750710227	7.93005300011885e-06	\\
6126.1962890625	8.29501771108177e-06	\\
6127.17507102273	9.25158265981474e-06	\\
6128.15385298295	1.08637126510099e-05	\\
6129.13263494318	8.83829180711423e-06	\\
6130.11141690341	7.50212288881511e-06	\\
6131.09019886364	7.91221582278533e-06	\\
6132.06898082386	7.53012975900916e-06	\\
6133.04776278409	8.89592648059635e-06	\\
6134.02654474432	8.60576919901801e-06	\\
6135.00532670455	6.27726838957153e-06	\\
6135.98410866477	9.40726486682945e-06	\\
6136.962890625	1.00048462337851e-05	\\
6137.94167258523	9.30960854097489e-06	\\
6138.92045454545	9.77805702948509e-06	\\
6139.89923650568	8.53779672614675e-06	\\
6140.87801846591	9.5457772717763e-06	\\
6141.85680042614	9.91672350848772e-06	\\
6142.83558238636	8.35238403248184e-06	\\
6143.81436434659	9.73392598622237e-06	\\
6144.79314630682	8.61639589953168e-06	\\
6145.77192826705	9.57277520887882e-06	\\
6146.75071022727	7.51722030388523e-06	\\
6147.7294921875	1.10587129034939e-05	\\
6148.70827414773	8.98814404795571e-06	\\
6149.68705610795	8.83072722793556e-06	\\
6150.66583806818	9.74013650292401e-06	\\
6151.64462002841	1.03241214222022e-05	\\
6152.62340198864	9.09166967480471e-06	\\
6153.60218394886	8.57967101315341e-06	\\
6154.58096590909	8.30631728406354e-06	\\
6155.55974786932	7.48722661667455e-06	\\
6156.53852982955	1.01286294701591e-05	\\
6157.51731178977	1.0707484302498e-05	\\
6158.49609375	8.93698669753542e-06	\\
6159.47487571023	8.21456800611828e-06	\\
6160.45365767045	9.93081512898914e-06	\\
6161.43243963068	8.86721033733329e-06	\\
6162.41122159091	9.595258984468e-06	\\
6163.39000355114	9.3611980164646e-06	\\
6164.36878551136	8.32824913911014e-06	\\
6165.34756747159	9.66500130772939e-06	\\
6166.32634943182	7.75058864481988e-06	\\
6167.30513139205	1.11186033855131e-05	\\
6168.28391335227	9.09932778475869e-06	\\
6169.2626953125	7.2511978972976e-06	\\
6170.24147727273	9.7269127356407e-06	\\
6171.22025923295	9.76280351295801e-06	\\
6172.19904119318	9.57227680815324e-06	\\
6173.17782315341	8.88857039950822e-06	\\
6174.15660511364	8.73406865347042e-06	\\
6175.13538707386	9.22202280919412e-06	\\
6176.11416903409	8.91740780882304e-06	\\
6177.09295099432	1.01972457679535e-05	\\
6178.07173295455	1.02087711065576e-05	\\
6179.05051491477	1.0150569673807e-05	\\
6180.029296875	1.11838654010654e-05	\\
6181.00807883523	9.80575429025174e-06	\\
6181.98686079545	8.62498224553618e-06	\\
6182.96564275568	1.11526809685661e-05	\\
6183.94442471591	1.02013633511637e-05	\\
6184.92320667614	8.70266900327966e-06	\\
6185.90198863636	9.51574258753799e-06	\\
6186.88077059659	1.03354481542657e-05	\\
6187.85955255682	8.72853584128419e-06	\\
6188.83833451705	8.77563974921167e-06	\\
6189.81711647727	1.04852165433096e-05	\\
6190.7958984375	9.04456023274418e-06	\\
6191.77468039773	8.943667368999e-06	\\
6192.75346235795	1.01722664015718e-05	\\
6193.73224431818	1.08749658284328e-05	\\
6194.71102627841	1.02692192797677e-05	\\
6195.68980823864	9.31598745628798e-06	\\
6196.66859019886	1.08978493950525e-05	\\
6197.64737215909	9.06548434048706e-06	\\
6198.62615411932	9.31974713517938e-06	\\
6199.60493607955	8.51683985622671e-06	\\
6200.58371803977	1.09821845096599e-05	\\
6201.5625	9.55432822294066e-06	\\
6202.54128196023	8.8312408658881e-06	\\
6203.52006392045	9.66978861034682e-06	\\
6204.49884588068	9.94512298238006e-06	\\
6205.47762784091	9.53002219262659e-06	\\
6206.45640980114	8.11609988473631e-06	\\
6207.43519176136	8.42953256640058e-06	\\
6208.41397372159	8.51381161493954e-06	\\
6209.39275568182	9.08912304230002e-06	\\
6210.37153764205	8.38831758670776e-06	\\
6211.35031960227	9.44325180838626e-06	\\
6212.3291015625	7.74096715572328e-06	\\
6213.30788352273	8.0876573254261e-06	\\
6214.28666548295	9.6681982539178e-06	\\
6215.26544744318	7.73601564976628e-06	\\
6216.24422940341	6.75801893481847e-06	\\
6217.22301136364	8.50089848398617e-06	\\
6218.20179332386	8.17248378937337e-06	\\
6219.18057528409	8.36672330427057e-06	\\
6220.15935724432	7.02817354063346e-06	\\
6221.13813920455	8.12475194016142e-06	\\
6222.11692116477	9.47056085869885e-06	\\
6223.095703125	1.01166879531668e-05	\\
6224.07448508523	6.72939721906557e-06	\\
6225.05326704545	8.40279473458372e-06	\\
6226.03204900568	8.94424930875237e-06	\\
6227.01083096591	6.49290417342215e-06	\\
6227.98961292614	9.68082700659218e-06	\\
6228.96839488636	8.606365827652e-06	\\
6229.94717684659	8.84449107097012e-06	\\
6230.92595880682	9.85477790864282e-06	\\
6231.90474076705	9.74804228080189e-06	\\
6232.88352272727	8.01258439985777e-06	\\
6233.8623046875	8.18961727633379e-06	\\
6234.84108664773	9.30582522988441e-06	\\
6235.81986860795	9.96886441491547e-06	\\
6236.79865056818	7.93330423588441e-06	\\
6237.77743252841	8.05283146619266e-06	\\
6238.75621448864	8.69295703293211e-06	\\
6239.73499644886	9.63165946268592e-06	\\
6240.71377840909	7.19738211789916e-06	\\
6241.69256036932	7.93594127656541e-06	\\
6242.67134232955	9.59461399568428e-06	\\
6243.65012428977	9.32942538935103e-06	\\
6244.62890625	8.22344134553775e-06	\\
6245.60768821023	9.12876952392022e-06	\\
6246.58647017045	9.68907027416338e-06	\\
6247.56525213068	8.86125825365139e-06	\\
6248.54403409091	9.07405982517984e-06	\\
6249.52281605114	8.89158958724129e-06	\\
6250.50159801136	1.2675343054917e-05	\\
6251.48037997159	9.70292700299525e-06	\\
6252.45916193182	9.4115654588967e-06	\\
6253.43794389205	1.09512803095052e-05	\\
6254.41672585227	9.85920595192773e-06	\\
6255.3955078125	8.75962115842567e-06	\\
6256.37428977273	8.77003834024329e-06	\\
6257.35307173295	9.72531184012167e-06	\\
6258.33185369318	1.03590367352936e-05	\\
6259.31063565341	9.75014446535197e-06	\\
6260.28941761364	8.60990367484358e-06	\\
6261.26819957386	1.08059238349355e-05	\\
6262.24698153409	9.25759508685716e-06	\\
6263.22576349432	9.26613239579253e-06	\\
6264.20454545455	9.70968298234748e-06	\\
6265.18332741477	9.83564529581551e-06	\\
6266.162109375	8.93505657387418e-06	\\
6267.14089133523	1.14354305162818e-05	\\
6268.11967329545	8.97455115413802e-06	\\
6269.09845525568	8.11736048009991e-06	\\
6270.07723721591	8.00189118843089e-06	\\
6271.05601917614	1.11490098981613e-05	\\
6272.03480113636	7.93499179672075e-06	\\
6273.01358309659	9.80560848304246e-06	\\
6273.99236505682	8.9698811392069e-06	\\
6274.97114701705	9.12639993985013e-06	\\
6275.94992897727	9.71719991533497e-06	\\
6276.9287109375	8.4738873532606e-06	\\
6277.90749289773	1.0672116751715e-05	\\
6278.88627485795	1.05226483343694e-05	\\
6279.86505681818	9.8596378567294e-06	\\
6280.84383877841	1.12057740706608e-05	\\
6281.82262073864	1.00857529975078e-05	\\
6282.80140269886	9.20575902330128e-06	\\
6283.78018465909	8.97120854192342e-06	\\
6284.75896661932	8.62927718124355e-06	\\
6285.73774857955	9.63222318715081e-06	\\
6286.71653053977	8.94752981332253e-06	\\
6287.6953125	9.85462493194414e-06	\\
6288.67409446023	7.81192077400789e-06	\\
6289.65287642045	8.04483785534612e-06	\\
6290.63165838068	1.05909462157904e-05	\\
6291.61044034091	1.04552793423252e-05	\\
6292.58922230114	1.01089999249936e-05	\\
6293.56800426136	9.57393258784269e-06	\\
6294.54678622159	1.03996974019626e-05	\\
6295.52556818182	1.00417096847338e-05	\\
6296.50435014205	1.08698401013016e-05	\\
6297.48313210227	9.82329062946656e-06	\\
6298.4619140625	9.94070327421994e-06	\\
6299.44069602273	1.07496949823172e-05	\\
6300.41947798295	1.19811097207315e-05	\\
6301.39825994318	1.0517351175524e-05	\\
6302.37704190341	1.04329978283385e-05	\\
6303.35582386364	9.99575346254727e-06	\\
6304.33460582386	9.78469332231141e-06	\\
6305.31338778409	9.42779465584762e-06	\\
6306.29216974432	8.4231377215323e-06	\\
6307.27095170455	1.11999280037123e-05	\\
6308.24973366477	8.88608363458445e-06	\\
6309.228515625	1.09166938236719e-05	\\
6310.20729758523	1.11044134634413e-05	\\
6311.18607954545	1.09705139166511e-05	\\
6312.16486150568	1.15368174472672e-05	\\
6313.14364346591	9.85875448362906e-06	\\
6314.12242542614	1.16680713924164e-05	\\
6315.10120738636	9.23270889774404e-06	\\
6316.07998934659	1.01996537122665e-05	\\
6317.05877130682	1.12865948637781e-05	\\
6318.03755326705	1.04112237950359e-05	\\
6319.01633522727	8.94975841187904e-06	\\
6319.9951171875	9.22518127622863e-06	\\
6320.97389914773	8.76769153868357e-06	\\
6321.95268110795	9.72944023426055e-06	\\
6322.93146306818	1.06873349975052e-05	\\
6323.91024502841	1.0661523639163e-05	\\
6324.88902698864	8.96919245474136e-06	\\
6325.86780894886	1.06152092137177e-05	\\
6326.84659090909	9.0539739937883e-06	\\
6327.82537286932	1.08455164601242e-05	\\
6328.80415482955	9.03761949807712e-06	\\
6329.78293678977	9.59020511206869e-06	\\
6330.76171875	1.06469982436533e-05	\\
6331.74050071023	1.10322337207436e-05	\\
6332.71928267045	1.06308091872447e-05	\\
6333.69806463068	7.89983563896885e-06	\\
6334.67684659091	1.01130935610541e-05	\\
6335.65562855114	9.63714178323274e-06	\\
6336.63441051136	1.24540089612248e-05	\\
6337.61319247159	1.1638491470695e-05	\\
6338.59197443182	9.89592737703139e-06	\\
6339.57075639205	1.09303473966715e-05	\\
6340.54953835227	1.09336612031665e-05	\\
6341.5283203125	9.31192395821949e-06	\\
6342.50710227273	1.06210110638758e-05	\\
6343.48588423295	1.21233717461419e-05	\\
6344.46466619318	1.10783743888029e-05	\\
6345.44344815341	1.17748767067942e-05	\\
6346.42223011364	1.03345251667387e-05	\\
6347.40101207386	1.03630009155192e-05	\\
6348.37979403409	8.90195122540262e-06	\\
6349.35857599432	9.76003299960721e-06	\\
6350.33735795455	9.90259510422571e-06	\\
6351.31613991477	7.72014978821977e-06	\\
6352.294921875	1.10889686105496e-05	\\
6353.27370383523	1.04803029031896e-05	\\
6354.25248579545	1.10229281013754e-05	\\
6355.23126775568	1.20806812889193e-05	\\
6356.21004971591	1.00256795540012e-05	\\
6357.18883167614	9.92764441745972e-06	\\
6358.16761363636	1.14327078407132e-05	\\
6359.14639559659	9.31206456643084e-06	\\
6360.12517755682	1.0944934330783e-05	\\
6361.10395951705	9.21627415629105e-06	\\
6362.08274147727	1.10808497522679e-05	\\
6363.0615234375	1.11368420221821e-05	\\
6364.04030539773	7.35600634833834e-06	\\
6365.01908735795	1.11027539050293e-05	\\
6365.99786931818	1.03228737119079e-05	\\
6366.97665127841	9.571815364797e-06	\\
6367.95543323864	1.0400762244411e-05	\\
6368.93421519886	1.07868396988604e-05	\\
6369.91299715909	1.0729825554414e-05	\\
6370.89177911932	1.04723558423604e-05	\\
6371.87056107955	1.23022325612574e-05	\\
6372.84934303977	9.39509403756763e-06	\\
6373.828125	1.00479274217585e-05	\\
6374.80690696023	1.01806257556481e-05	\\
6375.78568892045	1.1154514530894e-05	\\
6376.76447088068	1.18691801467672e-05	\\
6377.74325284091	9.88566341182709e-06	\\
6378.72203480114	9.21284674631958e-06	\\
6379.70081676136	1.13803925980183e-05	\\
6380.67959872159	1.04869825683514e-05	\\
6381.65838068182	1.15888929264046e-05	\\
6382.63716264205	1.18637363183611e-05	\\
6383.61594460227	1.16989739256013e-05	\\
6384.5947265625	1.09418921985359e-05	\\
6385.57350852273	9.53443679044535e-06	\\
6386.55229048295	9.91069272636556e-06	\\
6387.53107244318	9.17551999660984e-06	\\
6388.50985440341	9.96533253836317e-06	\\
6389.48863636364	8.40285890150948e-06	\\
6390.46741832386	9.72244927334872e-06	\\
6391.44620028409	1.27462712774085e-05	\\
6392.42498224432	1.09801730395312e-05	\\
6393.40376420455	1.01935121259305e-05	\\
6394.38254616477	1.03487625451309e-05	\\
6395.361328125	1.05192146064512e-05	\\
6396.34011008523	1.02124437363741e-05	\\
6397.31889204545	1.20179967843523e-05	\\
6398.29767400568	9.67683573947223e-06	\\
6399.27645596591	1.01047109514741e-05	\\
6400.25523792614	1.11726759322529e-05	\\
6401.23401988636	1.02959016641804e-05	\\
6402.21280184659	9.25970014384601e-06	\\
6403.19158380682	9.80762252363353e-06	\\
6404.17036576705	9.48202359727489e-06	\\
6405.14914772727	1.25073574535437e-05	\\
6406.1279296875	8.90161742598433e-06	\\
6407.10671164773	1.08040279955292e-05	\\
6408.08549360795	1.16765336638379e-05	\\
6409.06427556818	1.16875753316358e-05	\\
6410.04305752841	1.22126335587435e-05	\\
6411.02183948864	1.2614314159232e-05	\\
6412.00062144886	1.19749553169733e-05	\\
6412.97940340909	1.04050879619263e-05	\\
6413.95818536932	1.01103832167017e-05	\\
6414.93696732955	1.21104968893371e-05	\\
6415.91574928977	1.27726863896253e-05	\\
6416.89453125	1.00887121725971e-05	\\
6417.87331321023	1.19096384069506e-05	\\
6418.85209517045	1.22559211324314e-05	\\
6419.83087713068	1.1941534095934e-05	\\
6420.80965909091	1.21031707620299e-05	\\
6421.78844105114	1.14583334085348e-05	\\
6422.76722301136	1.04530155885658e-05	\\
6423.74600497159	1.08909874595043e-05	\\
6424.72478693182	1.25986141353416e-05	\\
6425.70356889205	1.27758884419201e-05	\\
6426.68235085227	1.18314987709472e-05	\\
6427.6611328125	1.15053144883815e-05	\\
6428.63991477273	1.05211626833765e-05	\\
6429.61869673295	1.19449577497703e-05	\\
6430.59747869318	1.01953697236045e-05	\\
6431.57626065341	1.15186506373441e-05	\\
6432.55504261364	1.2123684222804e-05	\\
6433.53382457386	9.79273672859635e-06	\\
6434.51260653409	1.11788447573032e-05	\\
6435.49138849432	1.12098129659966e-05	\\
6436.47017045455	1.23852183168548e-05	\\
6437.44895241477	9.02062069698073e-06	\\
6438.427734375	1.08493256085155e-05	\\
6439.40651633523	1.12261856923342e-05	\\
6440.38529829545	1.14185555899461e-05	\\
6441.36408025568	1.03606584779197e-05	\\
6442.34286221591	1.04101695215663e-05	\\
6443.32164417614	1.00686000437155e-05	\\
6444.30042613636	1.08712521585489e-05	\\
6445.27920809659	1.04980092875332e-05	\\
6446.25799005682	9.59424526023042e-06	\\
6447.23677201705	1.10722961558416e-05	\\
6448.21555397727	1.09851840058765e-05	\\
6449.1943359375	1.09776557661116e-05	\\
6450.17311789773	1.12861762048676e-05	\\
6451.15189985795	1.01256820259379e-05	\\
6452.13068181818	8.66432090683722e-06	\\
6453.10946377841	1.18429843258854e-05	\\
6454.08824573864	1.046576218379e-05	\\
6455.06702769886	1.07398887919549e-05	\\
6456.04580965909	1.13312599025356e-05	\\
6457.02459161932	9.14709037466739e-06	\\
6458.00337357955	1.12632712437586e-05	\\
6458.98215553977	1.08134598669562e-05	\\
6459.9609375	1.15868621348894e-05	\\
6460.93971946023	1.00179748185771e-05	\\
6461.91850142045	1.22686588448391e-05	\\
6462.89728338068	1.00761165015946e-05	\\
6463.87606534091	1.01771258535724e-05	\\
6464.85484730114	1.07212185369158e-05	\\
6465.83362926136	1.04901039391886e-05	\\
6466.81241122159	1.15989223735319e-05	\\
6467.79119318182	1.04966479786814e-05	\\
6468.76997514205	1.20543151116436e-05	\\
6469.74875710227	9.32035903295983e-06	\\
6470.7275390625	9.90320221422706e-06	\\
6471.70632102273	9.24163473579731e-06	\\
6472.68510298295	1.13068465797059e-05	\\
6473.66388494318	8.90538094774641e-06	\\
6474.64266690341	9.70730017300926e-06	\\
6475.62144886364	1.04818480864405e-05	\\
6476.60023082386	9.42113760022019e-06	\\
6477.57901278409	1.13511224481748e-05	\\
6478.55779474432	1.01330036690682e-05	\\
6479.53657670455	9.45186700868745e-06	\\
6480.51535866477	8.9438861799159e-06	\\
6481.494140625	1.01753969536674e-05	\\
6482.47292258523	1.00960771586755e-05	\\
6483.45170454545	1.10263855977816e-05	\\
6484.43048650568	1.19329798504392e-05	\\
6485.40926846591	1.03545270004786e-05	\\
6486.38805042614	1.02391207002672e-05	\\
6487.36683238636	1.19576146491379e-05	\\
6488.34561434659	1.05394210661115e-05	\\
6489.32439630682	1.22384437379514e-05	\\
6490.30317826705	1.24918838058156e-05	\\
6491.28196022727	1.17822227126401e-05	\\
6492.2607421875	8.24436870057312e-06	\\
6493.23952414773	9.8535398994028e-06	\\
6494.21830610795	9.74301875415291e-06	\\
6495.19708806818	9.47648310337573e-06	\\
6496.17587002841	1.10363965495903e-05	\\
6497.15465198864	9.34925874582167e-06	\\
6498.13343394886	1.07807040620109e-05	\\
6499.11221590909	7.4911607910725e-06	\\
6500.09099786932	1.20796016866131e-05	\\
6501.06977982955	1.08870604717252e-05	\\
6502.04856178977	1.17226644439546e-05	\\
6503.02734375	9.65822916349868e-06	\\
6504.00612571023	1.1316834332946e-05	\\
6504.98490767045	9.57943848666833e-06	\\
6505.96368963068	1.17775319373826e-05	\\
6506.94247159091	1.13109979610064e-05	\\
6507.92125355114	9.83958342355111e-06	\\
6508.90003551136	9.66809101584527e-06	\\
6509.87881747159	9.14118784187417e-06	\\
6510.85759943182	1.17121020222684e-05	\\
6511.83638139205	9.99619598446862e-06	\\
6512.81516335227	9.01387944382202e-06	\\
6513.7939453125	1.0334509285984e-05	\\
6514.77272727273	1.07111086933885e-05	\\
6515.75150923295	9.44798274054369e-06	\\
6516.73029119318	1.07818569268166e-05	\\
6517.70907315341	9.86795325434928e-06	\\
6518.68785511364	9.90340280475377e-06	\\
6519.66663707386	1.12375262077202e-05	\\
6520.64541903409	8.93439914246969e-06	\\
6521.62420099432	9.01004002499872e-06	\\
6522.60298295455	1.12119407352382e-05	\\
6523.58176491477	1.18104899996697e-05	\\
6524.560546875	1.00581571585313e-05	\\
6525.53932883523	9.73162791284576e-06	\\
6526.51811079545	9.70549608039152e-06	\\
6527.49689275568	9.66270184151972e-06	\\
6528.47567471591	9.44885201567444e-06	\\
6529.45445667614	1.01228862631768e-05	\\
6530.43323863636	1.14650041631869e-05	\\
6531.41202059659	1.12904151692859e-05	\\
6532.39080255682	8.93633499775863e-06	\\
6533.36958451705	7.95670537267756e-06	\\
6534.34836647727	1.133558540592e-05	\\
6535.3271484375	7.89064688613315e-06	\\
6536.30593039773	1.01791721627068e-05	\\
6537.28471235795	9.41891328164913e-06	\\
6538.26349431818	7.38001420918392e-06	\\
6539.24227627841	7.81767072353379e-06	\\
6540.22105823864	1.07876535371499e-05	\\
6541.19984019886	1.04750754081504e-05	\\
6542.17862215909	9.86521250431792e-06	\\
6543.15740411932	1.13241700984208e-05	\\
6544.13618607955	9.06494465459859e-06	\\
6545.11496803977	1.01817049361448e-05	\\
6546.09375	1.10430507667687e-05	\\
6547.07253196023	1.10926602228029e-05	\\
6548.05131392045	1.12496930490478e-05	\\
6549.03009588068	9.8310190037104e-06	\\
6550.00887784091	7.58577170346584e-06	\\
6550.98765980114	9.23842694083774e-06	\\
6551.96644176136	1.06751280668334e-05	\\
6552.94522372159	8.72237743272725e-06	\\
6553.92400568182	1.04074604493555e-05	\\
6554.90278764205	7.44802143329878e-06	\\
6555.88156960227	1.05763315532051e-05	\\
6556.8603515625	8.68714517666511e-06	\\
6557.83913352273	9.01987143565354e-06	\\
6558.81791548295	9.66337687359546e-06	\\
6559.79669744318	9.7365503823691e-06	\\
6560.77547940341	8.55259793768197e-06	\\
6561.75426136364	9.83438193673061e-06	\\
6562.73304332386	8.82972655355662e-06	\\
6563.71182528409	9.61630340399219e-06	\\
6564.69060724432	1.0488853747695e-05	\\
6565.66938920455	8.93997424233124e-06	\\
6566.64817116477	1.05991582712665e-05	\\
6567.626953125	8.634013258845e-06	\\
6568.60573508523	1.01770223083324e-05	\\
6569.58451704545	9.26665566629073e-06	\\
6570.56329900568	1.02290749121807e-05	\\
6571.54208096591	1.03071535964401e-05	\\
6572.52086292614	9.51603012614309e-06	\\
6573.49964488636	1.106727212745e-05	\\
6574.47842684659	7.81226306356776e-06	\\
6575.45720880682	1.04919234549829e-05	\\
6576.43599076705	8.76890355584427e-06	\\
6577.41477272727	8.02567015943106e-06	\\
6578.3935546875	8.84417112084272e-06	\\
6579.37233664773	1.08341030692809e-05	\\
6580.35111860795	7.92428007595391e-06	\\
6581.32990056818	8.70164303787301e-06	\\
6582.30868252841	9.94787152525188e-06	\\
6583.28746448864	1.0765571777113e-05	\\
6584.26624644886	9.18952453066364e-06	\\
6585.24502840909	9.65575594762394e-06	\\
6586.22381036932	1.18566657621575e-05	\\
6587.20259232955	9.56986929775039e-06	\\
6588.18137428977	8.99229795891344e-06	\\
6589.16015625	1.10941063283308e-05	\\
6590.13893821023	1.05639611177426e-05	\\
6591.11772017045	8.73267360294533e-06	\\
6592.09650213068	7.38493060661225e-06	\\
6593.07528409091	8.43793565854272e-06	\\
6594.05406605114	1.02178774682879e-05	\\
6595.03284801136	1.07372915474606e-05	\\
6596.01162997159	9.28340932447007e-06	\\
6596.99041193182	1.0576206122809e-05	\\
6597.96919389205	8.75147613599474e-06	\\
6598.94797585227	9.15027650715671e-06	\\
6599.9267578125	1.00760309810401e-05	\\
6600.90553977273	8.41378362216357e-06	\\
6601.88432173295	8.75672183853774e-06	\\
6602.86310369318	9.38712387552058e-06	\\
6603.84188565341	1.08632751314513e-05	\\
6604.82066761364	1.00237771192905e-05	\\
6605.79944957386	8.7366949956304e-06	\\
6606.77823153409	8.93482846373729e-06	\\
6607.75701349432	1.11341085331958e-05	\\
6608.73579545455	9.93688615949263e-06	\\
6609.71457741477	9.48606154589629e-06	\\
6610.693359375	6.90492414648985e-06	\\
6611.67214133523	1.08941114755877e-05	\\
6612.65092329545	1.15220685588074e-05	\\
6613.62970525568	9.54904041576369e-06	\\
6614.60848721591	1.06546371025948e-05	\\
6615.58726917614	9.27248998433246e-06	\\
6616.56605113636	1.03941924130596e-05	\\
6617.54483309659	8.99891315948358e-06	\\
6618.52361505682	1.1207639256795e-05	\\
6619.50239701705	9.08316740351739e-06	\\
6620.48117897727	9.54606543879888e-06	\\
6621.4599609375	1.11375438490071e-05	\\
6622.43874289773	9.17289939484285e-06	\\
6623.41752485795	1.25044331314231e-05	\\
6624.39630681818	9.52528074586188e-06	\\
6625.37508877841	9.34333135136956e-06	\\
6626.35387073864	1.20772847097035e-05	\\
6627.33265269886	1.05969357562381e-05	\\
6628.31143465909	1.02091313348063e-05	\\
6629.29021661932	9.92925741539242e-06	\\
6630.26899857955	1.03570336641043e-05	\\
6631.24778053977	1.08893570434591e-05	\\
6632.2265625	9.05844241060396e-06	\\
6633.20534446023	1.1836212396245e-05	\\
6634.18412642045	9.60429299290296e-06	\\
6635.16290838068	9.52771720788153e-06	\\
6636.14169034091	9.75329740721772e-06	\\
6637.12047230114	8.1581427239316e-06	\\
6638.09925426136	9.28744249367836e-06	\\
6639.07803622159	1.00402627546131e-05	\\
6640.05681818182	1.27306406447397e-05	\\
6641.03560014205	9.98072396816646e-06	\\
6642.01438210227	1.06752092349147e-05	\\
6642.9931640625	1.0058390450274e-05	\\
6643.97194602273	1.16515868954087e-05	\\
6644.95072798295	8.64485335244325e-06	\\
6645.92950994318	8.60062464244184e-06	\\
6646.90829190341	1.07740816878393e-05	\\
6647.88707386364	1.04964072112591e-05	\\
6648.86585582386	1.20984745553112e-05	\\
6649.84463778409	1.04127244251406e-05	\\
6650.82341974432	1.08807119816402e-05	\\
6651.80220170455	9.67238951448365e-06	\\
6652.78098366477	9.72409421540679e-06	\\
6653.759765625	9.60526077631119e-06	\\
6654.73854758523	8.68541953752701e-06	\\
6655.71732954545	1.06728234273659e-05	\\
6656.69611150568	9.19827459074944e-06	\\
6657.67489346591	8.5727286472194e-06	\\
6658.65367542614	1.00252074172898e-05	\\
6659.63245738636	9.83175456975615e-06	\\
6660.61123934659	9.65606561081815e-06	\\
6661.59002130682	8.66935572123222e-06	\\
6662.56880326705	1.14529424254817e-05	\\
6663.54758522727	1.18994167259939e-05	\\
6664.5263671875	9.7976230554374e-06	\\
6665.50514914773	1.02196326754741e-05	\\
6666.48393110795	1.06059240963756e-05	\\
6667.46271306818	9.08159882221243e-06	\\
6668.44149502841	9.64372685253848e-06	\\
6669.42027698864	1.08049696147971e-05	\\
6670.39905894886	8.77744372239721e-06	\\
6671.37784090909	8.87387499643085e-06	\\
6672.35662286932	1.08298923457039e-05	\\
6673.33540482955	1.05765875119328e-05	\\
6674.31418678977	9.21056148804963e-06	\\
6675.29296875	8.86724721817021e-06	\\
6676.27175071023	9.47011436766862e-06	\\
6677.25053267045	9.64834674865656e-06	\\
6678.22931463068	7.86576795797916e-06	\\
6679.20809659091	9.84501202371369e-06	\\
6680.18687855114	1.02926774394069e-05	\\
6681.16566051136	1.09209438822737e-05	\\
6682.14444247159	9.42082571464723e-06	\\
6683.12322443182	1.06617665700213e-05	\\
6684.10200639205	1.08266180691017e-05	\\
6685.08078835227	9.92320701925564e-06	\\
6686.0595703125	1.01248872429164e-05	\\
6687.03835227273	8.80043484878581e-06	\\
6688.01713423295	9.68724490832347e-06	\\
6688.99591619318	7.71556677881636e-06	\\
6689.97469815341	8.61455054644416e-06	\\
6690.95348011364	1.00199482511863e-05	\\
6691.93226207386	9.74108603975881e-06	\\
6692.91104403409	8.05217950242716e-06	\\
6693.88982599432	9.67401570525247e-06	\\
6694.86860795455	9.00617632575546e-06	\\
6695.84738991477	9.33976100473258e-06	\\
6696.826171875	1.00039730136193e-05	\\
6697.80495383523	9.27014967727539e-06	\\
6698.78373579545	8.46474993398969e-06	\\
6699.76251775568	1.21163091993361e-05	\\
6700.74129971591	8.60624661006163e-06	\\
6701.72008167614	8.96440655895696e-06	\\
6702.69886363636	1.06393850627745e-05	\\
6703.67764559659	8.84408653328434e-06	\\
6704.65642755682	9.73534244344964e-06	\\
6705.63520951705	9.19263456176653e-06	\\
6706.61399147727	7.23829287117992e-06	\\
6707.5927734375	1.07469294339182e-05	\\
6708.57155539773	7.7583020773419e-06	\\
6709.55033735795	6.62260684501762e-06	\\
6710.52911931818	9.31288469399555e-06	\\
6711.50790127841	8.84536268512212e-06	\\
6712.48668323864	1.022776192331e-05	\\
6713.46546519886	9.67279968320117e-06	\\
6714.44424715909	9.95128156277939e-06	\\
6715.42302911932	1.04085196345868e-05	\\
6716.40181107955	1.02892895226897e-05	\\
6717.38059303977	7.9703068367325e-06	\\
6718.359375	8.74424330293729e-06	\\
6719.33815696023	8.85582691090681e-06	\\
6720.31693892045	1.0168260145603e-05	\\
6721.29572088068	8.55308233320014e-06	\\
6722.27450284091	8.19985594381252e-06	\\
6723.25328480114	1.10576057639201e-05	\\
6724.23206676136	1.01174516949605e-05	\\
6725.21084872159	1.05451528258964e-05	\\
6726.18963068182	8.27478340964148e-06	\\
6727.16841264205	8.63054137796399e-06	\\
6728.14719460227	8.19441863084917e-06	\\
6729.1259765625	1.05244656327399e-05	\\
6730.10475852273	9.14309946364242e-06	\\
6731.08354048295	9.844390143243e-06	\\
6732.06232244318	8.14711263237529e-06	\\
6733.04110440341	1.00352127856001e-05	\\
6734.01988636364	9.99354687659644e-06	\\
6734.99866832386	1.02531065723532e-05	\\
6735.97745028409	1.05571260424312e-05	\\
6736.95623224432	8.40038004738296e-06	\\
6737.93501420455	9.90563138543158e-06	\\
6738.91379616477	1.15406850042343e-05	\\
6739.892578125	9.21129665805899e-06	\\
6740.87136008523	9.14435830841633e-06	\\
6741.85014204545	1.0192661872495e-05	\\
6742.82892400568	1.05105865708806e-05	\\
6743.80770596591	8.96705891369385e-06	\\
6744.78648792614	1.05554194082495e-05	\\
6745.76526988636	9.26685459998644e-06	\\
6746.74405184659	1.11154040242839e-05	\\
6747.72283380682	9.53451726382814e-06	\\
6748.70161576705	9.79331479679848e-06	\\
6749.68039772727	9.71146783098741e-06	\\
6750.6591796875	1.07829332089805e-05	\\
6751.63796164773	1.07538302806613e-05	\\
6752.61674360795	9.941311157474e-06	\\
6753.59552556818	1.0347976960432e-05	\\
6754.57430752841	9.08714144086891e-06	\\
6755.55308948864	8.6731573914278e-06	\\
6756.53187144886	9.58345410279519e-06	\\
6757.51065340909	1.18942493654369e-05	\\
6758.48943536932	1.08649321435988e-05	\\
6759.46821732955	1.0343254598511e-05	\\
6760.44699928977	9.38969751345229e-06	\\
6761.42578125	1.02014475584547e-05	\\
6762.40456321023	8.08273913100568e-06	\\
6763.38334517045	9.34507806214105e-06	\\
6764.36212713068	1.15301300690708e-05	\\
6765.34090909091	9.50063840155734e-06	\\
6766.31969105114	6.9119674760777e-06	\\
6767.29847301136	1.02130802398973e-05	\\
6768.27725497159	9.44449578580569e-06	\\
6769.25603693182	9.51969350357448e-06	\\
6770.23481889205	1.09437689708816e-05	\\
6771.21360085227	8.25833830520883e-06	\\
6772.1923828125	9.36726744366475e-06	\\
6773.17116477273	8.52905742643099e-06	\\
6774.14994673295	9.3891147961578e-06	\\
6775.12872869318	1.0971859247878e-05	\\
6776.10751065341	9.00339156253311e-06	\\
6777.08629261364	9.08246075228492e-06	\\
6778.06507457386	1.08803189715766e-05	\\
6779.04385653409	1.02738059827678e-05	\\
6780.02263849432	8.9962281413815e-06	\\
6781.00142045455	9.10558867240921e-06	\\
6781.98020241477	9.97748018869789e-06	\\
6782.958984375	1.01186796112958e-05	\\
6783.93776633523	1.18209902087948e-05	\\
6784.91654829545	1.02333762171423e-05	\\
6785.89533025568	8.41019672119945e-06	\\
6786.87411221591	8.38339914611323e-06	\\
6787.85289417614	9.49965129669802e-06	\\
6788.83167613636	9.77933796695317e-06	\\
6789.81045809659	9.27959618279247e-06	\\
6790.78924005682	9.0924790574116e-06	\\
6791.76802201705	9.07231432033201e-06	\\
6792.74680397727	8.0380034535093e-06	\\
6793.7255859375	8.82297063706853e-06	\\
6794.70436789773	1.11319950755916e-05	\\
6795.68314985795	1.10260304820986e-05	\\
6796.66193181818	7.96310725862634e-06	\\
6797.64071377841	1.00269202664532e-05	\\
6798.61949573864	1.04824722560887e-05	\\
6799.59827769886	8.19321730433703e-06	\\
6800.57705965909	8.62319598249204e-06	\\
6801.55584161932	1.02493151348023e-05	\\
6802.53462357955	9.26662113384368e-06	\\
6803.51340553977	1.01970783803324e-05	\\
6804.4921875	8.49140584265952e-06	\\
6805.47096946023	9.71238062138811e-06	\\
6806.44975142045	9.56501110293176e-06	\\
6807.42853338068	1.0930826141547e-05	\\
6808.40731534091	1.06278746380527e-05	\\
6809.38609730114	8.88505229857698e-06	\\
6810.36487926136	9.48308600842679e-06	\\
6811.34366122159	9.85721579418695e-06	\\
6812.32244318182	1.08530704275437e-05	\\
6813.30122514205	9.91513903866378e-06	\\
6814.28000710227	8.19952437752894e-06	\\
6815.2587890625	1.08783359221289e-05	\\
6816.23757102273	8.71267112554366e-06	\\
6817.21635298295	9.23871928320269e-06	\\
6818.19513494318	1.12971415562093e-05	\\
6819.17391690341	8.46961780899977e-06	\\
6820.15269886364	1.07205139215519e-05	\\
6821.13148082386	8.69418525795146e-06	\\
6822.11026278409	1.08096628972136e-05	\\
6823.08904474432	9.17061578893449e-06	\\
6824.06782670455	9.82682772225835e-06	\\
6825.04660866477	9.37135434907956e-06	\\
6826.025390625	9.24018210409258e-06	\\
6827.00417258523	8.93949578892115e-06	\\
6827.98295454545	1.08683805510451e-05	\\
6828.96173650568	9.2388431827356e-06	\\
6829.94051846591	8.48917922823713e-06	\\
6830.91930042614	9.7597348904039e-06	\\
6831.89808238636	8.59527235257778e-06	\\
6832.87686434659	9.39040977152764e-06	\\
6833.85564630682	7.7142257705185e-06	\\
6834.83442826705	7.84073537880776e-06	\\
6835.81321022727	7.69464368632543e-06	\\
6836.7919921875	9.8179034630573e-06	\\
6837.77077414773	1.00588753232353e-05	\\
6838.74955610795	1.02492339568958e-05	\\
6839.72833806818	9.07800174921569e-06	\\
6840.70712002841	1.02120225435867e-05	\\
6841.68590198864	9.23607878188403e-06	\\
6842.66468394886	1.13884509066025e-05	\\
6843.64346590909	9.8195008136711e-06	\\
6844.62224786932	8.8630983721806e-06	\\
6845.60102982955	9.8616684430908e-06	\\
6846.57981178977	1.01235245044894e-05	\\
6847.55859375	7.58495106702352e-06	\\
6848.53737571023	1.09267461738759e-05	\\
6849.51615767045	8.49904101631058e-06	\\
6850.49493963068	1.17560562579236e-05	\\
6851.47372159091	8.87282988064946e-06	\\
6852.45250355114	9.5121037967609e-06	\\
6853.43128551136	9.8022175590626e-06	\\
6854.41006747159	9.97934050798025e-06	\\
6855.38884943182	1.09698946287161e-05	\\
6856.36763139205	9.18244832943715e-06	\\
6857.34641335227	9.21358098440631e-06	\\
6858.3251953125	8.72808842958726e-06	\\
6859.30397727273	8.42474129676289e-06	\\
6860.28275923295	9.64251138279133e-06	\\
6861.26154119318	8.47464887955442e-06	\\
6862.24032315341	8.26134282434608e-06	\\
6863.21910511364	9.22862602151231e-06	\\
6864.19788707386	7.64340853197388e-06	\\
6865.17666903409	1.00747478712198e-05	\\
6866.15545099432	1.02464982656099e-05	\\
6867.13423295455	7.19327585647972e-06	\\
6868.11301491477	8.92419247900583e-06	\\
6869.091796875	8.8412768792666e-06	\\
6870.07057883523	1.05697648808878e-05	\\
6871.04936079545	1.01331049912772e-05	\\
6872.02814275568	9.28901291908604e-06	\\
6873.00692471591	1.00053798228432e-05	\\
6873.98570667614	1.03866166119185e-05	\\
6874.96448863636	9.07718815642694e-06	\\
6875.94327059659	7.87896865487354e-06	\\
6876.92205255682	1.08643785561331e-05	\\
6877.90083451705	7.36001068195214e-06	\\
6878.87961647727	1.0178425432694e-05	\\
6879.8583984375	1.07661642962295e-05	\\
6880.83718039773	9.92067207963429e-06	\\
6881.81596235795	1.06566127534164e-05	\\
6882.79474431818	1.01089741796325e-05	\\
6883.77352627841	9.09262026337794e-06	\\
6884.75230823864	1.08553407060873e-05	\\
6885.73109019886	7.22908774253638e-06	\\
6886.70987215909	7.5852183758129e-06	\\
6887.68865411932	9.88551107455513e-06	\\
6888.66743607955	1.01932547874952e-05	\\
6889.64621803977	8.24362109628806e-06	\\
6890.625	9.29258582281622e-06	\\
6891.60378196023	8.72428033607437e-06	\\
6892.58256392045	8.8151228879249e-06	\\
6893.56134588068	9.24168869775202e-06	\\
6894.54012784091	9.44278689112294e-06	\\
6895.51890980114	8.96183075003254e-06	\\
6896.49769176136	9.23292796466293e-06	\\
6897.47647372159	1.00829493006774e-05	\\
6898.45525568182	8.65060477833337e-06	\\
6899.43403764205	1.05392318192771e-05	\\
6900.41281960227	8.44041352181143e-06	\\
6901.3916015625	1.05532708762451e-05	\\
6902.37038352273	1.0921560324706e-05	\\
6903.34916548295	1.0101199172237e-05	\\
6904.32794744318	1.02198987322786e-05	\\
6905.30672940341	8.37679704188218e-06	\\
6906.28551136364	1.07472164462807e-05	\\
6907.26429332386	1.03298588080146e-05	\\
6908.24307528409	1.01458861258545e-05	\\
6909.22185724432	8.31462803022397e-06	\\
6910.20063920455	9.5006140856838e-06	\\
6911.17942116477	8.69338778282981e-06	\\
6912.158203125	8.01146520020537e-06	\\
6913.13698508523	1.11699417711737e-05	\\
6914.11576704545	7.77361763537613e-06	\\
6915.09454900568	8.02298186007617e-06	\\
6916.07333096591	8.37653669560228e-06	\\
6917.05211292614	9.00203344235214e-06	\\
6918.03089488636	8.82141556750206e-06	\\
6919.00967684659	1.11397530729292e-05	\\
6919.98845880682	9.68600902913366e-06	\\
6920.96724076705	1.02003517451901e-05	\\
6921.94602272727	1.01860711371339e-05	\\
6922.9248046875	8.99317318594291e-06	\\
6923.90358664773	1.06824369348199e-05	\\
6924.88236860795	8.97028175867884e-06	\\
6925.86115056818	9.22904649092858e-06	\\
6926.83993252841	9.69290798845811e-06	\\
6927.81871448864	9.49753940410925e-06	\\
6928.79749644886	6.98581029093473e-06	\\
6929.77627840909	8.72167971146345e-06	\\
6930.75506036932	8.04675614558912e-06	\\
6931.73384232955	1.01834203940139e-05	\\
6932.71262428977	7.49436976698643e-06	\\
6933.69140625	8.99933742317623e-06	\\
6934.67018821023	8.63070844514547e-06	\\
6935.64897017045	7.67420462893607e-06	\\
6936.62775213068	8.49327366611013e-06	\\
6937.60653409091	8.46480561456296e-06	\\
6938.58531605114	9.32364116767374e-06	\\
6939.56409801136	8.8607462248829e-06	\\
6940.54287997159	8.19274714860983e-06	\\
6941.52166193182	8.53711149533544e-06	\\
6942.50044389205	7.51987118591076e-06	\\
6943.47922585227	9.71922348726519e-06	\\
6944.4580078125	9.31166230298962e-06	\\
6945.43678977273	7.51601602234304e-06	\\
6946.41557173295	9.70222537775898e-06	\\
6947.39435369318	8.78612783424607e-06	\\
6948.37313565341	8.7281312304348e-06	\\
6949.35191761364	9.26575585199599e-06	\\
6950.33069957386	7.46388039895635e-06	\\
6951.30948153409	8.80755690808172e-06	\\
6952.28826349432	1.0297084435329e-05	\\
6953.26704545455	8.99308173190414e-06	\\
6954.24582741477	8.4028772467489e-06	\\
6955.224609375	8.68099944844862e-06	\\
6956.20339133523	7.36163813116578e-06	\\
6957.18217329545	8.47951655294338e-06	\\
6958.16095525568	9.23092128304258e-06	\\
6959.13973721591	8.28177102330892e-06	\\
6960.11851917614	7.09746588892635e-06	\\
6961.09730113636	9.00168047107274e-06	\\
6962.07608309659	9.07136724411174e-06	\\
6963.05486505682	8.32895115752831e-06	\\
6964.03364701705	8.97298813342166e-06	\\
6965.01242897727	8.81046407487184e-06	\\
6965.9912109375	9.11959004046721e-06	\\
6966.96999289773	8.71577777647897e-06	\\
6967.94877485795	9.28874842310321e-06	\\
6968.92755681818	9.80112814729426e-06	\\
6969.90633877841	1.09456602069691e-05	\\
6970.88512073864	9.08497834666059e-06	\\
6971.86390269886	8.59255720728475e-06	\\
6972.84268465909	7.99626937498038e-06	\\
6973.82146661932	9.72315530217442e-06	\\
6974.80024857955	8.61757862407703e-06	\\
6975.77903053977	8.90300079694572e-06	\\
6976.7578125	7.58593811037188e-06	\\
6977.73659446023	1.0022419169085e-05	\\
6978.71537642045	6.24783804648671e-06	\\
6979.69415838068	9.12519921721257e-06	\\
6980.67294034091	9.92741257887067e-06	\\
6981.65172230114	8.49717071867544e-06	\\
6982.63050426136	9.23356865819495e-06	\\
6983.60928622159	7.18212710333978e-06	\\
6984.58806818182	7.99808508894917e-06	\\
6985.56685014205	7.44995894994569e-06	\\
6986.54563210227	8.27395137749764e-06	\\
6987.5244140625	6.46705596761305e-06	\\
6988.50319602273	8.12982988163961e-06	\\
6989.48197798295	8.21347894498842e-06	\\
6990.46075994318	6.55092946677137e-06	\\
6991.43954190341	9.0873400585358e-06	\\
6992.41832386364	7.75235499387994e-06	\\
6993.39710582386	7.17076585925017e-06	\\
6994.37588778409	7.46161236239331e-06	\\
6995.35466974432	7.55413798790209e-06	\\
6996.33345170455	7.36009896695807e-06	\\
6997.31223366477	8.93632573742766e-06	\\
6998.291015625	7.76634498674089e-06	\\
6999.26979758523	8.57633627225182e-06	\\
7000.24857954545	8.87079832640022e-06	\\
7001.22736150568	9.82844048473676e-06	\\
7002.20614346591	9.40479335931138e-06	\\
7003.18492542614	9.32535449536314e-06	\\
7004.16370738636	7.92708132751626e-06	\\
7005.14248934659	8.11451249283504e-06	\\
7006.12127130682	7.4167853276853e-06	\\
7007.10005326705	8.45486729541963e-06	\\
7008.07883522727	9.56792534345002e-06	\\
7009.0576171875	7.23404720537692e-06	\\
7010.03639914773	7.28816541244749e-06	\\
7011.01518110795	7.7834259684121e-06	\\
7011.99396306818	7.30341144034841e-06	\\
7012.97274502841	7.76399948890574e-06	\\
7013.95152698864	6.76973294410322e-06	\\
7014.93030894886	8.17194475874678e-06	\\
7015.90909090909	9.06522000291515e-06	\\
7016.88787286932	8.87290043550987e-06	\\
7017.86665482955	6.75984623661704e-06	\\
7018.84543678977	6.64364429853281e-06	\\
7019.82421875	8.78920000468643e-06	\\
7020.80300071023	7.43891932936118e-06	\\
7021.78178267045	8.37496285335695e-06	\\
7022.76056463068	8.04564673270359e-06	\\
7023.73934659091	6.87282233332515e-06	\\
7024.71812855114	8.82112059277644e-06	\\
7025.69691051136	8.66931014608045e-06	\\
7026.67569247159	7.38876467298259e-06	\\
7027.65447443182	8.21892572797298e-06	\\
7028.63325639205	7.64816255502259e-06	\\
7029.61203835227	7.47336950654812e-06	\\
7030.5908203125	6.69810853064447e-06	\\
7031.56960227273	9.09038903764478e-06	\\
7032.54838423295	8.34377044444961e-06	\\
7033.52716619318	7.85167938186288e-06	\\
7034.50594815341	8.21348257870197e-06	\\
7035.48473011364	5.14814316730946e-06	\\
7036.46351207386	8.24111703676664e-06	\\
7037.44229403409	8.59479523774439e-06	\\
7038.42107599432	7.77048797198546e-06	\\
7039.39985795455	9.11314765243033e-06	\\
7040.37863991477	7.94639180864503e-06	\\
7041.357421875	8.29434870289765e-06	\\
7042.33620383523	6.42530339311496e-06	\\
7043.31498579545	6.85317376648453e-06	\\
7044.29376775568	7.28914975917798e-06	\\
7045.27254971591	8.23169489675528e-06	\\
7046.25133167614	8.85728441553021e-06	\\
7047.23011363636	9.62605041039973e-06	\\
7048.20889559659	7.95654977458009e-06	\\
7049.18767755682	7.12711149844373e-06	\\
7050.16645951705	9.77230723587304e-06	\\
7051.14524147727	7.48788955150821e-06	\\
7052.1240234375	8.49483442705152e-06	\\
7053.10280539773	9.32986717022498e-06	\\
7054.08158735795	6.69765307245801e-06	\\
7055.06036931818	8.23578046941164e-06	\\
7056.03915127841	9.21016668132798e-06	\\
7057.01793323864	8.09776152664792e-06	\\
7057.99671519886	1.02657956956678e-05	\\
7058.97549715909	1.03421246938909e-05	\\
7059.95427911932	8.98349981078117e-06	\\
7060.93306107955	8.72059985890183e-06	\\
7061.91184303977	7.83189091081941e-06	\\
7062.890625	8.87485961764247e-06	\\
7063.86940696023	5.20365264696553e-06	\\
7064.84818892045	6.07381548000757e-06	\\
7065.82697088068	7.46581583281739e-06	\\
7066.80575284091	7.25830531489625e-06	\\
7067.78453480114	7.65630503719013e-06	\\
7068.76331676136	7.32919445873412e-06	\\
7069.74209872159	8.60451664407381e-06	\\
7070.72088068182	8.54827420746701e-06	\\
7071.69966264205	6.16817558238271e-06	\\
7072.67844460227	6.09788685597578e-06	\\
7073.6572265625	7.81019664843505e-06	\\
7074.63600852273	8.0381226764478e-06	\\
7075.61479048295	8.23200434922432e-06	\\
7076.59357244318	7.06720866045392e-06	\\
7077.57235440341	8.82951068363649e-06	\\
7078.55113636364	9.6157427519997e-06	\\
7079.52991832386	6.86778083116649e-06	\\
7080.50870028409	7.12013179951134e-06	\\
7081.48748224432	7.34426885208812e-06	\\
7082.46626420455	8.86480579213838e-06	\\
7083.44504616477	9.09688148279616e-06	\\
7084.423828125	8.50460717088626e-06	\\
7085.40261008523	7.2454034237409e-06	\\
7086.38139204545	6.79340955235815e-06	\\
7087.36017400568	6.66483952597213e-06	\\
7088.33895596591	7.17010744103538e-06	\\
7089.31773792614	8.42722832204058e-06	\\
7090.29651988636	6.53529197226998e-06	\\
7091.27530184659	5.30042095354619e-06	\\
7092.25408380682	7.04320520345721e-06	\\
7093.23286576705	6.90132557031644e-06	\\
7094.21164772727	7.255072570526e-06	\\
7095.1904296875	6.82064111924879e-06	\\
7096.16921164773	7.43874071960211e-06	\\
7097.14799360795	5.39229078480982e-06	\\
7098.12677556818	8.44456323026646e-06	\\
7099.10555752841	7.12051467560178e-06	\\
7100.08433948864	1.06072978387776e-05	\\
7101.06312144886	6.75571418529659e-06	\\
7102.04190340909	6.02326800687085e-06	\\
7103.02068536932	6.85485049777401e-06	\\
7103.99946732955	8.18434822351605e-06	\\
7104.97824928977	6.92540660605786e-06	\\
7105.95703125	4.23045525030101e-06	\\
7106.93581321023	7.29650812167973e-06	\\
7107.91459517045	7.33495237912972e-06	\\
7108.89337713068	8.25129761552966e-06	\\
7109.87215909091	7.35924855132916e-06	\\
7110.85094105114	5.53470080144032e-06	\\
7111.82972301136	6.7195250744904e-06	\\
7112.80850497159	7.64502854441088e-06	\\
7113.78728693182	6.05406338560555e-06	\\
7114.76606889205	6.19047827160818e-06	\\
7115.74485085227	6.54730538337779e-06	\\
7116.7236328125	6.82304740189407e-06	\\
7117.70241477273	8.18352491359644e-06	\\
7118.68119673295	9.1984243085036e-06	\\
7119.65997869318	6.58507167394266e-06	\\
7120.63876065341	5.9592760723336e-06	\\
7121.61754261364	6.45531390502483e-06	\\
7122.59632457386	5.6504466301135e-06	\\
7123.57510653409	7.21929135702035e-06	\\
7124.55388849432	7.28247245617063e-06	\\
7125.53267045455	6.21928652954467e-06	\\
7126.51145241477	7.56855503300748e-06	\\
7127.490234375	7.74914816712916e-06	\\
7128.46901633523	7.41538567388927e-06	\\
7129.44779829545	7.70839866421907e-06	\\
7130.42658025568	8.15700079191428e-06	\\
7131.40536221591	6.47878261953394e-06	\\
7132.38414417614	7.83551053568814e-06	\\
7133.36292613636	6.93200039046396e-06	\\
7134.34170809659	6.34212407245676e-06	\\
7135.32049005682	7.83946718919924e-06	\\
7136.29927201705	8.44721352546233e-06	\\
7137.27805397727	7.85553201392156e-06	\\
7138.2568359375	6.89533337210247e-06	\\
7139.23561789773	7.42243583336104e-06	\\
7140.21439985795	5.52155355895944e-06	\\
7141.19318181818	7.27433471736708e-06	\\
7142.17196377841	4.64300110113398e-06	\\
7143.15074573864	6.02231257063746e-06	\\
7144.12952769886	6.59339382048789e-06	\\
7145.10830965909	7.5304164531287e-06	\\
7146.08709161932	7.50718741017012e-06	\\
7147.06587357955	6.28779183036578e-06	\\
7148.04465553977	6.63476282763165e-06	\\
7149.0234375	7.64926882342665e-06	\\
7150.00221946023	7.14635142567354e-06	\\
7150.98100142045	6.85116427219099e-06	\\
7151.95978338068	5.95194029451851e-06	\\
7152.93856534091	4.32513843623343e-06	\\
7153.91734730114	7.15464667012756e-06	\\
7154.89612926136	8.49693481973793e-06	\\
7155.87491122159	6.36368741139652e-06	\\
7156.85369318182	5.83366395503756e-06	\\
7157.83247514205	7.89113410913996e-06	\\
7158.81125710227	6.18422137420537e-06	\\
7159.7900390625	5.91061060308913e-06	\\
7160.76882102273	6.93859777556614e-06	\\
7161.74760298295	7.09731050924802e-06	\\
7162.72638494318	5.8369136368988e-06	\\
7163.70516690341	9.14313752330234e-06	\\
7164.68394886364	6.29433254772897e-06	\\
7165.66273082386	7.47254179461801e-06	\\
7166.64151278409	8.90974234978848e-06	\\
7167.62029474432	8.64227004670672e-06	\\
7168.59907670455	7.75841253788997e-06	\\
7169.57785866477	7.12648803974074e-06	\\
7170.556640625	7.11407844363775e-06	\\
7171.53542258523	8.67074961294718e-06	\\
7172.51420454545	7.04346038885424e-06	\\
7173.49298650568	7.00993475660987e-06	\\
7174.47176846591	7.25719045321533e-06	\\
7175.45055042614	7.05490683209751e-06	\\
7176.42933238636	7.88591648747968e-06	\\
7177.40811434659	8.80843180526465e-06	\\
7178.38689630682	8.45243179096142e-06	\\
7179.36567826705	7.01767543620656e-06	\\
7180.34446022727	5.44764372649933e-06	\\
7181.3232421875	6.18431350402293e-06	\\
7182.30202414773	7.04777377014604e-06	\\
7183.28080610795	6.1719514933079e-06	\\
7184.25958806818	7.81249625639423e-06	\\
7185.23837002841	7.81221746365987e-06	\\
7186.21715198864	7.2880023229128e-06	\\
7187.19593394886	7.03197449267639e-06	\\
7188.17471590909	7.34798700097976e-06	\\
7189.15349786932	8.17511408248838e-06	\\
7190.13227982955	6.50674960095784e-06	\\
7191.11106178977	6.91174330259568e-06	\\
7192.08984375	6.68679622827646e-06	\\
7193.06862571023	7.38775251478698e-06	\\
7194.04740767045	8.67138707455109e-06	\\
7195.02618963068	7.41678570623166e-06	\\
7196.00497159091	7.79704756933837e-06	\\
7196.98375355114	6.66674505983161e-06	\\
7197.96253551136	6.19471759690052e-06	\\
7198.94131747159	8.13458770625187e-06	\\
7199.92009943182	9.24680614621822e-06	\\
7200.89888139205	6.79965051638e-06	\\
7201.87766335227	7.97298583655694e-06	\\
7202.8564453125	7.44475342166574e-06	\\
7203.83522727273	7.12276890463046e-06	\\
7204.81400923295	7.82610948429572e-06	\\
7205.79279119318	7.38179560118472e-06	\\
7206.77157315341	5.82047959350547e-06	\\
7207.75035511364	7.79517052533959e-06	\\
7208.72913707386	6.95292260704011e-06	\\
7209.70791903409	6.34747031062573e-06	\\
7210.68670099432	5.81618181275752e-06	\\
7211.66548295455	6.72206188613146e-06	\\
7212.64426491477	7.40424955192906e-06	\\
7213.623046875	5.6901688363231e-06	\\
7214.60182883523	7.15949685390977e-06	\\
7215.58061079545	7.4554910806769e-06	\\
7216.55939275568	6.45783605652371e-06	\\
7217.53817471591	6.04126342541703e-06	\\
7218.51695667614	6.64700051891178e-06	\\
7219.49573863636	6.00364557122267e-06	\\
7220.47452059659	5.53289083092619e-06	\\
7221.45330255682	5.83931083165451e-06	\\
7222.43208451705	5.67203444822749e-06	\\
7223.41086647727	7.66004394122578e-06	\\
7224.3896484375	4.52825393625191e-06	\\
7225.36843039773	7.01158503074477e-06	\\
7226.34721235795	7.75192002061109e-06	\\
7227.32599431818	7.8287058389997e-06	\\
7228.30477627841	6.41056863149036e-06	\\
7229.28355823864	4.22612778975569e-06	\\
7230.26234019886	7.854569517969e-06	\\
7231.24112215909	5.78791793756544e-06	\\
7232.21990411932	6.27480475475747e-06	\\
7233.19868607955	8.16578409368431e-06	\\
7234.17746803977	3.33577170446266e-06	\\
7235.15625	6.29830834450768e-06	\\
7236.13503196023	7.12003526979881e-06	\\
7237.11381392045	6.71949917802931e-06	\\
7238.09259588068	7.87920873387898e-06	\\
7239.07137784091	5.3729366445403e-06	\\
7240.05015980114	6.94801747759374e-06	\\
7241.02894176136	5.76505436238659e-06	\\
7242.00772372159	5.20333674930356e-06	\\
7242.98650568182	7.43972270964175e-06	\\
7243.96528764205	7.46762313125893e-06	\\
7244.94406960227	6.52818103537212e-06	\\
7245.9228515625	6.44923201546063e-06	\\
7246.90163352273	6.88781159856256e-06	\\
7247.88041548295	6.78595053231277e-06	\\
7248.85919744318	6.00747094544544e-06	\\
7249.83797940341	3.90134954226128e-06	\\
7250.81676136364	4.45210568483488e-06	\\
7251.79554332386	4.83951943638927e-06	\\
7252.77432528409	8.86314796228757e-06	\\
7253.75310724432	6.37938320564234e-06	\\
7254.73188920455	5.28821327355609e-06	\\
7255.71067116477	5.36214195609306e-06	\\
7256.689453125	6.67660232302533e-06	\\
7257.66823508523	6.4836127661686e-06	\\
7258.64701704545	7.29558725188643e-06	\\
7259.62579900568	6.59645577891387e-06	\\
7260.60458096591	6.25304054444838e-06	\\
7261.58336292614	8.22337358894503e-06	\\
7262.56214488636	5.36777772080135e-06	\\
7263.54092684659	4.87126287621297e-06	\\
7264.51970880682	7.59737054797787e-06	\\
7265.49849076705	5.55078551710015e-06	\\
7266.47727272727	5.93826576243472e-06	\\
7267.4560546875	5.88707651721158e-06	\\
7268.43483664773	6.31351610604241e-06	\\
7269.41361860795	5.65970485058254e-06	\\
7270.39240056818	5.99122649754034e-06	\\
7271.37118252841	4.0653290986326e-06	\\
7272.34996448864	7.89233400325049e-06	\\
7273.32874644886	6.44028276979984e-06	\\
7274.30752840909	7.69410203386553e-06	\\
7275.28631036932	5.23825693155155e-06	\\
7276.26509232955	5.24364300361165e-06	\\
7277.24387428977	6.08451386080312e-06	\\
7278.22265625	7.75683085661251e-06	\\
7279.20143821023	5.28994317737088e-06	\\
7280.18022017045	6.29129775074323e-06	\\
7281.15900213068	6.04782243438953e-06	\\
7282.13778409091	6.61111805910105e-06	\\
7283.11656605114	5.84525388599957e-06	\\
7284.09534801136	6.92310088667202e-06	\\
7285.07412997159	5.97189554869655e-06	\\
7286.05291193182	5.65274390480751e-06	\\
7287.03169389205	6.54718028022432e-06	\\
7288.01047585227	5.19449161762797e-06	\\
7288.9892578125	5.90567241850227e-06	\\
7289.96803977273	7.04207992874677e-06	\\
7290.94682173295	5.19876324772159e-06	\\
7291.92560369318	5.90960939020471e-06	\\
7292.90438565341	5.18029364766841e-06	\\
7293.88316761364	4.97687869165066e-06	\\
7294.86194957386	8.08672907454759e-06	\\
7295.84073153409	3.84988465060479e-06	\\
7296.81951349432	4.79934633118096e-06	\\
7297.79829545455	7.01450720084982e-06	\\
7298.77707741477	6.53756474411316e-06	\\
7299.755859375	8.49609671558767e-06	\\
7300.73464133523	5.04356469764049e-06	\\
7301.71342329545	6.90456651377604e-06	\\
7302.69220525568	6.22949927544613e-06	\\
7303.67098721591	5.57007003100621e-06	\\
7304.64976917614	4.68340994711553e-06	\\
7305.62855113636	4.92618056942603e-06	\\
7306.60733309659	7.20560793479691e-06	\\
7307.58611505682	5.81579563841884e-06	\\
7308.56489701705	6.50550940970217e-06	\\
7309.54367897727	3.28520442838928e-06	\\
7310.5224609375	4.88257498783354e-06	\\
7311.50124289773	6.3856660969621e-06	\\
7312.48002485795	4.5354501185797e-06	\\
7313.45880681818	4.83877063129812e-06	\\
7314.43758877841	5.06046391725548e-06	\\
7315.41637073864	5.77113170149396e-06	\\
7316.39515269886	5.63103412880109e-06	\\
7317.37393465909	5.88362883326509e-06	\\
7318.35271661932	5.91932897876754e-06	\\
7319.33149857955	3.18845843102239e-06	\\
7320.31028053977	4.13797475812765e-06	\\
7321.2890625	5.21990516212425e-06	\\
7322.26784446023	6.78738919096852e-06	\\
7323.24662642045	6.47664576304779e-06	\\
7324.22540838068	5.30073041313878e-06	\\
7325.20419034091	5.22171980104129e-06	\\
7326.18297230114	6.11432584455409e-06	\\
7327.16175426136	4.11307042332488e-06	\\
7328.14053622159	3.67011802284032e-06	\\
7329.11931818182	5.73063726081646e-06	\\
7330.09810014205	6.81160196326671e-06	\\
7331.07688210227	5.82811211448095e-06	\\
7332.0556640625	5.09782188618115e-06	\\
7333.03444602273	6.20124130241944e-06	\\
7334.01322798295	4.54898597443027e-06	\\
7334.99200994318	5.49551013708553e-06	\\
7335.97079190341	4.97055406634593e-06	\\
7336.94957386364	4.06591964934346e-06	\\
7337.92835582386	3.86963500057275e-06	\\
7338.90713778409	5.60269344669455e-06	\\
7339.88591974432	4.7857867548349e-06	\\
7340.86470170455	5.08261708251055e-06	\\
7341.84348366477	6.48712397562185e-06	\\
7342.822265625	6.19034327857331e-06	\\
7343.80104758523	4.28842486001295e-06	\\
7344.77982954545	4.60197945084053e-06	\\
7345.75861150568	5.60045947911637e-06	\\
7346.73739346591	4.20253629876795e-06	\\
7347.71617542614	4.76664400145056e-06	\\
7348.69495738636	4.99510464890396e-06	\\
7349.67373934659	5.08706229065823e-06	\\
7350.65252130682	6.01850881682164e-06	\\
7351.63130326705	6.18657358749274e-06	\\
7352.61008522727	6.27049524797653e-06	\\
7353.5888671875	7.09958585065447e-06	\\
7354.56764914773	6.03582332259435e-06	\\
7355.54643110795	3.81126311520601e-06	\\
7356.52521306818	5.49096904570087e-06	\\
7357.50399502841	4.30644035293221e-06	\\
7358.48277698864	4.39287509297493e-06	\\
7359.46155894886	3.04246569646537e-06	\\
7360.44034090909	4.57223407750292e-06	\\
7361.41912286932	4.16266025856394e-06	\\
7362.39790482955	3.75900932752029e-06	\\
7363.37668678977	7.83766998504582e-06	\\
7364.35546875	5.85498453247868e-06	\\
7365.33425071023	3.75737222685654e-06	\\
7366.31303267045	5.13952327171812e-06	\\
7367.29181463068	5.47098540516169e-06	\\
7368.27059659091	3.77402135397824e-06	\\
7369.24937855114	7.28944575541746e-06	\\
7370.22816051136	5.82045405367903e-06	\\
7371.20694247159	6.59263739813131e-06	\\
7372.18572443182	3.40771502331951e-06	\\
7373.16450639205	4.2498745971454e-06	\\
7374.14328835227	3.53074214823303e-06	\\
7375.1220703125	5.66125343766712e-06	\\
7376.10085227273	6.11754641454362e-06	\\
7377.07963423295	5.92706766272616e-06	\\
7378.05841619318	3.50192850411342e-06	\\
7379.03719815341	4.12580538239628e-06	\\
7380.01598011364	4.47455363338233e-06	\\
7380.99476207386	4.62149698849273e-06	\\
7381.97354403409	4.62352974081509e-06	\\
7382.95232599432	6.3230757015239e-06	\\
7383.93110795455	4.43234492850008e-06	\\
7384.90988991477	4.80435939880114e-06	\\
7385.888671875	5.98806930515918e-06	\\
7386.86745383523	3.69214133286593e-06	\\
7387.84623579545	6.70297292936198e-06	\\
7388.82501775568	6.37441486510835e-06	\\
7389.80379971591	5.69470898964145e-06	\\
7390.78258167614	6.44060523333739e-06	\\
7391.76136363636	6.49572685809801e-06	\\
7392.74014559659	3.64376626670725e-06	\\
7393.71892755682	4.77704923642584e-06	\\
7394.69770951705	6.65271652507997e-06	\\
7395.67649147727	5.6349573243772e-06	\\
7396.6552734375	6.55776022743622e-06	\\
7397.63405539773	6.92971044259062e-06	\\
7398.61283735795	4.20280094473796e-06	\\
7399.59161931818	4.55267633569967e-06	\\
7400.57040127841	5.70531235421055e-06	\\
7401.54918323864	5.75256410081528e-06	\\
7402.52796519886	4.66271984936084e-06	\\
7403.50674715909	2.62168543003197e-06	\\
7404.48552911932	6.27548004172095e-06	\\
7405.46431107955	5.04685945558179e-06	\\
7406.44309303977	3.36659760262065e-06	\\
7407.421875	5.43545041997864e-06	\\
7408.40065696023	5.41174804651997e-06	\\
7409.37943892045	4.68829422373808e-06	\\
7410.35822088068	6.16133896477355e-06	\\
7411.33700284091	2.90278188898411e-06	\\
7412.31578480114	5.61550343410083e-06	\\
7413.29456676136	6.6216255841148e-06	\\
7414.27334872159	4.21790938294673e-06	\\
7415.25213068182	6.20523452022968e-06	\\
7416.23091264205	4.70483858937576e-06	\\
7417.20969460227	6.55954827066786e-06	\\
7418.1884765625	6.36301084061331e-06	\\
7419.16725852273	6.35807364942749e-06	\\
7420.14604048295	4.126986380686e-06	\\
7421.12482244318	5.17201211469855e-06	\\
7422.10360440341	5.71578879732527e-06	\\
7423.08238636364	4.99128006675008e-06	\\
7424.06116832386	6.72292317818973e-06	\\
7425.03995028409	6.37982177866927e-06	\\
7426.01873224432	5.71957847888348e-06	\\
7426.99751420455	5.58826091693333e-06	\\
7427.97629616477	5.98273482242881e-06	\\
7428.955078125	3.95095627943618e-06	\\
7429.93386008523	7.3771411616295e-06	\\
7430.91264204545	5.48260241264376e-06	\\
7431.89142400568	6.42144790067013e-06	\\
7432.87020596591	5.52678068559061e-06	\\
7433.84898792614	4.55286667398749e-06	\\
7434.82776988636	5.24917909419145e-06	\\
7435.80655184659	5.6806372251235e-06	\\
7436.78533380682	4.27351488872869e-06	\\
7437.76411576705	5.57216090692897e-06	\\
7438.74289772727	4.33433821883973e-06	\\
7439.7216796875	3.94991508187455e-06	\\
7440.70046164773	5.39426412212003e-06	\\
7441.67924360795	4.62018039085558e-06	\\
7442.65802556818	5.18133368194925e-06	\\
7443.63680752841	4.52587857424582e-06	\\
7444.61558948864	3.87403447504924e-06	\\
7445.59437144886	7.83956973457039e-06	\\
7446.57315340909	5.92785522511475e-06	\\
7447.55193536932	6.23683740365732e-06	\\
7448.53071732955	3.81359323343171e-06	\\
7449.50949928977	6.8777854758642e-06	\\
7450.48828125	2.33704168944561e-06	\\
7451.46706321023	5.36580021213891e-06	\\
7452.44584517045	5.32626703398889e-06	\\
7453.42462713068	6.03065127013799e-06	\\
7454.40340909091	4.23565793416046e-06	\\
7455.38219105114	5.32253441603504e-06	\\
7456.36097301136	4.4092718834956e-06	\\
7457.33975497159	6.06808225205653e-06	\\
7458.31853693182	3.54086846506586e-06	\\
7459.29731889205	3.07386192788218e-06	\\
7460.27610085227	3.17668808002343e-06	\\
7461.2548828125	4.27568691935e-06	\\
7462.23366477273	3.56903433638363e-06	\\
7463.21244673295	5.17103477258006e-06	\\
7464.19122869318	1.9883279240002e-06	\\
7465.17001065341	4.39733318804492e-06	\\
7466.14879261364	5.40363653642592e-06	\\
7467.12757457386	4.92270625918633e-06	\\
7468.10635653409	4.03171322755911e-06	\\
7469.08513849432	3.98570656752896e-06	\\
7470.06392045455	4.71379229883799e-06	\\
7471.04270241477	3.61124724209352e-06	\\
7472.021484375	4.14663681480613e-06	\\
7473.00026633523	3.64812872042375e-06	\\
7473.97904829545	3.08163089470188e-06	\\
7474.95783025568	4.92522709553059e-06	\\
7475.93661221591	4.02164660333881e-06	\\
7476.91539417614	4.33527575337374e-06	\\
7477.89417613636	2.63910397925408e-06	\\
7478.87295809659	2.83356114553346e-06	\\
7479.85174005682	4.59569171376656e-06	\\
7480.83052201705	4.98241793309592e-06	\\
7481.80930397727	3.62698730066872e-06	\\
7482.7880859375	3.91602254365692e-06	\\
7483.76686789773	3.78309470585254e-06	\\
7484.74564985795	3.55222883631918e-06	\\
7485.72443181818	4.53457735323544e-06	\\
7486.70321377841	3.32860649137197e-06	\\
7487.68199573864	4.35003716038716e-06	\\
7488.66077769886	3.82141619577163e-06	\\
7489.63955965909	3.01939800233069e-06	\\
7490.61834161932	4.31161958945985e-06	\\
7491.59712357955	4.07262639513861e-06	\\
7492.57590553977	3.59133088572032e-06	\\
7493.5546875	6.03389221024544e-06	\\
7494.53346946023	4.31643715422768e-06	\\
7495.51225142045	3.96782682636193e-06	\\
7496.49103338068	5.52882612795589e-06	\\
7497.46981534091	6.06597537715129e-06	\\
7498.44859730114	5.79082206642018e-06	\\
7499.42737926136	7.84357946747074e-06	\\
7500.40616122159	6.37017825905959e-06	\\
7501.38494318182	4.20956771407161e-06	\\
7502.36372514205	3.05014191476273e-06	\\
7503.34250710227	4.69982511451809e-06	\\
7504.3212890625	4.32524613302753e-06	\\
7505.30007102273	6.63834019971496e-06	\\
7506.27885298295	4.5128871899865e-06	\\
7507.25763494318	5.1692563935846e-06	\\
7508.23641690341	4.2580367154282e-06	\\
7509.21519886364	5.86940830232645e-06	\\
7510.19398082386	4.96296010632828e-06	\\
7511.17276278409	4.63557761045767e-06	\\
7512.15154474432	4.75851665710801e-06	\\
7513.13032670455	5.5078996742285e-06	\\
7514.10910866477	5.09758191158809e-06	\\
7515.087890625	6.26985183100388e-06	\\
7516.06667258523	4.65336632076806e-06	\\
7517.04545454545	4.46280736816697e-06	\\
7518.02423650568	4.22610143300588e-06	\\
7519.00301846591	5.43009235810254e-06	\\
7519.98180042614	6.13432373972132e-06	\\
7520.96058238636	5.33854233142122e-06	\\
7521.93936434659	5.21088325766533e-06	\\
7522.91814630682	5.46985458655243e-06	\\
7523.89692826705	3.75258689180738e-06	\\
7524.87571022727	5.01578295534371e-06	\\
7525.8544921875	4.62467006793725e-06	\\
7526.83327414773	5.61124368942171e-06	\\
7527.81205610795	5.92016705180006e-06	\\
7528.79083806818	4.22082893767708e-06	\\
7529.76962002841	4.40182230915195e-06	\\
7530.74840198864	4.58700038903706e-06	\\
7531.72718394886	5.52215496141287e-06	\\
7532.70596590909	5.57250773459329e-06	\\
7533.68474786932	6.68784551646466e-06	\\
7534.66352982955	4.44895194032244e-06	\\
7535.64231178977	4.51874871154152e-06	\\
7536.62109375	4.75454579507797e-06	\\
7537.59987571023	4.54316982089048e-06	\\
7538.57865767045	4.18498271610446e-06	\\
7539.55743963068	4.13849856737795e-06	\\
7540.53622159091	5.81106002718266e-06	\\
7541.51500355114	4.85411512719485e-06	\\
7542.49378551136	4.44765176472957e-06	\\
7543.47256747159	3.63069587958068e-06	\\
7544.45134943182	4.49201347873191e-06	\\
7545.43013139205	3.48066677186207e-06	\\
7546.40891335227	4.36514333777945e-06	\\
7547.3876953125	4.08433042306726e-06	\\
7548.36647727273	4.14128069456093e-06	\\
7549.34525923295	2.55097938407784e-06	\\
7550.32404119318	4.29817976431332e-06	\\
7551.30282315341	4.72949325606651e-06	\\
7552.28160511364	3.33939440990148e-06	\\
7553.26038707386	3.80483634281238e-06	\\
7554.23916903409	5.50154933041115e-06	\\
7555.21795099432	5.60487290659301e-06	\\
7556.19673295455	4.2219316832547e-06	\\
7557.17551491477	2.76900873159877e-06	\\
7558.154296875	4.18233147446881e-06	\\
7559.13307883523	2.48124243402737e-06	\\
7560.11186079545	2.50263250958826e-06	\\
7561.09064275568	3.93573168652844e-06	\\
7562.06942471591	5.11513910649809e-06	\\
7563.04820667614	3.1225217725807e-06	\\
7564.02698863636	4.94718032053381e-06	\\
7565.00577059659	5.73968874228248e-06	\\
7565.98455255682	2.7828822126662e-06	\\
7566.96333451705	4.65594915972451e-06	\\
7567.94211647727	3.5701466867e-06	\\
7568.9208984375	1.70194309848029e-06	\\
7569.89968039773	6.00360354399783e-06	\\
7570.87846235795	5.12725392454989e-06	\\
7571.85724431818	4.45077451533475e-06	\\
7572.83602627841	3.8719040703187e-06	\\
7573.81480823864	5.60508242442801e-06	\\
7574.79359019886	4.48000878755261e-06	\\
7575.77237215909	3.06551024123267e-06	\\
7576.75115411932	5.65359790407393e-06	\\
7577.72993607955	3.75967845040228e-06	\\
7578.70871803977	2.65425148217887e-06	\\
7579.6875	6.07974693784516e-06	\\
7580.66628196023	5.19652705067176e-06	\\
7581.64506392045	4.31599982339838e-06	\\
7582.62384588068	4.58896372991298e-06	\\
7583.60262784091	3.73214738875939e-06	\\
7584.58140980114	4.66180699097234e-06	\\
7585.56019176136	3.97733676116789e-06	\\
7586.53897372159	5.00796710321266e-06	\\
7587.51775568182	3.74990394916391e-06	\\
7588.49653764205	3.7520689306491e-06	\\
7589.47531960227	3.66176774506017e-06	\\
7590.4541015625	4.53206265567111e-06	\\
7591.43288352273	3.9855012501075e-06	\\
7592.41166548295	5.50764738586513e-06	\\
7593.39044744318	3.87004394508581e-06	\\
7594.36922940341	4.78464747404864e-06	\\
7595.34801136364	5.91710282389855e-06	\\
7596.32679332386	4.92337388321958e-06	\\
7597.30557528409	2.37427714411579e-06	\\
7598.28435724432	4.49087887697452e-06	\\
7599.26313920455	4.63510344992137e-06	\\
7600.24192116477	4.27135765210192e-06	\\
7601.220703125	5.89352219803403e-06	\\
7602.19948508523	3.74784057920321e-06	\\
7603.17826704545	4.18881037018395e-06	\\
7604.15704900568	4.17354926565721e-06	\\
7605.13583096591	3.92022497715903e-06	\\
7606.11461292614	4.71230342354168e-06	\\
7607.09339488636	5.40787266788987e-06	\\
7608.07217684659	3.82572633451429e-06	\\
7609.05095880682	4.84853802115915e-06	\\
7610.02974076705	4.06166698253436e-06	\\
7611.00852272727	3.4303957864252e-06	\\
7611.9873046875	5.14655931937152e-06	\\
7612.96608664773	5.76705411508653e-06	\\
7613.94486860795	3.67246612244696e-06	\\
7614.92365056818	6.41313204681477e-06	\\
7615.90243252841	5.56574013213049e-06	\\
7616.88121448864	3.94981227693274e-06	\\
7617.85999644886	4.02426121597567e-06	\\
7618.83877840909	3.99407895649451e-06	\\
7619.81756036932	3.59290603120155e-06	\\
7620.79634232955	2.46233482841621e-06	\\
7621.77512428977	3.6227120069341e-06	\\
7622.75390625	3.42360299898322e-06	\\
7623.73268821023	6.62740790107727e-06	\\
7624.71147017045	5.47611962933178e-06	\\
7625.69025213068	5.44979654787558e-06	\\
7626.66903409091	5.33775078500456e-06	\\
7627.64781605114	4.92922571483509e-06	\\
7628.62659801136	3.89992442850543e-06	\\
7629.60537997159	5.21435470953109e-06	\\
7630.58416193182	3.13579845275262e-06	\\
7631.56294389205	5.0777017296725e-06	\\
7632.54172585227	4.18981309727331e-06	\\
7633.5205078125	2.45509981747207e-06	\\
7634.49928977273	4.56186603792686e-06	\\
7635.47807173295	4.006786417681e-06	\\
7636.45685369318	4.5403922554975e-06	\\
7637.43563565341	2.42813646683603e-06	\\
7638.41441761364	4.61307410953056e-06	\\
7639.39319957386	4.6764788397044e-06	\\
7640.37198153409	4.10011638203443e-06	\\
7641.35076349432	5.62411805369633e-06	\\
7642.32954545455	3.6901464657307e-06	\\
7643.30832741477	5.35507849625666e-06	\\
7644.287109375	2.84948814230443e-06	\\
7645.26589133523	3.52417757031737e-06	\\
7646.24467329545	4.74788576829582e-06	\\
7647.22345525568	3.58830053875123e-06	\\
7648.20223721591	3.79753435933981e-06	\\
7649.18101917614	4.81642287414884e-06	\\
7650.15980113636	5.34532852713528e-06	\\
7651.13858309659	5.15685895543199e-06	\\
7652.11736505682	4.46567907122458e-06	\\
7653.09614701705	5.58706200820448e-06	\\
7654.07492897727	1.62777026192603e-06	\\
7655.0537109375	4.42849162943661e-06	\\
7656.03249289773	4.5850474256527e-06	\\
7657.01127485795	4.47754003783897e-06	\\
7657.99005681818	4.22191408084168e-06	\\
7658.96883877841	1.85442369018306e-06	\\
7659.94762073864	4.08155683028503e-06	\\
7660.92640269886	4.90447495583144e-06	\\
7661.90518465909	3.86708475906068e-06	\\
7662.88396661932	2.653884403519e-06	\\
7663.86274857955	4.78149509066182e-06	\\
7664.84153053977	3.69681556443122e-06	\\
7665.8203125	3.85621761547107e-06	\\
7666.79909446023	3.20133858984376e-06	\\
7667.77787642045	4.24744934960366e-06	\\
7668.75665838068	4.72210833159117e-06	\\
7669.73544034091	2.10107482764759e-06	\\
7670.71422230114	4.27529894237362e-06	\\
7671.69300426136	4.91426154338271e-06	\\
7672.67178622159	4.95470070926906e-06	\\
7673.65056818182	4.55422597174519e-06	\\
7674.62935014205	3.99977465602232e-06	\\
7675.60813210227	3.61409084716237e-06	\\
7676.5869140625	4.07065295723604e-06	\\
7677.56569602273	4.81614797464205e-06	\\
7678.54447798295	2.7340793909629e-06	\\
7679.52325994318	3.76061645585216e-06	\\
7680.50204190341	4.16520828496909e-06	\\
7681.48082386364	4.51297928333044e-06	\\
7682.45960582386	4.65200465296081e-06	\\
7683.43838778409	5.43337737793354e-06	\\
7684.41716974432	3.90078892768004e-06	\\
7685.39595170455	3.87465502019885e-06	\\
7686.37473366477	3.74238914890005e-06	\\
7687.353515625	3.93110414571915e-06	\\
7688.33229758523	4.28292976312151e-06	\\
7689.31107954545	4.45764977305961e-06	\\
7690.28986150568	3.12652210215256e-06	\\
7691.26864346591	4.20631812017111e-06	\\
7692.24742542614	4.09359689925947e-06	\\
7693.22620738636	5.12850908014277e-06	\\
7694.20498934659	4.76079014148617e-06	\\
7695.18377130682	3.38122594260053e-06	\\
7696.16255326705	4.64199081300388e-06	\\
7697.14133522727	3.96039221511818e-06	\\
7698.1201171875	4.09926944899665e-06	\\
7699.09889914773	4.76394164988522e-06	\\
7700.07768110795	3.90248259033142e-06	\\
7701.05646306818	3.33587846532945e-06	\\
7702.03524502841	3.95900197486213e-06	\\
7703.01402698864	4.19861099051873e-06	\\
7703.99280894886	2.94925393800252e-06	\\
7704.97159090909	2.98780881281544e-06	\\
7705.95037286932	4.0488013303604e-06	\\
7706.92915482955	2.95736774474916e-06	\\
7707.90793678977	4.20689719788065e-06	\\
7708.88671875	3.09109159917926e-06	\\
7709.86550071023	4.33667858424485e-06	\\
7710.84428267045	4.67956304707366e-06	\\
7711.82306463068	3.54165479877292e-06	\\
7712.80184659091	2.55244976078485e-06	\\
7713.78062855114	3.0625214286451e-06	\\
7714.75941051136	3.62581613470794e-06	\\
7715.73819247159	4.68486497142283e-06	\\
7716.71697443182	3.51065333079166e-06	\\
7717.69575639205	3.28512642460656e-06	\\
7718.67453835227	2.17203319423422e-06	\\
7719.6533203125	4.64098388946889e-06	\\
7720.63210227273	3.55249809493949e-06	\\
7721.61088423295	3.82844600385207e-06	\\
7722.58966619318	3.34591909470378e-06	\\
7723.56844815341	3.87251547214441e-06	\\
7724.54723011364	3.95879703107426e-06	\\
7725.52601207386	1.82972445443292e-06	\\
7726.50479403409	4.36759961762938e-06	\\
7727.48357599432	3.64380065293939e-06	\\
7728.46235795455	1.85608655273888e-06	\\
7729.44113991477	4.12729278171093e-06	\\
7730.419921875	3.86560248724129e-06	\\
7731.39870383523	4.41758134190466e-06	\\
7732.37748579545	4.87386689838634e-06	\\
7733.35626775568	4.73348105978403e-06	\\
7734.33504971591	2.94722066671542e-06	\\
7735.31383167614	4.32130862961888e-06	\\
7736.29261363636	2.22093269973854e-06	\\
7737.27139559659	2.47054544936932e-06	\\
7738.25017755682	4.90618001004555e-06	\\
7739.22895951705	4.31776830469163e-06	\\
7740.20774147727	1.95407818578341e-06	\\
7741.1865234375	3.83910962561585e-06	\\
7742.16530539773	4.18982431513199e-06	\\
7743.14408735795	5.37452121639737e-06	\\
7744.12286931818	4.18663628391368e-06	\\
7745.10165127841	3.12903631441394e-06	\\
7746.08043323864	3.10971572066128e-06	\\
7747.05921519886	4.40448886946479e-06	\\
7748.03799715909	4.48674439766442e-06	\\
7749.01677911932	4.29920168701701e-06	\\
7749.99556107955	3.83199644835725e-06	\\
7750.97434303977	4.24186058942224e-06	\\
7751.953125	5.11586571123422e-06	\\
7752.93190696023	5.20256728779281e-06	\\
7753.91068892045	3.24904514898141e-06	\\
7754.88947088068	4.56117961782315e-06	\\
7755.86825284091	4.98698700940848e-06	\\
7756.84703480114	3.84201648427402e-06	\\
7757.82581676136	3.06119674875592e-06	\\
7758.80459872159	4.17085415385711e-06	\\
7759.78338068182	4.36005036054809e-06	\\
7760.76216264205	4.34195690034981e-06	\\
7761.74094460227	3.41588146490222e-06	\\
7762.7197265625	3.48879632089484e-06	\\
7763.69850852273	4.2908025167518e-06	\\
7764.67729048295	4.33254538856977e-06	\\
7765.65607244318	4.73231363258992e-06	\\
7766.63485440341	3.340634070762e-06	\\
7767.61363636364	5.48242148983841e-06	\\
7768.59241832386	4.22996054833292e-06	\\
7769.57120028409	4.76552388859543e-06	\\
7770.54998224432	4.3517920142964e-06	\\
7771.52876420455	4.01541983840201e-06	\\
7772.50754616477	5.92871536771216e-06	\\
7773.486328125	4.08940857631113e-06	\\
7774.46511008523	4.25571874052913e-06	\\
7775.44389204545	4.64748760244117e-06	\\
7776.42267400568	6.03065366493948e-06	\\
7777.40145596591	4.99571749574305e-06	\\
7778.38023792614	5.49656103399419e-06	\\
7779.35901988636	5.25854656844097e-06	\\
7780.33780184659	5.33070870472397e-06	\\
7781.31658380682	4.09657110988685e-06	\\
7782.29536576705	5.29553235207457e-06	\\
7783.27414772727	4.25147801781855e-06	\\
7784.2529296875	3.16123612714862e-06	\\
7785.23171164773	5.72160298677731e-06	\\
7786.21049360795	6.31488479139916e-06	\\
7787.18927556818	4.22816546546588e-06	\\
7788.16805752841	4.68381464895215e-06	\\
7789.14683948864	6.15030287699029e-06	\\
7790.12562144886	4.72922637812534e-06	\\
7791.10440340909	4.78853931123531e-06	\\
7792.08318536932	5.01656048659397e-06	\\
7793.06196732955	4.26027806624917e-06	\\
7794.04074928977	4.13831220620899e-06	\\
7795.01953125	3.38433659547656e-06	\\
7795.99831321023	6.68220235450437e-06	\\
7796.97709517045	3.44766832771558e-06	\\
7797.95587713068	4.52745336627556e-06	\\
7798.93465909091	5.01111726225561e-06	\\
7799.91344105114	5.60383885396528e-06	\\
7800.89222301136	5.16791597304632e-06	\\
7801.87100497159	5.2615137244291e-06	\\
7802.84978693182	4.94946146148587e-06	\\
7803.82856889205	7.30207909570171e-06	\\
7804.80735085227	3.95198885871925e-06	\\
7805.7861328125	5.44079256392773e-06	\\
7806.76491477273	4.32038683162304e-06	\\
7807.74369673295	4.16479948913211e-06	\\
7808.72247869318	5.31230418221074e-06	\\
7809.70126065341	4.47432751198009e-06	\\
7810.68004261364	1.05662354843601e-06	\\
7811.65882457386	6.03627315050138e-06	\\
7812.63760653409	4.29622775479527e-06	\\
7813.61638849432	4.82427296066673e-06	\\
7814.59517045455	4.28893822995476e-06	\\
7815.57395241477	4.75136512487336e-06	\\
7816.552734375	5.02516756713952e-06	\\
7817.53151633523	4.60716237754002e-06	\\
7818.51029829545	4.60459162370192e-06	\\
7819.48908025568	4.67007304760345e-06	\\
7820.46786221591	3.88438070475803e-06	\\
7821.44664417614	5.43581427079734e-06	\\
7822.42542613636	3.81724133568985e-06	\\
7823.40420809659	5.215024576354e-06	\\
7824.38299005682	4.02097847888771e-06	\\
7825.36177201705	3.76984263592982e-06	\\
7826.34055397727	5.09778069092448e-06	\\
7827.3193359375	3.29946946313407e-06	\\
7828.29811789773	3.65911162150355e-06	\\
7829.27689985795	4.90525686658367e-06	\\
7830.25568181818	5.46550422187125e-06	\\
7831.23446377841	6.01714400250215e-06	\\
7832.21324573864	2.60059023043102e-06	\\
7833.19202769886	5.05402345523039e-06	\\
7834.17080965909	3.32428321093399e-06	\\
7835.14959161932	2.69123978248195e-06	\\
7836.12837357955	3.34110687847005e-06	\\
7837.10715553977	5.35470227180458e-06	\\
7838.0859375	3.45489915527688e-06	\\
7839.06471946023	4.17228088863576e-06	\\
7840.04350142045	3.80462733948569e-06	\\
7841.02228338068	3.83744345290518e-06	\\
7842.00106534091	3.4452403896727e-06	\\
7842.97984730114	2.66226442075913e-06	\\
7843.95862926136	3.26944912298455e-06	\\
7844.93741122159	3.63450352479181e-06	\\
7845.91619318182	4.33567264396821e-06	\\
7846.89497514205	4.58286646330611e-06	\\
7847.87375710227	4.81121396393302e-06	\\
7848.8525390625	4.56548341684755e-06	\\
7849.83132102273	4.04349579586436e-06	\\
7850.81010298295	5.37264180561708e-06	\\
7851.78888494318	3.35701019221998e-06	\\
7852.76766690341	2.70627707573686e-06	\\
7853.74644886364	4.76191538705938e-06	\\
7854.72523082386	3.00438357601906e-06	\\
7855.70401278409	4.05669250484929e-06	\\
7856.68279474432	4.43917052510581e-06	\\
7857.66157670455	5.55866431346123e-06	\\
7858.64035866477	5.73985382432565e-06	\\
7859.619140625	3.93839701373357e-06	\\
7860.59792258523	3.39315624063424e-06	\\
7861.57670454545	4.67808482227467e-06	\\
7862.55548650568	5.29594626830844e-06	\\
7863.53426846591	4.21716771807619e-06	\\
7864.51305042614	4.26041710821128e-06	\\
7865.49183238636	2.8657677212984e-06	\\
7866.47061434659	3.45680966403192e-06	\\
7867.44939630682	4.82442026495339e-06	\\
7868.42817826705	5.68634063832196e-06	\\
7869.40696022727	4.23048319252244e-06	\\
7870.3857421875	5.04416648641942e-06	\\
7871.36452414773	2.30835058821093e-06	\\
7872.34330610795	2.6381147051253e-06	\\
7873.32208806818	5.18064807638913e-06	\\
7874.30087002841	3.82007827386475e-06	\\
7875.27965198864	3.58124296217825e-06	\\
7876.25843394886	4.74346232858677e-06	\\
7877.23721590909	4.98376181473731e-06	\\
7878.21599786932	3.42170242140251e-06	\\
7879.19477982955	3.6027884504659e-06	\\
7880.17356178977	3.14022323890567e-06	\\
7881.15234375	4.37356788247811e-06	\\
7882.13112571023	5.24689366884665e-06	\\
7883.10990767045	4.02296427776396e-06	\\
7884.08868963068	4.76079929855204e-06	\\
7885.06747159091	4.5615459249884e-06	\\
7886.04625355114	3.58129482751424e-06	\\
7887.02503551136	3.67656809250265e-06	\\
7888.00381747159	2.2237178413394e-06	\\
7888.98259943182	3.72939120372255e-06	\\
7889.96138139205	4.31400509555247e-06	\\
7890.94016335227	4.43892323140954e-06	\\
7891.9189453125	2.13054613005661e-06	\\
7892.89772727273	2.16627975804845e-06	\\
7893.87650923295	4.40311548087774e-06	\\
7894.85529119318	2.69273403491728e-06	\\
7895.83407315341	3.08298890042333e-06	\\
7896.81285511364	3.33441878075491e-06	\\
7897.79163707386	3.41982050472028e-06	\\
7898.77041903409	3.63253078530786e-06	\\
7899.74920099432	1.98737999908575e-06	\\
7900.72798295455	3.7570033439445e-06	\\
7901.70676491477	3.45423245692985e-06	\\
7902.685546875	2.22676361909668e-06	\\
7903.66432883523	2.4114961782452e-06	\\
7904.64311079545	3.55650989672247e-06	\\
7905.62189275568	4.15375744028159e-06	\\
7906.60067471591	2.96490114178381e-06	\\
7907.57945667614	3.46528629083978e-06	\\
7908.55823863636	3.31361912942897e-06	\\
7909.53702059659	2.58838772370549e-06	\\
7910.51580255682	3.02838559010587e-06	\\
7911.49458451705	2.67458148705469e-06	\\
7912.47336647727	4.09749015083066e-06	\\
7913.4521484375	4.17047611007577e-06	\\
7914.43093039773	3.7759416224427e-06	\\
7915.40971235795	1.68794902395497e-06	\\
7916.38849431818	3.33801525067724e-06	\\
7917.36727627841	3.84574583332815e-06	\\
7918.34605823864	2.72962842979384e-06	\\
7919.32484019886	3.30674558688336e-06	\\
7920.30362215909	3.76206748918844e-06	\\
7921.28240411932	5.03786992992522e-06	\\
7922.26118607955	3.90418381623217e-06	\\
7923.23996803977	3.09769965760844e-06	\\
7924.21875	5.19482194964716e-06	\\
7925.19753196023	3.80938148357651e-06	\\
7926.17631392045	3.69221667566231e-06	\\
7927.15509588068	4.68190532859842e-06	\\
7928.13387784091	5.6444276499872e-06	\\
7929.11265980114	3.47410193061751e-06	\\
7930.09144176136	5.32811262722658e-06	\\
7931.07022372159	3.47199103849089e-06	\\
7932.04900568182	5.74678207241272e-06	\\
7933.02778764205	3.65979753347456e-06	\\
7934.00656960227	3.45351047591823e-06	\\
7934.9853515625	6.89839868215745e-06	\\
7935.96413352273	4.21584423939073e-06	\\
7936.94291548295	5.6882188897147e-06	\\
7937.92169744318	5.30606034973306e-06	\\
7938.90047940341	4.68189517278082e-06	\\
7939.87926136364	4.81250542816228e-06	\\
7940.85804332386	4.82359004478958e-06	\\
7941.83682528409	4.9562631930333e-06	\\
7942.81560724432	4.47000891104082e-06	\\
7943.79438920455	4.1962232950004e-06	\\
7944.77317116477	5.20323473914406e-06	\\
7945.751953125	4.86711435227581e-06	\\
7946.73073508523	5.74875796495384e-06	\\
7947.70951704545	5.86341465025652e-06	\\
7948.68829900568	3.79396494416971e-06	\\
7949.66708096591	4.60056786094266e-06	\\
7950.64586292614	4.31719600537898e-06	\\
7951.62464488636	5.18347723160764e-06	\\
7952.60342684659	5.2056011253162e-06	\\
7953.58220880682	4.50208135373816e-06	\\
7954.56099076705	6.04741935179723e-06	\\
7955.53977272727	4.65446863276287e-06	\\
7956.5185546875	7.58321507548155e-06	\\
7957.49733664773	5.97894527455636e-06	\\
7958.47611860795	5.97861498079418e-06	\\
7959.45490056818	5.94695153095406e-06	\\
7960.43368252841	5.29019053053121e-06	\\
7961.41246448864	4.8752560059011e-06	\\
7962.39124644886	5.53428180174318e-06	\\
7963.37002840909	5.71062640167147e-06	\\
7964.34881036932	5.25356004110589e-06	\\
7965.32759232955	4.07699634148717e-06	\\
7966.30637428977	5.56851302261156e-06	\\
7967.28515625	3.62124553838588e-06	\\
7968.26393821023	6.05922671271326e-06	\\
7969.24272017045	4.98931705154118e-06	\\
7970.22150213068	5.7461723022978e-06	\\
7971.20028409091	4.5260899812153e-06	\\
7972.17906605114	4.39148698028076e-06	\\
7973.15784801136	2.8713896167185e-06	\\
7974.13662997159	3.87346182653758e-06	\\
7975.11541193182	2.92412075139964e-06	\\
7976.09419389205	3.86527045271258e-06	\\
7977.07297585227	4.31960602637649e-06	\\
7978.0517578125	3.63623912354511e-06	\\
7979.03053977273	2.85993725539786e-06	\\
7980.00932173295	3.90932214001025e-06	\\
7980.98810369318	5.24114556742944e-06	\\
7981.96688565341	5.32998241164676e-06	\\
7982.94566761364	4.29583589003192e-06	\\
7983.92444957386	3.84469496522077e-06	\\
7984.90323153409	4.03385837989218e-06	\\
7985.88201349432	3.55400955602508e-06	\\
7986.86079545455	3.76231671906241e-06	\\
7987.83957741477	2.62485076377334e-06	\\
7988.818359375	4.89439975573399e-06	\\
7989.79714133523	3.04753515907302e-06	\\
7990.77592329545	2.29911428519417e-06	\\
7991.75470525568	1.62013376460382e-06	\\
7992.73348721591	3.2881576092079e-06	\\
7993.71226917614	2.83368761816081e-06	\\
7994.69105113636	2.36190814512752e-06	\\
7995.66983309659	4.9056321771598e-06	\\
7996.64861505682	3.28452277579537e-06	\\
7997.62739701705	9.36595472954119e-07	\\
7998.60617897727	3.51130687700943e-06	\\
7999.5849609375	8.93074460766869e-06	\\
8000.56374289773	7.6656306452505e-06	\\
8001.54252485795	2.96897702909853e-06	\\
8002.52130681818	2.79370767648692e-06	\\
8003.50008877841	2.79461876416412e-06	\\
8004.47887073864	2.68807800332974e-06	\\
8005.45765269886	2.90772930253132e-06	\\
8006.43643465909	3.45189065731284e-06	\\
8007.41521661932	2.61774098903825e-06	\\
8008.39399857955	2.83914457312905e-06	\\
8009.37278053977	2.8421961941267e-06	\\
8010.3515625	3.6421803227421e-06	\\
8011.33034446023	2.74824629297583e-06	\\
8012.30912642045	2.99651347399327e-06	\\
8013.28790838068	5.01148420162452e-07	\\
8014.26669034091	3.47179938842379e-06	\\
8015.24547230114	3.28148356521127e-06	\\
8016.22425426136	3.29576548179396e-06	\\
8017.20303622159	3.71679217118054e-06	\\
8018.18181818182	1.99822406918664e-06	\\
8019.16060014205	2.1193750663052e-06	\\
8020.13938210227	4.1427798111212e-06	\\
8021.1181640625	2.32258382408176e-06	\\
8022.09694602273	2.31000285380168e-06	\\
8023.07572798295	2.26994458801907e-06	\\
8024.05450994318	4.63298650397658e-06	\\
8025.03329190341	4.00976615898969e-06	\\
8026.01207386364	2.48634180626028e-06	\\
8026.99085582386	3.13862455826724e-06	\\
8027.96963778409	3.32266222162037e-06	\\
8028.94841974432	5.57586196730591e-06	\\
8029.92720170455	3.31814598757878e-06	\\
8030.90598366477	3.00151777797763e-06	\\
8031.884765625	4.35585284440794e-06	\\
8032.86354758523	2.77731548897425e-06	\\
8033.84232954545	2.53319660567421e-06	\\
8034.82111150568	2.67820757963718e-06	\\
8035.79989346591	2.71342547886521e-06	\\
8036.77867542614	3.8842607456487e-06	\\
8037.75745738636	3.77046479036915e-06	\\
8038.73623934659	3.98179792586316e-06	\\
8039.71502130682	1.83172487242726e-06	\\
8040.69380326705	5.16023572139366e-06	\\
8041.67258522727	4.24008848078004e-06	\\
8042.6513671875	2.66836457621912e-06	\\
8043.63014914773	3.62015081469506e-06	\\
8044.60893110795	5.31436629749645e-06	\\
8045.58771306818	3.31267899886258e-06	\\
8046.56649502841	3.38144435085433e-06	\\
8047.54527698864	3.46838948413013e-06	\\
8048.52405894886	4.11152369179559e-06	\\
8049.50284090909	3.31060849966303e-06	\\
8050.48162286932	2.47204710794422e-06	\\
8051.46040482955	2.92431516485942e-06	\\
8052.43918678977	2.37977928653006e-06	\\
8053.41796875	3.47518171718091e-06	\\
8054.39675071023	3.46726561831839e-06	\\
8055.37553267045	5.01848226886935e-06	\\
8056.35431463068	4.39494156489764e-06	\\
8057.33309659091	4.0947234939761e-06	\\
8058.31187855114	2.63693519284635e-06	\\
8059.29066051136	4.12588677897186e-06	\\
8060.26944247159	3.49785089099633e-06	\\
8061.24822443182	4.737876964188e-06	\\
8062.22700639205	4.05799331151515e-06	\\
8063.20578835227	4.0360142308523e-06	\\
8064.1845703125	3.62799653881025e-06	\\
8065.16335227273	4.3371790006387e-06	\\
8066.14213423295	4.43998216789173e-06	\\
8067.12091619318	3.41235154963275e-06	\\
8068.09969815341	5.41784542570842e-06	\\
8069.07848011364	3.33472684662752e-06	\\
8070.05726207386	5.33454281790793e-06	\\
8071.03604403409	3.97962588853393e-06	\\
8072.01482599432	5.17127529149464e-06	\\
8072.99360795455	4.21782919536291e-06	\\
8073.97238991477	2.53615077419106e-06	\\
8074.951171875	3.12672629885649e-06	\\
8075.92995383523	4.87037274741344e-06	\\
8076.90873579545	3.48418482071997e-06	\\
8077.88751775568	4.45109630089424e-06	\\
8078.86629971591	4.28512116253197e-06	\\
8079.84508167614	3.72673140906615e-06	\\
8080.82386363636	5.38569408524191e-06	\\
8081.80264559659	3.99908885211802e-06	\\
8082.78142755682	4.47579990660129e-06	\\
8083.76020951705	4.86227019578342e-06	\\
8084.73899147727	4.68895536225944e-06	\\
8085.7177734375	3.41175403429316e-06	\\
8086.69655539773	2.94023931046392e-06	\\
8087.67533735795	2.47031396772035e-06	\\
8088.65411931818	5.48866176398775e-06	\\
8089.63290127841	3.09119719494041e-06	\\
8090.61168323864	3.39968552321193e-06	\\
8091.59046519886	1.2169811648049e-06	\\
8092.56924715909	3.65500212990581e-06	\\
8093.54802911932	2.57117860253193e-06	\\
8094.52681107955	3.11647785043075e-06	\\
8095.50559303977	4.32693736370179e-06	\\
8096.484375	4.54354706494647e-06	\\
8097.46315696023	2.68311725433526e-06	\\
8098.44193892045	3.55900693823748e-06	\\
8099.42072088068	2.35616267727136e-06	\\
8100.39950284091	3.59757220760891e-06	\\
8101.37828480114	5.34561805457993e-06	\\
8102.35706676136	4.53256569387676e-06	\\
8103.33584872159	4.32177277983625e-06	\\
8104.31463068182	4.59249608166484e-06	\\
8105.29341264205	1.58903639765489e-06	\\
8106.27219460227	2.97434601929001e-06	\\
8107.2509765625	3.53734145518259e-06	\\
8108.22975852273	3.91199153274808e-06	\\
8109.20854048295	4.62189074947038e-06	\\
8110.18732244318	4.61152077680146e-06	\\
8111.16610440341	3.61809230501143e-06	\\
8112.14488636364	5.47075308283436e-06	\\
8113.12366832386	3.78252257721143e-06	\\
8114.10245028409	4.94123768592745e-06	\\
8115.08123224432	3.65831295706154e-06	\\
8116.06001420455	4.56468002852782e-06	\\
8117.03879616477	2.80033049212405e-06	\\
8118.017578125	2.91827364758733e-06	\\
8118.99636008523	1.9981601911624e-06	\\
8119.97514204545	3.28496995589459e-06	\\
8120.95392400568	3.86380343944485e-06	\\
8121.93270596591	2.12715196080639e-06	\\
8122.91148792614	1.95606479194772e-06	\\
8123.89026988636	2.31541126323269e-06	\\
8124.86905184659	2.11269274639469e-06	\\
8125.84783380682	4.04985957142157e-06	\\
8126.82661576705	3.47084219267295e-06	\\
8127.80539772727	3.56789256857298e-06	\\
8128.7841796875	4.55424274749987e-06	\\
8129.76296164773	4.49954665366131e-06	\\
8130.74174360795	3.55470958133233e-06	\\
8131.72052556818	4.43007612673432e-06	\\
8132.69930752841	4.26554101180591e-06	\\
8133.67808948864	2.9256057156691e-06	\\
8134.65687144886	4.28180622570597e-06	\\
8135.63565340909	2.06993158620054e-06	\\
8136.61443536932	3.32060156021312e-06	\\
8137.59321732955	3.65002998558843e-06	\\
8138.57199928977	3.93610797872437e-06	\\
8139.55078125	3.05544146715157e-06	\\
8140.52956321023	4.90279037917904e-06	\\
8141.50834517045	4.95749134587162e-06	\\
8142.48712713068	4.05508888282349e-06	\\
8143.46590909091	3.15770775634447e-06	\\
8144.44469105114	3.80171884922394e-06	\\
8145.42347301136	4.05106088345693e-06	\\
8146.40225497159	3.98427655642898e-06	\\
8147.38103693182	2.72863227980132e-06	\\
8148.35981889205	4.93734248994016e-06	\\
8149.33860085227	3.99358080300891e-06	\\
8150.3173828125	4.551939481779e-06	\\
8151.29616477273	3.04796799339356e-06	\\
8152.27494673295	4.00731977527929e-06	\\
8153.25372869318	3.35168028844656e-06	\\
8154.23251065341	7.15556613928182e-06	\\
8155.21129261364	5.66137681378543e-06	\\
8156.19007457386	6.04053946393796e-06	\\
8157.16885653409	2.50223017523717e-06	\\
8158.14763849432	3.10317244257895e-06	\\
8159.12642045455	3.68006237873378e-06	\\
8160.10520241477	5.68304052966202e-06	\\
8161.083984375	4.1994643118067e-06	\\
8162.06276633523	3.83073024001608e-06	\\
8163.04154829545	5.29633353876296e-06	\\
8164.02033025568	4.59609033914816e-06	\\
8164.99911221591	3.35568928212556e-06	\\
8165.97789417614	3.51286321818986e-06	\\
8166.95667613636	4.30535128041952e-06	\\
8167.93545809659	5.21642589459192e-06	\\
8168.91424005682	3.95431380793079e-06	\\
8169.89302201705	4.01508888143297e-06	\\
8170.87180397727	3.59150349721012e-06	\\
8171.8505859375	6.118269058491e-06	\\
8172.82936789773	4.26386116477686e-06	\\
8173.80814985795	4.53828799644435e-06	\\
8174.78693181818	4.40982060775676e-06	\\
8175.76571377841	4.30291502928423e-06	\\
8176.74449573864	4.71609218496146e-06	\\
8177.72327769886	5.51687825000007e-06	\\
8178.70205965909	3.37374518709655e-06	\\
8179.68084161932	4.11568346194605e-06	\\
8180.65962357955	4.63841444638597e-06	\\
8181.63840553977	4.1909045646948e-06	\\
8182.6171875	5.31565783557162e-06	\\
8183.59596946023	3.02495949643541e-06	\\
8184.57475142045	3.56333660228709e-06	\\
8185.55353338068	4.67511037122568e-06	\\
8186.53231534091	4.38117847901353e-06	\\
8187.51109730114	4.06412848901452e-06	\\
8188.48987926136	4.20786884875947e-06	\\
8189.46866122159	4.00440005064646e-06	\\
8190.44744318182	5.08874300248422e-06	\\
8191.42622514205	3.0699575478541e-06	\\
8192.40500710227	3.9923671664337e-06	\\
8193.3837890625	5.36744090518927e-06	\\
8194.36257102273	3.64126489131634e-06	\\
8195.34135298295	3.91364691425224e-06	\\
8196.32013494318	3.9493133203462e-06	\\
8197.29891690341	4.73764689294427e-06	\\
8198.27769886364	4.87200027312989e-06	\\
8199.25648082386	3.99088124799845e-06	\\
8200.23526278409	5.47829737998736e-06	\\
8201.21404474432	4.55807003376236e-06	\\
8202.19282670454	5.16216927257434e-06	\\
8203.17160866477	3.42256208441359e-06	\\
8204.150390625	2.13057339747495e-06	\\
8205.12917258523	4.20036467115612e-06	\\
8206.10795454545	4.72534938401061e-06	\\
8207.08673650568	5.34374142014117e-06	\\
8208.06551846591	4.20594809793508e-06	\\
8209.04430042614	4.04892894114637e-06	\\
8210.02308238636	3.71999690419774e-06	\\
8211.00186434659	5.12120382203815e-06	\\
8211.98064630682	4.09122431671904e-06	\\
8212.95942826704	3.3210031785172e-06	\\
8213.93821022727	5.0145194154808e-06	\\
8214.9169921875	4.68511687076236e-06	\\
8215.89577414773	4.12331185012413e-06	\\
8216.87455610795	5.24944279038826e-06	\\
8217.85333806818	4.58504476595839e-06	\\
8218.83212002841	4.37040324658822e-06	\\
8219.81090198864	3.77086686424597e-06	\\
8220.78968394886	3.95181047535169e-06	\\
8221.76846590909	4.94850446486526e-06	\\
8222.74724786932	4.16851133436525e-06	\\
8223.72602982954	5.23614413140266e-06	\\
8224.70481178977	4.4326445030219e-06	\\
8225.68359375	4.1670709284116e-06	\\
8226.66237571023	4.12846960540697e-06	\\
8227.64115767045	4.2620020741042e-06	\\
8228.61993963068	4.62486807739592e-06	\\
8229.59872159091	3.97710570208932e-06	\\
8230.57750355114	4.80461072737684e-06	\\
8231.55628551136	3.66471736858576e-06	\\
8232.53506747159	2.89991451136236e-06	\\
8233.51384943182	3.06516538720431e-06	\\
8234.49263139204	6.03714486525444e-06	\\
8235.47141335227	2.63351763875217e-06	\\
8236.4501953125	3.11080297632373e-06	\\
8237.42897727273	2.97333497500662e-06	\\
8238.40775923295	3.74053160856989e-06	\\
8239.38654119318	4.24481379577595e-06	\\
8240.36532315341	3.1437996598493e-06	\\
8241.34410511364	4.16909225441466e-06	\\
8242.32288707386	3.35841779872415e-06	\\
8243.30166903409	3.0830841436991e-06	\\
8244.28045099432	4.10830751536057e-06	\\
8245.25923295454	4.70492957312932e-06	\\
8246.23801491477	2.91174343403663e-06	\\
8247.216796875	3.08190632597718e-06	\\
8248.19557883523	2.72385387087176e-06	\\
8249.17436079545	2.45047547978565e-06	\\
8250.15314275568	1.44810896206441e-06	\\
8251.13192471591	4.79884481447568e-06	\\
8252.11070667614	4.46979008228949e-06	\\
8253.08948863636	2.66844277519147e-06	\\
8254.06827059659	4.0243374696911e-06	\\
8255.04705255682	4.60284551985926e-06	\\
8256.02583451704	2.28585074597904e-06	\\
8257.00461647727	4.68273116855704e-06	\\
8257.9833984375	4.61064678728154e-06	\\
8258.96218039773	3.45730389660909e-06	\\
8259.94096235795	3.96671305375378e-06	\\
8260.91974431818	2.66166726842328e-06	\\
8261.89852627841	4.36329687450477e-06	\\
8262.87730823864	2.53219133894125e-06	\\
8263.85609019886	3.34125399043517e-06	\\
8264.83487215909	2.68883070016863e-06	\\
8265.81365411932	3.34386389519262e-06	\\
8266.79243607954	3.38100995526953e-06	\\
8267.77121803977	3.47412006639646e-06	\\
8268.75	3.65938337503455e-06	\\
8269.72878196023	3.98295003574514e-06	\\
8270.70756392045	2.18234396759037e-06	\\
8271.68634588068	2.59382236076687e-06	\\
8272.66512784091	3.8741078118355e-06	\\
8273.64390980114	4.42446730095757e-06	\\
8274.62269176136	3.42214490973501e-06	\\
8275.60147372159	3.93477129592653e-06	\\
8276.58025568182	3.91077635439863e-06	\\
8277.55903764204	3.37698502175783e-06	\\
8278.53781960227	4.83241088552398e-06	\\
8279.5166015625	4.90071420116844e-06	\\
8280.49538352273	3.7306728169535e-06	\\
8281.47416548295	4.54360232545101e-06	\\
8282.45294744318	2.24749542352369e-06	\\
8283.43172940341	4.4825937642362e-06	\\
8284.41051136364	3.73212784014146e-06	\\
8285.38929332386	2.97048336444056e-06	\\
8286.36807528409	2.8545564396732e-06	\\
8287.34685724432	4.50319886487694e-06	\\
8288.32563920454	3.21859907686732e-06	\\
8289.30442116477	2.68826067696052e-06	\\
8290.283203125	4.06622761520832e-06	\\
8291.26198508523	4.38028152210444e-06	\\
8292.24076704545	3.75559809870672e-06	\\
8293.21954900568	3.7142428876889e-06	\\
8294.19833096591	2.38525038160611e-06	\\
8295.17711292614	3.05610029741626e-06	\\
8296.15589488636	3.70789471477184e-06	\\
8297.13467684659	3.39175323089224e-06	\\
8298.11345880682	3.50374705850787e-06	\\
8299.09224076704	3.12944184614309e-06	\\
8300.07102272727	3.31784423058742e-06	\\
8301.0498046875	5.00887698627318e-06	\\
8302.02858664773	3.34280771820485e-06	\\
8303.00736860795	3.7828635357981e-06	\\
8303.98615056818	4.45079677440787e-06	\\
8304.96493252841	3.53393825227458e-06	\\
8305.94371448864	3.27702048831936e-06	\\
8306.92249644886	3.44059688750784e-06	\\
8307.90127840909	3.71129941887126e-06	\\
8308.88006036932	2.0773052319464e-06	\\
8309.85884232954	2.77693137478642e-06	\\
8310.83762428977	3.6929399937579e-06	\\
8311.81640625	4.42405879001193e-06	\\
8312.79518821023	3.00705304256692e-06	\\
8313.77397017045	3.42414993241669e-06	\\
8314.75275213068	5.159708403271e-06	\\
8315.73153409091	3.77856767488408e-06	\\
8316.71031605114	2.07161112027053e-06	\\
8317.68909801136	5.46227739703373e-06	\\
8318.66787997159	3.83605382230752e-06	\\
8319.64666193182	3.62921663088758e-06	\\
8320.62544389204	1.78246677589139e-06	\\
8321.60422585227	4.87235209094209e-06	\\
8322.5830078125	4.20032470506118e-06	\\
8323.56178977273	3.43419103984659e-06	\\
8324.54057173295	3.55392861217828e-06	\\
8325.51935369318	3.92432170175196e-06	\\
8326.49813565341	3.39833758915808e-06	\\
8327.47691761364	2.78628690471846e-06	\\
8328.45569957386	4.99693153705762e-06	\\
8329.43448153409	4.25556228228178e-06	\\
8330.41326349432	4.33640291617984e-06	\\
8331.39204545454	4.27735069895042e-06	\\
8332.37082741477	3.91840773945176e-06	\\
8333.349609375	3.87904143927563e-06	\\
8334.32839133523	3.63614298684012e-06	\\
8335.30717329545	4.47899233027712e-06	\\
8336.28595525568	4.4353795054887e-06	\\
8337.26473721591	3.96906578971154e-06	\\
8338.24351917614	3.36124160414768e-06	\\
8339.22230113636	3.63120241477459e-06	\\
8340.20108309659	5.01856787153624e-06	\\
8341.17986505682	5.72077272624378e-06	\\
8342.15864701704	4.57385555434247e-06	\\
8343.13742897727	4.3086171549939e-06	\\
8344.1162109375	2.25345131709488e-06	\\
8345.09499289773	5.13151311098645e-06	\\
8346.07377485795	2.80134149958677e-06	\\
8347.05255681818	3.89596060796767e-06	\\
8348.03133877841	2.72787713193264e-06	\\
8349.01012073864	4.37123144745623e-06	\\
8349.98890269886	4.78992672196626e-06	\\
8350.96768465909	4.52249897361583e-06	\\
8351.94646661932	4.18550099695824e-06	\\
8352.92524857954	4.22844193040556e-06	\\
8353.90403053977	5.75080628631169e-06	\\
8354.8828125	4.0458963739989e-06	\\
8355.86159446023	3.70302499306216e-06	\\
8356.84037642045	3.51006936440622e-06	\\
8357.81915838068	2.8134758324076e-06	\\
8358.79794034091	4.09270263297388e-06	\\
8359.77672230114	2.6979625145736e-06	\\
8360.75550426136	3.15169743582535e-06	\\
8361.73428622159	2.31539728966751e-06	\\
8362.71306818182	3.68102762804321e-06	\\
8363.69185014204	4.24474142752469e-06	\\
8364.67063210227	5.17028795368443e-06	\\
8365.6494140625	2.6996666364539e-06	\\
8366.62819602273	4.10708370755097e-06	\\
8367.60697798295	4.42741677105964e-06	\\
8368.58575994318	4.85267005839422e-06	\\
8369.56454190341	3.87818909233984e-06	\\
8370.54332386364	3.74281212892694e-06	\\
8371.52210582386	2.71531232091999e-06	\\
8372.50088778409	4.7599485553678e-06	\\
8373.47966974432	4.9555934817398e-06	\\
8374.45845170454	3.69821317200647e-06	\\
8375.43723366477	3.49086510343283e-06	\\
8376.416015625	4.15782700868712e-06	\\
8377.39479758523	4.4008709709293e-06	\\
8378.37357954545	4.03270990920717e-06	\\
8379.35236150568	4.02199102004621e-06	\\
8380.33114346591	2.18115659826215e-06	\\
8381.30992542614	3.33360315829038e-06	\\
8382.28870738636	4.15252468788738e-06	\\
8383.26748934659	3.75914298724034e-06	\\
8384.24627130682	2.28338505882279e-06	\\
8385.22505326704	4.96872116478368e-06	\\
8386.20383522727	3.7156132539024e-06	\\
8387.1826171875	3.16458083885723e-06	\\
8388.16139914773	3.83351878075508e-06	\\
8389.14018110795	3.43705230420214e-06	\\
8390.11896306818	3.96348340051255e-06	\\
8391.09774502841	3.22924417518071e-06	\\
8392.07652698864	4.11905888044e-06	\\
8393.05530894886	3.30777327959902e-06	\\
8394.03409090909	3.48204781299924e-06	\\
8395.01287286932	3.53732602722362e-06	\\
8395.99165482954	4.45639850498184e-06	\\
8396.97043678977	2.50333035720446e-06	\\
8397.94921875	4.23657357762497e-06	\\
8398.92800071023	4.12955819150867e-06	\\
8399.90678267045	5.50941762384108e-06	\\
8400.88556463068	3.85987937691385e-06	\\
8401.86434659091	3.95962459837941e-06	\\
8402.84312855114	3.32914699455622e-06	\\
8403.82191051136	4.37752055508318e-06	\\
8404.80069247159	3.38868299515085e-06	\\
8405.77947443182	2.96274188737983e-06	\\
8406.75825639204	4.55567528943941e-06	\\
8407.73703835227	5.13564348102845e-06	\\
8408.7158203125	3.73038514774661e-06	\\
8409.69460227273	5.14246252410844e-06	\\
8410.67338423295	3.46297628790418e-06	\\
8411.65216619318	3.06530163967231e-06	\\
8412.63094815341	3.83495592592134e-06	\\
8413.60973011364	3.91932701955495e-06	\\
8414.58851207386	4.02415665955858e-06	\\
8415.56729403409	3.85281369869656e-06	\\
8416.54607599432	1.79449653251047e-06	\\
8417.52485795454	2.68094446460607e-06	\\
8418.50363991477	3.84606744030403e-06	\\
8419.482421875	3.97500516933704e-06	\\
8420.46120383523	4.18372707195538e-06	\\
8421.43998579545	3.59238523554745e-06	\\
8422.41876775568	2.38844510328113e-06	\\
8423.39754971591	4.07926009214457e-06	\\
8424.37633167614	4.03085583592749e-06	\\
8425.35511363636	2.80623936739656e-06	\\
8426.33389559659	2.7069144701764e-06	\\
8427.31267755682	2.92097022482183e-06	\\
8428.29145951704	4.61456000463167e-06	\\
8429.27024147727	3.60571147228688e-06	\\
8430.2490234375	2.67818309598508e-06	\\
8431.22780539773	2.16634633242642e-06	\\
8432.20658735795	3.22129074396842e-06	\\
8433.18536931818	3.32023050515714e-06	\\
8434.16415127841	3.33724152046893e-06	\\
8435.14293323864	2.74383837333887e-06	\\
8436.12171519886	4.75027130366023e-06	\\
8437.10049715909	5.24587348613625e-06	\\
8438.07927911932	3.60446269594013e-06	\\
8439.05806107954	2.67667815131344e-06	\\
8440.03684303977	3.62400961827291e-06	\\
8441.015625	3.33285135861253e-06	\\
8441.99440696023	4.167657904767e-06	\\
8442.97318892045	3.55788401926458e-06	\\
8443.95197088068	3.99071284092812e-06	\\
8444.93075284091	3.22237520820792e-06	\\
8445.90953480114	2.05941461376731e-06	\\
8446.88831676136	4.15708393189235e-06	\\
8447.86709872159	3.64334949257105e-06	\\
8448.84588068182	4.6354189321211e-06	\\
8449.82466264204	3.74853157980745e-06	\\
8450.80344460227	2.06515608856371e-06	\\
8451.7822265625	4.04621617910954e-06	\\
8452.76100852273	2.92633503725563e-06	\\
8453.73979048295	2.51943557070209e-06	\\
8454.71857244318	3.27428091813218e-06	\\
8455.69735440341	2.19221235941894e-06	\\
8456.67613636364	2.2732895318113e-06	\\
8457.65491832386	3.08173445725528e-06	\\
8458.63370028409	2.18217934398865e-06	\\
8459.61248224432	3.0721812643043e-06	\\
8460.59126420454	3.0223218682101e-06	\\
8461.57004616477	4.3109511018563e-06	\\
8462.548828125	3.85306373472417e-06	\\
8463.52761008523	2.53301171621846e-06	\\
8464.50639204545	3.4549909915899e-06	\\
8465.48517400568	3.90014374553533e-06	\\
8466.46395596591	2.49163749225215e-06	\\
8467.44273792614	3.28608711549321e-06	\\
8468.42151988636	4.4195275694735e-06	\\
8469.40030184659	3.19178491362545e-06	\\
8470.37908380682	3.00070582597243e-06	\\
8471.35786576704	4.28102313399466e-06	\\
8472.33664772727	4.41179131715913e-06	\\
8473.3154296875	4.73748471302194e-06	\\
8474.29421164773	2.41445450982072e-06	\\
8475.27299360795	4.00106014975804e-06	\\
8476.25177556818	3.40028002571498e-06	\\
8477.23055752841	2.71719986809223e-06	\\
8478.20933948864	3.50609710740318e-06	\\
8479.18812144886	2.95331306574765e-06	\\
8480.16690340909	3.28379835800295e-06	\\
8481.14568536932	4.10993729452668e-06	\\
8482.12446732954	2.82133440901103e-06	\\
8483.10324928977	4.16290571655629e-06	\\
8484.08203125	2.94719220236785e-06	\\
8485.06081321023	3.68871070140109e-06	\\
8486.03959517045	3.70148945637796e-06	\\
8487.01837713068	4.00718024139016e-06	\\
8487.99715909091	2.92088261826921e-06	\\
8488.97594105114	2.53938622527093e-06	\\
8489.95472301136	3.30572146040899e-06	\\
8490.93350497159	3.00464729658534e-06	\\
8491.91228693182	3.05591195670661e-06	\\
8492.89106889204	2.55700181159241e-06	\\
8493.86985085227	4.75739230000583e-06	\\
8494.8486328125	4.33780823568058e-06	\\
8495.82741477273	3.86363013583024e-06	\\
8496.80619673295	3.40632205709469e-06	\\
8497.78497869318	3.49153953547497e-06	\\
8498.76376065341	2.65170424238405e-06	\\
8499.74254261364	4.29328533728939e-06	\\
8500.72132457386	4.09435391526558e-06	\\
8501.70010653409	3.18483569504692e-06	\\
8502.67888849432	4.34668536513956e-06	\\
8503.65767045454	4.18255072356627e-06	\\
8504.63645241477	4.89633047591287e-06	\\
8505.615234375	3.14029270998502e-06	\\
8506.59401633523	4.34690548892798e-06	\\
8507.57279829545	4.24986817002066e-06	\\
8508.55158025568	4.55539890505747e-06	\\
8509.53036221591	3.0194846355406e-06	\\
8510.50914417614	5.46796599696347e-06	\\
8511.48792613636	3.95727952118881e-06	\\
8512.46670809659	3.91651643919912e-06	\\
8513.44549005682	2.54751807541059e-06	\\
8514.42427201704	5.21202408471458e-06	\\
8515.40305397727	4.42974249972951e-06	\\
8516.3818359375	4.76215499634798e-06	\\
8517.36061789773	4.38715793850193e-06	\\
8518.33939985795	4.23233200717902e-06	\\
8519.31818181818	3.82606876687092e-06	\\
8520.29696377841	3.97205765531748e-06	\\
8521.27574573864	3.55435068739517e-06	\\
8522.25452769886	4.77484109951697e-06	\\
8523.23330965909	2.82692331814064e-06	\\
8524.21209161932	4.82480802693169e-06	\\
8525.19087357954	2.58749086069944e-06	\\
8526.16965553977	3.00221415041378e-06	\\
8527.1484375	3.52760337331994e-06	\\
8528.12721946023	4.509813997395e-06	\\
8529.10600142045	4.37121932788782e-06	\\
8530.08478338068	5.12660960705058e-06	\\
8531.06356534091	3.9219255706459e-06	\\
8532.04234730114	4.0751911518794e-06	\\
8533.02112926136	4.71614451995033e-06	\\
8533.99991122159	3.98353114443226e-06	\\
8534.97869318182	2.98645990742925e-06	\\
8535.95747514204	3.51818172265178e-06	\\
8536.93625710227	4.55548591539163e-06	\\
8537.9150390625	3.09072286123444e-06	\\
8538.89382102273	3.25406828972341e-06	\\
8539.87260298295	3.74904546375019e-06	\\
8540.85138494318	3.4691984788449e-06	\\
8541.83016690341	5.07420841171186e-06	\\
8542.80894886364	4.07349492140676e-06	\\
8543.78773082386	3.27533463420181e-06	\\
8544.76651278409	4.94666809187827e-06	\\
8545.74529474432	2.18450356195563e-06	\\
8546.72407670454	2.85545611044929e-06	\\
8547.70285866477	3.20166927893619e-06	\\
8548.681640625	1.66207146206088e-06	\\
8549.66042258523	2.20972397395324e-06	\\
8550.63920454545	2.98421491873844e-06	\\
};
\addplot [color=blue,solid,forget plot]
  table[row sep=crcr]{
8550.63920454545	2.98421491873844e-06	\\
8551.61798650568	5.06146357010971e-06	\\
8552.59676846591	3.38063765383325e-06	\\
8553.57555042614	1.90245573644042e-06	\\
8554.55433238636	5.1461016006667e-06	\\
8555.53311434659	3.68933804179216e-06	\\
8556.51189630682	4.08818989605861e-06	\\
8557.49067826704	2.42885446582145e-06	\\
8558.46946022727	3.82852590967594e-06	\\
8559.4482421875	3.38941677230627e-06	\\
8560.42702414773	4.2432045242635e-06	\\
8561.40580610795	3.32684500255721e-06	\\
8562.38458806818	3.23393859578716e-06	\\
8563.36337002841	3.93955537125441e-06	\\
8564.34215198864	3.41191333346786e-06	\\
8565.32093394886	4.30182932395546e-06	\\
8566.29971590909	4.25807784951737e-06	\\
8567.27849786932	3.14230050872903e-06	\\
8568.25727982954	2.90126282665625e-06	\\
8569.23606178977	2.74717405772727e-06	\\
8570.21484375	3.42680261854997e-06	\\
8571.19362571023	4.08818677889495e-06	\\
8572.17240767045	3.48348654613146e-06	\\
8573.15118963068	2.94200497530058e-06	\\
8574.12997159091	3.08465812175217e-06	\\
8575.10875355114	3.55167503806865e-06	\\
8576.08753551136	2.76738650945508e-06	\\
8577.06631747159	3.0659997555447e-06	\\
8578.04509943182	3.98304050465736e-06	\\
8579.02388139204	3.00365611816968e-06	\\
8580.00266335227	4.23792432624831e-06	\\
8580.9814453125	4.05611994452719e-06	\\
8581.96022727273	2.33762875242731e-06	\\
8582.93900923295	3.66666993308931e-06	\\
8583.91779119318	2.76262726876622e-06	\\
8584.89657315341	3.52119880959946e-06	\\
8585.87535511364	3.10225523919898e-06	\\
8586.85413707386	2.95938859284197e-06	\\
8587.83291903409	2.64461383619663e-06	\\
8588.81170099432	3.28889218886515e-06	\\
8589.79048295454	3.39689025096771e-06	\\
8590.76926491477	3.00239140043313e-06	\\
8591.748046875	3.9861739700561e-06	\\
8592.72682883523	3.70161626284122e-06	\\
8593.70561079545	2.28214917617834e-06	\\
8594.68439275568	3.27812104240706e-06	\\
8595.66317471591	4.96045631420238e-06	\\
8596.64195667614	2.79802184845313e-06	\\
8597.62073863636	2.76266343310021e-06	\\
8598.59952059659	1.7071913807e-06	\\
8599.57830255682	3.31211644809859e-06	\\
8600.55708451704	2.93489239577708e-06	\\
8601.53586647727	3.04513262713238e-06	\\
8602.5146484375	2.87897318638937e-06	\\
8603.49343039773	3.94708399453755e-06	\\
8604.47221235795	2.4216297989274e-06	\\
8605.45099431818	3.29326977094098e-06	\\
8606.42977627841	2.55742423050093e-06	\\
8607.40855823864	2.18837883667202e-06	\\
8608.38734019886	2.32381052313553e-06	\\
8609.36612215909	2.79902019719089e-06	\\
8610.34490411932	2.93107215490785e-06	\\
8611.32368607954	3.31667110131612e-06	\\
8612.30246803977	2.95750755549884e-06	\\
8613.28125	3.65392424499478e-06	\\
8614.26003196023	1.27259265427105e-06	\\
8615.23881392045	3.47167396240222e-06	\\
8616.21759588068	3.12585818398557e-06	\\
8617.19637784091	2.09593072058824e-06	\\
8618.17515980114	3.74823460838237e-06	\\
8619.15394176136	2.64267136584899e-06	\\
8620.13272372159	3.43741570532331e-06	\\
8621.11150568182	1.77021669807264e-06	\\
8622.09028764204	3.28602307978567e-06	\\
8623.06906960227	4.11049093180834e-06	\\
8624.0478515625	1.66665273984968e-06	\\
8625.02663352273	3.59684753022792e-06	\\
8626.00541548295	2.28430798330806e-06	\\
8626.98419744318	1.80195375432261e-06	\\
8627.96297940341	1.47939043899555e-06	\\
8628.94176136364	2.73681013967799e-06	\\
8629.92054332386	3.41176225367732e-06	\\
8630.89932528409	3.56248707536917e-06	\\
8631.87810724432	2.49639934384753e-06	\\
8632.85688920454	3.9927999358026e-06	\\
8633.83567116477	2.66889707303301e-06	\\
8634.814453125	2.73585283645849e-06	\\
8635.79323508523	2.80664704261669e-06	\\
8636.77201704545	3.90484157194444e-06	\\
8637.75079900568	2.05377631664665e-06	\\
8638.72958096591	3.82970624660038e-06	\\
8639.70836292614	2.37400192414492e-06	\\
8640.68714488636	2.36108018195528e-06	\\
8641.66592684659	2.86321402328804e-06	\\
8642.64470880682	2.86052149540327e-06	\\
8643.62349076704	1.5773430957681e-06	\\
8644.60227272727	2.42900961985154e-06	\\
8645.5810546875	3.09769627229872e-06	\\
8646.55983664773	2.86277442015884e-06	\\
8647.53861860795	3.45747288509824e-06	\\
8648.51740056818	4.21578232273827e-06	\\
8649.49618252841	2.3877563412989e-06	\\
8650.47496448864	2.8373054945434e-06	\\
8651.45374644886	2.16292573074131e-06	\\
8652.43252840909	4.42669663721081e-06	\\
8653.41131036932	2.99439896452848e-06	\\
8654.39009232954	3.13756403820467e-06	\\
8655.36887428977	2.13002433709377e-06	\\
8656.34765625	2.02914690698014e-06	\\
8657.32643821023	3.11889039798168e-06	\\
8658.30522017045	8.22159070470364e-07	\\
8659.28400213068	2.59552482185387e-06	\\
8660.26278409091	2.48335243557367e-06	\\
8661.24156605114	1.62020492069873e-06	\\
8662.22034801136	2.24672937750524e-06	\\
8663.19912997159	2.56029766218544e-06	\\
8664.17791193182	3.09977042294155e-06	\\
8665.15669389204	2.93173085757296e-06	\\
8666.13547585227	2.89365191241982e-06	\\
8667.1142578125	2.26660714232643e-06	\\
8668.09303977273	2.55066683273874e-06	\\
8669.07182173295	2.57991036398732e-06	\\
8670.05060369318	3.15772910575126e-06	\\
8671.02938565341	2.44976935583511e-06	\\
8672.00816761364	2.81452160124163e-06	\\
8672.98694957386	2.07247586291306e-06	\\
8673.96573153409	3.27384449011975e-06	\\
8674.94451349432	3.12972792072986e-06	\\
8675.92329545454	2.73822387076835e-06	\\
8676.90207741477	3.83144371511104e-06	\\
8677.880859375	2.67398215817645e-06	\\
8678.85964133523	2.24359944090606e-06	\\
8679.83842329545	2.4966678792265e-06	\\
8680.81720525568	2.58360711790541e-06	\\
8681.79598721591	2.49679043481926e-06	\\
8682.77476917614	2.05613823687218e-06	\\
8683.75355113636	3.07419665041785e-06	\\
8684.73233309659	3.18576894487291e-06	\\
8685.71111505682	2.63669627400646e-06	\\
8686.68989701704	3.50610465611466e-06	\\
8687.66867897727	3.48049175995858e-06	\\
8688.6474609375	1.24106515651506e-06	\\
8689.62624289773	2.34395811705805e-06	\\
8690.60502485795	3.17971392170209e-06	\\
8691.58380681818	2.97490703133594e-06	\\
8692.56258877841	1.74697971271583e-06	\\
8693.54137073864	2.77894580645474e-06	\\
8694.52015269886	2.31254269898072e-06	\\
8695.49893465909	2.45347336882115e-06	\\
8696.47771661932	2.39486608798948e-06	\\
8697.45649857954	2.01214244822957e-06	\\
8698.43528053977	3.07550529445477e-06	\\
8699.4140625	2.26193080195523e-06	\\
8700.39284446023	3.68944423738639e-06	\\
8701.37162642045	2.8645033645048e-06	\\
8702.35040838068	3.60565079855287e-06	\\
8703.32919034091	1.67805675585822e-06	\\
8704.30797230114	1.21976262367136e-06	\\
8705.28675426136	1.89889214472334e-06	\\
8706.26553622159	1.86678606969537e-06	\\
8707.24431818182	2.99166379439728e-06	\\
8708.22310014204	3.06009588353424e-06	\\
8709.20188210227	2.42224864029916e-06	\\
8710.1806640625	3.00986830639377e-06	\\
8711.15944602273	2.62494080873288e-06	\\
8712.13822798295	3.39740230448375e-06	\\
8713.11700994318	2.67504420011627e-06	\\
8714.09579190341	2.84432873513897e-06	\\
8715.07457386364	2.73121666263706e-06	\\
8716.05335582386	1.6100101286333e-06	\\
8717.03213778409	4.05582528343673e-06	\\
8718.01091974432	1.13990905436083e-06	\\
8718.98970170454	3.24733743132851e-06	\\
8719.96848366477	3.14639964341097e-06	\\
8720.947265625	2.31227438015978e-06	\\
8721.92604758523	2.87750930836962e-06	\\
8722.90482954545	2.28590877211032e-06	\\
8723.88361150568	2.01357228227019e-06	\\
8724.86239346591	3.42031001044256e-06	\\
8725.84117542614	3.40586594823726e-06	\\
8726.81995738636	1.99766350450912e-06	\\
8727.79873934659	2.47437953741367e-06	\\
8728.77752130682	2.78572634403114e-06	\\
8729.75630326704	3.70245468932601e-06	\\
8730.73508522727	2.40260361345862e-06	\\
8731.7138671875	3.18728174127853e-06	\\
8732.69264914773	2.39959693813504e-06	\\
8733.67143110795	1.31390533577988e-06	\\
8734.65021306818	2.78443761779356e-06	\\
8735.62899502841	3.56118891121807e-06	\\
8736.60777698864	3.1462545120627e-06	\\
8737.58655894886	1.32781061859003e-06	\\
8738.56534090909	2.11606409128068e-06	\\
8739.54412286932	2.08129408383322e-06	\\
8740.52290482954	2.48839854393014e-06	\\
8741.50168678977	2.9344563602255e-06	\\
8742.48046875	2.55724205425525e-06	\\
8743.45925071023	2.33715795696452e-06	\\
8744.43803267045	3.03480942785112e-06	\\
8745.41681463068	1.90735935131734e-06	\\
8746.39559659091	2.01984985118422e-06	\\
8747.37437855114	3.65223851306969e-06	\\
8748.35316051136	1.98020848552783e-06	\\
8749.33194247159	3.50494856775953e-06	\\
8750.31072443182	4.39855941203281e-07	\\
8751.28950639204	2.6278163600248e-06	\\
8752.26828835227	2.36068564566267e-06	\\
8753.2470703125	2.44739622058649e-06	\\
8754.22585227273	3.13652799161212e-06	\\
8755.20463423295	2.04285467215576e-06	\\
8756.18341619318	2.37655507679856e-06	\\
8757.16219815341	2.26474464883409e-06	\\
8758.14098011364	3.13873809450203e-06	\\
8759.11976207386	2.34560177800176e-06	\\
8760.09854403409	3.93965352450115e-06	\\
8761.07732599432	4.3989538300432e-06	\\
8762.05610795454	2.00684143503431e-06	\\
8763.03488991477	1.84163331569974e-06	\\
8764.013671875	3.32551083495277e-06	\\
8764.99245383523	2.69362128586215e-06	\\
8765.97123579545	2.59373299222188e-06	\\
8766.95001775568	2.6214185056095e-06	\\
8767.92879971591	2.58415208596026e-06	\\
8768.90758167614	2.98530169243893e-06	\\
8769.88636363636	3.22278451358341e-06	\\
8770.86514559659	3.72461117536113e-06	\\
8771.84392755682	3.08089169908627e-06	\\
8772.82270951704	2.62803562737369e-06	\\
8773.80149147727	2.96546876780736e-06	\\
8774.7802734375	3.1224020170324e-06	\\
8775.75905539773	2.99917504169805e-06	\\
8776.73783735795	3.45519194863692e-06	\\
8777.71661931818	1.45374354678058e-06	\\
8778.69540127841	2.66455906180417e-06	\\
8779.67418323864	1.43009428381599e-06	\\
8780.65296519886	2.68212964856617e-06	\\
8781.63174715909	2.9660119547303e-06	\\
8782.61052911932	2.16114928248351e-06	\\
8783.58931107954	2.25200342196874e-06	\\
8784.56809303977	2.286617556116e-06	\\
8785.546875	2.96574671783003e-06	\\
8786.52565696023	2.43942778709396e-06	\\
8787.50443892045	1.73447474892679e-06	\\
8788.48322088068	3.26391116000665e-06	\\
8789.46200284091	2.37020560930862e-06	\\
8790.44078480114	1.18237758274433e-06	\\
8791.41956676136	2.53377693158204e-06	\\
8792.39834872159	2.07115346116652e-06	\\
8793.37713068182	2.10340118894317e-06	\\
8794.35591264204	2.83731814644813e-06	\\
8795.33469460227	2.42460568038357e-06	\\
8796.3134765625	2.60601268486776e-06	\\
8797.29225852273	1.48948858265697e-06	\\
8798.27104048295	2.40587049241343e-06	\\
8799.24982244318	3.03717722307558e-06	\\
8800.22860440341	3.34101137139337e-06	\\
8801.20738636364	3.04815161246522e-06	\\
8802.18616832386	2.69452927517449e-06	\\
8803.16495028409	2.16996278161085e-06	\\
8804.14373224432	2.45265058796585e-06	\\
8805.12251420454	2.6808117963073e-06	\\
8806.10129616477	3.75181182757445e-06	\\
8807.080078125	2.62095864964104e-06	\\
8808.05886008523	3.06903134217629e-06	\\
8809.03764204545	2.74853874560878e-06	\\
8810.01642400568	3.28595630479861e-06	\\
8810.99520596591	4.12412287592922e-06	\\
8811.97398792614	2.98577239704341e-06	\\
8812.95276988636	2.4615192657209e-06	\\
8813.93155184659	2.05847655423689e-06	\\
8814.91033380682	2.71296018208186e-06	\\
8815.88911576704	2.28296659059787e-06	\\
8816.86789772727	1.96828472858691e-06	\\
8817.8466796875	4.33077808187654e-06	\\
8818.82546164773	2.82101720132502e-06	\\
8819.80424360795	1.67209272789278e-06	\\
8820.78302556818	2.9013719632796e-06	\\
8821.76180752841	1.36989152137488e-06	\\
8822.74058948864	2.29113055180151e-06	\\
8823.71937144886	2.72108091751827e-06	\\
8824.69815340909	2.72907135903645e-06	\\
8825.67693536932	2.34525191165373e-06	\\
8826.65571732954	1.00286572000818e-06	\\
8827.63449928977	2.95034232706139e-06	\\
8828.61328125	3.59573454556529e-06	\\
8829.59206321023	2.14652422842305e-06	\\
8830.57084517045	3.80054260699873e-06	\\
8831.54962713068	2.44108651036182e-06	\\
8832.52840909091	1.8720994999813e-06	\\
8833.50719105114	3.14222462246649e-06	\\
8834.48597301136	3.23346063142948e-06	\\
8835.46475497159	3.67918441857658e-06	\\
8836.44353693182	2.34555967163408e-06	\\
8837.42231889204	1.39530932175276e-06	\\
8838.40110085227	2.37700484272566e-06	\\
8839.3798828125	1.2141623832896e-06	\\
8840.35866477273	1.18852732585131e-06	\\
8841.33744673295	2.31878346397848e-06	\\
8842.31622869318	3.0276025193212e-06	\\
8843.29501065341	4.04814291264881e-06	\\
8844.27379261364	1.74897874868165e-06	\\
8845.25257457386	2.69362576746287e-06	\\
8846.23135653409	2.65572333003283e-06	\\
8847.21013849432	2.32928793083313e-06	\\
8848.18892045454	1.34511154181103e-06	\\
8849.16770241477	1.96113228378063e-06	\\
8850.146484375	2.70636119805073e-06	\\
8851.12526633523	2.23120538379971e-06	\\
8852.10404829545	1.94114443301059e-06	\\
8853.08283025568	2.09453530723238e-06	\\
8854.06161221591	1.99882167748441e-06	\\
8855.04039417614	1.43514372983442e-06	\\
8856.01917613636	1.97679201948274e-06	\\
8856.99795809659	1.85650534128506e-06	\\
8857.97674005682	1.04551999625788e-06	\\
8858.95552201704	2.66573679078238e-06	\\
8859.93430397727	3.1107851255329e-06	\\
8860.9130859375	1.26427731486776e-06	\\
8861.89186789773	1.97014436205896e-06	\\
8862.87064985795	2.36117900226651e-06	\\
8863.84943181818	2.60050149504162e-06	\\
8864.82821377841	2.25679636095849e-06	\\
8865.80699573864	1.61339433000953e-06	\\
8866.78577769886	1.61911808679879e-06	\\
8867.76455965909	2.7543907648003e-06	\\
8868.74334161932	1.97575277531478e-06	\\
8869.72212357954	2.47187085756481e-06	\\
8870.70090553977	2.34826671596026e-06	\\
8871.6796875	2.07135303936208e-06	\\
8872.65846946023	2.15511681406188e-06	\\
8873.63725142045	2.1591640945514e-06	\\
8874.61603338068	1.81962810426555e-06	\\
8875.59481534091	3.06015623245974e-06	\\
8876.57359730114	2.15232118092061e-06	\\
8877.55237926136	1.46029740614205e-06	\\
8878.53116122159	3.22892283191207e-06	\\
8879.50994318182	1.90295376001105e-06	\\
8880.48872514204	3.66410058455237e-06	\\
8881.46750710227	2.79452874505e-06	\\
8882.4462890625	2.74195500513601e-06	\\
8883.42507102273	2.0979731256236e-06	\\
8884.40385298295	3.76450810234757e-06	\\
8885.38263494318	2.27833189337566e-06	\\
8886.36141690341	1.44266093099942e-06	\\
8887.34019886364	1.8591169763829e-06	\\
8888.31898082386	2.23726363230009e-06	\\
8889.29776278409	1.88277924056618e-06	\\
8890.27654474432	3.2582332449953e-06	\\
8891.25532670454	2.52951630199236e-06	\\
8892.23410866477	1.8148573074628e-06	\\
8893.212890625	2.23577295876895e-06	\\
8894.19167258523	2.38533444780504e-06	\\
8895.17045454545	2.72320502321358e-06	\\
8896.14923650568	7.38288918596365e-07	\\
8897.12801846591	3.07797461823195e-06	\\
8898.10680042614	1.89103123225885e-06	\\
8899.08558238636	2.12225794350329e-06	\\
8900.06436434659	4.05642169178121e-06	\\
8901.04314630682	2.77104291395959e-06	\\
8902.02192826704	1.38063353678966e-06	\\
8903.00071022727	1.7136433811069e-06	\\
8903.9794921875	1.13499318708122e-06	\\
8904.95827414773	2.16946081066233e-06	\\
8905.93705610795	1.0908334036817e-06	\\
8906.91583806818	9.37543095801631e-07	\\
8907.89462002841	1.14398463232444e-06	\\
8908.87340198864	1.33952308103243e-06	\\
8909.85218394886	1.78314144587565e-06	\\
8910.83096590909	2.7479275886799e-06	\\
8911.80974786932	1.53733341069004e-06	\\
8912.78852982954	1.55776233463834e-06	\\
8913.76731178977	1.42638345937911e-06	\\
8914.74609375	1.72732026662955e-06	\\
8915.72487571023	4.05229245263935e-06	\\
8916.70365767045	6.8384180537982e-07	\\
8917.68243963068	2.91810261814323e-06	\\
8918.66122159091	1.52139011141659e-06	\\
8919.64000355114	1.4007751698396e-06	\\
8920.61878551136	3.30924004763649e-06	\\
8921.59756747159	1.71495047707442e-06	\\
8922.57634943182	1.55049866397578e-06	\\
8923.55513139204	2.44700856030521e-06	\\
8924.53391335227	2.0954939838871e-06	\\
8925.5126953125	2.05034060881722e-06	\\
8926.49147727273	2.91306993462244e-06	\\
8927.47025923295	2.63831419353374e-06	\\
8928.44904119318	1.50227283697752e-06	\\
8929.42782315341	3.20425864002611e-06	\\
8930.40660511364	1.34550543150866e-07	\\
8931.38538707386	1.77167644615773e-06	\\
8932.36416903409	3.07601740862584e-06	\\
8933.34295099432	1.95070027131913e-06	\\
8934.32173295454	8.19202710472283e-07	\\
8935.30051491477	2.0760470129334e-06	\\
8936.279296875	9.84178691777908e-07	\\
8937.25807883523	1.91721577533782e-06	\\
8938.23686079545	1.84550289917676e-06	\\
8939.21564275568	4.08700227161784e-07	\\
8940.19442471591	1.75442478245662e-06	\\
8941.17320667614	1.39521643206244e-07	\\
8942.15198863636	2.31391044260718e-06	\\
8943.13077059659	2.49736304923193e-06	\\
8944.10955255682	1.04017040773862e-06	\\
8945.08833451704	1.87274930001004e-06	\\
8946.06711647727	1.07448036458306e-06	\\
8947.0458984375	2.09435087646751e-06	\\
8948.02468039773	1.21383426058361e-06	\\
8949.00346235795	1.26435953746364e-06	\\
8949.98224431818	3.58791358251981e-06	\\
8950.96102627841	7.75221675606493e-07	\\
8951.93980823864	2.22316539972519e-06	\\
8952.91859019886	2.09583772902445e-06	\\
8953.89737215909	1.92423193612808e-06	\\
8954.87615411932	3.13437351260756e-06	\\
8955.85493607954	2.85535950753474e-06	\\
8956.83371803977	1.55782977082778e-06	\\
8957.8125	2.07615353562826e-06	\\
8958.79128196023	1.87878262887084e-06	\\
8959.77006392045	2.91918302277644e-06	\\
8960.74884588068	2.69216188113444e-06	\\
8961.72762784091	2.48187312675573e-06	\\
8962.70640980114	3.93022071591879e-06	\\
8963.68519176136	3.01446701016822e-06	\\
8964.66397372159	1.59759791510215e-06	\\
8965.64275568182	2.01044060959087e-06	\\
8966.62153764204	3.05063833723406e-06	\\
8967.60031960227	9.13399819865814e-07	\\
8968.5791015625	2.3573055846851e-06	\\
8969.55788352273	2.33833874754307e-06	\\
8970.53666548295	1.75095600211994e-06	\\
8971.51544744318	2.0888605556908e-06	\\
8972.49422940341	1.56802996186119e-06	\\
8973.47301136364	1.00693880876899e-06	\\
8974.45179332386	2.26506965658796e-06	\\
8975.43057528409	2.11095970164975e-06	\\
8976.40935724432	2.04155216763524e-06	\\
8977.38813920454	2.41286205367693e-06	\\
8978.36692116477	3.16416591594629e-06	\\
8979.345703125	3.35885785694376e-06	\\
8980.32448508523	2.07125907042997e-06	\\
8981.30326704545	2.29608917160771e-06	\\
8982.28204900568	2.56254194375569e-06	\\
8983.26083096591	2.91020252177098e-06	\\
8984.23961292614	1.23698104198297e-06	\\
8985.21839488636	9.76319019853391e-07	\\
8986.19717684659	2.09918304212716e-06	\\
8987.17595880682	1.32565793120889e-06	\\
8988.15474076704	1.11860539044014e-06	\\
8989.13352272727	2.72461293228184e-06	\\
8990.1123046875	1.16339089257023e-06	\\
8991.09108664773	1.89384102680694e-06	\\
8992.06986860795	1.77453587662718e-06	\\
8993.04865056818	2.1521630400874e-06	\\
8994.02743252841	2.73159848629123e-06	\\
8995.00621448864	1.43733451668579e-06	\\
8995.98499644886	8.18896036525753e-07	\\
8996.96377840909	2.78364774944937e-06	\\
8997.94256036932	1.56418905517721e-06	\\
8998.92134232954	2.85455261884322e-06	\\
8999.90012428977	3.61968388493234e-06	\\
9000.87890625	1.99950261867047e-06	\\
9001.85768821023	1.78383354428274e-06	\\
9002.83647017045	1.37771389978565e-06	\\
9003.81525213068	2.12822916444477e-06	\\
9004.79403409091	1.70923884235473e-06	\\
9005.77281605114	2.04653991626266e-06	\\
9006.75159801136	1.91222895513168e-06	\\
9007.73037997159	1.481899255764e-06	\\
9008.70916193182	1.60400071805361e-06	\\
9009.68794389204	8.54753117364289e-07	\\
9010.66672585227	1.62260315934751e-06	\\
9011.6455078125	4.8272073074455e-07	\\
9012.62428977273	9.30850130537497e-07	\\
9013.60307173295	8.0849057497255e-07	\\
9014.58185369318	1.80187461485671e-06	\\
9015.56063565341	2.22317482906632e-06	\\
9016.53941761364	1.84229959474683e-06	\\
9017.51819957386	1.23759362954841e-06	\\
9018.49698153409	1.38964862306233e-06	\\
9019.47576349432	2.79439040214657e-06	\\
9020.45454545454	1.01019643922589e-06	\\
9021.43332741477	1.45619395138481e-06	\\
9022.412109375	1.41690501227685e-06	\\
9023.39089133523	1.89902449906691e-06	\\
9024.36967329545	6.83016914923635e-07	\\
9025.34845525568	3.79344531601801e-07	\\
9026.32723721591	1.09915478214168e-06	\\
9027.30601917614	2.56160016036411e-06	\\
9028.28480113636	6.0508190335382e-07	\\
9029.26358309659	1.14116091419598e-06	\\
9030.24236505682	1.94225183884979e-06	\\
9031.22114701704	2.01604689256199e-06	\\
9032.19992897727	8.94147112374634e-07	\\
9033.1787109375	4.04316945297541e-07	\\
9034.15749289773	1.36611496543957e-06	\\
9035.13627485795	5.56706195959081e-07	\\
9036.11505681818	7.6085698367954e-07	\\
9037.09383877841	6.76422129328237e-07	\\
9038.07262073864	5.65213260446919e-07	\\
9039.05140269886	1.73374528104319e-06	\\
9040.03018465909	1.49644273627409e-06	\\
9041.00896661932	3.29455953214679e-07	\\
9041.98774857954	1.30785529673424e-06	\\
9042.96653053977	1.31522588574906e-06	\\
9043.9453125	8.00024416522668e-07	\\
9044.92409446023	2.18623769162331e-06	\\
9045.90287642045	3.2596439361205e-07	\\
9046.88165838068	1.75876490579964e-06	\\
9047.86044034091	1.4582996458828e-06	\\
9048.83922230114	1.32426371441466e-06	\\
9049.81800426136	1.24002202359597e-06	\\
9050.79678622159	5.58908752625064e-07	\\
9051.77556818182	1.03074843494387e-06	\\
9052.75435014204	1.15572806825149e-06	\\
9053.73313210227	8.22080902937566e-07	\\
9054.7119140625	2.05885238982237e-06	\\
9055.69069602273	2.39046312805843e-06	\\
9056.66947798295	7.14300858662457e-07	\\
9057.64825994318	1.8223052140211e-06	\\
9058.62704190341	1.1706776955583e-06	\\
9059.60582386364	9.57214294656849e-07	\\
9060.58460582386	2.05624428034158e-06	\\
9061.56338778409	1.56159960417566e-06	\\
9062.54216974432	1.09844151704934e-06	\\
9063.52095170454	1.27668776486516e-06	\\
9064.49973366477	8.56531186160044e-07	\\
9065.478515625	1.18763251092793e-06	\\
9066.45729758523	1.35984679810746e-06	\\
9067.43607954545	1.30923070594408e-06	\\
9068.41486150568	1.73949409820695e-06	\\
9069.39364346591	1.38877984645899e-06	\\
9070.37242542614	1.04888868131937e-06	\\
9071.35120738636	1.8858517048757e-06	\\
9072.32998934659	1.34331155789774e-06	\\
9073.30877130682	1.65675852859406e-06	\\
9074.28755326704	1.38703412023419e-06	\\
9075.26633522727	1.48474215260783e-06	\\
9076.2451171875	1.26395626464931e-06	\\
9077.22389914773	1.9193342741854e-06	\\
9078.20268110795	2.16334267885535e-06	\\
9079.18146306818	9.50470847728033e-07	\\
9080.16024502841	1.00886202859803e-06	\\
9081.13902698864	1.21517037700969e-06	\\
9082.11780894886	9.78975013841631e-07	\\
9083.09659090909	2.04784155611674e-06	\\
9084.07537286932	2.94519992623528e-07	\\
9085.05415482954	1.38629138082078e-06	\\
9086.03293678977	2.26634818022439e-07	\\
9087.01171875	1.01236506721086e-06	\\
9087.99050071023	1.18305873161161e-06	\\
9088.96928267045	1.49041416725719e-06	\\
9089.94806463068	1.86108342654507e-06	\\
9090.92684659091	7.99889808206917e-07	\\
9091.90562855114	6.31772791088688e-07	\\
9092.88441051136	1.88385075535555e-06	\\
9093.86319247159	1.47854697488012e-06	\\
9094.84197443182	1.25693316135585e-06	\\
9095.82075639204	1.71882205420706e-06	\\
9096.79953835227	5.58770168482422e-07	\\
9097.7783203125	9.41081101950708e-07	\\
9098.75710227273	6.45278193920873e-07	\\
9099.73588423295	9.70368707482217e-07	\\
9100.71466619318	1.96769416009767e-06	\\
9101.69344815341	7.96859670629279e-07	\\
9102.67223011364	1.26064707734842e-06	\\
9103.65101207386	5.64621262125653e-07	\\
9104.62979403409	5.92559959913776e-07	\\
9105.60857599432	1.53484376644953e-06	\\
9106.58735795454	7.68665142031946e-07	\\
9107.56613991477	8.27159731264097e-07	\\
9108.544921875	1.03032716655128e-06	\\
9109.52370383523	5.08870544153114e-07	\\
9110.50248579545	7.47667805028555e-07	\\
9111.48126775568	1.60305135826334e-06	\\
9112.46004971591	1.29270223933388e-06	\\
9113.43883167614	1.10790930494926e-06	\\
9114.41761363636	1.60027394115551e-06	\\
9115.39639559659	1.71016537491295e-06	\\
9116.37517755682	8.35480915727792e-07	\\
9117.35395951704	1.24571608899988e-06	\\
9118.33274147727	7.66077220505724e-07	\\
9119.3115234375	1.22269927719994e-06	\\
9120.29030539773	6.36445037643075e-07	\\
9121.26908735795	2.056344523441e-06	\\
9122.24786931818	1.65217275860146e-06	\\
9123.22665127841	1.09555983375927e-06	\\
9124.20543323864	6.34246252828328e-07	\\
9125.18421519886	9.70868533327948e-07	\\
9126.16299715909	1.22916616616696e-06	\\
9127.14177911932	8.97941670624931e-07	\\
9128.12056107954	1.47334241002783e-06	\\
9129.09934303977	1.40512858063301e-06	\\
9130.078125	1.50860989681033e-06	\\
9131.05690696023	1.70069146802068e-06	\\
9132.03568892045	7.71533871589721e-07	\\
9133.01447088068	1.92042662713406e-06	\\
9133.99325284091	8.26713542400407e-07	\\
9134.97203480114	4.73053321392716e-07	\\
9135.95081676136	1.0175068914972e-06	\\
9136.92959872159	1.4216825485453e-06	\\
9137.90838068182	3.68283007085571e-07	\\
9138.88716264204	2.54855800753059e-07	\\
9139.86594460227	1.09911746441098e-06	\\
9140.8447265625	1.76925865908812e-06	\\
9141.82350852273	1.50993694968181e-06	\\
9142.80229048295	1.14792248995753e-06	\\
9143.78107244318	8.71556641034821e-07	\\
9144.75985440341	6.35993571098178e-07	\\
9145.73863636364	9.23492727821801e-07	\\
9146.71741832386	3.91398886126464e-07	\\
9147.69620028409	1.51187109142664e-06	\\
9148.67498224432	6.60209047344211e-07	\\
9149.65376420454	7.718706914153e-07	\\
9150.63254616477	9.04194936637502e-07	\\
9151.611328125	1.03003406438844e-06	\\
9152.59011008523	9.55987646079004e-07	\\
9153.56889204545	9.52002705352945e-07	\\
9154.54767400568	1.54076784319953e-06	\\
9155.52645596591	1.78012161184405e-07	\\
9156.50523792614	1.0875646943965e-06	\\
9157.48401988636	7.46816698823038e-07	\\
9158.46280184659	1.43986924054267e-06	\\
9159.44158380682	2.1176261818624e-07	\\
9160.42036576704	6.16194501321901e-07	\\
9161.39914772727	1.03290902245733e-07	\\
9162.3779296875	1.06998625877004e-06	\\
9163.35671164773	1.025701905352e-06	\\
9164.33549360795	4.16893118041511e-07	\\
9165.31427556818	2.11679701172265e-06	\\
9166.29305752841	3.28271395303942e-07	\\
9167.27183948864	1.54527694259899e-07	\\
9168.25062144886	1.77042471511417e-06	\\
9169.22940340909	2.02068966884092e-06	\\
9170.20818536932	1.21395355457095e-06	\\
9171.18696732954	2.59810583017099e-07	\\
9172.16574928977	3.02687288366555e-07	\\
9173.14453125	7.26261712421764e-07	\\
9174.12331321023	7.05842475450723e-07	\\
9175.10209517045	7.50201779533432e-07	\\
9176.08087713068	4.70555041240439e-07	\\
9177.05965909091	1.84529836173547e-06	\\
9178.03844105114	1.5615739964684e-06	\\
9179.01722301136	1.73219334194377e-07	\\
9179.99600497159	8.76374990263791e-07	\\
9180.97478693182	1.29254565984508e-06	\\
9181.95356889204	1.06675367147427e-06	\\
9182.93235085227	9.26420618548275e-07	\\
9183.9111328125	7.52051940114893e-07	\\
9184.88991477273	8.02022215231596e-07	\\
9185.86869673295	1.03000766973058e-06	\\
9186.84747869318	6.48908519866124e-08	\\
9187.82626065341	3.58988663535729e-07	\\
9188.80504261364	7.94737113267989e-07	\\
9189.78382457386	1.39872163106183e-06	\\
9190.76260653409	1.81673247471318e-06	\\
9191.74138849432	1.43104671915632e-06	\\
9192.72017045454	1.18411429524331e-06	\\
9193.69895241477	1.13509834536489e-06	\\
9194.677734375	4.69892454538563e-07	\\
9195.65651633523	7.85335210379425e-07	\\
9196.63529829545	1.05704460785399e-06	\\
9197.61408025568	6.93063610095174e-07	\\
9198.59286221591	6.97189693197062e-07	\\
9199.57164417614	1.53725388241587e-06	\\
9200.55042613636	7.67970713843733e-07	\\
9201.52920809659	1.52183622462631e-06	\\
9202.50799005682	1.46726283888877e-06	\\
9203.48677201704	6.87058023720824e-07	\\
9204.46555397727	5.90973171744822e-07	\\
9205.4443359375	8.97059554828987e-07	\\
9206.42311789773	4.69592329652699e-07	\\
9207.40189985795	1.89632873697649e-07	\\
9208.38068181818	2.84989986283952e-07	\\
9209.35946377841	8.77759041168074e-07	\\
9210.33824573864	1.09829390248981e-06	\\
9211.31702769886	8.56009714877834e-07	\\
9212.29580965909	1.77229636906379e-06	\\
9213.27459161932	1.61611736714171e-06	\\
9214.25337357954	4.68704838841551e-07	\\
9215.23215553977	2.06568815282174e-06	\\
9216.2109375	1.55678086422852e-06	\\
9217.18971946023	1.03119933508935e-06	\\
9218.16850142045	4.4026389430923e-07	\\
9219.14728338068	8.59907058166526e-07	\\
9220.12606534091	1.33248585621518e-06	\\
9221.10484730114	3.87407801800029e-07	\\
9222.08362926136	7.04102413560418e-07	\\
9223.06241122159	7.81266307551434e-07	\\
9224.04119318182	1.01748198416591e-06	\\
9225.01997514204	1.32847331333223e-06	\\
9225.99875710227	1.06114769728407e-06	\\
9226.9775390625	1.33096635798897e-06	\\
9227.95632102273	1.1326955866698e-06	\\
9228.93510298295	4.97227887400041e-07	\\
9229.91388494318	6.17755722759856e-07	\\
9230.89266690341	9.46939982387482e-07	\\
9231.87144886364	1.75964492463003e-06	\\
9232.85023082386	1.13766570950342e-06	\\
9233.82901278409	7.14930054792279e-07	\\
9234.80779474432	1.11955987654573e-06	\\
9235.78657670454	1.33901489715682e-06	\\
9236.76535866477	9.29335555554594e-07	\\
9237.744140625	9.66912241549525e-07	\\
9238.72292258523	4.09138640001689e-07	\\
9239.70170454545	4.12400583664712e-07	\\
9240.68048650568	6.77721876323977e-07	\\
9241.65926846591	8.34694559109992e-07	\\
9242.63805042614	5.04255061415902e-07	\\
9243.61683238636	1.55761484717326e-07	\\
9244.59561434659	1.31013404965679e-06	\\
9245.57439630682	1.33814774525271e-06	\\
9246.55317826704	1.37684862756222e-06	\\
9247.53196022727	3.62018831386706e-07	\\
9248.5107421875	2.6822396521005e-07	\\
9249.48952414773	3.19994538246584e-07	\\
9250.46830610795	8.44009335354947e-07	\\
9251.44708806818	5.81232072851467e-07	\\
9252.42587002841	6.49589326258631e-07	\\
9253.40465198864	1.68631537934412e-06	\\
9254.38343394886	7.87676841807818e-08	\\
9255.36221590909	5.06915259300238e-07	\\
9256.34099786932	6.22810337677897e-07	\\
9257.31977982954	6.33650222513339e-07	\\
9258.29856178977	7.02952957039027e-07	\\
9259.27734375	1.82742075650962e-06	\\
9260.25612571023	1.44824509616788e-06	\\
9261.23490767045	3.28174528498026e-07	\\
9262.21368963068	4.25690878040225e-07	\\
9263.19247159091	3.39399975835859e-07	\\
9264.17125355114	8.46843922950447e-07	\\
9265.15003551136	7.09983458011427e-07	\\
9266.12881747159	4.96052575075334e-07	\\
9267.10759943182	1.0610842977859e-06	\\
9268.08638139204	4.41109231637537e-07	\\
9269.06516335227	5.44617579284376e-07	\\
9270.0439453125	3.39992781905984e-07	\\
9271.02272727273	4.33758774449021e-07	\\
9272.00150923295	4.73968618783679e-07	\\
9272.98029119318	5.33230249863173e-07	\\
9273.95907315341	3.47795175177363e-07	\\
9274.93785511364	1.36110421812059e-06	\\
9275.91663707386	1.63569504008343e-06	\\
9276.89541903409	1.26723626576509e-06	\\
9277.87420099432	2.47403821075025e-07	\\
9278.85298295454	3.09773361100886e-07	\\
9279.83176491477	3.54416641356528e-07	\\
9280.810546875	6.32105705942981e-07	\\
9281.78932883523	2.01929972624301e-07	\\
9282.76811079545	6.66196723562332e-07	\\
9283.74689275568	1.68209758058059e-07	\\
9284.72567471591	7.88166087558649e-07	\\
9285.70445667614	1.15685850191773e-06	\\
9286.68323863636	5.01489034938593e-07	\\
9287.66202059659	1.19407496571074e-06	\\
9288.64080255682	8.36909783382818e-07	\\
9289.61958451704	5.53653850715051e-07	\\
9290.59836647727	8.54822445595341e-07	\\
9291.5771484375	4.98411394686139e-07	\\
9292.55593039773	8.54427364242348e-07	\\
9293.53471235795	1.02589874322834e-06	\\
9294.51349431818	8.25054553719788e-07	\\
9295.49227627841	4.74448756150737e-07	\\
9296.47105823864	3.8144553336652e-07	\\
9297.44984019886	9.62883710166045e-07	\\
9298.42862215909	4.36465734160233e-07	\\
9299.40740411932	9.53186881491565e-07	\\
9300.38618607954	1.2857790950678e-06	\\
9301.36496803977	8.39955239315413e-07	\\
9302.34375	6.36809716302977e-07	\\
9303.32253196023	1.00063146781546e-06	\\
9304.30131392045	4.38198288216869e-07	\\
9305.28009588068	3.4956491231497e-07	\\
9306.25887784091	9.9255640633435e-07	\\
9307.23765980114	4.61042995984844e-07	\\
9308.21644176136	1.51035991934998e-06	\\
9309.19522372159	1.40224572604593e-06	\\
9310.17400568182	9.54730774410535e-07	\\
9311.15278764204	9.42101074117874e-07	\\
9312.13156960227	1.51583399961555e-06	\\
9313.1103515625	9.64898720633358e-07	\\
9314.08913352273	7.40656576530351e-07	\\
9315.06791548295	3.86789672489791e-07	\\
9316.04669744318	2.73335330360526e-07	\\
9317.02547940341	9.74906469458958e-07	\\
9318.00426136364	2.43828340737185e-07	\\
9318.98304332386	9.82289769723645e-07	\\
9319.96182528409	1.99424911009697e-06	\\
9320.94060724432	1.28134293121225e-07	\\
9321.91938920454	5.24893946755281e-07	\\
9322.89817116477	2.34929523930397e-06	\\
9323.876953125	1.4789121558704e-06	\\
9324.85573508523	1.73556723467256e-06	\\
9325.83451704545	4.37495099465317e-07	\\
9326.81329900568	5.64620871767836e-07	\\
9327.79208096591	3.37182656335801e-07	\\
9328.77086292614	6.6422065190669e-07	\\
9329.74964488636	5.2421564898287e-07	\\
9330.72842684659	4.18592695312603e-07	\\
9331.70720880682	2.00636582506908e-08	\\
9332.68599076704	6.40744035743505e-07	\\
9333.66477272727	5.20843620926003e-07	\\
9334.6435546875	2.77816181959462e-07	\\
9335.62233664773	3.76158168583449e-07	\\
9336.60111860795	4.10371134314238e-07	\\
9337.57990056818	9.83027018347073e-07	\\
9338.55868252841	4.88620972262281e-07	\\
9339.53746448864	5.58568420411389e-07	\\
9340.51624644886	4.44479674295842e-07	\\
9341.49502840909	6.62113731953244e-07	\\
9342.47381036932	1.45625110830639e-06	\\
9343.45259232954	4.91272254513205e-07	\\
9344.43137428977	1.04353433169235e-06	\\
9345.41015625	1.10516564263216e-06	\\
9346.38893821023	8.19349842504997e-07	\\
9347.36772017045	3.45603257521548e-07	\\
9348.34650213068	7.26379417440243e-07	\\
9349.32528409091	1.04456342507218e-06	\\
9350.30406605114	7.68458306208263e-07	\\
9351.28284801136	1.27501438413465e-06	\\
9352.26162997159	8.13083767590717e-07	\\
9353.24041193182	1.38209181899039e-06	\\
9354.21919389204	1.21318367319133e-06	\\
9355.19797585227	1.69718893553471e-06	\\
9356.1767578125	9.53940846975855e-07	\\
9357.15553977273	7.37872568176912e-07	\\
9358.13432173295	7.40723044203781e-07	\\
9359.11310369318	4.09332916712071e-07	\\
9360.09188565341	1.04903400494758e-06	\\
9361.07066761364	3.92507291385461e-07	\\
9362.04944957386	9.14036550649078e-07	\\
9363.02823153409	1.14267535678151e-06	\\
9364.00701349432	1.16655751405632e-06	\\
9364.98579545454	1.63647064011503e-07	\\
9365.96457741477	1.11620400384662e-06	\\
9366.943359375	2.4462077381245e-07	\\
9367.92214133523	5.390843994041e-07	\\
9368.90092329545	1.08537384032427e-06	\\
9369.87970525568	6.9585148571009e-07	\\
9370.85848721591	6.33622590246485e-07	\\
9371.83726917614	4.54841616743854e-07	\\
9372.81605113636	7.46970000346113e-07	\\
9373.79483309659	1.11134283817704e-06	\\
9374.77361505682	1.10907819888185e-06	\\
9375.75239701704	1.48517974411769e-07	\\
9376.73117897727	3.47755993014192e-07	\\
9377.7099609375	9.92941748949503e-07	\\
9378.68874289773	6.51668534203339e-07	\\
9379.66752485795	9.2423735249465e-07	\\
9380.64630681818	7.9193936230922e-07	\\
9381.62508877841	1.7508616091789e-06	\\
9382.60387073864	5.99849677095614e-07	\\
9383.58265269886	5.55757559896576e-07	\\
9384.56143465909	5.51600926926797e-07	\\
9385.54021661932	1.82661164470884e-06	\\
9386.51899857954	9.34695743956055e-07	\\
9387.49778053977	8.00678561029293e-07	\\
9388.4765625	6.03508678343688e-07	\\
9389.45534446023	6.65202091255661e-07	\\
9390.43412642045	6.45904358170057e-07	\\
9391.41290838068	4.7345973308218e-07	\\
9392.39169034091	5.32587325538808e-07	\\
9393.37047230114	6.87601949893929e-07	\\
9394.34925426136	6.56297709865317e-07	\\
9395.32803622159	5.09085949520379e-07	\\
9396.30681818182	1.50353122560658e-06	\\
9397.28560014204	1.0090860793513e-06	\\
9398.26438210227	7.21201809527722e-07	\\
9399.2431640625	3.38867081352699e-07	\\
9400.22194602273	1.31001991104272e-06	\\
9401.20072798295	5.40631720806198e-07	\\
9402.17950994318	1.73938252613073e-06	\\
9403.15829190341	8.02223165678658e-07	\\
9404.13707386364	9.35971935272287e-07	\\
9405.11585582386	1.14535636697698e-06	\\
9406.09463778409	3.77520650545523e-07	\\
9407.07341974432	3.64716344758893e-07	\\
9408.05220170454	8.96573410638228e-07	\\
9409.03098366477	1.04183944009444e-06	\\
9410.009765625	1.11175445582646e-06	\\
9410.98854758523	5.55344168301636e-07	\\
9411.96732954545	4.49589113884894e-07	\\
9412.94611150568	4.17358723844697e-07	\\
9413.92489346591	6.22075340567346e-07	\\
9414.90367542614	1.01590222788588e-06	\\
9415.88245738636	3.33638816284809e-07	\\
9416.86123934659	1.13571773316186e-06	\\
9417.84002130682	1.26628860421959e-06	\\
9418.81880326704	6.5742108425336e-07	\\
9419.79758522727	3.89503260591257e-07	\\
9420.7763671875	1.19872032675867e-06	\\
9421.75514914773	9.91226915437376e-07	\\
9422.73393110795	1.25194847274705e-06	\\
9423.71271306818	6.24367740089579e-07	\\
9424.69149502841	7.92925825442631e-08	\\
9425.67027698864	8.9664969052057e-07	\\
9426.64905894886	7.69859605166571e-07	\\
9427.62784090909	1.00602115703271e-06	\\
9428.60662286932	5.68247383234004e-07	\\
9429.58540482954	4.75701555724256e-07	\\
9430.56418678977	5.13317689486358e-07	\\
9431.54296875	4.48410817000834e-07	\\
9432.52175071023	5.78071886870576e-07	\\
9433.50053267045	1.55604660661723e-06	\\
9434.47931463068	9.23222104944489e-07	\\
9435.45809659091	2.59618319541005e-07	\\
9436.43687855114	8.18960574830875e-07	\\
9437.41566051136	4.53379048245453e-07	\\
9438.39444247159	1.97370610498713e-06	\\
9439.37322443182	2.45456815432894e-07	\\
9440.35200639204	7.49079543799861e-07	\\
9441.33078835227	5.29462854278188e-07	\\
9442.3095703125	4.18464355764074e-07	\\
9443.28835227273	8.75417954291667e-07	\\
9444.26713423295	9.64493472466269e-07	\\
9445.24591619318	7.72545923899016e-07	\\
9446.22469815341	1.04605258623526e-06	\\
9447.20348011364	5.47628242473973e-07	\\
9448.18226207386	4.38627912510905e-07	\\
9449.16104403409	3.10505491537297e-07	\\
9450.13982599432	1.27818607090822e-06	\\
9451.11860795454	3.974141558128e-07	\\
9452.09738991477	1.40198472217627e-06	\\
9453.076171875	6.52919617555847e-07	\\
9454.05495383523	7.93218224056602e-07	\\
9455.03373579545	4.12944721021235e-08	\\
9456.01251775568	7.00131462425708e-07	\\
9456.99129971591	7.64373077434674e-07	\\
9457.97008167614	9.11547941027198e-07	\\
9458.94886363636	9.30592861699854e-07	\\
9459.92764559659	1.27727315038041e-06	\\
9460.90642755682	6.97531663029686e-07	\\
9461.88520951704	6.39014315555371e-07	\\
9462.86399147727	8.64221491777155e-07	\\
9463.8427734375	1.30529472630458e-06	\\
9464.82155539773	1.25542238892949e-06	\\
9465.80033735795	1.70909438342722e-06	\\
9466.77911931818	1.80146975010839e-06	\\
9467.75790127841	1.05076945125438e-06	\\
9468.73668323864	8.81214535251027e-07	\\
9469.71546519886	8.68481735362943e-07	\\
9470.69424715909	7.29366607567173e-07	\\
9471.67302911932	5.33996067512617e-07	\\
9472.65181107954	8.16262875656045e-08	\\
9473.63059303977	9.47482563772298e-07	\\
9474.609375	1.72739307398069e-06	\\
9475.58815696023	2.13117241210455e-06	\\
9476.56693892045	1.103490049112e-06	\\
9477.54572088068	7.99842919313981e-07	\\
9478.52450284091	3.39325080815367e-07	\\
9479.50328480114	8.48670901573024e-07	\\
9480.48206676136	7.28693270472058e-07	\\
9481.46084872159	9.0192589747548e-07	\\
9482.43963068182	6.97758043549694e-07	\\
9483.41841264204	1.50572377141598e-06	\\
9484.39719460227	7.44882154410025e-07	\\
9485.3759765625	5.67069843070656e-07	\\
9486.35475852273	1.46700568350115e-06	\\
9487.33354048295	1.12730733001443e-06	\\
9488.31232244318	1.70106658058823e-07	\\
9489.29110440341	9.0714727200529e-07	\\
9490.26988636364	1.09357549099589e-06	\\
9491.24866832386	1.14776211352107e-06	\\
9492.22745028409	9.59583925142852e-07	\\
9493.20623224432	4.17428710083882e-07	\\
9494.18501420454	6.81467978549313e-07	\\
9495.16379616477	1.17220154110032e-06	\\
9496.142578125	1.01716297952709e-06	\\
9497.12136008523	1.11138191291733e-06	\\
9498.10014204545	9.69127560834668e-07	\\
9499.07892400568	3.60072484794225e-07	\\
9500.05770596591	8.29177996762416e-07	\\
9501.03648792614	1.05757522159016e-06	\\
9502.01526988636	8.6169315242637e-07	\\
9502.99405184659	1.19991165390747e-06	\\
9503.97283380682	1.17693490938752e-06	\\
9504.95161576704	8.75229732697439e-07	\\
9505.93039772727	9.08991274386124e-07	\\
9506.9091796875	1.60758759921891e-06	\\
9507.88796164773	1.53307614785915e-06	\\
9508.86674360795	1.32419691592014e-06	\\
9509.84552556818	1.50348263476709e-06	\\
9510.82430752841	5.27416127663875e-07	\\
9511.80308948864	8.37200356907423e-07	\\
9512.78187144886	7.75180306081029e-07	\\
9513.76065340909	1.57648984749869e-06	\\
9514.73943536932	1.010507380137e-06	\\
9515.71821732954	9.02158931544055e-07	\\
9516.69699928977	1.10471156384117e-06	\\
9517.67578125	2.57160312847532e-07	\\
9518.65456321023	8.58748381081656e-07	\\
9519.63334517045	1.36872059013914e-06	\\
9520.61212713068	1.49365878841452e-06	\\
9521.59090909091	3.88828652091792e-07	\\
9522.56969105114	1.51204151647305e-06	\\
9523.54847301136	7.38239643578063e-07	\\
9524.52725497159	1.96623517379157e-06	\\
9525.50603693182	1.484253307511e-06	\\
9526.48481889204	5.31636061254308e-07	\\
9527.46360085227	1.40132790375885e-06	\\
9528.4423828125	2.58621483001598e-06	\\
9529.42116477273	1.12200124210212e-06	\\
9530.39994673295	1.11103747209238e-06	\\
9531.37872869318	2.07368810227091e-06	\\
9532.35751065341	1.51123195872057e-06	\\
9533.33629261364	1.58483368333141e-06	\\
9534.31507457386	9.75739476199934e-07	\\
9535.29385653409	1.18283508309512e-06	\\
9536.27263849432	1.66356302598516e-06	\\
9537.25142045454	1.56442937965666e-06	\\
9538.23020241477	1.05896105967678e-06	\\
9539.208984375	1.03063184888319e-06	\\
9540.18776633523	1.58102222310369e-06	\\
9541.16654829545	7.10915375419117e-07	\\
9542.14533025568	1.41062907651406e-06	\\
9543.12411221591	1.0916825353483e-06	\\
9544.10289417614	4.49318390111886e-07	\\
9545.08167613636	9.39628391041931e-07	\\
9546.06045809659	1.49106338270203e-06	\\
9547.03924005682	1.77950359350857e-06	\\
9548.01802201704	1.8863856101682e-06	\\
9548.99680397727	7.4680872721568e-07	\\
9549.9755859375	1.03398452122625e-06	\\
9550.95436789773	1.10039117122893e-06	\\
9551.93314985795	1.41339568614202e-06	\\
9552.91193181818	1.98054186327817e-06	\\
9553.89071377841	1.64768820469012e-06	\\
9554.86949573864	1.85949690827582e-06	\\
9555.84827769886	1.44351333639976e-06	\\
9556.82705965909	9.19854414759198e-07	\\
9557.80584161932	1.01901491012334e-06	\\
9558.78462357954	1.16752895690326e-06	\\
9559.76340553977	1.14080204979291e-06	\\
9560.7421875	7.8058396313261e-07	\\
9561.72096946023	1.36318890992215e-06	\\
9562.69975142045	1.0093751215643e-06	\\
9563.67853338068	1.17188901144031e-06	\\
9564.65731534091	2.27307901671776e-06	\\
9565.63609730114	6.45977486242379e-07	\\
9566.61487926136	1.2144688716167e-06	\\
9567.59366122159	1.34104717048946e-06	\\
9568.57244318182	9.41685117324801e-07	\\
9569.55122514204	1.02779420227023e-06	\\
9570.53000710227	1.86695984901689e-06	\\
9571.5087890625	1.45495605947522e-07	\\
9572.48757102273	3.78850230922092e-07	\\
9573.46635298295	1.54184046765412e-06	\\
9574.44513494318	1.89547991735777e-06	\\
9575.42391690341	1.47105212386152e-06	\\
9576.40269886364	6.12773599420247e-07	\\
9577.38148082386	1.89888560295986e-06	\\
9578.36026278409	1.66876671913501e-06	\\
9579.33904474432	1.31744997250482e-06	\\
9580.31782670454	9.23935947124265e-08	\\
9581.29660866477	5.97448194913892e-07	\\
9582.275390625	6.55070899904047e-07	\\
9583.25417258523	2.51546770318008e-07	\\
9584.23295454545	1.16581057813755e-06	\\
9585.21173650568	1.67014536435367e-06	\\
9586.19051846591	1.89785677155627e-06	\\
9587.16930042614	8.99642256211695e-07	\\
9588.14808238636	1.09143293240168e-06	\\
9589.12686434659	1.18310009871786e-06	\\
9590.10564630682	1.25067965840717e-06	\\
9591.08442826704	5.16292625775367e-07	\\
9592.06321022727	1.38975008796861e-06	\\
9593.0419921875	1.46041512619939e-06	\\
9594.02077414773	2.67629986767757e-07	\\
9594.99955610795	1.1401257658441e-06	\\
9595.97833806818	2.12946013064506e-06	\\
9596.95712002841	1.04627702110055e-06	\\
9597.93590198864	2.0599712049728e-06	\\
9598.91468394886	2.17605840231328e-06	\\
9599.89346590909	1.51452594609946e-06	\\
9600.87224786932	8.89667977458859e-07	\\
9601.85102982954	6.59220508236174e-07	\\
9602.82981178977	1.18033892756355e-06	\\
9603.80859375	8.20388254354482e-07	\\
9604.78737571023	1.54893543570149e-06	\\
9605.76615767045	5.65011131795598e-07	\\
9606.74493963068	1.12295601592369e-06	\\
9607.72372159091	7.27645680199084e-07	\\
9608.70250355114	1.41838248095949e-06	\\
9609.68128551136	1.38838137997117e-06	\\
9610.66006747159	1.52372770327425e-06	\\
9611.63884943182	1.03556522475811e-06	\\
9612.61763139204	1.35841398827186e-06	\\
9613.59641335227	1.15653985608989e-06	\\
9614.5751953125	2.26470374640113e-06	\\
9615.55397727273	1.62478009723808e-06	\\
9616.53275923295	1.83199588591984e-06	\\
9617.51154119318	3.88484499058578e-07	\\
9618.49032315341	2.25749943966176e-06	\\
9619.46910511364	2.02349042999547e-06	\\
9620.44788707386	1.21824813275503e-06	\\
9621.42666903409	1.31236948568782e-06	\\
9622.40545099432	1.34133305426725e-06	\\
9623.38423295454	1.86276153072211e-06	\\
9624.36301491477	5.40513325490104e-07	\\
9625.341796875	1.35833092177807e-06	\\
9626.32057883523	2.0290150908879e-06	\\
9627.29936079545	1.37126368272168e-06	\\
9628.27814275568	9.00495950882308e-07	\\
9629.25692471591	1.45835283887651e-06	\\
9630.23570667614	2.21701735106338e-07	\\
9631.21448863636	5.50511530068687e-07	\\
9632.19327059659	1.66179207014459e-06	\\
9633.17205255682	1.24858780913705e-06	\\
9634.15083451704	9.85159732706283e-07	\\
9635.12961647727	1.44512106366203e-06	\\
9636.1083984375	9.21034207017037e-07	\\
9637.08718039773	1.86353728963111e-06	\\
9638.06596235795	2.138518083784e-06	\\
9639.04474431818	1.0563715525059e-06	\\
9640.02352627841	2.09802279336388e-06	\\
9641.00230823864	2.52470416365474e-06	\\
9641.98109019886	1.37087804071297e-06	\\
9642.95987215909	1.93251044430958e-06	\\
9643.93865411932	1.54867701937238e-06	\\
9644.91743607954	6.11526523135712e-07	\\
9645.89621803977	1.09165163785271e-06	\\
9646.875	2.24927442922318e-06	\\
9647.85378196023	1.70708638302484e-06	\\
9648.83256392045	2.17545555868138e-06	\\
9649.81134588068	1.93262612268533e-06	\\
9650.79012784091	2.35194994636249e-06	\\
9651.76890980114	1.37355250262671e-06	\\
9652.74769176136	2.24990557558777e-06	\\
9653.72647372159	1.41255693919786e-06	\\
9654.70525568182	1.30553862713622e-06	\\
9655.68403764204	1.77396084739306e-06	\\
9656.66281960227	1.16469178813039e-06	\\
9657.6416015625	1.40181687460179e-06	\\
9658.62038352273	5.57620654040434e-07	\\
9659.59916548295	1.71747645051753e-06	\\
9660.57794744318	1.1898044673138e-06	\\
9661.55672940341	8.00247020255558e-07	\\
9662.53551136364	1.51883429635231e-06	\\
9663.51429332386	1.62994090594114e-06	\\
9664.49307528409	8.91044391274929e-07	\\
9665.47185724432	8.20910592027106e-07	\\
9666.45063920454	2.22848405140867e-06	\\
9667.42942116477	8.30038479769984e-07	\\
9668.408203125	1.31937489224826e-06	\\
9669.38698508523	2.12192376445461e-06	\\
9670.36576704545	1.43845033700891e-06	\\
9671.34454900568	1.53849844205465e-06	\\
9672.32333096591	1.80176349522415e-06	\\
9673.30211292614	4.33882706583638e-07	\\
9674.28089488636	1.38415012166851e-07	\\
9675.25967684659	1.4535915095941e-06	\\
9676.23845880682	2.76405702551088e-07	\\
9677.21724076704	8.16951030154323e-07	\\
9678.19602272727	1.30130821231406e-06	\\
9679.1748046875	2.12339989769078e-06	\\
9680.15358664773	1.2861245927256e-06	\\
9681.13236860795	2.0556320961744e-06	\\
9682.11115056818	1.12725108849928e-06	\\
9683.08993252841	9.75713194744584e-07	\\
9684.06871448864	8.87695273608508e-07	\\
9685.04749644886	8.28469280250547e-07	\\
9686.02627840909	2.13802317842111e-06	\\
9687.00506036932	1.50436804052755e-06	\\
9687.98384232954	1.16250424163057e-06	\\
9688.96262428977	1.19424340933482e-06	\\
9689.94140625	1.10454965600828e-06	\\
9690.92018821023	1.08642904088027e-06	\\
9691.89897017045	1.87200433990697e-06	\\
9692.87775213068	9.04002295923602e-07	\\
9693.85653409091	1.61274666593281e-06	\\
9694.83531605114	8.50416588515309e-07	\\
9695.81409801136	1.63342731615419e-06	\\
9696.79287997159	1.65772631728657e-06	\\
9697.77166193182	1.21734967949325e-06	\\
9698.75044389204	1.14689424311111e-06	\\
9699.72922585227	1.74973213326393e-06	\\
9700.7080078125	1.58476914860583e-06	\\
9701.68678977273	2.06829219223033e-06	\\
9702.66557173295	1.35398716968495e-06	\\
9703.64435369318	1.3971547445267e-06	\\
9704.62313565341	2.40855999241598e-06	\\
9705.60191761364	1.94288439188408e-06	\\
9706.58069957386	1.920846838143e-06	\\
9707.55948153409	1.38132869640979e-06	\\
9708.53826349432	1.77910019630234e-06	\\
9709.51704545454	1.90790202341455e-06	\\
9710.49582741477	2.30176722446918e-06	\\
9711.474609375	7.14354584093275e-07	\\
9712.45339133523	1.16825675873503e-06	\\
9713.43217329545	2.52582366924622e-06	\\
9714.41095525568	1.021317873416e-06	\\
9715.38973721591	1.59032503941331e-06	\\
9716.36851917614	1.24811299933887e-06	\\
9717.34730113636	8.98611063772713e-07	\\
9718.32608309659	1.79001270236022e-06	\\
9719.30486505682	1.16885144431099e-06	\\
9720.28364701704	9.70977202755725e-07	\\
9721.26242897727	1.81122560572319e-06	\\
9722.2412109375	1.65861223953161e-06	\\
9723.21999289773	1.35269680594346e-06	\\
9724.19877485795	1.26585677879369e-06	\\
9725.17755681818	1.6237405060779e-06	\\
9726.15633877841	1.72420766478434e-06	\\
9727.13512073864	1.5799858098041e-06	\\
9728.11390269886	1.43300478954693e-06	\\
9729.09268465909	1.40364950894192e-06	\\
9730.07146661932	2.27908462363731e-06	\\
9731.05024857954	1.77293599304373e-06	\\
9732.02903053977	1.23841329512432e-06	\\
9733.0078125	2.08105441108376e-06	\\
9733.98659446023	1.53816347463942e-06	\\
9734.96537642045	1.5404660725427e-06	\\
9735.94415838068	8.88815693268383e-07	\\
9736.92294034091	1.62647736543715e-06	\\
9737.90172230114	1.38896125380037e-06	\\
9738.88050426136	1.95229359719937e-06	\\
9739.85928622159	1.67917256554233e-06	\\
9740.83806818182	1.36684061374693e-06	\\
9741.81685014204	1.58810770279629e-06	\\
9742.79563210227	1.48003983275487e-06	\\
9743.7744140625	1.82048386478487e-06	\\
9744.75319602273	8.93580543650336e-07	\\
9745.73197798295	8.02108996414532e-07	\\
9746.71075994318	2.24582561288636e-06	\\
9747.68954190341	1.34562010584543e-06	\\
9748.66832386364	8.45108936510256e-07	\\
9749.64710582386	1.34237884282743e-06	\\
9750.62588778409	1.35168111296198e-06	\\
9751.60466974432	1.34817123286002e-06	\\
9752.58345170454	2.13674884825246e-06	\\
9753.56223366477	1.86362917106975e-06	\\
9754.541015625	1.20884635265388e-06	\\
9755.51979758523	1.79856299215516e-06	\\
9756.49857954545	1.90229845884032e-06	\\
9757.47736150568	1.20166743335836e-06	\\
9758.45614346591	1.8349609954564e-06	\\
9759.43492542614	1.47648368306146e-06	\\
9760.41370738636	8.54369464953844e-07	\\
9761.39248934659	2.40912475104534e-06	\\
9762.37127130682	1.11169803950079e-06	\\
9763.35005326704	1.53587376416227e-06	\\
9764.32883522727	9.13775295353676e-07	\\
9765.3076171875	2.08773945743463e-06	\\
9766.28639914773	1.45984913168075e-06	\\
9767.26518110795	1.96796571344117e-06	\\
9768.24396306818	1.1011655001397e-06	\\
9769.22274502841	2.05441102305968e-06	\\
9770.20152698864	2.22522039855496e-06	\\
9771.18030894886	7.11246180454093e-07	\\
9772.15909090909	1.51939624995448e-06	\\
9773.13787286932	4.85671726073883e-07	\\
9774.11665482954	1.17261471120273e-06	\\
9775.09543678977	2.47645881065887e-06	\\
9776.07421875	1.71245676379958e-06	\\
9777.05300071023	2.08582047475435e-06	\\
9778.03178267045	1.34407129316664e-06	\\
9779.01056463068	1.29778004370245e-06	\\
9779.98934659091	1.11050926639466e-06	\\
9780.96812855114	1.64416489604543e-06	\\
9781.94691051136	1.09795212432575e-06	\\
9782.92569247159	5.75479635068282e-07	\\
9783.90447443182	1.44175938843908e-06	\\
9784.88325639204	2.14702620769018e-06	\\
9785.86203835227	1.62170267374395e-06	\\
9786.8408203125	1.65464733132815e-06	\\
9787.81960227273	7.57171981812985e-07	\\
9788.79838423295	1.35502913479637e-06	\\
9789.77716619318	2.53907540915865e-06	\\
9790.75594815341	1.27019415189918e-06	\\
9791.73473011364	2.7729336609533e-06	\\
9792.71351207386	1.83348428227011e-06	\\
9793.69229403409	9.65657936752557e-07	\\
9794.67107599432	1.38947500909854e-06	\\
9795.64985795454	2.12614176216716e-06	\\
9796.62863991477	1.2458993325956e-06	\\
9797.607421875	1.32095796444311e-06	\\
9798.58620383523	1.65059684041485e-06	\\
9799.56498579545	1.61545897041491e-06	\\
9800.54376775568	1.85577684486983e-06	\\
9801.52254971591	1.33962599108963e-06	\\
9802.50133167614	1.86761587294018e-06	\\
9803.48011363636	1.97633057337287e-06	\\
9804.45889559659	1.04108437483607e-06	\\
9805.43767755682	1.08895614189472e-06	\\
9806.41645951704	1.47297234581018e-06	\\
9807.39524147727	1.00673365729159e-06	\\
9808.3740234375	1.2309412156927e-06	\\
9809.35280539773	2.29650507105461e-06	\\
9810.33158735795	1.12726451381219e-06	\\
9811.31036931818	2.18426128755967e-06	\\
9812.28915127841	1.16880787963046e-06	\\
9813.26793323864	1.340912247364e-06	\\
9814.24671519886	5.51776896602492e-07	\\
9815.22549715909	1.44121489293333e-06	\\
9816.20427911932	1.55402141249153e-06	\\
9817.18306107954	8.11262874233179e-07	\\
9818.16184303977	1.91073683434515e-06	\\
9819.140625	1.65404351296318e-06	\\
9820.11940696023	1.19035297069524e-06	\\
9821.09818892045	1.7978175916512e-06	\\
9822.07697088068	1.42595188991058e-06	\\
9823.05575284091	1.68104637623053e-06	\\
9824.03453480114	1.187280663546e-06	\\
9825.01331676136	1.03051434577327e-06	\\
9825.99209872159	1.56616368738718e-06	\\
9826.97088068182	9.7182107609655e-07	\\
9827.94966264204	1.76202544953166e-06	\\
9828.92844460227	1.69558599178319e-06	\\
9829.9072265625	1.87758437220119e-06	\\
9830.88600852273	1.68502911248876e-06	\\
9831.86479048295	1.65196914520551e-06	\\
9832.84357244318	1.38667137732996e-06	\\
9833.82235440341	1.32739445477457e-06	\\
9834.80113636364	2.10516966065206e-06	\\
9835.77991832386	1.6099924452849e-06	\\
9836.75870028409	1.07596969103616e-06	\\
9837.73748224432	2.01002257380222e-06	\\
9838.71626420454	1.65520753683063e-06	\\
9839.69504616477	1.20584085424133e-06	\\
9840.673828125	1.73276449966221e-06	\\
9841.65261008523	1.77028055770314e-06	\\
9842.63139204545	1.57445919843914e-06	\\
9843.61017400568	9.51623517419882e-07	\\
9844.58895596591	1.83410862229285e-06	\\
9845.56773792614	1.78975163609882e-06	\\
9846.54651988636	1.84725821520408e-06	\\
9847.52530184659	9.40649382322585e-07	\\
9848.50408380682	2.39863222313885e-06	\\
9849.48286576704	1.32900996589974e-06	\\
9850.46164772727	1.62136717328977e-06	\\
9851.4404296875	2.11725425552421e-06	\\
9852.41921164773	5.59199268543706e-07	\\
9853.39799360795	1.88921124462277e-06	\\
9854.37677556818	1.8782268558928e-06	\\
9855.35555752841	9.3230553125719e-07	\\
9856.33433948864	1.66800182824856e-06	\\
9857.31312144886	1.33377475618598e-06	\\
9858.29190340909	9.74279342871117e-07	\\
9859.27068536932	1.31131912968756e-06	\\
9860.24946732954	1.34723849734352e-06	\\
9861.22824928977	5.33792560908589e-07	\\
9862.20703125	1.85746579849873e-06	\\
9863.18581321023	9.70156541278394e-07	\\
9864.16459517045	4.35820251348085e-07	\\
9865.14337713068	1.82860810658841e-06	\\
9866.12215909091	1.6206460766485e-06	\\
9867.10094105114	1.06475838271299e-06	\\
9868.07972301136	2.49159025939222e-06	\\
9869.05850497159	1.55610768211346e-06	\\
9870.03728693182	1.73299283553783e-06	\\
9871.01606889204	2.21792619215352e-06	\\
9871.99485085227	1.43252281248932e-06	\\
9872.9736328125	1.64753622731443e-06	\\
9873.95241477273	1.49646746957777e-06	\\
9874.93119673295	9.82012113060812e-07	\\
9875.90997869318	8.5223095308202e-07	\\
9876.88876065341	1.47083651348394e-06	\\
9877.86754261364	1.34985113666536e-06	\\
9878.84632457386	2.04065461622126e-06	\\
9879.82510653409	1.47329966262208e-06	\\
9880.80388849432	1.70214190913209e-06	\\
9881.78267045454	1.80407247168487e-06	\\
9882.76145241477	1.11139450680212e-06	\\
9883.740234375	1.42994265384119e-06	\\
9884.71901633523	1.66569225542018e-06	\\
9885.69779829545	1.0130835567441e-06	\\
9886.67658025568	1.56516650880298e-06	\\
9887.65536221591	2.05555360845719e-06	\\
9888.63414417614	1.4146166860776e-06	\\
9889.61292613636	1.81156089517556e-06	\\
9890.59170809659	1.72115602083781e-06	\\
9891.57049005682	1.0551124893495e-06	\\
9892.54927201704	1.57406101954497e-06	\\
9893.52805397727	1.69384921337365e-06	\\
9894.5068359375	1.07896961402449e-06	\\
9895.48561789773	1.5964876991993e-06	\\
9896.46439985795	1.89292365591278e-06	\\
9897.44318181818	1.56467942445449e-06	\\
9898.42196377841	1.84307002625973e-06	\\
9899.40074573864	2.14522992317272e-06	\\
9900.37952769886	2.11261135382173e-06	\\
9901.35830965909	1.10197515879817e-06	\\
9902.33709161932	1.651545650273e-06	\\
9903.31587357954	2.38194276376568e-06	\\
9904.29465553977	2.57000767548497e-06	\\
9905.2734375	1.27866049866981e-06	\\
9906.25221946023	1.78258992517571e-06	\\
9907.23100142045	2.26555487602633e-06	\\
9908.20978338068	9.23387292746899e-07	\\
9909.18856534091	1.51183658451111e-06	\\
9910.16734730114	2.02940884020239e-06	\\
9911.14612926136	1.32857109926153e-06	\\
9912.12491122159	1.20049132055839e-06	\\
9913.10369318182	1.49754283770706e-06	\\
9914.08247514204	1.86290509039858e-06	\\
9915.06125710227	1.14867434636725e-06	\\
9916.0400390625	1.79659662846958e-06	\\
9917.01882102273	1.86561656186681e-06	\\
9917.99760298295	1.41541985232463e-06	\\
9918.97638494318	1.788908929738e-06	\\
9919.95516690341	2.25042407233974e-06	\\
9920.93394886364	1.71948772846501e-06	\\
9921.91273082386	1.9364533273872e-06	\\
9922.89151278409	1.28950220607963e-06	\\
9923.87029474432	1.6592892762739e-06	\\
9924.84907670454	2.7428053560269e-06	\\
9925.82785866477	1.37447444488919e-06	\\
9926.806640625	1.55596868247202e-06	\\
9927.78542258523	1.49225266112219e-06	\\
9928.76420454545	2.23172366036153e-06	\\
9929.74298650568	2.15317299674227e-06	\\
9930.72176846591	1.89522618830531e-06	\\
9931.70055042614	2.28142243869288e-06	\\
9932.67933238636	2.53030175682976e-06	\\
9933.65811434659	1.86038042335357e-06	\\
9934.63689630682	1.99201068678137e-06	\\
9935.61567826704	2.14149181398831e-06	\\
9936.59446022727	2.70551575190347e-06	\\
9937.5732421875	1.34991605461249e-06	\\
9938.55202414773	1.53515418758475e-06	\\
9939.53080610795	1.14004986776608e-06	\\
9940.50958806818	2.26202987429551e-06	\\
9941.48837002841	1.8784830158598e-06	\\
9942.46715198864	1.6962762223578e-06	\\
9943.44593394886	1.57224551521421e-06	\\
9944.42471590909	1.41479694422185e-06	\\
9945.40349786932	1.35536575022742e-06	\\
9946.38227982954	1.98651975218473e-06	\\
9947.36106178977	1.93151095982084e-06	\\
9948.33984375	1.78470453359896e-06	\\
9949.31862571023	2.25826198599252e-06	\\
9950.29740767045	1.91248891815801e-06	\\
9951.27618963068	1.89680555410271e-06	\\
9952.25497159091	2.09992736121899e-06	\\
9953.23375355114	1.37751186300261e-06	\\
9954.21253551136	1.20678484537295e-06	\\
9955.19131747159	1.81582109759585e-06	\\
9956.17009943182	7.51814638072277e-07	\\
9957.14888139204	1.26527928866907e-06	\\
9958.12766335227	1.8844853130007e-06	\\
9959.1064453125	1.74930424866717e-06	\\
9960.08522727273	1.8620932243118e-06	\\
9961.06400923295	1.81389153347209e-06	\\
9962.04279119318	2.35412117926493e-06	\\
9963.02157315341	1.42842219412778e-06	\\
9964.00035511364	1.91119317478071e-06	\\
9964.97913707386	8.56451329439525e-07	\\
9965.95791903409	1.05880108796178e-06	\\
9966.93670099432	2.18540027211576e-06	\\
9967.91548295454	1.66301716874401e-06	\\
9968.89426491477	1.53966382628731e-06	\\
9969.873046875	1.77281718091066e-06	\\
9970.85182883523	1.36013998817574e-06	\\
9971.83061079545	1.27399499561867e-06	\\
9972.80939275568	1.78998370291424e-06	\\
9973.78817471591	9.27860536225131e-07	\\
9974.76695667614	1.72342207040912e-06	\\
9975.74573863636	1.54764270004178e-06	\\
9976.72452059659	1.72230612336254e-06	\\
9977.70330255682	1.80781896110332e-06	\\
9978.68208451704	1.81134494826327e-06	\\
9979.66086647727	1.33824661184758e-06	\\
9980.6396484375	2.04812674952116e-06	\\
9981.61843039773	1.05286365795666e-06	\\
9982.59721235795	2.06294231607413e-06	\\
9983.57599431818	1.74868194248061e-06	\\
9984.55477627841	1.5407614195102e-06	\\
9985.53355823864	1.79287434910466e-06	\\
9986.51234019886	1.3865683918285e-06	\\
9987.49112215909	6.43325968613317e-07	\\
9988.46990411932	1.39442263018597e-06	\\
9989.44868607954	1.37752299454918e-06	\\
9990.42746803977	1.26647684287726e-06	\\
9991.40625	1.09719177934531e-06	\\
9992.38503196023	1.87563886015108e-06	\\
9993.36381392045	1.47773641055453e-06	\\
9994.34259588068	1.32495571587247e-06	\\
9995.32137784091	1.43770369084974e-06	\\
9996.30015980114	1.3976098947686e-06	\\
9997.27894176136	1.40055164319302e-06	\\
9998.25772372159	1.45280271927434e-06	\\
9999.23650568182	1.57202707577607e-06	\\
10000.215287642	2.4572876060175e-06	\\
10001.1940696023	1.41237672271975e-06	\\
10002.1728515625	1.63406633965975e-06	\\
10003.1516335227	1.69938646059155e-06	\\
10004.130415483	1.4607076999907e-06	\\
10005.1091974432	1.26164773205354e-06	\\
10006.0879794034	6.8417186764418e-07	\\
10007.0667613636	1.32537117358333e-06	\\
10008.0455433239	1.76149276159488e-06	\\
10009.0243252841	1.35028389032687e-06	\\
10010.0031072443	1.39089275312273e-06	\\
10010.9818892045	1.37294009503404e-06	\\
10011.9606711648	1.702483157836e-06	\\
10012.939453125	1.15081577899614e-06	\\
10013.9182350852	1.41996773194045e-06	\\
10014.8970170455	1.48383986926398e-06	\\
10015.8757990057	1.99379715227111e-06	\\
10016.8545809659	1.93346195085642e-06	\\
10017.8333629261	1.63450183684815e-06	\\
10018.8121448864	1.7392549578923e-06	\\
10019.7909268466	1.98558166855525e-06	\\
10020.7697088068	1.21393691706256e-06	\\
10021.748490767	1.53102391651306e-06	\\
10022.7272727273	1.4343296130075e-06	\\
10023.7060546875	1.22190432160141e-06	\\
10024.6848366477	1.59905458741629e-06	\\
10025.663618608	1.83970134058036e-06	\\
10026.6424005682	1.1749654631812e-06	\\
10027.6211825284	1.54885784340586e-06	\\
10028.5999644886	1.28553237227517e-06	\\
10029.5787464489	1.48515465698924e-06	\\
10030.5575284091	2.09006556450372e-06	\\
10031.5363103693	1.47652352527321e-06	\\
10032.5150923295	1.11679015527284e-06	\\
10033.4938742898	1.5940266103133e-06	\\
10034.47265625	1.49411589283414e-06	\\
10035.4514382102	1.030454899718e-06	\\
10036.4302201705	1.08686893910592e-06	\\
10037.4090021307	1.52368850771256e-06	\\
10038.3877840909	1.02937252499369e-06	\\
10039.3665660511	2.32635901640185e-06	\\
10040.3453480114	1.4674507616313e-06	\\
10041.3241299716	2.2003531590873e-06	\\
10042.3029119318	1.79005901559147e-06	\\
10043.281693892	1.43897869214689e-06	\\
10044.2604758523	1.5392244577712e-06	\\
10045.2392578125	2.27362423571298e-06	\\
10046.2180397727	2.09845351934725e-06	\\
10047.196821733	1.14925344464945e-06	\\
10048.1756036932	1.59935833153507e-06	\\
10049.1543856534	1.39572620340714e-06	\\
10050.1331676136	1.14920825199044e-06	\\
10051.1119495739	1.04896829403143e-06	\\
10052.0907315341	1.60429956976115e-06	\\
10053.0695134943	1.84999890201152e-06	\\
10054.0482954545	2.36559526880912e-06	\\
10055.0270774148	1.10378957328089e-06	\\
10056.005859375	1.84071915677085e-06	\\
10056.9846413352	2.07161424143985e-06	\\
10057.9634232955	1.7639160272708e-06	\\
10058.9422052557	1.66781668942868e-06	\\
10059.9209872159	1.69134322313535e-06	\\
10060.8997691761	1.810100227229e-06	\\
10061.8785511364	1.3656758332259e-06	\\
10062.8573330966	1.40361996086677e-06	\\
10063.8361150568	1.03259413255376e-06	\\
10064.814897017	2.10744195599665e-06	\\
10065.7936789773	1.04356705058007e-06	\\
10066.7724609375	1.41231171907538e-06	\\
10067.7512428977	1.33170708897234e-06	\\
10068.730024858	1.30704150456415e-06	\\
10069.7088068182	8.99943650093156e-07	\\
10070.6875887784	2.13344781825145e-06	\\
10071.6663707386	1.6711269303751e-06	\\
10072.6451526989	2.1618340655165e-06	\\
10073.6239346591	1.87695855929041e-06	\\
10074.6027166193	1.87593724469966e-06	\\
10075.5814985795	2.29318830755292e-06	\\
10076.5602805398	1.76997023844188e-06	\\
10077.5390625	2.1488929459564e-06	\\
10078.5178444602	2.07423059926138e-06	\\
10079.4966264205	1.52517799362045e-06	\\
10080.4754083807	2.09591385692122e-06	\\
10081.4541903409	1.96847057959137e-06	\\
10082.4329723011	1.76727373983116e-06	\\
10083.4117542614	1.14262500151097e-06	\\
10084.3905362216	2.68727627394215e-06	\\
10085.3693181818	1.88157332876931e-06	\\
10086.348100142	2.5452652863046e-06	\\
10087.3268821023	2.33822995975794e-06	\\
10088.3056640625	2.32248819469435e-06	\\
10089.2844460227	1.80139803740392e-06	\\
10090.263227983	2.36771722838426e-06	\\
10091.2420099432	1.48374535219234e-06	\\
10092.2207919034	2.25315987886969e-06	\\
10093.1995738636	1.6669750229126e-06	\\
10094.1783558239	2.01704962773827e-06	\\
10095.1571377841	1.69807519971341e-06	\\
10096.1359197443	1.83804923999216e-06	\\
10097.1147017045	1.54871082897008e-06	\\
10098.0934836648	1.84157027968977e-06	\\
10099.072265625	1.81219976080244e-06	\\
10100.0510475852	1.33054384753502e-06	\\
10101.0298295455	1.42501953912516e-06	\\
10102.0086115057	1.35534191510208e-06	\\
10102.9873934659	1.49488321268458e-06	\\
10103.9661754261	1.20842968852438e-06	\\
10104.9449573864	1.25366362973538e-06	\\
10105.9237393466	2.25862931363355e-06	\\
10106.9025213068	1.28141891807604e-06	\\
10107.881303267	1.65773352295257e-06	\\
10108.8600852273	1.67690038591826e-06	\\
10109.8388671875	1.62137758652208e-06	\\
10110.8176491477	2.0721237989898e-06	\\
10111.796431108	2.56525672693557e-06	\\
10112.7752130682	1.72765109014322e-06	\\
10113.7539950284	2.16214017201349e-06	\\
10114.7327769886	2.08782071990064e-06	\\
10115.7115589489	1.71216627188856e-06	\\
10116.6903409091	1.87672081055581e-06	\\
10117.6691228693	1.76099936791243e-06	\\
10118.6479048295	1.29632945950639e-06	\\
10119.6266867898	1.95889276387388e-06	\\
10120.60546875	1.70862135927718e-06	\\
10121.5842507102	1.32168584896414e-06	\\
10122.5630326705	1.89182283737386e-06	\\
10123.5418146307	2.02170029365266e-06	\\
10124.5205965909	1.27594200823243e-06	\\
10125.4993785511	1.77704372829531e-06	\\
10126.4781605114	1.50993518853895e-06	\\
10127.4569424716	1.41162451802147e-06	\\
10128.4357244318	1.62975141904103e-06	\\
10129.414506392	1.84165856557938e-06	\\
10130.3932883523	1.89083999720376e-06	\\
10131.3720703125	1.48854727598078e-06	\\
10132.3508522727	2.23953977244552e-06	\\
10133.329634233	1.62433269247766e-06	\\
10134.3084161932	1.89907918615481e-06	\\
10135.2871981534	1.65071019995429e-06	\\
10136.2659801136	2.03691897869917e-06	\\
10137.2447620739	1.55688645248827e-06	\\
10138.2235440341	1.85785968316452e-06	\\
10139.2023259943	1.98736538094721e-06	\\
10140.1811079545	1.97518576308396e-06	\\
10141.1598899148	1.99834180526714e-06	\\
10142.138671875	1.74783025052219e-06	\\
10143.1174538352	1.80024504880951e-06	\\
10144.0962357955	1.8591776830951e-06	\\
10145.0750177557	2.00633539567598e-06	\\
10146.0537997159	1.56413134039225e-06	\\
10147.0325816761	1.94155642555418e-06	\\
10148.0113636364	1.79150967777806e-06	\\
10148.9901455966	1.64471118796887e-06	\\
10149.9689275568	1.90033313831964e-06	\\
10150.947709517	2.01900204294618e-06	\\
10151.9264914773	1.94104022767843e-06	\\
10152.9052734375	1.92282304297415e-06	\\
10153.8840553977	1.74979059643192e-06	\\
10154.862837358	2.28585171392625e-06	\\
10155.8416193182	1.75863408504621e-06	\\
10156.8204012784	1.54115296679911e-06	\\
10157.7991832386	1.38458127710446e-06	\\
10158.7779651989	1.26477260251454e-06	\\
10159.7567471591	1.83192532109902e-06	\\
10160.7355291193	1.96093566906338e-06	\\
10161.7143110795	1.73005706759515e-06	\\
10162.6930930398	2.07647065693739e-06	\\
10163.671875	1.63871809056413e-06	\\
10164.6506569602	1.40250140163793e-06	\\
10165.6294389205	1.88732274996058e-06	\\
10166.6082208807	2.00303899097837e-06	\\
10167.5870028409	1.5744166338468e-06	\\
10168.5657848011	2.22458790559701e-06	\\
10169.5445667614	1.76955494291631e-06	\\
10170.5233487216	2.3528369299135e-06	\\
10171.5021306818	2.04678758702732e-06	\\
10172.480912642	2.03152445744683e-06	\\
10173.4596946023	2.15699248226619e-06	\\
10174.4384765625	2.47464091675153e-06	\\
10175.4172585227	1.79796458466486e-06	\\
10176.396040483	1.43061661613205e-06	\\
10177.3748224432	1.97695285829321e-06	\\
10178.3536044034	1.29224211208519e-06	\\
10179.3323863636	2.04936764744207e-06	\\
10180.3111683239	1.54050287901814e-06	\\
10181.2899502841	1.22664058181305e-06	\\
10182.2687322443	2.44668170258942e-06	\\
10183.2475142045	1.5606635158403e-06	\\
10184.2262961648	1.7790406574406e-06	\\
10185.205078125	1.95504837247752e-06	\\
10186.1838600852	1.40273960102498e-06	\\
10187.1626420455	1.79509051698951e-06	\\
10188.1414240057	1.75228766860849e-06	\\
10189.1202059659	2.1149492142902e-06	\\
10190.0989879261	1.54204219208685e-06	\\
10191.0777698864	2.54934471669789e-06	\\
10192.0565518466	1.30876125148961e-06	\\
10193.0353338068	1.68967132619387e-06	\\
10194.014115767	1.59030172813498e-06	\\
10194.9928977273	1.69378613223192e-06	\\
10195.9716796875	2.0244702156786e-06	\\
10196.9504616477	2.24459842158997e-06	\\
10197.929243608	2.33692834893597e-06	\\
10198.9080255682	2.21201810593555e-06	\\
10199.8868075284	1.89928072603351e-06	\\
10200.8655894886	1.51349976927675e-06	\\
10201.8443714489	1.98759578984544e-06	\\
10202.8231534091	1.18885338598283e-06	\\
10203.8019353693	1.61206330414218e-06	\\
10204.7807173295	1.55396668571957e-06	\\
10205.7594992898	7.42643307871499e-07	\\
10206.73828125	1.90576808670429e-06	\\
10207.7170632102	2.15085819578797e-06	\\
10208.6958451705	1.07224885950937e-06	\\
10209.6746271307	1.47801289480038e-06	\\
10210.6534090909	1.76170440830488e-06	\\
10211.6321910511	1.32843405096004e-06	\\
10212.6109730114	1.58779898331451e-06	\\
10213.5897549716	1.62702058559893e-06	\\
10214.5685369318	1.60289989790558e-06	\\
10215.547318892	2.0221932907106e-06	\\
10216.5261008523	1.79250427969321e-06	\\
10217.5048828125	1.44350176866637e-06	\\
10218.4836647727	1.43756075509535e-06	\\
10219.462446733	1.7006967381243e-06	\\
10220.4412286932	1.02925609870426e-06	\\
10221.4200106534	1.6358704260609e-06	\\
10222.3987926136	1.2196489588541e-06	\\
10223.3775745739	1.83707728516575e-06	\\
10224.3563565341	1.59439979861285e-06	\\
10225.3351384943	1.51537489292913e-06	\\
10226.3139204545	1.92547047369663e-06	\\
10227.2927024148	1.83582065156745e-06	\\
10228.271484375	2.03872802200006e-06	\\
10229.2502663352	1.70926861629057e-06	\\
10230.2290482955	2.21783557720396e-06	\\
10231.2078302557	1.18909660781325e-06	\\
10232.1866122159	1.80484755675179e-06	\\
10233.1653941761	1.59544248488088e-06	\\
10234.1441761364	1.27399423406514e-06	\\
10235.1229580966	1.40218633850984e-06	\\
10236.1017400568	1.86774336565074e-06	\\
10237.080522017	1.29235445347601e-06	\\
10238.0593039773	2.24378103457532e-06	\\
10239.0380859375	1.31235355154404e-06	\\
10240.0168678977	1.14570874692485e-06	\\
10240.995649858	2.57444032651588e-06	\\
10241.9744318182	1.37952787670656e-06	\\
10242.9532137784	1.56542832094154e-06	\\
10243.9319957386	1.99494429272584e-06	\\
10244.9107776989	1.31650491749884e-06	\\
10245.8895596591	1.90723421943378e-06	\\
10246.8683416193	1.4464428640108e-06	\\
10247.8471235795	1.21232736041134e-06	\\
10248.8259055398	1.80275292313791e-06	\\
10249.8046875	2.26050912699477e-06	\\
10250.7834694602	1.84342959127273e-06	\\
10251.7622514205	1.33957900653683e-06	\\
10252.7410333807	1.59378980967664e-06	\\
10253.7198153409	1.75783377233206e-06	\\
10254.6985973011	1.567877323037e-06	\\
10255.6773792614	1.62888171549363e-06	\\
10256.6561612216	1.19492106510154e-06	\\
10257.6349431818	1.33377051272189e-06	\\
10258.613725142	2.03269692098901e-06	\\
10259.5925071023	1.5545466701202e-06	\\
10260.5712890625	2.01481782465392e-06	\\
10261.5500710227	2.19322632069145e-06	\\
10262.528852983	1.13116630020535e-06	\\
10263.5076349432	1.35438257375095e-06	\\
10264.4864169034	1.78055584810209e-06	\\
10265.4651988636	1.70657752246686e-06	\\
10266.4439808239	1.93285837924309e-06	\\
10267.4227627841	1.71542478156377e-06	\\
10268.4015447443	1.63817785710535e-06	\\
10269.3803267045	2.11549716492843e-06	\\
10270.3591086648	1.65090194585472e-06	\\
10271.337890625	1.95352107092281e-06	\\
10272.3166725852	1.40011299330409e-06	\\
10273.2954545455	1.1591108373823e-06	\\
10274.2742365057	1.42135351734559e-06	\\
10275.2530184659	1.94394119076225e-06	\\
10276.2318004261	1.85434785696593e-06	\\
10277.2105823864	2.02095787318588e-06	\\
10278.1893643466	1.86833061776086e-06	\\
10279.1681463068	1.31336792229135e-06	\\
10280.146928267	2.1862714652184e-06	\\
10281.1257102273	1.51922213789902e-06	\\
10282.1044921875	1.56839599810674e-06	\\
10283.0832741477	1.66042059613721e-06	\\
10284.062056108	1.73538916027072e-06	\\
10285.0408380682	1.40189626690661e-06	\\
10286.0196200284	1.62571758649883e-06	\\
10286.9984019886	1.40662073214753e-06	\\
10287.9771839489	1.08269718153188e-06	\\
10288.9559659091	1.87960481621246e-06	\\
10289.9347478693	1.07409731364523e-06	\\
10290.9135298295	1.22318020868388e-06	\\
10291.8923117898	1.97818177161982e-06	\\
10292.87109375	1.76451205681251e-06	\\
10293.8498757102	1.49756554028129e-06	\\
10294.8286576705	1.39563233707183e-06	\\
10295.8074396307	1.84722519097224e-06	\\
10296.7862215909	1.63463317177082e-06	\\
10297.7650035511	2.33012968412464e-06	\\
10298.7437855114	1.40289627081351e-06	\\
10299.7225674716	1.79438797536231e-06	\\
10300.7013494318	1.93632188525465e-06	\\
10301.680131392	1.26281555540549e-06	\\
10302.6589133523	1.62350716045943e-06	\\
10303.6376953125	1.68640956516544e-06	\\
10304.6164772727	2.04670114336344e-06	\\
10305.595259233	1.55142568297982e-06	\\
10306.5740411932	1.26339969897287e-06	\\
10307.5528231534	1.35933569042787e-06	\\
10308.5316051136	1.60156610100995e-06	\\
10309.5103870739	2.26690959496047e-06	\\
10310.4891690341	1.11600925468009e-06	\\
10311.4679509943	1.75598981472755e-06	\\
10312.4467329545	1.13498288989404e-06	\\
10313.4255149148	1.1906518299569e-06	\\
10314.404296875	1.69292959161427e-06	\\
10315.3830788352	1.39475085896265e-06	\\
10316.3618607955	1.40800341290863e-06	\\
10317.3406427557	1.893017436584e-06	\\
10318.3194247159	1.89004772735831e-06	\\
10319.2982066761	1.89019040892719e-06	\\
10320.2769886364	1.82116850412722e-06	\\
10321.2557705966	1.44438971672969e-06	\\
10322.2345525568	1.40746932830217e-06	\\
10323.213334517	1.74656379452618e-06	\\
10324.1921164773	1.8663253310955e-06	\\
10325.1708984375	1.48058066084758e-06	\\
10326.1496803977	1.32176680191323e-06	\\
10327.128462358	1.57746611093448e-06	\\
10328.1072443182	1.93511727565013e-06	\\
10329.0860262784	1.64203294978426e-06	\\
10330.0648082386	1.39169554008796e-06	\\
10331.0435901989	1.76989225753911e-06	\\
10332.0223721591	1.26675635213519e-06	\\
10333.0011541193	1.37038934520702e-06	\\
10333.9799360795	2.28336860601104e-06	\\
10334.9587180398	1.94715142637092e-06	\\
10335.9375	1.62240964531603e-06	\\
10336.9162819602	1.70947921785809e-06	\\
10337.8950639205	1.30556018860012e-06	\\
10338.8738458807	1.45670003800833e-06	\\
10339.8526278409	1.80959813004777e-06	\\
10340.8314098011	1.54553960837218e-06	\\
10341.8101917614	1.52281230610365e-06	\\
10342.7889737216	1.63268609596052e-06	\\
10343.7677556818	1.65481430151932e-06	\\
10344.746537642	2.04602785677107e-06	\\
10345.7253196023	1.53457136863804e-06	\\
10346.7041015625	1.46589522686462e-06	\\
10347.6828835227	2.09589961244168e-06	\\
10348.661665483	1.46637056278732e-06	\\
10349.6404474432	1.66751678918013e-06	\\
10350.6192294034	1.68763793967536e-06	\\
10351.5980113636	1.17602274468545e-06	\\
10352.5767933239	1.21373783525054e-06	\\
10353.5555752841	1.2954936499962e-06	\\
10354.5343572443	1.14234240652059e-06	\\
10355.5131392045	1.28745802296968e-06	\\
10356.4919211648	1.64110427844474e-06	\\
10357.470703125	1.38795134996618e-06	\\
10358.4494850852	1.86117271617838e-06	\\
10359.4282670455	9.17315704634804e-07	\\
10360.4070490057	1.31641795946221e-06	\\
10361.3858309659	1.25393511816135e-06	\\
10362.3646129261	1.45393261070938e-06	\\
10363.3433948864	1.00018501633363e-06	\\
10364.3221768466	1.34099411010357e-06	\\
10365.3009588068	1.79367164681148e-06	\\
10366.279740767	1.28144695914991e-06	\\
10367.2585227273	1.31427144298718e-06	\\
10368.2373046875	1.62416635119106e-06	\\
10369.2160866477	1.24716749232911e-06	\\
10370.194868608	1.72103930452819e-06	\\
10371.1736505682	1.29905591644365e-06	\\
10372.1524325284	1.49511297451431e-06	\\
10373.1312144886	1.99551012119044e-06	\\
10374.1099964489	1.31056129371033e-06	\\
10375.0887784091	1.77572888543724e-06	\\
10376.0675603693	1.61113381621054e-06	\\
10377.0463423295	1.43001723899398e-06	\\
10378.0251242898	1.18414972797409e-06	\\
10379.00390625	1.10832192796323e-06	\\
10379.9826882102	1.73547778366712e-06	\\
10380.9614701705	1.79169937953518e-06	\\
10381.9402521307	1.9948136669159e-06	\\
10382.9190340909	1.72480917049025e-06	\\
10383.8978160511	1.08666587256066e-06	\\
10384.8765980114	1.51662016987692e-06	\\
10385.8553799716	1.25988700520168e-06	\\
10386.8341619318	1.56500550124786e-06	\\
10387.812943892	1.80187027398858e-06	\\
10388.7917258523	1.45629577999644e-06	\\
10389.7705078125	1.6588296985341e-06	\\
10390.7492897727	1.5037537729569e-06	\\
10391.728071733	1.14246068357966e-06	\\
10392.7068536932	1.53064003819727e-06	\\
10393.6856356534	1.54822173451921e-06	\\
10394.6644176136	1.76237929415876e-06	\\
10395.6431995739	1.54884634785666e-06	\\
10396.6219815341	1.66401634033086e-06	\\
10397.6007634943	1.37163290848173e-06	\\
10398.5795454545	2.03563192195858e-06	\\
10399.5583274148	1.83044769593654e-06	\\
10400.537109375	2.20273848335318e-06	\\
10401.5158913352	1.47760503400137e-06	\\
10402.4946732955	1.46897077331624e-06	\\
10403.4734552557	1.45372379722159e-06	\\
10404.4522372159	1.31635621877249e-06	\\
10405.4310191761	1.64071369952916e-06	\\
10406.4098011364	1.5946104694119e-06	\\
10407.3885830966	2.13200555078696e-06	\\
10408.3673650568	7.61395272325053e-07	\\
10409.346147017	1.58306665744298e-06	\\
10410.3249289773	1.37702829765246e-06	\\
10411.3037109375	1.23963171304834e-06	\\
10412.2824928977	1.7430474364242e-06	\\
10413.261274858	1.33197080257195e-06	\\
10414.2400568182	1.19145406561808e-06	\\
10415.2188387784	1.96342887083718e-06	\\
10416.1976207386	1.37951563431885e-06	\\
10417.1764026989	1.41456189295732e-06	\\
10418.1551846591	1.83228968553934e-06	\\
10419.1339666193	1.30581485794451e-06	\\
10420.1127485795	1.35525579784875e-06	\\
10421.0915305398	1.55245187053658e-06	\\
10422.0703125	1.65282849958927e-06	\\
10423.0490944602	1.20357996959744e-06	\\
10424.0278764205	1.42875867368428e-06	\\
10425.0066583807	1.55454317935813e-06	\\
10425.9854403409	1.19748543234866e-06	\\
10426.9642223011	1.7303787320276e-06	\\
10427.9430042614	1.66384967953221e-06	\\
10428.9217862216	1.61813098631531e-06	\\
10429.9005681818	8.61729833898584e-07	\\
10430.879350142	9.74240153746448e-07	\\
10431.8581321023	1.67231141467689e-06	\\
10432.8369140625	1.13933352360231e-06	\\
10433.8156960227	1.07722698394549e-06	\\
10434.794477983	9.33883325268369e-07	\\
10435.7732599432	1.34784546314927e-06	\\
10436.7520419034	9.41307297513778e-07	\\
10437.7308238636	1.56308787301999e-06	\\
10438.7096058239	9.35755282892836e-07	\\
10439.6883877841	1.27969564819966e-06	\\
10440.6671697443	1.20664400369035e-06	\\
10441.6459517045	9.16606104518415e-07	\\
10442.6247336648	8.37304113216928e-07	\\
10443.603515625	1.8436276431368e-06	\\
10444.5822975852	1.19853157652929e-06	\\
10445.5610795455	9.27841131111081e-07	\\
10446.5398615057	1.45300856780802e-06	\\
10447.5186434659	1.0843537971401e-06	\\
10448.4974254261	1.32845433584008e-06	\\
10449.4762073864	1.56614045394108e-06	\\
10450.4549893466	1.0406574721418e-06	\\
10451.4337713068	9.59531977685145e-07	\\
10452.412553267	1.37515315369505e-06	\\
10453.3913352273	9.77161990244998e-07	\\
10454.3701171875	9.54865512172458e-07	\\
10455.3488991477	1.05243209860579e-06	\\
10456.327681108	7.53630568451334e-07	\\
10457.3064630682	1.41238501821363e-06	\\
10458.2852450284	1.21935265374016e-06	\\
10459.2640269886	1.37533748270484e-06	\\
10460.2428089489	1.08095944031303e-06	\\
10461.2215909091	1.15462338571469e-06	\\
10462.2003728693	1.29264651977366e-06	\\
10463.1791548295	1.46553760129083e-06	\\
10464.1579367898	1.06367604452015e-06	\\
10465.13671875	1.10452562933725e-06	\\
10466.1155007102	1.29635873384947e-06	\\
10467.0942826705	1.5354358909569e-06	\\
10468.0730646307	1.20694919924027e-06	\\
10469.0518465909	1.49117087196611e-06	\\
10470.0306285511	7.08481560633555e-07	\\
10471.0094105114	9.01421005339458e-07	\\
10471.9881924716	1.43013686292969e-06	\\
10472.9669744318	1.17094627019572e-06	\\
10473.945756392	9.83599016104181e-07	\\
10474.9245383523	1.09436576676186e-06	\\
10475.9033203125	1.60360021914298e-06	\\
10476.8821022727	1.84177176023498e-06	\\
10477.860884233	1.82153627247388e-06	\\
10478.8396661932	1.06261607533897e-06	\\
10479.8184481534	1.23565216676767e-06	\\
10480.7972301136	1.53547887826848e-06	\\
10481.7760120739	1.38872678302744e-06	\\
10482.7547940341	1.19023733015013e-06	\\
10483.7335759943	9.60452861740211e-07	\\
10484.7123579545	1.7052427555296e-06	\\
10485.6911399148	1.7390620258724e-06	\\
10486.669921875	1.23560625549172e-06	\\
10487.6487038352	1.54114189439723e-06	\\
10488.6274857955	1.18218826232927e-06	\\
10489.6062677557	9.19746112979523e-07	\\
10490.5850497159	9.20817304765547e-07	\\
10491.5638316761	1.13429677804318e-06	\\
10492.5426136364	1.40861284887983e-06	\\
10493.5213955966	1.43735929816636e-06	\\
10494.5001775568	1.61754857288057e-06	\\
10495.478959517	1.45204561747332e-06	\\
10496.4577414773	1.61141144179997e-06	\\
10497.4365234375	1.39628807018193e-06	\\
10498.4153053977	1.45041049429501e-06	\\
10499.394087358	1.22920177995743e-06	\\
10500.3728693182	1.52011115617783e-06	\\
10501.3516512784	1.05327042260265e-06	\\
10502.3304332386	1.31737214506058e-06	\\
10503.3092151989	8.38583001546729e-07	\\
10504.2879971591	1.19908843903561e-06	\\
10505.2667791193	2.13140801492921e-06	\\
10506.2455610795	1.25792763526968e-06	\\
10507.2243430398	1.66376771936251e-06	\\
10508.203125	1.61764952375548e-06	\\
10509.1819069602	9.13698381152309e-07	\\
10510.1606889205	1.41873846339184e-06	\\
10511.1394708807	1.24045707667511e-06	\\
10512.1182528409	1.57208844883174e-06	\\
10513.0970348011	1.24021408227995e-06	\\
10514.0758167614	1.64327031580275e-06	\\
10515.0545987216	1.17371795018452e-06	\\
10516.0333806818	1.27533792766481e-06	\\
10517.012162642	1.19490705443689e-06	\\
10517.9909446023	1.12308909391826e-06	\\
10518.9697265625	1.51019186079285e-06	\\
10519.9485085227	1.35478700863775e-06	\\
10520.927290483	1.48340546667891e-06	\\
10521.9060724432	1.79284104179365e-06	\\
10522.8848544034	1.08271460800809e-06	\\
10523.8636363636	1.39807002894066e-06	\\
10524.8424183239	1.36213584860658e-06	\\
10525.8212002841	1.18070580432723e-06	\\
10526.7999822443	1.08772312752647e-06	\\
10527.7787642045	1.12265505962915e-06	\\
10528.7575461648	1.17115333904821e-06	\\
10529.736328125	1.5174703063069e-06	\\
10530.7151100852	9.85517702671571e-07	\\
10531.6938920455	9.54030039648895e-07	\\
10532.6726740057	1.23645360918493e-06	\\
10533.6514559659	1.13942264211354e-06	\\
10534.6302379261	1.38845348117079e-06	\\
10535.6090198864	1.40761106310146e-06	\\
10536.5878018466	1.03900207662713e-06	\\
10537.5665838068	1.12326784982441e-06	\\
10538.545365767	1.59447544796527e-06	\\
10539.5241477273	1.18048997628343e-06	\\
10540.5029296875	1.03803668348951e-06	\\
10541.4817116477	1.2978343001519e-06	\\
10542.460493608	1.30750145996755e-06	\\
10543.4392755682	1.29172119591371e-06	\\
10544.4180575284	1.00144025474843e-06	\\
10545.3968394886	1.63657043404615e-06	\\
10546.3756214489	1.20757201758105e-06	\\
10547.3544034091	1.325423066756e-06	\\
10548.3331853693	6.85364491248679e-07	\\
10549.3119673295	1.3181131682949e-06	\\
10550.2907492898	1.4190347646133e-06	\\
10551.26953125	9.90467894703602e-07	\\
10552.2483132102	8.59542286877475e-07	\\
10553.2270951705	1.30042167704746e-06	\\
10554.2058771307	1.51244637337325e-06	\\
10555.1846590909	1.20399993292218e-06	\\
10556.1634410511	8.02858016913831e-07	\\
10557.1422230114	6.63690803561799e-07	\\
10558.1210049716	1.2196593150056e-06	\\
10559.0997869318	1.08778466867816e-06	\\
10560.078568892	3.19491901482362e-07	\\
10561.0573508523	1.11359809668269e-06	\\
10562.0361328125	1.01882648557313e-06	\\
10563.0149147727	8.32403974675008e-07	\\
10563.993696733	1.24636640471882e-06	\\
10564.9724786932	9.9493496023969e-07	\\
10565.9512606534	8.83189165452641e-07	\\
10566.9300426136	1.47793030308563e-06	\\
10567.9088245739	6.36357619213095e-07	\\
10568.8876065341	1.11393481280282e-06	\\
10569.8663884943	1.27097869368886e-06	\\
10570.8451704545	1.06851844650108e-06	\\
10571.8239524148	9.48941978961943e-07	\\
10572.802734375	1.2614524481055e-06	\\
10573.7815163352	1.03079870322302e-06	\\
10574.7602982955	1.05505937170591e-06	\\
10575.7390802557	1.24668460541156e-06	\\
10576.7178622159	1.14224905518916e-06	\\
10577.6966441761	1.11237015129579e-06	\\
10578.6754261364	1.01931578274814e-06	\\
10579.6542080966	8.33021319766494e-07	\\
10580.6329900568	1.14242350253068e-06	\\
10581.611772017	8.79022689900505e-07	\\
10582.5905539773	1.46897048513464e-06	\\
10583.5693359375	9.6584340281318e-07	\\
10584.5481178977	1.21460284084855e-06	\\
10585.526899858	1.00864842735429e-06	\\
10586.5056818182	1.4409532423403e-06	\\
10587.4844637784	1.02468905725001e-06	\\
10588.4632457386	1.08784952150934e-06	\\
10589.4420276989	1.5198388422777e-06	\\
10590.4208096591	8.66751438917859e-07	\\
10591.3995916193	1.08601548181474e-06	\\
10592.3783735795	1.33814281760763e-06	\\
10593.3571555398	1.03864946303928e-06	\\
10594.3359375	1.05433240953476e-06	\\
10595.3147194602	5.06055280343126e-07	\\
10596.2935014205	1.15971917930522e-06	\\
10597.2722833807	1.11889012980423e-06	\\
10598.2510653409	1.38840037427003e-06	\\
10599.2298473011	7.22872342255598e-07	\\
10600.2086292614	9.0960797894706e-07	\\
10601.1874112216	9.42015647653366e-07	\\
10602.1661931818	9.71471333358039e-07	\\
10603.144975142	1.45258351453447e-06	\\
10604.1237571023	1.41015636597106e-06	\\
10605.1025390625	1.19728765756499e-06	\\
10606.0813210227	1.18525222584072e-06	\\
10607.060102983	9.38943116781527e-07	\\
10608.0388849432	1.4207735010112e-06	\\
10609.0176669034	1.30375284687674e-06	\\
10609.9964488636	1.2321937057779e-06	\\
10610.9752308239	1.43484799159533e-06	\\
10611.9540127841	1.19019361958979e-06	\\
10612.9327947443	6.48918097487538e-07	\\
10613.9115767045	1.03359244011182e-06	\\
10614.8903586648	1.16909419859495e-06	\\
10615.869140625	1.01458926753571e-06	\\
10616.8479225852	1.00211251461252e-06	\\
10617.8267045455	1.50593615115216e-06	\\
10618.8054865057	9.35223869244079e-07	\\
10619.7842684659	9.07786753637668e-07	\\
10620.7630504261	1.28588254536182e-06	\\
10621.7418323864	1.25034821278496e-06	\\
10622.7206143466	1.36163414050286e-06	\\
10623.6993963068	1.33340711213548e-06	\\
10624.678178267	9.43210451877194e-07	\\
10625.6569602273	1.20816306826399e-06	\\
10626.6357421875	1.68134617638534e-06	\\
10627.6145241477	1.04221264918413e-06	\\
10628.593306108	1.45713644199684e-06	\\
10629.5720880682	1.67469576781069e-06	\\
10630.5508700284	1.53538726701595e-06	\\
10631.5296519886	1.07991146387811e-06	\\
10632.5084339489	6.99553924543877e-07	\\
10633.4872159091	1.25263022450026e-06	\\
10634.4659978693	1.54085612074183e-06	\\
10635.4447798295	8.14251273460531e-07	\\
10636.4235617898	1.32489087895761e-06	\\
10637.40234375	1.23538656112414e-06	\\
10638.3811257102	8.36237558913689e-07	\\
10639.3599076705	1.14107522092087e-06	\\
10640.3386896307	1.04622196836616e-06	\\
10641.3174715909	9.86484173507697e-07	\\
10642.2962535511	1.67213195542445e-06	\\
10643.2750355114	1.06981373563705e-06	\\
10644.2538174716	8.59318732104206e-07	\\
10645.2325994318	1.35248146145722e-06	\\
10646.211381392	8.77123793627223e-07	\\
10647.1901633523	1.03955651612615e-06	\\
10648.1689453125	1.5920580318005e-06	\\
10649.1477272727	1.30015865384325e-06	\\
10650.126509233	1.23624085348074e-06	\\
10651.1052911932	1.24893784020052e-06	\\
10652.0840731534	8.45431746704895e-07	\\
10653.0628551136	1.17226001421102e-06	\\
10654.0416370739	1.75689678485848e-06	\\
10655.0204190341	1.19555162516852e-06	\\
10655.9992009943	1.01518673720785e-06	\\
10656.9779829545	1.18280287375864e-06	\\
10657.9567649148	7.41052020890727e-07	\\
10658.935546875	1.28313135465562e-06	\\
10659.9143288352	1.34086982362349e-06	\\
10660.8931107955	1.10459848261392e-06	\\
10661.8718927557	1.36286907441947e-06	\\
10662.8506747159	1.0456602518983e-06	\\
10663.8294566761	7.3457988474699e-07	\\
10664.8082386364	8.87205843093113e-07	\\
10665.7870205966	5.20103806841207e-07	\\
10666.7658025568	1.17136631768708e-06	\\
10667.744584517	1.28290109498472e-06	\\
10668.7233664773	1.26114731161669e-06	\\
10669.7021484375	1.00935951723266e-06	\\
10670.6809303977	1.43406267202839e-06	\\
10671.659712358	1.12385021868787e-06	\\
10672.6384943182	7.82433983010385e-07	\\
10673.6172762784	9.07972701734127e-07	\\
10674.5960582386	8.9964962012894e-07	\\
10675.5748401989	1.15199717534385e-06	\\
10676.5536221591	1.13604914166743e-06	\\
10677.5324041193	7.79445952617765e-07	\\
10678.5111860795	1.32940818444415e-06	\\
10679.4899680398	1.40641455173812e-06	\\
10680.46875	7.09693041570424e-07	\\
10681.4475319602	1.22174047458248e-06	\\
10682.4263139205	9.14727396223715e-07	\\
10683.4050958807	1.43690470680728e-06	\\
10684.3838778409	1.23327378579098e-06	\\
10685.3626598011	1.11523537644563e-06	\\
10686.3414417614	1.14978216564901e-06	\\
10687.3202237216	1.34255282217387e-06	\\
10688.2990056818	9.61596515841535e-07	\\
10689.277787642	9.8749644774865e-07	\\
10690.2565696023	1.28451444217955e-06	\\
10691.2353515625	1.05654462082832e-06	\\
10692.2141335227	1.09989014540179e-06	\\
10693.192915483	1.18171794931844e-06	\\
10694.1716974432	1.36007836085971e-06	\\
10695.1504794034	1.11163942058653e-06	\\
10696.1292613636	1.27667724575606e-06	\\
10697.1080433239	1.06715304944772e-06	\\
10698.0868252841	1.31872772539123e-06	\\
10699.0656072443	1.47982096722143e-06	\\
10700.0443892045	1.28981769560089e-06	\\
10701.0231711648	1.06435355867879e-06	\\
10702.001953125	1.00204901564185e-06	\\
10702.9807350852	8.05934928605465e-07	\\
10703.9595170455	1.21409413171094e-06	\\
10704.9382990057	1.5355813332062e-06	\\
10705.9170809659	1.00801082897222e-06	\\
10706.8958629261	1.49532724654364e-06	\\
10707.8746448864	1.10628316403492e-06	\\
10708.8534268466	7.84966039156301e-07	\\
10709.8322088068	1.55858915784376e-06	\\
10710.810990767	1.09678779038173e-06	\\
10711.7897727273	9.87711701996288e-07	\\
10712.7685546875	1.71050563173421e-06	\\
10713.7473366477	8.91558662790105e-07	\\
10714.726118608	9.22554544431068e-07	\\
10715.7049005682	1.42661635425057e-06	\\
10716.6836825284	1.19898167001622e-06	\\
10717.6624644886	1.23776265400079e-06	\\
10718.6412464489	9.79897653635245e-07	\\
10719.6200284091	6.95075847596755e-07	\\
10720.5988103693	1.35707056161204e-06	\\
10721.5775923295	1.35773878731774e-06	\\
10722.5563742898	9.67476640915045e-07	\\
10723.53515625	1.31238244300328e-06	\\
10724.5139382102	9.06345483654522e-07	\\
10725.4927201705	1.1480640749309e-06	\\
10726.4715021307	8.48109122152825e-07	\\
10727.4502840909	1.34387807354341e-06	\\
10728.4290660511	7.50340792460194e-07	\\
10729.4078480114	1.17017060701732e-06	\\
10730.3866299716	6.99607345391125e-07	\\
10731.3654119318	1.55325675150019e-06	\\
10732.344193892	1.09120153545394e-06	\\
10733.3229758523	1.04618094269504e-06	\\
10734.3017578125	9.26998063563484e-07	\\
10735.2805397727	9.33845839505711e-07	\\
10736.259321733	1.23624261945343e-06	\\
10737.2381036932	1.04123438022035e-06	\\
10738.2168856534	9.03953511656978e-07	\\
10739.1956676136	9.14207650926907e-07	\\
10740.1744495739	1.6930262037781e-06	\\
10741.1532315341	9.45292099083923e-07	\\
10742.1320134943	1.38306434156696e-06	\\
10743.1107954545	8.61462845099727e-07	\\
10744.0895774148	9.95719570311e-07	\\
10745.068359375	6.76336927508534e-07	\\
10746.0471413352	1.06170340098428e-06	\\
10747.0259232955	8.84394731261531e-07	\\
10748.0047052557	7.3997410551051e-07	\\
10748.9834872159	1.10816268212752e-06	\\
10749.9622691761	1.37236572005246e-06	\\
10750.9410511364	9.04196254831282e-07	\\
10751.9198330966	1.01017865102175e-06	\\
10752.8986150568	7.57880413938248e-07	\\
10753.877397017	1.30617087552714e-06	\\
10754.8561789773	1.20977176459295e-06	\\
10755.8349609375	4.65411424210861e-07	\\
10756.8137428977	1.01325082918025e-06	\\
10757.792524858	1.15263519191395e-06	\\
10758.7713068182	8.65894976415727e-07	\\
10759.7500887784	8.76914366698789e-07	\\
10760.7288707386	1.42366049393144e-06	\\
10761.7076526989	7.17190980794043e-07	\\
10762.6864346591	9.63737822168003e-07	\\
10763.6652166193	9.9947624968555e-07	\\
10764.6439985795	1.19010378951285e-06	\\
10765.6227805398	1.16049206420452e-06	\\
10766.6015625	1.2167560092127e-06	\\
10767.5803444602	6.80883342381876e-07	\\
10768.5591264205	9.05750985451857e-07	\\
10769.5379083807	4.709715291446e-07	\\
10770.5166903409	7.2142312408998e-07	\\
10771.4954723011	1.12148324744413e-06	\\
10772.4742542614	6.37785862078685e-07	\\
10773.4530362216	9.16509594128108e-07	\\
10774.4318181818	1.19158137404652e-06	\\
10775.410600142	8.31354731682029e-07	\\
10776.3893821023	7.74468086398063e-07	\\
10777.3681640625	8.10103337934255e-07	\\
10778.3469460227	8.98952511418278e-07	\\
10779.325727983	9.23657262390026e-07	\\
10780.3045099432	1.22582955132064e-06	\\
10781.2832919034	5.44522979713383e-07	\\
10782.2620738636	9.68898313441805e-07	\\
10783.2408558239	8.36506521018159e-07	\\
10784.2196377841	8.09724738745168e-07	\\
10785.1984197443	1.03061483464529e-06	\\
10786.1772017045	5.77356820323827e-07	\\
10787.1559836648	8.54530350230112e-07	\\
10788.134765625	1.07456009118104e-06	\\
10789.1135475852	5.90181402721571e-07	\\
10790.0923295455	5.43051366128641e-07	\\
10791.0711115057	8.96473604575744e-07	\\
10792.0498934659	7.07655615680922e-07	\\
10793.0286754261	7.57476181195963e-07	\\
10794.0074573864	1.01134114217704e-06	\\
10794.9862393466	7.07772163894269e-07	\\
10795.9650213068	8.06684882292214e-07	\\
10796.943803267	8.02523872989245e-07	\\
10797.9225852273	8.35296934625328e-07	\\
10798.9013671875	1.26933598138648e-06	\\
10799.8801491477	1.08692498453014e-06	\\
10800.858931108	5.14565524479341e-07	\\
10801.8377130682	1.14714839257254e-06	\\
10802.8164950284	1.11354400266057e-06	\\
10803.7952769886	6.03833451493279e-07	\\
10804.7740589489	1.31913514668045e-06	\\
10805.7528409091	8.825561153143e-07	\\
10806.7316228693	6.00523728449677e-07	\\
10807.7104048295	9.30485714164565e-07	\\
10808.6891867898	1.12425402729904e-06	\\
10809.66796875	5.54750063032532e-07	\\
10810.6467507102	8.84146323731053e-07	\\
10811.6255326705	6.54781615256993e-07	\\
10812.6043146307	7.92109779252718e-07	\\
10813.5830965909	1.33700141751051e-06	\\
10814.5618785511	1.04632859303039e-06	\\
10815.5406605114	1.01097722707375e-06	\\
10816.5194424716	7.36388750865701e-07	\\
10817.4982244318	8.13400820080715e-07	\\
10818.477006392	9.6143918340445e-07	\\
10819.4557883523	7.23185486714909e-07	\\
10820.4345703125	7.14762124818643e-07	\\
10821.4133522727	8.83573166573188e-07	\\
10822.392134233	8.18846617424669e-07	\\
10823.3709161932	7.78130579389848e-07	\\
10824.3496981534	9.14159374308403e-07	\\
10825.3284801136	9.14852668316345e-07	\\
10826.3072620739	5.25117844214487e-07	\\
10827.2860440341	7.77792514166894e-07	\\
10828.2648259943	8.72972670234353e-07	\\
10829.2436079545	5.45457516811204e-07	\\
10830.2223899148	1.14863269873427e-06	\\
10831.201171875	1.0102836998063e-06	\\
10832.1799538352	7.50256905904708e-07	\\
10833.1587357955	6.56654765943114e-07	\\
10834.1375177557	1.30037812644772e-06	\\
10835.1162997159	6.43856315769804e-07	\\
10836.0950816761	8.69408217702968e-07	\\
10837.0738636364	8.74805255483205e-07	\\
10838.0526455966	5.1962433538367e-07	\\
10839.0314275568	1.12462693047474e-06	\\
10840.010209517	1.00311529968671e-06	\\
10840.9889914773	9.66534720249135e-07	\\
10841.9677734375	8.39924187342018e-07	\\
10842.9465553977	5.33542506866333e-07	\\
10843.925337358	1.29378890742223e-06	\\
10844.9041193182	8.37366930109959e-07	\\
10845.8829012784	5.38277121138537e-07	\\
10846.8616832386	5.84661676594614e-07	\\
10847.8404651989	8.11345561115581e-07	\\
10848.8192471591	7.29668679196515e-07	\\
10849.7980291193	7.6362210288186e-07	\\
10850.7768110795	7.77520668048528e-07	\\
10851.7555930398	1.18862774727776e-06	\\
10852.734375	8.07866674918663e-07	\\
10853.7131569602	7.64831085392417e-07	\\
10854.6919389205	6.81610790757814e-07	\\
10855.6707208807	9.57787194635344e-07	\\
10856.6495028409	9.08087968543735e-07	\\
10857.6282848011	7.28965913243878e-07	\\
10858.6070667614	1.04136081214544e-06	\\
10859.5858487216	4.14982119352487e-07	\\
10860.5646306818	7.8816276234212e-07	\\
10861.543412642	6.82744183702614e-07	\\
10862.5221946023	9.78748728674991e-07	\\
10863.5009765625	5.42034368795125e-07	\\
10864.4797585227	8.61352952248241e-07	\\
10865.458540483	6.6917033496074e-07	\\
10866.4373224432	7.52616126902326e-07	\\
10867.4161044034	7.03148208619759e-07	\\
10868.3948863636	4.57794746396119e-07	\\
10869.3736683239	3.07625846824499e-07	\\
10870.3524502841	9.25737245707202e-07	\\
10871.3312322443	6.83779551939578e-07	\\
10872.3100142045	1.20434775744645e-06	\\
10873.2887961648	9.02434324185518e-07	\\
10874.267578125	1.0746633144105e-06	\\
10875.2463600852	5.83447758567659e-07	\\
10876.2251420455	6.97737087051715e-07	\\
10877.2039240057	8.6219441713004e-07	\\
10878.1827059659	1.04789348443228e-06	\\
10879.1614879261	7.54301498784416e-07	\\
10880.1402698864	7.10768654680752e-07	\\
10881.1190518466	1.01597102810192e-06	\\
10882.0978338068	6.65469414498454e-07	\\
10883.076615767	7.75632472687738e-07	\\
10884.0553977273	5.96978916425206e-07	\\
10885.0341796875	7.80029607991756e-07	\\
10886.0129616477	1.03227184208389e-06	\\
10886.991743608	5.24569189994584e-07	\\
10887.9705255682	3.34675979243743e-07	\\
10888.9493075284	1.1449962067618e-06	\\
10889.9280894886	4.83710156002149e-07	\\
10890.9068714489	3.15675630723434e-07	\\
10891.8856534091	6.45249431753882e-07	\\
10892.8644353693	6.67625211319515e-07	\\
10893.8432173295	8.84796051061147e-07	\\
10894.8219992898	9.28224776697126e-07	\\
10895.80078125	6.04795289327291e-07	\\
10896.7795632102	9.74020550319666e-07	\\
10897.7583451705	5.75094674539913e-07	\\
10898.7371271307	8.79266071726998e-07	\\
10899.7159090909	5.56504895820497e-07	\\
10900.6946910511	6.98024467903576e-07	\\
10901.6734730114	5.40993248066768e-07	\\
10902.6522549716	7.72738062376132e-07	\\
10903.6310369318	6.13151045357296e-07	\\
10904.609818892	5.00107642583979e-07	\\
10905.5886008523	4.94411068541262e-07	\\
10906.5673828125	4.12077675006109e-07	\\
10907.5461647727	6.11166942642351e-07	\\
10908.524946733	1.04860099206969e-06	\\
10909.5037286932	6.11313961949503e-07	\\
10910.4825106534	7.85813522141179e-07	\\
10911.4612926136	9.7079133691358e-07	\\
10912.4400745739	5.93732398534381e-07	\\
10913.4188565341	9.54166199072391e-07	\\
10914.3976384943	8.34005539781444e-07	\\
10915.3764204545	6.73292357463926e-07	\\
10916.3552024148	6.88386827450016e-07	\\
10917.333984375	4.42069446563874e-07	\\
10918.3127663352	5.6502654055575e-07	\\
10919.2915482955	7.23267567155897e-07	\\
10920.2703302557	9.71178968898342e-07	\\
10921.2491122159	5.74453449090531e-07	\\
10922.2278941761	5.5617019941668e-07	\\
10923.2066761364	6.50189240837747e-07	\\
10924.1854580966	4.82335893301549e-07	\\
10925.1642400568	7.136025924705e-07	\\
10926.143022017	5.84809908942327e-07	\\
10927.1218039773	8.40033686720347e-07	\\
10928.1005859375	3.08010546676463e-07	\\
10929.0793678977	8.3191988802879e-07	\\
10930.058149858	6.12671770913124e-07	\\
10931.0369318182	6.14189074188719e-07	\\
10932.0157137784	7.83739523268714e-07	\\
10932.9944957386	5.63410282131737e-07	\\
10933.9732776989	5.33447742975312e-07	\\
10934.9520596591	9.95970647008428e-07	\\
10935.9308416193	7.62240904634588e-07	\\
10936.9096235795	8.77806987783109e-07	\\
10937.8884055398	7.81603784225626e-07	\\
10938.8671875	5.01900690314978e-07	\\
10939.8459694602	1.16483709908782e-06	\\
10940.8247514205	6.77967993423971e-07	\\
10941.8035333807	9.90361901995299e-07	\\
10942.7823153409	7.10044209467946e-07	\\
10943.7610973011	1.8757160037276e-07	\\
10944.7398792614	1.34630451782182e-06	\\
10945.7186612216	6.20117002874425e-07	\\
10946.6974431818	5.18714713149959e-07	\\
10947.676225142	9.65620917433811e-07	\\
10948.6550071023	5.79428938278661e-07	\\
10949.6337890625	4.98372140683218e-07	\\
10950.6125710227	7.82266955984564e-07	\\
10951.591352983	5.12917927327227e-07	\\
10952.5701349432	6.84363260892501e-07	\\
10953.5489169034	7.95695279576832e-07	\\
10954.5276988636	5.90347423199851e-07	\\
10955.5064808239	6.76671128449939e-07	\\
10956.4852627841	8.37886263904176e-07	\\
10957.4640447443	4.50652362521072e-07	\\
10958.4428267045	3.66552065076097e-07	\\
10959.4216086648	9.02542396905141e-07	\\
10960.400390625	8.24635045792196e-07	\\
10961.3791725852	8.34753436857455e-07	\\
10962.3579545455	7.45255734575346e-07	\\
10963.3367365057	5.1657578057323e-07	\\
10964.3155184659	8.18638623542357e-07	\\
10965.2943004261	5.14473515938612e-07	\\
10966.2730823864	2.11916579525814e-07	\\
10967.2518643466	7.73821973113947e-07	\\
10968.2306463068	5.11876627730306e-07	\\
10969.209428267	5.42340080135558e-07	\\
10970.1882102273	7.65966906081159e-07	\\
10971.1669921875	6.13264380976273e-07	\\
10972.1457741477	4.41948214412574e-07	\\
10973.124556108	5.23754700393762e-07	\\
10974.1033380682	5.8087407174701e-07	\\
10975.0821200284	8.79022962721898e-07	\\
10976.0609019886	4.8664568252691e-07	\\
10977.0396839489	6.85975822721285e-07	\\
10978.0184659091	5.5411003513513e-07	\\
10978.9972478693	6.24712440753149e-07	\\
10979.9760298295	4.86824334808672e-07	\\
10980.9548117898	3.36985232072936e-07	\\
10981.93359375	2.36304189229501e-07	\\
10982.9123757102	5.77290528450632e-07	\\
10983.8911576705	8.44866672975862e-07	\\
10984.8699396307	6.30514722080956e-07	\\
10985.8487215909	3.54840054170456e-07	\\
10986.8275035511	9.40166585106247e-07	\\
10987.8062855114	7.84796934365969e-07	\\
10988.7850674716	5.70179462943642e-07	\\
10989.7638494318	5.24477945064403e-07	\\
10990.742631392	4.91477176994509e-07	\\
10991.7214133523	9.08856669005153e-07	\\
10992.7001953125	8.69916449273368e-07	\\
10993.6789772727	6.26512081115878e-07	\\
10994.657759233	3.24079661045065e-07	\\
10995.6365411932	6.19461014753136e-07	\\
10996.6153231534	2.67402629858814e-07	\\
10997.5941051136	5.20227982800681e-07	\\
10998.5728870739	2.76759769743835e-07	\\
10999.5516690341	4.45109737944723e-07	\\
11000.5304509943	6.98576254335477e-07	\\
11001.5092329545	4.41515246060294e-07	\\
11002.4880149148	2.90230669240145e-07	\\
11003.466796875	9.29672298585725e-07	\\
11004.4455788352	4.22328497432368e-07	\\
11005.4243607955	6.3371327302889e-07	\\
11006.4031427557	5.62219683936411e-07	\\
11007.3819247159	4.01912947284972e-07	\\
11008.3607066761	4.01454245613196e-07	\\
11009.3394886364	3.31157533927104e-07	\\
11010.3182705966	9.85929460001768e-07	\\
11011.2970525568	6.56794225468298e-07	\\
11012.275834517	6.75651850090316e-07	\\
11013.2546164773	2.6198148557011e-07	\\
11014.2333984375	3.42338821312317e-07	\\
11015.2121803977	9.91893348557525e-07	\\
11016.190962358	3.81838592842965e-07	\\
11017.1697443182	6.8621791813706e-07	\\
11018.1485262784	7.57313195523294e-07	\\
11019.1273082386	7.17533568927806e-07	\\
11020.1060901989	4.28458922583225e-07	\\
11021.0848721591	6.6011513147811e-07	\\
11022.0636541193	5.50907265193604e-07	\\
11023.0424360795	5.10337298220626e-07	\\
11024.0212180398	5.30959412617509e-07	\\
};
\end{axis}
\end{tikzpicture}%
	\caption{Impulse response at Fs\_TX = 4.000 Hz and Fs\_RX = 22.050 Hz.}
	\label{fig:response_2}
\end{figure}

\begin{figure}[H]
	\centering
	\setlength\figureheight{4cm}
    	\setlength\figurewidth{0.8\linewidth}
	% This file was created by matlab2tikz v0.4.6 running on MATLAB 8.2.
% Copyright (c) 2008--2014, Nico Schlömer <nico.schloemer@gmail.com>
% All rights reserved.
% Minimal pgfplots version: 1.3
% 
% The latest updates can be retrieved from
%   http://www.mathworks.com/matlabcentral/fileexchange/22022-matlab2tikz
% where you can also make suggestions and rate matlab2tikz.
% 
\begin{tikzpicture}

\begin{axis}[%
width=\figurewidth,
height=\figureheight,
scale only axis,
xmin=0,
xmax=1,
ymin=-0.4,
ymax=0.4,
name=plot1
]
\addplot [color=blue,solid,forget plot]
  table[row sep=crcr]{
0	-0.00115966796875	\\
0.000122085215480405	-0.002716064453125	\\
0.000244170430960811	-0.00262451171875	\\
0.000366255646441216	-0.00274658203125	\\
0.000488340861921621	-0.0020751953125	\\
0.000610426077402027	-0.00189208984375	\\
0.000732511292882432	-0.002593994140625	\\
0.000854596508362837	-0.00244140625	\\
0.000976681723843243	-0.00189208984375	\\
0.00109876693932365	-0.00177001953125	\\
0.00122085215480405	-0.001312255859375	\\
0.00134293737028446	-0.001129150390625	\\
0.00146502258576486	-0.00128173828125	\\
0.00158710780124527	-0.001007080078125	\\
0.00170919301672567	-0.001007080078125	\\
0.00183127823220608	-0.001190185546875	\\
0.00195336344768649	-0.001220703125	\\
0.00207544866316689	-0.00189208984375	\\
0.0021975338786473	-0.002410888671875	\\
0.0023196190941277	-0.002471923828125	\\
0.00244170430960811	-0.002410888671875	\\
0.00256378952508851	-0.002716064453125	\\
0.00268587474056892	-0.0025634765625	\\
0.00280795995604932	-0.002288818359375	\\
0.00293004517152973	-0.002197265625	\\
0.00305213038701013	-0.00213623046875	\\
0.00317421560249054	-0.002105712890625	\\
0.00329630081797094	-0.001953125	\\
0.00341838603345135	-0.0013427734375	\\
0.00354047124893175	-0.000732421875	\\
0.00366255646441216	-0.00067138671875	\\
0.00378464167989257	-0.00048828125	\\
0.00390672689537297	-0.000457763671875	\\
0.00402881211085338	-0.000518798828125	\\
0.00415089732633378	3.0517578125e-05	\\
0.00427298254181419	-9.1552734375e-05	\\
0.00439506775729459	-0.000640869140625	\\
0.004517152972775	-0.00115966796875	\\
0.0046392381882554	-0.00152587890625	\\
0.00476132340373581	-0.00152587890625	\\
0.00488340861921621	-0.00115966796875	\\
0.00500549383469662	-0.00079345703125	\\
0.00512757905017702	-0.001068115234375	\\
0.00524966426565743	-0.0008544921875	\\
0.00537174948113783	0.000152587890625	\\
0.00549383469661824	0.0008544921875	\\
0.00561591991209864	0.001007080078125	\\
0.00573800512757905	0.001312255859375	\\
0.00586009034305946	0.001800537109375	\\
0.00598217555853986	0.001678466796875	\\
0.00610426077402027	0.0013427734375	\\
0.00622634598950067	0.00128173828125	\\
0.00634843120498108	0.00091552734375	\\
0.00647051642046148	0.000396728515625	\\
0.00659260163594189	-0.00030517578125	\\
0.00671468685142229	-0.00103759765625	\\
0.0068367720669027	-0.0010986328125	\\
0.0069588572823831	-0.001190185546875	\\
0.00708094249786351	-0.0015869140625	\\
0.00720302771334391	-0.001800537109375	\\
0.00732511292882432	-0.002105712890625	\\
0.00744719814430472	-0.002044677734375	\\
0.00756928335978513	-0.001800537109375	\\
0.00769136857526554	-0.00146484375	\\
0.00781345379074594	-0.001220703125	\\
0.00793553900622635	-0.00128173828125	\\
0.00805762422170675	-0.00079345703125	\\
0.00817970943718716	-0.000762939453125	\\
0.00830179465266756	-0.000823974609375	\\
0.00842387986814797	-0.000335693359375	\\
0.00854596508362837	-0.00042724609375	\\
0.00866805029910878	-0.000701904296875	\\
0.00879013551458918	-0.00067138671875	\\
0.00891222073006959	-0.00091552734375	\\
0.00903430594554999	-0.00091552734375	\\
0.0091563911610304	-0.001190185546875	\\
0.0092784763765108	-0.00103759765625	\\
0.00940056159199121	-0.00067138671875	\\
0.00952264680747161	-0.000762939453125	\\
0.00964473202295202	-0.000579833984375	\\
0.00976681723843243	-0.00048828125	\\
0.00988890245391283	-0.000640869140625	\\
0.0100109876693932	-0.000335693359375	\\
0.0101330728848736	-0.00018310546875	\\
0.010255158100354	-0.0003662109375	\\
0.0103772433158345	-6.103515625e-05	\\
0.0104993285313149	0.000213623046875	\\
0.0106214137467953	6.103515625e-05	\\
0.0107434989622757	-0.00018310546875	\\
0.0108655841777561	-0.000152587890625	\\
0.0109876693932365	6.103515625e-05	\\
0.0111097546087169	0.000335693359375	\\
0.0112318398241973	0.00018310546875	\\
0.0113539250396777	0.00030517578125	\\
0.0114760102551581	0.000579833984375	\\
0.0115980954706385	0.000579833984375	\\
0.0117201806861189	0.00067138671875	\\
0.0118422659015993	0.0009765625	\\
0.0119643511170797	0.000946044921875	\\
0.0120864363325601	0.0008544921875	\\
0.0122085215480405	0.000762939453125	\\
0.0123306067635209	0.0006103515625	\\
0.0124526919790013	0.000946044921875	\\
0.0125747771944817	0.00079345703125	\\
0.0126968624099622	0.001068115234375	\\
0.0128189476254426	0.001434326171875	\\
0.012941032840923	0.001434326171875	\\
0.0130631180564034	0.0018310546875	\\
0.0131852032718838	0.00225830078125	\\
0.0133072884873642	0.00262451171875	\\
0.0134293737028446	0.002899169921875	\\
0.013551458918325	0.0025634765625	\\
0.0136735441338054	0.002410888671875	\\
0.0137956293492858	0.002044677734375	\\
0.0139177145647662	0.001922607421875	\\
0.0140397997802466	0.001739501953125	\\
0.014161884995727	0.000823974609375	\\
0.0142839702112074	0.000640869140625	\\
0.0144060554266878	0.000518798828125	\\
0.0145281406421682	0.0008544921875	\\
0.0146502258576486	0.001373291015625	\\
0.014772311073129	0.001556396484375	\\
0.0148943962886095	0.0018310546875	\\
0.0150164815040899	0.0020751953125	\\
0.0151385667195703	0.0028076171875	\\
0.0152606519350507	0.003021240234375	\\
0.0153827371505311	0.003173828125	\\
0.0155048223660115	0.00286865234375	\\
0.0156269075814919	0.00262451171875	\\
0.0157489927969723	0.002593994140625	\\
0.0158710780124527	0.002105712890625	\\
0.0159931632279331	0.001312255859375	\\
0.0161152484434135	0.0008544921875	\\
0.0162373336588939	0.000579833984375	\\
0.0163594188743743	-0.0003662109375	\\
0.0164815040898547	-0.000518798828125	\\
0.0166035893053351	-0.00030517578125	\\
0.0167256745208155	-0.000152587890625	\\
0.0168477597362959	-0.000396728515625	\\
0.0169698449517763	-0.000244140625	\\
0.0170919301672567	0.00042724609375	\\
0.0172140153827372	0.00079345703125	\\
0.0173361005982176	0.0009765625	\\
0.017458185813698	0.001617431640625	\\
0.0175802710291784	0.001434326171875	\\
0.0177023562446588	0.00091552734375	\\
0.0178244414601392	0.00079345703125	\\
0.0179465266756196	0.0006103515625	\\
0.0180686118911	0.000579833984375	\\
0.0181906971065804	-3.0517578125e-05	\\
0.0183127823220608	-0.0003662109375	\\
0.0184348675375412	-0.000640869140625	\\
0.0185569527530216	-0.000640869140625	\\
0.018679037968502	-6.103515625e-05	\\
0.0188011231839824	0.0001220703125	\\
0.0189232083994628	0.0003662109375	\\
0.0190452936149432	0.000885009765625	\\
0.0191673788304236	0.00115966796875	\\
0.019289464045904	0.001220703125	\\
0.0194115492613844	0.001922607421875	\\
0.0195336344768649	0.001800537109375	\\
0.0196557196923453	0.00164794921875	\\
0.0197778049078257	0.001861572265625	\\
0.0198998901233061	0.00146484375	\\
0.0200219753387865	0.001220703125	\\
0.0201440605542669	0.00140380859375	\\
0.0202661457697473	0.0009765625	\\
0.0203882309852277	0.0009765625	\\
0.0205103162007081	0.0009765625	\\
0.0206324014161885	0.000762939453125	\\
0.0207544866316689	0.00054931640625	\\
0.0208765718471493	0.000152587890625	\\
0.0209986570626297	0.000244140625	\\
0.0211207422781101	0.000213623046875	\\
0.0212428274935905	0	\\
0.0213649127090709	-0.00030517578125	\\
0.0214869979245513	0.0003662109375	\\
0.0216090831400317	0.0006103515625	\\
0.0217311683555121	-3.0517578125e-05	\\
0.0218532535709926	-0.00018310546875	\\
0.021975338786473	-0.00054931640625	\\
0.0220974240019534	-0.001007080078125	\\
0.0222195092174338	-0.00115966796875	\\
0.0223415944329142	-0.00152587890625	\\
0.0224636796483946	-0.001251220703125	\\
0.022585764863875	-0.001190185546875	\\
0.0227078500793554	-0.001190185546875	\\
0.0228299352948358	-0.0010986328125	\\
0.0229520205103162	-0.00115966796875	\\
0.0230741057257966	-0.001007080078125	\\
0.023196190941277	-0.000885009765625	\\
0.0233182761567574	-0.000640869140625	\\
0.0234403613722378	-0.000640869140625	\\
0.0235624465877182	-0.000579833984375	\\
0.0236845318031986	-0.00067138671875	\\
0.023806617018679	-0.00091552734375	\\
0.0239287022341594	-0.00115966796875	\\
0.0240507874496399	-0.00164794921875	\\
0.0241728726651203	-0.001861572265625	\\
0.0242949578806007	-0.001495361328125	\\
0.0244170430960811	-0.001434326171875	\\
0.0245391283115615	-0.0010986328125	\\
0.0246612135270419	-9.1552734375e-05	\\
0.0247832987425223	0.0006103515625	\\
0.0249053839580027	0.000885009765625	\\
0.0250274691734831	0.000823974609375	\\
0.0251495543889635	0.000579833984375	\\
0.0252716396044439	0.000274658203125	\\
0.0253937248199243	-0.000152587890625	\\
0.0255158100354047	-0.000244140625	\\
0.0256378952508851	-0.000518798828125	\\
0.0257599804663655	-0.001007080078125	\\
0.0258820656818459	-0.0010986328125	\\
0.0260041508973263	-0.001251220703125	\\
0.0261262361128067	-0.00103759765625	\\
0.0262483213282871	-0.00042724609375	\\
0.0263704065437676	6.103515625e-05	\\
0.026492491759248	0.000244140625	\\
0.0266145769747284	0.00042724609375	\\
0.0267366621902088	0.0006103515625	\\
0.0268587474056892	0.001007080078125	\\
0.0269808326211696	0.00140380859375	\\
0.02710291783665	0.0010986328125	\\
0.0272250030521304	0.000579833984375	\\
0.0273470882676108	3.0517578125e-05	\\
0.0274691734830912	-0.000579833984375	\\
0.0275912586985716	-0.000579833984375	\\
0.027713343914052	-0.0003662109375	\\
0.0278354291295324	-0.0003662109375	\\
0.0279575143450128	-0.000579833984375	\\
0.0280795995604932	-0.000213623046875	\\
0.0282016847759736	0.000244140625	\\
0.028323769991454	0.00042724609375	\\
0.0284458552069344	0.0006103515625	\\
0.0285679404224148	0.00128173828125	\\
0.0286900256378953	0.00177001953125	\\
0.0288121108533757	0.002044677734375	\\
0.0289341960688561	0.001617431640625	\\
0.0290562812843365	0.0015869140625	\\
0.0291783664998169	0.00152587890625	\\
0.0293004517152973	0.00103759765625	\\
0.0294225369307777	0.0008544921875	\\
0.0295446221462581	0.000762939453125	\\
0.0296667073617385	0.00079345703125	\\
0.0297887925772189	0.00067138671875	\\
0.0299108777926993	0.000274658203125	\\
0.0300329630081797	9.1552734375e-05	\\
0.0301550482236601	0.000457763671875	\\
0.0302771334391405	0.000518798828125	\\
0.0303992186546209	0.0008544921875	\\
0.0305213038701013	0.001129150390625	\\
0.0306433890855817	0.001190185546875	\\
0.0307654743010621	0.001739501953125	\\
0.0308875595165425	0.0018310546875	\\
0.031009644732023	0.00164794921875	\\
0.0311317299475034	0.001861572265625	\\
0.0312538151629838	0.0015869140625	\\
0.0313759003784642	0.0015869140625	\\
0.0314979855939446	0.00164794921875	\\
0.031620070809425	0.00103759765625	\\
0.0317421560249054	0.000701904296875	\\
0.0318642412403858	0.000885009765625	\\
0.0319863264558662	0.000762939453125	\\
0.0321084116713466	0.00079345703125	\\
0.032230496886827	0.00091552734375	\\
0.0323525821023074	0.0008544921875	\\
0.0324746673177878	0.000885009765625	\\
0.0325967525332682	0.000946044921875	\\
0.0327188377487486	0.00091552734375	\\
0.032840922964229	0.00115966796875	\\
0.0329630081797094	0.001495361328125	\\
0.0330850933951898	0.001495361328125	\\
0.0332071786106702	0.001220703125	\\
0.0333292638261507	0.001068115234375	\\
0.0334513490416311	0.00103759765625	\\
0.0335734342571115	0.00103759765625	\\
0.0336955194725919	0.00079345703125	\\
0.0338176046880723	0.000762939453125	\\
0.0339396899035527	0.001068115234375	\\
0.0340617751190331	0.0013427734375	\\
0.0341838603345135	0.001800537109375	\\
0.0343059455499939	0.001953125	\\
0.0344280307654743	0.001953125	\\
0.0345501159809547	0.002410888671875	\\
0.0346722011964351	0.00299072265625	\\
0.0347942864119155	0.0028076171875	\\
0.0349163716273959	0.00262451171875	\\
0.0350384568428763	0.00262451171875	\\
0.0351605420583567	0.00238037109375	\\
0.0352826272738371	0.002349853515625	\\
0.0354047124893175	0.002166748046875	\\
0.0355267977047979	0.002044677734375	\\
0.0356488829202784	0.002197265625	\\
0.0357709681357588	0.002288818359375	\\
0.0358930533512392	0.00250244140625	\\
0.0360151385667196	0.002777099609375	\\
0.0361372237822	0.002838134765625	\\
0.0362593089976804	0.00360107421875	\\
0.0363813942131608	0.00390625	\\
0.0365034794286412	0.003509521484375	\\
0.0366255646441216	0.003173828125	\\
0.036747649859602	0.00311279296875	\\
0.0368697350750824	0.003082275390625	\\
0.0369918202905628	0.0028076171875	\\
0.0371139055060432	0.0025634765625	\\
0.0372359907215236	0.002471923828125	\\
0.037358075937004	0.00201416015625	\\
0.0374801611524844	0.00225830078125	\\
0.0376022463679648	0.00213623046875	\\
0.0377243315834452	0.002197265625	\\
0.0378464167989256	0.002532958984375	\\
0.0379685020144061	0.002105712890625	\\
0.0380905872298865	0.0020751953125	\\
0.0382126724453669	0.00244140625	\\
0.0383347576608473	0.002716064453125	\\
0.0384568428763277	0.0025634765625	\\
0.0385789280918081	0.002685546875	\\
0.0387010133072885	0.00286865234375	\\
0.0388230985227689	0.00274658203125	\\
0.0389451837382493	0.002532958984375	\\
0.0390672689537297	0.0025634765625	\\
0.0391893541692101	0.00250244140625	\\
0.0393114393846905	0.00250244140625	\\
0.0394335246001709	0.002227783203125	\\
0.0395556098156513	0.00189208984375	\\
0.0396776950311317	0.001708984375	\\
0.0397997802466121	0.00152587890625	\\
0.0399218654620925	0.001434326171875	\\
0.0400439506775729	0.00140380859375	\\
0.0401660358930533	0.001190185546875	\\
0.0402881211085338	0.001312255859375	\\
0.0404102063240142	0.001800537109375	\\
0.0405322915394946	0.00213623046875	\\
0.040654376754975	0.002288818359375	\\
0.0407764619704554	0.002410888671875	\\
0.0408985471859358	0.00244140625	\\
0.0410206324014162	0.0025634765625	\\
0.0411427176168966	0.002227783203125	\\
0.041264802832377	0.002349853515625	\\
0.0413868880478574	0.002197265625	\\
0.0415089732633378	0.001922607421875	\\
0.0416310584788182	0.00152587890625	\\
0.0417531436942986	0.001434326171875	\\
0.041875228909779	0.001495361328125	\\
0.0419973141252594	0.00152587890625	\\
0.0421193993407398	0.001617431640625	\\
0.0422414845562202	0.00177001953125	\\
0.0423635697717006	0.001678466796875	\\
0.0424856549871811	0.001922607421875	\\
0.0426077402026615	0.002044677734375	\\
0.0427298254181419	0.00189208984375	\\
0.0428519106336223	0.0025634765625	\\
0.0429739958491027	0.00299072265625	\\
0.0430960810645831	0.003143310546875	\\
0.0432181662800635	0.003204345703125	\\
0.0433402514955439	0.00347900390625	\\
0.0434623367110243	0.00360107421875	\\
0.0435844219265047	0.0037841796875	\\
0.0437065071419851	0.00384521484375	\\
0.0438285923574655	0.0035400390625	\\
0.0439506775729459	0.003814697265625	\\
0.0440727627884263	0.003631591796875	\\
0.0441948480039067	0.00384521484375	\\
0.0443169332193871	0.003875732421875	\\
0.0444390184348675	0.0035400390625	\\
0.0445611036503479	0.0035400390625	\\
0.0446831888658283	0.003326416015625	\\
0.0448052740813088	0.00323486328125	\\
0.0449273592967892	0.003448486328125	\\
0.0450494445122696	0.00372314453125	\\
0.04517152972775	0.003875732421875	\\
0.0452936149432304	0.004058837890625	\\
0.0454157001587108	0.00390625	\\
0.0455377853741912	0.00323486328125	\\
0.0456598705896716	0.0032958984375	\\
0.045781955805152	0.002685546875	\\
0.0459040410206324	0.00213623046875	\\
0.0460261262361128	0.00238037109375	\\
0.0461482114515932	0.002349853515625	\\
0.0462702966670736	0.00213623046875	\\
0.046392381882554	0.001922607421875	\\
0.0465144670980344	0.00177001953125	\\
0.0466365523135148	0.00152587890625	\\
0.0467586375289952	0.001434326171875	\\
0.0468807227444756	0.001556396484375	\\
0.0470028079599561	0.001678466796875	\\
0.0471248931754365	0.00115966796875	\\
0.0472469783909169	0.001373291015625	\\
0.0473690636063973	0.001495361328125	\\
0.0474911488218777	0.001251220703125	\\
0.0476132340373581	0.000946044921875	\\
0.0477353192528385	0.0009765625	\\
0.0478574044683189	0.000946044921875	\\
0.0479794896837993	0.001312255859375	\\
0.0481015748992797	0.001373291015625	\\
0.0482236601147601	0.0008544921875	\\
0.0483457453302405	0.0010986328125	\\
0.0484678305457209	0.001434326171875	\\
0.0485899157612013	0.001708984375	\\
0.0487120009766817	0.001953125	\\
0.0488340861921621	0.001922607421875	\\
0.0489561714076425	0.00177001953125	\\
0.0490782566231229	0.002044677734375	\\
0.0492003418386033	0.00177001953125	\\
0.0493224270540838	0.001739501953125	\\
0.0494445122695642	0.0013427734375	\\
0.0495665974850446	0.001129150390625	\\
0.049688682700525	0.001373291015625	\\
0.0498107679160054	0.00103759765625	\\
0.0499328531314858	0.000885009765625	\\
0.0500549383469662	0.00140380859375	\\
0.0501770235624466	0.001983642578125	\\
0.050299108777927	0.002227783203125	\\
0.0504211939934074	0.002288818359375	\\
0.0505432792088878	0.002288818359375	\\
0.0506653644243682	0.00238037109375	\\
0.0507874496398486	0.00286865234375	\\
0.050909534855329	0.002777099609375	\\
0.0510316200708094	0.00238037109375	\\
0.0511537052862898	0.002685546875	\\
0.0512757905017702	0.002349853515625	\\
0.0513978757172506	0.001800537109375	\\
0.051519960932731	0.00164794921875	\\
0.0516420461482115	0.001373291015625	\\
0.0517641313636919	0.001556396484375	\\
0.0518862165791723	0.001983642578125	\\
0.0520083017946527	0.00250244140625	\\
0.0521303870101331	0.003021240234375	\\
0.0522524722256135	0.00347900390625	\\
0.0523745574410939	0.003448486328125	\\
0.0524966426565743	0.003570556640625	\\
0.0526187278720547	0.003936767578125	\\
0.0527408130875351	0.0040283203125	\\
0.0528628983030155	0.004150390625	\\
0.0529849835184959	0.00347900390625	\\
0.0531070687339763	0.003082275390625	\\
0.0532291539494567	0.002655029296875	\\
0.0533512391649371	0.001922607421875	\\
0.0534733243804175	0.001617431640625	\\
0.0535954095958979	0.001678466796875	\\
0.0537174948113783	0.001861572265625	\\
0.0538395800268587	0.002166748046875	\\
0.0539616652423392	0.002777099609375	\\
0.0540837504578196	0.002716064453125	\\
0.0542058356733	0.00323486328125	\\
0.0543279208887804	0.003570556640625	\\
0.0544500061042608	0.0032958984375	\\
0.0545720913197412	0.0037841796875	\\
0.0546941765352216	0.00408935546875	\\
0.054816261750702	0.004150390625	\\
0.0549383469661824	0.00372314453125	\\
0.0550604321816628	0.00372314453125	\\
0.0551825173971432	0.003204345703125	\\
0.0553046026126236	0.00262451171875	\\
0.055426687828104	0.002227783203125	\\
0.0555487730435844	0.002349853515625	\\
0.0556708582590648	0.00213623046875	\\
0.0557929434745452	0.00189208984375	\\
0.0559150286900256	0.001953125	\\
0.056037113905506	0.001983642578125	\\
0.0561591991209865	0.002166748046875	\\
0.0562812843364669	0.00201416015625	\\
0.0564033695519473	0.001922607421875	\\
0.0565254547674277	0.00201416015625	\\
0.0566475399829081	0.002655029296875	\\
0.0567696251983885	0.00244140625	\\
0.0568917104138689	0.002044677734375	\\
0.0570137956293493	0.002044677734375	\\
0.0571358808448297	0.001953125	\\
0.0572579660603101	0.00140380859375	\\
0.0573800512757905	0.0013427734375	\\
0.0575021364912709	0.000823974609375	\\
0.0576242217067513	0.000579833984375	\\
0.0577463069222317	0.000885009765625	\\
0.0578683921377121	0.000457763671875	\\
0.0579904773531925	0.00018310546875	\\
0.0581125625686729	0.00048828125	\\
0.0582346477841533	0.00067138671875	\\
0.0583567329996337	0.0010986328125	\\
0.0584788182151142	0.001251220703125	\\
0.0586009034305946	0.00140380859375	\\
0.058722988646075	0.001617431640625	\\
0.0588450738615554	0.00177001953125	\\
0.0589671590770358	0.002166748046875	\\
0.0590892442925162	0.0020751953125	\\
0.0592113295079966	0.001617431640625	\\
0.059333414723477	0.00152587890625	\\
0.0594554999389574	0.001129150390625	\\
0.0595775851544378	0.00091552734375	\\
0.0596996703699182	0.00115966796875	\\
0.0598217555853986	0.001373291015625	\\
0.059943840800879	0.00128173828125	\\
0.0600659260163594	0.00177001953125	\\
0.0601880112318398	0.002105712890625	\\
0.0603100964473202	0.002105712890625	\\
0.0604321816628006	0.0023193359375	\\
0.060554266878281	0.00244140625	\\
0.0606763520937614	0.00244140625	\\
0.0607984373092419	0.002471923828125	\\
0.0609205225247223	0.0023193359375	\\
0.0610426077402027	0.001953125	\\
0.0611646929556831	0.001739501953125	\\
0.0612867781711635	0.001434326171875	\\
0.0614088633866439	0.00140380859375	\\
0.0615309486021243	0.00164794921875	\\
0.0616530338176047	0.001983642578125	\\
0.0617751190330851	0.00238037109375	\\
0.0618972042485655	0.00262451171875	\\
0.0620192894640459	0.003021240234375	\\
0.0621413746795263	0.003173828125	\\
0.0622634598950067	0.003173828125	\\
0.0623855451104871	0.00335693359375	\\
0.0625076303259675	0.00311279296875	\\
0.0626297155414479	0.002471923828125	\\
0.0627518007569283	0.002532958984375	\\
0.0628738859724087	0.001983642578125	\\
0.0629959711878891	0.001983642578125	\\
0.0631180564033696	0.001739501953125	\\
0.06324014161885	0.001373291015625	\\
0.0633622268343304	0.001251220703125	\\
0.0634843120498108	0.00115966796875	\\
0.0636063972652912	0.001739501953125	\\
0.0637284824807716	0.001953125	\\
0.063850567696252	0.002349853515625	\\
0.0639726529117324	0.001983642578125	\\
0.0640947381272128	0.00189208984375	\\
0.0642168233426932	0.002166748046875	\\
0.0643389085581736	0.002166748046875	\\
0.064460993773654	0.0023193359375	\\
0.0645830789891344	0.00177001953125	\\
0.0647051642046148	0.0018310546875	\\
0.0648272494200952	0.001983642578125	\\
0.0649493346355756	0.001678466796875	\\
0.065071419851056	0.001220703125	\\
0.0651935050665364	0.000640869140625	\\
0.0653155902820168	0.00067138671875	\\
0.0654376754974973	0.000213623046875	\\
0.0655597607129777	-0.00018310546875	\\
0.0656818459284581	0.00018310546875	\\
0.0658039311439385	0.0006103515625	\\
0.0659260163594189	0.00042724609375	\\
0.0660481015748993	6.103515625e-05	\\
0.0661701867903797	0.000457763671875	\\
0.0662922720058601	0.00067138671875	\\
0.0664143572213405	0.00054931640625	\\
0.0665364424368209	0.000946044921875	\\
0.0666585276523013	0.001007080078125	\\
0.0667806128677817	0.000946044921875	\\
0.0669026980832621	0.000640869140625	\\
0.0670247832987425	0.00042724609375	\\
0.0671468685142229	0.00048828125	\\
0.0672689537297033	0.00030517578125	\\
0.0673910389451837	0.00030517578125	\\
0.0675131241606641	0	\\
0.0676352093761445	-9.1552734375e-05	\\
0.067757294591625	6.103515625e-05	\\
0.0678793798071054	0.000396728515625	\\
0.0680014650225858	0.0006103515625	\\
0.0681235502380662	0.0009765625	\\
0.0682456354535466	0.000823974609375	\\
0.068367720669027	0.0006103515625	\\
0.0684898058845074	0.0008544921875	\\
0.0686118910999878	0.00067138671875	\\
0.0687339763154682	0.000946044921875	\\
0.0688560615309486	0.00103759765625	\\
0.068978146746429	0.000885009765625	\\
0.0691002319619094	0.000701904296875	\\
0.0692223171773898	0.000396728515625	\\
0.0693444023928702	0.00048828125	\\
0.0694664876083506	0.000335693359375	\\
0.069588572823831	0.000274658203125	\\
0.0697106580393114	0.000640869140625	\\
0.0698327432547918	0.00091552734375	\\
0.0699548284702722	0.00091552734375	\\
0.0700769136857527	0.001129150390625	\\
0.0701989989012331	0.001007080078125	\\
0.0703210841167135	0.0008544921875	\\
0.0704431693321939	0.001129150390625	\\
0.0705652545476743	0.00103759765625	\\
0.0706873397631547	0.00054931640625	\\
0.0708094249786351	0.0001220703125	\\
0.0709315101941155	-0.000244140625	\\
0.0710535954095959	-0.000396728515625	\\
0.0711756806250763	-0.00067138671875	\\
0.0712977658405567	-0.000457763671875	\\
0.0714198510560371	-0.0003662109375	\\
0.0715419362715175	-0.00030517578125	\\
0.0716640214869979	0.00018310546875	\\
0.0717861067024783	0.0009765625	\\
0.0719081919179587	0.001251220703125	\\
0.0720302771334391	0.000518798828125	\\
0.0721523623489195	9.1552734375e-05	\\
0.0722744475643999	0.0001220703125	\\
0.0723965327798804	0.000152587890625	\\
0.0725186179953608	3.0517578125e-05	\\
0.0726407032108412	-0.00048828125	\\
0.0727627884263216	-0.0009765625	\\
0.072884873641802	-0.00128173828125	\\
0.0730069588572824	-0.001861572265625	\\
0.0731290440727628	-0.001617431640625	\\
0.0732511292882432	-0.00177001953125	\\
0.0733732145037236	-0.001617431640625	\\
0.073495299719204	-0.0010986328125	\\
0.0736173849346844	-0.001007080078125	\\
0.0737394701501648	-0.0006103515625	\\
0.0738615553656452	-0.00030517578125	\\
0.0739836405811256	0.00018310546875	\\
0.074105725796606	0.00030517578125	\\
0.0742278110120864	0.000518798828125	\\
0.0743498962275668	0.00030517578125	\\
0.0744719814430472	9.1552734375e-05	\\
0.0745940666585277	-0.0001220703125	\\
0.0747161518740081	-0.000244140625	\\
0.0748382370894885	-0.00054931640625	\\
0.0749603223049689	-0.001129150390625	\\
0.0750824075204493	-0.001312255859375	\\
0.0752044927359297	-0.001556396484375	\\
0.0753265779514101	-0.001373291015625	\\
0.0754486631668905	-0.001129150390625	\\
0.0755707483823709	-0.000823974609375	\\
0.0756928335978513	-0.000762939453125	\\
0.0758149188133317	-0.00115966796875	\\
0.0759370040288121	-0.000457763671875	\\
0.0760590892442925	-0.000152587890625	\\
0.0761811744597729	-0.0003662109375	\\
0.0763032596752533	-0.000335693359375	\\
0.0764253448907337	-0.000396728515625	\\
0.0765474301062141	-3.0517578125e-05	\\
0.0766695153216945	0.000244140625	\\
0.0767916005371749	6.103515625e-05	\\
0.0769136857526554	0.000213623046875	\\
0.0770357709681358	0.000396728515625	\\
0.0771578561836162	3.0517578125e-05	\\
0.0772799413990966	-9.1552734375e-05	\\
0.077402026614577	3.0517578125e-05	\\
0.0775241118300574	-3.0517578125e-05	\\
0.0776461970455378	3.0517578125e-05	\\
0.0777682822610182	0.00054931640625	\\
0.0778903674764986	0.0010986328125	\\
0.078012452691979	0.001312255859375	\\
0.0781345379074594	0.001617431640625	\\
0.0782566231229398	0.00140380859375	\\
0.0783787083384202	0.00091552734375	\\
0.0785007935539006	0.000701904296875	\\
0.078622878769381	0.000518798828125	\\
0.0787449639848614	-9.1552734375e-05	\\
0.0788670492003418	-0.0003662109375	\\
0.0789891344158222	3.0517578125e-05	\\
0.0791112196313026	6.103515625e-05	\\
0.0792333048467831	0.0001220703125	\\
0.0793553900622635	0.000274658203125	\\
0.0794774752777439	0.000640869140625	\\
0.0795995604932243	0.00128173828125	\\
0.0797216457087047	0.0018310546875	\\
0.0798437309241851	0.001953125	\\
0.0799658161396655	0.001983642578125	\\
0.0800879013551459	0.001068115234375	\\
0.0802099865706263	0.000732421875	\\
0.0803320717861067	0.000457763671875	\\
0.0804541570015871	-6.103515625e-05	\\
0.0805762422170675	-0.000396728515625	\\
0.0806983274325479	-0.00054931640625	\\
0.0808204126480283	-0.000762939453125	\\
0.0809424978635087	-0.001007080078125	\\
0.0810645830789891	-0.00067138671875	\\
0.0811866682944695	-0.000732421875	\\
0.0813087535099499	-0.0010986328125	\\
0.0814308387254303	-0.001007080078125	\\
0.0815529239409108	-0.000518798828125	\\
0.0816750091563912	-0.000335693359375	\\
0.0817970943718716	-0.00042724609375	\\
0.081919179587352	-0.00018310546875	\\
0.0820412648028324	0.0001220703125	\\
0.0821633500183128	0.000152587890625	\\
0.0822854352337932	-3.0517578125e-05	\\
0.0824075204492736	-0.0003662109375	\\
0.082529605664754	-0.00054931640625	\\
0.0826516908802344	-0.000518798828125	\\
0.0827737760957148	-0.00067138671875	\\
0.0828958613111952	-0.0006103515625	\\
0.0830179465266756	-0.00030517578125	\\
0.083140031742156	6.103515625e-05	\\
0.0832621169576364	0.00054931640625	\\
0.0833842021731168	0.00115966796875	\\
0.0835062873885972	0.001556396484375	\\
0.0836283726040776	0.00152587890625	\\
0.083750457819558	0.0020751953125	\\
0.0838725430350385	0.0025634765625	\\
0.0839946282505189	0.002288818359375	\\
0.0841167134659993	0.002471923828125	\\
0.0842387986814797	0.00244140625	\\
0.0843608838969601	0.002044677734375	\\
0.0844829691124405	0.002166748046875	\\
0.0846050543279209	0.00201416015625	\\
0.0847271395434013	0.001922607421875	\\
0.0848492247588817	0.001983642578125	\\
0.0849713099743621	0.001708984375	\\
0.0850933951898425	0.001617431640625	\\
0.0852154804053229	0.00213623046875	\\
0.0853375656208033	0.0020751953125	\\
0.0854596508362837	0.00177001953125	\\
0.0855817360517641	0.001678466796875	\\
0.0857038212672445	0.00201416015625	\\
0.0858259064827249	0.002105712890625	\\
0.0859479916982053	0.001800537109375	\\
0.0860700769136857	0.00238037109375	\\
0.0861921621291662	0.002044677734375	\\
0.0863142473446466	0.001495361328125	\\
0.086436332560127	0.001434326171875	\\
0.0865584177756074	0.0010986328125	\\
0.0866805029910878	0.001251220703125	\\
0.0868025882065682	0.00140380859375	\\
0.0869246734220486	0.00146484375	\\
0.087046758637529	0.00140380859375	\\
0.0871688438530094	0.001251220703125	\\
0.0872909290684898	0.00115966796875	\\
0.0874130142839702	0.001220703125	\\
0.0875350994994506	0.0013427734375	\\
0.087657184714931	0.001495361328125	\\
0.0877792699304114	0.001495361328125	\\
0.0879013551458918	0.001251220703125	\\
0.0880234403613722	0.000732421875	\\
0.0881455255768526	0.000396728515625	\\
0.088267610792333	0	\\
0.0883896960078134	-0.000579833984375	\\
0.0885117812232939	-0.000762939453125	\\
0.0886338664387743	-0.000640869140625	\\
0.0887559516542547	-0.00079345703125	\\
0.0888780368697351	-0.000946044921875	\\
0.0890001220852155	-0.00146484375	\\
0.0891222073006959	-0.001251220703125	\\
0.0892442925161763	-0.00079345703125	\\
0.0893663777316567	-0.000396728515625	\\
0.0894884629471371	-0.00030517578125	\\
0.0896105481626175	-0.00054931640625	\\
0.0897326333780979	-0.000335693359375	\\
0.0898547185935783	-0.00067138671875	\\
0.0899768038090587	-0.001068115234375	\\
0.0900988890245391	-0.001251220703125	\\
0.0902209742400195	-0.00146484375	\\
0.0903430594554999	-0.00140380859375	\\
0.0904651446709803	-0.001678466796875	\\
0.0905872298864607	-0.001739501953125	\\
0.0907093151019411	-0.0015869140625	\\
0.0908314003174216	-0.00164794921875	\\
0.090953485532902	-0.001556396484375	\\
0.0910755707483824	-0.001556396484375	\\
0.0911976559638628	-0.00128173828125	\\
0.0913197411793432	-0.00115966796875	\\
0.0914418263948236	-0.00091552734375	\\
0.091563911610304	-0.0008544921875	\\
0.0916859968257844	-0.000946044921875	\\
0.0918080820412648	-0.00091552734375	\\
0.0919301672567452	-0.001129150390625	\\
0.0920522524722256	-0.001495361328125	\\
0.092174337687706	-0.001739501953125	\\
0.0922964229031864	-0.00140380859375	\\
0.0924185081186668	-0.001739501953125	\\
0.0925405933341472	-0.002471923828125	\\
0.0926626785496276	-0.00238037109375	\\
0.092784763765108	-0.0020751953125	\\
0.0929068489805884	-0.00177001953125	\\
0.0930289341960688	-0.001312255859375	\\
0.0931510194115493	-0.00146484375	\\
0.0932731046270297	-0.00128173828125	\\
0.0933951898425101	-0.000732421875	\\
0.0935172750579905	-0.000640869140625	\\
0.0936393602734709	-0.000823974609375	\\
0.0937614454889513	-0.000885009765625	\\
0.0938835307044317	-0.000885009765625	\\
0.0940056159199121	-0.000732421875	\\
0.0941277011353925	-0.000762939453125	\\
0.0942497863508729	-0.000762939453125	\\
0.0943718715663533	-0.000640869140625	\\
0.0944939567818337	-0.000885009765625	\\
0.0946160419973141	-0.000823974609375	\\
0.0947381272127945	-0.0009765625	\\
0.0948602124282749	-0.000518798828125	\\
0.0949822976437553	-0.00048828125	\\
0.0951043828592358	-0.00091552734375	\\
0.0952264680747162	-0.000823974609375	\\
0.0953485532901966	-0.00115966796875	\\
0.095470638505677	-0.001495361328125	\\
0.0955927237211574	-0.001556396484375	\\
0.0957148089366378	-0.00152587890625	\\
0.0958368941521182	-0.00128173828125	\\
0.0959589793675986	-0.001556396484375	\\
0.096081064583079	-0.001953125	\\
0.0962031497985594	-0.001800537109375	\\
0.0963252350140398	-0.001953125	\\
0.0964473202295202	-0.001953125	\\
0.0965694054450006	-0.002197265625	\\
0.096691490660481	-0.002044677734375	\\
0.0968135758759614	-0.001312255859375	\\
0.0969356610914418	-0.001495361328125	\\
0.0970577463069222	-0.00189208984375	\\
0.0971798315224026	-0.0018310546875	\\
0.097301916737883	-0.001617431640625	\\
0.0974240019533635	-0.001312255859375	\\
0.0975460871688439	-0.001007080078125	\\
0.0976681723843243	-0.001007080078125	\\
0.0977902575998047	-0.00164794921875	\\
0.0979123428152851	-0.00177001953125	\\
0.0980344280307655	-0.001708984375	\\
0.0981565132462459	-0.002044677734375	\\
0.0982785984617263	-0.002777099609375	\\
0.0984006836772067	-0.00274658203125	\\
0.0985227688926871	-0.00286865234375	\\
0.0986448541081675	-0.00335693359375	\\
0.0987669393236479	-0.00299072265625	\\
0.0988890245391283	-0.00299072265625	\\
0.0990111097546087	-0.002593994140625	\\
0.0991331949700891	-0.001983642578125	\\
0.0992552801855695	-0.001617431640625	\\
0.0993773654010499	-0.001739501953125	\\
0.0994994506165303	-0.00201416015625	\\
0.0996215358320107	-0.00152587890625	\\
0.0997436210474912	-0.001373291015625	\\
0.0998657062629716	-0.001617431640625	\\
0.099987791478452	-0.00213623046875	\\
0.100109876693932	-0.002197265625	\\
0.100231961909413	-0.001861572265625	\\
0.100354047124893	-0.00189208984375	\\
0.100476132340374	-0.001953125	\\
0.100598217555854	-0.001495361328125	\\
0.100720302771334	-0.001190185546875	\\
0.100842387986815	-0.001190185546875	\\
0.100964473202295	-0.00152587890625	\\
0.101086558417776	-0.00152587890625	\\
0.101208643633256	-0.00128173828125	\\
0.101330728848736	-0.001251220703125	\\
0.101452814064217	-0.001007080078125	\\
0.101574899279697	-0.000701904296875	\\
0.101696984495178	-0.000732421875	\\
0.101819069710658	-0.0008544921875	\\
0.101941154926138	-0.0013427734375	\\
0.102063240141619	-0.00152587890625	\\
0.102185325357099	-0.00146484375	\\
0.10230741057258	-0.001983642578125	\\
0.10242949578806	-0.00213623046875	\\
0.10255158100354	-0.0018310546875	\\
0.102673666219021	-0.001739501953125	\\
0.102795751434501	-0.001708984375	\\
0.102917836649982	-0.0015869140625	\\
0.103039921865462	-0.00128173828125	\\
0.103162007080943	-0.0008544921875	\\
0.103284092296423	-0.000579833984375	\\
0.103406177511903	-0.0003662109375	\\
0.103528262727384	-0.000396728515625	\\
0.103650347942864	-0.00042724609375	\\
0.103772433158345	-0.000396728515625	\\
0.103894518373825	-0.000396728515625	\\
0.104016603589305	-0.00048828125	\\
0.104138688804786	-0.000457763671875	\\
0.104260774020266	-0.001251220703125	\\
0.104382859235747	-0.001739501953125	\\
0.104504944451227	-0.0013427734375	\\
0.104627029666707	-0.001068115234375	\\
0.104749114882188	-0.000732421875	\\
0.104871200097668	-0.0006103515625	\\
0.104993285313149	-0.000762939453125	\\
0.105115370528629	-0.0006103515625	\\
0.105237455744109	-0.000732421875	\\
0.10535954095959	-0.00048828125	\\
0.10548162617507	9.1552734375e-05	\\
0.105603711390551	-0.00079345703125	\\
0.105725796606031	-0.00128173828125	\\
0.105847881821511	-0.001373291015625	\\
0.105969967036992	-0.001312255859375	\\
0.106092052252472	-0.001556396484375	\\
0.106214137467953	-0.001800537109375	\\
0.106336222683433	-0.00164794921875	\\
0.106458307898913	-0.001861572265625	\\
0.106580393114394	-0.001708984375	\\
0.106702478329874	-0.001800537109375	\\
0.106824563545355	-0.001617431640625	\\
0.106946648760835	-0.001251220703125	\\
0.107068733976315	-0.001007080078125	\\
0.107190819191796	-0.000823974609375	\\
0.107312904407276	-0.000640869140625	\\
0.107434989622757	-0.00042724609375	\\
0.107557074838237	-0.00079345703125	\\
0.107679160053718	-0.0009765625	\\
0.107801245269198	-0.000885009765625	\\
0.107923330484678	-0.000946044921875	\\
0.108045415700159	-0.001373291015625	\\
0.108167500915639	-0.00128173828125	\\
0.10828958613112	-0.0013427734375	\\
0.1084116713466	-0.001556396484375	\\
0.10853375656208	-0.0013427734375	\\
0.108655841777561	-0.00115966796875	\\
0.108777926993041	-0.001007080078125	\\
0.108900012208522	-0.000732421875	\\
0.109022097424002	-0.000244140625	\\
0.109144182639482	-0.000396728515625	\\
0.109266267854963	-0.00030517578125	\\
0.109388353070443	-3.0517578125e-05	\\
0.109510438285924	-0.0001220703125	\\
0.109632523501404	-0.000396728515625	\\
0.109754608716884	-0.000213623046875	\\
0.109876693932365	-0.000274658203125	\\
0.109998779147845	-0.000823974609375	\\
0.110120864363326	-0.000946044921875	\\
0.110242949578806	-0.00140380859375	\\
0.110365034794286	-0.001251220703125	\\
0.110487120009767	-0.00103759765625	\\
0.110609205225247	-0.001312255859375	\\
0.110731290440728	-0.0008544921875	\\
0.110853375656208	-0.00103759765625	\\
0.110975460871688	-0.001068115234375	\\
0.111097546087169	-0.000762939453125	\\
0.111219631302649	-0.000946044921875	\\
0.11134171651813	-0.0009765625	\\
0.11146380173361	-0.00115966796875	\\
0.11158588694909	-0.00152587890625	\\
0.111707972164571	-0.001708984375	\\
0.111830057380051	-0.001434326171875	\\
0.111952142595532	-0.001739501953125	\\
0.112074227811012	-0.002166748046875	\\
0.112196313026492	-0.0023193359375	\\
0.112318398241973	-0.0025634765625	\\
0.112440483457453	-0.00250244140625	\\
0.112562568672934	-0.0023193359375	\\
0.112684653888414	-0.001739501953125	\\
0.112806739103895	-0.00152587890625	\\
0.112928824319375	-0.001373291015625	\\
0.113050909534855	-0.00115966796875	\\
0.113172994750336	-0.0010986328125	\\
0.113295079965816	-0.001220703125	\\
0.113417165181297	-0.001495361328125	\\
0.113539250396777	-0.00140380859375	\\
0.113661335612257	-0.00128173828125	\\
0.113783420827738	-0.0018310546875	\\
0.113905506043218	-0.001983642578125	\\
0.114027591258699	-0.0018310546875	\\
0.114149676474179	-0.0020751953125	\\
0.114271761689659	-0.002899169921875	\\
0.11439384690514	-0.002685546875	\\
0.11451593212062	-0.0023193359375	\\
0.114638017336101	-0.00225830078125	\\
0.114760102551581	-0.00213623046875	\\
0.114882187767061	-0.001800537109375	\\
0.115004272982542	-0.001678466796875	\\
0.115126358198022	-0.001922607421875	\\
0.115248443413503	-0.001800537109375	\\
0.115370528628983	-0.001739501953125	\\
0.115492613844463	-0.001953125	\\
0.115614699059944	-0.001708984375	\\
0.115736784275424	-0.002166748046875	\\
0.115858869490905	-0.002532958984375	\\
0.115980954706385	-0.002197265625	\\
0.116103039921865	-0.002288818359375	\\
0.116225125137346	-0.002227783203125	\\
0.116347210352826	-0.002288818359375	\\
0.116469295568307	-0.002197265625	\\
0.116591380783787	-0.001800537109375	\\
0.116713465999267	-0.001312255859375	\\
0.116835551214748	-0.0009765625	\\
0.116957636430228	-0.00091552734375	\\
0.117079721645709	-0.00079345703125	\\
0.117201806861189	-0.00042724609375	\\
0.11732389207667	-0.000335693359375	\\
0.11744597729215	-0.000244140625	\\
0.11756806250763	-0.00018310546875	\\
0.117690147723111	-0.00030517578125	\\
0.117812232938591	-0.000885009765625	\\
0.117934318154072	-0.0006103515625	\\
0.118056403369552	-0.00048828125	\\
0.118178488585032	-0.00091552734375	\\
0.118300573800513	-0.000762939453125	\\
0.118422659015993	-0.00030517578125	\\
0.118544744231474	0.000335693359375	\\
0.118666829446954	0.000579833984375	\\
0.118788914662434	0.000762939453125	\\
0.118910999877915	0.000885009765625	\\
0.119033085093395	0.001129150390625	\\
0.119155170308876	0.00115966796875	\\
0.119277255524356	0.00067138671875	\\
0.119399340739836	0.000335693359375	\\
0.119521425955317	0.00048828125	\\
0.119643511170797	0.000518798828125	\\
0.119765596386278	-6.103515625e-05	\\
0.119887681601758	-0.000244140625	\\
0.120009766817238	-0.000274658203125	\\
0.120131852032719	-0.000457763671875	\\
0.120253937248199	-0.000518798828125	\\
0.12037602246368	-0.000244140625	\\
0.12049810767916	-0.000396728515625	\\
0.12062019289464	-0.0008544921875	\\
0.120742278110121	-0.00067138671875	\\
0.120864363325601	-0.000457763671875	\\
0.120986448541082	-0.000396728515625	\\
0.121108533756562	-0.00042724609375	\\
0.121230618972042	-0.000152587890625	\\
0.121352704187523	-0.000244140625	\\
0.121474789403003	-0.000396728515625	\\
0.121596874618484	-0.0006103515625	\\
0.121718959833964	-0.00103759765625	\\
0.121841045049445	-0.001251220703125	\\
0.121963130264925	-0.00103759765625	\\
0.122085215480405	-0.000396728515625	\\
0.122207300695886	-0.000274658203125	\\
0.122329385911366	-0.00018310546875	\\
0.122451471126847	0.000335693359375	\\
0.122573556342327	0.000457763671875	\\
0.122695641557807	0.000579833984375	\\
0.122817726773288	0.00067138671875	\\
0.122939811988768	0.000244140625	\\
0.123061897204249	0.000335693359375	\\
0.123183982419729	0.001007080078125	\\
0.123306067635209	0.000640869140625	\\
0.12342815285069	0.000213623046875	\\
0.12355023806617	6.103515625e-05	\\
0.123672323281651	0.000213623046875	\\
0.123794408497131	0.000213623046875	\\
0.123916493712611	0	\\
0.124038578928092	0.0001220703125	\\
0.124160664143572	0.000335693359375	\\
0.124282749359053	0.000152587890625	\\
0.124404834574533	0.000579833984375	\\
0.124526919790013	0.000640869140625	\\
0.124649005005494	0.0006103515625	\\
0.124771090220974	0.001129150390625	\\
0.124893175436455	0.0010986328125	\\
0.125015260651935	0.00152587890625	\\
0.125137345867415	0.00128173828125	\\
0.125259431082896	0.001007080078125	\\
0.125381516298376	0.00115966796875	\\
0.125503601513857	0.0008544921875	\\
0.125625686729337	0.000579833984375	\\
0.125747771944817	0.0006103515625	\\
0.125869857160298	0.000762939453125	\\
0.125991942375778	0.00054931640625	\\
0.126114027591259	0.00067138671875	\\
0.126236112806739	0.0009765625	\\
0.12635819802222	0.00091552734375	\\
0.1264802832377	0.0009765625	\\
0.12660236845318	0.001190185546875	\\
0.126724453668661	0.001617431640625	\\
0.126846538884141	0.001556396484375	\\
0.126968624099622	0.00146484375	\\
0.127090709315102	0.00146484375	\\
0.127212794530582	0.001220703125	\\
0.127334879746063	0.001068115234375	\\
0.127456964961543	0.001190185546875	\\
0.127579050177024	0.000823974609375	\\
0.127701135392504	0.0003662109375	\\
0.127823220607984	0.000396728515625	\\
0.127945305823465	-9.1552734375e-05	\\
0.128067391038945	-0.00048828125	\\
0.128189476254426	-0.0006103515625	\\
0.128311561469906	-0.000396728515625	\\
0.128433646685386	0.0001220703125	\\
0.128555731900867	0.00018310546875	\\
0.128677817116347	-0.000335693359375	\\
0.128799902331828	-9.1552734375e-05	\\
0.128921987547308	0.00018310546875	\\
0.129044072762788	9.1552734375e-05	\\
0.129166157978269	-9.1552734375e-05	\\
0.129288243193749	-0.000457763671875	\\
0.12941032840923	-0.000640869140625	\\
0.12953241362471	-0.000762939453125	\\
0.12965449884019	-0.0009765625	\\
0.129776584055671	-0.001312255859375	\\
0.129898669271151	-0.001129150390625	\\
0.130020754486632	-0.000762939453125	\\
0.130142839702112	-0.000885009765625	\\
0.130264924917592	-0.000579833984375	\\
0.130387010133073	-0.000885009765625	\\
0.130509095348553	-0.001312255859375	\\
0.130631180564034	-0.001251220703125	\\
0.130753265779514	-0.00164794921875	\\
0.130875350994995	-0.00140380859375	\\
0.130997436210475	-0.001556396484375	\\
0.131119521425955	-0.00164794921875	\\
0.131241606641436	-0.001373291015625	\\
0.131363691856916	-0.0015869140625	\\
0.131485777072397	-0.001861572265625	\\
0.131607862287877	-0.00213623046875	\\
0.131729947503357	-0.00225830078125	\\
0.131852032718838	-0.002105712890625	\\
0.131974117934318	-0.00189208984375	\\
0.132096203149799	-0.00177001953125	\\
0.132218288365279	-0.001220703125	\\
0.132340373580759	-0.000701904296875	\\
0.13246245879624	-0.000579833984375	\\
0.13258454401172	-0.000579833984375	\\
0.132706629227201	-0.00054931640625	\\
0.132828714442681	0.0001220703125	\\
0.132950799658161	-9.1552734375e-05	\\
0.133072884873642	-0.0006103515625	\\
0.133194970089122	-0.000762939453125	\\
0.133317055304603	-0.001251220703125	\\
0.133439140520083	-0.00152587890625	\\
0.133561225735563	-0.00140380859375	\\
0.133683310951044	-0.001617431640625	\\
0.133805396166524	-0.00201416015625	\\
0.133927481382005	-0.001800537109375	\\
0.134049566597485	-0.001495361328125	\\
0.134171651812965	-0.00128173828125	\\
0.134293737028446	-0.001617431640625	\\
0.134415822243926	-0.00128173828125	\\
0.134537907459407	-0.000732421875	\\
0.134659992674887	-0.0010986328125	\\
0.134782077890367	-0.0010986328125	\\
0.134904163105848	-0.001220703125	\\
0.135026248321328	-0.0013427734375	\\
0.135148333536809	-0.0010986328125	\\
0.135270418752289	-0.00140380859375	\\
0.13539250396777	-0.001678466796875	\\
0.13551458918325	-0.0015869140625	\\
0.13563667439873	-0.00146484375	\\
0.135758759614211	-0.001678466796875	\\
0.135880844829691	-0.001739501953125	\\
0.136002930045172	-0.001678466796875	\\
0.136125015260652	-0.001953125	\\
0.136247100476132	-0.00164794921875	\\
0.136369185691613	-0.001251220703125	\\
0.136491270907093	-0.000946044921875	\\
0.136613356122574	-0.0008544921875	\\
0.136735441338054	-0.0006103515625	\\
0.136857526553534	-0.00048828125	\\
0.136979611769015	-0.000732421875	\\
0.137101696984495	-0.00128173828125	\\
0.137223782199976	-0.001678466796875	\\
0.137345867415456	-0.00164794921875	\\
0.137467952630936	-0.00146484375	\\
0.137590037846417	-0.00189208984375	\\
0.137712123061897	-0.001922607421875	\\
0.137834208277378	-0.00201416015625	\\
0.137956293492858	-0.00201416015625	\\
0.138078378708338	-0.001678466796875	\\
0.138200463923819	-0.00146484375	\\
0.138322549139299	-0.001495361328125	\\
0.13844463435478	-0.001129150390625	\\
0.13856671957026	-0.001129150390625	\\
0.13868880478574	-0.001373291015625	\\
0.138810890001221	-0.001312255859375	\\
0.138932975216701	-0.0013427734375	\\
0.139055060432182	-0.00146484375	\\
0.139177145647662	-0.00177001953125	\\
0.139299230863142	-0.002227783203125	\\
0.139421316078623	-0.00244140625	\\
0.139543401294103	-0.001861572265625	\\
0.139665486509584	-0.001678466796875	\\
0.139787571725064	-0.001800537109375	\\
0.139909656940545	-0.001434326171875	\\
0.140031742156025	-0.0010986328125	\\
0.140153827371505	-0.0006103515625	\\
0.140275912586986	-0.00048828125	\\
0.140397997802466	-0.000579833984375	\\
0.140520083017947	-0.000335693359375	\\
0.140642168233427	-0.000396728515625	\\
0.140764253448907	-0.00054931640625	\\
0.140886338664388	-0.0006103515625	\\
0.141008423879868	-0.0001220703125	\\
0.141130509095349	-0.000152587890625	\\
0.141252594310829	-0.000457763671875	\\
0.141374679526309	-0.000701904296875	\\
0.14149676474179	-0.0003662109375	\\
0.14161884995727	-0.0003662109375	\\
0.141740935172751	-0.00048828125	\\
0.141863020388231	0	\\
0.141985105603711	0.0003662109375	\\
0.142107190819192	0.00042724609375	\\
0.142229276034672	0.000640869140625	\\
0.142351361250153	0.00128173828125	\\
0.142473446465633	0.001190185546875	\\
0.142595531681113	0.0013427734375	\\
0.142717616896594	0.00152587890625	\\
0.142839702112074	0.00152587890625	\\
0.142961787327555	0.00115966796875	\\
0.143083872543035	0.000732421875	\\
0.143205957758515	0.00079345703125	\\
0.143328042973996	0.000396728515625	\\
0.143450128189476	0.00030517578125	\\
0.143572213404957	0.00048828125	\\
0.143694298620437	0.00054931640625	\\
0.143816383835917	0.000640869140625	\\
0.143938469051398	0.00067138671875	\\
0.144060554266878	0.000457763671875	\\
0.144182639482359	0.000701904296875	\\
0.144304724697839	0.00091552734375	\\
0.144426809913319	0.000579833984375	\\
0.1445488951288	0.000640869140625	\\
0.14467098034428	0.00079345703125	\\
0.144793065559761	0.000640869140625	\\
0.144915150775241	0.000518798828125	\\
0.145037235990722	0.000457763671875	\\
0.145159321206202	9.1552734375e-05	\\
0.145281406421682	0.000152587890625	\\
0.145403491637163	0.000274658203125	\\
0.145525576852643	0.000244140625	\\
0.145647662068124	0.000518798828125	\\
0.145769747283604	0.00054931640625	\\
0.145891832499084	0.0001220703125	\\
0.146013917714565	-6.103515625e-05	\\
0.146136002930045	0.000213623046875	\\
0.146258088145526	0.000579833984375	\\
0.146380173361006	0.00042724609375	\\
0.146502258576486	0.0001220703125	\\
0.146624343791967	0	\\
0.146746429007447	-0.00042724609375	\\
0.146868514222928	-0.0006103515625	\\
0.146990599438408	-0.000457763671875	\\
0.147112684653888	-0.00054931640625	\\
0.147234769869369	-0.000640869140625	\\
0.147356855084849	-0.00067138671875	\\
0.14747894030033	-0.0008544921875	\\
0.14760102551581	-0.00042724609375	\\
0.14772311073129	-0.00048828125	\\
0.147845195946771	-0.00048828125	\\
0.147967281162251	0.00018310546875	\\
0.148089366377732	0.000335693359375	\\
0.148211451593212	0.00030517578125	\\
0.148333536808692	0.000244140625	\\
0.148455622024173	0.00067138671875	\\
0.148577707239653	0.000396728515625	\\
0.148699792455134	0.000244140625	\\
0.148821877670614	6.103515625e-05	\\
0.148943962886094	-0.000518798828125	\\
0.149066048101575	-0.00042724609375	\\
0.149188133317055	-0.00115966796875	\\
0.149310218532536	-0.001373291015625	\\
0.149432303748016	-0.0010986328125	\\
0.149554388963497	-0.00128173828125	\\
0.149676474178977	-0.0010986328125	\\
0.149798559394457	-0.001068115234375	\\
0.149920644609938	-0.00115966796875	\\
0.150042729825418	-0.000885009765625	\\
0.150164815040899	-0.0006103515625	\\
0.150286900256379	-0.000213623046875	\\
0.150408985471859	-0.00042724609375	\\
0.15053107068734	-0.000732421875	\\
0.15065315590282	-0.000640869140625	\\
0.150775241118301	-0.00054931640625	\\
0.150897326333781	-0.00067138671875	\\
0.151019411549261	-0.00103759765625	\\
0.151141496764742	-0.00115966796875	\\
0.151263581980222	-0.0013427734375	\\
0.151385667195703	-0.00140380859375	\\
0.151507752411183	-0.00091552734375	\\
0.151629837626663	-0.000762939453125	\\
0.151751922842144	-0.000701904296875	\\
0.151874008057624	-0.00048828125	\\
0.151996093273105	-0.000244140625	\\
0.152118178488585	-0.000457763671875	\\
0.152240263704065	-0.000640869140625	\\
0.152362348919546	-0.000762939453125	\\
0.152484434135026	-0.000823974609375	\\
0.152606519350507	-0.0008544921875	\\
0.152728604565987	-0.001068115234375	\\
0.152850689781467	-0.0013427734375	\\
0.152972774996948	-0.001617431640625	\\
0.153094860212428	-0.0015869140625	\\
0.153216945427909	-0.001678466796875	\\
0.153339030643389	-0.001983642578125	\\
0.153461115858869	-0.001739501953125	\\
0.15358320107435	-0.00152587890625	\\
0.15370528628983	-0.001800537109375	\\
0.153827371505311	-0.00164794921875	\\
0.153949456720791	-0.001708984375	\\
0.154071541936272	-0.001739501953125	\\
0.154193627151752	-0.001220703125	\\
0.154315712367232	-0.001373291015625	\\
0.154437797582713	-0.0015869140625	\\
0.154559882798193	-0.001983642578125	\\
0.154681968013674	-0.002471923828125	\\
0.154804053229154	-0.00274658203125	\\
0.154926138444634	-0.002716064453125	\\
0.155048223660115	-0.002532958984375	\\
0.155170308875595	-0.0023193359375	\\
0.155292394091076	-0.00225830078125	\\
0.155414479306556	-0.0023193359375	\\
0.155536564522036	-0.002655029296875	\\
0.155658649737517	-0.002685546875	\\
0.155780734952997	-0.00262451171875	\\
0.155902820168478	-0.002532958984375	\\
0.156024905383958	-0.002471923828125	\\
0.156146990599438	-0.002227783203125	\\
0.156269075814919	-0.00189208984375	\\
0.156391161030399	-0.00177001953125	\\
0.15651324624588	-0.001678466796875	\\
0.15663533146136	-0.00201416015625	\\
0.15675741667684	-0.00213623046875	\\
0.156879501892321	-0.0020751953125	\\
0.157001587107801	-0.0020751953125	\\
0.157123672323282	-0.00250244140625	\\
0.157245757538762	-0.002655029296875	\\
0.157367842754242	-0.00262451171875	\\
0.157489927969723	-0.0023193359375	\\
0.157612013185203	-0.00164794921875	\\
0.157734098400684	-0.0015869140625	\\
0.157856183616164	-0.00177001953125	\\
0.157978268831644	-0.001617431640625	\\
0.158100354047125	-0.001312255859375	\\
0.158222439262605	-0.0010986328125	\\
0.158344524478086	-0.001068115234375	\\
0.158466609693566	-0.00128173828125	\\
0.158588694909047	-0.001556396484375	\\
0.158710780124527	-0.001495361328125	\\
0.158832865340007	-0.001861572265625	\\
0.158954950555488	-0.002166748046875	\\
0.159077035770968	-0.001953125	\\
0.159199120986449	-0.001708984375	\\
0.159321206201929	-0.00177001953125	\\
0.159443291417409	-0.001556396484375	\\
0.15956537663289	-0.001251220703125	\\
0.15968746184837	-0.00103759765625	\\
0.159809547063851	-0.0009765625	\\
0.159931632279331	-0.001068115234375	\\
0.160053717494811	-0.00115966796875	\\
0.160175802710292	-0.000640869140625	\\
0.160297887925772	-0.000640869140625	\\
0.160419973141253	-0.000823974609375	\\
0.160542058356733	-0.000701904296875	\\
0.160664143572213	-0.00103759765625	\\
0.160786228787694	-0.0010986328125	\\
0.160908314003174	-0.0013427734375	\\
0.161030399218655	-0.001312255859375	\\
0.161152484434135	-0.00146484375	\\
0.161274569649615	-0.0013427734375	\\
0.161396654865096	-0.00115966796875	\\
0.161518740080576	-0.0010986328125	\\
0.161640825296057	-0.00067138671875	\\
0.161762910511537	-0.000701904296875	\\
0.161884995727017	-0.000640869140625	\\
0.162007080942498	-0.00054931640625	\\
0.162129166157978	-0.000640869140625	\\
0.162251251373459	-0.000396728515625	\\
0.162373336588939	-0.0008544921875	\\
0.162495421804419	-0.0008544921875	\\
0.1626175070199	-0.000762939453125	\\
0.16273959223538	-0.001068115234375	\\
0.162861677450861	-0.001129150390625	\\
0.162983762666341	-0.00152587890625	\\
0.163105847881822	-0.001556396484375	\\
0.163227933097302	-0.001312255859375	\\
0.163350018312782	-0.0006103515625	\\
0.163472103528263	-0.0003662109375	\\
0.163594188743743	-0.00054931640625	\\
0.163716273959224	-0.00054931640625	\\
0.163838359174704	-0.00030517578125	\\
0.163960444390184	-0.000274658203125	\\
0.164082529605665	-0.000152587890625	\\
0.164204614821145	-3.0517578125e-05	\\
0.164326700036626	0.000152587890625	\\
0.164448785252106	6.103515625e-05	\\
0.164570870467586	6.103515625e-05	\\
0.164692955683067	-3.0517578125e-05	\\
0.164815040898547	-0.000244140625	\\
0.164937126114028	-0.00030517578125	\\
0.165059211329508	-0.000640869140625	\\
0.165181296544988	-0.000732421875	\\
0.165303381760469	-0.00048828125	\\
0.165425466975949	-0.00054931640625	\\
0.16554755219143	-0.000244140625	\\
0.16566963740691	-6.103515625e-05	\\
0.16579172262239	6.103515625e-05	\\
0.165913807837871	0.00030517578125	\\
0.166035893053351	0.0006103515625	\\
0.166157978268832	0.000579833984375	\\
0.166280063484312	0.000640869140625	\\
0.166402148699792	0.001068115234375	\\
0.166524233915273	0.000946044921875	\\
0.166646319130753	0.0006103515625	\\
0.166768404346234	0.00048828125	\\
0.166890489561714	0.00030517578125	\\
0.167012574777194	0.0003662109375	\\
0.167134659992675	0.000244140625	\\
0.167256745208155	0.00030517578125	\\
0.167378830423636	0.000701904296875	\\
0.167500915639116	0.000762939453125	\\
0.167623000854597	0.000823974609375	\\
0.167745086070077	0.000823974609375	\\
0.167867171285557	0.001068115234375	\\
0.167989256501038	0.00140380859375	\\
0.168111341716518	0.001678466796875	\\
0.168233426931999	0.0015869140625	\\
0.168355512147479	0.00091552734375	\\
0.168477597362959	0.000732421875	\\
0.16859968257844	0.000579833984375	\\
0.16872176779392	0.0003662109375	\\
0.168843853009401	0.00018310546875	\\
0.168965938224881	0.0001220703125	\\
0.169088023440361	6.103515625e-05	\\
0.169210108655842	0.000152587890625	\\
0.169332193871322	0.000274658203125	\\
0.169454279086803	0.000335693359375	\\
0.169576364302283	0.000579833984375	\\
0.169698449517763	0.000579833984375	\\
0.169820534733244	0.000579833984375	\\
0.169942619948724	0.00079345703125	\\
0.170064705164205	0.000762939453125	\\
0.170186790379685	0.000152587890625	\\
0.170308875595165	-0.0001220703125	\\
0.170430960810646	-0.00030517578125	\\
0.170553046026126	-0.000762939453125	\\
0.170675131241607	-0.0008544921875	\\
0.170797216457087	-0.000732421875	\\
0.170919301672567	-0.00079345703125	\\
0.171041386888048	-0.00054931640625	\\
0.171163472103528	-0.00067138671875	\\
0.171285557319009	-0.000640869140625	\\
0.171407642534489	-9.1552734375e-05	\\
0.171529727749969	0	\\
0.17165181296545	0	\\
0.17177389818093	-6.103515625e-05	\\
0.171895983396411	3.0517578125e-05	\\
0.172018068611891	0.000152587890625	\\
0.172140153827371	-3.0517578125e-05	\\
0.172262239042852	-0.000244140625	\\
0.172384324258332	-0.000732421875	\\
0.172506409473813	-0.000823974609375	\\
0.172628494689293	-0.000762939453125	\\
0.172750579904774	-0.0013427734375	\\
0.172872665120254	-0.0013427734375	\\
0.172994750335734	-0.000732421875	\\
0.173116835551215	-0.000335693359375	\\
0.173238920766695	-0.00048828125	\\
0.173361005982176	-0.0009765625	\\
0.173483091197656	-0.00079345703125	\\
0.173605176413136	-0.000335693359375	\\
0.173727261628617	0.00030517578125	\\
0.173849346844097	0.000518798828125	\\
0.173971432059578	0.0001220703125	\\
0.174093517275058	0.00018310546875	\\
0.174215602490538	-0.00018310546875	\\
0.174337687706019	-0.000244140625	\\
0.174459772921499	-0.000244140625	\\
0.17458185813698	-0.000579833984375	\\
0.17470394335246	-0.000762939453125	\\
0.17482602856794	-0.000762939453125	\\
0.174948113783421	-0.000579833984375	\\
0.175070198998901	-0.00048828125	\\
0.175192284214382	-0.000885009765625	\\
0.175314369429862	-0.000762939453125	\\
0.175436454645342	-0.000396728515625	\\
0.175558539860823	9.1552734375e-05	\\
0.175680625076303	-6.103515625e-05	\\
0.175802710291784	-0.00054931640625	\\
0.175924795507264	-0.00042724609375	\\
0.176046880722744	-0.000823974609375	\\
0.176168965938225	-0.001220703125	\\
0.176291051153705	-0.00164794921875	\\
0.176413136369186	-0.002044677734375	\\
0.176535221584666	-0.002197265625	\\
0.176657306800146	-0.0018310546875	\\
0.176779392015627	-0.00201416015625	\\
0.176901477231107	-0.002410888671875	\\
0.177023562446588	-0.002410888671875	\\
0.177145647662068	-0.0020751953125	\\
0.177267732877549	-0.001708984375	\\
0.177389818093029	-0.00201416015625	\\
0.177511903308509	-0.0018310546875	\\
0.17763398852399	-0.001373291015625	\\
0.17775607373947	-0.00128173828125	\\
0.177878158954951	-0.001495361328125	\\
0.178000244170431	-0.001861572265625	\\
0.178122329385911	-0.00189208984375	\\
0.178244414601392	-0.001953125	\\
0.178366499816872	-0.002410888671875	\\
0.178488585032353	-0.002593994140625	\\
0.178610670247833	-0.002899169921875	\\
0.178732755463313	-0.003143310546875	\\
0.178854840678794	-0.003082275390625	\\
0.178976925894274	-0.00335693359375	\\
0.179099011109755	-0.003387451171875	\\
0.179221096325235	-0.00347900390625	\\
0.179343181540715	-0.003387451171875	\\
0.179465266756196	-0.003265380859375	\\
0.179587351971676	-0.003143310546875	\\
0.179709437187157	-0.003082275390625	\\
0.179831522402637	-0.003082275390625	\\
0.179953607618117	-0.002960205078125	\\
0.180075692833598	-0.00299072265625	\\
0.180197778049078	-0.003143310546875	\\
0.180319863264559	-0.003448486328125	\\
0.180441948480039	-0.003753662109375	\\
0.180564033695519	-0.0035400390625	\\
0.180686118911	-0.00335693359375	\\
0.18080820412648	-0.003082275390625	\\
0.180930289341961	-0.002685546875	\\
0.181052374557441	-0.00225830078125	\\
0.181174459772921	-0.001983642578125	\\
0.181296544988402	-0.001678466796875	\\
0.181418630203882	-0.001617431640625	\\
0.181540715419363	-0.001739501953125	\\
0.181662800634843	-0.001678466796875	\\
0.181784885850324	-0.00146484375	\\
0.181906971065804	-0.0018310546875	\\
0.182029056281284	-0.00189208984375	\\
0.182151141496765	-0.00164794921875	\\
0.182273226712245	-0.001739501953125	\\
0.182395311927726	-0.0018310546875	\\
0.182517397143206	-0.0020751953125	\\
0.182639482358686	-0.001800537109375	\\
0.182761567574167	-0.0015869140625	\\
0.182883652789647	-0.00140380859375	\\
0.183005738005128	-0.001434326171875	\\
0.183127823220608	-0.00152587890625	\\
0.183249908436088	-0.0010986328125	\\
0.183371993651569	-0.001220703125	\\
0.183494078867049	-0.0013427734375	\\
0.18361616408253	-0.000823974609375	\\
0.18373824929801	-0.0008544921875	\\
0.18386033451349	-0.001251220703125	\\
0.183982419728971	-0.001556396484375	\\
0.184104504944451	-0.001129150390625	\\
0.184226590159932	-0.001129150390625	\\
0.184348675375412	-0.00103759765625	\\
0.184470760590892	-0.001068115234375	\\
0.184592845806373	-0.001373291015625	\\
0.184714931021853	-0.0009765625	\\
0.184837016237334	-0.001068115234375	\\
0.184959101452814	-0.00140380859375	\\
0.185081186668294	-0.001068115234375	\\
0.185203271883775	-0.000701904296875	\\
0.185325357099255	-0.00048828125	\\
0.185447442314736	-0.000335693359375	\\
0.185569527530216	-0.000335693359375	\\
0.185691612745696	-3.0517578125e-05	\\
0.185813697961177	0.000152587890625	\\
0.185935783176657	-0.00018310546875	\\
0.186057868392138	-0.00030517578125	\\
0.186179953607618	-0.0003662109375	\\
0.186302038823099	-0.000640869140625	\\
0.186424124038579	-0.00079345703125	\\
0.186546209254059	-0.0008544921875	\\
0.18666829446954	-0.0010986328125	\\
0.18679037968502	-0.000946044921875	\\
0.186912464900501	-0.00030517578125	\\
0.187034550115981	-0.00018310546875	\\
0.187156635331461	-0.000518798828125	\\
0.187278720546942	-0.00030517578125	\\
0.187400805762422	9.1552734375e-05	\\
0.187522890977903	9.1552734375e-05	\\
0.187644976193383	-0.000274658203125	\\
0.187767061408863	-0.000396728515625	\\
0.187889146624344	-0.000640869140625	\\
0.188011231839824	-0.00054931640625	\\
0.188133317055305	-0.000732421875	\\
0.188255402270785	-0.001068115234375	\\
0.188377487486265	-0.00128173828125	\\
0.188499572701746	-0.00152587890625	\\
0.188621657917226	-0.001556396484375	\\
0.188743743132707	-0.00146484375	\\
0.188865828348187	-0.001220703125	\\
0.188987913563667	-0.0009765625	\\
0.189109998779148	-0.000823974609375	\\
0.189232083994628	-0.00067138671875	\\
0.189354169210109	-0.000244140625	\\
0.189476254425589	9.1552734375e-05	\\
0.189598339641069	0.00018310546875	\\
0.18972042485655	-0.00018310546875	\\
0.18984251007203	-0.0003662109375	\\
0.189964595287511	-6.103515625e-05	\\
0.190086680502991	-0.00030517578125	\\
0.190208765718472	-0.000946044921875	\\
0.190330850933952	-0.000885009765625	\\
0.190452936149432	-0.000885009765625	\\
0.190575021364913	-0.001251220703125	\\
0.190697106580393	-0.00091552734375	\\
0.190819191795874	-0.000396728515625	\\
0.190941277011354	-0.000213623046875	\\
0.191063362226834	-3.0517578125e-05	\\
0.191185447442315	3.0517578125e-05	\\
0.191307532657795	0.000335693359375	\\
0.191429617873276	0.001068115234375	\\
0.191551703088756	0.001312255859375	\\
0.191673788304236	0.00091552734375	\\
0.191795873519717	0.001007080078125	\\
0.191917958735197	0.0009765625	\\
0.192040043950678	0.000946044921875	\\
0.192162129166158	0.001007080078125	\\
0.192284214381638	0.000885009765625	\\
0.192406299597119	0.0006103515625	\\
0.192528384812599	0.001251220703125	\\
0.19265047002808	0.00140380859375	\\
0.19277255524356	0.000823974609375	\\
0.19289464045904	0.00048828125	\\
0.193016725674521	0.0006103515625	\\
0.193138810890001	0.00030517578125	\\
0.193260896105482	0.000152587890625	\\
0.193382981320962	0.00048828125	\\
0.193505066536442	0.000274658203125	\\
0.193627151751923	0.000274658203125	\\
0.193749236967403	0.000213623046875	\\
0.193871322182884	0.000274658203125	\\
0.193993407398364	-0.000152587890625	\\
0.194115492613844	-0.000244140625	\\
0.194237577829325	-0.000274658203125	\\
0.194359663044805	-0.00048828125	\\
0.194481748260286	-0.00054931640625	\\
0.194603833475766	-0.000732421875	\\
0.194725918691247	-0.0008544921875	\\
0.194848003906727	-0.000640869140625	\\
0.194970089122207	-0.000396728515625	\\
0.195092174337688	-0.00048828125	\\
0.195214259553168	-0.000335693359375	\\
0.195336344768649	-0.000640869140625	\\
0.195458429984129	-0.000732421875	\\
0.195580515199609	-0.000701904296875	\\
0.19570260041509	-0.000823974609375	\\
0.19582468563057	-0.000946044921875	\\
0.195946770846051	-0.000640869140625	\\
0.196068856061531	-0.000946044921875	\\
0.196190941277011	-0.00115966796875	\\
0.196313026492492	-0.001220703125	\\
0.196435111707972	-0.00146484375	\\
0.196557196923453	-0.001251220703125	\\
0.196679282138933	-0.00079345703125	\\
0.196801367354413	-0.000457763671875	\\
0.196923452569894	-0.000213623046875	\\
0.197045537785374	-0.000396728515625	\\
0.197167623000855	-0.000335693359375	\\
0.197289708216335	-9.1552734375e-05	\\
0.197411793431815	-0.000457763671875	\\
0.197533878647296	-0.001068115234375	\\
0.197655963862776	-0.000885009765625	\\
0.197778049078257	-0.000823974609375	\\
0.197900134293737	-0.0010986328125	\\
0.198022219509217	-0.00103759765625	\\
0.198144304724698	-0.001556396484375	\\
0.198266389940178	-0.001434326171875	\\
0.198388475155659	-0.00146484375	\\
0.198510560371139	-0.001495361328125	\\
0.198632645586619	-0.0009765625	\\
0.1987547308021	-0.00079345703125	\\
0.19887681601758	-0.000457763671875	\\
0.198998901233061	-0.0006103515625	\\
0.199120986448541	-0.000213623046875	\\
0.199243071664022	0.00048828125	\\
0.199365156879502	0.0003662109375	\\
0.199487242094982	0	\\
0.199609327310463	0.000244140625	\\
0.199731412525943	0.0003662109375	\\
0.199853497741424	0.00042724609375	\\
0.199975582956904	-0.0001220703125	\\
0.200097668172384	-0.00018310546875	\\
0.200219753387865	0.000396728515625	\\
0.200341838603345	0.00030517578125	\\
0.200463923818826	0.000152587890625	\\
0.200586009034306	0.00054931640625	\\
0.200708094249786	0.000823974609375	\\
0.200830179465267	0.0009765625	\\
0.200952264680747	0.001373291015625	\\
0.201074349896228	0.001220703125	\\
0.201196435111708	0.001129150390625	\\
0.201318520327188	0.001007080078125	\\
0.201440605542669	0.00103759765625	\\
0.201562690758149	0.000579833984375	\\
0.20168477597363	3.0517578125e-05	\\
0.20180686118911	-0.00018310546875	\\
0.20192894640459	-0.000213623046875	\\
0.202051031620071	-0.000518798828125	\\
0.202173116835551	-0.0008544921875	\\
0.202295202051032	-0.000732421875	\\
0.202417287266512	-0.000274658203125	\\
0.202539372481992	0.0001220703125	\\
0.202661457697473	-0.0003662109375	\\
0.202783542912953	-0.000335693359375	\\
0.202905628128434	-0.000244140625	\\
0.203027713343914	-0.000244140625	\\
0.203149798559394	-9.1552734375e-05	\\
0.203271883774875	-0.00042724609375	\\
0.203393968990355	-0.00030517578125	\\
0.203516054205836	0.0001220703125	\\
0.203638139421316	-3.0517578125e-05	\\
0.203760224636796	-0.000274658203125	\\
0.203882309852277	-0.00048828125	\\
0.204004395067757	-0.000732421875	\\
0.204126480283238	-0.000701904296875	\\
0.204248565498718	-0.000518798828125	\\
0.204370650714199	-0.0003662109375	\\
0.204492735929679	-0.000701904296875	\\
0.204614821145159	-0.00091552734375	\\
0.20473690636064	-0.0009765625	\\
0.20485899157612	-0.00091552734375	\\
0.204981076791601	-0.00054931640625	\\
0.205103162007081	-0.000701904296875	\\
0.205225247222561	-0.0003662109375	\\
0.205347332438042	-0.0003662109375	\\
0.205469417653522	-0.000396728515625	\\
0.205591502869003	-0.000244140625	\\
0.205713588084483	-0.00042724609375	\\
0.205835673299963	-0.000274658203125	\\
0.205957758515444	-0.000579833984375	\\
0.206079843730924	-0.00030517578125	\\
0.206201928946405	0.000335693359375	\\
0.206324014161885	0.00054931640625	\\
0.206446099377365	0.001007080078125	\\
0.206568184592846	0.00140380859375	\\
0.206690269808326	0.001861572265625	\\
0.206812355023807	0.0018310546875	\\
0.206934440239287	0.001708984375	\\
0.207056525454767	0.00189208984375	\\
0.207178610670248	0.00164794921875	\\
0.207300695885728	0.001800537109375	\\
0.207422781101209	0.00189208984375	\\
0.207544866316689	0.00152587890625	\\
0.207666951532169	0.00115966796875	\\
0.20778903674765	0.000885009765625	\\
0.20791112196313	0.001007080078125	\\
0.208033207178611	0.00140380859375	\\
0.208155292394091	0.00177001953125	\\
0.208277377609571	0.001800537109375	\\
0.208399462825052	0.001800537109375	\\
0.208521548040532	0.00201416015625	\\
0.208643633256013	0.002288818359375	\\
0.208765718471493	0.002349853515625	\\
0.208887803686974	0.0023193359375	\\
0.209009888902454	0.002471923828125	\\
0.209131974117934	0.002655029296875	\\
0.209254059333415	0.002349853515625	\\
0.209376144548895	0.0018310546875	\\
0.209498229764376	0.0008544921875	\\
0.209620314979856	0.000579833984375	\\
0.209742400195336	9.1552734375e-05	\\
0.209864485410817	6.103515625e-05	\\
0.209986570626297	3.0517578125e-05	\\
0.210108655841778	-0.000213623046875	\\
0.210230741057258	0.00042724609375	\\
0.210352826272738	0.000701904296875	\\
0.210474911488219	0.00079345703125	\\
0.210596996703699	0.000946044921875	\\
0.21071908191918	0.00103759765625	\\
0.21084116713466	0.001007080078125	\\
0.21096325235014	0.001495361328125	\\
0.211085337565621	0.001220703125	\\
0.211207422781101	0.000823974609375	\\
0.211329507996582	0.000946044921875	\\
0.211451593212062	0.000579833984375	\\
0.211573678427542	-3.0517578125e-05	\\
0.211695763643023	-3.0517578125e-05	\\
0.211817848858503	-0.00018310546875	\\
0.211939934073984	-0.000640869140625	\\
0.212062019289464	-0.0003662109375	\\
0.212184104504944	-0.00042724609375	\\
0.212306189720425	-0.000335693359375	\\
0.212428274935905	9.1552734375e-05	\\
0.212550360151386	0.000396728515625	\\
0.212672445366866	0.00030517578125	\\
0.212794530582346	-0.000213623046875	\\
0.212916615797827	0.000244140625	\\
0.213038701013307	0.000244140625	\\
0.213160786228788	0.000274658203125	\\
0.213282871444268	0.00048828125	\\
0.213404956659749	0.00054931640625	\\
0.213527041875229	0.000457763671875	\\
0.213649127090709	6.103515625e-05	\\
0.21377121230619	0.000396728515625	\\
0.21389329752167	0.0003662109375	\\
0.214015382737151	0.000335693359375	\\
0.214137467952631	0.000274658203125	\\
0.214259553168111	-0.00030517578125	\\
0.214381638383592	-0.000396728515625	\\
0.214503723599072	-0.0001220703125	\\
0.214625808814553	-3.0517578125e-05	\\
0.214747894030033	-0.000274658203125	\\
0.214869979245513	-0.0001220703125	\\
0.214992064460994	0.000518798828125	\\
0.215114149676474	0.000579833984375	\\
0.215236234891955	0.000335693359375	\\
0.215358320107435	0.000457763671875	\\
0.215480405322915	0.000244140625	\\
0.215602490538396	0.000457763671875	\\
0.215724575753876	0.000518798828125	\\
0.215846660969357	-9.1552734375e-05	\\
0.215968746184837	0.0001220703125	\\
0.216090831400317	0.000274658203125	\\
0.216212916615798	0.0001220703125	\\
0.216335001831278	0.000732421875	\\
0.216457087046759	0.001007080078125	\\
0.216579172262239	0.00103759765625	\\
0.216701257477719	0.00164794921875	\\
0.2168233426932	0.0018310546875	\\
0.21694542790868	0.001708984375	\\
0.217067513124161	0.000885009765625	\\
0.217189598339641	0.000518798828125	\\
0.217311683555121	0.00054931640625	\\
0.217433768770602	0.000152587890625	\\
0.217555853986082	-3.0517578125e-05	\\
0.217677939201563	3.0517578125e-05	\\
0.217800024417043	-6.103515625e-05	\\
0.217922109632524	-0.000274658203125	\\
0.218044194848004	-0.00030517578125	\\
0.218166280063484	3.0517578125e-05	\\
0.218288365278965	0	\\
0.218410450494445	-6.103515625e-05	\\
0.218532535709926	6.103515625e-05	\\
0.218654620925406	9.1552734375e-05	\\
0.218776706140886	-3.0517578125e-05	\\
0.218898791356367	-0.000213623046875	\\
0.219020876571847	9.1552734375e-05	\\
0.219142961787328	-3.0517578125e-05	\\
0.219265047002808	-0.00030517578125	\\
0.219387132218288	-0.00042724609375	\\
0.219509217433769	-0.000732421875	\\
0.219631302649249	-0.0008544921875	\\
0.21975338786473	-0.001068115234375	\\
0.21987547308021	-0.000701904296875	\\
0.21999755829569	-0.000152587890625	\\
0.220119643511171	-0.0001220703125	\\
0.220241728726651	-0.000518798828125	\\
0.220363813942132	-0.00030517578125	\\
0.220485899157612	3.0517578125e-05	\\
0.220607984373092	-3.0517578125e-05	\\
0.220730069588573	9.1552734375e-05	\\
0.220852154804053	0.000213623046875	\\
0.220974240019534	-0.00030517578125	\\
0.221096325235014	-0.000213623046875	\\
0.221218410450494	-0.00048828125	\\
0.221340495665975	-0.000579833984375	\\
0.221462580881455	-0.000579833984375	\\
0.221584666096936	-0.000823974609375	\\
0.221706751312416	-0.000885009765625	\\
0.221828836527896	-0.000335693359375	\\
0.221950921743377	0.00030517578125	\\
0.222073006958857	0.00018310546875	\\
0.222195092174338	-9.1552734375e-05	\\
0.222317177389818	-3.0517578125e-05	\\
0.222439262605299	0.000457763671875	\\
0.222561347820779	0.0006103515625	\\
0.222683433036259	3.0517578125e-05	\\
0.22280551825174	-0.000244140625	\\
0.22292760346722	-0.0001220703125	\\
0.223049688682701	9.1552734375e-05	\\
0.223171773898181	0.000213623046875	\\
0.223293859113661	-0.0003662109375	\\
0.223415944329142	-0.00079345703125	\\
0.223538029544622	-0.000701904296875	\\
0.223660114760103	-0.000518798828125	\\
0.223782199975583	-0.000457763671875	\\
0.223904285191063	-0.000518798828125	\\
0.224026370406544	-0.00079345703125	\\
0.224148455622024	-0.000457763671875	\\
0.224270540837505	-0.0001220703125	\\
0.224392626052985	0.000152587890625	\\
0.224514711268465	0.000823974609375	\\
0.224636796483946	0.000885009765625	\\
0.224758881699426	0.000885009765625	\\
0.224880966914907	0.000732421875	\\
0.225003052130387	0.000762939453125	\\
0.225125137345867	0.00079345703125	\\
0.225247222561348	0.000274658203125	\\
0.225369307776828	0.000244140625	\\
0.225491392992309	0.000244140625	\\
0.225613478207789	-0.000244140625	\\
0.225735563423269	-0.000335693359375	\\
0.22585764863875	-0.000335693359375	\\
0.22597973385423	-0.000152587890625	\\
0.226101819069711	6.103515625e-05	\\
0.226223904285191	0.00030517578125	\\
0.226345989500671	0.000335693359375	\\
0.226468074716152	-0.000152587890625	\\
0.226590159931632	9.1552734375e-05	\\
0.226712245147113	0.000274658203125	\\
0.226834330362593	0.0003662109375	\\
0.226956415578074	0.000335693359375	\\
0.227078500793554	0.0001220703125	\\
0.227200586009034	3.0517578125e-05	\\
0.227322671224515	-0.000732421875	\\
0.227444756439995	-0.001129150390625	\\
0.227566841655476	-0.001068115234375	\\
0.227688926870956	-0.000823974609375	\\
0.227811012086436	-0.001220703125	\\
0.227933097301917	-0.00128173828125	\\
0.228055182517397	-0.0006103515625	\\
0.228177267732878	-0.00054931640625	\\
0.228299352948358	-0.000457763671875	\\
0.228421438163838	-0.00054931640625	\\
0.228543523379319	-0.000335693359375	\\
0.228665608594799	-0.000152587890625	\\
0.22878769381028	-0.0003662109375	\\
0.22890977902576	-0.000274658203125	\\
0.22903186424124	-0.00067138671875	\\
0.229153949456721	-0.000640869140625	\\
0.229276034672201	-0.000640869140625	\\
0.229398119887682	-0.00140380859375	\\
0.229520205103162	-0.00140380859375	\\
0.229642290318642	-0.000823974609375	\\
0.229764375534123	-0.0006103515625	\\
0.229886460749603	-0.00018310546875	\\
0.230008545965084	0.0001220703125	\\
0.230130631180564	0.000213623046875	\\
0.230252716396044	0.0006103515625	\\
0.230374801611525	0.000885009765625	\\
0.230496886827005	0.000640869140625	\\
0.230618972042486	0.000823974609375	\\
0.230741057257966	0.00054931640625	\\
0.230863142473446	0.0001220703125	\\
0.230985227688927	9.1552734375e-05	\\
0.231107312904407	0.000152587890625	\\
0.231229398119888	0.0001220703125	\\
0.231351483335368	-0.000244140625	\\
0.231473568550849	-0.000518798828125	\\
0.231595653766329	9.1552734375e-05	\\
0.231717738981809	0.00054931640625	\\
0.23183982419729	0.00048828125	\\
0.23196190941277	0.000457763671875	\\
0.232083994628251	0.000823974609375	\\
0.232206079843731	0.001190185546875	\\
0.232328165059211	0.00128173828125	\\
0.232450250274692	0.001068115234375	\\
0.232572335490172	0.00067138671875	\\
0.232694420705653	0.000732421875	\\
0.232816505921133	0.000762939453125	\\
0.232938591136613	0.00030517578125	\\
0.233060676352094	0.000335693359375	\\
0.233182761567574	0.000640869140625	\\
0.233304846783055	0.000579833984375	\\
0.233426931998535	0.000335693359375	\\
0.233549017214015	-0.0001220703125	\\
0.233671102429496	-0.00018310546875	\\
0.233793187644976	-0.00018310546875	\\
0.233915272860457	-0.000274658203125	\\
0.234037358075937	-0.000457763671875	\\
0.234159443291417	-0.000274658203125	\\
0.234281528506898	-0.000457763671875	\\
0.234403613722378	-6.103515625e-05	\\
0.234525698937859	0.000244140625	\\
0.234647784153339	6.103515625e-05	\\
0.234769869368819	0.0001220703125	\\
0.2348919545843	-0.00018310546875	\\
0.23501403979978	-3.0517578125e-05	\\
0.235136125015261	-0.000213623046875	\\
0.235258210230741	-0.000518798828125	\\
0.235380295446221	-0.0003662109375	\\
0.235502380661702	-0.000823974609375	\\
0.235624465877182	-0.00079345703125	\\
0.235746551092663	-6.103515625e-05	\\
0.235868636308143	-9.1552734375e-05	\\
0.235990721523623	-0.000335693359375	\\
0.236112806739104	9.1552734375e-05	\\
0.236234891954584	-6.103515625e-05	\\
0.236356977170065	-0.000244140625	\\
0.236479062385545	-6.103515625e-05	\\
0.236601147601026	-0.0003662109375	\\
0.236723232816506	-0.00079345703125	\\
0.236845318031986	-0.00067138671875	\\
0.236967403247467	-0.000518798828125	\\
0.237089488462947	-0.000579833984375	\\
0.237211573678428	-0.000701904296875	\\
0.237333658893908	-0.000732421875	\\
0.237455744109388	-0.000244140625	\\
0.237577829324869	-9.1552734375e-05	\\
0.237699914540349	-9.1552734375e-05	\\
0.23782199975583	0.000244140625	\\
0.23794408497131	0.000457763671875	\\
0.23806617018679	0.000457763671875	\\
0.238188255402271	0.000732421875	\\
0.238310340617751	0.00103759765625	\\
0.238432425833232	0.000579833984375	\\
0.238554511048712	0.000244140625	\\
0.238676596264192	0.0001220703125	\\
0.238798681479673	0.00018310546875	\\
0.238920766695153	0.0001220703125	\\
0.239042851910634	-6.103515625e-05	\\
0.239164937126114	-0.000213623046875	\\
0.239287022341594	-0.000152587890625	\\
0.239409107557075	0	\\
0.239531192772555	3.0517578125e-05	\\
0.239653277988036	-6.103515625e-05	\\
0.239775363203516	0.000274658203125	\\
0.239897448418996	0.000885009765625	\\
0.240019533634477	0.001434326171875	\\
0.240141618849957	0.002166748046875	\\
0.240263704065438	0.002532958984375	\\
0.240385789280918	0.002410888671875	\\
0.240507874496398	0.00225830078125	\\
0.240629959711879	0.001983642578125	\\
0.240752044927359	0.00201416015625	\\
0.24087413014284	0.0018310546875	\\
0.24099621535832	0.001312255859375	\\
0.241118300573801	0.001495361328125	\\
0.241240385789281	0.0018310546875	\\
0.241362471004761	0.002044677734375	\\
0.241484556220242	0.002349853515625	\\
0.241606641435722	0.0025634765625	\\
0.241728726651203	0.002471923828125	\\
0.241850811866683	0.001953125	\\
0.241972897082163	0.001953125	\\
0.242094982297644	0.001953125	\\
0.242217067513124	0.001953125	\\
0.242339152728605	0.00201416015625	\\
0.242461237944085	0.001953125	\\
0.242583323159565	0.001495361328125	\\
0.242705408375046	0.0013427734375	\\
0.242827493590526	0.0009765625	\\
0.242949578806007	0.00103759765625	\\
0.243071664021487	0.00146484375	\\
0.243193749236967	0.00115966796875	\\
0.243315834452448	0.001007080078125	\\
0.243437919667928	0.000946044921875	\\
0.243560004883409	0.0006103515625	\\
0.243682090098889	0.000946044921875	\\
0.243804175314369	0.001007080078125	\\
0.24392626052985	0.001068115234375	\\
0.24404834574533	0.00091552734375	\\
0.244170430960811	0.000946044921875	\\
0.244292516176291	0.001129150390625	\\
0.244414601391771	0.00067138671875	\\
0.244536686607252	0.0008544921875	\\
0.244658771822732	0.001129150390625	\\
0.244780857038213	0.000732421875	\\
0.244902942253693	0.0003662109375	\\
0.245025027469173	0.00079345703125	\\
0.245147112684654	0.00128173828125	\\
0.245269197900134	0.00152587890625	\\
0.245391283115615	0.001129150390625	\\
0.245513368331095	0.001220703125	\\
0.245635453546576	0.001068115234375	\\
0.245757538762056	0.00091552734375	\\
0.245879623977536	0.001220703125	\\
0.246001709193017	0.00128173828125	\\
0.246123794408497	0.0013427734375	\\
0.246245879623978	0.0013427734375	\\
0.246367964839458	0.001373291015625	\\
0.246490050054938	0.001068115234375	\\
0.246612135270419	0.001129150390625	\\
0.246734220485899	0.001190185546875	\\
0.24685630570138	0.000823974609375	\\
0.24697839091686	0.000579833984375	\\
0.24710047613234	0.00048828125	\\
0.247222561347821	0.00054931640625	\\
0.247344646563301	0.000946044921875	\\
0.247466731778782	0.00152587890625	\\
0.247588816994262	0.001220703125	\\
0.247710902209742	0.0010986328125	\\
0.247832987425223	0.00128173828125	\\
0.247955072640703	0.0018310546875	\\
0.248077157856184	0.002410888671875	\\
0.248199243071664	0.002105712890625	\\
0.248321328287144	0.0020751953125	\\
0.248443413502625	0.00177001953125	\\
0.248565498718105	0.001495361328125	\\
0.248687583933586	0.001434326171875	\\
0.248809669149066	0.001312255859375	\\
0.248931754364546	0.001220703125	\\
0.249053839580027	0.001312255859375	\\
0.249175924795507	0.001007080078125	\\
0.249298010010988	0.000885009765625	\\
0.249420095226468	0.0010986328125	\\
0.249542180441948	0.001434326171875	\\
0.249664265657429	0.0020751953125	\\
0.249786350872909	0.001617431640625	\\
0.24990843608839	0.001251220703125	\\
0.25003052130387	0.00140380859375	\\
0.250152606519351	0.001495361328125	\\
0.250274691734831	0.00128173828125	\\
0.250396776950311	0.000518798828125	\\
0.250518862165792	0.00054931640625	\\
0.250640947381272	0.0009765625	\\
0.250763032596753	0.00042724609375	\\
0.250885117812233	0.000457763671875	\\
0.251007203027713	0.000640869140625	\\
0.251129288243194	0.00091552734375	\\
0.251251373458674	0.001190185546875	\\
0.251373458674155	0.0008544921875	\\
0.251495543889635	0.000457763671875	\\
0.251617629105115	0.00054931640625	\\
0.251739714320596	0.0006103515625	\\
0.251861799536076	0.000457763671875	\\
0.251983884751557	-3.0517578125e-05	\\
0.252105969967037	-6.103515625e-05	\\
0.252228055182517	-0.000244140625	\\
0.252350140397998	-0.00042724609375	\\
0.252472225613478	-0.000640869140625	\\
0.252594310828959	-0.000396728515625	\\
0.252716396044439	-0.000396728515625	\\
0.252838481259919	-0.0009765625	\\
0.2529605664754	-0.000396728515625	\\
0.25308265169088	-0.000244140625	\\
0.253204736906361	-0.00079345703125	\\
0.253326822121841	-0.000823974609375	\\
0.253448907337321	-0.000244140625	\\
0.253570992552802	3.0517578125e-05	\\
0.253693077768282	6.103515625e-05	\\
0.253815162983763	9.1552734375e-05	\\
0.253937248199243	0.000274658203125	\\
0.254059333414723	0.000457763671875	\\
0.254181418630204	0.000244140625	\\
0.254303503845684	0.000244140625	\\
0.254425589061165	0.000457763671875	\\
0.254547674276645	3.0517578125e-05	\\
0.254669759492126	-0.000152587890625	\\
0.254791844707606	0.000335693359375	\\
0.254913929923086	0.000335693359375	\\
0.255036015138567	0.000274658203125	\\
0.255158100354047	0.001007080078125	\\
0.255280185569528	0.0010986328125	\\
0.255402270785008	0.001312255859375	\\
0.255524356000488	0.001983642578125	\\
0.255646441215969	0.00201416015625	\\
0.255768526431449	0.001953125	\\
0.25589061164693	0.0018310546875	\\
0.25601269686241	0.001800537109375	\\
0.25613478207789	0.001953125	\\
0.256256867293371	0.00225830078125	\\
0.256378952508851	0.002044677734375	\\
0.256501037724332	0.001434326171875	\\
0.256623122939812	0.00140380859375	\\
0.256745208155292	0.000946044921875	\\
0.256867293370773	0.00079345703125	\\
0.256989378586253	0.001007080078125	\\
0.257111463801734	0.000823974609375	\\
0.257233549017214	0.001129150390625	\\
0.257355634232694	0.00103759765625	\\
0.257477719448175	0.000885009765625	\\
0.257599804663655	0.0010986328125	\\
0.257721889879136	0.001190185546875	\\
0.257843975094616	0.001617431640625	\\
0.257966060310096	0.001953125	\\
0.258088145525577	0.002105712890625	\\
0.258210230741057	0.00238037109375	\\
0.258332315956538	0.002960205078125	\\
0.258454401172018	0.0020751953125	\\
0.258576486387498	0.001434326171875	\\
0.258698571602979	0.00152587890625	\\
0.258820656818459	0.000701904296875	\\
0.25894274203394	0.000274658203125	\\
0.25906482724942	0.0008544921875	\\
0.259186912464901	0.000885009765625	\\
0.259308997680381	0.00067138671875	\\
0.259431082895861	0.000885009765625	\\
0.259553168111342	0.0013427734375	\\
0.259675253326822	0.001861572265625	\\
0.259797338542303	0.001861572265625	\\
0.259919423757783	0.00201416015625	\\
0.260041508973263	0.00250244140625	\\
0.260163594188744	0.00238037109375	\\
0.260285679404224	0.001800537109375	\\
0.260407764619705	0.00152587890625	\\
0.260529849835185	0.0013427734375	\\
0.260651935050665	0.000518798828125	\\
0.260774020266146	-3.0517578125e-05	\\
0.260896105481626	-0.000213623046875	\\
0.261018190697107	-0.000762939453125	\\
0.261140275912587	-0.001251220703125	\\
0.261262361128067	-0.0009765625	\\
0.261384446343548	-0.001068115234375	\\
0.261506531559028	-0.00054931640625	\\
0.261628616774509	6.103515625e-05	\\
0.261750701989989	-0.000244140625	\\
0.261872787205469	9.1552734375e-05	\\
0.26199487242095	0.000732421875	\\
0.26211695763643	0.001007080078125	\\
0.262239042851911	0.001251220703125	\\
0.262361128067391	0.001129150390625	\\
0.262483213282871	0.00115966796875	\\
0.262605298498352	0.0015869140625	\\
0.262727383713832	0.001251220703125	\\
0.262849468929313	0.0006103515625	\\
0.262971554144793	0.00042724609375	\\
0.263093639360273	0.000152587890625	\\
0.263215724575754	0.000213623046875	\\
0.263337809791234	0.000885009765625	\\
0.263459895006715	0.0009765625	\\
0.263581980222195	0.001373291015625	\\
0.263704065437675	0.001556396484375	\\
0.263826150653156	0.0018310546875	\\
0.263948235868636	0.00201416015625	\\
0.264070321084117	0.0020751953125	\\
0.264192406299597	0.002227783203125	\\
0.264314491515078	0.001495361328125	\\
0.264436576730558	0.001129150390625	\\
0.264558661946038	0.001312255859375	\\
0.264680747161519	0.0013427734375	\\
0.264802832376999	0.001251220703125	\\
0.26492491759248	0.00140380859375	\\
0.26504700280796	0.0013427734375	\\
0.26516908802344	0.001556396484375	\\
0.265291173238921	0.001617431640625	\\
0.265413258454401	0.001678466796875	\\
0.265535343669882	0.00177001953125	\\
0.265657428885362	0.001678466796875	\\
0.265779514100842	0.001312255859375	\\
0.265901599316323	0.00115966796875	\\
0.266023684531803	0.001251220703125	\\
0.266145769747284	0.001190185546875	\\
0.266267854962764	0.001190185546875	\\
0.266389940178244	0.001312255859375	\\
0.266512025393725	0.000946044921875	\\
0.266634110609205	0.0010986328125	\\
0.266756195824686	0.001495361328125	\\
0.266878281040166	0.001434326171875	\\
0.267000366255646	0.001678466796875	\\
0.267122451471127	0.002197265625	\\
0.267244536686607	0.002197265625	\\
0.267366621902088	0.00189208984375	\\
0.267488707117568	0.00213623046875	\\
0.267610792333048	0.00189208984375	\\
0.267732877548529	0.001739501953125	\\
0.267854962764009	0.0013427734375	\\
0.26797704797949	0.00115966796875	\\
0.26809913319497	0.0009765625	\\
0.26822121841045	0.0001220703125	\\
0.268343303625931	-0.000274658203125	\\
0.268465388841411	-0.00079345703125	\\
0.268587474056892	-0.000640869140625	\\
0.268709559272372	6.103515625e-05	\\
0.268831644487853	0.000244140625	\\
0.268953729703333	0.00091552734375	\\
0.269075814918813	0.00177001953125	\\
0.269197900134294	0.0018310546875	\\
0.269319985349774	0.0015869140625	\\
0.269442070565255	0.00189208984375	\\
0.269564155780735	0.001922607421875	\\
0.269686240996215	0.001678466796875	\\
0.269808326211696	0.00128173828125	\\
0.269930411427176	0.00103759765625	\\
0.270052496642657	0.0008544921875	\\
0.270174581858137	0.000396728515625	\\
0.270296667073617	0.000213623046875	\\
0.270418752289098	0.000152587890625	\\
0.270540837504578	0.000244140625	\\
0.270662922720059	0.00079345703125	\\
0.270785007935539	0.00091552734375	\\
0.270907093151019	0.000946044921875	\\
0.2710291783665	0.001312255859375	\\
0.27115126358198	0.001953125	\\
0.271273348797461	0.00250244140625	\\
0.271395434012941	0.0029296875	\\
0.271517519228421	0.0032958984375	\\
0.271639604443902	0.003021240234375	\\
0.271761689659382	0.002777099609375	\\
0.271883774874863	0.002960205078125	\\
0.272005860090343	0.00262451171875	\\
0.272127945305823	0.001556396484375	\\
0.272250030521304	0.00152587890625	\\
0.272372115736784	0.001800537109375	\\
0.272494200952265	0.001861572265625	\\
0.272616286167745	0.001739501953125	\\
0.272738371383225	0.00140380859375	\\
0.272860456598706	0.0015869140625	\\
0.272982541814186	0.001190185546875	\\
0.273104627029667	0.001251220703125	\\
0.273226712245147	0.002471923828125	\\
0.273348797460628	0.00335693359375	\\
0.273470882676108	0.003570556640625	\\
0.273592967891588	0.003021240234375	\\
0.273715053107069	0.002105712890625	\\
0.273837138322549	0.002532958984375	\\
0.27395922353803	0.00335693359375	\\
0.27408130875351	0.003662109375	\\
0.27420339396899	0.0037841796875	\\
0.274325479184471	0.0032958984375	\\
0.274447564399951	0.003265380859375	\\
0.274569649615432	0.0029296875	\\
0.274691734830912	0.001922607421875	\\
0.274813820046392	0.000732421875	\\
0.274935905261873	0.000274658203125	\\
0.275057990477353	0.001190185546875	\\
0.275180075692834	0.00213623046875	\\
0.275302160908314	0.001953125	\\
0.275424246123794	0.001617431640625	\\
0.275546331339275	0.001373291015625	\\
0.275668416554755	0.00152587890625	\\
0.275790501770236	0.001434326171875	\\
0.275912586985716	0.00201416015625	\\
0.276034672201196	0.003082275390625	\\
0.276156757416677	0.00286865234375	\\
0.276278842632157	0.00177001953125	\\
0.276400927847638	0.00128173828125	\\
0.276523013063118	0.000701904296875	\\
0.276645098278598	0.00030517578125	\\
0.276767183494079	0.000701904296875	\\
0.276889268709559	0.0003662109375	\\
0.27701135392504	0.000946044921875	\\
0.27713343914052	0.001220703125	\\
0.277255524356	0.000946044921875	\\
0.277377609571481	0.001129150390625	\\
0.277499694786961	0.0020751953125	\\
0.277621780002442	0.003997802734375	\\
0.277743865217922	0.0040283203125	\\
0.277865950433403	0.00299072265625	\\
0.277988035648883	0.00299072265625	\\
0.278110120864363	0.003173828125	\\
0.278232206079844	0.0025634765625	\\
0.278354291295324	0.001861572265625	\\
0.278476376510805	0.00189208984375	\\
0.278598461726285	0.002044677734375	\\
0.278720546941765	0.00225830078125	\\
0.278842632157246	0.002410888671875	\\
0.278964717372726	0.002685546875	\\
0.279086802588207	0.00372314453125	\\
0.279208887803687	0.003997802734375	\\
0.279330973019167	0.00390625	\\
0.279453058234648	0.003326416015625	\\
0.279575143450128	0.0028076171875	\\
0.279697228665609	0.002349853515625	\\
0.279819313881089	0.001312255859375	\\
0.279941399096569	0.00103759765625	\\
0.28006348431205	0.001800537109375	\\
0.28018556952753	0.00225830078125	\\
0.280307654743011	0.0020751953125	\\
0.280429739958491	0.0018310546875	\\
0.280551825173971	0.0023193359375	\\
0.280673910389452	0.00274658203125	\\
0.280795995604932	0.003082275390625	\\
0.280918080820413	0.003448486328125	\\
0.281040166035893	0.003662109375	\\
0.281162251251373	0.003753662109375	\\
0.281284336466854	0.00274658203125	\\
0.281406421682334	0.002410888671875	\\
0.281528506897815	0.001861572265625	\\
0.281650592113295	0.001556396484375	\\
0.281772677328775	0.001617431640625	\\
0.281894762544256	0.001220703125	\\
0.282016847759736	0.00091552734375	\\
0.282138932975217	0.001129150390625	\\
0.282261018190697	0.002105712890625	\\
0.282383103406178	0.002349853515625	\\
0.282505188621658	0.00274658203125	\\
0.282627273837138	0.003265380859375	\\
0.282749359052619	0.00408935546875	\\
0.282871444268099	0.00445556640625	\\
0.28299352948358	0.004425048828125	\\
0.28311561469906	0.00390625	\\
0.28323769991454	0.003753662109375	\\
0.283359785130021	0.003570556640625	\\
0.283481870345501	0.002685546875	\\
0.283603955560982	0.001861572265625	\\
0.283726040776462	0.0015869140625	\\
0.283848125991942	0.00152587890625	\\
0.283970211207423	0.001129150390625	\\
0.284092296422903	0.0008544921875	\\
0.284214381638384	0.00140380859375	\\
0.284336466853864	0.00189208984375	\\
0.284458552069344	0.001556396484375	\\
0.284580637284825	0.00115966796875	\\
0.284702722500305	0.001068115234375	\\
0.284824807715786	0.00115966796875	\\
0.284946892931266	0.00115966796875	\\
0.285068978146746	0.00128173828125	\\
0.285191063362227	0.001068115234375	\\
0.285313148577707	0.000823974609375	\\
0.285435233793188	0.00042724609375	\\
0.285557319008668	0.000640869140625	\\
0.285679404224148	0.001068115234375	\\
0.285801489439629	0.0010986328125	\\
0.285923574655109	0.0010986328125	\\
0.28604565987059	0.00091552734375	\\
0.28616774508607	0.001434326171875	\\
0.28628983030155	0.002166748046875	\\
0.286411915517031	0.001922607421875	\\
0.286534000732511	0.0015869140625	\\
0.286656085947992	0.001953125	\\
0.286778171163472	0.002349853515625	\\
0.286900256378953	0.00286865234375	\\
0.287022341594433	0.00286865234375	\\
0.287144426809913	0.0028076171875	\\
0.287266512025394	0.002838134765625	\\
0.287388597240874	0.002532958984375	\\
0.287510682456355	0.0023193359375	\\
0.287632767671835	0.002685546875	\\
0.287754852887315	0.00274658203125	\\
0.287876938102796	0.002716064453125	\\
0.287999023318276	0.0023193359375	\\
0.288121108533757	0.00201416015625	\\
0.288243193749237	0.00225830078125	\\
0.288365278964717	0.002655029296875	\\
0.288487364180198	0.002685546875	\\
0.288609449395678	0.003204345703125	\\
0.288731534611159	0.003387451171875	\\
0.288853619826639	0.0035400390625	\\
0.288975705042119	0.0035400390625	\\
0.2890977902576	0.003173828125	\\
0.28921987547308	0.00372314453125	\\
0.289341960688561	0.00335693359375	\\
0.289464045904041	0.003448486328125	\\
0.289586131119521	0.004119873046875	\\
0.289708216335002	0.002899169921875	\\
0.289830301550482	0.002105712890625	\\
0.289952386765963	0.002166748046875	\\
0.290074471981443	0.001708984375	\\
0.290196557196923	0.00164794921875	\\
0.290318642412404	0.00164794921875	\\
0.290440727627884	0.001251220703125	\\
0.290562812843365	0.001251220703125	\\
0.290684898058845	0.001251220703125	\\
0.290806983274325	0.001129150390625	\\
0.290929068489806	0.001373291015625	\\
0.291051153705286	0.001495361328125	\\
0.291173238920767	0.001190185546875	\\
0.291295324136247	0.00091552734375	\\
0.291417409351728	0.000823974609375	\\
0.291539494567208	0.0010986328125	\\
0.291661579782688	0.001495361328125	\\
0.291783664998169	0.001434326171875	\\
0.291905750213649	0.001800537109375	\\
0.29202783542913	0.001617431640625	\\
0.29214992064461	0.00115966796875	\\
0.29227200586009	0.001495361328125	\\
0.292394091075571	0.00177001953125	\\
0.292516176291051	0.001983642578125	\\
0.292638261506532	0.002410888671875	\\
0.292760346722012	0.002899169921875	\\
0.292882431937492	0.002838134765625	\\
0.293004517152973	0.0025634765625	\\
0.293126602368453	0.002838134765625	\\
0.293248687583934	0.00311279296875	\\
0.293370772799414	0.002838134765625	\\
0.293492858014894	0.002227783203125	\\
0.293614943230375	0.00225830078125	\\
0.293737028445855	0.00201416015625	\\
0.293859113661336	0.00140380859375	\\
0.293981198876816	0.0010986328125	\\
0.294103284092296	0.001220703125	\\
0.294225369307777	0.000946044921875	\\
0.294347454523257	0.00079345703125	\\
0.294469539738738	0.001312255859375	\\
0.294591624954218	0.001861572265625	\\
0.294713710169698	0.002166748046875	\\
0.294835795385179	0.0030517578125	\\
0.294957880600659	0.003173828125	\\
0.29507996581614	0.002960205078125	\\
0.29520205103162	0.0030517578125	\\
0.2953241362471	0.003021240234375	\\
0.295446221462581	0.003021240234375	\\
0.295568306678061	0.002593994140625	\\
0.295690391893542	0.00250244140625	\\
0.295812477109022	0.002593994140625	\\
0.295934562324502	0.002349853515625	\\
0.296056647539983	0.002227783203125	\\
0.296178732755463	0.00238037109375	\\
0.296300817970944	0.002655029296875	\\
0.296422903186424	0.0028076171875	\\
0.296544988401905	0.003326416015625	\\
0.296667073617385	0.003326416015625	\\
0.296789158832865	0.003143310546875	\\
0.296911244048346	0.002532958984375	\\
0.297033329263826	0.00244140625	\\
0.297155414479307	0.00250244140625	\\
0.297277499694787	0.002227783203125	\\
0.297399584910267	0.002716064453125	\\
0.297521670125748	0.00250244140625	\\
0.297643755341228	0.00244140625	\\
0.297765840556709	0.002777099609375	\\
0.297887925772189	0.00286865234375	\\
0.298010010987669	0.003204345703125	\\
0.29813209620315	0.00396728515625	\\
0.29825418141863	0.004241943359375	\\
0.298376266634111	0.003997802734375	\\
0.298498351849591	0.003753662109375	\\
0.298620437065071	0.003631591796875	\\
0.298742522280552	0.003509521484375	\\
0.298864607496032	0.003265380859375	\\
0.298986692711513	0.002899169921875	\\
0.299108777926993	0.002899169921875	\\
0.299230863142473	0.002899169921875	\\
0.299352948357954	0.00299072265625	\\
0.299475033573434	0.002960205078125	\\
0.299597118788915	0.002471923828125	\\
0.299719204004395	0.002960205078125	\\
0.299841289219875	0.0029296875	\\
0.299963374435356	0.0029296875	\\
0.300085459650836	0.00286865234375	\\
0.300207544866317	0.00262451171875	\\
0.300329630081797	0.0018310546875	\\
0.300451715297277	0.00146484375	\\
0.300573800512758	0.002105712890625	\\
0.300695885728238	0.0023193359375	\\
0.300817970943719	0.002105712890625	\\
0.300940056159199	0.002197265625	\\
0.30106214137468	0.001983642578125	\\
0.30118422659016	0.001922607421875	\\
0.30130631180564	0.001953125	\\
0.301428397021121	0.00189208984375	\\
0.301550482236601	0.002044677734375	\\
0.301672567452082	0.00189208984375	\\
0.301794652667562	0.001312255859375	\\
0.301916737883042	0.001220703125	\\
0.302038823098523	0.000762939453125	\\
0.302160908314003	-3.0517578125e-05	\\
0.302282993529484	-0.000213623046875	\\
0.302405078744964	-3.0517578125e-05	\\
0.302527163960444	0.000244140625	\\
0.302649249175925	0.000640869140625	\\
0.302771334391405	0.001068115234375	\\
0.302893419606886	0.00091552734375	\\
0.303015504822366	0.000823974609375	\\
0.303137590037846	0.000885009765625	\\
0.303259675253327	0.0006103515625	\\
0.303381760468807	0.00018310546875	\\
0.303503845684288	-9.1552734375e-05	\\
0.303625930899768	-0.00018310546875	\\
0.303748016115248	-0.000335693359375	\\
0.303870101330729	-0.00018310546875	\\
0.303992186546209	0.000640869140625	\\
0.30411427176169	0.001007080078125	\\
0.30423635697717	0.000762939453125	\\
0.30435844219265	0.0013427734375	\\
0.304480527408131	0.00177001953125	\\
0.304602612623611	0.001495361328125	\\
0.304724697839092	0.001983642578125	\\
0.304846783054572	0.002655029296875	\\
0.304968868270052	0.00250244140625	\\
0.305090953485533	0.0018310546875	\\
0.305213038701013	0.001312255859375	\\
0.305335123916494	0.00079345703125	\\
0.305457209131974	0.000152587890625	\\
0.305579294347455	-0.000274658203125	\\
0.305701379562935	-0.00067138671875	\\
0.305823464778415	-0.000823974609375	\\
0.305945549993896	-0.00054931640625	\\
0.306067635209376	-0.000335693359375	\\
0.306189720424857	-0.000335693359375	\\
0.306311805640337	-6.103515625e-05	\\
0.306433890855817	0.0003662109375	\\
0.306555976071298	0.000885009765625	\\
0.306678061286778	0.00164794921875	\\
0.306800146502259	0.0013427734375	\\
0.306922231717739	0.000946044921875	\\
0.307044316933219	0.00079345703125	\\
0.3071664021487	0.0003662109375	\\
0.30728848736418	0	\\
0.307410572579661	-0.0006103515625	\\
0.307532657795141	-0.0009765625	\\
0.307654743010621	-0.001190185546875	\\
0.307776828226102	-0.00128173828125	\\
0.307898913441582	-0.00103759765625	\\
0.308020998657063	-0.000457763671875	\\
0.308143083872543	-0.000213623046875	\\
0.308265169088023	-6.103515625e-05	\\
0.308387254303504	0.00030517578125	\\
0.308509339518984	0.000396728515625	\\
0.308631424734465	0.0003662109375	\\
0.308753509949945	0.000335693359375	\\
0.308875595165425	0.0003662109375	\\
0.308997680380906	-0.000152587890625	\\
0.309119765596386	-0.000885009765625	\\
0.309241850811867	-0.0006103515625	\\
0.309363936027347	-0.00042724609375	\\
0.309486021242827	-0.0003662109375	\\
0.309608106458308	-0.000457763671875	\\
0.309730191673788	-0.00030517578125	\\
0.309852276889269	-0.000396728515625	\\
0.309974362104749	-0.000579833984375	\\
0.31009644732023	-0.000518798828125	\\
0.31021853253571	-0.000274658203125	\\
0.31034061775119	-0.000335693359375	\\
0.310462702966671	-0.000732421875	\\
0.310584788182151	-0.000885009765625	\\
0.310706873397632	-0.00048828125	\\
0.310828958613112	-0.000457763671875	\\
0.310951043828592	-0.000732421875	\\
0.311073129044073	-0.000885009765625	\\
0.311195214259553	-0.0013427734375	\\
0.311317299475034	-0.001495361328125	\\
0.311439384690514	-0.0015869140625	\\
0.311561469905994	-0.00152587890625	\\
0.311683555121475	-0.00146484375	\\
0.311805640336955	-0.001708984375	\\
0.311927725552436	-0.00152587890625	\\
0.312049810767916	-0.001495361328125	\\
0.312171895983396	-0.001220703125	\\
0.312293981198877	-0.001190185546875	\\
0.312416066414357	-0.001251220703125	\\
0.312538151629838	-0.000732421875	\\
0.312660236845318	-0.000579833984375	\\
0.312782322060798	-0.000396728515625	\\
0.312904407276279	9.1552734375e-05	\\
0.313026492491759	6.103515625e-05	\\
0.31314857770724	-0.000274658203125	\\
0.31327066292272	-0.000701904296875	\\
0.3133927481382	-0.000701904296875	\\
0.313514833353681	-9.1552734375e-05	\\
0.313636918569161	-0.000640869140625	\\
0.313759003784642	-0.000640869140625	\\
0.313881089000122	-0.000640869140625	\\
0.314003174215602	-0.00054931640625	\\
0.314125259431083	-0.000152587890625	\\
0.314247344646563	-0.000335693359375	\\
0.314369429862044	-0.0001220703125	\\
0.314491515077524	0.00030517578125	\\
0.314613600293005	0.000335693359375	\\
0.314735685508485	0.00054931640625	\\
0.314857770723965	0.00048828125	\\
0.314979855939446	0.00030517578125	\\
0.315101941154926	6.103515625e-05	\\
0.315224026370407	-0.0001220703125	\\
0.315346111585887	-0.0003662109375	\\
0.315468196801367	-0.000762939453125	\\
0.315590282016848	-0.001007080078125	\\
0.315712367232328	-0.00067138671875	\\
0.315834452447809	-0.000335693359375	\\
0.315956537663289	-0.00042724609375	\\
0.316078622878769	-0.0001220703125	\\
0.31620070809425	-0.000152587890625	\\
0.31632279330973	0.000213623046875	\\
0.316444878525211	0.000823974609375	\\
0.316566963740691	0.000762939453125	\\
0.316689048956171	0.000732421875	\\
0.316811134171652	0.000640869140625	\\
0.316933219387132	0.000244140625	\\
0.317055304602613	-0.000335693359375	\\
0.317177389818093	-0.00042724609375	\\
0.317299475033573	-0.0006103515625	\\
0.317421560249054	-0.001007080078125	\\
0.317543645464534	-0.00115966796875	\\
0.317665730680015	-0.00146484375	\\
0.317787815895495	-0.001220703125	\\
0.317909901110975	-0.000579833984375	\\
0.318031986326456	-0.000274658203125	\\
0.318154071541936	-0.000457763671875	\\
0.318276156757417	0.000457763671875	\\
0.318398241972897	0.000640869140625	\\
0.318520327188377	0.00030517578125	\\
0.318642412403858	0.0010986328125	\\
0.318764497619338	0.00128173828125	\\
0.318886582834819	0.0008544921875	\\
0.319008668050299	0.000762939453125	\\
0.31913075326578	0.000640869140625	\\
0.31925283848126	0.000518798828125	\\
0.31937492369674	0.000640869140625	\\
0.319497008912221	0.000579833984375	\\
0.319619094127701	0.00042724609375	\\
0.319741179343182	0.000213623046875	\\
0.319863264558662	0.000152587890625	\\
0.319985349774142	0.00018310546875	\\
0.320107434989623	0.00103759765625	\\
0.320229520205103	0.001373291015625	\\
0.320351605420584	0.001556396484375	\\
0.320473690636064	0.00146484375	\\
0.320595775851544	0.00146484375	\\
0.320717861067025	0.0015869140625	\\
0.320839946282505	0.001708984375	\\
0.320962031497986	0.001251220703125	\\
0.321084116713466	0.000701904296875	\\
0.321206201928946	0.00054931640625	\\
0.321328287144427	0.0006103515625	\\
0.321450372359907	0.00030517578125	\\
0.321572457575388	0.000152587890625	\\
0.321694542790868	0.000579833984375	\\
0.321816628006348	0.00079345703125	\\
0.321938713221829	0.001129150390625	\\
0.322060798437309	0.001678466796875	\\
0.32218288365279	0.001922607421875	\\
0.32230496886827	0.002532958984375	\\
0.32242705408375	0.0029296875	\\
0.322549139299231	0.00286865234375	\\
0.322671224514711	0.0028076171875	\\
0.322793309730192	0.00244140625	\\
0.322915394945672	0.001739501953125	\\
0.323037480161152	0.001495361328125	\\
0.323159565376633	0.001220703125	\\
0.323281650592113	0.001556396484375	\\
0.323403735807594	0.00177001953125	\\
0.323525821023074	0.001556396484375	\\
0.323647906238554	0.0013427734375	\\
0.323769991454035	0.001434326171875	\\
0.323892076669515	0.00152587890625	\\
0.324014161884996	0.001617431640625	\\
0.324136247100476	0.00177001953125	\\
0.324258332315957	0.001434326171875	\\
0.324380417531437	0.001007080078125	\\
0.324502502746917	0.001190185546875	\\
0.324624587962398	0.00103759765625	\\
0.324746673177878	0.000518798828125	\\
0.324868758393359	0	\\
0.324990843608839	-0.000579833984375	\\
0.325112928824319	-0.000823974609375	\\
0.3252350140398	-0.00042724609375	\\
0.32535709925528	-0.000640869140625	\\
0.325479184470761	-0.001251220703125	\\
0.325601269686241	-0.00140380859375	\\
0.325723354901721	-0.002288818359375	\\
0.325845440117202	-0.000213623046875	\\
0.325967525332682	0.007232666015625	\\
0.326089610548163	-0.02410888671875	\\
0.326211695763643	0.068328857421875	\\
0.326333780979123	-0.278656005859375	\\
0.326455866194604	-0.172027587890625	\\
0.326577951410084	0.236175537109375	\\
0.326700036625565	0.2545166015625	\\
0.326822121841045	0.06866455078125	\\
0.326944207056525	-0.014984130859375	\\
0.327066292272006	0.072845458984375	\\
0.327188377487486	0.0106201171875	\\
0.327310462702967	0.010894775390625	\\
0.327432547918447	-0.00897216796875	\\
0.327554633133927	-0.178192138671875	\\
0.327676718349408	-0.096588134765625	\\
0.327798803564888	0.0355224609375	\\
0.327920888780369	-0.05767822265625	\\
0.328042973995849	-0.110809326171875	\\
0.328165059211329	0.01104736328125	\\
0.32828714442681	0.0177001953125	\\
0.32840922964229	-0.00238037109375	\\
0.328531314857771	0.06787109375	\\
0.328653400073251	0.04840087890625	\\
0.328775485288732	0.01861572265625	\\
0.328897570504212	0.021514892578125	\\
0.329019655719692	0.02996826171875	\\
0.329141740935173	0.04376220703125	\\
0.329263826150653	-0.00701904296875	\\
0.329385911366134	-0.04364013671875	\\
0.329507996581614	-0.0208740234375	\\
0.329630081797094	-0.022216796875	\\
0.329752167012575	-0.03436279296875	\\
0.329874252228055	-0.02691650390625	\\
0.329996337443536	-0.0205078125	\\
0.330118422659016	-0.018890380859375	\\
0.330240507874496	-0.00390625	\\
0.330362593089977	0.012786865234375	\\
0.330484678305457	0.01513671875	\\
0.330606763520938	0.03814697265625	\\
0.330728848736418	0.020172119140625	\\
0.330850933951898	0.009124755859375	\\
0.330973019167379	0.02667236328125	\\
0.331095104382859	0.01922607421875	\\
0.33121718959834	0.0125732421875	\\
0.33133927481382	-0.001007080078125	\\
0.3314613600293	-0.03094482421875	\\
0.331583445244781	-0.016357421875	\\
0.331705530460261	0.003326416015625	\\
0.331827615675742	-0.028839111328125	\\
0.331949700891222	-0.024444580078125	\\
0.332071786106702	0.000457763671875	\\
0.332193871322183	-0.011199951171875	\\
0.332315956537663	-0.0155029296875	\\
0.332438041753144	0.007049560546875	\\
0.332560126968624	0.0047607421875	\\
0.332682212184104	-0.005462646484375	\\
0.332804297399585	-0.00421142578125	\\
0.332926382615065	0.0040283203125	\\
0.333048467830546	0.018768310546875	\\
0.333170553046026	0.00927734375	\\
0.333292638261507	-0.011444091796875	\\
0.333414723476987	0.00372314453125	\\
0.333536808692467	0.013214111328125	\\
0.333658893907948	-0.008941650390625	\\
0.333780979123428	-0.0081787109375	\\
0.333903064338909	-0.000701904296875	\\
0.334025149554389	-0.009124755859375	\\
0.334147234769869	-0.010528564453125	\\
0.33426931998535	-0.001983642578125	\\
0.33439140520083	-0.003753662109375	\\
0.334513490416311	-0.00738525390625	\\
0.334635575631791	-0.00421142578125	\\
0.334757660847271	0.005706787109375	\\
0.334879746062752	0.011932373046875	\\
0.335001831278232	0.003875732421875	\\
0.335123916493713	-0.00299072265625	\\
0.335246001709193	0.00384521484375	\\
0.335368086924673	0.004547119140625	\\
0.335490172140154	-0.002349853515625	\\
0.335612257355634	-0.006011962890625	\\
0.335734342571115	-0.00750732421875	\\
0.335856427786595	-0.00299072265625	\\
0.335978513002075	-0.003265380859375	\\
0.336100598217556	-0.010955810546875	\\
0.336222683433036	-0.011871337890625	\\
0.336344768648517	-0.0048828125	\\
0.336466853863997	-0.00341796875	\\
0.336588939079477	-0.000244140625	\\
0.336711024294958	0.001953125	\\
0.336833109510438	0.00146484375	\\
0.336955194725919	0.00311279296875	\\
0.337077279941399	0.005096435546875	\\
0.337199365156879	0.002105712890625	\\
0.33732145037236	0.003204345703125	\\
0.33744353558784	0.00384521484375	\\
0.337565620803321	-0.000823974609375	\\
0.337687706018801	0.003173828125	\\
0.337809791234282	0.00152587890625	\\
0.337931876449762	-0.007965087890625	\\
0.338053961665242	-0.008148193359375	\\
0.338176046880723	-0.0029296875	\\
0.338298132096203	-0.00262451171875	\\
0.338420217311684	-0.0015869140625	\\
0.338542302527164	0.000885009765625	\\
0.338664387742644	0.00042724609375	\\
0.338786472958125	0.00250244140625	\\
0.338908558173605	0.007354736328125	\\
0.339030643389086	0.007110595703125	\\
0.339152728604566	0.003753662109375	\\
0.339274813820046	0.0032958984375	\\
0.339396899035527	0.00225830078125	\\
0.339518984251007	0.000732421875	\\
0.339641069466488	0.000274658203125	\\
0.339763154681968	-0.002410888671875	\\
0.339885239897448	-0.005096435546875	\\
0.340007325112929	-0.005462646484375	\\
0.340129410328409	-0.00189208984375	\\
0.34025149554389	-0.00054931640625	\\
0.34037358075937	-0.001739501953125	\\
0.34049566597485	-0.000946044921875	\\
0.340617751190331	0.001861572265625	\\
0.340739836405811	0.004730224609375	\\
0.340861921621292	0.005584716796875	\\
0.340984006836772	0.004058837890625	\\
0.341106092052252	0.00177001953125	\\
0.341228177267733	0.002532958984375	\\
0.341350262483213	0.003082275390625	\\
0.341472347698694	0.0006103515625	\\
0.341594432914174	-0.000762939453125	\\
0.341716518129654	-0.00079345703125	\\
0.341838603345135	-0.001251220703125	\\
0.341960688560615	-0.001251220703125	\\
0.342082773776096	-0.00030517578125	\\
0.342204858991576	-0.0009765625	\\
0.342326944207057	-0.00225830078125	\\
0.342449029422537	-0.002105712890625	\\
0.342571114638017	-0.00091552734375	\\
0.342693199853498	0.001129150390625	\\
0.342815285068978	0.001556396484375	\\
0.342937370284459	0.00048828125	\\
0.343059455499939	0.001251220703125	\\
0.343181540715419	0.002593994140625	\\
0.3433036259309	0.0032958984375	\\
0.34342571114638	0.002960205078125	\\
0.343547796361861	0.00103759765625	\\
0.343669881577341	0.001312255859375	\\
0.343791966792821	0.000701904296875	\\
0.343914052008302	-0.00054931640625	\\
0.344036137223782	-0.000457763671875	\\
0.344158222439263	-0.001251220703125	\\
0.344280307654743	-0.00244140625	\\
0.344402392870223	-0.001068115234375	\\
0.344524478085704	0.00146484375	\\
0.344646563301184	0.0009765625	\\
0.344768648516665	-6.103515625e-05	\\
0.344890733732145	0.000762939453125	\\
0.345012818947625	0.0015869140625	\\
0.345134904163106	0.00213623046875	\\
0.345256989378586	0.00244140625	\\
0.345379074594067	0.001495361328125	\\
0.345501159809547	0.001190185546875	\\
0.345623245025027	0.00146484375	\\
0.345745330240508	0.002166748046875	\\
0.345867415455988	0.001922607421875	\\
0.345989500671469	0.001617431640625	\\
0.346111585886949	6.103515625e-05	\\
0.346233671102429	-0.001495361328125	\\
0.34635575631791	0.000640869140625	\\
0.34647784153339	0.00128173828125	\\
0.346599926748871	-0.000579833984375	\\
0.346722011964351	-0.001739501953125	\\
0.346844097179832	-0.000152587890625	\\
0.346966182395312	0.0003662109375	\\
0.347088267610792	0.000396728515625	\\
0.347210352826273	0.0013427734375	\\
0.347332438041753	0.00164794921875	\\
0.347454523257234	0.001495361328125	\\
0.347576608472714	0.002105712890625	\\
0.347698693688194	0.00238037109375	\\
0.347820778903675	0.002227783203125	\\
0.347942864119155	0.00140380859375	\\
0.348064949334636	6.103515625e-05	\\
0.348187034550116	-0.00030517578125	\\
0.348309119765596	-0.00054931640625	\\
0.348431204981077	9.1552734375e-05	\\
0.348553290196557	-0.001068115234375	\\
0.348675375412038	-0.0018310546875	\\
0.348797460627518	-0.0013427734375	\\
0.348919545842998	-0.001861572265625	\\
0.349041631058479	-0.0013427734375	\\
0.349163716273959	-0.001129150390625	\\
0.34928580148944	-0.000579833984375	\\
0.34940788670492	0.0006103515625	\\
0.3495299719204	0.00128173828125	\\
0.349652057135881	0.001495361328125	\\
0.349774142351361	0.00054931640625	\\
0.349896227566842	-0.0006103515625	\\
0.350018312782322	-0.0009765625	\\
0.350140397997802	-0.001434326171875	\\
0.350262483213283	-0.001861572265625	\\
0.350384568428763	-0.0015869140625	\\
0.350506653644244	-0.00189208984375	\\
0.350628738859724	-0.003021240234375	\\
0.350750824075204	-0.002349853515625	\\
0.350872909290685	-0.001556396484375	\\
0.350994994506165	-0.001312255859375	\\
0.351117079721646	-0.0006103515625	\\
0.351239164937126	-0.0006103515625	\\
0.351361250152607	9.1552734375e-05	\\
0.351483335368087	0.00103759765625	\\
0.351605420583567	0.000457763671875	\\
0.351727505799048	0.000335693359375	\\
0.351849591014528	0.000244140625	\\
0.351971676230009	-0.001007080078125	\\
0.352093761445489	-0.001556396484375	\\
0.352215846660969	-0.00091552734375	\\
0.35233793187645	-0.00128173828125	\\
0.35246001709193	-0.001800537109375	\\
0.352582102307411	-0.00152587890625	\\
0.352704187522891	-0.001434326171875	\\
0.352826272738371	-0.000885009765625	\\
0.352948357953852	-0.00018310546875	\\
0.353070443169332	-3.0517578125e-05	\\
0.353192528384813	0	\\
0.353314613600293	0.0006103515625	\\
0.353436698815773	0.001007080078125	\\
0.353558784031254	0.000732421875	\\
0.353680869246734	-0.000213623046875	\\
0.353802954462215	-0.000762939453125	\\
0.353925039677695	-0.000335693359375	\\
0.354047124893175	-6.103515625e-05	\\
0.354169210108656	-0.000946044921875	\\
0.354291295324136	-0.000946044921875	\\
0.354413380539617	-0.001251220703125	\\
0.354535465755097	-0.001678466796875	\\
0.354657550970577	-0.0006103515625	\\
0.354779636186058	-0.00054931640625	\\
0.354901721401538	-0.000335693359375	\\
0.355023806617019	0.0003662109375	\\
0.355145891832499	0.00018310546875	\\
0.355267977047979	0.0009765625	\\
0.35539006226346	0.001251220703125	\\
0.35551214747894	0.000823974609375	\\
0.355634232694421	0.000732421875	\\
0.355756317909901	0.000823974609375	\\
0.355878403125381	0.000396728515625	\\
0.356000488340862	3.0517578125e-05	\\
0.356122573556342	-3.0517578125e-05	\\
0.356244658771823	-0.001220703125	\\
0.356366743987303	-0.001495361328125	\\
0.356488829202784	-0.00146484375	\\
0.356610914418264	-0.001068115234375	\\
0.356732999633744	-0.000457763671875	\\
0.356855084849225	-0.00042724609375	\\
0.356977170064705	-6.103515625e-05	\\
0.357099255280186	0.000640869140625	\\
0.357221340495666	0.00091552734375	\\
0.357343425711146	0.001678466796875	\\
0.357465510926627	0.001983642578125	\\
0.357587596142107	0.0015869140625	\\
0.357709681357588	0.001495361328125	\\
0.357831766573068	0.001220703125	\\
0.357953851788548	0.000396728515625	\\
0.358075937004029	0.000213623046875	\\
0.358198022219509	0.0001220703125	\\
0.35832010743499	-0.00054931640625	\\
0.35844219265047	-0.000640869140625	\\
0.35856427786595	-0.000762939453125	\\
0.358686363081431	-0.001007080078125	\\
0.358808448296911	-0.00091552734375	\\
0.358930533512392	-0.0010986328125	\\
0.359052618727872	-0.000579833984375	\\
0.359174703943352	9.1552734375e-05	\\
0.359296789158833	0.00018310546875	\\
0.359418874374313	0.00030517578125	\\
0.359540959589794	0.00048828125	\\
0.359663044805274	0.001068115234375	\\
0.359785130020754	0.001220703125	\\
0.359907215236235	0.000946044921875	\\
0.360029300451715	0.001556396484375	\\
0.360151385667196	0.001800537109375	\\
0.360273470882676	0.0013427734375	\\
0.360395556098156	0.001495361328125	\\
0.360517641313637	0.00140380859375	\\
0.360639726529117	0.000823974609375	\\
0.360761811744598	0.0006103515625	\\
0.360883896960078	0.0010986328125	\\
0.361005982175559	0.000946044921875	\\
0.361128067391039	0.001068115234375	\\
0.361250152606519	0.001373291015625	\\
0.361372237822	0.00213623046875	\\
0.36149432303748	0.00274658203125	\\
0.361616408252961	0.002471923828125	\\
0.361738493468441	0.00262451171875	\\
0.361860578683921	0.003173828125	\\
0.361982663899402	0.0032958984375	\\
0.362104749114882	0.00274658203125	\\
0.362226834330363	0.002655029296875	\\
0.362348919545843	0.00286865234375	\\
0.362471004761323	0.003326416015625	\\
0.362593089976804	0.0035400390625	\\
0.362715175192284	0.0029296875	\\
0.362837260407765	0.002716064453125	\\
0.362959345623245	0.003021240234375	\\
0.363081430838725	0.00341796875	\\
0.363203516054206	0.003326416015625	\\
0.363325601269686	0.00299072265625	\\
0.363447686485167	0.00274658203125	\\
0.363569771700647	0.00286865234375	\\
0.363691856916127	0.0028076171875	\\
0.363813942131608	0.002899169921875	\\
0.363936027347088	0.002899169921875	\\
0.364058112562569	0.00238037109375	\\
0.364180197778049	0.0023193359375	\\
0.364302282993529	0.002166748046875	\\
0.36442436820901	0.002471923828125	\\
0.36454645342449	0.002288818359375	\\
0.364668538639971	0.0018310546875	\\
0.364790623855451	0.001953125	\\
0.364912709070931	0.001922607421875	\\
0.365034794286412	0.001983642578125	\\
0.365156879501892	0.00146484375	\\
0.365278964717373	0.001312255859375	\\
0.365401049932853	0.00140380859375	\\
0.365523135148334	0.0013427734375	\\
0.365645220363814	0.001434326171875	\\
0.365767305579294	0.00140380859375	\\
0.365889390794775	0.001007080078125	\\
0.366011476010255	0.000579833984375	\\
0.366133561225736	0.00048828125	\\
0.366255646441216	0.00054931640625	\\
0.366377731656696	0.000335693359375	\\
0.366499816872177	0.000335693359375	\\
0.366621902087657	-0.000244140625	\\
0.366743987303138	-0.00018310546875	\\
0.366866072518618	0.00042724609375	\\
0.366988157734098	0.000732421875	\\
0.367110242949579	0.000701904296875	\\
0.367232328165059	0.000579833984375	\\
0.36735441338054	0.000701904296875	\\
0.36747649859602	0.000396728515625	\\
0.3675985838115	0.0003662109375	\\
0.367720669026981	0.000213623046875	\\
0.367842754242461	-0.000244140625	\\
0.367964839457942	-0.00048828125	\\
0.368086924673422	-0.000213623046875	\\
0.368209009888902	-0.0008544921875	\\
0.368331095104383	-0.00079345703125	\\
0.368453180319863	-0.000732421875	\\
0.368575265535344	-0.00067138671875	\\
0.368697350750824	-0.00048828125	\\
0.368819435966304	-0.00042724609375	\\
0.368941521181785	0.000701904296875	\\
0.369063606397265	0.000823974609375	\\
0.369185691612746	0.000335693359375	\\
0.369307776828226	0.000396728515625	\\
0.369429862043706	0.000579833984375	\\
0.369551947259187	0.000640869140625	\\
0.369674032474667	0.00048828125	\\
0.369796117690148	0.0001220703125	\\
0.369918202905628	-0.000152587890625	\\
0.370040288121109	-0.000396728515625	\\
0.370162373336589	-0.000640869140625	\\
0.370284458552069	-0.000152587890625	\\
0.37040654376755	0.000274658203125	\\
0.37052862898303	0.00030517578125	\\
0.370650714198511	0.000518798828125	\\
0.370772799413991	0.000762939453125	\\
0.370894884629471	0.000640869140625	\\
0.371016969844952	0.00048828125	\\
0.371139055060432	0.000579833984375	\\
0.371261140275913	0.00042724609375	\\
0.371383225491393	-0.00018310546875	\\
0.371505310706873	-0.000152587890625	\\
0.371627395922354	-3.0517578125e-05	\\
0.371749481137834	-0.000579833984375	\\
0.371871566353315	-0.000701904296875	\\
0.371993651568795	-0.000213623046875	\\
0.372115736784275	-0.0001220703125	\\
0.372237821999756	0.000518798828125	\\
0.372359907215236	0.0008544921875	\\
0.372481992430717	0.0003662109375	\\
0.372604077646197	0.00018310546875	\\
0.372726162861677	0.0008544921875	\\
0.372848248077158	0.0008544921875	\\
0.372970333292638	0.000213623046875	\\
0.373092418508119	-0.00030517578125	\\
0.373214503723599	-0.00103759765625	\\
0.373336588939079	-0.001220703125	\\
0.37345867415456	-0.001708984375	\\
0.37358075937004	-0.00244140625	\\
0.373702844585521	-0.002288818359375	\\
0.373824929801001	-0.00244140625	\\
0.373947015016481	-0.002655029296875	\\
0.374069100231962	-0.002349853515625	\\
0.374191185447442	-0.001953125	\\
0.374313270662923	-0.002227783203125	\\
0.374435355878403	-0.00201416015625	\\
0.374557441093884	-0.0015869140625	\\
0.374679526309364	-0.001922607421875	\\
0.374801611524844	-0.002166748046875	\\
0.374923696740325	-0.002288818359375	\\
0.375045781955805	-0.00189208984375	\\
0.375167867171286	-0.001129150390625	\\
0.375289952386766	-0.001495361328125	\\
0.375412037602246	-0.001983642578125	\\
0.375534122817727	-0.00238037109375	\\
0.375656208033207	-0.00286865234375	\\
0.375778293248688	-0.003204345703125	\\
0.375900378464168	-0.003143310546875	\\
0.376022463679648	-0.00372314453125	\\
0.376144548895129	-0.00390625	\\
0.376266634110609	-0.00286865234375	\\
0.37638871932609	-0.001800537109375	\\
0.37651080454157	-0.00128173828125	\\
0.37663288975705	-0.000885009765625	\\
0.376754974972531	-0.000579833984375	\\
0.376877060188011	-0.000213623046875	\\
0.376999145403492	0.000213623046875	\\
0.377121230618972	0.00018310546875	\\
0.377243315834452	-0.000244140625	\\
0.377365401049933	-0.000244140625	\\
0.377487486265413	-0.000640869140625	\\
0.377609571480894	-0.001068115234375	\\
0.377731656696374	-0.001495361328125	\\
0.377853741911855	-0.001373291015625	\\
0.377975827127335	-0.001220703125	\\
0.378097912342815	-0.0013427734375	\\
0.378219997558296	-0.000885009765625	\\
0.378342082773776	-0.000762939453125	\\
0.378464167989257	-0.000579833984375	\\
0.378586253204737	-0.000244140625	\\
0.378708338420217	0.000396728515625	\\
0.378830423635698	0.000946044921875	\\
0.378952508851178	0.000579833984375	\\
0.379074594066659	0.000579833984375	\\
0.379196679282139	0.000885009765625	\\
0.379318764497619	0.001007080078125	\\
0.3794408497131	0.000823974609375	\\
0.37956293492858	0.000244140625	\\
0.379685020144061	0.00048828125	\\
0.379807105359541	0.00018310546875	\\
0.379929190575021	-0.00018310546875	\\
0.380051275790502	-0.0003662109375	\\
0.380173361005982	-0.0006103515625	\\
0.380295446221463	-0.000457763671875	\\
0.380417531436943	-0.000396728515625	\\
0.380539616652423	-0.000244140625	\\
0.380661701867904	9.1552734375e-05	\\
0.380783787083384	0.000396728515625	\\
0.380905872298865	0.001312255859375	\\
0.381027957514345	0.001739501953125	\\
0.381150042729825	0.001495361328125	\\
0.381272127945306	0.001312255859375	\\
0.381394213160786	0.00115966796875	\\
0.381516298376267	0.0006103515625	\\
0.381638383591747	0.00030517578125	\\
0.381760468807227	-6.103515625e-05	\\
0.381882554022708	-0.0001220703125	\\
0.382004639238188	0.0003662109375	\\
0.382126724453669	0.00048828125	\\
0.382248809669149	0.000701904296875	\\
0.38237089488463	0.000274658203125	\\
0.38249298010011	0.0003662109375	\\
0.38261506531559	0.001007080078125	\\
0.382737150531071	0.001373291015625	\\
0.382859235746551	0.00091552734375	\\
0.382981320962032	0.000335693359375	\\
0.383103406177512	0.0006103515625	\\
0.383225491392992	0.00042724609375	\\
0.383347576608473	0	\\
0.383469661823953	-0.000396728515625	\\
0.383591747039434	-0.000762939453125	\\
0.383713832254914	-0.001068115234375	\\
0.383835917470394	-0.001251220703125	\\
0.383958002685875	-0.0013427734375	\\
0.384080087901355	-0.001129150390625	\\
0.384202173116836	-0.001220703125	\\
0.384324258332316	-0.000946044921875	\\
0.384446343547796	-0.00067138671875	\\
0.384568428763277	-0.000457763671875	\\
0.384690513978757	-3.0517578125e-05	\\
0.384812599194238	0.00018310546875	\\
0.384934684409718	0.0003662109375	\\
0.385056769625198	9.1552734375e-05	\\
0.385178854840679	0.000640869140625	\\
0.385300940056159	0.00079345703125	\\
0.38542302527164	0.000579833984375	\\
0.38554511048712	0.000457763671875	\\
0.3856671957026	0.00030517578125	\\
0.385789280918081	-0.000244140625	\\
0.385911366133561	-0.00042724609375	\\
0.386033451349042	-0.000213623046875	\\
0.386155536564522	-0.00018310546875	\\
0.386277621780002	3.0517578125e-05	\\
0.386399706995483	-6.103515625e-05	\\
0.386521792210963	0.00054931640625	\\
0.386643877426444	0.000946044921875	\\
0.386765962641924	0.00091552734375	\\
0.386888047857404	0.001068115234375	\\
0.387010133072885	0.000701904296875	\\
0.387132218288365	0.000885009765625	\\
0.387254303503846	0.00128173828125	\\
0.387376388719326	0.001220703125	\\
0.387498473934807	0.00079345703125	\\
0.387620559150287	0.000640869140625	\\
0.387742644365767	0.000335693359375	\\
0.387864729581248	0.00030517578125	\\
0.387986814796728	6.103515625e-05	\\
0.388108900012209	-0.000762939453125	\\
0.388230985227689	-0.0008544921875	\\
0.388353070443169	-0.000579833984375	\\
0.38847515565865	-0.000396728515625	\\
0.38859724087413	-0.000396728515625	\\
0.388719326089611	-9.1552734375e-05	\\
0.388841411305091	-0.0001220703125	\\
0.388963496520571	-0.000457763671875	\\
0.389085581736052	-0.001007080078125	\\
0.389207666951532	-0.001007080078125	\\
0.389329752167013	-0.001251220703125	\\
0.389451837382493	-0.001617431640625	\\
0.389573922597973	-0.00128173828125	\\
0.389696007813454	-0.001373291015625	\\
0.389818093028934	-0.001739501953125	\\
0.389940178244415	-0.001953125	\\
0.390062263459895	-0.002105712890625	\\
0.390184348675375	-0.002227783203125	\\
0.390306433890856	-0.001922607421875	\\
0.390428519106336	-0.001953125	\\
0.390550604321817	-0.002532958984375	\\
0.390672689537297	-0.00238037109375	\\
0.390794774752777	-0.002288818359375	\\
0.390916859968258	-0.0029296875	\\
0.391038945183738	-0.00335693359375	\\
0.391161030399219	-0.00335693359375	\\
0.391283115614699	-0.00341796875	\\
0.391405200830179	-0.00372314453125	\\
0.39152728604566	-0.003753662109375	\\
0.39164937126114	-0.003173828125	\\
0.391771456476621	-0.002899169921875	\\
0.391893541692101	-0.002471923828125	\\
0.392015626907582	-0.00225830078125	\\
0.392137712123062	-0.002166748046875	\\
0.392259797338542	-0.002349853515625	\\
0.392381882554023	-0.0023193359375	\\
0.392503967769503	-0.0020751953125	\\
0.392626052984984	-0.001953125	\\
0.392748138200464	-0.00189208984375	\\
0.392870223415944	-0.002197265625	\\
0.392992308631425	-0.00262451171875	\\
0.393114393846905	-0.0028076171875	\\
0.393236479062386	-0.002960205078125	\\
0.393358564277866	-0.002838134765625	\\
0.393480649493346	-0.003204345703125	\\
0.393602734708827	-0.0035400390625	\\
0.393724819924307	-0.003326416015625	\\
0.393846905139788	-0.0029296875	\\
0.393968990355268	-0.0028076171875	\\
0.394091075570748	-0.00274658203125	\\
0.394213160786229	-0.002410888671875	\\
0.394335246001709	-0.002105712890625	\\
0.39445733121719	-0.002166748046875	\\
0.39457941643267	-0.00244140625	\\
0.39470150164815	-0.00238037109375	\\
0.394823586863631	-0.002105712890625	\\
0.394945672079111	-0.002349853515625	\\
0.395067757294592	-0.00274658203125	\\
0.395189842510072	-0.0025634765625	\\
0.395311927725552	-0.0028076171875	\\
0.395434012941033	-0.0030517578125	\\
0.395556098156513	-0.0028076171875	\\
0.395678183371994	-0.00244140625	\\
0.395800268587474	-0.002349853515625	\\
0.395922353802954	-0.00201416015625	\\
0.396044439018435	-0.001617431640625	\\
0.396166524233915	-0.0008544921875	\\
0.396288609449396	-0.00030517578125	\\
0.396410694664876	-0.000335693359375	\\
0.396532779880357	-0.00042724609375	\\
0.396654865095837	-0.000579833984375	\\
0.396776950311317	-0.000701904296875	\\
0.396899035526798	-0.00091552734375	\\
0.397021120742278	-0.001251220703125	\\
0.397143205957759	-0.001556396484375	\\
0.397265291173239	-0.001708984375	\\
0.397387376388719	-0.001983642578125	\\
0.3975094616042	-0.00189208984375	\\
0.39763154681968	-0.002044677734375	\\
0.397753632035161	-0.0023193359375	\\
0.397875717250641	-0.0020751953125	\\
0.397997802466121	-0.0020751953125	\\
0.398119887681602	-0.001495361328125	\\
0.398241972897082	-0.000732421875	\\
0.398364058112563	-0.00048828125	\\
0.398486143328043	-0.00048828125	\\
0.398608228543523	-0.00042724609375	\\
0.398730313759004	-0.000579833984375	\\
0.398852398974484	-0.00079345703125	\\
0.398974484189965	-0.00054931640625	\\
0.399096569405445	-0.0010986328125	\\
0.399218654620925	-0.00177001953125	\\
0.399340739836406	-0.001678466796875	\\
0.399462825051886	-0.00152587890625	\\
0.399584910267367	-0.001922607421875	\\
0.399706995482847	-0.002227783203125	\\
0.399829080698327	-0.001861572265625	\\
0.399951165913808	-0.001190185546875	\\
0.400073251129288	-0.000762939453125	\\
0.400195336344769	-0.000823974609375	\\
0.400317421560249	-0.0008544921875	\\
0.400439506775729	-0.000732421875	\\
0.40056159199121	-0.000274658203125	\\
0.40068367720669	0.0001220703125	\\
0.400805762422171	-0.0001220703125	\\
0.400927847637651	-0.00030517578125	\\
0.401049932853132	-0.000579833984375	\\
0.401172018068612	-0.000823974609375	\\
0.401294103284092	-0.0008544921875	\\
0.401416188499573	-0.000640869140625	\\
0.401538273715053	9.1552734375e-05	\\
0.401660358930534	3.0517578125e-05	\\
0.401782444146014	-0.00054931640625	\\
0.401904529361494	-6.103515625e-05	\\
0.402026614576975	0.00030517578125	\\
0.402148699792455	0.000640869140625	\\
0.402270785007936	0.00128173828125	\\
0.402392870223416	0.001190185546875	\\
0.402514955438896	0.00091552734375	\\
0.402637040654377	0.00115966796875	\\
0.402759125869857	0.00091552734375	\\
0.402881211085338	0.000579833984375	\\
0.403003296300818	0.000701904296875	\\
0.403125381516298	0.0003662109375	\\
0.403247466731779	0.000274658203125	\\
0.403369551947259	0.000640869140625	\\
0.40349163716274	0.0006103515625	\\
0.40361372237822	0.000579833984375	\\
0.4037358075937	0.000457763671875	\\
0.403857892809181	0.0006103515625	\\
0.403979978024661	0.00128173828125	\\
0.404102063240142	0.00152587890625	\\
0.404224148455622	0.001800537109375	\\
0.404346233671102	0.001495361328125	\\
0.404468318886583	0.0009765625	\\
0.404590404102063	0.000946044921875	\\
0.404712489317544	0.00067138671875	\\
0.404834574533024	0.00030517578125	\\
0.404956659748504	0	\\
0.405078744963985	-0.00030517578125	\\
0.405200830179465	-0.000396728515625	\\
0.405322915394946	-0.000335693359375	\\
0.405445000610426	0.0003662109375	\\
0.405567085825907	0.00018310546875	\\
0.405689171041387	0.0001220703125	\\
0.405811256256867	0.000518798828125	\\
0.405933341472348	0.000885009765625	\\
0.406055426687828	0.001434326171875	\\
0.406177511903309	0.00152587890625	\\
0.406299597118789	0.00146484375	\\
0.406421682334269	0.0013427734375	\\
0.40654376754975	0.001617431640625	\\
0.40666585276523	0.001678466796875	\\
0.406787937980711	0.001190185546875	\\
0.406910023196191	0.000518798828125	\\
0.407032108411671	0.000213623046875	\\
0.407154193627152	0.00030517578125	\\
0.407276278842632	-0.000152587890625	\\
0.407398364058113	-0.00018310546875	\\
0.407520449273593	-0.00042724609375	\\
0.407642534489073	-0.0006103515625	\\
0.407764619704554	-0.000579833984375	\\
0.407886704920034	-0.00054931640625	\\
0.408008790135515	-0.00054931640625	\\
0.408130875350995	-0.0003662109375	\\
0.408252960566475	-9.1552734375e-05	\\
0.408375045781956	0.00030517578125	\\
0.408497130997436	0.000457763671875	\\
0.408619216212917	-3.0517578125e-05	\\
0.408741301428397	-0.000396728515625	\\
0.408863386643877	-0.000396728515625	\\
0.408985471859358	-0.000518798828125	\\
0.409107557074838	-0.000518798828125	\\
0.409229642290319	-0.00048828125	\\
0.409351727505799	-0.00067138671875	\\
0.409473812721279	-0.000244140625	\\
0.40959589793676	-0.0003662109375	\\
0.40971798315224	-0.00018310546875	\\
0.409840068367721	0.00048828125	\\
0.409962153583201	0.0001220703125	\\
0.410084238798682	-0.00018310546875	\\
0.410206324014162	0.000244140625	\\
0.410328409229642	0	\\
0.410450494445123	-0.000213623046875	\\
0.410572579660603	0.00018310546875	\\
0.410694664876084	0	\\
0.410816750091564	-0.000579833984375	\\
0.410938835307044	-0.00054931640625	\\
0.411060920522525	-0.00054931640625	\\
0.411183005738005	-0.00042724609375	\\
0.411305090953486	-3.0517578125e-05	\\
0.411427176168966	3.0517578125e-05	\\
0.411549261384446	0	\\
0.411671346599927	0.00054931640625	\\
0.411793431815407	0.00103759765625	\\
0.411915517030888	0.000732421875	\\
0.412037602246368	0.00079345703125	\\
0.412159687461848	0.00091552734375	\\
0.412281772677329	0.00067138671875	\\
0.412403857892809	6.103515625e-05	\\
0.41252594310829	-0.00018310546875	\\
0.41264802832377	-0.000213623046875	\\
0.41277011353925	-0.00067138671875	\\
0.412892198754731	-0.000579833984375	\\
0.413014283970211	-0.000335693359375	\\
0.413136369185692	-0.000274658203125	\\
0.413258454401172	-0.00018310546875	\\
0.413380539616652	-6.103515625e-05	\\
0.413502624832133	-0.00018310546875	\\
0.413624710047613	-0.000152587890625	\\
0.413746795263094	0.000152587890625	\\
0.413868880478574	9.1552734375e-05	\\
0.413990965694054	-0.000274658203125	\\
0.414113050909535	-0.00048828125	\\
0.414235136125015	-0.000396728515625	\\
0.414357221340496	-0.00018310546875	\\
0.414479306555976	-0.000335693359375	\\
0.414601391771457	-0.000518798828125	\\
0.414723476986937	-0.0008544921875	\\
0.414845562202417	-0.001556396484375	\\
0.414967647417898	-0.00164794921875	\\
0.415089732633378	-0.001373291015625	\\
0.415211817848859	-0.00128173828125	\\
0.415333903064339	-0.0010986328125	\\
0.415455988279819	-0.0006103515625	\\
0.4155780734953	-0.000518798828125	\\
0.41570015871078	-0.0001220703125	\\
0.415822243926261	0.00018310546875	\\
0.415944329141741	0.00030517578125	\\
0.416066414357221	0.0001220703125	\\
0.416188499572702	0.000213623046875	\\
0.416310584788182	0.000335693359375	\\
0.416432670003663	0.000152587890625	\\
0.416554755219143	0.0001220703125	\\
0.416676840434623	0	\\
0.416798925650104	6.103515625e-05	\\
0.416921010865584	0.000213623046875	\\
0.417043096081065	0.000244140625	\\
0.417165181296545	0.00030517578125	\\
0.417287266512025	0.00018310546875	\\
0.417409351727506	0.00048828125	\\
0.417531436942986	0.0006103515625	\\
0.417653522158467	0.00079345703125	\\
0.417775607373947	0.001251220703125	\\
0.417897692589427	0.001373291015625	\\
0.418019777804908	0.001434326171875	\\
0.418141863020388	0.00115966796875	\\
0.418263948235869	0.001312255859375	\\
0.418386033451349	0.001617431640625	\\
0.418508118666829	0.0013427734375	\\
0.41863020388231	0.001220703125	\\
0.41875228909779	0.000823974609375	\\
0.418874374313271	0.000762939453125	\\
0.418996459528751	0.000946044921875	\\
0.419118544744231	0.000457763671875	\\
0.419240629959712	0.000335693359375	\\
0.419362715175192	0.000457763671875	\\
0.419484800390673	0.00054931640625	\\
0.419606885606153	0.00115966796875	\\
0.419728970821634	0.00189208984375	\\
0.419851056037114	0.00225830078125	\\
0.419973141252594	0.002197265625	\\
0.420095226468075	0.00177001953125	\\
0.420217311683555	0.001800537109375	\\
0.420339396899036	0.001708984375	\\
0.420461482114516	0.001251220703125	\\
0.420583567329996	0.000701904296875	\\
0.420705652545477	0.000244140625	\\
0.420827737760957	0.0003662109375	\\
0.420949822976438	0.000457763671875	\\
0.421071908191918	0.0006103515625	\\
0.421193993407398	0.000885009765625	\\
0.421316078622879	0.000885009765625	\\
0.421438163838359	0.00115966796875	\\
0.42156024905384	0.00177001953125	\\
0.42168233426932	0.00201416015625	\\
0.4218044194848	0.001708984375	\\
0.421926504700281	0.00189208984375	\\
0.422048589915761	0.00213623046875	\\
0.422170675131242	0.002044677734375	\\
0.422292760346722	0.001708984375	\\
0.422414845562202	0.00146484375	\\
0.422536930777683	0.001129150390625	\\
0.422659015993163	0.0009765625	\\
0.422781101208644	0.001190185546875	\\
0.422903186424124	0.00103759765625	\\
0.423025271639604	0.00067138671875	\\
0.423147356855085	0.0006103515625	\\
0.423269442070565	0.00030517578125	\\
0.423391527286046	0.000823974609375	\\
0.423513612501526	0.001495361328125	\\
0.423635697717006	0.000640869140625	\\
0.423757782932487	0.000335693359375	\\
0.423879868147967	0.00048828125	\\
0.424001953363448	0.000518798828125	\\
0.424124038578928	0.000457763671875	\\
0.424246123794409	0.000274658203125	\\
0.424368209009889	0.000579833984375	\\
0.424490294225369	0.00048828125	\\
0.42461237944085	-3.0517578125e-05	\\
0.42473446465633	-0.00018310546875	\\
0.424856549871811	-0.00018310546875	\\
0.424978635087291	-0.000335693359375	\\
0.425100720302771	3.0517578125e-05	\\
0.425222805518252	0.000274658203125	\\
0.425344890733732	0.00042724609375	\\
0.425466975949213	0.00067138671875	\\
0.425589061164693	0.000701904296875	\\
0.425711146380173	0.0010986328125	\\
0.425833231595654	0.001373291015625	\\
0.425955316811134	0.001495361328125	\\
0.426077402026615	0.001953125	\\
0.426199487242095	0.001861572265625	\\
0.426321572457575	0.0013427734375	\\
0.426443657673056	0.001251220703125	\\
0.426565742888536	0.001007080078125	\\
0.426687828104017	0.00079345703125	\\
0.426809913319497	0.00067138671875	\\
0.426931998534977	0.000732421875	\\
0.427054083750458	0.0009765625	\\
0.427176168965938	0.000732421875	\\
0.427298254181419	0.000946044921875	\\
0.427420339396899	0.001556396484375	\\
0.427542424612379	0.00189208984375	\\
0.42766450982786	0.001678466796875	\\
0.42778659504334	0.001434326171875	\\
0.427908680258821	0.00177001953125	\\
0.428030765474301	0.001800537109375	\\
0.428152850689781	0.001312255859375	\\
0.428274935905262	0.00115966796875	\\
0.428397021120742	0.00115966796875	\\
0.428519106336223	0.000946044921875	\\
0.428641191551703	0.00140380859375	\\
0.428763276767184	0.001373291015625	\\
0.428885361982664	0.0008544921875	\\
0.429007447198144	0.000762939453125	\\
0.429129532413625	0.000762939453125	\\
0.429251617629105	0.0006103515625	\\
0.429373702844586	0.00054931640625	\\
0.429495788060066	0.000396728515625	\\
0.429617873275546	0.00079345703125	\\
0.429739958491027	0.00115966796875	\\
0.429862043706507	0.0008544921875	\\
0.429984128921988	0.000701904296875	\\
0.430106214137468	0.00042724609375	\\
0.430228299352948	0.000213623046875	\\
0.430350384568429	-3.0517578125e-05	\\
0.430472469783909	-0.000457763671875	\\
0.43059455499939	-0.000335693359375	\\
0.43071664021487	-0.00030517578125	\\
0.43083872543035	-0.000396728515625	\\
0.430960810645831	-0.00018310546875	\\
0.431082895861311	3.0517578125e-05	\\
0.431204981076792	0.000213623046875	\\
0.431327066292272	0.0001220703125	\\
0.431449151507752	0.000274658203125	\\
0.431571236723233	0.0001220703125	\\
0.431693321938713	-9.1552734375e-05	\\
0.431815407154194	-0.00018310546875	\\
0.431937492369674	-0.000244140625	\\
0.432059577585154	-0.000213623046875	\\
0.432181662800635	-0.0006103515625	\\
0.432303748016115	-0.000823974609375	\\
0.432425833231596	-0.001312255859375	\\
0.432547918447076	-0.00164794921875	\\
0.432670003662556	-0.001068115234375	\\
0.432792088878037	-0.00091552734375	\\
0.432914174093517	-0.000640869140625	\\
0.433036259308998	-0.000701904296875	\\
0.433158344524478	-0.00091552734375	\\
0.433280429739959	-0.00030517578125	\\
0.433402514955439	-0.00018310546875	\\
0.433524600170919	-0.0001220703125	\\
0.4336466853864	0.00018310546875	\\
0.43376877060188	0.00030517578125	\\
0.433890855817361	0.000213623046875	\\
0.434012941032841	0.000335693359375	\\
0.434135026248321	0.000640869140625	\\
0.434257111463802	0.000518798828125	\\
0.434379196679282	-3.0517578125e-05	\\
0.434501281894763	-0.000640869140625	\\
0.434623367110243	-0.000701904296875	\\
0.434745452325723	-0.000213623046875	\\
0.434867537541204	-0.0003662109375	\\
0.434989622756684	-0.000244140625	\\
0.435111707972165	-0.000213623046875	\\
0.435233793187645	0.000457763671875	\\
0.435355878403125	0.001190185546875	\\
0.435477963618606	0.00128173828125	\\
0.435600048834086	0.0015869140625	\\
0.435722134049567	0.00164794921875	\\
0.435844219265047	0.0008544921875	\\
0.435966304480527	0.00048828125	\\
0.436088389696008	0.00030517578125	\\
0.436210474911488	3.0517578125e-05	\\
0.436332560126969	0	\\
0.436454645342449	3.0517578125e-05	\\
0.436576730557929	-0.0001220703125	\\
0.43669881577341	-0.00018310546875	\\
0.43682090098889	-6.103515625e-05	\\
0.436942986204371	-0.00018310546875	\\
0.437065071419851	-0.00048828125	\\
0.437187156635331	-0.000244140625	\\
0.437309241850812	0.000579833984375	\\
0.437431327066292	0.000701904296875	\\
0.437553412281773	0.00067138671875	\\
0.437675497497253	0.000640869140625	\\
0.437797582712734	0.00054931640625	\\
0.437919667928214	0.000244140625	\\
0.438041753143694	-6.103515625e-05	\\
0.438163838359175	-0.00054931640625	\\
0.438285923574655	-0.000701904296875	\\
0.438408008790136	-0.000762939453125	\\
0.438530094005616	-0.000732421875	\\
0.438652179221096	-0.00103759765625	\\
0.438774264436577	-0.00091552734375	\\
0.438896349652057	-0.00079345703125	\\
0.439018434867538	-0.00103759765625	\\
0.439140520083018	-0.00067138671875	\\
0.439262605298498	-0.0003662109375	\\
0.439384690513979	-0.00054931640625	\\
0.439506775729459	-0.000732421875	\\
0.43962886094494	-0.0006103515625	\\
0.43975094616042	-0.0006103515625	\\
0.4398730313759	-0.000762939453125	\\
0.439995116591381	-0.000823974609375	\\
0.440117201806861	-0.0010986328125	\\
0.440239287022342	-0.001312255859375	\\
0.440361372237822	-0.001190185546875	\\
0.440483457453302	-0.00091552734375	\\
0.440605542668783	-0.0009765625	\\
0.440727627884263	-0.00115966796875	\\
0.440849713099744	-0.00128173828125	\\
0.440971798315224	-0.00103759765625	\\
0.441093883530704	-0.000701904296875	\\
0.441215968746185	-0.00048828125	\\
0.441338053961665	-0.0001220703125	\\
0.441460139177146	-0.00030517578125	\\
0.441582224392626	-0.000396728515625	\\
0.441704309608106	0.00018310546875	\\
0.441826394823587	-0.000274658203125	\\
0.441948480039067	-0.000640869140625	\\
0.442070565254548	-0.0008544921875	\\
0.442192650470028	-0.001190185546875	\\
0.442314735685509	-0.0010986328125	\\
0.442436820900989	-0.001129150390625	\\
0.442558906116469	-0.00091552734375	\\
0.44268099133195	-0.000946044921875	\\
0.44280307654743	-0.00091552734375	\\
0.442925161762911	-0.00067138671875	\\
0.443047246978391	-0.00042724609375	\\
0.443169332193871	-0.000823974609375	\\
0.443291417409352	-0.000946044921875	\\
0.443413502624832	-0.000518798828125	\\
0.443535587840313	-0.0003662109375	\\
0.443657673055793	-0.000640869140625	\\
0.443779758271273	-0.00103759765625	\\
0.443901843486754	-0.0008544921875	\\
0.444023928702234	-0.001312255859375	\\
0.444146013917715	-0.00128173828125	\\
0.444268099133195	-0.001312255859375	\\
0.444390184348675	-0.0015869140625	\\
0.444512269564156	-0.0010986328125	\\
0.444634354779636	-0.001312255859375	\\
0.444756439995117	-0.00140380859375	\\
0.444878525210597	-0.000732421875	\\
0.445000610426077	-0.000244140625	\\
0.445122695641558	-0.000335693359375	\\
0.445244780857038	-0.000274658203125	\\
0.445366866072519	-0.000396728515625	\\
0.445488951287999	-0.000457763671875	\\
0.445611036503479	-0.000274658203125	\\
0.44573312171896	-0.000274658203125	\\
0.44585520693444	-0.00042724609375	\\
0.445977292149921	-0.0008544921875	\\
0.446099377365401	-0.00128173828125	\\
0.446221462580881	-0.00164794921875	\\
0.446343547796362	-0.001495361328125	\\
0.446465633011842	-0.001251220703125	\\
0.446587718227323	-0.001373291015625	\\
0.446709803442803	-0.001190185546875	\\
0.446831888658283	-0.00146484375	\\
0.446953973873764	-0.001251220703125	\\
0.447076059089244	-0.000701904296875	\\
0.447198144304725	-0.000823974609375	\\
0.447320229520205	-0.000885009765625	\\
0.447442314735686	-0.0008544921875	\\
0.447564399951166	-0.000732421875	\\
0.447686485166646	-0.000762939453125	\\
0.447808570382127	-0.000732421875	\\
0.447930655597607	-0.000762939453125	\\
0.448052740813088	-0.001251220703125	\\
0.448174826028568	-0.001708984375	\\
0.448296911244048	-0.001434326171875	\\
0.448418996459529	-0.001495361328125	\\
0.448541081675009	-0.0023193359375	\\
0.44866316689049	-0.0020751953125	\\
0.44878525210597	-0.001678466796875	\\
0.44890733732145	-0.00152587890625	\\
0.449029422536931	-0.001007080078125	\\
0.449151507752411	-0.00103759765625	\\
0.449273592967892	-0.000732421875	\\
0.449395678183372	0.0001220703125	\\
0.449517763398852	0	\\
0.449639848614333	0	\\
0.449761933829813	-0.0003662109375	\\
0.449884019045294	-0.00079345703125	\\
0.450006104260774	-0.00103759765625	\\
0.450128189476254	-0.00115966796875	\\
0.450250274691735	-0.001220703125	\\
0.450372359907215	-0.00115966796875	\\
0.450494445122696	-0.000823974609375	\\
0.450616530338176	-0.000579833984375	\\
0.450738615553656	-0.00079345703125	\\
0.450860700769137	-0.0010986328125	\\
0.450982785984617	-0.000640869140625	\\
0.451104871200098	-0.00048828125	\\
0.451226956415578	-0.00042724609375	\\
0.451349041631058	-0.00067138671875	\\
0.451471126846539	-0.00042724609375	\\
0.451593212062019	-0.000396728515625	\\
0.4517152972775	-0.000274658203125	\\
0.45183738249298	-0.000213623046875	\\
0.451959467708461	-0.0006103515625	\\
0.452081552923941	-0.000762939453125	\\
0.452203638139421	-0.00048828125	\\
0.452325723354902	-0.000579833984375	\\
0.452447808570382	-0.001068115234375	\\
0.452569893785863	-0.0010986328125	\\
0.452691979001343	-0.000579833984375	\\
0.452814064216823	-0.0001220703125	\\
0.452936149432304	-0.000518798828125	\\
0.453058234647784	-0.000640869140625	\\
0.453180319863265	-0.000457763671875	\\
0.453302405078745	-0.00054931640625	\\
0.453424490294225	-0.000732421875	\\
0.453546575509706	-0.0008544921875	\\
0.453668660725186	-0.001220703125	\\
0.453790745940667	-0.001708984375	\\
0.453912831156147	-0.00238037109375	\\
0.454034916371627	-0.002532958984375	\\
0.454157001587108	-0.002532958984375	\\
0.454279086802588	-0.00225830078125	\\
0.454401172018069	-0.001861572265625	\\
0.454523257233549	-0.00164794921875	\\
0.454645342449029	-0.001251220703125	\\
0.45476742766451	-0.0013427734375	\\
0.45488951287999	-0.001556396484375	\\
0.455011598095471	-0.001617431640625	\\
0.455133683310951	-0.001739501953125	\\
0.455255768526431	-0.002166748046875	\\
0.455377853741912	-0.0020751953125	\\
0.455499938957392	-0.002105712890625	\\
0.455622024172873	-0.002166748046875	\\
0.455744109388353	-0.002349853515625	\\
0.455866194603833	-0.002655029296875	\\
0.455988279819314	-0.0025634765625	\\
0.456110365034794	-0.0029296875	\\
0.456232450250275	-0.003204345703125	\\
0.456354535465755	-0.003173828125	\\
0.456476620681236	-0.003448486328125	\\
0.456598705896716	-0.0032958984375	\\
0.456720791112196	-0.00299072265625	\\
0.456842876327677	-0.002593994140625	\\
0.456964961543157	-0.002777099609375	\\
0.457087046758638	-0.0028076171875	\\
0.457209131974118	-0.00250244140625	\\
0.457331217189598	-0.002593994140625	\\
0.457453302405079	-0.002197265625	\\
0.457575387620559	-0.0020751953125	\\
0.45769747283604	-0.00244140625	\\
0.45781955805152	-0.00244140625	\\
0.457941643267	-0.002197265625	\\
0.458063728482481	-0.001800537109375	\\
0.458185813697961	-0.001861572265625	\\
0.458307898913442	-0.00201416015625	\\
0.458429984128922	-0.00213623046875	\\
0.458552069344402	-0.0023193359375	\\
0.458674154559883	-0.00238037109375	\\
0.458796239775363	-0.002044677734375	\\
0.458918324990844	-0.001953125	\\
0.459040410206324	-0.0013427734375	\\
0.459162495421804	-0.000701904296875	\\
0.459284580637285	-0.000885009765625	\\
0.459406665852765	-0.000885009765625	\\
0.459528751068246	-0.00128173828125	\\
0.459650836283726	-0.00177001953125	\\
0.459772921499206	-0.001800537109375	\\
0.459895006714687	-0.002105712890625	\\
0.460017091930167	-0.001739501953125	\\
0.460139177145648	-0.00152587890625	\\
0.460261262361128	-0.00128173828125	\\
0.460383347576608	-0.000823974609375	\\
0.460505432792089	-0.001220703125	\\
0.460627518007569	-0.001434326171875	\\
0.46074960322305	-0.001068115234375	\\
0.46087168843853	-0.000885009765625	\\
0.460993773654011	-0.0006103515625	\\
0.461115858869491	-0.0008544921875	\\
0.461237944084971	-0.0009765625	\\
0.461360029300452	-0.001007080078125	\\
0.461482114515932	-0.00128173828125	\\
0.461604199731413	-0.00164794921875	\\
0.461726284946893	-0.002288818359375	\\
0.461848370162373	-0.002410888671875	\\
0.461970455377854	-0.0020751953125	\\
0.462092540593334	-0.001953125	\\
0.462214625808815	-0.001800537109375	\\
0.462336711024295	-0.001739501953125	\\
0.462458796239775	-0.001983642578125	\\
0.462580881455256	-0.001861572265625	\\
0.462702966670736	-0.00164794921875	\\
0.462825051886217	-0.001678466796875	\\
0.462947137101697	-0.001434326171875	\\
0.463069222317177	-0.001373291015625	\\
0.463191307532658	-0.001678466796875	\\
0.463313392748138	-0.002044677734375	\\
0.463435477963619	-0.00238037109375	\\
0.463557563179099	-0.002227783203125	\\
0.463679648394579	-0.002471923828125	\\
0.46380173361006	-0.003265380859375	\\
0.46392381882554	-0.003326416015625	\\
0.464045904041021	-0.003082275390625	\\
0.464167989256501	-0.003448486328125	\\
0.464290074471981	-0.00347900390625	\\
0.464412159687462	-0.0029296875	\\
0.464534244902942	-0.002777099609375	\\
0.464656330118423	-0.0029296875	\\
0.464778415333903	-0.002838134765625	\\
0.464900500549383	-0.002593994140625	\\
0.465022585764864	-0.002288818359375	\\
0.465144670980344	-0.00250244140625	\\
0.465266756195825	-0.002716064453125	\\
0.465388841411305	-0.002716064453125	\\
0.465510926626786	-0.003021240234375	\\
0.465633011842266	-0.003326416015625	\\
0.465755097057746	-0.003570556640625	\\
0.465877182273227	-0.00390625	\\
0.465999267488707	-0.00396728515625	\\
0.466121352704188	-0.0040283203125	\\
0.466243437919668	-0.00396728515625	\\
0.466365523135148	-0.00372314453125	\\
0.466487608350629	-0.00311279296875	\\
0.466609693566109	-0.002838134765625	\\
0.46673177878159	-0.0025634765625	\\
0.46685386399707	-0.002288818359375	\\
0.46697594921255	-0.00238037109375	\\
0.467098034428031	-0.001983642578125	\\
0.467220119643511	-0.002288818359375	\\
0.467342204858992	-0.002838134765625	\\
0.467464290074472	-0.003082275390625	\\
0.467586375289952	-0.00347900390625	\\
0.467708460505433	-0.003082275390625	\\
0.467830545720913	-0.002899169921875	\\
0.467952630936394	-0.003021240234375	\\
0.468074716151874	-0.002899169921875	\\
0.468196801367354	-0.002716064453125	\\
0.468318886582835	-0.00286865234375	\\
0.468440971798315	-0.002899169921875	\\
0.468563057013796	-0.002288818359375	\\
0.468685142229276	-0.002166748046875	\\
0.468807227444756	-0.001922607421875	\\
0.468929312660237	-0.001739501953125	\\
0.469051397875717	-0.00189208984375	\\
0.469173483091198	-0.00213623046875	\\
0.469295568306678	-0.00244140625	\\
0.469417653522158	-0.00250244140625	\\
0.469539738737639	-0.002532958984375	\\
0.469661823953119	-0.002593994140625	\\
0.4697839091686	-0.00274658203125	\\
0.46990599438408	-0.00250244140625	\\
0.470028079599561	-0.002044677734375	\\
0.470150164815041	-0.00225830078125	\\
0.470272250030521	-0.001983642578125	\\
0.470394335246002	-0.001312255859375	\\
0.470516420461482	-0.00115966796875	\\
0.470638505676963	-0.001220703125	\\
0.470760590892443	-0.00128173828125	\\
0.470882676107923	-0.00164794921875	\\
0.471004761323404	-0.00189208984375	\\
0.471126846538884	-0.00238037109375	\\
0.471248931754365	-0.0029296875	\\
0.471371016969845	-0.003082275390625	\\
0.471493102185325	-0.0035400390625	\\
0.471615187400806	-0.003570556640625	\\
0.471737272616286	-0.003265380859375	\\
0.471859357831767	-0.003173828125	\\
0.471981443047247	-0.0029296875	\\
0.472103528262727	-0.0029296875	\\
0.472225613478208	-0.0029296875	\\
0.472347698693688	-0.0029296875	\\
0.472469783909169	-0.003387451171875	\\
0.472591869124649	-0.003387451171875	\\
0.472713954340129	-0.002685546875	\\
0.47283603955561	-0.002777099609375	\\
0.47295812477109	-0.003265380859375	\\
0.473080209986571	-0.00311279296875	\\
0.473202295202051	-0.003204345703125	\\
0.473324380417531	-0.00360107421875	\\
0.473446465633012	-0.004180908203125	\\
0.473568550848492	-0.004150390625	\\
0.473690636063973	-0.0042724609375	\\
0.473812721279453	-0.0040283203125	\\
0.473934806494933	-0.003173828125	\\
0.474056891710414	-0.003082275390625	\\
0.474178976925894	-0.00311279296875	\\
0.474301062141375	-0.00262451171875	\\
0.474423147356855	-0.002166748046875	\\
0.474545232572336	-0.0020751953125	\\
0.474667317787816	-0.002349853515625	\\
0.474789403003296	-0.002288818359375	\\
0.474911488218777	-0.002166748046875	\\
0.475033573434257	-0.00274658203125	\\
0.475155658649738	-0.002471923828125	\\
0.475277743865218	-0.002655029296875	\\
0.475399829080698	-0.002655029296875	\\
0.475521914296179	-0.002777099609375	\\
0.475643999511659	-0.0030517578125	\\
0.47576608472714	-0.0030517578125	\\
0.47588816994262	-0.002899169921875	\\
0.4760102551581	-0.00274658203125	\\
0.476132340373581	-0.00262451171875	\\
0.476254425589061	-0.00244140625	\\
0.476376510804542	-0.001800537109375	\\
0.476498596020022	-0.001708984375	\\
0.476620681235502	-0.0018310546875	\\
0.476742766450983	-0.001708984375	\\
0.476864851666463	-0.001739501953125	\\
0.476986936881944	-0.00177001953125	\\
0.477109022097424	-0.00201416015625	\\
0.477231107312904	-0.00244140625	\\
0.477353192528385	-0.0025634765625	\\
0.477475277743865	-0.002655029296875	\\
0.477597362959346	-0.00299072265625	\\
0.477719448174826	-0.003326416015625	\\
0.477841533390306	-0.0029296875	\\
0.477963618605787	-0.00286865234375	\\
0.478085703821267	-0.002960205078125	\\
0.478207789036748	-0.002349853515625	\\
0.478329874252228	-0.00189208984375	\\
0.478451959467708	-0.00189208984375	\\
0.478574044683189	-0.00146484375	\\
0.478696129898669	-0.001312255859375	\\
0.47881821511415	-0.00103759765625	\\
0.47894030032963	-0.00103759765625	\\
0.47906238554511	-0.001312255859375	\\
0.479184470760591	-0.001220703125	\\
0.479306555976071	-0.00140380859375	\\
0.479428641191552	-0.001922607421875	\\
0.479550726407032	-0.0020751953125	\\
0.479672811622513	-0.00201416015625	\\
0.479794896837993	-0.002288818359375	\\
0.479916982053473	-0.00274658203125	\\
0.480039067268954	-0.00262451171875	\\
0.480161152484434	-0.00244140625	\\
0.480283237699915	-0.002227783203125	\\
0.480405322915395	-0.002349853515625	\\
0.480527408130875	-0.002197265625	\\
0.480649493346356	-0.00164794921875	\\
0.480771578561836	-0.00146484375	\\
0.480893663777317	-0.00164794921875	\\
0.481015748992797	-0.00213623046875	\\
0.481137834208277	-0.00262451171875	\\
0.481259919423758	-0.002899169921875	\\
0.481382004639238	-0.00311279296875	\\
0.481504089854719	-0.002716064453125	\\
0.481626175070199	-0.003204345703125	\\
0.481748260285679	-0.003082275390625	\\
0.48187034550116	-0.00250244140625	\\
0.48199243071664	-0.00244140625	\\
0.482114515932121	-0.0023193359375	\\
0.482236601147601	-0.00262451171875	\\
0.482358686363081	-0.002105712890625	\\
0.482480771578562	-0.001678466796875	\\
0.482602856794042	-0.001251220703125	\\
0.482724942009523	-0.00115966796875	\\
0.482847027225003	-0.00140380859375	\\
0.482969112440483	-0.001007080078125	\\
0.483091197655964	-0.00091552734375	\\
0.483213282871444	-0.001800537109375	\\
0.483335368086925	-0.002197265625	\\
0.483457453302405	-0.002044677734375	\\
0.483579538517885	-0.002105712890625	\\
0.483701623733366	-0.001953125	\\
0.483823708948846	-0.001861572265625	\\
0.483945794164327	-0.001708984375	\\
0.484067879379807	-0.0015869140625	\\
0.484189964595288	-0.001373291015625	\\
0.484312049810768	-0.00146484375	\\
0.484434135026248	-0.001434326171875	\\
0.484556220241729	-0.00152587890625	\\
0.484678305457209	-0.001190185546875	\\
0.48480039067269	-0.001190185546875	\\
0.48492247588817	-0.00079345703125	\\
0.48504456110365	-0.000762939453125	\\
0.485166646319131	-0.000946044921875	\\
0.485288731534611	-0.0006103515625	\\
0.485410816750092	-0.000640869140625	\\
0.485532901965572	-0.0008544921875	\\
0.485654987181052	-0.001068115234375	\\
0.485777072396533	-0.001495361328125	\\
0.485899157612013	-0.001800537109375	\\
0.486021242827494	-0.00189208984375	\\
0.486143328042974	-0.001739501953125	\\
0.486265413258454	-0.001678466796875	\\
0.486387498473935	-0.001800537109375	\\
0.486509583689415	-0.001922607421875	\\
0.486631668904896	-0.0015869140625	\\
0.486753754120376	-0.001678466796875	\\
0.486875839335856	-0.001953125	\\
0.486997924551337	-0.0018310546875	\\
0.487120009766817	-0.002197265625	\\
0.487242094982298	-0.002044677734375	\\
0.487364180197778	-0.00177001953125	\\
0.487486265413258	-0.00146484375	\\
0.487608350628739	-0.00146484375	\\
0.487730435844219	-0.001739501953125	\\
0.4878525210597	-0.001617431640625	\\
0.48797460627518	-0.0015869140625	\\
0.48809669149066	-0.001739501953125	\\
0.488218776706141	-0.001922607421875	\\
0.488340861921621	-0.001800537109375	\\
};
\addplot [color=blue,solid,forget plot]
  table[row sep=crcr]{
0.488340861921621	-0.001800537109375	\\
0.488462947137102	-0.00177001953125	\\
0.488585032352582	-0.0018310546875	\\
0.488707117568063	-0.001617431640625	\\
0.488829202783543	-0.001708984375	\\
0.488951287999023	-0.00213623046875	\\
0.489073373214504	-0.002166748046875	\\
0.489195458429984	-0.001617431640625	\\
0.489317543645465	-0.001708984375	\\
0.489439628860945	-0.001739501953125	\\
0.489561714076425	-0.001861572265625	\\
0.489683799291906	-0.00177001953125	\\
0.489805884507386	-0.001220703125	\\
0.489927969722867	-0.000885009765625	\\
0.490050054938347	-0.000823974609375	\\
0.490172140153827	-0.000732421875	\\
0.490294225369308	-0.000732421875	\\
0.490416310584788	-0.001129150390625	\\
0.490538395800269	-0.001068115234375	\\
0.490660481015749	-0.0009765625	\\
0.490782566231229	-0.001373291015625	\\
0.49090465144671	-0.000701904296875	\\
0.49102673666219	-0.000762939453125	\\
0.491148821877671	-0.000885009765625	\\
0.491270907093151	-0.000579833984375	\\
0.491392992308631	-0.000732421875	\\
0.491515077524112	-0.000335693359375	\\
0.491637162739592	-0.000213623046875	\\
0.491759247955073	9.1552734375e-05	\\
0.491881333170553	0.0006103515625	\\
0.492003418386033	0.00042724609375	\\
0.492125503601514	0.000640869140625	\\
0.492247588816994	0.0006103515625	\\
0.492369674032475	0.000701904296875	\\
0.492491759247955	0.000885009765625	\\
0.492613844463435	0.0009765625	\\
0.492735929678916	0.00048828125	\\
0.492858014894396	0.0006103515625	\\
0.492980100109877	0.000640869140625	\\
0.493102185325357	0.00042724609375	\\
0.493224270540838	0.00042724609375	\\
0.493346355756318	0.000762939453125	\\
0.493468440971798	0.0009765625	\\
0.493590526187279	0.001190185546875	\\
0.493712611402759	0.001953125	\\
0.49383469661824	0.002288818359375	\\
0.49395678183372	0.002655029296875	\\
0.4940788670492	0.002716064453125	\\
0.494200952264681	0.0029296875	\\
0.494323037480161	0.00262451171875	\\
0.494445122695642	0.00238037109375	\\
0.494567207911122	0.002471923828125	\\
0.494689293126602	0.0018310546875	\\
0.494811378342083	0.001678466796875	\\
0.494933463557563	0.0013427734375	\\
0.495055548773044	0.00054931640625	\\
0.495177633988524	0.000701904296875	\\
0.495299719204004	0.0006103515625	\\
0.495421804419485	0.00042724609375	\\
0.495543889634965	0.000732421875	\\
0.495665974850446	0.0006103515625	\\
0.495788060065926	0.00091552734375	\\
0.495910145281406	0.00128173828125	\\
0.496032230496887	0.00164794921875	\\
0.496154315712367	0.00146484375	\\
0.496276400927848	0.001251220703125	\\
0.496398486143328	0.00152587890625	\\
0.496520571358808	0.001434326171875	\\
0.496642656574289	0.001251220703125	\\
0.496764741789769	0.0013427734375	\\
0.49688682700525	0.000885009765625	\\
0.49700891222073	0.000579833984375	\\
0.49713099743621	0.000213623046875	\\
0.497253082651691	9.1552734375e-05	\\
0.497375167867171	3.0517578125e-05	\\
0.497497253082652	-0.000152587890625	\\
0.497619338298132	-9.1552734375e-05	\\
0.497741423513613	-3.0517578125e-05	\\
0.497863508729093	0.000518798828125	\\
0.497985593944573	0.000640869140625	\\
0.498107679160054	0.000885009765625	\\
0.498229764375534	0.0009765625	\\
0.498351849591015	0.00042724609375	\\
0.498473934806495	0.000518798828125	\\
0.498596020021975	0.00030517578125	\\
0.498718105237456	0.000213623046875	\\
0.498840190452936	3.0517578125e-05	\\
0.498962275668417	-0.000762939453125	\\
0.499084360883897	-0.000762939453125	\\
0.499206446099377	-0.000213623046875	\\
0.499328531314858	-6.103515625e-05	\\
0.499450616530338	0.0003662109375	\\
0.499572701745819	0.000640869140625	\\
0.499694786961299	0.000579833984375	\\
0.499816872176779	0.000457763671875	\\
0.49993895739226	0.00067138671875	\\
0.50006104260774	0.000946044921875	\\
0.500183127823221	0.000762939453125	\\
0.500305213038701	0.001068115234375	\\
0.500427298254181	0.00103759765625	\\
0.500549383469662	0.0006103515625	\\
0.500671468685142	0.0006103515625	\\
0.500793553900623	0.000762939453125	\\
0.500915639116103	0.000579833984375	\\
0.501037724331583	0.000640869140625	\\
0.501159809547064	0.000518798828125	\\
0.501281894762544	0.000213623046875	\\
0.501403979978025	0.0006103515625	\\
0.501526065193505	0.0010986328125	\\
0.501648150408985	0.000762939453125	\\
0.501770235624466	0.00042724609375	\\
0.501892320839946	0.000823974609375	\\
0.502014406055427	0.000518798828125	\\
0.502136491270907	-3.0517578125e-05	\\
0.502258576486388	-0.000213623046875	\\
0.502380661701868	-0.0003662109375	\\
0.502502746917348	-0.000518798828125	\\
0.502624832132829	-0.000701904296875	\\
0.502746917348309	-0.00091552734375	\\
0.50286900256379	-0.001251220703125	\\
0.50299108777927	-0.001251220703125	\\
0.50311317299475	-0.00115966796875	\\
0.503235258210231	-0.0008544921875	\\
0.503357343425711	-0.00067138671875	\\
0.503479428641192	-0.0003662109375	\\
0.503601513856672	0	\\
0.503723599072152	-0.000732421875	\\
0.503845684287633	-0.00042724609375	\\
0.503967769503113	-0.0003662109375	\\
0.504089854718594	-0.000579833984375	\\
0.504211939934074	-0.00042724609375	\\
0.504334025149554	-0.00091552734375	\\
0.504456110365035	-0.00146484375	\\
0.504578195580515	-0.001434326171875	\\
0.504700280795996	-0.001312255859375	\\
0.504822366011476	-0.001556396484375	\\
0.504944451226956	-0.001739501953125	\\
0.505066536442437	-0.00164794921875	\\
0.505188621657917	-0.0015869140625	\\
0.505310706873398	-0.00189208984375	\\
0.505432792088878	-0.00201416015625	\\
0.505554877304358	-0.0020751953125	\\
0.505676962519839	-0.001373291015625	\\
0.505799047735319	-0.001373291015625	\\
0.5059211329508	-0.001708984375	\\
0.50604321816628	-0.00177001953125	\\
0.50616530338176	-0.001495361328125	\\
0.506287388597241	-0.001678466796875	\\
0.506409473812721	-0.001617431640625	\\
0.506531559028202	-0.001708984375	\\
0.506653644243682	-0.002166748046875	\\
0.506775729459162	-0.00213623046875	\\
0.506897814674643	-0.002044677734375	\\
0.507019899890123	-0.002349853515625	\\
0.507141985105604	-0.002655029296875	\\
0.507264070321084	-0.002685546875	\\
0.507386155536565	-0.002685546875	\\
0.507508240752045	-0.001953125	\\
0.507630325967525	-0.001495361328125	\\
0.507752411183006	-0.0015869140625	\\
0.507874496398486	-0.00115966796875	\\
0.507996581613967	-0.001129150390625	\\
0.508118666829447	-0.0008544921875	\\
0.508240752044927	-0.000396728515625	\\
0.508362837260408	-0.001007080078125	\\
0.508484922475888	-0.001251220703125	\\
0.508607007691369	-0.001220703125	\\
0.508729092906849	-0.00128173828125	\\
0.508851178122329	-0.00115966796875	\\
0.50897326333781	-0.00146484375	\\
0.50909534855329	-0.00115966796875	\\
0.509217433768771	-0.001129150390625	\\
0.509339518984251	-0.00091552734375	\\
0.509461604199731	-0.00048828125	\\
0.509583689415212	-0.000640869140625	\\
0.509705774630692	-0.00067138671875	\\
0.509827859846173	-0.00054931640625	\\
0.509949945061653	-0.000396728515625	\\
0.510072030277133	-6.103515625e-05	\\
0.510194115492614	-0.000396728515625	\\
0.510316200708094	-0.000457763671875	\\
0.510438285923575	-0.000335693359375	\\
0.510560371139055	-0.001007080078125	\\
0.510682456354535	-0.0009765625	\\
0.510804541570016	-0.0013427734375	\\
0.510926626785496	-0.0020751953125	\\
0.511048712000977	-0.001495361328125	\\
0.511170797216457	-0.001190185546875	\\
0.511292882431937	-0.001251220703125	\\
0.511414967647418	-0.001220703125	\\
0.511537052862898	-0.001129150390625	\\
0.511659138078379	-0.001251220703125	\\
0.511781223293859	-0.00128173828125	\\
0.51190330850934	-0.0010986328125	\\
0.51202539372482	-0.00103759765625	\\
0.5121474789403	-0.00115966796875	\\
0.512269564155781	-0.001434326171875	\\
0.512391649371261	-0.001708984375	\\
0.512513734586742	-0.00164794921875	\\
0.512635819802222	-0.00164794921875	\\
0.512757905017702	-0.0018310546875	\\
0.512879990233183	-0.001800537109375	\\
0.513002075448663	-0.00164794921875	\\
0.513124160664144	-0.0015869140625	\\
0.513246245879624	-0.00177001953125	\\
0.513368331095104	-0.00164794921875	\\
0.513490416310585	-0.001190185546875	\\
0.513612501526065	-0.0013427734375	\\
0.513734586741546	-0.001312255859375	\\
0.513856671957026	-0.0009765625	\\
0.513978757172506	-0.000885009765625	\\
0.514100842387987	-0.000579833984375	\\
0.514222927603467	-0.00018310546875	\\
0.514345012818948	-0.00030517578125	\\
0.514467098034428	-0.000335693359375	\\
0.514589183249908	-0.000213623046875	\\
0.514711268465389	-0.000457763671875	\\
0.514833353680869	-0.000885009765625	\\
0.51495543889635	-0.00042724609375	\\
0.51507752411183	-0.000457763671875	\\
0.51519960932731	-0.00042724609375	\\
0.515321694542791	0.000244140625	\\
0.515443779758271	0.00048828125	\\
0.515565864973752	0.000518798828125	\\
0.515687950189232	0.000335693359375	\\
0.515810035404712	0.0008544921875	\\
0.515932120620193	0.0006103515625	\\
0.516054205835673	0.00042724609375	\\
0.516176291051154	0.00067138671875	\\
0.516298376266634	0.000640869140625	\\
0.516420461482115	0.000732421875	\\
0.516542546697595	0.000946044921875	\\
0.516664631913075	0.001068115234375	\\
0.516786717128556	0.00103759765625	\\
0.516908802344036	0.001007080078125	\\
0.517030887559517	0.001220703125	\\
0.517152972774997	0.00152587890625	\\
0.517275057990477	0.00146484375	\\
0.517397143205958	0.00164794921875	\\
0.517519228421438	0.001617431640625	\\
0.517641313636919	0.001708984375	\\
0.517763398852399	0.001983642578125	\\
0.517885484067879	0.001922607421875	\\
0.51800756928336	0.001617431640625	\\
0.51812965449884	0.00201416015625	\\
0.518251739714321	0.002166748046875	\\
0.518373824929801	0.00140380859375	\\
0.518495910145281	0.0010986328125	\\
0.518617995360762	0.001312255859375	\\
0.518740080576242	0.001739501953125	\\
0.518862165791723	0.0015869140625	\\
0.518984251007203	0.001312255859375	\\
0.519106336222683	0.00140380859375	\\
0.519228421438164	0.001251220703125	\\
0.519350506653644	0.0010986328125	\\
0.519472591869125	0.001129150390625	\\
0.519594677084605	0.00103759765625	\\
0.519716762300085	0.001068115234375	\\
0.519838847515566	0.0006103515625	\\
0.519960932731046	0.00030517578125	\\
0.520083017946527	0.000274658203125	\\
0.520205103162007	3.0517578125e-05	\\
0.520327188377487	-0.00018310546875	\\
0.520449273592968	-0.000152587890625	\\
0.520571358808448	-0.000518798828125	\\
0.520693444023929	-0.00115966796875	\\
0.520815529239409	-0.001129150390625	\\
0.52093761445489	-0.00067138671875	\\
0.52105969967037	-0.00067138671875	\\
0.52118178488585	-0.000946044921875	\\
0.521303870101331	-0.000762939453125	\\
0.521425955316811	-0.0009765625	\\
0.521548040532292	-0.001220703125	\\
0.521670125747772	-0.001190185546875	\\
0.521792210963252	-0.001007080078125	\\
0.521914296178733	-0.000885009765625	\\
0.522036381394213	-0.001251220703125	\\
0.522158466609694	-0.001373291015625	\\
0.522280551825174	-0.00164794921875	\\
0.522402637040654	-0.001708984375	\\
0.522524722256135	-0.00152587890625	\\
0.522646807471615	-0.001434326171875	\\
0.522768892687096	-0.0015869140625	\\
0.522890977902576	-0.0015869140625	\\
0.523013063118056	-0.00115966796875	\\
0.523135148333537	-0.00091552734375	\\
0.523257233549017	-0.00091552734375	\\
0.523379318764498	-0.00091552734375	\\
0.523501403979978	-0.00079345703125	\\
0.523623489195458	-0.00103759765625	\\
0.523745574410939	-0.001007080078125	\\
0.523867659626419	-0.0010986328125	\\
0.5239897448419	-0.001434326171875	\\
0.52411183005738	-0.001373291015625	\\
0.52423391527286	-0.00177001953125	\\
0.524356000488341	-0.0018310546875	\\
0.524478085703821	-0.001708984375	\\
0.524600170919302	-0.001739501953125	\\
0.524722256134782	-0.00177001953125	\\
0.524844341350262	-0.001617431640625	\\
0.524966426565743	-0.001373291015625	\\
0.525088511781223	-0.00079345703125	\\
0.525210596996704	-0.00018310546875	\\
0.525332682212184	-0.00042724609375	\\
0.525454767427665	-0.000579833984375	\\
0.525576852643145	-0.0003662109375	\\
0.525698937858625	-0.00054931640625	\\
0.525821023074106	-0.000885009765625	\\
0.525943108289586	-0.0009765625	\\
0.526065193505067	-0.001556396484375	\\
0.526187278720547	-0.001800537109375	\\
0.526309363936027	-0.001678466796875	\\
0.526431449151508	-0.0018310546875	\\
0.526553534366988	-0.002197265625	\\
0.526675619582469	-0.002227783203125	\\
0.526797704797949	-0.00213623046875	\\
0.526919790013429	-0.002166748046875	\\
0.52704187522891	-0.00225830078125	\\
0.52716396044439	-0.001953125	\\
0.527286045659871	-0.001953125	\\
0.527408130875351	-0.0018310546875	\\
0.527530216090831	-0.001708984375	\\
0.527652301306312	-0.00201416015625	\\
0.527774386521792	-0.002349853515625	\\
0.527896471737273	-0.002410888671875	\\
0.528018556952753	-0.002685546875	\\
0.528140642168233	-0.00274658203125	\\
0.528262727383714	-0.003021240234375	\\
0.528384812599194	-0.0030517578125	\\
0.528506897814675	-0.00311279296875	\\
0.528628983030155	-0.003021240234375	\\
0.528751068245635	-0.00311279296875	\\
0.528873153461116	-0.003265380859375	\\
0.528995238676596	-0.00299072265625	\\
0.529117323892077	-0.00274658203125	\\
0.529239409107557	-0.002716064453125	\\
0.529361494323037	-0.002777099609375	\\
0.529483579538518	-0.003021240234375	\\
0.529605664753998	-0.00335693359375	\\
0.529727749969479	-0.003753662109375	\\
0.529849835184959	-0.003570556640625	\\
0.52997192040044	-0.0032958984375	\\
0.53009400561592	-0.00384521484375	\\
0.5302160908314	-0.004058837890625	\\
0.530338176046881	-0.00384521484375	\\
0.530460261262361	-0.00396728515625	\\
0.530582346477842	-0.003662109375	\\
0.530704431693322	-0.003662109375	\\
0.530826516908802	-0.0035400390625	\\
0.530948602124283	-0.003204345703125	\\
0.531070687339763	-0.002960205078125	\\
0.531192772555244	-0.00299072265625	\\
0.531314857770724	-0.0023193359375	\\
0.531436942986204	-0.0020751953125	\\
0.531559028201685	-0.00262451171875	\\
0.531681113417165	-0.002288818359375	\\
0.531803198632646	-0.00201416015625	\\
0.531925283848126	-0.002593994140625	\\
0.532047369063606	-0.00274658203125	\\
0.532169454279087	-0.00262451171875	\\
0.532291539494567	-0.00311279296875	\\
0.532413624710048	-0.00286865234375	\\
0.532535709925528	-0.002105712890625	\\
0.532657795141008	-0.001739501953125	\\
0.532779880356489	-0.001312255859375	\\
0.532901965571969	-0.001129150390625	\\
0.53302405078745	-0.00103759765625	\\
0.53314613600293	-0.00091552734375	\\
0.53326822121841	-0.000732421875	\\
0.533390306433891	-0.0003662109375	\\
0.533512391649371	-0.000152587890625	\\
0.533634476864852	-0.000274658203125	\\
0.533756562080332	-0.000244140625	\\
0.533878647295812	-0.000244140625	\\
0.534000732511293	-0.000213623046875	\\
0.534122817726773	-0.000518798828125	\\
0.534244902942254	-0.000732421875	\\
0.534366988157734	-0.000335693359375	\\
0.534489073373215	-0.000457763671875	\\
0.534611158588695	-0.000579833984375	\\
0.534733243804175	-0.000762939453125	\\
0.534855329019656	-0.000732421875	\\
0.534977414235136	-0.00018310546875	\\
0.535099499450617	6.103515625e-05	\\
0.535221584666097	-0.000213623046875	\\
0.535343669881577	-0.000213623046875	\\
0.535465755097058	-0.00018310546875	\\
0.535587840312538	-0.0006103515625	\\
0.535709925528019	-0.000457763671875	\\
0.535832010743499	-0.000335693359375	\\
0.535954095958979	-0.00042724609375	\\
0.53607618117446	-0.00042724609375	\\
0.53619826638994	-0.00018310546875	\\
0.536320351605421	-0.000274658203125	\\
0.536442436820901	-0.0003662109375	\\
0.536564522036381	-0.0006103515625	\\
0.536686607251862	-0.000274658203125	\\
0.536808692467342	0	\\
0.536930777682823	-0.000457763671875	\\
0.537052862898303	-0.000823974609375	\\
0.537174948113783	-0.000640869140625	\\
0.537297033329264	-0.001068115234375	\\
0.537419118544744	-0.001251220703125	\\
0.537541203760225	-0.0009765625	\\
0.537663288975705	-0.001251220703125	\\
0.537785374191185	-0.001495361328125	\\
0.537907459406666	-0.001617431640625	\\
0.538029544622146	-0.001739501953125	\\
0.538151629837627	-0.001495361328125	\\
0.538273715053107	-0.00103759765625	\\
0.538395800268587	-0.00103759765625	\\
0.538517885484068	-0.00103759765625	\\
0.538639970699548	-0.000823974609375	\\
0.538762055915029	-0.00042724609375	\\
0.538884141130509	-0.000396728515625	\\
0.539006226345989	-9.1552734375e-05	\\
0.53912831156147	9.1552734375e-05	\\
0.53925039677695	-0.00018310546875	\\
0.539372481992431	-0.000274658203125	\\
0.539494567207911	-0.00048828125	\\
0.539616652423392	-0.0008544921875	\\
0.539738737638872	-0.001129150390625	\\
0.539860822854352	-0.001373291015625	\\
0.539982908069833	-0.001251220703125	\\
0.540104993285313	-0.00067138671875	\\
0.540227078500794	-0.000701904296875	\\
0.540349163716274	-0.001312255859375	\\
0.540471248931754	-0.0010986328125	\\
0.540593334147235	-0.000732421875	\\
0.540715419362715	-0.000274658203125	\\
0.540837504578196	-0.00042724609375	\\
0.540959589793676	-0.00054931640625	\\
0.541081675009156	-0.00030517578125	\\
0.541203760224637	-0.00067138671875	\\
0.541325845440117	-0.00048828125	\\
0.541447930655598	-0.0006103515625	\\
0.541570015871078	-0.000701904296875	\\
0.541692101086558	-0.001220703125	\\
0.541814186302039	-0.001556396484375	\\
0.541936271517519	-0.001373291015625	\\
0.542058356733	-0.001922607421875	\\
0.54218044194848	-0.002044677734375	\\
0.54230252716396	-0.002197265625	\\
0.542424612379441	-0.00201416015625	\\
0.542546697594921	-0.001556396484375	\\
0.542668782810402	-0.001251220703125	\\
0.542790868025882	-0.001312255859375	\\
0.542912953241362	-0.001190185546875	\\
0.543035038456843	-0.0003662109375	\\
0.543157123672323	-0.00048828125	\\
0.543279208887804	-0.000732421875	\\
0.543401294103284	-0.00067138671875	\\
0.543523379318764	-0.000732421875	\\
0.543645464534245	-0.00115966796875	\\
0.543767549749725	-0.001434326171875	\\
0.543889634965206	-0.00140380859375	\\
0.544011720180686	-0.001983642578125	\\
0.544133805396167	-0.002288818359375	\\
0.544255890611647	-0.002410888671875	\\
0.544377975827127	-0.002716064453125	\\
0.544500061042608	-0.00274658203125	\\
0.544622146258088	-0.002288818359375	\\
0.544744231473569	-0.002227783203125	\\
0.544866316689049	-0.00225830078125	\\
0.544988401904529	-0.00225830078125	\\
0.54511048712001	-0.00201416015625	\\
0.54523257233549	-0.00244140625	\\
0.545354657550971	-0.00286865234375	\\
0.545476742766451	-0.003082275390625	\\
0.545598827981931	-0.0032958984375	\\
0.545720913197412	-0.00347900390625	\\
0.545842998412892	-0.00360107421875	\\
0.545965083628373	-0.00408935546875	\\
0.546087168843853	-0.00445556640625	\\
0.546209254059333	-0.004302978515625	\\
0.546331339274814	-0.0042724609375	\\
0.546453424490294	-0.00433349609375	\\
0.546575509705775	-0.003936767578125	\\
0.546697594921255	-0.0035400390625	\\
0.546819680136735	-0.00360107421875	\\
0.546941765352216	-0.0030517578125	\\
0.547063850567696	-0.002960205078125	\\
0.547185935783177	-0.00335693359375	\\
0.547308020998657	-0.00299072265625	\\
0.547430106214137	-0.00299072265625	\\
0.547552191429618	-0.002960205078125	\\
0.547674276645098	-0.003143310546875	\\
0.547796361860579	-0.003753662109375	\\
0.547918447076059	-0.003662109375	\\
0.548040532291539	-0.0035400390625	\\
0.54816261750702	-0.003265380859375	\\
0.5482847027225	-0.003387451171875	\\
0.548406787937981	-0.00335693359375	\\
0.548528873153461	-0.002777099609375	\\
0.548650958368942	-0.00213623046875	\\
0.548773043584422	-0.00189208984375	\\
0.548895128799902	-0.001708984375	\\
0.549017214015383	-0.001861572265625	\\
0.549139299230863	-0.0018310546875	\\
0.549261384446344	-0.001861572265625	\\
0.549383469661824	-0.002349853515625	\\
0.549505554877304	-0.002349853515625	\\
0.549627640092785	-0.00238037109375	\\
0.549749725308265	-0.00238037109375	\\
0.549871810523746	-0.002166748046875	\\
0.549993895739226	-0.001708984375	\\
0.550115980954706	-0.001708984375	\\
0.550238066170187	-0.001953125	\\
0.550360151385667	-0.001983642578125	\\
0.550482236601148	-0.001861572265625	\\
0.550604321816628	-0.0010986328125	\\
0.550726407032108	-0.001129150390625	\\
0.550848492247589	-0.001190185546875	\\
0.550970577463069	-0.001007080078125	\\
0.55109266267855	-0.001190185546875	\\
0.55121474789403	-0.001190185546875	\\
0.55133683310951	-0.001373291015625	\\
0.551458918324991	-0.001708984375	\\
0.551581003540471	-0.0020751953125	\\
0.551703088755952	-0.002349853515625	\\
0.551825173971432	-0.00250244140625	\\
0.551947259186912	-0.002471923828125	\\
0.552069344402393	-0.002197265625	\\
0.552191429617873	-0.0020751953125	\\
0.552313514833354	-0.001739501953125	\\
0.552435600048834	-0.001800537109375	\\
0.552557685264314	-0.001495361328125	\\
0.552679770479795	-0.00115966796875	\\
0.552801855695275	-0.001129150390625	\\
0.552923940910756	-0.00115966796875	\\
0.553046026126236	-0.00140380859375	\\
0.553168111341717	-0.00146484375	\\
0.553290196557197	-0.001556396484375	\\
0.553412281772677	-0.001739501953125	\\
0.553534366988158	-0.00177001953125	\\
0.553656452203638	-0.002044677734375	\\
0.553778537419119	-0.002044677734375	\\
0.553900622634599	-0.001922607421875	\\
0.554022707850079	-0.00152587890625	\\
0.55414479306556	-0.00152587890625	\\
0.55426687828104	-0.001953125	\\
0.554388963496521	-0.001678466796875	\\
0.554511048712001	-0.001129150390625	\\
0.554633133927481	-0.001068115234375	\\
0.554755219142962	-0.001190185546875	\\
0.554877304358442	-0.000946044921875	\\
0.554999389573923	-0.000946044921875	\\
0.555121474789403	-0.001007080078125	\\
0.555243560004883	-0.000518798828125	\\
0.555365645220364	-0.00067138671875	\\
0.555487730435844	-0.000762939453125	\\
0.555609815651325	-0.00030517578125	\\
0.555731900866805	-0.00030517578125	\\
0.555853986082285	-3.0517578125e-05	\\
0.555976071297766	0.00018310546875	\\
0.556098156513246	0	\\
0.556220241728727	0.000335693359375	\\
0.556342326944207	0.000518798828125	\\
0.556464412159687	0.000518798828125	\\
0.556586497375168	0.000885009765625	\\
0.556708582590648	0.00103759765625	\\
0.556830667806129	0.001190185546875	\\
0.556952753021609	0.001312255859375	\\
0.557074838237089	0.001434326171875	\\
0.55719692345257	0.001007080078125	\\
0.55731900866805	0.000823974609375	\\
0.557441093883531	0.000732421875	\\
0.557563179099011	0.000701904296875	\\
0.557685264314492	0.00128173828125	\\
0.557807349529972	0.001617431640625	\\
0.557929434745452	0.0015869140625	\\
0.558051519960933	0.0015869140625	\\
0.558173605176413	0.002197265625	\\
0.558295690391894	0.0025634765625	\\
0.558417775607374	0.002655029296875	\\
0.558539860822854	0.00286865234375	\\
0.558661946038335	0.0028076171875	\\
0.558784031253815	0.00274658203125	\\
0.558906116469296	0.0028076171875	\\
0.559028201684776	0.0025634765625	\\
0.559150286900256	0.002471923828125	\\
0.559272372115737	0.00213623046875	\\
0.559394457331217	0.0015869140625	\\
0.559516542546698	0.001190185546875	\\
0.559638627762178	0.00103759765625	\\
0.559760712977658	0.000885009765625	\\
0.559882798193139	0.0006103515625	\\
0.560004883408619	0.001312255859375	\\
0.5601269686241	0.001739501953125	\\
0.56024905383958	0.001312255859375	\\
0.56037113905506	0.00164794921875	\\
0.560493224270541	0.002105712890625	\\
0.560615309486021	0.00189208984375	\\
0.560737394701502	0.0020751953125	\\
0.560859479916982	0.001953125	\\
0.560981565132462	0.001556396484375	\\
0.561103650347943	0.001708984375	\\
0.561225735563423	0.001220703125	\\
0.561347820778904	0.00103759765625	\\
0.561469905994384	0.000640869140625	\\
0.561591991209864	0.0001220703125	\\
0.561714076425345	0.0003662109375	\\
0.561836161640825	0.000396728515625	\\
0.561958246856306	0.00054931640625	\\
0.562080332071786	0.00067138671875	\\
0.562202417287267	0.0001220703125	\\
0.562324502502747	0.00018310546875	\\
0.562446587718227	0.0008544921875	\\
0.562568672933708	0.001190185546875	\\
0.562690758149188	0.0010986328125	\\
0.562812843364669	0.001068115234375	\\
0.562934928580149	0.00103759765625	\\
0.563057013795629	0.00079345703125	\\
0.56317909901111	0.00079345703125	\\
0.56330118422659	0.000640869140625	\\
0.563423269442071	0.0006103515625	\\
0.563545354657551	0.000640869140625	\\
0.563667439873031	0.00048828125	\\
0.563789525088512	3.0517578125e-05	\\
0.563911610303992	-0.00030517578125	\\
0.564033695519473	-0.0003662109375	\\
0.564155780734953	-0.00042724609375	\\
0.564277865950433	6.103515625e-05	\\
0.564399951165914	0.0006103515625	\\
0.564522036381394	0.0008544921875	\\
0.564644121596875	0.00091552734375	\\
0.564766206812355	0.000701904296875	\\
0.564888292027835	0.00079345703125	\\
0.565010377243316	0.000701904296875	\\
0.565132462458796	0.000762939453125	\\
0.565254547674277	0.0008544921875	\\
0.565376632889757	0.000762939453125	\\
0.565498718105237	0.0008544921875	\\
0.565620803320718	0.001068115234375	\\
0.565742888536198	0.001220703125	\\
0.565864973751679	0.001251220703125	\\
0.565987058967159	0.00140380859375	\\
0.566109144182639	0.001800537109375	\\
0.56623122939812	0.0018310546875	\\
0.5663533146136	0.00177001953125	\\
0.566475399829081	0.001983642578125	\\
0.566597485044561	0.00201416015625	\\
0.566719570260041	0.0018310546875	\\
0.566841655475522	0.001495361328125	\\
0.566963740691002	0.001373291015625	\\
0.567085825906483	0.00140380859375	\\
0.567207911121963	0.0009765625	\\
0.567329996337444	0.00079345703125	\\
0.567452081552924	0.0009765625	\\
0.567574166768404	0.001129150390625	\\
0.567696251983885	0.00103759765625	\\
0.567818337199365	0.00079345703125	\\
0.567940422414846	0.00048828125	\\
0.568062507630326	0.00048828125	\\
0.568184592845806	0.000701904296875	\\
0.568306678061287	0.000946044921875	\\
0.568428763276767	0.000701904296875	\\
0.568550848492248	0.00048828125	\\
0.568672933707728	0.000885009765625	\\
0.568795018923208	0.0006103515625	\\
0.568917104138689	0.0001220703125	\\
0.569039189354169	-0.0001220703125	\\
0.56916127456965	-0.000152587890625	\\
0.56928335978513	-0.00018310546875	\\
0.56940544500061	-0.000732421875	\\
0.569527530216091	-0.00067138671875	\\
0.569649615431571	-0.00042724609375	\\
0.569771700647052	-0.0003662109375	\\
0.569893785862532	-0.000518798828125	\\
0.570015871078012	-0.000640869140625	\\
0.570137956293493	-0.001007080078125	\\
0.570260041508973	-0.000946044921875	\\
0.570382126724454	-0.000701904296875	\\
0.570504211939934	-0.000823974609375	\\
0.570626297155414	-0.000762939453125	\\
0.570748382370895	-0.000518798828125	\\
0.570870467586375	-0.0003662109375	\\
0.570992552801856	-0.000701904296875	\\
0.571114638017336	-0.000640869140625	\\
0.571236723232816	-0.000762939453125	\\
0.571358808448297	-0.00067138671875	\\
0.571480893663777	6.103515625e-05	\\
0.571602978879258	-0.000213623046875	\\
0.571725064094738	-0.000244140625	\\
0.571847149310219	0.00030517578125	\\
0.571969234525699	0.00054931640625	\\
0.572091319741179	0.00067138671875	\\
0.57221340495666	0.000885009765625	\\
0.57233549017214	0.000701904296875	\\
0.572457575387621	0.00042724609375	\\
0.572579660603101	0.000946044921875	\\
0.572701745818581	0.000885009765625	\\
0.572823831034062	0.00048828125	\\
0.572945916249542	0.0006103515625	\\
0.573068001465023	0.0006103515625	\\
0.573190086680503	0.000579833984375	\\
0.573312171895983	0.00079345703125	\\
0.573434257111464	0.001068115234375	\\
0.573556342326944	0.001129150390625	\\
0.573678427542425	0.001312255859375	\\
0.573800512757905	0.00115966796875	\\
0.573922597973385	0.001129150390625	\\
0.574044683188866	0.00140380859375	\\
0.574166768404346	0.00128173828125	\\
0.574288853619827	0.001495361328125	\\
0.574410938835307	0.001495361328125	\\
0.574533024050787	0.00146484375	\\
0.574655109266268	0.001373291015625	\\
0.574777194481748	0.00140380859375	\\
0.574899279697229	0.00177001953125	\\
0.575021364912709	0.00177001953125	\\
0.575143450128189	0.001556396484375	\\
0.57526553534367	0.0013427734375	\\
0.57538762055915	0.001068115234375	\\
0.575509705774631	0.001312255859375	\\
0.575631790990111	0.0010986328125	\\
0.575753876205591	0.00115966796875	\\
0.575875961421072	0.001678466796875	\\
0.575998046636552	0.001861572265625	\\
0.576120131852033	0.002044677734375	\\
0.576242217067513	0.002044677734375	\\
0.576364302282994	0.002288818359375	\\
0.576486387498474	0.00274658203125	\\
0.576608472713954	0.002349853515625	\\
0.576730557929435	0.00189208984375	\\
0.576852643144915	0.001617431640625	\\
0.576974728360396	0.001190185546875	\\
0.577096813575876	0.001220703125	\\
0.577218898791356	0.001068115234375	\\
0.577340984006837	0.00091552734375	\\
0.577463069222317	0.001007080078125	\\
0.577585154437798	0.000885009765625	\\
0.577707239653278	0.000762939453125	\\
0.577829324868758	0.00079345703125	\\
0.577951410084239	0.0013427734375	\\
0.578073495299719	0.0015869140625	\\
0.5781955805152	0.001495361328125	\\
0.57831766573068	0.0015869140625	\\
0.57843975094616	0.001708984375	\\
0.578561836161641	0.00140380859375	\\
0.578683921377121	0.001129150390625	\\
0.578806006592602	0.001007080078125	\\
0.578928091808082	0.000732421875	\\
0.579050177023562	0.00079345703125	\\
0.579172262239043	0.000823974609375	\\
0.579294347454523	0.000701904296875	\\
0.579416432670004	0.000762939453125	\\
0.579538517885484	0.000762939453125	\\
0.579660603100964	0.00048828125	\\
0.579782688316445	0.000732421875	\\
0.579904773531925	0.000579833984375	\\
0.580026858747406	0.00067138671875	\\
0.580148943962886	0.000701904296875	\\
0.580271029178366	0.00042724609375	\\
0.580393114393847	0.00030517578125	\\
0.580515199609327	0.000762939453125	\\
0.580637284824808	0.00054931640625	\\
0.580759370040288	0.00042724609375	\\
0.580881455255769	0.000457763671875	\\
0.581003540471249	3.0517578125e-05	\\
0.581125625686729	0	\\
0.58124771090221	0.000396728515625	\\
0.58136979611769	0.000640869140625	\\
0.581491881333171	0.000885009765625	\\
0.581613966548651	0.001312255859375	\\
0.581736051764131	0.00128173828125	\\
0.581858136979612	0.001678466796875	\\
0.581980222195092	0.0020751953125	\\
0.582102307410573	0.0020751953125	\\
0.582224392626053	0.00213623046875	\\
0.582346477841533	0.002288818359375	\\
0.582468563057014	0.001739501953125	\\
0.582590648272494	0.001617431640625	\\
0.582712733487975	0.00177001953125	\\
0.582834818703455	0.0013427734375	\\
0.582956903918935	0.0009765625	\\
0.583078989134416	0.000823974609375	\\
0.583201074349896	0.000885009765625	\\
0.583323159565377	0.000640869140625	\\
0.583445244780857	0.000732421875	\\
0.583567329996337	0.00054931640625	\\
0.583689415211818	0.00048828125	\\
0.583811500427298	0.000946044921875	\\
0.583933585642779	0.001190185546875	\\
0.584055670858259	0.0013427734375	\\
0.584177756073739	0.001220703125	\\
0.58429984128922	0.001068115234375	\\
0.5844219265047	0.000732421875	\\
0.584544011720181	0.00054931640625	\\
0.584666096935661	0.000396728515625	\\
0.584788182151141	0.0001220703125	\\
0.584910267366622	0	\\
0.585032352582102	-0.0003662109375	\\
0.585154437797583	-0.00067138671875	\\
0.585276523013063	-0.00079345703125	\\
0.585398608228544	-0.0009765625	\\
0.585520693444024	-0.000823974609375	\\
0.585642778659504	-0.000518798828125	\\
0.585764863874985	-0.000579833984375	\\
0.585886949090465	-0.000335693359375	\\
0.586009034305946	-0.000518798828125	\\
0.586131119521426	-0.000823974609375	\\
0.586253204736906	-0.00091552734375	\\
0.586375289952387	-0.000946044921875	\\
0.586497375167867	-0.0010986328125	\\
0.586619460383348	-0.00152587890625	\\
0.586741545598828	-0.001739501953125	\\
0.586863630814308	-0.00140380859375	\\
0.586985716029789	-0.0009765625	\\
0.587107801245269	-0.000946044921875	\\
0.58722988646075	-0.001434326171875	\\
0.58735197167623	-0.00146484375	\\
0.58747405689171	-0.0009765625	\\
0.587596142107191	-0.00115966796875	\\
0.587718227322671	-0.001220703125	\\
0.587840312538152	-0.001220703125	\\
0.587962397753632	-0.001190185546875	\\
0.588084482969112	-0.001068115234375	\\
0.588206568184593	-0.001220703125	\\
0.588328653400073	-0.001556396484375	\\
0.588450738615554	-0.0015869140625	\\
0.588572823831034	-0.001861572265625	\\
0.588694909046514	-0.001953125	\\
0.588816994261995	-0.00189208984375	\\
0.588939079477475	-0.002105712890625	\\
0.589061164692956	-0.002532958984375	\\
0.589183249908436	-0.002288818359375	\\
0.589305335123916	-0.0018310546875	\\
0.589427420339397	-0.001800537109375	\\
0.589549505554877	-0.00152587890625	\\
0.589671590770358	-0.001312255859375	\\
0.589793675985838	-0.001373291015625	\\
0.589915761201319	-0.00140380859375	\\
0.590037846416799	-0.00115966796875	\\
0.590159931632279	-0.00140380859375	\\
0.59028201684776	-0.001556396484375	\\
0.59040410206324	-0.00146484375	\\
0.590526187278721	-0.00164794921875	\\
0.590648272494201	-0.00177001953125	\\
0.590770357709681	-0.0015869140625	\\
0.590892442925162	-0.001708984375	\\
0.591014528140642	-0.00152587890625	\\
0.591136613356123	-0.001312255859375	\\
0.591258698571603	-0.0010986328125	\\
0.591380783787083	-0.00042724609375	\\
0.591502869002564	6.103515625e-05	\\
0.591624954218044	-3.0517578125e-05	\\
0.591747039433525	-0.0001220703125	\\
0.591869124649005	-0.000152587890625	\\
0.591991209864485	-3.0517578125e-05	\\
0.592113295079966	-0.00018310546875	\\
0.592235380295446	-0.000701904296875	\\
0.592357465510927	-0.000946044921875	\\
0.592479550726407	-0.00103759765625	\\
0.592601635941887	-0.0013427734375	\\
0.592723721157368	-0.00177001953125	\\
0.592845806372848	-0.001617431640625	\\
0.592967891588329	-0.00091552734375	\\
0.593089976803809	-0.0008544921875	\\
0.593212062019289	-0.001220703125	\\
0.59333414723477	-0.001312255859375	\\
0.59345623245025	-0.00067138671875	\\
0.593578317665731	0.00018310546875	\\
0.593700402881211	0.0001220703125	\\
0.593822488096691	-0.00018310546875	\\
0.593944573312172	-0.000244140625	\\
0.594066658527652	-0.00018310546875	\\
0.594188743743133	-0.00042724609375	\\
0.594310828958613	-0.000579833984375	\\
0.594432914174094	-0.000823974609375	\\
0.594554999389574	-0.00115966796875	\\
0.594677084605054	-0.00115966796875	\\
0.594799169820535	-0.001251220703125	\\
0.594921255036015	-0.001373291015625	\\
0.595043340251496	-0.00115966796875	\\
0.595165425466976	-0.001312255859375	\\
0.595287510682456	-0.001220703125	\\
0.595409595897937	-0.00103759765625	\\
0.595531681113417	-0.001007080078125	\\
0.595653766328898	-0.000885009765625	\\
0.595775851544378	-0.000823974609375	\\
0.595897936759858	-0.000762939453125	\\
0.596020021975339	-0.00042724609375	\\
0.596142107190819	-0.00030517578125	\\
0.5962641924063	-0.000579833984375	\\
0.59638627762178	-0.0008544921875	\\
0.59650836283726	-0.000732421875	\\
0.596630448052741	-0.000579833984375	\\
0.596752533268221	-0.00018310546875	\\
0.596874618483702	-0.0001220703125	\\
0.596996703699182	-0.00030517578125	\\
0.597118788914662	9.1552734375e-05	\\
0.597240874130143	0.000244140625	\\
0.597362959345623	0.000335693359375	\\
0.597485044561104	0.00030517578125	\\
0.597607129776584	0.00030517578125	\\
0.597729214992064	0.000640869140625	\\
0.597851300207545	0.000885009765625	\\
0.597973385423025	0.000579833984375	\\
0.598095470638506	0.0003662109375	\\
0.598217555853986	0.0001220703125	\\
0.598339641069466	-6.103515625e-05	\\
0.598461726284947	3.0517578125e-05	\\
0.598583811500427	-0.000152587890625	\\
0.598705896715908	-9.1552734375e-05	\\
0.598827981931388	0.00030517578125	\\
0.598950067146868	0.000213623046875	\\
0.599072152362349	0.000579833984375	\\
0.599194237577829	0.000396728515625	\\
0.59931632279331	0.000396728515625	\\
0.59943840800879	0.00067138671875	\\
0.599560493224271	0.000152587890625	\\
0.599682578439751	0.000213623046875	\\
0.599804663655231	0.000396728515625	\\
0.599926748870712	0.00018310546875	\\
0.600048834086192	6.103515625e-05	\\
0.600170919301673	-6.103515625e-05	\\
0.600293004517153	-0.00042724609375	\\
0.600415089732633	-0.00091552734375	\\
0.600537174948114	-0.00103759765625	\\
0.600659260163594	-0.001190185546875	\\
0.600781345379075	-0.00103759765625	\\
0.600903430594555	-0.0008544921875	\\
0.601025515810035	-0.001068115234375	\\
0.601147601025516	-0.00048828125	\\
0.601269686240996	-0.00018310546875	\\
0.601391771456477	-0.000274658203125	\\
0.601513856671957	-0.0001220703125	\\
0.601635941887437	0	\\
0.601758027102918	0.000335693359375	\\
0.601880112318398	-3.0517578125e-05	\\
0.602002197533879	-0.0003662109375	\\
0.602124282749359	-0.00048828125	\\
0.602246367964839	-0.000640869140625	\\
0.60236845318032	-0.0010986328125	\\
0.6024905383958	-0.00115966796875	\\
0.602612623611281	-0.000885009765625	\\
0.602734708826761	-0.000885009765625	\\
0.602856794042241	-0.001190185546875	\\
0.602978879257722	-0.001495361328125	\\
0.603100964473202	-0.001617431640625	\\
0.603223049688683	-0.00164794921875	\\
0.603345134904163	-0.00103759765625	\\
0.603467220119643	-0.001068115234375	\\
0.603589305335124	-0.00152587890625	\\
0.603711390550604	-0.001434326171875	\\
0.603833475766085	-0.001434326171875	\\
0.603955560981565	-0.001800537109375	\\
0.604077646197046	-0.00189208984375	\\
0.604199731412526	-0.00164794921875	\\
0.604321816628006	-0.001708984375	\\
0.604443901843487	-0.0015869140625	\\
0.604565987058967	-0.0015869140625	\\
0.604688072274448	-0.001983642578125	\\
0.604810157489928	-0.001983642578125	\\
0.604932242705408	-0.002166748046875	\\
0.605054327920889	-0.002197265625	\\
0.605176413136369	-0.001708984375	\\
0.60529849835185	-0.001373291015625	\\
0.60542058356733	-0.00128173828125	\\
0.60554266878281	-0.001495361328125	\\
0.605664753998291	-0.0013427734375	\\
0.605786839213771	-0.000885009765625	\\
0.605908924429252	-0.000823974609375	\\
0.606031009644732	-0.00128173828125	\\
0.606153094860212	-0.00140380859375	\\
0.606275180075693	-0.001190185546875	\\
0.606397265291173	-0.0010986328125	\\
0.606519350506654	-0.000946044921875	\\
0.606641435722134	-0.00128173828125	\\
0.606763520937614	-0.00140380859375	\\
0.606885606153095	-0.001739501953125	\\
0.607007691368575	-0.001739501953125	\\
0.607129776584056	-0.0015869140625	\\
0.607251861799536	-0.0015869140625	\\
0.607373947015016	-0.001708984375	\\
0.607496032230497	-0.00201416015625	\\
0.607618117445977	-0.0018310546875	\\
0.607740202661458	-0.001708984375	\\
0.607862287876938	-0.00146484375	\\
0.607984373092418	-0.001708984375	\\
0.608106458307899	-0.002410888671875	\\
0.608228543523379	-0.002349853515625	\\
0.60835062873886	-0.002777099609375	\\
0.60847271395434	-0.00299072265625	\\
0.608594799169821	-0.003082275390625	\\
0.608716884385301	-0.00335693359375	\\
0.608838969600781	-0.0032958984375	\\
0.608961054816262	-0.00335693359375	\\
0.609083140031742	-0.003387451171875	\\
0.609205225247223	-0.003448486328125	\\
0.609327310462703	-0.003143310546875	\\
0.609449395678183	-0.002777099609375	\\
0.609571480893664	-0.002471923828125	\\
0.609693566109144	-0.003021240234375	\\
0.609815651324625	-0.003387451171875	\\
0.609937736540105	-0.003265380859375	\\
0.610059821755585	-0.003509521484375	\\
0.610181906971066	-0.003662109375	\\
0.610303992186546	-0.004364013671875	\\
0.610426077402027	-0.0047607421875	\\
0.610548162617507	-0.0042724609375	\\
0.610670247832987	-0.004119873046875	\\
0.610792333048468	-0.00408935546875	\\
0.610914418263948	-0.00347900390625	\\
0.611036503479429	-0.003448486328125	\\
0.611158588694909	-0.003265380859375	\\
0.611280673910389	-0.002655029296875	\\
0.61140275912587	-0.001953125	\\
0.61152484434135	-0.001922607421875	\\
0.611646929556831	-0.001861572265625	\\
0.611769014772311	-0.00201416015625	\\
0.611891099987791	-0.002471923828125	\\
0.612013185203272	-0.002593994140625	\\
0.612135270418752	-0.002655029296875	\\
0.612257355634233	-0.00262451171875	\\
0.612379440849713	-0.00299072265625	\\
0.612501526065193	-0.00274658203125	\\
0.612623611280674	-0.002716064453125	\\
0.612745696496154	-0.002593994140625	\\
0.612867781711635	-0.00286865234375	\\
0.612989866927115	-0.00274658203125	\\
0.613111952142596	-0.0023193359375	\\
0.613234037358076	-0.002105712890625	\\
0.613356122573556	-0.002044677734375	\\
0.613478207789037	-0.002105712890625	\\
0.613600293004517	-0.001861572265625	\\
0.613722378219998	-0.001922607421875	\\
0.613844463435478	-0.001800537109375	\\
0.613966548650958	-0.0013427734375	\\
0.614088633866439	-0.001861572265625	\\
0.614210719081919	-0.00177001953125	\\
0.6143328042974	-0.00146484375	\\
0.61445488951288	-0.001190185546875	\\
0.61457697472836	-0.001129150390625	\\
0.614699059943841	-0.001220703125	\\
0.614821145159321	-0.00054931640625	\\
0.614943230374802	-0.000213623046875	\\
0.615065315590282	3.0517578125e-05	\\
0.615187400805762	0.000152587890625	\\
0.615309486021243	-3.0517578125e-05	\\
0.615431571236723	9.1552734375e-05	\\
0.615553656452204	-0.00042724609375	\\
0.615675741667684	-0.00079345703125	\\
0.615797826883164	-0.000762939453125	\\
0.615919912098645	-0.001007080078125	\\
0.616041997314125	-0.00103759765625	\\
0.616164082529606	-0.00115966796875	\\
0.616286167745086	-0.000823974609375	\\
0.616408252960566	-0.000518798828125	\\
0.616530338176047	-0.00091552734375	\\
0.616652423391527	-0.000732421875	\\
0.616774508607008	-0.000244140625	\\
0.616896593822488	0	\\
0.617018679037968	-0.000274658203125	\\
0.617140764253449	-0.000274658203125	\\
0.617262849468929	-0.00018310546875	\\
0.61738493468441	-0.000701904296875	\\
0.61750701989989	-0.000885009765625	\\
0.617629105115371	-0.00048828125	\\
0.617751190330851	-0.00067138671875	\\
0.617873275546331	-0.0010986328125	\\
0.617995360761812	-0.001312255859375	\\
0.618117445977292	-0.00140380859375	\\
0.618239531192773	-0.001007080078125	\\
0.618361616408253	-0.00115966796875	\\
0.618483701623733	-0.0015869140625	\\
0.618605786839214	-0.001373291015625	\\
0.618727872054694	-0.00079345703125	\\
0.618849957270175	-0.000640869140625	\\
0.618972042485655	-0.000579833984375	\\
0.619094127701135	-0.000457763671875	\\
0.619216212916616	-0.000274658203125	\\
0.619338298132096	-0.00030517578125	\\
0.619460383347577	-0.00030517578125	\\
0.619582468563057	-0.00054931640625	\\
0.619704553778537	-0.000732421875	\\
0.619826638994018	-0.001190185546875	\\
0.619948724209498	-0.001708984375	\\
0.620070809424979	-0.001495361328125	\\
0.620192894640459	-0.0010986328125	\\
0.620314979855939	-0.001190185546875	\\
0.62043706507142	-0.000732421875	\\
0.6205591502869	-0.00048828125	\\
0.620681235502381	-0.000213623046875	\\
0.620803320717861	-6.103515625e-05	\\
0.620925405933341	-0.0001220703125	\\
0.621047491148822	0.000244140625	\\
0.621169576364302	0.00030517578125	\\
0.621291661579783	0.00042724609375	\\
0.621413746795263	0.000762939453125	\\
0.621535832010743	0.000579833984375	\\
0.621657917226224	0.000274658203125	\\
0.621780002441704	0.00018310546875	\\
0.621902087657185	0.00054931640625	\\
0.622024172872665	0.0003662109375	\\
0.622146258088146	0.000396728515625	\\
0.622268343303626	0.0003662109375	\\
0.622390428519106	0.000701904296875	\\
0.622512513734587	0.000732421875	\\
0.622634598950067	0.00079345703125	\\
0.622756684165548	0.001129150390625	\\
0.622878769381028	0.001434326171875	\\
0.623000854596508	0.00146484375	\\
0.623122939811989	0.001373291015625	\\
0.623245025027469	0.00140380859375	\\
0.62336711024295	0.001190185546875	\\
0.62348919545843	0.001220703125	\\
0.62361128067391	0.001312255859375	\\
0.623733365889391	0.001190185546875	\\
0.623855451104871	0.001007080078125	\\
0.623977536320352	0.000732421875	\\
0.624099621535832	0.00079345703125	\\
0.624221706751312	0.000640869140625	\\
0.624343791966793	0.00018310546875	\\
0.624465877182273	0.000335693359375	\\
0.624587962397754	0.00091552734375	\\
0.624710047613234	0.001068115234375	\\
0.624832132828714	0.000885009765625	\\
0.624954218044195	0.00079345703125	\\
0.625076303259675	0.0010986328125	\\
0.625198388475156	0.0010986328125	\\
0.625320473690636	0.000457763671875	\\
0.625442558906116	0.000396728515625	\\
0.625564644121597	0.000274658203125	\\
0.625686729337077	3.0517578125e-05	\\
0.625808814552558	0	\\
0.625930899768038	-0.000518798828125	\\
0.626052984983518	-0.000396728515625	\\
0.626175070198999	-0.000335693359375	\\
0.626297155414479	-0.00048828125	\\
0.62641924062996	-0.0003662109375	\\
0.62654132584544	-0.0003662109375	\\
0.62666341106092	-0.000457763671875	\\
0.626785496276401	0	\\
0.626907581491881	0	\\
0.627029666707362	-0.0001220703125	\\
0.627151751922842	0.0001220703125	\\
0.627273837138323	-0.000152587890625	\\
0.627395922353803	-0.0003662109375	\\
0.627518007569283	-0.000640869140625	\\
0.627640092784764	-0.00067138671875	\\
0.627762178000244	-0.00042724609375	\\
0.627884263215725	-0.0008544921875	\\
0.628006348431205	-0.000946044921875	\\
0.628128433646685	-0.000946044921875	\\
0.628250518862166	-0.0009765625	\\
0.628372604077646	-0.00079345703125	\\
0.628494689293127	-0.00048828125	\\
0.628616774508607	-0.000244140625	\\
0.628738859724087	6.103515625e-05	\\
0.628860944939568	0.00018310546875	\\
0.628983030155048	0.000579833984375	\\
0.629105115370529	0.00067138671875	\\
0.629227200586009	0.000152587890625	\\
0.629349285801489	0.000152587890625	\\
0.62947137101697	0.0006103515625	\\
0.62959345623245	0.000457763671875	\\
0.629715541447931	9.1552734375e-05	\\
0.629837626663411	0.0001220703125	\\
0.629959711878891	-6.103515625e-05	\\
0.630081797094372	-0.000274658203125	\\
0.630203882309852	9.1552734375e-05	\\
0.630325967525333	0.000579833984375	\\
0.630448052740813	0.00079345703125	\\
0.630570137956293	0.00115966796875	\\
0.630692223171774	0.0008544921875	\\
0.630814308387254	0.000885009765625	\\
0.630936393602735	0.001251220703125	\\
0.631058478818215	0.0015869140625	\\
0.631180564033695	0.0013427734375	\\
0.631302649249176	0.00115966796875	\\
0.631424734464656	0.0015869140625	\\
0.631546819680137	0.001190185546875	\\
0.631668904895617	0.00103759765625	\\
0.631790990111098	0.0009765625	\\
0.631913075326578	0.000823974609375	\\
0.632035160542058	0.0009765625	\\
0.632157245757539	0.000518798828125	\\
0.632279330973019	0.000213623046875	\\
0.6324014161885	0.000518798828125	\\
0.63252350140398	0.000946044921875	\\
0.63264558661946	0.001556396484375	\\
0.632767671834941	0.001708984375	\\
0.632889757050421	0.00164794921875	\\
0.633011842265902	0.001495361328125	\\
0.633133927481382	0.001556396484375	\\
0.633256012696862	0.00146484375	\\
0.633378097912343	0.0015869140625	\\
0.633500183127823	0.00146484375	\\
0.633622268343304	0.001007080078125	\\
0.633744353558784	0.00067138671875	\\
0.633866438774264	0.000762939453125	\\
0.633988523989745	0.000823974609375	\\
0.634110609205225	0.00042724609375	\\
0.634232694420706	0.0003662109375	\\
0.634354779636186	0.000396728515625	\\
0.634476864851666	0.0006103515625	\\
0.634598950067147	0.000732421875	\\
0.634721035282627	0.00091552734375	\\
0.634843120498108	0.00140380859375	\\
0.634965205713588	0.001312255859375	\\
0.635087290929068	0.001373291015625	\\
0.635209376144549	0.001495361328125	\\
0.635331461360029	0.001129150390625	\\
0.63545354657551	0.00079345703125	\\
0.63557563179099	0.0008544921875	\\
0.63569771700647	0.0010986328125	\\
0.635819802221951	0.001068115234375	\\
0.635941887437431	0.001220703125	\\
0.636063972652912	0.00140380859375	\\
0.636186057868392	0.00128173828125	\\
0.636308143083873	0.001678466796875	\\
0.636430228299353	0.001800537109375	\\
0.636552313514833	0.001861572265625	\\
0.636674398730314	0.00189208984375	\\
0.636796483945794	0.001678466796875	\\
0.636918569161275	0.001922607421875	\\
0.637040654376755	0.00225830078125	\\
0.637162739592235	0.001861572265625	\\
0.637284824807716	0.00189208984375	\\
0.637406910023196	0.0020751953125	\\
0.637528995238677	0.002044677734375	\\
0.637651080454157	0.002288818359375	\\
0.637773165669637	0.002197265625	\\
0.637895250885118	0.00213623046875	\\
0.638017336100598	0.002197265625	\\
0.638139421316079	0.00250244140625	\\
0.638261506531559	0.0025634765625	\\
0.638383591747039	0.00262451171875	\\
0.63850567696252	0.0029296875	\\
0.638627762178	0.003265380859375	\\
0.638749847393481	0.003082275390625	\\
0.638871932608961	0.00341796875	\\
0.638994017824441	0.0037841796875	\\
0.639116103039922	0.0035400390625	\\
0.639238188255402	0.00372314453125	\\
0.639360273470883	0.003448486328125	\\
0.639482358686363	0.003143310546875	\\
0.639604443901843	0.002960205078125	\\
0.639726529117324	0.002685546875	\\
0.639848614332804	0.002777099609375	\\
0.639970699548285	0.0029296875	\\
0.640092784763765	0.003143310546875	\\
0.640214869979245	0.003570556640625	\\
0.640336955194726	0.003509521484375	\\
0.640459040410206	0.00347900390625	\\
0.640581125625687	0.003997802734375	\\
0.640703210841167	0.003875732421875	\\
0.640825296056648	0.00360107421875	\\
0.640947381272128	0.00372314453125	\\
0.641069466487608	0.003692626953125	\\
0.641191551703089	0.00311279296875	\\
0.641313636918569	0.00262451171875	\\
0.64143572213405	0.002349853515625	\\
0.64155780734953	0.002410888671875	\\
0.64167989256501	0.002471923828125	\\
0.641801977780491	0.002716064453125	\\
0.641924062995971	0.002960205078125	\\
0.642046148211452	0.0029296875	\\
0.642168233426932	0.00250244140625	\\
0.642290318642412	0.002593994140625	\\
0.642412403857893	0.003082275390625	\\
0.642534489073373	0.002960205078125	\\
0.642656574288854	0.002593994140625	\\
0.642778659504334	0.00274658203125	\\
0.642900744719814	0.0025634765625	\\
0.643022829935295	0.001983642578125	\\
0.643144915150775	0.00201416015625	\\
0.643267000366256	0.002227783203125	\\
0.643389085581736	0.00244140625	\\
0.643511170797216	0.002532958984375	\\
0.643633256012697	0.002410888671875	\\
0.643755341228177	0.0023193359375	\\
0.643877426443658	0.00244140625	\\
0.643999511659138	0.002288818359375	\\
0.644121596874618	0.001953125	\\
0.644243682090099	0.002349853515625	\\
0.644365767305579	0.0029296875	\\
0.64448785252106	0.002777099609375	\\
0.64460993773654	0.002960205078125	\\
0.64473202295202	0.002899169921875	\\
0.644854108167501	0.002685546875	\\
0.644976193382981	0.002685546875	\\
0.645098278598462	0.00274658203125	\\
0.645220363813942	0.002685546875	\\
0.645342449029423	0.00250244140625	\\
0.645464534244903	0.002288818359375	\\
0.645586619460383	0.00238037109375	\\
0.645708704675864	0.0028076171875	\\
0.645830789891344	0.002960205078125	\\
0.645952875106825	0.00299072265625	\\
0.646074960322305	0.003387451171875	\\
0.646197045537785	0.0035400390625	\\
0.646319130753266	0.004150390625	\\
0.646441215968746	0.004486083984375	\\
0.646563301184227	0.00439453125	\\
0.646685386399707	0.004730224609375	\\
0.646807471615187	0.004974365234375	\\
0.646929556830668	0.0045166015625	\\
0.647051642046148	0.004364013671875	\\
0.647173727261629	0.0042724609375	\\
0.647295812477109	0.0040283203125	\\
0.647417897692589	0.00408935546875	\\
0.64753998290807	0.004180908203125	\\
0.64766206812355	0.0037841796875	\\
0.647784153339031	0.0037841796875	\\
0.647906238554511	0.00390625	\\
0.648028323769991	0.00396728515625	\\
0.648150408985472	0.003814697265625	\\
0.648272494200952	0.003509521484375	\\
0.648394579416433	0.003448486328125	\\
0.648516664631913	0.003387451171875	\\
0.648638749847393	0.00347900390625	\\
0.648760835062874	0.003570556640625	\\
0.648882920278354	0.003326416015625	\\
0.649005005493835	0.00347900390625	\\
0.649127090709315	0.0035400390625	\\
0.649249175924795	0.002838134765625	\\
0.649371261140276	0.002471923828125	\\
0.649493346355756	0.002685546875	\\
0.649615431571237	0.002777099609375	\\
0.649737516786717	0.00274658203125	\\
0.649859602002198	0.003173828125	\\
0.649981687217678	0.0035400390625	\\
0.650103772433158	0.003631591796875	\\
0.650225857648639	0.003692626953125	\\
0.650347942864119	0.003753662109375	\\
0.6504700280796	0.003662109375	\\
0.65059211329508	0.003570556640625	\\
0.65071419851056	0.003387451171875	\\
0.650836283726041	0.00311279296875	\\
0.650958368941521	0.0028076171875	\\
0.651080454157002	0.00244140625	\\
0.651202539372482	0.00225830078125	\\
0.651324624587962	0.002349853515625	\\
0.651446709803443	0.002197265625	\\
0.651568795018923	0.001983642578125	\\
0.651690880234404	0.002166748046875	\\
0.651812965449884	0.0020751953125	\\
0.651935050665364	0.0015869140625	\\
0.652057135880845	0.00189208984375	\\
0.652179221096325	0.002471923828125	\\
0.652301306311806	0.002471923828125	\\
0.652423391527286	0.002593994140625	\\
0.652545476742766	0.002685546875	\\
0.652667561958247	0.002349853515625	\\
0.652789647173727	0.00225830078125	\\
0.652911732389208	0.00250244140625	\\
0.653033817604688	0.00238037109375	\\
0.653155902820168	0.00250244140625	\\
0.653277988035649	0.0025634765625	\\
0.653400073251129	0.00250244140625	\\
0.65352215846661	0.002410888671875	\\
0.65364424368209	0.0020751953125	\\
0.65376632889757	0.00189208984375	\\
0.653888414113051	0.0020751953125	\\
0.654010499328531	0.002288818359375	\\
0.654132584544012	0.00274658203125	\\
0.654254669759492	0.00311279296875	\\
0.654376754974973	0.003082275390625	\\
0.654498840190453	0.002899169921875	\\
0.654620925405933	0.002838134765625	\\
0.654743010621414	0.0030517578125	\\
0.654865095836894	0.0029296875	\\
0.654987181052375	0.00250244140625	\\
0.655109266267855	0.002716064453125	\\
0.655231351483335	0.002410888671875	\\
0.655353436698816	0.002471923828125	\\
0.655475521914296	0.00244140625	\\
0.655597607129777	0.002197265625	\\
0.655719692345257	0.0023193359375	\\
0.655841777560737	0.002593994140625	\\
0.655963862776218	0.003021240234375	\\
0.656085947991698	0.003204345703125	\\
0.656208033207179	0.00323486328125	\\
0.656330118422659	0.00311279296875	\\
0.656452203638139	0.002655029296875	\\
0.65657428885362	0.002960205078125	\\
0.6566963740691	0.0029296875	\\
0.656818459284581	0.002655029296875	\\
0.656940544500061	0.002777099609375	\\
0.657062629715541	0.00213623046875	\\
0.657184714931022	0.001800537109375	\\
0.657306800146502	0.002044677734375	\\
0.657428885361983	0.0015869140625	\\
0.657550970577463	0.001861572265625	\\
0.657673055792943	0.002349853515625	\\
0.657795141008424	0.002197265625	\\
0.657917226223904	0.00225830078125	\\
0.658039311439385	0.002288818359375	\\
0.658161396654865	0.00213623046875	\\
0.658283481870345	0.001922607421875	\\
0.658405567085826	0.001708984375	\\
0.658527652301306	0.001922607421875	\\
0.658649737516787	0.0018310546875	\\
0.658771822732267	0.001434326171875	\\
0.658893907947747	0.00115966796875	\\
0.659015993163228	0.000762939453125	\\
0.659138078378708	0.00042724609375	\\
0.659260163594189	0	\\
0.659382248809669	0	\\
0.65950433402515	0.0001220703125	\\
0.65962641924063	0	\\
0.65974850445611	0.00030517578125	\\
0.659870589671591	0.000762939453125	\\
0.659992674887071	0.00103759765625	\\
0.660114760102552	0.000762939453125	\\
0.660236845318032	0.00079345703125	\\
0.660358930533512	0.001190185546875	\\
0.660481015748993	0.000518798828125	\\
0.660603100964473	0.0006103515625	\\
0.660725186179954	0.0008544921875	\\
0.660847271395434	0.000701904296875	\\
0.660969356610914	0.000885009765625	\\
0.661091441826395	0.000762939453125	\\
0.661213527041875	0.000732421875	\\
0.661335612257356	0.000762939453125	\\
0.661457697472836	0.000579833984375	\\
0.661579782688316	0.000701904296875	\\
0.661701867903797	0.000640869140625	\\
0.661823953119277	0.000701904296875	\\
0.661946038334758	0.001251220703125	\\
0.662068123550238	0.001373291015625	\\
0.662190208765718	0.001373291015625	\\
0.662312293981199	0.001617431640625	\\
0.662434379196679	0.00164794921875	\\
0.66255646441216	0.001708984375	\\
0.66267854962764	0.00146484375	\\
0.66280063484312	0.00140380859375	\\
0.662922720058601	0.001678466796875	\\
0.663044805274081	0.001434326171875	\\
0.663166890489562	0.001739501953125	\\
0.663288975705042	0.001800537109375	\\
0.663411060920522	0.00103759765625	\\
0.663533146136003	0.00103759765625	\\
0.663655231351483	0.001434326171875	\\
0.663777316566964	0.001739501953125	\\
0.663899401782444	0.00201416015625	\\
0.664021486997925	0.001953125	\\
0.664143572213405	0.00213623046875	\\
0.664265657428885	0.001983642578125	\\
0.664387742644366	0.001983642578125	\\
0.664509827859846	0.0018310546875	\\
0.664631913075327	0.00164794921875	\\
0.664753998290807	0.0015869140625	\\
0.664876083506287	0.00128173828125	\\
0.664998168721768	0.00152587890625	\\
0.665120253937248	0.001678466796875	\\
0.665242339152729	0.001739501953125	\\
0.665364424368209	0.001495361328125	\\
0.665486509583689	0.001251220703125	\\
0.66560859479917	0.001708984375	\\
0.66573068001465	0.00177001953125	\\
0.665852765230131	0.0013427734375	\\
0.665974850445611	0.001678466796875	\\
0.666096935661091	0.00177001953125	\\
0.666219020876572	0.001556396484375	\\
0.666341106092052	0.0018310546875	\\
0.666463191307533	0.001739501953125	\\
0.666585276523013	0.00152587890625	\\
0.666707361738493	0.001800537109375	\\
0.666829446953974	0.00146484375	\\
0.666951532169454	0.001434326171875	\\
0.667073617384935	0.00128173828125	\\
0.667195702600415	0.00091552734375	\\
0.667317787815895	0.001007080078125	\\
0.667439873031376	0.000396728515625	\\
0.667561958246856	0.000518798828125	\\
0.667684043462337	0.000640869140625	\\
0.667806128677817	0.0003662109375	\\
0.667928213893297	0.00054931640625	\\
0.668050299108778	0.000640869140625	\\
0.668172384324258	0.00054931640625	\\
0.668294469539739	0.000732421875	\\
0.668416554755219	0.000701904296875	\\
0.6685386399707	0.000152587890625	\\
0.66866072518618	0.0001220703125	\\
0.66878281040166	3.0517578125e-05	\\
0.668904895617141	-0.000274658203125	\\
0.669026980832621	-0.00048828125	\\
0.669149066048102	0	\\
0.669271151263582	3.0517578125e-05	\\
0.669393236479062	-0.00030517578125	\\
0.669515321694543	0.00042724609375	\\
0.669637406910023	0.001190185546875	\\
0.669759492125504	0.001373291015625	\\
0.669881577340984	0.00146484375	\\
0.670003662556464	0.001617431640625	\\
0.670125747771945	0.0020751953125	\\
0.670247832987425	0.00244140625	\\
0.670369918202906	0.002410888671875	\\
0.670492003418386	0.00189208984375	\\
0.670614088633866	0.00164794921875	\\
0.670736173849347	0.001617431640625	\\
0.670858259064827	0.001129150390625	\\
0.670980344280308	0.0008544921875	\\
0.671102429495788	0.00091552734375	\\
0.671224514711268	0.001129150390625	\\
0.671346599926749	0.00140380859375	\\
0.671468685142229	0.0010986328125	\\
0.67159077035771	0.001068115234375	\\
0.67171285557319	0.001922607421875	\\
0.67183494078867	0.002166748046875	\\
0.671957026004151	0.00189208984375	\\
0.672079111219631	0.001678466796875	\\
0.672201196435112	0.001983642578125	\\
0.672323281650592	0.001983642578125	\\
0.672445366866072	0.001861572265625	\\
0.672567452081553	0.001739501953125	\\
0.672689537297033	0.00128173828125	\\
0.672811622512514	0.001190185546875	\\
0.672933707727994	0.00146484375	\\
0.673055792943475	0.000946044921875	\\
0.673177878158955	0.0006103515625	\\
0.673299963374435	0.000579833984375	\\
0.673422048589916	0.000885009765625	\\
0.673544133805396	0.001068115234375	\\
0.673666219020877	0.000885009765625	\\
0.673788304236357	0.001129150390625	\\
0.673910389451837	0.001220703125	\\
0.674032474667318	0.001007080078125	\\
0.674154559882798	0.000885009765625	\\
0.674276645098279	0.0006103515625	\\
0.674398730313759	0.000701904296875	\\
0.674520815529239	0.00048828125	\\
0.67464290074472	3.0517578125e-05	\\
0.6747649859602	0	\\
0.674887071175681	0.000244140625	\\
0.675009156391161	0.000335693359375	\\
0.675131241606641	0.000457763671875	\\
0.675253326822122	0.00030517578125	\\
0.675375412037602	0.000335693359375	\\
0.675497497253083	0.000335693359375	\\
0.675619582468563	6.103515625e-05	\\
0.675741667684043	0.0003662109375	\\
0.675863752899524	0.000457763671875	\\
0.675985838115004	0.0001220703125	\\
0.676107923330485	0.000213623046875	\\
0.676230008545965	0.00030517578125	\\
0.676352093761445	-6.103515625e-05	\\
0.676474178976926	-0.000396728515625	\\
0.676596264192406	-0.0010986328125	\\
0.676718349407887	-0.001556396484375	\\
0.676840434623367	-0.001312255859375	\\
0.676962519838847	-0.0013427734375	\\
0.677084605054328	-0.00140380859375	\\
0.677206690269808	-0.001556396484375	\\
0.677328775485289	-0.001617431640625	\\
0.677450860700769	-0.001434326171875	\\
0.67757294591625	-0.0010986328125	\\
0.67769503113173	-0.0008544921875	\\
0.67781711634721	-0.00128173828125	\\
0.677939201562691	-0.00164794921875	\\
0.678061286778171	-0.0015869140625	\\
0.678183371993652	-0.0013427734375	\\
0.678305457209132	-0.00128173828125	\\
0.678427542424612	-0.001556396484375	\\
0.678549627640093	-0.001434326171875	\\
0.678671712855573	-0.001251220703125	\\
0.678793798071054	-0.001495361328125	\\
0.678915883286534	-0.001495361328125	\\
0.679037968502014	-0.00128173828125	\\
0.679160053717495	-0.001007080078125	\\
0.679282138932975	-0.000701904296875	\\
0.679404224148456	-0.000213623046875	\\
0.679526309363936	-0.000152587890625	\\
0.679648394579416	0.000396728515625	\\
0.679770479794897	0.00091552734375	\\
0.679892565010377	0.000762939453125	\\
0.680014650225858	0.000946044921875	\\
0.680136735441338	0.001068115234375	\\
0.680258820656818	0.00115966796875	\\
0.680380905872299	0.000579833984375	\\
0.680502991087779	-0.00030517578125	\\
0.68062507630326	-0.0003662109375	\\
0.68074716151874	-0.000732421875	\\
0.68086924673422	-0.00115966796875	\\
0.680991331949701	-0.0008544921875	\\
0.681113417165181	-0.001068115234375	\\
0.681235502380662	-0.001678466796875	\\
0.681357587596142	-0.001190185546875	\\
0.681479672811622	-0.0010986328125	\\
0.681601758027103	-0.001129150390625	\\
0.681723843242583	-0.001373291015625	\\
0.681845928458064	-0.001373291015625	\\
0.681968013673544	-0.001739501953125	\\
0.682090098889025	-0.001953125	\\
0.682212184104505	-0.00152587890625	\\
0.682334269319985	-0.00128173828125	\\
0.682456354535466	-0.00177001953125	\\
0.682578439750946	-0.00177001953125	\\
0.682700524966427	-0.00128173828125	\\
0.682822610181907	-0.001708984375	\\
0.682944695397387	-0.00225830078125	\\
0.683066780612868	-0.002349853515625	\\
0.683188865828348	-0.00201416015625	\\
0.683310951043829	-0.001495361328125	\\
0.683433036259309	-0.0013427734375	\\
0.683555121474789	-0.0009765625	\\
0.68367720669027	-0.0009765625	\\
0.68379929190575	-0.001068115234375	\\
0.683921377121231	-0.00115966796875	\\
0.684043462336711	-0.0013427734375	\\
0.684165547552191	-0.00115966796875	\\
0.684287632767672	-0.001708984375	\\
0.684409717983152	-0.00189208984375	\\
0.684531803198633	-0.00164794921875	\\
0.684653888414113	-0.001220703125	\\
0.684775973629593	-0.001129150390625	\\
0.684898058845074	-0.001312255859375	\\
0.685020144060554	-0.001434326171875	\\
0.685142229276035	-0.001495361328125	\\
0.685264314491515	-0.001190185546875	\\
0.685386399706995	-0.001068115234375	\\
0.685508484922476	-0.001129150390625	\\
0.685630570137956	-0.001251220703125	\\
0.685752655353437	-0.0010986328125	\\
0.685874740568917	-0.001007080078125	\\
0.685996825784397	-0.0010986328125	\\
0.686118910999878	-0.001068115234375	\\
0.686240996215358	-0.001190185546875	\\
0.686363081430839	-0.001251220703125	\\
0.686485166646319	-0.001251220703125	\\
0.686607251861799	-0.001068115234375	\\
0.68672933707728	-0.000885009765625	\\
0.68685142229276	-0.001068115234375	\\
0.686973507508241	-0.0009765625	\\
0.687095592723721	-0.00054931640625	\\
0.687217677939202	-0.000579833984375	\\
0.687339763154682	-0.000396728515625	\\
0.687461848370162	0.00018310546875	\\
0.687583933585643	0.000396728515625	\\
0.687706018801123	0.000335693359375	\\
0.687828104016604	9.1552734375e-05	\\
0.687950189232084	6.103515625e-05	\\
0.688072274447564	0.00018310546875	\\
0.688194359663045	3.0517578125e-05	\\
0.688316444878525	0.000152587890625	\\
0.688438530094006	6.103515625e-05	\\
0.688560615309486	-9.1552734375e-05	\\
0.688682700524966	0.000213623046875	\\
0.688804785740447	0.000335693359375	\\
0.688926870955927	0.000823974609375	\\
0.689048956171408	0.001251220703125	\\
0.689171041386888	0.00164794921875	\\
0.689293126602368	0.001983642578125	\\
0.689415211817849	0.001617431640625	\\
0.689537297033329	0.00140380859375	\\
0.68965938224881	0.001434326171875	\\
0.68978146746429	0.00146484375	\\
0.68990355267977	0.000885009765625	\\
0.690025637895251	0.00042724609375	\\
0.690147723110731	0.000396728515625	\\
0.690269808326212	-0.000152587890625	\\
0.690391893541692	-0.000152587890625	\\
0.690513978757172	-3.0517578125e-05	\\
0.690636063972653	0.000152587890625	\\
0.690758149188133	0.00042724609375	\\
0.690880234403614	0.000213623046875	\\
0.691002319619094	-0.00018310546875	\\
0.691124404834574	0.0001220703125	\\
0.691246490050055	0.00079345703125	\\
0.691368575265535	0.000732421875	\\
0.691490660481016	0.0008544921875	\\
0.691612745696496	0.000640869140625	\\
0.691734830911977	6.103515625e-05	\\
0.691856916127457	0.000274658203125	\\
0.691979001342937	0	\\
0.692101086558418	-0.000518798828125	\\
0.692223171773898	-0.0003662109375	\\
0.692345256989379	-0.00048828125	\\
0.692467342204859	-0.00054931640625	\\
0.692589427420339	-0.000732421875	\\
0.69271151263582	-0.00103759765625	\\
0.6928335978513	-0.001251220703125	\\
0.692955683066781	-0.00115966796875	\\
0.693077768282261	-0.000579833984375	\\
0.693199853497741	-0.0001220703125	\\
0.693321938713222	-0.0001220703125	\\
0.693444023928702	-0.000152587890625	\\
0.693566109144183	-0.000274658203125	\\
0.693688194359663	-0.000335693359375	\\
0.693810279575143	-0.0001220703125	\\
0.693932364790624	-0.000335693359375	\\
0.694054450006104	-0.001068115234375	\\
0.694176535221585	-0.00115966796875	\\
0.694298620437065	-0.001129150390625	\\
0.694420705652545	-0.001190185546875	\\
0.694542790868026	-0.000579833984375	\\
0.694664876083506	-0.000579833984375	\\
0.694786961298987	-0.00079345703125	\\
0.694909046514467	-0.000335693359375	\\
0.695031131729947	0.000244140625	\\
0.695153216945428	0.00048828125	\\
0.695275302160908	0.0003662109375	\\
0.695397387376389	0.00042724609375	\\
0.695519472591869	0.000518798828125	\\
0.695641557807349	0.0006103515625	\\
0.69576364302283	0.000762939453125	\\
0.69588572823831	0.0003662109375	\\
0.696007813453791	-0.000152587890625	\\
0.696129898669271	-0.00042724609375	\\
0.696251983884752	-0.000579833984375	\\
0.696374069100232	-0.0008544921875	\\
0.696496154315712	-0.000640869140625	\\
0.696618239531193	-0.000885009765625	\\
0.696740324746673	-0.000732421875	\\
0.696862409962154	-0.000274658203125	\\
0.696984495177634	-0.000518798828125	\\
0.697106580393114	-0.000518798828125	\\
0.697228665608595	-0.00067138671875	\\
0.697350750824075	-0.000579833984375	\\
0.697472836039556	-0.000518798828125	\\
0.697594921255036	-0.001007080078125	\\
0.697717006470516	-0.0009765625	\\
0.697839091685997	-0.001312255859375	\\
0.697961176901477	-0.00164794921875	\\
0.698083262116958	-0.002105712890625	\\
0.698205347332438	-0.002593994140625	\\
0.698327432547918	-0.00250244140625	\\
0.698449517763399	-0.002288818359375	\\
0.698571602978879	-0.00225830078125	\\
0.69869368819436	-0.002349853515625	\\
0.69881577340984	-0.00225830078125	\\
0.69893785862532	-0.00201416015625	\\
0.699059943840801	-0.002105712890625	\\
0.699182029056281	-0.001983642578125	\\
0.699304114271762	-0.001556396484375	\\
0.699426199487242	-0.001678466796875	\\
0.699548284702722	-0.00213623046875	\\
0.699670369918203	-0.0020751953125	\\
0.699792455133683	-0.002197265625	\\
0.699914540349164	-0.0023193359375	\\
0.700036625564644	-0.0023193359375	\\
0.700158710780124	-0.00299072265625	\\
0.700280795995605	-0.0035400390625	\\
0.700402881211085	-0.003265380859375	\\
0.700524966426566	-0.002960205078125	\\
0.700647051642046	-0.003173828125	\\
0.700769136857527	-0.00311279296875	\\
0.700891222073007	-0.0028076171875	\\
0.701013307288487	-0.0025634765625	\\
0.701135392503968	-0.002410888671875	\\
0.701257477719448	-0.00244140625	\\
0.701379562934929	-0.00286865234375	\\
0.701501648150409	-0.00323486328125	\\
0.701623733365889	-0.0029296875	\\
0.70174581858137	-0.003082275390625	\\
0.70186790379685	-0.003265380859375	\\
0.701989989012331	-0.0028076171875	\\
0.702112074227811	-0.00286865234375	\\
0.702234159443291	-0.002716064453125	\\
0.702356244658772	-0.002349853515625	\\
0.702478329874252	-0.002288818359375	\\
0.702600415089733	-0.002532958984375	\\
0.702722500305213	-0.00262451171875	\\
0.702844585520693	-0.002197265625	\\
0.702966670736174	-0.001251220703125	\\
0.703088755951654	-0.001312255859375	\\
0.703210841167135	-0.001068115234375	\\
0.703332926382615	-0.001007080078125	\\
0.703455011598095	-0.00103759765625	\\
0.703577096813576	-0.000518798828125	\\
0.703699182029056	-0.000823974609375	\\
0.703821267244537	-0.000885009765625	\\
0.703943352460017	-0.000946044921875	\\
0.704065437675497	-0.00115966796875	\\
0.704187522890978	-0.000518798828125	\\
0.704309608106458	-9.1552734375e-05	\\
0.704431693321939	0.0001220703125	\\
0.704553778537419	0.000152587890625	\\
0.704675863752899	0.000457763671875	\\
0.70479794896838	0.00079345703125	\\
0.70492003418386	0.0010986328125	\\
0.705042119399341	0.0010986328125	\\
0.705164204614821	0.001129150390625	\\
0.705286289830302	0.00152587890625	\\
0.705408375045782	0.0015869140625	\\
0.705530460261262	0.00115966796875	\\
0.705652545476743	0.00054931640625	\\
0.705774630692223	0.000640869140625	\\
0.705896715907704	0.000396728515625	\\
0.706018801123184	0	\\
0.706140886338664	-0.0001220703125	\\
0.706262971554145	0.000244140625	\\
0.706385056769625	6.103515625e-05	\\
0.706507141985106	0.000244140625	\\
0.706629227200586	0.000732421875	\\
0.706751312416066	0.00048828125	\\
0.706873397631547	0.000396728515625	\\
0.706995482847027	0.0003662109375	\\
0.707117568062508	0.000213623046875	\\
0.707239653277988	0.00067138671875	\\
0.707361738493468	0.000701904296875	\\
0.707483823708949	0.000518798828125	\\
0.707605908924429	0.000396728515625	\\
0.70772799413991	0.00030517578125	\\
0.70785007935539	9.1552734375e-05	\\
0.70797216457087	-0.00030517578125	\\
0.708094249786351	-0.000274658203125	\\
0.708216335001831	6.103515625e-05	\\
0.708338420217312	0.000213623046875	\\
0.708460505432792	0.000640869140625	\\
0.708582590648272	0.000732421875	\\
0.708704675863753	0.00042724609375	\\
0.708826761079233	0.001007080078125	\\
0.708948846294714	0.001190185546875	\\
0.709070931510194	0.000732421875	\\
0.709193016725674	0.000762939453125	\\
0.709315101941155	0.000823974609375	\\
0.709437187156635	0.00067138671875	\\
0.709559272372116	0.000640869140625	\\
0.709681357587596	0.00054931640625	\\
0.709803442803077	0.000762939453125	\\
0.709925528018557	0.000946044921875	\\
0.710047613234037	0.000946044921875	\\
0.710169698449518	0.000885009765625	\\
0.710291783664998	0.00103759765625	\\
0.710413868880479	0.000885009765625	\\
0.710535954095959	0.001190185546875	\\
0.710658039311439	0.00177001953125	\\
0.71078012452692	0.0020751953125	\\
0.7109022097424	0.00201416015625	\\
0.711024294957881	0.00213623046875	\\
0.711146380173361	0.002471923828125	\\
0.711268465388841	0.00250244140625	\\
0.711390550604322	0.002105712890625	\\
0.711512635819802	0.002166748046875	\\
0.711634721035283	0.00189208984375	\\
0.711756806250763	0.00201416015625	\\
0.711878891466243	0.002227783203125	\\
0.712000976681724	0.001983642578125	\\
0.712123061897204	0.002288818359375	\\
0.712245147112685	0.002349853515625	\\
0.712367232328165	0.002532958984375	\\
0.712489317543645	0.002471923828125	\\
0.712611402759126	0.002685546875	\\
0.712733487974606	0.0030517578125	\\
0.712855573190087	0.003082275390625	\\
0.712977658405567	0.0035400390625	\\
0.713099743621047	0.003082275390625	\\
0.713221828836528	0.00250244140625	\\
0.713343914052008	0.00262451171875	\\
0.713465999267489	0.00238037109375	\\
0.713588084482969	0.0018310546875	\\
0.713710169698449	0.002044677734375	\\
0.71383225491393	0.0020751953125	\\
0.71395434012941	0.00189208984375	\\
0.714076425344891	0.00189208984375	\\
0.714198510560371	0.00189208984375	\\
0.714320595775852	0.002471923828125	\\
0.714442680991332	0.00244140625	\\
0.714564766206812	0.002227783203125	\\
0.714686851422293	0.002593994140625	\\
0.714808936637773	0.00244140625	\\
0.714931021853254	0.002288818359375	\\
0.715053107068734	0.002166748046875	\\
0.715175192284214	0.002044677734375	\\
0.715297277499695	0.002288818359375	\\
0.715419362715175	0.00213623046875	\\
0.715541447930656	0.002044677734375	\\
0.715663533146136	0.00177001953125	\\
0.715785618361616	0.000885009765625	\\
0.715907703577097	0.000732421875	\\
0.716029788792577	0.000946044921875	\\
0.716151874008058	0.000885009765625	\\
0.716273959223538	0.00079345703125	\\
0.716396044439018	0.00054931640625	\\
0.716518129654499	0.000732421875	\\
0.716640214869979	0.000640869140625	\\
0.71676230008546	0.000335693359375	\\
0.71688438530094	0.000213623046875	\\
0.71700647051642	0.00018310546875	\\
0.717128555731901	0.000335693359375	\\
0.717250640947381	0.00030517578125	\\
0.717372726162862	0.000152587890625	\\
0.717494811378342	0.00018310546875	\\
0.717616896593822	-0.00018310546875	\\
0.717738981809303	-0.000579833984375	\\
0.717861067024783	-0.000640869140625	\\
0.717983152240264	-0.000396728515625	\\
0.718105237455744	-0.00030517578125	\\
0.718227322671224	-0.0003662109375	\\
0.718349407886705	-0.000274658203125	\\
0.718471493102185	0.00018310546875	\\
0.718593578317666	6.103515625e-05	\\
0.718715663533146	0.00030517578125	\\
0.718837748748626	0.0008544921875	\\
0.718959833964107	0.00091552734375	\\
0.719081919179587	0.00054931640625	\\
0.719204004395068	0.000579833984375	\\
0.719326089610548	0.000762939453125	\\
0.719448174826029	0.000518798828125	\\
0.719570260041509	0.00018310546875	\\
0.719692345256989	0.000213623046875	\\
0.71981443047247	6.103515625e-05	\\
0.71993651568795	0.0001220703125	\\
0.720058600903431	-9.1552734375e-05	\\
0.720180686118911	-0.0003662109375	\\
0.720302771334391	-0.0006103515625	\\
0.720424856549872	-0.00054931640625	\\
0.720546941765352	-9.1552734375e-05	\\
0.720669026980833	0.0001220703125	\\
0.720791112196313	-3.0517578125e-05	\\
0.720913197411793	-0.0001220703125	\\
0.721035282627274	-0.000213623046875	\\
0.721157367842754	-0.00054931640625	\\
0.721279453058235	-0.0008544921875	\\
0.721401538273715	-0.001068115234375	\\
0.721523623489195	-0.001068115234375	\\
0.721645708704676	-0.00128173828125	\\
0.721767793920156	-0.001251220703125	\\
0.721889879135637	-0.00128173828125	\\
0.722011964351117	-0.00189208984375	\\
0.722134049566597	-0.00213623046875	\\
0.722256134782078	-0.00177001953125	\\
0.722378219997558	-0.00103759765625	\\
0.722500305213039	-0.000335693359375	\\
0.722622390428519	-0.000579833984375	\\
0.722744475643999	-0.000457763671875	\\
0.72286656085948	-0.00042724609375	\\
0.72298864607496	-0.00042724609375	\\
0.723110731290441	-0.000885009765625	\\
0.723232816505921	-0.001434326171875	\\
0.723354901721401	-0.001312255859375	\\
0.723476986936882	-0.001678466796875	\\
0.723599072152362	-0.00146484375	\\
0.723721157367843	-0.0015869140625	\\
0.723843242583323	-0.001617431640625	\\
0.723965327798804	-0.001495361328125	\\
0.724087413014284	-0.001495361328125	\\
0.724209498229764	-0.00152587890625	\\
0.724331583445245	-0.001495361328125	\\
0.724453668660725	-0.0009765625	\\
0.724575753876206	-0.0008544921875	\\
0.724697839091686	-0.001129150390625	\\
0.724819924307166	-0.001220703125	\\
0.724942009522647	-0.00103759765625	\\
0.725064094738127	-0.001190185546875	\\
0.725186179953608	-0.001556396484375	\\
0.725308265169088	-0.001739501953125	\\
0.725430350384568	-0.001739501953125	\\
0.725552435600049	-0.001739501953125	\\
0.725674520815529	-0.001800537109375	\\
0.72579660603101	-0.001373291015625	\\
0.72591869124649	-0.00140380859375	\\
0.72604077646197	-0.001800537109375	\\
0.726162861677451	-0.001312255859375	\\
0.726284946892931	-0.00091552734375	\\
0.726407032108412	-0.00103759765625	\\
0.726529117323892	-0.000732421875	\\
0.726651202539372	-0.000579833984375	\\
0.726773287754853	-0.00054931640625	\\
0.726895372970333	-0.00042724609375	\\
0.727017458185814	0	\\
0.727139543401294	0.000396728515625	\\
0.727261628616774	0.00054931640625	\\
0.727383713832255	0.000152587890625	\\
0.727505799047735	6.103515625e-05	\\
0.727627884263216	-0.00030517578125	\\
0.727749969478696	-0.000213623046875	\\
0.727872054694176	6.103515625e-05	\\
0.727994139909657	0.000152587890625	\\
0.728116225125137	0.000518798828125	\\
0.728238310340618	0.001068115234375	\\
0.728360395556098	0.001220703125	\\
0.728482480771579	0.001922607421875	\\
0.728604565987059	0.002410888671875	\\
0.728726651202539	0.00244140625	\\
0.72884873641802	0.0028076171875	\\
0.7289708216335	0.00250244140625	\\
0.729092906848981	0.002410888671875	\\
0.729214992064461	0.002197265625	\\
0.729337077279941	0.00152587890625	\\
0.729459162495422	0.000885009765625	\\
0.729581247710902	0.00103759765625	\\
0.729703332926383	0.001739501953125	\\
0.729825418141863	0.001708984375	\\
0.729947503357343	0.00128173828125	\\
0.730069588572824	0.00115966796875	\\
0.730191673788304	0.001190185546875	\\
0.730313759003785	0.001129150390625	\\
0.730435844219265	0.0013427734375	\\
0.730557929434745	0.00140380859375	\\
0.730680014650226	0.001617431640625	\\
0.730802099865706	0.001556396484375	\\
0.730924185081187	0.001373291015625	\\
0.731046270296667	0.001556396484375	\\
0.731168355512147	0.001434326171875	\\
0.731290440727628	0.001068115234375	\\
0.731412525943108	0.000885009765625	\\
0.731534611158589	0.001312255859375	\\
0.731656696374069	0.001373291015625	\\
0.731778781589549	0.00115966796875	\\
0.73190086680503	0.00103759765625	\\
0.73202295202051	0.0006103515625	\\
0.732145037235991	0.000762939453125	\\
0.732267122451471	0.001495361328125	\\
0.732389207666951	0.00140380859375	\\
0.732511292882432	0.001556396484375	\\
0.732633378097912	0.001617431640625	\\
0.732755463313393	0.001190185546875	\\
0.732877548528873	0.001220703125	\\
0.732999633744354	0.0008544921875	\\
0.733121718959834	0.0006103515625	\\
0.733243804175314	9.1552734375e-05	\\
0.733365889390795	-0.000244140625	\\
0.733487974606275	0.000274658203125	\\
0.733610059821756	0.00048828125	\\
0.733732145037236	0.000213623046875	\\
0.733854230252716	0.00018310546875	\\
0.733976315468197	0.000274658203125	\\
0.734098400683677	-0.000152587890625	\\
0.734220485899158	6.103515625e-05	\\
0.734342571114638	0.000213623046875	\\
0.734464656330118	-0.000152587890625	\\
0.734586741545599	0.00018310546875	\\
0.734708826761079	3.0517578125e-05	\\
0.73483091197656	-0.000274658203125	\\
0.73495299719204	-0.000274658203125	\\
0.73507508240752	-0.0006103515625	\\
0.735197167623001	-0.000640869140625	\\
0.735319252838481	-0.000518798828125	\\
0.735441338053962	-0.000701904296875	\\
0.735563423269442	-0.0009765625	\\
0.735685508484922	-0.000946044921875	\\
0.735807593700403	-0.000244140625	\\
0.735929678915883	-0.0001220703125	\\
0.736051764131364	-0.000152587890625	\\
0.736173849346844	0.00018310546875	\\
0.736295934562324	0.0001220703125	\\
0.736418019777805	0.00042724609375	\\
0.736540104993285	0.00079345703125	\\
0.736662190208766	0.00042724609375	\\
0.736784275424246	0.000640869140625	\\
0.736906360639726	0.00091552734375	\\
0.737028445855207	0.00079345703125	\\
0.737150531070687	0.0006103515625	\\
0.737272616286168	-3.0517578125e-05	\\
0.737394701501648	-0.0003662109375	\\
0.737516786717129	-0.00048828125	\\
0.737638871932609	-0.000823974609375	\\
0.737760957148089	-0.00103759765625	\\
0.73788304236357	-0.0010986328125	\\
0.73800512757905	-0.00079345703125	\\
0.738127212794531	-0.001129150390625	\\
0.738249298010011	-0.0008544921875	\\
0.738371383225491	-0.0003662109375	\\
0.738493468440972	-0.000640869140625	\\
0.738615553656452	-0.000823974609375	\\
0.738737638871933	-0.0009765625	\\
0.738859724087413	-0.00115966796875	\\
0.738981809302893	-0.0013427734375	\\
0.739103894518374	-0.001617431640625	\\
0.739225979733854	-0.001739501953125	\\
0.739348064949335	-0.00189208984375	\\
0.739470150164815	-0.002349853515625	\\
0.739592235380295	-0.00238037109375	\\
0.739714320595776	-0.0025634765625	\\
0.739836405811256	-0.0020751953125	\\
0.739958491026737	-0.001983642578125	\\
0.740080576242217	-0.0023193359375	\\
0.740202661457697	-0.001953125	\\
0.740324746673178	-0.0023193359375	\\
0.740446831888658	-0.002716064453125	\\
0.740568917104139	-0.002685546875	\\
0.740691002319619	-0.00238037109375	\\
0.740813087535099	-0.0029296875	\\
0.74093517275058	-0.003753662109375	\\
0.74105725796606	-0.00335693359375	\\
0.741179343181541	-0.003570556640625	\\
0.741301428397021	-0.004241943359375	\\
0.741423513612501	-0.004119873046875	\\
0.741545598827982	-0.00408935546875	\\
0.741667684043462	-0.004364013671875	\\
0.741789769258943	-0.004302978515625	\\
0.741911854474423	-0.004180908203125	\\
0.742033939689904	-0.003631591796875	\\
0.742156024905384	-0.003570556640625	\\
0.742278110120864	-0.00323486328125	\\
0.742400195336345	-0.002685546875	\\
0.742522280551825	-0.002685546875	\\
0.742644365767306	-0.002716064453125	\\
0.742766450982786	-0.0029296875	\\
0.742888536198266	-0.00311279296875	\\
0.743010621413747	-0.002777099609375	\\
0.743132706629227	-0.003082275390625	\\
0.743254791844708	-0.002899169921875	\\
0.743376877060188	-0.002532958984375	\\
0.743498962275668	-0.002593994140625	\\
0.743621047491149	-0.00238037109375	\\
0.743743132706629	-0.00213623046875	\\
0.74386521792211	-0.0018310546875	\\
0.74398730313759	-0.0015869140625	\\
0.74410938835307	-0.00152587890625	\\
0.744231473568551	-0.00140380859375	\\
0.744353558784031	-0.00140380859375	\\
0.744475643999512	-0.00128173828125	\\
0.744597729214992	-0.001708984375	\\
0.744719814430472	-0.00164794921875	\\
0.744841899645953	-0.00152587890625	\\
0.744963984861433	-0.00152587890625	\\
0.745086070076914	-0.001678466796875	\\
0.745208155292394	-0.0020751953125	\\
0.745330240507874	-0.00201416015625	\\
0.745452325723355	-0.001556396484375	\\
0.745574410938835	-0.00152587890625	\\
0.745696496154316	-0.00164794921875	\\
0.745818581369796	-0.001678466796875	\\
0.745940666585276	-0.001953125	\\
0.746062751800757	-0.001678466796875	\\
0.746184837016237	-0.001434326171875	\\
0.746306922231718	-0.001495361328125	\\
0.746429007447198	-0.001434326171875	\\
0.746551092662678	-0.00177001953125	\\
0.746673177878159	-0.00177001953125	\\
0.746795263093639	-0.001617431640625	\\
0.74691734830912	-0.002166748046875	\\
0.7470394335246	-0.002410888671875	\\
0.747161518740081	-0.002593994140625	\\
0.747283603955561	-0.002471923828125	\\
0.747405689171041	-0.00225830078125	\\
0.747527774386522	-0.00244140625	\\
0.747649859602002	-0.00213623046875	\\
0.747771944817483	-0.001708984375	\\
0.747894030032963	-0.00140380859375	\\
0.748016115248443	-0.0013427734375	\\
0.748138200463924	-0.001129150390625	\\
0.748260285679404	-0.000946044921875	\\
0.748382370894885	-0.000946044921875	\\
0.748504456110365	-0.0010986328125	\\
0.748626541325845	-0.001129150390625	\\
0.748748626541326	-0.001190185546875	\\
0.748870711756806	-0.001556396484375	\\
0.748992796972287	-0.001678466796875	\\
0.749114882187767	-0.001922607421875	\\
0.749236967403247	-0.002197265625	\\
0.749359052618728	-0.0023193359375	\\
0.749481137834208	-0.00225830078125	\\
0.749603223049689	-0.001953125	\\
0.749725308265169	-0.00177001953125	\\
0.749847393480649	-0.001373291015625	\\
0.74996947869613	-0.000885009765625	\\
0.75009156391161	-0.000701904296875	\\
0.750213649127091	-0.000457763671875	\\
0.750335734342571	-0.000732421875	\\
0.750457819558052	-0.000579833984375	\\
0.750579904773532	-0.0006103515625	\\
0.750701989989012	-0.000701904296875	\\
0.750824075204493	-0.000640869140625	\\
0.750946160419973	-0.000457763671875	\\
0.751068245635454	-0.0003662109375	\\
0.751190330850934	-0.000396728515625	\\
0.751312416066414	-0.00018310546875	\\
0.751434501281895	-0.000244140625	\\
0.751556586497375	-0.00030517578125	\\
0.751678671712856	0	\\
0.751800756928336	0.0003662109375	\\
0.751922842143816	0.000396728515625	\\
0.752044927359297	0.00048828125	\\
0.752167012574777	0.0003662109375	\\
0.752289097790258	0.000518798828125	\\
0.752411183005738	0.0008544921875	\\
0.752533268221218	0.000701904296875	\\
0.752655353436699	0.000457763671875	\\
0.752777438652179	0.00054931640625	\\
0.75289952386766	0.001190185546875	\\
0.75302160908314	0.001556396484375	\\
0.75314369429862	0.0013427734375	\\
0.753265779514101	0.001129150390625	\\
0.753387864729581	0.001190185546875	\\
0.753509949945062	0.00128173828125	\\
0.753632035160542	0.001678466796875	\\
0.753754120376023	0.0015869140625	\\
0.753876205591503	0.00140380859375	\\
0.753998290806983	0.0018310546875	\\
0.754120376022464	0.001800537109375	\\
0.754242461237944	0.001617431640625	\\
0.754364546453425	0.00140380859375	\\
0.754486631668905	0.000885009765625	\\
0.754608716884385	0.000946044921875	\\
0.754730802099866	0.000823974609375	\\
0.754852887315346	0.000640869140625	\\
0.754974972530827	0.0001220703125	\\
0.755097057746307	9.1552734375e-05	\\
0.755219142961787	0.0003662109375	\\
0.755341228177268	0.0003662109375	\\
0.755463313392748	0.00048828125	\\
0.755585398608229	0.000152587890625	\\
0.755707483823709	0.000396728515625	\\
0.755829569039189	0.000885009765625	\\
0.75595165425467	0.00042724609375	\\
0.75607373947015	-6.103515625e-05	\\
0.756195824685631	-3.0517578125e-05	\\
0.756317909901111	-9.1552734375e-05	\\
0.756439995116591	-0.000518798828125	\\
0.756562080332072	-0.00091552734375	\\
0.756684165547552	-0.001251220703125	\\
0.756806250763033	-0.001708984375	\\
0.756928335978513	-0.002105712890625	\\
0.757050421193993	-0.0018310546875	\\
0.757172506409474	-0.0015869140625	\\
0.757294591624954	-0.001556396484375	\\
0.757416676840435	-0.00103759765625	\\
0.757538762055915	-0.000701904296875	\\
0.757660847271395	-0.00079345703125	\\
0.757782932486876	-0.0006103515625	\\
0.757905017702356	-0.000335693359375	\\
0.758027102917837	-0.00018310546875	\\
0.758149188133317	-0.00042724609375	\\
0.758271273348798	-0.00054931640625	\\
0.758393358564278	-0.00054931640625	\\
0.758515443779758	-0.000640869140625	\\
0.758637528995239	-0.000946044921875	\\
0.758759614210719	-0.00128173828125	\\
0.7588816994262	-0.0010986328125	\\
0.75900378464168	-0.001312255859375	\\
0.75912586985716	-0.001373291015625	\\
0.759247955072641	-0.0015869140625	\\
0.759370040288121	-0.00152587890625	\\
0.759492125503602	-0.0006103515625	\\
0.759614210719082	-0.000335693359375	\\
0.759736295934562	-0.0006103515625	\\
0.759858381150043	-0.00042724609375	\\
0.759980466365523	-0.000457763671875	\\
0.760102551581004	-0.000518798828125	\\
0.760224636796484	-0.000213623046875	\\
0.760346722011964	-0.00030517578125	\\
0.760468807227445	-0.00048828125	\\
0.760590892442925	-0.0001220703125	\\
0.760712977658406	0.00042724609375	\\
0.760835062873886	0.000701904296875	\\
0.760957148089366	0.0003662109375	\\
0.761079233304847	0.000274658203125	\\
0.761201318520327	0.0003662109375	\\
0.761323403735808	0.00067138671875	\\
0.761445488951288	0.00079345703125	\\
0.761567574166768	0.000885009765625	\\
0.761689659382249	0.0006103515625	\\
0.761811744597729	0.000518798828125	\\
0.76193382981321	0.00054931640625	\\
0.76205591502869	0.000640869140625	\\
0.76217800024417	0.00054931640625	\\
0.762300085459651	0.00030517578125	\\
0.762422170675131	0.000274658203125	\\
0.762544255890612	0.0003662109375	\\
0.762666341106092	-6.103515625e-05	\\
0.762788426321573	-0.00067138671875	\\
0.762910511537053	-0.0006103515625	\\
0.763032596752533	-0.00042724609375	\\
0.763154681968014	-0.000213623046875	\\
0.763276767183494	0	\\
0.763398852398975	3.0517578125e-05	\\
0.763520937614455	-0.000213623046875	\\
0.763643022829935	-0.000213623046875	\\
0.763765108045416	-0.0001220703125	\\
0.763887193260896	0	\\
0.764009278476377	-0.00018310546875	\\
0.764131363691857	-0.000518798828125	\\
0.764253448907337	-0.000640869140625	\\
0.764375534122818	-0.001007080078125	\\
0.764497619338298	-0.00067138671875	\\
0.764619704553779	-0.000579833984375	\\
0.764741789769259	-0.000732421875	\\
0.764863874984739	-0.0008544921875	\\
0.76498596020022	-0.00115966796875	\\
0.7651080454157	-0.001007080078125	\\
0.765230130631181	-0.00103759765625	\\
0.765352215846661	-0.001129150390625	\\
0.765474301062141	-0.001007080078125	\\
0.765596386277622	-0.001068115234375	\\
0.765718471493102	-0.00115966796875	\\
0.765840556708583	-0.000885009765625	\\
0.765962641924063	-0.0010986328125	\\
0.766084727139543	-0.00115966796875	\\
0.766206812355024	-0.001556396484375	\\
0.766328897570504	-0.001922607421875	\\
0.766450982785985	-0.001708984375	\\
0.766573068001465	-0.00146484375	\\
0.766695153216945	-0.001678466796875	\\
0.766817238432426	-0.00152587890625	\\
0.766939323647906	-0.001129150390625	\\
0.767061408863387	-0.001068115234375	\\
0.767183494078867	-0.0010986328125	\\
0.767305579294348	-0.0006103515625	\\
0.767427664509828	-0.000457763671875	\\
0.767549749725308	-0.000640869140625	\\
0.767671834940789	3.0517578125e-05	\\
0.767793920156269	0.00018310546875	\\
0.76791600537175	0.000335693359375	\\
0.76803809058723	0.00048828125	\\
0.76816017580271	0.000244140625	\\
0.768282261018191	0.000396728515625	\\
0.768404346233671	0.00079345703125	\\
0.768526431449152	0.00067138671875	\\
0.768648516664632	0.00091552734375	\\
0.768770601880112	0.000823974609375	\\
0.768892687095593	0.0003662109375	\\
0.769014772311073	0.00042724609375	\\
0.769136857526554	0.0010986328125	\\
0.769258942742034	0.001373291015625	\\
0.769381027957514	0.00115966796875	\\
0.769503113172995	0.0015869140625	\\
0.769625198388475	0.00152587890625	\\
0.769747283603956	0.001556396484375	\\
0.769869368819436	0.00177001953125	\\
0.769991454034916	0.0018310546875	\\
0.770113539250397	0.00152587890625	\\
0.770235624465877	0.001068115234375	\\
0.770357709681358	0.0010986328125	\\
0.770479794896838	0.000946044921875	\\
0.770601880112318	0.0009765625	\\
0.770723965327799	0.00103759765625	\\
0.770846050543279	0.00103759765625	\\
0.77096813575876	0.0009765625	\\
0.77109022097424	0.000885009765625	\\
0.77121230618972	0.0010986328125	\\
0.771334391405201	0.0010986328125	\\
0.771456476620681	0.00146484375	\\
0.771578561836162	0.00152587890625	\\
0.771700647051642	0.001434326171875	\\
0.771822732267123	0.001556396484375	\\
0.771944817482603	0.00140380859375	\\
0.772066902698083	0.0013427734375	\\
0.772188987913564	0.000762939453125	\\
0.772311073129044	0.000732421875	\\
0.772433158344525	0.00067138671875	\\
0.772555243560005	0.00079345703125	\\
0.772677328775485	0.001129150390625	\\
0.772799413990966	0.00146484375	\\
0.772921499206446	0.001251220703125	\\
0.773043584421927	0.001129150390625	\\
0.773165669637407	0.001373291015625	\\
0.773287754852887	0.001495361328125	\\
0.773409840068368	0.001708984375	\\
0.773531925283848	0.00177001953125	\\
0.773654010499329	0.002044677734375	\\
0.773776095714809	0.00177001953125	\\
0.773898180930289	0.00152587890625	\\
0.77402026614577	0.001861572265625	\\
0.77414235136125	0.00152587890625	\\
0.774264436576731	0.001556396484375	\\
0.774386521792211	0.001617431640625	\\
0.774508607007691	0.001434326171875	\\
0.774630692223172	0.00140380859375	\\
0.774752777438652	0.001251220703125	\\
0.774874862654133	0.001190185546875	\\
0.774996947869613	0.0013427734375	\\
0.775119033085093	0.001861572265625	\\
0.775241118300574	0.00152587890625	\\
0.775363203516054	0.001373291015625	\\
0.775485288731535	0.002288818359375	\\
0.775607373947015	0.00201416015625	\\
0.775729459162495	0.001922607421875	\\
0.775851544377976	0.002166748046875	\\
0.775973629593456	0.00189208984375	\\
0.776095714808937	0.001739501953125	\\
0.776217800024417	0.001922607421875	\\
0.776339885239898	0.00177001953125	\\
0.776461970455378	0.0018310546875	\\
0.776584055670858	0.001556396484375	\\
0.776706140886339	0.00140380859375	\\
0.776828226101819	0.001617431640625	\\
0.7769503113173	0.001739501953125	\\
0.77707239653278	0.00225830078125	\\
0.77719448174826	0.0023193359375	\\
0.777316566963741	0.00201416015625	\\
0.777438652179221	0.0023193359375	\\
0.777560737394702	0.002655029296875	\\
0.777682822610182	0.002716064453125	\\
0.777804907825662	0.003021240234375	\\
0.777926993041143	0.002716064453125	\\
0.778049078256623	0.002227783203125	\\
0.778171163472104	0.00189208984375	\\
0.778293248687584	0.0020751953125	\\
0.778415333903064	0.002166748046875	\\
0.778537419118545	0.00146484375	\\
0.778659504334025	0.001129150390625	\\
0.778781589549506	0.001220703125	\\
0.778903674764986	0.001068115234375	\\
0.779025759980466	0.00115966796875	\\
0.779147845195947	0.00103759765625	\\
0.779269930411427	0.00115966796875	\\
0.779392015626908	0.00103759765625	\\
0.779514100842388	0.00048828125	\\
0.779636186057868	0.000518798828125	\\
0.779758271273349	0.00030517578125	\\
0.779880356488829	3.0517578125e-05	\\
0.78000244170431	-0.000244140625	\\
0.78012452691979	-0.000457763671875	\\
0.78024661213527	-0.000335693359375	\\
0.780368697350751	-0.0001220703125	\\
0.780490782566231	-0.000885009765625	\\
0.780612867781712	-0.0009765625	\\
0.780734952997192	-0.0010986328125	\\
0.780857038212672	-0.001007080078125	\\
0.780979123428153	-0.000701904296875	\\
0.781101208643633	-0.00103759765625	\\
0.781223293859114	-0.00115966796875	\\
0.781345379074594	-0.0010986328125	\\
0.781467464290075	-0.0008544921875	\\
0.781589549505555	-0.00067138671875	\\
0.781711634721035	-0.000823974609375	\\
0.781833719936516	-0.001251220703125	\\
0.781955805151996	-0.001800537109375	\\
0.782077890367477	-0.001495361328125	\\
0.782199975582957	-0.001373291015625	\\
0.782322060798437	-0.001434326171875	\\
0.782444146013918	-0.001678466796875	\\
0.782566231229398	-0.001800537109375	\\
0.782688316444879	-0.00128173828125	\\
0.782810401660359	-0.001190185546875	\\
0.782932486875839	-0.0009765625	\\
0.78305457209132	-0.00030517578125	\\
0.7831766573068	-0.00030517578125	\\
0.783298742522281	-0.000457763671875	\\
0.783420827737761	-0.000274658203125	\\
0.783542912953241	-0.00030517578125	\\
0.783664998168722	-0.000762939453125	\\
0.783787083384202	-0.0009765625	\\
0.783909168599683	-0.00091552734375	\\
0.784031253815163	-0.0006103515625	\\
0.784153339030643	-0.000518798828125	\\
0.784275424246124	-0.0006103515625	\\
0.784397509461604	-0.000274658203125	\\
0.784519594677085	0	\\
0.784641679892565	0.000152587890625	\\
0.784763765108045	-0.00018310546875	\\
0.784885850323526	-0.000244140625	\\
0.785007935539006	-0.000213623046875	\\
0.785130020754487	-0.000152587890625	\\
0.785252105969967	0.00042724609375	\\
0.785374191185447	0.00048828125	\\
0.785496276400928	0.000701904296875	\\
0.785618361616408	0.000762939453125	\\
0.785740446831889	0.00048828125	\\
0.785862532047369	0.000152587890625	\\
0.78598461726285	-0.000152587890625	\\
0.78610670247833	-0.000396728515625	\\
0.78622878769381	-0.000396728515625	\\
0.786350872909291	-0.000701904296875	\\
0.786472958124771	-0.00079345703125	\\
0.786595043340252	-0.0003662109375	\\
0.786717128555732	-0.000274658203125	\\
0.786839213771212	0	\\
0.786961298986693	0.000732421875	\\
0.787083384202173	0.00048828125	\\
0.787205469417654	0.000518798828125	\\
0.787327554633134	0.000732421875	\\
0.787449639848614	0.00030517578125	\\
0.787571725064095	0.0006103515625	\\
0.787693810279575	0.00054931640625	\\
0.787815895495056	0.000335693359375	\\
0.787937980710536	0.00018310546875	\\
0.788060065926016	-0.000701904296875	\\
0.788182151141497	-0.000274658203125	\\
0.788304236356977	-0.000152587890625	\\
0.788426321572458	-0.000885009765625	\\
0.788548406787938	-0.000640869140625	\\
0.788670492003418	-3.0517578125e-05	\\
0.788792577218899	0.000244140625	\\
0.788914662434379	0.000396728515625	\\
0.78903674764986	0.000213623046875	\\
0.78915883286534	6.103515625e-05	\\
0.78928091808082	0.000274658203125	\\
0.789403003296301	0.000274658203125	\\
0.789525088511781	0.000244140625	\\
0.789647173727262	-9.1552734375e-05	\\
0.789769258942742	-0.000213623046875	\\
0.789891344158222	-0.00042724609375	\\
0.790013429373703	-0.000701904296875	\\
0.790135514589183	-0.000946044921875	\\
0.790257599804664	-0.0009765625	\\
0.790379685020144	-0.0006103515625	\\
0.790501770235625	-0.000701904296875	\\
0.790623855451105	-0.00067138671875	\\
0.790745940666585	-0.000335693359375	\\
0.790868025882066	0.000213623046875	\\
0.790990111097546	0.00030517578125	\\
0.791112196313027	6.103515625e-05	\\
0.791234281528507	0.000274658203125	\\
0.791356366743987	0.00048828125	\\
0.791478451959468	0.000457763671875	\\
0.791600537174948	0.000213623046875	\\
0.791722622390429	9.1552734375e-05	\\
0.791844707605909	0.000457763671875	\\
0.791966792821389	0.000885009765625	\\
0.79208887803687	0.00103759765625	\\
0.79221096325235	0.00140380859375	\\
0.792333048467831	0.00164794921875	\\
0.792455133683311	0.001495361328125	\\
0.792577218898791	0.002044677734375	\\
0.792699304114272	0.002044677734375	\\
0.792821389329752	0.001983642578125	\\
0.792943474545233	0.002227783203125	\\
0.793065559760713	0.001983642578125	\\
0.793187644976193	0.001983642578125	\\
0.793309730191674	0.001983642578125	\\
0.793431815407154	0.0020751953125	\\
0.793553900622635	0.00238037109375	\\
0.793675985838115	0.002532958984375	\\
0.793798071053595	0.002349853515625	\\
0.793920156269076	0.002197265625	\\
0.794042241484556	0.00225830078125	\\
0.794164326700037	0.0020751953125	\\
0.794286411915517	0.00225830078125	\\
0.794408497130997	0.002227783203125	\\
0.794530582346478	0.002197265625	\\
0.794652667561958	0.001983642578125	\\
0.794774752777439	0.00201416015625	\\
0.794896837992919	0.002227783203125	\\
0.7950189232084	0.002349853515625	\\
0.79514100842388	0.002655029296875	\\
0.79526309363936	0.0020751953125	\\
0.795385178854841	0.00244140625	\\
0.795507264070321	0.0025634765625	\\
0.795629349285802	0.002166748046875	\\
0.795751434501282	0.00213623046875	\\
0.795873519716762	0.0018310546875	\\
0.795995604932243	0.00164794921875	\\
0.796117690147723	0.0013427734375	\\
0.796239775363204	0.00140380859375	\\
0.796361860578684	0.001434326171875	\\
0.796483945794164	0.001251220703125	\\
0.796606031009645	0.00128173828125	\\
0.796728116225125	0.00152587890625	\\
0.796850201440606	0.001708984375	\\
0.796972286656086	0.001983642578125	\\
0.797094371871566	0.001678466796875	\\
0.797216457087047	0.00140380859375	\\
0.797338542302527	0.001220703125	\\
0.797460627518008	0.00091552734375	\\
0.797582712733488	0.0008544921875	\\
0.797704797948968	0.000732421875	\\
0.797826883164449	0.000701904296875	\\
0.797948968379929	0.000518798828125	\\
0.79807105359541	6.103515625e-05	\\
0.79819313881089	-0.00030517578125	\\
0.79831522402637	-0.000274658203125	\\
0.798437309241851	-0.000274658203125	\\
0.798559394457331	-0.00030517578125	\\
0.798681479672812	-0.00018310546875	\\
0.798803564888292	-9.1552734375e-05	\\
0.798925650103772	0.00018310546875	\\
0.799047735319253	0.00018310546875	\\
0.799169820534733	-3.0517578125e-05	\\
0.799291905750214	-0.000244140625	\\
0.799413990965694	-0.000396728515625	\\
0.799536076181175	-0.00048828125	\\
0.799658161396655	-0.0003662109375	\\
0.799780246612135	-0.00054931640625	\\
0.799902331827616	-0.00018310546875	\\
0.800024417043096	0.000274658203125	\\
0.800146502258577	-9.1552734375e-05	\\
0.800268587474057	-0.000762939453125	\\
0.800390672689537	-0.000152587890625	\\
0.800512757905018	0.000335693359375	\\
0.800634843120498	0.000274658203125	\\
0.800756928335979	0.00054931640625	\\
0.800879013551459	0.000457763671875	\\
0.801001098766939	0.000946044921875	\\
0.80112318398242	0.00103759765625	\\
0.8012452691979	0.0006103515625	\\
0.801367354413381	0.0003662109375	\\
0.801489439628861	3.0517578125e-05	\\
0.801611524844341	3.0517578125e-05	\\
0.801733610059822	0.00042724609375	\\
0.801855695275302	0.000579833984375	\\
0.801977780490783	0.00048828125	\\
0.802099865706263	0.000335693359375	\\
0.802221950921743	0.00042724609375	\\
0.802344036137224	0.00054931640625	\\
0.802466121352704	0.000823974609375	\\
0.802588206568185	0.000701904296875	\\
0.802710291783665	0.000579833984375	\\
0.802832376999145	0.0009765625	\\
0.802954462214626	0.0009765625	\\
0.803076547430106	0.00067138671875	\\
0.803198632645587	0.0008544921875	\\
0.803320717861067	0.000213623046875	\\
0.803442803076547	-0.00018310546875	\\
0.803564888292028	-0.00042724609375	\\
0.803686973507508	-0.000274658203125	\\
0.803809058722989	-0.000823974609375	\\
0.803931143938469	-0.001129150390625	\\
0.80405322915395	-0.00103759765625	\\
0.80417531436943	-0.001190185546875	\\
0.80429739958491	-0.000885009765625	\\
0.804419484800391	-0.000640869140625	\\
0.804541570015871	-0.000640869140625	\\
0.804663655231352	-0.0003662109375	\\
0.804785740446832	-9.1552734375e-05	\\
0.804907825662312	-0.000274658203125	\\
0.805029910877793	-0.00048828125	\\
0.805151996093273	-0.000823974609375	\\
0.805274081308754	-0.00128173828125	\\
0.805396166524234	-0.001373291015625	\\
0.805518251739714	-0.00177001953125	\\
0.805640336955195	-0.001983642578125	\\
0.805762422170675	-0.002288818359375	\\
0.805884507386156	-0.00250244140625	\\
0.806006592601636	-0.002471923828125	\\
0.806128677817116	-0.00225830078125	\\
0.806250763032597	-0.00201416015625	\\
0.806372848248077	-0.00225830078125	\\
0.806494933463558	-0.001953125	\\
0.806617018679038	-0.001617431640625	\\
0.806739103894518	-0.001373291015625	\\
0.806861189109999	-0.001007080078125	\\
0.806983274325479	-0.00115966796875	\\
0.80710535954096	-0.0010986328125	\\
0.80722744475644	-0.001190185546875	\\
0.80734952997192	-0.001312255859375	\\
0.807471615187401	-0.00146484375	\\
0.807593700402881	-0.001708984375	\\
0.807715785618362	-0.002105712890625	\\
0.807837870833842	-0.00225830078125	\\
0.807959956049322	-0.002288818359375	\\
0.808082041264803	-0.00238037109375	\\
0.808204126480283	-0.002349853515625	\\
0.808326211695764	-0.00164794921875	\\
0.808448296911244	-0.00115966796875	\\
0.808570382126725	-0.00128173828125	\\
0.808692467342205	-0.000885009765625	\\
0.808814552557685	-0.000823974609375	\\
0.808936637773166	-0.001220703125	\\
0.809058722988646	-0.0009765625	\\
0.809180808204127	-0.0009765625	\\
0.809302893419607	-0.000701904296875	\\
0.809424978635087	-0.000244140625	\\
0.809547063850568	-0.00030517578125	\\
0.809669149066048	-0.00048828125	\\
0.809791234281529	-0.000396728515625	\\
0.809913319497009	-0.00042724609375	\\
0.810035404712489	-0.00054931640625	\\
0.81015748992797	-0.00048828125	\\
0.81027957514345	-0.00054931640625	\\
0.810401660358931	-0.000579833984375	\\
0.810523745574411	-0.00018310546875	\\
0.810645830789891	-0.000213623046875	\\
0.810767916005372	-0.0003662109375	\\
0.810890001220852	-9.1552734375e-05	\\
0.811012086436333	3.0517578125e-05	\\
0.811134171651813	3.0517578125e-05	\\
0.811256256867293	-0.000213623046875	\\
0.811378342082774	-0.00042724609375	\\
0.811500427298254	-0.0001220703125	\\
0.811622512513735	0.000335693359375	\\
0.811744597729215	6.103515625e-05	\\
0.811866682944695	-0.0001220703125	\\
0.811988768160176	3.0517578125e-05	\\
0.812110853375656	-0.00042724609375	\\
0.812232938591137	-0.000518798828125	\\
0.812355023806617	-0.0003662109375	\\
0.812477109022097	-0.0006103515625	\\
0.812599194237578	-0.00042724609375	\\
0.812721279453058	-0.00048828125	\\
0.812843364668539	-0.001007080078125	\\
0.812965449884019	-0.00115966796875	\\
0.813087535099499	-0.000640869140625	\\
0.81320962031498	-0.000885009765625	\\
0.81333170553046	-0.001434326171875	\\
0.813453790745941	-0.001312255859375	\\
0.813575875961421	-0.001312255859375	\\
0.813697961176902	-0.00177001953125	\\
0.813820046392382	-0.00201416015625	\\
0.813942131607862	-0.00189208984375	\\
0.814064216823343	-0.001678466796875	\\
0.814186302038823	-0.001617431640625	\\
0.814308387254304	-0.001495361328125	\\
0.814430472469784	-0.00128173828125	\\
0.814552557685264	-0.001373291015625	\\
0.814674642900745	-0.0010986328125	\\
0.814796728116225	-0.001068115234375	\\
0.814918813331706	-0.001434326171875	\\
0.815040898547186	-0.00128173828125	\\
0.815162983762666	-0.00140380859375	\\
0.815285068978147	-0.001861572265625	\\
0.815407154193627	-0.001922607421875	\\
0.815529239409108	-0.002044677734375	\\
0.815651324624588	-0.0018310546875	\\
0.815773409840068	-0.001556396484375	\\
0.815895495055549	-0.001220703125	\\
0.816017580271029	-0.0006103515625	\\
0.81613966548651	-0.00018310546875	\\
0.81626175070199	0	\\
0.81638383591747	-0.0001220703125	\\
0.816505921132951	9.1552734375e-05	\\
0.816628006348431	0.000152587890625	\\
0.816750091563912	0.000213623046875	\\
0.816872176779392	0.00042724609375	\\
0.816994261994872	0.0003662109375	\\
0.817116347210353	0.00018310546875	\\
0.817238432425833	-0.000152587890625	\\
0.817360517641314	-0.000396728515625	\\
0.817482602856794	-0.000732421875	\\
0.817604688072274	-0.000762939453125	\\
0.817726773287755	-0.0006103515625	\\
0.817848858503235	-0.000640869140625	\\
0.817970943718716	-0.00030517578125	\\
0.818093028934196	0.0001220703125	\\
0.818215114149677	0.000396728515625	\\
0.818337199365157	0.0006103515625	\\
0.818459284580637	0.001007080078125	\\
0.818581369796118	0.001220703125	\\
0.818703455011598	0.001190185546875	\\
0.818825540227079	0.000701904296875	\\
0.818947625442559	0.000885009765625	\\
0.819069710658039	0.000732421875	\\
0.81919179587352	0.00042724609375	\\
0.819313881089	0.000274658203125	\\
0.819435966304481	0	\\
0.819558051519961	-9.1552734375e-05	\\
0.819680136735441	-0.0003662109375	\\
0.819802221950922	-0.00042724609375	\\
0.819924307166402	-0.00030517578125	\\
0.820046392381883	0.000152587890625	\\
0.820168477597363	0.0003662109375	\\
0.820290562812843	0.000152587890625	\\
0.820412648028324	0.000274658203125	\\
0.820534733243804	0.00048828125	\\
0.820656818459285	0.000213623046875	\\
0.820778903674765	0.00018310546875	\\
0.820900988890245	-0.000213623046875	\\
0.821023074105726	-0.00103759765625	\\
0.821145159321206	-0.001068115234375	\\
0.821267244536687	-0.000762939453125	\\
0.821389329752167	-0.000640869140625	\\
0.821511414967647	-0.000885009765625	\\
0.821633500183128	-0.001068115234375	\\
0.821755585398608	-0.00103759765625	\\
0.821877670614089	-0.001373291015625	\\
0.821999755829569	-0.001129150390625	\\
0.822121841045049	-0.00103759765625	\\
0.82224392626053	-0.001373291015625	\\
0.82236601147601	-0.00140380859375	\\
0.822488096691491	-0.00152587890625	\\
0.822610181906971	-0.001678466796875	\\
0.822732267122452	-0.001983642578125	\\
0.822854352337932	-0.002410888671875	\\
0.822976437553412	-0.002105712890625	\\
0.823098522768893	-0.001739501953125	\\
0.823220607984373	-0.0020751953125	\\
0.823342693199854	-0.00213623046875	\\
0.823464778415334	-0.002349853515625	\\
0.823586863630814	-0.002044677734375	\\
0.823708948846295	-0.00177001953125	\\
0.823831034061775	-0.00201416015625	\\
0.823953119277256	-0.001983642578125	\\
0.824075204492736	-0.0023193359375	\\
0.824197289708216	-0.0025634765625	\\
0.824319374923697	-0.00244140625	\\
0.824441460139177	-0.002227783203125	\\
0.824563545354658	-0.002288818359375	\\
0.824685630570138	-0.00250244140625	\\
0.824807715785618	-0.002288818359375	\\
0.824929801001099	-0.001953125	\\
0.825051886216579	-0.002349853515625	\\
0.82517397143206	-0.0023193359375	\\
0.82529605664754	-0.002349853515625	\\
0.82541814186302	-0.0025634765625	\\
0.825540227078501	-0.002655029296875	\\
0.825662312293981	-0.0025634765625	\\
0.825784397509462	-0.001922607421875	\\
0.825906482724942	-0.001800537109375	\\
0.826028567940422	-0.001922607421875	\\
0.826150653155903	-0.00189208984375	\\
0.826272738371383	-0.001739501953125	\\
0.826394823586864	-0.001922607421875	\\
0.826516908802344	-0.001617431640625	\\
0.826638994017824	-0.001861572265625	\\
0.826761079233305	-0.002105712890625	\\
0.826883164448785	-0.00238037109375	\\
0.827005249664266	-0.002777099609375	\\
0.827127334879746	-0.0028076171875	\\
0.827249420095227	-0.00262451171875	\\
0.827371505310707	-0.002532958984375	\\
0.827493590526187	-0.0028076171875	\\
0.827615675741668	-0.002685546875	\\
0.827737760957148	-0.002044677734375	\\
0.827859846172629	-0.0018310546875	\\
0.827981931388109	-0.002227783203125	\\
0.828104016603589	-0.002197265625	\\
0.82822610181907	-0.0020751953125	\\
0.82834818703455	-0.00213623046875	\\
0.828470272250031	-0.0023193359375	\\
0.828592357465511	-0.001983642578125	\\
0.828714442680991	-0.0025634765625	\\
0.828836527896472	-0.0029296875	\\
0.828958613111952	-0.00347900390625	\\
0.829080698327433	-0.004119873046875	\\
0.829202783542913	-0.00408935546875	\\
0.829324868758393	-0.004241943359375	\\
0.829446953973874	-0.00445556640625	\\
0.829569039189354	-0.004119873046875	\\
0.829691124404835	-0.00390625	\\
0.829813209620315	-0.00408935546875	\\
0.829935294835795	-0.003570556640625	\\
0.830057380051276	-0.0029296875	\\
0.830179465266756	-0.002899169921875	\\
0.830301550482237	-0.002899169921875	\\
0.830423635697717	-0.00286865234375	\\
0.830545720913197	-0.0030517578125	\\
0.830667806128678	-0.003204345703125	\\
0.830789891344158	-0.003326416015625	\\
0.830911976559639	-0.003448486328125	\\
0.831034061775119	-0.00323486328125	\\
0.831156146990599	-0.0029296875	\\
0.83127823220608	-0.0035400390625	\\
0.83140031742156	-0.0035400390625	\\
0.831522402637041	-0.00299072265625	\\
0.831644487852521	-0.002777099609375	\\
0.831766573068002	-0.002960205078125	\\
0.831888658283482	-0.00286865234375	\\
0.832010743498962	-0.00262451171875	\\
0.832132828714443	-0.0028076171875	\\
0.832254913929923	-0.002532958984375	\\
0.832376999145404	-0.002227783203125	\\
0.832499084360884	-0.001861572265625	\\
0.832621169576364	-0.001922607421875	\\
0.832743254791845	-0.002349853515625	\\
0.832865340007325	-0.0018310546875	\\
0.832987425222806	-0.00177001953125	\\
0.833109510438286	-0.00189208984375	\\
0.833231595653766	-0.001922607421875	\\
0.833353680869247	-0.001861572265625	\\
0.833475766084727	-0.001556396484375	\\
0.833597851300208	-0.001678466796875	\\
0.833719936515688	-0.001708984375	\\
0.833842021731168	-0.001739501953125	\\
0.833964106946649	-0.00177001953125	\\
0.834086192162129	-0.001739501953125	\\
0.83420827737761	-0.001708984375	\\
0.83433036259309	-0.001556396484375	\\
0.83445244780857	-0.001861572265625	\\
0.834574533024051	-0.0020751953125	\\
0.834696618239531	-0.002044677734375	\\
0.834818703455012	-0.002044677734375	\\
0.834940788670492	-0.002197265625	\\
0.835062873885972	-0.00213623046875	\\
0.835184959101453	-0.002166748046875	\\
0.835307044316933	-0.002288818359375	\\
0.835429129532414	-0.002044677734375	\\
0.835551214747894	-0.001739501953125	\\
0.835673299963374	-0.001190185546875	\\
0.835795385178855	-0.00103759765625	\\
0.835917470394335	-0.00115966796875	\\
0.836039555609816	-0.0009765625	\\
0.836161640825296	-0.000701904296875	\\
0.836283726040777	-0.000701904296875	\\
0.836405811256257	-0.000762939453125	\\
0.836527896471737	-0.001068115234375	\\
0.836649981687218	-0.001068115234375	\\
0.836772066902698	-0.0013427734375	\\
0.836894152118179	-0.001434326171875	\\
0.837016237333659	-0.001220703125	\\
0.837138322549139	-0.001190185546875	\\
0.83726040776462	-0.001495361328125	\\
0.8373824929801	-0.00140380859375	\\
0.837504578195581	-0.0006103515625	\\
0.837626663411061	-0.000701904296875	\\
0.837748748626541	-0.000823974609375	\\
0.837870833842022	-0.001068115234375	\\
0.837992919057502	-0.000946044921875	\\
0.838115004272983	-0.000823974609375	\\
0.838237089488463	-0.001312255859375	\\
0.838359174703943	-0.00146484375	\\
0.838481259919424	-0.001708984375	\\
0.838603345134904	-0.00164794921875	\\
0.838725430350385	-0.00177001953125	\\
0.838847515565865	-0.001922607421875	\\
0.838969600781345	-0.00177001953125	\\
0.839091685996826	-0.001678466796875	\\
0.839213771212306	-0.00146484375	\\
0.839335856427787	-0.001312255859375	\\
0.839457941643267	-0.001434326171875	\\
0.839580026858747	-0.00146484375	\\
0.839702112074228	-0.000946044921875	\\
0.839824197289708	-0.0008544921875	\\
0.839946282505189	-0.00067138671875	\\
0.840068367720669	6.103515625e-05	\\
0.840190452936149	9.1552734375e-05	\\
0.84031253815163	-3.0517578125e-05	\\
0.84043462336711	-0.00067138671875	\\
0.840556708582591	-0.0010986328125	\\
0.840678793798071	-0.000885009765625	\\
0.840800879013551	-0.001190185546875	\\
0.840922964229032	-0.001220703125	\\
0.841045049444512	-0.001556396484375	\\
0.841167134659993	-0.001556396484375	\\
0.841289219875473	-0.001251220703125	\\
0.841411305090954	-0.00103759765625	\\
0.841533390306434	-0.001068115234375	\\
0.841655475521914	-0.00048828125	\\
0.841777560737395	0.000335693359375	\\
0.841899645952875	0.00042724609375	\\
0.842021731168356	0.000457763671875	\\
0.842143816383836	0.0003662109375	\\
0.842265901599316	0.0001220703125	\\
0.842387986814797	-9.1552734375e-05	\\
0.842510072030277	-9.1552734375e-05	\\
0.842632157245758	0.0001220703125	\\
0.842754242461238	-0.00018310546875	\\
0.842876327676718	-0.000335693359375	\\
0.842998412892199	-0.000274658203125	\\
0.843120498107679	-0.000457763671875	\\
0.84324258332316	-0.000732421875	\\
0.84336466853864	-0.00054931640625	\\
0.84348675375412	-0.000640869140625	\\
0.843608838969601	-0.001129150390625	\\
0.843730924185081	-0.00054931640625	\\
0.843853009400562	0	\\
0.843975094616042	0.0001220703125	\\
0.844097179831522	0.0001220703125	\\
0.844219265047003	-0.000244140625	\\
0.844341350262483	-0.00054931640625	\\
0.844463435477964	-0.000762939453125	\\
0.844585520693444	-0.001068115234375	\\
0.844707605908924	-0.001190185546875	\\
0.844829691124405	-0.001678466796875	\\
0.844951776339885	-0.00213623046875	\\
0.845073861555366	-0.00201416015625	\\
0.845195946770846	-0.001861572265625	\\
0.845318031986326	-0.0015869140625	\\
0.845440117201807	-0.00164794921875	\\
0.845562202417287	-0.00152587890625	\\
0.845684287632768	-0.0010986328125	\\
0.845806372848248	-0.00091552734375	\\
0.845928458063729	-0.00079345703125	\\
0.846050543279209	-0.000885009765625	\\
0.846172628494689	-0.001007080078125	\\
0.84629471371017	-0.001068115234375	\\
0.84641679892565	-0.000885009765625	\\
0.846538884141131	-0.001251220703125	\\
0.846660969356611	-0.001190185546875	\\
0.846783054572091	-0.000946044921875	\\
0.846905139787572	-0.001312255859375	\\
0.847027225003052	-0.00177001953125	\\
0.847149310218533	-0.001983642578125	\\
0.847271395434013	-0.001556396484375	\\
0.847393480649493	-0.00146484375	\\
0.847515565864974	-0.001434326171875	\\
0.847637651080454	-0.0013427734375	\\
0.847759736295935	-0.001434326171875	\\
0.847881821511415	-0.0013427734375	\\
0.848003906726895	-0.000885009765625	\\
0.848125991942376	-0.0009765625	\\
0.848248077157856	-0.0013427734375	\\
0.848370162373337	-0.00091552734375	\\
0.848492247588817	-0.00103759765625	\\
0.848614332804297	-0.001251220703125	\\
0.848736418019778	-0.00140380859375	\\
0.848858503235258	-0.001556396484375	\\
0.848980588450739	-0.0018310546875	\\
0.849102673666219	-0.001800537109375	\\
0.849224758881699	-0.001129150390625	\\
0.84934684409718	-0.001068115234375	\\
0.84946892931266	-0.001312255859375	\\
0.849591014528141	-0.000762939453125	\\
0.849713099743621	-0.0006103515625	\\
0.849835184959101	-0.000457763671875	\\
0.849957270174582	-0.00042724609375	\\
0.850079355390062	-0.00030517578125	\\
0.850201440605543	-0.00042724609375	\\
0.850323525821023	-0.000823974609375	\\
0.850445611036504	-0.000518798828125	\\
0.850567696251984	0.0001220703125	\\
0.850689781467464	0.000518798828125	\\
0.850811866682945	-0.00018310546875	\\
0.850933951898425	-0.000274658203125	\\
0.851056037113906	0	\\
0.851178122329386	-9.1552734375e-05	\\
0.851300207544866	0.0001220703125	\\
0.851422292760347	0.0003662109375	\\
0.851544377975827	0.000518798828125	\\
0.851666463191308	0.000274658203125	\\
0.851788548406788	0.00030517578125	\\
0.851910633622268	6.103515625e-05	\\
0.852032718837749	0	\\
0.852154804053229	9.1552734375e-05	\\
0.85227688926871	-9.1552734375e-05	\\
0.85239897448419	-0.0003662109375	\\
0.85252105969967	-0.000396728515625	\\
0.852643144915151	-0.0006103515625	\\
0.852765230130631	-0.00091552734375	\\
0.852887315346112	-0.0009765625	\\
0.853009400561592	-0.00054931640625	\\
0.853131485777072	-0.00030517578125	\\
0.853253570992553	-0.0003662109375	\\
0.853375656208033	-0.00030517578125	\\
0.853497741423514	3.0517578125e-05	\\
0.853619826638994	0.0001220703125	\\
0.853741911854474	3.0517578125e-05	\\
0.853863997069955	0	\\
0.853986082285435	-0.000152587890625	\\
0.854108167500916	-0.000152587890625	\\
0.854230252716396	-0.00079345703125	\\
0.854352337931876	-0.00091552734375	\\
0.854474423147357	-0.000946044921875	\\
0.854596508362837	-0.00128173828125	\\
0.854718593578318	-0.001495361328125	\\
0.854840678793798	-0.0013427734375	\\
0.854962764009279	-0.000885009765625	\\
0.855084849224759	-0.00067138671875	\\
0.855206934440239	-0.000823974609375	\\
0.85532901965572	-0.000701904296875	\\
0.8554511048712	-0.0003662109375	\\
0.855573190086681	0.000152587890625	\\
0.855695275302161	0.000213623046875	\\
0.855817360517641	-0.0001220703125	\\
0.855939445733122	-0.0001220703125	\\
0.856061530948602	-0.00018310546875	\\
0.856183616164083	-0.0006103515625	\\
0.856305701379563	-0.000946044921875	\\
0.856427786595043	-0.000885009765625	\\
0.856549871810524	-0.00067138671875	\\
0.856671957026004	-0.000640869140625	\\
0.856794042241485	-0.00091552734375	\\
0.856916127456965	-0.001068115234375	\\
0.857038212672445	-0.00054931640625	\\
0.857160297887926	0.000396728515625	\\
0.857282383103406	0.001068115234375	\\
0.857404468318887	0.0009765625	\\
0.857526553534367	0.000946044921875	\\
0.857648638749847	0.001190185546875	\\
0.857770723965328	0.001190185546875	\\
0.857892809180808	0.001129150390625	\\
0.858014894396289	0.001251220703125	\\
0.858136979611769	0.001312255859375	\\
0.858259064827249	0.001129150390625	\\
0.85838115004273	0.001556396484375	\\
0.85850323525821	0.001373291015625	\\
0.858625320473691	0.001129150390625	\\
0.858747405689171	0.001190185546875	\\
0.858869490904651	0.00128173828125	\\
0.858991576120132	0.001373291015625	\\
0.859113661335612	0.001678466796875	\\
0.859235746551093	0.00189208984375	\\
0.859357831766573	0.001800537109375	\\
0.859479916982054	0.002044677734375	\\
0.859602002197534	0.002227783203125	\\
0.859724087413014	0.002166748046875	\\
0.859846172628495	0.00238037109375	\\
0.859968257843975	0.002105712890625	\\
0.860090343059456	0.001800537109375	\\
0.860212428274936	0.00177001953125	\\
0.860334513490416	0.001983642578125	\\
0.860456598705897	0.002227783203125	\\
0.860578683921377	0.001922607421875	\\
0.860700769136858	0.0018310546875	\\
0.860822854352338	0.001953125	\\
0.860944939567818	0.002044677734375	\\
0.861067024783299	0.001922607421875	\\
0.861189109998779	0.0018310546875	\\
0.86131119521426	0.002166748046875	\\
0.86143328042974	0.001983642578125	\\
0.86155536564522	0.002288818359375	\\
0.861677450860701	0.00225830078125	\\
0.861799536076181	0.001861572265625	\\
0.861921621291662	0.001922607421875	\\
0.862043706507142	0.001922607421875	\\
0.862165791722622	0.00189208984375	\\
0.862287876938103	0.0018310546875	\\
0.862409962153583	0.001556396484375	\\
0.862532047369064	0.001678466796875	\\
0.862654132584544	0.00164794921875	\\
0.862776217800024	0.00146484375	\\
0.862898303015505	0.001861572265625	\\
0.863020388230985	0.0020751953125	\\
0.863142473446466	0.001953125	\\
0.863264558661946	0.00201416015625	\\
0.863386643877426	0.0020751953125	\\
0.863508729092907	0.0020751953125	\\
0.863630814308387	0.00238037109375	\\
0.863752899523868	0.00244140625	\\
0.863874984739348	0.0023193359375	\\
0.863997069954829	0.001861572265625	\\
0.864119155170309	0.001251220703125	\\
0.864241240385789	0.000946044921875	\\
0.86436332560127	0.00054931640625	\\
0.86448541081675	0.00079345703125	\\
0.864607496032231	0.0013427734375	\\
0.864729581247711	0.001617431640625	\\
0.864851666463191	0.0020751953125	\\
0.864973751678672	0.00274658203125	\\
0.865095836894152	0.003021240234375	\\
0.865217922109633	0.003082275390625	\\
0.865340007325113	0.00341796875	\\
0.865462092540593	0.003997802734375	\\
0.865584177756074	0.0037841796875	\\
0.865706262971554	0.003509521484375	\\
0.865828348187035	0.003265380859375	\\
0.865950433402515	0.00299072265625	\\
0.866072518617995	0.002471923828125	\\
0.866194603833476	0.0023193359375	\\
0.866316689048956	0.002166748046875	\\
0.866438774264437	0.001617431640625	\\
0.866560859479917	0.00164794921875	\\
0.866682944695397	0.00164794921875	\\
0.866805029910878	0.00146484375	\\
0.866927115126358	0.00177001953125	\\
0.867049200341839	0.00225830078125	\\
0.867171285557319	0.002410888671875	\\
0.867293370772799	0.00262451171875	\\
0.86741545598828	0.0030517578125	\\
0.86753754120376	0.0032958984375	\\
0.867659626419241	0.002960205078125	\\
0.867781711634721	0.002777099609375	\\
0.867903796850201	0.00250244140625	\\
0.868025882065682	0.00250244140625	\\
0.868147967281162	0.002227783203125	\\
0.868270052496643	0.002349853515625	\\
0.868392137712123	0.00213623046875	\\
0.868514222927604	0.00177001953125	\\
0.868636308143084	0.001556396484375	\\
0.868758393358564	0.002197265625	\\
0.868880478574045	0.0029296875	\\
0.869002563789525	0.00244140625	\\
0.869124649005006	0.002227783203125	\\
0.869246734220486	0.002288818359375	\\
0.869368819435966	0.00201416015625	\\
0.869490904651447	0.001922607421875	\\
0.869612989866927	0.0018310546875	\\
0.869735075082408	0.0015869140625	\\
0.869857160297888	0.001556396484375	\\
0.869979245513368	0.00128173828125	\\
0.870101330728849	0.001312255859375	\\
0.870223415944329	0.001556396484375	\\
0.87034550115981	0.001190185546875	\\
0.87046758637529	0.000640869140625	\\
0.87058967159077	0.000885009765625	\\
0.870711756806251	0.001068115234375	\\
0.870833842021731	0.001129150390625	\\
0.870955927237212	0.0010986328125	\\
0.871078012452692	0.001007080078125	\\
0.871200097668172	0.001129150390625	\\
0.871322182883653	0.001434326171875	\\
0.871444268099133	0.0018310546875	\\
0.871566353314614	0.00164794921875	\\
0.871688438530094	0.0013427734375	\\
0.871810523745574	0.001373291015625	\\
0.871932608961055	0.001312255859375	\\
0.872054694176535	0.001434326171875	\\
0.872176779392016	0.001312255859375	\\
0.872298864607496	0.00091552734375	\\
0.872420949822976	0.00079345703125	\\
0.872543035038457	0.000946044921875	\\
0.872665120253937	0.001373291015625	\\
0.872787205469418	0.001220703125	\\
0.872909290684898	0.000885009765625	\\
0.873031375900378	0.00128173828125	\\
0.873153461115859	0.001251220703125	\\
0.873275546331339	0.0013427734375	\\
0.87339763154682	0.00140380859375	\\
0.8735197167623	0.001708984375	\\
0.873641801977781	0.001678466796875	\\
0.873763887193261	0.0015869140625	\\
0.873885972408741	0.0015869140625	\\
0.874008057624222	0.001434326171875	\\
0.874130142839702	0.001617431640625	\\
0.874252228055183	0.00152587890625	\\
0.874374313270663	0.00140380859375	\\
0.874496398486143	0.00164794921875	\\
0.874618483701624	0.00189208984375	\\
0.874740568917104	0.00244140625	\\
0.874862654132585	0.002471923828125	\\
0.874984739348065	0.002288818359375	\\
0.875106824563545	0.00244140625	\\
0.875228909779026	0.002685546875	\\
0.875350994994506	0.0025634765625	\\
0.875473080209987	0.002655029296875	\\
0.875595165425467	0.00274658203125	\\
0.875717250640947	0.00244140625	\\
0.875839335856428	0.00213623046875	\\
0.875961421071908	0.0020751953125	\\
0.876083506287389	0.0020751953125	\\
0.876205591502869	0.00244140625	\\
0.876327676718349	0.002105712890625	\\
0.87644976193383	0.001617431640625	\\
0.87657184714931	0.0020751953125	\\
0.876693932364791	0.002410888671875	\\
0.876816017580271	0.002532958984375	\\
0.876938102795751	0.002227783203125	\\
0.877060188011232	0.002197265625	\\
0.877182273226712	0.00201416015625	\\
0.877304358442193	0.00152587890625	\\
0.877426443657673	0.00164794921875	\\
0.877548528873153	0.00128173828125	\\
0.877670614088634	0.0010986328125	\\
0.877792699304114	0.001678466796875	\\
0.877914784519595	0.001678466796875	\\
0.878036869735075	0.001495361328125	\\
0.878158954950556	0.001800537109375	\\
0.878281040166036	0.00177001953125	\\
0.878403125381516	0.001190185546875	\\
0.878525210596997	0.0010986328125	\\
0.878647295812477	0.001617431640625	\\
0.878769381027958	0.00177001953125	\\
0.878891466243438	0.001678466796875	\\
0.879013551458918	0.00177001953125	\\
0.879135636674399	0.00146484375	\\
0.879257721889879	0.001617431640625	\\
0.87937980710536	0.001800537109375	\\
0.87950189232084	0.001434326171875	\\
0.87962397753632	0.001007080078125	\\
0.879746062751801	0.00103759765625	\\
0.879868147967281	0.0009765625	\\
0.879990233182762	0.00079345703125	\\
0.880112318398242	0.00079345703125	\\
0.880234403613722	0.0010986328125	\\
0.880356488829203	0.001190185546875	\\
0.880478574044683	0.001220703125	\\
0.880600659260164	0.00152587890625	\\
0.880722744475644	0.001678466796875	\\
0.880844829691124	0.00201416015625	\\
0.880966914906605	0.001922607421875	\\
0.881089000122085	0.002288818359375	\\
0.881211085337566	0.00238037109375	\\
0.881333170553046	0.001922607421875	\\
0.881455255768526	0.002227783203125	\\
0.881577340984007	0.002197265625	\\
0.881699426199487	0.001953125	\\
0.881821511414968	0.002197265625	\\
0.881943596630448	0.002227783203125	\\
0.882065681845928	0.0018310546875	\\
0.882187767061409	0.0018310546875	\\
0.882309852276889	0.002197265625	\\
0.88243193749237	0.002471923828125	\\
0.88255402270785	0.00262451171875	\\
0.882676107923331	0.0029296875	\\
0.882798193138811	0.003143310546875	\\
0.882920278354291	0.00311279296875	\\
0.883042363569772	0.003662109375	\\
0.883164448785252	0.0035400390625	\\
0.883286534000733	0.003448486328125	\\
0.883408619216213	0.003692626953125	\\
0.883530704431693	0.00347900390625	\\
0.883652789647174	0.003448486328125	\\
0.883774874862654	0.003448486328125	\\
0.883896960078135	0.003204345703125	\\
0.884019045293615	0.003082275390625	\\
0.884141130509095	0.003021240234375	\\
0.884263215724576	0.003082275390625	\\
0.884385300940056	0.00323486328125	\\
0.884507386155537	0.003753662109375	\\
0.884629471371017	0.0040283203125	\\
0.884751556586497	0.004119873046875	\\
0.884873641801978	0.004486083984375	\\
0.884995727017458	0.004425048828125	\\
0.885117812232939	0.004608154296875	\\
0.885239897448419	0.004791259765625	\\
0.885361982663899	0.004669189453125	\\
0.88548406787938	0.004547119140625	\\
0.88560615309486	0.00421142578125	\\
0.885728238310341	0.004150390625	\\
0.885850323525821	0.0040283203125	\\
0.885972408741301	0.00347900390625	\\
0.886094493956782	0.0032958984375	\\
0.886216579172262	0.00311279296875	\\
0.886338664387743	0.00323486328125	\\
0.886460749603223	0.00323486328125	\\
0.886582834818703	0.00299072265625	\\
0.886704920034184	0.002777099609375	\\
0.886827005249664	0.00311279296875	\\
0.886949090465145	0.0029296875	\\
0.887071175680625	0.0029296875	\\
0.887193260896106	0.00335693359375	\\
0.887315346111586	0.00286865234375	\\
0.887437431327066	0.002838134765625	\\
0.887559516542547	0.002777099609375	\\
0.887681601758027	0.002349853515625	\\
0.887803686973508	0.001983642578125	\\
0.887925772188988	0.002044677734375	\\
0.888047857404468	0.002044677734375	\\
0.888169942619949	0.00177001953125	\\
0.888292027835429	0.001678466796875	\\
0.88841411305091	0.001678466796875	\\
0.88853619826639	0.001861572265625	\\
0.88865828348187	0.0023193359375	\\
0.888780368697351	0.00238037109375	\\
0.888902453912831	0.0020751953125	\\
0.889024539128312	0.001617431640625	\\
0.889146624343792	0.001434326171875	\\
0.889268709559272	0.001373291015625	\\
0.889390794774753	0.001251220703125	\\
0.889512879990233	0.0013427734375	\\
0.889634965205714	0.00140380859375	\\
0.889757050421194	0.001556396484375	\\
0.889879135636674	0.001739501953125	\\
0.890001220852155	0.001800537109375	\\
0.890123306067635	0.00189208984375	\\
0.890245391283116	0.002227783203125	\\
0.890367476498596	0.00244140625	\\
0.890489561714076	0.00238037109375	\\
0.890611646929557	0.002593994140625	\\
0.890733732145037	0.0025634765625	\\
0.890855817360518	0.00250244140625	\\
0.890977902575998	0.00250244140625	\\
0.891099987791478	0.00244140625	\\
0.891222073006959	0.002349853515625	\\
0.891344158222439	0.00213623046875	\\
0.89146624343792	0.001617431640625	\\
0.8915883286534	0.00152587890625	\\
0.891710413868881	0.001708984375	\\
0.891832499084361	0.001739501953125	\\
0.891954584299841	0.001739501953125	\\
0.892076669515322	0.001678466796875	\\
0.892198754730802	0.00152587890625	\\
0.892320839946283	0.0015869140625	\\
0.892442925161763	0.00146484375	\\
0.892565010377243	0.00146484375	\\
0.892687095592724	0.001708984375	\\
0.892809180808204	0.001190185546875	\\
0.892931266023685	0.0013427734375	\\
0.893053351239165	0.001678466796875	\\
0.893175436454645	0.00146484375	\\
0.893297521670126	0.001129150390625	\\
0.893419606885606	0.001373291015625	\\
0.893541692101087	0.001251220703125	\\
0.893663777316567	0.000701904296875	\\
0.893785862532047	0.000457763671875	\\
0.893907947747528	0.00054931640625	\\
0.894030032963008	0.00079345703125	\\
0.894152118178489	0.00079345703125	\\
0.894274203393969	0.00079345703125	\\
0.894396288609449	0.001220703125	\\
0.89451837382493	0.001190185546875	\\
0.89464045904041	0.001373291015625	\\
0.894762544255891	0.001953125	\\
0.894884629471371	0.001983642578125	\\
0.895006714686851	0.001953125	\\
0.895128799902332	0.002197265625	\\
0.895250885117812	0.001708984375	\\
0.895372970333293	0.0013427734375	\\
0.895495055548773	0.001129150390625	\\
0.895617140764253	0.001220703125	\\
0.895739225979734	0.00103759765625	\\
0.895861311195214	0.00054931640625	\\
0.895983396410695	0.0003662109375	\\
0.896105481626175	0.00048828125	\\
0.896227566841656	0.000518798828125	\\
0.896349652057136	0.000762939453125	\\
0.896471737272616	0.000701904296875	\\
0.896593822488097	0.000732421875	\\
0.896715907703577	0.000885009765625	\\
0.896837992919058	0.000701904296875	\\
0.896960078134538	0.00103759765625	\\
0.897082163350018	0.00091552734375	\\
0.897204248565499	0.000579833984375	\\
0.897326333780979	0.000457763671875	\\
0.89744841899646	0.000396728515625	\\
0.89757050421194	0.000396728515625	\\
0.89769258942742	0.00048828125	\\
0.897814674642901	0.000701904296875	\\
0.897936759858381	0.000640869140625	\\
0.898058845073862	0.000701904296875	\\
0.898180930289342	0.001129150390625	\\
0.898303015504822	0.00152587890625	\\
0.898425100720303	0.00189208984375	\\
0.898547185935783	0.001983642578125	\\
0.898669271151264	0.002197265625	\\
0.898791356366744	0.002044677734375	\\
0.898913441582224	0.00213623046875	\\
0.899035526797705	0.002288818359375	\\
0.899157612013185	0.00177001953125	\\
0.899279697228666	0.00177001953125	\\
0.899401782444146	0.0018310546875	\\
0.899523867659626	0.001800537109375	\\
0.899645952875107	0.001556396484375	\\
0.899768038090587	0.001190185546875	\\
0.899890123306068	0.000823974609375	\\
0.900012208521548	0.000946044921875	\\
0.900134293737028	0.001007080078125	\\
0.900256378952509	0.000946044921875	\\
0.900378464167989	0.00128173828125	\\
0.90050054938347	0.001617431640625	\\
0.90062263459895	0.001678466796875	\\
0.90074471981443	0.001739501953125	\\
0.900866805029911	0.00146484375	\\
0.900988890245391	0.00115966796875	\\
0.901110975460872	0.00103759765625	\\
0.901233060676352	0.0006103515625	\\
0.901355145891833	0.00048828125	\\
0.901477231107313	0.00048828125	\\
0.901599316322793	0.00048828125	\\
0.901721401538274	0.00054931640625	\\
0.901843486753754	0.0006103515625	\\
0.901965571969235	0.000762939453125	\\
0.902087657184715	0.000823974609375	\\
0.902209742400195	0.000946044921875	\\
0.902331827615676	0.000823974609375	\\
0.902453912831156	0.001251220703125	\\
0.902575998046637	0.001678466796875	\\
0.902698083262117	0.00164794921875	\\
0.902820168477597	0.00140380859375	\\
0.902942253693078	0.001129150390625	\\
0.903064338908558	0.000518798828125	\\
0.903186424124039	0.000213623046875	\\
0.903308509339519	0.0001220703125	\\
0.903430594554999	0	\\
0.90355267977048	9.1552734375e-05	\\
0.90367476498596	0.00079345703125	\\
0.903796850201441	0.001068115234375	\\
0.903918935416921	0.000640869140625	\\
0.904041020632401	0.00067138671875	\\
0.904163105847882	0.001190185546875	\\
0.904285191063362	0.0013427734375	\\
0.904407276278843	0.001434326171875	\\
0.904529361494323	0.001678466796875	\\
0.904651446709803	0.001556396484375	\\
0.904773531925284	0.001495361328125	\\
0.904895617140764	0.001617431640625	\\
0.905017702356245	0.00146484375	\\
0.905139787571725	0.001190185546875	\\
0.905261872787205	0.001190185546875	\\
0.905383958002686	0.001190185546875	\\
0.905506043218166	0.00146484375	\\
0.905628128433647	0.001495361328125	\\
0.905750213649127	0.001068115234375	\\
0.905872298864608	0.0015869140625	\\
0.905994384080088	0.002593994140625	\\
0.906116469295568	0.00262451171875	\\
0.906238554511049	0.0023193359375	\\
0.906360639726529	0.002532958984375	\\
0.90648272494201	0.002471923828125	\\
0.90660481015749	0.002532958984375	\\
0.90672689537297	0.002197265625	\\
0.906848980588451	0.0020751953125	\\
0.906971065803931	0.001922607421875	\\
0.907093151019412	0.0013427734375	\\
0.907215236234892	0.001434326171875	\\
0.907337321450372	0.001495361328125	\\
0.907459406665853	0.001129150390625	\\
0.907581491881333	0.00152587890625	\\
0.907703577096814	0.001861572265625	\\
0.907825662312294	0.00201416015625	\\
0.907947747527774	0.002105712890625	\\
0.908069832743255	0.00213623046875	\\
0.908191917958735	0.002410888671875	\\
0.908314003174216	0.00274658203125	\\
0.908436088389696	0.00262451171875	\\
0.908558173605176	0.002655029296875	\\
0.908680258820657	0.00262451171875	\\
0.908802344036137	0.002288818359375	\\
0.908924429251618	0.001708984375	\\
0.909046514467098	0.001800537109375	\\
0.909168599682578	0.00189208984375	\\
0.909290684898059	0.001617431640625	\\
0.909412770113539	0.001495361328125	\\
0.90953485532902	0.00128173828125	\\
0.9096569405445	0.001373291015625	\\
0.90977902575998	0.002105712890625	\\
0.909901110975461	0.002288818359375	\\
0.910023196190941	0.001922607421875	\\
0.910145281406422	0.00189208984375	\\
0.910267366621902	0.002227783203125	\\
0.910389451837383	0.002288818359375	\\
0.910511537052863	0.0020751953125	\\
0.910633622268343	0.001678466796875	\\
0.910755707483824	0.001495361328125	\\
0.910877792699304	0.001129150390625	\\
0.910999877914785	0.0006103515625	\\
0.911121963130265	0.000885009765625	\\
0.911244048345745	0.000885009765625	\\
0.911366133561226	0.00079345703125	\\
0.911488218776706	0.00103759765625	\\
0.911610303992187	0.000701904296875	\\
0.911732389207667	0.000152587890625	\\
0.911854474423147	0.00042724609375	\\
0.911976559638628	0.000701904296875	\\
0.912098644854108	0.000396728515625	\\
0.912220730069589	0.000457763671875	\\
0.912342815285069	0.000244140625	\\
0.912464900500549	0	\\
0.91258698571603	-0.0001220703125	\\
0.91270907093151	-0.000244140625	\\
0.912831156146991	-0.000213623046875	\\
0.912953241362471	-0.00030517578125	\\
0.913075326577951	-0.000518798828125	\\
0.913197411793432	-0.0010986328125	\\
0.913319497008912	-0.001068115234375	\\
0.913441582224393	-0.00079345703125	\\
0.913563667439873	-0.00103759765625	\\
0.913685752655353	-0.00048828125	\\
0.913807837870834	-0.0001220703125	\\
0.913929923086314	-0.000274658203125	\\
0.914052008301795	6.103515625e-05	\\
0.914174093517275	0.00048828125	\\
0.914296178732755	0.000701904296875	\\
0.914418263948236	0.0003662109375	\\
0.914540349163716	0.00054931640625	\\
0.914662434379197	0.00048828125	\\
0.914784519594677	0.000244140625	\\
0.914906604810158	0	\\
0.915028690025638	-0.00042724609375	\\
0.915150775241118	-0.000518798828125	\\
0.915272860456599	-0.000518798828125	\\
0.915394945672079	-0.000396728515625	\\
0.91551703088756	-0.00030517578125	\\
0.91563911610304	0.0001220703125	\\
0.91576120131852	0.000518798828125	\\
0.915883286534001	0.000457763671875	\\
0.916005371749481	6.103515625e-05	\\
0.916127456964962	0.00030517578125	\\
0.916249542180442	0.000457763671875	\\
0.916371627395922	0.000274658203125	\\
0.916493712611403	0.000274658203125	\\
0.916615797826883	-0.0001220703125	\\
0.916737883042364	-0.000213623046875	\\
0.916859968257844	3.0517578125e-05	\\
0.916982053473324	-0.00048828125	\\
0.917104138688805	-0.000335693359375	\\
0.917226223904285	0.000244140625	\\
0.917348309119766	0.00018310546875	\\
0.917470394335246	0	\\
0.917592479550726	0.00030517578125	\\
0.917714564766207	0.00054931640625	\\
0.917836649981687	0.000457763671875	\\
0.917958735197168	0.000457763671875	\\
0.918080820412648	0.000701904296875	\\
0.918202905628128	0.00091552734375	\\
0.918324990843609	0.00067138671875	\\
0.918447076059089	0.00054931640625	\\
0.91856916127457	3.0517578125e-05	\\
0.91869124649005	-0.0001220703125	\\
0.91881333170553	0	\\
0.918935416921011	3.0517578125e-05	\\
0.919057502136491	0.00018310546875	\\
0.919179587351972	-0.000213623046875	\\
0.919301672567452	-0.000518798828125	\\
0.919423757782933	-6.103515625e-05	\\
0.919545842998413	6.103515625e-05	\\
0.919667928213893	3.0517578125e-05	\\
0.919790013429374	0.00048828125	\\
0.919912098644854	0.0006103515625	\\
0.920034183860335	0.000457763671875	\\
0.920156269075815	0.00030517578125	\\
0.920278354291295	0	\\
0.920400439506776	0	\\
0.920522524722256	0.000152587890625	\\
0.920644609937737	-3.0517578125e-05	\\
0.920766695153217	-6.103515625e-05	\\
0.920888780368697	3.0517578125e-05	\\
0.921010865584178	-0.00018310546875	\\
0.921132950799658	-0.00030517578125	\\
0.921255036015139	6.103515625e-05	\\
0.921377121230619	-9.1552734375e-05	\\
0.921499206446099	0.000213623046875	\\
0.92162129166158	0.0003662109375	\\
0.92174337687706	0.000274658203125	\\
0.921865462092541	0.0003662109375	\\
0.921987547308021	0.000823974609375	\\
0.922109632523501	0.000732421875	\\
0.922231717738982	0.00048828125	\\
0.922353802954462	0.001068115234375	\\
0.922475888169943	0.001007080078125	\\
0.922597973385423	0.001068115234375	\\
0.922720058600903	0.001251220703125	\\
0.922842143816384	0.001007080078125	\\
0.922964229031864	0.0006103515625	\\
0.923086314247345	0.000518798828125	\\
0.923208399462825	0.00048828125	\\
0.923330484678305	0.00079345703125	\\
0.923452569893786	0.00152587890625	\\
0.923574655109266	0.002044677734375	\\
0.923696740324747	0.002044677734375	\\
0.923818825540227	0.002349853515625	\\
0.923940910755708	0.00286865234375	\\
0.924062995971188	0.0029296875	\\
0.924185081186668	0.002593994140625	\\
0.924307166402149	0.00262451171875	\\
0.924429251617629	0.00274658203125	\\
0.92455133683311	0.0025634765625	\\
0.92467342204859	0.002471923828125	\\
0.92479550726407	0.00244140625	\\
0.924917592479551	0.002716064453125	\\
0.925039677695031	0.002593994140625	\\
0.925161762910512	0.0023193359375	\\
0.925283848125992	0.001953125	\\
0.925405933341472	0.00225830078125	\\
0.925528018556953	0.00244140625	\\
0.925650103772433	0.002349853515625	\\
0.925772188987914	0.002288818359375	\\
0.925894274203394	0.0018310546875	\\
0.926016359418874	0.0018310546875	\\
0.926138444634355	0.001678466796875	\\
0.926260529849835	0.0018310546875	\\
0.926382615065316	0.001953125	\\
0.926504700280796	0.001739501953125	\\
0.926626785496276	0.00146484375	\\
0.926748870711757	0.00146484375	\\
0.926870955927237	0.001434326171875	\\
0.926993041142718	0.001373291015625	\\
0.927115126358198	0.00146484375	\\
0.927237211573678	0.00140380859375	\\
0.927359296789159	0.001068115234375	\\
0.927481382004639	0.001373291015625	\\
0.92760346722012	0.001739501953125	\\
0.9277255524356	0.001739501953125	\\
0.92784763765108	0.00201416015625	\\
0.927969722866561	0.001678466796875	\\
0.928091808082041	0.001434326171875	\\
0.928213893297522	0.0009765625	\\
0.928335978513002	0.000885009765625	\\
0.928458063728483	0.001312255859375	\\
0.928580148943963	0.001220703125	\\
0.928702234159443	0.00128173828125	\\
0.928824319374924	0.001312255859375	\\
0.928946404590404	0.00103759765625	\\
0.929068489805885	0.000946044921875	\\
0.929190575021365	0.0009765625	\\
0.929312660236845	0.000762939453125	\\
0.929434745452326	0.000885009765625	\\
0.929556830667806	0.00128173828125	\\
0.929678915883287	0.001434326171875	\\
0.929801001098767	0.00189208984375	\\
0.929923086314247	0.00213623046875	\\
0.930045171529728	0.001708984375	\\
0.930167256745208	0.001434326171875	\\
0.930289341960689	0.00177001953125	\\
0.930411427176169	0.0013427734375	\\
0.930533512391649	0.0009765625	\\
0.93065559760713	0.001129150390625	\\
0.93077768282261	0.000823974609375	\\
0.930899768038091	0.00103759765625	\\
0.931021853253571	0.00128173828125	\\
0.931143938469051	0.001007080078125	\\
0.931266023684532	0.0010986328125	\\
0.931388108900012	0.0009765625	\\
0.931510194115493	0.001007080078125	\\
0.931632279330973	0.001220703125	\\
0.931754364546453	0.00140380859375	\\
0.931876449761934	0.001373291015625	\\
0.931998534977414	0.0013427734375	\\
0.932120620192895	0.00146484375	\\
0.932242705408375	0.001373291015625	\\
0.932364790623855	0.00128173828125	\\
0.932486875839336	0.001220703125	\\
0.932608961054816	0.0009765625	\\
0.932731046270297	0.000518798828125	\\
0.932853131485777	0.0003662109375	\\
0.932975216701257	0.000335693359375	\\
0.933097301916738	0.00048828125	\\
0.933219387132218	0.00048828125	\\
0.933341472347699	0.000396728515625	\\
0.933463557563179	0.00091552734375	\\
0.93358564277866	0.00128173828125	\\
0.93370772799414	0.00128173828125	\\
0.93382981320962	0.00091552734375	\\
0.933951898425101	0.00067138671875	\\
0.934073983640581	0.000732421875	\\
0.934196068856062	0.00042724609375	\\
0.934318154071542	-3.0517578125e-05	\\
0.934440239287022	-0.000579833984375	\\
0.934562324502503	-0.00042724609375	\\
0.934684409717983	-0.0006103515625	\\
0.934806494933464	-0.0006103515625	\\
0.934928580148944	-0.000640869140625	\\
0.935050665364424	-0.000823974609375	\\
0.935172750579905	-0.000518798828125	\\
0.935294835795385	-0.0006103515625	\\
0.935416921010866	-0.000762939453125	\\
0.935539006226346	-0.001190185546875	\\
0.935661091441826	-0.00067138671875	\\
0.935783176657307	-0.000244140625	\\
0.935905261872787	-0.00103759765625	\\
0.936027347088268	-0.001129150390625	\\
0.936149432303748	-0.00115966796875	\\
0.936271517519228	-0.001373291015625	\\
0.936393602734709	-0.001190185546875	\\
0.936515687950189	-0.001251220703125	\\
0.93663777316567	-0.001861572265625	\\
0.93675985838115	-0.002349853515625	\\
0.93688194359663	-0.002410888671875	\\
0.937004028812111	-0.002349853515625	\\
0.937126114027591	-0.002410888671875	\\
0.937248199243072	-0.0020751953125	\\
0.937370284458552	-0.001312255859375	\\
0.937492369674032	-0.001220703125	\\
0.937614454889513	-0.000946044921875	\\
0.937736540104993	-0.001007080078125	\\
0.937858625320474	-0.001007080078125	\\
0.937980710535954	-0.00140380859375	\\
0.938102795751435	-0.001556396484375	\\
0.938224880966915	-0.001861572265625	\\
0.938346966182395	-0.00213623046875	\\
0.938469051397876	-0.00189208984375	\\
0.938591136613356	-0.001678466796875	\\
0.938713221828837	-0.001434326171875	\\
0.938835307044317	-0.001220703125	\\
0.938957392259797	-0.000701904296875	\\
0.939079477475278	-0.000762939453125	\\
0.939201562690758	-0.000946044921875	\\
0.939323647906239	-0.000701904296875	\\
0.939445733121719	-0.0006103515625	\\
0.939567818337199	-0.000335693359375	\\
0.93968990355268	3.0517578125e-05	\\
0.93981198876816	0.00030517578125	\\
0.939934073983641	6.103515625e-05	\\
0.940056159199121	-0.0001220703125	\\
0.940178244414601	0.00030517578125	\\
0.940300329630082	0.0001220703125	\\
0.940422414845562	-0.000457763671875	\\
0.940544500061043	-0.000152587890625	\\
0.940666585276523	0.000244140625	\\
0.940788670492003	0.000244140625	\\
0.940910755707484	0.0001220703125	\\
0.941032840922964	0.000396728515625	\\
0.941154926138445	0.000274658203125	\\
0.941277011353925	0.00030517578125	\\
0.941399096569405	0.00067138671875	\\
0.941521181784886	0.000640869140625	\\
0.941643267000366	0.0006103515625	\\
0.941765352215847	0.000518798828125	\\
0.941887437431327	0.000152587890625	\\
0.942009522646807	6.103515625e-05	\\
0.942131607862288	0.000213623046875	\\
0.942253693077768	0	\\
0.942375778293249	-0.000396728515625	\\
0.942497863508729	-0.000518798828125	\\
0.94261994872421	-0.000335693359375	\\
0.94274203393969	-0.000213623046875	\\
0.94286411915517	0.00018310546875	\\
0.942986204370651	0.000335693359375	\\
0.943108289586131	0.00018310546875	\\
0.943230374801612	0.000640869140625	\\
0.943352460017092	0.000823974609375	\\
0.943474545232572	0.000762939453125	\\
0.943596630448053	0.00091552734375	\\
0.943718715663533	0.000579833984375	\\
0.943840800879014	0.0003662109375	\\
0.943962886094494	0.000701904296875	\\
0.944084971309974	0.0006103515625	\\
0.944207056525455	0.00042724609375	\\
0.944329141740935	0.000732421875	\\
0.944451226956416	0.000732421875	\\
0.944573312171896	0.000732421875	\\
0.944695397387376	0.000244140625	\\
0.944817482602857	0.000213623046875	\\
0.944939567818337	0.000518798828125	\\
0.945061653033818	0.000518798828125	\\
0.945183738249298	0.000701904296875	\\
0.945305823464778	0.00054931640625	\\
0.945427908680259	0.000946044921875	\\
0.945549993895739	0.000885009765625	\\
0.94567207911122	0.00091552734375	\\
0.9457941643267	0.0009765625	\\
0.94591624954218	0.0006103515625	\\
0.946038334757661	0.00048828125	\\
0.946160419973141	0.000396728515625	\\
0.946282505188622	0.00030517578125	\\
0.946404590404102	0.000396728515625	\\
0.946526675619582	0.000701904296875	\\
0.946648760835063	0.000885009765625	\\
0.946770846050543	0.001129150390625	\\
0.946892931266024	0.000885009765625	\\
0.947015016481504	0.0010986328125	\\
0.947137101696985	0.00152587890625	\\
0.947259186912465	0.001983642578125	\\
0.947381272127945	0.0023193359375	\\
0.947503357343426	0.00225830078125	\\
0.947625442558906	0.00238037109375	\\
0.947747527774387	0.00201416015625	\\
0.947869612989867	0.00164794921875	\\
0.947991698205347	0.0013427734375	\\
0.948113783420828	0.001800537109375	\\
0.948235868636308	0.0023193359375	\\
0.948357953851789	0.001922607421875	\\
0.948480039067269	0.001434326171875	\\
0.948602124282749	0.00128173828125	\\
0.94872420949823	0.00146484375	\\
0.94884629471371	0.001617431640625	\\
0.948968379929191	0.0015869140625	\\
0.949090465144671	0.00244140625	\\
0.949212550360151	0.0028076171875	\\
0.949334635575632	0.002655029296875	\\
0.949456720791112	0.00262451171875	\\
0.949578806006593	0.00250244140625	\\
0.949700891222073	0.00238037109375	\\
0.949822976437553	0.001983642578125	\\
0.949945061653034	0.001556396484375	\\
0.950067146868514	0.001434326171875	\\
0.950189232083995	0.0010986328125	\\
0.950311317299475	0.00079345703125	\\
0.950433402514955	0.000640869140625	\\
0.950555487730436	0.000244140625	\\
0.950677572945916	0.000213623046875	\\
0.950799658161397	0.00048828125	\\
0.950921743376877	0.000152587890625	\\
0.951043828592357	3.0517578125e-05	\\
0.951165913807838	0.000213623046875	\\
0.951287999023318	-0.00018310546875	\\
0.951410084238799	-0.00030517578125	\\
0.951532169454279	-0.000274658203125	\\
0.95165425466976	-0.000152587890625	\\
0.95177633988524	-3.0517578125e-05	\\
0.95189842510072	-0.000335693359375	\\
0.952020510316201	-0.000732421875	\\
0.952142595531681	-0.001007080078125	\\
0.952264680747162	-0.001617431640625	\\
0.952386765962642	-0.001861572265625	\\
0.952508851178122	-0.001739501953125	\\
0.952630936393603	-0.00201416015625	\\
0.952753021609083	-0.002105712890625	\\
0.952875106824564	-0.001861572265625	\\
0.952997192040044	-0.001861572265625	\\
0.953119277255524	-0.001617431640625	\\
0.953241362471005	-0.001190185546875	\\
0.953363447686485	-0.00091552734375	\\
0.953485532901966	-0.000732421875	\\
0.953607618117446	-0.000762939453125	\\
0.953729703332926	-0.0009765625	\\
0.953851788548407	-0.001068115234375	\\
0.953973873763887	-0.001190185546875	\\
0.954095958979368	-0.00164794921875	\\
0.954218044194848	-0.001678466796875	\\
0.954340129410328	-0.001739501953125	\\
0.954462214625809	-0.001953125	\\
0.954584299841289	-0.00177001953125	\\
0.95470638505677	-0.00201416015625	\\
0.95482847027225	-0.001800537109375	\\
0.95495055548773	-0.00140380859375	\\
0.955072640703211	-0.0015869140625	\\
0.955194725918691	-0.00140380859375	\\
0.955316811134172	-0.00103759765625	\\
0.955438896349652	-0.001220703125	\\
0.955560981565132	-0.00140380859375	\\
0.955683066780613	-0.001708984375	\\
0.955805151996093	-0.002105712890625	\\
0.955927237211574	-0.002197265625	\\
0.956049322427054	-0.00225830078125	\\
0.956171407642535	-0.002288818359375	\\
0.956293492858015	-0.002105712890625	\\
0.956415578073495	-0.001983642578125	\\
0.956537663288976	-0.002471923828125	\\
0.956659748504456	-0.00213623046875	\\
0.956781833719937	-0.001800537109375	\\
0.956903918935417	-0.0018310546875	\\
0.957026004150897	-0.001953125	\\
0.957148089366378	-0.001922607421875	\\
0.957270174581858	-0.002044677734375	\\
0.957392259797339	-0.002593994140625	\\
0.957514345012819	-0.002960205078125	\\
0.957636430228299	-0.003021240234375	\\
0.95775851544378	-0.002960205078125	\\
0.95788060065926	-0.002777099609375	\\
0.958002685874741	-0.002838134765625	\\
0.958124771090221	-0.00323486328125	\\
0.958246856305701	-0.00311279296875	\\
0.958368941521182	-0.003326416015625	\\
0.958491026736662	-0.003631591796875	\\
0.958613111952143	-0.003509521484375	\\
0.958735197167623	-0.003570556640625	\\
0.958857282383103	-0.0035400390625	\\
0.958979367598584	-0.003509521484375	\\
0.959101452814064	-0.0035400390625	\\
0.959223538029545	-0.003448486328125	\\
0.959345623245025	-0.003387451171875	\\
0.959467708460505	-0.003509521484375	\\
0.959589793675986	-0.00396728515625	\\
0.959711878891466	-0.00433349609375	\\
0.959833964106947	-0.004241943359375	\\
0.959956049322427	-0.0042724609375	\\
0.960078134537907	-0.004364013671875	\\
0.960200219753388	-0.00445556640625	\\
0.960322304968868	-0.004364013671875	\\
0.960444390184349	-0.004180908203125	\\
0.960566475399829	-0.004486083984375	\\
0.960688560615309	-0.00469970703125	\\
0.96081064583079	-0.004791259765625	\\
0.96093273104627	-0.004638671875	\\
0.961054816261751	-0.004486083984375	\\
0.961176901477231	-0.004791259765625	\\
0.961298986692712	-0.0045166015625	\\
0.961421071908192	-0.00439453125	\\
0.961543157123672	-0.00445556640625	\\
0.961665242339153	-0.0045166015625	\\
0.961787327554633	-0.004638671875	\\
0.961909412770114	-0.004364013671875	\\
0.962031497985594	-0.004180908203125	\\
0.962153583201074	-0.004058837890625	\\
0.962275668416555	-0.003875732421875	\\
0.962397753632035	-0.003936767578125	\\
0.962519838847516	-0.003173828125	\\
0.962641924062996	-0.00286865234375	\\
0.962764009278476	-0.002960205078125	\\
0.962886094493957	-0.00286865234375	\\
0.963008179709437	-0.0028076171875	\\
0.963130264924918	-0.002410888671875	\\
0.963252350140398	-0.00244140625	\\
0.963374435355878	-0.002685546875	\\
0.963496520571359	-0.002471923828125	\\
0.963618605786839	-0.0023193359375	\\
0.96374069100232	-0.002655029296875	\\
0.9638627762178	-0.002899169921875	\\
0.96398486143328	-0.002899169921875	\\
0.964106946648761	-0.00262451171875	\\
0.964229031864241	-0.00262451171875	\\
0.964351117079722	-0.00250244140625	\\
0.964473202295202	-0.0015869140625	\\
0.964595287510682	-0.001556396484375	\\
0.964717372726163	-0.00189208984375	\\
0.964839457941643	-0.001861572265625	\\
0.964961543157124	-0.001556396484375	\\
0.965083628372604	-0.00140380859375	\\
0.965205713588084	-0.0015869140625	\\
0.965327798803565	-0.001251220703125	\\
0.965449884019045	-0.00115966796875	\\
0.965571969234526	-0.001495361328125	\\
0.965694054450006	-0.001800537109375	\\
0.965816139665487	-0.0020751953125	\\
0.965938224880967	-0.001739501953125	\\
0.966060310096447	-0.00189208984375	\\
0.966182395311928	-0.002227783203125	\\
0.966304480527408	-0.0020751953125	\\
0.966426565742889	-0.0020751953125	\\
0.966548650958369	-0.00250244140625	\\
0.966670736173849	-0.0020751953125	\\
0.96679282138933	-0.001373291015625	\\
0.96691490660481	-0.001373291015625	\\
0.967036991820291	-0.001678466796875	\\
0.967159077035771	-0.001983642578125	\\
0.967281162251251	-0.002197265625	\\
0.967403247466732	-0.0018310546875	\\
0.967525332682212	-0.0020751953125	\\
0.967647417897693	-0.001922607421875	\\
0.967769503113173	-0.00164794921875	\\
0.967891588328653	-0.001373291015625	\\
0.968013673544134	-0.001251220703125	\\
0.968135758759614	-0.0013427734375	\\
0.968257843975095	-0.001373291015625	\\
0.968379929190575	-0.0009765625	\\
0.968502014406055	-0.00042724609375	\\
0.968624099621536	-0.00067138671875	\\
0.968746184837016	-0.00067138671875	\\
0.968868270052497	-0.000457763671875	\\
0.968990355267977	-0.000244140625	\\
0.969112440483457	-0.000457763671875	\\
0.969234525698938	-0.000885009765625	\\
0.969356610914418	-0.0009765625	\\
0.969478696129899	-0.001129150390625	\\
0.969600781345379	-0.0013427734375	\\
0.969722866560859	-0.001434326171875	\\
0.96984495177634	-0.00152587890625	\\
0.96996703699182	-0.001617431640625	\\
0.970089122207301	-0.001678466796875	\\
0.970211207422781	-0.001373291015625	\\
0.970333292638262	-0.0010986328125	\\
0.970455377853742	-0.000946044921875	\\
0.970577463069222	-0.000518798828125	\\
0.970699548284703	-0.0006103515625	\\
0.970821633500183	-0.000732421875	\\
0.970943718715664	-0.000518798828125	\\
0.971065803931144	-0.0006103515625	\\
0.971187889146624	-0.000244140625	\\
0.971309974362105	-0.0003662109375	\\
0.971432059577585	-0.0003662109375	\\
0.971554144793066	-0.00067138671875	\\
0.971676230008546	-0.00115966796875	\\
0.971798315224026	-0.00115966796875	\\
0.971920400439507	-0.00115966796875	\\
0.972042485654987	-0.000701904296875	\\
0.972164570870468	-0.0006103515625	\\
0.972286656085948	-0.000823974609375	\\
0.972408741301428	-0.00030517578125	\\
0.972530826516909	0.00030517578125	\\
0.972652911732389	-3.0517578125e-05	\\
0.97277499694787	0.000152587890625	\\
0.97289708216335	0.00091552734375	\\
0.97301916737883	0.000518798828125	\\
0.973141252594311	0.00018310546875	\\
0.973263337809791	0.000335693359375	\\
0.973385423025272	0.000335693359375	\\
0.973507508240752	9.1552734375e-05	\\
0.973629593456232	-0.00030517578125	\\
0.973751678671713	-0.001007080078125	\\
0.973873763887193	-0.001251220703125	\\
0.973995849102674	-0.001251220703125	\\
0.974117934318154	-0.001556396484375	\\
0.974240019533634	-0.00115966796875	\\
0.974362104749115	-0.000885009765625	\\
0.974484189964595	-0.00054931640625	\\
0.974606275180076	-0.000457763671875	\\
0.974728360395556	-0.000274658203125	\\
0.974850445611037	-0.00042724609375	\\
0.974972530826517	-0.00067138671875	\\
0.975094616041997	-0.000274658203125	\\
0.975216701257478	-0.000946044921875	\\
0.975338786472958	-0.001190185546875	\\
0.975460871688439	-0.0009765625	\\
0.975582956903919	-0.00103759765625	\\
0.975705042119399	-0.00146484375	\\
0.97582712733488	-0.001708984375	\\
0.97594921255036	-0.00201416015625	\\
0.976071297765841	-0.001861572265625	\\
0.976193382981321	-0.001495361328125	\\
0.976315468196801	-0.00201416015625	\\
0.976437553412282	-0.001922607421875	\\
0.976559638627762	-0.00189208984375	\\
0.976681723843243	-0.002288818359375	\\
};
\addplot [color=blue,solid,forget plot]
  table[row sep=crcr]{
0.976681723843243	-0.002288818359375	\\
0.976803809058723	-0.001708984375	\\
0.976925894274203	-0.00189208984375	\\
0.977047979489684	-0.002227783203125	\\
0.977170064705164	-0.001953125	\\
0.977292149920645	-0.002227783203125	\\
0.977414235136125	-0.0029296875	\\
0.977536320351605	-0.003021240234375	\\
0.977658405567086	-0.003021240234375	\\
0.977780490782566	-0.003173828125	\\
0.977902575998047	-0.003448486328125	\\
0.978024661213527	-0.003204345703125	\\
0.978146746429007	-0.003143310546875	\\
0.978268831644488	-0.003173828125	\\
0.978390916859968	-0.00286865234375	\\
0.978513002075449	-0.003021240234375	\\
0.978635087290929	-0.002655029296875	\\
0.978757172506409	-0.002166748046875	\\
0.97887925772189	-0.002349853515625	\\
0.97900134293737	-0.002685546875	\\
0.979123428152851	-0.003204345703125	\\
0.979245513368331	-0.00341796875	\\
0.979367598583812	-0.003326416015625	\\
0.979489683799292	-0.003143310546875	\\
0.979611769014772	-0.0035400390625	\\
0.979733854230253	-0.00372314453125	\\
0.979855939445733	-0.00335693359375	\\
0.979978024661214	-0.00323486328125	\\
0.980100109876694	-0.00311279296875	\\
0.980222195092174	-0.003143310546875	\\
0.980344280307655	-0.002777099609375	\\
0.980466365523135	-0.002044677734375	\\
0.980588450738616	-0.00225830078125	\\
0.980710535954096	-0.001678466796875	\\
0.980832621169576	-0.00128173828125	\\
0.980954706385057	-0.001312255859375	\\
0.981076791600537	-0.001220703125	\\
0.981198876816018	-0.001617431640625	\\
0.981320962031498	-0.00201416015625	\\
0.981443047246978	-0.002105712890625	\\
0.981565132462459	-0.001861572265625	\\
0.981687217677939	-0.001983642578125	\\
0.98180930289342	-0.001922607421875	\\
0.9819313881089	-0.00201416015625	\\
0.98205347332438	-0.002105712890625	\\
0.982175558539861	-0.002197265625	\\
0.982297643755341	-0.0023193359375	\\
0.982419728970822	-0.001983642578125	\\
0.982541814186302	-0.00213623046875	\\
0.982663899401782	-0.002166748046875	\\
0.982785984617263	-0.001800537109375	\\
0.982908069832743	-0.002227783203125	\\
0.983030155048224	-0.00225830078125	\\
0.983152240263704	-0.00201416015625	\\
0.983274325479184	-0.002044677734375	\\
0.983396410694665	-0.00201416015625	\\
0.983518495910145	-0.002349853515625	\\
0.983640581125626	-0.002044677734375	\\
0.983762666341106	-0.00189208984375	\\
0.983884751556587	-0.002227783203125	\\
0.984006836772067	-0.0023193359375	\\
0.984128921987547	-0.00250244140625	\\
0.984251007203028	-0.002471923828125	\\
0.984373092418508	-0.0025634765625	\\
0.984495177633989	-0.0023193359375	\\
0.984617262849469	-0.00189208984375	\\
0.984739348064949	-0.00164794921875	\\
0.98486143328043	-0.001953125	\\
0.98498351849591	-0.002349853515625	\\
0.985105603711391	-0.0025634765625	\\
0.985227688926871	-0.003082275390625	\\
0.985349774142351	-0.00274658203125	\\
0.985471859357832	-0.00244140625	\\
0.985593944573312	-0.00244140625	\\
0.985716029788793	-0.0023193359375	\\
0.985838115004273	-0.002471923828125	\\
0.985960200219753	-0.002349853515625	\\
0.986082285435234	-0.00213623046875	\\
0.986204370650714	-0.002044677734375	\\
0.986326455866195	-0.0015869140625	\\
0.986448541081675	-0.001495361328125	\\
0.986570626297155	-0.00128173828125	\\
0.986692711512636	-0.001312255859375	\\
0.986814796728116	-0.0010986328125	\\
0.986936881943597	-0.001220703125	\\
0.987058967159077	-0.001953125	\\
0.987181052374557	-0.001495361328125	\\
0.987303137590038	-0.000946044921875	\\
0.987425222805518	-0.00115966796875	\\
0.987547308020999	-0.000946044921875	\\
0.987669393236479	-0.000885009765625	\\
0.987791478451959	-0.00128173828125	\\
0.98791356366744	-0.00103759765625	\\
0.98803564888292	-0.001007080078125	\\
0.988157734098401	-0.000518798828125	\\
0.988279819313881	-0.00042724609375	\\
0.988401904529362	-0.00067138671875	\\
0.988523989744842	-0.00030517578125	\\
0.988646074960322	-0.00030517578125	\\
0.988768160175803	-0.000274658203125	\\
0.988890245391283	-0.000335693359375	\\
0.989012330606764	-0.000274658203125	\\
0.989134415822244	-0.000152587890625	\\
0.989256501037724	0.000152587890625	\\
0.989378586253205	3.0517578125e-05	\\
0.989500671468685	-0.000152587890625	\\
0.989622756684166	9.1552734375e-05	\\
0.989744841899646	0.000152587890625	\\
0.989866927115126	0.000396728515625	\\
0.989989012330607	0.00018310546875	\\
0.990111097546087	0.000640869140625	\\
0.990233182761568	0.00103759765625	\\
0.990355267977048	0.000518798828125	\\
0.990477353192528	0.00042724609375	\\
0.990599438408009	0.000396728515625	\\
0.990721523623489	0.000579833984375	\\
0.99084360883897	0.000640869140625	\\
0.99096569405445	0.000701904296875	\\
0.99108777926993	0.000823974609375	\\
0.991209864485411	0.000579833984375	\\
0.991331949700891	0.000244140625	\\
0.991454034916372	6.103515625e-05	\\
0.991576120131852	0.000518798828125	\\
0.991698205347332	0.000701904296875	\\
0.991820290562813	0.0006103515625	\\
0.991942375778293	0.00067138671875	\\
0.992064460993774	0.00091552734375	\\
0.992186546209254	0.0009765625	\\
0.992308631424734	0.00103759765625	\\
0.992430716640215	0.00152587890625	\\
0.992552801855695	0.00140380859375	\\
0.992674887071176	0.00146484375	\\
0.992796972286656	0.00128173828125	\\
0.992919057502136	0.00067138671875	\\
0.993041142717617	0.00030517578125	\\
0.993163227933097	0.000274658203125	\\
0.993285313148578	0.000213623046875	\\
0.993407398364058	0.000274658203125	\\
0.993529483579539	-0.000152587890625	\\
0.993651568795019	-0.00018310546875	\\
0.993773654010499	0.000213623046875	\\
0.99389573922598	0.000457763671875	\\
0.99401782444146	0.00079345703125	\\
0.994139909656941	0.000701904296875	\\
0.994261994872421	0.00054931640625	\\
0.994384080087901	0.000823974609375	\\
0.994506165303382	0.00067138671875	\\
0.994628250518862	0.000701904296875	\\
0.994750335734343	0.000701904296875	\\
0.994872420949823	0.000579833984375	\\
0.994994506165303	0.00048828125	\\
0.995116591380784	0.0003662109375	\\
0.995238676596264	0.000701904296875	\\
0.995360761811745	0.000946044921875	\\
0.995482847027225	0.000823974609375	\\
0.995604932242705	0.000946044921875	\\
0.995727017458186	0.0008544921875	\\
0.995849102673666	0.00115966796875	\\
0.995971187889147	0.001495361328125	\\
0.996093273104627	0.001251220703125	\\
0.996215358320107	0.001129150390625	\\
0.996337443535588	0.001373291015625	\\
0.996459528751068	0.00189208984375	\\
0.996581613966549	0.00164794921875	\\
0.996703699182029	0.001220703125	\\
0.996825784397509	0.001220703125	\\
0.99694786961299	0.00140380859375	\\
0.99706995482847	0.0015869140625	\\
0.997192040043951	0.001495361328125	\\
0.997314125259431	0.00103759765625	\\
0.997436210474911	0.001068115234375	\\
0.997558295690392	0.0008544921875	\\
0.997680380905872	0.000885009765625	\\
0.997802466121353	0.001007080078125	\\
0.997924551336833	0.00115966796875	\\
0.998046636552314	0.00146484375	\\
0.998168721767794	0.00128173828125	\\
0.998290806983274	0.001190185546875	\\
0.998412892198755	0.00103759765625	\\
0.998534977414235	0.0009765625	\\
0.998657062629716	0.000885009765625	\\
0.998779147845196	0.0008544921875	\\
0.998901233060676	0.00018310546875	\\
0.999023318276157	0	\\
0.999145403491637	0.0001220703125	\\
0.999267488707118	-0.00042724609375	\\
0.999389573922598	-0.000701904296875	\\
0.999511659138078	-0.00054931640625	\\
0.999633744353559	-0.000579833984375	\\
0.999755829569039	-0.000823974609375	\\
0.99987791478452	-0.0010986328125	\\
1	-0.000732421875	\\
};
\end{axis}

\begin{axis}[%
width=\figurewidth,
height=\figureheight,
scale only axis,
xmin=-4000,
xmax=4000,
xlabel={Frequency (in hertz)},
ymin=0,
ymax=0.0006,
at=(plot1.below south west),
anchor=above north west,
title={Magnitude Response}
]
\addplot [color=blue,solid,forget plot]
  table[row sep=crcr]{
-4000	1.64620578289032e-05	\\
-3999.0234375	1.56451426923876e-05	\\
-3998.046875	1.54458246817485e-05	\\
-3997.0703125	1.5892931745498e-05	\\
-3996.09375	1.57969981340263e-05	\\
-3995.1171875	1.59158681212455e-05	\\
-3994.140625	1.57883887109514e-05	\\
-3993.1640625	1.61744263820056e-05	\\
-3992.1875	1.5349359519776e-05	\\
-3991.2109375	1.57738593640657e-05	\\
-3990.234375	1.58142377926051e-05	\\
-3989.2578125	1.5673738949017e-05	\\
-3988.28125	1.58765131575191e-05	\\
-3987.3046875	1.60630195794299e-05	\\
-3986.328125	1.62119008751163e-05	\\
-3985.3515625	1.58517129173588e-05	\\
-3984.375	1.62334572942354e-05	\\
-3983.3984375	1.55170345691891e-05	\\
-3982.421875	1.58135456543504e-05	\\
-3981.4453125	1.60644228550006e-05	\\
-3980.46875	1.58365279708055e-05	\\
-3979.4921875	1.61050179557507e-05	\\
-3978.515625	1.60694910634086e-05	\\
-3977.5390625	1.52848124566208e-05	\\
-3976.5625	1.60109098625366e-05	\\
-3975.5859375	1.56190260831568e-05	\\
-3974.609375	1.62258501509834e-05	\\
-3973.6328125	1.59273435492255e-05	\\
-3972.65625	1.57030884666649e-05	\\
-3971.6796875	1.53558862966417e-05	\\
-3970.703125	1.57123304988777e-05	\\
-3969.7265625	1.61834515882777e-05	\\
-3968.75	1.55713840785001e-05	\\
-3967.7734375	1.59322973494521e-05	\\
-3966.796875	1.60921723233144e-05	\\
-3965.8203125	1.54288034706621e-05	\\
-3964.84375	1.53551013917045e-05	\\
-3963.8671875	1.54235879885826e-05	\\
-3962.890625	1.5329246634832e-05	\\
-3961.9140625	1.56836482955388e-05	\\
-3960.9375	1.53383420174398e-05	\\
-3959.9609375	1.54290549798445e-05	\\
-3958.984375	1.48993093292318e-05	\\
-3958.0078125	1.59063167757275e-05	\\
-3957.03125	1.60403023410598e-05	\\
-3956.0546875	1.56637505862104e-05	\\
-3955.078125	1.55266135114261e-05	\\
-3954.1015625	1.56674481467462e-05	\\
-3953.125	1.51398980182307e-05	\\
-3952.1484375	1.58216141556451e-05	\\
-3951.171875	1.56664179110757e-05	\\
-3950.1953125	1.51422024564678e-05	\\
-3949.21875	1.54227007443591e-05	\\
-3948.2421875	1.60022791070149e-05	\\
-3947.265625	1.51851790334971e-05	\\
-3946.2890625	1.56792580910352e-05	\\
-3945.3125	1.52117117417096e-05	\\
-3944.3359375	1.55457064695777e-05	\\
-3943.359375	1.56429390465114e-05	\\
-3942.3828125	1.53210188016033e-05	\\
-3941.40625	1.57253757068203e-05	\\
-3940.4296875	1.54523023540047e-05	\\
-3939.453125	1.51869644438568e-05	\\
-3938.4765625	1.5259149740604e-05	\\
-3937.5	1.55197614830582e-05	\\
-3936.5234375	1.48181296245962e-05	\\
-3935.546875	1.46784994572223e-05	\\
-3934.5703125	1.54127011283834e-05	\\
-3933.59375	1.51176180477319e-05	\\
-3932.6171875	1.4960068532171e-05	\\
-3931.640625	1.55494209809792e-05	\\
-3930.6640625	1.53903506122711e-05	\\
-3929.6875	1.59432821941408e-05	\\
-3928.7109375	1.46832510324192e-05	\\
-3927.734375	1.56526893279337e-05	\\
-3926.7578125	1.50805706003963e-05	\\
-3925.78125	1.52913681950271e-05	\\
-3924.8046875	1.51649357841723e-05	\\
-3923.828125	1.55263555008389e-05	\\
-3922.8515625	1.53580446956709e-05	\\
-3921.875	1.52622355301139e-05	\\
-3920.8984375	1.55240419969502e-05	\\
-3919.921875	1.46595223258978e-05	\\
-3918.9453125	1.53505978339217e-05	\\
-3917.96875	1.45914797970047e-05	\\
-3916.9921875	1.52378970504707e-05	\\
-3916.015625	1.47587456572094e-05	\\
-3915.0390625	1.47398211155585e-05	\\
-3914.0625	1.50257731821823e-05	\\
-3913.0859375	1.51501445488737e-05	\\
-3912.109375	1.45857472005143e-05	\\
-3911.1328125	1.49279332248409e-05	\\
-3910.15625	1.45203782861644e-05	\\
-3909.1796875	1.49260118498315e-05	\\
-3908.203125	1.48925236710851e-05	\\
-3907.2265625	1.48835357727445e-05	\\
-3906.25	1.49014861047284e-05	\\
-3905.2734375	1.45475374739771e-05	\\
-3904.296875	1.4964528218545e-05	\\
-3903.3203125	1.45290883849308e-05	\\
-3902.34375	1.46106347102387e-05	\\
-3901.3671875	1.49955903702575e-05	\\
-3900.390625	1.36235747890921e-05	\\
-3899.4140625	1.41568561521075e-05	\\
-3898.4375	1.46912704840711e-05	\\
-3897.4609375	1.47575806733977e-05	\\
-3896.484375	1.5587412016138e-05	\\
-3895.5078125	1.45229819341497e-05	\\
-3894.53125	1.46089595920218e-05	\\
-3893.5546875	1.47143368265505e-05	\\
-3892.578125	1.41029257858097e-05	\\
-3891.6015625	1.45608276530717e-05	\\
-3890.625	1.47449330962771e-05	\\
-3889.6484375	1.45868830862842e-05	\\
-3888.671875	1.44087801878623e-05	\\
-3887.6953125	1.42201962380272e-05	\\
-3886.71875	1.4846588732479e-05	\\
-3885.7421875	1.3956722586474e-05	\\
-3884.765625	1.41505801045426e-05	\\
-3883.7890625	1.4832953376519e-05	\\
-3882.8125	1.47188433273299e-05	\\
-3881.8359375	1.3848347247264e-05	\\
-3880.859375	1.40681740251664e-05	\\
-3879.8828125	1.41072803857916e-05	\\
-3878.90625	1.4141675393237e-05	\\
-3877.9296875	1.46368501655168e-05	\\
-3876.953125	1.39928898079254e-05	\\
-3875.9765625	1.46403291764581e-05	\\
-3875	1.43920400964069e-05	\\
-3874.0234375	1.44283977298115e-05	\\
-3873.046875	1.45296561727935e-05	\\
-3872.0703125	1.47274208634802e-05	\\
-3871.09375	1.45768203687799e-05	\\
-3870.1171875	1.47102174561352e-05	\\
-3869.140625	1.51907876105855e-05	\\
-3868.1640625	1.38590998195222e-05	\\
-3867.1875	1.48287817404441e-05	\\
-3866.2109375	1.43644827166438e-05	\\
-3865.234375	1.4853055401043e-05	\\
-3864.2578125	1.4634258777616e-05	\\
-3863.28125	1.45403002384769e-05	\\
-3862.3046875	1.47005965103907e-05	\\
-3861.328125	1.46477867799091e-05	\\
-3860.3515625	1.45454030345069e-05	\\
-3859.375	1.53311945645483e-05	\\
-3858.3984375	1.45973148013151e-05	\\
-3857.421875	1.43390246884787e-05	\\
-3856.4453125	1.37902910224442e-05	\\
-3855.46875	1.52784288960373e-05	\\
-3854.4921875	1.48071105905975e-05	\\
-3853.515625	1.47861255310487e-05	\\
-3852.5390625	1.46639559477574e-05	\\
-3851.5625	1.48240065888495e-05	\\
-3850.5859375	1.48332271314152e-05	\\
-3849.609375	1.51768421079337e-05	\\
-3848.6328125	1.40110170065707e-05	\\
-3847.65625	1.45517833477141e-05	\\
-3846.6796875	1.42875173508222e-05	\\
-3845.703125	1.39043763660006e-05	\\
-3844.7265625	1.45851203682855e-05	\\
-3843.75	1.45945595683275e-05	\\
-3842.7734375	1.45631376998699e-05	\\
-3841.796875	1.40255033348193e-05	\\
-3840.8203125	1.43416048390741e-05	\\
-3839.84375	1.46549209590877e-05	\\
-3838.8671875	1.41846125219445e-05	\\
-3837.890625	1.40872427186241e-05	\\
-3836.9140625	1.54338338974143e-05	\\
-3835.9375	1.47534161916216e-05	\\
-3834.9609375	1.41775551868059e-05	\\
-3833.984375	1.44694548330263e-05	\\
-3833.0078125	1.4092378478042e-05	\\
-3832.03125	1.48242324169661e-05	\\
-3831.0546875	1.41993689530561e-05	\\
-3830.078125	1.40044029700268e-05	\\
-3829.1015625	1.48647051267393e-05	\\
-3828.125	1.47881201771511e-05	\\
-3827.1484375	1.42421163236669e-05	\\
-3826.171875	1.47735715698763e-05	\\
-3825.1953125	1.49183445734569e-05	\\
-3824.21875	1.47360741713597e-05	\\
-3823.2421875	1.49038592576817e-05	\\
-3822.265625	1.48413952293771e-05	\\
-3821.2890625	1.4678019053358e-05	\\
-3820.3125	1.46065760278308e-05	\\
-3819.3359375	1.4982409530077e-05	\\
-3818.359375	1.43720995974198e-05	\\
-3817.3828125	1.41057711282213e-05	\\
-3816.40625	1.41780084207187e-05	\\
-3815.4296875	1.43673968312625e-05	\\
-3814.453125	1.52806693009967e-05	\\
-3813.4765625	1.45995423846877e-05	\\
-3812.5	1.44961309509478e-05	\\
-3811.5234375	1.44308260961965e-05	\\
-3810.546875	1.4791180421814e-05	\\
-3809.5703125	1.53579855224387e-05	\\
-3808.59375	1.52843722290903e-05	\\
-3807.6171875	1.45732835678268e-05	\\
-3806.640625	1.48593266743384e-05	\\
-3805.6640625	1.45744869980002e-05	\\
-3804.6875	1.51614221926575e-05	\\
-3803.7109375	1.52087294880185e-05	\\
-3802.734375	1.56647721932493e-05	\\
-3801.7578125	1.54243398052913e-05	\\
-3800.78125	1.58374101995921e-05	\\
-3799.8046875	1.5777656178509e-05	\\
-3798.828125	1.619906602485e-05	\\
-3797.8515625	1.61648999479292e-05	\\
-3796.875	1.64727005523166e-05	\\
-3795.8984375	1.66914030156505e-05	\\
-3794.921875	1.65349419997366e-05	\\
-3793.9453125	1.59787985499946e-05	\\
-3792.96875	1.60017361699082e-05	\\
-3791.9921875	1.63890907893736e-05	\\
-3791.015625	1.66991514012908e-05	\\
-3790.0390625	1.70038159859729e-05	\\
-3789.0625	1.67369525023344e-05	\\
-3788.0859375	1.65681565193517e-05	\\
-3787.109375	1.70524984939564e-05	\\
-3786.1328125	1.63464493814223e-05	\\
-3785.15625	1.68150059973566e-05	\\
-3784.1796875	1.62051962762503e-05	\\
-3783.203125	1.66925857128778e-05	\\
-3782.2265625	1.69820776983899e-05	\\
-3781.25	1.72854276605271e-05	\\
-3780.2734375	1.73992363410533e-05	\\
-3779.296875	1.69111636824605e-05	\\
-3778.3203125	1.78605197577383e-05	\\
-3777.34375	1.74023113840408e-05	\\
-3776.3671875	1.73066886120989e-05	\\
-3775.390625	1.73666258438689e-05	\\
-3774.4140625	1.75443335501847e-05	\\
-3773.4375	1.75278494490078e-05	\\
-3772.4609375	1.7632879199689e-05	\\
-3771.484375	1.81553544232064e-05	\\
-3770.5078125	1.7591285532191e-05	\\
-3769.53125	1.74224631903819e-05	\\
-3768.5546875	1.67126267332047e-05	\\
-3767.578125	1.76390983247498e-05	\\
-3766.6015625	1.78857053724059e-05	\\
-3765.625	1.8273802906514e-05	\\
-3764.6484375	1.70594368088274e-05	\\
-3763.671875	1.83584190016792e-05	\\
-3762.6953125	1.70917964803589e-05	\\
-3761.71875	1.77969423778119e-05	\\
-3760.7421875	1.73346401064484e-05	\\
-3759.765625	1.83978276522438e-05	\\
-3758.7890625	1.71048868876176e-05	\\
-3757.8125	1.75856367111023e-05	\\
-3756.8359375	1.91306884915027e-05	\\
-3755.859375	1.80485095987372e-05	\\
-3754.8828125	1.81817082665137e-05	\\
-3753.90625	1.86097572048343e-05	\\
-3752.9296875	1.83359071708242e-05	\\
-3751.953125	1.81297976955236e-05	\\
-3750.9765625	1.78862954656723e-05	\\
-3750	1.84826925917205e-05	\\
-3749.0234375	1.80897001238412e-05	\\
-3748.046875	1.83774376218965e-05	\\
-3747.0703125	1.74122605703016e-05	\\
-3746.09375	1.82261524092195e-05	\\
-3745.1171875	1.78937609722655e-05	\\
-3744.140625	1.83259695878112e-05	\\
-3743.1640625	1.80349910300447e-05	\\
-3742.1875	1.77864384470882e-05	\\
-3741.2109375	1.69494673627626e-05	\\
-3740.234375	1.75652351615911e-05	\\
-3739.2578125	1.73722719067167e-05	\\
-3738.28125	1.66782148678204e-05	\\
-3737.3046875	1.59225035234054e-05	\\
-3736.328125	1.79400744260884e-05	\\
-3735.3515625	1.79538397791005e-05	\\
-3734.375	1.69666568828488e-05	\\
-3733.3984375	1.73064354686227e-05	\\
-3732.421875	1.79508988053692e-05	\\
-3731.4453125	1.67416472742208e-05	\\
-3730.46875	1.69329157968823e-05	\\
-3729.4921875	1.70372521977564e-05	\\
-3728.515625	1.8214050273521e-05	\\
-3727.5390625	1.71001247929897e-05	\\
-3726.5625	1.67499111114283e-05	\\
-3725.5859375	1.72180997708545e-05	\\
-3724.609375	1.63458712714612e-05	\\
-3723.6328125	1.54751959710771e-05	\\
-3722.65625	1.58468357455151e-05	\\
-3721.6796875	1.5553646150943e-05	\\
-3720.703125	1.66872509947263e-05	\\
-3719.7265625	1.54002033734685e-05	\\
-3718.75	1.63835123014987e-05	\\
-3717.7734375	1.6264971769953e-05	\\
-3716.796875	1.59368979365893e-05	\\
-3715.8203125	1.5163429906126e-05	\\
-3714.84375	1.48398416443675e-05	\\
-3713.8671875	1.54247034786513e-05	\\
-3712.890625	1.59026586851511e-05	\\
-3711.9140625	1.5798655246915e-05	\\
-3710.9375	1.64665120467083e-05	\\
-3709.9609375	1.57022138822951e-05	\\
-3708.984375	1.62616753824341e-05	\\
-3708.0078125	1.54480417145876e-05	\\
-3707.03125	1.47363149373438e-05	\\
-3706.0546875	1.52787544993479e-05	\\
-3705.078125	1.58664621994087e-05	\\
-3704.1015625	1.58791424818282e-05	\\
-3703.125	1.52068572224087e-05	\\
-3702.1484375	1.5372965357479e-05	\\
-3701.171875	1.49015116330575e-05	\\
-3700.1953125	1.46018538206819e-05	\\
-3699.21875	1.45606589613319e-05	\\
-3698.2421875	1.59635348361142e-05	\\
-3697.265625	1.62406770429405e-05	\\
-3696.2890625	1.56409035562468e-05	\\
-3695.3125	1.55856003595471e-05	\\
-3694.3359375	1.62247825889327e-05	\\
-3693.359375	1.55292881961839e-05	\\
-3692.3828125	1.51421350542422e-05	\\
-3691.40625	1.58715532954513e-05	\\
-3690.4296875	1.57471408597997e-05	\\
-3689.453125	1.5648893440328e-05	\\
-3688.4765625	1.62473864370582e-05	\\
-3687.5	1.54690645874093e-05	\\
-3686.5234375	1.54843850912926e-05	\\
-3685.546875	1.67309510831212e-05	\\
-3684.5703125	1.58442263990683e-05	\\
-3683.59375	1.59857565662551e-05	\\
-3682.6171875	1.60547872417792e-05	\\
-3681.640625	1.6223536585491e-05	\\
-3680.6640625	1.57354665554279e-05	\\
-3679.6875	1.67708510136604e-05	\\
-3678.7109375	1.59354624257141e-05	\\
-3677.734375	1.57898969987643e-05	\\
-3676.7578125	1.61483841668241e-05	\\
-3675.78125	1.55639790531253e-05	\\
-3674.8046875	1.55936045546827e-05	\\
-3673.828125	1.65216626438639e-05	\\
-3672.8515625	1.61465395729117e-05	\\
-3671.875	1.6031373268975e-05	\\
-3670.8984375	1.60739043579428e-05	\\
-3669.921875	1.65000650713951e-05	\\
-3668.9453125	1.56632719585299e-05	\\
-3667.96875	1.61505750825708e-05	\\
-3666.9921875	1.65521879112179e-05	\\
-3666.015625	1.6073615226878e-05	\\
-3665.0390625	1.65924292999244e-05	\\
-3664.0625	1.75438434282901e-05	\\
-3663.0859375	1.68896196493805e-05	\\
-3662.109375	1.69407732706698e-05	\\
-3661.1328125	1.67590538802405e-05	\\
-3660.15625	1.7565860808368e-05	\\
-3659.1796875	1.82739581668763e-05	\\
-3658.203125	1.74616773381305e-05	\\
-3657.2265625	1.78906094402472e-05	\\
-3656.25	1.72827495428591e-05	\\
-3655.2734375	1.78538177790379e-05	\\
-3654.296875	1.79505161713224e-05	\\
-3653.3203125	1.78395989950831e-05	\\
-3652.34375	1.82093777386316e-05	\\
-3651.3671875	1.81448091794705e-05	\\
-3650.390625	1.76914095134893e-05	\\
-3649.4140625	1.68485558511174e-05	\\
-3648.4375	1.83512699965168e-05	\\
-3647.4609375	1.8810583069891e-05	\\
-3646.484375	1.81128853475408e-05	\\
-3645.5078125	1.88817244121111e-05	\\
-3644.53125	1.90166312167296e-05	\\
-3643.5546875	1.90465479597217e-05	\\
-3642.578125	1.85542792018801e-05	\\
-3641.6015625	1.92790143581935e-05	\\
-3640.625	1.99659800137242e-05	\\
-3639.6484375	1.8682033425755e-05	\\
-3638.671875	1.9094462840902e-05	\\
-3637.6953125	1.92415162738866e-05	\\
-3636.71875	1.991280883289e-05	\\
-3635.7421875	1.86070233100803e-05	\\
-3634.765625	1.90324808329351e-05	\\
-3633.7890625	1.89264634106778e-05	\\
-3632.8125	2.0552076846031e-05	\\
-3631.8359375	1.89491518416333e-05	\\
-3630.859375	1.92261160130056e-05	\\
-3629.8828125	1.91270905874751e-05	\\
-3628.90625	1.94769419573787e-05	\\
-3627.9296875	1.94130303792687e-05	\\
-3626.953125	1.99788148104188e-05	\\
-3625.9765625	2.08675698154606e-05	\\
-3625	1.97095899286427e-05	\\
-3624.0234375	1.94518998401709e-05	\\
-3623.046875	2.04514926270164e-05	\\
-3622.0703125	2.09235604478151e-05	\\
-3621.09375	1.89530623093002e-05	\\
-3620.1171875	2.05391785204725e-05	\\
-3619.140625	2.10715030195758e-05	\\
-3618.1640625	1.99239538998269e-05	\\
-3617.1875	2.09908394831931e-05	\\
-3616.2109375	2.10846913421068e-05	\\
-3615.234375	2.08041381381829e-05	\\
-3614.2578125	2.07283185698832e-05	\\
-3613.28125	2.16681025702148e-05	\\
-3612.3046875	2.1766368214898e-05	\\
-3611.328125	2.17299717259297e-05	\\
-3610.3515625	2.31173922021997e-05	\\
-3609.375	2.26410176576813e-05	\\
-3608.3984375	2.26543197490855e-05	\\
-3607.421875	2.20831054545344e-05	\\
-3606.4453125	2.1333002467026e-05	\\
-3605.46875	2.29016255229517e-05	\\
-3604.4921875	2.15804499845683e-05	\\
-3603.515625	2.16701198580352e-05	\\
-3602.5390625	2.31843762343539e-05	\\
-3601.5625	2.29098731053432e-05	\\
-3600.5859375	2.33105914746023e-05	\\
-3599.609375	2.31971522733876e-05	\\
-3598.6328125	2.2702256206397e-05	\\
-3597.65625	2.4469171839236e-05	\\
-3596.6796875	2.35509122725122e-05	\\
-3595.703125	2.46697438324565e-05	\\
-3594.7265625	2.39624064907858e-05	\\
-3593.75	2.45873305022592e-05	\\
-3592.7734375	2.40397200591087e-05	\\
-3591.796875	2.50276146134867e-05	\\
-3590.8203125	2.35775216069152e-05	\\
-3589.84375	2.35220493645642e-05	\\
-3588.8671875	2.41162126874424e-05	\\
-3587.890625	2.45330787458609e-05	\\
-3586.9140625	2.45424543955302e-05	\\
-3585.9375	2.49738259378067e-05	\\
-3584.9609375	2.60303629244257e-05	\\
-3583.984375	2.51254294360023e-05	\\
-3583.0078125	2.41048984050123e-05	\\
-3582.03125	2.40103000094607e-05	\\
-3581.0546875	2.49622729750087e-05	\\
-3580.078125	2.49896024534018e-05	\\
-3579.1015625	2.52067587952385e-05	\\
-3578.125	2.58749563094018e-05	\\
-3577.1484375	2.53334540061946e-05	\\
-3576.171875	2.52133543979368e-05	\\
-3575.1953125	2.56954985665783e-05	\\
-3574.21875	2.62184935069361e-05	\\
-3573.2421875	2.61606935791299e-05	\\
-3572.265625	2.54830594644298e-05	\\
-3571.2890625	2.55093788101814e-05	\\
-3570.3125	2.56415906419501e-05	\\
-3569.3359375	2.69497708921976e-05	\\
-3568.359375	2.55099518116227e-05	\\
-3567.3828125	2.55218238248818e-05	\\
-3566.40625	2.5867928907515e-05	\\
-3565.4296875	2.55173040968558e-05	\\
-3564.453125	2.63505008769061e-05	\\
-3563.4765625	2.57954458101508e-05	\\
-3562.5	2.58157227071068e-05	\\
-3561.5234375	2.68158692782394e-05	\\
-3560.546875	2.69215744933778e-05	\\
-3559.5703125	2.66933030173061e-05	\\
-3558.59375	2.75796858157951e-05	\\
-3557.6171875	2.58718204258773e-05	\\
-3556.640625	2.5893278256248e-05	\\
-3555.6640625	2.63571305006605e-05	\\
-3554.6875	2.64831721872768e-05	\\
-3553.7109375	2.76023709425052e-05	\\
-3552.734375	2.56785523077602e-05	\\
-3551.7578125	2.61992763498294e-05	\\
-3550.78125	2.7746275581636e-05	\\
-3549.8046875	2.6114677162026e-05	\\
-3548.828125	2.64016588449196e-05	\\
-3547.8515625	2.71635037408651e-05	\\
-3546.875	2.68073698640289e-05	\\
-3545.8984375	2.62334311759134e-05	\\
-3544.921875	2.73170911070603e-05	\\
-3543.9453125	2.73258132418663e-05	\\
-3542.96875	2.6797238261703e-05	\\
-3541.9921875	2.66587104685725e-05	\\
-3541.015625	2.65178359163938e-05	\\
-3540.0390625	2.65564757689918e-05	\\
-3539.0625	2.64482811703165e-05	\\
-3538.0859375	2.64540159538362e-05	\\
-3537.109375	2.66583055201011e-05	\\
-3536.1328125	2.66979942972716e-05	\\
-3535.15625	2.71948986342349e-05	\\
-3534.1796875	2.78667568739484e-05	\\
-3533.203125	2.71396039715882e-05	\\
-3532.2265625	2.65957171436006e-05	\\
-3531.25	2.63042577801468e-05	\\
-3530.2734375	2.7060107471281e-05	\\
-3529.296875	2.70038110916542e-05	\\
-3528.3203125	2.74641680959581e-05	\\
-3527.34375	2.73617793693931e-05	\\
-3526.3671875	2.75140587197064e-05	\\
-3525.390625	2.79317142243568e-05	\\
-3524.4140625	2.74886596317046e-05	\\
-3523.4375	2.65978183703822e-05	\\
-3522.4609375	2.88376249176902e-05	\\
-3521.484375	2.61731873148945e-05	\\
-3520.5078125	2.91211967285981e-05	\\
-3519.53125	2.73815933378082e-05	\\
-3518.5546875	2.74214979635601e-05	\\
-3517.578125	2.81682927974711e-05	\\
-3516.6015625	2.70701095337557e-05	\\
-3515.625	2.81470083529714e-05	\\
-3514.6484375	2.82935290275875e-05	\\
-3513.671875	2.7312659054098e-05	\\
-3512.6953125	2.91786150653023e-05	\\
-3511.71875	2.95139970816127e-05	\\
-3510.7421875	2.80564641286628e-05	\\
-3509.765625	2.93273144060286e-05	\\
-3508.7890625	2.84315151869734e-05	\\
-3507.8125	2.927735799936e-05	\\
-3506.8359375	2.78861389363427e-05	\\
-3505.859375	2.81530440219582e-05	\\
-3504.8828125	2.99858628950313e-05	\\
-3503.90625	2.8905796528729e-05	\\
-3502.9296875	2.93592471067734e-05	\\
-3501.953125	2.78195637348859e-05	\\
-3500.9765625	3.04350926240595e-05	\\
-3500	3.02937736748853e-05	\\
-3499.0234375	3.00603302912293e-05	\\
-3498.046875	2.99556210871412e-05	\\
-3497.0703125	2.91324219057816e-05	\\
-3496.09375	2.87187240487806e-05	\\
-3495.1171875	3.10417113723058e-05	\\
-3494.140625	3.02090831264966e-05	\\
-3493.1640625	2.98311018598804e-05	\\
-3492.1875	3.09401201060462e-05	\\
-3491.2109375	2.98527021962979e-05	\\
-3490.234375	3.09016848455933e-05	\\
-3489.2578125	2.95545037157273e-05	\\
-3488.28125	2.90406044693606e-05	\\
-3487.3046875	2.88935015925713e-05	\\
-3486.328125	2.91187471653898e-05	\\
-3485.3515625	2.99371736301712e-05	\\
-3484.375	3.01963214767789e-05	\\
-3483.3984375	2.86937636249691e-05	\\
-3482.421875	2.91818314267942e-05	\\
-3481.4453125	3.07409344654108e-05	\\
-3480.46875	2.96444729318815e-05	\\
-3479.4921875	2.91317456948031e-05	\\
-3478.515625	3.08786101040272e-05	\\
-3477.5390625	3.03713525465576e-05	\\
-3476.5625	3.09543430485182e-05	\\
-3475.5859375	2.98724823857413e-05	\\
-3474.609375	3.09616735734911e-05	\\
-3473.6328125	2.99047030063258e-05	\\
-3472.65625	3.07855962580568e-05	\\
-3471.6796875	2.90426000350295e-05	\\
-3470.703125	2.99618737958329e-05	\\
-3469.7265625	3.09588297758646e-05	\\
-3468.75	3.12258881427906e-05	\\
-3467.7734375	3.09333273593496e-05	\\
-3466.796875	2.90918657881607e-05	\\
-3465.8203125	3.15581267346153e-05	\\
-3464.84375	3.14400094168104e-05	\\
-3463.8671875	3.06176906965113e-05	\\
-3462.890625	3.1871437166885e-05	\\
-3461.9140625	3.03956226860968e-05	\\
-3460.9375	3.10410771208765e-05	\\
-3459.9609375	3.2896556233663e-05	\\
-3458.984375	3.12222573389485e-05	\\
-3458.0078125	3.15007411011233e-05	\\
-3457.03125	3.1474710268274e-05	\\
-3456.0546875	3.12156064975916e-05	\\
-3455.078125	3.06562882044384e-05	\\
-3454.1015625	3.09099986077345e-05	\\
-3453.125	3.2031485113907e-05	\\
-3452.1484375	3.1309345951561e-05	\\
-3451.171875	3.23805194320225e-05	\\
-3450.1953125	3.2176110987613e-05	\\
-3449.21875	3.13967195180296e-05	\\
-3448.2421875	3.09086027888059e-05	\\
-3447.265625	3.10979715388824e-05	\\
-3446.2890625	3.20682573441534e-05	\\
-3445.3125	3.20927836653802e-05	\\
-3444.3359375	3.12588594035747e-05	\\
-3443.359375	3.11945604715771e-05	\\
-3442.3828125	3.19452926686306e-05	\\
-3441.40625	3.18938129609416e-05	\\
-3440.4296875	3.29404507093735e-05	\\
-3439.453125	3.28258998261889e-05	\\
-3438.4765625	3.19122013728678e-05	\\
-3437.5	3.22859510041257e-05	\\
-3436.5234375	3.36542829414412e-05	\\
-3435.546875	3.3068730618383e-05	\\
-3434.5703125	3.20191601413962e-05	\\
-3433.59375	3.28207258240002e-05	\\
-3432.6171875	3.29515603575206e-05	\\
-3431.640625	3.17874863036205e-05	\\
-3430.6640625	3.29571477649708e-05	\\
-3429.6875	3.28931873738366e-05	\\
-3428.7109375	3.41014517663729e-05	\\
-3427.734375	3.31983973197184e-05	\\
-3426.7578125	3.18826518105445e-05	\\
-3425.78125	3.41229928912918e-05	\\
-3424.8046875	3.4549615755368e-05	\\
-3423.828125	3.36085919199065e-05	\\
-3422.8515625	3.42099416073707e-05	\\
-3421.875	3.46572886436119e-05	\\
-3420.8984375	3.36260301382123e-05	\\
-3419.921875	3.39618186159044e-05	\\
-3418.9453125	3.39547115761469e-05	\\
-3417.96875	3.57357842170991e-05	\\
-3416.9921875	3.441664524047e-05	\\
-3416.015625	3.44017677777805e-05	\\
-3415.0390625	3.47000910207834e-05	\\
-3414.0625	3.6297154084104e-05	\\
-3413.0859375	3.40246656924705e-05	\\
-3412.109375	3.43858812072593e-05	\\
-3411.1328125	3.52509400620948e-05	\\
-3410.15625	3.40619155313936e-05	\\
-3409.1796875	3.47320617490851e-05	\\
-3408.203125	3.6065589754891e-05	\\
-3407.2265625	3.34653000433997e-05	\\
-3406.25	3.45797303993096e-05	\\
-3405.2734375	3.38325699745162e-05	\\
-3404.296875	3.68879046364181e-05	\\
-3403.3203125	3.46933401882232e-05	\\
-3402.34375	3.68415619303257e-05	\\
-3401.3671875	3.51589703142101e-05	\\
-3400.390625	3.69040105893881e-05	\\
-3399.4140625	3.55001434231666e-05	\\
-3398.4375	3.72970408785056e-05	\\
-3397.4609375	3.61614635416514e-05	\\
-3396.484375	3.8199290209176e-05	\\
-3395.5078125	3.67457865321347e-05	\\
-3394.53125	3.75218458947095e-05	\\
-3393.5546875	3.70604626672256e-05	\\
-3392.578125	3.57786195632903e-05	\\
-3391.6015625	3.81918327832318e-05	\\
-3390.625	3.7094535869522e-05	\\
-3389.6484375	3.63097723452058e-05	\\
-3388.671875	3.68023808628379e-05	\\
-3387.6953125	3.62956681088397e-05	\\
-3386.71875	3.73870266541703e-05	\\
-3385.7421875	3.55750811642845e-05	\\
-3384.765625	3.8426478823846e-05	\\
-3383.7890625	3.86563719094023e-05	\\
-3382.8125	3.7612705723241e-05	\\
-3381.8359375	3.83422434041369e-05	\\
-3380.859375	3.71926536152362e-05	\\
-3379.8828125	3.87070434272118e-05	\\
-3378.90625	3.8702033942484e-05	\\
-3377.9296875	3.71806185159744e-05	\\
-3376.953125	3.82242906519106e-05	\\
-3375.9765625	3.95915685019246e-05	\\
-3375	3.69713805521837e-05	\\
-3374.0234375	3.87958678377935e-05	\\
-3373.046875	3.85676877879649e-05	\\
-3372.0703125	3.82719238111177e-05	\\
-3371.09375	3.78923346055961e-05	\\
-3370.1171875	3.86129811705388e-05	\\
-3369.140625	3.8751078190476e-05	\\
-3368.1640625	3.79284756436893e-05	\\
-3367.1875	3.81590364922339e-05	\\
-3366.2109375	3.77476152942524e-05	\\
-3365.234375	3.97309017263763e-05	\\
-3364.2578125	3.85419983848458e-05	\\
-3363.28125	4.02623338266632e-05	\\
-3362.3046875	3.98343816178961e-05	\\
-3361.328125	3.86479276578325e-05	\\
-3360.3515625	4.01822709429561e-05	\\
-3359.375	4.01019050125598e-05	\\
-3358.3984375	4.0632855269099e-05	\\
-3357.421875	3.97460723956317e-05	\\
-3356.4453125	3.99327450187741e-05	\\
-3355.46875	3.95808473451207e-05	\\
-3354.4921875	3.98531611666528e-05	\\
-3353.515625	4.08973340057772e-05	\\
-3352.5390625	4.06117215065906e-05	\\
-3351.5625	3.87954470353341e-05	\\
-3350.5859375	3.92118435374847e-05	\\
-3349.609375	3.99134750433991e-05	\\
-3348.6328125	3.82849918441828e-05	\\
-3347.65625	3.82047998340897e-05	\\
-3346.6796875	4.06978114026772e-05	\\
-3345.703125	3.91244549342019e-05	\\
-3344.7265625	3.91032880034386e-05	\\
-3343.75	4.09115499906609e-05	\\
-3342.7734375	4.06782546427152e-05	\\
-3341.796875	4.06289250003642e-05	\\
-3340.8203125	4.09658539716829e-05	\\
-3339.84375	4.01855356331828e-05	\\
-3338.8671875	3.95975781924516e-05	\\
-3337.890625	3.97401900758627e-05	\\
-3336.9140625	3.95483549119248e-05	\\
-3335.9375	4.071466688403e-05	\\
-3334.9609375	4.03080678977584e-05	\\
-3333.984375	4.00337711271658e-05	\\
-3333.0078125	4.03056769390846e-05	\\
-3332.03125	4.06700895268924e-05	\\
-3331.0546875	4.10076258120307e-05	\\
-3330.078125	3.89141723104874e-05	\\
-3329.1015625	4.02332495383893e-05	\\
-3328.125	4.06350691329075e-05	\\
-3327.1484375	3.88620527207838e-05	\\
-3326.171875	4.17456338569046e-05	\\
-3325.1953125	4.09885113868334e-05	\\
-3324.21875	3.89318101987035e-05	\\
-3323.2421875	3.89255259429558e-05	\\
-3322.265625	3.96571572591754e-05	\\
-3321.2890625	4.02031498143316e-05	\\
-3320.3125	4.08522841915837e-05	\\
-3319.3359375	3.89978712614007e-05	\\
-3318.359375	4.05806715301381e-05	\\
-3317.3828125	3.93060444926692e-05	\\
-3316.40625	3.87479677440944e-05	\\
-3315.4296875	3.98777835703924e-05	\\
-3314.453125	3.97802981170387e-05	\\
-3313.4765625	4.0488939178489e-05	\\
-3312.5	4.04001303776818e-05	\\
-3311.5234375	4.02690308486859e-05	\\
-3310.546875	3.84708996906554e-05	\\
-3309.5703125	3.98527825497735e-05	\\
-3308.59375	3.89395235142131e-05	\\
-3307.6171875	3.8892838813977e-05	\\
-3306.640625	4.0380979765375e-05	\\
-3305.6640625	4.04604082026269e-05	\\
-3304.6875	3.91041533685109e-05	\\
-3303.7109375	4.04137678475603e-05	\\
-3302.734375	4.11220571829046e-05	\\
-3301.7578125	3.99303721523328e-05	\\
-3300.78125	4.16054373600512e-05	\\
-3299.8046875	4.08262002448703e-05	\\
-3298.828125	4.06516233670824e-05	\\
-3297.8515625	4.04022583370032e-05	\\
-3296.875	3.87029979220739e-05	\\
-3295.8984375	4.14731117315795e-05	\\
-3294.921875	4.1748717585997e-05	\\
-3293.9453125	4.07955081578331e-05	\\
-3292.96875	4.17392450771033e-05	\\
-3291.9921875	4.08196400711493e-05	\\
-3291.015625	4.1002762630365e-05	\\
-3290.0390625	3.90735098575523e-05	\\
-3289.0625	3.86582510429752e-05	\\
-3288.0859375	3.88082977367876e-05	\\
-3287.109375	4.04224779259354e-05	\\
-3286.1328125	3.96683274378273e-05	\\
-3285.15625	3.88478928694928e-05	\\
-3284.1796875	3.84874274634404e-05	\\
-3283.203125	3.96913318084003e-05	\\
-3282.2265625	3.80755378814036e-05	\\
-3281.25	3.89594967061912e-05	\\
-3280.2734375	3.81039101647778e-05	\\
-3279.296875	3.64944353836352e-05	\\
-3278.3203125	3.83348219395533e-05	\\
-3277.34375	3.78798183923987e-05	\\
-3276.3671875	3.85435522376935e-05	\\
-3275.390625	3.81363632367165e-05	\\
-3274.4140625	3.7757840625445e-05	\\
-3273.4375	3.65878142140555e-05	\\
-3272.4609375	3.75462160921998e-05	\\
-3271.484375	3.72934956202249e-05	\\
-3270.5078125	3.80245864209806e-05	\\
-3269.53125	3.79588488415076e-05	\\
-3268.5546875	3.68161480211044e-05	\\
-3267.578125	3.5994201713211e-05	\\
-3266.6015625	3.70332988266784e-05	\\
-3265.625	3.79583365192714e-05	\\
-3264.6484375	3.75948761471253e-05	\\
-3263.671875	3.75947664225051e-05	\\
-3262.6953125	3.49438936586157e-05	\\
-3261.71875	3.54117128361597e-05	\\
-3260.7421875	3.69506465674615e-05	\\
-3259.765625	3.77939662189371e-05	\\
-3258.7890625	3.62549496136296e-05	\\
-3257.8125	3.61577540464856e-05	\\
-3256.8359375	3.64614726969193e-05	\\
-3255.859375	3.65593010174272e-05	\\
-3254.8828125	3.7953050618249e-05	\\
-3253.90625	3.75355195157046e-05	\\
-3252.9296875	3.6714374789888e-05	\\
-3251.953125	3.69001712838974e-05	\\
-3250.9765625	3.70843005095033e-05	\\
-3250	3.53177932574838e-05	\\
-3249.0234375	3.51922479882223e-05	\\
-3248.046875	3.64751695035965e-05	\\
-3247.0703125	3.66774744525992e-05	\\
-3246.09375	3.64047660284017e-05	\\
-3245.1171875	3.77146506736707e-05	\\
-3244.140625	3.57334638762422e-05	\\
-3243.1640625	3.69465825709745e-05	\\
-3242.1875	3.71152847115539e-05	\\
-3241.2109375	3.58742676628016e-05	\\
-3240.234375	3.44338516462779e-05	\\
-3239.2578125	3.54755516198523e-05	\\
-3238.28125	3.59888651251129e-05	\\
-3237.3046875	3.49305807430575e-05	\\
-3236.328125	3.52926834537325e-05	\\
-3235.3515625	3.45037456987589e-05	\\
-3234.375	3.53274769343448e-05	\\
-3233.3984375	3.35298703369843e-05	\\
-3232.421875	3.49595498599268e-05	\\
-3231.4453125	3.5785850956898e-05	\\
-3230.46875	3.64794314219643e-05	\\
-3229.4921875	3.38101281594035e-05	\\
-3228.515625	3.53603730694673e-05	\\
-3227.5390625	3.40085912528969e-05	\\
-3226.5625	3.38157085636667e-05	\\
-3225.5859375	3.48107540721986e-05	\\
-3224.609375	3.53003380074278e-05	\\
-3223.6328125	3.40872487899431e-05	\\
-3222.65625	3.21761884253154e-05	\\
-3221.6796875	3.18617737992731e-05	\\
-3220.703125	3.35696030675657e-05	\\
-3219.7265625	3.25746175048444e-05	\\
-3218.75	3.48122780507803e-05	\\
-3217.7734375	3.26815862316216e-05	\\
-3216.796875	3.20020932947969e-05	\\
-3215.8203125	3.21559746994664e-05	\\
-3214.84375	3.29576039652999e-05	\\
-3213.8671875	3.32563502016417e-05	\\
-3212.890625	3.24167543208186e-05	\\
-3211.9140625	3.19893271293489e-05	\\
-3210.9375	3.14793421224717e-05	\\
-3209.9609375	3.10477207722022e-05	\\
-3208.984375	3.23557473007214e-05	\\
-3208.0078125	3.094503395752e-05	\\
-3207.03125	3.12411540061388e-05	\\
-3206.0546875	3.15860641670388e-05	\\
-3205.078125	2.92883470892242e-05	\\
-3204.1015625	3.15804219311689e-05	\\
-3203.125	3.12857841914466e-05	\\
-3202.1484375	2.92123228357661e-05	\\
-3201.171875	3.03768060955706e-05	\\
-3200.1953125	3.0723599786462e-05	\\
-3199.21875	2.9752630946501e-05	\\
-3198.2421875	2.95898027316756e-05	\\
-3197.265625	3.06177049096462e-05	\\
-3196.2890625	2.9672391996593e-05	\\
-3195.3125	2.67455935346269e-05	\\
-3194.3359375	2.90692311960411e-05	\\
-3193.359375	2.92219595917822e-05	\\
-3192.3828125	3.01109157210244e-05	\\
-3191.40625	2.83446591837197e-05	\\
-3190.4296875	2.88102249389653e-05	\\
-3189.453125	2.92798615499237e-05	\\
-3188.4765625	2.79925735607876e-05	\\
-3187.5	2.73240052380003e-05	\\
-3186.5234375	2.75736062941314e-05	\\
-3185.546875	2.6511570095167e-05	\\
-3184.5703125	2.69742271747599e-05	\\
-3183.59375	2.58994474601392e-05	\\
-3182.6171875	2.57732538812736e-05	\\
-3181.640625	2.61828255733802e-05	\\
-3180.6640625	2.7525249004268e-05	\\
-3179.6875	2.64622350651744e-05	\\
-3178.7109375	2.49029937798286e-05	\\
-3177.734375	2.68232185610116e-05	\\
-3176.7578125	2.77547025700335e-05	\\
-3175.78125	2.57998708782954e-05	\\
-3174.8046875	2.69640380987007e-05	\\
-3173.828125	2.6429380404242e-05	\\
-3172.8515625	2.79369534584664e-05	\\
-3171.875	2.67640520572693e-05	\\
-3170.8984375	2.76261737411976e-05	\\
-3169.921875	2.40884153102062e-05	\\
-3168.9453125	2.60655527956326e-05	\\
-3167.96875	2.76610099349505e-05	\\
-3166.9921875	2.57099608991133e-05	\\
-3166.015625	2.67155731447539e-05	\\
-3165.0390625	2.65162001041185e-05	\\
-3164.0625	2.57246331801171e-05	\\
-3163.0859375	2.59241580969057e-05	\\
-3162.109375	2.53788002911041e-05	\\
-3161.1328125	2.66743847422907e-05	\\
-3160.15625	2.80631801473207e-05	\\
-3159.1796875	2.64089301421992e-05	\\
-3158.203125	2.47780693945841e-05	\\
-3157.2265625	2.77702296774028e-05	\\
-3156.25	2.55600485843321e-05	\\
-3155.2734375	2.82606422646704e-05	\\
-3154.296875	2.78451155105723e-05	\\
-3153.3203125	2.83879965369916e-05	\\
-3152.34375	2.50729018263763e-05	\\
-3151.3671875	2.66380856252809e-05	\\
-3150.390625	2.76604929389653e-05	\\
-3149.4140625	2.9031304192437e-05	\\
-3148.4375	2.55618663732882e-05	\\
-3147.4609375	2.80169850053766e-05	\\
-3146.484375	2.81386496080252e-05	\\
-3145.5078125	2.78036182649591e-05	\\
-3144.53125	2.78253599895042e-05	\\
-3143.5546875	2.69924203666805e-05	\\
-3142.578125	2.57162115881023e-05	\\
-3141.6015625	2.83502099835554e-05	\\
-3140.625	2.72927740689705e-05	\\
-3139.6484375	3.00336805419217e-05	\\
-3138.671875	2.92334959895963e-05	\\
-3137.6953125	2.88872726083395e-05	\\
-3136.71875	2.81724049323859e-05	\\
-3135.7421875	2.89422921538947e-05	\\
-3134.765625	2.85780763360501e-05	\\
-3133.7890625	2.79933704583777e-05	\\
-3132.8125	3.07232806644629e-05	\\
-3131.8359375	2.68377195644519e-05	\\
-3130.859375	2.87494245878489e-05	\\
-3129.8828125	2.94701496867845e-05	\\
-3128.90625	2.91443626882768e-05	\\
-3127.9296875	2.95370162559283e-05	\\
-3126.953125	2.91652985290526e-05	\\
-3125.9765625	3.05766033692105e-05	\\
-3125	2.82484155753314e-05	\\
-3124.0234375	2.96342588522307e-05	\\
-3123.046875	3.10844568526773e-05	\\
-3122.0703125	3.10819128434833e-05	\\
-3121.09375	3.1010763623464e-05	\\
-3120.1171875	2.92415116124135e-05	\\
-3119.140625	3.06685835320542e-05	\\
-3118.1640625	3.06789783817847e-05	\\
-3117.1875	2.96373957534634e-05	\\
-3116.2109375	3.10614210683317e-05	\\
-3115.234375	3.14457078600826e-05	\\
-3114.2578125	2.9796545921878e-05	\\
-3113.28125	3.08350771623451e-05	\\
-3112.3046875	3.09991134540061e-05	\\
-3111.328125	2.84943615585759e-05	\\
-3110.3515625	3.36268779346974e-05	\\
-3109.375	3.10817095226146e-05	\\
-3108.3984375	2.88320482502937e-05	\\
-3107.421875	3.19114481641218e-05	\\
-3106.4453125	3.19450441975908e-05	\\
-3105.46875	2.92049975650396e-05	\\
-3104.4921875	2.96138820880346e-05	\\
-3103.515625	3.16178656057868e-05	\\
-3102.5390625	3.24149527976025e-05	\\
-3101.5625	3.25655648592288e-05	\\
-3100.5859375	3.42185220528014e-05	\\
-3099.609375	3.14833534273443e-05	\\
-3098.6328125	3.11693408324754e-05	\\
-3097.65625	3.15755433292843e-05	\\
-3096.6796875	3.02499844093457e-05	\\
-3095.703125	3.45605478740251e-05	\\
-3094.7265625	3.39029587052104e-05	\\
-3093.75	3.20985867580284e-05	\\
-3092.7734375	3.40308830852182e-05	\\
-3091.796875	3.21384550163284e-05	\\
-3090.8203125	3.39672265970908e-05	\\
-3089.84375	3.07868931641692e-05	\\
-3088.8671875	3.3632287564388e-05	\\
-3087.890625	3.44145822288836e-05	\\
-3086.9140625	3.51821672465833e-05	\\
-3085.9375	3.48680982701951e-05	\\
-3084.9609375	3.37636627808415e-05	\\
-3083.984375	3.30475541952206e-05	\\
-3083.0078125	3.54410869006974e-05	\\
-3082.03125	3.43414321921998e-05	\\
-3081.0546875	3.34075952583177e-05	\\
-3080.078125	3.44477221270424e-05	\\
-3079.1015625	3.36614499422377e-05	\\
-3078.125	3.60617782867738e-05	\\
-3077.1484375	3.5826160759985e-05	\\
-3076.171875	3.41115497549692e-05	\\
-3075.1953125	3.47986125147459e-05	\\
-3074.21875	3.57392920963748e-05	\\
-3073.2421875	3.45700122079313e-05	\\
-3072.265625	3.40161253841739e-05	\\
-3071.2890625	3.47226046528069e-05	\\
-3070.3125	3.57299125067907e-05	\\
-3069.3359375	3.56876658393038e-05	\\
-3068.359375	3.52958526483514e-05	\\
-3067.3828125	3.39937109992441e-05	\\
-3066.40625	3.42236192389287e-05	\\
-3065.4296875	3.55619422043933e-05	\\
-3064.453125	3.76081261133199e-05	\\
-3063.4765625	3.55117363274478e-05	\\
-3062.5	3.76098612476553e-05	\\
-3061.5234375	3.74229972536407e-05	\\
-3060.546875	3.7421747283896e-05	\\
-3059.5703125	3.55016997206352e-05	\\
-3058.59375	3.59599625711596e-05	\\
-3057.6171875	3.58153782832848e-05	\\
-3056.640625	3.72233044094653e-05	\\
-3055.6640625	3.69574408945514e-05	\\
-3054.6875	3.71894714224076e-05	\\
-3053.7109375	3.68046207733246e-05	\\
-3052.734375	3.71450331652948e-05	\\
-3051.7578125	3.49679275528105e-05	\\
-3050.78125	3.63059868902079e-05	\\
-3049.8046875	3.51486653939672e-05	\\
-3048.828125	3.48947948419627e-05	\\
-3047.8515625	3.54941329057378e-05	\\
-3046.875	3.433619717842e-05	\\
-3045.8984375	3.28077865380545e-05	\\
-3044.921875	3.59524671602896e-05	\\
-3043.9453125	3.66092683001541e-05	\\
-3042.96875	3.53320833703304e-05	\\
-3041.9921875	3.79134176871766e-05	\\
-3041.015625	3.5747749315582e-05	\\
-3040.0390625	3.54855754183135e-05	\\
-3039.0625	3.71384355975487e-05	\\
-3038.0859375	3.50306408486444e-05	\\
-3037.109375	3.50731598803292e-05	\\
-3036.1328125	3.49398921777003e-05	\\
-3035.15625	3.49693972864942e-05	\\
-3034.1796875	3.54139639272217e-05	\\
-3033.203125	3.4473100082272e-05	\\
-3032.2265625	3.58623507555009e-05	\\
-3031.25	3.56203751638042e-05	\\
-3030.2734375	3.51216317370242e-05	\\
-3029.296875	3.35896373377324e-05	\\
-3028.3203125	3.57016153431043e-05	\\
-3027.34375	3.6066673125719e-05	\\
-3026.3671875	3.5263555526352e-05	\\
-3025.390625	3.58648787573246e-05	\\
-3024.4140625	3.52722785918245e-05	\\
-3023.4375	3.65484150336351e-05	\\
-3022.4609375	3.72120973289318e-05	\\
-3021.484375	3.59286613708121e-05	\\
-3020.5078125	3.63614773667809e-05	\\
-3019.53125	3.52770618036743e-05	\\
-3018.5546875	3.4000098389291e-05	\\
-3017.578125	3.47727423991022e-05	\\
-3016.6015625	3.44820877662997e-05	\\
-3015.625	3.55084505561099e-05	\\
-3014.6484375	3.64134111260863e-05	\\
-3013.671875	3.42140629908495e-05	\\
-3012.6953125	3.58250049227718e-05	\\
-3011.71875	3.73291575856844e-05	\\
-3010.7421875	3.49023638447769e-05	\\
-3009.765625	3.56997770505389e-05	\\
-3008.7890625	3.49697925081318e-05	\\
-3007.8125	3.38007365785189e-05	\\
-3006.8359375	3.53079233682291e-05	\\
-3005.859375	3.67412313540254e-05	\\
-3004.8828125	3.4068339306054e-05	\\
-3003.90625	3.36116974678025e-05	\\
-3002.9296875	3.21144809627352e-05	\\
-3001.953125	3.64963667999297e-05	\\
-3000.9765625	3.50294236741751e-05	\\
-3000	3.26498445879872e-05	\\
-2999.0234375	3.45930411847851e-05	\\
-2998.046875	3.2980422593069e-05	\\
-2997.0703125	3.24197508982148e-05	\\
-2996.09375	3.37216642772986e-05	\\
-2995.1171875	3.44490950378859e-05	\\
-2994.140625	3.44354113321484e-05	\\
-2993.1640625	3.55798303786598e-05	\\
-2992.1875	3.38458231433487e-05	\\
-2991.2109375	3.66591970492321e-05	\\
-2990.234375	3.52067652195916e-05	\\
-2989.2578125	3.49955794604597e-05	\\
-2988.28125	3.43229181119845e-05	\\
-2987.3046875	3.37711198595923e-05	\\
-2986.328125	3.51722513444219e-05	\\
-2985.3515625	3.34209306064913e-05	\\
-2984.375	3.41738088449425e-05	\\
-2983.3984375	3.31876717927812e-05	\\
-2982.421875	3.55395070606943e-05	\\
-2981.4453125	3.45071335286888e-05	\\
-2980.46875	3.31092354932919e-05	\\
-2979.4921875	3.06653720073664e-05	\\
-2978.515625	3.50744233105892e-05	\\
-2977.5390625	3.20750176713171e-05	\\
-2976.5625	3.37842683727197e-05	\\
-2975.5859375	3.27451394553099e-05	\\
-2974.609375	3.006088407224e-05	\\
-2973.6328125	3.20838743279415e-05	\\
-2972.65625	3.35478250191845e-05	\\
-2971.6796875	3.02718449072941e-05	\\
-2970.703125	3.21119271639894e-05	\\
-2969.7265625	3.4647940248515e-05	\\
-2968.75	3.18275983399466e-05	\\
-2967.7734375	3.12633624641551e-05	\\
-2966.796875	3.16419855895328e-05	\\
-2965.8203125	3.10314397336931e-05	\\
-2964.84375	3.20206027412695e-05	\\
-2963.8671875	3.4211753234178e-05	\\
-2962.890625	3.33614105299385e-05	\\
-2961.9140625	3.17841212795832e-05	\\
-2960.9375	3.35909944086898e-05	\\
-2959.9609375	3.26270096809615e-05	\\
-2958.984375	3.24683028209159e-05	\\
-2958.0078125	2.91059458138059e-05	\\
-2957.03125	3.22840972593621e-05	\\
-2956.0546875	2.95491431165958e-05	\\
-2955.078125	3.21824445764179e-05	\\
-2954.1015625	3.03480574575185e-05	\\
-2953.125	3.16714631788315e-05	\\
-2952.1484375	3.0477265796708e-05	\\
-2951.171875	3.19512029179122e-05	\\
-2950.1953125	3.00467036549529e-05	\\
-2949.21875	2.90684638360936e-05	\\
-2948.2421875	3.07840225103946e-05	\\
-2947.265625	3.04956523652009e-05	\\
-2946.2890625	3.20563311524499e-05	\\
-2945.3125	3.00742920815551e-05	\\
-2944.3359375	3.02396345883722e-05	\\
-2943.359375	2.89798110588643e-05	\\
-2942.3828125	3.22190518538691e-05	\\
-2941.40625	3.11234162095148e-05	\\
-2940.4296875	3.21177097596684e-05	\\
-2939.453125	3.08174649986905e-05	\\
-2938.4765625	2.89866460609956e-05	\\
-2937.5	3.08167883916269e-05	\\
-2936.5234375	3.01528616472492e-05	\\
-2935.546875	2.96260426039432e-05	\\
-2934.5703125	3.1616122294341e-05	\\
-2933.59375	2.96363007133559e-05	\\
-2932.6171875	2.84319915149258e-05	\\
-2931.640625	2.83630078696926e-05	\\
-2930.6640625	2.78039471465397e-05	\\
-2929.6875	2.79792965013122e-05	\\
-2928.7109375	2.76427798676936e-05	\\
-2927.734375	2.59223276374013e-05	\\
-2926.7578125	2.84188396796918e-05	\\
-2925.78125	2.91456923923447e-05	\\
-2924.8046875	2.95154623226998e-05	\\
-2923.828125	2.81620550109213e-05	\\
-2922.8515625	2.71922689958023e-05	\\
-2921.875	2.52889115499297e-05	\\
-2920.8984375	2.69717188334307e-05	\\
-2919.921875	2.81698670344649e-05	\\
-2918.9453125	2.72273399597698e-05	\\
-2917.96875	2.64301198117838e-05	\\
-2916.9921875	2.90057703901702e-05	\\
-2916.015625	2.67735469471944e-05	\\
-2915.0390625	2.73676063977922e-05	\\
-2914.0625	2.80756130576166e-05	\\
-2913.0859375	2.59572095776142e-05	\\
-2912.109375	2.72870617082034e-05	\\
-2911.1328125	2.74240600317477e-05	\\
-2910.15625	2.89965254036668e-05	\\
-2909.1796875	2.73149568280668e-05	\\
-2908.203125	2.85664567509944e-05	\\
-2907.2265625	2.7411408984877e-05	\\
-2906.25	2.91695103712118e-05	\\
-2905.2734375	2.61109893210448e-05	\\
-2904.296875	2.69731387772598e-05	\\
-2903.3203125	2.82875236557546e-05	\\
-2902.34375	2.66499932861912e-05	\\
-2901.3671875	2.62898671442619e-05	\\
-2900.390625	2.91977893452364e-05	\\
-2899.4140625	2.56913600366839e-05	\\
-2898.4375	2.87578666099216e-05	\\
-2897.4609375	2.51371547219039e-05	\\
-2896.484375	2.40348037200593e-05	\\
-2895.5078125	2.72072516238822e-05	\\
-2894.53125	2.75973089465874e-05	\\
-2893.5546875	2.63675875714043e-05	\\
-2892.578125	2.37848453351513e-05	\\
-2891.6015625	2.81713875274938e-05	\\
-2890.625	2.57899006720213e-05	\\
-2889.6484375	2.64349509926439e-05	\\
-2888.671875	2.66496521406514e-05	\\
-2887.6953125	2.56467822682382e-05	\\
-2886.71875	2.36326647668961e-05	\\
-2885.7421875	2.64003829342426e-05	\\
-2884.765625	2.85854613054474e-05	\\
-2883.7890625	2.57887487932805e-05	\\
-2882.8125	2.57311011807572e-05	\\
-2881.8359375	2.48974878120132e-05	\\
-2880.859375	2.5491338309409e-05	\\
-2879.8828125	2.5894454712245e-05	\\
-2878.90625	2.4974225568679e-05	\\
-2877.9296875	2.24347293415341e-05	\\
-2876.953125	2.63354965106815e-05	\\
-2875.9765625	2.45884976249619e-05	\\
-2875	2.24207030476331e-05	\\
-2874.0234375	2.58824183168286e-05	\\
-2873.046875	2.49820116563126e-05	\\
-2872.0703125	2.5155250227266e-05	\\
-2871.09375	2.49145473849732e-05	\\
-2870.1171875	2.25030481752657e-05	\\
-2869.140625	2.29724685636056e-05	\\
-2868.1640625	2.04852485450852e-05	\\
-2867.1875	2.11104093681068e-05	\\
-2866.2109375	2.22965343833266e-05	\\
-2865.234375	2.32279093384023e-05	\\
-2864.2578125	2.21116145555482e-05	\\
-2863.28125	2.31062378217618e-05	\\
-2862.3046875	2.45190572178546e-05	\\
-2861.328125	2.28077067684659e-05	\\
-2860.3515625	2.5199249998613e-05	\\
-2859.375	2.07211815465533e-05	\\
-2858.3984375	2.25131275534423e-05	\\
-2857.421875	2.27318784877603e-05	\\
-2856.4453125	2.2777710820364e-05	\\
-2855.46875	2.31074731967126e-05	\\
-2854.4921875	1.97354202426443e-05	\\
-2853.515625	2.22896888899812e-05	\\
-2852.5390625	2.22874679961353e-05	\\
-2851.5625	2.41782136579354e-05	\\
-2850.5859375	2.14385094516985e-05	\\
-2849.609375	1.99860278318613e-05	\\
-2848.6328125	2.15999374816991e-05	\\
-2847.65625	2.15989688598337e-05	\\
-2846.6796875	2.19402666383716e-05	\\
-2845.703125	2.25174696299743e-05	\\
-2844.7265625	2.12695342462464e-05	\\
-2843.75	2.07191512405888e-05	\\
-2842.7734375	2.01861587995596e-05	\\
-2841.796875	1.9532532769943e-05	\\
-2840.8203125	2.22279319383981e-05	\\
-2839.84375	1.91810935473833e-05	\\
-2838.8671875	1.63137767573483e-05	\\
-2837.890625	1.9040926043453e-05	\\
-2836.9140625	1.8588421021303e-05	\\
-2835.9375	1.73891606146768e-05	\\
-2834.9609375	1.86573096084435e-05	\\
-2833.984375	2.0018075183407e-05	\\
-2833.0078125	1.86448011905218e-05	\\
-2832.03125	1.66270346502037e-05	\\
-2831.0546875	1.77717863070018e-05	\\
-2830.078125	1.84882080554215e-05	\\
-2829.1015625	1.50982462721522e-05	\\
-2828.125	1.69926152048181e-05	\\
-2827.1484375	1.69150781613706e-05	\\
-2826.171875	1.6430214099176e-05	\\
-2825.1953125	1.57504091190146e-05	\\
-2824.21875	1.5628463249399e-05	\\
-2823.2421875	1.76579919096841e-05	\\
-2822.265625	1.53274882067695e-05	\\
-2821.2890625	1.5472487022605e-05	\\
-2820.3125	1.55433702469114e-05	\\
-2819.3359375	1.56724254659534e-05	\\
-2818.359375	1.47868247737146e-05	\\
-2817.3828125	1.35819841673733e-05	\\
-2816.40625	1.25968040528604e-05	\\
-2815.4296875	1.20932023779201e-05	\\
-2814.453125	1.42819864437164e-05	\\
-2813.4765625	1.34617172041096e-05	\\
-2812.5	1.40301436298883e-05	\\
-2811.5234375	1.5706333366667e-05	\\
-2810.546875	1.26345082234531e-05	\\
-2809.5703125	1.23722718267516e-05	\\
-2808.59375	1.19210868592869e-05	\\
-2807.6171875	1.09404359442719e-05	\\
-2806.640625	1.22629709475767e-05	\\
-2805.6640625	1.07906132480791e-05	\\
-2804.6875	1.1248953724048e-05	\\
-2803.7109375	8.0417399004833e-06	\\
-2802.734375	8.83926203810687e-06	\\
-2801.7578125	9.29207722803193e-06	\\
-2800.78125	6.59141493185466e-06	\\
-2799.8046875	8.11876518259692e-06	\\
-2798.828125	8.52291840486389e-06	\\
-2797.8515625	9.06161809911397e-06	\\
-2796.875	6.68957298153794e-06	\\
-2795.8984375	7.04628804078772e-06	\\
-2794.921875	4.47805517102328e-06	\\
-2793.9453125	4.61256042334178e-06	\\
-2792.96875	6.46930012581207e-06	\\
-2791.9921875	3.95941763626369e-06	\\
-2791.015625	5.32058233318372e-06	\\
-2790.0390625	4.65238290419306e-06	\\
-2789.0625	4.76054769820793e-06	\\
-2788.0859375	3.4056938431505e-06	\\
-2787.109375	5.33525621829297e-06	\\
-2786.1328125	5.71295133130674e-06	\\
-2785.15625	5.75360819050716e-06	\\
-2784.1796875	6.60110303279168e-06	\\
-2783.203125	4.6488503339937e-06	\\
-2782.2265625	6.88067731256454e-06	\\
-2781.25	5.18795339126729e-06	\\
-2780.2734375	6.3531518841255e-06	\\
-2779.296875	6.90879827652349e-06	\\
-2778.3203125	6.96818596625469e-06	\\
-2777.34375	7.9908309304935e-06	\\
-2776.3671875	7.78941524144465e-06	\\
-2775.390625	9.73047541741963e-06	\\
-2774.4140625	9.00263886033803e-06	\\
-2773.4375	1.13418061221597e-05	\\
-2772.4609375	1.03970863084423e-05	\\
-2771.484375	1.2906971214151e-05	\\
-2770.5078125	1.43858671461849e-05	\\
-2769.53125	1.34762431949673e-05	\\
-2768.5546875	1.51244529771578e-05	\\
-2767.578125	1.53336104862513e-05	\\
-2766.6015625	1.72846259562388e-05	\\
-2765.625	1.79197473278086e-05	\\
-2764.6484375	1.6916361803015e-05	\\
-2763.671875	1.80055117435095e-05	\\
-2762.6953125	1.92636371961007e-05	\\
-2761.71875	2.01470425470103e-05	\\
-2760.7421875	2.41372920192767e-05	\\
-2759.765625	2.57116091881735e-05	\\
-2758.7890625	2.57999216699465e-05	\\
-2757.8125	2.4293384993106e-05	\\
-2756.8359375	2.47603055622402e-05	\\
-2755.859375	2.61524320664526e-05	\\
-2754.8828125	2.84602719316375e-05	\\
-2753.90625	2.61388339025144e-05	\\
-2752.9296875	2.95601502421917e-05	\\
-2751.953125	2.83916960134492e-05	\\
-2750.9765625	2.87358246067489e-05	\\
-2750	3.14563673733721e-05	\\
-2749.0234375	3.14264091319694e-05	\\
-2748.046875	3.1410894608532e-05	\\
-2747.0703125	3.45561577383664e-05	\\
-2746.09375	3.63062695923061e-05	\\
-2745.1171875	3.4756020510392e-05	\\
-2744.140625	3.65669817012403e-05	\\
-2743.1640625	3.80909150184898e-05	\\
-2742.1875	3.71402606772492e-05	\\
-2741.2109375	3.68128538454107e-05	\\
-2740.234375	4.10145131460844e-05	\\
-2739.2578125	4.04696694038125e-05	\\
-2738.28125	4.12983239008845e-05	\\
-2737.3046875	4.27843417189008e-05	\\
-2736.328125	4.24397695734672e-05	\\
-2735.3515625	4.50347217986838e-05	\\
-2734.375	4.40464329005898e-05	\\
-2733.3984375	4.79346618336419e-05	\\
-2732.421875	4.5379277327505e-05	\\
-2731.4453125	4.7882696811758e-05	\\
-2730.46875	4.65479803549679e-05	\\
-2729.4921875	4.84008791347662e-05	\\
-2728.515625	4.88320007938666e-05	\\
-2727.5390625	4.96562030829456e-05	\\
-2726.5625	5.06759606659665e-05	\\
-2725.5859375	4.92949004743246e-05	\\
-2724.609375	5.19165234564066e-05	\\
-2723.6328125	5.19262484279238e-05	\\
-2722.65625	5.13776111452637e-05	\\
-2721.6796875	5.54385122807349e-05	\\
-2720.703125	5.40475182386947e-05	\\
-2719.7265625	5.62549725860629e-05	\\
-2718.75	5.75060661684315e-05	\\
-2717.7734375	5.51643621046382e-05	\\
-2716.796875	5.90815675532316e-05	\\
-2715.8203125	5.70621654175155e-05	\\
-2714.84375	5.82528145075601e-05	\\
-2713.8671875	5.5061774605817e-05	\\
-2712.890625	5.89150566287514e-05	\\
-2711.9140625	6.03190674156566e-05	\\
-2710.9375	5.95500166320416e-05	\\
-2709.9609375	5.9019589856272e-05	\\
-2708.984375	6.04725907426022e-05	\\
-2708.0078125	5.92461677880397e-05	\\
-2707.03125	5.90078153252183e-05	\\
-2706.0546875	5.8228232338246e-05	\\
-2705.078125	6.3652546476803e-05	\\
-2704.1015625	6.15743857147424e-05	\\
-2703.125	5.95091481584636e-05	\\
-2702.1484375	5.98562470888091e-05	\\
-2701.171875	5.77924757883148e-05	\\
-2700.1953125	6.3589288802123e-05	\\
-2699.21875	6.47186761169003e-05	\\
-2698.2421875	6.02884953446552e-05	\\
-2697.265625	5.9542762896229e-05	\\
-2696.2890625	6.33219025270472e-05	\\
-2695.3125	6.27116925085907e-05	\\
-2694.3359375	6.05155094762387e-05	\\
-2693.359375	6.3191183253917e-05	\\
-2692.3828125	6.26809878600806e-05	\\
-2691.40625	6.14424595392286e-05	\\
-2690.4296875	6.21844522336876e-05	\\
-2689.453125	6.33509800589168e-05	\\
-2688.4765625	6.40628768043421e-05	\\
-2687.5	6.31991429245481e-05	\\
-2686.5234375	6.11389851911132e-05	\\
-2685.546875	6.36433044547776e-05	\\
-2684.5703125	6.29996090353164e-05	\\
-2683.59375	6.1110322690789e-05	\\
-2682.6171875	6.15926092567787e-05	\\
-2681.640625	6.11702506341275e-05	\\
-2680.6640625	6.23499743595944e-05	\\
-2679.6875	6.06451642588401e-05	\\
-2678.7109375	6.02398795798406e-05	\\
-2677.734375	6.22089538998144e-05	\\
-2676.7578125	6.23972416529461e-05	\\
-2675.78125	6.32135587239743e-05	\\
-2674.8046875	6.32142052703424e-05	\\
-2673.828125	6.27987149562551e-05	\\
-2672.8515625	6.00640861989559e-05	\\
-2671.875	6.2229752215193e-05	\\
-2670.8984375	6.08282493766155e-05	\\
-2669.921875	6.18543474309319e-05	\\
-2668.9453125	6.10402090064819e-05	\\
-2667.96875	6.05456594744584e-05	\\
-2666.9921875	5.97326057104017e-05	\\
-2666.015625	6.35205500626181e-05	\\
-2665.0390625	5.91332561262359e-05	\\
-2664.0625	6.19213450309736e-05	\\
-2663.0859375	6.00388612796659e-05	\\
-2662.109375	6.0990002674834e-05	\\
-2661.1328125	6.07532873545557e-05	\\
-2660.15625	5.99288091746598e-05	\\
-2659.1796875	5.95636987763452e-05	\\
-2658.203125	6.08177582622255e-05	\\
-2657.2265625	6.19350700928416e-05	\\
-2656.25	5.8487169602072e-05	\\
-2655.2734375	5.96184042860095e-05	\\
-2654.296875	5.91185238518446e-05	\\
-2653.3203125	5.79459346726071e-05	\\
-2652.34375	6.09600261141477e-05	\\
-2651.3671875	6.06169423833403e-05	\\
-2650.390625	5.95636766509965e-05	\\
-2649.4140625	5.88077534991932e-05	\\
-2648.4375	5.90997865889641e-05	\\
-2647.4609375	5.84944141644563e-05	\\
-2646.484375	6.12886965052709e-05	\\
-2645.5078125	5.87224116523434e-05	\\
-2644.53125	5.91540112509784e-05	\\
-2643.5546875	6.07339106908228e-05	\\
-2642.578125	5.89877785355458e-05	\\
-2641.6015625	6.12537525386514e-05	\\
-2640.625	5.92985881941495e-05	\\
-2639.6484375	5.91109302340683e-05	\\
-2638.671875	5.91523871597651e-05	\\
-2637.6953125	6.0033475931408e-05	\\
-2636.71875	5.89847305428511e-05	\\
-2635.7421875	5.91665159033875e-05	\\
-2634.765625	6.2240254496282e-05	\\
-2633.7890625	5.81114998926309e-05	\\
-2632.8125	6.1729917132938e-05	\\
-2631.8359375	6.07995523078548e-05	\\
-2630.859375	6.27867582863289e-05	\\
-2629.8828125	6.1536943372396e-05	\\
-2628.90625	6.17524567337805e-05	\\
-2627.9296875	6.34225034998852e-05	\\
-2626.953125	5.8379797982449e-05	\\
-2625.9765625	6.13349773658208e-05	\\
-2625	6.0879424309621e-05	\\
-2624.0234375	6.28645794137042e-05	\\
-2623.046875	6.20900933273558e-05	\\
-2622.0703125	6.04786271015264e-05	\\
-2621.09375	6.1667338730725e-05	\\
-2620.1171875	6.0488470707652e-05	\\
-2619.140625	6.32988771326159e-05	\\
-2618.1640625	6.25034087428042e-05	\\
-2617.1875	6.33639279678864e-05	\\
-2616.2109375	6.24049632808048e-05	\\
-2615.234375	6.41424328593078e-05	\\
-2614.2578125	6.39659861757619e-05	\\
-2613.28125	6.40050600057849e-05	\\
-2612.3046875	6.17580211535899e-05	\\
-2611.328125	6.67329567828135e-05	\\
-2610.3515625	6.34435554184043e-05	\\
-2609.375	6.38982074387444e-05	\\
-2608.3984375	6.56137266464299e-05	\\
-2607.421875	6.42963248202088e-05	\\
-2606.4453125	6.60313754464886e-05	\\
-2605.46875	6.27455348781003e-05	\\
-2604.4921875	6.39672016510536e-05	\\
-2603.515625	6.41770086405799e-05	\\
-2602.5390625	6.64103865287925e-05	\\
-2601.5625	6.85495931306272e-05	\\
-2600.5859375	6.4040514397562e-05	\\
-2599.609375	6.41463314192023e-05	\\
-2598.6328125	6.33344971949803e-05	\\
-2597.65625	6.73554499657101e-05	\\
-2596.6796875	6.50439206588159e-05	\\
-2595.703125	6.75577291480776e-05	\\
-2594.7265625	6.63981222897575e-05	\\
-2593.75	6.88498650415113e-05	\\
-2592.7734375	6.81466810426295e-05	\\
-2591.796875	6.6510108876538e-05	\\
-2590.8203125	6.79002125504311e-05	\\
-2589.84375	6.88454939586813e-05	\\
-2588.8671875	6.97944085856299e-05	\\
-2587.890625	6.88366210858384e-05	\\
-2586.9140625	6.88470207536479e-05	\\
-2585.9375	6.83073191699276e-05	\\
-2584.9609375	7.11932218481226e-05	\\
-2583.984375	6.90076234969254e-05	\\
-2583.0078125	7.07741604547593e-05	\\
-2582.03125	6.9844652975792e-05	\\
-2581.0546875	7.17559359775262e-05	\\
-2580.078125	7.44404283369459e-05	\\
-2579.1015625	7.10191034333761e-05	\\
-2578.125	7.38508084691082e-05	\\
-2577.1484375	7.32555977499827e-05	\\
-2576.171875	7.40252887020458e-05	\\
-2575.1953125	7.46726577142789e-05	\\
-2574.21875	7.30622307084084e-05	\\
-2573.2421875	7.61540903811316e-05	\\
-2572.265625	7.66114418546317e-05	\\
-2571.2890625	7.79030155512209e-05	\\
-2570.3125	7.62788727393795e-05	\\
-2569.3359375	7.81574261596619e-05	\\
-2568.359375	7.93813406584252e-05	\\
-2567.3828125	7.84391757981866e-05	\\
-2566.40625	8.25036102782437e-05	\\
-2565.4296875	8.0001786958775e-05	\\
-2564.453125	7.91778840820043e-05	\\
-2563.4765625	8.01726597524979e-05	\\
-2562.5	8.0721448724292e-05	\\
-2561.5234375	8.33630055417345e-05	\\
-2560.546875	8.52217855057187e-05	\\
-2559.5703125	8.44624320250347e-05	\\
-2558.59375	8.56725938179444e-05	\\
-2557.6171875	8.63639699419615e-05	\\
-2556.640625	8.40189765117115e-05	\\
-2555.6640625	8.53806560909767e-05	\\
-2554.6875	8.8408174293058e-05	\\
-2553.7109375	8.59155175944784e-05	\\
-2552.734375	8.75273950144497e-05	\\
-2551.7578125	8.94871956912889e-05	\\
-2550.78125	8.93910683361747e-05	\\
-2549.8046875	9.13361233522045e-05	\\
-2548.828125	8.85755259370299e-05	\\
-2547.8515625	9.14113646958314e-05	\\
-2546.875	9.23781330853302e-05	\\
-2545.8984375	9.08358788251489e-05	\\
-2544.921875	9.35495727646223e-05	\\
-2543.9453125	9.17526859852209e-05	\\
-2542.96875	9.36991144009285e-05	\\
-2541.9921875	9.51861566197496e-05	\\
-2541.015625	9.06244644064752e-05	\\
-2540.0390625	9.44267642004576e-05	\\
-2539.0625	9.27854536946706e-05	\\
-2538.0859375	9.37880521452454e-05	\\
-2537.109375	9.57815892831248e-05	\\
-2536.1328125	9.42345148949163e-05	\\
-2535.15625	9.70946914278222e-05	\\
-2534.1796875	9.44541247991584e-05	\\
-2533.203125	9.4385422426702e-05	\\
-2532.2265625	9.62247442302334e-05	\\
-2531.25	9.62084930439981e-05	\\
-2530.2734375	9.51526227601056e-05	\\
-2529.296875	9.89454583731273e-05	\\
-2528.3203125	9.76609749764191e-05	\\
-2527.34375	9.62730879983636e-05	\\
-2526.3671875	9.78124405442028e-05	\\
-2525.390625	9.70618682791148e-05	\\
-2524.4140625	9.77644514722345e-05	\\
-2523.4375	9.88884660033184e-05	\\
-2522.4609375	9.44862670909544e-05	\\
-2521.484375	9.92485190717121e-05	\\
-2520.5078125	9.83402580349909e-05	\\
-2519.53125	9.76189010208978e-05	\\
-2518.5546875	0.000100660494946379	\\
-2517.578125	0.000104214578288983	\\
-2516.6015625	9.95708579694111e-05	\\
-2515.625	9.9012350939589e-05	\\
-2514.6484375	0.000100086970998963	\\
-2513.671875	0.000103217681118998	\\
-2512.6953125	0.000100891417139867	\\
-2511.71875	9.91588367597672e-05	\\
-2510.7421875	9.89855997643318e-05	\\
-2509.765625	0.00010070009074617	\\
-2508.7890625	0.000100860810213673	\\
-2507.8125	0.000100116403591124	\\
-2506.8359375	0.000100079084938112	\\
-2505.859375	9.77810900161205e-05	\\
-2504.8828125	0.000100770047798484	\\
-2503.90625	9.72915243117223e-05	\\
-2502.9296875	9.79519636323625e-05	\\
-2501.953125	9.8193946751077e-05	\\
-2500.9765625	9.821227009821e-05	\\
-2500	0.000100859541030438	\\
-2499.0234375	9.61417392388778e-05	\\
-2498.046875	9.83844199476312e-05	\\
-2497.0703125	9.63466060407101e-05	\\
-2496.09375	9.74383622062499e-05	\\
-2495.1171875	9.84098718368251e-05	\\
-2494.140625	9.33582943444286e-05	\\
-2493.1640625	9.86823014567918e-05	\\
-2492.1875	9.4108765573268e-05	\\
-2491.2109375	9.60897506325271e-05	\\
-2490.234375	9.5743549818154e-05	\\
-2489.2578125	9.77372204364152e-05	\\
-2488.28125	9.27173002549813e-05	\\
-2487.3046875	9.41622512171405e-05	\\
-2486.328125	9.7547067168731e-05	\\
-2485.3515625	9.46489279320432e-05	\\
-2484.375	9.48285673234165e-05	\\
-2483.3984375	9.74229555698369e-05	\\
-2482.421875	9.52298477153067e-05	\\
-2481.4453125	9.83592936038912e-05	\\
-2480.46875	9.64511749475608e-05	\\
-2479.4921875	9.72918023537893e-05	\\
-2478.515625	9.89746274086731e-05	\\
-2477.5390625	0.000102378897780925	\\
-2476.5625	9.86118784782846e-05	\\
-2475.5859375	9.93687762512964e-05	\\
-2474.609375	9.6494798275672e-05	\\
-2473.6328125	0.000101877429124151	\\
-2472.65625	9.94044387699524e-05	\\
-2471.6796875	0.00010157253002456	\\
-2470.703125	9.79902657413933e-05	\\
-2469.7265625	9.83143516717594e-05	\\
-2468.75	0.000102299422014962	\\
-2467.7734375	0.000102554762379566	\\
-2466.796875	0.000102481380322052	\\
-2465.8203125	0.000102819797783118	\\
-2464.84375	0.000103053519967276	\\
-2463.8671875	0.000102044768122446	\\
-2462.890625	0.000103759981014685	\\
-2461.9140625	9.98847303555728e-05	\\
-2460.9375	0.000102797294664787	\\
-2459.9609375	0.0001026797811742	\\
-2458.984375	0.000103479201507491	\\
-2458.0078125	0.000101944762084101	\\
-2457.03125	0.000101681928969399	\\
-2456.0546875	0.00010282331909627	\\
-2455.078125	0.0001042187214992	\\
-2454.1015625	0.000104426040781318	\\
-2453.125	0.000103618836810762	\\
-2452.1484375	0.000105135602387606	\\
-2451.171875	0.000104558896260969	\\
-2450.1953125	0.000105561615378124	\\
-2449.21875	0.000103890721605507	\\
-2448.2421875	0.000105325635017849	\\
-2447.265625	0.000102720644228615	\\
-2446.2890625	0.000103855111209127	\\
-2445.3125	0.000105163157910068	\\
-2444.3359375	0.000101939907912143	\\
-2443.359375	0.000101598222198977	\\
-2442.3828125	0.000101675454276527	\\
-2441.40625	0.000104148805206431	\\
-2440.4296875	0.000102500515739036	\\
-2439.453125	0.000103825894759509	\\
-2438.4765625	0.000102160905153071	\\
-2437.5	0.000100278812976945	\\
-2436.5234375	0.000102847008067285	\\
-2435.546875	0.000100422170162636	\\
-2434.5703125	0.000104134681947604	\\
-2433.59375	0.000101174573136989	\\
-2432.6171875	0.000103339625348209	\\
-2431.640625	0.00010247910336141	\\
-2430.6640625	0.000103837366126435	\\
-2429.6875	0.000101374034607	\\
-2428.7109375	0.000102186177105851	\\
-2427.734375	0.000101161036697659	\\
-2426.7578125	0.000101598764670755	\\
-2425.78125	0.000101400149838794	\\
-2424.8046875	0.000100633575811849	\\
-2423.828125	9.97415500498726e-05	\\
-2422.8515625	0.000100384015072924	\\
-2421.875	0.000100461538931697	\\
-2420.8984375	9.91799589105114e-05	\\
-2419.921875	0.000102720368804156	\\
-2418.9453125	0.000101592067115656	\\
-2417.96875	0.000102409667788226	\\
-2416.9921875	0.000104713123185143	\\
-2416.015625	0.000103012601252424	\\
-2415.0390625	0.000101575799121024	\\
-2414.0625	0.000100753228377195	\\
-2413.0859375	0.000101599883689566	\\
-2412.109375	9.82128046156838e-05	\\
-2411.1328125	0.00010171852719281	\\
-2410.15625	0.000102270651839353	\\
-2409.1796875	0.000100715537334478	\\
-2408.203125	0.000102595845925478	\\
-2407.2265625	0.000100771541657144	\\
-2406.25	0.000102387102359045	\\
-2405.2734375	0.00010298984621674	\\
-2404.296875	9.799615217044e-05	\\
-2403.3203125	0.0001016562055442	\\
-2402.34375	0.000100778316162336	\\
-2401.3671875	0.000100303031603843	\\
-2400.390625	0.000104787098940278	\\
-2399.4140625	0.000102360235369762	\\
-2398.4375	0.000102383387451979	\\
-2397.4609375	0.000103080018294023	\\
-2396.484375	0.000101093771746326	\\
-2395.5078125	0.000101963834081676	\\
-2394.53125	0.000101111501920207	\\
-2393.5546875	0.00010145289809158	\\
-2392.578125	0.000103081613973264	\\
-2391.6015625	0.000100766135312356	\\
-2390.625	0.000100324026402956	\\
-2389.6484375	9.85136696744261e-05	\\
-2388.671875	0.000104128091282532	\\
-2387.6953125	9.96470865394166e-05	\\
-2386.71875	9.88893257132716e-05	\\
-2385.7421875	0.000102353130657187	\\
-2384.765625	0.000103291134096047	\\
-2383.7890625	9.94978763325869e-05	\\
-2382.8125	0.000102816859888173	\\
-2381.8359375	9.8468778216327e-05	\\
-2380.859375	0.000100583379719652	\\
-2379.8828125	0.000103712190790743	\\
-2378.90625	0.000102454005390934	\\
-2377.9296875	0.000102992343920228	\\
-2376.953125	0.000102512966001606	\\
-2375.9765625	0.000103440830263637	\\
-2375	0.000104205765148337	\\
-2374.0234375	0.000101873786541552	\\
-2373.046875	0.000104805654687831	\\
-2372.0703125	0.000102631163775313	\\
-2371.09375	0.000101795999053404	\\
-2370.1171875	0.000100095920708672	\\
-2369.140625	0.000100361195592612	\\
-2368.1640625	9.91562619097997e-05	\\
-2367.1875	0.000101357916859446	\\
-2366.2109375	0.000100315693337442	\\
-2365.234375	0.000101919862846937	\\
-2364.2578125	0.000101847739957337	\\
-2363.28125	0.000100752536914403	\\
-2362.3046875	9.94325151068287e-05	\\
-2361.328125	9.90152324257982e-05	\\
-2360.3515625	0.000100978946664375	\\
-2359.375	9.99446525307282e-05	\\
-2358.3984375	0.000100612314711143	\\
-2357.421875	9.82675179505143e-05	\\
-2356.4453125	9.93255004625544e-05	\\
-2355.46875	9.87036679310306e-05	\\
-2354.4921875	0.00010015920720545	\\
-2353.515625	9.7677656618399e-05	\\
-2352.5390625	9.95462851854999e-05	\\
-2351.5625	9.69936882478786e-05	\\
-2350.5859375	0.000100087046496921	\\
-2349.609375	9.67377036804223e-05	\\
-2348.6328125	9.9292865091531e-05	\\
-2347.65625	9.95654949951141e-05	\\
-2346.6796875	9.78933644023135e-05	\\
-2345.703125	9.81090194639605e-05	\\
-2344.7265625	9.67215073590529e-05	\\
-2343.75	9.69857043332011e-05	\\
-2342.7734375	9.75987325866401e-05	\\
-2341.796875	9.63462166868793e-05	\\
-2340.8203125	9.53299667398582e-05	\\
-2339.84375	9.63994818576914e-05	\\
-2338.8671875	9.79240259036626e-05	\\
-2337.890625	9.81591065641531e-05	\\
-2336.9140625	9.48386585721819e-05	\\
-2335.9375	9.730036908589e-05	\\
-2334.9609375	9.74627620713094e-05	\\
-2333.984375	9.60831908092495e-05	\\
-2333.0078125	9.66971813614318e-05	\\
-2332.03125	9.76804425369982e-05	\\
-2331.0546875	9.51883637490173e-05	\\
-2330.078125	9.50902630314888e-05	\\
-2329.1015625	9.37624646481355e-05	\\
-2328.125	9.75923729626399e-05	\\
-2327.1484375	9.46121532208164e-05	\\
-2326.171875	9.56220862502217e-05	\\
-2325.1953125	9.68713193990793e-05	\\
-2324.21875	9.52859419870966e-05	\\
-2323.2421875	9.71478768893694e-05	\\
-2322.265625	9.7486144367916e-05	\\
-2321.2890625	9.36207683806994e-05	\\
-2320.3125	9.59858550117263e-05	\\
-2319.3359375	9.53057567768954e-05	\\
-2318.359375	9.72240811488257e-05	\\
-2317.3828125	9.69635930508296e-05	\\
-2316.40625	9.77546614346884e-05	\\
-2315.4296875	9.68330469975362e-05	\\
-2314.453125	9.72542431967565e-05	\\
-2313.4765625	9.38454379983721e-05	\\
-2312.5	9.78842334322295e-05	\\
-2311.5234375	9.72949405122124e-05	\\
-2310.546875	9.70022206233319e-05	\\
-2309.5703125	9.43924768075106e-05	\\
-2308.59375	9.48647156587263e-05	\\
-2307.6171875	9.55254994434541e-05	\\
-2306.640625	9.66879350730257e-05	\\
-2305.6640625	9.5716329566122e-05	\\
-2304.6875	9.50616074625862e-05	\\
-2303.7109375	9.71964722527125e-05	\\
-2302.734375	9.61929683903822e-05	\\
-2301.7578125	9.9757764655337e-05	\\
-2300.78125	9.68070601644347e-05	\\
-2299.8046875	9.94147564582063e-05	\\
-2298.828125	9.31862353565122e-05	\\
-2297.8515625	9.84774677940812e-05	\\
-2296.875	9.75222819238834e-05	\\
-2295.8984375	9.66595052453248e-05	\\
-2294.921875	9.85107435856176e-05	\\
-2293.9453125	9.51425463715111e-05	\\
-2292.96875	9.56617478739416e-05	\\
-2291.9921875	9.62661024749451e-05	\\
-2291.015625	9.66692588362785e-05	\\
-2290.0390625	9.48449565120852e-05	\\
-2289.0625	9.83022565913137e-05	\\
-2288.0859375	9.59859808542955e-05	\\
-2287.109375	9.80354060436862e-05	\\
-2286.1328125	9.3855383917094e-05	\\
-2285.15625	9.36472768089131e-05	\\
-2284.1796875	9.50191914087404e-05	\\
-2283.203125	9.22211486077852e-05	\\
-2282.2265625	9.55525401740291e-05	\\
-2281.25	9.60188384497371e-05	\\
-2280.2734375	9.8008919103644e-05	\\
-2279.296875	9.32232714639678e-05	\\
-2278.3203125	9.5491912475743e-05	\\
-2277.34375	9.43089349175059e-05	\\
-2276.3671875	9.40517600991387e-05	\\
-2275.390625	9.46487105237932e-05	\\
-2274.4140625	9.24872542878518e-05	\\
-2273.4375	9.20953932853624e-05	\\
-2272.4609375	9.24825435231931e-05	\\
-2271.484375	8.92219488846054e-05	\\
-2270.5078125	9.12286050105703e-05	\\
-2269.53125	8.88653211696448e-05	\\
-2268.5546875	8.79610189605964e-05	\\
-2267.578125	8.72277285343336e-05	\\
-2266.6015625	8.69201478221907e-05	\\
-2265.625	8.8116332279915e-05	\\
-2264.6484375	8.59523151902612e-05	\\
-2263.671875	8.69825220854095e-05	\\
-2262.6953125	8.67317815786079e-05	\\
-2261.71875	8.73208107893347e-05	\\
-2260.7421875	8.59349553173324e-05	\\
-2259.765625	8.46246311097179e-05	\\
-2258.7890625	8.71824827697026e-05	\\
-2257.8125	8.38918840054419e-05	\\
-2256.8359375	8.3715453790133e-05	\\
-2255.859375	8.03634343515864e-05	\\
-2254.8828125	8.0005423959318e-05	\\
-2253.90625	8.11038953783321e-05	\\
-2252.9296875	8.28587224622753e-05	\\
-2251.953125	7.89323113768071e-05	\\
-2250.9765625	7.78701663490332e-05	\\
-2250	7.89804013756e-05	\\
-2249.0234375	7.28866095614558e-05	\\
-2248.046875	7.59850040599517e-05	\\
-2247.0703125	7.59731753650379e-05	\\
-2246.09375	7.50867607901345e-05	\\
-2245.1171875	7.43756114690142e-05	\\
-2244.140625	7.32380694957165e-05	\\
-2243.1640625	6.98773227050169e-05	\\
-2242.1875	7.15415113322703e-05	\\
-2241.2109375	6.74757786578053e-05	\\
-2240.234375	6.71791079295056e-05	\\
-2239.2578125	6.57984791118504e-05	\\
-2238.28125	7.00977979814775e-05	\\
-2237.3046875	6.49628937091456e-05	\\
-2236.328125	6.41269591064627e-05	\\
-2235.3515625	6.47217159192897e-05	\\
-2234.375	6.10460079383722e-05	\\
-2233.3984375	6.33643951667767e-05	\\
-2232.421875	6.19608125993476e-05	\\
-2231.4453125	6.12967168001668e-05	\\
-2230.46875	5.76829575840439e-05	\\
-2229.4921875	5.82215312237407e-05	\\
-2228.515625	5.76854458127513e-05	\\
-2227.5390625	5.57033160690357e-05	\\
-2226.5625	5.68918930848756e-05	\\
-2225.5859375	5.54934449626792e-05	\\
-2224.609375	5.26818844231877e-05	\\
-2223.6328125	5.20266489248034e-05	\\
-2222.65625	5.17448199018498e-05	\\
-2221.6796875	5.43966880702032e-05	\\
-2220.703125	5.14012430530683e-05	\\
-2219.7265625	4.98908362663553e-05	\\
-2218.75	5.26670315657852e-05	\\
-2217.7734375	5.01639237069749e-05	\\
-2216.796875	4.86958666023807e-05	\\
-2215.8203125	4.83421272022935e-05	\\
-2214.84375	4.75840447493512e-05	\\
-2213.8671875	4.67767044011369e-05	\\
-2212.890625	4.48931851959409e-05	\\
-2211.9140625	4.46795990963797e-05	\\
-2210.9375	4.28727056730064e-05	\\
-2209.9609375	4.28456777679685e-05	\\
-2208.984375	4.46625075450272e-05	\\
-2208.0078125	4.01359567097531e-05	\\
-2207.03125	4.14591886363265e-05	\\
-2206.0546875	4.07793930199317e-05	\\
-2205.078125	4.20968204853393e-05	\\
-2204.1015625	3.83497538983684e-05	\\
-2203.125	3.76608000566042e-05	\\
-2202.1484375	3.7010155398845e-05	\\
-2201.171875	3.71614534249172e-05	\\
-2200.1953125	3.4660255226251e-05	\\
-2199.21875	3.35102266578728e-05	\\
-2198.2421875	3.70229928142263e-05	\\
-2197.265625	3.48027833233695e-05	\\
-2196.2890625	3.43937919453396e-05	\\
-2195.3125	3.62691824225029e-05	\\
-2194.3359375	3.3223294422691e-05	\\
-2193.359375	3.57977046007997e-05	\\
-2192.3828125	3.23507053868935e-05	\\
-2191.40625	3.51745752549879e-05	\\
-2190.4296875	3.26119575127941e-05	\\
-2189.453125	3.77602770622084e-05	\\
-2188.4765625	3.46726456022483e-05	\\
-2187.5	3.21385327700624e-05	\\
-2186.5234375	3.67519683268806e-05	\\
-2185.546875	3.03164802342826e-05	\\
-2184.5703125	3.3947114492317e-05	\\
-2183.59375	3.38980078519184e-05	\\
-2182.6171875	3.34371752216082e-05	\\
-2181.640625	3.36904116896546e-05	\\
-2180.6640625	3.62940280873082e-05	\\
-2179.6875	3.64558122454385e-05	\\
-2178.7109375	3.37754951220909e-05	\\
-2177.734375	3.74438587468373e-05	\\
-2176.7578125	3.47397381308913e-05	\\
-2175.78125	3.59559120231284e-05	\\
-2174.8046875	3.54547798347632e-05	\\
-2173.828125	3.85959244418813e-05	\\
-2172.8515625	3.8336582006231e-05	\\
-2171.875	3.8105314852252e-05	\\
-2170.8984375	4.03429788107336e-05	\\
-2169.921875	3.93923314303165e-05	\\
-2168.9453125	3.80829102126855e-05	\\
-2167.96875	4.1730919915096e-05	\\
-2166.9921875	3.98656470337524e-05	\\
-2166.015625	4.15056506103026e-05	\\
-2165.0390625	4.00590479049212e-05	\\
-2164.0625	4.13164498527604e-05	\\
-2163.0859375	4.01368473334327e-05	\\
-2162.109375	4.2663242669412e-05	\\
-2161.1328125	4.53729919869218e-05	\\
-2160.15625	4.52038294257535e-05	\\
-2159.1796875	4.34580584089893e-05	\\
-2158.203125	4.46783456619518e-05	\\
-2157.2265625	4.14480659510549e-05	\\
-2156.25	4.51621780669581e-05	\\
-2155.2734375	4.14341940096069e-05	\\
-2154.296875	4.33312030222277e-05	\\
-2153.3203125	4.23913499628309e-05	\\
-2152.34375	4.36777259152111e-05	\\
-2151.3671875	4.61964024373542e-05	\\
-2150.390625	4.57052115222215e-05	\\
-2149.4140625	4.48044095535832e-05	\\
-2148.4375	4.59327631052793e-05	\\
-2147.4609375	4.48266323821542e-05	\\
-2146.484375	4.87977600742606e-05	\\
-2145.5078125	4.71274722909951e-05	\\
-2144.53125	4.67789565688556e-05	\\
-2143.5546875	4.80878740466477e-05	\\
-2142.578125	4.34197922510234e-05	\\
-2141.6015625	4.98850162778815e-05	\\
-2140.625	4.9203823965806e-05	\\
-2139.6484375	4.7050361639397e-05	\\
-2138.671875	4.93968852913047e-05	\\
-2137.6953125	5.11882026176662e-05	\\
-2136.71875	5.12698504806669e-05	\\
-2135.7421875	4.92806395767379e-05	\\
-2134.765625	4.69835870095165e-05	\\
-2133.7890625	5.01573459564202e-05	\\
-2132.8125	5.19575495865586e-05	\\
-2131.8359375	4.96810191256607e-05	\\
-2130.859375	5.13743895983584e-05	\\
-2129.8828125	4.85680176209039e-05	\\
-2128.90625	4.57402791052152e-05	\\
-2127.9296875	4.83918169458965e-05	\\
-2126.953125	5.1381511804256e-05	\\
-2125.9765625	5.04997405340511e-05	\\
-2125	4.84167660645333e-05	\\
-2124.0234375	5.08477861411733e-05	\\
-2123.046875	5.0006476769679e-05	\\
-2122.0703125	4.73736319437822e-05	\\
-2121.09375	4.68927623978026e-05	\\
-2120.1171875	5.10459997506985e-05	\\
-2119.140625	4.6464669044419e-05	\\
-2118.1640625	4.75028234433485e-05	\\
-2117.1875	4.92200907700446e-05	\\
-2116.2109375	4.98423865697535e-05	\\
-2115.234375	4.53817611895142e-05	\\
-2114.2578125	4.63712832382412e-05	\\
-2113.28125	4.33200104633018e-05	\\
-2112.3046875	4.84302591660997e-05	\\
-2111.328125	4.50549356189519e-05	\\
-2110.3515625	4.5928981400269e-05	\\
-2109.375	4.27689908830463e-05	\\
-2108.3984375	4.38382506759848e-05	\\
-2107.421875	4.05080810401405e-05	\\
-2106.4453125	4.27957429868868e-05	\\
-2105.46875	4.31343693408449e-05	\\
-2104.4921875	4.16343578826944e-05	\\
-2103.515625	4.08151459218673e-05	\\
-2102.5390625	3.60410815115795e-05	\\
-2101.5625	3.69543826095997e-05	\\
-2100.5859375	4.33321814262601e-05	\\
-2099.609375	3.8832074103747e-05	\\
-2098.6328125	3.95037863904351e-05	\\
-2097.65625	3.46992260561987e-05	\\
-2096.6796875	3.65974687220967e-05	\\
-2095.703125	3.41353233010847e-05	\\
-2094.7265625	3.64722322778896e-05	\\
-2093.75	3.34823664464201e-05	\\
-2092.7734375	3.47307623174561e-05	\\
-2091.796875	3.33449232463319e-05	\\
-2090.8203125	3.13557350110472e-05	\\
-2089.84375	3.57929281691663e-05	\\
-2088.8671875	3.26622284382464e-05	\\
-2087.890625	2.90486194132808e-05	\\
-2086.9140625	3.25358246806361e-05	\\
-2085.9375	3.26339954470388e-05	\\
-2084.9609375	3.17425062795112e-05	\\
-2083.984375	3.41662759040213e-05	\\
-2083.0078125	2.96126284013353e-05	\\
-2082.03125	3.20229887763462e-05	\\
-2081.0546875	3.20177959090973e-05	\\
-2080.078125	3.28100352740946e-05	\\
-2079.1015625	3.29155522056265e-05	\\
-2078.125	3.54461534911628e-05	\\
-2077.1484375	3.30576867467643e-05	\\
-2076.171875	3.42940263740739e-05	\\
-2075.1953125	3.47709605138249e-05	\\
-2074.21875	3.46933236000195e-05	\\
-2073.2421875	3.52316805588199e-05	\\
-2072.265625	3.53363878521999e-05	\\
-2071.2890625	3.76252987801829e-05	\\
-2070.3125	3.65179230233273e-05	\\
-2069.3359375	3.87105409077275e-05	\\
-2068.359375	3.62817334321908e-05	\\
-2067.3828125	3.82450179060434e-05	\\
-2066.40625	3.92156784904568e-05	\\
-2065.4296875	3.46728472204869e-05	\\
-2064.453125	3.82312216832789e-05	\\
-2063.4765625	3.80075814595717e-05	\\
-2062.5	4.05255618411346e-05	\\
-2061.5234375	3.8773002039269e-05	\\
-2060.546875	4.26716487036751e-05	\\
-2059.5703125	4.2996541760037e-05	\\
-2058.59375	4.19004534199197e-05	\\
-2057.6171875	4.45958264607971e-05	\\
-2056.640625	4.69152759673768e-05	\\
-2055.6640625	4.65059248087282e-05	\\
-2054.6875	5.13325287172084e-05	\\
-2053.7109375	5.25624003080617e-05	\\
-2052.734375	4.77291007930654e-05	\\
-2051.7578125	4.97498846527547e-05	\\
-2050.78125	5.32868636801919e-05	\\
-2049.8046875	5.20310316069988e-05	\\
-2048.828125	5.35762489010926e-05	\\
-2047.8515625	5.44336611530301e-05	\\
-2046.875	5.3278379389555e-05	\\
-2045.8984375	5.73239896246496e-05	\\
-2044.921875	6.02837423037885e-05	\\
-2043.9453125	5.72244682842368e-05	\\
-2042.96875	5.93114600022863e-05	\\
-2041.9921875	5.75581738414443e-05	\\
-2041.015625	6.42898820469134e-05	\\
-2040.0390625	6.12442141978293e-05	\\
-2039.0625	6.63216072033126e-05	\\
-2038.0859375	6.39750068278896e-05	\\
-2037.109375	6.60222468691509e-05	\\
-2036.1328125	7.08435678303005e-05	\\
-2035.15625	7.2137408224821e-05	\\
-2034.1796875	6.8270001278672e-05	\\
-2033.203125	7.20843349815028e-05	\\
-2032.2265625	7.27325323733426e-05	\\
-2031.25	7.43759141390474e-05	\\
-2030.2734375	7.64057451387006e-05	\\
-2029.296875	7.69444784952836e-05	\\
-2028.3203125	7.83903234701613e-05	\\
-2027.34375	7.71978408189353e-05	\\
-2026.3671875	7.91845218285216e-05	\\
-2025.390625	7.95357864437591e-05	\\
-2024.4140625	8.17996848361462e-05	\\
-2023.4375	8.21820444777774e-05	\\
-2022.4609375	8.67173397710433e-05	\\
-2021.484375	8.41148840984366e-05	\\
-2020.5078125	8.02713461577442e-05	\\
-2019.53125	8.53113159727498e-05	\\
-2018.5546875	8.48027400106208e-05	\\
-2017.578125	8.59126998761636e-05	\\
-2016.6015625	8.80070305821302e-05	\\
-2015.625	8.75080560145517e-05	\\
-2014.6484375	8.82899641058983e-05	\\
-2013.671875	9.20157820649349e-05	\\
-2012.6953125	8.96215057521745e-05	\\
-2011.71875	8.84462893016537e-05	\\
-2010.7421875	8.79817476704986e-05	\\
-2009.765625	9.12978174312735e-05	\\
-2008.7890625	9.12902426330043e-05	\\
-2007.8125	9.05283057505229e-05	\\
-2006.8359375	9.04089835117761e-05	\\
-2005.859375	9.15757321841421e-05	\\
-2004.8828125	9.02481915901383e-05	\\
-2003.90625	9.13169824145695e-05	\\
-2002.9296875	9.04832349455203e-05	\\
-2001.953125	9.0879339802752e-05	\\
-2000.9765625	8.89416950720669e-05	\\
-2000	9.06801793911246e-05	\\
-1999.0234375	9.04663192532254e-05	\\
-1998.046875	9.17803382150896e-05	\\
-1997.0703125	8.97600861480342e-05	\\
-1996.09375	9.46557138726662e-05	\\
-1995.1171875	9.01236014374602e-05	\\
-1994.140625	9.29897356656637e-05	\\
-1993.1640625	9.21458262141116e-05	\\
-1992.1875	9.18639753368399e-05	\\
-1991.2109375	9.22602449970016e-05	\\
-1990.234375	9.16065985765556e-05	\\
-1989.2578125	9.37819567387981e-05	\\
-1988.28125	9.35170832927778e-05	\\
-1987.3046875	9.18761720702572e-05	\\
-1986.328125	9.26648373642803e-05	\\
-1985.3515625	9.45433634571937e-05	\\
-1984.375	9.28555986655364e-05	\\
-1983.3984375	9.29012129435057e-05	\\
-1982.421875	9.17303051293725e-05	\\
-1981.4453125	9.26847307653979e-05	\\
-1980.46875	9.15373156239458e-05	\\
-1979.4921875	9.48017493129546e-05	\\
-1978.515625	9.07225621877017e-05	\\
-1977.5390625	9.08675846860354e-05	\\
-1976.5625	9.19165313376669e-05	\\
-1975.5859375	9.29480537599809e-05	\\
-1974.609375	8.79599867683246e-05	\\
-1973.6328125	9.00641068081225e-05	\\
-1972.65625	9.04545878503873e-05	\\
-1971.6796875	8.83667333336917e-05	\\
-1970.703125	9.29102278521813e-05	\\
-1969.7265625	9.2351600192384e-05	\\
-1968.75	9.04860623998688e-05	\\
-1967.7734375	9.20160268289978e-05	\\
-1966.796875	9.37772746918745e-05	\\
-1965.8203125	9.1720058695774e-05	\\
-1964.84375	9.1066691226639e-05	\\
-1963.8671875	9.05318177279546e-05	\\
-1962.890625	9.13590894777137e-05	\\
-1961.9140625	9.07748111522205e-05	\\
-1960.9375	8.91039657256826e-05	\\
-1959.9609375	8.66058201174048e-05	\\
-1958.984375	8.954828138643e-05	\\
-1958.0078125	9.12420778864304e-05	\\
-1957.03125	8.93768382551817e-05	\\
-1956.0546875	8.79523055814618e-05	\\
-1955.078125	9.13096058301114e-05	\\
-1954.1015625	9.16703069373332e-05	\\
-1953.125	9.12909200485107e-05	\\
-1952.1484375	9.06633970337155e-05	\\
-1951.171875	8.89830505920273e-05	\\
-1950.1953125	9.09914516501304e-05	\\
-1949.21875	9.02514372054311e-05	\\
-1948.2421875	8.72417881599986e-05	\\
-1947.265625	8.97401275470904e-05	\\
-1946.2890625	8.87005546469158e-05	\\
-1945.3125	8.60942761778237e-05	\\
-1944.3359375	8.80019796851244e-05	\\
-1943.359375	8.88165333851624e-05	\\
-1942.3828125	8.97767768811711e-05	\\
-1941.40625	8.98320966710122e-05	\\
-1940.4296875	8.9340869177868e-05	\\
-1939.453125	8.97040836211091e-05	\\
-1938.4765625	8.93310290348181e-05	\\
-1937.5	9.10369567411658e-05	\\
-1936.5234375	9.04043226166378e-05	\\
-1935.546875	9.3174507436734e-05	\\
-1934.5703125	9.15967142997813e-05	\\
-1933.59375	9.35109822299892e-05	\\
-1932.6171875	9.27825846906686e-05	\\
-1931.640625	9.39806050674572e-05	\\
-1930.6640625	8.94072000053504e-05	\\
-1929.6875	9.01231061210788e-05	\\
-1928.7109375	8.74585724167036e-05	\\
-1927.734375	9.20693709768403e-05	\\
-1926.7578125	8.88781011553631e-05	\\
-1925.78125	8.83289863244321e-05	\\
-1924.8046875	9.34947729866328e-05	\\
-1923.828125	8.56669147610826e-05	\\
-1922.8515625	8.61233219989611e-05	\\
-1921.875	8.91831818474238e-05	\\
-1920.8984375	9.02824346401885e-05	\\
-1919.921875	8.55655138810445e-05	\\
-1918.9453125	8.62227821652724e-05	\\
-1917.96875	8.77786376039023e-05	\\
-1916.9921875	9.08358814590421e-05	\\
-1916.015625	9.0666256696698e-05	\\
-1915.0390625	9.06719947873007e-05	\\
-1914.0625	9.12786677834146e-05	\\
-1913.0859375	9.09099421020954e-05	\\
-1912.109375	8.61320154499593e-05	\\
-1911.1328125	8.51909422553583e-05	\\
-1910.15625	8.69177918240257e-05	\\
-1909.1796875	8.73457339197992e-05	\\
-1908.203125	8.37112288596342e-05	\\
-1907.2265625	8.88113041633969e-05	\\
-1906.25	9.00639906841513e-05	\\
-1905.2734375	8.7964240255214e-05	\\
-1904.296875	8.68853396744819e-05	\\
-1903.3203125	8.59054842968387e-05	\\
-1902.34375	8.5067281289055e-05	\\
-1901.3671875	8.77014889720255e-05	\\
-1900.390625	8.37776646445666e-05	\\
-1899.4140625	8.41193918937399e-05	\\
-1898.4375	8.42703566141137e-05	\\
-1897.4609375	8.51705688885096e-05	\\
-1896.484375	8.82833448111666e-05	\\
-1895.5078125	8.95537013447696e-05	\\
-1894.53125	8.72027295354851e-05	\\
-1893.5546875	8.87833337678791e-05	\\
-1892.578125	9.01922456784955e-05	\\
-1891.6015625	8.86439236725281e-05	\\
-1890.625	8.83274425441474e-05	\\
-1889.6484375	8.80892929787839e-05	\\
-1888.671875	8.94126147634509e-05	\\
-1887.6953125	9.00133574943236e-05	\\
-1886.71875	8.78189071642815e-05	\\
-1885.7421875	8.77713264048513e-05	\\
-1884.765625	8.79692393838881e-05	\\
-1883.7890625	8.75611117804718e-05	\\
-1882.8125	8.46590377446625e-05	\\
-1881.8359375	8.35722991552598e-05	\\
-1880.859375	8.81466281909955e-05	\\
-1879.8828125	8.520827540838e-05	\\
-1878.90625	8.66859438846377e-05	\\
-1877.9296875	8.67705916019656e-05	\\
-1876.953125	8.67682565666945e-05	\\
-1875.9765625	8.71334519256327e-05	\\
-1875	9.00405806732701e-05	\\
-1874.0234375	8.49474046258507e-05	\\
-1873.046875	9.03052090341949e-05	\\
-1872.0703125	9.03774868262291e-05	\\
-1871.09375	8.87639453592289e-05	\\
-1870.1171875	8.74768523724309e-05	\\
-1869.140625	8.68651074800648e-05	\\
-1868.1640625	8.74457159981748e-05	\\
-1867.1875	8.65527729824974e-05	\\
-1866.2109375	8.81761403464153e-05	\\
-1865.234375	8.81113919389346e-05	\\
-1864.2578125	8.85292537331998e-05	\\
-1863.28125	8.7872450265669e-05	\\
-1862.3046875	8.68476301438373e-05	\\
-1861.328125	8.59350021901228e-05	\\
-1860.3515625	8.5928578502877e-05	\\
-1859.375	8.87069098633398e-05	\\
-1858.3984375	8.57794497327504e-05	\\
-1857.421875	8.77682624413867e-05	\\
-1856.4453125	8.6130446124041e-05	\\
-1855.46875	9.09191809979496e-05	\\
-1854.4921875	8.68677945192038e-05	\\
-1853.515625	9.01334670679761e-05	\\
-1852.5390625	8.73084632311838e-05	\\
-1851.5625	8.87203314720183e-05	\\
-1850.5859375	9.24145543230773e-05	\\
-1849.609375	8.91080721450163e-05	\\
-1848.6328125	9.19398840076446e-05	\\
-1847.65625	9.07298790181859e-05	\\
-1846.6796875	8.93727387665563e-05	\\
-1845.703125	8.64119052196069e-05	\\
-1844.7265625	9.15386189041087e-05	\\
-1843.75	8.63729523692262e-05	\\
-1842.7734375	9.07284860448467e-05	\\
-1841.796875	8.96710232117539e-05	\\
-1840.8203125	8.91572297710133e-05	\\
-1839.84375	8.90066451285003e-05	\\
-1838.8671875	9.50735287484824e-05	\\
-1837.890625	9.005899812363e-05	\\
-1836.9140625	9.43658257765968e-05	\\
-1835.9375	9.46112015872276e-05	\\
-1834.9609375	8.97209091640168e-05	\\
-1833.984375	9.4125419180389e-05	\\
-1833.0078125	9.32266211831091e-05	\\
-1832.03125	9.53252013316973e-05	\\
-1831.0546875	9.45958493224165e-05	\\
-1830.078125	9.60512019146778e-05	\\
-1829.1015625	9.63076480661012e-05	\\
-1828.125	9.41093022325463e-05	\\
-1827.1484375	9.42721812918304e-05	\\
-1826.171875	9.88203535168958e-05	\\
-1825.1953125	9.5655716564934e-05	\\
-1824.21875	9.65737819736162e-05	\\
-1823.2421875	9.91953620670616e-05	\\
-1822.265625	9.75780172389258e-05	\\
-1821.2890625	9.88981601303546e-05	\\
-1820.3125	9.67795295306084e-05	\\
-1819.3359375	0.000102372906185061	\\
-1818.359375	0.000102328871155867	\\
-1817.3828125	9.58377462561653e-05	\\
-1816.40625	0.000102071297655778	\\
-1815.4296875	0.000103945901132428	\\
-1814.453125	9.89363003685222e-05	\\
-1813.4765625	0.000103016328563547	\\
-1812.5	0.000103327953585357	\\
-1811.5234375	0.000103777334304082	\\
-1810.546875	0.000104343803397914	\\
-1809.5703125	0.000101785706133562	\\
-1808.59375	0.00010272699058423	\\
-1807.6171875	0.000100085953662788	\\
-1806.640625	0.000100204069489608	\\
-1805.6640625	0.000106998331772022	\\
-1804.6875	0.000104919175841358	\\
-1803.7109375	0.000103174868339667	\\
-1802.734375	0.000105966780834877	\\
-1801.7578125	0.000103009936473959	\\
-1800.78125	0.000108089337793804	\\
-1799.8046875	0.000108866277021583	\\
-1798.828125	0.00011119600812544	\\
-1797.8515625	0.000103089387524724	\\
-1796.875	0.000110678843846267	\\
-1795.8984375	0.000106165880069089	\\
-1794.921875	0.000105962157621524	\\
-1793.9453125	0.000108011079150562	\\
-1792.96875	0.000105895491176919	\\
-1791.9921875	0.000103778812530719	\\
-1791.015625	0.00010926712045325	\\
-1790.0390625	0.000106994830479337	\\
-1789.0625	0.00010940973488976	\\
-1788.0859375	0.000106726179901106	\\
-1787.109375	0.000106627457251632	\\
-1786.1328125	0.000110015665716735	\\
-1785.15625	0.000108371491207021	\\
-1784.1796875	0.000107319767140741	\\
-1783.203125	0.000109238573319262	\\
-1782.2265625	0.000108412262852302	\\
-1781.25	0.000105500039193221	\\
-1780.2734375	0.000110221736688422	\\
-1779.296875	0.000107334551837195	\\
-1778.3203125	0.000109368579914517	\\
-1777.34375	0.000105432463573963	\\
-1776.3671875	0.000107580749031816	\\
-1775.390625	0.00010646002783434	\\
-1774.4140625	0.000107187859018979	\\
-1773.4375	0.000107808478520791	\\
-1772.4609375	0.000107169974488407	\\
-1771.484375	0.000106563229270401	\\
-1770.5078125	0.000109312940497968	\\
-1769.53125	0.000105393190327654	\\
-1768.5546875	0.000107055392562425	\\
-1767.578125	0.000104324169374909	\\
-1766.6015625	0.000107166548562067	\\
-1765.625	0.000108213676091706	\\
-1764.6484375	0.000107271058429676	\\
-1763.671875	0.000108958615528744	\\
-1762.6953125	0.000108166821097622	\\
-1761.71875	0.000106882714074656	\\
-1760.7421875	0.000108684222645485	\\
-1759.765625	0.00011116430929049	\\
-1758.7890625	0.00010602908589005	\\
-1757.8125	0.000105366784106859	\\
-1756.8359375	0.000108624095756423	\\
-1755.859375	0.000108377115555173	\\
-1754.8828125	0.000109805492039095	\\
-1753.90625	0.000107430243623172	\\
-1752.9296875	0.000104955886113346	\\
-1751.953125	0.000106761617443954	\\
-1750.9765625	0.000107639694554568	\\
-1750	0.00010998149903398	\\
-1749.0234375	0.000108041335626125	\\
-1748.046875	0.000112188764936567	\\
-1747.0703125	0.000108524991873341	\\
-1746.09375	0.00010905474774075	\\
-1745.1171875	0.000111268791854708	\\
-1744.140625	0.000111341780813466	\\
-1743.1640625	0.000113734463249664	\\
-1742.1875	0.000108725072558081	\\
-1741.2109375	0.000108309324828898	\\
-1740.234375	0.000109006214018378	\\
-1739.2578125	0.000112215195950698	\\
-1738.28125	0.00011224304369932	\\
-1737.3046875	0.00011012197833871	\\
-1736.328125	0.000110254934488439	\\
-1735.3515625	0.000109182596041917	\\
-1734.375	0.000112761353293443	\\
-1733.3984375	0.000112016063574971	\\
-1732.421875	0.00011150502835293	\\
-1731.4453125	0.000110727055243006	\\
-1730.46875	0.000111602923080901	\\
-1729.4921875	0.000112444133878182	\\
-1728.515625	0.00011100476061813	\\
-1727.5390625	0.000112744893389965	\\
-1726.5625	0.00011380122976047	\\
-1725.5859375	0.000115830628173954	\\
-1724.609375	0.000115297832168182	\\
-1723.6328125	0.000116521189642771	\\
-1722.65625	0.000115348296092661	\\
-1721.6796875	0.000111500825329051	\\
-1720.703125	0.000115240059600894	\\
-1719.7265625	0.000113206340625051	\\
-1718.75	0.000116605250182503	\\
-1717.7734375	0.000119419950570677	\\
-1716.796875	0.000116724442315164	\\
-1715.8203125	0.000116770880490609	\\
-1714.84375	0.000118736963207191	\\
-1713.8671875	0.000119806913465584	\\
-1712.890625	0.000117718705852211	\\
-1711.9140625	0.00012245817037508	\\
-1710.9375	0.000121826918488289	\\
-1709.9609375	0.000120714527941923	\\
-1708.984375	0.000122485622077199	\\
-1708.0078125	0.000120830323821949	\\
-1707.03125	0.00012150000194669	\\
-1706.0546875	0.000124168044057583	\\
-1705.078125	0.000121895395459561	\\
-1704.1015625	0.000123868896374282	\\
-1703.125	0.000120645682702643	\\
-1702.1484375	0.000117034532352557	\\
-1701.171875	0.000118933409544003	\\
-1700.1953125	0.000121738849159626	\\
-1699.21875	0.000123832301307535	\\
-1698.2421875	0.00012272570570069	\\
-1697.265625	0.000124081144636815	\\
-1696.2890625	0.000123488799876012	\\
-1695.3125	0.000124364473970674	\\
-1694.3359375	0.000124308401209876	\\
-1693.359375	0.000127205094852116	\\
-1692.3828125	0.000127838446494535	\\
-1691.40625	0.000125359140244553	\\
-1690.4296875	0.00012622039808075	\\
-1689.453125	0.000129646223990115	\\
-1688.4765625	0.000127781534175412	\\
-1687.5	0.000129948347816542	\\
-1686.5234375	0.000130168701509493	\\
-1685.546875	0.00012929176245871	\\
-1684.5703125	0.000124907730615026	\\
-1683.59375	0.00012820087420842	\\
-1682.6171875	0.00012351467405748	\\
-1681.640625	0.000123848646809733	\\
-1680.6640625	0.000126109254654207	\\
-1679.6875	0.000125418544396074	\\
-1678.7109375	0.000131284418554807	\\
-1677.734375	0.000126614713104364	\\
-1676.7578125	0.000125743691509091	\\
-1675.78125	0.000127068807996972	\\
-1674.8046875	0.000129876219530083	\\
-1673.828125	0.000128237299103246	\\
-1672.8515625	0.00012685585398323	\\
-1671.875	0.000129332476435152	\\
-1670.8984375	0.000130328683041731	\\
-1669.921875	0.000124607931149136	\\
-1668.9453125	0.000126297450806879	\\
-1667.96875	0.000127948328452652	\\
-1666.9921875	0.000125105257430486	\\
-1666.015625	0.000126110154457296	\\
-1665.0390625	0.000124238067913462	\\
-1664.0625	0.00012288877503671	\\
-1663.0859375	0.000124234177848114	\\
-1662.109375	0.000124874125493467	\\
-1661.1328125	0.000123989690280405	\\
-1660.15625	0.00012119498363224	\\
-1659.1796875	0.000119524838560062	\\
-1658.203125	0.000123546154342277	\\
-1657.2265625	0.000121631099580923	\\
-1656.25	0.000124685389640273	\\
-1655.2734375	0.000121913197612773	\\
-1654.296875	0.000120598045577559	\\
-1653.3203125	0.000123791643752132	\\
-1652.34375	0.000120638305203735	\\
-1651.3671875	0.000117759144387339	\\
-1650.390625	0.000118854697741539	\\
-1649.4140625	0.000117335000817818	\\
-1648.4375	0.000117626659259281	\\
-1647.4609375	0.000117648034413845	\\
-1646.484375	0.000121564414338826	\\
-1645.5078125	0.000116531978461627	\\
-1644.53125	0.000114484187449808	\\
-1643.5546875	0.000112036182734507	\\
-1642.578125	0.000112164580715716	\\
-1641.6015625	0.000111395633761448	\\
-1640.625	0.000111790814825516	\\
-1639.6484375	0.000112004469332388	\\
-1638.671875	0.000111304224558038	\\
-1637.6953125	0.000111391247727031	\\
-1636.71875	0.000112999522479522	\\
-1635.7421875	0.000113130838941097	\\
-1634.765625	0.000112518644619469	\\
-1633.7890625	0.000110121796530808	\\
-1632.8125	0.000111821537472097	\\
-1631.8359375	0.0001097753182079	\\
-1630.859375	0.000109296067000989	\\
-1629.8828125	0.00010989233026856	\\
-1628.90625	0.000110510815554026	\\
-1627.9296875	0.000110731842699803	\\
-1626.953125	0.000108314639032473	\\
-1625.9765625	0.000106213532697408	\\
-1625	0.000109335325822241	\\
-1624.0234375	0.000108241611620272	\\
-1623.046875	0.000111352768785447	\\
-1622.0703125	0.000107930428745043	\\
-1621.09375	0.000105644681099727	\\
-1620.1171875	0.000107358926177327	\\
-1619.140625	0.000105136612207957	\\
-1618.1640625	0.000104900877136514	\\
-1617.1875	0.000107399907397796	\\
-1616.2109375	0.000106334384697957	\\
-1615.234375	0.000109500130740872	\\
-1614.2578125	0.000113977692275706	\\
-1613.28125	0.000110399176695926	\\
-1612.3046875	0.000109154536562048	\\
-1611.328125	0.000111518727645362	\\
-1610.3515625	0.000110428208969475	\\
-1609.375	0.000113737867217649	\\
-1608.3984375	0.000112478025785116	\\
-1607.421875	0.000106677236164781	\\
-1606.4453125	0.000113898248160207	\\
-1605.46875	0.000115850249567002	\\
-1604.4921875	0.000116276226261333	\\
-1603.515625	0.000114486277993485	\\
-1602.5390625	0.000116013052435048	\\
-1601.5625	0.000123274495127916	\\
-1600.5859375	0.000108807015159548	\\
-1599.609375	0.000116553529739617	\\
-1598.6328125	0.000118995013303126	\\
-1597.65625	0.000121424847348135	\\
-1596.6796875	0.000119637628543725	\\
-1595.703125	0.000120701914858438	\\
-1594.7265625	0.000122466203961396	\\
-1593.75	0.000122095089971202	\\
-1592.7734375	0.000121989007485419	\\
-1591.796875	0.000123416432301839	\\
-1590.8203125	0.000126343569368222	\\
-1589.84375	0.000122401139183964	\\
-1588.8671875	0.000122843413775096	\\
-1587.890625	0.000124425878680691	\\
-1586.9140625	0.000126437127135374	\\
-1585.9375	0.000123858586758542	\\
-1584.9609375	0.000128762357398776	\\
-1583.984375	0.000125636680499571	\\
-1583.0078125	0.000127991690425677	\\
-1582.03125	0.000125735431197653	\\
-1581.0546875	0.000128039099718054	\\
-1580.078125	0.000128572582353414	\\
-1579.1015625	0.00012957937776652	\\
-1578.125	0.000129272500537412	\\
-1577.1484375	0.000130830115676593	\\
-1576.171875	0.000129244267027533	\\
-1575.1953125	0.000127661506767172	\\
-1574.21875	0.0001328848014919	\\
-1573.2421875	0.000127307170255837	\\
-1572.265625	0.000131985136757263	\\
-1571.2890625	0.000129401611426034	\\
-1570.3125	0.000128707258806734	\\
-1569.3359375	0.000127871726452614	\\
-1568.359375	0.000123737352894491	\\
-1567.3828125	0.00012957755068051	\\
-1566.40625	0.000127402137453717	\\
-1565.4296875	0.000126296574589254	\\
-1564.453125	0.000127520877412948	\\
-1563.4765625	0.000126302053638221	\\
-1562.5	0.000120894203182126	\\
-1561.5234375	0.000122370443277614	\\
-1560.546875	0.000126380115815357	\\
-1559.5703125	0.000125972651078962	\\
-1558.59375	0.000120808695600407	\\
-1557.6171875	0.000121129228991483	\\
-1556.640625	0.000124015173229955	\\
-1555.6640625	0.000122564601416069	\\
-1554.6875	0.000124274700824892	\\
-1553.7109375	0.000120230055738904	\\
-1552.734375	0.000119652419980962	\\
-1551.7578125	0.000122403225353873	\\
-1550.78125	0.000120399787637843	\\
-1549.8046875	0.000122318087971169	\\
-1548.828125	0.000118186371037257	\\
-1547.8515625	0.00012135661683216	\\
-1546.875	0.000119186747345433	\\
-1545.8984375	0.000119932977102278	\\
-1544.921875	0.000118095110644696	\\
-1543.9453125	0.000121668586104863	\\
-1542.96875	0.000119383629430643	\\
-1541.9921875	0.000119529769883742	\\
-1541.015625	0.000121413872559384	\\
-1540.0390625	0.000118799066247295	\\
-1539.0625	0.000121985836264144	\\
-1538.0859375	0.000121036957833414	\\
-1537.109375	0.000120958324350456	\\
-1536.1328125	0.00012067231132516	\\
-1535.15625	0.000118787490887982	\\
-1534.1796875	0.000120554858700569	\\
-1533.203125	0.000123013716579948	\\
-1532.2265625	0.0001241731996661	\\
-1531.25	0.000122922319014893	\\
-1530.2734375	0.000121616559168395	\\
-1529.296875	0.000123014443750624	\\
-1528.3203125	0.000120095607206276	\\
-1527.34375	0.000126179055164711	\\
-1526.3671875	0.00012475273581387	\\
-1525.390625	0.000125147511790467	\\
-1524.4140625	0.000123462863738448	\\
-1523.4375	0.00012366131079251	\\
-1522.4609375	0.000122001731221195	\\
-1521.484375	0.000121749474213729	\\
-1520.5078125	0.000118256540064825	\\
-1519.53125	0.000117306406101938	\\
-1518.5546875	0.00012091829387455	\\
-1517.578125	0.000118295893984695	\\
-1516.6015625	0.000116338417729314	\\
-1515.625	0.000119261464076529	\\
-1514.6484375	0.000116875754143847	\\
-1513.671875	0.000115816095948994	\\
-1512.6953125	0.000113427610346648	\\
-1511.71875	0.000115588337784902	\\
-1510.7421875	0.000112865390055503	\\
-1509.765625	0.000110733519748504	\\
-1508.7890625	0.000112471381649754	\\
-1507.8125	0.000109555711913548	\\
-1506.8359375	0.000107560655433684	\\
-1505.859375	0.000106759308005145	\\
-1504.8828125	0.000107231440783852	\\
-1503.90625	0.000107343683791008	\\
-1502.9296875	0.000111946211549208	\\
-1501.953125	0.000111034613601343	\\
-1500.9765625	0.000108337131700521	\\
-1500	0.000116559051498441	\\
-1499.0234375	0.000106195506485573	\\
-1498.046875	0.000102085872813701	\\
-1497.0703125	0.000106976472875356	\\
-1496.09375	0.000101475762655641	\\
-1495.1171875	9.99357234033906e-05	\\
-1494.140625	0.000102265389254422	\\
-1493.1640625	9.99152833233974e-05	\\
-1492.1875	9.80250915764393e-05	\\
-1491.2109375	0.00010394380615476	\\
-1490.234375	9.71239773555574e-05	\\
-1489.2578125	0.000101142840460086	\\
-1488.28125	9.58002678754308e-05	\\
-1487.3046875	9.79938804524116e-05	\\
-1486.328125	9.48244533104882e-05	\\
-1485.3515625	9.71003912789966e-05	\\
-1484.375	9.46831152700439e-05	\\
-1483.3984375	9.73001485166558e-05	\\
-1482.421875	9.78379167922272e-05	\\
-1481.4453125	9.27151310194985e-05	\\
-1480.46875	9.55912550090849e-05	\\
-1479.4921875	9.27439271883458e-05	\\
-1478.515625	9.45221295972288e-05	\\
-1477.5390625	9.00627895163684e-05	\\
-1476.5625	9.05335368009879e-05	\\
-1475.5859375	9.17343973631241e-05	\\
-1474.609375	9.09881983366604e-05	\\
-1473.6328125	9.2161113449453e-05	\\
-1472.65625	8.93770585448036e-05	\\
-1471.6796875	8.65962355815777e-05	\\
-1470.703125	8.85354606418346e-05	\\
-1469.7265625	8.65374625712493e-05	\\
-1468.75	8.71072706243921e-05	\\
-1467.7734375	8.64028066362553e-05	\\
-1466.796875	8.16542070892097e-05	\\
-1465.8203125	8.43586461320699e-05	\\
-1464.84375	8.21370898500511e-05	\\
-1463.8671875	8.21977530522015e-05	\\
-1462.890625	8.33365044977349e-05	\\
-1461.9140625	8.02604913007952e-05	\\
-1460.9375	8.2655835433553e-05	\\
-1459.9609375	7.90921511709386e-05	\\
-1458.984375	7.74209870460302e-05	\\
-1458.0078125	7.758551769781e-05	\\
-1457.03125	7.87990163221553e-05	\\
-1456.0546875	7.84758563696597e-05	\\
-1455.078125	7.48777968638504e-05	\\
-1454.1015625	7.61068162973353e-05	\\
-1453.125	7.82970854409316e-05	\\
-1452.1484375	7.8695128385833e-05	\\
-1451.171875	7.60965684185346e-05	\\
-1450.1953125	7.43864778421112e-05	\\
-1449.21875	7.57725497916923e-05	\\
-1448.2421875	7.34005578772127e-05	\\
-1447.265625	7.5335852510673e-05	\\
-1446.2890625	7.83502034951454e-05	\\
-1445.3125	7.39544554641411e-05	\\
-1444.3359375	7.26578014743472e-05	\\
-1443.359375	7.16145666644951e-05	\\
-1442.3828125	7.00971529146556e-05	\\
-1441.40625	7.29144472534691e-05	\\
-1440.4296875	7.17902597811606e-05	\\
-1439.453125	6.74240917041058e-05	\\
-1438.4765625	6.83158130983193e-05	\\
-1437.5	7.2370207257806e-05	\\
-1436.5234375	7.34443940638404e-05	\\
-1435.546875	6.81122960545356e-05	\\
-1434.5703125	7.48859123450399e-05	\\
-1433.59375	7.53736544305388e-05	\\
-1432.6171875	7.08974107262045e-05	\\
-1431.640625	7.34469982724399e-05	\\
-1430.6640625	7.34514971213749e-05	\\
-1429.6875	7.36643164476299e-05	\\
-1428.7109375	6.93791424275273e-05	\\
-1427.734375	6.87954645505367e-05	\\
-1426.7578125	7.37805109906538e-05	\\
-1425.78125	6.931369376744e-05	\\
-1424.8046875	7.0560223672396e-05	\\
-1423.828125	6.92919935657025e-05	\\
-1422.8515625	6.74426502229268e-05	\\
-1421.875	6.6657924493826e-05	\\
-1420.8984375	6.37513118797598e-05	\\
-1419.921875	6.89820209088286e-05	\\
-1418.9453125	6.48440951795333e-05	\\
-1417.96875	6.793849845525e-05	\\
-1416.9921875	6.51378892035433e-05	\\
-1416.015625	6.70452570090251e-05	\\
-1415.0390625	6.23822152062584e-05	\\
-1414.0625	6.41806124294688e-05	\\
-1413.0859375	6.36630173253289e-05	\\
-1412.109375	6.97743061258551e-05	\\
-1411.1328125	6.70242702712006e-05	\\
-1410.15625	6.41096911058174e-05	\\
-1409.1796875	6.45347584130627e-05	\\
-1408.203125	6.67581552368028e-05	\\
-1407.2265625	6.60753801848849e-05	\\
-1406.25	6.33364536491678e-05	\\
-1405.2734375	6.48455583483387e-05	\\
-1404.296875	6.47010329382833e-05	\\
-1403.3203125	6.32665628347305e-05	\\
-1402.34375	6.24714571057972e-05	\\
-1401.3671875	6.32722317913688e-05	\\
-1400.390625	5.86533076096447e-05	\\
-1399.4140625	6.57242743387865e-05	\\
-1398.4375	6.14385207817319e-05	\\
-1397.4609375	6.00965126022514e-05	\\
-1396.484375	6.4731026641993e-05	\\
-1395.5078125	6.40218780220938e-05	\\
-1394.53125	6.07035262685473e-05	\\
-1393.5546875	6.26756450671761e-05	\\
-1392.578125	6.51744274446965e-05	\\
-1391.6015625	6.60401903603118e-05	\\
-1390.625	6.51352568314393e-05	\\
-1389.6484375	6.65228820519071e-05	\\
-1388.671875	6.33423797266171e-05	\\
-1387.6953125	6.55805637857821e-05	\\
-1386.71875	6.50488176927496e-05	\\
-1385.7421875	6.2034245526216e-05	\\
-1384.765625	6.23280059793866e-05	\\
-1383.7890625	6.05908813661378e-05	\\
-1382.8125	6.34824400696928e-05	\\
-1381.8359375	6.36928466503588e-05	\\
-1380.859375	6.45582455046432e-05	\\
-1379.8828125	6.33948982886355e-05	\\
-1378.90625	6.56836147566982e-05	\\
-1377.9296875	6.53127430331414e-05	\\
-1376.953125	6.38873765946827e-05	\\
-1375.9765625	6.45247064835191e-05	\\
-1375	6.5667556539795e-05	\\
-1374.0234375	6.36950855753925e-05	\\
-1373.046875	6.91481304062786e-05	\\
-1372.0703125	6.7579899371913e-05	\\
-1371.09375	6.48473465792706e-05	\\
-1370.1171875	6.65686397315042e-05	\\
-1369.140625	6.7994968786104e-05	\\
-1368.1640625	6.96060710171195e-05	\\
-1367.1875	6.42498776184753e-05	\\
-1366.2109375	6.53870339252002e-05	\\
-1365.234375	6.68939075334501e-05	\\
-1364.2578125	6.56118999252529e-05	\\
-1363.28125	6.5837244280497e-05	\\
-1362.3046875	6.45820091799972e-05	\\
-1361.328125	6.87311436460885e-05	\\
-1360.3515625	6.99253843239398e-05	\\
-1359.375	6.78018455954732e-05	\\
-1358.3984375	6.69894646863576e-05	\\
-1357.421875	6.56137043874455e-05	\\
-1356.4453125	7.02312253215074e-05	\\
-1355.46875	6.71260324968591e-05	\\
-1354.4921875	6.7517268705636e-05	\\
-1353.515625	6.93228486956672e-05	\\
-1352.5390625	6.83999720263898e-05	\\
-1351.5625	6.70928063656963e-05	\\
-1350.5859375	6.57344929635655e-05	\\
-1349.609375	6.83771511398777e-05	\\
-1348.6328125	6.55669505806988e-05	\\
-1347.65625	6.56561648024974e-05	\\
-1346.6796875	6.87887206203051e-05	\\
-1345.703125	6.82342076985609e-05	\\
-1344.7265625	6.95029445016341e-05	\\
-1343.75	6.87095211717615e-05	\\
-1342.7734375	6.64605673123184e-05	\\
-1341.796875	7.04635890886438e-05	\\
-1340.8203125	6.79781807181848e-05	\\
-1339.84375	6.58490950915478e-05	\\
-1338.8671875	6.67101475115517e-05	\\
-1337.890625	6.81581119217842e-05	\\
-1336.9140625	6.78832545478905e-05	\\
-1335.9375	7.06878654942634e-05	\\
-1334.9609375	6.81475738212234e-05	\\
-1333.984375	7.02538460269656e-05	\\
-1333.0078125	7.38891589467595e-05	\\
-1332.03125	6.95051102213233e-05	\\
-1331.0546875	7.01776702479196e-05	\\
-1330.078125	6.88350892377932e-05	\\
-1329.1015625	7.25150820050057e-05	\\
-1328.125	6.59242845534599e-05	\\
-1327.1484375	7.01559983222341e-05	\\
-1326.171875	6.8963408850741e-05	\\
-1325.1953125	6.44068227357736e-05	\\
-1324.21875	6.60685662262626e-05	\\
-1323.2421875	6.5419004578165e-05	\\
-1322.265625	6.7815766345406e-05	\\
-1321.2890625	7.05583748846293e-05	\\
-1320.3125	7.06180293908499e-05	\\
-1319.3359375	6.59705492036517e-05	\\
-1318.359375	7.12991097950039e-05	\\
-1317.3828125	6.92518726735722e-05	\\
-1316.40625	7.06334286651524e-05	\\
-1315.4296875	6.86757569705217e-05	\\
-1314.453125	6.58778809542842e-05	\\
-1313.4765625	6.43555949296255e-05	\\
-1312.5	6.81824359357388e-05	\\
-1311.5234375	6.87667701266491e-05	\\
-1310.546875	7.05855404789899e-05	\\
-1309.5703125	6.87629446122803e-05	\\
-1308.59375	6.57605482973842e-05	\\
-1307.6171875	6.6908017684161e-05	\\
-1306.640625	6.82256876689282e-05	\\
-1305.6640625	6.76318044371796e-05	\\
-1304.6875	7.00528158767789e-05	\\
-1303.7109375	6.17733545976853e-05	\\
-1302.734375	6.85718298498584e-05	\\
-1301.7578125	6.55875422488536e-05	\\
-1300.78125	6.59410084293307e-05	\\
-1299.8046875	6.7527196804861e-05	\\
-1298.828125	6.6006809172491e-05	\\
-1297.8515625	6.6570180010978e-05	\\
-1296.875	6.59150821089271e-05	\\
-1295.8984375	6.63202170074085e-05	\\
-1294.921875	6.92252941549353e-05	\\
-1293.9453125	6.63365364188209e-05	\\
-1292.96875	6.98217782254179e-05	\\
-1291.9921875	6.84226952075541e-05	\\
-1291.015625	7.31601352065427e-05	\\
-1290.0390625	7.05337747666699e-05	\\
-1289.0625	6.72771174631451e-05	\\
-1288.0859375	6.96775345843038e-05	\\
-1287.109375	6.62864245363469e-05	\\
-1286.1328125	6.86101961544285e-05	\\
-1285.15625	6.84274053370781e-05	\\
-1284.1796875	6.98577422777947e-05	\\
-1283.203125	6.63390143437494e-05	\\
-1282.2265625	6.92492644128387e-05	\\
-1281.25	6.88521755974399e-05	\\
-1280.2734375	7.01567046355731e-05	\\
-1279.296875	7.04905560221999e-05	\\
-1278.3203125	6.86514827682979e-05	\\
-1277.34375	7.05351347253344e-05	\\
-1276.3671875	6.80109972501767e-05	\\
-1275.390625	6.63777887413362e-05	\\
-1274.4140625	6.87451429344529e-05	\\
-1273.4375	6.84887458169861e-05	\\
-1272.4609375	6.6457786195029e-05	\\
-1271.484375	7.12070659873822e-05	\\
-1270.5078125	6.96965191510233e-05	\\
-1269.53125	6.97567601071068e-05	\\
-1268.5546875	7.09307080201771e-05	\\
-1267.578125	6.41488811460787e-05	\\
-1266.6015625	6.98968758095581e-05	\\
-1265.625	6.6186611650999e-05	\\
-1264.6484375	7.0039569335557e-05	\\
-1263.671875	6.77270227697897e-05	\\
-1262.6953125	6.83281005963804e-05	\\
-1261.71875	7.1722538534353e-05	\\
-1260.7421875	7.04542321423532e-05	\\
-1259.765625	6.96851007004959e-05	\\
-1258.7890625	6.88019113145735e-05	\\
-1257.8125	6.75260480464562e-05	\\
-1256.8359375	6.91060976508093e-05	\\
-1255.859375	6.97480829400829e-05	\\
-1254.8828125	7.16037449558508e-05	\\
-1253.90625	7.16329439334931e-05	\\
-1252.9296875	6.90473038104651e-05	\\
-1251.953125	6.4615329308258e-05	\\
-1250.9765625	7.14735488685434e-05	\\
-1250	6.64781413784053e-05	\\
-1249.0234375	6.71623312723465e-05	\\
-1248.046875	6.61474632870342e-05	\\
-1247.0703125	6.44357239103808e-05	\\
-1246.09375	6.82071264570812e-05	\\
-1245.1171875	6.88035718643392e-05	\\
-1244.140625	6.713632650417e-05	\\
-1243.1640625	6.49616837693749e-05	\\
-1242.1875	6.73683136311457e-05	\\
-1241.2109375	6.74499086296456e-05	\\
-1240.234375	6.52467472513109e-05	\\
-1239.2578125	6.80429209995823e-05	\\
-1238.28125	6.78691751347379e-05	\\
-1237.3046875	6.64326386309136e-05	\\
-1236.328125	6.97099061747371e-05	\\
-1235.3515625	6.65267791165447e-05	\\
-1234.375	6.56195042086053e-05	\\
-1233.3984375	6.85173701680674e-05	\\
-1232.421875	6.65050338514957e-05	\\
-1231.4453125	6.90592660542894e-05	\\
-1230.46875	6.79582561878122e-05	\\
-1229.4921875	6.5962919423388e-05	\\
-1228.515625	6.70255145000834e-05	\\
-1227.5390625	6.37460010383492e-05	\\
-1226.5625	6.76135261068323e-05	\\
-1225.5859375	6.47126633076921e-05	\\
-1224.609375	6.19015183896371e-05	\\
-1223.6328125	6.67342076049939e-05	\\
-1222.65625	6.26877687892543e-05	\\
-1221.6796875	6.6048105659517e-05	\\
-1220.703125	6.26405944943494e-05	\\
-1219.7265625	6.28736112215579e-05	\\
-1218.75	6.68729692302448e-05	\\
-1217.7734375	6.51675043175535e-05	\\
-1216.796875	6.32194579853002e-05	\\
-1215.8203125	6.83899261319679e-05	\\
-1214.84375	6.74153165381071e-05	\\
-1213.8671875	6.76664854876272e-05	\\
-1212.890625	6.82155205193998e-05	\\
-1211.9140625	6.59707875699144e-05	\\
-1210.9375	6.55732360050218e-05	\\
-1209.9609375	6.78967118300692e-05	\\
-1208.984375	6.71495481916372e-05	\\
-1208.0078125	6.57113902809526e-05	\\
-1207.03125	6.89219591262328e-05	\\
-1206.0546875	6.64651129450838e-05	\\
-1205.078125	6.56610516190351e-05	\\
-1204.1015625	6.76936609197741e-05	\\
-1203.125	6.6953879953473e-05	\\
-1202.1484375	7.19399533634852e-05	\\
-1201.171875	6.65489815899671e-05	\\
-1200.1953125	7.42308504139894e-05	\\
-1199.21875	6.79652464204832e-05	\\
-1198.2421875	6.63766692846154e-05	\\
-1197.265625	6.96758866621834e-05	\\
-1196.2890625	7.27762968111862e-05	\\
-1195.3125	6.54137769844863e-05	\\
-1194.3359375	6.88713724117622e-05	\\
-1193.359375	6.49440261903384e-05	\\
-1192.3828125	6.30069145705447e-05	\\
-1191.40625	6.37777926206248e-05	\\
-1190.4296875	6.7033004454306e-05	\\
-1189.453125	6.604257124235e-05	\\
-1188.4765625	6.64161832398737e-05	\\
-1187.5	6.59496683879167e-05	\\
-1186.5234375	6.34939945538749e-05	\\
-1185.546875	6.57271179973018e-05	\\
-1184.5703125	6.36322015530884e-05	\\
-1183.59375	7.0605775275637e-05	\\
-1182.6171875	6.29099897837659e-05	\\
-1181.640625	6.94711630729583e-05	\\
-1180.6640625	6.88772959004037e-05	\\
-1179.6875	7.28206953458705e-05	\\
-1178.7109375	6.41462815767788e-05	\\
-1177.734375	7.30107720998988e-05	\\
-1176.7578125	6.8838693059097e-05	\\
-1175.78125	6.95565091844482e-05	\\
-1174.8046875	7.04393400209118e-05	\\
-1173.828125	6.82201177215753e-05	\\
-1172.8515625	6.91472957196403e-05	\\
-1171.875	6.99138205280508e-05	\\
-1170.8984375	7.08172443268601e-05	\\
-1169.921875	6.86382580537582e-05	\\
-1168.9453125	6.48136541657187e-05	\\
-1167.96875	6.73738161310164e-05	\\
-1166.9921875	6.84954944928531e-05	\\
-1166.015625	6.48872280607072e-05	\\
-1165.0390625	6.45323019818727e-05	\\
-1164.0625	6.60441147987493e-05	\\
-1163.0859375	6.7207305044579e-05	\\
-1162.109375	6.44804364049596e-05	\\
-1161.1328125	6.79867818681255e-05	\\
-1160.15625	6.51992275674871e-05	\\
-1159.1796875	6.77030461193735e-05	\\
-1158.203125	6.70520111665015e-05	\\
-1157.2265625	6.60874518526667e-05	\\
-1156.25	7.09971319770534e-05	\\
-1155.2734375	6.36435264511356e-05	\\
-1154.296875	6.84609579350961e-05	\\
-1153.3203125	6.71656983258182e-05	\\
-1152.34375	6.26378774614573e-05	\\
-1151.3671875	6.3604669106545e-05	\\
-1150.390625	6.49662181617894e-05	\\
-1149.4140625	6.2001205097458e-05	\\
-1148.4375	6.59623675412944e-05	\\
-1147.4609375	6.3409427003136e-05	\\
-1146.484375	5.99776867723449e-05	\\
-1145.5078125	6.45515479279676e-05	\\
-1144.53125	5.82185150020489e-05	\\
-1143.5546875	6.04773953480733e-05	\\
-1142.578125	6.13309032126654e-05	\\
-1141.6015625	5.98106752960987e-05	\\
-1140.625	6.18091398639258e-05	\\
-1139.6484375	6.22062587312902e-05	\\
-1138.671875	6.3799193634555e-05	\\
-1137.6953125	6.91898908447183e-05	\\
-1136.71875	6.39121551318685e-05	\\
-1135.7421875	6.36487649178359e-05	\\
-1134.765625	6.54404634691754e-05	\\
-1133.7890625	6.55622317200848e-05	\\
-1132.8125	6.54795086878079e-05	\\
-1131.8359375	6.03486386837281e-05	\\
-1130.859375	6.27278035992702e-05	\\
-1129.8828125	6.23516163700128e-05	\\
-1128.90625	6.12560217095037e-05	\\
-1127.9296875	6.65451933584572e-05	\\
-1126.953125	5.97065083488191e-05	\\
-1125.9765625	5.97806809246344e-05	\\
-1125	6.08476233816929e-05	\\
-1124.0234375	6.07207449710612e-05	\\
-1123.046875	5.98242488928936e-05	\\
-1122.0703125	6.22053339975359e-05	\\
-1121.09375	6.64449609828207e-05	\\
-1120.1171875	6.35452356210098e-05	\\
-1119.140625	6.47526066464658e-05	\\
-1118.1640625	6.48976798603169e-05	\\
-1117.1875	6.35649021187401e-05	\\
-1116.2109375	6.30348972039558e-05	\\
-1115.234375	6.32587136090187e-05	\\
-1114.2578125	6.62266877608305e-05	\\
-1113.28125	6.58627775954683e-05	\\
-1112.3046875	6.62759519491931e-05	\\
-1111.328125	6.77106183613459e-05	\\
-1110.3515625	6.41198457653324e-05	\\
-1109.375	6.59678728311037e-05	\\
-1108.3984375	6.33566821855007e-05	\\
-1107.421875	6.78626423665579e-05	\\
-1106.4453125	6.30008991567295e-05	\\
-1105.46875	6.59698456179929e-05	\\
-1104.4921875	6.63748971005009e-05	\\
-1103.515625	6.78712215223263e-05	\\
-1102.5390625	6.4059607512419e-05	\\
-1101.5625	6.60752786336721e-05	\\
-1100.5859375	7.01258929973044e-05	\\
-1099.609375	7.65627778069728e-05	\\
-1098.6328125	7.08195888959784e-05	\\
-1097.65625	7.03112901341189e-05	\\
-1096.6796875	6.90674315041414e-05	\\
-1095.703125	6.93095712577861e-05	\\
-1094.7265625	6.99991801015524e-05	\\
-1093.75	6.80596703651228e-05	\\
-1092.7734375	6.81633474588662e-05	\\
-1091.796875	6.74632218338153e-05	\\
-1090.8203125	6.85930691760539e-05	\\
-1089.84375	6.60497069703568e-05	\\
-1088.8671875	6.39181351958004e-05	\\
-1087.890625	7.24392655123061e-05	\\
-1086.9140625	6.25746153950905e-05	\\
-1085.9375	6.66836431808738e-05	\\
-1084.9609375	7.01098677291221e-05	\\
-1083.984375	7.29996857909591e-05	\\
-1083.0078125	7.21392288580908e-05	\\
-1082.03125	7.08591025262047e-05	\\
-1081.0546875	6.95169051551765e-05	\\
-1080.078125	7.54731667599213e-05	\\
-1079.1015625	6.98283429156833e-05	\\
-1078.125	7.36731282692959e-05	\\
-1077.1484375	6.83463797232351e-05	\\
-1076.171875	7.09214376430789e-05	\\
-1075.1953125	7.06855431675193e-05	\\
-1074.21875	6.48234675613764e-05	\\
-1073.2421875	6.9622504876607e-05	\\
-1072.265625	6.6649555419675e-05	\\
-1071.2890625	6.71727304299139e-05	\\
-1070.3125	7.17433502010239e-05	\\
-1069.3359375	6.8543963371752e-05	\\
-1068.359375	6.96149440466915e-05	\\
-1067.3828125	7.04890409267202e-05	\\
-1066.40625	6.85365337238876e-05	\\
-1065.4296875	6.93085576872288e-05	\\
-1064.453125	6.75026081066393e-05	\\
-1063.4765625	7.01683221874523e-05	\\
-1062.5	6.8433771676788e-05	\\
-1061.5234375	6.74112185546912e-05	\\
-1060.546875	6.72915975262106e-05	\\
-1059.5703125	6.6029313877598e-05	\\
-1058.59375	6.86738197869905e-05	\\
-1057.6171875	6.34236424619112e-05	\\
-1056.640625	6.32140457668692e-05	\\
-1055.6640625	6.56509905458114e-05	\\
-1054.6875	6.26317713419106e-05	\\
-1053.7109375	6.31940735388848e-05	\\
-1052.734375	6.24716713617839e-05	\\
-1051.7578125	6.17189332520848e-05	\\
-1050.78125	6.15230151793205e-05	\\
-1049.8046875	6.11526533013724e-05	\\
-1048.828125	6.02337845798586e-05	\\
-1047.8515625	6.41869044078451e-05	\\
-1046.875	6.51538395812037e-05	\\
-1045.8984375	6.20625238744242e-05	\\
-1044.921875	6.52198178971121e-05	\\
-1043.9453125	6.12999893081955e-05	\\
-1042.96875	5.84419125370752e-05	\\
-1041.9921875	6.59692208656773e-05	\\
-1041.015625	6.51556088876854e-05	\\
-1040.0390625	6.45492203966491e-05	\\
-1039.0625	6.2323645074751e-05	\\
-1038.0859375	6.33090844600278e-05	\\
-1037.109375	6.04527228841909e-05	\\
-1036.1328125	5.91238012500008e-05	\\
-1035.15625	5.65641821030007e-05	\\
-1034.1796875	5.94008221411568e-05	\\
-1033.203125	6.16065000681489e-05	\\
-1032.2265625	6.01996969062148e-05	\\
-1031.25	6.44534061967717e-05	\\
-1030.2734375	6.16857578922188e-05	\\
-1029.296875	6.65519289824898e-05	\\
-1028.3203125	6.00011102101786e-05	\\
-1027.34375	6.42530816589476e-05	\\
-1026.3671875	5.99991677993778e-05	\\
-1025.390625	6.23979966529052e-05	\\
-1024.4140625	6.24614975416455e-05	\\
-1023.4375	6.35577867212347e-05	\\
-1022.4609375	6.3879619833322e-05	\\
-1021.484375	6.27209113338744e-05	\\
-1020.5078125	5.64056989752272e-05	\\
-1019.53125	5.85415381332116e-05	\\
-1018.5546875	5.63474670887001e-05	\\
-1017.578125	5.96909061133426e-05	\\
-1016.6015625	5.78217440508319e-05	\\
-1015.625	6.04270082282083e-05	\\
-1014.6484375	5.51267004148029e-05	\\
-1013.671875	5.93018074675025e-05	\\
-1012.6953125	5.69024837059391e-05	\\
-1011.71875	5.52986291703605e-05	\\
-1010.7421875	5.96715584085966e-05	\\
-1009.765625	6.24224504810131e-05	\\
-1008.7890625	6.30276913338236e-05	\\
-1007.8125	5.97954639704782e-05	\\
-1006.8359375	6.06504993278994e-05	\\
-1005.859375	6.47690231986976e-05	\\
-1004.8828125	6.20533977090868e-05	\\
-1003.90625	6.23964798482685e-05	\\
-1002.9296875	6.44363872569815e-05	\\
-1001.953125	6.28788949259345e-05	\\
-1000.9765625	6.18446188300364e-05	\\
-1000	5.47588424150285e-05	\\
-999.0234375	6.6853721700997e-05	\\
-998.046875	6.16304403435153e-05	\\
-997.0703125	6.09079881185355e-05	\\
-996.09375	6.56652743013789e-05	\\
-995.1171875	6.45262731978916e-05	\\
-994.140625	6.42860557335978e-05	\\
-993.1640625	6.56516763657906e-05	\\
-992.1875	6.26607654570264e-05	\\
-991.2109375	6.3632400780754e-05	\\
-990.234375	6.51458030383616e-05	\\
-989.2578125	6.74739824787806e-05	\\
-988.28125	7.04243378609903e-05	\\
-987.3046875	6.47178106392196e-05	\\
-986.328125	6.64523197304349e-05	\\
-985.3515625	7.12366275491977e-05	\\
-984.375	6.88198215306538e-05	\\
-983.3984375	6.84734769447627e-05	\\
-982.421875	6.73665443450159e-05	\\
-981.4453125	7.19277093035157e-05	\\
-980.46875	7.20867103066934e-05	\\
-979.4921875	6.60663289076446e-05	\\
-978.515625	6.69660538792972e-05	\\
-977.5390625	6.40127038505934e-05	\\
-976.5625	7.08042038874808e-05	\\
-975.5859375	6.13100431963276e-05	\\
-974.609375	6.38256981096458e-05	\\
-973.6328125	6.67900577223603e-05	\\
-972.65625	6.6867195444038e-05	\\
-971.6796875	6.48458065856741e-05	\\
-970.703125	6.72927464134353e-05	\\
-969.7265625	7.29196057781077e-05	\\
-968.75	6.76707718051758e-05	\\
-967.7734375	6.73629391424961e-05	\\
-966.796875	6.82825197036614e-05	\\
-965.8203125	6.16400583426501e-05	\\
-964.84375	6.59696586405282e-05	\\
-963.8671875	6.88282645447363e-05	\\
-962.890625	6.14994555154238e-05	\\
-961.9140625	6.71813311073519e-05	\\
-960.9375	6.34252879146914e-05	\\
-959.9609375	6.47013801775955e-05	\\
-958.984375	6.64468654485878e-05	\\
-958.0078125	6.67773991712554e-05	\\
-957.03125	6.65714297509414e-05	\\
-956.0546875	6.1672003779588e-05	\\
-955.078125	6.43579595035376e-05	\\
-954.1015625	6.55120159502242e-05	\\
-953.125	6.53690908900551e-05	\\
-952.1484375	6.36855311300202e-05	\\
-951.171875	6.56074936668609e-05	\\
-950.1953125	7.01502843126279e-05	\\
-949.21875	6.70202917437575e-05	\\
-948.2421875	6.51311620397466e-05	\\
-947.265625	6.97417652678032e-05	\\
-946.2890625	6.77294351228078e-05	\\
-945.3125	7.08785524107703e-05	\\
-944.3359375	7.0876657097409e-05	\\
-943.359375	6.53818106963872e-05	\\
-942.3828125	6.51205244368735e-05	\\
-941.40625	6.90176274263389e-05	\\
-940.4296875	6.77272739404746e-05	\\
-939.453125	6.68759698222319e-05	\\
-938.4765625	6.54528370687628e-05	\\
-937.5	6.62623157955611e-05	\\
-936.5234375	6.53916931384321e-05	\\
-935.546875	6.35891296327325e-05	\\
-934.5703125	6.97272914768325e-05	\\
-933.59375	6.50775776152462e-05	\\
-932.6171875	7.05448371272833e-05	\\
-931.640625	6.69106672583777e-05	\\
-930.6640625	6.83536266571696e-05	\\
-929.6875	6.64645463300953e-05	\\
-928.7109375	6.99467784791981e-05	\\
-927.734375	6.98726776000354e-05	\\
-926.7578125	6.50239301327616e-05	\\
-925.78125	6.90330057128334e-05	\\
-924.8046875	6.77252921353276e-05	\\
-923.828125	6.61127654243313e-05	\\
-922.8515625	6.67976576833254e-05	\\
-921.875	6.7598559338298e-05	\\
-920.8984375	6.89090059379345e-05	\\
-919.921875	6.46775935811846e-05	\\
-918.9453125	6.79621297948662e-05	\\
-917.96875	6.79536680527615e-05	\\
-916.9921875	6.71394157389019e-05	\\
-916.015625	7.33347460437006e-05	\\
-915.0390625	7.11939792337511e-05	\\
-914.0625	7.14108671057831e-05	\\
-913.0859375	6.63264125000402e-05	\\
-912.109375	7.02520855197225e-05	\\
-911.1328125	7.11866304506146e-05	\\
-910.15625	6.91691344835967e-05	\\
-909.1796875	6.96569686911364e-05	\\
-908.203125	7.09834687958036e-05	\\
-907.2265625	6.64579064677853e-05	\\
-906.25	6.79909240671382e-05	\\
-905.2734375	7.10531221535746e-05	\\
-904.296875	6.7982730985679e-05	\\
-903.3203125	7.03594389473756e-05	\\
-902.34375	7.53297849785126e-05	\\
-901.3671875	6.75856426857327e-05	\\
-900.390625	8.95440330767631e-05	\\
-899.4140625	7.11948253014517e-05	\\
-898.4375	7.67067712867883e-05	\\
-897.4609375	7.19687364363074e-05	\\
-896.484375	6.6927796023877e-05	\\
-895.5078125	7.42153383850339e-05	\\
-894.53125	7.36875158630846e-05	\\
-893.5546875	7.01773458111854e-05	\\
-892.578125	7.26958410178915e-05	\\
-891.6015625	6.74416339710723e-05	\\
-890.625	6.6878849237599e-05	\\
-889.6484375	7.04721183923458e-05	\\
-888.671875	6.9695125708277e-05	\\
-887.6953125	7.23618149774335e-05	\\
-886.71875	6.79822696026146e-05	\\
-885.7421875	6.94685264665923e-05	\\
-884.765625	6.88636124087122e-05	\\
-883.7890625	7.10565229504782e-05	\\
-882.8125	6.77591260829007e-05	\\
-881.8359375	7.23513530837827e-05	\\
-880.859375	6.97430059507621e-05	\\
-879.8828125	6.97225934567152e-05	\\
-878.90625	6.99656403416864e-05	\\
-877.9296875	6.70578338138251e-05	\\
-876.953125	6.50111643546536e-05	\\
-875.9765625	6.25747441128518e-05	\\
-875	6.44399343643494e-05	\\
-874.0234375	6.49205391500694e-05	\\
-873.046875	6.99197044172686e-05	\\
-872.0703125	6.74708799729535e-05	\\
-871.09375	7.341082811468e-05	\\
-870.1171875	7.20486218282481e-05	\\
-869.140625	6.80045463252837e-05	\\
-868.1640625	7.21611448718813e-05	\\
-867.1875	6.77043097438515e-05	\\
-866.2109375	6.52626963946067e-05	\\
-865.234375	6.71480890338329e-05	\\
-864.2578125	7.08505463544328e-05	\\
-863.28125	6.87891772840725e-05	\\
-862.3046875	6.91786327602385e-05	\\
-861.328125	6.96557578751355e-05	\\
-860.3515625	6.82305950305343e-05	\\
-859.375	6.81478160875427e-05	\\
-858.3984375	7.12919463103486e-05	\\
-857.421875	6.86744855440357e-05	\\
-856.4453125	6.72396445359345e-05	\\
-855.46875	6.63275267983291e-05	\\
-854.4921875	7.04881608561013e-05	\\
-853.515625	6.82202760111455e-05	\\
-852.5390625	6.45332923678176e-05	\\
-851.5625	6.62103455799304e-05	\\
-850.5859375	6.89993398703786e-05	\\
-849.609375	7.11026642318626e-05	\\
-848.6328125	6.96773365249183e-05	\\
-847.65625	6.75466526102453e-05	\\
-846.6796875	6.71601031645592e-05	\\
-845.703125	6.86805189109188e-05	\\
-844.7265625	6.99058977963653e-05	\\
-843.75	7.07488549669287e-05	\\
-842.7734375	6.73652541141961e-05	\\
-841.796875	6.54654817103887e-05	\\
-840.8203125	6.80999737554835e-05	\\
-839.84375	6.46505293529778e-05	\\
-838.8671875	6.74612194454303e-05	\\
-837.890625	7.10770757767057e-05	\\
-836.9140625	6.96239730628935e-05	\\
-835.9375	7.08447452311935e-05	\\
-834.9609375	6.66908816543824e-05	\\
-833.984375	6.39651627796424e-05	\\
-833.0078125	6.90041941877119e-05	\\
-832.03125	6.75624644408072e-05	\\
-831.0546875	6.77106107153587e-05	\\
-830.078125	6.50045023384774e-05	\\
-829.1015625	6.75769662204421e-05	\\
-828.125	6.60957146828018e-05	\\
-827.1484375	7.42968853825066e-05	\\
-826.171875	7.1691012210365e-05	\\
-825.1953125	7.14753051467898e-05	\\
-824.21875	7.04700845177272e-05	\\
-823.2421875	6.90707092707262e-05	\\
-822.265625	6.75042842579013e-05	\\
-821.2890625	7.05118254748193e-05	\\
-820.3125	6.82694192000847e-05	\\
-819.3359375	7.30231819063131e-05	\\
-818.359375	7.04844774435743e-05	\\
-817.3828125	6.7878693068829e-05	\\
-816.40625	6.78187870466832e-05	\\
-815.4296875	7.42514074319071e-05	\\
-814.453125	7.16280524976806e-05	\\
-813.4765625	7.01368057188717e-05	\\
-812.5	7.13289582454411e-05	\\
-811.5234375	7.14677424923977e-05	\\
-810.546875	6.67344848947082e-05	\\
-809.5703125	7.18771643957909e-05	\\
-808.59375	6.60098923583405e-05	\\
-807.6171875	7.1656501718164e-05	\\
-806.640625	6.86164268614558e-05	\\
-805.6640625	6.60557929277833e-05	\\
-804.6875	7.25828525795934e-05	\\
-803.7109375	6.81460468187831e-05	\\
-802.734375	7.4035724258426e-05	\\
-801.7578125	7.55821035083963e-05	\\
-800.78125	6.67355441040047e-05	\\
-799.8046875	6.84681542078107e-05	\\
-798.828125	7.04093223606068e-05	\\
-797.8515625	6.62788122404719e-05	\\
-796.875	7.53795962843891e-05	\\
-795.8984375	6.85699175874229e-05	\\
-794.921875	7.12436295208083e-05	\\
-793.9453125	7.13659324493596e-05	\\
-792.96875	7.00602533369677e-05	\\
-791.9921875	7.16883153276089e-05	\\
-791.015625	6.86648110643859e-05	\\
-790.0390625	7.19363034915858e-05	\\
-789.0625	7.54428639871377e-05	\\
-788.0859375	7.60040315252318e-05	\\
-787.109375	7.27694329394658e-05	\\
-786.1328125	6.76557018593731e-05	\\
-785.15625	7.38749702277021e-05	\\
-784.1796875	6.798550366956e-05	\\
-783.203125	7.52539695521641e-05	\\
-782.2265625	8.01147675551712e-05	\\
-781.25	7.50269954176608e-05	\\
-780.2734375	7.86179036919471e-05	\\
-779.296875	7.58856176340719e-05	\\
-778.3203125	7.68903111244779e-05	\\
-777.34375	7.75261470687972e-05	\\
-776.3671875	7.683769907192e-05	\\
-775.390625	8.04835426445661e-05	\\
-774.4140625	7.80888948922121e-05	\\
-773.4375	7.79129871049322e-05	\\
-772.4609375	7.81313724581926e-05	\\
-771.484375	6.81241422301854e-05	\\
-770.5078125	7.95313795708572e-05	\\
-769.53125	7.53669681287147e-05	\\
-768.5546875	7.85915320627036e-05	\\
-767.578125	7.51388581553248e-05	\\
-766.6015625	7.51846577648262e-05	\\
-765.625	8.07173474088082e-05	\\
-764.6484375	8.24059163793652e-05	\\
-763.671875	7.60058937890154e-05	\\
-762.6953125	8.22668455185644e-05	\\
-761.71875	7.44641962619934e-05	\\
-760.7421875	7.94401928560239e-05	\\
-759.765625	7.86677599350962e-05	\\
-758.7890625	8.22607641913667e-05	\\
-757.8125	8.02125578781262e-05	\\
-756.8359375	8.38448621800098e-05	\\
-755.859375	7.89823894407827e-05	\\
-754.8828125	8.58060819290417e-05	\\
-753.90625	7.76713370401765e-05	\\
-752.9296875	7.6700077696191e-05	\\
-751.953125	7.31789682289427e-05	\\
-750.9765625	7.81384195566519e-05	\\
-750	7.65371557530473e-05	\\
-749.0234375	7.6336230577654e-05	\\
-748.046875	7.70731774500044e-05	\\
-747.0703125	8.22826741837437e-05	\\
-746.09375	8.08958761303337e-05	\\
-745.1171875	8.13018254551484e-05	\\
-744.140625	7.95372750406881e-05	\\
-743.1640625	7.95356613176687e-05	\\
-742.1875	8.37216987109264e-05	\\
-741.2109375	8.0516178238286e-05	\\
-740.234375	8.54232329621048e-05	\\
-739.2578125	8.24826070188931e-05	\\
-738.28125	8.26406275227646e-05	\\
-737.3046875	7.37035511152562e-05	\\
-736.328125	8.43845342831024e-05	\\
-735.3515625	8.89315053218026e-05	\\
-734.375	8.71333132609698e-05	\\
-733.3984375	8.10763269565648e-05	\\
-732.421875	7.96132473204573e-05	\\
-731.4453125	8.33930487737855e-05	\\
-730.46875	8.45284543655367e-05	\\
-729.4921875	8.25625443913624e-05	\\
-728.515625	8.20529876824677e-05	\\
-727.5390625	8.58737745774042e-05	\\
-726.5625	8.48775379215344e-05	\\
-725.5859375	8.11572632848969e-05	\\
-724.609375	8.77243727557097e-05	\\
-723.6328125	8.39038828762613e-05	\\
-722.65625	8.42244139693051e-05	\\
-721.6796875	8.66878140509249e-05	\\
-720.703125	7.9642359133523e-05	\\
-719.7265625	8.83633580903672e-05	\\
-718.75	8.42047667012381e-05	\\
-717.7734375	8.49628820795954e-05	\\
-716.796875	8.12844728662888e-05	\\
-715.8203125	7.93785174410866e-05	\\
-714.84375	8.48300253425636e-05	\\
-713.8671875	7.86313297540235e-05	\\
-712.890625	8.3548846840709e-05	\\
-711.9140625	8.66868773278306e-05	\\
-710.9375	8.46276730643292e-05	\\
-709.9609375	8.67041765676348e-05	\\
-708.984375	8.30483703765321e-05	\\
-708.0078125	8.73230286142442e-05	\\
-707.03125	8.44108725927891e-05	\\
-706.0546875	7.82364994256376e-05	\\
-705.078125	8.10458588470221e-05	\\
-704.1015625	8.28366928465563e-05	\\
-703.125	7.88250600496612e-05	\\
-702.1484375	7.93318974701037e-05	\\
-701.171875	8.1761362278978e-05	\\
-700.1953125	9.36338004692343e-05	\\
-699.21875	7.99499974417047e-05	\\
-698.2421875	8.7007958196228e-05	\\
-697.265625	7.6890280246582e-05	\\
-696.2890625	7.50391297920927e-05	\\
-695.3125	8.12500309730263e-05	\\
-694.3359375	7.737924877368e-05	\\
-693.359375	7.56710654246821e-05	\\
-692.3828125	7.7345653044079e-05	\\
-691.40625	7.83695154667629e-05	\\
-690.4296875	8.31857455407127e-05	\\
-689.453125	7.71483804000178e-05	\\
-688.4765625	7.6360986914415e-05	\\
-687.5	7.26276615192441e-05	\\
-686.5234375	7.31458342588154e-05	\\
-685.546875	7.85557236100321e-05	\\
-684.5703125	7.64484088619277e-05	\\
-683.59375	7.88246615696383e-05	\\
-682.6171875	8.18492059440863e-05	\\
-681.640625	7.83439450109525e-05	\\
-680.6640625	7.72306604552676e-05	\\
-679.6875	7.5310443699612e-05	\\
-678.7109375	8.50346260081138e-05	\\
-677.734375	7.60356172276647e-05	\\
-676.7578125	7.5705935919925e-05	\\
-675.78125	7.48001240955915e-05	\\
-674.8046875	7.4765208030725e-05	\\
-673.828125	7.67244836071644e-05	\\
-672.8515625	7.99112205819859e-05	\\
-671.875	8.14112416406041e-05	\\
-670.8984375	7.78320096518232e-05	\\
-669.921875	7.99819738935617e-05	\\
-668.9453125	8.51547851670088e-05	\\
-667.96875	7.61167835089823e-05	\\
-666.9921875	7.72989059773217e-05	\\
-666.015625	8.1467284629172e-05	\\
-665.0390625	8.03957122985936e-05	\\
-664.0625	7.91934886708126e-05	\\
-663.0859375	8.15729496250901e-05	\\
-662.109375	7.47548734531583e-05	\\
-661.1328125	7.55048003514787e-05	\\
-660.15625	8.24165249736848e-05	\\
-659.1796875	8.01670291900142e-05	\\
-658.203125	7.7561242526877e-05	\\
-657.2265625	8.88045551848492e-05	\\
-656.25	8.51679657393198e-05	\\
-655.2734375	8.01256414230914e-05	\\
-654.296875	8.26208283266832e-05	\\
-653.3203125	8.7820028559817e-05	\\
-652.34375	8.40099947856051e-05	\\
-651.3671875	8.68871803784865e-05	\\
-650.390625	8.78408147565718e-05	\\
-649.4140625	8.45304298956547e-05	\\
-648.4375	9.86922423891199e-05	\\
-647.4609375	9.04575593921075e-05	\\
-646.484375	9.41957749023373e-05	\\
-645.5078125	9.0033249271762e-05	\\
-644.53125	9.72269710009691e-05	\\
-643.5546875	9.00757322003627e-05	\\
-642.578125	8.73496839842843e-05	\\
-641.6015625	9.7410458999655e-05	\\
-640.625	9.00891700266774e-05	\\
-639.6484375	8.98212393905302e-05	\\
-638.671875	8.95521816982996e-05	\\
-637.6953125	9.31450314862085e-05	\\
-636.71875	8.94174283712485e-05	\\
-635.7421875	8.79114860769018e-05	\\
-634.765625	9.59486666239664e-05	\\
-633.7890625	8.62177239415606e-05	\\
-632.8125	8.96329109969644e-05	\\
-631.8359375	9.07386771772197e-05	\\
-630.859375	9.49181633558847e-05	\\
-629.8828125	9.2381143419937e-05	\\
-628.90625	8.8195335246493e-05	\\
-627.9296875	9.32285632055811e-05	\\
-626.953125	8.79502571825146e-05	\\
-625.9765625	8.88473235794753e-05	\\
-625	9.11495648049478e-05	\\
-624.0234375	8.49278882813787e-05	\\
-623.046875	8.89677687670182e-05	\\
-622.0703125	9.41406360221423e-05	\\
-621.09375	9.70707222585133e-05	\\
-620.1171875	0.000100798448448989	\\
-619.140625	0.000101343062025395	\\
-618.1640625	9.57003969015275e-05	\\
-617.1875	0.000100764147184561	\\
-616.2109375	9.23189397003675e-05	\\
-615.234375	0.000103837849579806	\\
-614.2578125	0.000101691329327943	\\
-613.28125	9.58064991214552e-05	\\
-612.3046875	0.00010426949202809	\\
-611.328125	0.000103712880035242	\\
-610.3515625	9.80665564381802e-05	\\
-609.375	9.99813709023521e-05	\\
-608.3984375	0.000103938225123689	\\
-607.421875	9.73109824610167e-05	\\
-606.4453125	9.239529407623e-05	\\
-605.46875	0.00010045993757962	\\
-604.4921875	0.000110113800265706	\\
-603.515625	9.5832158195198e-05	\\
-602.5390625	0.000100413853025696	\\
-601.5625	0.000104444869073848	\\
-600.5859375	0.000102442844749196	\\
-599.609375	9.26203610437632e-05	\\
-598.6328125	0.000109168925297951	\\
-597.65625	0.000117020998015368	\\
-596.6796875	0.000114256142876867	\\
-595.703125	0.000110618477338483	\\
-594.7265625	0.000110970731437298	\\
-593.75	0.000108328472127119	\\
-592.7734375	0.000104718976018817	\\
-591.796875	0.00010616959219433	\\
-590.8203125	0.000107529421813304	\\
-589.84375	0.00011084443922736	\\
-588.8671875	0.000105031988533047	\\
-587.890625	0.00010938720471658	\\
-586.9140625	0.000110647939363798	\\
-585.9375	0.000120362046163597	\\
-584.9609375	0.000109619146735627	\\
-583.984375	0.000115180943622521	\\
-583.0078125	0.000118518180705962	\\
-582.03125	0.00011895006469227	\\
-581.0546875	0.000115859576717374	\\
-580.078125	0.00012649714235931	\\
-579.1015625	0.000121885691497669	\\
-578.125	0.000125216820418991	\\
-577.1484375	0.000130603183172555	\\
-576.171875	0.000135140639201665	\\
-575.1953125	0.000132373188180436	\\
-574.21875	0.000128870331024554	\\
-573.2421875	0.000132026746252888	\\
-572.265625	0.000138033248088782	\\
-571.2890625	0.000126095660876482	\\
-570.3125	0.000126987686316666	\\
-569.3359375	0.000124650174447538	\\
-568.359375	0.00012143240440063	\\
-567.3828125	0.000119584072180596	\\
-566.40625	0.000127822730701422	\\
-565.4296875	0.000118487279799562	\\
-564.453125	0.000122999336630295	\\
-563.4765625	0.000122234428365286	\\
-562.5	0.000120447750243848	\\
-561.5234375	0.000116610084518873	\\
-560.546875	0.000125505069855698	\\
-559.5703125	0.000134494541021666	\\
-558.59375	0.00013691702649593	\\
-557.6171875	0.000132470326040561	\\
-556.640625	0.000134149311064473	\\
-555.6640625	0.000133499932340804	\\
-554.6875	0.000142669801201252	\\
-553.7109375	0.000131329380780105	\\
-552.734375	0.000130223908852405	\\
-551.7578125	0.000132877611535138	\\
-550.78125	0.000123144980123309	\\
-549.8046875	0.000128615360979393	\\
-548.828125	0.000127397784887523	\\
-547.8515625	0.000121400324832612	\\
-546.875	0.000119951981259982	\\
-545.8984375	0.000127617376229444	\\
-544.921875	0.000122580019733621	\\
-543.9453125	0.000123622639848901	\\
-542.96875	0.000124206198323686	\\
-541.9921875	0.00012805205270312	\\
-541.015625	0.000125733511019605	\\
-540.0390625	0.00011919738249584	\\
-539.0625	0.000121727360833229	\\
-538.0859375	0.000124610513303601	\\
-537.109375	0.000125298128129049	\\
-536.1328125	0.000120674602279962	\\
-535.15625	0.000125937498285834	\\
-534.1796875	0.000125876402597088	\\
-533.203125	0.000117955672824248	\\
-532.2265625	0.00011844391713097	\\
-531.25	0.000132150919763452	\\
-530.2734375	0.000134606409823943	\\
-529.296875	0.00012491047805622	\\
-528.3203125	0.000127700770282869	\\
-527.34375	0.000133186811241018	\\
-526.3671875	0.000124868527084752	\\
-525.390625	0.000116360406982228	\\
-524.4140625	0.000135168789011112	\\
-523.4375	0.000121274567577714	\\
-522.4609375	0.000122297043100894	\\
-521.484375	0.000122627541240661	\\
-520.5078125	0.000131610430229945	\\
-519.53125	0.000121269526467468	\\
-518.5546875	0.000124177761445872	\\
-517.578125	0.000145901440623652	\\
-516.6015625	0.000134473212255969	\\
-515.625	0.000125280055601252	\\
-514.6484375	0.000147283499490093	\\
-513.671875	0.000144368608803723	\\
-512.6953125	0.0001314732660576	\\
-511.71875	0.000139383286192448	\\
-510.7421875	0.000146661812202005	\\
-509.765625	0.000138369275934094	\\
-508.7890625	0.000140655735575329	\\
-507.8125	0.000145489050840206	\\
-506.8359375	0.000142988824612314	\\
-505.859375	0.000149652754070551	\\
-504.8828125	0.000148510493015892	\\
-503.90625	0.000149255325881289	\\
-502.9296875	0.000147368561282073	\\
-501.953125	0.000169493245343257	\\
-500.9765625	0.000142858955822049	\\
-500	0.000192510697231583	\\
-499.0234375	0.000130786323153755	\\
-498.046875	0.000169657574597628	\\
-497.0703125	0.000153435965371445	\\
-496.09375	0.000145975194289466	\\
-495.1171875	0.000167512140484853	\\
-494.140625	0.000165766175294255	\\
-493.1640625	0.000161584305374303	\\
-492.1875	0.000165668418961052	\\
-491.2109375	0.000173391158748671	\\
-490.234375	0.000158769809150336	\\
-489.2578125	0.000167342517433179	\\
-488.28125	0.000177921244829601	\\
-487.3046875	0.000172512757765062	\\
-486.328125	0.000171533492922521	\\
-485.3515625	0.000178992647852694	\\
-484.375	0.000174709628666874	\\
-483.3984375	0.000181151251544531	\\
-482.421875	0.000176465878248154	\\
-481.4453125	0.00017632576029744	\\
-480.46875	0.000182037397571893	\\
-479.4921875	0.000178818021470093	\\
-478.515625	0.00017538047305446	\\
-477.5390625	0.000176118692535445	\\
-476.5625	0.000169908153015317	\\
-475.5859375	0.000172228166874569	\\
-474.609375	0.000181008555591869	\\
-473.6328125	0.000177717380114034	\\
-472.65625	0.000168910340252127	\\
-471.6796875	0.000183625295047535	\\
-470.703125	0.000170785729169803	\\
-469.7265625	0.000171187892027408	\\
-468.75	0.000183933940097566	\\
-467.7734375	0.000180806885256685	\\
-466.796875	0.000172836656438737	\\
-465.8203125	0.000172900430350052	\\
-464.84375	0.000157876551048696	\\
-463.8671875	0.000169343033556901	\\
-462.890625	0.000161628409612299	\\
-461.9140625	0.000162339972381576	\\
-460.9375	0.000158686922118638	\\
-459.9609375	0.000147536249673683	\\
-458.984375	0.00014333209882431	\\
-458.0078125	0.000145729872419009	\\
-457.03125	0.000131446152189194	\\
-456.0546875	0.000136780186315355	\\
-455.078125	0.000139901218005887	\\
-454.1015625	0.000117661738853567	\\
-453.125	0.000126702205542897	\\
-452.1484375	0.000124253733447229	\\
-451.171875	0.000131120527749436	\\
-450.1953125	0.000129926087326326	\\
-449.21875	0.00013869322985632	\\
-448.2421875	0.000134037165649966	\\
-447.265625	0.000139925976060546	\\
-446.2890625	0.000126524328790327	\\
-445.3125	0.000137067788430962	\\
-444.3359375	0.0001223023284652	\\
-443.359375	0.000125161048604526	\\
-442.3828125	0.000141732226142642	\\
-441.40625	0.000132168174243133	\\
-440.4296875	0.000105117423357357	\\
-439.453125	0.000110283763748698	\\
-438.4765625	0.000117616811796288	\\
-437.5	0.000106526020508991	\\
-436.5234375	0.000110111197935794	\\
-435.546875	9.80445697787286e-05	\\
-434.5703125	9.21738462961723e-05	\\
-433.59375	9.28646853985357e-05	\\
-432.6171875	0.000101696713851658	\\
-431.640625	9.71519865995132e-05	\\
-430.6640625	0.000109620728674336	\\
-429.6875	0.000120259524239122	\\
-428.7109375	0.000100096297572913	\\
-427.734375	0.000108937238058265	\\
-426.7578125	0.000109761742914639	\\
-425.78125	8.93725126665433e-05	\\
-424.8046875	9.59771620749605e-05	\\
-423.828125	0.000103084589296464	\\
-422.8515625	9.63687830571634e-05	\\
-421.875	7.31035043062355e-05	\\
-420.8984375	0.000106270507366382	\\
-419.921875	8.54691140314757e-05	\\
-418.9453125	7.16138995094108e-05	\\
-417.96875	8.46956131708631e-05	\\
-416.9921875	9.9663635139712e-05	\\
-416.015625	6.78213935199465e-05	\\
-415.0390625	9.355745709413e-05	\\
-414.0625	0.000104354521174701	\\
-413.0859375	8.50978921439517e-05	\\
-412.109375	8.9097312266362e-05	\\
-411.1328125	9.22543940366173e-05	\\
-410.15625	8.46262668867822e-05	\\
-409.1796875	9.09645702885232e-05	\\
-408.203125	9.31275673168977e-05	\\
-407.2265625	9.31598089872681e-05	\\
-406.25	8.7109489777046e-05	\\
-405.2734375	8.47750135555258e-05	\\
-404.296875	8.55452694138191e-05	\\
-403.3203125	8.10697326349083e-05	\\
-402.34375	8.66593614073627e-05	\\
-401.3671875	7.20393326530275e-05	\\
-400.390625	9.6501289643104e-05	\\
-399.4140625	7.15746665385294e-05	\\
-398.4375	7.23304683667219e-05	\\
-397.4609375	5.80707349539247e-05	\\
-396.484375	7.01471481214424e-05	\\
-395.5078125	6.58947664965902e-05	\\
-394.53125	5.74182208741402e-05	\\
-393.5546875	5.69559977656469e-05	\\
-392.578125	5.68598266062425e-05	\\
-391.6015625	6.45471261387108e-05	\\
-390.625	6.02730472175325e-05	\\
-389.6484375	6.01442715901982e-05	\\
-388.671875	4.85367105351906e-05	\\
-387.6953125	7.12546911671136e-05	\\
-386.71875	7.24624363164503e-05	\\
-385.7421875	5.0840900487861e-05	\\
-384.765625	5.00770739311747e-05	\\
-383.7890625	7.17673437008384e-05	\\
-382.8125	5.80730591677586e-05	\\
-381.8359375	5.17994285106407e-05	\\
-380.859375	4.98596344304154e-05	\\
-379.8828125	5.5285507866308e-05	\\
-378.90625	5.9083352130038e-05	\\
-377.9296875	5.29404943017949e-05	\\
-376.953125	4.88046484267339e-05	\\
-375.9765625	5.12515418693312e-05	\\
-375	4.72732652419262e-05	\\
-374.0234375	4.58666637442162e-05	\\
-373.046875	5.38492707420389e-05	\\
-372.0703125	4.65541917758266e-05	\\
-371.09375	4.02461282273771e-05	\\
-370.1171875	3.39073558319184e-05	\\
-369.140625	3.67147942836088e-05	\\
-368.1640625	4.29666108340014e-05	\\
-367.1875	4.80564071904314e-05	\\
-366.2109375	4.82408783021087e-05	\\
-365.234375	4.23017400096584e-05	\\
-364.2578125	4.60577772546571e-05	\\
-363.28125	5.03773648532554e-05	\\
-362.3046875	5.09889312211145e-05	\\
-361.328125	3.04843575845669e-05	\\
-360.3515625	2.01946071211084e-05	\\
-359.375	3.06401451910024e-05	\\
-358.3984375	3.37665668356188e-05	\\
-357.421875	3.89615292229264e-05	\\
-356.4453125	4.59543417507938e-05	\\
-355.46875	3.25228774541245e-05	\\
-354.4921875	4.08264790123115e-05	\\
-353.515625	4.02551683406993e-05	\\
-352.5390625	3.43586649652947e-05	\\
-351.5625	3.1304849998552e-05	\\
-350.5859375	3.63511294339827e-05	\\
-349.609375	4.03547142520708e-05	\\
-348.6328125	3.42759694266904e-05	\\
-347.65625	4.07966012389524e-05	\\
-346.6796875	2.68226289267262e-05	\\
-345.703125	3.02709119707949e-05	\\
-344.7265625	4.19873246621054e-05	\\
-343.75	3.09085871749736e-05	\\
-342.7734375	3.6205280880896e-05	\\
-341.796875	3.43506388422346e-05	\\
-340.8203125	3.63217315873103e-05	\\
-339.84375	2.88210913179573e-05	\\
-338.8671875	3.48555351560344e-05	\\
-337.890625	2.61917387596942e-05	\\
-336.9140625	2.70445524879045e-05	\\
-335.9375	2.5052996458113e-05	\\
-334.9609375	2.92145857788575e-05	\\
-333.984375	1.58830418905999e-05	\\
-333.0078125	3.04359131188758e-05	\\
-332.03125	1.85646003916347e-05	\\
-331.0546875	1.71201249729755e-05	\\
-330.078125	1.13583985141433e-05	\\
-329.1015625	2.07885993332942e-05	\\
-328.125	1.92620111499445e-05	\\
-327.1484375	1.64967993616328e-05	\\
-326.171875	1.6155927323561e-05	\\
-325.1953125	3.64860006623852e-05	\\
-324.21875	2.46123298331049e-05	\\
-323.2421875	2.09339505615728e-05	\\
-322.265625	2.35404247708284e-05	\\
-321.2890625	3.01636546459971e-05	\\
-320.3125	2.83679098327555e-05	\\
-319.3359375	3.3888024964724e-05	\\
-318.359375	3.39880481433425e-05	\\
-317.3828125	3.55647923948945e-05	\\
-316.40625	2.67965313689949e-05	\\
-315.4296875	2.80301696489167e-05	\\
-314.453125	3.124290469137e-05	\\
-313.4765625	3.56658239121104e-05	\\
-312.5	2.38451376816011e-05	\\
-311.5234375	2.95419696381749e-05	\\
-310.546875	3.26788240862855e-05	\\
-309.5703125	2.65631674253829e-05	\\
-308.59375	2.10296015638265e-05	\\
-307.6171875	3.06940753637191e-05	\\
-306.640625	2.59208076659082e-05	\\
-305.6640625	2.23979656970141e-05	\\
-304.6875	1.47012242417948e-05	\\
-303.7109375	2.27223654370756e-05	\\
-302.734375	2.67343927100232e-05	\\
-301.7578125	3.12084821778605e-05	\\
-300.78125	2.58367803913633e-05	\\
-299.8046875	9.96168891967017e-06	\\
-298.828125	3.83320319236552e-05	\\
-297.8515625	3.13763477767729e-05	\\
-296.875	2.5916503017615e-05	\\
-295.8984375	2.9821736910079e-05	\\
-294.921875	3.40408976080305e-05	\\
-293.9453125	3.77209419308307e-05	\\
-292.96875	2.81261648045329e-05	\\
-291.9921875	2.99622867145511e-05	\\
-291.015625	2.511724860973e-05	\\
-290.0390625	3.38118035215084e-05	\\
-289.0625	2.99505746955742e-05	\\
-288.0859375	2.60241237979022e-05	\\
-287.109375	2.96181638963043e-05	\\
-286.1328125	3.21478171494118e-05	\\
-285.15625	3.65586881577751e-05	\\
-284.1796875	2.64029855516337e-05	\\
-283.203125	4.07013279249842e-05	\\
-282.2265625	2.36661423317752e-05	\\
-281.25	4.39655412077424e-05	\\
-280.2734375	3.02228153559455e-05	\\
-279.296875	3.10147510027345e-05	\\
-278.3203125	3.94788502610931e-05	\\
-277.34375	4.02579469091278e-05	\\
-276.3671875	3.68002866615813e-05	\\
-275.390625	1.89152311414637e-05	\\
-274.4140625	3.65046864444371e-05	\\
-273.4375	3.91056481066595e-05	\\
-272.4609375	1.62568365461119e-05	\\
-271.484375	2.77418323276413e-05	\\
-270.5078125	3.6733888485225e-05	\\
-269.53125	3.52715528421873e-05	\\
-268.5546875	3.58766373576899e-05	\\
-267.578125	2.98263305167537e-05	\\
-266.6015625	2.46378775882196e-05	\\
-265.625	3.06053894053383e-05	\\
-264.6484375	5.39475647964369e-05	\\
-263.671875	3.42990445871609e-05	\\
-262.6953125	4.95689968686244e-05	\\
-261.71875	3.94224217499578e-05	\\
-260.7421875	2.57663382119569e-05	\\
-259.765625	2.83696049581245e-05	\\
-258.7890625	2.65680550597342e-05	\\
-257.8125	2.88403160456793e-05	\\
-256.8359375	2.6189366056346e-05	\\
-255.859375	3.55940616209542e-05	\\
-254.8828125	1.91045845050525e-05	\\
-253.90625	2.21085329762448e-05	\\
-252.9296875	3.33433312270898e-05	\\
-251.953125	2.47271296500246e-05	\\
-250.9765625	3.59133604393089e-05	\\
-250	2.88049564733143e-05	\\
-249.0234375	3.64737654520798e-05	\\
-248.046875	3.11940751222666e-05	\\
-247.0703125	3.00875917879362e-05	\\
-246.09375	3.1483236054451e-05	\\
-245.1171875	2.84582942887014e-05	\\
-244.140625	2.14978014046596e-05	\\
-243.1640625	3.90418169191301e-05	\\
-242.1875	2.58703011382134e-05	\\
-241.2109375	2.12382900646823e-05	\\
-240.234375	3.43025449651353e-05	\\
-239.2578125	5.99423530953459e-06	\\
-238.28125	2.52122591368723e-05	\\
-237.3046875	2.36431983802213e-05	\\
-236.328125	1.14775873589185e-05	\\
-235.3515625	3.65058400127941e-05	\\
-234.375	2.38748889190781e-05	\\
-233.3984375	2.98105584405381e-05	\\
-232.421875	3.25108786068957e-05	\\
-231.4453125	2.91204365699625e-05	\\
-230.46875	2.62607190865777e-05	\\
-229.4921875	3.67294031469894e-05	\\
-228.515625	3.21433973332268e-05	\\
-227.5390625	6.52251462600577e-06	\\
-226.5625	2.69490662303848e-05	\\
-225.5859375	2.7505205311947e-05	\\
-224.609375	2.34364734689974e-05	\\
-223.6328125	3.63557263071923e-05	\\
-222.65625	2.15040832466219e-05	\\
-221.6796875	2.26458374859966e-05	\\
-220.703125	1.61259462695025e-05	\\
-219.7265625	1.60636004888052e-05	\\
-218.75	1.39257738896448e-05	\\
-217.7734375	2.38465427124546e-05	\\
-216.796875	1.31892701751726e-05	\\
-215.8203125	2.67421281881492e-05	\\
-214.84375	7.99529424286936e-07	\\
-213.8671875	5.85908906890059e-06	\\
-212.890625	1.32731960011404e-05	\\
-211.9140625	1.48703861489189e-05	\\
-210.9375	1.77585321540791e-05	\\
-209.9609375	1.44534037424891e-05	\\
-208.984375	2.3036110731584e-05	\\
-208.0078125	2.21846117346874e-05	\\
-207.03125	2.92795697104554e-05	\\
-206.0546875	1.82601084626696e-05	\\
-205.078125	2.11298052864383e-05	\\
-204.1015625	2.78380826794182e-05	\\
-203.125	5.97222839442134e-06	\\
-202.1484375	2.00338754579229e-05	\\
-201.171875	2.61153841736081e-05	\\
-200.1953125	5.6468965308937e-05	\\
-199.21875	2.36699630636279e-05	\\
-198.2421875	2.41775539079791e-05	\\
-197.265625	1.93687622327982e-05	\\
-196.2890625	1.4528857866187e-05	\\
-195.3125	8.15421583467432e-06	\\
-194.3359375	2.09144783331031e-05	\\
-193.359375	5.20220906102635e-06	\\
-192.3828125	1.28351013937023e-05	\\
-191.40625	2.07337035550492e-05	\\
-190.4296875	2.77835378066735e-05	\\
-189.453125	2.45791490171402e-05	\\
-188.4765625	1.8383863576532e-05	\\
-187.5	2.00279476753637e-05	\\
-186.5234375	2.1968185071623e-05	\\
-185.546875	1.55392784256432e-05	\\
-184.5703125	1.70886020520026e-05	\\
-183.59375	1.36401012957908e-05	\\
-182.6171875	2.41414028335862e-05	\\
-181.640625	2.29931706203064e-05	\\
-180.6640625	1.16507444057475e-05	\\
-179.6875	1.68315564311292e-05	\\
-178.7109375	1.21503962810091e-05	\\
-177.734375	1.34326286466557e-05	\\
-176.7578125	8.6493813473295e-06	\\
-175.78125	7.97882152247433e-06	\\
-174.8046875	1.71057343730738e-05	\\
-173.828125	1.01023402794834e-05	\\
-172.8515625	6.68042185218841e-06	\\
-171.875	1.74624055098104e-05	\\
-170.8984375	2.05901070417998e-06	\\
-169.921875	1.95059062055932e-05	\\
-168.9453125	7.97278441838165e-06	\\
-167.96875	1.38668548578879e-05	\\
-166.9921875	1.56390746666862e-05	\\
-166.015625	7.50257643580012e-06	\\
-165.0390625	5.54785015517329e-06	\\
-164.0625	3.1750700147737e-06	\\
-163.0859375	7.33626655394422e-06	\\
-162.109375	1.61392606161334e-05	\\
-161.1328125	9.72872652498293e-06	\\
-160.15625	8.71471222099851e-06	\\
-159.1796875	9.79150987167823e-06	\\
-158.203125	6.19414258436874e-06	\\
-157.2265625	4.32462019941417e-06	\\
-156.25	1.83636367011601e-05	\\
-155.2734375	1.53863106410079e-05	\\
-154.296875	8.21930293444014e-06	\\
-153.3203125	1.2205447235974e-05	\\
-152.34375	1.46868660893641e-05	\\
-151.3671875	1.42011754557149e-05	\\
-150.390625	1.3051740013196e-05	\\
-149.4140625	1.37754123262382e-05	\\
-148.4375	8.76552123429408e-06	\\
-147.4609375	9.40183926293967e-06	\\
-146.484375	1.73004887727451e-05	\\
-145.5078125	1.25429820078806e-05	\\
-144.53125	6.76953667131177e-06	\\
-143.5546875	9.58729022810646e-06	\\
-142.578125	2.07783666559833e-05	\\
-141.6015625	9.50000687802149e-06	\\
-140.625	5.76399941390045e-06	\\
-139.6484375	9.06159972265759e-06	\\
-138.671875	7.71632580731657e-06	\\
-137.6953125	7.34241109785358e-06	\\
-136.71875	6.67213570894095e-06	\\
-135.7421875	2.78455372777121e-05	\\
-134.765625	2.43055646215856e-05	\\
-133.7890625	8.13062134549789e-06	\\
-132.8125	7.06104721171237e-06	\\
-131.8359375	2.22149209416877e-05	\\
-130.859375	1.7087924682597e-05	\\
-129.8828125	5.18293656961571e-06	\\
-128.90625	6.39521197675245e-06	\\
-127.9296875	2.10347233449978e-05	\\
-126.953125	2.20879669039866e-05	\\
-125.9765625	2.06141857576728e-05	\\
-125	3.40998620891899e-05	\\
-124.0234375	2.45479353964272e-05	\\
-123.046875	1.74555931412121e-05	\\
-122.0703125	1.59641052107367e-05	\\
-121.09375	4.71193150446854e-05	\\
-120.1171875	0.000297302548604463	\\
-119.140625	4.79937293716925e-05	\\
-118.1640625	2.9832160233152e-05	\\
-117.1875	1.88089217138076e-05	\\
-116.2109375	6.50030915169324e-06	\\
-115.234375	1.60988726580709e-05	\\
-114.2578125	3.66663614008006e-06	\\
-113.28125	2.85423440224929e-05	\\
-112.3046875	7.90536807217933e-06	\\
-111.328125	1.67162951768103e-05	\\
-110.3515625	2.52326230888302e-05	\\
-109.375	1.10326972680217e-05	\\
-108.3984375	2.5456072676504e-05	\\
-107.421875	1.31160990857005e-05	\\
-106.4453125	1.51202879238703e-05	\\
-105.46875	1.5371487194134e-05	\\
-104.4921875	3.25203600445679e-05	\\
-103.515625	5.60356585649428e-06	\\
-102.5390625	1.87515310872283e-05	\\
-101.5625	1.25070395865531e-05	\\
-100.5859375	2.08444270466054e-05	\\
-99.609375	3.19869569093075e-05	\\
-98.6328125	2.20405734780393e-05	\\
-97.65625	2.65494332098259e-05	\\
-96.6796875	2.45420155529489e-05	\\
-95.703125	2.51158997061731e-05	\\
-94.7265625	4.24324472005371e-06	\\
-93.75	1.11492251937654e-05	\\
};
\addplot [color=blue,solid,forget plot]
  table[row sep=crcr]{
-93.75	1.11492251937654e-05	\\
-92.7734375	4.58798171260305e-06	\\
-91.796875	2.02723361009577e-05	\\
-90.8203125	1.8613388359395e-05	\\
-89.84375	2.81901948021766e-05	\\
-88.8671875	1.73499729178438e-05	\\
-87.890625	4.90215472222717e-05	\\
-86.9140625	4.01782655733822e-05	\\
-85.9375	1.33484290668961e-05	\\
-84.9609375	2.07643896813888e-05	\\
-83.984375	1.39138829983463e-05	\\
-83.0078125	1.02782149295754e-05	\\
-82.03125	2.4527997263925e-05	\\
-81.0546875	2.38022321993813e-05	\\
-80.078125	2.33540461051041e-05	\\
-79.1015625	5.86611215924602e-06	\\
-78.125	2.28135283481691e-05	\\
-77.1484375	1.72545519795242e-05	\\
-76.171875	1.17965193779205e-05	\\
-75.1953125	1.24531084216402e-05	\\
-74.21875	2.07037921010836e-05	\\
-73.2421875	1.13663458204602e-05	\\
-72.265625	1.08638803907974e-05	\\
-71.2890625	1.07742582613996e-05	\\
-70.3125	8.77559215728116e-06	\\
-69.3359375	2.7088599753468e-05	\\
-68.359375	7.84300091022284e-06	\\
-67.3828125	2.01283702059581e-05	\\
-66.40625	2.34462618975387e-05	\\
-65.4296875	1.99735337918613e-05	\\
-64.453125	1.79055670926034e-05	\\
-63.4765625	1.46130133949538e-05	\\
-62.5	2.12785427975892e-05	\\
-61.5234375	4.177634625733e-06	\\
-60.546875	2.48776409398009e-05	\\
-59.5703125	2.21023938061524e-05	\\
-58.59375	5.00961471822974e-06	\\
-57.6171875	1.08921848285067e-05	\\
-56.640625	3.82151914271136e-05	\\
-55.6640625	3.50252202394273e-05	\\
-54.6875	3.38388398397861e-05	\\
-53.7109375	2.67364308380467e-05	\\
-52.734375	1.27564301851124e-05	\\
-51.7578125	4.29653089471246e-05	\\
-50.78125	8.30137725021766e-05	\\
-49.8046875	0.00023899427942632	\\
-48.828125	2.17108306235569e-05	\\
-47.8515625	1.57486183550273e-05	\\
-46.875	3.1293338428603e-05	\\
-45.8984375	7.49632363423823e-05	\\
-44.921875	0.000401143946259114	\\
-43.9453125	7.37727811490267e-05	\\
-42.96875	5.6591895221285e-05	\\
-41.9921875	5.27009355789155e-05	\\
-41.015625	2.31032528905053e-05	\\
-40.0390625	3.36276351100634e-05	\\
-39.0625	0.000111875564169349	\\
-38.0859375	2.97535015120493e-05	\\
-37.109375	1.19983371927651e-05	\\
-36.1328125	7.35734866491224e-05	\\
-35.15625	2.58360187614826e-06	\\
-34.1796875	2.98124521970379e-05	\\
-33.203125	4.7131625289951e-05	\\
-32.2265625	1.24245259564049e-05	\\
-31.25	2.38265832642402e-05	\\
-30.2734375	3.35489440655118e-05	\\
-29.296875	2.76967777427884e-05	\\
-28.3203125	4.17588013033734e-05	\\
-27.34375	2.3398085014362e-05	\\
-26.3671875	6.96332583880767e-05	\\
-25.390625	2.75826639267708e-05	\\
-24.4140625	0.000107943339362843	\\
-23.4375	4.52596998650867e-05	\\
-22.4609375	5.81493523243024e-05	\\
-21.484375	5.18151537761058e-05	\\
-20.5078125	3.36973800637359e-05	\\
-19.53125	1.31998227711918e-05	\\
-18.5546875	6.4615909739858e-05	\\
-17.578125	5.81645778397861e-05	\\
-16.6015625	9.0741380983127e-05	\\
-15.625	5.7305008957124e-05	\\
-14.6484375	0.000101159639884315	\\
-13.671875	0.000374089243759698	\\
-12.6953125	0.000242887233337364	\\
-11.71875	7.14828391975088e-05	\\
-10.7421875	5.66404592408324e-05	\\
-9.765625	7.63252833379431e-05	\\
-8.7890625	7.07755990141284e-05	\\
-7.8125	0.00036528688825495	\\
-6.8359375	0.000157394738110839	\\
-5.859375	0.000219091125477408	\\
-4.8828125	0.000307365359750025	\\
-3.90625	0.000136563342802336	\\
-2.9296875	0.000299038548982331	\\
-1.953125	0.000173269057876063	\\
-0.9765625	0.000124109428175657	\\
0	9.93534922599792e-06	\\
0.9765625	0.000124109428175657	\\
1.953125	0.000173269057876063	\\
2.9296875	0.000299038548982331	\\
3.90625	0.000136563342802336	\\
4.8828125	0.000307365359750025	\\
5.859375	0.000219091125477408	\\
6.8359375	0.000157394738110839	\\
7.8125	0.00036528688825495	\\
8.7890625	7.07755990141284e-05	\\
9.765625	7.63252833379431e-05	\\
10.7421875	5.66404592408324e-05	\\
11.71875	7.14828391975088e-05	\\
12.6953125	0.000242887233337364	\\
13.671875	0.000374089243759698	\\
14.6484375	0.000101159639884315	\\
15.625	5.7305008957124e-05	\\
16.6015625	9.0741380983127e-05	\\
17.578125	5.81645778397861e-05	\\
18.5546875	6.4615909739858e-05	\\
19.53125	1.31998227711918e-05	\\
20.5078125	3.36973800637359e-05	\\
21.484375	5.18151537761058e-05	\\
22.4609375	5.81493523243024e-05	\\
23.4375	4.52596998650867e-05	\\
24.4140625	0.000107943339362843	\\
25.390625	2.75826639267708e-05	\\
26.3671875	6.96332583880767e-05	\\
27.34375	2.3398085014362e-05	\\
28.3203125	4.17588013033734e-05	\\
29.296875	2.76967777427884e-05	\\
30.2734375	3.35489440655118e-05	\\
31.25	2.38265832642402e-05	\\
32.2265625	1.24245259564049e-05	\\
33.203125	4.7131625289951e-05	\\
34.1796875	2.98124521970379e-05	\\
35.15625	2.58360187614826e-06	\\
36.1328125	7.35734866491224e-05	\\
37.109375	1.19983371927651e-05	\\
38.0859375	2.97535015120493e-05	\\
39.0625	0.000111875564169349	\\
40.0390625	3.36276351100634e-05	\\
41.015625	2.31032528905053e-05	\\
41.9921875	5.27009355789155e-05	\\
42.96875	5.6591895221285e-05	\\
43.9453125	7.37727811490267e-05	\\
44.921875	0.000401143946259114	\\
45.8984375	7.49632363423823e-05	\\
46.875	3.1293338428603e-05	\\
47.8515625	1.57486183550273e-05	\\
48.828125	2.17108306235569e-05	\\
49.8046875	0.00023899427942632	\\
50.78125	8.30137725021766e-05	\\
51.7578125	4.29653089471246e-05	\\
52.734375	1.27564301851124e-05	\\
53.7109375	2.67364308380467e-05	\\
54.6875	3.38388398397861e-05	\\
55.6640625	3.50252202394273e-05	\\
56.640625	3.82151914271136e-05	\\
57.6171875	1.08921848285067e-05	\\
58.59375	5.00961471822974e-06	\\
59.5703125	2.21023938061524e-05	\\
60.546875	2.48776409398009e-05	\\
61.5234375	4.177634625733e-06	\\
62.5	2.12785427975892e-05	\\
63.4765625	1.46130133949538e-05	\\
64.453125	1.79055670926034e-05	\\
65.4296875	1.99735337918613e-05	\\
66.40625	2.34462618975387e-05	\\
67.3828125	2.01283702059581e-05	\\
68.359375	7.84300091022284e-06	\\
69.3359375	2.7088599753468e-05	\\
70.3125	8.77559215728116e-06	\\
71.2890625	1.07742582613996e-05	\\
72.265625	1.08638803907974e-05	\\
73.2421875	1.13663458204602e-05	\\
74.21875	2.07037921010836e-05	\\
75.1953125	1.24531084216402e-05	\\
76.171875	1.17965193779205e-05	\\
77.1484375	1.72545519795242e-05	\\
78.125	2.28135283481691e-05	\\
79.1015625	5.86611215924602e-06	\\
80.078125	2.33540461051041e-05	\\
81.0546875	2.38022321993813e-05	\\
82.03125	2.4527997263925e-05	\\
83.0078125	1.02782149295754e-05	\\
83.984375	1.39138829983463e-05	\\
84.9609375	2.07643896813888e-05	\\
85.9375	1.33484290668961e-05	\\
86.9140625	4.01782655733822e-05	\\
87.890625	4.90215472222717e-05	\\
88.8671875	1.73499729178438e-05	\\
89.84375	2.81901948021766e-05	\\
90.8203125	1.8613388359395e-05	\\
91.796875	2.02723361009577e-05	\\
92.7734375	4.58798171260305e-06	\\
93.75	1.11492251937654e-05	\\
94.7265625	4.24324472005371e-06	\\
95.703125	2.51158997061731e-05	\\
96.6796875	2.45420155529489e-05	\\
97.65625	2.65494332098259e-05	\\
98.6328125	2.20405734780393e-05	\\
99.609375	3.19869569093075e-05	\\
100.5859375	2.08444270466054e-05	\\
101.5625	1.25070395865531e-05	\\
102.5390625	1.87515310872283e-05	\\
103.515625	5.60356585649428e-06	\\
104.4921875	3.25203600445679e-05	\\
105.46875	1.5371487194134e-05	\\
106.4453125	1.51202879238703e-05	\\
107.421875	1.31160990857005e-05	\\
108.3984375	2.5456072676504e-05	\\
109.375	1.10326972680217e-05	\\
110.3515625	2.52326230888302e-05	\\
111.328125	1.67162951768103e-05	\\
112.3046875	7.90536807217933e-06	\\
113.28125	2.85423440224929e-05	\\
114.2578125	3.66663614008006e-06	\\
115.234375	1.60988726580709e-05	\\
116.2109375	6.50030915169324e-06	\\
117.1875	1.88089217138076e-05	\\
118.1640625	2.9832160233152e-05	\\
119.140625	4.79937293716925e-05	\\
120.1171875	0.000297302548604463	\\
121.09375	4.71193150446854e-05	\\
122.0703125	1.59641052107367e-05	\\
123.046875	1.74555931412121e-05	\\
124.0234375	2.45479353964272e-05	\\
125	3.40998620891899e-05	\\
125.9765625	2.06141857576728e-05	\\
126.953125	2.20879669039866e-05	\\
127.9296875	2.10347233449978e-05	\\
128.90625	6.39521197675245e-06	\\
129.8828125	5.18293656961571e-06	\\
130.859375	1.7087924682597e-05	\\
131.8359375	2.22149209416877e-05	\\
132.8125	7.06104721171237e-06	\\
133.7890625	8.13062134549789e-06	\\
134.765625	2.43055646215856e-05	\\
135.7421875	2.78455372777121e-05	\\
136.71875	6.67213570894095e-06	\\
137.6953125	7.34241109785358e-06	\\
138.671875	7.71632580731657e-06	\\
139.6484375	9.06159972265759e-06	\\
140.625	5.76399941390045e-06	\\
141.6015625	9.50000687802149e-06	\\
142.578125	2.07783666559833e-05	\\
143.5546875	9.58729022810646e-06	\\
144.53125	6.76953667131177e-06	\\
145.5078125	1.25429820078806e-05	\\
146.484375	1.73004887727451e-05	\\
147.4609375	9.40183926293967e-06	\\
148.4375	8.76552123429408e-06	\\
149.4140625	1.37754123262382e-05	\\
150.390625	1.3051740013196e-05	\\
151.3671875	1.42011754557149e-05	\\
152.34375	1.46868660893641e-05	\\
153.3203125	1.2205447235974e-05	\\
154.296875	8.21930293444014e-06	\\
155.2734375	1.53863106410079e-05	\\
156.25	1.83636367011601e-05	\\
157.2265625	4.32462019941417e-06	\\
158.203125	6.19414258436874e-06	\\
159.1796875	9.79150987167823e-06	\\
160.15625	8.71471222099851e-06	\\
161.1328125	9.72872652498293e-06	\\
162.109375	1.61392606161334e-05	\\
163.0859375	7.33626655394422e-06	\\
164.0625	3.1750700147737e-06	\\
165.0390625	5.54785015517329e-06	\\
166.015625	7.50257643580012e-06	\\
166.9921875	1.56390746666862e-05	\\
167.96875	1.38668548578879e-05	\\
168.9453125	7.97278441838165e-06	\\
169.921875	1.95059062055932e-05	\\
170.8984375	2.05901070417998e-06	\\
171.875	1.74624055098104e-05	\\
172.8515625	6.68042185218841e-06	\\
173.828125	1.01023402794834e-05	\\
174.8046875	1.71057343730738e-05	\\
175.78125	7.97882152247433e-06	\\
176.7578125	8.6493813473295e-06	\\
177.734375	1.34326286466557e-05	\\
178.7109375	1.21503962810091e-05	\\
179.6875	1.68315564311292e-05	\\
180.6640625	1.16507444057475e-05	\\
181.640625	2.29931706203064e-05	\\
182.6171875	2.41414028335862e-05	\\
183.59375	1.36401012957908e-05	\\
184.5703125	1.70886020520026e-05	\\
185.546875	1.55392784256432e-05	\\
186.5234375	2.1968185071623e-05	\\
187.5	2.00279476753637e-05	\\
188.4765625	1.8383863576532e-05	\\
189.453125	2.45791490171402e-05	\\
190.4296875	2.77835378066735e-05	\\
191.40625	2.07337035550492e-05	\\
192.3828125	1.28351013937023e-05	\\
193.359375	5.20220906102635e-06	\\
194.3359375	2.09144783331031e-05	\\
195.3125	8.15421583467432e-06	\\
196.2890625	1.4528857866187e-05	\\
197.265625	1.93687622327982e-05	\\
198.2421875	2.41775539079791e-05	\\
199.21875	2.36699630636279e-05	\\
200.1953125	5.6468965308937e-05	\\
201.171875	2.61153841736081e-05	\\
202.1484375	2.00338754579229e-05	\\
203.125	5.97222839442134e-06	\\
204.1015625	2.78380826794182e-05	\\
205.078125	2.11298052864383e-05	\\
206.0546875	1.82601084626696e-05	\\
207.03125	2.92795697104554e-05	\\
208.0078125	2.21846117346874e-05	\\
208.984375	2.3036110731584e-05	\\
209.9609375	1.44534037424891e-05	\\
210.9375	1.77585321540791e-05	\\
211.9140625	1.48703861489189e-05	\\
212.890625	1.32731960011404e-05	\\
213.8671875	5.85908906890059e-06	\\
214.84375	7.99529424286936e-07	\\
215.8203125	2.67421281881492e-05	\\
216.796875	1.31892701751726e-05	\\
217.7734375	2.38465427124546e-05	\\
218.75	1.39257738896448e-05	\\
219.7265625	1.60636004888052e-05	\\
220.703125	1.61259462695025e-05	\\
221.6796875	2.26458374859966e-05	\\
222.65625	2.15040832466219e-05	\\
223.6328125	3.63557263071923e-05	\\
224.609375	2.34364734689974e-05	\\
225.5859375	2.7505205311947e-05	\\
226.5625	2.69490662303848e-05	\\
227.5390625	6.52251462600577e-06	\\
228.515625	3.21433973332268e-05	\\
229.4921875	3.67294031469894e-05	\\
230.46875	2.62607190865777e-05	\\
231.4453125	2.91204365699625e-05	\\
232.421875	3.25108786068957e-05	\\
233.3984375	2.98105584405381e-05	\\
234.375	2.38748889190781e-05	\\
235.3515625	3.65058400127941e-05	\\
236.328125	1.14775873589185e-05	\\
237.3046875	2.36431983802213e-05	\\
238.28125	2.52122591368723e-05	\\
239.2578125	5.99423530953459e-06	\\
240.234375	3.43025449651353e-05	\\
241.2109375	2.12382900646823e-05	\\
242.1875	2.58703011382134e-05	\\
243.1640625	3.90418169191301e-05	\\
244.140625	2.14978014046596e-05	\\
245.1171875	2.84582942887014e-05	\\
246.09375	3.1483236054451e-05	\\
247.0703125	3.00875917879362e-05	\\
248.046875	3.11940751222666e-05	\\
249.0234375	3.64737654520798e-05	\\
250	2.88049564733143e-05	\\
250.9765625	3.59133604393089e-05	\\
251.953125	2.47271296500246e-05	\\
252.9296875	3.33433312270898e-05	\\
253.90625	2.21085329762448e-05	\\
254.8828125	1.91045845050525e-05	\\
255.859375	3.55940616209542e-05	\\
256.8359375	2.6189366056346e-05	\\
257.8125	2.88403160456793e-05	\\
258.7890625	2.65680550597342e-05	\\
259.765625	2.83696049581245e-05	\\
260.7421875	2.57663382119569e-05	\\
261.71875	3.94224217499578e-05	\\
262.6953125	4.95689968686244e-05	\\
263.671875	3.42990445871609e-05	\\
264.6484375	5.39475647964369e-05	\\
265.625	3.06053894053383e-05	\\
266.6015625	2.46378775882196e-05	\\
267.578125	2.98263305167537e-05	\\
268.5546875	3.58766373576899e-05	\\
269.53125	3.52715528421873e-05	\\
270.5078125	3.6733888485225e-05	\\
271.484375	2.77418323276413e-05	\\
272.4609375	1.62568365461119e-05	\\
273.4375	3.91056481066595e-05	\\
274.4140625	3.65046864444371e-05	\\
275.390625	1.89152311414637e-05	\\
276.3671875	3.68002866615813e-05	\\
277.34375	4.02579469091278e-05	\\
278.3203125	3.94788502610931e-05	\\
279.296875	3.10147510027345e-05	\\
280.2734375	3.02228153559455e-05	\\
281.25	4.39655412077424e-05	\\
282.2265625	2.36661423317752e-05	\\
283.203125	4.07013279249842e-05	\\
284.1796875	2.64029855516337e-05	\\
285.15625	3.65586881577751e-05	\\
286.1328125	3.21478171494118e-05	\\
287.109375	2.96181638963043e-05	\\
288.0859375	2.60241237979022e-05	\\
289.0625	2.99505746955742e-05	\\
290.0390625	3.38118035215084e-05	\\
291.015625	2.511724860973e-05	\\
291.9921875	2.99622867145511e-05	\\
292.96875	2.81261648045329e-05	\\
293.9453125	3.77209419308307e-05	\\
294.921875	3.40408976080305e-05	\\
295.8984375	2.9821736910079e-05	\\
296.875	2.5916503017615e-05	\\
297.8515625	3.13763477767729e-05	\\
298.828125	3.83320319236552e-05	\\
299.8046875	9.96168891967017e-06	\\
300.78125	2.58367803913633e-05	\\
301.7578125	3.12084821778605e-05	\\
302.734375	2.67343927100232e-05	\\
303.7109375	2.27223654370756e-05	\\
304.6875	1.47012242417948e-05	\\
305.6640625	2.23979656970141e-05	\\
306.640625	2.59208076659082e-05	\\
307.6171875	3.06940753637191e-05	\\
308.59375	2.10296015638265e-05	\\
309.5703125	2.65631674253829e-05	\\
310.546875	3.26788240862855e-05	\\
311.5234375	2.95419696381749e-05	\\
312.5	2.38451376816011e-05	\\
313.4765625	3.56658239121104e-05	\\
314.453125	3.124290469137e-05	\\
315.4296875	2.80301696489167e-05	\\
316.40625	2.67965313689949e-05	\\
317.3828125	3.55647923948945e-05	\\
318.359375	3.39880481433425e-05	\\
319.3359375	3.3888024964724e-05	\\
320.3125	2.83679098327555e-05	\\
321.2890625	3.01636546459971e-05	\\
322.265625	2.35404247708284e-05	\\
323.2421875	2.09339505615728e-05	\\
324.21875	2.46123298331049e-05	\\
325.1953125	3.64860006623852e-05	\\
326.171875	1.6155927323561e-05	\\
327.1484375	1.64967993616328e-05	\\
328.125	1.92620111499445e-05	\\
329.1015625	2.07885993332942e-05	\\
330.078125	1.13583985141433e-05	\\
331.0546875	1.71201249729755e-05	\\
332.03125	1.85646003916347e-05	\\
333.0078125	3.04359131188758e-05	\\
333.984375	1.58830418905999e-05	\\
334.9609375	2.92145857788575e-05	\\
335.9375	2.5052996458113e-05	\\
336.9140625	2.70445524879045e-05	\\
337.890625	2.61917387596942e-05	\\
338.8671875	3.48555351560344e-05	\\
339.84375	2.88210913179573e-05	\\
340.8203125	3.63217315873103e-05	\\
341.796875	3.43506388422346e-05	\\
342.7734375	3.6205280880896e-05	\\
343.75	3.09085871749736e-05	\\
344.7265625	4.19873246621054e-05	\\
345.703125	3.02709119707949e-05	\\
346.6796875	2.68226289267262e-05	\\
347.65625	4.07966012389524e-05	\\
348.6328125	3.42759694266904e-05	\\
349.609375	4.03547142520708e-05	\\
350.5859375	3.63511294339827e-05	\\
351.5625	3.1304849998552e-05	\\
352.5390625	3.43586649652947e-05	\\
353.515625	4.02551683406993e-05	\\
354.4921875	4.08264790123115e-05	\\
355.46875	3.25228774541245e-05	\\
356.4453125	4.59543417507938e-05	\\
357.421875	3.89615292229264e-05	\\
358.3984375	3.37665668356188e-05	\\
359.375	3.06401451910024e-05	\\
360.3515625	2.01946071211084e-05	\\
361.328125	3.04843575845669e-05	\\
362.3046875	5.09889312211145e-05	\\
363.28125	5.03773648532554e-05	\\
364.2578125	4.60577772546571e-05	\\
365.234375	4.23017400096584e-05	\\
366.2109375	4.82408783021087e-05	\\
367.1875	4.80564071904314e-05	\\
368.1640625	4.29666108340014e-05	\\
369.140625	3.67147942836088e-05	\\
370.1171875	3.39073558319184e-05	\\
371.09375	4.02461282273771e-05	\\
372.0703125	4.65541917758266e-05	\\
373.046875	5.38492707420389e-05	\\
374.0234375	4.58666637442162e-05	\\
375	4.72732652419262e-05	\\
375.9765625	5.12515418693312e-05	\\
376.953125	4.88046484267339e-05	\\
377.9296875	5.29404943017949e-05	\\
378.90625	5.9083352130038e-05	\\
379.8828125	5.5285507866308e-05	\\
380.859375	4.98596344304154e-05	\\
381.8359375	5.17994285106407e-05	\\
382.8125	5.80730591677586e-05	\\
383.7890625	7.17673437008384e-05	\\
384.765625	5.00770739311747e-05	\\
385.7421875	5.0840900487861e-05	\\
386.71875	7.24624363164503e-05	\\
387.6953125	7.12546911671136e-05	\\
388.671875	4.85367105351906e-05	\\
389.6484375	6.01442715901982e-05	\\
390.625	6.02730472175325e-05	\\
391.6015625	6.45471261387108e-05	\\
392.578125	5.68598266062425e-05	\\
393.5546875	5.69559977656469e-05	\\
394.53125	5.74182208741402e-05	\\
395.5078125	6.58947664965902e-05	\\
396.484375	7.01471481214424e-05	\\
397.4609375	5.80707349539247e-05	\\
398.4375	7.23304683667219e-05	\\
399.4140625	7.15746665385294e-05	\\
400.390625	9.6501289643104e-05	\\
401.3671875	7.20393326530275e-05	\\
402.34375	8.66593614073627e-05	\\
403.3203125	8.10697326349083e-05	\\
404.296875	8.55452694138191e-05	\\
405.2734375	8.47750135555258e-05	\\
406.25	8.7109489777046e-05	\\
407.2265625	9.31598089872681e-05	\\
408.203125	9.31275673168977e-05	\\
409.1796875	9.09645702885232e-05	\\
410.15625	8.46262668867822e-05	\\
411.1328125	9.22543940366173e-05	\\
412.109375	8.9097312266362e-05	\\
413.0859375	8.50978921439517e-05	\\
414.0625	0.000104354521174701	\\
415.0390625	9.355745709413e-05	\\
416.015625	6.78213935199465e-05	\\
416.9921875	9.9663635139712e-05	\\
417.96875	8.46956131708631e-05	\\
418.9453125	7.16138995094108e-05	\\
419.921875	8.54691140314757e-05	\\
420.8984375	0.000106270507366382	\\
421.875	7.31035043062355e-05	\\
422.8515625	9.63687830571634e-05	\\
423.828125	0.000103084589296464	\\
424.8046875	9.59771620749605e-05	\\
425.78125	8.93725126665433e-05	\\
426.7578125	0.000109761742914639	\\
427.734375	0.000108937238058265	\\
428.7109375	0.000100096297572913	\\
429.6875	0.000120259524239122	\\
430.6640625	0.000109620728674336	\\
431.640625	9.71519865995132e-05	\\
432.6171875	0.000101696713851658	\\
433.59375	9.28646853985357e-05	\\
434.5703125	9.21738462961723e-05	\\
435.546875	9.80445697787286e-05	\\
436.5234375	0.000110111197935794	\\
437.5	0.000106526020508991	\\
438.4765625	0.000117616811796288	\\
439.453125	0.000110283763748698	\\
440.4296875	0.000105117423357357	\\
441.40625	0.000132168174243133	\\
442.3828125	0.000141732226142642	\\
443.359375	0.000125161048604526	\\
444.3359375	0.0001223023284652	\\
445.3125	0.000137067788430962	\\
446.2890625	0.000126524328790327	\\
447.265625	0.000139925976060546	\\
448.2421875	0.000134037165649966	\\
449.21875	0.00013869322985632	\\
450.1953125	0.000129926087326326	\\
451.171875	0.000131120527749436	\\
452.1484375	0.000124253733447229	\\
453.125	0.000126702205542897	\\
454.1015625	0.000117661738853567	\\
455.078125	0.000139901218005887	\\
456.0546875	0.000136780186315355	\\
457.03125	0.000131446152189194	\\
458.0078125	0.000145729872419009	\\
458.984375	0.00014333209882431	\\
459.9609375	0.000147536249673683	\\
460.9375	0.000158686922118638	\\
461.9140625	0.000162339972381576	\\
462.890625	0.000161628409612299	\\
463.8671875	0.000169343033556901	\\
464.84375	0.000157876551048696	\\
465.8203125	0.000172900430350052	\\
466.796875	0.000172836656438737	\\
467.7734375	0.000180806885256685	\\
468.75	0.000183933940097566	\\
469.7265625	0.000171187892027408	\\
470.703125	0.000170785729169803	\\
471.6796875	0.000183625295047535	\\
472.65625	0.000168910340252127	\\
473.6328125	0.000177717380114034	\\
474.609375	0.000181008555591869	\\
475.5859375	0.000172228166874569	\\
476.5625	0.000169908153015317	\\
477.5390625	0.000176118692535445	\\
478.515625	0.00017538047305446	\\
479.4921875	0.000178818021470093	\\
480.46875	0.000182037397571893	\\
481.4453125	0.00017632576029744	\\
482.421875	0.000176465878248154	\\
483.3984375	0.000181151251544531	\\
484.375	0.000174709628666874	\\
485.3515625	0.000178992647852694	\\
486.328125	0.000171533492922521	\\
487.3046875	0.000172512757765062	\\
488.28125	0.000177921244829601	\\
489.2578125	0.000167342517433179	\\
490.234375	0.000158769809150336	\\
491.2109375	0.000173391158748671	\\
492.1875	0.000165668418961052	\\
493.1640625	0.000161584305374303	\\
494.140625	0.000165766175294255	\\
495.1171875	0.000167512140484853	\\
496.09375	0.000145975194289466	\\
497.0703125	0.000153435965371445	\\
498.046875	0.000169657574597628	\\
499.0234375	0.000130786323153755	\\
500	0.000192510697231583	\\
500.9765625	0.000142858955822049	\\
501.953125	0.000169493245343257	\\
502.9296875	0.000147368561282073	\\
503.90625	0.000149255325881289	\\
504.8828125	0.000148510493015892	\\
505.859375	0.000149652754070551	\\
506.8359375	0.000142988824612314	\\
507.8125	0.000145489050840206	\\
508.7890625	0.000140655735575329	\\
509.765625	0.000138369275934094	\\
510.7421875	0.000146661812202005	\\
511.71875	0.000139383286192448	\\
512.6953125	0.0001314732660576	\\
513.671875	0.000144368608803723	\\
514.6484375	0.000147283499490093	\\
515.625	0.000125280055601252	\\
516.6015625	0.000134473212255969	\\
517.578125	0.000145901440623652	\\
518.5546875	0.000124177761445872	\\
519.53125	0.000121269526467468	\\
520.5078125	0.000131610430229945	\\
521.484375	0.000122627541240661	\\
522.4609375	0.000122297043100894	\\
523.4375	0.000121274567577714	\\
524.4140625	0.000135168789011112	\\
525.390625	0.000116360406982228	\\
526.3671875	0.000124868527084752	\\
527.34375	0.000133186811241018	\\
528.3203125	0.000127700770282869	\\
529.296875	0.00012491047805622	\\
530.2734375	0.000134606409823943	\\
531.25	0.000132150919763452	\\
532.2265625	0.00011844391713097	\\
533.203125	0.000117955672824248	\\
534.1796875	0.000125876402597088	\\
535.15625	0.000125937498285834	\\
536.1328125	0.000120674602279962	\\
537.109375	0.000125298128129049	\\
538.0859375	0.000124610513303601	\\
539.0625	0.000121727360833229	\\
540.0390625	0.00011919738249584	\\
541.015625	0.000125733511019605	\\
541.9921875	0.00012805205270312	\\
542.96875	0.000124206198323686	\\
543.9453125	0.000123622639848901	\\
544.921875	0.000122580019733621	\\
545.8984375	0.000127617376229444	\\
546.875	0.000119951981259982	\\
547.8515625	0.000121400324832612	\\
548.828125	0.000127397784887523	\\
549.8046875	0.000128615360979393	\\
550.78125	0.000123144980123309	\\
551.7578125	0.000132877611535138	\\
552.734375	0.000130223908852405	\\
553.7109375	0.000131329380780105	\\
554.6875	0.000142669801201252	\\
555.6640625	0.000133499932340804	\\
556.640625	0.000134149311064473	\\
557.6171875	0.000132470326040561	\\
558.59375	0.00013691702649593	\\
559.5703125	0.000134494541021666	\\
560.546875	0.000125505069855698	\\
561.5234375	0.000116610084518873	\\
562.5	0.000120447750243848	\\
563.4765625	0.000122234428365286	\\
564.453125	0.000122999336630295	\\
565.4296875	0.000118487279799562	\\
566.40625	0.000127822730701422	\\
567.3828125	0.000119584072180596	\\
568.359375	0.00012143240440063	\\
569.3359375	0.000124650174447538	\\
570.3125	0.000126987686316666	\\
571.2890625	0.000126095660876482	\\
572.265625	0.000138033248088782	\\
573.2421875	0.000132026746252888	\\
574.21875	0.000128870331024554	\\
575.1953125	0.000132373188180436	\\
576.171875	0.000135140639201665	\\
577.1484375	0.000130603183172555	\\
578.125	0.000125216820418991	\\
579.1015625	0.000121885691497669	\\
580.078125	0.00012649714235931	\\
581.0546875	0.000115859576717374	\\
582.03125	0.00011895006469227	\\
583.0078125	0.000118518180705962	\\
583.984375	0.000115180943622521	\\
584.9609375	0.000109619146735627	\\
585.9375	0.000120362046163597	\\
586.9140625	0.000110647939363798	\\
587.890625	0.00010938720471658	\\
588.8671875	0.000105031988533047	\\
589.84375	0.00011084443922736	\\
590.8203125	0.000107529421813304	\\
591.796875	0.00010616959219433	\\
592.7734375	0.000104718976018817	\\
593.75	0.000108328472127119	\\
594.7265625	0.000110970731437298	\\
595.703125	0.000110618477338483	\\
596.6796875	0.000114256142876867	\\
597.65625	0.000117020998015368	\\
598.6328125	0.000109168925297951	\\
599.609375	9.26203610437632e-05	\\
600.5859375	0.000102442844749196	\\
601.5625	0.000104444869073848	\\
602.5390625	0.000100413853025696	\\
603.515625	9.5832158195198e-05	\\
604.4921875	0.000110113800265706	\\
605.46875	0.00010045993757962	\\
606.4453125	9.239529407623e-05	\\
607.421875	9.73109824610167e-05	\\
608.3984375	0.000103938225123689	\\
609.375	9.99813709023521e-05	\\
610.3515625	9.80665564381802e-05	\\
611.328125	0.000103712880035242	\\
612.3046875	0.00010426949202809	\\
613.28125	9.58064991214552e-05	\\
614.2578125	0.000101691329327943	\\
615.234375	0.000103837849579806	\\
616.2109375	9.23189397003675e-05	\\
617.1875	0.000100764147184561	\\
618.1640625	9.57003969015275e-05	\\
619.140625	0.000101343062025395	\\
620.1171875	0.000100798448448989	\\
621.09375	9.70707222585133e-05	\\
622.0703125	9.41406360221423e-05	\\
623.046875	8.89677687670182e-05	\\
624.0234375	8.49278882813787e-05	\\
625	9.11495648049478e-05	\\
625.9765625	8.88473235794753e-05	\\
626.953125	8.79502571825146e-05	\\
627.9296875	9.32285632055811e-05	\\
628.90625	8.8195335246493e-05	\\
629.8828125	9.2381143419937e-05	\\
630.859375	9.49181633558847e-05	\\
631.8359375	9.07386771772197e-05	\\
632.8125	8.96329109969644e-05	\\
633.7890625	8.62177239415606e-05	\\
634.765625	9.59486666239664e-05	\\
635.7421875	8.79114860769018e-05	\\
636.71875	8.94174283712485e-05	\\
637.6953125	9.31450314862085e-05	\\
638.671875	8.95521816982996e-05	\\
639.6484375	8.98212393905302e-05	\\
640.625	9.00891700266774e-05	\\
641.6015625	9.7410458999655e-05	\\
642.578125	8.73496839842843e-05	\\
643.5546875	9.00757322003627e-05	\\
644.53125	9.72269710009691e-05	\\
645.5078125	9.0033249271762e-05	\\
646.484375	9.41957749023373e-05	\\
647.4609375	9.04575593921075e-05	\\
648.4375	9.86922423891199e-05	\\
649.4140625	8.45304298956547e-05	\\
650.390625	8.78408147565718e-05	\\
651.3671875	8.68871803784865e-05	\\
652.34375	8.40099947856051e-05	\\
653.3203125	8.7820028559817e-05	\\
654.296875	8.26208283266832e-05	\\
655.2734375	8.01256414230914e-05	\\
656.25	8.51679657393198e-05	\\
657.2265625	8.88045551848492e-05	\\
658.203125	7.7561242526877e-05	\\
659.1796875	8.01670291900142e-05	\\
660.15625	8.24165249736848e-05	\\
661.1328125	7.55048003514787e-05	\\
662.109375	7.47548734531583e-05	\\
663.0859375	8.15729496250901e-05	\\
664.0625	7.91934886708126e-05	\\
665.0390625	8.03957122985936e-05	\\
666.015625	8.1467284629172e-05	\\
666.9921875	7.72989059773217e-05	\\
667.96875	7.61167835089823e-05	\\
668.9453125	8.51547851670088e-05	\\
669.921875	7.99819738935617e-05	\\
670.8984375	7.78320096518232e-05	\\
671.875	8.14112416406041e-05	\\
672.8515625	7.99112205819859e-05	\\
673.828125	7.67244836071644e-05	\\
674.8046875	7.4765208030725e-05	\\
675.78125	7.48001240955915e-05	\\
676.7578125	7.5705935919925e-05	\\
677.734375	7.60356172276647e-05	\\
678.7109375	8.50346260081138e-05	\\
679.6875	7.5310443699612e-05	\\
680.6640625	7.72306604552676e-05	\\
681.640625	7.83439450109525e-05	\\
682.6171875	8.18492059440863e-05	\\
683.59375	7.88246615696383e-05	\\
684.5703125	7.64484088619277e-05	\\
685.546875	7.85557236100321e-05	\\
686.5234375	7.31458342588154e-05	\\
687.5	7.26276615192441e-05	\\
688.4765625	7.6360986914415e-05	\\
689.453125	7.71483804000178e-05	\\
690.4296875	8.31857455407127e-05	\\
691.40625	7.83695154667629e-05	\\
692.3828125	7.7345653044079e-05	\\
693.359375	7.56710654246821e-05	\\
694.3359375	7.737924877368e-05	\\
695.3125	8.12500309730263e-05	\\
696.2890625	7.50391297920927e-05	\\
697.265625	7.6890280246582e-05	\\
698.2421875	8.7007958196228e-05	\\
699.21875	7.99499974417047e-05	\\
700.1953125	9.36338004692343e-05	\\
701.171875	8.1761362278978e-05	\\
702.1484375	7.93318974701037e-05	\\
703.125	7.88250600496612e-05	\\
704.1015625	8.28366928465563e-05	\\
705.078125	8.10458588470221e-05	\\
706.0546875	7.82364994256376e-05	\\
707.03125	8.44108725927891e-05	\\
708.0078125	8.73230286142442e-05	\\
708.984375	8.30483703765321e-05	\\
709.9609375	8.67041765676348e-05	\\
710.9375	8.46276730643292e-05	\\
711.9140625	8.66868773278306e-05	\\
712.890625	8.3548846840709e-05	\\
713.8671875	7.86313297540235e-05	\\
714.84375	8.48300253425636e-05	\\
715.8203125	7.93785174410866e-05	\\
716.796875	8.12844728662888e-05	\\
717.7734375	8.49628820795954e-05	\\
718.75	8.42047667012381e-05	\\
719.7265625	8.83633580903672e-05	\\
720.703125	7.9642359133523e-05	\\
721.6796875	8.66878140509249e-05	\\
722.65625	8.42244139693051e-05	\\
723.6328125	8.39038828762613e-05	\\
724.609375	8.77243727557097e-05	\\
725.5859375	8.11572632848969e-05	\\
726.5625	8.48775379215344e-05	\\
727.5390625	8.58737745774042e-05	\\
728.515625	8.20529876824677e-05	\\
729.4921875	8.25625443913624e-05	\\
730.46875	8.45284543655367e-05	\\
731.4453125	8.33930487737855e-05	\\
732.421875	7.96132473204573e-05	\\
733.3984375	8.10763269565648e-05	\\
734.375	8.71333132609698e-05	\\
735.3515625	8.89315053218026e-05	\\
736.328125	8.43845342831024e-05	\\
737.3046875	7.37035511152562e-05	\\
738.28125	8.26406275227646e-05	\\
739.2578125	8.24826070188931e-05	\\
740.234375	8.54232329621048e-05	\\
741.2109375	8.0516178238286e-05	\\
742.1875	8.37216987109264e-05	\\
743.1640625	7.95356613176687e-05	\\
744.140625	7.95372750406881e-05	\\
745.1171875	8.13018254551484e-05	\\
746.09375	8.08958761303337e-05	\\
747.0703125	8.22826741837437e-05	\\
748.046875	7.70731774500044e-05	\\
749.0234375	7.6336230577654e-05	\\
750	7.65371557530473e-05	\\
750.9765625	7.81384195566519e-05	\\
751.953125	7.31789682289427e-05	\\
752.9296875	7.6700077696191e-05	\\
753.90625	7.76713370401765e-05	\\
754.8828125	8.58060819290417e-05	\\
755.859375	7.89823894407827e-05	\\
756.8359375	8.38448621800098e-05	\\
757.8125	8.02125578781262e-05	\\
758.7890625	8.22607641913667e-05	\\
759.765625	7.86677599350962e-05	\\
760.7421875	7.94401928560239e-05	\\
761.71875	7.44641962619934e-05	\\
762.6953125	8.22668455185644e-05	\\
763.671875	7.60058937890154e-05	\\
764.6484375	8.24059163793652e-05	\\
765.625	8.07173474088082e-05	\\
766.6015625	7.51846577648262e-05	\\
767.578125	7.51388581553248e-05	\\
768.5546875	7.85915320627036e-05	\\
769.53125	7.53669681287147e-05	\\
770.5078125	7.95313795708572e-05	\\
771.484375	6.81241422301854e-05	\\
772.4609375	7.81313724581926e-05	\\
773.4375	7.79129871049322e-05	\\
774.4140625	7.80888948922121e-05	\\
775.390625	8.04835426445661e-05	\\
776.3671875	7.683769907192e-05	\\
777.34375	7.75261470687972e-05	\\
778.3203125	7.68903111244779e-05	\\
779.296875	7.58856176340719e-05	\\
780.2734375	7.86179036919471e-05	\\
781.25	7.50269954176608e-05	\\
782.2265625	8.01147675551712e-05	\\
783.203125	7.52539695521641e-05	\\
784.1796875	6.798550366956e-05	\\
785.15625	7.38749702277021e-05	\\
786.1328125	6.76557018593731e-05	\\
787.109375	7.27694329394658e-05	\\
788.0859375	7.60040315252318e-05	\\
789.0625	7.54428639871377e-05	\\
790.0390625	7.19363034915858e-05	\\
791.015625	6.86648110643859e-05	\\
791.9921875	7.16883153276089e-05	\\
792.96875	7.00602533369677e-05	\\
793.9453125	7.13659324493596e-05	\\
794.921875	7.12436295208083e-05	\\
795.8984375	6.85699175874229e-05	\\
796.875	7.53795962843891e-05	\\
797.8515625	6.62788122404719e-05	\\
798.828125	7.04093223606068e-05	\\
799.8046875	6.84681542078107e-05	\\
800.78125	6.67355441040047e-05	\\
801.7578125	7.55821035083963e-05	\\
802.734375	7.4035724258426e-05	\\
803.7109375	6.81460468187831e-05	\\
804.6875	7.25828525795934e-05	\\
805.6640625	6.60557929277833e-05	\\
806.640625	6.86164268614558e-05	\\
807.6171875	7.1656501718164e-05	\\
808.59375	6.60098923583405e-05	\\
809.5703125	7.18771643957909e-05	\\
810.546875	6.67344848947082e-05	\\
811.5234375	7.14677424923977e-05	\\
812.5	7.13289582454411e-05	\\
813.4765625	7.01368057188717e-05	\\
814.453125	7.16280524976806e-05	\\
815.4296875	7.42514074319071e-05	\\
816.40625	6.78187870466832e-05	\\
817.3828125	6.7878693068829e-05	\\
818.359375	7.04844774435743e-05	\\
819.3359375	7.30231819063131e-05	\\
820.3125	6.82694192000847e-05	\\
821.2890625	7.05118254748193e-05	\\
822.265625	6.75042842579013e-05	\\
823.2421875	6.90707092707262e-05	\\
824.21875	7.04700845177272e-05	\\
825.1953125	7.14753051467898e-05	\\
826.171875	7.1691012210365e-05	\\
827.1484375	7.42968853825066e-05	\\
828.125	6.60957146828018e-05	\\
829.1015625	6.75769662204421e-05	\\
830.078125	6.50045023384774e-05	\\
831.0546875	6.77106107153587e-05	\\
832.03125	6.75624644408072e-05	\\
833.0078125	6.90041941877119e-05	\\
833.984375	6.39651627796424e-05	\\
834.9609375	6.66908816543824e-05	\\
835.9375	7.08447452311935e-05	\\
836.9140625	6.96239730628935e-05	\\
837.890625	7.10770757767057e-05	\\
838.8671875	6.74612194454303e-05	\\
839.84375	6.46505293529778e-05	\\
840.8203125	6.80999737554835e-05	\\
841.796875	6.54654817103887e-05	\\
842.7734375	6.73652541141961e-05	\\
843.75	7.07488549669287e-05	\\
844.7265625	6.99058977963653e-05	\\
845.703125	6.86805189109188e-05	\\
846.6796875	6.71601031645592e-05	\\
847.65625	6.75466526102453e-05	\\
848.6328125	6.96773365249183e-05	\\
849.609375	7.11026642318626e-05	\\
850.5859375	6.89993398703786e-05	\\
851.5625	6.62103455799304e-05	\\
852.5390625	6.45332923678176e-05	\\
853.515625	6.82202760111455e-05	\\
854.4921875	7.04881608561013e-05	\\
855.46875	6.63275267983291e-05	\\
856.4453125	6.72396445359345e-05	\\
857.421875	6.86744855440357e-05	\\
858.3984375	7.12919463103486e-05	\\
859.375	6.81478160875427e-05	\\
860.3515625	6.82305950305343e-05	\\
861.328125	6.96557578751355e-05	\\
862.3046875	6.91786327602385e-05	\\
863.28125	6.87891772840725e-05	\\
864.2578125	7.08505463544328e-05	\\
865.234375	6.71480890338329e-05	\\
866.2109375	6.52626963946067e-05	\\
867.1875	6.77043097438515e-05	\\
868.1640625	7.21611448718813e-05	\\
869.140625	6.80045463252837e-05	\\
870.1171875	7.20486218282481e-05	\\
871.09375	7.341082811468e-05	\\
872.0703125	6.74708799729535e-05	\\
873.046875	6.99197044172686e-05	\\
874.0234375	6.49205391500694e-05	\\
875	6.44399343643494e-05	\\
875.9765625	6.25747441128518e-05	\\
876.953125	6.50111643546536e-05	\\
877.9296875	6.70578338138251e-05	\\
878.90625	6.99656403416864e-05	\\
879.8828125	6.97225934567152e-05	\\
880.859375	6.97430059507621e-05	\\
881.8359375	7.23513530837827e-05	\\
882.8125	6.77591260829007e-05	\\
883.7890625	7.10565229504782e-05	\\
884.765625	6.88636124087122e-05	\\
885.7421875	6.94685264665923e-05	\\
886.71875	6.79822696026146e-05	\\
887.6953125	7.23618149774335e-05	\\
888.671875	6.9695125708277e-05	\\
889.6484375	7.04721183923458e-05	\\
890.625	6.6878849237599e-05	\\
891.6015625	6.74416339710723e-05	\\
892.578125	7.26958410178915e-05	\\
893.5546875	7.01773458111854e-05	\\
894.53125	7.36875158630846e-05	\\
895.5078125	7.42153383850339e-05	\\
896.484375	6.6927796023877e-05	\\
897.4609375	7.19687364363074e-05	\\
898.4375	7.67067712867883e-05	\\
899.4140625	7.11948253014517e-05	\\
900.390625	8.95440330767631e-05	\\
901.3671875	6.75856426857327e-05	\\
902.34375	7.53297849785126e-05	\\
903.3203125	7.03594389473756e-05	\\
904.296875	6.7982730985679e-05	\\
905.2734375	7.10531221535746e-05	\\
906.25	6.79909240671382e-05	\\
907.2265625	6.64579064677853e-05	\\
908.203125	7.09834687958036e-05	\\
909.1796875	6.96569686911364e-05	\\
910.15625	6.91691344835967e-05	\\
911.1328125	7.11866304506146e-05	\\
912.109375	7.02520855197225e-05	\\
913.0859375	6.63264125000402e-05	\\
914.0625	7.14108671057831e-05	\\
915.0390625	7.11939792337511e-05	\\
916.015625	7.33347460437006e-05	\\
916.9921875	6.71394157389019e-05	\\
917.96875	6.79536680527615e-05	\\
918.9453125	6.79621297948662e-05	\\
919.921875	6.46775935811846e-05	\\
920.8984375	6.89090059379345e-05	\\
921.875	6.7598559338298e-05	\\
922.8515625	6.67976576833254e-05	\\
923.828125	6.61127654243313e-05	\\
924.8046875	6.77252921353276e-05	\\
925.78125	6.90330057128334e-05	\\
926.7578125	6.50239301327616e-05	\\
927.734375	6.98726776000354e-05	\\
928.7109375	6.99467784791981e-05	\\
929.6875	6.64645463300953e-05	\\
930.6640625	6.83536266571696e-05	\\
931.640625	6.69106672583777e-05	\\
932.6171875	7.05448371272833e-05	\\
933.59375	6.50775776152462e-05	\\
934.5703125	6.97272914768325e-05	\\
935.546875	6.35891296327325e-05	\\
936.5234375	6.53916931384321e-05	\\
937.5	6.62623157955611e-05	\\
938.4765625	6.54528370687628e-05	\\
939.453125	6.68759698222319e-05	\\
940.4296875	6.77272739404746e-05	\\
941.40625	6.90176274263389e-05	\\
942.3828125	6.51205244368735e-05	\\
943.359375	6.53818106963872e-05	\\
944.3359375	7.0876657097409e-05	\\
945.3125	7.08785524107703e-05	\\
946.2890625	6.77294351228078e-05	\\
947.265625	6.97417652678032e-05	\\
948.2421875	6.51311620397466e-05	\\
949.21875	6.70202917437575e-05	\\
950.1953125	7.01502843126279e-05	\\
951.171875	6.56074936668609e-05	\\
952.1484375	6.36855311300202e-05	\\
953.125	6.53690908900551e-05	\\
954.1015625	6.55120159502242e-05	\\
955.078125	6.43579595035376e-05	\\
956.0546875	6.1672003779588e-05	\\
957.03125	6.65714297509414e-05	\\
958.0078125	6.67773991712554e-05	\\
958.984375	6.64468654485878e-05	\\
959.9609375	6.47013801775955e-05	\\
960.9375	6.34252879146914e-05	\\
961.9140625	6.71813311073519e-05	\\
962.890625	6.14994555154238e-05	\\
963.8671875	6.88282645447363e-05	\\
964.84375	6.59696586405282e-05	\\
965.8203125	6.16400583426501e-05	\\
966.796875	6.82825197036614e-05	\\
967.7734375	6.73629391424961e-05	\\
968.75	6.76707718051758e-05	\\
969.7265625	7.29196057781077e-05	\\
970.703125	6.72927464134353e-05	\\
971.6796875	6.48458065856741e-05	\\
972.65625	6.6867195444038e-05	\\
973.6328125	6.67900577223603e-05	\\
974.609375	6.38256981096458e-05	\\
975.5859375	6.13100431963276e-05	\\
976.5625	7.08042038874808e-05	\\
977.5390625	6.40127038505934e-05	\\
978.515625	6.69660538792972e-05	\\
979.4921875	6.60663289076446e-05	\\
980.46875	7.20867103066934e-05	\\
981.4453125	7.19277093035157e-05	\\
982.421875	6.73665443450159e-05	\\
983.3984375	6.84734769447627e-05	\\
984.375	6.88198215306538e-05	\\
985.3515625	7.12366275491977e-05	\\
986.328125	6.64523197304349e-05	\\
987.3046875	6.47178106392196e-05	\\
988.28125	7.04243378609903e-05	\\
989.2578125	6.74739824787806e-05	\\
990.234375	6.51458030383616e-05	\\
991.2109375	6.3632400780754e-05	\\
992.1875	6.26607654570264e-05	\\
993.1640625	6.56516763657906e-05	\\
994.140625	6.42860557335978e-05	\\
995.1171875	6.45262731978916e-05	\\
996.09375	6.56652743013789e-05	\\
997.0703125	6.09079881185355e-05	\\
998.046875	6.16304403435153e-05	\\
999.0234375	6.6853721700997e-05	\\
1000	5.47588424150285e-05	\\
1000.9765625	6.18446188300364e-05	\\
1001.953125	6.28788949259345e-05	\\
1002.9296875	6.44363872569815e-05	\\
1003.90625	6.23964798482685e-05	\\
1004.8828125	6.20533977090868e-05	\\
1005.859375	6.47690231986976e-05	\\
1006.8359375	6.06504993278994e-05	\\
1007.8125	5.97954639704782e-05	\\
1008.7890625	6.30276913338236e-05	\\
1009.765625	6.24224504810131e-05	\\
1010.7421875	5.96715584085966e-05	\\
1011.71875	5.52986291703605e-05	\\
1012.6953125	5.69024837059391e-05	\\
1013.671875	5.93018074675025e-05	\\
1014.6484375	5.51267004148029e-05	\\
1015.625	6.04270082282083e-05	\\
1016.6015625	5.78217440508319e-05	\\
1017.578125	5.96909061133426e-05	\\
1018.5546875	5.63474670887001e-05	\\
1019.53125	5.85415381332116e-05	\\
1020.5078125	5.64056989752272e-05	\\
1021.484375	6.27209113338744e-05	\\
1022.4609375	6.3879619833322e-05	\\
1023.4375	6.35577867212347e-05	\\
1024.4140625	6.24614975416455e-05	\\
1025.390625	6.23979966529052e-05	\\
1026.3671875	5.99991677993778e-05	\\
1027.34375	6.42530816589476e-05	\\
1028.3203125	6.00011102101786e-05	\\
1029.296875	6.65519289824898e-05	\\
1030.2734375	6.16857578922188e-05	\\
1031.25	6.44534061967717e-05	\\
1032.2265625	6.01996969062148e-05	\\
1033.203125	6.16065000681489e-05	\\
1034.1796875	5.94008221411568e-05	\\
1035.15625	5.65641821030007e-05	\\
1036.1328125	5.91238012500008e-05	\\
1037.109375	6.04527228841909e-05	\\
1038.0859375	6.33090844600278e-05	\\
1039.0625	6.2323645074751e-05	\\
1040.0390625	6.45492203966491e-05	\\
1041.015625	6.51556088876854e-05	\\
1041.9921875	6.59692208656773e-05	\\
1042.96875	5.84419125370752e-05	\\
1043.9453125	6.12999893081955e-05	\\
1044.921875	6.52198178971121e-05	\\
1045.8984375	6.20625238744242e-05	\\
1046.875	6.51538395812037e-05	\\
1047.8515625	6.41869044078451e-05	\\
1048.828125	6.02337845798586e-05	\\
1049.8046875	6.11526533013724e-05	\\
1050.78125	6.15230151793205e-05	\\
1051.7578125	6.17189332520848e-05	\\
1052.734375	6.24716713617839e-05	\\
1053.7109375	6.31940735388848e-05	\\
1054.6875	6.26317713419106e-05	\\
1055.6640625	6.56509905458114e-05	\\
1056.640625	6.32140457668692e-05	\\
1057.6171875	6.34236424619112e-05	\\
1058.59375	6.86738197869905e-05	\\
1059.5703125	6.6029313877598e-05	\\
1060.546875	6.72915975262106e-05	\\
1061.5234375	6.74112185546912e-05	\\
1062.5	6.8433771676788e-05	\\
1063.4765625	7.01683221874523e-05	\\
1064.453125	6.75026081066393e-05	\\
1065.4296875	6.93085576872288e-05	\\
1066.40625	6.85365337238876e-05	\\
1067.3828125	7.04890409267202e-05	\\
1068.359375	6.96149440466915e-05	\\
1069.3359375	6.8543963371752e-05	\\
1070.3125	7.17433502010239e-05	\\
1071.2890625	6.71727304299139e-05	\\
1072.265625	6.6649555419675e-05	\\
1073.2421875	6.9622504876607e-05	\\
1074.21875	6.48234675613764e-05	\\
1075.1953125	7.06855431675193e-05	\\
1076.171875	7.09214376430789e-05	\\
1077.1484375	6.83463797232351e-05	\\
1078.125	7.36731282692959e-05	\\
1079.1015625	6.98283429156833e-05	\\
1080.078125	7.54731667599213e-05	\\
1081.0546875	6.95169051551765e-05	\\
1082.03125	7.08591025262047e-05	\\
1083.0078125	7.21392288580908e-05	\\
1083.984375	7.29996857909591e-05	\\
1084.9609375	7.01098677291221e-05	\\
1085.9375	6.66836431808738e-05	\\
1086.9140625	6.25746153950905e-05	\\
1087.890625	7.24392655123061e-05	\\
1088.8671875	6.39181351958004e-05	\\
1089.84375	6.60497069703568e-05	\\
1090.8203125	6.85930691760539e-05	\\
1091.796875	6.74632218338153e-05	\\
1092.7734375	6.81633474588662e-05	\\
1093.75	6.80596703651228e-05	\\
1094.7265625	6.99991801015524e-05	\\
1095.703125	6.93095712577861e-05	\\
1096.6796875	6.90674315041414e-05	\\
1097.65625	7.03112901341189e-05	\\
1098.6328125	7.08195888959784e-05	\\
1099.609375	7.65627778069728e-05	\\
1100.5859375	7.01258929973044e-05	\\
1101.5625	6.60752786336721e-05	\\
1102.5390625	6.4059607512419e-05	\\
1103.515625	6.78712215223263e-05	\\
1104.4921875	6.63748971005009e-05	\\
1105.46875	6.59698456179929e-05	\\
1106.4453125	6.30008991567295e-05	\\
1107.421875	6.78626423665579e-05	\\
1108.3984375	6.33566821855007e-05	\\
1109.375	6.59678728311037e-05	\\
1110.3515625	6.41198457653324e-05	\\
1111.328125	6.77106183613459e-05	\\
1112.3046875	6.62759519491931e-05	\\
1113.28125	6.58627775954683e-05	\\
1114.2578125	6.62266877608305e-05	\\
1115.234375	6.32587136090187e-05	\\
1116.2109375	6.30348972039558e-05	\\
1117.1875	6.35649021187401e-05	\\
1118.1640625	6.48976798603169e-05	\\
1119.140625	6.47526066464658e-05	\\
1120.1171875	6.35452356210098e-05	\\
1121.09375	6.64449609828207e-05	\\
1122.0703125	6.22053339975359e-05	\\
1123.046875	5.98242488928936e-05	\\
1124.0234375	6.07207449710612e-05	\\
1125	6.08476233816929e-05	\\
1125.9765625	5.97806809246344e-05	\\
1126.953125	5.97065083488191e-05	\\
1127.9296875	6.65451933584572e-05	\\
1128.90625	6.12560217095037e-05	\\
1129.8828125	6.23516163700128e-05	\\
1130.859375	6.27278035992702e-05	\\
1131.8359375	6.03486386837281e-05	\\
1132.8125	6.54795086878079e-05	\\
1133.7890625	6.55622317200848e-05	\\
1134.765625	6.54404634691754e-05	\\
1135.7421875	6.36487649178359e-05	\\
1136.71875	6.39121551318685e-05	\\
1137.6953125	6.91898908447183e-05	\\
1138.671875	6.3799193634555e-05	\\
1139.6484375	6.22062587312902e-05	\\
1140.625	6.18091398639258e-05	\\
1141.6015625	5.98106752960987e-05	\\
1142.578125	6.13309032126654e-05	\\
1143.5546875	6.04773953480733e-05	\\
1144.53125	5.82185150020489e-05	\\
1145.5078125	6.45515479279676e-05	\\
1146.484375	5.99776867723449e-05	\\
1147.4609375	6.3409427003136e-05	\\
1148.4375	6.59623675412944e-05	\\
1149.4140625	6.2001205097458e-05	\\
1150.390625	6.49662181617894e-05	\\
1151.3671875	6.3604669106545e-05	\\
1152.34375	6.26378774614573e-05	\\
1153.3203125	6.71656983258182e-05	\\
1154.296875	6.84609579350961e-05	\\
1155.2734375	6.36435264511356e-05	\\
1156.25	7.09971319770534e-05	\\
1157.2265625	6.60874518526667e-05	\\
1158.203125	6.70520111665015e-05	\\
1159.1796875	6.77030461193735e-05	\\
1160.15625	6.51992275674871e-05	\\
1161.1328125	6.79867818681255e-05	\\
1162.109375	6.44804364049596e-05	\\
1163.0859375	6.7207305044579e-05	\\
1164.0625	6.60441147987493e-05	\\
1165.0390625	6.45323019818727e-05	\\
1166.015625	6.48872280607072e-05	\\
1166.9921875	6.84954944928531e-05	\\
1167.96875	6.73738161310164e-05	\\
1168.9453125	6.48136541657187e-05	\\
1169.921875	6.86382580537582e-05	\\
1170.8984375	7.08172443268601e-05	\\
1171.875	6.99138205280508e-05	\\
1172.8515625	6.91472957196403e-05	\\
1173.828125	6.82201177215753e-05	\\
1174.8046875	7.04393400209118e-05	\\
1175.78125	6.95565091844482e-05	\\
1176.7578125	6.8838693059097e-05	\\
1177.734375	7.30107720998988e-05	\\
1178.7109375	6.41462815767788e-05	\\
1179.6875	7.28206953458705e-05	\\
1180.6640625	6.88772959004037e-05	\\
1181.640625	6.94711630729583e-05	\\
1182.6171875	6.29099897837659e-05	\\
1183.59375	7.0605775275637e-05	\\
1184.5703125	6.36322015530884e-05	\\
1185.546875	6.57271179973018e-05	\\
1186.5234375	6.34939945538749e-05	\\
1187.5	6.59496683879167e-05	\\
1188.4765625	6.64161832398737e-05	\\
1189.453125	6.604257124235e-05	\\
1190.4296875	6.7033004454306e-05	\\
1191.40625	6.37777926206248e-05	\\
1192.3828125	6.30069145705447e-05	\\
1193.359375	6.49440261903384e-05	\\
1194.3359375	6.88713724117622e-05	\\
1195.3125	6.54137769844863e-05	\\
1196.2890625	7.27762968111862e-05	\\
1197.265625	6.96758866621834e-05	\\
1198.2421875	6.63766692846154e-05	\\
1199.21875	6.79652464204832e-05	\\
1200.1953125	7.42308504139894e-05	\\
1201.171875	6.65489815899671e-05	\\
1202.1484375	7.19399533634852e-05	\\
1203.125	6.6953879953473e-05	\\
1204.1015625	6.76936609197741e-05	\\
1205.078125	6.56610516190351e-05	\\
1206.0546875	6.64651129450838e-05	\\
1207.03125	6.89219591262328e-05	\\
1208.0078125	6.57113902809526e-05	\\
1208.984375	6.71495481916372e-05	\\
1209.9609375	6.78967118300692e-05	\\
1210.9375	6.55732360050218e-05	\\
1211.9140625	6.59707875699144e-05	\\
1212.890625	6.82155205193998e-05	\\
1213.8671875	6.76664854876272e-05	\\
1214.84375	6.74153165381071e-05	\\
1215.8203125	6.83899261319679e-05	\\
1216.796875	6.32194579853002e-05	\\
1217.7734375	6.51675043175535e-05	\\
1218.75	6.68729692302448e-05	\\
1219.7265625	6.28736112215579e-05	\\
1220.703125	6.26405944943494e-05	\\
1221.6796875	6.6048105659517e-05	\\
1222.65625	6.26877687892543e-05	\\
1223.6328125	6.67342076049939e-05	\\
1224.609375	6.19015183896371e-05	\\
1225.5859375	6.47126633076921e-05	\\
1226.5625	6.76135261068323e-05	\\
1227.5390625	6.37460010383492e-05	\\
1228.515625	6.70255145000834e-05	\\
1229.4921875	6.5962919423388e-05	\\
1230.46875	6.79582561878122e-05	\\
1231.4453125	6.90592660542894e-05	\\
1232.421875	6.65050338514957e-05	\\
1233.3984375	6.85173701680674e-05	\\
1234.375	6.56195042086053e-05	\\
1235.3515625	6.65267791165447e-05	\\
1236.328125	6.97099061747371e-05	\\
1237.3046875	6.64326386309136e-05	\\
1238.28125	6.78691751347379e-05	\\
1239.2578125	6.80429209995823e-05	\\
1240.234375	6.52467472513109e-05	\\
1241.2109375	6.74499086296456e-05	\\
1242.1875	6.73683136311457e-05	\\
1243.1640625	6.49616837693749e-05	\\
1244.140625	6.713632650417e-05	\\
1245.1171875	6.88035718643392e-05	\\
1246.09375	6.82071264570812e-05	\\
1247.0703125	6.44357239103808e-05	\\
1248.046875	6.61474632870342e-05	\\
1249.0234375	6.71623312723465e-05	\\
1250	6.64781413784053e-05	\\
1250.9765625	7.14735488685434e-05	\\
1251.953125	6.4615329308258e-05	\\
1252.9296875	6.90473038104651e-05	\\
1253.90625	7.16329439334931e-05	\\
1254.8828125	7.16037449558508e-05	\\
1255.859375	6.97480829400829e-05	\\
1256.8359375	6.91060976508093e-05	\\
1257.8125	6.75260480464562e-05	\\
1258.7890625	6.88019113145735e-05	\\
1259.765625	6.96851007004959e-05	\\
1260.7421875	7.04542321423532e-05	\\
1261.71875	7.1722538534353e-05	\\
1262.6953125	6.83281005963804e-05	\\
1263.671875	6.77270227697897e-05	\\
1264.6484375	7.0039569335557e-05	\\
1265.625	6.6186611650999e-05	\\
1266.6015625	6.98968758095581e-05	\\
1267.578125	6.41488811460787e-05	\\
1268.5546875	7.09307080201771e-05	\\
1269.53125	6.97567601071068e-05	\\
1270.5078125	6.96965191510233e-05	\\
1271.484375	7.12070659873822e-05	\\
1272.4609375	6.6457786195029e-05	\\
1273.4375	6.84887458169861e-05	\\
1274.4140625	6.87451429344529e-05	\\
1275.390625	6.63777887413362e-05	\\
1276.3671875	6.80109972501767e-05	\\
1277.34375	7.05351347253344e-05	\\
1278.3203125	6.86514827682979e-05	\\
1279.296875	7.04905560221999e-05	\\
1280.2734375	7.01567046355731e-05	\\
1281.25	6.88521755974399e-05	\\
1282.2265625	6.92492644128387e-05	\\
1283.203125	6.63390143437494e-05	\\
1284.1796875	6.98577422777947e-05	\\
1285.15625	6.84274053370781e-05	\\
1286.1328125	6.86101961544285e-05	\\
1287.109375	6.62864245363469e-05	\\
1288.0859375	6.96775345843038e-05	\\
1289.0625	6.72771174631451e-05	\\
1290.0390625	7.05337747666699e-05	\\
1291.015625	7.31601352065427e-05	\\
1291.9921875	6.84226952075541e-05	\\
1292.96875	6.98217782254179e-05	\\
1293.9453125	6.63365364188209e-05	\\
1294.921875	6.92252941549353e-05	\\
1295.8984375	6.63202170074085e-05	\\
1296.875	6.59150821089271e-05	\\
1297.8515625	6.6570180010978e-05	\\
1298.828125	6.6006809172491e-05	\\
1299.8046875	6.7527196804861e-05	\\
1300.78125	6.59410084293307e-05	\\
1301.7578125	6.55875422488536e-05	\\
1302.734375	6.85718298498584e-05	\\
1303.7109375	6.17733545976853e-05	\\
1304.6875	7.00528158767789e-05	\\
1305.6640625	6.76318044371796e-05	\\
1306.640625	6.82256876689282e-05	\\
1307.6171875	6.6908017684161e-05	\\
1308.59375	6.57605482973842e-05	\\
1309.5703125	6.87629446122803e-05	\\
1310.546875	7.05855404789899e-05	\\
1311.5234375	6.87667701266491e-05	\\
1312.5	6.81824359357388e-05	\\
1313.4765625	6.43555949296255e-05	\\
1314.453125	6.58778809542842e-05	\\
1315.4296875	6.86757569705217e-05	\\
1316.40625	7.06334286651524e-05	\\
1317.3828125	6.92518726735722e-05	\\
1318.359375	7.12991097950039e-05	\\
1319.3359375	6.59705492036517e-05	\\
1320.3125	7.06180293908499e-05	\\
1321.2890625	7.05583748846293e-05	\\
1322.265625	6.7815766345406e-05	\\
1323.2421875	6.5419004578165e-05	\\
1324.21875	6.60685662262626e-05	\\
1325.1953125	6.44068227357736e-05	\\
1326.171875	6.8963408850741e-05	\\
1327.1484375	7.01559983222341e-05	\\
1328.125	6.59242845534599e-05	\\
1329.1015625	7.25150820050057e-05	\\
1330.078125	6.88350892377932e-05	\\
1331.0546875	7.01776702479196e-05	\\
1332.03125	6.95051102213233e-05	\\
1333.0078125	7.38891589467595e-05	\\
1333.984375	7.02538460269656e-05	\\
1334.9609375	6.81475738212234e-05	\\
1335.9375	7.06878654942634e-05	\\
1336.9140625	6.78832545478905e-05	\\
1337.890625	6.81581119217842e-05	\\
1338.8671875	6.67101475115517e-05	\\
1339.84375	6.58490950915478e-05	\\
1340.8203125	6.79781807181848e-05	\\
1341.796875	7.04635890886438e-05	\\
1342.7734375	6.64605673123184e-05	\\
1343.75	6.87095211717615e-05	\\
1344.7265625	6.95029445016341e-05	\\
1345.703125	6.82342076985609e-05	\\
1346.6796875	6.87887206203051e-05	\\
1347.65625	6.56561648024974e-05	\\
1348.6328125	6.55669505806988e-05	\\
1349.609375	6.83771511398777e-05	\\
1350.5859375	6.57344929635655e-05	\\
1351.5625	6.70928063656963e-05	\\
1352.5390625	6.83999720263898e-05	\\
1353.515625	6.93228486956672e-05	\\
1354.4921875	6.7517268705636e-05	\\
1355.46875	6.71260324968591e-05	\\
1356.4453125	7.02312253215074e-05	\\
1357.421875	6.56137043874455e-05	\\
1358.3984375	6.69894646863576e-05	\\
1359.375	6.78018455954732e-05	\\
1360.3515625	6.99253843239398e-05	\\
1361.328125	6.87311436460885e-05	\\
1362.3046875	6.45820091799972e-05	\\
1363.28125	6.5837244280497e-05	\\
1364.2578125	6.56118999252529e-05	\\
1365.234375	6.68939075334501e-05	\\
1366.2109375	6.53870339252002e-05	\\
1367.1875	6.42498776184753e-05	\\
1368.1640625	6.96060710171195e-05	\\
1369.140625	6.7994968786104e-05	\\
1370.1171875	6.65686397315042e-05	\\
1371.09375	6.48473465792706e-05	\\
1372.0703125	6.7579899371913e-05	\\
1373.046875	6.91481304062786e-05	\\
1374.0234375	6.36950855753925e-05	\\
1375	6.5667556539795e-05	\\
1375.9765625	6.45247064835191e-05	\\
1376.953125	6.38873765946827e-05	\\
1377.9296875	6.53127430331414e-05	\\
1378.90625	6.56836147566982e-05	\\
1379.8828125	6.33948982886355e-05	\\
1380.859375	6.45582455046432e-05	\\
1381.8359375	6.36928466503588e-05	\\
1382.8125	6.34824400696928e-05	\\
1383.7890625	6.05908813661378e-05	\\
1384.765625	6.23280059793866e-05	\\
1385.7421875	6.2034245526216e-05	\\
1386.71875	6.50488176927496e-05	\\
1387.6953125	6.55805637857821e-05	\\
1388.671875	6.33423797266171e-05	\\
1389.6484375	6.65228820519071e-05	\\
1390.625	6.51352568314393e-05	\\
1391.6015625	6.60401903603118e-05	\\
1392.578125	6.51744274446965e-05	\\
1393.5546875	6.26756450671761e-05	\\
1394.53125	6.07035262685473e-05	\\
1395.5078125	6.40218780220938e-05	\\
1396.484375	6.4731026641993e-05	\\
1397.4609375	6.00965126022514e-05	\\
1398.4375	6.14385207817319e-05	\\
1399.4140625	6.57242743387865e-05	\\
1400.390625	5.86533076096447e-05	\\
1401.3671875	6.32722317913688e-05	\\
1402.34375	6.24714571057972e-05	\\
1403.3203125	6.32665628347305e-05	\\
1404.296875	6.47010329382833e-05	\\
1405.2734375	6.48455583483387e-05	\\
1406.25	6.33364536491678e-05	\\
1407.2265625	6.60753801848849e-05	\\
1408.203125	6.67581552368028e-05	\\
1409.1796875	6.45347584130627e-05	\\
1410.15625	6.41096911058174e-05	\\
1411.1328125	6.70242702712006e-05	\\
1412.109375	6.97743061258551e-05	\\
1413.0859375	6.36630173253289e-05	\\
1414.0625	6.41806124294688e-05	\\
1415.0390625	6.23822152062584e-05	\\
1416.015625	6.70452570090251e-05	\\
1416.9921875	6.51378892035433e-05	\\
1417.96875	6.793849845525e-05	\\
1418.9453125	6.48440951795333e-05	\\
1419.921875	6.89820209088286e-05	\\
1420.8984375	6.37513118797598e-05	\\
1421.875	6.6657924493826e-05	\\
1422.8515625	6.74426502229268e-05	\\
1423.828125	6.92919935657025e-05	\\
1424.8046875	7.0560223672396e-05	\\
1425.78125	6.931369376744e-05	\\
1426.7578125	7.37805109906538e-05	\\
1427.734375	6.87954645505367e-05	\\
1428.7109375	6.93791424275273e-05	\\
1429.6875	7.36643164476299e-05	\\
1430.6640625	7.34514971213749e-05	\\
1431.640625	7.34469982724399e-05	\\
1432.6171875	7.08974107262045e-05	\\
1433.59375	7.53736544305388e-05	\\
1434.5703125	7.48859123450399e-05	\\
1435.546875	6.81122960545356e-05	\\
1436.5234375	7.34443940638404e-05	\\
1437.5	7.2370207257806e-05	\\
1438.4765625	6.83158130983193e-05	\\
1439.453125	6.74240917041058e-05	\\
1440.4296875	7.17902597811606e-05	\\
1441.40625	7.29144472534691e-05	\\
1442.3828125	7.00971529146556e-05	\\
1443.359375	7.16145666644951e-05	\\
1444.3359375	7.26578014743472e-05	\\
1445.3125	7.39544554641411e-05	\\
1446.2890625	7.83502034951454e-05	\\
1447.265625	7.5335852510673e-05	\\
1448.2421875	7.34005578772127e-05	\\
1449.21875	7.57725497916923e-05	\\
1450.1953125	7.43864778421112e-05	\\
1451.171875	7.60965684185346e-05	\\
1452.1484375	7.8695128385833e-05	\\
1453.125	7.82970854409316e-05	\\
1454.1015625	7.61068162973353e-05	\\
1455.078125	7.48777968638504e-05	\\
1456.0546875	7.84758563696597e-05	\\
1457.03125	7.87990163221553e-05	\\
1458.0078125	7.758551769781e-05	\\
1458.984375	7.74209870460302e-05	\\
1459.9609375	7.90921511709386e-05	\\
1460.9375	8.2655835433553e-05	\\
1461.9140625	8.02604913007952e-05	\\
1462.890625	8.33365044977349e-05	\\
1463.8671875	8.21977530522015e-05	\\
1464.84375	8.21370898500511e-05	\\
1465.8203125	8.43586461320699e-05	\\
1466.796875	8.16542070892097e-05	\\
1467.7734375	8.64028066362553e-05	\\
1468.75	8.71072706243921e-05	\\
1469.7265625	8.65374625712493e-05	\\
1470.703125	8.85354606418346e-05	\\
1471.6796875	8.65962355815777e-05	\\
1472.65625	8.93770585448036e-05	\\
1473.6328125	9.2161113449453e-05	\\
1474.609375	9.09881983366604e-05	\\
1475.5859375	9.17343973631241e-05	\\
1476.5625	9.05335368009879e-05	\\
1477.5390625	9.00627895163684e-05	\\
1478.515625	9.45221295972288e-05	\\
1479.4921875	9.27439271883458e-05	\\
1480.46875	9.55912550090849e-05	\\
1481.4453125	9.27151310194985e-05	\\
1482.421875	9.78379167922272e-05	\\
1483.3984375	9.73001485166558e-05	\\
1484.375	9.46831152700439e-05	\\
1485.3515625	9.71003912789966e-05	\\
1486.328125	9.48244533104882e-05	\\
1487.3046875	9.79938804524116e-05	\\
1488.28125	9.58002678754308e-05	\\
1489.2578125	0.000101142840460086	\\
1490.234375	9.71239773555574e-05	\\
1491.2109375	0.00010394380615476	\\
1492.1875	9.80250915764393e-05	\\
1493.1640625	9.99152833233974e-05	\\
1494.140625	0.000102265389254422	\\
1495.1171875	9.99357234033906e-05	\\
1496.09375	0.000101475762655641	\\
1497.0703125	0.000106976472875356	\\
1498.046875	0.000102085872813701	\\
1499.0234375	0.000106195506485573	\\
1500	0.000116559051498441	\\
1500.9765625	0.000108337131700521	\\
1501.953125	0.000111034613601343	\\
1502.9296875	0.000111946211549208	\\
1503.90625	0.000107343683791008	\\
1504.8828125	0.000107231440783852	\\
1505.859375	0.000106759308005145	\\
1506.8359375	0.000107560655433684	\\
1507.8125	0.000109555711913548	\\
1508.7890625	0.000112471381649754	\\
1509.765625	0.000110733519748504	\\
1510.7421875	0.000112865390055503	\\
1511.71875	0.000115588337784902	\\
1512.6953125	0.000113427610346648	\\
1513.671875	0.000115816095948994	\\
1514.6484375	0.000116875754143847	\\
1515.625	0.000119261464076529	\\
1516.6015625	0.000116338417729314	\\
1517.578125	0.000118295893984695	\\
1518.5546875	0.00012091829387455	\\
1519.53125	0.000117306406101938	\\
1520.5078125	0.000118256540064825	\\
1521.484375	0.000121749474213729	\\
1522.4609375	0.000122001731221195	\\
1523.4375	0.00012366131079251	\\
1524.4140625	0.000123462863738448	\\
1525.390625	0.000125147511790467	\\
1526.3671875	0.00012475273581387	\\
1527.34375	0.000126179055164711	\\
1528.3203125	0.000120095607206276	\\
1529.296875	0.000123014443750624	\\
1530.2734375	0.000121616559168395	\\
1531.25	0.000122922319014893	\\
1532.2265625	0.0001241731996661	\\
1533.203125	0.000123013716579948	\\
1534.1796875	0.000120554858700569	\\
1535.15625	0.000118787490887982	\\
1536.1328125	0.00012067231132516	\\
1537.109375	0.000120958324350456	\\
1538.0859375	0.000121036957833414	\\
1539.0625	0.000121985836264144	\\
1540.0390625	0.000118799066247295	\\
1541.015625	0.000121413872559384	\\
1541.9921875	0.000119529769883742	\\
1542.96875	0.000119383629430643	\\
1543.9453125	0.000121668586104863	\\
1544.921875	0.000118095110644696	\\
1545.8984375	0.000119932977102278	\\
1546.875	0.000119186747345433	\\
1547.8515625	0.00012135661683216	\\
1548.828125	0.000118186371037257	\\
1549.8046875	0.000122318087971169	\\
1550.78125	0.000120399787637843	\\
1551.7578125	0.000122403225353873	\\
1552.734375	0.000119652419980962	\\
1553.7109375	0.000120230055738904	\\
1554.6875	0.000124274700824892	\\
1555.6640625	0.000122564601416069	\\
1556.640625	0.000124015173229955	\\
1557.6171875	0.000121129228991483	\\
1558.59375	0.000120808695600407	\\
1559.5703125	0.000125972651078962	\\
1560.546875	0.000126380115815357	\\
1561.5234375	0.000122370443277614	\\
1562.5	0.000120894203182126	\\
1563.4765625	0.000126302053638221	\\
1564.453125	0.000127520877412948	\\
1565.4296875	0.000126296574589254	\\
1566.40625	0.000127402137453717	\\
1567.3828125	0.00012957755068051	\\
1568.359375	0.000123737352894491	\\
1569.3359375	0.000127871726452614	\\
1570.3125	0.000128707258806734	\\
1571.2890625	0.000129401611426034	\\
1572.265625	0.000131985136757263	\\
1573.2421875	0.000127307170255837	\\
1574.21875	0.0001328848014919	\\
1575.1953125	0.000127661506767172	\\
1576.171875	0.000129244267027533	\\
1577.1484375	0.000130830115676593	\\
1578.125	0.000129272500537412	\\
1579.1015625	0.00012957937776652	\\
1580.078125	0.000128572582353414	\\
1581.0546875	0.000128039099718054	\\
1582.03125	0.000125735431197653	\\
1583.0078125	0.000127991690425677	\\
1583.984375	0.000125636680499571	\\
1584.9609375	0.000128762357398776	\\
1585.9375	0.000123858586758542	\\
1586.9140625	0.000126437127135374	\\
1587.890625	0.000124425878680691	\\
1588.8671875	0.000122843413775096	\\
1589.84375	0.000122401139183964	\\
1590.8203125	0.000126343569368222	\\
1591.796875	0.000123416432301839	\\
1592.7734375	0.000121989007485419	\\
1593.75	0.000122095089971202	\\
1594.7265625	0.000122466203961396	\\
1595.703125	0.000120701914858438	\\
1596.6796875	0.000119637628543725	\\
1597.65625	0.000121424847348135	\\
1598.6328125	0.000118995013303126	\\
1599.609375	0.000116553529739617	\\
1600.5859375	0.000108807015159548	\\
1601.5625	0.000123274495127916	\\
1602.5390625	0.000116013052435048	\\
1603.515625	0.000114486277993485	\\
1604.4921875	0.000116276226261333	\\
1605.46875	0.000115850249567002	\\
1606.4453125	0.000113898248160207	\\
1607.421875	0.000106677236164781	\\
1608.3984375	0.000112478025785116	\\
1609.375	0.000113737867217649	\\
1610.3515625	0.000110428208969475	\\
1611.328125	0.000111518727645362	\\
1612.3046875	0.000109154536562048	\\
1613.28125	0.000110399176695926	\\
1614.2578125	0.000113977692275706	\\
1615.234375	0.000109500130740872	\\
1616.2109375	0.000106334384697957	\\
1617.1875	0.000107399907397796	\\
1618.1640625	0.000104900877136514	\\
1619.140625	0.000105136612207957	\\
1620.1171875	0.000107358926177327	\\
1621.09375	0.000105644681099727	\\
1622.0703125	0.000107930428745043	\\
1623.046875	0.000111352768785447	\\
1624.0234375	0.000108241611620272	\\
1625	0.000109335325822241	\\
1625.9765625	0.000106213532697408	\\
1626.953125	0.000108314639032473	\\
1627.9296875	0.000110731842699803	\\
1628.90625	0.000110510815554026	\\
1629.8828125	0.00010989233026856	\\
1630.859375	0.000109296067000989	\\
1631.8359375	0.0001097753182079	\\
1632.8125	0.000111821537472097	\\
1633.7890625	0.000110121796530808	\\
1634.765625	0.000112518644619469	\\
1635.7421875	0.000113130838941097	\\
1636.71875	0.000112999522479522	\\
1637.6953125	0.000111391247727031	\\
1638.671875	0.000111304224558038	\\
1639.6484375	0.000112004469332388	\\
1640.625	0.000111790814825516	\\
1641.6015625	0.000111395633761448	\\
1642.578125	0.000112164580715716	\\
1643.5546875	0.000112036182734507	\\
1644.53125	0.000114484187449808	\\
1645.5078125	0.000116531978461627	\\
1646.484375	0.000121564414338826	\\
1647.4609375	0.000117648034413845	\\
1648.4375	0.000117626659259281	\\
1649.4140625	0.000117335000817818	\\
1650.390625	0.000118854697741539	\\
1651.3671875	0.000117759144387339	\\
1652.34375	0.000120638305203735	\\
1653.3203125	0.000123791643752132	\\
1654.296875	0.000120598045577559	\\
1655.2734375	0.000121913197612773	\\
1656.25	0.000124685389640273	\\
1657.2265625	0.000121631099580923	\\
1658.203125	0.000123546154342277	\\
1659.1796875	0.000119524838560062	\\
1660.15625	0.00012119498363224	\\
1661.1328125	0.000123989690280405	\\
1662.109375	0.000124874125493467	\\
1663.0859375	0.000124234177848114	\\
1664.0625	0.00012288877503671	\\
1665.0390625	0.000124238067913462	\\
1666.015625	0.000126110154457296	\\
1666.9921875	0.000125105257430486	\\
1667.96875	0.000127948328452652	\\
1668.9453125	0.000126297450806879	\\
1669.921875	0.000124607931149136	\\
1670.8984375	0.000130328683041731	\\
1671.875	0.000129332476435152	\\
1672.8515625	0.00012685585398323	\\
1673.828125	0.000128237299103246	\\
1674.8046875	0.000129876219530083	\\
1675.78125	0.000127068807996972	\\
1676.7578125	0.000125743691509091	\\
1677.734375	0.000126614713104364	\\
1678.7109375	0.000131284418554807	\\
1679.6875	0.000125418544396074	\\
1680.6640625	0.000126109254654207	\\
1681.640625	0.000123848646809733	\\
1682.6171875	0.00012351467405748	\\
1683.59375	0.00012820087420842	\\
1684.5703125	0.000124907730615026	\\
1685.546875	0.00012929176245871	\\
1686.5234375	0.000130168701509493	\\
1687.5	0.000129948347816542	\\
1688.4765625	0.000127781534175412	\\
1689.453125	0.000129646223990115	\\
1690.4296875	0.00012622039808075	\\
1691.40625	0.000125359140244553	\\
1692.3828125	0.000127838446494535	\\
1693.359375	0.000127205094852116	\\
1694.3359375	0.000124308401209876	\\
1695.3125	0.000124364473970674	\\
1696.2890625	0.000123488799876012	\\
1697.265625	0.000124081144636815	\\
1698.2421875	0.00012272570570069	\\
1699.21875	0.000123832301307535	\\
1700.1953125	0.000121738849159626	\\
1701.171875	0.000118933409544003	\\
1702.1484375	0.000117034532352557	\\
1703.125	0.000120645682702643	\\
1704.1015625	0.000123868896374282	\\
1705.078125	0.000121895395459561	\\
1706.0546875	0.000124168044057583	\\
1707.03125	0.00012150000194669	\\
1708.0078125	0.000120830323821949	\\
1708.984375	0.000122485622077199	\\
1709.9609375	0.000120714527941923	\\
1710.9375	0.000121826918488289	\\
1711.9140625	0.00012245817037508	\\
1712.890625	0.000117718705852211	\\
1713.8671875	0.000119806913465584	\\
1714.84375	0.000118736963207191	\\
1715.8203125	0.000116770880490609	\\
1716.796875	0.000116724442315164	\\
1717.7734375	0.000119419950570677	\\
1718.75	0.000116605250182503	\\
1719.7265625	0.000113206340625051	\\
1720.703125	0.000115240059600894	\\
1721.6796875	0.000111500825329051	\\
1722.65625	0.000115348296092661	\\
1723.6328125	0.000116521189642771	\\
1724.609375	0.000115297832168182	\\
1725.5859375	0.000115830628173954	\\
1726.5625	0.00011380122976047	\\
1727.5390625	0.000112744893389965	\\
1728.515625	0.00011100476061813	\\
1729.4921875	0.000112444133878182	\\
1730.46875	0.000111602923080901	\\
1731.4453125	0.000110727055243006	\\
1732.421875	0.00011150502835293	\\
1733.3984375	0.000112016063574971	\\
1734.375	0.000112761353293443	\\
1735.3515625	0.000109182596041917	\\
1736.328125	0.000110254934488439	\\
1737.3046875	0.00011012197833871	\\
1738.28125	0.00011224304369932	\\
1739.2578125	0.000112215195950698	\\
1740.234375	0.000109006214018378	\\
1741.2109375	0.000108309324828898	\\
1742.1875	0.000108725072558081	\\
1743.1640625	0.000113734463249664	\\
1744.140625	0.000111341780813466	\\
1745.1171875	0.000111268791854708	\\
1746.09375	0.00010905474774075	\\
1747.0703125	0.000108524991873341	\\
1748.046875	0.000112188764936567	\\
1749.0234375	0.000108041335626125	\\
1750	0.00010998149903398	\\
1750.9765625	0.000107639694554568	\\
1751.953125	0.000106761617443954	\\
1752.9296875	0.000104955886113346	\\
1753.90625	0.000107430243623172	\\
1754.8828125	0.000109805492039095	\\
1755.859375	0.000108377115555173	\\
1756.8359375	0.000108624095756423	\\
1757.8125	0.000105366784106859	\\
1758.7890625	0.00010602908589005	\\
1759.765625	0.00011116430929049	\\
1760.7421875	0.000108684222645485	\\
1761.71875	0.000106882714074656	\\
1762.6953125	0.000108166821097622	\\
1763.671875	0.000108958615528744	\\
1764.6484375	0.000107271058429676	\\
1765.625	0.000108213676091706	\\
1766.6015625	0.000107166548562067	\\
1767.578125	0.000104324169374909	\\
1768.5546875	0.000107055392562425	\\
1769.53125	0.000105393190327654	\\
1770.5078125	0.000109312940497968	\\
1771.484375	0.000106563229270401	\\
1772.4609375	0.000107169974488407	\\
1773.4375	0.000107808478520791	\\
1774.4140625	0.000107187859018979	\\
1775.390625	0.00010646002783434	\\
1776.3671875	0.000107580749031816	\\
1777.34375	0.000105432463573963	\\
1778.3203125	0.000109368579914517	\\
1779.296875	0.000107334551837195	\\
1780.2734375	0.000110221736688422	\\
1781.25	0.000105500039193221	\\
1782.2265625	0.000108412262852302	\\
1783.203125	0.000109238573319262	\\
1784.1796875	0.000107319767140741	\\
1785.15625	0.000108371491207021	\\
1786.1328125	0.000110015665716735	\\
1787.109375	0.000106627457251632	\\
1788.0859375	0.000106726179901106	\\
1789.0625	0.00010940973488976	\\
1790.0390625	0.000106994830479337	\\
1791.015625	0.00010926712045325	\\
1791.9921875	0.000103778812530719	\\
1792.96875	0.000105895491176919	\\
1793.9453125	0.000108011079150562	\\
1794.921875	0.000105962157621524	\\
1795.8984375	0.000106165880069089	\\
1796.875	0.000110678843846267	\\
1797.8515625	0.000103089387524724	\\
1798.828125	0.00011119600812544	\\
1799.8046875	0.000108866277021583	\\
1800.78125	0.000108089337793804	\\
1801.7578125	0.000103009936473959	\\
1802.734375	0.000105966780834877	\\
1803.7109375	0.000103174868339667	\\
1804.6875	0.000104919175841358	\\
1805.6640625	0.000106998331772022	\\
1806.640625	0.000100204069489608	\\
1807.6171875	0.000100085953662788	\\
1808.59375	0.00010272699058423	\\
1809.5703125	0.000101785706133562	\\
1810.546875	0.000104343803397914	\\
1811.5234375	0.000103777334304082	\\
1812.5	0.000103327953585357	\\
1813.4765625	0.000103016328563547	\\
1814.453125	9.89363003685222e-05	\\
1815.4296875	0.000103945901132428	\\
1816.40625	0.000102071297655778	\\
1817.3828125	9.58377462561653e-05	\\
1818.359375	0.000102328871155867	\\
1819.3359375	0.000102372906185061	\\
1820.3125	9.67795295306084e-05	\\
1821.2890625	9.88981601303546e-05	\\
1822.265625	9.75780172389258e-05	\\
1823.2421875	9.91953620670616e-05	\\
1824.21875	9.65737819736162e-05	\\
1825.1953125	9.5655716564934e-05	\\
1826.171875	9.88203535168958e-05	\\
1827.1484375	9.42721812918304e-05	\\
1828.125	9.41093022325463e-05	\\
1829.1015625	9.63076480661012e-05	\\
1830.078125	9.60512019146778e-05	\\
1831.0546875	9.45958493224165e-05	\\
1832.03125	9.53252013316973e-05	\\
1833.0078125	9.32266211831091e-05	\\
1833.984375	9.4125419180389e-05	\\
1834.9609375	8.97209091640168e-05	\\
1835.9375	9.46112015872276e-05	\\
1836.9140625	9.43658257765968e-05	\\
1837.890625	9.005899812363e-05	\\
1838.8671875	9.50735287484824e-05	\\
1839.84375	8.90066451285003e-05	\\
1840.8203125	8.91572297710133e-05	\\
1841.796875	8.96710232117539e-05	\\
1842.7734375	9.07284860448467e-05	\\
1843.75	8.63729523692262e-05	\\
1844.7265625	9.15386189041087e-05	\\
1845.703125	8.64119052196069e-05	\\
1846.6796875	8.93727387665563e-05	\\
1847.65625	9.07298790181859e-05	\\
1848.6328125	9.19398840076446e-05	\\
1849.609375	8.91080721450163e-05	\\
1850.5859375	9.24145543230773e-05	\\
1851.5625	8.87203314720183e-05	\\
1852.5390625	8.73084632311838e-05	\\
1853.515625	9.01334670679761e-05	\\
1854.4921875	8.68677945192038e-05	\\
1855.46875	9.09191809979496e-05	\\
1856.4453125	8.6130446124041e-05	\\
1857.421875	8.77682624413867e-05	\\
1858.3984375	8.57794497327504e-05	\\
1859.375	8.87069098633398e-05	\\
1860.3515625	8.5928578502877e-05	\\
1861.328125	8.59350021901228e-05	\\
1862.3046875	8.68476301438373e-05	\\
1863.28125	8.7872450265669e-05	\\
1864.2578125	8.85292537331998e-05	\\
1865.234375	8.81113919389346e-05	\\
1866.2109375	8.81761403464153e-05	\\
1867.1875	8.65527729824974e-05	\\
1868.1640625	8.74457159981748e-05	\\
1869.140625	8.68651074800648e-05	\\
1870.1171875	8.74768523724309e-05	\\
1871.09375	8.87639453592289e-05	\\
1872.0703125	9.03774868262291e-05	\\
1873.046875	9.03052090341949e-05	\\
1874.0234375	8.49474046258507e-05	\\
1875	9.00405806732701e-05	\\
1875.9765625	8.71334519256327e-05	\\
1876.953125	8.67682565666945e-05	\\
1877.9296875	8.67705916019656e-05	\\
1878.90625	8.66859438846377e-05	\\
1879.8828125	8.520827540838e-05	\\
1880.859375	8.81466281909955e-05	\\
1881.8359375	8.35722991552598e-05	\\
1882.8125	8.46590377446625e-05	\\
1883.7890625	8.75611117804718e-05	\\
1884.765625	8.79692393838881e-05	\\
1885.7421875	8.77713264048513e-05	\\
1886.71875	8.78189071642815e-05	\\
1887.6953125	9.00133574943236e-05	\\
1888.671875	8.94126147634509e-05	\\
1889.6484375	8.80892929787839e-05	\\
1890.625	8.83274425441474e-05	\\
1891.6015625	8.86439236725281e-05	\\
1892.578125	9.01922456784955e-05	\\
1893.5546875	8.87833337678791e-05	\\
1894.53125	8.72027295354851e-05	\\
1895.5078125	8.95537013447696e-05	\\
1896.484375	8.82833448111666e-05	\\
1897.4609375	8.51705688885096e-05	\\
1898.4375	8.42703566141137e-05	\\
1899.4140625	8.41193918937399e-05	\\
1900.390625	8.37776646445666e-05	\\
1901.3671875	8.77014889720255e-05	\\
1902.34375	8.5067281289055e-05	\\
1903.3203125	8.59054842968387e-05	\\
1904.296875	8.68853396744819e-05	\\
1905.2734375	8.7964240255214e-05	\\
1906.25	9.00639906841513e-05	\\
1907.2265625	8.88113041633969e-05	\\
1908.203125	8.37112288596342e-05	\\
1909.1796875	8.73457339197992e-05	\\
1910.15625	8.69177918240257e-05	\\
1911.1328125	8.51909422553583e-05	\\
1912.109375	8.61320154499593e-05	\\
1913.0859375	9.09099421020954e-05	\\
1914.0625	9.12786677834146e-05	\\
1915.0390625	9.06719947873007e-05	\\
1916.015625	9.0666256696698e-05	\\
1916.9921875	9.08358814590421e-05	\\
1917.96875	8.77786376039023e-05	\\
1918.9453125	8.62227821652724e-05	\\
1919.921875	8.55655138810445e-05	\\
1920.8984375	9.02824346401885e-05	\\
1921.875	8.91831818474238e-05	\\
1922.8515625	8.61233219989611e-05	\\
1923.828125	8.56669147610826e-05	\\
1924.8046875	9.34947729866328e-05	\\
1925.78125	8.83289863244321e-05	\\
1926.7578125	8.88781011553631e-05	\\
1927.734375	9.20693709768403e-05	\\
1928.7109375	8.74585724167036e-05	\\
1929.6875	9.01231061210788e-05	\\
1930.6640625	8.94072000053504e-05	\\
1931.640625	9.39806050674572e-05	\\
1932.6171875	9.27825846906686e-05	\\
1933.59375	9.35109822299892e-05	\\
1934.5703125	9.15967142997813e-05	\\
1935.546875	9.3174507436734e-05	\\
1936.5234375	9.04043226166378e-05	\\
1937.5	9.10369567411658e-05	\\
1938.4765625	8.93310290348181e-05	\\
1939.453125	8.97040836211091e-05	\\
1940.4296875	8.9340869177868e-05	\\
1941.40625	8.98320966710122e-05	\\
1942.3828125	8.97767768811711e-05	\\
1943.359375	8.88165333851624e-05	\\
1944.3359375	8.80019796851244e-05	\\
1945.3125	8.60942761778237e-05	\\
1946.2890625	8.87005546469158e-05	\\
1947.265625	8.97401275470904e-05	\\
1948.2421875	8.72417881599986e-05	\\
1949.21875	9.02514372054311e-05	\\
1950.1953125	9.09914516501304e-05	\\
1951.171875	8.89830505920273e-05	\\
1952.1484375	9.06633970337155e-05	\\
1953.125	9.12909200485107e-05	\\
1954.1015625	9.16703069373332e-05	\\
1955.078125	9.13096058301114e-05	\\
1956.0546875	8.79523055814618e-05	\\
1957.03125	8.93768382551817e-05	\\
1958.0078125	9.12420778864304e-05	\\
1958.984375	8.954828138643e-05	\\
1959.9609375	8.66058201174048e-05	\\
1960.9375	8.91039657256826e-05	\\
1961.9140625	9.07748111522205e-05	\\
1962.890625	9.13590894777137e-05	\\
1963.8671875	9.05318177279546e-05	\\
1964.84375	9.1066691226639e-05	\\
1965.8203125	9.1720058695774e-05	\\
1966.796875	9.37772746918745e-05	\\
1967.7734375	9.20160268289978e-05	\\
1968.75	9.04860623998688e-05	\\
1969.7265625	9.2351600192384e-05	\\
1970.703125	9.29102278521813e-05	\\
1971.6796875	8.83667333336917e-05	\\
1972.65625	9.04545878503873e-05	\\
1973.6328125	9.00641068081225e-05	\\
1974.609375	8.79599867683246e-05	\\
1975.5859375	9.29480537599809e-05	\\
1976.5625	9.19165313376669e-05	\\
1977.5390625	9.08675846860354e-05	\\
1978.515625	9.07225621877017e-05	\\
1979.4921875	9.48017493129546e-05	\\
1980.46875	9.15373156239458e-05	\\
1981.4453125	9.26847307653979e-05	\\
1982.421875	9.17303051293725e-05	\\
1983.3984375	9.29012129435057e-05	\\
1984.375	9.28555986655364e-05	\\
1985.3515625	9.45433634571937e-05	\\
1986.328125	9.26648373642803e-05	\\
1987.3046875	9.18761720702572e-05	\\
1988.28125	9.35170832927778e-05	\\
1989.2578125	9.37819567387981e-05	\\
1990.234375	9.16065985765556e-05	\\
1991.2109375	9.22602449970016e-05	\\
1992.1875	9.18639753368399e-05	\\
1993.1640625	9.21458262141116e-05	\\
1994.140625	9.29897356656637e-05	\\
1995.1171875	9.01236014374602e-05	\\
1996.09375	9.46557138726662e-05	\\
1997.0703125	8.97600861480342e-05	\\
1998.046875	9.17803382150896e-05	\\
1999.0234375	9.04663192532254e-05	\\
2000	9.06801793911246e-05	\\
2000.9765625	8.89416950720669e-05	\\
2001.953125	9.0879339802752e-05	\\
2002.9296875	9.04832349455203e-05	\\
2003.90625	9.13169824145695e-05	\\
2004.8828125	9.02481915901383e-05	\\
2005.859375	9.15757321841421e-05	\\
2006.8359375	9.04089835117761e-05	\\
2007.8125	9.05283057505229e-05	\\
2008.7890625	9.12902426330043e-05	\\
2009.765625	9.12978174312735e-05	\\
2010.7421875	8.79817476704986e-05	\\
2011.71875	8.84462893016537e-05	\\
2012.6953125	8.96215057521745e-05	\\
2013.671875	9.20157820649349e-05	\\
2014.6484375	8.82899641058983e-05	\\
2015.625	8.75080560145517e-05	\\
2016.6015625	8.80070305821302e-05	\\
2017.578125	8.59126998761636e-05	\\
2018.5546875	8.48027400106208e-05	\\
2019.53125	8.53113159727498e-05	\\
2020.5078125	8.02713461577442e-05	\\
2021.484375	8.41148840984366e-05	\\
2022.4609375	8.67173397710433e-05	\\
2023.4375	8.21820444777774e-05	\\
2024.4140625	8.17996848361462e-05	\\
2025.390625	7.95357864437591e-05	\\
2026.3671875	7.91845218285216e-05	\\
2027.34375	7.71978408189353e-05	\\
2028.3203125	7.83903234701613e-05	\\
2029.296875	7.69444784952836e-05	\\
2030.2734375	7.64057451387006e-05	\\
2031.25	7.43759141390474e-05	\\
2032.2265625	7.27325323733426e-05	\\
2033.203125	7.20843349815028e-05	\\
2034.1796875	6.8270001278672e-05	\\
2035.15625	7.2137408224821e-05	\\
2036.1328125	7.08435678303005e-05	\\
2037.109375	6.60222468691509e-05	\\
2038.0859375	6.39750068278896e-05	\\
2039.0625	6.63216072033126e-05	\\
2040.0390625	6.12442141978293e-05	\\
2041.015625	6.42898820469134e-05	\\
2041.9921875	5.75581738414443e-05	\\
2042.96875	5.93114600022863e-05	\\
2043.9453125	5.72244682842368e-05	\\
2044.921875	6.02837423037885e-05	\\
2045.8984375	5.73239896246496e-05	\\
2046.875	5.3278379389555e-05	\\
2047.8515625	5.44336611530301e-05	\\
2048.828125	5.35762489010926e-05	\\
2049.8046875	5.20310316069988e-05	\\
2050.78125	5.32868636801919e-05	\\
2051.7578125	4.97498846527547e-05	\\
2052.734375	4.77291007930654e-05	\\
2053.7109375	5.25624003080617e-05	\\
2054.6875	5.13325287172084e-05	\\
2055.6640625	4.65059248087282e-05	\\
2056.640625	4.69152759673768e-05	\\
2057.6171875	4.45958264607971e-05	\\
2058.59375	4.19004534199197e-05	\\
2059.5703125	4.2996541760037e-05	\\
2060.546875	4.26716487036751e-05	\\
2061.5234375	3.8773002039269e-05	\\
2062.5	4.05255618411346e-05	\\
2063.4765625	3.80075814595717e-05	\\
2064.453125	3.82312216832789e-05	\\
2065.4296875	3.46728472204869e-05	\\
2066.40625	3.92156784904568e-05	\\
2067.3828125	3.82450179060434e-05	\\
2068.359375	3.62817334321908e-05	\\
2069.3359375	3.87105409077275e-05	\\
2070.3125	3.65179230233273e-05	\\
2071.2890625	3.76252987801829e-05	\\
2072.265625	3.53363878521999e-05	\\
2073.2421875	3.52316805588199e-05	\\
2074.21875	3.46933236000195e-05	\\
2075.1953125	3.47709605138249e-05	\\
2076.171875	3.42940263740739e-05	\\
2077.1484375	3.30576867467643e-05	\\
2078.125	3.54461534911628e-05	\\
2079.1015625	3.29155522056265e-05	\\
2080.078125	3.28100352740946e-05	\\
2081.0546875	3.20177959090973e-05	\\
2082.03125	3.20229887763462e-05	\\
2083.0078125	2.96126284013353e-05	\\
2083.984375	3.41662759040213e-05	\\
2084.9609375	3.17425062795112e-05	\\
2085.9375	3.26339954470388e-05	\\
2086.9140625	3.25358246806361e-05	\\
2087.890625	2.90486194132808e-05	\\
2088.8671875	3.26622284382464e-05	\\
2089.84375	3.57929281691663e-05	\\
2090.8203125	3.13557350110472e-05	\\
2091.796875	3.33449232463319e-05	\\
2092.7734375	3.47307623174561e-05	\\
2093.75	3.34823664464201e-05	\\
2094.7265625	3.64722322778896e-05	\\
2095.703125	3.41353233010847e-05	\\
2096.6796875	3.65974687220967e-05	\\
2097.65625	3.46992260561987e-05	\\
2098.6328125	3.95037863904351e-05	\\
2099.609375	3.8832074103747e-05	\\
2100.5859375	4.33321814262601e-05	\\
2101.5625	3.69543826095997e-05	\\
2102.5390625	3.60410815115795e-05	\\
2103.515625	4.08151459218673e-05	\\
2104.4921875	4.16343578826944e-05	\\
2105.46875	4.31343693408449e-05	\\
2106.4453125	4.27957429868868e-05	\\
2107.421875	4.05080810401405e-05	\\
2108.3984375	4.38382506759848e-05	\\
2109.375	4.27689908830463e-05	\\
2110.3515625	4.5928981400269e-05	\\
2111.328125	4.50549356189519e-05	\\
2112.3046875	4.84302591660997e-05	\\
2113.28125	4.33200104633018e-05	\\
2114.2578125	4.63712832382412e-05	\\
2115.234375	4.53817611895142e-05	\\
2116.2109375	4.98423865697535e-05	\\
2117.1875	4.92200907700446e-05	\\
2118.1640625	4.75028234433485e-05	\\
2119.140625	4.6464669044419e-05	\\
2120.1171875	5.10459997506985e-05	\\
2121.09375	4.68927623978026e-05	\\
2122.0703125	4.73736319437822e-05	\\
2123.046875	5.0006476769679e-05	\\
2124.0234375	5.08477861411733e-05	\\
2125	4.84167660645333e-05	\\
2125.9765625	5.04997405340511e-05	\\
2126.953125	5.1381511804256e-05	\\
2127.9296875	4.83918169458965e-05	\\
2128.90625	4.57402791052152e-05	\\
2129.8828125	4.85680176209039e-05	\\
2130.859375	5.13743895983584e-05	\\
2131.8359375	4.96810191256607e-05	\\
2132.8125	5.19575495865586e-05	\\
2133.7890625	5.01573459564202e-05	\\
2134.765625	4.69835870095165e-05	\\
2135.7421875	4.92806395767379e-05	\\
2136.71875	5.12698504806669e-05	\\
2137.6953125	5.11882026176662e-05	\\
2138.671875	4.93968852913047e-05	\\
2139.6484375	4.7050361639397e-05	\\
2140.625	4.9203823965806e-05	\\
2141.6015625	4.98850162778815e-05	\\
2142.578125	4.34197922510234e-05	\\
2143.5546875	4.80878740466477e-05	\\
2144.53125	4.67789565688556e-05	\\
2145.5078125	4.71274722909951e-05	\\
2146.484375	4.87977600742606e-05	\\
2147.4609375	4.48266323821542e-05	\\
2148.4375	4.59327631052793e-05	\\
2149.4140625	4.48044095535832e-05	\\
2150.390625	4.57052115222215e-05	\\
2151.3671875	4.61964024373542e-05	\\
2152.34375	4.36777259152111e-05	\\
2153.3203125	4.23913499628309e-05	\\
2154.296875	4.33312030222277e-05	\\
2155.2734375	4.14341940096069e-05	\\
2156.25	4.51621780669581e-05	\\
2157.2265625	4.14480659510549e-05	\\
2158.203125	4.46783456619518e-05	\\
2159.1796875	4.34580584089893e-05	\\
2160.15625	4.52038294257535e-05	\\
2161.1328125	4.53729919869218e-05	\\
2162.109375	4.2663242669412e-05	\\
2163.0859375	4.01368473334327e-05	\\
2164.0625	4.13164498527604e-05	\\
2165.0390625	4.00590479049212e-05	\\
2166.015625	4.15056506103026e-05	\\
2166.9921875	3.98656470337524e-05	\\
2167.96875	4.1730919915096e-05	\\
2168.9453125	3.80829102126855e-05	\\
2169.921875	3.93923314303165e-05	\\
2170.8984375	4.03429788107336e-05	\\
2171.875	3.8105314852252e-05	\\
2172.8515625	3.8336582006231e-05	\\
2173.828125	3.85959244418813e-05	\\
2174.8046875	3.54547798347632e-05	\\
2175.78125	3.59559120231284e-05	\\
2176.7578125	3.47397381308913e-05	\\
2177.734375	3.74438587468373e-05	\\
2178.7109375	3.37754951220909e-05	\\
2179.6875	3.64558122454385e-05	\\
2180.6640625	3.62940280873082e-05	\\
2181.640625	3.36904116896546e-05	\\
2182.6171875	3.34371752216082e-05	\\
2183.59375	3.38980078519184e-05	\\
2184.5703125	3.3947114492317e-05	\\
2185.546875	3.03164802342826e-05	\\
2186.5234375	3.67519683268806e-05	\\
2187.5	3.21385327700624e-05	\\
2188.4765625	3.46726456022483e-05	\\
2189.453125	3.77602770622084e-05	\\
2190.4296875	3.26119575127941e-05	\\
2191.40625	3.51745752549879e-05	\\
2192.3828125	3.23507053868935e-05	\\
2193.359375	3.57977046007997e-05	\\
2194.3359375	3.3223294422691e-05	\\
2195.3125	3.62691824225029e-05	\\
2196.2890625	3.43937919453396e-05	\\
2197.265625	3.48027833233695e-05	\\
2198.2421875	3.70229928142263e-05	\\
2199.21875	3.35102266578728e-05	\\
2200.1953125	3.4660255226251e-05	\\
2201.171875	3.71614534249172e-05	\\
2202.1484375	3.7010155398845e-05	\\
2203.125	3.76608000566042e-05	\\
2204.1015625	3.83497538983684e-05	\\
2205.078125	4.20968204853393e-05	\\
2206.0546875	4.07793930199317e-05	\\
2207.03125	4.14591886363265e-05	\\
2208.0078125	4.01359567097531e-05	\\
2208.984375	4.46625075450272e-05	\\
2209.9609375	4.28456777679685e-05	\\
2210.9375	4.28727056730064e-05	\\
2211.9140625	4.46795990963797e-05	\\
2212.890625	4.48931851959409e-05	\\
2213.8671875	4.67767044011369e-05	\\
2214.84375	4.75840447493512e-05	\\
2215.8203125	4.83421272022935e-05	\\
2216.796875	4.86958666023807e-05	\\
2217.7734375	5.01639237069749e-05	\\
2218.75	5.26670315657852e-05	\\
2219.7265625	4.98908362663553e-05	\\
2220.703125	5.14012430530683e-05	\\
2221.6796875	5.43966880702032e-05	\\
2222.65625	5.17448199018498e-05	\\
2223.6328125	5.20266489248034e-05	\\
2224.609375	5.26818844231877e-05	\\
2225.5859375	5.54934449626792e-05	\\
2226.5625	5.68918930848756e-05	\\
2227.5390625	5.57033160690357e-05	\\
2228.515625	5.76854458127513e-05	\\
2229.4921875	5.82215312237407e-05	\\
2230.46875	5.76829575840439e-05	\\
2231.4453125	6.12967168001668e-05	\\
2232.421875	6.19608125993476e-05	\\
2233.3984375	6.33643951667767e-05	\\
2234.375	6.10460079383722e-05	\\
2235.3515625	6.47217159192897e-05	\\
2236.328125	6.41269591064627e-05	\\
2237.3046875	6.49628937091456e-05	\\
2238.28125	7.00977979814775e-05	\\
2239.2578125	6.57984791118504e-05	\\
2240.234375	6.71791079295056e-05	\\
2241.2109375	6.74757786578053e-05	\\
2242.1875	7.15415113322703e-05	\\
2243.1640625	6.98773227050169e-05	\\
2244.140625	7.32380694957165e-05	\\
2245.1171875	7.43756114690142e-05	\\
2246.09375	7.50867607901345e-05	\\
2247.0703125	7.59731753650379e-05	\\
2248.046875	7.59850040599517e-05	\\
2249.0234375	7.28866095614558e-05	\\
2250	7.89804013756e-05	\\
2250.9765625	7.78701663490332e-05	\\
2251.953125	7.89323113768071e-05	\\
2252.9296875	8.28587224622753e-05	\\
2253.90625	8.11038953783321e-05	\\
2254.8828125	8.0005423959318e-05	\\
2255.859375	8.03634343515864e-05	\\
2256.8359375	8.3715453790133e-05	\\
2257.8125	8.38918840054419e-05	\\
2258.7890625	8.71824827697026e-05	\\
2259.765625	8.46246311097179e-05	\\
2260.7421875	8.59349553173324e-05	\\
2261.71875	8.73208107893347e-05	\\
2262.6953125	8.67317815786079e-05	\\
2263.671875	8.69825220854095e-05	\\
2264.6484375	8.59523151902612e-05	\\
2265.625	8.8116332279915e-05	\\
2266.6015625	8.69201478221907e-05	\\
2267.578125	8.72277285343336e-05	\\
2268.5546875	8.79610189605964e-05	\\
2269.53125	8.88653211696448e-05	\\
2270.5078125	9.12286050105703e-05	\\
2271.484375	8.92219488846054e-05	\\
2272.4609375	9.24825435231931e-05	\\
2273.4375	9.20953932853624e-05	\\
2274.4140625	9.24872542878518e-05	\\
2275.390625	9.46487105237932e-05	\\
2276.3671875	9.40517600991387e-05	\\
2277.34375	9.43089349175059e-05	\\
2278.3203125	9.5491912475743e-05	\\
2279.296875	9.32232714639678e-05	\\
2280.2734375	9.8008919103644e-05	\\
2281.25	9.60188384497371e-05	\\
2282.2265625	9.55525401740291e-05	\\
2283.203125	9.22211486077852e-05	\\
2284.1796875	9.50191914087404e-05	\\
2285.15625	9.36472768089131e-05	\\
2286.1328125	9.3855383917094e-05	\\
2287.109375	9.80354060436862e-05	\\
2288.0859375	9.59859808542955e-05	\\
2289.0625	9.83022565913137e-05	\\
2290.0390625	9.48449565120852e-05	\\
2291.015625	9.66692588362785e-05	\\
2291.9921875	9.62661024749451e-05	\\
2292.96875	9.56617478739416e-05	\\
2293.9453125	9.51425463715111e-05	\\
2294.921875	9.85107435856176e-05	\\
2295.8984375	9.66595052453248e-05	\\
2296.875	9.75222819238834e-05	\\
2297.8515625	9.84774677940812e-05	\\
2298.828125	9.31862353565122e-05	\\
2299.8046875	9.94147564582063e-05	\\
2300.78125	9.68070601644347e-05	\\
2301.7578125	9.9757764655337e-05	\\
2302.734375	9.61929683903822e-05	\\
2303.7109375	9.71964722527125e-05	\\
2304.6875	9.50616074625862e-05	\\
2305.6640625	9.5716329566122e-05	\\
2306.640625	9.66879350730257e-05	\\
2307.6171875	9.55254994434541e-05	\\
2308.59375	9.48647156587263e-05	\\
2309.5703125	9.43924768075106e-05	\\
2310.546875	9.70022206233319e-05	\\
2311.5234375	9.72949405122124e-05	\\
2312.5	9.78842334322295e-05	\\
2313.4765625	9.38454379983721e-05	\\
2314.453125	9.72542431967565e-05	\\
2315.4296875	9.68330469975362e-05	\\
2316.40625	9.77546614346884e-05	\\
2317.3828125	9.69635930508296e-05	\\
2318.359375	9.72240811488257e-05	\\
2319.3359375	9.53057567768954e-05	\\
2320.3125	9.59858550117263e-05	\\
2321.2890625	9.36207683806994e-05	\\
2322.265625	9.7486144367916e-05	\\
2323.2421875	9.71478768893694e-05	\\
2324.21875	9.52859419870966e-05	\\
2325.1953125	9.68713193990793e-05	\\
2326.171875	9.56220862502217e-05	\\
2327.1484375	9.46121532208164e-05	\\
2328.125	9.75923729626399e-05	\\
2329.1015625	9.37624646481355e-05	\\
2330.078125	9.50902630314888e-05	\\
2331.0546875	9.51883637490173e-05	\\
2332.03125	9.76804425369982e-05	\\
2333.0078125	9.66971813614318e-05	\\
2333.984375	9.60831908092495e-05	\\
2334.9609375	9.74627620713094e-05	\\
2335.9375	9.730036908589e-05	\\
2336.9140625	9.48386585721819e-05	\\
2337.890625	9.81591065641531e-05	\\
2338.8671875	9.79240259036626e-05	\\
2339.84375	9.63994818576914e-05	\\
2340.8203125	9.53299667398582e-05	\\
2341.796875	9.63462166868793e-05	\\
2342.7734375	9.75987325866401e-05	\\
2343.75	9.69857043332011e-05	\\
2344.7265625	9.67215073590529e-05	\\
2345.703125	9.81090194639605e-05	\\
2346.6796875	9.78933644023135e-05	\\
2347.65625	9.95654949951141e-05	\\
2348.6328125	9.9292865091531e-05	\\
2349.609375	9.67377036804223e-05	\\
2350.5859375	0.000100087046496921	\\
2351.5625	9.69936882478786e-05	\\
2352.5390625	9.95462851854999e-05	\\
2353.515625	9.7677656618399e-05	\\
2354.4921875	0.00010015920720545	\\
2355.46875	9.87036679310306e-05	\\
2356.4453125	9.93255004625544e-05	\\
2357.421875	9.82675179505143e-05	\\
2358.3984375	0.000100612314711143	\\
2359.375	9.99446525307282e-05	\\
2360.3515625	0.000100978946664375	\\
2361.328125	9.90152324257982e-05	\\
2362.3046875	9.94325151068287e-05	\\
2363.28125	0.000100752536914403	\\
2364.2578125	0.000101847739957337	\\
2365.234375	0.000101919862846937	\\
2366.2109375	0.000100315693337442	\\
2367.1875	0.000101357916859446	\\
2368.1640625	9.91562619097997e-05	\\
2369.140625	0.000100361195592612	\\
2370.1171875	0.000100095920708672	\\
2371.09375	0.000101795999053404	\\
2372.0703125	0.000102631163775313	\\
2373.046875	0.000104805654687831	\\
2374.0234375	0.000101873786541552	\\
2375	0.000104205765148337	\\
2375.9765625	0.000103440830263637	\\
2376.953125	0.000102512966001606	\\
2377.9296875	0.000102992343920228	\\
2378.90625	0.000102454005390934	\\
2379.8828125	0.000103712190790743	\\
2380.859375	0.000100583379719652	\\
2381.8359375	9.8468778216327e-05	\\
2382.8125	0.000102816859888173	\\
2383.7890625	9.94978763325869e-05	\\
2384.765625	0.000103291134096047	\\
2385.7421875	0.000102353130657187	\\
2386.71875	9.88893257132716e-05	\\
2387.6953125	9.96470865394166e-05	\\
2388.671875	0.000104128091282532	\\
2389.6484375	9.85136696744261e-05	\\
2390.625	0.000100324026402956	\\
2391.6015625	0.000100766135312356	\\
2392.578125	0.000103081613973264	\\
2393.5546875	0.00010145289809158	\\
2394.53125	0.000101111501920207	\\
2395.5078125	0.000101963834081676	\\
2396.484375	0.000101093771746326	\\
2397.4609375	0.000103080018294023	\\
2398.4375	0.000102383387451979	\\
2399.4140625	0.000102360235369762	\\
2400.390625	0.000104787098940278	\\
2401.3671875	0.000100303031603843	\\
2402.34375	0.000100778316162336	\\
2403.3203125	0.0001016562055442	\\
2404.296875	9.799615217044e-05	\\
2405.2734375	0.00010298984621674	\\
2406.25	0.000102387102359045	\\
2407.2265625	0.000100771541657144	\\
2408.203125	0.000102595845925478	\\
2409.1796875	0.000100715537334478	\\
2410.15625	0.000102270651839353	\\
2411.1328125	0.00010171852719281	\\
2412.109375	9.82128046156838e-05	\\
2413.0859375	0.000101599883689566	\\
2414.0625	0.000100753228377195	\\
2415.0390625	0.000101575799121024	\\
2416.015625	0.000103012601252424	\\
2416.9921875	0.000104713123185143	\\
2417.96875	0.000102409667788226	\\
2418.9453125	0.000101592067115656	\\
2419.921875	0.000102720368804156	\\
2420.8984375	9.91799589105114e-05	\\
2421.875	0.000100461538931697	\\
2422.8515625	0.000100384015072924	\\
2423.828125	9.97415500498726e-05	\\
2424.8046875	0.000100633575811849	\\
2425.78125	0.000101400149838794	\\
2426.7578125	0.000101598764670755	\\
2427.734375	0.000101161036697659	\\
2428.7109375	0.000102186177105851	\\
2429.6875	0.000101374034607	\\
2430.6640625	0.000103837366126435	\\
2431.640625	0.00010247910336141	\\
2432.6171875	0.000103339625348209	\\
2433.59375	0.000101174573136989	\\
2434.5703125	0.000104134681947604	\\
2435.546875	0.000100422170162636	\\
2436.5234375	0.000102847008067285	\\
2437.5	0.000100278812976945	\\
2438.4765625	0.000102160905153071	\\
2439.453125	0.000103825894759509	\\
2440.4296875	0.000102500515739036	\\
2441.40625	0.000104148805206431	\\
2442.3828125	0.000101675454276527	\\
2443.359375	0.000101598222198977	\\
2444.3359375	0.000101939907912143	\\
2445.3125	0.000105163157910068	\\
2446.2890625	0.000103855111209127	\\
2447.265625	0.000102720644228615	\\
2448.2421875	0.000105325635017849	\\
2449.21875	0.000103890721605507	\\
2450.1953125	0.000105561615378124	\\
2451.171875	0.000104558896260969	\\
2452.1484375	0.000105135602387606	\\
2453.125	0.000103618836810762	\\
2454.1015625	0.000104426040781318	\\
2455.078125	0.0001042187214992	\\
2456.0546875	0.00010282331909627	\\
2457.03125	0.000101681928969399	\\
2458.0078125	0.000101944762084101	\\
2458.984375	0.000103479201507491	\\
2459.9609375	0.0001026797811742	\\
2460.9375	0.000102797294664787	\\
2461.9140625	9.98847303555728e-05	\\
2462.890625	0.000103759981014685	\\
2463.8671875	0.000102044768122446	\\
2464.84375	0.000103053519967276	\\
2465.8203125	0.000102819797783118	\\
2466.796875	0.000102481380322052	\\
2467.7734375	0.000102554762379566	\\
2468.75	0.000102299422014962	\\
2469.7265625	9.83143516717594e-05	\\
2470.703125	9.79902657413933e-05	\\
2471.6796875	0.00010157253002456	\\
2472.65625	9.94044387699524e-05	\\
2473.6328125	0.000101877429124151	\\
2474.609375	9.6494798275672e-05	\\
2475.5859375	9.93687762512964e-05	\\
2476.5625	9.86118784782846e-05	\\
2477.5390625	0.000102378897780925	\\
2478.515625	9.89746274086731e-05	\\
2479.4921875	9.72918023537893e-05	\\
2480.46875	9.64511749475608e-05	\\
2481.4453125	9.83592936038912e-05	\\
2482.421875	9.52298477153067e-05	\\
2483.3984375	9.74229555698369e-05	\\
2484.375	9.48285673234165e-05	\\
2485.3515625	9.46489279320432e-05	\\
2486.328125	9.7547067168731e-05	\\
2487.3046875	9.41622512171405e-05	\\
2488.28125	9.27173002549813e-05	\\
2489.2578125	9.77372204364152e-05	\\
2490.234375	9.5743549818154e-05	\\
2491.2109375	9.60897506325271e-05	\\
2492.1875	9.4108765573268e-05	\\
2493.1640625	9.86823014567918e-05	\\
2494.140625	9.33582943444286e-05	\\
2495.1171875	9.84098718368251e-05	\\
2496.09375	9.74383622062499e-05	\\
2497.0703125	9.63466060407101e-05	\\
2498.046875	9.83844199476312e-05	\\
2499.0234375	9.61417392388778e-05	\\
2500	0.000100859541030438	\\
2500.9765625	9.821227009821e-05	\\
2501.953125	9.8193946751077e-05	\\
2502.9296875	9.79519636323625e-05	\\
2503.90625	9.72915243117223e-05	\\
2504.8828125	0.000100770047798484	\\
2505.859375	9.77810900161205e-05	\\
2506.8359375	0.000100079084938112	\\
2507.8125	0.000100116403591124	\\
2508.7890625	0.000100860810213673	\\
2509.765625	0.00010070009074617	\\
2510.7421875	9.89855997643318e-05	\\
2511.71875	9.91588367597672e-05	\\
2512.6953125	0.000100891417139867	\\
2513.671875	0.000103217681118998	\\
2514.6484375	0.000100086970998963	\\
2515.625	9.9012350939589e-05	\\
2516.6015625	9.95708579694111e-05	\\
2517.578125	0.000104214578288983	\\
2518.5546875	0.000100660494946379	\\
2519.53125	9.76189010208978e-05	\\
2520.5078125	9.83402580349909e-05	\\
2521.484375	9.92485190717121e-05	\\
2522.4609375	9.44862670909544e-05	\\
2523.4375	9.88884660033184e-05	\\
2524.4140625	9.77644514722345e-05	\\
2525.390625	9.70618682791148e-05	\\
2526.3671875	9.78124405442028e-05	\\
2527.34375	9.62730879983636e-05	\\
2528.3203125	9.76609749764191e-05	\\
2529.296875	9.89454583731273e-05	\\
2530.2734375	9.51526227601056e-05	\\
2531.25	9.62084930439981e-05	\\
2532.2265625	9.62247442302334e-05	\\
2533.203125	9.4385422426702e-05	\\
2534.1796875	9.44541247991584e-05	\\
2535.15625	9.70946914278222e-05	\\
2536.1328125	9.42345148949163e-05	\\
2537.109375	9.57815892831248e-05	\\
2538.0859375	9.37880521452454e-05	\\
2539.0625	9.27854536946706e-05	\\
2540.0390625	9.44267642004576e-05	\\
2541.015625	9.06244644064752e-05	\\
2541.9921875	9.51861566197496e-05	\\
2542.96875	9.36991144009285e-05	\\
2543.9453125	9.17526859852209e-05	\\
2544.921875	9.35495727646223e-05	\\
2545.8984375	9.08358788251489e-05	\\
2546.875	9.23781330853302e-05	\\
2547.8515625	9.14113646958314e-05	\\
2548.828125	8.85755259370299e-05	\\
2549.8046875	9.13361233522045e-05	\\
2550.78125	8.93910683361747e-05	\\
2551.7578125	8.94871956912889e-05	\\
2552.734375	8.75273950144497e-05	\\
2553.7109375	8.59155175944784e-05	\\
2554.6875	8.8408174293058e-05	\\
2555.6640625	8.53806560909767e-05	\\
2556.640625	8.40189765117115e-05	\\
2557.6171875	8.63639699419615e-05	\\
2558.59375	8.56725938179444e-05	\\
2559.5703125	8.44624320250347e-05	\\
2560.546875	8.52217855057187e-05	\\
2561.5234375	8.33630055417345e-05	\\
2562.5	8.0721448724292e-05	\\
2563.4765625	8.01726597524979e-05	\\
2564.453125	7.91778840820043e-05	\\
2565.4296875	8.0001786958775e-05	\\
2566.40625	8.25036102782437e-05	\\
2567.3828125	7.84391757981866e-05	\\
2568.359375	7.93813406584252e-05	\\
2569.3359375	7.81574261596619e-05	\\
2570.3125	7.62788727393795e-05	\\
2571.2890625	7.79030155512209e-05	\\
2572.265625	7.66114418546317e-05	\\
2573.2421875	7.61540903811316e-05	\\
2574.21875	7.30622307084084e-05	\\
2575.1953125	7.46726577142789e-05	\\
2576.171875	7.40252887020458e-05	\\
2577.1484375	7.32555977499827e-05	\\
2578.125	7.38508084691082e-05	\\
2579.1015625	7.10191034333761e-05	\\
2580.078125	7.44404283369459e-05	\\
2581.0546875	7.17559359775262e-05	\\
2582.03125	6.9844652975792e-05	\\
2583.0078125	7.07741604547593e-05	\\
2583.984375	6.90076234969254e-05	\\
2584.9609375	7.11932218481226e-05	\\
2585.9375	6.83073191699276e-05	\\
2586.9140625	6.88470207536479e-05	\\
2587.890625	6.88366210858384e-05	\\
2588.8671875	6.97944085856299e-05	\\
2589.84375	6.88454939586813e-05	\\
2590.8203125	6.79002125504311e-05	\\
2591.796875	6.6510108876538e-05	\\
2592.7734375	6.81466810426295e-05	\\
2593.75	6.88498650415113e-05	\\
2594.7265625	6.63981222897575e-05	\\
2595.703125	6.75577291480776e-05	\\
2596.6796875	6.50439206588159e-05	\\
2597.65625	6.73554499657101e-05	\\
2598.6328125	6.33344971949803e-05	\\
2599.609375	6.41463314192023e-05	\\
2600.5859375	6.4040514397562e-05	\\
2601.5625	6.85495931306272e-05	\\
2602.5390625	6.64103865287925e-05	\\
2603.515625	6.41770086405799e-05	\\
2604.4921875	6.39672016510536e-05	\\
2605.46875	6.27455348781003e-05	\\
2606.4453125	6.60313754464886e-05	\\
2607.421875	6.42963248202088e-05	\\
2608.3984375	6.56137266464299e-05	\\
2609.375	6.38982074387444e-05	\\
2610.3515625	6.34435554184043e-05	\\
2611.328125	6.67329567828135e-05	\\
2612.3046875	6.17580211535899e-05	\\
2613.28125	6.40050600057849e-05	\\
2614.2578125	6.39659861757619e-05	\\
2615.234375	6.41424328593078e-05	\\
2616.2109375	6.24049632808048e-05	\\
2617.1875	6.33639279678864e-05	\\
2618.1640625	6.25034087428042e-05	\\
2619.140625	6.32988771326159e-05	\\
2620.1171875	6.0488470707652e-05	\\
2621.09375	6.1667338730725e-05	\\
2622.0703125	6.04786271015264e-05	\\
2623.046875	6.20900933273558e-05	\\
2624.0234375	6.28645794137042e-05	\\
2625	6.0879424309621e-05	\\
2625.9765625	6.13349773658208e-05	\\
2626.953125	5.8379797982449e-05	\\
2627.9296875	6.34225034998852e-05	\\
2628.90625	6.17524567337805e-05	\\
2629.8828125	6.1536943372396e-05	\\
2630.859375	6.27867582863289e-05	\\
2631.8359375	6.07995523078548e-05	\\
2632.8125	6.1729917132938e-05	\\
2633.7890625	5.81114998926309e-05	\\
2634.765625	6.2240254496282e-05	\\
2635.7421875	5.91665159033875e-05	\\
2636.71875	5.89847305428511e-05	\\
2637.6953125	6.0033475931408e-05	\\
2638.671875	5.91523871597651e-05	\\
2639.6484375	5.91109302340683e-05	\\
2640.625	5.92985881941495e-05	\\
2641.6015625	6.12537525386514e-05	\\
2642.578125	5.89877785355458e-05	\\
2643.5546875	6.07339106908228e-05	\\
2644.53125	5.91540112509784e-05	\\
2645.5078125	5.87224116523434e-05	\\
2646.484375	6.12886965052709e-05	\\
2647.4609375	5.84944141644563e-05	\\
2648.4375	5.90997865889641e-05	\\
2649.4140625	5.88077534991932e-05	\\
2650.390625	5.95636766509965e-05	\\
2651.3671875	6.06169423833403e-05	\\
2652.34375	6.09600261141477e-05	\\
2653.3203125	5.79459346726071e-05	\\
2654.296875	5.91185238518446e-05	\\
2655.2734375	5.96184042860095e-05	\\
2656.25	5.8487169602072e-05	\\
2657.2265625	6.19350700928416e-05	\\
2658.203125	6.08177582622255e-05	\\
2659.1796875	5.95636987763452e-05	\\
2660.15625	5.99288091746598e-05	\\
2661.1328125	6.07532873545557e-05	\\
2662.109375	6.0990002674834e-05	\\
2663.0859375	6.00388612796659e-05	\\
2664.0625	6.19213450309736e-05	\\
2665.0390625	5.91332561262359e-05	\\
2666.015625	6.35205500626181e-05	\\
2666.9921875	5.97326057104017e-05	\\
2667.96875	6.05456594744584e-05	\\
2668.9453125	6.10402090064819e-05	\\
2669.921875	6.18543474309319e-05	\\
2670.8984375	6.08282493766155e-05	\\
2671.875	6.2229752215193e-05	\\
2672.8515625	6.00640861989559e-05	\\
2673.828125	6.27987149562551e-05	\\
2674.8046875	6.32142052703424e-05	\\
2675.78125	6.32135587239743e-05	\\
2676.7578125	6.23972416529461e-05	\\
2677.734375	6.22089538998144e-05	\\
2678.7109375	6.02398795798406e-05	\\
2679.6875	6.06451642588401e-05	\\
2680.6640625	6.23499743595944e-05	\\
2681.640625	6.11702506341275e-05	\\
2682.6171875	6.15926092567787e-05	\\
2683.59375	6.1110322690789e-05	\\
2684.5703125	6.29996090353164e-05	\\
2685.546875	6.36433044547776e-05	\\
2686.5234375	6.11389851911132e-05	\\
2687.5	6.31991429245481e-05	\\
2688.4765625	6.40628768043421e-05	\\
2689.453125	6.33509800589168e-05	\\
2690.4296875	6.21844522336876e-05	\\
2691.40625	6.14424595392286e-05	\\
2692.3828125	6.26809878600806e-05	\\
2693.359375	6.3191183253917e-05	\\
2694.3359375	6.05155094762387e-05	\\
2695.3125	6.27116925085907e-05	\\
2696.2890625	6.33219025270472e-05	\\
2697.265625	5.9542762896229e-05	\\
2698.2421875	6.02884953446552e-05	\\
2699.21875	6.47186761169003e-05	\\
2700.1953125	6.3589288802123e-05	\\
2701.171875	5.77924757883148e-05	\\
2702.1484375	5.98562470888091e-05	\\
2703.125	5.95091481584636e-05	\\
2704.1015625	6.15743857147424e-05	\\
2705.078125	6.3652546476803e-05	\\
2706.0546875	5.8228232338246e-05	\\
2707.03125	5.90078153252183e-05	\\
2708.0078125	5.92461677880397e-05	\\
2708.984375	6.04725907426022e-05	\\
2709.9609375	5.9019589856272e-05	\\
2710.9375	5.95500166320416e-05	\\
2711.9140625	6.03190674156566e-05	\\
2712.890625	5.89150566287514e-05	\\
2713.8671875	5.5061774605817e-05	\\
2714.84375	5.82528145075601e-05	\\
2715.8203125	5.70621654175155e-05	\\
2716.796875	5.90815675532316e-05	\\
2717.7734375	5.51643621046382e-05	\\
2718.75	5.75060661684315e-05	\\
2719.7265625	5.62549725860629e-05	\\
2720.703125	5.40475182386947e-05	\\
2721.6796875	5.54385122807349e-05	\\
2722.65625	5.13776111452637e-05	\\
2723.6328125	5.19262484279238e-05	\\
2724.609375	5.19165234564066e-05	\\
2725.5859375	4.92949004743246e-05	\\
2726.5625	5.06759606659665e-05	\\
2727.5390625	4.96562030829456e-05	\\
2728.515625	4.88320007938666e-05	\\
2729.4921875	4.84008791347662e-05	\\
2730.46875	4.65479803549679e-05	\\
2731.4453125	4.7882696811758e-05	\\
2732.421875	4.5379277327505e-05	\\
2733.3984375	4.79346618336419e-05	\\
2734.375	4.40464329005898e-05	\\
2735.3515625	4.50347217986838e-05	\\
2736.328125	4.24397695734672e-05	\\
2737.3046875	4.27843417189008e-05	\\
2738.28125	4.12983239008845e-05	\\
2739.2578125	4.04696694038125e-05	\\
2740.234375	4.10145131460844e-05	\\
2741.2109375	3.68128538454107e-05	\\
2742.1875	3.71402606772492e-05	\\
2743.1640625	3.80909150184898e-05	\\
2744.140625	3.65669817012403e-05	\\
2745.1171875	3.4756020510392e-05	\\
2746.09375	3.63062695923061e-05	\\
2747.0703125	3.45561577383664e-05	\\
2748.046875	3.1410894608532e-05	\\
2749.0234375	3.14264091319694e-05	\\
2750	3.14563673733721e-05	\\
2750.9765625	2.87358246067489e-05	\\
2751.953125	2.83916960134492e-05	\\
2752.9296875	2.95601502421917e-05	\\
2753.90625	2.61388339025144e-05	\\
2754.8828125	2.84602719316375e-05	\\
2755.859375	2.61524320664526e-05	\\
2756.8359375	2.47603055622402e-05	\\
2757.8125	2.4293384993106e-05	\\
2758.7890625	2.57999216699465e-05	\\
2759.765625	2.57116091881735e-05	\\
2760.7421875	2.41372920192767e-05	\\
2761.71875	2.01470425470103e-05	\\
2762.6953125	1.92636371961007e-05	\\
2763.671875	1.80055117435095e-05	\\
2764.6484375	1.6916361803015e-05	\\
2765.625	1.79197473278086e-05	\\
2766.6015625	1.72846259562388e-05	\\
2767.578125	1.53336104862513e-05	\\
2768.5546875	1.51244529771578e-05	\\
2769.53125	1.34762431949673e-05	\\
2770.5078125	1.43858671461849e-05	\\
2771.484375	1.2906971214151e-05	\\
2772.4609375	1.03970863084423e-05	\\
2773.4375	1.13418061221597e-05	\\
2774.4140625	9.00263886033803e-06	\\
2775.390625	9.73047541741963e-06	\\
2776.3671875	7.78941524144465e-06	\\
2777.34375	7.9908309304935e-06	\\
2778.3203125	6.96818596625469e-06	\\
2779.296875	6.90879827652349e-06	\\
2780.2734375	6.3531518841255e-06	\\
2781.25	5.18795339126729e-06	\\
2782.2265625	6.88067731256454e-06	\\
2783.203125	4.6488503339937e-06	\\
2784.1796875	6.60110303279168e-06	\\
2785.15625	5.75360819050716e-06	\\
2786.1328125	5.71295133130674e-06	\\
2787.109375	5.33525621829297e-06	\\
2788.0859375	3.4056938431505e-06	\\
2789.0625	4.76054769820793e-06	\\
2790.0390625	4.65238290419306e-06	\\
2791.015625	5.32058233318372e-06	\\
2791.9921875	3.95941763626369e-06	\\
2792.96875	6.46930012581207e-06	\\
2793.9453125	4.61256042334178e-06	\\
2794.921875	4.47805517102328e-06	\\
2795.8984375	7.04628804078772e-06	\\
2796.875	6.68957298153794e-06	\\
2797.8515625	9.06161809911397e-06	\\
2798.828125	8.52291840486389e-06	\\
2799.8046875	8.11876518259692e-06	\\
2800.78125	6.59141493185466e-06	\\
2801.7578125	9.29207722803193e-06	\\
2802.734375	8.83926203810687e-06	\\
2803.7109375	8.0417399004833e-06	\\
2804.6875	1.1248953724048e-05	\\
2805.6640625	1.07906132480791e-05	\\
2806.640625	1.22629709475767e-05	\\
2807.6171875	1.09404359442719e-05	\\
2808.59375	1.19210868592869e-05	\\
2809.5703125	1.23722718267516e-05	\\
2810.546875	1.26345082234531e-05	\\
2811.5234375	1.5706333366667e-05	\\
2812.5	1.40301436298883e-05	\\
2813.4765625	1.34617172041096e-05	\\
2814.453125	1.42819864437164e-05	\\
2815.4296875	1.20932023779201e-05	\\
2816.40625	1.25968040528604e-05	\\
2817.3828125	1.35819841673733e-05	\\
2818.359375	1.47868247737146e-05	\\
2819.3359375	1.56724254659534e-05	\\
2820.3125	1.55433702469114e-05	\\
2821.2890625	1.5472487022605e-05	\\
2822.265625	1.53274882067695e-05	\\
2823.2421875	1.76579919096841e-05	\\
2824.21875	1.5628463249399e-05	\\
2825.1953125	1.57504091190146e-05	\\
2826.171875	1.6430214099176e-05	\\
2827.1484375	1.69150781613706e-05	\\
2828.125	1.69926152048181e-05	\\
2829.1015625	1.50982462721522e-05	\\
2830.078125	1.84882080554215e-05	\\
2831.0546875	1.77717863070018e-05	\\
2832.03125	1.66270346502037e-05	\\
2833.0078125	1.86448011905218e-05	\\
2833.984375	2.0018075183407e-05	\\
2834.9609375	1.86573096084435e-05	\\
2835.9375	1.73891606146768e-05	\\
2836.9140625	1.8588421021303e-05	\\
2837.890625	1.9040926043453e-05	\\
2838.8671875	1.63137767573483e-05	\\
2839.84375	1.91810935473833e-05	\\
2840.8203125	2.22279319383981e-05	\\
2841.796875	1.9532532769943e-05	\\
2842.7734375	2.01861587995596e-05	\\
2843.75	2.07191512405888e-05	\\
2844.7265625	2.12695342462464e-05	\\
2845.703125	2.25174696299743e-05	\\
2846.6796875	2.19402666383716e-05	\\
2847.65625	2.15989688598337e-05	\\
2848.6328125	2.15999374816991e-05	\\
2849.609375	1.99860278318613e-05	\\
2850.5859375	2.14385094516985e-05	\\
2851.5625	2.41782136579354e-05	\\
2852.5390625	2.22874679961353e-05	\\
2853.515625	2.22896888899812e-05	\\
2854.4921875	1.97354202426443e-05	\\
2855.46875	2.31074731967126e-05	\\
2856.4453125	2.2777710820364e-05	\\
2857.421875	2.27318784877603e-05	\\
2858.3984375	2.25131275534423e-05	\\
2859.375	2.07211815465533e-05	\\
2860.3515625	2.5199249998613e-05	\\
2861.328125	2.28077067684659e-05	\\
2862.3046875	2.45190572178546e-05	\\
2863.28125	2.31062378217618e-05	\\
2864.2578125	2.21116145555482e-05	\\
2865.234375	2.32279093384023e-05	\\
2866.2109375	2.22965343833266e-05	\\
2867.1875	2.11104093681068e-05	\\
2868.1640625	2.04852485450852e-05	\\
2869.140625	2.29724685636056e-05	\\
2870.1171875	2.25030481752657e-05	\\
2871.09375	2.49145473849732e-05	\\
2872.0703125	2.5155250227266e-05	\\
2873.046875	2.49820116563126e-05	\\
2874.0234375	2.58824183168286e-05	\\
2875	2.24207030476331e-05	\\
2875.9765625	2.45884976249619e-05	\\
2876.953125	2.63354965106815e-05	\\
2877.9296875	2.24347293415341e-05	\\
2878.90625	2.4974225568679e-05	\\
2879.8828125	2.5894454712245e-05	\\
2880.859375	2.5491338309409e-05	\\
2881.8359375	2.48974878120132e-05	\\
2882.8125	2.57311011807572e-05	\\
2883.7890625	2.57887487932805e-05	\\
2884.765625	2.85854613054474e-05	\\
2885.7421875	2.64003829342426e-05	\\
2886.71875	2.36326647668961e-05	\\
2887.6953125	2.56467822682382e-05	\\
2888.671875	2.66496521406514e-05	\\
2889.6484375	2.64349509926439e-05	\\
2890.625	2.57899006720213e-05	\\
2891.6015625	2.81713875274938e-05	\\
2892.578125	2.37848453351513e-05	\\
2893.5546875	2.63675875714043e-05	\\
2894.53125	2.75973089465874e-05	\\
2895.5078125	2.72072516238822e-05	\\
2896.484375	2.40348037200593e-05	\\
2897.4609375	2.51371547219039e-05	\\
2898.4375	2.87578666099216e-05	\\
2899.4140625	2.56913600366839e-05	\\
2900.390625	2.91977893452364e-05	\\
2901.3671875	2.62898671442619e-05	\\
2902.34375	2.66499932861912e-05	\\
2903.3203125	2.82875236557546e-05	\\
2904.296875	2.69731387772598e-05	\\
2905.2734375	2.61109893210448e-05	\\
2906.25	2.91695103712118e-05	\\
2907.2265625	2.7411408984877e-05	\\
2908.203125	2.85664567509944e-05	\\
2909.1796875	2.73149568280668e-05	\\
2910.15625	2.89965254036668e-05	\\
2911.1328125	2.74240600317477e-05	\\
2912.109375	2.72870617082034e-05	\\
2913.0859375	2.59572095776142e-05	\\
2914.0625	2.80756130576166e-05	\\
2915.0390625	2.73676063977922e-05	\\
2916.015625	2.67735469471944e-05	\\
2916.9921875	2.90057703901702e-05	\\
2917.96875	2.64301198117838e-05	\\
2918.9453125	2.72273399597698e-05	\\
2919.921875	2.81698670344649e-05	\\
2920.8984375	2.69717188334307e-05	\\
2921.875	2.52889115499297e-05	\\
2922.8515625	2.71922689958023e-05	\\
2923.828125	2.81620550109213e-05	\\
2924.8046875	2.95154623226998e-05	\\
2925.78125	2.91456923923447e-05	\\
2926.7578125	2.84188396796918e-05	\\
2927.734375	2.59223276374013e-05	\\
2928.7109375	2.76427798676936e-05	\\
2929.6875	2.79792965013122e-05	\\
2930.6640625	2.78039471465397e-05	\\
2931.640625	2.83630078696926e-05	\\
2932.6171875	2.84319915149258e-05	\\
2933.59375	2.96363007133559e-05	\\
2934.5703125	3.1616122294341e-05	\\
2935.546875	2.96260426039432e-05	\\
2936.5234375	3.01528616472492e-05	\\
2937.5	3.08167883916269e-05	\\
2938.4765625	2.89866460609956e-05	\\
2939.453125	3.08174649986905e-05	\\
2940.4296875	3.21177097596684e-05	\\
2941.40625	3.11234162095148e-05	\\
2942.3828125	3.22190518538691e-05	\\
2943.359375	2.89798110588643e-05	\\
2944.3359375	3.02396345883722e-05	\\
2945.3125	3.00742920815551e-05	\\
2946.2890625	3.20563311524499e-05	\\
2947.265625	3.04956523652009e-05	\\
2948.2421875	3.07840225103946e-05	\\
2949.21875	2.90684638360936e-05	\\
2950.1953125	3.00467036549529e-05	\\
2951.171875	3.19512029179122e-05	\\
2952.1484375	3.0477265796708e-05	\\
2953.125	3.16714631788315e-05	\\
2954.1015625	3.03480574575185e-05	\\
2955.078125	3.21824445764179e-05	\\
2956.0546875	2.95491431165958e-05	\\
2957.03125	3.22840972593621e-05	\\
2958.0078125	2.91059458138059e-05	\\
2958.984375	3.24683028209159e-05	\\
2959.9609375	3.26270096809615e-05	\\
2960.9375	3.35909944086898e-05	\\
2961.9140625	3.17841212795832e-05	\\
2962.890625	3.33614105299385e-05	\\
2963.8671875	3.4211753234178e-05	\\
2964.84375	3.20206027412695e-05	\\
2965.8203125	3.10314397336931e-05	\\
2966.796875	3.16419855895328e-05	\\
2967.7734375	3.12633624641551e-05	\\
2968.75	3.18275983399466e-05	\\
2969.7265625	3.4647940248515e-05	\\
2970.703125	3.21119271639894e-05	\\
2971.6796875	3.02718449072941e-05	\\
2972.65625	3.35478250191845e-05	\\
2973.6328125	3.20838743279415e-05	\\
2974.609375	3.006088407224e-05	\\
2975.5859375	3.27451394553099e-05	\\
2976.5625	3.37842683727197e-05	\\
2977.5390625	3.20750176713171e-05	\\
2978.515625	3.50744233105892e-05	\\
2979.4921875	3.06653720073664e-05	\\
2980.46875	3.31092354932919e-05	\\
2981.4453125	3.45071335286888e-05	\\
2982.421875	3.55395070606943e-05	\\
2983.3984375	3.31876717927812e-05	\\
2984.375	3.41738088449425e-05	\\
2985.3515625	3.34209306064913e-05	\\
2986.328125	3.51722513444219e-05	\\
2987.3046875	3.37711198595923e-05	\\
2988.28125	3.43229181119845e-05	\\
2989.2578125	3.49955794604597e-05	\\
2990.234375	3.52067652195916e-05	\\
2991.2109375	3.66591970492321e-05	\\
2992.1875	3.38458231433487e-05	\\
2993.1640625	3.55798303786598e-05	\\
2994.140625	3.44354113321484e-05	\\
2995.1171875	3.44490950378859e-05	\\
2996.09375	3.37216642772986e-05	\\
2997.0703125	3.24197508982148e-05	\\
2998.046875	3.2980422593069e-05	\\
2999.0234375	3.45930411847851e-05	\\
3000	3.26498445879872e-05	\\
3000.9765625	3.50294236741751e-05	\\
3001.953125	3.64963667999297e-05	\\
3002.9296875	3.21144809627352e-05	\\
3003.90625	3.36116974678025e-05	\\
3004.8828125	3.4068339306054e-05	\\
3005.859375	3.67412313540254e-05	\\
3006.8359375	3.53079233682291e-05	\\
3007.8125	3.38007365785189e-05	\\
3008.7890625	3.49697925081318e-05	\\
3009.765625	3.56997770505389e-05	\\
3010.7421875	3.49023638447769e-05	\\
3011.71875	3.73291575856844e-05	\\
3012.6953125	3.58250049227718e-05	\\
3013.671875	3.42140629908495e-05	\\
3014.6484375	3.64134111260863e-05	\\
3015.625	3.55084505561099e-05	\\
3016.6015625	3.44820877662997e-05	\\
3017.578125	3.47727423991022e-05	\\
3018.5546875	3.4000098389291e-05	\\
3019.53125	3.52770618036743e-05	\\
3020.5078125	3.63614773667809e-05	\\
3021.484375	3.59286613708121e-05	\\
3022.4609375	3.72120973289318e-05	\\
3023.4375	3.65484150336351e-05	\\
3024.4140625	3.52722785918245e-05	\\
3025.390625	3.58648787573246e-05	\\
3026.3671875	3.5263555526352e-05	\\
3027.34375	3.6066673125719e-05	\\
3028.3203125	3.57016153431043e-05	\\
3029.296875	3.35896373377324e-05	\\
3030.2734375	3.51216317370242e-05	\\
3031.25	3.56203751638042e-05	\\
3032.2265625	3.58623507555009e-05	\\
3033.203125	3.4473100082272e-05	\\
3034.1796875	3.54139639272217e-05	\\
3035.15625	3.49693972864942e-05	\\
3036.1328125	3.49398921777003e-05	\\
3037.109375	3.50731598803292e-05	\\
3038.0859375	3.50306408486444e-05	\\
3039.0625	3.71384355975487e-05	\\
3040.0390625	3.54855754183135e-05	\\
3041.015625	3.5747749315582e-05	\\
3041.9921875	3.79134176871766e-05	\\
3042.96875	3.53320833703304e-05	\\
3043.9453125	3.66092683001541e-05	\\
3044.921875	3.59524671602896e-05	\\
3045.8984375	3.28077865380545e-05	\\
3046.875	3.433619717842e-05	\\
3047.8515625	3.54941329057378e-05	\\
3048.828125	3.48947948419627e-05	\\
3049.8046875	3.51486653939672e-05	\\
3050.78125	3.63059868902079e-05	\\
3051.7578125	3.49679275528105e-05	\\
3052.734375	3.71450331652948e-05	\\
3053.7109375	3.68046207733246e-05	\\
3054.6875	3.71894714224076e-05	\\
3055.6640625	3.69574408945514e-05	\\
3056.640625	3.72233044094653e-05	\\
3057.6171875	3.58153782832848e-05	\\
3058.59375	3.59599625711596e-05	\\
3059.5703125	3.55016997206352e-05	\\
3060.546875	3.7421747283896e-05	\\
3061.5234375	3.74229972536407e-05	\\
3062.5	3.76098612476553e-05	\\
3063.4765625	3.55117363274478e-05	\\
3064.453125	3.76081261133199e-05	\\
3065.4296875	3.55619422043933e-05	\\
3066.40625	3.42236192389287e-05	\\
3067.3828125	3.39937109992441e-05	\\
3068.359375	3.52958526483514e-05	\\
3069.3359375	3.56876658393038e-05	\\
3070.3125	3.57299125067907e-05	\\
3071.2890625	3.47226046528069e-05	\\
3072.265625	3.40161253841739e-05	\\
3073.2421875	3.45700122079313e-05	\\
3074.21875	3.57392920963748e-05	\\
3075.1953125	3.47986125147459e-05	\\
3076.171875	3.41115497549692e-05	\\
3077.1484375	3.5826160759985e-05	\\
3078.125	3.60617782867738e-05	\\
3079.1015625	3.36614499422377e-05	\\
3080.078125	3.44477221270424e-05	\\
3081.0546875	3.34075952583177e-05	\\
3082.03125	3.43414321921998e-05	\\
3083.0078125	3.54410869006974e-05	\\
3083.984375	3.30475541952206e-05	\\
3084.9609375	3.37636627808415e-05	\\
3085.9375	3.48680982701951e-05	\\
3086.9140625	3.51821672465833e-05	\\
3087.890625	3.44145822288836e-05	\\
3088.8671875	3.3632287564388e-05	\\
3089.84375	3.07868931641692e-05	\\
3090.8203125	3.39672265970908e-05	\\
3091.796875	3.21384550163284e-05	\\
3092.7734375	3.40308830852182e-05	\\
3093.75	3.20985867580284e-05	\\
3094.7265625	3.39029587052104e-05	\\
3095.703125	3.45605478740251e-05	\\
3096.6796875	3.02499844093457e-05	\\
3097.65625	3.15755433292843e-05	\\
3098.6328125	3.11693408324754e-05	\\
3099.609375	3.14833534273443e-05	\\
3100.5859375	3.42185220528014e-05	\\
3101.5625	3.25655648592288e-05	\\
3102.5390625	3.24149527976025e-05	\\
3103.515625	3.16178656057868e-05	\\
3104.4921875	2.96138820880346e-05	\\
3105.46875	2.92049975650396e-05	\\
3106.4453125	3.19450441975908e-05	\\
3107.421875	3.19114481641218e-05	\\
3108.3984375	2.88320482502937e-05	\\
3109.375	3.10817095226146e-05	\\
3110.3515625	3.36268779346974e-05	\\
3111.328125	2.84943615585759e-05	\\
3112.3046875	3.09991134540061e-05	\\
3113.28125	3.08350771623451e-05	\\
3114.2578125	2.9796545921878e-05	\\
3115.234375	3.14457078600826e-05	\\
3116.2109375	3.10614210683317e-05	\\
3117.1875	2.96373957534634e-05	\\
3118.1640625	3.06789783817847e-05	\\
3119.140625	3.06685835320542e-05	\\
3120.1171875	2.92415116124135e-05	\\
3121.09375	3.1010763623464e-05	\\
3122.0703125	3.10819128434833e-05	\\
3123.046875	3.10844568526773e-05	\\
3124.0234375	2.96342588522307e-05	\\
3125	2.82484155753314e-05	\\
3125.9765625	3.05766033692105e-05	\\
3126.953125	2.91652985290526e-05	\\
3127.9296875	2.95370162559283e-05	\\
3128.90625	2.91443626882768e-05	\\
3129.8828125	2.94701496867845e-05	\\
3130.859375	2.87494245878489e-05	\\
3131.8359375	2.68377195644519e-05	\\
3132.8125	3.07232806644629e-05	\\
3133.7890625	2.79933704583777e-05	\\
3134.765625	2.85780763360501e-05	\\
3135.7421875	2.89422921538947e-05	\\
3136.71875	2.81724049323859e-05	\\
3137.6953125	2.88872726083395e-05	\\
3138.671875	2.92334959895963e-05	\\
3139.6484375	3.00336805419217e-05	\\
3140.625	2.72927740689705e-05	\\
3141.6015625	2.83502099835554e-05	\\
3142.578125	2.57162115881023e-05	\\
3143.5546875	2.69924203666805e-05	\\
3144.53125	2.78253599895042e-05	\\
3145.5078125	2.78036182649591e-05	\\
3146.484375	2.81386496080252e-05	\\
3147.4609375	2.80169850053766e-05	\\
3148.4375	2.55618663732882e-05	\\
3149.4140625	2.9031304192437e-05	\\
3150.390625	2.76604929389653e-05	\\
3151.3671875	2.66380856252809e-05	\\
3152.34375	2.50729018263763e-05	\\
3153.3203125	2.83879965369916e-05	\\
3154.296875	2.78451155105723e-05	\\
3155.2734375	2.82606422646704e-05	\\
3156.25	2.55600485843321e-05	\\
3157.2265625	2.77702296774028e-05	\\
3158.203125	2.47780693945841e-05	\\
3159.1796875	2.64089301421992e-05	\\
3160.15625	2.80631801473207e-05	\\
3161.1328125	2.66743847422907e-05	\\
3162.109375	2.53788002911041e-05	\\
3163.0859375	2.59241580969057e-05	\\
3164.0625	2.57246331801171e-05	\\
3165.0390625	2.65162001041185e-05	\\
3166.015625	2.67155731447539e-05	\\
3166.9921875	2.57099608991133e-05	\\
3167.96875	2.76610099349505e-05	\\
3168.9453125	2.60655527956326e-05	\\
3169.921875	2.40884153102062e-05	\\
3170.8984375	2.76261737411976e-05	\\
3171.875	2.67640520572693e-05	\\
3172.8515625	2.79369534584664e-05	\\
3173.828125	2.6429380404242e-05	\\
3174.8046875	2.69640380987007e-05	\\
3175.78125	2.57998708782954e-05	\\
3176.7578125	2.77547025700335e-05	\\
3177.734375	2.68232185610116e-05	\\
3178.7109375	2.49029937798286e-05	\\
3179.6875	2.64622350651744e-05	\\
3180.6640625	2.7525249004268e-05	\\
3181.640625	2.61828255733802e-05	\\
3182.6171875	2.57732538812736e-05	\\
3183.59375	2.58994474601392e-05	\\
3184.5703125	2.69742271747599e-05	\\
3185.546875	2.6511570095167e-05	\\
3186.5234375	2.75736062941314e-05	\\
3187.5	2.73240052380003e-05	\\
3188.4765625	2.79925735607876e-05	\\
3189.453125	2.92798615499237e-05	\\
3190.4296875	2.88102249389653e-05	\\
3191.40625	2.83446591837197e-05	\\
3192.3828125	3.01109157210244e-05	\\
3193.359375	2.92219595917822e-05	\\
3194.3359375	2.90692311960411e-05	\\
3195.3125	2.67455935346269e-05	\\
3196.2890625	2.9672391996593e-05	\\
3197.265625	3.06177049096462e-05	\\
3198.2421875	2.95898027316756e-05	\\
3199.21875	2.9752630946501e-05	\\
3200.1953125	3.0723599786462e-05	\\
3201.171875	3.03768060955706e-05	\\
3202.1484375	2.92123228357661e-05	\\
3203.125	3.12857841914466e-05	\\
3204.1015625	3.15804219311689e-05	\\
3205.078125	2.92883470892242e-05	\\
3206.0546875	3.15860641670388e-05	\\
3207.03125	3.12411540061388e-05	\\
3208.0078125	3.094503395752e-05	\\
3208.984375	3.23557473007214e-05	\\
3209.9609375	3.10477207722022e-05	\\
3210.9375	3.14793421224717e-05	\\
3211.9140625	3.19893271293489e-05	\\
3212.890625	3.24167543208186e-05	\\
3213.8671875	3.32563502016417e-05	\\
3214.84375	3.29576039652999e-05	\\
3215.8203125	3.21559746994664e-05	\\
3216.796875	3.20020932947969e-05	\\
3217.7734375	3.26815862316216e-05	\\
3218.75	3.48122780507803e-05	\\
3219.7265625	3.25746175048444e-05	\\
3220.703125	3.35696030675657e-05	\\
3221.6796875	3.18617737992731e-05	\\
3222.65625	3.21761884253154e-05	\\
3223.6328125	3.40872487899431e-05	\\
3224.609375	3.53003380074278e-05	\\
3225.5859375	3.48107540721986e-05	\\
3226.5625	3.38157085636667e-05	\\
3227.5390625	3.40085912528969e-05	\\
3228.515625	3.53603730694673e-05	\\
3229.4921875	3.38101281594035e-05	\\
3230.46875	3.64794314219643e-05	\\
3231.4453125	3.5785850956898e-05	\\
3232.421875	3.49595498599268e-05	\\
3233.3984375	3.35298703369843e-05	\\
3234.375	3.53274769343448e-05	\\
3235.3515625	3.45037456987589e-05	\\
3236.328125	3.52926834537325e-05	\\
3237.3046875	3.49305807430575e-05	\\
3238.28125	3.59888651251129e-05	\\
3239.2578125	3.54755516198523e-05	\\
3240.234375	3.44338516462779e-05	\\
3241.2109375	3.58742676628016e-05	\\
3242.1875	3.71152847115539e-05	\\
3243.1640625	3.69465825709745e-05	\\
3244.140625	3.57334638762422e-05	\\
3245.1171875	3.77146506736707e-05	\\
3246.09375	3.64047660284017e-05	\\
3247.0703125	3.66774744525992e-05	\\
3248.046875	3.64751695035965e-05	\\
3249.0234375	3.51922479882223e-05	\\
3250	3.53177932574838e-05	\\
3250.9765625	3.70843005095033e-05	\\
3251.953125	3.69001712838974e-05	\\
3252.9296875	3.6714374789888e-05	\\
3253.90625	3.75355195157046e-05	\\
3254.8828125	3.7953050618249e-05	\\
3255.859375	3.65593010174272e-05	\\
3256.8359375	3.64614726969193e-05	\\
3257.8125	3.61577540464856e-05	\\
3258.7890625	3.62549496136296e-05	\\
3259.765625	3.77939662189371e-05	\\
3260.7421875	3.69506465674615e-05	\\
3261.71875	3.54117128361597e-05	\\
3262.6953125	3.49438936586157e-05	\\
3263.671875	3.75947664225051e-05	\\
3264.6484375	3.75948761471253e-05	\\
3265.625	3.79583365192714e-05	\\
3266.6015625	3.70332988266784e-05	\\
3267.578125	3.5994201713211e-05	\\
3268.5546875	3.68161480211044e-05	\\
3269.53125	3.79588488415076e-05	\\
3270.5078125	3.80245864209806e-05	\\
3271.484375	3.72934956202249e-05	\\
3272.4609375	3.75462160921998e-05	\\
3273.4375	3.65878142140555e-05	\\
3274.4140625	3.7757840625445e-05	\\
3275.390625	3.81363632367165e-05	\\
3276.3671875	3.85435522376935e-05	\\
3277.34375	3.78798183923987e-05	\\
3278.3203125	3.83348219395533e-05	\\
3279.296875	3.64944353836352e-05	\\
3280.2734375	3.81039101647778e-05	\\
3281.25	3.89594967061912e-05	\\
3282.2265625	3.80755378814036e-05	\\
3283.203125	3.96913318084003e-05	\\
3284.1796875	3.84874274634404e-05	\\
3285.15625	3.88478928694928e-05	\\
3286.1328125	3.96683274378273e-05	\\
3287.109375	4.04224779259354e-05	\\
3288.0859375	3.88082977367876e-05	\\
3289.0625	3.86582510429752e-05	\\
3290.0390625	3.90735098575523e-05	\\
3291.015625	4.1002762630365e-05	\\
3291.9921875	4.08196400711493e-05	\\
3292.96875	4.17392450771033e-05	\\
3293.9453125	4.07955081578331e-05	\\
3294.921875	4.1748717585997e-05	\\
3295.8984375	4.14731117315795e-05	\\
3296.875	3.87029979220739e-05	\\
3297.8515625	4.04022583370032e-05	\\
3298.828125	4.06516233670824e-05	\\
3299.8046875	4.08262002448703e-05	\\
3300.78125	4.16054373600512e-05	\\
3301.7578125	3.99303721523328e-05	\\
3302.734375	4.11220571829046e-05	\\
3303.7109375	4.04137678475603e-05	\\
3304.6875	3.91041533685109e-05	\\
3305.6640625	4.04604082026269e-05	\\
3306.640625	4.0380979765375e-05	\\
3307.6171875	3.8892838813977e-05	\\
3308.59375	3.89395235142131e-05	\\
3309.5703125	3.98527825497735e-05	\\
3310.546875	3.84708996906554e-05	\\
3311.5234375	4.02690308486859e-05	\\
3312.5	4.04001303776818e-05	\\
3313.4765625	4.0488939178489e-05	\\
3314.453125	3.97802981170387e-05	\\
3315.4296875	3.98777835703924e-05	\\
3316.40625	3.87479677440944e-05	\\
3317.3828125	3.93060444926692e-05	\\
3318.359375	4.05806715301381e-05	\\
3319.3359375	3.89978712614007e-05	\\
3320.3125	4.08522841915837e-05	\\
3321.2890625	4.02031498143316e-05	\\
3322.265625	3.96571572591754e-05	\\
3323.2421875	3.89255259429558e-05	\\
3324.21875	3.89318101987035e-05	\\
3325.1953125	4.09885113868334e-05	\\
3326.171875	4.17456338569046e-05	\\
3327.1484375	3.88620527207838e-05	\\
3328.125	4.06350691329075e-05	\\
3329.1015625	4.02332495383893e-05	\\
3330.078125	3.89141723104874e-05	\\
3331.0546875	4.10076258120307e-05	\\
3332.03125	4.06700895268924e-05	\\
3333.0078125	4.03056769390846e-05	\\
3333.984375	4.00337711271658e-05	\\
3334.9609375	4.03080678977584e-05	\\
3335.9375	4.071466688403e-05	\\
3336.9140625	3.95483549119248e-05	\\
3337.890625	3.97401900758627e-05	\\
3338.8671875	3.95975781924516e-05	\\
3339.84375	4.01855356331828e-05	\\
3340.8203125	4.09658539716829e-05	\\
3341.796875	4.06289250003642e-05	\\
3342.7734375	4.06782546427152e-05	\\
3343.75	4.09115499906609e-05	\\
3344.7265625	3.91032880034386e-05	\\
3345.703125	3.91244549342019e-05	\\
3346.6796875	4.06978114026772e-05	\\
3347.65625	3.82047998340897e-05	\\
3348.6328125	3.82849918441828e-05	\\
3349.609375	3.99134750433991e-05	\\
3350.5859375	3.92118435374847e-05	\\
3351.5625	3.87954470353341e-05	\\
3352.5390625	4.06117215065906e-05	\\
3353.515625	4.08973340057772e-05	\\
3354.4921875	3.98531611666528e-05	\\
3355.46875	3.95808473451207e-05	\\
3356.4453125	3.99327450187741e-05	\\
3357.421875	3.97460723956317e-05	\\
3358.3984375	4.0632855269099e-05	\\
3359.375	4.01019050125598e-05	\\
3360.3515625	4.01822709429561e-05	\\
3361.328125	3.86479276578325e-05	\\
3362.3046875	3.98343816178961e-05	\\
3363.28125	4.02623338266632e-05	\\
3364.2578125	3.85419983848458e-05	\\
3365.234375	3.97309017263763e-05	\\
3366.2109375	3.77476152942524e-05	\\
3367.1875	3.81590364922339e-05	\\
3368.1640625	3.79284756436893e-05	\\
3369.140625	3.8751078190476e-05	\\
3370.1171875	3.86129811705388e-05	\\
3371.09375	3.78923346055961e-05	\\
3372.0703125	3.82719238111177e-05	\\
3373.046875	3.85676877879649e-05	\\
3374.0234375	3.87958678377935e-05	\\
3375	3.69713805521837e-05	\\
3375.9765625	3.95915685019246e-05	\\
3376.953125	3.82242906519106e-05	\\
3377.9296875	3.71806185159744e-05	\\
3378.90625	3.8702033942484e-05	\\
3379.8828125	3.87070434272118e-05	\\
3380.859375	3.71926536152362e-05	\\
3381.8359375	3.83422434041369e-05	\\
3382.8125	3.7612705723241e-05	\\
3383.7890625	3.86563719094023e-05	\\
3384.765625	3.8426478823846e-05	\\
3385.7421875	3.55750811642845e-05	\\
3386.71875	3.73870266541703e-05	\\
3387.6953125	3.62956681088397e-05	\\
3388.671875	3.68023808628379e-05	\\
3389.6484375	3.63097723452058e-05	\\
3390.625	3.7094535869522e-05	\\
3391.6015625	3.81918327832318e-05	\\
3392.578125	3.57786195632903e-05	\\
3393.5546875	3.70604626672256e-05	\\
3394.53125	3.75218458947095e-05	\\
3395.5078125	3.67457865321347e-05	\\
3396.484375	3.8199290209176e-05	\\
3397.4609375	3.61614635416514e-05	\\
3398.4375	3.72970408785056e-05	\\
3399.4140625	3.55001434231666e-05	\\
3400.390625	3.69040105893881e-05	\\
3401.3671875	3.51589703142101e-05	\\
3402.34375	3.68415619303257e-05	\\
3403.3203125	3.46933401882232e-05	\\
3404.296875	3.68879046364181e-05	\\
3405.2734375	3.38325699745162e-05	\\
3406.25	3.45797303993096e-05	\\
3407.2265625	3.34653000433997e-05	\\
3408.203125	3.6065589754891e-05	\\
3409.1796875	3.47320617490851e-05	\\
3410.15625	3.40619155313936e-05	\\
3411.1328125	3.52509400620948e-05	\\
3412.109375	3.43858812072593e-05	\\
3413.0859375	3.40246656924705e-05	\\
3414.0625	3.6297154084104e-05	\\
3415.0390625	3.47000910207834e-05	\\
3416.015625	3.44017677777805e-05	\\
3416.9921875	3.441664524047e-05	\\
3417.96875	3.57357842170991e-05	\\
3418.9453125	3.39547115761469e-05	\\
3419.921875	3.39618186159044e-05	\\
3420.8984375	3.36260301382123e-05	\\
3421.875	3.46572886436119e-05	\\
3422.8515625	3.42099416073707e-05	\\
3423.828125	3.36085919199065e-05	\\
3424.8046875	3.4549615755368e-05	\\
3425.78125	3.41229928912918e-05	\\
3426.7578125	3.18826518105445e-05	\\
3427.734375	3.31983973197184e-05	\\
3428.7109375	3.41014517663729e-05	\\
3429.6875	3.28931873738366e-05	\\
3430.6640625	3.29571477649708e-05	\\
3431.640625	3.17874863036205e-05	\\
3432.6171875	3.29515603575206e-05	\\
3433.59375	3.28207258240002e-05	\\
3434.5703125	3.20191601413962e-05	\\
3435.546875	3.3068730618383e-05	\\
3436.5234375	3.36542829414412e-05	\\
3437.5	3.22859510041257e-05	\\
3438.4765625	3.19122013728678e-05	\\
3439.453125	3.28258998261889e-05	\\
3440.4296875	3.29404507093735e-05	\\
3441.40625	3.18938129609416e-05	\\
3442.3828125	3.19452926686306e-05	\\
3443.359375	3.11945604715771e-05	\\
3444.3359375	3.12588594035747e-05	\\
3445.3125	3.20927836653802e-05	\\
3446.2890625	3.20682573441534e-05	\\
3447.265625	3.10979715388824e-05	\\
3448.2421875	3.09086027888059e-05	\\
3449.21875	3.13967195180296e-05	\\
3450.1953125	3.2176110987613e-05	\\
3451.171875	3.23805194320225e-05	\\
3452.1484375	3.1309345951561e-05	\\
3453.125	3.2031485113907e-05	\\
3454.1015625	3.09099986077345e-05	\\
3455.078125	3.06562882044384e-05	\\
3456.0546875	3.12156064975916e-05	\\
3457.03125	3.1474710268274e-05	\\
3458.0078125	3.15007411011233e-05	\\
3458.984375	3.12222573389485e-05	\\
3459.9609375	3.2896556233663e-05	\\
3460.9375	3.10410771208765e-05	\\
3461.9140625	3.03956226860968e-05	\\
3462.890625	3.1871437166885e-05	\\
3463.8671875	3.06176906965113e-05	\\
3464.84375	3.14400094168104e-05	\\
3465.8203125	3.15581267346153e-05	\\
3466.796875	2.90918657881607e-05	\\
3467.7734375	3.09333273593496e-05	\\
3468.75	3.12258881427906e-05	\\
3469.7265625	3.09588297758646e-05	\\
3470.703125	2.99618737958329e-05	\\
3471.6796875	2.90426000350295e-05	\\
3472.65625	3.07855962580568e-05	\\
3473.6328125	2.99047030063258e-05	\\
3474.609375	3.09616735734911e-05	\\
3475.5859375	2.98724823857413e-05	\\
3476.5625	3.09543430485182e-05	\\
3477.5390625	3.03713525465576e-05	\\
3478.515625	3.08786101040272e-05	\\
3479.4921875	2.91317456948031e-05	\\
3480.46875	2.96444729318815e-05	\\
3481.4453125	3.07409344654108e-05	\\
3482.421875	2.91818314267942e-05	\\
3483.3984375	2.86937636249691e-05	\\
3484.375	3.01963214767789e-05	\\
3485.3515625	2.99371736301712e-05	\\
3486.328125	2.91187471653898e-05	\\
3487.3046875	2.88935015925713e-05	\\
3488.28125	2.90406044693606e-05	\\
3489.2578125	2.95545037157273e-05	\\
3490.234375	3.09016848455933e-05	\\
3491.2109375	2.98527021962979e-05	\\
3492.1875	3.09401201060462e-05	\\
3493.1640625	2.98311018598804e-05	\\
3494.140625	3.02090831264966e-05	\\
3495.1171875	3.10417113723058e-05	\\
3496.09375	2.87187240487806e-05	\\
3497.0703125	2.91324219057816e-05	\\
3498.046875	2.99556210871412e-05	\\
3499.0234375	3.00603302912293e-05	\\
3500	3.02937736748853e-05	\\
3500.9765625	3.04350926240595e-05	\\
3501.953125	2.78195637348859e-05	\\
3502.9296875	2.93592471067734e-05	\\
3503.90625	2.8905796528729e-05	\\
3504.8828125	2.99858628950313e-05	\\
3505.859375	2.81530440219582e-05	\\
3506.8359375	2.78861389363427e-05	\\
3507.8125	2.927735799936e-05	\\
3508.7890625	2.84315151869734e-05	\\
3509.765625	2.93273144060286e-05	\\
3510.7421875	2.80564641286628e-05	\\
3511.71875	2.95139970816127e-05	\\
3512.6953125	2.91786150653023e-05	\\
3513.671875	2.7312659054098e-05	\\
3514.6484375	2.82935290275875e-05	\\
3515.625	2.81470083529714e-05	\\
3516.6015625	2.70701095337557e-05	\\
3517.578125	2.81682927974711e-05	\\
3518.5546875	2.74214979635601e-05	\\
3519.53125	2.73815933378082e-05	\\
3520.5078125	2.91211967285981e-05	\\
3521.484375	2.61731873148945e-05	\\
3522.4609375	2.88376249176902e-05	\\
3523.4375	2.65978183703822e-05	\\
3524.4140625	2.74886596317046e-05	\\
3525.390625	2.79317142243568e-05	\\
3526.3671875	2.75140587197064e-05	\\
3527.34375	2.73617793693931e-05	\\
3528.3203125	2.74641680959581e-05	\\
3529.296875	2.70038110916542e-05	\\
3530.2734375	2.7060107471281e-05	\\
3531.25	2.63042577801468e-05	\\
3532.2265625	2.65957171436006e-05	\\
3533.203125	2.71396039715882e-05	\\
3534.1796875	2.78667568739484e-05	\\
3535.15625	2.71948986342349e-05	\\
3536.1328125	2.66979942972716e-05	\\
3537.109375	2.66583055201011e-05	\\
3538.0859375	2.64540159538362e-05	\\
3539.0625	2.64482811703165e-05	\\
3540.0390625	2.65564757689918e-05	\\
3541.015625	2.65178359163938e-05	\\
3541.9921875	2.66587104685725e-05	\\
3542.96875	2.6797238261703e-05	\\
3543.9453125	2.73258132418663e-05	\\
3544.921875	2.73170911070603e-05	\\
3545.8984375	2.62334311759134e-05	\\
3546.875	2.68073698640289e-05	\\
3547.8515625	2.71635037408651e-05	\\
3548.828125	2.64016588449196e-05	\\
3549.8046875	2.6114677162026e-05	\\
3550.78125	2.7746275581636e-05	\\
3551.7578125	2.61992763498294e-05	\\
3552.734375	2.56785523077602e-05	\\
3553.7109375	2.76023709425052e-05	\\
3554.6875	2.64831721872768e-05	\\
3555.6640625	2.63571305006605e-05	\\
3556.640625	2.5893278256248e-05	\\
3557.6171875	2.58718204258773e-05	\\
3558.59375	2.75796858157951e-05	\\
3559.5703125	2.66933030173061e-05	\\
3560.546875	2.69215744933778e-05	\\
3561.5234375	2.68158692782394e-05	\\
3562.5	2.58157227071068e-05	\\
3563.4765625	2.57954458101508e-05	\\
3564.453125	2.63505008769061e-05	\\
3565.4296875	2.55173040968558e-05	\\
3566.40625	2.5867928907515e-05	\\
3567.3828125	2.55218238248818e-05	\\
3568.359375	2.55099518116227e-05	\\
3569.3359375	2.69497708921976e-05	\\
3570.3125	2.56415906419501e-05	\\
3571.2890625	2.55093788101814e-05	\\
3572.265625	2.54830594644298e-05	\\
3573.2421875	2.61606935791299e-05	\\
3574.21875	2.62184935069361e-05	\\
3575.1953125	2.56954985665783e-05	\\
3576.171875	2.52133543979368e-05	\\
3577.1484375	2.53334540061946e-05	\\
3578.125	2.58749563094018e-05	\\
3579.1015625	2.52067587952385e-05	\\
3580.078125	2.49896024534018e-05	\\
3581.0546875	2.49622729750087e-05	\\
3582.03125	2.40103000094607e-05	\\
3583.0078125	2.41048984050123e-05	\\
3583.984375	2.51254294360023e-05	\\
3584.9609375	2.60303629244257e-05	\\
3585.9375	2.49738259378067e-05	\\
3586.9140625	2.45424543955302e-05	\\
3587.890625	2.45330787458609e-05	\\
3588.8671875	2.41162126874424e-05	\\
3589.84375	2.35220493645642e-05	\\
3590.8203125	2.35775216069152e-05	\\
3591.796875	2.50276146134867e-05	\\
3592.7734375	2.40397200591087e-05	\\
3593.75	2.45873305022592e-05	\\
3594.7265625	2.39624064907858e-05	\\
3595.703125	2.46697438324565e-05	\\
3596.6796875	2.35509122725122e-05	\\
3597.65625	2.4469171839236e-05	\\
3598.6328125	2.2702256206397e-05	\\
3599.609375	2.31971522733876e-05	\\
3600.5859375	2.33105914746023e-05	\\
3601.5625	2.29098731053432e-05	\\
3602.5390625	2.31843762343539e-05	\\
3603.515625	2.16701198580352e-05	\\
3604.4921875	2.15804499845683e-05	\\
3605.46875	2.29016255229517e-05	\\
3606.4453125	2.1333002467026e-05	\\
3607.421875	2.20831054545344e-05	\\
3608.3984375	2.26543197490855e-05	\\
3609.375	2.26410176576813e-05	\\
3610.3515625	2.31173922021997e-05	\\
3611.328125	2.17299717259297e-05	\\
3612.3046875	2.1766368214898e-05	\\
3613.28125	2.16681025702148e-05	\\
3614.2578125	2.07283185698832e-05	\\
3615.234375	2.08041381381829e-05	\\
3616.2109375	2.10846913421068e-05	\\
3617.1875	2.09908394831931e-05	\\
3618.1640625	1.99239538998269e-05	\\
3619.140625	2.10715030195758e-05	\\
3620.1171875	2.05391785204725e-05	\\
3621.09375	1.89530623093002e-05	\\
3622.0703125	2.09235604478151e-05	\\
3623.046875	2.04514926270164e-05	\\
3624.0234375	1.94518998401709e-05	\\
3625	1.97095899286427e-05	\\
3625.9765625	2.08675698154606e-05	\\
3626.953125	1.99788148104188e-05	\\
3627.9296875	1.94130303792687e-05	\\
3628.90625	1.94769419573787e-05	\\
3629.8828125	1.91270905874751e-05	\\
3630.859375	1.92261160130056e-05	\\
3631.8359375	1.89491518416333e-05	\\
3632.8125	2.0552076846031e-05	\\
3633.7890625	1.89264634106778e-05	\\
3634.765625	1.90324808329351e-05	\\
3635.7421875	1.86070233100803e-05	\\
3636.71875	1.991280883289e-05	\\
3637.6953125	1.92415162738866e-05	\\
3638.671875	1.9094462840902e-05	\\
3639.6484375	1.8682033425755e-05	\\
3640.625	1.99659800137242e-05	\\
3641.6015625	1.92790143581935e-05	\\
3642.578125	1.85542792018801e-05	\\
3643.5546875	1.90465479597217e-05	\\
3644.53125	1.90166312167296e-05	\\
3645.5078125	1.88817244121111e-05	\\
3646.484375	1.81128853475408e-05	\\
3647.4609375	1.8810583069891e-05	\\
3648.4375	1.83512699965168e-05	\\
3649.4140625	1.68485558511174e-05	\\
3650.390625	1.76914095134893e-05	\\
3651.3671875	1.81448091794705e-05	\\
3652.34375	1.82093777386316e-05	\\
3653.3203125	1.78395989950831e-05	\\
3654.296875	1.79505161713224e-05	\\
3655.2734375	1.78538177790379e-05	\\
3656.25	1.72827495428591e-05	\\
3657.2265625	1.78906094402472e-05	\\
3658.203125	1.74616773381305e-05	\\
3659.1796875	1.82739581668763e-05	\\
3660.15625	1.7565860808368e-05	\\
3661.1328125	1.67590538802405e-05	\\
3662.109375	1.69407732706698e-05	\\
3663.0859375	1.68896196493805e-05	\\
3664.0625	1.75438434282901e-05	\\
3665.0390625	1.65924292999244e-05	\\
3666.015625	1.6073615226878e-05	\\
3666.9921875	1.65521879112179e-05	\\
3667.96875	1.61505750825708e-05	\\
3668.9453125	1.56632719585299e-05	\\
3669.921875	1.65000650713951e-05	\\
3670.8984375	1.60739043579428e-05	\\
3671.875	1.6031373268975e-05	\\
3672.8515625	1.61465395729117e-05	\\
3673.828125	1.65216626438639e-05	\\
3674.8046875	1.55936045546827e-05	\\
3675.78125	1.55639790531253e-05	\\
3676.7578125	1.61483841668241e-05	\\
3677.734375	1.57898969987643e-05	\\
3678.7109375	1.59354624257141e-05	\\
3679.6875	1.67708510136604e-05	\\
3680.6640625	1.57354665554279e-05	\\
3681.640625	1.6223536585491e-05	\\
3682.6171875	1.60547872417792e-05	\\
3683.59375	1.59857565662551e-05	\\
3684.5703125	1.58442263990683e-05	\\
3685.546875	1.67309510831212e-05	\\
3686.5234375	1.54843850912926e-05	\\
3687.5	1.54690645874093e-05	\\
3688.4765625	1.62473864370582e-05	\\
3689.453125	1.5648893440328e-05	\\
3690.4296875	1.57471408597997e-05	\\
3691.40625	1.58715532954513e-05	\\
3692.3828125	1.51421350542422e-05	\\
3693.359375	1.55292881961839e-05	\\
3694.3359375	1.62247825889327e-05	\\
3695.3125	1.55856003595471e-05	\\
3696.2890625	1.56409035562468e-05	\\
3697.265625	1.62406770429405e-05	\\
3698.2421875	1.59635348361142e-05	\\
3699.21875	1.45606589613319e-05	\\
3700.1953125	1.46018538206819e-05	\\
3701.171875	1.49015116330575e-05	\\
3702.1484375	1.5372965357479e-05	\\
3703.125	1.52068572224087e-05	\\
3704.1015625	1.58791424818282e-05	\\
3705.078125	1.58664621994087e-05	\\
3706.0546875	1.52787544993479e-05	\\
3707.03125	1.47363149373438e-05	\\
3708.0078125	1.54480417145876e-05	\\
3708.984375	1.62616753824341e-05	\\
3709.9609375	1.57022138822951e-05	\\
3710.9375	1.64665120467083e-05	\\
3711.9140625	1.5798655246915e-05	\\
3712.890625	1.59026586851511e-05	\\
3713.8671875	1.54247034786513e-05	\\
3714.84375	1.48398416443675e-05	\\
3715.8203125	1.5163429906126e-05	\\
3716.796875	1.59368979365893e-05	\\
3717.7734375	1.6264971769953e-05	\\
3718.75	1.63835123014987e-05	\\
3719.7265625	1.54002033734685e-05	\\
3720.703125	1.66872509947263e-05	\\
3721.6796875	1.5553646150943e-05	\\
3722.65625	1.58468357455151e-05	\\
3723.6328125	1.54751959710771e-05	\\
3724.609375	1.63458712714612e-05	\\
3725.5859375	1.72180997708545e-05	\\
3726.5625	1.67499111114283e-05	\\
3727.5390625	1.71001247929897e-05	\\
3728.515625	1.8214050273521e-05	\\
3729.4921875	1.70372521977564e-05	\\
3730.46875	1.69329157968823e-05	\\
3731.4453125	1.67416472742208e-05	\\
3732.421875	1.79508988053692e-05	\\
3733.3984375	1.73064354686227e-05	\\
3734.375	1.69666568828488e-05	\\
3735.3515625	1.79538397791005e-05	\\
3736.328125	1.79400744260884e-05	\\
3737.3046875	1.59225035234054e-05	\\
3738.28125	1.66782148678204e-05	\\
3739.2578125	1.73722719067167e-05	\\
3740.234375	1.75652351615911e-05	\\
3741.2109375	1.69494673627626e-05	\\
3742.1875	1.77864384470882e-05	\\
3743.1640625	1.80349910300447e-05	\\
3744.140625	1.83259695878112e-05	\\
3745.1171875	1.78937609722655e-05	\\
3746.09375	1.82261524092195e-05	\\
3747.0703125	1.74122605703016e-05	\\
3748.046875	1.83774376218965e-05	\\
3749.0234375	1.80897001238412e-05	\\
3750	1.84826925917205e-05	\\
3750.9765625	1.78862954656723e-05	\\
3751.953125	1.81297976955236e-05	\\
3752.9296875	1.83359071708242e-05	\\
3753.90625	1.86097572048343e-05	\\
3754.8828125	1.81817082665137e-05	\\
3755.859375	1.80485095987372e-05	\\
3756.8359375	1.91306884915027e-05	\\
3757.8125	1.75856367111023e-05	\\
3758.7890625	1.71048868876176e-05	\\
3759.765625	1.83978276522438e-05	\\
3760.7421875	1.73346401064484e-05	\\
3761.71875	1.77969423778119e-05	\\
3762.6953125	1.70917964803589e-05	\\
3763.671875	1.83584190016792e-05	\\
3764.6484375	1.70594368088274e-05	\\
3765.625	1.8273802906514e-05	\\
3766.6015625	1.78857053724059e-05	\\
3767.578125	1.76390983247498e-05	\\
3768.5546875	1.67126267332047e-05	\\
3769.53125	1.74224631903819e-05	\\
3770.5078125	1.7591285532191e-05	\\
3771.484375	1.81553544232064e-05	\\
3772.4609375	1.7632879199689e-05	\\
3773.4375	1.75278494490078e-05	\\
3774.4140625	1.75443335501847e-05	\\
3775.390625	1.73666258438689e-05	\\
3776.3671875	1.73066886120989e-05	\\
3777.34375	1.74023113840408e-05	\\
3778.3203125	1.78605197577383e-05	\\
3779.296875	1.69111636824605e-05	\\
3780.2734375	1.73992363410533e-05	\\
3781.25	1.72854276605271e-05	\\
3782.2265625	1.69820776983899e-05	\\
3783.203125	1.66925857128778e-05	\\
3784.1796875	1.62051962762503e-05	\\
3785.15625	1.68150059973566e-05	\\
3786.1328125	1.63464493814223e-05	\\
3787.109375	1.70524984939564e-05	\\
3788.0859375	1.65681565193517e-05	\\
3789.0625	1.67369525023344e-05	\\
3790.0390625	1.70038159859729e-05	\\
3791.015625	1.66991514012908e-05	\\
3791.9921875	1.63890907893736e-05	\\
3792.96875	1.60017361699082e-05	\\
3793.9453125	1.59787985499946e-05	\\
3794.921875	1.65349419997366e-05	\\
3795.8984375	1.66914030156505e-05	\\
3796.875	1.64727005523166e-05	\\
3797.8515625	1.61648999479292e-05	\\
3798.828125	1.619906602485e-05	\\
3799.8046875	1.5777656178509e-05	\\
3800.78125	1.58374101995921e-05	\\
3801.7578125	1.54243398052913e-05	\\
3802.734375	1.56647721932493e-05	\\
3803.7109375	1.52087294880185e-05	\\
3804.6875	1.51614221926575e-05	\\
3805.6640625	1.45744869980002e-05	\\
3806.640625	1.48593266743384e-05	\\
3807.6171875	1.45732835678268e-05	\\
3808.59375	1.52843722290903e-05	\\
3809.5703125	1.53579855224387e-05	\\
3810.546875	1.4791180421814e-05	\\
3811.5234375	1.44308260961965e-05	\\
3812.5	1.44961309509478e-05	\\
};
\addplot [color=blue,solid,forget plot]
  table[row sep=crcr]{
3812.5	1.44961309509478e-05	\\
3813.4765625	1.45995423846877e-05	\\
3814.453125	1.52806693009967e-05	\\
3815.4296875	1.43673968312625e-05	\\
3816.40625	1.41780084207187e-05	\\
3817.3828125	1.41057711282213e-05	\\
3818.359375	1.43720995974198e-05	\\
3819.3359375	1.4982409530077e-05	\\
3820.3125	1.46065760278308e-05	\\
3821.2890625	1.4678019053358e-05	\\
3822.265625	1.48413952293771e-05	\\
3823.2421875	1.49038592576817e-05	\\
3824.21875	1.47360741713597e-05	\\
3825.1953125	1.49183445734569e-05	\\
3826.171875	1.47735715698763e-05	\\
3827.1484375	1.42421163236669e-05	\\
3828.125	1.47881201771511e-05	\\
3829.1015625	1.48647051267393e-05	\\
3830.078125	1.40044029700268e-05	\\
3831.0546875	1.41993689530561e-05	\\
3832.03125	1.48242324169661e-05	\\
3833.0078125	1.4092378478042e-05	\\
3833.984375	1.44694548330263e-05	\\
3834.9609375	1.41775551868059e-05	\\
3835.9375	1.47534161916216e-05	\\
3836.9140625	1.54338338974143e-05	\\
3837.890625	1.40872427186241e-05	\\
3838.8671875	1.41846125219445e-05	\\
3839.84375	1.46549209590877e-05	\\
3840.8203125	1.43416048390741e-05	\\
3841.796875	1.40255033348193e-05	\\
3842.7734375	1.45631376998699e-05	\\
3843.75	1.45945595683275e-05	\\
3844.7265625	1.45851203682855e-05	\\
3845.703125	1.39043763660006e-05	\\
3846.6796875	1.42875173508222e-05	\\
3847.65625	1.45517833477141e-05	\\
3848.6328125	1.40110170065707e-05	\\
3849.609375	1.51768421079337e-05	\\
3850.5859375	1.48332271314152e-05	\\
3851.5625	1.48240065888495e-05	\\
3852.5390625	1.46639559477574e-05	\\
3853.515625	1.47861255310487e-05	\\
3854.4921875	1.48071105905975e-05	\\
3855.46875	1.52784288960373e-05	\\
3856.4453125	1.37902910224442e-05	\\
3857.421875	1.43390246884787e-05	\\
3858.3984375	1.45973148013151e-05	\\
3859.375	1.53311945645483e-05	\\
3860.3515625	1.45454030345069e-05	\\
3861.328125	1.46477867799091e-05	\\
3862.3046875	1.47005965103907e-05	\\
3863.28125	1.45403002384769e-05	\\
3864.2578125	1.4634258777616e-05	\\
3865.234375	1.4853055401043e-05	\\
3866.2109375	1.43644827166438e-05	\\
3867.1875	1.48287817404441e-05	\\
3868.1640625	1.38590998195222e-05	\\
3869.140625	1.51907876105855e-05	\\
3870.1171875	1.47102174561352e-05	\\
3871.09375	1.45768203687799e-05	\\
3872.0703125	1.47274208634802e-05	\\
3873.046875	1.45296561727935e-05	\\
3874.0234375	1.44283977298115e-05	\\
3875	1.43920400964069e-05	\\
3875.9765625	1.46403291764581e-05	\\
3876.953125	1.39928898079254e-05	\\
3877.9296875	1.46368501655168e-05	\\
3878.90625	1.4141675393237e-05	\\
3879.8828125	1.41072803857916e-05	\\
3880.859375	1.40681740251664e-05	\\
3881.8359375	1.3848347247264e-05	\\
3882.8125	1.47188433273299e-05	\\
3883.7890625	1.4832953376519e-05	\\
3884.765625	1.41505801045426e-05	\\
3885.7421875	1.3956722586474e-05	\\
3886.71875	1.4846588732479e-05	\\
3887.6953125	1.42201962380272e-05	\\
3888.671875	1.44087801878623e-05	\\
3889.6484375	1.45868830862842e-05	\\
3890.625	1.47449330962771e-05	\\
3891.6015625	1.45608276530717e-05	\\
3892.578125	1.41029257858097e-05	\\
3893.5546875	1.47143368265505e-05	\\
3894.53125	1.46089595920218e-05	\\
3895.5078125	1.45229819341497e-05	\\
3896.484375	1.5587412016138e-05	\\
3897.4609375	1.47575806733977e-05	\\
3898.4375	1.46912704840711e-05	\\
3899.4140625	1.41568561521075e-05	\\
3900.390625	1.36235747890921e-05	\\
3901.3671875	1.49955903702575e-05	\\
3902.34375	1.46106347102387e-05	\\
3903.3203125	1.45290883849308e-05	\\
3904.296875	1.4964528218545e-05	\\
3905.2734375	1.45475374739771e-05	\\
3906.25	1.49014861047284e-05	\\
3907.2265625	1.48835357727445e-05	\\
3908.203125	1.48925236710851e-05	\\
3909.1796875	1.49260118498315e-05	\\
3910.15625	1.45203782861644e-05	\\
3911.1328125	1.49279332248409e-05	\\
3912.109375	1.45857472005143e-05	\\
3913.0859375	1.51501445488737e-05	\\
3914.0625	1.50257731821823e-05	\\
3915.0390625	1.47398211155585e-05	\\
3916.015625	1.47587456572094e-05	\\
3916.9921875	1.52378970504707e-05	\\
3917.96875	1.45914797970047e-05	\\
3918.9453125	1.53505978339217e-05	\\
3919.921875	1.46595223258978e-05	\\
3920.8984375	1.55240419969502e-05	\\
3921.875	1.52622355301139e-05	\\
3922.8515625	1.53580446956709e-05	\\
3923.828125	1.55263555008389e-05	\\
3924.8046875	1.51649357841723e-05	\\
3925.78125	1.52913681950271e-05	\\
3926.7578125	1.50805706003963e-05	\\
3927.734375	1.56526893279337e-05	\\
3928.7109375	1.46832510324192e-05	\\
3929.6875	1.59432821941408e-05	\\
3930.6640625	1.53903506122711e-05	\\
3931.640625	1.55494209809792e-05	\\
3932.6171875	1.4960068532171e-05	\\
3933.59375	1.51176180477319e-05	\\
3934.5703125	1.54127011283834e-05	\\
3935.546875	1.46784994572223e-05	\\
3936.5234375	1.48181296245962e-05	\\
3937.5	1.55197614830582e-05	\\
3938.4765625	1.5259149740604e-05	\\
3939.453125	1.51869644438568e-05	\\
3940.4296875	1.54523023540047e-05	\\
3941.40625	1.57253757068203e-05	\\
3942.3828125	1.53210188016033e-05	\\
3943.359375	1.56429390465114e-05	\\
3944.3359375	1.55457064695777e-05	\\
3945.3125	1.52117117417096e-05	\\
3946.2890625	1.56792580910352e-05	\\
3947.265625	1.51851790334971e-05	\\
3948.2421875	1.60022791070149e-05	\\
3949.21875	1.54227007443591e-05	\\
3950.1953125	1.51422024564678e-05	\\
3951.171875	1.56664179110757e-05	\\
3952.1484375	1.58216141556451e-05	\\
3953.125	1.51398980182307e-05	\\
3954.1015625	1.56674481467462e-05	\\
3955.078125	1.55266135114261e-05	\\
3956.0546875	1.56637505862104e-05	\\
3957.03125	1.60403023410598e-05	\\
3958.0078125	1.59063167757275e-05	\\
3958.984375	1.48993093292318e-05	\\
3959.9609375	1.54290549798445e-05	\\
3960.9375	1.53383420174398e-05	\\
3961.9140625	1.56836482955388e-05	\\
3962.890625	1.5329246634832e-05	\\
3963.8671875	1.54235879885826e-05	\\
3964.84375	1.53551013917045e-05	\\
3965.8203125	1.54288034706621e-05	\\
3966.796875	1.60921723233144e-05	\\
3967.7734375	1.59322973494521e-05	\\
3968.75	1.55713840785001e-05	\\
3969.7265625	1.61834515882777e-05	\\
3970.703125	1.57123304988777e-05	\\
3971.6796875	1.53558862966417e-05	\\
3972.65625	1.57030884666649e-05	\\
3973.6328125	1.59273435492255e-05	\\
3974.609375	1.62258501509834e-05	\\
3975.5859375	1.56190260831568e-05	\\
3976.5625	1.60109098625366e-05	\\
3977.5390625	1.52848124566208e-05	\\
3978.515625	1.60694910634086e-05	\\
3979.4921875	1.61050179557507e-05	\\
3980.46875	1.58365279708055e-05	\\
3981.4453125	1.60644228550006e-05	\\
3982.421875	1.58135456543504e-05	\\
3983.3984375	1.55170345691891e-05	\\
3984.375	1.62334572942354e-05	\\
3985.3515625	1.58517129173588e-05	\\
3986.328125	1.62119008751163e-05	\\
3987.3046875	1.60630195794299e-05	\\
3988.28125	1.58765131575191e-05	\\
3989.2578125	1.5673738949017e-05	\\
3990.234375	1.58142377926051e-05	\\
3991.2109375	1.57738593640657e-05	\\
3992.1875	1.5349359519776e-05	\\
3993.1640625	1.61744263820056e-05	\\
3994.140625	1.57883887109514e-05	\\
3995.1171875	1.59158681212455e-05	\\
3996.09375	1.57969981340263e-05	\\
3997.0703125	1.5892931745498e-05	\\
3998.046875	1.54458246817485e-05	\\
3999.0234375	1.56451426923876e-05	\\
};
\end{axis}
\end{tikzpicture}%
	\caption{Impulse response at Fs\_TX = 22.050 Hz and Fs\_RX = 8.000 Hz.}
	\label{fig:response_3}
\end{figure}

\begin{figure}[H]
	\centering
	\setlength\figureheight{4cm}
    	\setlength\figurewidth{0.8\linewidth}
	% This file was created by matlab2tikz v0.4.6 running on MATLAB 8.2.
% Copyright (c) 2008--2014, Nico Schlömer <nico.schloemer@gmail.com>
% All rights reserved.
% Minimal pgfplots version: 1.3
% 
% The latest updates can be retrieved from
%   http://www.mathworks.com/matlabcentral/fileexchange/22022-matlab2tikz
% where you can also make suggestions and rate matlab2tikz.
% 
\begin{tikzpicture}

\begin{axis}[%
width=\figurewidth,
height=\figureheight,
scale only axis,
xmin=0,
xmax=1,
ymin=-0.2,
ymax=0.2,
name=plot1
]
\addplot [color=blue,solid,forget plot]
  table[row sep=crcr]{
0	0.000152587890625	\\
4.43911750344032e-05	-0.000244140625	\\
8.87823500688063e-05	0.001434326171875	\\
0.000133173525103209	0.001007080078125	\\
0.000177564700137613	0.000823974609375	\\
0.000221955875172016	0.0010986328125	\\
0.000266347050206419	0.001068115234375	\\
0.000310738225240822	0.000579833984375	\\
0.000355129400275225	0.0009765625	\\
0.000399520575309628	0.0008544921875	\\
0.000443911750344032	0.001220703125	\\
0.000488302925378435	0.0010986328125	\\
0.000532694100412838	0.001007080078125	\\
0.000577085275447241	0.00177001953125	\\
0.000621476450481644	0.001434326171875	\\
0.000665867625516047	0.001617431640625	\\
0.000710258800550451	0.001800537109375	\\
0.000754649975584854	0.00152587890625	\\
0.000799041150619257	0.001617431640625	\\
0.00084343232565366	0.001708984375	\\
0.000887823500688063	0.001434326171875	\\
0.000932214675722466	0.001861572265625	\\
0.000976605850756869	0.002105712890625	\\
0.00102099702579127	0.00238037109375	\\
0.00106538820082568	0.002593994140625	\\
0.00110977937586008	0.002166748046875	\\
0.00115417055089448	0.002960205078125	\\
0.00119856172592889	0.003387451171875	\\
0.00124295290096329	0.003021240234375	\\
0.00128734407599769	0.00286865234375	\\
0.00133173525103209	0.0029296875	\\
0.0013761264260665	0.002838134765625	\\
0.0014205176011009	0.002960205078125	\\
0.0014649087761353	0.0030517578125	\\
0.00150929995116971	0.00286865234375	\\
0.00155369112620411	0.00286865234375	\\
0.00159808230123851	0.00286865234375	\\
0.00164247347627292	0.003204345703125	\\
0.00168686465130732	0.003448486328125	\\
0.00173125582634172	0.003387451171875	\\
0.00177564700137613	0.003173828125	\\
0.00182003817641053	0.00286865234375	\\
0.00186442935144493	0.002838134765625	\\
0.00190882052647934	0.003082275390625	\\
0.00195321170151374	0.003509521484375	\\
0.00199760287654814	0.003265380859375	\\
0.00204199405158255	0.003204345703125	\\
0.00208638522661695	0.00335693359375	\\
0.00213077640165135	0.0032958984375	\\
0.00217516757668575	0.003265380859375	\\
0.00221955875172016	0.00341796875	\\
0.00226394992675456	0.003326416015625	\\
0.00230834110178896	0.00347900390625	\\
0.00235273227682337	0.003265380859375	\\
0.00239712345185777	0.00262451171875	\\
0.00244151462689217	0.002777099609375	\\
0.00248590580192658	0.0025634765625	\\
0.00253029697696098	0.00250244140625	\\
0.00257468815199538	0.0028076171875	\\
0.00261907932702979	0.002471923828125	\\
0.00266347050206419	0.00244140625	\\
0.00270786167709859	0.0025634765625	\\
0.002752252852133	0.002471923828125	\\
0.0027966440271674	0.002471923828125	\\
0.0028410352022018	0.002685546875	\\
0.00288542637723621	0.002593994140625	\\
0.00292981755227061	0.00311279296875	\\
0.00297420872730501	0.00311279296875	\\
0.00301859990233941	0.00262451171875	\\
0.00306299107737382	0.002532958984375	\\
0.00310738225240822	0.00274658203125	\\
0.00315177342744262	0.003173828125	\\
0.00319616460247703	0.00299072265625	\\
0.00324055577751143	0.002899169921875	\\
0.00328494695254583	0.002655029296875	\\
0.00332933812758024	0.00286865234375	\\
0.00337372930261464	0.003021240234375	\\
0.00341812047764904	0.002838134765625	\\
0.00346251165268345	0.0028076171875	\\
0.00350690282771785	0.0030517578125	\\
0.00355129400275225	0.0030517578125	\\
0.00359568517778666	0.00262451171875	\\
0.00364007635282106	0.00238037109375	\\
0.00368446752785546	0.0023193359375	\\
0.00372885870288987	0.001739501953125	\\
0.00377324987792427	0.001708984375	\\
0.00381764105295867	0.001922607421875	\\
0.00386203222799308	0.001983642578125	\\
0.00390642340302748	0.0018310546875	\\
0.00395081457806188	0.0013427734375	\\
0.00399520575309628	0.001556396484375	\\
0.00403959692813069	0.00140380859375	\\
0.00408398810316509	0.0010986328125	\\
0.00412837927819949	0.00128173828125	\\
0.0041727704532339	0.001129150390625	\\
0.0042171616282683	0.001373291015625	\\
0.0042615528033027	0.001373291015625	\\
0.00430594397833711	0.001251220703125	\\
0.00435033515337151	0.001190185546875	\\
0.00439472632840591	0.001007080078125	\\
0.00443911750344032	0.000732421875	\\
0.00448350867847472	0.000701904296875	\\
0.00452789985350912	0.00079345703125	\\
0.00457229102854353	0.0006103515625	\\
0.00461668220357793	0.00048828125	\\
0.00466107337861233	0.000457763671875	\\
0.00470546455364673	0.000701904296875	\\
0.00474985572868114	0.00067138671875	\\
0.00479424690371554	0.00067138671875	\\
0.00483863807874994	0.000579833984375	\\
0.00488302925378435	0.0003662109375	\\
0.00492742042881875	0.00048828125	\\
0.00497181160385315	0.000396728515625	\\
0.00501620277888756	0.000244140625	\\
0.00506059395392196	0.000457763671875	\\
0.00510498512895636	0.000396728515625	\\
0.00514937630399077	0.000457763671875	\\
0.00519376747902517	0.00054931640625	\\
0.00523815865405957	0.000701904296875	\\
0.00528254982909398	0.000579833984375	\\
0.00532694100412838	0.000885009765625	\\
0.00537133217916278	0.00103759765625	\\
0.00541572335419719	0.00079345703125	\\
0.00546011452923159	0.00115966796875	\\
0.00550450570426599	0.001129150390625	\\
0.00554889687930039	0.00115966796875	\\
0.0055932880543348	0.001068115234375	\\
0.0056376792293692	0.0009765625	\\
0.0056820704044036	0.000762939453125	\\
0.00572646157943801	0.00079345703125	\\
0.00577085275447241	0.00091552734375	\\
0.00581524392950681	0.000640869140625	\\
0.00585963510454122	0.00048828125	\\
0.00590402627957562	0.00048828125	\\
0.00594841745461002	0.00042724609375	\\
0.00599280862964443	0.0001220703125	\\
0.00603719980467883	-0.0001220703125	\\
0.00608159097971323	3.0517578125e-05	\\
0.00612598215474764	0.0001220703125	\\
0.00617037332978204	9.1552734375e-05	\\
0.00621476450481644	-0.000244140625	\\
0.00625915567985085	-0.0003662109375	\\
0.00630354685488525	-0.000213623046875	\\
0.00634793802991965	-0.00079345703125	\\
0.00639232920495406	-0.0010986328125	\\
0.00643672037998846	-0.001068115234375	\\
0.00648111155502286	-0.00103759765625	\\
0.00652550273005726	-0.001007080078125	\\
0.00656989390509167	-0.00103759765625	\\
0.00661428508012607	-0.000885009765625	\\
0.00665867625516047	-0.0010986328125	\\
0.00670306743019488	-0.00103759765625	\\
0.00674745860522928	-0.00103759765625	\\
0.00679184978026368	-0.001190185546875	\\
0.00683624095529809	-0.000946044921875	\\
0.00688063213033249	-0.001068115234375	\\
0.00692502330536689	-0.001373291015625	\\
0.0069694144804013	-0.001312255859375	\\
0.0070138056554357	-0.001129150390625	\\
0.0070581968304701	-0.000640869140625	\\
0.00710258800550451	-0.00018310546875	\\
0.00714697918053891	-0.000152587890625	\\
0.00719137035557331	-6.103515625e-05	\\
0.00723576153060772	-0.000213623046875	\\
0.00728015270564212	-0.000732421875	\\
0.00732454388067652	-0.000732421875	\\
0.00736893505571092	-0.00048828125	\\
0.00741332623074533	-0.000457763671875	\\
0.00745771740577973	-0.000152587890625	\\
0.00750210858081413	-3.0517578125e-05	\\
0.00754649975584854	3.0517578125e-05	\\
0.00759089093088294	-3.0517578125e-05	\\
0.00763528210591734	-0.000457763671875	\\
0.00767967328095175	3.0517578125e-05	\\
0.00772406445598615	0.0001220703125	\\
0.00776845563102055	-0.000244140625	\\
0.00781284680605496	0.000244140625	\\
0.00785723798108936	-0.000152587890625	\\
0.00790162915612376	-0.0006103515625	\\
0.00794602033115816	-0.000213623046875	\\
0.00799041150619257	-0.00030517578125	\\
0.00803480268122697	-0.000244140625	\\
0.00807919385626137	-0.000457763671875	\\
0.00812358503129578	-0.000457763671875	\\
0.00816797620633018	-0.000152587890625	\\
0.00821236738136458	-0.000274658203125	\\
0.00825675855639899	-0.000335693359375	\\
0.00830114973143339	-0.000213623046875	\\
0.00834554090646779	-6.103515625e-05	\\
0.0083899320815022	-0.000518798828125	\\
0.0084343232565366	-0.00054931640625	\\
0.008478714431571	-0.000213623046875	\\
0.00852310560660541	-0.000335693359375	\\
0.00856749678163981	-0.00030517578125	\\
0.00861188795667421	-0.00048828125	\\
0.00865627913170862	-0.000457763671875	\\
0.00870067030674302	-0.00018310546875	\\
0.00874506148177742	-0.0003662109375	\\
0.00878945265681183	-0.000152587890625	\\
0.00883384383184623	-0.000152587890625	\\
0.00887823500688063	-0.000335693359375	\\
0.00892262618191503	-0.000244140625	\\
0.00896701735694944	-0.000518798828125	\\
0.00901140853198384	-0.0001220703125	\\
0.00905579970701824	0.000396728515625	\\
0.00910019088205265	0.000244140625	\\
0.00914458205708705	-3.0517578125e-05	\\
0.00918897323212145	-0.000152587890625	\\
0.00923336440715586	9.1552734375e-05	\\
0.00927775558219026	0.00030517578125	\\
0.00932214675722466	0.00042724609375	\\
0.00936653793225907	0.00042724609375	\\
0.00941092910729347	0.000335693359375	\\
0.00945532028232787	0.000640869140625	\\
0.00949971145736228	0.00048828125	\\
0.00954410263239668	-6.103515625e-05	\\
0.00958849380743108	-3.0517578125e-05	\\
0.00963288498246549	0.00018310546875	\\
0.00967727615749989	-0.000152587890625	\\
0.00972166733253429	0.0003662109375	\\
0.00976605850756869	0.00054931640625	\\
0.0098104496826031	0.000335693359375	\\
0.0098548408576375	0.0001220703125	\\
0.0098992320326719	-0.000152587890625	\\
0.00994362320770631	6.103515625e-05	\\
0.00998801438274071	0.000274658203125	\\
0.0100324055577751	0.000274658203125	\\
0.0100767967328095	9.1552734375e-05	\\
0.0101211879078439	0	\\
0.0101655790828783	9.1552734375e-05	\\
0.0102099702579127	-3.0517578125e-05	\\
0.0102543614329471	-0.000152587890625	\\
0.0102987526079815	9.1552734375e-05	\\
0.0103431437830159	-0.0001220703125	\\
0.0103875349580503	-0.00030517578125	\\
0.0104319261330847	-0.000396728515625	\\
0.0104763173081191	-0.00042724609375	\\
0.0105207084831535	-0.0003662109375	\\
0.010565099658188	0	\\
0.0106094908332224	-3.0517578125e-05	\\
0.0106538820082568	-9.1552734375e-05	\\
0.0106982731832912	0.00048828125	\\
0.0107426643583256	0.00018310546875	\\
0.01078705553336	0.000335693359375	\\
0.0108314467083944	0.00054931640625	\\
0.0108758378834288	0.00048828125	\\
0.0109202290584632	0.000518798828125	\\
0.0109646202334976	0.000396728515625	\\
0.011009011408532	0.000396728515625	\\
0.0110534025835664	-0.000213623046875	\\
0.0110977937586008	0.000152587890625	\\
0.0111421849336352	0.000396728515625	\\
0.0111865761086696	3.0517578125e-05	\\
0.011230967283704	3.0517578125e-05	\\
0.0112753584587384	0.0001220703125	\\
0.0113197496337728	0.000396728515625	\\
0.0113641408088072	0.000274658203125	\\
0.0114085319838416	0.000396728515625	\\
0.011452923158876	0.000701904296875	\\
0.0114973143339104	0.0006103515625	\\
0.0115417055089448	0.0008544921875	\\
0.0115860966839792	0.00091552734375	\\
0.0116304878590136	0.0006103515625	\\
0.011674879034048	0.000701904296875	\\
0.0117192702090824	0.0008544921875	\\
0.0117636613841168	0.00091552734375	\\
0.0118080525591512	0.00067138671875	\\
0.0118524437341856	0.00091552734375	\\
0.01189683490922	0.00067138671875	\\
0.0119412260842545	0.0003662109375	\\
0.0119856172592889	0.000701904296875	\\
0.0120300084343233	0.000640869140625	\\
0.0120743996093577	0.0003662109375	\\
0.0121187907843921	0.001007080078125	\\
0.0121631819594265	0.001220703125	\\
0.0122075731344609	0.0010986328125	\\
0.0122519643094953	0.0009765625	\\
0.0122963554845297	0.0009765625	\\
0.0123407466595641	0.001007080078125	\\
0.0123851378345985	0.00103759765625	\\
0.0124295290096329	0.000701904296875	\\
0.0124739201846673	0.000579833984375	\\
0.0125183113597017	0.0008544921875	\\
0.0125627025347361	0.000823974609375	\\
0.0126070937097705	0.00103759765625	\\
0.0126514848848049	0.0010986328125	\\
0.0126958760598393	0.0013427734375	\\
0.0127402672348737	0.0010986328125	\\
0.0127846584099081	0.00128173828125	\\
0.0128290495849425	0.001495361328125	\\
0.0128734407599769	0.001190185546875	\\
0.0129178319350113	0.0015869140625	\\
0.0129622231100457	0.001129150390625	\\
0.0130066142850801	0.001129150390625	\\
0.0130510054601145	0.001495361328125	\\
0.0130953966351489	0.001373291015625	\\
0.0131397878101833	0.001251220703125	\\
0.0131841789852177	0.00146484375	\\
0.0132285701602521	0.001739501953125	\\
0.0132729613352865	0.00189208984375	\\
0.0133173525103209	0.00225830078125	\\
0.0133617436853554	0.00189208984375	\\
0.0134061348603898	0.00177001953125	\\
0.0134505260354242	0.00213623046875	\\
0.0134949172104586	0.0018310546875	\\
0.013539308385493	0.0013427734375	\\
0.0135836995605274	0.001068115234375	\\
0.0136280907355618	0.001129150390625	\\
0.0136724819105962	0.00146484375	\\
0.0137168730856306	0.00115966796875	\\
0.013761264260665	0.00146484375	\\
0.0138056554356994	0.0015869140625	\\
0.0138500466107338	0.00115966796875	\\
0.0138944377857682	0.00091552734375	\\
0.0139388289608026	0.00067138671875	\\
0.013983220135837	0.00030517578125	\\
0.0140276113108714	0.000274658203125	\\
0.0140720024859058	9.1552734375e-05	\\
0.0141163936609402	0.00042724609375	\\
0.0141607848359746	0.000274658203125	\\
0.014205176011009	0	\\
0.0142495671860434	0.000274658203125	\\
0.0142939583610778	-0.000244140625	\\
0.0143383495361122	-3.0517578125e-05	\\
0.0143827407111466	0.000244140625	\\
0.014427131886181	0.000213623046875	\\
0.0144715230612154	0.000152587890625	\\
0.0145159142362498	-0.000274658203125	\\
0.0145603054112842	0	\\
0.0146046965863186	0	\\
0.014649087761353	-0.0003662109375	\\
0.0146934789363874	-0.000213623046875	\\
0.0147378701114218	-6.103515625e-05	\\
0.0147822612864563	-0.0001220703125	\\
0.0148266524614907	0	\\
0.0148710436365251	0.00018310546875	\\
0.0149154348115595	-0.000152587890625	\\
0.0149598259865939	-0.0001220703125	\\
0.0150042171616283	-0.0003662109375	\\
0.0150486083366627	-0.000518798828125	\\
0.0150929995116971	0.000213623046875	\\
0.0151373906867315	0.000885009765625	\\
0.0151817818617659	0.0008544921875	\\
0.0152261730368003	0.000396728515625	\\
0.0152705642118347	0.0006103515625	\\
0.0153149553868691	0.000732421875	\\
0.0153593465619035	0.000579833984375	\\
0.0154037377369379	0.000457763671875	\\
0.0154481289119723	0.00018310546875	\\
0.0154925200870067	0.0003662109375	\\
0.0155369112620411	0.00018310546875	\\
0.0155813024370755	-3.0517578125e-05	\\
0.0156256936121099	0.00030517578125	\\
0.0156700847871443	0.00054931640625	\\
0.0157144759621787	0.00042724609375	\\
0.0157588671372131	0.000762939453125	\\
0.0158032583122475	0.001007080078125	\\
0.0158476494872819	0.000946044921875	\\
0.0158920406623163	0.000885009765625	\\
0.0159364318373507	0.000335693359375	\\
0.0159808230123851	0.000274658203125	\\
0.0160252141874195	0.00042724609375	\\
0.0160696053624539	0.000457763671875	\\
0.0161139965374883	3.0517578125e-05	\\
0.0161583877125227	-0.00030517578125	\\
0.0162027788875572	-0.000335693359375	\\
0.0162471700625916	0.000152587890625	\\
0.016291561237626	6.103515625e-05	\\
0.0163359524126604	-0.00018310546875	\\
0.0163803435876948	0	\\
0.0164247347627292	-0.00018310546875	\\
0.0164691259377636	-3.0517578125e-05	\\
0.016513517112798	0.000244140625	\\
0.0165579082878324	0.00018310546875	\\
0.0166022994628668	0.000518798828125	\\
0.0166466906379012	0.000823974609375	\\
0.0166910818129356	0.00079345703125	\\
0.01673547298797	0.0008544921875	\\
0.0167798641630044	0.00103759765625	\\
0.0168242553380388	0.001373291015625	\\
0.0168686465130732	0.0010986328125	\\
0.0169130376881076	0.001220703125	\\
0.016957428863142	0.001251220703125	\\
0.0170018200381764	0.001312255859375	\\
0.0170462112132108	0.00201416015625	\\
0.0170906023882452	0.002227783203125	\\
0.0171349935632796	0.00225830078125	\\
0.017179384738314	0.0023193359375	\\
0.0172237759133484	0.0025634765625	\\
0.0172681670883828	0.002349853515625	\\
0.0173125582634172	0.00201416015625	\\
0.0173569494384516	0.001953125	\\
0.017401340613486	0.001739501953125	\\
0.0174457317885204	0.001739501953125	\\
0.0174901229635548	0.001708984375	\\
0.0175345141385892	0.001617431640625	\\
0.0175789053136237	0.001953125	\\
0.0176232964886581	0.00238037109375	\\
0.0176676876636925	0.002197265625	\\
0.0177120788387269	0.001922607421875	\\
0.0177564700137613	0.001800537109375	\\
0.0178008611887957	0.001953125	\\
0.0178452523638301	0.002044677734375	\\
0.0178896435388645	0.00164794921875	\\
0.0179340347138989	0.0013427734375	\\
0.0179784258889333	0.001129150390625	\\
0.0180228170639677	0.0015869140625	\\
0.0180672082390021	0.0015869140625	\\
0.0181115994140365	0.001251220703125	\\
0.0181559905890709	0.001312255859375	\\
0.0182003817641053	0.00140380859375	\\
0.0182447729391397	0.001007080078125	\\
0.0182891641141741	0.001373291015625	\\
0.0183335552892085	0.001312255859375	\\
0.0183779464642429	0.001373291015625	\\
0.0184223376392773	0.00115966796875	\\
0.0184667288143117	0.0010986328125	\\
0.0185111199893461	0.001068115234375	\\
0.0185555111643805	0.00091552734375	\\
0.0185999023394149	0.0015869140625	\\
0.0186442935144493	0.001190185546875	\\
0.0186886846894837	0.001312255859375	\\
0.0187330758645181	0.001495361328125	\\
0.0187774670395525	0.0013427734375	\\
0.0188218582145869	0.00152587890625	\\
0.0188662493896213	0.001068115234375	\\
0.0189106405646557	0.001373291015625	\\
0.0189550317396902	0.001434326171875	\\
0.0189994229147246	0.001739501953125	\\
0.019043814089759	0.002044677734375	\\
0.0190882052647934	0.0018310546875	\\
0.0191325964398278	0.002197265625	\\
0.0191769876148622	0.002044677734375	\\
0.0192213787898966	0.0020751953125	\\
0.019265769964931	0.001922607421875	\\
0.0193101611399654	0.00189208984375	\\
0.0193545523149998	0.002166748046875	\\
0.0193989434900342	0.00225830078125	\\
0.0194433346650686	0.00189208984375	\\
0.019487725840103	0.00164794921875	\\
0.0195321170151374	0.00177001953125	\\
0.0195765081901718	0.001922607421875	\\
0.0196208993652062	0.002044677734375	\\
0.0196652905402406	0.001953125	\\
0.019709681715275	0.0015869140625	\\
0.0197540728903094	0.0015869140625	\\
0.0197984640653438	0.001861572265625	\\
0.0198428552403782	0.0018310546875	\\
0.0198872464154126	0.001678466796875	\\
0.019931637590447	0.00189208984375	\\
0.0199760287654814	0.00164794921875	\\
0.0200204199405158	0.00164794921875	\\
0.0200648111155502	0.00164794921875	\\
0.0201092022905846	0.00146484375	\\
0.020153593465619	0.00164794921875	\\
0.0201979846406534	0.0015869140625	\\
0.0202423758156878	0.001220703125	\\
0.0202867669907222	0.00140380859375	\\
0.0203311581657566	0.00152587890625	\\
0.0203755493407911	0.001373291015625	\\
0.0204199405158255	0.00164794921875	\\
0.0204643316908599	0.001739501953125	\\
0.0205087228658943	0.00140380859375	\\
0.0205531140409287	0.0015869140625	\\
0.0205975052159631	0.001983642578125	\\
0.0206418963909975	0.00189208984375	\\
0.0206862875660319	0.001678466796875	\\
0.0207306787410663	0.001800537109375	\\
0.0207750699161007	0.00201416015625	\\
0.0208194610911351	0.001983642578125	\\
0.0208638522661695	0.00189208984375	\\
0.0209082434412039	0.001800537109375	\\
0.0209526346162383	0.0013427734375	\\
0.0209970257912727	0.00128173828125	\\
0.0210414169663071	0.001373291015625	\\
0.0210858081413415	0.00152587890625	\\
0.0211301993163759	0.0015869140625	\\
0.0211745904914103	0.001708984375	\\
0.0212189816664447	0.0018310546875	\\
0.0212633728414791	0.00164794921875	\\
0.0213077640165135	0.0015869140625	\\
0.0213521551915479	0.001251220703125	\\
0.0213965463665823	0.001190185546875	\\
0.0214409375416167	0.00140380859375	\\
0.0214853287166511	0.0015869140625	\\
0.0215297198916855	0.001373291015625	\\
0.0215741110667199	0.001190185546875	\\
0.0216185022417543	0.00103759765625	\\
0.0216628934167887	0.0010986328125	\\
0.0217072845918231	0.00103759765625	\\
0.0217516757668575	0.000946044921875	\\
0.021796066941892	0.000946044921875	\\
0.0218404581169264	0.001129150390625	\\
0.0218848492919608	0.00140380859375	\\
0.0219292404669952	0.00140380859375	\\
0.0219736316420296	0.001251220703125	\\
0.022018022817064	0.00128173828125	\\
0.0220624139920984	0.00164794921875	\\
0.0221068051671328	0.001434326171875	\\
0.0221511963421672	0.00115966796875	\\
0.0221955875172016	0.00115966796875	\\
0.022239978692236	0.001373291015625	\\
0.0222843698672704	0.001251220703125	\\
0.0223287610423048	0.000823974609375	\\
0.0223731522173392	0.000823974609375	\\
0.0224175433923736	0.0009765625	\\
0.022461934567408	0.000885009765625	\\
0.0225063257424424	0.000701904296875	\\
0.0225507169174768	0.000732421875	\\
0.0225951080925112	0.00115966796875	\\
0.0226394992675456	0.000885009765625	\\
0.02268389044258	0.00079345703125	\\
0.0227282816176144	0.000762939453125	\\
0.0227726727926488	0.000701904296875	\\
0.0228170639676832	0.000946044921875	\\
0.0228614551427176	0.0013427734375	\\
0.022905846317752	0.00115966796875	\\
0.0229502374927864	0.0006103515625	\\
0.0229946286678208	0.000946044921875	\\
0.0230390198428552	0.001129150390625	\\
0.0230834110178896	0.001251220703125	\\
0.023127802192924	0.00152587890625	\\
0.0231721933679584	0.001129150390625	\\
0.0232165845429929	0.000885009765625	\\
0.0232609757180273	0.001190185546875	\\
0.0233053668930617	0.001373291015625	\\
0.0233497580680961	0.001434326171875	\\
0.0233941492431305	0.0013427734375	\\
0.0234385404181649	0.001434326171875	\\
0.0234829315931993	0.0010986328125	\\
0.0235273227682337	0.0009765625	\\
0.0235717139432681	0.001373291015625	\\
0.0236161051183025	0.001373291015625	\\
0.0236604962933369	0.001251220703125	\\
0.0237048874683713	0.001617431640625	\\
0.0237492786434057	0.001800537109375	\\
0.0237936698184401	0.001251220703125	\\
0.0238380609934745	0.000701904296875	\\
0.0238824521685089	0.00048828125	\\
0.0239268433435433	0.000396728515625	\\
0.0239712345185777	0.00079345703125	\\
0.0240156256936121	0.00030517578125	\\
0.0240600168686465	0.0001220703125	\\
0.0241044080436809	0.000640869140625	\\
0.0241487992187153	0.00054931640625	\\
0.0241931903937497	9.1552734375e-05	\\
0.0242375815687841	0.00054931640625	\\
0.0242819727438185	0.0008544921875	\\
0.0243263639188529	0.00042724609375	\\
0.0243707550938873	0.00042724609375	\\
0.0244151462689217	0.000701904296875	\\
0.0244595374439561	0.001220703125	\\
0.0245039286189905	0.001434326171875	\\
0.0245483197940249	0.001068115234375	\\
0.0245927109690594	0.000701904296875	\\
0.0246371021440938	0.000579833984375	\\
0.0246814933191282	0.0003662109375	\\
0.0247258844941626	0.00067138671875	\\
0.024770275669197	0.000823974609375	\\
0.0248146668442314	0.00067138671875	\\
0.0248590580192658	0.00079345703125	\\
0.0249034491943002	0.000762939453125	\\
0.0249478403693346	0.00103759765625	\\
0.024992231544369	0.000823974609375	\\
0.0250366227194034	0.001007080078125	\\
0.0250810138944378	0.001220703125	\\
0.0251254050694722	0.00140380859375	\\
0.0251697962445066	0.001617431640625	\\
0.025214187419541	0.001739501953125	\\
0.0252585785945754	0.00140380859375	\\
0.0253029697696098	0.001434326171875	\\
0.0253473609446442	0.001556396484375	\\
0.0253917521196786	0.00140380859375	\\
0.025436143294713	0.001190185546875	\\
0.0254805344697474	0.001129150390625	\\
0.0255249256447818	0.001220703125	\\
0.0255693168198162	0.001373291015625	\\
0.0256137079948506	0.00140380859375	\\
0.025658099169885	0.001373291015625	\\
0.0257024903449194	0.001251220703125	\\
0.0257468815199538	0.000701904296875	\\
0.0257912726949882	0.001190185546875	\\
0.0258356638700226	0.001495361328125	\\
0.025880055045057	0.0010986328125	\\
0.0259244462200914	0.00079345703125	\\
0.0259688373951259	0.00079345703125	\\
0.0260132285701603	0.000640869140625	\\
0.0260576197451947	0.001007080078125	\\
0.0261020109202291	0.0013427734375	\\
0.0261464020952635	0.000732421875	\\
0.0261907932702979	0.000732421875	\\
0.0262351844453323	0.001251220703125	\\
0.0262795756203667	0.001190185546875	\\
0.0263239667954011	0.000457763671875	\\
0.0263683579704355	0.00042724609375	\\
0.0264127491454699	0.00079345703125	\\
0.0264571403205043	0.000701904296875	\\
0.0265015314955387	0.00067138671875	\\
0.0265459226705731	0.000762939453125	\\
0.0265903138456075	0.001007080078125	\\
0.0266347050206419	0.00103759765625	\\
0.0266790961956763	0.001068115234375	\\
0.0267234873707107	0.0010986328125	\\
0.0267678785457451	0.001220703125	\\
0.0268122697207795	0.001708984375	\\
0.0268566608958139	0.0015869140625	\\
0.0269010520708483	0.001251220703125	\\
0.0269454432458827	0.00128173828125	\\
0.0269898344209171	0.001190185546875	\\
0.0270342255959515	0.001373291015625	\\
0.0270786167709859	0.00152587890625	\\
0.0271230079460203	0.00146484375	\\
0.0271673991210547	0.00177001953125	\\
0.0272117902960891	0.0013427734375	\\
0.0272561814711235	0.00091552734375	\\
0.0273005726461579	0.001373291015625	\\
0.0273449638211923	0.00152587890625	\\
0.0273893549962268	0.001190185546875	\\
0.0274337461712612	0.00079345703125	\\
0.0274781373462956	0.00115966796875	\\
0.02752252852133	0.00091552734375	\\
0.0275669196963644	0.000335693359375	\\
0.0276113108713988	0.000762939453125	\\
0.0276557020464332	0.00079345703125	\\
0.0277000932214676	0.000244140625	\\
0.027744484396502	0.000274658203125	\\
0.0277888755715364	0.000335693359375	\\
0.0278332667465708	0.000457763671875	\\
0.0278776579216052	0.000335693359375	\\
0.0279220490966396	0.000152587890625	\\
0.027966440271674	0.000213623046875	\\
0.0280108314467084	6.103515625e-05	\\
0.0280552226217428	0.000152587890625	\\
0.0280996137967772	6.103515625e-05	\\
0.0281440049718116	-0.00042724609375	\\
0.028188396146846	-0.00054931640625	\\
0.0282327873218804	-0.000396728515625	\\
0.0282771784969148	0.000274658203125	\\
0.0283215696719492	0.00042724609375	\\
0.0283659608469836	-9.1552734375e-05	\\
0.028410352022018	-0.00042724609375	\\
0.0284547431970524	-0.000244140625	\\
0.0284991343720868	-3.0517578125e-05	\\
0.0285435255471212	-0.00018310546875	\\
0.0285879167221556	-0.000244140625	\\
0.02863230789719	3.0517578125e-05	\\
0.0286766990722244	0	\\
0.0287210902472588	6.103515625e-05	\\
0.0287654814222932	0.000213623046875	\\
0.0288098725973277	9.1552734375e-05	\\
0.0288542637723621	-6.103515625e-05	\\
0.0288986549473965	-0.000244140625	\\
0.0289430461224309	0.000274658203125	\\
0.0289874372974653	0.0001220703125	\\
0.0290318284724997	-0.00030517578125	\\
0.0290762196475341	-0.000244140625	\\
0.0291206108225685	-0.000457763671875	\\
0.0291650019976029	-0.00048828125	\\
0.0292093931726373	-0.0001220703125	\\
0.0292537843476717	-0.00054931640625	\\
0.0292981755227061	-0.00091552734375	\\
0.0293425666977405	-0.001373291015625	\\
0.0293869578727749	-0.001312255859375	\\
0.0294313490478093	-0.0013427734375	\\
0.0294757402228437	-0.001678466796875	\\
0.0295201313978781	-0.001495361328125	\\
0.0295645225729125	-0.00115966796875	\\
0.0296089137479469	-0.001129150390625	\\
0.0296533049229813	-0.00128173828125	\\
0.0296976960980157	-0.00140380859375	\\
0.0297420872730501	-0.001007080078125	\\
0.0297864784480845	-0.001251220703125	\\
0.0298308696231189	-0.00140380859375	\\
0.0298752607981533	-0.00140380859375	\\
0.0299196519731877	-0.001678466796875	\\
0.0299640431482221	-0.001708984375	\\
0.0300084343232565	-0.0018310546875	\\
0.0300528254982909	-0.00201416015625	\\
0.0300972166733253	-0.001708984375	\\
0.0301416078483597	-0.00103759765625	\\
0.0301859990233941	-0.00091552734375	\\
0.0302303901984286	-0.001251220703125	\\
0.030274781373463	-0.00091552734375	\\
0.0303191725484974	-0.00091552734375	\\
0.0303635637235318	-0.001220703125	\\
0.0304079548985662	-0.0010986328125	\\
0.0304523460736006	-0.000946044921875	\\
0.030496737248635	-0.00067138671875	\\
0.0305411284236694	-0.0006103515625	\\
0.0305855195987038	-0.000823974609375	\\
0.0306299107737382	-0.000732421875	\\
0.0306743019487726	-0.0006103515625	\\
0.030718693123807	-0.00079345703125	\\
0.0307630842988414	-0.000640869140625	\\
0.0308074754738758	-0.00042724609375	\\
0.0308518666489102	-0.000579833984375	\\
0.0308962578239446	-0.000579833984375	\\
0.030940648998979	-0.000946044921875	\\
0.0309850401740134	-0.00091552734375	\\
0.0310294313490478	-0.000640869140625	\\
0.0310738225240822	-0.001068115234375	\\
0.0311182136991166	-0.0010986328125	\\
0.031162604874151	-0.000823974609375	\\
0.0312069960491854	-0.00048828125	\\
0.0312513872242198	-0.00054931640625	\\
0.0312957783992542	-0.0010986328125	\\
0.0313401695742886	-0.000640869140625	\\
0.031384560749323	-0.00079345703125	\\
0.0314289519243574	-0.00128173828125	\\
0.0314733430993918	-0.00103759765625	\\
0.0315177342744262	-0.0009765625	\\
0.0315621254494606	-0.0009765625	\\
0.031606516624495	-0.00115966796875	\\
0.0316509077995295	-0.001434326171875	\\
0.0316952989745639	-0.00140380859375	\\
0.0317396901495983	-0.0013427734375	\\
0.0317840813246327	-0.001373291015625	\\
0.0318284724996671	-0.001220703125	\\
0.0318728636747015	-0.0009765625	\\
0.0319172548497359	-0.000946044921875	\\
0.0319616460247703	-0.001251220703125	\\
0.0320060371998047	-0.00146484375	\\
0.0320504283748391	-0.001556396484375	\\
0.0320948195498735	-0.001617431640625	\\
0.0321392107249079	-0.001312255859375	\\
0.0321836018999423	-0.000946044921875	\\
0.0322279930749767	-0.00128173828125	\\
0.0322723842500111	-0.00140380859375	\\
0.0323167754250455	-0.0009765625	\\
0.0323611666000799	-0.001312255859375	\\
0.0324055577751143	-0.001220703125	\\
0.0324499489501487	-0.000823974609375	\\
0.0324943401251831	-0.000701904296875	\\
0.0325387313002175	-0.00067138671875	\\
0.0325831224752519	-0.001007080078125	\\
0.0326275136502863	-0.0008544921875	\\
0.0326719048253207	-0.000518798828125	\\
0.0327162960003551	-0.000701904296875	\\
0.0327606871753895	-0.0006103515625	\\
0.0328050783504239	-0.000396728515625	\\
0.0328494695254583	-0.00048828125	\\
0.0328938607004927	-0.000640869140625	\\
0.0329382518755271	-0.0006103515625	\\
0.0329826430505615	-0.000885009765625	\\
0.0330270342255959	-0.0008544921875	\\
0.0330714254006304	-0.000762939453125	\\
0.0331158165756648	-0.00048828125	\\
0.0331602077506992	-0.000335693359375	\\
0.0332045989257336	-0.00042724609375	\\
0.033248990100768	-0.000640869140625	\\
0.0332933812758024	-0.00067138671875	\\
0.0333377724508368	-0.0008544921875	\\
0.0333821636258712	-0.000701904296875	\\
0.0334265548009056	-0.000732421875	\\
0.03347094597594	-0.000885009765625	\\
0.0335153371509744	-3.0517578125e-05	\\
0.0335597283260088	9.1552734375e-05	\\
0.0336041195010432	-0.000152587890625	\\
0.0336485106760776	-0.000396728515625	\\
0.033692901851112	-0.00042724609375	\\
0.0337372930261464	-0.000579833984375	\\
0.0337816842011808	-6.103515625e-05	\\
0.0338260753762152	-0.000213623046875	\\
0.0338704665512496	-9.1552734375e-05	\\
0.033914857726284	0.000213623046875	\\
0.0339592489013184	-0.00048828125	\\
0.0340036400763528	-0.00018310546875	\\
0.0340480312513872	3.0517578125e-05	\\
0.0340924224264216	-0.0001220703125	\\
0.034136813601456	-3.0517578125e-05	\\
0.0341812047764904	0.00030517578125	\\
0.0342255959515248	0.000518798828125	\\
0.0342699871265592	0.000579833984375	\\
0.0343143783015936	0.001068115234375	\\
0.034358769476628	0.000640869140625	\\
0.0344031606516624	0.000396728515625	\\
0.0344475518266969	0.000274658203125	\\
0.0344919430017313	0.0006103515625	\\
0.0345363341767657	0.0010986328125	\\
0.0345807253518001	0.000885009765625	\\
0.0346251165268345	0.00103759765625	\\
0.0346695077018689	0.001434326171875	\\
0.0347138988769033	0.001861572265625	\\
0.0347582900519377	0.0018310546875	\\
0.0348026812269721	0.001556396484375	\\
0.0348470724020065	0.00146484375	\\
0.0348914635770409	0.001434326171875	\\
0.0349358547520753	0.001861572265625	\\
0.0349802459271097	0.00146484375	\\
0.0350246371021441	0.000885009765625	\\
0.0350690282771785	0.001617431640625	\\
0.0351134194522129	0.001495361328125	\\
0.0351578106272473	0.001434326171875	\\
0.0352022018022817	0.001617431640625	\\
0.0352465929773161	0.00164794921875	\\
0.0352909841523505	0.00189208984375	\\
0.0353353753273849	0.001220703125	\\
0.0353797665024193	0.001190185546875	\\
0.0354241576774537	0.00152587890625	\\
0.0354685488524881	0.0010986328125	\\
0.0355129400275225	0.00140380859375	\\
0.0355573312025569	0.0015869140625	\\
0.0356017223775913	0.00103759765625	\\
0.0356461135526257	0.001190185546875	\\
0.0356905047276601	0.00091552734375	\\
0.0357348959026945	0.000946044921875	\\
0.0357792870777289	0.00079345703125	\\
0.0358236782527634	0.000885009765625	\\
0.0358680694277978	0.001251220703125	\\
0.0359124606028322	0.000579833984375	\\
0.0359568517778666	0.000701904296875	\\
0.036001242952901	0.000732421875	\\
0.0360456341279354	0.0006103515625	\\
0.0360900253029698	0.000640869140625	\\
0.0361344164780042	0.00042724609375	\\
0.0361788076530386	0.0006103515625	\\
0.036223198828073	0.0006103515625	\\
0.0362675900031074	0.000579833984375	\\
0.0363119811781418	0.001007080078125	\\
0.0363563723531762	0.001251220703125	\\
0.0364007635282106	0.001312255859375	\\
0.036445154703245	0.00103759765625	\\
0.0364895458782794	0.001129150390625	\\
0.0365339370533138	0.001495361328125	\\
0.0365783282283482	0.001373291015625	\\
0.0366227194033826	0.0015869140625	\\
0.036667110578417	0.001617431640625	\\
0.0367115017534514	0.001495361328125	\\
0.0367558929284858	0.00140380859375	\\
0.0368002841035202	0.001617431640625	\\
0.0368446752785546	0.001617431640625	\\
0.036889066453589	0.00201416015625	\\
0.0369334576286234	0.00177001953125	\\
0.0369778488036578	0.001251220703125	\\
0.0370222399786922	0.001312255859375	\\
0.0370666311537266	0.001312255859375	\\
0.037111022328761	0.00140380859375	\\
0.0371554135037954	0.001373291015625	\\
0.0371998046788298	0.00146484375	\\
0.0372441958538643	0.00146484375	\\
0.0372885870288987	0.001678466796875	\\
0.0373329782039331	0.001434326171875	\\
0.0373773693789675	0.00140380859375	\\
0.0374217605540019	0.00152587890625	\\
0.0374661517290363	0.001739501953125	\\
0.0375105429040707	0.001373291015625	\\
0.0375549340791051	0.001617431640625	\\
0.0375993252541395	0.00146484375	\\
0.0376437164291739	0.00115966796875	\\
0.0376881076042083	0.00140380859375	\\
0.0377324987792427	0.00091552734375	\\
0.0377768899542771	0.00091552734375	\\
0.0378212811293115	0.000701904296875	\\
0.0378656723043459	0.00054931640625	\\
0.0379100634793803	0.0006103515625	\\
0.0379544546544147	0.000518798828125	\\
0.0379988458294491	0.000396728515625	\\
0.0380432370044835	0.00048828125	\\
0.0380876281795179	0.000579833984375	\\
0.0381320193545523	0.000640869140625	\\
0.0381764105295867	0.00042724609375	\\
0.0382208017046211	0	\\
0.0382651928796555	0	\\
0.0383095840546899	-6.103515625e-05	\\
0.0383539752297243	-3.0517578125e-05	\\
0.0383983664047587	0.0001220703125	\\
0.0384427575797931	0.000640869140625	\\
0.0384871487548275	0.000457763671875	\\
0.0385315399298619	0.000335693359375	\\
0.0385759311048963	0.000640869140625	\\
0.0386203222799308	0.00018310546875	\\
0.0386647134549652	0.000274658203125	\\
0.0387091046299996	0	\\
0.038753495805034	-0.000396728515625	\\
0.0387978869800684	-0.00018310546875	\\
0.0388422781551028	-0.000335693359375	\\
0.0388866693301372	3.0517578125e-05	\\
0.0389310605051716	0	\\
0.038975451680206	-9.1552734375e-05	\\
0.0390198428552404	-0.000213623046875	\\
0.0390642340302748	-6.103515625e-05	\\
0.0391086252053092	0.00048828125	\\
0.0391530163803436	0.0003662109375	\\
0.039197407555378	0.000701904296875	\\
0.0392417987304124	0.00067138671875	\\
0.0392861899054468	0.00048828125	\\
0.0393305810804812	0.000701904296875	\\
0.0393749722555156	0.0006103515625	\\
0.03941936343055	0.000244140625	\\
0.0394637546055844	0.000396728515625	\\
0.0395081457806188	0.000335693359375	\\
0.0395525369556532	0.00042724609375	\\
0.0395969281306876	0.00054931640625	\\
0.039641319305722	0.000823974609375	\\
0.0396857104807564	0.0006103515625	\\
0.0397301016557908	0.000762939453125	\\
0.0397744928308252	0.00103759765625	\\
0.0398188840058596	0.00079345703125	\\
0.039863275180894	0.000762939453125	\\
0.0399076663559284	0.00067138671875	\\
0.0399520575309628	0.000457763671875	\\
0.0399964487059973	0.00054931640625	\\
0.0400408398810317	0.000762939453125	\\
0.0400852310560661	0.000335693359375	\\
0.0401296222311005	0.0001220703125	\\
0.0401740134061349	-9.1552734375e-05	\\
0.0402184045811693	-0.00018310546875	\\
0.0402627957562037	3.0517578125e-05	\\
0.0403071869312381	-6.103515625e-05	\\
0.0403515781062725	0.000152587890625	\\
0.0403959692813069	0.00048828125	\\
0.0404403604563413	0.0003662109375	\\
0.0404847516313757	0.000518798828125	\\
0.0405291428064101	0.000640869140625	\\
0.0405735339814445	0.000701904296875	\\
0.0406179251564789	0.000946044921875	\\
0.0406623163315133	0.000701904296875	\\
0.0407067075065477	0.000946044921875	\\
0.0407510986815821	0.001434326171875	\\
0.0407954898566165	0.0009765625	\\
0.0408398810316509	0.00079345703125	\\
0.0408842722066853	0.001251220703125	\\
0.0409286633817197	0.00164794921875	\\
0.0409730545567541	0.00189208984375	\\
0.0410174457317885	0.001495361328125	\\
0.0410618369068229	0.001434326171875	\\
0.0411062280818573	0.001312255859375	\\
0.0411506192568917	0.0010986328125	\\
0.0411950104319261	0.001373291015625	\\
0.0412394016069605	0.001251220703125	\\
0.0412837927819949	0.001251220703125	\\
0.0413281839570293	0.001861572265625	\\
0.0413725751320637	0.001708984375	\\
0.0414169663070982	0.001251220703125	\\
0.0414613574821326	0.001251220703125	\\
0.041505748657167	0.001556396484375	\\
0.0415501398322014	0.001739501953125	\\
0.0415945310072358	0.00164794921875	\\
0.0416389221822702	0.00140380859375	\\
0.0416833133573046	0.00152587890625	\\
0.041727704532339	0.0018310546875	\\
0.0417720957073734	0.001373291015625	\\
0.0418164868824078	0.0010986328125	\\
0.0418608780574422	0.001251220703125	\\
0.0419052692324766	0.00146484375	\\
0.041949660407511	0.001556396484375	\\
0.0419940515825454	0.0009765625	\\
0.0420384427575798	0.0010986328125	\\
0.0420828339326142	0.001617431640625	\\
0.0421272251076486	0.001190185546875	\\
0.042171616282683	0.001312255859375	\\
0.0422160074577174	0.001373291015625	\\
0.0422603986327518	0.001190185546875	\\
0.0423047898077862	0.00152587890625	\\
0.0423491809828206	0.001434326171875	\\
0.042393572157855	0.001251220703125	\\
0.0424379633328894	0.001556396484375	\\
0.0424823545079238	0.001129150390625	\\
0.0425267456829582	0.000885009765625	\\
0.0425711368579926	0.001129150390625	\\
0.042615528033027	0.001190185546875	\\
0.0426599192080614	0.0009765625	\\
0.0427043103830958	0.001068115234375	\\
0.0427487015581302	0.001220703125	\\
0.0427930927331646	0.001190185546875	\\
0.0428374839081991	0.00115966796875	\\
0.0428818750832335	0.001251220703125	\\
0.0429262662582679	0.00103759765625	\\
0.0429706574333023	0.001190185546875	\\
0.0430150486083367	0.001495361328125	\\
0.0430594397833711	0.00103759765625	\\
0.0431038309584055	0.00115966796875	\\
0.0431482221334399	0.00140380859375	\\
0.0431926133084743	0.000701904296875	\\
0.0432370044835087	0.000732421875	\\
0.0432813956585431	0.00103759765625	\\
0.0433257868335775	0.000946044921875	\\
0.0433701780086119	0.000762939453125	\\
0.0434145691836463	0.000885009765625	\\
0.0434589603586807	0.0008544921875	\\
0.0435033515337151	0.00054931640625	\\
0.0435477427087495	0.000701904296875	\\
0.0435921338837839	0.000335693359375	\\
0.0436365250588183	-0.000244140625	\\
0.0436809162338527	3.0517578125e-05	\\
0.0437253074088871	0.000518798828125	\\
0.0437696985839215	-9.1552734375e-05	\\
0.0438140897589559	-0.00030517578125	\\
0.0438584809339903	0.000244140625	\\
0.0439028721090247	0.00018310546875	\\
0.0439472632840591	-0.000274658203125	\\
0.0439916544590935	-0.000152587890625	\\
0.0440360456341279	-0.000518798828125	\\
0.0440804368091623	-0.00054931640625	\\
0.0441248279841967	-0.000244140625	\\
0.0441692191592311	-0.00048828125	\\
0.0442136103342655	-0.00030517578125	\\
0.0442580015093	-0.00030517578125	\\
0.0443023926843344	-0.00030517578125	\\
0.0443467838593688	-3.0517578125e-05	\\
0.0443911750344032	-0.00042724609375	\\
0.0444355662094376	-0.000732421875	\\
0.044479957384472	-0.000732421875	\\
0.0445243485595064	-0.000732421875	\\
0.0445687397345408	-0.000518798828125	\\
0.0446131309095752	-0.000274658203125	\\
0.0446575220846096	-0.000762939453125	\\
0.044701913259644	-0.000701904296875	\\
0.0447463044346784	-0.00054931640625	\\
0.0447906956097128	-0.000701904296875	\\
0.0448350867847472	-0.000946044921875	\\
0.0448794779597816	-0.00103759765625	\\
0.044923869134816	-0.0009765625	\\
0.0449682603098504	-0.00079345703125	\\
0.0450126514848848	-0.000579833984375	\\
0.0450570426599192	-0.00048828125	\\
0.0451014338349536	-0.00048828125	\\
0.045145825009988	-0.000701904296875	\\
0.0451902161850224	-0.0008544921875	\\
0.0452346073600568	-0.0009765625	\\
0.0452789985350912	-0.001220703125	\\
0.0453233897101256	-0.001190185546875	\\
0.04536778088516	-0.000823974609375	\\
0.0454121720601944	-0.00103759765625	\\
0.0454565632352288	-0.00115966796875	\\
0.0455009544102632	-0.001495361328125	\\
0.0455453455852976	-0.00140380859375	\\
0.045589736760332	-0.001373291015625	\\
0.0456341279353664	-0.001556396484375	\\
0.0456785191104008	-0.00146484375	\\
0.0457229102854353	-0.00177001953125	\\
0.0457673014604697	-0.001922607421875	\\
0.0458116926355041	-0.001495361328125	\\
0.0458560838105385	-0.0015869140625	\\
0.0459004749855729	-0.00189208984375	\\
0.0459448661606073	-0.00152587890625	\\
0.0459892573356417	-0.001434326171875	\\
0.0460336485106761	-0.001556396484375	\\
0.0460780396857105	-0.0013427734375	\\
0.0461224308607449	-0.001434326171875	\\
0.0461668220357793	-0.00164794921875	\\
0.0462112132108137	-0.001495361328125	\\
0.0462556043858481	-0.0010986328125	\\
0.0462999955608825	-0.00128173828125	\\
0.0463443867359169	-0.001373291015625	\\
0.0463887779109513	-0.000946044921875	\\
0.0464331690859857	-0.00079345703125	\\
0.0464775602610201	-0.000579833984375	\\
0.0465219514360545	-0.0008544921875	\\
0.0465663426110889	-0.00079345703125	\\
0.0466107337861233	-0.0006103515625	\\
0.0466551249611577	-0.000946044921875	\\
0.0466995161361921	-0.001129150390625	\\
0.0467439073112265	-0.0008544921875	\\
0.0467882984862609	-0.000885009765625	\\
0.0468326896612953	-0.0013427734375	\\
0.0468770808363297	-0.00115966796875	\\
0.0469214720113641	-0.001129150390625	\\
0.0469658631863985	-0.001129150390625	\\
0.0470102543614329	-0.001190185546875	\\
0.0470546455364673	-0.001708984375	\\
0.0470990367115018	-0.0015869140625	\\
0.0471434278865362	-0.0010986328125	\\
0.0471878190615706	-0.000885009765625	\\
0.047232210236605	-0.001312255859375	\\
0.0472766014116394	-0.0018310546875	\\
0.0473209925866738	-0.001861572265625	\\
0.0473653837617082	-0.00177001953125	\\
0.0474097749367426	-0.001434326171875	\\
0.047454166111777	-0.001800537109375	\\
0.0474985572868114	-0.00189208984375	\\
0.0475429484618458	-0.00177001953125	\\
0.0475873396368802	-0.002044677734375	\\
0.0476317308119146	-0.00213623046875	\\
0.047676121986949	-0.00213623046875	\\
0.0477205131619834	-0.00177001953125	\\
0.0477649043370178	-0.0018310546875	\\
0.0478092955120522	-0.001953125	\\
0.0478536866870866	-0.0020751953125	\\
0.047898077862121	-0.002044677734375	\\
0.0479424690371554	-0.00189208984375	\\
0.0479868602121898	-0.001922607421875	\\
0.0480312513872242	-0.002166748046875	\\
0.0480756425622586	-0.001678466796875	\\
0.048120033737293	-0.0020751953125	\\
0.0481644249123274	-0.00213623046875	\\
0.0482088160873618	-0.0018310546875	\\
0.0482532072623962	-0.0018310546875	\\
0.0482975984374306	-0.001556396484375	\\
0.048341989612465	-0.001739501953125	\\
0.0483863807874994	-0.00177001953125	\\
0.0484307719625338	-0.001495361328125	\\
0.0484751631375683	-0.00146484375	\\
0.0485195543126027	-0.00103759765625	\\
0.0485639454876371	-0.001617431640625	\\
0.0486083366626715	-0.00146484375	\\
0.0486527278377059	-0.00079345703125	\\
0.0486971190127403	-0.0009765625	\\
0.0487415101877747	-0.000885009765625	\\
0.0487859013628091	-0.000946044921875	\\
0.0488302925378435	-0.001190185546875	\\
0.0488746837128779	-0.001312255859375	\\
0.0489190748879123	-0.001434326171875	\\
0.0489634660629467	-0.001373291015625	\\
0.0490078572379811	-0.00140380859375	\\
0.0490522484130155	-0.001129150390625	\\
0.0490966395880499	-0.00091552734375	\\
0.0491410307630843	-0.001068115234375	\\
0.0491854219381187	-0.001068115234375	\\
0.0492298131131531	-0.00115966796875	\\
0.0492742042881875	-0.001129150390625	\\
0.0493185954632219	-0.00079345703125	\\
0.0493629866382563	-0.0009765625	\\
0.0494073778132907	-0.001220703125	\\
0.0494517689883251	-0.001251220703125	\\
0.0494961601633595	-0.0008544921875	\\
0.0495405513383939	-0.000946044921875	\\
0.0495849425134283	-0.001190185546875	\\
0.0496293336884627	-0.00103759765625	\\
0.0496737248634971	-0.00140380859375	\\
0.0497181160385315	-0.00152587890625	\\
0.0497625072135659	-0.001312255859375	\\
0.0498068983886003	-0.001678466796875	\\
0.0498512895636348	-0.00201416015625	\\
0.0498956807386692	-0.002044677734375	\\
0.0499400719137036	-0.0015869140625	\\
0.049984463088738	-0.001678466796875	\\
0.0500288542637724	-0.0018310546875	\\
0.0500732454388068	-0.0015869140625	\\
0.0501176366138412	-0.001556396484375	\\
0.0501620277888756	-0.00164794921875	\\
0.05020641896391	-0.00177001953125	\\
0.0502508101389444	-0.001434326171875	\\
0.0502952013139788	-0.00140380859375	\\
0.0503395924890132	-0.001312255859375	\\
0.0503839836640476	-0.001556396484375	\\
0.050428374839082	-0.001220703125	\\
0.0504727660141164	-0.0009765625	\\
0.0505171571891508	-0.000823974609375	\\
0.0505615483641852	-0.0010986328125	\\
0.0506059395392196	-0.001708984375	\\
0.050650330714254	-0.001007080078125	\\
0.0506947218892884	-0.00067138671875	\\
0.0507391130643228	-0.001312255859375	\\
0.0507835042393572	-0.00164794921875	\\
0.0508278954143916	-0.001007080078125	\\
0.050872286589426	-0.00091552734375	\\
0.0509166777644604	-0.00115966796875	\\
0.0509610689394948	-0.000640869140625	\\
0.0510054601145292	-0.000823974609375	\\
0.0510498512895636	-0.00091552734375	\\
0.051094242464598	-0.000640869140625	\\
0.0511386336396324	-0.000885009765625	\\
0.0511830248146668	-0.00103759765625	\\
0.0512274159897013	-0.0013427734375	\\
0.0512718071647357	-0.001251220703125	\\
0.0513161983397701	-0.001251220703125	\\
0.0513605895148045	-0.001007080078125	\\
0.0514049806898389	-0.00115966796875	\\
0.0514493718648733	-0.000946044921875	\\
0.0514937630399077	-0.000946044921875	\\
0.0515381542149421	-0.001220703125	\\
0.0515825453899765	-0.0010986328125	\\
0.0516269365650109	-0.000946044921875	\\
0.0516713277400453	-0.0009765625	\\
0.0517157189150797	-0.001007080078125	\\
0.0517601100901141	-0.00091552734375	\\
0.0518045012651485	-0.00091552734375	\\
0.0518488924401829	-0.00067138671875	\\
0.0518932836152173	-0.000701904296875	\\
0.0519376747902517	-0.00103759765625	\\
0.0519820659652861	-0.0010986328125	\\
0.0520264571403205	-0.001220703125	\\
0.0520708483153549	-0.00128173828125	\\
0.0521152394903893	-0.00079345703125	\\
0.0521596306654237	-0.000946044921875	\\
0.0522040218404581	-0.001220703125	\\
0.0522484130154925	-0.000946044921875	\\
0.0522928041905269	-0.000885009765625	\\
0.0523371953655613	-0.001007080078125	\\
0.0523815865405957	-0.001617431640625	\\
0.0524259777156301	-0.001708984375	\\
0.0524703688906645	-0.001373291015625	\\
0.0525147600656989	-0.001556396484375	\\
0.0525591512407333	-0.00146484375	\\
0.0526035424157677	-0.001495361328125	\\
0.0526479335908022	-0.001739501953125	\\
0.0526923247658366	-0.001983642578125	\\
0.052736715940871	-0.0015869140625	\\
0.0527811071159054	-0.00128173828125	\\
0.0528254982909398	-0.0009765625	\\
0.0528698894659742	-0.0008544921875	\\
0.0529142806410086	-0.0010986328125	\\
0.052958671816043	-0.0013427734375	\\
0.0530030629910774	-0.001007080078125	\\
0.0530474541661118	-0.001312255859375	\\
0.0530918453411462	-0.0015869140625	\\
0.0531362365161806	-0.00128173828125	\\
0.053180627691215	-0.001312255859375	\\
0.0532250188662494	-0.00152587890625	\\
0.0532694100412838	-0.001556396484375	\\
0.0533138012163182	-0.0018310546875	\\
0.0533581923913526	-0.00164794921875	\\
0.053402583566387	-0.00140380859375	\\
0.0534469747414214	-0.0013427734375	\\
0.0534913659164558	-0.001708984375	\\
0.0535357570914902	-0.001800537109375	\\
0.0535801482665246	-0.00146484375	\\
0.053624539441559	-0.0015869140625	\\
0.0536689306165934	-0.00152587890625	\\
0.0537133217916278	-0.001190185546875	\\
0.0537577129666622	-0.00115966796875	\\
0.0538021041416966	-0.0008544921875	\\
0.053846495316731	-0.0013427734375	\\
0.0538908864917654	-0.001800537109375	\\
0.0539352776667998	-0.001434326171875	\\
0.0539796688418342	-0.001312255859375	\\
0.0540240600168686	-0.001190185546875	\\
0.0540684511919031	-0.001068115234375	\\
0.0541128423669375	-0.000823974609375	\\
0.0541572335419719	-0.001190185546875	\\
0.0542016247170063	-0.001068115234375	\\
0.0542460158920407	-0.001495361328125	\\
0.0542904070670751	-0.001495361328125	\\
0.0543347982421095	-0.001495361328125	\\
0.0543791894171439	-0.001800537109375	\\
0.0544235805921783	-0.001678466796875	\\
0.0544679717672127	-0.001617431640625	\\
0.0545123629422471	-0.001678466796875	\\
0.0545567541172815	-0.0013427734375	\\
0.0546011452923159	-0.000885009765625	\\
0.0546455364673503	-0.001312255859375	\\
0.0546899276423847	-0.00152587890625	\\
0.0547343188174191	-0.001556396484375	\\
0.0547787099924535	-0.001251220703125	\\
0.0548231011674879	-0.00115966796875	\\
0.0548674923425223	-0.001220703125	\\
0.0549118835175567	-0.001251220703125	\\
0.0549562746925911	-0.0013427734375	\\
0.0550006658676255	-0.00146484375	\\
0.0550450570426599	-0.001678466796875	\\
0.0550894482176943	-0.001953125	\\
0.0551338393927287	-0.001800537109375	\\
0.0551782305677631	-0.001739501953125	\\
0.0552226217427975	-0.001953125	\\
0.0552670129178319	-0.002349853515625	\\
0.0553114040928663	-0.00274658203125	\\
0.0553557952679007	-0.002471923828125	\\
0.0554001864429351	-0.002197265625	\\
0.0554445776179695	-0.002288818359375	\\
0.055488968793004	-0.0023193359375	\\
0.0555333599680384	-0.0015869140625	\\
0.0555777511430728	-0.0015869140625	\\
0.0556221423181072	-0.00189208984375	\\
0.0556665334931416	-0.001373291015625	\\
0.055710924668176	-0.001434326171875	\\
0.0557553158432104	-0.00103759765625	\\
0.0557997070182448	-0.000885009765625	\\
0.0558440981932792	-0.001190185546875	\\
0.0558884893683136	-0.0008544921875	\\
0.055932880543348	-0.00103759765625	\\
0.0559772717183824	-0.001007080078125	\\
0.0560216628934168	-0.00103759765625	\\
0.0560660540684512	-0.0008544921875	\\
0.0561104452434856	-0.000640869140625	\\
0.05615483641852	-0.00079345703125	\\
0.0561992275935544	-0.000213623046875	\\
0.0562436187685888	-0.00030517578125	\\
0.0562880099436232	6.103515625e-05	\\
0.0563324011186576	0.000579833984375	\\
0.056376792293692	0.00018310546875	\\
0.0564211834687264	0.000244140625	\\
0.0564655746437608	0.000335693359375	\\
0.0565099658187952	9.1552734375e-05	\\
0.0565543569938296	0.000518798828125	\\
0.056598748168864	0.00030517578125	\\
0.0566431393438984	-3.0517578125e-05	\\
0.0566875305189328	0	\\
0.0567319216939672	-9.1552734375e-05	\\
0.0567763128690016	-0.00018310546875	\\
0.056820704044036	0.000335693359375	\\
0.0568650952190704	0.000518798828125	\\
0.0569094863941049	0.000579833984375	\\
0.0569538775691393	0.00030517578125	\\
0.0569982687441737	3.0517578125e-05	\\
0.0570426599192081	0.000213623046875	\\
0.0570870510942425	-3.0517578125e-05	\\
0.0571314422692769	0.000244140625	\\
0.0571758334443113	0.000274658203125	\\
0.0572202246193457	-0.000579833984375	\\
0.0572646157943801	-0.0006103515625	\\
0.0573090069694145	-0.000396728515625	\\
0.0573533981444489	-0.000732421875	\\
0.0573977893194833	-0.00054931640625	\\
0.0574421804945177	-3.0517578125e-05	\\
0.0574865716695521	-0.00030517578125	\\
0.0575309628445865	0.00018310546875	\\
0.0575753540196209	0.000274658203125	\\
0.0576197451946553	-0.000244140625	\\
0.0576641363696897	0	\\
0.0577085275447241	0.0001220703125	\\
0.0577529187197585	6.103515625e-05	\\
0.0577973098947929	-0.00018310546875	\\
0.0578417010698273	-0.000396728515625	\\
0.0578860922448617	9.1552734375e-05	\\
0.0579304834198961	0.0001220703125	\\
0.0579748745949305	0	\\
0.0580192657699649	0.0001220703125	\\
0.0580636569449993	0.00048828125	\\
0.0581080481200337	0.000823974609375	\\
0.0581524392950681	0.0008544921875	\\
0.0581968304701025	0.00103759765625	\\
0.0582412216451369	0.00115966796875	\\
0.0582856128201713	0.000579833984375	\\
0.0583300039952058	0.00042724609375	\\
0.0583743951702402	0.0006103515625	\\
0.0584187863452746	0.00103759765625	\\
0.058463177520309	0.00079345703125	\\
0.0585075686953434	0.00048828125	\\
0.0585519598703778	0.000823974609375	\\
0.0585963510454122	0.00048828125	\\
0.0586407422204466	6.103515625e-05	\\
0.058685133395481	0.000335693359375	\\
0.0587295245705154	0.000335693359375	\\
0.0587739157455498	0.000732421875	\\
0.0588183069205842	0.00140380859375	\\
0.0588626980956186	0.00079345703125	\\
0.058907089270653	0.000885009765625	\\
0.0589514804456874	0.0009765625	\\
0.0589958716207218	0.0009765625	\\
0.0590402627957562	0.00079345703125	\\
0.0590846539707906	0.000823974609375	\\
0.059129045145825	0.000823974609375	\\
0.0591734363208594	0.0009765625	\\
0.0592178274958938	0.001220703125	\\
0.0592622186709282	0.000885009765625	\\
0.0593066098459626	0.000762939453125	\\
0.059351001020997	0.000457763671875	\\
0.0593953921960314	0.0003662109375	\\
0.0594397833710658	0.000457763671875	\\
0.0594841745461002	0.000213623046875	\\
0.0595285657211346	0	\\
0.059572956896169	0.00030517578125	\\
0.0596173480712034	0.0003662109375	\\
0.0596617392462378	0.000396728515625	\\
0.0597061304212722	0.000579833984375	\\
0.0597505215963067	0.000396728515625	\\
0.0597949127713411	0.000579833984375	\\
0.0598393039463755	0.000762939453125	\\
0.0598836951214099	0.000701904296875	\\
0.0599280862964443	0.000732421875	\\
0.0599724774714787	0.000762939453125	\\
0.0600168686465131	0.0008544921875	\\
0.0600612598215475	0.00030517578125	\\
0.0601056509965819	0.00054931640625	\\
0.0601500421716163	0.000946044921875	\\
0.0601944333466507	0.00042724609375	\\
0.0602388245216851	0.000518798828125	\\
0.0602832156967195	0.000335693359375	\\
0.0603276068717539	0.00042724609375	\\
0.0603719980467883	0.000213623046875	\\
0.0604163892218227	0.000335693359375	\\
0.0604607803968571	0.00042724609375	\\
0.0605051715718915	0.00030517578125	\\
0.0605495627469259	0.000640869140625	\\
0.0605939539219603	0.000396728515625	\\
0.0606383450969947	0.000335693359375	\\
0.0606827362720291	0.000457763671875	\\
0.0607271274470635	0.00048828125	\\
0.0607715186220979	6.103515625e-05	\\
0.0608159097971323	0.000213623046875	\\
0.0608603009721667	-0.000244140625	\\
0.0609046921472011	-0.000640869140625	\\
0.0609490833222355	-9.1552734375e-05	\\
0.0609934744972699	-0.000518798828125	\\
0.0610378656723043	-0.00048828125	\\
0.0610822568473387	-0.00054931640625	\\
0.0611266480223732	-0.00048828125	\\
0.0611710391974076	-0.00030517578125	\\
0.061215430372442	-0.000244140625	\\
0.0612598215474764	-0.00067138671875	\\
0.0613042127225108	-0.000457763671875	\\
0.0613486038975452	-0.00018310546875	\\
0.0613929950725796	-0.000457763671875	\\
0.061437386247614	-0.00018310546875	\\
0.0614817774226484	-0.000762939453125	\\
0.0615261685976828	-0.00091552734375	\\
0.0615705597727172	-0.000518798828125	\\
0.0616149509477516	-0.00103759765625	\\
0.061659342122786	-0.001007080078125	\\
0.0617037332978204	-0.001129150390625	\\
0.0617481244728548	-0.001129150390625	\\
0.0617925156478892	-0.000823974609375	\\
0.0618369068229236	-0.00140380859375	\\
0.061881297997958	-0.0015869140625	\\
0.0619256891729924	-0.001373291015625	\\
0.0619700803480268	-0.001678466796875	\\
0.0620144715230612	-0.001190185546875	\\
0.0620588626980956	-0.000823974609375	\\
0.06210325387313	-0.000946044921875	\\
0.0621476450481644	-0.000732421875	\\
0.0621920362231988	-0.00091552734375	\\
0.0622364273982332	-0.000946044921875	\\
0.0622808185732676	-0.000701904296875	\\
0.062325209748302	-0.00091552734375	\\
0.0623696009233364	-0.000946044921875	\\
0.0624139920983708	-0.0010986328125	\\
0.0624583832734052	-0.001129150390625	\\
0.0625027744484396	-0.00091552734375	\\
0.0625471656234741	-0.00115966796875	\\
0.0625915567985085	-0.00140380859375	\\
0.0626359479735429	-0.00146484375	\\
0.0626803391485773	-0.001068115234375	\\
0.0627247303236117	-0.000732421875	\\
0.0627691214986461	-0.00103759765625	\\
0.0628135126736805	-0.00103759765625	\\
0.0628579038487149	-0.00079345703125	\\
0.0629022950237493	-0.000518798828125	\\
0.0629466861987837	-0.000640869140625	\\
0.0629910773738181	-0.001068115234375	\\
0.0630354685488525	-0.00128173828125	\\
0.0630798597238869	-0.001251220703125	\\
0.0631242508989213	-0.0008544921875	\\
0.0631686420739557	-0.0013427734375	\\
0.0632130332489901	-0.00128173828125	\\
0.0632574244240245	-0.000885009765625	\\
0.0633018155990589	-0.000885009765625	\\
0.0633462067740933	-0.001007080078125	\\
0.0633905979491277	-0.00054931640625	\\
0.0634349891241621	-0.000823974609375	\\
0.0634793802991965	-0.001251220703125	\\
0.0635237714742309	-0.001434326171875	\\
0.0635681626492653	-0.0013427734375	\\
0.0636125538242997	-0.001190185546875	\\
0.0636569449993341	-0.001312255859375	\\
0.0637013361743685	-0.00140380859375	\\
0.0637457273494029	-0.00128173828125	\\
0.0637901185244373	-0.0013427734375	\\
0.0638345096994717	-0.00128173828125	\\
0.0638789008745062	-0.00128173828125	\\
0.0639232920495405	-0.001068115234375	\\
0.063967683224575	-0.000946044921875	\\
0.0640120743996094	-0.0010986328125	\\
0.0640564655746438	-0.0008544921875	\\
0.0641008567496782	-0.00079345703125	\\
0.0641452479247126	-0.00091552734375	\\
0.064189639099747	-0.001190185546875	\\
0.0642340302747814	-0.001434326171875	\\
0.0642784214498158	-0.001251220703125	\\
0.0643228126248502	-0.000762939453125	\\
0.0643672037998846	-0.001129150390625	\\
0.064411594974919	-0.0009765625	\\
0.0644559861499534	-0.000823974609375	\\
0.0645003773249878	-0.00128173828125	\\
0.0645447685000222	-0.001190185546875	\\
0.0645891596750566	-0.001220703125	\\
0.064633550850091	-0.001312255859375	\\
0.0646779420251254	-0.001068115234375	\\
0.0647223332001598	-0.001312255859375	\\
0.0647667243751942	-0.00140380859375	\\
0.0648111155502286	-0.0013427734375	\\
0.064855506725263	-0.001068115234375	\\
0.0648998979002974	-0.000823974609375	\\
0.0649442890753318	-0.00079345703125	\\
0.0649886802503662	-0.000885009765625	\\
0.0650330714254006	-0.001434326171875	\\
0.065077462600435	-0.00128173828125	\\
0.0651218537754694	-0.001068115234375	\\
0.0651662449505038	-0.001007080078125	\\
0.0652106361255382	-0.001129150390625	\\
0.0652550273005726	-0.0008544921875	\\
0.0652994184756071	-0.000579833984375	\\
0.0653438096506414	-0.000823974609375	\\
0.0653882008256759	-0.0008544921875	\\
0.0654325920007103	-0.000762939453125	\\
0.0654769831757447	-0.000640869140625	\\
0.0655213743507791	-0.000732421875	\\
0.0655657655258135	-0.0006103515625	\\
0.0656101567008479	-0.000335693359375	\\
0.0656545478758823	-0.00018310546875	\\
0.0656989390509167	-6.103515625e-05	\\
0.0657433302259511	-6.103515625e-05	\\
0.0657877214009855	0.00018310546875	\\
0.0658321125760199	0.00054931640625	\\
0.0658765037510543	0.000274658203125	\\
0.0659208949260887	0.000213623046875	\\
0.0659652861011231	0.00091552734375	\\
0.0660096772761575	0.00067138671875	\\
0.0660540684511919	0.0006103515625	\\
0.0660984596262263	0.00079345703125	\\
0.0661428508012607	0.000640869140625	\\
0.0661872419762951	0.001251220703125	\\
0.0662316331513295	0.00103759765625	\\
0.0662760243263639	0.000701904296875	\\
0.0663204155013983	0.0008544921875	\\
0.0663648066764327	0.0008544921875	\\
0.0664091978514671	0.00067138671875	\\
0.0664535890265015	0.00067138671875	\\
0.0664979802015359	0.000579833984375	\\
0.0665423713765703	0.0006103515625	\\
0.0665867625516047	0.0003662109375	\\
0.0666311537266392	0.000579833984375	\\
0.0666755449016735	0.000457763671875	\\
0.066719936076708	0.0001220703125	\\
0.0667643272517423	0.0001220703125	\\
0.0668087184267768	6.103515625e-05	\\
0.0668531096018112	-3.0517578125e-05	\\
0.0668975007768456	-0.000244140625	\\
0.06694189195188	-0.000244140625	\\
0.0669862831269144	-0.000396728515625	\\
0.0670306743019488	-0.000396728515625	\\
0.0670750654769832	-0.000396728515625	\\
0.0671194566520176	-0.0003662109375	\\
0.067163847827052	-0.00018310546875	\\
0.0672082390020864	-0.00018310546875	\\
0.0672526301771208	-0.000152587890625	\\
0.0672970213521552	-0.0003662109375	\\
0.0673414125271896	-0.000396728515625	\\
0.067385803702224	-0.000274658203125	\\
0.0674301948772584	-0.000701904296875	\\
0.0674745860522928	-0.00079345703125	\\
0.0675189772273272	-0.000640869140625	\\
0.0675633684023616	-0.00079345703125	\\
0.067607759577396	-0.000518798828125	\\
0.0676521507524304	-0.00042724609375	\\
0.0676965419274648	-0.000335693359375	\\
0.0677409331024992	0.00018310546875	\\
0.0677853242775336	-0.000274658203125	\\
0.067829715452568	0	\\
0.0678741066276024	0.000579833984375	\\
0.0679184978026368	0.000274658203125	\\
0.0679628889776712	0.0001220703125	\\
0.0680072801527056	0.00048828125	\\
0.0680516713277401	0.0001220703125	\\
0.0680960625027744	-0.0001220703125	\\
0.0681404536778089	0.000213623046875	\\
0.0681848448528432	-9.1552734375e-05	\\
0.0682292360278777	-0.000640869140625	\\
0.0682736272029121	-0.00018310546875	\\
0.0683180183779465	0.000335693359375	\\
0.0683624095529809	0.000213623046875	\\
0.0684068007280153	-0.000335693359375	\\
0.0684511919030497	-0.00042724609375	\\
0.0684955830780841	-0.0006103515625	\\
0.0685399742531185	-0.00067138671875	\\
0.0685843654281529	-0.000152587890625	\\
0.0686287566031873	-0.000213623046875	\\
0.0686731477782217	-0.00030517578125	\\
0.0687175389532561	-0.000244140625	\\
0.0687619301282905	-0.000457763671875	\\
0.0688063213033249	-0.000823974609375	\\
0.0688507124783593	-0.000823974609375	\\
0.0688951036533937	-0.000640869140625	\\
0.0689394948284281	-0.000732421875	\\
0.0689838860034625	-0.00079345703125	\\
0.0690282771784969	-0.000518798828125	\\
0.0690726683535313	-0.000762939453125	\\
0.0691170595285657	-0.001068115234375	\\
0.0691614507036001	-0.001190185546875	\\
0.0692058418786345	-0.000946044921875	\\
0.0692502330536689	-0.000762939453125	\\
0.0692946242287033	-0.000946044921875	\\
0.0693390154037377	-0.00103759765625	\\
0.0693834065787721	-0.001312255859375	\\
0.0694277977538065	-0.0009765625	\\
0.069472188928841	-0.000732421875	\\
0.0695165801038753	-0.000701904296875	\\
0.0695609712789098	-0.000823974609375	\\
0.0696053624539442	-0.00115966796875	\\
0.0696497536289786	-0.001312255859375	\\
0.069694144804013	-0.001129150390625	\\
0.0697385359790474	-0.00079345703125	\\
0.0697829271540818	-0.00103759765625	\\
0.0698273183291162	-0.001220703125	\\
0.0698717095041506	-0.000885009765625	\\
0.069916100679185	-0.000762939453125	\\
0.0699604918542194	-0.001007080078125	\\
0.0700048830292538	-0.000946044921875	\\
0.0700492742042882	-0.001129150390625	\\
0.0700936653793226	-0.0013427734375	\\
0.070138056554357	-0.00140380859375	\\
0.0701824477293914	-0.00140380859375	\\
0.0702268389044258	-0.0010986328125	\\
0.0702712300794602	-0.000732421875	\\
0.0703156212544946	-0.00091552734375	\\
0.070360012429529	-0.00103759765625	\\
0.0704044036045634	-0.00115966796875	\\
0.0704487947795978	-0.001129150390625	\\
0.0704931859546322	-0.001495361328125	\\
0.0705375771296666	-0.001434326171875	\\
0.070581968304701	-0.00103759765625	\\
0.0706263594797354	-0.001434326171875	\\
0.0706707506547698	-0.00177001953125	\\
0.0707151418298042	-0.00164794921875	\\
0.0707595330048386	-0.001617431640625	\\
0.070803924179873	-0.00152587890625	\\
0.0708483153549074	-0.001190185546875	\\
0.0708927065299419	-0.0018310546875	\\
0.0709370977049762	-0.00238037109375	\\
0.0709814888800107	-0.00189208984375	\\
0.0710258800550451	-0.00238037109375	\\
0.0710702712300795	-0.002471923828125	\\
0.0711146624051139	-0.002105712890625	\\
0.0711590535801483	-0.002166748046875	\\
0.0712034447551827	-0.00244140625	\\
0.0712478359302171	-0.002593994140625	\\
0.0712922271052515	-0.002471923828125	\\
0.0713366182802859	-0.00262451171875	\\
0.0713810094553203	-0.002838134765625	\\
0.0714254006303547	-0.002532958984375	\\
0.0714697918053891	-0.001983642578125	\\
0.0715141829804235	-0.001983642578125	\\
0.0715585741554579	-0.002166748046875	\\
0.0716029653304923	-0.001953125	\\
0.0716473565055267	-0.001953125	\\
0.0716917476805611	-0.00189208984375	\\
0.0717361388555955	-0.002044677734375	\\
0.0717805300306299	-0.00177001953125	\\
0.0718249212056643	-0.001068115234375	\\
0.0718693123806987	-0.001312255859375	\\
0.0719137035557331	-0.00146484375	\\
0.0719580947307675	-0.001251220703125	\\
0.0720024859058019	-0.001678466796875	\\
0.0720468770808363	-0.001312255859375	\\
0.0720912682558707	-0.0009765625	\\
0.0721356594309051	-0.001312255859375	\\
0.0721800506059395	-0.001068115234375	\\
0.0722244417809739	-0.000946044921875	\\
0.0722688329560083	-0.00115966796875	\\
0.0723132241310428	-0.00140380859375	\\
0.0723576153060772	-0.0013427734375	\\
0.0724020064811116	-0.001434326171875	\\
0.072446397656146	-0.00146484375	\\
0.0724907888311804	-0.000946044921875	\\
0.0725351800062148	-0.0009765625	\\
0.0725795711812492	-0.001190185546875	\\
0.0726239623562836	-0.001617431640625	\\
0.072668353531318	-0.001739501953125	\\
0.0727127447063524	-0.00146484375	\\
0.0727571358813868	-0.00189208984375	\\
0.0728015270564212	-0.001861572265625	\\
0.0728459182314556	-0.001617431640625	\\
0.07289030940649	-0.001739501953125	\\
0.0729347005815244	-0.001617431640625	\\
0.0729790917565588	-0.001373291015625	\\
0.0730234829315932	-0.001373291015625	\\
0.0730678741066276	-0.0018310546875	\\
0.073112265281662	-0.001922607421875	\\
0.0731566564566964	-0.001953125	\\
0.0732010476317308	-0.002197265625	\\
0.0732454388067652	-0.001922607421875	\\
0.0732898299817996	-0.002044677734375	\\
0.073334221156834	-0.00201416015625	\\
0.0733786123318684	-0.0018310546875	\\
0.0734230035069028	-0.002197265625	\\
0.0734673946819372	-0.0020751953125	\\
0.0735117858569716	-0.002166748046875	\\
0.073556177032006	-0.001861572265625	\\
0.0736005682070404	-0.001739501953125	\\
0.0736449593820748	-0.00177001953125	\\
0.0736893505571092	-0.001312255859375	\\
0.0737337417321437	-0.001312255859375	\\
0.0737781329071781	-0.001373291015625	\\
0.0738225240822125	-0.00152587890625	\\
0.0738669152572469	-0.00152587890625	\\
0.0739113064322813	-0.00103759765625	\\
0.0739556976073157	-0.001129150390625	\\
0.0740000887823501	-0.00091552734375	\\
0.0740444799573845	-0.00067138671875	\\
0.0740888711324189	-0.001007080078125	\\
0.0741332623074533	-0.00067138671875	\\
0.0741776534824877	-0.00054931640625	\\
0.0742220446575221	-0.000579833984375	\\
0.0742664358325565	-0.000640869140625	\\
0.0743108270075909	-0.000762939453125	\\
0.0743552181826253	-6.103515625e-05	\\
0.0743996093576597	-0.000335693359375	\\
0.0744440005326941	-0.00048828125	\\
0.0744883917077285	-0.000457763671875	\\
0.0745327828827629	-0.000457763671875	\\
0.0745771740577973	-0.000335693359375	\\
0.0746215652328317	3.0517578125e-05	\\
0.0746659564078661	-0.000213623046875	\\
0.0747103475829005	-0.000396728515625	\\
0.0747547387579349	-0.000457763671875	\\
0.0747991299329693	-0.00054931640625	\\
0.0748435211080037	-0.00067138671875	\\
0.0748879122830381	-0.00091552734375	\\
0.0749323034580725	-0.001251220703125	\\
0.0749766946331069	-0.0006103515625	\\
0.0750210858081413	-0.000396728515625	\\
0.0750654769831757	-0.000640869140625	\\
0.0751098681582102	-0.0006103515625	\\
0.0751542593332445	-0.000579833984375	\\
0.075198650508279	-0.000244140625	\\
0.0752430416833134	-3.0517578125e-05	\\
0.0752874328583478	9.1552734375e-05	\\
0.0753318240333822	0.000152587890625	\\
0.0753762152084166	-9.1552734375e-05	\\
0.075420606383451	0.000244140625	\\
0.0754649975584854	0.00018310546875	\\
0.0755093887335198	9.1552734375e-05	\\
0.0755537799085542	0.00048828125	\\
0.0755981710835886	0.000457763671875	\\
0.075642562258623	0.000396728515625	\\
0.0756869534336574	0.000701904296875	\\
0.0757313446086918	0.001129150390625	\\
0.0757757357837262	0.00091552734375	\\
0.0758201269587606	0.0010986328125	\\
0.075864518133795	0.001739501953125	\\
0.0759089093088294	0.00164794921875	\\
0.0759533004838638	0.001739501953125	\\
0.0759976916588982	0.001678466796875	\\
0.0760420828339326	0.00146484375	\\
0.076086474008967	0.001708984375	\\
0.0761308651840014	0.001617431640625	\\
0.0761752563590358	0.001617431640625	\\
0.0762196475340702	0.001861572265625	\\
0.0762640387091046	0.00146484375	\\
0.076308429884139	0.001953125	\\
0.0763528210591734	0.00225830078125	\\
0.0763972122342078	0.0015869140625	\\
0.0764416034092422	0.00128173828125	\\
0.0764859945842766	0.001007080078125	\\
0.0765303857593111	0.00067138671875	\\
0.0765747769343454	0.00079345703125	\\
0.0766191681093799	0.000823974609375	\\
0.0766635592844143	0.00054931640625	\\
0.0767079504594487	0.00030517578125	\\
0.0767523416344831	0.000244140625	\\
0.0767967328095175	0.000946044921875	\\
0.0768411239845519	0.001068115234375	\\
0.0768855151595863	0.001129150390625	\\
0.0769299063346207	0.0010986328125	\\
0.0769742975096551	0.001312255859375	\\
0.0770186886846895	0.0010986328125	\\
0.0770630798597239	0.000946044921875	\\
0.0771074710347583	0.0009765625	\\
0.0771518622097927	0.0001220703125	\\
0.0771962533848271	0.000274658203125	\\
0.0772406445598615	0.0006103515625	\\
0.0772850357348959	0.00091552734375	\\
0.0773294269099303	0.00091552734375	\\
0.0773738180849647	0.00048828125	\\
0.0774182092599991	0.0006103515625	\\
0.0774626004350335	0.000762939453125	\\
0.0775069916100679	0.00079345703125	\\
0.0775513827851023	0.001373291015625	\\
0.0775957739601367	0.0008544921875	\\
0.0776401651351711	0.000335693359375	\\
0.0776845563102055	0.000274658203125	\\
0.0777289474852399	3.0517578125e-05	\\
0.0777733386602743	0.00018310546875	\\
0.0778177298353087	0.00048828125	\\
0.0778621210103431	0.000640869140625	\\
0.0779065121853775	0.000640869140625	\\
0.077950903360412	0.0006103515625	\\
0.0779952945354463	0.00091552734375	\\
0.0780396857104808	0.00128173828125	\\
0.0780840768855152	0.0010986328125	\\
0.0781284680605496	0.0008544921875	\\
0.078172859235584	0.00042724609375	\\
0.0782172504106184	0.000396728515625	\\
0.0782616415856528	0.000244140625	\\
0.0783060327606872	0.00030517578125	\\
0.0783504239357216	0.0003662109375	\\
0.078394815110756	6.103515625e-05	\\
0.0784392062857904	0.000213623046875	\\
0.0784835974608248	0.000274658203125	\\
0.0785279886358592	0.000457763671875	\\
0.0785723798108936	0.000274658203125	\\
0.078616770985928	0.00079345703125	\\
0.0786611621609624	0.001007080078125	\\
0.0787055533359968	0.000885009765625	\\
0.0787499445110312	0.00115966796875	\\
0.0787943356860656	0.001312255859375	\\
0.0788387268611	0.00115966796875	\\
0.0788831180361344	0.0010986328125	\\
0.0789275092111688	0.000885009765625	\\
0.0789719003862032	0.000823974609375	\\
0.0790162915612376	0.001556396484375	\\
0.079060682736272	0.001708984375	\\
0.0791050739113064	0.00189208984375	\\
0.0791494650863408	0.00213623046875	\\
0.0791938562613752	0.002044677734375	\\
0.0792382474364096	0.00189208984375	\\
0.0792826386114441	0.001708984375	\\
0.0793270297864784	0.001495361328125	\\
0.0793714209615129	0.001556396484375	\\
0.0794158121365472	0.00140380859375	\\
0.0794602033115817	0.0015869140625	\\
0.0795045944866161	0.00164794921875	\\
0.0795489856616505	0.00189208984375	\\
0.0795933768366849	0.002105712890625	\\
0.0796377680117193	0.00164794921875	\\
0.0796821591867537	0.0018310546875	\\
0.0797265503617881	0.0015869140625	\\
0.0797709415368225	0.001800537109375	\\
0.0798153327118569	0.0018310546875	\\
0.0798597238868913	0.001556396484375	\\
0.0799041150619257	0.00140380859375	\\
0.0799485062369601	0.001251220703125	\\
0.0799928974119945	0.0010986328125	\\
0.0800372885870289	0.001190185546875	\\
0.0800816797620633	0.001251220703125	\\
0.0801260709370977	0.001007080078125	\\
0.0801704621121321	0.001434326171875	\\
0.0802148532871665	0.00152587890625	\\
0.0802592444622009	0.001434326171875	\\
0.0803036356372353	0.001312255859375	\\
0.0803480268122697	0.001068115234375	\\
0.0803924179873041	0.001373291015625	\\
0.0804368091623385	0.00091552734375	\\
0.0804812003373729	0.000946044921875	\\
0.0805255915124073	0.0013427734375	\\
0.0805699826874417	0.001129150390625	\\
0.0806143738624761	0.001129150390625	\\
0.0806587650375105	0.0013427734375	\\
0.080703156212545	0.0015869140625	\\
0.0807475473875793	0.00140380859375	\\
0.0807919385626138	0.00128173828125	\\
0.0808363297376481	0.0010986328125	\\
0.0808807209126826	0.000762939453125	\\
0.080925112087717	0.000518798828125	\\
0.0809695032627514	0.000152587890625	\\
0.0810138944377858	0.000396728515625	\\
0.0810582856128202	0.000701904296875	\\
0.0811026767878546	0.000457763671875	\\
0.081147067962889	0.000946044921875	\\
0.0811914591379234	0.001251220703125	\\
0.0812358503129578	0.00079345703125	\\
0.0812802414879922	0.000640869140625	\\
0.0813246326630266	0.000701904296875	\\
0.081369023838061	0.001068115234375	\\
0.0814134150130954	0.0010986328125	\\
0.0814578061881298	0.000762939453125	\\
0.0815021973631642	0.000823974609375	\\
0.0815465885381986	0.000885009765625	\\
0.081590979713233	0.001007080078125	\\
0.0816353708882674	0.000762939453125	\\
0.0816797620633018	0.000946044921875	\\
0.0817241532383362	0.0009765625	\\
0.0817685444133706	0.001220703125	\\
0.081812935588405	0.00115966796875	\\
0.0818573267634394	0.001129150390625	\\
0.0819017179384738	0.001220703125	\\
0.0819461091135082	0.001373291015625	\\
0.0819905002885426	0.001495361328125	\\
0.082034891463577	0.00146484375	\\
0.0820792826386114	0.001068115234375	\\
0.0821236738136459	0.001312255859375	\\
0.0821680649886802	0.001373291015625	\\
0.0822124561637147	0.000946044921875	\\
0.0822568473387491	0.00091552734375	\\
0.0823012385137835	0.001190185546875	\\
0.0823456296888179	0.001373291015625	\\
0.0823900208638523	0.001312255859375	\\
0.0824344120388867	0.0010986328125	\\
0.0824788032139211	0.00091552734375	\\
0.0825231943889555	0.001068115234375	\\
0.0825675855639899	0.001190185546875	\\
0.0826119767390243	0.0013427734375	\\
0.0826563679140587	0.0013427734375	\\
0.0827007590890931	0.001220703125	\\
0.0827451502641275	0.0010986328125	\\
0.0827895414391619	0.00103759765625	\\
0.0828339326141963	0.00115966796875	\\
0.0828783237892307	0.000946044921875	\\
0.0829227149642651	0.001129150390625	\\
0.0829671061392995	0.001190185546875	\\
0.0830114973143339	0.00091552734375	\\
0.0830558884893683	0.0009765625	\\
0.0831002796644027	0.0009765625	\\
0.0831446708394371	0.0008544921875	\\
0.0831890620144715	0.0009765625	\\
0.0832334531895059	0.0009765625	\\
0.0832778443645403	0.00079345703125	\\
0.0833222355395747	0.001068115234375	\\
0.0833666267146091	0.001220703125	\\
0.0834110178896435	0.001251220703125	\\
0.0834554090646779	0.001129150390625	\\
0.0834998002397123	0.000823974609375	\\
0.0835441914147468	0.000640869140625	\\
0.0835885825897811	0.00067138671875	\\
0.0836329737648156	0.00067138671875	\\
0.08367736493985	0.0003662109375	\\
0.0837217561148844	0.000946044921875	\\
0.0837661472899188	0.00079345703125	\\
0.0838105384649532	0.00030517578125	\\
0.0838549296399876	0.00091552734375	\\
0.083899320815022	0.000732421875	\\
0.0839437119900564	0.00048828125	\\
0.0839881031650908	0.00091552734375	\\
0.0840324943401252	0.000885009765625	\\
0.0840768855151596	0.0008544921875	\\
0.084121276690194	0.000640869140625	\\
0.0841656678652284	0.000762939453125	\\
0.0842100590402628	0.0009765625	\\
0.0842544502152972	0.000640869140625	\\
0.0842988413903316	0.000579833984375	\\
0.084343232565366	0.00042724609375	\\
0.0843876237404004	0.000244140625	\\
0.0844320149154348	0.000579833984375	\\
0.0844764060904692	0.000885009765625	\\
0.0845207972655036	0.0009765625	\\
0.084565188440538	0.000946044921875	\\
0.0846095796155724	0.00048828125	\\
0.0846539707906068	0.000274658203125	\\
0.0846983619656412	0.000274658203125	\\
0.0847427531406756	0.000335693359375	\\
0.08478714431571	0.000244140625	\\
0.0848315354907444	9.1552734375e-05	\\
0.0848759266657788	0.000152587890625	\\
0.0849203178408132	-0.000152587890625	\\
0.0849647090158477	-0.00018310546875	\\
0.0850091001908821	-0.000244140625	\\
0.0850534913659165	-0.00018310546875	\\
0.0850978825409509	0.000213623046875	\\
0.0851422737159853	0.00018310546875	\\
0.0851866648910197	0.00042724609375	\\
0.0852310560660541	0.000335693359375	\\
0.0852754472410885	0.000213623046875	\\
0.0853198384161229	0.000335693359375	\\
0.0853642295911573	0	\\
0.0854086207661917	9.1552734375e-05	\\
0.0854530119412261	3.0517578125e-05	\\
0.0854974031162605	-0.000244140625	\\
0.0855417942912949	0	\\
0.0855861854663293	6.103515625e-05	\\
0.0856305766413637	-6.103515625e-05	\\
0.0856749678163981	3.0517578125e-05	\\
0.0857193589914325	9.1552734375e-05	\\
0.0857637501664669	9.1552734375e-05	\\
0.0858081413415013	-0.000213623046875	\\
0.0858525325165357	-3.0517578125e-05	\\
0.0858969236915701	-6.103515625e-05	\\
0.0859413148666045	-0.000335693359375	\\
0.0859857060416389	3.0517578125e-05	\\
0.0860300972166733	0.000274658203125	\\
0.0860744883917077	-0.0001220703125	\\
0.0861188795667421	-0.000274658203125	\\
0.0861632707417765	-0.00048828125	\\
0.0862076619168109	-0.000640869140625	\\
0.0862520530918453	-0.00079345703125	\\
0.0862964442668797	-0.00128173828125	\\
0.0863408354419141	-0.001434326171875	\\
0.0863852266169486	-0.001556396484375	\\
0.086429617791983	-0.00164794921875	\\
0.0864740089670174	-0.001495361328125	\\
0.0865184001420518	-0.00146484375	\\
0.0865627913170862	-0.0009765625	\\
0.0866071824921206	-0.000732421875	\\
0.086651573667155	-0.00103759765625	\\
0.0866959648421894	-0.00079345703125	\\
0.0867403560172238	-0.000518798828125	\\
0.0867847471922582	-0.001251220703125	\\
0.0868291383672926	-0.001129150390625	\\
0.086873529542327	-0.0010986328125	\\
0.0869179207173614	-0.00140380859375	\\
0.0869623118923958	-0.00079345703125	\\
0.0870067030674302	-0.001129150390625	\\
0.0870510942424646	-0.001312255859375	\\
0.087095485417499	-0.0010986328125	\\
0.0871398765925334	-0.001007080078125	\\
0.0871842677675678	-0.0010986328125	\\
0.0872286589426022	-0.0008544921875	\\
0.0872730501176366	-0.000701904296875	\\
0.087317441292671	-0.000762939453125	\\
0.0873618324677054	-0.000762939453125	\\
0.0874062236427398	-0.000732421875	\\
0.0874506148177742	-0.00042724609375	\\
0.0874950059928086	-0.0003662109375	\\
0.087539397167843	-0.0001220703125	\\
0.0875837883428774	-0.00048828125	\\
0.0876281795179118	-0.00030517578125	\\
0.0876725706929462	-0.000396728515625	\\
0.0877169618679806	-0.000823974609375	\\
0.0877613530430151	-0.000640869140625	\\
0.0878057442180495	-0.000518798828125	\\
0.0878501353930839	-0.000396728515625	\\
0.0878945265681183	-0.000244140625	\\
0.0879389177431527	-0.000213623046875	\\
0.0879833089181871	-0.000396728515625	\\
0.0880277000932215	-0.000518798828125	\\
0.0880720912682559	-0.00030517578125	\\
0.0881164824432903	-0.00054931640625	\\
0.0881608736183247	-0.000244140625	\\
0.0882052647933591	-0.00048828125	\\
0.0882496559683935	-0.00067138671875	\\
0.0882940471434279	-0.0003662109375	\\
0.0883384383184623	-0.000518798828125	\\
0.0883828294934967	-0.000640869140625	\\
0.0884272206685311	-0.000762939453125	\\
0.0884716118435655	-0.000762939453125	\\
0.0885160030185999	-0.00067138671875	\\
0.0885603941936343	-0.000518798828125	\\
0.0886047853686687	-0.000213623046875	\\
0.0886491765437031	-0.000244140625	\\
0.0886935677187375	-6.103515625e-05	\\
0.0887379588937719	-0.000518798828125	\\
0.0887823500688063	-0.00042724609375	\\
0.0888267412438407	-0.000457763671875	\\
0.0888711324188751	-0.000640869140625	\\
0.0889155235939095	-0.000244140625	\\
0.0889599147689439	-0.000823974609375	\\
0.0890043059439783	-0.00103759765625	\\
0.0890486971190127	-0.00091552734375	\\
0.0890930882940471	-0.001129150390625	\\
0.0891374794690815	-0.001220703125	\\
0.089181870644116	-0.001068115234375	\\
0.0892262618191504	-0.00091552734375	\\
0.0892706529941848	-0.001007080078125	\\
0.0893150441692192	-0.001373291015625	\\
0.0893594353442536	-0.001861572265625	\\
0.089403826519288	-0.001617431640625	\\
0.0894482176943224	-0.00146484375	\\
0.0894926088693568	-0.0015869140625	\\
0.0895370000443912	-0.0013427734375	\\
0.0895813912194256	-0.000701904296875	\\
0.08962578239446	-0.00128173828125	\\
0.0896701735694944	-0.001434326171875	\\
0.0897145647445288	-0.0009765625	\\
0.0897589559195632	-0.0015869140625	\\
0.0898033470945976	-0.00128173828125	\\
0.089847738269632	-0.00115966796875	\\
0.0898921294446664	-0.00152587890625	\\
0.0899365206197008	-0.00115966796875	\\
0.0899809117947352	-0.001434326171875	\\
0.0900253029697696	-0.0013427734375	\\
0.090069694144804	-0.00067138671875	\\
0.0901140853198384	-0.0009765625	\\
0.0901584764948728	-0.00091552734375	\\
0.0902028676699072	-0.001007080078125	\\
0.0902472588449416	-0.001190185546875	\\
0.090291650019976	-0.0009765625	\\
0.0903360411950104	-0.000946044921875	\\
0.0903804323700448	-0.0009765625	\\
0.0904248235450792	-0.0009765625	\\
0.0904692147201136	-0.00067138671875	\\
0.0905136058951481	-0.000518798828125	\\
0.0905579970701824	-0.000762939453125	\\
0.0906023882452169	-0.00079345703125	\\
0.0906467794202512	-0.000701904296875	\\
0.0906911705952857	-0.0003662109375	\\
0.0907355617703201	-0.000244140625	\\
0.0907799529453545	-0.000946044921875	\\
0.0908243441203889	-0.00048828125	\\
0.0908687352954233	6.103515625e-05	\\
0.0909131264704577	-0.0001220703125	\\
0.0909575176454921	-0.0001220703125	\\
0.0910019088205265	-0.000335693359375	\\
0.0910462999955609	0.0001220703125	\\
0.0910906911705953	-0.000152587890625	\\
0.0911350823456297	-0.00030517578125	\\
0.0911794735206641	-3.0517578125e-05	\\
0.0912238646956985	0.000152587890625	\\
0.0912682558707329	0.000152587890625	\\
0.0913126470457673	0.000213623046875	\\
0.0913570382208017	3.0517578125e-05	\\
0.0914014293958361	0.0001220703125	\\
0.0914458205708705	0.00030517578125	\\
0.0914902117459049	0.000762939453125	\\
0.0915346029209393	0.00103759765625	\\
0.0915789940959737	0.000457763671875	\\
0.0916233852710081	0.000762939453125	\\
0.0916677764460425	0.00091552734375	\\
0.0917121676210769	6.103515625e-05	\\
0.0917565587961113	0.0003662109375	\\
0.0918009499711457	0.000518798828125	\\
0.0918453411461801	0.0003662109375	\\
0.0918897323212145	-0.0001220703125	\\
0.091934123496249	-3.0517578125e-05	\\
0.0919785146712833	0.000213623046875	\\
0.0920229058463178	0.0001220703125	\\
0.0920672970213521	0.000274658203125	\\
0.0921116881963866	0	\\
0.092156079371421	3.0517578125e-05	\\
0.0922004705464554	-0.0001220703125	\\
0.0922448617214898	-0.00048828125	\\
0.0922892528965242	-0.00048828125	\\
0.0923336440715586	-0.00054931640625	\\
0.092378035246593	-0.00067138671875	\\
0.0924224264216274	-0.000579833984375	\\
0.0924668175966618	-0.000396728515625	\\
0.0925112087716962	-0.00042724609375	\\
0.0925555999467306	-0.00048828125	\\
0.092599991121765	-0.000335693359375	\\
0.0926443822967994	-0.000335693359375	\\
0.0926887734718338	-0.000732421875	\\
0.0927331646468682	-0.000396728515625	\\
0.0927775558219026	-0.000457763671875	\\
0.092821946996937	-0.00067138671875	\\
0.0928663381719714	-0.000701904296875	\\
0.0929107293470058	-0.0008544921875	\\
0.0929551205220402	-0.00067138671875	\\
0.0929995116970746	-0.00042724609375	\\
0.093043902872109	-0.00018310546875	\\
0.0930882940471434	-0.000457763671875	\\
0.0931326852221778	-0.00054931640625	\\
0.0931770763972122	-3.0517578125e-05	\\
0.0932214675722466	-0.00030517578125	\\
0.093265858747281	-0.000396728515625	\\
0.0933102499223154	-0.00030517578125	\\
0.0933546410973499	-0.00042724609375	\\
0.0933990322723842	-0.000762939453125	\\
0.0934434234474187	-0.000762939453125	\\
0.0934878146224531	-0.000274658203125	\\
0.0935322057974875	-0.000518798828125	\\
0.0935765969725219	-0.000579833984375	\\
0.0936209881475563	-0.0003662109375	\\
0.0936653793225907	-0.00042724609375	\\
0.0937097704976251	-0.000213623046875	\\
0.0937541616726595	-3.0517578125e-05	\\
0.0937985528476939	-0.000579833984375	\\
0.0938429440227283	-0.00048828125	\\
0.0938873351977627	-0.00042724609375	\\
0.0939317263727971	-0.000457763671875	\\
0.0939761175478315	-0.00042724609375	\\
0.0940205087228659	-0.000335693359375	\\
0.0940648998979003	-0.00030517578125	\\
0.0941092910729347	-0.000213623046875	\\
0.0941536822479691	-0.00030517578125	\\
0.0941980734230035	-0.0003662109375	\\
0.0942424645980379	-0.000152587890625	\\
0.0942868557730723	-0.000640869140625	\\
0.0943312469481067	-0.000823974609375	\\
0.0943756381231411	-0.00091552734375	\\
0.0944200292981755	-0.001068115234375	\\
0.0944644204732099	-0.001373291015625	\\
0.0945088116482443	-0.001190185546875	\\
0.0945532028232787	-0.000579833984375	\\
0.0945975939983131	-0.0010986328125	\\
0.0946419851733475	-0.00128173828125	\\
0.0946863763483819	-0.00140380859375	\\
0.0947307675234163	-0.0015869140625	\\
0.0947751586984508	-0.0015869140625	\\
0.0948195498734851	-0.001495361328125	\\
0.0948639410485196	-0.00164794921875	\\
0.094908332223554	-0.00201416015625	\\
0.0949527233985884	-0.001678466796875	\\
0.0949971145736228	-0.001617431640625	\\
0.0950415057486572	-0.00146484375	\\
0.0950858969236916	-0.00152587890625	\\
0.095130288098726	-0.0018310546875	\\
0.0951746792737604	-0.001068115234375	\\
0.0952190704487948	-0.001129150390625	\\
0.0952634616238292	-0.001434326171875	\\
0.0953078527988636	-0.000762939453125	\\
0.095352243973898	-0.00067138671875	\\
0.0953966351489324	-0.000732421875	\\
0.0954410263239668	-0.0006103515625	\\
0.0954854174990012	-0.00115966796875	\\
0.0955298086740356	-0.00128173828125	\\
0.09557419984907	-0.000885009765625	\\
0.0956185910241044	-0.00128173828125	\\
0.0956629821991388	-0.001068115234375	\\
0.0957073733741732	-0.000885009765625	\\
0.0957517645492076	-0.001495361328125	\\
0.095796155724242	-0.00146484375	\\
0.0958405468992764	-0.00140380859375	\\
0.0958849380743108	-0.001129150390625	\\
0.0959293292493452	-0.001068115234375	\\
0.0959737204243796	-0.00067138671875	\\
0.096018111599414	-0.000885009765625	\\
0.0960625027744484	-0.001220703125	\\
0.0961068939494828	-0.000885009765625	\\
0.0961512851245172	-0.000701904296875	\\
0.0961956762995517	-0.00079345703125	\\
0.0962400674745861	-0.0006103515625	\\
0.0962844586496205	-0.00018310546875	\\
0.0963288498246549	-0.000457763671875	\\
0.0963732409996893	-0.000335693359375	\\
0.0964176321747237	0.0001220703125	\\
0.0964620233497581	-0.00030517578125	\\
0.0965064145247925	-0.000244140625	\\
0.0965508056998269	-0.00030517578125	\\
0.0965951968748613	-0.0006103515625	\\
0.0966395880498957	-0.000732421875	\\
0.0966839792249301	-0.0008544921875	\\
0.0967283703999645	-0.000579833984375	\\
0.0967727615749989	-0.000457763671875	\\
0.0968171527500333	-0.001007080078125	\\
0.0968615439250677	-0.001220703125	\\
0.0969059351001021	-0.000946044921875	\\
0.0969503262751365	-0.001251220703125	\\
0.0969947174501709	-0.00146484375	\\
0.0970391086252053	-0.00115966796875	\\
0.0970834998002397	-0.00067138671875	\\
0.0971278909752741	-0.0003662109375	\\
0.0971722821503085	-0.000244140625	\\
0.0972166733253429	-0.0006103515625	\\
0.0972610645003773	-0.000579833984375	\\
0.0973054556754117	-0.00042724609375	\\
0.0973498468504461	-0.000274658203125	\\
0.0973942380254805	-0.0001220703125	\\
0.0974386292005149	-0.00030517578125	\\
0.0974830203755493	-0.00018310546875	\\
0.0975274115505837	0.000396728515625	\\
0.0975718027256181	0.00030517578125	\\
0.0976161939006526	0.00018310546875	\\
0.097660585075687	0.000335693359375	\\
0.0977049762507214	0.000244140625	\\
0.0977493674257558	9.1552734375e-05	\\
0.0977937586007902	0	\\
0.0978381497758246	0.00018310546875	\\
0.097882540950859	-6.103515625e-05	\\
0.0979269321258934	6.103515625e-05	\\
0.0979713233009278	0.0001220703125	\\
0.0980157144759622	0.000640869140625	\\
0.0980601056509966	0.000640869140625	\\
0.098104496826031	0.000396728515625	\\
0.0981488880010654	0.000823974609375	\\
0.0981932791760998	0.0013427734375	\\
0.0982376703511342	0.001190185546875	\\
0.0982820615261686	0.00128173828125	\\
0.098326452701203	0.001373291015625	\\
0.0983708438762374	0.00128173828125	\\
0.0984152350512718	0.001495361328125	\\
0.0984596262263062	0.00140380859375	\\
0.0985040174013406	0.001251220703125	\\
0.098548408576375	0.0008544921875	\\
0.0985927997514094	0.001007080078125	\\
0.0986371909264438	0.001220703125	\\
0.0986815821014782	0.00079345703125	\\
0.0987259732765126	0.0009765625	\\
0.098770364451547	0.0013427734375	\\
0.0988147556265814	0.0008544921875	\\
0.0988591468016158	0.000946044921875	\\
0.0989035379766502	0.000732421875	\\
0.0989479291516846	0.000457763671875	\\
0.098992320326719	0.00054931640625	\\
0.0990367115017535	0.00054931640625	\\
0.0990811026767879	0.00048828125	\\
0.0991254938518223	0.000457763671875	\\
0.0991698850268567	0.00054931640625	\\
0.0992142762018911	0.000946044921875	\\
0.0992586673769255	0.000946044921875	\\
0.0993030585519599	0.000885009765625	\\
0.0993474497269943	0.001373291015625	\\
0.0993918409020287	0.0010986328125	\\
0.0994362320770631	0.000885009765625	\\
0.0994806232520975	0.001190185546875	\\
0.0995250144271319	0.001129150390625	\\
0.0995694056021663	0.001007080078125	\\
0.0996137967772007	0.00091552734375	\\
0.0996581879522351	0.00128173828125	\\
0.0997025791272695	0.00140380859375	\\
0.0997469703023039	0.001220703125	\\
0.0997913614773383	0.0013427734375	\\
0.0998357526523727	0.00152587890625	\\
0.0998801438274071	0.001495361328125	\\
0.0999245350024415	0.00152587890625	\\
0.0999689261774759	0.001556396484375	\\
0.10001331735251	0.0015869140625	\\
0.100057708527545	0.001556396484375	\\
0.100102099702579	0.001708984375	\\
0.100146490877614	0.001983642578125	\\
0.100190882052648	0.001861572265625	\\
0.100235273227682	0.001678466796875	\\
0.100279664402717	0.002044677734375	\\
0.100324055577751	0.002166748046875	\\
0.100368446752786	0.001739501953125	\\
0.10041283792782	0.00140380859375	\\
0.100457229102854	0.00146484375	\\
0.100501620277889	0.001861572265625	\\
0.100546011452923	0.00189208984375	\\
0.100590402627958	0.001861572265625	\\
0.100634793802992	0.001434326171875	\\
0.100679184978026	0.0015869140625	\\
0.100723576153061	0.001434326171875	\\
0.100767967328095	0.000885009765625	\\
0.10081235850313	0.0013427734375	\\
0.100856749678164	0.00177001953125	\\
0.100901140853198	0.00128173828125	\\
0.100945532028233	0.001495361328125	\\
0.100989923203267	0.001617431640625	\\
0.101034314378302	0.00128173828125	\\
0.101078705553336	0.001312255859375	\\
0.10112309672837	0.001068115234375	\\
0.101167487903405	0.00103759765625	\\
0.101211879078439	0.001068115234375	\\
0.101256270253474	0.001251220703125	\\
0.101300661428508	0.0015869140625	\\
0.101345052603542	0.001495361328125	\\
0.101389443778577	0.001007080078125	\\
0.101433834953611	0.000946044921875	\\
0.101478226128646	0.001220703125	\\
0.10152261730368	0.001068115234375	\\
0.101567008478714	0.00115966796875	\\
0.101611399653749	0.00140380859375	\\
0.101655790828783	0.001495361328125	\\
0.101700182003818	0.001373291015625	\\
0.101744573178852	0.001922607421875	\\
0.101788964353886	0.001739501953125	\\
0.101833355528921	0.00128173828125	\\
0.101877746703955	0.001434326171875	\\
0.10192213787899	0.001434326171875	\\
0.101966529054024	0.001312255859375	\\
0.102010920229058	0.001220703125	\\
0.102055311404093	0.00103759765625	\\
0.102099702579127	0.0010986328125	\\
0.102144093754162	0.0010986328125	\\
0.102188484929196	0.0008544921875	\\
0.10223287610423	0.000885009765625	\\
0.102277267279265	0.001068115234375	\\
0.102321658454299	0.001007080078125	\\
0.102366049629334	0.0010986328125	\\
0.102410440804368	0.000885009765625	\\
0.102454831979403	0.00067138671875	\\
0.102499223154437	0.000640869140625	\\
0.102543614329471	0.000457763671875	\\
0.102588005504506	0.00030517578125	\\
0.10263239667954	0.000274658203125	\\
0.102676787854575	6.103515625e-05	\\
0.102721179029609	-0.00018310546875	\\
0.102765570204643	-0.00018310546875	\\
0.102809961379678	-0.00048828125	\\
0.102854352554712	-0.000457763671875	\\
0.102898743729747	-0.000640869140625	\\
0.102943134904781	-0.00054931640625	\\
0.102987526079815	-0.000244140625	\\
0.10303191725485	-9.1552734375e-05	\\
0.103076308429884	-0.00018310546875	\\
0.103120699604919	-0.00054931640625	\\
0.103165090779953	-0.000640869140625	\\
0.103209481954987	-0.000518798828125	\\
0.103253873130022	-0.00048828125	\\
0.103298264305056	-0.000335693359375	\\
0.103342655480091	-0.00042724609375	\\
0.103387046655125	-0.000701904296875	\\
0.103431437830159	-0.000885009765625	\\
0.103475829005194	-0.00079345703125	\\
0.103520220180228	-0.000396728515625	\\
0.103564611355263	-0.000274658203125	\\
0.103609002530297	-0.000457763671875	\\
0.103653393705331	-0.0006103515625	\\
0.103697784880366	-0.00042724609375	\\
0.1037421760554	-0.00042724609375	\\
0.103786567230435	-0.0003662109375	\\
0.103830958405469	-0.00054931640625	\\
0.103875349580503	-0.000396728515625	\\
0.103919740755538	-0.000335693359375	\\
0.103964131930572	-0.000732421875	\\
0.104008523105607	-0.00048828125	\\
0.104052914280641	-0.00054931640625	\\
0.104097305455675	-0.000701904296875	\\
0.10414169663071	-0.000823974609375	\\
0.104186087805744	-0.001007080078125	\\
0.104230478980779	-0.000762939453125	\\
0.104274870155813	-0.00079345703125	\\
0.104319261330847	-0.000885009765625	\\
0.104363652505882	-0.00115966796875	\\
0.104408043680916	-0.0013427734375	\\
0.104452434855951	-0.001190185546875	\\
0.104496826030985	-0.001312255859375	\\
0.104541217206019	-0.00128173828125	\\
0.104585608381054	-0.001251220703125	\\
0.104629999556088	-0.00115966796875	\\
0.104674390731123	-0.00103759765625	\\
0.104718781906157	-0.000732421875	\\
0.104763173081191	-0.000640869140625	\\
0.104807564256226	-0.00091552734375	\\
0.10485195543126	-0.000946044921875	\\
0.104896346606295	-0.000762939453125	\\
0.104940737781329	-0.00067138671875	\\
0.104985128956363	-0.0001220703125	\\
0.105029520131398	0.000152587890625	\\
0.105073911306432	-0.00054931640625	\\
0.105118302481467	-0.00048828125	\\
0.105162693656501	-0.000579833984375	\\
0.105207084831535	-0.000274658203125	\\
0.10525147600657	3.0517578125e-05	\\
0.105295867181604	-0.000244140625	\\
0.105340258356639	-0.0001220703125	\\
0.105384649531673	9.1552734375e-05	\\
0.105429040706708	0.000244140625	\\
0.105473431881742	0.00054931640625	\\
0.105517823056776	-6.103515625e-05	\\
0.105562214231811	9.1552734375e-05	\\
0.105606605406845	0.00042724609375	\\
0.10565099658188	0.000244140625	\\
0.105695387756914	0.000457763671875	\\
0.105739778931948	3.0517578125e-05	\\
0.105784170106983	-0.0001220703125	\\
0.105828561282017	0.000213623046875	\\
0.105872952457052	-0.000213623046875	\\
0.105917343632086	-0.00030517578125	\\
0.10596173480712	-0.000396728515625	\\
0.106006125982155	-0.00079345703125	\\
0.106050517157189	-0.00054931640625	\\
0.106094908332224	-0.000457763671875	\\
0.106139299507258	-0.0001220703125	\\
0.106183690682292	-0.00054931640625	\\
0.106228081857327	-0.001068115234375	\\
0.106272473032361	-0.00128173828125	\\
0.106316864207396	-0.00103759765625	\\
0.10636125538243	-0.000762939453125	\\
0.106405646557464	-0.000701904296875	\\
0.106450037732499	-0.000823974609375	\\
0.106494428907533	-0.00103759765625	\\
0.106538820082568	-0.00115966796875	\\
0.106583211257602	-0.000823974609375	\\
0.106627602432636	-0.000946044921875	\\
0.106671993607671	-0.00115966796875	\\
0.106716384782705	-0.000885009765625	\\
0.10676077595774	-0.001190185546875	\\
0.106805167132774	-0.00146484375	\\
0.106849558307808	-0.001129150390625	\\
0.106893949482843	-0.00140380859375	\\
0.106938340657877	-0.0013427734375	\\
0.106982731832912	-0.001556396484375	\\
0.107027123007946	-0.001220703125	\\
0.10707151418298	-0.000640869140625	\\
0.107115905358015	-0.001556396484375	\\
0.107160296533049	-0.00152587890625	\\
0.107204687708084	-0.0010986328125	\\
0.107249078883118	-0.001190185546875	\\
0.107293470058152	-0.0008544921875	\\
0.107337861233187	-0.0009765625	\\
0.107382252408221	-0.001129150390625	\\
0.107426643583256	-0.000732421875	\\
0.10747103475829	-0.0008544921875	\\
0.107515425933324	-0.00091552734375	\\
0.107559817108359	-0.001129150390625	\\
0.107604208283393	-0.00146484375	\\
0.107648599458428	-0.000823974609375	\\
0.107692990633462	-0.0010986328125	\\
0.107737381808496	-0.00140380859375	\\
0.107781772983531	-0.0008544921875	\\
0.107826164158565	-0.00091552734375	\\
0.1078705553336	-0.00115966796875	\\
0.107914946508634	-0.001312255859375	\\
0.107959337683668	-0.001068115234375	\\
0.108003728858703	-0.00067138671875	\\
0.108048120033737	-0.00115966796875	\\
0.108092511208772	-0.00091552734375	\\
0.108136902383806	-0.00067138671875	\\
0.108181293558841	-0.000823974609375	\\
0.108225684733875	-0.00048828125	\\
0.108270075908909	-0.00067138671875	\\
0.108314467083944	-0.001129150390625	\\
0.108358858258978	-0.0010986328125	\\
0.108403249434013	-0.001190185546875	\\
0.108447640609047	-0.001495361328125	\\
0.108492031784081	-0.001953125	\\
0.108536422959116	-0.001861572265625	\\
0.10858081413415	-0.001739501953125	\\
0.108625205309185	-0.00152587890625	\\
0.108669596484219	-0.001495361328125	\\
0.108713987659253	-0.001678466796875	\\
0.108758378834288	-0.00152587890625	\\
0.108802770009322	-0.00164794921875	\\
0.108847161184357	-0.001312255859375	\\
0.108891552359391	-0.00140380859375	\\
0.108935943534425	-0.001800537109375	\\
0.10898033470946	-0.001495361328125	\\
0.109024725884494	-0.00128173828125	\\
0.109069117059529	-0.001495361328125	\\
0.109113508234563	-0.0013427734375	\\
0.109157899409597	-0.000823974609375	\\
0.109202290584632	-0.0009765625	\\
0.109246681759666	-0.001068115234375	\\
0.109291072934701	-0.001190185546875	\\
0.109335464109735	-0.001678466796875	\\
0.109379855284769	-0.001373291015625	\\
0.109424246459804	-0.001190185546875	\\
0.109468637634838	-0.001739501953125	\\
0.109513028809873	-0.00146484375	\\
0.109557419984907	-0.001617431640625	\\
0.109601811159941	-0.001922607421875	\\
0.109646202334976	-0.001312255859375	\\
0.10969059351001	-0.00146484375	\\
0.109734984685045	-0.001495361328125	\\
0.109779375860079	-0.00091552734375	\\
0.109823767035113	-0.00128173828125	\\
0.109868158210148	-0.002105712890625	\\
0.109912549385182	-0.0018310546875	\\
0.109956940560217	-0.001556396484375	\\
0.110001331735251	-0.002105712890625	\\
0.110045722910285	-0.00201416015625	\\
0.11009011408532	-0.00189208984375	\\
0.110134505260354	-0.00177001953125	\\
0.110178896435389	-0.001678466796875	\\
0.110223287610423	-0.001556396484375	\\
0.110267678785457	-0.001495361328125	\\
0.110312069960492	-0.001617431640625	\\
0.110356461135526	-0.00189208984375	\\
0.110400852310561	-0.0018310546875	\\
0.110445243485595	-0.001678466796875	\\
0.110489634660629	-0.0018310546875	\\
0.110534025835664	-0.001922607421875	\\
0.110578417010698	-0.0020751953125	\\
0.110622808185733	-0.002197265625	\\
0.110667199360767	-0.00201416015625	\\
0.110711590535801	-0.00201416015625	\\
0.110755981710836	-0.002197265625	\\
0.11080037288587	-0.00177001953125	\\
0.110844764060905	-0.001861572265625	\\
0.110889155235939	-0.0020751953125	\\
0.110933546410974	-0.00164794921875	\\
0.110977937586008	-0.00189208984375	\\
0.111022328761042	-0.001800537109375	\\
0.111066719936077	-0.001678466796875	\\
0.111111111111111	-0.00152587890625	\\
0.111155502286146	-0.00152587890625	\\
0.11119989346118	-0.00140380859375	\\
0.111244284636214	-0.001190185546875	\\
0.111288675811249	-0.00128173828125	\\
0.111333066986283	-0.00128173828125	\\
0.111377458161318	-0.001373291015625	\\
0.111421849336352	-0.001251220703125	\\
0.111466240511386	-0.001251220703125	\\
0.111510631686421	-0.001373291015625	\\
0.111555022861455	-0.001190185546875	\\
0.11159941403649	-0.00140380859375	\\
0.111643805211524	-0.001251220703125	\\
0.111688196386558	-0.000946044921875	\\
0.111732587561593	-0.00128173828125	\\
0.111776978736627	-0.001434326171875	\\
0.111821369911662	-0.001373291015625	\\
0.111865761086696	-0.00115966796875	\\
0.11191015226173	-0.00115966796875	\\
0.111954543436765	-0.0018310546875	\\
0.111998934611799	-0.002105712890625	\\
0.112043325786834	-0.001678466796875	\\
0.112087716961868	-0.0015869140625	\\
0.112132108136902	-0.001739501953125	\\
0.112176499311937	-0.001708984375	\\
0.112220890486971	-0.001861572265625	\\
0.112265281662006	-0.001800537109375	\\
0.11230967283704	-0.00128173828125	\\
0.112354064012074	-0.00164794921875	\\
0.112398455187109	-0.0015869140625	\\
0.112442846362143	-0.001312255859375	\\
0.112487237537178	-0.001495361328125	\\
0.112531628712212	-0.001617431640625	\\
0.112576019887246	-0.00177001953125	\\
0.112620411062281	-0.00213623046875	\\
0.112664802237315	-0.002166748046875	\\
0.11270919341235	-0.001953125	\\
0.112753584587384	-0.00238037109375	\\
0.112797975762418	-0.00225830078125	\\
0.112842366937453	-0.00164794921875	\\
0.112886758112487	-0.001922607421875	\\
0.112931149287522	-0.00177001953125	\\
0.112975540462556	-0.00146484375	\\
0.11301993163759	-0.001861572265625	\\
0.113064322812625	-0.00189208984375	\\
0.113108713987659	-0.00201416015625	\\
0.113153105162694	-0.00177001953125	\\
0.113197496337728	-0.002166748046875	\\
0.113241887512762	-0.002471923828125	\\
0.113286278687797	-0.002227783203125	\\
0.113330669862831	-0.00238037109375	\\
0.113375061037866	-0.002288818359375	\\
0.1134194522129	-0.00244140625	\\
0.113463843387934	-0.002471923828125	\\
0.113508234562969	-0.00250244140625	\\
0.113552625738003	-0.0025634765625	\\
0.113597016913038	-0.00262451171875	\\
0.113641408088072	-0.002410888671875	\\
0.113685799263107	-0.0023193359375	\\
0.113730190438141	-0.00250244140625	\\
0.113774581613175	-0.002227783203125	\\
0.11381897278821	-0.002044677734375	\\
0.113863363963244	-0.0020751953125	\\
0.113907755138279	-0.002166748046875	\\
0.113952146313313	-0.002685546875	\\
0.113996537488347	-0.00299072265625	\\
0.114040928663382	-0.002838134765625	\\
0.114085319838416	-0.002838134765625	\\
0.114129711013451	-0.0025634765625	\\
0.114174102188485	-0.0025634765625	\\
0.114218493363519	-0.002685546875	\\
0.114262884538554	-0.00225830078125	\\
0.114307275713588	-0.0023193359375	\\
0.114351666888623	-0.002288818359375	\\
0.114396058063657	-0.001983642578125	\\
0.114440449238691	-0.002349853515625	\\
0.114484840413726	-0.0023193359375	\\
0.11452923158876	-0.00201416015625	\\
0.114573622763795	-0.00213623046875	\\
0.114618013938829	-0.0020751953125	\\
0.114662405113863	-0.001922607421875	\\
0.114706796288898	-0.001678466796875	\\
0.114751187463932	-0.001678466796875	\\
0.114795578638967	-0.001129150390625	\\
0.114839969814001	-0.001190185546875	\\
0.114884360989035	-0.001251220703125	\\
0.11492875216407	-0.000823974609375	\\
0.114973143339104	-0.000762939453125	\\
0.115017534514139	-0.000762939453125	\\
0.115061925689173	-0.0008544921875	\\
0.115106316864207	-0.000732421875	\\
0.115150708039242	-0.000701904296875	\\
0.115195099214276	-0.00067138671875	\\
0.115239490389311	-0.000396728515625	\\
0.115283881564345	-0.000701904296875	\\
0.115328272739379	-0.000701904296875	\\
0.115372663914414	-0.00091552734375	\\
0.115417055089448	-0.000823974609375	\\
0.115461446264483	-0.000762939453125	\\
0.115505837439517	-0.000762939453125	\\
0.115550228614551	-0.0006103515625	\\
0.115594619789586	-0.00048828125	\\
0.11563901096462	-0.000579833984375	\\
0.115683402139655	-0.000946044921875	\\
0.115727793314689	-0.0010986328125	\\
0.115772184489723	-0.00128173828125	\\
0.115816575664758	-0.000885009765625	\\
0.115860966839792	-0.00079345703125	\\
0.115905358014827	-0.00128173828125	\\
0.115949749189861	-0.0013427734375	\\
0.115994140364895	-0.0010986328125	\\
0.11603853153993	-0.001495361328125	\\
0.116082922714964	-0.001495361328125	\\
0.116127313889999	-0.001251220703125	\\
0.116171705065033	-0.001434326171875	\\
0.116216096240067	-0.001251220703125	\\
0.116260487415102	-0.0013427734375	\\
0.116304878590136	-0.0013427734375	\\
0.116349269765171	-0.00128173828125	\\
0.116393660940205	-0.00128173828125	\\
0.116438052115239	-0.00152587890625	\\
0.116482443290274	-0.00115966796875	\\
0.116526834465308	-0.00128173828125	\\
0.116571225640343	-0.001007080078125	\\
0.116615616815377	-0.00103759765625	\\
0.116660007990412	-0.00091552734375	\\
0.116704399165446	-0.000946044921875	\\
0.11674879034048	-0.00115966796875	\\
0.116793181515515	-0.000579833984375	\\
0.116837572690549	-0.000885009765625	\\
0.116881963865584	-0.000732421875	\\
0.116926355040618	-0.000274658203125	\\
0.116970746215652	-0.000396728515625	\\
0.117015137390687	-9.1552734375e-05	\\
0.117059528565721	0	\\
0.117103919740756	0.0001220703125	\\
0.11714831091579	0.0003662109375	\\
0.117192702090824	0.000274658203125	\\
0.117237093265859	0.000152587890625	\\
0.117281484440893	-0.0001220703125	\\
0.117325875615928	-0.0003662109375	\\
0.117370266790962	-0.00030517578125	\\
0.117414657965996	-0.00018310546875	\\
0.117459049141031	-0.000244140625	\\
0.117503440316065	0.000244140625	\\
0.1175478314911	-0.000152587890625	\\
0.117592222666134	-0.00018310546875	\\
0.117636613841168	3.0517578125e-05	\\
0.117681005016203	6.103515625e-05	\\
0.117725396191237	0.000274658203125	\\
0.117769787366272	0.0001220703125	\\
0.117814178541306	-0.0001220703125	\\
0.11785856971634	0	\\
0.117902960891375	0.0003662109375	\\
0.117947352066409	-0.000213623046875	\\
0.117991743241444	-0.000457763671875	\\
0.118036134416478	-0.000152587890625	\\
0.118080525591512	-0.00067138671875	\\
0.118124916766547	-0.0009765625	\\
0.118169307941581	-0.000213623046875	\\
0.118213699116616	-3.0517578125e-05	\\
0.11825809029165	-0.000274658203125	\\
0.118302481466684	-0.00030517578125	\\
0.118346872641719	-0.000518798828125	\\
0.118391263816753	-0.000701904296875	\\
0.118435654991788	-0.0008544921875	\\
0.118480046166822	-0.00067138671875	\\
0.118524437341856	-0.000457763671875	\\
0.118568828516891	-0.000823974609375	\\
0.118613219691925	-0.000946044921875	\\
0.11865761086696	-0.000213623046875	\\
0.118702002041994	3.0517578125e-05	\\
0.118746393217028	-0.0003662109375	\\
0.118790784392063	-9.1552734375e-05	\\
0.118835175567097	6.103515625e-05	\\
0.118879566742132	-0.000152587890625	\\
0.118923957917166	0.0001220703125	\\
0.1189683490922	-3.0517578125e-05	\\
0.119012740267235	-6.103515625e-05	\\
0.119057131442269	-0.000244140625	\\
0.119101522617304	-0.000274658203125	\\
0.119145913792338	-0.000396728515625	\\
0.119190304967372	-0.0003662109375	\\
0.119234696142407	-0.000335693359375	\\
0.119279087317441	-0.000152587890625	\\
0.119323478492476	-3.0517578125e-05	\\
0.11936786966751	-0.00018310546875	\\
0.119412260842545	-9.1552734375e-05	\\
0.119456652017579	-0.000274658203125	\\
0.119501043192613	-0.0001220703125	\\
0.119545434367648	0.000152587890625	\\
0.119589825542682	-0.000457763671875	\\
0.119634216717717	-0.00067138671875	\\
0.119678607892751	-0.000579833984375	\\
0.119722999067785	-0.001220703125	\\
0.11976739024282	-0.0013427734375	\\
0.119811781417854	-0.001068115234375	\\
0.119856172592889	-0.00146484375	\\
0.119900563767923	-0.001251220703125	\\
0.119944954942957	-0.000885009765625	\\
0.119989346117992	-0.000732421875	\\
0.120033737293026	-0.000885009765625	\\
0.120078128468061	-0.001190185546875	\\
0.120122519643095	-0.00146484375	\\
0.120166910818129	-0.0013427734375	\\
0.120211301993164	-0.0009765625	\\
0.120255693168198	-0.0009765625	\\
0.120300084343233	-0.00140380859375	\\
0.120344475518267	-0.001220703125	\\
0.120388866693301	-0.0013427734375	\\
0.120433257868336	-0.00146484375	\\
0.12047764904337	-0.001190185546875	\\
0.120522040218405	-0.000885009765625	\\
0.120566431393439	-0.00079345703125	\\
0.120610822568473	-0.000823974609375	\\
0.120655213743508	-0.001190185546875	\\
0.120699604918542	-0.0013427734375	\\
0.120743996093577	-0.000732421875	\\
0.120788387268611	-0.000396728515625	\\
0.120832778443645	-0.0006103515625	\\
0.12087716961868	-0.000518798828125	\\
0.120921560793714	-0.000244140625	\\
0.120965951968749	-0.0003662109375	\\
0.121010343143783	-0.000457763671875	\\
0.121054734318817	-0.000396728515625	\\
0.121099125493852	-0.00067138671875	\\
0.121143516668886	-0.0006103515625	\\
0.121187907843921	-0.000701904296875	\\
0.121232299018955	-0.000762939453125	\\
0.121276690193989	-0.00054931640625	\\
0.121321081369024	-0.00030517578125	\\
0.121365472544058	-0.000335693359375	\\
0.121409863719093	-0.0003662109375	\\
0.121454254894127	3.0517578125e-05	\\
0.121498646069161	-0.000244140625	\\
0.121543037244196	-9.1552734375e-05	\\
0.12158742841923	0.000213623046875	\\
0.121631819594265	-0.00018310546875	\\
0.121676210769299	-0.00042724609375	\\
0.121720601944333	-0.0001220703125	\\
0.121764993119368	-0.000244140625	\\
0.121809384294402	-0.000213623046875	\\
0.121853775469437	-0.000213623046875	\\
0.121898166644471	-0.000335693359375	\\
0.121942557819505	-9.1552734375e-05	\\
0.12198694899454	6.103515625e-05	\\
0.122031340169574	3.0517578125e-05	\\
0.122075731344609	-9.1552734375e-05	\\
0.122120122519643	3.0517578125e-05	\\
0.122164513694678	0.0001220703125	\\
0.122208904869712	3.0517578125e-05	\\
0.122253296044746	-6.103515625e-05	\\
0.122297687219781	0.000335693359375	\\
0.122342078394815	0.00048828125	\\
0.12238646956985	0.000213623046875	\\
0.122430860744884	0.00042724609375	\\
0.122475251919918	0.000701904296875	\\
0.122519643094953	0.000640869140625	\\
0.122564034269987	0.00042724609375	\\
0.122608425445022	0.000579833984375	\\
0.122652816620056	0.000518798828125	\\
0.12269720779509	0.00103759765625	\\
0.122741598970125	0.001251220703125	\\
0.122785990145159	0.001373291015625	\\
0.122830381320194	0.00140380859375	\\
0.122874772495228	0.001495361328125	\\
0.122919163670262	0.001495361328125	\\
0.122963554845297	0.001495361328125	\\
0.123007946020331	0.00146484375	\\
0.123052337195366	0.00103759765625	\\
0.1230967283704	0.000946044921875	\\
0.123141119545434	0.001312255859375	\\
0.123185510720469	0.001068115234375	\\
0.123229901895503	0.00091552734375	\\
0.123274293070538	0.001190185546875	\\
0.123318684245572	0.001556396484375	\\
0.123363075420606	0.001220703125	\\
0.123407466595641	0.000946044921875	\\
0.123451857770675	0.001373291015625	\\
0.12349624894571	0.00091552734375	\\
0.123540640120744	0.00048828125	\\
0.123585031295778	0.000701904296875	\\
0.123629422470813	0.001220703125	\\
0.123673813645847	0.001495361328125	\\
0.123718204820882	0.00103759765625	\\
0.123762595995916	0.000823974609375	\\
0.12380698717095	0.00079345703125	\\
0.123851378345985	0.000732421875	\\
0.123895769521019	0.0003662109375	\\
0.123940160696054	0.000396728515625	\\
0.123984551871088	0.000579833984375	\\
0.124028943046122	0.0006103515625	\\
0.124073334221157	0.0010986328125	\\
0.124117725396191	0.001129150390625	\\
0.124162116571226	0.000946044921875	\\
0.12420650774626	0.00079345703125	\\
0.124250898921294	0.000457763671875	\\
0.124295290096329	0.000732421875	\\
0.124339681271363	0.000701904296875	\\
0.124384072446398	0.000579833984375	\\
0.124428463621432	0.000335693359375	\\
0.124472854796466	0.000396728515625	\\
0.124517245971501	0.000335693359375	\\
0.124561637146535	0.00054931640625	\\
0.12460602832157	0.0008544921875	\\
0.124650419496604	0.00115966796875	\\
0.124694810671638	0.00103759765625	\\
0.124739201846673	0.00048828125	\\
0.124783593021707	0.000579833984375	\\
0.124827984196742	0.000335693359375	\\
0.124872375371776	0.000396728515625	\\
0.124916766546811	0.000518798828125	\\
0.124961157721845	0.00054931640625	\\
0.125005548896879	0.00054931640625	\\
0.125049940071914	0.000518798828125	\\
0.125094331246948	0.00091552734375	\\
0.125138722421983	0.000823974609375	\\
0.125183113597017	0.000518798828125	\\
0.125227504772051	0.000701904296875	\\
0.125271895947086	0.00048828125	\\
0.12531628712212	0.000732421875	\\
0.125360678297155	0.0008544921875	\\
0.125405069472189	0.00091552734375	\\
0.125449460647223	0.000885009765625	\\
0.125493851822258	0.000335693359375	\\
0.125538242997292	0.00054931640625	\\
0.125582634172327	0.000274658203125	\\
0.125627025347361	-6.103515625e-05	\\
0.125671416522395	-0.0001220703125	\\
0.12571580769743	-0.00030517578125	\\
0.125760198872464	-0.00030517578125	\\
0.125804590047499	-0.000213623046875	\\
0.125848981222533	-0.000213623046875	\\
0.125893372397567	-0.000213623046875	\\
0.125937763572602	-0.000213623046875	\\
0.125982154747636	-9.1552734375e-05	\\
0.126026545922671	-0.000396728515625	\\
0.126070937097705	-0.000518798828125	\\
0.126115328272739	-0.00030517578125	\\
0.126159719447774	-0.000396728515625	\\
0.126204110622808	-0.00030517578125	\\
0.126248501797843	-0.00079345703125	\\
0.126292892972877	-0.0006103515625	\\
0.126337284147911	-0.00079345703125	\\
0.126381675322946	-0.00054931640625	\\
0.12642606649798	-0.000457763671875	\\
0.126470457673015	-0.000640869140625	\\
0.126514848848049	-0.00018310546875	\\
0.126559240023083	-0.000152587890625	\\
0.126603631198118	-3.0517578125e-05	\\
0.126648022373152	-0.0001220703125	\\
0.126692413548187	3.0517578125e-05	\\
0.126736804723221	9.1552734375e-05	\\
0.126781195898255	-9.1552734375e-05	\\
0.12682558707329	0.000579833984375	\\
0.126869978248324	0.000457763671875	\\
0.126914369423359	-6.103515625e-05	\\
0.126958760598393	-9.1552734375e-05	\\
0.127003151773427	-0.00030517578125	\\
0.127047542948462	-9.1552734375e-05	\\
0.127091934123496	0.0001220703125	\\
0.127136325298531	-0.00042724609375	\\
0.127180716473565	-0.0003662109375	\\
0.127225107648599	-0.000457763671875	\\
0.127269498823634	-0.0008544921875	\\
0.127313889998668	-0.0008544921875	\\
0.127358281173703	-0.00042724609375	\\
0.127402672348737	-0.00030517578125	\\
0.127447063523771	-0.000518798828125	\\
0.127491454698806	-0.00048828125	\\
0.12753584587384	-0.000457763671875	\\
0.127580237048875	-0.000457763671875	\\
0.127624628223909	-0.00067138671875	\\
0.127669019398943	-0.000732421875	\\
0.127713410573978	-0.000762939453125	\\
0.127757801749012	-0.001007080078125	\\
0.127802192924047	-0.001129150390625	\\
0.127846584099081	-0.0009765625	\\
0.127890975274116	-0.001007080078125	\\
0.12793536644915	-0.001190185546875	\\
0.127979757624184	-0.000823974609375	\\
0.128024148799219	-0.000518798828125	\\
0.128068539974253	-0.000701904296875	\\
0.128112931149288	-0.001007080078125	\\
0.128157322324322	-0.0008544921875	\\
0.128201713499356	-0.000762939453125	\\
0.128246104674391	-0.001312255859375	\\
0.128290495849425	-0.00140380859375	\\
0.12833488702446	-0.0013427734375	\\
0.128379278199494	-0.00128173828125	\\
0.128423669374528	-0.000762939453125	\\
0.128468060549563	-0.000946044921875	\\
0.128512451724597	-0.001068115234375	\\
0.128556842899632	-0.000396728515625	\\
0.128601234074666	-0.0003662109375	\\
0.1286456252497	-0.000518798828125	\\
0.128690016424735	-0.000457763671875	\\
0.128734407599769	-0.00042724609375	\\
0.128778798774804	-0.000396728515625	\\
0.128823189949838	-0.000396728515625	\\
0.128867581124872	-0.000640869140625	\\
0.128911972299907	-0.0006103515625	\\
0.128956363474941	-0.00042724609375	\\
0.129000754649976	-0.00054931640625	\\
0.12904514582501	-0.000823974609375	\\
0.129089537000044	-0.000640869140625	\\
0.129133928175079	-0.00079345703125	\\
0.129178319350113	-0.001007080078125	\\
0.129222710525148	-0.0006103515625	\\
0.129267101700182	-0.00091552734375	\\
0.129311492875216	-0.0010986328125	\\
0.129355884050251	-0.000732421875	\\
0.129400275225285	-0.000946044921875	\\
0.12944466640032	-0.001312255859375	\\
0.129489057575354	-0.001434326171875	\\
0.129533448750388	-0.00128173828125	\\
0.129577839925423	-0.00115966796875	\\
0.129622231100457	-0.0015869140625	\\
0.129666622275492	-0.00152587890625	\\
0.129711013450526	-0.00128173828125	\\
0.12975540462556	-0.001495361328125	\\
0.129799795800595	-0.001617431640625	\\
0.129844186975629	-0.001708984375	\\
0.129888578150664	-0.00201416015625	\\
0.129932969325698	-0.002288818359375	\\
0.129977360500732	-0.002197265625	\\
0.130021751675767	-0.0018310546875	\\
0.130066142850801	-0.001983642578125	\\
0.130110534025836	-0.00201416015625	\\
0.13015492520087	-0.001678466796875	\\
0.130199316375904	-0.00140380859375	\\
0.130243707550939	-0.00146484375	\\
0.130288098725973	-0.001495361328125	\\
0.130332489901008	-0.0013427734375	\\
0.130376881076042	-0.00115966796875	\\
0.130421272251077	-0.001068115234375	\\
0.130465663426111	-0.000823974609375	\\
0.130510054601145	-0.0008544921875	\\
0.13055444577618	-0.001190185546875	\\
0.130598836951214	-0.001220703125	\\
0.130643228126249	-0.0010986328125	\\
0.130687619301283	-0.001068115234375	\\
0.130732010476317	-0.000885009765625	\\
0.130776401651352	-0.00091552734375	\\
0.130820792826386	-0.000946044921875	\\
0.130865184001421	-0.0008544921875	\\
0.130909575176455	-0.000762939453125	\\
0.130953966351489	-0.00091552734375	\\
0.130998357526524	-0.00048828125	\\
0.131042748701558	-0.000457763671875	\\
0.131087139876593	-0.000762939453125	\\
0.131131531051627	-0.00067138671875	\\
0.131175922226661	-0.00079345703125	\\
0.131220313401696	-0.000823974609375	\\
0.13126470457673	-0.000701904296875	\\
0.131309095751765	-0.001007080078125	\\
0.131353486926799	-0.00115966796875	\\
0.131397878101833	-0.001068115234375	\\
0.131442269276868	-0.000885009765625	\\
0.131486660451902	-0.001129150390625	\\
0.131531051626937	-0.001190185546875	\\
0.131575442801971	-0.0006103515625	\\
0.131619833977005	-0.000579833984375	\\
0.13166422515204	-0.00048828125	\\
0.131708616327074	-0.00091552734375	\\
0.131753007502109	-0.00103759765625	\\
0.131797398677143	-0.000274658203125	\\
0.131841789852177	-0.00067138671875	\\
0.131886181027212	-0.000762939453125	\\
0.131930572202246	-0.00054931640625	\\
0.131974963377281	-0.00054931640625	\\
0.132019354552315	-0.000518798828125	\\
0.132063745727349	-3.0517578125e-05	\\
0.132108136902384	-0.000335693359375	\\
0.132152528077418	-0.00048828125	\\
0.132196919252453	-0.000579833984375	\\
0.132241310427487	-0.000946044921875	\\
0.132285701602521	-0.00128173828125	\\
0.132330092777556	-0.001220703125	\\
0.13237448395259	-0.000701904296875	\\
0.132418875127625	-0.000946044921875	\\
0.132463266302659	-0.001251220703125	\\
0.132507657477693	-0.000946044921875	\\
0.132552048652728	-0.0009765625	\\
0.132596439827762	-0.00140380859375	\\
0.132640831002797	-0.000640869140625	\\
0.132685222177831	-0.00042724609375	\\
0.132729613352865	-0.001068115234375	\\
0.1327740045279	-0.00091552734375	\\
0.132818395702934	-0.000823974609375	\\
0.132862786877969	-0.000640869140625	\\
0.132907178053003	-0.000885009765625	\\
0.132951569228037	-0.00067138671875	\\
0.132995960403072	-0.00018310546875	\\
0.133040351578106	-0.000396728515625	\\
0.133084742753141	-0.00054931640625	\\
0.133129133928175	-6.103515625e-05	\\
0.133173525103209	-0.000274658203125	\\
0.133217916278244	-0.000457763671875	\\
0.133262307453278	-0.00091552734375	\\
0.133306698628313	-0.000823974609375	\\
0.133351089803347	-6.103515625e-05	\\
0.133395480978382	-0.000396728515625	\\
0.133439872153416	-0.00042724609375	\\
0.13348426332845	-0.000732421875	\\
0.133528654503485	-0.000762939453125	\\
0.133573045678519	-0.000244140625	\\
0.133617436853554	-0.00067138671875	\\
0.133661828028588	-0.001007080078125	\\
0.133706219203622	-0.001434326171875	\\
0.133750610378657	-0.001800537109375	\\
0.133795001553691	-0.001617431640625	\\
0.133839392728726	-0.001373291015625	\\
0.13388378390376	-0.001312255859375	\\
0.133928175078794	-0.0013427734375	\\
0.133972566253829	-0.001708984375	\\
0.134016957428863	-0.0015869140625	\\
0.134061348603898	-0.0013427734375	\\
0.134105739778932	-0.001434326171875	\\
0.134150130953966	-0.00128173828125	\\
0.134194522129001	-0.001556396484375	\\
0.134238913304035	-0.001373291015625	\\
0.13428330447907	-0.001312255859375	\\
0.134327695654104	-0.001556396484375	\\
0.134372086829138	-0.001312255859375	\\
0.134416478004173	-0.001220703125	\\
0.134460869179207	-0.00140380859375	\\
0.134505260354242	-0.00152587890625	\\
0.134549651529276	-0.00146484375	\\
0.13459404270431	-0.001220703125	\\
0.134638433879345	-0.00115966796875	\\
0.134682825054379	-0.00152587890625	\\
0.134727216229414	-0.001434326171875	\\
0.134771607404448	-0.001251220703125	\\
0.134815998579482	-0.001312255859375	\\
0.134860389754517	-0.001373291015625	\\
0.134904780929551	-0.001434326171875	\\
0.134949172104586	-0.001312255859375	\\
0.13499356327962	-0.000518798828125	\\
0.135037954454654	-0.00042724609375	\\
0.135082345629689	-0.000823974609375	\\
0.135126736804723	-0.001129150390625	\\
0.135171127979758	-0.001556396484375	\\
0.135215519154792	-0.001495361328125	\\
0.135259910329826	-0.001434326171875	\\
0.135304301504861	-0.001708984375	\\
0.135348692679895	-0.001617431640625	\\
0.13539308385493	-0.001434326171875	\\
0.135437475029964	-0.00152587890625	\\
0.135481866204998	-0.001739501953125	\\
0.135526257380033	-0.001678466796875	\\
0.135570648555067	-0.0015869140625	\\
0.135615039730102	-0.00152587890625	\\
0.135659430905136	-0.00146484375	\\
0.13570382208017	-0.0013427734375	\\
0.135748213255205	-0.001220703125	\\
0.135792604430239	-0.0010986328125	\\
0.135836995605274	-0.00115966796875	\\
0.135881386780308	-0.001068115234375	\\
0.135925777955342	-0.000946044921875	\\
0.135970169130377	-0.0008544921875	\\
0.136014560305411	-0.0008544921875	\\
0.136058951480446	-0.0006103515625	\\
0.13610334265548	-0.00067138671875	\\
0.136147733830514	-0.000885009765625	\\
0.136192125005549	-0.00054931640625	\\
0.136236516180583	-0.00018310546875	\\
0.136280907355618	-0.000152587890625	\\
0.136325298530652	-0.000335693359375	\\
0.136369689705687	-0.0008544921875	\\
0.136414080880721	-0.0006103515625	\\
0.136458472055755	-9.1552734375e-05	\\
0.13650286323079	-0.00042724609375	\\
0.136547254405824	0.000244140625	\\
0.136591645580859	0.000335693359375	\\
0.136636036755893	0.000213623046875	\\
0.136680427930927	0.000701904296875	\\
0.136724819105962	0.000335693359375	\\
0.136769210280996	0	\\
0.136813601456031	-0.0001220703125	\\
0.136857992631065	-0.000152587890625	\\
0.136902383806099	-0.00030517578125	\\
0.136946774981134	-0.000274658203125	\\
0.136991166156168	-0.00018310546875	\\
0.137035557331203	6.103515625e-05	\\
0.137079948506237	9.1552734375e-05	\\
0.137124339681271	-0.000152587890625	\\
0.137168730856306	-0.0006103515625	\\
0.13721312203134	-0.00091552734375	\\
0.137257513206375	-0.000457763671875	\\
0.137301904381409	-0.000152587890625	\\
0.137346295556443	0	\\
0.137390686731478	6.103515625e-05	\\
0.137435077906512	0.000518798828125	\\
0.137479469081547	0.0001220703125	\\
0.137523860256581	-0.000213623046875	\\
0.137568251431615	9.1552734375e-05	\\
0.13761264260665	0.000335693359375	\\
0.137657033781684	0.00048828125	\\
0.137701424956719	9.1552734375e-05	\\
0.137745816131753	-0.00018310546875	\\
0.137790207306787	0.000213623046875	\\
0.137834598481822	-3.0517578125e-05	\\
0.137878989656856	-0.000732421875	\\
0.137923380831891	-0.000579833984375	\\
0.137967772006925	-0.00042724609375	\\
0.138012163181959	-0.001007080078125	\\
0.138056554356994	-0.001190185546875	\\
0.138100945532028	-0.0009765625	\\
0.138145336707063	-0.001068115234375	\\
0.138189727882097	-0.001220703125	\\
0.138234119057131	-0.001068115234375	\\
0.138278510232166	-0.00115966796875	\\
0.1383229014072	-0.001190185546875	\\
0.138367292582235	-0.00115966796875	\\
0.138411683757269	-0.000885009765625	\\
0.138456074932303	-0.000762939453125	\\
0.138500466107338	-0.00054931640625	\\
0.138544857282372	-0.000640869140625	\\
0.138589248457407	-0.00067138671875	\\
0.138633639632441	-0.00079345703125	\\
0.138678030807475	-0.00048828125	\\
0.13872242198251	-0.000213623046875	\\
0.138766813157544	0.0001220703125	\\
0.138811204332579	0.00042724609375	\\
0.138855595507613	0.0001220703125	\\
0.138899986682648	6.103515625e-05	\\
0.138944377857682	0.00030517578125	\\
0.138988769032716	3.0517578125e-05	\\
0.139033160207751	-0.00048828125	\\
0.139077551382785	-6.103515625e-05	\\
0.13912194255782	0.00030517578125	\\
0.139166333732854	0.000244140625	\\
0.139210724907888	0.00042724609375	\\
0.139255116082923	0.00091552734375	\\
0.139299507257957	0.001068115234375	\\
0.139343898432992	0.00091552734375	\\
0.139388289608026	0.001556396484375	\\
0.13943268078306	0.00146484375	\\
0.139477071958095	0.001434326171875	\\
0.139521463133129	0.001190185546875	\\
0.139565854308164	0.0009765625	\\
0.139610245483198	0.001495361328125	\\
0.139654636658232	0.001739501953125	\\
0.139699027833267	0.001220703125	\\
0.139743419008301	0.00146484375	\\
0.139787810183336	0.001617431640625	\\
0.13983220135837	0.001708984375	\\
0.139876592533404	0.00225830078125	\\
0.139920983708439	0.001983642578125	\\
0.139965374883473	0.001708984375	\\
0.140009766058508	0.00152587890625	\\
0.140054157233542	0.00164794921875	\\
0.140098548408576	0.00146484375	\\
0.140142939583611	0.00103759765625	\\
0.140187330758645	0.00103759765625	\\
0.14023172193368	0.00152587890625	\\
0.140276113108714	0.001922607421875	\\
0.140320504283748	0.002349853515625	\\
0.140364895458783	0.002166748046875	\\
0.140409286633817	0.001922607421875	\\
0.140453677808852	0.00238037109375	\\
0.140498068983886	0.002685546875	\\
0.14054246015892	0.0028076171875	\\
0.140586851333955	0.002960205078125	\\
0.140631242508989	0.00262451171875	\\
0.140675633684024	0.00250244140625	\\
0.140720024859058	0.00250244140625	\\
0.140764416034092	0.0023193359375	\\
0.140808807209127	0.002471923828125	\\
0.140853198384161	0.001953125	\\
0.140897589559196	0.001800537109375	\\
0.14094198073423	0.002288818359375	\\
0.140986371909264	0.0023193359375	\\
0.141030763084299	0.002227783203125	\\
0.141075154259333	0.00213623046875	\\
0.141119545434368	0.0020751953125	\\
0.141163936609402	0.0015869140625	\\
0.141208327784436	0.00128173828125	\\
0.141252718959471	0.001190185546875	\\
0.141297110134505	0.000946044921875	\\
0.14134150130954	0.00091552734375	\\
0.141385892484574	0.0009765625	\\
0.141430283659608	0.000885009765625	\\
0.141474674834643	0.0008544921875	\\
0.141519066009677	0.000762939453125	\\
0.141563457184712	0.000335693359375	\\
0.141607848359746	0.000457763671875	\\
0.14165223953478	0.000579833984375	\\
0.141696630709815	0.0001220703125	\\
0.141741021884849	0.000152587890625	\\
0.141785413059884	0.00054931640625	\\
0.141829804234918	0.000732421875	\\
0.141874195409953	0.000518798828125	\\
0.141918586584987	0.0001220703125	\\
0.141962977760021	6.103515625e-05	\\
0.142007368935056	0.000335693359375	\\
0.14205176011009	0.000885009765625	\\
0.142096151285125	0.00128173828125	\\
0.142140542460159	0.001190185546875	\\
0.142184933635193	0.0008544921875	\\
0.142229324810228	0.00091552734375	\\
0.142273715985262	0.001068115234375	\\
0.142318107160297	0.000732421875	\\
0.142362498335331	0.00091552734375	\\
0.142406889510365	0.000946044921875	\\
0.1424512806854	0.00103759765625	\\
0.142495671860434	0.0010986328125	\\
0.142540063035469	0.0008544921875	\\
0.142584454210503	0.0009765625	\\
0.142628845385537	0.0008544921875	\\
0.142673236560572	0.000762939453125	\\
0.142717627735606	0.00103759765625	\\
0.142762018910641	0.0009765625	\\
0.142806410085675	0.000946044921875	\\
0.142850801260709	0.0008544921875	\\
0.142895192435744	0.000396728515625	\\
0.142939583610778	0.000579833984375	\\
0.142983974785813	0.000701904296875	\\
0.143028365960847	0.0006103515625	\\
0.143072757135881	0.000274658203125	\\
0.143117148310916	0.0001220703125	\\
0.14316153948595	0.000823974609375	\\
0.143205930660985	0.000732421875	\\
0.143250321836019	0.00018310546875	\\
0.143294713011053	0.000335693359375	\\
0.143339104186088	0.000885009765625	\\
0.143383495361122	0.000701904296875	\\
0.143427886536157	0.0003662109375	\\
0.143472277711191	0.000518798828125	\\
0.143516668886225	0.00042724609375	\\
0.14356106006126	0.000152587890625	\\
0.143605451236294	0.000701904296875	\\
0.143649842411329	0.0003662109375	\\
0.143694233586363	9.1552734375e-05	\\
0.143738624761397	0.00048828125	\\
0.143783015936432	0.00091552734375	\\
0.143827407111466	0.001129150390625	\\
0.143871798286501	0.00115966796875	\\
0.143916189461535	0.001007080078125	\\
0.143960580636569	0.0008544921875	\\
0.144004971811604	0.00128173828125	\\
0.144049362986638	0.001190185546875	\\
0.144093754161673	0.001800537109375	\\
0.144138145336707	0.002410888671875	\\
0.144182536511741	0.002044677734375	\\
0.144226927686776	0.002166748046875	\\
0.14427131886181	0.002532958984375	\\
0.144315710036845	0.002349853515625	\\
0.144360101211879	0.00250244140625	\\
0.144404492386913	0.002655029296875	\\
0.144448883561948	0.002227783203125	\\
0.144493274736982	0.002166748046875	\\
0.144537665912017	0.002716064453125	\\
0.144582057087051	0.002655029296875	\\
0.144626448262086	0.002532958984375	\\
0.14467083943712	0.002655029296875	\\
0.144715230612154	0.002655029296875	\\
0.144759621787189	0.002532958984375	\\
0.144804012962223	0.002716064453125	\\
0.144848404137258	0.0029296875	\\
0.144892795312292	0.002838134765625	\\
0.144937186487326	0.00274658203125	\\
0.144981577662361	0.002899169921875	\\
0.145025968837395	0.0028076171875	\\
0.14507036001243	0.00244140625	\\
0.145114751187464	0.002288818359375	\\
0.145159142362498	0.002044677734375	\\
0.145203533537533	0.002105712890625	\\
0.145247924712567	0.001800537109375	\\
0.145292315887602	0.00115966796875	\\
0.145336707062636	0.001556396484375	\\
0.14538109823767	0.0013427734375	\\
0.145425489412705	0.001373291015625	\\
0.145469880587739	0.001251220703125	\\
0.145514271762774	0.001068115234375	\\
0.145558662937808	0.00146484375	\\
0.145603054112842	0.001068115234375	\\
0.145647445287877	0.0006103515625	\\
0.145691836462911	0.00103759765625	\\
0.145736227637946	0.000732421875	\\
0.14578061881298	0.000152587890625	\\
0.145825009988014	0.000396728515625	\\
0.145869401163049	0.000457763671875	\\
0.145913792338083	0.00018310546875	\\
0.145958183513118	0.000152587890625	\\
0.146002574688152	0.00048828125	\\
0.146046965863186	0.000885009765625	\\
0.146091357038221	0.000335693359375	\\
0.146135748213255	0.000244140625	\\
0.14618013938829	3.0517578125e-05	\\
0.146224530563324	-0.00048828125	\\
0.146268921738358	-0.000213623046875	\\
0.146313312913393	0.0001220703125	\\
0.146357704088427	-6.103515625e-05	\\
0.146402095263462	-9.1552734375e-05	\\
0.146446486438496	-0.0001220703125	\\
0.14649087761353	-0.000213623046875	\\
0.146535268788565	0.00018310546875	\\
0.146579659963599	-3.0517578125e-05	\\
0.146624051138634	-0.000274658203125	\\
0.146668442313668	0.000244140625	\\
0.146712833488702	0.000152587890625	\\
0.146757224663737	0.000579833984375	\\
0.146801615838771	0.00054931640625	\\
0.146846007013806	0.000152587890625	\\
0.14689039818884	0.000244140625	\\
0.146934789363874	0.000457763671875	\\
0.146979180538909	0.000244140625	\\
0.147023571713943	0.00042724609375	\\
0.147067962888978	0.000732421875	\\
0.147112354064012	0.000457763671875	\\
0.147156745239046	0.000579833984375	\\
0.147201136414081	0.000885009765625	\\
0.147245527589115	0.000518798828125	\\
0.14728991876415	0.000213623046875	\\
0.147334309939184	0.00030517578125	\\
0.147378701114219	0.00042724609375	\\
0.147423092289253	0.000244140625	\\
0.147467483464287	0.00042724609375	\\
0.147511874639322	0.00054931640625	\\
0.147556265814356	-3.0517578125e-05	\\
0.147600656989391	-0.00030517578125	\\
0.147645048164425	-0.000213623046875	\\
0.147689439339459	-0.00030517578125	\\
0.147733830514494	-0.0001220703125	\\
0.147778221689528	9.1552734375e-05	\\
0.147822612864563	0.000152587890625	\\
0.147867004039597	0.00048828125	\\
0.147911395214631	0.000579833984375	\\
0.147955786389666	0.000732421875	\\
0.1480001775647	0.0010986328125	\\
0.148044568739735	0.00140380859375	\\
0.148088959914769	0.00115966796875	\\
0.148133351089803	0.001373291015625	\\
0.148177742264838	0.00164794921875	\\
0.148222133439872	0.0015869140625	\\
0.148266524614907	0.002044677734375	\\
0.148310915789941	0.001678466796875	\\
0.148355306964975	0.001678466796875	\\
0.14839969814001	0.002044677734375	\\
0.148444089315044	0.002349853515625	\\
0.148488480490079	0.002532958984375	\\
0.148532871665113	0.002685546875	\\
0.148577262840147	0.002838134765625	\\
0.148621654015182	0.0028076171875	\\
0.148666045190216	0.002838134765625	\\
0.148710436365251	0.0030517578125	\\
0.148754827540285	0.00262451171875	\\
0.148799218715319	0.002471923828125	\\
0.148843609890354	0.002899169921875	\\
0.148888001065388	0.00299072265625	\\
0.148932392240423	0.00262451171875	\\
0.148976783415457	0.00274658203125	\\
0.149021174590491	0.0023193359375	\\
0.149065565765526	0.001953125	\\
0.14910995694056	0.001922607421875	\\
0.149154348115595	0.0015869140625	\\
0.149198739290629	0.00201416015625	\\
0.149243130465663	0.00250244140625	\\
0.149287521640698	0.002197265625	\\
0.149331912815732	0.00213623046875	\\
0.149376303990767	0.00213623046875	\\
0.149420695165801	0.002105712890625	\\
0.149465086340835	0.00189208984375	\\
0.14950947751587	0.002044677734375	\\
0.149553868690904	0.001617431640625	\\
0.149598259865939	0.00103759765625	\\
0.149642651040973	0.001495361328125	\\
0.149687042216007	0.00128173828125	\\
0.149731433391042	0.000823974609375	\\
0.149775824566076	0.000946044921875	\\
0.149820215741111	0.000885009765625	\\
0.149864606916145	0.001068115234375	\\
0.149908998091179	0.001129150390625	\\
0.149953389266214	0.00103759765625	\\
0.149997780441248	0.00091552734375	\\
0.150042171616283	0.001190185546875	\\
0.150086562791317	0.00128173828125	\\
0.150130953966351	0.000946044921875	\\
0.150175345141386	0.000823974609375	\\
0.15021973631642	0.000946044921875	\\
0.150264127491455	0.00103759765625	\\
0.150308518666489	0.0010986328125	\\
0.150352909841524	0.001129150390625	\\
0.150397301016558	0.001251220703125	\\
0.150441692191592	0.001373291015625	\\
0.150486083366627	0.00146484375	\\
0.150530474541661	0.00146484375	\\
0.150574865716696	0.0015869140625	\\
0.15061925689173	0.001190185546875	\\
0.150663648066764	0.000885009765625	\\
0.150708039241799	0.000946044921875	\\
0.150752430416833	0.000946044921875	\\
0.150796821591868	0.000732421875	\\
0.150841212766902	0.00067138671875	\\
0.150885603941936	0.00079345703125	\\
0.150929995116971	0.000274658203125	\\
0.150974386292005	0.000640869140625	\\
0.15101877746704	0.000885009765625	\\
0.151063168642074	0.00018310546875	\\
0.151107559817108	0.000152587890625	\\
0.151151950992143	0.000518798828125	\\
0.151196342167177	0.000885009765625	\\
0.151240733342212	0.0006103515625	\\
0.151285124517246	0.00030517578125	\\
0.15132951569228	0.000244140625	\\
0.151373906867315	0.000274658203125	\\
0.151418298042349	-6.103515625e-05	\\
0.151462689217384	-0.00054931640625	\\
0.151507080392418	-0.000579833984375	\\
0.151551471567452	-0.00048828125	\\
0.151595862742487	-0.000701904296875	\\
0.151640253917521	-0.000640869140625	\\
0.151684645092556	-0.000701904296875	\\
0.15172903626759	-0.000640869140625	\\
0.151773427442624	0	\\
0.151817818617659	-0.0001220703125	\\
0.151862209792693	-0.00067138671875	\\
0.151906600967728	-0.0006103515625	\\
0.151950992142762	-0.000396728515625	\\
0.151995383317796	-0.000213623046875	\\
0.152039774492831	-0.000152587890625	\\
0.152084165667865	-3.0517578125e-05	\\
0.1521285568429	0.000152587890625	\\
0.152172948017934	0.000213623046875	\\
0.152217339192968	0.0003662109375	\\
0.152261730368003	0.00030517578125	\\
0.152306121543037	0.00067138671875	\\
0.152350512718072	0.00079345703125	\\
0.152394903893106	0.001007080078125	\\
0.15243929506814	0.001617431640625	\\
0.152483686243175	0.00115966796875	\\
0.152528077418209	0.000823974609375	\\
0.152572468593244	0.001190185546875	\\
0.152616859768278	0.001007080078125	\\
0.152661250943312	0.000823974609375	\\
0.152705642118347	0.000701904296875	\\
0.152750033293381	0.00079345703125	\\
0.152794424468416	0.00079345703125	\\
0.15283881564345	0.00054931640625	\\
0.152883206818485	0.0003662109375	\\
0.152927597993519	0.000457763671875	\\
0.152971989168553	0.000335693359375	\\
0.153016380343588	9.1552734375e-05	\\
0.153060771518622	-0.00018310546875	\\
0.153105162693657	-0.000274658203125	\\
0.153149553868691	-0.000396728515625	\\
0.153193945043725	-0.00067138671875	\\
0.15323833621876	-0.000579833984375	\\
0.153282727393794	-0.000732421875	\\
0.153327118568829	-0.000946044921875	\\
0.153371509743863	-0.0008544921875	\\
0.153415900918897	-0.001190185546875	\\
0.153460292093932	-0.000946044921875	\\
0.153504683268966	-0.000946044921875	\\
0.153549074444001	-0.001312255859375	\\
0.153593465619035	-0.001129150390625	\\
0.153637856794069	-0.000946044921875	\\
0.153682247969104	-0.000732421875	\\
0.153726639144138	-0.000701904296875	\\
0.153771030319173	-0.000732421875	\\
0.153815421494207	-0.00115966796875	\\
0.153859812669241	-0.00140380859375	\\
0.153904203844276	-0.001220703125	\\
0.15394859501931	-0.001068115234375	\\
0.153992986194345	-0.001007080078125	\\
0.154037377369379	-0.001190185546875	\\
0.154081768544413	-0.001068115234375	\\
0.154126159719448	-0.00079345703125	\\
0.154170550894482	-0.000823974609375	\\
0.154214942069517	-0.00079345703125	\\
0.154259333244551	-0.000579833984375	\\
0.154303724419585	-0.00091552734375	\\
0.15434811559462	-0.000762939453125	\\
0.154392506769654	-0.00054931640625	\\
0.154436897944689	-0.000762939453125	\\
0.154481289119723	-0.00048828125	\\
0.154525680294757	0.0001220703125	\\
0.154570071469792	0.00030517578125	\\
0.154614462644826	0.000518798828125	\\
0.154658853819861	0.00048828125	\\
0.154703244994895	0.0010986328125	\\
0.154747636169929	0.00146484375	\\
0.154792027344964	0.001678466796875	\\
0.154836418519998	0.00140380859375	\\
0.154880809695033	0.001190185546875	\\
0.154925200870067	0.0015869140625	\\
0.154969592045101	0.000823974609375	\\
0.155013983220136	0.00048828125	\\
0.15505837439517	0.000640869140625	\\
0.155102765570205	0.0008544921875	\\
0.155147156745239	0.0010986328125	\\
0.155191547920273	0.001190185546875	\\
0.155235939095308	0.001251220703125	\\
0.155280330270342	0.001312255859375	\\
0.155324721445377	0.0009765625	\\
0.155369112620411	0.001220703125	\\
0.155413503795445	0.000823974609375	\\
0.15545789497048	-0.00030517578125	\\
0.155502286145514	-0.00042724609375	\\
0.155546677320549	-0.000640869140625	\\
0.155591068495583	-0.001129150390625	\\
0.155635459670617	-0.000946044921875	\\
0.155679850845652	-0.0009765625	\\
0.155724242020686	-0.000701904296875	\\
0.155768633195721	-0.000396728515625	\\
0.155813024370755	-0.000244140625	\\
0.15585741554579	-0.00048828125	\\
0.155901806720824	-0.000640869140625	\\
0.155946197895858	-0.000396728515625	\\
0.155990589070893	-0.000335693359375	\\
0.156034980245927	-6.103515625e-05	\\
0.156079371420962	-0.0001220703125	\\
0.156123762595996	-0.00048828125	\\
0.15616815377103	-0.0006103515625	\\
0.156212544946065	-0.000244140625	\\
0.156256936121099	0.000244140625	\\
0.156301327296134	0.00030517578125	\\
0.156345718471168	0.000732421875	\\
0.156390109646202	0.001190185546875	\\
0.156434500821237	0.00152587890625	\\
0.156478891996271	0.0023193359375	\\
0.156523283171306	0.002777099609375	\\
0.15656767434634	0.00274658203125	\\
0.156612065521374	0.002655029296875	\\
0.156656456696409	0.00262451171875	\\
0.156700847871443	0.0029296875	\\
0.156745239046478	0.002899169921875	\\
0.156789630221512	0.00262451171875	\\
0.156834021396546	0.002685546875	\\
0.156878412571581	0.00286865234375	\\
0.156922803746615	0.002777099609375	\\
0.15696719492165	0.0029296875	\\
0.157011586096684	0.003082275390625	\\
0.157055977271718	0.003173828125	\\
0.157100368446753	0.00299072265625	\\
0.157144759621787	0.002593994140625	\\
0.157189150796822	0.002197265625	\\
0.157233541971856	0.001617431640625	\\
0.15727793314689	0.00140380859375	\\
0.157322324321925	0.001220703125	\\
0.157366715496959	0.001068115234375	\\
0.157411106671994	0.000579833984375	\\
0.157455497847028	0.000457763671875	\\
0.157499889022062	0.0006103515625	\\
0.157544280197097	0.000579833984375	\\
0.157588671372131	0.00115966796875	\\
0.157633062547166	0.00048828125	\\
0.1576774537222	0.000518798828125	\\
0.157721844897234	0.000457763671875	\\
0.157766236072269	0.00018310546875	\\
0.157810627247303	0.000518798828125	\\
0.157855018422338	0.00054931640625	\\
0.157899409597372	0.00042724609375	\\
0.157943800772406	0.000579833984375	\\
0.157988191947441	0.000946044921875	\\
0.158032583122475	0.00103759765625	\\
0.15807697429751	0.001068115234375	\\
0.158121365472544	0.00140380859375	\\
0.158165756647578	0.001556396484375	\\
0.158210147822613	0.0020751953125	\\
0.158254538997647	0.00262451171875	\\
0.158298930172682	0.002532958984375	\\
0.158343321347716	0.002685546875	\\
0.15838771252275	0.002838134765625	\\
0.158432103697785	0.00244140625	\\
0.158476494872819	0.00244140625	\\
0.158520886047854	0.00244140625	\\
0.158565277222888	0.00244140625	\\
0.158609668397922	0.002532958984375	\\
0.158654059572957	0.00250244140625	\\
0.158698450747991	0.002105712890625	\\
0.158742841923026	0.002166748046875	\\
0.15878723309806	0.002044677734375	\\
0.158831624273095	0.00177001953125	\\
0.158876015448129	0.001617431640625	\\
0.158920406623163	0.0015869140625	\\
0.158964797798198	0.001739501953125	\\
0.159009188973232	0.00152587890625	\\
0.159053580148267	0.00115966796875	\\
0.159097971323301	0.001007080078125	\\
0.159142362498335	0.00067138671875	\\
0.15918675367337	0.000579833984375	\\
0.159231144848404	0.000579833984375	\\
0.159275536023439	0.0001220703125	\\
0.159319927198473	-3.0517578125e-05	\\
0.159364318373507	-0.00054931640625	\\
0.159408709548542	-0.00079345703125	\\
0.159453100723576	-0.000885009765625	\\
0.159497491898611	-0.00030517578125	\\
0.159541883073645	-0.000244140625	\\
0.159586274248679	-0.00048828125	\\
0.159630665423714	-0.000335693359375	\\
0.159675056598748	-0.000152587890625	\\
0.159719447773783	-0.000152587890625	\\
0.159763838948817	-0.00018310546875	\\
0.159808230123851	0.000762939453125	\\
0.159852621298886	0.000701904296875	\\
0.15989701247392	0.001129150390625	\\
0.159941403648955	0.00146484375	\\
0.159985794823989	0.0013427734375	\\
0.160030185999023	0.00164794921875	\\
0.160074577174058	0.001495361328125	\\
0.160118968349092	0.00140380859375	\\
0.160163359524127	0.001861572265625	\\
0.160207750699161	0.00201416015625	\\
0.160252141874195	0.002410888671875	\\
0.16029653304923	0.002716064453125	\\
0.160340924224264	0.0023193359375	\\
0.160385315399299	0.002288818359375	\\
0.160429706574333	0.003082275390625	\\
0.160474097749367	0.003173828125	\\
0.160518488924402	0.00286865234375	\\
0.160562880099436	0.00299072265625	\\
0.160607271274471	0.003204345703125	\\
0.160651662449505	0.002838134765625	\\
0.160696053624539	0.002593994140625	\\
0.160740444799574	0.0023193359375	\\
0.160784835974608	0.002105712890625	\\
0.160829227149643	0.0020751953125	\\
0.160873618324677	0.00140380859375	\\
0.160918009499711	0.00152587890625	\\
0.160962400674746	0.001434326171875	\\
0.16100679184978	0.000762939453125	\\
0.161051183024815	0.000579833984375	\\
0.161095574199849	0.000244140625	\\
0.161139965374883	-0.00030517578125	\\
0.161184356549918	-0.000518798828125	\\
0.161228747724952	-0.000396728515625	\\
0.161273138899987	-0.000701904296875	\\
0.161317530075021	-0.001068115234375	\\
0.161361921250056	-0.001007080078125	\\
0.16140631242509	-0.00115966796875	\\
0.161450703600124	-0.001007080078125	\\
0.161495094775159	-0.0013427734375	\\
0.161539485950193	-0.0015869140625	\\
0.161583877125228	-0.001556396484375	\\
0.161628268300262	-0.00164794921875	\\
0.161672659475296	-0.00164794921875	\\
0.161717050650331	-0.001678466796875	\\
0.161761441825365	-0.001617431640625	\\
0.1618058330004	-0.00146484375	\\
0.161850224175434	-0.001312255859375	\\
0.161894615350468	-0.00140380859375	\\
0.161939006525503	-0.00146484375	\\
0.161983397700537	-0.0013427734375	\\
0.162027788875572	-0.0015869140625	\\
0.162072180050606	-0.001556396484375	\\
0.16211657122564	-0.001007080078125	\\
0.162160962400675	-0.00048828125	\\
0.162205353575709	-0.00018310546875	\\
0.162249744750744	0.000213623046875	\\
0.162294135925778	0.0006103515625	\\
0.162338527100812	0.00091552734375	\\
0.162382918275847	0.0010986328125	\\
0.162427309450881	0.001129150390625	\\
0.162471700625916	0.00140380859375	\\
0.16251609180095	0.001556396484375	\\
0.162560482975984	0.001739501953125	\\
0.162604874151019	0.001861572265625	\\
0.162649265326053	0.00103759765625	\\
0.162693656501088	0	\\
0.162738047676122	-0.0001220703125	\\
0.162782438851156	0.000152587890625	\\
0.162826830026191	0.0003662109375	\\
0.162871221201225	0.00030517578125	\\
0.16291561237626	0.000274658203125	\\
0.162960003551294	0.0008544921875	\\
0.163004394726328	0.000732421875	\\
0.163048785901363	6.103515625e-05	\\
0.163093177076397	-0.000396728515625	\\
0.163137568251432	-0.0008544921875	\\
0.163181959426466	-0.001373291015625	\\
0.1632263506015	-0.001953125	\\
0.163270741776535	-0.002288818359375	\\
0.163315132951569	-0.002410888671875	\\
0.163359524126604	-0.002288818359375	\\
0.163403915301638	-0.002166748046875	\\
0.163448306476672	-0.002197265625	\\
0.163492697651707	-0.001373291015625	\\
0.163537088826741	-0.001220703125	\\
0.163581480001776	-0.0015869140625	\\
0.16362587117681	-0.00091552734375	\\
0.163670262351844	-0.00054931640625	\\
0.163714653526879	-0.000244140625	\\
0.163759044701913	-0.000244140625	\\
0.163803435876948	-0.000640869140625	\\
0.163847827051982	3.0517578125e-05	\\
0.163892218227016	-9.1552734375e-05	\\
0.163936609402051	6.103515625e-05	\\
0.163981000577085	0.0009765625	\\
0.16402539175212	0.00140380859375	\\
0.164069782927154	0.00225830078125	\\
0.164114174102188	0.002471923828125	\\
0.164158565277223	0.002899169921875	\\
0.164202956452257	0.0035400390625	\\
0.164247347627292	0.003662109375	\\
0.164291738802326	0.003997802734375	\\
0.164336129977361	0.004150390625	\\
0.164380521152395	0.004119873046875	\\
0.164424912327429	0.004241943359375	\\
0.164469303502464	0.004241943359375	\\
0.164513694677498	0.00372314453125	\\
0.164558085852533	0.00311279296875	\\
0.164602477027567	0.003021240234375	\\
0.164646868202601	0.00323486328125	\\
0.164691259377636	0.003082275390625	\\
0.16473565055267	0.00286865234375	\\
0.164780041727705	0.00274658203125	\\
0.164824432902739	0.00250244140625	\\
0.164868824077773	0.002105712890625	\\
0.164913215252808	0.001251220703125	\\
0.164957606427842	0.000518798828125	\\
0.165001997602877	-0.0001220703125	\\
0.165046388777911	-0.0006103515625	\\
0.165090779952945	-0.000701904296875	\\
0.16513517112798	-0.001190185546875	\\
0.165179562303014	-0.001190185546875	\\
0.165223953478049	-0.001129150390625	\\
0.165268344653083	-0.001190185546875	\\
0.165312735828117	-0.000946044921875	\\
0.165357127003152	-0.0009765625	\\
0.165401518178186	-0.00146484375	\\
0.165445909353221	-0.001373291015625	\\
0.165490300528255	-0.00079345703125	\\
0.165534691703289	-0.000518798828125	\\
0.165579082878324	-0.000457763671875	\\
0.165623474053358	-0.000732421875	\\
0.165667865228393	-0.0003662109375	\\
0.165712256403427	-0.00030517578125	\\
0.165756647578461	-0.000274658203125	\\
0.165801038753496	0.00048828125	\\
0.16584542992853	0.000701904296875	\\
0.165889821103565	0.001373291015625	\\
0.165934212278599	0.00152587890625	\\
0.165978603453633	0.001434326171875	\\
0.166022994628668	0.002197265625	\\
0.166067385803702	0.00238037109375	\\
0.166111776978737	0.002105712890625	\\
0.166156168153771	0.002410888671875	\\
0.166200559328805	0.002227783203125	\\
0.16624495050384	0.00213623046875	\\
0.166289341678874	0.002227783203125	\\
0.166333732853909	0.002349853515625	\\
0.166378124028943	0.002227783203125	\\
0.166422515203977	0.00189208984375	\\
0.166466906379012	0.002044677734375	\\
0.166511297554046	0.002105712890625	\\
0.166555688729081	0.001953125	\\
0.166600079904115	0.00115966796875	\\
0.166644471079149	0.00091552734375	\\
0.166688862254184	0.00103759765625	\\
0.166733253429218	0.00079345703125	\\
0.166777644604253	0.000946044921875	\\
0.166822035779287	3.0517578125e-05	\\
0.166866426954321	-0.000396728515625	\\
0.166910818129356	-0.000579833984375	\\
0.16695520930439	-0.001129150390625	\\
0.166999600479425	-0.001251220703125	\\
0.167043991654459	-0.0013427734375	\\
0.167088382829494	-0.001617431640625	\\
0.167132774004528	-0.00177001953125	\\
0.167177165179562	-0.001434326171875	\\
0.167221556354597	-0.0013427734375	\\
0.167265947529631	-0.00146484375	\\
0.167310338704666	-0.001617431640625	\\
0.1673547298797	-0.0010986328125	\\
0.167399121054734	-0.000579833984375	\\
0.167443512229769	-0.000701904296875	\\
0.167487903404803	-0.00048828125	\\
0.167532294579838	-3.0517578125e-05	\\
0.167576685754872	0.000396728515625	\\
0.167621076929906	-3.0517578125e-05	\\
0.167665468104941	-0.000152587890625	\\
0.167709859279975	0.0006103515625	\\
0.16775425045501	0.0009765625	\\
0.167798641630044	0.0009765625	\\
0.167843032805078	0.001495361328125	\\
0.167887423980113	0.002471923828125	\\
0.167931815155147	0.002227783203125	\\
0.167976206330182	0.00225830078125	\\
0.168020597505216	0.00213623046875	\\
0.16806498868025	0.001953125	\\
0.168109379855285	0.002044677734375	\\
0.168153771030319	0.002288818359375	\\
0.168198162205354	0.002288818359375	\\
0.168242553380388	0.0025634765625	\\
0.168286944555422	0.002410888671875	\\
0.168331335730457	0.00189208984375	\\
0.168375726905491	0.001800537109375	\\
0.168420118080526	0.00189208984375	\\
0.16846450925556	0.001312255859375	\\
0.168508900430594	0.000762939453125	\\
0.168553291605629	0.001007080078125	\\
0.168597682780663	0.000640869140625	\\
0.168642073955698	0.000335693359375	\\
0.168686465130732	0.000152587890625	\\
0.168730856305766	-0.00048828125	\\
0.168775247480801	-0.00042724609375	\\
0.168819638655835	-0.000579833984375	\\
0.16886402983087	-0.00115966796875	\\
0.168908421005904	-0.001434326171875	\\
0.168952812180938	-0.00164794921875	\\
0.168997203355973	-0.00128173828125	\\
0.169041594531007	-0.0009765625	\\
0.169085985706042	-0.0010986328125	\\
0.169130376881076	-0.001434326171875	\\
0.16917476805611	-0.001708984375	\\
0.169219159231145	-0.00201416015625	\\
0.169263550406179	-0.002349853515625	\\
0.169307941581214	-0.00244140625	\\
0.169352332756248	-0.0028076171875	\\
0.169396723931282	-0.002685546875	\\
0.169441115106317	-0.002288818359375	\\
0.169485506281351	-0.002166748046875	\\
0.169529897456386	-0.001861572265625	\\
0.16957428863142	-0.001708984375	\\
0.169618679806454	-0.001861572265625	\\
0.169663070981489	-0.00189208984375	\\
0.169707462156523	-0.0013427734375	\\
0.169751853331558	-0.001312255859375	\\
0.169796244506592	-0.00128173828125	\\
0.169840635681627	-0.00152587890625	\\
0.169885026856661	-0.0013427734375	\\
0.169929418031695	-0.000946044921875	\\
0.16997380920673	-0.0003662109375	\\
0.170018200381764	0	\\
0.170062591556799	9.1552734375e-05	\\
0.170106982731833	0.000579833984375	\\
0.170151373906867	0.0008544921875	\\
0.170195765081902	0.001373291015625	\\
0.170240156256936	0.0013427734375	\\
0.170284547431971	0.001220703125	\\
0.170328938607005	0.00115966796875	\\
0.170373329782039	0.00067138671875	\\
0.170417720957074	0.000152587890625	\\
0.170462112132108	-0.000396728515625	\\
0.170506503307143	-0.0006103515625	\\
0.170550894482177	-0.00030517578125	\\
0.170595285657211	0	\\
0.170639676832246	-0.000213623046875	\\
0.17068406800728	-0.000335693359375	\\
0.170728459182315	-0.000518798828125	\\
0.170772850357349	-0.000762939453125	\\
0.170817241532383	-0.00091552734375	\\
0.170861632707418	-0.00152587890625	\\
0.170906023882452	-0.00213623046875	\\
0.170950415057487	-0.002593994140625	\\
0.170994806232521	-0.00286865234375	\\
0.171039197407555	-0.002471923828125	\\
0.17108358858259	-0.00238037109375	\\
0.171127979757624	-0.002685546875	\\
0.171172370932659	-0.002471923828125	\\
0.171216762107693	-0.00213623046875	\\
0.171261153282727	-0.00152587890625	\\
0.171305544457762	-0.00152587890625	\\
0.171349935632796	-0.00128173828125	\\
0.171394326807831	-0.000946044921875	\\
0.171438717982865	-0.00091552734375	\\
0.171483109157899	-0.001129150390625	\\
0.171527500332934	-0.001251220703125	\\
0.171571891507968	-0.001220703125	\\
0.171616282683003	-0.00103759765625	\\
0.171660673858037	-9.1552734375e-05	\\
0.171705065033071	-0.000244140625	\\
0.171749456208106	0.00030517578125	\\
0.17179384738314	0.0013427734375	\\
0.171838238558175	0.001373291015625	\\
0.171882629733209	0.00146484375	\\
0.171927020908243	0.001556396484375	\\
0.171971412083278	0.001434326171875	\\
0.172015803258312	0.001861572265625	\\
0.172060194433347	0.002197265625	\\
0.172104585608381	0.001953125	\\
0.172148976783415	0.001922607421875	\\
0.17219336795845	0.00177001953125	\\
0.172237759133484	0.00164794921875	\\
0.172282150308519	0.000885009765625	\\
0.172326541483553	0.000274658203125	\\
0.172370932658587	0.000396728515625	\\
0.172415323833622	0.00042724609375	\\
0.172459715008656	0.000152587890625	\\
0.172504106183691	9.1552734375e-05	\\
0.172548497358725	0.000579833984375	\\
0.172592888533759	-0.000244140625	\\
0.172637279708794	-0.001220703125	\\
0.172681670883828	-0.00164794921875	\\
0.172726062058863	-0.00238037109375	\\
0.172770453233897	-0.00274658203125	\\
0.172814844408932	-0.00225830078125	\\
0.172859235583966	-0.00213623046875	\\
0.172903626759	-0.002197265625	\\
0.172948017934035	-0.001800537109375	\\
0.172992409109069	-0.00225830078125	\\
0.173036800284104	-0.002349853515625	\\
0.173081191459138	-0.001922607421875	\\
0.173125582634172	-0.001739501953125	\\
0.173169973809207	-0.00189208984375	\\
0.173214364984241	-0.00164794921875	\\
0.173258756159276	-0.00091552734375	\\
0.17330314733431	-0.000518798828125	\\
0.173347538509344	-0.0001220703125	\\
0.173391929684379	0.000213623046875	\\
0.173436320859413	0.000335693359375	\\
0.173480712034448	0.000213623046875	\\
0.173525103209482	0.00048828125	\\
0.173569494384516	0.001373291015625	\\
0.173613885559551	0.001983642578125	\\
0.173658276734585	0.0018310546875	\\
0.17370266790962	0.002349853515625	\\
0.173747059084654	0.0025634765625	\\
0.173791450259688	0.0023193359375	\\
0.173835841434723	0.00238037109375	\\
0.173880232609757	0.00238037109375	\\
0.173924623784792	0.0020751953125	\\
0.173969014959826	0.001617431640625	\\
0.17401340613486	0.001800537109375	\\
0.174057797309895	0.001708984375	\\
0.174102188484929	0.001220703125	\\
0.174146579659964	0.000579833984375	\\
0.174190970834998	0.000640869140625	\\
0.174235362010032	6.103515625e-05	\\
0.174279753185067	-0.000518798828125	\\
0.174324144360101	0	\\
0.174368535535136	-0.000335693359375	\\
0.17441292671017	-0.0006103515625	\\
0.174457317885204	-0.000457763671875	\\
0.174501709060239	-0.000732421875	\\
0.174546100235273	-0.00140380859375	\\
0.174590491410308	-0.001617431640625	\\
0.174634882585342	-0.001861572265625	\\
0.174679273760376	-0.002349853515625	\\
0.174723664935411	-0.0020751953125	\\
0.174768056110445	-0.002166748046875	\\
0.17481244728548	-0.00213623046875	\\
0.174856838460514	-0.001953125	\\
0.174901229635548	-0.0015869140625	\\
0.174945620810583	-0.001068115234375	\\
0.174990011985617	-0.001220703125	\\
0.175034403160652	-0.00115966796875	\\
0.175078794335686	-0.00067138671875	\\
0.17512318551072	-0.00042724609375	\\
0.175167576685755	-0.00018310546875	\\
0.175211967860789	0.0003662109375	\\
0.175256359035824	0.000640869140625	\\
0.175300750210858	0.000732421875	\\
0.175345141385892	0.000640869140625	\\
0.175389532560927	0.000885009765625	\\
0.175433923735961	0.00079345703125	\\
0.175478314910996	0.001068115234375	\\
0.17552270608603	0.00128173828125	\\
0.175567097261065	0.001922607421875	\\
0.175611488436099	0.002471923828125	\\
0.175655879611133	0.002471923828125	\\
0.175700270786168	0.002716064453125	\\
0.175744661961202	0.002288818359375	\\
0.175789053136237	0.00164794921875	\\
0.175833444311271	0.00152587890625	\\
0.175877835486305	0.001190185546875	\\
0.17592222666134	0.000823974609375	\\
0.175966617836374	0.000885009765625	\\
0.176011009011409	0.00079345703125	\\
0.176055400186443	0.000335693359375	\\
0.176099791361477	0.000274658203125	\\
0.176144182536512	-6.103515625e-05	\\
0.176188573711546	-0.000701904296875	\\
0.176232964886581	-0.000732421875	\\
0.176277356061615	-0.000701904296875	\\
0.176321747236649	-0.0008544921875	\\
0.176366138411684	-0.001068115234375	\\
0.176410529586718	-0.000762939453125	\\
0.176454920761753	-0.001251220703125	\\
0.176499311936787	-0.001495361328125	\\
0.176543703111821	-0.001556396484375	\\
0.176588094286856	-0.001800537109375	\\
0.17663248546189	-0.0015869140625	\\
0.176676876636925	-0.001190185546875	\\
0.176721267811959	-0.001708984375	\\
0.176765658986993	-0.002105712890625	\\
0.176810050162028	-0.002044677734375	\\
0.176854441337062	-0.002227783203125	\\
0.176898832512097	-0.00213623046875	\\
0.176943223687131	-0.002471923828125	\\
0.176987614862165	-0.00299072265625	\\
0.1770320060372	-0.0029296875	\\
0.177076397212234	-0.002655029296875	\\
0.177120788387269	-0.002777099609375	\\
0.177165179562303	-0.003265380859375	\\
0.177209570737337	-0.003204345703125	\\
0.177253961912372	-0.00347900390625	\\
0.177298353087406	-0.003509521484375	\\
0.177342744262441	-0.003326416015625	\\
0.177387135437475	-0.003143310546875	\\
0.177431526612509	-0.002349853515625	\\
0.177475917787544	-0.002349853515625	\\
0.177520308962578	-0.0020751953125	\\
0.177564700137613	-0.001556396484375	\\
};
\addplot [color=blue,solid,forget plot]
  table[row sep=crcr]{
0.177564700137613	-0.001556396484375	\\
0.177609091312647	-0.001617431640625	\\
0.177653482487681	-0.001190185546875	\\
0.177697873662716	-0.0010986328125	\\
0.17774226483775	-0.000701904296875	\\
0.177786656012785	-0.000823974609375	\\
0.177831047187819	-0.000885009765625	\\
0.177875438362853	-0.000579833984375	\\
0.177919829537888	-0.00103759765625	\\
0.177964220712922	-0.001220703125	\\
0.178008611887957	-0.00146484375	\\
0.178053003062991	-0.001678466796875	\\
0.178097394238025	-0.00189208984375	\\
0.17814178541306	-0.001739501953125	\\
0.178186176588094	-0.00140380859375	\\
0.178230567763129	-0.001617431640625	\\
0.178274958938163	-0.001495361328125	\\
0.178319350113197	-0.00140380859375	\\
0.178363741288232	-0.00177001953125	\\
0.178408132463266	-0.002105712890625	\\
0.178452523638301	-0.002197265625	\\
0.178496914813335	-0.00311279296875	\\
0.17854130598837	-0.00299072265625	\\
0.178585697163404	-0.003204345703125	\\
0.178630088338438	-0.00372314453125	\\
0.178674479513473	-0.003387451171875	\\
0.178718870688507	-0.0029296875	\\
0.178763261863542	-0.00250244140625	\\
0.178807653038576	-0.00238037109375	\\
0.17885204421361	-0.002227783203125	\\
0.178896435388645	-0.00244140625	\\
0.178940826563679	-0.001953125	\\
0.178985217738714	-0.00128173828125	\\
0.179029608913748	-0.00177001953125	\\
0.179074000088782	-0.001678466796875	\\
0.179118391263817	-0.0018310546875	\\
0.179162782438851	-0.002227783203125	\\
0.179207173613886	-0.00177001953125	\\
0.17925156478892	-0.001312255859375	\\
0.179295955963954	-0.00140380859375	\\
0.179340347138989	-0.000885009765625	\\
0.179384738314023	-0.000244140625	\\
0.179429129489058	6.103515625e-05	\\
0.179473520664092	0.0009765625	\\
0.179517911839126	0.0013427734375	\\
0.179562303014161	0.001312255859375	\\
0.179606694189195	0.00146484375	\\
0.17965108536423	0.001190185546875	\\
0.179695476539264	0.0010986328125	\\
0.179739867714298	0.001373291015625	\\
0.179784258889333	0.000732421875	\\
0.179828650064367	0.000152587890625	\\
0.179873041239402	-0.00018310546875	\\
0.179917432414436	-0.00048828125	\\
0.17996182358947	-0.00091552734375	\\
0.180006214764505	-0.00140380859375	\\
0.180050605939539	-0.00128173828125	\\
0.180094997114574	-0.001007080078125	\\
0.180139388289608	-0.001251220703125	\\
0.180183779464642	-0.0015869140625	\\
0.180228170639677	-0.001861572265625	\\
0.180272561814711	-0.002410888671875	\\
0.180316952989746	-0.00286865234375	\\
0.18036134416478	-0.00341796875	\\
0.180405735339814	-0.003692626953125	\\
0.180450126514849	-0.003570556640625	\\
0.180494517689883	-0.003387451171875	\\
0.180538908864918	-0.003204345703125	\\
0.180583300039952	-0.003143310546875	\\
0.180627691214986	-0.0028076171875	\\
0.180672082390021	-0.00262451171875	\\
0.180716473565055	-0.002349853515625	\\
0.18076086474009	-0.0018310546875	\\
0.180805255915124	-0.001434326171875	\\
0.180849647090158	-0.000946044921875	\\
0.180894038265193	-0.000823974609375	\\
0.180938429440227	-0.00030517578125	\\
0.180982820615262	0	\\
0.181027211790296	-0.00054931640625	\\
0.18107160296533	-0.000457763671875	\\
0.181115994140365	9.1552734375e-05	\\
0.181160385315399	0.000274658203125	\\
0.181204776490434	0.000732421875	\\
0.181249167665468	0.00128173828125	\\
0.181293558840502	0.00146484375	\\
0.181337950015537	0.001434326171875	\\
0.181382341190571	0.001312255859375	\\
0.181426732365606	0.001495361328125	\\
0.18147112354064	0.001373291015625	\\
0.181515514715675	0.00103759765625	\\
0.181559905890709	0.00079345703125	\\
0.181604297065743	0.000335693359375	\\
0.181648688240778	-3.0517578125e-05	\\
0.181693079415812	-0.000396728515625	\\
0.181737470590847	-0.00079345703125	\\
0.181781861765881	-0.0015869140625	\\
0.181826252940915	-0.002197265625	\\
0.18187064411595	-0.001953125	\\
0.181915035290984	-0.001983642578125	\\
0.181959426466019	-0.0018310546875	\\
0.182003817641053	-0.002197265625	\\
0.182048208816087	-0.002532958984375	\\
0.182092599991122	-0.001953125	\\
0.182136991166156	-0.001922607421875	\\
0.182181382341191	-0.00201416015625	\\
0.182225773516225	-0.002288818359375	\\
0.182270164691259	-0.002593994140625	\\
0.182314555866294	-0.0023193359375	\\
0.182358947041328	-0.002197265625	\\
0.182403338216363	-0.0023193359375	\\
0.182447729391397	-0.00177001953125	\\
0.182492120566431	-0.0015869140625	\\
0.182536511741466	-0.000885009765625	\\
0.1825809029165	-9.1552734375e-05	\\
0.182625294091535	9.1552734375e-05	\\
0.182669685266569	0.0003662109375	\\
0.182714076441603	0.000244140625	\\
0.182758467616638	0.0003662109375	\\
0.182802858791672	0.00030517578125	\\
0.182847249966707	0.000732421875	\\
0.182891641141741	0.00115966796875	\\
0.182936032316775	0.00054931640625	\\
0.18298042349181	0.000518798828125	\\
0.183024814666844	0.000885009765625	\\
0.183069205841879	0.000335693359375	\\
0.183113597016913	0.0006103515625	\\
0.183157988191947	0.0010986328125	\\
0.183202379366982	0.001373291015625	\\
0.183246770542016	0.001800537109375	\\
0.183291161717051	0.002471923828125	\\
0.183335552892085	0.002532958984375	\\
0.183379944067119	0.001983642578125	\\
0.183424335242154	0.001953125	\\
0.183468726417188	0.001129150390625	\\
0.183513117592223	0.00042724609375	\\
0.183557508767257	0.00018310546875	\\
0.183601899942291	-0.0006103515625	\\
0.183646291117326	-0.0003662109375	\\
0.18369068229236	-0.0001220703125	\\
0.183735073467395	-0.00067138671875	\\
0.183779464642429	-0.00067138671875	\\
0.183823855817463	-0.000213623046875	\\
0.183868246992498	-0.000396728515625	\\
0.183912638167532	-0.000335693359375	\\
0.183957029342567	-0.0001220703125	\\
0.184001420517601	-0.0001220703125	\\
0.184045811692636	-0.00042724609375	\\
0.18409020286767	-0.00042724609375	\\
0.184134594042704	-0.00079345703125	\\
0.184178985217739	-0.001312255859375	\\
0.184223376392773	-0.0009765625	\\
0.184267767567808	-0.001129150390625	\\
0.184312158742842	-0.001220703125	\\
0.184356549917876	-0.001251220703125	\\
0.184400941092911	-0.001251220703125	\\
0.184445332267945	-0.00091552734375	\\
0.18448972344298	-0.00152587890625	\\
0.184534114618014	-0.001556396484375	\\
0.184578505793048	-0.001434326171875	\\
0.184622896968083	-0.00189208984375	\\
0.184667288143117	-0.001861572265625	\\
0.184711679318152	-0.001983642578125	\\
0.184756070493186	-0.00201416015625	\\
0.18480046166822	-0.001861572265625	\\
0.184844852843255	-0.001739501953125	\\
0.184889244018289	-0.0020751953125	\\
0.184933635193324	-0.002471923828125	\\
0.184978026368358	-0.00213623046875	\\
0.185022417543392	-0.001983642578125	\\
0.185066808718427	-0.00152587890625	\\
0.185111199893461	-0.001373291015625	\\
0.185155591068496	-0.000823974609375	\\
0.18519998224353	-0.0001220703125	\\
0.185244373418564	-0.000244140625	\\
0.185288764593599	-0.00030517578125	\\
0.185333155768633	0	\\
0.185377546943668	-9.1552734375e-05	\\
0.185421938118702	0.000152587890625	\\
0.185466329293736	0.000213623046875	\\
0.185510720468771	-0.000518798828125	\\
0.185555111643805	-0.00079345703125	\\
0.18559950281884	-0.001312255859375	\\
0.185643893993874	-0.00177001953125	\\
0.185688285168908	-0.00146484375	\\
0.185732676343943	-0.00146484375	\\
0.185777067518977	-0.00140380859375	\\
0.185821458694012	-0.00140380859375	\\
0.185865849869046	-0.0013427734375	\\
0.18591024104408	-0.001556396484375	\\
0.185954632219115	-0.001953125	\\
0.185999023394149	-0.00250244140625	\\
0.186043414569184	-0.0032958984375	\\
0.186087805744218	-0.00372314453125	\\
0.186132196919252	-0.003875732421875	\\
0.186176588094287	-0.00360107421875	\\
0.186220979269321	-0.002685546875	\\
0.186265370444356	-0.002288818359375	\\
0.18630976161939	-0.002410888671875	\\
0.186354152794424	-0.002655029296875	\\
0.186398543969459	-0.0025634765625	\\
0.186442935144493	-0.00225830078125	\\
0.186487326319528	-0.001678466796875	\\
0.186531717494562	-0.00103759765625	\\
0.186576108669596	-0.001007080078125	\\
0.186620499844631	-0.00079345703125	\\
0.186664891019665	-0.001068115234375	\\
0.1867092821947	-0.001251220703125	\\
0.186753673369734	-0.001190185546875	\\
0.186798064544768	-0.00115966796875	\\
0.186842455719803	-0.0008544921875	\\
0.186886846894837	-0.000457763671875	\\
0.186931238069872	0.000213623046875	\\
0.186975629244906	0.000823974609375	\\
0.187020020419941	0.001251220703125	\\
0.187064411594975	0.00201416015625	\\
0.187108802770009	0.00225830078125	\\
0.187153193945044	0.00244140625	\\
0.187197585120078	0.00238037109375	\\
0.187241976295113	0.0020751953125	\\
0.187286367470147	0.00152587890625	\\
0.187330758645181	0.001190185546875	\\
0.187375149820216	0.001129150390625	\\
0.18741954099525	0.000732421875	\\
0.187463932170285	3.0517578125e-05	\\
0.187508323345319	-0.0006103515625	\\
0.187552714520353	-0.000732421875	\\
0.187597105695388	-0.000701904296875	\\
0.187641496870422	-0.0009765625	\\
0.187685888045457	-0.001190185546875	\\
0.187730279220491	-0.00164794921875	\\
0.187774670395525	-0.002227783203125	\\
0.18781906157056	-0.0025634765625	\\
0.187863452745594	-0.00299072265625	\\
0.187907843920629	-0.00286865234375	\\
0.187952235095663	-0.00274658203125	\\
0.187996626270697	-0.003082275390625	\\
0.188041017445732	-0.00335693359375	\\
0.188085408620766	-0.0029296875	\\
0.188129799795801	-0.002593994140625	\\
0.188174190970835	-0.001983642578125	\\
0.188218582145869	-0.002044677734375	\\
0.188262973320904	-0.001739501953125	\\
0.188307364495938	-0.00079345703125	\\
0.188351755670973	-0.000518798828125	\\
0.188396146846007	9.1552734375e-05	\\
0.188440538021041	0.000213623046875	\\
0.188484929196076	9.1552734375e-05	\\
0.18852932037111	0.00042724609375	\\
0.188573711546145	0.000640869140625	\\
0.188618102721179	0.000213623046875	\\
0.188662493896213	0.000152587890625	\\
0.188706885071248	0.000518798828125	\\
0.188751276246282	0.00079345703125	\\
0.188795667421317	0.001373291015625	\\
0.188840058596351	0.0013427734375	\\
0.188884449771385	0.00140380859375	\\
0.18892884094642	0.001129150390625	\\
0.188973232121454	0.000823974609375	\\
0.189017623296489	0.000579833984375	\\
0.189062014471523	-9.1552734375e-05	\\
0.189106405646557	-0.000640869140625	\\
0.189150796821592	-0.0009765625	\\
0.189195187996626	-0.00146484375	\\
0.189239579171661	-0.001556396484375	\\
0.189283970346695	-0.00225830078125	\\
0.189328361521729	-0.0023193359375	\\
0.189372752696764	-0.002197265625	\\
0.189417143871798	-0.002593994140625	\\
0.189461535046833	-0.003021240234375	\\
0.189505926221867	-0.002899169921875	\\
0.189550317396902	-0.002716064453125	\\
0.189594708571936	-0.00286865234375	\\
0.18963909974697	-0.002716064453125	\\
0.189683490922005	-0.00299072265625	\\
0.189727882097039	-0.002593994140625	\\
0.189772273272074	-0.002349853515625	\\
0.189816664447108	-0.002655029296875	\\
0.189861055622142	-0.002716064453125	\\
0.189905446797177	-0.002410888671875	\\
0.189949837972211	-0.001708984375	\\
0.189994229147246	-0.00140380859375	\\
0.19003862032228	-0.0009765625	\\
0.190083011497314	-0.00054931640625	\\
0.190127402672349	-0.000579833984375	\\
0.190171793847383	-3.0517578125e-05	\\
0.190216185022418	0.00115966796875	\\
0.190260576197452	0.001373291015625	\\
0.190304967372486	0.001617431640625	\\
0.190349358547521	0.001983642578125	\\
0.190393749722555	0.002166748046875	\\
0.19043814089759	0.00250244140625	\\
0.190482532072624	0.002166748046875	\\
0.190526923247658	0.001983642578125	\\
0.190571314422693	0.002105712890625	\\
0.190615705597727	0.002288818359375	\\
0.190660096772762	0.002227783203125	\\
0.190704487947796	0.0025634765625	\\
0.19074887912283	0.00238037109375	\\
0.190793270297865	0.001953125	\\
0.190837661472899	0.002349853515625	\\
0.190882052647934	0.001739501953125	\\
0.190926443822968	0.000885009765625	\\
0.190970834998002	0.000396728515625	\\
0.191015226173037	-3.0517578125e-05	\\
0.191059617348071	-0.000335693359375	\\
0.191104008523106	-0.000335693359375	\\
0.19114839969814	0.000274658203125	\\
0.191192790873174	-0.000244140625	\\
0.191237182048209	-0.00030517578125	\\
0.191281573223243	-0.000396728515625	\\
0.191325964398278	-0.000823974609375	\\
0.191370355573312	-0.000701904296875	\\
0.191414746748346	-0.000396728515625	\\
0.191459137923381	-0.000396728515625	\\
0.191503529098415	-0.00048828125	\\
0.19154792027345	-0.000152587890625	\\
0.191592311448484	-0.00030517578125	\\
0.191636702623518	-0.000518798828125	\\
0.191681093798553	-0.000518798828125	\\
0.191725484973587	-0.000732421875	\\
0.191769876148622	-0.00067138671875	\\
0.191814267323656	-0.000946044921875	\\
0.19185865849869	-0.000823974609375	\\
0.191903049673725	-0.000213623046875	\\
0.191947440848759	-0.0003662109375	\\
0.191991832023794	-0.000213623046875	\\
0.192036223198828	-0.000335693359375	\\
0.192080614373862	-0.000823974609375	\\
0.192125005548897	-0.001434326171875	\\
0.192169396723931	-0.001953125	\\
0.192213787898966	-0.0015869140625	\\
0.192258179074	-0.001922607421875	\\
0.192302570249034	-0.002593994140625	\\
0.192346961424069	-0.002685546875	\\
0.192391352599103	-0.0029296875	\\
0.192435743774138	-0.0025634765625	\\
0.192480134949172	-0.002227783203125	\\
0.192524526124207	-0.001800537109375	\\
0.192568917299241	-0.000762939453125	\\
0.192613308474275	-0.00030517578125	\\
0.19265769964931	0.0006103515625	\\
0.192702090824344	0.00048828125	\\
0.192746481999379	0.000274658203125	\\
0.192790873174413	0.0008544921875	\\
0.192835264349447	0.000823974609375	\\
0.192879655524482	0.00067138671875	\\
0.192924046699516	0.000701904296875	\\
0.192968437874551	0.000213623046875	\\
0.193012829049585	-0.000244140625	\\
0.193057220224619	-0.00048828125	\\
0.193101611399654	-0.000579833984375	\\
0.193146002574688	-0.000457763671875	\\
0.193190393749723	0.000274658203125	\\
0.193234784924757	0.000732421875	\\
0.193279176099791	0.000579833984375	\\
0.193323567274826	0.001068115234375	\\
0.19336795844986	0.00091552734375	\\
0.193412349624895	0.00030517578125	\\
0.193456740799929	9.1552734375e-05	\\
0.193501131974963	-0.00079345703125	\\
0.193545523149998	-0.0015869140625	\\
0.193589914325032	-0.002166748046875	\\
0.193634305500067	-0.001861572265625	\\
0.193678696675101	-0.0015869140625	\\
0.193723087850135	-0.00146484375	\\
0.19376747902517	-0.00067138671875	\\
0.193811870200204	-0.00048828125	\\
0.193856261375239	-0.0006103515625	\\
0.193900652550273	-0.00030517578125	\\
0.193945043725307	0.000213623046875	\\
0.193989434900342	0.00042724609375	\\
0.194033826075376	0.0009765625	\\
0.194078217250411	0.001251220703125	\\
0.194122608425445	0.000762939453125	\\
0.194166999600479	0.00048828125	\\
0.194211390775514	0.000274658203125	\\
0.194255781950548	0.000213623046875	\\
0.194300173125583	0.000640869140625	\\
0.194344564300617	0.001220703125	\\
0.194388955475651	0.001678466796875	\\
0.194433346650686	0.00201416015625	\\
0.19447773782572	0.0028076171875	\\
0.194522129000755	0.003448486328125	\\
0.194566520175789	0.00372314453125	\\
0.194610911350823	0.0040283203125	\\
0.194655302525858	0.004119873046875	\\
0.194699693700892	0.00408935546875	\\
0.194744084875927	0.0037841796875	\\
0.194788476050961	0.00299072265625	\\
0.194832867225995	0.00244140625	\\
0.19487725840103	0.0020751953125	\\
0.194921649576064	0.001495361328125	\\
0.194966040751099	0.000762939453125	\\
0.195010431926133	0.001007080078125	\\
0.195054823101167	0.00103759765625	\\
0.195099214276202	0.000732421875	\\
0.195143605451236	0.0006103515625	\\
0.195187996626271	0.000579833984375	\\
0.195232387801305	-0.00018310546875	\\
0.195276778976339	-0.00054931640625	\\
0.195321170151374	-0.0009765625	\\
0.195365561326408	-0.001739501953125	\\
0.195409952501443	-0.00152587890625	\\
0.195454343676477	-0.001495361328125	\\
0.195498734851512	-0.00152587890625	\\
0.195543126026546	-0.001434326171875	\\
0.19558751720158	-0.00140380859375	\\
0.195631908376615	-0.000732421875	\\
0.195676299551649	0.000152587890625	\\
0.195720690726684	0.000274658203125	\\
0.195765081901718	0.00048828125	\\
0.195809473076752	0.00091552734375	\\
0.195853864251787	0.001983642578125	\\
0.195898255426821	0.002777099609375	\\
0.195942646601856	0.00262451171875	\\
0.19598703777689	0.00244140625	\\
0.196031428951924	0.00286865234375	\\
0.196075820126959	0.003021240234375	\\
0.196120211301993	0.00274658203125	\\
0.196164602477028	0.00323486328125	\\
0.196208993652062	0.00323486328125	\\
0.196253384827096	0.00347900390625	\\
0.196297776002131	0.003997802734375	\\
0.196342167177165	0.003814697265625	\\
0.1963865583522	0.0035400390625	\\
0.196430949527234	0.003173828125	\\
0.196475340702268	0.003173828125	\\
0.196519731877303	0.002960205078125	\\
0.196564123052337	0.002716064453125	\\
0.196608514227372	0.00225830078125	\\
0.196652905402406	0.001495361328125	\\
0.19669729657744	0.00079345703125	\\
0.196741687752475	6.103515625e-05	\\
0.196786078927509	-0.00042724609375	\\
0.196830470102544	-0.00079345703125	\\
0.196874861277578	-0.000732421875	\\
0.196919252452612	-0.0008544921875	\\
0.196963643627647	-0.001251220703125	\\
0.197008034802681	-0.001373291015625	\\
0.197052425977716	-0.001708984375	\\
0.19709681715275	-0.001739501953125	\\
0.197141208327784	-0.001495361328125	\\
0.197185599502819	-0.001495361328125	\\
0.197229990677853	-0.001373291015625	\\
0.197274381852888	-0.00140380859375	\\
0.197318773027922	-0.001190185546875	\\
0.197363164202956	-0.000701904296875	\\
0.197407555377991	-0.000518798828125	\\
0.197451946553025	0.000152587890625	\\
0.19749633772806	0.000335693359375	\\
0.197540728903094	-3.0517578125e-05	\\
0.197585120078128	0.00018310546875	\\
0.197629511253163	0.0008544921875	\\
0.197673902428197	0.001556396484375	\\
0.197718293603232	0.00225830078125	\\
0.197762684778266	0.002716064453125	\\
0.1978070759533	0.0029296875	\\
0.197851467128335	0.002899169921875	\\
0.197895858303369	0.003204345703125	\\
0.197940249478404	0.0035400390625	\\
0.197984640653438	0.003570556640625	\\
0.198029031828473	0.0030517578125	\\
0.198073423003507	0.002960205078125	\\
0.198117814178541	0.002716064453125	\\
0.198162205353576	0.0029296875	\\
0.19820659652861	0.0030517578125	\\
0.198250987703645	0.0030517578125	\\
0.198295378878679	0.0028076171875	\\
0.198339770053713	0.002777099609375	\\
0.198384161228748	0.0025634765625	\\
0.198428552403782	0.002197265625	\\
0.198472943578817	0.00225830078125	\\
0.198517334753851	0.001800537109375	\\
0.198561725928885	0.00177001953125	\\
0.19860611710392	0.001495361328125	\\
0.198650508278954	0.001220703125	\\
0.198694899453989	0.0015869140625	\\
0.198739290629023	0.00103759765625	\\
0.198783681804057	0.001373291015625	\\
0.198828072979092	0.001861572265625	\\
0.198872464154126	0.001678466796875	\\
0.198916855329161	0.0020751953125	\\
0.198961246504195	0.002197265625	\\
0.199005637679229	0.001983642578125	\\
0.199050028854264	0.00201416015625	\\
0.199094420029298	0.002288818359375	\\
0.199138811204333	0.00225830078125	\\
0.199183202379367	0.001800537109375	\\
0.199227593554401	0.00140380859375	\\
0.199271984729436	0.00146484375	\\
0.19931637590447	0.001434326171875	\\
0.199360767079505	0.001495361328125	\\
0.199405158254539	0.00115966796875	\\
0.199449549429573	0.000457763671875	\\
0.199493940604608	0.000274658203125	\\
0.199538331779642	0.000274658203125	\\
0.199582722954677	0.00018310546875	\\
0.199627114129711	0.000244140625	\\
0.199671505304745	-0.000244140625	\\
0.19971589647978	-6.103515625e-05	\\
0.199760287654814	-0.000152587890625	\\
0.199804678829849	-0.00103759765625	\\
0.199849070004883	-0.001739501953125	\\
0.199893461179917	-0.0018310546875	\\
0.199937852354952	-0.0018310546875	\\
0.199982243529986	-0.001373291015625	\\
0.200026634705021	-0.000732421875	\\
0.200071025880055	-0.00042724609375	\\
0.200115417055089	3.0517578125e-05	\\
0.200159808230124	0.000885009765625	\\
0.200204199405158	0.001495361328125	\\
0.200248590580193	0.00201416015625	\\
0.200292981755227	0.002532958984375	\\
0.200337372930261	0.0025634765625	\\
0.200381764105296	0.00262451171875	\\
0.20042615528033	0.001953125	\\
0.200470546455365	0.001220703125	\\
0.200514937630399	0.00091552734375	\\
0.200559328805433	0.000213623046875	\\
0.200603719980468	0.000396728515625	\\
0.200648111155502	0.00103759765625	\\
0.200692502330537	0.00128173828125	\\
0.200736893505571	0.001708984375	\\
0.200781284680605	0.002105712890625	\\
0.20082567585564	0.001708984375	\\
0.200870067030674	0.001220703125	\\
0.200914458205709	0.000579833984375	\\
0.200958849380743	-0.0003662109375	\\
0.201003240555778	-0.00152587890625	\\
0.201047631730812	-0.002197265625	\\
0.201092022905846	-0.00286865234375	\\
0.201136414080881	-0.00286865234375	\\
0.201180805255915	-0.002410888671875	\\
0.20122519643095	-0.001983642578125	\\
0.201269587605984	-0.001983642578125	\\
0.201313978781018	-0.001861572265625	\\
0.201358369956053	-0.00128173828125	\\
0.201402761131087	-0.0008544921875	\\
0.201447152306122	-0.00042724609375	\\
0.201491543481156	0.000457763671875	\\
0.20153593465619	0.001068115234375	\\
0.201580325831225	0.000396728515625	\\
0.201624717006259	-0.000152587890625	\\
0.201669108181294	-9.1552734375e-05	\\
0.201713499356328	-6.103515625e-05	\\
0.201757890531362	3.0517578125e-05	\\
0.201802281706397	0.000457763671875	\\
0.201846672881431	0.00152587890625	\\
0.201891064056466	0.002227783203125	\\
0.2019354552315	0.003326416015625	\\
0.201979846406534	0.004180908203125	\\
0.202024237581569	0.00445556640625	\\
0.202068628756603	0.004730224609375	\\
0.202113019931638	0.004791259765625	\\
0.202157411106672	0.00433349609375	\\
0.202201802281706	0.004058837890625	\\
0.202246193456741	0.003570556640625	\\
0.202290584631775	0.00323486328125	\\
0.20233497580681	0.002593994140625	\\
0.202379366981844	0.001434326171875	\\
0.202423758156878	0.000244140625	\\
0.202468149331913	-0.00042724609375	\\
0.202512540506947	-0.00054931640625	\\
0.202556931681982	-0.000579833984375	\\
0.202601322857016	-0.000640869140625	\\
0.20264571403205	-0.000579833984375	\\
0.202690105207085	-0.00048828125	\\
0.202734496382119	-0.000823974609375	\\
0.202778887557154	-0.001129150390625	\\
0.202823278732188	-0.00128173828125	\\
0.202867669907222	-0.0018310546875	\\
0.202912061082257	-0.0023193359375	\\
0.202956452257291	-0.00238037109375	\\
0.203000843432326	-0.002227783203125	\\
0.20304523460736	-0.001708984375	\\
0.203089625782394	-0.001556396484375	\\
0.203134016957429	-0.0013427734375	\\
0.203178408132463	-0.0006103515625	\\
0.203222799307498	0.00018310546875	\\
0.203267190482532	0.0008544921875	\\
0.203311581657566	0.00177001953125	\\
0.203355972832601	0.0023193359375	\\
0.203400364007635	0.00238037109375	\\
0.20344475518267	0.002838134765625	\\
0.203489146357704	0.002899169921875	\\
0.203533537532738	0.002685546875	\\
0.203577928707773	0.002471923828125	\\
0.203622319882807	0.00238037109375	\\
0.203666711057842	0.002166748046875	\\
0.203711102232876	0.0023193359375	\\
0.203755493407911	0.0029296875	\\
0.203799884582945	0.002716064453125	\\
0.203844275757979	0.00262451171875	\\
0.203888666933014	0.00225830078125	\\
0.203933058108048	0.0020751953125	\\
0.203977449283083	0.002105712890625	\\
0.204021840458117	0.0013427734375	\\
0.204066231633151	0.00048828125	\\
0.204110622808186	-0.000213623046875	\\
0.20415501398322	-0.000732421875	\\
0.204199405158255	-0.000640869140625	\\
0.204243796333289	-0.0018310546875	\\
0.204288187508323	-0.002532958984375	\\
0.204332578683358	-0.00274658203125	\\
0.204376969858392	-0.0025634765625	\\
0.204421361033427	-0.002197265625	\\
0.204465752208461	-0.001983642578125	\\
0.204510143383495	-0.001800537109375	\\
0.20455453455853	-0.00189208984375	\\
0.204598925733564	-0.001922607421875	\\
0.204643316908599	-0.001312255859375	\\
0.204687708083633	-0.0015869140625	\\
0.204732099258667	-0.001556396484375	\\
0.204776490433702	-0.001495361328125	\\
0.204820881608736	-0.00146484375	\\
0.204865272783771	-0.00103759765625	\\
0.204909663958805	-0.0006103515625	\\
0.204954055133839	0.00018310546875	\\
0.204998446308874	0.000732421875	\\
0.205042837483908	0.001312255859375	\\
0.205087228658943	0.00152587890625	\\
0.205131619833977	0.001953125	\\
0.205176011009011	0.002349853515625	\\
0.205220402184046	0.00299072265625	\\
0.20526479335908	0.003326416015625	\\
0.205309184534115	0.00311279296875	\\
0.205353575709149	0.0029296875	\\
0.205397966884183	0.002716064453125	\\
0.205442358059218	0.002471923828125	\\
0.205486749234252	0.002410888671875	\\
0.205531140409287	0.0020751953125	\\
0.205575531584321	0.00177001953125	\\
0.205619922759355	0.001739501953125	\\
0.20566431393439	0.001434326171875	\\
0.205708705109424	0.001007080078125	\\
0.205753096284459	0.0009765625	\\
0.205797487459493	0.00128173828125	\\
0.205841878634527	0.001495361328125	\\
0.205886269809562	0.001739501953125	\\
0.205930660984596	0.00164794921875	\\
0.205975052159631	0.0010986328125	\\
0.206019443334665	0.001556396484375	\\
0.206063834509699	0.001373291015625	\\
0.206108225684734	0.000885009765625	\\
0.206152616859768	0.00079345703125	\\
0.206197008034803	0.000213623046875	\\
0.206241399209837	0.00042724609375	\\
0.206285790384871	0.00103759765625	\\
0.206330181559906	0.00140380859375	\\
0.20637457273494	0.001373291015625	\\
0.206418963909975	0.001495361328125	\\
0.206463355085009	0.001861572265625	\\
0.206507746260044	0.001739501953125	\\
0.206552137435078	0.002044677734375	\\
0.206596528610112	0.002105712890625	\\
0.206640919785147	0.002044677734375	\\
0.206685310960181	0.001953125	\\
0.206729702135216	0.001708984375	\\
0.20677409331025	0.001251220703125	\\
0.206818484485284	0.000762939453125	\\
0.206862875660319	0.0001220703125	\\
0.206907266835353	-0.00030517578125	\\
0.206951658010388	-0.000213623046875	\\
0.206996049185422	-0.0003662109375	\\
0.207040440360456	-0.000885009765625	\\
0.207084831535491	-0.000823974609375	\\
0.207129222710525	-0.00079345703125	\\
0.20717361388556	-0.001678466796875	\\
0.207218005060594	-0.001617431640625	\\
0.207262396235628	-0.001922607421875	\\
0.207306787410663	-0.0025634765625	\\
0.207351178585697	-0.002716064453125	\\
0.207395569760732	-0.002593994140625	\\
0.207439960935766	-0.002288818359375	\\
0.2074843521108	-0.00152587890625	\\
0.207528743285835	-0.001007080078125	\\
0.207573134460869	-0.0009765625	\\
0.207617525635904	-6.103515625e-05	\\
0.207661916810938	0.000335693359375	\\
0.207706307985972	0.001129150390625	\\
0.207750699161007	0.00213623046875	\\
0.207795090336041	0.00177001953125	\\
0.207839481511076	0.001434326171875	\\
0.20788387268611	0.001373291015625	\\
0.207928263861144	0.000946044921875	\\
0.207972655036179	0.000335693359375	\\
0.208017046211213	6.103515625e-05	\\
0.208061437386248	0.000213623046875	\\
0.208105828561282	0.00042724609375	\\
0.208150219736316	0.000640869140625	\\
0.208194610911351	0.0013427734375	\\
0.208239002086385	0.00115966796875	\\
0.20828339326142	0.000732421875	\\
0.208327784436454	0.001007080078125	\\
0.208372175611488	0.00048828125	\\
0.208416566786523	6.103515625e-05	\\
0.208460957961557	-0.000823974609375	\\
0.208505349136592	-0.001708984375	\\
0.208549740311626	-0.002410888671875	\\
0.20859413148666	-0.00286865234375	\\
0.208638522661695	-0.003021240234375	\\
0.208682913836729	-0.003204345703125	\\
0.208727305011764	-0.002593994140625	\\
0.208771696186798	-0.002197265625	\\
0.208816087361832	-0.00189208984375	\\
0.208860478536867	-0.0009765625	\\
0.208904869711901	-0.00018310546875	\\
0.208949260886936	0.00048828125	\\
0.20899365206197	0.000762939453125	\\
0.209038043237004	0.000518798828125	\\
0.209082434412039	0.0001220703125	\\
0.209126825587073	-0.000244140625	\\
0.209171216762108	-3.0517578125e-05	\\
0.209215607937142	0.00030517578125	\\
0.209259999112176	0.00054931640625	\\
0.209304390287211	0.001220703125	\\
0.209348781462245	0.001953125	\\
0.20939317263728	0.00274658203125	\\
0.209437563812314	0.00323486328125	\\
0.209481954987349	0.003875732421875	\\
0.209526346162383	0.0045166015625	\\
0.209570737337417	0.004425048828125	\\
0.209615128512452	0.004852294921875	\\
0.209659519687486	0.0045166015625	\\
0.209703910862521	0.003936767578125	\\
0.209748302037555	0.003448486328125	\\
0.209792693212589	0.0025634765625	\\
0.209837084387624	0.001983642578125	\\
0.209881475562658	0.0010986328125	\\
0.209925866737693	0.000213623046875	\\
0.209970257912727	-0.00030517578125	\\
0.210014649087761	-0.000640869140625	\\
0.210059040262796	-0.0006103515625	\\
0.21010343143783	-0.000244140625	\\
0.210147822612865	-0.00042724609375	\\
0.210192213787899	-0.001251220703125	\\
0.210236604962933	-0.001861572265625	\\
0.210280996137968	-0.0023193359375	\\
0.210325387313002	-0.002593994140625	\\
0.210369778488037	-0.0029296875	\\
0.210414169663071	-0.003021240234375	\\
0.210458560838105	-0.002685546875	\\
0.21050295201314	-0.00286865234375	\\
0.210547343188174	-0.00238037109375	\\
0.210591734363209	-0.001678466796875	\\
0.210636125538243	-0.00115966796875	\\
0.210680516713277	-0.000335693359375	\\
0.210724907888312	3.0517578125e-05	\\
0.210769299063346	0.00048828125	\\
0.210813690238381	0.00128173828125	\\
0.210858081413415	0.00189208984375	\\
0.210902472588449	0.002044677734375	\\
0.210946863763484	0.0025634765625	\\
0.210991254938518	0.002838134765625	\\
0.211035646113553	0.0025634765625	\\
0.211080037288587	0.002685546875	\\
0.211124428463621	0.00262451171875	\\
0.211168819638656	0.00250244140625	\\
0.21121321081369	0.0020751953125	\\
0.211257601988725	0.002227783203125	\\
0.211301993163759	0.002197265625	\\
0.211346384338793	0.001556396484375	\\
0.211390775513828	0.001739501953125	\\
0.211435166688862	0.001312255859375	\\
0.211479557863897	0.000732421875	\\
0.211523949038931	0.000518798828125	\\
0.211568340213965	-0.000396728515625	\\
0.211612731389	-0.001068115234375	\\
0.211657122564034	-0.0015869140625	\\
0.211701513739069	-0.002532958984375	\\
0.211745904914103	-0.002960205078125	\\
0.211790296089137	-0.003326416015625	\\
0.211834687264172	-0.003631591796875	\\
0.211879078439206	-0.00396728515625	\\
0.211923469614241	-0.00372314453125	\\
0.211967860789275	-0.00360107421875	\\
0.21201225196431	-0.00323486328125	\\
0.212056643139344	-0.002838134765625	\\
0.212101034314378	-0.002960205078125	\\
0.212145425489413	-0.002960205078125	\\
0.212189816664447	-0.0029296875	\\
0.212234207839482	-0.0028076171875	\\
0.212278599014516	-0.002166748046875	\\
0.21232299018955	-0.0018310546875	\\
0.212367381364585	-0.00140380859375	\\
0.212411772539619	-0.000579833984375	\\
0.212456163714654	-0.000152587890625	\\
0.212500554889688	0.000640869140625	\\
0.212544946064722	0.001129150390625	\\
0.212589337239757	0.002044677734375	\\
0.212633728414791	0.002532958984375	\\
0.212678119589826	0.002349853515625	\\
0.21272251076486	0.003021240234375	\\
0.212766901939894	0.003265380859375	\\
0.212811293114929	0.003143310546875	\\
0.212855684289963	0.0032958984375	\\
0.212900075464998	0.0032958984375	\\
0.212944466640032	0.00341796875	\\
0.212988857815066	0.003143310546875	\\
0.213033248990101	0.002838134765625	\\
0.213077640165135	0.0029296875	\\
0.21312203134017	0.003021240234375	\\
0.213166422515204	0.002655029296875	\\
0.213210813690238	0.002532958984375	\\
0.213255204865273	0.00274658203125	\\
0.213299596040307	0.0025634765625	\\
0.213343987215342	0.002105712890625	\\
0.213388378390376	0.00189208984375	\\
0.21343276956541	0.001739501953125	\\
0.213477160740445	0.00164794921875	\\
0.213521551915479	0.0020751953125	\\
0.213565943090514	0.00140380859375	\\
0.213610334265548	0.001068115234375	\\
0.213654725440582	0.00146484375	\\
0.213699116615617	0.001739501953125	\\
0.213743507790651	0.0015869140625	\\
0.213787898965686	0.001922607421875	\\
0.21383229014072	0.0025634765625	\\
0.213876681315754	0.00274658203125	\\
0.213921072490789	0.0028076171875	\\
0.213965463665823	0.00286865234375	\\
0.214009854840858	0.003204345703125	\\
0.214054246015892	0.003021240234375	\\
0.214098637190926	0.002777099609375	\\
0.214143028365961	0.00244140625	\\
0.214187419540995	0.002044677734375	\\
0.21423181071603	0.001739501953125	\\
0.214276201891064	0.00152587890625	\\
0.214320593066098	0.00128173828125	\\
0.214364984241133	0.001007080078125	\\
0.214409375416167	0.000396728515625	\\
0.214453766591202	0.000396728515625	\\
0.214498157766236	6.103515625e-05	\\
0.21454254894127	0	\\
0.214586940116305	-0.000396728515625	\\
0.214631331291339	-0.001190185546875	\\
0.214675722466374	-0.00152587890625	\\
0.214720113641408	-0.0015869140625	\\
0.214764504816442	-0.002227783203125	\\
0.214808895991477	-0.00213623046875	\\
0.214853287166511	-0.0020751953125	\\
0.214897678341546	-0.00244140625	\\
0.21494206951658	-0.002197265625	\\
0.214986460691615	-0.001678466796875	\\
0.215030851866649	-0.001251220703125	\\
0.215075243041683	-0.00054931640625	\\
0.215119634216718	0.0003662109375	\\
0.215164025391752	0.00115966796875	\\
0.215208416566787	0.001922607421875	\\
0.215252807741821	0.0025634765625	\\
0.215297198916855	0.002593994140625	\\
0.21534159009189	0.0023193359375	\\
0.215385981266924	0.002655029296875	\\
0.215430372441959	0.002288818359375	\\
0.215474763616993	0.00164794921875	\\
0.215519154792027	0.00115966796875	\\
0.215563545967062	0.0010986328125	\\
0.215607937142096	0.0013427734375	\\
0.215652328317131	0.001556396484375	\\
0.215696719492165	0.001739501953125	\\
0.215741110667199	0.0018310546875	\\
0.215785501842234	0.00177001953125	\\
0.215829893017268	0.0015869140625	\\
0.215874284192303	0.001800537109375	\\
0.215918675367337	0.001220703125	\\
0.215963066542371	0.000396728515625	\\
0.216007457717406	-9.1552734375e-05	\\
0.21605184889244	-0.001068115234375	\\
0.216096240067475	-0.0020751953125	\\
0.216140631242509	-0.00244140625	\\
0.216185022417543	-0.002227783203125	\\
0.216229413592578	-0.001678466796875	\\
0.216273804767612	-0.001434326171875	\\
0.216318195942647	-0.001068115234375	\\
0.216362587117681	-0.000762939453125	\\
0.216406978292715	-0.000213623046875	\\
0.21645136946775	0.0008544921875	\\
0.216495760642784	0.00140380859375	\\
0.216540151817819	0.001312255859375	\\
0.216584542992853	0.001129150390625	\\
0.216628934167887	0.0013427734375	\\
0.216673325342922	0.0013427734375	\\
0.216717716517956	0.001251220703125	\\
0.216762107692991	0.001495361328125	\\
0.216806498868025	0.001739501953125	\\
0.216850890043059	0.002197265625	\\
0.216895281218094	0.00286865234375	\\
0.216939672393128	0.003631591796875	\\
0.216984063568163	0.004364013671875	\\
0.217028454743197	0.00494384765625	\\
0.217072845918231	0.005218505859375	\\
0.217117237093266	0.00531005859375	\\
0.2171616282683	0.005126953125	\\
0.217206019443335	0.004486083984375	\\
0.217250410618369	0.003875732421875	\\
0.217294801793403	0.003692626953125	\\
0.217339192968438	0.002197265625	\\
0.217383584143472	0.000946044921875	\\
0.217427975318507	0.00030517578125	\\
0.217472366493541	-0.000518798828125	\\
0.217516757668575	-0.00067138671875	\\
0.21756114884361	-0.0008544921875	\\
0.217605540018644	-0.00128173828125	\\
0.217649931193679	-0.001251220703125	\\
0.217694322368713	-0.001495361328125	\\
0.217738713543747	-0.001678466796875	\\
0.217783104718782	-0.001678466796875	\\
0.217827495893816	-0.00225830078125	\\
0.217871887068851	-0.002777099609375	\\
0.217916278243885	-0.00286865234375	\\
0.21796066941892	-0.0028076171875	\\
0.218005060593954	-0.002838134765625	\\
0.218049451768988	-0.002655029296875	\\
0.218093842944023	-0.0023193359375	\\
0.218138234119057	-0.00189208984375	\\
0.218182625294092	-0.001068115234375	\\
0.218227016469126	9.1552734375e-05	\\
0.21827140764416	0.00103759765625	\\
0.218315798819195	0.001434326171875	\\
0.218360189994229	0.002166748046875	\\
0.218404581169264	0.00299072265625	\\
0.218448972344298	0.002593994140625	\\
0.218493363519332	0.002471923828125	\\
0.218537754694367	0.00311279296875	\\
0.218582145869401	0.003173828125	\\
0.218626537044436	0.00274658203125	\\
0.21867092821947	0.00244140625	\\
0.218715319394504	0.002716064453125	\\
0.218759710569539	0.003173828125	\\
0.218804101744573	0.002838134765625	\\
0.218848492919608	0.00213623046875	\\
0.218892884094642	0.002166748046875	\\
0.218937275269676	0.00213623046875	\\
0.218981666444711	0.00189208984375	\\
0.219026057619745	0.001495361328125	\\
0.21907044879478	0.000762939453125	\\
0.219114839969814	-0.000335693359375	\\
0.219159231144848	-0.001190185546875	\\
0.219203622319883	-0.001312255859375	\\
0.219248013494917	-0.001922607421875	\\
0.219292404669952	-0.0029296875	\\
0.219336795844986	-0.003173828125	\\
0.21938118702002	-0.003875732421875	\\
0.219425578195055	-0.004241943359375	\\
0.219469969370089	-0.00390625	\\
0.219514360545124	-0.0037841796875	\\
0.219558751720158	-0.003173828125	\\
0.219603142895192	-0.002685546875	\\
0.219647534070227	-0.002105712890625	\\
0.219691925245261	-0.0020751953125	\\
0.219736316420296	-0.0020751953125	\\
0.21978070759533	-0.001922607421875	\\
0.219825098770364	-0.00201416015625	\\
0.219869489945399	-0.001800537109375	\\
0.219913881120433	-0.001312255859375	\\
0.219958272295468	-0.0008544921875	\\
0.220002663470502	-0.0008544921875	\\
0.220047054645536	-0.00042724609375	\\
0.220091445820571	0.00042724609375	\\
0.220135836995605	0.00140380859375	\\
0.22018022817064	0.002349853515625	\\
0.220224619345674	0.002716064453125	\\
0.220269010520708	0.002777099609375	\\
0.220313401695743	0.002685546875	\\
0.220357792870777	0.002349853515625	\\
0.220402184045812	0.00250244140625	\\
0.220446575220846	0.0028076171875	\\
0.220490966395881	0.002838134765625	\\
0.220535357570915	0.002777099609375	\\
0.220579748745949	0.00250244140625	\\
0.220624139920984	0.002349853515625	\\
0.220668531096018	0.002288818359375	\\
0.220712922271053	0.001953125	\\
0.220757313446087	0.001617431640625	\\
0.220801704621121	0.00140380859375	\\
0.220846095796156	0.001007080078125	\\
0.22089048697119	0.00128173828125	\\
0.220934878146225	0.001190185546875	\\
0.220979269321259	0.001251220703125	\\
0.221023660496293	0.001068115234375	\\
0.221068051671328	0.000823974609375	\\
0.221112442846362	0.001251220703125	\\
0.221156834021397	0.001251220703125	\\
0.221201225196431	0.000823974609375	\\
0.221245616371465	0.000457763671875	\\
0.2212900075465	6.103515625e-05	\\
0.221334398721534	0.000244140625	\\
0.221378789896569	0.0006103515625	\\
0.221423181071603	0.000823974609375	\\
0.221467572246637	0.000640869140625	\\
0.221511963421672	0.000701904296875	\\
0.221556354596706	0.0009765625	\\
0.221600745771741	0.00103759765625	\\
0.221645136946775	0.00115966796875	\\
0.221689528121809	0.001190185546875	\\
0.221733919296844	0.000762939453125	\\
0.221778310471878	0.000518798828125	\\
0.221822701646913	0.000396728515625	\\
0.221867092821947	0.00042724609375	\\
0.221911483996981	6.103515625e-05	\\
0.221955875172016	-0.000732421875	\\
0.22200026634705	-0.0009765625	\\
0.222044657522085	-0.0009765625	\\
0.222089048697119	-0.001129150390625	\\
0.222133439872153	-0.001251220703125	\\
0.222177831047188	-0.001861572265625	\\
0.222222222222222	-0.00213623046875	\\
0.222266613397257	-0.002166748046875	\\
0.222311004572291	-0.002471923828125	\\
0.222355395747325	-0.00238037109375	\\
0.22239978692236	-0.0025634765625	\\
0.222444178097394	-0.00274658203125	\\
0.222488569272429	-0.0025634765625	\\
0.222532960447463	-0.002685546875	\\
0.222577351622497	-0.0025634765625	\\
0.222621742797532	-0.00238037109375	\\
0.222666133972566	-0.001739501953125	\\
0.222710525147601	-0.00067138671875	\\
0.222754916322635	-0.000701904296875	\\
0.222799307497669	0.000244140625	\\
0.222843698672704	0.001434326171875	\\
0.222888089847738	0.001953125	\\
0.222932481022773	0.002471923828125	\\
0.222976872197807	0.00250244140625	\\
0.223021263372841	0.00244140625	\\
0.223065654547876	0.0020751953125	\\
0.22311004572291	0.0015869140625	\\
0.223154436897945	0.00140380859375	\\
0.223198828072979	0.00140380859375	\\
0.223243219248013	0.001312255859375	\\
0.223287610423048	0.000946044921875	\\
0.223332001598082	0.00091552734375	\\
0.223376392773117	0.00128173828125	\\
0.223420783948151	0.00128173828125	\\
0.223465175123186	0.0008544921875	\\
0.22350956629822	0.000274658203125	\\
0.223553957473254	-3.0517578125e-05	\\
0.223598348648289	0.00018310546875	\\
0.223642739823323	-0.00067138671875	\\
0.223687130998358	-0.001495361328125	\\
0.223731522173392	-0.002166748046875	\\
0.223775913348426	-0.002685546875	\\
0.223820304523461	-0.002685546875	\\
0.223864695698495	-0.003173828125	\\
0.22390908687353	-0.002685546875	\\
0.223953478048564	-0.0030517578125	\\
0.223997869223598	-0.002593994140625	\\
0.224042260398633	-0.001434326171875	\\
0.224086651573667	-0.001220703125	\\
0.224131042748702	-0.00030517578125	\\
0.224175433923736	0.000244140625	\\
0.22421982509877	0.0006103515625	\\
0.224264216273805	0.00067138671875	\\
0.224308607448839	0.000732421875	\\
0.224352998623874	0.000823974609375	\\
0.224397389798908	0.000732421875	\\
0.224441780973942	0.001251220703125	\\
0.224486172148977	0.0013427734375	\\
0.224530563324011	0.001739501953125	\\
0.224574954499046	0.00238037109375	\\
0.22461934567408	0.002777099609375	\\
0.224663736849114	0.0030517578125	\\
0.224708128024149	0.003662109375	\\
0.224752519199183	0.004852294921875	\\
0.224796910374218	0.004730224609375	\\
0.224841301549252	0.004730224609375	\\
0.224885692724286	0.004638671875	\\
0.224930083899321	0.00384521484375	\\
0.224974475074355	0.002960205078125	\\
0.22501886624939	0.0018310546875	\\
0.225063257424424	0.00140380859375	\\
0.225107648599458	0.0008544921875	\\
0.225152039774493	0.0001220703125	\\
0.225196430949527	-0.000152587890625	\\
0.225240822124562	-0.000518798828125	\\
0.225285213299596	-0.001068115234375	\\
0.22532960447463	-0.001251220703125	\\
0.225373995649665	-0.00140380859375	\\
0.225418386824699	-0.001495361328125	\\
0.225462777999734	-0.001953125	\\
0.225507169174768	-0.001953125	\\
0.225551560349802	-0.001739501953125	\\
0.225595951524837	-0.001983642578125	\\
0.225640342699871	-0.00238037109375	\\
0.225684733874906	-0.002410888671875	\\
0.22572912504994	-0.002471923828125	\\
0.225773516224974	-0.00225830078125	\\
0.225817907400009	-0.001556396484375	\\
0.225862298575043	-0.001190185546875	\\
0.225906689750078	-0.000518798828125	\\
0.225951080925112	0.0003662109375	\\
0.225995472100146	0.00152587890625	\\
0.226039863275181	0.00238037109375	\\
0.226084254450215	0.002532958984375	\\
0.22612864562525	0.0029296875	\\
0.226173036800284	0.003173828125	\\
0.226217427975319	0.0030517578125	\\
0.226261819150353	0.002410888671875	\\
0.226306210325387	0.001983642578125	\\
0.226350601500422	0.002288818359375	\\
0.226394992675456	0.0020751953125	\\
0.226439383850491	0.001922607421875	\\
0.226483775025525	0.001983642578125	\\
0.226528166200559	0.00201416015625	\\
0.226572557375594	0.002471923828125	\\
0.226616948550628	0.002197265625	\\
0.226661339725663	0.001556396484375	\\
0.226705730900697	0.001556396484375	\\
0.226750122075731	0.001068115234375	\\
0.226794513250766	0.00030517578125	\\
0.2268389044258	-0.00030517578125	\\
0.226883295600835	-0.001251220703125	\\
0.226927686775869	-0.00225830078125	\\
0.226972077950903	-0.003204345703125	\\
0.227016469125938	-0.00396728515625	\\
0.227060860300972	-0.00408935546875	\\
0.227105251476007	-0.00421142578125	\\
0.227149642651041	-0.004119873046875	\\
0.227194033826075	-0.00384521484375	\\
0.22723842500111	-0.00347900390625	\\
0.227282816176144	-0.002716064453125	\\
0.227327207351179	-0.00189208984375	\\
0.227371598526213	-0.00189208984375	\\
0.227415989701247	-0.0013427734375	\\
0.227460380876282	-0.000732421875	\\
0.227504772051316	-0.00048828125	\\
0.227549163226351	-0.00054931640625	\\
0.227593554401385	-0.0006103515625	\\
0.227637945576419	-3.0517578125e-05	\\
0.227682336751454	0.00048828125	\\
0.227726727926488	0.00054931640625	\\
0.227771119101523	0.0008544921875	\\
0.227815510276557	0.0018310546875	\\
0.227859901451591	0.002685546875	\\
0.227904292626626	0.003326416015625	\\
0.22794868380166	0.004364013671875	\\
0.227993074976695	0.00457763671875	\\
0.228037466151729	0.004302978515625	\\
0.228081857326763	0.00390625	\\
0.228126248501798	0.003173828125	\\
0.228170639676832	0.002838134765625	\\
0.228215030851867	0.00274658203125	\\
0.228259422026901	0.002838134765625	\\
0.228303813201935	0.001953125	\\
0.22834820437697	0.001678466796875	\\
0.228392595552004	0.001556396484375	\\
0.228436986727039	0.0010986328125	\\
0.228481377902073	0.001129150390625	\\
0.228525769077107	0.001129150390625	\\
0.228570160252142	0.00103759765625	\\
0.228614551427176	0.001068115234375	\\
0.228658942602211	0.00115966796875	\\
0.228703333777245	0.000946044921875	\\
0.228747724952279	0.00018310546875	\\
0.228792116127314	-0.00042724609375	\\
0.228836507302348	-0.0003662109375	\\
0.228880898477383	-0.00018310546875	\\
0.228925289652417	-0.00030517578125	\\
0.228969680827452	-0.000335693359375	\\
0.229014072002486	0.00030517578125	\\
0.22905846317752	0.0008544921875	\\
0.229102854352555	0.000701904296875	\\
0.229147245527589	0.00103759765625	\\
0.229191636702624	0.00152587890625	\\
0.229236027877658	0.0013427734375	\\
0.229280419052692	0.001953125	\\
0.229324810227727	0.00189208984375	\\
0.229369201402761	0.001983642578125	\\
0.229413592577796	0.00244140625	\\
0.22945798375283	0.00238037109375	\\
0.229502374927864	0.002166748046875	\\
0.229546766102899	0.0018310546875	\\
0.229591157277933	0.001678466796875	\\
0.229635548452968	0.001495361328125	\\
0.229679939628002	0.001251220703125	\\
0.229724330803036	0.001068115234375	\\
0.229768721978071	0.000640869140625	\\
0.229813113153105	0.0001220703125	\\
0.22985750432814	-0.00018310546875	\\
0.229901895503174	-0.0008544921875	\\
0.229946286678208	-0.00115966796875	\\
0.229990677853243	-0.001251220703125	\\
0.230035069028277	-0.00152587890625	\\
0.230079460203312	-0.002105712890625	\\
0.230123851378346	-0.002197265625	\\
0.23016824255338	-0.00213623046875	\\
0.230212633728415	-0.002655029296875	\\
0.230257024903449	-0.0023193359375	\\
0.230301416078484	-0.002044677734375	\\
0.230345807253518	-0.0025634765625	\\
0.230390198428552	-0.002349853515625	\\
0.230434589603587	-0.00164794921875	\\
0.230478980778621	-0.000946044921875	\\
0.230523371953656	0.000213623046875	\\
0.23056776312869	0.001617431640625	\\
0.230612154303724	0.002227783203125	\\
0.230656545478759	0.002532958984375	\\
0.230700936653793	0.002655029296875	\\
0.230745327828828	0.00274658203125	\\
0.230789719003862	0.0028076171875	\\
0.230834110178896	0.0025634765625	\\
0.230878501353931	0.0020751953125	\\
0.230922892528965	0.0018310546875	\\
0.230967283704	0.001708984375	\\
0.231011674879034	0.001953125	\\
0.231056066054068	0.00189208984375	\\
0.231100457229103	0.001861572265625	\\
0.231144848404137	0.00213623046875	\\
0.231189239579172	0.001983642578125	\\
0.231233630754206	0.001922607421875	\\
0.23127802192924	0.002105712890625	\\
0.231322413104275	0.00201416015625	\\
0.231366804279309	0.001068115234375	\\
0.231411195454344	-9.1552734375e-05	\\
0.231455586629378	-0.000732421875	\\
0.231499977804412	-0.001708984375	\\
0.231544368979447	-0.00238037109375	\\
0.231588760154481	-0.001922607421875	\\
0.231633151329516	-0.00189208984375	\\
0.23167754250455	-0.001953125	\\
0.231721933679584	-0.00152587890625	\\
0.231766324854619	-0.000946044921875	\\
0.231810716029653	-0.000518798828125	\\
0.231855107204688	0.00079345703125	\\
0.231899498379722	0.0010986328125	\\
0.231943889554757	0.001373291015625	\\
0.231988280729791	0.00152587890625	\\
0.232032671904825	0.001495361328125	\\
0.23207706307986	0.001708984375	\\
0.232121454254894	0.001861572265625	\\
0.232165845429929	0.002166748046875	\\
0.232210236604963	0.002044677734375	\\
0.232254627779997	0.002349853515625	\\
0.232299018955032	0.002716064453125	\\
0.232343410130066	0.003875732421875	\\
0.232387801305101	0.00433349609375	\\
0.232432192480135	0.00482177734375	\\
0.232476583655169	0.00604248046875	\\
0.232520974830204	0.0062255859375	\\
0.232565366005238	0.0059814453125	\\
0.232609757180273	0.0062255859375	\\
0.232654148355307	0.006317138671875	\\
0.232698539530341	0.00543212890625	\\
0.232742930705376	0.00439453125	\\
0.23278732188041	0.00341796875	\\
0.232831713055445	0.002349853515625	\\
0.232876104230479	0.00128173828125	\\
0.232920495405513	0.000640869140625	\\
0.232964886580548	0.000823974609375	\\
0.233009277755582	0.000274658203125	\\
0.233053668930617	-0.0003662109375	\\
0.233098060105651	-0.0001220703125	\\
0.233142451280685	-0.000152587890625	\\
0.23318684245572	-0.00030517578125	\\
0.233231233630754	-0.00079345703125	\\
0.233275624805789	-0.00128173828125	\\
0.233320015980823	-0.001495361328125	\\
0.233364407155857	-0.0015869140625	\\
0.233408798330892	-0.002044677734375	\\
0.233453189505926	-0.0020751953125	\\
0.233497580680961	-0.001861572265625	\\
0.233541971855995	-0.00201416015625	\\
0.233586363031029	-0.0013427734375	\\
0.233630754206064	-0.000335693359375	\\
0.233675145381098	0.0008544921875	\\
0.233719536556133	0.001678466796875	\\
0.233763927731167	0.0018310546875	\\
0.233808318906201	0.00286865234375	\\
0.233852710081236	0.00335693359375	\\
0.23389710125627	0.0032958984375	\\
0.233941492431305	0.003326416015625	\\
0.233985883606339	0.002777099609375	\\
0.234030274781373	0.00262451171875	\\
0.234074665956408	0.00201416015625	\\
0.234119057131442	0.00128173828125	\\
0.234163448306477	0.001434326171875	\\
0.234207839481511	0.001312255859375	\\
0.234252230656545	0.0013427734375	\\
0.23429662183158	0.001861572265625	\\
0.234341013006614	0.001983642578125	\\
0.234385404181649	0.001800537109375	\\
0.234429795356683	0.001953125	\\
0.234474186531718	0.00128173828125	\\
0.234518577706752	0.0006103515625	\\
0.234562968881786	6.103515625e-05	\\
0.234607360056821	-0.000885009765625	\\
0.234651751231855	-0.001861572265625	\\
0.23469614240689	-0.00311279296875	\\
0.234740533581924	-0.0037841796875	\\
0.234784924756958	-0.0042724609375	\\
0.234829315931993	-0.003997802734375	\\
0.234873707107027	-0.003387451171875	\\
0.234918098282062	-0.00347900390625	\\
0.234962489457096	-0.002899169921875	\\
0.23500688063213	-0.001953125	\\
0.235051271807165	-0.00128173828125	\\
0.235095662982199	-0.00091552734375	\\
0.235140054157234	-0.000823974609375	\\
0.235184445332268	-0.0006103515625	\\
0.235228836507302	-0.00042724609375	\\
0.235273227682337	-0.000335693359375	\\
0.235317618857371	-0.000396728515625	\\
0.235362010032406	-0.000152587890625	\\
0.23540640120744	0.000640869140625	\\
0.235450792382474	0.001129150390625	\\
0.235495183557509	0.0018310546875	\\
0.235539574732543	0.002685546875	\\
0.235583965907578	0.003509521484375	\\
0.235628357082612	0.0040283203125	\\
0.235672748257646	0.0047607421875	\\
0.235717139432681	0.0048828125	\\
0.235761530607715	0.0048828125	\\
0.23580592178275	0.004638671875	\\
0.235850312957784	0.0037841796875	\\
0.235894704132818	0.0035400390625	\\
0.235939095307853	0.0035400390625	\\
0.235983486482887	0.0030517578125	\\
0.236027877657922	0.002532958984375	\\
0.236072268832956	0.0025634765625	\\
0.23611666000799	0.00189208984375	\\
0.236161051183025	0.001434326171875	\\
0.236205442358059	0.00164794921875	\\
0.236249833533094	0.001739501953125	\\
0.236294224708128	0.001220703125	\\
0.236338615883162	0.00115966796875	\\
0.236383007058197	0.001068115234375	\\
0.236427398233231	0.000396728515625	\\
0.236471789408266	0.000335693359375	\\
0.2365161805833	0.000213623046875	\\
0.236560571758334	-0.000579833984375	\\
0.236604962933369	-0.0006103515625	\\
0.236649354108403	-9.1552734375e-05	\\
0.236693745283438	-0.00018310546875	\\
0.236738136458472	-0.000152587890625	\\
0.236782527633506	0.0001220703125	\\
0.236826918808541	0.00079345703125	\\
0.236871309983575	0.000823974609375	\\
0.23691570115861	0.001007080078125	\\
0.236960092333644	0.0010986328125	\\
0.237004483508678	0.001007080078125	\\
0.237048874683713	0.000946044921875	\\
0.237093265858747	0.001068115234375	\\
0.237137657033782	0.00115966796875	\\
0.237182048208816	0.001708984375	\\
0.23722643938385	0.001190185546875	\\
0.237270830558885	0.001068115234375	\\
0.237315221733919	0.001190185546875	\\
0.237359612908954	0.00079345703125	\\
0.237404004083988	0.00054931640625	\\
0.237448395259023	0.000640869140625	\\
0.237492786434057	0.000213623046875	\\
0.237537177609091	-0.000335693359375	\\
0.237581568784126	-0.000457763671875	\\
0.23762595995916	-0.000396728515625	\\
0.237670351134195	-0.00042724609375	\\
0.237714742309229	-0.00067138671875	\\
0.237759133484263	-0.00146484375	\\
0.237803524659298	-0.001556396484375	\\
0.237847915834332	-0.001495361328125	\\
0.237892307009367	-0.002166748046875	\\
0.237936698184401	-0.002349853515625	\\
0.237981089359435	-0.001983642578125	\\
0.23802548053447	-0.00225830078125	\\
0.238069871709504	-0.00213623046875	\\
0.238114262884539	-0.001220703125	\\
0.238158654059573	-0.000640869140625	\\
0.238203045234607	-0.0001220703125	\\
0.238247436409642	0.00030517578125	\\
0.238291827584676	0.001007080078125	\\
0.238336218759711	0.002197265625	\\
0.238380609934745	0.00262451171875	\\
0.238425001109779	0.0028076171875	\\
0.238469392284814	0.003326416015625	\\
0.238513783459848	0.003143310546875	\\
0.238558174634883	0.002685546875	\\
0.238602565809917	0.001953125	\\
0.238646956984951	0.001953125	\\
0.238691348159986	0.00164794921875	\\
0.23873573933502	0.001373291015625	\\
0.238780130510055	0.00201416015625	\\
0.238824521685089	0.00213623046875	\\
0.238868912860123	0.00244140625	\\
0.238913304035158	0.002838134765625	\\
0.238957695210192	0.0025634765625	\\
0.239002086385227	0.002197265625	\\
0.239046477560261	0.002166748046875	\\
0.239090868735295	0.00146484375	\\
0.23913525991033	0.00054931640625	\\
0.239179651085364	-0.000701904296875	\\
0.239224042260399	-0.00146484375	\\
0.239268433435433	-0.00152587890625	\\
0.239312824610467	-0.00146484375	\\
0.239357215785502	-0.001739501953125	\\
0.239401606960536	-0.001800537109375	\\
0.239445998135571	-0.00177001953125	\\
0.239490389310605	-0.001617431640625	\\
0.239534780485639	-0.000732421875	\\
0.239579171660674	9.1552734375e-05	\\
0.239623562835708	0.00042724609375	\\
0.239667954010743	0.000762939453125	\\
0.239712345185777	0.001190185546875	\\
0.239756736360811	0.00103759765625	\\
0.239801127535846	0.000885009765625	\\
0.23984551871088	0.000457763671875	\\
0.239889909885915	0.000518798828125	\\
0.239934301060949	0.0010986328125	\\
0.239978692235983	0.001373291015625	\\
0.240023083411018	0.002197265625	\\
0.240067474586052	0.0025634765625	\\
0.240111865761087	0.00299072265625	\\
0.240156256936121	0.003753662109375	\\
0.240200648111155	0.004364013671875	\\
0.24024503928619	0.00506591796875	\\
0.240289430461224	0.00537109375	\\
0.240333821636259	0.00543212890625	\\
0.240378212811293	0.0052490234375	\\
0.240422603986328	0.004364013671875	\\
0.240466995161362	0.003509521484375	\\
0.240511386336396	0.002716064453125	\\
0.240555777511431	0.0018310546875	\\
0.240600168686465	0.000885009765625	\\
0.2406445598615	0.000457763671875	\\
0.240688951036534	0.000274658203125	\\
0.240733342211568	-0.000244140625	\\
0.240777733386603	-0.000396728515625	\\
0.240822124561637	-0.00042724609375	\\
0.240866515736672	-0.000579833984375	\\
0.240910906911706	-0.00067138671875	\\
0.24095529808674	-0.000946044921875	\\
0.240999689261775	-0.001678466796875	\\
0.241044080436809	-0.001861572265625	\\
0.241088471611844	-0.002227783203125	\\
0.241132862786878	-0.00299072265625	\\
0.241177253961912	-0.002899169921875	\\
0.241221645136947	-0.002227783203125	\\
0.241266036311981	-0.00201416015625	\\
0.241310427487016	-0.0015869140625	\\
0.24135481866205	-0.00079345703125	\\
0.241399209837084	-0.00042724609375	\\
0.241443601012119	0.000274658203125	\\
0.241487992187153	0.000762939453125	\\
0.241532383362188	0.000946044921875	\\
0.241576774537222	0.001434326171875	\\
0.241621165712256	0.001708984375	\\
0.241665556887291	0.001861572265625	\\
0.241709948062325	0.00177001953125	\\
0.24175433923736	0.00177001953125	\\
0.241798730412394	0.00152587890625	\\
0.241843121587428	0.001556396484375	\\
0.241887512762463	0.001708984375	\\
0.241931903937497	0.00164794921875	\\
0.241976295112532	0.001556396484375	\\
0.242020686287566	0.001800537109375	\\
0.2420650774626	0.002105712890625	\\
0.242109468637635	0.0020751953125	\\
0.242153859812669	0.002105712890625	\\
0.242198250987704	0.0018310546875	\\
0.242242642162738	0.00146484375	\\
0.242287033337772	0.00079345703125	\\
0.242331424512807	6.103515625e-05	\\
0.242375815687841	-0.000732421875	\\
0.242420206862876	-0.00164794921875	\\
0.24246459803791	-0.00299072265625	\\
0.242508989212944	-0.00341796875	\\
0.242553380387979	-0.003570556640625	\\
0.242597771563013	-0.0040283203125	\\
0.242642162738048	-0.003753662109375	\\
0.242686553913082	-0.00323486328125	\\
0.242730945088116	-0.00323486328125	\\
0.242775336263151	-0.003021240234375	\\
0.242819727438185	-0.002716064453125	\\
0.24286411861322	-0.002166748046875	\\
0.242908509788254	-0.00152587890625	\\
0.242952900963289	-0.001678466796875	\\
0.242997292138323	-0.0018310546875	\\
0.243041683313357	-0.001800537109375	\\
0.243086074488392	-0.001373291015625	\\
0.243130465663426	-0.00115966796875	\\
0.243174856838461	-0.001007080078125	\\
0.243219248013495	-9.1552734375e-05	\\
0.243263639188529	0.000885009765625	\\
0.243308030363564	0.001739501953125	\\
0.243352421538598	0.002777099609375	\\
0.243396812713633	0.00347900390625	\\
0.243441203888667	0.00372314453125	\\
0.243485595063701	0.003997802734375	\\
0.243529986238736	0.004150390625	\\
0.24357437741377	0.00360107421875	\\
0.243618768588805	0.0030517578125	\\
0.243663159763839	0.002197265625	\\
0.243707550938873	0.00164794921875	\\
0.243751942113908	0.001312255859375	\\
0.243796333288942	0.00115966796875	\\
0.243840724463977	0.001220703125	\\
0.243885115639011	0.00128173828125	\\
0.243929506814045	0.001251220703125	\\
0.24397389798908	0.001220703125	\\
0.244018289164114	0.00189208984375	\\
0.244062680339149	0.0013427734375	\\
0.244107071514183	0.000823974609375	\\
0.244151462689217	0.000640869140625	\\
0.244195853864252	-6.103515625e-05	\\
0.244240245039286	-0.0008544921875	\\
0.244284636214321	-0.001068115234375	\\
0.244329027389355	-0.001220703125	\\
0.244373418564389	-0.001251220703125	\\
0.244417809739424	-0.001129150390625	\\
0.244462200914458	-0.00146484375	\\
0.244506592089493	-0.00115966796875	\\
0.244550983264527	-0.000396728515625	\\
0.244595374439561	-0.00048828125	\\
0.244639765614596	-9.1552734375e-05	\\
0.24468415678963	0.0001220703125	\\
0.244728547964665	-0.0001220703125	\\
0.244772939139699	0.00048828125	\\
0.244817330314733	0.000732421875	\\
0.244861721489768	0.001129150390625	\\
0.244906112664802	0.000885009765625	\\
0.244950503839837	0.00030517578125	\\
0.244994895014871	0.00030517578125	\\
0.245039286189905	0.000274658203125	\\
0.24508367736494	0.0008544921875	\\
0.245128068539974	0.000885009765625	\\
0.245172459715009	0.0009765625	\\
0.245216850890043	0.000885009765625	\\
0.245261242065077	0.0003662109375	\\
0.245305633240112	0.00079345703125	\\
0.245350024415146	0.00103759765625	\\
0.245394415590181	0.0001220703125	\\
0.245438806765215	0.000152587890625	\\
0.245483197940249	0.000457763671875	\\
0.245527589115284	-3.0517578125e-05	\\
0.245571980290318	-0.00042724609375	\\
0.245616371465353	-0.001373291015625	\\
0.245660762640387	-0.001556396484375	\\
0.245705153815421	-0.001678466796875	\\
0.245749544990456	-0.002532958984375	\\
0.24579393616549	-0.00225830078125	\\
0.245838327340525	-0.00201416015625	\\
0.245882718515559	-0.00177001953125	\\
0.245927109690594	-0.001220703125	\\
0.245971500865628	-0.0006103515625	\\
0.246015892040662	0.000457763671875	\\
0.246060283215697	0.001007080078125	\\
0.246104674390731	0.000823974609375	\\
0.246149065565766	0.0013427734375	\\
0.2461934567408	0.002288818359375	\\
0.246237847915834	0.0028076171875	\\
0.246282239090869	0.0028076171875	\\
0.246326630265903	0.00274658203125	\\
0.246371021440938	0.0020751953125	\\
0.246415412615972	0.00140380859375	\\
0.246459803791006	0.001617431640625	\\
0.246504194966041	0.001434326171875	\\
0.246548586141075	0.00164794921875	\\
0.24659297731611	0.002044677734375	\\
0.246637368491144	0.001922607421875	\\
0.246681759666178	0.002227783203125	\\
0.246726150841213	0.002410888671875	\\
0.246770542016247	0.0018310546875	\\
0.246814933191282	0.00115966796875	\\
0.246859324366316	0.0009765625	\\
0.24690371554135	0.00048828125	\\
0.246948106716385	0.00030517578125	\\
0.246992497891419	-0.000518798828125	\\
0.247036889066454	-0.001434326171875	\\
0.247081280241488	-0.00177001953125	\\
0.247125671416522	-0.0018310546875	\\
0.247170062591557	-0.001678466796875	\\
0.247214453766591	-0.00164794921875	\\
0.247258844941626	-0.001708984375	\\
0.24730323611666	-0.00103759765625	\\
0.247347627291694	0	\\
0.247392018466729	0.0003662109375	\\
0.247436409641763	0.000732421875	\\
0.247480800816798	0.001220703125	\\
0.247525191991832	0.00067138671875	\\
0.247569583166866	0.000762939453125	\\
0.247613974341901	0.00103759765625	\\
0.247658365516935	0.001190185546875	\\
0.24770275669197	0.00146484375	\\
0.247747147867004	0.00140380859375	\\
0.247791539042038	0.001678466796875	\\
0.247835930217073	0.002044677734375	\\
0.247880321392107	0.002685546875	\\
0.247924712567142	0.003570556640625	\\
0.247969103742176	0.004150390625	\\
0.24801349491721	0.004547119140625	\\
0.248057886092245	0.00537109375	\\
0.248102277267279	0.005889892578125	\\
0.248146668442314	0.00592041015625	\\
0.248191059617348	0.005859375	\\
0.248235450792382	0.005035400390625	\\
0.248279841967417	0.004058837890625	\\
0.248324233142451	0.003326416015625	\\
0.248368624317486	0.002593994140625	\\
0.24841301549252	0.002197265625	\\
0.248457406667555	0.0015869140625	\\
0.248501797842589	0.00140380859375	\\
0.248546189017623	0.001556396484375	\\
0.248590580192658	0.00128173828125	\\
0.248634971367692	0.0006103515625	\\
0.248679362542727	0.000518798828125	\\
0.248723753717761	0.000244140625	\\
0.248768144892795	-0.0003662109375	\\
0.24881253606783	-0.00042724609375	\\
0.248856927242864	-0.00048828125	\\
0.248901318417899	-0.000518798828125	\\
0.248945709592933	-0.001251220703125	\\
0.248990100767967	-0.00152587890625	\\
0.249034491943002	-0.00164794921875	\\
0.249078883118036	-0.001312255859375	\\
0.249123274293071	-0.0009765625	\\
0.249167665468105	-0.000823974609375	\\
0.249212056643139	-0.000274658203125	\\
0.249256447818174	3.0517578125e-05	\\
0.249300838993208	0.0006103515625	\\
0.249345230168243	0.00177001953125	\\
0.249389621343277	0.002899169921875	\\
0.249434012518311	0.003021240234375	\\
0.249478403693346	0.003173828125	\\
0.24952279486838	0.0025634765625	\\
0.249567186043415	0.00244140625	\\
0.249611577218449	0.00299072265625	\\
0.249655968393483	0.00286865234375	\\
0.249700359568518	0.003326416015625	\\
0.249744750743552	0.003021240234375	\\
0.249789141918587	0.002532958984375	\\
0.249833533093621	0.002593994140625	\\
0.249877924268655	0.002960205078125	\\
0.24992231544369	0.003021240234375	\\
0.249966706618724	0.00299072265625	\\
0.250011097793759	0.00286865234375	\\
0.250055488968793	0.002471923828125	\\
0.250099880143827	0.002532958984375	\\
0.250144271318862	0.00128173828125	\\
0.250188662493896	6.103515625e-05	\\
0.250233053668931	-9.1552734375e-05	\\
0.250277444843965	-0.00079345703125	\\
0.250321836018999	-0.001739501953125	\\
0.250366227194034	-0.00177001953125	\\
0.250410618369068	-0.001556396484375	\\
0.250455009544103	-0.001251220703125	\\
0.250499400719137	-0.001312255859375	\\
0.250543791894171	-0.0006103515625	\\
0.250588183069206	-0.0003662109375	\\
0.25063257424424	-0.000885009765625	\\
0.250676965419275	-0.0010986328125	\\
0.250721356594309	-0.001312255859375	\\
0.250765747769343	-0.0010986328125	\\
0.250810138944378	-0.00140380859375	\\
0.250854530119412	-0.001373291015625	\\
0.250898921294447	-0.00054931640625	\\
0.250943312469481	-0.000335693359375	\\
0.250987703644515	3.0517578125e-05	\\
0.25103209481955	0.000885009765625	\\
0.251076485994584	0.00103759765625	\\
0.251120877169619	0.00140380859375	\\
0.251165268344653	0.001708984375	\\
0.251209659519687	0.002410888671875	\\
0.251254050694722	0.0025634765625	\\
0.251298441869756	0.002838134765625	\\
0.251342833044791	0.003082275390625	\\
0.251387224219825	0.003143310546875	\\
0.251431615394859	0.00286865234375	\\
0.251476006569894	0.00213623046875	\\
0.251520397744928	0.002166748046875	\\
0.251564788919963	0.002288818359375	\\
0.251609180094997	0.002410888671875	\\
0.251653571270032	0.002716064453125	\\
0.251697962445066	0.0029296875	\\
0.2517423536201	0.003448486328125	\\
0.251786744795135	0.00286865234375	\\
0.251831135970169	0.0023193359375	\\
0.251875527145204	0.00274658203125	\\
0.251919918320238	0.00250244140625	\\
0.251964309495272	0.002166748046875	\\
0.252008700670307	0.001678466796875	\\
0.252053091845341	0.0010986328125	\\
0.252097483020376	0.000518798828125	\\
0.25214187419541	0.000396728515625	\\
0.252186265370444	0.000946044921875	\\
0.252230656545479	0.00054931640625	\\
0.252275047720513	0.00018310546875	\\
0.252319438895548	0.000732421875	\\
0.252363830070582	0.000732421875	\\
0.252408221245616	0.000946044921875	\\
0.252452612420651	0.000885009765625	\\
0.252497003595685	0.000885009765625	\\
0.25254139477072	0.001129150390625	\\
0.252585785945754	0.00103759765625	\\
0.252630177120788	0.001129150390625	\\
0.252674568295823	0.0010986328125	\\
0.252718959470857	0.0010986328125	\\
0.252763350645892	0.001251220703125	\\
0.252807741820926	0.001495361328125	\\
0.25285213299596	0.00164794921875	\\
0.252896524170995	0.00164794921875	\\
0.252940915346029	0.001678466796875	\\
0.252985306521064	0.00152587890625	\\
0.253029697696098	0.0015869140625	\\
0.253074088871132	0.001922607421875	\\
0.253118480046167	0.001708984375	\\
0.253162871221201	0.00189208984375	\\
0.253207262396236	0.001708984375	\\
0.25325165357127	0.001800537109375	\\
0.253296044746304	0.00201416015625	\\
0.253340435921339	0.001922607421875	\\
0.253384827096373	0.001129150390625	\\
0.253429218271408	0.00091552734375	\\
0.253473609446442	0.001007080078125	\\
0.253518000621476	0.0009765625	\\
0.253562391796511	0.0013427734375	\\
0.253606782971545	0.00091552734375	\\
0.25365117414658	0.0006103515625	\\
0.253695565321614	0.00103759765625	\\
0.253739956496648	0.00115966796875	\\
0.253784347671683	0.001373291015625	\\
0.253828738846717	0.001495361328125	\\
0.253873130021752	0.002105712890625	\\
0.253917521196786	0.002288818359375	\\
0.25396191237182	0.00244140625	\\
0.254006303546855	0.00323486328125	\\
0.254050694721889	0.003173828125	\\
0.254095085896924	0.003631591796875	\\
0.254139477071958	0.0042724609375	\\
0.254183868246993	0.004180908203125	\\
0.254228259422027	0.004791259765625	\\
0.254272650597061	0.004486083984375	\\
0.254317041772096	0.003753662109375	\\
0.25436143294713	0.003509521484375	\\
0.254405824122165	0.003082275390625	\\
0.254450215297199	0.00311279296875	\\
0.254494606472233	0.003021240234375	\\
0.254538997647268	0.002899169921875	\\
0.254583388822302	0.00323486328125	\\
0.254627779997337	0.0028076171875	\\
0.254672171172371	0.00299072265625	\\
0.254716562347405	0.00323486328125	\\
0.25476095352244	0.003082275390625	\\
0.254805344697474	0.002349853515625	\\
0.254849735872509	0.00213623046875	\\
0.254894127047543	0.001922607421875	\\
0.254938518222577	0.0009765625	\\
0.254982909397612	0.0006103515625	\\
0.255027300572646	0.0001220703125	\\
0.255071691747681	0.000274658203125	\\
0.255116082922715	0.00042724609375	\\
0.255160474097749	0.00018310546875	\\
0.255204865272784	0.000457763671875	\\
0.255249256447818	0.000946044921875	\\
0.255293647622853	0.00140380859375	\\
0.255338038797887	0.001739501953125	\\
0.255382429972921	0.00225830078125	\\
0.255426821147956	0.0020751953125	\\
0.25547121232299	0.001922607421875	\\
0.255515603498025	0.0018310546875	\\
0.255559994673059	0.0018310546875	\\
0.255604385848093	0.00177001953125	\\
0.255648777023128	0.00201416015625	\\
0.255693168198162	0.00238037109375	\\
0.255737559373197	0.003265380859375	\\
0.255781950548231	0.00390625	\\
0.255826341723265	0.003997802734375	\\
0.2558707328983	0.004730224609375	\\
0.255915124073334	0.005340576171875	\\
0.255959515248369	0.005706787109375	\\
0.256003906423403	0.005889892578125	\\
0.256048297598437	0.006134033203125	\\
0.256092688773472	0.00634765625	\\
0.256137079948506	0.005767822265625	\\
0.256181471123541	0.005218505859375	\\
0.256225862298575	0.0047607421875	\\
0.256270253473609	0.00384521484375	\\
0.256314644648644	0.003173828125	\\
0.256359035823678	0.002593994140625	\\
0.256403426998713	0.00225830078125	\\
0.256447818173747	0.00244140625	\\
0.256492209348781	0.002044677734375	\\
0.256536600523816	0.001800537109375	\\
0.25658099169885	0.00152587890625	\\
0.256625382873885	0.001068115234375	\\
0.256669774048919	0.00079345703125	\\
0.256714165223953	0.00067138671875	\\
0.256758556398988	0.00030517578125	\\
0.256802947574022	-0.0001220703125	\\
0.256847338749057	-0.000396728515625	\\
0.256891729924091	-0.0006103515625	\\
0.256936121099126	-0.0006103515625	\\
0.25698051227416	-0.0009765625	\\
0.257024903449194	-0.0009765625	\\
0.257069294624229	-0.000579833984375	\\
0.257113685799263	-6.103515625e-05	\\
0.257158076974297	0.000579833984375	\\
0.257202468149332	0.001007080078125	\\
0.257246859324366	0.0018310546875	\\
0.257291250499401	0.001922607421875	\\
0.257335641674435	0.002288818359375	\\
0.25738003284947	0.00286865234375	\\
0.257424424024504	0.00311279296875	\\
0.257468815199538	0.0029296875	\\
0.257513206374573	0.002716064453125	\\
0.257557597549607	0.00299072265625	\\
0.257601988724642	0.00323486328125	\\
0.257646379899676	0.0029296875	\\
0.25769077107471	0.003173828125	\\
0.257735162249745	0.003570556640625	\\
0.257779553424779	0.003631591796875	\\
0.257823944599814	0.003509521484375	\\
0.257868335774848	0.00390625	\\
0.257912726949882	0.00433349609375	\\
0.257957118124917	0.00396728515625	\\
0.258001509299951	0.003143310546875	\\
0.258045900474986	0.0025634765625	\\
0.25809029165002	0.00164794921875	\\
0.258134682825054	0.00115966796875	\\
0.258179074000089	0.000701904296875	\\
0.258223465175123	-0.000274658203125	\\
0.258267856350158	-0.00042724609375	\\
0.258312247525192	-0.000244140625	\\
0.258356638700226	-0.00054931640625	\\
0.258401029875261	-0.000701904296875	\\
0.258445421050295	-0.000457763671875	\\
0.25848981222533	-0.00054931640625	\\
0.258534203400364	-0.00079345703125	\\
0.258578594575398	-0.000701904296875	\\
0.258622985750433	-0.00067138671875	\\
0.258667376925467	-0.0008544921875	\\
0.258711768100502	-0.00115966796875	\\
0.258756159275536	-0.0010986328125	\\
0.25880055045057	-0.000762939453125	\\
0.258844941625605	-0.00103759765625	\\
0.258889332800639	-0.001129150390625	\\
0.258933723975674	-0.00091552734375	\\
0.258978115150708	-0.0006103515625	\\
0.259022506325742	0.000244140625	\\
0.259066897500777	0.000701904296875	\\
0.259111288675811	0.00103759765625	\\
0.259155679850846	0.001800537109375	\\
0.25920007102588	0.0020751953125	\\
0.259244462200914	0.00213623046875	\\
0.259288853375949	0.00225830078125	\\
0.259333244550983	0.001861572265625	\\
0.259377635726018	0.00177001953125	\\
0.259422026901052	0.00152587890625	\\
0.259466418076086	0.001312255859375	\\
0.259510809251121	0.001434326171875	\\
0.259555200426155	0.00189208984375	\\
0.25959959160119	0.00201416015625	\\
0.259643982776224	0.0023193359375	\\
0.259688373951258	0.00250244140625	\\
0.259732765126293	0.0020751953125	\\
0.259777156301327	0.0020751953125	\\
0.259821547476362	0.00213623046875	\\
0.259865938651396	0.002166748046875	\\
0.259910329826431	0.001800537109375	\\
0.259954721001465	0.000885009765625	\\
0.259999112176499	0.000701904296875	\\
0.260043503351534	0.000732421875	\\
0.260087894526568	0.00042724609375	\\
0.260132285701603	0.00030517578125	\\
0.260176676876637	0.00018310546875	\\
0.260221068051671	0.0001220703125	\\
0.260265459226706	0.00054931640625	\\
0.26030985040174	0.000762939453125	\\
0.260354241576775	0.00079345703125	\\
0.260398632751809	0.000885009765625	\\
0.260443023926843	0.000946044921875	\\
0.260487415101878	0.0008544921875	\\
0.260531806276912	0.00067138671875	\\
0.260576197451947	0.00042724609375	\\
0.260620588626981	0.00054931640625	\\
0.260664979802015	0.000274658203125	\\
0.26070937097705	0	\\
0.260753762152084	-0.00018310546875	\\
0.260798153327119	-9.1552734375e-05	\\
0.260842544502153	3.0517578125e-05	\\
0.260886935677187	-6.103515625e-05	\\
0.260931326852222	-0.00018310546875	\\
0.260975718027256	-0.00042724609375	\\
0.261020109202291	-0.00018310546875	\\
0.261064500377325	6.103515625e-05	\\
0.261108891552359	-0.000579833984375	\\
0.261153282727394	-0.00103759765625	\\
0.261197673902428	-0.000732421875	\\
0.261242065077463	-0.0008544921875	\\
0.261286456252497	-0.001220703125	\\
0.261330847427531	-0.000701904296875	\\
0.261375238602566	-0.000823974609375	\\
0.2614196297776	-0.001373291015625	\\
0.261464020952635	-0.0010986328125	\\
0.261508412127669	-0.000946044921875	\\
0.261552803302703	-0.00091552734375	\\
0.261597194477738	-0.00054931640625	\\
0.261641585652772	-0.0003662109375	\\
0.261685976827807	0.000244140625	\\
0.261730368002841	0.000640869140625	\\
0.261774759177875	0.000152587890625	\\
0.26181915035291	0.001129150390625	\\
0.261863541527944	0.001678466796875	\\
0.261907932702979	0.001678466796875	\\
0.261952323878013	0.00225830078125	\\
0.261996715053047	0.0020751953125	\\
0.262041106228082	0.002105712890625	\\
0.262085497403116	0.0023193359375	\\
0.262129888578151	0.002197265625	\\
0.262174279753185	0.001708984375	\\
0.262218670928219	0.0015869140625	\\
0.262263062103254	0.0010986328125	\\
0.262307453278288	0.000701904296875	\\
0.262351844453323	0.001068115234375	\\
0.262396235628357	0.00115966796875	\\
0.262440626803391	0.00146484375	\\
0.262485017978426	0.001739501953125	\\
0.26252940915346	0.00189208984375	\\
0.262573800328495	0.001434326171875	\\
0.262618191503529	0.000701904296875	\\
0.262662582678564	0.000335693359375	\\
0.262706973853598	-0.00030517578125	\\
0.262751365028632	-0.00030517578125	\\
0.262795756203667	-0.000732421875	\\
0.262840147378701	-0.0009765625	\\
0.262884538553736	-0.00164794921875	\\
0.26292892972877	-0.002471923828125	\\
0.262973320903804	-0.002349853515625	\\
0.263017712078839	-0.00250244140625	\\
0.263062103253873	-0.002471923828125	\\
0.263106494428908	-0.001983642578125	\\
0.263150885603942	-0.00177001953125	\\
0.263195276778976	-0.001678466796875	\\
0.263239667954011	-0.00146484375	\\
0.263284059129045	-0.00128173828125	\\
0.26332845030408	-0.00128173828125	\\
0.263372841479114	-0.000762939453125	\\
0.263417232654148	-0.00091552734375	\\
0.263461623829183	-0.000946044921875	\\
0.263506015004217	-0.00042724609375	\\
0.263550406179252	-0.000244140625	\\
0.263594797354286	9.1552734375e-05	\\
0.26363918852932	0.000518798828125	\\
0.263683579704355	0.0008544921875	\\
0.263727970879389	0.001007080078125	\\
0.263772362054424	0.0015869140625	\\
0.263816753229458	0.00201416015625	\\
0.263861144404492	0.00262451171875	\\
0.263905535579527	0.003143310546875	\\
0.263949926754561	0.0032958984375	\\
0.263994317929596	0.003326416015625	\\
0.26403870910463	0.003204345703125	\\
0.264083100279664	0.00244140625	\\
0.264127491454699	0.0015869140625	\\
0.264171882629733	0.00146484375	\\
0.264216273804768	0.001007080078125	\\
0.264260664979802	0.000762939453125	\\
0.264305056154836	0.000335693359375	\\
0.264349447329871	-0.000518798828125	\\
0.264393838504905	-0.000762939453125	\\
0.26443822967994	-0.000396728515625	\\
0.264482620854974	-0.000762939453125	\\
0.264527012030008	-0.001373291015625	\\
0.264571403205043	-0.0015869140625	\\
0.264615794380077	-0.001953125	\\
0.264660185555112	-0.00213623046875	\\
0.264704576730146	-0.002349853515625	\\
0.26474896790518	-0.002655029296875	\\
0.264793359080215	-0.00311279296875	\\
0.264837750255249	-0.00341796875	\\
0.264882141430284	-0.003204345703125	\\
0.264926532605318	-0.0032958984375	\\
0.264970923780352	-0.003326416015625	\\
0.265015314955387	-0.003143310546875	\\
0.265059706130421	-0.002685546875	\\
0.265104097305456	-0.002349853515625	\\
0.26514848848049	-0.002288818359375	\\
0.265192879655524	-0.0018310546875	\\
0.265237270830559	-0.001129150390625	\\
0.265281662005593	-0.001007080078125	\\
0.265326053180628	-0.00146484375	\\
0.265370444355662	-0.001220703125	\\
0.265414835530697	-0.00067138671875	\\
0.265459226705731	-0.000762939453125	\\
0.265503617880765	-0.001068115234375	\\
0.2655480090558	-0.001190185546875	\\
0.265592400230834	-0.00067138671875	\\
0.265636791405868	-0.000640869140625	\\
0.265681182580903	-0.000732421875	\\
0.265725573755937	-0.000335693359375	\\
0.265769964930972	-0.00018310546875	\\
0.265814356106006	-0.00030517578125	\\
0.265858747281041	9.1552734375e-05	\\
0.265903138456075	0.000244140625	\\
0.265947529631109	-0.0006103515625	\\
0.265991920806144	-0.000762939453125	\\
0.266036311981178	-0.0009765625	\\
0.266080703156213	-0.00146484375	\\
0.266125094331247	-0.001678466796875	\\
0.266169485506281	-0.0020751953125	\\
0.266213876681316	-0.002349853515625	\\
0.26625826785635	-0.002532958984375	\\
0.266302659031385	-0.00250244140625	\\
0.266347050206419	-0.002716064453125	\\
0.266391441381453	-0.002716064453125	\\
0.266435832556488	-0.002532958984375	\\
0.266480223731522	-0.002838134765625	\\
0.266524614906557	-0.002716064453125	\\
0.266569006081591	-0.0025634765625	\\
0.266613397256625	-0.00250244140625	\\
0.26665778843166	-0.0028076171875	\\
0.266702179606694	-0.003082275390625	\\
0.266746570781729	-0.00244140625	\\
0.266790961956763	-0.002288818359375	\\
0.266835353131797	-0.00250244140625	\\
0.266879744306832	-0.0020751953125	\\
0.266924135481866	-0.001800537109375	\\
0.266968526656901	-0.001617431640625	\\
0.267012917831935	-0.0010986328125	\\
0.267057309006969	-0.000762939453125	\\
0.267101700182004	-0.000274658203125	\\
0.267146091357038	-0.0003662109375	\\
0.267190482532073	-0.0006103515625	\\
0.267234873707107	-6.103515625e-05	\\
0.267279264882141	-0.0001220703125	\\
0.267323656057176	-3.0517578125e-05	\\
0.26736804723221	0.000274658203125	\\
0.267412438407245	-3.0517578125e-05	\\
0.267456829582279	-0.000152587890625	\\
0.267501220757313	0.000244140625	\\
0.267545611932348	0.00030517578125	\\
0.267590003107382	0.00030517578125	\\
0.267634394282417	0.001251220703125	\\
0.267678785457451	0.00128173828125	\\
0.267723176632485	0.0008544921875	\\
0.26776756780752	0.001251220703125	\\
0.267811958982554	0.00128173828125	\\
0.267856350157589	0.000885009765625	\\
0.267900741332623	0.0009765625	\\
0.267945132507658	0.0008544921875	\\
0.267989523682692	0.00091552734375	\\
0.268033914857726	0.00030517578125	\\
0.268078306032761	0.00018310546875	\\
0.268122697207795	0.00048828125	\\
0.268167088382829	3.0517578125e-05	\\
0.268211479557864	-0.000274658203125	\\
0.268255870732898	-0.000274658203125	\\
0.268300261907933	0.000274658203125	\\
0.268344653082967	6.103515625e-05	\\
0.268389044258002	-0.000335693359375	\\
0.268433435433036	0.0001220703125	\\
0.26847782660807	0.000213623046875	\\
0.268522217783105	-0.000274658203125	\\
0.268566608958139	-0.00048828125	\\
0.268611000133174	-0.0009765625	\\
0.268655391308208	-0.0009765625	\\
0.268699782483242	-0.0010986328125	\\
0.268744173658277	-0.001373291015625	\\
0.268788564833311	-0.00164794921875	\\
0.268832956008346	-0.00146484375	\\
0.26887734718338	-0.0013427734375	\\
0.268921738358414	-0.00152587890625	\\
0.268966129533449	-0.0015869140625	\\
0.269010520708483	-0.001251220703125	\\
0.269054911883518	-0.001312255859375	\\
0.269099303058552	-0.001708984375	\\
0.269143694233586	-0.00164794921875	\\
0.269188085408621	-0.001129150390625	\\
0.269232476583655	-0.00128173828125	\\
0.26927686775869	-0.001495361328125	\\
0.269321258933724	-0.00146484375	\\
0.269365650108758	-0.001617431640625	\\
0.269410041283793	-0.001708984375	\\
0.269454432458827	-0.00152587890625	\\
0.269498823633862	-0.00152587890625	\\
0.269543214808896	-0.001556396484375	\\
0.26958760598393	-0.001129150390625	\\
0.269631997158965	-0.000579833984375	\\
0.269676388333999	-3.0517578125e-05	\\
0.269720779509034	0.000640869140625	\\
0.269765170684068	0.00146484375	\\
0.269809561859102	0.0015869140625	\\
0.269853953034137	0.001220703125	\\
0.269898344209171	0.00201416015625	\\
0.269942735384206	0.00238037109375	\\
0.26998712655924	0.002044677734375	\\
0.270031517734274	0.0025634765625	\\
0.270075908909309	0.00213623046875	\\
0.270120300084343	0.001861572265625	\\
0.270164691259378	0.001800537109375	\\
0.270209082434412	0.001068115234375	\\
0.270253473609446	0.001312255859375	\\
0.270297864784481	0.00128173828125	\\
0.270342255959515	0.001312255859375	\\
0.27038664713455	0.001190185546875	\\
0.270431038309584	0.00103759765625	\\
0.270475429484618	0.00103759765625	\\
0.270519820659653	0.000396728515625	\\
0.270564211834687	0.000335693359375	\\
0.270608603009722	0.000244140625	\\
0.270652994184756	-0.000579833984375	\\
0.27069738535979	-0.0009765625	\\
0.270741776534825	-0.0010986328125	\\
0.270786167709859	-0.0009765625	\\
0.270830558884894	-0.001007080078125	\\
0.270874950059928	-0.001251220703125	\\
0.270919341234962	-0.00146484375	\\
0.270963732409997	-0.001495361328125	\\
0.271008123585031	-0.0010986328125	\\
0.271052514760066	-0.001251220703125	\\
0.2710969059351	-0.000701904296875	\\
0.271141297110135	6.103515625e-05	\\
0.271185688285169	0.000396728515625	\\
0.271230079460203	0.0006103515625	\\
0.271274470635238	0.00115966796875	\\
0.271318861810272	0.0018310546875	\\
0.271363252985307	0.001678466796875	\\
0.271407644160341	0.00201416015625	\\
0.271452035335375	0.002349853515625	\\
0.27149642651041	0.0020751953125	\\
0.271540817685444	0.00250244140625	\\
0.271585208860479	0.00311279296875	\\
0.271629600035513	0.003326416015625	\\
0.271673991210547	0.00384521484375	\\
0.271718382385582	0.004638671875	\\
0.271762773560616	0.0050048828125	\\
0.271807164735651	0.00537109375	\\
0.271851555910685	0.00592041015625	\\
0.271895947085719	0.0059814453125	\\
0.271940338260754	0.006072998046875	\\
0.271984729435788	0.005615234375	\\
0.272029120610823	0.00518798828125	\\
0.272073511785857	0.004730224609375	\\
0.272117902960891	0.004486083984375	\\
0.272162294135926	0.00396728515625	\\
0.27220668531096	0.00360107421875	\\
0.272251076485995	0.0032958984375	\\
0.272295467661029	0.002960205078125	\\
0.272339858836063	0.002471923828125	\\
0.272384250011098	0.00238037109375	\\
0.272428641186132	0.002105712890625	\\
0.272473032361167	0.00238037109375	\\
0.272517423536201	0.002349853515625	\\
0.272561814711235	0.001922607421875	\\
0.27260620588627	0.00164794921875	\\
0.272650597061304	0.001129150390625	\\
0.272694988236339	0.00054931640625	\\
0.272739379411373	0.00030517578125	\\
0.272783770586407	0.000396728515625	\\
0.272828161761442	0.00030517578125	\\
0.272872552936476	-0.00018310546875	\\
0.272916944111511	-9.1552734375e-05	\\
0.272961335286545	0.000244140625	\\
0.273005726461579	0.00042724609375	\\
0.273050117636614	0.000457763671875	\\
0.273094508811648	0.00079345703125	\\
0.273138899986683	0.001068115234375	\\
0.273183291161717	0.00115966796875	\\
0.273227682336751	0.0015869140625	\\
0.273272073511786	0.00164794921875	\\
0.27331646468682	0.001800537109375	\\
0.273360855861855	0.00177001953125	\\
0.273405247036889	0.002044677734375	\\
0.273449638211923	0.002655029296875	\\
0.273494029386958	0.002532958984375	\\
0.273538420561992	0.001953125	\\
0.273582811737027	0.0025634765625	\\
0.273627202912061	0.0028076171875	\\
0.273671594087095	0.002960205078125	\\
0.27371598526213	0.003448486328125	\\
0.273760376437164	0.003387451171875	\\
0.273804767612199	0.00311279296875	\\
0.273849158787233	0.002532958984375	\\
0.273893549962268	0.0023193359375	\\
0.273937941137302	0.002197265625	\\
0.273982332312336	0.001861572265625	\\
0.274026723487371	0.001800537109375	\\
0.274071114662405	0.001678466796875	\\
0.27411550583744	0.00164794921875	\\
0.274159897012474	0.00152587890625	\\
0.274204288187508	0.001739501953125	\\
0.274248679362543	0.001617431640625	\\
0.274293070537577	0.001800537109375	\\
0.274337461712612	0.0020751953125	\\
0.274381852887646	0.001861572265625	\\
0.27442624406268	0.001708984375	\\
0.274470635237715	0.001708984375	\\
0.274515026412749	0.00140380859375	\\
0.274559417587784	0.0010986328125	\\
0.274603808762818	0.0009765625	\\
0.274648199937852	0.000885009765625	\\
0.274692591112887	0.000701904296875	\\
0.274736982287921	0.0009765625	\\
0.274781373462956	0.00103759765625	\\
0.27482576463799	0.000885009765625	\\
0.274870155813024	0.001220703125	\\
0.274914546988059	0.001312255859375	\\
0.274958938163093	0.0020751953125	\\
0.275003329338128	0.00225830078125	\\
0.275047720513162	0.002197265625	\\
0.275092111688196	0.0028076171875	\\
0.275136502863231	0.00262451171875	\\
0.275180894038265	0.002410888671875	\\
0.2752252852133	0.0028076171875	\\
0.275269676388334	0.0029296875	\\
0.275314067563368	0.002838134765625	\\
0.275358458738403	0.003265380859375	\\
0.275402849913437	0.002960205078125	\\
0.275447241088472	0.0030517578125	\\
0.275491632263506	0.00408935546875	\\
0.27553602343854	0.00421142578125	\\
0.275580414613575	0.004058837890625	\\
0.275624805788609	0.004425048828125	\\
0.275669196963644	0.00439453125	\\
0.275713588138678	0.00439453125	\\
0.275757979313712	0.0045166015625	\\
0.275802370488747	0.004547119140625	\\
0.275846761663781	0.004150390625	\\
0.275891152838816	0.0042724609375	\\
0.27593554401385	0.0042724609375	\\
0.275979935188884	0.00390625	\\
0.276024326363919	0.0040283203125	\\
0.276068717538953	0.00421142578125	\\
0.276113108713988	0.003814697265625	\\
0.276157499889022	0.003448486328125	\\
0.276201891064056	0.00347900390625	\\
0.276246282239091	0.0032958984375	\\
0.276290673414125	0.00372314453125	\\
0.27633506458916	0.003326416015625	\\
0.276379455764194	0.0030517578125	\\
0.276423846939229	0.00286865234375	\\
0.276468238114263	0.00244140625	\\
0.276512629289297	0.0023193359375	\\
0.276557020464332	0.001983642578125	\\
0.276601411639366	0.001800537109375	\\
0.2766458028144	0.001190185546875	\\
0.276690193989435	0.000885009765625	\\
0.276734585164469	0.001068115234375	\\
0.276778976339504	0.001068115234375	\\
0.276823367514538	0.00152587890625	\\
0.276867758689573	0.00091552734375	\\
0.276912149864607	0.0008544921875	\\
0.276956541039641	0.00079345703125	\\
0.277000932214676	0.000457763671875	\\
0.27704532338971	0.00091552734375	\\
0.277089714564745	0.00079345703125	\\
0.277134105739779	0.000579833984375	\\
0.277178496914813	0.000762939453125	\\
0.277222888089848	0.0009765625	\\
0.277267279264882	0.00140380859375	\\
0.277311670439917	0.0010986328125	\\
0.277356061614951	0.00091552734375	\\
0.277400452789985	0.001495361328125	\\
0.27744484396502	0.001556396484375	\\
0.277489235140054	0.001739501953125	\\
0.277533626315089	0.00225830078125	\\
0.277578017490123	0.00244140625	\\
0.277622408665157	0.00250244140625	\\
0.277666799840192	0.00286865234375	\\
0.277711191015226	0.00286865234375	\\
0.277755582190261	0.00360107421875	\\
0.277799973365295	0.00360107421875	\\
0.277844364540329	0.00384521484375	\\
0.277888755715364	0.00421142578125	\\
0.277933146890398	0.00408935546875	\\
0.277977538065433	0.00445556640625	\\
0.278021929240467	0.004241943359375	\\
0.278066320415501	0.0042724609375	\\
0.278110711590536	0.00408935546875	\\
0.27815510276557	0.00433349609375	\\
0.278199493940605	0.0037841796875	\\
0.278243885115639	0.00311279296875	\\
0.278288276290673	0.003173828125	\\
0.278332667465708	0.003387451171875	\\
0.278377058640742	0.003509521484375	\\
0.278421449815777	0.0030517578125	\\
0.278465840990811	0.0028076171875	\\
0.278510232165845	0.00299072265625	\\
0.27855462334088	0.003143310546875	\\
0.278599014515914	0.002349853515625	\\
0.278643405690949	0.002166748046875	\\
0.278687796865983	0.0023193359375	\\
0.278732188041017	0.001953125	\\
0.278776579216052	0.001312255859375	\\
0.278820970391086	0.001007080078125	\\
0.278865361566121	0.0013427734375	\\
0.278909752741155	0.001190185546875	\\
0.278954143916189	0.00091552734375	\\
0.278998535091224	0.0010986328125	\\
0.279042926266258	0.001220703125	\\
0.279087317441293	0.001556396484375	\\
0.279131708616327	0.001373291015625	\\
0.279176099791361	0.00152587890625	\\
0.279220490966396	0.00201416015625	\\
0.27926488214143	0.002105712890625	\\
0.279309273316465	0.00201416015625	\\
0.279353664491499	0.001800537109375	\\
0.279398055666533	0.002227783203125	\\
0.279442446841568	0.00262451171875	\\
0.279486838016602	0.00244140625	\\
0.279531229191637	0.002685546875	\\
0.279575620366671	0.0028076171875	\\
0.279620011541706	0.002532958984375	\\
0.27966440271674	0.00299072265625	\\
0.279708793891774	0.003509521484375	\\
0.279753185066809	0.00384521484375	\\
0.279797576241843	0.00469970703125	\\
0.279841967416878	0.005035400390625	\\
0.279886358591912	0.004852294921875	\\
0.279930749766946	0.004669189453125	\\
0.279975140941981	0.00445556640625	\\
0.280019532117015	0.004364013671875	\\
0.28006392329205	0.003875732421875	\\
0.280108314467084	0.003326416015625	\\
0.280152705642118	0.003265380859375	\\
0.280197096817153	0.0030517578125	\\
0.280241487992187	0.002288818359375	\\
0.280285879167222	0.002410888671875	\\
0.280330270342256	0.002349853515625	\\
0.28037466151729	0.001800537109375	\\
0.280419052692325	0.001983642578125	\\
0.280463443867359	0.0025634765625	\\
0.280507835042394	0.002197265625	\\
0.280552226217428	0.001617431640625	\\
0.280596617392462	0.001556396484375	\\
0.280641008567497	0.00115966796875	\\
0.280685399742531	0.00067138671875	\\
0.280729790917566	0.000244140625	\\
0.2807741820926	9.1552734375e-05	\\
0.280818573267634	-0.000274658203125	\\
0.280862964442669	-0.000762939453125	\\
0.280907355617703	-0.000640869140625	\\
0.280951746792738	-0.000518798828125	\\
0.280996137967772	-0.000244140625	\\
0.281040529142806	-3.0517578125e-05	\\
0.281084920317841	0.00018310546875	\\
0.281129311492875	0.000152587890625	\\
0.28117370266791	0.0003662109375	\\
0.281218093842944	0.0009765625	\\
0.281262485017978	0.000823974609375	\\
0.281306876193013	0.000885009765625	\\
0.281351267368047	0.000762939453125	\\
0.281395658543082	0.0006103515625	\\
0.281440049718116	0.000762939453125	\\
0.28148444089315	0.00091552734375	\\
0.281528832068185	0.001220703125	\\
0.281573223243219	0.00128173828125	\\
0.281617614418254	0.0013427734375	\\
0.281662005593288	0.0018310546875	\\
0.281706396768322	0.002197265625	\\
0.281750787943357	0.0020751953125	\\
0.281795179118391	0.001922607421875	\\
0.281839570293426	0.002166748046875	\\
0.28188396146846	0.00225830078125	\\
0.281928352643494	0.00177001953125	\\
0.281972743818529	0.0020751953125	\\
0.282017134993563	0.00177001953125	\\
0.282061526168598	0.001129150390625	\\
0.282105917343632	0.00103759765625	\\
0.282150308518667	0.000579833984375	\\
0.282194699693701	0.000579833984375	\\
0.282239090868735	0.000579833984375	\\
0.28228348204377	0.000579833984375	\\
0.282327873218804	0.00042724609375	\\
0.282372264393839	0.00018310546875	\\
0.282416655568873	0.000213623046875	\\
0.282461046743907	0.0006103515625	\\
0.282505437918942	0.000762939453125	\\
0.282549829093976	-0.0001220703125	\\
0.282594220269011	-0.000579833984375	\\
0.282638611444045	-0.0006103515625	\\
0.282683002619079	-0.000640869140625	\\
0.282727393794114	-0.000701904296875	\\
0.282771784969148	-0.0006103515625	\\
0.282816176144183	-0.0009765625	\\
0.282860567319217	-0.001190185546875	\\
0.282904958494251	-0.0009765625	\\
0.282949349669286	-0.00115966796875	\\
0.28299374084432	-0.000885009765625	\\
0.283038132019355	-0.000457763671875	\\
0.283082523194389	-0.000640869140625	\\
0.283126914369423	-0.00054931640625	\\
0.283171305544458	0	\\
0.283215696719492	-3.0517578125e-05	\\
0.283260087894527	-0.000152587890625	\\
0.283304479069561	-3.0517578125e-05	\\
0.283348870244595	0.000152587890625	\\
0.28339326141963	0.000244140625	\\
0.283437652594664	-0.000274658203125	\\
0.283482043769699	-0.000152587890625	\\
0.283526434944733	9.1552734375e-05	\\
0.283570826119767	0.00042724609375	\\
0.283615217294802	0.000885009765625	\\
0.283659608469836	0.001068115234375	\\
0.283703999644871	0.001251220703125	\\
0.283748390819905	0.000579833984375	\\
0.283792781994939	0.000518798828125	\\
0.283837173169974	0.001068115234375	\\
0.283881564345008	0.000946044921875	\\
0.283925955520043	0.0009765625	\\
0.283970346695077	0.000885009765625	\\
0.284014737870111	0.00067138671875	\\
0.284059129045146	0.00079345703125	\\
0.28410352022018	0.000732421875	\\
0.284147911395215	0.00067138671875	\\
0.284192302570249	0.000762939453125	\\
0.284236693745283	0.000640869140625	\\
0.284281084920318	0.0003662109375	\\
0.284325476095352	0.000396728515625	\\
0.284369867270387	0.00048828125	\\
0.284414258445421	6.103515625e-05	\\
0.284458649620455	0	\\
0.28450304079549	0.0001220703125	\\
0.284547431970524	3.0517578125e-05	\\
0.284591823145559	-0.00048828125	\\
0.284636214320593	-0.001068115234375	\\
0.284680605495627	-0.001190185546875	\\
0.284724996670662	-0.001190185546875	\\
0.284769387845696	-0.0010986328125	\\
0.284813779020731	-0.001190185546875	\\
0.284858170195765	-0.001190185546875	\\
0.2849025613708	-0.0013427734375	\\
0.284946952545834	-0.00140380859375	\\
0.284991343720868	-0.00146484375	\\
0.285035734895903	-0.00164794921875	\\
0.285080126070937	-0.002166748046875	\\
0.285124517245971	-0.002410888671875	\\
0.285168908421006	-0.002532958984375	\\
0.28521329959604	-0.002655029296875	\\
0.285257690771075	-0.00262451171875	\\
0.285302081946109	-0.003204345703125	\\
0.285346473121144	-0.002777099609375	\\
0.285390864296178	-0.00262451171875	\\
0.285435255471212	-0.002899169921875	\\
0.285479646646247	-0.00244140625	\\
0.285524037821281	-0.002410888671875	\\
0.285568428996316	-0.001983642578125	\\
0.28561282017135	-0.00146484375	\\
0.285657211346384	-0.0008544921875	\\
0.285701602521419	-0.001007080078125	\\
0.285745993696453	-0.0010986328125	\\
0.285790384871488	-0.000732421875	\\
0.285834776046522	-0.0003662109375	\\
0.285879167221556	9.1552734375e-05	\\
0.285923558396591	0.000244140625	\\
0.285967949571625	0.00048828125	\\
0.28601234074666	0.00042724609375	\\
0.286056731921694	0.000335693359375	\\
0.286101123096728	0.00018310546875	\\
0.286145514271763	-0.00018310546875	\\
0.286189905446797	-0.0001220703125	\\
0.286234296621832	0.00018310546875	\\
0.286278687796866	0.00067138671875	\\
0.2863230789719	0.001007080078125	\\
0.286367470146935	0.000518798828125	\\
0.286411861321969	0.00042724609375	\\
0.286456252497004	0.00042724609375	\\
0.286500643672038	9.1552734375e-05	\\
0.286545034847072	-0.000457763671875	\\
0.286589426022107	-0.000946044921875	\\
0.286633817197141	-0.001068115234375	\\
0.286678208372176	-0.001495361328125	\\
0.28672259954721	-0.00164794921875	\\
0.286766990722244	-0.001983642578125	\\
0.286811381897279	-0.001983642578125	\\
0.286855773072313	-0.00152587890625	\\
0.286900164247348	-0.001800537109375	\\
0.286944555422382	-0.001953125	\\
0.286988946597416	-0.00225830078125	\\
0.287033337772451	-0.002105712890625	\\
0.287077728947485	-0.001739501953125	\\
0.28712212012252	-0.001251220703125	\\
0.287166511297554	-0.0015869140625	\\
0.287210902472588	-0.001708984375	\\
0.287255293647623	-0.00189208984375	\\
0.287299684822657	-0.0015869140625	\\
0.287344075997692	-0.0018310546875	\\
0.287388467172726	-0.001556396484375	\\
0.28743285834776	-0.00079345703125	\\
0.287477249522795	-0.0010986328125	\\
0.287521640697829	-0.000640869140625	\\
0.287566031872864	-0.000244140625	\\
0.287610423047898	0.00048828125	\\
0.287654814222932	0.000823974609375	\\
0.287699205397967	0.001220703125	\\
0.287743596573001	0.00201416015625	\\
0.287787987748036	0.002532958984375	\\
0.28783237892307	0.002716064453125	\\
0.287876770098104	0.002960205078125	\\
0.287921161273139	0.003082275390625	\\
0.287965552448173	0.0025634765625	\\
0.288009943623208	0.0023193359375	\\
0.288054334798242	0.001983642578125	\\
0.288098725973277	0.001617431640625	\\
0.288143117148311	0.001373291015625	\\
0.288187508323345	0.000701904296875	\\
0.28823189949838	0.000732421875	\\
0.288276290673414	0.001068115234375	\\
0.288320681848449	0.00079345703125	\\
0.288365073023483	0.00067138671875	\\
0.288409464198517	0.000762939453125	\\
0.288453855373552	0.00067138671875	\\
0.288498246548586	0.00042724609375	\\
0.288542637723621	-0.0001220703125	\\
0.288587028898655	-0.00030517578125	\\
0.288631420073689	-0.00115966796875	\\
0.288675811248724	-0.00225830078125	\\
0.288720202423758	-0.002899169921875	\\
0.288764593598793	-0.00360107421875	\\
0.288808984773827	-0.003997802734375	\\
0.288853375948861	-0.004058837890625	\\
0.288897767123896	-0.0042724609375	\\
0.28894215829893	-0.004364013671875	\\
0.288986549473965	-0.00390625	\\
0.289030940648999	-0.003662109375	\\
0.289075331824033	-0.003570556640625	\\
0.289119722999068	-0.003326416015625	\\
0.289164114174102	-0.00347900390625	\\
0.289208505349137	-0.003387451171875	\\
0.289252896524171	-0.003143310546875	\\
0.289297287699205	-0.003814697265625	\\
0.28934167887424	-0.004119873046875	\\
0.289386070049274	-0.003326416015625	\\
0.289430461224309	-0.003662109375	\\
0.289474852399343	-0.00335693359375	\\
0.289519243574377	-0.00335693359375	\\
0.289563634749412	-0.003631591796875	\\
0.289608025924446	-0.00335693359375	\\
0.289652417099481	-0.003326416015625	\\
0.289696808274515	-0.002960205078125	\\
0.289741199449549	-0.002410888671875	\\
0.289785590624584	-0.001861572265625	\\
0.289829981799618	-0.0023193359375	\\
0.289874372974653	-0.00262451171875	\\
0.289918764149687	-0.00238037109375	\\
0.289963155324721	-0.002655029296875	\\
0.290007546499756	-0.003387451171875	\\
0.29005193767479	-0.00335693359375	\\
0.290096328849825	-0.0032958984375	\\
0.290140720024859	-0.00311279296875	\\
0.290185111199893	-0.002838134765625	\\
0.290229502374928	-0.003082275390625	\\
0.290273893549962	-0.0029296875	\\
0.290318284724997	-0.0029296875	\\
0.290362675900031	-0.002838134765625	\\
0.290407067075065	-0.00311279296875	\\
0.2904514582501	-0.003173828125	\\
0.290495849425134	-0.003204345703125	\\
0.290540240600169	-0.003265380859375	\\
0.290584631775203	-0.0029296875	\\
0.290629022950238	-0.00311279296875	\\
0.290673414125272	-0.003173828125	\\
0.290717805300306	-0.002655029296875	\\
0.290762196475341	-0.0023193359375	\\
0.290806587650375	-0.002685546875	\\
0.29085097882541	-0.002716064453125	\\
0.290895370000444	-0.002227783203125	\\
0.290939761175478	-0.001495361328125	\\
0.290984152350513	-0.000823974609375	\\
0.291028543525547	-0.000732421875	\\
0.291072934700582	-0.000213623046875	\\
0.291117325875616	-0.00054931640625	\\
0.29116171705065	-0.0009765625	\\
0.291206108225685	-0.000823974609375	\\
0.291250499400719	-0.0008544921875	\\
0.291294890575754	-0.000579833984375	\\
0.291339281750788	-0.000518798828125	\\
0.291383672925822	-0.000457763671875	\\
0.291428064100857	-6.103515625e-05	\\
0.291472455275891	-0.000274658203125	\\
0.291516846450926	-0.000457763671875	\\
0.29156123762596	-9.1552734375e-05	\\
0.291605628800994	-0.000579833984375	\\
0.291650019976029	-0.000640869140625	\\
0.291694411151063	-0.00018310546875	\\
0.291738802326098	-0.000518798828125	\\
0.291783193501132	-0.00048828125	\\
0.291827584676166	-0.000396728515625	\\
0.291871975851201	-0.00042724609375	\\
0.291916367026235	-0.000152587890625	\\
0.29196075820127	0.00042724609375	\\
0.292005149376304	0.000396728515625	\\
0.292049540551338	0.00048828125	\\
0.292093931726373	0.000457763671875	\\
0.292138322901407	0.000457763671875	\\
0.292182714076442	0.000640869140625	\\
0.292227105251476	0.000701904296875	\\
0.29227149642651	0.0010986328125	\\
0.292315887601545	0.001678466796875	\\
0.292360278776579	0.001373291015625	\\
0.292404669951614	0.001373291015625	\\
0.292449061126648	0.002105712890625	\\
0.292493452301682	0.001953125	\\
0.292537843476717	0.00146484375	\\
0.292582234651751	0.001220703125	\\
0.292626625826786	0.0003662109375	\\
0.29267101700182	-0.00042724609375	\\
0.292715408176854	-0.00140380859375	\\
0.292759799351889	-0.002532958984375	\\
0.292804190526923	-0.003143310546875	\\
0.292848581701958	-0.0032958984375	\\
0.292892972876992	-0.00335693359375	\\
0.292937364052026	-0.003631591796875	\\
0.292981755227061	-0.00341796875	\\
0.293026146402095	-0.002899169921875	\\
0.29307053757713	-0.00311279296875	\\
0.293114928752164	-0.003570556640625	\\
0.293159319927198	-0.003570556640625	\\
0.293203711102233	-0.00384521484375	\\
0.293248102277267	-0.003936767578125	\\
0.293292493452302	-0.004150390625	\\
0.293336884627336	-0.004302978515625	\\
0.293381275802371	-0.0037841796875	\\
0.293425666977405	-0.003387451171875	\\
0.293470058152439	-0.002655029296875	\\
0.293514449327474	-0.001678466796875	\\
0.293558840502508	-0.000732421875	\\
0.293603231677542	-0.000335693359375	\\
0.293647622852577	-0.0001220703125	\\
0.293692014027611	0.000885009765625	\\
0.293736405202646	0.00115966796875	\\
0.29378079637768	0.001007080078125	\\
0.293825187552715	0.00152587890625	\\
0.293869578727749	0.00152587890625	\\
0.293913969902783	0.00164794921875	\\
0.293958361077818	0.002288818359375	\\
0.294002752252852	0.002716064453125	\\
0.294047143427887	0.0030517578125	\\
0.294091534602921	0.00341796875	\\
0.294135925777955	0.003692626953125	\\
0.29418031695299	0.00360107421875	\\
0.294224708128024	0.00335693359375	\\
0.294269099303059	0.0035400390625	\\
0.294313490478093	0.00311279296875	\\
0.294357881653127	0.002777099609375	\\
0.294402272828162	0.00286865234375	\\
0.294446664003196	0.002685546875	\\
0.294491055178231	0.002685546875	\\
0.294535446353265	0.002471923828125	\\
0.294579837528299	0.001800537109375	\\
0.294624228703334	0.0015869140625	\\
0.294668619878368	0.001678466796875	\\
0.294713011053403	0.00140380859375	\\
0.294757402228437	0.001068115234375	\\
0.294801793403471	0.001007080078125	\\
0.294846184578506	0.000701904296875	\\
0.29489057575354	0.000457763671875	\\
0.294934966928575	0.0009765625	\\
0.294979358103609	0.000732421875	\\
0.295023749278643	0.000518798828125	\\
0.295068140453678	0.001068115234375	\\
0.295112531628712	0.001312255859375	\\
0.295156922803747	0.00177001953125	\\
0.295201313978781	0.002349853515625	\\
0.295245705153815	0.003082275390625	\\
0.29529009632885	0.003692626953125	\\
0.295334487503884	0.0032958984375	\\
0.295378878678919	0.00323486328125	\\
0.295423269853953	0.002899169921875	\\
0.295467661028987	0.0025634765625	\\
0.295512052204022	0.002777099609375	\\
0.295556443379056	0.002838134765625	\\
0.295600834554091	0.002899169921875	\\
0.295645225729125	0.003021240234375	\\
0.295689616904159	0.0030517578125	\\
0.295734008079194	0.002960205078125	\\
0.295778399254228	0.002532958984375	\\
0.295822790429263	0.00225830078125	\\
0.295867181604297	0.001678466796875	\\
0.295911572779331	0.001617431640625	\\
0.295955963954366	0.00213623046875	\\
0.2960003551294	0.001953125	\\
0.296044746304435	0.00189208984375	\\
0.296089137479469	0.002105712890625	\\
0.296133528654503	0.002044677734375	\\
0.296177919829538	0.001708984375	\\
0.296222311004572	0.001708984375	\\
0.296266702179607	0.001556396484375	\\
0.296311093354641	0.0015869140625	\\
0.296355484529676	0.002197265625	\\
0.29639987570471	0.0023193359375	\\
0.296444266879744	0.002899169921875	\\
0.296488658054779	0.003509521484375	\\
0.296533049229813	0.003448486328125	\\
0.296577440404848	0.003448486328125	\\
0.296621831579882	0.003509521484375	\\
0.296666222754916	0.002838134765625	\\
0.296710613929951	0.002685546875	\\
0.296755005104985	0.002716064453125	\\
0.29679939628002	0.002227783203125	\\
0.296843787455054	0.00164794921875	\\
0.296888178630088	0.00140380859375	\\
0.296932569805123	0.001312255859375	\\
0.296976960980157	0.001220703125	\\
0.297021352155192	0.001190185546875	\\
0.297065743330226	0.001312255859375	\\
0.29711013450526	0.001007080078125	\\
0.297154525680295	0.0013427734375	\\
0.297198916855329	0.00128173828125	\\
0.297243308030364	0.000732421875	\\
0.297287699205398	0.0001220703125	\\
0.297332090380432	-0.000946044921875	\\
0.297376481555467	-0.001312255859375	\\
0.297420872730501	-0.0009765625	\\
0.297465263905536	-0.001312255859375	\\
0.29750965508057	-0.001983642578125	\\
0.297554046255604	-0.001251220703125	\\
0.297598437430639	-0.00103759765625	\\
0.297642828605673	-0.000640869140625	\\
0.297687219780708	-0.000274658203125	\\
0.297731610955742	-0.0006103515625	\\
0.297776002130776	0.000213623046875	\\
0.297820393305811	0.000823974609375	\\
0.297864784480845	0.001007080078125	\\
0.29790917565588	0.0013427734375	\\
0.297953566830914	0.002288818359375	\\
0.297997958005948	0.002197265625	\\
0.298042349180983	0.0025634765625	\\
0.298086740356017	0.002777099609375	\\
0.298131131531052	0.002716064453125	\\
0.298175522706086	0.002899169921875	\\
0.29821991388112	0.003204345703125	\\
0.298264305056155	0.003570556640625	\\
0.298308696231189	0.0035400390625	\\
0.298353087406224	0.00384521484375	\\
0.298397478581258	0.004119873046875	\\
0.298441869756292	0.003631591796875	\\
0.298486260931327	0.003173828125	\\
0.298530652106361	0.003021240234375	\\
0.298575043281396	0.0028076171875	\\
0.29861943445643	0.0023193359375	\\
0.298663825631464	0.001678466796875	\\
0.298708216806499	0.0013427734375	\\
0.298752607981533	0.0010986328125	\\
0.298796999156568	0.0006103515625	\\
0.298841390331602	0.000396728515625	\\
0.298885781506636	0.0003662109375	\\
0.298930172681671	0.000244140625	\\
0.298974563856705	0.00048828125	\\
0.29901895503174	0.0003662109375	\\
0.299063346206774	0.000335693359375	\\
0.299107737381809	0.00054931640625	\\
0.299152128556843	0.000518798828125	\\
0.299196519731877	0.000762939453125	\\
0.299240910906912	0.001007080078125	\\
0.299285302081946	0.00140380859375	\\
0.299329693256981	0.00128173828125	\\
0.299374084432015	0.001220703125	\\
0.299418475607049	0.001678466796875	\\
0.299462866782084	0.00238037109375	\\
0.299507257957118	0.002960205078125	\\
0.299551649132153	0.0030517578125	\\
0.299596040307187	0.003265380859375	\\
0.299640431482221	0.003387451171875	\\
0.299684822657256	0.003173828125	\\
0.29972921383229	0.003082275390625	\\
0.299773605007325	0.003143310546875	\\
0.299817996182359	0.00311279296875	\\
0.299862387357393	0.0028076171875	\\
0.299906778532428	0.0028076171875	\\
0.299951169707462	0.002288818359375	\\
0.299995560882497	0.001861572265625	\\
0.300039952057531	0.001708984375	\\
0.300084343232565	0.001007080078125	\\
0.3001287344076	0.000885009765625	\\
0.300173125582634	0.00079345703125	\\
0.300217516757669	0.00091552734375	\\
0.300261907932703	0.001190185546875	\\
0.300306299107737	0.001312255859375	\\
0.300350690282772	0.001190185546875	\\
0.300395081457806	0.000701904296875	\\
0.300439472632841	6.103515625e-05	\\
0.300483863807875	-0.000152587890625	\\
0.300528254982909	0.000152587890625	\\
0.300572646157944	-3.0517578125e-05	\\
0.300617037332978	-0.0001220703125	\\
0.300661428508013	0.00042724609375	\\
0.300705819683047	0.000244140625	\\
0.300750210858081	0.000152587890625	\\
0.300794602033116	0.00018310546875	\\
0.30083899320815	-0.0001220703125	\\
0.300883384383185	-3.0517578125e-05	\\
0.300927775558219	0	\\
0.300972166733253	3.0517578125e-05	\\
0.301016557908288	0.000213623046875	\\
0.301060949083322	-6.103515625e-05	\\
0.301105340258357	-0.000244140625	\\
0.301149731433391	0	\\
0.301194122608425	0.000244140625	\\
0.30123851378346	0.000213623046875	\\
0.301282904958494	0.000823974609375	\\
0.301327296133529	0.001312255859375	\\
0.301371687308563	0.00091552734375	\\
0.301416078483597	0.0006103515625	\\
0.301460469658632	0.00103759765625	\\
0.301504860833666	0.00079345703125	\\
0.301549252008701	0.000152587890625	\\
0.301593643183735	0.000732421875	\\
0.301638034358769	0.001068115234375	\\
0.301682425533804	0.00091552734375	\\
0.301726816708838	0.0010986328125	\\
0.301771207883873	0.0008544921875	\\
0.301815599058907	0.00140380859375	\\
0.301859990233942	0.0015869140625	\\
0.301904381408976	0.00164794921875	\\
0.30194877258401	0.001953125	\\
0.301993163759045	0.002044677734375	\\
0.302037554934079	0.001861572265625	\\
0.302081946109113	0.00177001953125	\\
0.302126337284148	0.001708984375	\\
0.302170728459182	0.001495361328125	\\
0.302215119634217	0.0020751953125	\\
0.302259510809251	0.002716064453125	\\
0.302303901984286	0.003204345703125	\\
0.30234829315932	0.00372314453125	\\
0.302392684334354	0.00341796875	\\
0.302437075509389	0.0029296875	\\
0.302481466684423	0.00286865234375	\\
0.302525857859458	0.002685546875	\\
0.302570249034492	0.002197265625	\\
0.302614640209526	0.002044677734375	\\
0.302659031384561	0.0020751953125	\\
0.302703422559595	0.002166748046875	\\
0.30274781373463	0.001983642578125	\\
0.302792204909664	0.00177001953125	\\
0.302836596084698	0.00177001953125	\\
0.302880987259733	0.001251220703125	\\
0.302925378434767	0.000396728515625	\\
0.302969769609802	-0.00042724609375	\\
0.303014160784836	-0.00067138671875	\\
0.30305855195987	-0.001068115234375	\\
0.303102943134905	-0.00152587890625	\\
0.303147334309939	-0.00177001953125	\\
0.303191725484974	-0.00164794921875	\\
0.303236116660008	-0.001495361328125	\\
0.303280507835042	-0.0009765625	\\
0.303324899010077	-0.000823974609375	\\
0.303369290185111	-6.103515625e-05	\\
0.303413681360146	0.000244140625	\\
0.30345807253518	6.103515625e-05	\\
0.303502463710214	0	\\
0.303546854885249	0.000274658203125	\\
0.303591246060283	0.00030517578125	\\
0.303635637235318	-0.0001220703125	\\
0.303680028410352	-3.0517578125e-05	\\
0.303724419585386	0.000244140625	\\
0.303768810760421	0.000885009765625	\\
0.303813201935455	0.001312255859375	\\
0.30385759311049	0.001251220703125	\\
0.303901984285524	0.0018310546875	\\
0.303946375460558	0.00177001953125	\\
0.303990766635593	0.001800537109375	\\
0.304035157810627	0.001861572265625	\\
0.304079548985662	0.00128173828125	\\
0.304123940160696	0.0006103515625	\\
0.30416833133573	0.000457763671875	\\
0.304212722510765	0.00054931640625	\\
0.304257113685799	-0.00048828125	\\
0.304301504860834	-0.000335693359375	\\
0.304345896035868	0.000244140625	\\
0.304390287210903	-0.000946044921875	\\
0.304434678385937	-0.00164794921875	\\
0.304479069560971	-0.002044677734375	\\
0.304523460736006	-0.002227783203125	\\
0.30456785191104	-0.001922607421875	\\
0.304612243086074	-0.002197265625	\\
0.304656634261109	-0.002288818359375	\\
0.304701025436143	-0.001983642578125	\\
0.304745416611178	-0.001739501953125	\\
0.304789807786212	-0.001434326171875	\\
0.304834198961247	-0.00146484375	\\
0.304878590136281	-0.00146484375	\\
0.304922981311315	-0.00079345703125	\\
0.30496737248635	-0.000274658203125	\\
0.305011763661384	0.000274658203125	\\
0.305056154836419	0.00067138671875	\\
0.305100546011453	0.0001220703125	\\
0.305144937186487	0.000274658203125	\\
0.305189328361522	0.00030517578125	\\
0.305233719536556	0.000213623046875	\\
0.305278110711591	0.00079345703125	\\
0.305322501886625	0.000823974609375	\\
0.305366893061659	0.0009765625	\\
0.305411284236694	0.0009765625	\\
0.305455675411728	0.000885009765625	\\
0.305500066586763	0.00067138671875	\\
0.305544457761797	0.001190185546875	\\
0.305588848936831	0.001678466796875	\\
0.305633240111866	0.00146484375	\\
0.3056776312869	0.00177001953125	\\
0.305722022461935	0.002166748046875	\\
0.305766413636969	0.001861572265625	\\
0.305810804812003	0.000762939453125	\\
0.305855195987038	0.000213623046875	\\
0.305899587162072	-0.000274658203125	\\
0.305943978337107	-0.001251220703125	\\
0.305988369512141	-0.001251220703125	\\
0.306032760687175	-0.001068115234375	\\
0.30607715186221	-0.001434326171875	\\
0.306121543037244	-0.001861572265625	\\
0.306165934212279	-0.002471923828125	\\
0.306210325387313	-0.00238037109375	\\
0.306254716562347	-0.002197265625	\\
0.306299107737382	-0.00244140625	\\
0.306343498912416	-0.001983642578125	\\
0.306387890087451	-0.00146484375	\\
0.306432281262485	-0.001495361328125	\\
0.306476672437519	-0.0015869140625	\\
0.306521063612554	-0.001556396484375	\\
0.306565454787588	-0.0010986328125	\\
0.306609845962623	-0.00128173828125	\\
0.306654237137657	-0.001190185546875	\\
0.306698628312691	-0.00042724609375	\\
0.306743019487726	0.000457763671875	\\
0.30678741066276	0.001556396484375	\\
0.306831801837795	0.00250244140625	\\
0.306876193012829	0.002471923828125	\\
0.306920584187863	0.001922607421875	\\
0.306964975362898	0.001495361328125	\\
0.307009366537932	0.0010986328125	\\
0.307053757712967	0.000885009765625	\\
0.307098148888001	0.000518798828125	\\
0.307142540063035	-0.000152587890625	\\
0.30718693123807	-0.00091552734375	\\
0.307231322413104	-0.001495361328125	\\
0.307275713588139	-0.00146484375	\\
0.307320104763173	-0.001708984375	\\
0.307364495938207	-0.002593994140625	\\
0.307408887113242	-0.0030517578125	\\
0.307453278288276	-0.003509521484375	\\
0.307497669463311	-0.00341796875	\\
0.307542060638345	-0.00408935546875	\\
0.30758645181338	-0.005157470703125	\\
0.307630842988414	-0.00555419921875	\\
0.307675234163448	-0.00665283203125	\\
0.307719625338483	-0.007354736328125	\\
0.307764016513517	-0.00732421875	\\
0.307808407688552	-0.005828857421875	\\
0.307852798863586	-0.004058837890625	\\
0.30789719003862	-0.003021240234375	\\
0.307941581213655	-0.002105712890625	\\
0.307985972388689	-0.001373291015625	\\
0.308030363563724	-0.00128173828125	\\
0.308074754738758	-0.00164794921875	\\
0.308119145913792	-0.00152587890625	\\
0.308163537088827	-0.000640869140625	\\
0.308207928263861	0.000518798828125	\\
0.308252319438896	0.00128173828125	\\
0.30829671061393	0.001739501953125	\\
0.308341101788964	0.00225830078125	\\
0.308385492963999	0.0029296875	\\
0.308429884139033	0.0040283203125	\\
0.308474275314068	0.004302978515625	\\
0.308518666489102	0.004150390625	\\
0.308563057664136	0.004364013671875	\\
0.308607448839171	0.00372314453125	\\
0.308651840014205	0.003143310546875	\\
0.30869623118924	0.002716064453125	\\
0.308740622364274	0.00128173828125	\\
0.308785013539308	-0.00054931640625	\\
0.308829404714343	-0.001434326171875	\\
0.308873795889377	-0.002044677734375	\\
0.308918187064412	-0.002349853515625	\\
0.308962578239446	-0.00238037109375	\\
0.30900696941448	-0.002899169921875	\\
0.309051360589515	-0.003326416015625	\\
0.309095751764549	-0.00408935546875	\\
0.309140142939584	-0.005340576171875	\\
0.309184534114618	-0.00616455078125	\\
0.309228925289652	-0.006622314453125	\\
0.309273316464687	-0.00775146484375	\\
0.309317707639721	-0.008941650390625	\\
0.309362098814756	-0.008575439453125	\\
0.30940648998979	-0.0076904296875	\\
0.309450881164824	-0.007476806640625	\\
0.309495272339859	-0.006439208984375	\\
0.309539663514893	-0.00537109375	\\
0.309584054689928	-0.00421142578125	\\
0.309628445864962	-0.00347900390625	\\
0.309672837039996	-0.002899169921875	\\
0.309717228215031	-0.00225830078125	\\
0.309761619390065	-0.001953125	\\
0.3098060105651	-0.001434326171875	\\
0.309850401740134	-0.001007080078125	\\
0.309894792915168	9.1552734375e-05	\\
0.309939184090203	0.00103759765625	\\
0.309983575265237	0.002105712890625	\\
0.310027966440272	0.0030517578125	\\
0.310072357615306	0.0032958984375	\\
0.31011674879034	0.00372314453125	\\
0.310161139965375	0.002716064453125	\\
0.310205531140409	0.00177001953125	\\
0.310249922315444	0.00146484375	\\
0.310294313490478	0.00115966796875	\\
0.310338704665513	0.000885009765625	\\
0.310383095840547	0	\\
0.310427487015581	-0.001007080078125	\\
0.310471878190616	-0.002044677734375	\\
0.31051626936565	-0.00323486328125	\\
0.310560660540684	-0.00396728515625	\\
0.310605051715719	-0.004425048828125	\\
0.310649442890753	-0.004547119140625	\\
0.310693834065788	-0.004852294921875	\\
0.310738225240822	-0.004913330078125	\\
0.310782616415857	-0.004425048828125	\\
0.310827007590891	-0.0040283203125	\\
0.310871398765925	-0.00445556640625	\\
0.31091578994096	-0.00537109375	\\
0.310960181115994	-0.0062255859375	\\
0.311004572291029	-0.006683349609375	\\
0.311048963466063	-0.006011962890625	\\
0.311093354641097	-0.00531005859375	\\
0.311137745816132	-0.004180908203125	\\
0.311182136991166	-0.00262451171875	\\
0.311226528166201	-0.00164794921875	\\
0.311270919341235	-0.0003662109375	\\
0.311315310516269	0.00018310546875	\\
0.311359701691304	0.0006103515625	\\
0.311404092866338	0.0008544921875	\\
0.311448484041373	0.000946044921875	\\
0.311492875216407	0.001220703125	\\
0.311537266391441	0.001495361328125	\\
0.311581657566476	0.001861572265625	\\
0.31162604874151	0.002593994140625	\\
0.311670439916545	0.002105712890625	\\
0.311714831091579	0.001678466796875	\\
0.311759222266613	0.00164794921875	\\
0.311803613441648	0.00189208984375	\\
0.311848004616682	0.0018310546875	\\
0.311892395791717	0.001373291015625	\\
0.311936786966751	0.001495361328125	\\
0.311981178141785	0.00152587890625	\\
0.31202556931682	0.0010986328125	\\
0.312069960491854	-0.000274658203125	\\
0.312114351666889	-0.00152587890625	\\
0.312158742841923	-0.003082275390625	\\
0.312203134016957	-0.004241943359375	\\
0.312247525191992	-0.004547119140625	\\
0.312291916367026	-0.004547119140625	\\
0.312336307542061	-0.00445556640625	\\
0.312380698717095	-0.0040283203125	\\
0.312425089892129	-0.003631591796875	\\
0.312469481067164	-0.003265380859375	\\
0.312513872242198	-0.00286865234375	\\
0.312558263417233	-0.00286865234375	\\
0.312602654592267	-0.0023193359375	\\
0.312647045767301	-0.0018310546875	\\
0.312691436942336	-0.001190185546875	\\
0.31273582811737	-0.000396728515625	\\
0.312780219292405	0.0003662109375	\\
0.312824610467439	0.001251220703125	\\
0.312869001642474	0.001312255859375	\\
0.312913392817508	0.001312255859375	\\
0.312957783992542	0.00146484375	\\
0.313002175167577	0.001312255859375	\\
0.313046566342611	0.00189208984375	\\
0.313090957517645	0.002593994140625	\\
0.31313534869268	0.00323486328125	\\
0.313179739867714	0.003387451171875	\\
0.313224131042749	0.002471923828125	\\
0.313268522217783	0.0023193359375	\\
0.313312913392818	0.001434326171875	\\
0.313357304567852	9.1552734375e-05	\\
0.313401695742886	-0.000457763671875	\\
0.313446086917921	-0.000946044921875	\\
0.313490478092955	-0.001251220703125	\\
0.31353486926799	-0.00177001953125	\\
0.313579260443024	-0.002227783203125	\\
0.313623651618058	-0.002532958984375	\\
0.313668042793093	-0.003143310546875	\\
0.313712433968127	-0.0032958984375	\\
0.313756825143162	-0.003387451171875	\\
0.313801216318196	-0.00311279296875	\\
0.31384560749323	-0.003570556640625	\\
0.313889998668265	-0.00445556640625	\\
0.313934389843299	-0.005126953125	\\
0.313978781018334	-0.005126953125	\\
0.314023172193368	-0.004364013671875	\\
0.314067563368402	-0.0029296875	\\
0.314111954543437	-0.002197265625	\\
0.314156345718471	-0.001556396484375	\\
0.314200736893506	-0.00054931640625	\\
0.31424512806854	-6.103515625e-05	\\
0.314289519243574	0.000213623046875	\\
0.314333910418609	0.00042724609375	\\
0.314378301593643	6.103515625e-05	\\
0.314422692768678	-0.000396728515625	\\
0.314467083943712	-0.000152587890625	\\
0.314511475118746	-3.0517578125e-05	\\
0.314555866293781	-0.000274658203125	\\
0.314600257468815	0.000823974609375	\\
0.31464464864385	0.00146484375	\\
0.314689039818884	0.00140380859375	\\
0.314733430993918	0.0009765625	\\
0.314777822168953	0.00048828125	\\
0.314822213343987	-0.0003662109375	\\
0.314866604519022	-0.0013427734375	\\
0.314910995694056	-0.00213623046875	\\
0.31495538686909	-0.003204345703125	\\
0.314999778044125	-0.003204345703125	\\
0.315044169219159	-0.003387451171875	\\
0.315088560394194	-0.003631591796875	\\
0.315132951569228	-0.00384521484375	\\
0.315177342744262	-0.004119873046875	\\
0.315221733919297	-0.0045166015625	\\
0.315266125094331	-0.004913330078125	\\
0.315310516269366	-0.004638671875	\\
0.3153549074444	-0.004302978515625	\\
0.315399298619434	-0.004150390625	\\
0.315443689794469	-0.004150390625	\\
0.315488080969503	-0.004119873046875	\\
0.315532472144538	-0.00384521484375	\\
0.315576863319572	-0.0035400390625	\\
0.315621254494606	-0.003173828125	\\
0.315665645669641	-0.002197265625	\\
0.315710036844675	-0.000823974609375	\\
0.31575442801971	0.000457763671875	\\
0.315798819194744	0.001129150390625	\\
0.315843210369778	0.00103759765625	\\
0.315887601544813	0.0009765625	\\
0.315931992719847	0.0008544921875	\\
0.315976383894882	0.001007080078125	\\
0.316020775069916	0.001617431640625	\\
0.316065166244951	0.001678466796875	\\
0.316109557419985	0.0025634765625	\\
0.316153948595019	0.00396728515625	\\
0.316198339770054	0.0045166015625	\\
0.316242730945088	0.004241943359375	\\
0.316287122120123	0.00384521484375	\\
0.316331513295157	0.003082275390625	\\
0.316375904470191	0.0023193359375	\\
0.316420295645226	0.00164794921875	\\
0.31646468682026	0.00054931640625	\\
0.316509077995295	0.000732421875	\\
0.316553469170329	0.001007080078125	\\
0.316597860345363	0.00054931640625	\\
0.316642251520398	0.000946044921875	\\
0.316686642695432	0.000244140625	\\
0.316731033870467	-0.000244140625	\\
0.316775425045501	-0.000885009765625	\\
0.316819816220535	-0.00262451171875	\\
0.31686420739557	-0.002960205078125	\\
0.316908598570604	-0.0023193359375	\\
0.316952989745639	-0.001861572265625	\\
0.316997380920673	-0.00201416015625	\\
0.317041772095707	-0.002288818359375	\\
0.317086163270742	-0.00189208984375	\\
0.317130554445776	-0.001312255859375	\\
0.317174945620811	-0.000457763671875	\\
0.317219336795845	0.00030517578125	\\
0.317263727970879	3.0517578125e-05	\\
0.317308119145914	-0.00030517578125	\\
0.317352510320948	-0.000946044921875	\\
0.317396901495983	-0.001617431640625	\\
0.317441292671017	-0.00067138671875	\\
0.317485683846051	-0.001068115234375	\\
0.317530075021086	-0.001129150390625	\\
0.31757446619612	-0.000579833984375	\\
0.317618857371155	-0.00042724609375	\\
0.317663248546189	-0.0001220703125	\\
0.317707639721223	-0.00018310546875	\\
0.317752030896258	-3.0517578125e-05	\\
0.317796422071292	-9.1552734375e-05	\\
0.317840813246327	-0.0003662109375	\\
0.317885204421361	-0.00079345703125	\\
0.317929595596395	-0.00042724609375	\\
0.31797398677143	-0.0003662109375	\\
0.318018377946464	-0.000244140625	\\
0.318062769121499	0.00030517578125	\\
0.318107160296533	6.103515625e-05	\\
0.318151551471567	-0.000579833984375	\\
0.318195942646602	-0.0008544921875	\\
0.318240333821636	-0.00115966796875	\\
0.318284724996671	-0.000762939453125	\\
0.318329116171705	0.00030517578125	\\
0.318373507346739	0.000152587890625	\\
0.318417898521774	0.001220703125	\\
0.318462289696808	0.002227783203125	\\
0.318506680871843	0.0015869140625	\\
0.318551072046877	0.001434326171875	\\
0.318595463221911	0.001678466796875	\\
0.318639854396946	0.002349853515625	\\
0.31868424557198	0.0025634765625	\\
0.318728636747015	0.002655029296875	\\
0.318773027922049	0.00347900390625	\\
0.318817419097084	0.004669189453125	\\
0.318861810272118	0.004913330078125	\\
0.318906201447152	0.004425048828125	\\
0.318950592622187	0.00323486328125	\\
0.318994983797221	0.00128173828125	\\
0.319039374972256	-0.000274658203125	\\
0.31908376614729	-0.0008544921875	\\
0.319128157322324	-0.00054931640625	\\
0.319172548497359	-0.0010986328125	\\
0.319216939672393	-0.001495361328125	\\
0.319261330847428	-0.00140380859375	\\
0.319305722022462	-0.00164794921875	\\
0.319350113197496	-0.001617431640625	\\
0.319394504372531	-0.00140380859375	\\
0.319438895547565	-0.00213623046875	\\
0.3194832867226	-0.0028076171875	\\
0.319527677897634	-0.002777099609375	\\
0.319572069072668	-0.003326416015625	\\
0.319616460247703	-0.00341796875	\\
0.319660851422737	-0.003448486328125	\\
0.319705242597772	-0.002960205078125	\\
0.319749633772806	-0.001983642578125	\\
0.31979402494784	-0.00079345703125	\\
0.319838416122875	-0.000152587890625	\\
0.319882807297909	0.000885009765625	\\
0.319927198472944	0.0020751953125	\\
0.319971589647978	0.002349853515625	\\
0.320015980823012	0.0029296875	\\
0.320060371998047	0.0028076171875	\\
0.320104763173081	0.00238037109375	\\
0.320149154348116	0.00323486328125	\\
0.32019354552315	0.0045166015625	\\
0.320237936698184	0.005035400390625	\\
0.320282327873219	0.00579833984375	\\
0.320326719048253	0.00604248046875	\\
0.320371110223288	0.00543212890625	\\
0.320415501398322	0.00543212890625	\\
0.320459892573356	0.00531005859375	\\
0.320504283748391	0.004608154296875	\\
0.320548674923425	0.004180908203125	\\
0.32059306609846	0.00396728515625	\\
0.320637457273494	0.00421142578125	\\
0.320681848448528	0.00469970703125	\\
0.320726239623563	0.0048828125	\\
0.320770630798597	0.004852294921875	\\
0.320815021973632	0.004638671875	\\
0.320859413148666	0.00347900390625	\\
0.3209038043237	0.0023193359375	\\
0.320948195498735	0.001129150390625	\\
0.320992586673769	-6.103515625e-05	\\
0.321036977848804	-0.000396728515625	\\
0.321081369023838	-0.001312255859375	\\
0.321125760198872	-0.00164794921875	\\
0.321170151373907	-0.000823974609375	\\
0.321214542548941	-3.0517578125e-05	\\
0.321258933723976	0.001129150390625	\\
0.32130332489901	0.00140380859375	\\
0.321347716074045	0.0015869140625	\\
0.321392107249079	0.001495361328125	\\
0.321436498424113	0.001007080078125	\\
0.321480889599148	0.0010986328125	\\
0.321525280774182	0.000732421875	\\
0.321569671949216	0.000732421875	\\
0.321614063124251	0.001495361328125	\\
0.321658454299285	0.001800537109375	\\
0.32170284547432	0.0015869140625	\\
0.321747236649354	0.001922607421875	\\
0.321791627824389	0.002471923828125	\\
0.321836018999423	0.002899169921875	\\
0.321880410174457	0.00274658203125	\\
0.321924801349492	0.002349853515625	\\
0.321969192524526	0.001861572265625	\\
0.322013583699561	0.001190185546875	\\
0.322057974874595	0.000701904296875	\\
0.322102366049629	-0.000762939453125	\\
0.322146757224664	-0.001861572265625	\\
0.322191148399698	-0.00262451171875	\\
0.322235539574733	-0.00299072265625	\\
0.322279930749767	-0.00341796875	\\
0.322324321924801	-0.00390625	\\
0.322368713099836	-0.004425048828125	\\
0.32241310427487	-0.004974365234375	\\
0.322457495449905	-0.00469970703125	\\
0.322501886624939	-0.00457763671875	\\
0.322546277799973	-0.00537109375	\\
0.322590668975008	-0.005279541015625	\\
0.322635060150042	-0.00457763671875	\\
0.322679451325077	-0.0040283203125	\\
0.322723842500111	-0.003204345703125	\\
0.322768233675145	-0.00274658203125	\\
0.32281262485018	-0.001953125	\\
0.322857016025214	-0.00103759765625	\\
0.322901407200249	-0.00054931640625	\\
0.322945798375283	-0.0001220703125	\\
0.322990189550317	-0.00018310546875	\\
0.323034580725352	0.000152587890625	\\
0.323078971900386	0.001251220703125	\\
0.323123363075421	0.00213623046875	\\
0.323167754250455	0.00274658203125	\\
0.323212145425489	0.003265380859375	\\
0.323256536600524	0.003448486328125	\\
0.323300927775558	0.00323486328125	\\
0.323345318950593	0.003326416015625	\\
0.323389710125627	0.00384521484375	\\
0.323434101300661	0.002899169921875	\\
0.323478492475696	0.002349853515625	\\
0.32352288365073	0.00250244140625	\\
0.323567274825765	0.002227783203125	\\
0.323611666000799	0.002532958984375	\\
0.323656057175833	0.0018310546875	\\
0.323700448350868	0.00091552734375	\\
0.323744839525902	0.000396728515625	\\
0.323789230700937	-0.0008544921875	\\
0.323833621875971	-0.001800537109375	\\
0.323878013051005	-0.00189208984375	\\
0.32392240422604	-0.002655029296875	\\
0.323966795401074	-0.002593994140625	\\
0.324011186576109	-0.00201416015625	\\
0.324055577751143	-0.001495361328125	\\
0.324099968926177	-0.000579833984375	\\
0.324144360101212	-0.00018310546875	\\
0.324188751276246	-0.000823974609375	\\
0.324233142451281	-0.0015869140625	\\
0.324277533626315	-0.001800537109375	\\
0.324321924801349	-0.0023193359375	\\
0.324366315976384	-0.00250244140625	\\
0.324410707151418	-0.002838134765625	\\
0.324455098326453	-0.003082275390625	\\
0.324499489501487	-0.002716064453125	\\
0.324543880676522	-0.002197265625	\\
0.324588271851556	-0.001312255859375	\\
0.32463266302659	-0.000244140625	\\
0.324677054201625	0.00091552734375	\\
0.324721445376659	0.0015869140625	\\
0.324765836551694	0.002044677734375	\\
0.324810227726728	0.001983642578125	\\
0.324854618901762	0.001678466796875	\\
0.324899010076797	0.0013427734375	\\
0.324943401251831	0.000762939453125	\\
0.324987792426866	0.001251220703125	\\
0.3250321836019	0.001739501953125	\\
0.325076574776934	0.002349853515625	\\
0.325120965951969	0.002899169921875	\\
0.325165357127003	0.00244140625	\\
0.325209748302038	0.00164794921875	\\
0.325254139477072	0.001068115234375	\\
0.325298530652106	0.00018310546875	\\
0.325342921827141	-0.001129150390625	\\
0.325387313002175	-0.0018310546875	\\
0.32543170417721	-0.0028076171875	\\
0.325476095352244	-0.003021240234375	\\
0.325520486527278	-0.002593994140625	\\
0.325564877702313	-0.00286865234375	\\
0.325609268877347	-0.0023193359375	\\
0.325653660052382	-0.002166748046875	\\
0.325698051227416	-0.002288818359375	\\
0.32574244240245	-0.00238037109375	\\
0.325786833577485	-0.002166748046875	\\
0.325831224752519	-0.001434326171875	\\
0.325875615927554	-0.001434326171875	\\
0.325920007102588	-0.000762939453125	\\
0.325964398277622	-0.00018310546875	\\
0.326008789452657	0.001129150390625	\\
0.326053180627691	0.00189208984375	\\
0.326097571802726	0.001434326171875	\\
0.32614196297776	0.001708984375	\\
0.326186354152794	0.002044677734375	\\
0.326230745327829	0.001983642578125	\\
0.326275136502863	0.001068115234375	\\
0.326319527677898	0.0001220703125	\\
0.326363918852932	-0.000396728515625	\\
0.326408310027966	-0.000885009765625	\\
0.326452701203001	-0.000885009765625	\\
0.326497092378035	-0.000885009765625	\\
0.32654148355307	-0.000701904296875	\\
0.326585874728104	-0.00079345703125	\\
0.326630265903138	-0.0008544921875	\\
0.326674657078173	-0.000885009765625	\\
0.326719048253207	-0.000640869140625	\\
0.326763439428242	-0.000396728515625	\\
0.326807830603276	-0.000732421875	\\
0.32685222177831	-0.001678466796875	\\
0.326896612953345	-0.001434326171875	\\
0.326941004128379	0	\\
0.326985395303414	0.00054931640625	\\
0.327029786478448	0.00042724609375	\\
0.327074177653483	0.000274658203125	\\
0.327118568828517	0.000213623046875	\\
0.327162960003551	-0.000213623046875	\\
0.327207351178586	-0.001129150390625	\\
0.32725174235362	-0.001373291015625	\\
0.327296133528655	-0.00091552734375	\\
0.327340524703689	-0.001220703125	\\
0.327384915878723	-0.001617431640625	\\
0.327429307053758	-0.001800537109375	\\
0.327473698228792	-0.001708984375	\\
0.327518089403827	-0.001495361328125	\\
0.327562480578861	-0.00128173828125	\\
0.327606871753895	-0.0008544921875	\\
0.32765126292893	-0.000762939453125	\\
0.327695654103964	-0.000762939453125	\\
0.327740045278999	-0.000335693359375	\\
0.327784436454033	-0.000396728515625	\\
0.327828827629067	-0.000457763671875	\\
0.327873218804102	-0.000518798828125	\\
0.327917609979136	-0.001556396484375	\\
0.327962001154171	0.002227783203125	\\
0.328006392329205	-0.000335693359375	\\
0.328050783504239	-0.000823974609375	\\
0.328095174679274	0.008026123046875	\\
0.328139565854308	-0.1192626953125	\\
0.328183957029343	-0.05877685546875	\\
0.328228348204377	0.058746337890625	\\
0.328272739379411	0.023101806640625	\\
0.328317130554446	0.009796142578125	\\
0.32836152172948	0.0087890625	\\
0.328405912904515	0.0443115234375	\\
0.328450304079549	0.04095458984375	\\
0.328494695254583	0.043731689453125	\\
0.328539086429618	0.022491455078125	\\
0.328583477604652	-0.017730712890625	\\
0.328627868779687	-0.00372314453125	\\
0.328672259954721	0.008544921875	\\
0.328716651129755	0.005401611328125	\\
0.32876104230479	0.006134033203125	\\
0.328805433479824	-0.003936767578125	\\
0.328849824654859	0.004058837890625	\\
0.328894215829893	0.014862060546875	\\
0.328938607004927	0.013092041015625	\\
0.328982998179962	-0.0018310546875	\\
0.329027389354996	-0.00408935546875	\\
0.329071780530031	-0.0028076171875	\\
0.329116171705065	0.011962890625	\\
0.329160562880099	0.01153564453125	\\
0.329204954055134	-0.007537841796875	\\
0.329249345230168	-0.016571044921875	\\
0.329293736405203	-0.02093505859375	\\
0.329338127580237	-0.02178955078125	\\
0.329382518755271	-0.018096923828125	\\
0.329426909930306	-0.015472412109375	\\
0.32947130110534	-0.012420654296875	\\
0.329515692280375	-0.001495361328125	\\
0.329560083455409	0.003753662109375	\\
0.329604474630443	-0.0032958984375	\\
0.329648865805478	-0.007232666015625	\\
0.329693256980512	-0.009979248046875	\\
0.329737648155547	-0.019073486328125	\\
0.329782039330581	-0.018096923828125	\\
0.329826430505616	-0.013702392578125	\\
0.32987082168065	-0.00701904296875	\\
0.329915212855684	-0.0067138671875	\\
0.329959604030719	-0.001983642578125	\\
0.330003995205753	0.003753662109375	\\
0.330048386380787	0.00177001953125	\\
0.330092777555822	-0.0050048828125	\\
0.330137168730856	-0.006134033203125	\\
0.330181559905891	0.00152587890625	\\
0.330225951080925	0.00537109375	\\
0.33027034225596	0.0087890625	\\
0.330314733430994	0.009368896484375	\\
0.330359124606028	0.00616455078125	\\
0.330403515781063	0.005706787109375	\\
0.330447906956097	0.005859375	\\
0.330492298131132	0.003662109375	\\
0.330536689306166	0.000457763671875	\\
0.3305810804812	0.000823974609375	\\
0.330625471656235	0.001190185546875	\\
0.330669862831269	0.0040283203125	\\
0.330714254006304	0.002044677734375	\\
0.330758645181338	-0.001953125	\\
0.330803036356372	-0.000213623046875	\\
0.330847427531407	0.00274658203125	\\
0.330891818706441	0.00494384765625	\\
0.330936209881476	0.002960205078125	\\
0.33098060105651	-0.001068115234375	\\
0.331024992231544	-0.00286865234375	\\
0.331069383406579	-0.001190185546875	\\
0.331113774581613	-0.00274658203125	\\
0.331158165756648	-0.006744384765625	\\
0.331202556931682	-0.006072998046875	\\
0.331246948106716	-0.002838134765625	\\
0.331291339281751	-0.001678466796875	\\
0.331335730456785	-0.001861572265625	\\
0.33138012163182	-0.00274658203125	\\
0.331424512806854	-0.003570556640625	\\
0.331468903981888	-0.002227783203125	\\
0.331513295156923	-0.001922607421875	\\
0.331557686331957	-0.00213623046875	\\
0.331602077506992	-0.00372314453125	\\
0.331646468682026	-0.004119873046875	\\
0.33169085985706	-0.001190185546875	\\
0.331735251032095	-0.0018310546875	\\
0.331779642207129	-0.0050048828125	\\
0.331824033382164	-0.00390625	\\
0.331868424557198	-0.0020751953125	\\
0.331912815732232	-0.0040283203125	\\
0.331957206907267	-0.0040283203125	\\
0.332001598082301	-0.002044677734375	\\
0.332045989257336	-0.00115966796875	\\
0.33209038043237	0.001495361328125	\\
0.332134771607404	0.005096435546875	\\
0.332179162782439	0.0040283203125	\\
0.332223553957473	0.002899169921875	\\
0.332267945132508	0.00360107421875	\\
0.332312336307542	0.00439453125	\\
0.332356727482576	0.0057373046875	\\
0.332401118657611	0.004791259765625	\\
0.332445509832645	0.003265380859375	\\
0.33248990100768	0.001312255859375	\\
0.332534292182714	-0.000762939453125	\\
0.332578683357748	-0.000518798828125	\\
0.332623074532783	0.000885009765625	\\
0.332667465707817	0.00146484375	\\
0.332711856882852	0.001922607421875	\\
0.332756248057886	0.002593994140625	\\
0.33280063923292	0.001007080078125	\\
0.332845030407955	-6.103515625e-05	\\
0.332889421582989	-0.000885009765625	\\
0.332933812758024	-0.00189208984375	\\
0.332978203933058	-0.00225830078125	\\
0.333022595108093	-0.002227783203125	\\
0.333066986283127	-0.002777099609375	\\
0.333111377458161	-0.00347900390625	\\
0.333155768633196	-0.004302978515625	\\
0.33320015980823	-0.005523681640625	\\
0.333244550983265	-0.005706787109375	\\
0.333288942158299	-0.004974365234375	\\
0.333333333333333	-0.00396728515625	\\
0.333377724508368	-0.003143310546875	\\
0.333422115683402	-0.00225830078125	\\
0.333466506858437	-0.003204345703125	\\
0.333510898033471	-0.005157470703125	\\
0.333555289208505	-0.007354736328125	\\
0.33359968038354	-0.008575439453125	\\
0.333644071558574	-0.008056640625	\\
0.333688462733609	-0.007720947265625	\\
0.333732853908643	-0.006591796875	\\
0.333777245083677	-0.004669189453125	\\
0.333821636258712	-0.00335693359375	\\
0.333866027433746	-0.003173828125	\\
0.333910418608781	-0.00384521484375	\\
0.333954809783815	-0.004730224609375	\\
0.333999200958849	-0.0045166015625	\\
0.334043592133884	-0.004364013671875	\\
0.334087983308918	-0.003631591796875	\\
0.334132374483953	-0.00152587890625	\\
0.334176765658987	0.000335693359375	\\
0.334221156834021	0.001953125	\\
0.334265548009056	0.003082275390625	\\
0.33430993918409	0.0023193359375	\\
0.334354330359125	0.000885009765625	\\
0.334398721534159	0.000396728515625	\\
0.334443112709193	-0.000823974609375	\\
0.334487503884228	-0.00128173828125	\\
0.334531895059262	-0.00054931640625	\\
0.334576286234297	-0.00048828125	\\
0.334620677409331	0.000244140625	\\
0.334665068584365	0.002471923828125	\\
0.3347094597594	0.004425048828125	\\
0.334753850934434	0.004791259765625	\\
0.334798242109469	0.00433349609375	\\
0.334842633284503	0.003936767578125	\\
0.334887024459537	0.002593994140625	\\
0.334931415634572	0.001190185546875	\\
0.334975806809606	-0.00030517578125	\\
0.335020197984641	-0.001922607421875	\\
0.335064589159675	-0.002227783203125	\\
0.335108980334709	-0.00177001953125	\\
0.335153371509744	-6.103515625e-05	\\
0.335197762684778	0.00103759765625	\\
0.335242153859813	0.00067138671875	\\
0.335286545034847	0.000213623046875	\\
0.335330936209881	-0.00128173828125	\\
0.335375327384916	-0.00341796875	\\
0.33541971855995	-0.0050048828125	\\
0.335464109734985	-0.0062255859375	\\
0.335508500910019	-0.00531005859375	\\
0.335552892085054	-0.004058837890625	\\
0.335597283260088	-0.003509521484375	\\
0.335641674435122	-0.002471923828125	\\
0.335686065610157	-0.00164794921875	\\
0.335730456785191	-0.001953125	\\
0.335774847960226	-0.0029296875	\\
0.33581923913526	-0.0032958984375	\\
0.335863630310294	-0.00372314453125	\\
0.335908021485329	-0.00347900390625	\\
0.335952412660363	-0.002288818359375	\\
0.335996803835398	-0.0008544921875	\\
0.336041195010432	0.0003662109375	\\
0.336085586185466	0.000946044921875	\\
0.336129977360501	0.00067138671875	\\
0.336174368535535	-0.00030517578125	\\
0.33621875971057	-0.00018310546875	\\
0.336263150885604	-0.0006103515625	\\
0.336307542060638	-0.0010986328125	\\
0.336351933235673	-0.000518798828125	\\
0.336396324410707	0.00018310546875	\\
0.336440715585742	0.001922607421875	\\
0.336485106760776	0.0032958984375	\\
0.33652949793581	0.00384521484375	\\
0.336573889110845	0.004241943359375	\\
0.336618280285879	0.00335693359375	\\
0.336662671460914	0.00225830078125	\\
0.336707062635948	0.001861572265625	\\
0.336751453810982	0.000885009765625	\\
0.336795844986017	-0.000274658203125	\\
0.336840236161051	-0.0009765625	\\
0.336884627336086	-0.00128173828125	\\
0.33692901851112	-0.000885009765625	\\
0.336973409686154	-0.000579833984375	\\
0.337017800861189	-0.0003662109375	\\
0.337062192036223	-0.00030517578125	\\
0.337106583211258	-0.000640869140625	\\
0.337150974386292	-0.00115966796875	\\
0.337195365561326	-0.0010986328125	\\
0.337239756736361	-0.001190185546875	\\
0.337284147911395	-0.00213623046875	\\
0.33732853908643	-0.002471923828125	\\
0.337372930261464	-0.001953125	\\
0.337417321436498	-0.002227783203125	\\
0.337461712611533	-0.00244140625	\\
0.337506103786567	-0.00128173828125	\\
0.337550494961602	-0.00079345703125	\\
0.337594886136636	-0.00067138671875	\\
0.33763927731167	-0.000457763671875	\\
0.337683668486705	-0.000701904296875	\\
0.337728059661739	-0.002410888671875	\\
0.337772450836774	-0.003326416015625	\\
0.337816842011808	-0.0040283203125	\\
0.337861233186842	-0.004425048828125	\\
0.337905624361877	-0.00439453125	\\
0.337950015536911	-0.0045166015625	\\
0.337994406711946	-0.004547119140625	\\
0.33803879788698	-0.00421142578125	\\
0.338083189062014	-0.00299072265625	\\
0.338127580237049	-0.003021240234375	\\
0.338171971412083	-0.00347900390625	\\
0.338216362587118	-0.00341796875	\\
0.338260753762152	-0.002777099609375	\\
0.338305144937187	-0.002227783203125	\\
0.338349536112221	-0.001617431640625	\\
0.338393927287255	-0.00128173828125	\\
0.33843831846229	-0.001495361328125	\\
0.338482709637324	-0.0015869140625	\\
0.338527100812358	-0.001434326171875	\\
0.338571491987393	-0.001617431640625	\\
0.338615883162427	-0.00201416015625	\\
0.338660274337462	-0.002471923828125	\\
0.338704665512496	-0.002685546875	\\
0.338749056687531	-0.002044677734375	\\
0.338793447862565	-0.001220703125	\\
0.338837839037599	-0.001220703125	\\
0.338882230212634	-0.00030517578125	\\
0.338926621387668	0.00048828125	\\
0.338971012562703	-0.0003662109375	\\
0.339015403737737	-0.000762939453125	\\
0.339059794912771	-0.000579833984375	\\
0.339104186087806	-0.000885009765625	\\
0.33914857726284	-0.00079345703125	\\
0.339192968437875	-0.000152587890625	\\
0.339237359612909	-0.00042724609375	\\
0.339281750787943	-0.0013427734375	\\
0.339326141962978	-0.00103759765625	\\
0.339370533138012	-0.000732421875	\\
0.339414924313047	-0.001007080078125	\\
0.339459315488081	-0.000823974609375	\\
0.339503706663115	-0.000457763671875	\\
0.33954809783815	-0.000213623046875	\\
0.339592489013184	-0.000244140625	\\
0.339636880188219	-0.0003662109375	\\
0.339681271363253	-0.001312255859375	\\
0.339725662538287	-0.00250244140625	\\
0.339770053713322	-0.003204345703125	\\
0.339814444888356	-0.0035400390625	\\
0.339858836063391	-0.003448486328125	\\
0.339903227238425	-0.0032958984375	\\
0.339947618413459	-0.002838134765625	\\
0.339992009588494	-0.00250244140625	\\
0.340036400763528	-0.001678466796875	\\
0.340080791938563	-0.00128173828125	\\
0.340125183113597	-0.000885009765625	\\
0.340169574288631	-9.1552734375e-05	\\
0.340213965463666	-0.000396728515625	\\
0.3402583566387	-0.000701904296875	\\
0.340302747813735	-0.0015869140625	\\
0.340347138988769	-0.001678466796875	\\
0.340391530163803	-0.001190185546875	\\
0.340435921338838	-0.001312255859375	\\
0.340480312513872	-0.001922607421875	\\
0.340524703688907	-0.00238037109375	\\
0.340569094863941	-0.002166748046875	\\
0.340613486038975	-0.00201416015625	\\
0.34065787721401	-0.00177001953125	\\
0.340702268389044	-0.001068115234375	\\
0.340746659564079	-0.000335693359375	\\
0.340791050739113	-0.000701904296875	\\
0.340835441914147	-0.00054931640625	\\
0.340879833089182	-0.0006103515625	\\
0.340924224264216	-0.00091552734375	\\
0.340968615439251	-0.000579833984375	\\
0.341013006614285	-0.0006103515625	\\
0.341057397789319	-0.000457763671875	\\
0.341101788964354	0.000152587890625	\\
0.341146180139388	0.000579833984375	\\
0.341190571314423	0.001007080078125	\\
0.341234962489457	0.00042724609375	\\
0.341279353664492	-0.000396728515625	\\
0.341323744839526	-0.0009765625	\\
0.34136813601456	-0.001495361328125	\\
0.341412527189595	-0.00286865234375	\\
0.341456918364629	-0.003265380859375	\\
0.341501309539664	-0.003143310546875	\\
0.341545700714698	-0.0032958984375	\\
0.341590091889732	-0.00323486328125	\\
0.341634483064767	-0.003082275390625	\\
0.341678874239801	-0.00335693359375	\\
0.341723265414836	-0.00384521484375	\\
0.34176765658987	-0.0040283203125	\\
0.341812047764904	-0.00445556640625	\\
0.341856438939939	-0.0045166015625	\\
0.341900830114973	-0.0042724609375	\\
0.341945221290008	-0.002960205078125	\\
0.341989612465042	-0.0010986328125	\\
0.342034003640076	-9.1552734375e-05	\\
0.342078394815111	0.00042724609375	\\
0.342122785990145	0.000244140625	\\
0.34216717716518	-0.000457763671875	\\
0.342211568340214	-0.000274658203125	\\
0.342255959515248	-0.000396728515625	\\
0.342300350690283	-0.001251220703125	\\
0.342344741865317	-0.00152587890625	\\
0.342389133040352	-0.00115966796875	\\
0.342433524215386	-0.000885009765625	\\
0.34247791539042	-0.00067138671875	\\
0.342522306565455	-0.00054931640625	\\
0.342566697740489	-6.103515625e-05	\\
0.342611088915524	0.000885009765625	\\
0.342655480090558	0.00189208984375	\\
0.342699871265592	0.00244140625	\\
0.342744262440627	0.002532958984375	\\
0.342788653615661	0.00146484375	\\
0.342833044790696	0.000946044921875	\\
0.34287743596573	-0.000152587890625	\\
0.342921827140764	-0.0003662109375	\\
0.342966218315799	-0.00030517578125	\\
0.343010609490833	-0.0006103515625	\\
0.343055000665868	-0.000640869140625	\\
0.343099391840902	-0.00115966796875	\\
0.343143783015936	-0.001495361328125	\\
0.343188174190971	-0.001190185546875	\\
0.343232565366005	-0.00091552734375	\\
0.34327695654104	-0.001312255859375	\\
0.343321347716074	-0.0015869140625	\\
0.343365738891108	-0.002227783203125	\\
0.343410130066143	-0.00262451171875	\\
0.343454521241177	-0.001983642578125	\\
0.343498912416212	-0.001556396484375	\\
0.343543303591246	-0.001556396484375	\\
0.34358769476628	-0.001190185546875	\\
0.343632085941315	-0.000701904296875	\\
0.343676477116349	-0.0010986328125	\\
0.343720868291384	-0.00140380859375	\\
0.343765259466418	-0.001708984375	\\
0.343809650641452	-0.001708984375	\\
0.343854041816487	-0.00164794921875	\\
0.343898432991521	-0.00146484375	\\
0.343942824166556	-0.00079345703125	\\
0.34398721534159	-0.001190185546875	\\
0.344031606516625	-0.001434326171875	\\
0.344075997691659	-0.0010986328125	\\
0.344120388866693	-0.00128173828125	\\
0.344164780041728	-0.000640869140625	\\
0.344209171216762	-3.0517578125e-05	\\
0.344253562391797	0.000396728515625	\\
0.344297953566831	0.00091552734375	\\
0.344342344741865	0.000885009765625	\\
0.3443867359169	0.001220703125	\\
0.344431127091934	0.00189208984375	\\
0.344475518266969	0.00225830078125	\\
0.344519909442003	0.00177001953125	\\
0.344564300617037	0.00079345703125	\\
0.344608691792072	-6.103515625e-05	\\
0.344653082967106	-0.00091552734375	\\
0.344697474142141	-0.0013427734375	\\
0.344741865317175	-0.001800537109375	\\
0.344786256492209	-0.00225830078125	\\
0.344830647667244	-0.002838134765625	\\
0.344875038842278	-0.0025634765625	\\
0.344919430017313	-0.001953125	\\
0.344963821192347	-0.002227783203125	\\
0.345008212367381	-0.002227783203125	\\
0.345052603542416	-0.00225830078125	\\
0.34509699471745	-0.002288818359375	\\
0.345141385892485	-0.00189208984375	\\
0.345185777067519	-0.001373291015625	\\
0.345230168242553	-0.001708984375	\\
0.345274559417588	-0.001678466796875	\\
0.345318950592622	-0.001373291015625	\\
0.345363341767657	-0.001953125	\\
0.345407732942691	-0.001556396484375	\\
0.345452124117725	-0.002044677734375	\\
0.34549651529276	-0.002227783203125	\\
0.345540906467794	-0.002288818359375	\\
0.345585297642829	-0.002960205078125	\\
0.345629688817863	-0.001983642578125	\\
0.345674079992897	-0.001434326171875	\\
0.345718471167932	-0.001190185546875	\\
0.345762862342966	-0.00128173828125	\\
0.345807253518001	-0.00189208984375	\\
0.345851644693035	-0.00146484375	\\
0.345896035868069	-0.001617431640625	\\
0.345940427043104	-0.001953125	\\
0.345984818218138	-0.00201416015625	\\
0.346029209393173	-0.00250244140625	\\
0.346073600568207	-0.00177001953125	\\
0.346117991743241	-0.000885009765625	\\
0.346162382918276	-0.001373291015625	\\
0.34620677409331	-0.00128173828125	\\
0.346251165268345	-0.00140380859375	\\
0.346295556443379	-0.00213623046875	\\
0.346339947618413	-0.002227783203125	\\
0.346384338793448	-0.001953125	\\
0.346428729968482	-0.002593994140625	\\
0.346473121143517	-0.003265380859375	\\
0.346517512318551	-0.00341796875	\\
0.346561903493585	-0.002899169921875	\\
0.34660629466862	-0.002777099609375	\\
0.346650685843654	-0.002777099609375	\\
0.346695077018689	-0.00201416015625	\\
0.346739468193723	-0.002227783203125	\\
0.346783859368758	-0.002197265625	\\
0.346828250543792	-0.002227783203125	\\
0.346872641718826	-0.002166748046875	\\
0.346917032893861	-0.001556396484375	\\
0.346961424068895	-0.000762939453125	\\
0.347005815243929	-9.1552734375e-05	\\
0.347050206418964	-0.0003662109375	\\
0.347094597593998	6.103515625e-05	\\
0.347138988769033	6.103515625e-05	\\
0.347183379944067	-0.000244140625	\\
0.347227771119102	-0.000518798828125	\\
0.347272162294136	-0.001312255859375	\\
0.34731655346917	-0.001617431640625	\\
0.347360944644205	-0.001953125	\\
0.347405335819239	-0.002197265625	\\
0.347449726994274	-0.00189208984375	\\
0.347494118169308	-0.00213623046875	\\
0.347538509344342	-0.0023193359375	\\
0.347582900519377	-0.001953125	\\
0.347627291694411	-0.00213623046875	\\
0.347671682869446	-0.00286865234375	\\
0.34771607404448	-0.003143310546875	\\
0.347760465219514	-0.003021240234375	\\
0.347804856394549	-0.003082275390625	\\
0.347849247569583	-0.003021240234375	\\
0.347893638744618	-0.0030517578125	\\
0.347938029919652	-0.00311279296875	\\
0.347982421094686	-0.00347900390625	\\
0.348026812269721	-0.00390625	\\
0.348071203444755	-0.00384521484375	\\
0.34811559461979	-0.00347900390625	\\
0.348159985794824	-0.00384521484375	\\
0.348204376969858	-0.003662109375	\\
0.348248768144893	-0.00274658203125	\\
0.348293159319927	-0.00201416015625	\\
0.348337550494962	-0.001251220703125	\\
0.348381941669996	-0.00091552734375	\\
0.34842633284503	-0.0003662109375	\\
0.348470724020065	-0.0010986328125	\\
0.348515115195099	-0.001678466796875	\\
0.348559506370134	-0.0015869140625	\\
0.348603897545168	-0.0023193359375	\\
0.348648288720202	-0.002197265625	\\
0.348692679895237	-0.00189208984375	\\
0.348737071070271	-0.0015869140625	\\
0.348781462245306	-0.001861572265625	\\
0.34882585342034	-0.001983642578125	\\
0.348870244595374	-0.001861572265625	\\
0.348914635770409	-0.001708984375	\\
0.348959026945443	-0.0020751953125	\\
0.349003418120478	-0.002532958984375	\\
0.349047809295512	-0.00286865234375	\\
0.349092200470546	-0.00335693359375	\\
0.349136591645581	-0.00347900390625	\\
0.349180982820615	-0.0037841796875	\\
0.34922537399565	-0.003936767578125	\\
0.349269765170684	-0.00372314453125	\\
0.349314156345719	-0.003662109375	\\
0.349358547520753	-0.003692626953125	\\
0.349402938695787	-0.0037841796875	\\
0.349447329870822	-0.00335693359375	\\
0.349491721045856	-0.002655029296875	\\
0.34953611222089	-0.002838134765625	\\
0.349580503395925	-0.00311279296875	\\
0.349624894570959	-0.0030517578125	\\
0.349669285745994	-0.002838134765625	\\
0.349713676921028	-0.00213623046875	\\
0.349758068096063	-0.00201416015625	\\
0.349802459271097	-0.0018310546875	\\
0.349846850446131	-0.00115966796875	\\
0.349891241621166	-0.001190185546875	\\
0.3499356327962	-0.000335693359375	\\
0.349980023971235	0.0006103515625	\\
0.350024415146269	0.000946044921875	\\
0.350068806321303	0.0009765625	\\
0.350113197496338	0.000274658203125	\\
0.350157588671372	0.000244140625	\\
0.350201979846407	0.000579833984375	\\
0.350246371021441	0.0006103515625	\\
0.350290762196475	0.000396728515625	\\
0.35033515337151	0.000579833984375	\\
0.350379544546544	0.00128173828125	\\
0.350423935721579	0.00146484375	\\
0.350468326896613	0.00128173828125	\\
0.350512718071647	0.0006103515625	\\
0.350557109246682	-6.103515625e-05	\\
0.350601500421716	-0.000885009765625	\\
0.350645891596751	-0.001678466796875	\\
0.350690282771785	-0.00146484375	\\
0.350734673946819	-0.001495361328125	\\
0.350779065121854	-0.001495361328125	\\
0.350823456296888	-0.001068115234375	\\
0.350867847471923	-0.000885009765625	\\
0.350912238646957	-0.00103759765625	\\
0.350956629821991	-0.001495361328125	\\
0.351001020997026	-0.002105712890625	\\
0.35104541217206	-0.002899169921875	\\
0.351089803347095	-0.0030517578125	\\
0.351134194522129	-0.002349853515625	\\
0.351178585697163	-0.001861572265625	\\
0.351222976872198	-0.00146484375	\\
0.351267368047232	-0.00201416015625	\\
0.351311759222267	-0.00225830078125	\\
0.351356150397301	-0.001617431640625	\\
0.351400541572335	-0.001617431640625	\\
0.35144493274737	-0.001220703125	\\
0.351489323922404	-0.001220703125	\\
0.351533715097439	-0.00177001953125	\\
0.351578106272473	-0.001007080078125	\\
0.351622497447507	-0.000732421875	\\
0.351666888622542	-0.000762939453125	\\
0.351711279797576	-6.103515625e-05	\\
0.351755670972611	0.000701904296875	\\
0.351800062147645	0.00054931640625	\\
0.351844453322679	-9.1552734375e-05	\\
0.351888844497714	-9.1552734375e-05	\\
0.351933235672748	-0.000335693359375	\\
0.351977626847783	-0.0008544921875	\\
0.352022018022817	-0.000640869140625	\\
0.352066409197851	-0.001007080078125	\\
0.352110800372886	-0.001495361328125	\\
0.35215519154792	-0.00152587890625	\\
0.352199582722955	-0.00146484375	\\
0.352243973897989	-0.000823974609375	\\
0.352288365073023	-0.000579833984375	\\
0.352332756248058	-0.000732421875	\\
0.352377147423092	-0.000701904296875	\\
0.352421538598127	-0.00079345703125	\\
0.352465929773161	-0.000518798828125	\\
0.352510320948196	-0.000335693359375	\\
0.35255471212323	-0.000579833984375	\\
0.352599103298264	-0.00054931640625	\\
0.352643494473299	-0.00042724609375	\\
0.352687885648333	-0.000762939453125	\\
0.352732276823368	-0.00067138671875	\\
0.352776667998402	-9.1552734375e-05	\\
0.352821059173436	-0.000396728515625	\\
0.352865450348471	-0.000640869140625	\\
0.352909841523505	-0.00067138671875	\\
0.35295423269854	-0.000640869140625	\\
0.352998623873574	-0.00018310546875	\\
0.353043015048608	-0.00067138671875	\\
0.353087406223643	-0.000701904296875	\\
0.353131797398677	-0.000335693359375	\\
0.353176188573712	-9.1552734375e-05	\\
0.353220579748746	0.000213623046875	\\
0.35326497092378	0.000274658203125	\\
0.353309362098815	0.000335693359375	\\
0.353353753273849	-0.000335693359375	\\
0.353398144448884	-0.000518798828125	\\
0.353442535623918	-0.0003662109375	\\
0.353486926798952	-0.0006103515625	\\
0.353531317973987	-0.000457763671875	\\
0.353575709149021	-0.000335693359375	\\
0.353620100324056	-0.000335693359375	\\
0.35366449149909	-0.000579833984375	\\
0.353708882674124	-0.00079345703125	\\
0.353753273849159	-0.000762939453125	\\
0.353797665024193	-0.000640869140625	\\
0.353842056199228	-0.000885009765625	\\
0.353886447374262	-0.001007080078125	\\
0.353930838549296	-0.000732421875	\\
0.353975229724331	-0.001220703125	\\
0.354019620899365	-0.001220703125	\\
0.3540640120744	-0.00103759765625	\\
0.354108403249434	-0.001190185546875	\\
0.354152794424468	-0.001434326171875	\\
0.354197185599503	-0.001495361328125	\\
0.354241576774537	-0.001220703125	\\
0.354285967949572	-0.001495361328125	\\
0.354330359124606	-0.00146484375	\\
0.35437475029964	-0.00128173828125	\\
0.354419141474675	-0.00115966796875	\\
0.354463532649709	-0.0001220703125	\\
0.354507923824744	0.000152587890625	\\
0.354552314999778	0.0001220703125	\\
0.354596706174812	0.00042724609375	\\
0.354641097349847	-3.0517578125e-05	\\
0.354685488524881	0	\\
0.354729879699916	-9.1552734375e-05	\\
0.35477427087495	-0.000244140625	\\
0.354818662049984	-0.00018310546875	\\
0.354863053225019	-0.0003662109375	\\
0.354907444400053	-0.000457763671875	\\
0.354951835575088	-0.0008544921875	\\
0.354996226750122	-0.000885009765625	\\
0.355040617925156	-0.00103759765625	\\
0.355085009100191	-0.00048828125	\\
0.355129400275225	-0.0001220703125	\\
};
\addplot [color=blue,solid,forget plot]
  table[row sep=crcr]{
0.355129400275225	-0.0001220703125	\\
0.35517379145026	-0.00091552734375	\\
0.355218182625294	-0.0013427734375	\\
0.355262573800329	-0.00091552734375	\\
0.355306964975363	-0.001373291015625	\\
0.355351356150397	-0.002105712890625	\\
0.355395747325432	-0.001617431640625	\\
0.355440138500466	-0.0018310546875	\\
0.355484529675501	-0.00213623046875	\\
0.355528920850535	-0.00177001953125	\\
0.355573312025569	-0.00079345703125	\\
0.355617703200604	-0.0013427734375	\\
0.355662094375638	-0.001556396484375	\\
0.355706485550673	-0.0010986328125	\\
0.355750876725707	-0.00146484375	\\
0.355795267900741	-0.001373291015625	\\
0.355839659075776	-0.001708984375	\\
0.35588405025081	-0.002105712890625	\\
0.355928441425845	-0.00238037109375	\\
0.355972832600879	-0.00177001953125	\\
0.356017223775913	-0.000762939453125	\\
0.356061614950948	-0.001495361328125	\\
0.356106006125982	-0.000946044921875	\\
0.356150397301017	-0.000732421875	\\
0.356194788476051	-0.00140380859375	\\
0.356239179651085	-0.001373291015625	\\
0.35628357082612	-0.0015869140625	\\
0.356327962001154	-0.001922607421875	\\
0.356372353176189	-0.00213623046875	\\
0.356416744351223	-0.001861572265625	\\
0.356461135526257	-0.001739501953125	\\
0.356505526701292	-0.002288818359375	\\
0.356549917876326	-0.002197265625	\\
0.356594309051361	-0.0020751953125	\\
0.356638700226395	-0.001495361328125	\\
0.356683091401429	-0.001922607421875	\\
0.356727482576464	-0.002105712890625	\\
0.356771873751498	-0.0015869140625	\\
0.356816264926533	-0.0015869140625	\\
0.356860656101567	-0.001556396484375	\\
0.356905047276601	-0.00189208984375	\\
0.356949438451636	-0.00189208984375	\\
0.35699382962667	-0.001617431640625	\\
0.357038220801705	-0.001678466796875	\\
0.357082611976739	-0.001708984375	\\
0.357127003151773	-0.001739501953125	\\
0.357171394326808	-0.00128173828125	\\
0.357215785501842	-0.00146484375	\\
0.357260176676877	-0.00103759765625	\\
0.357304567851911	-0.0003662109375	\\
0.357348959026945	-0.0003662109375	\\
0.35739335020198	-0.000213623046875	\\
0.357437741377014	-0.000732421875	\\
0.357482132552049	-0.0003662109375	\\
0.357526523727083	-6.103515625e-05	\\
0.357570914902117	-0.000396728515625	\\
0.357615306077152	0.0001220703125	\\
0.357659697252186	0.000701904296875	\\
0.357704088427221	0.000518798828125	\\
0.357748479602255	0.000518798828125	\\
0.35779287077729	0.0006103515625	\\
0.357837261952324	-6.103515625e-05	\\
0.357881653127358	-0.000396728515625	\\
0.357926044302393	0.00018310546875	\\
0.357970435477427	0.000244140625	\\
0.358014826652461	-0.000244140625	\\
0.358059217827496	-0.00018310546875	\\
0.35810360900253	9.1552734375e-05	\\
0.358148000177565	0.0008544921875	\\
0.358192391352599	0.00128173828125	\\
0.358236782527634	0.000946044921875	\\
0.358281173702668	0.001129150390625	\\
0.358325564877702	0.001556396484375	\\
0.358369956052737	0.00146484375	\\
0.358414347227771	0.00103759765625	\\
0.358458738402806	0.000274658203125	\\
0.35850312957784	-0.0001220703125	\\
0.358547520752874	-0.00054931640625	\\
0.358591911927909	-0.000762939453125	\\
0.358636303102943	-0.000518798828125	\\
0.358680694277978	-0.000762939453125	\\
0.358725085453012	-0.000335693359375	\\
0.358769476628046	6.103515625e-05	\\
0.358813867803081	0.000244140625	\\
0.358858258978115	0.00030517578125	\\
0.35890265015315	-0.000213623046875	\\
0.358947041328184	-0.000640869140625	\\
0.358991432503218	-0.00054931640625	\\
0.359035823678253	-0.000579833984375	\\
0.359080214853287	-0.001068115234375	\\
0.359124606028322	-0.001129150390625	\\
0.359168997203356	-0.001251220703125	\\
0.35921338837839	-0.001129150390625	\\
0.359257779553425	-0.001007080078125	\\
0.359302170728459	-0.00115966796875	\\
0.359346561903494	-0.00067138671875	\\
0.359390953078528	-0.001190185546875	\\
0.359435344253562	-0.001739501953125	\\
0.359479735428597	-0.00115966796875	\\
0.359524126603631	-0.00177001953125	\\
0.359568517778666	-0.001922607421875	\\
0.3596129089537	-0.00128173828125	\\
0.359657300128734	-0.001190185546875	\\
0.359701691303769	-0.001190185546875	\\
0.359746082478803	-0.0009765625	\\
0.359790473653838	-0.00042724609375	\\
0.359834864828872	-0.000518798828125	\\
0.359879256003906	-0.000640869140625	\\
0.359923647178941	-0.00048828125	\\
0.359968038353975	-0.000335693359375	\\
0.36001242952901	-0.000244140625	\\
0.360056820704044	-0.000518798828125	\\
0.360101211879078	-0.00042724609375	\\
0.360145603054113	-6.103515625e-05	\\
0.360189994229147	9.1552734375e-05	\\
0.360234385404182	0.000152587890625	\\
0.360278776579216	0.00091552734375	\\
0.36032316775425	0.001068115234375	\\
0.360367558929285	0.001251220703125	\\
0.360411950104319	0.00115966796875	\\
0.360456341279354	0.00067138671875	\\
0.360500732454388	0.000732421875	\\
0.360545123629422	0.0003662109375	\\
0.360589514804457	0.0001220703125	\\
0.360633905979491	-0.00018310546875	\\
0.360678297154526	3.0517578125e-05	\\
0.36072268832956	-0.0001220703125	\\
0.360767079504594	9.1552734375e-05	\\
0.360811470679629	0.0006103515625	\\
0.360855861854663	0.000244140625	\\
0.360900253029698	0.0001220703125	\\
0.360944644204732	-9.1552734375e-05	\\
0.360989035379767	-3.0517578125e-05	\\
0.361033426554801	-9.1552734375e-05	\\
0.361077817729835	-0.000579833984375	\\
0.36112220890487	-6.103515625e-05	\\
0.361166600079904	-0.000335693359375	\\
0.361210991254939	-0.00067138671875	\\
0.361255382429973	-0.000335693359375	\\
0.361299773605007	-0.0006103515625	\\
0.361344164780042	-0.000701904296875	\\
0.361388555955076	-0.001708984375	\\
0.361432947130111	-0.00146484375	\\
0.361477338305145	-0.001434326171875	\\
0.361521729480179	-0.001953125	\\
0.361566120655214	-0.001495361328125	\\
0.361610511830248	-0.001678466796875	\\
0.361654903005283	-0.00140380859375	\\
0.361699294180317	-0.000762939453125	\\
0.361743685355351	-0.0010986328125	\\
0.361788076530386	-0.00140380859375	\\
0.36183246770542	-0.00146484375	\\
0.361876858880455	-0.001190185546875	\\
0.361921250055489	-0.000885009765625	\\
0.361965641230523	-0.00115966796875	\\
0.362010032405558	-0.000396728515625	\\
0.362054423580592	-0.00067138671875	\\
0.362098814755627	-0.000885009765625	\\
0.362143205930661	-0.000335693359375	\\
0.362187597105695	-0.000396728515625	\\
0.36223198828073	-0.000518798828125	\\
0.362276379455764	-0.000518798828125	\\
0.362320770630799	-0.0001220703125	\\
0.362365161805833	-0.000152587890625	\\
0.362409552980867	0.000274658203125	\\
0.362453944155902	0.0003662109375	\\
0.362498335330936	0.0001220703125	\\
0.362542726505971	0.000335693359375	\\
0.362587117681005	-0.00054931640625	\\
0.362631508856039	-0.00146484375	\\
0.362675900031074	-0.001556396484375	\\
0.362720291206108	-0.001708984375	\\
0.362764682381143	-0.001922607421875	\\
0.362809073556177	-0.00091552734375	\\
0.362853464731211	-0.001220703125	\\
0.362897855906246	-0.001861572265625	\\
0.36294224708128	-0.001739501953125	\\
0.362986638256315	-0.00177001953125	\\
0.363031029431349	-0.001861572265625	\\
0.363075420606383	-0.00152587890625	\\
0.363119811781418	-0.001953125	\\
0.363164202956452	-0.002685546875	\\
0.363208594131487	-0.002685546875	\\
0.363252985306521	-0.0030517578125	\\
0.363297376481555	-0.0030517578125	\\
0.36334176765659	-0.002838134765625	\\
0.363386158831624	-0.003204345703125	\\
0.363430550006659	-0.0029296875	\\
0.363474941181693	-0.00244140625	\\
0.363519332356728	-0.002349853515625	\\
0.363563723531762	-0.0025634765625	\\
0.363608114706796	-0.002716064453125	\\
0.363652505881831	-0.00250244140625	\\
0.363696897056865	-0.00286865234375	\\
0.3637412882319	-0.002685546875	\\
0.363785679406934	-0.002197265625	\\
0.363830070581968	-0.002349853515625	\\
0.363874461757003	-0.001678466796875	\\
0.363918852932037	-0.0010986328125	\\
0.363963244107072	-0.001708984375	\\
0.364007635282106	-0.001800537109375	\\
0.36405202645714	-0.001556396484375	\\
0.364096417632175	-0.00146484375	\\
0.364140808807209	-0.001739501953125	\\
0.364185199982244	-0.001434326171875	\\
0.364229591157278	-0.000640869140625	\\
0.364273982332312	-0.0006103515625	\\
0.364318373507347	-0.00030517578125	\\
0.364362764682381	0.000244140625	\\
0.364407155857416	0.000396728515625	\\
0.36445154703245	-0.000152587890625	\\
0.364495938207484	-3.0517578125e-05	\\
0.364540329382519	-0.00048828125	\\
0.364584720557553	-0.001373291015625	\\
0.364629111732588	-0.00146484375	\\
0.364673502907622	-0.00164794921875	\\
0.364717894082656	-0.001953125	\\
0.364762285257691	-0.0020751953125	\\
0.364806676432725	-0.001800537109375	\\
0.36485106760776	-0.001373291015625	\\
0.364895458782794	-0.0013427734375	\\
0.364939849957828	-0.001373291015625	\\
0.364984241132863	-0.001251220703125	\\
0.365028632307897	-0.00146484375	\\
0.365073023482932	-0.00152587890625	\\
0.365117414657966	-0.001190185546875	\\
0.365161805833	-0.001007080078125	\\
0.365206197008035	-0.0013427734375	\\
0.365250588183069	-0.00164794921875	\\
0.365294979358104	-0.00115966796875	\\
0.365339370533138	-0.001068115234375	\\
0.365383761708172	-0.0009765625	\\
0.365428152883207	-0.000732421875	\\
0.365472544058241	-0.00042724609375	\\
0.365516935233276	-0.000701904296875	\\
0.36556132640831	-0.00054931640625	\\
0.365605717583344	-0.000152587890625	\\
0.365650108758379	9.1552734375e-05	\\
0.365694499933413	-0.000213623046875	\\
0.365738891108448	-0.000732421875	\\
0.365783282283482	6.103515625e-05	\\
0.365827673458516	0.000274658203125	\\
0.365872064633551	0.000152587890625	\\
0.365916455808585	-0.00054931640625	\\
0.36596084698362	-0.001129150390625	\\
0.366005238158654	-0.00115966796875	\\
0.366049629333688	-0.000701904296875	\\
0.366094020508723	-0.000579833984375	\\
0.366138411683757	-0.000823974609375	\\
0.366182802858792	-0.000946044921875	\\
0.366227194033826	-0.00103759765625	\\
0.366271585208861	-0.001251220703125	\\
0.366315976383895	-0.001190185546875	\\
0.366360367558929	-0.001617431640625	\\
0.366404758733964	-0.00164794921875	\\
0.366449149908998	-0.00164794921875	\\
0.366493541084032	-0.00164794921875	\\
0.366537932259067	-0.0010986328125	\\
0.366582323434101	-0.001190185546875	\\
0.366626714609136	-0.001190185546875	\\
0.36667110578417	-0.001129150390625	\\
0.366715496959205	-0.001251220703125	\\
0.366759888134239	-0.001129150390625	\\
0.366804279309273	-0.001312255859375	\\
0.366848670484308	-0.0018310546875	\\
0.366893061659342	-0.001251220703125	\\
0.366937452834377	-0.0009765625	\\
0.366981844009411	-0.001007080078125	\\
0.367026235184445	-0.000762939453125	\\
0.36707062635948	-0.00103759765625	\\
0.367115017534514	-0.00103759765625	\\
0.367159408709549	-0.001190185546875	\\
0.367203799884583	-0.00146484375	\\
0.367248191059617	-0.00189208984375	\\
0.367292582234652	-0.00177001953125	\\
0.367336973409686	-0.001190185546875	\\
0.367381364584721	-0.001678466796875	\\
0.367425755759755	-0.001739501953125	\\
0.367470146934789	-0.0020751953125	\\
0.367514538109824	-0.0020751953125	\\
0.367558929284858	-0.00189208984375	\\
0.367603320459893	-0.002044677734375	\\
0.367647711634927	-0.00213623046875	\\
0.367692102809961	-0.0025634765625	\\
0.367736493984996	-0.00244140625	\\
0.36778088516003	-0.00225830078125	\\
0.367825276335065	-0.002532958984375	\\
0.367869667510099	-0.00213623046875	\\
0.367914058685133	-0.002044677734375	\\
0.367958449860168	-0.001953125	\\
0.368002841035202	-0.00177001953125	\\
0.368047232210237	-0.001708984375	\\
0.368091623385271	-0.002410888671875	\\
0.368136014560305	-0.00201416015625	\\
0.36818040573534	-0.001251220703125	\\
0.368224796910374	-0.00140380859375	\\
0.368269188085409	-0.001068115234375	\\
0.368313579260443	-0.00079345703125	\\
0.368357970435477	-0.000335693359375	\\
0.368402361610512	-0.000640869140625	\\
0.368446752785546	-0.000732421875	\\
0.368491143960581	-0.000640869140625	\\
0.368535535135615	-0.0010986328125	\\
0.368579926310649	-0.000946044921875	\\
0.368624317485684	-0.001068115234375	\\
0.368668708660718	-0.001068115234375	\\
0.368713099835753	-0.001556396484375	\\
0.368757491010787	-0.00152587890625	\\
0.368801882185821	-0.001251220703125	\\
0.368846273360856	-0.00103759765625	\\
0.36889066453589	-0.001556396484375	\\
0.368935055710925	-0.001678466796875	\\
0.368979446885959	-0.001434326171875	\\
0.369023838060993	-0.00146484375	\\
0.369068229236028	-0.001922607421875	\\
0.369112620411062	-0.00225830078125	\\
0.369157011586097	-0.001708984375	\\
0.369201402761131	-0.00164794921875	\\
0.369245793936165	-0.001434326171875	\\
0.3692901851112	-0.001556396484375	\\
0.369334576286234	-0.001129150390625	\\
0.369378967461269	-0.00146484375	\\
0.369423358636303	-0.00213623046875	\\
0.369467749811338	-0.001617431640625	\\
0.369512140986372	-0.00146484375	\\
0.369556532161406	-0.00201416015625	\\
0.369600923336441	-0.001922607421875	\\
0.369645314511475	-0.001495361328125	\\
0.36968970568651	-0.00128173828125	\\
0.369734096861544	-0.001129150390625	\\
0.369778488036578	-0.000762939453125	\\
0.369822879211613	-0.00018310546875	\\
0.369867270386647	0.00054931640625	\\
0.369911661561682	0.0003662109375	\\
0.369956052736716	0.00018310546875	\\
0.37000044391175	0.000335693359375	\\
0.370044835086785	-0.0001220703125	\\
0.370089226261819	-3.0517578125e-05	\\
0.370133617436854	-0.00018310546875	\\
0.370178008611888	-0.000396728515625	\\
0.370222399786922	-0.000244140625	\\
0.370266790961957	-0.0003662109375	\\
0.370311182136991	-0.00054931640625	\\
0.370355573312026	-0.000518798828125	\\
0.37039996448706	-0.00079345703125	\\
0.370444355662094	-0.000518798828125	\\
0.370488746837129	-0.00030517578125	\\
0.370533138012163	-0.000335693359375	\\
0.370577529187198	-0.000518798828125	\\
0.370621920362232	-0.00091552734375	\\
0.370666311537266	-0.00152587890625	\\
0.370710702712301	-0.001220703125	\\
0.370755093887335	-0.001739501953125	\\
0.37079948506237	-0.002471923828125	\\
0.370843876237404	-0.0015869140625	\\
0.370888267412438	-0.00115966796875	\\
0.370932658587473	-0.001495361328125	\\
0.370977049762507	-0.00140380859375	\\
0.371021440937542	-0.001495361328125	\\
0.371065832112576	-0.002227783203125	\\
0.37111022328761	-0.001708984375	\\
0.371154614462645	-0.00128173828125	\\
0.371199005637679	-0.001953125	\\
0.371243396812714	-0.00140380859375	\\
0.371287787987748	-0.001129150390625	\\
0.371332179162782	-0.00103759765625	\\
0.371376570337817	-0.000579833984375	\\
0.371420961512851	-0.00048828125	\\
0.371465352687886	-0.00054931640625	\\
0.37150974386292	-0.00042724609375	\\
0.371554135037954	-0.000732421875	\\
0.371598526212989	-0.00054931640625	\\
0.371642917388023	-0.00048828125	\\
0.371687308563058	-0.0006103515625	\\
0.371731699738092	-0.000274658203125	\\
0.371776090913126	-0.000274658203125	\\
0.371820482088161	-0.000396728515625	\\
0.371864873263195	-3.0517578125e-05	\\
0.37190926443823	0.000274658203125	\\
0.371953655613264	-0.000335693359375	\\
0.371998046788299	-0.001190185546875	\\
0.372042437963333	-0.00115966796875	\\
0.372086829138367	-0.00091552734375	\\
0.372131220313402	-0.001312255859375	\\
0.372175611488436	-0.00177001953125	\\
0.372220002663471	-0.001800537109375	\\
0.372264393838505	-0.001800537109375	\\
0.372308785013539	-0.001678466796875	\\
0.372353176188574	-0.001220703125	\\
0.372397567363608	-0.0010986328125	\\
0.372441958538643	-0.00140380859375	\\
0.372486349713677	-0.0010986328125	\\
0.372530740888711	-0.0009765625	\\
0.372575132063746	-0.00115966796875	\\
0.37261952323878	-0.001220703125	\\
0.372663914413815	-0.000335693359375	\\
0.372708305588849	9.1552734375e-05	\\
0.372752696763883	9.1552734375e-05	\\
0.372797087938918	0.0003662109375	\\
0.372841479113952	9.1552734375e-05	\\
0.372885870288987	0.000152587890625	\\
0.372930261464021	0.000762939453125	\\
0.372974652639055	0.000335693359375	\\
0.37301904381409	9.1552734375e-05	\\
0.373063434989124	0.0003662109375	\\
0.373107826164159	0.000152587890625	\\
0.373152217339193	-0.0001220703125	\\
0.373196608514227	0.000244140625	\\
0.373240999689262	0.000457763671875	\\
0.373285390864296	9.1552734375e-05	\\
0.373329782039331	-0.000152587890625	\\
0.373374173214365	-0.000274658203125	\\
0.373418564389399	-0.000946044921875	\\
0.373462955564434	-0.0015869140625	\\
0.373507346739468	-0.001007080078125	\\
0.373551737914503	-0.001373291015625	\\
0.373596129089537	-0.001434326171875	\\
0.373640520264571	-0.000823974609375	\\
0.373684911439606	-0.001129150390625	\\
0.37372930261464	-0.00189208984375	\\
0.373773693789675	-0.001495361328125	\\
0.373818084964709	-0.001708984375	\\
0.373862476139743	-0.00244140625	\\
0.373906867314778	-0.0028076171875	\\
0.373951258489812	-0.00323486328125	\\
0.373995649664847	-0.003082275390625	\\
0.374040040839881	-0.00286865234375	\\
0.374084432014915	-0.002471923828125	\\
0.37412882318995	-0.00225830078125	\\
0.374173214364984	-0.00177001953125	\\
0.374217605540019	-0.001495361328125	\\
0.374261996715053	-0.001129150390625	\\
0.374306387890087	-0.00067138671875	\\
0.374350779065122	-0.000213623046875	\\
0.374395170240156	-0.0001220703125	\\
0.374439561415191	-0.000152587890625	\\
0.374483952590225	0.000518798828125	\\
0.374528343765259	0.000518798828125	\\
0.374572734940294	0.000244140625	\\
0.374617126115328	0.000701904296875	\\
0.374661517290363	0.000732421875	\\
0.374705908465397	0.00054931640625	\\
0.374750299640431	0.001007080078125	\\
0.374794690815466	0.001190185546875	\\
0.3748390819905	0.001617431640625	\\
0.374883473165535	0.00091552734375	\\
0.374927864340569	0.00018310546875	\\
0.374972255515603	0.00030517578125	\\
0.375016646690638	-0.000244140625	\\
0.375061037865672	-0.000701904296875	\\
0.375105429040707	-0.000701904296875	\\
0.375149820215741	-0.000518798828125	\\
0.375194211390776	-0.000396728515625	\\
0.37523860256581	-0.00048828125	\\
0.375282993740844	-0.000701904296875	\\
0.375327384915879	-0.000946044921875	\\
0.375371776090913	-0.000732421875	\\
0.375416167265948	-0.00067138671875	\\
0.375460558440982	-0.000885009765625	\\
0.375504949616016	-0.00042724609375	\\
0.375549340791051	-0.000518798828125	\\
0.375593731966085	-0.000244140625	\\
0.37563812314112	-0.00048828125	\\
0.375682514316154	-0.001007080078125	\\
0.375726905491188	-0.000579833984375	\\
0.375771296666223	-0.000274658203125	\\
0.375815687841257	-6.103515625e-05	\\
0.375860079016292	-0.000244140625	\\
0.375904470191326	-0.000396728515625	\\
0.37594886136636	-3.0517578125e-05	\\
0.375993252541395	0.00048828125	\\
0.376037643716429	0.00042724609375	\\
0.376082034891464	0.000457763671875	\\
0.376126426066498	0.000396728515625	\\
0.376170817241532	0.000335693359375	\\
0.376215208416567	0.000732421875	\\
0.376259599591601	0.000823974609375	\\
0.376303990766636	0.000579833984375	\\
0.37634838194167	0.000213623046875	\\
0.376392773116704	0.00067138671875	\\
0.376437164291739	0.000762939453125	\\
0.376481555466773	0.00091552734375	\\
0.376525946641808	0.000457763671875	\\
0.376570337816842	9.1552734375e-05	\\
0.376614728991876	-0.000213623046875	\\
0.376659120166911	-0.000274658203125	\\
0.376703511341945	-6.103515625e-05	\\
0.37674790251698	-0.00018310546875	\\
0.376792293692014	-0.00030517578125	\\
0.376836684867048	-0.00103759765625	\\
0.376881076042083	-0.00054931640625	\\
0.376925467217117	-0.000701904296875	\\
0.376969858392152	-0.000762939453125	\\
0.377014249567186	-0.00079345703125	\\
0.37705864074222	-0.000946044921875	\\
0.377103031917255	-0.00079345703125	\\
0.377147423092289	-0.000946044921875	\\
0.377191814267324	-0.00140380859375	\\
0.377236205442358	-0.001007080078125	\\
0.377280596617392	-0.0009765625	\\
0.377324987792427	-0.00067138671875	\\
0.377369378967461	-0.000244140625	\\
0.377413770142496	0.000335693359375	\\
0.37745816131753	0.000335693359375	\\
0.377502552492564	0.0006103515625	\\
0.377546943667599	0.001007080078125	\\
0.377591334842633	6.103515625e-05	\\
0.377635726017668	-0.000457763671875	\\
0.377680117192702	-0.000274658203125	\\
0.377724508367736	-0.000152587890625	\\
0.377768899542771	0.000335693359375	\\
0.377813290717805	0.000579833984375	\\
0.37785768189284	-3.0517578125e-05	\\
0.377902073067874	-0.0003662109375	\\
0.377946464242909	0.00030517578125	\\
0.377990855417943	0.001129150390625	\\
0.378035246592977	0.001007080078125	\\
0.378079637768012	0.000823974609375	\\
0.378124028943046	0.000274658203125	\\
0.378168420118081	-9.1552734375e-05	\\
0.378212811293115	-0.00018310546875	\\
0.378257202468149	-0.000335693359375	\\
0.378301593643184	-0.00067138671875	\\
0.378345984818218	-0.00115966796875	\\
0.378390375993253	-0.000579833984375	\\
0.378434767168287	0.00018310546875	\\
0.378479158343321	6.103515625e-05	\\
0.378523549518356	-0.0003662109375	\\
0.37856794069339	-0.000762939453125	\\
0.378612331868425	6.103515625e-05	\\
0.378656723043459	-6.103515625e-05	\\
0.378701114218493	-0.0003662109375	\\
0.378745505393528	0.000274658203125	\\
0.378789896568562	0.00054931640625	\\
0.378834287743597	0.0006103515625	\\
0.378878678918631	0.001007080078125	\\
0.378923070093665	0.0003662109375	\\
0.3789674612687	0.000274658203125	\\
0.379011852443734	0.000518798828125	\\
0.379056243618769	0.0009765625	\\
0.379100634793803	0.001068115234375	\\
0.379145025968837	0.000640869140625	\\
0.379189417143872	0.001007080078125	\\
0.379233808318906	0.0008544921875	\\
0.379278199493941	0.00140380859375	\\
0.379322590668975	0.001495361328125	\\
0.379366981844009	0.001251220703125	\\
0.379411373019044	0.001495361328125	\\
0.379455764194078	0.002044677734375	\\
0.379500155369113	0.00225830078125	\\
0.379544546544147	0.001708984375	\\
0.379588937719181	0.00140380859375	\\
0.379633328894216	0.001617431640625	\\
0.37967772006925	0.001708984375	\\
0.379722111244285	0.00115966796875	\\
0.379766502419319	0.001251220703125	\\
0.379810893594353	0.0008544921875	\\
0.379855284769388	0.000823974609375	\\
0.379899675944422	0.001251220703125	\\
0.379944067119457	0.000823974609375	\\
0.379988458294491	0.00091552734375	\\
0.380032849469525	0.00103759765625	\\
0.38007724064456	0.00115966796875	\\
0.380121631819594	0.0008544921875	\\
0.380166022994629	0.001220703125	\\
0.380210414169663	0.00140380859375	\\
0.380254805344697	0.00128173828125	\\
0.380299196519732	0.001251220703125	\\
0.380343587694766	0.0008544921875	\\
0.380387978869801	0.000762939453125	\\
0.380432370044835	0.0009765625	\\
0.38047676121987	0.0009765625	\\
0.380521152394904	0.000946044921875	\\
0.380565543569938	0.00042724609375	\\
0.380609934744973	0.000885009765625	\\
0.380654325920007	0.001251220703125	\\
0.380698717095041	0.000701904296875	\\
0.380743108270076	0.000457763671875	\\
0.38078749944511	0.00054931640625	\\
0.380831890620145	0.00048828125	\\
0.380876281795179	0.000335693359375	\\
0.380920672970214	0.000274658203125	\\
0.380965064145248	0.000457763671875	\\
0.381009455320282	0.0006103515625	\\
0.381053846495317	0.000640869140625	\\
0.381098237670351	0.000457763671875	\\
0.381142628845386	0.00054931640625	\\
0.38118702002042	-0.00018310546875	\\
0.381231411195454	-0.000946044921875	\\
0.381275802370489	-0.000823974609375	\\
0.381320193545523	-0.000946044921875	\\
0.381364584720558	-0.0003662109375	\\
0.381408975895592	-0.00067138671875	\\
0.381453367070626	-0.000518798828125	\\
0.381497758245661	-0.0003662109375	\\
0.381542149420695	-0.000518798828125	\\
0.38158654059573	-0.00048828125	\\
0.381630931770764	-0.0006103515625	\\
0.381675322945798	-0.00054931640625	\\
0.381719714120833	-0.000732421875	\\
0.381764105295867	-0.000701904296875	\\
0.381808496470902	-0.000396728515625	\\
0.381852887645936	0.000457763671875	\\
0.38189727882097	0.000640869140625	\\
0.381941669996005	0.000518798828125	\\
0.381986061171039	0.001434326171875	\\
0.382030452346074	0.00140380859375	\\
0.382074843521108	0.001251220703125	\\
0.382119234696142	0.001617431640625	\\
0.382163625871177	0.00177001953125	\\
0.382208017046211	0.0020751953125	\\
0.382252408221246	0.002044677734375	\\
0.38229679939628	0.002227783203125	\\
0.382341190571314	0.001953125	\\
0.382385581746349	0.001861572265625	\\
0.382429972921383	0.00164794921875	\\
0.382474364096418	0.00128173828125	\\
0.382518755271452	0.001373291015625	\\
0.382563146446486	0.001129150390625	\\
0.382607537621521	0.0013427734375	\\
0.382651928796555	0.001434326171875	\\
0.38269631997159	0.001068115234375	\\
0.382740711146624	0.0009765625	\\
0.382785102321658	0.00103759765625	\\
0.382829493496693	0.000701904296875	\\
0.382873884671727	0.000732421875	\\
0.382918275846762	0.000457763671875	\\
0.382962667021796	0.00018310546875	\\
0.38300705819683	0.000701904296875	\\
0.383051449371865	0.00048828125	\\
0.383095840546899	-0.000152587890625	\\
0.383140231721934	9.1552734375e-05	\\
0.383184622896968	0.00067138671875	\\
0.383229014072002	0.000823974609375	\\
0.383273405247037	0.000244140625	\\
0.383317796422071	0.00042724609375	\\
0.383362187597106	0.000823974609375	\\
0.38340657877214	0.0003662109375	\\
0.383450969947174	0.00048828125	\\
0.383495361122209	0.000732421875	\\
0.383539752297243	0.000152587890625	\\
0.383584143472278	0.00048828125	\\
0.383628534647312	0.00030517578125	\\
0.383672925822347	0.00091552734375	\\
0.383717316997381	0.001068115234375	\\
0.383761708172415	0.0003662109375	\\
0.38380609934745	0.00103759765625	\\
0.383850490522484	0.00103759765625	\\
0.383894881697519	0.00091552734375	\\
0.383939272872553	0.001434326171875	\\
0.383983664047587	0.00128173828125	\\
0.384028055222622	0.00177001953125	\\
0.384072446397656	0.00152587890625	\\
0.384116837572691	0.0013427734375	\\
0.384161228747725	0.001708984375	\\
0.384205619922759	0.001800537109375	\\
0.384250011097794	0.001617431640625	\\
0.384294402272828	0.00128173828125	\\
0.384338793447863	0.001708984375	\\
0.384383184622897	0.001556396484375	\\
0.384427575797931	0.0013427734375	\\
0.384471966972966	0.001434326171875	\\
0.384516358148	0.0015869140625	\\
0.384560749323035	0.001373291015625	\\
0.384605140498069	0.0008544921875	\\
0.384649531673103	0.000457763671875	\\
0.384693922848138	0.001251220703125	\\
0.384738314023172	0.000823974609375	\\
0.384782705198207	0.001312255859375	\\
0.384827096373241	0.0013427734375	\\
0.384871487548275	0.001617431640625	\\
0.38491587872331	0.001800537109375	\\
0.384960269898344	0.001800537109375	\\
0.385004661073379	0.001953125	\\
0.385049052248413	0.001678466796875	\\
0.385093443423447	0.00152587890625	\\
0.385137834598482	0.0015869140625	\\
0.385182225773516	0.0013427734375	\\
0.385226616948551	0.00079345703125	\\
0.385271008123585	0.001068115234375	\\
0.385315399298619	0.001556396484375	\\
0.385359790473654	0.00152587890625	\\
0.385404181648688	0.001007080078125	\\
0.385448572823723	0.00128173828125	\\
0.385492963998757	0.00189208984375	\\
0.385537355173791	0.00177001953125	\\
0.385581746348826	0.0008544921875	\\
0.38562613752386	0.000396728515625	\\
0.385670528698895	0.000885009765625	\\
0.385714919873929	0.000579833984375	\\
0.385759311048963	0.00054931640625	\\
0.385803702223998	0.00091552734375	\\
0.385848093399032	0.000701904296875	\\
0.385892484574067	0.000885009765625	\\
0.385936875749101	0.000579833984375	\\
0.385981266924135	0.000335693359375	\\
0.38602565809917	0.00103759765625	\\
0.386070049274204	0	\\
0.386114440449239	-6.103515625e-05	\\
0.386158831624273	0.000701904296875	\\
0.386203222799308	0.000579833984375	\\
0.386247613974342	0.000396728515625	\\
0.386292005149376	-0.00018310546875	\\
0.386336396324411	-6.103515625e-05	\\
0.386380787499445	0.0003662109375	\\
0.38642517867448	0.000152587890625	\\
0.386469569849514	0.000244140625	\\
0.386513961024548	0.0008544921875	\\
0.386558352199583	0.000885009765625	\\
0.386602743374617	0.00054931640625	\\
0.386647134549652	0.000701904296875	\\
0.386691525724686	0.000762939453125	\\
0.38673591689972	0.000640869140625	\\
0.386780308074755	0.00054931640625	\\
0.386824699249789	0.000579833984375	\\
0.386869090424824	0.00091552734375	\\
0.386913481599858	0.001312255859375	\\
0.386957872774892	0.0003662109375	\\
0.387002263949927	0.000640869140625	\\
0.387046655124961	0.00103759765625	\\
0.387091046299996	0.0003662109375	\\
0.38713543747503	0.001007080078125	\\
0.387179828650064	0.000457763671875	\\
0.387224219825099	3.0517578125e-05	\\
0.387268611000133	0.0009765625	\\
0.387313002175168	0.0008544921875	\\
0.387357393350202	0.001251220703125	\\
0.387401784525236	0.0010986328125	\\
0.387446175700271	0.0009765625	\\
0.387490566875305	0.000762939453125	\\
0.38753495805034	0.00042724609375	\\
0.387579349225374	0.00018310546875	\\
0.387623740400408	6.103515625e-05	\\
0.387668131575443	0.00067138671875	\\
0.387712522750477	0.0001220703125	\\
0.387756913925512	-0.0003662109375	\\
0.387801305100546	0.00018310546875	\\
0.38784569627558	-0.00048828125	\\
0.387890087450615	-0.000640869140625	\\
0.387934478625649	0.000396728515625	\\
0.387978869800684	0.000244140625	\\
0.388023260975718	0.000274658203125	\\
0.388067652150752	0.000396728515625	\\
0.388112043325787	-0.000213623046875	\\
0.388156434500821	-6.103515625e-05	\\
0.388200825675856	-0.00030517578125	\\
0.38824521685089	-0.00079345703125	\\
0.388289608025924	-0.00103759765625	\\
0.388333999200959	-0.00079345703125	\\
0.388378390375993	-0.00048828125	\\
0.388422781551028	-0.00030517578125	\\
0.388467172726062	-0.000701904296875	\\
0.388511563901096	-0.001251220703125	\\
0.388555955076131	-0.00048828125	\\
0.388600346251165	9.1552734375e-05	\\
0.3886447374262	-0.00018310546875	\\
0.388689128601234	-0.000274658203125	\\
0.388733519776268	-0.000518798828125	\\
0.388777910951303	-0.000640869140625	\\
0.388822302126337	-0.000213623046875	\\
0.388866693301372	6.103515625e-05	\\
0.388911084476406	-0.000213623046875	\\
0.388955475651441	-0.000518798828125	\\
0.388999866826475	0.000244140625	\\
0.389044258001509	0.000701904296875	\\
0.389088649176544	0	\\
0.389133040351578	0.000244140625	\\
0.389177431526612	0.000457763671875	\\
0.389221822701647	0.000762939453125	\\
0.389266213876681	0.001068115234375	\\
0.389310605051716	0.0010986328125	\\
0.38935499622675	0.00146484375	\\
0.389399387401785	0.000946044921875	\\
0.389443778576819	0.001495361328125	\\
0.389488169751853	0.0020751953125	\\
0.389532560926888	0.002105712890625	\\
0.389576952101922	0.002044677734375	\\
0.389621343276957	0.002197265625	\\
0.389665734451991	0.001739501953125	\\
0.389710125627025	0.001373291015625	\\
0.38975451680206	0.001007080078125	\\
0.389798907977094	0.0013427734375	\\
0.389843299152129	0.001495361328125	\\
0.389887690327163	0.001922607421875	\\
0.389932081502197	0.002288818359375	\\
0.389976472677232	0.00140380859375	\\
0.390020863852266	0.0018310546875	\\
0.390065255027301	0.002044677734375	\\
0.390109646202335	0.001373291015625	\\
0.390154037377369	0.0009765625	\\
0.390198428552404	0.00048828125	\\
0.390242819727438	0.000762939453125	\\
0.390287210902473	-6.103515625e-05	\\
0.390331602077507	0.00018310546875	\\
0.390375993252541	9.1552734375e-05	\\
0.390420384427576	-0.00067138671875	\\
0.39046477560261	-0.000152587890625	\\
0.390509166777645	0.00030517578125	\\
0.390553557952679	-0.000579833984375	\\
0.390597949127713	-0.000946044921875	\\
0.390642340302748	-0.000274658203125	\\
0.390686731477782	-0.000579833984375	\\
0.390731122652817	-0.000640869140625	\\
0.390775513827851	-0.0013427734375	\\
0.390819905002885	-0.00128173828125	\\
0.39086429617792	-0.001190185546875	\\
0.390908687352954	-0.0008544921875	\\
0.390953078527989	-0.000762939453125	\\
0.390997469703023	-0.000762939453125	\\
0.391041860878057	-0.00042724609375	\\
0.391086252053092	-0.000335693359375	\\
0.391130643228126	-0.0003662109375	\\
0.391175034403161	-0.000213623046875	\\
0.391219425578195	-0.000335693359375	\\
0.391263816753229	-0.000213623046875	\\
0.391308207928264	-3.0517578125e-05	\\
0.391352599103298	0.0001220703125	\\
0.391396990278333	0.000396728515625	\\
0.391441381453367	0.000823974609375	\\
0.391485772628401	0.0009765625	\\
0.391530163803436	0.0010986328125	\\
0.39157455497847	0.001373291015625	\\
0.391618946153505	0.001373291015625	\\
0.391663337328539	0.001800537109375	\\
0.391707728503573	0.002105712890625	\\
0.391752119678608	0.0013427734375	\\
0.391796510853642	0.00164794921875	\\
0.391840902028677	0.00238037109375	\\
0.391885293203711	0.001922607421875	\\
0.391929684378745	0.001129150390625	\\
0.39197407555378	0.000762939453125	\\
0.392018466728814	0.001220703125	\\
0.392062857903849	0.0001220703125	\\
0.392107249078883	9.1552734375e-05	\\
0.392151640253918	0.001068115234375	\\
0.392196031428952	0.00048828125	\\
0.392240422603986	0.000579833984375	\\
0.392284813779021	0.0015869140625	\\
0.392329204954055	0.001922607421875	\\
0.39237359612909	0.002105712890625	\\
0.392417987304124	0.001922607421875	\\
0.392462378479158	0.00164794921875	\\
0.392506769654193	0.00140380859375	\\
0.392551160829227	0.001373291015625	\\
0.392595552004262	0.00115966796875	\\
0.392639943179296	0.0010986328125	\\
0.39268433435433	0.0010986328125	\\
0.392728725529365	0.001312255859375	\\
0.392773116704399	0.001312255859375	\\
0.392817507879434	0.001220703125	\\
0.392861899054468	0.00079345703125	\\
0.392906290229502	0.0010986328125	\\
0.392950681404537	0.001556396484375	\\
0.392995072579571	0.001556396484375	\\
0.393039463754606	0.001678466796875	\\
0.39308385492964	0.001220703125	\\
0.393128246104674	0.00115966796875	\\
0.393172637279709	0.001434326171875	\\
0.393217028454743	0.00140380859375	\\
0.393261419629778	0.001007080078125	\\
0.393305810804812	0.001129150390625	\\
0.393350201979846	0.0018310546875	\\
0.393394593154881	0.00152587890625	\\
0.393438984329915	0.0013427734375	\\
0.39348337550495	0.001434326171875	\\
0.393527766679984	0.001373291015625	\\
0.393572157855018	0.0015869140625	\\
0.393616549030053	0.00225830078125	\\
0.393660940205087	0.001739501953125	\\
0.393705331380122	0.0009765625	\\
0.393749722555156	0.000885009765625	\\
0.39379411373019	0.00128173828125	\\
0.393838504905225	0.00103759765625	\\
0.393882896080259	0.000518798828125	\\
0.393927287255294	0.00042724609375	\\
0.393971678430328	0.00079345703125	\\
0.394016069605362	0.000946044921875	\\
0.394060460780397	0.000732421875	\\
0.394104851955431	0.000640869140625	\\
0.394149243130466	0.001251220703125	\\
0.3941936343055	0.001068115234375	\\
0.394238025480534	0.00128173828125	\\
0.394282416655569	0.001617431640625	\\
0.394326807830603	0.001556396484375	\\
0.394371199005638	0.001373291015625	\\
0.394415590180672	0.000946044921875	\\
0.394459981355706	0.00146484375	\\
0.394504372530741	0.00152587890625	\\
0.394548763705775	0.001312255859375	\\
0.39459315488081	0.00164794921875	\\
0.394637546055844	0.00115966796875	\\
0.394681937230879	0.001068115234375	\\
0.394726328405913	0.001007080078125	\\
0.394770719580947	0.00079345703125	\\
0.394815110755982	0.001220703125	\\
0.394859501931016	0.001861572265625	\\
0.394903893106051	0.0013427734375	\\
0.394948284281085	0.00067138671875	\\
0.394992675456119	0.001434326171875	\\
0.395037066631154	0.001708984375	\\
0.395081457806188	0.001220703125	\\
0.395125848981223	0.00128173828125	\\
0.395170240156257	0.000885009765625	\\
0.395214631331291	0.001251220703125	\\
0.395259022506326	0.00128173828125	\\
0.39530341368136	0.001220703125	\\
0.395347804856395	0.0018310546875	\\
0.395392196031429	0.001800537109375	\\
0.395436587206463	0.001373291015625	\\
0.395480978381498	0.00146484375	\\
0.395525369556532	0.001800537109375	\\
0.395569760731567	0.002044677734375	\\
0.395614151906601	0.001312255859375	\\
0.395658543081635	0.0008544921875	\\
0.39570293425667	0.001251220703125	\\
0.395747325431704	0.000762939453125	\\
0.395791716606739	0.00128173828125	\\
0.395836107781773	0.00140380859375	\\
0.395880498956807	0.000640869140625	\\
0.395924890131842	0.00048828125	\\
0.395969281306876	0.000762939453125	\\
0.396013672481911	0.000823974609375	\\
0.396058063656945	0.001007080078125	\\
0.396102454831979	0.000946044921875	\\
0.396146846007014	0.00030517578125	\\
0.396191237182048	0.000701904296875	\\
0.396235628357083	0.000823974609375	\\
0.396280019532117	0.0010986328125	\\
0.396324410707151	0.00103759765625	\\
0.396368801882186	0.00079345703125	\\
0.39641319305722	0.00067138671875	\\
0.396457584232255	0.000518798828125	\\
0.396501975407289	9.1552734375e-05	\\
0.396546366582323	-0.0001220703125	\\
0.396590757757358	0.001007080078125	\\
0.396635148932392	0.000762939453125	\\
0.396679540107427	0.00079345703125	\\
0.396723931282461	0.000335693359375	\\
0.396768322457495	0.000885009765625	\\
0.39681271363253	0.001190185546875	\\
0.396857104807564	0.000579833984375	\\
0.396901495982599	0.000457763671875	\\
0.396945887157633	-6.103515625e-05	\\
0.396990278332667	-0.0006103515625	\\
0.397034669507702	-0.000152587890625	\\
0.397079060682736	0.0006103515625	\\
0.397123451857771	-0.000152587890625	\\
0.397167843032805	-0.00018310546875	\\
0.397212234207839	0.00054931640625	\\
0.397256625382874	0.0003662109375	\\
0.397301016557908	0.000762939453125	\\
0.397345407732943	0.0009765625	\\
0.397389798907977	0.000732421875	\\
0.397434190083012	0.000946044921875	\\
0.397478581258046	0.00067138671875	\\
0.39752297243308	0.000823974609375	\\
0.397567363608115	0.0010986328125	\\
0.397611754783149	0.000244140625	\\
0.397656145958183	0.00042724609375	\\
0.397700537133218	0.00079345703125	\\
0.397744928308252	0.000457763671875	\\
0.397789319483287	-6.103515625e-05	\\
0.397833710658321	0.0006103515625	\\
0.397878101833356	0.00079345703125	\\
0.39792249300839	0.000701904296875	\\
0.397966884183424	0.00048828125	\\
0.398011275358459	0.00030517578125	\\
0.398055666533493	0.00042724609375	\\
0.398100057708528	0.00054931640625	\\
0.398144448883562	-0.000152587890625	\\
0.398188840058596	-0.0001220703125	\\
0.398233231233631	0.000244140625	\\
0.398277622408665	0.000213623046875	\\
0.3983220135837	0.000335693359375	\\
0.398366404758734	0.00030517578125	\\
0.398410795933768	0.0001220703125	\\
0.398455187108803	-0.000274658203125	\\
0.398499578283837	-0.0001220703125	\\
0.398543969458872	0.00030517578125	\\
0.398588360633906	0.000762939453125	\\
0.39863275180894	0.00067138671875	\\
0.398677142983975	0.00048828125	\\
0.398721534159009	0.0013427734375	\\
0.398765925334044	0.001434326171875	\\
0.398810316509078	0.0010986328125	\\
0.398854707684112	0.001129150390625	\\
0.398899098859147	0.00164794921875	\\
0.398943490034181	0.0015869140625	\\
0.398987881209216	0.00140380859375	\\
0.39903227238425	0.001312255859375	\\
0.399076663559284	0.001495361328125	\\
0.399121054734319	0.00201416015625	\\
0.399165445909353	0.001495361328125	\\
0.399209837084388	0.0009765625	\\
0.399254228259422	0.000885009765625	\\
0.399298619434456	0.001190185546875	\\
0.399343010609491	0.001983642578125	\\
0.399387401784525	0.001373291015625	\\
0.39943179295956	0.001251220703125	\\
0.399476184134594	0.001251220703125	\\
0.399520575309628	0.000823974609375	\\
0.399564966484663	0.00146484375	\\
0.399609357659697	0.00103759765625	\\
0.399653748834732	0.000640869140625	\\
0.399698140009766	0.001434326171875	\\
0.3997425311848	0.001678466796875	\\
0.399786922359835	0.00115966796875	\\
0.399831313534869	0.000823974609375	\\
0.399875704709904	0.000885009765625	\\
0.399920095884938	0.0006103515625	\\
0.399964487059972	0.000762939453125	\\
0.400008878235007	0.0003662109375	\\
0.400053269410041	-3.0517578125e-05	\\
0.400097660585076	0.000396728515625	\\
0.40014205176011	0.00042724609375	\\
0.400186442935144	0.000732421875	\\
0.400230834110179	0.000457763671875	\\
0.400275225285213	-6.103515625e-05	\\
0.400319616460248	0.000457763671875	\\
0.400364007635282	0.00054931640625	\\
0.400408398810317	-0.000152587890625	\\
0.400452789985351	-9.1552734375e-05	\\
0.400497181160385	0.000335693359375	\\
0.40054157233542	0.00067138671875	\\
0.400585963510454	0.000579833984375	\\
0.400630354685489	0.00054931640625	\\
0.400674745860523	0.00054931640625	\\
0.400719137035557	0.0015869140625	\\
0.400763528210592	0.001434326171875	\\
0.400807919385626	0.000640869140625	\\
0.400852310560661	0.001190185546875	\\
0.400896701735695	0.001129150390625	\\
0.400941092910729	0.001129150390625	\\
0.400985484085764	0.00152587890625	\\
0.401029875260798	0.00152587890625	\\
0.401074266435833	0.001312255859375	\\
0.401118657610867	0.002044677734375	\\
0.401163048785901	0.0018310546875	\\
0.401207439960936	0.000946044921875	\\
0.40125183113597	0.000946044921875	\\
0.401296222311005	0.00128173828125	\\
0.401340613486039	0.00115966796875	\\
0.401385004661073	0.00091552734375	\\
0.401429395836108	0.00091552734375	\\
0.401473787011142	0.001312255859375	\\
0.401518178186177	0.00152587890625	\\
0.401562569361211	0.00140380859375	\\
0.401606960536245	0.001373291015625	\\
0.40165135171128	0.00079345703125	\\
0.401695742886314	0.00115966796875	\\
0.401740134061349	0.0009765625	\\
0.401784525236383	0.00103759765625	\\
0.401828916411417	0.001220703125	\\
0.401873307586452	0.00128173828125	\\
0.401917698761486	0.00140380859375	\\
0.401962089936521	0.00177001953125	\\
0.402006481111555	0.001953125	\\
0.402050872286589	0.001434326171875	\\
0.402095263461624	0.001495361328125	\\
0.402139654636658	0.00164794921875	\\
0.402184045811693	0.00140380859375	\\
0.402228436986727	0.000762939453125	\\
0.402272828161761	0.000885009765625	\\
0.402317219336796	0.000946044921875	\\
0.40236161051183	0.0008544921875	\\
0.402406001686865	0.000640869140625	\\
0.402450392861899	0.00091552734375	\\
0.402494784036933	0.001312255859375	\\
0.402539175211968	0.001190185546875	\\
0.402583566387002	0.001434326171875	\\
0.402627957562037	0.00152587890625	\\
0.402672348737071	0.001434326171875	\\
0.402716739912105	0.00054931640625	\\
0.40276113108714	0.00079345703125	\\
0.402805522262174	0.001739501953125	\\
0.402849913437209	0.001373291015625	\\
0.402894304612243	0.000762939453125	\\
0.402938695787277	0.000946044921875	\\
0.402983086962312	0.000946044921875	\\
0.403027478137346	0.00140380859375	\\
0.403071869312381	0.001556396484375	\\
0.403116260487415	0.001434326171875	\\
0.40316065166245	0.001129150390625	\\
0.403205042837484	0.001495361328125	\\
0.403249434012518	0.002166748046875	\\
0.403293825187553	0.002105712890625	\\
0.403338216362587	0.00146484375	\\
0.403382607537622	0.001190185546875	\\
0.403426998712656	0.002410888671875	\\
0.40347138988769	0.002288818359375	\\
0.403515781062725	0.001434326171875	\\
0.403560172237759	0.001068115234375	\\
0.403604563412794	0.0008544921875	\\
0.403648954587828	0.001312255859375	\\
0.403693345762862	0.001220703125	\\
0.403737736937897	0.000274658203125	\\
0.403782128112931	0.000579833984375	\\
0.403826519287966	0.00115966796875	\\
0.403870910463	0.001373291015625	\\
0.403915301638034	0.001007080078125	\\
0.403959692813069	0.000823974609375	\\
0.404004083988103	0.001312255859375	\\
0.404048475163138	0.001220703125	\\
0.404092866338172	0.001312255859375	\\
0.404137257513206	0.000885009765625	\\
0.404181648688241	-6.103515625e-05	\\
0.404226039863275	0.000244140625	\\
0.40427043103831	0.001007080078125	\\
0.404314822213344	0.00067138671875	\\
0.404359213388378	-0.00067138671875	\\
0.404403604563413	0	\\
0.404447995738447	0.0008544921875	\\
0.404492386913482	0.0003662109375	\\
0.404536778088516	0.000457763671875	\\
0.40458116926355	9.1552734375e-05	\\
0.404625560438585	9.1552734375e-05	\\
0.404669951613619	0.00018310546875	\\
0.404714342788654	0.000213623046875	\\
0.404758733963688	0.000244140625	\\
0.404803125138722	-0.00067138671875	\\
0.404847516313757	-0.000335693359375	\\
0.404891907488791	0.00030517578125	\\
0.404936298663826	0.000762939453125	\\
0.40498068983886	0.00115966796875	\\
0.405025081013894	0.0003662109375	\\
0.405069472188929	0.000213623046875	\\
0.405113863363963	0.00079345703125	\\
0.405158254538998	0.0013427734375	\\
0.405202645714032	0.00054931640625	\\
0.405247036889066	-0.0001220703125	\\
0.405291428064101	0.00042724609375	\\
0.405335819239135	0.000762939453125	\\
0.40538021041417	0.00048828125	\\
0.405424601589204	0.000701904296875	\\
0.405468992764238	3.0517578125e-05	\\
0.405513383939273	0.000518798828125	\\
0.405557775114307	0.000946044921875	\\
0.405602166289342	0.000640869140625	\\
0.405646557464376	0.000274658203125	\\
0.40569094863941	0.0003662109375	\\
0.405735339814445	0.000762939453125	\\
0.405779730989479	0.0009765625	\\
0.405824122164514	0.00103759765625	\\
0.405868513339548	0.00054931640625	\\
0.405912904514583	0.0008544921875	\\
0.405957295689617	0.001007080078125	\\
0.406001686864651	0.0008544921875	\\
0.406046078039686	0.001007080078125	\\
0.40609046921472	0.0010986328125	\\
0.406134860389754	0.00091552734375	\\
0.406179251564789	0.00067138671875	\\
0.406223642739823	0.000335693359375	\\
0.406268033914858	0.0003662109375	\\
0.406312425089892	0.000335693359375	\\
0.406356816264927	0.000274658203125	\\
0.406401207439961	0.000640869140625	\\
0.406445598614995	0.000244140625	\\
0.40648998979003	0.000396728515625	\\
0.406534380965064	0.00030517578125	\\
0.406578772140099	0.00079345703125	\\
0.406623163315133	0.001312255859375	\\
0.406667554490167	0.0009765625	\\
0.406711945665202	0.001068115234375	\\
0.406756336840236	0.001312255859375	\\
0.406800728015271	0.001007080078125	\\
0.406845119190305	0.000701904296875	\\
0.406889510365339	0.000701904296875	\\
0.406933901540374	0.001220703125	\\
0.406978292715408	0.000946044921875	\\
0.407022683890443	0.00048828125	\\
0.407067075065477	0.001068115234375	\\
0.407111466240511	0.0008544921875	\\
0.407155857415546	0.000640869140625	\\
0.40720024859058	0.000885009765625	\\
0.407244639765615	0.0009765625	\\
0.407289030940649	0.001007080078125	\\
0.407333422115683	0.0006103515625	\\
0.407377813290718	0.000457763671875	\\
0.407422204465752	0.000823974609375	\\
0.407466595640787	0.0006103515625	\\
0.407510986815821	0.000946044921875	\\
0.407555377990855	0.000701904296875	\\
0.40759976916589	0.0006103515625	\\
0.407644160340924	0.000579833984375	\\
0.407688551515959	0.000579833984375	\\
0.407732942690993	0.000701904296875	\\
0.407777333866027	0.000579833984375	\\
0.407821725041062	0.000885009765625	\\
0.407866116216096	0.000701904296875	\\
0.407910507391131	0.0006103515625	\\
0.407954898566165	0.000946044921875	\\
0.407999289741199	0.00128173828125	\\
0.408043680916234	0.00079345703125	\\
0.408088072091268	0.0008544921875	\\
0.408132463266303	0.001068115234375	\\
0.408176854441337	0.001373291015625	\\
0.408221245616371	0.001739501953125	\\
0.408265636791406	0.001190185546875	\\
0.40831002796644	0.001373291015625	\\
0.408354419141475	0.00189208984375	\\
0.408398810316509	0.001800537109375	\\
0.408443201491544	0.002197265625	\\
0.408487592666578	0.0020751953125	\\
0.408531983841612	0.00201416015625	\\
0.408576375016647	0.002044677734375	\\
0.408620766191681	0.001983642578125	\\
0.408665157366715	0.001739501953125	\\
0.40870954854175	0.001434326171875	\\
0.408753939716784	0.00238037109375	\\
0.408798330891819	0.00238037109375	\\
0.408842722066853	0.002044677734375	\\
0.408887113241888	0.00201416015625	\\
0.408931504416922	0.001922607421875	\\
0.408975895591956	0.00213623046875	\\
0.409020286766991	0.002655029296875	\\
0.409064677942025	0.00250244140625	\\
0.40910906911706	0.00201416015625	\\
0.409153460292094	0.001922607421875	\\
0.409197851467128	0.001953125	\\
0.409242242642163	0.00225830078125	\\
0.409286633817197	0.001983642578125	\\
0.409331024992232	0.001800537109375	\\
0.409375416167266	0.00201416015625	\\
0.4094198073423	0.002166748046875	\\
0.409464198517335	0.001953125	\\
0.409508589692369	0.001556396484375	\\
0.409552980867404	0.00201416015625	\\
0.409597372042438	0.002349853515625	\\
0.409641763217472	0.0020751953125	\\
0.409686154392507	0.00189208984375	\\
0.409730545567541	0.002166748046875	\\
0.409774936742576	0.00225830078125	\\
0.40981932791761	0.002044677734375	\\
0.409863719092644	0.002105712890625	\\
0.409908110267679	0.002197265625	\\
0.409952501442713	0.002471923828125	\\
0.409996892617748	0.002349853515625	\\
0.410041283792782	0.001983642578125	\\
0.410085674967816	0.001434326171875	\\
0.410130066142851	0.0009765625	\\
0.410174457317885	0.001068115234375	\\
0.41021884849292	0.001678466796875	\\
0.410263239667954	0.001708984375	\\
0.410307630842988	0.001312255859375	\\
0.410352022018023	0.001678466796875	\\
0.410396413193057	0.002288818359375	\\
0.410440804368092	0.001922607421875	\\
0.410485195543126	0.00238037109375	\\
0.41052958671816	0.002349853515625	\\
0.410573977893195	0.00189208984375	\\
0.410618369068229	0.002532958984375	\\
0.410662760243264	0.00213623046875	\\
0.410707151418298	0.001739501953125	\\
0.410751542593332	0.00244140625	\\
0.410795933768367	0.002288818359375	\\
0.410840324943401	0.0023193359375	\\
0.410884716118436	0.0028076171875	\\
0.41092910729347	0.002471923828125	\\
0.410973498468504	0.002471923828125	\\
0.411017889643539	0.002105712890625	\\
0.411062280818573	0.002288818359375	\\
0.411106671993608	0.00225830078125	\\
0.411151063168642	0.002227783203125	\\
0.411195454343676	0.002105712890625	\\
0.411239845518711	0.00225830078125	\\
0.411284236693745	0.002288818359375	\\
0.41132862786878	0.00244140625	\\
0.411373019043814	0.00274658203125	\\
0.411417410218848	0.0029296875	\\
0.411461801393883	0.003173828125	\\
0.411506192568917	0.002960205078125	\\
0.411550583743952	0.002593994140625	\\
0.411594974918986	0.0018310546875	\\
0.411639366094021	0.0015869140625	\\
0.411683757269055	0.001922607421875	\\
0.411728148444089	0.00225830078125	\\
0.411772539619124	0.001708984375	\\
0.411816930794158	0.0010986328125	\\
0.411861321969193	0.001251220703125	\\
0.411905713144227	0.0018310546875	\\
0.411950104319261	0.002197265625	\\
0.411994495494296	0.001373291015625	\\
0.41203888666933	0.000762939453125	\\
0.412083277844365	0.00115966796875	\\
0.412127669019399	0.001617431640625	\\
0.412172060194433	0.001220703125	\\
0.412216451369468	0.000701904296875	\\
0.412260842544502	0.00091552734375	\\
0.412305233719537	0.001007080078125	\\
0.412349624894571	0.001007080078125	\\
0.412394016069605	0.00115966796875	\\
0.41243840724464	0.000518798828125	\\
0.412482798419674	-0.000213623046875	\\
0.412527189594709	9.1552734375e-05	\\
0.412571580769743	0.000762939453125	\\
0.412615971944777	0.00048828125	\\
0.412660363119812	0.000274658203125	\\
0.412704754294846	0.000518798828125	\\
0.412749145469881	0.000640869140625	\\
0.412793536644915	0.000823974609375	\\
0.412837927819949	0.000762939453125	\\
0.412882318994984	0.0003662109375	\\
0.412926710170018	0.001251220703125	\\
0.412971101345053	0.002288818359375	\\
0.413015492520087	0.001708984375	\\
0.413059883695121	0.00128173828125	\\
0.413104274870156	0.001922607421875	\\
0.41314866604519	0.002288818359375	\\
0.413193057220225	0.002410888671875	\\
0.413237448395259	0.0020751953125	\\
0.413281839570293	0.001678466796875	\\
0.413326230745328	0.0013427734375	\\
0.413370621920362	0.00189208984375	\\
0.413415013095397	0.0028076171875	\\
0.413459404270431	0.0020751953125	\\
0.413503795445465	0.001617431640625	\\
0.4135481866205	0.00146484375	\\
0.413592577795534	0.00189208984375	\\
0.413636968970569	0.002044677734375	\\
0.413681360145603	0.00140380859375	\\
0.413725751320637	0.00164794921875	\\
0.413770142495672	0.00140380859375	\\
0.413814533670706	0.00128173828125	\\
0.413858924845741	0.001434326171875	\\
0.413903316020775	0.000946044921875	\\
0.413947707195809	0.0009765625	\\
0.413992098370844	0.001007080078125	\\
0.414036489545878	0.00079345703125	\\
0.414080880720913	0.000823974609375	\\
0.414125271895947	0.000701904296875	\\
0.414169663070981	0.00054931640625	\\
0.414214054246016	0.0003662109375	\\
0.41425844542105	0.000244140625	\\
0.414302836596085	6.103515625e-05	\\
0.414347227771119	3.0517578125e-05	\\
0.414391618946154	0.000244140625	\\
0.414436010121188	-6.103515625e-05	\\
0.414480401296222	-0.00018310546875	\\
0.414524792471257	0.0003662109375	\\
0.414569183646291	0	\\
0.414613574821326	0.00018310546875	\\
0.41465796599636	0.000274658203125	\\
0.414702357171394	0.00030517578125	\\
0.414746748346429	0.00030517578125	\\
0.414791139521463	-0.000335693359375	\\
0.414835530696498	0.000152587890625	\\
0.414879921871532	0.00115966796875	\\
0.414924313046566	0.001251220703125	\\
0.414968704221601	0.000762939453125	\\
0.415013095396635	0.000823974609375	\\
0.41505748657167	0.000518798828125	\\
0.415101877746704	0.00079345703125	\\
0.415146268921738	0.001068115234375	\\
0.415190660096773	0.000579833984375	\\
0.415235051271807	0.000762939453125	\\
0.415279442446842	0.000732421875	\\
0.415323833621876	0.00048828125	\\
0.41536822479691	0.000640869140625	\\
0.415412615971945	0.001068115234375	\\
0.415457007146979	0.000579833984375	\\
0.415501398322014	6.103515625e-05	\\
0.415545789497048	0.000518798828125	\\
0.415590180672082	0.000946044921875	\\
0.415634571847117	0.001068115234375	\\
0.415678963022151	0.00103759765625	\\
0.415723354197186	0.000640869140625	\\
0.41576774537222	6.103515625e-05	\\
0.415812136547254	0.000579833984375	\\
0.415856527722289	0.001007080078125	\\
0.415900918897323	0.000244140625	\\
0.415945310072358	-6.103515625e-05	\\
0.415989701247392	6.103515625e-05	\\
0.416034092422426	0.0006103515625	\\
0.416078483597461	0.00048828125	\\
0.416122874772495	-0.00018310546875	\\
0.41616726594753	-0.0001220703125	\\
0.416211657122564	0.00054931640625	\\
0.416256048297598	0.00054931640625	\\
0.416300439472633	0.00048828125	\\
0.416344830647667	0.00042724609375	\\
0.416389221822702	0.000274658203125	\\
0.416433612997736	0.000244140625	\\
0.41647800417277	0.000518798828125	\\
0.416522395347805	0.000244140625	\\
0.416566786522839	0.0001220703125	\\
0.416611177697874	0.000152587890625	\\
0.416655568872908	0.00042724609375	\\
0.416699960047942	0.00115966796875	\\
0.416744351222977	0.000762939453125	\\
0.416788742398011	-0.0001220703125	\\
0.416833133573046	0.00030517578125	\\
0.41687752474808	0.000762939453125	\\
0.416921915923115	0.000335693359375	\\
0.416966307098149	-0.000213623046875	\\
0.417010698273183	-9.1552734375e-05	\\
0.417055089448218	0.000244140625	\\
0.417099480623252	0.0001220703125	\\
0.417143871798286	-6.103515625e-05	\\
0.417188262973321	-0.000457763671875	\\
0.417232654148355	-0.000457763671875	\\
0.41727704532339	-0.00018310546875	\\
0.417321436498424	0.00042724609375	\\
0.417365827673459	0.00030517578125	\\
0.417410218848493	-0.000244140625	\\
0.417454610023527	-0.00018310546875	\\
0.417499001198562	-0.000213623046875	\\
0.417543392373596	-0.000396728515625	\\
0.417587783548631	-0.000244140625	\\
0.417632174723665	-0.000213623046875	\\
0.417676565898699	-0.000885009765625	\\
0.417720957073734	-0.001007080078125	\\
0.417765348248768	-0.000762939453125	\\
0.417809739423803	-0.0010986328125	\\
0.417854130598837	-0.0006103515625	\\
0.417898521773871	-0.000244140625	\\
0.417942912948906	-3.0517578125e-05	\\
0.41798730412394	-0.000701904296875	\\
0.418031695298975	-0.000579833984375	\\
0.418076086474009	0.00054931640625	\\
0.418120477649043	0.000579833984375	\\
0.418164868824078	6.103515625e-05	\\
0.418209259999112	0.000274658203125	\\
0.418253651174147	0.000335693359375	\\
0.418298042349181	0.00018310546875	\\
0.418342433524215	0.000762939453125	\\
0.41838682469925	0.000732421875	\\
0.418431215874284	-0.00018310546875	\\
0.418475607049319	0.00048828125	\\
0.418519998224353	0.0008544921875	\\
0.418564389399387	0.000885009765625	\\
0.418608780574422	0.0013427734375	\\
0.418653171749456	0.000946044921875	\\
0.418697562924491	0.0006103515625	\\
0.418741954099525	0.00054931640625	\\
0.418786345274559	0.000640869140625	\\
0.418830736449594	0.000457763671875	\\
0.418875127624628	0.00030517578125	\\
0.418919518799663	0.00054931640625	\\
0.418963909974697	0.000701904296875	\\
0.419008301149731	0.000701904296875	\\
0.419052692324766	0.000152587890625	\\
0.4190970834998	0.00054931640625	\\
0.419141474674835	0.000579833984375	\\
0.419185865849869	-6.103515625e-05	\\
0.419230257024903	0.000457763671875	\\
0.419274648199938	0.00030517578125	\\
0.419319039374972	0.000274658203125	\\
0.419363430550007	0.000823974609375	\\
0.419407821725041	0.000274658203125	\\
0.419452212900075	9.1552734375e-05	\\
0.41949660407511	0.000244140625	\\
0.419540995250144	-0.000457763671875	\\
0.419585386425179	-0.000640869140625	\\
0.419629777600213	6.103515625e-05	\\
0.419674168775247	-0.000244140625	\\
0.419718559950282	-0.000396728515625	\\
0.419762951125316	-0.000396728515625	\\
0.419807342300351	-6.103515625e-05	\\
0.419851733475385	0.0003662109375	\\
0.419896124650419	-0.0001220703125	\\
0.419940515825454	9.1552734375e-05	\\
0.419984907000488	0.000396728515625	\\
0.420029298175523	0	\\
0.420073689350557	-6.103515625e-05	\\
0.420118080525592	0.00018310546875	\\
0.420162471700626	0.00054931640625	\\
0.42020686287566	0.000335693359375	\\
0.420251254050695	0.00048828125	\\
0.420295645225729	-3.0517578125e-05	\\
0.420340036400764	-0.00018310546875	\\
0.420384427575798	0.0001220703125	\\
0.420428818750832	0.000152587890625	\\
0.420473209925867	0.000244140625	\\
0.420517601100901	-0.00030517578125	\\
0.420561992275936	-0.00042724609375	\\
0.42060638345097	-0.00054931640625	\\
0.420650774626004	-0.000885009765625	\\
0.420695165801039	-0.001220703125	\\
0.420739556976073	-0.001922607421875	\\
0.420783948151108	-0.002288818359375	\\
0.420828339326142	-0.00152587890625	\\
0.420872730501176	-0.001495361328125	\\
0.420917121676211	-0.0015869140625	\\
0.420961512851245	-0.001434326171875	\\
0.42100590402628	-0.00201416015625	\\
0.421050295201314	-0.0018310546875	\\
0.421094686376348	-0.001983642578125	\\
0.421139077551383	-0.00201416015625	\\
0.421183468726417	-0.00201416015625	\\
0.421227859901452	-0.002166748046875	\\
0.421272251076486	-0.001953125	\\
0.42131664225152	-0.00140380859375	\\
0.421361033426555	-0.00115966796875	\\
0.421405424601589	-0.001312255859375	\\
0.421449815776624	-0.0013427734375	\\
0.421494206951658	-0.001251220703125	\\
0.421538598126692	-0.0008544921875	\\
0.421582989301727	-0.00146484375	\\
0.421627380476761	-0.0018310546875	\\
0.421671771651796	-0.00164794921875	\\
0.42171616282683	-0.00146484375	\\
0.421760554001864	-0.00140380859375	\\
0.421804945176899	-0.001556396484375	\\
0.421849336351933	-0.00146484375	\\
0.421893727526968	-0.001556396484375	\\
0.421938118702002	-0.00152587890625	\\
0.421982509877036	-0.00115966796875	\\
0.422026901052071	-0.001251220703125	\\
0.422071292227105	-0.001708984375	\\
0.42211568340214	-0.001434326171875	\\
0.422160074577174	-0.00152587890625	\\
0.422204465752208	-0.001556396484375	\\
0.422248856927243	-0.00146484375	\\
0.422293248102277	-0.001800537109375	\\
0.422337639277312	-0.001861572265625	\\
0.422382030452346	-0.0018310546875	\\
0.42242642162738	-0.0018310546875	\\
0.422470812802415	-0.001983642578125	\\
0.422515203977449	-0.001953125	\\
0.422559595152484	-0.00189208984375	\\
0.422603986327518	-0.001983642578125	\\
0.422648377502553	-0.001861572265625	\\
0.422692768677587	-0.002044677734375	\\
0.422737159852621	-0.00250244140625	\\
0.422781551027656	-0.0020751953125	\\
0.42282594220269	-0.0015869140625	\\
0.422870333377725	-0.00244140625	\\
0.422914724552759	-0.002349853515625	\\
0.422959115727793	-0.00201416015625	\\
0.423003506902828	-0.001708984375	\\
0.423047898077862	-0.001251220703125	\\
0.423092289252897	-0.0018310546875	\\
0.423136680427931	-0.00189208984375	\\
0.423181071602965	-0.001739501953125	\\
0.423225462778	-0.001129150390625	\\
0.423269853953034	-0.0008544921875	\\
0.423314245128069	-0.0010986328125	\\
0.423358636303103	-0.001251220703125	\\
0.423403027478137	-0.00115966796875	\\
0.423447418653172	-0.000885009765625	\\
0.423491809828206	-0.001007080078125	\\
0.423536201003241	-0.001007080078125	\\
0.423580592178275	-0.000732421875	\\
0.423624983353309	-0.0006103515625	\\
0.423669374528344	-0.00042724609375	\\
0.423713765703378	-0.00054931640625	\\
0.423758156878413	-0.00115966796875	\\
0.423802548053447	-0.001312255859375	\\
0.423846939228481	-0.001129150390625	\\
0.423891330403516	-0.0008544921875	\\
0.42393572157855	-0.001068115234375	\\
0.423980112753585	-0.001190185546875	\\
0.424024503928619	-0.00091552734375	\\
0.424068895103653	-0.000518798828125	\\
0.424113286278688	-9.1552734375e-05	\\
0.424157677453722	-0.0003662109375	\\
0.424202068628757	-0.00067138671875	\\
0.424246459803791	9.1552734375e-05	\\
0.424290850978825	0.000518798828125	\\
0.42433524215386	0.000335693359375	\\
0.424379633328894	0.0001220703125	\\
0.424424024503929	0	\\
0.424468415678963	-0.000244140625	\\
0.424512806853997	-0.000396728515625	\\
0.424557198029032	0.0001220703125	\\
0.424601589204066	-0.000213623046875	\\
0.424645980379101	0	\\
0.424690371554135	0.00042724609375	\\
0.424734762729169	0.000579833984375	\\
0.424779153904204	0.00048828125	\\
0.424823545079238	-9.1552734375e-05	\\
0.424867936254273	-0.000274658203125	\\
0.424912327429307	-6.103515625e-05	\\
0.424956718604341	9.1552734375e-05	\\
0.425001109779376	-0.000396728515625	\\
0.42504550095441	-0.00067138671875	\\
0.425089892129445	-0.000335693359375	\\
0.425134283304479	0	\\
0.425178674479513	0.00048828125	\\
0.425223065654548	0.000152587890625	\\
0.425267456829582	-0.000152587890625	\\
0.425311848004617	-0.0001220703125	\\
0.425356239179651	0.000335693359375	\\
0.425400630354686	0.00030517578125	\\
0.42544502152972	-0.00042724609375	\\
0.425489412704754	-0.00048828125	\\
0.425533803879789	-0.000274658203125	\\
0.425578195054823	6.103515625e-05	\\
0.425622586229857	3.0517578125e-05	\\
0.425666977404892	-0.000396728515625	\\
0.425711368579926	-0.0006103515625	\\
0.425755759754961	-0.0003662109375	\\
0.425800150929995	-9.1552734375e-05	\\
0.42584454210503	-0.000244140625	\\
0.425888933280064	-0.000457763671875	\\
0.425933324455098	-0.000457763671875	\\
0.425977715630133	-0.00042724609375	\\
0.426022106805167	-0.000152587890625	\\
0.426066497980202	-0.000457763671875	\\
0.426110889155236	-0.000640869140625	\\
0.42615528033027	-0.00048828125	\\
0.426199671505305	-0.00030517578125	\\
0.426244062680339	-0.000213623046875	\\
0.426288453855374	9.1552734375e-05	\\
0.426332845030408	0.000244140625	\\
0.426377236205442	0.000244140625	\\
0.426421627380477	6.103515625e-05	\\
0.426466018555511	-0.000244140625	\\
0.426510409730546	-6.103515625e-05	\\
0.42655480090558	0.000244140625	\\
0.426599192080614	0.00091552734375	\\
0.426643583255649	0.0009765625	\\
0.426687974430683	0.0009765625	\\
0.426732365605718	0.00128173828125	\\
0.426776756780752	0.000640869140625	\\
0.426821147955786	0.001007080078125	\\
0.426865539130821	0.00140380859375	\\
0.426909930305855	0.00128173828125	\\
0.42695432148089	0.000946044921875	\\
0.426998712655924	0.0008544921875	\\
0.427043103830958	0.001068115234375	\\
0.427087495005993	0.000885009765625	\\
0.427131886181027	0.00091552734375	\\
0.427176277356062	0.00067138671875	\\
0.427220668531096	0.000579833984375	\\
0.42726505970613	0.00079345703125	\\
0.427309450881165	0.000274658203125	\\
0.427353842056199	9.1552734375e-05	\\
0.427398233231234	-9.1552734375e-05	\\
0.427442624406268	-0.000213623046875	\\
0.427487015581302	0.000213623046875	\\
0.427531406756337	-0.000213623046875	\\
0.427575797931371	-0.000518798828125	\\
0.427620189106406	-0.000213623046875	\\
0.42766458028144	0.00018310546875	\\
0.427708971456474	0.00030517578125	\\
0.427753362631509	-0.00030517578125	\\
0.427797753806543	-3.0517578125e-05	\\
0.427842144981578	6.103515625e-05	\\
0.427886536156612	-0.000152587890625	\\
0.427930927331646	0.00054931640625	\\
0.427975318506681	0.000213623046875	\\
0.428019709681715	-0.00030517578125	\\
0.42806410085675	0	\\
0.428108492031784	0.00018310546875	\\
0.428152883206818	0.00054931640625	\\
0.428197274381853	0.000823974609375	\\
0.428241665556887	0.00079345703125	\\
0.428286056731922	0.00054931640625	\\
0.428330447906956	0.0006103515625	\\
0.42837483908199	0.001007080078125	\\
0.428419230257025	0.000946044921875	\\
0.428463621432059	0.000640869140625	\\
0.428508012607094	0.001251220703125	\\
0.428552403782128	0.00103759765625	\\
0.428596794957163	0.000885009765625	\\
0.428641186132197	0.00103759765625	\\
0.428685577307231	0.00140380859375	\\
0.428729968482266	0.001129150390625	\\
0.4287743596573	0.0006103515625	\\
0.428818750832335	0.000885009765625	\\
0.428863142007369	0.000518798828125	\\
0.428907533182403	0.000213623046875	\\
0.428951924357438	-3.0517578125e-05	\\
0.428996315532472	0.00048828125	\\
0.429040706707507	0.000274658203125	\\
0.429085097882541	-0.000244140625	\\
0.429129489057575	0.000274658203125	\\
0.42917388023261	6.103515625e-05	\\
0.429218271407644	0.00018310546875	\\
0.429262662582679	0.000335693359375	\\
0.429307053757713	0.000244140625	\\
0.429351444932747	0.000335693359375	\\
0.429395836107782	9.1552734375e-05	\\
0.429440227282816	0.000396728515625	\\
0.429484618457851	0.000579833984375	\\
0.429529009632885	0.000640869140625	\\
0.429573400807919	0.000701904296875	\\
0.429617791982954	0.0008544921875	\\
0.429662183157988	0.001007080078125	\\
0.429706574333023	0.0003662109375	\\
0.429750965508057	0.000244140625	\\
0.429795356683091	0.000457763671875	\\
0.429839747858126	0.0003662109375	\\
0.42988413903316	0.00042724609375	\\
0.429928530208195	6.103515625e-05	\\
0.429972921383229	-0.00067138671875	\\
0.430017312558263	-0.0006103515625	\\
0.430061703733298	-0.0001220703125	\\
0.430106094908332	-0.000274658203125	\\
0.430150486083367	-0.000274658203125	\\
0.430194877258401	-0.00018310546875	\\
0.430239268433435	-9.1552734375e-05	\\
0.43028365960847	0.00018310546875	\\
0.430328050783504	0.000244140625	\\
0.430372441958539	6.103515625e-05	\\
0.430416833133573	-0.0001220703125	\\
0.430461224308607	-0.000213623046875	\\
0.430505615483642	-6.103515625e-05	\\
0.430550006658676	0	\\
0.430594397833711	-0.0003662109375	\\
0.430638789008745	-0.00018310546875	\\
0.430683180183779	0.000274658203125	\\
0.430727571358814	0	\\
0.430771962533848	0	\\
0.430816353708883	0.00042724609375	\\
0.430860744883917	0.000244140625	\\
0.430905136058951	0.00048828125	\\
0.430949527233986	0.000335693359375	\\
0.43099391840902	0.0001220703125	\\
0.431038309584055	0.000885009765625	\\
0.431082700759089	0.00079345703125	\\
0.431127091934124	0.00067138671875	\\
0.431171483109158	0.00042724609375	\\
0.431215874284192	0.000701904296875	\\
0.431260265459227	0.00079345703125	\\
0.431304656634261	0.0009765625	\\
0.431349047809296	0.000640869140625	\\
0.43139343898433	0.00018310546875	\\
0.431437830159364	0.000640869140625	\\
0.431482221334399	0.00091552734375	\\
0.431526612509433	0.000946044921875	\\
0.431571003684468	0.000640869140625	\\
0.431615394859502	0.00030517578125	\\
0.431659786034536	0.000823974609375	\\
0.431704177209571	0.001617431640625	\\
0.431748568384605	0.001129150390625	\\
0.43179295955964	0.00054931640625	\\
0.431837350734674	0.001129150390625	\\
0.431881741909708	0.001251220703125	\\
0.431926133084743	0.000640869140625	\\
0.431970524259777	0.0008544921875	\\
0.432014915434812	0.0010986328125	\\
0.432059306609846	0.001312255859375	\\
0.43210369778488	0.001556396484375	\\
0.432148088959915	0.001617431640625	\\
0.432192480134949	0.00091552734375	\\
0.432236871309984	0.000396728515625	\\
0.432281262485018	0.00067138671875	\\
0.432325653660052	0.000640869140625	\\
0.432370044835087	0.00054931640625	\\
0.432414436010121	0.00067138671875	\\
0.432458827185156	0.00067138671875	\\
0.43250321836019	0.000885009765625	\\
0.432547609535224	0.000885009765625	\\
0.432592000710259	0.0006103515625	\\
0.432636391885293	0.00079345703125	\\
0.432680783060328	0.000762939453125	\\
0.432725174235362	0.00042724609375	\\
0.432769565410396	0.000701904296875	\\
0.432813956585431	0.001373291015625	\\
0.432858347760465	0.000946044921875	\\
0.4329027389355	0.0008544921875	\\
0.432947130110534	0.000946044921875	\\
0.432991521285568	0.001007080078125	\\
0.433035912460603	0.0008544921875	\\
0.433080303635637	0.000335693359375	\\
0.433124694810672	0.000732421875	\\
0.433169085985706	0.001190185546875	\\
0.43321347716074	0.000640869140625	\\
0.433257868335775	0.0006103515625	\\
0.433302259510809	0.00091552734375	\\
0.433346650685844	0.001190185546875	\\
0.433391041860878	0.001373291015625	\\
0.433435433035912	0.00115966796875	\\
0.433479824210947	0.00140380859375	\\
0.433524215385981	0.001373291015625	\\
0.433568606561016	0.001068115234375	\\
0.43361299773605	0.00054931640625	\\
0.433657388911084	0.001007080078125	\\
0.433701780086119	0.000946044921875	\\
0.433746171261153	0.000701904296875	\\
0.433790562436188	0.000457763671875	\\
0.433834953611222	3.0517578125e-05	\\
0.433879344786257	0.00115966796875	\\
0.433923735961291	0.00128173828125	\\
0.433968127136325	0.00128173828125	\\
0.43401251831136	0.00164794921875	\\
0.434056909486394	0.001739501953125	\\
0.434101300661428	0.00189208984375	\\
0.434145691836463	0.001739501953125	\\
0.434190083011497	0.00128173828125	\\
0.434234474186532	0.001007080078125	\\
0.434278865361566	0.001129150390625	\\
0.434323256536601	0.001708984375	\\
0.434367647711635	0.00177001953125	\\
0.434412038886669	0.0015869140625	\\
0.434456430061704	0.001617431640625	\\
0.434500821236738	0.001861572265625	\\
0.434545212411773	0.002166748046875	\\
0.434589603586807	0.001617431640625	\\
0.434633994761841	0.001739501953125	\\
0.434678385936876	0.002105712890625	\\
0.43472277711191	0.0018310546875	\\
0.434767168286945	0.00244140625	\\
0.434811559461979	0.002197265625	\\
0.434855950637013	0.001953125	\\
0.434900341812048	0.00177001953125	\\
0.434944732987082	0.0015869140625	\\
0.434989124162117	0.0018310546875	\\
0.435033515337151	0.00177001953125	\\
0.435077906512185	0.00146484375	\\
0.43512229768722	0.001312255859375	\\
0.435166688862254	0.001373291015625	\\
0.435211080037289	0.001800537109375	\\
0.435255471212323	0.00146484375	\\
0.435299862387357	0.00146484375	\\
0.435344253562392	0.001220703125	\\
0.435388644737426	0.000946044921875	\\
0.435433035912461	0.00128173828125	\\
0.435477427087495	0.000946044921875	\\
0.435521818262529	0.001220703125	\\
0.435566209437564	0.001495361328125	\\
0.435610600612598	0.00128173828125	\\
0.435654991787633	0.00140380859375	\\
0.435699382962667	0.0015869140625	\\
0.435743774137701	0.001373291015625	\\
0.435788165312736	0.001129150390625	\\
0.43583255648777	0.000579833984375	\\
0.435876947662805	0.0006103515625	\\
0.435921338837839	0.000823974609375	\\
0.435965730012873	0.000946044921875	\\
0.436010121187908	0.000946044921875	\\
0.436054512362942	0.001007080078125	\\
0.436098903537977	0.001220703125	\\
0.436143294713011	0.001068115234375	\\
0.436187685888045	0.000762939453125	\\
0.43623207706308	0.000885009765625	\\
0.436276468238114	0.000701904296875	\\
0.436320859413149	0.00030517578125	\\
0.436365250588183	9.1552734375e-05	\\
0.436409641763217	0.000244140625	\\
0.436454032938252	0.000518798828125	\\
0.436498424113286	0.0003662109375	\\
0.436542815288321	0.000152587890625	\\
0.436587206463355	0.0001220703125	\\
0.436631597638389	0.000335693359375	\\
0.436675988813424	0.000335693359375	\\
0.436720379988458	-0.000152587890625	\\
0.436764771163493	-0.00042724609375	\\
0.436809162338527	-0.0006103515625	\\
0.436853553513561	-0.00054931640625	\\
0.436897944688596	-0.00048828125	\\
0.43694233586363	-0.000732421875	\\
0.436986727038665	-0.000823974609375	\\
0.437031118213699	-0.00054931640625	\\
0.437075509388734	-0.00079345703125	\\
0.437119900563768	-0.00091552734375	\\
0.437164291738802	-0.00067138671875	\\
0.437208682913837	-0.001190185546875	\\
0.437253074088871	-0.001434326171875	\\
0.437297465263906	-0.0015869140625	\\
0.43734185643894	-0.00146484375	\\
0.437386247613974	-0.001007080078125	\\
0.437430638789009	-0.001007080078125	\\
0.437475029964043	-0.000946044921875	\\
0.437519421139078	-0.001129150390625	\\
0.437563812314112	-0.001373291015625	\\
0.437608203489146	-0.0015869140625	\\
0.437652594664181	-0.001220703125	\\
0.437696985839215	-0.001190185546875	\\
0.43774137701425	-0.00164794921875	\\
0.437785768189284	-0.001068115234375	\\
0.437830159364318	-0.000579833984375	\\
0.437874550539353	-0.000518798828125	\\
0.437918941714387	-0.000640869140625	\\
0.437963332889422	-0.00048828125	\\
0.438007724064456	-0.0003662109375	\\
0.43805211523949	-0.00042724609375	\\
0.438096506414525	-0.000335693359375	\\
0.438140897589559	-0.000244140625	\\
0.438185288764594	-0.000396728515625	\\
0.438229679939628	-0.000579833984375	\\
0.438274071114662	-0.0008544921875	\\
0.438318462289697	-0.00115966796875	\\
0.438362853464731	-0.000732421875	\\
0.438407244639766	-0.00030517578125	\\
0.4384516358148	-0.000518798828125	\\
0.438496026989834	-0.000335693359375	\\
0.438540418164869	-0.000274658203125	\\
0.438584809339903	-0.00042724609375	\\
0.438629200514938	-0.00054931640625	\\
0.438673591689972	-0.000701904296875	\\
0.438717982865006	-0.000640869140625	\\
0.438762374040041	-0.000396728515625	\\
0.438806765215075	-0.000274658203125	\\
0.43885115639011	-0.000335693359375	\\
0.438895547565144	-0.00042724609375	\\
0.438939938740178	-0.00048828125	\\
0.438984329915213	-0.000457763671875	\\
0.439028721090247	-3.0517578125e-05	\\
0.439073112265282	-0.00042724609375	\\
0.439117503440316	-0.000640869140625	\\
0.43916189461535	-0.000701904296875	\\
0.439206285790385	-0.000457763671875	\\
0.439250676965419	-9.1552734375e-05	\\
0.439295068140454	-0.000701904296875	\\
0.439339459315488	-0.0010986328125	\\
0.439383850490522	-0.00079345703125	\\
0.439428241665557	-0.000823974609375	\\
0.439472632840591	-0.0010986328125	\\
0.439517024015626	-0.0010986328125	\\
0.43956141519066	-0.001007080078125	\\
0.439605806365695	-0.000823974609375	\\
0.439650197540729	-0.0003662109375	\\
0.439694588715763	-0.00030517578125	\\
0.439738979890798	-0.00091552734375	\\
0.439783371065832	-0.001251220703125	\\
0.439827762240867	-0.001312255859375	\\
0.439872153415901	-0.001495361328125	\\
0.439916544590935	-0.001220703125	\\
0.43996093576597	-0.0015869140625	\\
0.440005326941004	-0.002166748046875	\\
0.440049718116039	-0.001861572265625	\\
0.440094109291073	-0.001556396484375	\\
0.440138500466107	-0.001678466796875	\\
0.440182891641142	-0.002044677734375	\\
0.440227282816176	-0.00177001953125	\\
0.440271673991211	-0.00128173828125	\\
0.440316065166245	-0.0013427734375	\\
0.440360456341279	-0.001617431640625	\\
0.440404847516314	-0.001861572265625	\\
0.440449238691348	-0.00213623046875	\\
0.440493629866383	-0.002166748046875	\\
0.440538021041417	-0.001800537109375	\\
0.440582412216451	-0.001434326171875	\\
0.440626803391486	-0.0018310546875	\\
0.44067119456652	-0.001861572265625	\\
0.440715585741555	-0.00152587890625	\\
0.440759976916589	-0.001190185546875	\\
0.440804368091623	-0.001068115234375	\\
0.440848759266658	-0.00146484375	\\
0.440893150441692	-0.001373291015625	\\
0.440937541616727	-0.001220703125	\\
0.440981932791761	-0.001556396484375	\\
0.441026323966795	-0.001312255859375	\\
0.44107071514183	-0.00128173828125	\\
0.441115106316864	-0.001617431640625	\\
0.441159497491899	-0.0018310546875	\\
0.441203888666933	-0.001708984375	\\
0.441248279841967	-0.00177001953125	\\
0.441292671017002	-0.0015869140625	\\
0.441337062192036	-0.001068115234375	\\
0.441381453367071	-0.001739501953125	\\
0.441425844542105	-0.001739501953125	\\
0.441470235717139	-0.0015869140625	\\
0.441514626892174	-0.002197265625	\\
0.441559018067208	-0.002410888671875	\\
0.441603409242243	-0.00286865234375	\\
0.441647800417277	-0.00299072265625	\\
0.441692191592311	-0.00286865234375	\\
0.441736582767346	-0.00262451171875	\\
0.44178097394238	-0.002166748046875	\\
0.441825365117415	-0.002105712890625	\\
0.441869756292449	-0.0020751953125	\\
0.441914147467483	-0.001922607421875	\\
0.441958538642518	-0.001556396484375	\\
0.442002929817552	-0.00201416015625	\\
0.442047320992587	-0.00238037109375	\\
0.442091712167621	-0.002227783203125	\\
0.442136103342655	-0.00274658203125	\\
0.44218049451769	-0.002716064453125	\\
0.442224885692724	-0.002410888671875	\\
0.442269276867759	-0.002044677734375	\\
0.442313668042793	-0.001739501953125	\\
0.442358059217828	-0.001220703125	\\
0.442402450392862	-0.00091552734375	\\
0.442446841567896	-0.00091552734375	\\
0.442491232742931	-0.00042724609375	\\
0.442535623917965	-0.000518798828125	\\
0.442580015092999	-0.0013427734375	\\
0.442624406268034	-0.00146484375	\\
0.442668797443068	-0.00140380859375	\\
0.442713188618103	-0.001678466796875	\\
0.442757579793137	-0.001708984375	\\
0.442801970968172	-0.001739501953125	\\
0.442846362143206	-0.00152587890625	\\
0.44289075331824	-0.0015869140625	\\
0.442935144493275	-0.0018310546875	\\
0.442979535668309	-0.001861572265625	\\
0.443023926843344	-0.001708984375	\\
0.443068318018378	-0.00164794921875	\\
0.443112709193412	-0.001983642578125	\\
0.443157100368447	-0.001373291015625	\\
0.443201491543481	-0.001251220703125	\\
0.443245882718516	-0.002227783203125	\\
0.44329027389355	-0.001739501953125	\\
0.443334665068584	-0.001556396484375	\\
0.443379056243619	-0.0020751953125	\\
0.443423447418653	-0.001708984375	\\
0.443467838593688	-0.001739501953125	\\
0.443512229768722	-0.0020751953125	\\
0.443556620943756	-0.002288818359375	\\
0.443601012118791	-0.001678466796875	\\
0.443645403293825	-0.001739501953125	\\
0.44368979446886	-0.001739501953125	\\
0.443734185643894	-0.00140380859375	\\
0.443778576818928	-0.001922607421875	\\
0.443822967993963	-0.001678466796875	\\
0.443867359168997	-0.0015869140625	\\
0.443911750344032	-0.002197265625	\\
0.443956141519066	-0.00244140625	\\
0.4440005326941	-0.0023193359375	\\
0.444044923869135	-0.00238037109375	\\
0.444089315044169	-0.00244140625	\\
0.444133706219204	-0.002532958984375	\\
0.444178097394238	-0.002593994140625	\\
0.444222488569272	-0.0020751953125	\\
0.444266879744307	-0.002044677734375	\\
0.444311270919341	-0.00213623046875	\\
0.444355662094376	-0.00201416015625	\\
0.44440005326941	-0.002410888671875	\\
0.444444444444444	-0.002288818359375	\\
0.444488835619479	-0.002197265625	\\
0.444533226794513	-0.002593994140625	\\
0.444577617969548	-0.002655029296875	\\
0.444622009144582	-0.002685546875	\\
0.444666400319616	-0.00299072265625	\\
0.444710791494651	-0.003082275390625	\\
0.444755182669685	-0.002777099609375	\\
0.44479957384472	-0.00286865234375	\\
0.444843965019754	-0.003021240234375	\\
0.444888356194789	-0.002960205078125	\\
0.444932747369823	-0.00286865234375	\\
0.444977138544857	-0.00244140625	\\
0.445021529719892	-0.002471923828125	\\
0.445065920894926	-0.002716064453125	\\
0.44511031206996	-0.002685546875	\\
0.445154703244995	-0.002685546875	\\
0.445199094420029	-0.0028076171875	\\
0.445243485595064	-0.003173828125	\\
0.445287876770098	-0.0030517578125	\\
0.445332267945133	-0.002410888671875	\\
0.445376659120167	-0.002655029296875	\\
0.445421050295201	-0.0023193359375	\\
0.445465441470236	-0.002471923828125	\\
0.44550983264527	-0.002593994140625	\\
0.445554223820305	-0.002777099609375	\\
0.445598614995339	-0.002960205078125	\\
0.445643006170373	-0.002655029296875	\\
0.445687397345408	-0.003448486328125	\\
0.445731788520442	-0.00323486328125	\\
0.445776179695477	-0.002899169921875	\\
0.445820570870511	-0.003021240234375	\\
0.445864962045545	-0.002593994140625	\\
0.44590935322058	-0.002410888671875	\\
0.445953744395614	-0.002288818359375	\\
0.445998135570649	-0.002532958984375	\\
0.446042526745683	-0.00299072265625	\\
0.446086917920717	-0.002960205078125	\\
0.446131309095752	-0.002960205078125	\\
0.446175700270786	-0.003326416015625	\\
0.446220091445821	-0.00323486328125	\\
0.446264482620855	-0.00311279296875	\\
0.446308873795889	-0.00311279296875	\\
0.446353264970924	-0.0029296875	\\
0.446397656145958	-0.002960205078125	\\
0.446442047320993	-0.0028076171875	\\
0.446486438496027	-0.002410888671875	\\
0.446530829671061	-0.002532958984375	\\
0.446575220846096	-0.002716064453125	\\
0.44661961202113	-0.002655029296875	\\
0.446664003196165	-0.0025634765625	\\
0.446708394371199	-0.002349853515625	\\
0.446752785546233	-0.002288818359375	\\
0.446797176721268	-0.002349853515625	\\
0.446841567896302	-0.002044677734375	\\
0.446885959071337	-0.002166748046875	\\
0.446930350246371	-0.00274658203125	\\
0.446974741421405	-0.0023193359375	\\
0.44701913259644	-0.0023193359375	\\
0.447063523771474	-0.002410888671875	\\
0.447107914946509	-0.00225830078125	\\
0.447152306121543	-0.002349853515625	\\
0.447196697296577	-0.002471923828125	\\
0.447241088471612	-0.002838134765625	\\
0.447285479646646	-0.00250244140625	\\
0.447329870821681	-0.002716064453125	\\
0.447374261996715	-0.002593994140625	\\
0.447418653171749	-0.002227783203125	\\
0.447463044346784	-0.00238037109375	\\
0.447507435521818	-0.002288818359375	\\
0.447551826696853	-0.001983642578125	\\
0.447596217871887	-0.00213623046875	\\
0.447640609046921	-0.0020751953125	\\
0.447685000221956	-0.001678466796875	\\
0.44772939139699	-0.00201416015625	\\
0.447773782572025	-0.002471923828125	\\
0.447818173747059	-0.00286865234375	\\
0.447862564922093	-0.0029296875	\\
0.447906956097128	-0.002166748046875	\\
0.447951347272162	-0.00201416015625	\\
0.447995738447197	-0.001922607421875	\\
0.448040129622231	-0.001953125	\\
0.448084520797266	-0.00177001953125	\\
0.4481289119723	-0.001617431640625	\\
0.448173303147334	-0.001708984375	\\
0.448217694322369	-0.002166748046875	\\
0.448262085497403	-0.00262451171875	\\
0.448306476672438	-0.002532958984375	\\
0.448350867847472	-0.002197265625	\\
0.448395259022506	-0.00225830078125	\\
0.448439650197541	-0.00201416015625	\\
0.448484041372575	-0.0015869140625	\\
0.44852843254761	-0.001312255859375	\\
0.448572823722644	-0.001373291015625	\\
0.448617214897678	-0.001495361328125	\\
0.448661606072713	-0.001373291015625	\\
0.448705997247747	-0.001312255859375	\\
0.448750388422782	-0.00164794921875	\\
0.448794779597816	-0.001861572265625	\\
0.44883917077285	-0.001861572265625	\\
0.448883561947885	-0.001739501953125	\\
0.448927953122919	-0.001220703125	\\
0.448972344297954	-0.00146484375	\\
0.449016735472988	-0.001373291015625	\\
0.449061126648022	-0.0008544921875	\\
0.449105517823057	-0.001190185546875	\\
0.449149908998091	-0.001190185546875	\\
0.449194300173126	-0.00103759765625	\\
0.44923869134816	-0.001220703125	\\
0.449283082523194	-0.001129150390625	\\
0.449327473698229	-0.001007080078125	\\
0.449371864873263	-0.0010986328125	\\
0.449416256048298	-0.00115966796875	\\
0.449460647223332	-0.000762939453125	\\
0.449505038398366	-0.001068115234375	\\
0.449549429573401	-0.00115966796875	\\
0.449593820748435	-0.00091552734375	\\
0.44963821192347	-0.001190185546875	\\
0.449682603098504	-0.000885009765625	\\
0.449726994273538	-0.0013427734375	\\
0.449771385448573	-0.00115966796875	\\
0.449815776623607	-0.000885009765625	\\
0.449860167798642	-0.0006103515625	\\
0.449904558973676	-0.000244140625	\\
0.44994895014871	-9.1552734375e-05	\\
0.449993341323745	0.0001220703125	\\
0.450037732498779	0.00030517578125	\\
0.450082123673814	0.0006103515625	\\
0.450126514848848	9.1552734375e-05	\\
0.450170906023882	0	\\
0.450215297198917	0.00030517578125	\\
0.450259688373951	0.000457763671875	\\
0.450304079548986	0.000152587890625	\\
0.45034847072402	9.1552734375e-05	\\
0.450392861899054	0.00042724609375	\\
0.450437253074089	0.000579833984375	\\
0.450481644249123	0.000457763671875	\\
0.450526035424158	0.0003662109375	\\
0.450570426599192	0.00048828125	\\
0.450614817774226	0.00054931640625	\\
0.450659208949261	0.000518798828125	\\
0.450703600124295	0.000335693359375	\\
0.45074799129933	0.0003662109375	\\
0.450792382474364	0.0003662109375	\\
0.450836773649399	-9.1552734375e-05	\\
0.450881164824433	-3.0517578125e-05	\\
0.450925555999467	0.000213623046875	\\
0.450969947174502	0.00042724609375	\\
0.451014338349536	0.000244140625	\\
0.45105872952457	0.000335693359375	\\
0.451103120699605	0.00042724609375	\\
0.451147511874639	0.00048828125	\\
0.451191903049674	0.000244140625	\\
0.451236294224708	0.00042724609375	\\
0.451280685399743	0.00054931640625	\\
0.451325076574777	0.00042724609375	\\
0.451369467749811	0.00048828125	\\
0.451413858924846	0.000244140625	\\
0.45145825009988	0.000457763671875	\\
0.451502641274915	0.00079345703125	\\
0.451547032449949	0.0009765625	\\
0.451591423624983	0.00067138671875	\\
0.451635814800018	0.000518798828125	\\
0.451680205975052	0.000823974609375	\\
0.451724597150087	0.000946044921875	\\
0.451768988325121	0.00091552734375	\\
0.451813379500155	0.00067138671875	\\
0.45185777067519	0.000579833984375	\\
0.451902161850224	0.000640869140625	\\
0.451946553025259	0.000885009765625	\\
0.451990944200293	0.001068115234375	\\
0.452035335375327	0.00128173828125	\\
0.452079726550362	0.001556396484375	\\
0.452124117725396	0.001434326171875	\\
0.452168508900431	0.0015869140625	\\
0.452212900075465	0.0010986328125	\\
0.452257291250499	0.000701904296875	\\
0.452301682425534	0.000823974609375	\\
0.452346073600568	0.000457763671875	\\
0.452390464775603	0.000579833984375	\\
0.452434855950637	0.00079345703125	\\
0.452479247125671	0.000457763671875	\\
0.452523638300706	0.0008544921875	\\
0.45256802947574	0.000762939453125	\\
0.452612420650775	0.00048828125	\\
0.452656811825809	0.000732421875	\\
0.452701203000843	0.000213623046875	\\
0.452745594175878	6.103515625e-05	\\
0.452789985350912	0.0003662109375	\\
0.452834376525947	0.00018310546875	\\
0.452878767700981	0.000213623046875	\\
0.452923158876015	0.00048828125	\\
0.45296755005105	0.000579833984375	\\
0.453011941226084	0.000823974609375	\\
0.453056332401119	0.000640869140625	\\
0.453100723576153	0.000213623046875	\\
0.453145114751187	-0.000244140625	\\
0.453189505926222	-0.0003662109375	\\
0.453233897101256	-0.000457763671875	\\
0.453278288276291	-0.00048828125	\\
0.453322679451325	-0.000244140625	\\
0.45336707062636	-0.000244140625	\\
0.453411461801394	-0.00067138671875	\\
0.453455852976428	-0.00030517578125	\\
0.453500244151463	-0.00042724609375	\\
0.453544635326497	-0.00079345703125	\\
0.453589026501531	-0.001190185546875	\\
0.453633417676566	-0.001708984375	\\
0.4536778088516	-0.001800537109375	\\
0.453722200026635	-0.002044677734375	\\
0.453766591201669	-0.0023193359375	\\
0.453810982376704	-0.002716064453125	\\
0.453855373551738	-0.0029296875	\\
0.453899764726772	-0.00250244140625	\\
0.453944155901807	-0.001617431640625	\\
0.453988547076841	-0.0020751953125	\\
0.454032938251876	-0.002471923828125	\\
0.45407732942691	-0.002349853515625	\\
0.454121720601944	-0.0030517578125	\\
0.454166111776979	-0.003570556640625	\\
0.454210502952013	-0.00421142578125	\\
0.454254894127048	-0.004150390625	\\
0.454299285302082	-0.0037841796875	\\
0.454343676477116	-0.003692626953125	\\
0.454388067652151	-0.00311279296875	\\
0.454432458827185	-0.002410888671875	\\
0.45447685000222	-0.0018310546875	\\
0.454521241177254	-0.002044677734375	\\
0.454565632352288	-0.00213623046875	\\
0.454610023527323	-0.002288818359375	\\
0.454654414702357	-0.002716064453125	\\
0.454698805877392	-0.00274658203125	\\
0.454743197052426	-0.0030517578125	\\
0.45478758822746	-0.0028076171875	\\
0.454831979402495	-0.002716064453125	\\
0.454876370577529	-0.0023193359375	\\
0.454920761752564	-0.002044677734375	\\
0.454965152927598	-0.00164794921875	\\
0.455009544102632	-0.00128173828125	\\
0.455053935277667	-0.001373291015625	\\
0.455098326452701	-0.001129150390625	\\
0.455142717627736	-0.001190185546875	\\
0.45518710880277	-0.001739501953125	\\
0.455231499977804	-0.00201416015625	\\
0.455275891152839	-0.001708984375	\\
0.455320282327873	-0.001617431640625	\\
0.455364673502908	-0.001434326171875	\\
0.455409064677942	-0.000640869140625	\\
0.455453455852976	-0.000152587890625	\\
0.455497847028011	-0.000640869140625	\\
0.455542238203045	-0.000823974609375	\\
0.45558662937808	-0.00067138671875	\\
0.455631020553114	-0.000457763671875	\\
0.455675411728148	-0.000701904296875	\\
0.455719802903183	-0.001129150390625	\\
0.455764194078217	-0.00128173828125	\\
0.455808585253252	-0.001007080078125	\\
0.455852976428286	-0.0006103515625	\\
0.45589736760332	-0.000946044921875	\\
0.455941758778355	-0.000640869140625	\\
0.455986149953389	0.0001220703125	\\
0.456030541128424	-0.000335693359375	\\
0.456074932303458	-0.000640869140625	\\
0.456119323478492	-0.000244140625	\\
0.456163714653527	-0.000213623046875	\\
0.456208105828561	-0.0001220703125	\\
0.456252497003596	-9.1552734375e-05	\\
0.45629688817863	6.103515625e-05	\\
0.456341279353664	-0.000396728515625	\\
0.456385670528699	-0.00054931640625	\\
0.456430061703733	-0.00042724609375	\\
0.456474452878768	-0.00054931640625	\\
0.456518844053802	-0.000457763671875	\\
0.456563235228837	-0.00067138671875	\\
0.456607626403871	-0.00079345703125	\\
0.456652017578905	-0.00091552734375	\\
0.45669640875394	-0.000213623046875	\\
0.456740799928974	-6.103515625e-05	\\
0.456785191104009	-0.000732421875	\\
0.456829582279043	-9.1552734375e-05	\\
0.456873973454077	0.00042724609375	\\
0.456918364629112	0.000335693359375	\\
0.456962755804146	0.00048828125	\\
0.457007146979181	0.00042724609375	\\
0.457051538154215	-0.000213623046875	\\
0.457095929329249	-0.000274658203125	\\
0.457140320504284	-0.000244140625	\\
0.457184711679318	-0.000640869140625	\\
0.457229102854353	-0.000640869140625	\\
0.457273494029387	-0.00042724609375	\\
0.457317885204421	-6.103515625e-05	\\
0.457362276379456	-0.00018310546875	\\
0.45740666755449	0.000335693359375	\\
0.457451058729525	0.00018310546875	\\
0.457495449904559	9.1552734375e-05	\\
0.457539841079593	0.000335693359375	\\
0.457584232254628	0.00018310546875	\\
0.457628623429662	-0.000213623046875	\\
0.457673014604697	-0.000762939453125	\\
0.457717405779731	-0.00067138671875	\\
0.457761796954765	-0.000732421875	\\
0.4578061881298	-0.00042724609375	\\
0.457850579304834	0.000244140625	\\
0.457894970479869	0.000579833984375	\\
0.457939361654903	0.000518798828125	\\
0.457983752829937	0.000244140625	\\
0.458028144004972	0.0003662109375	\\
0.458072535180006	0.000335693359375	\\
0.458116926355041	-0.00042724609375	\\
0.458161317530075	-0.0006103515625	\\
0.458205708705109	-0.000335693359375	\\
0.458250099880144	-0.000701904296875	\\
0.458294491055178	-0.000885009765625	\\
0.458338882230213	-0.001556396484375	\\
0.458383273405247	-0.001007080078125	\\
0.458427664580281	-0.000579833984375	\\
0.458472055755316	-0.0009765625	\\
0.45851644693035	-0.0008544921875	\\
0.458560838105385	-0.0010986328125	\\
0.458605229280419	-0.001373291015625	\\
0.458649620455453	-0.001556396484375	\\
0.458694011630488	-0.001617431640625	\\
0.458738402805522	-0.00189208984375	\\
0.458782793980557	-0.002593994140625	\\
0.458827185155591	-0.002685546875	\\
0.458871576330625	-0.002410888671875	\\
0.45891596750566	-0.002960205078125	\\
0.458960358680694	-0.002471923828125	\\
0.459004749855729	-0.002197265625	\\
0.459049141030763	-0.0020751953125	\\
0.459093532205797	-0.002044677734375	\\
0.459137923380832	-0.00244140625	\\
0.459182314555866	-0.002593994140625	\\
0.459226705730901	-0.002410888671875	\\
0.459271096905935	-0.00274658203125	\\
0.45931548808097	-0.0032958984375	\\
0.459359879256004	-0.003509521484375	\\
0.459404270431038	-0.003021240234375	\\
0.459448661606073	-0.002471923828125	\\
0.459493052781107	-0.002471923828125	\\
0.459537443956142	-0.00244140625	\\
0.459581835131176	-0.002197265625	\\
0.45962622630621	-0.002044677734375	\\
0.459670617481245	-0.00146484375	\\
0.459715008656279	-0.00146484375	\\
0.459759399831314	-0.00238037109375	\\
0.459803791006348	-0.0030517578125	\\
0.459848182181382	-0.002960205078125	\\
0.459892573356417	-0.002777099609375	\\
0.459936964531451	-0.00299072265625	\\
0.459981355706486	-0.00335693359375	\\
0.46002574688152	-0.00372314453125	\\
0.460070138056554	-0.00360107421875	\\
0.460114529231589	-0.002838134765625	\\
0.460158920406623	-0.002899169921875	\\
0.460203311581658	-0.00311279296875	\\
0.460247702756692	-0.003326416015625	\\
0.460292093931726	-0.003204345703125	\\
0.460336485106761	-0.003448486328125	\\
0.460380876281795	-0.00347900390625	\\
0.46042526745683	-0.00360107421875	\\
0.460469658631864	-0.00347900390625	\\
0.460514049806898	-0.00335693359375	\\
0.460558440981933	-0.00372314453125	\\
0.460602832156967	-0.004119873046875	\\
0.460647223332002	-0.00408935546875	\\
0.460691614507036	-0.00384521484375	\\
0.46073600568207	-0.003448486328125	\\
0.460780396857105	-0.003875732421875	\\
0.460824788032139	-0.003662109375	\\
0.460869179207174	-0.003082275390625	\\
0.460913570382208	-0.003173828125	\\
0.460957961557242	-0.00323486328125	\\
0.461002352732277	-0.003173828125	\\
0.461046743907311	-0.003173828125	\\
0.461091135082346	-0.00323486328125	\\
0.46113552625738	-0.0035400390625	\\
0.461179917432414	-0.003753662109375	\\
0.461224308607449	-0.003509521484375	\\
0.461268699782483	-0.00311279296875	\\
0.461313090957518	-0.00262451171875	\\
0.461357482132552	-0.002105712890625	\\
0.461401873307586	-0.001312255859375	\\
0.461446264482621	-0.00115966796875	\\
0.461490655657655	-0.001434326171875	\\
0.46153504683269	-0.001007080078125	\\
0.461579438007724	-0.001434326171875	\\
0.461623829182758	-0.00164794921875	\\
0.461668220357793	-0.00164794921875	\\
0.461712611532827	-0.00201416015625	\\
0.461757002707862	-0.0018310546875	\\
0.461801393882896	-0.001678466796875	\\
0.461845785057931	-0.00103759765625	\\
0.461890176232965	-0.00115966796875	\\
0.461934567407999	-0.00152587890625	\\
0.461978958583034	-0.001068115234375	\\
0.462023349758068	-0.001068115234375	\\
0.462067740933102	-0.001312255859375	\\
0.462112132108137	-0.001800537109375	\\
0.462156523283171	-0.001708984375	\\
0.462200914458206	-0.00140380859375	\\
0.46224530563324	-0.001922607421875	\\
0.462289696808275	-0.001922607421875	\\
0.462334087983309	-0.00146484375	\\
0.462378479158343	-0.002105712890625	\\
0.462422870333378	-0.002349853515625	\\
0.462467261508412	-0.0020751953125	\\
0.462511652683447	-0.002410888671875	\\
0.462556043858481	-0.002532958984375	\\
0.462600435033515	-0.0029296875	\\
0.46264482620855	-0.002716064453125	\\
0.462689217383584	-0.0023193359375	\\
0.462733608558619	-0.001953125	\\
0.462777999733653	-0.0020751953125	\\
0.462822390908687	-0.00152587890625	\\
0.462866782083722	-0.000885009765625	\\
0.462911173258756	-0.00164794921875	\\
0.462955564433791	-0.001861572265625	\\
0.462999955608825	-0.001708984375	\\
0.463044346783859	-0.002197265625	\\
0.463088737958894	-0.002044677734375	\\
0.463133129133928	-0.001708984375	\\
0.463177520308963	-0.001953125	\\
0.463221911483997	-0.00177001953125	\\
0.463266302659031	-0.001983642578125	\\
0.463310693834066	-0.001800537109375	\\
0.4633550850091	-0.002166748046875	\\
0.463399476184135	-0.00244140625	\\
0.463443867359169	-0.0029296875	\\
0.463488258534203	-0.00341796875	\\
0.463532649709238	-0.003143310546875	\\
0.463577040884272	-0.003265380859375	\\
0.463621432059307	-0.00347900390625	\\
0.463665823234341	-0.003326416015625	\\
0.463710214409375	-0.003509521484375	\\
0.46375460558441	-0.00372314453125	\\
0.463798996759444	-0.003448486328125	\\
0.463843387934479	-0.003448486328125	\\
0.463887779109513	-0.00360107421875	\\
0.463932170284547	-0.003326416015625	\\
0.463976561459582	-0.003387451171875	\\
0.464020952634616	-0.003692626953125	\\
0.464065343809651	-0.003936767578125	\\
0.464109734984685	-0.004302978515625	\\
0.464154126159719	-0.004241943359375	\\
0.464198517334754	-0.004058837890625	\\
0.464242908509788	-0.00372314453125	\\
0.464287299684823	-0.00341796875	\\
0.464331690859857	-0.003570556640625	\\
0.464376082034891	-0.00335693359375	\\
0.464420473209926	-0.00299072265625	\\
0.46446486438496	-0.003021240234375	\\
0.464509255559995	-0.002960205078125	\\
0.464553646735029	-0.0029296875	\\
0.464598037910063	-0.003326416015625	\\
0.464642429085098	-0.003692626953125	\\
0.464686820260132	-0.00396728515625	\\
0.464731211435167	-0.00390625	\\
0.464775602610201	-0.003448486328125	\\
0.464819993785235	-0.0035400390625	\\
0.46486438496027	-0.0035400390625	\\
0.464908776135304	-0.003265380859375	\\
0.464953167310339	-0.003662109375	\\
0.464997558485373	-0.003692626953125	\\
0.465041949660408	-0.003631591796875	\\
0.465086340835442	-0.0037841796875	\\
0.465130732010476	-0.004150390625	\\
0.465175123185511	-0.004119873046875	\\
0.465219514360545	-0.00439453125	\\
0.46526390553558	-0.005218505859375	\\
0.465308296710614	-0.005462646484375	\\
0.465352687885648	-0.00531005859375	\\
0.465397079060683	-0.005615234375	\\
0.465441470235717	-0.0057373046875	\\
0.465485861410752	-0.005401611328125	\\
0.465530252585786	-0.005035400390625	\\
0.46557464376082	-0.004608154296875	\\
0.465619034935855	-0.004547119140625	\\
0.465663426110889	-0.004364013671875	\\
0.465707817285924	-0.00482177734375	\\
0.465752208460958	-0.00518798828125	\\
0.465796599635992	-0.0050048828125	\\
0.465840990811027	-0.004608154296875	\\
0.465885381986061	-0.004180908203125	\\
0.465929773161096	-0.004302978515625	\\
0.46597416433613	-0.00390625	\\
0.466018555511164	-0.00341796875	\\
0.466062946686199	-0.002777099609375	\\
0.466107337861233	-0.00244140625	\\
0.466151729036268	-0.002410888671875	\\
0.466196120211302	-0.002471923828125	\\
0.466240511386336	-0.002777099609375	\\
0.466284902561371	-0.0032958984375	\\
0.466329293736405	-0.0035400390625	\\
0.46637368491144	-0.00341796875	\\
0.466418076086474	-0.00390625	\\
0.466462467261508	-0.00396728515625	\\
0.466506858436543	-0.003570556640625	\\
0.466551249611577	-0.0032958984375	\\
0.466595640786612	-0.0032958984375	\\
0.466640031961646	-0.003509521484375	\\
0.46668442313668	-0.003692626953125	\\
0.466728814311715	-0.003936767578125	\\
0.466773205486749	-0.004058837890625	\\
0.466817596661784	-0.003875732421875	\\
0.466861987836818	-0.00408935546875	\\
0.466906379011852	-0.00390625	\\
0.466950770186887	-0.00433349609375	\\
0.466995161361921	-0.00439453125	\\
0.467039552536956	-0.00421142578125	\\
0.46708394371199	-0.004486083984375	\\
0.467128334887024	-0.003875732421875	\\
0.467172726062059	-0.003875732421875	\\
0.467217117237093	-0.00390625	\\
0.467261508412128	-0.00286865234375	\\
0.467305899587162	-0.002685546875	\\
0.467350290762196	-0.002716064453125	\\
0.467394681937231	-0.002532958984375	\\
0.467439073112265	-0.00238037109375	\\
0.4674834642873	-0.00177001953125	\\
0.467527855462334	-0.00164794921875	\\
0.467572246637369	-0.001739501953125	\\
0.467616637812403	-0.00201416015625	\\
0.467661028987437	-0.001800537109375	\\
0.467705420162472	-0.001495361328125	\\
0.467749811337506	-0.00128173828125	\\
0.467794202512541	-0.000823974609375	\\
0.467838593687575	-0.00018310546875	\\
0.467882984862609	3.0517578125e-05	\\
0.467927376037644	3.0517578125e-05	\\
0.467971767212678	0	\\
0.468016158387713	-0.000244140625	\\
0.468060549562747	-0.0003662109375	\\
0.468104940737781	-0.0003662109375	\\
0.468149331912816	-0.000640869140625	\\
0.46819372308785	-0.0009765625	\\
0.468238114262885	-0.000762939453125	\\
0.468282505437919	-0.000823974609375	\\
0.468326896612953	-0.000457763671875	\\
0.468371287787988	-0.00018310546875	\\
0.468415678963022	-0.000518798828125	\\
0.468460070138057	-0.0006103515625	\\
0.468504461313091	-0.000762939453125	\\
0.468548852488125	-0.0010986328125	\\
0.46859324366316	-0.001739501953125	\\
0.468637634838194	-0.00177001953125	\\
0.468682026013229	-0.001617431640625	\\
0.468726417188263	-0.001556396484375	\\
0.468770808363297	-0.001220703125	\\
0.468815199538332	-0.0010986328125	\\
0.468859590713366	-0.0006103515625	\\
0.468903981888401	-0.00030517578125	\\
0.468948373063435	-0.000823974609375	\\
0.468992764238469	-0.00152587890625	\\
0.469037155413504	-0.001922607421875	\\
0.469081546588538	-0.00213623046875	\\
0.469125937763573	-0.00225830078125	\\
0.469170328938607	-0.001708984375	\\
0.469214720113641	-0.001312255859375	\\
0.469259111288676	-0.000946044921875	\\
0.46930350246371	-0.000732421875	\\
0.469347893638745	-0.000579833984375	\\
0.469392284813779	3.0517578125e-05	\\
0.469436675988813	3.0517578125e-05	\\
0.469481067163848	-6.103515625e-05	\\
0.469525458338882	-0.00030517578125	\\
0.469569849513917	-0.000732421875	\\
0.469614240688951	-0.000640869140625	\\
0.469658631863985	-0.0006103515625	\\
0.46970302303902	-0.0003662109375	\\
0.469747414214054	0.0001220703125	\\
0.469791805389089	-0.0003662109375	\\
0.469836196564123	-0.000762939453125	\\
0.469880587739157	-0.000457763671875	\\
0.469924978914192	-0.0003662109375	\\
0.469969370089226	-0.000640869140625	\\
0.470013761264261	-0.0010986328125	\\
0.470058152439295	-0.001617431640625	\\
0.470102543614329	-0.00152587890625	\\
0.470146934789364	-0.001373291015625	\\
0.470191325964398	-0.001617431640625	\\
0.470235717139433	-0.001922607421875	\\
0.470280108314467	-0.001220703125	\\
0.470324499489502	-0.0010986328125	\\
0.470368890664536	-0.0013427734375	\\
0.47041328183957	-0.00091552734375	\\
0.470457673014605	-0.00103759765625	\\
0.470502064189639	-0.001556396484375	\\
0.470546455364673	-0.001922607421875	\\
0.470590846539708	-0.00213623046875	\\
0.470635237714742	-0.002349853515625	\\
0.470679628889777	-0.001861572265625	\\
0.470724020064811	-0.001800537109375	\\
0.470768411239846	-0.001983642578125	\\
0.47081280241488	-0.001800537109375	\\
0.470857193589914	-0.001922607421875	\\
0.470901584764949	-0.0020751953125	\\
0.470945975939983	-0.0018310546875	\\
0.470990367115018	-0.00189208984375	\\
0.471034758290052	-0.001678466796875	\\
0.471079149465086	-0.00189208984375	\\
0.471123540640121	-0.00250244140625	\\
0.471167931815155	-0.00250244140625	\\
0.47121232299019	-0.00250244140625	\\
0.471256714165224	-0.002777099609375	\\
0.471301105340258	-0.002471923828125	\\
0.471345496515293	-0.00299072265625	\\
0.471389887690327	-0.003509521484375	\\
0.471434278865362	-0.003326416015625	\\
0.471478670040396	-0.0029296875	\\
0.47152306121543	-0.003143310546875	\\
0.471567452390465	-0.002960205078125	\\
0.471611843565499	-0.003143310546875	\\
0.471656234740534	-0.003387451171875	\\
0.471700625915568	-0.00390625	\\
0.471745017090602	-0.004364013671875	\\
0.471789408265637	-0.00457763671875	\\
0.471833799440671	-0.004791259765625	\\
0.471878190615706	-0.00457763671875	\\
0.47192258179074	-0.0042724609375	\\
0.471966972965774	-0.003570556640625	\\
0.472011364140809	-0.003082275390625	\\
0.472055755315843	-0.002685546875	\\
0.472100146490878	-0.002471923828125	\\
0.472144537665912	-0.00225830078125	\\
0.472188928840946	-0.0018310546875	\\
0.472233320015981	-0.00189208984375	\\
0.472277711191015	-0.00244140625	\\
0.47232210236605	-0.002227783203125	\\
0.472366493541084	-0.001739501953125	\\
0.472410884716118	-0.0020751953125	\\
0.472455275891153	-0.0013427734375	\\
0.472499667066187	-0.001251220703125	\\
0.472544058241222	-0.00115966796875	\\
0.472588449416256	-0.000885009765625	\\
0.47263284059129	-0.000823974609375	\\
0.472677231766325	-0.000823974609375	\\
0.472721622941359	-0.000823974609375	\\
0.472766014116394	-0.000762939453125	\\
0.472810405291428	-0.001068115234375	\\
0.472854796466462	-0.001220703125	\\
0.472899187641497	-0.001068115234375	\\
0.472943578816531	-0.001129150390625	\\
0.472987969991566	-0.001007080078125	\\
0.4730323611666	-0.000946044921875	\\
0.473076752341634	-0.001007080078125	\\
0.473121143516669	-0.000946044921875	\\
0.473165534691703	-0.00079345703125	\\
0.473209925866738	-0.000823974609375	\\
0.473254317041772	-0.00091552734375	\\
0.473298708216806	-0.001556396484375	\\
0.473343099391841	-0.0020751953125	\\
0.473387490566875	-0.002166748046875	\\
0.47343188174191	-0.002288818359375	\\
0.473476272916944	-0.001983642578125	\\
0.473520664091979	-0.001312255859375	\\
0.473565055267013	-0.00091552734375	\\
0.473609446442047	-0.000640869140625	\\
0.473653837617082	-9.1552734375e-05	\\
0.473698228792116	0	\\
0.473742619967151	6.103515625e-05	\\
0.473787011142185	0.00030517578125	\\
0.473831402317219	0.000213623046875	\\
0.473875793492254	-0.000213623046875	\\
0.473920184667288	-0.00018310546875	\\
0.473964575842323	-6.103515625e-05	\\
0.474008967017357	-0.0006103515625	\\
0.474053358192391	-0.000213623046875	\\
0.474097749367426	0.00042724609375	\\
0.47414214054246	0.00042724609375	\\
0.474186531717495	0.000701904296875	\\
0.474230922892529	0.001312255859375	\\
0.474275314067563	0.0018310546875	\\
0.474319705242598	0.0020751953125	\\
0.474364096417632	0.00152587890625	\\
0.474408487592667	0.00152587890625	\\
0.474452878767701	0.001373291015625	\\
0.474497269942735	0.000823974609375	\\
0.47454166111777	0.000457763671875	\\
0.474586052292804	-9.1552734375e-05	\\
0.474630443467839	0.000244140625	\\
0.474674834642873	0.000152587890625	\\
0.474719225817907	0.00042724609375	\\
0.474763616992942	0.00079345703125	\\
0.474808008167976	0.00067138671875	\\
0.474852399343011	0.000701904296875	\\
0.474896790518045	0.00103759765625	\\
0.474941181693079	0.000823974609375	\\
0.474985572868114	0.000457763671875	\\
0.475029964043148	-3.0517578125e-05	\\
0.475074355218183	-0.00042724609375	\\
0.475118746393217	-0.000152587890625	\\
0.475163137568251	3.0517578125e-05	\\
0.475207528743286	0.000457763671875	\\
0.47525191991832	0.0003662109375	\\
0.475296311093355	0.000213623046875	\\
0.475340702268389	6.103515625e-05	\\
0.475385093443423	0.000213623046875	\\
0.475429484618458	0.00067138671875	\\
0.475473875793492	0.000274658203125	\\
0.475518266968527	0.000213623046875	\\
0.475562658143561	0.0003662109375	\\
0.475607049318595	0.000701904296875	\\
0.47565144049363	0.001007080078125	\\
0.475695831668664	0.00103759765625	\\
0.475740222843699	0.001617431640625	\\
0.475784614018733	0.001861572265625	\\
0.475829005193767	0.00164794921875	\\
0.475873396368802	0.001739501953125	\\
0.475917787543836	0.001556396484375	\\
0.475962178718871	0.0013427734375	\\
0.476006569893905	0.001129150390625	\\
0.47605096106894	0.00103759765625	\\
0.476095352243974	0.000823974609375	\\
0.476139743419008	0.000396728515625	\\
0.476184134594043	0.000518798828125	\\
0.476228525769077	0.00048828125	\\
0.476272916944112	0.0003662109375	\\
0.476317308119146	0.0003662109375	\\
0.47636169929418	0.000152587890625	\\
0.476406090469215	-0.000335693359375	\\
0.476450481644249	-0.000732421875	\\
0.476494872819284	-0.000732421875	\\
0.476539263994318	-0.000579833984375	\\
0.476583655169352	-0.00079345703125	\\
0.476628046344387	-0.000762939453125	\\
0.476672437519421	-0.0006103515625	\\
0.476716828694456	-0.001129150390625	\\
0.47676121986949	-0.00146484375	\\
0.476805611044524	-0.00152587890625	\\
0.476850002219559	-0.001068115234375	\\
0.476894393394593	-0.001312255859375	\\
0.476938784569628	-0.0013427734375	\\
0.476983175744662	-0.00140380859375	\\
0.477027566919696	-0.00164794921875	\\
0.477071958094731	-0.001678466796875	\\
0.477116349269765	-0.002197265625	\\
0.4771607404448	-0.00177001953125	\\
0.477205131619834	-0.00177001953125	\\
0.477249522794868	-0.001678466796875	\\
0.477293913969903	-0.00128173828125	\\
0.477338305144937	-0.00140380859375	\\
0.477382696319972	-0.00152587890625	\\
0.477427087495006	-0.001220703125	\\
0.47747147867004	-0.0013427734375	\\
0.477515869845075	-0.0015869140625	\\
0.477560261020109	-0.00213623046875	\\
0.477604652195144	-0.002288818359375	\\
0.477649043370178	-0.001983642578125	\\
0.477693434545212	-0.002349853515625	\\
0.477737825720247	-0.00262451171875	\\
0.477782216895281	-0.0023193359375	\\
0.477826608070316	-0.0020751953125	\\
0.47787099924535	-0.002044677734375	\\
0.477915390420384	-0.001983642578125	\\
0.477959781595419	-0.001983642578125	\\
0.478004172770453	-0.00238037109375	\\
0.478048563945488	-0.00286865234375	\\
0.478092955120522	-0.003082275390625	\\
0.478137346295556	-0.003509521484375	\\
0.478181737470591	-0.0037841796875	\\
0.478226128645625	-0.00396728515625	\\
0.47827051982066	-0.003875732421875	\\
0.478314910995694	-0.003631591796875	\\
0.478359302170728	-0.002593994140625	\\
0.478403693345763	-0.0023193359375	\\
0.478448084520797	-0.002288818359375	\\
0.478492475695832	-0.001556396484375	\\
0.478536866870866	-0.001220703125	\\
0.4785812580459	-0.0010986328125	\\
0.478625649220935	-0.001220703125	\\
0.478670040395969	-0.00146484375	\\
0.478714431571004	-0.002227783203125	\\
0.478758822746038	-0.002227783203125	\\
0.478803213921073	-0.002166748046875	\\
0.478847605096107	-0.001953125	\\
0.478891996271141	-0.001556396484375	\\
0.478936387446176	-0.00146484375	\\
0.47898077862121	-0.001007080078125	\\
0.479025169796244	-0.000732421875	\\
0.479069560971279	-0.000457763671875	\\
0.479113952146313	-0.0003662109375	\\
0.479158343321348	-0.000823974609375	\\
0.479202734496382	-0.001220703125	\\
0.479247125671417	-0.00140380859375	\\
0.479291516846451	-0.001495361328125	\\
0.479335908021485	-0.00146484375	\\
0.47938029919652	-0.001434326171875	\\
0.479424690371554	-0.001220703125	\\
0.479469081546589	-0.001129150390625	\\
0.479513472721623	-0.00146484375	\\
0.479557863896657	-0.001617431640625	\\
0.479602255071692	-0.0013427734375	\\
0.479646646246726	-0.001190185546875	\\
0.479691037421761	-0.00140380859375	\\
0.479735428596795	-0.00164794921875	\\
0.479779819771829	-0.002044677734375	\\
0.479824210946864	-0.001953125	\\
0.479868602121898	-0.00164794921875	\\
0.479912993296933	-0.001800537109375	\\
0.479957384471967	-0.001953125	\\
0.480001775647001	-0.002166748046875	\\
0.480046166822036	-0.002044677734375	\\
0.48009055799707	-0.001800537109375	\\
0.480134949172105	-0.001434326171875	\\
0.480179340347139	-0.001373291015625	\\
0.480223731522173	-0.00164794921875	\\
0.480268122697208	-0.001434326171875	\\
0.480312513872242	-0.001678466796875	\\
0.480356905047277	-0.001922607421875	\\
0.480401296222311	-0.002349853515625	\\
0.480445687397345	-0.002593994140625	\\
0.48049007857238	-0.002685546875	\\
0.480534469747414	-0.00262451171875	\\
0.480578860922449	-0.002227783203125	\\
0.480623252097483	-0.00189208984375	\\
0.480667643272517	-0.001495361328125	\\
0.480712034447552	-0.00140380859375	\\
0.480756425622586	-0.0010986328125	\\
0.480800816797621	-0.00103759765625	\\
0.480845207972655	-0.001739501953125	\\
0.480889599147689	-0.001708984375	\\
0.480933990322724	-0.001983642578125	\\
0.480978381497758	-0.002227783203125	\\
0.481022772672793	-0.00244140625	\\
0.481067163847827	-0.00238037109375	\\
0.481111555022861	-0.001678466796875	\\
0.481155946197896	-0.00152587890625	\\
0.48120033737293	-0.000732421875	\\
0.481244728547965	0	\\
0.481289119722999	-0.00018310546875	\\
0.481333510898033	-0.000152587890625	\\
0.481377902073068	-0.000274658203125	\\
0.481422293248102	-0.000732421875	\\
0.481466684423137	-0.001220703125	\\
0.481511075598171	-0.001708984375	\\
0.481555466773205	-0.001495361328125	\\
0.48159985794824	-0.0010986328125	\\
0.481644249123274	-0.001007080078125	\\
0.481688640298309	-0.000579833984375	\\
0.481733031473343	0	\\
0.481777422648378	3.0517578125e-05	\\
0.481821813823412	-0.000213623046875	\\
0.481866204998446	-0.0001220703125	\\
0.481910596173481	-0.00048828125	\\
0.481954987348515	-0.00042724609375	\\
0.48199937852355	-0.000579833984375	\\
0.482043769698584	-0.00018310546875	\\
0.482088160873618	-0.00018310546875	\\
0.482132552048653	-0.000335693359375	\\
0.482176943223687	-9.1552734375e-05	\\
0.482221334398722	-0.0001220703125	\\
0.482265725573756	0.000213623046875	\\
0.48231011674879	0.000274658203125	\\
0.482354507923825	6.103515625e-05	\\
0.482398899098859	6.103515625e-05	\\
0.482443290273894	-0.00018310546875	\\
0.482487681448928	-0.00079345703125	\\
0.482532072623962	-0.00140380859375	\\
0.482576463798997	-0.00115966796875	\\
0.482620854974031	-0.001220703125	\\
0.482665246149066	-0.001190185546875	\\
0.4827096373241	-0.0008544921875	\\
0.482754028499134	-0.000885009765625	\\
0.482798419674169	-0.00079345703125	\\
0.482842810849203	-0.000823974609375	\\
0.482887202024238	-0.001007080078125	\\
0.482931593199272	-0.0010986328125	\\
0.482975984374306	-0.001556396484375	\\
0.483020375549341	-0.00146484375	\\
0.483064766724375	-0.0010986328125	\\
0.48310915789941	-0.001434326171875	\\
0.483153549074444	-0.001953125	\\
0.483197940249478	-0.00177001953125	\\
0.483242331424513	-0.0018310546875	\\
0.483286722599547	-0.001495361328125	\\
0.483331113774582	-0.00177001953125	\\
0.483375504949616	-0.001708984375	\\
0.48341989612465	-0.001922607421875	\\
0.483464287299685	-0.002349853515625	\\
0.483508678474719	-0.002197265625	\\
0.483553069649754	-0.003082275390625	\\
0.483597460824788	-0.003326416015625	\\
0.483641851999822	-0.00347900390625	\\
0.483686243174857	-0.003265380859375	\\
0.483730634349891	-0.003204345703125	\\
0.483775025524926	-0.003265380859375	\\
0.48381941669996	-0.0030517578125	\\
0.483863807874994	-0.00286865234375	\\
0.483908199050029	-0.002838134765625	\\
0.483952590225063	-0.003631591796875	\\
0.483996981400098	-0.00408935546875	\\
0.484041372575132	-0.004180908203125	\\
0.484085763750166	-0.0045166015625	\\
0.484130154925201	-0.0045166015625	\\
0.484174546100235	-0.004119873046875	\\
0.48421893727527	-0.00396728515625	\\
0.484263328450304	-0.003936767578125	\\
0.484307719625338	-0.00335693359375	\\
0.484352110800373	-0.003082275390625	\\
0.484396501975407	-0.003082275390625	\\
0.484440893150442	-0.003082275390625	\\
0.484485284325476	-0.00323486328125	\\
0.484529675500511	-0.003326416015625	\\
0.484574066675545	-0.003387451171875	\\
0.484618457850579	-0.00299072265625	\\
0.484662849025614	-0.00250244140625	\\
0.484707240200648	-0.00274658203125	\\
0.484751631375683	-0.0023193359375	\\
0.484796022550717	-0.001495361328125	\\
0.484840413725751	-0.001251220703125	\\
0.484884804900786	-0.000823974609375	\\
0.48492919607582	-0.00079345703125	\\
0.484973587250855	-0.000579833984375	\\
0.485017978425889	-0.00115966796875	\\
0.485062369600923	-0.001861572265625	\\
0.485106760775958	-0.00177001953125	\\
0.485151151950992	-0.00152587890625	\\
0.485195543126027	-0.001373291015625	\\
0.485239934301061	-0.00128173828125	\\
0.485284325476095	-0.000762939453125	\\
0.48532871665113	-0.0008544921875	\\
0.485373107826164	-0.000732421875	\\
0.485417499001199	-0.000518798828125	\\
0.485461890176233	-0.000579833984375	\\
0.485506281351267	-0.0008544921875	\\
0.485550672526302	-0.001708984375	\\
0.485595063701336	-0.001953125	\\
0.485639454876371	-0.001678466796875	\\
0.485683846051405	-0.00140380859375	\\
0.485728237226439	-0.001495361328125	\\
0.485772628401474	-0.001312255859375	\\
0.485817019576508	-0.00103759765625	\\
0.485861410751543	-0.001251220703125	\\
0.485905801926577	-0.001220703125	\\
0.485950193101611	-0.00115966796875	\\
0.485994584276646	-0.001373291015625	\\
0.48603897545168	-0.00079345703125	\\
0.486083366626715	-0.00146484375	\\
0.486127757801749	-0.002197265625	\\
0.486172148976783	-0.002044677734375	\\
0.486216540151818	-0.001739501953125	\\
0.486260931326852	-0.002105712890625	\\
0.486305322501887	-0.002593994140625	\\
0.486349713676921	-0.002838134765625	\\
0.486394104851955	-0.003143310546875	\\
0.48643849602699	-0.003326416015625	\\
0.486482887202024	-0.002716064453125	\\
0.486527278377059	-0.002593994140625	\\
0.486571669552093	-0.0023193359375	\\
0.486616060727127	-0.00177001953125	\\
0.486660451902162	-0.00189208984375	\\
0.486704843077196	-0.001434326171875	\\
0.486749234252231	-0.00115966796875	\\
0.486793625427265	-0.001373291015625	\\
0.486838016602299	-0.0015869140625	\\
0.486882407777334	-0.002410888671875	\\
0.486926798952368	-0.00286865234375	\\
0.486971190127403	-0.002777099609375	\\
0.487015581302437	-0.002288818359375	\\
0.487059972477471	-0.0015869140625	\\
0.487104363652506	-0.00091552734375	\\
0.48714875482754	-0.000885009765625	\\
0.487193146002575	-0.00030517578125	\\
0.487237537177609	9.1552734375e-05	\\
0.487281928352644	-0.000823974609375	\\
0.487326319527678	-0.000946044921875	\\
0.487370710702712	-0.001068115234375	\\
0.487415101877747	-0.00164794921875	\\
0.487459493052781	-0.00128173828125	\\
0.487503884227815	-0.000823974609375	\\
0.48754827540285	-0.000244140625	\\
0.487592666577884	0.000457763671875	\\
0.487637057752919	0.00054931640625	\\
0.487681448927953	0.00091552734375	\\
0.487725840102988	0.001495361328125	\\
0.487770231278022	0.00177001953125	\\
0.487814622453056	0.0013427734375	\\
0.487859013628091	0.0010986328125	\\
0.487903404803125	0.000762939453125	\\
0.48794779597816	0.000579833984375	\\
0.487992187153194	0.0008544921875	\\
0.488036578328228	0.00054931640625	\\
0.488080969503263	0.0008544921875	\\
0.488125360678297	0.001556396484375	\\
0.488169751853332	0.002288818359375	\\
0.488214143028366	0.002166748046875	\\
0.4882585342034	0.0020751953125	\\
0.488302925378435	0.001800537109375	\\
0.488347316553469	0.001220703125	\\
0.488391707728504	0.000640869140625	\\
0.488436098903538	0.000396728515625	\\
0.488480490078572	3.0517578125e-05	\\
0.488524881253607	-0.0003662109375	\\
0.488569272428641	-0.000152587890625	\\
0.488613663603676	-0.00018310546875	\\
0.48865805477871	3.0517578125e-05	\\
0.488702445953744	0.00030517578125	\\
0.488746837128779	0.000213623046875	\\
0.488791228303813	-0.000152587890625	\\
0.488835619478848	-0.0003662109375	\\
0.488880010653882	-0.000640869140625	\\
0.488924401828916	-0.0003662109375	\\
0.488968793003951	-0.000732421875	\\
0.489013184178985	-0.0009765625	\\
0.48905757535402	-0.0010986328125	\\
0.489101966529054	-0.000946044921875	\\
0.489146357704088	-0.000946044921875	\\
0.489190748879123	-0.0010986328125	\\
0.489235140054157	-0.00091552734375	\\
0.489279531229192	-0.00079345703125	\\
0.489323922404226	-0.000640869140625	\\
0.48936831357926	-0.00067138671875	\\
0.489412704754295	-0.00115966796875	\\
0.489457095929329	-0.00152587890625	\\
0.489501487104364	-0.00146484375	\\
0.489545878279398	-0.0018310546875	\\
0.489590269454432	-0.001434326171875	\\
0.489634660629467	-0.000701904296875	\\
0.489679051804501	-0.000152587890625	\\
0.489723442979536	-0.000335693359375	\\
0.48976783415457	-3.0517578125e-05	\\
0.489812225329605	0.000640869140625	\\
0.489856616504639	-0.000335693359375	\\
0.489901007679673	-0.000732421875	\\
0.489945398854708	-0.0013427734375	\\
0.489989790029742	-0.001953125	\\
0.490034181204776	-0.001983642578125	\\
0.490078572379811	-0.0023193359375	\\
0.490122963554845	-0.002166748046875	\\
0.49016735472988	-0.001678466796875	\\
0.490211745904914	-0.001617431640625	\\
0.490256137079949	-0.001068115234375	\\
0.490300528254983	-0.000946044921875	\\
0.490344919430017	-0.00103759765625	\\
0.490389310605052	-0.001251220703125	\\
0.490433701780086	-0.0020751953125	\\
0.490478092955121	-0.00225830078125	\\
0.490522484130155	-0.001800537109375	\\
0.490566875305189	-0.00177001953125	\\
0.490611266480224	-0.001312255859375	\\
0.490655657655258	-0.00067138671875	\\
0.490700048830293	9.1552734375e-05	\\
0.490744440005327	0.00042724609375	\\
0.490788831180361	0.001007080078125	\\
0.490833222355396	0.00152587890625	\\
0.49087761353043	0.001220703125	\\
0.490922004705465	0.00140380859375	\\
0.490966395880499	0.001007080078125	\\
0.491010787055533	0.000579833984375	\\
0.491055178230568	0.000823974609375	\\
0.491099569405602	0.00091552734375	\\
0.491143960580637	0.001220703125	\\
0.491188351755671	0.001922607421875	\\
0.491232742930705	0.002227783203125	\\
0.49127713410574	0.002288818359375	\\
0.491321525280774	0.001953125	\\
0.491365916455809	0.001800537109375	\\
0.491410307630843	0.001739501953125	\\
0.491454698805877	0.0010986328125	\\
0.491499089980912	0.0010986328125	\\
0.491543481155946	0.00079345703125	\\
0.491587872330981	0.000732421875	\\
0.491632263506015	0.001251220703125	\\
0.491676654681049	0.001373291015625	\\
0.491721045856084	0.00140380859375	\\
0.491765437031118	0.00146484375	\\
0.491809828206153	0.00140380859375	\\
0.491854219381187	0.001739501953125	\\
0.491898610556221	0.001800537109375	\\
0.491943001731256	0.001007080078125	\\
0.49198739290629	0.001190185546875	\\
0.492031784081325	0.001617431640625	\\
0.492076175256359	0.0015869140625	\\
0.492120566431393	0.001556396484375	\\
0.492164957606428	0.001373291015625	\\
0.492209348781462	0.00177001953125	\\
0.492253739956497	0.002166748046875	\\
0.492298131131531	0.002655029296875	\\
0.492342522306565	0.0025634765625	\\
0.4923869134816	0.002166748046875	\\
0.492431304656634	0.002197265625	\\
0.492475695831669	0.00238037109375	\\
0.492520087006703	0.002655029296875	\\
0.492564478181737	0.002960205078125	\\
0.492608869356772	0.002777099609375	\\
0.492653260531806	0.002593994140625	\\
0.492697651706841	0.0025634765625	\\
0.492742042881875	0.00213623046875	\\
0.492786434056909	0.001708984375	\\
0.492830825231944	0.001678466796875	\\
0.492875216406978	0.001251220703125	\\
0.492919607582013	0.001373291015625	\\
0.492963998757047	0.002410888671875	\\
0.493008389932082	0.00213623046875	\\
0.493052781107116	0.002197265625	\\
0.49309717228215	0.002044677734375	\\
0.493141563457185	0.001312255859375	\\
0.493185954632219	0.001220703125	\\
0.493230345807254	0.00067138671875	\\
0.493274736982288	0.0003662109375	\\
0.493319128157322	-6.103515625e-05	\\
0.493363519332357	3.0517578125e-05	\\
0.493407910507391	0.00048828125	\\
0.493452301682426	0.001007080078125	\\
0.49349669285746	0.001861572265625	\\
0.493541084032494	0.0018310546875	\\
0.493585475207529	0.001861572265625	\\
0.493629866382563	0.002532958984375	\\
0.493674257557598	0.0025634765625	\\
0.493718648732632	0.002349853515625	\\
0.493763039907666	0.00201416015625	\\
0.493807431082701	0.001953125	\\
0.493851822257735	0.002349853515625	\\
0.49389621343277	0.00238037109375	\\
0.493940604607804	0.002899169921875	\\
0.493984995782838	0.0032958984375	\\
0.494029386957873	0.00341796875	\\
0.494073778132907	0.004302978515625	\\
0.494118169307942	0.004364013671875	\\
0.494162560482976	0.004058837890625	\\
0.49420695165801	0.004241943359375	\\
0.494251342833045	0.0035400390625	\\
0.494295734008079	0.0032958984375	\\
0.494340125183114	0.003082275390625	\\
0.494384516358148	0.002655029296875	\\
0.494428907533182	0.002899169921875	\\
0.494473298708217	0.0029296875	\\
0.494517689883251	0.003204345703125	\\
0.494562081058286	0.00390625	\\
0.49460647223332	0.004180908203125	\\
0.494650863408354	0.003570556640625	\\
0.494695254583389	0.002899169921875	\\
0.494739645758423	0.00274658203125	\\
0.494784036933458	0.001922607421875	\\
0.494828428108492	0.000823974609375	\\
0.494872819283526	0.000579833984375	\\
0.494917210458561	0.000244140625	\\
0.494961601633595	-0.00018310546875	\\
0.49500599280863	-0.000152587890625	\\
0.495050383983664	-6.103515625e-05	\\
0.495094775158698	0.00048828125	\\
0.495139166333733	0.000823974609375	\\
0.495183557508767	0.000640869140625	\\
0.495227948683802	0.000457763671875	\\
0.495272339858836	0.000518798828125	\\
0.49531673103387	-0.000152587890625	\\
0.495361122208905	-0.0006103515625	\\
0.495405513383939	-0.000762939453125	\\
0.495449904558974	-0.000885009765625	\\
0.495494295734008	-0.000457763671875	\\
0.495538686909042	-9.1552734375e-05	\\
0.495583078084077	-0.000274658203125	\\
0.495627469259111	-9.1552734375e-05	\\
0.495671860434146	0.00030517578125	\\
0.49571625160918	-0.000152587890625	\\
0.495760642784215	-9.1552734375e-05	\\
0.495805033959249	-0.0003662109375	\\
0.495849425134283	-0.001190185546875	\\
0.495893816309318	-0.0013427734375	\\
0.495938207484352	-0.001129150390625	\\
0.495982598659386	-0.00042724609375	\\
0.496026989834421	-0.00048828125	\\
0.496071381009455	-3.0517578125e-05	\\
0.49611577218449	0.000457763671875	\\
0.496160163359524	0.000732421875	\\
0.496204554534559	0.00103759765625	\\
0.496248945709593	0.000274658203125	\\
0.496293336884627	-0.00067138671875	\\
0.496337728059662	-0.00128173828125	\\
0.496382119234696	-0.002197265625	\\
0.496426510409731	-0.002227783203125	\\
0.496470901584765	-0.002044677734375	\\
0.496515292759799	-0.00140380859375	\\
0.496559683934834	-0.001007080078125	\\
0.496604075109868	-0.00042724609375	\\
0.496648466284903	0.000213623046875	\\
0.496692857459937	9.1552734375e-05	\\
0.496737248634971	-0.000335693359375	\\
0.496781639810006	-0.000823974609375	\\
0.49682603098504	-0.000823974609375	\\
0.496870422160075	-0.001007080078125	\\
0.496914813335109	-0.00091552734375	\\
0.496959204510143	-0.000701904296875	\\
0.497003595685178	-3.0517578125e-05	\\
0.497047986860212	0.000640869140625	\\
0.497092378035247	0.001068115234375	\\
0.497136769210281	0.001312255859375	\\
0.497181160385315	0.00177001953125	\\
0.49722555156035	0.001861572265625	\\
0.497269942735384	0.00177001953125	\\
0.497314333910419	0.00238037109375	\\
0.497358725085453	0.002288818359375	\\
0.497403116260487	0.001953125	\\
0.497447507435522	0.001953125	\\
0.497491898610556	0.002288818359375	\\
0.497536289785591	0.00238037109375	\\
0.497580680960625	0.00213623046875	\\
0.497625072135659	0.00164794921875	\\
0.497669463310694	0.000946044921875	\\
0.497713854485728	0.0006103515625	\\
0.497758245660763	0.00042724609375	\\
0.497802636835797	0.000701904296875	\\
0.497847028010831	0.000732421875	\\
0.497891419185866	0.00140380859375	\\
0.4979358103609	0.001922607421875	\\
0.497980201535935	0.001678466796875	\\
0.498024592710969	0.001678466796875	\\
0.498068983886003	0.001373291015625	\\
0.498113375061038	0.001312255859375	\\
0.498157766236072	0.001434326171875	\\
0.498202157411107	0.001068115234375	\\
0.498246548586141	0.001007080078125	\\
0.498290939761176	0.001434326171875	\\
0.49833533093621	0.00164794921875	\\
0.498379722111244	0.00189208984375	\\
0.498424113286279	0.00177001953125	\\
0.498468504461313	0.00164794921875	\\
0.498512895636347	0.001739501953125	\\
0.498557286811382	0.001953125	\\
0.498601677986416	0.00201416015625	\\
0.498646069161451	0.00189208984375	\\
0.498690460336485	0.001800537109375	\\
0.49873485151152	0.00238037109375	\\
0.498779242686554	0.002349853515625	\\
0.498823633861588	0.00244140625	\\
0.498868025036623	0.00250244140625	\\
0.498912416211657	0.0023193359375	\\
0.498956807386692	0.00225830078125	\\
0.499001198561726	0.001861572265625	\\
0.49904558973676	0.001434326171875	\\
0.499089980911795	0.00067138671875	\\
0.499134372086829	-6.103515625e-05	\\
0.499178763261864	6.103515625e-05	\\
0.499223154436898	6.103515625e-05	\\
0.499267545611932	6.103515625e-05	\\
0.499311936786967	0.000335693359375	\\
0.499356327962001	0.000762939453125	\\
0.499400719137036	0.001190185546875	\\
0.49944511031207	0.000946044921875	\\
0.499489501487104	0.00042724609375	\\
0.499533892662139	-0.000244140625	\\
0.499578283837173	-0.0009765625	\\
0.499622675012208	-0.00152587890625	\\
0.499667066187242	-0.00146484375	\\
0.499711457362276	-0.0008544921875	\\
0.499755848537311	0	\\
0.499800239712345	0.0006103515625	\\
0.49984463088738	0.001007080078125	\\
0.499889022062414	0.001251220703125	\\
0.499933413237448	0.00115966796875	\\
0.499977804412483	0.000946044921875	\\
0.500022195587517	0.000885009765625	\\
0.500066586762552	0.000701904296875	\\
0.500110977937586	0.000946044921875	\\
0.50015536911262	0.001739501953125	\\
0.500199760287655	0.00238037109375	\\
0.500244151462689	0.003173828125	\\
0.500288542637724	0.003265380859375	\\
0.500332933812758	0.003662109375	\\
0.500377324987792	0.0042724609375	\\
0.500421716162827	0.00445556640625	\\
0.500466107337861	0.004425048828125	\\
0.500510498512896	0.004364013671875	\\
0.50055488968793	0.004425048828125	\\
0.500599280862964	0.003814697265625	\\
0.500643672037999	0.00360107421875	\\
0.500688063213033	0.004119873046875	\\
0.500732454388068	0.00323486328125	\\
0.500776845563102	0.003173828125	\\
0.500821236738137	0.004058837890625	\\
0.500865627913171	0.00384521484375	\\
0.500910019088205	0.003692626953125	\\
0.50095441026324	0.00347900390625	\\
0.500998801438274	0.00286865234375	\\
0.501043192613308	0.002197265625	\\
0.501087583788343	0.00189208984375	\\
0.501131974963377	0.00128173828125	\\
0.501176366138412	0.00054931640625	\\
0.501220757313446	0.00048828125	\\
0.50126514848848	9.1552734375e-05	\\
0.501309539663515	-0.0006103515625	\\
0.501353930838549	-0.00091552734375	\\
0.501398322013584	-0.000885009765625	\\
0.501442713188618	-0.0010986328125	\\
0.501487104363653	-0.001220703125	\\
0.501531495538687	-0.00128173828125	\\
0.501575886713721	-0.001434326171875	\\
0.501620277888756	-0.0013427734375	\\
0.50166466906379	-0.0013427734375	\\
0.501709060238825	-0.001678466796875	\\
0.501753451413859	-0.00177001953125	\\
0.501797842588893	-0.001922607421875	\\
0.501842233763928	-0.00238037109375	\\
0.501886624938962	-0.001922607421875	\\
0.501931016113996	-0.001983642578125	\\
0.501975407289031	-0.002166748046875	\\
0.502019798464065	-0.001495361328125	\\
0.5020641896391	-0.001678466796875	\\
0.502108580814134	-0.00164794921875	\\
0.502152971989169	-0.001220703125	\\
0.502197363164203	-0.001678466796875	\\
0.502241754339237	-0.0020751953125	\\
0.502286145514272	-0.00189208984375	\\
0.502330536689306	-0.001556396484375	\\
0.502374927864341	-0.001220703125	\\
0.502419319039375	-0.001068115234375	\\
0.502463710214409	-0.0008544921875	\\
0.502508101389444	-0.000457763671875	\\
0.502552492564478	-0.00042724609375	\\
0.502596883739513	-0.001068115234375	\\
0.502641274914547	-0.001434326171875	\\
0.502685666089581	-0.001983642578125	\\
0.502730057264616	-0.00225830078125	\\
0.50277444843965	-0.002685546875	\\
0.502818839614685	-0.00250244140625	\\
0.502863230789719	-0.002532958984375	\\
0.502907621964753	-0.002471923828125	\\
0.502952013139788	-0.001800537109375	\\
0.502996404314822	-0.001800537109375	\\
0.503040795489857	-0.00164794921875	\\
0.503085186664891	-0.00201416015625	\\
0.503129577839925	-0.002410888671875	\\
0.50317396901496	-0.00244140625	\\
0.503218360189994	-0.00262451171875	\\
0.503262751365029	-0.00225830078125	\\
0.503307142540063	-0.0025634765625	\\
0.503351533715097	-0.002471923828125	\\
0.503395924890132	-0.001983642578125	\\
0.503440316065166	-0.001861572265625	\\
0.503484707240201	-0.001556396484375	\\
0.503529098415235	-0.00091552734375	\\
0.503573489590269	-0.00048828125	\\
0.503617880765304	0.000244140625	\\
0.503662271940338	0.000885009765625	\\
0.503706663115373	0.00079345703125	\\
0.503751054290407	0.001007080078125	\\
0.503795445465441	0.001007080078125	\\
0.503839836640476	0.000732421875	\\
0.50388422781551	0.000396728515625	\\
0.503928618990545	0.0003662109375	\\
0.503973010165579	0.000274658203125	\\
0.504017401340613	-0.00030517578125	\\
0.504061792515648	-0.000885009765625	\\
0.504106183690682	-0.000946044921875	\\
0.504150574865717	-0.00146484375	\\
0.504194966040751	-0.0015869140625	\\
0.504239357215786	-0.0013427734375	\\
0.50428374839082	-0.0018310546875	\\
0.504328139565854	-0.00115966796875	\\
0.504372530740889	-0.0010986328125	\\
0.504416921915923	-0.001495361328125	\\
0.504461313090958	-0.001434326171875	\\
0.504505704265992	-0.001800537109375	\\
0.504550095441026	-0.001800537109375	\\
0.504594486616061	-0.00115966796875	\\
0.504638877791095	-0.0008544921875	\\
0.50468326896613	-0.00042724609375	\\
0.504727660141164	0	\\
0.504772051316198	3.0517578125e-05	\\
0.504816442491233	0.000152587890625	\\
0.504860833666267	-0.000457763671875	\\
0.504905224841302	-0.000274658203125	\\
0.504949616016336	-0.000213623046875	\\
0.50499400719137	-0.00079345703125	\\
0.505038398366405	-0.000335693359375	\\
0.505082789541439	0.000152587890625	\\
0.505127180716474	0.000274658203125	\\
0.505171571891508	0.000579833984375	\\
0.505215963066542	0.001068115234375	\\
0.505260354241577	0.0009765625	\\
0.505304745416611	0.000457763671875	\\
0.505349136591646	0.0001220703125	\\
0.50539352776668	-0.0001220703125	\\
0.505437918941714	-0.000457763671875	\\
0.505482310116749	-0.000732421875	\\
0.505526701291783	-0.00054931640625	\\
0.505571092466818	-0.000732421875	\\
0.505615483641852	-0.00042724609375	\\
0.505659874816886	0.000152587890625	\\
0.505704265991921	0	\\
0.505748657166955	-0.00030517578125	\\
0.50579304834199	-0.000732421875	\\
0.505837439517024	-0.001556396484375	\\
0.505881830692058	-0.002044677734375	\\
0.505926221867093	-0.002777099609375	\\
0.505970613042127	-0.003204345703125	\\
0.506015004217162	-0.00299072265625	\\
0.506059395392196	-0.002471923828125	\\
0.50610378656723	-0.00177001953125	\\
0.506148177742265	-0.00140380859375	\\
0.506192568917299	-0.0006103515625	\\
0.506236960092334	0	\\
0.506281351267368	-3.0517578125e-05	\\
0.506325742442402	-0.00042724609375	\\
0.506370133617437	-0.00018310546875	\\
0.506414524792471	0.000213623046875	\\
0.506458915967506	0.00091552734375	\\
0.50650330714254	0.0015869140625	\\
0.506547698317575	0.001922607421875	\\
0.506592089492609	0.003021240234375	\\
0.506636480667643	0.003631591796875	\\
0.506680871842678	0.004119873046875	\\
0.506725263017712	0.0042724609375	\\
0.506769654192746	0.004791259765625	\\
0.506814045367781	0.004791259765625	\\
0.506858436542815	0.0037841796875	\\
0.50690282771785	0.003875732421875	\\
0.506947218892884	0.004241943359375	\\
0.506991610067918	0.0037841796875	\\
0.507036001242953	0.003875732421875	\\
0.507080392417987	0.004669189453125	\\
0.507124783593022	0.004425048828125	\\
0.507169174768056	0.00396728515625	\\
0.507213565943091	0.0037841796875	\\
0.507257957118125	0.00347900390625	\\
0.507302348293159	0.0035400390625	\\
0.507346739468194	0.0029296875	\\
0.507391130643228	0.002166748046875	\\
0.507435521818263	0.001495361328125	\\
0.507479912993297	0.00152587890625	\\
0.507524304168331	0.00128173828125	\\
0.507568695343366	0.0010986328125	\\
0.5076130865184	0.001068115234375	\\
0.507657477693435	0.000274658203125	\\
0.507701868868469	-0.000457763671875	\\
0.507746260043503	-0.000762939453125	\\
0.507790651218538	-0.001129150390625	\\
0.507835042393572	-0.001922607421875	\\
0.507879433568607	-0.001617431640625	\\
0.507923824743641	-0.001800537109375	\\
0.507968215918675	-0.00189208984375	\\
0.50801260709371	-0.001495361328125	\\
0.508056998268744	-0.0018310546875	\\
0.508101389443779	-0.00177001953125	\\
0.508145780618813	-0.00140380859375	\\
0.508190171793847	-0.0010986328125	\\
0.508234562968882	-0.0013427734375	\\
0.508278954143916	-0.00164794921875	\\
0.508323345318951	-0.0018310546875	\\
0.508367736493985	-0.00164794921875	\\
0.508412127669019	-0.0013427734375	\\
0.508456518844054	-0.001129150390625	\\
0.508500910019088	-0.00030517578125	\\
0.508545301194123	3.0517578125e-05	\\
0.508589692369157	-0.000244140625	\\
0.508634083544191	3.0517578125e-05	\\
0.508678474719226	9.1552734375e-05	\\
0.50872286589426	-0.000274658203125	\\
0.508767257069295	-0.0003662109375	\\
0.508811648244329	-0.000762939453125	\\
0.508856039419363	-0.00079345703125	\\
0.508900430594398	-0.00048828125	\\
0.508944821769432	-0.000335693359375	\\
0.508989212944467	-0.0001220703125	\\
0.509033604119501	0.0006103515625	\\
0.509077995294535	0.00067138671875	\\
0.50912238646957	0.00018310546875	\\
0.509166777644604	-0.00030517578125	\\
0.509211168819639	-0.000946044921875	\\
0.509255559994673	-0.001434326171875	\\
0.509299951169708	-0.0018310546875	\\
0.509344342344742	-0.001495361328125	\\
0.509388733519776	-0.001617431640625	\\
0.509433124694811	-0.001434326171875	\\
0.509477515869845	-0.0008544921875	\\
0.509521907044879	-0.000518798828125	\\
0.509566298219914	-0.00018310546875	\\
0.509610689394948	-0.000274658203125	\\
0.509655080569983	-0.000335693359375	\\
0.509699471745017	-0.000640869140625	\\
0.509743862920051	-0.001312255859375	\\
0.509788254095086	-0.00201416015625	\\
0.50983264527012	-0.0023193359375	\\
0.509877036445155	-0.001617431640625	\\
0.509921427620189	-0.000518798828125	\\
0.509965818795224	0.00067138671875	\\
0.510010209970258	0.001434326171875	\\
0.510054601145292	0.001983642578125	\\
0.510098992320327	0.002593994140625	\\
0.510143383495361	0.003021240234375	\\
0.510187774670396	0.002655029296875	\\
0.51023216584543	0.001953125	\\
0.510276557020464	0.00115966796875	\\
0.510320948195499	0.001068115234375	\\
0.510365339370533	0.00115966796875	\\
0.510409730545567	0.000640869140625	\\
0.510454121720602	0.00018310546875	\\
0.510498512895636	-0.000244140625	\\
0.510542904070671	-0.000244140625	\\
0.510587295245705	0.000152587890625	\\
0.51063168642074	-0.000213623046875	\\
0.510676077595774	-6.103515625e-05	\\
0.510720468770808	-6.103515625e-05	\\
0.510764859945843	-6.103515625e-05	\\
0.510809251120877	-0.000152587890625	\\
0.510853642295912	-3.0517578125e-05	\\
0.510898033470946	0.000244140625	\\
0.51094242464598	0.000274658203125	\\
0.510986815821015	0.00048828125	\\
0.511031206996049	0.00030517578125	\\
0.511075598171084	0.0006103515625	\\
0.511119989346118	0.00067138671875	\\
0.511164380521152	0.000244140625	\\
0.511208771696187	0.000640869140625	\\
0.511253162871221	0.00030517578125	\\
0.511297554046256	0.000244140625	\\
0.51134194522129	0.00030517578125	\\
0.511386336396324	0.0009765625	\\
0.511430727571359	0.001861572265625	\\
0.511475118746393	0.002288818359375	\\
0.511519509921428	0.002227783203125	\\
0.511563901096462	0.002166748046875	\\
0.511608292271496	0.002288818359375	\\
0.511652683446531	0.0018310546875	\\
0.511697074621565	0.001861572265625	\\
0.5117414657966	0.001617431640625	\\
0.511785856971634	0.00164794921875	\\
0.511830248146668	0.0020751953125	\\
0.511874639321703	0.00164794921875	\\
0.511919030496737	0.0015869140625	\\
0.511963421671772	0.00146484375	\\
0.512007812846806	0.00146484375	\\
0.51205220402184	0.00140380859375	\\
0.512096595196875	0.00091552734375	\\
0.512140986371909	0.000701904296875	\\
0.512185377546944	0.001129150390625	\\
0.512229768721978	0.00128173828125	\\
0.512274159897013	0.00128173828125	\\
0.512318551072047	0.0013427734375	\\
0.512362942247081	0.000701904296875	\\
0.512407333422116	0.00054931640625	\\
0.51245172459715	0.000335693359375	\\
0.512496115772184	0.000457763671875	\\
0.512540506947219	0.000823974609375	\\
0.512584898122253	0.0006103515625	\\
0.512629289297288	0.0008544921875	\\
0.512673680472322	0.00103759765625	\\
0.512718071647357	0.00146484375	\\
0.512762462822391	0.001678466796875	\\
0.512806853997425	0.001983642578125	\\
0.51285124517246	0.002685546875	\\
0.512895636347494	0.00274658203125	\\
0.512940027522529	0.002899169921875	\\
0.512984418697563	0.00347900390625	\\
0.513028809872597	0.003692626953125	\\
0.513073201047632	0.003570556640625	\\
0.513117592222666	0.003570556640625	\\
0.513161983397701	0.0037841796875	\\
0.513206374572735	0.004150390625	\\
0.513250765747769	0.004241943359375	\\
0.513295156922804	0.004302978515625	\\
0.513339548097838	0.004547119140625	\\
0.513383939272873	0.00457763671875	\\
0.513428330447907	0.00439453125	\\
0.513472721622941	0.0045166015625	\\
0.513517112797976	0.004180908203125	\\
0.51356150397301	0.003997802734375	\\
0.513605895148045	0.004119873046875	\\
0.513650286323079	0.003570556640625	\\
0.513694677498113	0.0030517578125	\\
0.513739068673148	0.003021240234375	\\
0.513783459848182	0.002532958984375	\\
0.513827851023217	0.0020751953125	\\
0.513872242198251	0.0015869140625	\\
0.513916633373285	0.001129150390625	\\
0.51396102454832	0.00140380859375	\\
0.514005415723354	0.001373291015625	\\
0.514049806898389	0.000762939453125	\\
0.514094198073423	0.000274658203125	\\
0.514138589248457	-0.000457763671875	\\
0.514182980423492	-0.00091552734375	\\
0.514227371598526	-0.001251220703125	\\
0.514271762773561	-0.001007080078125	\\
0.514316153948595	-0.00042724609375	\\
0.514360545123629	3.0517578125e-05	\\
0.514404936298664	0.000518798828125	\\
0.514449327473698	0.000152587890625	\\
0.514493718648733	-0.00018310546875	\\
0.514538109823767	-0.000396728515625	\\
0.514582500998801	-0.000579833984375	\\
0.514626892173836	-0.000518798828125	\\
0.51467128334887	-0.000579833984375	\\
0.514715674523905	-0.0003662109375	\\
0.514760065698939	0.0001220703125	\\
0.514804456873973	0.00042724609375	\\
0.514848848049008	0.001007080078125	\\
0.514893239224042	0.00128173828125	\\
0.514937630399077	0.001373291015625	\\
0.514982021574111	0.00177001953125	\\
0.515026412749146	0.00164794921875	\\
0.51507080392418	0.001708984375	\\
0.515115195099214	0.00140380859375	\\
0.515159586274249	0.000946044921875	\\
0.515203977449283	0.001251220703125	\\
0.515248368624317	0.001373291015625	\\
0.515292759799352	0.00140380859375	\\
0.515337150974386	0.001556396484375	\\
0.515381542149421	0.002105712890625	\\
0.515425933324455	0.002197265625	\\
0.515470324499489	0.002044677734375	\\
0.515514715674524	0.001007080078125	\\
0.515559106849558	-0.00018310546875	\\
0.515603498024593	-0.000457763671875	\\
0.515647889199627	-0.000579833984375	\\
0.515692280374662	-0.000396728515625	\\
0.515736671549696	-0.0003662109375	\\
0.51578106272473	0.000244140625	\\
0.515825453899765	0.0009765625	\\
0.515869845074799	0.00079345703125	\\
0.515914236249834	0.000885009765625	\\
0.515958627424868	0.0008544921875	\\
0.516003018599902	0.00048828125	\\
0.516047409774937	3.0517578125e-05	\\
0.516091800949971	0.0001220703125	\\
0.516136192125006	0.0003662109375	\\
0.51618058330004	0.000457763671875	\\
0.516224974475074	0.001007080078125	\\
0.516269365650109	0.0018310546875	\\
0.516313756825143	0.002471923828125	\\
0.516358148000178	0.0029296875	\\
0.516402539175212	0.003387451171875	\\
0.516446930350246	0.003631591796875	\\
0.516491321525281	0.003997802734375	\\
0.516535712700315	0.0035400390625	\\
0.51658010387535	0.0035400390625	\\
0.516624495050384	0.003509521484375	\\
0.516668886225418	0.00299072265625	\\
0.516713277400453	0.002716064453125	\\
0.516757668575487	0.002716064453125	\\
0.516802059750522	0.00262451171875	\\
0.516846450925556	0.00201416015625	\\
0.51689084210059	0.0015869140625	\\
0.516935233275625	0.001007080078125	\\
0.516979624450659	0.000640869140625	\\
0.517024015625694	0.0006103515625	\\
0.517068406800728	3.0517578125e-05	\\
0.517112797975762	-3.0517578125e-05	\\
0.517157189150797	0.0001220703125	\\
0.517201580325831	0.0001220703125	\\
0.517245971500866	0.00067138671875	\\
0.5172903626759	0.000946044921875	\\
0.517334753850934	0.0006103515625	\\
0.517379145025969	0.000152587890625	\\
0.517423536201003	0	\\
0.517467927376038	3.0517578125e-05	\\
0.517512318551072	-0.0001220703125	\\
0.517556709726106	0.000732421875	\\
0.517601100901141	0.001068115234375	\\
0.517645492076175	0.000762939453125	\\
0.51768988325121	0.001251220703125	\\
0.517734274426244	0.001708984375	\\
0.517778665601279	0.0015869140625	\\
0.517823056776313	0.001922607421875	\\
0.517867447951347	0.002105712890625	\\
0.517911839126382	0.002197265625	\\
0.517956230301416	0.002593994140625	\\
0.51800062147645	0.003021240234375	\\
0.518045012651485	0.002899169921875	\\
0.518089403826519	0.00323486328125	\\
0.518133795001554	0.003662109375	\\
0.518178186176588	0.0035400390625	\\
0.518222577351622	0.003875732421875	\\
0.518266968526657	0.003387451171875	\\
0.518311359701691	0.002838134765625	\\
0.518355750876726	0.003021240234375	\\
0.51840014205176	0.002593994140625	\\
0.518444533226795	0.002593994140625	\\
0.518488924401829	0.002655029296875	\\
0.518533315576863	0.00286865234375	\\
0.518577706751898	0.002960205078125	\\
0.518622097926932	0.00299072265625	\\
0.518666489101967	0.002410888671875	\\
0.518710880277001	0.0020751953125	\\
0.518755271452035	0.00128173828125	\\
0.51879966262707	0.000335693359375	\\
0.518844053802104	-0.0003662109375	\\
0.518888444977139	-0.000640869140625	\\
0.518932836152173	-0.00048828125	\\
0.518977227327207	0.000274658203125	\\
0.519021618502242	0.00079345703125	\\
0.519066009677276	0.0008544921875	\\
0.519110400852311	0.001220703125	\\
0.519154792027345	0.001678466796875	\\
0.519199183202379	0.001953125	\\
0.519243574377414	0.001708984375	\\
0.519287965552448	0.001861572265625	\\
0.519332356727483	0.0015869140625	\\
0.519376747902517	0.001739501953125	\\
0.519421139077551	0.00189208984375	\\
0.519465530252586	0.00201416015625	\\
0.51950992142762	0.002410888671875	\\
0.519554312602655	0.0029296875	\\
0.519598703777689	0.00341796875	\\
0.519643094952723	0.003814697265625	\\
0.519687486127758	0.003692626953125	\\
0.519731877302792	0.0035400390625	\\
0.519776268477827	0.003662109375	\\
0.519820659652861	0.003021240234375	\\
0.519865050827895	0.0029296875	\\
0.51990944200293	0.0025634765625	\\
0.519953833177964	0.0025634765625	\\
0.519998224352999	0.002685546875	\\
0.520042615528033	0.00250244140625	\\
0.520087006703067	0.002838134765625	\\
0.520131397878102	0.002655029296875	\\
0.520175789053136	0.001800537109375	\\
0.520220180228171	0.001068115234375	\\
0.520264571403205	0.000640869140625	\\
0.520308962578239	6.103515625e-05	\\
0.520353353753274	-0.000701904296875	\\
0.520397744928308	-0.0010986328125	\\
0.520442136103343	-0.00146484375	\\
0.520486527278377	-0.001495361328125	\\
0.520530918453411	-0.001678466796875	\\
0.520575309628446	-0.001434326171875	\\
0.52061970080348	-0.001739501953125	\\
0.520664091978515	-0.001617431640625	\\
0.520708483153549	-0.001678466796875	\\
0.520752874328584	-0.002197265625	\\
0.520797265503618	-0.002532958984375	\\
0.520841656678652	-0.003265380859375	\\
0.520886047853687	-0.002655029296875	\\
0.520930439028721	-0.002471923828125	\\
0.520974830203755	-0.002655029296875	\\
0.52101922137879	-0.00250244140625	\\
0.521063612553824	-0.00244140625	\\
0.521108003728859	-0.00177001953125	\\
0.521152394903893	-0.00140380859375	\\
0.521196786078928	-0.00128173828125	\\
0.521241177253962	-0.00164794921875	\\
0.521285568428996	-0.001739501953125	\\
0.521329959604031	-0.00146484375	\\
0.521374350779065	-0.00128173828125	\\
0.5214187419541	-0.00115966796875	\\
0.521463133129134	-0.000732421875	\\
0.521507524304168	-9.1552734375e-05	\\
0.521551915479203	0.0003662109375	\\
0.521596306654237	0.00091552734375	\\
0.521640697829272	0.0008544921875	\\
0.521685089004306	0.000396728515625	\\
0.52172948017934	0.000518798828125	\\
0.521773871354375	0.000244140625	\\
0.521818262529409	-9.1552734375e-05	\\
0.521862653704444	3.0517578125e-05	\\
0.521907044879478	-0.000732421875	\\
0.521951436054512	-0.00054931640625	\\
0.521995827229547	-0.00030517578125	\\
0.522040218404581	-0.00030517578125	\\
0.522084609579616	0.00018310546875	\\
0.52212900075465	0.00054931640625	\\
0.522173391929684	0.00054931640625	\\
0.522217783104719	0.000244140625	\\
0.522262174279753	-0.000152587890625	\\
0.522306565454788	-0.000396728515625	\\
0.522350956629822	-0.0009765625	\\
0.522395347804856	-0.000640869140625	\\
0.522439738979891	-0.000213623046875	\\
0.522484130154925	-0.00054931640625	\\
0.52252852132996	-0.000274658203125	\\
0.522572912504994	-0.000152587890625	\\
0.522617303680028	6.103515625e-05	\\
0.522661694855063	0.000640869140625	\\
0.522706086030097	0.00042724609375	\\
0.522750477205132	0.00030517578125	\\
0.522794868380166	0.0008544921875	\\
0.5228392595552	0.000946044921875	\\
0.522883650730235	0.0013427734375	\\
0.522928041905269	0.002044677734375	\\
0.522972433080304	0.001922607421875	\\
0.523016824255338	0.001922607421875	\\
0.523061215430372	0.002044677734375	\\
0.523105606605407	0.00146484375	\\
0.523149997780441	0.001495361328125	\\
0.523194388955476	0.001373291015625	\\
0.52323878013051	0.00054931640625	\\
0.523283171305544	0.000335693359375	\\
0.523327562480579	-0.00048828125	\\
0.523371953655613	-0.00115966796875	\\
0.523416344830648	-0.0009765625	\\
0.523460736005682	-0.000457763671875	\\
0.523505127180717	-0.000457763671875	\\
0.523549518355751	-0.00091552734375	\\
0.523593909530785	-0.000396728515625	\\
0.52363830070582	-0.00042724609375	\\
0.523682691880854	-0.00048828125	\\
0.523727083055888	-0.000579833984375	\\
0.523771474230923	-0.000579833984375	\\
0.523815865405957	-0.00091552734375	\\
0.523860256580992	-0.00140380859375	\\
0.523904647756026	-0.000701904296875	\\
0.52394903893106	-9.1552734375e-05	\\
0.523993430106095	-9.1552734375e-05	\\
0.524037821281129	0.00030517578125	\\
0.524082212456164	0.00054931640625	\\
0.524126603631198	0.000396728515625	\\
0.524170994806233	0.00054931640625	\\
0.524215385981267	0.00079345703125	\\
0.524259777156301	0.000640869140625	\\
0.524304168331336	0.00103759765625	\\
0.52434855950637	0.001312255859375	\\
0.524392950681405	0.00140380859375	\\
0.524437341856439	0.00189208984375	\\
0.524481733031473	0.002655029296875	\\
0.524526124206508	0.002777099609375	\\
0.524570515381542	0.002410888671875	\\
0.524614906556577	0.002471923828125	\\
0.524659297731611	0.002716064453125	\\
0.524703688906645	0.002655029296875	\\
0.52474808008168	0.00244140625	\\
0.524792471256714	0.00213623046875	\\
0.524836862431749	0.002655029296875	\\
0.524881253606783	0.003082275390625	\\
0.524925644781817	0.00311279296875	\\
0.524970035956852	0.003204345703125	\\
0.525014427131886	0.002838134765625	\\
0.525058818306921	0.00274658203125	\\
0.525103209481955	0.002349853515625	\\
0.525147600656989	0.001556396484375	\\
0.525191991832024	0.0006103515625	\\
0.525236383007058	-0.000396728515625	\\
0.525280774182093	-0.000762939453125	\\
0.525325165357127	-0.00042724609375	\\
0.525369556532161	-0.000946044921875	\\
0.525413947707196	-0.00067138671875	\\
0.52545833888223	-0.000152587890625	\\
0.525502730057265	-0.000213623046875	\\
0.525547121232299	9.1552734375e-05	\\
0.525591512407333	-6.103515625e-05	\\
0.525635903582368	-0.000213623046875	\\
0.525680294757402	0.00030517578125	\\
0.525724685932437	0.00018310546875	\\
0.525769077107471	0.00054931640625	\\
0.525813468282505	0.0010986328125	\\
0.52585785945754	0.000732421875	\\
0.525902250632574	0.001373291015625	\\
0.525946641807609	0.001861572265625	\\
0.525991032982643	0.001861572265625	\\
0.526035424157677	0.001922607421875	\\
0.526079815332712	0.001861572265625	\\
0.526124206507746	0.001922607421875	\\
0.526168597682781	0.00152587890625	\\
0.526212988857815	0.0020751953125	\\
0.52625738003285	0.002410888671875	\\
0.526301771207884	0.002166748046875	\\
0.526346162382918	0.002593994140625	\\
0.526390553557953	0.0028076171875	\\
0.526434944732987	0.003143310546875	\\
0.526479335908022	0.00286865234375	\\
0.526523727083056	0.002532958984375	\\
0.52656811825809	0.002105712890625	\\
0.526612509433125	0.001434326171875	\\
0.526656900608159	0.001495361328125	\\
0.526701291783193	0.00103759765625	\\
0.526745682958228	0.00030517578125	\\
0.526790074133262	0.000335693359375	\\
0.526834465308297	0.000244140625	\\
0.526878856483331	-9.1552734375e-05	\\
0.526923247658366	-0.000244140625	\\
0.5269676388334	-0.00079345703125	\\
0.527012030008434	-0.000701904296875	\\
0.527056421183469	-0.001434326171875	\\
0.527100812358503	-0.00225830078125	\\
0.527145203533538	-0.002197265625	\\
0.527189594708572	-0.002593994140625	\\
0.527233985883606	-0.00238037109375	\\
0.527278377058641	-0.002288818359375	\\
0.527322768233675	-0.001983642578125	\\
0.52736715940871	-0.00152587890625	\\
0.527411550583744	-0.00201416015625	\\
0.527455941758778	-0.0020751953125	\\
0.527500332933813	-0.002166748046875	\\
0.527544724108847	-0.002349853515625	\\
0.527589115283882	-0.002197265625	\\
0.527633506458916	-0.00225830078125	\\
0.52767789763395	-0.002227783203125	\\
0.527722288808985	-0.00213623046875	\\
0.527766679984019	-0.002105712890625	\\
0.527811071159054	-0.00140380859375	\\
0.527855462334088	-0.00103759765625	\\
0.527899853509122	-0.00079345703125	\\
0.527944244684157	-3.0517578125e-05	\\
0.527988635859191	0.000244140625	\\
0.528033027034226	-3.0517578125e-05	\\
0.52807741820926	-0.000213623046875	\\
0.528121809384294	-0.000396728515625	\\
0.528166200559329	-0.000579833984375	\\
0.528210591734363	-0.00048828125	\\
0.528254982909398	-0.0008544921875	\\
0.528299374084432	-0.0008544921875	\\
0.528343765259466	-0.000457763671875	\\
0.528388156434501	-0.00042724609375	\\
0.528432547609535	-0.001068115234375	\\
0.52847693878457	-0.00091552734375	\\
0.528521329959604	-0.000946044921875	\\
0.528565721134638	-0.001617431640625	\\
0.528610112309673	-0.00128173828125	\\
0.528654503484707	-0.001007080078125	\\
0.528698894659742	-0.000640869140625	\\
0.528743285834776	-0.001434326171875	\\
0.52878767700981	-0.001495361328125	\\
0.528832068184845	-0.001068115234375	\\
0.528876459359879	-0.0013427734375	\\
0.528920850534914	-0.0010986328125	\\
0.528965241709948	-0.0009765625	\\
0.529009632884982	-0.000518798828125	\\
0.529054024060017	-0.00048828125	\\
0.529098415235051	-0.000335693359375	\\
0.529142806410086	9.1552734375e-05	\\
0.52918719758512	0.000213623046875	\\
0.529231588760155	0.001129150390625	\\
0.529275979935189	0.001434326171875	\\
0.529320371110223	0.001708984375	\\
0.529364762285258	0.001953125	\\
0.529409153460292	0.001220703125	\\
0.529453544635326	0.001251220703125	\\
0.529497935810361	0.00140380859375	\\
0.529542326985395	0.001678466796875	\\
0.52958671816043	0.001312255859375	\\
0.529631109335464	0.00091552734375	\\
0.529675500510499	0.0003662109375	\\
0.529719891685533	-3.0517578125e-05	\\
0.529764282860567	-0.0001220703125	\\
0.529808674035602	-0.000274658203125	\\
0.529853065210636	-0.000244140625	\\
0.529897456385671	-0.00048828125	\\
0.529941847560705	-0.001007080078125	\\
0.529986238735739	-0.001068115234375	\\
0.530030629910774	-0.0003662109375	\\
0.530075021085808	-0.00030517578125	\\
0.530119412260843	-0.000213623046875	\\
0.530163803435877	0.00018310546875	\\
0.530208194610911	-3.0517578125e-05	\\
0.530252585785946	-3.0517578125e-05	\\
0.53029697696098	0.000244140625	\\
0.530341368136015	0.0001220703125	\\
0.530385759311049	0.000335693359375	\\
0.530430150486083	0.000244140625	\\
0.530474541661118	0.00054931640625	\\
0.530518932836152	0.0008544921875	\\
0.530563324011187	0.000457763671875	\\
0.530607715186221	0.00079345703125	\\
0.530652106361255	0.001556396484375	\\
0.53069649753629	0.002044677734375	\\
0.530740888711324	0.001983642578125	\\
0.530785279886359	0.002716064453125	\\
0.530829671061393	0.0032958984375	\\
0.530874062236427	0.00274658203125	\\
0.530918453411462	0.0030517578125	\\
0.530962844586496	0.00299072265625	\\
0.531007235761531	0.002655029296875	\\
0.531051626936565	0.002960205078125	\\
0.531096018111599	0.003082275390625	\\
0.531140409286634	0.003448486328125	\\
0.531184800461668	0.003936767578125	\\
0.531229191636703	0.004486083984375	\\
0.531273582811737	0.0045166015625	\\
0.531317973986771	0.00421142578125	\\
0.531362365161806	0.004150390625	\\
0.53140675633684	0.00372314453125	\\
0.531451147511875	0.003143310546875	\\
0.531495538686909	0.002685546875	\\
0.531539929861943	0.002471923828125	\\
0.531584321036978	0.002532958984375	\\
0.531628712212012	0.002532958984375	\\
0.531673103387047	0.001953125	\\
0.531717494562081	0.001312255859375	\\
0.531761885737115	0.00177001953125	\\
0.53180627691215	0.00189208984375	\\
0.531850668087184	0.001739501953125	\\
0.531895059262219	0.00146484375	\\
0.531939450437253	0.00103759765625	\\
0.531983841612288	0.0013427734375	\\
0.532028232787322	0.001556396484375	\\
0.532072623962356	0.001678466796875	\\
0.532117015137391	0.0018310546875	\\
0.532161406312425	0.002166748046875	\\
0.532205797487459	0.002471923828125	\\
0.532250188662494	0.002960205078125	\\
0.532294579837528	0.003143310546875	\\
0.532338971012563	0.003021240234375	\\
0.532383362187597	0.00323486328125	\\
0.532427753362631	0.003265380859375	\\
0.532472144537666	0.003173828125	\\
0.5325165357127	0.003143310546875	\\
0.532560926887735	0.0035400390625	\\
0.532605318062769	0.004150390625	\\
0.532649709237804	0.004364013671875	\\
0.532694100412838	0.0042724609375	\\
};
\addplot [color=blue,solid,forget plot]
  table[row sep=crcr]{
0.532694100412838	0.0042724609375	\\
0.532738491587872	0.004241943359375	\\
0.532782882762907	0.003936767578125	\\
0.532827273937941	0.003662109375	\\
0.532871665112976	0.003448486328125	\\
0.53291605628801	0.00372314453125	\\
0.532960447463044	0.003326416015625	\\
0.533004838638079	0.003204345703125	\\
0.533049229813113	0.003021240234375	\\
0.533093620988148	0.00250244140625	\\
0.533138012163182	0.002197265625	\\
0.533182403338216	0.001800537109375	\\
0.533226794513251	0.001373291015625	\\
0.533271185688285	0.001434326171875	\\
0.53331557686332	0.001373291015625	\\
0.533359968038354	0.001190185546875	\\
0.533404359213388	0.0009765625	\\
0.533448750388423	0.000823974609375	\\
0.533493141563457	0.0008544921875	\\
0.533537532738492	0.0003662109375	\\
0.533581923913526	-0.000244140625	\\
0.53362631508856	-0.00042724609375	\\
0.533670706263595	-0.00042724609375	\\
0.533715097438629	-0.000518798828125	\\
0.533759488613664	-0.000579833984375	\\
0.533803879788698	-0.001007080078125	\\
0.533848270963732	-0.001007080078125	\\
0.533892662138767	-0.00103759765625	\\
0.533937053313801	-0.000946044921875	\\
0.533981444488836	-0.000335693359375	\\
0.53402583566387	-0.000244140625	\\
0.534070226838904	-0.000213623046875	\\
0.534114618013939	3.0517578125e-05	\\
0.534159009188973	-6.103515625e-05	\\
0.534203400364008	-0.000152587890625	\\
0.534247791539042	-0.0001220703125	\\
0.534292182714076	-3.0517578125e-05	\\
0.534336573889111	0	\\
0.534380965064145	0.00030517578125	\\
0.53442535623918	0.00067138671875	\\
0.534469747414214	0.0006103515625	\\
0.534514138589248	0.000640869140625	\\
0.534558529764283	0.000946044921875	\\
0.534602920939317	0.000640869140625	\\
0.534647312114352	0.000244140625	\\
0.534691703289386	0.000213623046875	\\
0.534736094464421	0	\\
0.534780485639455	-0.0001220703125	\\
0.534824876814489	-0.00042724609375	\\
0.534869267989524	-0.000518798828125	\\
0.534913659164558	-0.00079345703125	\\
0.534958050339593	-0.000823974609375	\\
0.535002441514627	-0.000640869140625	\\
0.535046832689661	-0.00115966796875	\\
0.535091223864696	-0.00140380859375	\\
0.53513561503973	-0.00115966796875	\\
0.535180006214764	-0.00103759765625	\\
0.535224397389799	-0.001373291015625	\\
0.535268788564833	-0.00152587890625	\\
0.535313179739868	-0.001678466796875	\\
0.535357570914902	-0.001708984375	\\
0.535401962089937	-0.00128173828125	\\
0.535446353264971	-0.00103759765625	\\
0.535490744440005	-0.000732421875	\\
0.53553513561504	-0.000152587890625	\\
0.535579526790074	0.00018310546875	\\
0.535623917965109	0.000152587890625	\\
0.535668309140143	0.000946044921875	\\
0.535712700315177	0.00128173828125	\\
0.535757091490212	0.0013427734375	\\
0.535801482665246	0.001434326171875	\\
0.535845873840281	0.00152587890625	\\
0.535890265015315	0.001556396484375	\\
0.535934656190349	0.001373291015625	\\
0.535979047365384	0.001617431640625	\\
0.536023438540418	0.002044677734375	\\
0.536067829715453	0.002227783203125	\\
0.536112220890487	0.002349853515625	\\
0.536156612065521	0.00244140625	\\
0.536201003240556	0.00164794921875	\\
0.53624539441559	0.00128173828125	\\
0.536289785590625	0.0010986328125	\\
0.536334176765659	0.000640869140625	\\
0.536378567940693	0.001007080078125	\\
0.536422959115728	0.001190185546875	\\
0.536467350290762	0.000823974609375	\\
0.536511741465797	0.0008544921875	\\
0.536556132640831	0.0013427734375	\\
0.536600523815865	0.001708984375	\\
0.5366449149909	0.00152587890625	\\
0.536689306165934	0.00164794921875	\\
0.536733697340969	0.0013427734375	\\
0.536778088516003	0.000946044921875	\\
0.536822479691037	0.00042724609375	\\
0.536866870866072	6.103515625e-05	\\
0.536911262041106	0.00030517578125	\\
0.536955653216141	0.000518798828125	\\
0.537000044391175	0.000732421875	\\
0.537044435566209	0.0010986328125	\\
0.537088826741244	0.00067138671875	\\
0.537133217916278	0.00067138671875	\\
0.537177609091313	0.001190185546875	\\
0.537222000266347	0.00128173828125	\\
0.537266391441381	0.001220703125	\\
0.537310782616416	0.0009765625	\\
0.53735517379145	0.001068115234375	\\
0.537399564966485	0.001373291015625	\\
0.537443956141519	0.00146484375	\\
0.537488347316553	0.00201416015625	\\
0.537532738491588	0.002471923828125	\\
0.537577129666622	0.002685546875	\\
0.537621520841657	0.003021240234375	\\
0.537665912016691	0.003173828125	\\
0.537710303191726	0.002471923828125	\\
0.53775469436676	0.002655029296875	\\
0.537799085541794	0.002960205078125	\\
0.537843476716829	0.002685546875	\\
0.537887867891863	0.002685546875	\\
0.537932259066897	0.002349853515625	\\
0.537976650241932	0.002899169921875	\\
0.538021041416966	0.002685546875	\\
0.538065432592001	0.001983642578125	\\
0.538109823767035	0.001861572265625	\\
0.53815421494207	0.00103759765625	\\
0.538198606117104	0.00042724609375	\\
0.538242997292138	0.000396728515625	\\
0.538287388467173	0.000274658203125	\\
0.538331779642207	0.0003662109375	\\
0.538376170817242	0.00018310546875	\\
0.538420561992276	0.000213623046875	\\
0.53846495316731	0.0009765625	\\
0.538509344342345	0.00091552734375	\\
0.538553735517379	0.000762939453125	\\
0.538598126692414	0.00128173828125	\\
0.538642517867448	0.001129150390625	\\
0.538686909042482	0.000946044921875	\\
0.538731300217517	0.00115966796875	\\
0.538775691392551	0.001495361328125	\\
0.538820082567586	0.001617431640625	\\
0.53886447374262	0.0018310546875	\\
0.538908864917654	0.002349853515625	\\
0.538953256092689	0.002471923828125	\\
0.538997647267723	0.002166748046875	\\
0.539042038442758	0.00250244140625	\\
0.539086429617792	0.002777099609375	\\
0.539130820792826	0.00250244140625	\\
0.539175211967861	0.002532958984375	\\
0.539219603142895	0.0025634765625	\\
0.53926399431793	0.00225830078125	\\
0.539308385492964	0.002044677734375	\\
0.539352776667998	0.002197265625	\\
0.539397167843033	0.002410888671875	\\
0.539441559018067	0.0023193359375	\\
0.539485950193102	0.00201416015625	\\
0.539530341368136	0.001708984375	\\
0.53957473254317	0.001678466796875	\\
0.539619123718205	0.001312255859375	\\
0.539663514893239	0.001220703125	\\
0.539707906068274	0.001495361328125	\\
0.539752297243308	0.001556396484375	\\
0.539796688418342	0.0018310546875	\\
0.539841079593377	0.00146484375	\\
0.539885470768411	0.001007080078125	\\
0.539929861943446	0.000946044921875	\\
0.53997425311848	0.000457763671875	\\
0.540018644293514	9.1552734375e-05	\\
0.540063035468549	0	\\
0.540107426643583	-0.000457763671875	\\
0.540151817818618	-6.103515625e-05	\\
0.540196208993652	0.000335693359375	\\
0.540240600168686	0.000213623046875	\\
0.540284991343721	9.1552734375e-05	\\
0.540329382518755	0.00042724609375	\\
0.54037377369379	0.0006103515625	\\
0.540418164868824	0.000274658203125	\\
0.540462556043859	0.000457763671875	\\
0.540506947218893	0.000213623046875	\\
0.540551338393927	0.0001220703125	\\
0.540595729568962	6.103515625e-05	\\
0.540640120743996	0	\\
0.54068451191903	-9.1552734375e-05	\\
0.540728903094065	0.00018310546875	\\
0.540773294269099	0.001068115234375	\\
0.540817685444134	0.0010986328125	\\
0.540862076619168	0.001373291015625	\\
0.540906467794202	0.001495361328125	\\
0.540950858969237	0.001251220703125	\\
0.540995250144271	0.00140380859375	\\
0.541039641319306	0.00146484375	\\
0.54108403249434	0.001007080078125	\\
0.541128423669375	0.000701904296875	\\
0.541172814844409	0.0006103515625	\\
0.541217206019443	0.000335693359375	\\
0.541261597194478	0.00079345703125	\\
0.541305988369512	0.0009765625	\\
0.541350379544547	0.00030517578125	\\
0.541394770719581	9.1552734375e-05	\\
0.541439161894615	0.00042724609375	\\
0.54148355306965	0.0006103515625	\\
0.541527944244684	0.00042724609375	\\
0.541572335419719	0.00042724609375	\\
0.541616726594753	0.000213623046875	\\
0.541661117769787	0.0001220703125	\\
0.541705508944822	0.00030517578125	\\
0.541749900119856	0.00054931640625	\\
0.541794291294891	0.000335693359375	\\
0.541838682469925	9.1552734375e-05	\\
0.541883073644959	0.000640869140625	\\
0.541927464819994	0.000396728515625	\\
0.541971855995028	0.00030517578125	\\
0.542016247170063	0.000762939453125	\\
0.542060638345097	0.000823974609375	\\
0.542105029520131	0.00140380859375	\\
0.542149420695166	0.0015869140625	\\
0.5421938118702	0.002044677734375	\\
0.542238203045235	0.002593994140625	\\
0.542282594220269	0.00262451171875	\\
0.542326985395303	0.0028076171875	\\
0.542371376570338	0.0025634765625	\\
0.542415767745372	0.002410888671875	\\
0.542460158920407	0.002410888671875	\\
0.542504550095441	0.00238037109375	\\
0.542548941270475	0.002227783203125	\\
0.54259333244551	0.002410888671875	\\
0.542637723620544	0.002227783203125	\\
0.542682114795579	0.001861572265625	\\
0.542726505970613	0.001617431640625	\\
0.542770897145647	0.00164794921875	\\
0.542815288320682	0.001434326171875	\\
0.542859679495716	0.0013427734375	\\
0.542904070670751	0.001495361328125	\\
0.542948461845785	0.001251220703125	\\
0.542992853020819	0.000762939453125	\\
0.543037244195854	0.000823974609375	\\
0.543081635370888	0.00079345703125	\\
0.543126026545923	0.00042724609375	\\
0.543170417720957	0.000457763671875	\\
0.543214808895992	0.0003662109375	\\
0.543259200071026	-0.000152587890625	\\
0.54330359124606	-0.0001220703125	\\
0.543347982421095	0.000335693359375	\\
0.543392373596129	0.00042724609375	\\
0.543436764771164	0.00018310546875	\\
0.543481155946198	0.000152587890625	\\
0.543525547121232	0.00030517578125	\\
0.543569938296267	0.00018310546875	\\
0.543614329471301	0.00042724609375	\\
0.543658720646335	0.00042724609375	\\
0.54370311182137	0.000274658203125	\\
0.543747502996404	0.000701904296875	\\
0.543791894171439	0.000823974609375	\\
0.543836285346473	0.000762939453125	\\
0.543880676521508	0.00091552734375	\\
0.543925067696542	0.001068115234375	\\
0.543969458871576	0.001007080078125	\\
0.544013850046611	0.001251220703125	\\
0.544058241221645	0.0020751953125	\\
0.54410263239668	0.001953125	\\
0.544147023571714	0.00189208984375	\\
0.544191414746748	0.002197265625	\\
0.544235805921783	0.00189208984375	\\
0.544280197096817	0.001739501953125	\\
0.544324588271852	0.001739501953125	\\
0.544368979446886	0.001556396484375	\\
0.54441337062192	0.00140380859375	\\
0.544457761796955	0.001434326171875	\\
0.544502152971989	0.001495361328125	\\
0.544546544147024	0.000946044921875	\\
0.544590935322058	0.00091552734375	\\
0.544635326497092	0.000701904296875	\\
0.544679717672127	0.000732421875	\\
0.544724108847161	0.0008544921875	\\
0.544768500022196	0.000274658203125	\\
0.54481289119723	0.000244140625	\\
0.544857282372264	0	\\
0.544901673547299	0.000213623046875	\\
0.544946064722333	0.000457763671875	\\
0.544990455897368	0.0003662109375	\\
0.545034847072402	0.000335693359375	\\
0.545079238247436	0.00103759765625	\\
0.545123629422471	0.00140380859375	\\
0.545168020597505	0.001129150390625	\\
0.54521241177254	0.001190185546875	\\
0.545256802947574	0.0009765625	\\
0.545301194122608	0.001190185546875	\\
0.545345585297643	0.001312255859375	\\
0.545389976472677	0.0009765625	\\
0.545434367647712	0.00103759765625	\\
0.545478758822746	0.001556396484375	\\
0.54552314999778	0.001708984375	\\
0.545567541172815	0.001983642578125	\\
0.545611932347849	0.001739501953125	\\
0.545656323522884	0.00128173828125	\\
0.545700714697918	0.001190185546875	\\
0.545745105872952	0.001434326171875	\\
0.545789497047987	0.00164794921875	\\
0.545833888223021	0.00177001953125	\\
0.545878279398056	0.00152587890625	\\
0.54592267057309	0.00152587890625	\\
0.545967061748124	0.001708984375	\\
0.546011452923159	0.001861572265625	\\
0.546055844098193	0.001800537109375	\\
0.546100235273228	0.001739501953125	\\
0.546144626448262	0.001220703125	\\
0.546189017623297	0.000762939453125	\\
0.546233408798331	0.000885009765625	\\
0.546277799973365	0.000823974609375	\\
0.5463221911484	0.000335693359375	\\
0.546366582323434	-9.1552734375e-05	\\
0.546410973498468	6.103515625e-05	\\
0.546455364673503	0.000274658203125	\\
0.546499755848537	0.000640869140625	\\
0.546544147023572	0.000396728515625	\\
0.546588538198606	0.00018310546875	\\
0.546632929373641	0.00018310546875	\\
0.546677320548675	0.000274658203125	\\
0.546721711723709	0.000396728515625	\\
0.546766102898744	-0.000152587890625	\\
0.546810494073778	-0.000396728515625	\\
0.546854885248813	-0.00018310546875	\\
0.546899276423847	0.0001220703125	\\
0.546943667598881	0.000640869140625	\\
0.546988058773916	0.0006103515625	\\
0.54703244994895	0.0003662109375	\\
0.547076841123985	0.00067138671875	\\
0.547121232299019	0.00018310546875	\\
0.547165623474053	0.00048828125	\\
0.547210014649088	0.000244140625	\\
0.547254405824122	0	\\
0.547298796999157	0.000274658203125	\\
0.547343188174191	0.000335693359375	\\
0.547387579349225	0.000701904296875	\\
0.54743197052426	0.00067138671875	\\
0.547476361699294	0.00054931640625	\\
0.547520752874329	0.000396728515625	\\
0.547565144049363	0.000457763671875	\\
0.547609535224397	0.0010986328125	\\
0.547653926399432	0.00115966796875	\\
0.547698317574466	0.000946044921875	\\
0.547742708749501	0.0008544921875	\\
0.547787099924535	0.001068115234375	\\
0.547831491099569	0.00103759765625	\\
0.547875882274604	0.001251220703125	\\
0.547920273449638	0.00103759765625	\\
0.547964664624673	0.001220703125	\\
0.548009055799707	0.000762939453125	\\
0.548053446974741	0.000762939453125	\\
0.548097838149776	0.000701904296875	\\
0.54814222932481	0.00018310546875	\\
0.548186620499845	0.0001220703125	\\
0.548231011674879	0.000244140625	\\
0.548275402849913	3.0517578125e-05	\\
0.548319794024948	0.00018310546875	\\
0.548364185199982	0.00091552734375	\\
0.548408576375017	0.0013427734375	\\
0.548452967550051	0.0018310546875	\\
0.548497358725085	0.001800537109375	\\
0.54854174990012	0.0020751953125	\\
0.548586141075154	0.002410888671875	\\
0.548630532250189	0.001556396484375	\\
0.548674923425223	0.002044677734375	\\
0.548719314600257	0.00250244140625	\\
0.548763705775292	0.002166748046875	\\
0.548808096950326	0.00244140625	\\
0.548852488125361	0.0028076171875	\\
0.548896879300395	0.002960205078125	\\
0.54894127047543	0.003082275390625	\\
0.548985661650464	0.002685546875	\\
0.549030052825498	0.002471923828125	\\
0.549074444000533	0.00250244140625	\\
0.549118835175567	0.00250244140625	\\
0.549163226350602	0.002197265625	\\
0.549207617525636	0.00152587890625	\\
0.54925200870067	0.00140380859375	\\
0.549296399875705	0.001678466796875	\\
0.549340791050739	0.001617431640625	\\
0.549385182225773	0.00140380859375	\\
0.549429573400808	0.00103759765625	\\
0.549473964575842	0.001312255859375	\\
0.549518355750877	0.001495361328125	\\
0.549562746925911	0.00177001953125	\\
0.549607138100946	0.001983642578125	\\
0.54965152927598	0.00201416015625	\\
0.549695920451014	0.002288818359375	\\
0.549740311626049	0.002532958984375	\\
0.549784702801083	0.0025634765625	\\
0.549829093976118	0.00238037109375	\\
0.549873485151152	0.002197265625	\\
0.549917876326186	0.002716064453125	\\
0.549962267501221	0.002899169921875	\\
0.550006658676255	0.002716064453125	\\
0.55005104985129	0.00274658203125	\\
0.550095441026324	0.002410888671875	\\
0.550139832201358	0.002471923828125	\\
0.550184223376393	0.0020751953125	\\
0.550228614551427	0.001983642578125	\\
0.550273005726462	0.002197265625	\\
0.550317396901496	0.002288818359375	\\
0.55036178807653	0.00262451171875	\\
0.550406179251565	0.002716064453125	\\
0.550450570426599	0.002532958984375	\\
0.550494961601634	0.002655029296875	\\
0.550539352776668	0.00286865234375	\\
0.550583743951702	0.002593994140625	\\
0.550628135126737	0.0023193359375	\\
0.550672526301771	0.002655029296875	\\
0.550716917476806	0.00262451171875	\\
0.55076130865184	0.002471923828125	\\
0.550805699826874	0.002655029296875	\\
0.550850091001909	0.002838134765625	\\
0.550894482176943	0.003265380859375	\\
0.550938873351978	0.00323486328125	\\
0.550983264527012	0.0029296875	\\
0.551027655702046	0.00250244140625	\\
0.551072046877081	0.0023193359375	\\
0.551116438052115	0.001861572265625	\\
0.55116082922715	0.001556396484375	\\
0.551205220402184	0.001495361328125	\\
0.551249611577218	0.001617431640625	\\
0.551294002752253	0.002044677734375	\\
0.551338393927287	0.00213623046875	\\
0.551382785102322	0.002532958984375	\\
0.551427176277356	0.002716064453125	\\
0.55147156745239	0.00286865234375	\\
0.551515958627425	0.002197265625	\\
0.551560349802459	0.00213623046875	\\
0.551604740977494	0.0025634765625	\\
0.551649132152528	0.00244140625	\\
0.551693523327563	0.002227783203125	\\
0.551737914502597	0.00213623046875	\\
0.551782305677631	0.002227783203125	\\
0.551826696852666	0.001922607421875	\\
0.5518710880277	0.002197265625	\\
0.551915479202735	0.002288818359375	\\
0.551959870377769	0.00225830078125	\\
0.552004261552803	0.00274658203125	\\
0.552048652727838	0.002532958984375	\\
0.552093043902872	0.0020751953125	\\
0.552137435077906	0.0018310546875	\\
0.552181826252941	0.001739501953125	\\
0.552226217427975	0.0013427734375	\\
0.55227060860301	0.00164794921875	\\
0.552314999778044	0.001708984375	\\
0.552359390953079	0.0015869140625	\\
0.552403782128113	0.0013427734375	\\
0.552448173303147	0.001220703125	\\
0.552492564478182	0.0015869140625	\\
0.552536955653216	0.001220703125	\\
0.552581346828251	0.001190185546875	\\
0.552625738003285	0.001007080078125	\\
0.552670129178319	0.0008544921875	\\
0.552714520353354	0.00067138671875	\\
0.552758911528388	0.000579833984375	\\
0.552803302703423	0.00042724609375	\\
0.552847693878457	0.00048828125	\\
0.552892085053491	0.0009765625	\\
0.552936476228526	0.000885009765625	\\
0.55298086740356	0.00054931640625	\\
0.553025258578595	0.0006103515625	\\
0.553069649753629	0.00054931640625	\\
0.553114040928663	0.000457763671875	\\
0.553158432103698	0.000518798828125	\\
0.553202823278732	0.000885009765625	\\
0.553247214453767	0.001373291015625	\\
0.553291605628801	0.001617431640625	\\
0.553335996803835	0.00146484375	\\
0.55338038797887	0.001373291015625	\\
0.553424779153904	0.001556396484375	\\
0.553469170328939	0.001434326171875	\\
0.553513561503973	0.001556396484375	\\
0.553557952679007	0.001800537109375	\\
0.553602343854042	0.001556396484375	\\
0.553646735029076	0.00164794921875	\\
0.553691126204111	0.001861572265625	\\
0.553735517379145	0.002227783203125	\\
0.553779908554179	0.001800537109375	\\
0.553824299729214	0.001373291015625	\\
0.553868690904248	0.001617431640625	\\
0.553913082079283	0.00146484375	\\
0.553957473254317	0.001434326171875	\\
0.554001864429351	0.0013427734375	\\
0.554046255604386	0.000823974609375	\\
0.55409064677942	0.0006103515625	\\
0.554135037954455	0.00079345703125	\\
0.554179429129489	0.000701904296875	\\
0.554223820304523	0.00067138671875	\\
0.554268211479558	0.00115966796875	\\
0.554312602654592	0.00091552734375	\\
0.554356993829627	0.000701904296875	\\
0.554401385004661	0.0006103515625	\\
0.554445776179695	0.000885009765625	\\
0.55449016735473	0.00103759765625	\\
0.554534558529764	0.000885009765625	\\
0.554578949704799	0.000885009765625	\\
0.554623340879833	0.00018310546875	\\
0.554667732054868	0.000274658203125	\\
0.554712123229902	0.000701904296875	\\
0.554756514404936	0.000762939453125	\\
0.554800905579971	0.000732421875	\\
0.554845296755005	0.000579833984375	\\
0.554889687930039	0.000701904296875	\\
0.554934079105074	0.000762939453125	\\
0.554978470280108	0.0009765625	\\
0.555022861455143	0.001220703125	\\
0.555067252630177	0.00128173828125	\\
0.555111643805212	0.00140380859375	\\
0.555156034980246	0.001739501953125	\\
0.55520042615528	0.001739501953125	\\
0.555244817330315	0.00225830078125	\\
0.555289208505349	0.00238037109375	\\
0.555333599680384	0.00238037109375	\\
0.555377990855418	0.002960205078125	\\
0.555422382030452	0.0030517578125	\\
0.555466773205487	0.0029296875	\\
0.555511164380521	0.003143310546875	\\
0.555555555555556	0.00286865234375	\\
0.55559994673059	0.003448486328125	\\
0.555644337905624	0.003692626953125	\\
0.555688729080659	0.003204345703125	\\
0.555733120255693	0.003814697265625	\\
0.555777511430728	0.003814697265625	\\
0.555821902605762	0.003448486328125	\\
0.555866293780796	0.003753662109375	\\
0.555910684955831	0.003448486328125	\\
0.555955076130865	0.0030517578125	\\
0.5559994673059	0.002838134765625	\\
0.556043858480934	0.003021240234375	\\
0.556088249655968	0.00335693359375	\\
0.556132640831003	0.0030517578125	\\
0.556177032006037	0.003143310546875	\\
0.556221423181072	0.003387451171875	\\
0.556265814356106	0.003204345703125	\\
0.55631020553114	0.003631591796875	\\
0.556354596706175	0.003997802734375	\\
0.556398987881209	0.003631591796875	\\
0.556443379056244	0.003875732421875	\\
0.556487770231278	0.0040283203125	\\
0.556532161406312	0.003387451171875	\\
0.556576552581347	0.003326416015625	\\
0.556620943756381	0.00360107421875	\\
0.556665334931416	0.003814697265625	\\
0.55670972610645	0.0037841796875	\\
0.556754117281485	0.00384521484375	\\
0.556798508456519	0.00390625	\\
0.556842899631553	0.0037841796875	\\
0.556887290806588	0.003570556640625	\\
0.556931681981622	0.00360107421875	\\
0.556976073156656	0.003692626953125	\\
0.557020464331691	0.003173828125	\\
0.557064855506725	0.003021240234375	\\
0.55710924668176	0.003631591796875	\\
0.557153637856794	0.004119873046875	\\
0.557198029031828	0.00396728515625	\\
0.557242420206863	0.00372314453125	\\
0.557286811381897	0.00384521484375	\\
0.557331202556932	0.004364013671875	\\
0.557375593731966	0.004119873046875	\\
0.557419984907001	0.003997802734375	\\
0.557464376082035	0.004058837890625	\\
0.557508767257069	0.003662109375	\\
0.557553158432104	0.00384521484375	\\
0.557597549607138	0.0037841796875	\\
0.557641940782173	0.0032958984375	\\
0.557686331957207	0.003173828125	\\
0.557730723132241	0.003204345703125	\\
0.557775114307276	0.00299072265625	\\
0.55781950548231	0.002960205078125	\\
0.557863896657344	0.003143310546875	\\
0.557908287832379	0.00286865234375	\\
0.557952679007413	0.0029296875	\\
0.557997070182448	0.003204345703125	\\
0.558041461357482	0.002777099609375	\\
0.558085852532517	0.002593994140625	\\
0.558130243707551	0.002777099609375	\\
0.558174634882585	0.002685546875	\\
0.55821902605762	0.002655029296875	\\
0.558263417232654	0.00335693359375	\\
0.558307808407689	0.00360107421875	\\
0.558352199582723	0.00323486328125	\\
0.558396590757757	0.003173828125	\\
0.558440981932792	0.0032958984375	\\
0.558485373107826	0.003143310546875	\\
0.558529764282861	0.002410888671875	\\
0.558574155457895	0.002532958984375	\\
0.558618546632929	0.0029296875	\\
0.558662937807964	0.002593994140625	\\
0.558707328982998	0.0023193359375	\\
0.558751720158033	0.002166748046875	\\
0.558796111333067	0.00225830078125	\\
0.558840502508101	0.002532958984375	\\
0.558884893683136	0.00244140625	\\
0.55892928485817	0.002410888671875	\\
0.558973676033205	0.00238037109375	\\
0.559018067208239	0.001983642578125	\\
0.559062458383273	0.0020751953125	\\
0.559106849558308	0.00201416015625	\\
0.559151240733342	0.002044677734375	\\
0.559195631908377	0.002227783203125	\\
0.559240023083411	0.002410888671875	\\
0.559284414258445	0.002716064453125	\\
0.55932880543348	0.002899169921875	\\
0.559373196608514	0.00286865234375	\\
0.559417587783549	0.00323486328125	\\
0.559461978958583	0.002899169921875	\\
0.559506370133617	0.002532958984375	\\
0.559550761308652	0.002960205078125	\\
0.559595152483686	0.002685546875	\\
0.559639543658721	0.002716064453125	\\
0.559683934833755	0.00286865234375	\\
0.559728326008789	0.002716064453125	\\
0.559772717183824	0.002899169921875	\\
0.559817108358858	0.002685546875	\\
0.559861499533893	0.00262451171875	\\
0.559905890708927	0.002716064453125	\\
0.559950281883961	0.002685546875	\\
0.559994673058996	0.0028076171875	\\
0.56003906423403	0.0030517578125	\\
0.560083455409065	0.002960205078125	\\
0.560127846584099	0.00286865234375	\\
0.560172237759134	0.002960205078125	\\
0.560216628934168	0.003082275390625	\\
0.560261020109202	0.002593994140625	\\
0.560305411284237	0.002655029296875	\\
0.560349802459271	0.003082275390625	\\
0.560394193634306	0.002197265625	\\
0.56043858480934	0.002197265625	\\
0.560482975984374	0.002471923828125	\\
0.560527367159409	0.002197265625	\\
0.560571758334443	0.00244140625	\\
0.560616149509477	0.00250244140625	\\
0.560660540684512	0.00213623046875	\\
0.560704931859546	0.0023193359375	\\
0.560749323034581	0.00225830078125	\\
0.560793714209615	0.001922607421875	\\
0.56083810538465	0.001861572265625	\\
0.560882496559684	0.0020751953125	\\
0.560926887734718	0.00225830078125	\\
0.560971278909753	0.0025634765625	\\
0.561015670084787	0.002227783203125	\\
0.561060061259822	0.0023193359375	\\
0.561104452434856	0.0025634765625	\\
0.56114884360989	0.00238037109375	\\
0.561193234784925	0.002655029296875	\\
0.561237625959959	0.0023193359375	\\
0.561282017134994	0.00201416015625	\\
0.561326408310028	0.00189208984375	\\
0.561370799485062	0.001953125	\\
0.561415190660097	0.00225830078125	\\
0.561459581835131	0.001922607421875	\\
0.561503973010166	0.00225830078125	\\
0.5615483641852	0.00274658203125	\\
0.561592755360234	0.00286865234375	\\
0.561637146535269	0.0030517578125	\\
0.561681537710303	0.002593994140625	\\
0.561725928885338	0.0023193359375	\\
0.561770320060372	0.00244140625	\\
0.561814711235406	0.00225830078125	\\
0.561859102410441	0.0020751953125	\\
0.561903493585475	0.0023193359375	\\
0.56194788476051	0.002288818359375	\\
0.561992275935544	0.0025634765625	\\
0.562036667110578	0.002349853515625	\\
0.562081058285613	0.001861572265625	\\
0.562125449460647	0.002197265625	\\
0.562169840635682	0.00250244140625	\\
0.562214231810716	0.0025634765625	\\
0.56225862298575	0.00299072265625	\\
0.562303014160785	0.0030517578125	\\
0.562347405335819	0.0028076171875	\\
0.562391796510854	0.003387451171875	\\
0.562436187685888	0.003509521484375	\\
0.562480578860922	0.00323486328125	\\
0.562524970035957	0.002899169921875	\\
0.562569361210991	0.002655029296875	\\
0.562613752386026	0.002593994140625	\\
0.56265814356106	0.002532958984375	\\
0.562702534736094	0.00213623046875	\\
0.562746925911129	0.002227783203125	\\
0.562791317086163	0.002838134765625	\\
0.562835708261198	0.002685546875	\\
0.562880099436232	0.0025634765625	\\
0.562924490611266	0.002685546875	\\
0.562968881786301	0.00274658203125	\\
0.563013272961335	0.002471923828125	\\
0.56305766413637	0.0028076171875	\\
0.563102055311404	0.0032958984375	\\
0.563146446486439	0.002899169921875	\\
0.563190837661473	0.0028076171875	\\
0.563235228836507	0.0029296875	\\
0.563279620011542	0.00299072265625	\\
0.563324011186576	0.003143310546875	\\
0.563368402361611	0.0030517578125	\\
0.563412793536645	0.003326416015625	\\
0.563457184711679	0.003387451171875	\\
0.563501575886714	0.003082275390625	\\
0.563545967061748	0.00335693359375	\\
0.563590358236783	0.00335693359375	\\
0.563634749411817	0.00347900390625	\\
0.563679140586851	0.00372314453125	\\
0.563723531761886	0.003387451171875	\\
0.56376792293692	0.00347900390625	\\
0.563812314111955	0.00396728515625	\\
0.563856705286989	0.003753662109375	\\
0.563901096462023	0.003570556640625	\\
0.563945487637058	0.0035400390625	\\
0.563989878812092	0.0035400390625	\\
0.564034269987127	0.003662109375	\\
0.564078661162161	0.003570556640625	\\
0.564123052337195	0.0032958984375	\\
0.56416744351223	0.00299072265625	\\
0.564211834687264	0.003021240234375	\\
0.564256225862299	0.00341796875	\\
0.564300617037333	0.003021240234375	\\
0.564345008212367	0.002532958984375	\\
0.564389399387402	0.002471923828125	\\
0.564433790562436	0.00250244140625	\\
0.564478181737471	0.002685546875	\\
0.564522572912505	0.002838134765625	\\
0.564566964087539	0.00250244140625	\\
0.564611355262574	0.00311279296875	\\
0.564655746437608	0.003265380859375	\\
0.564700137612643	0.002899169921875	\\
0.564744528787677	0.003173828125	\\
0.564788919962711	0.0035400390625	\\
0.564833311137746	0.003387451171875	\\
0.56487770231278	0.003082275390625	\\
0.564922093487815	0.003387451171875	\\
0.564966484662849	0.003173828125	\\
0.565010875837883	0.003662109375	\\
0.565055267012918	0.00396728515625	\\
0.565099658187952	0.003662109375	\\
0.565144049362987	0.0037841796875	\\
0.565188440538021	0.003936767578125	\\
0.565232831713055	0.004486083984375	\\
0.56527722288809	0.004730224609375	\\
0.565321614063124	0.00469970703125	\\
0.565366005238159	0.004852294921875	\\
0.565410396413193	0.005157470703125	\\
0.565454787588227	0.005218505859375	\\
0.565499178763262	0.00543212890625	\\
0.565543569938296	0.005340576171875	\\
0.565587961113331	0.00537109375	\\
0.565632352288365	0.00537109375	\\
0.565676743463399	0.00518798828125	\\
0.565721134638434	0.005218505859375	\\
0.565765525813468	0.005126953125	\\
0.565809916988503	0.0048828125	\\
0.565854308163537	0.005035400390625	\\
0.565898699338572	0.005035400390625	\\
0.565943090513606	0.005157470703125	\\
0.56598748168864	0.005340576171875	\\
0.566031872863675	0.005462646484375	\\
0.566076264038709	0.005462646484375	\\
0.566120655213744	0.005096435546875	\\
0.566165046388778	0.004669189453125	\\
0.566209437563812	0.00445556640625	\\
0.566253828738847	0.004608154296875	\\
0.566298219913881	0.00482177734375	\\
0.566342611088915	0.00457763671875	\\
0.56638700226395	0.00433349609375	\\
0.566431393438984	0.00457763671875	\\
0.566475784614019	0.00457763671875	\\
0.566520175789053	0.0042724609375	\\
0.566564566964088	0.003997802734375	\\
0.566608958139122	0.004058837890625	\\
0.566653349314156	0.00390625	\\
0.566697740489191	0.00341796875	\\
0.566742131664225	0.004150390625	\\
0.56678652283926	0.0037841796875	\\
0.566830914014294	0.003265380859375	\\
0.566875305189328	0.003204345703125	\\
0.566919696364363	0.0030517578125	\\
0.566964087539397	0.003570556640625	\\
0.567008478714432	0.003082275390625	\\
0.567052869889466	0.003204345703125	\\
0.5670972610645	0.003143310546875	\\
0.567141652239535	0.003143310546875	\\
0.567186043414569	0.00335693359375	\\
0.567230434589604	0.003173828125	\\
0.567274825764638	0.00341796875	\\
0.567319216939672	0.00311279296875	\\
0.567363608114707	0.003021240234375	\\
0.567407999289741	0.003265380859375	\\
0.567452390464776	0.00299072265625	\\
0.56749678163981	0.0032958984375	\\
0.567541172814844	0.00286865234375	\\
0.567585563989879	0.002410888671875	\\
0.567629955164913	0.002685546875	\\
0.567674346339948	0.002227783203125	\\
0.567718737514982	0.002166748046875	\\
0.567763128690016	0.002471923828125	\\
0.567807519865051	0.00238037109375	\\
0.567851911040085	0.002349853515625	\\
0.56789630221512	0.002593994140625	\\
0.567940693390154	0.00225830078125	\\
0.567985084565188	0.00225830078125	\\
0.568029475740223	0.00244140625	\\
0.568073866915257	0.0023193359375	\\
0.568118258090292	0.00225830078125	\\
0.568162649265326	0.002227783203125	\\
0.56820704044036	0.00244140625	\\
0.568251431615395	0.003021240234375	\\
0.568295822790429	0.003509521484375	\\
0.568340213965464	0.00311279296875	\\
0.568384605140498	0.002777099609375	\\
0.568428996315532	0.002593994140625	\\
0.568473387490567	0.00250244140625	\\
0.568517778665601	0.00189208984375	\\
0.568562169840636	0.0015869140625	\\
0.56860656101567	0.001678466796875	\\
0.568650952190705	0.001739501953125	\\
0.568695343365739	0.002044677734375	\\
0.568739734540773	0.002197265625	\\
0.568784125715808	0.001953125	\\
0.568828516890842	0.001953125	\\
0.568872908065877	0.002349853515625	\\
0.568917299240911	0.00262451171875	\\
0.568961690415945	0.00262451171875	\\
0.56900608159098	0.002166748046875	\\
0.569050472766014	0.002227783203125	\\
0.569094863941048	0.002716064453125	\\
0.569139255116083	0.002410888671875	\\
0.569183646291117	0.00274658203125	\\
0.569228037466152	0.00311279296875	\\
0.569272428641186	0.0028076171875	\\
0.569316819816221	0.0030517578125	\\
0.569361210991255	0.00299072265625	\\
0.569405602166289	0.002899169921875	\\
0.569449993341324	0.003326416015625	\\
0.569494384516358	0.002899169921875	\\
0.569538775691393	0.002716064453125	\\
0.569583166866427	0.00311279296875	\\
0.569627558041461	0.003265380859375	\\
0.569671949216496	0.003814697265625	\\
0.56971634039153	0.003448486328125	\\
0.569760731566565	0.00323486328125	\\
0.569805122741599	0.00360107421875	\\
0.569849513916633	0.003326416015625	\\
0.569893905091668	0.003570556640625	\\
0.569938296266702	0.003143310546875	\\
0.569982687441737	0.00274658203125	\\
0.570027078616771	0.003143310546875	\\
0.570071469791805	0.003204345703125	\\
0.57011586096684	0.0030517578125	\\
0.570160252141874	0.003021240234375	\\
0.570204643316909	0.003082275390625	\\
0.570249034491943	0.003143310546875	\\
0.570293425666977	0.002960205078125	\\
0.570337816842012	0.002716064453125	\\
0.570382208017046	0.002685546875	\\
0.570426599192081	0.002410888671875	\\
0.570470990367115	0.002960205078125	\\
0.570515381542149	0.003326416015625	\\
0.570559772717184	0.002471923828125	\\
0.570604163892218	0.00250244140625	\\
0.570648555067253	0.002410888671875	\\
0.570692946242287	0.0025634765625	\\
0.570737337417321	0.002655029296875	\\
0.570781728592356	0.002105712890625	\\
0.57082611976739	0.001953125	\\
0.570870510942425	0.001953125	\\
0.570914902117459	0.0020751953125	\\
0.570959293292493	0.002593994140625	\\
0.571003684467528	0.00244140625	\\
0.571048075642562	0.00238037109375	\\
0.571092466817597	0.00238037109375	\\
0.571136857992631	0.002655029296875	\\
0.571181249167665	0.00286865234375	\\
0.5712256403427	0.0028076171875	\\
0.571270031517734	0.0028076171875	\\
0.571314422692769	0.003143310546875	\\
0.571358813867803	0.003509521484375	\\
0.571403205042837	0.00323486328125	\\
0.571447596217872	0.00323486328125	\\
0.571491987392906	0.0032958984375	\\
0.571536378567941	0.00286865234375	\\
0.571580769742975	0.003082275390625	\\
0.57162516091801	0.003143310546875	\\
0.571669552093044	0.00274658203125	\\
0.571713943268078	0.00250244140625	\\
0.571758334443113	0.003021240234375	\\
0.571802725618147	0.0029296875	\\
0.571847116793182	0.00286865234375	\\
0.571891507968216	0.002899169921875	\\
0.57193589914325	0.0029296875	\\
0.571980290318285	0.00311279296875	\\
0.572024681493319	0.00299072265625	\\
0.572069072668354	0.0032958984375	\\
0.572113463843388	0.003570556640625	\\
0.572157855018422	0.00347900390625	\\
0.572202246193457	0.003570556640625	\\
0.572246637368491	0.003814697265625	\\
0.572291028543526	0.004180908203125	\\
0.57233541971856	0.004364013671875	\\
0.572379810893594	0.003570556640625	\\
0.572424202068629	0.003387451171875	\\
0.572468593243663	0.003326416015625	\\
0.572512984418698	0.00274658203125	\\
0.572557375593732	0.002838134765625	\\
0.572601766768766	0.002960205078125	\\
0.572646157943801	0.00335693359375	\\
0.572690549118835	0.00341796875	\\
0.57273494029387	0.00311279296875	\\
0.572779331468904	0.003509521484375	\\
0.572823722643938	0.003265380859375	\\
0.572868113818973	0.003082275390625	\\
0.572912504994007	0.003173828125	\\
0.572956896169042	0.00323486328125	\\
0.573001287344076	0.0030517578125	\\
0.57304567851911	0.0028076171875	\\
0.573090069694145	0.003265380859375	\\
0.573134460869179	0.003509521484375	\\
0.573178852044214	0.003570556640625	\\
0.573223243219248	0.00372314453125	\\
0.573267634394282	0.003509521484375	\\
0.573312025569317	0.0029296875	\\
0.573356416744351	0.003143310546875	\\
0.573400807919386	0.003143310546875	\\
0.57344519909442	0.003509521484375	\\
0.573489590269454	0.00360107421875	\\
0.573533981444489	0.003173828125	\\
0.573578372619523	0.0032958984375	\\
0.573622763794558	0.00341796875	\\
0.573667154969592	0.00299072265625	\\
0.573711546144626	0.002899169921875	\\
0.573755937319661	0.002685546875	\\
0.573800328494695	0.002655029296875	\\
0.57384471966973	0.00262451171875	\\
0.573889110844764	0.0023193359375	\\
0.573933502019798	0.002044677734375	\\
0.573977893194833	0.001861572265625	\\
0.574022284369867	0.001953125	\\
0.574066675544902	0.00189208984375	\\
0.574111066719936	0.00152587890625	\\
0.57415545789497	0.001556396484375	\\
0.574199849070005	0.0015869140625	\\
0.574244240245039	0.001190185546875	\\
0.574288631420074	0.001129150390625	\\
0.574333022595108	0.000732421875	\\
0.574377413770143	0.00091552734375	\\
0.574421804945177	0.0013427734375	\\
0.574466196120211	0.001251220703125	\\
0.574510587295246	0.0010986328125	\\
0.57455497847028	0.001190185546875	\\
0.574599369645315	0.001007080078125	\\
0.574643760820349	0.000885009765625	\\
0.574688151995383	0.001312255859375	\\
0.574732543170418	0.00146484375	\\
0.574776934345452	0.000946044921875	\\
0.574821325520486	0.001373291015625	\\
0.574865716695521	0.00152587890625	\\
0.574910107870555	0.000885009765625	\\
0.57495449904559	0.00103759765625	\\
0.574998890220624	0.001190185546875	\\
0.575043281395659	0.0010986328125	\\
0.575087672570693	0.001922607421875	\\
0.575132063745727	0.002105712890625	\\
0.575176454920762	0.002105712890625	\\
0.575220846095796	0.002166748046875	\\
0.575265237270831	0.00177001953125	\\
0.575309628445865	0.001495361328125	\\
0.575354019620899	0.00140380859375	\\
0.575398410795934	0.000946044921875	\\
0.575442801970968	0.00128173828125	\\
0.575487193146003	0.00128173828125	\\
0.575531584321037	0.000946044921875	\\
0.575575975496071	0.000885009765625	\\
0.575620366671106	0.0009765625	\\
0.57566475784614	0.000885009765625	\\
0.575709149021175	0.000518798828125	\\
0.575753540196209	0.000823974609375	\\
0.575797931371243	0.00091552734375	\\
0.575842322546278	0.0008544921875	\\
0.575886713721312	0.001312255859375	\\
0.575931104896347	0.001129150390625	\\
0.575975496071381	0.000732421875	\\
0.576019887246415	0.00042724609375	\\
0.57606427842145	0.00030517578125	\\
0.576108669596484	0.0003662109375	\\
0.576153060771519	0.00067138671875	\\
0.576197451946553	0.001007080078125	\\
0.576241843121587	0.000701904296875	\\
0.576286234296622	0.00048828125	\\
0.576330625471656	0.00048828125	\\
0.576375016646691	0.00030517578125	\\
0.576419407821725	0.000640869140625	\\
0.576463798996759	0.000640869140625	\\
0.576508190171794	6.103515625e-05	\\
0.576552581346828	3.0517578125e-05	\\
0.576596972521863	9.1552734375e-05	\\
0.576641363696897	-0.000335693359375	\\
0.576685754871931	6.103515625e-05	\\
0.576730146046966	9.1552734375e-05	\\
0.576774537222	-0.000213623046875	\\
0.576818928397035	-3.0517578125e-05	\\
0.576863319572069	0.0003662109375	\\
0.576907710747103	0.00054931640625	\\
0.576952101922138	0.0003662109375	\\
0.576996493097172	0.00042724609375	\\
0.577040884272207	0.000152587890625	\\
0.577085275447241	0	\\
0.577129666622275	0.000274658203125	\\
0.57717405779731	0.0001220703125	\\
0.577218448972344	-0.00018310546875	\\
0.577262840147379	-0.0001220703125	\\
0.577307231322413	-0.000152587890625	\\
0.577351622497448	-0.000244140625	\\
0.577396013672482	-0.000396728515625	\\
0.577440404847516	-0.000335693359375	\\
0.577484796022551	-9.1552734375e-05	\\
0.577529187197585	3.0517578125e-05	\\
0.57757357837262	-0.0003662109375	\\
0.577617969547654	-0.000762939453125	\\
0.577662360722688	-0.00067138671875	\\
0.577706751897723	-0.000274658203125	\\
0.577751143072757	0	\\
0.577795534247792	-3.0517578125e-05	\\
0.577839925422826	-0.0001220703125	\\
0.57788431659786	6.103515625e-05	\\
0.577928707772895	9.1552734375e-05	\\
0.577973098947929	-9.1552734375e-05	\\
0.578017490122964	-0.000213623046875	\\
0.578061881297998	-0.0001220703125	\\
0.578106272473032	0.000457763671875	\\
0.578150663648067	0.001007080078125	\\
0.578195054823101	0.00048828125	\\
0.578239445998136	0.000579833984375	\\
0.57828383717317	0.000579833984375	\\
0.578328228348204	0.0003662109375	\\
0.578372619523239	0.000732421875	\\
0.578417010698273	0.0003662109375	\\
0.578461401873308	0.00018310546875	\\
0.578505793048342	6.103515625e-05	\\
0.578550184223376	0.0001220703125	\\
0.578594575398411	-6.103515625e-05	\\
0.578638966573445	-0.000244140625	\\
0.57868335774848	0.0001220703125	\\
0.578727748923514	3.0517578125e-05	\\
0.578772140098548	-0.000335693359375	\\
0.578816531273583	-0.00018310546875	\\
0.578860922448617	-9.1552734375e-05	\\
0.578905313623652	-0.000335693359375	\\
0.578949704798686	-9.1552734375e-05	\\
0.57899409597372	-3.0517578125e-05	\\
0.579038487148755	-0.000335693359375	\\
0.579082878323789	-0.00018310546875	\\
0.579127269498824	-0.0001220703125	\\
0.579171660673858	-0.000335693359375	\\
0.579216051848892	-0.000335693359375	\\
0.579260443023927	0.0001220703125	\\
0.579304834198961	0.000640869140625	\\
0.579349225373996	0.000244140625	\\
0.57939361654903	-0.000244140625	\\
0.579438007724065	6.103515625e-05	\\
0.579482398899099	0.0003662109375	\\
0.579526790074133	0.000579833984375	\\
0.579571181249168	0.000244140625	\\
0.579615572424202	-0.000152587890625	\\
0.579659963599236	0.000244140625	\\
0.579704354774271	0.000518798828125	\\
0.579748745949305	0.00042724609375	\\
0.57979313712434	0.00018310546875	\\
0.579837528299374	9.1552734375e-05	\\
0.579881919474408	0.00018310546875	\\
0.579926310649443	0.0003662109375	\\
0.579970701824477	0.001220703125	\\
0.580015092999512	0.00103759765625	\\
0.580059484174546	0.0009765625	\\
0.580103875349581	0.000823974609375	\\
0.580148266524615	0.000732421875	\\
0.580192657699649	0.000732421875	\\
0.580237048874684	0.000213623046875	\\
0.580281440049718	0.00042724609375	\\
0.580325831224753	0.00054931640625	\\
0.580370222399787	0.000640869140625	\\
0.580414613574821	0.00103759765625	\\
0.580459004749856	0.000579833984375	\\
0.58050339592489	0.00048828125	\\
0.580547787099924	0.000640869140625	\\
0.580592178274959	0.000213623046875	\\
0.580636569449993	3.0517578125e-05	\\
0.580680960625028	3.0517578125e-05	\\
0.580725351800062	-0.000244140625	\\
0.580769742975097	-0.000274658203125	\\
0.580814134150131	-3.0517578125e-05	\\
0.580858525325165	-6.103515625e-05	\\
0.5809029165002	-3.0517578125e-05	\\
0.580947307675234	0.00018310546875	\\
0.580991698850269	0.00018310546875	\\
0.581036090025303	0	\\
0.581080481200337	0.000579833984375	\\
0.581124872375372	0.00054931640625	\\
0.581169263550406	0.0006103515625	\\
0.581213654725441	0.0006103515625	\\
0.581258045900475	0.00042724609375	\\
0.581302437075509	0.000701904296875	\\
0.581346828250544	0.000762939453125	\\
0.581391219425578	0.000701904296875	\\
0.581435610600613	0.00018310546875	\\
0.581480001775647	-0.000244140625	\\
0.581524392950681	9.1552734375e-05	\\
0.581568784125716	0.000213623046875	\\
0.58161317530075	-0.00042724609375	\\
0.581657566475785	-0.000518798828125	\\
0.581701957650819	-0.00054931640625	\\
0.581746348825853	-0.000701904296875	\\
0.581790740000888	-0.000244140625	\\
0.581835131175922	-0.000274658203125	\\
0.581879522350957	-0.00048828125	\\
0.581923913525991	-0.000762939453125	\\
0.581968304701025	-0.00140380859375	\\
0.58201269587606	-0.001312255859375	\\
0.582057087051094	-0.0015869140625	\\
0.582101478226129	-0.001068115234375	\\
0.582145869401163	-0.00067138671875	\\
0.582190260576197	-0.000946044921875	\\
0.582234651751232	-0.001129150390625	\\
0.582279042926266	-0.001190185546875	\\
0.582323434101301	-0.00128173828125	\\
0.582367825276335	-0.001434326171875	\\
0.582412216451369	-0.0009765625	\\
0.582456607626404	-0.00146484375	\\
0.582500998801438	-0.00177001953125	\\
0.582545389976473	-0.001678466796875	\\
0.582589781151507	-0.001739501953125	\\
0.582634172326541	-0.001617431640625	\\
0.582678563501576	-0.00140380859375	\\
0.58272295467661	-0.001708984375	\\
0.582767345851645	-0.001739501953125	\\
0.582811737026679	-0.002197265625	\\
0.582856128201714	-0.001434326171875	\\
0.582900519376748	-0.000823974609375	\\
0.582944910551782	-0.001068115234375	\\
0.582989301726817	-0.000885009765625	\\
0.583033692901851	-0.000640869140625	\\
0.583078084076886	-0.00103759765625	\\
0.58312247525192	-0.001190185546875	\\
0.583166866426954	-0.0008544921875	\\
0.583211257601989	-0.001312255859375	\\
0.583255648777023	-0.001068115234375	\\
0.583300039952057	-0.00128173828125	\\
0.583344431127092	-0.00146484375	\\
0.583388822302126	-0.001495361328125	\\
0.583433213477161	-0.001922607421875	\\
0.583477604652195	-0.0018310546875	\\
0.58352199582723	-0.002166748046875	\\
0.583566387002264	-0.001922607421875	\\
0.583610778177298	-0.0015869140625	\\
0.583655169352333	-0.002044677734375	\\
0.583699560527367	-0.00189208984375	\\
0.583743951702402	-0.001800537109375	\\
0.583788342877436	-0.002105712890625	\\
0.58383273405247	-0.001922607421875	\\
0.583877125227505	-0.0020751953125	\\
0.583921516402539	-0.0018310546875	\\
0.583965907577574	-0.001861572265625	\\
0.584010298752608	-0.00189208984375	\\
0.584054689927642	-0.00189208984375	\\
0.584099081102677	-0.00250244140625	\\
0.584143472277711	-0.00213623046875	\\
0.584187863452746	-0.001861572265625	\\
0.58423225462778	-0.00189208984375	\\
0.584276645802814	-0.001922607421875	\\
0.584321036977849	-0.001800537109375	\\
0.584365428152883	-0.001556396484375	\\
0.584409819327918	-0.00164794921875	\\
0.584454210502952	-0.001739501953125	\\
0.584498601677986	-0.0015869140625	\\
0.584542992853021	-0.001556396484375	\\
0.584587384028055	-0.001983642578125	\\
0.58463177520309	-0.001617431640625	\\
0.584676166378124	-0.001190185546875	\\
0.584720557553158	-0.00177001953125	\\
0.584764948728193	-0.0013427734375	\\
0.584809339903227	-0.00103759765625	\\
0.584853731078262	-0.001220703125	\\
0.584898122253296	-0.001068115234375	\\
0.58494251342833	-0.0009765625	\\
0.584986904603365	-0.001190185546875	\\
0.585031295778399	-0.001312255859375	\\
0.585075686953434	-0.0010986328125	\\
0.585120078128468	-0.001251220703125	\\
0.585164469303502	-0.001220703125	\\
0.585208860478537	-0.001220703125	\\
0.585253251653571	-0.00140380859375	\\
0.585297642828606	-0.00103759765625	\\
0.58534203400364	-0.001129150390625	\\
0.585386425178674	-0.00128173828125	\\
0.585430816353709	-0.00079345703125	\\
0.585475207528743	-0.001251220703125	\\
0.585519598703778	-0.001617431640625	\\
0.585563989878812	-0.001678466796875	\\
0.585608381053846	-0.001556396484375	\\
0.585652772228881	-0.00146484375	\\
0.585697163403915	-0.001556396484375	\\
0.58574155457895	-0.00146484375	\\
0.585785945753984	-0.00152587890625	\\
0.585830336929019	-0.001739501953125	\\
0.585874728104053	-0.001434326171875	\\
0.585919119279087	-0.0015869140625	\\
0.585963510454122	-0.0020751953125	\\
0.586007901629156	-0.001708984375	\\
0.586052292804191	-0.001617431640625	\\
0.586096683979225	-0.0018310546875	\\
0.586141075154259	-0.00140380859375	\\
0.586185466329294	-0.0013427734375	\\
0.586229857504328	-0.001800537109375	\\
0.586274248679363	-0.001739501953125	\\
0.586318639854397	-0.001251220703125	\\
0.586363031029431	-0.00115966796875	\\
0.586407422204466	-0.0013427734375	\\
0.5864518133795	-0.00128173828125	\\
0.586496204554535	-0.00128173828125	\\
0.586540595729569	-0.00146484375	\\
0.586584986904603	-0.001434326171875	\\
0.586629378079638	-0.001434326171875	\\
0.586673769254672	-0.001434326171875	\\
0.586718160429707	-0.001312255859375	\\
0.586762551604741	-0.00140380859375	\\
0.586806942779775	-0.001434326171875	\\
0.58685133395481	-0.001129150390625	\\
0.586895725129844	-0.00103759765625	\\
0.586940116304879	-0.0009765625	\\
0.586984507479913	-0.001312255859375	\\
0.587028898654947	-0.0013427734375	\\
0.587073289829982	-0.001068115234375	\\
0.587117681005016	-0.001251220703125	\\
0.587162072180051	-0.00115966796875	\\
0.587206463355085	-0.000946044921875	\\
0.587250854530119	-0.001129150390625	\\
0.587295245705154	-0.00146484375	\\
0.587339636880188	-0.001251220703125	\\
0.587384028055223	-0.000946044921875	\\
0.587428419230257	-0.0008544921875	\\
0.587472810405291	-0.000762939453125	\\
0.587517201580326	-0.0008544921875	\\
0.58756159275536	-0.00067138671875	\\
0.587605983930395	-0.00079345703125	\\
0.587650375105429	-0.0009765625	\\
0.587694766280463	-0.001007080078125	\\
0.587739157455498	-0.00091552734375	\\
0.587783548630532	-0.0008544921875	\\
0.587827939805567	-0.001129150390625	\\
0.587872330980601	-0.00079345703125	\\
0.587916722155636	-0.001129150390625	\\
0.58796111333067	-0.00146484375	\\
0.588005504505704	-0.001312255859375	\\
0.588049895680739	-0.00115966796875	\\
0.588094286855773	-0.0010986328125	\\
0.588138678030807	-0.001190185546875	\\
0.588183069205842	-0.0009765625	\\
0.588227460380876	-0.001373291015625	\\
0.588271851555911	-0.001739501953125	\\
0.588316242730945	-0.001495361328125	\\
0.588360633905979	-0.001495361328125	\\
0.588405025081014	-0.00164794921875	\\
0.588449416256048	-0.001800537109375	\\
0.588493807431083	-0.00225830078125	\\
0.588538198606117	-0.001800537109375	\\
0.588582589781152	-0.001495361328125	\\
0.588626980956186	-0.0015869140625	\\
0.58867137213122	-0.001708984375	\\
0.588715763306255	-0.002105712890625	\\
0.588760154481289	-0.00225830078125	\\
0.588804545656324	-0.001861572265625	\\
0.588848936831358	-0.00177001953125	\\
0.588893328006392	-0.001739501953125	\\
0.588937719181427	-0.00177001953125	\\
0.588982110356461	-0.001678466796875	\\
0.589026501531495	-0.001495361328125	\\
0.58907089270653	-0.00201416015625	\\
0.589115283881564	-0.0018310546875	\\
0.589159675056599	-0.0018310546875	\\
0.589204066231633	-0.00225830078125	\\
0.589248457406668	-0.002227783203125	\\
0.589292848581702	-0.001983642578125	\\
0.589337239756736	-0.00189208984375	\\
0.589381630931771	-0.001922607421875	\\
0.589426022106805	-0.001953125	\\
0.58947041328184	-0.002044677734375	\\
0.589514804456874	-0.001953125	\\
0.589559195631908	-0.001800537109375	\\
0.589603586806943	-0.0018310546875	\\
0.589647977981977	-0.002166748046875	\\
0.589692369157012	-0.001739501953125	\\
0.589736760332046	-0.001190185546875	\\
0.58978115150708	-0.001251220703125	\\
0.589825542682115	-0.001190185546875	\\
0.589869933857149	-0.001373291015625	\\
0.589914325032184	-0.001434326171875	\\
0.589958716207218	-0.00103759765625	\\
0.590003107382252	-0.001068115234375	\\
0.590047498557287	-0.001708984375	\\
0.590091889732321	-0.001617431640625	\\
0.590136280907356	-0.0015869140625	\\
0.59018067208239	-0.001922607421875	\\
0.590225063257424	-0.0015869140625	\\
0.590269454432459	-0.001678466796875	\\
0.590313845607493	-0.001922607421875	\\
0.590358236782528	-0.001861572265625	\\
0.590402627957562	-0.00189208984375	\\
0.590447019132596	-0.001434326171875	\\
0.590491410307631	-0.00164794921875	\\
0.590535801482665	-0.001800537109375	\\
0.5905801926577	-0.00177001953125	\\
0.590624583832734	-0.001678466796875	\\
0.590668975007768	-0.0013427734375	\\
0.590713366182803	-0.00115966796875	\\
0.590757757357837	-0.00103759765625	\\
0.590802148532872	-0.000823974609375	\\
0.590846539707906	-0.00048828125	\\
0.59089093088294	-0.0009765625	\\
0.590935322057975	-0.001251220703125	\\
0.590979713233009	-0.0008544921875	\\
0.591024104408044	-0.00128173828125	\\
0.591068495583078	-0.001007080078125	\\
0.591112886758112	-0.00128173828125	\\
0.591157277933147	-0.001708984375	\\
0.591201669108181	-0.00128173828125	\\
0.591246060283216	-0.001556396484375	\\
0.59129045145825	-0.00152587890625	\\
0.591334842633285	-0.001708984375	\\
0.591379233808319	-0.002044677734375	\\
0.591423624983353	-0.001434326171875	\\
0.591468016158388	-0.001190185546875	\\
0.591512407333422	-0.001373291015625	\\
0.591556798508457	-0.001129150390625	\\
0.591601189683491	-0.00140380859375	\\
0.591645580858525	-0.001312255859375	\\
0.59168997203356	-0.001220703125	\\
0.591734363208594	-0.001220703125	\\
0.591778754383628	-0.001190185546875	\\
0.591823145558663	-0.00140380859375	\\
0.591867536733697	-0.001983642578125	\\
0.591911927908732	-0.00238037109375	\\
0.591956319083766	-0.002471923828125	\\
0.592000710258801	-0.002593994140625	\\
0.592045101433835	-0.0028076171875	\\
0.592089492608869	-0.002960205078125	\\
0.592133883783904	-0.0030517578125	\\
0.592178274958938	-0.003143310546875	\\
0.592222666133973	-0.00286865234375	\\
0.592267057309007	-0.003021240234375	\\
0.592311448484041	-0.0029296875	\\
0.592355839659076	-0.0029296875	\\
0.59240023083411	-0.00323486328125	\\
0.592444622009145	-0.003173828125	\\
0.592489013184179	-0.003143310546875	\\
0.592533404359213	-0.002838134765625	\\
0.592577795534248	-0.002685546875	\\
0.592622186709282	-0.00274658203125	\\
0.592666577884317	-0.002471923828125	\\
0.592710969059351	-0.002471923828125	\\
0.592755360234385	-0.0028076171875	\\
0.59279975140942	-0.002655029296875	\\
0.592844142584454	-0.002777099609375	\\
0.592888533759489	-0.002410888671875	\\
0.592932924934523	-0.00238037109375	\\
0.592977316109557	-0.0023193359375	\\
0.593021707284592	-0.002227783203125	\\
0.593066098459626	-0.002655029296875	\\
0.593110489634661	-0.002471923828125	\\
0.593154880809695	-0.001953125	\\
0.593199271984729	-0.002532958984375	\\
0.593243663159764	-0.002685546875	\\
0.593288054334798	-0.0023193359375	\\
0.593332445509833	-0.002685546875	\\
0.593376836684867	-0.003204345703125	\\
0.593421227859901	-0.003173828125	\\
0.593465619034936	-0.002777099609375	\\
0.59351001020997	-0.0025634765625	\\
0.593554401385005	-0.002471923828125	\\
0.593598792560039	-0.002197265625	\\
0.593643183735074	-0.002044677734375	\\
0.593687574910108	-0.002410888671875	\\
0.593731966085142	-0.002288818359375	\\
0.593776357260177	-0.00244140625	\\
0.593820748435211	-0.002960205078125	\\
0.593865139610245	-0.00311279296875	\\
0.59390953078528	-0.0029296875	\\
0.593953921960314	-0.00262451171875	\\
0.593998313135349	-0.002716064453125	\\
0.594042704310383	-0.0029296875	\\
0.594087095485417	-0.0030517578125	\\
0.594131486660452	-0.002899169921875	\\
0.594175877835486	-0.0028076171875	\\
0.594220269010521	-0.003082275390625	\\
0.594264660185555	-0.002532958984375	\\
0.59430905136059	-0.002655029296875	\\
0.594353442535624	-0.003143310546875	\\
0.594397833710658	-0.0028076171875	\\
0.594442224885693	-0.003082275390625	\\
0.594486616060727	-0.00286865234375	\\
0.594531007235762	-0.00244140625	\\
0.594575398410796	-0.00244140625	\\
0.59461978958583	-0.00225830078125	\\
0.594664180760865	-0.002349853515625	\\
0.594708571935899	-0.002105712890625	\\
0.594752963110934	-0.001495361328125	\\
0.594797354285968	-0.001861572265625	\\
0.594841745461002	-0.00189208984375	\\
0.594886136636037	-0.001861572265625	\\
0.594930527811071	-0.002471923828125	\\
0.594974918986106	-0.002532958984375	\\
0.59501931016114	-0.0023193359375	\\
0.595063701336174	-0.00225830078125	\\
0.595108092511209	-0.002471923828125	\\
0.595152483686243	-0.002593994140625	\\
0.595196874861278	-0.00244140625	\\
0.595241266036312	-0.002410888671875	\\
0.595285657211346	-0.00225830078125	\\
0.595330048386381	-0.0028076171875	\\
0.595374439561415	-0.002899169921875	\\
0.59541883073645	-0.003082275390625	\\
0.595463221911484	-0.00341796875	\\
0.595507613086518	-0.003631591796875	\\
0.595552004261553	-0.0040283203125	\\
0.595596395436587	-0.003997802734375	\\
0.595640786611622	-0.004486083984375	\\
0.595685177786656	-0.004364013671875	\\
0.59572956896169	-0.0042724609375	\\
0.595773960136725	-0.0042724609375	\\
0.595818351311759	-0.004364013671875	\\
0.595862742486794	-0.00433349609375	\\
0.595907133661828	-0.004180908203125	\\
0.595951524836862	-0.003814697265625	\\
0.595995916011897	-0.003204345703125	\\
0.596040307186931	-0.00347900390625	\\
0.596084698361966	-0.003204345703125	\\
0.596129089537	-0.003082275390625	\\
0.596173480712034	-0.003265380859375	\\
0.596217871887069	-0.0025634765625	\\
0.596262263062103	-0.002685546875	\\
0.596306654237138	-0.002655029296875	\\
0.596351045412172	-0.00274658203125	\\
0.596395436587207	-0.002655029296875	\\
0.596439827762241	-0.002288818359375	\\
0.596484218937275	-0.002593994140625	\\
0.59652861011231	-0.002349853515625	\\
0.596573001287344	-0.002532958984375	\\
0.596617392462378	-0.00286865234375	\\
0.596661783637413	-0.002685546875	\\
0.596706174812447	-0.002532958984375	\\
0.596750565987482	-0.002532958984375	\\
0.596794957162516	-0.002716064453125	\\
0.59683934833755	-0.002593994140625	\\
0.596883739512585	-0.00250244140625	\\
0.596928130687619	-0.00238037109375	\\
0.596972521862654	-0.0023193359375	\\
0.597016913037688	-0.0023193359375	\\
0.597061304212723	-0.002532958984375	\\
0.597105695387757	-0.00250244140625	\\
0.597150086562791	-0.001739501953125	\\
0.597194477737826	-0.00189208984375	\\
0.59723886891286	-0.001953125	\\
0.597283260087895	-0.001434326171875	\\
0.597327651262929	-0.001373291015625	\\
0.597372042437963	-0.001617431640625	\\
0.597416433612998	-0.001739501953125	\\
0.597460824788032	-0.001617431640625	\\
0.597505215963066	-0.0015869140625	\\
0.597549607138101	-0.00177001953125	\\
0.597593998313135	-0.002044677734375	\\
0.59763838948817	-0.002410888671875	\\
0.597682780663204	-0.0020751953125	\\
0.597727171838239	-0.002044677734375	\\
0.597771563013273	-0.002105712890625	\\
0.597815954188307	-0.001922607421875	\\
0.597860345363342	-0.001739501953125	\\
0.597904736538376	-0.00177001953125	\\
0.597949127713411	-0.00140380859375	\\
0.597993518888445	-0.00091552734375	\\
0.598037910063479	-0.00067138671875	\\
0.598082301238514	-0.00018310546875	\\
0.598126692413548	-0.00042724609375	\\
0.598171083588583	-0.000518798828125	\\
0.598215474763617	-0.00048828125	\\
0.598259865938651	-0.000732421875	\\
0.598304257113686	-0.000518798828125	\\
0.59834864828872	-0.001007080078125	\\
0.598393039463755	-0.00128173828125	\\
0.598437430638789	-0.001190185546875	\\
0.598481821813823	-0.001007080078125	\\
0.598526212988858	-0.001190185546875	\\
0.598570604163892	-0.001190185546875	\\
0.598614995338927	-0.001495361328125	\\
0.598659386513961	-0.001190185546875	\\
0.598703777688995	-0.001312255859375	\\
0.59874816886403	-0.00189208984375	\\
0.598792560039064	-0.002105712890625	\\
0.598836951214099	-0.0025634765625	\\
0.598881342389133	-0.00244140625	\\
0.598925733564167	-0.002349853515625	\\
0.598970124739202	-0.00262451171875	\\
0.599014515914236	-0.00262451171875	\\
0.599058907089271	-0.002532958984375	\\
0.599103298264305	-0.002166748046875	\\
0.599147689439339	-0.002105712890625	\\
0.599192080614374	-0.002105712890625	\\
0.599236471789408	-0.001983642578125	\\
0.599280862964443	-0.00213623046875	\\
0.599325254139477	-0.002166748046875	\\
0.599369645314511	-0.002166748046875	\\
0.599414036489546	-0.002227783203125	\\
0.59945842766458	-0.00244140625	\\
0.599502818839615	-0.002532958984375	\\
0.599547210014649	-0.002410888671875	\\
0.599591601189683	-0.0020751953125	\\
0.599635992364718	-0.001861572265625	\\
0.599680383539752	-0.002197265625	\\
0.599724774714787	-0.002227783203125	\\
0.599769165889821	-0.00238037109375	\\
0.599813557064856	-0.00244140625	\\
0.59985794823989	-0.001800537109375	\\
0.599902339414924	-0.00177001953125	\\
0.599946730589959	-0.002105712890625	\\
0.599991121764993	-0.0023193359375	\\
0.600035512940028	-0.0023193359375	\\
0.600079904115062	-0.002044677734375	\\
0.600124295290096	-0.001739501953125	\\
0.600168686465131	-0.001617431640625	\\
0.600213077640165	-0.00164794921875	\\
0.6002574688152	-0.001251220703125	\\
0.600301859990234	-0.0013427734375	\\
0.600346251165268	-0.001678466796875	\\
0.600390642340303	-0.001434326171875	\\
0.600435033515337	-0.001556396484375	\\
0.600479424690372	-0.0015869140625	\\
0.600523815865406	-0.001373291015625	\\
0.60056820704044	-0.001251220703125	\\
0.600612598215475	-0.001556396484375	\\
0.600656989390509	-0.001007080078125	\\
0.600701380565544	-0.00079345703125	\\
0.600745771740578	-0.001190185546875	\\
0.600790162915612	-0.00152587890625	\\
0.600834554090647	-0.001373291015625	\\
0.600878945265681	-0.001190185546875	\\
0.600923336440716	-0.001678466796875	\\
0.60096772761575	-0.00201416015625	\\
0.601012118790784	-0.001953125	\\
0.601056509965819	-0.001708984375	\\
0.601100901140853	-0.00189208984375	\\
0.601145292315888	-0.002044677734375	\\
0.601189683490922	-0.00250244140625	\\
0.601234074665956	-0.00244140625	\\
0.601278465840991	-0.002044677734375	\\
0.601322857016025	-0.002166748046875	\\
0.60136724819106	-0.00201416015625	\\
0.601411639366094	-0.0023193359375	\\
0.601456030541128	-0.00250244140625	\\
0.601500421716163	-0.002777099609375	\\
0.601544812891197	-0.003082275390625	\\
0.601589204066232	-0.00274658203125	\\
0.601633595241266	-0.003021240234375	\\
0.6016779864163	-0.0032958984375	\\
0.601722377591335	-0.00347900390625	\\
0.601766768766369	-0.00408935546875	\\
0.601811159941404	-0.003875732421875	\\
0.601855551116438	-0.0037841796875	\\
0.601899942291472	-0.003875732421875	\\
0.601944333466507	-0.003509521484375	\\
0.601988724641541	-0.00390625	\\
0.602033115816576	-0.004180908203125	\\
0.60207750699161	-0.004180908203125	\\
0.602121898166645	-0.00390625	\\
0.602166289341679	-0.00396728515625	\\
0.602210680516713	-0.00396728515625	\\
0.602255071691748	-0.00372314453125	\\
0.602299462866782	-0.00433349609375	\\
0.602343854041816	-0.004058837890625	\\
0.602388245216851	-0.0037841796875	\\
0.602432636391885	-0.00347900390625	\\
0.60247702756692	-0.002532958984375	\\
0.602521418741954	-0.002227783203125	\\
0.602565809916988	-0.0025634765625	\\
0.602610201092023	-0.00262451171875	\\
0.602654592267057	-0.00311279296875	\\
0.602698983442092	-0.003204345703125	\\
0.602743374617126	-0.002777099609375	\\
0.602787765792161	-0.002593994140625	\\
0.602832156967195	-0.00225830078125	\\
0.602876548142229	-0.0018310546875	\\
0.602920939317264	-0.001861572265625	\\
0.602965330492298	-0.001861572265625	\\
0.603009721667333	-0.001678466796875	\\
0.603054112842367	-0.00213623046875	\\
0.603098504017401	-0.0023193359375	\\
0.603142895192436	-0.002288818359375	\\
0.60318728636747	-0.0028076171875	\\
0.603231677542505	-0.0030517578125	\\
0.603276068717539	-0.002471923828125	\\
0.603320459892573	-0.002166748046875	\\
0.603364851067608	-0.002166748046875	\\
0.603409242242642	-0.001983642578125	\\
0.603453633417677	-0.00225830078125	\\
0.603498024592711	-0.002227783203125	\\
0.603542415767745	-0.001800537109375	\\
0.60358680694278	-0.0018310546875	\\
0.603631198117814	-0.002349853515625	\\
0.603675589292849	-0.00244140625	\\
0.603719980467883	-0.002349853515625	\\
0.603764371642917	-0.002197265625	\\
0.603808762817952	-0.0018310546875	\\
0.603853153992986	-0.0020751953125	\\
0.603897545168021	-0.001922607421875	\\
0.603941936343055	-0.0013427734375	\\
0.603986327518089	-0.001617431640625	\\
0.604030718693124	-0.0015869140625	\\
0.604075109868158	-0.001068115234375	\\
0.604119501043193	-0.000885009765625	\\
0.604163892218227	-0.001220703125	\\
0.604208283393261	-0.0015869140625	\\
0.604252674568296	-0.0008544921875	\\
0.60429706574333	-0.000152587890625	\\
0.604341456918365	-0.000274658203125	\\
0.604385848093399	-0.0001220703125	\\
0.604430239268433	0.00042724609375	\\
0.604474630443468	0.000518798828125	\\
0.604519021618502	0.0006103515625	\\
0.604563412793537	0.000946044921875	\\
0.604607803968571	0.000762939453125	\\
0.604652195143605	0.0008544921875	\\
0.60469658631864	0.001220703125	\\
0.604740977493674	0.001373291015625	\\
0.604785368668709	0.00146484375	\\
0.604829759843743	0.001312255859375	\\
0.604874151018778	0.001129150390625	\\
0.604918542193812	0.001220703125	\\
0.604962933368846	0.00115966796875	\\
0.605007324543881	0.000457763671875	\\
0.605051715718915	0.000457763671875	\\
0.605096106893949	6.103515625e-05	\\
0.605140498068984	-0.00048828125	\\
0.605184889244018	-0.000518798828125	\\
0.605229280419053	-0.000640869140625	\\
0.605273671594087	-0.000732421875	\\
0.605318062769121	-0.0009765625	\\
0.605362453944156	-0.00079345703125	\\
0.60540684511919	-0.00079345703125	\\
0.605451236294225	-0.00067138671875	\\
0.605495627469259	0.00018310546875	\\
0.605540018644294	-0.000152587890625	\\
0.605584409819328	-0.000457763671875	\\
0.605628800994362	-6.103515625e-05	\\
0.605673192169397	-6.103515625e-05	\\
0.605717583344431	-0.000396728515625	\\
0.605761974519466	0.000396728515625	\\
0.6058063656945	0.000762939453125	\\
0.605850756869534	9.1552734375e-05	\\
0.605895148044569	-0.000274658203125	\\
0.605939539219603	-0.00048828125	\\
0.605983930394637	-0.00042724609375	\\
0.606028321569672	-6.103515625e-05	\\
0.606072712744706	0.000274658203125	\\
0.606117103919741	0.0003662109375	\\
0.606161495094775	0.00048828125	\\
0.60620588626981	0.000518798828125	\\
0.606250277444844	0.000457763671875	\\
0.606294668619878	0.0003662109375	\\
0.606339059794913	0.000335693359375	\\
0.606383450969947	0.000732421875	\\
0.606427842144982	0.000579833984375	\\
0.606472233320016	0.000457763671875	\\
0.60651662449505	0.00103759765625	\\
0.606561015670085	0.0010986328125	\\
0.606605406845119	0.001129150390625	\\
0.606649798020154	0.00128173828125	\\
0.606694189195188	0.001220703125	\\
0.606738580370222	0.000885009765625	\\
0.606782971545257	0.000732421875	\\
0.606827362720291	0.000396728515625	\\
0.606871753895326	-6.103515625e-05	\\
0.60691614507036	-0.0001220703125	\\
0.606960536245394	-3.0517578125e-05	\\
0.607004927420429	6.103515625e-05	\\
0.607049318595463	-0.0001220703125	\\
0.607093709770498	-0.0001220703125	\\
0.607138100945532	-9.1552734375e-05	\\
0.607182492120566	0.000457763671875	\\
0.607226883295601	0.000640869140625	\\
0.607271274470635	0.001007080078125	\\
0.60731566564567	0.000823974609375	\\
0.607360056820704	0.00067138671875	\\
0.607404447995738	0.000762939453125	\\
0.607448839170773	0.000518798828125	\\
0.607493230345807	0.000274658203125	\\
0.607537621520842	-0.00018310546875	\\
0.607582012695876	-0.000335693359375	\\
0.60762640387091	-0.00067138671875	\\
0.607670795045945	-0.00030517578125	\\
0.607715186220979	0.0001220703125	\\
0.607759577396014	0.000213623046875	\\
0.607803968571048	0.00030517578125	\\
0.607848359746083	0	\\
0.607892750921117	6.103515625e-05	\\
0.607937142096151	0	\\
0.607981533271186	-0.00042724609375	\\
0.60802592444622	0	\\
0.608070315621254	-0.00018310546875	\\
0.608114706796289	-0.00042724609375	\\
0.608159097971323	-9.1552734375e-05	\\
0.608203489146358	-0.000152587890625	\\
0.608247880321392	-0.00030517578125	\\
0.608292271496427	-0.00054931640625	\\
0.608336662671461	-0.000823974609375	\\
0.608381053846495	-0.0010986328125	\\
0.60842544502153	-0.001220703125	\\
0.608469836196564	-0.00115966796875	\\
0.608514227371599	-0.001495361328125	\\
0.608558618546633	-0.001434326171875	\\
0.608603009721667	-0.00091552734375	\\
0.608647400896702	-0.00103759765625	\\
0.608691792071736	-0.001220703125	\\
0.608736183246771	-0.001251220703125	\\
0.608780574421805	-0.001617431640625	\\
0.608824965596839	-0.0015869140625	\\
0.608869356771874	-0.00128173828125	\\
0.608913747946908	-0.000701904296875	\\
0.608958139121943	-0.000244140625	\\
0.609002530296977	-0.0003662109375	\\
0.609046921472011	-0.000457763671875	\\
0.609091312647046	-0.000274658203125	\\
0.60913570382208	-0.00018310546875	\\
0.609180094997115	-0.000396728515625	\\
0.609224486172149	-0.00079345703125	\\
0.609268877347183	-0.000457763671875	\\
0.609313268522218	-0.000152587890625	\\
0.609357659697252	-0.000244140625	\\
0.609402050872287	-0.0001220703125	\\
0.609446442047321	-0.0001220703125	\\
0.609490833222355	-0.0001220703125	\\
0.60953522439739	-0.0003662109375	\\
0.609579615572424	-0.00042724609375	\\
0.609624006747459	-0.000244140625	\\
0.609668397922493	-0.00030517578125	\\
0.609712789097527	-0.00048828125	\\
0.609757180272562	-0.000244140625	\\
0.609801571447596	0	\\
0.609845962622631	0.000396728515625	\\
0.609890353797665	0.000244140625	\\
0.609934744972699	-0.000244140625	\\
0.609979136147734	-0.000457763671875	\\
0.610023527322768	-0.000274658203125	\\
0.610067918497803	0.0001220703125	\\
0.610112309672837	3.0517578125e-05	\\
0.610156700847871	0.00030517578125	\\
0.610201092022906	0.00048828125	\\
0.61024548319794	0.00079345703125	\\
0.610289874372975	0.00091552734375	\\
0.610334265548009	0.000701904296875	\\
0.610378656723043	0.001007080078125	\\
0.610423047898078	0.001190185546875	\\
0.610467439073112	0.001007080078125	\\
0.610511830248147	0.000946044921875	\\
0.610556221423181	0.001129150390625	\\
0.610600612598216	0.0008544921875	\\
0.61064500377325	0.000762939453125	\\
0.610689394948284	0.000946044921875	\\
0.610733786123319	0.000885009765625	\\
0.610778177298353	0.00128173828125	\\
0.610822568473387	0.001617431640625	\\
0.610866959648422	0.001190185546875	\\
0.610911350823456	0.001556396484375	\\
0.610955741998491	0.001739501953125	\\
0.611000133173525	0.001495361328125	\\
0.611044524348559	0.00152587890625	\\
0.611088915523594	0.0015869140625	\\
0.611133306698628	0.001739501953125	\\
0.611177697873663	0.001556396484375	\\
0.611222089048697	0.00177001953125	\\
0.611266480223732	0.001708984375	\\
0.611310871398766	0.001312255859375	\\
0.6113552625738	0.001068115234375	\\
0.611399653748835	0.0010986328125	\\
0.611444044923869	0.001068115234375	\\
0.611488436098904	0.000579833984375	\\
0.611532827273938	0.00067138671875	\\
0.611577218448972	0.000640869140625	\\
0.611621609624007	0.00042724609375	\\
0.611666000799041	0.000457763671875	\\
0.611710391974076	0.000274658203125	\\
0.61175478314911	0.000335693359375	\\
0.611799174324144	0.000457763671875	\\
0.611843565499179	0.000640869140625	\\
0.611887956674213	0.000823974609375	\\
0.611932347849248	0.00115966796875	\\
0.611976739024282	0.001068115234375	\\
0.612021130199316	0.001556396484375	\\
0.612065521374351	0.001708984375	\\
0.612109912549385	0.00189208984375	\\
0.61215430372442	0.00213623046875	\\
0.612198694899454	0.002105712890625	\\
0.612243086074488	0.002166748046875	\\
0.612287477249523	0.001739501953125	\\
0.612331868424557	0.001312255859375	\\
0.612376259599592	0.000579833984375	\\
0.612420650774626	0.000732421875	\\
0.61246504194966	0.001190185546875	\\
0.612509433124695	0.000946044921875	\\
0.612553824299729	0.0008544921875	\\
0.612598215474764	0.000732421875	\\
0.612642606649798	0.00048828125	\\
0.612686997824832	0.000457763671875	\\
0.612731388999867	0.000457763671875	\\
0.612775780174901	0.00054931640625	\\
0.612820171349936	0.000335693359375	\\
0.61286456252497	6.103515625e-05	\\
0.612908953700004	0.00018310546875	\\
0.612953344875039	6.103515625e-05	\\
0.612997736050073	-6.103515625e-05	\\
0.613042127225108	0.000274658203125	\\
0.613086518400142	0.000335693359375	\\
0.613130909575176	0.00018310546875	\\
0.613175300750211	-3.0517578125e-05	\\
0.613219691925245	-0.000335693359375	\\
0.61326408310028	-0.000335693359375	\\
0.613308474275314	-0.000213623046875	\\
0.613352865450349	-0.000335693359375	\\
0.613397256625383	-0.000885009765625	\\
0.613441647800417	-0.00103759765625	\\
0.613486038975452	-0.000762939453125	\\
0.613530430150486	-0.000732421875	\\
0.61357482132552	-0.000946044921875	\\
0.613619212500555	-0.00048828125	\\
0.613663603675589	0.000244140625	\\
0.613707994850624	0.000213623046875	\\
0.613752386025658	0.000213623046875	\\
0.613796777200692	-0.000244140625	\\
0.613841168375727	3.0517578125e-05	\\
0.613885559550761	-0.0001220703125	\\
0.613929950725796	-0.000244140625	\\
0.61397434190083	0.000457763671875	\\
0.614018733075865	0.000579833984375	\\
0.614063124250899	0.00067138671875	\\
0.614107515425933	0.000579833984375	\\
0.614151906600968	0.000885009765625	\\
0.614196297776002	0.000946044921875	\\
0.614240688951037	0.000946044921875	\\
0.614285080126071	0.000396728515625	\\
0.614329471301105	-6.103515625e-05	\\
0.61437386247614	0.000213623046875	\\
0.614418253651174	0.000335693359375	\\
0.614462644826209	0.0001220703125	\\
0.614507036001243	3.0517578125e-05	\\
0.614551427176277	9.1552734375e-05	\\
0.614595818351312	0.00018310546875	\\
0.614640209526346	0.000152587890625	\\
0.614684600701381	0.00048828125	\\
0.614728991876415	0.000335693359375	\\
0.614773383051449	0.00030517578125	\\
0.614817774226484	0.000152587890625	\\
0.614862165401518	6.103515625e-05	\\
0.614906556576553	6.103515625e-05	\\
0.614950947751587	-0.0009765625	\\
0.614995338926621	-0.00091552734375	\\
0.615039730101656	-0.00067138671875	\\
0.61508412127669	-0.000946044921875	\\
0.615128512451725	-0.001251220703125	\\
0.615172903626759	-0.001190185546875	\\
0.615217294801793	-0.000640869140625	\\
0.615261685976828	-0.0009765625	\\
0.615306077151862	-0.0010986328125	\\
0.615350468326897	-0.00115966796875	\\
0.615394859501931	-0.000946044921875	\\
0.615439250676965	-0.00048828125	\\
0.615483641852	-0.000640869140625	\\
0.615528033027034	-0.000579833984375	\\
0.615572424202069	-0.000946044921875	\\
0.615616815377103	-0.00128173828125	\\
0.615661206552137	-0.000701904296875	\\
0.615705597727172	-0.000762939453125	\\
0.615749988902206	-0.0008544921875	\\
0.615794380077241	-0.0008544921875	\\
0.615838771252275	-0.00079345703125	\\
0.615883162427309	-0.00030517578125	\\
0.615927553602344	-0.000274658203125	\\
0.615971944777378	-0.000213623046875	\\
0.616016335952413	0	\\
0.616060727127447	-6.103515625e-05	\\
0.616105118302481	-0.000274658203125	\\
0.616149509477516	-0.000396728515625	\\
0.61619390065255	-0.00054931640625	\\
0.616238291827585	-0.00048828125	\\
0.616282683002619	-0.00048828125	\\
0.616327074177654	-9.1552734375e-05	\\
0.616371465352688	3.0517578125e-05	\\
0.616415856527722	-6.103515625e-05	\\
0.616460247702757	0.00067138671875	\\
0.616504638877791	0.00079345703125	\\
0.616549030052825	0.000152587890625	\\
0.61659342122786	0.000396728515625	\\
0.616637812402894	0.00079345703125	\\
0.616682203577929	0.000335693359375	\\
0.616726594752963	-0.00018310546875	\\
0.616770985927998	0.000244140625	\\
0.616815377103032	0.001007080078125	\\
0.616859768278066	0.001190185546875	\\
0.616904159453101	0.0009765625	\\
0.616948550628135	0.00042724609375	\\
0.61699294180317	3.0517578125e-05	\\
0.617037332978204	0.000823974609375	\\
0.617081724153238	0.000152587890625	\\
0.617126115328273	6.103515625e-05	\\
0.617170506503307	0.000701904296875	\\
0.617214897678342	0.0008544921875	\\
0.617259288853376	0.00128173828125	\\
0.61730368002841	0.001068115234375	\\
0.617348071203445	0.001007080078125	\\
0.617392462378479	0.000823974609375	\\
0.617436853553514	0.0009765625	\\
0.617481244728548	0.001495361328125	\\
0.617525635903582	0.00140380859375	\\
0.617570027078617	0.000885009765625	\\
0.617614418253651	0.001129150390625	\\
0.617658809428686	0.001129150390625	\\
0.61770320060372	0.00091552734375	\\
0.617747591778754	0.00128173828125	\\
0.617791982953789	0.00067138671875	\\
0.617836374128823	0.000152587890625	\\
0.617880765303858	-0.000152587890625	\\
0.617925156478892	-0.000457763671875	\\
0.617969547653926	-9.1552734375e-05	\\
0.618013938828961	0.000274658203125	\\
0.618058330003995	0.000396728515625	\\
0.61810272117903	-0.000579833984375	\\
0.618147112354064	-0.00054931640625	\\
0.618191503529098	0.00018310546875	\\
0.618235894704133	-6.103515625e-05	\\
0.618280285879167	-0.000152587890625	\\
0.618324677054202	-3.0517578125e-05	\\
0.618369068229236	9.1552734375e-05	\\
0.61841345940427	0.000244140625	\\
0.618457850579305	-3.0517578125e-05	\\
0.618502241754339	0.000152587890625	\\
0.618546632929374	9.1552734375e-05	\\
0.618591024104408	3.0517578125e-05	\\
0.618635415279442	0.000579833984375	\\
0.618679806454477	0.000579833984375	\\
0.618724197629511	0.00048828125	\\
0.618768588804546	0.0006103515625	\\
0.61881297997958	0.000762939453125	\\
0.618857371154614	0.000946044921875	\\
0.618901762329649	0.000762939453125	\\
0.618946153504683	0.00042724609375	\\
0.618990544679718	0.000579833984375	\\
0.619034935854752	0.0006103515625	\\
0.619079327029787	0.0006103515625	\\
0.619123718204821	0.00048828125	\\
0.619168109379855	0.000640869140625	\\
0.61921250055489	0.00091552734375	\\
0.619256891729924	0.000640869140625	\\
0.619301282904958	0.000640869140625	\\
0.619345674079993	0.0006103515625	\\
0.619390065255027	0.000335693359375	\\
0.619434456430062	0.000152587890625	\\
0.619478847605096	0.00067138671875	\\
0.61952323878013	0.000762939453125	\\
0.619567629955165	0.00018310546875	\\
0.619612021130199	0.000152587890625	\\
0.619656412305234	0.000335693359375	\\
0.619700803480268	0.000244140625	\\
0.619745194655303	0.000244140625	\\
0.619789585830337	0.000213623046875	\\
0.619833977005371	0.0001220703125	\\
0.619878368180406	0.00054931640625	\\
0.61992275935544	0.00048828125	\\
0.619967150530475	0.00048828125	\\
0.620011541705509	0.001007080078125	\\
0.620055932880543	0.000762939453125	\\
0.620100324055578	0.000885009765625	\\
0.620144715230612	0.000946044921875	\\
0.620189106405647	0.00091552734375	\\
0.620233497580681	0.000823974609375	\\
0.620277888755715	0.000823974609375	\\
0.62032227993075	0.001068115234375	\\
0.620366671105784	0.000885009765625	\\
0.620411062280819	0.00079345703125	\\
0.620455453455853	0.00054931640625	\\
0.620499844630887	0.000457763671875	\\
0.620544235805922	0.0008544921875	\\
0.620588626980956	0.000946044921875	\\
0.620633018155991	0.001251220703125	\\
0.620677409331025	0.00115966796875	\\
0.620721800506059	0.000885009765625	\\
0.620766191681094	0.00067138671875	\\
0.620810582856128	0.000518798828125	\\
0.620854974031163	0.0009765625	\\
0.620899365206197	0.00091552734375	\\
0.620943756381231	0.001007080078125	\\
0.620988147556266	0.00091552734375	\\
0.6210325387313	0.000762939453125	\\
0.621076929906335	0.00079345703125	\\
0.621121321081369	0.000946044921875	\\
0.621165712256403	0.000823974609375	\\
0.621210103431438	0.000885009765625	\\
0.621254494606472	0.000640869140625	\\
0.621298885781507	0.00067138671875	\\
0.621343276956541	0.001068115234375	\\
0.621387668131575	0.00079345703125	\\
0.62143205930661	0.00054931640625	\\
0.621476450481644	0.000457763671875	\\
0.621520841656679	0.000518798828125	\\
0.621565232831713	0.00079345703125	\\
0.621609624006747	0.0010986328125	\\
0.621654015181782	0.0008544921875	\\
0.621698406356816	0.000946044921875	\\
0.621742797531851	0.001129150390625	\\
0.621787188706885	0.001129150390625	\\
0.62183157988192	0.001190185546875	\\
0.621875971056954	0.00091552734375	\\
0.621920362231988	0.0008544921875	\\
0.621964753407023	0.000762939453125	\\
0.622009144582057	0.000213623046875	\\
0.622053535757091	0.000579833984375	\\
0.622097926932126	0.00091552734375	\\
0.62214231810716	0.00079345703125	\\
0.622186709282195	0.001129150390625	\\
0.622231100457229	0.000885009765625	\\
0.622275491632263	0.00079345703125	\\
0.622319882807298	0.00079345703125	\\
0.622364273982332	0.0009765625	\\
0.622408665157367	0.001129150390625	\\
0.622453056332401	0.00079345703125	\\
0.622497447507436	0.000579833984375	\\
0.62254183868247	0.00115966796875	\\
0.622586229857504	0.001129150390625	\\
0.622630621032539	0.000701904296875	\\
0.622675012207573	0.001068115234375	\\
0.622719403382608	0.0009765625	\\
0.622763794557642	0.0009765625	\\
0.622808185732676	0.001190185546875	\\
0.622852576907711	0.000732421875	\\
0.622896968082745	3.0517578125e-05	\\
0.62294135925778	0.000244140625	\\
0.622985750432814	0.000396728515625	\\
0.623030141607848	0.000152587890625	\\
0.623074532782883	0.000396728515625	\\
0.623118923957917	0.000762939453125	\\
0.623163315132952	0.0003662109375	\\
0.623207706307986	0.0006103515625	\\
0.62325209748302	0.001220703125	\\
0.623296488658055	0.001068115234375	\\
0.623340879833089	0.000946044921875	\\
0.623385271008124	0.00091552734375	\\
0.623429662183158	0.00054931640625	\\
0.623474053358192	0.000518798828125	\\
0.623518444533227	0.0006103515625	\\
0.623562835708261	0.000457763671875	\\
0.623607226883296	0.00067138671875	\\
0.62365161805833	0.00115966796875	\\
0.623696009233364	0.00146484375	\\
0.623740400408399	0.001495361328125	\\
0.623784791583433	0.001495361328125	\\
0.623829182758468	0.001678466796875	\\
0.623873573933502	0.002044677734375	\\
0.623917965108536	0.00244140625	\\
0.623962356283571	0.002593994140625	\\
0.624006747458605	0.002410888671875	\\
0.62405113863364	0.002349853515625	\\
0.624095529808674	0.001983642578125	\\
0.624139920983708	0.001922607421875	\\
0.624184312158743	0.002044677734375	\\
0.624228703333777	0.001922607421875	\\
0.624273094508812	0.001953125	\\
0.624317485683846	0.00213623046875	\\
0.62436187685888	0.002227783203125	\\
0.624406268033915	0.002410888671875	\\
0.624450659208949	0.002655029296875	\\
0.624495050383984	0.002685546875	\\
0.624539441559018	0.002410888671875	\\
0.624583832734052	0.001708984375	\\
0.624628223909087	0.001495361328125	\\
0.624672615084121	0.001312255859375	\\
0.624717006259156	0.00115966796875	\\
0.62476139743419	0.0010986328125	\\
0.624805788609225	0.000762939453125	\\
0.624850179784259	0.001129150390625	\\
0.624894570959293	0.001739501953125	\\
0.624938962134328	0.002197265625	\\
0.624983353309362	0.002349853515625	\\
0.625027744484396	0.00152587890625	\\
0.625072135659431	0.001556396484375	\\
0.625116526834465	0.00177001953125	\\
0.6251609180095	0.001739501953125	\\
0.625205309184534	0.00164794921875	\\
0.625249700359569	0.001678466796875	\\
0.625294091534603	0.001617431640625	\\
0.625338482709637	0.00189208984375	\\
0.625382873884672	0.00189208984375	\\
0.625427265059706	0.001617431640625	\\
0.625471656234741	0.0013427734375	\\
0.625516047409775	0.001220703125	\\
0.625560438584809	0.000823974609375	\\
0.625604829759844	0.00042724609375	\\
0.625649220934878	0.000762939453125	\\
0.625693612109913	0.000823974609375	\\
0.625738003284947	0.0003662109375	\\
0.625782394459981	0.00079345703125	\\
0.625826785635016	0.00091552734375	\\
0.62587117681005	0.00091552734375	\\
0.625915567985085	0.000640869140625	\\
0.625959959160119	0.0006103515625	\\
0.626004350335153	0.0008544921875	\\
0.626048741510188	0.00042724609375	\\
0.626093132685222	0.000579833984375	\\
0.626137523860257	0.000701904296875	\\
0.626181915035291	0.00054931640625	\\
0.626226306210325	0.0008544921875	\\
0.62627069738536	0.000885009765625	\\
0.626315088560394	0.0010986328125	\\
0.626359479735429	0.000946044921875	\\
0.626403870910463	0.00091552734375	\\
0.626448262085497	0.000396728515625	\\
0.626492653260532	0.00054931640625	\\
0.626537044435566	0.0008544921875	\\
0.626581435610601	0.000457763671875	\\
0.626625826785635	0.00042724609375	\\
0.626670217960669	0.000579833984375	\\
0.626714609135704	0.000152587890625	\\
0.626759000310738	0.00042724609375	\\
0.626803391485773	0.00103759765625	\\
0.626847782660807	0.001129150390625	\\
0.626892173835841	0.00079345703125	\\
0.626936565010876	0.00042724609375	\\
0.62698095618591	0.000885009765625	\\
0.627025347360945	0.001068115234375	\\
0.627069738535979	0.00054931640625	\\
0.627114129711013	-9.1552734375e-05	\\
0.627158520886048	6.103515625e-05	\\
0.627202912061082	0	\\
0.627247303236117	-0.000518798828125	\\
0.627291694411151	-0.000579833984375	\\
0.627336085586185	-0.000244140625	\\
0.62738047676122	-6.103515625e-05	\\
0.627424867936254	-0.000457763671875	\\
0.627469259111289	-6.103515625e-05	\\
0.627513650286323	-9.1552734375e-05	\\
0.627558041461358	-0.00048828125	\\
0.627602432636392	-0.000701904296875	\\
0.627646823811426	-0.0008544921875	\\
0.627691214986461	-0.000885009765625	\\
0.627735606161495	-0.000579833984375	\\
0.627779997336529	-0.00067138671875	\\
0.627824388511564	-0.00042724609375	\\
0.627868779686598	0.000213623046875	\\
0.627913170861633	-9.1552734375e-05	\\
0.627957562036667	6.103515625e-05	\\
0.628001953211701	0.00054931640625	\\
0.628046344386736	0.00048828125	\\
0.62809073556177	0.000274658203125	\\
0.628135126736805	0.00054931640625	\\
0.628179517911839	0.000762939453125	\\
0.628223909086874	0.000518798828125	\\
0.628268300261908	0.000640869140625	\\
0.628312691436942	0.00103759765625	\\
0.628357082611977	0.001220703125	\\
0.628401473787011	0.001129150390625	\\
0.628445864962046	0.001312255859375	\\
0.62849025613708	0.001312255859375	\\
0.628534647312114	0.001434326171875	\\
0.628579038487149	0.001678466796875	\\
0.628623429662183	0.001678466796875	\\
0.628667820837218	0.0018310546875	\\
0.628712212012252	0.001708984375	\\
0.628756603187286	0.001861572265625	\\
0.628800994362321	0.0018310546875	\\
0.628845385537355	0.001800537109375	\\
0.62888977671239	0.001556396484375	\\
0.628934167887424	0.0013427734375	\\
0.628978559062458	0.0015869140625	\\
0.629022950237493	0.00177001953125	\\
0.629067341412527	0.001983642578125	\\
0.629111732587562	0.002471923828125	\\
0.629156123762596	0.00213623046875	\\
0.62920051493763	0.00201416015625	\\
0.629244906112665	0.002227783203125	\\
0.629289297287699	0.001678466796875	\\
0.629333688462734	0.00146484375	\\
0.629378079637768	0.00152587890625	\\
0.629422470812802	0.00128173828125	\\
0.629466861987837	0.001556396484375	\\
0.629511253162871	0.001220703125	\\
0.629555644337906	0.001190185546875	\\
0.62960003551294	0.0013427734375	\\
0.629644426687974	0.001220703125	\\
0.629688817863009	0.001220703125	\\
0.629733209038043	0.001251220703125	\\
0.629777600213078	0.001007080078125	\\
0.629821991388112	0.000701904296875	\\
0.629866382563146	0.00103759765625	\\
0.629910773738181	0.00115966796875	\\
0.629955164913215	0.00128173828125	\\
0.62999955608825	0.001708984375	\\
0.630043947263284	0.00128173828125	\\
0.630088338438318	0.001068115234375	\\
0.630132729613353	0.00103759765625	\\
0.630177120788387	0.00140380859375	\\
0.630221511963422	0.00140380859375	\\
0.630265903138456	0.001434326171875	\\
0.630310294313491	0.001739501953125	\\
0.630354685488525	0.0015869140625	\\
0.630399076663559	0.001861572265625	\\
0.630443467838594	0.00189208984375	\\
0.630487859013628	0.001953125	\\
0.630532250188663	0.002166748046875	\\
0.630576641363697	0.002166748046875	\\
0.630621032538731	0.001922607421875	\\
0.630665423713766	0.001861572265625	\\
0.6307098148888	0.001953125	\\
0.630754206063834	0.001617431640625	\\
0.630798597238869	0.0015869140625	\\
0.630842988413903	0.0018310546875	\\
0.630887379588938	0.002227783203125	\\
0.630931770763972	0.002166748046875	\\
0.630976161939007	0.002044677734375	\\
0.631020553114041	0.002044677734375	\\
0.631064944289075	0.002166748046875	\\
0.63110933546411	0.002044677734375	\\
0.631153726639144	0.001861572265625	\\
0.631198117814179	0.0020751953125	\\
0.631242508989213	0.001953125	\\
0.631286900164247	0.0018310546875	\\
0.631331291339282	0.002166748046875	\\
0.631375682514316	0.002349853515625	\\
0.631420073689351	0.001953125	\\
0.631464464864385	0.001312255859375	\\
0.631508856039419	0.00146484375	\\
0.631553247214454	0.001312255859375	\\
0.631597638389488	0.001190185546875	\\
0.631642029564523	0.001251220703125	\\
0.631686420739557	0.001251220703125	\\
0.631730811914591	0.00152587890625	\\
0.631775203089626	0.001495361328125	\\
0.63181959426466	0.001190185546875	\\
0.631863985439695	0.00128173828125	\\
0.631908376614729	0.001068115234375	\\
0.631952767789763	0.00079345703125	\\
0.631997158964798	0.00115966796875	\\
0.632041550139832	0.0003662109375	\\
0.632085941314867	-3.0517578125e-05	\\
0.632130332489901	0.000396728515625	\\
0.632174723664935	0.00091552734375	\\
0.63221911483997	0.001312255859375	\\
0.632263506015004	0.00091552734375	\\
0.632307897190039	0.0008544921875	\\
0.632352288365073	0.00067138671875	\\
0.632396679540107	6.103515625e-05	\\
0.632441070715142	9.1552734375e-05	\\
0.632485461890176	-0.000274658203125	\\
0.632529853065211	-0.00079345703125	\\
0.632574244240245	-0.00067138671875	\\
0.632618635415279	-0.00042724609375	\\
0.632663026590314	-0.000457763671875	\\
0.632707417765348	-0.00048828125	\\
0.632751808940383	-0.000762939453125	\\
0.632796200115417	-0.00054931640625	\\
0.632840591290451	-0.000640869140625	\\
0.632884982465486	-0.00067138671875	\\
0.63292937364052	-0.000396728515625	\\
0.632973764815555	-0.000579833984375	\\
0.633018155990589	-0.000762939453125	\\
0.633062547165623	-0.000732421875	\\
0.633106938340658	-0.00042724609375	\\
0.633151329515692	-0.000335693359375	\\
0.633195720690727	-0.000518798828125	\\
0.633240111865761	-0.000335693359375	\\
0.633284503040796	-0.000152587890625	\\
0.63332889421583	-0.00030517578125	\\
0.633373285390864	-0.000579833984375	\\
0.633417676565899	-0.00048828125	\\
0.633462067740933	-0.00048828125	\\
0.633506458915967	-0.000885009765625	\\
0.633550850091002	-0.000579833984375	\\
0.633595241266036	-0.0003662109375	\\
0.633639632441071	-0.000274658203125	\\
0.633684023616105	0.0001220703125	\\
0.63372841479114	-0.000457763671875	\\
0.633772805966174	-0.000335693359375	\\
0.633817197141208	0.000213623046875	\\
0.633861588316243	3.0517578125e-05	\\
0.633905979491277	0.000213623046875	\\
0.633950370666312	0.000152587890625	\\
0.633994761841346	9.1552734375e-05	\\
0.63403915301638	3.0517578125e-05	\\
0.634083544191415	0.000213623046875	\\
0.634127935366449	-9.1552734375e-05	\\
0.634172326541484	-0.00048828125	\\
0.634216717716518	-9.1552734375e-05	\\
0.634261108891552	-0.00018310546875	\\
0.634305500066587	-9.1552734375e-05	\\
0.634349891241621	-0.0001220703125	\\
0.634394282416656	-0.000518798828125	\\
0.63443867359169	-0.000244140625	\\
0.634483064766724	-0.000396728515625	\\
0.634527455941759	-0.00067138671875	\\
0.634571847116793	-0.0008544921875	\\
0.634616238291828	-0.00042724609375	\\
0.634660629466862	-6.103515625e-05	\\
0.634705020641896	-0.00030517578125	\\
0.634749411816931	-0.0003662109375	\\
0.634793802991965	-3.0517578125e-05	\\
0.634838194167	0.000244140625	\\
0.634882585342034	-0.000152587890625	\\
0.634926976517068	-0.000457763671875	\\
0.634971367692103	-0.000396728515625	\\
0.635015758867137	-0.0003662109375	\\
0.635060150042172	-0.000274658203125	\\
0.635104541217206	-0.0003662109375	\\
0.63514893239224	-0.000213623046875	\\
0.635193323567275	-0.000274658203125	\\
0.635237714742309	-0.00054931640625	\\
0.635282105917344	0	\\
0.635326497092378	3.0517578125e-05	\\
0.635370888267412	-0.000244140625	\\
0.635415279442447	-0.000213623046875	\\
0.635459670617481	6.103515625e-05	\\
0.635504061792516	0	\\
0.63554845296755	-0.000152587890625	\\
0.635592844142584	0.000244140625	\\
0.635637235317619	0.000213623046875	\\
0.635681626492653	-0.000274658203125	\\
0.635726017667688	0	\\
0.635770408842722	0.000579833984375	\\
0.635814800017756	0.000244140625	\\
0.635859191192791	0.00030517578125	\\
0.635903582367825	0.000640869140625	\\
0.63594797354286	0.001007080078125	\\
0.635992364717894	0.000885009765625	\\
0.636036755892929	0.000396728515625	\\
0.636081147067963	0.000396728515625	\\
0.636125538242997	-9.1552734375e-05	\\
0.636169929418032	-0.000274658203125	\\
0.636214320593066	-0.00042724609375	\\
0.6362587117681	-0.00042724609375	\\
0.636303102943135	0.00018310546875	\\
0.636347494118169	-6.103515625e-05	\\
0.636391885293204	-0.000335693359375	\\
0.636436276468238	-0.000396728515625	\\
0.636480667643272	-0.000274658203125	\\
0.636525058818307	-0.000518798828125	\\
0.636569449993341	-0.000762939453125	\\
0.636613841168376	-0.000701904296875	\\
0.63665823234341	-0.00067138671875	\\
0.636702623518445	-0.000396728515625	\\
0.636747014693479	-0.000335693359375	\\
0.636791405868513	-0.00042724609375	\\
0.636835797043548	0	\\
0.636880188218582	3.0517578125e-05	\\
0.636924579393617	0	\\
0.636968970568651	-0.000213623046875	\\
0.637013361743685	-0.000457763671875	\\
0.63705775291872	-0.00018310546875	\\
0.637102144093754	-9.1552734375e-05	\\
0.637146535268789	-0.00018310546875	\\
0.637190926443823	-0.000152587890625	\\
0.637235317618857	-9.1552734375e-05	\\
0.637279708793892	0.000244140625	\\
0.637324099968926	0.000335693359375	\\
0.637368491143961	0.00042724609375	\\
0.637412882318995	0.000885009765625	\\
0.637457273494029	0.000885009765625	\\
0.637501664669064	0.0006103515625	\\
0.637546055844098	0.000335693359375	\\
0.637590447019133	0.000885009765625	\\
0.637634838194167	0.00146484375	\\
0.637679229369201	0.0010986328125	\\
0.637723620544236	0.00152587890625	\\
0.63776801171927	0.001678466796875	\\
0.637812402894305	0.0015869140625	\\
0.637856794069339	0.001617431640625	\\
0.637901185244373	0.00115966796875	\\
0.637945576419408	0.0018310546875	\\
0.637989967594442	0.001708984375	\\
0.638034358769477	0.00152587890625	\\
0.638078749944511	0.00225830078125	\\
0.638123141119545	0.0025634765625	\\
0.63816753229458	0.002197265625	\\
0.638211923469614	0.002197265625	\\
0.638256314644649	0.00238037109375	\\
0.638300705819683	0.00225830078125	\\
0.638345096994717	0.001861572265625	\\
0.638389488169752	0.00146484375	\\
0.638433879344786	0.001312255859375	\\
0.638478270519821	0.00152587890625	\\
0.638522661694855	0.001800537109375	\\
0.638567052869889	0.00140380859375	\\
0.638611444044924	0.001190185546875	\\
0.638655835219958	0.001373291015625	\\
0.638700226394993	0.001220703125	\\
0.638744617570027	0.00164794921875	\\
0.638789008745062	0.001312255859375	\\
0.638833399920096	0.000946044921875	\\
0.63887779109513	0.0010986328125	\\
0.638922182270165	0.000823974609375	\\
0.638966573445199	0.00079345703125	\\
0.639010964620234	0.001220703125	\\
0.639055355795268	0.000946044921875	\\
0.639099746970302	0.00115966796875	\\
0.639144138145337	0.0009765625	\\
0.639188529320371	0.000823974609375	\\
0.639232920495405	0.00048828125	\\
0.63927731167044	0.000518798828125	\\
0.639321702845474	0.000946044921875	\\
0.639366094020509	0.00103759765625	\\
0.639410485195543	0.001068115234375	\\
0.639454876370578	0.0009765625	\\
0.639499267545612	0.001312255859375	\\
0.639543658720646	0.001373291015625	\\
0.639588049895681	0.00115966796875	\\
0.639632441070715	0.001678466796875	\\
0.63967683224575	0.00152587890625	\\
0.639721223420784	0.001190185546875	\\
0.639765614595818	0.00146484375	\\
0.639810005770853	0.001495361328125	\\
0.639854396945887	0.001708984375	\\
0.639898788120922	0.00201416015625	\\
0.639943179295956	0.002227783203125	\\
0.63998757047099	0.002166748046875	\\
0.640031961646025	0.001800537109375	\\
0.640076352821059	0.001617431640625	\\
0.640120743996094	0.00177001953125	\\
0.640165135171128	0.00164794921875	\\
0.640209526346162	0.00128173828125	\\
0.640253917521197	0.001708984375	\\
0.640298308696231	0.001678466796875	\\
0.640342699871266	0.00115966796875	\\
0.6403870910463	0.00079345703125	\\
0.640431482221334	0.00079345703125	\\
0.640475873396369	0.00091552734375	\\
0.640520264571403	0.00079345703125	\\
0.640564655746438	0.000701904296875	\\
0.640609046921472	0.000701904296875	\\
0.640653438096506	0.0006103515625	\\
0.640697829271541	0.000335693359375	\\
0.640742220446575	0.00067138671875	\\
0.64078661162161	0.000518798828125	\\
0.640831002796644	3.0517578125e-05	\\
0.640875393971678	3.0517578125e-05	\\
0.640919785146713	-0.00042724609375	\\
0.640964176321747	-0.000335693359375	\\
0.641008567496782	-0.000244140625	\\
0.641052958671816	-9.1552734375e-05	\\
0.64109734984685	0.000335693359375	\\
0.641141741021885	-0.00018310546875	\\
0.641186132196919	-0.000396728515625	\\
0.641230523371954	9.1552734375e-05	\\
0.641274914546988	-0.000335693359375	\\
0.641319305722022	-0.00091552734375	\\
0.641363696897057	-0.000885009765625	\\
0.641408088072091	-0.00067138671875	\\
0.641452479247126	-0.000396728515625	\\
0.64149687042216	-0.000274658203125	\\
0.641541261597194	-0.00054931640625	\\
0.641585652772229	-0.0001220703125	\\
0.641630043947263	-3.0517578125e-05	\\
0.641674435122298	-0.000518798828125	\\
0.641718826297332	-0.000244140625	\\
0.641763217472367	-0.00048828125	\\
0.641807608647401	-0.000946044921875	\\
0.641851999822435	-0.000518798828125	\\
0.64189639099747	-0.000335693359375	\\
0.641940782172504	-0.0003662109375	\\
0.641985173347538	-0.00042724609375	\\
0.642029564522573	-0.00030517578125	\\
0.642073955697607	-0.000152587890625	\\
0.642118346872642	-0.000518798828125	\\
0.642162738047676	-0.00042724609375	\\
0.642207129222711	-0.000396728515625	\\
0.642251520397745	-0.000457763671875	\\
0.642295911572779	-0.000335693359375	\\
0.642340302747814	-0.00018310546875	\\
0.642384693922848	-0.000244140625	\\
0.642429085097883	0.000213623046875	\\
0.642473476272917	0	\\
0.642517867447951	-6.103515625e-05	\\
0.642562258622986	0.00018310546875	\\
0.64260664979802	-3.0517578125e-05	\\
0.642651040973055	0.000213623046875	\\
0.642695432148089	0.000335693359375	\\
0.642739823323123	0.000274658203125	\\
0.642784214498158	0.0001220703125	\\
0.642828605673192	0.000335693359375	\\
0.642872996848227	0.00018310546875	\\
0.642917388023261	0	\\
0.642961779198295	-0.000152587890625	\\
0.64300617037333	0.0003662109375	\\
0.643050561548364	0.00067138671875	\\
0.643094952723399	0.00067138671875	\\
0.643139343898433	0.001220703125	\\
0.643183735073467	0.001129150390625	\\
0.643228126248502	0.0010986328125	\\
0.643272517423536	0.00103759765625	\\
0.643316908598571	0.000946044921875	\\
0.643361299773605	0.0009765625	\\
0.643405690948639	0.00140380859375	\\
0.643450082123674	0.0015869140625	\\
0.643494473298708	0.001556396484375	\\
0.643538864473743	0.00189208984375	\\
0.643583255648777	0.001800537109375	\\
0.643627646823811	0.00152587890625	\\
0.643672037998846	0.0015869140625	\\
0.64371642917388	0.0009765625	\\
0.643760820348915	0.00079345703125	\\
0.643805211523949	0.00115966796875	\\
0.643849602698983	0.00103759765625	\\
0.643893993874018	0.0010986328125	\\
0.643938385049052	0.00140380859375	\\
0.643982776224087	0.001495361328125	\\
0.644027167399121	0.001251220703125	\\
0.644071558574155	0.001251220703125	\\
0.64411594974919	0.0010986328125	\\
0.644160340924224	0.001129150390625	\\
0.644204732099259	0.00067138671875	\\
0.644249123274293	0.000152587890625	\\
0.644293514449327	0.000762939453125	\\
0.644337905624362	0.001129150390625	\\
0.644382296799396	0.001373291015625	\\
0.644426687974431	0.001495361328125	\\
0.644471079149465	0.00152587890625	\\
0.6445154703245	0.001953125	\\
0.644559861499534	0.00177001953125	\\
0.644604252674568	0.00115966796875	\\
0.644648643849603	0.0013427734375	\\
0.644693035024637	0.001678466796875	\\
0.644737426199672	0.001495361328125	\\
0.644781817374706	0.001739501953125	\\
0.64482620854974	0.00225830078125	\\
0.644870599724775	0.0020751953125	\\
0.644914990899809	0.002166748046875	\\
0.644959382074843	0.00177001953125	\\
0.645003773249878	0.00146484375	\\
0.645048164424912	0.001312255859375	\\
0.645092555599947	0.001129150390625	\\
0.645136946774981	0.00115966796875	\\
0.645181337950016	0.001129150390625	\\
0.64522572912505	0.001617431640625	\\
0.645270120300084	0.00177001953125	\\
0.645314511475119	0.0018310546875	\\
0.645358902650153	0.002288818359375	\\
0.645403293825188	0.002471923828125	\\
0.645447685000222	0.00225830078125	\\
0.645492076175256	0.00201416015625	\\
0.645536467350291	0.00250244140625	\\
0.645580858525325	0.002471923828125	\\
0.64562524970036	0.002593994140625	\\
0.645669640875394	0.003173828125	\\
0.645714032050428	0.003448486328125	\\
0.645758423225463	0.003326416015625	\\
0.645802814400497	0.002899169921875	\\
0.645847205575532	0.0028076171875	\\
0.645891596750566	0.002593994140625	\\
0.6459359879256	0.00250244140625	\\
0.645980379100635	0.002777099609375	\\
0.646024770275669	0.00250244140625	\\
0.646069161450704	0.002655029296875	\\
0.646113552625738	0.0030517578125	\\
0.646157943800772	0.00360107421875	\\
0.646202334975807	0.00384521484375	\\
0.646246726150841	0.003631591796875	\\
0.646291117325876	0.0037841796875	\\
0.64633550850091	0.00372314453125	\\
0.646379899675944	0.003143310546875	\\
0.646424290850979	0.00299072265625	\\
0.646468682026013	0.002777099609375	\\
0.646513073201048	0.002960205078125	\\
0.646557464376082	0.003204345703125	\\
0.646601855551116	0.003143310546875	\\
0.646646246726151	0.003021240234375	\\
0.646690637901185	0.00286865234375	\\
0.64673502907622	0.0028076171875	\\
0.646779420251254	0.002716064453125	\\
0.646823811426288	0.002471923828125	\\
0.646868202601323	0.00250244140625	\\
0.646912593776357	0.00238037109375	\\
0.646956984951392	0.002349853515625	\\
0.647001376126426	0.002471923828125	\\
0.64704576730146	0.00250244140625	\\
0.647090158476495	0.002410888671875	\\
0.647134549651529	0.0020751953125	\\
0.647178940826564	0.001617431640625	\\
0.647223332001598	0.001373291015625	\\
0.647267723176633	0.00128173828125	\\
0.647312114351667	0.00177001953125	\\
0.647356505526701	0.001617431640625	\\
0.647400896701736	0.001739501953125	\\
0.64744528787677	0.002166748046875	\\
0.647489679051805	0.002197265625	\\
0.647534070226839	0.002410888671875	\\
0.647578461401873	0.002105712890625	\\
0.647622852576908	0.001953125	\\
0.647667243751942	0.001800537109375	\\
0.647711634926976	0.001739501953125	\\
0.647756026102011	0.001800537109375	\\
0.647800417277045	0.00152587890625	\\
0.64784480845208	0.001708984375	\\
0.647889199627114	0.001983642578125	\\
0.647933590802149	0.002593994140625	\\
0.647977981977183	0.00250244140625	\\
0.648022373152217	0.002197265625	\\
0.648066764327252	0.001983642578125	\\
0.648111155502286	0.00177001953125	\\
0.648155546677321	0.001434326171875	\\
0.648199937852355	0.001556396484375	\\
0.648244329027389	0.00152587890625	\\
0.648288720202424	0.0015869140625	\\
0.648333111377458	0.001953125	\\
0.648377502552493	0.00225830078125	\\
0.648421893727527	0.0023193359375	\\
0.648466284902561	0.0020751953125	\\
0.648510676077596	0.00164794921875	\\
0.64855506725263	0.001373291015625	\\
0.648599458427665	0.0015869140625	\\
0.648643849602699	0.00128173828125	\\
0.648688240777733	0.00177001953125	\\
0.648732631952768	0.001861572265625	\\
0.648777023127802	0.001708984375	\\
0.648821414302837	0.002044677734375	\\
0.648865805477871	0.00164794921875	\\
0.648910196652905	0.001495361328125	\\
0.64895458782794	0.00115966796875	\\
0.648998979002974	0.000885009765625	\\
0.649043370178009	0.000732421875	\\
0.649087761353043	0.000946044921875	\\
0.649132152528077	0.00079345703125	\\
0.649176543703112	0.000732421875	\\
0.649220934878146	0.001373291015625	\\
0.649265326053181	0.001708984375	\\
0.649309717228215	0.00244140625	\\
0.649354108403249	0.00299072265625	\\
0.649398499578284	0.002593994140625	\\
0.649442890753318	0.002716064453125	\\
0.649487281928353	0.002197265625	\\
0.649531673103387	0.001739501953125	\\
0.649576064278421	0.00140380859375	\\
0.649620455453456	0.001556396484375	\\
0.64966484662849	0.002105712890625	\\
0.649709237803525	0.00189208984375	\\
0.649753628978559	0.002471923828125	\\
0.649798020153593	0.00299072265625	\\
0.649842411328628	0.002410888671875	\\
0.649886802503662	0.00238037109375	\\
0.649931193678697	0.00244140625	\\
0.649975584853731	0.00189208984375	\\
0.650019976028765	0.001678466796875	\\
0.6500643672038	0.001953125	\\
0.650108758378834	0.00201416015625	\\
0.650153149553869	0.002197265625	\\
0.650197540728903	0.00225830078125	\\
0.650241931903938	0.00213623046875	\\
0.650286323078972	0.00177001953125	\\
0.650330714254006	0.0010986328125	\\
0.650375105429041	0.00042724609375	\\
0.650419496604075	-0.000152587890625	\\
0.650463887779109	-0.00018310546875	\\
0.650508278954144	-0.000213623046875	\\
0.650552670129178	-0.000213623046875	\\
0.650597061304213	0.000335693359375	\\
0.650641452479247	0.00042724609375	\\
0.650685843654282	0.00042724609375	\\
0.650730234829316	0.0006103515625	\\
0.65077462600435	0.000244140625	\\
0.650819017179385	-0.000213623046875	\\
0.650863408354419	-0.000579833984375	\\
0.650907799529454	-0.0008544921875	\\
0.650952190704488	-0.000885009765625	\\
0.650996581879522	-0.0006103515625	\\
0.651040973054557	-0.00067138671875	\\
0.651085364229591	-0.000579833984375	\\
0.651129755404626	-3.0517578125e-05	\\
0.65117414657966	6.103515625e-05	\\
0.651218537754694	0.000152587890625	\\
0.651262928929729	0.000457763671875	\\
0.651307320104763	0.00067138671875	\\
0.651351711279798	0.0010986328125	\\
0.651396102454832	0.001495361328125	\\
0.651440493629866	0.001617431640625	\\
0.651484884804901	0.001434326171875	\\
0.651529275979935	0.00177001953125	\\
0.65157366715497	0.001800537109375	\\
0.651618058330004	0.00189208984375	\\
0.651662449505038	0.00177001953125	\\
0.651706840680073	0.000762939453125	\\
0.651751231855107	0.00048828125	\\
0.651795623030142	-0.0001220703125	\\
0.651840014205176	-0.00030517578125	\\
0.65188440538021	0.00048828125	\\
0.651928796555245	0.0009765625	\\
0.651973187730279	0.001373291015625	\\
0.652017578905314	0.0013427734375	\\
0.652061970080348	0.000946044921875	\\
0.652106361255382	0.0008544921875	\\
0.652150752430417	0.00048828125	\\
0.652195143605451	0.0006103515625	\\
0.652239534780486	0.000396728515625	\\
0.65228392595552	0.00042724609375	\\
0.652328317130554	0.000701904296875	\\
0.652372708305589	0.000640869140625	\\
0.652417099480623	0.000457763671875	\\
0.652461490655658	9.1552734375e-05	\\
0.652505881830692	-0.000152587890625	\\
0.652550273005726	-0.000335693359375	\\
0.652594664180761	-0.00054931640625	\\
0.652639055355795	-0.00018310546875	\\
0.65268344653083	0.00018310546875	\\
0.652727837705864	0.00054931640625	\\
0.652772228880898	0.0008544921875	\\
0.652816620055933	0.0009765625	\\
0.652861011230967	0.000762939453125	\\
0.652905402406002	0.000701904296875	\\
0.652949793581036	0.000396728515625	\\
0.652994184756071	-6.103515625e-05	\\
0.653038575931105	-0.000274658203125	\\
0.653082967106139	-0.00030517578125	\\
0.653127358281174	0.000518798828125	\\
0.653171749456208	0.000579833984375	\\
0.653216140631243	0.000518798828125	\\
0.653260531806277	0.000946044921875	\\
0.653304922981311	0.000762939453125	\\
0.653349314156346	0.0013427734375	\\
0.65339370533138	0.00146484375	\\
0.653438096506414	0.0010986328125	\\
0.653482487681449	0.00152587890625	\\
0.653526878856483	0.001495361328125	\\
0.653571270031518	0.001953125	\\
0.653615661206552	0.0023193359375	\\
0.653660052381587	0.001861572265625	\\
0.653704443556621	0.00177001953125	\\
0.653748834731655	0.00140380859375	\\
0.65379322590669	0.000579833984375	\\
0.653837617081724	0.000701904296875	\\
0.653882008256759	0.000274658203125	\\
0.653926399431793	-0.0001220703125	\\
0.653970790606827	-0.00030517578125	\\
0.654015181781862	0	\\
0.654059572956896	0.000640869140625	\\
0.654103964131931	0.000457763671875	\\
0.654148355306965	0.000579833984375	\\
0.654192746481999	0.000732421875	\\
0.654237137657034	0.000244140625	\\
0.654281528832068	-3.0517578125e-05	\\
0.654325920007103	0.000152587890625	\\
0.654370311182137	0	\\
0.654414702357171	-0.0003662109375	\\
0.654459093532206	-3.0517578125e-05	\\
0.65450348470724	3.0517578125e-05	\\
0.654547875882275	-0.000244140625	\\
0.654592267057309	-0.000396728515625	\\
0.654636658232343	-0.0008544921875	\\
0.654681049407378	-0.001220703125	\\
0.654725440582412	-0.001190185546875	\\
0.654769831757447	-0.001251220703125	\\
0.654814222932481	-0.001220703125	\\
0.654858614107515	-0.00115966796875	\\
0.65490300528255	-0.000640869140625	\\
0.654947396457584	0	\\
0.654991787632619	-0.000335693359375	\\
0.655036178807653	-0.000457763671875	\\
0.655080569982687	-0.000457763671875	\\
0.655124961157722	-0.000579833984375	\\
0.655169352332756	-0.00091552734375	\\
0.655213743507791	-0.00128173828125	\\
0.655258134682825	-0.001739501953125	\\
0.655302525857859	-0.001251220703125	\\
0.655346917032894	-0.0008544921875	\\
0.655391308207928	-0.00091552734375	\\
0.655435699382963	-0.000640869140625	\\
0.655480090557997	-0.00091552734375	\\
0.655524481733031	-0.0009765625	\\
0.655568872908066	-0.00091552734375	\\
0.6556132640831	-0.00067138671875	\\
0.655657655258135	-0.001007080078125	\\
0.655702046433169	-0.0010986328125	\\
0.655746437608204	-0.000823974609375	\\
0.655790828783238	-0.001251220703125	\\
0.655835219958272	-0.00140380859375	\\
0.655879611133307	-0.001190185546875	\\
0.655924002308341	-0.0010986328125	\\
0.655968393483376	-0.000579833984375	\\
0.65601278465841	0	\\
0.656057175833444	-0.00018310546875	\\
0.656101567008479	-0.0001220703125	\\
0.656145958183513	0.000244140625	\\
0.656190349358547	-0.000213623046875	\\
0.656234740533582	-0.0006103515625	\\
0.656279131708616	-0.001068115234375	\\
0.656323522883651	-0.001251220703125	\\
0.656367914058685	-0.00115966796875	\\
0.65641230523372	-0.001220703125	\\
0.656456696408754	-0.00128173828125	\\
0.656501087583788	-0.00115966796875	\\
0.656545478758823	-0.001251220703125	\\
0.656589869933857	-0.0010986328125	\\
0.656634261108892	-0.001617431640625	\\
0.656678652283926	-0.001983642578125	\\
0.65672304345896	-0.002471923828125	\\
0.656767434633995	-0.002166748046875	\\
0.656811825809029	-0.001800537109375	\\
0.656856216984064	-0.00225830078125	\\
0.656900608159098	-0.00201416015625	\\
0.656944999334132	-0.002288818359375	\\
0.656989390509167	-0.003021240234375	\\
0.657033781684201	-0.003265380859375	\\
0.657078172859236	-0.00390625	\\
0.65712256403427	-0.003753662109375	\\
0.657166955209304	-0.003570556640625	\\
0.657211346384339	-0.003143310546875	\\
0.657255737559373	-0.002838134765625	\\
0.657300128734408	-0.002655029296875	\\
0.657344519909442	-0.002655029296875	\\
0.657388911084476	-0.002716064453125	\\
0.657433302259511	-0.002532958984375	\\
0.657477693434545	-0.0032958984375	\\
0.65752208460958	-0.00347900390625	\\
0.657566475784614	-0.003570556640625	\\
0.657610866959648	-0.003814697265625	\\
0.657655258134683	-0.003662109375	\\
0.657699649309717	-0.003448486328125	\\
0.657744040484752	-0.003143310546875	\\
0.657788431659786	-0.0028076171875	\\
0.65783282283482	-0.00238037109375	\\
0.657877214009855	-0.0028076171875	\\
0.657921605184889	-0.002655029296875	\\
0.657965996359924	-0.001983642578125	\\
0.658010387534958	-0.002471923828125	\\
0.658054778709992	-0.00238037109375	\\
0.658099169885027	-0.0023193359375	\\
0.658143561060061	-0.00250244140625	\\
0.658187952235096	-0.00244140625	\\
0.65823234341013	-0.00238037109375	\\
0.658276734585164	-0.002349853515625	\\
0.658321125760199	-0.002593994140625	\\
0.658365516935233	-0.002777099609375	\\
0.658409908110268	-0.002716064453125	\\
0.658454299285302	-0.00323486328125	\\
0.658498690460336	-0.00384521484375	\\
0.658543081635371	-0.0040283203125	\\
0.658587472810405	-0.0035400390625	\\
0.65863186398544	-0.003509521484375	\\
0.658676255160474	-0.003387451171875	\\
0.658720646335509	-0.0035400390625	\\
0.658765037510543	-0.00421142578125	\\
0.658809428685577	-0.00421142578125	\\
0.658853819860612	-0.004302978515625	\\
0.658898211035646	-0.0045166015625	\\
0.65894260221068	-0.004364013671875	\\
0.658986993385715	-0.004669189453125	\\
0.659031384560749	-0.004669189453125	\\
0.659075775735784	-0.00396728515625	\\
0.659120166910818	-0.0037841796875	\\
0.659164558085853	-0.003509521484375	\\
0.659208949260887	-0.00360107421875	\\
0.659253340435921	-0.003509521484375	\\
0.659297731610956	-0.003173828125	\\
0.65934212278599	-0.003387451171875	\\
0.659386513961025	-0.003448486328125	\\
0.659430905136059	-0.003448486328125	\\
0.659475296311093	-0.003143310546875	\\
0.659519687486128	-0.00213623046875	\\
0.659564078661162	-0.00250244140625	\\
0.659608469836197	-0.00244140625	\\
0.659652861011231	-0.0020751953125	\\
0.659697252186265	-0.002349853515625	\\
0.6597416433613	-0.002655029296875	\\
0.659786034536334	-0.002716064453125	\\
0.659830425711369	-0.002685546875	\\
0.659874816886403	-0.002685546875	\\
0.659919208061437	-0.002532958984375	\\
0.659963599236472	-0.00244140625	\\
0.660007990411506	-0.002044677734375	\\
0.660052381586541	-0.00189208984375	\\
0.660096772761575	-0.0018310546875	\\
0.660141163936609	-0.002044677734375	\\
0.660185555111644	-0.002288818359375	\\
0.660229946286678	-0.00225830078125	\\
0.660274337461713	-0.00238037109375	\\
0.660318728636747	-0.002349853515625	\\
0.660363119811781	-0.002532958984375	\\
0.660407510986816	-0.002777099609375	\\
0.66045190216185	-0.00250244140625	\\
0.660496293336885	-0.002593994140625	\\
0.660540684511919	-0.00299072265625	\\
0.660585075686953	-0.003509521484375	\\
0.660629466861988	-0.003265380859375	\\
0.660673858037022	-0.003387451171875	\\
0.660718249212057	-0.003509521484375	\\
0.660762640387091	-0.0037841796875	\\
0.660807031562126	-0.003875732421875	\\
0.66085142273716	-0.003692626953125	\\
0.660895813912194	-0.003814697265625	\\
0.660940205087229	-0.003631591796875	\\
0.660984596262263	-0.003936767578125	\\
0.661028987437297	-0.004241943359375	\\
0.661073378612332	-0.003875732421875	\\
0.661117769787366	-0.003509521484375	\\
0.661162160962401	-0.003662109375	\\
0.661206552137435	-0.003662109375	\\
0.661250943312469	-0.00408935546875	\\
0.661295334487504	-0.00421142578125	\\
0.661339725662538	-0.0037841796875	\\
0.661384116837573	-0.0035400390625	\\
0.661428508012607	-0.003326416015625	\\
0.661472899187642	-0.002655029296875	\\
0.661517290362676	-0.002716064453125	\\
0.66156168153771	-0.002899169921875	\\
0.661606072712745	-0.00299072265625	\\
0.661650463887779	-0.003204345703125	\\
0.661694855062814	-0.00335693359375	\\
0.661739246237848	-0.003265380859375	\\
0.661783637412882	-0.003173828125	\\
0.661828028587917	-0.002716064453125	\\
0.661872419762951	-0.002410888671875	\\
0.661916810937985	-0.002777099609375	\\
0.66196120211302	-0.002777099609375	\\
0.662005593288054	-0.003021240234375	\\
0.662049984463089	-0.003021240234375	\\
0.662094375638123	-0.002899169921875	\\
0.662138766813158	-0.003173828125	\\
0.662183157988192	-0.0032958984375	\\
0.662227549163226	-0.003326416015625	\\
0.662271940338261	-0.003082275390625	\\
0.662316331513295	-0.003509521484375	\\
0.66236072268833	-0.0040283203125	\\
0.662405113863364	-0.0037841796875	\\
0.662449505038398	-0.004058837890625	\\
0.662493896213433	-0.00408935546875	\\
0.662538287388467	-0.00384521484375	\\
0.662582678563502	-0.003326416015625	\\
0.662627069738536	-0.002777099609375	\\
0.66267146091357	-0.002227783203125	\\
0.662715852088605	-0.00152587890625	\\
0.662760243263639	-0.0013427734375	\\
0.662804634438674	-0.001190185546875	\\
0.662849025613708	-0.0015869140625	\\
0.662893416788742	-0.002166748046875	\\
0.662937807963777	-0.0020751953125	\\
0.662982199138811	-0.002471923828125	\\
0.663026590313846	-0.002288818359375	\\
0.66307098148888	-0.00189208984375	\\
0.663115372663914	-0.001708984375	\\
0.663159763838949	-0.001434326171875	\\
0.663204155013983	-0.00140380859375	\\
0.663248546189018	-0.001495361328125	\\
0.663292937364052	-0.00152587890625	\\
0.663337328539086	-0.001312255859375	\\
0.663381719714121	-0.00189208984375	\\
0.663426110889155	-0.00189208984375	\\
0.66347050206419	-0.001495361328125	\\
0.663514893239224	-0.00189208984375	\\
0.663559284414258	-0.00164794921875	\\
0.663603675589293	-0.001373291015625	\\
0.663648066764327	-0.00140380859375	\\
0.663692457939362	-0.00201416015625	\\
0.663736849114396	-0.002777099609375	\\
0.66378124028943	-0.003082275390625	\\
0.663825631464465	-0.003143310546875	\\
0.663870022639499	-0.002960205078125	\\
0.663914413814534	-0.002532958984375	\\
0.663958804989568	-0.001708984375	\\
0.664003196164602	-0.001220703125	\\
0.664047587339637	-0.0013427734375	\\
0.664091978514671	-0.001312255859375	\\
0.664136369689706	-0.001373291015625	\\
0.66418076086474	-0.001220703125	\\
0.664225152039775	-0.00177001953125	\\
0.664269543214809	-0.001983642578125	\\
0.664313934389843	-0.00225830078125	\\
0.664358325564878	-0.002044677734375	\\
0.664402716739912	-0.00128173828125	\\
0.664447107914947	-0.000518798828125	\\
0.664491499089981	-0.000640869140625	\\
0.664535890265015	-0.000823974609375	\\
0.66458028144005	-0.000457763671875	\\
0.664624672615084	-0.00018310546875	\\
0.664669063790118	0.0001220703125	\\
0.664713454965153	0.0001220703125	\\
0.664757846140187	3.0517578125e-05	\\
0.664802237315222	6.103515625e-05	\\
0.664846628490256	0.0003662109375	\\
0.664891019665291	0.0003662109375	\\
0.664935410840325	0.00042724609375	\\
0.664979802015359	0.00048828125	\\
0.665024193190394	0.00054931640625	\\
0.665068584365428	-9.1552734375e-05	\\
0.665112975540463	-0.000457763671875	\\
0.665157366715497	-0.001190185546875	\\
0.665201757890531	-0.00177001953125	\\
0.665246149065566	-0.00115966796875	\\
0.6652905402406	-0.0009765625	\\
0.665334931415635	-0.001007080078125	\\
0.665379322590669	-0.0008544921875	\\
0.665423713765703	-0.001220703125	\\
0.665468104940738	-0.001129150390625	\\
0.665512496115772	-0.00115966796875	\\
0.665556887290807	-0.0008544921875	\\
0.665601278465841	-0.001617431640625	\\
0.665645669640875	-0.002716064453125	\\
0.66569006081591	-0.002349853515625	\\
0.665734451990944	-0.002044677734375	\\
0.665778843165979	-0.00225830078125	\\
0.665823234341013	-0.0023193359375	\\
0.665867625516047	-0.0023193359375	\\
0.665912016691082	-0.001800537109375	\\
0.665956407866116	-0.001617431640625	\\
0.666000799041151	-0.001708984375	\\
0.666045190216185	-0.00146484375	\\
0.666089581391219	-0.001495361328125	\\
0.666133972566254	-0.000946044921875	\\
0.666178363741288	-0.000732421875	\\
0.666222754916323	-0.00048828125	\\
0.666267146091357	-0.000732421875	\\
0.666311537266391	-0.000640869140625	\\
0.666355928441426	-0.000518798828125	\\
0.66640031961646	-0.0009765625	\\
0.666444710791495	-0.0010986328125	\\
0.666489101966529	-0.00146484375	\\
0.666533493141563	-0.001708984375	\\
0.666577884316598	-0.0010986328125	\\
0.666622275491632	-0.000732421875	\\
0.666666666666667	-0.000396728515625	\\
0.666711057841701	-0.000732421875	\\
0.666755449016735	-0.0006103515625	\\
0.66679984019177	-0.000762939453125	\\
0.666844231366804	-0.0008544921875	\\
0.666888622541839	-0.000244140625	\\
0.666933013716873	-0.000640869140625	\\
0.666977404891907	-0.000701904296875	\\
0.667021796066942	-0.0010986328125	\\
0.667066187241976	-0.001312255859375	\\
0.667110578417011	-0.001190185546875	\\
0.667154969592045	-0.00140380859375	\\
0.66719936076708	-0.0015869140625	\\
0.667243751942114	-0.001861572265625	\\
0.667288143117148	-0.001678466796875	\\
0.667332534292183	-0.001617431640625	\\
0.667376925467217	-0.001678466796875	\\
0.667421316642252	-0.00177001953125	\\
0.667465707817286	-0.002197265625	\\
0.66751009899232	-0.00262451171875	\\
0.667554490167355	-0.002227783203125	\\
0.667598881342389	-0.00238037109375	\\
0.667643272517424	-0.002685546875	\\
0.667687663692458	-0.002410888671875	\\
0.667732054867492	-0.002197265625	\\
0.667776446042527	-0.001922607421875	\\
0.667820837217561	-0.0023193359375	\\
0.667865228392596	-0.002899169921875	\\
0.66790961956763	-0.002960205078125	\\
0.667954010742664	-0.0028076171875	\\
0.667998401917699	-0.00274658203125	\\
0.668042793092733	-0.002532958984375	\\
0.668087184267768	-0.00213623046875	\\
0.668131575442802	-0.00140380859375	\\
0.668175966617836	-0.001434326171875	\\
0.668220357792871	-0.00128173828125	\\
0.668264748967905	-0.001007080078125	\\
0.66830914014294	-0.001190185546875	\\
0.668353531317974	-0.001556396484375	\\
0.668397922493008	-0.00201416015625	\\
0.668442313668043	-0.002288818359375	\\
0.668486704843077	-0.002288818359375	\\
0.668531096018112	-0.001953125	\\
0.668575487193146	-0.002044677734375	\\
0.66861987836818	-0.00164794921875	\\
0.668664269543215	-0.001495361328125	\\
0.668708660718249	-0.001678466796875	\\
0.668753051893284	-0.001617431640625	\\
0.668797443068318	-0.00140380859375	\\
0.668841834243352	-0.00177001953125	\\
0.668886225418387	-0.00146484375	\\
0.668930616593421	-0.001312255859375	\\
0.668975007768456	-0.001220703125	\\
0.66901939894349	-0.000823974609375	\\
0.669063790118524	-0.0010986328125	\\
0.669108181293559	-0.000640869140625	\\
0.669152572468593	-0.00079345703125	\\
0.669196963643628	-0.001739501953125	\\
0.669241354818662	-0.001373291015625	\\
0.669285745993697	-0.0009765625	\\
0.669330137168731	-0.0010986328125	\\
0.669374528343765	-0.00048828125	\\
0.6694189195188	-0.000152587890625	\\
0.669463310693834	0.000457763671875	\\
0.669507701868868	0.000885009765625	\\
0.669552093043903	0.001220703125	\\
0.669596484218937	0.00091552734375	\\
0.669640875393972	0.00042724609375	\\
0.669685266569006	0.00018310546875	\\
0.66972965774404	-0.00018310546875	\\
0.669774048919075	0	\\
0.669818440094109	0.0003662109375	\\
0.669862831269144	0.00054931640625	\\
0.669907222444178	0.0006103515625	\\
0.669951613619213	0.00091552734375	\\
0.669996004794247	0.001220703125	\\
0.670040395969281	0.000732421875	\\
0.670084787144316	6.103515625e-05	\\
0.67012917831935	0.00067138671875	\\
0.670173569494385	0.000396728515625	\\
0.670217960669419	-0.00018310546875	\\
0.670262351844453	3.0517578125e-05	\\
0.670306743019488	3.0517578125e-05	\\
0.670351134194522	0.00018310546875	\\
0.670395525369556	3.0517578125e-05	\\
0.670439916544591	-0.000244140625	\\
0.670484307719625	-0.000732421875	\\
0.67052869889466	-0.00128173828125	\\
0.670573090069694	-0.001190185546875	\\
0.670617481244729	-0.001251220703125	\\
0.670661872419763	-0.0009765625	\\
0.670706263594797	-0.0008544921875	\\
0.670750654769832	-0.0003662109375	\\
0.670795045944866	0.000579833984375	\\
0.670839437119901	0.001007080078125	\\
0.670883828294935	0.000946044921875	\\
0.670928219469969	0.00067138671875	\\
0.670972610645004	0.000823974609375	\\
0.671017001820038	0.000732421875	\\
0.671061392995073	0.00018310546875	\\
0.671105784170107	0	\\
0.671150175345141	0.00030517578125	\\
0.671194566520176	0.0008544921875	\\
0.67123895769521	0.001129150390625	\\
0.671283348870245	0.0015869140625	\\
0.671327740045279	0.001953125	\\
0.671372131220313	0.00189208984375	\\
0.671416522395348	0.002288818359375	\\
0.671460913570382	0.0025634765625	\\
0.671505304745417	0.002349853515625	\\
0.671549695920451	0.00250244140625	\\
0.671594087095485	0.00262451171875	\\
0.67163847827052	0.00244140625	\\
0.671682869445554	0.002288818359375	\\
0.671727260620589	0.002227783203125	\\
0.671771651795623	0.0018310546875	\\
0.671816042970657	0.0010986328125	\\
0.671860434145692	0.00091552734375	\\
0.671904825320726	0.0003662109375	\\
0.671949216495761	-0.000274658203125	\\
0.671993607670795	-0.00067138671875	\\
0.672037998845829	-0.00054931640625	\\
0.672082390020864	-0.000457763671875	\\
0.672126781195898	-0.00054931640625	\\
0.672171172370933	-0.0006103515625	\\
0.672215563545967	-0.00067138671875	\\
0.672259954721001	-0.000701904296875	\\
0.672304345896036	-0.00067138671875	\\
0.67234873707107	-0.000701904296875	\\
0.672393128246105	-0.001007080078125	\\
0.672437519421139	-0.001373291015625	\\
0.672481910596173	-0.001251220703125	\\
0.672526301771208	-0.001678466796875	\\
0.672570692946242	-0.001800537109375	\\
0.672615084121277	-0.001434326171875	\\
0.672659475296311	-0.001983642578125	\\
0.672703866471346	-0.0018310546875	\\
0.67274825764638	-0.00146484375	\\
0.672792648821414	-0.001312255859375	\\
0.672837039996449	-0.000823974609375	\\
0.672881431171483	-0.000274658203125	\\
0.672925822346518	3.0517578125e-05	\\
0.672970213521552	0	\\
0.673014604696586	0.000152587890625	\\
0.673058995871621	0.000152587890625	\\
0.673103387046655	0	\\
0.673147778221689	0.00030517578125	\\
0.673192169396724	6.103515625e-05	\\
0.673236560571758	-0.0001220703125	\\
0.673280951746793	0.000274658203125	\\
0.673325342921827	0.000244140625	\\
0.673369734096862	0.000640869140625	\\
0.673414125271896	0.001007080078125	\\
0.67345851644693	0.000640869140625	\\
0.673502907621965	0.000732421875	\\
0.673547298796999	0.000579833984375	\\
0.673591689972034	0.001007080078125	\\
0.673636081147068	0.00201416015625	\\
0.673680472322102	0.001495361328125	\\
0.673724863497137	0.0008544921875	\\
0.673769254672171	0.000518798828125	\\
0.673813645847206	-9.1552734375e-05	\\
0.67385803702224	-0.000701904296875	\\
0.673902428197274	-0.0010986328125	\\
0.673946819372309	-0.00103759765625	\\
0.673991210547343	-0.001190185546875	\\
0.674035601722378	-0.001068115234375	\\
0.674079992897412	-0.001678466796875	\\
0.674124384072446	-0.001617431640625	\\
0.674168775247481	-0.0013427734375	\\
0.674213166422515	-0.002166748046875	\\
0.67425755759755	-0.002532958984375	\\
0.674301948772584	-0.00225830078125	\\
0.674346339947618	-0.002532958984375	\\
0.674390731122653	-0.002655029296875	\\
0.674435122297687	-0.00225830078125	\\
0.674479513472722	-0.002349853515625	\\
0.674523904647756	-0.002166748046875	\\
0.67456829582279	-0.002685546875	\\
0.674612686997825	-0.003204345703125	\\
0.674657078172859	-0.00347900390625	\\
0.674701469347894	-0.003875732421875	\\
0.674745860522928	-0.003082275390625	\\
0.674790251697962	-0.002716064453125	\\
0.674834642872997	-0.00262451171875	\\
0.674879034048031	-0.002227783203125	\\
0.674923425223066	-0.0020751953125	\\
0.6749678163981	-0.002410888671875	\\
0.675012207573135	-0.00262451171875	\\
0.675056598748169	-0.0025634765625	\\
0.675100989923203	-0.002532958984375	\\
0.675145381098238	-0.002655029296875	\\
0.675189772273272	-0.0025634765625	\\
0.675234163448306	-0.002777099609375	\\
0.675278554623341	-0.003204345703125	\\
0.675322945798375	-0.00286865234375	\\
0.67536733697341	-0.00274658203125	\\
0.675411728148444	-0.00299072265625	\\
0.675456119323478	-0.00347900390625	\\
0.675500510498513	-0.003509521484375	\\
0.675544901673547	-0.003265380859375	\\
0.675589292848582	-0.002899169921875	\\
0.675633684023616	-0.00311279296875	\\
0.675678075198651	-0.003143310546875	\\
0.675722466373685	-0.00299072265625	\\
0.675766857548719	-0.002838134765625	\\
0.675811248723754	-0.002960205078125	\\
0.675855639898788	-0.003143310546875	\\
0.675900031073823	-0.003631591796875	\\
0.675944422248857	-0.0042724609375	\\
0.675988813423891	-0.004302978515625	\\
0.676033204598926	-0.0042724609375	\\
0.67607759577396	-0.004638671875	\\
0.676121986948995	-0.004913330078125	\\
0.676166378124029	-0.004547119140625	\\
0.676210769299063	-0.003814697265625	\\
0.676255160474098	-0.0032958984375	\\
0.676299551649132	-0.0025634765625	\\
0.676343942824167	-0.002288818359375	\\
0.676388333999201	-0.0020751953125	\\
0.676432725174235	-0.002410888671875	\\
0.67647711634927	-0.00311279296875	\\
0.676521507524304	-0.002593994140625	\\
0.676565898699339	-0.002838134765625	\\
0.676610289874373	-0.00347900390625	\\
0.676654681049407	-0.003173828125	\\
0.676699072224442	-0.002838134765625	\\
0.676743463399476	-0.002899169921875	\\
0.676787854574511	-0.0029296875	\\
0.676832245749545	-0.002166748046875	\\
0.676876636924579	-0.002227783203125	\\
0.676921028099614	-0.002655029296875	\\
0.676965419274648	-0.002593994140625	\\
0.677009810449683	-0.002349853515625	\\
0.677054201624717	-0.002105712890625	\\
0.677098592799751	-0.002197265625	\\
0.677142983974786	-0.001953125	\\
0.67718737514982	-0.002105712890625	\\
0.677231766324855	-0.002166748046875	\\
0.677276157499889	-0.00201416015625	\\
0.677320548674923	-0.002685546875	\\
0.677364939849958	-0.003265380859375	\\
0.677409331024992	-0.00360107421875	\\
0.677453722200027	-0.003570556640625	\\
0.677498113375061	-0.0032958984375	\\
0.677542504550095	-0.00250244140625	\\
0.67758689572513	-0.001953125	\\
0.677631286900164	-0.00238037109375	\\
0.677675678075199	-0.002166748046875	\\
0.677720069250233	-0.001922607421875	\\
0.677764460425268	-0.002044677734375	\\
0.677808851600302	-0.0018310546875	\\
0.677853242775336	-0.001983642578125	\\
0.677897633950371	-0.002105712890625	\\
0.677942025125405	-0.0023193359375	\\
0.677986416300439	-0.002532958984375	\\
0.678030807475474	-0.00225830078125	\\
0.678075198650508	-0.00244140625	\\
0.678119589825543	-0.0023193359375	\\
0.678163981000577	-0.001983642578125	\\
0.678208372175611	-0.001556396484375	\\
0.678252763350646	-0.0008544921875	\\
0.67829715452568	-0.00079345703125	\\
0.678341545700715	-0.00054931640625	\\
0.678385936875749	3.0517578125e-05	\\
0.678430328050784	-0.00018310546875	\\
0.678474719225818	-0.000732421875	\\
0.678519110400852	-0.00091552734375	\\
0.678563501575887	-0.00103759765625	\\
0.678607892750921	-0.00140380859375	\\
0.678652283925956	-0.00146484375	\\
0.67869667510099	-0.00140380859375	\\
0.678741066276024	-0.0020751953125	\\
0.678785457451059	-0.001983642578125	\\
0.678829848626093	-0.00250244140625	\\
0.678874239801127	-0.0032958984375	\\
0.678918630976162	-0.002838134765625	\\
0.678963022151196	-0.002685546875	\\
0.679007413326231	-0.002685546875	\\
0.679051804501265	-0.002685546875	\\
0.6790961956763	-0.002471923828125	\\
0.679140586851334	-0.00250244140625	\\
0.679184978026368	-0.0028076171875	\\
0.679229369201403	-0.00311279296875	\\
0.679273760376437	-0.003448486328125	\\
0.679318151551472	-0.0035400390625	\\
0.679362542726506	-0.0040283203125	\\
0.67940693390154	-0.00408935546875	\\
0.679451325076575	-0.0042724609375	\\
0.679495716251609	-0.0045166015625	\\
0.679540107426644	-0.00360107421875	\\
0.679584498601678	-0.003082275390625	\\
0.679628889776712	-0.002960205078125	\\
0.679673280951747	-0.002685546875	\\
0.679717672126781	-0.002838134765625	\\
0.679762063301816	-0.00262451171875	\\
0.67980645447685	-0.002197265625	\\
0.679850845651884	-0.001953125	\\
0.679895236826919	-0.0018310546875	\\
0.679939628001953	-0.0018310546875	\\
0.679984019176988	-0.001861572265625	\\
0.680028410352022	-0.00189208984375	\\
0.680072801527056	-0.001617431640625	\\
0.680117192702091	-0.001708984375	\\
0.680161583877125	-0.001922607421875	\\
0.68020597505216	-0.001861572265625	\\
0.680250366227194	-0.001708984375	\\
0.680294757402228	-0.001556396484375	\\
0.680339148577263	-0.001678466796875	\\
0.680383539752297	-0.001800537109375	\\
0.680427930927332	-0.00189208984375	\\
0.680472322102366	-0.00250244140625	\\
0.6805167132774	-0.002838134765625	\\
0.680561104452435	-0.00250244140625	\\
0.680605495627469	-0.00262451171875	\\
0.680649886802504	-0.002685546875	\\
0.680694277977538	-0.0029296875	\\
0.680738669152572	-0.003021240234375	\\
0.680783060327607	-0.003448486328125	\\
0.680827451502641	-0.003631591796875	\\
0.680871842677676	-0.003692626953125	\\
0.68091623385271	-0.003936767578125	\\
0.680960625027744	-0.00396728515625	\\
0.681005016202779	-0.003662109375	\\
0.681049407377813	-0.00372314453125	\\
0.681093798552848	-0.003570556640625	\\
0.681138189727882	-0.00390625	\\
0.681182580902917	-0.004180908203125	\\
0.681226972077951	-0.0045166015625	\\
0.681271363252985	-0.0047607421875	\\
0.68131575442802	-0.004669189453125	\\
0.681360145603054	-0.0048828125	\\
0.681404536778089	-0.00469970703125	\\
0.681448927953123	-0.00457763671875	\\
0.681493319128157	-0.0045166015625	\\
0.681537710303192	-0.004180908203125	\\
0.681582101478226	-0.0040283203125	\\
0.681626492653261	-0.004180908203125	\\
0.681670883828295	-0.0042724609375	\\
0.681715275003329	-0.004608154296875	\\
0.681759666178364	-0.004547119140625	\\
0.681804057353398	-0.00457763671875	\\
0.681848448528433	-0.005096435546875	\\
0.681892839703467	-0.005096435546875	\\
0.681937230878501	-0.004241943359375	\\
0.681981622053536	-0.00408935546875	\\
0.68202601322857	-0.00445556640625	\\
0.682070404403605	-0.004730224609375	\\
0.682114795578639	-0.004730224609375	\\
0.682159186753673	-0.00457763671875	\\
0.682203577928708	-0.00494384765625	\\
0.682247969103742	-0.004669189453125	\\
0.682292360278777	-0.00494384765625	\\
0.682336751453811	-0.004791259765625	\\
0.682381142628845	-0.004302978515625	\\
0.68242553380388	-0.00445556640625	\\
0.682469924978914	-0.004180908203125	\\
0.682514316153949	-0.004058837890625	\\
0.682558707328983	-0.00408935546875	\\
0.682603098504017	-0.004425048828125	\\
0.682647489679052	-0.004119873046875	\\
0.682691880854086	-0.004119873046875	\\
0.682736272029121	-0.0042724609375	\\
0.682780663204155	-0.004486083984375	\\
0.682825054379189	-0.004547119140625	\\
0.682869445554224	-0.00439453125	\\
0.682913836729258	-0.004547119140625	\\
0.682958227904293	-0.004974365234375	\\
0.683002619079327	-0.00506591796875	\\
0.683047010254361	-0.005462646484375	\\
0.683091401429396	-0.0057373046875	\\
0.68313579260443	-0.0054931640625	\\
0.683180183779465	-0.005126953125	\\
0.683224574954499	-0.004638671875	\\
0.683268966129533	-0.0042724609375	\\
0.683313357304568	-0.003173828125	\\
0.683357748479602	-0.0032958984375	\\
0.683402139654637	-0.002960205078125	\\
0.683446530829671	-0.002288818359375	\\
0.683490922004706	-0.00238037109375	\\
0.68353531317974	-0.002655029296875	\\
0.683579704354774	-0.003509521484375	\\
0.683624095529809	-0.00372314453125	\\
0.683668486704843	-0.00390625	\\
0.683712877879877	-0.003753662109375	\\
0.683757269054912	-0.0035400390625	\\
0.683801660229946	-0.003570556640625	\\
0.683846051404981	-0.003082275390625	\\
0.683890442580015	-0.003173828125	\\
0.683934833755049	-0.003021240234375	\\
0.683979224930084	-0.003021240234375	\\
0.684023616105118	-0.00323486328125	\\
0.684068007280153	-0.003204345703125	\\
0.684112398455187	-0.003387451171875	\\
0.684156789630222	-0.003570556640625	\\
0.684201180805256	-0.003204345703125	\\
0.68424557198029	-0.00360107421875	\\
0.684289963155325	-0.003387451171875	\\
0.684334354330359	-0.003509521484375	\\
0.684378745505394	-0.00445556640625	\\
0.684423136680428	-0.004364013671875	\\
0.684467527855462	-0.004119873046875	\\
0.684511919030497	-0.004364013671875	\\
0.684556310205531	-0.00439453125	\\
0.684600701380566	-0.00421142578125	\\
0.6846450925556	-0.00396728515625	\\
0.684689483730634	-0.003204345703125	\\
0.684733874905669	-0.002777099609375	\\
0.684778266080703	-0.002777099609375	\\
0.684822657255738	-0.002227783203125	\\
0.684867048430772	-0.002288818359375	\\
0.684911439605806	-0.00238037109375	\\
0.684955830780841	-0.00250244140625	\\
0.685000221955875	-0.002777099609375	\\
0.68504461313091	-0.0023193359375	\\
0.685089004305944	-0.00250244140625	\\
0.685133395480978	-0.002410888671875	\\
0.685177786656013	-0.002197265625	\\
0.685222177831047	-0.00177001953125	\\
0.685266569006082	-0.001007080078125	\\
0.685310960181116	-0.000885009765625	\\
0.68535535135615	-0.0003662109375	\\
0.685399742531185	0.000244140625	\\
0.685444133706219	0.000244140625	\\
0.685488524881254	-6.103515625e-05	\\
0.685532916056288	-0.000579833984375	\\
0.685577307231322	-0.00054931640625	\\
0.685621698406357	-0.000732421875	\\
0.685666089581391	-0.000946044921875	\\
0.685710480756426	-0.00103759765625	\\
0.68575487193146	-0.00152587890625	\\
0.685799263106494	-0.00225830078125	\\
0.685843654281529	-0.00238037109375	\\
0.685888045456563	-0.00244140625	\\
0.685932436631598	-0.00311279296875	\\
0.685976827806632	-0.00311279296875	\\
0.686021218981666	-0.002471923828125	\\
0.686065610156701	-0.002593994140625	\\
0.686110001331735	-0.002838134765625	\\
0.68615439250677	-0.00238037109375	\\
0.686198783681804	-0.002044677734375	\\
0.686243174856839	-0.002288818359375	\\
0.686287566031873	-0.002197265625	\\
0.686331957206907	-0.00225830078125	\\
0.686376348381942	-0.002593994140625	\\
0.686420739556976	-0.00262451171875	\\
0.68646513073201	-0.002655029296875	\\
0.686509521907045	-0.00189208984375	\\
0.686553913082079	-0.00146484375	\\
0.686598304257114	-0.001556396484375	\\
0.686642695432148	-0.00128173828125	\\
0.686687086607182	-0.00103759765625	\\
0.686731477782217	-0.0008544921875	\\
0.686775868957251	-0.00030517578125	\\
0.686820260132286	-0.00042724609375	\\
0.68686465130732	-0.000244140625	\\
0.686909042482355	0.0001220703125	\\
0.686953433657389	-0.000213623046875	\\
0.686997824832423	-0.00018310546875	\\
0.687042216007458	-3.0517578125e-05	\\
0.687086607182492	0.000213623046875	\\
0.687130998357527	9.1552734375e-05	\\
0.687175389532561	0	\\
0.687219780707595	-0.000152587890625	\\
0.68726417188263	-3.0517578125e-05	\\
0.687308563057664	-9.1552734375e-05	\\
0.687352954232698	-0.00048828125	\\
0.687397345407733	-0.00042724609375	\\
0.687441736582767	-0.000457763671875	\\
0.687486127757802	-0.00054931640625	\\
0.687530518932836	-0.00079345703125	\\
0.687574910107871	-0.000732421875	\\
0.687619301282905	-0.000885009765625	\\
0.687663692457939	-0.00103759765625	\\
0.687708083632974	-0.001373291015625	\\
0.687752474808008	-0.00140380859375	\\
0.687796865983043	-0.001312255859375	\\
0.687841257158077	-0.001800537109375	\\
0.687885648333111	-0.00189208984375	\\
0.687930039508146	-0.002655029296875	\\
0.68797443068318	-0.0028076171875	\\
0.688018821858215	-0.00299072265625	\\
0.688063213033249	-0.00335693359375	\\
0.688107604208283	-0.003021240234375	\\
0.688151995383318	-0.002685546875	\\
0.688196386558352	-0.0028076171875	\\
0.688240777733387	-0.00274658203125	\\
0.688285168908421	-0.002716064453125	\\
0.688329560083455	-0.0023193359375	\\
0.68837395125849	-0.002471923828125	\\
0.688418342433524	-0.0025634765625	\\
0.688462733608559	-0.00213623046875	\\
0.688507124783593	-0.001861572265625	\\
0.688551515958627	-0.0015869140625	\\
0.688595907133662	-0.00146484375	\\
0.688640298308696	-0.00146484375	\\
0.688684689483731	-0.001556396484375	\\
0.688729080658765	-0.00146484375	\\
0.688773471833799	-0.0010986328125	\\
0.688817863008834	-0.0013427734375	\\
0.688862254183868	-0.001007080078125	\\
0.688906645358903	-0.0003662109375	\\
0.688951036533937	-0.000396728515625	\\
0.688995427708971	3.0517578125e-05	\\
0.689039818884006	-3.0517578125e-05	\\
0.68908421005904	-0.000152587890625	\\
0.689128601234075	0.000152587890625	\\
0.689172992409109	-0.0003662109375	\\
0.689217383584143	-0.000885009765625	\\
0.689261774759178	-0.000579833984375	\\
0.689306165934212	-0.000579833984375	\\
0.689350557109247	-0.000701904296875	\\
0.689394948284281	-0.00079345703125	\\
0.689439339459315	-0.001220703125	\\
0.68948373063435	-0.001220703125	\\
0.689528121809384	-0.0015869140625	\\
0.689572512984419	-0.0015869140625	\\
0.689616904159453	-0.0015869140625	\\
0.689661295334488	-0.00189208984375	\\
0.689705686509522	-0.0018310546875	\\
0.689750077684556	-0.00164794921875	\\
0.689794468859591	-0.001220703125	\\
0.689838860034625	-0.001708984375	\\
0.68988325120966	-0.002349853515625	\\
0.689927642384694	-0.0028076171875	\\
0.689972033559728	-0.00311279296875	\\
0.690016424734763	-0.003326416015625	\\
0.690060815909797	-0.00384521484375	\\
0.690105207084832	-0.004119873046875	\\
0.690149598259866	-0.00445556640625	\\
0.6901939894349	-0.004669189453125	\\
0.690238380609935	-0.004425048828125	\\
0.690282771784969	-0.004180908203125	\\
0.690327162960004	-0.00408935546875	\\
0.690371554135038	-0.003662109375	\\
0.690415945310072	-0.0029296875	\\
0.690460336485107	-0.001983642578125	\\
0.690504727660141	-0.001251220703125	\\
0.690549118835176	-0.000823974609375	\\
0.69059351001021	-0.00054931640625	\\
0.690637901185244	-0.00042724609375	\\
0.690682292360279	-0.000518798828125	\\
0.690726683535313	-0.001434326171875	\\
0.690771074710348	-0.001556396484375	\\
0.690815465885382	-0.0015869140625	\\
0.690859857060416	-0.001708984375	\\
0.690904248235451	-0.00146484375	\\
0.690948639410485	-0.001617431640625	\\
0.69099303058552	-0.001251220703125	\\
0.691037421760554	-0.000762939453125	\\
0.691081812935588	-0.001129150390625	\\
0.691126204110623	-0.001129150390625	\\
0.691170595285657	-0.000640869140625	\\
0.691214986460692	-0.0009765625	\\
0.691259377635726	-0.0013427734375	\\
0.69130376881076	-0.001251220703125	\\
0.691348159985795	-0.0009765625	\\
0.691392551160829	-0.00140380859375	\\
0.691436942335864	-0.002044677734375	\\
0.691481333510898	-0.002532958984375	\\
0.691525724685932	-0.00299072265625	\\
0.691570115860967	-0.003204345703125	\\
0.691614507036001	-0.003753662109375	\\
0.691658898211036	-0.0042724609375	\\
0.69170328938607	-0.004058837890625	\\
0.691747680561104	-0.00396728515625	\\
0.691792071736139	-0.00384521484375	\\
0.691836462911173	-0.0032958984375	\\
0.691880854086208	-0.003265380859375	\\
0.691925245261242	-0.003265380859375	\\
0.691969636436277	-0.003662109375	\\
0.692014027611311	-0.003662109375	\\
0.692058418786345	-0.00311279296875	\\
0.69210280996138	-0.003021240234375	\\
0.692147201136414	-0.003143310546875	\\
0.692191592311448	-0.003448486328125	\\
0.692235983486483	-0.0040283203125	\\
0.692280374661517	-0.003448486328125	\\
0.692324765836552	-0.002838134765625	\\
0.692369157011586	-0.002227783203125	\\
0.69241354818662	-0.00177001953125	\\
0.692457939361655	-0.001434326171875	\\
0.692502330536689	-0.000823974609375	\\
0.692546721711724	-0.000213623046875	\\
0.692591112886758	-6.103515625e-05	\\
0.692635504061793	0.00079345703125	\\
0.692679895236827	0.0006103515625	\\
0.692724286411861	-6.103515625e-05	\\
0.692768677586896	0	\\
0.69281306876193	-0.00054931640625	\\
0.692857459936965	-0.000579833984375	\\
0.692901851111999	-0.000885009765625	\\
0.692946242287033	-0.001220703125	\\
0.692990633462068	-0.0015869140625	\\
0.693035024637102	-0.002227783203125	\\
0.693079415812137	-0.00225830078125	\\
0.693123806987171	-0.002197265625	\\
0.693168198162205	-0.00189208984375	\\
0.69321258933724	-0.001312255859375	\\
0.693256980512274	-0.0013427734375	\\
0.693301371687309	-0.0010986328125	\\
0.693345762862343	-0.000701904296875	\\
0.693390154037377	-0.001190185546875	\\
0.693434545212412	-0.001678466796875	\\
0.693478936387446	-0.001983642578125	\\
0.693523327562481	-0.002532958984375	\\
0.693567718737515	-0.002838134765625	\\
0.693612109912549	-0.0030517578125	\\
0.693656501087584	-0.003204345703125	\\
0.693700892262618	-0.00286865234375	\\
0.693745283437653	-0.002593994140625	\\
0.693789674612687	-0.002685546875	\\
0.693834065787721	-0.002105712890625	\\
0.693878456962756	-0.00201416015625	\\
0.69392284813779	-0.0018310546875	\\
0.693967239312825	-0.001007080078125	\\
0.694011630487859	-0.000701904296875	\\
0.694056021662893	-0.00048828125	\\
0.694100412837928	-0.00067138671875	\\
0.694144804012962	-0.001190185546875	\\
0.694189195187997	-0.0008544921875	\\
0.694233586363031	-0.0008544921875	\\
0.694277977538065	-0.00140380859375	\\
0.6943223687131	-0.001708984375	\\
0.694366759888134	-0.001617431640625	\\
0.694411151063169	-0.0018310546875	\\
0.694455542238203	-0.001556396484375	\\
0.694499933413237	-0.001220703125	\\
0.694544324588272	-0.001129150390625	\\
0.694588715763306	-0.001007080078125	\\
0.694633106938341	-0.001220703125	\\
0.694677498113375	-0.0013427734375	\\
0.69472188928841	-0.001495361328125	\\
0.694766280463444	-0.001434326171875	\\
0.694810671638478	-0.0010986328125	\\
0.694855062813513	-0.00146484375	\\
0.694899453988547	-0.001373291015625	\\
0.694943845163581	-0.00079345703125	\\
0.694988236338616	-0.0018310546875	\\
0.69503262751365	-0.0029296875	\\
0.695077018688685	-0.00311279296875	\\
0.695121409863719	-0.00311279296875	\\
0.695165801038753	-0.003662109375	\\
0.695210192213788	-0.003662109375	\\
0.695254583388822	-0.00311279296875	\\
0.695298974563857	-0.0029296875	\\
0.695343365738891	-0.002716064453125	\\
0.695387756913926	-0.002227783203125	\\
0.69543214808896	-0.001922607421875	\\
0.695476539263994	-0.00201416015625	\\
0.695520930439029	-0.00250244140625	\\
0.695565321614063	-0.002471923828125	\\
0.695609712789098	-0.0023193359375	\\
0.695654103964132	-0.00262451171875	\\
0.695698495139166	-0.002593994140625	\\
0.695742886314201	-0.0030517578125	\\
0.695787277489235	-0.002899169921875	\\
0.69583166866427	-0.003173828125	\\
0.695876059839304	-0.003570556640625	\\
0.695920451014338	-0.0035400390625	\\
0.695964842189373	-0.003173828125	\\
0.696009233364407	-0.002471923828125	\\
0.696053624539442	-0.002410888671875	\\
0.696098015714476	-0.002166748046875	\\
0.69614240688951	-0.001953125	\\
0.696186798064545	-0.0018310546875	\\
0.696231189239579	-0.001373291015625	\\
0.696275580414614	-0.001678466796875	\\
0.696319971589648	-0.001678466796875	\\
0.696364362764682	-0.001739501953125	\\
0.696408753939717	-0.001556396484375	\\
0.696453145114751	-0.001556396484375	\\
0.696497536289786	-0.001495361328125	\\
0.69654192746482	-0.00115966796875	\\
0.696586318639854	-0.00140380859375	\\
0.696630709814889	-0.001800537109375	\\
0.696675100989923	-0.001251220703125	\\
0.696719492164958	-0.001129150390625	\\
0.696763883339992	-0.001251220703125	\\
0.696808274515026	-0.001434326171875	\\
0.696852665690061	-0.001251220703125	\\
0.696897056865095	-0.001068115234375	\\
0.69694144804013	-0.00128173828125	\\
0.696985839215164	-0.0010986328125	\\
0.697030230390198	-0.001617431640625	\\
0.697074621565233	-0.002471923828125	\\
0.697119012740267	-0.002532958984375	\\
0.697163403915302	-0.0029296875	\\
0.697207795090336	-0.00311279296875	\\
0.69725218626537	-0.003509521484375	\\
0.697296577440405	-0.00372314453125	\\
0.697340968615439	-0.004058837890625	\\
0.697385359790474	-0.0047607421875	\\
0.697429750965508	-0.00457763671875	\\
0.697474142140542	-0.004791259765625	\\
0.697518533315577	-0.00469970703125	\\
0.697562924490611	-0.004486083984375	\\
0.697607315665646	-0.004150390625	\\
0.69765170684068	-0.003173828125	\\
0.697696098015715	-0.00225830078125	\\
0.697740489190749	-0.002044677734375	\\
0.697784880365783	-0.00140380859375	\\
0.697829271540818	-0.0009765625	\\
0.697873662715852	-0.001068115234375	\\
0.697918053890886	-0.001678466796875	\\
0.697962445065921	-0.002105712890625	\\
0.698006836240955	-0.0023193359375	\\
0.69805122741599	-0.0028076171875	\\
0.698095618591024	-0.002777099609375	\\
0.698140009766059	-0.002685546875	\\
0.698184400941093	-0.002288818359375	\\
0.698228792116127	-0.002105712890625	\\
0.698273183291162	-0.00177001953125	\\
0.698317574466196	-0.001708984375	\\
0.698361965641231	-0.0018310546875	\\
0.698406356816265	-0.001708984375	\\
0.698450747991299	-0.002166748046875	\\
0.698495139166334	-0.002471923828125	\\
0.698539530341368	-0.002410888671875	\\
0.698583921516403	-0.002899169921875	\\
0.698628312691437	-0.003387451171875	\\
0.698672703866471	-0.003692626953125	\\
0.698717095041506	-0.00384521484375	\\
0.69876148621654	-0.004150390625	\\
0.698805877391575	-0.00482177734375	\\
0.698850268566609	-0.005218505859375	\\
0.698894659741643	-0.0057373046875	\\
0.698939050916678	-0.005828857421875	\\
0.698983442091712	-0.005401611328125	\\
0.699027833266747	-0.004547119140625	\\
0.699072224441781	-0.004486083984375	\\
0.699116615616815	-0.00433349609375	\\
0.69916100679185	-0.004058837890625	\\
0.699205397966884	-0.004150390625	\\
0.699249789141919	-0.003814697265625	\\
0.699294180316953	-0.003814697265625	\\
0.699338571491987	-0.00384521484375	\\
0.699382962667022	-0.004058837890625	\\
0.699427353842056	-0.004302978515625	\\
0.699471745017091	-0.004241943359375	\\
0.699516136192125	-0.00408935546875	\\
0.699560527367159	-0.00347900390625	\\
0.699604918542194	-0.003021240234375	\\
0.699649309717228	-0.002410888671875	\\
0.699693700892263	-0.001739501953125	\\
0.699738092067297	-0.000885009765625	\\
0.699782483242331	-0.000335693359375	\\
0.699826874417366	-0.000335693359375	\\
0.6998712655924	-0.000396728515625	\\
0.699915656767435	-0.000396728515625	\\
0.699960047942469	-0.000518798828125	\\
0.700004439117503	-0.000762939453125	\\
0.700048830292538	-0.00140380859375	\\
0.700093221467572	-0.001495361328125	\\
0.700137612642607	-0.00164794921875	\\
0.700182003817641	-0.0025634765625	\\
0.700226394992675	-0.00311279296875	\\
0.70027078616771	-0.00335693359375	\\
0.700315177342744	-0.00384521484375	\\
0.700359568517779	-0.003692626953125	\\
0.700403959692813	-0.003448486328125	\\
0.700448350867848	-0.00311279296875	\\
0.700492742042882	-0.003265380859375	\\
0.700537133217916	-0.0030517578125	\\
0.700581524392951	-0.002777099609375	\\
0.700625915567985	-0.00286865234375	\\
0.700670306743019	-0.003509521484375	\\
0.700714697918054	-0.004730224609375	\\
0.700759089093088	-0.004974365234375	\\
0.700803480268123	-0.004913330078125	\\
0.700847871443157	-0.00537109375	\\
0.700892262618191	-0.005218505859375	\\
0.700936653793226	-0.004913330078125	\\
0.70098104496826	-0.005126953125	\\
0.701025436143295	-0.0050048828125	\\
0.701069827318329	-0.004669189453125	\\
0.701114218493364	-0.00469970703125	\\
0.701158609668398	-0.004638671875	\\
0.701203000843432	-0.004150390625	\\
0.701247392018467	-0.0032958984375	\\
0.701291783193501	-0.00274658203125	\\
0.701336174368536	-0.002899169921875	\\
0.70138056554357	-0.002716064453125	\\
0.701424956718604	-0.003173828125	\\
0.701469347893639	-0.00347900390625	\\
0.701513739068673	-0.003570556640625	\\
0.701558130243708	-0.003997802734375	\\
0.701602521418742	-0.00396728515625	\\
0.701646912593776	-0.003814697265625	\\
0.701691303768811	-0.00384521484375	\\
0.701735694943845	-0.003265380859375	\\
0.70178008611888	-0.002777099609375	\\
0.701824477293914	-0.0028076171875	\\
0.701868868468948	-0.002838134765625	\\
0.701913259643983	-0.003173828125	\\
0.701957650819017	-0.002838134765625	\\
0.702002041994052	-0.002593994140625	\\
0.702046433169086	-0.002685546875	\\
0.70209082434412	-0.002593994140625	\\
0.702135215519155	-0.00286865234375	\\
0.702179606694189	-0.003387451171875	\\
0.702223997869224	-0.003814697265625	\\
0.702268389044258	-0.003753662109375	\\
0.702312780219292	-0.004150390625	\\
0.702357171394327	-0.004486083984375	\\
0.702401562569361	-0.004425048828125	\\
0.702445953744396	-0.00457763671875	\\
0.70249034491943	-0.0045166015625	\\
0.702534736094464	-0.00433349609375	\\
0.702579127269499	-0.0042724609375	\\
0.702623518444533	-0.003753662109375	\\
0.702667909619568	-0.003387451171875	\\
0.702712300794602	-0.00360107421875	\\
0.702756691969636	-0.003692626953125	\\
0.702801083144671	-0.003692626953125	\\
0.702845474319705	-0.0037841796875	\\
0.70288986549474	-0.00347900390625	\\
0.702934256669774	-0.003936767578125	\\
0.702978647844808	-0.004180908203125	\\
0.703023039019843	-0.003936767578125	\\
0.703067430194877	-0.00390625	\\
0.703111821369912	-0.00384521484375	\\
0.703156212544946	-0.003143310546875	\\
0.703200603719981	-0.0030517578125	\\
0.703244994895015	-0.003021240234375	\\
0.703289386070049	-0.002166748046875	\\
0.703333777245084	-0.0015869140625	\\
0.703378168420118	-0.001495361328125	\\
0.703422559595152	-0.00128173828125	\\
0.703466950770187	-0.000732421875	\\
0.703511341945221	-0.000335693359375	\\
0.703555733120256	-0.000213623046875	\\
0.70360012429529	-0.000457763671875	\\
0.703644515470324	-0.000152587890625	\\
0.703688906645359	0.0001220703125	\\
0.703733297820393	-0.0001220703125	\\
0.703777688995428	-0.000457763671875	\\
0.703822080170462	-0.00079345703125	\\
0.703866471345497	-0.000152587890625	\\
0.703910862520531	0.000518798828125	\\
0.703955253695565	0.000213623046875	\\
0.7039996448706	0.00018310546875	\\
0.704044036045634	6.103515625e-05	\\
0.704088427220669	0.000244140625	\\
0.704132818395703	0.0003662109375	\\
0.704177209570737	-0.000213623046875	\\
0.704221600745772	-0.000823974609375	\\
0.704265991920806	-0.00115966796875	\\
0.704310383095841	-0.00177001953125	\\
0.704354774270875	-0.002593994140625	\\
0.704399165445909	-0.002899169921875	\\
0.704443556620944	-0.00384521484375	\\
0.704487947795978	-0.00439453125	\\
0.704532338971013	-0.004638671875	\\
0.704576730146047	-0.00494384765625	\\
0.704621121321081	-0.005157470703125	\\
0.704665512496116	-0.005584716796875	\\
0.70470990367115	-0.005889892578125	\\
0.704754294846185	-0.00628662109375	\\
0.704798686021219	-0.00634765625	\\
0.704843077196253	-0.0064697265625	\\
0.704887468371288	-0.00653076171875	\\
0.704931859546322	-0.0067138671875	\\
0.704976250721357	-0.006439208984375	\\
0.705020641896391	-0.006195068359375	\\
0.705065033071425	-0.00579833984375	\\
0.70510942424646	-0.00482177734375	\\
0.705153815421494	-0.004364013671875	\\
0.705198206596529	-0.003814697265625	\\
0.705242597771563	-0.00335693359375	\\
0.705286988946597	-0.002960205078125	\\
0.705331380121632	-0.002777099609375	\\
0.705375771296666	-0.0028076171875	\\
0.705420162471701	-0.002838134765625	\\
0.705464553646735	-0.002655029296875	\\
0.705508944821769	-0.002197265625	\\
0.705553335996804	-0.002349853515625	\\
0.705597727171838	-0.00213623046875	\\
0.705642118346873	-0.001220703125	\\
0.705686509521907	-0.000946044921875	\\
0.705730900696941	-0.001251220703125	\\
0.705775291871976	-0.001983642578125	\\
0.70581968304701	-0.002044677734375	\\
0.705864074222045	-0.002166748046875	\\
0.705908465397079	-0.00274658203125	\\
0.705952856572113	-0.00299072265625	\\
0.705997247747148	-0.0032958984375	\\
0.706041638922182	-0.003173828125	\\
0.706086030097217	-0.003387451171875	\\
0.706130421272251	-0.003662109375	\\
0.706174812447286	-0.00396728515625	\\
0.70621920362232	-0.00433349609375	\\
0.706263594797354	-0.004425048828125	\\
0.706307985972389	-0.004486083984375	\\
0.706352377147423	-0.0048828125	\\
0.706396768322457	-0.004547119140625	\\
0.706441159497492	-0.004302978515625	\\
0.706485550672526	-0.004425048828125	\\
0.706529941847561	-0.0040283203125	\\
0.706574333022595	-0.00360107421875	\\
0.70661872419763	-0.003509521484375	\\
0.706663115372664	-0.003570556640625	\\
0.706707506547698	-0.00396728515625	\\
0.706751897722733	-0.003662109375	\\
0.706796288897767	-0.003387451171875	\\
0.706840680072802	-0.003448486328125	\\
0.706885071247836	-0.00299072265625	\\
0.70692946242287	-0.002410888671875	\\
0.706973853597905	-0.00189208984375	\\
0.707018244772939	-0.001800537109375	\\
0.707062635947974	-0.00128173828125	\\
0.707107027123008	-0.000762939453125	\\
0.707151418298042	-0.000396728515625	\\
0.707195809473077	-0.000335693359375	\\
0.707240200648111	-0.000244140625	\\
0.707284591823146	0.0001220703125	\\
0.70732898299818	-0.000244140625	\\
0.707373374173214	-0.000457763671875	\\
0.707417765348249	-0.000274658203125	\\
0.707462156523283	-0.0003662109375	\\
0.707506547698318	-0.0006103515625	\\
0.707550938873352	-0.00079345703125	\\
0.707595330048386	-0.000946044921875	\\
0.707639721223421	-0.001708984375	\\
0.707684112398455	-0.00213623046875	\\
0.70772850357349	-0.002410888671875	\\
0.707772894748524	-0.002532958984375	\\
0.707817285923558	-0.00250244140625	\\
0.707861677098593	-0.002593994140625	\\
0.707906068273627	-0.00238037109375	\\
0.707950459448662	-0.002471923828125	\\
0.707994850623696	-0.0028076171875	\\
0.70803924179873	-0.0030517578125	\\
0.708083632973765	-0.003143310546875	\\
0.708128024148799	-0.0035400390625	\\
0.708172415323834	-0.003875732421875	\\
0.708216806498868	-0.004180908203125	\\
0.708261197673902	-0.00421142578125	\\
0.708305588848937	-0.00421142578125	\\
0.708349980023971	-0.0042724609375	\\
0.708394371199006	-0.00408935546875	\\
0.70843876237404	-0.003753662109375	\\
0.708483153549074	-0.00335693359375	\\
0.708527544724109	-0.003173828125	\\
0.708571935899143	-0.003143310546875	\\
0.708616327074178	-0.003082275390625	\\
0.708660718249212	-0.002685546875	\\
0.708705109424246	-0.002685546875	\\
0.708749500599281	-0.001953125	\\
0.708793891774315	-0.00177001953125	\\
0.70883828294935	-0.001922607421875	\\
0.708882674124384	-0.0015869140625	\\
0.708927065299419	-0.00146484375	\\
0.708971456474453	-0.001983642578125	\\
0.709015847649487	-0.002166748046875	\\
0.709060238824522	-0.00213623046875	\\
0.709104629999556	-0.002166748046875	\\
0.70914902117459	-0.002044677734375	\\
0.709193412349625	-0.00225830078125	\\
0.709237803524659	-0.002532958984375	\\
0.709282194699694	-0.002471923828125	\\
0.709326585874728	-0.0025634765625	\\
0.709370977049762	-0.002777099609375	\\
0.709415368224797	-0.002838134765625	\\
0.709459759399831	-0.0035400390625	\\
0.709504150574866	-0.003936767578125	\\
0.7095485417499	-0.004180908203125	\\
0.709592932924935	-0.00445556640625	\\
0.709637324099969	-0.0047607421875	\\
0.709681715275003	-0.0048828125	\\
0.709726106450038	-0.00457763671875	\\
0.709770497625072	-0.00482177734375	\\
0.709814888800107	-0.00469970703125	\\
0.709859279975141	-0.004547119140625	\\
0.709903671150175	-0.0047607421875	\\
0.70994806232521	-0.004425048828125	\\
0.709992453500244	-0.004364013671875	\\
0.710036844675279	-0.004486083984375	\\
0.710081235850313	-0.004913330078125	\\
0.710125627025347	-0.0048828125	\\
0.710170018200382	-0.004425048828125	\\
0.710214409375416	-0.0045166015625	\\
0.710258800550451	-0.0040283203125	\\
};
\addplot [color=blue,solid,forget plot]
  table[row sep=crcr]{
0.710258800550451	-0.0040283203125	\\
0.710303191725485	-0.003662109375	\\
0.710347582900519	-0.003387451171875	\\
0.710391974075554	-0.002593994140625	\\
0.710436365250588	-0.00189208984375	\\
0.710480756425623	-0.0020751953125	\\
0.710525147600657	-0.00164794921875	\\
0.710569538775691	-0.001129150390625	\\
0.710613929950726	-0.00103759765625	\\
0.71065832112576	-0.000244140625	\\
0.710702712300795	0.000518798828125	\\
0.710747103475829	0.00079345703125	\\
0.710791494650863	0.000701904296875	\\
0.710835885825898	0.000701904296875	\\
0.710880277000932	0.0008544921875	\\
0.710924668175967	9.1552734375e-05	\\
0.710969059351001	9.1552734375e-05	\\
0.711013450526035	0.00054931640625	\\
0.71105784170107	0.00042724609375	\\
0.711102232876104	0.000701904296875	\\
0.711146624051139	0.0010986328125	\\
0.711191015226173	0.00103759765625	\\
0.711235406401207	0.0010986328125	\\
0.711279797576242	0.00079345703125	\\
0.711324188751276	0.00030517578125	\\
0.711368579926311	0.000213623046875	\\
0.711412971101345	-0.0001220703125	\\
0.711457362276379	-0.0003662109375	\\
0.711501753451414	-0.000152587890625	\\
0.711546144626448	-0.000823974609375	\\
0.711590535801483	-0.00140380859375	\\
0.711634926976517	-0.0018310546875	\\
0.711679318151552	-0.00244140625	\\
0.711723709326586	-0.002471923828125	\\
0.71176810050162	-0.00299072265625	\\
0.711812491676655	-0.0029296875	\\
0.711856882851689	-0.0030517578125	\\
0.711901274026724	-0.00347900390625	\\
0.711945665201758	-0.003509521484375	\\
0.711990056376792	-0.00408935546875	\\
0.712034447551827	-0.004791259765625	\\
0.712078838726861	-0.004486083984375	\\
0.712123229901895	-0.00408935546875	\\
0.71216762107693	-0.00494384765625	\\
0.712212012251964	-0.004974365234375	\\
0.712256403426999	-0.0047607421875	\\
0.712300794602033	-0.005035400390625	\\
0.712345185777068	-0.004638671875	\\
0.712389576952102	-0.00421142578125	\\
0.712433968127136	-0.00384521484375	\\
0.712478359302171	-0.003173828125	\\
0.712522750477205	-0.00262451171875	\\
0.71256714165224	-0.00177001953125	\\
0.712611532827274	-0.000885009765625	\\
0.712655924002308	0	\\
0.712700315177343	0.00042724609375	\\
0.712744706352377	0.000274658203125	\\
0.712789097527412	-0.0001220703125	\\
0.712833488702446	-0.000823974609375	\\
0.71287787987748	-0.001129150390625	\\
0.712922271052515	-0.0010986328125	\\
0.712966662227549	-0.0010986328125	\\
0.713011053402584	-0.000640869140625	\\
0.713055444577618	-0.00042724609375	\\
0.713099835752652	0.00018310546875	\\
0.713144226927687	0.0001220703125	\\
0.713188618102721	-0.00030517578125	\\
0.713233009277756	-0.000457763671875	\\
0.71327740045279	-0.00079345703125	\\
0.713321791627824	-0.000823974609375	\\
0.713366182802859	-0.001129150390625	\\
0.713410573977893	-0.00164794921875	\\
0.713454965152928	-0.0018310546875	\\
0.713499356327962	-0.002288818359375	\\
0.713543747502996	-0.00311279296875	\\
0.713588138678031	-0.003448486328125	\\
0.713632529853065	-0.003875732421875	\\
0.7136769210281	-0.004241943359375	\\
0.713721312203134	-0.004486083984375	\\
0.713765703378168	-0.0047607421875	\\
0.713810094553203	-0.004241943359375	\\
0.713854485728237	-0.003692626953125	\\
0.713898876903272	-0.00390625	\\
0.713943268078306	-0.00335693359375	\\
0.71398765925334	-0.002777099609375	\\
0.714032050428375	-0.00244140625	\\
0.714076441603409	-0.00250244140625	\\
0.714120832778444	-0.0028076171875	\\
0.714165223953478	-0.00225830078125	\\
0.714209615128512	-0.002105712890625	\\
0.714254006303547	-0.001953125	\\
0.714298397478581	-0.001617431640625	\\
0.714342788653616	-0.001556396484375	\\
0.71438717982865	-0.001373291015625	\\
0.714431571003684	-0.001129150390625	\\
0.714475962178719	-0.0008544921875	\\
0.714520353353753	3.0517578125e-05	\\
0.714564744528788	0.0003662109375	\\
0.714609135703822	0.000518798828125	\\
0.714653526878857	0.00103759765625	\\
0.714697918053891	0.001617431640625	\\
0.714742309228925	0.001800537109375	\\
0.71478670040396	0.001953125	\\
0.714831091578994	0.001678466796875	\\
0.714875482754028	0.000823974609375	\\
0.714919873929063	0.000885009765625	\\
0.714964265104097	0.000457763671875	\\
0.715008656279132	0	\\
0.715053047454166	-0.0009765625	\\
0.715097438629201	-0.001312255859375	\\
0.715141829804235	-0.00146484375	\\
0.715186220979269	-0.001708984375	\\
0.715230612154304	-0.0013427734375	\\
0.715275003329338	-0.001373291015625	\\
0.715319394504373	-0.0010986328125	\\
0.715363785679407	-0.0010986328125	\\
0.715408176854441	-0.0010986328125	\\
0.715452568029476	-0.001251220703125	\\
0.71549695920451	-0.0013427734375	\\
0.715541350379545	-0.001434326171875	\\
0.715585741554579	-0.00213623046875	\\
0.715630132729613	-0.0020751953125	\\
0.715674523904648	-0.001739501953125	\\
0.715718915079682	-0.001922607421875	\\
0.715763306254717	-0.001983642578125	\\
0.715807697429751	-0.001800537109375	\\
0.715852088604785	-0.002044677734375	\\
0.71589647977982	-0.001708984375	\\
0.715940870954854	-0.001434326171875	\\
0.715985262129889	-0.001007080078125	\\
0.716029653304923	-0.001068115234375	\\
0.716074044479957	-0.00115966796875	\\
0.716118435654992	-0.0008544921875	\\
0.716162826830026	-0.000396728515625	\\
0.716207218005061	-3.0517578125e-05	\\
0.716251609180095	-0.00042724609375	\\
0.716296000355129	-0.00067138671875	\\
0.716340391530164	-0.000701904296875	\\
0.716384782705198	-0.001007080078125	\\
0.716429173880233	-0.0015869140625	\\
0.716473565055267	-0.00128173828125	\\
0.716517956230301	-0.001556396484375	\\
0.716562347405336	-0.001708984375	\\
0.71660673858037	-0.00164794921875	\\
0.716651129755405	-0.00201416015625	\\
0.716695520930439	-0.002105712890625	\\
0.716739912105473	-0.00244140625	\\
0.716784303280508	-0.00244140625	\\
0.716828694455542	-0.00238037109375	\\
0.716873085630577	-0.00286865234375	\\
0.716917476805611	-0.0028076171875	\\
0.716961867980645	-0.003448486328125	\\
0.71700625915568	-0.00372314453125	\\
0.717050650330714	-0.003631591796875	\\
0.717095041505749	-0.003814697265625	\\
0.717139432680783	-0.0037841796875	\\
0.717183823855817	-0.003936767578125	\\
0.717228215030852	-0.00390625	\\
0.717272606205886	-0.004119873046875	\\
0.717316997380921	-0.004364013671875	\\
0.717361388555955	-0.004150390625	\\
0.71740577973099	-0.00421142578125	\\
0.717450170906024	-0.00372314453125	\\
0.717494562081058	-0.00311279296875	\\
0.717538953256093	-0.00299072265625	\\
0.717583344431127	-0.002288818359375	\\
0.717627735606161	-0.00152587890625	\\
0.717672126781196	-0.00152587890625	\\
0.71771651795623	-0.00146484375	\\
0.717760909131265	-0.00115966796875	\\
0.717805300306299	-0.001220703125	\\
0.717849691481333	-0.00091552734375	\\
0.717894082656368	-6.103515625e-05	\\
0.717938473831402	-3.0517578125e-05	\\
0.717982865006437	-0.000335693359375	\\
0.718027256181471	-0.000396728515625	\\
0.718071647356506	-0.000213623046875	\\
0.71811603853154	-9.1552734375e-05	\\
0.718160429706574	9.1552734375e-05	\\
0.718204820881609	0.000457763671875	\\
0.718249212056643	0.0006103515625	\\
0.718293603231678	0.000640869140625	\\
0.718337994406712	0.000762939453125	\\
0.718382385581746	0.000518798828125	\\
0.718426776756781	0.000579833984375	\\
0.718471167931815	0.0006103515625	\\
0.71851555910685	0.0006103515625	\\
0.718559950281884	0.00042724609375	\\
0.718604341456918	-3.0517578125e-05	\\
0.718648732631953	-0.000396728515625	\\
0.718693123806987	-0.000518798828125	\\
0.718737514982022	-0.0006103515625	\\
0.718781906157056	-0.00030517578125	\\
0.71882629733209	-0.00091552734375	\\
0.718870688507125	-0.001373291015625	\\
0.718915079682159	-0.001007080078125	\\
0.718959470857194	-0.001129150390625	\\
0.719003862032228	-0.001678466796875	\\
0.719048253207262	-0.001953125	\\
0.719092644382297	-0.002197265625	\\
0.719137035557331	-0.002593994140625	\\
0.719181426732366	-0.00250244140625	\\
0.7192258179074	-0.0028076171875	\\
0.719270209082434	-0.003662109375	\\
0.719314600257469	-0.003814697265625	\\
0.719358991432503	-0.00372314453125	\\
0.719403382607538	-0.004180908203125	\\
0.719447773782572	-0.00421142578125	\\
0.719492164957606	-0.004180908203125	\\
0.719536556132641	-0.004547119140625	\\
0.719580947307675	-0.005126953125	\\
0.71962533848271	-0.005126953125	\\
0.719669729657744	-0.00494384765625	\\
0.719714120832778	-0.004913330078125	\\
0.719758512007813	-0.004241943359375	\\
0.719802903182847	-0.0040283203125	\\
0.719847294357882	-0.003875732421875	\\
0.719891685532916	-0.0035400390625	\\
0.71993607670795	-0.002960205078125	\\
0.719980467882985	-0.002471923828125	\\
0.720024859058019	-0.0018310546875	\\
0.720069250233054	-0.001068115234375	\\
0.720113641408088	-0.0003662109375	\\
0.720158032583123	0.000396728515625	\\
0.720202423758157	0.000732421875	\\
0.720246814933191	0.000457763671875	\\
0.720291206108226	3.0517578125e-05	\\
0.72033559728326	-0.000579833984375	\\
0.720379988458295	-0.0009765625	\\
0.720424379633329	-0.000701904296875	\\
0.720468770808363	-0.00079345703125	\\
0.720513161983398	-0.000762939453125	\\
0.720557553158432	-0.000396728515625	\\
0.720601944333466	-0.00067138671875	\\
0.720646335508501	-0.000762939453125	\\
0.720690726683535	-0.0010986328125	\\
0.72073511785857	-0.001678466796875	\\
0.720779509033604	-0.00213623046875	\\
0.720823900208639	-0.002532958984375	\\
0.720868291383673	-0.00250244140625	\\
0.720912682558707	-0.002838134765625	\\
0.720957073733742	-0.002716064453125	\\
0.721001464908776	-0.003021240234375	\\
0.721045856083811	-0.003509521484375	\\
0.721090247258845	-0.003509521484375	\\
0.721134638433879	-0.0042724609375	\\
0.721179029608914	-0.004852294921875	\\
0.721223420783948	-0.005035400390625	\\
0.721267811958983	-0.005096435546875	\\
0.721312203134017	-0.004547119140625	\\
0.721356594309051	-0.003814697265625	\\
0.721400985484086	-0.0028076171875	\\
0.72144537665912	-0.002288818359375	\\
0.721489767834155	-0.00201416015625	\\
0.721534159009189	-0.001556396484375	\\
0.721578550184223	-0.001373291015625	\\
0.721622941359258	-0.001220703125	\\
0.721667332534292	-0.001007080078125	\\
0.721711723709327	-0.00067138671875	\\
0.721756114884361	-0.000396728515625	\\
0.721800506059395	-3.0517578125e-05	\\
0.72184489723443	0.000457763671875	\\
0.721889288409464	0.001434326171875	\\
0.721933679584499	0.001800537109375	\\
0.721978070759533	0.00244140625	\\
0.722022461934567	0.00311279296875	\\
0.722066853109602	0.0032958984375	\\
0.722111244284636	0.003692626953125	\\
0.722155635459671	0.00372314453125	\\
0.722200026634705	0.003173828125	\\
0.722244417809739	0.003204345703125	\\
0.722288808984774	0.0029296875	\\
0.722333200159808	0.0025634765625	\\
0.722377591334843	0.002288818359375	\\
0.722421982509877	0.0018310546875	\\
0.722466373684911	0.000762939453125	\\
0.722510764859946	-0.0003662109375	\\
0.72255515603498	-0.00091552734375	\\
0.722599547210015	-0.001739501953125	\\
0.722643938385049	-0.0020751953125	\\
0.722688329560083	-0.001861572265625	\\
0.722732720735118	-0.001983642578125	\\
0.722777111910152	-0.001708984375	\\
0.722821503085187	-0.001434326171875	\\
0.722865894260221	-0.00140380859375	\\
0.722910285435255	-0.00164794921875	\\
0.72295467661029	-0.00201416015625	\\
0.722999067785324	-0.002105712890625	\\
0.723043458960359	-0.00262451171875	\\
0.723087850135393	-0.002655029296875	\\
0.723132241310428	-0.002227783203125	\\
0.723176632485462	-0.002655029296875	\\
0.723221023660496	-0.0025634765625	\\
0.723265414835531	-0.00225830078125	\\
0.723309806010565	-0.002105712890625	\\
0.723354197185599	-0.001678466796875	\\
0.723398588360634	-0.00128173828125	\\
0.723442979535668	-0.001007080078125	\\
0.723487370710703	-0.00079345703125	\\
0.723531761885737	-0.00030517578125	\\
0.723576153060772	0.000457763671875	\\
0.723620544235806	0.000762939453125	\\
0.72366493541084	0.000762939453125	\\
0.723709326585875	0.00091552734375	\\
0.723753717760909	0.0010986328125	\\
0.723798108935944	0.000762939453125	\\
0.723842500110978	0.0010986328125	\\
0.723886891286012	0.000762939453125	\\
0.723931282461047	0.000396728515625	\\
0.723975673636081	0.000762939453125	\\
0.724020064811116	0.000274658203125	\\
0.72406445598615	9.1552734375e-05	\\
0.724108847161184	3.0517578125e-05	\\
0.724153238336219	-0.000152587890625	\\
0.724197629511253	-0.000640869140625	\\
0.724242020686288	-0.0006103515625	\\
0.724286411861322	-0.001434326171875	\\
0.724330803036356	-0.001983642578125	\\
0.724375194211391	-0.0018310546875	\\
0.724419585386425	-0.00213623046875	\\
0.72446397656146	-0.001983642578125	\\
0.724508367736494	-0.00238037109375	\\
0.724552758911528	-0.003173828125	\\
0.724597150086563	-0.0030517578125	\\
0.724641541261597	-0.00347900390625	\\
0.724685932436632	-0.003570556640625	\\
0.724730323611666	-0.00347900390625	\\
0.7247747147867	-0.0037841796875	\\
0.724819105961735	-0.00335693359375	\\
0.724863497136769	-0.003387451171875	\\
0.724907888311804	-0.00360107421875	\\
0.724952279486838	-0.00335693359375	\\
0.724996670661872	-0.003326416015625	\\
0.725041061836907	-0.003204345703125	\\
0.725085453011941	-0.00274658203125	\\
0.725129844186976	-0.00274658203125	\\
0.72517423536201	-0.002532958984375	\\
0.725218626537044	-0.001983642578125	\\
0.725263017712079	-0.00152587890625	\\
0.725307408887113	-0.00140380859375	\\
0.725351800062148	-0.001556396484375	\\
0.725396191237182	-0.001495361328125	\\
0.725440582412216	-0.001434326171875	\\
0.725484973587251	-0.001068115234375	\\
0.725529364762285	-0.00091552734375	\\
0.72557375593732	-0.00054931640625	\\
0.725618147112354	0.00030517578125	\\
0.725662538287388	0.00018310546875	\\
0.725706929462423	6.103515625e-05	\\
0.725751320637457	-3.0517578125e-05	\\
0.725795711812492	0.000518798828125	\\
0.725840102987526	0.0006103515625	\\
0.725884494162561	0.00030517578125	\\
0.725928885337595	0.000335693359375	\\
0.725973276512629	3.0517578125e-05	\\
0.726017667687664	0.000244140625	\\
0.726062058862698	0.0003662109375	\\
0.726106450037733	0.0003662109375	\\
0.726150841212767	-3.0517578125e-05	\\
0.726195232387801	-0.000335693359375	\\
0.726239623562836	-0.0001220703125	\\
0.72628401473787	-0.000457763671875	\\
0.726328405912904	-0.00067138671875	\\
0.726372797087939	-0.000518798828125	\\
0.726417188262973	-3.0517578125e-05	\\
0.726461579438008	-0.000244140625	\\
0.726505970613042	-0.00042724609375	\\
0.726550361788077	-0.00042724609375	\\
0.726594752963111	-0.000762939453125	\\
0.726639144138145	-0.0008544921875	\\
0.72668353531318	-0.0008544921875	\\
0.726727926488214	-0.00177001953125	\\
0.726772317663249	-0.002166748046875	\\
0.726816708838283	-0.00238037109375	\\
0.726861100013317	-0.00238037109375	\\
0.726905491188352	-0.002410888671875	\\
0.726949882363386	-0.002197265625	\\
0.726994273538421	-0.0020751953125	\\
0.727038664713455	-0.002532958984375	\\
0.727083055888489	-0.0023193359375	\\
0.727127447063524	-0.00213623046875	\\
0.727171838238558	-0.00225830078125	\\
0.727216229413593	-0.002471923828125	\\
0.727260620588627	-0.002685546875	\\
0.727305011763661	-0.003204345703125	\\
0.727349402938696	-0.00262451171875	\\
0.72739379411373	-0.00244140625	\\
0.727438185288765	-0.002716064453125	\\
0.727482576463799	-0.002471923828125	\\
0.727526967638833	-0.00164794921875	\\
0.727571358813868	-0.00140380859375	\\
0.727615749988902	-0.0008544921875	\\
0.727660141163937	-0.000152587890625	\\
0.727704532338971	-3.0517578125e-05	\\
0.727748923514005	0.000946044921875	\\
0.72779331468904	0.001495361328125	\\
0.727837705864074	0.001617431640625	\\
0.727882097039109	0.001861572265625	\\
0.727926488214143	0.00152587890625	\\
0.727970879389177	0.001220703125	\\
0.728015270564212	0.00079345703125	\\
0.728059661739246	0.000335693359375	\\
0.728104052914281	0.000274658203125	\\
0.728148444089315	0.00054931640625	\\
0.728192835264349	0.00067138671875	\\
0.728237226439384	0.000457763671875	\\
0.728281617614418	0.0003662109375	\\
0.728326008789453	0.000457763671875	\\
0.728370399964487	0.000518798828125	\\
0.728414791139521	6.103515625e-05	\\
0.728459182314556	-0.00067138671875	\\
0.72850357348959	-0.00079345703125	\\
0.728547964664625	-0.001068115234375	\\
0.728592355839659	-0.001495361328125	\\
0.728636747014694	-0.00146484375	\\
0.728681138189728	-0.001800537109375	\\
0.728725529364762	-0.002044677734375	\\
0.728769920539797	-0.002227783203125	\\
0.728814311714831	-0.002716064453125	\\
0.728858702889866	-0.002655029296875	\\
0.7289030940649	-0.002197265625	\\
0.728947485239934	-0.00146484375	\\
0.728991876414969	-0.00152587890625	\\
0.729036267590003	-0.001678466796875	\\
0.729080658765037	-0.001373291015625	\\
0.729125049940072	-0.001251220703125	\\
0.729169441115106	-0.000579833984375	\\
0.729213832290141	-0.000274658203125	\\
0.729258223465175	-0.00018310546875	\\
0.72930261464021	0.000457763671875	\\
0.729347005815244	0.000640869140625	\\
0.729391396990278	0.000701904296875	\\
0.729435788165313	0.001220703125	\\
0.729480179340347	0.001800537109375	\\
0.729524570515382	0.0018310546875	\\
0.729568961690416	0.001922607421875	\\
0.72961335286545	0.002227783203125	\\
0.729657744040485	0.002532958984375	\\
0.729702135215519	0.0025634765625	\\
0.729746526390554	0.002777099609375	\\
0.729790917565588	0.00299072265625	\\
0.729835308740622	0.00286865234375	\\
0.729879699915657	0.0025634765625	\\
0.729924091090691	0.002471923828125	\\
0.729968482265726	0.002532958984375	\\
0.73001287344076	0.002227783203125	\\
0.730057264615794	0.001708984375	\\
0.730101655790829	0.00091552734375	\\
0.730146046965863	0.000274658203125	\\
0.730190438140898	0	\\
0.730234829315932	-0.0006103515625	\\
0.730279220490966	-0.0008544921875	\\
0.730323611666001	-0.001007080078125	\\
0.730368002841035	-0.001617431640625	\\
0.73041239401607	-0.001617431640625	\\
0.730456785191104	-0.001495361328125	\\
0.730501176366138	-0.00189208984375	\\
0.730545567541173	-0.001708984375	\\
0.730589958716207	-0.00146484375	\\
0.730634349891242	-0.001953125	\\
0.730678741066276	-0.0020751953125	\\
0.73072313224131	-0.0015869140625	\\
0.730767523416345	-0.00164794921875	\\
0.730811914591379	-0.0018310546875	\\
0.730856305766414	-0.00152587890625	\\
0.730900696941448	-0.001800537109375	\\
0.730945088116482	-0.002105712890625	\\
0.730989479291517	-0.00164794921875	\\
0.731033870466551	-0.001617431640625	\\
0.731078261641586	-0.001678466796875	\\
0.73112265281662	-0.0009765625	\\
0.731167043991654	-0.0006103515625	\\
0.731211435166689	-0.00048828125	\\
0.731255826341723	-0.000244140625	\\
0.731300217516758	-0.0001220703125	\\
0.731344608691792	-3.0517578125e-05	\\
0.731388999866826	0	\\
0.731433391041861	0.00018310546875	\\
0.731477782216895	-0.0001220703125	\\
0.73152217339193	-0.00018310546875	\\
0.731566564566964	0.0003662109375	\\
0.731610955741999	0.00042724609375	\\
0.731655346917033	0.0003662109375	\\
0.731699738092067	0.00030517578125	\\
0.731744129267102	-9.1552734375e-05	\\
0.731788520442136	-9.1552734375e-05	\\
0.73183291161717	-0.000640869140625	\\
0.731877302792205	-0.001220703125	\\
0.731921693967239	-0.0013427734375	\\
0.731966085142274	-0.001434326171875	\\
0.732010476317308	-0.001800537109375	\\
0.732054867492343	-0.002532958984375	\\
0.732099258667377	-0.00274658203125	\\
0.732143649842411	-0.0029296875	\\
0.732188041017446	-0.00323486328125	\\
0.73223243219248	-0.003631591796875	\\
0.732276823367515	-0.003997802734375	\\
0.732321214542549	-0.004180908203125	\\
0.732365605717583	-0.004364013671875	\\
0.732409996892618	-0.004638671875	\\
0.732454388067652	-0.004852294921875	\\
0.732498779242687	-0.004486083984375	\\
0.732543170417721	-0.00396728515625	\\
0.732587561592755	-0.00347900390625	\\
0.73263195276779	-0.00341796875	\\
0.732676343942824	-0.003265380859375	\\
0.732720735117859	-0.003021240234375	\\
0.732765126292893	-0.00262451171875	\\
0.732809517467927	-0.00201416015625	\\
0.732853908642962	-0.0023193359375	\\
0.732898299817996	-0.001739501953125	\\
0.732942690993031	-0.001373291015625	\\
0.732987082168065	-0.001220703125	\\
0.733031473343099	-0.000732421875	\\
0.733075864518134	-9.1552734375e-05	\\
0.733120255693168	0.0003662109375	\\
0.733164646868203	0.00042724609375	\\
0.733209038043237	0.000579833984375	\\
0.733253429218271	0.000762939453125	\\
0.733297820393306	0.000701904296875	\\
0.73334221156834	0.000732421875	\\
0.733386602743375	0.00140380859375	\\
0.733430993918409	0.001495361328125	\\
0.733475385093443	0.001708984375	\\
0.733519776268478	0.00164794921875	\\
0.733564167443512	0.001251220703125	\\
0.733608558618547	0.001617431640625	\\
0.733652949793581	0.00128173828125	\\
0.733697340968615	0.00140380859375	\\
0.73374173214365	0.0010986328125	\\
0.733786123318684	0.000946044921875	\\
0.733830514493719	0.001251220703125	\\
0.733874905668753	0.000244140625	\\
0.733919296843787	0.000640869140625	\\
0.733963688018822	0.0008544921875	\\
0.734008079193856	9.1552734375e-05	\\
0.734052470368891	0.00018310546875	\\
0.734096861543925	-0.000213623046875	\\
0.734141252718959	6.103515625e-05	\\
0.734185643893994	-3.0517578125e-05	\\
0.734230035069028	-0.00067138671875	\\
0.734274426244063	-0.000274658203125	\\
0.734318817419097	-0.00067138671875	\\
0.734363208594132	-0.000885009765625	\\
0.734407599769166	-0.00091552734375	\\
0.7344519909442	-0.000946044921875	\\
0.734496382119235	-0.001220703125	\\
0.734540773294269	-0.001495361328125	\\
0.734585164469304	-0.001800537109375	\\
0.734629555644338	-0.002227783203125	\\
0.734673946819372	-0.002593994140625	\\
0.734718337994407	-0.00244140625	\\
0.734762729169441	-0.00262451171875	\\
0.734807120344475	-0.002960205078125	\\
0.73485151151951	-0.002777099609375	\\
0.734895902694544	-0.003173828125	\\
0.734940293869579	-0.0032958984375	\\
0.734984685044613	-0.0029296875	\\
0.735029076219648	-0.002899169921875	\\
0.735073467394682	-0.002716064453125	\\
0.735117858569716	-0.002471923828125	\\
0.735162249744751	-0.002288818359375	\\
0.735206640919785	-0.0023193359375	\\
0.73525103209482	-0.0020751953125	\\
0.735295423269854	-0.001373291015625	\\
0.735339814444888	-0.001190185546875	\\
0.735384205619923	-0.000823974609375	\\
0.735428596794957	0.00030517578125	\\
0.735472987969992	0.0006103515625	\\
0.735517379145026	0.00030517578125	\\
0.73556177032006	0.00042724609375	\\
0.735606161495095	0.0003662109375	\\
0.735650552670129	-3.0517578125e-05	\\
0.735694943845164	3.0517578125e-05	\\
0.735739335020198	-6.103515625e-05	\\
0.735783726195232	-0.000244140625	\\
0.735828117370267	0.00030517578125	\\
0.735872508545301	9.1552734375e-05	\\
0.735916899720336	0.000244140625	\\
0.73596129089537	0.00054931640625	\\
0.736005682070404	0.00042724609375	\\
0.736050073245439	0.000335693359375	\\
0.736094464420473	0	\\
0.736138855595508	-0.000396728515625	\\
0.736183246770542	-0.000274658203125	\\
0.736227637945576	-0.000518798828125	\\
0.736272029120611	-0.00128173828125	\\
0.736316420295645	-0.00146484375	\\
0.73636081147068	-0.001190185546875	\\
0.736405202645714	-0.0015869140625	\\
0.736449593820748	-0.001678466796875	\\
0.736493984995783	-0.001495361328125	\\
0.736538376170817	-0.001495361328125	\\
0.736582767345852	-0.001220703125	\\
0.736627158520886	-0.00115966796875	\\
0.73667154969592	-0.000823974609375	\\
0.736715940870955	-0.00042724609375	\\
0.736760332045989	0.0001220703125	\\
0.736804723221024	0.000823974609375	\\
0.736849114396058	0.00115966796875	\\
0.736893505571092	0.00091552734375	\\
0.736937896746127	0.001251220703125	\\
0.736982287921161	0.0015869140625	\\
0.737026679096196	0.001373291015625	\\
0.73707107027123	0.001556396484375	\\
0.737115461446265	0.001678466796875	\\
0.737159852621299	0.002105712890625	\\
0.737204243796333	0.00238037109375	\\
0.737248634971368	0.002532958984375	\\
0.737293026146402	0.002899169921875	\\
0.737337417321437	0.003265380859375	\\
0.737381808496471	0.003173828125	\\
0.737426199671505	0.003265380859375	\\
0.73747059084654	0.003143310546875	\\
0.737514982021574	0.002960205078125	\\
0.737559373196608	0.003387451171875	\\
0.737603764371643	0.0029296875	\\
0.737648155546677	0.002716064453125	\\
0.737692546721712	0.00299072265625	\\
0.737736937896746	0.002349853515625	\\
0.737781329071781	0.001434326171875	\\
0.737825720246815	0.000518798828125	\\
0.737870111421849	-0.000152587890625	\\
0.737914502596884	-0.000732421875	\\
0.737958893771918	-0.0009765625	\\
0.738003284946953	-0.0008544921875	\\
0.738047676121987	-0.00115966796875	\\
0.738092067297021	-0.001220703125	\\
0.738136458472056	-0.000732421875	\\
0.73818084964709	-0.001190185546875	\\
0.738225240822125	-0.001739501953125	\\
0.738269631997159	-0.001800537109375	\\
0.738314023172193	-0.00213623046875	\\
0.738358414347228	-0.0020751953125	\\
0.738402805522262	-0.00201416015625	\\
0.738447196697297	-0.002044677734375	\\
0.738491587872331	-0.00201416015625	\\
0.738535979047365	-0.001861572265625	\\
0.7385803702224	-0.0015869140625	\\
0.738624761397434	-0.001678466796875	\\
0.738669152572469	-0.00152587890625	\\
0.738713543747503	-0.001373291015625	\\
0.738757934922537	-0.00091552734375	\\
0.738802326097572	-0.000762939453125	\\
0.738846717272606	-0.00018310546875	\\
0.738891108447641	-0.000274658203125	\\
0.738935499622675	-0.000579833984375	\\
0.738979890797709	-0.00030517578125	\\
0.739024281972744	-0.0001220703125	\\
0.739068673147778	-0.00042724609375	\\
0.739113064322813	-0.0008544921875	\\
0.739157455497847	-0.001068115234375	\\
0.739201846672881	-0.001068115234375	\\
0.739246237847916	-0.000762939453125	\\
0.73929062902295	-0.000823974609375	\\
0.739335020197985	-0.001190185546875	\\
0.739379411373019	-0.00103759765625	\\
0.739423802548053	-0.00128173828125	\\
0.739468193723088	-0.001617431640625	\\
0.739512584898122	-0.0018310546875	\\
0.739556976073157	-0.00238037109375	\\
0.739601367248191	-0.00250244140625	\\
0.739645758423225	-0.00274658203125	\\
0.73969014959826	-0.002899169921875	\\
0.739734540773294	-0.0029296875	\\
0.739778931948329	-0.00311279296875	\\
0.739823323123363	-0.00323486328125	\\
0.739867714298397	-0.00372314453125	\\
0.739912105473432	-0.003936767578125	\\
0.739956496648466	-0.00372314453125	\\
0.740000887823501	-0.004302978515625	\\
0.740045278998535	-0.004486083984375	\\
0.74008967017357	-0.004241943359375	\\
0.740134061348604	-0.004119873046875	\\
0.740178452523638	-0.003936767578125	\\
0.740222843698673	-0.0032958984375	\\
0.740267234873707	-0.00335693359375	\\
0.740311626048741	-0.00347900390625	\\
0.740356017223776	-0.002838134765625	\\
0.74040040839881	-0.00274658203125	\\
0.740444799573845	-0.001953125	\\
0.740489190748879	-0.001739501953125	\\
0.740533581923914	-0.002105712890625	\\
0.740577973098948	-0.001495361328125	\\
0.740622364273982	-0.00115966796875	\\
0.740666755449017	-0.001007080078125	\\
0.740711146624051	-0.0006103515625	\\
0.740755537799086	-0.000518798828125	\\
0.74079992897412	-0.00018310546875	\\
0.740844320149154	-9.1552734375e-05	\\
0.740888711324189	0.000152587890625	\\
0.740933102499223	0.000244140625	\\
0.740977493674258	0.00030517578125	\\
0.741021884849292	0.00079345703125	\\
0.741066276024326	0.000518798828125	\\
0.741110667199361	0.000518798828125	\\
0.741155058374395	0.00054931640625	\\
0.74119944954943	0.0006103515625	\\
0.741243840724464	0.00103759765625	\\
0.741288231899498	0.001251220703125	\\
0.741332623074533	0.0009765625	\\
0.741377014249567	0.00079345703125	\\
0.741421405424602	0.0009765625	\\
0.741465796599636	0.000732421875	\\
0.74151018777467	0.000579833984375	\\
0.741554578949705	-0.00018310546875	\\
0.741598970124739	-0.00054931640625	\\
0.741643361299774	6.103515625e-05	\\
0.741687752474808	6.103515625e-05	\\
0.741732143649842	0.0001220703125	\\
0.741776534824877	-0.000244140625	\\
0.741820925999911	-0.0006103515625	\\
0.741865317174946	-0.0009765625	\\
0.74190970834998	-0.0013427734375	\\
0.741954099525014	-0.00146484375	\\
0.741998490700049	-0.001434326171875	\\
0.742042881875083	-0.00177001953125	\\
0.742087273050118	-0.00177001953125	\\
0.742131664225152	-0.0020751953125	\\
0.742176055400187	-0.002655029296875	\\
0.742220446575221	-0.002685546875	\\
0.742264837750255	-0.0028076171875	\\
0.74230922892529	-0.002960205078125	\\
0.742353620100324	-0.003143310546875	\\
0.742398011275358	-0.003143310546875	\\
0.742442402450393	-0.00335693359375	\\
0.742486793625427	-0.00341796875	\\
0.742531184800462	-0.003753662109375	\\
0.742575575975496	-0.003814697265625	\\
0.74261996715053	-0.004058837890625	\\
0.742664358325565	-0.0040283203125	\\
0.742708749500599	-0.003814697265625	\\
0.742753140675634	-0.003753662109375	\\
0.742797531850668	-0.003509521484375	\\
0.742841923025703	-0.00341796875	\\
0.742886314200737	-0.003509521484375	\\
0.742930705375771	-0.003448486328125	\\
0.742975096550806	-0.00286865234375	\\
0.74301948772584	-0.00274658203125	\\
0.743063878900875	-0.00286865234375	\\
0.743108270075909	-0.002197265625	\\
0.743152661250943	-0.001312255859375	\\
0.743197052425978	-0.000457763671875	\\
0.743241443601012	-0.00079345703125	\\
0.743285834776046	-0.000640869140625	\\
0.743330225951081	-0.00042724609375	\\
0.743374617126115	-0.000823974609375	\\
0.74341900830115	-0.000885009765625	\\
0.743463399476184	-0.0008544921875	\\
0.743507790651219	-0.0008544921875	\\
0.743552181826253	-0.000701904296875	\\
0.743596573001287	-0.0009765625	\\
0.743640964176322	-0.000701904296875	\\
0.743685355351356	-0.000152587890625	\\
0.743729746526391	0	\\
0.743774137701425	-0.00018310546875	\\
0.743818528876459	-0.000213623046875	\\
0.743862920051494	-6.103515625e-05	\\
0.743907311226528	-0.000823974609375	\\
0.743951702401563	-0.000885009765625	\\
0.743996093576597	-0.000762939453125	\\
0.744040484751631	-0.000732421875	\\
0.744084875926666	-0.00054931640625	\\
0.7441292671017	-0.000885009765625	\\
0.744173658276735	-0.000640869140625	\\
0.744218049451769	-0.000457763671875	\\
0.744262440626803	-0.00067138671875	\\
0.744306831801838	-0.000579833984375	\\
0.744351222976872	-0.0001220703125	\\
0.744395614151907	0.000274658203125	\\
0.744440005326941	0.000213623046875	\\
0.744484396501975	0.00048828125	\\
0.74452878767701	0.0006103515625	\\
0.744573178852044	0.000701904296875	\\
0.744617570027079	0.001251220703125	\\
0.744661961202113	0.001190185546875	\\
0.744706352377147	0.000823974609375	\\
0.744750743552182	0.000946044921875	\\
0.744795134727216	0.0009765625	\\
0.744839525902251	0.000946044921875	\\
0.744883917077285	0.000823974609375	\\
0.744928308252319	0.001068115234375	\\
0.744972699427354	0.001220703125	\\
0.745017090602388	0.0013427734375	\\
0.745061481777423	0.001739501953125	\\
0.745105872952457	0.001800537109375	\\
0.745150264127491	0.00201416015625	\\
0.745194655302526	0.002349853515625	\\
0.74523904647756	0.002532958984375	\\
0.745283437652595	0.0023193359375	\\
0.745327828827629	0.001983642578125	\\
0.745372220002663	0.001953125	\\
0.745416611177698	0.0018310546875	\\
0.745461002352732	0.001708984375	\\
0.745505393527767	0.001190185546875	\\
0.745549784702801	0.000701904296875	\\
0.745594175877836	0.00042724609375	\\
0.74563856705287	0.00018310546875	\\
0.745682958227904	-3.0517578125e-05	\\
0.745727349402939	-0.000518798828125	\\
0.745771740577973	-0.0008544921875	\\
0.745816131753008	-0.000701904296875	\\
0.745860522928042	-0.000732421875	\\
0.745904914103076	-0.000701904296875	\\
0.745949305278111	-0.000946044921875	\\
0.745993696453145	-0.00103759765625	\\
0.746038087628179	-0.0009765625	\\
0.746082478803214	-0.000732421875	\\
0.746126869978248	-0.001068115234375	\\
0.746171261153283	-0.00103759765625	\\
0.746215652328317	-0.001129150390625	\\
0.746260043503352	-0.00152587890625	\\
0.746304434678386	-0.00164794921875	\\
0.74634882585342	-0.0018310546875	\\
0.746393217028455	-0.001800537109375	\\
0.746437608203489	-0.001708984375	\\
0.746481999378524	-0.00146484375	\\
0.746526390553558	-0.001373291015625	\\
0.746570781728592	-0.001190185546875	\\
0.746615172903627	-0.000762939453125	\\
0.746659564078661	-0.001129150390625	\\
0.746703955253696	-0.0008544921875	\\
0.74674834642873	-0.000762939453125	\\
0.746792737603764	-0.00079345703125	\\
0.746837128778799	-0.000762939453125	\\
0.746881519953833	-0.001129150390625	\\
0.746925911128868	-0.00103759765625	\\
0.746970302303902	-0.0013427734375	\\
0.747014693478936	-0.00152587890625	\\
0.747059084653971	-0.001495361328125	\\
0.747103475829005	-0.001190185546875	\\
0.74714786700404	-0.00128173828125	\\
0.747192258179074	-0.001190185546875	\\
0.747236649354108	-0.0010986328125	\\
0.747281040529143	-0.00152587890625	\\
0.747325431704177	-0.001373291015625	\\
0.747369822879212	-0.0013427734375	\\
0.747414214054246	-0.00152587890625	\\
0.74745860522928	-0.00146484375	\\
0.747502996404315	-0.00152587890625	\\
0.747547387579349	-0.00140380859375	\\
0.747591778754384	-0.0009765625	\\
0.747636169929418	-0.0009765625	\\
0.747680561104452	-0.001373291015625	\\
0.747724952279487	-0.001678466796875	\\
0.747769343454521	-0.001739501953125	\\
0.747813734629556	-0.00140380859375	\\
0.74785812580459	-0.00164794921875	\\
0.747902516979624	-0.001251220703125	\\
0.747946908154659	-0.00042724609375	\\
0.747991299329693	-0.000396728515625	\\
0.748035690504728	-0.000518798828125	\\
0.748080081679762	-0.000274658203125	\\
0.748124472854796	-3.0517578125e-05	\\
0.748168864029831	-0.000244140625	\\
0.748213255204865	-0.000274658203125	\\
0.7482576463799	0.0003662109375	\\
0.748302037554934	0.0006103515625	\\
0.748346428729968	0.000701904296875	\\
0.748390819905003	0.000732421875	\\
0.748435211080037	0.0009765625	\\
0.748479602255072	0.00146484375	\\
0.748523993430106	0.001495361328125	\\
0.748568384605141	0.001556396484375	\\
0.748612775780175	0.001251220703125	\\
0.748657166955209	0.0008544921875	\\
0.748701558130244	0.001220703125	\\
0.748745949305278	0.00164794921875	\\
0.748790340480313	0.00213623046875	\\
0.748834731655347	0.001800537109375	\\
0.748879122830381	0.001617431640625	\\
0.748923514005416	0.0015869140625	\\
0.74896790518045	0.001556396484375	\\
0.749012296355485	0.001861572265625	\\
0.749056687530519	0.001739501953125	\\
0.749101078705553	0.00146484375	\\
0.749145469880588	0.00128173828125	\\
0.749189861055622	0.00103759765625	\\
0.749234252230657	0.000946044921875	\\
0.749278643405691	0.0006103515625	\\
0.749323034580725	0.000579833984375	\\
0.74936742575576	0.000457763671875	\\
0.749411816930794	-6.103515625e-05	\\
0.749456208105829	0.000457763671875	\\
0.749500599280863	0.00048828125	\\
0.749544990455897	-3.0517578125e-05	\\
0.749589381630932	-0.0001220703125	\\
0.749633772805966	-0.0001220703125	\\
0.749678163981001	-0.00054931640625	\\
0.749722555156035	-0.000762939453125	\\
0.749766946331069	-0.001007080078125	\\
0.749811337506104	-0.001190185546875	\\
0.749855728681138	-0.0010986328125	\\
0.749900119856173	-0.0013427734375	\\
0.749944511031207	-0.00201416015625	\\
0.749988902206241	-0.002410888671875	\\
0.750033293381276	-0.002532958984375	\\
0.75007768455631	-0.00250244140625	\\
0.750122075731345	-0.002838134765625	\\
0.750166466906379	-0.003143310546875	\\
0.750210858081413	-0.002838134765625	\\
0.750255249256448	-0.002471923828125	\\
0.750299640431482	-0.0025634765625	\\
0.750344031606517	-0.002197265625	\\
0.750388422781551	-0.0020751953125	\\
0.750432813956585	-0.002288818359375	\\
0.75047720513162	-0.002349853515625	\\
0.750521596306654	-0.002197265625	\\
0.750565987481689	-0.002471923828125	\\
0.750610378656723	-0.002044677734375	\\
0.750654769831758	-0.00164794921875	\\
0.750699161006792	-0.001708984375	\\
0.750743552181826	-0.001373291015625	\\
0.750787943356861	-0.0013427734375	\\
0.750832334531895	-0.00054931640625	\\
0.750876725706929	0.000152587890625	\\
0.750921116881964	0.0003662109375	\\
0.750965508056998	0.001434326171875	\\
0.751009899232033	0.001953125	\\
0.751054290407067	0.002197265625	\\
0.751098681582101	0.00225830078125	\\
0.751143072757136	0.002105712890625	\\
0.75118746393217	0.001739501953125	\\
0.751231855107205	0.00177001953125	\\
0.751276246282239	0.001953125	\\
0.751320637457274	0.0015869140625	\\
0.751365028632308	0.002105712890625	\\
0.751409419807342	0.00250244140625	\\
0.751453810982377	0.002777099609375	\\
0.751498202157411	0.002777099609375	\\
0.751542593332446	0.003021240234375	\\
0.75158698450748	0.003082275390625	\\
0.751631375682514	0.002777099609375	\\
0.751675766857549	0.002685546875	\\
0.751720158032583	0.002685546875	\\
0.751764549207617	0.002227783203125	\\
0.751808940382652	0.0018310546875	\\
0.751853331557686	0.00164794921875	\\
0.751897722732721	0.002044677734375	\\
0.751942113907755	0.001983642578125	\\
0.75198650508279	0.001220703125	\\
0.752030896257824	0.000762939453125	\\
0.752075287432858	0.001220703125	\\
0.752119678607893	0.001434326171875	\\
0.752164069782927	0.001556396484375	\\
0.752208460957962	0.001800537109375	\\
0.752252852132996	0.00164794921875	\\
0.75229724330803	0.00177001953125	\\
0.752341634483065	0.00201416015625	\\
0.752386025658099	0.001861572265625	\\
0.752430416833134	0.001556396484375	\\
0.752474808008168	0.0020751953125	\\
0.752519199183202	0.0015869140625	\\
0.752563590358237	0.00115966796875	\\
0.752607981533271	0.00140380859375	\\
0.752652372708306	0.00152587890625	\\
0.75269676388334	0.001739501953125	\\
0.752741155058374	0.001495361328125	\\
0.752785546233409	0.002197265625	\\
0.752829937408443	0.003173828125	\\
0.752874328583478	0.003173828125	\\
0.752918719758512	0.0037841796875	\\
0.752963110933546	0.00390625	\\
0.753007502108581	0.004150390625	\\
0.753051893283615	0.004425048828125	\\
0.75309628445865	0.00439453125	\\
0.753140675633684	0.0037841796875	\\
0.753185066808718	0.003753662109375	\\
0.753229457983753	0.003631591796875	\\
0.753273849158787	0.0030517578125	\\
0.753318240333822	0.0030517578125	\\
0.753362631508856	0.002593994140625	\\
0.75340702268389	0.002593994140625	\\
0.753451413858925	0.0025634765625	\\
0.753495805033959	0.001739501953125	\\
0.753540196208994	0.001617431640625	\\
0.753584587384028	0.00146484375	\\
0.753628978559062	0.000885009765625	\\
0.753673369734097	0.00079345703125	\\
0.753717760909131	0.000762939453125	\\
0.753762152084166	0.00067138671875	\\
0.7538065432592	0.000518798828125	\\
0.753850934434234	0.000213623046875	\\
0.753895325609269	-0.0001220703125	\\
0.753939716784303	-3.0517578125e-05	\\
0.753984107959338	-0.000457763671875	\\
0.754028499134372	-0.00115966796875	\\
0.754072890309407	-0.0010986328125	\\
0.754117281484441	-0.00115966796875	\\
0.754161672659475	-0.001495361328125	\\
0.75420606383451	-0.0013427734375	\\
0.754250455009544	-0.0013427734375	\\
0.754294846184579	-0.00140380859375	\\
0.754339237359613	-0.001190185546875	\\
0.754383628534647	-0.000732421875	\\
0.754428019709682	-0.000885009765625	\\
0.754472410884716	-0.00054931640625	\\
0.75451680205975	-6.103515625e-05	\\
0.754561193234785	-0.000152587890625	\\
0.754605584409819	-0.0006103515625	\\
0.754649975584854	-0.000885009765625	\\
0.754694366759888	-0.000732421875	\\
0.754738757934923	-0.00054931640625	\\
0.754783149109957	-0.00091552734375	\\
0.754827540284991	-0.000762939453125	\\
0.754871931460026	-0.0003662109375	\\
0.75491632263506	-9.1552734375e-05	\\
0.754960713810095	-0.000274658203125	\\
0.755005104985129	-9.1552734375e-05	\\
0.755049496160163	0.00018310546875	\\
0.755093887335198	-0.00018310546875	\\
0.755138278510232	-6.103515625e-05	\\
0.755182669685267	6.103515625e-05	\\
0.755227060860301	-3.0517578125e-05	\\
0.755271452035335	0.000396728515625	\\
0.75531584321037	0.000640869140625	\\
0.755360234385404	0.00042724609375	\\
0.755404625560439	0.00048828125	\\
0.755449016735473	0.00054931640625	\\
0.755493407910507	0.00054931640625	\\
0.755537799085542	3.0517578125e-05	\\
0.755582190260576	-0.000396728515625	\\
0.755626581435611	0.0003662109375	\\
0.755670972610645	0.00091552734375	\\
0.755715363785679	0.000946044921875	\\
0.755759754960714	0.000640869140625	\\
0.755804146135748	0.00091552734375	\\
0.755848537310783	0.001251220703125	\\
0.755892928485817	0.00140380859375	\\
0.755937319660851	0.001220703125	\\
0.755981710835886	0.000732421875	\\
0.75602610201092	0.000946044921875	\\
0.756070493185955	0.00079345703125	\\
0.756114884360989	0.00115966796875	\\
0.756159275536023	0.001251220703125	\\
0.756203666711058	0.0009765625	\\
0.756248057886092	0.0015869140625	\\
0.756292449061127	0.001434326171875	\\
0.756336840236161	0.001220703125	\\
0.756381231411196	0.001678466796875	\\
0.75642562258623	0.00164794921875	\\
0.756470013761264	0.00152587890625	\\
0.756514404936299	0.001434326171875	\\
0.756558796111333	0.00213623046875	\\
0.756603187286367	0.00213623046875	\\
0.756647578461402	0.001800537109375	\\
0.756691969636436	0.00189208984375	\\
0.756736360811471	0.0020751953125	\\
0.756780751986505	0.001495361328125	\\
0.756825143161539	0.00146484375	\\
0.756869534336574	0.00189208984375	\\
0.756913925511608	0.0018310546875	\\
0.756958316686643	0.00189208984375	\\
0.757002707861677	0.00177001953125	\\
0.757047099036712	0.001800537109375	\\
0.757091490211746	0.0018310546875	\\
0.75713588138678	0.001312255859375	\\
0.757180272561815	0.001708984375	\\
0.757224663736849	0.0018310546875	\\
0.757269054911884	0.001190185546875	\\
0.757313446086918	0.001190185546875	\\
0.757357837261952	0.000946044921875	\\
0.757402228436987	9.1552734375e-05	\\
0.757446619612021	-3.0517578125e-05	\\
0.757491010787056	0.000152587890625	\\
0.75753540196209	-0.000213623046875	\\
0.757579793137124	0.00018310546875	\\
0.757624184312159	0.000213623046875	\\
0.757668575487193	-0.0001220703125	\\
0.757712966662228	3.0517578125e-05	\\
0.757757357837262	-0.000457763671875	\\
0.757801749012296	-0.001190185546875	\\
0.757846140187331	-0.0015869140625	\\
0.757890531362365	-0.001556396484375	\\
0.7579349225374	-0.001983642578125	\\
0.757979313712434	-0.001678466796875	\\
0.758023704887468	-0.00152587890625	\\
0.758068096062503	-0.001953125	\\
0.758112487237537	-0.001739501953125	\\
0.758156878412572	-0.0018310546875	\\
0.758201269587606	-0.001708984375	\\
0.75824566076264	-0.001068115234375	\\
0.758290051937675	-0.000732421875	\\
0.758334443112709	-0.000640869140625	\\
0.758378834287744	-0.001312255859375	\\
0.758423225462778	-0.0015869140625	\\
0.758467616637812	-0.000946044921875	\\
0.758512007812847	-0.001068115234375	\\
0.758556398987881	-0.00115966796875	\\
0.758600790162916	-0.00091552734375	\\
0.75864518133795	-0.000946044921875	\\
0.758689572512984	-0.00018310546875	\\
0.758733963688019	0.000518798828125	\\
0.758778354863053	0.000457763671875	\\
0.758822746038088	0.001220703125	\\
0.758867137213122	0.001556396484375	\\
0.758911528388156	0.00164794921875	\\
0.758955919563191	0.001739501953125	\\
0.759000310738225	0.001617431640625	\\
0.75904470191326	0.00164794921875	\\
0.759089093088294	0.001708984375	\\
0.759133484263329	0.002227783203125	\\
0.759177875438363	0.002349853515625	\\
0.759222266613397	0.00244140625	\\
0.759266657788432	0.002471923828125	\\
0.759311048963466	0.002838134765625	\\
0.7593554401385	0.0030517578125	\\
0.759399831313535	0.0030517578125	\\
0.759444222488569	0.003326416015625	\\
0.759488613663604	0.003173828125	\\
0.759533004838638	0.00286865234375	\\
0.759577396013672	0.00274658203125	\\
0.759621787188707	0.002655029296875	\\
0.759666178363741	0.002593994140625	\\
0.759710569538776	0.00286865234375	\\
0.75975496071381	0.003173828125	\\
0.759799351888845	0.002471923828125	\\
0.759843743063879	0.001922607421875	\\
0.759888134238913	0.001800537109375	\\
0.759932525413948	0.001800537109375	\\
0.759976916588982	0.001800537109375	\\
0.760021307764017	0.002288818359375	\\
0.760065698939051	0.0023193359375	\\
0.760110090114085	0.002044677734375	\\
0.76015448128912	0.002197265625	\\
0.760198872464154	0.002197265625	\\
0.760243263639188	0.001953125	\\
0.760287654814223	0.0023193359375	\\
0.760332045989257	0.0025634765625	\\
0.760376437164292	0.00244140625	\\
0.760420828339326	0.002532958984375	\\
0.760465219514361	0.002655029296875	\\
0.760509610689395	0.00238037109375	\\
0.760554001864429	0.00299072265625	\\
0.760598393039464	0.002838134765625	\\
0.760642784214498	0.002838134765625	\\
0.760687175389533	0.003631591796875	\\
0.760731566564567	0.0037841796875	\\
0.760775957739601	0.004364013671875	\\
0.760820348914636	0.004547119140625	\\
0.76086474008967	0.004669189453125	\\
0.760909131264705	0.004852294921875	\\
0.760953522439739	0.004913330078125	\\
0.760997913614773	0.004852294921875	\\
0.761042304789808	0.004852294921875	\\
0.761086695964842	0.004425048828125	\\
0.761131087139877	0.00433349609375	\\
0.761175478314911	0.0042724609375	\\
0.761219869489945	0.00396728515625	\\
0.76126426066498	0.003692626953125	\\
0.761308651840014	0.003143310546875	\\
0.761353043015049	0.0025634765625	\\
0.761397434190083	0.002685546875	\\
0.761441825365117	0.0025634765625	\\
0.761486216540152	0.002227783203125	\\
0.761530607715186	0.001678466796875	\\
0.761574998890221	0.0020751953125	\\
0.761619390065255	0.002044677734375	\\
0.761663781240289	0.001220703125	\\
0.761708172415324	0.0010986328125	\\
0.761752563590358	0.000823974609375	\\
0.761796954765393	0.000396728515625	\\
0.761841345940427	0.000213623046875	\\
0.761885737115461	-9.1552734375e-05	\\
0.761930128290496	-0.00042724609375	\\
0.76197451946553	-0.000396728515625	\\
0.762018910640565	0.000274658203125	\\
0.762063301815599	0.000396728515625	\\
0.762107692990633	0.00042724609375	\\
0.762152084165668	0.00079345703125	\\
0.762196475340702	0.000885009765625	\\
0.762240866515737	0.0008544921875	\\
0.762285257690771	0.00091552734375	\\
0.762329648865805	0.00079345703125	\\
0.76237404004084	0.000732421875	\\
0.762418431215874	0.001495361328125	\\
0.762462822390909	0.001068115234375	\\
0.762507213565943	0.00079345703125	\\
0.762551604740978	0.0010986328125	\\
0.762595995916012	0.00103759765625	\\
0.762640387091046	0.001434326171875	\\
0.762684778266081	0.001495361328125	\\
0.762729169441115	0.00189208984375	\\
0.76277356061615	0.001678466796875	\\
0.762817951791184	0.001922607421875	\\
0.762862342966218	0.002838134765625	\\
0.762906734141253	0.003021240234375	\\
0.762951125316287	0.003143310546875	\\
0.762995516491322	0.003265380859375	\\
0.763039907666356	0.0029296875	\\
0.76308429884139	0.002899169921875	\\
0.763128690016425	0.00250244140625	\\
0.763173081191459	0.002410888671875	\\
0.763217472366494	0.00274658203125	\\
0.763261863541528	0.003082275390625	\\
0.763306254716562	0.003082275390625	\\
0.763350645891597	0.002899169921875	\\
0.763395037066631	0.00238037109375	\\
0.763439428241666	0.00213623046875	\\
0.7634838194167	0.002410888671875	\\
0.763528210591734	0.002532958984375	\\
0.763572601766769	0.002471923828125	\\
0.763616992941803	0.002838134765625	\\
0.763661384116838	0.0029296875	\\
0.763705775291872	0.002777099609375	\\
0.763750166466906	0.00238037109375	\\
0.763794557641941	0.00262451171875	\\
0.763838948816975	0.002777099609375	\\
0.76388333999201	0.002197265625	\\
0.763927731167044	0.002288818359375	\\
0.763972122342078	0.002349853515625	\\
0.764016513517113	0.00244140625	\\
0.764060904692147	0.00274658203125	\\
0.764105295867182	0.002471923828125	\\
0.764149687042216	0.002044677734375	\\
0.76419407821725	0.002227783203125	\\
0.764238469392285	0.002105712890625	\\
0.764282860567319	0.00250244140625	\\
0.764327251742354	0.00262451171875	\\
0.764371642917388	0.00213623046875	\\
0.764416034092422	0.002471923828125	\\
0.764460425267457	0.00250244140625	\\
0.764504816442491	0.00213623046875	\\
0.764549207617526	0.002227783203125	\\
0.76459359879256	0.0023193359375	\\
0.764637989967594	0.002197265625	\\
0.764682381142629	0.0015869140625	\\
0.764726772317663	0.00128173828125	\\
0.764771163492698	0.001800537109375	\\
0.764815554667732	0.002288818359375	\\
0.764859945842767	0.002349853515625	\\
0.764904337017801	0.00225830078125	\\
0.764948728192835	0.00274658203125	\\
0.76499311936787	0.002410888671875	\\
0.765037510542904	0.002288818359375	\\
0.765081901717938	0.00262451171875	\\
0.765126292892973	0.002166748046875	\\
0.765170684068007	0.002044677734375	\\
0.765215075243042	0.00213623046875	\\
0.765259466418076	0.002044677734375	\\
0.76530385759311	0.0018310546875	\\
0.765348248768145	0.001312255859375	\\
0.765392639943179	0.00103759765625	\\
0.765437031118214	0.001251220703125	\\
0.765481422293248	0.001434326171875	\\
0.765525813468283	0.001251220703125	\\
0.765570204643317	0.00091552734375	\\
0.765614595818351	0.00103759765625	\\
0.765658986993386	0.001312255859375	\\
0.76570337816842	0.000885009765625	\\
0.765747769343455	0.0008544921875	\\
0.765792160518489	0.001007080078125	\\
0.765836551693523	0.00091552734375	\\
0.765880942868558	0.001007080078125	\\
0.765925334043592	0.0009765625	\\
0.765969725218627	0.000518798828125	\\
0.766014116393661	0.000823974609375	\\
0.766058507568695	0.0009765625	\\
0.76610289874373	0.0009765625	\\
0.766147289918764	0.001068115234375	\\
0.766191681093799	0.00030517578125	\\
0.766236072268833	-3.0517578125e-05	\\
0.766280463443867	0.000152587890625	\\
0.766324854618902	0.0003662109375	\\
0.766369245793936	0.00115966796875	\\
0.766413636968971	0.001800537109375	\\
0.766458028144005	0.001861572265625	\\
0.766502419319039	0.00152587890625	\\
0.766546810494074	0.001495361328125	\\
0.766591201669108	0.00164794921875	\\
0.766635592844143	0.001983642578125	\\
0.766679984019177	0.002838134765625	\\
0.766724375194211	0.0028076171875	\\
0.766768766369246	0.00335693359375	\\
0.76681315754428	0.003570556640625	\\
0.766857548719315	0.002960205078125	\\
0.766901939894349	0.00384521484375	\\
0.766946331069383	0.004180908203125	\\
0.766990722244418	0.004119873046875	\\
0.767035113419452	0.00433349609375	\\
0.767079504594487	0.00421142578125	\\
0.767123895769521	0.004364013671875	\\
0.767168286944555	0.00457763671875	\\
0.76721267811959	0.004302978515625	\\
0.767257069294624	0.004608154296875	\\
0.767301460469659	0.004913330078125	\\
0.767345851644693	0.005035400390625	\\
0.767390242819727	0.00506591796875	\\
0.767434633994762	0.004058837890625	\\
0.767479025169796	0.00384521484375	\\
0.767523416344831	0.00360107421875	\\
0.767567807519865	0.003143310546875	\\
0.767612198694899	0.00341796875	\\
0.767656589869934	0.002838134765625	\\
0.767700981044968	0.00286865234375	\\
0.767745372220003	0.002960205078125	\\
0.767789763395037	0.00286865234375	\\
0.767834154570071	0.002777099609375	\\
0.767878545745106	0.002288818359375	\\
0.76792293692014	0.002166748046875	\\
0.767967328095175	0.002105712890625	\\
0.768011719270209	0.002044677734375	\\
0.768056110445243	0.002166748046875	\\
0.768100501620278	0.00201416015625	\\
0.768144892795312	0.001953125	\\
0.768189283970347	0.001800537109375	\\
0.768233675145381	0.001678466796875	\\
0.768278066320416	0.001495361328125	\\
0.76832245749545	0.001434326171875	\\
0.768366848670484	0.001617431640625	\\
0.768411239845519	0.001739501953125	\\
0.768455631020553	0.001739501953125	\\
0.768500022195588	0.002410888671875	\\
0.768544413370622	0.0028076171875	\\
0.768588804545656	0.002288818359375	\\
0.768633195720691	0.002471923828125	\\
0.768677586895725	0.0025634765625	\\
0.768721978070759	0.0025634765625	\\
0.768766369245794	0.003143310546875	\\
0.768810760420828	0.003143310546875	\\
0.768855151595863	0.0032958984375	\\
0.768899542770897	0.003753662109375	\\
0.768943933945932	0.002716064453125	\\
0.768988325120966	0.002532958984375	\\
0.769032716296	0.00274658203125	\\
0.769077107471035	0.002197265625	\\
0.769121498646069	0.002197265625	\\
0.769165889821104	0.002471923828125	\\
0.769210280996138	0.002655029296875	\\
0.769254672171172	0.002655029296875	\\
0.769299063346207	0.002777099609375	\\
0.769343454521241	0.002899169921875	\\
0.769387845696276	0.0030517578125	\\
0.76943223687131	0.003204345703125	\\
0.769476628046344	0.00335693359375	\\
0.769521019221379	0.002899169921875	\\
0.769565410396413	0.002227783203125	\\
0.769609801571448	0.001861572265625	\\
0.769654192746482	0.001800537109375	\\
0.769698583921516	0.001739501953125	\\
0.769742975096551	0.001739501953125	\\
0.769787366271585	0.001556396484375	\\
0.76983175744662	0.001373291015625	\\
0.769876148621654	0.001434326171875	\\
0.769920539796688	0.001129150390625	\\
0.769964930971723	0.001190185546875	\\
0.770009322146757	0.001251220703125	\\
0.770053713321792	0.00146484375	\\
0.770098104496826	0.001953125	\\
0.77014249567186	0.001739501953125	\\
0.770186886846895	0.00164794921875	\\
0.770231278021929	0.001617431640625	\\
0.770275669196964	0.0009765625	\\
0.770320060371998	0.00103759765625	\\
0.770364451547032	0.001434326171875	\\
0.770408842722067	0.00164794921875	\\
0.770453233897101	0.001678466796875	\\
0.770497625072136	0.00146484375	\\
0.77054201624717	0.001708984375	\\
0.770586407422204	0.00244140625	\\
0.770630798597239	0.002716064453125	\\
0.770675189772273	0.002777099609375	\\
0.770719580947308	0.003173828125	\\
0.770763972122342	0.0035400390625	\\
0.770808363297376	0.0035400390625	\\
0.770852754472411	0.003173828125	\\
0.770897145647445	0.00347900390625	\\
0.77094153682248	0.003082275390625	\\
0.770985927997514	0.00323486328125	\\
0.771030319172549	0.003662109375	\\
0.771074710347583	0.00341796875	\\
0.771119101522617	0.003631591796875	\\
0.771163492697652	0.003387451171875	\\
0.771207883872686	0.0030517578125	\\
0.771252275047721	0.003326416015625	\\
0.771296666222755	0.00335693359375	\\
0.771341057397789	0.002838134765625	\\
0.771385448572824	0.002838134765625	\\
0.771429839747858	0.002685546875	\\
0.771474230922893	0.00238037109375	\\
0.771518622097927	0.002288818359375	\\
0.771563013272961	0.0025634765625	\\
0.771607404447996	0.00274658203125	\\
0.77165179562303	0.0023193359375	\\
0.771696186798065	0.002044677734375	\\
0.771740577973099	0.001983642578125	\\
0.771784969148133	0.00140380859375	\\
0.771829360323168	0.0006103515625	\\
0.771873751498202	0.00030517578125	\\
0.771918142673237	0.00042724609375	\\
0.771962533848271	0.000701904296875	\\
0.772006925023305	0.00042724609375	\\
0.77205131619834	0.00042724609375	\\
0.772095707373374	0.000518798828125	\\
0.772140098548409	0.00091552734375	\\
0.772184489723443	0.00115966796875	\\
0.772228880898477	0.001068115234375	\\
0.772273272073512	0.001190185546875	\\
0.772317663248546	0.00152587890625	\\
0.772362054423581	0.0018310546875	\\
0.772406445598615	0.00146484375	\\
0.772450836773649	0.001312255859375	\\
0.772495227948684	0.001983642578125	\\
0.772539619123718	0.001953125	\\
0.772584010298753	0.002044677734375	\\
0.772628401473787	0.00225830078125	\\
0.772672792648821	0.00201416015625	\\
0.772717183823856	0.002288818359375	\\
0.77276157499889	0.00250244140625	\\
0.772805966173925	0.00299072265625	\\
0.772850357348959	0.00360107421875	\\
0.772894748523993	0.0037841796875	\\
0.772939139699028	0.0032958984375	\\
0.772983530874062	0.002655029296875	\\
0.773027922049097	0.0028076171875	\\
0.773072313224131	0.0028076171875	\\
0.773116704399165	0.002777099609375	\\
0.7731610955742	0.00262451171875	\\
0.773205486749234	0.00262451171875	\\
0.773249877924269	0.00274658203125	\\
0.773294269099303	0.002685546875	\\
0.773338660274338	0.002532958984375	\\
0.773383051449372	0.003021240234375	\\
0.773427442624406	0.00299072265625	\\
0.773471833799441	0.003448486328125	\\
0.773516224974475	0.003692626953125	\\
0.773560616149509	0.002777099609375	\\
0.773605007324544	0.002655029296875	\\
0.773649398499578	0.002777099609375	\\
0.773693789674613	0.002410888671875	\\
0.773738180849647	0.002105712890625	\\
0.773782572024681	0.0023193359375	\\
0.773826963199716	0.001800537109375	\\
0.77387135437475	0.001708984375	\\
0.773915745549785	0.0018310546875	\\
0.773960136724819	0.001739501953125	\\
0.774004527899854	0.0018310546875	\\
0.774048919074888	0.001556396484375	\\
0.774093310249922	0.0015869140625	\\
0.774137701424957	0.001373291015625	\\
0.774182092599991	0.001220703125	\\
0.774226483775026	0.001220703125	\\
0.77427087495006	0.001068115234375	\\
0.774315266125094	0.001068115234375	\\
0.774359657300129	0.00152587890625	\\
0.774404048475163	0.00164794921875	\\
0.774448439650198	0.00201416015625	\\
0.774492830825232	0.0018310546875	\\
0.774537222000266	0.00128173828125	\\
0.774581613175301	0.00152587890625	\\
0.774626004350335	0.001861572265625	\\
0.77467039552537	0.002044677734375	\\
0.774714786700404	0.002471923828125	\\
0.774759177875438	0.002899169921875	\\
0.774803569050473	0.002410888671875	\\
0.774847960225507	0.002838134765625	\\
0.774892351400542	0.0030517578125	\\
0.774936742575576	0.003082275390625	\\
0.77498113375061	0.00372314453125	\\
0.775025524925645	0.00408935546875	\\
0.775069916100679	0.004150390625	\\
0.775114307275714	0.004150390625	\\
0.775158698450748	0.003204345703125	\\
0.775203089625782	0.002471923828125	\\
0.775247480800817	0.00274658203125	\\
0.775291871975851	0.00323486328125	\\
0.775336263150886	0.003265380859375	\\
0.77538065432592	0.002777099609375	\\
0.775425045500954	0.002685546875	\\
0.775469436675989	0.00244140625	\\
0.775513827851023	0.00250244140625	\\
0.775558219026058	0.002532958984375	\\
0.775602610201092	0.002593994140625	\\
0.775647001376126	0.002899169921875	\\
0.775691392551161	0.002471923828125	\\
0.775735783726195	0.0015869140625	\\
0.77578017490123	0.00146484375	\\
0.775824566076264	0.0013427734375	\\
0.775868957251298	0.00128173828125	\\
0.775913348426333	0.00140380859375	\\
0.775957739601367	0.00146484375	\\
0.776002130776402	0.001312255859375	\\
0.776046521951436	0.000946044921875	\\
0.77609091312647	0.0008544921875	\\
0.776135304301505	0.00091552734375	\\
0.776179695476539	0.001129150390625	\\
0.776224086651574	0.001556396484375	\\
0.776268477826608	0.001678466796875	\\
0.776312869001642	0.001739501953125	\\
0.776357260176677	0.001678466796875	\\
0.776401651351711	0.00103759765625	\\
0.776446042526746	0.001190185546875	\\
0.77649043370178	0.001953125	\\
0.776534824876814	0.002777099609375	\\
0.776579216051849	0.002685546875	\\
0.776623607226883	0.00238037109375	\\
0.776667998401918	0.002471923828125	\\
0.776712389576952	0.002838134765625	\\
0.776756780751987	0.0032958984375	\\
0.776801171927021	0.0037841796875	\\
0.776845563102055	0.004302978515625	\\
0.77688995427709	0.004638671875	\\
0.776934345452124	0.004302978515625	\\
0.776978736627159	0.003997802734375	\\
0.777023127802193	0.004486083984375	\\
0.777067518977227	0.004547119140625	\\
0.777111910152262	0.004730224609375	\\
0.777156301327296	0.0048828125	\\
0.77720069250233	0.0045166015625	\\
0.777245083677365	0.00457763671875	\\
0.777289474852399	0.004608154296875	\\
0.777333866027434	0.004241943359375	\\
0.777378257202468	0.00390625	\\
0.777422648377503	0.004150390625	\\
0.777467039552537	0.004364013671875	\\
0.777511430727571	0.004608154296875	\\
0.777555821902606	0.00408935546875	\\
0.77760021307764	0.003570556640625	\\
0.777644604252675	0.0037841796875	\\
0.777688995427709	0.00384521484375	\\
0.777733386602743	0.0035400390625	\\
0.777777777777778	0.00323486328125	\\
0.777822168952812	0.002777099609375	\\
0.777866560127847	0.002471923828125	\\
0.777910951302881	0.002593994140625	\\
0.777955342477915	0.0025634765625	\\
0.77799973365295	0.002288818359375	\\
0.778044124827984	0.00274658203125	\\
0.778088516003019	0.002960205078125	\\
0.778132907178053	0.002685546875	\\
0.778177298353087	0.002044677734375	\\
0.778221689528122	0.001708984375	\\
0.778266080703156	0.001922607421875	\\
0.778310471878191	0.001708984375	\\
0.778354863053225	0.0018310546875	\\
0.778399254228259	0.00238037109375	\\
0.778443645403294	0.00250244140625	\\
0.778488036578328	0.001922607421875	\\
0.778532427753363	0.002471923828125	\\
0.778576818928397	0.0020751953125	\\
0.778621210103431	0.0015869140625	\\
0.778665601278466	0.002044677734375	\\
0.7787099924535	0.001739501953125	\\
0.778754383628535	0.001739501953125	\\
0.778798774803569	0.00225830078125	\\
0.778843165978603	0.0023193359375	\\
0.778887557153638	0.002471923828125	\\
0.778931948328672	0.002410888671875	\\
0.778976339503707	0.0023193359375	\\
0.779020730678741	0.0023193359375	\\
0.779065121853776	0.002227783203125	\\
0.77910951302881	0.002349853515625	\\
0.779153904203844	0.00146484375	\\
0.779198295378879	0.001190185546875	\\
0.779242686553913	0.00140380859375	\\
0.779287077728947	0.001312255859375	\\
0.779331468903982	0.00177001953125	\\
0.779375860079016	0.001190185546875	\\
0.779420251254051	0.000762939453125	\\
0.779464642429085	0.0008544921875	\\
0.779509033604119	0.001007080078125	\\
0.779553424779154	0.00079345703125	\\
0.779597815954188	0.000335693359375	\\
0.779642207129223	0.000213623046875	\\
0.779686598304257	0.000244140625	\\
0.779730989479292	0.000274658203125	\\
0.779775380654326	9.1552734375e-05	\\
0.77981977182936	0.00030517578125	\\
0.779864163004395	0.000274658203125	\\
0.779908554179429	0	\\
0.779952945354464	0.00030517578125	\\
0.779997336529498	0.000457763671875	\\
0.780041727704532	0.000396728515625	\\
0.780086118879567	0	\\
0.780130510054601	6.103515625e-05	\\
0.780174901229636	0.00042724609375	\\
0.78021929240467	0.00042724609375	\\
0.780263683579704	0.000946044921875	\\
0.780308074754739	0.001129150390625	\\
0.780352465929773	0.00128173828125	\\
0.780396857104808	0.001617431640625	\\
0.780441248279842	0.001708984375	\\
0.780485639454876	0.00201416015625	\\
0.780530030629911	0.00201416015625	\\
0.780574421804945	0.00189208984375	\\
0.78061881297998	0.00225830078125	\\
0.780663204155014	0.002655029296875	\\
0.780707595330048	0.002960205078125	\\
0.780751986505083	0.00274658203125	\\
0.780796377680117	0.002960205078125	\\
0.780840768855152	0.003631591796875	\\
0.780885160030186	0.003814697265625	\\
0.78092955120522	0.00384521484375	\\
0.780973942380255	0.0045166015625	\\
0.781018333555289	0.0045166015625	\\
0.781062724730324	0.004180908203125	\\
0.781107115905358	0.0045166015625	\\
0.781151507080392	0.004730224609375	\\
0.781195898255427	0.003997802734375	\\
0.781240289430461	0.004058837890625	\\
0.781284680605496	0.0040283203125	\\
0.78132907178053	0.003387451171875	\\
0.781373462955564	0.003448486328125	\\
0.781417854130599	0.00341796875	\\
0.781462245305633	0.00347900390625	\\
0.781506636480668	0.003387451171875	\\
0.781551027655702	0.003387451171875	\\
0.781595418830736	0.003570556640625	\\
0.781639810005771	0.00323486328125	\\
0.781684201180805	0.002777099609375	\\
0.78172859235584	0.002838134765625	\\
0.781772983530874	0.0029296875	\\
0.781817374705909	0.00286865234375	\\
0.781861765880943	0.003021240234375	\\
0.781906157055977	0.002655029296875	\\
0.781950548231012	0.0025634765625	\\
0.781994939406046	0.00201416015625	\\
0.78203933058108	0.00152587890625	\\
0.782083721756115	0.001800537109375	\\
0.782128112931149	0.001373291015625	\\
0.782172504106184	0.001434326171875	\\
0.782216895281218	0.001068115234375	\\
0.782261286456252	0.000640869140625	\\
0.782305677631287	0.000762939453125	\\
0.782350068806321	0.000640869140625	\\
0.782394459981356	0.000457763671875	\\
0.78243885115639	0.000579833984375	\\
0.782483242331425	0.000885009765625	\\
0.782527633506459	0.00048828125	\\
0.782572024681493	9.1552734375e-05	\\
0.782616415856528	0	\\
0.782660807031562	0.000152587890625	\\
0.782705198206597	-0.000244140625	\\
0.782749589381631	-0.000152587890625	\\
0.782793980556665	0.000274658203125	\\
0.7828383717317	0.000244140625	\\
0.782882762906734	3.0517578125e-05	\\
0.782927154081768	0.000274658203125	\\
0.782971545256803	0.000640869140625	\\
0.783015936431837	0.00103759765625	\\
0.783060327606872	0.001007080078125	\\
0.783104718781906	0.0008544921875	\\
0.783149109956941	0.00079345703125	\\
0.783193501131975	0.000579833984375	\\
0.783237892307009	-6.103515625e-05	\\
0.783282283482044	-0.0003662109375	\\
0.783326674657078	-9.1552734375e-05	\\
0.783371065832113	0.00030517578125	\\
0.783415457007147	-3.0517578125e-05	\\
0.783459848182181	-6.103515625e-05	\\
0.783504239357216	0.000732421875	\\
0.78354863053225	0.00115966796875	\\
0.783593021707285	0.00091552734375	\\
0.783637412882319	0.001068115234375	\\
0.783681804057353	0.00115966796875	\\
0.783726195232388	0.000701904296875	\\
0.783770586407422	0.000335693359375	\\
0.783814977582457	0.000335693359375	\\
0.783859368757491	0.000152587890625	\\
0.783903759932525	-0.00042724609375	\\
0.78394815110756	-6.103515625e-05	\\
0.783992542282594	0.000518798828125	\\
0.784036933457629	0.0001220703125	\\
0.784081324632663	0.00042724609375	\\
0.784125715807697	0.00067138671875	\\
0.784170106982732	0.0003662109375	\\
0.784214498157766	0.001007080078125	\\
0.784258889332801	0.0010986328125	\\
0.784303280507835	0.00054931640625	\\
0.784347671682869	0.00115966796875	\\
0.784392062857904	0.000885009765625	\\
0.784436454032938	0.00103759765625	\\
0.784480845207973	0.001617431640625	\\
0.784525236383007	0.0015869140625	\\
0.784569627558041	0.00213623046875	\\
0.784614018733076	0.0023193359375	\\
0.78465840990811	0.001861572265625	\\
0.784702801083145	0.002044677734375	\\
0.784747192258179	0.0023193359375	\\
0.784791583433213	0.002197265625	\\
0.784835974608248	0.002716064453125	\\
0.784880365783282	0.002777099609375	\\
0.784924756958317	0.002960205078125	\\
0.784969148133351	0.002593994140625	\\
0.785013539308385	0.002838134765625	\\
0.78505793048342	0.00299072265625	\\
0.785102321658454	0.002685546875	\\
0.785146712833489	0.003082275390625	\\
0.785191104008523	0.002838134765625	\\
0.785235495183558	0.002838134765625	\\
0.785279886358592	0.003021240234375	\\
0.785324277533626	0.00299072265625	\\
0.785368668708661	0.00311279296875	\\
0.785413059883695	0.002960205078125	\\
0.78545745105873	0.00274658203125	\\
0.785501842233764	0.003021240234375	\\
0.785546233408798	0.003204345703125	\\
0.785590624583833	0.002685546875	\\
0.785635015758867	0.002349853515625	\\
0.785679406933902	0.00225830078125	\\
0.785723798108936	0.0023193359375	\\
0.78576818928397	0.002655029296875	\\
0.785812580459005	0.00250244140625	\\
0.785856971634039	0.002044677734375	\\
0.785901362809074	0.001434326171875	\\
0.785945753984108	0.001312255859375	\\
0.785990145159142	0.001129150390625	\\
0.786034536334177	0.0006103515625	\\
0.786078927509211	0.000579833984375	\\
0.786123318684246	0.00048828125	\\
0.78616770985928	0.0003662109375	\\
0.786212101034314	0.000396728515625	\\
0.786256492209349	0.000152587890625	\\
0.786300883384383	-0.0001220703125	\\
0.786345274559418	0.00018310546875	\\
0.786389665734452	0.000579833984375	\\
0.786434056909486	0.00030517578125	\\
0.786478448084521	-0.000335693359375	\\
0.786522839259555	-0.000640869140625	\\
0.78656723043459	-6.103515625e-05	\\
0.786611621609624	0.00067138671875	\\
0.786656012784658	0.000396728515625	\\
0.786700403959693	0.000732421875	\\
0.786744795134727	0.0008544921875	\\
0.786789186309762	0.00048828125	\\
0.786833577484796	0.000946044921875	\\
0.78687796865983	0.000579833984375	\\
0.786922359834865	0.00018310546875	\\
0.786966751009899	0.000518798828125	\\
0.787011142184934	0.000579833984375	\\
0.787055533359968	0.00079345703125	\\
0.787099924535002	0.00030517578125	\\
0.787144315710037	0.000152587890625	\\
0.787188706885071	-0.0001220703125	\\
0.787233098060106	0	\\
0.78727748923514	0.000701904296875	\\
0.787321880410174	-9.1552734375e-05	\\
0.787366271585209	-0.000335693359375	\\
0.787410662760243	6.103515625e-05	\\
0.787455053935278	-0.000152587890625	\\
0.787499445110312	9.1552734375e-05	\\
0.787543836285347	-6.103515625e-05	\\
0.787588227460381	-0.000396728515625	\\
0.787632618635415	-0.000152587890625	\\
0.78767700981045	-0.000244140625	\\
0.787721400985484	-0.000213623046875	\\
0.787765792160518	-0.000457763671875	\\
0.787810183335553	-0.00018310546875	\\
0.787854574510587	6.103515625e-05	\\
0.787898965685622	-0.000274658203125	\\
0.787943356860656	-9.1552734375e-05	\\
0.78798774803569	0.000244140625	\\
0.788032139210725	-9.1552734375e-05	\\
0.788076530385759	0.00018310546875	\\
0.788120921560794	0.000335693359375	\\
0.788165312735828	0.0006103515625	\\
0.788209703910863	0.000701904296875	\\
0.788254095085897	-6.103515625e-05	\\
0.788298486260931	0.00067138671875	\\
0.788342877435966	0.00103759765625	\\
0.788387268611	0.000946044921875	\\
0.788431659786035	0.00152587890625	\\
0.788476050961069	0.001800537109375	\\
0.788520442136103	0.00164794921875	\\
0.788564833311138	0.001983642578125	\\
0.788609224486172	0.002655029296875	\\
0.788653615661207	0.002410888671875	\\
0.788698006836241	0.00213623046875	\\
0.788742398011275	0.002197265625	\\
0.78878678918631	0.002044677734375	\\
0.788831180361344	0.002349853515625	\\
0.788875571536379	0.00238037109375	\\
0.788919962711413	0.002716064453125	\\
0.788964353886447	0.0023193359375	\\
0.789008745061482	0.0025634765625	\\
0.789053136236516	0.00341796875	\\
0.789097527411551	0.00286865234375	\\
0.789141918586585	0.0025634765625	\\
0.789186309761619	0.0028076171875	\\
0.789230700936654	0.002593994140625	\\
0.789275092111688	0.00244140625	\\
0.789319483286723	0.002349853515625	\\
0.789363874461757	0.001739501953125	\\
0.789408265636791	0.001434326171875	\\
0.789452656811826	0.0013427734375	\\
0.78949704798686	0.001434326171875	\\
0.789541439161895	0.0010986328125	\\
0.789585830336929	0.00079345703125	\\
0.789630221511963	0.00091552734375	\\
0.789674612686998	0.0006103515625	\\
0.789719003862032	0.0003662109375	\\
0.789763395037067	-0.00030517578125	\\
0.789807786212101	-0.00079345703125	\\
0.789852177387135	-0.00091552734375	\\
0.78989656856217	-0.00103759765625	\\
0.789940959737204	-0.000732421875	\\
0.789985350912239	-0.000518798828125	\\
0.790029742087273	-0.000640869140625	\\
0.790074133262307	-0.001068115234375	\\
0.790118524437342	-0.001739501953125	\\
0.790162915612376	-0.001434326171875	\\
0.790207306787411	-0.001220703125	\\
0.790251697962445	-0.00140380859375	\\
0.79029608913748	-0.00146484375	\\
0.790340480312514	-0.001922607421875	\\
0.790384871487548	-0.001190185546875	\\
0.790429262662583	-0.00054931640625	\\
0.790473653837617	-0.001007080078125	\\
0.790518045012651	-0.000701904296875	\\
0.790562436187686	-0.00079345703125	\\
0.79060682736272	-0.001190185546875	\\
0.790651218537755	-0.000732421875	\\
0.790695609712789	-0.000946044921875	\\
0.790740000887823	-0.00128173828125	\\
0.790784392062858	-0.00103759765625	\\
0.790828783237892	-0.000640869140625	\\
0.790873174412927	-0.000213623046875	\\
0.790917565587961	3.0517578125e-05	\\
0.790961956762996	-0.000244140625	\\
0.79100634793803	-0.000274658203125	\\
0.791050739113064	0.000152587890625	\\
0.791095130288099	6.103515625e-05	\\
0.791139521463133	-0.0001220703125	\\
0.791183912638168	0	\\
0.791228303813202	0.0001220703125	\\
0.791272694988236	-0.000152587890625	\\
0.791317086163271	-0.00030517578125	\\
0.791361477338305	-0.000244140625	\\
0.791405868513339	-0.00030517578125	\\
0.791450259688374	0.000335693359375	\\
0.791494650863408	0.000823974609375	\\
0.791539042038443	0.000732421875	\\
0.791583433213477	0.00067138671875	\\
0.791627824388512	6.103515625e-05	\\
0.791672215563546	-6.103515625e-05	\\
0.79171660673858	6.103515625e-05	\\
0.791760997913615	0.000213623046875	\\
0.791805389088649	3.0517578125e-05	\\
0.791849780263684	0.00030517578125	\\
0.791894171438718	0.000946044921875	\\
0.791938562613752	0.0008544921875	\\
0.791982953788787	0.001220703125	\\
0.792027344963821	0.00103759765625	\\
0.792071736138856	0.001312255859375	\\
0.79211612731389	0.000823974609375	\\
0.792160518488924	0.00030517578125	\\
0.792204909663959	0.000396728515625	\\
0.792249300838993	0	\\
0.792293692014028	0.000823974609375	\\
0.792338083189062	0.00103759765625	\\
0.792382474364096	0.00079345703125	\\
0.792426865539131	0.000579833984375	\\
0.792471256714165	0.00048828125	\\
0.7925156478892	0.00054931640625	\\
0.792560039064234	0.00042724609375	\\
0.792604430239268	0.0008544921875	\\
0.792648821414303	0.001007080078125	\\
0.792693212589337	0.000762939453125	\\
0.792737603764372	0.000335693359375	\\
0.792781994939406	-0.000244140625	\\
0.79282638611444	-0.00018310546875	\\
0.792870777289475	3.0517578125e-05	\\
0.792915168464509	0.00042724609375	\\
0.792959559639544	0.00042724609375	\\
0.793003950814578	0.000396728515625	\\
0.793048341989612	0.000396728515625	\\
0.793092733164647	0.000579833984375	\\
0.793137124339681	0.00067138671875	\\
0.793181515514716	0.00067138671875	\\
0.79322590668975	0.00128173828125	\\
0.793270297864785	0.001251220703125	\\
0.793314689039819	0.001312255859375	\\
0.793359080214853	0.0015869140625	\\
0.793403471389888	0.000701904296875	\\
0.793447862564922	0.000518798828125	\\
0.793492253739956	0.00048828125	\\
0.793536644914991	0.001190185546875	\\
0.793581036090025	0.001129150390625	\\
0.79362542726506	0.000885009765625	\\
0.793669818440094	0.001251220703125	\\
0.793714209615129	0.00103759765625	\\
0.793758600790163	0.0006103515625	\\
0.793802991965197	0.001373291015625	\\
0.793847383140232	0.001922607421875	\\
0.793891774315266	0.000732421875	\\
0.793936165490301	0.0003662109375	\\
0.793980556665335	0.000762939453125	\\
0.794024947840369	0.000579833984375	\\
0.794069339015404	0.00079345703125	\\
0.794113730190438	0.00067138671875	\\
0.794158121365473	0.000701904296875	\\
0.794202512540507	0.0003662109375	\\
0.794246903715541	0.000152587890625	\\
0.794291294890576	6.103515625e-05	\\
0.79433568606561	3.0517578125e-05	\\
0.794380077240645	0.000335693359375	\\
0.794424468415679	0.000213623046875	\\
0.794468859590713	0.00048828125	\\
0.794513250765748	0.00030517578125	\\
0.794557641940782	0.000396728515625	\\
0.794602033115817	0.00030517578125	\\
0.794646424290851	0.000457763671875	\\
0.794690815465885	0.0006103515625	\\
0.79473520664092	-6.103515625e-05	\\
0.794779597815954	3.0517578125e-05	\\
0.794823988990989	-9.1552734375e-05	\\
0.794868380166023	-0.000823974609375	\\
0.794912771341057	-0.000244140625	\\
0.794957162516092	0.000152587890625	\\
0.795001553691126	-0.000335693359375	\\
0.795045944866161	-0.000885009765625	\\
0.795090336041195	-0.000640869140625	\\
0.795134727216229	-0.0008544921875	\\
0.795179118391264	-0.000946044921875	\\
0.795223509566298	-0.000396728515625	\\
0.795267900741333	0.000244140625	\\
0.795312291916367	0.000213623046875	\\
0.795356683091401	-0.000152587890625	\\
0.795401074266436	6.103515625e-05	\\
0.79544546544147	0.00018310546875	\\
0.795489856616505	-0.00018310546875	\\
0.795534247791539	-0.000518798828125	\\
0.795578638966573	-3.0517578125e-05	\\
0.795623030141608	-0.0001220703125	\\
0.795667421316642	-0.000732421875	\\
0.795711812491677	-0.00054931640625	\\
0.795756203666711	-0.000701904296875	\\
0.795800594841745	-0.00067138671875	\\
0.79584498601678	-0.00067138671875	\\
0.795889377191814	-0.000152587890625	\\
0.795933768366849	3.0517578125e-05	\\
0.795978159541883	3.0517578125e-05	\\
0.796022550716918	0.00018310546875	\\
0.796066941891952	-0.0003662109375	\\
0.796111333066986	-0.000244140625	\\
0.796155724242021	3.0517578125e-05	\\
0.796200115417055	0.000335693359375	\\
0.796244506592089	0.0003662109375	\\
0.796288897767124	-0.000335693359375	\\
0.796333288942158	-0.000457763671875	\\
0.796377680117193	-0.000518798828125	\\
0.796422071292227	-0.000701904296875	\\
0.796466462467261	-0.00018310546875	\\
0.796510853642296	0	\\
0.79655524481733	0.00018310546875	\\
0.796599635992365	0.000244140625	\\
0.796644027167399	-3.0517578125e-05	\\
0.796688418342434	-0.00018310546875	\\
0.796732809517468	-0.000244140625	\\
0.796777200692502	-0.0001220703125	\\
0.796821591867537	0.000213623046875	\\
0.796865983042571	0	\\
0.796910374217606	-0.000244140625	\\
0.79695476539264	-0.000213623046875	\\
0.796999156567674	-0.00042724609375	\\
0.797043547742709	-0.00067138671875	\\
0.797087938917743	-0.000518798828125	\\
0.797132330092778	-0.00054931640625	\\
0.797176721267812	-0.001220703125	\\
0.797221112442846	-0.0010986328125	\\
0.797265503617881	-0.0013427734375	\\
0.797309894792915	-0.001556396484375	\\
0.79735428596795	-0.00152587890625	\\
0.797398677142984	-0.001190185546875	\\
0.797443068318018	-0.0008544921875	\\
0.797487459493053	-0.001007080078125	\\
0.797531850668087	-0.000946044921875	\\
0.797576241843122	-0.001373291015625	\\
0.797620633018156	-0.00128173828125	\\
0.79766502419319	-0.00103759765625	\\
0.797709415368225	-0.000946044921875	\\
0.797753806543259	-0.001129150390625	\\
0.797798197718294	-0.000701904296875	\\
0.797842588893328	-0.00067138671875	\\
0.797886980068362	-0.0010986328125	\\
0.797931371243397	-0.00103759765625	\\
0.797975762418431	-0.001190185546875	\\
0.798020153593466	-0.00067138671875	\\
0.7980645447685	-0.000762939453125	\\
0.798108935943534	-0.000946044921875	\\
0.798153327118569	-0.001068115234375	\\
0.798197718293603	-0.001190185546875	\\
0.798242109468638	-0.001251220703125	\\
0.798286500643672	-0.001220703125	\\
0.798330891818706	-0.000396728515625	\\
0.798375282993741	-0.000244140625	\\
0.798419674168775	-0.000457763671875	\\
0.79846406534381	-0.000457763671875	\\
0.798508456518844	-0.000579833984375	\\
0.798552847693878	-0.0003662109375	\\
0.798597238868913	-0.000213623046875	\\
0.798641630043947	-0.00042724609375	\\
0.798686021218982	-0.000701904296875	\\
0.798730412394016	-0.0008544921875	\\
0.798774803569051	-0.000457763671875	\\
0.798819194744085	-0.000457763671875	\\
0.798863585919119	-0.000457763671875	\\
0.798907977094154	-0.000244140625	\\
0.798952368269188	-0.000335693359375	\\
0.798996759444222	-0.000885009765625	\\
0.799041150619257	-0.001220703125	\\
0.799085541794291	-0.00115966796875	\\
0.799129932969326	-0.00103759765625	\\
0.79917432414436	-0.000152587890625	\\
0.799218715319394	-3.0517578125e-05	\\
0.799263106494429	-0.000732421875	\\
0.799307497669463	-0.000579833984375	\\
0.799351888844498	-0.000762939453125	\\
0.799396280019532	-0.00115966796875	\\
0.799440671194567	-0.0008544921875	\\
0.799485062369601	-0.00091552734375	\\
0.799529453544635	-0.001190185546875	\\
0.79957384471967	-0.00152587890625	\\
0.799618235894704	-0.0013427734375	\\
0.799662627069739	-0.00103759765625	\\
0.799707018244773	-0.000762939453125	\\
0.799751409419807	-0.000457763671875	\\
0.799795800594842	-3.0517578125e-05	\\
0.799840191769876	0.000274658203125	\\
0.799884582944911	-0.000274658203125	\\
0.799928974119945	-0.000213623046875	\\
0.799973365294979	-0.000274658203125	\\
0.800017756470014	-0.0003662109375	\\
0.800062147645048	-0.000244140625	\\
0.800106538820083	0.000213623046875	\\
0.800150929995117	0.00048828125	\\
0.800195321170151	0.00054931640625	\\
0.800239712345186	0.00048828125	\\
0.80028410352022	0.000579833984375	\\
0.800328494695255	0.000823974609375	\\
0.800372885870289	0.001251220703125	\\
0.800417277045323	0.001068115234375	\\
0.800461668220358	0.00079345703125	\\
0.800506059395392	0.000518798828125	\\
0.800550450570427	0.000640869140625	\\
0.800594841745461	0.00048828125	\\
0.800639232920495	0.000518798828125	\\
0.80068362409553	0.00103759765625	\\
0.800728015270564	0.000823974609375	\\
0.800772406445599	0.000946044921875	\\
0.800816797620633	0.001190185546875	\\
0.800861188795667	0.000885009765625	\\
0.800905579970702	0.000213623046875	\\
0.800949971145736	0.00018310546875	\\
0.800994362320771	0.00054931640625	\\
0.801038753495805	0.0009765625	\\
0.801083144670839	0.000579833984375	\\
0.801127535845874	0.000457763671875	\\
0.801171927020908	0.000762939453125	\\
0.801216318195943	0.000396728515625	\\
0.801260709370977	0.0009765625	\\
0.801305100546011	0.001312255859375	\\
0.801349491721046	0.00079345703125	\\
0.80139388289608	0.000396728515625	\\
0.801438274071115	-9.1552734375e-05	\\
0.801482665246149	-0.000152587890625	\\
0.801527056421183	9.1552734375e-05	\\
0.801571447596218	0.00042724609375	\\
0.801615838771252	0.00079345703125	\\
0.801660229946287	0.0008544921875	\\
0.801704621121321	0.000762939453125	\\
0.801749012296356	0.000640869140625	\\
0.80179340347139	0.000518798828125	\\
0.801837794646424	0.00079345703125	\\
0.801882185821459	0.0008544921875	\\
0.801926576996493	0.000885009765625	\\
0.801970968171527	0.000457763671875	\\
0.802015359346562	0.000762939453125	\\
0.802059750521596	0.00054931640625	\\
0.802104141696631	-0.00018310546875	\\
0.802148532871665	0.000213623046875	\\
0.8021929240467	0.0006103515625	\\
0.802237315221734	0.001007080078125	\\
0.802281706396768	0.000946044921875	\\
0.802326097571803	0.000335693359375	\\
0.802370488746837	0.000274658203125	\\
0.802414879921872	0.000885009765625	\\
0.802459271096906	0.001007080078125	\\
0.80250366227194	0.0009765625	\\
0.802548053446975	0.00067138671875	\\
0.802592444622009	0.0003662109375	\\
0.802636835797044	0.00018310546875	\\
0.802681226972078	0.0001220703125	\\
0.802725618147112	0.000152587890625	\\
0.802770009322147	0.000274658203125	\\
0.802814400497181	0.000885009765625	\\
0.802858791672216	0.0010986328125	\\
0.80290318284725	0.001007080078125	\\
0.802947574022284	0.001251220703125	\\
0.802991965197319	0.00115966796875	\\
0.803036356372353	0.0010986328125	\\
0.803080747547388	0.001129150390625	\\
0.803125138722422	0.001007080078125	\\
0.803169529897456	0.00048828125	\\
0.803213921072491	0.00030517578125	\\
0.803258312247525	-0.000213623046875	\\
0.80330270342256	-0.0003662109375	\\
0.803347094597594	-0.0003662109375	\\
0.803391485772628	-0.000396728515625	\\
0.803435876947663	-0.000244140625	\\
0.803480268122697	9.1552734375e-05	\\
0.803524659297732	-0.0001220703125	\\
0.803569050472766	-0.00067138671875	\\
0.8036134416478	-0.000335693359375	\\
0.803657832822835	0.00042724609375	\\
0.803702223997869	9.1552734375e-05	\\
0.803746615172904	-0.000579833984375	\\
0.803791006347938	-0.0008544921875	\\
0.803835397522972	-0.00103759765625	\\
0.803879788698007	-0.001190185546875	\\
0.803924179873041	-0.00079345703125	\\
0.803968571048076	-0.001007080078125	\\
0.80401296222311	-0.0008544921875	\\
0.804057353398144	-0.000732421875	\\
0.804101744573179	-0.001129150390625	\\
0.804146135748213	-0.001373291015625	\\
0.804190526923248	-0.001678466796875	\\
0.804234918098282	-0.00115966796875	\\
0.804279309273316	-0.001556396484375	\\
0.804323700448351	-0.001739501953125	\\
0.804368091623385	-0.00164794921875	\\
0.80441248279842	-0.002197265625	\\
0.804456873973454	-0.002044677734375	\\
0.804501265148489	-0.0018310546875	\\
0.804545656323523	-0.001861572265625	\\
0.804590047498557	-0.001739501953125	\\
0.804634438673592	-0.0020751953125	\\
0.804678829848626	-0.001708984375	\\
0.80472322102366	-0.00128173828125	\\
0.804767612198695	-0.00164794921875	\\
0.804812003373729	-0.001495361328125	\\
0.804856394548764	-0.000823974609375	\\
0.804900785723798	-0.0009765625	\\
0.804945176898832	-0.00048828125	\\
0.804989568073867	-0.00030517578125	\\
0.805033959248901	-0.000885009765625	\\
0.805078350423936	-0.00091552734375	\\
0.80512274159897	-0.000701904296875	\\
0.805167132774005	-0.000244140625	\\
0.805211523949039	3.0517578125e-05	\\
0.805255915124073	-3.0517578125e-05	\\
0.805300306299108	-3.0517578125e-05	\\
0.805344697474142	-0.00030517578125	\\
0.805389088649177	-0.000457763671875	\\
0.805433479824211	3.0517578125e-05	\\
0.805477870999245	0.000213623046875	\\
0.80552226217428	-3.0517578125e-05	\\
0.805566653349314	0.000244140625	\\
0.805611044524349	-0.0001220703125	\\
0.805655435699383	-0.00030517578125	\\
0.805699826874417	-6.103515625e-05	\\
0.805744218049452	-0.000274658203125	\\
0.805788609224486	-0.00030517578125	\\
0.805833000399521	3.0517578125e-05	\\
0.805877391574555	-6.103515625e-05	\\
0.805921782749589	-0.000274658203125	\\
0.805966173924624	-0.00030517578125	\\
0.806010565099658	-0.000396728515625	\\
0.806054956274693	-0.000244140625	\\
0.806099347449727	6.103515625e-05	\\
0.806143738624761	3.0517578125e-05	\\
0.806188129799796	-0.00018310546875	\\
0.80623252097483	-0.000213623046875	\\
0.806276912149865	-0.000396728515625	\\
0.806321303324899	-0.0003662109375	\\
0.806365694499933	-0.000457763671875	\\
0.806410085674968	-0.00054931640625	\\
0.806454476850002	-0.00054931640625	\\
0.806498868025037	-0.000579833984375	\\
0.806543259200071	-0.000518798828125	\\
0.806587650375105	-0.000701904296875	\\
0.80663204155014	-0.00079345703125	\\
0.806676432725174	-0.0006103515625	\\
0.806720823900209	-0.000457763671875	\\
0.806765215075243	-0.000579833984375	\\
0.806809606250277	-0.00079345703125	\\
0.806853997425312	-0.000885009765625	\\
0.806898388600346	-0.0009765625	\\
0.806942779775381	-0.000823974609375	\\
0.806987170950415	-0.001007080078125	\\
0.807031562125449	-0.000885009765625	\\
0.807075953300484	-0.000762939453125	\\
0.807120344475518	-0.000732421875	\\
0.807164735650553	-0.00091552734375	\\
0.807209126825587	-0.001007080078125	\\
0.807253518000622	-0.00067138671875	\\
0.807297909175656	-0.000152587890625	\\
0.80734230035069	-0.00042724609375	\\
0.807386691525725	-0.00054931640625	\\
0.807431082700759	0.0001220703125	\\
0.807475473875793	-0.00018310546875	\\
0.807519865050828	-0.000152587890625	\\
0.807564256225862	0.000213623046875	\\
0.807608647400897	-6.103515625e-05	\\
0.807653038575931	-0.000274658203125	\\
0.807697429750965	0.000152587890625	\\
0.807741820926	-3.0517578125e-05	\\
0.807786212101034	-0.00054931640625	\\
0.807830603276069	-0.000213623046875	\\
0.807874994451103	-6.103515625e-05	\\
0.807919385626138	-0.000244140625	\\
0.807963776801172	-0.0001220703125	\\
0.808008167976206	-0.000396728515625	\\
0.808052559151241	-0.00048828125	\\
0.808096950326275	-0.000274658203125	\\
0.80814134150131	3.0517578125e-05	\\
0.808185732676344	-3.0517578125e-05	\\
0.808230123851378	0.000396728515625	\\
0.808274515026413	6.103515625e-05	\\
0.808318906201447	-0.000885009765625	\\
0.808363297376482	-0.000579833984375	\\
0.808407688551516	-0.0008544921875	\\
0.80845207972655	-0.001129150390625	\\
0.808496470901585	-0.00115966796875	\\
0.808540862076619	-0.001495361328125	\\
0.808585253251654	-0.00128173828125	\\
0.808629644426688	-0.00103759765625	\\
0.808674035601722	-0.001312255859375	\\
0.808718426776757	-0.001190185546875	\\
0.808762817951791	-0.000823974609375	\\
0.808807209126826	-0.00146484375	\\
0.80885160030186	-0.0020751953125	\\
0.808895991476894	-0.0028076171875	\\
0.808940382651929	-0.003326416015625	\\
0.808984773826963	-0.002593994140625	\\
0.809029165001998	-0.002716064453125	\\
0.809073556177032	-0.003173828125	\\
0.809117947352066	-0.003204345703125	\\
0.809162338527101	-0.002777099609375	\\
0.809206729702135	-0.00341796875	\\
0.80925112087717	-0.00390625	\\
0.809295512052204	-0.003509521484375	\\
0.809339903227238	-0.003448486328125	\\
0.809384294402273	-0.0032958984375	\\
0.809428685577307	-0.00341796875	\\
0.809473076752342	-0.0030517578125	\\
0.809517467927376	-0.002716064453125	\\
0.80956185910241	-0.002655029296875	\\
0.809606250277445	-0.002227783203125	\\
0.809650641452479	-0.001953125	\\
0.809695032627514	-0.001739501953125	\\
0.809739423802548	-0.001312255859375	\\
0.809783814977582	-0.0013427734375	\\
0.809828206152617	-0.000732421875	\\
0.809872597327651	-0.000335693359375	\\
0.809916988502686	-0.0003662109375	\\
0.80996137967772	-3.0517578125e-05	\\
0.810005770852754	9.1552734375e-05	\\
0.810050162027789	0.0001220703125	\\
0.810094553202823	0.00030517578125	\\
0.810138944377858	0.0006103515625	\\
0.810183335552892	0.001068115234375	\\
0.810227726727927	0.00103759765625	\\
0.810272117902961	0.00103759765625	\\
0.810316509077995	0.000732421875	\\
0.81036090025303	0.00079345703125	\\
0.810405291428064	0.001007080078125	\\
0.810449682603098	0.00128173828125	\\
0.810494073778133	0.001739501953125	\\
0.810538464953167	0.001953125	\\
0.810582856128202	0.00177001953125	\\
0.810627247303236	0.001373291015625	\\
0.810671638478271	0.0013427734375	\\
0.810716029653305	0.000946044921875	\\
0.810760420828339	0.0006103515625	\\
0.810804812003374	0.00091552734375	\\
0.810849203178408	0.000701904296875	\\
0.810893594353443	0.000335693359375	\\
0.810937985528477	0.000213623046875	\\
0.810982376703511	-0.000244140625	\\
0.811026767878546	-0.000213623046875	\\
0.81107115905358	-0.0003662109375	\\
0.811115550228615	-0.00048828125	\\
0.811159941403649	-0.000762939453125	\\
0.811204332578683	-0.000823974609375	\\
0.811248723753718	-0.001495361328125	\\
0.811293114928752	-0.00152587890625	\\
0.811337506103787	-0.00115966796875	\\
0.811381897278821	-0.00164794921875	\\
0.811426288453855	-0.00164794921875	\\
0.81147067962889	-0.0018310546875	\\
0.811515070803924	-0.002288818359375	\\
0.811559461978959	-0.0029296875	\\
0.811603853153993	-0.00286865234375	\\
0.811648244329027	-0.0020751953125	\\
0.811692635504062	-0.002410888671875	\\
0.811737026679096	-0.00274658203125	\\
0.811781417854131	-0.00244140625	\\
0.811825809029165	-0.002685546875	\\
0.811870200204199	-0.00262451171875	\\
0.811914591379234	-0.00213623046875	\\
0.811958982554268	-0.002105712890625	\\
0.812003373729303	-0.0025634765625	\\
0.812047764904337	-0.003021240234375	\\
0.812092156079371	-0.002838134765625	\\
0.812136547254406	-0.002777099609375	\\
0.81218093842944	-0.002655029296875	\\
0.812225329604475	-0.00238037109375	\\
0.812269720779509	-0.002471923828125	\\
0.812314111954543	-0.002655029296875	\\
0.812358503129578	-0.002685546875	\\
0.812402894304612	-0.0030517578125	\\
0.812447285479647	-0.0029296875	\\
0.812491676654681	-0.00274658203125	\\
0.812536067829715	-0.00250244140625	\\
0.81258045900475	-0.00244140625	\\
0.812624850179784	-0.002105712890625	\\
0.812669241354819	-0.001800537109375	\\
0.812713632529853	-0.00244140625	\\
0.812758023704887	-0.0025634765625	\\
0.812802414879922	-0.002105712890625	\\
0.812846806054956	-0.001983642578125	\\
0.812891197229991	-0.001617431640625	\\
0.812935588405025	-0.001617431640625	\\
0.81297997958006	-0.001708984375	\\
0.813024370755094	-0.001953125	\\
0.813068761930128	-0.00225830078125	\\
0.813113153105163	-0.001983642578125	\\
0.813157544280197	-0.00244140625	\\
0.813201935455231	-0.00189208984375	\\
0.813246326630266	-0.00140380859375	\\
0.8132907178053	-0.0015869140625	\\
0.813335108980335	-0.001708984375	\\
0.813379500155369	-0.00152587890625	\\
0.813423891330403	-0.001190185546875	\\
0.813468282505438	-0.001800537109375	\\
0.813512673680472	-0.0020751953125	\\
0.813557064855507	-0.002197265625	\\
0.813601456030541	-0.002593994140625	\\
0.813645847205576	-0.00262451171875	\\
0.81369023838061	-0.00225830078125	\\
0.813734629555644	-0.00189208984375	\\
0.813779020730679	-0.001373291015625	\\
0.813823411905713	-0.001678466796875	\\
0.813867803080748	-0.0015869140625	\\
0.813912194255782	-0.00128173828125	\\
0.813956585430816	-0.00177001953125	\\
0.814000976605851	-0.0013427734375	\\
0.814045367780885	-0.001434326171875	\\
0.81408975895592	-0.001953125	\\
0.814134150130954	-0.001129150390625	\\
0.814178541305988	-0.001129150390625	\\
0.814222932481023	-0.000885009765625	\\
0.814267323656057	-0.00079345703125	\\
0.814311714831092	-0.00128173828125	\\
0.814356106006126	-0.00115966796875	\\
0.81440049718116	-0.000640869140625	\\
0.814444888356195	-0.00048828125	\\
0.814489279531229	-0.0003662109375	\\
0.814533670706264	-0.000579833984375	\\
0.814578061881298	-0.00030517578125	\\
0.814622453056332	-0.00054931640625	\\
0.814666844231367	-0.000457763671875	\\
0.814711235406401	-0.00042724609375	\\
0.814755626581436	-0.000701904296875	\\
0.81480001775647	-0.000946044921875	\\
0.814844408931504	-0.000579833984375	\\
0.814888800106539	-0.00067138671875	\\
0.814933191281573	-0.0008544921875	\\
0.814977582456608	-0.00079345703125	\\
0.815021973631642	-0.00048828125	\\
0.815066364806676	-0.000518798828125	\\
0.815110755981711	-0.000640869140625	\\
0.815155147156745	-0.000457763671875	\\
0.81519953833178	-0.000457763671875	\\
0.815243929506814	-0.000457763671875	\\
0.815288320681848	-0.000701904296875	\\
0.815332711856883	-0.000823974609375	\\
0.815377103031917	-0.000701904296875	\\
0.815421494206952	-0.001068115234375	\\
0.815465885381986	-0.00164794921875	\\
0.81551027655702	-0.00146484375	\\
0.815554667732055	-0.001190185546875	\\
0.815599058907089	-0.001129150390625	\\
0.815643450082124	-0.00079345703125	\\
0.815687841257158	-0.000885009765625	\\
0.815732232432193	-0.000823974609375	\\
0.815776623607227	-0.001007080078125	\\
0.815821014782261	-0.001373291015625	\\
0.815865405957296	-0.00128173828125	\\
0.81590979713233	-0.00128173828125	\\
0.815954188307365	-0.001312255859375	\\
0.815998579482399	-0.001556396484375	\\
0.816042970657433	-0.001678466796875	\\
0.816087361832468	-0.00177001953125	\\
0.816131753007502	-0.002044677734375	\\
0.816176144182536	-0.001739501953125	\\
0.816220535357571	-0.001556396484375	\\
0.816264926532605	-0.0015869140625	\\
0.81630931770764	-0.00164794921875	\\
0.816353708882674	-0.00177001953125	\\
0.816398100057709	-0.001678466796875	\\
0.816442491232743	-0.00164794921875	\\
0.816486882407777	-0.00177001953125	\\
0.816531273582812	-0.002044677734375	\\
0.816575664757846	-0.00201416015625	\\
0.816620055932881	-0.002166748046875	\\
0.816664447107915	-0.002166748046875	\\
0.816708838282949	-0.002288818359375	\\
0.816753229457984	-0.0023193359375	\\
0.816797620633018	-0.002288818359375	\\
0.816842011808053	-0.002197265625	\\
0.816886402983087	-0.002471923828125	\\
0.816930794158121	-0.002593994140625	\\
0.816975185333156	-0.00244140625	\\
0.81701957650819	-0.0025634765625	\\
0.817063967683225	-0.002532958984375	\\
0.817108358858259	-0.001861572265625	\\
0.817152750033293	-0.001983642578125	\\
0.817197141208328	-0.001800537109375	\\
0.817241532383362	-0.001739501953125	\\
0.817285923558397	-0.001983642578125	\\
0.817330314733431	-0.001800537109375	\\
0.817374705908465	-0.00177001953125	\\
0.8174190970835	-0.001556396484375	\\
0.817463488258534	-0.001556396484375	\\
0.817507879433569	-0.001983642578125	\\
0.817552270608603	-0.0018310546875	\\
0.817596661783637	-0.001556396484375	\\
0.817641052958672	-0.001068115234375	\\
0.817685444133706	-0.000213623046875	\\
0.817729835308741	-0.0001220703125	\\
0.817774226483775	-0.0003662109375	\\
0.817818617658809	-0.00030517578125	\\
0.817863008833844	-0.000732421875	\\
0.817907400008878	-0.000732421875	\\
0.817951791183913	-0.00048828125	\\
0.817996182358947	-0.000579833984375	\\
0.818040573533981	-0.000335693359375	\\
0.818084964709016	6.103515625e-05	\\
0.81812935588405	0.000274658203125	\\
0.818173747059085	0.000732421875	\\
0.818218138234119	0.000823974609375	\\
0.818262529409153	0.000579833984375	\\
0.818306920584188	0.000732421875	\\
0.818351311759222	0.000244140625	\\
0.818395702934257	6.103515625e-05	\\
0.818440094109291	0.0003662109375	\\
0.818484485284325	0.0008544921875	\\
0.81852887645936	0.001312255859375	\\
0.818573267634394	0.00091552734375	\\
0.818617658809429	0.00079345703125	\\
0.818662049984463	0.00048828125	\\
0.818706441159498	-0.00018310546875	\\
0.818750832334532	-3.0517578125e-05	\\
0.818795223509566	3.0517578125e-05	\\
0.818839614684601	-0.00042724609375	\\
0.818884005859635	-0.00042724609375	\\
0.818928397034669	-0.000457763671875	\\
0.818972788209704	-0.000244140625	\\
0.819017179384738	-0.000152587890625	\\
0.819061570559773	-0.00030517578125	\\
0.819105961734807	-0.00054931640625	\\
0.819150352909842	-0.000946044921875	\\
0.819194744084876	-0.00091552734375	\\
0.81923913525991	-0.001129150390625	\\
0.819283526434945	-0.001190185546875	\\
0.819327917609979	-0.001251220703125	\\
0.819372308785014	-0.00091552734375	\\
0.819416699960048	-0.00067138671875	\\
0.819461091135082	-0.000823974609375	\\
0.819505482310117	-0.000823974609375	\\
0.819549873485151	-0.001129150390625	\\
0.819594264660186	-0.001312255859375	\\
0.81963865583522	-0.001129150390625	\\
0.819683047010254	-0.001068115234375	\\
0.819727438185289	-0.00067138671875	\\
0.819771829360323	-0.00018310546875	\\
0.819816220535358	-0.0003662109375	\\
0.819860611710392	-0.000335693359375	\\
0.819905002885426	-0.000274658203125	\\
0.819949394060461	-0.00048828125	\\
0.819993785235495	-0.00048828125	\\
0.82003817641053	-0.00018310546875	\\
0.820082567585564	3.0517578125e-05	\\
0.820126958760598	3.0517578125e-05	\\
0.820171349935633	-0.000457763671875	\\
0.820215741110667	-0.000946044921875	\\
0.820260132285702	-0.000885009765625	\\
0.820304523460736	-0.001007080078125	\\
0.82034891463577	-0.001190185546875	\\
0.820393305810805	-0.000946044921875	\\
0.820437696985839	-0.000946044921875	\\
0.820482088160874	-0.001190185546875	\\
0.820526479335908	-0.000762939453125	\\
0.820570870510942	-0.0006103515625	\\
0.820615261685977	-0.00048828125	\\
0.820659652861011	-0.000518798828125	\\
0.820704044036046	-0.001190185546875	\\
0.82074843521108	-0.0010986328125	\\
0.820792826386114	-0.0009765625	\\
0.820837217561149	-0.001129150390625	\\
0.820881608736183	-0.00189208984375	\\
0.820925999911218	-0.00177001953125	\\
0.820970391086252	-0.00152587890625	\\
0.821014782261286	-0.00189208984375	\\
0.821059173436321	-0.001617431640625	\\
0.821103564611355	-0.001556396484375	\\
0.82114795578639	-0.001556396484375	\\
0.821192346961424	-0.00152587890625	\\
0.821236738136458	-0.00140380859375	\\
0.821281129311493	-0.001220703125	\\
0.821325520486527	-0.001251220703125	\\
0.821369911661562	-0.00152587890625	\\
0.821414302836596	-0.00152587890625	\\
0.821458694011631	-0.00103759765625	\\
0.821503085186665	-0.0009765625	\\
0.821547476361699	-0.00103759765625	\\
0.821591867536734	-0.00103759765625	\\
0.821636258711768	-0.00054931640625	\\
0.821680649886802	-6.103515625e-05	\\
0.821725041061837	0.00018310546875	\\
0.821769432236871	0.000274658203125	\\
0.821813823411906	0.000396728515625	\\
0.82185821458694	0.000518798828125	\\
0.821902605761974	6.103515625e-05	\\
0.821946996937009	-0.000152587890625	\\
0.821991388112043	0.000396728515625	\\
0.822035779287078	0.00048828125	\\
0.822080170462112	0.000335693359375	\\
0.822124561637147	0.000152587890625	\\
0.822168952812181	0.000732421875	\\
0.822213343987215	0.000640869140625	\\
0.82225773516225	0.000885009765625	\\
0.822302126337284	0.000885009765625	\\
0.822346517512319	0.0001220703125	\\
0.822390908687353	0.000213623046875	\\
0.822435299862387	-0.000213623046875	\\
0.822479691037422	0.000213623046875	\\
0.822524082212456	0.000518798828125	\\
0.822568473387491	0.000274658203125	\\
0.822612864562525	0.000701904296875	\\
0.822657255737559	0.00042724609375	\\
0.822701646912594	0.0003662109375	\\
0.822746038087628	0.00048828125	\\
0.822790429262663	0.00048828125	\\
0.822834820437697	0.000457763671875	\\
0.822879211612731	3.0517578125e-05	\\
0.822923602787766	-0.00042724609375	\\
0.8229679939628	-0.00048828125	\\
0.823012385137835	-0.0006103515625	\\
0.823056776312869	-0.000396728515625	\\
0.823101167487903	-0.000244140625	\\
0.823145558662938	-0.000244140625	\\
0.823189949837972	-0.00018310546875	\\
0.823234341013007	-0.00067138671875	\\
0.823278732188041	-0.000579833984375	\\
0.823323123363075	-0.000518798828125	\\
0.82336751453811	-0.000732421875	\\
0.823411905713144	-0.00048828125	\\
0.823456296888179	-0.000579833984375	\\
0.823500688063213	-0.000213623046875	\\
0.823545079238247	0.000152587890625	\\
0.823589470413282	-0.00048828125	\\
0.823633861588316	9.1552734375e-05	\\
0.823678252763351	0.000396728515625	\\
0.823722643938385	0.000396728515625	\\
0.823767035113419	0.001190185546875	\\
0.823811426288454	0.001220703125	\\
0.823855817463488	0.00079345703125	\\
0.823900208638523	0.000762939453125	\\
0.823944599813557	0.000823974609375	\\
0.823988990988591	0.00067138671875	\\
0.824033382163626	0.00115966796875	\\
0.82407777333866	0.001251220703125	\\
0.824122164513695	0.00067138671875	\\
0.824166555688729	0.000946044921875	\\
0.824210946863764	0.001373291015625	\\
0.824255338038798	0.001556396484375	\\
0.824299729213832	0.001739501953125	\\
0.824344120388867	0.00152587890625	\\
0.824388511563901	0.001678466796875	\\
0.824432902738936	0.001953125	\\
0.82447729391397	0.00177001953125	\\
0.824521685089004	0.00128173828125	\\
0.824566076264039	0.00146484375	\\
0.824610467439073	0.00140380859375	\\
0.824654858614107	0.00146484375	\\
0.824699249789142	0.00079345703125	\\
0.824743640964176	0.00067138671875	\\
0.824788032139211	0.000885009765625	\\
0.824832423314245	0.00054931640625	\\
0.82487681448928	0.0003662109375	\\
0.824921205664314	0.000213623046875	\\
0.824965596839348	0.00048828125	\\
0.825009988014383	0.00018310546875	\\
0.825054379189417	0	\\
0.825098770364452	6.103515625e-05	\\
0.825143161539486	-0.000396728515625	\\
0.82518755271452	-0.000396728515625	\\
0.825231943889555	-0.000396728515625	\\
0.825276335064589	-0.000396728515625	\\
0.825320726239624	0.00018310546875	\\
0.825365117414658	-0.00018310546875	\\
0.825409508589692	-0.0003662109375	\\
0.825453899764727	-0.000274658203125	\\
0.825498290939761	-0.00030517578125	\\
0.825542682114796	-0.0006103515625	\\
0.82558707328983	-0.0006103515625	\\
0.825631464464864	-9.1552734375e-05	\\
0.825675855639899	3.0517578125e-05	\\
0.825720246814933	3.0517578125e-05	\\
0.825764637989968	0.000335693359375	\\
0.825809029165002	-0.000152587890625	\\
0.825853420340036	-0.000335693359375	\\
0.825897811515071	-0.000213623046875	\\
0.825942202690105	-0.000640869140625	\\
0.82598659386514	-0.000762939453125	\\
0.826030985040174	-0.00067138671875	\\
0.826075376215208	-0.0003662109375	\\
0.826119767390243	-0.00030517578125	\\
0.826164158565277	-0.00018310546875	\\
0.826208549740312	0.00018310546875	\\
0.826252940915346	0.000213623046875	\\
0.82629733209038	6.103515625e-05	\\
0.826341723265415	0.000396728515625	\\
0.826386114440449	0.000244140625	\\
0.826430505615484	-6.103515625e-05	\\
0.826474896790518	0.000213623046875	\\
0.826519287965552	6.103515625e-05	\\
0.826563679140587	0.000213623046875	\\
0.826608070315621	0.000762939453125	\\
0.826652461490656	0.00103759765625	\\
0.82669685266569	0.000762939453125	\\
0.826741243840724	0.00054931640625	\\
0.826785635015759	0.000396728515625	\\
0.826830026190793	-0.000213623046875	\\
0.826874417365828	-0.000396728515625	\\
0.826918808540862	-0.0003662109375	\\
0.826963199715896	-0.0003662109375	\\
0.827007590890931	-0.000213623046875	\\
0.827051982065965	-9.1552734375e-05	\\
0.827096373241	-6.103515625e-05	\\
0.827140764416034	0.000213623046875	\\
0.827185155591069	0.0001220703125	\\
0.827229546766103	-9.1552734375e-05	\\
0.827273937941137	0	\\
0.827318329116172	-0.000335693359375	\\
0.827362720291206	-0.000762939453125	\\
0.82740711146624	-0.00054931640625	\\
0.827451502641275	-0.000274658203125	\\
0.827495893816309	6.103515625e-05	\\
0.827540284991344	9.1552734375e-05	\\
0.827584676166378	-0.00018310546875	\\
0.827629067341413	-6.103515625e-05	\\
0.827673458516447	0.000457763671875	\\
0.827717849691481	0.00054931640625	\\
0.827762240866516	0.00030517578125	\\
0.82780663204155	0	\\
0.827851023216585	0.000274658203125	\\
0.827895414391619	3.0517578125e-05	\\
0.827939805566653	0.000244140625	\\
0.827984196741688	0.000213623046875	\\
0.828028587916722	0.00030517578125	\\
0.828072979091757	0.00067138671875	\\
0.828117370266791	0.00018310546875	\\
0.828161761441825	0.000579833984375	\\
0.82820615261686	0.0003662109375	\\
0.828250543791894	-9.1552734375e-05	\\
0.828294934966929	0.000213623046875	\\
0.828339326141963	0.00018310546875	\\
0.828383717316997	-0.000152587890625	\\
0.828428108492032	-0.00048828125	\\
0.828472499667066	-0.00042724609375	\\
0.828516890842101	0	\\
0.828561282017135	-9.1552734375e-05	\\
0.828605673192169	-0.000152587890625	\\
0.828650064367204	0.000244140625	\\
0.828694455542238	0.0001220703125	\\
0.828738846717273	-9.1552734375e-05	\\
0.828783237892307	-0.000335693359375	\\
0.828827629067341	-0.000244140625	\\
0.828872020242376	-0.000244140625	\\
0.82891641141741	-0.000823974609375	\\
0.828960802592445	-0.0009765625	\\
0.829005193767479	-0.00054931640625	\\
0.829049584942513	-0.000701904296875	\\
0.829093976117548	-0.0008544921875	\\
0.829138367292582	-0.0008544921875	\\
0.829182758467617	-0.00091552734375	\\
0.829227149642651	-0.00067138671875	\\
0.829271540817685	-0.00103759765625	\\
0.82931593199272	-0.00103759765625	\\
0.829360323167754	-0.001220703125	\\
0.829404714342789	-0.000823974609375	\\
0.829449105517823	-0.0008544921875	\\
0.829493496692857	-0.001495361328125	\\
0.829537887867892	-0.001251220703125	\\
0.829582279042926	-0.001190185546875	\\
0.829626670217961	-0.000946044921875	\\
0.829671061392995	-0.000396728515625	\\
0.829715452568029	0.000152587890625	\\
0.829759843743064	-0.000213623046875	\\
0.829804234918098	-0.000213623046875	\\
0.829848626093133	-0.000579833984375	\\
0.829893017268167	-0.001434326171875	\\
0.829937408443202	-0.0015869140625	\\
0.829981799618236	-0.0015869140625	\\
0.83002619079327	-0.00152587890625	\\
0.830070581968305	-0.001617431640625	\\
0.830114973143339	-0.001220703125	\\
0.830159364318374	-0.000885009765625	\\
0.830203755493408	-0.0013427734375	\\
0.830248146668442	-0.0010986328125	\\
0.830292537843477	-0.000335693359375	\\
0.830336929018511	-0.000244140625	\\
0.830381320193545	-0.00042724609375	\\
0.83042571136858	-0.00054931640625	\\
0.830470102543614	-0.00091552734375	\\
0.830514493718649	-0.00091552734375	\\
0.830558884893683	-0.000762939453125	\\
0.830603276068718	-0.000579833984375	\\
0.830647667243752	-0.000579833984375	\\
0.830692058418786	-0.000701904296875	\\
0.830736449593821	-0.000640869140625	\\
0.830780840768855	-0.000152587890625	\\
0.83082523194389	-0.000274658203125	\\
0.830869623118924	-0.0008544921875	\\
0.830914014293958	-0.0006103515625	\\
0.830958405468993	-0.00079345703125	\\
0.831002796644027	-0.001129150390625	\\
0.831047187819062	-0.001068115234375	\\
0.831091578994096	-0.000823974609375	\\
0.83113597016913	-0.00048828125	\\
0.831180361344165	-0.00030517578125	\\
0.831224752519199	-0.00067138671875	\\
0.831269143694234	-0.000518798828125	\\
0.831313534869268	-0.00048828125	\\
0.831357926044302	-9.1552734375e-05	\\
0.831402317219337	0.000396728515625	\\
0.831446708394371	-6.103515625e-05	\\
0.831491099569406	-9.1552734375e-05	\\
0.83153549074444	-0.000213623046875	\\
0.831579881919474	-0.00018310546875	\\
0.831624273094509	-9.1552734375e-05	\\
0.831668664269543	0.000213623046875	\\
0.831713055444578	0.000396728515625	\\
0.831757446619612	0.000213623046875	\\
0.831801837794646	0.00042724609375	\\
0.831846228969681	0.000518798828125	\\
0.831890620144715	0.00054931640625	\\
0.83193501131975	3.0517578125e-05	\\
0.831979402494784	-0.000152587890625	\\
0.832023793669818	-0.0003662109375	\\
0.832068184844853	-0.000152587890625	\\
0.832112576019887	0.000396728515625	\\
0.832156967194922	3.0517578125e-05	\\
0.832201358369956	-6.103515625e-05	\\
0.83224574954499	0.000579833984375	\\
0.832290140720025	0.0006103515625	\\
0.832334531895059	0.000518798828125	\\
0.832378923070094	0.000518798828125	\\
0.832423314245128	0.000335693359375	\\
0.832467705420162	0.0003662109375	\\
0.832512096595197	0.0001220703125	\\
0.832556487770231	-0.000274658203125	\\
0.832600878945266	0.000213623046875	\\
0.8326452701203	9.1552734375e-05	\\
0.832689661295335	-0.000274658203125	\\
0.832734052470369	-0.000152587890625	\\
0.832778443645403	-6.103515625e-05	\\
0.832822834820438	-0.000335693359375	\\
0.832867225995472	-0.000701904296875	\\
0.832911617170507	-0.00128173828125	\\
0.832956008345541	-0.00128173828125	\\
0.833000399520575	-0.0010986328125	\\
0.83304479069561	-0.00164794921875	\\
0.833089181870644	-0.00189208984375	\\
0.833133573045678	-0.0015869140625	\\
0.833177964220713	-0.001708984375	\\
0.833222355395747	-0.00164794921875	\\
0.833266746570782	-0.00164794921875	\\
0.833311137745816	-0.001495361328125	\\
0.833355528920851	-0.001129150390625	\\
0.833399920095885	-0.00146484375	\\
0.833444311270919	-0.0018310546875	\\
0.833488702445954	-0.00152587890625	\\
0.833533093620988	-0.00128173828125	\\
0.833577484796023	-0.0013427734375	\\
0.833621875971057	-0.001220703125	\\
0.833666267146091	-0.00152587890625	\\
0.833710658321126	-0.001373291015625	\\
0.83375504949616	-0.0013427734375	\\
0.833799440671195	-0.0015869140625	\\
0.833843831846229	-0.00152587890625	\\
0.833888223021263	-0.001861572265625	\\
0.833932614196298	-0.001922607421875	\\
0.833977005371332	-0.00164794921875	\\
0.834021396546367	-0.001678466796875	\\
0.834065787721401	-0.00177001953125	\\
0.834110178896435	-0.001800537109375	\\
0.83415457007147	-0.001739501953125	\\
0.834198961246504	-0.0020751953125	\\
0.834243352421539	-0.001983642578125	\\
0.834287743596573	-0.00146484375	\\
0.834332134771607	-0.001373291015625	\\
0.834376525946642	-0.00152587890625	\\
0.834420917121676	-0.001373291015625	\\
0.834465308296711	-0.00146484375	\\
0.834509699471745	-0.001983642578125	\\
0.834554090646779	-0.0018310546875	\\
0.834598481821814	-0.001953125	\\
0.834642872996848	-0.00213623046875	\\
0.834687264171883	-0.0018310546875	\\
0.834731655346917	-0.002105712890625	\\
0.834776046521951	-0.001556396484375	\\
0.834820437696986	-0.001220703125	\\
0.83486482887202	-0.00140380859375	\\
0.834909220047055	-0.001129150390625	\\
0.834953611222089	-0.00152587890625	\\
0.834998002397123	-0.001953125	\\
0.835042393572158	-0.001739501953125	\\
0.835086784747192	-0.001800537109375	\\
0.835131175922227	-0.0018310546875	\\
0.835175567097261	-0.001434326171875	\\
0.835219958272295	-0.001251220703125	\\
0.83526434944733	-0.00146484375	\\
0.835308740622364	-0.00091552734375	\\
0.835353131797399	-0.000640869140625	\\
0.835397522972433	-0.001434326171875	\\
0.835441914147467	-0.00146484375	\\
0.835486305322502	-0.0015869140625	\\
0.835530696497536	-0.001556396484375	\\
0.835575087672571	-0.0013427734375	\\
0.835619478847605	-0.00115966796875	\\
0.83566387002264	-0.000823974609375	\\
0.835708261197674	-0.000518798828125	\\
0.835752652372708	-0.00018310546875	\\
0.835797043547743	-0.000335693359375	\\
0.835841434722777	-0.000457763671875	\\
0.835885825897811	-0.000732421875	\\
0.835930217072846	-0.001007080078125	\\
0.83597460824788	-0.001190185546875	\\
0.836018999422915	-0.001312255859375	\\
0.836063390597949	-0.001373291015625	\\
0.836107781772984	-0.001129150390625	\\
0.836152172948018	-0.0009765625	\\
0.836196564123052	-0.001007080078125	\\
0.836240955298087	-0.000885009765625	\\
0.836285346473121	-0.000701904296875	\\
0.836329737648156	-0.001007080078125	\\
0.83637412882319	-0.00128173828125	\\
0.836418519998224	-0.00164794921875	\\
0.836462911173259	-0.001495361328125	\\
0.836507302348293	-0.001068115234375	\\
0.836551693523328	-0.001220703125	\\
0.836596084698362	-0.001220703125	\\
0.836640475873396	-0.001312255859375	\\
0.836684867048431	-0.001251220703125	\\
0.836729258223465	-0.0010986328125	\\
0.8367736493985	-0.0013427734375	\\
0.836818040573534	-0.001373291015625	\\
0.836862431748568	-0.00091552734375	\\
0.836906822923603	-0.00140380859375	\\
0.836951214098637	-0.001708984375	\\
0.836995605273672	-0.00177001953125	\\
0.837039996448706	-0.0018310546875	\\
0.83708438762374	-0.0015869140625	\\
0.837128778798775	-0.00140380859375	\\
0.837173169973809	-0.001495361328125	\\
0.837217561148844	-0.00140380859375	\\
0.837261952323878	-0.00103759765625	\\
0.837306343498912	-0.0008544921875	\\
0.837350734673947	-0.001373291015625	\\
0.837395125848981	-0.001617431640625	\\
0.837439517024016	-0.001678466796875	\\
0.83748390819905	-0.001739501953125	\\
0.837528299374084	-0.00177001953125	\\
0.837572690549119	-0.001861572265625	\\
0.837617081724153	-0.001739501953125	\\
0.837661472899188	-0.001983642578125	\\
0.837705864074222	-0.001617431640625	\\
0.837750255249256	-0.001556396484375	\\
0.837794646424291	-0.00140380859375	\\
0.837839037599325	-0.00115966796875	\\
0.83788342877436	-0.001312255859375	\\
0.837927819949394	-0.001312255859375	\\
0.837972211124428	-0.001800537109375	\\
0.838016602299463	-0.00213623046875	\\
0.838060993474497	-0.002044677734375	\\
0.838105384649532	-0.002593994140625	\\
0.838149775824566	-0.002410888671875	\\
0.8381941669996	-0.002044677734375	\\
0.838238558174635	-0.001861572265625	\\
0.838282949349669	-0.00164794921875	\\
0.838327340524704	-0.001495361328125	\\
0.838371731699738	-0.001678466796875	\\
0.838416122874773	-0.00177001953125	\\
0.838460514049807	-0.001861572265625	\\
0.838504905224841	-0.002288818359375	\\
0.838549296399876	-0.00213623046875	\\
0.83859368757491	-0.00189208984375	\\
0.838638078749945	-0.00201416015625	\\
0.838682469924979	-0.001861572265625	\\
0.838726861100013	-0.001922607421875	\\
0.838771252275048	-0.00213623046875	\\
0.838815643450082	-0.001495361328125	\\
0.838860034625116	-0.001495361328125	\\
0.838904425800151	-0.001617431640625	\\
0.838948816975185	-0.001373291015625	\\
0.83899320815022	-0.001190185546875	\\
0.839037599325254	-0.00140380859375	\\
0.839081990500289	-0.001373291015625	\\
0.839126381675323	-0.000946044921875	\\
0.839170772850357	-0.000274658203125	\\
0.839215164025392	0.00018310546875	\\
0.839259555200426	-3.0517578125e-05	\\
0.839303946375461	0.00030517578125	\\
0.839348337550495	0.00048828125	\\
0.839392728725529	-0.000152587890625	\\
0.839437119900564	0.000274658203125	\\
0.839481511075598	0.0003662109375	\\
0.839525902250633	0.000396728515625	\\
0.839570293425667	0.00048828125	\\
0.839614684600701	0.00030517578125	\\
0.839659075775736	0.0008544921875	\\
0.83970346695077	0.000518798828125	\\
0.839747858125805	0.000244140625	\\
0.839792249300839	0.000518798828125	\\
0.839836640475873	0.0008544921875	\\
0.839881031650908	0.00079345703125	\\
0.839925422825942	0.00042724609375	\\
0.839969814000977	0.0008544921875	\\
0.840014205176011	0.000732421875	\\
0.840058596351045	0.0009765625	\\
0.84010298752608	0.001312255859375	\\
0.840147378701114	0.001007080078125	\\
0.840191769876149	0.00146484375	\\
0.840236161051183	0.00152587890625	\\
0.840280552226217	0.001617431640625	\\
0.840324943401252	0.0010986328125	\\
0.840369334576286	0.001129150390625	\\
0.840413725751321	0.00079345703125	\\
0.840458116926355	0.00079345703125	\\
0.840502508101389	0.001068115234375	\\
0.840546899276424	0.00030517578125	\\
0.840591290451458	0.00030517578125	\\
0.840635681626493	0.000213623046875	\\
0.840680072801527	0.000457763671875	\\
0.840724463976561	0.0008544921875	\\
0.840768855151596	0.00091552734375	\\
0.84081324632663	0.000946044921875	\\
0.840857637501665	0.000732421875	\\
0.840902028676699	0.00048828125	\\
0.840946419851733	0.000579833984375	\\
0.840990811026768	0.000946044921875	\\
0.841035202201802	0.000640869140625	\\
0.841079593376837	0.000579833984375	\\
0.841123984551871	0.000335693359375	\\
0.841168375726906	0.00048828125	\\
0.84121276690194	0.000457763671875	\\
0.841257158076974	-6.103515625e-05	\\
0.841301549252009	0.00030517578125	\\
0.841345940427043	0.000457763671875	\\
0.841390331602078	6.103515625e-05	\\
0.841434722777112	0.00018310546875	\\
0.841479113952146	0.000518798828125	\\
0.841523505127181	0.000274658203125	\\
0.841567896302215	0.000396728515625	\\
0.841612287477249	0.000640869140625	\\
0.841656678652284	0.000457763671875	\\
0.841701069827318	0.00042724609375	\\
0.841745461002353	0.00042724609375	\\
0.841789852177387	0.00042724609375	\\
0.841834243352422	0.00018310546875	\\
0.841878634527456	0.00042724609375	\\
0.84192302570249	0.00067138671875	\\
0.841967416877525	0.000579833984375	\\
0.842011808052559	0.000518798828125	\\
0.842056199227594	0.00048828125	\\
0.842100590402628	0.000640869140625	\\
0.842144981577662	0.000640869140625	\\
0.842189372752697	0.000518798828125	\\
0.842233763927731	0.000274658203125	\\
0.842278155102766	0.000732421875	\\
0.8423225462778	0.0003662109375	\\
0.842366937452834	0.000335693359375	\\
0.842411328627869	0.000732421875	\\
0.842455719802903	0.000274658203125	\\
0.842500110977938	0.00067138671875	\\
0.842544502152972	0.001129150390625	\\
0.842588893328006	0.000946044921875	\\
0.842633284503041	0.000518798828125	\\
0.842677675678075	0.000274658203125	\\
0.84272206685311	0.00018310546875	\\
0.842766458028144	-0.0001220703125	\\
0.842810849203178	-0.0003662109375	\\
0.842855240378213	-0.00018310546875	\\
0.842899631553247	0	\\
0.842944022728282	0.00018310546875	\\
0.842988413903316	0.000701904296875	\\
0.84303280507835	0.001068115234375	\\
0.843077196253385	0.000640869140625	\\
0.843121587428419	0.0006103515625	\\
0.843165978603454	0.00048828125	\\
0.843210369778488	0.000335693359375	\\
0.843254760953522	0.000640869140625	\\
0.843299152128557	0.00103759765625	\\
0.843343543303591	0.0010986328125	\\
0.843387934478626	0.000946044921875	\\
0.84343232565366	0.001068115234375	\\
0.843476716828694	0.0013427734375	\\
0.843521108003729	0.0009765625	\\
0.843565499178763	6.103515625e-05	\\
0.843609890353798	0.0006103515625	\\
0.843654281528832	0.0009765625	\\
0.843698672703866	0.000823974609375	\\
0.843743063878901	0.0009765625	\\
0.843787455053935	0.001007080078125	\\
0.84383184622897	0.0010986328125	\\
0.843876237404004	0.001007080078125	\\
0.843920628579038	0.0008544921875	\\
0.843965019754073	0.000946044921875	\\
0.844009410929107	0.0009765625	\\
0.844053802104142	0.0009765625	\\
0.844098193279176	0.000579833984375	\\
0.844142584454211	0.000640869140625	\\
0.844186975629245	0.0009765625	\\
0.844231366804279	0.000701904296875	\\
0.844275757979314	0.000823974609375	\\
0.844320149154348	0.001220703125	\\
0.844364540329383	0.001434326171875	\\
0.844408931504417	0.001495361328125	\\
0.844453322679451	0.00115966796875	\\
0.844497713854486	0.00103759765625	\\
0.84454210502952	0.00164794921875	\\
0.844586496204555	0.0015869140625	\\
0.844630887379589	0.001220703125	\\
0.844675278554623	0.0010986328125	\\
0.844719669729658	0.00115966796875	\\
0.844764060904692	0.001495361328125	\\
0.844808452079727	0.00189208984375	\\
0.844852843254761	0.001861572265625	\\
0.844897234429795	0.001922607421875	\\
0.84494162560483	0.0020751953125	\\
0.844986016779864	0.001800537109375	\\
0.845030407954899	0.001708984375	\\
0.845074799129933	0.00164794921875	\\
0.845119190304967	0.001953125	\\
0.845163581480002	0.0018310546875	\\
0.845207972655036	0.001861572265625	\\
0.845252363830071	0.002288818359375	\\
0.845296755005105	0.00274658203125	\\
0.845341146180139	0.002899169921875	\\
0.845385537355174	0.00262451171875	\\
0.845429928530208	0.002899169921875	\\
0.845474319705243	0.003021240234375	\\
0.845518710880277	0.0025634765625	\\
0.845563102055311	0.002593994140625	\\
0.845607493230346	0.00262451171875	\\
0.84565188440538	0.00213623046875	\\
0.845696275580415	0.00238037109375	\\
0.845740666755449	0.002777099609375	\\
0.845785057930483	0.00286865234375	\\
0.845829449105518	0.003021240234375	\\
0.845873840280552	0.002471923828125	\\
0.845918231455587	0.002899169921875	\\
0.845962622630621	0.002716064453125	\\
0.846007013805655	0.001953125	\\
0.84605140498069	0.001861572265625	\\
0.846095796155724	0.001983642578125	\\
0.846140187330759	0.001922607421875	\\
0.846184578505793	0.001678466796875	\\
0.846228969680828	0.001678466796875	\\
0.846273360855862	0.00115966796875	\\
0.846317752030896	0.001251220703125	\\
0.846362143205931	0.0013427734375	\\
0.846406534380965	0.001251220703125	\\
0.846450925555999	0.0010986328125	\\
0.846495316731034	0.001129150390625	\\
0.846539707906068	0.000732421875	\\
0.846584099081103	0.00054931640625	\\
0.846628490256137	0.0008544921875	\\
0.846672881431171	0.000579833984375	\\
0.846717272606206	0.000274658203125	\\
0.84676166378124	0.000335693359375	\\
0.846806054956275	0.00067138671875	\\
0.846850446131309	0.000518798828125	\\
0.846894837306344	0.000457763671875	\\
0.846939228481378	0.00054931640625	\\
0.846983619656412	0.000274658203125	\\
0.847028010831447	3.0517578125e-05	\\
0.847072402006481	-0.000335693359375	\\
0.847116793181516	-0.000701904296875	\\
0.84716118435655	-0.00054931640625	\\
0.847205575531584	-0.00030517578125	\\
0.847249966706619	-0.000213623046875	\\
0.847294357881653	-3.0517578125e-05	\\
0.847338749056687	-0.000152587890625	\\
0.847383140231722	-0.000244140625	\\
0.847427531406756	0.000396728515625	\\
0.847471922581791	0.000732421875	\\
0.847516313756825	0.00030517578125	\\
0.84756070493186	0.000152587890625	\\
0.847605096106894	-0.000152587890625	\\
0.847649487281928	0	\\
0.847693878456963	0.000640869140625	\\
0.847738269631997	0.00042724609375	\\
0.847782660807032	0.00067138671875	\\
0.847827051982066	0.000732421875	\\
0.8478714431571	0.000457763671875	\\
0.847915834332135	0.00018310546875	\\
0.847960225507169	-0.000213623046875	\\
0.848004616682204	0.000152587890625	\\
0.848049007857238	0.000457763671875	\\
0.848093399032272	0.000244140625	\\
0.848137790207307	0.000701904296875	\\
0.848182181382341	0.000579833984375	\\
0.848226572557376	0.000640869140625	\\
0.84827096373241	0.000579833984375	\\
0.848315354907444	0.0003662109375	\\
0.848359746082479	0.001007080078125	\\
0.848404137257513	0.0008544921875	\\
0.848448528432548	0.000762939453125	\\
0.848492919607582	0.00115966796875	\\
0.848537310782616	0.000701904296875	\\
0.848581701957651	0.000762939453125	\\
0.848626093132685	0.000762939453125	\\
0.84867048430772	0.0008544921875	\\
0.848714875482754	0.001220703125	\\
0.848759266657788	0.001434326171875	\\
0.848803657832823	0.002044677734375	\\
0.848848049007857	0.00213623046875	\\
0.848892440182892	0.002288818359375	\\
0.848936831357926	0.002288818359375	\\
0.84898122253296	0.0023193359375	\\
0.849025613707995	0.00250244140625	\\
0.849070004883029	0.001953125	\\
0.849114396058064	0.002227783203125	\\
0.849158787233098	0.002593994140625	\\
0.849203178408132	0.002471923828125	\\
0.849247569583167	0.002227783203125	\\
0.849291960758201	0.00250244140625	\\
0.849336351933236	0.00286865234375	\\
0.84938074310827	0.00286865234375	\\
0.849425134283304	0.002532958984375	\\
0.849469525458339	0.00262451171875	\\
0.849513916633373	0.002655029296875	\\
0.849558307808408	0.002044677734375	\\
0.849602698983442	0.001983642578125	\\
0.849647090158477	0.001708984375	\\
0.849691481333511	0.002044677734375	\\
0.849735872508545	0.0020751953125	\\
0.84978026368358	0.001800537109375	\\
0.849824654858614	0.0020751953125	\\
0.849869046033649	0.0020751953125	\\
0.849913437208683	0.0020751953125	\\
0.849957828383717	0.00177001953125	\\
0.850002219558752	0.001495361328125	\\
0.850046610733786	0.001220703125	\\
0.85009100190882	0.000701904296875	\\
0.850135393083855	0.0001220703125	\\
0.850179784258889	0	\\
0.850224175433924	-0.000152587890625	\\
0.850268566608958	-0.000244140625	\\
0.850312957783993	-0.00048828125	\\
0.850357348959027	-0.000640869140625	\\
0.850401740134061	-0.000518798828125	\\
0.850446131309096	-0.000885009765625	\\
0.85049052248413	-0.001312255859375	\\
0.850534913659165	-0.001312255859375	\\
0.850579304834199	-0.001617431640625	\\
0.850623696009233	-0.00225830078125	\\
0.850668087184268	-0.00238037109375	\\
0.850712478359302	-0.00189208984375	\\
0.850756869534337	-0.0020751953125	\\
0.850801260709371	-0.002166748046875	\\
0.850845651884405	-0.002105712890625	\\
0.85089004305944	-0.0020751953125	\\
0.850934434234474	-0.002166748046875	\\
0.850978825409509	-0.00189208984375	\\
0.851023216584543	-0.001861572265625	\\
0.851067607759577	-0.001739501953125	\\
0.851111998934612	-0.00146484375	\\
0.851156390109646	-0.001739501953125	\\
0.851200781284681	-0.001373291015625	\\
0.851245172459715	-0.000946044921875	\\
0.851289563634749	-0.00103759765625	\\
0.851333954809784	-0.000946044921875	\\
0.851378345984818	-0.001007080078125	\\
0.851422737159853	-0.000946044921875	\\
0.851467128334887	-0.0013427734375	\\
0.851511519509921	-0.001190185546875	\\
0.851555910684956	-0.000701904296875	\\
0.85160030185999	-0.001007080078125	\\
0.851644693035025	-0.000823974609375	\\
0.851689084210059	-0.000946044921875	\\
0.851733475385093	-0.0015869140625	\\
0.851777866560128	-0.001220703125	\\
0.851822257735162	-0.00146484375	\\
0.851866648910197	-0.00115966796875	\\
0.851911040085231	-0.000640869140625	\\
0.851955431260265	-0.000579833984375	\\
0.8519998224353	-0.000213623046875	\\
0.852044213610334	-0.000274658203125	\\
0.852088604785369	-0.0006103515625	\\
0.852132995960403	-0.000732421875	\\
0.852177387135437	-0.00054931640625	\\
0.852221778310472	-0.0008544921875	\\
0.852266169485506	-0.001068115234375	\\
0.852310560660541	-0.001251220703125	\\
0.852354951835575	-0.00140380859375	\\
0.852399343010609	-0.001220703125	\\
0.852443734185644	-0.001068115234375	\\
0.852488125360678	-0.000732421875	\\
0.852532516535713	-0.00067138671875	\\
0.852576907710747	-0.00128173828125	\\
0.852621298885782	-0.001312255859375	\\
0.852665690060816	-0.001434326171875	\\
0.85271008123585	-0.0013427734375	\\
0.852754472410885	-0.001068115234375	\\
0.852798863585919	-0.001373291015625	\\
0.852843254760954	-0.00164794921875	\\
0.852887645935988	-0.001434326171875	\\
0.852932037111022	-0.00079345703125	\\
0.852976428286057	-0.00067138671875	\\
0.853020819461091	-0.00067138671875	\\
0.853065210636126	-0.00054931640625	\\
0.85310960181116	-0.000885009765625	\\
0.853153992986194	-0.001068115234375	\\
0.853198384161229	-0.0009765625	\\
0.853242775336263	-0.001190185546875	\\
0.853287166511298	-0.00103759765625	\\
0.853331557686332	-0.000946044921875	\\
0.853375948861366	-0.001190185546875	\\
0.853420340036401	-0.0009765625	\\
0.853464731211435	-0.000335693359375	\\
0.85350912238647	6.103515625e-05	\\
0.853553513561504	-3.0517578125e-05	\\
0.853597904736538	-0.000274658203125	\\
0.853642295911573	-9.1552734375e-05	\\
0.853686687086607	-0.00018310546875	\\
0.853731078261642	-0.000152587890625	\\
0.853775469436676	-0.00054931640625	\\
0.85381986061171	-0.000701904296875	\\
0.853864251786745	-0.000274658203125	\\
0.853908642961779	-0.000396728515625	\\
0.853953034136814	0.000152587890625	\\
0.853997425311848	-0.000152587890625	\\
0.854041816486882	-0.00054931640625	\\
0.854086207661917	-0.000274658203125	\\
0.854130598836951	-0.00048828125	\\
0.854174990011986	-0.000244140625	\\
0.85421938118702	0	\\
0.854263772362054	-0.000244140625	\\
0.854308163537089	-0.000396728515625	\\
0.854352554712123	-0.00079345703125	\\
0.854396945887158	-0.000518798828125	\\
0.854441337062192	-0.0003662109375	\\
0.854485728237226	-0.00115966796875	\\
0.854530119412261	-0.00128173828125	\\
0.854574510587295	-0.00128173828125	\\
0.85461890176233	-0.001495361328125	\\
0.854663292937364	-0.001495361328125	\\
0.854707684112399	-0.001495361328125	\\
0.854752075287433	-0.0018310546875	\\
0.854796466462467	-0.002288818359375	\\
0.854840857637502	-0.0023193359375	\\
0.854885248812536	-0.002044677734375	\\
0.85492963998757	-0.00177001953125	\\
0.854974031162605	-0.00146484375	\\
0.855018422337639	-0.001983642578125	\\
0.855062813512674	-0.001983642578125	\\
0.855107204687708	-0.001800537109375	\\
0.855151595862742	-0.002197265625	\\
0.855195987037777	-0.00189208984375	\\
0.855240378212811	-0.001983642578125	\\
0.855284769387846	-0.002471923828125	\\
0.85532916056288	-0.002105712890625	\\
0.855373551737915	-0.002288818359375	\\
0.855417942912949	-0.00201416015625	\\
0.855462334087983	-0.0018310546875	\\
0.855506725263018	-0.002349853515625	\\
0.855551116438052	-0.001861572265625	\\
0.855595507613087	-0.001495361328125	\\
0.855639898788121	-0.001556396484375	\\
0.855684289963155	-0.00189208984375	\\
0.85572868113819	-0.001861572265625	\\
0.855773072313224	-0.00201416015625	\\
0.855817463488258	-0.0018310546875	\\
0.855861854663293	-0.001190185546875	\\
0.855906245838327	-0.001007080078125	\\
0.855950637013362	-0.0010986328125	\\
0.855995028188396	-0.001129150390625	\\
0.856039419363431	-0.001129150390625	\\
0.856083810538465	-0.001678466796875	\\
0.856128201713499	-0.001861572265625	\\
0.856172592888534	-0.00164794921875	\\
0.856216984063568	-0.001495361328125	\\
0.856261375238603	-0.001617431640625	\\
0.856305766413637	-0.001983642578125	\\
0.856350157588671	-0.001708984375	\\
0.856394548763706	-0.001190185546875	\\
0.85643893993874	-0.00140380859375	\\
0.856483331113775	-0.001220703125	\\
0.856527722288809	-0.001220703125	\\
0.856572113463843	-0.00152587890625	\\
0.856616504638878	-0.0013427734375	\\
0.856660895813912	-0.001129150390625	\\
0.856705286988947	-0.00140380859375	\\
0.856749678163981	-0.0013427734375	\\
0.856794069339015	-0.00115966796875	\\
0.85683846051405	-0.001495361328125	\\
0.856882851689084	-0.00140380859375	\\
0.856927242864119	-0.00103759765625	\\
0.856971634039153	-0.00054931640625	\\
0.857016025214187	-0.0006103515625	\\
0.857060416389222	3.0517578125e-05	\\
0.857104807564256	0.0003662109375	\\
0.857149198739291	0	\\
0.857193589914325	0.0003662109375	\\
0.857237981089359	-0.0001220703125	\\
0.857282372264394	-0.000396728515625	\\
0.857326763439428	0.00042724609375	\\
0.857371154614463	0.0006103515625	\\
0.857415545789497	0.000518798828125	\\
0.857459936964531	0.00128173828125	\\
0.857504328139566	0.001617431640625	\\
0.8575487193146	0.001800537109375	\\
0.857593110489635	0.001953125	\\
0.857637501664669	0.0018310546875	\\
0.857681892839703	0.001373291015625	\\
0.857726284014738	0.001190185546875	\\
0.857770675189772	0.001220703125	\\
0.857815066364807	0.0009765625	\\
0.857859457539841	0.00115966796875	\\
0.857903848714875	0.00128173828125	\\
0.85794823988991	0.00103759765625	\\
0.857992631064944	0.0010986328125	\\
0.858037022239979	0.001708984375	\\
0.858081413415013	0.001739501953125	\\
0.858125804590048	0.00152587890625	\\
0.858170195765082	0.00128173828125	\\
0.858214586940116	0.000946044921875	\\
0.858258978115151	0.000579833984375	\\
0.858303369290185	0.000244140625	\\
0.85834776046522	-0.000152587890625	\\
0.858392151640254	-0.00018310546875	\\
0.858436542815288	-0.0003662109375	\\
0.858480933990323	-0.000640869140625	\\
0.858525325165357	-0.000274658203125	\\
0.858569716340391	0.00018310546875	\\
0.858614107515426	6.103515625e-05	\\
0.85865849869046	-0.00030517578125	\\
0.858702889865495	-0.000640869140625	\\
0.858747281040529	-0.000396728515625	\\
0.858791672215564	-0.00042724609375	\\
0.858836063390598	-0.001068115234375	\\
0.858880454565632	-0.000885009765625	\\
0.858924845740667	-0.00048828125	\\
0.858969236915701	-0.00042724609375	\\
0.859013628090736	-0.0006103515625	\\
0.85905801926577	-0.0006103515625	\\
0.859102410440804	-0.0003662109375	\\
0.859146801615839	-0.0008544921875	\\
0.859191192790873	-0.00103759765625	\\
0.859235583965908	-0.00103759765625	\\
0.859279975140942	-0.001220703125	\\
0.859324366315976	-0.0013427734375	\\
0.859368757491011	-0.001220703125	\\
0.859413148666045	-0.001007080078125	\\
0.85945753984108	-0.000885009765625	\\
0.859501931016114	-0.0010986328125	\\
0.859546322191148	-0.0010986328125	\\
0.859590713366183	-0.000885009765625	\\
0.859635104541217	-0.000885009765625	\\
0.859679495716252	-0.0013427734375	\\
0.859723886891286	-0.0015869140625	\\
0.85976827806632	-0.001190185546875	\\
0.859812669241355	-0.00091552734375	\\
0.859857060416389	-0.00091552734375	\\
0.859901451591424	-0.000701904296875	\\
0.859945842766458	-0.000762939453125	\\
0.859990233941492	-0.000701904296875	\\
0.860034625116527	-0.000518798828125	\\
0.860079016291561	-0.000823974609375	\\
0.860123407466596	-0.000335693359375	\\
0.86016779864163	0.000152587890625	\\
0.860212189816664	-0.000244140625	\\
0.860256580991699	-0.000244140625	\\
0.860300972166733	-0.00018310546875	\\
0.860345363341768	-0.000244140625	\\
0.860389754516802	-6.103515625e-05	\\
0.860434145691837	-3.0517578125e-05	\\
0.860478536866871	0.00054931640625	\\
0.860522928041905	0.000579833984375	\\
0.86056731921694	0.000213623046875	\\
0.860611710391974	0.00042724609375	\\
0.860656101567008	0.000274658203125	\\
0.860700492742043	0.00018310546875	\\
0.860744883917077	0.000518798828125	\\
0.860789275092112	0.000335693359375	\\
0.860833666267146	9.1552734375e-05	\\
0.86087805744218	0.000152587890625	\\
0.860922448617215	0.0003662109375	\\
0.860966839792249	0.00048828125	\\
0.861011230967284	0.0003662109375	\\
0.861055622142318	0.00048828125	\\
0.861100013317353	0.000335693359375	\\
0.861144404492387	0.000244140625	\\
0.861188795667421	0.00030517578125	\\
0.861233186842456	-0.00018310546875	\\
0.86127757801749	-0.000152587890625	\\
0.861321969192525	-3.0517578125e-05	\\
0.861366360367559	-3.0517578125e-05	\\
0.861410751542593	-0.000152587890625	\\
0.861455142717628	-0.000152587890625	\\
0.861499533892662	0.000244140625	\\
0.861543925067697	9.1552734375e-05	\\
0.861588316242731	-0.000518798828125	\\
0.861632707417765	-0.00091552734375	\\
0.8616770985928	-0.000946044921875	\\
0.861721489767834	-0.000518798828125	\\
0.861765880942869	-0.0001220703125	\\
0.861810272117903	-0.0001220703125	\\
0.861854663292937	3.0517578125e-05	\\
0.861899054467972	0	\\
0.861943445643006	-0.000335693359375	\\
0.861987836818041	-3.0517578125e-05	\\
0.862032227993075	0.000274658203125	\\
0.862076619168109	0.000457763671875	\\
0.862121010343144	0.000335693359375	\\
0.862165401518178	-3.0517578125e-05	\\
0.862209792693213	0.000213623046875	\\
0.862254183868247	0.000244140625	\\
0.862298575043281	-0.000244140625	\\
0.862342966218316	-0.000274658203125	\\
0.86238735739335	-0.00018310546875	\\
0.862431748568385	-3.0517578125e-05	\\
0.862476139743419	-0.0001220703125	\\
0.862520530918453	-0.000213623046875	\\
0.862564922093488	-0.0001220703125	\\
0.862609313268522	0.0001220703125	\\
0.862653704443557	0	\\
0.862698095618591	-0.000213623046875	\\
0.862742486793625	0.00054931640625	\\
0.86278687796866	0.00030517578125	\\
0.862831269143694	-0.000213623046875	\\
0.862875660318729	6.103515625e-05	\\
0.862920051493763	-0.000152587890625	\\
0.862964442668797	-0.000396728515625	\\
0.863008833843832	-0.0006103515625	\\
0.863053225018866	-0.000274658203125	\\
0.863097616193901	-0.000244140625	\\
0.863142007368935	-9.1552734375e-05	\\
0.86318639854397	-9.1552734375e-05	\\
0.863230789719004	-6.103515625e-05	\\
0.863275180894038	-0.000244140625	\\
0.863319572069073	-0.00048828125	\\
0.863363963244107	-0.0003662109375	\\
0.863408354419141	-0.000823974609375	\\
0.863452745594176	-0.000335693359375	\\
0.86349713676921	-3.0517578125e-05	\\
0.863541527944245	-0.000518798828125	\\
0.863585919119279	0.00018310546875	\\
0.863630310294313	0.00018310546875	\\
0.863674701469348	-0.000274658203125	\\
0.863719092644382	0.000213623046875	\\
0.863763483819417	0.00030517578125	\\
0.863807874994451	0.000579833984375	\\
0.863852266169486	0.000640869140625	\\
0.86389665734452	0.000396728515625	\\
0.863941048519554	0.000244140625	\\
0.863985439694589	0.0001220703125	\\
0.864029830869623	0.000213623046875	\\
0.864074222044658	0.000457763671875	\\
0.864118613219692	0.0008544921875	\\
0.864163004394726	0.000518798828125	\\
0.864207395569761	0.000579833984375	\\
0.864251786744795	0.00054931640625	\\
0.864296177919829	0.00067138671875	\\
0.864340569094864	0.000762939453125	\\
0.864384960269898	0.000518798828125	\\
0.864429351444933	0.0006103515625	\\
0.864473742619967	0.00048828125	\\
0.864518133795002	0.001068115234375	\\
0.864562524970036	0.00177001953125	\\
0.86460691614507	0.00146484375	\\
0.864651307320105	0.00177001953125	\\
0.864695698495139	0.001983642578125	\\
0.864740089670174	0.001708984375	\\
0.864784480845208	0.001739501953125	\\
0.864828872020242	0.00164794921875	\\
0.864873263195277	0.00140380859375	\\
0.864917654370311	0.00146484375	\\
0.864962045545346	0.001800537109375	\\
0.86500643672038	0.0020751953125	\\
0.865050827895414	0.001983642578125	\\
0.865095219070449	0.0023193359375	\\
0.865139610245483	0.00244140625	\\
0.865184001420518	0.0023193359375	\\
0.865228392595552	0.00274658203125	\\
0.865272783770586	0.00250244140625	\\
0.865317174945621	0.001953125	\\
0.865361566120655	0.001953125	\\
0.86540595729569	0.001556396484375	\\
0.865450348470724	0.001739501953125	\\
0.865494739645758	0.001983642578125	\\
0.865539130820793	0.00177001953125	\\
0.865583521995827	0.002288818359375	\\
0.865627913170862	0.0025634765625	\\
0.865672304345896	0.002288818359375	\\
0.86571669552093	0.002227783203125	\\
0.865761086695965	0.002410888671875	\\
0.865805477870999	0.00225830078125	\\
0.865849869046034	0.002410888671875	\\
0.865894260221068	0.002593994140625	\\
0.865938651396102	0.002532958984375	\\
0.865983042571137	0.0023193359375	\\
0.866027433746171	0.002227783203125	\\
0.866071824921206	0.002777099609375	\\
0.86611621609624	0.0028076171875	\\
0.866160607271274	0.003173828125	\\
0.866204998446309	0.002960205078125	\\
0.866249389621343	0.002410888671875	\\
0.866293780796378	0.00250244140625	\\
0.866338171971412	0.002716064453125	\\
0.866382563146446	0.0028076171875	\\
0.866426954321481	0.002838134765625	\\
0.866471345496515	0.00262451171875	\\
0.86651573667155	0.002685546875	\\
0.866560127846584	0.0029296875	\\
0.866604519021619	0.002899169921875	\\
0.866648910196653	0.002716064453125	\\
0.866693301371687	0.002899169921875	\\
0.866737692546722	0.002685546875	\\
0.866782083721756	0.0020751953125	\\
0.866826474896791	0.00225830078125	\\
0.866870866071825	0.00250244140625	\\
0.866915257246859	0.002197265625	\\
0.866959648421894	0.0018310546875	\\
0.867004039596928	0.00201416015625	\\
0.867048430771963	0.002227783203125	\\
0.867092821946997	0.002593994140625	\\
0.867137213122031	0.002227783203125	\\
0.867181604297066	0.00213623046875	\\
0.8672259954721	0.0020751953125	\\
0.867270386647135	0.001556396484375	\\
0.867314777822169	0.00079345703125	\\
0.867359168997203	0.001068115234375	\\
0.867403560172238	0.00115966796875	\\
0.867447951347272	0.00091552734375	\\
0.867492342522307	0.00091552734375	\\
0.867536733697341	0.000946044921875	\\
0.867581124872375	0.0010986328125	\\
0.86762551604741	0.000823974609375	\\
0.867669907222444	0.00048828125	\\
0.867714298397479	0.000152587890625	\\
0.867758689572513	-0.000152587890625	\\
0.867803080747547	-0.00018310546875	\\
0.867847471922582	-0.000518798828125	\\
0.867891863097616	-0.00079345703125	\\
0.867936254272651	-0.0008544921875	\\
0.867980645447685	-0.001068115234375	\\
0.868025036622719	-0.001312255859375	\\
0.868069427797754	-0.001068115234375	\\
0.868113818972788	-0.0008544921875	\\
0.868158210147823	-0.00067138671875	\\
0.868202601322857	-0.000640869140625	\\
0.868246992497891	-0.000823974609375	\\
0.868291383672926	-0.00067138671875	\\
0.86833577484796	-0.000701904296875	\\
0.868380166022995	-0.000579833984375	\\
0.868424557198029	-0.000335693359375	\\
0.868468948373063	-0.00018310546875	\\
0.868513339548098	-0.0001220703125	\\
0.868557730723132	-0.000274658203125	\\
0.868602121898167	-0.00030517578125	\\
0.868646513073201	-0.000213623046875	\\
0.868690904248235	0.000274658203125	\\
0.86873529542327	0.000701904296875	\\
0.868779686598304	0.000701904296875	\\
0.868824077773339	0.00128173828125	\\
0.868868468948373	0.001434326171875	\\
0.868912860123408	0.001678466796875	\\
0.868957251298442	0.0013427734375	\\
0.869001642473476	0.0008544921875	\\
0.869046033648511	0.00115966796875	\\
0.869090424823545	0.0010986328125	\\
0.869134815998579	0.000732421875	\\
0.869179207173614	0.0010986328125	\\
0.869223598348648	0.001251220703125	\\
0.869267989523683	0.00128173828125	\\
0.869312380698717	0.001739501953125	\\
0.869356771873751	0.001617431640625	\\
0.869401163048786	0.0013427734375	\\
0.86944555422382	0.0010986328125	\\
0.869489945398855	0.0009765625	\\
0.869534336573889	0.000946044921875	\\
0.869578727748924	0.000396728515625	\\
0.869623118923958	0.000640869140625	\\
0.869667510098992	0.00079345703125	\\
0.869711901274027	0.000396728515625	\\
0.869756292449061	0.000335693359375	\\
0.869800683624096	0.0001220703125	\\
0.86984507479913	0	\\
0.869889465974164	-3.0517578125e-05	\\
0.869933857149199	-0.000274658203125	\\
0.869978248324233	-0.0003662109375	\\
0.870022639499268	-0.000244140625	\\
0.870067030674302	0.000518798828125	\\
0.870111421849336	0.000274658203125	\\
0.870155813024371	0.000213623046875	\\
0.870200204199405	0.00030517578125	\\
0.87024459537444	3.0517578125e-05	\\
0.870288986549474	0.000396728515625	\\
0.870333377724508	0.00030517578125	\\
0.870377768899543	-6.103515625e-05	\\
0.870422160074577	-0.000244140625	\\
0.870466551249612	-0.0003662109375	\\
0.870510942424646	-0.00054931640625	\\
0.87055533359968	-0.000640869140625	\\
0.870599724774715	-0.00048828125	\\
0.870644115949749	9.1552734375e-05	\\
0.870688507124784	0.000152587890625	\\
0.870732898299818	0.000244140625	\\
0.870777289474852	0.000244140625	\\
0.870821680649887	9.1552734375e-05	\\
0.870866071824921	0.000396728515625	\\
0.870910462999956	-3.0517578125e-05	\\
0.87095485417499	-0.000244140625	\\
0.870999245350024	-0.0003662109375	\\
0.871043636525059	-0.00067138671875	\\
0.871088027700093	-0.000396728515625	\\
0.871132418875128	3.0517578125e-05	\\
0.871176810050162	0.00018310546875	\\
0.871221201225196	0.000244140625	\\
0.871265592400231	3.0517578125e-05	\\
0.871309983575265	3.0517578125e-05	\\
0.8713543747503	0.000152587890625	\\
0.871398765925334	0.000152587890625	\\
0.871443157100368	-0.000335693359375	\\
0.871487548275403	-6.103515625e-05	\\
0.871531939450437	0.000335693359375	\\
0.871576330625472	6.103515625e-05	\\
0.871620721800506	3.0517578125e-05	\\
0.871665112975541	0.000244140625	\\
0.871709504150575	0.000396728515625	\\
0.871753895325609	0.000396728515625	\\
0.871798286500644	-6.103515625e-05	\\
0.871842677675678	-0.000213623046875	\\
0.871887068850712	-0.000152587890625	\\
0.871931460025747	-0.000152587890625	\\
0.871975851200781	-0.000152587890625	\\
0.872020242375816	-0.000457763671875	\\
0.87206463355085	-0.000579833984375	\\
0.872109024725884	-0.0006103515625	\\
0.872153415900919	-0.000762939453125	\\
0.872197807075953	-0.00048828125	\\
0.872242198250988	-0.00067138671875	\\
0.872286589426022	-0.000396728515625	\\
0.872330980601057	-0.000518798828125	\\
0.872375371776091	-0.000885009765625	\\
0.872419762951125	-0.0006103515625	\\
0.87246415412616	-0.000640869140625	\\
0.872508545301194	-0.00091552734375	\\
0.872552936476229	-0.00067138671875	\\
0.872597327651263	-0.000396728515625	\\
0.872641718826297	-0.000518798828125	\\
0.872686110001332	-0.0003662109375	\\
0.872730501176366	-6.103515625e-05	\\
0.8727748923514	-9.1552734375e-05	\\
0.872819283526435	-0.000518798828125	\\
0.872863674701469	-0.000579833984375	\\
0.872908065876504	-0.000335693359375	\\
0.872952457051538	-0.000518798828125	\\
0.872996848226573	-0.000244140625	\\
0.873041239401607	-0.00018310546875	\\
0.873085630576641	3.0517578125e-05	\\
0.873130021751676	-9.1552734375e-05	\\
0.87317441292671	-0.000335693359375	\\
0.873218804101745	-0.000213623046875	\\
0.873263195276779	-6.103515625e-05	\\
0.873307586451813	0.0003662109375	\\
0.873351977626848	0.000274658203125	\\
0.873396368801882	0.000244140625	\\
0.873440759976917	0.0001220703125	\\
0.873485151151951	0.00030517578125	\\
0.873529542326985	0.0006103515625	\\
0.87357393350202	0.000152587890625	\\
0.873618324677054	0.00030517578125	\\
0.873662715852089	0.000457763671875	\\
0.873707107027123	0.000244140625	\\
0.873751498202157	0.000244140625	\\
0.873795889377192	0.000244140625	\\
0.873840280552226	0.000274658203125	\\
0.873884671727261	0.000274658203125	\\
0.873929062902295	0.000457763671875	\\
0.873973454077329	0.00054931640625	\\
0.874017845252364	9.1552734375e-05	\\
0.874062236427398	0.000274658203125	\\
0.874106627602433	0.00067138671875	\\
0.874151018777467	0.000396728515625	\\
0.874195409952501	0.00018310546875	\\
0.874239801127536	0.000152587890625	\\
0.87428419230257	0.000213623046875	\\
0.874328583477605	6.103515625e-05	\\
0.874372974652639	0.00048828125	\\
0.874417365827673	0.000732421875	\\
0.874461757002708	0.000457763671875	\\
0.874506148177742	0.000885009765625	\\
0.874550539352777	0.0010986328125	\\
0.874594930527811	0.001190185546875	\\
0.874639321702846	0.001251220703125	\\
0.87468371287788	0.001373291015625	\\
0.874728104052914	0.001708984375	\\
0.874772495227949	0.001495361328125	\\
0.874816886402983	0.00177001953125	\\
0.874861277578017	0.002227783203125	\\
0.874905668753052	0.00177001953125	\\
0.874950059928086	0.00115966796875	\\
0.874994451103121	0.00164794921875	\\
0.875038842278155	0.0018310546875	\\
0.87508323345319	0.001678466796875	\\
0.875127624628224	0.00177001953125	\\
0.875172015803258	0.00189208984375	\\
0.875216406978293	0.00201416015625	\\
0.875260798153327	0.0018310546875	\\
0.875305189328362	0.0018310546875	\\
0.875349580503396	0.001983642578125	\\
0.87539397167843	0.002044677734375	\\
0.875438362853465	0.001678466796875	\\
0.875482754028499	0.001495361328125	\\
0.875527145203534	0.001861572265625	\\
0.875571536378568	0.001708984375	\\
0.875615927553602	0.00146484375	\\
0.875660318728637	0.00140380859375	\\
0.875704709903671	0.001251220703125	\\
0.875749101078706	0.001190185546875	\\
0.87579349225374	0.0010986328125	\\
0.875837883428774	0.00091552734375	\\
0.875882274603809	0.000640869140625	\\
0.875926665778843	0.00042724609375	\\
0.875971056953878	-6.103515625e-05	\\
0.876015448128912	-9.1552734375e-05	\\
0.876059839303946	-0.0003662109375	\\
0.876104230478981	-0.000823974609375	\\
0.876148621654015	-0.00054931640625	\\
0.87619301282905	-0.0003662109375	\\
0.876237404004084	-0.00079345703125	\\
0.876281795179118	-0.000518798828125	\\
0.876326186354153	-0.0003662109375	\\
0.876370577529187	-0.00091552734375	\\
0.876414968704222	-0.0008544921875	\\
0.876459359879256	-0.000885009765625	\\
0.87650375105429	-0.0006103515625	\\
0.876548142229325	-0.000823974609375	\\
0.876592533404359	-0.001495361328125	\\
0.876636924579394	-0.001312255859375	\\
0.876681315754428	-0.00103759765625	\\
0.876725706929462	-0.001434326171875	\\
0.876770098104497	-0.001434326171875	\\
0.876814489279531	-0.001190185546875	\\
0.876858880454566	-0.001556396484375	\\
0.8769032716296	-0.001708984375	\\
0.876947662804634	-0.001068115234375	\\
0.876992053979669	-0.001068115234375	\\
0.877036445154703	-0.0013427734375	\\
0.877080836329738	-0.000946044921875	\\
0.877125227504772	-0.00115966796875	\\
0.877169618679806	-0.001434326171875	\\
0.877214009854841	-0.001495361328125	\\
0.877258401029875	-0.001617431640625	\\
0.87730279220491	-0.001953125	\\
0.877347183379944	-0.002105712890625	\\
0.877391574554979	-0.001800537109375	\\
0.877435965730013	-0.001495361328125	\\
0.877480356905047	-0.001190185546875	\\
0.877524748080082	-0.001495361328125	\\
0.877569139255116	-0.001922607421875	\\
0.87761353043015	-0.00152587890625	\\
0.877657921605185	-0.00146484375	\\
0.877702312780219	-0.001800537109375	\\
0.877746703955254	-0.001800537109375	\\
0.877791095130288	-0.002227783203125	\\
0.877835486305322	-0.00201416015625	\\
0.877879877480357	-0.00152587890625	\\
0.877924268655391	-0.00189208984375	\\
0.877968659830426	-0.001922607421875	\\
0.87801305100546	-0.00146484375	\\
0.878057442180495	-0.001129150390625	\\
0.878101833355529	-0.001220703125	\\
0.878146224530563	-0.00140380859375	\\
0.878190615705598	-0.000946044921875	\\
0.878235006880632	-0.00042724609375	\\
0.878279398055667	-0.000701904296875	\\
0.878323789230701	-0.0006103515625	\\
0.878368180405735	-0.000335693359375	\\
0.87841257158077	-0.00018310546875	\\
0.878456962755804	0.00042724609375	\\
0.878501353930839	0.000701904296875	\\
0.878545745105873	0.000213623046875	\\
0.878590136280907	0.00054931640625	\\
0.878634527455942	0.000640869140625	\\
0.878678918630976	0.000579833984375	\\
0.878723309806011	0.000579833984375	\\
0.878767700981045	3.0517578125e-05	\\
0.878812092156079	-3.0517578125e-05	\\
0.878856483331114	0.000640869140625	\\
0.878900874506148	0.000579833984375	\\
0.878945265681183	0.00091552734375	\\
0.878989656856217	0.00091552734375	\\
0.879034048031251	0.00091552734375	\\
0.879078439206286	0.00091552734375	\\
0.87912283038132	0.000885009765625	\\
0.879167221556355	0.000518798828125	\\
0.879211612731389	0.000457763671875	\\
0.879256003906423	0.000457763671875	\\
0.879300395081458	0.000457763671875	\\
0.879344786256492	0.000396728515625	\\
0.879389177431527	3.0517578125e-05	\\
0.879433568606561	0.000213623046875	\\
0.879477959781595	-3.0517578125e-05	\\
0.87952235095663	0.00042724609375	\\
0.879566742131664	0.00030517578125	\\
0.879611133306699	-0.00030517578125	\\
0.879655524481733	-0.000213623046875	\\
0.879699915656767	-0.000274658203125	\\
0.879744306831802	-0.000518798828125	\\
0.879788698006836	-0.000518798828125	\\
0.879833089181871	-0.000213623046875	\\
0.879877480356905	-0.000762939453125	\\
0.879921871531939	-0.001129150390625	\\
0.879966262706974	-0.00115966796875	\\
0.880010653882008	-0.001068115234375	\\
0.880055045057043	-0.000823974609375	\\
0.880099436232077	-0.001068115234375	\\
0.880143827407112	-0.001190185546875	\\
0.880188218582146	-0.001220703125	\\
0.88023260975718	-0.0013427734375	\\
0.880277000932215	-0.001312255859375	\\
0.880321392107249	-0.00103759765625	\\
0.880365783282283	-0.0006103515625	\\
0.880410174457318	-0.0010986328125	\\
0.880454565632352	-0.0010986328125	\\
0.880498956807387	-0.001068115234375	\\
0.880543347982421	-0.001373291015625	\\
0.880587739157455	-0.001312255859375	\\
0.88063213033249	-0.001373291015625	\\
0.880676521507524	-0.001007080078125	\\
0.880720912682559	-0.000457763671875	\\
0.880765303857593	-0.0003662109375	\\
0.880809695032628	-0.000518798828125	\\
0.880854086207662	-0.00054931640625	\\
0.880898477382696	-0.000335693359375	\\
0.880942868557731	-0.00018310546875	\\
0.880987259732765	-6.103515625e-05	\\
0.8810316509078	0.0003662109375	\\
0.881076042082834	0.00054931640625	\\
0.881120433257868	0.000335693359375	\\
0.881164824432903	0.000640869140625	\\
0.881209215607937	0.0009765625	\\
0.881253606782972	0.00128173828125	\\
0.881297997958006	0.001312255859375	\\
0.88134238913304	0.001373291015625	\\
0.881386780308075	0.00213623046875	\\
0.881431171483109	0.00189208984375	\\
0.881475562658144	0.001800537109375	\\
0.881519953833178	0.002227783203125	\\
0.881564345008212	0.0020751953125	\\
0.881608736183247	0.001800537109375	\\
0.881653127358281	0.0018310546875	\\
0.881697518533316	0.001953125	\\
0.88174190970835	0.002044677734375	\\
0.881786300883384	0.00213623046875	\\
0.881830692058419	0.00238037109375	\\
0.881875083233453	0.002960205078125	\\
0.881919474408488	0.0029296875	\\
0.881963865583522	0.00286865234375	\\
0.882008256758556	0.00299072265625	\\
0.882052647933591	0.002685546875	\\
0.882097039108625	0.002899169921875	\\
0.88214143028366	0.002838134765625	\\
0.882185821458694	0.002716064453125	\\
0.882230212633728	0.002685546875	\\
0.882274603808763	0.00274658203125	\\
0.882318994983797	0.002349853515625	\\
0.882363386158832	0.002105712890625	\\
0.882407777333866	0.001922607421875	\\
0.8824521685089	0.002197265625	\\
0.882496559683935	0.0023193359375	\\
0.882540950858969	0.0025634765625	\\
0.882585342034004	0.003082275390625	\\
0.882629733209038	0.002655029296875	\\
0.882674124384072	0.002471923828125	\\
0.882718515559107	0.00262451171875	\\
0.882762906734141	0.002288818359375	\\
0.882807297909176	0.002532958984375	\\
0.88285168908421	0.002777099609375	\\
0.882896080259244	0.002532958984375	\\
0.882940471434279	0.002593994140625	\\
0.882984862609313	0.00244140625	\\
0.883029253784348	0.002288818359375	\\
0.883073644959382	0.002288818359375	\\
0.883118036134417	0.00225830078125	\\
0.883162427309451	0.002227783203125	\\
0.883206818484485	0.00164794921875	\\
0.88325120965952	0.001800537109375	\\
0.883295600834554	0.002410888671875	\\
0.883339992009588	0.002655029296875	\\
0.883384383184623	0.0030517578125	\\
0.883428774359657	0.00311279296875	\\
0.883473165534692	0.002655029296875	\\
0.883517556709726	0.002655029296875	\\
0.883561947884761	0.00262451171875	\\
0.883606339059795	0.0029296875	\\
0.883650730234829	0.00274658203125	\\
0.883695121409864	0.0023193359375	\\
0.883739512584898	0.002685546875	\\
0.883783903759933	0.002471923828125	\\
0.883828294934967	0.0023193359375	\\
0.883872686110001	0.002471923828125	\\
0.883917077285036	0.00213623046875	\\
0.88396146846007	0.002349853515625	\\
0.884005859635105	0.0025634765625	\\
0.884050250810139	0.002288818359375	\\
0.884094641985173	0.002166748046875	\\
0.884139033160208	0.001953125	\\
0.884183424335242	0.001983642578125	\\
0.884227815510277	0.00201416015625	\\
0.884272206685311	0.00225830078125	\\
0.884316597860345	0.00164794921875	\\
0.88436098903538	0.0018310546875	\\
0.884405380210414	0.0015869140625	\\
0.884449771385449	0.0015869140625	\\
0.884494162560483	0.002197265625	\\
0.884538553735517	0.001953125	\\
0.884582944910552	0.00189208984375	\\
0.884627336085586	0.001617431640625	\\
0.884671727260621	0.00152587890625	\\
0.884716118435655	0.0015869140625	\\
0.884760509610689	0.001373291015625	\\
0.884804900785724	0.00128173828125	\\
0.884849291960758	0.001373291015625	\\
0.884893683135793	0.00079345703125	\\
0.884938074310827	0.001312255859375	\\
0.884982465485861	0.001495361328125	\\
0.885026856660896	0.001251220703125	\\
0.88507124783593	0.001220703125	\\
0.885115639010965	0.00091552734375	\\
0.885160030185999	0.00091552734375	\\
0.885204421361033	0.000823974609375	\\
0.885248812536068	0.00067138671875	\\
0.885293203711102	0.0008544921875	\\
0.885337594886137	0.001068115234375	\\
0.885381986061171	0.0010986328125	\\
0.885426377236205	0.00103759765625	\\
0.88547076841124	0.000579833984375	\\
0.885515159586274	0.000152587890625	\\
0.885559550761309	9.1552734375e-05	\\
0.885603941936343	0.0001220703125	\\
0.885648333111377	3.0517578125e-05	\\
0.885692724286412	0.00018310546875	\\
0.885737115461446	0.000396728515625	\\
0.885781506636481	-0.00018310546875	\\
0.885825897811515	-0.000152587890625	\\
0.88587028898655	0.00048828125	\\
0.885914680161584	0.00067138671875	\\
0.885959071336618	0.0003662109375	\\
0.886003462511653	0.000335693359375	\\
0.886047853686687	0.00030517578125	\\
0.886092244861721	0.00042724609375	\\
0.886136636036756	0.000518798828125	\\
0.88618102721179	0.000274658203125	\\
0.886225418386825	0.000335693359375	\\
0.886269809561859	0.000732421875	\\
0.886314200736893	0.000762939453125	\\
0.886358591911928	0.000244140625	\\
0.886402983086962	0.000457763671875	\\
0.886447374261997	0.000640869140625	\\
0.886491765437031	0.00079345703125	\\
0.886536156612066	0.000946044921875	\\
0.8865805477871	0.000732421875	\\
0.886624938962134	0.00054931640625	\\
0.886669330137169	0.000335693359375	\\
0.886713721312203	0.00042724609375	\\
0.886758112487238	0.000762939453125	\\
0.886802503662272	0.000732421875	\\
0.886846894837306	0.000762939453125	\\
0.886891286012341	0.001068115234375	\\
0.886935677187375	0.0010986328125	\\
0.88698006836241	0.001007080078125	\\
0.887024459537444	0.000457763671875	\\
0.887068850712478	0.00042724609375	\\
0.887113241887513	0.000885009765625	\\
0.887157633062547	0.000732421875	\\
0.887202024237582	0.0006103515625	\\
0.887246415412616	0.0010986328125	\\
0.88729080658765	0.00103759765625	\\
0.887335197762685	0.000762939453125	\\
0.887379588937719	0.000762939453125	\\
0.887423980112754	0.000762939453125	\\
0.887468371287788	0.000701904296875	\\
0.887512762462822	0.000732421875	\\
0.887557153637857	0.00091552734375	\\
0.887601544812891	0.0010986328125	\\
0.887645935987926	0.000762939453125	\\
0.88769032716296	0.00067138671875	\\
0.887734718337994	0.00091552734375	\\
0.887779109513029	0.001068115234375	\\
0.887823500688063	0.001007080078125	\\
};
\addplot [color=blue,solid,forget plot]
  table[row sep=crcr]{
0.887823500688063	0.001007080078125	\\
0.887867891863098	0.001312255859375	\\
0.887912283038132	0.00140380859375	\\
0.887956674213166	0.0013427734375	\\
0.888001065388201	0.00146484375	\\
0.888045456563235	0.00128173828125	\\
0.88808984773827	0.00140380859375	\\
0.888134238913304	0.001556396484375	\\
0.888178630088338	0.001556396484375	\\
0.888223021263373	0.000946044921875	\\
0.888267412438407	0.00140380859375	\\
0.888311803613442	0.001861572265625	\\
0.888356194788476	0.001373291015625	\\
0.88840058596351	0.001708984375	\\
0.888444977138545	0.001800537109375	\\
0.888489368313579	0.001495361328125	\\
0.888533759488614	0.001708984375	\\
0.888578150663648	0.00189208984375	\\
0.888622541838683	0.001861572265625	\\
0.888666933013717	0.001495361328125	\\
0.888711324188751	0.001708984375	\\
0.888755715363786	0.002288818359375	\\
0.88880010653882	0.0018310546875	\\
0.888844497713854	0.0018310546875	\\
0.888888888888889	0.00201416015625	\\
0.888933280063923	0.002105712890625	\\
0.888977671238958	0.001678466796875	\\
0.889022062413992	0.0010986328125	\\
0.889066453589026	0.001129150390625	\\
0.889110844764061	0.000885009765625	\\
0.889155235939095	0.000518798828125	\\
0.88919962711413	0.00030517578125	\\
0.889244018289164	0.000640869140625	\\
0.889288409464199	0.000946044921875	\\
0.889332800639233	0.000885009765625	\\
0.889377191814267	0.00079345703125	\\
0.889421582989302	0.000946044921875	\\
0.889465974164336	0.001251220703125	\\
0.889510365339371	0.001434326171875	\\
0.889554756514405	0.001129150390625	\\
0.889599147689439	0.00115966796875	\\
0.889643538864474	0.0010986328125	\\
0.889687930039508	0.00091552734375	\\
0.889732321214543	0.001556396484375	\\
0.889776712389577	0.00177001953125	\\
0.889821103564611	0.001312255859375	\\
0.889865494739646	0.001220703125	\\
0.88990988591468	0.00164794921875	\\
0.889954277089715	0.001983642578125	\\
0.889998668264749	0.0020751953125	\\
0.890043059439783	0.002288818359375	\\
0.890087450614818	0.002227783203125	\\
0.890131841789852	0.00201416015625	\\
0.890176232964887	0.00213623046875	\\
0.890220624139921	0.0023193359375	\\
0.890265015314955	0.002593994140625	\\
0.89030940648999	0.00244140625	\\
0.890353797665024	0.00244140625	\\
0.890398188840059	0.002685546875	\\
0.890442580015093	0.002655029296875	\\
0.890486971190127	0.002899169921875	\\
0.890531362365162	0.0029296875	\\
0.890575753540196	0.00274658203125	\\
0.890620144715231	0.002288818359375	\\
0.890664535890265	0.002166748046875	\\
0.890708927065299	0.0020751953125	\\
0.890753318240334	0.00189208984375	\\
0.890797709415368	0.001953125	\\
0.890842100590403	0.001922607421875	\\
0.890886491765437	0.00238037109375	\\
0.890930882940471	0.002655029296875	\\
0.890975274115506	0.002685546875	\\
0.89101966529054	0.0028076171875	\\
0.891064056465575	0.002777099609375	\\
0.891108447640609	0.00274658203125	\\
0.891152838815643	0.002838134765625	\\
0.891197229990678	0.002777099609375	\\
0.891241621165712	0.002593994140625	\\
0.891286012340747	0.002471923828125	\\
0.891330403515781	0.0028076171875	\\
0.891374794690815	0.002655029296875	\\
0.89141918586585	0.002685546875	\\
0.891463577040884	0.002838134765625	\\
0.891507968215919	0.0025634765625	\\
0.891552359390953	0.002777099609375	\\
0.891596750565988	0.00250244140625	\\
0.891641141741022	0.00213623046875	\\
0.891685532916056	0.00225830078125	\\
0.891729924091091	0.002197265625	\\
0.891774315266125	0.002166748046875	\\
0.891818706441159	0.001861572265625	\\
0.891863097616194	0.00201416015625	\\
0.891907488791228	0.0023193359375	\\
0.891951879966263	0.002044677734375	\\
0.891996271141297	0.001739501953125	\\
0.892040662316332	0.001556396484375	\\
0.892085053491366	0.00146484375	\\
0.8921294446664	0.001495361328125	\\
0.892173835841435	0.001068115234375	\\
0.892218227016469	0.001251220703125	\\
0.892262618191504	0.0009765625	\\
0.892307009366538	0.000732421875	\\
0.892351400541572	0.00128173828125	\\
0.892395791716607	0.00091552734375	\\
0.892440182891641	0.000732421875	\\
0.892484574066676	0.0003662109375	\\
0.89252896524171	0.00018310546875	\\
0.892573356416744	0.00030517578125	\\
0.892617747591779	0.00018310546875	\\
0.892662138766813	0.0003662109375	\\
0.892706529941848	0.0001220703125	\\
0.892750921116882	0.00048828125	\\
0.892795312291916	0.00042724609375	\\
0.892839703466951	0.00018310546875	\\
0.892884094641985	0.000213623046875	\\
0.89292848581702	-0.00054931640625	\\
0.892972876992054	0	\\
0.893017268167088	0.00048828125	\\
0.893061659342123	0.000244140625	\\
0.893106050517157	6.103515625e-05	\\
0.893150441692192	6.103515625e-05	\\
0.893194832867226	0.0001220703125	\\
0.89323922404226	0.000274658203125	\\
0.893283615217295	9.1552734375e-05	\\
0.893328006392329	0.00018310546875	\\
0.893372397567364	0.000518798828125	\\
0.893416788742398	0.000518798828125	\\
0.893461179917432	0.000274658203125	\\
0.893505571092467	0	\\
0.893549962267501	0.000213623046875	\\
0.893594353442536	0.00018310546875	\\
0.89363874461757	0	\\
0.893683135792604	-9.1552734375e-05	\\
0.893727526967639	-0.000335693359375	\\
0.893771918142673	0.00018310546875	\\
0.893816309317708	0.00030517578125	\\
0.893860700492742	0.000213623046875	\\
0.893905091667776	0.000274658203125	\\
0.893949482842811	3.0517578125e-05	\\
0.893993874017845	0.0001220703125	\\
0.89403826519288	0.000244140625	\\
0.894082656367914	0.000885009765625	\\
0.894127047542948	0.00067138671875	\\
0.894171438717983	0.00018310546875	\\
0.894215829893017	0.00030517578125	\\
0.894260221068052	0.00048828125	\\
0.894304612243086	0.000244140625	\\
0.894349003418121	-6.103515625e-05	\\
0.894393394593155	0.000244140625	\\
0.894437785768189	0.000274658203125	\\
0.894482176943224	0	\\
0.894526568118258	9.1552734375e-05	\\
0.894570959293292	0.000152587890625	\\
0.894615350468327	0.00018310546875	\\
0.894659741643361	9.1552734375e-05	\\
0.894704132818396	-0.00030517578125	\\
0.89474852399343	-0.0001220703125	\\
0.894792915168464	-0.00030517578125	\\
0.894837306343499	-0.0008544921875	\\
0.894881697518533	-0.00048828125	\\
0.894926088693568	-0.0006103515625	\\
0.894970479868602	-0.000457763671875	\\
0.895014871043637	-0.000274658203125	\\
0.895059262218671	-0.0003662109375	\\
0.895103653393705	-0.000640869140625	\\
0.89514804456874	-0.000701904296875	\\
0.895192435743774	-9.1552734375e-05	\\
0.895236826918809	0	\\
0.895281218093843	-0.000518798828125	\\
0.895325609268877	-0.00067138671875	\\
0.895370000443912	-0.000701904296875	\\
0.895414391618946	-0.00079345703125	\\
0.895458782793981	-0.000640869140625	\\
0.895503173969015	-0.000518798828125	\\
0.895547565144049	-0.000640869140625	\\
0.895591956319084	-0.00042724609375	\\
0.895636347494118	-0.00048828125	\\
0.895680738669153	-0.00054931640625	\\
0.895725129844187	-0.000213623046875	\\
0.895769521019221	-6.103515625e-05	\\
0.895813912194256	-0.000213623046875	\\
0.89585830336929	-0.00042724609375	\\
0.895902694544325	-0.00018310546875	\\
0.895947085719359	-0.0001220703125	\\
0.895991476894393	-0.000396728515625	\\
0.896035868069428	3.0517578125e-05	\\
0.896080259244462	0.0006103515625	\\
0.896124650419497	0.000396728515625	\\
0.896169041594531	-6.103515625e-05	\\
0.896213432769565	-0.000244140625	\\
0.8962578239446	-3.0517578125e-05	\\
0.896302215119634	0.000244140625	\\
0.896346606294669	9.1552734375e-05	\\
0.896390997469703	-0.000152587890625	\\
0.896435388644737	-0.000152587890625	\\
0.896479779819772	-0.0003662109375	\\
0.896524170994806	-0.00042724609375	\\
0.896568562169841	-6.103515625e-05	\\
0.896612953344875	0.0001220703125	\\
0.896657344519909	-3.0517578125e-05	\\
0.896701735694944	-0.000152587890625	\\
0.896746126869978	3.0517578125e-05	\\
0.896790518045013	3.0517578125e-05	\\
0.896834909220047	-0.0003662109375	\\
0.896879300395081	9.1552734375e-05	\\
0.896923691570116	0.000579833984375	\\
0.89696808274515	0.000396728515625	\\
0.897012473920185	6.103515625e-05	\\
0.897056865095219	0.0001220703125	\\
0.897101256270254	3.0517578125e-05	\\
0.897145647445288	-0.000213623046875	\\
0.897190038620322	-0.000244140625	\\
0.897234429795357	-0.0003662109375	\\
0.897278820970391	0	\\
0.897323212145426	0	\\
0.89736760332046	3.0517578125e-05	\\
0.897411994495494	-0.000274658203125	\\
0.897456385670529	-0.000518798828125	\\
0.897500776845563	-0.0001220703125	\\
0.897545168020597	-3.0517578125e-05	\\
0.897589559195632	0.000274658203125	\\
0.897633950370666	9.1552734375e-05	\\
0.897678341545701	9.1552734375e-05	\\
0.897722732720735	0.000152587890625	\\
0.89776712389577	0.000213623046875	\\
0.897811515070804	0.000701904296875	\\
0.897855906245838	0.001220703125	\\
0.897900297420873	0.001068115234375	\\
0.897944688595907	0.000823974609375	\\
0.897989079770942	0.001068115234375	\\
0.898033470945976	0.00091552734375	\\
0.89807786212101	0.000701904296875	\\
0.898122253296045	0.00128173828125	\\
0.898166644471079	0.00146484375	\\
0.898211035646114	0.0013427734375	\\
0.898255426821148	0.001434326171875	\\
0.898299817996182	0.00115966796875	\\
0.898344209171217	0.0010986328125	\\
0.898388600346251	0.001800537109375	\\
0.898432991521286	0.001922607421875	\\
0.89847738269632	0.00164794921875	\\
0.898521773871354	0.001922607421875	\\
0.898566165046389	0.001739501953125	\\
0.898610556221423	0.001373291015625	\\
0.898654947396458	0.00164794921875	\\
0.898699338571492	0.00164794921875	\\
0.898743729746526	0.0010986328125	\\
0.898788120921561	0.00128173828125	\\
0.898832512096595	0.0010986328125	\\
0.89887690327163	0.00091552734375	\\
0.898921294446664	0.0013427734375	\\
0.898965685621698	0.001953125	\\
0.899010076796733	0.002105712890625	\\
0.899054467971767	0.0020751953125	\\
0.899098859146802	0.001953125	\\
0.899143250321836	0.002349853515625	\\
0.89918764149687	0.002593994140625	\\
0.899232032671905	0.0015869140625	\\
0.899276423846939	0.00177001953125	\\
0.899320815021974	0.001983642578125	\\
0.899365206197008	0.002044677734375	\\
0.899409597372042	0.001922607421875	\\
0.899453988547077	0.002288818359375	\\
0.899498379722111	0.0023193359375	\\
0.899542770897146	0.002044677734375	\\
0.89958716207218	0.002105712890625	\\
0.899631553247214	0.001861572265625	\\
0.899675944422249	0.001983642578125	\\
0.899720335597283	0.002166748046875	\\
0.899764726772318	0.001800537109375	\\
0.899809117947352	0.002044677734375	\\
0.899853509122386	0.002105712890625	\\
0.899897900297421	0.002197265625	\\
0.899942291472455	0.002166748046875	\\
0.89998668264749	0.001678466796875	\\
0.900031073822524	0.0018310546875	\\
0.900075464997559	0.002166748046875	\\
0.900119856172593	0.0020751953125	\\
0.900164247347627	0.002227783203125	\\
0.900208638522662	0.002349853515625	\\
0.900253029697696	0.002288818359375	\\
0.90029742087273	0.0020751953125	\\
0.900341812047765	0.001800537109375	\\
0.900386203222799	0.00152587890625	\\
0.900430594397834	0.001190185546875	\\
0.900474985572868	0.001312255859375	\\
0.900519376747903	0.001220703125	\\
0.900563767922937	0.00091552734375	\\
0.900608159097971	0.0013427734375	\\
0.900652550273006	0.001556396484375	\\
0.90069694144804	0.0013427734375	\\
0.900741332623075	0.001220703125	\\
0.900785723798109	0.001220703125	\\
0.900830114973143	0.001251220703125	\\
0.900874506148178	0.001861572265625	\\
0.900918897323212	0.00164794921875	\\
0.900963288498247	0.000946044921875	\\
0.901007679673281	0.000823974609375	\\
0.901052070848315	0.000885009765625	\\
0.90109646202335	0.001251220703125	\\
0.901140853198384	0.001129150390625	\\
0.901185244373419	0.000732421875	\\
0.901229635548453	0.00091552734375	\\
0.901274026723487	0.00091552734375	\\
0.901318417898522	0.000885009765625	\\
0.901362809073556	0.000732421875	\\
0.901407200248591	0.000885009765625	\\
0.901451591423625	0.00067138671875	\\
0.901495982598659	0.000640869140625	\\
0.901540373773694	0.000274658203125	\\
0.901584764948728	0	\\
0.901629156123763	0.000396728515625	\\
0.901673547298797	-6.103515625e-05	\\
0.901717938473831	9.1552734375e-05	\\
0.901762329648866	0	\\
0.9018067208239	-0.000396728515625	\\
0.901851111998935	-0.00042724609375	\\
0.901895503173969	-0.00048828125	\\
0.901939894349003	-3.0517578125e-05	\\
0.901984285524038	0.0001220703125	\\
0.902028676699072	0.000457763671875	\\
0.902073067874107	0.0006103515625	\\
0.902117459049141	0.00067138671875	\\
0.902161850224175	0.000396728515625	\\
0.90220624139921	-0.000244140625	\\
0.902250632574244	0.000213623046875	\\
0.902295023749279	0.00018310546875	\\
0.902339414924313	0.00018310546875	\\
0.902383806099347	0.000457763671875	\\
0.902428197274382	0.00048828125	\\
0.902472588449416	0.0006103515625	\\
0.902516979624451	0.00048828125	\\
0.902561370799485	0.000457763671875	\\
0.902605761974519	0.0006103515625	\\
0.902650153149554	0.000732421875	\\
0.902694544324588	0.00079345703125	\\
0.902738935499623	0.000518798828125	\\
0.902783326674657	0.000518798828125	\\
0.902827717849692	0.00054931640625	\\
0.902872109024726	0.001434326171875	\\
0.90291650019976	0.001251220703125	\\
0.902960891374795	0.0008544921875	\\
0.903005282549829	0.00115966796875	\\
0.903049673724863	0.0009765625	\\
0.903094064899898	0.001007080078125	\\
0.903138456074932	0.0013427734375	\\
0.903182847249967	0.00128173828125	\\
0.903227238425001	0.001190185546875	\\
0.903271629600035	0.0013427734375	\\
0.90331602077507	0.001129150390625	\\
0.903360411950104	0.0015869140625	\\
0.903404803125139	0.0018310546875	\\
0.903449194300173	0.00152587890625	\\
0.903493585475208	0.001739501953125	\\
0.903537976650242	0.001556396484375	\\
0.903582367825276	0.001556396484375	\\
0.903626759000311	0.001800537109375	\\
0.903671150175345	0.00146484375	\\
0.90371554135038	0.001251220703125	\\
0.903759932525414	0.00128173828125	\\
0.903804323700448	0.0015869140625	\\
0.903848714875483	0.0018310546875	\\
0.903893106050517	0.001495361328125	\\
0.903937497225552	0.001068115234375	\\
0.903981888400586	0.00115966796875	\\
0.90402627957562	0.001220703125	\\
0.904070670750655	0.00128173828125	\\
0.904115061925689	0.0015869140625	\\
0.904159453100724	0.00152587890625	\\
0.904203844275758	0.00128173828125	\\
0.904248235450792	0.001373291015625	\\
0.904292626625827	0.001312255859375	\\
0.904337017800861	0.001312255859375	\\
0.904381408975896	0.001190185546875	\\
0.90442580015093	0.001434326171875	\\
0.904470191325964	0.00115966796875	\\
0.904514582500999	0.00128173828125	\\
0.904558973676033	0.001312255859375	\\
0.904603364851068	0.001251220703125	\\
0.904647756026102	0.000823974609375	\\
0.904692147201136	0.000518798828125	\\
0.904736538376171	0.000732421875	\\
0.904780929551205	0.00103759765625	\\
0.90482532072624	0.00103759765625	\\
0.904869711901274	0.001068115234375	\\
0.904914103076308	0.001220703125	\\
0.904958494251343	0.00164794921875	\\
0.905002885426377	0.001190185546875	\\
0.905047276601412	0.001007080078125	\\
0.905091667776446	0.00146484375	\\
0.90513605895148	0.00146484375	\\
0.905180450126515	0.001373291015625	\\
0.905224841301549	0.00103759765625	\\
0.905269232476584	0.001007080078125	\\
0.905313623651618	0.00146484375	\\
0.905358014826652	0.001495361328125	\\
0.905402406001687	0.0013427734375	\\
0.905446797176721	0.001312255859375	\\
0.905491188351756	0.00128173828125	\\
0.90553557952679	0.00177001953125	\\
0.905579970701825	0.001983642578125	\\
0.905624361876859	0.002166748046875	\\
0.905668753051893	0.002166748046875	\\
0.905713144226928	0.001953125	\\
0.905757535401962	0.001739501953125	\\
0.905801926576997	0.0020751953125	\\
0.905846317752031	0.00225830078125	\\
0.905890708927065	0.00213623046875	\\
0.9059351001021	0.002227783203125	\\
0.905979491277134	0.0020751953125	\\
0.906023882452168	0.00262451171875	\\
0.906068273627203	0.00341796875	\\
0.906112664802237	0.0032958984375	\\
0.906157055977272	0.003326416015625	\\
0.906201447152306	0.0029296875	\\
0.906245838327341	0.002716064453125	\\
0.906290229502375	0.002899169921875	\\
0.906334620677409	0.0025634765625	\\
0.906379011852444	0.00323486328125	\\
0.906423403027478	0.003265380859375	\\
0.906467794202513	0.003143310546875	\\
0.906512185377547	0.003448486328125	\\
0.906556576552581	0.003143310546875	\\
0.906600967727616	0.002655029296875	\\
0.90664535890265	0.00286865234375	\\
0.906689750077685	0.002960205078125	\\
0.906734141252719	0.002685546875	\\
0.906778532427753	0.002716064453125	\\
0.906822923602788	0.00274658203125	\\
0.906867314777822	0.0025634765625	\\
0.906911705952857	0.002471923828125	\\
0.906956097127891	0.002044677734375	\\
0.907000488302925	0.00201416015625	\\
0.90704487947796	0.002349853515625	\\
0.907089270652994	0.001800537109375	\\
0.907133661828029	0.001983642578125	\\
0.907178053003063	0.00213623046875	\\
0.907222444178097	0.001861572265625	\\
0.907266835353132	0.001739501953125	\\
0.907311226528166	0.002105712890625	\\
0.907355617703201	0.001983642578125	\\
0.907400008878235	0.00177001953125	\\
0.907444400053269	0.002044677734375	\\
0.907488791228304	0.002044677734375	\\
0.907533182403338	0.002288818359375	\\
0.907577573578373	0.00250244140625	\\
0.907621964753407	0.00238037109375	\\
0.907666355928441	0.002105712890625	\\
0.907710747103476	0.00262451171875	\\
0.90775513827851	0.002777099609375	\\
0.907799529453545	0.00250244140625	\\
0.907843920628579	0.00274658203125	\\
0.907888311803613	0.0030517578125	\\
0.907932702978648	0.00286865234375	\\
0.907977094153682	0.00274658203125	\\
0.908021485328717	0.002960205078125	\\
0.908065876503751	0.0032958984375	\\
0.908110267678785	0.003082275390625	\\
0.90815465885382	0.0029296875	\\
0.908199050028854	0.0030517578125	\\
0.908243441203889	0.0032958984375	\\
0.908287832378923	0.00323486328125	\\
0.908332223553957	0.0029296875	\\
0.908376614728992	0.003143310546875	\\
0.908421005904026	0.003143310546875	\\
0.908465397079061	0.003021240234375	\\
0.908509788254095	0.002716064453125	\\
0.90855417942913	0.003204345703125	\\
0.908598570604164	0.003082275390625	\\
0.908642961779198	0.00311279296875	\\
0.908687352954233	0.00347900390625	\\
0.908731744129267	0.00323486328125	\\
0.908776135304301	0.002899169921875	\\
0.908820526479336	0.00360107421875	\\
0.90886491765437	0.004058837890625	\\
0.908909308829405	0.00341796875	\\
0.908953700004439	0.003143310546875	\\
0.908998091179474	0.002593994140625	\\
0.909042482354508	0.00238037109375	\\
0.909086873529542	0.002655029296875	\\
0.909131264704577	0.0023193359375	\\
0.909175655879611	0.00225830078125	\\
0.909220047054646	0.002105712890625	\\
0.90926443822968	0.001678466796875	\\
0.909308829404714	0.002410888671875	\\
0.909353220579749	0.0025634765625	\\
0.909397611754783	0.00238037109375	\\
0.909442002929818	0.002655029296875	\\
0.909486394104852	0.002685546875	\\
0.909530785279886	0.002166748046875	\\
0.909575176454921	0.001312255859375	\\
0.909619567629955	0.001556396484375	\\
0.90966395880499	0.001861572265625	\\
0.909708349980024	0.00164794921875	\\
0.909752741155058	0.001739501953125	\\
0.909797132330093	0.00140380859375	\\
0.909841523505127	0.001129150390625	\\
0.909885914680162	0.0018310546875	\\
0.909930305855196	0.0018310546875	\\
0.90997469703023	0.00164794921875	\\
0.910019088205265	0.001373291015625	\\
0.910063479380299	0.000946044921875	\\
0.910107870555334	0.00103759765625	\\
0.910152261730368	0.000823974609375	\\
0.910196652905402	0.00079345703125	\\
0.910241044080437	0.001068115234375	\\
0.910285435255471	0.00103759765625	\\
0.910329826430506	0.00048828125	\\
0.91037421760554	0.00067138671875	\\
0.910418608780574	0.001190185546875	\\
0.910462999955609	0.00079345703125	\\
0.910507391130643	0.000457763671875	\\
0.910551782305678	0.000823974609375	\\
0.910596173480712	0.0008544921875	\\
0.910640564655746	0.000274658203125	\\
0.910684955830781	6.103515625e-05	\\
0.910729347005815	3.0517578125e-05	\\
0.91077373818085	-0.00048828125	\\
0.910818129355884	-0.000213623046875	\\
0.910862520530918	-0.000335693359375	\\
0.910906911705953	-0.000396728515625	\\
0.910951302880987	-0.000274658203125	\\
0.910995694056022	-0.00079345703125	\\
0.911040085231056	-0.0006103515625	\\
0.91108447640609	-0.00042724609375	\\
0.911128867581125	-0.00054931640625	\\
0.911173258756159	-0.0008544921875	\\
0.911217649931194	-0.000518798828125	\\
0.911262041106228	-0.000335693359375	\\
0.911306432281263	-0.000457763671875	\\
0.911350823456297	-0.00067138671875	\\
0.911395214631331	-0.001190185546875	\\
0.911439605806366	-0.001190185546875	\\
0.9114839969814	-0.001129150390625	\\
0.911528388156435	-0.001007080078125	\\
0.911572779331469	-0.00103759765625	\\
0.911617170506503	-0.001007080078125	\\
0.911661561681538	-0.00091552734375	\\
0.911705952856572	-0.00079345703125	\\
0.911750344031606	-0.00048828125	\\
0.911794735206641	-0.000457763671875	\\
0.911839126381675	-0.000274658203125	\\
0.91188351755671	-0.000152587890625	\\
0.911927908731744	-0.000152587890625	\\
0.911972299906779	-6.103515625e-05	\\
0.912016691081813	-0.000335693359375	\\
0.912061082256847	-0.000640869140625	\\
0.912105473431882	-0.000457763671875	\\
0.912149864606916	9.1552734375e-05	\\
0.912194255781951	0.00042724609375	\\
0.912238646956985	0.0006103515625	\\
0.912283038132019	0.00067138671875	\\
0.912327429307054	0.000579833984375	\\
0.912371820482088	0.000335693359375	\\
0.912416211657123	0.000335693359375	\\
0.912460602832157	0.000640869140625	\\
0.912504994007191	0.0003662109375	\\
0.912549385182226	0.000274658203125	\\
0.91259377635726	0.0006103515625	\\
0.912638167532295	0.000335693359375	\\
0.912682558707329	0.000213623046875	\\
0.912726949882363	-6.103515625e-05	\\
0.912771341057398	0.00018310546875	\\
0.912815732232432	0.000152587890625	\\
0.912860123407467	-0.00018310546875	\\
0.912904514582501	0.000274658203125	\\
0.912948905757535	0.00030517578125	\\
0.91299329693257	6.103515625e-05	\\
0.913037688107604	-3.0517578125e-05	\\
0.913082079282639	-9.1552734375e-05	\\
0.913126470457673	-0.0003662109375	\\
0.913170861632707	-0.00048828125	\\
0.913215252807742	-0.00042724609375	\\
0.913259643982776	-0.0003662109375	\\
0.913304035157811	-9.1552734375e-05	\\
0.913348426332845	-0.0006103515625	\\
0.913392817507879	-0.00054931640625	\\
0.913437208682914	-0.00054931640625	\\
0.913481599857948	-0.0010986328125	\\
0.913525991032983	-0.0010986328125	\\
0.913570382208017	-0.000885009765625	\\
0.913614773383051	-0.000732421875	\\
0.913659164558086	-0.001220703125	\\
0.91370355573312	-0.00115966796875	\\
0.913747946908155	-0.000640869140625	\\
0.913792338083189	-0.0008544921875	\\
0.913836729258223	-0.00103759765625	\\
0.913881120433258	-0.000946044921875	\\
0.913925511608292	-0.001007080078125	\\
0.913969902783327	-0.001129150390625	\\
0.914014293958361	-0.00079345703125	\\
0.914058685133396	-0.000518798828125	\\
0.91410307630843	-0.00030517578125	\\
0.914147467483464	-0.000946044921875	\\
0.914191858658499	-0.0009765625	\\
0.914236249833533	-0.00048828125	\\
0.914280641008568	-0.00079345703125	\\
0.914325032183602	-0.00103759765625	\\
0.914369423358636	-0.00128173828125	\\
0.914413814533671	-0.0009765625	\\
0.914458205708705	-0.000762939453125	\\
0.914502596883739	-0.001007080078125	\\
0.914546988058774	-0.000762939453125	\\
0.914591379233808	-0.000274658203125	\\
0.914635770408843	-0.000335693359375	\\
0.914680161583877	-0.00048828125	\\
0.914724552758912	-0.000457763671875	\\
0.914768943933946	-0.0006103515625	\\
0.91481333510898	-0.0006103515625	\\
0.914857726284015	-0.000396728515625	\\
0.914902117459049	-0.000640869140625	\\
0.914946508634084	-0.000640869140625	\\
0.914990899809118	-0.001007080078125	\\
0.915035290984152	-0.001220703125	\\
0.915079682159187	-0.001190185546875	\\
0.915124073334221	-0.001007080078125	\\
0.915168464509256	-0.001373291015625	\\
0.91521285568429	-0.0018310546875	\\
0.915257246859324	-0.001739501953125	\\
0.915301638034359	-0.00128173828125	\\
0.915346029209393	-0.00146484375	\\
0.915390420384428	-0.00128173828125	\\
0.915434811559462	-0.0006103515625	\\
0.915479202734496	-0.00091552734375	\\
0.915523593909531	-0.001068115234375	\\
0.915567985084565	-0.001068115234375	\\
0.9156123762596	-0.001129150390625	\\
0.915656767434634	-0.000762939453125	\\
0.915701158609668	6.103515625e-05	\\
0.915745549784703	9.1552734375e-05	\\
0.915789940959737	-0.0001220703125	\\
0.915834332134772	6.103515625e-05	\\
0.915878723309806	-0.00030517578125	\\
0.91592311448484	-0.0001220703125	\\
0.915967505659875	0.000274658203125	\\
0.916011896834909	0.000335693359375	\\
0.916056288009944	0.00067138671875	\\
0.916100679184978	0.000579833984375	\\
0.916145070360012	0.000732421875	\\
0.916189461535047	0.00103759765625	\\
0.916233852710081	0.0010986328125	\\
0.916278243885116	0.00079345703125	\\
0.91632263506015	0.00067138671875	\\
0.916367026235184	0.00067138671875	\\
0.916411417410219	0.00067138671875	\\
0.916455808585253	0.0010986328125	\\
0.916500199760288	0.00115966796875	\\
0.916544590935322	0.000579833984375	\\
0.916588982110356	0.00067138671875	\\
0.916633373285391	0.000579833984375	\\
0.916677764460425	0.000579833984375	\\
0.91672215563546	0.000396728515625	\\
0.916766546810494	3.0517578125e-05	\\
0.916810937985528	0.000274658203125	\\
0.916855329160563	6.103515625e-05	\\
0.916899720335597	-0.000396728515625	\\
0.916944111510632	-0.0003662109375	\\
0.916988502685666	-0.000457763671875	\\
0.917032893860701	-0.0008544921875	\\
0.917077285035735	-0.000762939453125	\\
0.917121676210769	-0.000579833984375	\\
0.917166067385804	-0.000396728515625	\\
0.917210458560838	-0.000274658203125	\\
0.917254849735872	0.000244140625	\\
0.917299240910907	0.00079345703125	\\
0.917343632085941	0.000579833984375	\\
0.917388023260976	9.1552734375e-05	\\
0.91743241443601	0	\\
0.917476805611045	0.000274658203125	\\
0.917521196786079	6.103515625e-05	\\
0.917565587961113	3.0517578125e-05	\\
0.917609979136148	9.1552734375e-05	\\
0.917654370311182	0.000335693359375	\\
0.917698761486217	0.000396728515625	\\
0.917743152661251	0.00103759765625	\\
0.917787543836285	0.00091552734375	\\
0.91783193501132	0.000518798828125	\\
0.917876326186354	0.0010986328125	\\
0.917920717361389	0.0013427734375	\\
0.917965108536423	0.00146484375	\\
0.918009499711457	0.001678466796875	\\
0.918053890886492	0.00177001953125	\\
0.918098282061526	0.001739501953125	\\
0.918142673236561	0.001373291015625	\\
0.918187064411595	0.000946044921875	\\
0.918231455586629	0.001129150390625	\\
0.918275846761664	0.00067138671875	\\
0.918320237936698	0.00103759765625	\\
0.918364629111733	0.001068115234375	\\
0.918409020286767	0.000518798828125	\\
0.918453411461801	0.0008544921875	\\
0.918497802636836	0.0010986328125	\\
0.91854219381187	0.001190185546875	\\
0.918586584986905	0.0010986328125	\\
0.918630976161939	0.000335693359375	\\
0.918675367336973	0.000213623046875	\\
0.918719758512008	0.0001220703125	\\
0.918764149687042	-0.00048828125	\\
0.918808540862077	-0.000213623046875	\\
0.918852932037111	-0.0003662109375	\\
0.918897323212145	-0.00067138671875	\\
0.91894171438718	-0.0009765625	\\
0.918986105562214	-0.001190185546875	\\
0.919030496737249	-0.000823974609375	\\
0.919074887912283	-0.000946044921875	\\
0.919119279087317	-0.000946044921875	\\
0.919163670262352	-0.00128173828125	\\
0.919208061437386	-0.00152587890625	\\
0.919252452612421	-0.001251220703125	\\
0.919296843787455	-0.001556396484375	\\
0.919341234962489	-0.001922607421875	\\
0.919385626137524	-0.0015869140625	\\
0.919430017312558	-0.001434326171875	\\
0.919474408487593	-0.001708984375	\\
0.919518799662627	-0.0018310546875	\\
0.919563190837661	-0.001983642578125	\\
0.919607582012696	-0.001861572265625	\\
0.91965197318773	-0.001708984375	\\
0.919696364362765	-0.00177001953125	\\
0.919740755537799	-0.001708984375	\\
0.919785146712834	-0.00201416015625	\\
0.919829537887868	-0.00213623046875	\\
0.919873929062902	-0.00152587890625	\\
0.919918320237937	-0.00152587890625	\\
0.919962711412971	-0.001708984375	\\
0.920007102588006	-0.00146484375	\\
0.92005149376304	-0.001617431640625	\\
0.920095884938074	-0.001800537109375	\\
0.920140276113109	-0.00115966796875	\\
0.920184667288143	-0.001190185546875	\\
0.920229058463177	-0.001373291015625	\\
0.920273449638212	-0.00091552734375	\\
0.920317840813246	-0.000762939453125	\\
0.920362231988281	-0.00103759765625	\\
0.920406623163315	-0.00091552734375	\\
0.92045101433835	-0.0008544921875	\\
0.920495405513384	-0.001068115234375	\\
0.920539796688418	-0.000885009765625	\\
0.920584187863453	-0.000946044921875	\\
0.920628579038487	-0.001373291015625	\\
0.920672970213522	-0.0008544921875	\\
0.920717361388556	-0.000701904296875	\\
0.92076175256359	-0.001129150390625	\\
0.920806143738625	-0.001068115234375	\\
0.920850534913659	-0.001495361328125	\\
0.920894926088694	-0.001373291015625	\\
0.920939317263728	-0.001251220703125	\\
0.920983708438762	-0.0009765625	\\
0.921028099613797	-0.000885009765625	\\
0.921072490788831	-0.000946044921875	\\
0.921116881963866	-0.00115966796875	\\
0.9211612731389	-0.00128173828125	\\
0.921205664313934	-0.000946044921875	\\
0.921250055488969	-0.000823974609375	\\
0.921294446664003	-0.00115966796875	\\
0.921338837839038	-0.000885009765625	\\
0.921383229014072	-0.00042724609375	\\
0.921427620189106	-0.00030517578125	\\
0.921472011364141	-0.00018310546875	\\
0.921516402539175	-0.00048828125	\\
0.92156079371421	-0.00054931640625	\\
0.921605184889244	-0.00067138671875	\\
0.921649576064278	-0.00067138671875	\\
0.921693967239313	-0.00030517578125	\\
0.921738358414347	-0.00048828125	\\
0.921782749589382	-0.000732421875	\\
0.921827140764416	-0.0003662109375	\\
0.92187153193945	-0.0003662109375	\\
0.921915923114485	-0.000396728515625	\\
0.921960314289519	6.103515625e-05	\\
0.922004705464554	-0.00054931640625	\\
0.922049096639588	-0.000457763671875	\\
0.922093487814622	-0.00030517578125	\\
0.922137878989657	-0.00067138671875	\\
0.922182270164691	-0.00067138671875	\\
0.922226661339726	-0.0009765625	\\
0.92227105251476	-0.0003662109375	\\
0.922315443689794	-0.00042724609375	\\
0.922359834864829	-0.000885009765625	\\
0.922404226039863	-0.00048828125	\\
0.922448617214898	-0.00018310546875	\\
0.922493008389932	-0.00018310546875	\\
0.922537399564967	-0.00030517578125	\\
0.922581790740001	-3.0517578125e-05	\\
0.922626181915035	-9.1552734375e-05	\\
0.92267057309007	9.1552734375e-05	\\
0.922714964265104	0.00030517578125	\\
0.922759355440139	-0.000457763671875	\\
0.922803746615173	-0.00048828125	\\
0.922848137790207	-0.00048828125	\\
0.922892528965242	-0.000244140625	\\
0.922936920140276	3.0517578125e-05	\\
0.92298131131531	-0.00030517578125	\\
0.923025702490345	-0.0001220703125	\\
0.923070093665379	6.103515625e-05	\\
0.923114484840414	0	\\
0.923158876015448	0.000152587890625	\\
0.923203267190483	0.000152587890625	\\
0.923247658365517	0.000244140625	\\
0.923292049540551	0.00067138671875	\\
0.923336440715586	0.00048828125	\\
0.92338083189062	0.00048828125	\\
0.923425223065655	0.0006103515625	\\
0.923469614240689	0.0006103515625	\\
0.923514005415723	0.000732421875	\\
0.923558396590758	0.00042724609375	\\
0.923602787765792	0.000701904296875	\\
0.923647178940827	0.001007080078125	\\
0.923691570115861	0.001007080078125	\\
0.923735961290895	0.001220703125	\\
0.92378035246593	0.00128173828125	\\
0.923824743640964	0.00103759765625	\\
0.923869134815999	0.00115966796875	\\
0.923913525991033	0.001373291015625	\\
0.923957917166067	0.0015869140625	\\
0.924002308341102	0.001434326171875	\\
0.924046699516136	0.00103759765625	\\
0.924091090691171	0.001495361328125	\\
0.924135481866205	0.00152587890625	\\
0.924179873041239	0.00115966796875	\\
0.924224264216274	0.00115966796875	\\
0.924268655391308	0.00128173828125	\\
0.924313046566343	0.0013427734375	\\
0.924357437741377	0.00152587890625	\\
0.924401828916411	0.001617431640625	\\
0.924446220091446	0.001708984375	\\
0.92449061126648	0.001556396484375	\\
0.924535002441515	0.00140380859375	\\
0.924579393616549	0.001220703125	\\
0.924623784791583	0.001190185546875	\\
0.924668175966618	0.00128173828125	\\
0.924712567141652	0.00103759765625	\\
0.924756958316687	0.00079345703125	\\
0.924801349491721	0.000579833984375	\\
0.924845740666755	0.000640869140625	\\
0.92489013184179	0.00079345703125	\\
0.924934523016824	0.000946044921875	\\
0.924978914191859	0.001251220703125	\\
0.925023305366893	0.0015869140625	\\
0.925067696541927	0.00128173828125	\\
0.925112087716962	0.001251220703125	\\
0.925156478891996	0.001007080078125	\\
0.925200870067031	0.000885009765625	\\
0.925245261242065	0.00140380859375	\\
0.925289652417099	0.001220703125	\\
0.925334043592134	0.001190185546875	\\
0.925378434767168	0.0015869140625	\\
0.925422825942203	0.001373291015625	\\
0.925467217117237	0.0010986328125	\\
0.925511608292272	0.00115966796875	\\
0.925555999467306	0.001495361328125	\\
0.92560039064234	0.00201416015625	\\
0.925644781817375	0.002227783203125	\\
0.925689172992409	0.001556396484375	\\
0.925733564167443	0.00152587890625	\\
0.925777955342478	0.00164794921875	\\
0.925822346517512	0.001434326171875	\\
0.925866737692547	0.001312255859375	\\
0.925911128867581	0.00128173828125	\\
0.925955520042616	0.00146484375	\\
0.92599991121765	0.001678466796875	\\
0.926044302392684	0.0013427734375	\\
0.926088693567719	0.00140380859375	\\
0.926133084742753	0.001617431640625	\\
0.926177475917788	0.00140380859375	\\
0.926221867092822	0.00146484375	\\
0.926266258267856	0.00140380859375	\\
0.926310649442891	0.0010986328125	\\
0.926355040617925	0.00091552734375	\\
0.92639943179296	0.000396728515625	\\
0.926443822967994	0.000457763671875	\\
0.926488214143028	0.000946044921875	\\
0.926532605318063	0.0006103515625	\\
0.926576996493097	0.00042724609375	\\
0.926621387668132	0.000823974609375	\\
0.926665778843166	0.000152587890625	\\
0.9267101700182	-0.000518798828125	\\
0.926754561193235	0.0001220703125	\\
0.926798952368269	0.0001220703125	\\
0.926843343543304	-0.00030517578125	\\
0.926887734718338	-0.00018310546875	\\
0.926932125893372	-6.103515625e-05	\\
0.926976517068407	-0.000244140625	\\
0.927020908243441	-0.000396728515625	\\
0.927065299418476	0	\\
0.92710969059351	0	\\
0.927154081768544	-6.103515625e-05	\\
0.927198472943579	-3.0517578125e-05	\\
0.927242864118613	-0.00048828125	\\
0.927287255293648	-0.000396728515625	\\
0.927331646468682	-6.103515625e-05	\\
0.927376037643716	-3.0517578125e-05	\\
0.927420428818751	0.000335693359375	\\
0.927464819993785	0.000335693359375	\\
0.92750921116882	-0.000213623046875	\\
0.927553602343854	-0.000244140625	\\
0.927597993518889	-0.000732421875	\\
0.927642384693923	3.0517578125e-05	\\
0.927686775868957	0.000213623046875	\\
0.927731167043992	0	\\
0.927775558219026	0.0006103515625	\\
0.92781994939406	0.00018310546875	\\
0.927864340569095	9.1552734375e-05	\\
0.927908731744129	0.0003662109375	\\
0.927953122919164	0.0001220703125	\\
0.927997514094198	-0.000213623046875	\\
0.928041905269232	-0.000640869140625	\\
0.928086296444267	-0.0001220703125	\\
0.928130687619301	0.00048828125	\\
0.928175078794336	0.00042724609375	\\
0.92821946996937	0.0001220703125	\\
0.928263861144405	-0.000152587890625	\\
0.928308252319439	0.00030517578125	\\
0.928352643494473	9.1552734375e-05	\\
0.928397034669508	-0.000152587890625	\\
0.928441425844542	-0.00018310546875	\\
0.928485817019577	-0.0001220703125	\\
0.928530208194611	0.000579833984375	\\
0.928574599369645	0.000396728515625	\\
0.92861899054468	0.0001220703125	\\
0.928663381719714	-0.00018310546875	\\
0.928707772894748	-0.000579833984375	\\
0.928752164069783	-0.00042724609375	\\
0.928796555244817	-0.000274658203125	\\
0.928840946419852	-0.000396728515625	\\
0.928885337594886	-0.00042724609375	\\
0.928929728769921	-0.000518798828125	\\
0.928974119944955	-0.00079345703125	\\
0.929018511119989	-0.00054931640625	\\
0.929062902295024	-0.00042724609375	\\
0.929107293470058	-0.000732421875	\\
0.929151684645093	-0.00091552734375	\\
0.929196075820127	-0.000762939453125	\\
0.929240466995161	-0.00054931640625	\\
0.929284858170196	-0.000640869140625	\\
0.92932924934523	-0.00042724609375	\\
0.929373640520265	-0.0001220703125	\\
0.929418031695299	0.00018310546875	\\
0.929462422870333	0.00030517578125	\\
0.929506814045368	0.000457763671875	\\
0.929551205220402	0.00030517578125	\\
0.929595596395437	-6.103515625e-05	\\
0.929639987570471	0.000274658203125	\\
0.929684378745505	0.000579833984375	\\
0.92972876992054	0.00042724609375	\\
0.929773161095574	0.00030517578125	\\
0.929817552270609	-3.0517578125e-05	\\
0.929861943445643	0.000396728515625	\\
0.929906334620677	0.000518798828125	\\
0.929950725795712	-6.103515625e-05	\\
0.929995116970746	6.103515625e-05	\\
0.930039508145781	0.000274658203125	\\
0.930083899320815	0.000335693359375	\\
0.930128290495849	0.000518798828125	\\
0.930172681670884	0.000823974609375	\\
0.930217072845918	0.00067138671875	\\
0.930261464020953	0.000457763671875	\\
0.930305855195987	0.000640869140625	\\
0.930350246371021	0.000335693359375	\\
0.930394637546056	0.0003662109375	\\
0.93043902872109	0.00030517578125	\\
0.930483419896125	0.00067138671875	\\
0.930527811071159	0.0006103515625	\\
0.930572202246193	0.0001220703125	\\
0.930616593421228	0.00067138671875	\\
0.930660984596262	0.0006103515625	\\
0.930705375771297	0.000213623046875	\\
0.930749766946331	0.000274658203125	\\
0.930794158121365	0	\\
0.9308385492964	0.0001220703125	\\
0.930882940471434	0.00018310546875	\\
0.930927331646469	0.000213623046875	\\
0.930971722821503	-6.103515625e-05	\\
0.931016113996538	-0.000457763671875	\\
0.931060505171572	-0.000152587890625	\\
0.931104896346606	-0.00018310546875	\\
0.931149287521641	6.103515625e-05	\\
0.931193678696675	-0.00018310546875	\\
0.93123806987171	-0.00030517578125	\\
0.931282461046744	-0.000152587890625	\\
0.931326852221778	-9.1552734375e-05	\\
0.931371243396813	-0.000335693359375	\\
0.931415634571847	-0.000213623046875	\\
0.931460025746881	-0.000274658203125	\\
0.931504416921916	-0.000213623046875	\\
0.93154880809695	0	\\
0.931593199271985	0.00018310546875	\\
0.931637590447019	-0.00018310546875	\\
0.931681981622054	0.000213623046875	\\
0.931726372797088	0.000396728515625	\\
0.931770763972122	0.000396728515625	\\
0.931815155147157	0.000732421875	\\
0.931859546322191	0.0006103515625	\\
0.931903937497226	0.000640869140625	\\
0.93194832867226	0.00048828125	\\
0.931992719847294	0.000823974609375	\\
0.932037111022329	0.001434326171875	\\
0.932081502197363	0.0010986328125	\\
0.932125893372398	0.000701904296875	\\
0.932170284547432	0.000732421875	\\
0.932214675722466	0.00054931640625	\\
0.932259066897501	0.00091552734375	\\
0.932303458072535	0.00115966796875	\\
0.93234784924757	0.00103759765625	\\
0.932392240422604	0.00115966796875	\\
0.932436631597638	0.001220703125	\\
0.932481022772673	0.001434326171875	\\
0.932525413947707	0.001251220703125	\\
0.932569805122742	0.001617431640625	\\
0.932614196297776	0.001800537109375	\\
0.93265858747281	0.00115966796875	\\
0.932702978647845	0.001220703125	\\
0.932747369822879	0.00079345703125	\\
0.932791760997914	0.00067138671875	\\
0.932836152172948	0.000732421875	\\
0.932880543347982	0.0003662109375	\\
0.932924934523017	0.000579833984375	\\
0.932969325698051	0.00042724609375	\\
0.933013716873086	0.000457763671875	\\
0.93305810804812	0.000396728515625	\\
0.933102499223154	0.000335693359375	\\
0.933146890398189	0.00042724609375	\\
0.933191281573223	0.000335693359375	\\
0.933235672748258	0.0008544921875	\\
0.933280063923292	0.001007080078125	\\
0.933324455098326	0.00042724609375	\\
0.933368846273361	0.0003662109375	\\
0.933413237448395	-0.000244140625	\\
0.93345762862343	6.103515625e-05	\\
0.933502019798464	0.00048828125	\\
0.933546410973498	0.00030517578125	\\
0.933590802148533	0.000396728515625	\\
0.933635193323567	0.000213623046875	\\
0.933679584498602	9.1552734375e-05	\\
0.933723975673636	0	\\
0.93376836684867	6.103515625e-05	\\
0.933812758023705	0.00030517578125	\\
0.933857149198739	0.000244140625	\\
0.933901540373774	0.000244140625	\\
0.933945931548808	0.00030517578125	\\
0.933990322723843	-9.1552734375e-05	\\
0.934034713898877	-0.000335693359375	\\
0.934079105073911	0.0001220703125	\\
0.934123496248946	0.000244140625	\\
0.93416788742398	0.000274658203125	\\
0.934212278599015	0.0001220703125	\\
0.934256669774049	0.00030517578125	\\
0.934301060949083	0.000579833984375	\\
0.934345452124118	0.000762939453125	\\
0.934389843299152	0.00048828125	\\
0.934434234474187	0.00030517578125	\\
0.934478625649221	0.000274658203125	\\
0.934523016824255	0.0001220703125	\\
0.93456740799929	-0.00018310546875	\\
0.934611799174324	-0.0003662109375	\\
0.934656190349359	-0.000579833984375	\\
0.934700581524393	-0.0001220703125	\\
0.934744972699427	0.0003662109375	\\
0.934789363874462	9.1552734375e-05	\\
0.934833755049496	-6.103515625e-05	\\
0.934878146224531	0.000213623046875	\\
0.934922537399565	9.1552734375e-05	\\
0.934966928574599	-0.000396728515625	\\
0.935011319749634	-0.00042724609375	\\
0.935055710924668	-0.00018310546875	\\
0.935100102099703	-0.000335693359375	\\
0.935144493274737	0.00018310546875	\\
0.935188884449771	-0.000274658203125	\\
0.935233275624806	-0.00042724609375	\\
0.93527766679984	3.0517578125e-05	\\
0.935322057974875	-0.000152587890625	\\
0.935366449149909	-0.000396728515625	\\
0.935410840324943	-0.00054931640625	\\
0.935455231499978	-0.0006103515625	\\
0.935499622675012	-9.1552734375e-05	\\
0.935544013850047	-0.000152587890625	\\
0.935588405025081	-0.000457763671875	\\
0.935632796200115	-0.00048828125	\\
0.93567718737515	-0.000579833984375	\\
0.935721578550184	-0.00030517578125	\\
0.935765969725219	-0.000579833984375	\\
0.935810360900253	-0.000457763671875	\\
0.935854752075287	-0.00054931640625	\\
0.935899143250322	-0.0008544921875	\\
0.935943534425356	-0.00030517578125	\\
0.935987925600391	-0.000457763671875	\\
0.936032316775425	-0.0006103515625	\\
0.93607670795046	-0.000579833984375	\\
0.936121099125494	-0.000732421875	\\
0.936165490300528	-0.000946044921875	\\
0.936209881475563	-0.001007080078125	\\
0.936254272650597	-0.0010986328125	\\
0.936298663825631	-0.0010986328125	\\
0.936343055000666	-0.00079345703125	\\
0.9363874461757	-0.000732421875	\\
0.936431837350735	-0.001007080078125	\\
0.936476228525769	-0.00103759765625	\\
0.936520619700803	-0.0009765625	\\
0.936565010875838	-0.001190185546875	\\
0.936609402050872	-0.000885009765625	\\
0.936653793225907	-0.00079345703125	\\
0.936698184400941	-0.000823974609375	\\
0.936742575575976	-0.000823974609375	\\
0.93678696675101	-0.0009765625	\\
0.936831357926044	-0.00128173828125	\\
0.936875749101079	-0.00091552734375	\\
0.936920140276113	-0.000701904296875	\\
0.936964531451148	-0.0008544921875	\\
0.937008922626182	-0.000457763671875	\\
0.937053313801216	-0.00048828125	\\
0.937097704976251	-0.00054931640625	\\
0.937142096151285	-0.00030517578125	\\
0.937186487326319	-6.103515625e-05	\\
0.937230878501354	-3.0517578125e-05	\\
0.937275269676388	0.000152587890625	\\
0.937319660851423	0.00018310546875	\\
0.937364052026457	0.00018310546875	\\
0.937408443201492	0.0001220703125	\\
0.937452834376526	0.000396728515625	\\
0.93749722555156	0.00042724609375	\\
0.937541616726595	-3.0517578125e-05	\\
0.937586007901629	0.00018310546875	\\
0.937630399076664	-3.0517578125e-05	\\
0.937674790251698	0	\\
0.937719181426732	0.00042724609375	\\
0.937763572601767	0.000518798828125	\\
0.937807963776801	0.00048828125	\\
0.937852354951836	0.0003662109375	\\
0.93789674612687	0.00079345703125	\\
0.937941137301904	0.00067138671875	\\
0.937985528476939	0.000640869140625	\\
0.938029919651973	0.000946044921875	\\
0.938074310827008	0.000701904296875	\\
0.938118702002042	0.0006103515625	\\
0.938163093177076	0.000457763671875	\\
0.938207484352111	0.000335693359375	\\
0.938251875527145	0.000244140625	\\
0.93829626670218	0.000244140625	\\
0.938340657877214	0.00042724609375	\\
0.938385049052248	0.00042724609375	\\
0.938429440227283	0.00042724609375	\\
0.938473831402317	0.000579833984375	\\
0.938518222577352	0.0006103515625	\\
0.938562613752386	0.000518798828125	\\
0.93860700492742	0.00091552734375	\\
0.938651396102455	0.000946044921875	\\
0.938695787277489	0.000762939453125	\\
0.938740178452524	0.001007080078125	\\
0.938784569627558	0.001007080078125	\\
0.938828960802592	0.000701904296875	\\
0.938873351977627	0.000823974609375	\\
0.938917743152661	0.00048828125	\\
0.938962134327696	0.000762939453125	\\
0.93900652550273	0.000885009765625	\\
0.939050916677764	0.001190185546875	\\
0.939095307852799	0.00140380859375	\\
0.939139699027833	0.001373291015625	\\
0.939184090202868	0.001800537109375	\\
0.939228481377902	0.001312255859375	\\
0.939272872552936	0.001434326171875	\\
0.939317263727971	0.001922607421875	\\
0.939361654903005	0.001495361328125	\\
0.93940604607804	0.001861572265625	\\
0.939450437253074	0.0018310546875	\\
0.939494828428109	0.001678466796875	\\
0.939539219603143	0.0013427734375	\\
0.939583610778177	0.00140380859375	\\
0.939628001953212	0.00140380859375	\\
0.939672393128246	0.001312255859375	\\
0.939716784303281	0.001495361328125	\\
0.939761175478315	0.001220703125	\\
0.939805566653349	0.001495361328125	\\
0.939849957828384	0.002349853515625	\\
0.939894349003418	0.001983642578125	\\
0.939938740178452	0.00189208984375	\\
0.939983131353487	0.00238037109375	\\
0.940027522528521	0.001953125	\\
0.940071913703556	0.002532958984375	\\
0.94011630487859	0.002410888671875	\\
0.940160696053625	0.001739501953125	\\
0.940205087228659	0.001953125	\\
0.940249478403693	0.00189208984375	\\
0.940293869578728	0.002166748046875	\\
0.940338260753762	0.001922607421875	\\
0.940382651928797	0.001495361328125	\\
0.940427043103831	0.001556396484375	\\
0.940471434278865	0.001434326171875	\\
0.9405158254539	0.001220703125	\\
0.940560216628934	0.001495361328125	\\
0.940604607803969	0.001922607421875	\\
0.940648998979003	0.001190185546875	\\
0.940693390154037	0.00079345703125	\\
0.940737781329072	0.001312255859375	\\
0.940782172504106	0.001434326171875	\\
0.940826563679141	0.00128173828125	\\
0.940870954854175	0.000640869140625	\\
0.940915346029209	0.000823974609375	\\
0.940959737204244	0.001220703125	\\
0.941004128379278	0.00103759765625	\\
0.941048519554313	0.00115966796875	\\
0.941092910729347	0.000823974609375	\\
0.941137301904381	0.00103759765625	\\
0.941181693079416	0.001220703125	\\
0.94122608425445	0.001373291015625	\\
0.941270475429485	0.001190185546875	\\
0.941314866604519	0.00140380859375	\\
0.941359257779553	0.00177001953125	\\
0.941403648954588	0.001434326171875	\\
0.941448040129622	0.00152587890625	\\
0.941492431304657	0.001800537109375	\\
0.941536822479691	0.001708984375	\\
0.941581213654725	0.0015869140625	\\
0.94162560482976	0.001708984375	\\
0.941669996004794	0.001678466796875	\\
0.941714387179829	0.001708984375	\\
0.941758778354863	0.00152587890625	\\
0.941803169529898	0.001617431640625	\\
0.941847560704932	0.001739501953125	\\
0.941891951879966	0.001983642578125	\\
0.941936343055001	0.002044677734375	\\
0.941980734230035	0.001556396484375	\\
0.942025125405069	0.002166748046875	\\
0.942069516580104	0.001953125	\\
0.942113907755138	0.001922607421875	\\
0.942158298930173	0.001983642578125	\\
0.942202690105207	0.001556396484375	\\
0.942247081280241	0.00213623046875	\\
0.942291472455276	0.001953125	\\
0.94233586363031	0.001678466796875	\\
0.942380254805345	0.001739501953125	\\
0.942424645980379	0.00177001953125	\\
0.942469037155414	0.001800537109375	\\
0.942513428330448	0.00146484375	\\
0.942557819505482	0.00140380859375	\\
0.942602210680517	0.00128173828125	\\
0.942646601855551	0.001190185546875	\\
0.942690993030586	0.001434326171875	\\
0.94273538420562	0.0015869140625	\\
0.942779775380654	0.001220703125	\\
0.942824166555689	0.0008544921875	\\
0.942868557730723	0.000823974609375	\\
0.942912948905758	0.001251220703125	\\
0.942957340080792	0.001190185546875	\\
0.943001731255826	0.00128173828125	\\
0.943046122430861	0.001678466796875	\\
0.943090513605895	0.001922607421875	\\
0.94313490478093	0.001678466796875	\\
0.943179295955964	0.00091552734375	\\
0.943223687130998	0.0010986328125	\\
0.943268078306033	0.000762939453125	\\
0.943312469481067	0.000732421875	\\
0.943356860656102	0.001220703125	\\
0.943401251831136	0.00067138671875	\\
0.94344564300617	0.00128173828125	\\
0.943490034181205	0.0013427734375	\\
0.943534425356239	0.001434326171875	\\
0.943578816531274	0.0015869140625	\\
0.943623207706308	0.0015869140625	\\
0.943667598881342	0.001556396484375	\\
0.943711990056377	0.0008544921875	\\
0.943756381231411	0.001190185546875	\\
0.943800772406446	0.001190185546875	\\
0.94384516358148	0.001007080078125	\\
0.943889554756514	0.001007080078125	\\
0.943933945931549	0.000701904296875	\\
0.943978337106583	0.0008544921875	\\
0.944022728281618	0.000885009765625	\\
0.944067119456652	0.00067138671875	\\
0.944111510631686	0.00042724609375	\\
0.944155901806721	0.000823974609375	\\
0.944200292981755	0.0009765625	\\
0.94424468415679	0.000396728515625	\\
0.944289075331824	9.1552734375e-05	\\
0.944333466506858	0.000335693359375	\\
0.944377857681893	0.0003662109375	\\
0.944422248856927	0	\\
0.944466640031962	-0.000335693359375	\\
0.944511031206996	9.1552734375e-05	\\
0.944555422382031	0.00018310546875	\\
0.944599813557065	-0.000274658203125	\\
0.944644204732099	-0.000335693359375	\\
0.944688595907134	-0.000244140625	\\
0.944732987082168	-0.000518798828125	\\
0.944777378257202	-0.000701904296875	\\
0.944821769432237	-0.00018310546875	\\
0.944866160607271	-0.000274658203125	\\
0.944910551782306	6.103515625e-05	\\
0.94495494295734	0.000213623046875	\\
0.944999334132374	0	\\
0.945043725307409	-0.000213623046875	\\
0.945088116482443	-0.00048828125	\\
0.945132507657478	-0.000152587890625	\\
0.945176898832512	-0.000274658203125	\\
0.945221290007547	-0.0001220703125	\\
0.945265681182581	-0.00030517578125	\\
0.945310072357615	-0.000152587890625	\\
0.94535446353265	-3.0517578125e-05	\\
0.945398854707684	-6.103515625e-05	\\
0.945443245882719	3.0517578125e-05	\\
0.945487637057753	0.000244140625	\\
0.945532028232787	-0.00018310546875	\\
0.945576419407822	-0.00018310546875	\\
0.945620810582856	-0.000335693359375	\\
0.94566520175789	-0.000518798828125	\\
0.945709592932925	-0.000579833984375	\\
0.945753984107959	-0.000579833984375	\\
0.945798375282994	-6.103515625e-05	\\
0.945842766458028	0.000213623046875	\\
0.945887157633063	0.00042724609375	\\
0.945931548808097	0.0001220703125	\\
0.945975939983131	0	\\
0.946020331158166	-0.000396728515625	\\
0.9460647223332	-0.00054931640625	\\
0.946109113508235	-0.000335693359375	\\
0.946153504683269	-0.000457763671875	\\
0.946197895858303	-0.000885009765625	\\
0.946242287033338	-0.000518798828125	\\
0.946286678208372	-0.000579833984375	\\
0.946331069383407	-0.000579833984375	\\
0.946375460558441	-0.000335693359375	\\
0.946419851733475	-3.0517578125e-05	\\
0.94646424290851	9.1552734375e-05	\\
0.946508634083544	0.000213623046875	\\
0.946553025258579	-3.0517578125e-05	\\
0.946597416433613	-0.00067138671875	\\
0.946641807608647	-0.00048828125	\\
0.946686198783682	-0.000152587890625	\\
0.946730589958716	-0.000335693359375	\\
0.946774981133751	0.0001220703125	\\
0.946819372308785	0.000457763671875	\\
0.946863763483819	0.00042724609375	\\
0.946908154658854	0.000244140625	\\
0.946952545833888	0.000640869140625	\\
0.946996937008923	0.000732421875	\\
0.947041328183957	0.000457763671875	\\
0.947085719358991	0.000579833984375	\\
0.947130110534026	0.000640869140625	\\
0.94717450170906	0.00079345703125	\\
0.947218892884095	0.001068115234375	\\
0.947263284059129	0.001251220703125	\\
0.947307675234163	0.00128173828125	\\
0.947352066409198	0.001251220703125	\\
0.947396457584232	0.001251220703125	\\
0.947440848759267	0.001312255859375	\\
0.947485239934301	0.001068115234375	\\
0.947529631109335	0.001251220703125	\\
0.94757402228437	0.001739501953125	\\
0.947618413459404	0.001068115234375	\\
0.947662804634439	0.0013427734375	\\
0.947707195809473	0.000885009765625	\\
0.947751586984507	0.00128173828125	\\
0.947795978159542	0.001495361328125	\\
0.947840369334576	0.00079345703125	\\
0.947884760509611	0.0008544921875	\\
0.947929151684645	0.00054931640625	\\
0.94797354285968	0.000457763671875	\\
0.948017934034714	0.000335693359375	\\
0.948062325209748	-3.0517578125e-05	\\
0.948106716384783	0.000152587890625	\\
0.948151107559817	-0.00018310546875	\\
0.948195498734852	-0.0003662109375	\\
0.948239889909886	0.000457763671875	\\
0.94828428108492	0.000396728515625	\\
0.948328672259955	0.000640869140625	\\
0.948373063434989	0.000732421875	\\
0.948417454610024	0.00091552734375	\\
0.948461845785058	0.000732421875	\\
0.948506236960092	0.00030517578125	\\
0.948550628135127	0.0006103515625	\\
0.948595019310161	0.00030517578125	\\
0.948639410485196	0.000579833984375	\\
0.94868380166023	0.000518798828125	\\
0.948728192835264	0.0010986328125	\\
0.948772584010299	0.001007080078125	\\
0.948816975185333	0.000518798828125	\\
0.948861366360368	0.000732421875	\\
0.948905757535402	0.00048828125	\\
0.948950148710436	0.0006103515625	\\
0.948994539885471	0.00042724609375	\\
0.949038931060505	0.000213623046875	\\
0.94908332223554	0.0006103515625	\\
0.949127713410574	0.000762939453125	\\
0.949172104585608	0.00048828125	\\
0.949216495760643	0.000518798828125	\\
0.949260886935677	0.000335693359375	\\
0.949305278110712	0	\\
0.949349669285746	0.0001220703125	\\
0.94939406046078	9.1552734375e-05	\\
0.949438451635815	0.000244140625	\\
0.949482842810849	-6.103515625e-05	\\
0.949527233985884	0.000701904296875	\\
0.949571625160918	0.000701904296875	\\
0.949616016335952	0.00048828125	\\
0.949660407510987	0.0003662109375	\\
0.949704798686021	6.103515625e-05	\\
0.949749189861056	0.000396728515625	\\
0.94979358103609	0.00091552734375	\\
0.949837972211124	0.001220703125	\\
0.949882363386159	0.000518798828125	\\
0.949926754561193	0.00042724609375	\\
0.949971145736228	0.00091552734375	\\
0.950015536911262	0.00091552734375	\\
0.950059928086296	0.000762939453125	\\
0.950104319261331	0.000946044921875	\\
0.950148710436365	0.0009765625	\\
0.9501931016114	0.001068115234375	\\
0.950237492786434	0.0008544921875	\\
0.950281883961469	0.000579833984375	\\
0.950326275136503	0.000640869140625	\\
0.950370666311537	0.0008544921875	\\
0.950415057486572	0.0008544921875	\\
0.950459448661606	0.000335693359375	\\
0.95050383983664	0.00042724609375	\\
0.950548231011675	0.000823974609375	\\
0.950592622186709	0.000762939453125	\\
0.950637013361744	0.000518798828125	\\
0.950681404536778	0.000640869140625	\\
0.950725795711812	0.000946044921875	\\
0.950770186886847	0.00115966796875	\\
0.950814578061881	0.000640869140625	\\
0.950858969236916	0.000518798828125	\\
0.95090336041195	0.000762939453125	\\
0.950947751586985	0.000946044921875	\\
0.950992142762019	0.00067138671875	\\
0.951036533937053	0.000701904296875	\\
0.951080925112088	0.000701904296875	\\
0.951125316287122	0.00042724609375	\\
0.951169707462157	0.0008544921875	\\
0.951214098637191	0.000762939453125	\\
0.951258489812225	0.000396728515625	\\
0.95130288098726	0.0006103515625	\\
0.951347272162294	0.000396728515625	\\
0.951391663337329	0.000640869140625	\\
0.951436054512363	0.00054931640625	\\
0.951480445687397	0.000457763671875	\\
0.951524836862432	0.0001220703125	\\
0.951569228037466	-0.000244140625	\\
0.951613619212501	-0.000244140625	\\
0.951658010387535	-0.00042724609375	\\
0.951702401562569	-3.0517578125e-05	\\
0.951746792737604	-0.0003662109375	\\
0.951791183912638	-0.000244140625	\\
0.951835575087673	-0.00018310546875	\\
0.951879966262707	-0.000152587890625	\\
0.951924357437741	-0.00067138671875	\\
0.951968748612776	-0.000335693359375	\\
0.95201313978781	-0.000274658203125	\\
0.952057530962845	-0.00030517578125	\\
0.952101922137879	-9.1552734375e-05	\\
0.952146313312913	-0.0003662109375	\\
0.952190704487948	-0.00030517578125	\\
0.952235095662982	-0.000701904296875	\\
0.952279486838017	-0.0008544921875	\\
0.952323878013051	-0.000701904296875	\\
0.952368269188085	-0.00054931640625	\\
0.95241266036312	-0.000701904296875	\\
0.952457051538154	-0.00048828125	\\
0.952501442713189	-0.000823974609375	\\
0.952545833888223	-0.001129150390625	\\
0.952590225063257	-0.000701904296875	\\
0.952634616238292	-0.000823974609375	\\
0.952679007413326	-0.000640869140625	\\
0.952723398588361	-0.000885009765625	\\
0.952767789763395	-0.000946044921875	\\
0.952812180938429	-0.000701904296875	\\
0.952856572113464	-0.00091552734375	\\
0.952900963288498	-0.001251220703125	\\
0.952945354463533	-0.001220703125	\\
0.952989745638567	-0.0010986328125	\\
0.953034136813602	-0.00067138671875	\\
0.953078527988636	-0.000701904296875	\\
0.95312291916367	-0.0013427734375	\\
0.953167310338705	-0.001190185546875	\\
0.953211701513739	-0.0008544921875	\\
0.953256092688773	-0.001129150390625	\\
0.953300483863808	-0.0008544921875	\\
0.953344875038842	-0.00091552734375	\\
0.953389266213877	-0.001373291015625	\\
0.953433657388911	-0.001495361328125	\\
0.953478048563945	-0.001190185546875	\\
0.95352243973898	-0.001007080078125	\\
0.953566830914014	-0.00140380859375	\\
0.953611222089049	-0.001129150390625	\\
0.953655613264083	-0.001068115234375	\\
0.953700004439118	-0.000579833984375	\\
0.953744395614152	-0.000457763671875	\\
0.953788786789186	-0.000640869140625	\\
0.953833177964221	-0.00018310546875	\\
0.953877569139255	-0.000335693359375	\\
0.95392196031429	-0.0010986328125	\\
0.953966351489324	-0.000701904296875	\\
0.954010742664358	-0.000885009765625	\\
0.954055133839393	-0.001251220703125	\\
0.954099525014427	-0.0008544921875	\\
0.954143916189461	-0.000885009765625	\\
0.954188307364496	-0.001007080078125	\\
0.95423269853953	-0.001007080078125	\\
0.954277089714565	-0.00054931640625	\\
0.954321480889599	-0.00091552734375	\\
0.954365872064634	-0.000701904296875	\\
0.954410263239668	-0.00079345703125	\\
0.954454654414702	-0.001007080078125	\\
0.954499045589737	-0.00067138671875	\\
0.954543436764771	-0.000946044921875	\\
0.954587827939806	-0.00091552734375	\\
0.95463221911484	-0.000701904296875	\\
0.954676610289874	-0.000946044921875	\\
0.954721001464909	-0.000946044921875	\\
0.954765392639943	-0.001190185546875	\\
0.954809783814978	-0.00079345703125	\\
0.954854174990012	-0.000457763671875	\\
0.954898566165046	-0.00079345703125	\\
0.954942957340081	-0.00067138671875	\\
0.954987348515115	-3.0517578125e-05	\\
0.95503173969015	-0.000152587890625	\\
0.955076130865184	-0.000732421875	\\
0.955120522040218	-0.0006103515625	\\
0.955164913215253	-0.000213623046875	\\
0.955209304390287	-0.000213623046875	\\
0.955253695565322	-0.000274658203125	\\
0.955298086740356	-0.00018310546875	\\
0.95534247791539	-0.00018310546875	\\
0.955386869090425	-9.1552734375e-05	\\
0.955431260265459	-0.000244140625	\\
0.955475651440494	-0.0001220703125	\\
0.955520042615528	-9.1552734375e-05	\\
0.955564433790562	-0.000640869140625	\\
0.955608824965597	-0.000457763671875	\\
0.955653216140631	-0.000335693359375	\\
0.955697607315666	-0.000274658203125	\\
0.9557419984907	-0.00054931640625	\\
0.955786389665734	-0.00091552734375	\\
0.955830780840769	-0.00079345703125	\\
0.955875172015803	-0.00067138671875	\\
0.955919563190838	-0.0008544921875	\\
0.955963954365872	-0.001007080078125	\\
0.956008345540906	-0.001129150390625	\\
0.956052736715941	-0.000823974609375	\\
0.956097127890975	-0.001373291015625	\\
0.95614151906601	-0.00164794921875	\\
0.956185910241044	-0.001617431640625	\\
0.956230301416078	-0.001434326171875	\\
0.956274692591113	-0.00115966796875	\\
0.956319083766147	-0.001007080078125	\\
0.956363474941182	-0.001007080078125	\\
0.956407866116216	-0.000701904296875	\\
0.956452257291251	-0.00054931640625	\\
0.956496648466285	-0.000640869140625	\\
0.956541039641319	-0.00042724609375	\\
0.956585430816354	-0.000396728515625	\\
0.956629821991388	-0.00042724609375	\\
0.956674213166423	-9.1552734375e-05	\\
0.956718604341457	-3.0517578125e-05	\\
0.956762995516491	-0.000152587890625	\\
0.956807386691526	-0.0001220703125	\\
0.95685177786656	-0.0001220703125	\\
0.956896169041595	0.0003662109375	\\
0.956940560216629	9.1552734375e-05	\\
0.956984951391663	-3.0517578125e-05	\\
0.957029342566698	9.1552734375e-05	\\
0.957073733741732	3.0517578125e-05	\\
0.957118124916767	0.00018310546875	\\
0.957162516091801	0.0001220703125	\\
0.957206907266835	-0.00018310546875	\\
0.95725129844187	0.00018310546875	\\
0.957295689616904	0.000213623046875	\\
0.957340080791939	0.0003662109375	\\
0.957384471966973	-0.0001220703125	\\
0.957428863142007	-0.000457763671875	\\
0.957473254317042	-0.000640869140625	\\
0.957517645492076	-6.103515625e-05	\\
0.957562036667111	0.000213623046875	\\
0.957606427842145	-0.000152587890625	\\
0.957650819017179	9.1552734375e-05	\\
0.957695210192214	-9.1552734375e-05	\\
0.957739601367248	-0.000396728515625	\\
0.957783992542283	-0.00079345703125	\\
0.957828383717317	-0.000885009765625	\\
0.957872774892351	-0.00048828125	\\
0.957917166067386	-0.000579833984375	\\
0.95796155724242	-0.000762939453125	\\
0.958005948417455	-0.000823974609375	\\
0.958050339592489	-0.000518798828125	\\
0.958094730767523	-0.000274658203125	\\
0.958139121942558	-0.000579833984375	\\
0.958183513117592	-0.001068115234375	\\
0.958227904292627	-0.00128173828125	\\
0.958272295467661	-0.00079345703125	\\
0.958316686642695	-0.001068115234375	\\
0.95836107781773	-0.001068115234375	\\
0.958405468992764	-0.001007080078125	\\
0.958449860167799	-0.00128173828125	\\
0.958494251342833	-0.001007080078125	\\
0.958538642517867	-0.001220703125	\\
0.958583033692902	-0.00115966796875	\\
0.958627424867936	-0.000946044921875	\\
0.958671816042971	-0.001251220703125	\\
0.958716207218005	-0.001007080078125	\\
0.95876059839304	-0.000701904296875	\\
0.958804989568074	-0.001190185546875	\\
0.958849380743108	-0.00140380859375	\\
0.958893771918143	-0.001556396484375	\\
0.958938163093177	-0.001983642578125	\\
0.958982554268211	-0.00201416015625	\\
0.959026945443246	-0.001861572265625	\\
0.95907133661828	-0.001953125	\\
0.959115727793315	-0.001922607421875	\\
0.959160118968349	-0.001739501953125	\\
0.959204510143383	-0.00164794921875	\\
0.959248901318418	-0.00213623046875	\\
0.959293292493452	-0.00238037109375	\\
0.959337683668487	-0.0025634765625	\\
0.959382074843521	-0.00244140625	\\
0.959426466018556	-0.00238037109375	\\
0.95947085719359	-0.002166748046875	\\
0.959515248368624	-0.001739501953125	\\
0.959559639543659	-0.00177001953125	\\
0.959604030718693	-0.00177001953125	\\
0.959648421893728	-0.00189208984375	\\
0.959692813068762	-0.001953125	\\
0.959737204243796	-0.001190185546875	\\
0.959781595418831	-0.00128173828125	\\
0.959825986593865	-0.001495361328125	\\
0.9598703777689	-0.001007080078125	\\
0.959914768943934	-0.001007080078125	\\
0.959959160118968	-0.001556396484375	\\
0.960003551294003	-0.001434326171875	\\
0.960047942469037	-0.001434326171875	\\
0.960092333644072	-0.001556396484375	\\
0.960136724819106	-0.0013427734375	\\
0.96018111599414	-0.000823974609375	\\
0.960225507169175	-0.000762939453125	\\
0.960269898344209	-0.00067138671875	\\
0.960314289519244	-0.0013427734375	\\
0.960358680694278	-0.001068115234375	\\
0.960403071869312	-0.000732421875	\\
0.960447463044347	-0.001373291015625	\\
0.960491854219381	-0.001922607421875	\\
0.960536245394416	-0.002410888671875	\\
0.96058063656945	-0.002044677734375	\\
0.960625027744484	-0.00189208984375	\\
0.960669418919519	-0.001678466796875	\\
0.960713810094553	-0.001373291015625	\\
0.960758201269588	-0.001556396484375	\\
0.960802592444622	-0.001922607421875	\\
0.960846983619656	-0.0018310546875	\\
0.960891374794691	-0.0018310546875	\\
0.960935765969725	-0.001708984375	\\
0.96098015714476	-0.00164794921875	\\
0.961024548319794	-0.001708984375	\\
0.961068939494828	-0.001922607421875	\\
0.961113330669863	-0.002197265625	\\
0.961157721844897	-0.001800537109375	\\
0.961202113019932	-0.001617431640625	\\
0.961246504194966	-0.002166748046875	\\
0.96129089537	-0.002655029296875	\\
0.961335286545035	-0.00189208984375	\\
0.961379677720069	-0.00213623046875	\\
0.961424068895104	-0.00225830078125	\\
0.961468460070138	-0.0015869140625	\\
0.961512851245173	-0.001953125	\\
0.961557242420207	-0.0018310546875	\\
0.961601633595241	-0.001800537109375	\\
0.961646024770276	-0.001739501953125	\\
0.96169041594531	-0.001922607421875	\\
0.961734807120344	-0.001556396484375	\\
0.961779198295379	-0.001373291015625	\\
0.961823589470413	-0.001983642578125	\\
0.961867980645448	-0.00140380859375	\\
0.961912371820482	-0.00128173828125	\\
0.961956762995516	-0.00189208984375	\\
0.962001154170551	-0.001434326171875	\\
0.962045545345585	-0.000823974609375	\\
0.96208993652062	-0.0013427734375	\\
0.962134327695654	-0.001434326171875	\\
0.962178718870689	-0.00128173828125	\\
0.962223110045723	-0.000885009765625	\\
0.962267501220757	-0.00042724609375	\\
0.962311892395792	-0.000579833984375	\\
0.962356283570826	-0.00103759765625	\\
0.962400674745861	-0.000518798828125	\\
0.962445065920895	0	\\
0.962489457095929	-0.000457763671875	\\
0.962533848270964	-0.00103759765625	\\
0.962578239445998	-0.00103759765625	\\
0.962622630621033	-0.000946044921875	\\
0.962667021796067	-0.000244140625	\\
0.962711412971101	-0.001129150390625	\\
0.962755804146136	-0.001068115234375	\\
0.96280019532117	-0.000396728515625	\\
0.962844586496205	-0.000762939453125	\\
0.962888977671239	-0.0003662109375	\\
0.962933368846273	-0.00042724609375	\\
0.962977760021308	-0.00067138671875	\\
0.963022151196342	-0.000701904296875	\\
0.963066542371377	-0.0003662109375	\\
0.963110933546411	-0.00042724609375	\\
0.963155324721445	-0.000335693359375	\\
0.96319971589648	-0.000335693359375	\\
0.963244107071514	-0.000579833984375	\\
0.963288498246549	-0.00042724609375	\\
0.963332889421583	-0.0006103515625	\\
0.963377280596617	-0.0006103515625	\\
0.963421671771652	-0.000244140625	\\
0.963466062946686	-0.000518798828125	\\
0.963510454121721	-0.00042724609375	\\
0.963554845296755	-0.0001220703125	\\
0.963599236471789	-0.000274658203125	\\
0.963643627646824	-0.0003662109375	\\
0.963688018821858	-0.000823974609375	\\
0.963732409996893	-0.000823974609375	\\
0.963776801171927	-0.0003662109375	\\
0.963821192346961	-0.000701904296875	\\
0.963865583521996	-0.00079345703125	\\
0.96390997469703	-0.00079345703125	\\
0.963954365872065	-0.000946044921875	\\
0.963998757047099	-0.0006103515625	\\
0.964043148222133	-0.00048828125	\\
0.964087539397168	-0.0006103515625	\\
0.964131930572202	-0.000244140625	\\
0.964176321747237	0	\\
0.964220712922271	-0.0001220703125	\\
0.964265104097305	-0.000579833984375	\\
0.96430949527234	0.000152587890625	\\
0.964353886447374	0.0001220703125	\\
0.964398277622409	-0.000396728515625	\\
0.964442668797443	-0.0001220703125	\\
0.964487059972478	-0.000762939453125	\\
0.964531451147512	0.00018310546875	\\
0.964575842322546	0.000274658203125	\\
0.964620233497581	-0.00030517578125	\\
0.964664624672615	0.0003662109375	\\
0.964709015847649	0.0003662109375	\\
0.964753407022684	-0.000396728515625	\\
0.964797798197718	-0.000701904296875	\\
0.964842189372753	-0.000244140625	\\
0.964886580547787	-0.000244140625	\\
0.964930971722822	-0.000274658203125	\\
0.964975362897856	0.00067138671875	\\
0.96501975407289	0.000335693359375	\\
0.965064145247925	0.0001220703125	\\
0.965108536422959	0.000396728515625	\\
0.965152927597994	0.000152587890625	\\
0.965197318773028	0.000244140625	\\
0.965241709948062	-0.0003662109375	\\
0.965286101123097	3.0517578125e-05	\\
0.965330492298131	0.0001220703125	\\
0.965374883473166	-0.000213623046875	\\
0.9654192746482	-0.00030517578125	\\
0.965463665823234	0	\\
0.965508056998269	0.000457763671875	\\
0.965552448173303	-0.0001220703125	\\
0.965596839348338	-6.103515625e-05	\\
0.965641230523372	-0.0003662109375	\\
0.965685621698406	-0.000701904296875	\\
0.965730012873441	-0.000701904296875	\\
0.965774404048475	-0.001251220703125	\\
0.96581879522351	-0.001007080078125	\\
0.965863186398544	-0.000732421875	\\
0.965907577573578	-0.0010986328125	\\
0.965951968748613	-0.00079345703125	\\
0.965996359923647	-0.000732421875	\\
0.966040751098682	-0.001373291015625	\\
0.966085142273716	-0.001007080078125	\\
0.96612953344875	-0.00067138671875	\\
0.966173924623785	-0.00067138671875	\\
0.966218315798819	-0.001129150390625	\\
0.966262706973854	-0.001220703125	\\
0.966307098148888	-0.00091552734375	\\
0.966351489323922	-0.000732421875	\\
0.966395880498957	-0.0006103515625	\\
0.966440271673991	-0.001068115234375	\\
0.966484662849026	-0.00140380859375	\\
0.96652905402406	-0.0013427734375	\\
0.966573445199094	-0.00079345703125	\\
0.966617836374129	-0.000640869140625	\\
0.966662227549163	-0.0003662109375	\\
0.966706618724198	0.00018310546875	\\
0.966751009899232	-0.000335693359375	\\
0.966795401074266	-0.000274658203125	\\
0.966839792249301	-6.103515625e-05	\\
0.966884183424335	-0.000518798828125	\\
0.96692857459937	-0.0003662109375	\\
0.966972965774404	-0.0003662109375	\\
0.967017356949438	0.000152587890625	\\
0.967061748124473	0.000213623046875	\\
0.967106139299507	-0.000152587890625	\\
0.967150530474542	9.1552734375e-05	\\
0.967194921649576	-0.000396728515625	\\
0.967239312824611	-0.000518798828125	\\
0.967283703999645	-0.000396728515625	\\
0.967328095174679	-0.00048828125	\\
0.967372486349714	-0.000762939453125	\\
0.967416877524748	-0.00103759765625	\\
0.967461268699782	-0.0006103515625	\\
0.967505659874817	-0.00042724609375	\\
0.967550051049851	-0.000885009765625	\\
0.967594442224886	-0.000335693359375	\\
0.96763883339992	-0.000640869140625	\\
0.967683224574954	-0.00091552734375	\\
0.967727615749989	-0.000396728515625	\\
0.967772006925023	-0.000579833984375	\\
0.967816398100058	-0.00067138671875	\\
0.967860789275092	-0.000335693359375	\\
0.967905180450127	-0.000579833984375	\\
0.967949571625161	-0.000823974609375	\\
0.967993962800195	-0.000518798828125	\\
0.96803835397523	-0.00091552734375	\\
0.968082745150264	-0.0009765625	\\
0.968127136325299	-0.000823974609375	\\
0.968171527500333	-0.00079345703125	\\
0.968215918675367	-0.000762939453125	\\
0.968260309850402	-0.000946044921875	\\
0.968304701025436	-0.0010986328125	\\
0.968349092200471	-0.001251220703125	\\
0.968393483375505	-0.000946044921875	\\
0.968437874550539	-0.0001220703125	\\
0.968482265725574	-0.000244140625	\\
0.968526656900608	-0.0006103515625	\\
0.968571048075643	-0.00018310546875	\\
0.968615439250677	-0.000335693359375	\\
0.968659830425711	-0.000213623046875	\\
0.968704221600746	-0.000213623046875	\\
0.96874861277578	-0.000457763671875	\\
0.968793003950815	-0.000518798828125	\\
0.968837395125849	-0.001007080078125	\\
0.968881786300883	-0.000640869140625	\\
0.968926177475918	-0.00067138671875	\\
0.968970568650952	-0.001129150390625	\\
0.969014959825987	-0.000762939453125	\\
0.969059351001021	-0.0010986328125	\\
0.969103742176055	-0.00128173828125	\\
0.96914813335109	-0.000885009765625	\\
0.969192524526124	-0.00128173828125	\\
0.969236915701159	-0.000732421875	\\
0.969281306876193	-0.000579833984375	\\
0.969325698051227	-0.000823974609375	\\
0.969370089226262	-0.000640869140625	\\
0.969414480401296	-0.000457763671875	\\
0.969458871576331	-0.000213623046875	\\
0.969503262751365	-0.00030517578125	\\
0.969547653926399	-0.00048828125	\\
0.969592045101434	-0.000701904296875	\\
0.969636436276468	-0.000732421875	\\
0.969680827451503	-0.000732421875	\\
0.969725218626537	-0.000457763671875	\\
0.969769609801571	-0.000823974609375	\\
0.969814000976606	-0.000946044921875	\\
0.96985839215164	-0.001434326171875	\\
0.969902783326675	-0.0010986328125	\\
0.969947174501709	-0.000518798828125	\\
0.969991565676743	-0.000640869140625	\\
0.970035956851778	-0.00115966796875	\\
0.970080348026812	-0.00128173828125	\\
0.970124739201847	-0.001068115234375	\\
0.970169130376881	-0.001129150390625	\\
0.970213521551915	-0.00079345703125	\\
0.97025791272695	-0.00091552734375	\\
0.970302303901984	-0.001129150390625	\\
0.970346695077019	-0.001007080078125	\\
0.970391086252053	-0.001220703125	\\
0.970435477427087	-0.001129150390625	\\
0.970479868602122	-0.001068115234375	\\
0.970524259777156	-0.00128173828125	\\
0.970568650952191	-0.001678466796875	\\
0.970613042127225	-0.00103759765625	\\
0.97065743330226	-0.001220703125	\\
0.970701824477294	-0.00128173828125	\\
0.970746215652328	-0.00103759765625	\\
0.970790606827363	-0.00091552734375	\\
0.970834998002397	-0.000457763671875	\\
0.970879389177432	-0.001129150390625	\\
0.970923780352466	-0.000457763671875	\\
0.9709681715275	-0.00048828125	\\
0.971012562702535	-0.000579833984375	\\
0.971056953877569	-0.000457763671875	\\
0.971101345052604	-0.00079345703125	\\
0.971145736227638	-0.0003662109375	\\
0.971190127402672	-0.00018310546875	\\
0.971234518577707	-0.000244140625	\\
0.971278909752741	0.00018310546875	\\
0.971323300927776	0.000244140625	\\
0.97136769210281	0.00054931640625	\\
0.971412083277844	0.0001220703125	\\
0.971456474452879	0	\\
0.971500865627913	0.00067138671875	\\
0.971545256802948	0.00067138671875	\\
0.971589647977982	0.0008544921875	\\
0.971634039153016	0.000579833984375	\\
0.971678430328051	0.00030517578125	\\
0.971722821503085	0.000518798828125	\\
0.97176721267812	0.000274658203125	\\
0.971811603853154	0.00054931640625	\\
0.971855995028188	0.000274658203125	\\
0.971900386203223	-0.00048828125	\\
0.971944777378257	0	\\
0.971989168553292	-0.00018310546875	\\
0.972033559728326	-0.000213623046875	\\
0.97207795090336	3.0517578125e-05	\\
0.972122342078395	0.000213623046875	\\
0.972166733253429	-0.000152587890625	\\
0.972211124428464	3.0517578125e-05	\\
0.972255515603498	3.0517578125e-05	\\
0.972299906778532	-3.0517578125e-05	\\
0.972344297953567	0.000244140625	\\
0.972388689128601	3.0517578125e-05	\\
0.972433080303636	0.000274658203125	\\
0.97247747147867	0.0009765625	\\
0.972521862653704	0.0008544921875	\\
0.972566253828739	0.000823974609375	\\
0.972610645003773	0.00067138671875	\\
0.972655036178808	0.00048828125	\\
0.972699427353842	0.000732421875	\\
0.972743818528876	0.0009765625	\\
0.972788209703911	0.00091552734375	\\
0.972832600878945	0.000762939453125	\\
0.97287699205398	0.001068115234375	\\
0.972921383229014	0.000823974609375	\\
0.972965774404049	0.00115966796875	\\
0.973010165579083	0.0018310546875	\\
0.973054556754117	0.001129150390625	\\
0.973098947929152	0.0009765625	\\
0.973143339104186	0.0010986328125	\\
0.97318773027922	0.000732421875	\\
0.973232121454255	0.000885009765625	\\
0.973276512629289	0.000885009765625	\\
0.973320903804324	0.00103759765625	\\
0.973365294979358	0.001556396484375	\\
0.973409686154393	0.001617431640625	\\
0.973454077329427	0.001495361328125	\\
0.973498468504461	0.0008544921875	\\
0.973542859679496	0.00103759765625	\\
0.97358725085453	0.000885009765625	\\
0.973631642029565	0.000946044921875	\\
0.973676033204599	0.001312255859375	\\
0.973720424379633	0.00030517578125	\\
0.973764815554668	0.00079345703125	\\
0.973809206729702	0.00091552734375	\\
0.973853597904737	0.000579833984375	\\
0.973897989079771	0.001373291015625	\\
0.973942380254805	0.000762939453125	\\
0.97398677142984	0.000518798828125	\\
0.974031162604874	0.0009765625	\\
0.974075553779909	0.00030517578125	\\
0.974119944954943	6.103515625e-05	\\
0.974164336129977	0.000518798828125	\\
0.974208727305012	0.00054931640625	\\
0.974253118480046	0.000457763671875	\\
0.974297509655081	0.000885009765625	\\
0.974341900830115	0.000823974609375	\\
0.974386292005149	0.000396728515625	\\
0.974430683180184	0.0009765625	\\
0.974475074355218	0.001129150390625	\\
0.974519465530253	0.000823974609375	\\
0.974563856705287	0.000946044921875	\\
0.974608247880321	0.000823974609375	\\
0.974652639055356	0.000885009765625	\\
0.97469703023039	0.0013427734375	\\
0.974741421405425	0.00115966796875	\\
0.974785812580459	0.000762939453125	\\
0.974830203755493	0.000701904296875	\\
0.974874594930528	0.001068115234375	\\
0.974918986105562	0.001190185546875	\\
0.974963377280597	0.001129150390625	\\
0.975007768455631	0.0013427734375	\\
0.975052159630665	0.000946044921875	\\
0.9750965508057	0.001251220703125	\\
0.975140941980734	0.0009765625	\\
0.975185333155769	0.00103759765625	\\
0.975229724330803	0.00140380859375	\\
0.975274115505837	0.00103759765625	\\
0.975318506680872	0.000946044921875	\\
0.975362897855906	0.000885009765625	\\
0.975407289030941	0.000885009765625	\\
0.975451680205975	0.00042724609375	\\
0.975496071381009	0.0003662109375	\\
0.975540462556044	9.1552734375e-05	\\
0.975584853731078	3.0517578125e-05	\\
0.975629244906113	0.00042724609375	\\
0.975673636081147	-0.000274658203125	\\
0.975718027256182	-0.00018310546875	\\
0.975762418431216	-0.000213623046875	\\
0.97580680960625	-0.0006103515625	\\
0.975851200781285	-0.000335693359375	\\
0.975895591956319	-0.00042724609375	\\
0.975939983131353	-0.000213623046875	\\
0.975984374306388	-0.000213623046875	\\
0.976028765481422	-0.000213623046875	\\
0.976073156656457	0.000335693359375	\\
0.976117547831491	0.000152587890625	\\
0.976161939006525	0.000274658203125	\\
0.97620633018156	0.000762939453125	\\
0.976250721356594	0.00079345703125	\\
0.976295112531629	0.0008544921875	\\
0.976339503706663	0.000518798828125	\\
0.976383894881698	0.00091552734375	\\
0.976428286056732	0.000701904296875	\\
0.976472677231766	0.000762939453125	\\
0.976517068406801	0.000823974609375	\\
0.976561459581835	0.000701904296875	\\
0.97660585075687	0.001312255859375	\\
0.976650241931904	0.001678466796875	\\
0.976694633106938	0.001708984375	\\
0.976739024281973	0.001434326171875	\\
0.976783415457007	0.0015869140625	\\
0.976827806632042	0.001922607421875	\\
0.976872197807076	0.0015869140625	\\
0.97691658898211	0.00152587890625	\\
0.976960980157145	0.0013427734375	\\
0.977005371332179	0.001129150390625	\\
0.977049762507214	0.00152587890625	\\
0.977094153682248	0.000946044921875	\\
0.977138544857282	0.001434326171875	\\
0.977182936032317	0.001678466796875	\\
0.977227327207351	0.001434326171875	\\
0.977271718382386	0.001373291015625	\\
0.97731610955742	0.001617431640625	\\
0.977360500732454	0.0015869140625	\\
0.977404891907489	0.00152587890625	\\
0.977449283082523	0.001373291015625	\\
0.977493674257558	0.001190185546875	\\
0.977538065432592	0.000701904296875	\\
0.977582456607626	0.00128173828125	\\
0.977626847782661	0.00146484375	\\
0.977671238957695	0.000762939453125	\\
0.97771563013273	0.000732421875	\\
0.977760021307764	0.00091552734375	\\
0.977804412482798	0.000518798828125	\\
0.977848803657833	0.000579833984375	\\
0.977893194832867	0.00079345703125	\\
0.977937586007902	0.001007080078125	\\
0.977981977182936	0.001007080078125	\\
0.97802636835797	0.000762939453125	\\
0.978070759533005	0.000701904296875	\\
0.978115150708039	0.00067138671875	\\
0.978159541883074	0.000762939453125	\\
0.978203933058108	0.001068115234375	\\
0.978248324233142	0.00091552734375	\\
0.978292715408177	0.00091552734375	\\
0.978337106583211	0.0010986328125	\\
0.978381497758246	0.001068115234375	\\
0.97842588893328	0.001068115234375	\\
0.978470280108314	0.00103759765625	\\
0.978514671283349	0.0009765625	\\
0.978559062458383	0.001220703125	\\
0.978603453633418	0.0015869140625	\\
0.978647844808452	0.001617431640625	\\
0.978692235983487	0.00152587890625	\\
0.978736627158521	0.001129150390625	\\
0.978781018333555	0.0015869140625	\\
0.97882540950859	0.001251220703125	\\
0.978869800683624	0.0009765625	\\
0.978914191858658	0.001068115234375	\\
0.978958583033693	0.00140380859375	\\
0.979002974208727	0.00164794921875	\\
0.979047365383762	0.00177001953125	\\
0.979091756558796	0.00152587890625	\\
0.979136147733831	0.001678466796875	\\
0.979180538908865	0.00177001953125	\\
0.979224930083899	0.00152587890625	\\
0.979269321258934	0.00115966796875	\\
0.979313712433968	0.00115966796875	\\
0.979358103609003	0.001312255859375	\\
0.979402494784037	0.0010986328125	\\
0.979446885959071	0.000946044921875	\\
0.979491277134106	0.000946044921875	\\
0.97953566830914	0.001190185546875	\\
0.979580059484175	0.001190185546875	\\
0.979624450659209	0.000946044921875	\\
0.979668841834243	0.00115966796875	\\
0.979713233009278	0.00128173828125	\\
0.979757624184312	0.001129150390625	\\
0.979802015359347	0.0015869140625	\\
0.979846406534381	0.00177001953125	\\
0.979890797709415	0.001617431640625	\\
0.97993518888445	0.001068115234375	\\
0.979979580059484	0.000946044921875	\\
0.980023971234519	0.0010986328125	\\
0.980068362409553	0.00091552734375	\\
0.980112753584587	0.001190185546875	\\
0.980157144759622	0.00128173828125	\\
0.980201535934656	0.001861572265625	\\
0.980245927109691	0.001922607421875	\\
0.980290318284725	0.001495361328125	\\
0.980334709459759	0.00189208984375	\\
0.980379100634794	0.001556396484375	\\
0.980423491809828	0.001617431640625	\\
0.980467882984863	0.00189208984375	\\
0.980512274159897	0.001708984375	\\
0.980556665334931	0.001678466796875	\\
0.980601056509966	0.001922607421875	\\
0.980645447685	0.001922607421875	\\
0.980689838860035	0.001617431640625	\\
0.980734230035069	0.00177001953125	\\
0.980778621210103	0.001922607421875	\\
0.980823012385138	0.0018310546875	\\
0.980867403560172	0.001800537109375	\\
0.980911794735207	0.001861572265625	\\
0.980956185910241	0.001922607421875	\\
0.981000577085275	0.001800537109375	\\
0.98104496826031	0.00128173828125	\\
0.981089359435344	0.0015869140625	\\
0.981133750610379	0.00177001953125	\\
0.981178141785413	0.00201416015625	\\
0.981222532960447	0.0025634765625	\\
0.981266924135482	0.0025634765625	\\
0.981311315310516	0.002197265625	\\
0.981355706485551	0.00189208984375	\\
0.981400097660585	0.002166748046875	\\
0.98144448883562	0.002105712890625	\\
0.981488880010654	0.00164794921875	\\
0.981533271185688	0.00213623046875	\\
0.981577662360723	0.0020751953125	\\
0.981622053535757	0.002227783203125	\\
0.981666444710791	0.002105712890625	\\
0.981710835885826	0.001251220703125	\\
0.98175522706086	0.001495361328125	\\
0.981799618235895	0.001495361328125	\\
0.981844009410929	0.001373291015625	\\
0.981888400585963	0.001312255859375	\\
0.981932791760998	0.001251220703125	\\
0.981977182936032	0.001678466796875	\\
0.982021574111067	0.001800537109375	\\
0.982065965286101	0.00164794921875	\\
0.982110356461136	0.00152587890625	\\
0.98215474763617	0.0015869140625	\\
0.982199138811204	0.001312255859375	\\
0.982243529986239	0.0008544921875	\\
0.982287921161273	0.000823974609375	\\
0.982332312336308	0.00067138671875	\\
0.982376703511342	0.00079345703125	\\
0.982421094686376	0.00079345703125	\\
0.982465485861411	0.00030517578125	\\
0.982509877036445	0.000762939453125	\\
0.98255426821148	0.00091552734375	\\
0.982598659386514	0.000762939453125	\\
0.982643050561548	0.000518798828125	\\
0.982687441736583	0.000732421875	\\
0.982731832911617	0.000274658203125	\\
0.982776224086652	0.00018310546875	\\
0.982820615261686	0.000732421875	\\
0.98286500643672	0.00030517578125	\\
0.982909397611755	0.000244140625	\\
0.982953788786789	0.000274658203125	\\
0.982998179961824	0.000274658203125	\\
0.983042571136858	0.00079345703125	\\
0.983086962311892	0.00103759765625	\\
0.983131353486927	0.00079345703125	\\
0.983175744661961	0.000885009765625	\\
0.983220135836996	0.00048828125	\\
0.98326452701203	0.00048828125	\\
0.983308918187064	0.000274658203125	\\
0.983353309362099	0.000457763671875	\\
0.983397700537133	0.000640869140625	\\
0.983442091712168	0.00079345703125	\\
0.983486482887202	0.00054931640625	\\
0.983530874062236	0.000213623046875	\\
0.983575265237271	0.000640869140625	\\
0.983619656412305	0.000274658203125	\\
0.98366404758734	0.000335693359375	\\
0.983708438762374	0.00067138671875	\\
0.983752829937408	0.000274658203125	\\
0.983797221112443	0.000701904296875	\\
0.983841612287477	0.000518798828125	\\
0.983886003462512	0.000579833984375	\\
0.983930394637546	0.000579833984375	\\
0.98397478581258	0.000274658203125	\\
0.984019176987615	0.00054931640625	\\
0.984063568162649	9.1552734375e-05	\\
0.984107959337684	0.00030517578125	\\
0.984152350512718	0.000396728515625	\\
0.984196741687753	0	\\
0.984241132862787	6.103515625e-05	\\
0.984285524037821	6.103515625e-05	\\
0.984329915212856	-0.0001220703125	\\
0.98437430638789	-3.0517578125e-05	\\
0.984418697562924	0.0006103515625	\\
0.984463088737959	0.000244140625	\\
0.984507479912993	-0.000152587890625	\\
0.984551871088028	-3.0517578125e-05	\\
0.984596262263062	9.1552734375e-05	\\
0.984640653438096	-9.1552734375e-05	\\
0.984685044613131	-0.0001220703125	\\
0.984729435788165	-0.000518798828125	\\
0.9847738269632	-0.0006103515625	\\
0.984818218138234	-0.00048828125	\\
0.984862609313269	-0.00079345703125	\\
0.984907000488303	-0.0008544921875	\\
0.984951391663337	-0.00048828125	\\
0.984995782838372	-0.000640869140625	\\
0.985040174013406	-0.000518798828125	\\
0.985084565188441	-0.00054931640625	\\
0.985128956363475	-0.00067138671875	\\
0.985173347538509	-0.000579833984375	\\
0.985217738713544	-0.00103759765625	\\
0.985262129888578	-0.001068115234375	\\
0.985306521063613	-0.00115966796875	\\
0.985350912238647	-0.00140380859375	\\
0.985395303413681	-0.001678466796875	\\
0.985439694588716	-0.00152587890625	\\
0.98548408576375	-0.001708984375	\\
0.985528476938785	-0.001556396484375	\\
0.985572868113819	-0.001678466796875	\\
0.985617259288853	-0.0013427734375	\\
0.985661650463888	-0.00146484375	\\
0.985706041638922	-0.001953125	\\
0.985750432813957	-0.00128173828125	\\
0.985794823988991	-0.000732421875	\\
0.985839215164025	-0.00048828125	\\
0.98588360633906	-0.0013427734375	\\
0.985927997514094	-0.001495361328125	\\
0.985972388689129	-0.00091552734375	\\
0.986016779864163	-0.000885009765625	\\
0.986061171039197	-0.00054931640625	\\
0.986105562214232	-0.000732421875	\\
0.986149953389266	-0.000701904296875	\\
0.986194344564301	-0.000701904296875	\\
0.986238735739335	-0.0008544921875	\\
0.986283126914369	-0.000244140625	\\
0.986327518089404	-0.00048828125	\\
0.986371909264438	-0.000946044921875	\\
0.986416300439473	-0.00030517578125	\\
0.986460691614507	6.103515625e-05	\\
0.986505082789541	-0.000213623046875	\\
0.986549473964576	-0.000213623046875	\\
0.98659386513961	-6.103515625e-05	\\
0.986638256314645	-0.000579833984375	\\
0.986682647489679	-0.000457763671875	\\
0.986727038664713	-3.0517578125e-05	\\
0.986771429839748	-0.000213623046875	\\
0.986815821014782	-0.0006103515625	\\
0.986860212189817	-9.1552734375e-05	\\
0.986904603364851	-0.0001220703125	\\
0.986948994539885	-0.000457763671875	\\
0.98699338571492	-0.000732421875	\\
0.987037776889954	-0.000640869140625	\\
0.987082168064989	-0.000244140625	\\
0.987126559240023	-0.000457763671875	\\
0.987170950415058	-0.0001220703125	\\
0.987215341590092	-3.0517578125e-05	\\
0.987259732765126	-0.00030517578125	\\
0.987304123940161	-0.000213623046875	\\
0.987348515115195	-0.000396728515625	\\
0.987392906290229	-0.000457763671875	\\
0.987437297465264	-0.00048828125	\\
0.987481688640298	-0.000732421875	\\
0.987526079815333	-0.00042724609375	\\
0.987570470990367	-0.000152587890625	\\
0.987614862165402	-0.000335693359375	\\
0.987659253340436	-0.0006103515625	\\
0.98770364451547	-0.000579833984375	\\
0.987748035690505	-0.00042724609375	\\
0.987792426865539	-0.000457763671875	\\
0.987836818040574	-9.1552734375e-05	\\
0.987881209215608	0.000152587890625	\\
0.987925600390642	9.1552734375e-05	\\
0.987969991565677	0.000152587890625	\\
0.988014382740711	0.000396728515625	\\
0.988058773915746	-0.0001220703125	\\
0.98810316509078	-0.00018310546875	\\
0.988147556265814	9.1552734375e-05	\\
0.988191947440849	-0.000274658203125	\\
0.988236338615883	-0.00054931640625	\\
0.988280729790918	-0.000335693359375	\\
0.988325120965952	-0.000213623046875	\\
0.988369512140986	-0.000274658203125	\\
0.988413903316021	-9.1552734375e-05	\\
0.988458294491055	-0.00030517578125	\\
0.98850268566609	-0.000701904296875	\\
0.988547076841124	-3.0517578125e-05	\\
0.988591468016158	6.103515625e-05	\\
0.988635859191193	-0.000335693359375	\\
0.988680250366227	-0.000213623046875	\\
0.988724641541262	-0.00018310546875	\\
0.988769032716296	-3.0517578125e-05	\\
0.98881342389133	9.1552734375e-05	\\
0.988857815066365	0.00018310546875	\\
0.988902206241399	0.000244140625	\\
0.988946597416434	0.000335693359375	\\
0.988990988591468	0	\\
0.989035379766502	-0.00048828125	\\
0.989079770941537	-0.000244140625	\\
0.989124162116571	6.103515625e-05	\\
0.989168553291606	0.00018310546875	\\
0.98921294446664	0.0001220703125	\\
0.989257335641674	-0.000244140625	\\
0.989301726816709	-0.000152587890625	\\
0.989346117991743	-0.000213623046875	\\
0.989390509166778	9.1552734375e-05	\\
0.989434900341812	0.00042724609375	\\
0.989479291516846	3.0517578125e-05	\\
0.989523682691881	-0.000152587890625	\\
0.989568073866915	-0.000213623046875	\\
0.98961246504195	-0.0003662109375	\\
0.989656856216984	-0.000244140625	\\
0.989701247392018	-0.000244140625	\\
0.989745638567053	-0.000518798828125	\\
0.989790029742087	-0.0009765625	\\
0.989834420917122	-0.000885009765625	\\
0.989878812092156	-0.000457763671875	\\
0.989923203267191	-0.00030517578125	\\
0.989967594442225	-0.000732421875	\\
0.990011985617259	-0.000701904296875	\\
0.990056376792294	-0.000152587890625	\\
0.990100767967328	6.103515625e-05	\\
0.990145159142362	-0.000335693359375	\\
0.990189550317397	-0.000396728515625	\\
0.990233941492431	0	\\
0.990278332667466	0.000152587890625	\\
0.9903227238425	0.000518798828125	\\
0.990367115017534	0.000518798828125	\\
0.990411506192569	6.103515625e-05	\\
0.990455897367603	-0.00018310546875	\\
0.990500288542638	-0.0001220703125	\\
0.990544679717672	3.0517578125e-05	\\
0.990589070892707	0.000213623046875	\\
0.990633462067741	-0.000213623046875	\\
0.990677853242775	-9.1552734375e-05	\\
0.99072224441781	9.1552734375e-05	\\
0.990766635592844	0.000213623046875	\\
0.990811026767879	0.00054931640625	\\
0.990855417942913	0.0003662109375	\\
0.990899809117947	0.00030517578125	\\
0.990944200292982	0.000274658203125	\\
0.990988591468016	0.000457763671875	\\
0.991032982643051	0.000335693359375	\\
0.991077373818085	0.000213623046875	\\
0.991121764993119	0.0001220703125	\\
0.991166156168154	-0.0001220703125	\\
0.991210547343188	-0.000152587890625	\\
0.991254938518223	-0.000274658203125	\\
0.991299329693257	-0.000274658203125	\\
0.991343720868291	0.00030517578125	\\
0.991388112043326	-0.00030517578125	\\
0.99143250321836	-0.000518798828125	\\
0.991476894393395	-0.00042724609375	\\
0.991521285568429	-0.000244140625	\\
0.991565676743463	-0.000244140625	\\
0.991610067918498	-0.000457763671875	\\
0.991654459093532	-0.00018310546875	\\
0.991698850268567	-0.000244140625	\\
0.991743241443601	-0.00054931640625	\\
0.991787632618635	-0.000518798828125	\\
0.99183202379367	-0.00048828125	\\
0.991876414968704	-0.000213623046875	\\
0.991920806143739	-0.00018310546875	\\
0.991965197318773	0.00018310546875	\\
0.992009588493807	3.0517578125e-05	\\
0.992053979668842	-0.000152587890625	\\
0.992098370843876	-0.0001220703125	\\
0.992142762018911	-0.00018310546875	\\
0.992187153193945	-0.000244140625	\\
0.992231544368979	-0.00018310546875	\\
0.992275935544014	-0.00054931640625	\\
0.992320326719048	-0.00103759765625	\\
0.992364717894083	-0.000946044921875	\\
0.992409109069117	-0.0006103515625	\\
0.992453500244151	-0.00079345703125	\\
0.992497891419186	-0.000732421875	\\
0.99254228259422	-0.000640869140625	\\
0.992586673769255	-0.000732421875	\\
0.992631064944289	-0.000823974609375	\\
0.992675456119324	-0.00067138671875	\\
0.992719847294358	-0.000732421875	\\
0.992764238469392	-0.0008544921875	\\
0.992808629644427	-0.00091552734375	\\
0.992853020819461	-0.001007080078125	\\
0.992897411994496	-0.000579833984375	\\
0.99294180316953	-0.000732421875	\\
0.992986194344564	-0.001007080078125	\\
0.993030585519599	-0.000823974609375	\\
0.993074976694633	-0.000762939453125	\\
0.993119367869667	-0.00079345703125	\\
0.993163759044702	-0.000823974609375	\\
0.993208150219736	-0.000885009765625	\\
0.993252541394771	-0.0009765625	\\
0.993296932569805	-0.0008544921875	\\
0.99334132374484	-0.0006103515625	\\
0.993385714919874	-0.00067138671875	\\
0.993430106094908	-0.000335693359375	\\
0.993474497269943	-0.000457763671875	\\
0.993518888444977	-0.000946044921875	\\
0.993563279620012	-0.00103759765625	\\
0.993607670795046	-0.000823974609375	\\
0.99365206197008	-0.00103759765625	\\
0.993696453145115	-0.00140380859375	\\
0.993740844320149	-0.00091552734375	\\
0.993785235495184	-0.0008544921875	\\
0.993829626670218	-0.00115966796875	\\
0.993874017845252	-0.000762939453125	\\
0.993918409020287	-0.000579833984375	\\
0.993962800195321	-0.000274658203125	\\
0.994007191370356	-0.000335693359375	\\
0.99405158254539	-0.000640869140625	\\
0.994095973720424	-0.00054931640625	\\
0.994140364895459	-0.000518798828125	\\
0.994184756070493	-0.000335693359375	\\
0.994229147245528	-0.000457763671875	\\
0.994273538420562	-0.000244140625	\\
0.994317929595596	-0.000274658203125	\\
0.994362320770631	-0.000701904296875	\\
0.994406711945665	-0.00048828125	\\
0.9944511031207	-0.000518798828125	\\
0.994495494295734	-0.0003662109375	\\
0.994539885470768	9.1552734375e-05	\\
0.994584276645803	-0.000152587890625	\\
0.994628667820837	-0.000152587890625	\\
0.994673058995872	9.1552734375e-05	\\
0.994717450170906	0	\\
0.99476184134594	0.00030517578125	\\
0.994806232520975	0.0006103515625	\\
0.994850623696009	0.000274658203125	\\
0.994895014871044	-0.000213623046875	\\
0.994939406046078	-0.000518798828125	\\
0.994983797221112	-0.000396728515625	\\
0.995028188396147	-0.00030517578125	\\
0.995072579571181	-0.000152587890625	\\
0.995116970746216	-0.000244140625	\\
0.99516136192125	3.0517578125e-05	\\
0.995205753096284	-0.0003662109375	\\
0.995250144271319	-0.000640869140625	\\
0.995294535446353	-0.000396728515625	\\
0.995338926621388	-0.0003662109375	\\
0.995383317796422	-3.0517578125e-05	\\
0.995427708971456	-0.000213623046875	\\
0.995472100146491	0.0001220703125	\\
0.995516491321525	0.000579833984375	\\
0.99556088249656	9.1552734375e-05	\\
0.995605273671594	0.0006103515625	\\
0.995649664846629	0.00079345703125	\\
0.995694056021663	0.000457763671875	\\
0.995738447196697	0.000732421875	\\
0.995782838371732	0.00067138671875	\\
0.995827229546766	0.000640869140625	\\
0.9958716207218	0.00067138671875	\\
0.995916011896835	0.000823974609375	\\
0.995960403071869	0.000823974609375	\\
0.996004794246904	0.001129150390625	\\
0.996049185421938	0.00091552734375	\\
0.996093576596973	0.001007080078125	\\
0.996137967772007	0.001434326171875	\\
0.996182358947041	0.001007080078125	\\
0.996226750122076	0.001129150390625	\\
0.99627114129711	0.001678466796875	\\
0.996315532472145	0.00213623046875	\\
0.996359923647179	0.001708984375	\\
0.996404314822213	0.0013427734375	\\
0.996448705997248	0.001434326171875	\\
0.996493097172282	0.001312255859375	\\
0.996537488347317	0.00146484375	\\
0.996581879522351	0.0018310546875	\\
0.996626270697385	0.001495361328125	\\
0.99667066187242	0.001434326171875	\\
0.996715053047454	0.001556396484375	\\
0.996759444222489	0.001983642578125	\\
0.996803835397523	0.001922607421875	\\
0.996848226572557	0.001556396484375	\\
0.996892617747592	0.001556396484375	\\
0.996937008922626	0.001678466796875	\\
0.996981400097661	0.001220703125	\\
0.997025791272695	0.001312255859375	\\
0.997070182447729	0.001617431640625	\\
0.997114573622764	0.001373291015625	\\
0.997158964797798	0.00115966796875	\\
0.997203355972833	0.00079345703125	\\
0.997247747147867	0.00115966796875	\\
0.997292138322901	0.00115966796875	\\
0.997336529497936	0.000823974609375	\\
0.99738092067297	0.00115966796875	\\
0.997425311848005	0.00079345703125	\\
0.997469703023039	0.0009765625	\\
0.997514094198073	0.00146484375	\\
0.997558485373108	0.001312255859375	\\
0.997602876548142	0.0009765625	\\
0.997647267723177	0.0010986328125	\\
0.997691658898211	0.00115966796875	\\
0.997736050073245	0.000701904296875	\\
0.99778044124828	0.0010986328125	\\
0.997824832423314	0.001251220703125	\\
0.997869223598349	0.000823974609375	\\
0.997913614773383	0.001007080078125	\\
0.997958005948417	0.00091552734375	\\
0.998002397123452	0.0008544921875	\\
0.998046788298486	0.001434326171875	\\
0.998091179473521	0.001678466796875	\\
0.998135570648555	0.00103759765625	\\
0.998179961823589	0.00115966796875	\\
0.998224352998624	0.0013427734375	\\
0.998268744173658	0.00091552734375	\\
0.998313135348693	0.000823974609375	\\
0.998357526523727	0.001007080078125	\\
0.998401917698762	0.000823974609375	\\
0.998446308873796	0.0009765625	\\
0.99849070004883	0.001434326171875	\\
0.998535091223865	0.00146484375	\\
0.998579482398899	0.00128173828125	\\
0.998623873573933	0.0010986328125	\\
0.998668264748968	0.000640869140625	\\
0.998712655924002	0.000152587890625	\\
0.998757047099037	0.00128173828125	\\
0.998801438274071	0.001434326171875	\\
0.998845829449105	0.000732421875	\\
0.99889022062414	0.000518798828125	\\
0.998934611799174	0.00048828125	\\
0.998979002974209	0.0006103515625	\\
0.999023394149243	0.0006103515625	\\
0.999067785324278	0.0001220703125	\\
0.999112176499312	9.1552734375e-05	\\
0.999156567674346	6.103515625e-05	\\
0.999200958849381	-0.000396728515625	\\
0.999245350024415	-0.0003662109375	\\
0.99928974119945	0	\\
0.999334132374484	3.0517578125e-05	\\
0.999378523549518	-0.000213623046875	\\
0.999422914724553	-0.000518798828125	\\
0.999467305899587	-0.0003662109375	\\
0.999511697074622	-9.1552734375e-05	\\
0.999556088249656	-0.000396728515625	\\
0.99960047942469	-0.000335693359375	\\
0.999644870599725	-0.000274658203125	\\
0.999689261774759	-0.000213623046875	\\
0.999733652949794	-0.0003662109375	\\
0.999778044124828	-0.000335693359375	\\
0.999822435299862	-0.00030517578125	\\
0.999866826474897	-0.000335693359375	\\
0.999911217649931	-0.000579833984375	\\
0.999955608824966	-0.000457763671875	\\
1	-0.000335693359375	\\
};
\end{axis}

\begin{axis}[%
width=\figurewidth,
height=\figureheight,
scale only axis,
xmin=-15000,
xmax=15000,
xlabel={Frequency (in hertz)},
ymin=0,
ymax=0.0006,
at=(plot1.below south west),
anchor=above north west,
title={Magnitude Response}
]
\addplot [color=blue,solid,forget plot]
  table[row sep=crcr]{
-11025	3.65755774758079e-08	\\
-11024.0212180398	8.5283295000321e-07	\\
-11023.0424360795	3.28821786877167e-07	\\
-11022.0636541193	9.00510737428282e-08	\\
-11021.0848721591	2.29444384043337e-07	\\
-11020.1060901989	3.79275570135795e-07	\\
-11019.1273082386	2.19229584342878e-07	\\
-11018.1485262784	4.98346804249498e-07	\\
-11017.1697443182	3.64168679122739e-07	\\
-11016.190962358	2.10027945197449e-07	\\
-11015.2121803977	2.25089304174186e-07	\\
-11014.2333984375	4.04670548691998e-07	\\
-11013.2546164773	3.70594569581616e-07	\\
-11012.275834517	2.83200744248436e-07	\\
-11011.2970525568	8.47734481404522e-07	\\
-11010.3182705966	3.04171828736175e-07	\\
-11009.3394886364	3.44841127045598e-07	\\
-11008.3607066761	2.42362992681971e-07	\\
-11007.3819247159	1.52838968460744e-07	\\
-11006.4031427557	2.51140361592898e-07	\\
-11005.4243607955	7.57859302145074e-07	\\
-11004.4455788352	2.06200203960869e-07	\\
-11003.466796875	3.78434951901723e-07	\\
-11002.4880149148	2.80178814408448e-07	\\
-11001.5092329545	4.94982095651224e-07	\\
-11000.5304509943	5.27118194910882e-07	\\
-10999.5516690341	4.88493713635106e-07	\\
-10998.5728870739	2.4764631479494e-07	\\
-10997.5941051136	3.7411185714794e-07	\\
-10996.6153231534	4.32986560159093e-07	\\
-10995.6365411932	3.28363532472297e-07	\\
-10994.657759233	5.41636748686034e-07	\\
-10993.6789772727	5.97634328985684e-07	\\
-10992.7001953125	3.2306652927595e-07	\\
-10991.7214133523	2.89554152672428e-07	\\
-10990.742631392	2.9977458020177e-07	\\
-10989.7638494318	1.55699688157437e-07	\\
-10988.7850674716	3.7925395990644e-07	\\
-10987.8062855114	7.81840294287668e-08	\\
-10986.8275035511	3.14915091473333e-07	\\
-10985.8487215909	2.51175881719678e-07	\\
-10984.8699396307	9.11216474088275e-08	\\
-10983.8911576705	6.05003077357967e-07	\\
-10982.9123757102	3.64330793654919e-07	\\
-10981.93359375	4.15286305556761e-07	\\
-10980.9548117898	4.38144840426988e-07	\\
-10979.9760298295	3.16368330504967e-07	\\
-10978.9972478693	5.92382681262396e-07	\\
-10978.0184659091	4.36734092733169e-07	\\
-10977.0396839489	3.68769151934774e-07	\\
-10976.0609019886	2.8016231071612e-07	\\
-10975.0821200284	6.63015433203995e-07	\\
-10974.1033380682	4.55046982334573e-07	\\
-10973.124556108	4.51753168754172e-07	\\
-10972.1457741477	2.25916731747243e-07	\\
-10971.1669921875	5.45190301795267e-07	\\
-10970.1882102273	3.59008223369039e-07	\\
-10969.209428267	5.88185803391748e-07	\\
-10968.2306463068	1.94406626403223e-07	\\
-10967.2518643466	2.76131671337772e-07	\\
-10966.2730823864	5.40308800731858e-07	\\
-10965.2943004261	2.19294898741985e-07	\\
-10964.3155184659	5.03716996860143e-07	\\
-10963.3367365057	5.37347074259336e-07	\\
-10962.3579545455	7.86232658267853e-07	\\
-10961.3791725852	3.63516591028026e-07	\\
-10960.400390625	3.1618288420306e-07	\\
-10959.4216086648	3.49172638169653e-07	\\
-10958.4428267045	2.69393449974596e-07	\\
-10957.4640447443	1.4699708428986e-07	\\
-10956.4852627841	3.02889149056414e-07	\\
-10955.5064808239	2.97311296107927e-07	\\
-10954.5276988636	4.8568511542049e-07	\\
-10953.5489169034	4.83050489112396e-07	\\
-10952.5701349432	3.89218641022002e-08	\\
-10951.591352983	6.38419933873003e-07	\\
-10950.6125710227	6.13701644370175e-07	\\
-10949.6337890625	7.66868575529385e-07	\\
-10948.6550071023	5.47005566094221e-07	\\
-10947.676225142	5.35361028450959e-07	\\
-10946.6974431818	6.29323114139498e-07	\\
-10945.7186612216	5.13463792217618e-07	\\
-10944.7398792614	3.99119321171895e-08	\\
-10943.7610973011	5.90865511610815e-07	\\
-10942.7823153409	9.13713992697528e-08	\\
-10941.8035333807	2.9910247133764e-07	\\
-10940.8247514205	2.63173222766562e-07	\\
-10939.8459694602	7.65161402308435e-07	\\
-10938.8671875	8.27242355275961e-07	\\
-10937.8884055398	3.87152537930741e-07	\\
-10936.9096235795	5.63561389852634e-07	\\
-10935.9308416193	5.99821574399464e-07	\\
-10934.9520596591	2.37898750687519e-07	\\
-10933.9732776989	4.3680195714707e-07	\\
-10932.9944957386	2.32420893860526e-07	\\
-10932.0157137784	1.34206815890962e-07	\\
-10931.0369318182	5.86867656464031e-07	\\
-10930.058149858	5.88049385406655e-07	\\
-10929.0793678977	4.3904419415184e-07	\\
-10928.1005859375	7.0322877463093e-08	\\
-10927.1218039773	1.8953780199444e-07	\\
-10926.143022017	1.33370385223249e-07	\\
-10925.1642400568	4.97462545054451e-07	\\
-10924.1854580966	2.26397627486676e-07	\\
-10923.2066761364	4.07724092991029e-07	\\
-10922.2278941761	2.59732049416056e-07	\\
-10921.2491122159	3.48859536625154e-07	\\
-10920.2703302557	2.48217804632954e-07	\\
-10919.2915482955	4.82287304258407e-07	\\
-10918.3127663352	5.50606053662159e-07	\\
-10917.333984375	6.8649309601978e-07	\\
-10916.3552024148	6.56072801927349e-07	\\
-10915.3764204545	5.35859938765717e-07	\\
-10914.3976384943	3.4991520457526e-07	\\
-10913.4188565341	3.94948024819285e-07	\\
-10912.4400745739	5.34896746315888e-07	\\
-10911.4612926136	5.24227070451293e-07	\\
-10910.4825106534	1.95105673252692e-07	\\
-10909.5037286932	7.60894441375143e-07	\\
-10908.524946733	2.78564345437554e-07	\\
-10907.5461647727	8.04454991260035e-07	\\
-10906.5673828125	6.76054239537419e-07	\\
-10905.5886008523	2.38189625279387e-07	\\
-10904.609818892	3.03958655208556e-07	\\
-10903.6310369318	5.87264891644337e-07	\\
-10902.6522549716	9.58814699669128e-08	\\
-10901.6734730114	3.84503011268006e-07	\\
-10900.6946910511	3.150818198516e-07	\\
-10899.7159090909	1.26913728752086e-07	\\
-10898.7371271307	1.36189620053965e-07	\\
-10897.7583451705	6.31096313942254e-07	\\
-10896.7795632102	3.00504664130681e-07	\\
-10895.80078125	6.70845407387737e-07	\\
-10894.8219992898	2.29044159263299e-07	\\
-10893.8432173295	4.70012836830117e-07	\\
-10892.8644353693	7.30345933734798e-07	\\
-10891.8856534091	5.51075430947492e-07	\\
-10890.9068714489	5.88778256131791e-07	\\
-10889.9280894886	4.42842907738643e-07	\\
-10888.9493075284	4.77692693085698e-07	\\
-10887.9705255682	2.43753081324796e-07	\\
-10886.991743608	3.28615826678686e-07	\\
-10886.0129616477	1.68850598689432e-07	\\
-10885.0341796875	7.53055617548117e-07	\\
-10884.0553977273	6.2489734208721e-07	\\
-10883.076615767	5.58255797821834e-07	\\
-10882.0978338068	4.19936792273434e-07	\\
-10881.1190518466	2.48676846059071e-07	\\
-10880.1402698864	6.03145984343775e-07	\\
-10879.1614879261	4.93264163967701e-07	\\
-10878.1827059659	8.48252847154293e-07	\\
-10877.2039240057	7.85609066137346e-07	\\
-10876.2251420455	3.96068921274439e-07	\\
-10875.2463600852	7.46532657953056e-07	\\
-10874.267578125	2.88640285242046e-07	\\
-10873.2887961648	2.03316264337818e-07	\\
-10872.3100142045	3.29795002794599e-07	\\
-10871.3312322443	5.91916879355285e-07	\\
-10870.3524502841	6.41844635139583e-07	\\
-10869.3736683239	4.33188977100943e-07	\\
-10868.3948863636	7.05773836439776e-07	\\
-10867.4161044034	7.46158012096973e-07	\\
-10866.4373224432	4.19935160752029e-07	\\
-10865.458540483	2.30633768052115e-07	\\
-10864.4797585227	7.40011797289545e-07	\\
-10863.5009765625	6.80043900321805e-07	\\
-10862.5221946023	4.16772791784408e-07	\\
-10861.543412642	1.02544596008071e-06	\\
-10860.5646306818	3.1372765012197e-07	\\
-10859.5858487216	3.02032101171246e-07	\\
-10858.6070667614	3.44030558553592e-07	\\
-10857.6282848011	2.71982413021743e-07	\\
-10856.6495028409	4.28987198998434e-07	\\
-10855.6707208807	7.43166779237395e-07	\\
-10854.6919389205	8.38073794026177e-07	\\
-10853.7131569602	4.027936235081e-07	\\
-10852.734375	3.21115611862156e-07	\\
-10851.7555930398	6.89993773908024e-07	\\
-10850.7768110795	4.63742977635082e-07	\\
-10849.7980291193	7.64619722206479e-07	\\
-10848.8192471591	5.60211827639894e-07	\\
-10847.8404651989	4.31976853330059e-07	\\
-10846.8616832386	8.1603598247136e-07	\\
-10845.8829012784	8.14705024574643e-07	\\
-10844.9041193182	4.90984803187125e-07	\\
-10843.925337358	1.51458532260998e-07	\\
-10842.9465553977	5.22611664638098e-07	\\
-10841.9677734375	4.52116324578669e-07	\\
-10840.9889914773	8.5334829071835e-07	\\
-10840.010209517	8.18811016847649e-07	\\
-10839.0314275568	7.36623950824119e-07	\\
-10838.0526455966	7.97791239967988e-07	\\
-10837.0738636364	7.13681714127199e-07	\\
-10836.0950816761	6.94885743559264e-07	\\
-10835.1162997159	1.05103283187549e-06	\\
-10834.1375177557	6.10547384243788e-07	\\
-10833.1587357955	5.63088671267615e-07	\\
-10832.1799538352	5.24986712663484e-07	\\
-10831.201171875	1.01368905557281e-06	\\
-10830.2223899148	6.37210441287078e-07	\\
-10829.2436079545	3.91798467704527e-07	\\
-10828.2648259943	6.40165685555542e-07	\\
-10827.2860440341	6.65744402706821e-07	\\
-10826.3072620739	6.07710959482608e-07	\\
-10825.3284801136	6.26533714006038e-07	\\
-10824.3496981534	4.91330536468657e-07	\\
-10823.3709161932	6.36323526818899e-07	\\
-10822.392134233	4.64780117783069e-07	\\
-10821.4133522727	2.14744448249433e-07	\\
-10820.4345703125	9.84719777862699e-07	\\
-10819.4557883523	9.85101185056617e-07	\\
-10818.477006392	5.67679017546585e-07	\\
-10817.4982244318	5.65283842419274e-07	\\
-10816.5194424716	9.68176054751299e-07	\\
-10815.5406605114	4.93968265294121e-07	\\
-10814.5618785511	6.16404224311963e-07	\\
-10813.5830965909	4.49319212657914e-07	\\
-10812.6043146307	3.91064709042655e-07	\\
-10811.6255326705	9.18531631868213e-07	\\
-10810.6467507102	8.88175003335139e-07	\\
-10809.66796875	8.15055610375864e-07	\\
-10808.6891867898	7.59067984190636e-07	\\
-10807.7104048295	1.04405085926928e-06	\\
-10806.7316228693	5.05692172860619e-07	\\
-10805.7528409091	6.37694475557133e-07	\\
-10804.7740589489	6.8418080544613e-07	\\
-10803.7952769886	9.47217945023172e-07	\\
-10802.8164950284	7.79905223922409e-07	\\
-10801.8377130682	1.00312275617065e-06	\\
-10800.858931108	9.04437895402009e-07	\\
-10799.8801491477	9.00064849056588e-07	\\
-10798.9013671875	1.01734522230241e-06	\\
-10797.9225852273	6.44058227423301e-07	\\
-10796.943803267	9.00724782989925e-07	\\
-10795.9650213068	7.08030233286154e-07	\\
-10794.9862393466	5.48007835111423e-07	\\
-10794.0074573864	2.77467707456456e-07	\\
-10793.0286754261	9.34126328161991e-07	\\
-10792.0498934659	5.45726871474695e-07	\\
-10791.0711115057	4.62837141528411e-07	\\
-10790.0923295455	8.2898047102129e-07	\\
-10789.1135475852	9.1545983557036e-07	\\
-10788.134765625	5.14305522986097e-07	\\
-10787.1559836648	8.38853580613185e-07	\\
-10786.1772017045	5.12862090619045e-07	\\
-10785.1984197443	7.43618786478437e-07	\\
-10784.2196377841	7.57683451436334e-07	\\
-10783.2408558239	9.11587911226437e-07	\\
-10782.2620738636	3.69589733666551e-07	\\
-10781.2832919034	1.05558642656358e-06	\\
-10780.3045099432	1.00898331661666e-06	\\
-10779.325727983	6.0571065717384e-07	\\
-10778.3469460227	6.62429837006691e-07	\\
-10777.3681640625	7.87145854193836e-07	\\
-10776.3893821023	1.0254048914695e-06	\\
-10775.410600142	8.39457474046841e-07	\\
-10774.4318181818	9.98632292006535e-07	\\
-10773.4530362216	9.7054106468102e-07	\\
-10772.4742542614	1.24884504218281e-06	\\
-10771.4954723011	7.58365737155929e-07	\\
-10770.5166903409	5.79041739675611e-07	\\
-10769.5379083807	1.32888618034065e-06	\\
-10768.5591264205	3.75995221469226e-07	\\
-10767.5803444602	8.7895403890997e-07	\\
-10766.6015625	1.27908398953577e-06	\\
-10765.6227805398	8.54839848130122e-07	\\
-10764.6439985795	1.11318584689428e-06	\\
-10763.6652166193	1.08975162695273e-06	\\
-10762.6864346591	1.00509560634686e-06	\\
-10761.7076526989	9.3864135657436e-07	\\
-10760.7288707386	1.26095263104637e-06	\\
-10759.7500887784	8.44990221001518e-07	\\
-10758.7713068182	1.2386821277501e-06	\\
-10757.792524858	9.61227745237156e-07	\\
-10756.8137428977	9.91631887930021e-07	\\
-10755.8349609375	1.11934745179089e-06	\\
-10754.8561789773	1.23612195886143e-06	\\
-10753.877397017	6.5244706607216e-07	\\
-10752.8986150568	1.24573427612309e-06	\\
-10751.9198330966	1.18355389043263e-06	\\
-10750.9410511364	1.12815183605704e-06	\\
-10749.9622691761	3.65427176783356e-07	\\
-10748.9834872159	1.28860895663453e-06	\\
-10748.0047052557	6.16235708044255e-07	\\
-10747.0259232955	9.37489975277976e-07	\\
-10746.0471413352	1.19037927423629e-06	\\
-10745.068359375	9.82420166418912e-07	\\
-10744.0895774148	1.15393071181095e-06	\\
-10743.1107954545	6.07183425561055e-07	\\
-10742.1320134943	1.30468772522717e-06	\\
-10741.1532315341	1.10372502340791e-06	\\
-10740.1744495739	9.89732816769167e-07	\\
-10739.1956676136	9.57740969131759e-07	\\
-10738.2168856534	1.22219487599005e-06	\\
-10737.2381036932	9.75337098596015e-07	\\
-10736.259321733	1.12916416041815e-06	\\
-10735.2805397727	1.30930716740117e-06	\\
-10734.3017578125	8.48855373767657e-07	\\
-10733.3229758523	1.36715216441716e-06	\\
-10732.344193892	8.45755299461503e-07	\\
-10731.3654119318	9.97537204553161e-07	\\
-10730.3866299716	1.04038242996062e-06	\\
-10729.4078480114	1.08725862536098e-06	\\
-10728.4290660511	1.2927941868331e-06	\\
-10727.4502840909	1.4671861958572e-06	\\
-10726.4715021307	1.19721932179289e-06	\\
-10725.4927201705	8.60392844247021e-07	\\
-10724.5139382102	1.03222422547113e-06	\\
-10723.53515625	1.50026177305607e-06	\\
-10722.5563742898	9.67659603305513e-07	\\
-10721.5775923295	1.63259095041092e-06	\\
-10720.5988103693	1.07180565794112e-06	\\
-10719.6200284091	1.40502744721593e-06	\\
-10718.6412464489	1.06741651737953e-06	\\
-10717.6624644886	8.14409704676653e-07	\\
-10716.6836825284	1.26965958092721e-06	\\
-10715.7049005682	1.23167654442333e-06	\\
-10714.726118608	1.2601101267707e-06	\\
-10713.7473366477	9.77522194505134e-07	\\
-10712.7685546875	1.29442665097337e-06	\\
-10711.7897727273	1.41966596743361e-06	\\
-10710.810990767	9.16240131189627e-07	\\
-10709.8322088068	1.73058090228391e-06	\\
-10708.8534268466	1.08877229927846e-06	\\
-10707.8746448864	1.14187862905322e-06	\\
-10706.8958629261	1.30572386954599e-06	\\
-10705.9170809659	9.86821755476455e-07	\\
-10704.9382990057	1.65227972084526e-06	\\
-10703.9595170455	1.19302614795384e-06	\\
-10702.9807350852	1.04588835517016e-06	\\
-10702.001953125	1.10173874293199e-06	\\
-10701.0231711648	1.33825834521639e-06	\\
-10700.0443892045	1.67536500901278e-06	\\
-10699.0656072443	1.06781913475083e-06	\\
-10698.0868252841	1.43398048567928e-06	\\
-10697.1080433239	1.69358676703615e-06	\\
-10696.1292613636	1.20052852309726e-06	\\
-10695.1504794034	1.33957335216401e-06	\\
-10694.1716974432	1.50662164067877e-06	\\
-10693.192915483	8.82110626604122e-07	\\
-10692.2141335227	1.16942198000866e-06	\\
-10691.2353515625	1.47617688360457e-06	\\
-10690.2565696023	1.19285764255101e-06	\\
-10689.277787642	1.57649054641036e-06	\\
-10688.2990056818	1.65850159980157e-06	\\
-10687.3202237216	7.71060806346728e-07	\\
-10686.3414417614	1.42683345294343e-06	\\
-10685.3626598011	1.4055994687876e-06	\\
-10684.3838778409	1.16170991135865e-06	\\
-10683.4050958807	1.26193765566298e-06	\\
-10682.4263139205	1.2417232157558e-06	\\
-10681.4475319602	1.30353660362046e-06	\\
-10680.46875	1.73687770777158e-06	\\
-10679.4899680398	1.67759378519346e-06	\\
-10678.5111860795	1.48822303604508e-06	\\
-10677.5324041193	1.52866183864089e-06	\\
-10676.5536221591	1.75335145002272e-06	\\
-10675.5748401989	1.37734936430779e-06	\\
-10674.5960582386	1.1902995533346e-06	\\
-10673.6172762784	1.28383629926929e-06	\\
-10672.6384943182	1.32387846833265e-06	\\
-10671.659712358	1.24067853248755e-06	\\
-10670.6809303977	1.57159653213359e-06	\\
-10669.7021484375	9.13608177688431e-07	\\
-10668.7233664773	1.14894797155956e-06	\\
-10667.744584517	1.42157940369882e-06	\\
-10666.7658025568	8.30211881466399e-07	\\
-10665.7870205966	1.61957233524244e-06	\\
-10664.8082386364	1.73167638854661e-06	\\
-10663.8294566761	1.1704491146475e-06	\\
-10662.8506747159	1.33912049861322e-06	\\
-10661.8718927557	1.63228575335038e-06	\\
-10660.8931107955	1.00225970230419e-06	\\
-10659.9143288352	1.70821490130484e-06	\\
-10658.935546875	2.02742921219631e-06	\\
-10657.9567649148	1.09822409293743e-06	\\
-10656.9779829545	1.65564341705581e-06	\\
-10655.9992009943	1.76994153740273e-06	\\
-10655.0204190341	9.04253723720971e-07	\\
-10654.0416370739	1.48145288745097e-06	\\
-10653.0628551136	1.69379324969293e-06	\\
-10652.0840731534	1.13634803490281e-06	\\
-10651.1052911932	1.03176986725912e-06	\\
-10650.126509233	1.3227965836687e-06	\\
-10649.1477272727	1.62001105224543e-06	\\
-10648.1689453125	1.10155790472186e-06	\\
-10647.1901633523	1.69944367946981e-06	\\
-10646.211381392	1.78498748488861e-06	\\
-10645.2325994318	1.39733523735248e-06	\\
-10644.2538174716	1.63794178017728e-06	\\
-10643.2750355114	1.5822261099024e-06	\\
-10642.2962535511	1.34799907033629e-06	\\
-10641.3174715909	1.50195924231419e-06	\\
-10640.3386896307	1.45013669199126e-06	\\
-10639.3599076705	1.17352013548551e-06	\\
-10638.3811257102	1.67141288859236e-06	\\
-10637.40234375	1.58298195874135e-06	\\
-10636.4235617898	1.37782937329565e-06	\\
-10635.4447798295	2.43899219215802e-06	\\
-10634.4659978693	1.64527107556353e-06	\\
-10633.4872159091	1.71720617946389e-06	\\
-10632.5084339489	1.50594866617581e-06	\\
-10631.5296519886	1.42159984404236e-06	\\
-10630.5508700284	1.44027713413677e-06	\\
-10629.5720880682	1.59374127379842e-06	\\
-10628.593306108	1.75919750075172e-06	\\
-10627.6145241477	1.73680857112471e-06	\\
-10626.6357421875	1.45845676301864e-06	\\
-10625.6569602273	1.48377893627388e-06	\\
-10624.678178267	1.66334350744885e-06	\\
-10623.6993963068	1.65917104633973e-06	\\
-10622.7206143466	1.43116468578192e-06	\\
-10621.7418323864	1.40748208389579e-06	\\
-10620.7630504261	1.81988073883918e-06	\\
-10619.7842684659	1.54629727745514e-06	\\
-10618.8054865057	1.85143342192056e-06	\\
-10617.8267045455	1.84879501770855e-06	\\
-10616.8479225852	1.72223216506588e-06	\\
-10615.869140625	1.35831363988905e-06	\\
-10614.8903586648	1.67087806847268e-06	\\
-10613.9115767045	1.76786411640753e-06	\\
-10612.9327947443	1.61209612370618e-06	\\
-10611.9540127841	1.45543331468415e-06	\\
-10610.9752308239	1.30868830222978e-06	\\
-10609.9964488636	1.58169332205749e-06	\\
-10609.0176669034	1.59830242209859e-06	\\
-10608.0388849432	1.61745979601552e-06	\\
-10607.060102983	1.50586123865788e-06	\\
-10606.0813210227	2.06441481955153e-06	\\
-10605.1025390625	1.77429271611499e-06	\\
-10604.1237571023	1.57719320930431e-06	\\
-10603.144975142	1.54051096391297e-06	\\
-10602.1661931818	1.37931916080208e-06	\\
-10601.1874112216	2.55876260308575e-06	\\
-10600.2086292614	1.47885338727188e-06	\\
-10599.2298473011	1.91574313396376e-06	\\
-10598.2510653409	1.46210526982057e-06	\\
-10597.2722833807	1.64294225257713e-06	\\
-10596.2935014205	1.72248970949588e-06	\\
-10595.3147194602	1.49524356242612e-06	\\
-10594.3359375	1.14000201910193e-06	\\
-10593.3571555398	1.86232504300243e-06	\\
-10592.3783735795	1.45089503595251e-06	\\
-10591.3995916193	1.39165076899595e-06	\\
-10590.4208096591	1.90554294916754e-06	\\
-10589.4420276989	1.90082477775857e-06	\\
-10588.4632457386	1.19165064689219e-06	\\
-10587.4844637784	2.30307373238927e-06	\\
-10586.5056818182	1.3117248057685e-06	\\
-10585.526899858	1.46883951925181e-06	\\
-10584.5481178977	2.12835687483518e-06	\\
-10583.5693359375	1.48374859686759e-06	\\
-10582.5905539773	2.00278413242613e-06	\\
-10581.611772017	1.71819033698338e-06	\\
-10580.6329900568	1.46104833053755e-06	\\
-10579.6542080966	1.7288715544405e-06	\\
-10578.6754261364	2.04451361386174e-06	\\
-10577.6966441761	1.81075638205239e-06	\\
-10576.7178622159	1.61892714126375e-06	\\
-10575.7390802557	1.60353550543598e-06	\\
-10574.7602982955	1.46691679872918e-06	\\
-10573.7815163352	1.60794393269024e-06	\\
-10572.802734375	1.43826831659107e-06	\\
-10571.8239524148	1.78755521389096e-06	\\
-10570.8451704545	1.27289277175245e-06	\\
-10569.8663884943	1.79976433661595e-06	\\
-10568.8876065341	2.00151808201127e-06	\\
-10567.9088245739	1.83651665590039e-06	\\
-10566.9300426136	1.95800806752246e-06	\\
-10565.9512606534	1.72066864263081e-06	\\
-10564.9724786932	1.50751787757431e-06	\\
-10563.993696733	1.73865792310315e-06	\\
-10563.0149147727	1.7079295999536e-06	\\
-10562.0361328125	1.56911657378034e-06	\\
-10561.0573508523	1.93427893414865e-06	\\
-10560.078568892	1.54715937334094e-06	\\
-10559.0997869318	1.32586769240541e-06	\\
-10558.1210049716	1.73973629471459e-06	\\
-10557.1422230114	1.97279421608903e-06	\\
-10556.1634410511	1.73001210881436e-06	\\
-10555.1846590909	2.0593936367061e-06	\\
-10554.2058771307	1.9096135257042e-06	\\
-10553.2270951705	1.63109776282007e-06	\\
-10552.2483132102	1.7539742857666e-06	\\
-10551.26953125	1.64806835553641e-06	\\
-10550.2907492898	1.64253696557035e-06	\\
-10549.3119673295	2.27651178028024e-06	\\
-10548.3331853693	1.75239698208185e-06	\\
-10547.3544034091	1.6573231066071e-06	\\
-10546.3756214489	1.89412806325096e-06	\\
-10545.3968394886	1.94617151503851e-06	\\
-10544.4180575284	1.00804812541745e-06	\\
-10543.4392755682	2.00604548448206e-06	\\
-10542.460493608	1.85380301159261e-06	\\
-10541.4817116477	1.69297823386156e-06	\\
-10540.5029296875	2.04907273414275e-06	\\
-10539.5241477273	1.7259209035327e-06	\\
-10538.545365767	2.05114433196914e-06	\\
-10537.5665838068	2.4937771185121e-06	\\
-10536.5878018466	1.89259335064691e-06	\\
-10535.6090198864	2.06096215584305e-06	\\
-10534.6302379261	1.5325850599811e-06	\\
-10533.6514559659	1.53080710993024e-06	\\
-10532.6726740057	1.90798682365903e-06	\\
-10531.6938920455	2.09081303021845e-06	\\
-10530.7151100852	1.56922587907805e-06	\\
-10529.736328125	1.77357047291501e-06	\\
-10528.7575461648	1.59891568811187e-06	\\
-10527.7787642045	1.87567099147689e-06	\\
-10526.7999822443	1.54185105432741e-06	\\
-10525.8212002841	1.96744628516578e-06	\\
-10524.8424183239	2.38250716215462e-06	\\
-10523.8636363636	1.68849861550254e-06	\\
-10522.8848544034	1.95761307584148e-06	\\
-10521.9060724432	1.7890693814349e-06	\\
-10520.927290483	1.90349769447096e-06	\\
-10519.9485085227	2.0279131242014e-06	\\
-10518.9697265625	1.90744568001248e-06	\\
-10517.9909446023	2.10684812900786e-06	\\
-10517.012162642	2.1221974600909e-06	\\
-10516.0333806818	2.0937641449879e-06	\\
-10515.0545987216	2.41722838076789e-06	\\
-10514.0758167614	1.67781614873954e-06	\\
-10513.0970348011	1.64154917672432e-06	\\
-10512.1182528409	1.85402740865634e-06	\\
-10511.1394708807	1.44005354266562e-06	\\
-10510.1606889205	1.38708737531321e-06	\\
-10509.1819069602	1.88214720945666e-06	\\
-10508.203125	1.85251105231136e-06	\\
-10507.2243430398	1.66513998234346e-06	\\
-10506.2455610795	1.82219947557196e-06	\\
-10505.2667791193	1.92461915715642e-06	\\
-10504.2879971591	1.90690203538433e-06	\\
-10503.3092151989	1.69508196611449e-06	\\
-10502.3304332386	1.7443824829823e-06	\\
-10501.3516512784	1.87314622120651e-06	\\
-10500.3728693182	2.31090043133152e-06	\\
-10499.394087358	2.25246304470249e-06	\\
-10498.4153053977	2.23693199975753e-06	\\
-10497.4365234375	2.02158435195398e-06	\\
-10496.4577414773	2.12382648724224e-06	\\
-10495.478959517	1.6792255041539e-06	\\
-10494.5001775568	1.63977085464777e-06	\\
-10493.5213955966	1.50623999577085e-06	\\
-10492.5426136364	1.99168488626207e-06	\\
-10491.5638316761	1.79594048718245e-06	\\
-10490.5850497159	2.36379676318765e-06	\\
-10489.6062677557	1.94982897623267e-06	\\
-10488.6274857955	1.79225035302076e-06	\\
-10487.6487038352	1.80034254667223e-06	\\
-10486.669921875	2.3908916666765e-06	\\
-10485.6911399148	2.13449104219217e-06	\\
-10484.7123579545	1.68368110309474e-06	\\
-10483.7335759943	1.72999797190455e-06	\\
-10482.7547940341	1.87682759803976e-06	\\
-10481.7760120739	1.82400454236073e-06	\\
-10480.7972301136	1.86820138560678e-06	\\
-10479.8184481534	1.71847988761578e-06	\\
-10478.8396661932	1.99690379224035e-06	\\
-10477.860884233	2.13157061002837e-06	\\
-10476.8821022727	2.41271188441388e-06	\\
-10475.9033203125	1.9077567159746e-06	\\
-10474.9245383523	2.13151777759476e-06	\\
-10473.945756392	2.11948796926615e-06	\\
-10472.9669744318	1.9845099047221e-06	\\
-10471.9881924716	1.79300272099752e-06	\\
-10471.0094105114	1.69926125555523e-06	\\
-10470.0306285511	1.72145756952633e-06	\\
-10469.0518465909	1.89126818047171e-06	\\
-10468.0730646307	1.99985772740872e-06	\\
-10467.0942826705	1.75960001751464e-06	\\
-10466.1155007102	2.18737727244655e-06	\\
-10465.13671875	2.26593446982134e-06	\\
-10464.1579367898	2.01106067982199e-06	\\
-10463.1791548295	1.48619549125721e-06	\\
-10462.2003728693	2.32649404832251e-06	\\
-10461.2215909091	2.31857855997642e-06	\\
-10460.2428089489	1.68990866628866e-06	\\
-10459.2640269886	1.67806984363951e-06	\\
-10458.2852450284	1.84199829004792e-06	\\
-10457.3064630682	1.83945490693255e-06	\\
-10456.327681108	2.21657509505075e-06	\\
-10455.3488991477	2.14264510530933e-06	\\
-10454.3701171875	2.11940271044818e-06	\\
-10453.3913352273	1.9771414483537e-06	\\
-10452.412553267	2.09644193934759e-06	\\
-10451.4337713068	1.93702240599655e-06	\\
-10450.4549893466	2.29946075601831e-06	\\
-10449.4762073864	1.75485419171595e-06	\\
-10448.4974254261	1.61749219950834e-06	\\
-10447.5186434659	1.9648980946582e-06	\\
-10446.5398615057	1.295936828462e-06	\\
-10445.5610795455	1.6761686897998e-06	\\
-10444.5822975852	2.00872210630403e-06	\\
-10443.603515625	2.04587828352794e-06	\\
-10442.6247336648	2.12926770642553e-06	\\
-10441.6459517045	2.71626639648261e-06	\\
-10440.6671697443	1.59382291829971e-06	\\
-10439.6883877841	2.06485033389644e-06	\\
-10438.7096058239	1.93451740011059e-06	\\
-10437.7308238636	1.87074018006343e-06	\\
-10436.7520419034	1.64658982985393e-06	\\
-10435.7732599432	1.39796155747328e-06	\\
-10434.794477983	1.53687855259184e-06	\\
-10433.8156960227	1.55629558460762e-06	\\
-10432.8369140625	1.6241775637427e-06	\\
-10431.8581321023	2.01609018014906e-06	\\
-10430.879350142	1.66934441379518e-06	\\
-10429.9005681818	2.0706490637826e-06	\\
-10428.9217862216	1.86394204980011e-06	\\
-10427.9430042614	2.17997338741191e-06	\\
-10426.9642223011	1.99459753900232e-06	\\
-10425.9854403409	1.90008038970703e-06	\\
-10425.0066583807	1.89914166182355e-06	\\
-10424.0278764205	2.17089043601485e-06	\\
-10423.0490944602	1.90248044401349e-06	\\
-10422.0703125	2.20568231122686e-06	\\
-10421.0915305398	2.1582262285267e-06	\\
-10420.1127485795	2.37752834647565e-06	\\
-10419.1339666193	1.58954047070267e-06	\\
-10418.1551846591	1.95312519471257e-06	\\
-10417.1764026989	2.03870860857843e-06	\\
-10416.1976207386	2.2774767991873e-06	\\
-10415.2188387784	1.81860405341296e-06	\\
-10414.2400568182	2.47427796430381e-06	\\
-10413.261274858	1.76782517185852e-06	\\
-10412.2824928977	2.09333752566575e-06	\\
-10411.3037109375	2.37819788627662e-06	\\
-10410.3249289773	1.37604177245095e-06	\\
-10409.346147017	2.27722900515775e-06	\\
-10408.3673650568	1.45893871527753e-06	\\
-10407.3885830966	2.29396977876003e-06	\\
-10406.4098011364	2.04933431022064e-06	\\
-10405.4310191761	1.57129077071015e-06	\\
-10404.4522372159	1.88502290376284e-06	\\
-10403.4734552557	2.34674740869461e-06	\\
-10402.4946732955	2.43119679076184e-06	\\
-10401.5158913352	1.60125291743462e-06	\\
-10400.537109375	1.58946917865114e-06	\\
-10399.5583274148	2.19805401836381e-06	\\
-10398.5795454545	1.9412879256614e-06	\\
-10397.6007634943	1.8470893127294e-06	\\
-10396.6219815341	2.12439513036623e-06	\\
-10395.6431995739	2.02600872231517e-06	\\
-10394.6644176136	1.92965753490477e-06	\\
-10393.6856356534	2.3368886203502e-06	\\
-10392.7068536932	2.05491850534977e-06	\\
-10391.728071733	1.71344359920891e-06	\\
-10390.7492897727	1.74400277322347e-06	\\
-10389.7705078125	2.14069106622479e-06	\\
-10388.7917258523	2.07328811762132e-06	\\
-10387.812943892	2.91213044357128e-06	\\
-10386.8341619318	1.22213613505095e-06	\\
-10385.8553799716	1.66494001635356e-06	\\
-10384.8765980114	2.18733436753311e-06	\\
-10383.8978160511	1.82316552308046e-06	\\
-10382.9190340909	2.15329135710369e-06	\\
-10381.9402521307	1.79599472763622e-06	\\
-10380.9614701705	1.86762654335031e-06	\\
-10379.9826882102	1.88927080029905e-06	\\
-10379.00390625	1.61095371591237e-06	\\
-10378.0251242898	1.76154592213975e-06	\\
-10377.0463423295	1.95022072570821e-06	\\
-10376.0675603693	1.88389533993972e-06	\\
-10375.0887784091	2.10181906722066e-06	\\
-10374.1099964489	1.8802432519703e-06	\\
-10373.1312144886	1.99886104734561e-06	\\
-10372.1524325284	2.12296748976549e-06	\\
-10371.1736505682	2.03724468041465e-06	\\
-10370.194868608	2.20379210400479e-06	\\
-10369.2160866477	1.86451797880794e-06	\\
-10368.2373046875	2.03714346941514e-06	\\
-10367.2585227273	1.58754858768246e-06	\\
-10366.279740767	1.39091928404822e-06	\\
-10365.3009588068	2.04482045444518e-06	\\
-10364.3221768466	2.20435657283867e-06	\\
-10363.3433948864	1.82025060536747e-06	\\
-10362.3646129261	2.33766774385575e-06	\\
-10361.3858309659	1.803002478628e-06	\\
-10360.4070490057	2.41942862293146e-06	\\
-10359.4282670455	1.82291368978908e-06	\\
-10358.4494850852	1.38064850555395e-06	\\
-10357.470703125	1.92107139491533e-06	\\
-10356.4919211648	1.84781969467031e-06	\\
-10355.5131392045	1.72245829029305e-06	\\
-10354.5343572443	1.99938900371112e-06	\\
-10353.5555752841	1.77036879553025e-06	\\
-10352.5767933239	1.96761880183494e-06	\\
-10351.5980113636	1.76814889275941e-06	\\
-10350.6192294034	1.9983199513911e-06	\\
-10349.6404474432	1.78906537951947e-06	\\
-10348.661665483	1.56904607135669e-06	\\
-10347.6828835227	1.46677080474936e-06	\\
-10346.7041015625	2.48089937904529e-06	\\
-10345.7253196023	2.37536653450773e-06	\\
-10344.746537642	1.75930292778029e-06	\\
-10343.7677556818	1.73502264377882e-06	\\
-10342.7889737216	2.11527458655091e-06	\\
-10341.8101917614	2.27131083060394e-06	\\
-10340.8314098011	1.27757914569393e-06	\\
-10339.8526278409	1.57354134182175e-06	\\
-10338.8738458807	1.84264368825974e-06	\\
-10337.8950639205	1.66623519545092e-06	\\
-10336.9162819602	2.10517278010249e-06	\\
-10335.9375	1.52923552970648e-06	\\
-10334.9587180398	1.94289020100174e-06	\\
-10333.9799360795	1.71158111036115e-06	\\
-10333.0011541193	1.64195356503353e-06	\\
-10332.0223721591	1.78166470187337e-06	\\
-10331.0435901989	2.27586026636388e-06	\\
-10330.0648082386	1.49896199892016e-06	\\
-10329.0860262784	2.21392618318129e-06	\\
-10328.1072443182	1.97686625726545e-06	\\
-10327.128462358	1.70382967408191e-06	\\
-10326.1496803977	2.05157499701558e-06	\\
-10325.1708984375	2.59833225169795e-06	\\
-10324.1921164773	1.68374263880182e-06	\\
-10323.213334517	2.2951141185652e-06	\\
-10322.2345525568	2.17923600038963e-06	\\
-10321.2557705966	2.00125688763511e-06	\\
-10320.2769886364	1.94143249081654e-06	\\
-10319.2982066761	1.80748978159559e-06	\\
-10318.3194247159	1.79017506224195e-06	\\
-10317.3406427557	1.52767352892546e-06	\\
-10316.3618607955	2.51947935757138e-06	\\
-10315.3830788352	1.62957970981376e-06	\\
-10314.404296875	1.75944295892217e-06	\\
-10313.4255149148	1.85430101873421e-06	\\
-10312.4467329545	1.51508957757401e-06	\\
-10311.4679509943	1.99917947247535e-06	\\
-10310.4891690341	1.82711767214395e-06	\\
-10309.5103870739	1.68253719026782e-06	\\
-10308.5316051136	1.928621682019e-06	\\
-10307.5528231534	1.9248716599569e-06	\\
-10306.5740411932	2.27315624471268e-06	\\
-10305.595259233	1.70437727587565e-06	\\
-10304.6164772727	1.94253039521926e-06	\\
-10303.6376953125	1.92005233307997e-06	\\
-10302.6589133523	2.09527753373258e-06	\\
-10301.680131392	1.94636664522068e-06	\\
-10300.7013494318	1.96405934490764e-06	\\
-10299.7225674716	2.02644549317247e-06	\\
-10298.7437855114	1.99796976995394e-06	\\
-10297.7650035511	1.5225052455562e-06	\\
-10296.7862215909	1.90813137089084e-06	\\
-10295.8074396307	1.81754359677246e-06	\\
-10294.8286576705	1.65357141528115e-06	\\
-10293.8498757102	2.43276683968884e-06	\\
-10292.87109375	2.04519195380728e-06	\\
-10291.8923117898	2.77677882286673e-06	\\
-10290.9135298295	2.23905860570596e-06	\\
-10289.9347478693	1.97614025281659e-06	\\
-10288.9559659091	1.75421068692166e-06	\\
-10287.9771839489	1.9876727043776e-06	\\
-10286.9984019886	1.91107426718195e-06	\\
-10286.0196200284	2.29693202755815e-06	\\
-10285.0408380682	1.82283505526796e-06	\\
-10284.062056108	1.4035937242189e-06	\\
-10283.0832741477	1.72995425061329e-06	\\
-10282.1044921875	1.96072114906278e-06	\\
-10281.1257102273	1.76100069747544e-06	\\
-10280.146928267	2.14330146238264e-06	\\
-10279.1681463068	2.05689371489325e-06	\\
-10278.1893643466	1.79011044895909e-06	\\
-10277.2105823864	1.55836278594057e-06	\\
-10276.2318004261	2.47039676189452e-06	\\
-10275.2530184659	1.64113382143156e-06	\\
-10274.2742365057	1.82311522009118e-06	\\
-10273.2954545455	1.91569566921105e-06	\\
-10272.3166725852	1.73252217121408e-06	\\
-10271.337890625	1.75484296562711e-06	\\
-10270.3591086648	1.81829325848833e-06	\\
-10269.3803267045	1.7085063248879e-06	\\
-10268.4015447443	1.93593400318131e-06	\\
-10267.4227627841	1.89161112932676e-06	\\
-10266.4439808239	1.06564524882717e-06	\\
-10265.4651988636	2.56318252937736e-06	\\
-10264.4864169034	1.54940784842982e-06	\\
-10263.5076349432	1.703084934626e-06	\\
-10262.528852983	2.38270285093102e-06	\\
-10261.5500710227	1.66479762191958e-06	\\
-10260.5712890625	1.89155748723782e-06	\\
-10259.5925071023	1.66951515134229e-06	\\
-10258.613725142	1.66239200999288e-06	\\
-10257.6349431818	1.24014354585776e-06	\\
-10256.6561612216	2.55613052094937e-06	\\
-10255.6773792614	1.47808809006571e-06	\\
-10254.6985973011	2.52866665945742e-06	\\
-10253.7198153409	2.20622626115443e-06	\\
-10252.7410333807	2.10546589931067e-06	\\
-10251.7622514205	1.85481797275604e-06	\\
-10250.7834694602	2.08442788504247e-06	\\
-10249.8046875	1.39541611420097e-06	\\
-10248.8259055398	9.55429119800128e-07	\\
-10247.8471235795	2.01915217175673e-06	\\
-10246.8683416193	8.25998763249426e-07	\\
-10245.8895596591	1.55465638481799e-06	\\
-10244.9107776989	1.99031432993743e-06	\\
-10243.9319957386	2.01751032524318e-06	\\
-10242.9532137784	1.3136978520724e-06	\\
-10241.9744318182	2.08197212805658e-06	\\
-10240.995649858	1.98823938067438e-06	\\
-10240.0168678977	1.90357489104002e-06	\\
-10239.0380859375	2.32771646938912e-06	\\
-10238.0593039773	1.66709760143298e-06	\\
-10237.080522017	2.02902820860798e-06	\\
-10236.1017400568	1.89004154997862e-06	\\
-10235.1229580966	2.27838117378514e-06	\\
-10234.1441761364	2.07366487046309e-06	\\
-10233.1653941761	1.26415328316555e-06	\\
-10232.1866122159	2.30013068921341e-06	\\
-10231.2078302557	1.87054596786473e-06	\\
-10230.2290482955	1.58665316934805e-06	\\
-10229.2502663352	2.04138473019821e-06	\\
-10228.271484375	2.03594736316914e-06	\\
-10227.2927024148	1.85652778569955e-06	\\
-10226.3139204545	1.36299721958888e-06	\\
-10225.3351384943	2.35820298056017e-06	\\
-10224.3563565341	1.85301444797343e-06	\\
-10223.3775745739	2.11152777057365e-06	\\
-10222.3987926136	1.73438588442216e-06	\\
-10221.4200106534	1.89129659631529e-06	\\
-10220.4412286932	1.93825665398911e-06	\\
-10219.462446733	2.22300430516815e-06	\\
-10218.4836647727	2.01126866084897e-06	\\
-10217.5048828125	2.58433504237565e-06	\\
-10216.5261008523	2.26031052364427e-06	\\
-10215.547318892	1.63915457711768e-06	\\
-10214.5685369318	2.07526037820571e-06	\\
-10213.5897549716	2.34836246043039e-06	\\
-10212.6109730114	2.29312020753979e-06	\\
-10211.6321910511	1.72724436310112e-06	\\
-10210.6534090909	1.70659272602427e-06	\\
-10209.6746271307	1.56889328159802e-06	\\
-10208.6958451705	2.11379902829619e-06	\\
-10207.7170632102	1.84804570049087e-06	\\
-10206.73828125	1.02854033911937e-06	\\
-10205.7594992898	2.18234902871545e-06	\\
-10204.7807173295	1.68331141825129e-06	\\
-10203.8019353693	1.67102027887893e-06	\\
-10202.8231534091	1.7565124587856e-06	\\
-10201.8443714489	1.85451223849816e-06	\\
-10200.8655894886	2.28385215646203e-06	\\
-10199.8868075284	1.51199571617165e-06	\\
-10198.9080255682	1.78461998773043e-06	\\
-10197.929243608	1.23128508049931e-06	\\
-10196.9504616477	1.63557279696279e-06	\\
-10195.9716796875	1.53864248617473e-06	\\
-10194.9928977273	1.92177612513862e-06	\\
-10194.014115767	1.72612481761162e-06	\\
-10193.0353338068	1.66335333772497e-06	\\
-10192.0565518466	2.28606617728866e-06	\\
-10191.0777698864	1.84964176006202e-06	\\
-10190.0989879261	1.76354364779884e-06	\\
-10189.1202059659	2.04011925777003e-06	\\
-10188.1414240057	2.51921612838506e-06	\\
-10187.1626420455	1.86354568642128e-06	\\
-10186.1838600852	1.80582502347982e-06	\\
-10185.205078125	1.66150072523535e-06	\\
-10184.2262961648	1.97615970716247e-06	\\
-10183.2475142045	8.41097468348862e-07	\\
-10182.2687322443	1.39976017670943e-06	\\
-10181.2899502841	1.80247099219049e-06	\\
-10180.3111683239	1.48417227330905e-06	\\
-10179.3323863636	1.81259715773617e-06	\\
-10178.3536044034	1.72866877948653e-06	\\
-10177.3748224432	2.29956307651601e-06	\\
-10176.396040483	1.73410025522907e-06	\\
-10175.4172585227	2.33360745838286e-06	\\
-10174.4384765625	1.27319667724923e-06	\\
-10173.4596946023	2.14998400054823e-06	\\
-10172.480912642	2.0192582819232e-06	\\
-10171.5021306818	1.37326874081715e-06	\\
-10170.5233487216	1.92711084867354e-06	\\
-10169.5445667614	2.23341955815951e-06	\\
-10168.5657848011	7.78735084290895e-07	\\
-10167.5870028409	2.23715843404199e-06	\\
-10166.6082208807	1.84128169816367e-06	\\
-10165.6294389205	1.54062277973108e-06	\\
-10164.6506569602	2.58699545896512e-06	\\
-10163.671875	1.85627691775359e-06	\\
-10162.6930930398	1.85693740607407e-06	\\
-10161.7143110795	2.49216799946897e-06	\\
-10160.7355291193	1.95974895821371e-06	\\
-10159.7567471591	2.45214653550708e-06	\\
-10158.7779651989	2.34594182048706e-06	\\
-10157.7991832386	2.01674607411119e-06	\\
-10156.8204012784	2.04490448307847e-06	\\
-10155.8416193182	1.20393838901734e-06	\\
-10154.862837358	2.4698184171821e-06	\\
-10153.8840553977	1.98638121105334e-06	\\
-10152.9052734375	1.91482794606022e-06	\\
-10151.9264914773	1.81952490656783e-06	\\
-10150.947709517	2.32829681279706e-06	\\
-10149.9689275568	1.72761934243673e-06	\\
-10148.9901455966	1.92307475305495e-06	\\
-10148.0113636364	2.21168795532408e-06	\\
-10147.0325816761	1.64152665276087e-06	\\
-10146.0537997159	1.92325293073136e-06	\\
-10145.0750177557	2.09221990611463e-06	\\
-10144.0962357955	1.15165377831651e-06	\\
-10143.1174538352	1.41478227970936e-06	\\
-10142.138671875	1.36821237450406e-06	\\
-10141.1598899148	1.29707774604514e-06	\\
-10140.1811079545	1.50960798719909e-06	\\
-10139.2023259943	1.34238410789601e-06	\\
-10138.2235440341	1.33589525779128e-06	\\
-10137.2447620739	2.01250525696861e-06	\\
-10136.2659801136	1.75878764868041e-06	\\
-10135.2871981534	2.91148542392545e-06	\\
-10134.3084161932	2.81752019459734e-06	\\
-10133.329634233	2.28896743196643e-06	\\
-10132.3508522727	2.45041004839601e-06	\\
-10131.3720703125	1.95731245943036e-06	\\
-10130.3932883523	1.29595943280113e-06	\\
-10129.414506392	2.14132608202344e-06	\\
-10128.4357244318	1.22890803151218e-06	\\
-10127.4569424716	1.55887539039687e-06	\\
-10126.4781605114	2.30791360805085e-06	\\
-10125.4993785511	1.32100902528561e-06	\\
-10124.5205965909	1.6619509115675e-06	\\
-10123.5418146307	1.7903909290154e-06	\\
-10122.5630326705	2.02081297277709e-06	\\
-10121.5842507102	2.32067273086919e-06	\\
-10120.60546875	1.72651444407429e-06	\\
-10119.6266867898	1.99894324793625e-06	\\
-10118.6479048295	2.26546147311843e-06	\\
-10117.6691228693	1.70861676989465e-06	\\
-10116.6903409091	1.61077652784843e-06	\\
-10115.7115589489	2.14650895112384e-06	\\
-10114.7327769886	1.96397778473582e-06	\\
-10113.7539950284	1.81827264759204e-06	\\
-10112.7752130682	1.81899581645945e-06	\\
-10111.796431108	2.16696596016044e-06	\\
-10110.8176491477	2.29213946330303e-06	\\
-10109.8388671875	1.69326697198698e-06	\\
-10108.8600852273	2.50066335286689e-06	\\
-10107.881303267	2.1071991357617e-06	\\
-10106.9025213068	2.12507243508872e-06	\\
-10105.9237393466	1.63481729957461e-06	\\
-10104.9449573864	1.77343651768415e-06	\\
-10103.9661754261	1.95133104446985e-06	\\
-10102.9873934659	1.97400356037986e-06	\\
-10102.0086115057	2.05965982229192e-06	\\
-10101.0298295455	1.78989762410383e-06	\\
-10100.0510475852	1.92915141297361e-06	\\
-10099.072265625	1.96867178445553e-06	\\
-10098.0934836648	2.59478536054512e-06	\\
-10097.1147017045	1.82672396856251e-06	\\
-10096.1359197443	2.33499467806052e-06	\\
-10095.1571377841	1.86426684784634e-06	\\
-10094.1783558239	2.29716404707202e-06	\\
-10093.1995738636	1.66934612970067e-06	\\
-10092.2207919034	1.83980759183877e-06	\\
-10091.2420099432	2.25484274389013e-06	\\
-10090.263227983	1.68393673312873e-06	\\
-10089.2844460227	1.88783748785917e-06	\\
-10088.3056640625	1.89895010072753e-06	\\
-10087.3268821023	2.00088182333659e-06	\\
-10086.348100142	1.93732468631128e-06	\\
-10085.3693181818	9.15713166630672e-07	\\
-10084.3905362216	1.72118745991447e-06	\\
-10083.4117542614	1.71901571936919e-06	\\
-10082.4329723011	1.58622947365599e-06	\\
-10081.4541903409	1.90740708892435e-06	\\
-10080.4754083807	1.96842670851478e-06	\\
-10079.4966264205	1.57734439075427e-06	\\
-10078.5178444602	2.07869624153411e-06	\\
-10077.5390625	2.14805298550294e-06	\\
-10076.5602805398	1.82039439008807e-06	\\
-10075.5814985795	1.8175199845789e-06	\\
-10074.6027166193	1.9849423912467e-06	\\
-10073.6239346591	1.98529080331392e-06	\\
-10072.6451526989	8.43287775617854e-07	\\
-10071.6663707386	1.96906786045056e-06	\\
-10070.6875887784	1.41190654792223e-06	\\
-10069.7088068182	2.1469957428506e-06	\\
-10068.730024858	2.18554696329388e-06	\\
-10067.7512428977	2.09816449384367e-06	\\
-10066.7724609375	2.26902750511224e-06	\\
-10065.7936789773	1.41186687649839e-06	\\
-10064.814897017	1.94586388496108e-06	\\
-10063.8361150568	1.68464547840305e-06	\\
-10062.8573330966	1.50605128970887e-06	\\
-10061.8785511364	1.93890173659124e-06	\\
-10060.8997691761	2.501496162101e-06	\\
-10059.9209872159	2.31276171035064e-06	\\
-10058.9422052557	1.56297986126993e-06	\\
-10057.9634232955	1.68914811447247e-06	\\
-10056.9846413352	1.71770047138869e-06	\\
-10056.005859375	2.21687568494121e-06	\\
-10055.0270774148	1.66746803008199e-06	\\
-10054.0482954545	2.76651193596734e-06	\\
-10053.0695134943	2.37247597036506e-06	\\
-10052.0907315341	1.46079151732559e-06	\\
-10051.1119495739	2.24745926112757e-06	\\
-10050.1331676136	1.63891795389273e-06	\\
-10049.1543856534	1.38897053930558e-06	\\
-10048.1756036932	1.69028796429751e-06	\\
-10047.196821733	2.47184537673521e-06	\\
-10046.2180397727	1.68174752038896e-06	\\
-10045.2392578125	1.55432187573409e-06	\\
-10044.2604758523	1.97210371189792e-06	\\
-10043.281693892	1.31392498584515e-06	\\
-10042.3029119318	1.50094138209941e-06	\\
-10041.3241299716	2.59344567163593e-06	\\
-10040.3453480114	2.25818593202593e-06	\\
-10039.3665660511	2.04689675637194e-06	\\
-10038.3877840909	2.103308346403e-06	\\
-10037.4090021307	2.43106524164159e-06	\\
-10036.4302201705	2.28844296290912e-06	\\
-10035.4514382102	2.70567483646493e-06	\\
-10034.47265625	2.4313255449576e-06	\\
-10033.4938742898	2.50369125399667e-06	\\
-10032.5150923295	2.2296741829779e-06	\\
-10031.5363103693	2.26029221333974e-06	\\
-10030.5575284091	2.24123075554053e-06	\\
-10029.5787464489	1.79684988419574e-06	\\
-10028.5999644886	1.5110462657492e-06	\\
-10027.6211825284	2.19162187058022e-06	\\
-10026.6424005682	1.51485894693966e-06	\\
-10025.663618608	1.81254183942795e-06	\\
-10024.6848366477	2.22118154255679e-06	\\
-10023.7060546875	2.39129260260541e-06	\\
-10022.7272727273	1.9659795190725e-06	\\
-10021.748490767	1.93685091845707e-06	\\
-10020.7697088068	2.25707080154917e-06	\\
-10019.7909268466	2.28808030204406e-06	\\
-10018.8121448864	1.15080730732168e-06	\\
-10017.8333629261	2.49791394881612e-06	\\
-10016.8545809659	2.70240548886919e-06	\\
-10015.8757990057	2.09182214281679e-06	\\
-10014.8970170455	3.02689559555739e-06	\\
-10013.9182350852	2.01726618987032e-06	\\
-10012.939453125	2.77400278611069e-06	\\
-10011.9606711648	2.61807146272724e-06	\\
-10010.9818892045	2.34224627048861e-06	\\
-10010.0031072443	2.41004175270349e-06	\\
-10009.0243252841	2.84069063289107e-06	\\
-10008.0455433239	2.7802607601345e-06	\\
-10007.0667613636	1.84264445580353e-06	\\
-10006.0879794034	2.33986455986189e-06	\\
-10005.1091974432	2.66770230259368e-06	\\
-10004.130415483	2.13823827841272e-06	\\
-10003.1516335227	2.19508530833088e-06	\\
-10002.1728515625	3.3770920477197e-06	\\
-10001.1940696023	1.93336366122183e-06	\\
-10000.215287642	2.47196692098044e-06	\\
-9999.23650568182	2.13967485427828e-06	\\
-9998.25772372159	2.56619591642114e-06	\\
-9997.27894176136	2.13851159886855e-06	\\
-9996.30015980114	2.46060519525825e-06	\\
-9995.32137784091	1.60565055055078e-06	\\
-9994.34259588068	2.40992089111654e-06	\\
-9993.36381392045	2.16270917060561e-06	\\
-9992.38503196023	2.42639854450536e-06	\\
-9991.40625	2.04424971010734e-06	\\
-9990.42746803977	1.97748800855244e-06	\\
-9989.44868607955	2.1016858584882e-06	\\
-9988.46990411932	2.0637896500537e-06	\\
-9987.49112215909	2.62036288683511e-06	\\
-9986.51234019886	3.02114592304271e-06	\\
-9985.53355823864	2.43691903327748e-06	\\
-9984.55477627841	3.13194625534649e-06	\\
-9983.57599431818	1.95959349581596e-06	\\
-9982.59721235795	2.2595096093985e-06	\\
-9981.61843039773	3.03286506398686e-06	\\
-9980.6396484375	2.38839326629676e-06	\\
-9979.66086647727	2.46483699084904e-06	\\
-9978.68208451705	1.43720040954144e-06	\\
-9977.70330255682	2.77139917979205e-06	\\
-9976.72452059659	2.56837061383589e-06	\\
-9975.74573863636	2.63193667503996e-06	\\
-9974.76695667614	2.42869452507859e-06	\\
-9973.78817471591	2.88358320035705e-06	\\
-9972.80939275568	2.50154970640904e-06	\\
-9971.83061079545	1.54312035449903e-06	\\
-9970.85182883523	1.37322681496261e-06	\\
-9969.873046875	2.5479036968077e-06	\\
-9968.89426491477	2.41968520261556e-06	\\
-9967.91548295455	2.61632860190979e-06	\\
-9966.93670099432	2.18377049083302e-06	\\
-9965.95791903409	2.31231631338631e-06	\\
-9964.97913707386	1.9818172560796e-06	\\
-9964.00035511364	2.54617162049855e-06	\\
-9963.02157315341	2.98892215059984e-06	\\
-9962.04279119318	2.46802826037257e-06	\\
-9961.06400923295	2.86092338394215e-06	\\
-9960.08522727273	3.24665175348792e-06	\\
-9959.1064453125	2.63144444453386e-06	\\
-9958.12766335227	2.45749564919018e-06	\\
-9957.14888139205	1.43324503326665e-06	\\
-9956.17009943182	2.07638500495093e-06	\\
-9955.19131747159	2.6932151102175e-06	\\
-9954.21253551136	2.32293581013305e-06	\\
-9953.23375355114	2.41528609939469e-06	\\
-9952.25497159091	2.4587505274784e-06	\\
-9951.27618963068	2.37367703879817e-06	\\
-9950.29740767045	2.09650842226432e-06	\\
-9949.31862571023	2.22067114458657e-06	\\
-9948.33984375	2.56790495944956e-06	\\
-9947.36106178977	2.68422222252965e-06	\\
-9946.38227982955	3.02531235465939e-06	\\
-9945.40349786932	2.72808811656243e-06	\\
-9944.42471590909	1.95050872005905e-06	\\
-9943.44593394886	2.79968941190061e-06	\\
-9942.46715198864	3.22249142494352e-06	\\
-9941.48837002841	2.39479845300508e-06	\\
-9940.50958806818	3.1266298509823e-06	\\
-9939.53080610795	2.43654535531496e-06	\\
-9938.55202414773	2.35627610790995e-06	\\
-9937.5732421875	3.4559505400846e-06	\\
-9936.59446022727	3.0046674266144e-06	\\
-9935.61567826705	3.18651512374696e-06	\\
-9934.63689630682	2.62371513828321e-06	\\
-9933.65811434659	2.6524570677771e-06	\\
-9932.67933238636	3.02243324507901e-06	\\
-9931.70055042614	2.88882208084087e-06	\\
-9930.72176846591	2.56670829688898e-06	\\
-9929.74298650568	2.52749134885285e-06	\\
-9928.76420454545	2.02960710406292e-06	\\
-9927.78542258523	3.31090539336275e-06	\\
-9926.806640625	3.04880880779369e-06	\\
-9925.82785866477	2.73765279757632e-06	\\
-9924.84907670455	3.30902897711535e-06	\\
-9923.87029474432	2.02272197056687e-06	\\
-9922.89151278409	2.59178675521705e-06	\\
-9921.91273082386	2.81047307077155e-06	\\
-9920.93394886364	2.70325411287237e-06	\\
-9919.95516690341	3.25384480786614e-06	\\
-9918.97638494318	3.18314753265995e-06	\\
-9917.99760298295	2.80223064203542e-06	\\
-9917.01882102273	2.6719068397138e-06	\\
-9916.0400390625	2.88751518846321e-06	\\
-9915.06125710227	3.06435733954816e-06	\\
-9914.08247514205	3.41917485769125e-06	\\
-9913.10369318182	2.56907576195779e-06	\\
-9912.12491122159	3.38703974725743e-06	\\
-9911.14612926136	3.30589638744867e-06	\\
-9910.16734730114	3.32119382641368e-06	\\
-9909.18856534091	4.00439385892615e-06	\\
-9908.20978338068	2.45049675324952e-06	\\
-9907.23100142045	2.96433872796851e-06	\\
-9906.25221946023	2.60224738430626e-06	\\
-9905.2734375	3.51717549190133e-06	\\
-9904.29465553977	2.47459045571051e-06	\\
-9903.31587357955	3.36668316907564e-06	\\
-9902.33709161932	3.35676282737161e-06	\\
-9901.35830965909	3.33001375007492e-06	\\
-9900.37952769886	2.76288479786596e-06	\\
-9899.40074573864	2.87709288812207e-06	\\
-9898.42196377841	2.96563751417427e-06	\\
-9897.44318181818	3.67552280123812e-06	\\
-9896.46439985795	2.8372768530669e-06	\\
-9895.48561789773	3.34943254291621e-06	\\
-9894.5068359375	4.04947486769978e-06	\\
-9893.52805397727	3.03424957275248e-06	\\
-9892.54927201705	3.24465589858941e-06	\\
-9891.57049005682	4.12252741751594e-06	\\
-9890.59170809659	2.62022473632063e-06	\\
-9889.61292613636	3.43218883503504e-06	\\
-9888.63414417614	2.53553454903247e-06	\\
-9887.65536221591	2.67518786316963e-06	\\
-9886.67658025568	2.97303770074169e-06	\\
-9885.69779829545	3.28563368679534e-06	\\
-9884.71901633523	2.07289854946289e-06	\\
-9883.740234375	2.76401757683157e-06	\\
-9882.76145241477	1.75953028733261e-06	\\
-9881.78267045455	2.39471692919413e-06	\\
-9880.80388849432	2.65090261156666e-06	\\
-9879.82510653409	2.48876779676326e-06	\\
-9878.84632457386	2.58991640500877e-06	\\
-9877.86754261364	3.35172642873408e-06	\\
-9876.88876065341	3.10182725859633e-06	\\
-9875.90997869318	2.68497330215358e-06	\\
-9874.93119673295	3.03895880681162e-06	\\
-9873.95241477273	2.57996062860507e-06	\\
-9872.9736328125	2.86492637968834e-06	\\
-9871.99485085227	2.81939547248323e-06	\\
-9871.01606889205	3.44916350780556e-06	\\
-9870.03728693182	2.68030989130199e-06	\\
-9869.05850497159	2.54684886370402e-06	\\
-9868.07972301136	2.68236330147428e-06	\\
-9867.10094105114	2.45732670296062e-06	\\
-9866.12215909091	3.18550231303193e-06	\\
-9865.14337713068	3.92281312479059e-06	\\
-9864.16459517045	3.38094295817555e-06	\\
-9863.18581321023	2.67214725244911e-06	\\
-9862.20703125	3.53171120546296e-06	\\
-9861.22824928977	3.40109213512336e-06	\\
-9860.24946732955	2.30835955013129e-06	\\
-9859.27068536932	2.99596851706197e-06	\\
-9858.29190340909	2.82608929705857e-06	\\
-9857.31312144886	1.86937951715461e-06	\\
-9856.33433948864	2.93744678604207e-06	\\
-9855.35555752841	2.96298149006537e-06	\\
-9854.37677556818	3.1441254590843e-06	\\
-9853.39799360795	2.39993957586474e-06	\\
-9852.41921164773	2.40925237335061e-06	\\
-9851.4404296875	2.86393182914036e-06	\\
-9850.46164772727	2.62951694175497e-06	\\
-9849.48286576705	2.43377229889112e-06	\\
-9848.50408380682	3.02130673913117e-06	\\
-9847.52530184659	3.4016976416554e-06	\\
-9846.54651988636	4.19742507656718e-06	\\
-9845.56773792614	3.23492796034129e-06	\\
-9844.58895596591	3.16662158497722e-06	\\
-9843.61017400568	2.67050996023939e-06	\\
-9842.63139204545	2.45225735747442e-06	\\
-9841.65261008523	2.86685137595989e-06	\\
-9840.673828125	2.83644274652374e-06	\\
-9839.69504616477	2.85225861073642e-06	\\
-9838.71626420455	3.13727240613111e-06	\\
-9837.73748224432	2.80044659731798e-06	\\
-9836.75870028409	3.42604901551655e-06	\\
-9835.77991832386	2.74126410480431e-06	\\
-9834.80113636364	3.46631067174668e-06	\\
-9833.82235440341	2.51472350335913e-06	\\
-9832.84357244318	3.05701977501812e-06	\\
-9831.86479048295	3.45711334824908e-06	\\
-9830.88600852273	3.3032831455564e-06	\\
-9829.9072265625	3.13014388086702e-06	\\
-9828.92844460227	3.33118017099946e-06	\\
-9827.94966264205	3.15209565622503e-06	\\
-9826.97088068182	2.74070238517121e-06	\\
-9825.99209872159	3.83160273108356e-06	\\
-9825.01331676136	3.18091176267496e-06	\\
-9824.03453480114	3.64324873693837e-06	\\
-9823.05575284091	2.73952336169553e-06	\\
-9822.07697088068	2.5628909754316e-06	\\
-9821.09818892045	3.25070889843274e-06	\\
-9820.11940696023	2.56880630916838e-06	\\
-9819.140625	1.72255227702505e-06	\\
-9818.16184303977	2.98592174531417e-06	\\
-9817.18306107955	3.05055017998517e-06	\\
-9816.20427911932	3.27158104751861e-06	\\
-9815.22549715909	3.42738327424551e-06	\\
-9814.24671519886	2.88333604621216e-06	\\
-9813.26793323864	2.98136992709899e-06	\\
-9812.28915127841	3.33775363191987e-06	\\
-9811.31036931818	3.55849028919845e-06	\\
-9810.33158735795	3.14935492812077e-06	\\
-9809.35280539773	2.38969418439205e-06	\\
-9808.3740234375	2.95526691147262e-06	\\
-9807.39524147727	3.65913066702103e-06	\\
-9806.41645951705	3.43674978052606e-06	\\
-9805.43767755682	2.666513820649e-06	\\
-9804.45889559659	3.38557449043859e-06	\\
-9803.48011363636	2.90269149979534e-06	\\
-9802.50133167614	2.56648771914927e-06	\\
-9801.52254971591	2.51366810017348e-06	\\
-9800.54376775568	2.21736670650404e-06	\\
-9799.56498579545	3.02856990268593e-06	\\
-9798.58620383523	3.36520174543066e-06	\\
-9797.607421875	3.55499839721868e-06	\\
-9796.62863991477	2.62365411311334e-06	\\
-9795.64985795455	2.65401054957692e-06	\\
-9794.67107599432	2.51176375098109e-06	\\
-9793.69229403409	3.61432736721851e-06	\\
-9792.71351207386	3.06225378776651e-06	\\
-9791.73473011364	2.26754126502173e-06	\\
-9790.75594815341	2.76767968596101e-06	\\
-9789.77716619318	3.47295038456473e-06	\\
-9788.79838423295	3.23261335062803e-06	\\
-9787.81960227273	2.67935214751936e-06	\\
-9786.8408203125	2.75578990720649e-06	\\
-9785.86203835227	3.20039829344057e-06	\\
-9784.88325639205	3.68146748856411e-06	\\
-9783.90447443182	3.4429546056804e-06	\\
-9782.92569247159	3.45405856582388e-06	\\
-9781.94691051136	3.27668227486702e-06	\\
-9780.96812855114	3.06282525379861e-06	\\
-9779.98934659091	1.74579779012925e-06	\\
-9779.01056463068	2.40974949479344e-06	\\
-9778.03178267045	2.75657542862718e-06	\\
-9777.05300071023	2.56469627588548e-06	\\
-9776.07421875	2.89357017918107e-06	\\
-9775.09543678977	3.58105313724534e-06	\\
-9774.11665482955	2.42958933576066e-06	\\
-9773.13787286932	3.47009874799926e-06	\\
-9772.15909090909	2.91932052861172e-06	\\
-9771.18030894886	3.16313550543043e-06	\\
-9770.20152698864	3.04835773480564e-06	\\
-9769.22274502841	2.71047691014902e-06	\\
-9768.24396306818	2.1365605706677e-06	\\
-9767.26518110795	2.59573837128044e-06	\\
-9766.28639914773	2.53332477242513e-06	\\
-9765.3076171875	3.19211419295647e-06	\\
-9764.32883522727	2.8163027249347e-06	\\
-9763.35005326705	3.39808569435838e-06	\\
-9762.37127130682	3.12699137641828e-06	\\
-9761.39248934659	2.18827741469466e-06	\\
-9760.41370738636	2.38582269149341e-06	\\
-9759.43492542614	3.17328143555303e-06	\\
-9758.45614346591	2.03564641180625e-06	\\
-9757.47736150568	3.94669068327888e-06	\\
-9756.49857954545	2.98133849149566e-06	\\
-9755.51979758523	2.45394135869741e-06	\\
-9754.541015625	2.80367622408203e-06	\\
-9753.56223366477	2.6758649755234e-06	\\
-9752.58345170455	4.08761397865492e-06	\\
-9751.60466974432	3.12942051103446e-06	\\
-9750.62588778409	3.83331081933299e-06	\\
-9749.64710582386	3.25824903753838e-06	\\
-9748.66832386364	3.52008841895498e-06	\\
-9747.68954190341	2.40506280902511e-06	\\
-9746.71075994318	2.97391609773224e-06	\\
-9745.73197798295	3.10672955599901e-06	\\
-9744.75319602273	2.86942482821341e-06	\\
-9743.7744140625	2.59621046102724e-06	\\
-9742.79563210227	2.93554890265854e-06	\\
-9741.81685014205	3.36401806218765e-06	\\
-9740.83806818182	2.33258510103589e-06	\\
-9739.85928622159	3.18833539516255e-06	\\
-9738.88050426136	2.30593429888925e-06	\\
-9737.90172230114	3.20206524948466e-06	\\
-9736.92294034091	3.52920246561844e-06	\\
-9735.94415838068	2.89548086704648e-06	\\
-9734.96537642045	3.21179838066762e-06	\\
-9733.98659446023	3.76107906235332e-06	\\
-9733.0078125	3.62932616147058e-06	\\
-9732.02903053977	3.48398673512055e-06	\\
-9731.05024857955	2.6248275020334e-06	\\
-9730.07146661932	2.85095628982678e-06	\\
-9729.09268465909	2.9093149660539e-06	\\
-9728.11390269886	2.7893439574722e-06	\\
-9727.13512073864	3.85861644437448e-06	\\
-9726.15633877841	3.40022317120256e-06	\\
-9725.17755681818	3.49864694489694e-06	\\
-9724.19877485795	3.65056824856238e-06	\\
-9723.21999289773	2.86740231304191e-06	\\
-9722.2412109375	2.92697834804081e-06	\\
-9721.26242897727	2.96723687328732e-06	\\
-9720.28364701705	3.66978231484694e-06	\\
-9719.30486505682	4.69932658526163e-06	\\
-9718.32608309659	3.73001544806083e-06	\\
-9717.34730113636	2.94221972915371e-06	\\
-9716.36851917614	2.69981898854297e-06	\\
-9715.38973721591	2.29427175293864e-06	\\
-9714.41095525568	3.46964963521605e-06	\\
-9713.43217329545	3.45045087400546e-06	\\
-9712.45339133523	2.78137231592049e-06	\\
-9711.474609375	3.3303995704966e-06	\\
-9710.49582741477	2.56458679484096e-06	\\
-9709.51704545455	2.65944793192677e-06	\\
-9708.53826349432	3.44145432256257e-06	\\
-9707.55948153409	3.00239463913765e-06	\\
-9706.58069957386	2.70571989107061e-06	\\
-9705.60191761364	3.72614029288689e-06	\\
-9704.62313565341	3.56315828870147e-06	\\
-9703.64435369318	3.08156392846908e-06	\\
-9702.66557173295	3.3576972146777e-06	\\
-9701.68678977273	2.85386320007567e-06	\\
-9700.7080078125	3.87168329663511e-06	\\
-9699.72922585227	2.78699857944507e-06	\\
-9698.75044389205	3.15121159540536e-06	\\
-9697.77166193182	2.85030869010138e-06	\\
-9696.79287997159	3.12637862548099e-06	\\
-9695.81409801136	2.59400193789516e-06	\\
-9694.83531605114	2.81998614823198e-06	\\
-9693.85653409091	2.53441859733947e-06	\\
-9692.87775213068	2.34462334733271e-06	\\
-9691.89897017045	2.62201055573905e-06	\\
-9690.92018821023	3.04648685492987e-06	\\
-9689.94140625	3.54406656530696e-06	\\
-9688.96262428977	3.19622508247135e-06	\\
-9687.98384232955	3.18410484747839e-06	\\
-9687.00506036932	3.76951382693163e-06	\\
-9686.02627840909	2.60560561326318e-06	\\
-9685.04749644886	2.76729380846987e-06	\\
-9684.06871448864	3.35970030894665e-06	\\
-9683.08993252841	3.26425662348743e-06	\\
-9682.11115056818	2.87254444902707e-06	\\
-9681.13236860795	3.05685443943778e-06	\\
-9680.15358664773	3.01738382070053e-06	\\
-9679.1748046875	3.90475327013748e-06	\\
-9678.19602272727	3.03423524118623e-06	\\
-9677.21724076705	3.59267581409738e-06	\\
-9676.23845880682	3.95471765306047e-06	\\
-9675.25967684659	3.14479733634139e-06	\\
-9674.28089488636	3.68899661674434e-06	\\
-9673.30211292614	3.11888372166604e-06	\\
-9672.32333096591	3.35413384003001e-06	\\
-9671.34454900568	3.05497668634349e-06	\\
-9670.36576704545	2.35003177816742e-06	\\
-9669.38698508523	2.55468898211008e-06	\\
-9668.408203125	3.41562623222865e-06	\\
-9667.42942116477	2.24752604123741e-06	\\
-9666.45063920455	3.80004277887694e-06	\\
-9665.47185724432	2.57530586294207e-06	\\
-9664.49307528409	2.99825892411335e-06	\\
-9663.51429332386	2.19229447226101e-06	\\
-9662.53551136364	3.73392680324732e-06	\\
-9661.55672940341	4.07843597733521e-06	\\
-9660.57794744318	3.56139554771791e-06	\\
-9659.59916548295	3.04473866063385e-06	\\
-9658.62038352273	3.30348473333363e-06	\\
-9657.6416015625	2.48179526837459e-06	\\
-9656.66281960227	2.76257142172355e-06	\\
-9655.68403764205	2.50336876561691e-06	\\
-9654.70525568182	2.44699572074021e-06	\\
-9653.72647372159	2.57228610759202e-06	\\
-9652.74769176136	3.91361408821884e-06	\\
-9651.76890980114	2.78272759864254e-06	\\
-9650.79012784091	2.97271322258097e-06	\\
-9649.81134588068	3.82231810661807e-06	\\
-9648.83256392045	4.04787121040937e-06	\\
-9647.85378196023	3.90067916464622e-06	\\
-9646.875	2.70549503244824e-06	\\
-9645.89621803977	3.88013039498225e-06	\\
-9644.91743607955	2.9182950119366e-06	\\
-9643.93865411932	3.08015290076757e-06	\\
-9642.95987215909	3.651286260454e-06	\\
-9641.98109019886	3.87842538998609e-06	\\
-9641.00230823864	3.45009488460365e-06	\\
-9640.02352627841	3.48611988951843e-06	\\
-9639.04474431818	3.11298376624705e-06	\\
-9638.06596235795	3.73469796968328e-06	\\
-9637.08718039773	2.97153405406957e-06	\\
-9636.1083984375	3.39355660027366e-06	\\
-9635.12961647727	3.00077562519061e-06	\\
-9634.15083451705	3.73328824046485e-06	\\
-9633.17205255682	2.96888907987339e-06	\\
-9632.19327059659	3.00239933099355e-06	\\
-9631.21448863636	3.72838111931455e-06	\\
-9630.23570667614	3.41095103940283e-06	\\
-9629.25692471591	3.43335191374385e-06	\\
-9628.27814275568	4.17472853198279e-06	\\
-9627.29936079545	4.19625752094573e-06	\\
-9626.32057883523	3.88253594797666e-06	\\
-9625.341796875	2.96519342481835e-06	\\
-9624.36301491477	2.97349065112043e-06	\\
-9623.38423295455	2.7910419735374e-06	\\
-9622.40545099432	2.82541209126587e-06	\\
-9621.42666903409	2.88445124943637e-06	\\
-9620.44788707386	4.1576393827266e-06	\\
-9619.46910511364	3.17457091230487e-06	\\
-9618.49032315341	3.61584421904221e-06	\\
-9617.51154119318	4.07301891836052e-06	\\
-9616.53275923295	2.91163754086063e-06	\\
-9615.55397727273	3.23280883508697e-06	\\
-9614.5751953125	3.7684103213126e-06	\\
-9613.59641335227	4.03535891506719e-06	\\
-9612.61763139205	3.80905537964432e-06	\\
-9611.63884943182	2.93282841415371e-06	\\
-9610.66006747159	3.43080642815034e-06	\\
-9609.68128551136	2.83869813112839e-06	\\
-9608.70250355114	3.06996015500014e-06	\\
-9607.72372159091	3.00178354466449e-06	\\
-9606.74493963068	3.36435696130086e-06	\\
-9605.76615767045	3.53978318682928e-06	\\
-9604.78737571023	3.08532715145033e-06	\\
-9603.80859375	3.75135239970392e-06	\\
-9602.82981178977	4.01093427980045e-06	\\
-9601.85102982955	4.81659640959983e-06	\\
-9600.87224786932	4.08665268258707e-06	\\
-9599.89346590909	3.27609300900731e-06	\\
-9598.91468394886	3.00992478269719e-06	\\
-9597.93590198864	3.21100854414804e-06	\\
-9596.95712002841	3.06670005055334e-06	\\
-9595.97833806818	4.16674710745133e-06	\\
-9594.99955610795	3.00300827915516e-06	\\
-9594.02077414773	3.17020315488004e-06	\\
-9593.0419921875	3.03969285701067e-06	\\
-9592.06321022727	3.73648711312426e-06	\\
-9591.08442826705	3.62047859184796e-06	\\
-9590.10564630682	4.24270676398935e-06	\\
-9589.12686434659	2.7966971297856e-06	\\
-9588.14808238636	3.77359041869273e-06	\\
-9587.16930042614	3.40919922543321e-06	\\
-9586.19051846591	4.12041568481738e-06	\\
-9585.21173650568	4.46078474406468e-06	\\
-9584.23295454545	4.23987389906606e-06	\\
-9583.25417258523	3.7606364951724e-06	\\
-9582.275390625	3.70866555669843e-06	\\
-9581.29660866477	4.63987003424412e-06	\\
-9580.31782670455	3.73075125071543e-06	\\
-9579.33904474432	4.62413344891489e-06	\\
-9578.36026278409	3.77421372218696e-06	\\
-9577.38148082386	3.86461331305077e-06	\\
-9576.40269886364	4.14701602338035e-06	\\
-9575.42391690341	3.9730374556894e-06	\\
-9574.44513494318	5.33064515503088e-06	\\
-9573.46635298295	3.51673350040046e-06	\\
-9572.48757102273	3.66079222902092e-06	\\
-9571.5087890625	2.82937915430641e-06	\\
-9570.53000710227	3.94691035539564e-06	\\
-9569.55122514205	4.56006240908397e-06	\\
-9568.57244318182	4.04100853859982e-06	\\
-9567.59366122159	4.18203531052221e-06	\\
-9566.61487926136	3.85542718528089e-06	\\
-9565.63609730114	3.34507724602072e-06	\\
-9564.65731534091	3.85232427967867e-06	\\
-9563.67853338068	3.15390369338836e-06	\\
-9562.69975142045	4.16110234394627e-06	\\
-9561.72096946023	3.02785144163166e-06	\\
-9560.7421875	4.26024166016978e-06	\\
-9559.76340553977	4.77193921906823e-06	\\
-9558.78462357955	4.84517330992083e-06	\\
-9557.80584161932	3.96328622294749e-06	\\
-9556.82705965909	3.81719722400868e-06	\\
-9555.84827769886	3.94679766195851e-06	\\
-9554.86949573864	3.72072738655326e-06	\\
-9553.89071377841	4.44616767959791e-06	\\
-9552.91193181818	3.38162282550129e-06	\\
-9551.93314985795	4.53835288910986e-06	\\
-9550.95436789773	4.16697450236014e-06	\\
-9549.9755859375	5.06880901152009e-06	\\
-9548.99680397727	4.87713121866935e-06	\\
-9548.01802201705	4.12680542167389e-06	\\
-9547.03924005682	3.62663217350908e-06	\\
-9546.06045809659	4.1182073788898e-06	\\
-9545.08167613636	3.48681895561831e-06	\\
-9544.10289417614	4.12096552845093e-06	\\
-9543.12411221591	4.58950614384245e-06	\\
-9542.14533025568	4.13757507012853e-06	\\
-9541.16654829545	3.46761182059264e-06	\\
-9540.18776633523	4.4345091102044e-06	\\
-9539.208984375	4.16147715251491e-06	\\
-9538.23020241477	3.55873335154326e-06	\\
-9537.25142045455	4.17496984901526e-06	\\
-9536.27263849432	3.44097732543096e-06	\\
-9535.29385653409	3.64731947883319e-06	\\
-9534.31507457386	4.70816128864773e-06	\\
-9533.33629261364	3.47011979721155e-06	\\
-9532.35751065341	4.78483620356422e-06	\\
-9531.37872869318	3.6586280418518e-06	\\
-9530.39994673295	3.78087734764868e-06	\\
-9529.42116477273	4.76414295875074e-06	\\
-9528.4423828125	5.37068162111542e-06	\\
-9527.46360085227	4.36672541946506e-06	\\
-9526.48481889205	5.28692868054551e-06	\\
-9525.50603693182	4.60799475862112e-06	\\
-9524.52725497159	3.7227306194232e-06	\\
-9523.54847301136	2.99986463094884e-06	\\
-9522.56969105114	5.15795962700059e-06	\\
-9521.59090909091	5.00380519078113e-06	\\
-9520.61212713068	3.20537635940116e-06	\\
-9519.63334517045	4.27952412547766e-06	\\
-9518.65456321023	4.50774547595574e-06	\\
-9517.67578125	4.87045420015379e-06	\\
-9516.69699928977	4.51826355341214e-06	\\
-9515.71821732955	4.32144683351971e-06	\\
-9514.73943536932	3.80429632708133e-06	\\
-9513.76065340909	4.53967078588929e-06	\\
-9512.78187144886	4.59631031589872e-06	\\
-9511.80308948864	4.3056709584796e-06	\\
-9510.82430752841	4.52347635636284e-06	\\
-9509.84552556818	5.65427494310651e-06	\\
-9508.86674360795	4.68436471127456e-06	\\
-9507.88796164773	4.61605883445888e-06	\\
-9506.9091796875	4.11385221757394e-06	\\
-9505.93039772727	5.59643781134623e-06	\\
-9504.95161576705	5.27172949773912e-06	\\
-9503.97283380682	5.2760020675021e-06	\\
-9502.99405184659	4.55926982672462e-06	\\
-9502.01526988636	4.06866686679507e-06	\\
-9501.03648792614	4.48666085646316e-06	\\
-9500.05770596591	5.87511135812812e-06	\\
-9499.07892400568	5.59688359237308e-06	\\
-9498.10014204545	4.31109446220072e-06	\\
-9497.12136008523	4.19284999577339e-06	\\
-9496.142578125	3.74421364667994e-06	\\
-9495.16379616477	4.61431758785104e-06	\\
-9494.18501420455	3.88411225669931e-06	\\
-9493.20623224432	4.38113177913825e-06	\\
-9492.22745028409	4.55916629412818e-06	\\
-9491.24866832386	4.75791140116896e-06	\\
-9490.26988636364	4.47551746627528e-06	\\
-9489.29110440341	5.66386028795969e-06	\\
-9488.31232244318	3.82102503834224e-06	\\
-9487.33354048295	5.44744365407714e-06	\\
-9486.35475852273	4.6873735041858e-06	\\
-9485.3759765625	5.85753706298016e-06	\\
-9484.39719460227	5.01476170911124e-06	\\
-9483.41841264205	5.45931931227674e-06	\\
-9482.43963068182	4.40886895795288e-06	\\
-9481.46084872159	4.53268962459865e-06	\\
-9480.48206676136	4.89465747211898e-06	\\
-9479.50328480114	4.47101555363936e-06	\\
-9478.52450284091	6.02491018085751e-06	\\
-9477.54572088068	5.75362750261626e-06	\\
-9476.56693892045	4.5850716524951e-06	\\
-9475.58815696023	3.91416812305468e-06	\\
-9474.609375	3.12975469761305e-06	\\
-9473.63059303977	4.71352723208783e-06	\\
-9472.65181107955	3.31868609172422e-06	\\
-9471.67302911932	5.29827673779217e-06	\\
-9470.69424715909	3.58986747475239e-06	\\
-9469.71546519886	4.83431937991181e-06	\\
-9468.73668323864	5.04577771649345e-06	\\
-9467.75790127841	5.4819706584226e-06	\\
-9466.77911931818	5.04325394411596e-06	\\
-9465.80033735795	5.86383876744166e-06	\\
-9464.82155539773	5.18973708638571e-06	\\
-9463.8427734375	4.74295767420234e-06	\\
-9462.86399147727	5.31137739633236e-06	\\
-9461.88520951705	4.4068756762602e-06	\\
-9460.90642755682	4.46369401219398e-06	\\
-9459.92764559659	5.37460575527835e-06	\\
-9458.94886363636	4.261861876362e-06	\\
-9457.97008167614	4.98480669973301e-06	\\
-9456.99129971591	4.71016105761761e-06	\\
-9456.01251775568	5.29671503529109e-06	\\
-9455.03373579545	4.08987248091298e-06	\\
-9454.05495383523	4.93576118858457e-06	\\
-9453.076171875	4.74674622742593e-06	\\
-9452.09738991477	5.28064574186348e-06	\\
-9451.11860795455	4.52421626490787e-06	\\
-9450.13982599432	4.42945747429855e-06	\\
-9449.16104403409	5.32310568469689e-06	\\
-9448.18226207386	5.00398743898621e-06	\\
-9447.20348011364	3.95087507050266e-06	\\
-9446.22469815341	4.86512134619459e-06	\\
-9445.24591619318	5.3381727495596e-06	\\
-9444.26713423295	5.17859121864971e-06	\\
-9443.28835227273	4.14003018877488e-06	\\
-9442.3095703125	5.01888463818636e-06	\\
-9441.33078835227	4.07769117457979e-06	\\
-9440.35200639205	5.19422785590928e-06	\\
-9439.37322443182	4.59196020798319e-06	\\
-9438.39444247159	4.47217719687301e-06	\\
-9437.41566051136	4.75355225847305e-06	\\
-9436.43687855114	5.4278591264343e-06	\\
-9435.45809659091	4.93019199531174e-06	\\
-9434.47931463068	5.25117038783119e-06	\\
-9433.50053267045	6.34449172238907e-06	\\
-9432.52175071023	5.83472611758136e-06	\\
-9431.54296875	5.43148104506273e-06	\\
-9430.56418678977	5.83366493505865e-06	\\
-9429.58540482955	5.36061089073215e-06	\\
-9428.60662286932	5.39215986510256e-06	\\
-9427.62784090909	6.14879083494405e-06	\\
-9426.64905894886	5.20457223734284e-06	\\
-9425.67027698864	6.04958429389608e-06	\\
-9424.69149502841	4.73016750047113e-06	\\
-9423.71271306818	4.64822066153428e-06	\\
-9422.73393110795	6.22174075214624e-06	\\
-9421.75514914773	6.31258235526976e-06	\\
-9420.7763671875	4.39757548076527e-06	\\
-9419.79758522727	4.80770301526102e-06	\\
-9418.81880326705	4.87403116708943e-06	\\
-9417.84002130682	4.4293094458738e-06	\\
-9416.86123934659	6.30335421097674e-06	\\
-9415.88245738636	5.92490553831679e-06	\\
-9414.90367542614	3.63577902511525e-06	\\
-9413.92489346591	5.07266776037243e-06	\\
-9412.94611150568	5.4512421031395e-06	\\
-9411.96732954545	4.71214029326691e-06	\\
-9410.98854758523	5.46403032675779e-06	\\
-9410.009765625	4.96480612442166e-06	\\
-9409.03098366477	5.38883421835138e-06	\\
-9408.05220170455	5.68927672902503e-06	\\
-9407.07341974432	4.53888837297448e-06	\\
-9406.09463778409	4.56381266714499e-06	\\
-9405.11585582386	5.34001808096279e-06	\\
-9404.13707386364	5.19171809769303e-06	\\
-9403.15829190341	4.35463613739718e-06	\\
-9402.17950994318	4.55922325408704e-06	\\
-9401.20072798295	3.77265215439584e-06	\\
-9400.22194602273	3.89412814953351e-06	\\
-9399.2431640625	4.31696655851378e-06	\\
-9398.26438210227	5.84685671495795e-06	\\
-9397.28560014205	5.05195490142456e-06	\\
-9396.30681818182	6.21023731828139e-06	\\
-9395.32803622159	5.1987767801174e-06	\\
-9394.34925426136	5.3989175470512e-06	\\
-9393.37047230114	6.02962117679177e-06	\\
-9392.39169034091	5.5052386115806e-06	\\
-9391.41290838068	5.10470240363595e-06	\\
-9390.43412642045	4.52511886097225e-06	\\
-9389.45534446023	4.88666663159616e-06	\\
-9388.4765625	5.84146776174208e-06	\\
-9387.49778053977	5.21180193103597e-06	\\
-9386.51899857955	6.49684610923187e-06	\\
-9385.54021661932	5.14826461312831e-06	\\
-9384.56143465909	4.64103330962922e-06	\\
-9383.58265269886	4.6025726796741e-06	\\
-9382.60387073864	4.61397164631848e-06	\\
-9381.62508877841	4.86537369212674e-06	\\
-9380.64630681818	4.58980671268127e-06	\\
-9379.66752485795	5.31333824832826e-06	\\
-9378.68874289773	5.47965777444141e-06	\\
-9377.7099609375	5.75950725456782e-06	\\
-9376.73117897727	5.54836339267696e-06	\\
-9375.75239701705	4.99978686583182e-06	\\
-9374.77361505682	5.36283431093309e-06	\\
-9373.79483309659	4.50299016493033e-06	\\
-9372.81605113636	4.3726581155822e-06	\\
-9371.83726917614	5.77276406110603e-06	\\
-9370.85848721591	5.30913467273389e-06	\\
-9369.87970525568	4.83056503646442e-06	\\
-9368.90092329545	5.45165309053084e-06	\\
-9367.92214133523	5.30189595983296e-06	\\
-9366.943359375	5.51267704799876e-06	\\
-9365.96457741477	5.1395935128371e-06	\\
-9364.98579545455	4.11041666940964e-06	\\
-9364.00701349432	4.87612594396416e-06	\\
-9363.02823153409	5.55976203562327e-06	\\
-9362.04944957386	5.59318620350679e-06	\\
-9361.07066761364	5.23976008093915e-06	\\
-9360.09188565341	5.53989473615014e-06	\\
-9359.11310369318	4.75085575880427e-06	\\
-9358.13432173295	5.53037431082698e-06	\\
-9357.15553977273	5.78340787253626e-06	\\
-9356.1767578125	4.98959812811227e-06	\\
-9355.19797585227	5.53032042610767e-06	\\
-9354.21919389205	5.55587428664905e-06	\\
-9353.24041193182	4.76692457602398e-06	\\
-9352.26162997159	5.55088011978654e-06	\\
-9351.28284801136	4.69519957634691e-06	\\
-9350.30406605114	4.70125900944143e-06	\\
-9349.32528409091	4.83969184185899e-06	\\
-9348.34650213068	5.03568248574615e-06	\\
-9347.36772017045	5.49353290371248e-06	\\
-9346.38893821023	4.93904098265032e-06	\\
-9345.41015625	4.28964714127904e-06	\\
-9344.43137428977	4.57075049619143e-06	\\
-9343.45259232955	4.70874890129521e-06	\\
-9342.47381036932	4.35172418548703e-06	\\
-9341.49502840909	4.70984345084001e-06	\\
-9340.51624644886	3.64792593464692e-06	\\
-9339.53746448864	4.82227730265636e-06	\\
-9338.55868252841	5.61127699784859e-06	\\
-9337.57990056818	4.20431601992473e-06	\\
-9336.60111860795	4.41787660808266e-06	\\
-9335.62233664773	4.88705400479799e-06	\\
-9334.6435546875	5.62367214895774e-06	\\
-9333.66477272727	5.45236452621173e-06	\\
-9332.68599076705	4.66969837182922e-06	\\
-9331.70720880682	4.79295824871387e-06	\\
-9330.72842684659	5.00515044543286e-06	\\
-9329.74964488636	4.50482105571145e-06	\\
-9328.77086292614	5.0230606356717e-06	\\
-9327.79208096591	5.6412478069223e-06	\\
-9326.81329900568	5.42066090834454e-06	\\
-9325.83451704545	5.13773073976366e-06	\\
-9324.85573508523	5.12253904799755e-06	\\
-9323.876953125	4.77459934889986e-06	\\
-9322.89817116477	5.6635613898197e-06	\\
-9321.91938920455	4.70695227710319e-06	\\
-9320.94060724432	5.40792444742352e-06	\\
-9319.96182528409	5.65247606792088e-06	\\
-9318.98304332386	4.62725656593997e-06	\\
-9318.00426136364	5.59322136616128e-06	\\
-9317.02547940341	5.61691398244959e-06	\\
-9316.04669744318	4.58279300586489e-06	\\
-9315.06791548295	5.17536429528334e-06	\\
-9314.08913352273	6.19726811028026e-06	\\
-9313.1103515625	5.80076680557721e-06	\\
-9312.13156960227	5.72803165712207e-06	\\
-9311.15278764205	4.84094276080213e-06	\\
-9310.17400568182	5.29919135552489e-06	\\
-9309.19522372159	4.94273996823947e-06	\\
-9308.21644176136	5.45489388600153e-06	\\
-9307.23765980114	4.77250646789562e-06	\\
-9306.25887784091	3.83458971970542e-06	\\
-9305.28009588068	5.55881750685459e-06	\\
-9304.30131392045	4.72490226984649e-06	\\
-9303.32253196023	5.55984562032017e-06	\\
-9302.34375	6.04404645379965e-06	\\
-9301.36496803977	5.74853907255074e-06	\\
-9300.38618607955	4.98634251183207e-06	\\
-9299.40740411932	6.07173298788315e-06	\\
-9298.42862215909	4.43678349325298e-06	\\
-9297.44984019886	5.41022908732486e-06	\\
-9296.47105823864	3.25446326845273e-06	\\
-9295.49227627841	4.95591575454857e-06	\\
-9294.51349431818	4.55822420095588e-06	\\
-9293.53471235795	4.83462269730134e-06	\\
-9292.55593039773	4.99906868684343e-06	\\
-9291.5771484375	4.83867749753432e-06	\\
-9290.59836647727	4.617694666881e-06	\\
-9289.61958451705	5.38103310397655e-06	\\
-9288.64080255682	3.61504369231233e-06	\\
-9287.66202059659	5.0980584251628e-06	\\
-9286.68323863636	3.70812849191505e-06	\\
-9285.70445667614	4.80732385905253e-06	\\
-9284.72567471591	4.40964292671832e-06	\\
-9283.74689275568	4.42498411692546e-06	\\
-9282.76811079545	5.46774478577175e-06	\\
-9281.78932883523	5.04578310585597e-06	\\
-9280.810546875	4.63023587068498e-06	\\
-9279.83176491477	5.29336588371147e-06	\\
-9278.85298295455	4.75532930894563e-06	\\
-9277.87420099432	5.75789695197842e-06	\\
-9276.89541903409	3.57412118482675e-06	\\
-9275.91663707386	6.18602463438984e-06	\\
-9274.93785511364	5.39006466647443e-06	\\
-9273.95907315341	5.0923069601073e-06	\\
-9272.98029119318	5.5538039850751e-06	\\
-9272.00150923295	4.42424464930642e-06	\\
-9271.02272727273	5.75725768186161e-06	\\
-9270.0439453125	4.97339165213942e-06	\\
-9269.06516335227	7.16116886323878e-06	\\
-9268.08638139205	3.27417533724591e-06	\\
-9267.10759943182	4.87853888284753e-06	\\
-9266.12881747159	5.05991744527868e-06	\\
-9265.15003551136	4.51808134565292e-06	\\
-9264.17125355114	4.51264077908372e-06	\\
-9263.19247159091	4.96051660658008e-06	\\
-9262.21368963068	5.61212119602386e-06	\\
-9261.23490767045	5.07665486504217e-06	\\
-9260.25612571023	5.89854085203035e-06	\\
-9259.27734375	6.02075250388669e-06	\\
-9258.29856178977	5.00909944465599e-06	\\
-9257.31977982955	5.48219994146444e-06	\\
-9256.34099786932	5.71033232096696e-06	\\
-9255.36221590909	5.07849861343709e-06	\\
-9254.38343394886	4.84854011243215e-06	\\
-9253.40465198864	4.52895233613479e-06	\\
-9252.42587002841	4.59599339359263e-06	\\
-9251.44708806818	5.0950111061877e-06	\\
-9250.46830610795	4.54163523275032e-06	\\
-9249.48952414773	4.75282239632388e-06	\\
-9248.5107421875	4.99050188620963e-06	\\
-9247.53196022727	4.66825074034192e-06	\\
-9246.55317826705	5.26541084195725e-06	\\
-9245.57439630682	5.10107295039556e-06	\\
-9244.59561434659	6.11791890818277e-06	\\
-9243.61683238636	6.33830837476412e-06	\\
-9242.63805042614	3.86871991782643e-06	\\
-9241.65926846591	5.71840351567686e-06	\\
-9240.68048650568	4.24366565446122e-06	\\
-9239.70170454545	3.81777768929471e-06	\\
-9238.72292258523	4.7383367314261e-06	\\
-9237.744140625	4.30176857911018e-06	\\
-9236.76535866477	5.29589074879935e-06	\\
-9235.78657670455	5.8974621968583e-06	\\
-9234.80779474432	4.57616313650755e-06	\\
-9233.82901278409	4.82941898189609e-06	\\
-9232.85023082386	5.05026544053505e-06	\\
-9231.87144886364	4.42726376641217e-06	\\
-9230.89266690341	5.56599082228855e-06	\\
-9229.91388494318	3.92038115628484e-06	\\
-9228.93510298295	5.4652681795066e-06	\\
-9227.95632102273	5.35975721639576e-06	\\
-9226.9775390625	4.66857957993096e-06	\\
-9225.99875710227	5.53890849990195e-06	\\
-9225.01997514205	4.99679320382383e-06	\\
-9224.04119318182	4.86978182937699e-06	\\
-9223.06241122159	4.981109549677e-06	\\
-9222.08362926136	4.45035171734903e-06	\\
-9221.10484730114	5.64831886813786e-06	\\
-9220.12606534091	4.48018023797615e-06	\\
-9219.14728338068	4.83600468876244e-06	\\
-9218.16850142045	5.35688504108375e-06	\\
-9217.18971946023	5.15302069404285e-06	\\
-9216.2109375	5.61178075569749e-06	\\
-9215.23215553977	5.93811610143364e-06	\\
-9214.25337357955	4.65935192139692e-06	\\
-9213.27459161932	5.59308652813364e-06	\\
-9212.29580965909	4.45174463955495e-06	\\
-9211.31702769886	4.93333582081454e-06	\\
-9210.33824573864	4.42746879422842e-06	\\
-9209.35946377841	4.66698034158296e-06	\\
-9208.38068181818	6.11986221217826e-06	\\
-9207.40189985795	5.10898834233942e-06	\\
-9206.42311789773	5.32166198904376e-06	\\
-9205.4443359375	6.17858973874704e-06	\\
-9204.46555397727	5.48104033493509e-06	\\
-9203.48677201705	5.81262608912134e-06	\\
-9202.50799005682	5.28253898565395e-06	\\
-9201.52920809659	4.04559971900927e-06	\\
-9200.55042613636	4.65389345921793e-06	\\
-9199.57164417614	5.15415918090055e-06	\\
-9198.59286221591	4.18501552323819e-06	\\
-9197.61408025568	4.95451480189456e-06	\\
-9196.63529829545	4.05980394722698e-06	\\
-9195.65651633523	5.2392644460839e-06	\\
-9194.677734375	3.75386999507248e-06	\\
-9193.69895241477	5.16315944208516e-06	\\
-9192.72017045455	5.20265071538648e-06	\\
-9191.74138849432	4.63017609412563e-06	\\
-9190.76260653409	4.90775939424823e-06	\\
-9189.78382457386	4.54039889883155e-06	\\
-9188.80504261364	4.99991588554653e-06	\\
-9187.82626065341	5.39561187502881e-06	\\
-9186.84747869318	6.69572309770228e-06	\\
-9185.86869673295	6.02074180111082e-06	\\
-9184.88991477273	4.86185895910712e-06	\\
-9183.9111328125	5.34354943679787e-06	\\
-9182.93235085227	5.5572901981581e-06	\\
-9181.95356889205	3.75635483325114e-06	\\
-9180.97478693182	4.5320964861573e-06	\\
-9179.99600497159	4.7787447002274e-06	\\
-9179.01722301136	4.63495669113249e-06	\\
-9178.03844105114	5.91962981074164e-06	\\
-9177.05965909091	5.09902612880914e-06	\\
-9176.08087713068	3.31952616039504e-06	\\
-9175.10209517045	4.45071642439133e-06	\\
-9174.12331321023	4.31827330621758e-06	\\
-9173.14453125	4.32661829722891e-06	\\
-9172.16574928977	5.77390824578775e-06	\\
-9171.18696732955	4.84344978715131e-06	\\
-9170.20818536932	5.17012071515821e-06	\\
-9169.22940340909	4.89567122394031e-06	\\
-9168.25062144886	5.2895735656431e-06	\\
-9167.27183948864	5.27355402137619e-06	\\
-9166.29305752841	4.91445831099855e-06	\\
-9165.31427556818	4.37753304863494e-06	\\
-9164.33549360795	4.48888038741811e-06	\\
-9163.35671164773	5.11305881941593e-06	\\
-9162.3779296875	5.35652851698607e-06	\\
-9161.39914772727	3.88537282474509e-06	\\
-9160.42036576705	4.70138261478539e-06	\\
-9159.44158380682	5.88535395447183e-06	\\
-9158.46280184659	5.31526492911218e-06	\\
-9157.48401988636	5.00697848848496e-06	\\
-9156.50523792614	5.04197848923864e-06	\\
-9155.52645596591	5.03281451590845e-06	\\
-9154.54767400568	4.87751211251433e-06	\\
-9153.56889204545	6.46182481496955e-06	\\
-9152.59011008523	5.61175404622053e-06	\\
-9151.611328125	4.23835171841367e-06	\\
-9150.63254616477	5.89288616155984e-06	\\
-9149.65376420455	5.12886886043402e-06	\\
-9148.67498224432	4.69513998938121e-06	\\
-9147.69620028409	5.62946873837977e-06	\\
-9146.71741832386	4.43119265111068e-06	\\
-9145.73863636364	4.17506724014637e-06	\\
-9144.75985440341	5.57799347864833e-06	\\
-9143.78107244318	5.23458893135872e-06	\\
-9142.80229048295	4.92014428714672e-06	\\
-9141.82350852273	4.62884582783893e-06	\\
-9140.8447265625	6.95779518260947e-06	\\
-9139.86594460227	4.50665870950913e-06	\\
-9138.88716264205	4.43509829047124e-06	\\
-9137.90838068182	4.75706499241167e-06	\\
-9136.92959872159	5.42060775907465e-06	\\
-9135.95081676136	5.01284738396714e-06	\\
-9134.97203480114	4.77078788285531e-06	\\
-9133.99325284091	6.08041538537382e-06	\\
-9133.01447088068	5.65132174725236e-06	\\
-9132.03568892045	4.12652525513543e-06	\\
-9131.05690696023	4.56863091265958e-06	\\
-9130.078125	5.17996721878991e-06	\\
-9129.09934303977	6.35248894468795e-06	\\
-9128.12056107955	4.28613262811258e-06	\\
-9127.14177911932	5.38662909413593e-06	\\
-9126.16299715909	4.63552037664371e-06	\\
-9125.18421519886	4.74245072153304e-06	\\
-9124.20543323864	4.39537297361547e-06	\\
-9123.22665127841	4.46654939403946e-06	\\
-9122.24786931818	4.72377866710379e-06	\\
-9121.26908735795	5.87967213532302e-06	\\
-9120.29030539773	4.57132544777601e-06	\\
-9119.3115234375	5.28704468417043e-06	\\
-9118.33274147727	5.07259618692623e-06	\\
-9117.35395951705	4.91378618798168e-06	\\
-9116.37517755682	4.57801068260165e-06	\\
-9115.39639559659	4.80539776724446e-06	\\
-9114.41761363636	4.20900188015859e-06	\\
-9113.43883167614	5.40892356020955e-06	\\
-9112.46004971591	4.39061202993842e-06	\\
-9111.48126775568	4.03334631213694e-06	\\
-9110.50248579545	4.16508997333669e-06	\\
-9109.52370383523	5.37816407680803e-06	\\
-9108.544921875	5.65111067432142e-06	\\
-9107.56613991477	4.90472806833066e-06	\\
-9106.58735795455	5.24351360435113e-06	\\
-9105.60857599432	4.30623996423928e-06	\\
-9104.62979403409	5.92981710346055e-06	\\
-9103.65101207386	5.13159247401723e-06	\\
-9102.67223011364	5.34924741999786e-06	\\
-9101.69344815341	5.83451683360434e-06	\\
-9100.71466619318	5.61088783740228e-06	\\
-9099.73588423295	4.24002506567166e-06	\\
-9098.75710227273	5.01136020185328e-06	\\
-9097.7783203125	3.44714178595524e-06	\\
-9096.79953835227	4.61737751210393e-06	\\
-9095.82075639205	4.37034628886602e-06	\\
-9094.84197443182	5.12478084229025e-06	\\
-9093.86319247159	5.86477667649871e-06	\\
-9092.88441051136	5.85614195295112e-06	\\
-9091.90562855114	5.11461425085537e-06	\\
-9090.92684659091	5.72086832811003e-06	\\
-9089.94806463068	5.00771380031694e-06	\\
-9088.96928267045	4.08315174417924e-06	\\
-9087.99050071023	5.57957318374702e-06	\\
-9087.01171875	4.9725179819891e-06	\\
-9086.03293678977	5.39860344878225e-06	\\
-9085.05415482955	6.23096017303163e-06	\\
-9084.07537286932	5.01684740037186e-06	\\
-9083.09659090909	6.12956054081735e-06	\\
-9082.11780894886	5.14477382277203e-06	\\
-9081.13902698864	4.84457372618688e-06	\\
-9080.16024502841	4.24803781357596e-06	\\
-9079.18146306818	5.01425040487155e-06	\\
-9078.20268110795	4.49323509784139e-06	\\
-9077.22389914773	5.78796172505333e-06	\\
-9076.2451171875	5.1032968420385e-06	\\
-9075.26633522727	3.74705939418957e-06	\\
-9074.28755326705	4.35419139818033e-06	\\
-9073.30877130682	4.2053949119024e-06	\\
-9072.32998934659	3.32229305918777e-06	\\
-9071.35120738636	5.19685300310223e-06	\\
-9070.37242542614	3.9299828064537e-06	\\
-9069.39364346591	4.76989220736131e-06	\\
-9068.41486150568	4.68713628638695e-06	\\
-9067.43607954545	5.53524390451664e-06	\\
-9066.45729758523	5.08002054745013e-06	\\
-9065.478515625	5.94069911181038e-06	\\
-9064.49973366477	6.30244104121468e-06	\\
-9063.52095170455	5.11806134322819e-06	\\
-9062.54216974432	4.7141637300597e-06	\\
-9061.56338778409	4.74374432525448e-06	\\
-9060.58460582386	4.68309393732763e-06	\\
-9059.60582386364	4.47834581368376e-06	\\
-9058.62704190341	3.18886433071147e-06	\\
-9057.64825994318	4.60981114731081e-06	\\
-9056.66947798295	5.05036333314179e-06	\\
-9055.69069602273	3.79058974396643e-06	\\
-9054.7119140625	4.58299317664415e-06	\\
-9053.73313210227	5.38867628286564e-06	\\
-9052.75435014205	4.82438520903565e-06	\\
-9051.77556818182	6.2613836634685e-06	\\
-9050.79678622159	6.02003152923557e-06	\\
-9049.81800426136	4.81927578166854e-06	\\
-9048.83922230114	5.37483762671077e-06	\\
-9047.86044034091	4.69557542300515e-06	\\
-9046.88165838068	4.798866887177e-06	\\
-9045.90287642045	6.46259532958803e-06	\\
-9044.92409446023	4.66350589432837e-06	\\
-9043.9453125	4.41141155068554e-06	\\
-9042.96653053977	5.24014072121217e-06	\\
-9041.98774857955	5.29116901766464e-06	\\
-9041.00896661932	5.77283595143322e-06	\\
-9040.03018465909	7.12827284134849e-06	\\
-9039.05140269886	4.5195191702634e-06	\\
-9038.07262073864	4.96926665122229e-06	\\
-9037.09383877841	7.09239682540161e-06	\\
-9036.11505681818	4.11779706079963e-06	\\
-9035.13627485795	5.51000115065948e-06	\\
-9034.15749289773	3.06818612200116e-06	\\
-9033.1787109375	3.89773032892667e-06	\\
-9032.19992897727	4.66068550466239e-06	\\
-9031.22114701705	3.35149537776366e-06	\\
-9030.24236505682	5.67888718660043e-06	\\
-9029.26358309659	6.01078990424308e-06	\\
-9028.28480113636	4.49362597980237e-06	\\
-9027.30601917614	5.28515143333155e-06	\\
-9026.32723721591	4.75548499428507e-06	\\
-9025.34845525568	4.80320230255392e-06	\\
-9024.36967329545	5.10947764730024e-06	\\
-9023.39089133523	4.81881585210566e-06	\\
-9022.412109375	6.02395121566628e-06	\\
-9021.43332741477	4.38378598651854e-06	\\
-9020.45454545455	3.56919828742374e-06	\\
-9019.47576349432	5.66829719150625e-06	\\
-9018.49698153409	4.53323396194986e-06	\\
-9017.51819957386	4.89731820421174e-06	\\
-9016.53941761364	4.54885243890767e-06	\\
-9015.56063565341	5.27204509792741e-06	\\
-9014.58185369318	5.00907512926972e-06	\\
-9013.60307173295	5.04282600066073e-06	\\
-9012.62428977273	5.94648274984025e-06	\\
-9011.6455078125	5.54671077627446e-06	\\
-9010.66672585227	5.30293762757814e-06	\\
-9009.68794389205	5.84118521509079e-06	\\
-9008.70916193182	3.35917063849214e-06	\\
-9007.73037997159	5.7204020852829e-06	\\
-9006.75159801136	6.06168228438749e-06	\\
-9005.77281605114	4.32887144286913e-06	\\
-9004.79403409091	5.80184934065584e-06	\\
-9003.81525213068	4.59638684631431e-06	\\
-9002.83647017045	6.23205702683961e-06	\\
-9001.85768821023	5.96626905079164e-06	\\
-9000.87890625	4.98402970333462e-06	\\
-8999.90012428977	9.20017800101613e-06	\\
-8998.92134232955	5.44295369553428e-06	\\
-8997.94256036932	4.94276732605543e-06	\\
-8996.96377840909	5.6811434009049e-06	\\
-8995.98499644886	3.50033613054182e-06	\\
-8995.00621448864	4.3618565984904e-06	\\
-8994.02743252841	4.8923540051355e-06	\\
-8993.04865056818	4.79360984230599e-06	\\
-8992.06986860795	4.85046698206299e-06	\\
-8991.09108664773	7.25699642185806e-06	\\
-8990.1123046875	3.7421110875351e-06	\\
-8989.13352272727	5.75763282308094e-06	\\
-8988.15474076705	5.76426660794542e-06	\\
-8987.17595880682	4.0633333675413e-06	\\
-8986.19717684659	5.56405546846368e-06	\\
-8985.21839488636	4.75591798655495e-06	\\
-8984.23961292614	5.38959114272744e-06	\\
-8983.26083096591	5.83689397220074e-06	\\
-8982.28204900568	5.93958292794391e-06	\\
-8981.30326704545	5.18766957260431e-06	\\
-8980.32448508523	5.32223737324068e-06	\\
-8979.345703125	5.59446612615311e-06	\\
-8978.36692116477	4.95851109493748e-06	\\
-8977.38813920455	6.01838220307463e-06	\\
-8976.40935724432	5.65332412973948e-06	\\
-8975.43057528409	6.47914168792374e-06	\\
-8974.45179332386	5.93178095752483e-06	\\
-8973.47301136364	5.53452016860948e-06	\\
-8972.49422940341	6.04023764558533e-06	\\
-8971.51544744318	5.38067558652558e-06	\\
-8970.53666548295	3.54339589135186e-06	\\
-8969.55788352273	6.45169743799963e-06	\\
-8968.5791015625	4.94645227869997e-06	\\
-8967.60031960227	3.09104323045239e-06	\\
-8966.62153764205	4.983014072779e-06	\\
-8965.64275568182	4.71596455242027e-06	\\
-8964.66397372159	4.71715410629877e-06	\\
-8963.68519176136	4.90900844872261e-06	\\
-8962.70640980114	5.57678378720222e-06	\\
-8961.72762784091	5.26911312909375e-06	\\
-8960.74884588068	5.51239183755559e-06	\\
-8959.77006392045	4.55521382435242e-06	\\
-8958.79128196023	4.09185356504115e-06	\\
-8957.8125	4.6704058748406e-06	\\
-8956.83371803977	5.01458250879674e-06	\\
-8955.85493607955	4.42921656863391e-06	\\
-8954.87615411932	5.51114825404315e-06	\\
-8953.89737215909	4.85895429242867e-06	\\
-8952.91859019886	6.14191204841484e-06	\\
-8951.93980823864	4.02686086731329e-06	\\
-8950.96102627841	5.63349138559112e-06	\\
-8949.98224431818	5.27078276807174e-06	\\
-8949.00346235795	6.42046645925175e-06	\\
-8948.02468039773	5.04109413670563e-06	\\
-8947.0458984375	4.71921253270085e-06	\\
-8946.06711647727	6.5710396256658e-06	\\
-8945.08833451705	4.19356463451881e-06	\\
-8944.10955255682	4.67529951824958e-06	\\
-8943.13077059659	4.73085215644368e-06	\\
-8942.15198863636	5.88497766272743e-06	\\
-8941.17320667614	4.61812529105089e-06	\\
-8940.19442471591	5.42586160834275e-06	\\
-8939.21564275568	5.42467867394399e-06	\\
-8938.23686079545	5.02610077088277e-06	\\
-8937.25807883523	4.36096604854964e-06	\\
-8936.279296875	5.72708756608524e-06	\\
-8935.30051491477	5.12398460813016e-06	\\
-8934.32173295455	4.81139371085434e-06	\\
-8933.34295099432	4.98288489327948e-06	\\
-8932.36416903409	4.06555148413832e-06	\\
-8931.38538707386	4.76544980188549e-06	\\
-8930.40660511364	3.50338121986012e-06	\\
-8929.42782315341	3.30830255660645e-06	\\
-8928.44904119318	4.39758798601594e-06	\\
-8927.47025923295	4.8493834437686e-06	\\
-8926.49147727273	5.69978109893986e-06	\\
-8925.5126953125	5.66807528999042e-06	\\
-8924.53391335227	5.21476626778185e-06	\\
-8923.55513139205	4.62393517667597e-06	\\
-8922.57634943182	4.33674509298107e-06	\\
-8921.59756747159	7.24602601388784e-06	\\
-8920.61878551136	4.35107744204901e-06	\\
-8919.64000355114	5.730969079273e-06	\\
-8918.66122159091	5.53300746208762e-06	\\
-8917.68243963068	5.16772211995033e-06	\\
-8916.70365767045	5.38331065540017e-06	\\
-8915.72487571023	5.87104203517743e-06	\\
-8914.74609375	4.73444816112341e-06	\\
-8913.76731178977	5.75797216057975e-06	\\
-8912.78852982955	4.15886664256827e-06	\\
-8911.80974786932	5.70611088412839e-06	\\
-8910.83096590909	6.20881555112556e-06	\\
-8909.85218394886	5.74347087380894e-06	\\
-8908.87340198864	5.86482875348004e-06	\\
-8907.89462002841	4.34825796213425e-06	\\
-8906.91583806818	5.3832609079008e-06	\\
-8905.93705610795	5.44286626701213e-06	\\
-8904.95827414773	4.65811649688935e-06	\\
-8903.9794921875	5.92917332375804e-06	\\
-8903.00071022727	5.25749618999513e-06	\\
-8902.02192826705	5.28050898252092e-06	\\
-8901.04314630682	5.53147659064713e-06	\\
-8900.06436434659	4.43576995761636e-06	\\
-8899.08558238636	5.9107242003536e-06	\\
-8898.10680042614	3.45200151677818e-06	\\
-8897.12801846591	5.18275902055884e-06	\\
-8896.14923650568	5.04432265640867e-06	\\
-8895.17045454545	4.73202458944053e-06	\\
-8894.19167258523	4.55798357969295e-06	\\
-8893.212890625	4.14013728174692e-06	\\
-8892.23410866477	5.73037171940363e-06	\\
-8891.25532670455	6.35624338981941e-06	\\
-8890.27654474432	5.61981573063817e-06	\\
-8889.29776278409	5.54729904633099e-06	\\
-8888.31898082386	4.22484152891355e-06	\\
-8887.34019886364	4.87300793480821e-06	\\
-8886.36141690341	6.13584837299344e-06	\\
-8885.38263494318	4.91646493288184e-06	\\
-8884.40385298295	4.82573240045008e-06	\\
-8883.42507102273	5.70329952583767e-06	\\
-8882.4462890625	3.9844992341782e-06	\\
-8881.46750710227	5.41697590423137e-06	\\
-8880.48872514205	6.08481406574694e-06	\\
-8879.50994318182	3.88474916800515e-06	\\
-8878.53116122159	4.4300348003722e-06	\\
-8877.55237926136	4.26407971734252e-06	\\
-8876.57359730114	5.02456280285274e-06	\\
-8875.59481534091	5.57941044838729e-06	\\
-8874.61603338068	4.87137157553499e-06	\\
-8873.63725142045	3.60553385831958e-06	\\
-8872.65846946023	6.08479514659063e-06	\\
-8871.6796875	4.26115718262926e-06	\\
-8870.70090553977	4.61006374986599e-06	\\
-8869.72212357955	5.24172198557914e-06	\\
-8868.74334161932	5.38311867778586e-06	\\
-8867.76455965909	5.30166701746981e-06	\\
-8866.78577769886	4.79189314373915e-06	\\
-8865.80699573864	3.71149675138284e-06	\\
-8864.82821377841	5.58610775064256e-06	\\
-8863.84943181818	5.58708715330326e-06	\\
-8862.87064985795	5.78410812668717e-06	\\
-8861.89186789773	4.98336296870812e-06	\\
-8860.9130859375	5.76840672600965e-06	\\
-8859.93430397727	5.01698166874538e-06	\\
-8858.95552201705	4.31266688722923e-06	\\
-8857.97674005682	4.67439382773905e-06	\\
-8856.99795809659	6.26276566273172e-06	\\
-8856.01917613636	4.47755578171238e-06	\\
-8855.04039417614	4.65286174160619e-06	\\
-8854.06161221591	5.30866625245924e-06	\\
-8853.08283025568	4.05913892480968e-06	\\
-8852.10404829545	4.73965160045619e-06	\\
-8851.12526633523	4.99289041401849e-06	\\
-8850.146484375	5.57885759354921e-06	\\
-8849.16770241477	5.80631565293791e-06	\\
-8848.18892045455	4.80532088256886e-06	\\
-8847.21013849432	5.57619634161027e-06	\\
-8846.23135653409	5.15882986698629e-06	\\
-8845.25257457386	4.97412953607617e-06	\\
-8844.27379261364	3.74396756850586e-06	\\
-8843.29501065341	4.36566471212351e-06	\\
-8842.31622869318	4.66177000267773e-06	\\
-8841.33744673295	3.67420826950667e-06	\\
-8840.35866477273	3.42109866151561e-06	\\
-8839.3798828125	4.53451311807459e-06	\\
-8838.40110085227	4.03098194418211e-06	\\
-8837.42231889205	5.73599098705917e-06	\\
-8836.44353693182	5.74645185249564e-06	\\
-8835.46475497159	7.09296716512204e-06	\\
-8834.48597301136	6.36669374575805e-06	\\
-8833.50719105114	6.33809310702661e-06	\\
-8832.52840909091	5.2514904860959e-06	\\
-8831.54962713068	6.70171787796547e-06	\\
-8830.57084517045	4.72100589438894e-06	\\
-8829.59206321023	5.63527860281678e-06	\\
-8828.61328125	4.62930615575548e-06	\\
-8827.63449928977	3.94091858358413e-06	\\
-8826.65571732955	4.07738018789593e-06	\\
-8825.67693536932	6.67646200944167e-06	\\
-8824.69815340909	6.47641437411411e-06	\\
-8823.71937144886	5.74849176404554e-06	\\
-8822.74058948864	5.92490895336867e-06	\\
-8821.76180752841	5.43387851463692e-06	\\
-8820.78302556818	3.97332205360606e-06	\\
-8819.80424360795	4.40051140809314e-06	\\
-8818.82546164773	4.10526568691948e-06	\\
-8817.8466796875	4.92448487852394e-06	\\
-8816.86789772727	7.11925275066623e-06	\\
-8815.88911576705	4.96342667249711e-06	\\
-8814.91033380682	4.75145390317275e-06	\\
-8813.93155184659	3.79755801056656e-06	\\
-8812.95276988636	4.65572960032991e-06	\\
-8811.97398792614	4.58006671454262e-06	\\
-8810.99520596591	4.62271426944368e-06	\\
-8810.01642400568	5.27148137030667e-06	\\
-8809.03764204545	4.57371771850079e-06	\\
-8808.05886008523	4.80908889785003e-06	\\
-8807.080078125	3.58246072088697e-06	\\
-8806.10129616477	6.05411456145554e-06	\\
-8805.12251420455	4.40649736093813e-06	\\
-8804.14373224432	5.66802528120313e-06	\\
-8803.16495028409	6.2830053327552e-06	\\
-8802.18616832386	6.55931195467283e-06	\\
-8801.20738636364	6.22384114454755e-06	\\
-8800.22860440341	5.78545053589228e-06	\\
-8799.24982244318	5.76675710109506e-06	\\
-8798.27104048295	5.20633363605439e-06	\\
-8797.29225852273	5.51806775450344e-06	\\
-8796.3134765625	5.8241306960424e-06	\\
-8795.33469460227	5.60769505912521e-06	\\
-8794.35591264205	4.35395884849444e-06	\\
-8793.37713068182	5.23355464674796e-06	\\
-8792.39834872159	5.39942746217128e-06	\\
-8791.41956676136	4.42865319625188e-06	\\
-8790.44078480114	4.72284692925792e-06	\\
-8789.46200284091	6.03838886741663e-06	\\
-8788.48322088068	5.95033240267217e-06	\\
-8787.50443892045	5.42234604801913e-06	\\
-8786.52565696023	5.49673834460517e-06	\\
-8785.546875	5.96291035681047e-06	\\
-8784.56809303977	5.69927582755864e-06	\\
-8783.58931107955	5.44280849843286e-06	\\
-8782.61052911932	5.43036622145142e-06	\\
-8781.63174715909	4.59984542085477e-06	\\
-8780.65296519886	3.36713145117808e-06	\\
-8779.67418323864	5.28895712201528e-06	\\
-8778.69540127841	3.92138996898838e-06	\\
-8777.71661931818	4.22220445123614e-06	\\
-8776.73783735795	5.08805688217413e-06	\\
-8775.75905539773	6.56023397901518e-06	\\
-8774.7802734375	5.42928782397259e-06	\\
-8773.80149147727	6.39667194350365e-06	\\
-8772.82270951705	5.21726150194333e-06	\\
-8771.84392755682	5.16495003719901e-06	\\
-8770.86514559659	5.94620308569072e-06	\\
-8769.88636363636	5.31837005367821e-06	\\
-8768.90758167614	5.79703279902566e-06	\\
-8767.92879971591	6.02311804897406e-06	\\
-8766.95001775568	5.74781145845821e-06	\\
-8765.97123579545	5.68777613029393e-06	\\
-8764.99245383523	5.21005811583703e-06	\\
-8764.013671875	4.3732786989491e-06	\\
-8763.03488991477	6.36620268512147e-06	\\
-8762.05610795455	4.7992004382159e-06	\\
-8761.07732599432	3.80822818056109e-06	\\
-8760.09854403409	4.93705842135856e-06	\\
-8759.11976207386	5.50816300088392e-06	\\
-8758.14098011364	3.73657335752306e-06	\\
-8757.16219815341	6.68235210550091e-06	\\
-8756.18341619318	6.01697350332285e-06	\\
-8755.20463423295	5.5330991960484e-06	\\
-8754.22585227273	5.93305348737411e-06	\\
-8753.2470703125	5.30337105256542e-06	\\
-8752.26828835227	6.62443703537445e-06	\\
-8751.28950639205	4.39314045306659e-06	\\
-8750.31072443182	9.66020526937903e-06	\\
-8749.33194247159	5.25986604679302e-06	\\
-8748.35316051136	6.83871180181088e-06	\\
-8747.37437855114	4.57188058752673e-06	\\
-8746.39559659091	5.37366856132545e-06	\\
-8745.41681463068	5.51387179368294e-06	\\
-8744.43803267045	7.26782938298378e-06	\\
-8743.45925071023	5.301992128602e-06	\\
-8742.48046875	6.11233996212533e-06	\\
-8741.50168678977	4.49137664320842e-06	\\
-8740.52290482955	4.03494831612742e-06	\\
-8739.54412286932	4.78373872804511e-06	\\
-8738.56534090909	5.33382207588166e-06	\\
-8737.58655894886	6.30228403069766e-06	\\
-8736.60777698864	5.75189460228251e-06	\\
-8735.62899502841	6.14628356077525e-06	\\
-8734.65021306818	5.15149678700942e-06	\\
-8733.67143110795	6.54435708462141e-06	\\
-8732.69264914773	7.28861567606394e-06	\\
-8731.7138671875	4.76863302446755e-06	\\
-8730.73508522727	6.12425660958809e-06	\\
-8729.75630326705	6.27684333634683e-06	\\
-8728.77752130682	6.02730505994664e-06	\\
-8727.79873934659	5.96531145223848e-06	\\
-8726.81995738636	4.8538074923795e-06	\\
-8725.84117542614	4.56683764752042e-06	\\
-8724.86239346591	5.53203036070596e-06	\\
-8723.88361150568	5.09283250787385e-06	\\
-8722.90482954545	4.66020653006422e-06	\\
-8721.92604758523	5.52850565392071e-06	\\
-8720.947265625	5.19915919991196e-06	\\
-8719.96848366477	5.35510914748584e-06	\\
-8718.98970170455	5.17865200896462e-06	\\
-8718.01091974432	5.23785314041512e-06	\\
-8717.03213778409	5.34080722925346e-06	\\
-8716.05335582386	7.31952788546477e-06	\\
-8715.07457386364	5.89617373226856e-06	\\
-8714.09579190341	5.78164411385372e-06	\\
-8713.11700994318	5.03787828072039e-06	\\
-8712.13822798295	5.17717987769227e-06	\\
-8711.15944602273	5.75404511572662e-06	\\
-8710.1806640625	6.0164907713327e-06	\\
-8709.20188210227	5.82905578914333e-06	\\
-8708.22310014205	7.27586378966641e-06	\\
-8707.24431818182	5.06786032329942e-06	\\
-8706.26553622159	4.82544168338126e-06	\\
-8705.28675426136	6.31352887293708e-06	\\
-8704.30797230114	6.26935509629304e-06	\\
-8703.32919034091	4.95947014794087e-06	\\
-8702.35040838068	4.57039804757716e-06	\\
-8701.37162642045	6.16734444648064e-06	\\
-8700.39284446023	6.0822855566778e-06	\\
-8699.4140625	6.54741064008378e-06	\\
-8698.43528053977	4.24824744157032e-06	\\
-8697.45649857955	5.65714365783909e-06	\\
-8696.47771661932	5.24034159383047e-06	\\
-8695.49893465909	4.79934330967686e-06	\\
-8694.52015269886	4.58711964362581e-06	\\
-8693.54137073864	5.7092842727823e-06	\\
-8692.56258877841	6.73241156548067e-06	\\
-8691.58380681818	4.63986941037758e-06	\\
-8690.60502485795	4.45605506176183e-06	\\
-8689.62624289773	6.26971070682613e-06	\\
-8688.6474609375	5.13130804544696e-06	\\
-8687.66867897727	4.48891918318677e-06	\\
-8686.68989701705	5.33280013779725e-06	\\
-8685.71111505682	5.00474722743891e-06	\\
-8684.73233309659	7.86277630933032e-06	\\
-8683.75355113636	4.86265064339824e-06	\\
-8682.77476917614	6.09666635313098e-06	\\
-8681.79598721591	6.15217451244296e-06	\\
-8680.81720525568	4.16180770349086e-06	\\
-8679.83842329545	5.83569840939729e-06	\\
-8678.85964133523	5.15403556971326e-06	\\
-8677.880859375	5.21439583522837e-06	\\
-8676.90207741477	4.6771000049633e-06	\\
-8675.92329545455	5.79468047885017e-06	\\
-8674.94451349432	5.11372132011726e-06	\\
-8673.96573153409	5.06909555061175e-06	\\
-8672.98694957386	5.64762134647138e-06	\\
-8672.00816761364	5.61538813349067e-06	\\
-8671.02938565341	5.38627057681604e-06	\\
-8670.05060369318	5.18777488654112e-06	\\
-8669.07182173295	4.31487570606644e-06	\\
-8668.09303977273	6.03414679484918e-06	\\
-8667.1142578125	5.78525675316477e-06	\\
-8666.13547585227	4.68391160599964e-06	\\
-8665.15669389205	6.88344475626499e-06	\\
-8664.17791193182	3.05230265939304e-06	\\
-8663.19912997159	4.46134772704234e-06	\\
-8662.22034801136	5.86799026983104e-06	\\
-8661.24156605114	6.8339347842702e-06	\\
-8660.26278409091	4.13541525146483e-06	\\
-8659.28400213068	5.76635673770719e-06	\\
-8658.30522017045	5.54325401739832e-06	\\
-8657.32643821023	4.80203507459865e-06	\\
-8656.34765625	5.20040363287387e-06	\\
-8655.36887428977	5.54365683843613e-06	\\
-8654.39009232955	7.04479996371526e-06	\\
-8653.41131036932	5.7282185244286e-06	\\
-8652.43252840909	5.07974609732137e-06	\\
-8651.45374644886	6.78988785485756e-06	\\
-8650.47496448864	5.60022388051925e-06	\\
-8649.49618252841	6.74203601717758e-06	\\
-8648.51740056818	6.75775305388137e-06	\\
-8647.53861860795	5.71980563591302e-06	\\
-8646.55983664773	4.7315805499235e-06	\\
-8645.5810546875	7.10782938515429e-06	\\
-8644.60227272727	5.41039117885248e-06	\\
-8643.62349076705	5.39011188592242e-06	\\
-8642.64470880682	5.60273496365144e-06	\\
-8641.66592684659	5.72539287005212e-06	\\
-8640.68714488636	4.99387831137429e-06	\\
-8639.70836292614	4.7755595569559e-06	\\
-8638.72958096591	5.6592423839212e-06	\\
-8637.75079900568	5.25645716603245e-06	\\
-8636.77201704545	5.0423604296972e-06	\\
-8635.79323508523	5.05287167175521e-06	\\
-8634.814453125	6.30244319477694e-06	\\
-8633.83567116477	4.44645584907217e-06	\\
-8632.85688920455	3.21395484956499e-06	\\
-8631.87810724432	6.22941354960873e-06	\\
-8630.89932528409	5.06356693671996e-06	\\
-8629.92054332386	6.02365369780979e-06	\\
-8628.94176136364	5.06649670309938e-06	\\
-8627.96297940341	4.66593544070034e-06	\\
-8626.98419744318	6.81097180080661e-06	\\
-8626.00541548295	5.31119075739469e-06	\\
-8625.02663352273	5.32531109349911e-06	\\
-8624.0478515625	5.67165742036094e-06	\\
-8623.06906960227	4.87215461920182e-06	\\
-8622.09028764205	6.28934592198935e-06	\\
-8621.11150568182	5.37077068196064e-06	\\
-8620.13272372159	4.64015531382344e-06	\\
-8619.15394176136	5.20633049563682e-06	\\
-8618.17515980114	5.98482556878363e-06	\\
-8617.19637784091	4.55083248024213e-06	\\
-8616.21759588068	5.56134116723208e-06	\\
-8615.23881392045	6.1481338498136e-06	\\
-8614.26003196023	5.608686139921e-06	\\
-8613.28125	4.97218343252285e-06	\\
-8612.30246803977	5.69925103653245e-06	\\
-8611.32368607955	5.01463364868784e-06	\\
-8610.34490411932	4.98658132352625e-06	\\
-8609.36612215909	5.35398590832717e-06	\\
-8608.38734019886	5.17711652730667e-06	\\
-8607.40855823864	5.53624515160678e-06	\\
-8606.42977627841	2.70936899745534e-06	\\
-8605.45099431818	5.38627124861789e-06	\\
-8604.47221235795	6.09289498323208e-06	\\
-8603.49343039773	5.19283213403767e-06	\\
-8602.5146484375	5.60607761104895e-06	\\
-8601.53586647727	5.00841239179353e-06	\\
-8600.55708451705	7.69372510439241e-06	\\
-8599.57830255682	5.07217339406235e-06	\\
-8598.59952059659	5.6343020061413e-06	\\
-8597.62073863636	4.81262197270607e-06	\\
-8596.64195667614	6.88567812377062e-06	\\
-8595.66317471591	5.14151027699135e-06	\\
-8594.68439275568	5.34661361930728e-06	\\
-8593.70561079545	5.44402405860557e-06	\\
-8592.72682883523	5.04667375482436e-06	\\
-8591.748046875	4.3533289418104e-06	\\
-8590.76926491477	5.65342730676499e-06	\\
-8589.79048295455	5.38641932295823e-06	\\
-8588.81170099432	5.56557893873912e-06	\\
-8587.83291903409	6.48990575296338e-06	\\
-8586.85413707386	5.24004628589225e-06	\\
-8585.87535511364	4.28306392943268e-06	\\
-8584.89657315341	5.06031799908859e-06	\\
-8583.91779119318	4.3308916300407e-06	\\
-8582.93900923295	4.83187886053071e-06	\\
-8581.96022727273	4.59176188522058e-06	\\
-8580.9814453125	6.5756071457651e-06	\\
-8580.00266335227	5.61530458763069e-06	\\
-8579.02388139205	6.09350381076581e-06	\\
-8578.04509943182	3.61536547142684e-06	\\
-8577.06631747159	6.06681381832313e-06	\\
-8576.08753551136	6.14754927379786e-06	\\
-8575.10875355114	5.47334079978452e-06	\\
-8574.12997159091	5.65568056651491e-06	\\
-8573.15118963068	4.80467634369899e-06	\\
-8572.17240767045	4.92923639289478e-06	\\
-8571.19362571023	6.43349887491649e-06	\\
-8570.21484375	6.1287294805711e-06	\\
-8569.23606178977	6.36250317953984e-06	\\
-8568.25727982955	4.47572933769975e-06	\\
-8567.27849786932	4.0099172510908e-06	\\
-8566.29971590909	4.94956182766422e-06	\\
-8565.32093394886	4.84703314234015e-06	\\
-8564.34215198864	7.56619085133061e-06	\\
-8563.36337002841	5.2345505949587e-06	\\
-8562.38458806818	3.11722452615378e-06	\\
-8561.40580610795	5.66537440083665e-06	\\
-8560.42702414773	5.8685553837714e-06	\\
-8559.4482421875	3.76532450912798e-06	\\
-8558.46946022727	5.0109165784718e-06	\\
-8557.49067826705	4.02110630329736e-06	\\
-8556.51189630682	6.79595913016726e-06	\\
-8555.53311434659	4.0471471210066e-06	\\
-8554.55433238636	4.30208102552254e-06	\\
-8553.57555042614	5.46233318689677e-06	\\
-8552.59676846591	6.05568446495179e-06	\\
-8551.61798650568	5.28507864329868e-06	\\
-8550.63920454545	4.50283360994326e-06	\\
-8549.66042258523	5.54322918208629e-06	\\
-8548.681640625	5.6694372269357e-06	\\
-8547.70285866477	5.41549048597735e-06	\\
-8546.72407670455	5.57520764868439e-06	\\
-8545.74529474432	4.67866662841051e-06	\\
-8544.76651278409	5.30093386275348e-06	\\
-8543.78773082386	4.82943343139512e-06	\\
-8542.80894886364	4.23052039442061e-06	\\
-8541.83016690341	5.41408972568649e-06	\\
-8540.85138494318	5.53261517405009e-06	\\
-8539.87260298295	5.81453134303812e-06	\\
-8538.89382102273	6.38324942790593e-06	\\
-8537.9150390625	6.52291502940891e-06	\\
-8536.93625710227	4.99426533117099e-06	\\
-8535.95747514205	6.3233270692388e-06	\\
-8534.97869318182	5.47717440409322e-06	\\
-8533.99991122159	5.15968349372955e-06	\\
-8533.02112926136	4.60152396567058e-06	\\
-8532.04234730114	4.37524429467817e-06	\\
-8531.06356534091	4.15094176514102e-06	\\
-8530.08478338068	4.59577165211168e-06	\\
-8529.10600142045	5.62273513692873e-06	\\
-8528.12721946023	6.08955645717257e-06	\\
-8527.1484375	5.29830962126851e-06	\\
-8526.16965553977	5.38936909842444e-06	\\
-8525.19087357955	6.19409879981441e-06	\\
-8524.21209161932	5.00583216029358e-06	\\
-8523.23330965909	6.16772629234693e-06	\\
-8522.25452769886	4.27881578004933e-06	\\
-8521.27574573864	3.81021230319034e-06	\\
-8520.29696377841	5.51325484198603e-06	\\
-8519.31818181818	4.87217498496042e-06	\\
-8518.33939985795	4.14124237680001e-06	\\
-8517.36061789773	5.20360685916014e-06	\\
-8516.3818359375	7.21343480900687e-06	\\
-8515.40305397727	7.38873352518045e-06	\\
-8514.42427201705	4.70835946998762e-06	\\
-8513.44549005682	4.56160219680114e-06	\\
-8512.46670809659	4.77464948061342e-06	\\
-8511.48792613636	4.8893989898377e-06	\\
-8510.50914417614	4.41011312011919e-06	\\
-8509.53036221591	3.78750312889126e-06	\\
-8508.55158025568	5.93925687695452e-06	\\
-8507.57279829545	4.56572642762801e-06	\\
-8506.59401633523	4.12057559373661e-06	\\
-8505.615234375	4.47839709543295e-06	\\
-8504.63645241477	4.77548130052502e-06	\\
-8503.65767045455	6.39419039845813e-06	\\
-8502.67888849432	5.46868586779261e-06	\\
-8501.70010653409	5.06159491131062e-06	\\
-8500.72132457386	7.64204559214928e-06	\\
-8499.74254261364	4.31346024328387e-06	\\
-8498.76376065341	4.97365808941426e-06	\\
-8497.78497869318	4.23835230120636e-06	\\
-8496.80619673295	5.84873866128141e-06	\\
-8495.82741477273	6.62717752108849e-06	\\
-8494.8486328125	4.35606719876596e-06	\\
-8493.86985085227	3.87718844942375e-06	\\
-8492.89106889205	5.09449487691581e-06	\\
-8491.91228693182	5.05143410398106e-06	\\
-8490.93350497159	4.08569854404416e-06	\\
-8489.95472301136	4.41115450112621e-06	\\
-8488.97594105114	3.57199920826842e-06	\\
-8487.99715909091	5.12912979200791e-06	\\
-8487.01837713068	5.67011210598624e-06	\\
-8486.03959517045	4.51109492886919e-06	\\
-8485.06081321023	4.69880685167271e-06	\\
-8484.08203125	4.2106581482528e-06	\\
-8483.10324928977	5.41806228739826e-06	\\
-8482.12446732955	4.86791357236153e-06	\\
-8481.14568536932	5.36126635102337e-06	\\
-8480.16690340909	5.44311682857402e-06	\\
-8479.18812144886	5.59605733576802e-06	\\
-8478.20933948864	4.67724621118625e-06	\\
-8477.23055752841	4.35654571773099e-06	\\
-8476.25177556818	4.53739716849803e-06	\\
-8475.27299360795	3.9832490143528e-06	\\
-8474.29421164773	3.69109942985802e-06	\\
-8473.3154296875	5.01382667907784e-06	\\
-8472.33664772727	5.26725062995097e-06	\\
-8471.35786576705	5.23138938741361e-06	\\
-8470.37908380682	4.4546652205221e-06	\\
-8469.40030184659	4.60300607914995e-06	\\
-8468.42151988636	5.92643879546657e-06	\\
-8467.44273792614	6.67400669087726e-06	\\
-8466.46395596591	4.21039815951899e-06	\\
-8465.48517400568	3.83255702132187e-06	\\
-8464.50639204545	5.48116167883599e-06	\\
-8463.52761008523	4.81130612171139e-06	\\
-8462.548828125	5.92160993190788e-06	\\
-8461.57004616477	4.94462979698626e-06	\\
-8460.59126420455	5.62355730161947e-06	\\
-8459.61248224432	3.27060122007752e-06	\\
-8458.63370028409	5.43907648430145e-06	\\
-8457.65491832386	6.23207960933716e-06	\\
-8456.67613636364	4.0267260533187e-06	\\
-8455.69735440341	6.17761146361442e-06	\\
-8454.71857244318	4.69720578795749e-06	\\
-8453.73979048295	6.03143819716291e-06	\\
-8452.76100852273	5.72475148838072e-06	\\
-8451.7822265625	5.15870430615918e-06	\\
-8450.80344460227	5.86273379173293e-06	\\
-8449.82466264205	5.47897108607348e-06	\\
-8448.84588068182	4.85868759108935e-06	\\
-8447.86709872159	5.38475718965571e-06	\\
-8446.88831676136	4.74254509041988e-06	\\
-8445.90953480114	4.94021832985131e-06	\\
-8444.93075284091	4.49033053295165e-06	\\
-8443.95197088068	6.31473642736915e-06	\\
-8442.97318892045	4.94307606681226e-06	\\
-8441.99440696023	4.80491035384268e-06	\\
-8441.015625	5.13540128907095e-06	\\
-8440.03684303977	5.13575659112296e-06	\\
-8439.05806107955	3.66289139456254e-06	\\
-8438.07927911932	4.52189756284648e-06	\\
-8437.10049715909	4.37655659580758e-06	\\
-8436.12171519886	4.88343879996402e-06	\\
-8435.14293323864	4.2218070794778e-06	\\
-8434.16415127841	4.86820215407126e-06	\\
-8433.18536931818	5.15037352801527e-06	\\
-8432.20658735795	5.75791001669641e-06	\\
-8431.22780539773	6.1749944141467e-06	\\
-8430.2490234375	4.94806819871318e-06	\\
-8429.27024147727	4.03325065088156e-06	\\
-8428.29145951705	6.6470808609104e-06	\\
-8427.31267755682	4.99814536682856e-06	\\
-8426.33389559659	5.05382316314769e-06	\\
-8425.35511363636	4.45845557464349e-06	\\
-8424.37633167614	6.4859597398862e-06	\\
-8423.39754971591	3.13821344381376e-06	\\
-8422.41876775568	6.51833196973623e-06	\\
-8421.43998579545	5.75755720060261e-06	\\
-8420.46120383523	6.85422033884648e-06	\\
-8419.482421875	5.72984848851098e-06	\\
-8418.50363991477	5.4646120591351e-06	\\
-8417.52485795455	5.2653027638298e-06	\\
-8416.54607599432	5.2834346102731e-06	\\
-8415.56729403409	5.55267242011482e-06	\\
-8414.58851207386	4.9461742547238e-06	\\
-8413.60973011364	5.9772980213353e-06	\\
-8412.63094815341	4.33512444730005e-06	\\
-8411.65216619318	5.2551306611612e-06	\\
-8410.67338423295	6.030402153509e-06	\\
-8409.69460227273	6.04661905319171e-06	\\
-8408.7158203125	5.32274357518576e-06	\\
-8407.73703835227	6.3892864451178e-06	\\
-8406.75825639205	4.63854987246667e-06	\\
-8405.77947443182	4.67812360587818e-06	\\
-8404.80069247159	4.91611658004466e-06	\\
-8403.82191051136	4.18666441694668e-06	\\
-8402.84312855114	5.28746363249511e-06	\\
-8401.86434659091	5.09419193251939e-06	\\
-8400.88556463068	5.14285072976974e-06	\\
-8399.90678267045	4.66785733674442e-06	\\
-8398.92800071023	5.33492881963164e-06	\\
-8397.94921875	4.58981367922114e-06	\\
-8396.97043678977	4.08434408344563e-06	\\
-8395.99165482955	4.33805545511495e-06	\\
-8395.01287286932	4.33363754865213e-06	\\
-8394.03409090909	3.47540676875468e-06	\\
-8393.05530894886	4.45365222434051e-06	\\
-8392.07652698864	5.86613186670447e-06	\\
-8391.09774502841	4.88339179598991e-06	\\
-8390.11896306818	7.53237058579631e-06	\\
-8389.14018110795	5.07864857076395e-06	\\
-8388.16139914773	4.91053117917713e-06	\\
-8387.1826171875	3.91021019125885e-06	\\
-8386.20383522727	5.46028219866802e-06	\\
-8385.22505326705	4.73696026261489e-06	\\
-8384.24627130682	5.78460792492836e-06	\\
-8383.26748934659	5.73243220799325e-06	\\
-8382.28870738636	4.12070132886172e-06	\\
-8381.30992542614	5.71881460259013e-06	\\
-8380.33114346591	4.79318497399853e-06	\\
-8379.35236150568	5.83837213364287e-06	\\
-8378.37357954545	5.26833895416819e-06	\\
-8377.39479758523	4.52038707218062e-06	\\
-8376.416015625	5.94519429946438e-06	\\
-8375.43723366477	3.70759440631366e-06	\\
-8374.45845170455	6.87819422460147e-06	\\
-8373.47966974432	5.1298873728259e-06	\\
-8372.50088778409	4.00578228531342e-06	\\
-8371.52210582386	5.79723151435165e-06	\\
-8370.54332386364	4.61488470890837e-06	\\
-8369.56454190341	4.38514889941662e-06	\\
-8368.58575994318	5.40014745575697e-06	\\
-8367.60697798295	5.13962445133426e-06	\\
-8366.62819602273	6.20615523627562e-06	\\
-8365.6494140625	4.15760147170708e-06	\\
-8364.67063210227	5.78424164496744e-06	\\
-8363.69185014205	5.24274021455948e-06	\\
-8362.71306818182	5.03947054921853e-06	\\
-8361.73428622159	5.31178457137664e-06	\\
-8360.75550426136	6.1970334690942e-06	\\
-8359.77672230114	5.68446246553269e-06	\\
-8358.79794034091	4.36594062975103e-06	\\
-8357.81915838068	6.11412414564932e-06	\\
-8356.84037642045	4.61501311123946e-06	\\
-8355.86159446023	3.19743776834304e-06	\\
-8354.8828125	3.48533376159153e-06	\\
-8353.90403053977	4.56620214707051e-06	\\
-8352.92524857955	5.66052760144552e-06	\\
-8351.94646661932	4.46183021431782e-06	\\
-8350.96768465909	4.35155129140728e-06	\\
-8349.98890269886	3.73767015398263e-06	\\
-8349.01012073864	4.46022500354725e-06	\\
-8348.03133877841	4.26038397445366e-06	\\
-8347.05255681818	3.88022962111197e-06	\\
-8346.07377485795	5.37435781675378e-06	\\
-8345.09499289773	3.55066011886187e-06	\\
-8344.1162109375	3.89782641704067e-06	\\
-8343.13742897727	5.25507489930529e-06	\\
-8342.15864701705	4.30459062710084e-06	\\
-8341.17986505682	5.46761488494529e-06	\\
-8340.20108309659	4.59440859619521e-06	\\
-8339.22230113636	5.64557343135846e-06	\\
-8338.24351917614	4.98562533590352e-06	\\
-8337.26473721591	4.43734928465653e-06	\\
-8336.28595525568	4.52116865130172e-06	\\
-8335.30717329545	5.72205368613416e-06	\\
-8334.32839133523	4.35298650686651e-06	\\
-8333.349609375	5.50720817856813e-06	\\
-8332.37082741477	4.0216033390027e-06	\\
-8331.39204545455	4.61904395668464e-06	\\
-8330.41326349432	5.95869782958224e-06	\\
-8329.43448153409	4.98546545558552e-06	\\
-8328.45569957386	5.26729736189236e-06	\\
-8327.47691761364	5.33946987045495e-06	\\
-8326.49813565341	6.12574848069471e-06	\\
-8325.51935369318	4.33551493676886e-06	\\
-8324.54057173295	5.53391847926754e-06	\\
-8323.56178977273	4.9059195804434e-06	\\
-8322.5830078125	6.27523978074576e-06	\\
-8321.60422585227	4.00520464880153e-06	\\
-8320.62544389205	4.35345112409146e-06	\\
-8319.64666193182	5.26818443569006e-06	\\
-8318.66787997159	4.53525637825842e-06	\\
-8317.68909801136	5.09436088189645e-06	\\
-8316.71031605114	4.20472996398816e-06	\\
-8315.73153409091	3.6570258714325e-06	\\
-8314.75275213068	5.55821962915119e-06	\\
-8313.77397017045	6.1417871215923e-06	\\
-8312.79518821023	5.54496810732774e-06	\\
-8311.81640625	3.83028373360051e-06	\\
-8310.83762428977	4.47406729560157e-06	\\
-8309.85884232955	5.11367104458784e-06	\\
-8308.88006036932	5.3557922393624e-06	\\
-8307.90127840909	4.28375400414154e-06	\\
-8306.92249644886	5.57662320196367e-06	\\
-8305.94371448864	5.24157211204743e-06	\\
-8304.96493252841	6.50790491267692e-06	\\
-8303.98615056818	2.92846606502063e-06	\\
-8303.00736860795	4.13939324512993e-06	\\
-8302.02858664773	5.27382520115631e-06	\\
-8301.0498046875	4.18410085934921e-06	\\
-8300.07102272727	5.35599614983905e-06	\\
-8299.09224076705	4.04851669958278e-06	\\
-8298.11345880682	5.58507354374956e-06	\\
-8297.13467684659	4.56422395734335e-06	\\
-8296.15589488636	5.5647556073654e-06	\\
-8295.17711292614	5.12871944020726e-06	\\
-8294.19833096591	5.46342489032508e-06	\\
-8293.21954900568	5.02749863735475e-06	\\
-8292.24076704545	4.0055969806618e-06	\\
-8291.26198508523	5.54182080136792e-06	\\
-8290.283203125	4.13983357946788e-06	\\
-8289.30442116477	6.49325180859172e-06	\\
-8288.32563920455	4.16171000988208e-06	\\
-8287.34685724432	4.28721274995559e-06	\\
-8286.36807528409	4.33008847649265e-06	\\
-8285.38929332386	4.86950099006534e-06	\\
-8284.41051136364	6.63630031639827e-06	\\
-8283.43172940341	4.86731017429731e-06	\\
-8282.45294744318	4.01733020602376e-06	\\
-8281.47416548295	4.26307984295083e-06	\\
-8280.49538352273	5.68938713163483e-06	\\
-8279.5166015625	4.68001013128388e-06	\\
-8278.53781960227	4.96982534903962e-06	\\
-8277.55903764205	6.25596840006164e-06	\\
-8276.58025568182	4.65399796362037e-06	\\
-8275.60147372159	4.26034291787368e-06	\\
-8274.62269176136	5.49313766926662e-06	\\
-8273.64390980114	5.23505391155859e-06	\\
-8272.66512784091	4.80510955971343e-06	\\
-8271.68634588068	5.3416852431813e-06	\\
-8270.70756392045	4.93983489051059e-06	\\
-8269.72878196023	4.77261940895389e-06	\\
-8268.75	5.51519542928774e-06	\\
-8267.77121803977	4.92935428453507e-06	\\
-8266.79243607955	4.88995300750996e-06	\\
-8265.81365411932	3.29254783256433e-06	\\
-8264.83487215909	4.8770207509354e-06	\\
-8263.85609019886	6.21949132089841e-06	\\
-8262.87730823864	3.95801028525885e-06	\\
-8261.89852627841	3.90800755115448e-06	\\
-8260.91974431818	2.4602783801082e-06	\\
-8259.94096235795	4.50595227947818e-06	\\
-8258.96218039773	4.3535342147413e-06	\\
-8257.9833984375	5.40524445701223e-06	\\
-8257.00461647727	4.32506128621285e-06	\\
-8256.02583451705	4.59054875870544e-06	\\
-8255.04705255682	4.19587377409342e-06	\\
-8254.06827059659	5.158252296822e-06	\\
-8253.08948863636	4.4602743926845e-06	\\
-8252.11070667614	4.6295571611347e-06	\\
-8251.13192471591	4.92860600499585e-06	\\
-8250.15314275568	3.29535735293789e-06	\\
-8249.17436079545	4.97647802892557e-06	\\
-8248.19557883523	5.38936343918496e-06	\\
-8247.216796875	3.21154614956079e-06	\\
-8246.23801491477	4.83561883413703e-06	\\
-8245.25923295455	4.10443658521602e-06	\\
-8244.28045099432	4.70287531153924e-06	\\
-8243.30166903409	4.91915032718663e-06	\\
-8242.32288707386	5.90763001564869e-06	\\
-8241.34410511364	4.28846450191472e-06	\\
-8240.36532315341	4.38164213460486e-06	\\
-8239.38654119318	2.14140547906102e-06	\\
-8238.40775923295	3.7112764277803e-06	\\
-8237.42897727273	4.56416517766745e-06	\\
-8236.4501953125	6.46028089540198e-06	\\
-8235.47141335227	2.98634926006608e-06	\\
-8234.49263139205	5.21092310788401e-06	\\
-8233.51384943182	5.89273102922816e-06	\\
-8232.53506747159	4.72185872235623e-06	\\
-8231.55628551136	3.34633319030818e-06	\\
-8230.57750355114	4.52021036362256e-06	\\
-8229.59872159091	5.26480656012172e-06	\\
-8228.61993963068	3.93086044303536e-06	\\
-8227.64115767045	5.57860735111391e-06	\\
-8226.66237571023	5.19723672031051e-06	\\
-8225.68359375	3.70672697863155e-06	\\
-8224.70481178977	5.76568478133273e-06	\\
-8223.72602982955	4.00405161102705e-06	\\
-8222.74724786932	3.69351929907987e-06	\\
-8221.76846590909	4.32764519696913e-06	\\
-8220.78968394886	5.14859154588613e-06	\\
-8219.81090198864	3.3327568149657e-06	\\
-8218.83212002841	4.29637060311739e-06	\\
-8217.85333806818	3.55680462968343e-06	\\
-8216.87455610795	3.33902718627006e-06	\\
-8215.89577414773	4.45138910630929e-06	\\
-8214.9169921875	3.46544909457475e-06	\\
-8213.93821022727	4.1845524013746e-06	\\
-8212.95942826705	5.99201036480033e-06	\\
-8211.98064630682	5.45085840661458e-06	\\
-8211.00186434659	2.99504676112736e-06	\\
-8210.02308238636	4.98019878131205e-06	\\
-8209.04430042614	5.04146813892379e-06	\\
-8208.06551846591	6.70005428367048e-06	\\
-8207.08673650568	5.69361923199232e-06	\\
-8206.10795454545	4.59957614470925e-06	\\
-8205.12917258523	5.11358968660484e-06	\\
-8204.150390625	6.10231312895081e-06	\\
-8203.17160866477	4.45419803507784e-06	\\
-8202.19282670455	4.04840255411202e-06	\\
-8201.21404474432	5.01440053389218e-06	\\
-8200.23526278409	5.98977565550527e-06	\\
-8199.25648082386	4.6089105765229e-06	\\
-8198.27769886364	3.19109552184629e-06	\\
-8197.29891690341	4.35521568657518e-06	\\
-8196.32013494318	6.13629617764857e-06	\\
-8195.34135298295	2.73684377101808e-06	\\
-8194.36257102273	4.33291881416986e-06	\\
-8193.3837890625	5.3359294832428e-06	\\
-8192.40500710227	4.94553760632364e-06	\\
-8191.42622514205	4.66408325523609e-06	\\
-8190.44744318182	4.03982887391404e-06	\\
-8189.46866122159	4.08565696884722e-06	\\
-8188.48987926136	3.49868437169636e-06	\\
-8187.51109730114	4.35699100094392e-06	\\
-8186.53231534091	4.06058192620992e-06	\\
-8185.55353338068	4.99625677152953e-06	\\
-8184.57475142045	3.88649662950786e-06	\\
-8183.59596946023	4.32508632049794e-06	\\
-8182.6171875	3.00328751575603e-06	\\
-8181.63840553977	3.78680419205978e-06	\\
-8180.65962357955	4.53823748333751e-06	\\
-8179.68084161932	4.88327441613139e-06	\\
-8178.70205965909	5.12042797479466e-06	\\
-8177.72327769886	4.03730009258854e-06	\\
-8176.74449573864	2.95694969312595e-06	\\
-8175.76571377841	3.269150967487e-06	\\
-8174.78693181818	4.12626364085101e-06	\\
-8173.80814985795	5.45094553671614e-06	\\
-8172.82936789773	4.89945787126506e-06	\\
-8171.8505859375	5.01827096980088e-06	\\
-8170.87180397727	4.51076970493982e-06	\\
-8169.89302201705	3.28323957723685e-06	\\
-8168.91424005682	4.06477924994011e-06	\\
-8167.93545809659	4.59542061885925e-06	\\
-8166.95667613636	4.07913248593109e-06	\\
-8165.97789417614	3.37941557207839e-06	\\
-8164.99911221591	4.20791726152974e-06	\\
-8164.02033025568	3.29675545631273e-06	\\
-8163.04154829545	5.32719960241873e-06	\\
-8162.06276633523	2.69744159948873e-06	\\
-8161.083984375	3.8459792030523e-06	\\
-8160.10520241477	3.80856062856947e-06	\\
-8159.12642045455	4.01300942808633e-06	\\
-8158.14763849432	5.51203421847094e-06	\\
-8157.16885653409	3.73231026778724e-06	\\
-8156.19007457386	3.98881072458805e-06	\\
-8155.21129261364	3.70950470298931e-06	\\
-8154.23251065341	4.70873305484945e-06	\\
-8153.25372869318	1.97643582699815e-06	\\
-8152.27494673295	4.39395228551738e-06	\\
-8151.29616477273	4.01327441650646e-06	\\
-8150.3173828125	2.54171135219467e-06	\\
-8149.33860085227	3.55459071708549e-06	\\
-8148.35981889205	5.26880823836059e-06	\\
-8147.38103693182	6.04786290277335e-06	\\
-8146.40225497159	3.12024642836601e-06	\\
-8145.42347301136	3.94507851407682e-06	\\
-8144.44469105114	4.59407688841778e-06	\\
-8143.46590909091	4.51801453935094e-06	\\
-8142.48712713068	4.57264348213947e-06	\\
-8141.50834517045	4.65097412653512e-06	\\
-8140.52956321023	3.85275528416859e-06	\\
-8139.55078125	5.05986006816006e-06	\\
-8138.57199928977	4.14728564302583e-06	\\
-8137.59321732955	4.78575146817209e-06	\\
-8136.61443536932	4.26138898630806e-06	\\
-8135.63565340909	4.55466958172613e-06	\\
-8134.65687144886	4.61965636846361e-06	\\
-8133.67808948864	2.81432680046688e-06	\\
-8132.69930752841	4.44036403781381e-06	\\
-8131.72052556818	3.52565095390847e-06	\\
-8130.74174360795	5.2059718577381e-06	\\
-8129.76296164773	3.4639693940226e-06	\\
-8128.7841796875	3.3530402281994e-06	\\
-8127.80539772727	4.37621005517839e-06	\\
-8126.82661576705	4.04008089774567e-06	\\
-8125.84783380682	3.10815758624244e-06	\\
-8124.86905184659	3.74426201857356e-06	\\
-8123.89026988636	4.14386142629164e-06	\\
-8122.91148792614	3.73688423536435e-06	\\
-8121.93270596591	3.71214497289621e-06	\\
-8120.95392400568	4.57766183498732e-06	\\
-8119.97514204545	2.94881444230097e-06	\\
-8118.99636008523	4.82550188661479e-06	\\
-8118.017578125	3.94635017118025e-06	\\
-8117.03879616477	4.44421800335101e-06	\\
-8116.06001420455	2.53151967918453e-06	\\
-8115.08123224432	4.22008727048276e-06	\\
-8114.10245028409	4.50869542304615e-06	\\
-8113.12366832386	4.9675432218807e-06	\\
-8112.14488636364	2.56379885768763e-06	\\
-8111.16610440341	4.67572299221883e-06	\\
-8110.18732244318	4.04565055248211e-06	\\
-8109.20854048295	5.38648781439355e-06	\\
-8108.22975852273	4.43012749996012e-06	\\
-8107.2509765625	4.04876050923706e-06	\\
-8106.27219460227	2.91430409575783e-06	\\
-8105.29341264205	2.71320253366768e-06	\\
-8104.31463068182	4.15401309910184e-06	\\
-8103.33584872159	2.9867356783272e-06	\\
-8102.35706676136	4.18548100260882e-06	\\
-8101.37828480114	2.61889989301495e-06	\\
-8100.39950284091	3.28485953670172e-06	\\
-8099.42072088068	4.27990075033163e-06	\\
-8098.44193892045	4.47600796213504e-06	\\
-8097.46315696023	4.57513976720849e-06	\\
-8096.484375	2.92466935399949e-06	\\
-8095.50559303977	4.46119896526228e-06	\\
-8094.52681107955	5.27771552403493e-06	\\
-8093.54802911932	4.93107066555452e-06	\\
-8092.56924715909	3.82842263088139e-06	\\
-8091.59046519886	4.87929773624579e-06	\\
-8090.61168323864	4.62710852818874e-06	\\
-8089.63290127841	4.29895692180268e-06	\\
-8088.65411931818	2.28499529905001e-06	\\
-8087.67533735795	5.22970750738002e-06	\\
-8086.69655539773	5.41117945616053e-06	\\
-8085.7177734375	2.64568510478175e-06	\\
-8084.73899147727	2.55675481308884e-06	\\
-8083.76020951705	3.39665258603124e-06	\\
-8082.78142755682	4.64748916137918e-06	\\
-8081.80264559659	2.91436285600819e-06	\\
-8080.82386363636	3.00003686822757e-06	\\
-8079.84508167614	4.97408179895941e-06	\\
-8078.86629971591	4.31193027327848e-06	\\
-8077.88751775568	3.58235655856222e-06	\\
-8076.90873579545	4.82984135806849e-06	\\
-8075.92995383523	4.38219509857772e-06	\\
-8074.951171875	4.81002445435867e-06	\\
-8073.97238991477	3.77376384692493e-06	\\
-8072.99360795455	3.47072975744409e-06	\\
-8072.01482599432	4.17743552745482e-06	\\
-8071.03604403409	3.65083155313178e-06	\\
-8070.05726207386	3.071418460503e-06	\\
-8069.07848011364	3.50905985151349e-06	\\
-8068.09969815341	3.34163777212918e-06	\\
-8067.12091619318	3.42770585157396e-06	\\
-8066.14213423295	3.39706995738639e-06	\\
-8065.16335227273	4.76322497739292e-06	\\
-8064.1845703125	4.25507222279825e-06	\\
-8063.20578835227	3.81050054350117e-06	\\
-8062.22700639205	3.25866309791357e-06	\\
-8061.24822443182	4.22892074948213e-06	\\
-8060.26944247159	2.83166794789411e-06	\\
-8059.29066051136	3.92782771635521e-06	\\
-8058.31187855114	3.53670162524078e-06	\\
-8057.33309659091	4.03855999030906e-06	\\
-8056.35431463068	4.55849378823948e-06	\\
-8055.37553267045	3.71177131681179e-06	\\
-8054.39675071023	2.63740544850867e-06	\\
-8053.41796875	3.6793893003978e-06	\\
-8052.43918678977	2.90056641690687e-06	\\
-8051.46040482955	3.41257585029993e-06	\\
-8050.48162286932	2.61150719411345e-06	\\
-8049.50284090909	4.6727719556017e-06	\\
-8048.52405894886	3.53239449395818e-06	\\
-8047.54527698864	3.51191618819054e-06	\\
-8046.56649502841	3.44762873509083e-06	\\
-8045.58771306818	3.94803492170395e-06	\\
-8044.60893110795	5.26557914899808e-06	\\
-8043.63014914773	3.62324525551733e-06	\\
-8042.6513671875	3.58985093067775e-06	\\
-8041.67258522727	4.01214603475224e-06	\\
-8040.69380326705	3.70419671155206e-06	\\
-8039.71502130682	5.35510970919901e-06	\\
-8038.73623934659	5.18356720880103e-06	\\
-8037.75745738636	2.81993339026157e-06	\\
-8036.77867542614	3.2969451898938e-06	\\
-8035.79989346591	4.80026790602997e-06	\\
-8034.82111150568	5.44701496117757e-06	\\
-8033.84232954545	4.53495289147857e-06	\\
-8032.86354758523	4.59809768679915e-06	\\
-8031.884765625	3.53761179479627e-06	\\
-8030.90598366477	3.32125475017796e-06	\\
-8029.92720170455	4.4822252692288e-06	\\
-8028.94841974432	4.41292812490894e-06	\\
-8027.96963778409	4.80884208356422e-06	\\
-8026.99085582386	3.29594528763224e-06	\\
-8026.01207386364	3.50196217473219e-06	\\
-8025.03329190341	4.02400776791614e-06	\\
-8024.05450994318	5.15741633155279e-06	\\
-8023.07572798295	2.41129207170674e-06	\\
-8022.09694602273	3.89880488965504e-06	\\
-8021.1181640625	5.02569317775881e-06	\\
-8020.13938210227	4.70172520633138e-06	\\
-8019.16060014205	4.67316395572673e-06	\\
-8018.18181818182	3.16814853760717e-06	\\
-8017.20303622159	4.39326384960259e-06	\\
-8016.22425426136	4.13264651085915e-06	\\
-8015.24547230114	4.24287552883818e-06	\\
-8014.26669034091	3.06594183777083e-06	\\
-8013.28790838068	4.44543444599418e-06	\\
-8012.30912642045	3.49329313027521e-06	\\
-8011.33034446023	3.8244526644694e-06	\\
-8010.3515625	5.37030305355594e-06	\\
-8009.37278053977	3.54246017830848e-06	\\
-8008.39399857955	4.19739848384101e-06	\\
-8007.41521661932	3.1044242436884e-06	\\
-8006.43643465909	3.34249834309951e-06	\\
-8005.45765269886	4.50161520981577e-06	\\
-8004.47887073864	1.74769639623138e-06	\\
-8003.50008877841	3.75937494529901e-06	\\
-8002.52130681818	3.59027036660695e-06	\\
-8001.54252485795	4.63126869272127e-06	\\
-8000.56374289773	4.44251039839488e-06	\\
-7999.5849609375	3.03887993544027e-06	\\
-7998.60617897727	5.71794625996286e-06	\\
-7997.62739701705	3.56780564519416e-06	\\
-7996.64861505682	3.35552665201132e-06	\\
-7995.66983309659	3.20527517584608e-06	\\
-7994.69105113636	4.09350292985641e-06	\\
-7993.71226917614	3.39713501680835e-06	\\
-7992.73348721591	3.62876214525578e-06	\\
-7991.75470525568	3.41307558511231e-06	\\
-7990.77592329545	5.58981962938395e-06	\\
-7989.79714133523	4.53588629763509e-06	\\
-7988.818359375	3.92010950105877e-06	\\
-7987.83957741477	3.08898547855069e-06	\\
-7986.86079545455	4.47812934144121e-06	\\
-7985.88201349432	4.35196021693426e-06	\\
-7984.90323153409	3.09931560236571e-06	\\
-7983.92444957386	3.72726887055853e-06	\\
-7982.94566761364	3.97143616485623e-06	\\
-7981.96688565341	1.27832928492625e-06	\\
-7980.98810369318	3.19380000916739e-06	\\
-7980.00932173295	3.85876432755136e-06	\\
-7979.03053977273	3.89207343305756e-06	\\
-7978.0517578125	3.86293712220544e-06	\\
-7977.07297585227	3.91230335544477e-06	\\
-7976.09419389205	4.70363848363661e-06	\\
-7975.11541193182	2.46871913365275e-06	\\
-7974.13662997159	4.05320184297331e-06	\\
-7973.15784801136	3.88113569616388e-06	\\
-7972.17906605114	2.44735897684603e-06	\\
-7971.20028409091	3.67103100613394e-06	\\
-7970.22150213068	3.23949498666755e-06	\\
-7969.24272017045	3.67250611898623e-06	\\
-7968.26393821023	4.5888045523696e-06	\\
-7967.28515625	3.06001826300116e-06	\\
-7966.30637428977	2.78722333475591e-06	\\
-7965.32759232955	3.92954877267243e-06	\\
-7964.34881036932	4.33154108363576e-06	\\
-7963.37002840909	3.35785389494215e-06	\\
-7962.39124644886	4.3376246187587e-06	\\
-7961.41246448864	2.6538860348127e-06	\\
-7960.43368252841	3.26803614189969e-06	\\
-7959.45490056818	3.65086239113443e-06	\\
-7958.47611860795	4.04091269586761e-06	\\
-7957.49733664773	2.99970901363757e-06	\\
-7956.5185546875	3.37339469444079e-06	\\
-7955.53977272727	3.20390917113015e-06	\\
-7954.56099076705	3.1635335879019e-06	\\
-7953.58220880682	3.94092482755572e-06	\\
-7952.60342684659	4.83030798136125e-06	\\
-7951.62464488636	3.53640550807829e-06	\\
-7950.64586292614	3.41236486741708e-06	\\
-7949.66708096591	2.81433903006974e-06	\\
-7948.68829900568	2.18968994638805e-06	\\
-7947.70951704545	4.71354348507402e-06	\\
-7946.73073508523	4.95699119846174e-06	\\
-7945.751953125	3.76530445306376e-06	\\
-7944.77317116477	3.37564478568882e-06	\\
-7943.79438920455	4.00595962304644e-06	\\
-7942.81560724432	2.18615769928759e-06	\\
-7941.83682528409	2.89948089961253e-06	\\
-7940.85804332386	3.78144869436532e-06	\\
-7939.87926136364	3.26767423527912e-06	\\
-7938.90047940341	3.60502321980776e-06	\\
-7937.92169744318	4.18740120321855e-06	\\
-7936.94291548295	4.89955633551966e-06	\\
-7935.96413352273	3.62695131655287e-06	\\
-7934.9853515625	3.41590844738599e-06	\\
-7934.00656960227	3.29858252088545e-06	\\
-7933.02778764205	2.72870457385e-06	\\
-7932.04900568182	3.72649975360295e-06	\\
-7931.07022372159	3.00633533766275e-06	\\
-7930.09144176136	1.38587022973972e-06	\\
-7929.11265980114	3.01735762917463e-06	\\
-7928.13387784091	2.51040065857917e-06	\\
-7927.15509588068	5.27833696146433e-06	\\
-7926.17631392045	2.76422947892919e-06	\\
-7925.19753196023	4.23736440734904e-06	\\
-7924.21875	4.14547556593873e-06	\\
-7923.23996803977	4.16593617190586e-06	\\
-7922.26118607955	2.41866584735107e-06	\\
-7921.28240411932	3.37682674560866e-06	\\
-7920.30362215909	4.7616812062691e-06	\\
-7919.32484019886	3.29653714683274e-06	\\
-7918.34605823864	4.27180015172469e-06	\\
-7917.36727627841	3.51464189631606e-06	\\
-7916.38849431818	5.45164600761575e-06	\\
-7915.40971235795	3.63871344734474e-06	\\
-7914.43093039773	4.27301659472962e-06	\\
-7913.4521484375	3.83611226844203e-06	\\
-7912.47336647727	3.93389779952201e-06	\\
-7911.49458451705	4.0553884477117e-06	\\
-7910.51580255682	3.72211804153845e-06	\\
-7909.53702059659	1.99727705904525e-06	\\
-7908.55823863636	2.92868815753547e-06	\\
-7907.57945667614	4.06706573292002e-06	\\
-7906.60067471591	3.31719430583596e-06	\\
-7905.62189275568	4.72725938749642e-06	\\
-7904.64311079545	3.37383859773537e-06	\\
-7903.66432883523	4.03586311858842e-06	\\
-7902.685546875	3.93473879321527e-06	\\
-7901.70676491477	3.96810621142491e-06	\\
-7900.72798295455	2.41304726947166e-06	\\
-7899.74920099432	3.83762576344222e-06	\\
-7898.77041903409	4.21483657116399e-06	\\
-7897.79163707386	3.77907234699388e-06	\\
-7896.81285511364	2.8249270826369e-06	\\
-7895.83407315341	3.69811038467497e-06	\\
-7894.85529119318	3.83968164365047e-06	\\
-7893.87650923295	3.39751900267477e-06	\\
-7892.89772727273	2.28198997097511e-06	\\
-7891.9189453125	3.94092294464532e-06	\\
-7890.94016335227	3.14185413419907e-06	\\
-7889.96138139205	4.18924818985261e-06	\\
-7888.98259943182	4.73815509876598e-06	\\
-7888.00381747159	3.54825962657295e-06	\\
-7887.02503551136	5.35248012635357e-06	\\
-7886.04625355114	3.61696895272724e-06	\\
-7885.06747159091	3.9253518144002e-06	\\
-7884.08868963068	4.95149069240185e-06	\\
-7883.10990767045	4.44494776767233e-06	\\
-7882.13112571023	3.566388138502e-06	\\
-7881.15234375	3.62472023224045e-06	\\
-7880.17356178977	2.94201200121667e-06	\\
-7879.19477982955	3.12871200641967e-06	\\
-7878.21599786932	3.33183043643221e-06	\\
-7877.23721590909	2.34861420238292e-06	\\
-7876.25843394886	3.42362734064623e-06	\\
-7875.27965198864	2.95018584560756e-06	\\
-7874.30087002841	3.61501837908496e-06	\\
-7873.32208806818	4.01834261960214e-06	\\
-7872.34330610795	5.36544818040026e-06	\\
-7871.36452414773	4.00318615490693e-06	\\
-7870.3857421875	4.4874793305152e-06	\\
-7869.40696022727	4.42252611606238e-06	\\
-7868.42817826705	9.00894019958579e-07	\\
-7867.44939630682	2.27298779656195e-06	\\
-7866.47061434659	3.90423779981432e-06	\\
-7865.49183238636	4.24321070083151e-06	\\
-7864.51305042614	1.55943218386319e-06	\\
-7863.53426846591	3.23209938552023e-06	\\
-7862.55548650568	5.55745873736866e-06	\\
-7861.57670454545	2.34389859331812e-06	\\
-7860.59792258523	4.22430158423687e-06	\\
-7859.619140625	2.7194689296941e-06	\\
-7858.64035866477	2.50443179980614e-06	\\
-7857.66157670455	3.77106035178296e-06	\\
-7856.68279474432	2.54772355397379e-06	\\
-7855.70401278409	3.91175731457813e-06	\\
-7854.72523082386	3.17820741199374e-06	\\
-7853.74644886364	3.47038284254136e-06	\\
-7852.76766690341	1.88775827353219e-06	\\
-7851.78888494318	4.53112600564621e-06	\\
-7850.81010298295	3.06795122984076e-06	\\
-7849.83132102273	4.03259463282086e-06	\\
-7848.8525390625	3.26920459740048e-06	\\
-7847.87375710227	4.34235168322207e-06	\\
-7846.89497514205	2.48010505603734e-06	\\
-7845.91619318182	3.63358934971427e-06	\\
-7844.93741122159	2.66271006972323e-06	\\
-7843.95862926136	1.98384281503681e-06	\\
-7842.97984730114	5.66948885032192e-06	\\
-7842.00106534091	3.79524121566402e-06	\\
-7841.02228338068	2.52995785573012e-06	\\
-7840.04350142045	3.99909973376876e-06	\\
-7839.06471946023	3.20886041772986e-06	\\
-7838.0859375	2.57161779430506e-06	\\
-7837.10715553977	3.87787776103139e-06	\\
-7836.12837357955	1.76728015249982e-06	\\
-7835.14959161932	2.96773692696266e-06	\\
-7834.17080965909	3.03649126018546e-06	\\
-7833.19202769886	3.55147586555627e-06	\\
-7832.21324573864	2.43059183554301e-06	\\
-7831.23446377841	2.79012366910416e-06	\\
-7830.25568181818	4.40169990800988e-06	\\
-7829.27689985795	4.13753973906491e-06	\\
-7828.29811789773	3.64373679733774e-06	\\
-7827.3193359375	3.15088025771416e-06	\\
-7826.34055397727	4.06057755858525e-06	\\
-7825.36177201705	3.97969257335113e-06	\\
-7824.38299005682	3.25869922050144e-06	\\
-7823.40420809659	4.3848608440929e-06	\\
-7822.42542613636	2.71500590488951e-06	\\
-7821.44664417614	4.40333987367945e-06	\\
-7820.46786221591	3.19316456872522e-06	\\
-7819.48908025568	4.38928145125663e-06	\\
-7818.51029829545	4.46909175986283e-06	\\
-7817.53151633523	4.39084836512184e-06	\\
-7816.552734375	4.25831187341305e-06	\\
-7815.57395241477	4.7487039059019e-06	\\
-7814.59517045455	4.33552349069138e-06	\\
-7813.61638849432	1.94888468668369e-06	\\
-7812.63760653409	1.47401371725073e-06	\\
-7811.65882457386	2.57847746203861e-06	\\
-7810.68004261364	1.65058464929906e-06	\\
-7809.70126065341	4.16553571015625e-06	\\
-7808.72247869318	3.54209152927826e-06	\\
-7807.74369673295	4.12638489328942e-06	\\
-7806.76491477273	3.55896971078205e-06	\\
-7805.7861328125	4.85597855965815e-06	\\
-7804.80735085227	3.47558803962718e-06	\\
-7803.82856889205	5.38849390415163e-06	\\
-7802.84978693182	2.30775808645629e-06	\\
-7801.87100497159	2.44060257318552e-06	\\
-7800.89222301136	4.12353504885894e-06	\\
-7799.91344105114	3.26991083691671e-06	\\
-7798.93465909091	2.76062842082488e-06	\\
-7797.95587713068	3.20044495704438e-06	\\
-7796.97709517045	3.47607992783407e-06	\\
-7795.99831321023	3.83507132676822e-06	\\
-7795.01953125	2.04255888083659e-06	\\
-7794.04074928977	2.84333530266306e-06	\\
-7793.06196732955	3.42450377412855e-06	\\
-7792.08318536932	4.39474984134586e-06	\\
-7791.10440340909	3.97704616407036e-06	\\
-7790.12562144886	4.58641737830382e-06	\\
-7789.14683948864	4.46538145758194e-06	\\
-7788.16805752841	4.88978986635758e-06	\\
-7787.18927556818	3.50514447604355e-06	\\
-7786.21049360795	5.3716472938838e-06	\\
-7785.23171164773	4.28567461262167e-06	\\
-7784.2529296875	3.03079563551214e-06	\\
-7783.27414772727	4.66829523753721e-06	\\
-7782.29536576705	4.9901739462921e-06	\\
-7781.31658380682	2.89751528064451e-06	\\
-7780.33780184659	2.94660909968651e-06	\\
-7779.35901988636	3.97366702333337e-06	\\
-7778.38023792614	4.30926646980156e-06	\\
-7777.40145596591	4.704707175092e-06	\\
-7776.42267400568	2.05207672703529e-06	\\
-7775.44389204545	4.3761211903018e-06	\\
-7774.46511008523	2.55810095062111e-06	\\
-7773.486328125	2.57541235848789e-06	\\
-7772.50754616477	5.66876846482382e-06	\\
-7771.52876420455	4.20816074144361e-06	\\
-7770.54998224432	3.09852371282497e-06	\\
-7769.57120028409	3.86639880665644e-06	\\
-7768.59241832386	2.35565160773926e-06	\\
-7767.61363636364	3.24202242351867e-06	\\
-7766.63485440341	4.69175638892565e-06	\\
-7765.65607244318	2.49708146367973e-06	\\
-7764.67729048295	3.15158582434796e-06	\\
-7763.69850852273	4.64673022156849e-06	\\
-7762.7197265625	2.77853237603825e-06	\\
-7761.74094460227	5.23856152643546e-06	\\
-7760.76216264205	3.03592437789862e-06	\\
-7759.78338068182	3.04868549865611e-06	\\
-7758.80459872159	4.53282848491678e-06	\\
-7757.82581676136	3.17475952910244e-06	\\
-7756.84703480114	4.6887340891579e-06	\\
-7755.86825284091	3.3995764936337e-06	\\
-7754.88947088068	3.33742792487266e-06	\\
-7753.91068892045	4.13026916526733e-06	\\
-7752.93190696023	3.68398819148277e-06	\\
-7751.953125	3.26667644933801e-06	\\
-7750.97434303977	3.97881840481995e-06	\\
-7749.99556107955	1.66053674438838e-06	\\
-7749.01677911932	3.5775152174894e-06	\\
-7748.03799715909	4.59040401872365e-06	\\
-7747.05921519886	3.49712650221152e-06	\\
-7746.08043323864	4.31213948834941e-06	\\
-7745.10165127841	4.48525539055014e-06	\\
-7744.12286931818	4.06163148425977e-06	\\
-7743.14408735795	4.33297072794537e-06	\\
-7742.16530539773	5.5365907795035e-06	\\
-7741.1865234375	3.30603324204554e-06	\\
-7740.20774147727	3.98018656960505e-06	\\
-7739.22895951705	2.63483440818099e-06	\\
-7738.25017755682	4.64006031443928e-06	\\
-7737.27139559659	4.46860274962234e-06	\\
-7736.29261363636	3.96284838567332e-06	\\
-7735.31383167614	3.75273033614892e-06	\\
-7734.33504971591	3.9242667373465e-06	\\
-7733.35626775568	3.18959238393451e-06	\\
-7732.37748579545	3.97678744388561e-06	\\
-7731.39870383523	4.25500761265776e-06	\\
-7730.419921875	3.92744278322672e-06	\\
-7729.44113991477	3.66598413096813e-06	\\
-7728.46235795455	3.07387914434218e-06	\\
-7727.48357599432	2.74080503198868e-06	\\
-7726.50479403409	3.63239139613376e-06	\\
-7725.52601207386	3.88340443762949e-06	\\
-7724.54723011364	2.99318940643374e-06	\\
-7723.56844815341	4.33677194827124e-06	\\
-7722.58966619318	4.91708730741669e-06	\\
-7721.61088423295	4.59339410908266e-06	\\
-7720.63210227273	5.15354065603946e-06	\\
-7719.6533203125	4.34787109090943e-06	\\
-7718.67453835227	4.53711491203318e-06	\\
-7717.69575639205	4.05958205053319e-06	\\
-7716.71697443182	2.15665256120763e-06	\\
-7715.73819247159	3.27290128549053e-06	\\
-7714.75941051136	1.51853460006145e-06	\\
-7713.78062855114	3.79628854242176e-06	\\
-7712.80184659091	2.82649215963583e-06	\\
-7711.82306463068	4.08472328620087e-06	\\
-7710.84428267045	5.34358124404989e-06	\\
-7709.86550071023	4.74758004932762e-06	\\
-7708.88671875	3.48539254480888e-06	\\
-7707.90793678977	5.24274127801731e-06	\\
-7706.92915482955	5.03070577715373e-06	\\
-7705.95037286932	4.83260376313358e-06	\\
-7704.97159090909	3.19330294082684e-06	\\
-7703.99280894886	3.88009469483656e-06	\\
-7703.01402698864	4.05556700210488e-06	\\
-7702.03524502841	4.31964903516843e-06	\\
-7701.05646306818	3.77492911368548e-06	\\
-7700.07768110795	3.05350712531e-06	\\
-7699.09889914773	4.25170077713861e-06	\\
-7698.1201171875	2.81509050362654e-06	\\
-7697.14133522727	5.00128738962619e-06	\\
-7696.16255326705	3.77733567206431e-06	\\
-7695.18377130682	4.09568265218462e-06	\\
-7694.20498934659	5.15388732503472e-06	\\
-7693.22620738636	4.69468622634202e-06	\\
-7692.24742542614	4.36083323385526e-06	\\
-7691.26864346591	3.64608893103212e-06	\\
-7690.28986150568	2.87685507119979e-06	\\
-7689.31107954545	6.01083193093993e-06	\\
-7688.33229758523	4.48984394711966e-06	\\
-7687.353515625	5.0630719303608e-06	\\
-7686.37473366477	5.37650239905651e-06	\\
-7685.39595170455	4.44167094152018e-06	\\
-7684.41716974432	4.00129722290943e-06	\\
-7683.43838778409	5.83074279726098e-06	\\
-7682.45960582386	4.05833843468889e-06	\\
-7681.48082386364	4.31115157039585e-06	\\
-7680.50204190341	3.90266432025923e-06	\\
-7679.52325994318	2.82529679524227e-06	\\
-7678.54447798295	3.57798018224165e-06	\\
-7677.56569602273	5.44103099701593e-06	\\
-7676.5869140625	2.54184632294639e-06	\\
-7675.60813210227	3.50548139226347e-06	\\
-7674.62935014205	3.3618182545351e-06	\\
-7673.65056818182	2.69466771968836e-06	\\
-7672.67178622159	4.88160145527746e-06	\\
-7671.69300426136	3.77123256832892e-06	\\
-7670.71422230114	5.98507731111559e-06	\\
-7669.73544034091	5.5855422738997e-06	\\
-7668.75665838068	4.25540280683599e-06	\\
-7667.77787642045	5.22685951511536e-06	\\
-7666.79909446023	4.49397216454742e-06	\\
-7665.8203125	4.9761983453304e-06	\\
-7664.84153053977	5.98784057568508e-06	\\
-7663.86274857955	2.73445794681647e-06	\\
-7662.88396661932	4.25088829865226e-06	\\
-7661.90518465909	4.1424313259752e-06	\\
-7660.92640269886	3.87267396056824e-06	\\
-7659.94762073864	5.3098737646332e-06	\\
-7658.96883877841	5.32419712540573e-06	\\
-7657.99005681818	5.41589633193471e-06	\\
-7657.01127485795	4.78770826667975e-06	\\
-7656.03249289773	3.20205969274503e-06	\\
-7655.0537109375	4.17196879598944e-06	\\
-7654.07492897727	4.67996803957967e-06	\\
-7653.09614701705	5.07905774025917e-06	\\
-7652.11736505682	3.24144558285382e-06	\\
-7651.13858309659	4.50765112630874e-06	\\
-7650.15980113636	2.99693898284437e-06	\\
-7649.18101917614	4.83615306558564e-06	\\
-7648.20223721591	4.70113238325432e-06	\\
-7647.22345525568	3.80392029385765e-06	\\
-7646.24467329545	6.09715095441022e-06	\\
-7645.26589133523	5.55730456004123e-06	\\
-7644.287109375	5.13418603943752e-06	\\
-7643.30832741477	4.05226632831659e-06	\\
-7642.32954545455	4.03064129981226e-06	\\
-7641.35076349432	6.67481869808479e-06	\\
-7640.37198153409	6.84785210041862e-06	\\
-7639.39319957386	6.07619156424662e-06	\\
-7638.41441761364	5.30765213824625e-06	\\
-7637.43563565341	5.1731523560586e-06	\\
-7636.45685369318	5.67292967274385e-06	\\
-7635.47807173295	6.08657052044428e-06	\\
-7634.49928977273	5.4506809838553e-06	\\
-7633.5205078125	4.95803533747211e-06	\\
-7632.54172585227	5.11556269144475e-06	\\
-7631.56294389205	4.15152324450572e-06	\\
-7630.58416193182	3.69914643173711e-06	\\
-7629.60537997159	3.80472956478782e-06	\\
-7628.62659801136	5.58302221737194e-06	\\
-7627.64781605114	4.76631908100352e-06	\\
-7626.66903409091	6.1694736039856e-06	\\
-7625.69025213068	5.76616718520973e-06	\\
-7624.71147017045	7.79033011588099e-06	\\
-7623.73268821023	7.71198683370993e-06	\\
-7622.75390625	6.91272685317164e-06	\\
-7621.77512428977	5.74252293136901e-06	\\
-7620.79634232955	6.9488645783449e-06	\\
-7619.81756036932	5.05265108674038e-06	\\
-7618.83877840909	7.36900500621398e-06	\\
-7617.85999644886	4.91044104550229e-06	\\
-7616.88121448864	6.46856430947506e-06	\\
-7615.90243252841	5.0996100677006e-06	\\
-7614.92365056818	7.17188444250222e-06	\\
-7613.94486860795	7.05821942569202e-06	\\
-7612.96608664773	7.45609864995948e-06	\\
-7611.9873046875	8.33198436191315e-06	\\
-7611.00852272727	5.7356187350788e-06	\\
-7610.02974076705	5.9675104498806e-06	\\
-7609.05095880682	6.46652506540509e-06	\\
-7608.07217684659	7.38435619633634e-06	\\
-7607.09339488636	7.30422306427926e-06	\\
-7606.11461292614	6.72258226749676e-06	\\
-7605.13583096591	8.23515277431503e-06	\\
-7604.15704900568	6.2274355831115e-06	\\
-7603.17826704545	7.60541317773608e-06	\\
-7602.19948508523	5.45887233496927e-06	\\
-7601.220703125	4.64583061840163e-06	\\
-7600.24192116477	6.79368416621141e-06	\\
-7599.26313920455	5.31546760515215e-06	\\
-7598.28435724432	4.8261201968168e-06	\\
-7597.30557528409	5.06367642407786e-06	\\
-7596.32679332386	7.53840389060061e-06	\\
-7595.34801136364	5.39398775075865e-06	\\
-7594.36922940341	7.8915995674237e-06	\\
-7593.39044744318	5.41153178703232e-06	\\
-7592.41166548295	5.44463469359055e-06	\\
-7591.43288352273	6.12115904288309e-06	\\
-7590.4541015625	6.5175831603318e-06	\\
-7589.47531960227	6.6515691651733e-06	\\
-7588.49653764205	6.00288088791016e-06	\\
-7587.51775568182	5.6490303555562e-06	\\
-7586.53897372159	6.5610036900658e-06	\\
-7585.56019176136	3.95886496712862e-06	\\
-7584.58140980114	5.24530373430364e-06	\\
-7583.60262784091	6.66025754355871e-06	\\
-7582.62384588068	6.45583405483114e-06	\\
-7581.64506392045	6.42364036202871e-06	\\
-7580.66628196023	6.07437317742088e-06	\\
-7579.6875	8.37095730861036e-06	\\
-7578.70871803977	6.55636169526542e-06	\\
-7577.72993607955	9.00765608535061e-06	\\
-7576.75115411932	8.14683870347221e-06	\\
-7575.77237215909	7.14348044405348e-06	\\
-7574.79359019886	5.5930216508371e-06	\\
-7573.81480823864	6.15305436187199e-06	\\
-7572.83602627841	6.55899548809531e-06	\\
-7571.85724431818	6.4859435244625e-06	\\
-7570.87846235795	6.94342812787511e-06	\\
-7569.89968039773	5.15525540125987e-06	\\
-7568.9208984375	8.50892011990256e-06	\\
-7567.94211647727	6.04177033023357e-06	\\
-7566.96333451705	6.33666970423971e-06	\\
-7565.98455255682	6.61559803308883e-06	\\
-7565.00577059659	6.9440652959557e-06	\\
-7564.02698863636	4.99090532982548e-06	\\
-7563.04820667614	7.24520383708401e-06	\\
-7562.06942471591	6.83150026015207e-06	\\
-7561.09064275568	5.11042203347127e-06	\\
-7560.11186079545	6.07444570697254e-06	\\
-7559.13307883523	7.88240948043696e-06	\\
-7558.154296875	6.50506891226993e-06	\\
-7557.17551491477	7.69385780228501e-06	\\
-7556.19673295455	6.06488922825748e-06	\\
-7555.21795099432	8.67882937963078e-06	\\
-7554.23916903409	5.80883585277148e-06	\\
-7553.26038707386	6.41532061102203e-06	\\
-7552.28160511364	6.38210468842217e-06	\\
-7551.30282315341	5.78305764142678e-06	\\
-7550.32404119318	5.84519310039552e-06	\\
-7549.34525923295	6.46360876159158e-06	\\
-7548.36647727273	5.82575962880233e-06	\\
-7547.3876953125	7.23109190298096e-06	\\
-7546.40891335227	6.47479452342416e-06	\\
-7545.43013139205	7.21386861949282e-06	\\
-7544.45134943182	5.08086456350578e-06	\\
-7543.47256747159	7.34635109593241e-06	\\
-7542.49378551136	6.01423593402618e-06	\\
-7541.51500355114	6.81804845066364e-06	\\
-7540.53622159091	6.86761354552136e-06	\\
-7539.55743963068	7.58761704093343e-06	\\
-7538.57865767045	6.68371595521747e-06	\\
-7537.59987571023	7.41271124781164e-06	\\
-7536.62109375	8.63983448889906e-06	\\
-7535.64231178977	7.62301135672183e-06	\\
-7534.66352982955	7.23410396185401e-06	\\
-7533.68474786932	8.75959821563163e-06	\\
-7532.70596590909	6.16065156032711e-06	\\
-7531.72718394886	8.78514732862057e-06	\\
-7530.74840198864	8.37516350096574e-06	\\
-7529.76962002841	7.0319002436185e-06	\\
-7528.79083806818	6.39195272244843e-06	\\
-7527.81205610795	7.30526614339531e-06	\\
-7526.83327414773	7.12353457430497e-06	\\
-7525.8544921875	7.06361096969432e-06	\\
-7524.87571022727	6.43431012781147e-06	\\
-7523.89692826705	8.26929877323565e-06	\\
-7522.91814630682	7.74365811927892e-06	\\
-7521.93936434659	7.09070093795218e-06	\\
-7520.96058238636	6.21516523871878e-06	\\
-7519.98180042614	7.7855715097702e-06	\\
-7519.00301846591	9.36955716265797e-06	\\
-7518.02423650568	7.776812872464e-06	\\
-7517.04545454545	7.66370782668082e-06	\\
-7516.06667258523	7.04764066519115e-06	\\
-7515.087890625	6.89989538774475e-06	\\
-7514.10910866477	5.31184785757744e-06	\\
-7513.13032670455	4.34336914401322e-06	\\
-7512.15154474432	6.12276568950959e-06	\\
-7511.17276278409	6.56881415758782e-06	\\
-7510.19398082386	6.72450956403874e-06	\\
-7509.21519886364	7.17605520090065e-06	\\
-7508.23641690341	6.94657960894001e-06	\\
-7507.25763494318	5.98754142305147e-06	\\
-7506.27885298295	7.53657110186183e-06	\\
-7505.30007102273	7.40802950003423e-06	\\
-7504.3212890625	5.54492051261883e-06	\\
-7503.34250710227	7.29136048747605e-06	\\
-7502.36372514205	9.8783464073172e-06	\\
-7501.38494318182	8.29441208187783e-06	\\
-7500.40616122159	5.48078151146432e-06	\\
-7499.42737926136	5.23134679130635e-06	\\
-7498.44859730114	7.26306928434436e-06	\\
-7497.46981534091	8.78292037802042e-06	\\
-7496.49103338068	7.17111740411291e-06	\\
-7495.51225142045	6.98979261588375e-06	\\
-7494.53346946023	7.99516790922754e-06	\\
-7493.5546875	7.82541092349935e-06	\\
-7492.57590553977	5.25730301398761e-06	\\
-7491.59712357955	6.73117409602761e-06	\\
-7490.61834161932	6.80452592244987e-06	\\
-7489.63955965909	8.34211599271624e-06	\\
-7488.66077769886	6.96275415600699e-06	\\
-7487.68199573864	5.64098570294191e-06	\\
-7486.70321377841	6.1857491522079e-06	\\
-7485.72443181818	6.3277351796544e-06	\\
-7484.74564985795	8.03769981980214e-06	\\
-7483.76686789773	7.09816318086824e-06	\\
-7482.7880859375	6.43065933961594e-06	\\
-7481.80930397727	6.96120796443884e-06	\\
-7480.83052201705	7.19859544304399e-06	\\
-7479.85174005682	7.58949711483941e-06	\\
-7478.87295809659	3.71592858978315e-06	\\
-7477.89417613636	7.06486906252365e-06	\\
-7476.91539417614	7.42273716809142e-06	\\
-7475.93661221591	7.49313136432587e-06	\\
-7474.95783025568	8.0097328767653e-06	\\
-7473.97904829545	7.89889408717336e-06	\\
-7473.00026633523	6.54982869432187e-06	\\
-7472.021484375	7.2060567617614e-06	\\
-7471.04270241477	5.88991664024032e-06	\\
-7470.06392045455	6.88528148953993e-06	\\
-7469.08513849432	8.22368575269257e-06	\\
-7468.10635653409	5.04923625150578e-06	\\
-7467.12757457386	6.44465653735513e-06	\\
-7466.14879261364	9.34953809608573e-06	\\
-7465.17001065341	7.0085849600365e-06	\\
-7464.19122869318	6.97414797832275e-06	\\
-7463.21244673295	7.94020862880358e-06	\\
-7462.23366477273	5.4310953277899e-06	\\
-7461.2548828125	8.56528970008593e-06	\\
-7460.27610085227	8.91906039416276e-06	\\
-7459.29731889205	8.24095115533304e-06	\\
-7458.31853693182	7.47936167792316e-06	\\
-7457.33975497159	5.98331073353354e-06	\\
-7456.36097301136	6.55008423377766e-06	\\
-7455.38219105114	6.94536518874204e-06	\\
-7454.40340909091	6.34031783424859e-06	\\
-7453.42462713068	5.07611855330074e-06	\\
-7452.44584517045	7.86044114656058e-06	\\
-7451.46706321023	7.86920504162509e-06	\\
-7450.48828125	7.6245097043836e-06	\\
-7449.50949928977	7.12704027649723e-06	\\
-7448.53071732955	7.14707839780892e-06	\\
-7447.55193536932	7.26148440035081e-06	\\
-7446.57315340909	5.80004917709387e-06	\\
-7445.59437144886	5.57290441061874e-06	\\
-7444.61558948864	7.95235727351174e-06	\\
-7443.63680752841	7.14578054382375e-06	\\
-7442.65802556818	7.31563827385396e-06	\\
-7441.67924360795	7.31987094840749e-06	\\
-7440.70046164773	7.75212742891379e-06	\\
-7439.7216796875	6.91509097710764e-06	\\
-7438.74289772727	7.63120737914648e-06	\\
-7437.76411576705	5.80283217417246e-06	\\
-7436.78533380682	5.15752916752832e-06	\\
-7435.80655184659	6.81333977605015e-06	\\
-7434.82776988636	7.33172941246108e-06	\\
-7433.84898792614	7.28663534341646e-06	\\
-7432.87020596591	7.67690099177946e-06	\\
-7431.89142400568	5.50218980892996e-06	\\
-7430.91264204545	7.62817906903421e-06	\\
-7429.93386008523	8.87624115084493e-06	\\
-7428.955078125	6.01400267504653e-06	\\
-7427.97629616477	7.13282060601689e-06	\\
-7426.99751420455	7.21864572192426e-06	\\
-7426.01873224432	6.50253078207591e-06	\\
-7425.03995028409	7.59703134936223e-06	\\
-7424.06116832386	6.06335206947496e-06	\\
-7423.08238636364	7.74815816631365e-06	\\
-7422.10360440341	8.19540128758041e-06	\\
-7421.12482244318	7.06225885218698e-06	\\
-7420.14604048295	6.1253085794485e-06	\\
-7419.16725852273	5.77564870148018e-06	\\
-7418.1884765625	7.29561093522968e-06	\\
-7417.20969460227	6.12923205522308e-06	\\
-7416.23091264205	9.7640954389016e-06	\\
-7415.25213068182	5.62439187539682e-06	\\
-7414.27334872159	7.46792079054806e-06	\\
-7413.29456676136	6.50009269197249e-06	\\
-7412.31578480114	8.09847195768214e-06	\\
-7411.33700284091	8.26338940070401e-06	\\
-7410.35822088068	8.13420656528205e-06	\\
-7409.37943892045	7.49806571200374e-06	\\
-7408.40065696023	7.16230640777292e-06	\\
-7407.421875	7.54036950044218e-06	\\
-7406.44309303977	6.67538302376173e-06	\\
-7405.46431107955	8.0755355688736e-06	\\
-7404.48552911932	5.19404248403118e-06	\\
-7403.50674715909	6.44787705213248e-06	\\
-7402.52796519886	7.02353853487544e-06	\\
-7401.54918323864	6.60572055698279e-06	\\
-7400.57040127841	5.6852271792752e-06	\\
-7399.59161931818	5.91346774263805e-06	\\
-7398.61283735795	5.73434540396136e-06	\\
-7397.63405539773	7.08249707477436e-06	\\
-7396.6552734375	6.50259624754593e-06	\\
-7395.67649147727	6.52540490642452e-06	\\
-7394.69770951705	7.47194473945103e-06	\\
-7393.71892755682	5.76732990509443e-06	\\
-7392.74014559659	6.95886049068948e-06	\\
-7391.76136363636	6.89333769606559e-06	\\
-7390.78258167614	5.88404766395498e-06	\\
-7389.80379971591	7.73902325261901e-06	\\
-7388.82501775568	6.0569791015925e-06	\\
-7387.84623579545	7.56204426849086e-06	\\
-7386.86745383523	6.84515601611667e-06	\\
-7385.888671875	6.12874931738588e-06	\\
-7384.90988991477	7.61304159498715e-06	\\
-7383.93110795455	6.34679682091453e-06	\\
-7382.95232599432	7.27433698359921e-06	\\
-7381.97354403409	8.36261260389242e-06	\\
-7380.99476207386	6.99173289586734e-06	\\
-7380.01598011364	6.5427793046986e-06	\\
-7379.03719815341	7.81475247297311e-06	\\
-7378.05841619318	5.94136912027241e-06	\\
-7377.07963423295	6.32593845237242e-06	\\
-7376.10085227273	8.06668304390303e-06	\\
-7375.1220703125	7.15155376893296e-06	\\
-7374.14328835227	6.98686092155367e-06	\\
-7373.16450639205	6.49374502472284e-06	\\
-7372.18572443182	7.08837676473322e-06	\\
-7371.20694247159	5.44990720520275e-06	\\
-7370.22816051136	7.70413806850497e-06	\\
-7369.24937855114	6.52520007337011e-06	\\
-7368.27059659091	5.59598358582301e-06	\\
-7367.29181463068	8.36055301836536e-06	\\
-7366.31303267045	7.6286378450734e-06	\\
-7365.33425071023	7.75755753887573e-06	\\
-7364.35546875	5.9132173193907e-06	\\
-7363.37668678977	6.58612170407381e-06	\\
-7362.39790482955	7.09929345711125e-06	\\
-7361.41912286932	6.62432300877102e-06	\\
-7360.44034090909	5.71151725701256e-06	\\
-7359.46155894886	6.56909513478965e-06	\\
-7358.48277698864	6.11220985793961e-06	\\
-7357.50399502841	5.90499242918471e-06	\\
-7356.52521306818	6.99544913784337e-06	\\
-7355.54643110795	9.11106621818492e-06	\\
-7354.56764914773	8.37190644591138e-06	\\
-7353.5888671875	5.75371583787116e-06	\\
-7352.61008522727	5.59877281359956e-06	\\
-7351.63130326705	6.21107058241052e-06	\\
-7350.65252130682	7.17485033284121e-06	\\
-7349.67373934659	7.41482716539655e-06	\\
-7348.69495738636	7.49752919131796e-06	\\
-7347.71617542614	6.72359441571493e-06	\\
-7346.73739346591	5.47688015064364e-06	\\
-7345.75861150568	7.17360500436683e-06	\\
-7344.77982954545	6.27180728712046e-06	\\
-7343.80104758523	8.59326966755238e-06	\\
-7342.822265625	6.80033072961104e-06	\\
-7341.84348366477	6.29385114467489e-06	\\
-7340.86470170455	6.98389042424355e-06	\\
-7339.88591974432	8.08276315935173e-06	\\
-7338.90713778409	9.68493799396602e-06	\\
-7337.92835582386	6.78757285384708e-06	\\
-7336.94957386364	6.9439778914668e-06	\\
-7335.97079190341	6.40156298872383e-06	\\
-7334.99200994318	6.65943591401843e-06	\\
-7334.01322798295	5.795250898385e-06	\\
-7333.03444602273	8.97680353051118e-06	\\
-7332.0556640625	6.86794538690958e-06	\\
-7331.07688210227	7.74095122925083e-06	\\
-7330.09810014205	8.52640026400574e-06	\\
-7329.11931818182	7.11193318063158e-06	\\
-7328.14053622159	5.83539775362214e-06	\\
-7327.16175426136	7.87237890873382e-06	\\
-7326.18297230114	6.80449365831686e-06	\\
-7325.20419034091	7.55568886626585e-06	\\
-7324.22540838068	7.02733683295656e-06	\\
-7323.24662642045	7.1033244971599e-06	\\
-7322.26784446023	9.20003044805411e-06	\\
-7321.2890625	6.42594374662987e-06	\\
-7320.31028053977	9.7004420507217e-06	\\
-7319.33149857955	6.73080050377506e-06	\\
-7318.35271661932	7.12647961954999e-06	\\
-7317.37393465909	7.1297593572239e-06	\\
-7316.39515269886	5.64600274123838e-06	\\
-7315.41637073864	7.89345473434266e-06	\\
-7314.43758877841	7.5669056200834e-06	\\
-7313.45880681818	8.44680292645823e-06	\\
-7312.48002485795	9.1196080518218e-06	\\
-7311.50124289773	6.89364223590414e-06	\\
-7310.5224609375	7.37090817547145e-06	\\
-7309.54367897727	6.7755110860804e-06	\\
-7308.56489701705	6.84857045488597e-06	\\
-7307.58611505682	7.03974686286206e-06	\\
-7306.60733309659	8.93266701196232e-06	\\
-7305.62855113636	9.04104692112966e-06	\\
-7304.64976917614	7.08229289220203e-06	\\
-7303.67098721591	8.19318005178325e-06	\\
-7302.69220525568	6.49489098962554e-06	\\
-7301.71342329545	6.96416108119708e-06	\\
-7300.73464133523	7.58861290579415e-06	\\
-7299.755859375	9.03179529860268e-06	\\
-7298.77707741477	8.1442895764761e-06	\\
-7297.79829545455	6.68257998410375e-06	\\
-7296.81951349432	6.30597025088627e-06	\\
-7295.84073153409	8.5404546650854e-06	\\
-7294.86194957386	7.67408460298161e-06	\\
-7293.88316761364	5.77832263952599e-06	\\
-7292.90438565341	6.38325222370823e-06	\\
-7291.92560369318	5.1590980039015e-06	\\
-7290.94682173295	7.47010371955684e-06	\\
-7289.96803977273	9.91154679672859e-06	\\
-7288.9892578125	5.8306283041132e-06	\\
-7288.01047585227	7.95807082278338e-06	\\
-7287.03169389205	7.50411370344459e-06	\\
-7286.05291193182	8.32366829018674e-06	\\
-7285.07412997159	5.79328871512702e-06	\\
-7284.09534801136	7.65582232420925e-06	\\
-7283.11656605114	9.35102670115488e-06	\\
-7282.13778409091	7.2217431541526e-06	\\
-7281.15900213068	7.28647543177025e-06	\\
-7280.18022017045	8.69913359207085e-06	\\
-7279.20143821023	9.4712302371885e-06	\\
-7278.22265625	7.92283853064536e-06	\\
-7277.24387428977	7.42986980555274e-06	\\
-7276.26509232955	6.9565522488552e-06	\\
-7275.28631036932	9.4428511525001e-06	\\
-7274.30752840909	8.98902903933722e-06	\\
-7273.32874644886	7.64172742495545e-06	\\
-7272.34996448864	7.3636525272792e-06	\\
-7271.37118252841	7.45181333429815e-06	\\
-7270.39240056818	7.09846144525353e-06	\\
-7269.41361860795	6.44408886417046e-06	\\
-7268.43483664773	7.11178318522132e-06	\\
-7267.4560546875	8.44500139113236e-06	\\
-7266.47727272727	7.29132412412152e-06	\\
-7265.49849076705	7.67273026609362e-06	\\
-7264.51970880682	7.25195907612798e-06	\\
-7263.54092684659	7.77486537280873e-06	\\
-7262.56214488636	8.72495637189335e-06	\\
-7261.58336292614	8.02279938989176e-06	\\
-7260.60458096591	6.26288930075115e-06	\\
-7259.62579900568	6.84758839257821e-06	\\
-7258.64701704545	7.10816346032765e-06	\\
-7257.66823508523	8.25682620390124e-06	\\
-7256.689453125	6.81724997065809e-06	\\
-7255.71067116477	8.69812118740948e-06	\\
-7254.73188920455	9.6820104707271e-06	\\
-7253.75310724432	7.53275351728116e-06	\\
-7252.77432528409	7.69696712328209e-06	\\
-7251.79554332386	7.31767394636423e-06	\\
-7250.81676136364	5.66487254058786e-06	\\
-7249.83797940341	7.43470362748739e-06	\\
-7248.85919744318	4.93874623043235e-06	\\
-7247.88041548295	7.86056109171741e-06	\\
-7246.90163352273	7.35629635106711e-06	\\
-7245.9228515625	7.09157429836578e-06	\\
-7244.94406960227	8.92515110135373e-06	\\
-7243.96528764205	7.57190769252675e-06	\\
-7242.98650568182	8.23777139579437e-06	\\
-7242.00772372159	8.89178775029283e-06	\\
-7241.02894176136	9.2676227821999e-06	\\
-7240.05015980114	6.74042362993081e-06	\\
-7239.07137784091	6.22306907302989e-06	\\
-7238.09259588068	5.15008037516752e-06	\\
-7237.11381392045	7.60472975506526e-06	\\
-7236.13503196023	7.65752498840226e-06	\\
-7235.15625	8.22177818094696e-06	\\
-7234.17746803977	7.033351804729e-06	\\
-7233.19868607955	9.48079982406965e-06	\\
-7232.21990411932	7.61687710290394e-06	\\
-7231.24112215909	7.41147527101019e-06	\\
-7230.26234019886	7.06653579893607e-06	\\
-7229.28355823864	7.16703915449778e-06	\\
-7228.30477627841	6.20953723160516e-06	\\
-7227.32599431818	8.69209985734058e-06	\\
-7226.34721235795	8.90637680127492e-06	\\
-7225.36843039773	8.37195928217809e-06	\\
-7224.3896484375	7.2349139305588e-06	\\
-7223.41086647727	6.87395501718881e-06	\\
-7222.43208451705	7.40913880093945e-06	\\
-7221.45330255682	6.22495415137758e-06	\\
-7220.47452059659	7.56754054126153e-06	\\
-7219.49573863636	7.30750514333571e-06	\\
-7218.51695667614	8.9903049899673e-06	\\
-7217.53817471591	7.11178145807949e-06	\\
-7216.55939275568	7.51532622076678e-06	\\
-7215.58061079545	6.59007988005851e-06	\\
-7214.60182883523	8.53183592144357e-06	\\
-7213.623046875	8.03025107124466e-06	\\
-7212.64426491477	1.01000522595184e-05	\\
-7211.66548295455	7.42216468194724e-06	\\
-7210.68670099432	7.74116705654962e-06	\\
-7209.70791903409	6.18688171235894e-06	\\
-7208.72913707386	7.90620523685174e-06	\\
-7207.75035511364	8.61539734940119e-06	\\
-7206.77157315341	7.91249854689741e-06	\\
-7205.79279119318	7.24075324393625e-06	\\
-7204.81400923295	6.06405219925214e-06	\\
-7203.83522727273	8.21856729011441e-06	\\
-7202.8564453125	8.87178859962746e-06	\\
-7201.87766335227	6.73516960660537e-06	\\
-7200.89888139205	6.37543476274329e-06	\\
-7199.92009943182	8.39886153463822e-06	\\
-7198.94131747159	6.34473526908624e-06	\\
-7197.96253551136	7.47942912373917e-06	\\
-7196.98375355114	7.14864384531712e-06	\\
-7196.00497159091	7.63316793979957e-06	\\
-7195.02618963068	7.08745690410591e-06	\\
-7194.04740767045	7.65913259313612e-06	\\
-7193.06862571023	6.07342555475604e-06	\\
-7192.08984375	8.5756250916233e-06	\\
-7191.11106178977	5.80677128137462e-06	\\
-7190.13227982955	6.10634169223411e-06	\\
-7189.15349786932	4.74846954919541e-06	\\
-7188.17471590909	7.22635251794979e-06	\\
-7187.19593394886	8.89968055793045e-06	\\
-7186.21715198864	7.8721745920631e-06	\\
-7185.23837002841	8.56788938681453e-06	\\
-7184.25958806818	8.74049350198356e-06	\\
-7183.28080610795	8.64537816982061e-06	\\
-7182.30202414773	7.18374785826592e-06	\\
-7181.3232421875	7.89160263472128e-06	\\
-7180.34446022727	8.2947711218696e-06	\\
-7179.36567826705	8.04330205674493e-06	\\
-7178.38689630682	5.67421526169271e-06	\\
-7177.40811434659	8.06839296011227e-06	\\
-7176.42933238636	7.35960066877867e-06	\\
-7175.45055042614	6.45477074006246e-06	\\
-7174.47176846591	6.190879848825e-06	\\
-7173.49298650568	7.93466103467323e-06	\\
-7172.51420454545	6.21172359577567e-06	\\
-7171.53542258523	6.57258590548881e-06	\\
-7170.556640625	9.64352416676435e-06	\\
-7169.57785866477	6.09735125703818e-06	\\
-7168.59907670455	8.47535756300803e-06	\\
-7167.62029474432	7.32656469873057e-06	\\
-7166.64151278409	7.63607071314866e-06	\\
-7165.66273082386	7.48985471561147e-06	\\
-7164.68394886364	6.77392553126582e-06	\\
-7163.70516690341	6.49536359087921e-06	\\
-7162.72638494318	7.22024483218773e-06	\\
-7161.74760298295	8.12522746791528e-06	\\
-7160.76882102273	6.19649540355815e-06	\\
-7159.7900390625	8.05295850586202e-06	\\
-7158.81125710227	6.32990049114985e-06	\\
-7157.83247514205	7.51062823682084e-06	\\
-7156.85369318182	7.90854877340173e-06	\\
-7155.87491122159	7.75048685508946e-06	\\
-7154.89612926136	5.87987391127324e-06	\\
-7153.91734730114	9.10100552272711e-06	\\
-7152.93856534091	9.67009030487546e-06	\\
-7151.95978338068	8.36803740379342e-06	\\
-7150.98100142045	6.33180808047232e-06	\\
-7150.00221946023	6.56241550955294e-06	\\
-7149.0234375	7.45880086292809e-06	\\
-7148.04465553977	8.11849879113746e-06	\\
-7147.06587357955	6.72837885550174e-06	\\
-7146.08709161932	5.86564712572309e-06	\\
-7145.10830965909	6.36263289604195e-06	\\
-7144.12952769886	5.29502879022268e-06	\\
-7143.15074573864	6.55518836088163e-06	\\
-7142.17196377841	7.01727232825603e-06	\\
-7141.19318181818	7.26077280029249e-06	\\
-7140.21439985795	6.75271538040343e-06	\\
-7139.23561789773	7.62331818537281e-06	\\
-7138.2568359375	7.69673787473052e-06	\\
-7137.27805397727	8.39607848231639e-06	\\
-7136.29927201705	7.17494080498194e-06	\\
-7135.32049005682	8.22337848866845e-06	\\
-7134.34170809659	7.0563995019826e-06	\\
-7133.36292613636	6.82899385348623e-06	\\
-7132.38414417614	6.41056212521217e-06	\\
-7131.40536221591	7.71190995978677e-06	\\
-7130.42658025568	8.55926609874135e-06	\\
-7129.44779829545	8.43486020743645e-06	\\
-7128.46901633523	7.94846534690065e-06	\\
-7127.490234375	7.82296212700184e-06	\\
-7126.51145241477	8.24538343534834e-06	\\
-7125.53267045455	7.04406111629905e-06	\\
-7124.55388849432	9.83293249660877e-06	\\
-7123.57510653409	8.59802229411554e-06	\\
-7122.59632457386	7.11753049773195e-06	\\
-7121.61754261364	7.2976200577416e-06	\\
-7120.63876065341	8.55700786420301e-06	\\
-7119.65997869318	7.50645387781097e-06	\\
-7118.68119673295	7.39163225169661e-06	\\
-7117.70241477273	8.83339092084226e-06	\\
-7116.7236328125	7.61914334974998e-06	\\
-7115.74485085227	6.89448867793391e-06	\\
-7114.76606889205	6.31767583086869e-06	\\
-7113.78728693182	8.00295941952578e-06	\\
-7112.80850497159	7.45901460621416e-06	\\
-7111.82972301136	8.87170505040418e-06	\\
-7110.85094105114	8.96515683554984e-06	\\
-7109.87215909091	6.74904365828668e-06	\\
};
\addplot [color=blue,solid,forget plot]
  table[row sep=crcr]{
-7109.87215909091	6.74904365828668e-06	\\
-7108.89337713068	7.01798188225225e-06	\\
-7107.91459517045	6.18784265429866e-06	\\
-7106.93581321023	6.36603054493093e-06	\\
-7105.95703125	8.60331574247562e-06	\\
-7104.97824928977	6.8778189146414e-06	\\
-7103.99946732955	7.22763062162162e-06	\\
-7103.02068536932	8.05026968050711e-06	\\
-7102.04190340909	6.35794857878241e-06	\\
-7101.06312144886	7.7593460994324e-06	\\
-7100.08433948864	6.81194114296618e-06	\\
-7099.10555752841	6.44996163395307e-06	\\
-7098.12677556818	6.14829869970677e-06	\\
-7097.14799360795	8.0233439362491e-06	\\
-7096.16921164773	8.18788892010978e-06	\\
-7095.1904296875	5.71361407030767e-06	\\
-7094.21164772727	9.24605604418464e-06	\\
-7093.23286576705	7.67880221611285e-06	\\
-7092.25408380682	7.26621739810602e-06	\\
-7091.27530184659	7.48714787871677e-06	\\
-7090.29651988636	5.43662557228974e-06	\\
-7089.31773792614	7.74608390033138e-06	\\
-7088.33895596591	7.60654874615164e-06	\\
-7087.36017400568	7.84481680670191e-06	\\
-7086.38139204545	7.02325016741782e-06	\\
-7085.40261008523	6.0381687668839e-06	\\
-7084.423828125	8.58172338354708e-06	\\
-7083.44504616477	7.34413870059563e-06	\\
-7082.46626420455	5.75027259681208e-06	\\
-7081.48748224432	7.6820729209556e-06	\\
-7080.50870028409	6.14273472337957e-06	\\
-7079.52991832386	7.15793310451504e-06	\\
-7078.55113636364	7.64625002286116e-06	\\
-7077.57235440341	7.60427328330904e-06	\\
-7076.59357244318	8.32279177368947e-06	\\
-7075.61479048295	7.19640220700735e-06	\\
-7074.63600852273	6.59291178858422e-06	\\
-7073.6572265625	6.09799558802595e-06	\\
-7072.67844460227	7.40464278935221e-06	\\
-7071.69966264205	7.20631154071716e-06	\\
-7070.72088068182	5.96068517823382e-06	\\
-7069.74209872159	6.33472628198794e-06	\\
-7068.76331676136	6.28991222761221e-06	\\
-7067.78453480114	8.10034391424023e-06	\\
-7066.80575284091	8.49406889325202e-06	\\
-7065.82697088068	7.81921406654561e-06	\\
-7064.84818892045	7.05085328696641e-06	\\
-7063.86940696023	7.17101471776314e-06	\\
-7062.890625	9.39609849855319e-06	\\
-7061.91184303977	8.194811718247e-06	\\
-7060.93306107955	6.08809368363363e-06	\\
-7059.95427911932	6.62007008067286e-06	\\
-7058.97549715909	8.48947484421978e-06	\\
-7057.99671519886	6.50868919350289e-06	\\
-7057.01793323864	6.3397903568603e-06	\\
-7056.03915127841	9.6963491379321e-06	\\
-7055.06036931818	8.14986664002446e-06	\\
-7054.08158735795	8.82626749012948e-06	\\
-7053.10280539773	7.68953160628301e-06	\\
-7052.1240234375	7.19203479357354e-06	\\
-7051.14524147727	6.4306516279798e-06	\\
-7050.16645951705	9.63793717854709e-06	\\
-7049.18767755682	8.44026016725964e-06	\\
-7048.20889559659	8.37172940509821e-06	\\
-7047.23011363636	7.82446875851799e-06	\\
-7046.25133167614	9.02157223651989e-06	\\
-7045.27254971591	9.50087333413767e-06	\\
-7044.29376775568	6.16989112251398e-06	\\
-7043.31498579545	7.97832069114485e-06	\\
-7042.33620383523	6.81766720292981e-06	\\
-7041.357421875	6.00941322265071e-06	\\
-7040.37863991477	7.04040470728757e-06	\\
-7039.39985795455	6.5414371994137e-06	\\
-7038.42107599432	8.51368054342633e-06	\\
-7037.44229403409	7.19035577878124e-06	\\
-7036.46351207386	5.72170917820912e-06	\\
-7035.48473011364	6.08244498726991e-06	\\
-7034.50594815341	7.63879514666616e-06	\\
-7033.52716619318	6.97694805701585e-06	\\
-7032.54838423295	7.51295850034926e-06	\\
-7031.56960227273	8.26750953404539e-06	\\
-7030.5908203125	7.08569820323663e-06	\\
-7029.61203835227	7.14848958720797e-06	\\
-7028.63325639205	6.57826624437797e-06	\\
-7027.65447443182	8.80056983473404e-06	\\
-7026.67569247159	7.17231608515409e-06	\\
-7025.69691051136	8.34711457304737e-06	\\
-7024.71812855114	7.54795260979798e-06	\\
-7023.73934659091	6.34474867049268e-06	\\
-7022.76056463068	6.88786878433944e-06	\\
-7021.78178267045	6.42095404251673e-06	\\
-7020.80300071023	6.63929767161254e-06	\\
-7019.82421875	7.18831410357202e-06	\\
-7018.84543678977	5.65474131380739e-06	\\
-7017.86665482955	7.97015811829325e-06	\\
-7016.88787286932	6.97428069322005e-06	\\
-7015.90909090909	7.76300719149643e-06	\\
-7014.93030894886	7.10774079856555e-06	\\
-7013.95152698864	8.01637604024937e-06	\\
-7012.97274502841	7.49395796366304e-06	\\
-7011.99396306818	6.31981215526694e-06	\\
-7011.01518110795	8.22758837968694e-06	\\
-7010.03639914773	6.44912145501791e-06	\\
-7009.0576171875	7.56977363421178e-06	\\
-7008.07883522727	8.32579820900407e-06	\\
-7007.10005326705	7.06216201938104e-06	\\
-7006.12127130682	6.88670606235209e-06	\\
-7005.14248934659	7.2483925750742e-06	\\
-7004.16370738636	7.99133800194756e-06	\\
-7003.18492542614	6.26388428265914e-06	\\
-7002.20614346591	7.68052734757269e-06	\\
-7001.22736150568	8.44156895375992e-06	\\
-7000.24857954545	6.02978011259439e-06	\\
-6999.26979758523	7.1751089506814e-06	\\
-6998.291015625	6.33224357818361e-06	\\
-6997.31223366477	8.00017641561377e-06	\\
-6996.33345170455	8.25694771021647e-06	\\
-6995.35466974432	6.00070126224224e-06	\\
-6994.37588778409	6.04849683373395e-06	\\
-6993.39710582386	7.7566320937903e-06	\\
-6992.41832386364	5.67321039758199e-06	\\
-6991.43954190341	7.31065204143368e-06	\\
-6990.46075994318	7.69618047315033e-06	\\
-6989.48197798295	7.76193656143744e-06	\\
-6988.50319602273	6.61618535393836e-06	\\
-6987.5244140625	9.23759903888994e-06	\\
-6986.54563210227	6.62195324100094e-06	\\
-6985.56685014205	7.47983014874244e-06	\\
-6984.58806818182	6.9474852251888e-06	\\
-6983.60928622159	9.07180553513619e-06	\\
-6982.63050426136	8.36205725334228e-06	\\
-6981.65172230114	9.66590316837131e-06	\\
-6980.67294034091	7.27849839535736e-06	\\
-6979.69415838068	7.43376903813898e-06	\\
-6978.71537642045	8.07169887827452e-06	\\
-6977.73659446023	9.73129811288489e-06	\\
-6976.7578125	7.65584481207799e-06	\\
-6975.77903053977	7.37427594013429e-06	\\
-6974.80024857955	7.85028280955507e-06	\\
-6973.82146661932	6.10307889772207e-06	\\
-6972.84268465909	7.61290419568891e-06	\\
-6971.86390269886	6.81693722130165e-06	\\
-6970.88512073864	7.32664185660932e-06	\\
-6969.90633877841	6.84890769142607e-06	\\
-6968.92755681818	5.06234118134543e-06	\\
-6967.94877485795	8.53669865915874e-06	\\
-6966.96999289773	7.76106205941881e-06	\\
-6965.9912109375	6.89294496954417e-06	\\
-6965.01242897727	7.73409370388112e-06	\\
-6964.03364701705	9.12971745664778e-06	\\
-6963.05486505682	9.18888765655051e-06	\\
-6962.07608309659	7.8025191396427e-06	\\
-6961.09730113636	6.21076420358392e-06	\\
-6960.11851917614	8.49823311642144e-06	\\
-6959.13973721591	7.64717899811418e-06	\\
-6958.16095525568	8.0273446546867e-06	\\
-6957.18217329545	7.91896954526421e-06	\\
-6956.20339133523	6.30653935246946e-06	\\
-6955.224609375	8.9466101838131e-06	\\
-6954.24582741477	7.7751319840246e-06	\\
-6953.26704545455	7.24469630639574e-06	\\
-6952.28826349432	8.4685796414059e-06	\\
-6951.30948153409	7.66973923804105e-06	\\
-6950.33069957386	8.63171142741645e-06	\\
-6949.35191761364	7.76176988295512e-06	\\
-6948.37313565341	7.93372788385336e-06	\\
-6947.39435369318	7.16589481424747e-06	\\
-6946.41557173295	8.78807011527112e-06	\\
-6945.43678977273	7.99713108028011e-06	\\
-6944.4580078125	6.13553105589811e-06	\\
-6943.47922585227	7.39859321719608e-06	\\
-6942.50044389205	5.60406993002682e-06	\\
-6941.52166193182	6.03778120984488e-06	\\
-6940.54287997159	1.04984731875081e-05	\\
-6939.56409801136	9.03352394046385e-06	\\
-6938.58531605114	7.41910268711874e-06	\\
-6937.60653409091	7.22310094243999e-06	\\
-6936.62775213068	8.62508669667896e-06	\\
-6935.64897017045	7.50511983145775e-06	\\
-6934.67018821023	6.97510896641872e-06	\\
-6933.69140625	7.52837088840779e-06	\\
-6932.71262428977	7.10837381022336e-06	\\
-6931.73384232955	5.70846563500666e-06	\\
-6930.75506036932	8.30173240269811e-06	\\
-6929.77627840909	9.77957160016899e-06	\\
-6928.79749644886	7.06417647686342e-06	\\
-6927.81871448864	6.83976779748324e-06	\\
-6926.83993252841	6.29088657245195e-06	\\
-6925.86115056818	7.06542248822119e-06	\\
-6924.88236860796	8.51119593773321e-06	\\
-6923.90358664773	6.13782858001602e-06	\\
-6922.9248046875	8.65582585680134e-06	\\
-6921.94602272727	7.821179319207e-06	\\
-6920.96724076705	8.03908144732432e-06	\\
-6919.98845880682	5.60922736364991e-06	\\
-6919.00967684659	6.17257222450178e-06	\\
-6918.03089488636	7.7162451847681e-06	\\
-6917.05211292614	8.16566985631184e-06	\\
-6916.07333096591	7.77002952489715e-06	\\
-6915.09454900568	8.15001460020523e-06	\\
-6914.11576704546	8.62613494410436e-06	\\
-6913.13698508523	7.52916644661108e-06	\\
-6912.158203125	8.40438889670031e-06	\\
-6911.17942116477	6.70554480401016e-06	\\
-6910.20063920455	7.31279931728524e-06	\\
-6909.22185724432	7.90708370099686e-06	\\
-6908.24307528409	6.45318486092633e-06	\\
-6907.26429332386	8.43425351677292e-06	\\
-6906.28551136364	6.47852106222336e-06	\\
-6905.30672940341	6.81628411593292e-06	\\
-6904.32794744318	5.54526537493576e-06	\\
-6903.34916548296	6.5251071387942e-06	\\
-6902.37038352273	6.53290037283445e-06	\\
-6901.3916015625	7.34439448680706e-06	\\
-6900.41281960227	6.76615754161179e-06	\\
-6899.43403764205	4.16091648161686e-06	\\
-6898.45525568182	6.8596943345537e-06	\\
-6897.47647372159	8.28782613681746e-06	\\
-6896.49769176136	6.52433486770997e-06	\\
-6895.51890980114	6.35996821127284e-06	\\
-6894.54012784091	7.44694543667209e-06	\\
-6893.56134588068	6.74505698089759e-06	\\
-6892.58256392046	6.39830919597452e-06	\\
-6891.60378196023	5.88365262489208e-06	\\
-6890.625	5.30430374918714e-06	\\
-6889.64621803977	7.53856431930615e-06	\\
-6888.66743607955	5.95482020938743e-06	\\
-6887.68865411932	8.20401175615727e-06	\\
-6886.70987215909	7.61865858812679e-06	\\
-6885.73109019886	7.44547177700889e-06	\\
-6884.75230823864	8.04125702193286e-06	\\
-6883.77352627841	6.42286432627453e-06	\\
-6882.79474431818	5.7662793406581e-06	\\
-6881.81596235796	6.72203228895045e-06	\\
-6880.83718039773	6.22245618678354e-06	\\
-6879.8583984375	9.59793177986405e-06	\\
-6878.87961647727	6.32946621540681e-06	\\
-6877.90083451705	6.58082581182403e-06	\\
-6876.92205255682	5.87888607465419e-06	\\
-6875.94327059659	7.27755756803895e-06	\\
-6874.96448863636	7.66048680218732e-06	\\
-6873.98570667614	7.7851801529936e-06	\\
-6873.00692471591	8.3121676198323e-06	\\
-6872.02814275568	8.51803823156275e-06	\\
-6871.04936079546	7.7770282013016e-06	\\
-6870.07057883523	7.49977544732995e-06	\\
-6869.091796875	7.24514952084331e-06	\\
-6868.11301491477	7.0319611103747e-06	\\
-6867.13423295455	5.55559177016742e-06	\\
-6866.15545099432	7.02344518188825e-06	\\
-6865.17666903409	6.83015485746695e-06	\\
-6864.19788707386	7.03023640242426e-06	\\
-6863.21910511364	7.46045500552109e-06	\\
-6862.24032315341	8.08965907703404e-06	\\
-6861.26154119318	9.21470775375984e-06	\\
-6860.28275923296	6.62493701932556e-06	\\
-6859.30397727273	7.97822454975081e-06	\\
-6858.3251953125	9.41048283972432e-06	\\
-6857.34641335227	9.77933297274445e-06	\\
-6856.36763139205	7.61898562854361e-06	\\
-6855.38884943182	5.71154624174535e-06	\\
-6854.41006747159	6.41207411717823e-06	\\
-6853.43128551136	8.52062813460133e-06	\\
-6852.45250355114	8.40790091639147e-06	\\
-6851.47372159091	7.16420745553849e-06	\\
-6850.49493963068	8.03155212611065e-06	\\
-6849.51615767046	8.27182796917994e-06	\\
-6848.53737571023	7.73003662401777e-06	\\
-6847.55859375	8.5107140219819e-06	\\
-6846.57981178977	8.74413538624448e-06	\\
-6845.60102982955	8.21642941630348e-06	\\
-6844.62224786932	7.00589890889143e-06	\\
-6843.64346590909	6.49686546720685e-06	\\
-6842.66468394886	7.92868280509009e-06	\\
-6841.68590198864	8.82386278231353e-06	\\
-6840.70712002841	8.27364608855933e-06	\\
-6839.72833806818	7.13073229860623e-06	\\
-6838.74955610796	9.46429243213288e-06	\\
-6837.77077414773	6.54728051300121e-06	\\
-6836.7919921875	7.2489831472051e-06	\\
-6835.81321022727	8.68843997664126e-06	\\
-6834.83442826705	9.16412524850033e-06	\\
-6833.85564630682	9.59290230276642e-06	\\
-6832.87686434659	7.72843583504009e-06	\\
-6831.89808238636	8.37838188065292e-06	\\
-6830.91930042614	8.15065271848219e-06	\\
-6829.94051846591	8.25112082196884e-06	\\
-6828.96173650568	7.98139380312738e-06	\\
-6827.98295454546	1.139977376691e-05	\\
-6827.00417258523	7.89592783391418e-06	\\
-6826.025390625	8.33431434096703e-06	\\
-6825.04660866477	7.63673730057234e-06	\\
-6824.06782670455	6.53155173308957e-06	\\
-6823.08904474432	7.61873401293869e-06	\\
-6822.11026278409	7.23394064332118e-06	\\
-6821.13148082386	8.36378054464616e-06	\\
-6820.15269886364	8.54820371171399e-06	\\
-6819.17391690341	9.21756918090596e-06	\\
-6818.19513494318	7.56504181196659e-06	\\
-6817.21635298296	7.98621837334569e-06	\\
-6816.23757102273	7.40424984577076e-06	\\
-6815.2587890625	6.83719069089451e-06	\\
-6814.28000710227	8.257032016532e-06	\\
-6813.30122514205	5.53759477217105e-06	\\
-6812.32244318182	1.13313552837764e-05	\\
-6811.34366122159	7.75623358569051e-06	\\
-6810.36487926136	8.12973218164748e-06	\\
-6809.38609730114	7.74565876989236e-06	\\
-6808.40731534091	8.55431391173055e-06	\\
-6807.42853338068	9.48053331178769e-06	\\
-6806.44975142046	8.55255148142962e-06	\\
-6805.47096946023	8.67614935499807e-06	\\
-6804.4921875	7.97993840959528e-06	\\
-6803.51340553977	8.94018265151139e-06	\\
-6802.53462357955	8.4733445672691e-06	\\
-6801.55584161932	8.2012626234329e-06	\\
-6800.57705965909	7.16103128485307e-06	\\
-6799.59827769886	6.90892068583812e-06	\\
-6798.61949573864	9.26766059407189e-06	\\
-6797.64071377841	9.4930213600646e-06	\\
-6796.66193181818	8.14040294210955e-06	\\
-6795.68314985796	8.01008294912373e-06	\\
-6794.70436789773	9.46815365964995e-06	\\
-6793.7255859375	7.79376959505928e-06	\\
-6792.74680397727	9.11157167859733e-06	\\
-6791.76802201705	9.44395516150094e-06	\\
-6790.78924005682	6.2709257186579e-06	\\
-6789.81045809659	8.75496524883175e-06	\\
-6788.83167613636	1.03152119649128e-05	\\
-6787.85289417614	8.55487348624896e-06	\\
-6786.87411221591	8.37712367855412e-06	\\
-6785.89533025568	6.97868211902366e-06	\\
-6784.91654829546	9.78295159960482e-06	\\
-6783.93776633523	8.80302846223532e-06	\\
-6782.958984375	6.68825702621699e-06	\\
-6781.98020241477	8.87511859053381e-06	\\
-6781.00142045455	8.23047033956252e-06	\\
-6780.02263849432	9.00912788103188e-06	\\
-6779.04385653409	1.01016728761476e-05	\\
-6778.06507457386	8.73247318683444e-06	\\
-6777.08629261364	8.8366680914724e-06	\\
-6776.10751065341	8.01691380189595e-06	\\
-6775.12872869318	8.29598966838552e-06	\\
-6774.14994673296	7.97558379028607e-06	\\
-6773.17116477273	7.89422992773412e-06	\\
-6772.1923828125	8.41405139629978e-06	\\
-6771.21360085227	8.78108609253138e-06	\\
-6770.23481889205	6.99582648074864e-06	\\
-6769.25603693182	8.82763810548976e-06	\\
-6768.27725497159	8.89748738547843e-06	\\
-6767.29847301136	1.02183910058552e-05	\\
-6766.31969105114	9.08105047594054e-06	\\
-6765.34090909091	9.93547755755087e-06	\\
-6764.36212713068	9.29763289987138e-06	\\
-6763.38334517046	8.67787575141346e-06	\\
-6762.40456321023	7.71106444254252e-06	\\
-6761.42578125	9.6590671751116e-06	\\
-6760.44699928977	8.0489293150954e-06	\\
-6759.46821732955	8.88656427974344e-06	\\
-6758.48943536932	8.47436560048372e-06	\\
-6757.51065340909	9.41467103202498e-06	\\
-6756.53187144886	8.29703280638344e-06	\\
-6755.55308948864	8.18490101537639e-06	\\
-6754.57430752841	9.26031704656207e-06	\\
-6753.59552556818	8.42317733047447e-06	\\
-6752.61674360796	9.18750976604998e-06	\\
-6751.63796164773	7.62915238233601e-06	\\
-6750.6591796875	7.65950813514811e-06	\\
-6749.68039772727	6.87768519564833e-06	\\
-6748.70161576705	8.57309546027787e-06	\\
-6747.72283380682	7.86436707401889e-06	\\
-6746.74405184659	1.00576068554721e-05	\\
-6745.76526988636	9.03680126875122e-06	\\
-6744.78648792614	9.86657339680656e-06	\\
-6743.80770596591	1.06238466810489e-05	\\
-6742.82892400568	9.53256137231177e-06	\\
-6741.85014204546	1.02500067137294e-05	\\
-6740.87136008523	7.73934785464288e-06	\\
-6739.892578125	7.58460890703911e-06	\\
-6738.91379616477	9.31288324620154e-06	\\
-6737.93501420455	8.66054273419572e-06	\\
-6736.95623224432	8.13399361852449e-06	\\
-6735.97745028409	1.12484031208391e-05	\\
-6734.99866832386	9.19378504151034e-06	\\
-6734.01988636364	9.50621818795659e-06	\\
-6733.04110440341	8.91704156237316e-06	\\
-6732.06232244318	9.18535241685117e-06	\\
-6731.08354048296	9.13443321865745e-06	\\
-6730.10475852273	7.69842280518846e-06	\\
-6729.1259765625	7.17542157815719e-06	\\
-6728.14719460227	7.54499121152242e-06	\\
-6727.16841264205	9.40342978836344e-06	\\
-6726.18963068182	1.04567505716889e-05	\\
-6725.21084872159	9.6417326680159e-06	\\
-6724.23206676136	9.05483872578321e-06	\\
-6723.25328480114	8.5482542052135e-06	\\
-6722.27450284091	1.08057861579675e-05	\\
-6721.29572088068	9.35762130780926e-06	\\
-6720.31693892046	8.7009045352236e-06	\\
-6719.33815696023	8.50957696880592e-06	\\
-6718.359375	1.00617319266304e-05	\\
-6717.38059303977	8.29945171932981e-06	\\
-6716.40181107955	9.39694587538718e-06	\\
-6715.42302911932	7.23796230695458e-06	\\
-6714.44424715909	9.13117657749985e-06	\\
-6713.46546519886	6.50653036636333e-06	\\
-6712.48668323864	7.56136144483642e-06	\\
-6711.50790127841	8.46743482965245e-06	\\
-6710.52911931818	8.13958747268908e-06	\\
-6709.55033735796	7.76108624486727e-06	\\
-6708.57155539773	8.89968500351665e-06	\\
-6707.5927734375	9.06063005085155e-06	\\
-6706.61399147727	9.09152926412029e-06	\\
-6705.63520951705	9.21839600768347e-06	\\
-6704.65642755682	9.29213411559192e-06	\\
-6703.67764559659	9.05353007018672e-06	\\
-6702.69886363636	9.5100819156547e-06	\\
-6701.72008167614	1.01642988624734e-05	\\
-6700.74129971591	7.70571971459989e-06	\\
-6699.76251775568	9.97369715756393e-06	\\
-6698.78373579546	8.625765164534e-06	\\
-6697.80495383523	9.45209775696591e-06	\\
-6696.826171875	1.07537321797601e-05	\\
-6695.84738991477	8.45258653050449e-06	\\
-6694.86860795455	8.96657795585254e-06	\\
-6693.88982599432	9.84713598775908e-06	\\
-6692.91104403409	1.00030613808072e-05	\\
-6691.93226207386	9.24786299509809e-06	\\
-6690.95348011364	8.6642092885751e-06	\\
-6689.97469815341	9.61386196258446e-06	\\
-6688.99591619318	9.83853821375799e-06	\\
-6688.01713423296	9.94371776018916e-06	\\
-6687.03835227273	8.74743520518248e-06	\\
-6686.0595703125	7.15866206899075e-06	\\
-6685.08078835227	8.20259786646787e-06	\\
-6684.10200639205	9.38021758792692e-06	\\
-6683.12322443182	8.73944900374353e-06	\\
-6682.14444247159	9.5372180173227e-06	\\
-6681.16566051136	8.88705451606145e-06	\\
-6680.18687855114	8.65715636515426e-06	\\
-6679.20809659091	1.03953841852965e-05	\\
-6678.22931463068	9.81352586478657e-06	\\
-6677.25053267046	8.99927179975542e-06	\\
-6676.27175071023	1.01348988518755e-05	\\
-6675.29296875	8.33235065675407e-06	\\
-6674.31418678977	7.5399276479446e-06	\\
-6673.33540482955	7.63999271808142e-06	\\
-6672.35662286932	8.63325774640022e-06	\\
-6671.37784090909	8.33879239564638e-06	\\
-6670.39905894886	8.3284056311269e-06	\\
-6669.42027698864	9.56542948509985e-06	\\
-6668.44149502841	9.30252257361185e-06	\\
-6667.46271306818	1.07111681654632e-05	\\
-6666.48393110796	8.17682204161159e-06	\\
-6665.50514914773	1.00939322795894e-05	\\
-6664.5263671875	1.08178504556466e-05	\\
-6663.54758522727	8.93284984438267e-06	\\
-6662.56880326705	9.74627139596369e-06	\\
-6661.59002130682	9.10886248731164e-06	\\
-6660.61123934659	9.27622888415378e-06	\\
-6659.63245738636	9.53907946731429e-06	\\
-6658.65367542614	1.03180391285343e-05	\\
-6657.67489346591	9.53826344719118e-06	\\
-6656.69611150568	9.50994680490533e-06	\\
-6655.71732954546	9.00452994781881e-06	\\
-6654.73854758523	8.6181664051648e-06	\\
-6653.759765625	8.58278679681963e-06	\\
-6652.78098366477	7.7467750675083e-06	\\
-6651.80220170455	8.906199239007e-06	\\
-6650.82341974432	8.10160158892743e-06	\\
-6649.84463778409	8.48018324896323e-06	\\
-6648.86585582386	6.6163440214099e-06	\\
-6647.88707386364	9.16867215180989e-06	\\
-6646.90829190341	9.66821416685944e-06	\\
-6645.92950994318	9.14660236306365e-06	\\
-6644.95072798296	8.40509803097052e-06	\\
-6643.97194602273	8.35942221762775e-06	\\
-6642.9931640625	9.42905946568649e-06	\\
-6642.01438210227	8.4710847789605e-06	\\
-6641.03560014205	9.04188792731244e-06	\\
-6640.05681818182	7.7393432131257e-06	\\
-6639.07803622159	8.15301604772825e-06	\\
-6638.09925426136	8.24478470264844e-06	\\
-6637.12047230114	8.40698640033759e-06	\\
-6636.14169034091	8.93031469429665e-06	\\
-6635.16290838068	8.14496854481195e-06	\\
-6634.18412642046	1.03021922785626e-05	\\
-6633.20534446023	9.75324129964824e-06	\\
-6632.2265625	7.77723977103694e-06	\\
-6631.24778053977	8.83840065427446e-06	\\
-6630.26899857955	8.33022665529308e-06	\\
-6629.29021661932	1.0237007986772e-05	\\
-6628.31143465909	9.98213592108313e-06	\\
-6627.33265269886	1.04374360540913e-05	\\
-6626.35387073864	1.10226612271393e-05	\\
-6625.37508877841	8.70313221660756e-06	\\
-6624.39630681818	8.07429684988261e-06	\\
-6623.41752485796	7.69657575636551e-06	\\
-6622.43874289773	8.29309374561701e-06	\\
-6621.4599609375	8.21160029901711e-06	\\
-6620.48117897727	7.70889018220709e-06	\\
-6619.50239701705	9.04568493133063e-06	\\
-6618.52361505682	8.12614037068377e-06	\\
-6617.54483309659	7.57792674501564e-06	\\
-6616.56605113636	9.92513271534477e-06	\\
-6615.58726917614	9.41341313656236e-06	\\
-6614.60848721591	9.52794152934053e-06	\\
-6613.62970525568	7.78115080433031e-06	\\
-6612.65092329546	1.17643139526075e-05	\\
-6611.67214133523	8.35842317640187e-06	\\
-6610.693359375	7.48929338727429e-06	\\
-6609.71457741477	7.35679165065665e-06	\\
-6608.73579545455	7.13131820465463e-06	\\
-6607.75701349432	8.97438428024854e-06	\\
-6606.77823153409	7.09676777452328e-06	\\
-6605.79944957386	8.12163410055809e-06	\\
-6604.82066761364	9.22771843312164e-06	\\
-6603.84188565341	8.85075019182923e-06	\\
-6602.86310369318	1.01087980100824e-05	\\
-6601.88432173296	8.8745005073157e-06	\\
-6600.90553977273	7.95410176061368e-06	\\
-6599.9267578125	5.65903141519818e-06	\\
-6598.94797585227	8.14788150422876e-06	\\
-6597.96919389205	6.91657322637595e-06	\\
-6596.99041193182	8.09183498573784e-06	\\
-6596.01162997159	9.63453627230418e-06	\\
-6595.03284801136	1.14789339636648e-05	\\
-6594.05406605114	7.10114854744351e-06	\\
-6593.07528409091	8.20847021204979e-06	\\
-6592.09650213068	7.59033299021468e-06	\\
-6591.11772017046	9.27863801851659e-06	\\
-6590.13893821023	8.79926311434447e-06	\\
-6589.16015625	9.00555280217679e-06	\\
-6588.18137428977	8.82884323829954e-06	\\
-6587.20259232955	1.02482621282389e-05	\\
-6586.22381036932	8.35645381106002e-06	\\
-6585.24502840909	7.75969122227994e-06	\\
-6584.26624644886	9.24470737733824e-06	\\
-6583.28746448864	8.3024960393563e-06	\\
-6582.30868252841	8.2409505120804e-06	\\
-6581.32990056818	1.01283393842752e-05	\\
-6580.35111860796	8.42466626733721e-06	\\
-6579.37233664773	6.68727781334698e-06	\\
-6578.3935546875	8.41752192007868e-06	\\
-6577.41477272727	8.08740786141078e-06	\\
-6576.43599076705	8.89771774861252e-06	\\
-6575.45720880682	8.73122623699127e-06	\\
-6574.47842684659	9.12370079215286e-06	\\
-6573.49964488636	8.75044017543056e-06	\\
-6572.52086292614	7.81449843525046e-06	\\
-6571.54208096591	8.8587783981447e-06	\\
-6570.56329900568	7.22722833449079e-06	\\
-6569.58451704546	1.01213840468923e-05	\\
-6568.60573508523	1.12847712780401e-05	\\
-6567.626953125	1.0416299244038e-05	\\
-6566.64817116477	1.02687550282193e-05	\\
-6565.66938920455	9.58427694908057e-06	\\
-6564.69060724432	9.16976158895402e-06	\\
-6563.71182528409	8.52691619065622e-06	\\
-6562.73304332386	8.10573348649552e-06	\\
-6561.75426136364	8.26083172672959e-06	\\
-6560.77547940341	9.49314576026799e-06	\\
-6559.79669744318	8.15359150786684e-06	\\
-6558.81791548296	8.69769414337929e-06	\\
-6557.83913352273	8.80855291260796e-06	\\
-6556.8603515625	1.0175420923132e-05	\\
-6555.88156960227	8.54904579613728e-06	\\
-6554.90278764205	1.0003177898552e-05	\\
-6553.92400568182	9.96281249763447e-06	\\
-6552.94522372159	1.13845633527229e-05	\\
-6551.96644176136	9.59272178712896e-06	\\
-6550.98765980114	1.02897451129021e-05	\\
-6550.00887784091	8.57678352760411e-06	\\
-6549.03009588068	9.41235340636219e-06	\\
-6548.05131392046	1.05192962655264e-05	\\
-6547.07253196023	8.43804176603709e-06	\\
-6546.09375	8.79974553011078e-06	\\
-6545.11496803977	9.20829129593363e-06	\\
-6544.13618607955	8.45031138595918e-06	\\
-6543.15740411932	9.64909565984739e-06	\\
-6542.17862215909	9.70887426840726e-06	\\
-6541.19984019886	8.21469974300767e-06	\\
-6540.22105823864	8.45184870876614e-06	\\
-6539.24227627841	9.59019654374868e-06	\\
-6538.26349431818	1.16726981774021e-05	\\
-6537.28471235796	1.04484472578761e-05	\\
-6536.30593039773	8.24973122335049e-06	\\
-6535.3271484375	8.64021563711814e-06	\\
-6534.34836647727	9.34865385146934e-06	\\
-6533.36958451705	9.06414100225618e-06	\\
-6532.39080255682	8.64150955898057e-06	\\
-6531.41202059659	9.11231671641368e-06	\\
-6530.43323863636	7.3620096103317e-06	\\
-6529.45445667614	7.48110275956129e-06	\\
-6528.47567471591	9.67018695479775e-06	\\
-6527.49689275568	9.79077582036073e-06	\\
-6526.51811079546	1.05985175622609e-05	\\
-6525.53932883523	1.07930832450268e-05	\\
-6524.560546875	9.34871731055096e-06	\\
-6523.58176491477	8.23750065142247e-06	\\
-6522.60298295455	9.901205259367e-06	\\
-6521.62420099432	1.04470435675682e-05	\\
-6520.64541903409	8.38683204907569e-06	\\
-6519.66663707386	1.07802365588633e-05	\\
-6518.68785511364	1.07903820501664e-05	\\
-6517.70907315341	9.43861480147141e-06	\\
-6516.73029119318	9.41377421510318e-06	\\
-6515.75150923296	1.02656799121609e-05	\\
-6514.77272727273	8.06764880733132e-06	\\
-6513.7939453125	1.06979393273903e-05	\\
-6512.81516335227	1.09412346494908e-05	\\
-6511.83638139205	1.16211439579469e-05	\\
-6510.85759943182	9.22912203459517e-06	\\
-6509.87881747159	8.28208591800287e-06	\\
-6508.90003551136	9.65315991526928e-06	\\
-6507.92125355114	8.58821603993299e-06	\\
-6506.94247159091	9.12949424815988e-06	\\
-6505.96368963068	9.10576276828587e-06	\\
-6504.98490767046	9.45452183307324e-06	\\
-6504.00612571023	7.78273050888696e-06	\\
-6503.02734375	8.49964070631768e-06	\\
-6502.04856178977	9.03356701199011e-06	\\
-6501.06977982955	9.73636805426708e-06	\\
-6500.09099786932	1.22796794014604e-05	\\
-6499.11221590909	1.15768215093137e-05	\\
-6498.13343394886	7.52924891642413e-06	\\
-6497.15465198864	1.029466696794e-05	\\
-6496.17587002841	1.03278520339438e-05	\\
-6495.19708806818	9.23487875570879e-06	\\
-6494.21830610796	9.20149045948414e-06	\\
-6493.23952414773	7.22651160392155e-06	\\
-6492.2607421875	1.18864016934386e-05	\\
-6491.28196022727	8.58559591688729e-06	\\
-6490.30317826705	8.0492534977045e-06	\\
-6489.32439630682	1.09221544637319e-05	\\
-6488.34561434659	9.67178127281977e-06	\\
-6487.36683238636	9.93276290196641e-06	\\
-6486.38805042614	8.07963956163226e-06	\\
-6485.40926846591	9.49275321031118e-06	\\
-6484.43048650568	9.28031297823287e-06	\\
-6483.45170454546	9.15410596406701e-06	\\
-6482.47292258523	1.09103714991432e-05	\\
-6481.494140625	1.07661608045333e-05	\\
-6480.51535866477	1.09146552900389e-05	\\
-6479.53657670455	8.28654147042861e-06	\\
-6478.55779474432	1.00033993673836e-05	\\
-6477.57901278409	9.01380063522072e-06	\\
-6476.60023082386	8.73243588722933e-06	\\
-6475.62144886364	1.01536174175021e-05	\\
-6474.64266690341	9.40785636001547e-06	\\
-6473.66388494318	1.09302208749721e-05	\\
-6472.68510298296	7.44745021465221e-06	\\
-6471.70632102273	9.81571785258344e-06	\\
-6470.7275390625	8.95934360087105e-06	\\
-6469.74875710227	8.89424640870987e-06	\\
-6468.76997514205	9.28399655498786e-06	\\
-6467.79119318182	7.60870194634005e-06	\\
-6466.81241122159	8.89416264256334e-06	\\
-6465.83362926136	7.51010461908783e-06	\\
-6464.85484730114	7.88007996642692e-06	\\
-6463.87606534091	8.54752504752738e-06	\\
-6462.89728338068	7.55873401822311e-06	\\
-6461.91850142046	9.75135195978501e-06	\\
-6460.93971946023	9.76863719420982e-06	\\
-6459.9609375	6.89885962848442e-06	\\
-6458.98215553977	1.05596723133305e-05	\\
-6458.00337357955	7.76041344748866e-06	\\
-6457.02459161932	9.37473325749555e-06	\\
-6456.04580965909	7.72721005143105e-06	\\
-6455.06702769886	9.03920983899803e-06	\\
-6454.08824573864	9.06047051056444e-06	\\
-6453.10946377841	1.0151160758812e-05	\\
-6452.13068181818	6.86096698096596e-06	\\
-6451.15189985796	9.92687210141526e-06	\\
-6450.17311789773	9.15252751253098e-06	\\
-6449.1943359375	8.5334213109348e-06	\\
-6448.21555397727	8.19282398540594e-06	\\
-6447.23677201705	1.03569913836335e-05	\\
-6446.25799005682	1.07527240415624e-05	\\
-6445.27920809659	7.78000656000061e-06	\\
-6444.30042613636	9.03738630849419e-06	\\
-6443.32164417614	9.35583432939379e-06	\\
-6442.34286221591	9.21116196524774e-06	\\
-6441.36408025568	8.44510797762435e-06	\\
-6440.38529829546	9.95485280411251e-06	\\
-6439.40651633523	8.95760169509512e-06	\\
-6438.427734375	9.15393043895657e-06	\\
-6437.44895241477	9.06504838858732e-06	\\
-6436.47017045455	9.10607607429404e-06	\\
-6435.49138849432	8.96735958678298e-06	\\
-6434.51260653409	8.89720418957706e-06	\\
-6433.53382457386	6.60992965782825e-06	\\
-6432.55504261364	8.73172233101893e-06	\\
-6431.57626065341	1.06798910095109e-05	\\
-6430.59747869318	9.56050497560511e-06	\\
-6429.61869673296	8.95776911406732e-06	\\
-6428.63991477273	9.14078078422633e-06	\\
-6427.6611328125	8.84143168727729e-06	\\
-6426.68235085227	7.12643627844071e-06	\\
-6425.70356889205	9.1044784212695e-06	\\
-6424.72478693182	7.06767925577544e-06	\\
-6423.74600497159	7.22591735019307e-06	\\
-6422.76722301136	7.94916299660805e-06	\\
-6421.78844105114	7.13283109029337e-06	\\
-6420.80965909091	8.18380669992689e-06	\\
-6419.83087713068	8.72937740805095e-06	\\
-6418.85209517046	7.60955051393078e-06	\\
-6417.87331321023	8.90548302641017e-06	\\
-6416.89453125	8.02901999354197e-06	\\
-6415.91574928977	8.28241405692516e-06	\\
-6414.93696732955	9.11365241848967e-06	\\
-6413.95818536932	9.81119728256284e-06	\\
-6412.97940340909	9.76318996230859e-06	\\
-6412.00062144886	1.05621889318567e-05	\\
-6411.02183948864	8.24854901059863e-06	\\
-6410.04305752841	7.13274487534369e-06	\\
-6409.06427556818	8.66855189608797e-06	\\
-6408.08549360796	9.01363077918889e-06	\\
-6407.10671164773	7.2428743469579e-06	\\
-6406.1279296875	8.53245111239864e-06	\\
-6405.14914772727	8.28587237923274e-06	\\
-6404.17036576705	7.12529422432021e-06	\\
-6403.19158380682	7.36905802998727e-06	\\
-6402.21280184659	1.01871006457568e-05	\\
-6401.23401988636	9.31896600113932e-06	\\
-6400.25523792614	7.5097261680601e-06	\\
-6399.27645596591	6.74210748225763e-06	\\
-6398.29767400568	8.76479771141405e-06	\\
-6397.31889204546	8.5373037348107e-06	\\
-6396.34011008523	8.15568264315858e-06	\\
-6395.361328125	5.98246322048289e-06	\\
-6394.38254616477	9.00257539835349e-06	\\
-6393.40376420455	7.4278058814387e-06	\\
-6392.42498224432	9.36792898465939e-06	\\
-6391.44620028409	6.06185843121162e-06	\\
-6390.46741832386	7.17287536090897e-06	\\
-6389.48863636364	6.7837419653776e-06	\\
-6388.50985440341	6.8654262149285e-06	\\
-6387.53107244318	5.83630337861044e-06	\\
-6386.55229048296	6.28648159544853e-06	\\
-6385.57350852273	7.15250565176501e-06	\\
-6384.5947265625	8.0303636112127e-06	\\
-6383.61594460227	5.46409450583429e-06	\\
-6382.63716264205	6.0222262280122e-06	\\
-6381.65838068182	5.38029996256066e-06	\\
-6380.67959872159	7.9849743396737e-06	\\
-6379.70081676136	5.38656964936777e-06	\\
-6378.72203480114	5.09750842808369e-06	\\
-6377.74325284091	6.75400713349549e-06	\\
-6376.76447088068	7.1144162729288e-06	\\
-6375.78568892046	6.95088543545446e-06	\\
-6374.80690696023	5.51478064192761e-06	\\
-6373.828125	5.78790231498287e-06	\\
-6372.84934303977	7.22952629570996e-06	\\
-6371.87056107955	7.11875156770047e-06	\\
-6370.89177911932	7.29401400806477e-06	\\
-6369.91299715909	7.24356589016993e-06	\\
-6368.93421519886	8.1132984388784e-06	\\
-6367.95543323864	6.41381128617023e-06	\\
-6366.97665127841	6.67693321100091e-06	\\
-6365.99786931818	6.50027301944308e-06	\\
-6365.01908735796	6.13092351230047e-06	\\
-6364.04030539773	6.82167470182136e-06	\\
-6363.0615234375	6.5491239638106e-06	\\
-6362.08274147727	8.17143453746824e-06	\\
-6361.10395951705	6.34426167165391e-06	\\
-6360.12517755682	8.49743969658923e-06	\\
-6359.14639559659	7.69919991796736e-06	\\
-6358.16761363636	7.23319474943024e-06	\\
-6357.18883167614	8.36351051280417e-06	\\
-6356.21004971591	6.4316780934188e-06	\\
-6355.23126775568	8.03585769941896e-06	\\
-6354.25248579546	9.69188667736395e-06	\\
-6353.27370383523	8.01928436016067e-06	\\
-6352.294921875	7.6186784719683e-06	\\
-6351.31613991477	8.24524045895433e-06	\\
-6350.33735795455	7.4705057473295e-06	\\
-6349.35857599432	4.24525622054517e-06	\\
-6348.37979403409	7.02885152074367e-06	\\
-6347.40101207386	7.73841344705279e-06	\\
-6346.42223011364	7.52041792449378e-06	\\
-6345.44344815341	5.71318595627649e-06	\\
-6344.46466619318	5.47861915017882e-06	\\
-6343.48588423296	8.19790527577625e-06	\\
-6342.50710227273	8.49281418069927e-06	\\
-6341.5283203125	9.55388027158276e-06	\\
-6340.54953835227	8.09962986322696e-06	\\
-6339.57075639205	7.09945352964307e-06	\\
-6338.59197443182	6.47822069521606e-06	\\
-6337.61319247159	8.74576087385979e-06	\\
-6336.63441051136	7.81352692845203e-06	\\
-6335.65562855114	7.79566506975927e-06	\\
-6334.67684659091	6.26226504015065e-06	\\
-6333.69806463068	7.28920970605638e-06	\\
-6332.71928267046	6.1121670838958e-06	\\
-6331.74050071023	7.61146041777013e-06	\\
-6330.76171875	8.18964953646279e-06	\\
-6329.78293678977	8.67346290708575e-06	\\
-6328.80415482955	7.01834676790718e-06	\\
-6327.82537286932	8.95598259109087e-06	\\
-6326.84659090909	7.7512105561204e-06	\\
-6325.86780894886	8.52884945271739e-06	\\
-6324.88902698864	7.17809929802723e-06	\\
-6323.91024502841	6.74037532316748e-06	\\
-6322.93146306818	6.69068692484868e-06	\\
-6321.95268110796	6.9239666316266e-06	\\
-6320.97389914773	8.02124733939307e-06	\\
-6319.9951171875	5.01773686829645e-06	\\
-6319.01633522727	7.83940420284915e-06	\\
-6318.03755326705	8.49611690046071e-06	\\
-6317.05877130682	7.80730916990234e-06	\\
-6316.07998934659	6.05912222357082e-06	\\
-6315.10120738636	9.17079271451842e-06	\\
-6314.12242542614	7.81713457259381e-06	\\
-6313.14364346591	7.39021252037268e-06	\\
-6312.16486150568	7.23465067900596e-06	\\
-6311.18607954546	7.75330301976241e-06	\\
-6310.20729758523	6.87199378933206e-06	\\
-6309.228515625	8.35160813405132e-06	\\
-6308.24973366477	7.87140514733996e-06	\\
-6307.27095170455	6.75825062631293e-06	\\
-6306.29216974432	7.77361568859704e-06	\\
-6305.31338778409	7.09910118459119e-06	\\
-6304.33460582386	6.77639006759838e-06	\\
-6303.35582386364	5.940539734588e-06	\\
-6302.37704190341	7.17601256671316e-06	\\
-6301.39825994318	6.45493305788944e-06	\\
-6300.41947798296	6.55491093843706e-06	\\
-6299.44069602273	7.28212210666105e-06	\\
-6298.4619140625	6.61493523459627e-06	\\
-6297.48313210227	7.74811408901633e-06	\\
-6296.50435014205	7.86200482637019e-06	\\
-6295.52556818182	8.82386252015503e-06	\\
-6294.54678622159	8.2957470615779e-06	\\
-6293.56800426136	6.48831950638278e-06	\\
-6292.58922230114	7.38784513793305e-06	\\
-6291.61044034091	9.11081562555637e-06	\\
-6290.63165838068	6.40921762727842e-06	\\
-6289.65287642046	7.2671490985047e-06	\\
-6288.67409446023	7.24676146823928e-06	\\
-6287.6953125	8.22245715929146e-06	\\
-6286.71653053977	6.22862502586917e-06	\\
-6285.73774857955	7.28259758270276e-06	\\
-6284.75896661932	6.93219204029637e-06	\\
-6283.78018465909	6.71747041126272e-06	\\
-6282.80140269886	8.82616761029675e-06	\\
-6281.82262073864	8.42117454860918e-06	\\
-6280.84383877841	6.60935504427577e-06	\\
-6279.86505681818	7.00615233029949e-06	\\
-6278.88627485796	6.91528054971029e-06	\\
-6277.90749289773	5.77668685610293e-06	\\
-6276.9287109375	7.77654791133683e-06	\\
-6275.94992897727	8.22329036549148e-06	\\
-6274.97114701705	7.19385337313961e-06	\\
-6273.99236505682	6.05052738855298e-06	\\
-6273.01358309659	7.69753265171221e-06	\\
-6272.03480113636	8.37174644820524e-06	\\
-6271.05601917614	7.7075336940978e-06	\\
-6270.07723721591	7.89211119945611e-06	\\
-6269.09845525568	6.64993240336296e-06	\\
-6268.11967329546	7.72528129749765e-06	\\
-6267.14089133523	8.72092810050389e-06	\\
-6266.162109375	7.16860458343325e-06	\\
-6265.18332741477	6.91805412631722e-06	\\
-6264.20454545455	7.94061845636357e-06	\\
-6263.22576349432	7.66915905409323e-06	\\
-6262.24698153409	8.80545377892153e-06	\\
-6261.26819957386	7.91867629045205e-06	\\
-6260.28941761364	4.91700644796767e-06	\\
-6259.31063565341	8.03264672068604e-06	\\
-6258.33185369318	7.33703589377089e-06	\\
-6257.35307173296	7.90960498724363e-06	\\
-6256.37428977273	1.0229189603271e-05	\\
-6255.3955078125	6.77854431658808e-06	\\
-6254.41672585227	9.02541383274001e-06	\\
-6253.43794389205	8.15404085732937e-06	\\
-6252.45916193182	7.57434015904764e-06	\\
-6251.48037997159	8.40276894865099e-06	\\
-6250.50159801136	6.71670473466694e-06	\\
-6249.52281605114	6.74502492603545e-06	\\
-6248.54403409091	7.4407936718572e-06	\\
-6247.56525213068	9.77172437124157e-06	\\
-6246.58647017046	5.67946654235819e-06	\\
-6245.60768821023	8.43062139885919e-06	\\
-6244.62890625	8.43067508735705e-06	\\
-6243.65012428977	6.85675644357349e-06	\\
-6242.67134232955	8.81976207479148e-06	\\
-6241.69256036932	7.42631602500037e-06	\\
-6240.71377840909	8.09391104398845e-06	\\
-6239.73499644886	7.79996062585549e-06	\\
-6238.75621448864	7.21057397765391e-06	\\
-6237.77743252841	9.27984637132758e-06	\\
-6236.79865056818	7.52878573802313e-06	\\
-6235.81986860796	8.26921199802208e-06	\\
-6234.84108664773	7.09869253907384e-06	\\
-6233.8623046875	8.96441714133727e-06	\\
-6232.88352272727	8.71046032856703e-06	\\
-6231.90474076705	8.62026047507746e-06	\\
-6230.92595880682	1.01563745385386e-05	\\
-6229.94717684659	6.75365199502606e-06	\\
-6228.96839488636	1.10966248564076e-05	\\
-6227.98961292614	7.5105998372225e-06	\\
-6227.01083096591	9.07636881513476e-06	\\
-6226.03204900568	8.33110407149913e-06	\\
-6225.05326704546	9.64205141034742e-06	\\
-6224.07448508523	9.44773701519894e-06	\\
-6223.095703125	6.01311315434339e-06	\\
-6222.11692116477	9.3716518823986e-06	\\
-6221.13813920455	7.96570543017326e-06	\\
-6220.15935724432	9.42007246778163e-06	\\
-6219.18057528409	8.32373698045655e-06	\\
-6218.20179332386	8.9480758646265e-06	\\
-6217.22301136364	8.276560466567e-06	\\
-6216.24422940341	9.61593077802284e-06	\\
-6215.26544744318	8.0030536263224e-06	\\
-6214.28666548296	7.9998673592457e-06	\\
-6213.30788352273	1.01587655814866e-05	\\
-6212.3291015625	6.78475912458131e-06	\\
-6211.35031960227	7.84938090847747e-06	\\
-6210.37153764205	8.54003780767734e-06	\\
-6209.39275568182	6.77852651815311e-06	\\
-6208.41397372159	8.47177814545485e-06	\\
-6207.43519176136	9.18693334154491e-06	\\
-6206.45640980114	7.83868691340482e-06	\\
-6205.47762784091	8.10843830627694e-06	\\
-6204.49884588068	7.67088227919641e-06	\\
-6203.52006392046	1.04974335888085e-05	\\
-6202.54128196023	8.51355153883332e-06	\\
-6201.5625	7.94459443527765e-06	\\
-6200.58371803977	7.73558188095617e-06	\\
-6199.60493607955	8.63678681296248e-06	\\
-6198.62615411932	8.77667682504028e-06	\\
-6197.64737215909	1.01030573201252e-05	\\
-6196.66859019886	9.93623199852423e-06	\\
-6195.68980823864	7.79115179304732e-06	\\
-6194.71102627841	9.50975837381775e-06	\\
-6193.73224431818	9.72838028164871e-06	\\
-6192.75346235796	7.53327110957291e-06	\\
-6191.77468039773	7.66539511209318e-06	\\
-6190.7958984375	6.69271486423742e-06	\\
-6189.81711647727	8.02124595762338e-06	\\
-6188.83833451705	9.7168873808354e-06	\\
-6187.85955255682	9.50874309305719e-06	\\
-6186.88077059659	7.77871269091838e-06	\\
-6185.90198863636	7.73738224265874e-06	\\
-6184.92320667614	7.71908100710342e-06	\\
-6183.94442471591	8.6653170411902e-06	\\
-6182.96564275568	9.66235729024191e-06	\\
-6181.98686079546	7.78170332997037e-06	\\
-6181.00807883523	8.16339974763274e-06	\\
-6180.029296875	7.89241477760934e-06	\\
-6179.05051491477	8.09572495082672e-06	\\
-6178.07173295455	9.03342519623578e-06	\\
-6177.09295099432	1.03258591703767e-05	\\
-6176.11416903409	7.46785866774221e-06	\\
-6175.13538707386	7.47398229798129e-06	\\
-6174.15660511364	7.87098294463293e-06	\\
-6173.17782315341	8.33566136665778e-06	\\
-6172.19904119318	8.73280615877353e-06	\\
-6171.22025923296	6.79067922547555e-06	\\
-6170.24147727273	8.28008615687095e-06	\\
-6169.2626953125	7.61709816219417e-06	\\
-6168.28391335227	1.04501176870062e-05	\\
-6167.30513139205	8.20728205547058e-06	\\
-6166.32634943182	9.90104543623886e-06	\\
-6165.34756747159	8.84643214380653e-06	\\
-6164.36878551136	7.14187310528431e-06	\\
-6163.39000355114	7.21547181031573e-06	\\
-6162.41122159091	6.22935081357252e-06	\\
-6161.43243963068	8.41142390790972e-06	\\
-6160.45365767046	9.17612898553175e-06	\\
-6159.47487571023	9.59233441650178e-06	\\
-6158.49609375	7.64185273267211e-06	\\
-6157.51731178977	9.08740694151217e-06	\\
-6156.53852982955	9.55015179790634e-06	\\
-6155.55974786932	7.46631351268818e-06	\\
-6154.58096590909	6.79394080189256e-06	\\
-6153.60218394886	7.71282487844554e-06	\\
-6152.62340198864	8.80962978504366e-06	\\
-6151.64462002841	9.50183314775993e-06	\\
-6150.66583806818	8.83055780365871e-06	\\
-6149.68705610796	7.11815389652692e-06	\\
-6148.70827414773	9.52315130554397e-06	\\
-6147.7294921875	7.89033126910565e-06	\\
-6146.75071022727	8.88380493698267e-06	\\
-6145.77192826705	7.21685866863199e-06	\\
-6144.79314630682	1.00083075262611e-05	\\
-6143.81436434659	8.66466379088035e-06	\\
-6142.83558238636	7.73679427477782e-06	\\
-6141.85680042614	9.67512828639098e-06	\\
-6140.87801846591	8.94624184340623e-06	\\
-6139.89923650568	8.11415023196335e-06	\\
-6138.92045454546	9.20553656506597e-06	\\
-6137.94167258523	1.04816686688478e-05	\\
-6136.962890625	7.57872972732188e-06	\\
-6135.98410866477	8.54149027641312e-06	\\
-6135.00532670455	8.35095200428086e-06	\\
-6134.02654474432	5.49992777958933e-06	\\
-6133.04776278409	8.7164269394512e-06	\\
-6132.06898082386	8.75817026261942e-06	\\
-6131.09019886364	7.51745731484965e-06	\\
-6130.11141690341	7.8111655624583e-06	\\
-6129.13263494318	8.68957557210668e-06	\\
-6128.15385298296	7.71862278607849e-06	\\
-6127.17507102273	8.64406055455101e-06	\\
-6126.1962890625	8.87886198645326e-06	\\
-6125.21750710227	8.96655956436552e-06	\\
-6124.23872514205	8.26788331879427e-06	\\
-6123.25994318182	8.35489316611074e-06	\\
-6122.28116122159	8.95541400224902e-06	\\
-6121.30237926136	6.85538218089143e-06	\\
-6120.32359730114	8.38631724014092e-06	\\
-6119.34481534091	9.3141051260966e-06	\\
-6118.36603338068	9.52745087246285e-06	\\
-6117.38725142046	6.79892234742269e-06	\\
-6116.40846946023	9.33293269534537e-06	\\
-6115.4296875	7.8008384783219e-06	\\
-6114.45090553977	8.56859151494014e-06	\\
-6113.47212357955	8.45194534890776e-06	\\
-6112.49334161932	8.56153377460774e-06	\\
-6111.51455965909	8.43527899843307e-06	\\
-6110.53577769886	7.07167122198332e-06	\\
-6109.55699573864	8.99085928952513e-06	\\
-6108.57821377841	8.15823318082357e-06	\\
-6107.59943181818	8.06827237222756e-06	\\
-6106.62064985796	9.78064832458424e-06	\\
-6105.64186789773	7.39204597011677e-06	\\
-6104.6630859375	7.18964144820881e-06	\\
-6103.68430397727	8.94503010813554e-06	\\
-6102.70552201705	8.67955302153254e-06	\\
-6101.72674005682	8.62814798896719e-06	\\
-6100.74795809659	9.1255505147828e-06	\\
-6099.76917613636	8.6531816483689e-06	\\
-6098.79039417614	7.97323947225522e-06	\\
-6097.81161221591	6.88730489606175e-06	\\
-6096.83283025568	8.72971835593521e-06	\\
-6095.85404829546	1.01754194764948e-05	\\
-6094.87526633523	9.39665940099445e-06	\\
-6093.896484375	9.06408841888738e-06	\\
-6092.91770241477	9.39163697353558e-06	\\
-6091.93892045455	9.08612142662527e-06	\\
-6090.96013849432	7.98411167569872e-06	\\
-6089.98135653409	9.22290871665319e-06	\\
-6089.00257457386	8.00218794211504e-06	\\
-6088.02379261364	8.83116730657439e-06	\\
-6087.04501065341	7.69717402881291e-06	\\
-6086.06622869318	9.6591122487773e-06	\\
-6085.08744673296	8.62052019693221e-06	\\
-6084.10866477273	7.54826283692827e-06	\\
-6083.1298828125	6.52962005284858e-06	\\
-6082.15110085227	7.3484229040308e-06	\\
-6081.17231889205	9.10670165441524e-06	\\
-6080.19353693182	8.23179495412411e-06	\\
-6079.21475497159	7.65583134007616e-06	\\
-6078.23597301136	8.38712785127622e-06	\\
-6077.25719105114	8.88375125294378e-06	\\
-6076.27840909091	9.68383394797998e-06	\\
-6075.29962713068	8.19820167243453e-06	\\
-6074.32084517046	7.44274624741008e-06	\\
-6073.34206321023	1.04486959897592e-05	\\
-6072.36328125	8.44327676448803e-06	\\
-6071.38449928977	9.12349465312862e-06	\\
-6070.40571732955	8.7869718317272e-06	\\
-6069.42693536932	7.90530293702816e-06	\\
-6068.44815340909	6.54840471315952e-06	\\
-6067.46937144886	8.85289016650787e-06	\\
-6066.49058948864	8.72020518482622e-06	\\
-6065.51180752841	8.8210439468182e-06	\\
-6064.53302556818	9.83983386681171e-06	\\
-6063.55424360796	9.46496787826227e-06	\\
-6062.57546164773	9.15942743772423e-06	\\
-6061.5966796875	7.7320752600795e-06	\\
-6060.61789772727	8.47262562802268e-06	\\
-6059.63911576705	7.59579445871175e-06	\\
-6058.66033380682	7.23930295443347e-06	\\
-6057.68155184659	6.89879048410289e-06	\\
-6056.70276988636	6.86863437185286e-06	\\
-6055.72398792614	8.11293399847671e-06	\\
-6054.74520596591	7.4714531475871e-06	\\
-6053.76642400568	1.03156963390199e-05	\\
-6052.78764204546	8.68722161665737e-06	\\
-6051.80886008523	8.29509189770029e-06	\\
-6050.830078125	9.95726380616756e-06	\\
-6049.85129616477	9.91478684422801e-06	\\
-6048.87251420455	8.54907905348612e-06	\\
-6047.89373224432	8.43895340479614e-06	\\
-6046.91495028409	8.48116170246811e-06	\\
-6045.93616832386	8.10931185894386e-06	\\
-6044.95738636364	1.00818461089743e-05	\\
-6043.97860440341	9.84950278024922e-06	\\
-6042.99982244318	8.1958740875374e-06	\\
-6042.02104048296	9.63152899315457e-06	\\
-6041.04225852273	5.6089694761902e-06	\\
-6040.0634765625	8.78883476408924e-06	\\
-6039.08469460227	5.49835829163957e-06	\\
-6038.10591264205	1.0718507569977e-05	\\
-6037.12713068182	7.48016525938875e-06	\\
-6036.14834872159	8.0893194323516e-06	\\
-6035.16956676136	8.31495455978052e-06	\\
-6034.19078480114	9.10965134910341e-06	\\
-6033.21200284091	8.7018637531526e-06	\\
-6032.23322088068	9.17161481519981e-06	\\
-6031.25443892046	9.22820306560407e-06	\\
-6030.27565696023	8.76204704168679e-06	\\
-6029.296875	8.13068662174157e-06	\\
-6028.31809303977	7.8582227039866e-06	\\
-6027.33931107955	9.87424860210265e-06	\\
-6026.36052911932	9.01009671837525e-06	\\
-6025.38174715909	9.49750764755093e-06	\\
-6024.40296519886	9.60915933904235e-06	\\
-6023.42418323864	7.32522685519402e-06	\\
-6022.44540127841	7.59401340792951e-06	\\
-6021.46661931818	8.61849481291439e-06	\\
-6020.48783735796	8.28576919770566e-06	\\
-6019.50905539773	1.01989991625977e-05	\\
-6018.5302734375	7.12555528716112e-06	\\
-6017.55149147727	7.66345961556644e-06	\\
-6016.57270951705	8.88169847191417e-06	\\
-6015.59392755682	7.42948876427766e-06	\\
-6014.61514559659	9.00684604390887e-06	\\
-6013.63636363636	7.78845076519534e-06	\\
-6012.65758167614	9.55064220525834e-06	\\
-6011.67879971591	9.87252811925522e-06	\\
-6010.70001775568	9.10940529871812e-06	\\
-6009.72123579546	9.1890184811648e-06	\\
-6008.74245383523	9.72979128704791e-06	\\
-6007.763671875	8.77747595502606e-06	\\
-6006.78488991477	8.30826952686472e-06	\\
-6005.80610795455	8.43707757121326e-06	\\
-6004.82732599432	7.4643911515606e-06	\\
-6003.84854403409	9.03542755994775e-06	\\
-6002.86976207386	7.84319341066985e-06	\\
-6001.89098011364	9.71972589688494e-06	\\
-6000.91219815341	8.51420004203077e-06	\\
-5999.93341619318	1.08330461778995e-05	\\
-5998.95463423296	1.03263529134998e-05	\\
-5997.97585227273	8.92460651390156e-06	\\
-5996.9970703125	8.71133997088525e-06	\\
-5996.01828835227	8.80159289830043e-06	\\
-5995.03950639205	8.53508473304187e-06	\\
-5994.06072443182	6.78942056109616e-06	\\
-5993.08194247159	7.02346791960969e-06	\\
-5992.10316051136	7.18881507006961e-06	\\
-5991.12437855114	7.7289416773606e-06	\\
-5990.14559659091	5.26499793599272e-06	\\
-5989.16681463068	6.75579906219729e-06	\\
-5988.18803267046	9.56812619502918e-06	\\
-5987.20925071023	9.81154199770383e-06	\\
-5986.23046875	1.2140991997882e-05	\\
-5985.25168678977	1.03016055932953e-05	\\
-5984.27290482955	8.49097562378013e-06	\\
-5983.29412286932	9.32958096868588e-06	\\
-5982.31534090909	8.79477837215145e-06	\\
-5981.33655894886	9.27927010293702e-06	\\
-5980.35777698864	7.40412778406176e-06	\\
-5979.37899502841	9.43754588755134e-06	\\
-5978.40021306818	7.87576101700696e-06	\\
-5977.42143110796	7.31495058551165e-06	\\
-5976.44264914773	8.72489722145053e-06	\\
-5975.4638671875	1.00903207331068e-05	\\
-5974.48508522727	8.71461733180578e-06	\\
-5973.50630326705	9.83440725948509e-06	\\
-5972.52752130682	9.46243812275668e-06	\\
-5971.54873934659	9.41928237191086e-06	\\
-5970.56995738636	1.06332055277957e-05	\\
-5969.59117542614	9.48357896356559e-06	\\
-5968.61239346591	8.69465744938435e-06	\\
-5967.63361150568	8.42152970666712e-06	\\
-5966.65482954546	7.69163978836223e-06	\\
-5965.67604758523	8.92235190531943e-06	\\
-5964.697265625	8.23451050861988e-06	\\
-5963.71848366477	8.53582622770392e-06	\\
-5962.73970170455	9.15457584872317e-06	\\
-5961.76091974432	7.81956656828701e-06	\\
-5960.78213778409	8.37343235088499e-06	\\
-5959.80335582386	9.39939990333123e-06	\\
-5958.82457386364	7.27336371334554e-06	\\
-5957.84579190341	9.06785170018081e-06	\\
-5956.86700994318	7.79996212670406e-06	\\
-5955.88822798296	9.12154295370033e-06	\\
-5954.90944602273	9.37766420559098e-06	\\
-5953.9306640625	8.56815130946995e-06	\\
-5952.95188210227	9.69283392241095e-06	\\
-5951.97310014205	9.94784431012099e-06	\\
-5950.99431818182	8.82271657713419e-06	\\
-5950.01553622159	7.97159765414216e-06	\\
-5949.03675426136	8.56690422502817e-06	\\
-5948.05797230114	9.13350400598846e-06	\\
-5947.07919034091	8.8436693337199e-06	\\
-5946.10040838068	7.93350311055876e-06	\\
-5945.12162642046	8.04597575642265e-06	\\
-5944.14284446023	8.53425702902362e-06	\\
-5943.1640625	6.71521299229814e-06	\\
-5942.18528053977	1.10333104459396e-05	\\
-5941.20649857955	9.5116549950652e-06	\\
-5940.22771661932	6.00483346254427e-06	\\
-5939.24893465909	7.73442101838954e-06	\\
-5938.27015269886	7.60368685462455e-06	\\
-5937.29137073864	6.89260184673044e-06	\\
-5936.31258877841	6.24459510557829e-06	\\
-5935.33380681818	8.15118958571981e-06	\\
-5934.35502485796	6.96312037534146e-06	\\
-5933.37624289773	7.37093607079582e-06	\\
-5932.3974609375	9.7983152029166e-06	\\
-5931.41867897727	9.98114941497434e-06	\\
-5930.43989701705	1.00994066404377e-05	\\
-5929.46111505682	8.59939891454508e-06	\\
-5928.48233309659	8.81629307269889e-06	\\
-5927.50355113636	9.13531781941093e-06	\\
-5926.52476917614	8.82275407395023e-06	\\
-5925.54598721591	7.88660438889262e-06	\\
-5924.56720525568	7.38714131315417e-06	\\
-5923.58842329546	8.3111865316939e-06	\\
-5922.60964133523	7.32840771519959e-06	\\
-5921.630859375	8.59589958438164e-06	\\
-5920.65207741477	6.93706096267481e-06	\\
-5919.67329545455	8.07443414022325e-06	\\
-5918.69451349432	7.98732607047629e-06	\\
-5917.71573153409	9.72503293364263e-06	\\
-5916.73694957386	9.56151715125615e-06	\\
-5915.75816761364	7.85008771010231e-06	\\
-5914.77938565341	8.10069642963233e-06	\\
-5913.80060369318	7.35919587432865e-06	\\
-5912.82182173296	8.76513156089719e-06	\\
-5911.84303977273	8.42937405179508e-06	\\
-5910.8642578125	9.97455007853833e-06	\\
-5909.88547585227	1.035572126399e-05	\\
-5908.90669389205	1.04660665955869e-05	\\
-5907.92791193182	8.88741887744968e-06	\\
-5906.94912997159	8.64986404352258e-06	\\
-5905.97034801136	8.83706612861185e-06	\\
-5904.99156605114	7.1749074931365e-06	\\
-5904.01278409091	9.87879940017653e-06	\\
-5903.03400213068	7.99909339132907e-06	\\
-5902.05522017046	6.83782122291177e-06	\\
-5901.07643821023	8.39062456618262e-06	\\
-5900.09765625	9.10075178623528e-06	\\
-5899.11887428977	7.05619457236488e-06	\\
-5898.14009232955	8.09920608497249e-06	\\
-5897.16131036932	8.87318301596282e-06	\\
-5896.18252840909	8.92722198639029e-06	\\
-5895.20374644886	9.44655503893726e-06	\\
-5894.22496448864	8.2356347289575e-06	\\
-5893.24618252841	1.0766755213951e-05	\\
-5892.26740056818	7.56542868364299e-06	\\
-5891.28861860796	6.92820199233132e-06	\\
-5890.30983664773	8.03667624977494e-06	\\
-5889.3310546875	7.06319653082843e-06	\\
-5888.35227272727	6.14165329290167e-06	\\
-5887.37349076705	8.36139754586691e-06	\\
-5886.39470880682	7.13437800027722e-06	\\
-5885.41592684659	7.606613518173e-06	\\
-5884.43714488636	1.01082170692385e-05	\\
-5883.45836292614	8.32204148398853e-06	\\
-5882.47958096591	6.54279789214261e-06	\\
-5881.50079900568	8.88873421010924e-06	\\
-5880.52201704546	9.59321681315643e-06	\\
-5879.54323508523	9.94143733320656e-06	\\
-5878.564453125	6.3105097263178e-06	\\
-5877.58567116477	8.51620885559246e-06	\\
-5876.60688920455	8.15784784517758e-06	\\
-5875.62810724432	8.33674553331096e-06	\\
-5874.64932528409	8.42866380567546e-06	\\
-5873.67054332386	7.90128726611043e-06	\\
-5872.69176136364	6.32471471514557e-06	\\
-5871.71297940341	9.05509132290311e-06	\\
-5870.73419744318	6.9297229080762e-06	\\
-5869.75541548296	6.15651099082327e-06	\\
-5868.77663352273	5.40084688843036e-06	\\
-5867.7978515625	8.92983578501621e-06	\\
-5866.81906960227	6.39425417703703e-06	\\
-5865.84028764205	8.08504395707541e-06	\\
-5864.86150568182	5.69604180733096e-06	\\
-5863.88272372159	8.68722956920943e-06	\\
-5862.90394176136	7.62593735064642e-06	\\
-5861.92515980114	8.2879112064763e-06	\\
-5860.94637784091	9.88546236745001e-06	\\
-5859.96759588068	9.90040626309306e-06	\\
-5858.98881392046	1.06241765509692e-05	\\
-5858.01003196023	8.20417769968665e-06	\\
-5857.03125	8.50222282913738e-06	\\
-5856.05246803977	8.30278979037571e-06	\\
-5855.07368607955	7.5104895004824e-06	\\
-5854.09490411932	7.42046869835998e-06	\\
-5853.11612215909	5.32830917383758e-06	\\
-5852.13734019886	5.78814709675924e-06	\\
-5851.15855823864	6.00777589473901e-06	\\
-5850.17977627841	8.14730136782997e-06	\\
-5849.20099431818	7.04564509728525e-06	\\
-5848.22221235796	8.27400187044176e-06	\\
-5847.24343039773	8.42323670140813e-06	\\
-5846.2646484375	9.33852464116683e-06	\\
-5845.28586647727	8.63517526382443e-06	\\
-5844.30708451705	6.83065684462893e-06	\\
-5843.32830255682	7.43249930121091e-06	\\
-5842.34952059659	5.05642593844648e-06	\\
-5841.37073863636	6.00267263930268e-06	\\
-5840.39195667614	5.35390850481462e-06	\\
-5839.41317471591	6.56318214460837e-06	\\
-5838.43439275568	5.41154437414517e-06	\\
-5837.45561079546	6.47784632289774e-06	\\
-5836.47682883523	5.88696100304937e-06	\\
-5835.498046875	6.73968702419865e-06	\\
-5834.51926491477	7.68747035043833e-06	\\
-5833.54048295455	5.77252342837141e-06	\\
-5832.56170099432	7.99722730718995e-06	\\
-5831.58291903409	6.5447372438199e-06	\\
-5830.60413707386	8.07069214112344e-06	\\
-5829.62535511364	8.86272357766044e-06	\\
-5828.64657315341	1.0246394656993e-05	\\
-5827.66779119318	8.25313287894406e-06	\\
-5826.68900923296	8.09316400247747e-06	\\
-5825.71022727273	8.84675905835662e-06	\\
-5824.7314453125	8.28223698978867e-06	\\
-5823.75266335227	8.18631472228047e-06	\\
-5822.77388139205	9.45051897538155e-06	\\
-5821.79509943182	7.87168234079382e-06	\\
-5820.81631747159	6.2820719332643e-06	\\
-5819.83753551136	5.74904165518301e-06	\\
-5818.85875355114	7.11706363590764e-06	\\
-5817.87997159091	7.28214922193027e-06	\\
-5816.90118963068	8.02443786032651e-06	\\
-5815.92240767046	7.9471430109666e-06	\\
-5814.94362571023	6.19683313716727e-06	\\
-5813.96484375	5.15731966784553e-06	\\
-5812.98606178977	5.23415182775223e-06	\\
-5812.00727982955	8.29333735975066e-06	\\
-5811.02849786932	8.387846524259e-06	\\
-5810.04971590909	5.30669195193211e-06	\\
-5809.07093394886	8.41686529190938e-06	\\
-5808.09215198864	7.19381941730821e-06	\\
-5807.11337002841	6.56439257662582e-06	\\
-5806.13458806818	7.19907084766314e-06	\\
-5805.15580610796	6.93112673404598e-06	\\
-5804.17702414773	5.14600203849905e-06	\\
-5803.1982421875	6.60649785553575e-06	\\
-5802.21946022727	8.39626905225873e-06	\\
-5801.24067826705	5.38040120092793e-06	\\
-5800.26189630682	8.30627900868918e-06	\\
-5799.28311434659	5.48578482294834e-06	\\
-5798.30433238636	6.42366259478178e-06	\\
-5797.32555042614	6.84670324351555e-06	\\
-5796.34676846591	6.21758556061013e-06	\\
-5795.36798650568	8.16810616076204e-06	\\
-5794.38920454546	5.70035449756405e-06	\\
-5793.41042258523	9.12209730582729e-06	\\
-5792.431640625	9.31501786907083e-06	\\
-5791.45285866477	6.37529777490405e-06	\\
-5790.47407670455	8.43309375893771e-06	\\
-5789.49529474432	7.89989429305534e-06	\\
-5788.51651278409	8.56833612896587e-06	\\
-5787.53773082386	6.91788764238639e-06	\\
-5786.55894886364	8.4820006258918e-06	\\
-5785.58016690341	8.96236939236433e-06	\\
-5784.60138494318	7.4264408564825e-06	\\
-5783.62260298296	7.14000129821797e-06	\\
-5782.64382102273	9.65329260217544e-06	\\
-5781.6650390625	6.72123050148982e-06	\\
-5780.68625710227	7.51577779610769e-06	\\
-5779.70747514205	7.26227893083497e-06	\\
-5778.72869318182	6.05855605428744e-06	\\
-5777.74991122159	7.15336044599255e-06	\\
-5776.77112926136	8.21233632880374e-06	\\
-5775.79234730114	9.60796347999967e-06	\\
-5774.81356534091	7.69014399640762e-06	\\
-5773.83478338068	6.7318787465837e-06	\\
-5772.85600142046	6.44407588429885e-06	\\
-5771.87721946023	7.20926128637086e-06	\\
-5770.8984375	5.68921624911577e-06	\\
-5769.91965553977	6.3796179235442e-06	\\
-5768.94087357955	8.5448072104836e-06	\\
-5767.96209161932	5.38537776990823e-06	\\
-5766.98330965909	6.22948802669485e-06	\\
-5766.00452769886	7.6187594389346e-06	\\
-5765.02574573864	5.12157596524673e-06	\\
-5764.04696377841	7.64159432973649e-06	\\
-5763.06818181818	6.6705132411376e-06	\\
-5762.08939985796	7.15915346381103e-06	\\
-5761.11061789773	7.59485626132942e-06	\\
-5760.1318359375	8.08327860395295e-06	\\
-5759.15305397727	7.40692763683647e-06	\\
-5758.17427201705	6.58521799657093e-06	\\
-5757.19549005682	5.96234929089903e-06	\\
-5756.21670809659	7.72262497871762e-06	\\
-5755.23792613636	8.37974153257094e-06	\\
-5754.25914417614	5.37931561114956e-06	\\
-5753.28036221591	7.81312276295781e-06	\\
-5752.30158025568	7.14470235915506e-06	\\
-5751.32279829546	8.95036609449744e-06	\\
-5750.34401633523	8.56231726023723e-06	\\
-5749.365234375	6.56367732280706e-06	\\
-5748.38645241477	5.33121588606663e-06	\\
-5747.40767045455	6.39683597981397e-06	\\
-5746.42888849432	7.48287451080451e-06	\\
-5745.45010653409	4.84699398323089e-06	\\
-5744.47132457386	7.50039538173247e-06	\\
-5743.49254261364	4.41459205443628e-06	\\
-5742.51376065341	5.02065098010585e-06	\\
-5741.53497869318	7.78628062841043e-06	\\
-5740.55619673296	5.65342297640459e-06	\\
-5739.57741477273	7.92232499234992e-06	\\
-5738.5986328125	7.30381442491134e-06	\\
-5737.61985085227	6.56375519019292e-06	\\
-5736.64106889205	8.58023472243378e-06	\\
-5735.66228693182	6.05098525220246e-06	\\
-5734.68350497159	5.8179554570477e-06	\\
-5733.70472301136	7.30253204251474e-06	\\
-5732.72594105114	7.71473189600009e-06	\\
-5731.74715909091	5.71817405357124e-06	\\
-5730.76837713068	8.00734963156176e-06	\\
-5729.78959517046	7.18798367431306e-06	\\
-5728.81081321023	5.84492832759377e-06	\\
-5727.83203125	5.69519122830688e-06	\\
-5726.85324928977	5.99298834752869e-06	\\
-5725.87446732955	6.94693082977962e-06	\\
-5724.89568536932	7.24022309251325e-06	\\
-5723.91690340909	3.28836147801541e-06	\\
-5722.93812144886	9.21229344209647e-06	\\
-5721.95933948864	7.23504847772072e-06	\\
-5720.98055752841	6.82157227854224e-06	\\
-5720.00177556818	6.84304396031522e-06	\\
-5719.02299360796	7.49282624327731e-06	\\
-5718.04421164773	6.78476638560728e-06	\\
-5717.0654296875	9.80472618720502e-06	\\
-5716.08664772727	7.03440661809105e-06	\\
-5715.10786576705	8.14398934446144e-06	\\
-5714.12908380682	9.19517430333152e-06	\\
-5713.15030184659	7.40658595670469e-06	\\
-5712.17151988636	7.54297600825087e-06	\\
-5711.19273792614	6.54423946671202e-06	\\
-5710.21395596591	6.18271663968278e-06	\\
-5709.23517400568	8.11030198616268e-06	\\
-5708.25639204546	7.33825395229834e-06	\\
-5707.27761008523	8.21461034171284e-06	\\
-5706.298828125	8.70811547770537e-06	\\
-5705.32004616477	5.6686504135105e-06	\\
-5704.34126420455	7.42351526993449e-06	\\
-5703.36248224432	7.36475713541003e-06	\\
-5702.38370028409	6.68846875475659e-06	\\
-5701.40491832386	7.31059734839621e-06	\\
-5700.42613636364	6.05253239108354e-06	\\
-5699.44735440341	9.55859613679268e-06	\\
-5698.46857244318	5.248820577781e-06	\\
-5697.48979048296	6.10775706871781e-06	\\
-5696.51100852273	7.73719045249874e-06	\\
-5695.5322265625	9.49544237348205e-06	\\
-5694.55344460227	7.99270081002528e-06	\\
-5693.57466264205	6.72239429291951e-06	\\
-5692.59588068182	6.06703796875342e-06	\\
-5691.61709872159	7.44366872814727e-06	\\
-5690.63831676136	6.64128266972902e-06	\\
-5689.65953480114	6.34352514324264e-06	\\
-5688.68075284091	7.06626028556874e-06	\\
-5687.70197088068	7.94932623325868e-06	\\
-5686.72318892046	5.12311615524387e-06	\\
-5685.74440696023	7.78885632348897e-06	\\
-5684.765625	8.62740184082476e-06	\\
-5683.78684303977	6.83860725753578e-06	\\
-5682.80806107955	6.15398805154446e-06	\\
-5681.82927911932	5.24167646716185e-06	\\
-5680.85049715909	5.12808183504838e-06	\\
-5679.87171519886	7.80355119865047e-06	\\
-5678.89293323864	6.75875313759106e-06	\\
-5677.91415127841	5.69189517612339e-06	\\
-5676.93536931818	6.66854392622281e-06	\\
-5675.95658735796	7.23600336996619e-06	\\
-5674.97780539773	6.13823592206911e-06	\\
-5673.9990234375	8.80852024320202e-06	\\
-5673.02024147727	7.79722720204006e-06	\\
-5672.04145951705	8.0902869773856e-06	\\
-5671.06267755682	9.69376567514404e-06	\\
-5670.08389559659	9.28005651919318e-06	\\
-5669.10511363636	7.18636369664249e-06	\\
-5668.12633167614	7.0533963430917e-06	\\
-5667.14754971591	6.80379572830267e-06	\\
-5666.16876775568	6.94307851717903e-06	\\
-5665.18998579546	7.11924660993444e-06	\\
-5664.21120383523	9.92867537790441e-06	\\
-5663.232421875	6.97410235981594e-06	\\
-5662.25363991477	9.39400558344155e-06	\\
-5661.27485795455	7.12212432109611e-06	\\
-5660.29607599432	6.07705800334933e-06	\\
-5659.31729403409	7.45112264382493e-06	\\
-5658.33851207386	7.36802038592535e-06	\\
-5657.35973011364	7.68738607505099e-06	\\
-5656.38094815341	7.59088826197896e-06	\\
-5655.40216619318	8.89812939407896e-06	\\
-5654.42338423296	7.34749448496465e-06	\\
-5653.44460227273	9.50397674000963e-06	\\
-5652.4658203125	7.15815809772967e-06	\\
-5651.48703835227	9.55084784448268e-06	\\
-5650.50825639205	8.08753807381241e-06	\\
-5649.52947443182	5.72251343873859e-06	\\
-5648.55069247159	7.06463270241948e-06	\\
-5647.57191051136	7.96896218790281e-06	\\
-5646.59312855114	7.80235922497764e-06	\\
-5645.61434659091	8.80721893493021e-06	\\
-5644.63556463068	7.77942812742356e-06	\\
-5643.65678267046	7.71585338807403e-06	\\
-5642.67800071023	7.03388422124048e-06	\\
-5641.69921875	9.12288986441363e-06	\\
-5640.72043678977	7.09941885884782e-06	\\
-5639.74165482955	6.34918652297286e-06	\\
-5638.76287286932	7.81661095561205e-06	\\
-5637.78409090909	8.08930832902346e-06	\\
-5636.80530894886	6.24997257485397e-06	\\
-5635.82652698864	7.66606374389693e-06	\\
-5634.84774502841	7.15301319398137e-06	\\
-5633.86896306818	7.13411787806073e-06	\\
-5632.89018110796	7.02396183701151e-06	\\
-5631.91139914773	7.10466662388755e-06	\\
-5630.9326171875	8.89382748408087e-06	\\
-5629.95383522727	7.73783981331255e-06	\\
-5628.97505326705	6.81924070835676e-06	\\
-5627.99627130682	7.14604795508384e-06	\\
-5627.01748934659	8.99662418015358e-06	\\
-5626.03870738636	7.23065567238822e-06	\\
-5625.05992542614	8.57623832254547e-06	\\
-5624.08114346591	9.54456689373816e-06	\\
-5623.10236150568	8.6824214822886e-06	\\
-5622.12357954546	9.18246219057293e-06	\\
-5621.14479758523	9.51493795334429e-06	\\
-5620.166015625	9.81384569702205e-06	\\
-5619.18723366477	8.26200406121262e-06	\\
-5618.20845170455	8.54636834177695e-06	\\
-5617.22966974432	7.81293579948754e-06	\\
-5616.25088778409	9.20806851990519e-06	\\
-5615.27210582386	7.80280915136069e-06	\\
-5614.29332386364	9.54435025422976e-06	\\
-5613.31454190341	8.47674040029166e-06	\\
-5612.33575994318	7.81547961631669e-06	\\
-5611.35697798296	9.25244724402367e-06	\\
-5610.37819602273	9.27281594268045e-06	\\
-5609.3994140625	1.13047870183638e-05	\\
-5608.42063210227	8.48779770808253e-06	\\
-5607.44185014205	8.91929899829567e-06	\\
-5606.46306818182	8.31478074754317e-06	\\
-5605.48428622159	7.3997168458223e-06	\\
-5604.50550426136	8.0423199691125e-06	\\
-5603.52672230114	8.26352331768177e-06	\\
-5602.54794034091	9.08656255579249e-06	\\
-5601.56915838068	6.29765957818814e-06	\\
-5600.59037642046	8.19542849827406e-06	\\
-5599.61159446023	8.64418626326104e-06	\\
-5598.6328125	6.414730998855e-06	\\
-5597.65403053977	7.89093577562777e-06	\\
-5596.67524857955	9.42118524834807e-06	\\
-5595.69646661932	8.28620926003911e-06	\\
-5594.71768465909	6.84577409771674e-06	\\
-5593.73890269886	8.37406016141994e-06	\\
-5592.76012073864	7.71245088462782e-06	\\
-5591.78133877841	8.3366113710208e-06	\\
-5590.80255681818	7.98752366920742e-06	\\
-5589.82377485796	8.00590104126811e-06	\\
-5588.84499289773	8.29150417567127e-06	\\
-5587.8662109375	7.86135868619664e-06	\\
-5586.88742897727	8.12192319117511e-06	\\
-5585.90864701705	9.1622506998711e-06	\\
-5584.92986505682	9.05448695458014e-06	\\
-5583.95108309659	7.26680420834436e-06	\\
-5582.97230113636	8.31956691981693e-06	\\
-5581.99351917614	6.36044884756137e-06	\\
-5581.01473721591	6.90192351927307e-06	\\
-5580.03595525568	6.45216060050777e-06	\\
-5579.05717329546	9.31141209002375e-06	\\
-5578.07839133523	6.77859241014654e-06	\\
-5577.099609375	7.63307651040654e-06	\\
-5576.12082741477	9.86073948570604e-06	\\
-5575.14204545455	7.98355488065266e-06	\\
-5574.16326349432	8.65118779217404e-06	\\
-5573.18448153409	6.84504879283887e-06	\\
-5572.20569957386	8.82442825523268e-06	\\
-5571.22691761364	7.97451698145147e-06	\\
-5570.24813565341	9.69069508146333e-06	\\
-5569.26935369318	8.8403265328971e-06	\\
-5568.29057173296	1.00003006458355e-05	\\
-5567.31178977273	8.18725274713226e-06	\\
-5566.3330078125	8.2170020881666e-06	\\
-5565.35422585227	8.45160362700437e-06	\\
-5564.37544389205	6.17927947379914e-06	\\
-5563.39666193182	6.65047912274907e-06	\\
-5562.41787997159	9.8156873257035e-06	\\
-5561.43909801136	8.26918519680624e-06	\\
-5560.46031605114	9.09076003040722e-06	\\
-5559.48153409091	7.58043719400288e-06	\\
-5558.50275213068	8.60678802391948e-06	\\
-5557.52397017046	6.78026480255864e-06	\\
-5556.54518821023	9.79789103970882e-06	\\
-5555.56640625	1.00636991495326e-05	\\
-5554.58762428977	1.03389287912508e-05	\\
-5553.60884232955	8.70584954856209e-06	\\
-5552.63006036932	8.23871881850799e-06	\\
-5551.65127840909	8.0701733184336e-06	\\
-5550.67249644886	9.078488352773e-06	\\
-5549.69371448864	7.00104253637419e-06	\\
-5548.71493252841	9.04445360965193e-06	\\
-5547.73615056818	6.59370527061258e-06	\\
-5546.75736860796	8.36835023327698e-06	\\
-5545.77858664773	8.66180230841894e-06	\\
-5544.7998046875	8.02556489975785e-06	\\
-5543.82102272727	7.25624505203638e-06	\\
-5542.84224076705	9.11319429691396e-06	\\
-5541.86345880682	7.73761200097007e-06	\\
-5540.88467684659	6.20182008820301e-06	\\
-5539.90589488636	1.08071707499693e-05	\\
-5538.92711292614	8.43137900336932e-06	\\
-5537.94833096591	7.15403506106862e-06	\\
-5536.96954900568	1.06698657811484e-05	\\
-5535.99076704546	9.49572331527772e-06	\\
-5535.01198508523	8.68943661907151e-06	\\
-5534.033203125	9.23936347664788e-06	\\
-5533.05442116477	8.44520846845817e-06	\\
-5532.07563920455	7.86593290889638e-06	\\
-5531.09685724432	7.47242723067611e-06	\\
-5530.11807528409	6.58332485348684e-06	\\
-5529.13929332386	6.07941876508506e-06	\\
-5528.16051136364	9.41571820809184e-06	\\
-5527.18172940341	8.56635728720405e-06	\\
-5526.20294744318	9.89404283770028e-06	\\
-5525.22416548296	1.07644191984448e-05	\\
-5524.24538352273	9.87843924048649e-06	\\
-5523.2666015625	8.64544108140292e-06	\\
-5522.28781960227	8.99363564481373e-06	\\
-5521.30903764205	8.69785137037465e-06	\\
-5520.33025568182	1.07928373694824e-05	\\
-5519.35147372159	8.85334129066177e-06	\\
-5518.37269176136	7.57640666823337e-06	\\
-5517.39390980114	8.19959233693752e-06	\\
-5516.41512784091	8.14536055543572e-06	\\
-5515.43634588068	9.44940741593479e-06	\\
-5514.45756392046	8.22077662526047e-06	\\
-5513.47878196023	9.80113697556251e-06	\\
-5512.5	7.75942270045401e-06	\\
-5511.52121803977	1.08475251575137e-05	\\
-5510.54243607955	8.31769244552777e-06	\\
-5509.56365411932	1.14094006508225e-05	\\
-5508.58487215909	7.52746687601049e-06	\\
-5507.60609019886	9.29463104064888e-06	\\
-5506.62730823864	9.63465537060783e-06	\\
-5505.64852627841	9.00887775323198e-06	\\
-5504.66974431818	6.22695563603441e-06	\\
-5503.69096235796	9.79320457978944e-06	\\
-5502.71218039773	6.5713297045241e-06	\\
-5501.7333984375	9.04637093697869e-06	\\
-5500.75461647727	9.16910815452703e-06	\\
-5499.77583451705	7.8315560974154e-06	\\
-5498.79705255682	9.19411982989202e-06	\\
-5497.81827059659	8.77770342024246e-06	\\
-5496.83948863636	9.40121240721641e-06	\\
-5495.86070667614	1.02822476134776e-05	\\
-5494.88192471591	7.89288522242321e-06	\\
-5493.90314275568	8.08164636827185e-06	\\
-5492.92436079546	8.65399116996356e-06	\\
-5491.94557883523	8.71349756613204e-06	\\
-5490.966796875	7.78861213923931e-06	\\
-5489.98801491477	1.02063759884717e-05	\\
-5489.00923295455	9.28890193641709e-06	\\
-5488.03045099432	8.11159763799632e-06	\\
-5487.05166903409	9.20208262898142e-06	\\
-5486.07288707386	8.63330276575073e-06	\\
-5485.09410511364	9.14276682674347e-06	\\
-5484.11532315341	1.031268200291e-05	\\
-5483.13654119318	9.04063635021089e-06	\\
-5482.15775923296	8.06859871233594e-06	\\
-5481.17897727273	7.47108663558744e-06	\\
-5480.2001953125	9.23664340868887e-06	\\
-5479.22141335227	7.42822793238309e-06	\\
-5478.24263139205	9.1825682927204e-06	\\
-5477.26384943182	1.11014769993393e-05	\\
-5476.28506747159	1.01426264265235e-05	\\
-5475.30628551136	9.29208868295674e-06	\\
-5474.32750355114	9.59917316117451e-06	\\
-5473.34872159091	9.49072105924636e-06	\\
-5472.36993963068	8.50416208427903e-06	\\
-5471.39115767046	1.12639416900065e-05	\\
-5470.41237571023	9.41275990503388e-06	\\
-5469.43359375	1.06156610052932e-05	\\
-5468.45481178977	9.02020980424895e-06	\\
-5467.47602982955	8.45974125528905e-06	\\
-5466.49724786932	1.00428596896594e-05	\\
-5465.51846590909	7.6735345856728e-06	\\
-5464.53968394886	9.58248933549989e-06	\\
-5463.56090198864	9.02727389502509e-06	\\
-5462.58212002841	7.23741946713068e-06	\\
-5461.60333806818	7.77030203512502e-06	\\
-5460.62455610796	9.31639241232702e-06	\\
-5459.64577414773	8.57911026681932e-06	\\
-5458.6669921875	9.99325160543463e-06	\\
-5457.68821022727	7.58714675579026e-06	\\
-5456.70942826705	1.06080317724758e-05	\\
-5455.73064630682	1.13789594490745e-05	\\
-5454.75186434659	1.02236121252297e-05	\\
-5453.77308238636	9.35933129311176e-06	\\
-5452.79430042614	9.67531014124929e-06	\\
-5451.81551846591	8.82603215550732e-06	\\
-5450.83673650568	8.95909802739424e-06	\\
-5449.85795454546	9.38925579469578e-06	\\
-5448.87917258523	8.8742353070695e-06	\\
-5447.900390625	7.68396887927032e-06	\\
-5446.92160866477	9.17090271255702e-06	\\
-5445.94282670455	8.21612184377132e-06	\\
-5444.96404474432	9.64209920242776e-06	\\
-5443.98526278409	8.23069582401685e-06	\\
-5443.00648082386	9.87064741833496e-06	\\
-5442.02769886364	9.29465417939093e-06	\\
-5441.04891690341	9.19491059133349e-06	\\
-5440.07013494318	8.83872661976548e-06	\\
-5439.09135298296	1.19807984298201e-05	\\
-5438.11257102273	9.56269435612711e-06	\\
-5437.1337890625	1.03537510137318e-05	\\
-5436.15500710227	7.55603901143683e-06	\\
-5435.17622514205	1.07037867599477e-05	\\
-5434.19744318182	8.4756503150417e-06	\\
-5433.21866122159	1.06918034777652e-05	\\
-5432.23987926136	7.93425826285661e-06	\\
-5431.26109730114	8.46511876967273e-06	\\
-5430.28231534091	8.85957612717574e-06	\\
-5429.30353338068	7.34581798921106e-06	\\
-5428.32475142046	8.77704825201701e-06	\\
-5427.34596946023	7.76608586724797e-06	\\
-5426.3671875	8.87146754114212e-06	\\
-5425.38840553977	9.76966787317158e-06	\\
-5424.40962357955	7.93546619021292e-06	\\
-5423.43084161932	8.43358550629655e-06	\\
-5422.45205965909	9.24495451965917e-06	\\
-5421.47327769886	8.41506770315747e-06	\\
-5420.49449573864	1.10394760911994e-05	\\
-5419.51571377841	1.00820148505994e-05	\\
-5418.53693181818	8.1110769895159e-06	\\
-5417.55814985796	9.9626245453791e-06	\\
-5416.57936789773	9.03476700189894e-06	\\
-5415.6005859375	1.02265912041731e-05	\\
-5414.62180397727	8.14842905044829e-06	\\
-5413.64302201705	8.496859975637e-06	\\
-5412.66424005682	9.04744004755447e-06	\\
-5411.68545809659	8.85126299861911e-06	\\
-5410.70667613636	9.48974404060667e-06	\\
-5409.72789417614	7.86504973787992e-06	\\
-5408.74911221591	8.81401250772707e-06	\\
-5407.77033025568	1.02467696260799e-05	\\
-5406.79154829546	1.04106657290248e-05	\\
-5405.81276633523	8.15706454596066e-06	\\
-5404.833984375	7.95173646582813e-06	\\
-5403.85520241477	8.97608191295236e-06	\\
-5402.87642045455	7.96185071801296e-06	\\
-5401.89763849432	9.52524860266844e-06	\\
-5400.91885653409	1.14757906704561e-05	\\
-5399.94007457386	8.32170493214466e-06	\\
-5398.96129261364	9.8121903465167e-06	\\
-5397.98251065341	7.8697748354563e-06	\\
-5397.00372869318	9.97325213410763e-06	\\
-5396.02494673296	7.1992525292381e-06	\\
-5395.04616477273	9.36798200331122e-06	\\
-5394.0673828125	8.12808857707864e-06	\\
-5393.08860085227	8.35981084719891e-06	\\
-5392.10981889205	8.80863148965092e-06	\\
-5391.13103693182	1.11107720745187e-05	\\
-5390.15225497159	8.84436619457893e-06	\\
-5389.17347301136	9.05526998683944e-06	\\
-5388.19469105114	8.87094971849551e-06	\\
-5387.21590909091	8.5542188597003e-06	\\
-5386.23712713068	8.74725605523066e-06	\\
-5385.25834517046	1.01695459351716e-05	\\
-5384.27956321023	9.15453063430921e-06	\\
-5383.30078125	9.10170447475577e-06	\\
-5382.32199928977	7.66303304977202e-06	\\
-5381.34321732955	1.03448732349504e-05	\\
-5380.36443536932	8.47140673561575e-06	\\
-5379.38565340909	8.82036359758389e-06	\\
-5378.40687144886	8.44882028484989e-06	\\
-5377.42808948864	9.13575875280591e-06	\\
-5376.44930752841	7.6340816224029e-06	\\
-5375.47052556818	8.12241303153375e-06	\\
-5374.49174360796	8.40728534512886e-06	\\
-5373.51296164773	1.11437327306681e-05	\\
-5372.5341796875	1.00422895874485e-05	\\
-5371.55539772727	9.89738547634236e-06	\\
-5370.57661576705	1.00714599959039e-05	\\
-5369.59783380682	9.62539652163878e-06	\\
-5368.61905184659	7.11778865964146e-06	\\
-5367.64026988636	9.10229509126717e-06	\\
-5366.66148792614	9.64078726755401e-06	\\
-5365.68270596591	7.82609171916827e-06	\\
-5364.70392400568	9.09105883762514e-06	\\
-5363.72514204546	8.14088890589276e-06	\\
-5362.74636008523	9.16479833162918e-06	\\
-5361.767578125	7.87645475393572e-06	\\
-5360.78879616477	6.61756364512733e-06	\\
-5359.81001420455	1.03311915742875e-05	\\
-5358.83123224432	7.80065981895321e-06	\\
-5357.85245028409	8.16243661854647e-06	\\
-5356.87366832386	9.52158121763384e-06	\\
-5355.89488636364	7.90013211971153e-06	\\
-5354.91610440341	8.51355720589259e-06	\\
-5353.93732244318	8.20293178670255e-06	\\
-5352.95854048296	8.07721961416958e-06	\\
-5351.97975852273	7.6191619143393e-06	\\
-5351.0009765625	6.71894436828662e-06	\\
-5350.02219460227	8.08432604576649e-06	\\
-5349.04341264205	7.76368834492415e-06	\\
-5348.06463068182	9.32194097087498e-06	\\
-5347.08584872159	5.88432648517859e-06	\\
-5346.10706676136	9.67849944785975e-06	\\
-5345.12828480114	9.7112493171124e-06	\\
-5344.14950284091	7.11212387540659e-06	\\
-5343.17072088068	7.82508512626973e-06	\\
-5342.19193892046	1.04167880631718e-05	\\
-5341.21315696023	9.11296756747923e-06	\\
-5340.234375	7.20437881209281e-06	\\
-5339.25559303977	6.90838803351859e-06	\\
-5338.27681107955	9.74131815101024e-06	\\
-5337.29802911932	6.22684722422725e-06	\\
-5336.31924715909	1.09290059649985e-05	\\
-5335.34046519886	9.27931367052755e-06	\\
-5334.36168323864	8.13102510799038e-06	\\
-5333.38290127841	9.54728584565103e-06	\\
-5332.40411931818	9.76991786357191e-06	\\
-5331.42533735796	8.5756781562037e-06	\\
-5330.44655539773	8.6101651271752e-06	\\
-5329.4677734375	8.29368185752396e-06	\\
-5328.48899147727	7.84761332079441e-06	\\
-5327.51020951705	6.91054493265729e-06	\\
-5326.53142755682	7.00722838146655e-06	\\
-5325.55264559659	8.88653077854695e-06	\\
-5324.57386363636	9.27322145780399e-06	\\
-5323.59508167614	6.72906746928045e-06	\\
-5322.61629971591	8.56046743503807e-06	\\
-5321.63751775568	7.64141644926917e-06	\\
-5320.65873579546	8.17595052761708e-06	\\
-5319.67995383523	7.99910949082449e-06	\\
-5318.701171875	9.49524270642114e-06	\\
-5317.72238991477	8.0782752501772e-06	\\
-5316.74360795455	6.57173677037214e-06	\\
-5315.76482599432	7.46162387711074e-06	\\
-5314.78604403409	8.20645126795558e-06	\\
-5313.80726207386	8.91534749951764e-06	\\
-5312.82848011364	7.63869908877315e-06	\\
-5311.84969815341	9.44569941141897e-06	\\
-5310.87091619318	8.89544288357068e-06	\\
-5309.89213423296	7.53664446403221e-06	\\
-5308.91335227273	7.60713976776061e-06	\\
-5307.9345703125	9.99208509846551e-06	\\
-5306.95578835227	8.60596260559423e-06	\\
-5305.97700639205	8.60820227199834e-06	\\
-5304.99822443182	8.29095760064946e-06	\\
-5304.01944247159	7.59391264577895e-06	\\
-5303.04066051136	7.25493199933045e-06	\\
-5302.06187855114	9.2699165906606e-06	\\
-5301.08309659091	9.07115139308193e-06	\\
-5300.10431463068	8.90543552678781e-06	\\
-5299.12553267046	9.40805228635244e-06	\\
-5298.14675071023	7.80096497321993e-06	\\
-5297.16796875	8.61707983147454e-06	\\
-5296.18918678977	9.10953801860528e-06	\\
-5295.21040482955	7.522915999259e-06	\\
-5294.23162286932	8.48446990037615e-06	\\
-5293.25284090909	8.7610542700465e-06	\\
-5292.27405894886	7.66912905992299e-06	\\
-5291.29527698864	5.99834983608545e-06	\\
-5290.31649502841	9.74684560558207e-06	\\
-5289.33771306818	9.58031035990339e-06	\\
-5288.35893110796	9.42504430108827e-06	\\
-5287.38014914773	8.76252454925919e-06	\\
-5286.4013671875	7.93195883434156e-06	\\
-5285.42258522727	8.74774649682182e-06	\\
-5284.44380326705	8.35083440997113e-06	\\
-5283.46502130682	8.18434510042745e-06	\\
-5282.48623934659	7.99616797467859e-06	\\
-5281.50745738636	7.32370742115438e-06	\\
-5280.52867542614	7.9909028772505e-06	\\
-5279.54989346591	8.34519103620906e-06	\\
-5278.57111150568	9.67043116086462e-06	\\
-5277.59232954546	8.62548715207373e-06	\\
-5276.61354758523	7.30347624565518e-06	\\
-5275.634765625	8.47167219283816e-06	\\
-5274.65598366477	8.77896123945593e-06	\\
-5273.67720170455	8.46919352897768e-06	\\
-5272.69841974432	7.25489097125691e-06	\\
-5271.71963778409	8.59862151510115e-06	\\
-5270.74085582386	8.07557793351492e-06	\\
-5269.76207386364	8.75162979636575e-06	\\
-5268.78329190341	8.65082356817021e-06	\\
-5267.80450994318	7.71884673960921e-06	\\
-5266.82572798296	8.82505195381551e-06	\\
-5265.84694602273	5.9138624910447e-06	\\
-5264.8681640625	8.80582104456351e-06	\\
-5263.88938210227	7.63416837030354e-06	\\
-5262.91060014205	9.65643802343528e-06	\\
-5261.93181818182	7.4924317062673e-06	\\
-5260.95303622159	7.37907102521138e-06	\\
-5259.97425426136	5.67428016007129e-06	\\
-5258.99547230114	7.57133681739243e-06	\\
-5258.01669034091	8.51818935705612e-06	\\
-5257.03790838068	8.30259949039176e-06	\\
-5256.05912642046	8.2438140966367e-06	\\
-5255.08034446023	1.11372455224845e-05	\\
-5254.1015625	7.20239739426943e-06	\\
-5253.12278053977	8.94889793387531e-06	\\
-5252.14399857955	8.67194945403376e-06	\\
-5251.16521661932	8.94254947097428e-06	\\
-5250.18643465909	7.71114567778052e-06	\\
-5249.20765269886	9.79735903078226e-06	\\
-5248.22887073864	7.87915361246961e-06	\\
-5247.25008877841	8.63243713828865e-06	\\
-5246.27130681818	9.97333227636139e-06	\\
-5245.29252485796	6.88392362381353e-06	\\
-5244.31374289773	7.33251860899111e-06	\\
-5243.3349609375	6.55671556877101e-06	\\
-5242.35617897727	8.9276005872219e-06	\\
-5241.37739701705	1.00324748321117e-05	\\
-5240.39861505682	9.08987948239351e-06	\\
-5239.41983309659	1.03348052791812e-05	\\
-5238.44105113636	6.57879471438677e-06	\\
-5237.46226917614	1.05313931937313e-05	\\
-5236.48348721591	7.49810646063605e-06	\\
-5235.50470525568	7.71667790722711e-06	\\
-5234.52592329546	8.23171021146475e-06	\\
-5233.54714133523	8.36146766524242e-06	\\
-5232.568359375	9.69848292162057e-06	\\
-5231.58957741477	6.64072982877077e-06	\\
-5230.61079545455	8.7146992802948e-06	\\
-5229.63201349432	7.83840031078195e-06	\\
-5228.65323153409	7.08076194751497e-06	\\
-5227.67444957386	6.16669517410933e-06	\\
-5226.69566761364	8.58617669204845e-06	\\
-5225.71688565341	7.59557908739271e-06	\\
-5224.73810369318	5.53271803540207e-06	\\
-5223.75932173296	7.6069645698206e-06	\\
-5222.78053977273	6.86422887421272e-06	\\
-5221.8017578125	6.87564519831841e-06	\\
-5220.82297585227	8.60210225797341e-06	\\
-5219.84419389205	6.98318344755452e-06	\\
-5218.86541193182	9.29435178389021e-06	\\
-5217.88662997159	7.84438182207977e-06	\\
-5216.90784801136	1.06676120818242e-05	\\
-5215.92906605114	8.31918392959238e-06	\\
-5214.95028409091	7.64760756370114e-06	\\
-5213.97150213068	8.24643341837701e-06	\\
-5212.99272017046	8.00683232749394e-06	\\
-5212.01393821023	7.46542130630605e-06	\\
-5211.03515625	6.77751623282918e-06	\\
-5210.05637428977	6.59622782501871e-06	\\
-5209.07759232955	9.43807624941167e-06	\\
-5208.09881036932	7.99281492283255e-06	\\
-5207.12002840909	8.02495513820324e-06	\\
-5206.14124644886	7.99620658824171e-06	\\
-5205.16246448864	8.24913967082515e-06	\\
-5204.18368252841	7.44140391036538e-06	\\
-5203.20490056818	6.88867783368647e-06	\\
-5202.22611860796	8.14922741489244e-06	\\
-5201.24733664773	8.94553856067469e-06	\\
-5200.2685546875	6.8183547342159e-06	\\
-5199.28977272727	7.11093980302971e-06	\\
-5198.31099076705	8.43089154920028e-06	\\
-5197.33220880682	6.83495078962145e-06	\\
-5196.35342684659	8.71271010066179e-06	\\
-5195.37464488636	9.46666839989384e-06	\\
-5194.39586292614	6.04829712331949e-06	\\
-5193.41708096591	7.99847461496359e-06	\\
-5192.43829900568	8.35135781305735e-06	\\
-5191.45951704546	9.37632518599404e-06	\\
-5190.48073508523	6.05339145538711e-06	\\
-5189.501953125	5.67220818935436e-06	\\
-5188.52317116477	8.78673655478624e-06	\\
-5187.54438920455	6.55202347844403e-06	\\
-5186.56560724432	8.45373363955598e-06	\\
-5185.58682528409	7.92641283091972e-06	\\
-5184.60804332386	6.4740294085731e-06	\\
-5183.62926136364	6.9776192086247e-06	\\
-5182.65047940341	8.52176875377532e-06	\\
-5181.67169744318	7.09897347227096e-06	\\
-5180.69291548296	8.06774577596298e-06	\\
-5179.71413352273	8.84632659642221e-06	\\
-5178.7353515625	6.99038800669629e-06	\\
-5177.75656960227	9.37498750464157e-06	\\
-5176.77778764205	6.41045120205035e-06	\\
-5175.79900568182	8.80825246974273e-06	\\
-5174.82022372159	1.02168275402686e-05	\\
-5173.84144176136	8.05324434823216e-06	\\
-5172.86265980114	7.86978576798707e-06	\\
-5171.88387784091	8.60148369860295e-06	\\
-5170.90509588068	8.0803418898787e-06	\\
-5169.92631392046	7.5766181735827e-06	\\
-5168.94753196023	8.92988565423054e-06	\\
-5167.96875	6.5459249954962e-06	\\
-5166.98996803977	6.95200675660646e-06	\\
-5166.01118607955	9.87714616734625e-06	\\
-5165.03240411932	6.09919946059081e-06	\\
-5164.05362215909	5.72635243792534e-06	\\
-5163.07484019886	7.41239127118088e-06	\\
-5162.09605823864	8.51538758082722e-06	\\
-5161.11727627841	8.5017625587375e-06	\\
-5160.13849431818	7.1617213109117e-06	\\
-5159.15971235796	8.21838534217875e-06	\\
-5158.18093039773	9.47464030437439e-06	\\
-5157.2021484375	7.02045608197433e-06	\\
-5156.22336647727	7.31963648839347e-06	\\
-5155.24458451705	7.14345469158154e-06	\\
-5154.26580255682	6.86408835868438e-06	\\
-5153.28702059659	6.78567828042357e-06	\\
-5152.30823863636	8.18790112091477e-06	\\
-5151.32945667614	6.97348365791416e-06	\\
-5150.35067471591	8.07359825019843e-06	\\
-5149.37189275568	4.43476659273457e-06	\\
-5148.39311079546	7.84399409593909e-06	\\
-5147.41432883523	6.51308206667024e-06	\\
-5146.435546875	7.68688466916858e-06	\\
-5145.45676491477	6.35042367280139e-06	\\
-5144.47798295455	7.72139675224822e-06	\\
-5143.49920099432	7.60520593581658e-06	\\
-5142.52041903409	7.50333038995521e-06	\\
-5141.54163707386	7.28187671748626e-06	\\
-5140.56285511364	9.80369694808362e-06	\\
-5139.58407315341	8.79058918413729e-06	\\
-5138.60529119318	6.13305184830366e-06	\\
-5137.62650923296	8.64183368207665e-06	\\
-5136.64772727273	5.59327440885298e-06	\\
-5135.6689453125	7.1146102525537e-06	\\
-5134.69016335227	6.32478230512791e-06	\\
-5133.71138139205	5.98073789588582e-06	\\
-5132.73259943182	7.49746766815215e-06	\\
-5131.75381747159	5.95583233800254e-06	\\
-5130.77503551136	8.16761735781307e-06	\\
-5129.79625355114	8.17915695444466e-06	\\
-5128.81747159091	6.77047462111056e-06	\\
-5127.83868963068	8.69572910704127e-06	\\
-5126.85990767046	9.26160526304607e-06	\\
-5125.88112571023	6.2092862540075e-06	\\
-5124.90234375	8.94006739102329e-06	\\
-5123.92356178977	7.16473259999578e-06	\\
-5122.94477982955	6.64674334669412e-06	\\
-5121.96599786932	7.14199286840638e-06	\\
-5120.98721590909	8.83274926577621e-06	\\
-5120.00843394886	6.73433451811126e-06	\\
-5119.02965198864	7.0491092606139e-06	\\
-5118.05087002841	9.48950924828005e-06	\\
-5117.07208806818	7.16385761578261e-06	\\
-5116.09330610796	7.06025236837951e-06	\\
-5115.11452414773	7.29726772774885e-06	\\
-5114.1357421875	7.43864868677814e-06	\\
-5113.15696022727	9.05208325306562e-06	\\
-5112.17817826705	7.99515472786609e-06	\\
-5111.19939630682	8.90270270366266e-06	\\
-5110.22061434659	8.25747413591213e-06	\\
-5109.24183238636	9.22171325588871e-06	\\
-5108.26305042614	7.85139432437269e-06	\\
-5107.28426846591	8.18730056832272e-06	\\
-5106.30548650568	6.05417493909132e-06	\\
-5105.32670454546	8.90325897914384e-06	\\
-5104.34792258523	7.13659909296607e-06	\\
-5103.369140625	6.0644740569963e-06	\\
-5102.39035866477	8.69678361476378e-06	\\
-5101.41157670455	5.85096325103247e-06	\\
-5100.43279474432	7.72494596911493e-06	\\
-5099.45401278409	6.57678026531586e-06	\\
-5098.47523082386	7.10195978625448e-06	\\
-5097.49644886364	6.98826317850596e-06	\\
-5096.51766690341	6.18978060956487e-06	\\
-5095.53888494318	7.88447945329059e-06	\\
-5094.56010298296	6.58433300361638e-06	\\
-5093.58132102273	8.48025989415704e-06	\\
-5092.6025390625	6.68227616754616e-06	\\
-5091.62375710227	1.03277845074154e-05	\\
-5090.64497514205	6.7557733834411e-06	\\
-5089.66619318182	7.45275229593342e-06	\\
-5088.68741122159	8.02661433077731e-06	\\
-5087.70862926136	7.44057148310603e-06	\\
-5086.72984730114	8.00349087543183e-06	\\
-5085.75106534091	7.52003221867027e-06	\\
-5084.77228338068	5.83904210652652e-06	\\
-5083.79350142046	7.16544175152532e-06	\\
-5082.81471946023	6.38605688721591e-06	\\
-5081.8359375	9.70527654593624e-06	\\
-5080.85715553977	9.76321272514862e-06	\\
-5079.87837357955	8.79439725220502e-06	\\
-5078.89959161932	7.14030000551392e-06	\\
-5077.92080965909	7.4096739153834e-06	\\
-5076.94202769886	7.40955309534496e-06	\\
-5075.96324573864	8.05924817966051e-06	\\
-5074.98446377841	5.61540696061174e-06	\\
-5074.00568181818	7.03297980644123e-06	\\
-5073.02689985796	7.5326845496402e-06	\\
-5072.04811789773	7.30953576752008e-06	\\
-5071.0693359375	6.85361817584858e-06	\\
-5070.09055397727	6.25644615365221e-06	\\
-5069.11177201705	6.36949424583164e-06	\\
-5068.13299005682	7.61260678609996e-06	\\
-5067.15420809659	7.13609018277036e-06	\\
-5066.17542613636	5.6294083965258e-06	\\
-5065.19664417614	7.06119664809912e-06	\\
-5064.21786221591	8.99376818428724e-06	\\
-5063.23908025568	9.43833544133005e-06	\\
-5062.26029829546	7.62120519216944e-06	\\
-5061.28151633523	7.45379944114043e-06	\\
-5060.302734375	5.90455848837134e-06	\\
-5059.32395241477	6.92305948635725e-06	\\
-5058.34517045455	5.85478942389507e-06	\\
-5057.36638849432	6.25448742077713e-06	\\
-5056.38760653409	6.48833276295214e-06	\\
-5055.40882457386	5.28469280211515e-06	\\
-5054.43004261364	6.43645103031238e-06	\\
-5053.45126065341	7.86761999039693e-06	\\
-5052.47247869318	6.39894943516352e-06	\\
-5051.49369673296	7.36705850485745e-06	\\
-5050.51491477273	7.15759065670867e-06	\\
-5049.5361328125	8.10682395156649e-06	\\
-5048.55735085227	6.32322270928378e-06	\\
-5047.57856889205	7.00104821801918e-06	\\
-5046.59978693182	7.58175182386609e-06	\\
-5045.62100497159	6.4349641808221e-06	\\
-5044.64222301136	7.38602215510775e-06	\\
-5043.66344105114	8.72035998070684e-06	\\
-5042.68465909091	9.10714927031413e-06	\\
-5041.70587713068	7.27890315388062e-06	\\
-5040.72709517046	6.43717456176772e-06	\\
-5039.74831321023	8.08675681199535e-06	\\
-5038.76953125	9.36999841682469e-06	\\
-5037.79074928977	7.0976606210466e-06	\\
-5036.81196732955	8.37707922416941e-06	\\
-5035.83318536932	7.13651794708593e-06	\\
-5034.85440340909	7.79498847021578e-06	\\
-5033.87562144886	5.43072317442514e-06	\\
-5032.89683948864	4.99525580434888e-06	\\
-5031.91805752841	8.69046804805974e-06	\\
-5030.93927556818	7.98093954096452e-06	\\
-5029.96049360796	6.12363618444592e-06	\\
-5028.98171164773	8.7260499621603e-06	\\
-5028.0029296875	5.81302101507151e-06	\\
-5027.02414772727	6.48527354518384e-06	\\
-5026.04536576705	8.15931209834887e-06	\\
-5025.06658380682	8.4125922907876e-06	\\
-5024.08780184659	8.53597644343846e-06	\\
-5023.10901988636	6.18697721002216e-06	\\
-5022.13023792614	8.94250121856011e-06	\\
-5021.15145596591	6.35672999970194e-06	\\
-5020.17267400568	6.6221740768135e-06	\\
-5019.19389204546	8.62618077641628e-06	\\
-5018.21511008523	6.38741520775821e-06	\\
-5017.236328125	6.26930068725712e-06	\\
-5016.25754616477	6.44527688078739e-06	\\
-5015.27876420455	7.63010698836545e-06	\\
-5014.29998224432	6.44973714048103e-06	\\
-5013.32120028409	8.44501335984426e-06	\\
-5012.34241832386	5.73301825160849e-06	\\
-5011.36363636364	7.30881903908967e-06	\\
-5010.38485440341	7.71384378444748e-06	\\
-5009.40607244318	8.45416791743408e-06	\\
-5008.42729048296	7.04178643914382e-06	\\
-5007.44850852273	8.03949508358043e-06	\\
-5006.4697265625	6.82258678879628e-06	\\
-5005.49094460227	8.7598201737315e-06	\\
-5004.51216264205	8.39072490937106e-06	\\
-5003.53338068182	9.45883058408736e-06	\\
-5002.55459872159	6.96271197114398e-06	\\
-5001.57581676136	6.99246664809339e-06	\\
-5000.59703480114	6.78942588010952e-06	\\
-4999.61825284091	7.75487304096141e-06	\\
-4998.63947088068	7.60021169317726e-06	\\
-4997.66068892046	9.05157468224637e-06	\\
-4996.68190696023	7.41928803340327e-06	\\
-4995.703125	6.99476151549582e-06	\\
-4994.72434303977	7.65543473544417e-06	\\
-4993.74556107955	6.75951569541622e-06	\\
-4992.76677911932	7.02031502049983e-06	\\
-4991.78799715909	6.65329071920974e-06	\\
-4990.80921519886	7.00611685284259e-06	\\
-4989.83043323864	7.59405938986297e-06	\\
-4988.85165127841	7.50450920888504e-06	\\
-4987.87286931818	8.35615572151019e-06	\\
-4986.89408735796	6.46067491723915e-06	\\
-4985.91530539773	7.07926841440945e-06	\\
-4984.9365234375	8.11396251052671e-06	\\
-4983.95774147727	5.80869309359245e-06	\\
-4982.97895951705	8.89464623297542e-06	\\
-4982.00017755682	9.5958711165143e-06	\\
-4981.02139559659	8.18796424239227e-06	\\
-4980.04261363636	6.36913798696135e-06	\\
-4979.06383167614	7.35074413824726e-06	\\
-4978.08504971591	6.70967265740833e-06	\\
-4977.10626775568	6.11887215601011e-06	\\
-4976.12748579546	9.57152728174064e-06	\\
-4975.14870383523	6.94351122969324e-06	\\
-4974.169921875	6.83579874401634e-06	\\
-4973.19113991477	7.72307267066434e-06	\\
-4972.21235795455	9.36222436421921e-06	\\
-4971.23357599432	7.41551417493428e-06	\\
-4970.25479403409	8.90195560440437e-06	\\
-4969.27601207386	8.81617687465272e-06	\\
-4968.29723011364	1.00697878709285e-05	\\
-4967.31844815341	7.48137441927013e-06	\\
-4966.33966619318	6.45262017300932e-06	\\
-4965.36088423296	7.67370375371776e-06	\\
-4964.38210227273	4.86200681376202e-06	\\
-4963.4033203125	7.32931854565119e-06	\\
-4962.42453835227	7.07356559429033e-06	\\
-4961.44575639205	7.8553053842271e-06	\\
-4960.46697443182	6.28398523171834e-06	\\
-4959.48819247159	9.16433184968245e-06	\\
-4958.50941051136	9.54625057219927e-06	\\
-4957.53062855114	8.67518858016521e-06	\\
-4956.55184659091	9.19785303142549e-06	\\
-4955.57306463068	9.83760715296547e-06	\\
-4954.59428267046	9.19664868362562e-06	\\
-4953.61550071023	8.05940572143585e-06	\\
-4952.63671875	7.87331167247765e-06	\\
-4951.65793678977	8.46596689857982e-06	\\
-4950.67915482955	7.21098664852924e-06	\\
-4949.70037286932	5.87922198131815e-06	\\
-4948.72159090909	8.72074610505896e-06	\\
-4947.74280894886	9.64711697494598e-06	\\
-4946.76402698864	9.91858282803304e-06	\\
-4945.78524502841	8.20730692507965e-06	\\
-4944.80646306818	6.2848032898334e-06	\\
-4943.82768110796	1.04136860826686e-05	\\
-4942.84889914773	8.1116978200214e-06	\\
-4941.8701171875	8.94885812544027e-06	\\
-4940.89133522727	9.44889642615394e-06	\\
-4939.91255326705	8.16457186308285e-06	\\
-4938.93377130682	9.86838752617485e-06	\\
-4937.95498934659	9.81083246565566e-06	\\
-4936.97620738636	9.63929153944974e-06	\\
-4935.99742542614	8.28473723427957e-06	\\
-4935.01864346591	9.66778584066736e-06	\\
-4934.03986150568	7.51002514953506e-06	\\
-4933.06107954546	9.34100345639326e-06	\\
-4932.08229758523	1.08276996859015e-05	\\
-4931.103515625	7.88436123924725e-06	\\
-4930.12473366477	7.57232487659181e-06	\\
-4929.14595170455	9.88879828959616e-06	\\
-4928.16716974432	9.89872825326578e-06	\\
-4927.18838778409	6.77983082506897e-06	\\
-4926.20960582386	9.73329659527453e-06	\\
-4925.23082386364	7.86588333379331e-06	\\
-4924.25204190341	9.8740123996761e-06	\\
-4923.27325994318	8.42558072749034e-06	\\
-4922.29447798296	9.43154432989536e-06	\\
-4921.31569602273	9.4237185818296e-06	\\
-4920.3369140625	8.27392188096918e-06	\\
-4919.35813210227	9.7563198315933e-06	\\
-4918.37935014205	9.38115206518869e-06	\\
-4917.40056818182	8.52945075248364e-06	\\
-4916.42178622159	7.69121067746188e-06	\\
-4915.44300426136	1.01335341425838e-05	\\
-4914.46422230114	1.02350403350423e-05	\\
-4913.48544034091	9.9699619130871e-06	\\
-4912.50665838068	1.03935952104458e-05	\\
-4911.52787642046	1.02450587109124e-05	\\
-4910.54909446023	9.35459196422183e-06	\\
-4909.5703125	1.0095013556751e-05	\\
-4908.59153053977	9.36291469743398e-06	\\
-4907.61274857955	9.58012163020546e-06	\\
-4906.63396661932	1.00429700202881e-05	\\
-4905.65518465909	7.46455022890146e-06	\\
-4904.67640269886	9.26701815150679e-06	\\
-4903.69762073864	9.38754994361692e-06	\\
-4902.71883877841	9.52140483424217e-06	\\
-4901.74005681818	9.48222589878468e-06	\\
-4900.76127485796	1.21722497827136e-05	\\
-4899.78249289773	9.08058913816619e-06	\\
-4898.8037109375	1.16347398398049e-05	\\
-4897.82492897727	9.85573410254756e-06	\\
-4896.84614701705	9.47719813960598e-06	\\
-4895.86736505682	1.02311044769428e-05	\\
-4894.88858309659	9.52660526143369e-06	\\
-4893.90980113636	1.19824365057878e-05	\\
-4892.93101917614	1.12637930066908e-05	\\
-4891.95223721591	1.24706027936533e-05	\\
-4890.97345525568	7.76899831967786e-06	\\
-4889.99467329546	1.07396655954383e-05	\\
-4889.01589133523	9.35283717434455e-06	\\
-4888.037109375	9.33366949619457e-06	\\
-4887.05832741477	1.06439957297945e-05	\\
-4886.07954545455	9.45919155856541e-06	\\
-4885.10076349432	1.14396237600639e-05	\\
-4884.12198153409	1.13336179436171e-05	\\
-4883.14319957386	9.1954623648048e-06	\\
-4882.16441761364	1.10003290896562e-05	\\
-4881.18563565341	1.08835298526768e-05	\\
-4880.20685369318	1.06873483915012e-05	\\
-4879.22807173296	8.5869393268662e-06	\\
-4878.24928977273	1.19095612715543e-05	\\
-4877.2705078125	1.11935094566775e-05	\\
-4876.29172585227	1.00056392677533e-05	\\
-4875.31294389205	1.2279658487588e-05	\\
-4874.33416193182	1.26112268998373e-05	\\
-4873.35537997159	1.12042320581002e-05	\\
-4872.37659801136	1.1513386507245e-05	\\
-4871.39781605114	9.81048954618568e-06	\\
-4870.41903409091	1.06500703332869e-05	\\
-4869.44025213068	9.04190417707405e-06	\\
-4868.46147017046	1.03532833207527e-05	\\
-4867.48268821023	9.66340036402348e-06	\\
-4866.50390625	8.5635309014844e-06	\\
-4865.52512428977	1.13795554704073e-05	\\
-4864.54634232955	1.14976834707083e-05	\\
-4863.56756036932	1.09565641848061e-05	\\
-4862.58877840909	1.27803617761564e-05	\\
-4861.60999644886	1.27682048825806e-05	\\
-4860.63121448864	9.12490803927217e-06	\\
-4859.65243252841	8.98889117447824e-06	\\
-4858.67365056818	1.13564935000181e-05	\\
-4857.69486860796	1.01617211105331e-05	\\
-4856.71608664773	1.23970564336567e-05	\\
-4855.7373046875	8.14805620117427e-06	\\
-4854.75852272727	8.92946961025094e-06	\\
-4853.77974076705	1.13547060984898e-05	\\
-4852.80095880682	1.26945590594675e-05	\\
-4851.82217684659	1.04500739254578e-05	\\
-4850.84339488636	9.10054843246285e-06	\\
-4849.86461292614	1.38807895047779e-05	\\
-4848.88583096591	1.20064387624106e-05	\\
-4847.90704900568	9.7865178250415e-06	\\
-4846.92826704546	1.03519540294863e-05	\\
-4845.94948508523	9.91231286861612e-06	\\
-4844.970703125	1.06879858481987e-05	\\
-4843.99192116477	1.15153771234104e-05	\\
-4843.01313920455	1.20307209851029e-05	\\
-4842.03435724432	1.03669055016429e-05	\\
-4841.05557528409	1.12346130225655e-05	\\
-4840.07679332386	9.3260400207255e-06	\\
-4839.09801136364	1.21732337384922e-05	\\
-4838.11922940341	1.06631743447384e-05	\\
-4837.14044744318	1.22880139570411e-05	\\
-4836.16166548296	1.3527920060554e-05	\\
-4835.18288352273	1.17256556564389e-05	\\
-4834.2041015625	1.23017798537662e-05	\\
-4833.22531960227	1.08696766674155e-05	\\
-4832.24653764205	1.16646223487682e-05	\\
-4831.26775568182	8.96544964765991e-06	\\
-4830.28897372159	1.15461846707481e-05	\\
-4829.31019176136	1.01643830959776e-05	\\
-4828.33140980114	1.24267595118362e-05	\\
-4827.35262784091	1.15204166114314e-05	\\
-4826.37384588068	1.11220621361489e-05	\\
-4825.39506392046	1.03559910196801e-05	\\
-4824.41628196023	1.10356683237411e-05	\\
-4823.4375	1.17391838923043e-05	\\
-4822.45871803977	1.15598039645713e-05	\\
-4821.47993607955	1.04622551773851e-05	\\
-4820.50115411932	1.11286792786862e-05	\\
-4819.52237215909	1.13655371872356e-05	\\
-4818.54359019886	8.69964991594862e-06	\\
-4817.56480823864	1.45602417340911e-05	\\
-4816.58602627841	1.02618759859223e-05	\\
-4815.60724431818	1.24918296327377e-05	\\
-4814.62846235796	1.10117700607381e-05	\\
-4813.64968039773	1.060739157072e-05	\\
-4812.6708984375	9.68661759812314e-06	\\
-4811.69211647727	1.00669266741402e-05	\\
-4810.71333451705	1.03741877452897e-05	\\
-4809.73455255682	8.97275415638386e-06	\\
-4808.75577059659	8.75683655415047e-06	\\
-4807.77698863636	1.10203244876839e-05	\\
-4806.79820667614	1.14332902812258e-05	\\
-4805.81942471591	1.05882380501137e-05	\\
-4804.84064275568	1.17213944453319e-05	\\
-4803.86186079546	9.68229368046728e-06	\\
-4802.88307883523	1.13097129733085e-05	\\
-4801.904296875	9.32165461373686e-06	\\
-4800.92551491477	8.35353141977128e-06	\\
-4799.94673295455	1.29487283535644e-05	\\
-4798.96795099432	1.10880570925187e-05	\\
-4797.98916903409	1.00129340196054e-05	\\
-4797.01038707386	1.08633390189706e-05	\\
-4796.03160511364	1.05187759879279e-05	\\
-4795.05282315341	1.08244209257709e-05	\\
-4794.07404119318	1.10618865491124e-05	\\
-4793.09525923296	1.07424833140186e-05	\\
-4792.11647727273	1.13117315053345e-05	\\
-4791.1376953125	1.0569000375177e-05	\\
-4790.15891335227	9.20212819244525e-06	\\
-4789.18013139205	1.0854192160534e-05	\\
-4788.20134943182	1.03198618506469e-05	\\
-4787.22256747159	1.10389552582046e-05	\\
-4786.24378551136	1.03473568609947e-05	\\
-4785.26500355114	9.76345089212971e-06	\\
-4784.28622159091	8.91487180517925e-06	\\
-4783.30743963068	1.09205808505396e-05	\\
-4782.32865767046	1.15982744313325e-05	\\
-4781.34987571023	1.12191195178717e-05	\\
-4780.37109375	1.10024209958498e-05	\\
-4779.39231178977	1.30996199651408e-05	\\
-4778.41352982955	1.35845140926982e-05	\\
-4777.43474786932	1.03418645724996e-05	\\
-4776.45596590909	1.09369771946297e-05	\\
-4775.47718394886	1.23872889763587e-05	\\
-4774.49840198864	1.05041475568944e-05	\\
-4773.51962002841	9.66162256469933e-06	\\
-4772.54083806818	1.09993344668004e-05	\\
-4771.56205610796	8.28632370082625e-06	\\
-4770.58327414773	1.08809525043841e-05	\\
-4769.6044921875	1.03230713617182e-05	\\
-4768.62571022727	1.21244082138877e-05	\\
-4767.64692826705	1.20177049616716e-05	\\
-4766.66814630682	9.71825236608622e-06	\\
-4765.68936434659	1.2305679030704e-05	\\
-4764.71058238636	1.09031553434492e-05	\\
-4763.73180042614	1.21667982123073e-05	\\
-4762.75301846591	1.17452228976147e-05	\\
-4761.77423650568	9.91873733666837e-06	\\
-4760.79545454546	1.22438359877588e-05	\\
-4759.81667258523	1.10194974621012e-05	\\
-4758.837890625	1.12701980157286e-05	\\
-4757.85910866477	1.17866097629698e-05	\\
-4756.88032670455	1.11075178790162e-05	\\
-4755.90154474432	9.21071728089462e-06	\\
-4754.92276278409	9.83178044977691e-06	\\
-4753.94398082386	1.20965649110057e-05	\\
-4752.96519886364	1.02158016210396e-05	\\
-4751.98641690341	1.4132465736522e-05	\\
-4751.00763494318	1.41621088642122e-05	\\
-4750.02885298296	1.45395999379348e-05	\\
-4749.05007102273	1.31074677436304e-05	\\
-4748.0712890625	1.10946367846244e-05	\\
-4747.09250710227	1.10268038550139e-05	\\
-4746.11372514205	9.40101613075242e-06	\\
-4745.13494318182	9.57629968119853e-06	\\
-4744.15616122159	8.62212682363643e-06	\\
-4743.17737926136	7.01004157103132e-06	\\
-4742.19859730114	6.93044796433632e-06	\\
-4741.21981534091	9.55223410265336e-06	\\
-4740.24103338068	9.91017890608127e-06	\\
-4739.26225142046	1.03134086987951e-05	\\
-4738.28346946023	1.10798078120563e-05	\\
-4737.3046875	1.3982163813786e-05	\\
-4736.32590553977	1.22726712389289e-05	\\
-4735.34712357955	1.23782483054906e-05	\\
-4734.36834161932	1.05681028244615e-05	\\
-4733.38955965909	9.59532679802444e-06	\\
-4732.41077769886	9.50519955523024e-06	\\
-4731.43199573864	9.43321008409998e-06	\\
-4730.45321377841	8.11162652781963e-06	\\
-4729.47443181818	8.24645103891086e-06	\\
-4728.49564985796	9.91246182203302e-06	\\
-4727.51686789773	1.17402666373427e-05	\\
-4726.5380859375	9.98925936180515e-06	\\
-4725.55930397727	1.21667163357676e-05	\\
-4724.58052201705	1.07392754877583e-05	\\
-4723.60174005682	9.51810100338754e-06	\\
-4722.62295809659	1.21323270822959e-05	\\
-4721.64417613636	9.80420665420544e-06	\\
-4720.66539417614	7.79742325054341e-06	\\
-4719.68661221591	9.24899383479051e-06	\\
-4718.70783025568	1.07738367407501e-05	\\
-4717.72904829546	9.04857957856564e-06	\\
-4716.75026633523	9.09492028421486e-06	\\
-4715.771484375	1.00294503736709e-05	\\
-4714.79270241477	1.03186637098628e-05	\\
-4713.81392045455	9.61240152172669e-06	\\
-4712.83513849432	9.2083377570879e-06	\\
-4711.85635653409	1.00561523082711e-05	\\
-4710.87757457386	1.21195505092015e-05	\\
-4709.89879261364	9.02569191070446e-06	\\
-4708.92001065341	1.13614854922779e-05	\\
-4707.94122869318	1.191118672893e-05	\\
-4706.96244673296	9.65850318926734e-06	\\
-4705.98366477273	1.04610921018053e-05	\\
-4705.0048828125	9.91536706058716e-06	\\
-4704.02610085227	1.20103646336524e-05	\\
-4703.04731889205	1.24898404374073e-05	\\
-4702.06853693182	1.38550607633288e-05	\\
-4701.08975497159	1.11262197392301e-05	\\
-4700.11097301136	1.04320959160327e-05	\\
-4699.13219105114	1.04086579656771e-05	\\
-4698.15340909091	9.60968069560253e-06	\\
-4697.17462713068	1.04078648642201e-05	\\
-4696.19584517046	9.83387858060986e-06	\\
-4695.21706321023	1.14524057342088e-05	\\
-4694.23828125	9.15382685371652e-06	\\
-4693.25949928977	1.0018566775943e-05	\\
-4692.28071732955	1.158714229717e-05	\\
-4691.30193536932	8.64068014971036e-06	\\
-4690.32315340909	1.04314970525828e-05	\\
-4689.34437144886	1.12482678017824e-05	\\
-4688.36558948864	9.0768880080296e-06	\\
-4687.38680752841	1.02960363675484e-05	\\
-4686.40802556818	1.00377316207757e-05	\\
-4685.42924360796	1.28659563855421e-05	\\
-4684.45046164773	1.04649585811095e-05	\\
-4683.4716796875	1.13396062671045e-05	\\
-4682.49289772727	1.07194226272422e-05	\\
-4681.51411576705	9.62133424293075e-06	\\
-4680.53533380682	9.7824111196514e-06	\\
-4679.55655184659	1.12938696973068e-05	\\
-4678.57776988636	1.03890304655039e-05	\\
-4677.59898792614	9.39067446326479e-06	\\
-4676.62020596591	8.69535592092008e-06	\\
-4675.64142400568	9.31121827026855e-06	\\
-4674.66264204546	9.175512587855e-06	\\
-4673.68386008523	8.84412252511396e-06	\\
-4672.705078125	9.21354868387791e-06	\\
-4671.72629616477	9.30564431663402e-06	\\
-4670.74751420455	8.90525546780543e-06	\\
-4669.76873224432	1.19097217000777e-05	\\
-4668.78995028409	9.65537950036649e-06	\\
-4667.81116832386	1.07043800430847e-05	\\
-4666.83238636364	1.05034526711286e-05	\\
-4665.85360440341	1.03497360387337e-05	\\
-4664.87482244318	9.72992543151683e-06	\\
-4663.89604048296	1.00594519441445e-05	\\
-4662.91725852273	9.46056446002972e-06	\\
-4661.9384765625	9.93329704530005e-06	\\
-4660.95969460227	1.10520440441331e-05	\\
-4659.98091264205	1.17043116897066e-05	\\
-4659.00213068182	1.182848813933e-05	\\
-4658.02334872159	1.01930645795204e-05	\\
-4657.04456676136	1.12200299565403e-05	\\
-4656.06578480114	8.77352397271198e-06	\\
-4655.08700284091	1.28795912441028e-05	\\
-4654.10822088068	1.0111667558442e-05	\\
-4653.12943892046	9.40383841954944e-06	\\
-4652.15065696023	6.69736819938051e-06	\\
-4651.171875	7.01680503362248e-06	\\
-4650.19309303977	1.09387413147018e-05	\\
-4649.21431107955	1.12232098721855e-05	\\
-4648.23552911932	1.2818099977866e-05	\\
-4647.25674715909	1.08267905609028e-05	\\
-4646.27796519886	1.10811276080848e-05	\\
-4645.29918323864	1.29197380649567e-05	\\
-4644.32040127841	1.02691583445439e-05	\\
-4643.34161931818	1.13537384459511e-05	\\
-4642.36283735796	9.84287688536614e-06	\\
-4641.38405539773	9.11261644765067e-06	\\
-4640.4052734375	1.01273224679945e-05	\\
-4639.42649147727	9.79599350848733e-06	\\
-4638.44770951705	1.29762489024275e-05	\\
-4637.46892755682	1.17239233994675e-05	\\
-4636.49014559659	1.00567691624594e-05	\\
-4635.51136363636	1.16016703584717e-05	\\
-4634.53258167614	1.03517130549524e-05	\\
-4633.55379971591	1.04468001215063e-05	\\
-4632.57501775568	1.36759074348775e-05	\\
-4631.59623579546	1.22455875948801e-05	\\
-4630.61745383523	1.22101958213903e-05	\\
-4629.638671875	1.02188550302004e-05	\\
-4628.65988991477	1.02172887162331e-05	\\
-4627.68110795455	1.21537165415916e-05	\\
-4626.70232599432	1.01131713780693e-05	\\
-4625.72354403409	1.15218187373091e-05	\\
-4624.74476207386	1.12312636278293e-05	\\
-4623.76598011364	1.3529225388728e-05	\\
-4622.78719815341	1.27597526033213e-05	\\
-4621.80841619318	1.3363192312115e-05	\\
-4620.82963423296	1.21055286351159e-05	\\
-4619.85085227273	1.07312704399129e-05	\\
-4618.8720703125	1.26012171063315e-05	\\
-4617.89328835227	1.18266127217874e-05	\\
-4616.91450639205	1.10220558976533e-05	\\
-4615.93572443182	1.17866935245835e-05	\\
-4614.95694247159	1.23782716090755e-05	\\
-4613.97816051136	1.07013810153099e-05	\\
-4612.99937855114	1.11324507456174e-05	\\
-4612.02059659091	1.1678908497518e-05	\\
-4611.04181463068	1.27103702635781e-05	\\
-4610.06303267046	1.42010733041818e-05	\\
-4609.08425071023	1.44336330800106e-05	\\
-4608.10546875	1.41301627381916e-05	\\
-4607.12668678977	1.38028567454699e-05	\\
-4606.14790482955	1.38193710695843e-05	\\
-4605.16912286932	1.46052908989918e-05	\\
-4604.19034090909	1.33229011761999e-05	\\
-4603.21155894886	1.14252371026593e-05	\\
-4602.23277698864	1.16516618826375e-05	\\
-4601.25399502841	1.02721114386681e-05	\\
-4600.27521306818	9.84835666239027e-06	\\
-4599.29643110796	1.22502353929052e-05	\\
-4598.31764914773	1.5272837679639e-05	\\
-4597.3388671875	1.58207654706145e-05	\\
-4596.36008522727	1.68365162464609e-05	\\
-4595.38130326705	1.53970018594342e-05	\\
-4594.40252130682	1.42757856934393e-05	\\
-4593.42373934659	1.35763206082845e-05	\\
-4592.44495738636	1.1821004636407e-05	\\
-4591.46617542614	1.28679292768547e-05	\\
-4590.48739346591	1.0801770992063e-05	\\
-4589.50861150568	1.43088878117981e-05	\\
-4588.52982954546	1.47960387732771e-05	\\
-4587.55104758523	1.55636066701422e-05	\\
-4586.572265625	1.69153396679121e-05	\\
-4585.59348366477	1.58184787163659e-05	\\
-4584.61470170455	1.60352883284118e-05	\\
-4583.63591974432	1.30505881848682e-05	\\
-4582.65713778409	1.574264975214e-05	\\
-4581.67835582386	1.42148597873465e-05	\\
-4580.69957386364	1.36877228371749e-05	\\
-4579.72079190341	1.15860589265107e-05	\\
-4578.74200994318	1.3427554986411e-05	\\
-4577.76322798296	1.36671030989828e-05	\\
-4576.78444602273	1.44091194150583e-05	\\
-4575.8056640625	1.23652744099735e-05	\\
-4574.82688210227	1.29997269495931e-05	\\
-4573.84810014205	1.45277059335038e-05	\\
-4572.86931818182	1.40007047180244e-05	\\
-4571.89053622159	1.49005197011432e-05	\\
-4570.91175426136	1.37047349658784e-05	\\
-4569.93297230114	1.5549653974196e-05	\\
-4568.95419034091	1.44598183737345e-05	\\
-4567.97540838068	1.39134861872788e-05	\\
-4566.99662642046	1.52005391995863e-05	\\
-4566.01784446023	1.29447006020406e-05	\\
-4565.0390625	1.56166989552795e-05	\\
-4564.06028053977	1.60385325479085e-05	\\
-4563.08149857955	1.48174410072558e-05	\\
-4562.10271661932	1.52978305805066e-05	\\
-4561.12393465909	1.18010836443579e-05	\\
-4560.14515269886	1.30392592472319e-05	\\
-4559.16637073864	1.60576282931187e-05	\\
-4558.18758877841	1.67297854421431e-05	\\
-4557.20880681818	1.60969563017066e-05	\\
-4556.23002485796	1.58406540948202e-05	\\
-4555.25124289773	1.40196289391925e-05	\\
-4554.2724609375	1.60772031351588e-05	\\
-4553.29367897727	1.60623886922391e-05	\\
-4552.31489701705	1.51087846471714e-05	\\
-4551.33611505682	1.65273525093455e-05	\\
-4550.35733309659	1.38963071638314e-05	\\
-4549.37855113636	1.51784669509958e-05	\\
-4548.39976917614	1.56480304932285e-05	\\
-4547.42098721591	1.42259149635356e-05	\\
-4546.44220525568	1.65070426586688e-05	\\
-4545.46342329546	1.51409098775556e-05	\\
-4544.48464133523	1.57738393070406e-05	\\
-4543.505859375	1.759817649251e-05	\\
-4542.52707741477	1.49512780452707e-05	\\
-4541.54829545455	1.66199314729079e-05	\\
-4540.56951349432	1.65058267832908e-05	\\
-4539.59073153409	1.52423369618258e-05	\\
-4538.61194957386	1.47632141201282e-05	\\
-4537.63316761364	1.4870122466199e-05	\\
-4536.65438565341	1.48769089238086e-05	\\
-4535.67560369318	1.37848352192047e-05	\\
-4534.69682173296	1.53247750510492e-05	\\
-4533.71803977273	1.49562620323065e-05	\\
-4532.7392578125	1.36778045025369e-05	\\
-4531.76047585227	1.59516157921121e-05	\\
-4530.78169389205	1.60144627237845e-05	\\
-4529.80291193182	1.55620032476466e-05	\\
-4528.82412997159	1.41255700878464e-05	\\
-4527.84534801136	1.51368489747039e-05	\\
-4526.86656605114	1.79072485313828e-05	\\
-4525.88778409091	1.72807655312826e-05	\\
-4524.90900213068	1.78339841925699e-05	\\
-4523.93022017046	1.48734784443299e-05	\\
-4522.95143821023	1.53559501799424e-05	\\
-4521.97265625	1.29696733916577e-05	\\
-4520.99387428977	1.57912254419988e-05	\\
-4520.01509232955	1.71248935688981e-05	\\
-4519.03631036932	1.4655575160649e-05	\\
-4518.05752840909	1.57994582492039e-05	\\
-4517.07874644886	1.47045696193241e-05	\\
-4516.09996448864	1.66788453438626e-05	\\
-4515.12118252841	1.78979566380407e-05	\\
-4514.14240056818	1.23872703902068e-05	\\
-4513.16361860796	1.62993403441132e-05	\\
-4512.18483664773	1.48639681398667e-05	\\
-4511.2060546875	1.68872131245196e-05	\\
-4510.22727272727	1.54743435410433e-05	\\
-4509.24849076705	1.657557411154e-05	\\
-4508.26970880682	1.63775558800677e-05	\\
-4507.29092684659	1.71279750653605e-05	\\
-4506.31214488636	1.79414673791525e-05	\\
-4505.33336292614	1.78315540691403e-05	\\
-4504.35458096591	1.69300258811506e-05	\\
-4503.37579900568	1.70329833661226e-05	\\
-4502.39701704546	1.57395633977633e-05	\\
-4501.41823508523	1.5313748207907e-05	\\
-4500.439453125	1.37267181778652e-05	\\
-4499.46067116477	1.3653739039242e-05	\\
-4498.48188920455	1.37460346961066e-05	\\
-4497.50310724432	1.63949141333622e-05	\\
-4496.52432528409	1.4116624069237e-05	\\
-4495.54554332386	1.73669907398786e-05	\\
-4494.56676136364	1.69544876618744e-05	\\
-4493.58797940341	1.74079360861591e-05	\\
-4492.60919744318	1.69652116748495e-05	\\
-4491.63041548296	1.68028918203611e-05	\\
-4490.65163352273	1.60902855087754e-05	\\
-4489.6728515625	1.75402990247154e-05	\\
-4488.69406960227	1.75087617763441e-05	\\
-4487.71528764205	1.36823841787485e-05	\\
-4486.73650568182	1.60296479605872e-05	\\
-4485.75772372159	1.60371304447972e-05	\\
-4484.77894176136	1.50373447862124e-05	\\
-4483.80015980114	1.7891425559367e-05	\\
-4482.82137784091	1.5255982181362e-05	\\
-4481.84259588068	1.95056635659587e-05	\\
-4480.86381392046	1.92079577988525e-05	\\
-4479.88503196023	1.88324239749026e-05	\\
-4478.90625	1.62906242066105e-05	\\
-4477.92746803977	1.59433977256241e-05	\\
-4476.94868607955	1.62067391140196e-05	\\
-4475.96990411932	1.65939350634601e-05	\\
-4474.99112215909	1.47238202120624e-05	\\
-4474.01234019886	1.6368675529536e-05	\\
-4473.03355823864	1.53809134263647e-05	\\
-4472.05477627841	1.67060451665483e-05	\\
-4471.07599431818	1.63840663770239e-05	\\
-4470.09721235796	1.61647663078227e-05	\\
-4469.11843039773	1.70455754316298e-05	\\
-4468.1396484375	1.77096016218172e-05	\\
-4467.16086647727	1.3787430881827e-05	\\
-4466.18208451705	1.50933689555145e-05	\\
-4465.20330255682	1.50180751057702e-05	\\
-4464.22452059659	1.58823291096624e-05	\\
-4463.24573863636	1.6775239224629e-05	\\
-4462.26695667614	1.37047686785706e-05	\\
-4461.28817471591	1.49618037584951e-05	\\
-4460.30939275568	1.77555278278804e-05	\\
-4459.33061079546	1.34972341548035e-05	\\
-4458.35182883523	1.52816632563168e-05	\\
-4457.373046875	1.52257439660771e-05	\\
-4456.39426491477	1.56773947776381e-05	\\
-4455.41548295455	1.6329398365522e-05	\\
-4454.43670099432	1.5455473742385e-05	\\
-4453.45791903409	1.54259666583399e-05	\\
-4452.47913707386	1.66036133568989e-05	\\
-4451.50035511364	1.42987892532779e-05	\\
-4450.52157315341	1.65290455408773e-05	\\
-4449.54279119318	1.81221914341796e-05	\\
-4448.56400923296	1.732989186706e-05	\\
-4447.58522727273	1.68233815845374e-05	\\
-4446.6064453125	1.69052219075642e-05	\\
-4445.62766335227	1.72458054276327e-05	\\
-4444.64888139205	1.59633506288687e-05	\\
-4443.67009943182	1.77976910592092e-05	\\
-4442.69131747159	1.45477223449352e-05	\\
-4441.71253551136	1.54762145078829e-05	\\
-4440.73375355114	1.60669234746963e-05	\\
-4439.75497159091	1.54472480027238e-05	\\
-4438.77618963068	1.54750313039777e-05	\\
-4437.79740767046	1.4489881573226e-05	\\
-4436.81862571023	1.60796668255039e-05	\\
-4435.83984375	1.3826371574318e-05	\\
-4434.86106178977	1.48797666272328e-05	\\
-4433.88227982955	1.62628121785321e-05	\\
-4432.90349786932	1.45907026814488e-05	\\
-4431.92471590909	1.57076433054888e-05	\\
-4430.94593394886	1.28844933442869e-05	\\
-4429.96715198864	1.47641117882876e-05	\\
-4428.98837002841	1.77222315639283e-05	\\
-4428.00958806818	1.59473255426041e-05	\\
-4427.03080610796	1.73607711346004e-05	\\
-4426.05202414773	1.54684494188454e-05	\\
-4425.0732421875	1.78170367574247e-05	\\
-4424.09446022727	1.41353735740395e-05	\\
-4423.11567826705	1.49369126953509e-05	\\
-4422.13689630682	1.67067473231358e-05	\\
-4421.15811434659	1.41705695578491e-05	\\
-4420.17933238636	1.70243682963574e-05	\\
-4419.20055042614	1.41533048830349e-05	\\
-4418.22176846591	1.57412794676157e-05	\\
-4417.24298650568	1.60551009156427e-05	\\
-4416.26420454546	1.63081981760412e-05	\\
-4415.28542258523	1.75848393796476e-05	\\
-4414.306640625	1.5148533842561e-05	\\
-4413.32785866477	1.66825094047072e-05	\\
-4412.34907670455	1.59539135663402e-05	\\
-4411.37029474432	1.70450324441997e-05	\\
-4410.39151278409	1.65498935549474e-05	\\
-4409.41273082386	1.78711996968529e-05	\\
-4408.43394886364	1.73227179381713e-05	\\
-4407.45516690341	1.70829121114927e-05	\\
-4406.47638494318	1.38680006540444e-05	\\
-4405.49760298296	1.79110846128345e-05	\\
-4404.51882102273	1.57120455753236e-05	\\
-4403.5400390625	1.4611186808441e-05	\\
-4402.56125710227	1.56982204193047e-05	\\
-4401.58247514205	1.85692258406789e-05	\\
-4400.60369318182	1.63851786239871e-05	\\
-4399.62491122159	1.60643770092334e-05	\\
-4398.64612926136	1.5951229219214e-05	\\
-4397.66734730114	1.20490500315092e-05	\\
-4396.68856534091	1.58652430563737e-05	\\
-4395.70978338068	1.73804917423205e-05	\\
-4394.73100142046	1.73155946723708e-05	\\
-4393.75221946023	1.55951505286584e-05	\\
-4392.7734375	1.49445564316832e-05	\\
-4391.79465553977	1.66555504808113e-05	\\
-4390.81587357955	1.83034610066342e-05	\\
-4389.83709161932	1.84772891454723e-05	\\
-4388.85830965909	1.47136815070584e-05	\\
-4387.87952769886	1.66854287213355e-05	\\
-4386.90074573864	1.43890176975976e-05	\\
-4385.92196377841	1.4724894218708e-05	\\
-4384.94318181818	1.65064735279101e-05	\\
-4383.96439985796	1.53999099201294e-05	\\
-4382.98561789773	1.56495554676382e-05	\\
-4382.0068359375	1.67034063880698e-05	\\
-4381.02805397727	1.4719497328459e-05	\\
-4380.04927201705	1.63217986953598e-05	\\
-4379.07049005682	1.81557229921118e-05	\\
-4378.09170809659	1.3587118736194e-05	\\
-4377.11292613636	1.37478992634179e-05	\\
-4376.13414417614	1.55095590070767e-05	\\
-4375.15536221591	1.6172116922316e-05	\\
-4374.17658025568	1.39028785867173e-05	\\
-4373.19779829546	1.570365475987e-05	\\
-4372.21901633523	1.44035433546429e-05	\\
-4371.240234375	1.4744145185843e-05	\\
-4370.26145241477	1.34056870474243e-05	\\
-4369.28267045455	1.61825577650816e-05	\\
-4368.30388849432	1.45471990026112e-05	\\
-4367.32510653409	1.41129086266143e-05	\\
-4366.34632457386	1.42451837060182e-05	\\
-4365.36754261364	1.42518024056544e-05	\\
-4364.38876065341	1.47145063827137e-05	\\
-4363.40997869318	1.45735643851033e-05	\\
-4362.43119673296	1.35917195698443e-05	\\
-4361.45241477273	1.4449729234825e-05	\\
-4360.4736328125	1.64584321523647e-05	\\
-4359.49485085227	1.47041435710996e-05	\\
-4358.51606889205	1.53658919333444e-05	\\
-4357.53728693182	1.22036525167831e-05	\\
-4356.55850497159	1.49769444512159e-05	\\
-4355.57972301136	1.388884457937e-05	\\
-4354.60094105114	1.56032971284287e-05	\\
-4353.62215909091	1.37527241765393e-05	\\
-4352.64337713068	1.43384706261157e-05	\\
-4351.66459517046	1.26758084337496e-05	\\
-4350.68581321023	1.61709973973551e-05	\\
-4349.70703125	1.60228309792272e-05	\\
-4348.72824928977	1.40678910916191e-05	\\
-4347.74946732955	1.53553910839086e-05	\\
-4346.77068536932	1.4436564759492e-05	\\
-4345.79190340909	1.49020442409913e-05	\\
-4344.81312144886	1.34384217805246e-05	\\
-4343.83433948864	1.48165852552907e-05	\\
-4342.85555752841	1.51929996127179e-05	\\
-4341.87677556818	1.50619317450472e-05	\\
-4340.89799360796	1.52071267720625e-05	\\
-4339.91921164773	1.22287196020843e-05	\\
-4338.9404296875	1.23863918536904e-05	\\
-4337.96164772727	1.49475368926684e-05	\\
-4336.98286576705	1.4138242657227e-05	\\
-4336.00408380682	1.38494309388004e-05	\\
-4335.02530184659	1.51043321844459e-05	\\
-4334.04651988636	1.25377168846452e-05	\\
-4333.06773792614	1.29767558490533e-05	\\
-4332.08895596591	1.39036113380999e-05	\\
-4331.11017400568	1.36795682768847e-05	\\
-4330.13139204546	1.24859472466491e-05	\\
-4329.15261008523	1.16699498083748e-05	\\
-4328.173828125	1.41362493845656e-05	\\
-4327.19504616477	1.53713605636459e-05	\\
-4326.21626420455	1.39525310627767e-05	\\
-4325.23748224432	1.32270708035396e-05	\\
-4324.25870028409	1.90941317251519e-05	\\
-4323.27991832386	1.36702847559731e-05	\\
-4322.30113636364	1.43020730643358e-05	\\
-4321.32235440341	1.34252233072938e-05	\\
-4320.34357244318	1.3862859818126e-05	\\
-4319.36479048296	1.30772966411557e-05	\\
-4318.38600852273	1.4510674319956e-05	\\
-4317.4072265625	1.17511926001574e-05	\\
-4316.42844460227	1.40176471352453e-05	\\
-4315.44966264205	1.12870184375547e-05	\\
-4314.47088068182	1.49667791455175e-05	\\
-4313.49209872159	1.40701277151751e-05	\\
-4312.51331676136	1.4617525899897e-05	\\
-4311.53453480114	1.40742420782767e-05	\\
-4310.55575284091	1.58373383511507e-05	\\
-4309.57697088068	1.32245338539851e-05	\\
-4308.59818892046	1.20121313208458e-05	\\
-4307.61940696023	1.41482281601716e-05	\\
-4306.640625	1.17224714274689e-05	\\
-4305.66184303977	1.35780023154777e-05	\\
-4304.68306107955	1.32364651371891e-05	\\
-4303.70427911932	1.30261981585436e-05	\\
-4302.72549715909	1.29243280653397e-05	\\
-4301.74671519886	1.47180626228427e-05	\\
-4300.76793323864	1.39653432162333e-05	\\
-4299.78915127841	1.02325422504729e-05	\\
-4298.81036931818	1.17833855839009e-05	\\
-4297.83158735796	1.46315437229106e-05	\\
-4296.85280539773	1.50770217847123e-05	\\
-4295.8740234375	1.59948309106453e-05	\\
-4294.89524147727	1.54590526505366e-05	\\
-4293.91645951705	1.06046256489932e-05	\\
-4292.93767755682	1.22472828879235e-05	\\
-4291.95889559659	1.44705073782445e-05	\\
-4290.98011363636	1.2775481960122e-05	\\
-4290.00133167614	1.40205779215077e-05	\\
-4289.02254971591	1.44529289144466e-05	\\
-4288.04376775568	1.26375099924106e-05	\\
-4287.06498579546	1.35053682117847e-05	\\
-4286.08620383523	1.49369620665036e-05	\\
-4285.107421875	1.1494655095981e-05	\\
-4284.12863991477	1.47700925679563e-05	\\
-4283.14985795455	1.48519912066101e-05	\\
-4282.17107599432	1.46705876271801e-05	\\
-4281.19229403409	1.36363315157979e-05	\\
-4280.21351207386	1.24089026850587e-05	\\
-4279.23473011364	1.19337207919821e-05	\\
-4278.25594815341	1.65897813082253e-05	\\
-4277.27716619318	1.30257328670778e-05	\\
-4276.29838423296	1.46477875520175e-05	\\
-4275.31960227273	1.5091913839565e-05	\\
-4274.3408203125	1.40065144523935e-05	\\
-4273.36203835227	1.30897332616294e-05	\\
-4272.38325639205	1.55265402921814e-05	\\
-4271.40447443182	1.28574660870952e-05	\\
-4270.42569247159	1.4750011846666e-05	\\
-4269.44691051136	1.53397173673237e-05	\\
-4268.46812855114	1.56859029355873e-05	\\
-4267.48934659091	1.32672056960965e-05	\\
-4266.51056463068	1.66510976411546e-05	\\
-4265.53178267046	1.41150831660057e-05	\\
-4264.55300071023	1.16753782091548e-05	\\
-4263.57421875	1.36490699202063e-05	\\
-4262.59543678977	1.35230497282459e-05	\\
-4261.61665482955	1.49116459215433e-05	\\
-4260.63787286932	1.62237288953798e-05	\\
-4259.65909090909	1.70407716699945e-05	\\
-4258.68030894886	1.4283729792454e-05	\\
-4257.70152698864	1.2165558000099e-05	\\
-4256.72274502841	1.2531896270357e-05	\\
-4255.74396306818	1.35187666318313e-05	\\
-4254.76518110796	1.31511264090908e-05	\\
-4253.78639914773	1.19561305884976e-05	\\
-4252.8076171875	1.39131866401437e-05	\\
-4251.82883522727	1.47092977832458e-05	\\
-4250.85005326705	1.36951974877354e-05	\\
-4249.87127130682	1.22185211248293e-05	\\
-4248.89248934659	1.40436634593796e-05	\\
-4247.91370738636	1.31742117886756e-05	\\
-4246.93492542614	1.28456536687676e-05	\\
-4245.95614346591	1.3643036617237e-05	\\
-4244.97736150568	1.18660576085027e-05	\\
-4243.99857954546	1.31276182429458e-05	\\
-4243.01979758523	1.17498565348344e-05	\\
-4242.041015625	1.36235023702166e-05	\\
-4241.06223366477	1.19503548607894e-05	\\
-4240.08345170455	1.335303992634e-05	\\
-4239.10466974432	1.36719755808748e-05	\\
-4238.12588778409	1.23319883949703e-05	\\
-4237.14710582386	1.28292634888181e-05	\\
-4236.16832386364	1.63813914957792e-05	\\
-4235.18954190341	1.19547251047829e-05	\\
-4234.21075994318	1.44296327230277e-05	\\
-4233.23197798296	1.08424459218431e-05	\\
-4232.25319602273	9.72221854442556e-06	\\
-4231.2744140625	1.28872964358222e-05	\\
-4230.29563210227	1.21988355281096e-05	\\
-4229.31685014205	1.45750433753446e-05	\\
-4228.33806818182	1.07513757938244e-05	\\
-4227.35928622159	1.13684870063481e-05	\\
-4226.38050426136	1.1762272668596e-05	\\
-4225.40172230114	1.3529956393443e-05	\\
-4224.42294034091	1.31072778007384e-05	\\
-4223.44415838068	1.370843903221e-05	\\
-4222.46537642046	1.31841919247081e-05	\\
-4221.48659446023	1.25542502392762e-05	\\
-4220.5078125	1.18212306509521e-05	\\
-4219.52903053977	1.17234396053416e-05	\\
-4218.55024857955	1.29814590539764e-05	\\
-4217.57146661932	1.37849459045128e-05	\\
-4216.59268465909	1.23544872080358e-05	\\
-4215.61390269886	1.36560061278686e-05	\\
-4214.63512073864	1.35668215187657e-05	\\
-4213.65633877841	1.31482729540747e-05	\\
-4212.67755681818	1.24912346831755e-05	\\
-4211.69877485796	1.15349841631701e-05	\\
-4210.71999289773	1.07371645170878e-05	\\
-4209.7412109375	1.0504666282009e-05	\\
-4208.76242897727	1.1457484157346e-05	\\
-4207.78364701705	1.07453253417655e-05	\\
-4206.80486505682	1.18985329215013e-05	\\
-4205.82608309659	1.26179251537947e-05	\\
-4204.84730113636	1.0766279364051e-05	\\
-4203.86851917614	1.14994153272405e-05	\\
-4202.88973721591	1.14729608501828e-05	\\
-4201.91095525568	1.19556614973088e-05	\\
-4200.93217329546	1.19687288378638e-05	\\
-4199.95339133523	1.33156790302718e-05	\\
-4198.974609375	1.24144725485177e-05	\\
-4197.99582741477	1.06193570727773e-05	\\
-4197.01704545455	1.06494413822352e-05	\\
-4196.03826349432	1.11188893169775e-05	\\
-4195.05948153409	1.2838390771041e-05	\\
-4194.08069957386	1.0392153031901e-05	\\
-4193.10191761364	1.22388140734921e-05	\\
-4192.12313565341	1.32874614363111e-05	\\
-4191.14435369318	1.18289928145772e-05	\\
-4190.16557173296	1.43202058356213e-05	\\
-4189.18678977273	1.11002929057443e-05	\\
-4188.2080078125	1.32096553235174e-05	\\
-4187.22922585227	1.01134640049407e-05	\\
-4186.25044389205	1.08657075123939e-05	\\
-4185.27166193182	1.19250862424839e-05	\\
-4184.29287997159	1.22671986836403e-05	\\
-4183.31409801136	1.35924560598968e-05	\\
-4182.33531605114	1.2219467266745e-05	\\
-4181.35653409091	1.19414208214525e-05	\\
-4180.37775213068	9.97445680456294e-06	\\
-4179.39897017046	1.32819980037166e-05	\\
-4178.42018821023	9.29075982669993e-06	\\
-4177.44140625	1.14026867260127e-05	\\
-4176.46262428977	1.19025424010018e-05	\\
-4175.48384232955	1.18117088736481e-05	\\
-4174.50506036932	8.55057993904553e-06	\\
-4173.52627840909	1.08655001824134e-05	\\
-4172.54749644886	1.26840151474693e-05	\\
-4171.56871448864	1.06686928848553e-05	\\
-4170.58993252841	1.43791564378082e-05	\\
-4169.61115056818	1.08648343702586e-05	\\
-4168.63236860796	1.17460656410257e-05	\\
-4167.65358664773	1.18732659030451e-05	\\
-4166.6748046875	1.25087513433873e-05	\\
-4165.69602272727	1.14904804066785e-05	\\
-4164.71724076705	1.20952654002389e-05	\\
-4163.73845880682	1.21805594007441e-05	\\
-4162.75967684659	1.1590257875549e-05	\\
-4161.78089488636	1.32900938706933e-05	\\
-4160.80211292614	1.412569659052e-05	\\
-4159.82333096591	1.14703338170725e-05	\\
-4158.84454900568	1.19460322648927e-05	\\
-4157.86576704546	1.24703118674464e-05	\\
-4156.88698508523	1.04753387405038e-05	\\
-4155.908203125	9.34759427699646e-06	\\
-4154.92942116477	9.80306346587507e-06	\\
-4153.95063920455	1.23180699580101e-05	\\
-4152.97185724432	1.21040979738762e-05	\\
-4151.99307528409	1.0125698405106e-05	\\
-4151.01429332386	1.25895087817326e-05	\\
-4150.03551136364	1.1742406087558e-05	\\
-4149.05672940341	1.08691931018388e-05	\\
-4148.07794744318	1.33377046654394e-05	\\
-4147.09916548296	1.186392395498e-05	\\
-4146.12038352273	9.09998962612039e-06	\\
-4145.1416015625	1.05397921328088e-05	\\
-4144.16281960227	1.06710309805444e-05	\\
-4143.18403764205	1.19156136283258e-05	\\
-4142.20525568182	1.17405770376332e-05	\\
-4141.22647372159	9.40697643708478e-06	\\
-4140.24769176136	1.27127612831837e-05	\\
-4139.26890980114	8.51858338733507e-06	\\
-4138.29012784091	1.08448168063588e-05	\\
-4137.31134588068	9.41892882935086e-06	\\
-4136.33256392046	8.47442539654971e-06	\\
-4135.35378196023	9.19783072486974e-06	\\
-4134.375	8.94449036491734e-06	\\
-4133.39621803977	1.16947672497275e-05	\\
-4132.41743607955	9.80721189088986e-06	\\
-4131.43865411932	1.14427031803659e-05	\\
-4130.45987215909	1.24758584067877e-05	\\
-4129.48109019886	1.0974639773991e-05	\\
-4128.50230823864	8.71166830250762e-06	\\
-4127.52352627841	1.16934786535151e-05	\\
-4126.54474431818	1.02804085182505e-05	\\
-4125.56596235796	9.29390186755175e-06	\\
-4124.58718039773	1.15672893975283e-05	\\
-4123.6083984375	7.68380010720509e-06	\\
-4122.62961647727	9.668670728909e-06	\\
-4121.65083451705	1.13749391157078e-05	\\
-4120.67205255682	1.16363053061724e-05	\\
-4119.69327059659	1.08063666575099e-05	\\
-4118.71448863636	1.1021688294297e-05	\\
-4117.73570667614	1.06648882877419e-05	\\
-4116.75692471591	1.10945983672383e-05	\\
-4115.77814275568	1.02948771997733e-05	\\
-4114.79936079546	1.09039478835185e-05	\\
-4113.82057883523	8.9176105023947e-06	\\
-4112.841796875	1.09455897370838e-05	\\
-4111.86301491477	1.03990078034365e-05	\\
-4110.88423295455	1.08342294252425e-05	\\
-4109.90545099432	1.12495251540513e-05	\\
-4108.92666903409	1.08525571896947e-05	\\
-4107.94788707386	1.14160107561145e-05	\\
-4106.96910511364	9.89490029265186e-06	\\
-4105.99032315341	1.19235766930969e-05	\\
-4105.01154119318	1.31514687920957e-05	\\
-4104.03275923296	1.16934890609912e-05	\\
-4103.05397727273	1.14790526107521e-05	\\
-4102.0751953125	9.07345879996485e-06	\\
-4101.09641335227	1.09068804544851e-05	\\
-4100.11763139205	1.14368446292876e-05	\\
-4099.13884943182	1.04555163723671e-05	\\
-4098.16006747159	1.10838969166808e-05	\\
-4097.18128551136	1.10155977052829e-05	\\
-4096.20250355114	1.08974118103334e-05	\\
-4095.22372159091	1.36890116401201e-05	\\
-4094.24493963068	1.40315979125516e-05	\\
-4093.26615767046	1.1372544370678e-05	\\
-4092.28737571023	1.00934344108589e-05	\\
-4091.30859375	8.84074522250661e-06	\\
-4090.32981178977	1.02020168269994e-05	\\
-4089.35102982955	1.09088624578118e-05	\\
-4088.37224786932	1.10843177612133e-05	\\
-4087.39346590909	1.30481047570285e-05	\\
-4086.41468394886	1.25416814093437e-05	\\
-4085.43590198864	1.04694896379022e-05	\\
-4084.45712002841	1.08322707483025e-05	\\
-4083.47833806818	1.05669926506852e-05	\\
-4082.49955610796	1.10860896801584e-05	\\
-4081.52077414773	9.78341245881712e-06	\\
-4080.5419921875	1.26607554335241e-05	\\
-4079.56321022727	1.08061190698275e-05	\\
-4078.58442826705	9.60179398453324e-06	\\
-4077.60564630682	1.12066791840553e-05	\\
-4076.62686434659	8.25319018861143e-06	\\
-4075.64808238636	1.0543431237054e-05	\\
-4074.66930042614	8.90803897045945e-06	\\
-4073.69051846591	1.0544103149984e-05	\\
-4072.71173650568	1.16001491601473e-05	\\
-4071.73295454546	1.26777763874439e-05	\\
-4070.75417258523	8.37289271574246e-06	\\
-4069.775390625	1.26702043880495e-05	\\
-4068.79660866477	1.11634155826021e-05	\\
-4067.81782670455	1.13983461314451e-05	\\
-4066.83904474432	1.28937522703447e-05	\\
-4065.86026278409	7.9134922870763e-06	\\
-4064.88148082386	8.4693975260634e-06	\\
-4063.90269886364	1.30909492257425e-05	\\
-4062.92391690341	1.03332142790475e-05	\\
-4061.94513494318	1.20138327810634e-05	\\
-4060.96635298296	1.21767838824387e-05	\\
-4059.98757102273	1.17504474362056e-05	\\
-4059.0087890625	1.34322156221587e-05	\\
-4058.03000710227	1.31016891098849e-05	\\
-4057.05122514205	1.18959466518834e-05	\\
-4056.07244318182	1.13518486136589e-05	\\
-4055.09366122159	1.17243630609351e-05	\\
-4054.11487926136	1.24270564090067e-05	\\
-4053.13609730114	1.2933323354412e-05	\\
-4052.15731534091	1.12773514435456e-05	\\
-4051.17853338068	1.07568740057213e-05	\\
-4050.19975142046	1.1127234428636e-05	\\
-4049.22096946023	1.45390729321661e-05	\\
-4048.2421875	9.91518324562202e-06	\\
-4047.26340553977	1.09471496444522e-05	\\
-4046.28462357955	1.04541673662591e-05	\\
-4045.30584161932	1.20084980383104e-05	\\
-4044.32705965909	1.02524918990524e-05	\\
-4043.34827769886	1.03101540928406e-05	\\
-4042.36949573864	1.10880309184718e-05	\\
-4041.39071377841	1.08680839187282e-05	\\
-4040.41193181818	1.10572636738499e-05	\\
-4039.43314985796	9.30214367614368e-06	\\
-4038.45436789773	1.20013978245303e-05	\\
-4037.4755859375	1.08103483322091e-05	\\
-4036.49680397727	1.12555575835861e-05	\\
-4035.51802201705	1.07471654914406e-05	\\
-4034.53924005682	1.32647747773481e-05	\\
-4033.56045809659	1.17820884917585e-05	\\
-4032.58167613636	1.14913405097279e-05	\\
-4031.60289417614	1.03312632466407e-05	\\
-4030.62411221591	9.67481867884016e-06	\\
-4029.64533025568	8.23350106508396e-06	\\
-4028.66654829546	1.07147569955245e-05	\\
-4027.68776633523	8.21506298264876e-06	\\
-4026.708984375	9.44809073761858e-06	\\
-4025.73020241477	9.43754362348227e-06	\\
-4024.75142045455	1.04937082284295e-05	\\
-4023.77263849432	1.16140590515971e-05	\\
-4022.79385653409	9.29858415502736e-06	\\
-4021.81507457386	1.14807036800381e-05	\\
-4020.83629261364	8.7277999746391e-06	\\
-4019.85751065341	1.17590881415268e-05	\\
-4018.87872869318	9.8723158586036e-06	\\
-4017.89994673296	1.00645063460259e-05	\\
-4016.92116477273	8.36376652839303e-06	\\
-4015.9423828125	9.01394529945753e-06	\\
-4014.96360085227	1.02160626857584e-05	\\
-4013.98481889205	1.14636299423217e-05	\\
-4013.00603693182	1.125603534379e-05	\\
-4012.02725497159	8.94603084466913e-06	\\
-4011.04847301136	1.04694983230607e-05	\\
-4010.06969105114	1.27057713505833e-05	\\
-4009.09090909091	9.64732318838002e-06	\\
-4008.11212713068	1.12125823711796e-05	\\
-4007.13334517046	1.06863691103344e-05	\\
-4006.15456321023	1.05280442224277e-05	\\
-4005.17578125	8.20972953857546e-06	\\
-4004.19699928977	9.25768994848809e-06	\\
-4003.21821732955	1.13601488519502e-05	\\
-4002.23943536932	9.50262607140787e-06	\\
-4001.26065340909	1.15430363753465e-05	\\
-4000.28187144886	9.92335105908427e-06	\\
-3999.30308948864	1.14941370340387e-05	\\
-3998.32430752841	1.12299628328105e-05	\\
-3997.34552556818	1.1562265544328e-05	\\
-3996.36674360796	1.22475459425331e-05	\\
-3995.38796164773	1.27528488794303e-05	\\
-3994.4091796875	1.03339732154899e-05	\\
-3993.43039772727	9.12259928957269e-06	\\
-3992.45161576705	1.05452817037713e-05	\\
-3991.47283380682	9.29905156781531e-06	\\
-3990.49405184659	8.60242300611007e-06	\\
-3989.51526988636	1.0247734786807e-05	\\
-3988.53648792614	9.94574859625661e-06	\\
-3987.55770596591	8.90391422826929e-06	\\
-3986.57892400568	1.15824783355002e-05	\\
-3985.60014204546	9.94395968193527e-06	\\
-3984.62136008523	1.11512575975874e-05	\\
-3983.642578125	8.16984183814062e-06	\\
-3982.66379616477	7.89022803672603e-06	\\
-3981.68501420455	8.56034056482279e-06	\\
-3980.70623224432	1.08007600881681e-05	\\
-3979.72745028409	7.35387202523408e-06	\\
-3978.74866832386	9.08015093504987e-06	\\
-3977.76988636364	1.177671650048e-05	\\
-3976.79110440341	1.2646003268147e-05	\\
-3975.81232244318	1.07206654469847e-05	\\
-3974.83354048296	1.00727680241133e-05	\\
-3973.85475852273	9.12004759594979e-06	\\
-3972.8759765625	8.68978602348306e-06	\\
-3971.89719460227	8.99965090729867e-06	\\
-3970.91841264205	1.06771772431933e-05	\\
-3969.93963068182	9.94385231375369e-06	\\
-3968.96084872159	8.43182268198775e-06	\\
-3967.98206676136	1.0777296194703e-05	\\
-3967.00328480114	9.98918437327275e-06	\\
-3966.02450284091	7.7246002589774e-06	\\
-3965.04572088068	9.02087107722898e-06	\\
-3964.06693892046	8.83136799660985e-06	\\
-3963.08815696023	9.75944796098467e-06	\\
-3962.109375	9.39956774681559e-06	\\
-3961.13059303977	6.10172819792929e-06	\\
-3960.15181107955	8.56733497604389e-06	\\
-3959.17302911932	6.39968338464726e-06	\\
-3958.19424715909	8.02523297902352e-06	\\
-3957.21546519886	8.06414135273062e-06	\\
-3956.23668323864	1.00317598988915e-05	\\
-3955.25790127841	1.07218115181319e-05	\\
-3954.27911931818	8.70989616150159e-06	\\
-3953.30033735796	1.00577961253319e-05	\\
-3952.32155539773	9.00214686752608e-06	\\
-3951.3427734375	1.08316114926778e-05	\\
-3950.36399147727	9.38661591606604e-06	\\
-3949.38520951705	6.73490992421154e-06	\\
-3948.40642755682	9.47866328853399e-06	\\
-3947.42764559659	7.61330604942051e-06	\\
-3946.44886363636	5.30136176719612e-06	\\
-3945.47008167614	7.01331566528407e-06	\\
-3944.49129971591	5.61522095450719e-06	\\
-3943.51251775568	9.51801601965828e-06	\\
-3942.53373579546	8.45404381876273e-06	\\
-3941.55495383523	1.00230663847045e-05	\\
-3940.576171875	8.15936566244288e-06	\\
-3939.59738991477	1.11433008466709e-05	\\
-3938.61860795455	9.93937139144146e-06	\\
-3937.63982599432	7.78673408654932e-06	\\
-3936.66104403409	9.96637091793384e-06	\\
-3935.68226207386	9.42422162956417e-06	\\
-3934.70348011364	6.01012391083549e-06	\\
-3933.72469815341	9.6098342386548e-06	\\
-3932.74591619318	1.09164889536639e-05	\\
-3931.76713423296	1.04082480173022e-05	\\
-3930.78835227273	8.61116182301095e-06	\\
-3929.8095703125	7.51472631717706e-06	\\
-3928.83078835227	8.70486356456537e-06	\\
-3927.85200639205	1.05597369284565e-05	\\
-3926.87322443182	6.53964303345654e-06	\\
-3925.89444247159	8.1827516288427e-06	\\
-3924.91566051136	9.48739426815318e-06	\\
-3923.93687855114	8.03908908924064e-06	\\
-3922.95809659091	1.13283453980459e-05	\\
-3921.97931463068	9.6174055187616e-06	\\
-3921.00053267046	9.52002547864409e-06	\\
-3920.02175071023	9.52941769374174e-06	\\
-3919.04296875	8.40503129324333e-06	\\
-3918.06418678977	6.79295597503101e-06	\\
-3917.08540482955	9.0553196361685e-06	\\
-3916.10662286932	1.04606503828543e-05	\\
-3915.12784090909	8.32150091134132e-06	\\
-3914.14905894886	6.93701805232434e-06	\\
-3913.17027698864	8.71212486740328e-06	\\
-3912.19149502841	7.52718260075098e-06	\\
-3911.21271306818	9.75989038802006e-06	\\
-3910.23393110796	8.8920378316292e-06	\\
-3909.25514914773	8.89650135743458e-06	\\
-3908.2763671875	8.4941538356873e-06	\\
-3907.29758522727	8.51197866191796e-06	\\
-3906.31880326705	6.92685376715838e-06	\\
-3905.34002130682	6.45685479606579e-06	\\
-3904.36123934659	6.12408650947004e-06	\\
-3903.38245738636	7.1770146034025e-06	\\
-3902.40367542614	3.99140557224369e-06	\\
-3901.42489346591	7.16153978682968e-06	\\
-3900.44611150568	9.11682255379947e-06	\\
-3899.46732954546	7.87005942221386e-06	\\
-3898.48854758523	8.06903628202201e-06	\\
-3897.509765625	9.52499878310146e-06	\\
-3896.53098366477	5.72232590382659e-06	\\
-3895.55220170455	7.80095813483149e-06	\\
-3894.57341974432	6.67614633257298e-06	\\
-3893.59463778409	7.37538689419954e-06	\\
-3892.61585582386	8.22389845314521e-06	\\
-3891.63707386364	6.60379644763585e-06	\\
-3890.65829190341	7.22775779172843e-06	\\
-3889.67950994318	5.79608664530167e-06	\\
-3888.70072798296	7.45031270801566e-06	\\
-3887.72194602273	6.62076767470328e-06	\\
-3886.7431640625	7.53799630336068e-06	\\
-3885.76438210227	9.11674065795103e-06	\\
-3884.78560014205	7.5730150572875e-06	\\
-3883.80681818182	7.83622913301947e-06	\\
-3882.82803622159	7.25938957137229e-06	\\
-3881.84925426136	1.02843845229786e-05	\\
-3880.87047230114	7.86134842458639e-06	\\
-3879.89169034091	8.59219215533931e-06	\\
-3878.91290838068	8.3306252562705e-06	\\
-3877.93412642046	8.95984988144346e-06	\\
-3876.95534446023	9.62683686978885e-06	\\
-3875.9765625	7.48509845601575e-06	\\
-3874.99778053977	7.85814896273965e-06	\\
-3874.01899857955	8.93451822919817e-06	\\
-3873.04021661932	8.39084687068638e-06	\\
-3872.06143465909	7.17088350485163e-06	\\
-3871.08265269886	6.88990864420562e-06	\\
-3870.10387073864	8.0679753807878e-06	\\
-3869.12508877841	7.47121091342427e-06	\\
-3868.14630681818	6.91078766357372e-06	\\
-3867.16752485796	8.39905263050578e-06	\\
-3866.18874289773	5.88507260969034e-06	\\
-3865.2099609375	8.3654703420113e-06	\\
-3864.23117897727	5.51012615676596e-06	\\
-3863.25239701705	8.75498248382689e-06	\\
-3862.27361505682	6.9094483842241e-06	\\
-3861.29483309659	7.319276893695e-06	\\
-3860.31605113636	8.55974040892545e-06	\\
-3859.33726917614	7.22052342170207e-06	\\
-3858.35848721591	5.89152284029728e-06	\\
-3857.37970525568	5.39256150487343e-06	\\
-3856.40092329546	6.65085788862528e-06	\\
-3855.42214133523	7.35078103408328e-06	\\
-3854.443359375	9.27437479797837e-06	\\
-3853.46457741477	7.72593424018709e-06	\\
-3852.48579545455	6.94971957282237e-06	\\
-3851.50701349432	7.40519062976217e-06	\\
-3850.52823153409	6.24148905093687e-06	\\
-3849.54944957386	8.9201928521041e-06	\\
-3848.57066761364	6.7411215237972e-06	\\
-3847.59188565341	6.965308565192e-06	\\
-3846.61310369318	6.8854467461958e-06	\\
-3845.63432173296	6.21286318626633e-06	\\
-3844.65553977273	6.05713807714762e-06	\\
-3843.6767578125	5.78163513699088e-06	\\
-3842.69797585227	4.68500405337881e-06	\\
-3841.71919389205	8.47018865624219e-06	\\
-3840.74041193182	8.21812855767188e-06	\\
-3839.76162997159	6.4033595841847e-06	\\
-3838.78284801136	6.40270606706117e-06	\\
-3837.80406605114	7.75858078366535e-06	\\
-3836.82528409091	7.84454196120125e-06	\\
-3835.84650213068	9.10474944346829e-06	\\
-3834.86772017046	7.43693365518207e-06	\\
-3833.88893821023	6.34417980428824e-06	\\
-3832.91015625	8.2710086621407e-06	\\
-3831.93137428977	5.16684529421786e-06	\\
-3830.95259232955	6.99177598335416e-06	\\
-3829.97381036932	8.4554522990679e-06	\\
-3828.99502840909	6.443844659156e-06	\\
-3828.01624644886	8.09359889222887e-06	\\
-3827.03746448864	1.01102218195791e-05	\\
-3826.05868252841	7.05714581322515e-06	\\
-3825.07990056818	8.20237535616737e-06	\\
-3824.10111860796	5.22918531017707e-06	\\
-3823.12233664773	6.85764581792156e-06	\\
-3822.1435546875	6.88908209704542e-06	\\
-3821.16477272727	7.04180687941928e-06	\\
-3820.18599076705	7.40405679798361e-06	\\
-3819.20720880682	7.24040423380109e-06	\\
-3818.22842684659	5.02528869152769e-06	\\
-3817.24964488636	6.73390842355412e-06	\\
-3816.27086292614	8.35067998315766e-06	\\
-3815.29208096591	5.11486805533566e-06	\\
-3814.31329900568	5.7884619710033e-06	\\
-3813.33451704546	6.64718700994743e-06	\\
-3812.35573508523	8.07101567790571e-06	\\
-3811.376953125	8.67368376125557e-06	\\
-3810.39817116477	8.65656331779332e-06	\\
-3809.41938920455	7.68061467755089e-06	\\
-3808.44060724432	7.77233703580377e-06	\\
-3807.46182528409	6.75761637291902e-06	\\
-3806.48304332386	8.57846101076394e-06	\\
-3805.50426136364	8.24816007979119e-06	\\
-3804.52547940341	4.70065848765479e-06	\\
-3803.54669744318	7.62763589872142e-06	\\
-3802.56791548296	9.1002021749213e-06	\\
-3801.58913352273	7.78145102904643e-06	\\
-3800.6103515625	5.86783311547373e-06	\\
-3799.63156960227	8.84279645506618e-06	\\
-3798.65278764205	9.08270262441714e-06	\\
-3797.67400568182	8.52692526746663e-06	\\
-3796.69522372159	8.81989470235938e-06	\\
-3795.71644176136	5.92267210482019e-06	\\
-3794.73765980114	8.25196121966765e-06	\\
-3793.75887784091	8.96466735529499e-06	\\
-3792.78009588068	8.85845552516095e-06	\\
-3791.80131392046	8.94150270743503e-06	\\
-3790.82253196023	1.05088179989209e-05	\\
-3789.84375	8.46427396638988e-06	\\
-3788.86496803977	8.50093952819187e-06	\\
-3787.88618607955	5.80013307336796e-06	\\
-3786.90740411932	8.16528423118959e-06	\\
-3785.92862215909	9.97753944646779e-06	\\
-3784.94984019886	5.19865395407313e-06	\\
-3783.97105823864	9.33872554869944e-06	\\
-3782.99227627841	7.55988467943172e-06	\\
-3782.01349431818	7.30418108141037e-06	\\
-3781.03471235796	6.56657235374613e-06	\\
-3780.05593039773	9.70850041827951e-06	\\
-3779.0771484375	8.04172728522899e-06	\\
-3778.09836647727	8.84390640009846e-06	\\
-3777.11958451705	7.49956776070063e-06	\\
-3776.14080255682	7.11642562356757e-06	\\
-3775.16202059659	5.85443251347548e-06	\\
-3774.18323863636	7.33012398719955e-06	\\
-3773.20445667614	6.775951664583e-06	\\
-3772.22567471591	7.14056788792626e-06	\\
-3771.24689275568	8.14432033252387e-06	\\
-3770.26811079546	8.12152549195699e-06	\\
-3769.28932883523	7.47961240178046e-06	\\
-3768.310546875	7.53867953585914e-06	\\
-3767.33176491477	7.56675591010734e-06	\\
-3766.35298295455	9.01616840721988e-06	\\
-3765.37420099432	6.80271255194504e-06	\\
-3764.39541903409	7.63857775116751e-06	\\
-3763.41663707386	5.8654703407583e-06	\\
-3762.43785511364	7.94033811277302e-06	\\
-3761.45907315341	7.1596473690305e-06	\\
-3760.48029119318	6.15354084823305e-06	\\
-3759.50150923296	7.10680009677556e-06	\\
-3758.52272727273	8.60961107909023e-06	\\
-3757.5439453125	8.19469248215412e-06	\\
-3756.56516335227	6.85557930103005e-06	\\
-3755.58638139205	7.80178404470043e-06	\\
-3754.60759943182	8.25283575928271e-06	\\
-3753.62881747159	5.6994733422277e-06	\\
-3752.65003551136	5.33515844392986e-06	\\
-3751.67125355114	6.42329527717843e-06	\\
-3750.69247159091	7.32401762690729e-06	\\
-3749.71368963068	9.08077399513466e-06	\\
-3748.73490767046	7.19916764270522e-06	\\
-3747.75612571023	7.56720891523585e-06	\\
-3746.77734375	7.3051897894652e-06	\\
-3745.79856178977	6.08249190252342e-06	\\
-3744.81977982955	6.6160319831123e-06	\\
-3743.84099786932	5.86305685837589e-06	\\
-3742.86221590909	5.5778126083664e-06	\\
-3741.88343394886	4.35385517961982e-06	\\
-3740.90465198864	6.60800169360869e-06	\\
-3739.92587002841	7.18038959765674e-06	\\
-3738.94708806818	5.91309380573564e-06	\\
-3737.96830610796	6.34714736867587e-06	\\
-3736.98952414773	7.17140010532719e-06	\\
-3736.0107421875	6.27101611096258e-06	\\
-3735.03196022727	6.04409324903174e-06	\\
-3734.05317826705	4.75054621273942e-06	\\
-3733.07439630682	4.92031158702522e-06	\\
-3732.09561434659	4.87548579538227e-06	\\
-3731.11683238636	5.42201023002457e-06	\\
-3730.13805042614	5.60130875226364e-06	\\
-3729.15926846591	7.46729790048491e-06	\\
-3728.18048650568	5.53926694232788e-06	\\
-3727.20170454546	4.25195111997762e-06	\\
-3726.22292258523	4.18715429872618e-06	\\
-3725.244140625	5.33140498760509e-06	\\
-3724.26535866477	4.84560769546973e-06	\\
-3723.28657670455	6.22838127850222e-06	\\
-3722.30779474432	6.53953915401352e-06	\\
-3721.32901278409	3.91258612541529e-06	\\
-3720.35023082386	4.4551527767481e-06	\\
-3719.37144886364	4.99311258283933e-06	\\
-3718.39266690341	5.30223385313839e-06	\\
-3717.41388494318	9.13093712096936e-06	\\
-3716.43510298296	4.35604739927719e-06	\\
-3715.45632102273	6.74798190642368e-06	\\
-3714.4775390625	4.59406345080459e-06	\\
-3713.49875710227	4.00615039142145e-06	\\
-3712.51997514205	7.62027025344297e-06	\\
-3711.54119318182	6.75473190831321e-06	\\
-3710.56241122159	5.07475277034169e-06	\\
-3709.58362926136	6.15493007413029e-06	\\
-3708.60484730114	4.88668033060319e-06	\\
-3707.62606534091	6.09502166183094e-06	\\
-3706.64728338068	5.13817564166087e-06	\\
-3705.66850142046	7.49996424404987e-06	\\
-3704.68971946023	4.66892048823575e-06	\\
-3703.7109375	6.69216942003197e-06	\\
-3702.73215553977	3.65558284974845e-06	\\
-3701.75337357955	5.38666624323178e-06	\\
-3700.77459161932	4.77309933025888e-06	\\
-3699.79580965909	3.92089243203462e-06	\\
-3698.81702769886	5.49574328477703e-06	\\
-3697.83824573864	8.15862764318605e-06	\\
-3696.85946377841	3.81701023934439e-06	\\
-3695.88068181818	7.52492284928144e-06	\\
-3694.90189985796	5.71957391363233e-06	\\
-3693.92311789773	5.86319819679038e-06	\\
-3692.9443359375	4.76847147200634e-06	\\
-3691.96555397727	4.7274957755584e-06	\\
-3690.98677201705	5.06407752635715e-06	\\
-3690.00799005682	3.8067320202121e-06	\\
-3689.02920809659	5.6065707565433e-06	\\
-3688.05042613636	3.90819090327007e-06	\\
-3687.07164417614	5.75163953015437e-06	\\
-3686.09286221591	7.20194412730285e-06	\\
-3685.11408025568	7.12010427111734e-06	\\
-3684.13529829546	3.62014286048305e-06	\\
-3683.15651633523	4.09644727983965e-06	\\
-3682.177734375	4.59061189445033e-06	\\
-3681.19895241477	5.66029139252349e-06	\\
-3680.22017045455	5.96340542173075e-06	\\
-3679.24138849432	3.90886532921651e-06	\\
-3678.26260653409	4.86834197120754e-06	\\
-3677.28382457386	7.19125332991926e-06	\\
-3676.30504261364	6.81139890244367e-06	\\
-3675.32626065341	5.06958711753731e-06	\\
-3674.34747869318	6.33391242737011e-06	\\
-3673.36869673296	4.11787227326751e-06	\\
-3672.38991477273	6.75397528384377e-06	\\
-3671.4111328125	5.15332550428465e-06	\\
-3670.43235085227	5.58735033763114e-06	\\
-3669.45356889205	5.84821585264568e-06	\\
-3668.47478693182	6.64737887194797e-06	\\
-3667.49600497159	5.39490944936203e-06	\\
-3666.51722301136	6.52136226211795e-06	\\
-3665.53844105114	6.11418399035535e-06	\\
-3664.55965909091	3.86748561937351e-06	\\
-3663.58087713068	6.12531949998876e-06	\\
-3662.60209517046	5.58712043698536e-06	\\
-3661.62331321023	7.77939167455003e-06	\\
-3660.64453125	6.8394902281859e-06	\\
-3659.66574928977	7.89673484851145e-06	\\
-3658.68696732955	6.23621181427428e-06	\\
-3657.70818536932	6.40643770274655e-06	\\
-3656.72940340909	6.02573403868961e-06	\\
-3655.75062144886	4.67409217687299e-06	\\
-3654.77183948864	7.62115988399315e-06	\\
-3653.79305752841	8.17099170096971e-06	\\
-3652.81427556818	7.86017399730018e-06	\\
-3651.83549360796	5.26532393395139e-06	\\
-3650.85671164773	6.9299470912518e-06	\\
-3649.8779296875	7.64387113133748e-06	\\
-3648.89914772727	4.96389740660618e-06	\\
-3647.92036576705	7.88484541807491e-06	\\
-3646.94158380682	6.02572332233436e-06	\\
-3645.96280184659	6.71308140232794e-06	\\
-3644.98401988636	6.12223441147365e-06	\\
-3644.00523792614	5.20241292324569e-06	\\
-3643.02645596591	6.12220707004126e-06	\\
-3642.04767400568	5.45744039960745e-06	\\
-3641.06889204546	4.47196671760675e-06	\\
-3640.09011008523	6.89186865074403e-06	\\
-3639.111328125	6.75242286026626e-06	\\
-3638.13254616477	5.29377007366807e-06	\\
-3637.15376420455	3.68491790948206e-06	\\
-3636.17498224432	9.15248479075253e-06	\\
-3635.19620028409	6.71931340916732e-06	\\
-3634.21741832386	7.78938204161221e-06	\\
-3633.23863636364	5.38363618962568e-06	\\
-3632.25985440341	5.9645818142036e-06	\\
-3631.28107244318	5.25666662424234e-06	\\
-3630.30229048296	5.01255514787651e-06	\\
-3629.32350852273	5.95504141928408e-06	\\
-3628.3447265625	5.82179618095775e-06	\\
-3627.36594460227	4.74086522628021e-06	\\
-3626.38716264205	4.92493759445532e-06	\\
-3625.40838068182	5.50486988483935e-06	\\
-3624.42959872159	6.35167278916737e-06	\\
-3623.45081676136	4.70330272530917e-06	\\
-3622.47203480114	4.05586413804813e-06	\\
-3621.49325284091	6.12473671093464e-06	\\
-3620.51447088068	5.79615003468929e-06	\\
-3619.53568892046	5.46307065124631e-06	\\
-3618.55690696023	6.57732300753424e-06	\\
-3617.578125	4.92495974880354e-06	\\
-3616.59934303977	5.44975327539255e-06	\\
-3615.62056107955	6.04280435679032e-06	\\
-3614.64177911932	5.75779809032343e-06	\\
-3613.66299715909	7.48438988790588e-06	\\
-3612.68421519886	9.03922422165704e-06	\\
-3611.70543323864	5.91679835515188e-06	\\
-3610.72665127841	3.41798654756105e-06	\\
-3609.74786931818	5.29597860804442e-06	\\
-3608.76908735796	3.59827076733747e-06	\\
-3607.79030539773	6.55130005852284e-06	\\
-3606.8115234375	5.76897207609718e-06	\\
-3605.83274147727	7.72752760225253e-06	\\
-3604.85395951705	7.72416717880155e-06	\\
-3603.87517755682	6.81193132689793e-06	\\
-3602.89639559659	4.28608386311376e-06	\\
-3601.91761363636	6.64376728240701e-06	\\
-3600.93883167614	5.80251236657074e-06	\\
-3599.96004971591	5.02070764360607e-06	\\
-3598.98126775568	6.65215618990689e-06	\\
-3598.00248579546	7.54311835447507e-06	\\
-3597.02370383523	5.66301928656613e-06	\\
-3596.044921875	8.47443878300234e-06	\\
-3595.06613991477	5.72604309645326e-06	\\
-3594.08735795455	6.46703231561407e-06	\\
-3593.10857599432	7.22254495203417e-06	\\
-3592.12979403409	6.04415202550902e-06	\\
-3591.15101207386	6.51812507423034e-06	\\
-3590.17223011364	5.65985600905342e-06	\\
-3589.19344815341	6.30118921875191e-06	\\
-3588.21466619318	4.19455655358598e-06	\\
-3587.23588423296	6.14589652224486e-06	\\
-3586.25710227273	5.00701361053067e-06	\\
-3585.2783203125	4.70605684611637e-06	\\
-3584.29953835227	7.84262346558585e-06	\\
-3583.32075639205	5.01995996316489e-06	\\
-3582.34197443182	6.43811540214592e-06	\\
-3581.36319247159	5.71872848971829e-06	\\
-3580.38441051136	4.98120767346133e-06	\\
-3579.40562855114	5.1534677504994e-06	\\
-3578.42684659091	4.98167705915481e-06	\\
-3577.44806463068	7.14470155295548e-06	\\
-3576.46928267046	5.08588306267143e-06	\\
-3575.49050071023	6.85058334370493e-06	\\
-3574.51171875	5.80918137154211e-06	\\
-3573.53293678977	8.36639597986128e-06	\\
-3572.55415482955	7.25472186039056e-06	\\
-3571.57537286932	4.31363442005656e-06	\\
-3570.59659090909	6.92915597776963e-06	\\
-3569.61780894886	6.10840765271297e-06	\\
-3568.63902698864	6.1205731034157e-06	\\
-3567.66024502841	5.19934884238498e-06	\\
-3566.68146306818	7.47392702735576e-06	\\
-3565.70268110796	4.32675996489402e-06	\\
-3564.72389914773	8.87286185486794e-06	\\
-3563.7451171875	8.54380140155201e-06	\\
-3562.76633522727	6.16721068332521e-06	\\
-3561.78755326705	8.70048584898104e-06	\\
-3560.80877130682	5.83947649533358e-06	\\
-3559.82998934659	5.28956856910582e-06	\\
-3558.85120738636	5.91291438474065e-06	\\
-3557.87242542614	5.52961088485144e-06	\\
-3556.89364346591	8.76616784128419e-06	\\
-3555.91486150568	8.43982977479424e-06	\\
-3554.93607954546	9.25510773000291e-06	\\
-3553.95729758523	4.90507372523561e-06	\\
-3552.978515625	4.89367580688912e-06	\\
-3551.99973366477	8.75833153100587e-06	\\
-3551.02095170455	8.49271721214546e-06	\\
-3550.04216974432	8.12269646650254e-06	\\
-3549.06338778409	8.10287584775513e-06	\\
-3548.08460582386	7.71673948580948e-06	\\
-3547.10582386364	6.97464003906052e-06	\\
-3546.12704190341	7.27798061806614e-06	\\
-3545.14825994318	7.30551390303694e-06	\\
-3544.16947798296	9.90093808301727e-06	\\
-3543.19069602273	8.34808497765795e-06	\\
-3542.2119140625	7.99447174397667e-06	\\
-3541.23313210227	4.89159192713225e-06	\\
-3540.25435014205	7.79935656664089e-06	\\
-3539.27556818182	4.74014165364509e-06	\\
-3538.29678622159	7.57442799221758e-06	\\
-3537.31800426136	8.33968208461058e-06	\\
-3536.33922230114	6.3448567851218e-06	\\
-3535.36044034091	8.27001353433307e-06	\\
-3534.38165838068	6.98460749172944e-06	\\
-3533.40287642046	7.59124480229111e-06	\\
-3532.42409446023	5.0157085668709e-06	\\
-3531.4453125	4.52192493253271e-06	\\
-3530.46653053977	7.36375619001882e-06	\\
-3529.48774857955	7.5449020616378e-06	\\
-3528.50896661932	4.11935426963018e-06	\\
-3527.53018465909	6.69794389323451e-06	\\
-3526.55140269886	7.45693559061545e-06	\\
-3525.57262073864	6.71920546619903e-06	\\
-3524.59383877841	6.9374026792154e-06	\\
-3523.61505681818	7.02431779212775e-06	\\
-3522.63627485796	6.47630058402033e-06	\\
-3521.65749289773	6.41315430571118e-06	\\
-3520.6787109375	5.80448584381825e-06	\\
-3519.69992897727	7.66115410192093e-06	\\
-3518.72114701705	5.22946469483049e-06	\\
-3517.74236505682	4.26743351707681e-06	\\
-3516.76358309659	6.46713330843624e-06	\\
-3515.78480113636	6.39244716031695e-06	\\
-3514.80601917614	5.6647106985432e-06	\\
-3513.82723721591	9.81152796961525e-06	\\
-3512.84845525568	6.93970322255143e-06	\\
-3511.86967329546	6.01836949631538e-06	\\
-3510.89089133523	8.55272800008483e-06	\\
-3509.912109375	7.92330995635317e-06	\\
-3508.93332741477	8.84278733180434e-06	\\
-3507.95454545455	6.15560592537302e-06	\\
-3506.97576349432	3.7676813679772e-06	\\
-3505.99698153409	7.91607107761979e-06	\\
-3505.01819957386	8.23758327219558e-06	\\
-3504.03941761364	4.96805033259464e-06	\\
-3503.06063565341	7.85702879399076e-06	\\
-3502.08185369318	7.88531456908246e-06	\\
-3501.10307173296	8.84491700526281e-06	\\
-3500.12428977273	5.18233475439163e-06	\\
-3499.1455078125	6.21429183950036e-06	\\
-3498.16672585227	8.02797698097798e-06	\\
-3497.18794389205	5.21155919160867e-06	\\
-3496.20916193182	8.20902125809212e-06	\\
-3495.23037997159	6.04673970297001e-06	\\
-3494.25159801136	8.25085686115338e-06	\\
-3493.27281605114	7.21614046725465e-06	\\
-3492.29403409091	3.78010079014488e-06	\\
-3491.31525213068	5.45426441292021e-06	\\
-3490.33647017046	4.77971675010694e-06	\\
-3489.35768821023	5.17599179465239e-06	\\
-3488.37890625	5.52648340046677e-06	\\
-3487.40012428977	6.36702807454397e-06	\\
-3486.42134232955	7.36955719161516e-06	\\
-3485.44256036932	6.5838534475994e-06	\\
-3484.46377840909	5.056437607108e-06	\\
-3483.48499644886	6.16744448126104e-06	\\
-3482.50621448864	3.00136556224579e-06	\\
-3481.52743252841	6.13412453330947e-06	\\
-3480.54865056818	7.8157263363365e-06	\\
-3479.56986860796	4.27263229519662e-06	\\
-3478.59108664773	6.3802847236994e-06	\\
-3477.6123046875	6.71390251053503e-06	\\
-3476.63352272727	8.28003508496028e-06	\\
-3475.65474076705	5.16461960748422e-06	\\
-3474.67595880682	4.87350696913155e-06	\\
-3473.69717684659	4.46652921305547e-06	\\
-3472.71839488636	4.41711686018104e-06	\\
-3471.73961292614	7.05253169286347e-06	\\
-3470.76083096591	6.38168303016065e-06	\\
-3469.78204900568	6.68251709828621e-06	\\
-3468.80326704546	6.50949404761308e-06	\\
-3467.82448508523	5.49836158129102e-06	\\
-3466.845703125	7.28894221915631e-06	\\
-3465.86692116477	6.92564279519691e-06	\\
-3464.88813920455	6.94680488782122e-06	\\
-3463.90935724432	7.5619126173186e-06	\\
-3462.93057528409	1.01161041584801e-05	\\
-3461.95179332386	6.86022284061155e-06	\\
-3460.97301136364	9.29149528477892e-06	\\
-3459.99422940341	7.41940785194909e-06	\\
-3459.01544744318	7.87804523221647e-06	\\
-3458.03666548296	6.76257004747967e-06	\\
-3457.05788352273	5.82556940238407e-06	\\
-3456.0791015625	7.64587142366812e-06	\\
-3455.10031960227	7.6659976717771e-06	\\
-3454.12153764205	9.69394550351038e-06	\\
-3453.14275568182	4.77793722699523e-06	\\
-3452.16397372159	8.30299497918e-06	\\
-3451.18519176136	8.80808763860331e-06	\\
-3450.20640980114	7.35790532559217e-06	\\
-3449.22762784091	7.70351548227938e-06	\\
-3448.24884588068	8.14071883116257e-06	\\
-3447.27006392046	6.9378933647699e-06	\\
-3446.29128196023	3.70200641989561e-06	\\
-3445.3125	5.99173790827754e-06	\\
-3444.33371803977	7.53278680441074e-06	\\
-3443.35493607955	6.54087288283437e-06	\\
-3442.37615411932	7.72460782252245e-06	\\
-3441.39737215909	6.84836154620129e-06	\\
-3440.41859019886	5.05856236041813e-06	\\
-3439.43980823864	8.10601030548511e-06	\\
-3438.46102627841	7.13970581025488e-06	\\
-3437.48224431818	8.29242763924627e-06	\\
-3436.50346235796	6.07576725849794e-06	\\
-3435.52468039773	8.50651891895515e-06	\\
-3434.5458984375	6.66392720869497e-06	\\
-3433.56711647727	6.79846231388591e-06	\\
-3432.58833451705	6.30229051593783e-06	\\
-3431.60955255682	4.73776766269984e-06	\\
-3430.63077059659	2.77482619174596e-06	\\
-3429.65198863636	6.28377305330906e-06	\\
-3428.67320667614	7.03433153620489e-06	\\
-3427.69442471591	7.00796822398318e-06	\\
-3426.71564275568	7.99995705540487e-06	\\
-3425.73686079546	8.30646375958038e-06	\\
-3424.75807883523	6.43160740907779e-06	\\
-3423.779296875	7.65786070626034e-06	\\
-3422.80051491477	5.62834548355789e-06	\\
-3421.82173295455	4.43239442074157e-06	\\
-3420.84295099432	4.83030759570343e-06	\\
-3419.86416903409	7.76509448476902e-06	\\
-3418.88538707386	7.10026395781727e-06	\\
-3417.90660511364	4.40146831901494e-06	\\
-3416.92782315341	3.99686819027672e-06	\\
-3415.94904119318	6.71143783038658e-06	\\
-3414.97025923296	2.38283455407305e-06	\\
-3413.99147727273	9.16232725202475e-06	\\
-3413.0126953125	6.54592566917883e-06	\\
-3412.03391335227	7.99293070183558e-06	\\
-3411.05513139205	6.60121476417006e-06	\\
-3410.07634943182	7.20635509411527e-06	\\
-3409.09756747159	4.0366987377356e-06	\\
-3408.11878551136	6.90188943552882e-06	\\
-3407.14000355114	3.77433889043678e-06	\\
-3406.16122159091	7.26339198684622e-06	\\
-3405.18243963068	7.48842392090886e-06	\\
-3404.20365767046	1.02619505309727e-05	\\
-3403.22487571023	4.81408353765791e-06	\\
-3402.24609375	5.37891938583579e-06	\\
-3401.26731178977	5.83378188966146e-06	\\
-3400.28852982955	5.70550447471887e-06	\\
-3399.30974786932	7.11092579527137e-06	\\
-3398.33096590909	7.86951994067151e-06	\\
-3397.35218394886	7.9190667227137e-06	\\
-3396.37340198864	8.21994853591322e-06	\\
-3395.39462002841	7.45587492044357e-06	\\
-3394.41583806818	6.37732724761621e-06	\\
-3393.43705610796	6.84890177200744e-06	\\
-3392.45827414773	5.79316356799543e-06	\\
-3391.4794921875	6.82790216426201e-06	\\
-3390.50071022727	6.58484964340971e-06	\\
-3389.52192826705	9.33550395247543e-06	\\
-3388.54314630682	6.71737322204307e-06	\\
-3387.56436434659	6.57631789533697e-06	\\
-3386.58558238636	5.76898927266532e-06	\\
-3385.60680042614	4.7049351539577e-06	\\
-3384.62801846591	6.22585919867337e-06	\\
-3383.64923650568	9.69782900792454e-06	\\
-3382.67045454546	7.71705747928608e-06	\\
-3381.69167258523	7.39343881204883e-06	\\
-3380.712890625	7.07200914265782e-06	\\
-3379.73410866477	7.15441903640344e-06	\\
-3378.75532670455	6.94838989868597e-06	\\
-3377.77654474432	1.05686705840513e-05	\\
-3376.79776278409	7.60692380105086e-06	\\
-3375.81898082386	8.44065631619855e-06	\\
-3374.84019886364	8.55197052134713e-06	\\
-3373.86141690341	9.93265218941174e-06	\\
-3372.88263494318	7.82487888422349e-06	\\
-3371.90385298296	7.19673519688336e-06	\\
-3370.92507102273	7.44558206478955e-06	\\
-3369.9462890625	6.89944727841426e-06	\\
-3368.96750710227	9.64371153771517e-06	\\
-3367.98872514205	7.44402493791908e-06	\\
-3367.00994318182	6.84039083965697e-06	\\
-3366.03116122159	7.29350871201282e-06	\\
-3365.05237926136	6.85539787566859e-06	\\
-3364.07359730114	6.98775215765831e-06	\\
-3363.09481534091	6.69830633197007e-06	\\
-3362.11603338068	7.30792026793953e-06	\\
-3361.13725142046	8.30738297076493e-06	\\
-3360.15846946023	5.48839108913006e-06	\\
-3359.1796875	8.07424375401736e-06	\\
-3358.20090553977	8.36871221273323e-06	\\
-3357.22212357955	9.43830640792177e-06	\\
-3356.24334161932	6.31870904327666e-06	\\
-3355.26455965909	8.24038575027765e-06	\\
-3354.28577769886	9.07323722850592e-06	\\
-3353.30699573864	6.85492585015592e-06	\\
-3352.32821377841	6.41640383978907e-06	\\
-3351.34943181818	7.40080677453706e-06	\\
-3350.37064985796	8.11650210632133e-06	\\
-3349.39186789773	5.67994348308526e-06	\\
-3348.4130859375	6.80450518706188e-06	\\
-3347.43430397727	6.3165992972136e-06	\\
-3346.45552201705	5.55713189251655e-06	\\
-3345.47674005682	6.78035793684003e-06	\\
-3344.49795809659	7.44174954045407e-06	\\
-3343.51917613636	6.48592964833874e-06	\\
-3342.54039417614	6.14978028850856e-06	\\
-3341.56161221591	6.9730682994108e-06	\\
-3340.58283025568	8.95637699944769e-06	\\
-3339.60404829546	6.58308444298448e-06	\\
-3338.62526633523	6.90509342064109e-06	\\
-3337.646484375	7.05863278835874e-06	\\
-3336.66770241477	7.46061936564249e-06	\\
-3335.68892045455	5.4677146489931e-06	\\
-3334.71013849432	8.41799297286671e-06	\\
-3333.73135653409	8.22333149309792e-06	\\
-3332.75257457386	1.01399206451832e-05	\\
-3331.77379261364	6.29219206593473e-06	\\
-3330.79501065341	7.69444990738781e-06	\\
-3329.81622869318	9.83154696703001e-06	\\
-3328.83744673296	6.49174316445218e-06	\\
-3327.85866477273	6.69771063151008e-06	\\
-3326.8798828125	7.01575146654042e-06	\\
-3325.90110085227	6.35983347770488e-06	\\
-3324.92231889205	6.16208006587392e-06	\\
-3323.94353693182	6.69808778450492e-06	\\
-3322.96475497159	8.70111819049835e-06	\\
-3321.98597301136	6.941028674256e-06	\\
-3321.00719105114	6.20932637759786e-06	\\
-3320.02840909091	8.43358883167105e-06	\\
-3319.04962713068	9.81091692586088e-06	\\
-3318.07084517046	7.84756333079582e-06	\\
-3317.09206321023	6.15846599281555e-06	\\
-3316.11328125	7.45895704143135e-06	\\
-3315.13449928977	7.32208244607607e-06	\\
-3314.15571732955	5.2731999162103e-06	\\
-3313.17693536932	8.34645314049038e-06	\\
-3312.19815340909	9.81418959686322e-06	\\
-3311.21937144886	6.50644268970866e-06	\\
-3310.24058948864	8.69830457491726e-06	\\
-3309.26180752841	6.44529993999361e-06	\\
-3308.28302556818	9.76868552133638e-06	\\
-3307.30424360796	7.27683917550227e-06	\\
-3306.32546164773	9.88758830207203e-06	\\
-3305.3466796875	6.28498355641954e-06	\\
-3304.36789772727	8.93289980575135e-06	\\
-3303.38911576705	4.77191837013829e-06	\\
-3302.41033380682	8.88523462359483e-06	\\
-3301.43155184659	6.13210353526283e-06	\\
-3300.45276988636	6.96508575429446e-06	\\
-3299.47398792614	3.37697839336722e-06	\\
-3298.49520596591	6.41106619698526e-06	\\
-3297.51642400568	8.91892451367944e-06	\\
-3296.53764204546	8.19329331255562e-06	\\
-3295.55886008523	6.52904387827444e-06	\\
-3294.580078125	6.97655445231526e-06	\\
-3293.60129616477	6.58410975446255e-06	\\
-3292.62251420455	8.37592060157944e-06	\\
-3291.64373224432	7.99675204099443e-06	\\
-3290.66495028409	9.11573731878824e-06	\\
-3289.68616832386	6.83219437516752e-06	\\
-3288.70738636364	6.52671439327849e-06	\\
-3287.72860440341	7.6044077536189e-06	\\
-3286.74982244318	5.58690297865913e-06	\\
-3285.77104048296	8.01764773017378e-06	\\
-3284.79225852273	4.10121181473996e-06	\\
-3283.8134765625	1.16129092628454e-05	\\
-3282.83469460227	3.02621981215116e-06	\\
-3281.85591264205	9.12278384178588e-06	\\
-3280.87713068182	5.3011361950394e-06	\\
-3279.89834872159	8.96647267906354e-06	\\
-3278.91956676136	7.45959678344316e-06	\\
-3277.94078480114	7.8361886983984e-06	\\
-3276.96200284091	7.00865555213517e-06	\\
-3275.98322088068	9.3234866500227e-06	\\
-3275.00443892046	5.88843685975198e-06	\\
-3274.02565696023	5.8997042667548e-06	\\
-3273.046875	9.21340192649073e-06	\\
-3272.06809303977	6.17312732549603e-06	\\
-3271.08931107955	7.27303686302001e-06	\\
-3270.11052911932	6.28473592185866e-06	\\
-3269.13174715909	8.12762327739322e-06	\\
-3268.15296519886	3.98092860014282e-06	\\
-3267.17418323864	8.83533842087064e-06	\\
-3266.19540127841	3.77132191679055e-06	\\
-3265.21661931818	8.14366679430762e-06	\\
-3264.23783735796	5.896319115889e-06	\\
-3263.25905539773	7.27511933645924e-06	\\
-3262.2802734375	7.70378348542392e-06	\\
-3261.30149147727	7.97544534242314e-06	\\
-3260.32270951705	8.58483172141497e-06	\\
-3259.34392755682	7.46520938618689e-06	\\
-3258.36514559659	8.24241200361249e-06	\\
-3257.38636363636	6.38058332918237e-06	\\
-3256.40758167614	6.58610450167089e-06	\\
-3255.42879971591	4.94800954546705e-06	\\
-3254.45001775568	6.47638190431502e-06	\\
-3253.47123579546	6.27393754920069e-06	\\
-3252.49245383523	5.38038127032368e-06	\\
-3251.513671875	5.34980346144889e-06	\\
-3250.53488991477	7.56900304889741e-06	\\
-3249.55610795455	5.38593840326826e-06	\\
-3248.57732599432	6.86715071978326e-06	\\
-3247.59854403409	6.47201133766624e-06	\\
-3246.61976207386	7.02191099656472e-06	\\
-3245.64098011364	5.4023890445851e-06	\\
-3244.66219815341	6.35209212516419e-06	\\
-3243.68341619318	8.80652896629794e-06	\\
-3242.70463423296	4.99770944644396e-06	\\
-3241.72585227273	5.8399363041359e-06	\\
-3240.7470703125	7.41429341354414e-06	\\
-3239.76828835227	6.41501026780166e-06	\\
-3238.78950639205	5.31539093117702e-06	\\
-3237.81072443182	6.67282010671431e-06	\\
-3236.83194247159	5.87315121357321e-06	\\
-3235.85316051136	6.96644191113979e-06	\\
-3234.87437855114	5.65380206501836e-06	\\
-3233.89559659091	7.7988769166354e-06	\\
-3232.91681463068	4.27143515873373e-06	\\
-3231.93803267046	8.42630357298141e-06	\\
-3230.95925071023	5.19444498787262e-06	\\
-3229.98046875	4.63197998282777e-06	\\
-3229.00168678977	3.74226794524826e-06	\\
-3228.02290482955	5.21252061453378e-06	\\
-3227.04412286932	3.44674267910026e-06	\\
-3226.06534090909	8.11493957546307e-06	\\
-3225.08655894886	6.09772358396485e-06	\\
-3224.10777698864	4.53852866545889e-06	\\
-3223.12899502841	4.19828861450633e-06	\\
-3222.15021306818	1.31814296328779e-06	\\
-3221.17143110796	3.38353570221822e-06	\\
-3220.19264914773	4.80400218116881e-06	\\
-3219.2138671875	4.26801218791639e-06	\\
-3218.23508522727	3.78930275975588e-06	\\
-3217.25630326705	5.72424426459969e-06	\\
-3216.27752130682	6.04400741603136e-06	\\
-3215.29873934659	5.25764662041371e-06	\\
-3214.31995738636	3.90845173629924e-06	\\
-3213.34117542614	6.32496101982953e-06	\\
-3212.36239346591	6.29389241809466e-06	\\
-3211.38361150568	4.50996995634303e-06	\\
-3210.40482954546	5.922983937322e-06	\\
-3209.42604758523	5.85501017484608e-06	\\
-3208.447265625	5.30826621119794e-06	\\
-3207.46848366477	7.3289012997337e-06	\\
-3206.48970170455	3.87100036576368e-06	\\
-3205.51091974432	6.38983194150087e-06	\\
-3204.53213778409	5.13027314589624e-06	\\
-3203.55335582386	6.71907155191186e-06	\\
-3202.57457386364	4.00331687122691e-06	\\
-3201.59579190341	7.24295798028006e-06	\\
-3200.61700994318	5.65065333180539e-06	\\
-3199.63822798296	3.35948090814521e-06	\\
-3198.65944602273	6.68029339717283e-06	\\
-3197.6806640625	3.75299305943797e-06	\\
-3196.70188210227	7.62083168257051e-06	\\
-3195.72310014205	3.30159094907509e-06	\\
-3194.74431818182	5.26842358455125e-06	\\
};
\addplot [color=blue,solid,forget plot]
  table[row sep=crcr]{
-3194.74431818182	5.26842358455125e-06	\\
-3193.76553622159	4.14581693859445e-06	\\
-3192.78675426136	4.47924597823485e-06	\\
-3191.80797230114	5.30937663456191e-06	\\
-3190.82919034091	3.86969701534095e-06	\\
-3189.85040838068	5.10850888346366e-06	\\
-3188.87162642046	4.93100652199406e-06	\\
-3187.89284446023	5.4014278979364e-06	\\
-3186.9140625	4.26334923423353e-06	\\
-3185.93528053977	3.74672987438624e-06	\\
-3184.95649857955	3.91493549296007e-06	\\
-3183.97771661932	4.2611333504769e-06	\\
-3182.99893465909	3.92937849069316e-06	\\
-3182.02015269886	3.80215428140113e-06	\\
-3181.04137073864	6.80275632431413e-06	\\
-3180.06258877841	3.88195510588894e-06	\\
-3179.08380681818	6.14632015408303e-06	\\
-3178.10502485796	2.07661331041894e-06	\\
-3177.12624289773	6.20161050068553e-06	\\
-3176.1474609375	4.16890905899354e-06	\\
-3175.16867897727	4.99128864475251e-06	\\
-3174.18989701705	6.37427472030012e-06	\\
-3173.21111505682	2.61730714338528e-06	\\
-3172.23233309659	7.35726866469139e-06	\\
-3171.25355113636	2.89274074733574e-06	\\
-3170.27476917614	3.35548880146266e-06	\\
-3169.29598721591	2.44179472436013e-06	\\
-3168.31720525568	4.83775389975338e-06	\\
-3167.33842329546	6.92386584535704e-07	\\
-3166.35964133523	5.8347712304792e-06	\\
-3165.380859375	2.54558590505924e-06	\\
-3164.40207741477	2.25894235515106e-06	\\
-3163.42329545455	3.504581701879e-06	\\
-3162.44451349432	2.44509192529568e-06	\\
-3161.46573153409	5.20334689195883e-06	\\
-3160.48694957386	4.23528587489164e-06	\\
-3159.50816761364	3.86478533396869e-06	\\
-3158.52938565341	1.93466575315706e-06	\\
-3157.55060369318	4.39503041438791e-06	\\
-3156.57182173296	1.31694150956772e-06	\\
-3155.59303977273	4.62484261884579e-06	\\
-3154.6142578125	4.8384274932039e-06	\\
-3153.63547585227	5.52918833596567e-07	\\
-3152.65669389205	4.8873373069387e-06	\\
-3151.67791193182	4.07656546535735e-06	\\
-3150.69912997159	4.56611065192147e-06	\\
-3149.72034801136	2.11595300079999e-06	\\
-3148.74156605114	5.50477495927385e-06	\\
-3147.76278409091	3.00660776283667e-06	\\
-3146.78400213068	4.24183282357497e-06	\\
-3145.80522017046	4.15039680453441e-06	\\
-3144.82643821023	3.59669263342718e-06	\\
-3143.84765625	4.39922190486844e-06	\\
-3142.86887428977	4.28017040964383e-06	\\
-3141.89009232955	4.64987552474542e-06	\\
-3140.91131036932	3.48070552145457e-06	\\
-3139.93252840909	3.27826903338788e-06	\\
-3138.95374644886	2.2181824934219e-06	\\
-3137.97496448864	3.84024440642592e-06	\\
-3136.99618252841	5.88830456517709e-06	\\
-3136.01740056818	3.15048583968066e-06	\\
-3135.03861860796	3.91332490416465e-06	\\
-3134.05983664773	4.58261838591392e-06	\\
-3133.0810546875	6.55303643369652e-06	\\
-3132.10227272727	3.14842383348496e-06	\\
-3131.12349076705	6.1407940956206e-06	\\
-3130.14470880682	6.63759880024894e-06	\\
-3129.16592684659	7.01780035230176e-06	\\
-3128.18714488636	8.3165836448752e-06	\\
-3127.20836292614	8.2314597083897e-06	\\
-3126.22958096591	4.79715981792659e-06	\\
-3125.25079900568	5.03323834727701e-06	\\
-3124.27201704546	7.18213707472833e-06	\\
-3123.29323508523	4.90283969654908e-06	\\
-3122.314453125	5.06203116678668e-06	\\
-3121.33567116477	4.83206776196961e-06	\\
-3120.35688920455	5.58751328965117e-06	\\
-3119.37810724432	5.62127855370219e-06	\\
-3118.39932528409	4.84325478090718e-06	\\
-3117.42054332386	5.06004826381039e-06	\\
-3116.44176136364	6.13042248394637e-06	\\
-3115.46297940341	4.92181354024062e-06	\\
-3114.48419744318	6.35988451615635e-06	\\
-3113.50541548296	6.09732819536721e-06	\\
-3112.52663352273	3.86130979154733e-06	\\
-3111.5478515625	4.76942534507713e-06	\\
-3110.56906960227	5.24860672565812e-06	\\
-3109.59028764205	3.94338233659148e-06	\\
-3108.61150568182	3.45418633026757e-06	\\
-3107.63272372159	6.5523408539336e-06	\\
-3106.65394176136	6.14230578216623e-06	\\
-3105.67515980114	5.75234955307357e-06	\\
-3104.69637784091	6.89318862244485e-06	\\
-3103.71759588068	6.3752328472931e-06	\\
-3102.73881392046	5.88285614509227e-06	\\
-3101.76003196023	4.03622601535408e-06	\\
-3100.78125	4.13795609168738e-06	\\
-3099.80246803977	5.716255763521e-06	\\
-3098.82368607955	5.60379654193843e-06	\\
-3097.84490411932	3.79965420500559e-06	\\
-3096.86612215909	4.91443653087087e-06	\\
-3095.88734019886	4.72951997772396e-06	\\
-3094.90855823864	2.91023619559124e-06	\\
-3093.92977627841	5.92590631531703e-06	\\
-3092.95099431818	3.91108215047e-06	\\
-3091.97221235796	6.26767565928252e-06	\\
-3090.99343039773	4.80931924058817e-06	\\
-3090.0146484375	4.78895660877867e-06	\\
-3089.03586647727	3.15874983446296e-06	\\
-3088.05708451705	6.13488375111366e-06	\\
-3087.07830255682	6.91260470269919e-06	\\
-3086.09952059659	4.39741355688163e-06	\\
-3085.12073863636	4.48104840137389e-06	\\
-3084.14195667614	6.89975836021229e-06	\\
-3083.16317471591	4.80700168885917e-06	\\
-3082.18439275568	4.3097912681052e-06	\\
-3081.20561079546	7.16794683336276e-06	\\
-3080.22682883523	7.19283563096231e-06	\\
-3079.248046875	4.99117503026635e-06	\\
-3078.26926491477	4.50650761891058e-06	\\
-3077.29048295455	4.03314450671596e-06	\\
-3076.31170099432	5.83976997745681e-06	\\
-3075.33291903409	6.72143386986436e-06	\\
-3074.35413707386	4.63910051189407e-06	\\
-3073.37535511364	4.99819616327353e-06	\\
-3072.39657315341	7.94173973072607e-06	\\
-3071.41779119318	4.05005750045961e-06	\\
-3070.43900923296	4.34366795876617e-06	\\
-3069.46022727273	6.97628990249466e-06	\\
-3068.4814453125	5.13973477305773e-06	\\
-3067.50266335227	7.64314738720353e-06	\\
-3066.52388139205	5.99584015493847e-06	\\
-3065.54509943182	7.34643234563853e-06	\\
-3064.56631747159	6.74089900647425e-06	\\
-3063.58753551136	6.16614101989388e-06	\\
-3062.60875355114	7.62448785756824e-06	\\
-3061.62997159091	4.40797325449266e-06	\\
-3060.65118963068	4.95524585464617e-06	\\
-3059.67240767046	6.38301873149283e-06	\\
-3058.69362571023	5.18059787071521e-06	\\
-3057.71484375	8.99406576059675e-06	\\
-3056.73606178977	4.57672620483507e-06	\\
-3055.75727982955	5.70402873131691e-06	\\
-3054.77849786932	6.55200856719435e-06	\\
-3053.79971590909	7.80983543395229e-06	\\
-3052.82093394886	6.23413952135587e-06	\\
-3051.84215198864	6.91826876059586e-06	\\
-3050.86337002841	6.5539356739255e-06	\\
-3049.88458806818	7.10925875889012e-06	\\
-3048.90580610796	5.44252418188539e-06	\\
-3047.92702414773	7.15231682236398e-06	\\
-3046.9482421875	4.74456073142441e-06	\\
-3045.96946022727	6.30380882610487e-06	\\
-3044.99067826705	7.48443629275991e-06	\\
-3044.01189630682	6.4685169090079e-06	\\
-3043.03311434659	4.94156561625081e-06	\\
-3042.05433238636	6.09295689741792e-06	\\
-3041.07555042614	4.50357429637167e-06	\\
-3040.09676846591	4.76507462967633e-06	\\
-3039.11798650568	4.76570786738601e-06	\\
-3038.13920454546	6.83207274856037e-06	\\
-3037.16042258523	7.55896600664917e-06	\\
-3036.181640625	3.95415933687225e-06	\\
-3035.20285866477	7.11386620666693e-06	\\
-3034.22407670455	4.15233764386129e-06	\\
-3033.24529474432	8.02517934212748e-06	\\
-3032.26651278409	3.26478119336771e-06	\\
-3031.28773082386	3.95786455990871e-06	\\
-3030.30894886364	6.38204936454737e-06	\\
-3029.33016690341	9.76890835918335e-06	\\
-3028.35138494318	6.65704149755269e-06	\\
-3027.37260298296	6.50356365308631e-06	\\
-3026.39382102273	5.11842748206802e-06	\\
-3025.4150390625	8.95968854541193e-06	\\
-3024.43625710227	7.42581627653565e-06	\\
-3023.45747514205	6.75424562872583e-06	\\
-3022.47869318182	6.7511241383729e-06	\\
-3021.49991122159	1.73369035447683e-06	\\
-3020.52112926136	3.75980663615043e-06	\\
-3019.54234730114	4.78746059693251e-06	\\
-3018.56356534091	4.77031186486822e-06	\\
-3017.58478338068	7.28427797428333e-06	\\
-3016.60600142046	4.50677072741703e-06	\\
-3015.62721946023	3.22051895849985e-06	\\
-3014.6484375	6.93497334844756e-06	\\
-3013.66965553977	3.95736566800053e-06	\\
-3012.69087357955	4.14878870285875e-06	\\
-3011.71209161932	4.79666728689112e-06	\\
-3010.73330965909	3.34249448947731e-06	\\
-3009.75452769886	3.71854966701996e-06	\\
-3008.77574573864	6.57463491177955e-06	\\
-3007.79696377841	3.74843769540633e-06	\\
-3006.81818181818	6.26296518425085e-06	\\
-3005.83939985796	5.57595289620855e-06	\\
-3004.86061789773	5.90310017340944e-06	\\
-3003.8818359375	3.55826281260897e-06	\\
-3002.90305397727	3.46430751067959e-06	\\
-3001.92427201705	6.85813192147649e-06	\\
-3000.94549005682	2.23574794677416e-06	\\
-2999.96670809659	6.40525675055424e-06	\\
-2998.98792613636	2.7407189377701e-06	\\
-2998.00914417614	4.17149542361306e-06	\\
-2997.03036221591	7.24558250375745e-06	\\
-2996.05158025568	3.83673608123943e-06	\\
-2995.07279829546	3.60725257510452e-06	\\
-2994.09401633523	5.14795531856298e-06	\\
-2993.115234375	4.17595292668247e-06	\\
-2992.13645241477	4.07656011686612e-06	\\
-2991.15767045455	3.83309843970224e-06	\\
-2990.17888849432	5.21034120045495e-06	\\
-2989.20010653409	8.01787406254111e-06	\\
-2988.22132457386	6.25624310488878e-06	\\
-2987.24254261364	4.51312584341107e-06	\\
-2986.26376065341	3.68583354080117e-06	\\
-2985.28497869318	4.34361036684229e-06	\\
-2984.30619673296	5.85557524334713e-06	\\
-2983.32741477273	5.28662885067508e-06	\\
-2982.3486328125	6.63718397978908e-06	\\
-2981.36985085227	6.6624671804147e-06	\\
-2980.39106889205	4.01169831846422e-06	\\
-2979.41228693182	3.64103329898829e-06	\\
-2978.43350497159	6.56275685653722e-06	\\
-2977.45472301136	5.24876827769003e-06	\\
-2976.47594105114	3.09955232241895e-06	\\
-2975.49715909091	5.42688352851547e-06	\\
-2974.51837713068	2.59331451040508e-06	\\
-2973.53959517046	3.1910029726617e-06	\\
-2972.56081321023	5.19828171191463e-06	\\
-2971.58203125	5.25940643803338e-06	\\
-2970.60324928977	4.48308426521495e-06	\\
-2969.62446732955	5.59483004811013e-06	\\
-2968.64568536932	4.92031879789632e-06	\\
-2967.66690340909	4.96185575705528e-06	\\
-2966.68812144886	5.52752037377181e-06	\\
-2965.70933948864	6.01948733784033e-06	\\
-2964.73055752841	3.85003314293153e-06	\\
-2963.75177556818	3.44401737823946e-06	\\
-2962.77299360796	4.78991878222869e-06	\\
-2961.79421164773	4.31291612438827e-06	\\
-2960.8154296875	4.37946934566222e-06	\\
-2959.83664772727	4.68990286927971e-06	\\
-2958.85786576705	4.72991655727233e-06	\\
-2957.87908380682	5.22037402065988e-06	\\
-2956.90030184659	3.31947403444169e-06	\\
-2955.92151988636	3.29564801074687e-06	\\
-2954.94273792614	2.50797806199041e-06	\\
-2953.96395596591	2.70241878333915e-06	\\
-2952.98517400568	3.8798952534642e-06	\\
-2952.00639204546	4.52144252413908e-06	\\
-2951.02761008523	3.05270818269617e-06	\\
-2950.048828125	4.844544800359e-06	\\
-2949.07004616477	5.48659209099511e-06	\\
-2948.09126420455	5.11049428954381e-06	\\
-2947.11248224432	5.58954472096685e-06	\\
-2946.13370028409	3.28286142459796e-06	\\
-2945.15491832386	6.66049527862821e-06	\\
-2944.17613636364	5.46291311776296e-06	\\
-2943.19735440341	3.11544554910849e-06	\\
-2942.21857244318	4.94815518732998e-06	\\
-2941.23979048296	3.84401733424272e-06	\\
-2940.26100852273	2.45216038571283e-06	\\
-2939.2822265625	4.15964384854581e-06	\\
-2938.30344460227	3.99989255694006e-06	\\
-2937.32466264205	5.09640106753759e-06	\\
-2936.34588068182	5.10661221411358e-06	\\
-2935.36709872159	7.48682290166127e-06	\\
-2934.38831676136	3.30588604586103e-06	\\
-2933.40953480114	3.28564659675688e-06	\\
-2932.43075284091	3.52979009744975e-06	\\
-2931.45197088068	3.91757781208955e-06	\\
-2930.47318892046	3.16049207536663e-06	\\
-2929.49440696023	5.46910199624583e-06	\\
-2928.515625	4.18838468315718e-06	\\
-2927.53684303977	4.3928195794944e-06	\\
-2926.55806107955	5.17341349752293e-06	\\
-2925.57927911932	4.23508907657349e-06	\\
-2924.60049715909	4.89496171366531e-06	\\
-2923.62171519886	4.78729973692923e-06	\\
-2922.64293323864	5.48564533852651e-06	\\
-2921.66415127841	2.00822218720313e-06	\\
-2920.68536931818	1.97618330670144e-06	\\
-2919.70658735796	4.41970124587756e-06	\\
-2918.72780539773	2.92487026976474e-06	\\
-2917.7490234375	5.29696196939571e-06	\\
-2916.77024147727	6.06062596616099e-06	\\
-2915.79145951705	5.23728218121983e-06	\\
-2914.81267755682	4.17560744258949e-06	\\
-2913.83389559659	5.41954042483807e-06	\\
-2912.85511363636	3.86398810959201e-06	\\
-2911.87633167614	3.56962565778673e-06	\\
-2910.89754971591	2.94732468990434e-06	\\
-2909.91876775568	5.44392904901348e-06	\\
-2908.93998579546	4.38423101610561e-06	\\
-2907.96120383523	4.34301164002499e-06	\\
-2906.982421875	7.62821366026552e-06	\\
-2906.00363991477	5.41452697692217e-06	\\
-2905.02485795455	4.44937112285067e-06	\\
-2904.04607599432	4.11882712930982e-06	\\
-2903.06729403409	6.4361986385113e-06	\\
-2902.08851207386	6.79298054295202e-06	\\
-2901.10973011364	1.67821786853684e-06	\\
-2900.13094815341	3.30605503038348e-06	\\
-2899.15216619318	2.58397604649771e-06	\\
-2898.17338423296	5.03838235510546e-06	\\
-2897.19460227273	3.42832990708021e-06	\\
-2896.2158203125	4.49474218965829e-06	\\
-2895.23703835227	3.95532203589187e-06	\\
-2894.25825639205	3.64553546186186e-06	\\
-2893.27947443182	4.59441439193861e-06	\\
-2892.30069247159	3.49259102268966e-06	\\
-2891.32191051136	3.8671092843816e-06	\\
-2890.34312855114	3.54364437954377e-06	\\
-2889.36434659091	4.98920494662752e-06	\\
-2888.38556463068	5.44631775694933e-06	\\
-2887.40678267046	2.54106939512331e-06	\\
-2886.42800071023	5.27067323555189e-06	\\
-2885.44921875	3.02902380829889e-06	\\
-2884.47043678977	2.69365196020493e-06	\\
-2883.49165482955	2.67218524875853e-06	\\
-2882.51287286932	2.04967306841465e-06	\\
-2881.53409090909	2.44744225685431e-06	\\
-2880.55530894886	8.03927698675367e-07	\\
-2879.57652698864	2.05408395302597e-06	\\
-2878.59774502841	3.87743066263443e-06	\\
-2877.61896306818	2.28246609459498e-06	\\
-2876.64018110796	2.9497489876962e-06	\\
-2875.66139914773	5.28485361692664e-06	\\
-2874.6826171875	4.77183524178971e-06	\\
-2873.70383522727	3.70726635832435e-06	\\
-2872.72505326705	3.05980155120563e-06	\\
-2871.74627130682	4.99846922843471e-06	\\
-2870.76748934659	3.47073919790119e-06	\\
-2869.78870738636	5.39473416599955e-06	\\
-2868.80992542614	2.79377730234606e-06	\\
-2867.83114346591	3.34978274246577e-06	\\
-2866.85236150568	6.04919430234132e-06	\\
-2865.87357954546	8.25865559303358e-06	\\
-2864.89479758523	1.65351596450654e-06	\\
-2863.916015625	5.59833466334693e-06	\\
-2862.93723366477	2.77105042165666e-06	\\
-2861.95845170455	4.35052409132747e-06	\\
-2860.97966974432	2.15715583078548e-06	\\
-2860.00088778409	4.14739956473991e-06	\\
-2859.02210582386	3.29516435970086e-06	\\
-2858.04332386364	4.77157149405267e-06	\\
-2857.06454190341	1.49010761969156e-06	\\
-2856.08575994318	4.15647646013232e-06	\\
-2855.10697798296	2.62618488771587e-06	\\
-2854.12819602273	2.86318881054772e-06	\\
-2853.1494140625	8.58395836151411e-07	\\
-2852.17063210227	2.79732543971194e-06	\\
-2851.19185014205	1.90131841291536e-06	\\
-2850.21306818182	3.80876490912884e-06	\\
-2849.23428622159	3.63521956374567e-06	\\
-2848.25550426136	2.45475214621159e-06	\\
-2847.27672230114	2.73498388587658e-06	\\
-2846.29794034091	4.05553100846361e-06	\\
-2845.31915838068	2.29416710915538e-06	\\
-2844.34037642046	2.6698091836681e-06	\\
-2843.36159446023	3.98644303517036e-06	\\
-2842.3828125	2.8693626330884e-06	\\
-2841.40403053977	2.86392063973937e-06	\\
-2840.42524857955	3.31605206882318e-06	\\
-2839.44646661932	4.4732839956391e-06	\\
-2838.46768465909	3.3434869427996e-06	\\
-2837.48890269886	4.80974071573523e-06	\\
-2836.51012073864	3.37146832475975e-06	\\
-2835.53133877841	4.41154723073322e-06	\\
-2834.55255681818	1.46085813543219e-06	\\
-2833.57377485796	4.61100630473622e-06	\\
-2832.59499289773	3.46811766398455e-06	\\
-2831.6162109375	5.34785859087092e-06	\\
-2830.63742897727	4.04516392038333e-06	\\
-2829.65864701705	3.81292278948782e-06	\\
-2828.67986505682	5.1910715616594e-06	\\
-2827.70108309659	3.37845963880854e-06	\\
-2826.72230113636	4.27507175115144e-06	\\
-2825.74351917614	3.60306567645143e-06	\\
-2824.76473721591	3.59827982273e-06	\\
-2823.78595525568	3.97817927657384e-06	\\
-2822.80717329545	4.69076383249551e-06	\\
-2821.82839133523	2.79233689233618e-06	\\
-2820.849609375	2.36146531941351e-06	\\
-2819.87082741477	3.39534979297913e-06	\\
-2818.89204545455	3.09877272480753e-07	\\
-2817.91326349432	1.79799230299238e-06	\\
-2816.93448153409	1.42124681175411e-06	\\
-2815.95569957386	6.52547567769548e-07	\\
-2814.97691761364	1.77121001143617e-06	\\
-2813.99813565341	1.47575865883504e-06	\\
-2813.01935369318	1.11134853469604e-06	\\
-2812.04057173295	3.41824661840372e-07	\\
-2811.06178977273	1.40371762113629e-06	\\
-2810.0830078125	3.35927737466208e-06	\\
-2809.10422585227	2.73401986561309e-06	\\
-2808.12544389205	2.38604592423503e-06	\\
-2807.14666193182	1.65402393974081e-06	\\
-2806.16787997159	1.46039807683165e-06	\\
-2805.18909801136	1.86273418190857e-06	\\
-2804.21031605114	8.93601172735078e-07	\\
-2803.23153409091	2.10922820488248e-06	\\
-2802.25275213068	2.62394854801383e-06	\\
-2801.27397017045	1.90590019104556e-06	\\
-2800.29518821023	1.66620154450953e-06	\\
-2799.31640625	6.40393828085902e-07	\\
-2798.33762428977	2.47690142333158e-06	\\
-2797.35884232955	3.09032463285475e-06	\\
-2796.38006036932	1.36600481763574e-06	\\
-2795.40127840909	2.13660905036377e-06	\\
-2794.42249644886	2.62402206308109e-06	\\
-2793.44371448864	1.63285502503068e-06	\\
-2792.46493252841	1.50707857172838e-06	\\
-2791.48615056818	1.52155058698765e-06	\\
-2790.50736860795	3.5166520348458e-06	\\
-2789.52858664773	5.50707647691607e-07	\\
-2788.5498046875	3.24132264270955e-06	\\
-2787.57102272727	1.30111118129765e-06	\\
-2786.59224076705	1.20544273020633e-06	\\
-2785.61345880682	1.19976750433772e-06	\\
-2784.63467684659	1.3472694814085e-06	\\
-2783.65589488636	1.47211144352771e-06	\\
-2782.67711292614	2.50660289748446e-06	\\
-2781.69833096591	1.30969552001911e-06	\\
-2780.71954900568	7.48851336311492e-07	\\
-2779.74076704545	3.7800327259833e-06	\\
-2778.76198508523	2.41425873615509e-06	\\
-2777.783203125	4.81457742157517e-07	\\
-2776.80442116477	1.56503764002716e-06	\\
-2775.82563920455	1.58823957386081e-06	\\
-2774.84685724432	3.76267962968488e-06	\\
-2773.86807528409	2.13724197103523e-06	\\
-2772.88929332386	2.75425049358554e-06	\\
-2771.91051136364	5.02453330557031e-07	\\
-2770.93172940341	2.34776478652176e-06	\\
-2769.95294744318	5.59565495064372e-07	\\
-2768.97416548295	3.97987355237858e-06	\\
-2767.99538352273	3.65889565349057e-06	\\
-2767.0166015625	2.94343263591024e-06	\\
-2766.03781960227	4.13455050868395e-06	\\
-2765.05903764205	4.80991741704654e-06	\\
-2764.08025568182	3.32816655179175e-06	\\
-2763.10147372159	3.51653307309765e-06	\\
-2762.12269176136	3.43420896592292e-06	\\
-2761.14390980114	2.14809136085652e-06	\\
-2760.16512784091	3.27505259061264e-06	\\
-2759.18634588068	6.48150705117231e-06	\\
-2758.20756392045	3.7857086885082e-06	\\
-2757.22878196023	2.78272137545811e-06	\\
-2756.25	2.27665449064888e-06	\\
-2755.27121803977	2.91489806344309e-06	\\
-2754.29243607955	2.07591322996028e-06	\\
-2753.31365411932	6.56936858777892e-06	\\
-2752.33487215909	4.37866826482226e-06	\\
-2751.35609019886	2.41840681220149e-06	\\
-2750.37730823864	1.68329149694193e-06	\\
-2749.39852627841	2.64957686461998e-06	\\
-2748.41974431818	3.53827328462157e-06	\\
-2747.44096235795	6.56540359293604e-06	\\
-2746.46218039773	6.8773818785314e-06	\\
-2745.4833984375	4.62905209203989e-06	\\
-2744.50461647727	8.24480572213219e-07	\\
-2743.52583451705	4.18711277068443e-06	\\
-2742.54705255682	5.69526340498197e-06	\\
-2741.56827059659	6.24178338298713e-06	\\
-2740.58948863636	5.82802545830845e-06	\\
-2739.61070667614	2.72797881208903e-06	\\
-2738.63192471591	5.25993073408675e-06	\\
-2737.65314275568	7.49695584247444e-06	\\
-2736.67436079545	4.89833366769579e-06	\\
-2735.69557883523	3.94321059626331e-06	\\
-2734.716796875	3.66961420925272e-06	\\
-2733.73801491477	4.69601498828271e-06	\\
-2732.75923295455	6.22400194055147e-06	\\
-2731.78045099432	7.93584617578885e-06	\\
-2730.80166903409	7.86563897858116e-06	\\
-2729.82288707386	1.83193723303623e-06	\\
-2728.84410511364	2.02716576113554e-06	\\
-2727.86532315341	6.91220750456317e-06	\\
-2726.88654119318	8.00386085814599e-06	\\
-2725.90775923295	7.49361844763953e-06	\\
-2724.92897727273	6.09635418088786e-06	\\
-2723.9501953125	6.30520226203253e-06	\\
-2722.97141335227	3.95413675455279e-06	\\
-2721.99263139205	7.91249333647524e-06	\\
-2721.01384943182	9.15768973332119e-06	\\
-2720.03506747159	1.09937604651641e-05	\\
-2719.05628551136	5.07868752460706e-06	\\
-2718.07750355114	1.03720489809125e-05	\\
-2717.09872159091	8.04541703143695e-06	\\
-2716.11993963068	8.6943346222023e-06	\\
-2715.14115767045	5.20732800295995e-06	\\
-2714.16237571023	6.56174353791964e-06	\\
-2713.18359375	5.2718117078937e-06	\\
-2712.20481178977	8.67067257322456e-06	\\
-2711.22602982955	7.030740942015e-06	\\
-2710.24724786932	5.32101088562841e-06	\\
-2709.26846590909	6.8379980598573e-06	\\
-2708.28968394886	5.7130239051283e-06	\\
-2707.31090198864	4.00703277020468e-06	\\
-2706.33212002841	7.94040389189657e-06	\\
-2705.35333806818	6.12401990553033e-06	\\
-2704.37455610795	8.05266045970128e-06	\\
-2703.39577414773	5.40780451245773e-06	\\
-2702.4169921875	6.61823282256381e-06	\\
-2701.43821022727	6.06485397629639e-06	\\
-2700.45942826705	9.80048438810412e-06	\\
-2699.48064630682	5.92164507868871e-06	\\
-2698.50186434659	4.37319468335196e-06	\\
-2697.52308238636	5.23039346413709e-06	\\
-2696.54430042614	2.72518386293727e-06	\\
-2695.56551846591	8.77499759143943e-06	\\
-2694.58673650568	7.56896044745794e-06	\\
-2693.60795454545	7.82209947606149e-06	\\
-2692.62917258523	8.35908202828073e-06	\\
-2691.650390625	4.96089894590821e-06	\\
-2690.67160866477	7.69409706866232e-06	\\
-2689.69282670455	6.98093819993718e-06	\\
-2688.71404474432	6.9358071595968e-06	\\
-2687.73526278409	9.22428299084042e-06	\\
-2686.75648082386	6.40124384284549e-06	\\
-2685.77769886364	7.74183037863941e-06	\\
-2684.79891690341	6.49056178258718e-06	\\
-2683.82013494318	5.47135369597753e-06	\\
-2682.84135298295	8.50379433119764e-06	\\
-2681.86257102273	1.06591803110307e-05	\\
-2680.8837890625	7.74775665210904e-06	\\
-2679.90500710227	9.97781631097798e-06	\\
-2678.92622514205	7.52701102599433e-06	\\
-2677.94744318182	1.04967048800068e-05	\\
-2676.96866122159	7.13892848178111e-06	\\
-2675.98987926136	9.66984964070978e-06	\\
-2675.01109730114	5.90860212740751e-06	\\
-2674.03231534091	9.25553915933191e-06	\\
-2673.05353338068	1.29141189350765e-05	\\
-2672.07475142045	8.44909776478114e-06	\\
-2671.09596946023	9.23235812526987e-06	\\
-2670.1171875	1.00752855817841e-05	\\
-2669.13840553977	1.16768454356727e-05	\\
-2668.15962357955	9.01703957702175e-06	\\
-2667.18084161932	8.6265697880826e-06	\\
-2666.20205965909	3.56937233047156e-06	\\
-2665.22327769886	6.5140299586603e-06	\\
-2664.24449573864	7.28067630878025e-06	\\
-2663.26571377841	9.14232366385177e-06	\\
-2662.28693181818	7.21071771030407e-06	\\
-2661.30814985795	8.82258248865961e-06	\\
-2660.32936789773	7.33246871490005e-06	\\
-2659.3505859375	9.00279402269212e-06	\\
-2658.37180397727	9.42011054122724e-06	\\
-2657.39302201705	8.12528517946348e-06	\\
-2656.41424005682	1.01195304852176e-05	\\
-2655.43545809659	8.8264274115568e-06	\\
-2654.45667613636	5.37920453335535e-06	\\
-2653.47789417614	6.90048903306781e-06	\\
-2652.49911221591	6.80423806093495e-06	\\
-2651.52033025568	7.49209075202238e-06	\\
-2650.54154829545	7.60204227025801e-06	\\
-2649.56276633523	7.46600078214287e-06	\\
-2648.583984375	9.03832709681233e-06	\\
-2647.60520241477	9.92637314508742e-06	\\
-2646.62642045455	6.99134210342693e-06	\\
-2645.64763849432	1.1189907202075e-05	\\
-2644.66885653409	8.24328732263467e-06	\\
-2643.69007457386	7.72724030090873e-06	\\
-2642.71129261364	6.82545181407289e-06	\\
-2641.73251065341	9.55873463304659e-06	\\
-2640.75372869318	8.53381174503931e-06	\\
-2639.77494673295	1.36101262168017e-05	\\
-2638.79616477273	1.01727392999639e-05	\\
-2637.8173828125	1.27409486452016e-05	\\
-2636.83860085227	7.58726662456945e-06	\\
-2635.85981889205	8.3845175803049e-06	\\
-2634.88103693182	6.37583961681608e-06	\\
-2633.90225497159	7.13547865450581e-06	\\
-2632.92347301136	1.25483221312679e-05	\\
-2631.94469105114	1.00665025509934e-05	\\
-2630.96590909091	1.11178167545335e-05	\\
-2629.98712713068	5.88418890548162e-06	\\
-2629.00834517045	7.37743733290187e-06	\\
-2628.02956321023	7.34974537688955e-06	\\
-2627.05078125	4.62344119762142e-06	\\
-2626.07199928977	7.45014755649668e-06	\\
-2625.09321732955	9.54503977001142e-06	\\
-2624.11443536932	9.00450995499237e-06	\\
-2623.13565340909	7.05004767081643e-06	\\
-2622.15687144886	9.2233715115999e-06	\\
-2621.17808948864	7.95875299463989e-06	\\
-2620.19930752841	3.75817220000155e-06	\\
-2619.22052556818	9.23131635188674e-06	\\
-2618.24174360795	4.85198223905322e-06	\\
-2617.26296164773	5.47532164138058e-06	\\
-2616.2841796875	1.05143713663192e-05	\\
-2615.30539772727	7.22588366185077e-06	\\
-2614.32661576705	8.15700064552104e-06	\\
-2613.34783380682	8.45364341266012e-06	\\
-2612.36905184659	5.32665585678252e-06	\\
-2611.39026988636	4.65573351810625e-06	\\
-2610.41148792614	6.8863595902438e-06	\\
-2609.43270596591	8.58195842274819e-06	\\
-2608.45392400568	6.31111488540669e-06	\\
-2607.47514204545	7.19461446884648e-06	\\
-2606.49636008523	3.83672692841114e-06	\\
-2605.517578125	4.17952480942147e-06	\\
-2604.53879616477	2.88538297795258e-06	\\
-2603.56001420455	8.62456727506564e-06	\\
-2602.58123224432	1.1033075454989e-05	\\
-2601.60245028409	9.34452172307638e-06	\\
-2600.62366832386	8.65338842041828e-06	\\
-2599.64488636364	6.80218617791554e-06	\\
-2598.66610440341	6.15571980497798e-06	\\
-2597.68732244318	7.77485508207652e-06	\\
-2596.70854048295	5.61574830801038e-06	\\
-2595.72975852273	6.08157206795503e-06	\\
-2594.7509765625	5.86504377967524e-06	\\
-2593.77219460227	5.52954525945763e-06	\\
-2592.79341264205	1.06534342956051e-05	\\
-2591.81463068182	9.33603217192033e-06	\\
-2590.83584872159	1.07332580761661e-05	\\
-2589.85706676136	9.3433123586389e-06	\\
-2588.87828480114	6.52389355979953e-06	\\
-2587.89950284091	1.20822610193775e-05	\\
-2586.92072088068	9.39603793213243e-06	\\
-2585.94193892045	1.07253245520665e-05	\\
-2584.96315696023	8.97775252252712e-06	\\
-2583.984375	7.74161008673369e-06	\\
-2583.00559303977	1.12858115885358e-05	\\
-2582.02681107955	1.17902823165449e-05	\\
-2581.04802911932	1.03887793227839e-05	\\
-2580.06924715909	1.1036659948732e-05	\\
-2579.09046519886	8.11516907886775e-06	\\
-2578.11168323864	1.04755127513138e-05	\\
-2577.13290127841	1.2523850446189e-05	\\
-2576.15411931818	1.25254865263412e-05	\\
-2575.17533735795	1.33487136327705e-05	\\
-2574.19655539773	1.10594948924882e-05	\\
-2573.2177734375	1.16375271478506e-05	\\
-2572.23899147727	8.02095011434817e-06	\\
-2571.26020951705	9.46869165727952e-06	\\
-2570.28142755682	1.3503916523033e-05	\\
-2569.30264559659	1.13553096664194e-05	\\
-2568.32386363636	1.35157686585148e-05	\\
-2567.34508167614	1.25461450286407e-05	\\
-2566.36629971591	7.84201601104799e-06	\\
-2565.38751775568	1.02901656298179e-05	\\
-2564.40873579545	8.53230731649721e-06	\\
-2563.42995383523	1.42429325036642e-05	\\
-2562.451171875	1.4567312498378e-05	\\
-2561.47238991477	1.30851966964052e-05	\\
-2560.49360795455	1.124413739682e-05	\\
-2559.51482599432	1.06200456341965e-05	\\
-2558.53604403409	9.19258307718049e-06	\\
-2557.55726207386	9.81679793928961e-06	\\
-2556.57848011364	1.3134185276538e-05	\\
-2555.59969815341	1.30016508203182e-05	\\
-2554.62091619318	8.77869837235778e-06	\\
-2553.64213423295	9.18332414808979e-06	\\
-2552.66335227273	9.6403291415825e-06	\\
-2551.6845703125	9.5751769541088e-06	\\
-2550.70578835227	9.70804790209553e-06	\\
-2549.72700639205	1.20569360935071e-05	\\
-2548.74822443182	1.25775352959022e-05	\\
-2547.76944247159	1.13461533904966e-05	\\
-2546.79066051136	9.61255056679938e-06	\\
-2545.81187855114	9.26662352511788e-06	\\
-2544.83309659091	1.36779978049829e-05	\\
-2543.85431463068	9.5786792017844e-06	\\
-2542.87553267045	1.2687587879665e-05	\\
-2541.89675071023	1.30828713388349e-05	\\
-2540.91796875	1.07720467294631e-05	\\
-2539.93918678977	1.23630684081609e-05	\\
-2538.96040482955	1.13465476893299e-05	\\
-2537.98162286932	1.31451801263953e-05	\\
-2537.00284090909	1.16139737771759e-05	\\
-2536.02405894886	1.0513328742645e-05	\\
-2535.04527698864	1.02845563306765e-05	\\
-2534.06649502841	9.60237978652877e-06	\\
-2533.08771306818	1.05227492732501e-05	\\
-2532.10893110795	1.13737414641285e-05	\\
-2531.13014914773	1.28935970238767e-05	\\
-2530.1513671875	7.62654880391048e-06	\\
-2529.17258522727	1.23915070475643e-05	\\
-2528.19380326705	9.65742098643429e-06	\\
-2527.21502130682	1.14341198801488e-05	\\
-2526.23623934659	1.58419435920574e-05	\\
-2525.25745738636	1.400433818524e-05	\\
-2524.27867542614	1.21591197194834e-05	\\
-2523.29989346591	1.2295125625366e-05	\\
-2522.32111150568	1.05835599204902e-05	\\
-2521.34232954545	1.36576120359163e-05	\\
-2520.36354758523	1.00100452374513e-05	\\
-2519.384765625	1.522269411294e-05	\\
-2518.40598366477	1.3925586576509e-05	\\
-2517.42720170455	1.37903021093818e-05	\\
-2516.44841974432	1.90273789401122e-05	\\
-2515.46963778409	1.10863219203196e-05	\\
-2514.49085582386	1.56631044305564e-05	\\
-2513.51207386364	1.24551858129238e-05	\\
-2512.53329190341	1.23611407461894e-05	\\
-2511.55450994318	1.34724288620022e-05	\\
-2510.57572798295	1.46821389791948e-05	\\
-2509.59694602273	1.05248554120014e-05	\\
-2508.6181640625	1.63605658840124e-05	\\
-2507.63938210227	1.70037117257888e-05	\\
-2506.66060014205	1.65982949149232e-05	\\
-2505.68181818182	1.67750002510302e-05	\\
-2504.70303622159	1.42638784803625e-05	\\
-2503.72425426136	9.92312637693691e-06	\\
-2502.74547230114	1.88065597316314e-05	\\
-2501.76669034091	1.05846449279785e-05	\\
-2500.78790838068	1.60830096240532e-05	\\
-2499.80912642045	1.23423961563045e-05	\\
-2498.83034446023	1.10480930740472e-05	\\
-2497.8515625	1.0444532769733e-05	\\
-2496.87278053977	1.17522134706582e-05	\\
-2495.89399857955	1.19387758801284e-05	\\
-2494.91521661932	1.28898515351656e-05	\\
-2493.93643465909	1.75987049032965e-05	\\
-2492.95765269886	1.28843092311765e-05	\\
-2491.97887073864	1.46182295581329e-05	\\
-2491.00008877841	1.3220007479457e-05	\\
-2490.02130681818	1.23030040593245e-05	\\
-2489.04252485795	1.27953296217664e-05	\\
-2488.06374289773	1.37465082197053e-05	\\
-2487.0849609375	9.64976796759365e-06	\\
-2486.10617897727	1.38397577004886e-05	\\
-2485.12739701705	1.09197161563871e-05	\\
-2484.14861505682	1.33330862992937e-05	\\
-2483.16983309659	1.14551401863731e-05	\\
-2482.19105113636	1.20392357919938e-05	\\
-2481.21226917614	1.07744491572926e-05	\\
-2480.23348721591	1.50070917796501e-05	\\
-2479.25470525568	1.02954347512395e-05	\\
-2478.27592329545	1.02220240638125e-05	\\
-2477.29714133523	1.37772888744459e-05	\\
-2476.318359375	1.11017399028128e-05	\\
-2475.33957741477	1.308364749912e-05	\\
-2474.36079545455	8.38221612499108e-06	\\
-2473.38201349432	1.17036668864691e-05	\\
-2472.40323153409	1.23944828718623e-05	\\
-2471.42444957386	9.94744716261625e-06	\\
-2470.44566761364	1.18802341328419e-05	\\
-2469.46688565341	1.12075605344456e-05	\\
-2468.48810369318	8.14327497826777e-06	\\
-2467.50932173295	1.38842205787105e-05	\\
-2466.53053977273	1.30681827758706e-05	\\
-2465.5517578125	1.00800621711629e-05	\\
-2464.57297585227	1.17634986731376e-05	\\
-2463.59419389205	1.21556947848075e-05	\\
-2462.61541193182	1.16969945313014e-05	\\
-2461.63662997159	1.33824512684582e-05	\\
-2460.65784801136	8.70161457290056e-06	\\
-2459.67906605114	1.55645163361909e-05	\\
-2458.70028409091	8.98747441483013e-06	\\
-2457.72150213068	9.32539439331602e-06	\\
-2456.74272017045	9.00213167101684e-06	\\
-2455.76393821023	1.37677789210611e-05	\\
-2454.78515625	1.34046428029915e-05	\\
-2453.80637428977	1.39850108329484e-05	\\
-2452.82759232955	1.48605433605499e-05	\\
-2451.84881036932	1.3720061605075e-05	\\
-2450.87002840909	1.61231556439175e-05	\\
-2449.89124644886	1.6719173156617e-05	\\
-2448.91246448864	1.6062372435979e-05	\\
-2447.93368252841	1.46722351080737e-05	\\
-2446.95490056818	1.12547014444918e-05	\\
-2445.97611860795	1.70652671880259e-05	\\
-2444.99733664773	1.0912489210281e-05	\\
-2444.0185546875	1.70730294238341e-05	\\
-2443.03977272727	1.30327660954118e-05	\\
-2442.06099076705	1.36705843624822e-05	\\
-2441.08220880682	1.48879304313828e-05	\\
-2440.10342684659	1.59353532551997e-05	\\
-2439.12464488636	1.25183623596444e-05	\\
-2438.14586292614	1.90804867380124e-05	\\
-2437.16708096591	1.4586181047714e-05	\\
-2436.18829900568	1.42125100692022e-05	\\
-2435.20951704545	1.46165291608628e-05	\\
-2434.23073508523	1.40022049188288e-05	\\
-2433.251953125	1.04935851784015e-05	\\
-2432.27317116477	1.45130064592897e-05	\\
-2431.29438920455	1.51541863818808e-05	\\
-2430.31560724432	1.32795262760411e-05	\\
-2429.33682528409	1.14409187342529e-05	\\
-2428.35804332386	1.01203793508877e-05	\\
-2427.37926136364	1.17520720362205e-05	\\
-2426.40047940341	1.210040790732e-05	\\
-2425.42169744318	1.05310684632227e-05	\\
-2424.44291548295	1.31321428684912e-05	\\
-2423.46413352273	1.20920822456478e-05	\\
-2422.4853515625	8.83835131329433e-06	\\
-2421.50656960227	1.31925207863567e-05	\\
-2420.52778764205	7.00633643185449e-06	\\
-2419.54900568182	1.50222321779968e-05	\\
-2418.57022372159	1.28104623689339e-05	\\
-2417.59144176136	1.25086257472886e-05	\\
-2416.61265980114	1.20873787276672e-05	\\
-2415.63387784091	1.43341497486834e-05	\\
-2414.65509588068	1.12714092780936e-05	\\
-2413.67631392045	9.96677116493869e-06	\\
-2412.69753196023	1.06979606514353e-05	\\
-2411.71875	1.0674730118919e-05	\\
-2410.73996803977	1.29100671309624e-05	\\
-2409.76118607955	1.38248866373234e-05	\\
-2408.78240411932	1.21126659962943e-05	\\
-2407.80362215909	1.13216990552699e-05	\\
-2406.82484019886	1.25366098174034e-05	\\
-2405.84605823864	1.60341006385682e-05	\\
-2404.86727627841	1.30339417285951e-05	\\
-2403.88849431818	1.53232785997131e-05	\\
-2402.90971235795	1.32368545734626e-05	\\
-2401.93093039773	1.11761306698093e-05	\\
-2400.9521484375	1.15972672045454e-05	\\
-2399.97336647727	1.15406661960588e-05	\\
-2398.99458451705	1.33815020146956e-05	\\
-2398.01580255682	1.3405314544094e-05	\\
-2397.03702059659	1.17583488758016e-05	\\
-2396.05823863636	1.34794837314849e-05	\\
-2395.07945667614	1.07089221484535e-05	\\
-2394.10067471591	1.18543208739252e-05	\\
-2393.12189275568	1.01265523008739e-05	\\
-2392.14311079545	1.31929963249887e-05	\\
-2391.16432883523	1.05475931696297e-05	\\
-2390.185546875	1.11518890778064e-05	\\
-2389.20676491477	1.18071724302909e-05	\\
-2388.22798295455	1.2900899594381e-05	\\
-2387.24920099432	1.18500193168334e-05	\\
-2386.27041903409	9.15098141480253e-06	\\
-2385.29163707386	1.04029311164239e-05	\\
-2384.31285511364	1.11940922186062e-05	\\
-2383.33407315341	1.04932374963135e-05	\\
-2382.35529119318	8.41217933850615e-06	\\
-2381.37650923295	8.22697537600947e-06	\\
-2380.39772727273	8.79600385537138e-06	\\
-2379.4189453125	9.68136983306031e-06	\\
-2378.44016335227	9.65377268042556e-06	\\
-2377.46138139205	1.13867327662863e-05	\\
-2376.48259943182	1.01012026951165e-05	\\
-2375.50381747159	7.16009935968107e-06	\\
-2374.52503551136	1.11811697197271e-05	\\
-2373.54625355114	6.90753420694246e-06	\\
-2372.56747159091	9.59406555279268e-06	\\
-2371.58868963068	1.05741110134304e-05	\\
-2370.60990767045	1.13337276612157e-05	\\
-2369.63112571023	7.14851979089286e-06	\\
-2368.65234375	8.52594685351857e-06	\\
-2367.67356178977	1.19333418626326e-05	\\
-2366.69477982955	9.55301396656965e-06	\\
-2365.71599786932	9.76067333604412e-06	\\
-2364.73721590909	1.2622802560616e-05	\\
-2363.75843394886	1.14348667979891e-05	\\
-2362.77965198864	1.31048421121442e-05	\\
-2361.80087002841	1.23758546492067e-05	\\
-2360.82208806818	1.06562167964211e-05	\\
-2359.84330610795	1.15438924253775e-05	\\
-2358.86452414773	9.07744697881029e-06	\\
-2357.8857421875	1.27049430734602e-05	\\
-2356.90696022727	1.11481738998377e-05	\\
-2355.92817826705	1.52930209867912e-05	\\
-2354.94939630682	1.39265782939805e-05	\\
-2353.97061434659	1.19515985194932e-05	\\
-2352.99183238636	1.72939738612485e-05	\\
-2352.01305042614	1.11075953220586e-05	\\
-2351.03426846591	9.46069460473123e-06	\\
-2350.05548650568	1.87022827318072e-05	\\
-2349.07670454545	1.31822755949023e-05	\\
-2348.09792258523	1.13811465603336e-05	\\
-2347.119140625	1.29704983667432e-05	\\
-2346.14035866477	1.09246166404823e-05	\\
-2345.16157670455	1.06402594861197e-05	\\
-2344.18279474432	1.64394304304421e-05	\\
-2343.20401278409	1.65536043675439e-05	\\
-2342.22523082386	8.5259705756403e-06	\\
-2341.24644886364	1.56655104616389e-05	\\
-2340.26766690341	1.51406864085172e-05	\\
-2339.28888494318	1.03801408484007e-05	\\
-2338.31010298295	1.52677753053666e-05	\\
-2337.33132102273	1.19181637415286e-05	\\
-2336.3525390625	1.17814224888857e-05	\\
-2335.37375710227	1.61563392445596e-05	\\
-2334.39497514205	1.26029884559675e-05	\\
-2333.41619318182	9.85381546339629e-06	\\
-2332.43741122159	1.55246880398417e-05	\\
-2331.45862926136	8.0906352996333e-06	\\
-2330.47984730114	1.05280978736704e-05	\\
-2329.50106534091	1.19898950811809e-05	\\
-2328.52228338068	1.19353088715187e-05	\\
-2327.54350142045	6.15521603792809e-06	\\
-2326.56471946023	1.05087943445844e-05	\\
-2325.5859375	1.0054443085818e-05	\\
-2324.60715553977	1.0239631740076e-05	\\
-2323.62837357955	9.97550675853848e-06	\\
-2322.64959161932	1.14550913738169e-05	\\
-2321.67080965909	7.16441155936494e-06	\\
-2320.69202769886	1.19493346207555e-05	\\
-2319.71324573864	1.03148595677466e-05	\\
-2318.73446377841	9.49631464391728e-06	\\
-2317.75568181818	1.08498181926699e-05	\\
-2316.77689985795	1.09793662599682e-05	\\
-2315.79811789773	1.1921305409653e-05	\\
-2314.8193359375	1.06367200056811e-05	\\
-2313.84055397727	8.99291604544703e-06	\\
-2312.86177201705	1.05534782863396e-05	\\
-2311.88299005682	1.02979956533513e-05	\\
-2310.90420809659	1.08396974047936e-05	\\
-2309.92542613636	1.35771466211046e-05	\\
-2308.94664417614	1.01647242397532e-05	\\
-2307.96786221591	1.09916937477769e-05	\\
-2306.98908025568	1.34991714775306e-05	\\
-2306.01029829545	1.14306336807305e-05	\\
-2305.03151633523	1.16857625951727e-05	\\
-2304.052734375	1.38730352031501e-05	\\
-2303.07395241477	1.75169522215065e-05	\\
-2302.09517045455	1.16544237338649e-05	\\
-2301.11638849432	1.24677578835774e-05	\\
-2300.13760653409	1.69807809917693e-05	\\
-2299.15882457386	1.20516437766496e-05	\\
-2298.18004261364	1.558757644892e-05	\\
-2297.20126065341	1.38846594763373e-05	\\
-2296.22247869318	1.54598376392559e-05	\\
-2295.24369673295	1.40261570818371e-05	\\
-2294.26491477273	1.39486468726326e-05	\\
-2293.2861328125	1.36696517228093e-05	\\
-2292.30735085227	1.42088044181936e-05	\\
-2291.32856889205	1.44392279060748e-05	\\
-2290.34978693182	1.64365438941813e-05	\\
-2289.37100497159	1.0890933192144e-05	\\
-2288.39222301136	1.33830874358607e-05	\\
-2287.41344105114	1.78056991875551e-05	\\
-2286.43465909091	1.52578042035758e-05	\\
-2285.45587713068	1.49810026018697e-05	\\
-2284.47709517045	1.25074353107287e-05	\\
-2283.49831321023	1.57104856838896e-05	\\
-2282.51953125	1.1762892655996e-05	\\
-2281.54074928977	1.1816091676574e-05	\\
-2280.56196732955	1.92945058644749e-05	\\
-2279.58318536932	1.02153560639429e-05	\\
-2278.60440340909	1.53504783137568e-05	\\
-2277.62562144886	1.28880604923445e-05	\\
-2276.64683948864	1.0281361953896e-05	\\
-2275.66805752841	9.2265533789135e-06	\\
-2274.68927556818	1.38634462855256e-05	\\
-2273.71049360795	1.16687316288638e-05	\\
-2272.73171164773	1.12107429068589e-05	\\
-2271.7529296875	1.13393032047157e-05	\\
-2270.77414772727	1.08912807384184e-05	\\
-2269.79536576705	9.17304839983014e-06	\\
-2268.81658380682	1.11423413156724e-05	\\
-2267.83780184659	9.67386014599937e-06	\\
-2266.85901988636	1.23324463797453e-05	\\
-2265.88023792614	9.45547436182029e-06	\\
-2264.90145596591	8.51336215768043e-06	\\
-2263.92267400568	1.29344752331205e-05	\\
-2262.94389204545	1.44391616700878e-05	\\
-2261.96511008523	1.19853034481267e-05	\\
-2260.986328125	1.06139729788509e-05	\\
-2260.00754616477	1.06254739590319e-05	\\
-2259.02876420455	1.17821411470889e-05	\\
-2258.04998224432	7.66434257850735e-06	\\
-2257.07120028409	1.12490504368562e-05	\\
-2256.09241832386	1.05312356224001e-05	\\
-2255.11363636364	1.09629494142538e-05	\\
-2254.13485440341	8.22082232964495e-06	\\
-2253.15607244318	9.97612544511641e-06	\\
-2252.17729048295	9.23540836109104e-06	\\
-2251.19850852273	1.10277002456324e-05	\\
-2250.2197265625	1.40735342168786e-05	\\
-2249.24094460227	8.62109753191805e-06	\\
-2248.26216264205	1.27009600192486e-05	\\
-2247.28338068182	1.09636116435671e-05	\\
-2246.30459872159	9.72097860730166e-06	\\
-2245.32581676136	1.19131725540486e-05	\\
-2244.34703480114	9.50585443834524e-06	\\
-2243.36825284091	1.03770737800806e-05	\\
-2242.38947088068	6.0323203574983e-06	\\
-2241.41068892045	1.14712947316337e-05	\\
-2240.43190696023	9.19355062289341e-06	\\
-2239.453125	6.53462009691684e-06	\\
-2238.47434303977	1.15471672034157e-05	\\
-2237.49556107955	6.06893477622947e-06	\\
-2236.51677911932	4.06209571250789e-06	\\
-2235.53799715909	8.24370324708715e-06	\\
-2234.55921519886	5.90362968189001e-06	\\
-2233.58043323864	7.63508897225743e-06	\\
-2232.60165127841	6.96309458741855e-06	\\
-2231.62286931818	6.60224953992934e-06	\\
-2230.64408735795	6.09663105878553e-06	\\
-2229.66530539773	5.09222297583826e-06	\\
-2228.6865234375	6.61062645859292e-06	\\
-2227.70774147727	1.08460761612662e-05	\\
-2226.72895951705	4.60388793033797e-06	\\
-2225.75017755682	8.05263448487694e-06	\\
-2224.77139559659	4.22521317779544e-06	\\
-2223.79261363636	6.30006723872493e-06	\\
-2222.81383167614	8.76813242881283e-06	\\
-2221.83504971591	3.65653894282158e-06	\\
-2220.85626775568	4.95175989694408e-06	\\
-2219.87748579545	7.97325052778663e-06	\\
-2218.89870383523	4.39008435920138e-06	\\
-2217.919921875	4.90063835220926e-06	\\
-2216.94113991477	3.12433846156002e-06	\\
-2215.96235795455	4.22643883432139e-06	\\
-2214.98357599432	3.89898508280812e-06	\\
-2214.00479403409	6.4928298133582e-06	\\
-2213.02601207386	6.41220109308028e-06	\\
-2212.04723011364	4.95783273183574e-06	\\
-2211.06844815341	3.88572859500735e-06	\\
-2210.08966619318	1.14332562761587e-06	\\
-2209.11088423295	5.60294569565107e-06	\\
-2208.13210227273	7.7047428904389e-06	\\
-2207.1533203125	7.25881024914961e-06	\\
-2206.17453835227	6.02073156030347e-06	\\
-2205.19575639205	7.75693494658822e-06	\\
-2204.21697443182	7.30701169993781e-06	\\
-2203.23819247159	4.62775324084444e-06	\\
-2202.25941051136	6.68588680324955e-06	\\
-2201.28062855114	4.65205395753544e-06	\\
-2200.30184659091	2.55066718187661e-06	\\
-2199.32306463068	7.39768244506249e-06	\\
-2198.34428267045	1.31483343527775e-05	\\
-2197.36550071023	6.97225838653162e-06	\\
-2196.38671875	4.50437779117457e-06	\\
-2195.40793678977	8.75637975664492e-06	\\
-2194.42915482955	3.80303057421436e-06	\\
-2193.45037286932	4.9105319684144e-06	\\
-2192.47159090909	9.39442715141743e-06	\\
-2191.49280894886	7.44818402275269e-06	\\
-2190.51402698864	8.90999454240555e-06	\\
-2189.53524502841	1.02269105590572e-05	\\
-2188.55646306818	6.68180568754173e-06	\\
-2187.57768110795	3.29389949586036e-06	\\
-2186.59889914773	9.97848338872294e-06	\\
-2185.6201171875	1.09470380861247e-05	\\
-2184.64133522727	3.30770171045642e-06	\\
-2183.66255326705	9.55968809962669e-06	\\
-2182.68377130682	4.36054413658582e-06	\\
-2181.70498934659	9.12559784271053e-06	\\
-2180.72620738636	1.80574333519365e-05	\\
-2179.74742542614	7.16719007021488e-06	\\
-2178.76864346591	8.52150824757419e-06	\\
-2177.78986150568	1.00958158531724e-05	\\
-2176.81107954545	2.86409997739697e-06	\\
-2175.83229758523	1.00684862989118e-05	\\
-2174.853515625	1.15236088085572e-05	\\
-2173.87473366477	1.10225994121293e-05	\\
-2172.89595170455	3.50012882418382e-06	\\
-2171.91716974432	8.14265516629599e-06	\\
-2170.93838778409	1.06226791068985e-05	\\
-2169.95960582386	9.41941655972841e-06	\\
-2168.98082386364	7.59664019825025e-06	\\
-2168.00204190341	5.2759708112585e-06	\\
-2167.02325994318	5.14371693214004e-06	\\
-2166.04447798295	9.88365683601138e-06	\\
-2165.06569602273	4.26308159397868e-06	\\
-2164.0869140625	5.928692127035e-06	\\
-2163.10813210227	8.58024515432925e-06	\\
-2162.12935014205	9.49118880501068e-06	\\
-2161.15056818182	6.70687310419462e-06	\\
-2160.17178622159	9.31544117762571e-06	\\
-2159.19300426136	1.16279290206983e-05	\\
-2158.21422230114	1.03466049810562e-05	\\
-2157.23544034091	1.12424705682557e-05	\\
-2156.25665838068	6.69200280830997e-06	\\
-2155.27787642045	6.45131786655435e-06	\\
-2154.29909446023	9.58901887238305e-06	\\
-2153.3203125	1.24103871141505e-05	\\
-2152.34153053977	1.29849853126034e-05	\\
-2151.36274857955	1.29513373262479e-05	\\
-2150.38396661932	7.52528154116722e-06	\\
-2149.40518465909	5.11412429452319e-06	\\
-2148.42640269886	8.04168359837703e-06	\\
-2147.44762073864	9.23312041713968e-06	\\
-2146.46883877841	1.21085664775934e-05	\\
-2145.49005681818	1.58450348882851e-05	\\
-2144.51127485795	8.91375939521211e-06	\\
-2143.53249289773	5.51900507560249e-06	\\
-2142.5537109375	6.25773068989781e-06	\\
-2141.57492897727	1.07668032052318e-05	\\
-2140.59614701705	1.07162225553464e-05	\\
-2139.61736505682	1.22391027682435e-05	\\
-2138.63858309659	7.90045380679537e-06	\\
-2137.65980113636	8.48864510146427e-06	\\
-2136.68101917614	6.62361750213522e-06	\\
-2135.70223721591	8.95812157256428e-06	\\
-2134.72345525568	1.44290677082355e-05	\\
-2133.74467329545	1.00852606631735e-05	\\
-2132.76589133523	8.46427905404173e-06	\\
-2131.787109375	7.54151363781225e-06	\\
-2130.80832741477	6.82127703326502e-06	\\
-2129.82954545455	6.09356295308642e-06	\\
-2128.85076349432	1.00125298086225e-05	\\
-2127.87198153409	9.86371201931969e-06	\\
-2126.89319957386	5.85969877235296e-06	\\
-2125.91441761364	1.24514743848341e-05	\\
-2124.93563565341	1.31184523912336e-05	\\
-2123.95685369318	1.61277789007024e-05	\\
-2122.97807173295	7.23088514688085e-06	\\
-2121.99928977273	8.58533960126685e-06	\\
-2121.0205078125	3.89828173202376e-06	\\
-2120.04172585227	5.6260730017591e-06	\\
-2119.06294389205	9.86179598042668e-06	\\
-2118.08416193182	8.50131263859708e-06	\\
-2117.10537997159	1.11229992688978e-05	\\
-2116.12659801136	5.09675120254464e-06	\\
-2115.14781605114	6.12278192989239e-06	\\
-2114.16903409091	9.87558268755534e-06	\\
-2113.19025213068	9.45792982119867e-06	\\
-2112.21147017045	5.15490090615047e-06	\\
-2111.23268821023	1.32719729783374e-05	\\
-2110.25390625	3.82142073224439e-06	\\
-2109.27512428977	1.33693599086108e-05	\\
-2108.29634232955	7.81298476476947e-06	\\
-2107.31756036932	1.36368330525389e-06	\\
-2106.33877840909	8.64398467573944e-06	\\
-2105.35999644886	4.2756243030387e-06	\\
-2104.38121448864	2.94399173951683e-06	\\
-2103.40243252841	6.41613045462651e-06	\\
-2102.42365056818	3.75005923171832e-06	\\
-2101.44486860795	4.37124876756518e-06	\\
-2100.46608664773	6.08889479325245e-06	\\
-2099.4873046875	3.9432243976415e-06	\\
-2098.50852272727	5.30716708979216e-06	\\
-2097.52974076705	7.27600074119795e-06	\\
-2096.55095880682	9.87519515416853e-06	\\
-2095.57217684659	9.5238213053883e-06	\\
-2094.59339488636	8.00868914238196e-06	\\
-2093.61461292614	3.78382790879595e-06	\\
-2092.63583096591	3.90223258440293e-06	\\
-2091.65704900568	4.0223479312074e-06	\\
-2090.67826704545	5.16043404198381e-06	\\
-2089.69948508523	5.50653034494106e-06	\\
-2088.720703125	8.6605945245811e-06	\\
-2087.74192116477	5.72409845990738e-06	\\
-2086.76313920455	6.32698234982391e-06	\\
-2085.78435724432	7.77771007836846e-06	\\
-2084.80557528409	8.58703976759977e-06	\\
-2083.82679332386	2.9303119945883e-06	\\
-2082.84801136364	5.6857213988755e-06	\\
-2081.86922940341	5.693146178263e-06	\\
-2080.89044744318	6.26070614293695e-06	\\
-2079.91166548295	2.43355584460901e-06	\\
-2078.93288352273	8.89762477168118e-06	\\
-2077.9541015625	5.58154409553787e-06	\\
-2076.97531960227	9.3654581335439e-06	\\
-2075.99653764205	6.69702806330472e-06	\\
-2075.01775568182	5.9519754720871e-06	\\
-2074.03897372159	8.53130929574489e-06	\\
-2073.06019176136	9.49215522532696e-06	\\
-2072.08140980114	7.28478938576503e-06	\\
-2071.10262784091	9.84933272862228e-06	\\
-2070.12384588068	9.40938702577624e-06	\\
-2069.14506392045	9.59003105649133e-06	\\
-2068.16628196023	1.13860857937157e-05	\\
-2067.1875	7.4427582513008e-06	\\
-2066.20871803977	8.01248950199946e-06	\\
-2065.22993607955	1.14267612967262e-05	\\
-2064.25115411932	8.05275325801951e-06	\\
-2063.27237215909	9.78989852070145e-06	\\
-2062.29359019886	1.56420192779084e-05	\\
-2061.31480823864	8.12181795289017e-06	\\
-2060.33602627841	1.45791015922583e-05	\\
-2059.35724431818	1.40209169095327e-05	\\
-2058.37846235795	7.30696198810194e-06	\\
-2057.39968039773	1.15504119244389e-05	\\
-2056.4208984375	1.44504930423958e-05	\\
-2055.44211647727	1.01447153835253e-05	\\
-2054.46333451705	1.30931440557917e-05	\\
-2053.48455255682	1.40521773814718e-05	\\
-2052.50577059659	1.18155965045567e-05	\\
-2051.52698863636	1.15247963546881e-05	\\
-2050.54820667614	1.3087440340532e-05	\\
-2049.56942471591	1.05791205769395e-05	\\
-2048.59064275568	1.15711376947935e-05	\\
-2047.61186079545	9.89162011989905e-06	\\
-2046.63307883523	1.48059993672879e-05	\\
-2045.654296875	6.54810670360145e-06	\\
-2044.67551491477	7.04296793119206e-06	\\
-2043.69673295455	1.19066912666882e-05	\\
-2042.71795099432	1.39804967437247e-05	\\
-2041.73916903409	1.0265594829775e-05	\\
-2040.76038707386	1.17793417545122e-05	\\
-2039.78160511364	2.15191166947662e-06	\\
-2038.80282315341	8.6504920120632e-06	\\
-2037.82404119318	1.19913467876653e-05	\\
-2036.84525923295	1.38492157923552e-05	\\
-2035.86647727273	1.03199227778699e-05	\\
-2034.8876953125	1.11245932429092e-05	\\
-2033.90891335227	9.32607579585942e-06	\\
-2032.93013139205	1.55821998451553e-05	\\
-2031.95134943182	2.35296429274789e-05	\\
-2030.97256747159	1.34445733944483e-05	\\
-2029.99378551136	1.56694234497058e-06	\\
-2029.01500355114	7.93183814403049e-06	\\
-2028.03622159091	1.12472078839186e-05	\\
-2027.05743963068	1.8285416203708e-05	\\
-2026.07865767045	1.63116288188631e-05	\\
-2025.09987571023	1.04591608750718e-05	\\
-2024.12109375	6.9377170470418e-06	\\
-2023.14231178977	1.12724295986904e-05	\\
-2022.16352982955	1.91805804086029e-05	\\
-2021.18474786932	1.22804731618298e-05	\\
-2020.20596590909	1.65411668881646e-05	\\
-2019.22718394886	1.1263904997825e-05	\\
-2018.24840198864	1.01049998062263e-05	\\
-2017.26962002841	8.77926056130003e-06	\\
-2016.29083806818	1.46854736364667e-05	\\
-2015.31205610795	1.36230105440286e-05	\\
-2014.33327414773	1.08000434070178e-05	\\
-2013.3544921875	6.31026160246397e-06	\\
-2012.37571022727	1.04390493549777e-05	\\
-2011.39692826705	1.21551362476789e-05	\\
-2010.41814630682	1.77471352223243e-05	\\
-2009.43936434659	1.57325697544742e-05	\\
-2008.46058238636	1.47770634265064e-05	\\
-2007.48180042614	1.01859595390405e-05	\\
-2006.50301846591	1.58658810016204e-05	\\
-2005.52423650568	9.23043057316933e-06	\\
-2004.54545454545	1.00904526720266e-05	\\
-2003.56667258523	2.16921821187787e-05	\\
-2002.587890625	1.27898179872726e-05	\\
-2001.60910866477	1.69487422939793e-05	\\
-2000.63032670455	1.5380474912724e-05	\\
-1999.65154474432	1.15770952016e-05	\\
-1998.67276278409	7.0454215240975e-06	\\
-1997.69398082386	1.13583341002894e-05	\\
-1996.71519886364	1.22661764799915e-05	\\
-1995.73641690341	1.14863461504324e-05	\\
-1994.75763494318	1.08151597827359e-05	\\
-1993.77885298295	8.68515656384366e-06	\\
-1992.80007102273	1.32167680465161e-05	\\
-1991.8212890625	1.25534074579674e-05	\\
-1990.84250710227	1.19236302903921e-05	\\
-1989.86372514205	1.10372895571537e-05	\\
-1988.88494318182	1.16366126295665e-05	\\
-1987.90616122159	9.42223666359802e-06	\\
-1986.92737926136	8.48240367056398e-06	\\
-1985.94859730114	5.69977600125921e-06	\\
-1984.96981534091	9.73972985479646e-06	\\
-1983.99103338068	9.5455650342174e-06	\\
-1983.01225142045	9.20903452284273e-06	\\
-1982.03346946023	5.18966448839485e-06	\\
-1981.0546875	3.21624182515013e-06	\\
-1980.07590553977	9.72391672467896e-06	\\
-1979.09712357955	1.05072951722797e-05	\\
-1978.11834161932	1.1774061313896e-05	\\
-1977.13955965909	7.41937834527636e-06	\\
-1976.16077769886	1.03332588497384e-05	\\
-1975.18199573864	1.047117297922e-06	\\
-1974.20321377841	3.12590756125873e-06	\\
-1973.22443181818	4.57573642290179e-06	\\
-1972.24564985795	1.39817650815909e-05	\\
-1971.26686789773	1.50794153960939e-05	\\
-1970.2880859375	1.40321330342596e-05	\\
-1969.30930397727	1.03141485032882e-05	\\
-1968.33052201705	1.37811655022303e-06	\\
-1967.35174005682	3.63437986569035e-06	\\
-1966.37295809659	1.2710318869156e-05	\\
-1965.39417613636	1.37240193272603e-05	\\
-1964.41539417614	1.21028005598686e-05	\\
-1963.43661221591	9.21432628302775e-06	\\
-1962.45783025568	7.81144536088841e-06	\\
-1961.47904829545	6.65796921176516e-06	\\
-1960.50026633523	1.06225294746132e-05	\\
-1959.521484375	8.73073249349637e-06	\\
-1958.54270241477	9.20731159324451e-06	\\
-1957.56392045455	1.03871664177912e-05	\\
-1956.58513849432	1.26064925749797e-05	\\
-1955.60635653409	1.42468468072331e-05	\\
-1954.62757457386	1.10785053387341e-05	\\
-1953.64879261364	1.0440135148296e-05	\\
-1952.67001065341	7.43329099311322e-06	\\
-1951.69122869318	5.96408348591512e-06	\\
-1950.71244673295	9.12530765263172e-06	\\
-1949.73366477273	6.59487493888109e-06	\\
-1948.7548828125	8.81036607037955e-06	\\
-1947.77610085227	9.78770200094604e-06	\\
-1946.79731889205	9.98774060304465e-06	\\
-1945.81853693182	6.41655426716495e-06	\\
-1944.83975497159	5.51001241479547e-06	\\
-1943.86097301136	9.74071780698492e-06	\\
-1942.88219105114	1.0499010963137e-05	\\
-1941.90340909091	1.28651567462797e-05	\\
-1940.92462713068	1.41086013266635e-05	\\
-1939.94584517045	1.17921557056788e-05	\\
-1938.96706321023	1.32163663697268e-05	\\
-1937.98828125	9.67699027136935e-06	\\
-1937.00949928977	1.04033242277572e-05	\\
-1936.03071732955	7.90549425260135e-06	\\
-1935.05193536932	2.04058762713608e-05	\\
-1934.07315340909	1.14605302247808e-05	\\
-1933.09437144886	1.70657416739231e-05	\\
-1932.11558948864	6.63659620427054e-06	\\
-1931.13680752841	1.70293443731104e-05	\\
-1930.15802556818	7.35341120083535e-06	\\
-1929.17924360795	1.42533090981033e-05	\\
-1928.20046164773	5.78773892572841e-06	\\
-1927.2216796875	1.50919851968584e-05	\\
-1926.24289772727	9.01300874180936e-06	\\
-1925.26411576705	1.4352890775907e-05	\\
-1924.28533380682	1.30249677188531e-05	\\
-1923.30655184659	1.34402737427926e-05	\\
-1922.32776988636	1.47085373526635e-05	\\
-1921.34898792614	1.11075096062954e-05	\\
-1920.37020596591	1.23794344593587e-05	\\
-1919.39142400568	1.18889242271997e-05	\\
-1918.41264204545	1.22919496164178e-05	\\
-1917.43386008523	1.36932364673084e-05	\\
-1916.455078125	1.56304345882397e-05	\\
-1915.47629616477	1.4623154723641e-05	\\
-1914.49751420455	1.46554639140251e-05	\\
-1913.51873224432	1.11227654807394e-05	\\
-1912.53995028409	1.23362480399745e-05	\\
-1911.56116832386	1.50050895005838e-05	\\
-1910.58238636364	1.56103218382655e-05	\\
-1909.60360440341	1.49417468089668e-05	\\
-1908.62482244318	1.11280517776506e-05	\\
-1907.64604048295	1.1781114947476e-05	\\
-1906.66725852273	8.73370688352453e-06	\\
-1905.6884765625	1.00092970326083e-05	\\
-1904.70969460227	9.04772319499703e-06	\\
-1903.73091264205	1.17457027414302e-05	\\
-1902.75213068182	6.96824950646837e-06	\\
-1901.77334872159	1.24569833231102e-05	\\
-1900.79456676136	1.05859342929557e-05	\\
-1899.81578480114	1.18835702870953e-05	\\
-1898.83700284091	3.43295232072828e-06	\\
-1897.85822088068	5.41875172061408e-06	\\
-1896.87943892045	2.51283310894254e-06	\\
-1895.90065696023	2.05284665453561e-06	\\
-1894.921875	6.66646431123459e-06	\\
-1893.94309303977	8.67738408353248e-06	\\
-1892.96431107955	3.92059769664307e-06	\\
-1891.98552911932	3.27981693193491e-06	\\
-1891.00674715909	4.64945295232852e-06	\\
-1890.02796519886	1.00121533059993e-05	\\
-1889.04918323864	1.1777710816031e-05	\\
-1888.07040127841	9.48257002721501e-06	\\
-1887.09161931818	4.52674232493319e-06	\\
-1886.11283735795	3.31242216476026e-06	\\
-1885.13405539773	8.6572700335283e-06	\\
-1884.1552734375	7.54631051353716e-06	\\
-1883.17649147727	6.13400756663844e-06	\\
-1882.19770951705	6.11201169380559e-06	\\
-1881.21892755682	8.71175480338794e-06	\\
-1880.24014559659	8.39116307242371e-06	\\
-1879.26136363636	1.40158940639543e-05	\\
-1878.28258167614	1.08297561032908e-05	\\
-1877.30379971591	9.13248277665128e-06	\\
-1876.32501775568	5.77460368311088e-06	\\
-1875.34623579545	1.50042583199962e-05	\\
-1874.36745383523	1.30764400757106e-05	\\
-1873.388671875	1.62479762812261e-05	\\
-1872.40988991477	1.90100981845312e-05	\\
-1871.43110795455	1.47608716508858e-05	\\
-1870.45232599432	1.03654602724547e-05	\\
-1869.47354403409	9.81374057227526e-06	\\
-1868.49476207386	1.28465831169991e-05	\\
-1867.51598011364	1.22587123684802e-05	\\
-1866.53719815341	1.23655554265726e-05	\\
-1865.55841619318	1.19790854330182e-05	\\
-1864.57963423295	1.38905656024591e-05	\\
-1863.60085227273	1.47922695104962e-05	\\
-1862.6220703125	1.73305726788919e-05	\\
-1861.64328835227	1.52363649198625e-05	\\
-1860.66450639205	1.43171754771398e-05	\\
-1859.68572443182	1.08020763813365e-05	\\
-1858.70694247159	1.28617676390692e-05	\\
-1857.72816051136	1.37300532555533e-05	\\
-1856.74937855114	1.41544558163218e-05	\\
-1855.77059659091	1.51170948049344e-05	\\
-1854.79181463068	1.57247285343539e-05	\\
-1853.81303267045	1.20315459903378e-05	\\
-1852.83425071023	1.42722113867383e-05	\\
-1851.85546875	1.1744545259099e-05	\\
-1850.87668678977	1.04282005665665e-05	\\
-1849.89790482955	1.53320161022013e-05	\\
-1848.91912286932	1.70728078822682e-05	\\
-1847.94034090909	1.32183302435999e-05	\\
-1846.96155894886	1.1645041275018e-05	\\
-1845.98277698864	6.06535239856543e-06	\\
-1845.00399502841	1.37226575997763e-05	\\
-1844.02521306818	1.64573497378197e-05	\\
-1843.04643110795	2.06038472091339e-05	\\
-1842.06764914773	1.6517119464482e-05	\\
-1841.0888671875	7.54945815409171e-06	\\
-1840.11008522727	1.51553226026645e-05	\\
-1839.13130326705	2.44546517688232e-05	\\
-1838.15252130682	2.05832594813853e-05	\\
-1837.17373934659	2.39896883086052e-05	\\
-1836.19495738636	1.62768966538732e-05	\\
-1835.21617542614	8.93039832495832e-06	\\
-1834.23739346591	1.4504778347474e-05	\\
-1833.25861150568	2.37549819217119e-05	\\
-1832.27982954545	2.57982909830735e-05	\\
-1831.30104758523	2.11662784433356e-05	\\
-1830.322265625	8.79186222238605e-06	\\
-1829.34348366477	8.49684697359181e-06	\\
-1828.36470170455	9.0967082036575e-06	\\
-1827.38591974432	1.5430428578406e-05	\\
-1826.40713778409	2.44979175602787e-05	\\
-1825.42835582386	2.65815106939053e-05	\\
-1824.44957386364	1.41413922479818e-05	\\
-1823.47079190341	1.16602193846953e-05	\\
-1822.49200994318	7.13731762525929e-06	\\
-1821.51322798295	2.08139282947399e-05	\\
-1820.53444602273	1.87910709807992e-05	\\
-1819.5556640625	2.15132567762822e-05	\\
-1818.57688210227	2.03095588873801e-05	\\
-1817.59810014205	1.5791777981223e-05	\\
-1816.61931818182	1.02953795710986e-05	\\
-1815.64053622159	1.30582326388498e-05	\\
-1814.66175426136	1.42994070217469e-05	\\
-1813.68297230114	1.63447119304696e-05	\\
-1812.70419034091	1.97872705111454e-05	\\
-1811.72540838068	1.33201196617015e-05	\\
-1810.74662642045	1.61621700989054e-05	\\
-1809.76784446023	1.20816149501292e-05	\\
-1808.7890625	1.24231291837468e-05	\\
-1807.81028053977	1.66743816413002e-05	\\
-1806.83149857955	1.38443091684051e-05	\\
-1805.85271661932	1.94802422585846e-05	\\
-1804.87393465909	1.18324669052725e-05	\\
-1803.89515269886	1.45801661621206e-05	\\
-1802.91637073864	1.04192896771255e-05	\\
-1801.93758877841	1.32903206471309e-05	\\
-1800.95880681818	1.82706763271189e-05	\\
-1799.98002485795	1.94620466084177e-05	\\
-1799.00124289773	1.50463780978563e-05	\\
-1798.0224609375	1.70844976614562e-05	\\
-1797.04367897727	1.08302052025927e-05	\\
-1796.06489701705	1.02561349521878e-05	\\
-1795.08611505682	1.38152674558002e-05	\\
-1794.10733309659	1.98817610246029e-05	\\
-1793.12855113636	1.54746434725341e-05	\\
-1792.14976917614	1.64814837640952e-05	\\
-1791.17098721591	1.18550243717252e-05	\\
-1790.19220525568	1.09795945531786e-05	\\
-1789.21342329545	1.273946824838e-05	\\
-1788.23464133523	1.19405857788129e-05	\\
-1787.255859375	1.10809081846092e-05	\\
-1786.27707741477	1.89562541553201e-05	\\
-1785.29829545455	1.06903928572017e-05	\\
-1784.31951349432	1.15703595365459e-05	\\
-1783.34073153409	1.63417838557502e-05	\\
-1782.36194957386	1.64362268929462e-05	\\
-1781.38316761364	1.35889064415367e-05	\\
-1780.40438565341	1.15493884573921e-05	\\
-1779.42560369318	1.29585352577986e-05	\\
-1778.44682173295	1.23553065874618e-05	\\
-1777.46803977273	1.60780161631468e-05	\\
-1776.4892578125	1.94134287133079e-05	\\
-1775.51047585227	2.33549780068716e-05	\\
-1774.53169389205	1.62901218283655e-05	\\
-1773.55291193182	1.94563340523767e-05	\\
-1772.57412997159	1.57347671466349e-05	\\
-1771.59534801136	1.87070829849616e-05	\\
-1770.61656605114	2.29687968349095e-05	\\
-1769.63778409091	1.79365119121817e-05	\\
-1768.65900213068	1.6955848748824e-05	\\
-1767.68022017045	1.7903418193089e-05	\\
-1766.70143821023	1.70633514919759e-05	\\
-1765.72265625	1.49473865969641e-05	\\
-1764.74387428977	1.95615235992782e-05	\\
-1763.76509232955	1.220788947099e-05	\\
-1762.78631036932	1.86899911986928e-05	\\
-1761.80752840909	1.52464368338241e-05	\\
-1760.82874644886	1.31938466127946e-05	\\
-1759.84996448864	1.66240978619053e-05	\\
-1758.87118252841	8.83320916980561e-06	\\
-1757.89240056818	7.64658007024304e-06	\\
-1756.91361860795	8.86760422009559e-06	\\
-1755.93483664773	9.75094569877141e-06	\\
-1754.9560546875	1.19082643486791e-05	\\
-1753.97727272727	7.57207905301432e-06	\\
-1752.99849076705	1.23059238245812e-05	\\
-1752.01970880682	9.67330184558685e-06	\\
-1751.04092684659	1.3271715620079e-05	\\
-1750.06214488636	1.36422842538776e-05	\\
-1749.08336292614	8.02899721694715e-06	\\
-1748.10458096591	8.773451383533e-06	\\
-1747.12579900568	6.89351996956909e-06	\\
-1746.14701704545	1.15212546794788e-05	\\
-1745.16823508523	6.26577003371419e-06	\\
-1744.189453125	9.47940598029618e-06	\\
-1743.21067116477	8.76208703999715e-06	\\
-1742.23188920455	1.28187872134601e-05	\\
-1741.25310724432	8.96944758599654e-06	\\
-1740.27432528409	1.34157543137618e-05	\\
-1739.29554332386	1.09009618363295e-05	\\
-1738.31676136364	1.20185243463733e-05	\\
-1737.33797940341	9.16053648658617e-06	\\
-1736.35919744318	9.54662950345803e-06	\\
-1735.38041548295	1.4735119950186e-05	\\
-1734.40163352273	9.83492450137827e-06	\\
-1733.4228515625	1.34032376870602e-05	\\
-1732.44406960227	1.45920181120544e-05	\\
-1731.46528764205	1.04062011692483e-05	\\
-1730.48650568182	1.38926682758435e-05	\\
-1729.50772372159	1.65717447765496e-05	\\
-1728.52894176136	1.57465092408788e-05	\\
-1727.55015980114	1.59847163108369e-05	\\
-1726.57137784091	1.63291794300285e-05	\\
-1725.59259588068	1.53696021939816e-05	\\
-1724.61381392045	1.85191298385434e-05	\\
-1723.63503196023	1.89908332867789e-05	\\
-1722.65625	1.75010669181981e-05	\\
-1721.67746803977	1.58388026271429e-05	\\
-1720.69868607955	1.38280376797999e-05	\\
-1719.71990411932	1.87217678976336e-05	\\
-1718.74112215909	1.95547408156666e-05	\\
-1717.76234019886	1.70114575193964e-05	\\
-1716.78355823864	2.09636669323439e-05	\\
-1715.80477627841	1.66946561478505e-05	\\
-1714.82599431818	1.86463100822531e-05	\\
-1713.84721235795	1.86035991551693e-05	\\
-1712.86843039773	1.90097718269021e-05	\\
-1711.8896484375	1.74041949680393e-05	\\
-1710.91086647727	5.59110539991613e-06	\\
-1709.93208451705	1.6395556650061e-05	\\
-1708.95330255682	1.52806847179042e-05	\\
-1707.97452059659	1.41058241509242e-05	\\
-1706.99573863636	1.07989847452543e-05	\\
-1706.01695667614	1.46492791847909e-05	\\
-1705.03817471591	9.87464612371688e-06	\\
-1704.05939275568	8.15665407308678e-06	\\
-1703.08061079545	1.11910682301529e-05	\\
-1702.10182883523	1.33064640121073e-05	\\
-1701.123046875	9.96284231272349e-06	\\
-1700.14426491477	5.55261718356504e-06	\\
-1699.16548295455	1.02446793254025e-05	\\
-1698.18670099432	1.64139755438252e-05	\\
-1697.20791903409	1.54166199535175e-05	\\
-1696.22913707386	2.20129099141463e-05	\\
-1695.25035511364	1.41520816335329e-05	\\
-1694.27157315341	8.04450000098705e-06	\\
-1693.29279119318	9.35389992337399e-06	\\
-1692.31400923295	5.45335183308364e-06	\\
-1691.33522727273	1.41312166294005e-05	\\
-1690.3564453125	1.70573175281045e-05	\\
-1689.37766335227	1.9474408266739e-05	\\
-1688.39888139205	1.66819029281719e-05	\\
-1687.42009943182	1.3036874122569e-05	\\
-1686.44131747159	1.2558027701556e-05	\\
-1685.46253551136	1.45662671607647e-05	\\
-1684.48375355114	1.86427418235533e-05	\\
-1683.50497159091	1.48234249669665e-05	\\
-1682.52618963068	1.58970591663059e-05	\\
-1681.54740767045	2.09064615918393e-05	\\
-1680.56862571023	1.73075801665593e-05	\\
-1679.58984375	9.75688899324438e-06	\\
-1678.61106178977	1.54809546181247e-05	\\
-1677.63227982955	1.66104908556069e-05	\\
-1676.65349786932	2.11069556912036e-05	\\
-1675.67471590909	1.8828243495898e-05	\\
-1674.69593394886	1.46540089489827e-05	\\
-1673.71715198864	1.68744083714705e-05	\\
-1672.73837002841	1.54745051260241e-05	\\
-1671.75958806818	1.99495853356779e-05	\\
-1670.78080610795	1.76730103810649e-05	\\
-1669.80202414773	2.29981137096437e-05	\\
-1668.8232421875	2.22135284919658e-05	\\
-1667.84446022727	1.46842776130288e-05	\\
-1666.86567826705	1.59974273386179e-05	\\
-1665.88689630682	1.38068118177693e-05	\\
-1664.90811434659	1.95562494976692e-05	\\
-1663.92933238636	2.55549013503757e-05	\\
-1662.95055042614	2.57501025034283e-05	\\
-1661.97176846591	2.25965854091161e-05	\\
-1660.99298650568	1.36633316579147e-05	\\
-1660.01420454545	8.48334466607352e-06	\\
-1659.03542258523	1.5592701028559e-05	\\
-1658.056640625	1.81525515103464e-05	\\
-1657.07785866477	2.09897989377685e-05	\\
-1656.09907670455	1.88828630142245e-05	\\
-1655.12029474432	1.33762048302199e-05	\\
-1654.14151278409	1.64953968906613e-05	\\
-1653.16273082386	1.53137224276326e-05	\\
-1652.18394886364	1.55329191528135e-05	\\
-1651.20516690341	1.43734410540413e-05	\\
-1650.22638494318	1.41157378418197e-05	\\
-1649.24760298295	1.08101960183942e-05	\\
-1648.26882102273	1.16537190468102e-05	\\
-1647.2900390625	1.46908836033493e-05	\\
-1646.31125710227	2.19309087751843e-05	\\
-1645.33247514205	2.31017007077471e-05	\\
-1644.35369318182	1.91916006511593e-05	\\
-1643.37491122159	1.11242105323893e-05	\\
-1642.39612926136	1.30005290587784e-05	\\
-1641.41734730114	7.50917475568137e-06	\\
-1640.43856534091	1.13946536442038e-05	\\
-1639.45978338068	1.45173307652988e-05	\\
-1638.48100142045	1.48072612235371e-05	\\
-1637.50221946023	1.63796171740116e-05	\\
-1636.5234375	1.93932960022442e-05	\\
-1635.54465553977	2.07084850455064e-05	\\
-1634.56587357955	2.08347383618891e-05	\\
-1633.58709161932	2.09676319933214e-05	\\
-1632.60830965909	1.74672248928841e-05	\\
-1631.62952769886	9.97475145009483e-06	\\
-1630.65074573864	9.50371705728536e-06	\\
-1629.67196377841	7.10188062513394e-06	\\
-1628.69318181818	7.68715941560196e-06	\\
-1627.71439985795	1.62334132854353e-05	\\
-1626.73561789773	1.33437764525738e-05	\\
-1625.7568359375	1.6228168732758e-05	\\
-1624.77805397727	1.50190531486105e-05	\\
-1623.79927201705	1.67202443691765e-05	\\
-1622.82049005682	1.85806008404246e-05	\\
-1621.84170809659	1.73039693922641e-05	\\
-1620.86292613636	1.75294814520834e-05	\\
-1619.88414417614	1.12498646726125e-05	\\
-1618.90536221591	1.23025840013357e-05	\\
-1617.92658025568	5.71227116755015e-06	\\
-1616.94779829545	8.89346715775711e-06	\\
-1615.96901633523	1.34660361866575e-05	\\
-1614.990234375	1.33743043310649e-05	\\
-1614.01145241477	1.31106962972763e-05	\\
-1613.03267045455	1.22929597510427e-05	\\
-1612.05388849432	1.64900234537368e-05	\\
-1611.07510653409	2.087687031918e-05	\\
-1610.09632457386	2.01678391383544e-05	\\
-1609.11754261364	1.34363146337308e-05	\\
-1608.13876065341	1.43396709477571e-05	\\
-1607.15997869318	2.02375323630248e-05	\\
-1606.18119673295	1.62855463518043e-05	\\
-1605.20241477273	1.32234454295438e-05	\\
-1604.2236328125	1.65450996717634e-05	\\
-1603.24485085227	1.43369798818184e-05	\\
-1602.26606889205	1.45244743834684e-05	\\
-1601.28728693182	9.38122348021035e-06	\\
-1600.30850497159	1.729636552229e-05	\\
-1599.32972301136	1.29693209914932e-05	\\
-1598.35094105114	2.1111435840236e-05	\\
-1597.37215909091	2.5445814096633e-05	\\
-1596.39337713068	2.18410180795548e-05	\\
-1595.41459517045	2.06981100904331e-05	\\
-1594.43581321023	2.40161536018553e-05	\\
-1593.45703125	2.22745172854821e-05	\\
-1592.47824928977	1.40845691951689e-05	\\
-1591.49946732955	1.80340530570256e-05	\\
-1590.52068536932	1.73614614516466e-05	\\
-1589.54190340909	1.75863759251109e-05	\\
-1588.56312144886	1.67489172536202e-05	\\
-1587.58433948864	1.50838922124659e-05	\\
-1586.60555752841	1.71026017614557e-05	\\
-1585.62677556818	1.17150077098985e-05	\\
-1584.64799360795	1.48234967521148e-05	\\
-1583.66921164773	1.87752500623316e-05	\\
-1582.6904296875	1.9564795726097e-05	\\
-1581.71164772727	1.88141700239547e-05	\\
-1580.73286576705	2.32105163858559e-05	\\
-1579.75408380682	2.55893787119335e-05	\\
-1578.77530184659	2.16684024888468e-05	\\
-1577.79651988636	1.65235329800607e-05	\\
-1576.81773792614	1.51555456007696e-05	\\
-1575.83895596591	1.85529100405017e-05	\\
-1574.86017400568	2.5068581896136e-05	\\
-1573.88139204545	3.29838897075912e-05	\\
-1572.90261008523	2.78453271930331e-05	\\
-1571.923828125	3.11535106019386e-05	\\
-1570.94504616477	2.05978647791171e-05	\\
-1569.96626420455	1.22250493521802e-05	\\
-1568.98748224432	1.10658049626311e-05	\\
-1568.00870028409	1.87886179737843e-05	\\
-1567.02991832386	1.78577496379486e-05	\\
-1566.05113636364	2.82307360883587e-05	\\
-1565.07235440341	2.82688455450186e-05	\\
-1564.09357244318	2.17311336664738e-05	\\
-1563.11479048295	1.71958104125918e-05	\\
-1562.13600852273	5.26507562211998e-06	\\
-1561.1572265625	1.22104377543614e-05	\\
-1560.17844460227	2.21222464982048e-05	\\
-1559.19966264205	2.70397964805347e-05	\\
-1558.22088068182	2.91717885478558e-05	\\
-1557.24209872159	3.32823178613597e-05	\\
-1556.26331676136	2.10603572200706e-05	\\
-1555.28453480114	1.48399430789605e-05	\\
-1554.30575284091	9.71268289738322e-06	\\
-1553.32697088068	1.32627082173132e-05	\\
-1552.34818892045	2.07464956971275e-05	\\
-1551.36940696023	2.39756802893506e-05	\\
-1550.390625	2.50097289917526e-05	\\
-1549.41184303977	1.61914643922065e-05	\\
-1548.43306107955	1.55302232550524e-05	\\
-1547.45427911932	8.39909043731134e-06	\\
-1546.47549715909	1.54896167123283e-05	\\
-1545.49671519886	1.42926840057762e-05	\\
-1544.51793323864	1.94249293259608e-05	\\
-1543.53915127841	2.22778891142745e-05	\\
-1542.56036931818	2.20869002532667e-05	\\
-1541.58158735795	2.05774908949805e-05	\\
-1540.60280539773	2.30674743100717e-05	\\
-1539.6240234375	1.75491685868404e-05	\\
-1538.64524147727	1.89452259078486e-05	\\
-1537.66645951705	1.0535718449821e-05	\\
-1536.68767755682	1.22169183780118e-05	\\
-1535.70889559659	8.83095787167224e-06	\\
-1534.73011363636	1.42348336514987e-05	\\
-1533.75133167614	2.56280934962812e-05	\\
-1532.77254971591	2.55605510733408e-05	\\
-1531.79376775568	2.53239932690914e-05	\\
-1530.81498579545	2.81963388497733e-05	\\
-1529.83620383523	2.39048616274868e-05	\\
-1528.857421875	2.07466185874893e-05	\\
-1527.87863991477	7.24677805175799e-06	\\
-1526.89985795455	7.06470806603996e-06	\\
-1525.92107599432	1.29036162000831e-05	\\
-1524.94229403409	9.86001549142846e-06	\\
-1523.96351207386	2.63285641591811e-05	\\
-1522.98473011364	2.32823832712849e-05	\\
-1522.00594815341	2.37994211065541e-05	\\
-1521.02716619318	2.38040406905119e-05	\\
-1520.04838423295	2.25260832048143e-05	\\
-1519.06960227273	2.59539734577793e-05	\\
-1518.0908203125	1.6512858310654e-05	\\
-1517.11203835227	1.98089581260454e-06	\\
-1516.13325639205	6.95337113640151e-06	\\
-1515.15447443182	1.36170678108091e-05	\\
-1514.17569247159	2.59828926316464e-05	\\
-1513.19691051136	2.95368105316311e-05	\\
-1512.21812855114	2.35081398108955e-05	\\
-1511.23934659091	2.28237924324403e-05	\\
-1510.26056463068	2.44423069968239e-05	\\
-1509.28178267045	1.6612472761053e-05	\\
-1508.30300071023	1.36245684862314e-05	\\
-1507.32421875	1.1311809063092e-05	\\
-1506.34543678977	3.00260150847455e-06	\\
-1505.36665482955	1.17941862168305e-05	\\
-1504.38787286932	1.82438318808055e-05	\\
-1503.40909090909	1.83251825627845e-05	\\
-1502.43030894886	2.68582641650215e-05	\\
-1501.45152698864	1.96723725180896e-05	\\
-1500.47274502841	2.34249648598153e-05	\\
-1499.49396306818	2.74735815179028e-05	\\
-1498.51518110795	1.63960154842176e-05	\\
-1497.53639914773	1.60669670196638e-05	\\
-1496.5576171875	1.2967397429587e-05	\\
-1495.57883522727	8.27993488248434e-06	\\
-1494.60005326705	7.17444992958073e-06	\\
-1493.62127130682	1.57261095550443e-05	\\
-1492.64248934659	1.15021287751728e-05	\\
-1491.66370738636	1.85208671827666e-05	\\
-1490.68492542614	2.29649696228892e-05	\\
-1489.70614346591	1.4422755853057e-05	\\
-1488.72736150568	1.70877522566214e-05	\\
-1487.74857954545	1.4982300137973e-05	\\
-1486.76979758523	1.59747118832971e-05	\\
-1485.791015625	1.92711986064622e-05	\\
-1484.81223366477	1.55566214683729e-05	\\
-1483.83345170455	8.30104544087155e-06	\\
-1482.85466974432	1.83826473366787e-06	\\
-1481.87588778409	2.34521396060846e-06	\\
-1480.89710582386	3.84849018768893e-06	\\
-1479.91832386364	8.21507004973125e-06	\\
-1478.93954190341	8.5338369661574e-06	\\
-1477.96075994318	1.47928677073933e-05	\\
-1476.98197798295	1.5114612038281e-05	\\
-1476.00319602273	1.46050735432495e-05	\\
-1475.0244140625	2.05838133376699e-05	\\
-1474.04563210227	1.6255914911469e-05	\\
-1473.06685014205	2.01841756607369e-05	\\
-1472.08806818182	2.3262532250379e-05	\\
-1471.10928622159	1.41344004039836e-05	\\
-1470.13050426136	1.69691503453902e-05	\\
-1469.15172230114	1.46391233215622e-05	\\
-1468.17294034091	5.2560804606111e-06	\\
-1467.19415838068	7.13845960222458e-06	\\
-1466.21537642045	1.10083394166795e-05	\\
-1465.23659446023	3.67107440601806e-06	\\
-1464.2578125	1.17289228168592e-05	\\
-1463.27903053977	9.11162813060243e-06	\\
-1462.30024857955	8.96724543242796e-06	\\
-1461.32146661932	7.46074755792332e-06	\\
-1460.34268465909	6.15503425414521e-06	\\
-1459.36390269886	1.66840535423278e-05	\\
-1458.38512073864	6.39872165104173e-06	\\
-1457.40633877841	1.25744248030577e-05	\\
-1456.42755681818	1.37244014471747e-05	\\
-1455.44877485795	2.5717862311352e-06	\\
-1454.46999289773	1.49134175638691e-05	\\
-1453.4912109375	1.42216006771302e-05	\\
-1452.51242897727	3.88220715560474e-06	\\
-1451.53364701705	1.71915916431356e-05	\\
-1450.55486505682	1.32478774891059e-05	\\
-1449.57608309659	1.30117100147861e-05	\\
-1448.59730113636	1.75670970852486e-05	\\
-1447.61851917614	4.4346216084493e-06	\\
-1446.63973721591	9.45310835676255e-06	\\
-1445.66095525568	9.06207344560402e-06	\\
-1444.68217329545	4.54234227167521e-06	\\
-1443.70339133523	1.01258663538878e-05	\\
-1442.724609375	1.61081967510552e-05	\\
-1441.74582741477	1.65540719614706e-05	\\
-1440.76704545455	1.7393918065167e-05	\\
-1439.78826349432	8.56943022131039e-06	\\
-1438.80948153409	4.61541925274933e-06	\\
-1437.83069957386	8.57604988485241e-06	\\
-1436.85191761364	8.37729682395601e-06	\\
-1435.87313565341	1.24637817888222e-05	\\
-1434.89435369318	8.1298841801468e-06	\\
-1433.91557173295	1.78601364907279e-05	\\
-1432.93678977273	4.50994487130407e-06	\\
-1431.9580078125	1.21969728381494e-05	\\
-1430.97922585227	1.07509877208572e-05	\\
-1430.00044389205	4.54242071105006e-06	\\
-1429.02166193182	1.36684802562368e-05	\\
-1428.04287997159	1.01329106470359e-05	\\
-1427.06409801136	1.07545087958006e-05	\\
-1426.08531605114	1.71328241431698e-05	\\
-1425.10653409091	8.34629526359392e-06	\\
-1424.12775213068	1.08362036329299e-05	\\
-1423.14897017045	1.29616889345405e-05	\\
-1422.17018821023	6.66656750256353e-06	\\
-1421.19140625	1.23302486032693e-05	\\
-1420.21262428977	1.13743617519873e-05	\\
-1419.23384232955	1.15843475665597e-05	\\
-1418.25506036932	1.67134570849099e-05	\\
-1417.27627840909	1.49008863415614e-05	\\
-1416.29749644886	1.67070626744324e-05	\\
-1415.31871448864	8.64840713133058e-06	\\
-1414.33993252841	2.08932022782432e-05	\\
-1413.36115056818	1.04944219994761e-05	\\
-1412.38236860795	9.99986572149664e-06	\\
-1411.40358664773	1.70264698229727e-05	\\
-1410.4248046875	4.87743413551185e-06	\\
-1409.44602272727	1.8074317564431e-05	\\
-1408.46724076705	1.9004387029762e-05	\\
-1407.48845880682	6.7130869561937e-06	\\
-1406.50967684659	1.2498493296002e-05	\\
-1405.53089488636	1.08360980159186e-05	\\
-1404.55211292614	3.12264728368405e-06	\\
-1403.57333096591	9.53787960603162e-06	\\
-1402.59454900568	7.27173246378989e-06	\\
-1401.61576704545	9.5499852551395e-06	\\
-1400.63698508523	1.215739924904e-05	\\
-1399.658203125	1.10777582328441e-05	\\
-1398.67942116477	7.88891717811061e-06	\\
-1397.70063920455	4.78203669913371e-06	\\
-1396.72185724432	1.21007004808183e-05	\\
-1395.74307528409	4.70158674800897e-06	\\
-1394.76429332386	1.36260062125536e-05	\\
-1393.78551136364	1.84101974215517e-05	\\
-1392.80672940341	2.17827391110651e-06	\\
-1391.82794744318	1.44476844103297e-05	\\
-1390.84916548295	5.19681556349809e-06	\\
-1389.87038352273	2.16200614725503e-05	\\
-1388.8916015625	1.40198616526493e-05	\\
-1387.91281960227	8.43153889081596e-06	\\
-1386.93403764205	8.93106439743154e-06	\\
-1385.95525568182	3.26175891798859e-06	\\
-1384.97647372159	1.01085034811057e-05	\\
-1383.99769176136	1.33424445504317e-05	\\
-1383.01890980114	1.60061734511357e-05	\\
-1382.04012784091	6.45376838532068e-06	\\
-1381.06134588068	1.84157674820218e-05	\\
-1380.08256392045	8.25809771603293e-06	\\
-1379.10378196023	1.54646700714568e-05	\\
-1378.125	2.63039989066248e-05	\\
-1377.14621803977	5.85176813177187e-06	\\
-1376.16743607955	1.66832100534302e-05	\\
-1375.18865411932	1.15023782831645e-05	\\
-1374.20987215909	9.14012054327563e-06	\\
-1373.23109019886	2.19287854159847e-05	\\
-1372.25230823864	1.59960070800277e-05	\\
-1371.27352627841	5.16480256727829e-06	\\
-1370.29474431818	1.07749616706695e-05	\\
-1369.31596235795	9.69418910361557e-06	\\
-1368.33718039773	2.18534614455876e-05	\\
-1367.3583984375	2.1750443644102e-05	\\
-1366.37961647727	2.14646271819215e-06	\\
-1365.40083451705	1.71831025872622e-05	\\
-1364.42205255682	1.06105502110972e-05	\\
-1363.44327059659	9.90896407123318e-06	\\
-1362.46448863636	1.4381598561615e-05	\\
-1361.48570667614	7.88973796584853e-06	\\
-1360.50692471591	9.98451326460565e-06	\\
-1359.52814275568	1.17024115194635e-05	\\
-1358.54936079545	2.98001214465288e-06	\\
-1357.57057883523	1.07386706690433e-05	\\
-1356.591796875	1.21172207977791e-05	\\
-1355.61301491477	6.5854386123324e-06	\\
-1354.63423295455	1.79722246024609e-05	\\
-1353.65545099432	1.10592306814643e-05	\\
-1352.67666903409	1.0749931279033e-05	\\
-1351.69788707386	1.31188669446476e-05	\\
-1350.71910511364	3.88992594896253e-06	\\
-1349.74032315341	1.47226026183624e-05	\\
-1348.76154119318	1.35173153535846e-05	\\
-1347.78275923295	8.28942091281758e-06	\\
-1346.80397727273	7.29821844587202e-06	\\
-1345.8251953125	1.02505483133125e-05	\\
-1344.84641335227	5.70306191079194e-06	\\
-1343.86763139205	1.09564316792703e-05	\\
-1342.88884943182	1.22977752595915e-05	\\
-1341.91006747159	7.1993825030783e-06	\\
-1340.93128551136	1.00222876366329e-05	\\
-1339.95250355114	6.95914871761922e-06	\\
-1338.97372159091	6.80299766012819e-06	\\
-1337.99493963068	1.25130994780762e-05	\\
-1337.01615767045	3.99276606859801e-06	\\
-1336.03737571023	9.59530908588798e-06	\\
-1335.05859375	1.2896993047217e-05	\\
-1334.07981178977	9.45825982005391e-06	\\
-1333.10102982955	1.54361891246914e-05	\\
-1332.12224786932	7.26938254752299e-06	\\
-1331.14346590909	6.74115187480742e-06	\\
-1330.16468394886	1.56698391874349e-05	\\
-1329.18590198864	8.34141230201277e-06	\\
-1328.20712002841	6.9391273289002e-06	\\
-1327.22833806818	1.45813522724804e-05	\\
-1326.24955610795	4.05905126957879e-06	\\
-1325.27077414773	1.30972578581124e-05	\\
-1324.2919921875	8.26131044480428e-06	\\
-1323.31321022727	8.14313415468121e-06	\\
-1322.33442826705	1.42564817849537e-05	\\
-1321.35564630682	3.71312682198514e-06	\\
-1320.37686434659	1.53372282281687e-05	\\
-1319.39808238636	8.58132032354649e-06	\\
-1318.41930042614	1.48155543721676e-05	\\
-1317.44051846591	1.25302887660247e-05	\\
-1316.46173650568	5.92581519135898e-06	\\
-1315.48295454545	1.08657943037746e-05	\\
-1314.50417258523	7.85919461724079e-06	\\
-1313.525390625	5.98021370073383e-06	\\
-1312.54660866477	1.21726123823182e-05	\\
-1311.56782670455	1.39204788772802e-05	\\
-1310.58904474432	1.22399982671906e-05	\\
-1309.61026278409	1.28211725348765e-05	\\
-1308.63148082386	1.07528815499336e-05	\\
-1307.65269886364	5.66578538976327e-06	\\
-1306.67391690341	1.0092824336911e-05	\\
-1305.69513494318	8.90669604449999e-06	\\
-1304.71635298295	1.63632736118051e-05	\\
-1303.73757102273	1.23754509504651e-05	\\
-1302.7587890625	1.743083900991e-05	\\
-1301.78000710227	1.20197476640348e-05	\\
-1300.80122514205	9.54464639559961e-06	\\
-1299.82244318182	1.47173956412356e-05	\\
-1298.84366122159	7.64792392708457e-06	\\
-1297.86487926136	1.06606600894048e-05	\\
-1296.88609730114	1.79839573507866e-05	\\
-1295.90731534091	7.41825615099329e-06	\\
-1294.92853338068	1.501156664068e-05	\\
-1293.94975142045	9.15301941833103e-06	\\
-1292.97096946023	3.27444612271e-06	\\
-1291.9921875	7.90400596450826e-06	\\
-1291.01340553977	1.0079786804988e-05	\\
-1290.03462357955	9.11374246656892e-06	\\
-1289.05584161932	1.24150823175604e-05	\\
-1288.07705965909	1.11048026379912e-05	\\
-1287.09827769886	1.24090758407343e-05	\\
-1286.11949573864	5.23146610846226e-06	\\
-1285.14071377841	1.24845905553285e-05	\\
-1284.16193181818	7.84999947897632e-06	\\
-1283.18314985795	5.21740916096715e-06	\\
-1282.20436789773	1.12287799284438e-05	\\
-1281.2255859375	8.17612205048223e-06	\\
-1280.24680397727	1.34414106073748e-05	\\
-1279.26802201705	1.23964491625031e-05	\\
-1278.28924005682	1.07302308881419e-05	\\
-1277.31045809659	1.39645707098146e-05	\\
-1276.33167613636	7.79402865017555e-06	\\
-1275.35289417614	1.22705309386352e-05	\\
-1274.37411221591	3.92301581841256e-06	\\
-1273.39533025568	2.61547810354191e-06	\\
-1272.41654829545	1.05841820413776e-05	\\
-1271.43776633523	8.91658475668634e-06	\\
-1270.458984375	6.64556798131797e-06	\\
-1269.48020241477	8.26844027826793e-06	\\
-1268.50142045455	9.03531799411671e-06	\\
-1267.52263849432	7.7250705168885e-06	\\
-1266.54385653409	4.75801340708133e-06	\\
-1265.56507457386	8.41378788455722e-06	\\
-1264.58629261364	4.75346865066084e-06	\\
-1263.60751065341	3.73749969196998e-06	\\
-1262.62872869318	5.56480276619485e-06	\\
-1261.64994673295	6.60205456033036e-06	\\
-1260.67116477273	1.07028291628836e-05	\\
-1259.6923828125	1.07075481628958e-05	\\
-1258.71360085227	7.14191966560934e-06	\\
-1257.73481889205	2.94168417028414e-06	\\
-1256.75603693182	4.47002435550051e-06	\\
-1255.77725497159	9.80469147955778e-06	\\
-1254.79847301136	5.50654927332256e-06	\\
-1253.81969105114	8.11085437508601e-06	\\
-1252.84090909091	7.05656783110871e-06	\\
-1251.86212713068	7.1001525867978e-06	\\
-1250.88334517045	7.10908093769383e-06	\\
-1249.90456321023	1.33274503220765e-05	\\
-1248.92578125	1.62017163903201e-05	\\
-1247.94699928977	7.11427830935232e-06	\\
-1246.96821732955	1.29488692792403e-05	\\
-1245.98943536932	6.67187439269918e-06	\\
-1245.01065340909	6.30631477846083e-06	\\
-1244.03187144886	3.54547920866195e-06	\\
-1243.05308948864	7.44470598324761e-06	\\
-1242.07430752841	7.41632434115994e-06	\\
-1241.09552556818	9.02313336417152e-06	\\
-1240.11674360795	7.80834092628696e-06	\\
-1239.13796164773	9.0192428684559e-06	\\
-1238.1591796875	1.20576664660085e-05	\\
-1237.18039772727	5.13913238738955e-06	\\
-1236.20161576705	7.19117750455514e-06	\\
-1235.22283380682	9.71144153973643e-06	\\
-1234.24405184659	7.52699605295662e-06	\\
-1233.26526988636	6.75581608285547e-06	\\
-1232.28648792614	1.50179906982567e-05	\\
-1231.30770596591	1.09409455104174e-05	\\
-1230.32892400568	1.12015039384939e-05	\\
-1229.35014204545	9.45024265226184e-06	\\
-1228.37136008523	1.10002773272833e-05	\\
-1227.392578125	1.1718703925947e-05	\\
-1226.41379616477	1.54858412794454e-05	\\
-1225.43501420455	1.37186234487702e-05	\\
-1224.45623224432	1.60905849018784e-05	\\
-1223.47745028409	1.09797922178017e-05	\\
-1222.49866832386	1.08078190618122e-05	\\
-1221.51988636364	1.68580595065481e-05	\\
-1220.54110440341	1.287977751018e-05	\\
-1219.56232244318	7.70756382849721e-06	\\
-1218.58354048295	1.2307174761543e-05	\\
-1217.60475852273	1.10652440907007e-05	\\
-1216.6259765625	1.23439101064888e-05	\\
-1215.64719460227	1.14952845229068e-05	\\
-1214.66841264205	1.22312677904045e-05	\\
-1213.68963068182	1.45745304549215e-05	\\
-1212.71084872159	1.37553219752686e-05	\\
-1211.73206676136	9.23395306417326e-06	\\
-1210.75328480114	1.36180895316908e-05	\\
-1209.77450284091	7.39072679648876e-06	\\
-1208.79572088068	1.08417194280262e-05	\\
-1207.81693892045	1.20530346708422e-05	\\
-1206.83815696023	1.167857950408e-05	\\
-1205.859375	1.03280479290556e-05	\\
-1204.88059303977	1.13282191969398e-05	\\
-1203.90181107955	9.4553543504466e-06	\\
-1202.92302911932	6.22127297037917e-06	\\
-1201.94424715909	1.03606257309679e-05	\\
-1200.96546519886	1.25141105171177e-05	\\
-1199.98668323864	1.45988253950282e-05	\\
-1199.00790127841	6.21375615653072e-06	\\
-1198.02911931818	5.29957414348622e-06	\\
-1197.05033735795	6.5503472612136e-06	\\
-1196.07155539773	4.62544699639083e-06	\\
-1195.0927734375	9.0180767349642e-06	\\
-1194.11399147727	9.07301657207746e-06	\\
-1193.13520951705	5.90404081533451e-06	\\
-1192.15642755682	9.37910441634512e-06	\\
-1191.17764559659	1.07718161197994e-05	\\
-1190.19886363636	1.00306973759267e-05	\\
-1189.22008167614	7.46349598163707e-06	\\
-1188.24129971591	6.62280063914188e-06	\\
-1187.26251775568	6.55169905222148e-06	\\
-1186.28373579545	8.80224401641635e-06	\\
-1185.30495383523	5.97335451516298e-06	\\
-1184.326171875	9.03074382496242e-06	\\
-1183.34738991477	9.33171999275498e-06	\\
-1182.36860795455	2.62550575535645e-06	\\
-1181.38982599432	8.40080543875603e-06	\\
-1180.41104403409	1.22895106731155e-05	\\
-1179.43226207386	1.13266872226906e-05	\\
-1178.45348011364	1.08558614627667e-05	\\
-1177.47469815341	1.09046356981769e-05	\\
-1176.49591619318	1.37788061789986e-05	\\
-1175.51713423295	1.26197221787241e-05	\\
-1174.53835227273	1.0618445288138e-05	\\
-1173.5595703125	8.55984136744832e-06	\\
-1172.58078835227	1.62001850362918e-05	\\
-1171.60200639205	1.21403713088325e-05	\\
-1170.62322443182	9.58331056188795e-06	\\
-1169.64444247159	1.09876594153717e-05	\\
-1168.66566051136	1.03501038205385e-05	\\
-1167.68687855114	7.90523146412728e-06	\\
-1166.70809659091	1.01482050961201e-05	\\
-1165.72931463068	1.4190913927122e-05	\\
-1164.75053267045	1.33944313464387e-05	\\
-1163.77175071023	1.17918509856363e-05	\\
-1162.79296875	7.03722436739402e-06	\\
-1161.81418678977	1.35885563311806e-05	\\
-1160.83540482955	9.90096900071875e-06	\\
-1159.85662286932	1.10729563615659e-05	\\
-1158.87784090909	1.03474490290572e-05	\\
-1157.89905894886	7.77435885600993e-06	\\
-1156.92027698864	1.03789280854802e-05	\\
-1155.94149502841	1.02315168202171e-05	\\
-1154.96271306818	1.33992442884348e-05	\\
-1153.98393110795	5.95638175355593e-06	\\
-1153.00514914773	8.04357259957754e-06	\\
-1152.0263671875	7.0872550895243e-06	\\
-1151.04758522727	7.58180836149154e-06	\\
-1150.06880326705	1.17408686618369e-05	\\
-1149.09002130682	6.6013025482515e-06	\\
-1148.11123934659	8.23872230546995e-06	\\
-1147.13245738636	8.69459081193106e-06	\\
-1146.15367542614	6.0557370504897e-06	\\
-1145.17489346591	6.71494207843649e-06	\\
-1144.19611150568	1.1366099302399e-05	\\
-1143.21732954545	7.55918600872317e-06	\\
-1142.23854758523	7.5413996434982e-06	\\
-1141.259765625	7.14060742093188e-06	\\
-1140.28098366477	8.54165826205653e-06	\\
-1139.30220170455	2.07590172991307e-06	\\
-1138.32341974432	5.04395376623217e-06	\\
-1137.34463778409	4.92291385677401e-06	\\
-1136.36585582386	3.5202698730979e-06	\\
-1135.38707386364	2.47234786193467e-06	\\
-1134.40829190341	7.15657179245051e-06	\\
-1133.42950994318	9.61351808703352e-06	\\
-1132.45072798295	7.78368615846101e-06	\\
-1131.47194602273	6.53073278824043e-06	\\
-1130.4931640625	7.49979159816417e-06	\\
-1129.51438210227	1.6201167961502e-06	\\
-1128.53560014205	3.65742634702118e-06	\\
-1127.55681818182	6.18353591680978e-06	\\
-1126.57803622159	1.34380267679161e-05	\\
-1125.59925426136	7.76875381876684e-06	\\
-1124.62047230114	1.0554451657729e-05	\\
-1123.64169034091	7.50229890154943e-06	\\
-1122.66290838068	9.05133627972013e-06	\\
-1121.68412642045	1.2389531794371e-05	\\
-1120.70534446023	7.25011580552962e-06	\\
-1119.7265625	9.30477284311147e-06	\\
-1118.74778053977	1.04006930973648e-05	\\
-1117.76899857955	7.11004247747096e-06	\\
-1116.79021661932	1.26803989112612e-05	\\
-1115.81143465909	1.12174577273591e-05	\\
-1114.83265269886	1.24481790539709e-05	\\
-1113.85387073864	7.44919140101904e-06	\\
-1112.87508877841	8.2039876853354e-06	\\
-1111.89630681818	1.68676379079747e-05	\\
-1110.91752485795	1.16988641478981e-05	\\
-1109.93874289773	8.45860406562998e-06	\\
-1108.9599609375	1.2993249312068e-05	\\
-1107.98117897727	1.02388664349503e-05	\\
-1107.00239701705	1.09926098550895e-05	\\
-1106.02361505682	9.66200391958416e-06	\\
-1105.04483309659	1.30044097356681e-05	\\
-1104.06605113636	1.20781794123939e-05	\\
-1103.08726917614	1.1025398260549e-05	\\
-1102.10848721591	1.46823360433482e-05	\\
-1101.12970525568	9.74290488858655e-06	\\
-1100.15092329545	9.99536150111613e-06	\\
-1099.17214133523	9.96473727722851e-06	\\
-1098.193359375	1.02152080242537e-05	\\
-1097.21457741477	1.84113583550455e-05	\\
-1096.23579545455	7.5415234489666e-06	\\
-1095.25701349432	9.21746023096635e-06	\\
-1094.27823153409	1.16363430470791e-05	\\
-1093.29944957386	9.9460232941549e-06	\\
-1092.32066761364	8.89609476890923e-06	\\
-1091.34188565341	8.95872108738702e-06	\\
-1090.36310369318	1.06731873950392e-05	\\
-1089.38432173295	1.3570863976925e-05	\\
-1088.40553977273	1.42912957478714e-05	\\
-1087.4267578125	1.18325200416979e-05	\\
-1086.44797585227	1.10213458288785e-05	\\
-1085.46919389205	1.04611596183083e-05	\\
-1084.49041193182	1.19111958972825e-05	\\
-1083.51162997159	1.62993667790829e-05	\\
-1082.53284801136	1.30912664704722e-05	\\
-1081.55406605114	1.18370849360404e-05	\\
-1080.57528409091	1.00409913243755e-05	\\
-1079.59650213068	2.62371605669031e-06	\\
-1078.61772017045	1.00484127129623e-05	\\
-1077.63893821023	1.2058748629306e-05	\\
-1076.66015625	1.067639913683e-05	\\
-1075.68137428977	5.74909722272262e-06	\\
-1074.70259232955	6.93929221601208e-06	\\
-1073.72381036932	6.54938440392944e-06	\\
-1072.74502840909	9.72545215778811e-06	\\
-1071.76624644886	8.49277911734537e-06	\\
-1070.78746448864	6.97805861416286e-06	\\
-1069.80868252841	8.33159410581573e-06	\\
-1068.82990056818	1.74133217228282e-06	\\
-1067.85111860795	1.17154579051536e-05	\\
-1066.87233664773	1.58861465466736e-05	\\
-1065.8935546875	1.17039980382201e-05	\\
-1064.91477272727	9.14145436160171e-07	\\
-1063.93599076705	3.74800158625016e-06	\\
-1062.95720880682	1.34102590276068e-05	\\
-1061.97842684659	1.51214755045394e-05	\\
-1060.99964488636	8.65733714526618e-06	\\
-1060.02086292614	8.94662853170955e-06	\\
-1059.04208096591	6.37479497232501e-06	\\
-1058.06329900568	1.09357966116547e-05	\\
-1057.08451704545	1.4003347603734e-05	\\
-1056.10573508523	1.8384278264639e-05	\\
-1055.126953125	1.11117215388105e-05	\\
-1054.14817116477	5.64686988704405e-06	\\
-1053.16938920455	9.90228462633708e-06	\\
-1052.19060724432	1.66705735130663e-05	\\
-1051.21182528409	9.16632641461136e-06	\\
-1050.23304332386	7.32274509468659e-06	\\
-1049.25426136364	7.50773248330855e-06	\\
-1048.27547940341	7.26836068760702e-06	\\
-1047.29669744318	8.32025182488744e-06	\\
-1046.31791548295	8.86481055539548e-06	\\
-1045.33913352273	1.33404294245084e-05	\\
-1044.3603515625	1.08435069010592e-05	\\
-1043.38156960227	5.89075220969072e-06	\\
-1042.40278764205	5.96024651236381e-06	\\
-1041.42400568182	7.9723203592375e-06	\\
-1040.44522372159	6.13860416699048e-06	\\
-1039.46644176136	1.30239767106442e-05	\\
-1038.48765980114	1.62984640903105e-05	\\
-1037.50887784091	1.15172346922684e-05	\\
-1036.53009588068	7.11440073575909e-06	\\
-1035.55131392045	2.77664646603062e-06	\\
-1034.57253196023	4.13500890530391e-06	\\
-1033.59375	1.1486063684764e-05	\\
-1032.61496803977	6.85967196569204e-06	\\
-1031.63618607955	1.18974314642561e-05	\\
-1030.65740411932	5.12765956899441e-06	\\
-1029.67862215909	7.79169588058177e-06	\\
-1028.69984019886	2.47733701149496e-06	\\
-1027.72105823864	8.79581034794135e-06	\\
-1026.74227627841	5.71180009269906e-06	\\
-1025.76349431818	9.36009347372494e-06	\\
-1024.78471235795	5.91996493474573e-06	\\
-1023.80593039773	5.15387337207124e-06	\\
-1022.8271484375	8.35389036141812e-06	\\
-1021.84836647727	1.29736515803825e-05	\\
-1020.86958451705	1.06082774763624e-05	\\
-1019.89080255682	1.05567700610577e-05	\\
-1018.91202059659	9.22311486772722e-06	\\
-1017.93323863636	4.97038618875776e-06	\\
-1016.95445667614	3.81217599002186e-06	\\
-1015.97567471591	2.79363091380843e-06	\\
-1014.99689275568	7.18207842963344e-06	\\
-1014.01811079545	6.89990647264027e-06	\\
-1013.03932883523	1.10299291599532e-05	\\
-1012.060546875	2.12956362244634e-05	\\
-1011.08176491477	9.13867764038104e-06	\\
-1010.10298295455	9.89061019896993e-06	\\
-1009.12420099432	1.15481690405207e-05	\\
-1008.14541903409	9.46886774360455e-06	\\
-1007.16663707386	6.05926570944155e-06	\\
-1006.18785511364	5.59360903441178e-06	\\
-1005.20907315341	7.08374495453258e-06	\\
-1004.23029119318	8.91500010118696e-06	\\
-1003.25150923295	1.28762168960721e-05	\\
-1002.27272727273	9.59019589746454e-06	\\
-1001.2939453125	3.6030158802965e-06	\\
-1000.31516335227	2.81135100722844e-05	\\
-999.336381392046	1.19841111879004e-05	\\
-998.357599431818	1.26354971824416e-05	\\
-997.378817471592	7.96860184991757e-06	\\
-996.400035511364	9.5663946485362e-06	\\
-995.421253551136	1.00956098708823e-05	\\
-994.44247159091	3.47047166456709e-06	\\
-993.463689630682	2.2880905946585e-06	\\
-992.484907670454	7.98759681729701e-06	\\
-991.506125710228	8.24152103198878e-06	\\
-990.52734375	1.14838441803864e-05	\\
-989.548561789772	6.24723418387067e-06	\\
-988.569779829546	9.88103254735878e-06	\\
-987.590997869318	7.23500558350733e-06	\\
-986.612215909092	1.19164712066043e-05	\\
-985.633433948864	3.64720921283309e-06	\\
-984.654651988636	9.41729334561859e-06	\\
-983.67587002841	8.61169088676424e-06	\\
-982.697088068182	7.09966934649343e-06	\\
-981.718306107954	6.05729616669828e-06	\\
-980.739524147728	8.11387077006538e-06	\\
-979.7607421875	6.60025971903095e-06	\\
-978.781960227272	9.14304566576554e-06	\\
-977.803178267046	7.9810595185391e-06	\\
-976.824396306818	3.03785000776773e-06	\\
-975.845614346592	1.09660647860393e-05	\\
-974.866832386364	8.28602823009156e-06	\\
-973.888050426136	5.10022674741144e-06	\\
-972.90926846591	9.4137673958113e-06	\\
-971.930486505682	4.64200652663155e-06	\\
-970.951704545454	3.15799498132154e-06	\\
-969.972922585228	7.63718557378045e-06	\\
-968.994140625	3.35694260032223e-06	\\
-968.015358664772	2.23229877566085e-06	\\
-967.036576704546	1.05611979986364e-05	\\
-966.057794744318	3.80243562126873e-06	\\
-965.079012784092	4.47596969549496e-06	\\
-964.100230823864	2.27799677269552e-06	\\
-963.121448863636	2.23185613528177e-06	\\
-962.14266690341	6.00825602023125e-06	\\
-961.163884943182	1.68674312911515e-06	\\
-960.185102982954	4.8052303497627e-06	\\
-959.206321022728	8.82650260557858e-06	\\
-958.2275390625	6.79716159695662e-06	\\
-957.248757102272	7.76371501742069e-06	\\
-956.269975142046	8.61171027389707e-06	\\
-955.291193181818	6.25821214318357e-06	\\
-954.312411221592	5.52629458217159e-06	\\
-953.333629261364	7.23860107670819e-06	\\
-952.354847301136	3.71738949730098e-06	\\
-951.37606534091	9.96586590734124e-06	\\
-950.397283380682	7.80215360052003e-06	\\
-949.418501420454	6.58318757590015e-06	\\
-948.439719460228	9.97314640009068e-06	\\
-947.4609375	4.42392119074486e-06	\\
-946.482155539772	4.32911002125061e-06	\\
-945.503373579546	1.08230112284406e-05	\\
-944.524591619318	7.71839739557625e-06	\\
-943.545809659092	4.74903120790488e-06	\\
-942.567027698864	5.62414319801276e-06	\\
-941.588245738636	4.34095481894753e-06	\\
-940.60946377841	6.60931757342586e-06	\\
-939.630681818182	8.23630020828433e-06	\\
-938.651899857954	5.99574363847741e-06	\\
-937.673117897728	7.5013032624823e-06	\\
-936.6943359375	5.60881191339676e-06	\\
-935.715553977272	3.84688392737244e-06	\\
-934.736772017046	3.39853639698561e-06	\\
-933.757990056818	2.5419981396885e-06	\\
-932.779208096592	1.05346519528849e-05	\\
-931.800426136364	1.02418888693076e-05	\\
-930.821644176136	7.3727708488426e-06	\\
-929.84286221591	9.686889731899e-06	\\
-928.864080255682	1.0635564058816e-06	\\
-927.885298295454	6.56289656677167e-06	\\
-926.906516335228	8.76797838192572e-06	\\
-925.927734375	7.80371452169384e-06	\\
-924.948952414772	1.49822631674967e-05	\\
-923.970170454546	1.6677414849699e-05	\\
-922.991388494318	6.12571938401897e-06	\\
-922.012606534092	1.55076581679202e-05	\\
-921.033824573864	9.86015790418805e-06	\\
-920.055042613636	7.14992755115382e-06	\\
-919.07626065341	6.42987258456014e-06	\\
-918.097478693182	7.06650811059016e-06	\\
-917.118696732954	1.57671173570567e-05	\\
-916.139914772728	1.70321928982999e-05	\\
-915.1611328125	9.13400840901863e-06	\\
-914.182350852272	5.33014444320465e-06	\\
-913.203568892046	9.49710073485067e-06	\\
-912.224786931818	1.77544562193731e-06	\\
-911.246004971592	9.40084959916275e-06	\\
-910.267223011364	1.83279624635255e-05	\\
-909.288441051136	1.38524535814339e-05	\\
-908.30965909091	7.86560175106121e-06	\\
-907.330877130682	1.94387091883678e-05	\\
-906.352095170454	1.04503885731939e-05	\\
-905.373313210228	3.6803965022991e-06	\\
-904.39453125	2.07729696269395e-06	\\
-903.415749289772	5.64412119477902e-06	\\
-902.436967329546	1.40257295111483e-05	\\
-901.458185369318	1.30031263296906e-05	\\
-900.479403409092	3.53377594816043e-06	\\
-899.500621448864	8.11982938295713e-06	\\
-898.521839488636	5.21911396041434e-06	\\
-897.54305752841	1.22331018754663e-05	\\
-896.564275568182	1.02772349048963e-05	\\
-895.585493607954	5.52800032878963e-06	\\
-894.606711647728	8.71079306677644e-06	\\
-893.6279296875	8.51589504182375e-06	\\
-892.649147727272	1.29556171611014e-05	\\
-891.670365767046	9.7125192984708e-06	\\
-890.691583806818	9.18189969971419e-06	\\
-889.712801846592	1.00789146411308e-05	\\
-888.734019886364	1.02090831310253e-05	\\
-887.755237926136	8.47688519320335e-06	\\
-886.77645596591	2.28083928315843e-06	\\
-885.797674005682	2.57132124182708e-06	\\
-884.818892045454	8.15647290107978e-06	\\
-883.840110085228	2.0948347298975e-06	\\
-882.861328125	6.83074962917844e-06	\\
-881.882546164772	1.04047662820388e-05	\\
-880.903764204546	7.61101209874841e-06	\\
-879.924982244318	9.12129521098305e-06	\\
-878.946200284092	1.18420852003443e-05	\\
-877.967418323864	1.01063741154809e-05	\\
-876.988636363636	4.1460515174794e-06	\\
-876.00985440341	8.80307639490868e-06	\\
-875.031072443182	4.6887358587395e-06	\\
-874.052290482954	6.93498737709038e-06	\\
-873.073508522728	1.14035412758635e-05	\\
-872.0947265625	6.54372684504269e-06	\\
-871.115944602272	1.24633243093069e-05	\\
-870.137162642046	1.33237228382307e-05	\\
-869.158380681818	8.14351404860428e-06	\\
-868.179598721592	7.7154437303769e-06	\\
-867.200816761364	1.59537943236991e-05	\\
-866.222034801136	1.35161783712881e-05	\\
-865.24325284091	9.97730239881319e-06	\\
-864.264470880682	9.86001420612713e-06	\\
-863.285688920454	1.05057686041331e-05	\\
-862.306906960228	7.09819770185197e-06	\\
-861.328125	1.05624338729901e-05	\\
-860.349343039772	9.51448266952116e-06	\\
-859.370561079546	5.61879584313254e-06	\\
-858.391779119318	7.35775983015573e-06	\\
-857.412997159092	7.29183154241366e-06	\\
-856.434215198864	6.35350439931475e-06	\\
-855.455433238636	8.08985542764708e-06	\\
-854.47665127841	8.56194620101915e-06	\\
-853.497869318182	3.45028650008394e-06	\\
-852.519087357954	1.37841405655506e-05	\\
-851.540305397728	7.94820100905439e-06	\\
-850.5615234375	7.34244273935134e-06	\\
-849.582741477272	5.02827705615746e-06	\\
-848.603959517046	4.77330395230023e-06	\\
-847.625177556818	3.36325542917691e-06	\\
-846.646395596592	1.0462015866135e-05	\\
-845.667613636364	6.17173773126315e-06	\\
-844.688831676136	8.06600804984206e-06	\\
-843.71004971591	7.25138096342439e-06	\\
-842.731267755682	6.63696171390803e-06	\\
-841.752485795454	6.7520970236428e-06	\\
-840.773703835228	4.86009704747846e-06	\\
-839.794921875	1.0501388535616e-05	\\
-838.816139914772	7.29971108787024e-06	\\
-837.837357954546	5.57289290581596e-06	\\
-836.858575994318	1.04654006941378e-05	\\
-835.879794034092	6.17340176548879e-06	\\
-834.901012073864	1.05688487664085e-05	\\
-833.922230113636	9.47536142703511e-06	\\
-832.94344815341	4.10904443837068e-06	\\
-831.964666193182	6.25929369639964e-06	\\
-830.985884232954	7.30693273336828e-06	\\
-830.007102272728	7.92585522916879e-06	\\
-829.0283203125	3.98231260996256e-06	\\
-828.049538352272	6.69210652852351e-06	\\
-827.070756392046	5.10995121612773e-06	\\
-826.091974431818	5.29948983779813e-06	\\
-825.113192471592	7.19966921146201e-06	\\
-824.134410511364	9.24067726780044e-06	\\
-823.155628551136	9.83754319824707e-06	\\
-822.17684659091	9.14174717331691e-06	\\
-821.198064630682	4.74952048032989e-06	\\
-820.219282670454	7.94000689128859e-06	\\
-819.240500710228	8.88282809790017e-06	\\
-818.26171875	6.90159181943916e-06	\\
-817.282936789772	8.31170978395677e-06	\\
-816.304154829546	1.31966734368966e-05	\\
-815.325372869318	8.42284721780795e-06	\\
-814.346590909092	7.47131279362935e-06	\\
-813.367808948864	1.07368420737707e-05	\\
-812.389026988636	3.82637059619624e-06	\\
-811.41024502841	7.56290878858539e-06	\\
-810.431463068182	7.13981337560786e-06	\\
-809.452681107954	1.26019975784314e-05	\\
-808.473899147728	1.16467964971281e-05	\\
-807.4951171875	8.08677236997562e-06	\\
-806.516335227272	2.56363748448601e-06	\\
-805.537553267046	1.04620329928171e-05	\\
-804.558771306818	1.10014090297123e-05	\\
-803.579989346592	1.02380632490192e-05	\\
-802.601207386364	9.46728870650038e-06	\\
-801.622425426136	8.95857181944531e-06	\\
-800.64364346591	9.59103710857123e-06	\\
-799.664861505682	8.48799139271876e-06	\\
-798.686079545454	7.16062111769029e-06	\\
-797.707297585228	9.75069468704481e-06	\\
-796.728515625	1.17489098899105e-05	\\
-795.749733664772	1.0459760694485e-05	\\
-794.770951704546	1.24867463309669e-05	\\
-793.792169744318	9.67479118083649e-06	\\
-792.813387784092	1.0075578053821e-05	\\
-791.834605823864	1.25417431521347e-05	\\
-790.855823863636	7.94201797354474e-06	\\
-789.87704190341	9.98480349246685e-06	\\
-788.898259943182	1.15349548544361e-05	\\
-787.919477982954	1.45110835145044e-05	\\
-786.940696022728	1.33695330016656e-05	\\
-785.9619140625	1.74643393666025e-05	\\
-784.983132102272	1.33503240911817e-05	\\
-784.004350142046	1.07022562994121e-05	\\
-783.025568181818	8.58026006011166e-06	\\
-782.046786221592	1.22444145869418e-05	\\
-781.068004261364	1.43647468893023e-05	\\
-780.089222301136	6.84679135046073e-06	\\
-779.11044034091	1.65478762538527e-05	\\
-778.131658380682	1.00633774506968e-05	\\
-777.152876420454	1.76901512190201e-05	\\
-776.174094460228	1.43128742273163e-05	\\
-775.1953125	5.63094223916358e-06	\\
-774.216530539772	1.17235646985265e-05	\\
-773.237748579546	7.89904604263783e-06	\\
-772.258966619318	1.79746346577852e-05	\\
-771.280184659092	1.69899835800057e-05	\\
-770.301402698864	1.57844621878945e-05	\\
-769.322620738636	1.17541215389862e-05	\\
-768.34383877841	2.39118485118223e-06	\\
-767.365056818182	8.8434797253929e-06	\\
-766.386274857954	1.38062844386122e-05	\\
-765.407492897728	1.21339582967254e-05	\\
-764.4287109375	1.32694250404478e-05	\\
-763.449928977272	8.19791235965212e-06	\\
-762.471147017046	1.15919788573932e-05	\\
-761.492365056818	1.01619822289372e-05	\\
-760.513583096592	1.60405822575918e-05	\\
-759.534801136364	9.37722105167135e-06	\\
-758.556019176136	8.07042682065754e-06	\\
-757.57723721591	5.07556077128489e-06	\\
-756.598455255682	6.31505515251901e-06	\\
-755.619673295454	1.31892141425952e-05	\\
-754.640891335228	1.54251529336085e-05	\\
-753.662109375	6.4667713698425e-06	\\
-752.683327414772	5.64397193730172e-06	\\
-751.704545454546	8.27709036382596e-06	\\
-750.725763494318	6.73701796500205e-06	\\
-749.746981534092	2.23523901379029e-05	\\
-748.768199573864	1.05520611732442e-05	\\
-747.789417613636	1.14366819100006e-05	\\
-746.81063565341	8.04445481949173e-06	\\
-745.831853693182	1.00941836160963e-05	\\
-744.853071732954	1.47230950880362e-05	\\
-743.874289772728	8.75158212938686e-06	\\
-742.8955078125	1.63409467253887e-05	\\
-741.916725852272	1.26334767766154e-05	\\
-740.937943892046	1.53104355190422e-05	\\
-739.959161931818	1.14003993906336e-05	\\
-738.980379971592	1.6297196046816e-05	\\
-738.001598011364	1.41354423661959e-05	\\
-737.022816051136	1.43108634668959e-05	\\
-736.04403409091	1.47683769301327e-05	\\
-735.065252130682	1.56967639718519e-05	\\
-734.086470170454	1.29977973898349e-05	\\
-733.107688210228	1.18382152647408e-05	\\
-732.12890625	1.40907273885728e-05	\\
-731.150124289772	1.05729468016188e-05	\\
-730.171342329546	9.18360658672625e-06	\\
-729.192560369318	7.94905825786957e-06	\\
-728.213778409092	1.91710687756567e-05	\\
-727.234996448864	9.63395895941414e-06	\\
-726.256214488636	1.19695461837847e-05	\\
-725.27743252841	1.1402824159983e-05	\\
-724.298650568182	5.12825935312144e-06	\\
-723.319868607954	9.03123275924938e-06	\\
-722.341086647728	6.5817086973121e-06	\\
-721.3623046875	9.92753004690333e-06	\\
-720.383522727272	2.86893848250273e-06	\\
-719.404740767046	7.7792294325851e-06	\\
-718.425958806818	1.20598248980198e-05	\\
-717.447176846592	9.98723980284729e-06	\\
-716.468394886364	1.19594030744756e-05	\\
-715.489612926136	1.11786356872025e-05	\\
-714.51083096591	1.21886312432749e-05	\\
-713.532049005682	1.11461991630525e-05	\\
-712.553267045454	1.29994479099606e-05	\\
-711.574485085228	1.82953632879265e-05	\\
-710.595703125	1.13954632102812e-05	\\
-709.616921164772	1.37548231646215e-05	\\
-708.638139204546	1.35182217215366e-05	\\
-707.659357244318	1.53425393734741e-05	\\
-706.680575284092	1.55196647938232e-05	\\
-705.701793323864	1.7906289149152e-05	\\
-704.723011363636	1.89492677963594e-05	\\
-703.74422940341	1.70236945959639e-05	\\
-702.765447443182	1.87089329557709e-05	\\
-701.786665482954	1.91518656312496e-05	\\
-700.807883522728	1.13192555010796e-05	\\
-699.8291015625	3.33142741701868e-05	\\
-698.850319602272	1.44169774599281e-05	\\
-697.871537642046	1.69594843515416e-05	\\
-696.892755681818	2.00248170728229e-05	\\
-695.913973721592	4.90738560964978e-06	\\
-694.935191761364	1.50597907773392e-05	\\
-693.956409801136	1.7769203575069e-05	\\
-692.97762784091	1.1934819413856e-05	\\
-691.998845880682	1.98007977301244e-05	\\
-691.020063920454	1.97269679679445e-05	\\
-690.041281960228	1.98469080438256e-05	\\
-689.0625	2.35305450702349e-05	\\
-688.083718039772	2.56658736056629e-05	\\
-687.104936079546	2.17783371478421e-05	\\
-686.126154119318	2.46687643086356e-05	\\
-685.147372159092	2.27162318826423e-05	\\
-684.168590198864	1.92394059263728e-05	\\
-683.189808238636	3.79897401223547e-05	\\
-682.21102627841	3.69580097685436e-05	\\
-681.232244318182	3.3523392193044e-05	\\
-680.253462357954	3.5996285117813e-05	\\
-679.274680397728	2.98605332031963e-05	\\
-678.2958984375	3.22140394735008e-05	\\
-677.317116477272	3.53223147888389e-05	\\
-676.338334517046	2.31464024786107e-05	\\
-675.359552556818	3.00759321735865e-05	\\
-674.380770596592	2.88902647770207e-05	\\
-673.401988636364	3.48981774794996e-05	\\
-672.423206676136	4.0428570941312e-05	\\
-671.44442471591	3.35264607019896e-05	\\
-670.465642755682	5.20662332828683e-05	\\
-669.486860795454	3.0090201487546e-05	\\
-668.508078835228	2.90741829719006e-05	\\
-667.529296875	2.59255568312027e-05	\\
-666.550514914772	1.62260166657232e-05	\\
-665.571732954546	3.53688829578907e-05	\\
-664.592950994318	4.86846528015266e-05	\\
-663.614169034092	4.70629115614526e-05	\\
-662.635387073864	5.62274621976392e-05	\\
-661.656605113636	3.91640411796941e-05	\\
-660.67782315341	4.30419522016438e-05	\\
-659.699041193182	2.6092581273184e-05	\\
-658.720259232954	2.10363999467356e-05	\\
-657.741477272728	1.23073612397053e-05	\\
-656.7626953125	2.34442969382713e-05	\\
-655.783913352272	3.71060138956302e-05	\\
-654.805131392046	5.30088928637181e-05	\\
-653.826349431818	5.0908063400814e-05	\\
-652.847567471592	6.56071061023532e-05	\\
-651.868785511364	4.30700836945287e-05	\\
-650.890003551136	4.75674883739542e-05	\\
-649.91122159091	2.54383608988532e-05	\\
-648.932439630682	3.00381043169861e-05	\\
-647.953657670454	4.47240745200298e-05	\\
-646.974875710228	4.72802508038985e-05	\\
-645.99609375	4.39620030936809e-05	\\
-645.017311789772	3.83768043148119e-05	\\
-644.038529829546	5.24663017157696e-05	\\
-643.059747869318	4.65424916489808e-05	\\
-642.080965909092	5.33111106885104e-05	\\
-641.102183948864	4.79748815441946e-05	\\
-640.123401988636	4.69085516156146e-05	\\
-639.14462002841	3.88059682012847e-05	\\
-638.165838068182	3.54024323279945e-06	\\
-637.187056107954	1.08430713596323e-05	\\
-636.208274147728	4.80790219843066e-05	\\
-635.2294921875	4.14979976505967e-05	\\
-634.250710227272	7.00523081128043e-05	\\
-633.271928267046	6.19956924876555e-05	\\
-632.293146306818	8.46246179664273e-05	\\
-631.314364346592	7.71576795047126e-05	\\
-630.335582386364	4.27023322098507e-05	\\
-629.356800426136	3.15550947102572e-05	\\
-628.37801846591	4.64549251042397e-05	\\
-627.399236505682	1.7800226035812e-05	\\
-626.420454545454	2.85767721659096e-05	\\
-625.441672585228	2.58469761618378e-05	\\
-624.462890625	6.03718472755304e-05	\\
-623.484108664772	4.95281759626478e-05	\\
-622.505326704546	4.6683716459238e-05	\\
-621.526544744318	4.51476293152924e-05	\\
-620.547762784092	8.474733476124e-05	\\
-619.568980823864	7.70947070073431e-05	\\
-618.590198863636	4.8670095681629e-05	\\
-617.61141690341	1.80094786569487e-05	\\
-616.632634943182	4.33560956344769e-05	\\
-615.653852982954	4.80761065513496e-05	\\
-614.675071022728	2.55397071988027e-05	\\
-613.6962890625	1.38968972589705e-05	\\
-612.717507102272	3.6285311660664e-05	\\
-611.738725142046	3.65001309682791e-05	\\
-610.759943181818	1.16926521722773e-05	\\
-609.781161221592	2.44574118886887e-05	\\
-608.802379261364	3.94852626886653e-05	\\
-607.823597301136	2.84363494091401e-05	\\
-606.84481534091	2.82583238752652e-05	\\
-605.866033380682	3.68609596641049e-05	\\
-604.887251420454	2.91397885674968e-05	\\
-603.908469460228	2.91285068637163e-05	\\
-602.9296875	3.48181226476419e-05	\\
-601.950905539772	4.45038608615637e-05	\\
-600.972123579546	4.06343861140496e-05	\\
-599.993341619318	4.96006241116551e-05	\\
-599.014559659092	2.27176790993403e-05	\\
-598.035777698864	2.78263896326602e-05	\\
-597.056995738636	2.30778697742872e-05	\\
-596.07821377841	2.03464932859695e-05	\\
-595.099431818182	1.20796727703192e-05	\\
-594.120649857954	1.79902016990671e-05	\\
-593.141867897728	2.79310622313067e-05	\\
-592.1630859375	9.53045579038187e-06	\\
-591.184303977272	1.6108182942686e-05	\\
-590.205522017046	2.14148269650298e-05	\\
-589.226740056818	2.79690456982833e-05	\\
-588.247958096592	3.24021470963711e-05	\\
-587.269176136364	3.40012483892889e-05	\\
-586.290394176136	8.89546553708131e-06	\\
-585.31161221591	3.03821979093671e-05	\\
-584.332830255682	4.14149178464728e-05	\\
-583.354048295454	1.7818674951908e-05	\\
-582.375266335228	3.08575185490934e-05	\\
-581.396484375	3.52891344987248e-05	\\
-580.417702414772	1.74658563216748e-05	\\
-579.438920454546	3.94374598334336e-05	\\
-578.460138494318	3.42637241572767e-05	\\
-577.481356534092	2.7390410946029e-05	\\
-576.502574573864	4.09270111090116e-05	\\
-575.523792613636	3.29787221036711e-05	\\
-574.54501065341	2.05342243811788e-05	\\
-573.566228693182	3.55167139967328e-05	\\
-572.587446732954	2.40746921568808e-05	\\
-571.608664772728	2.48275618499474e-05	\\
-570.6298828125	3.38174746681426e-05	\\
-569.651100852272	2.37470565782437e-05	\\
-568.672318892046	2.47633453998708e-05	\\
-567.693536931818	3.07337584854144e-05	\\
-566.714754971592	1.12135866803532e-05	\\
-565.735973011364	2.5799577639608e-05	\\
-564.757191051136	2.82272126210094e-05	\\
-563.77840909091	1.16552662498608e-05	\\
-562.799627130682	1.90338826233471e-05	\\
-561.820845170454	1.90609718061809e-05	\\
-560.842063210228	1.23041515579495e-05	\\
-559.86328125	2.04631812384846e-05	\\
-558.884499289772	2.19707647568518e-05	\\
-557.905717329546	1.99508817820513e-05	\\
-556.926935369318	2.33876895691924e-05	\\
-555.948153409092	2.35053280226301e-05	\\
-554.969371448864	1.08490207962905e-05	\\
-553.990589488636	2.86197242707103e-05	\\
-553.01180752841	3.76541989468524e-05	\\
-552.033025568182	1.20648976649242e-05	\\
-551.054243607954	3.35245823152734e-05	\\
-550.075461647728	2.12470618551356e-05	\\
-549.0966796875	1.87439158601075e-05	\\
-548.117897727272	5.16900426272555e-05	\\
-547.139115767046	1.48710560972903e-05	\\
-546.160333806818	2.27469101711632e-05	\\
-545.181551846592	3.8059919702583e-05	\\
-544.202769886364	1.17800607532863e-05	\\
-543.223987926136	3.03104852868693e-05	\\
-542.24520596591	4.48040055982982e-05	\\
-541.266424005682	7.64642820214035e-06	\\
-540.287642045454	3.5152531720138e-05	\\
-539.308860085228	2.65781617811646e-05	\\
-538.330078125	2.61755915893194e-05	\\
-537.351296164772	4.1575322985102e-05	\\
-536.372514204546	1.31056336485542e-05	\\
-535.393732244318	3.81039939794128e-05	\\
-534.414950284092	4.00627926179068e-05	\\
-533.436168323864	1.92231552268929e-05	\\
-532.457386363636	6.62911663494356e-05	\\
-531.47860440341	4.14078379656109e-05	\\
-530.499822443182	6.48756283014716e-05	\\
-529.521040482954	3.51302554420836e-05	\\
-528.542258522728	4.23051813461252e-05	\\
-527.5634765625	2.30983522331134e-05	\\
-526.584694602272	6.5202951190305e-05	\\
-525.605912642046	3.040680522178e-05	\\
-524.627130681818	8.80800831075143e-05	\\
-523.648348721592	2.41052066906766e-05	\\
-522.669566761364	7.37533034143245e-05	\\
-521.690784801136	2.16944025587372e-05	\\
-520.71200284091	6.46418257680432e-05	\\
-519.733220880682	5.72483850499938e-05	\\
-518.754438920454	7.03272435116099e-05	\\
-517.775656960228	9.00052853302719e-05	\\
-516.796875	5.39636377284538e-05	\\
-515.818093039772	5.41605947081512e-05	\\
-514.839311079546	3.44309285864134e-05	\\
-513.860529119318	3.90231594594913e-05	\\
-512.881747159092	1.64554135004095e-05	\\
-511.902965198864	5.43247534669367e-05	\\
-510.924183238636	2.38044561385011e-05	\\
-509.94540127841	4.57413564351459e-05	\\
-508.966619318182	7.01010582692416e-06	\\
-507.987837357954	2.75755539121847e-05	\\
-507.009055397728	4.12354022067647e-05	\\
-506.0302734375	3.42111766087766e-05	\\
-505.051491477272	3.93183706881735e-05	\\
-504.072709517046	2.16840859456447e-05	\\
-503.093927556818	1.56365101647897e-05	\\
-502.115145596592	4.44718521878237e-05	\\
-501.136363636364	3.22222328187962e-05	\\
-500.157581676136	0.000130969263230924	\\
-499.17879971591	4.46079946950742e-05	\\
-498.200017755682	6.97205650387336e-05	\\
-497.221235795454	2.56150558501165e-05	\\
-496.242453835228	6.01599748122015e-05	\\
-495.263671875	4.56325345534788e-05	\\
-494.284889914772	3.06396989521162e-05	\\
-493.306107954546	3.3333888960352e-05	\\
-492.327325994318	4.85676108008966e-05	\\
-491.348544034092	1.62646453379164e-05	\\
-490.369762073864	6.94126130073485e-05	\\
-489.390980113636	4.59260244118082e-05	\\
-488.41219815341	7.96667849202736e-05	\\
-487.433416193182	5.26932252755515e-05	\\
-486.454634232954	7.13469889744685e-05	\\
-485.475852272728	5.60129454742192e-05	\\
-484.4970703125	8.93850005272537e-05	\\
-483.518288352272	7.56552188874683e-05	\\
-482.539506392046	6.94837015763498e-05	\\
-481.560724431818	6.00198735271217e-05	\\
-480.581942471592	4.50703911205368e-05	\\
-479.603160511364	3.96098602271075e-05	\\
-478.624378551136	3.56133425800825e-05	\\
-477.64559659091	4.09936913353376e-05	\\
-476.666814630682	8.7955304160795e-06	\\
-475.688032670454	2.56018469446477e-05	\\
-474.709250710228	1.35603178890851e-05	\\
-473.73046875	8.732735598268e-06	\\
-472.751686789772	2.04196799684033e-05	\\
-471.772904829546	2.94107853891617e-05	\\
-470.794122869318	2.92198521445361e-05	\\
-469.815340909092	2.338270342572e-05	\\
-468.836558948864	2.57373547704728e-05	\\
-467.857776988636	1.36892276831569e-05	\\
-466.87899502841	2.27962261040069e-05	\\
-465.900213068182	3.48432726564353e-05	\\
-464.921431107954	2.09931109724103e-05	\\
-463.942649147728	2.08527366115649e-05	\\
-462.9638671875	1.74224814373323e-05	\\
-461.985085227272	3.29804746195547e-05	\\
-461.006303267046	2.08692438538631e-05	\\
-460.027521306818	4.59755835269218e-05	\\
-459.048739346592	2.01859472765485e-05	\\
-458.069957386364	1.96408063500061e-05	\\
-457.091175426136	1.65816990291289e-05	\\
-456.11239346591	2.50123313400442e-05	\\
-455.133611505682	1.75008280699147e-05	\\
-454.154829545454	3.53342871245187e-05	\\
-453.176047585228	2.26400049812571e-05	\\
-452.197265625	2.70391070954276e-05	\\
-451.218483664772	2.36782847192999e-05	\\
-450.239701704546	1.84496525487653e-05	\\
-449.260919744318	2.08945904851187e-05	\\
-448.282137784092	2.98221535794225e-05	\\
-447.303355823864	3.8824878326675e-05	\\
-446.324573863636	1.09385902288311e-05	\\
-445.34579190341	2.69318550752992e-05	\\
-444.367009943182	2.06392317369812e-05	\\
-443.388227982954	1.05886275167522e-05	\\
-442.409446022728	2.62610876242944e-05	\\
-441.4306640625	2.5130653603561e-05	\\
-440.451882102272	1.61228734890267e-05	\\
-439.473100142046	2.90161870348627e-05	\\
-438.494318181818	2.09691373712557e-05	\\
-437.515536221592	2.04397286209161e-05	\\
-436.536754261364	1.90956746183606e-05	\\
-435.557972301136	1.77855781725579e-05	\\
-434.57919034091	1.91415289371544e-05	\\
-433.600408380682	2.11582126710971e-05	\\
-432.621626420454	2.27517197200282e-05	\\
-431.642844460228	1.62155224361541e-05	\\
-430.6640625	3.08946389808813e-05	\\
-429.685280539772	1.91542623478082e-05	\\
-428.706498579546	1.25471805430808e-05	\\
-427.727716619318	2.24206907137664e-05	\\
-426.748934659092	1.78001134681849e-05	\\
-425.770152698864	1.2653360627531e-05	\\
-424.791370738636	2.77553173216628e-05	\\
-423.81258877841	2.44000718863259e-05	\\
-422.833806818182	2.11748166604592e-06	\\
-421.855024857954	2.20977725464773e-05	\\
-420.876242897728	1.86481493749015e-06	\\
-419.8974609375	1.15946904274664e-05	\\
-418.918678977272	1.81598048219225e-05	\\
-417.939897017046	1.65001940575678e-05	\\
-416.961115056818	3.37988150654553e-05	\\
-415.982333096592	2.8929451188255e-05	\\
-415.003551136364	8.328225000479e-06	\\
-414.024769176136	4.0579183443878e-05	\\
-413.04598721591	2.16985931344009e-05	\\
-412.067205255682	2.35909594835477e-05	\\
-411.088423295454	1.58740313290351e-05	\\
-410.109641335228	2.90074929670606e-05	\\
-409.130859375	1.2025662122116e-05	\\
-408.152077414772	4.56179151882429e-06	\\
-407.173295454546	1.82971058331054e-05	\\
-406.194513494318	1.2411423605753e-05	\\
-405.215731534092	8.40545996008974e-06	\\
-404.236949573864	2.13651225882448e-05	\\
-403.258167613636	2.66287842692663e-05	\\
-402.27938565341	1.19085885138437e-05	\\
-401.300603693182	3.42072468919475e-05	\\
-400.321821732954	6.04602593955394e-05	\\
-399.343039772728	3.7653517136784e-05	\\
-398.3642578125	3.62436010558513e-05	\\
-397.385475852272	8.36918867146531e-06	\\
-396.406693892046	1.6184660407078e-05	\\
-395.427911931818	2.15463817101798e-05	\\
-394.449129971592	4.04273978503686e-06	\\
-393.470348011364	3.86449358447537e-05	\\
-392.491566051136	1.2441349258049e-05	\\
-391.51278409091	4.173158486299e-05	\\
-390.534002130682	6.79995720009616e-06	\\
-389.555220170454	3.28717309841561e-05	\\
-388.576438210228	8.70316114047077e-06	\\
-387.59765625	2.13664266588334e-05	\\
-386.618874289772	3.65364782346507e-06	\\
-385.640092329546	2.77652681196906e-05	\\
-384.661310369318	1.84097438476794e-05	\\
-383.682528409092	2.87727216720275e-05	\\
-382.703746448864	2.13232393829329e-05	\\
-381.724964488636	3.60682345400663e-05	\\
-380.74618252841	1.19157665286088e-05	\\
-379.767400568182	3.00618298483492e-05	\\
-378.788618607954	1.20528250189644e-05	\\
-377.809836647728	1.24325393311607e-05	\\
-376.8310546875	1.09325905483223e-05	\\
-375.852272727272	7.24054975518773e-06	\\
-374.873490767046	3.93120901681637e-06	\\
-373.894708806818	8.76030017084216e-06	\\
-372.915926846592	9.61296039360289e-06	\\
-371.937144886364	2.63746143821971e-05	\\
-370.958362926136	1.79050467285536e-05	\\
-369.97958096591	4.11142475938092e-05	\\
-369.000799005682	1.73792538178191e-05	\\
-368.022017045454	3.22606936923239e-05	\\
-367.043235085228	2.70383057605704e-05	\\
-366.064453125	3.1929279004848e-05	\\
-365.085671164772	1.25882721207841e-05	\\
-364.106889204546	2.16419315220194e-05	\\
-363.128107244318	1.49298046430255e-05	\\
-362.149325284092	1.26497149873934e-05	\\
-361.170543323864	1.44666837423144e-05	\\
-360.191761363636	1.39859729433855e-05	\\
-359.21297940341	7.6082592328465e-06	\\
-358.234197443182	7.19052320332949e-06	\\
-357.255415482954	2.02655363106647e-05	\\
-356.276633522728	4.23527466023761e-06	\\
-355.2978515625	1.88094442692302e-05	\\
-354.319069602272	8.4552766049052e-06	\\
-353.340287642046	1.54025115026994e-05	\\
-352.361505681818	4.64950957913899e-06	\\
-351.382723721592	7.43947709120639e-06	\\
-350.403941761364	1.17440655347494e-05	\\
-349.425159801136	9.19657062435103e-06	\\
-348.44637784091	1.40258818502151e-05	\\
-347.467595880682	1.6088993159044e-05	\\
-346.488813920454	5.51451577784912e-06	\\
-345.510031960228	2.08951481632729e-05	\\
-344.53125	3.99925403105809e-06	\\
-343.552468039772	1.39775288248528e-05	\\
-342.573686079546	1.37838367137351e-05	\\
-341.594904119318	7.59356385285709e-06	\\
-340.616122159092	8.14488995948204e-06	\\
-339.637340198864	1.24353982890749e-05	\\
-338.658558238636	3.19292806946732e-05	\\
-337.67977627841	9.18577216026783e-06	\\
-336.700994318182	1.77171936684852e-05	\\
-335.722212357954	1.10801764289532e-06	\\
-334.743430397728	1.33134052065015e-05	\\
-333.7646484375	6.8603868844628e-06	\\
-332.785866477272	1.41722033477618e-05	\\
-331.807084517046	2.1708457518514e-05	\\
-330.828302556818	1.85851148823771e-05	\\
-329.849520596592	1.05711351220216e-05	\\
-328.870738636364	3.45458544109195e-05	\\
-327.891956676136	2.96967239945343e-05	\\
-326.91317471591	9.77255762249018e-06	\\
-325.934392755682	3.76835687145884e-05	\\
-324.955610795454	7.92766292335289e-06	\\
-323.976828835228	3.476031613438e-05	\\
-322.998046875	3.51386279251785e-05	\\
-322.019264914772	1.99872882727814e-05	\\
-321.040482954546	2.0086834343715e-05	\\
-320.061700994318	1.37914960867479e-05	\\
-319.082919034092	8.62696193122421e-06	\\
-318.104137073864	2.49254632402696e-05	\\
-317.125355113636	2.83132089112279e-05	\\
-316.14657315341	3.85300971242984e-05	\\
-315.167791193182	2.26008579642086e-05	\\
-314.189009232954	2.68603971966999e-05	\\
-313.210227272728	4.20308347829773e-05	\\
-312.2314453125	2.19864593277865e-05	\\
-311.252663352272	2.81096918357582e-05	\\
-310.273881392046	1.60462166635347e-05	\\
-309.295099431818	2.55504093736838e-05	\\
-308.316317471592	2.61262692563052e-05	\\
-307.337535511364	8.13313423452475e-06	\\
-306.358753551136	3.7792138029606e-05	\\
-305.37997159091	2.14115713923657e-05	\\
-304.401189630682	1.91424767825495e-05	\\
-303.422407670454	1.83818898930935e-05	\\
-302.443625710228	2.21861612307738e-05	\\
-301.46484375	1.48270272201638e-05	\\
-300.486061789772	3.39107458589602e-05	\\
-299.507279829546	3.91972243982016e-05	\\
-298.528497869318	2.2116721441332e-05	\\
-297.549715909092	9.42719946945514e-06	\\
-296.570933948864	1.21935950030542e-05	\\
-295.592151988636	5.92551914578856e-06	\\
-294.61337002841	1.154978467155e-05	\\
-293.634588068182	1.27826806295396e-05	\\
-292.655806107954	2.50294193840866e-05	\\
-291.677024147728	4.19068183028717e-05	\\
-290.6982421875	1.56494494847877e-05	\\
-289.719460227272	2.73921074112423e-05	\\
-288.740678267046	5.13619982060891e-06	\\
-287.761896306818	2.04358625605352e-05	\\
-286.783114346592	1.34041393797076e-05	\\
-285.804332386364	1.55472935997907e-05	\\
-284.825550426136	2.29454252080895e-05	\\
-283.84676846591	1.81098393706871e-05	\\
-282.867986505682	3.51166197764103e-05	\\
-281.889204545454	2.48478782159631e-05	\\
-280.910422585228	4.16439566832465e-06	\\
-279.931640625	6.39610606551748e-06	\\
-278.952858664772	9.9082431888612e-06	\\
-277.974076704546	1.73484930201359e-05	\\
-276.995294744318	5.70307519005701e-06	\\
-276.016512784092	2.40627525496543e-05	\\
-275.037730823864	1.7690539421406e-05	\\
-274.058948863636	3.01614092843799e-05	\\
-273.08016690341	1.03183777142486e-05	\\
-272.101384943182	4.42639682156849e-05	\\
-271.122602982954	1.15433001952868e-05	\\
-270.143821022728	3.60291897586908e-05	\\
-269.1650390625	3.29796460946354e-05	\\
-268.186257102272	4.51572990935286e-05	\\
-267.207475142046	6.45299667426242e-05	\\
-266.228693181818	4.26906092366121e-05	\\
-265.249911221592	8.18481053173843e-05	\\
-264.271129261364	1.98626083919987e-05	\\
-263.292347301136	9.289516772263e-05	\\
-262.31356534091	1.29844308109402e-05	\\
-261.334783380682	6.76087548266302e-05	\\
-260.356001420454	3.17575858189231e-05	\\
-259.377219460228	3.18673062221763e-05	\\
-258.3984375	3.64122976559287e-05	\\
-257.419655539772	5.70201977934665e-05	\\
-256.440873579546	5.88190629005486e-05	\\
-255.462091619318	6.53834175527575e-05	\\
-254.483309659092	4.77455222342127e-05	\\
-253.504527698864	2.67494161053071e-05	\\
-252.525745738636	4.63615762360056e-05	\\
-251.54696377841	7.23783739575299e-06	\\
-250.568181818182	3.92379748566252e-05	\\
-249.589399857954	1.25329830758903e-05	\\
-248.610617897728	2.73491303287982e-05	\\
-247.6318359375	1.81310956775447e-05	\\
-246.653053977272	3.84630771214899e-05	\\
-245.674272017046	5.02176562983801e-05	\\
-244.695490056818	4.0113909257441e-05	\\
-243.716708096592	3.92189278675291e-05	\\
-242.737926136364	2.07660958487851e-05	\\
-241.759144176136	2.697109532332e-05	\\
-240.78036221591	1.89791715062025e-05	\\
-239.801580255682	4.85471477468203e-05	\\
-238.822798295454	2.46045301094908e-05	\\
-237.844016335228	2.46821631649741e-05	\\
-236.865234375	1.38602202393396e-05	\\
-235.886452414772	2.88138873207529e-05	\\
-234.907670454546	3.99364451982745e-05	\\
-233.928888494318	1.95461246503883e-05	\\
-232.950106534092	3.8249536521609e-05	\\
-231.971324573864	5.57774012268065e-05	\\
-230.992542613636	3.95205659003675e-05	\\
-230.01376065341	3.66071252854613e-05	\\
-229.034978693182	5.34135961449024e-05	\\
-228.056196732954	5.02107654144006e-05	\\
-227.077414772728	3.17022378754794e-05	\\
-226.0986328125	1.46692109278644e-05	\\
-225.119850852272	3.09015873846992e-05	\\
-224.141068892046	1.50705562359059e-05	\\
-223.162286931818	3.44589437436934e-06	\\
-222.183504971592	1.51246637184174e-05	\\
-221.204723011364	1.41157305814114e-05	\\
-220.225941051136	8.5288797157819e-06	\\
-219.24715909091	1.5552334445662e-05	\\
-218.268377130682	1.41981475935065e-05	\\
-217.289595170454	1.55673345298219e-05	\\
-216.310813210228	1.75608682031696e-05	\\
-215.33203125	9.39442334160392e-06	\\
-214.353249289772	1.56733715678703e-05	\\
-213.374467329546	5.66192252815796e-06	\\
-212.395685369318	7.75417330559615e-06	\\
-211.416903409092	6.76375046003672e-06	\\
-210.438121448864	1.18997089248747e-05	\\
-209.459339488636	6.55787102796824e-06	\\
-208.48055752841	5.19264186566088e-06	\\
-207.501775568182	1.29091359528132e-05	\\
-206.522993607954	1.10807182753532e-05	\\
-205.544211647728	1.00168712482835e-05	\\
-204.5654296875	5.98802989297657e-06	\\
-203.586647727272	3.47801295321321e-06	\\
-202.607865767046	9.7069142424629e-06	\\
-201.629083806818	1.05581252582755e-05	\\
-200.650301846592	3.16223774527958e-05	\\
-199.671519886364	7.28316011147416e-05	\\
-198.692737926136	8.14640191954207e-06	\\
-197.71395596591	2.1807078980877e-05	\\
-196.735174005682	1.11352349855654e-05	\\
-195.756392045454	8.20645454448385e-06	\\
-194.777610085228	4.73877712326437e-06	\\
-193.798828125	9.72325806195491e-06	\\
-192.820046164772	1.29796883909067e-05	\\
-191.841264204546	1.3796843190178e-05	\\
-190.862482244318	8.97860754538367e-06	\\
-189.883700284092	8.61265566176781e-06	\\
-188.904918323864	7.34498662193486e-06	\\
-187.926136363636	1.20047064622242e-05	\\
-186.94735440341	2.07188329107203e-05	\\
-185.968572443182	1.09033240461692e-05	\\
-184.989790482954	1.13426166356622e-05	\\
-184.011008522728	1.02662049249819e-05	\\
-183.0322265625	4.88132107897527e-06	\\
-182.053444602272	1.35550561324212e-05	\\
-181.074662642046	1.41695774290054e-05	\\
-180.095880681818	1.31613741138617e-05	\\
-179.117098721592	1.81982419260857e-05	\\
-178.138316761364	2.82534893735357e-05	\\
-177.159534801136	7.06506819712772e-06	\\
-176.18075284091	1.10468926354479e-05	\\
-175.201970880682	3.88702194929647e-06	\\
-174.223188920454	2.17239018274287e-05	\\
-173.244406960228	9.28351336847598e-06	\\
-172.265625	2.82175638963322e-05	\\
-171.286843039772	1.86050349538003e-05	\\
-170.308061079546	3.42212533415409e-05	\\
-169.329279119318	2.45585862514606e-05	\\
-168.350497159092	7.98844349337095e-06	\\
-167.371715198864	6.81112685399622e-06	\\
-166.392933238636	1.66262350440288e-05	\\
-165.41415127841	2.46346362860781e-05	\\
-164.435369318182	1.23783001898061e-05	\\
-163.456587357954	2.78271362158894e-05	\\
-162.477805397728	1.93525015348355e-05	\\
-161.4990234375	3.66559293165992e-05	\\
-160.520241477272	4.4024348982379e-05	\\
-159.541459517046	6.47153149925507e-05	\\
-158.562677556818	7.05051602184135e-05	\\
-157.583895596592	6.43147062047171e-05	\\
-156.605113636364	6.39290161512581e-05	\\
-155.626331676136	4.13102898643477e-05	\\
-154.64754971591	2.84434231868932e-05	\\
-153.668767755682	5.31582418916027e-05	\\
-152.689985795454	2.97035746648914e-05	\\
-151.711203835228	3.53198007105797e-05	\\
-150.732421875	5.10688745802231e-05	\\
-149.753639914772	3.09211641080535e-05	\\
-148.774857954546	2.78441252869315e-05	\\
-147.796075994318	5.48511208522788e-05	\\
-146.817294034092	4.08349610132392e-05	\\
-145.838512073864	2.72178767003353e-05	\\
-144.859730113636	4.33478706555545e-05	\\
-143.88094815341	2.16324888903508e-05	\\
-142.902166193182	1.87789484142735e-05	\\
-141.923384232954	8.79768469064442e-06	\\
-140.944602272728	4.18872142003834e-06	\\
-139.9658203125	1.76976497351131e-05	\\
-138.987038352272	2.17168584275959e-05	\\
-138.008256392046	5.45065265212867e-06	\\
-137.029474431818	2.85827541288956e-05	\\
-136.050692471592	3.93227314808885e-05	\\
-135.071910511364	2.6919639462056e-05	\\
-134.093128551136	3.20898092972733e-05	\\
-133.11434659091	8.13396690475214e-06	\\
-132.135564630682	4.59302017897939e-05	\\
-131.156782670454	8.94100765664174e-06	\\
-130.178000710228	2.44736558942742e-05	\\
-129.19921875	3.05633660461322e-05	\\
-128.220436789772	3.41723532897975e-05	\\
-127.241654829546	3.0916355350889e-05	\\
-126.262872869318	3.13024908166173e-05	\\
-125.284090909092	2.65562724383258e-05	\\
-124.305308948864	4.74109785674371e-05	\\
-123.326526988636	2.44241738492531e-05	\\
-122.34774502841	3.44385999691e-05	\\
-121.368963068182	6.68386556499142e-05	\\
-120.390181107954	0.000183402220674602	\\
-119.411399147728	0.000115428226167538	\\
-118.4326171875	2.52564969479373e-05	\\
-117.453835227272	5.64796748736523e-05	\\
-116.475053267046	7.08698101833869e-05	\\
-115.496271306818	3.86325579945224e-05	\\
-114.517489346592	3.19640444172076e-05	\\
-113.538707386364	3.07540494838393e-05	\\
-112.559925426136	2.37395708065664e-05	\\
-111.58114346591	3.52234186940656e-05	\\
-110.602361505682	2.26422306859738e-05	\\
-109.623579545454	1.45217207796687e-05	\\
-108.644797585228	3.94335878769456e-06	\\
-107.666015625	9.94287833886007e-06	\\
-106.687233664772	2.61082412649895e-05	\\
-105.708451704546	1.25723911923993e-05	\\
-104.729669744318	2.97095724801576e-06	\\
-103.750887784092	9.4247247940924e-06	\\
-102.772105823864	1.26194973236601e-05	\\
-101.793323863636	1.97628492205355e-05	\\
-100.81454190341	1.19192461345947e-05	\\
-99.835759943182	4.08369201791848e-05	\\
-98.856977982954	5.44582633838578e-06	\\
-97.8781960227279	1.09345541049335e-05	\\
-96.8994140625	2.41558896469016e-06	\\
-95.9206321022721	8.61067280912433e-06	\\
-94.941850142046	7.69359449917742e-06	\\
-93.963068181818	1.14732535767905e-05	\\
-92.9842862215919	2.13445486263602e-05	\\
-92.005504261364	3.44100206942339e-05	\\
-91.026722301136	2.39047189344426e-05	\\
-90.0479403409099	2.44035356171091e-05	\\
-89.069158380682	1.19097905614108e-05	\\
-88.090376420454	7.18880747404295e-06	\\
-87.1115944602279	3.00901464099457e-05	\\
-86.1328125	2.50638176603149e-05	\\
-85.1540305397721	3.17285307702192e-05	\\
-84.175248579546	1.01193755398153e-05	\\
-83.196466619318	1.31278201780318e-05	\\
-82.2176846590919	1.60244863497492e-05	\\
-81.238902698864	2.74348105797664e-05	\\
-80.260120738636	3.12899853004517e-05	\\
-79.2813387784099	1.46200596247737e-05	\\
-78.302556818182	1.35078478324237e-05	\\
-77.323774857954	3.01919521062761e-05	\\
-76.3449928977279	9.09509292368172e-06	\\
-75.3662109375	2.17745490755338e-05	\\
-74.3874289772721	3.42071724999347e-05	\\
-73.408647017046	1.36781517234922e-06	\\
-72.429865056818	1.24851573845692e-05	\\
-71.4510830965919	2.14918697016701e-05	\\
-70.472301136364	8.95550655521802e-06	\\
-69.493519176136	1.86569489604144e-05	\\
-68.5147372159099	1.58175081609895e-05	\\
-67.535955255682	2.20126147045435e-05	\\
-66.557173295454	1.91871231565798e-05	\\
-65.5783913352279	1.14882758026603e-05	\\
-64.599609375	1.17561809311698e-05	\\
-63.6208274147721	5.74385712163431e-06	\\
-62.642045454546	1.73110763584889e-05	\\
-61.663263494318	3.01241280724148e-05	\\
-60.6844815340919	2.05959113159336e-05	\\
-59.705699573864	4.01438773807614e-06	\\
-58.726917613636	5.18593135969852e-06	\\
-57.7481356534099	1.92844220647153e-05	\\
-56.769353693182	1.12533014886035e-05	\\
-55.790571732954	1.72342193429085e-05	\\
-54.8117897727279	5.25383519370752e-06	\\
-53.8330078125	2.09696468996496e-05	\\
-52.8542258522721	1.57204590994937e-05	\\
-51.875443892046	4.28636883224161e-05	\\
-50.896661931818	1.9982015933332e-05	\\
-49.9178799715919	0.000278336777207723	\\
-48.939098011364	4.04219831583503e-05	\\
-47.960316051136	1.72767118435135e-05	\\
-46.9815340909099	1.62778717983559e-05	\\
-46.002752130682	3.99385237433288e-05	\\
-45.023970170454	0.000272727187095252	\\
-44.0451882102279	7.41888202020597e-05	\\
-43.06640625	4.9824108800565e-05	\\
-42.0876242897721	2.96925451503548e-05	\\
-41.108842329546	1.48281744501973e-05	\\
-40.130060369318	5.31988888174729e-05	\\
-39.1512784090919	0.000106676983155763	\\
-38.172496448864	7.56858701522522e-05	\\
-37.193714488636	0.000108910897777014	\\
-36.2149325284099	7.06840541156879e-05	\\
-35.236150568182	2.61455479797426e-05	\\
-34.257368607954	2.67904731802128e-05	\\
-33.2785866477279	3.10560620535806e-05	\\
-32.2998046875	3.99314737017302e-05	\\
-31.3210227272721	2.5898485127819e-05	\\
-30.342240767046	3.62361908829788e-05	\\
-29.363458806818	3.98731221333011e-05	\\
-28.3846768465919	3.092308271974e-05	\\
-27.405894886364	3.5146304831197e-05	\\
-26.427112926136	9.18671139681714e-05	\\
-25.4483309659099	1.92907382509066e-05	\\
-24.469549005682	7.50818917366806e-05	\\
-23.490767045454	0.00010308813573723	\\
-22.5119850852279	3.95648231655836e-05	\\
-21.533203125	4.64880288853449e-05	\\
-20.5544211647721	3.14738939698776e-05	\\
-19.575639204546	1.97519363881597e-05	\\
-18.596857244318	0.00012201591503076	\\
-17.6180752840919	6.93878039338928e-05	\\
-16.639293323864	0.000116577548827564	\\
-15.660511363636	0.000124950521497279	\\
-14.6817294034099	0.000148413952547398	\\
-13.702947443182	0.000162667925080096	\\
-12.724165482954	0.000156653203004795	\\
-11.7453835227279	9.84786172065733e-05	\\
-10.7666015625	6.65595279102e-05	\\
-9.78781960227207	5.45368539838865e-05	\\
-8.80903764204595	0.000214774783260589	\\
-7.83025568181802	0.000401247105823017	\\
-6.8514737215919	4.37022066515436e-05	\\
-5.87269176136397	0.000194878879447117	\\
-4.89390980113603	0.000210523233704347	\\
-3.91512784090992	0.000189954109445323	\\
-2.93634588068198	0.000278405866334292	\\
-1.95756392045405	8.62641963054009e-05	\\
-0.978781960227934	9.29256465987957e-05	\\
0	9.48662107641047e-06	\\
0.978781960226115	9.29256465987957e-05	\\
1.95756392045405	8.62641963054009e-05	\\
2.93634588068198	0.000278405866334292	\\
3.9151278409081	0.000189954109445323	\\
4.89390980113603	0.000210523233704347	\\
5.87269176136397	0.000194878879447117	\\
6.85147372159008	4.37022066515436e-05	\\
7.83025568181802	0.000401247105823017	\\
8.80903764204413	0.000214774783260589	\\
9.78781960227207	5.45368539838865e-05	\\
10.7666015625	6.65595279102e-05	\\
11.7453835227261	9.84786172065733e-05	\\
12.724165482954	0.000156653203004795	\\
13.702947443182	0.000162667925080096	\\
14.6817294034081	0.000148413952547398	\\
15.660511363636	0.000124950521497279	\\
16.639293323864	0.000116577548827564	\\
17.6180752840901	6.93878039338928e-05	\\
18.596857244318	0.00012201591503076	\\
19.5756392045441	1.97519363881597e-05	\\
20.5544211647721	3.14738939698776e-05	\\
21.533203125	4.64880288853449e-05	\\
22.5119850852261	3.95648231655836e-05	\\
23.490767045454	0.00010308813573723	\\
24.469549005682	7.50818917366806e-05	\\
25.4483309659081	1.92907382509066e-05	\\
26.427112926136	9.18671139681714e-05	\\
27.405894886364	3.5146304831197e-05	\\
28.3846768465901	3.092308271974e-05	\\
29.363458806818	3.98731221333011e-05	\\
30.3422407670441	3.62361908829788e-05	\\
31.3210227272721	2.5898485127819e-05	\\
32.2998046875	3.99314737017302e-05	\\
33.2785866477261	3.10560620535806e-05	\\
34.257368607954	2.67904731802128e-05	\\
35.236150568182	2.61455479797426e-05	\\
36.2149325284081	7.06840541156879e-05	\\
37.193714488636	0.000108910897777014	\\
38.172496448864	7.56858701522522e-05	\\
39.1512784090901	0.000106676983155763	\\
40.130060369318	5.31988888174729e-05	\\
41.1088423295441	1.48281744501973e-05	\\
42.0876242897721	2.96925451503548e-05	\\
43.06640625	4.9824108800565e-05	\\
44.0451882102261	7.41888202020597e-05	\\
45.023970170454	0.000272727187095252	\\
46.002752130682	3.99385237433288e-05	\\
46.9815340909081	1.62778717983559e-05	\\
47.960316051136	1.72767118435135e-05	\\
48.939098011364	4.04219831583503e-05	\\
49.9178799715901	0.000278336777207723	\\
50.896661931818	1.9982015933332e-05	\\
51.8754438920441	4.28636883224161e-05	\\
52.8542258522721	1.57204590994937e-05	\\
53.8330078125	2.09696468996496e-05	\\
54.8117897727261	5.25383519370752e-06	\\
55.790571732954	1.72342193429085e-05	\\
56.769353693182	1.12533014886035e-05	\\
57.7481356534081	1.92844220647153e-05	\\
58.726917613636	5.18593135969852e-06	\\
59.705699573864	4.01438773807614e-06	\\
60.6844815340901	2.05959113159336e-05	\\
61.663263494318	3.01241280724148e-05	\\
62.6420454545441	1.73110763584889e-05	\\
63.6208274147721	5.74385712163431e-06	\\
64.599609375	1.17561809311698e-05	\\
65.5783913352261	1.14882758026603e-05	\\
66.557173295454	1.91871231565798e-05	\\
67.535955255682	2.20126147045435e-05	\\
68.5147372159081	1.58175081609895e-05	\\
69.493519176136	1.86569489604144e-05	\\
70.472301136364	8.95550655521802e-06	\\
71.4510830965901	2.14918697016701e-05	\\
72.429865056818	1.24851573845692e-05	\\
73.4086470170441	1.36781517234922e-06	\\
74.3874289772721	3.42071724999347e-05	\\
75.3662109375	2.17745490755338e-05	\\
76.3449928977261	9.09509292368172e-06	\\
77.323774857954	3.01919521062761e-05	\\
78.302556818182	1.35078478324237e-05	\\
79.2813387784081	1.46200596247737e-05	\\
80.260120738636	3.12899853004517e-05	\\
81.238902698864	2.74348105797664e-05	\\
82.2176846590901	1.60244863497492e-05	\\
83.196466619318	1.31278201780318e-05	\\
84.1752485795441	1.01193755398153e-05	\\
85.1540305397721	3.17285307702192e-05	\\
86.1328125	2.50638176603149e-05	\\
87.1115944602261	3.00901464099457e-05	\\
88.090376420454	7.18880747404295e-06	\\
89.069158380682	1.19097905614108e-05	\\
90.0479403409081	2.44035356171091e-05	\\
91.026722301136	2.39047189344426e-05	\\
92.005504261364	3.44100206942339e-05	\\
92.9842862215901	2.13445486263602e-05	\\
93.963068181818	1.14732535767905e-05	\\
94.9418501420441	7.69359449917742e-06	\\
95.9206321022721	8.61067280912433e-06	\\
96.8994140625	2.41558896469016e-06	\\
97.8781960227261	1.09345541049335e-05	\\
98.856977982954	5.44582633838578e-06	\\
99.835759943182	4.08369201791848e-05	\\
100.814541903408	1.19192461345947e-05	\\
101.793323863636	1.97628492205355e-05	\\
102.772105823864	1.26194973236601e-05	\\
103.75088778409	9.4247247940924e-06	\\
104.729669744318	2.97095724801576e-06	\\
105.708451704544	1.25723911923993e-05	\\
106.687233664772	2.61082412649895e-05	\\
107.666015625	9.94287833886007e-06	\\
108.644797585226	3.94335878769456e-06	\\
109.623579545454	1.45217207796687e-05	\\
110.602361505682	2.26422306859738e-05	\\
111.581143465908	3.52234186940656e-05	\\
112.559925426136	2.37395708065664e-05	\\
113.538707386364	3.07540494838393e-05	\\
114.51748934659	3.19640444172076e-05	\\
115.496271306818	3.86325579945224e-05	\\
116.475053267044	7.08698101833869e-05	\\
117.453835227272	5.64796748736523e-05	\\
118.4326171875	2.52564969479373e-05	\\
119.411399147726	0.000115428226167538	\\
120.390181107954	0.000183402220674602	\\
121.368963068182	6.68386556499142e-05	\\
122.347745028408	3.44385999691e-05	\\
123.326526988636	2.44241738492531e-05	\\
124.305308948864	4.74109785674371e-05	\\
125.28409090909	2.65562724383258e-05	\\
126.262872869318	3.13024908166173e-05	\\
127.241654829544	3.0916355350889e-05	\\
128.220436789772	3.41723532897975e-05	\\
129.19921875	3.05633660461322e-05	\\
130.178000710226	2.44736558942742e-05	\\
131.156782670454	8.94100765664174e-06	\\
132.135564630682	4.59302017897939e-05	\\
133.114346590908	8.13396690475214e-06	\\
134.093128551136	3.20898092972733e-05	\\
135.071910511364	2.6919639462056e-05	\\
136.05069247159	3.93227314808885e-05	\\
137.029474431818	2.85827541288956e-05	\\
138.008256392044	5.45065265212867e-06	\\
138.987038352272	2.17168584275959e-05	\\
139.9658203125	1.76976497351131e-05	\\
140.944602272726	4.18872142003834e-06	\\
141.923384232954	8.79768469064442e-06	\\
142.902166193182	1.87789484142735e-05	\\
143.880948153408	2.16324888903508e-05	\\
144.859730113636	4.33478706555545e-05	\\
145.838512073864	2.72178767003353e-05	\\
146.81729403409	4.08349610132392e-05	\\
147.796075994318	5.48511208522788e-05	\\
148.774857954544	2.78441252869315e-05	\\
149.753639914772	3.09211641080535e-05	\\
150.732421875	5.10688745802231e-05	\\
151.711203835226	3.53198007105797e-05	\\
152.689985795454	2.97035746648914e-05	\\
153.668767755682	5.31582418916027e-05	\\
154.647549715908	2.84434231868932e-05	\\
155.626331676136	4.13102898643477e-05	\\
156.605113636364	6.39290161512581e-05	\\
157.58389559659	6.43147062047171e-05	\\
158.562677556818	7.05051602184135e-05	\\
159.541459517044	6.47153149925507e-05	\\
160.520241477272	4.4024348982379e-05	\\
161.4990234375	3.66559293165992e-05	\\
162.477805397726	1.93525015348355e-05	\\
163.456587357954	2.78271362158894e-05	\\
164.435369318182	1.23783001898061e-05	\\
165.414151278408	2.46346362860781e-05	\\
166.392933238636	1.66262350440288e-05	\\
167.371715198864	6.81112685399622e-06	\\
168.35049715909	7.98844349337095e-06	\\
169.329279119318	2.45585862514606e-05	\\
170.308061079544	3.42212533415409e-05	\\
171.286843039772	1.86050349538003e-05	\\
172.265625	2.82175638963322e-05	\\
173.244406960226	9.28351336847598e-06	\\
174.223188920454	2.17239018274287e-05	\\
175.201970880682	3.88702194929647e-06	\\
176.180752840908	1.10468926354479e-05	\\
177.159534801136	7.06506819712772e-06	\\
178.138316761364	2.82534893735357e-05	\\
179.11709872159	1.81982419260857e-05	\\
180.095880681818	1.31613741138617e-05	\\
181.074662642044	1.41695774290054e-05	\\
182.053444602272	1.35550561324212e-05	\\
183.0322265625	4.88132107897527e-06	\\
184.011008522726	1.02662049249819e-05	\\
184.989790482954	1.13426166356622e-05	\\
185.968572443182	1.09033240461692e-05	\\
186.947354403408	2.07188329107203e-05	\\
187.926136363636	1.20047064622242e-05	\\
188.904918323864	7.34498662193486e-06	\\
189.88370028409	8.61265566176781e-06	\\
190.862482244318	8.97860754538367e-06	\\
191.841264204544	1.3796843190178e-05	\\
192.820046164772	1.29796883909067e-05	\\
193.798828125	9.72325806195491e-06	\\
194.777610085226	4.73877712326437e-06	\\
195.756392045454	8.20645454448385e-06	\\
196.735174005682	1.11352349855654e-05	\\
197.713955965908	2.1807078980877e-05	\\
198.692737926136	8.14640191954207e-06	\\
199.671519886364	7.28316011147416e-05	\\
200.65030184659	3.16223774527958e-05	\\
201.629083806818	1.05581252582755e-05	\\
202.607865767044	9.7069142424629e-06	\\
203.586647727272	3.47801295321321e-06	\\
204.5654296875	5.98802989297657e-06	\\
205.544211647726	1.00168712482835e-05	\\
206.522993607954	1.10807182753532e-05	\\
207.501775568182	1.29091359528132e-05	\\
208.480557528408	5.19264186566088e-06	\\
209.459339488636	6.55787102796824e-06	\\
210.438121448864	1.18997089248747e-05	\\
211.41690340909	6.76375046003672e-06	\\
212.395685369318	7.75417330559615e-06	\\
213.374467329544	5.66192252815796e-06	\\
214.353249289772	1.56733715678703e-05	\\
215.33203125	9.39442334160392e-06	\\
216.310813210226	1.75608682031696e-05	\\
217.289595170454	1.55673345298219e-05	\\
218.268377130682	1.41981475935065e-05	\\
219.247159090908	1.5552334445662e-05	\\
220.225941051136	8.5288797157819e-06	\\
221.204723011364	1.41157305814114e-05	\\
222.18350497159	1.51246637184174e-05	\\
223.162286931818	3.44589437436934e-06	\\
224.141068892044	1.50705562359059e-05	\\
225.119850852272	3.09015873846992e-05	\\
226.0986328125	1.46692109278644e-05	\\
227.077414772726	3.17022378754794e-05	\\
228.056196732954	5.02107654144006e-05	\\
229.034978693182	5.34135961449024e-05	\\
230.013760653408	3.66071252854613e-05	\\
230.992542613636	3.95205659003675e-05	\\
231.971324573864	5.57774012268065e-05	\\
232.95010653409	3.8249536521609e-05	\\
233.928888494318	1.95461246503883e-05	\\
234.907670454544	3.99364451982745e-05	\\
235.886452414772	2.88138873207529e-05	\\
236.865234375	1.38602202393396e-05	\\
237.844016335226	2.46821631649741e-05	\\
238.822798295454	2.46045301094908e-05	\\
239.801580255682	4.85471477468203e-05	\\
240.780362215908	1.89791715062025e-05	\\
241.759144176136	2.697109532332e-05	\\
242.737926136364	2.07660958487851e-05	\\
243.71670809659	3.92189278675291e-05	\\
244.695490056818	4.0113909257441e-05	\\
245.674272017044	5.02176562983801e-05	\\
246.653053977272	3.84630771214899e-05	\\
247.6318359375	1.81310956775447e-05	\\
248.610617897726	2.73491303287982e-05	\\
249.589399857954	1.25329830758903e-05	\\
250.568181818182	3.92379748566252e-05	\\
251.546963778408	7.23783739575299e-06	\\
252.525745738636	4.63615762360056e-05	\\
253.504527698864	2.67494161053071e-05	\\
254.48330965909	4.77455222342127e-05	\\
255.462091619318	6.53834175527575e-05	\\
256.440873579544	5.88190629005486e-05	\\
257.419655539772	5.70201977934665e-05	\\
258.3984375	3.64122976559287e-05	\\
259.377219460226	3.18673062221763e-05	\\
260.356001420454	3.17575858189231e-05	\\
261.334783380682	6.76087548266302e-05	\\
262.313565340908	1.29844308109402e-05	\\
263.292347301136	9.289516772263e-05	\\
264.271129261364	1.98626083919987e-05	\\
265.24991122159	8.18481053173843e-05	\\
266.228693181818	4.26906092366121e-05	\\
267.207475142044	6.45299667426242e-05	\\
268.186257102272	4.51572990935286e-05	\\
269.1650390625	3.29796460946354e-05	\\
270.143821022726	3.60291897586908e-05	\\
271.122602982954	1.15433001952868e-05	\\
272.101384943182	4.42639682156849e-05	\\
273.080166903408	1.03183777142486e-05	\\
274.058948863636	3.01614092843799e-05	\\
275.037730823864	1.7690539421406e-05	\\
276.01651278409	2.40627525496543e-05	\\
276.995294744318	5.70307519005701e-06	\\
277.974076704544	1.73484930201359e-05	\\
278.952858664772	9.9082431888612e-06	\\
279.931640625	6.39610606551748e-06	\\
280.910422585226	4.16439566832465e-06	\\
281.889204545454	2.48478782159631e-05	\\
282.867986505682	3.51166197764103e-05	\\
283.846768465908	1.81098393706871e-05	\\
284.825550426136	2.29454252080895e-05	\\
285.804332386364	1.55472935997907e-05	\\
286.78311434659	1.34041393797076e-05	\\
287.761896306818	2.04358625605352e-05	\\
288.740678267044	5.13619982060891e-06	\\
289.719460227272	2.73921074112423e-05	\\
290.6982421875	1.56494494847877e-05	\\
291.677024147726	4.19068183028717e-05	\\
292.655806107954	2.50294193840866e-05	\\
293.634588068182	1.27826806295396e-05	\\
294.613370028408	1.154978467155e-05	\\
295.592151988636	5.92551914578856e-06	\\
296.570933948864	1.21935950030542e-05	\\
297.54971590909	9.42719946945514e-06	\\
298.528497869318	2.2116721441332e-05	\\
299.507279829544	3.91972243982016e-05	\\
300.486061789772	3.39107458589602e-05	\\
301.46484375	1.48270272201638e-05	\\
302.443625710226	2.21861612307738e-05	\\
303.422407670454	1.83818898930935e-05	\\
304.401189630682	1.91424767825495e-05	\\
305.379971590908	2.14115713923657e-05	\\
306.358753551136	3.7792138029606e-05	\\
307.337535511364	8.13313423452475e-06	\\
308.31631747159	2.61262692563052e-05	\\
309.295099431818	2.55504093736838e-05	\\
310.273881392044	1.60462166635347e-05	\\
311.252663352272	2.81096918357582e-05	\\
312.2314453125	2.19864593277865e-05	\\
313.210227272726	4.20308347829773e-05	\\
314.189009232954	2.68603971966999e-05	\\
315.167791193182	2.26008579642086e-05	\\
316.146573153408	3.85300971242984e-05	\\
317.125355113636	2.83132089112279e-05	\\
318.104137073864	2.49254632402696e-05	\\
319.08291903409	8.62696193122421e-06	\\
320.061700994318	1.37914960867479e-05	\\
321.040482954544	2.0086834343715e-05	\\
322.019264914772	1.99872882727814e-05	\\
322.998046875	3.51386279251785e-05	\\
323.976828835226	3.476031613438e-05	\\
324.955610795454	7.92766292335289e-06	\\
325.934392755682	3.76835687145884e-05	\\
326.913174715908	9.77255762249018e-06	\\
327.891956676136	2.96967239945343e-05	\\
328.870738636364	3.45458544109195e-05	\\
329.84952059659	1.05711351220216e-05	\\
330.828302556818	1.85851148823771e-05	\\
331.807084517044	2.1708457518514e-05	\\
332.785866477272	1.41722033477618e-05	\\
333.7646484375	6.8603868844628e-06	\\
334.743430397726	1.33134052065015e-05	\\
335.722212357954	1.10801764289532e-06	\\
336.700994318182	1.77171936684852e-05	\\
337.679776278408	9.18577216026783e-06	\\
338.658558238636	3.19292806946732e-05	\\
339.637340198864	1.24353982890749e-05	\\
340.61612215909	8.14488995948204e-06	\\
341.594904119318	7.59356385285709e-06	\\
342.573686079544	1.37838367137351e-05	\\
343.552468039772	1.39775288248528e-05	\\
344.53125	3.99925403105809e-06	\\
345.510031960226	2.08951481632729e-05	\\
346.488813920454	5.51451577784912e-06	\\
347.467595880682	1.6088993159044e-05	\\
348.446377840908	1.40258818502151e-05	\\
349.425159801136	9.19657062435103e-06	\\
350.403941761364	1.17440655347494e-05	\\
351.38272372159	7.43947709120639e-06	\\
352.361505681818	4.64950957913899e-06	\\
353.340287642044	1.54025115026994e-05	\\
354.319069602272	8.4552766049052e-06	\\
355.2978515625	1.88094442692302e-05	\\
356.276633522726	4.23527466023761e-06	\\
357.255415482954	2.02655363106647e-05	\\
358.234197443182	7.19052320332949e-06	\\
359.212979403408	7.6082592328465e-06	\\
360.191761363636	1.39859729433855e-05	\\
361.170543323864	1.44666837423144e-05	\\
362.14932528409	1.26497149873934e-05	\\
363.128107244318	1.49298046430255e-05	\\
364.106889204544	2.16419315220194e-05	\\
365.085671164772	1.25882721207841e-05	\\
366.064453125	3.1929279004848e-05	\\
367.043235085226	2.70383057605704e-05	\\
368.022017045454	3.22606936923239e-05	\\
369.000799005682	1.73792538178191e-05	\\
369.979580965908	4.11142475938092e-05	\\
370.958362926136	1.79050467285536e-05	\\
371.937144886364	2.63746143821971e-05	\\
372.91592684659	9.61296039360289e-06	\\
373.894708806818	8.76030017084216e-06	\\
374.873490767044	3.93120901681637e-06	\\
375.852272727272	7.24054975518773e-06	\\
376.8310546875	1.09325905483223e-05	\\
377.809836647726	1.24325393311607e-05	\\
378.788618607954	1.20528250189644e-05	\\
379.767400568182	3.00618298483492e-05	\\
380.746182528408	1.19157665286088e-05	\\
381.724964488636	3.60682345400663e-05	\\
382.703746448864	2.13232393829329e-05	\\
383.68252840909	2.87727216720275e-05	\\
384.661310369318	1.84097438476794e-05	\\
385.640092329544	2.77652681196906e-05	\\
386.618874289772	3.65364782346507e-06	\\
387.59765625	2.13664266588334e-05	\\
388.576438210226	8.70316114047077e-06	\\
389.555220170454	3.28717309841561e-05	\\
390.534002130682	6.79995720009616e-06	\\
391.512784090908	4.173158486299e-05	\\
392.491566051136	1.2441349258049e-05	\\
393.470348011364	3.86449358447537e-05	\\
394.44912997159	4.04273978503686e-06	\\
395.427911931818	2.15463817101798e-05	\\
396.406693892044	1.6184660407078e-05	\\
397.385475852272	8.36918867146531e-06	\\
398.3642578125	3.62436010558513e-05	\\
399.343039772726	3.7653517136784e-05	\\
400.321821732954	6.04602593955394e-05	\\
401.300603693182	3.42072468919475e-05	\\
402.279385653408	1.19085885138437e-05	\\
403.258167613636	2.66287842692663e-05	\\
404.236949573864	2.13651225882448e-05	\\
405.21573153409	8.40545996008974e-06	\\
406.194513494318	1.2411423605753e-05	\\
407.173295454544	1.82971058331054e-05	\\
408.152077414772	4.56179151882429e-06	\\
409.130859375	1.2025662122116e-05	\\
410.109641335226	2.90074929670606e-05	\\
411.088423295454	1.58740313290351e-05	\\
412.067205255682	2.35909594835477e-05	\\
413.045987215908	2.16985931344009e-05	\\
414.024769176136	4.0579183443878e-05	\\
415.003551136364	8.328225000479e-06	\\
415.98233309659	2.8929451188255e-05	\\
416.961115056818	3.37988150654553e-05	\\
417.939897017044	1.65001940575678e-05	\\
418.918678977272	1.81598048219225e-05	\\
419.8974609375	1.15946904274664e-05	\\
420.876242897726	1.86481493749015e-06	\\
421.855024857954	2.20977725464773e-05	\\
422.833806818182	2.11748166604592e-06	\\
423.812588778408	2.44000718863259e-05	\\
424.791370738636	2.77553173216628e-05	\\
425.770152698864	1.2653360627531e-05	\\
426.74893465909	1.78001134681849e-05	\\
427.727716619318	2.24206907137664e-05	\\
428.706498579544	1.25471805430808e-05	\\
429.685280539772	1.91542623478082e-05	\\
430.6640625	3.08946389808813e-05	\\
431.642844460226	1.62155224361541e-05	\\
432.621626420454	2.27517197200282e-05	\\
433.600408380682	2.11582126710971e-05	\\
434.579190340908	1.91415289371544e-05	\\
435.557972301136	1.77855781725579e-05	\\
436.536754261364	1.90956746183606e-05	\\
437.51553622159	2.04397286209161e-05	\\
438.494318181818	2.09691373712557e-05	\\
439.473100142044	2.90161870348627e-05	\\
440.451882102272	1.61228734890267e-05	\\
441.4306640625	2.5130653603561e-05	\\
442.409446022726	2.62610876242944e-05	\\
443.388227982954	1.05886275167522e-05	\\
444.367009943182	2.06392317369812e-05	\\
445.345791903408	2.69318550752992e-05	\\
446.324573863636	1.09385902288311e-05	\\
447.303355823864	3.8824878326675e-05	\\
448.28213778409	2.98221535794225e-05	\\
449.260919744318	2.08945904851187e-05	\\
450.239701704544	1.84496525487653e-05	\\
451.218483664772	2.36782847192999e-05	\\
452.197265625	2.70391070954276e-05	\\
453.176047585226	2.26400049812571e-05	\\
454.154829545454	3.53342871245187e-05	\\
455.133611505682	1.75008280699147e-05	\\
456.112393465908	2.50123313400442e-05	\\
457.091175426136	1.65816990291289e-05	\\
458.069957386364	1.96408063500061e-05	\\
459.04873934659	2.01859472765485e-05	\\
460.027521306818	4.59755835269218e-05	\\
461.006303267044	2.08692438538631e-05	\\
461.985085227272	3.29804746195547e-05	\\
462.9638671875	1.74224814373323e-05	\\
463.942649147726	2.08527366115649e-05	\\
464.921431107954	2.09931109724103e-05	\\
465.900213068182	3.48432726564353e-05	\\
466.878995028408	2.27962261040069e-05	\\
467.857776988636	1.36892276831569e-05	\\
468.836558948864	2.57373547704728e-05	\\
469.81534090909	2.338270342572e-05	\\
470.794122869318	2.92198521445361e-05	\\
471.772904829544	2.94107853891617e-05	\\
472.751686789772	2.04196799684033e-05	\\
473.73046875	8.732735598268e-06	\\
474.709250710226	1.35603178890851e-05	\\
475.688032670454	2.56018469446477e-05	\\
476.666814630682	8.7955304160795e-06	\\
477.645596590908	4.09936913353376e-05	\\
478.624378551136	3.56133425800825e-05	\\
479.603160511364	3.96098602271075e-05	\\
480.58194247159	4.50703911205368e-05	\\
481.560724431818	6.00198735271217e-05	\\
482.539506392044	6.94837015763498e-05	\\
483.518288352272	7.56552188874683e-05	\\
484.4970703125	8.93850005272537e-05	\\
485.475852272726	5.60129454742192e-05	\\
486.454634232954	7.13469889744685e-05	\\
487.433416193182	5.26932252755515e-05	\\
488.412198153408	7.96667849202736e-05	\\
489.390980113636	4.59260244118082e-05	\\
490.369762073864	6.94126130073485e-05	\\
491.34854403409	1.62646453379164e-05	\\
492.327325994318	4.85676108008966e-05	\\
493.306107954544	3.3333888960352e-05	\\
494.284889914772	3.06396989521162e-05	\\
495.263671875	4.56325345534788e-05	\\
496.242453835226	6.01599748122015e-05	\\
497.221235795454	2.56150558501165e-05	\\
498.200017755682	6.97205650387336e-05	\\
499.178799715908	4.46079946950742e-05	\\
500.157581676136	0.000130969263230924	\\
501.136363636364	3.22222328187962e-05	\\
502.11514559659	4.44718521878237e-05	\\
503.093927556818	1.56365101647897e-05	\\
504.072709517044	2.16840859456447e-05	\\
505.051491477272	3.93183706881735e-05	\\
506.0302734375	3.42111766087766e-05	\\
507.009055397726	4.12354022067647e-05	\\
507.987837357954	2.75755539121847e-05	\\
508.966619318182	7.01010582692416e-06	\\
509.945401278408	4.57413564351459e-05	\\
510.924183238636	2.38044561385011e-05	\\
511.902965198864	5.43247534669367e-05	\\
512.88174715909	1.64554135004095e-05	\\
513.860529119318	3.90231594594913e-05	\\
514.839311079544	3.44309285864134e-05	\\
515.818093039772	5.41605947081512e-05	\\
516.796875	5.39636377284538e-05	\\
517.775656960226	9.00052853302719e-05	\\
518.754438920454	7.03272435116099e-05	\\
519.733220880682	5.72483850499938e-05	\\
520.712002840908	6.46418257680432e-05	\\
521.690784801136	2.16944025587372e-05	\\
522.669566761364	7.37533034143245e-05	\\
523.64834872159	2.41052066906766e-05	\\
524.627130681818	8.80800831075143e-05	\\
525.605912642044	3.040680522178e-05	\\
526.584694602272	6.5202951190305e-05	\\
527.5634765625	2.30983522331134e-05	\\
528.542258522726	4.23051813461252e-05	\\
529.521040482954	3.51302554420836e-05	\\
530.499822443182	6.48756283014716e-05	\\
531.478604403408	4.14078379656109e-05	\\
532.457386363636	6.62911663494356e-05	\\
533.436168323864	1.92231552268929e-05	\\
534.41495028409	4.00627926179068e-05	\\
535.393732244318	3.81039939794128e-05	\\
536.372514204544	1.31056336485542e-05	\\
537.351296164772	4.1575322985102e-05	\\
538.330078125	2.61755915893194e-05	\\
539.308860085226	2.65781617811646e-05	\\
540.287642045454	3.5152531720138e-05	\\
541.266424005682	7.64642820214035e-06	\\
542.245205965908	4.48040055982982e-05	\\
543.223987926136	3.03104852868693e-05	\\
544.202769886364	1.17800607532863e-05	\\
545.18155184659	3.8059919702583e-05	\\
546.160333806818	2.27469101711632e-05	\\
547.139115767044	1.48710560972903e-05	\\
548.117897727272	5.16900426272555e-05	\\
549.0966796875	1.87439158601075e-05	\\
550.075461647726	2.12470618551356e-05	\\
551.054243607954	3.35245823152734e-05	\\
552.033025568182	1.20648976649242e-05	\\
553.011807528408	3.76541989468524e-05	\\
553.990589488636	2.86197242707103e-05	\\
554.969371448864	1.08490207962905e-05	\\
555.94815340909	2.35053280226301e-05	\\
556.926935369318	2.33876895691924e-05	\\
557.905717329544	1.99508817820513e-05	\\
558.884499289772	2.19707647568518e-05	\\
559.86328125	2.04631812384846e-05	\\
560.842063210226	1.23041515579495e-05	\\
561.820845170454	1.90609718061809e-05	\\
562.799627130682	1.90338826233471e-05	\\
563.778409090908	1.16552662498608e-05	\\
564.757191051136	2.82272126210094e-05	\\
565.735973011364	2.5799577639608e-05	\\
566.71475497159	1.12135866803532e-05	\\
567.693536931818	3.07337584854144e-05	\\
568.672318892044	2.47633453998708e-05	\\
569.651100852272	2.37470565782437e-05	\\
570.6298828125	3.38174746681426e-05	\\
571.608664772726	2.48275618499474e-05	\\
572.587446732954	2.40746921568808e-05	\\
573.566228693182	3.55167139967328e-05	\\
574.545010653408	2.05342243811788e-05	\\
575.523792613636	3.29787221036711e-05	\\
576.502574573864	4.09270111090116e-05	\\
577.48135653409	2.7390410946029e-05	\\
578.460138494318	3.42637241572767e-05	\\
579.438920454544	3.94374598334336e-05	\\
580.417702414772	1.74658563216748e-05	\\
581.396484375	3.52891344987248e-05	\\
582.375266335226	3.08575185490934e-05	\\
583.354048295454	1.7818674951908e-05	\\
584.332830255682	4.14149178464728e-05	\\
585.311612215908	3.03821979093671e-05	\\
586.290394176136	8.89546553708131e-06	\\
587.269176136364	3.40012483892889e-05	\\
588.24795809659	3.24021470963711e-05	\\
589.226740056818	2.79690456982833e-05	\\
590.205522017044	2.14148269650298e-05	\\
591.184303977272	1.6108182942686e-05	\\
592.1630859375	9.53045579038187e-06	\\
593.141867897726	2.79310622313067e-05	\\
594.120649857954	1.79902016990671e-05	\\
595.099431818182	1.20796727703192e-05	\\
596.078213778408	2.03464932859695e-05	\\
597.056995738636	2.30778697742872e-05	\\
598.035777698864	2.78263896326602e-05	\\
599.01455965909	2.27176790993403e-05	\\
599.993341619318	4.96006241116551e-05	\\
600.972123579544	4.06343861140496e-05	\\
601.950905539772	4.45038608615637e-05	\\
602.9296875	3.48181226476419e-05	\\
603.908469460226	2.91285068637163e-05	\\
604.887251420454	2.91397885674968e-05	\\
605.866033380682	3.68609596641049e-05	\\
606.844815340908	2.82583238752652e-05	\\
607.823597301136	2.84363494091401e-05	\\
608.802379261364	3.94852626886653e-05	\\
609.78116122159	2.44574118886887e-05	\\
610.759943181818	1.16926521722773e-05	\\
611.738725142044	3.65001309682791e-05	\\
612.717507102272	3.6285311660664e-05	\\
613.6962890625	1.38968972589705e-05	\\
614.675071022726	2.55397071988027e-05	\\
615.653852982954	4.80761065513496e-05	\\
616.632634943182	4.33560956344769e-05	\\
617.611416903408	1.80094786569487e-05	\\
618.590198863636	4.8670095681629e-05	\\
619.568980823864	7.70947070073431e-05	\\
620.54776278409	8.474733476124e-05	\\
621.526544744318	4.51476293152924e-05	\\
622.505326704544	4.6683716459238e-05	\\
623.484108664772	4.95281759626478e-05	\\
624.462890625	6.03718472755304e-05	\\
625.441672585226	2.58469761618378e-05	\\
626.420454545454	2.85767721659096e-05	\\
627.399236505682	1.7800226035812e-05	\\
628.378018465908	4.64549251042397e-05	\\
629.356800426136	3.15550947102572e-05	\\
630.335582386364	4.27023322098507e-05	\\
631.31436434659	7.71576795047126e-05	\\
632.293146306818	8.46246179664273e-05	\\
633.271928267044	6.19956924876555e-05	\\
634.250710227272	7.00523081128043e-05	\\
635.2294921875	4.14979976505967e-05	\\
636.208274147726	4.80790219843066e-05	\\
637.187056107954	1.08430713596323e-05	\\
638.165838068182	3.54024323279945e-06	\\
639.144620028408	3.88059682012847e-05	\\
640.123401988636	4.69085516156146e-05	\\
641.102183948864	4.79748815441946e-05	\\
642.08096590909	5.33111106885104e-05	\\
643.059747869318	4.65424916489808e-05	\\
644.038529829544	5.24663017157696e-05	\\
645.017311789772	3.83768043148119e-05	\\
645.99609375	4.39620030936809e-05	\\
646.974875710226	4.72802508038985e-05	\\
647.953657670454	4.47240745200298e-05	\\
648.932439630682	3.00381043169861e-05	\\
649.911221590908	2.54383608988532e-05	\\
650.890003551136	4.75674883739542e-05	\\
651.868785511364	4.30700836945287e-05	\\
652.84756747159	6.56071061023532e-05	\\
653.826349431818	5.0908063400814e-05	\\
654.805131392044	5.30088928637181e-05	\\
655.783913352272	3.71060138956302e-05	\\
656.7626953125	2.34442969382713e-05	\\
657.741477272726	1.23073612397053e-05	\\
658.720259232954	2.10363999467356e-05	\\
659.699041193182	2.6092581273184e-05	\\
660.677823153408	4.30419522016438e-05	\\
661.656605113636	3.91640411796941e-05	\\
662.635387073864	5.62274621976392e-05	\\
663.61416903409	4.70629115614526e-05	\\
664.592950994318	4.86846528015266e-05	\\
665.571732954544	3.53688829578907e-05	\\
666.550514914772	1.62260166657232e-05	\\
667.529296875	2.59255568312027e-05	\\
668.508078835226	2.90741829719006e-05	\\
669.486860795454	3.0090201487546e-05	\\
670.465642755682	5.20662332828683e-05	\\
671.444424715908	3.35264607019896e-05	\\
672.423206676136	4.0428570941312e-05	\\
673.401988636364	3.48981774794996e-05	\\
674.38077059659	2.88902647770207e-05	\\
675.359552556818	3.00759321735865e-05	\\
676.338334517044	2.31464024786107e-05	\\
677.317116477272	3.53223147888389e-05	\\
678.2958984375	3.22140394735008e-05	\\
679.274680397726	2.98605332031963e-05	\\
680.253462357954	3.5996285117813e-05	\\
681.232244318182	3.3523392193044e-05	\\
682.211026278408	3.69580097685436e-05	\\
683.189808238636	3.79897401223547e-05	\\
684.168590198864	1.92394059263728e-05	\\
685.14737215909	2.27162318826423e-05	\\
686.126154119318	2.46687643086356e-05	\\
687.104936079544	2.17783371478421e-05	\\
688.083718039772	2.56658736056629e-05	\\
689.0625	2.35305450702349e-05	\\
690.041281960226	1.98469080438256e-05	\\
691.020063920454	1.97269679679445e-05	\\
691.998845880682	1.98007977301244e-05	\\
692.977627840908	1.1934819413856e-05	\\
693.956409801136	1.7769203575069e-05	\\
694.935191761364	1.50597907773392e-05	\\
695.91397372159	4.90738560964978e-06	\\
696.892755681818	2.00248170728229e-05	\\
697.871537642044	1.69594843515416e-05	\\
698.850319602272	1.44169774599281e-05	\\
699.8291015625	3.33142741701868e-05	\\
700.807883522726	1.13192555010796e-05	\\
701.786665482954	1.91518656312496e-05	\\
702.765447443182	1.87089329557709e-05	\\
703.744229403408	1.70236945959639e-05	\\
704.723011363636	1.89492677963594e-05	\\
705.701793323864	1.7906289149152e-05	\\
706.68057528409	1.55196647938232e-05	\\
707.659357244318	1.53425393734741e-05	\\
708.638139204544	1.35182217215366e-05	\\
709.616921164772	1.37548231646215e-05	\\
710.595703125	1.13954632102812e-05	\\
711.574485085226	1.82953632879265e-05	\\
712.553267045454	1.29994479099606e-05	\\
713.532049005682	1.11461991630525e-05	\\
714.510830965908	1.21886312432749e-05	\\
715.489612926136	1.11786356872025e-05	\\
716.468394886364	1.19594030744756e-05	\\
717.44717684659	9.98723980284729e-06	\\
718.425958806818	1.20598248980198e-05	\\
719.404740767044	7.7792294325851e-06	\\
720.383522727272	2.86893848250273e-06	\\
};
\addplot [color=blue,solid,forget plot]
  table[row sep=crcr]{
720.383522727272	2.86893848250273e-06	\\
721.3623046875	9.92753004690333e-06	\\
722.341086647726	6.5817086973121e-06	\\
723.319868607954	9.03123275924938e-06	\\
724.298650568182	5.12825935312144e-06	\\
725.277432528408	1.1402824159983e-05	\\
726.256214488636	1.19695461837847e-05	\\
727.234996448864	9.63395895941414e-06	\\
728.21377840909	1.91710687756567e-05	\\
729.192560369318	7.94905825786957e-06	\\
730.171342329544	9.18360658672625e-06	\\
731.150124289772	1.05729468016188e-05	\\
732.12890625	1.40907273885728e-05	\\
733.107688210226	1.18382152647408e-05	\\
734.086470170454	1.29977973898349e-05	\\
735.065252130682	1.56967639718519e-05	\\
736.044034090908	1.47683769301327e-05	\\
737.022816051136	1.43108634668959e-05	\\
738.001598011364	1.41354423661959e-05	\\
738.98037997159	1.6297196046816e-05	\\
739.959161931818	1.14003993906336e-05	\\
740.937943892044	1.53104355190422e-05	\\
741.916725852272	1.26334767766154e-05	\\
742.8955078125	1.63409467253887e-05	\\
743.874289772726	8.75158212938686e-06	\\
744.853071732954	1.47230950880362e-05	\\
745.831853693182	1.00941836160963e-05	\\
746.810635653408	8.04445481949173e-06	\\
747.789417613636	1.14366819100006e-05	\\
748.768199573864	1.05520611732442e-05	\\
749.74698153409	2.23523901379029e-05	\\
750.725763494318	6.73701796500205e-06	\\
751.704545454544	8.27709036382596e-06	\\
752.683327414772	5.64397193730172e-06	\\
753.662109375	6.4667713698425e-06	\\
754.640891335226	1.54251529336085e-05	\\
755.619673295454	1.31892141425952e-05	\\
756.598455255682	6.31505515251901e-06	\\
757.577237215908	5.07556077128489e-06	\\
758.556019176136	8.07042682065754e-06	\\
759.534801136364	9.37722105167135e-06	\\
760.51358309659	1.60405822575918e-05	\\
761.492365056818	1.01619822289372e-05	\\
762.471147017044	1.15919788573932e-05	\\
763.449928977272	8.19791235965212e-06	\\
764.4287109375	1.32694250404478e-05	\\
765.407492897726	1.21339582967254e-05	\\
766.386274857954	1.38062844386122e-05	\\
767.365056818182	8.8434797253929e-06	\\
768.343838778408	2.39118485118223e-06	\\
769.322620738636	1.17541215389862e-05	\\
770.301402698864	1.57844621878945e-05	\\
771.28018465909	1.69899835800057e-05	\\
772.258966619318	1.79746346577852e-05	\\
773.237748579544	7.89904604263783e-06	\\
774.216530539772	1.17235646985265e-05	\\
775.1953125	5.63094223916358e-06	\\
776.174094460226	1.43128742273163e-05	\\
777.152876420454	1.76901512190201e-05	\\
778.131658380682	1.00633774506968e-05	\\
779.110440340908	1.65478762538527e-05	\\
780.089222301136	6.84679135046073e-06	\\
781.068004261364	1.43647468893023e-05	\\
782.04678622159	1.22444145869418e-05	\\
783.025568181818	8.58026006011166e-06	\\
784.004350142044	1.07022562994121e-05	\\
784.983132102272	1.33503240911817e-05	\\
785.9619140625	1.74643393666025e-05	\\
786.940696022726	1.33695330016656e-05	\\
787.919477982954	1.45110835145044e-05	\\
788.898259943182	1.15349548544361e-05	\\
789.877041903408	9.98480349246685e-06	\\
790.855823863636	7.94201797354474e-06	\\
791.834605823864	1.25417431521347e-05	\\
792.81338778409	1.0075578053821e-05	\\
793.792169744318	9.67479118083649e-06	\\
794.770951704544	1.24867463309669e-05	\\
795.749733664772	1.0459760694485e-05	\\
796.728515625	1.17489098899105e-05	\\
797.707297585226	9.75069468704481e-06	\\
798.686079545454	7.16062111769029e-06	\\
799.664861505682	8.48799139271876e-06	\\
800.643643465908	9.59103710857123e-06	\\
801.622425426136	8.95857181944531e-06	\\
802.601207386364	9.46728870650038e-06	\\
803.57998934659	1.02380632490192e-05	\\
804.558771306818	1.10014090297123e-05	\\
805.537553267044	1.04620329928171e-05	\\
806.516335227272	2.56363748448601e-06	\\
807.4951171875	8.08677236997562e-06	\\
808.473899147726	1.16467964971281e-05	\\
809.452681107954	1.26019975784314e-05	\\
810.431463068182	7.13981337560786e-06	\\
811.410245028408	7.56290878858539e-06	\\
812.389026988636	3.82637059619624e-06	\\
813.367808948864	1.07368420737707e-05	\\
814.34659090909	7.47131279362935e-06	\\
815.325372869318	8.42284721780795e-06	\\
816.304154829544	1.31966734368966e-05	\\
817.282936789772	8.31170978395677e-06	\\
818.26171875	6.90159181943916e-06	\\
819.240500710226	8.88282809790017e-06	\\
820.219282670454	7.94000689128859e-06	\\
821.198064630682	4.74952048032989e-06	\\
822.176846590908	9.14174717331691e-06	\\
823.155628551136	9.83754319824707e-06	\\
824.134410511364	9.24067726780044e-06	\\
825.11319247159	7.19966921146201e-06	\\
826.091974431818	5.29948983779813e-06	\\
827.070756392044	5.10995121612773e-06	\\
828.049538352272	6.69210652852351e-06	\\
829.0283203125	3.98231260996256e-06	\\
830.007102272726	7.92585522916879e-06	\\
830.985884232954	7.30693273336828e-06	\\
831.964666193182	6.25929369639964e-06	\\
832.943448153408	4.10904443837068e-06	\\
833.922230113636	9.47536142703511e-06	\\
834.901012073864	1.05688487664085e-05	\\
835.87979403409	6.17340176548879e-06	\\
836.858575994318	1.04654006941378e-05	\\
837.837357954544	5.57289290581596e-06	\\
838.816139914772	7.29971108787024e-06	\\
839.794921875	1.0501388535616e-05	\\
840.773703835226	4.86009704747846e-06	\\
841.752485795454	6.7520970236428e-06	\\
842.731267755682	6.63696171390803e-06	\\
843.710049715908	7.25138096342439e-06	\\
844.688831676136	8.06600804984206e-06	\\
845.667613636364	6.17173773126315e-06	\\
846.64639559659	1.0462015866135e-05	\\
847.625177556818	3.36325542917691e-06	\\
848.603959517044	4.77330395230023e-06	\\
849.582741477272	5.02827705615746e-06	\\
850.5615234375	7.34244273935134e-06	\\
851.540305397726	7.94820100905439e-06	\\
852.519087357954	1.37841405655506e-05	\\
853.497869318182	3.45028650008394e-06	\\
854.476651278408	8.56194620101915e-06	\\
855.455433238636	8.08985542764708e-06	\\
856.434215198864	6.35350439931475e-06	\\
857.41299715909	7.29183154241366e-06	\\
858.391779119318	7.35775983015573e-06	\\
859.370561079544	5.61879584313254e-06	\\
860.349343039772	9.51448266952116e-06	\\
861.328125	1.05624338729901e-05	\\
862.306906960226	7.09819770185197e-06	\\
863.285688920454	1.05057686041331e-05	\\
864.264470880682	9.86001420612713e-06	\\
865.243252840908	9.97730239881319e-06	\\
866.222034801136	1.35161783712881e-05	\\
867.200816761364	1.59537943236991e-05	\\
868.17959872159	7.7154437303769e-06	\\
869.158380681818	8.14351404860428e-06	\\
870.137162642044	1.33237228382307e-05	\\
871.115944602272	1.24633243093069e-05	\\
872.0947265625	6.54372684504269e-06	\\
873.073508522726	1.14035412758635e-05	\\
874.052290482954	6.93498737709038e-06	\\
875.031072443182	4.6887358587395e-06	\\
876.009854403408	8.80307639490868e-06	\\
876.988636363636	4.1460515174794e-06	\\
877.967418323864	1.01063741154809e-05	\\
878.94620028409	1.18420852003443e-05	\\
879.924982244318	9.12129521098305e-06	\\
880.903764204544	7.61101209874841e-06	\\
881.882546164772	1.04047662820388e-05	\\
882.861328125	6.83074962917844e-06	\\
883.840110085226	2.0948347298975e-06	\\
884.818892045454	8.15647290107978e-06	\\
885.797674005682	2.57132124182708e-06	\\
886.776455965908	2.28083928315843e-06	\\
887.755237926136	8.47688519320335e-06	\\
888.734019886364	1.02090831310253e-05	\\
889.71280184659	1.00789146411308e-05	\\
890.691583806818	9.18189969971419e-06	\\
891.670365767044	9.7125192984708e-06	\\
892.649147727272	1.29556171611014e-05	\\
893.6279296875	8.51589504182375e-06	\\
894.606711647726	8.71079306677644e-06	\\
895.585493607954	5.52800032878963e-06	\\
896.564275568182	1.02772349048963e-05	\\
897.543057528408	1.22331018754663e-05	\\
898.521839488636	5.21911396041434e-06	\\
899.500621448864	8.11982938295713e-06	\\
900.47940340909	3.53377594816043e-06	\\
901.458185369318	1.30031263296906e-05	\\
902.436967329544	1.40257295111483e-05	\\
903.415749289772	5.64412119477902e-06	\\
904.39453125	2.07729696269395e-06	\\
905.373313210226	3.6803965022991e-06	\\
906.352095170454	1.04503885731939e-05	\\
907.330877130682	1.94387091883678e-05	\\
908.309659090908	7.86560175106121e-06	\\
909.288441051136	1.38524535814339e-05	\\
910.267223011364	1.83279624635255e-05	\\
911.24600497159	9.40084959916275e-06	\\
912.224786931818	1.77544562193731e-06	\\
913.203568892044	9.49710073485067e-06	\\
914.182350852272	5.33014444320465e-06	\\
915.1611328125	9.13400840901863e-06	\\
916.139914772726	1.70321928982999e-05	\\
917.118696732954	1.57671173570567e-05	\\
918.097478693182	7.06650811059016e-06	\\
919.076260653408	6.42987258456014e-06	\\
920.055042613636	7.14992755115382e-06	\\
921.033824573864	9.86015790418805e-06	\\
922.01260653409	1.55076581679202e-05	\\
922.991388494318	6.12571938401897e-06	\\
923.970170454544	1.6677414849699e-05	\\
924.948952414772	1.49822631674967e-05	\\
925.927734375	7.80371452169384e-06	\\
926.906516335226	8.76797838192572e-06	\\
927.885298295454	6.56289656677167e-06	\\
928.864080255682	1.0635564058816e-06	\\
929.842862215908	9.686889731899e-06	\\
930.821644176136	7.3727708488426e-06	\\
931.800426136364	1.02418888693076e-05	\\
932.77920809659	1.05346519528849e-05	\\
933.757990056818	2.5419981396885e-06	\\
934.736772017044	3.39853639698561e-06	\\
935.715553977272	3.84688392737244e-06	\\
936.6943359375	5.60881191339676e-06	\\
937.673117897726	7.5013032624823e-06	\\
938.651899857954	5.99574363847741e-06	\\
939.630681818182	8.23630020828433e-06	\\
940.609463778408	6.60931757342586e-06	\\
941.588245738636	4.34095481894753e-06	\\
942.567027698864	5.62414319801276e-06	\\
943.54580965909	4.74903120790488e-06	\\
944.524591619318	7.71839739557625e-06	\\
945.503373579544	1.08230112284406e-05	\\
946.482155539772	4.32911002125061e-06	\\
947.4609375	4.42392119074486e-06	\\
948.439719460226	9.97314640009068e-06	\\
949.418501420454	6.58318757590015e-06	\\
950.397283380682	7.80215360052003e-06	\\
951.376065340908	9.96586590734124e-06	\\
952.354847301136	3.71738949730098e-06	\\
953.333629261364	7.23860107670819e-06	\\
954.31241122159	5.52629458217159e-06	\\
955.291193181818	6.25821214318357e-06	\\
956.269975142044	8.61171027389707e-06	\\
957.248757102272	7.76371501742069e-06	\\
958.2275390625	6.79716159695662e-06	\\
959.206321022726	8.82650260557858e-06	\\
960.185102982954	4.8052303497627e-06	\\
961.163884943182	1.68674312911515e-06	\\
962.142666903408	6.00825602023125e-06	\\
963.121448863636	2.23185613528177e-06	\\
964.100230823864	2.27799677269552e-06	\\
965.07901278409	4.47596969549496e-06	\\
966.057794744318	3.80243562126873e-06	\\
967.036576704544	1.05611979986364e-05	\\
968.015358664772	2.23229877566085e-06	\\
968.994140625	3.35694260032223e-06	\\
969.972922585226	7.63718557378045e-06	\\
970.951704545454	3.15799498132154e-06	\\
971.930486505682	4.64200652663155e-06	\\
972.909268465908	9.4137673958113e-06	\\
973.888050426136	5.10022674741144e-06	\\
974.866832386364	8.28602823009156e-06	\\
975.84561434659	1.09660647860393e-05	\\
976.824396306818	3.03785000776773e-06	\\
977.803178267044	7.9810595185391e-06	\\
978.781960227272	9.14304566576554e-06	\\
979.7607421875	6.60025971903095e-06	\\
980.739524147726	8.11387077006538e-06	\\
981.718306107954	6.05729616669828e-06	\\
982.697088068182	7.09966934649343e-06	\\
983.675870028408	8.61169088676424e-06	\\
984.654651988636	9.41729334561859e-06	\\
985.633433948864	3.64720921283309e-06	\\
986.61221590909	1.19164712066043e-05	\\
987.590997869318	7.23500558350733e-06	\\
988.569779829544	9.88103254735878e-06	\\
989.548561789772	6.24723418387067e-06	\\
990.52734375	1.14838441803864e-05	\\
991.506125710226	8.24152103198878e-06	\\
992.484907670454	7.98759681729701e-06	\\
993.463689630682	2.2880905946585e-06	\\
994.442471590908	3.47047166456709e-06	\\
995.421253551136	1.00956098708823e-05	\\
996.400035511364	9.5663946485362e-06	\\
997.37881747159	7.96860184991757e-06	\\
998.357599431818	1.26354971824416e-05	\\
999.336381392044	1.19841111879004e-05	\\
1000.31516335227	2.81135100722844e-05	\\
1001.2939453125	3.6030158802965e-06	\\
1002.27272727273	9.59019589746454e-06	\\
1003.25150923295	1.28762168960721e-05	\\
1004.23029119318	8.91500010118696e-06	\\
1005.20907315341	7.08374495453258e-06	\\
1006.18785511364	5.59360903441178e-06	\\
1007.16663707386	6.05926570944155e-06	\\
1008.14541903409	9.46886774360455e-06	\\
1009.12420099432	1.15481690405207e-05	\\
1010.10298295454	9.89061019896993e-06	\\
1011.08176491477	9.13867764038104e-06	\\
1012.060546875	2.12956362244634e-05	\\
1013.03932883523	1.10299291599532e-05	\\
1014.01811079545	6.89990647264027e-06	\\
1014.99689275568	7.18207842963344e-06	\\
1015.97567471591	2.79363091380843e-06	\\
1016.95445667614	3.81217599002186e-06	\\
1017.93323863636	4.97038618875776e-06	\\
1018.91202059659	9.22311486772722e-06	\\
1019.89080255682	1.05567700610577e-05	\\
1020.86958451704	1.06082774763624e-05	\\
1021.84836647727	1.29736515803825e-05	\\
1022.8271484375	8.35389036141812e-06	\\
1023.80593039773	5.15387337207124e-06	\\
1024.78471235795	5.91996493474573e-06	\\
1025.76349431818	9.36009347372494e-06	\\
1026.74227627841	5.71180009269906e-06	\\
1027.72105823864	8.79581034794135e-06	\\
1028.69984019886	2.47733701149496e-06	\\
1029.67862215909	7.79169588058177e-06	\\
1030.65740411932	5.12765956899441e-06	\\
1031.63618607954	1.18974314642561e-05	\\
1032.61496803977	6.85967196569204e-06	\\
1033.59375	1.1486063684764e-05	\\
1034.57253196023	4.13500890530391e-06	\\
1035.55131392045	2.77664646603062e-06	\\
1036.53009588068	7.11440073575909e-06	\\
1037.50887784091	1.15172346922684e-05	\\
1038.48765980114	1.62984640903105e-05	\\
1039.46644176136	1.30239767106442e-05	\\
1040.44522372159	6.13860416699048e-06	\\
1041.42400568182	7.9723203592375e-06	\\
1042.40278764204	5.96024651236381e-06	\\
1043.38156960227	5.89075220969072e-06	\\
1044.3603515625	1.08435069010592e-05	\\
1045.33913352273	1.33404294245084e-05	\\
1046.31791548295	8.86481055539548e-06	\\
1047.29669744318	8.32025182488744e-06	\\
1048.27547940341	7.26836068760702e-06	\\
1049.25426136364	7.50773248330855e-06	\\
1050.23304332386	7.32274509468659e-06	\\
1051.21182528409	9.16632641461136e-06	\\
1052.19060724432	1.66705735130663e-05	\\
1053.16938920454	9.90228462633708e-06	\\
1054.14817116477	5.64686988704405e-06	\\
1055.126953125	1.11117215388105e-05	\\
1056.10573508523	1.8384278264639e-05	\\
1057.08451704545	1.4003347603734e-05	\\
1058.06329900568	1.09357966116547e-05	\\
1059.04208096591	6.37479497232501e-06	\\
1060.02086292614	8.94662853170955e-06	\\
1060.99964488636	8.65733714526618e-06	\\
1061.97842684659	1.51214755045394e-05	\\
1062.95720880682	1.34102590276068e-05	\\
1063.93599076704	3.74800158625016e-06	\\
1064.91477272727	9.14145436160171e-07	\\
1065.8935546875	1.17039980382201e-05	\\
1066.87233664773	1.58861465466736e-05	\\
1067.85111860795	1.17154579051536e-05	\\
1068.82990056818	1.74133217228282e-06	\\
1069.80868252841	8.33159410581573e-06	\\
1070.78746448864	6.97805861416286e-06	\\
1071.76624644886	8.49277911734537e-06	\\
1072.74502840909	9.72545215778811e-06	\\
1073.72381036932	6.54938440392944e-06	\\
1074.70259232954	6.93929221601208e-06	\\
1075.68137428977	5.74909722272262e-06	\\
1076.66015625	1.067639913683e-05	\\
1077.63893821023	1.2058748629306e-05	\\
1078.61772017045	1.00484127129623e-05	\\
1079.59650213068	2.62371605669031e-06	\\
1080.57528409091	1.00409913243755e-05	\\
1081.55406605114	1.18370849360404e-05	\\
1082.53284801136	1.30912664704722e-05	\\
1083.51162997159	1.62993667790829e-05	\\
1084.49041193182	1.19111958972825e-05	\\
1085.46919389204	1.04611596183083e-05	\\
1086.44797585227	1.10213458288785e-05	\\
1087.4267578125	1.18325200416979e-05	\\
1088.40553977273	1.42912957478714e-05	\\
1089.38432173295	1.3570863976925e-05	\\
1090.36310369318	1.06731873950392e-05	\\
1091.34188565341	8.95872108738702e-06	\\
1092.32066761364	8.89609476890923e-06	\\
1093.29944957386	9.9460232941549e-06	\\
1094.27823153409	1.16363430470791e-05	\\
1095.25701349432	9.21746023096635e-06	\\
1096.23579545454	7.5415234489666e-06	\\
1097.21457741477	1.84113583550455e-05	\\
1098.193359375	1.02152080242537e-05	\\
1099.17214133523	9.96473727722851e-06	\\
1100.15092329545	9.99536150111613e-06	\\
1101.12970525568	9.74290488858655e-06	\\
1102.10848721591	1.46823360433482e-05	\\
1103.08726917614	1.1025398260549e-05	\\
1104.06605113636	1.20781794123939e-05	\\
1105.04483309659	1.30044097356681e-05	\\
1106.02361505682	9.66200391958416e-06	\\
1107.00239701704	1.09926098550895e-05	\\
1107.98117897727	1.02388664349503e-05	\\
1108.9599609375	1.2993249312068e-05	\\
1109.93874289773	8.45860406562998e-06	\\
1110.91752485795	1.16988641478981e-05	\\
1111.89630681818	1.68676379079747e-05	\\
1112.87508877841	8.2039876853354e-06	\\
1113.85387073864	7.44919140101904e-06	\\
1114.83265269886	1.24481790539709e-05	\\
1115.81143465909	1.12174577273591e-05	\\
1116.79021661932	1.26803989112612e-05	\\
1117.76899857954	7.11004247747096e-06	\\
1118.74778053977	1.04006930973648e-05	\\
1119.7265625	9.30477284311147e-06	\\
1120.70534446023	7.25011580552962e-06	\\
1121.68412642045	1.2389531794371e-05	\\
1122.66290838068	9.05133627972013e-06	\\
1123.64169034091	7.50229890154943e-06	\\
1124.62047230114	1.0554451657729e-05	\\
1125.59925426136	7.76875381876684e-06	\\
1126.57803622159	1.34380267679161e-05	\\
1127.55681818182	6.18353591680978e-06	\\
1128.53560014204	3.65742634702118e-06	\\
1129.51438210227	1.6201167961502e-06	\\
1130.4931640625	7.49979159816417e-06	\\
1131.47194602273	6.53073278824043e-06	\\
1132.45072798295	7.78368615846101e-06	\\
1133.42950994318	9.61351808703352e-06	\\
1134.40829190341	7.15657179245051e-06	\\
1135.38707386364	2.47234786193467e-06	\\
1136.36585582386	3.5202698730979e-06	\\
1137.34463778409	4.92291385677401e-06	\\
1138.32341974432	5.04395376623217e-06	\\
1139.30220170454	2.07590172991307e-06	\\
1140.28098366477	8.54165826205653e-06	\\
1141.259765625	7.14060742093188e-06	\\
1142.23854758523	7.5413996434982e-06	\\
1143.21732954545	7.55918600872317e-06	\\
1144.19611150568	1.1366099302399e-05	\\
1145.17489346591	6.71494207843649e-06	\\
1146.15367542614	6.0557370504897e-06	\\
1147.13245738636	8.69459081193106e-06	\\
1148.11123934659	8.23872230546995e-06	\\
1149.09002130682	6.6013025482515e-06	\\
1150.06880326704	1.17408686618369e-05	\\
1151.04758522727	7.58180836149154e-06	\\
1152.0263671875	7.0872550895243e-06	\\
1153.00514914773	8.04357259957754e-06	\\
1153.98393110795	5.95638175355593e-06	\\
1154.96271306818	1.33992442884348e-05	\\
1155.94149502841	1.02315168202171e-05	\\
1156.92027698864	1.03789280854802e-05	\\
1157.89905894886	7.77435885600993e-06	\\
1158.87784090909	1.03474490290572e-05	\\
1159.85662286932	1.10729563615659e-05	\\
1160.83540482954	9.90096900071875e-06	\\
1161.81418678977	1.35885563311806e-05	\\
1162.79296875	7.03722436739402e-06	\\
1163.77175071023	1.17918509856363e-05	\\
1164.75053267045	1.33944313464387e-05	\\
1165.72931463068	1.4190913927122e-05	\\
1166.70809659091	1.01482050961201e-05	\\
1167.68687855114	7.90523146412728e-06	\\
1168.66566051136	1.03501038205385e-05	\\
1169.64444247159	1.09876594153717e-05	\\
1170.62322443182	9.58331056188795e-06	\\
1171.60200639204	1.21403713088325e-05	\\
1172.58078835227	1.62001850362918e-05	\\
1173.5595703125	8.55984136744832e-06	\\
1174.53835227273	1.0618445288138e-05	\\
1175.51713423295	1.26197221787241e-05	\\
1176.49591619318	1.37788061789986e-05	\\
1177.47469815341	1.09046356981769e-05	\\
1178.45348011364	1.08558614627667e-05	\\
1179.43226207386	1.13266872226906e-05	\\
1180.41104403409	1.22895106731155e-05	\\
1181.38982599432	8.40080543875603e-06	\\
1182.36860795454	2.62550575535645e-06	\\
1183.34738991477	9.33171999275498e-06	\\
1184.326171875	9.03074382496242e-06	\\
1185.30495383523	5.97335451516298e-06	\\
1186.28373579545	8.80224401641635e-06	\\
1187.26251775568	6.55169905222148e-06	\\
1188.24129971591	6.62280063914188e-06	\\
1189.22008167614	7.46349598163707e-06	\\
1190.19886363636	1.00306973759267e-05	\\
1191.17764559659	1.07718161197994e-05	\\
1192.15642755682	9.37910441634512e-06	\\
1193.13520951704	5.90404081533451e-06	\\
1194.11399147727	9.07301657207746e-06	\\
1195.0927734375	9.0180767349642e-06	\\
1196.07155539773	4.62544699639083e-06	\\
1197.05033735795	6.5503472612136e-06	\\
1198.02911931818	5.29957414348622e-06	\\
1199.00790127841	6.21375615653072e-06	\\
1199.98668323864	1.45988253950282e-05	\\
1200.96546519886	1.25141105171177e-05	\\
1201.94424715909	1.03606257309679e-05	\\
1202.92302911932	6.22127297037917e-06	\\
1203.90181107954	9.4553543504466e-06	\\
1204.88059303977	1.13282191969398e-05	\\
1205.859375	1.03280479290556e-05	\\
1206.83815696023	1.167857950408e-05	\\
1207.81693892045	1.20530346708422e-05	\\
1208.79572088068	1.08417194280262e-05	\\
1209.77450284091	7.39072679648876e-06	\\
1210.75328480114	1.36180895316908e-05	\\
1211.73206676136	9.23395306417326e-06	\\
1212.71084872159	1.37553219752686e-05	\\
1213.68963068182	1.45745304549215e-05	\\
1214.66841264204	1.22312677904045e-05	\\
1215.64719460227	1.14952845229068e-05	\\
1216.6259765625	1.23439101064888e-05	\\
1217.60475852273	1.10652440907007e-05	\\
1218.58354048295	1.2307174761543e-05	\\
1219.56232244318	7.70756382849721e-06	\\
1220.54110440341	1.287977751018e-05	\\
1221.51988636364	1.68580595065481e-05	\\
1222.49866832386	1.08078190618122e-05	\\
1223.47745028409	1.09797922178017e-05	\\
1224.45623224432	1.60905849018784e-05	\\
1225.43501420454	1.37186234487702e-05	\\
1226.41379616477	1.54858412794454e-05	\\
1227.392578125	1.1718703925947e-05	\\
1228.37136008523	1.10002773272833e-05	\\
1229.35014204545	9.45024265226184e-06	\\
1230.32892400568	1.12015039384939e-05	\\
1231.30770596591	1.09409455104174e-05	\\
1232.28648792614	1.50179906982567e-05	\\
1233.26526988636	6.75581608285547e-06	\\
1234.24405184659	7.52699605295662e-06	\\
1235.22283380682	9.71144153973643e-06	\\
1236.20161576704	7.19117750455514e-06	\\
1237.18039772727	5.13913238738955e-06	\\
1238.1591796875	1.20576664660085e-05	\\
1239.13796164773	9.0192428684559e-06	\\
1240.11674360795	7.80834092628696e-06	\\
1241.09552556818	9.02313336417152e-06	\\
1242.07430752841	7.41632434115994e-06	\\
1243.05308948864	7.44470598324761e-06	\\
1244.03187144886	3.54547920866195e-06	\\
1245.01065340909	6.30631477846083e-06	\\
1245.98943536932	6.67187439269918e-06	\\
1246.96821732954	1.29488692792403e-05	\\
1247.94699928977	7.11427830935232e-06	\\
1248.92578125	1.62017163903201e-05	\\
1249.90456321023	1.33274503220765e-05	\\
1250.88334517045	7.10908093769383e-06	\\
1251.86212713068	7.1001525867978e-06	\\
1252.84090909091	7.05656783110871e-06	\\
1253.81969105114	8.11085437508601e-06	\\
1254.79847301136	5.50654927332256e-06	\\
1255.77725497159	9.80469147955778e-06	\\
1256.75603693182	4.47002435550051e-06	\\
1257.73481889204	2.94168417028414e-06	\\
1258.71360085227	7.14191966560934e-06	\\
1259.6923828125	1.07075481628958e-05	\\
1260.67116477273	1.07028291628836e-05	\\
1261.64994673295	6.60205456033036e-06	\\
1262.62872869318	5.56480276619485e-06	\\
1263.60751065341	3.73749969196998e-06	\\
1264.58629261364	4.75346865066084e-06	\\
1265.56507457386	8.41378788455722e-06	\\
1266.54385653409	4.75801340708133e-06	\\
1267.52263849432	7.7250705168885e-06	\\
1268.50142045454	9.03531799411671e-06	\\
1269.48020241477	8.26844027826793e-06	\\
1270.458984375	6.64556798131797e-06	\\
1271.43776633523	8.91658475668634e-06	\\
1272.41654829545	1.05841820413776e-05	\\
1273.39533025568	2.61547810354191e-06	\\
1274.37411221591	3.92301581841256e-06	\\
1275.35289417614	1.22705309386352e-05	\\
1276.33167613636	7.79402865017555e-06	\\
1277.31045809659	1.39645707098146e-05	\\
1278.28924005682	1.07302308881419e-05	\\
1279.26802201704	1.23964491625031e-05	\\
1280.24680397727	1.34414106073748e-05	\\
1281.2255859375	8.17612205048223e-06	\\
1282.20436789773	1.12287799284438e-05	\\
1283.18314985795	5.21740916096715e-06	\\
1284.16193181818	7.84999947897632e-06	\\
1285.14071377841	1.24845905553285e-05	\\
1286.11949573864	5.23146610846226e-06	\\
1287.09827769886	1.24090758407343e-05	\\
1288.07705965909	1.11048026379912e-05	\\
1289.05584161932	1.24150823175604e-05	\\
1290.03462357954	9.11374246656892e-06	\\
1291.01340553977	1.0079786804988e-05	\\
1291.9921875	7.90400596450826e-06	\\
1292.97096946023	3.27444612271e-06	\\
1293.94975142045	9.15301941833103e-06	\\
1294.92853338068	1.501156664068e-05	\\
1295.90731534091	7.41825615099329e-06	\\
1296.88609730114	1.79839573507866e-05	\\
1297.86487926136	1.06606600894048e-05	\\
1298.84366122159	7.64792392708457e-06	\\
1299.82244318182	1.47173956412356e-05	\\
1300.80122514204	9.54464639559961e-06	\\
1301.78000710227	1.20197476640348e-05	\\
1302.7587890625	1.743083900991e-05	\\
1303.73757102273	1.23754509504651e-05	\\
1304.71635298295	1.63632736118051e-05	\\
1305.69513494318	8.90669604449999e-06	\\
1306.67391690341	1.0092824336911e-05	\\
1307.65269886364	5.66578538976327e-06	\\
1308.63148082386	1.07528815499336e-05	\\
1309.61026278409	1.28211725348765e-05	\\
1310.58904474432	1.22399982671906e-05	\\
1311.56782670454	1.39204788772802e-05	\\
1312.54660866477	1.21726123823182e-05	\\
1313.525390625	5.98021370073383e-06	\\
1314.50417258523	7.85919461724079e-06	\\
1315.48295454545	1.08657943037746e-05	\\
1316.46173650568	5.92581519135898e-06	\\
1317.44051846591	1.25302887660247e-05	\\
1318.41930042614	1.48155543721676e-05	\\
1319.39808238636	8.58132032354649e-06	\\
1320.37686434659	1.53372282281687e-05	\\
1321.35564630682	3.71312682198514e-06	\\
1322.33442826704	1.42564817849537e-05	\\
1323.31321022727	8.14313415468121e-06	\\
1324.2919921875	8.26131044480428e-06	\\
1325.27077414773	1.30972578581124e-05	\\
1326.24955610795	4.05905126957879e-06	\\
1327.22833806818	1.45813522724804e-05	\\
1328.20712002841	6.9391273289002e-06	\\
1329.18590198864	8.34141230201277e-06	\\
1330.16468394886	1.56698391874349e-05	\\
1331.14346590909	6.74115187480742e-06	\\
1332.12224786932	7.26938254752299e-06	\\
1333.10102982954	1.54361891246914e-05	\\
1334.07981178977	9.45825982005391e-06	\\
1335.05859375	1.2896993047217e-05	\\
1336.03737571023	9.59530908588798e-06	\\
1337.01615767045	3.99276606859801e-06	\\
1337.99493963068	1.25130994780762e-05	\\
1338.97372159091	6.80299766012819e-06	\\
1339.95250355114	6.95914871761922e-06	\\
1340.93128551136	1.00222876366329e-05	\\
1341.91006747159	7.1993825030783e-06	\\
1342.88884943182	1.22977752595915e-05	\\
1343.86763139204	1.09564316792703e-05	\\
1344.84641335227	5.70306191079194e-06	\\
1345.8251953125	1.02505483133125e-05	\\
1346.80397727273	7.29821844587202e-06	\\
1347.78275923295	8.28942091281758e-06	\\
1348.76154119318	1.35173153535846e-05	\\
1349.74032315341	1.47226026183624e-05	\\
1350.71910511364	3.88992594896253e-06	\\
1351.69788707386	1.31188669446476e-05	\\
1352.67666903409	1.0749931279033e-05	\\
1353.65545099432	1.10592306814643e-05	\\
1354.63423295454	1.79722246024609e-05	\\
1355.61301491477	6.5854386123324e-06	\\
1356.591796875	1.21172207977791e-05	\\
1357.57057883523	1.07386706690433e-05	\\
1358.54936079545	2.98001214465288e-06	\\
1359.52814275568	1.17024115194635e-05	\\
1360.50692471591	9.98451326460565e-06	\\
1361.48570667614	7.88973796584853e-06	\\
1362.46448863636	1.4381598561615e-05	\\
1363.44327059659	9.90896407123318e-06	\\
1364.42205255682	1.06105502110972e-05	\\
1365.40083451704	1.71831025872622e-05	\\
1366.37961647727	2.14646271819215e-06	\\
1367.3583984375	2.1750443644102e-05	\\
1368.33718039773	2.18534614455876e-05	\\
1369.31596235795	9.69418910361557e-06	\\
1370.29474431818	1.07749616706695e-05	\\
1371.27352627841	5.16480256727829e-06	\\
1372.25230823864	1.59960070800277e-05	\\
1373.23109019886	2.19287854159847e-05	\\
1374.20987215909	9.14012054327563e-06	\\
1375.18865411932	1.15023782831645e-05	\\
1376.16743607954	1.66832100534302e-05	\\
1377.14621803977	5.85176813177187e-06	\\
1378.125	2.63039989066248e-05	\\
1379.10378196023	1.54646700714568e-05	\\
1380.08256392045	8.25809771603293e-06	\\
1381.06134588068	1.84157674820218e-05	\\
1382.04012784091	6.45376838532068e-06	\\
1383.01890980114	1.60061734511357e-05	\\
1383.99769176136	1.33424445504317e-05	\\
1384.97647372159	1.01085034811057e-05	\\
1385.95525568182	3.26175891798859e-06	\\
1386.93403764204	8.93106439743154e-06	\\
1387.91281960227	8.43153889081596e-06	\\
1388.8916015625	1.40198616526493e-05	\\
1389.87038352273	2.16200614725503e-05	\\
1390.84916548295	5.19681556349809e-06	\\
1391.82794744318	1.44476844103297e-05	\\
1392.80672940341	2.17827391110651e-06	\\
1393.78551136364	1.84101974215517e-05	\\
1394.76429332386	1.36260062125536e-05	\\
1395.74307528409	4.70158674800897e-06	\\
1396.72185724432	1.21007004808183e-05	\\
1397.70063920454	4.78203669913371e-06	\\
1398.67942116477	7.88891717811061e-06	\\
1399.658203125	1.10777582328441e-05	\\
1400.63698508523	1.215739924904e-05	\\
1401.61576704545	9.5499852551395e-06	\\
1402.59454900568	7.27173246378989e-06	\\
1403.57333096591	9.53787960603162e-06	\\
1404.55211292614	3.12264728368405e-06	\\
1405.53089488636	1.08360980159186e-05	\\
1406.50967684659	1.2498493296002e-05	\\
1407.48845880682	6.7130869561937e-06	\\
1408.46724076704	1.9004387029762e-05	\\
1409.44602272727	1.8074317564431e-05	\\
1410.4248046875	4.87743413551185e-06	\\
1411.40358664773	1.70264698229727e-05	\\
1412.38236860795	9.99986572149664e-06	\\
1413.36115056818	1.04944219994761e-05	\\
1414.33993252841	2.08932022782432e-05	\\
1415.31871448864	8.64840713133058e-06	\\
1416.29749644886	1.67070626744324e-05	\\
1417.27627840909	1.49008863415614e-05	\\
1418.25506036932	1.67134570849099e-05	\\
1419.23384232954	1.15843475665597e-05	\\
1420.21262428977	1.13743617519873e-05	\\
1421.19140625	1.23302486032693e-05	\\
1422.17018821023	6.66656750256353e-06	\\
1423.14897017045	1.29616889345405e-05	\\
1424.12775213068	1.08362036329299e-05	\\
1425.10653409091	8.34629526359392e-06	\\
1426.08531605114	1.71328241431698e-05	\\
1427.06409801136	1.07545087958006e-05	\\
1428.04287997159	1.01329106470359e-05	\\
1429.02166193182	1.36684802562368e-05	\\
1430.00044389204	4.54242071105006e-06	\\
1430.97922585227	1.07509877208572e-05	\\
1431.9580078125	1.21969728381494e-05	\\
1432.93678977273	4.50994487130407e-06	\\
1433.91557173295	1.78601364907279e-05	\\
1434.89435369318	8.1298841801468e-06	\\
1435.87313565341	1.24637817888222e-05	\\
1436.85191761364	8.37729682395601e-06	\\
1437.83069957386	8.57604988485241e-06	\\
1438.80948153409	4.61541925274933e-06	\\
1439.78826349432	8.56943022131039e-06	\\
1440.76704545454	1.7393918065167e-05	\\
1441.74582741477	1.65540719614706e-05	\\
1442.724609375	1.61081967510552e-05	\\
1443.70339133523	1.01258663538878e-05	\\
1444.68217329545	4.54234227167521e-06	\\
1445.66095525568	9.06207344560402e-06	\\
1446.63973721591	9.45310835676255e-06	\\
1447.61851917614	4.4346216084493e-06	\\
1448.59730113636	1.75670970852486e-05	\\
1449.57608309659	1.30117100147861e-05	\\
1450.55486505682	1.32478774891059e-05	\\
1451.53364701704	1.71915916431356e-05	\\
1452.51242897727	3.88220715560474e-06	\\
1453.4912109375	1.42216006771302e-05	\\
1454.46999289773	1.49134175638691e-05	\\
1455.44877485795	2.5717862311352e-06	\\
1456.42755681818	1.37244014471747e-05	\\
1457.40633877841	1.25744248030577e-05	\\
1458.38512073864	6.39872165104173e-06	\\
1459.36390269886	1.66840535423278e-05	\\
1460.34268465909	6.15503425414521e-06	\\
1461.32146661932	7.46074755792332e-06	\\
1462.30024857954	8.96724543242796e-06	\\
1463.27903053977	9.11162813060243e-06	\\
1464.2578125	1.17289228168592e-05	\\
1465.23659446023	3.67107440601806e-06	\\
1466.21537642045	1.10083394166795e-05	\\
1467.19415838068	7.13845960222458e-06	\\
1468.17294034091	5.2560804606111e-06	\\
1469.15172230114	1.46391233215622e-05	\\
1470.13050426136	1.69691503453902e-05	\\
1471.10928622159	1.41344004039836e-05	\\
1472.08806818182	2.3262532250379e-05	\\
1473.06685014204	2.01841756607369e-05	\\
1474.04563210227	1.6255914911469e-05	\\
1475.0244140625	2.05838133376699e-05	\\
1476.00319602273	1.46050735432495e-05	\\
1476.98197798295	1.5114612038281e-05	\\
1477.96075994318	1.47928677073933e-05	\\
1478.93954190341	8.5338369661574e-06	\\
1479.91832386364	8.21507004973125e-06	\\
1480.89710582386	3.84849018768893e-06	\\
1481.87588778409	2.34521396060846e-06	\\
1482.85466974432	1.83826473366787e-06	\\
1483.83345170454	8.30104544087155e-06	\\
1484.81223366477	1.55566214683729e-05	\\
1485.791015625	1.92711986064622e-05	\\
1486.76979758523	1.59747118832971e-05	\\
1487.74857954545	1.4982300137973e-05	\\
1488.72736150568	1.70877522566214e-05	\\
1489.70614346591	1.4422755853057e-05	\\
1490.68492542614	2.29649696228892e-05	\\
1491.66370738636	1.85208671827666e-05	\\
1492.64248934659	1.15021287751728e-05	\\
1493.62127130682	1.57261095550443e-05	\\
1494.60005326704	7.17444992958073e-06	\\
1495.57883522727	8.27993488248434e-06	\\
1496.5576171875	1.2967397429587e-05	\\
1497.53639914773	1.60669670196638e-05	\\
1498.51518110795	1.63960154842176e-05	\\
1499.49396306818	2.74735815179028e-05	\\
1500.47274502841	2.34249648598153e-05	\\
1501.45152698864	1.96723725180896e-05	\\
1502.43030894886	2.68582641650215e-05	\\
1503.40909090909	1.83251825627845e-05	\\
1504.38787286932	1.82438318808055e-05	\\
1505.36665482954	1.17941862168305e-05	\\
1506.34543678977	3.00260150847455e-06	\\
1507.32421875	1.1311809063092e-05	\\
1508.30300071023	1.36245684862314e-05	\\
1509.28178267045	1.6612472761053e-05	\\
1510.26056463068	2.44423069968239e-05	\\
1511.23934659091	2.28237924324403e-05	\\
1512.21812855114	2.35081398108955e-05	\\
1513.19691051136	2.95368105316311e-05	\\
1514.17569247159	2.59828926316464e-05	\\
1515.15447443182	1.36170678108091e-05	\\
1516.13325639204	6.95337113640151e-06	\\
1517.11203835227	1.98089581260454e-06	\\
1518.0908203125	1.6512858310654e-05	\\
1519.06960227273	2.59539734577793e-05	\\
1520.04838423295	2.25260832048143e-05	\\
1521.02716619318	2.38040406905119e-05	\\
1522.00594815341	2.37994211065541e-05	\\
1522.98473011364	2.32823832712849e-05	\\
1523.96351207386	2.63285641591811e-05	\\
1524.94229403409	9.86001549142846e-06	\\
1525.92107599432	1.29036162000831e-05	\\
1526.89985795454	7.06470806603996e-06	\\
1527.87863991477	7.24677805175799e-06	\\
1528.857421875	2.07466185874893e-05	\\
1529.83620383523	2.39048616274868e-05	\\
1530.81498579545	2.81963388497733e-05	\\
1531.79376775568	2.53239932690914e-05	\\
1532.77254971591	2.55605510733408e-05	\\
1533.75133167614	2.56280934962812e-05	\\
1534.73011363636	1.42348336514987e-05	\\
1535.70889559659	8.83095787167224e-06	\\
1536.68767755682	1.22169183780118e-05	\\
1537.66645951704	1.0535718449821e-05	\\
1538.64524147727	1.89452259078486e-05	\\
1539.6240234375	1.75491685868404e-05	\\
1540.60280539773	2.30674743100717e-05	\\
1541.58158735795	2.05774908949805e-05	\\
1542.56036931818	2.20869002532667e-05	\\
1543.53915127841	2.22778891142745e-05	\\
1544.51793323864	1.94249293259608e-05	\\
1545.49671519886	1.42926840057762e-05	\\
1546.47549715909	1.54896167123283e-05	\\
1547.45427911932	8.39909043731134e-06	\\
1548.43306107954	1.55302232550524e-05	\\
1549.41184303977	1.61914643922065e-05	\\
1550.390625	2.50097289917526e-05	\\
1551.36940696023	2.39756802893506e-05	\\
1552.34818892045	2.07464956971275e-05	\\
1553.32697088068	1.32627082173132e-05	\\
1554.30575284091	9.71268289738322e-06	\\
1555.28453480114	1.48399430789605e-05	\\
1556.26331676136	2.10603572200706e-05	\\
1557.24209872159	3.32823178613597e-05	\\
1558.22088068182	2.91717885478558e-05	\\
1559.19966264204	2.70397964805347e-05	\\
1560.17844460227	2.21222464982048e-05	\\
1561.1572265625	1.22104377543614e-05	\\
1562.13600852273	5.26507562211998e-06	\\
1563.11479048295	1.71958104125918e-05	\\
1564.09357244318	2.17311336664738e-05	\\
1565.07235440341	2.82688455450186e-05	\\
1566.05113636364	2.82307360883587e-05	\\
1567.02991832386	1.78577496379486e-05	\\
1568.00870028409	1.87886179737843e-05	\\
1568.98748224432	1.10658049626311e-05	\\
1569.96626420454	1.22250493521802e-05	\\
1570.94504616477	2.05978647791171e-05	\\
1571.923828125	3.11535106019386e-05	\\
1572.90261008523	2.78453271930331e-05	\\
1573.88139204545	3.29838897075912e-05	\\
1574.86017400568	2.5068581896136e-05	\\
1575.83895596591	1.85529100405017e-05	\\
1576.81773792614	1.51555456007696e-05	\\
1577.79651988636	1.65235329800607e-05	\\
1578.77530184659	2.16684024888468e-05	\\
1579.75408380682	2.55893787119335e-05	\\
1580.73286576704	2.32105163858559e-05	\\
1581.71164772727	1.88141700239547e-05	\\
1582.6904296875	1.9564795726097e-05	\\
1583.66921164773	1.87752500623316e-05	\\
1584.64799360795	1.48234967521148e-05	\\
1585.62677556818	1.17150077098985e-05	\\
1586.60555752841	1.71026017614557e-05	\\
1587.58433948864	1.50838922124659e-05	\\
1588.56312144886	1.67489172536202e-05	\\
1589.54190340909	1.75863759251109e-05	\\
1590.52068536932	1.73614614516466e-05	\\
1591.49946732954	1.80340530570256e-05	\\
1592.47824928977	1.40845691951689e-05	\\
1593.45703125	2.22745172854821e-05	\\
1594.43581321023	2.40161536018553e-05	\\
1595.41459517045	2.06981100904331e-05	\\
1596.39337713068	2.18410180795548e-05	\\
1597.37215909091	2.5445814096633e-05	\\
1598.35094105114	2.1111435840236e-05	\\
1599.32972301136	1.29693209914932e-05	\\
1600.30850497159	1.729636552229e-05	\\
1601.28728693182	9.38122348021035e-06	\\
1602.26606889204	1.45244743834684e-05	\\
1603.24485085227	1.43369798818184e-05	\\
1604.2236328125	1.65450996717634e-05	\\
1605.20241477273	1.32234454295438e-05	\\
1606.18119673295	1.62855463518043e-05	\\
1607.15997869318	2.02375323630248e-05	\\
1608.13876065341	1.43396709477571e-05	\\
1609.11754261364	1.34363146337308e-05	\\
1610.09632457386	2.01678391383544e-05	\\
1611.07510653409	2.087687031918e-05	\\
1612.05388849432	1.64900234537368e-05	\\
1613.03267045454	1.22929597510427e-05	\\
1614.01145241477	1.31106962972763e-05	\\
1614.990234375	1.33743043310649e-05	\\
1615.96901633523	1.34660361866575e-05	\\
1616.94779829545	8.89346715775711e-06	\\
1617.92658025568	5.71227116755015e-06	\\
1618.90536221591	1.23025840013357e-05	\\
1619.88414417614	1.12498646726125e-05	\\
1620.86292613636	1.75294814520834e-05	\\
1621.84170809659	1.73039693922641e-05	\\
1622.82049005682	1.85806008404246e-05	\\
1623.79927201704	1.67202443691765e-05	\\
1624.77805397727	1.50190531486105e-05	\\
1625.7568359375	1.6228168732758e-05	\\
1626.73561789773	1.33437764525738e-05	\\
1627.71439985795	1.62334132854353e-05	\\
1628.69318181818	7.68715941560196e-06	\\
1629.67196377841	7.10188062513394e-06	\\
1630.65074573864	9.50371705728536e-06	\\
1631.62952769886	9.97475145009483e-06	\\
1632.60830965909	1.74672248928841e-05	\\
1633.58709161932	2.09676319933214e-05	\\
1634.56587357954	2.08347383618891e-05	\\
1635.54465553977	2.07084850455064e-05	\\
1636.5234375	1.93932960022442e-05	\\
1637.50221946023	1.63796171740116e-05	\\
1638.48100142045	1.48072612235371e-05	\\
1639.45978338068	1.45173307652988e-05	\\
1640.43856534091	1.13946536442038e-05	\\
1641.41734730114	7.50917475568137e-06	\\
1642.39612926136	1.30005290587784e-05	\\
1643.37491122159	1.11242105323893e-05	\\
1644.35369318182	1.91916006511593e-05	\\
1645.33247514204	2.31017007077471e-05	\\
1646.31125710227	2.19309087751843e-05	\\
1647.2900390625	1.46908836033493e-05	\\
1648.26882102273	1.16537190468102e-05	\\
1649.24760298295	1.08101960183942e-05	\\
1650.22638494318	1.41157378418197e-05	\\
1651.20516690341	1.43734410540413e-05	\\
1652.18394886364	1.55329191528135e-05	\\
1653.16273082386	1.53137224276326e-05	\\
1654.14151278409	1.64953968906613e-05	\\
1655.12029474432	1.33762048302199e-05	\\
1656.09907670454	1.88828630142245e-05	\\
1657.07785866477	2.09897989377685e-05	\\
1658.056640625	1.81525515103464e-05	\\
1659.03542258523	1.5592701028559e-05	\\
1660.01420454545	8.48334466607352e-06	\\
1660.99298650568	1.36633316579147e-05	\\
1661.97176846591	2.25965854091161e-05	\\
1662.95055042614	2.57501025034283e-05	\\
1663.92933238636	2.55549013503757e-05	\\
1664.90811434659	1.95562494976692e-05	\\
1665.88689630682	1.38068118177693e-05	\\
1666.86567826704	1.59974273386179e-05	\\
1667.84446022727	1.46842776130288e-05	\\
1668.8232421875	2.22135284919658e-05	\\
1669.80202414773	2.29981137096437e-05	\\
1670.78080610795	1.76730103810649e-05	\\
1671.75958806818	1.99495853356779e-05	\\
1672.73837002841	1.54745051260241e-05	\\
1673.71715198864	1.68744083714705e-05	\\
1674.69593394886	1.46540089489827e-05	\\
1675.67471590909	1.8828243495898e-05	\\
1676.65349786932	2.11069556912036e-05	\\
1677.63227982954	1.66104908556069e-05	\\
1678.61106178977	1.54809546181247e-05	\\
1679.58984375	9.75688899324438e-06	\\
1680.56862571023	1.73075801665593e-05	\\
1681.54740767045	2.09064615918393e-05	\\
1682.52618963068	1.58970591663059e-05	\\
1683.50497159091	1.48234249669665e-05	\\
1684.48375355114	1.86427418235533e-05	\\
1685.46253551136	1.45662671607647e-05	\\
1686.44131747159	1.2558027701556e-05	\\
1687.42009943182	1.3036874122569e-05	\\
1688.39888139204	1.66819029281719e-05	\\
1689.37766335227	1.9474408266739e-05	\\
1690.3564453125	1.70573175281045e-05	\\
1691.33522727273	1.41312166294005e-05	\\
1692.31400923295	5.45335183308364e-06	\\
1693.29279119318	9.35389992337399e-06	\\
1694.27157315341	8.04450000098705e-06	\\
1695.25035511364	1.41520816335329e-05	\\
1696.22913707386	2.20129099141463e-05	\\
1697.20791903409	1.54166199535175e-05	\\
1698.18670099432	1.64139755438252e-05	\\
1699.16548295454	1.02446793254025e-05	\\
1700.14426491477	5.55261718356504e-06	\\
1701.123046875	9.96284231272349e-06	\\
1702.10182883523	1.33064640121073e-05	\\
1703.08061079545	1.11910682301529e-05	\\
1704.05939275568	8.15665407308678e-06	\\
1705.03817471591	9.87464612371688e-06	\\
1706.01695667614	1.46492791847909e-05	\\
1706.99573863636	1.07989847452543e-05	\\
1707.97452059659	1.41058241509242e-05	\\
1708.95330255682	1.52806847179042e-05	\\
1709.93208451704	1.6395556650061e-05	\\
1710.91086647727	5.59110539991613e-06	\\
1711.8896484375	1.74041949680393e-05	\\
1712.86843039773	1.90097718269021e-05	\\
1713.84721235795	1.86035991551693e-05	\\
1714.82599431818	1.86463100822531e-05	\\
1715.80477627841	1.66946561478505e-05	\\
1716.78355823864	2.09636669323439e-05	\\
1717.76234019886	1.70114575193964e-05	\\
1718.74112215909	1.95547408156666e-05	\\
1719.71990411932	1.87217678976336e-05	\\
1720.69868607954	1.38280376797999e-05	\\
1721.67746803977	1.58388026271429e-05	\\
1722.65625	1.75010669181981e-05	\\
1723.63503196023	1.89908332867789e-05	\\
1724.61381392045	1.85191298385434e-05	\\
1725.59259588068	1.53696021939816e-05	\\
1726.57137784091	1.63291794300285e-05	\\
1727.55015980114	1.59847163108369e-05	\\
1728.52894176136	1.57465092408788e-05	\\
1729.50772372159	1.65717447765496e-05	\\
1730.48650568182	1.38926682758435e-05	\\
1731.46528764204	1.04062011692483e-05	\\
1732.44406960227	1.45920181120544e-05	\\
1733.4228515625	1.34032376870602e-05	\\
1734.40163352273	9.83492450137827e-06	\\
1735.38041548295	1.4735119950186e-05	\\
1736.35919744318	9.54662950345803e-06	\\
1737.33797940341	9.16053648658617e-06	\\
1738.31676136364	1.20185243463733e-05	\\
1739.29554332386	1.09009618363295e-05	\\
1740.27432528409	1.34157543137618e-05	\\
1741.25310724432	8.96944758599654e-06	\\
1742.23188920454	1.28187872134601e-05	\\
1743.21067116477	8.76208703999715e-06	\\
1744.189453125	9.47940598029618e-06	\\
1745.16823508523	6.26577003371419e-06	\\
1746.14701704545	1.15212546794788e-05	\\
1747.12579900568	6.89351996956909e-06	\\
1748.10458096591	8.773451383533e-06	\\
1749.08336292614	8.02899721694715e-06	\\
1750.06214488636	1.36422842538776e-05	\\
1751.04092684659	1.3271715620079e-05	\\
1752.01970880682	9.67330184558685e-06	\\
1752.99849076704	1.23059238245812e-05	\\
1753.97727272727	7.57207905301432e-06	\\
1754.9560546875	1.19082643486791e-05	\\
1755.93483664773	9.75094569877141e-06	\\
1756.91361860795	8.86760422009559e-06	\\
1757.89240056818	7.64658007024304e-06	\\
1758.87118252841	8.83320916980561e-06	\\
1759.84996448864	1.66240978619053e-05	\\
1760.82874644886	1.31938466127946e-05	\\
1761.80752840909	1.52464368338241e-05	\\
1762.78631036932	1.86899911986928e-05	\\
1763.76509232954	1.220788947099e-05	\\
1764.74387428977	1.95615235992782e-05	\\
1765.72265625	1.49473865969641e-05	\\
1766.70143821023	1.70633514919759e-05	\\
1767.68022017045	1.7903418193089e-05	\\
1768.65900213068	1.6955848748824e-05	\\
1769.63778409091	1.79365119121817e-05	\\
1770.61656605114	2.29687968349095e-05	\\
1771.59534801136	1.87070829849616e-05	\\
1772.57412997159	1.57347671466349e-05	\\
1773.55291193182	1.94563340523767e-05	\\
1774.53169389204	1.62901218283655e-05	\\
1775.51047585227	2.33549780068716e-05	\\
1776.4892578125	1.94134287133079e-05	\\
1777.46803977273	1.60780161631468e-05	\\
1778.44682173295	1.23553065874618e-05	\\
1779.42560369318	1.29585352577986e-05	\\
1780.40438565341	1.15493884573921e-05	\\
1781.38316761364	1.35889064415367e-05	\\
1782.36194957386	1.64362268929462e-05	\\
1783.34073153409	1.63417838557502e-05	\\
1784.31951349432	1.15703595365459e-05	\\
1785.29829545454	1.06903928572017e-05	\\
1786.27707741477	1.89562541553201e-05	\\
1787.255859375	1.10809081846092e-05	\\
1788.23464133523	1.19405857788129e-05	\\
1789.21342329545	1.273946824838e-05	\\
1790.19220525568	1.09795945531786e-05	\\
1791.17098721591	1.18550243717252e-05	\\
1792.14976917614	1.64814837640952e-05	\\
1793.12855113636	1.54746434725341e-05	\\
1794.10733309659	1.98817610246029e-05	\\
1795.08611505682	1.38152674558002e-05	\\
1796.06489701704	1.02561349521878e-05	\\
1797.04367897727	1.08302052025927e-05	\\
1798.0224609375	1.70844976614562e-05	\\
1799.00124289773	1.50463780978563e-05	\\
1799.98002485795	1.94620466084177e-05	\\
1800.95880681818	1.82706763271189e-05	\\
1801.93758877841	1.32903206471309e-05	\\
1802.91637073864	1.04192896771255e-05	\\
1803.89515269886	1.45801661621206e-05	\\
1804.87393465909	1.18324669052725e-05	\\
1805.85271661932	1.94802422585846e-05	\\
1806.83149857954	1.38443091684051e-05	\\
1807.81028053977	1.66743816413002e-05	\\
1808.7890625	1.24231291837468e-05	\\
1809.76784446023	1.20816149501292e-05	\\
1810.74662642045	1.61621700989054e-05	\\
1811.72540838068	1.33201196617015e-05	\\
1812.70419034091	1.97872705111454e-05	\\
1813.68297230114	1.63447119304696e-05	\\
1814.66175426136	1.42994070217469e-05	\\
1815.64053622159	1.30582326388498e-05	\\
1816.61931818182	1.02953795710986e-05	\\
1817.59810014204	1.5791777981223e-05	\\
1818.57688210227	2.03095588873801e-05	\\
1819.5556640625	2.15132567762822e-05	\\
1820.53444602273	1.87910709807992e-05	\\
1821.51322798295	2.08139282947399e-05	\\
1822.49200994318	7.13731762525929e-06	\\
1823.47079190341	1.16602193846953e-05	\\
1824.44957386364	1.41413922479818e-05	\\
1825.42835582386	2.65815106939053e-05	\\
1826.40713778409	2.44979175602787e-05	\\
1827.38591974432	1.5430428578406e-05	\\
1828.36470170454	9.0967082036575e-06	\\
1829.34348366477	8.49684697359181e-06	\\
1830.322265625	8.79186222238605e-06	\\
1831.30104758523	2.11662784433356e-05	\\
1832.27982954545	2.57982909830735e-05	\\
1833.25861150568	2.37549819217119e-05	\\
1834.23739346591	1.4504778347474e-05	\\
1835.21617542614	8.93039832495832e-06	\\
1836.19495738636	1.62768966538732e-05	\\
1837.17373934659	2.39896883086052e-05	\\
1838.15252130682	2.05832594813853e-05	\\
1839.13130326704	2.44546517688232e-05	\\
1840.11008522727	1.51553226026645e-05	\\
1841.0888671875	7.54945815409171e-06	\\
1842.06764914773	1.6517119464482e-05	\\
1843.04643110795	2.06038472091339e-05	\\
1844.02521306818	1.64573497378197e-05	\\
1845.00399502841	1.37226575997763e-05	\\
1845.98277698864	6.06535239856543e-06	\\
1846.96155894886	1.1645041275018e-05	\\
1847.94034090909	1.32183302435999e-05	\\
1848.91912286932	1.70728078822682e-05	\\
1849.89790482954	1.53320161022013e-05	\\
1850.87668678977	1.04282005665665e-05	\\
1851.85546875	1.1744545259099e-05	\\
1852.83425071023	1.42722113867383e-05	\\
1853.81303267045	1.20315459903378e-05	\\
1854.79181463068	1.57247285343539e-05	\\
1855.77059659091	1.51170948049344e-05	\\
1856.74937855114	1.41544558163218e-05	\\
1857.72816051136	1.37300532555533e-05	\\
1858.70694247159	1.28617676390692e-05	\\
1859.68572443182	1.08020763813365e-05	\\
1860.66450639204	1.43171754771398e-05	\\
1861.64328835227	1.52363649198625e-05	\\
1862.6220703125	1.73305726788919e-05	\\
1863.60085227273	1.47922695104962e-05	\\
1864.57963423295	1.38905656024591e-05	\\
1865.55841619318	1.19790854330182e-05	\\
1866.53719815341	1.23655554265726e-05	\\
1867.51598011364	1.22587123684802e-05	\\
1868.49476207386	1.28465831169991e-05	\\
1869.47354403409	9.81374057227526e-06	\\
1870.45232599432	1.03654602724547e-05	\\
1871.43110795454	1.47608716508858e-05	\\
1872.40988991477	1.90100981845312e-05	\\
1873.388671875	1.62479762812261e-05	\\
1874.36745383523	1.30764400757106e-05	\\
1875.34623579545	1.50042583199962e-05	\\
1876.32501775568	5.77460368311088e-06	\\
1877.30379971591	9.13248277665128e-06	\\
1878.28258167614	1.08297561032908e-05	\\
1879.26136363636	1.40158940639543e-05	\\
1880.24014559659	8.39116307242371e-06	\\
1881.21892755682	8.71175480338794e-06	\\
1882.19770951704	6.11201169380559e-06	\\
1883.17649147727	6.13400756663844e-06	\\
1884.1552734375	7.54631051353716e-06	\\
1885.13405539773	8.6572700335283e-06	\\
1886.11283735795	3.31242216476026e-06	\\
1887.09161931818	4.52674232493319e-06	\\
1888.07040127841	9.48257002721501e-06	\\
1889.04918323864	1.1777710816031e-05	\\
1890.02796519886	1.00121533059993e-05	\\
1891.00674715909	4.64945295232852e-06	\\
1891.98552911932	3.27981693193491e-06	\\
1892.96431107954	3.92059769664307e-06	\\
1893.94309303977	8.67738408353248e-06	\\
1894.921875	6.66646431123459e-06	\\
1895.90065696023	2.05284665453561e-06	\\
1896.87943892045	2.51283310894254e-06	\\
1897.85822088068	5.41875172061408e-06	\\
1898.83700284091	3.43295232072828e-06	\\
1899.81578480114	1.18835702870953e-05	\\
1900.79456676136	1.05859342929557e-05	\\
1901.77334872159	1.24569833231102e-05	\\
1902.75213068182	6.96824950646837e-06	\\
1903.73091264204	1.17457027414302e-05	\\
1904.70969460227	9.04772319499703e-06	\\
1905.6884765625	1.00092970326083e-05	\\
1906.66725852273	8.73370688352453e-06	\\
1907.64604048295	1.1781114947476e-05	\\
1908.62482244318	1.11280517776506e-05	\\
1909.60360440341	1.49417468089668e-05	\\
1910.58238636364	1.56103218382655e-05	\\
1911.56116832386	1.50050895005838e-05	\\
1912.53995028409	1.23362480399745e-05	\\
1913.51873224432	1.11227654807394e-05	\\
1914.49751420454	1.46554639140251e-05	\\
1915.47629616477	1.4623154723641e-05	\\
1916.455078125	1.56304345882397e-05	\\
1917.43386008523	1.36932364673084e-05	\\
1918.41264204545	1.22919496164178e-05	\\
1919.39142400568	1.18889242271997e-05	\\
1920.37020596591	1.23794344593587e-05	\\
1921.34898792614	1.11075096062954e-05	\\
1922.32776988636	1.47085373526635e-05	\\
1923.30655184659	1.34402737427926e-05	\\
1924.28533380682	1.30249677188531e-05	\\
1925.26411576704	1.4352890775907e-05	\\
1926.24289772727	9.01300874180936e-06	\\
1927.2216796875	1.50919851968584e-05	\\
1928.20046164773	5.78773892572841e-06	\\
1929.17924360795	1.42533090981033e-05	\\
1930.15802556818	7.35341120083535e-06	\\
1931.13680752841	1.70293443731104e-05	\\
1932.11558948864	6.63659620427054e-06	\\
1933.09437144886	1.70657416739231e-05	\\
1934.07315340909	1.14605302247808e-05	\\
1935.05193536932	2.04058762713608e-05	\\
1936.03071732954	7.90549425260135e-06	\\
1937.00949928977	1.04033242277572e-05	\\
1937.98828125	9.67699027136935e-06	\\
1938.96706321023	1.32163663697268e-05	\\
1939.94584517045	1.17921557056788e-05	\\
1940.92462713068	1.41086013266635e-05	\\
1941.90340909091	1.28651567462797e-05	\\
1942.88219105114	1.0499010963137e-05	\\
1943.86097301136	9.74071780698492e-06	\\
1944.83975497159	5.51001241479547e-06	\\
1945.81853693182	6.41655426716495e-06	\\
1946.79731889204	9.98774060304465e-06	\\
1947.77610085227	9.78770200094604e-06	\\
1948.7548828125	8.81036607037955e-06	\\
1949.73366477273	6.59487493888109e-06	\\
1950.71244673295	9.12530765263172e-06	\\
1951.69122869318	5.96408348591512e-06	\\
1952.67001065341	7.43329099311322e-06	\\
1953.64879261364	1.0440135148296e-05	\\
1954.62757457386	1.10785053387341e-05	\\
1955.60635653409	1.42468468072331e-05	\\
1956.58513849432	1.26064925749797e-05	\\
1957.56392045454	1.03871664177912e-05	\\
1958.54270241477	9.20731159324451e-06	\\
1959.521484375	8.73073249349637e-06	\\
1960.50026633523	1.06225294746132e-05	\\
1961.47904829545	6.65796921176516e-06	\\
1962.45783025568	7.81144536088841e-06	\\
1963.43661221591	9.21432628302775e-06	\\
1964.41539417614	1.21028005598686e-05	\\
1965.39417613636	1.37240193272603e-05	\\
1966.37295809659	1.2710318869156e-05	\\
1967.35174005682	3.63437986569035e-06	\\
1968.33052201704	1.37811655022303e-06	\\
1969.30930397727	1.03141485032882e-05	\\
1970.2880859375	1.40321330342596e-05	\\
1971.26686789773	1.50794153960939e-05	\\
1972.24564985795	1.39817650815909e-05	\\
1973.22443181818	4.57573642290179e-06	\\
1974.20321377841	3.12590756125873e-06	\\
1975.18199573864	1.047117297922e-06	\\
1976.16077769886	1.03332588497384e-05	\\
1977.13955965909	7.41937834527636e-06	\\
1978.11834161932	1.1774061313896e-05	\\
1979.09712357954	1.05072951722797e-05	\\
1980.07590553977	9.72391672467896e-06	\\
1981.0546875	3.21624182515013e-06	\\
1982.03346946023	5.18966448839485e-06	\\
1983.01225142045	9.20903452284273e-06	\\
1983.99103338068	9.5455650342174e-06	\\
1984.96981534091	9.73972985479646e-06	\\
1985.94859730114	5.69977600125921e-06	\\
1986.92737926136	8.48240367056398e-06	\\
1987.90616122159	9.42223666359802e-06	\\
1988.88494318182	1.16366126295665e-05	\\
1989.86372514204	1.10372895571537e-05	\\
1990.84250710227	1.19236302903921e-05	\\
1991.8212890625	1.25534074579674e-05	\\
1992.80007102273	1.32167680465161e-05	\\
1993.77885298295	8.68515656384366e-06	\\
1994.75763494318	1.08151597827359e-05	\\
1995.73641690341	1.14863461504324e-05	\\
1996.71519886364	1.22661764799915e-05	\\
1997.69398082386	1.13583341002894e-05	\\
1998.67276278409	7.0454215240975e-06	\\
1999.65154474432	1.15770952016e-05	\\
2000.63032670454	1.5380474912724e-05	\\
2001.60910866477	1.69487422939793e-05	\\
2002.587890625	1.27898179872726e-05	\\
2003.56667258523	2.16921821187787e-05	\\
2004.54545454545	1.00904526720266e-05	\\
2005.52423650568	9.23043057316933e-06	\\
2006.50301846591	1.58658810016204e-05	\\
2007.48180042614	1.01859595390405e-05	\\
2008.46058238636	1.47770634265064e-05	\\
2009.43936434659	1.57325697544742e-05	\\
2010.41814630682	1.77471352223243e-05	\\
2011.39692826704	1.21551362476789e-05	\\
2012.37571022727	1.04390493549777e-05	\\
2013.3544921875	6.31026160246397e-06	\\
2014.33327414773	1.08000434070178e-05	\\
2015.31205610795	1.36230105440286e-05	\\
2016.29083806818	1.46854736364667e-05	\\
2017.26962002841	8.77926056130003e-06	\\
2018.24840198864	1.01049998062263e-05	\\
2019.22718394886	1.1263904997825e-05	\\
2020.20596590909	1.65411668881646e-05	\\
2021.18474786932	1.22804731618298e-05	\\
2022.16352982954	1.91805804086029e-05	\\
2023.14231178977	1.12724295986904e-05	\\
2024.12109375	6.9377170470418e-06	\\
2025.09987571023	1.04591608750718e-05	\\
2026.07865767045	1.63116288188631e-05	\\
2027.05743963068	1.8285416203708e-05	\\
2028.03622159091	1.12472078839186e-05	\\
2029.01500355114	7.93183814403049e-06	\\
2029.99378551136	1.56694234497058e-06	\\
2030.97256747159	1.34445733944483e-05	\\
2031.95134943182	2.35296429274789e-05	\\
2032.93013139204	1.55821998451553e-05	\\
2033.90891335227	9.32607579585942e-06	\\
2034.8876953125	1.11245932429092e-05	\\
2035.86647727273	1.03199227778699e-05	\\
2036.84525923295	1.38492157923552e-05	\\
2037.82404119318	1.19913467876653e-05	\\
2038.80282315341	8.6504920120632e-06	\\
2039.78160511364	2.15191166947662e-06	\\
2040.76038707386	1.17793417545122e-05	\\
2041.73916903409	1.0265594829775e-05	\\
2042.71795099432	1.39804967437247e-05	\\
2043.69673295454	1.19066912666882e-05	\\
2044.67551491477	7.04296793119206e-06	\\
2045.654296875	6.54810670360145e-06	\\
2046.63307883523	1.48059993672879e-05	\\
2047.61186079545	9.89162011989905e-06	\\
2048.59064275568	1.15711376947935e-05	\\
2049.56942471591	1.05791205769395e-05	\\
2050.54820667614	1.3087440340532e-05	\\
2051.52698863636	1.15247963546881e-05	\\
2052.50577059659	1.18155965045567e-05	\\
2053.48455255682	1.40521773814718e-05	\\
2054.46333451704	1.30931440557917e-05	\\
2055.44211647727	1.01447153835253e-05	\\
2056.4208984375	1.44504930423958e-05	\\
2057.39968039773	1.15504119244389e-05	\\
2058.37846235795	7.30696198810194e-06	\\
2059.35724431818	1.40209169095327e-05	\\
2060.33602627841	1.45791015922583e-05	\\
2061.31480823864	8.12181795289017e-06	\\
2062.29359019886	1.56420192779084e-05	\\
2063.27237215909	9.78989852070145e-06	\\
2064.25115411932	8.05275325801951e-06	\\
2065.22993607954	1.14267612967262e-05	\\
2066.20871803977	8.01248950199946e-06	\\
2067.1875	7.4427582513008e-06	\\
2068.16628196023	1.13860857937157e-05	\\
2069.14506392045	9.59003105649133e-06	\\
2070.12384588068	9.40938702577624e-06	\\
2071.10262784091	9.84933272862228e-06	\\
2072.08140980114	7.28478938576503e-06	\\
2073.06019176136	9.49215522532696e-06	\\
2074.03897372159	8.53130929574489e-06	\\
2075.01775568182	5.9519754720871e-06	\\
2075.99653764204	6.69702806330472e-06	\\
2076.97531960227	9.3654581335439e-06	\\
2077.9541015625	5.58154409553787e-06	\\
2078.93288352273	8.89762477168118e-06	\\
2079.91166548295	2.43355584460901e-06	\\
2080.89044744318	6.26070614293695e-06	\\
2081.86922940341	5.693146178263e-06	\\
2082.84801136364	5.6857213988755e-06	\\
2083.82679332386	2.9303119945883e-06	\\
2084.80557528409	8.58703976759977e-06	\\
2085.78435724432	7.77771007836846e-06	\\
2086.76313920454	6.32698234982391e-06	\\
2087.74192116477	5.72409845990738e-06	\\
2088.720703125	8.6605945245811e-06	\\
2089.69948508523	5.50653034494106e-06	\\
2090.67826704545	5.16043404198381e-06	\\
2091.65704900568	4.0223479312074e-06	\\
2092.63583096591	3.90223258440293e-06	\\
2093.61461292614	3.78382790879595e-06	\\
2094.59339488636	8.00868914238196e-06	\\
2095.57217684659	9.5238213053883e-06	\\
2096.55095880682	9.87519515416853e-06	\\
2097.52974076704	7.27600074119795e-06	\\
2098.50852272727	5.30716708979216e-06	\\
2099.4873046875	3.9432243976415e-06	\\
2100.46608664773	6.08889479325245e-06	\\
2101.44486860795	4.37124876756518e-06	\\
2102.42365056818	3.75005923171832e-06	\\
2103.40243252841	6.41613045462651e-06	\\
2104.38121448864	2.94399173951683e-06	\\
2105.35999644886	4.2756243030387e-06	\\
2106.33877840909	8.64398467573944e-06	\\
2107.31756036932	1.36368330525389e-06	\\
2108.29634232954	7.81298476476947e-06	\\
2109.27512428977	1.33693599086108e-05	\\
2110.25390625	3.82142073224439e-06	\\
2111.23268821023	1.32719729783374e-05	\\
2112.21147017045	5.15490090615047e-06	\\
2113.19025213068	9.45792982119867e-06	\\
2114.16903409091	9.87558268755534e-06	\\
2115.14781605114	6.12278192989239e-06	\\
2116.12659801136	5.09675120254464e-06	\\
2117.10537997159	1.11229992688978e-05	\\
2118.08416193182	8.50131263859708e-06	\\
2119.06294389204	9.86179598042668e-06	\\
2120.04172585227	5.6260730017591e-06	\\
2121.0205078125	3.89828173202376e-06	\\
2121.99928977273	8.58533960126685e-06	\\
2122.97807173295	7.23088514688085e-06	\\
2123.95685369318	1.61277789007024e-05	\\
2124.93563565341	1.31184523912336e-05	\\
2125.91441761364	1.24514743848341e-05	\\
2126.89319957386	5.85969877235296e-06	\\
2127.87198153409	9.86371201931969e-06	\\
2128.85076349432	1.00125298086225e-05	\\
2129.82954545454	6.09356295308642e-06	\\
2130.80832741477	6.82127703326502e-06	\\
2131.787109375	7.54151363781225e-06	\\
2132.76589133523	8.46427905404173e-06	\\
2133.74467329545	1.00852606631735e-05	\\
2134.72345525568	1.44290677082355e-05	\\
2135.70223721591	8.95812157256428e-06	\\
2136.68101917614	6.62361750213522e-06	\\
2137.65980113636	8.48864510146427e-06	\\
2138.63858309659	7.90045380679537e-06	\\
2139.61736505682	1.22391027682435e-05	\\
2140.59614701704	1.07162225553464e-05	\\
2141.57492897727	1.07668032052318e-05	\\
2142.5537109375	6.25773068989781e-06	\\
2143.53249289773	5.51900507560249e-06	\\
2144.51127485795	8.91375939521211e-06	\\
2145.49005681818	1.58450348882851e-05	\\
2146.46883877841	1.21085664775934e-05	\\
2147.44762073864	9.23312041713968e-06	\\
2148.42640269886	8.04168359837703e-06	\\
2149.40518465909	5.11412429452319e-06	\\
2150.38396661932	7.52528154116722e-06	\\
2151.36274857954	1.29513373262479e-05	\\
2152.34153053977	1.29849853126034e-05	\\
2153.3203125	1.24103871141505e-05	\\
2154.29909446023	9.58901887238305e-06	\\
2155.27787642045	6.45131786655435e-06	\\
2156.25665838068	6.69200280830997e-06	\\
2157.23544034091	1.12424705682557e-05	\\
2158.21422230114	1.03466049810562e-05	\\
2159.19300426136	1.16279290206983e-05	\\
2160.17178622159	9.31544117762571e-06	\\
2161.15056818182	6.70687310419462e-06	\\
2162.12935014204	9.49118880501068e-06	\\
2163.10813210227	8.58024515432925e-06	\\
2164.0869140625	5.928692127035e-06	\\
2165.06569602273	4.26308159397868e-06	\\
2166.04447798295	9.88365683601138e-06	\\
2167.02325994318	5.14371693214004e-06	\\
2168.00204190341	5.2759708112585e-06	\\
2168.98082386364	7.59664019825025e-06	\\
2169.95960582386	9.41941655972841e-06	\\
2170.93838778409	1.06226791068985e-05	\\
2171.91716974432	8.14265516629599e-06	\\
2172.89595170454	3.50012882418382e-06	\\
2173.87473366477	1.10225994121293e-05	\\
2174.853515625	1.15236088085572e-05	\\
2175.83229758523	1.00684862989118e-05	\\
2176.81107954545	2.86409997739697e-06	\\
2177.78986150568	1.00958158531724e-05	\\
2178.76864346591	8.52150824757419e-06	\\
2179.74742542614	7.16719007021488e-06	\\
2180.72620738636	1.80574333519365e-05	\\
2181.70498934659	9.12559784271053e-06	\\
2182.68377130682	4.36054413658582e-06	\\
2183.66255326704	9.55968809962669e-06	\\
2184.64133522727	3.30770171045642e-06	\\
2185.6201171875	1.09470380861247e-05	\\
2186.59889914773	9.97848338872294e-06	\\
2187.57768110795	3.29389949586036e-06	\\
2188.55646306818	6.68180568754173e-06	\\
2189.53524502841	1.02269105590572e-05	\\
2190.51402698864	8.90999454240555e-06	\\
2191.49280894886	7.44818402275269e-06	\\
2192.47159090909	9.39442715141743e-06	\\
2193.45037286932	4.9105319684144e-06	\\
2194.42915482954	3.80303057421436e-06	\\
2195.40793678977	8.75637975664492e-06	\\
2196.38671875	4.50437779117457e-06	\\
2197.36550071023	6.97225838653162e-06	\\
2198.34428267045	1.31483343527775e-05	\\
2199.32306463068	7.39768244506249e-06	\\
2200.30184659091	2.55066718187661e-06	\\
2201.28062855114	4.65205395753544e-06	\\
2202.25941051136	6.68588680324955e-06	\\
2203.23819247159	4.62775324084444e-06	\\
2204.21697443182	7.30701169993781e-06	\\
2205.19575639204	7.75693494658822e-06	\\
2206.17453835227	6.02073156030347e-06	\\
2207.1533203125	7.25881024914961e-06	\\
2208.13210227273	7.7047428904389e-06	\\
2209.11088423295	5.60294569565107e-06	\\
2210.08966619318	1.14332562761587e-06	\\
2211.06844815341	3.88572859500735e-06	\\
2212.04723011364	4.95783273183574e-06	\\
2213.02601207386	6.41220109308028e-06	\\
2214.00479403409	6.4928298133582e-06	\\
2214.98357599432	3.89898508280812e-06	\\
2215.96235795454	4.22643883432139e-06	\\
2216.94113991477	3.12433846156002e-06	\\
2217.919921875	4.90063835220926e-06	\\
2218.89870383523	4.39008435920138e-06	\\
2219.87748579545	7.97325052778663e-06	\\
2220.85626775568	4.95175989694408e-06	\\
2221.83504971591	3.65653894282158e-06	\\
2222.81383167614	8.76813242881283e-06	\\
2223.79261363636	6.30006723872493e-06	\\
2224.77139559659	4.22521317779544e-06	\\
2225.75017755682	8.05263448487694e-06	\\
2226.72895951704	4.60388793033797e-06	\\
2227.70774147727	1.08460761612662e-05	\\
2228.6865234375	6.61062645859292e-06	\\
2229.66530539773	5.09222297583826e-06	\\
2230.64408735795	6.09663105878553e-06	\\
2231.62286931818	6.60224953992934e-06	\\
2232.60165127841	6.96309458741855e-06	\\
2233.58043323864	7.63508897225743e-06	\\
2234.55921519886	5.90362968189001e-06	\\
2235.53799715909	8.24370324708715e-06	\\
2236.51677911932	4.06209571250789e-06	\\
2237.49556107954	6.06893477622947e-06	\\
2238.47434303977	1.15471672034157e-05	\\
2239.453125	6.53462009691684e-06	\\
2240.43190696023	9.19355062289341e-06	\\
2241.41068892045	1.14712947316337e-05	\\
2242.38947088068	6.0323203574983e-06	\\
2243.36825284091	1.03770737800806e-05	\\
2244.34703480114	9.50585443834524e-06	\\
2245.32581676136	1.19131725540486e-05	\\
2246.30459872159	9.72097860730166e-06	\\
2247.28338068182	1.09636116435671e-05	\\
2248.26216264204	1.27009600192486e-05	\\
2249.24094460227	8.62109753191805e-06	\\
2250.2197265625	1.40735342168786e-05	\\
2251.19850852273	1.10277002456324e-05	\\
2252.17729048295	9.23540836109104e-06	\\
2253.15607244318	9.97612544511641e-06	\\
2254.13485440341	8.22082232964495e-06	\\
2255.11363636364	1.09629494142538e-05	\\
2256.09241832386	1.05312356224001e-05	\\
2257.07120028409	1.12490504368562e-05	\\
2258.04998224432	7.66434257850735e-06	\\
2259.02876420454	1.17821411470889e-05	\\
2260.00754616477	1.06254739590319e-05	\\
2260.986328125	1.06139729788509e-05	\\
2261.96511008523	1.19853034481267e-05	\\
2262.94389204545	1.44391616700878e-05	\\
2263.92267400568	1.29344752331205e-05	\\
2264.90145596591	8.51336215768043e-06	\\
2265.88023792614	9.45547436182029e-06	\\
2266.85901988636	1.23324463797453e-05	\\
2267.83780184659	9.67386014599937e-06	\\
2268.81658380682	1.11423413156724e-05	\\
2269.79536576704	9.17304839983014e-06	\\
2270.77414772727	1.08912807384184e-05	\\
2271.7529296875	1.13393032047157e-05	\\
2272.73171164773	1.12107429068589e-05	\\
2273.71049360795	1.16687316288638e-05	\\
2274.68927556818	1.38634462855256e-05	\\
2275.66805752841	9.2265533789135e-06	\\
2276.64683948864	1.0281361953896e-05	\\
2277.62562144886	1.28880604923445e-05	\\
2278.60440340909	1.53504783137568e-05	\\
2279.58318536932	1.02153560639429e-05	\\
2280.56196732954	1.92945058644749e-05	\\
2281.54074928977	1.1816091676574e-05	\\
2282.51953125	1.1762892655996e-05	\\
2283.49831321023	1.57104856838896e-05	\\
2284.47709517045	1.25074353107287e-05	\\
2285.45587713068	1.49810026018697e-05	\\
2286.43465909091	1.52578042035758e-05	\\
2287.41344105114	1.78056991875551e-05	\\
2288.39222301136	1.33830874358607e-05	\\
2289.37100497159	1.0890933192144e-05	\\
2290.34978693182	1.64365438941813e-05	\\
2291.32856889204	1.44392279060748e-05	\\
2292.30735085227	1.42088044181936e-05	\\
2293.2861328125	1.36696517228093e-05	\\
2294.26491477273	1.39486468726326e-05	\\
2295.24369673295	1.40261570818371e-05	\\
2296.22247869318	1.54598376392559e-05	\\
2297.20126065341	1.38846594763373e-05	\\
2298.18004261364	1.558757644892e-05	\\
2299.15882457386	1.20516437766496e-05	\\
2300.13760653409	1.69807809917693e-05	\\
2301.11638849432	1.24677578835774e-05	\\
2302.09517045454	1.16544237338649e-05	\\
2303.07395241477	1.75169522215065e-05	\\
2304.052734375	1.38730352031501e-05	\\
2305.03151633523	1.16857625951727e-05	\\
2306.01029829545	1.14306336807305e-05	\\
2306.98908025568	1.34991714775306e-05	\\
2307.96786221591	1.09916937477769e-05	\\
2308.94664417614	1.01647242397532e-05	\\
2309.92542613636	1.35771466211046e-05	\\
2310.90420809659	1.08396974047936e-05	\\
2311.88299005682	1.02979956533513e-05	\\
2312.86177201704	1.05534782863396e-05	\\
2313.84055397727	8.99291604544703e-06	\\
2314.8193359375	1.06367200056811e-05	\\
2315.79811789773	1.1921305409653e-05	\\
2316.77689985795	1.09793662599682e-05	\\
2317.75568181818	1.08498181926699e-05	\\
2318.73446377841	9.49631464391728e-06	\\
2319.71324573864	1.03148595677466e-05	\\
2320.69202769886	1.19493346207555e-05	\\
2321.67080965909	7.16441155936494e-06	\\
2322.64959161932	1.14550913738169e-05	\\
2323.62837357954	9.97550675853848e-06	\\
2324.60715553977	1.0239631740076e-05	\\
2325.5859375	1.0054443085818e-05	\\
2326.56471946023	1.05087943445844e-05	\\
2327.54350142045	6.15521603792809e-06	\\
2328.52228338068	1.19353088715187e-05	\\
2329.50106534091	1.19898950811809e-05	\\
2330.47984730114	1.05280978736704e-05	\\
2331.45862926136	8.0906352996333e-06	\\
2332.43741122159	1.55246880398417e-05	\\
2333.41619318182	9.85381546339629e-06	\\
2334.39497514204	1.26029884559675e-05	\\
2335.37375710227	1.61563392445596e-05	\\
2336.3525390625	1.17814224888857e-05	\\
2337.33132102273	1.19181637415286e-05	\\
2338.31010298295	1.52677753053666e-05	\\
2339.28888494318	1.03801408484007e-05	\\
2340.26766690341	1.51406864085172e-05	\\
2341.24644886364	1.56655104616389e-05	\\
2342.22523082386	8.5259705756403e-06	\\
2343.20401278409	1.65536043675439e-05	\\
2344.18279474432	1.64394304304421e-05	\\
2345.16157670454	1.06402594861197e-05	\\
2346.14035866477	1.09246166404823e-05	\\
2347.119140625	1.29704983667432e-05	\\
2348.09792258523	1.13811465603336e-05	\\
2349.07670454545	1.31822755949023e-05	\\
2350.05548650568	1.87022827318072e-05	\\
2351.03426846591	9.46069460473123e-06	\\
2352.01305042614	1.11075953220586e-05	\\
2352.99183238636	1.72939738612485e-05	\\
2353.97061434659	1.19515985194932e-05	\\
2354.94939630682	1.39265782939805e-05	\\
2355.92817826704	1.52930209867912e-05	\\
2356.90696022727	1.11481738998377e-05	\\
2357.8857421875	1.27049430734602e-05	\\
2358.86452414773	9.07744697881029e-06	\\
2359.84330610795	1.15438924253775e-05	\\
2360.82208806818	1.06562167964211e-05	\\
2361.80087002841	1.23758546492067e-05	\\
2362.77965198864	1.31048421121442e-05	\\
2363.75843394886	1.14348667979891e-05	\\
2364.73721590909	1.2622802560616e-05	\\
2365.71599786932	9.76067333604412e-06	\\
2366.69477982954	9.55301396656965e-06	\\
2367.67356178977	1.19333418626326e-05	\\
2368.65234375	8.52594685351857e-06	\\
2369.63112571023	7.14851979089286e-06	\\
2370.60990767045	1.13337276612157e-05	\\
2371.58868963068	1.05741110134304e-05	\\
2372.56747159091	9.59406555279268e-06	\\
2373.54625355114	6.90753420694246e-06	\\
2374.52503551136	1.11811697197271e-05	\\
2375.50381747159	7.16009935968107e-06	\\
2376.48259943182	1.01012026951165e-05	\\
2377.46138139204	1.13867327662863e-05	\\
2378.44016335227	9.65377268042556e-06	\\
2379.4189453125	9.68136983306031e-06	\\
2380.39772727273	8.79600385537138e-06	\\
2381.37650923295	8.22697537600947e-06	\\
2382.35529119318	8.41217933850615e-06	\\
2383.33407315341	1.04932374963135e-05	\\
2384.31285511364	1.11940922186062e-05	\\
2385.29163707386	1.04029311164239e-05	\\
2386.27041903409	9.15098141480253e-06	\\
2387.24920099432	1.18500193168334e-05	\\
2388.22798295454	1.2900899594381e-05	\\
2389.20676491477	1.18071724302909e-05	\\
2390.185546875	1.11518890778064e-05	\\
2391.16432883523	1.05475931696297e-05	\\
2392.14311079545	1.31929963249887e-05	\\
2393.12189275568	1.01265523008739e-05	\\
2394.10067471591	1.18543208739252e-05	\\
2395.07945667614	1.07089221484535e-05	\\
2396.05823863636	1.34794837314849e-05	\\
2397.03702059659	1.17583488758016e-05	\\
2398.01580255682	1.3405314544094e-05	\\
2398.99458451704	1.33815020146956e-05	\\
2399.97336647727	1.15406661960588e-05	\\
2400.9521484375	1.15972672045454e-05	\\
2401.93093039773	1.11761306698093e-05	\\
2402.90971235795	1.32368545734626e-05	\\
2403.88849431818	1.53232785997131e-05	\\
2404.86727627841	1.30339417285951e-05	\\
2405.84605823864	1.60341006385682e-05	\\
2406.82484019886	1.25366098174034e-05	\\
2407.80362215909	1.13216990552699e-05	\\
2408.78240411932	1.21126659962943e-05	\\
2409.76118607954	1.38248866373234e-05	\\
2410.73996803977	1.29100671309624e-05	\\
2411.71875	1.0674730118919e-05	\\
2412.69753196023	1.06979606514353e-05	\\
2413.67631392045	9.96677116493869e-06	\\
2414.65509588068	1.12714092780936e-05	\\
2415.63387784091	1.43341497486834e-05	\\
2416.61265980114	1.20873787276672e-05	\\
2417.59144176136	1.25086257472886e-05	\\
2418.57022372159	1.28104623689339e-05	\\
2419.54900568182	1.50222321779968e-05	\\
2420.52778764204	7.00633643185449e-06	\\
2421.50656960227	1.31925207863567e-05	\\
2422.4853515625	8.83835131329433e-06	\\
2423.46413352273	1.20920822456478e-05	\\
2424.44291548295	1.31321428684912e-05	\\
2425.42169744318	1.05310684632227e-05	\\
2426.40047940341	1.210040790732e-05	\\
2427.37926136364	1.17520720362205e-05	\\
2428.35804332386	1.01203793508877e-05	\\
2429.33682528409	1.14409187342529e-05	\\
2430.31560724432	1.32795262760411e-05	\\
2431.29438920454	1.51541863818808e-05	\\
2432.27317116477	1.45130064592897e-05	\\
2433.251953125	1.04935851784015e-05	\\
2434.23073508523	1.40022049188288e-05	\\
2435.20951704545	1.46165291608628e-05	\\
2436.18829900568	1.42125100692022e-05	\\
2437.16708096591	1.4586181047714e-05	\\
2438.14586292614	1.90804867380124e-05	\\
2439.12464488636	1.25183623596444e-05	\\
2440.10342684659	1.59353532551997e-05	\\
2441.08220880682	1.48879304313828e-05	\\
2442.06099076704	1.36705843624822e-05	\\
2443.03977272727	1.30327660954118e-05	\\
2444.0185546875	1.70730294238341e-05	\\
2444.99733664773	1.0912489210281e-05	\\
2445.97611860795	1.70652671880259e-05	\\
2446.95490056818	1.12547014444918e-05	\\
2447.93368252841	1.46722351080737e-05	\\
2448.91246448864	1.6062372435979e-05	\\
2449.89124644886	1.6719173156617e-05	\\
2450.87002840909	1.61231556439175e-05	\\
2451.84881036932	1.3720061605075e-05	\\
2452.82759232954	1.48605433605499e-05	\\
2453.80637428977	1.39850108329484e-05	\\
2454.78515625	1.34046428029915e-05	\\
2455.76393821023	1.37677789210611e-05	\\
2456.74272017045	9.00213167101684e-06	\\
2457.72150213068	9.32539439331602e-06	\\
2458.70028409091	8.98747441483013e-06	\\
2459.67906605114	1.55645163361909e-05	\\
2460.65784801136	8.70161457290056e-06	\\
2461.63662997159	1.33824512684582e-05	\\
2462.61541193182	1.16969945313014e-05	\\
2463.59419389204	1.21556947848075e-05	\\
2464.57297585227	1.17634986731376e-05	\\
2465.5517578125	1.00800621711629e-05	\\
2466.53053977273	1.30681827758706e-05	\\
2467.50932173295	1.38842205787105e-05	\\
2468.48810369318	8.14327497826777e-06	\\
2469.46688565341	1.12075605344456e-05	\\
2470.44566761364	1.18802341328419e-05	\\
2471.42444957386	9.94744716261625e-06	\\
2472.40323153409	1.23944828718623e-05	\\
2473.38201349432	1.17036668864691e-05	\\
2474.36079545454	8.38221612499108e-06	\\
2475.33957741477	1.308364749912e-05	\\
2476.318359375	1.11017399028128e-05	\\
2477.29714133523	1.37772888744459e-05	\\
2478.27592329545	1.02220240638125e-05	\\
2479.25470525568	1.02954347512395e-05	\\
2480.23348721591	1.50070917796501e-05	\\
2481.21226917614	1.07744491572926e-05	\\
2482.19105113636	1.20392357919938e-05	\\
2483.16983309659	1.14551401863731e-05	\\
2484.14861505682	1.33330862992937e-05	\\
2485.12739701704	1.09197161563871e-05	\\
2486.10617897727	1.38397577004886e-05	\\
2487.0849609375	9.64976796759365e-06	\\
2488.06374289773	1.37465082197053e-05	\\
2489.04252485795	1.27953296217664e-05	\\
2490.02130681818	1.23030040593245e-05	\\
2491.00008877841	1.3220007479457e-05	\\
2491.97887073864	1.46182295581329e-05	\\
2492.95765269886	1.28843092311765e-05	\\
2493.93643465909	1.75987049032965e-05	\\
2494.91521661932	1.28898515351656e-05	\\
2495.89399857954	1.19387758801284e-05	\\
2496.87278053977	1.17522134706582e-05	\\
2497.8515625	1.0444532769733e-05	\\
2498.83034446023	1.10480930740472e-05	\\
2499.80912642045	1.23423961563045e-05	\\
2500.78790838068	1.60830096240532e-05	\\
2501.76669034091	1.05846449279785e-05	\\
2502.74547230114	1.88065597316314e-05	\\
2503.72425426136	9.92312637693691e-06	\\
2504.70303622159	1.42638784803625e-05	\\
2505.68181818182	1.67750002510302e-05	\\
2506.66060014204	1.65982949149232e-05	\\
2507.63938210227	1.70037117257888e-05	\\
2508.6181640625	1.63605658840124e-05	\\
2509.59694602273	1.05248554120014e-05	\\
2510.57572798295	1.46821389791948e-05	\\
2511.55450994318	1.34724288620022e-05	\\
2512.53329190341	1.23611407461894e-05	\\
2513.51207386364	1.24551858129238e-05	\\
2514.49085582386	1.56631044305564e-05	\\
2515.46963778409	1.10863219203196e-05	\\
2516.44841974432	1.90273789401122e-05	\\
2517.42720170454	1.37903021093818e-05	\\
2518.40598366477	1.3925586576509e-05	\\
2519.384765625	1.522269411294e-05	\\
2520.36354758523	1.00100452374513e-05	\\
2521.34232954545	1.36576120359163e-05	\\
2522.32111150568	1.05835599204902e-05	\\
2523.29989346591	1.2295125625366e-05	\\
2524.27867542614	1.21591197194834e-05	\\
2525.25745738636	1.400433818524e-05	\\
2526.23623934659	1.58419435920574e-05	\\
2527.21502130682	1.14341198801488e-05	\\
2528.19380326704	9.65742098643429e-06	\\
2529.17258522727	1.23915070475643e-05	\\
2530.1513671875	7.62654880391048e-06	\\
2531.13014914773	1.28935970238767e-05	\\
2532.10893110795	1.13737414641285e-05	\\
2533.08771306818	1.05227492732501e-05	\\
2534.06649502841	9.60237978652877e-06	\\
2535.04527698864	1.02845563306765e-05	\\
2536.02405894886	1.0513328742645e-05	\\
2537.00284090909	1.16139737771759e-05	\\
2537.98162286932	1.31451801263953e-05	\\
2538.96040482954	1.13465476893299e-05	\\
2539.93918678977	1.23630684081609e-05	\\
2540.91796875	1.07720467294631e-05	\\
2541.89675071023	1.30828713388349e-05	\\
2542.87553267045	1.2687587879665e-05	\\
2543.85431463068	9.5786792017844e-06	\\
2544.83309659091	1.36779978049829e-05	\\
2545.81187855114	9.26662352511788e-06	\\
2546.79066051136	9.61255056679938e-06	\\
2547.76944247159	1.13461533904966e-05	\\
2548.74822443182	1.25775352959022e-05	\\
2549.72700639204	1.20569360935071e-05	\\
2550.70578835227	9.70804790209553e-06	\\
2551.6845703125	9.5751769541088e-06	\\
2552.66335227273	9.6403291415825e-06	\\
2553.64213423295	9.18332414808979e-06	\\
2554.62091619318	8.77869837235778e-06	\\
2555.59969815341	1.30016508203182e-05	\\
2556.57848011364	1.3134185276538e-05	\\
2557.55726207386	9.81679793928961e-06	\\
2558.53604403409	9.19258307718049e-06	\\
2559.51482599432	1.06200456341965e-05	\\
2560.49360795454	1.124413739682e-05	\\
2561.47238991477	1.30851966964052e-05	\\
2562.451171875	1.4567312498378e-05	\\
2563.42995383523	1.42429325036642e-05	\\
2564.40873579545	8.53230731649721e-06	\\
2565.38751775568	1.02901656298179e-05	\\
2566.36629971591	7.84201601104799e-06	\\
2567.34508167614	1.25461450286407e-05	\\
2568.32386363636	1.35157686585148e-05	\\
2569.30264559659	1.13553096664194e-05	\\
2570.28142755682	1.3503916523033e-05	\\
2571.26020951704	9.46869165727952e-06	\\
2572.23899147727	8.02095011434817e-06	\\
2573.2177734375	1.16375271478506e-05	\\
2574.19655539773	1.10594948924882e-05	\\
2575.17533735795	1.33487136327705e-05	\\
2576.15411931818	1.25254865263412e-05	\\
2577.13290127841	1.2523850446189e-05	\\
2578.11168323864	1.04755127513138e-05	\\
2579.09046519886	8.11516907886775e-06	\\
2580.06924715909	1.1036659948732e-05	\\
2581.04802911932	1.03887793227839e-05	\\
2582.02681107954	1.17902823165449e-05	\\
2583.00559303977	1.12858115885358e-05	\\
2583.984375	7.74161008673369e-06	\\
2584.96315696023	8.97775252252712e-06	\\
2585.94193892045	1.07253245520665e-05	\\
2586.92072088068	9.39603793213243e-06	\\
2587.89950284091	1.20822610193775e-05	\\
2588.87828480114	6.52389355979953e-06	\\
2589.85706676136	9.3433123586389e-06	\\
2590.83584872159	1.07332580761661e-05	\\
2591.81463068182	9.33603217192033e-06	\\
2592.79341264204	1.06534342956051e-05	\\
2593.77219460227	5.52954525945763e-06	\\
2594.7509765625	5.86504377967524e-06	\\
2595.72975852273	6.08157206795503e-06	\\
2596.70854048295	5.61574830801038e-06	\\
2597.68732244318	7.77485508207652e-06	\\
2598.66610440341	6.15571980497798e-06	\\
2599.64488636364	6.80218617791554e-06	\\
2600.62366832386	8.65338842041828e-06	\\
2601.60245028409	9.34452172307638e-06	\\
2602.58123224432	1.1033075454989e-05	\\
2603.56001420454	8.62456727506564e-06	\\
2604.53879616477	2.88538297795258e-06	\\
2605.517578125	4.17952480942147e-06	\\
2606.49636008523	3.83672692841114e-06	\\
2607.47514204545	7.19461446884648e-06	\\
2608.45392400568	6.31111488540669e-06	\\
2609.43270596591	8.58195842274819e-06	\\
2610.41148792614	6.8863595902438e-06	\\
2611.39026988636	4.65573351810625e-06	\\
2612.36905184659	5.32665585678252e-06	\\
2613.34783380682	8.45364341266012e-06	\\
2614.32661576704	8.15700064552104e-06	\\
2615.30539772727	7.22588366185077e-06	\\
2616.2841796875	1.05143713663192e-05	\\
2617.26296164773	5.47532164138058e-06	\\
2618.24174360795	4.85198223905322e-06	\\
2619.22052556818	9.23131635188674e-06	\\
2620.19930752841	3.75817220000155e-06	\\
2621.17808948864	7.95875299463989e-06	\\
2622.15687144886	9.2233715115999e-06	\\
2623.13565340909	7.05004767081643e-06	\\
2624.11443536932	9.00450995499237e-06	\\
2625.09321732954	9.54503977001142e-06	\\
2626.07199928977	7.45014755649668e-06	\\
2627.05078125	4.62344119762142e-06	\\
2628.02956321023	7.34974537688955e-06	\\
2629.00834517045	7.37743733290187e-06	\\
2629.98712713068	5.88418890548162e-06	\\
2630.96590909091	1.11178167545335e-05	\\
2631.94469105114	1.00665025509934e-05	\\
2632.92347301136	1.25483221312679e-05	\\
2633.90225497159	7.13547865450581e-06	\\
2634.88103693182	6.37583961681608e-06	\\
2635.85981889204	8.3845175803049e-06	\\
2636.83860085227	7.58726662456945e-06	\\
2637.8173828125	1.27409486452016e-05	\\
2638.79616477273	1.01727392999639e-05	\\
2639.77494673295	1.36101262168017e-05	\\
2640.75372869318	8.53381174503931e-06	\\
2641.73251065341	9.55873463304659e-06	\\
2642.71129261364	6.82545181407289e-06	\\
2643.69007457386	7.72724030090873e-06	\\
2644.66885653409	8.24328732263467e-06	\\
2645.64763849432	1.1189907202075e-05	\\
2646.62642045454	6.99134210342693e-06	\\
2647.60520241477	9.92637314508742e-06	\\
2648.583984375	9.03832709681233e-06	\\
2649.56276633523	7.46600078214287e-06	\\
2650.54154829545	7.60204227025801e-06	\\
2651.52033025568	7.49209075202238e-06	\\
2652.49911221591	6.80423806093495e-06	\\
2653.47789417614	6.90048903306781e-06	\\
2654.45667613636	5.37920453335535e-06	\\
2655.43545809659	8.8264274115568e-06	\\
2656.41424005682	1.01195304852176e-05	\\
2657.39302201704	8.12528517946348e-06	\\
2658.37180397727	9.42011054122724e-06	\\
2659.3505859375	9.00279402269212e-06	\\
2660.32936789773	7.33246871490005e-06	\\
2661.30814985795	8.82258248865961e-06	\\
2662.28693181818	7.21071771030407e-06	\\
2663.26571377841	9.14232366385177e-06	\\
2664.24449573864	7.28067630878025e-06	\\
2665.22327769886	6.5140299586603e-06	\\
2666.20205965909	3.56937233047156e-06	\\
2667.18084161932	8.6265697880826e-06	\\
2668.15962357954	9.01703957702175e-06	\\
2669.13840553977	1.16768454356727e-05	\\
2670.1171875	1.00752855817841e-05	\\
2671.09596946023	9.23235812526987e-06	\\
2672.07475142045	8.44909776478114e-06	\\
2673.05353338068	1.29141189350765e-05	\\
2674.03231534091	9.25553915933191e-06	\\
2675.01109730114	5.90860212740751e-06	\\
2675.98987926136	9.66984964070978e-06	\\
2676.96866122159	7.13892848178111e-06	\\
2677.94744318182	1.04967048800068e-05	\\
2678.92622514204	7.52701102599433e-06	\\
2679.90500710227	9.97781631097798e-06	\\
2680.8837890625	7.74775665210904e-06	\\
2681.86257102273	1.06591803110307e-05	\\
2682.84135298295	8.50379433119764e-06	\\
2683.82013494318	5.47135369597753e-06	\\
2684.79891690341	6.49056178258718e-06	\\
2685.77769886364	7.74183037863941e-06	\\
2686.75648082386	6.40124384284549e-06	\\
2687.73526278409	9.22428299084042e-06	\\
2688.71404474432	6.9358071595968e-06	\\
2689.69282670454	6.98093819993718e-06	\\
2690.67160866477	7.69409706866232e-06	\\
2691.650390625	4.96089894590821e-06	\\
2692.62917258523	8.35908202828073e-06	\\
2693.60795454545	7.82209947606149e-06	\\
2694.58673650568	7.56896044745794e-06	\\
2695.56551846591	8.77499759143943e-06	\\
2696.54430042614	2.72518386293727e-06	\\
2697.52308238636	5.23039346413709e-06	\\
2698.50186434659	4.37319468335196e-06	\\
2699.48064630682	5.92164507868871e-06	\\
2700.45942826704	9.80048438810412e-06	\\
2701.43821022727	6.06485397629639e-06	\\
2702.4169921875	6.61823282256381e-06	\\
2703.39577414773	5.40780451245773e-06	\\
2704.37455610795	8.05266045970128e-06	\\
2705.35333806818	6.12401990553033e-06	\\
2706.33212002841	7.94040389189657e-06	\\
2707.31090198864	4.00703277020468e-06	\\
2708.28968394886	5.7130239051283e-06	\\
2709.26846590909	6.8379980598573e-06	\\
2710.24724786932	5.32101088562841e-06	\\
2711.22602982954	7.030740942015e-06	\\
2712.20481178977	8.67067257322456e-06	\\
2713.18359375	5.2718117078937e-06	\\
2714.16237571023	6.56174353791964e-06	\\
2715.14115767045	5.20732800295995e-06	\\
2716.11993963068	8.6943346222023e-06	\\
2717.09872159091	8.04541703143695e-06	\\
2718.07750355114	1.03720489809125e-05	\\
2719.05628551136	5.07868752460706e-06	\\
2720.03506747159	1.09937604651641e-05	\\
2721.01384943182	9.15768973332119e-06	\\
2721.99263139204	7.91249333647524e-06	\\
2722.97141335227	3.95413675455279e-06	\\
2723.9501953125	6.30520226203253e-06	\\
2724.92897727273	6.09635418088786e-06	\\
2725.90775923295	7.49361844763953e-06	\\
2726.88654119318	8.00386085814599e-06	\\
2727.86532315341	6.91220750456317e-06	\\
2728.84410511364	2.02716576113554e-06	\\
2729.82288707386	1.83193723303623e-06	\\
2730.80166903409	7.86563897858116e-06	\\
2731.78045099432	7.93584617578885e-06	\\
2732.75923295454	6.22400194055147e-06	\\
2733.73801491477	4.69601498828271e-06	\\
2734.716796875	3.66961420925272e-06	\\
2735.69557883523	3.94321059626331e-06	\\
2736.67436079545	4.89833366769579e-06	\\
2737.65314275568	7.49695584247444e-06	\\
2738.63192471591	5.25993073408675e-06	\\
2739.61070667614	2.72797881208903e-06	\\
2740.58948863636	5.82802545830845e-06	\\
2741.56827059659	6.24178338298713e-06	\\
2742.54705255682	5.69526340498197e-06	\\
2743.52583451704	4.18711277068443e-06	\\
2744.50461647727	8.24480572213219e-07	\\
2745.4833984375	4.62905209203989e-06	\\
2746.46218039773	6.8773818785314e-06	\\
2747.44096235795	6.56540359293604e-06	\\
2748.41974431818	3.53827328462157e-06	\\
2749.39852627841	2.64957686461998e-06	\\
2750.37730823864	1.68329149694193e-06	\\
2751.35609019886	2.41840681220149e-06	\\
2752.33487215909	4.37866826482226e-06	\\
2753.31365411932	6.56936858777892e-06	\\
2754.29243607954	2.07591322996028e-06	\\
2755.27121803977	2.91489806344309e-06	\\
2756.25	2.27665449064888e-06	\\
2757.22878196023	2.78272137545811e-06	\\
2758.20756392045	3.7857086885082e-06	\\
2759.18634588068	6.48150705117231e-06	\\
2760.16512784091	3.27505259061264e-06	\\
2761.14390980114	2.14809136085652e-06	\\
2762.12269176136	3.43420896592292e-06	\\
2763.10147372159	3.51653307309765e-06	\\
2764.08025568182	3.32816655179175e-06	\\
2765.05903764204	4.80991741704654e-06	\\
2766.03781960227	4.13455050868395e-06	\\
2767.0166015625	2.94343263591024e-06	\\
2767.99538352273	3.65889565349057e-06	\\
2768.97416548295	3.97987355237858e-06	\\
2769.95294744318	5.59565495064372e-07	\\
2770.93172940341	2.34776478652176e-06	\\
2771.91051136364	5.02453330557031e-07	\\
2772.88929332386	2.75425049358554e-06	\\
2773.86807528409	2.13724197103523e-06	\\
2774.84685724432	3.76267962968488e-06	\\
2775.82563920454	1.58823957386081e-06	\\
2776.80442116477	1.56503764002716e-06	\\
2777.783203125	4.81457742157517e-07	\\
2778.76198508523	2.41425873615509e-06	\\
2779.74076704545	3.7800327259833e-06	\\
2780.71954900568	7.48851336311492e-07	\\
2781.69833096591	1.30969552001911e-06	\\
2782.67711292614	2.50660289748446e-06	\\
2783.65589488636	1.47211144352771e-06	\\
2784.63467684659	1.3472694814085e-06	\\
2785.61345880682	1.19976750433772e-06	\\
2786.59224076704	1.20544273020633e-06	\\
2787.57102272727	1.30111118129765e-06	\\
2788.5498046875	3.24132264270955e-06	\\
2789.52858664773	5.50707647691607e-07	\\
2790.50736860795	3.5166520348458e-06	\\
2791.48615056818	1.52155058698765e-06	\\
2792.46493252841	1.50707857172838e-06	\\
2793.44371448864	1.63285502503068e-06	\\
2794.42249644886	2.62402206308109e-06	\\
2795.40127840909	2.13660905036377e-06	\\
2796.38006036932	1.36600481763574e-06	\\
2797.35884232954	3.09032463285475e-06	\\
2798.33762428977	2.47690142333158e-06	\\
2799.31640625	6.40393828085902e-07	\\
2800.29518821023	1.66620154450953e-06	\\
2801.27397017045	1.90590019104556e-06	\\
2802.25275213068	2.62394854801383e-06	\\
2803.23153409091	2.10922820488248e-06	\\
2804.21031605114	8.93601172735078e-07	\\
2805.18909801136	1.86273418190857e-06	\\
2806.16787997159	1.46039807683165e-06	\\
2807.14666193182	1.65402393974081e-06	\\
2808.12544389204	2.38604592423503e-06	\\
2809.10422585227	2.73401986561309e-06	\\
2810.0830078125	3.35927737466208e-06	\\
2811.06178977273	1.40371762113629e-06	\\
2812.04057173295	3.41824661840372e-07	\\
2813.01935369318	1.11134853469604e-06	\\
2813.99813565341	1.47575865883504e-06	\\
2814.97691761364	1.77121001143617e-06	\\
2815.95569957386	6.52547567769548e-07	\\
2816.93448153409	1.42124681175411e-06	\\
2817.91326349432	1.79799230299238e-06	\\
2818.89204545454	3.09877272480753e-07	\\
2819.87082741477	3.39534979297913e-06	\\
2820.849609375	2.36146531941351e-06	\\
2821.82839133523	2.79233689233618e-06	\\
2822.80717329545	4.69076383249551e-06	\\
2823.78595525568	3.97817927657384e-06	\\
2824.76473721591	3.59827982273e-06	\\
2825.74351917614	3.60306567645143e-06	\\
2826.72230113636	4.27507175115144e-06	\\
2827.70108309659	3.37845963880854e-06	\\
2828.67986505682	5.1910715616594e-06	\\
2829.65864701704	3.81292278948782e-06	\\
2830.63742897727	4.04516392038333e-06	\\
2831.6162109375	5.34785859087092e-06	\\
2832.59499289773	3.46811766398455e-06	\\
2833.57377485795	4.61100630473622e-06	\\
2834.55255681818	1.46085813543219e-06	\\
2835.53133877841	4.41154723073322e-06	\\
2836.51012073864	3.37146832475975e-06	\\
2837.48890269886	4.80974071573523e-06	\\
2838.46768465909	3.3434869427996e-06	\\
2839.44646661932	4.4732839956391e-06	\\
2840.42524857955	3.31605206882318e-06	\\
2841.40403053977	2.86392063973937e-06	\\
2842.3828125	2.8693626330884e-06	\\
2843.36159446023	3.98644303517036e-06	\\
2844.34037642045	2.6698091836681e-06	\\
2845.31915838068	2.29416710915538e-06	\\
2846.29794034091	4.05553100846361e-06	\\
2847.27672230114	2.73498388587658e-06	\\
2848.25550426136	2.45475214621159e-06	\\
2849.23428622159	3.63521956374567e-06	\\
2850.21306818182	3.80876490912884e-06	\\
2851.19185014205	1.90131841291536e-06	\\
2852.17063210227	2.79732543971194e-06	\\
2853.1494140625	8.58395836151411e-07	\\
2854.12819602273	2.86318881054772e-06	\\
2855.10697798295	2.62618488771587e-06	\\
2856.08575994318	4.15647646013232e-06	\\
2857.06454190341	1.49010761969156e-06	\\
2858.04332386364	4.77157149405267e-06	\\
2859.02210582386	3.29516435970086e-06	\\
2860.00088778409	4.14739956473991e-06	\\
2860.97966974432	2.15715583078548e-06	\\
2861.95845170455	4.35052409132747e-06	\\
2862.93723366477	2.77105042165666e-06	\\
2863.916015625	5.59833466334693e-06	\\
2864.89479758523	1.65351596450654e-06	\\
2865.87357954545	8.25865559303358e-06	\\
2866.85236150568	6.04919430234132e-06	\\
2867.83114346591	3.34978274246577e-06	\\
2868.80992542614	2.79377730234606e-06	\\
2869.78870738636	5.39473416599955e-06	\\
2870.76748934659	3.47073919790119e-06	\\
2871.74627130682	4.99846922843471e-06	\\
2872.72505326705	3.05980155120563e-06	\\
2873.70383522727	3.70726635832435e-06	\\
2874.6826171875	4.77183524178971e-06	\\
2875.66139914773	5.28485361692664e-06	\\
2876.64018110795	2.9497489876962e-06	\\
2877.61896306818	2.28246609459498e-06	\\
2878.59774502841	3.87743066263443e-06	\\
2879.57652698864	2.05408395302597e-06	\\
2880.55530894886	8.03927698675367e-07	\\
2881.53409090909	2.44744225685431e-06	\\
2882.51287286932	2.04967306841465e-06	\\
2883.49165482955	2.67218524875853e-06	\\
2884.47043678977	2.69365196020493e-06	\\
2885.44921875	3.02902380829889e-06	\\
2886.42800071023	5.27067323555189e-06	\\
2887.40678267045	2.54106939512331e-06	\\
2888.38556463068	5.44631775694933e-06	\\
2889.36434659091	4.98920494662752e-06	\\
2890.34312855114	3.54364437954377e-06	\\
2891.32191051136	3.8671092843816e-06	\\
2892.30069247159	3.49259102268966e-06	\\
2893.27947443182	4.59441439193861e-06	\\
2894.25825639205	3.64553546186186e-06	\\
2895.23703835227	3.95532203589187e-06	\\
2896.2158203125	4.49474218965829e-06	\\
2897.19460227273	3.42832990708021e-06	\\
2898.17338423295	5.03838235510546e-06	\\
2899.15216619318	2.58397604649771e-06	\\
2900.13094815341	3.30605503038348e-06	\\
2901.10973011364	1.67821786853684e-06	\\
2902.08851207386	6.79298054295202e-06	\\
2903.06729403409	6.4361986385113e-06	\\
2904.04607599432	4.11882712930982e-06	\\
2905.02485795455	4.44937112285067e-06	\\
2906.00363991477	5.41452697692217e-06	\\
2906.982421875	7.62821366026552e-06	\\
2907.96120383523	4.34301164002499e-06	\\
2908.93998579545	4.38423101610561e-06	\\
2909.91876775568	5.44392904901348e-06	\\
2910.89754971591	2.94732468990434e-06	\\
2911.87633167614	3.56962565778673e-06	\\
2912.85511363636	3.86398810959201e-06	\\
2913.83389559659	5.41954042483807e-06	\\
2914.81267755682	4.17560744258949e-06	\\
2915.79145951705	5.23728218121983e-06	\\
2916.77024147727	6.06062596616099e-06	\\
2917.7490234375	5.29696196939571e-06	\\
2918.72780539773	2.92487026976474e-06	\\
2919.70658735795	4.41970124587756e-06	\\
2920.68536931818	1.97618330670144e-06	\\
2921.66415127841	2.00822218720313e-06	\\
2922.64293323864	5.48564533852651e-06	\\
2923.62171519886	4.78729973692923e-06	\\
2924.60049715909	4.89496171366531e-06	\\
2925.57927911932	4.23508907657349e-06	\\
2926.55806107955	5.17341349752293e-06	\\
2927.53684303977	4.3928195794944e-06	\\
2928.515625	4.18838468315718e-06	\\
2929.49440696023	5.46910199624583e-06	\\
2930.47318892045	3.16049207536663e-06	\\
2931.45197088068	3.91757781208955e-06	\\
2932.43075284091	3.52979009744975e-06	\\
2933.40953480114	3.28564659675688e-06	\\
2934.38831676136	3.30588604586103e-06	\\
2935.36709872159	7.48682290166127e-06	\\
2936.34588068182	5.10661221411358e-06	\\
2937.32466264205	5.09640106753759e-06	\\
2938.30344460227	3.99989255694006e-06	\\
2939.2822265625	4.15964384854581e-06	\\
2940.26100852273	2.45216038571283e-06	\\
2941.23979048295	3.84401733424272e-06	\\
2942.21857244318	4.94815518732998e-06	\\
2943.19735440341	3.11544554910849e-06	\\
2944.17613636364	5.46291311776296e-06	\\
2945.15491832386	6.66049527862821e-06	\\
2946.13370028409	3.28286142459796e-06	\\
2947.11248224432	5.58954472096685e-06	\\
2948.09126420455	5.11049428954381e-06	\\
2949.07004616477	5.48659209099511e-06	\\
2950.048828125	4.844544800359e-06	\\
2951.02761008523	3.05270818269617e-06	\\
2952.00639204545	4.52144252413908e-06	\\
2952.98517400568	3.8798952534642e-06	\\
2953.96395596591	2.70241878333915e-06	\\
2954.94273792614	2.50797806199041e-06	\\
2955.92151988636	3.29564801074687e-06	\\
2956.90030184659	3.31947403444169e-06	\\
2957.87908380682	5.22037402065988e-06	\\
2958.85786576705	4.72991655727233e-06	\\
2959.83664772727	4.68990286927971e-06	\\
2960.8154296875	4.37946934566222e-06	\\
2961.79421164773	4.31291612438827e-06	\\
2962.77299360795	4.78991878222869e-06	\\
2963.75177556818	3.44401737823946e-06	\\
2964.73055752841	3.85003314293153e-06	\\
2965.70933948864	6.01948733784033e-06	\\
2966.68812144886	5.52752037377181e-06	\\
2967.66690340909	4.96185575705528e-06	\\
2968.64568536932	4.92031879789632e-06	\\
2969.62446732955	5.59483004811013e-06	\\
2970.60324928977	4.48308426521495e-06	\\
2971.58203125	5.25940643803338e-06	\\
2972.56081321023	5.19828171191463e-06	\\
2973.53959517045	3.1910029726617e-06	\\
2974.51837713068	2.59331451040508e-06	\\
2975.49715909091	5.42688352851547e-06	\\
2976.47594105114	3.09955232241895e-06	\\
2977.45472301136	5.24876827769003e-06	\\
2978.43350497159	6.56275685653722e-06	\\
2979.41228693182	3.64103329898829e-06	\\
2980.39106889205	4.01169831846422e-06	\\
2981.36985085227	6.6624671804147e-06	\\
2982.3486328125	6.63718397978908e-06	\\
2983.32741477273	5.28662885067508e-06	\\
2984.30619673295	5.85557524334713e-06	\\
2985.28497869318	4.34361036684229e-06	\\
2986.26376065341	3.68583354080117e-06	\\
2987.24254261364	4.51312584341107e-06	\\
2988.22132457386	6.25624310488878e-06	\\
2989.20010653409	8.01787406254111e-06	\\
2990.17888849432	5.21034120045495e-06	\\
2991.15767045455	3.83309843970224e-06	\\
2992.13645241477	4.07656011686612e-06	\\
2993.115234375	4.17595292668247e-06	\\
2994.09401633523	5.14795531856298e-06	\\
2995.07279829545	3.60725257510452e-06	\\
2996.05158025568	3.83673608123943e-06	\\
2997.03036221591	7.24558250375745e-06	\\
2998.00914417614	4.17149542361306e-06	\\
2998.98792613636	2.7407189377701e-06	\\
2999.96670809659	6.40525675055424e-06	\\
3000.94549005682	2.23574794677416e-06	\\
3001.92427201705	6.85813192147649e-06	\\
3002.90305397727	3.46430751067959e-06	\\
3003.8818359375	3.55826281260897e-06	\\
3004.86061789773	5.90310017340944e-06	\\
3005.83939985795	5.57595289620855e-06	\\
3006.81818181818	6.26296518425085e-06	\\
3007.79696377841	3.74843769540633e-06	\\
3008.77574573864	6.57463491177955e-06	\\
3009.75452769886	3.71854966701996e-06	\\
3010.73330965909	3.34249448947731e-06	\\
3011.71209161932	4.79666728689112e-06	\\
3012.69087357955	4.14878870285875e-06	\\
3013.66965553977	3.95736566800053e-06	\\
3014.6484375	6.93497334844756e-06	\\
3015.62721946023	3.22051895849985e-06	\\
3016.60600142045	4.50677072741703e-06	\\
3017.58478338068	7.28427797428333e-06	\\
3018.56356534091	4.77031186486822e-06	\\
3019.54234730114	4.78746059693251e-06	\\
3020.52112926136	3.75980663615043e-06	\\
3021.49991122159	1.73369035447683e-06	\\
3022.47869318182	6.7511241383729e-06	\\
3023.45747514205	6.75424562872583e-06	\\
3024.43625710227	7.42581627653565e-06	\\
3025.4150390625	8.95968854541193e-06	\\
3026.39382102273	5.11842748206802e-06	\\
3027.37260298295	6.50356365308631e-06	\\
3028.35138494318	6.65704149755269e-06	\\
3029.33016690341	9.76890835918335e-06	\\
3030.30894886364	6.38204936454737e-06	\\
3031.28773082386	3.95786455990871e-06	\\
3032.26651278409	3.26478119336771e-06	\\
3033.24529474432	8.02517934212748e-06	\\
3034.22407670455	4.15233764386129e-06	\\
3035.20285866477	7.11386620666693e-06	\\
3036.181640625	3.95415933687225e-06	\\
3037.16042258523	7.55896600664917e-06	\\
3038.13920454545	6.83207274856037e-06	\\
3039.11798650568	4.76570786738601e-06	\\
3040.09676846591	4.76507462967633e-06	\\
3041.07555042614	4.50357429637167e-06	\\
3042.05433238636	6.09295689741792e-06	\\
3043.03311434659	4.94156561625081e-06	\\
3044.01189630682	6.4685169090079e-06	\\
3044.99067826705	7.48443629275991e-06	\\
3045.96946022727	6.30380882610487e-06	\\
3046.9482421875	4.74456073142441e-06	\\
3047.92702414773	7.15231682236398e-06	\\
3048.90580610795	5.44252418188539e-06	\\
3049.88458806818	7.10925875889012e-06	\\
3050.86337002841	6.5539356739255e-06	\\
3051.84215198864	6.91826876059586e-06	\\
3052.82093394886	6.23413952135587e-06	\\
3053.79971590909	7.80983543395229e-06	\\
3054.77849786932	6.55200856719435e-06	\\
3055.75727982955	5.70402873131691e-06	\\
3056.73606178977	4.57672620483507e-06	\\
3057.71484375	8.99406576059675e-06	\\
3058.69362571023	5.18059787071521e-06	\\
3059.67240767045	6.38301873149283e-06	\\
3060.65118963068	4.95524585464617e-06	\\
3061.62997159091	4.40797325449266e-06	\\
3062.60875355114	7.62448785756824e-06	\\
3063.58753551136	6.16614101989388e-06	\\
3064.56631747159	6.74089900647425e-06	\\
3065.54509943182	7.34643234563853e-06	\\
3066.52388139205	5.99584015493847e-06	\\
3067.50266335227	7.64314738720353e-06	\\
3068.4814453125	5.13973477305773e-06	\\
3069.46022727273	6.97628990249466e-06	\\
3070.43900923295	4.34366795876617e-06	\\
3071.41779119318	4.05005750045961e-06	\\
3072.39657315341	7.94173973072607e-06	\\
3073.37535511364	4.99819616327353e-06	\\
3074.35413707386	4.63910051189407e-06	\\
3075.33291903409	6.72143386986436e-06	\\
3076.31170099432	5.83976997745681e-06	\\
3077.29048295455	4.03314450671596e-06	\\
3078.26926491477	4.50650761891058e-06	\\
3079.248046875	4.99117503026635e-06	\\
3080.22682883523	7.19283563096231e-06	\\
3081.20561079545	7.16794683336276e-06	\\
3082.18439275568	4.3097912681052e-06	\\
3083.16317471591	4.80700168885917e-06	\\
3084.14195667614	6.89975836021229e-06	\\
3085.12073863636	4.48104840137389e-06	\\
3086.09952059659	4.39741355688163e-06	\\
3087.07830255682	6.91260470269919e-06	\\
3088.05708451705	6.13488375111366e-06	\\
3089.03586647727	3.15874983446296e-06	\\
3090.0146484375	4.78895660877867e-06	\\
3090.99343039773	4.80931924058817e-06	\\
3091.97221235795	6.26767565928252e-06	\\
3092.95099431818	3.91108215047e-06	\\
3093.92977627841	5.92590631531703e-06	\\
3094.90855823864	2.91023619559124e-06	\\
3095.88734019886	4.72951997772396e-06	\\
3096.86612215909	4.91443653087087e-06	\\
3097.84490411932	3.79965420500559e-06	\\
3098.82368607955	5.60379654193843e-06	\\
3099.80246803977	5.716255763521e-06	\\
3100.78125	4.13795609168738e-06	\\
3101.76003196023	4.03622601535408e-06	\\
3102.73881392045	5.88285614509227e-06	\\
3103.71759588068	6.3752328472931e-06	\\
3104.69637784091	6.89318862244485e-06	\\
3105.67515980114	5.75234955307357e-06	\\
3106.65394176136	6.14230578216623e-06	\\
3107.63272372159	6.5523408539336e-06	\\
3108.61150568182	3.45418633026757e-06	\\
3109.59028764205	3.94338233659148e-06	\\
3110.56906960227	5.24860672565812e-06	\\
3111.5478515625	4.76942534507713e-06	\\
3112.52663352273	3.86130979154733e-06	\\
3113.50541548295	6.09732819536721e-06	\\
3114.48419744318	6.35988451615635e-06	\\
3115.46297940341	4.92181354024062e-06	\\
3116.44176136364	6.13042248394637e-06	\\
3117.42054332386	5.06004826381039e-06	\\
3118.39932528409	4.84325478090718e-06	\\
3119.37810724432	5.62127855370219e-06	\\
3120.35688920455	5.58751328965117e-06	\\
3121.33567116477	4.83206776196961e-06	\\
3122.314453125	5.06203116678668e-06	\\
3123.29323508523	4.90283969654908e-06	\\
3124.27201704545	7.18213707472833e-06	\\
3125.25079900568	5.03323834727701e-06	\\
3126.22958096591	4.79715981792659e-06	\\
3127.20836292614	8.2314597083897e-06	\\
3128.18714488636	8.3165836448752e-06	\\
3129.16592684659	7.01780035230176e-06	\\
3130.14470880682	6.63759880024894e-06	\\
3131.12349076705	6.1407940956206e-06	\\
3132.10227272727	3.14842383348496e-06	\\
3133.0810546875	6.55303643369652e-06	\\
3134.05983664773	4.58261838591392e-06	\\
3135.03861860795	3.91332490416465e-06	\\
3136.01740056818	3.15048583968066e-06	\\
3136.99618252841	5.88830456517709e-06	\\
3137.97496448864	3.84024440642592e-06	\\
3138.95374644886	2.2181824934219e-06	\\
3139.93252840909	3.27826903338788e-06	\\
3140.91131036932	3.48070552145457e-06	\\
3141.89009232955	4.64987552474542e-06	\\
3142.86887428977	4.28017040964383e-06	\\
3143.84765625	4.39922190486844e-06	\\
3144.82643821023	3.59669263342718e-06	\\
3145.80522017045	4.15039680453441e-06	\\
3146.78400213068	4.24183282357497e-06	\\
3147.76278409091	3.00660776283667e-06	\\
3148.74156605114	5.50477495927385e-06	\\
3149.72034801136	2.11595300079999e-06	\\
3150.69912997159	4.56611065192147e-06	\\
3151.67791193182	4.07656546535735e-06	\\
3152.65669389205	4.8873373069387e-06	\\
3153.63547585227	5.52918833596567e-07	\\
3154.6142578125	4.8384274932039e-06	\\
3155.59303977273	4.62484261884579e-06	\\
3156.57182173295	1.31694150956772e-06	\\
3157.55060369318	4.39503041438791e-06	\\
3158.52938565341	1.93466575315706e-06	\\
3159.50816761364	3.86478533396869e-06	\\
3160.48694957386	4.23528587489164e-06	\\
3161.46573153409	5.20334689195883e-06	\\
3162.44451349432	2.44509192529568e-06	\\
3163.42329545455	3.504581701879e-06	\\
3164.40207741477	2.25894235515106e-06	\\
3165.380859375	2.54558590505924e-06	\\
3166.35964133523	5.8347712304792e-06	\\
3167.33842329545	6.92386584535704e-07	\\
3168.31720525568	4.83775389975338e-06	\\
3169.29598721591	2.44179472436013e-06	\\
3170.27476917614	3.35548880146266e-06	\\
3171.25355113636	2.89274074733574e-06	\\
3172.23233309659	7.35726866469139e-06	\\
3173.21111505682	2.61730714338528e-06	\\
3174.18989701705	6.37427472030012e-06	\\
3175.16867897727	4.99128864475251e-06	\\
3176.1474609375	4.16890905899354e-06	\\
3177.12624289773	6.20161050068553e-06	\\
3178.10502485795	2.07661331041894e-06	\\
3179.08380681818	6.14632015408303e-06	\\
3180.06258877841	3.88195510588894e-06	\\
3181.04137073864	6.80275632431413e-06	\\
3182.02015269886	3.80215428140113e-06	\\
3182.99893465909	3.92937849069316e-06	\\
3183.97771661932	4.2611333504769e-06	\\
3184.95649857955	3.91493549296007e-06	\\
3185.93528053977	3.74672987438624e-06	\\
3186.9140625	4.26334923423353e-06	\\
3187.89284446023	5.4014278979364e-06	\\
3188.87162642045	4.93100652199406e-06	\\
3189.85040838068	5.10850888346366e-06	\\
3190.82919034091	3.86969701534095e-06	\\
3191.80797230114	5.30937663456191e-06	\\
3192.78675426136	4.47924597823485e-06	\\
3193.76553622159	4.14581693859445e-06	\\
3194.74431818182	5.26842358455125e-06	\\
3195.72310014205	3.30159094907509e-06	\\
3196.70188210227	7.62083168257051e-06	\\
3197.6806640625	3.75299305943797e-06	\\
3198.65944602273	6.68029339717283e-06	\\
3199.63822798295	3.35948090814521e-06	\\
3200.61700994318	5.65065333180539e-06	\\
3201.59579190341	7.24295798028006e-06	\\
3202.57457386364	4.00331687122691e-06	\\
3203.55335582386	6.71907155191186e-06	\\
3204.53213778409	5.13027314589624e-06	\\
3205.51091974432	6.38983194150087e-06	\\
3206.48970170455	3.87100036576368e-06	\\
3207.46848366477	7.3289012997337e-06	\\
3208.447265625	5.30826621119794e-06	\\
3209.42604758523	5.85501017484608e-06	\\
3210.40482954545	5.922983937322e-06	\\
3211.38361150568	4.50996995634303e-06	\\
3212.36239346591	6.29389241809466e-06	\\
3213.34117542614	6.32496101982953e-06	\\
3214.31995738636	3.90845173629924e-06	\\
3215.29873934659	5.25764662041371e-06	\\
3216.27752130682	6.04400741603136e-06	\\
3217.25630326705	5.72424426459969e-06	\\
3218.23508522727	3.78930275975588e-06	\\
3219.2138671875	4.26801218791639e-06	\\
3220.19264914773	4.80400218116881e-06	\\
3221.17143110795	3.38353570221822e-06	\\
3222.15021306818	1.31814296328779e-06	\\
3223.12899502841	4.19828861450633e-06	\\
3224.10777698864	4.53852866545889e-06	\\
3225.08655894886	6.09772358396485e-06	\\
3226.06534090909	8.11493957546307e-06	\\
3227.04412286932	3.44674267910026e-06	\\
3228.02290482955	5.21252061453378e-06	\\
3229.00168678977	3.74226794524826e-06	\\
3229.98046875	4.63197998282777e-06	\\
3230.95925071023	5.19444498787262e-06	\\
3231.93803267045	8.42630357298141e-06	\\
3232.91681463068	4.27143515873373e-06	\\
3233.89559659091	7.7988769166354e-06	\\
3234.87437855114	5.65380206501836e-06	\\
3235.85316051136	6.96644191113979e-06	\\
3236.83194247159	5.87315121357321e-06	\\
3237.81072443182	6.67282010671431e-06	\\
3238.78950639205	5.31539093117702e-06	\\
3239.76828835227	6.41501026780166e-06	\\
3240.7470703125	7.41429341354414e-06	\\
3241.72585227273	5.8399363041359e-06	\\
3242.70463423295	4.99770944644396e-06	\\
3243.68341619318	8.80652896629794e-06	\\
3244.66219815341	6.35209212516419e-06	\\
3245.64098011364	5.4023890445851e-06	\\
3246.61976207386	7.02191099656472e-06	\\
3247.59854403409	6.47201133766624e-06	\\
3248.57732599432	6.86715071978326e-06	\\
3249.55610795455	5.38593840326826e-06	\\
3250.53488991477	7.56900304889741e-06	\\
3251.513671875	5.34980346144889e-06	\\
3252.49245383523	5.38038127032368e-06	\\
3253.47123579545	6.27393754920069e-06	\\
3254.45001775568	6.47638190431502e-06	\\
3255.42879971591	4.94800954546705e-06	\\
3256.40758167614	6.58610450167089e-06	\\
3257.38636363636	6.38058332918237e-06	\\
3258.36514559659	8.24241200361249e-06	\\
3259.34392755682	7.46520938618689e-06	\\
3260.32270951705	8.58483172141497e-06	\\
3261.30149147727	7.97544534242314e-06	\\
3262.2802734375	7.70378348542392e-06	\\
3263.25905539773	7.27511933645924e-06	\\
3264.23783735795	5.896319115889e-06	\\
3265.21661931818	8.14366679430762e-06	\\
3266.19540127841	3.77132191679055e-06	\\
3267.17418323864	8.83533842087064e-06	\\
3268.15296519886	3.98092860014282e-06	\\
3269.13174715909	8.12762327739322e-06	\\
3270.11052911932	6.28473592185866e-06	\\
3271.08931107955	7.27303686302001e-06	\\
3272.06809303977	6.17312732549603e-06	\\
3273.046875	9.21340192649073e-06	\\
3274.02565696023	5.8997042667548e-06	\\
3275.00443892045	5.88843685975198e-06	\\
3275.98322088068	9.3234866500227e-06	\\
3276.96200284091	7.00865555213517e-06	\\
3277.94078480114	7.8361886983984e-06	\\
3278.91956676136	7.45959678344316e-06	\\
3279.89834872159	8.96647267906354e-06	\\
3280.87713068182	5.3011361950394e-06	\\
3281.85591264205	9.12278384178588e-06	\\
3282.83469460227	3.02621981215116e-06	\\
3283.8134765625	1.16129092628454e-05	\\
3284.79225852273	4.10121181473996e-06	\\
3285.77104048295	8.01764773017378e-06	\\
3286.74982244318	5.58690297865913e-06	\\
3287.72860440341	7.6044077536189e-06	\\
3288.70738636364	6.52671439327849e-06	\\
3289.68616832386	6.83219437516752e-06	\\
3290.66495028409	9.11573731878824e-06	\\
3291.64373224432	7.99675204099443e-06	\\
3292.62251420455	8.37592060157944e-06	\\
3293.60129616477	6.58410975446255e-06	\\
3294.580078125	6.97655445231526e-06	\\
3295.55886008523	6.52904387827444e-06	\\
3296.53764204545	8.19329331255562e-06	\\
3297.51642400568	8.91892451367944e-06	\\
3298.49520596591	6.41106619698526e-06	\\
3299.47398792614	3.37697839336722e-06	\\
3300.45276988636	6.96508575429446e-06	\\
3301.43155184659	6.13210353526283e-06	\\
3302.41033380682	8.88523462359483e-06	\\
3303.38911576705	4.77191837013829e-06	\\
3304.36789772727	8.93289980575135e-06	\\
3305.3466796875	6.28498355641954e-06	\\
3306.32546164773	9.88758830207203e-06	\\
3307.30424360795	7.27683917550227e-06	\\
3308.28302556818	9.76868552133638e-06	\\
3309.26180752841	6.44529993999361e-06	\\
3310.24058948864	8.69830457491726e-06	\\
3311.21937144886	6.50644268970866e-06	\\
3312.19815340909	9.81418959686322e-06	\\
3313.17693536932	8.34645314049038e-06	\\
3314.15571732955	5.2731999162103e-06	\\
3315.13449928977	7.32208244607607e-06	\\
3316.11328125	7.45895704143135e-06	\\
3317.09206321023	6.15846599281555e-06	\\
3318.07084517045	7.84756333079582e-06	\\
3319.04962713068	9.81091692586088e-06	\\
3320.02840909091	8.43358883167105e-06	\\
3321.00719105114	6.20932637759786e-06	\\
3321.98597301136	6.941028674256e-06	\\
3322.96475497159	8.70111819049835e-06	\\
3323.94353693182	6.69808778450492e-06	\\
3324.92231889205	6.16208006587392e-06	\\
3325.90110085227	6.35983347770488e-06	\\
3326.8798828125	7.01575146654042e-06	\\
3327.85866477273	6.69771063151008e-06	\\
3328.83744673295	6.49174316445218e-06	\\
3329.81622869318	9.83154696703001e-06	\\
3330.79501065341	7.69444990738781e-06	\\
3331.77379261364	6.29219206593473e-06	\\
3332.75257457386	1.01399206451832e-05	\\
3333.73135653409	8.22333149309792e-06	\\
3334.71013849432	8.41799297286671e-06	\\
3335.68892045455	5.4677146489931e-06	\\
3336.66770241477	7.46061936564249e-06	\\
3337.646484375	7.05863278835874e-06	\\
3338.62526633523	6.90509342064109e-06	\\
3339.60404829545	6.58308444298448e-06	\\
3340.58283025568	8.95637699944769e-06	\\
3341.56161221591	6.9730682994108e-06	\\
3342.54039417614	6.14978028850856e-06	\\
3343.51917613636	6.48592964833874e-06	\\
3344.49795809659	7.44174954045407e-06	\\
3345.47674005682	6.78035793684003e-06	\\
3346.45552201705	5.55713189251655e-06	\\
3347.43430397727	6.3165992972136e-06	\\
3348.4130859375	6.80450518706188e-06	\\
3349.39186789773	5.67994348308526e-06	\\
3350.37064985795	8.11650210632133e-06	\\
3351.34943181818	7.40080677453706e-06	\\
3352.32821377841	6.41640383978907e-06	\\
3353.30699573864	6.85492585015592e-06	\\
3354.28577769886	9.07323722850592e-06	\\
3355.26455965909	8.24038575027765e-06	\\
3356.24334161932	6.31870904327666e-06	\\
3357.22212357955	9.43830640792177e-06	\\
3358.20090553977	8.36871221273323e-06	\\
3359.1796875	8.07424375401736e-06	\\
3360.15846946023	5.48839108913006e-06	\\
3361.13725142045	8.30738297076493e-06	\\
3362.11603338068	7.30792026793953e-06	\\
3363.09481534091	6.69830633197007e-06	\\
3364.07359730114	6.98775215765831e-06	\\
3365.05237926136	6.85539787566859e-06	\\
3366.03116122159	7.29350871201282e-06	\\
3367.00994318182	6.84039083965697e-06	\\
3367.98872514205	7.44402493791908e-06	\\
3368.96750710227	9.64371153771517e-06	\\
3369.9462890625	6.89944727841426e-06	\\
3370.92507102273	7.44558206478955e-06	\\
3371.90385298295	7.19673519688336e-06	\\
3372.88263494318	7.82487888422349e-06	\\
3373.86141690341	9.93265218941174e-06	\\
3374.84019886364	8.55197052134713e-06	\\
3375.81898082386	8.44065631619855e-06	\\
3376.79776278409	7.60692380105086e-06	\\
3377.77654474432	1.05686705840513e-05	\\
3378.75532670455	6.94838989868597e-06	\\
3379.73410866477	7.15441903640344e-06	\\
3380.712890625	7.07200914265782e-06	\\
3381.69167258523	7.39343881204883e-06	\\
3382.67045454545	7.71705747928608e-06	\\
3383.64923650568	9.69782900792454e-06	\\
3384.62801846591	6.22585919867337e-06	\\
3385.60680042614	4.7049351539577e-06	\\
3386.58558238636	5.76898927266532e-06	\\
3387.56436434659	6.57631789533697e-06	\\
3388.54314630682	6.71737322204307e-06	\\
3389.52192826705	9.33550395247543e-06	\\
3390.50071022727	6.58484964340971e-06	\\
3391.4794921875	6.82790216426201e-06	\\
3392.45827414773	5.79316356799543e-06	\\
3393.43705610795	6.84890177200744e-06	\\
3394.41583806818	6.37732724761621e-06	\\
3395.39462002841	7.45587492044357e-06	\\
3396.37340198864	8.21994853591322e-06	\\
3397.35218394886	7.9190667227137e-06	\\
3398.33096590909	7.86951994067151e-06	\\
3399.30974786932	7.11092579527137e-06	\\
3400.28852982955	5.70550447471887e-06	\\
3401.26731178977	5.83378188966146e-06	\\
3402.24609375	5.37891938583579e-06	\\
3403.22487571023	4.81408353765791e-06	\\
3404.20365767045	1.02619505309727e-05	\\
3405.18243963068	7.48842392090886e-06	\\
3406.16122159091	7.26339198684622e-06	\\
3407.14000355114	3.77433889043678e-06	\\
3408.11878551136	6.90188943552882e-06	\\
3409.09756747159	4.0366987377356e-06	\\
3410.07634943182	7.20635509411527e-06	\\
3411.05513139205	6.60121476417006e-06	\\
3412.03391335227	7.99293070183558e-06	\\
3413.0126953125	6.54592566917883e-06	\\
3413.99147727273	9.16232725202475e-06	\\
3414.97025923295	2.38283455407305e-06	\\
3415.94904119318	6.71143783038658e-06	\\
3416.92782315341	3.99686819027672e-06	\\
3417.90660511364	4.40146831901494e-06	\\
3418.88538707386	7.10026395781727e-06	\\
3419.86416903409	7.76509448476902e-06	\\
3420.84295099432	4.83030759570343e-06	\\
3421.82173295455	4.43239442074157e-06	\\
3422.80051491477	5.62834548355789e-06	\\
3423.779296875	7.65786070626034e-06	\\
3424.75807883523	6.43160740907779e-06	\\
3425.73686079545	8.30646375958038e-06	\\
3426.71564275568	7.99995705540487e-06	\\
3427.69442471591	7.00796822398318e-06	\\
3428.67320667614	7.03433153620489e-06	\\
3429.65198863636	6.28377305330906e-06	\\
3430.63077059659	2.77482619174596e-06	\\
3431.60955255682	4.73776766269984e-06	\\
3432.58833451705	6.30229051593783e-06	\\
3433.56711647727	6.79846231388591e-06	\\
3434.5458984375	6.66392720869497e-06	\\
3435.52468039773	8.50651891895515e-06	\\
3436.50346235795	6.07576725849794e-06	\\
3437.48224431818	8.29242763924627e-06	\\
3438.46102627841	7.13970581025488e-06	\\
3439.43980823864	8.10601030548511e-06	\\
3440.41859019886	5.05856236041813e-06	\\
3441.39737215909	6.84836154620129e-06	\\
3442.37615411932	7.72460782252245e-06	\\
3443.35493607955	6.54087288283437e-06	\\
3444.33371803977	7.53278680441074e-06	\\
3445.3125	5.99173790827754e-06	\\
3446.29128196023	3.70200641989561e-06	\\
3447.27006392045	6.9378933647699e-06	\\
3448.24884588068	8.14071883116257e-06	\\
3449.22762784091	7.70351548227938e-06	\\
3450.20640980114	7.35790532559217e-06	\\
3451.18519176136	8.80808763860331e-06	\\
3452.16397372159	8.30299497918e-06	\\
3453.14275568182	4.77793722699523e-06	\\
3454.12153764205	9.69394550351038e-06	\\
3455.10031960227	7.6659976717771e-06	\\
3456.0791015625	7.64587142366812e-06	\\
3457.05788352273	5.82556940238407e-06	\\
3458.03666548295	6.76257004747967e-06	\\
3459.01544744318	7.87804523221647e-06	\\
3459.99422940341	7.41940785194909e-06	\\
3460.97301136364	9.29149528477892e-06	\\
3461.95179332386	6.86022284061155e-06	\\
3462.93057528409	1.01161041584801e-05	\\
3463.90935724432	7.5619126173186e-06	\\
3464.88813920455	6.94680488782122e-06	\\
3465.86692116477	6.92564279519691e-06	\\
3466.845703125	7.28894221915631e-06	\\
3467.82448508523	5.49836158129102e-06	\\
3468.80326704545	6.50949404761308e-06	\\
3469.78204900568	6.68251709828621e-06	\\
3470.76083096591	6.38168303016065e-06	\\
3471.73961292614	7.05253169286347e-06	\\
3472.71839488636	4.41711686018104e-06	\\
3473.69717684659	4.46652921305547e-06	\\
3474.67595880682	4.87350696913155e-06	\\
3475.65474076705	5.16461960748422e-06	\\
3476.63352272727	8.28003508496028e-06	\\
3477.6123046875	6.71390251053503e-06	\\
3478.59108664773	6.3802847236994e-06	\\
3479.56986860795	4.27263229519662e-06	\\
3480.54865056818	7.8157263363365e-06	\\
3481.52743252841	6.13412453330947e-06	\\
3482.50621448864	3.00136556224579e-06	\\
3483.48499644886	6.16744448126104e-06	\\
3484.46377840909	5.056437607108e-06	\\
3485.44256036932	6.5838534475994e-06	\\
3486.42134232955	7.36955719161516e-06	\\
3487.40012428977	6.36702807454397e-06	\\
3488.37890625	5.52648340046677e-06	\\
3489.35768821023	5.17599179465239e-06	\\
3490.33647017045	4.77971675010694e-06	\\
3491.31525213068	5.45426441292021e-06	\\
3492.29403409091	3.78010079014488e-06	\\
3493.27281605114	7.21614046725465e-06	\\
3494.25159801136	8.25085686115338e-06	\\
3495.23037997159	6.04673970297001e-06	\\
3496.20916193182	8.20902125809212e-06	\\
3497.18794389205	5.21155919160867e-06	\\
3498.16672585227	8.02797698097798e-06	\\
3499.1455078125	6.21429183950036e-06	\\
3500.12428977273	5.18233475439163e-06	\\
3501.10307173295	8.84491700526281e-06	\\
3502.08185369318	7.88531456908246e-06	\\
3503.06063565341	7.85702879399076e-06	\\
3504.03941761364	4.96805033259464e-06	\\
3505.01819957386	8.23758327219558e-06	\\
3505.99698153409	7.91607107761979e-06	\\
3506.97576349432	3.7676813679772e-06	\\
3507.95454545455	6.15560592537302e-06	\\
3508.93332741477	8.84278733180434e-06	\\
3509.912109375	7.92330995635317e-06	\\
3510.89089133523	8.55272800008483e-06	\\
3511.86967329545	6.01836949631538e-06	\\
3512.84845525568	6.93970322255143e-06	\\
3513.82723721591	9.81152796961525e-06	\\
3514.80601917614	5.6647106985432e-06	\\
3515.78480113636	6.39244716031695e-06	\\
3516.76358309659	6.46713330843624e-06	\\
3517.74236505682	4.26743351707681e-06	\\
3518.72114701705	5.22946469483049e-06	\\
3519.69992897727	7.66115410192093e-06	\\
3520.6787109375	5.80448584381825e-06	\\
3521.65749289773	6.41315430571118e-06	\\
3522.63627485795	6.47630058402033e-06	\\
3523.61505681818	7.02431779212775e-06	\\
3524.59383877841	6.9374026792154e-06	\\
3525.57262073864	6.71920546619903e-06	\\
3526.55140269886	7.45693559061545e-06	\\
3527.53018465909	6.69794389323451e-06	\\
3528.50896661932	4.11935426963018e-06	\\
3529.48774857955	7.5449020616378e-06	\\
3530.46653053977	7.36375619001882e-06	\\
3531.4453125	4.52192493253271e-06	\\
3532.42409446023	5.0157085668709e-06	\\
3533.40287642045	7.59124480229111e-06	\\
3534.38165838068	6.98460749172944e-06	\\
3535.36044034091	8.27001353433307e-06	\\
3536.33922230114	6.3448567851218e-06	\\
3537.31800426136	8.33968208461058e-06	\\
3538.29678622159	7.57442799221758e-06	\\
3539.27556818182	4.74014165364509e-06	\\
3540.25435014205	7.79935656664089e-06	\\
3541.23313210227	4.89159192713225e-06	\\
3542.2119140625	7.99447174397667e-06	\\
3543.19069602273	8.34808497765795e-06	\\
3544.16947798295	9.90093808301727e-06	\\
3545.14825994318	7.30551390303694e-06	\\
3546.12704190341	7.27798061806614e-06	\\
3547.10582386364	6.97464003906052e-06	\\
3548.08460582386	7.71673948580948e-06	\\
3549.06338778409	8.10287584775513e-06	\\
3550.04216974432	8.12269646650254e-06	\\
3551.02095170455	8.49271721214546e-06	\\
3551.99973366477	8.75833153100587e-06	\\
3552.978515625	4.89367580688912e-06	\\
3553.95729758523	4.90507372523561e-06	\\
3554.93607954545	9.25510773000291e-06	\\
3555.91486150568	8.43982977479424e-06	\\
3556.89364346591	8.76616784128419e-06	\\
3557.87242542614	5.52961088485144e-06	\\
3558.85120738636	5.91291438474065e-06	\\
3559.82998934659	5.28956856910582e-06	\\
3560.80877130682	5.83947649533358e-06	\\
3561.78755326705	8.70048584898104e-06	\\
3562.76633522727	6.16721068332521e-06	\\
3563.7451171875	8.54380140155201e-06	\\
3564.72389914773	8.87286185486794e-06	\\
3565.70268110795	4.32675996489402e-06	\\
3566.68146306818	7.47392702735576e-06	\\
3567.66024502841	5.19934884238498e-06	\\
3568.63902698864	6.1205731034157e-06	\\
3569.61780894886	6.10840765271297e-06	\\
3570.59659090909	6.92915597776963e-06	\\
3571.57537286932	4.31363442005656e-06	\\
3572.55415482955	7.25472186039056e-06	\\
3573.53293678977	8.36639597986128e-06	\\
3574.51171875	5.80918137154211e-06	\\
3575.49050071023	6.85058334370493e-06	\\
3576.46928267045	5.08588306267143e-06	\\
3577.44806463068	7.14470155295548e-06	\\
3578.42684659091	4.98167705915481e-06	\\
3579.40562855114	5.1534677504994e-06	\\
3580.38441051136	4.98120767346133e-06	\\
3581.36319247159	5.71872848971829e-06	\\
3582.34197443182	6.43811540214592e-06	\\
3583.32075639205	5.01995996316489e-06	\\
3584.29953835227	7.84262346558585e-06	\\
3585.2783203125	4.70605684611637e-06	\\
3586.25710227273	5.00701361053067e-06	\\
3587.23588423295	6.14589652224486e-06	\\
3588.21466619318	4.19455655358598e-06	\\
3589.19344815341	6.30118921875191e-06	\\
3590.17223011364	5.65985600905342e-06	\\
3591.15101207386	6.51812507423034e-06	\\
3592.12979403409	6.04415202550902e-06	\\
3593.10857599432	7.22254495203417e-06	\\
3594.08735795455	6.46703231561407e-06	\\
3595.06613991477	5.72604309645326e-06	\\
3596.044921875	8.47443878300234e-06	\\
3597.02370383523	5.66301928656613e-06	\\
3598.00248579545	7.54311835447507e-06	\\
3598.98126775568	6.65215618990689e-06	\\
3599.96004971591	5.02070764360607e-06	\\
3600.93883167614	5.80251236657074e-06	\\
3601.91761363636	6.64376728240701e-06	\\
3602.89639559659	4.28608386311376e-06	\\
3603.87517755682	6.81193132689793e-06	\\
3604.85395951705	7.72416717880155e-06	\\
3605.83274147727	7.72752760225253e-06	\\
3606.8115234375	5.76897207609718e-06	\\
3607.79030539773	6.55130005852284e-06	\\
3608.76908735795	3.59827076733747e-06	\\
3609.74786931818	5.29597860804442e-06	\\
3610.72665127841	3.41798654756105e-06	\\
3611.70543323864	5.91679835515188e-06	\\
3612.68421519886	9.03922422165704e-06	\\
3613.66299715909	7.48438988790588e-06	\\
3614.64177911932	5.75779809032343e-06	\\
3615.62056107955	6.04280435679032e-06	\\
3616.59934303977	5.44975327539255e-06	\\
3617.578125	4.92495974880354e-06	\\
3618.55690696023	6.57732300753424e-06	\\
3619.53568892045	5.46307065124631e-06	\\
3620.51447088068	5.79615003468929e-06	\\
3621.49325284091	6.12473671093464e-06	\\
3622.47203480114	4.05586413804813e-06	\\
3623.45081676136	4.70330272530917e-06	\\
3624.42959872159	6.35167278916737e-06	\\
3625.40838068182	5.50486988483935e-06	\\
3626.38716264205	4.92493759445532e-06	\\
3627.36594460227	4.74086522628021e-06	\\
3628.3447265625	5.82179618095775e-06	\\
3629.32350852273	5.95504141928408e-06	\\
3630.30229048295	5.01255514787651e-06	\\
3631.28107244318	5.25666662424234e-06	\\
3632.25985440341	5.9645818142036e-06	\\
3633.23863636364	5.38363618962568e-06	\\
3634.21741832386	7.78938204161221e-06	\\
3635.19620028409	6.71931340916732e-06	\\
3636.17498224432	9.15248479075253e-06	\\
3637.15376420455	3.68491790948206e-06	\\
3638.13254616477	5.29377007366807e-06	\\
3639.111328125	6.75242286026626e-06	\\
3640.09011008523	6.89186865074403e-06	\\
3641.06889204545	4.47196671760675e-06	\\
3642.04767400568	5.45744039960745e-06	\\
3643.02645596591	6.12220707004126e-06	\\
3644.00523792614	5.20241292324569e-06	\\
3644.98401988636	6.12223441147365e-06	\\
3645.96280184659	6.71308140232794e-06	\\
3646.94158380682	6.02572332233436e-06	\\
3647.92036576705	7.88484541807491e-06	\\
3648.89914772727	4.96389740660618e-06	\\
3649.8779296875	7.64387113133748e-06	\\
3650.85671164773	6.9299470912518e-06	\\
3651.83549360795	5.26532393395139e-06	\\
3652.81427556818	7.86017399730018e-06	\\
3653.79305752841	8.17099170096971e-06	\\
3654.77183948864	7.62115988399315e-06	\\
3655.75062144886	4.67409217687299e-06	\\
3656.72940340909	6.02573403868961e-06	\\
3657.70818536932	6.40643770274655e-06	\\
3658.68696732955	6.23621181427428e-06	\\
3659.66574928977	7.89673484851145e-06	\\
3660.64453125	6.8394902281859e-06	\\
3661.62331321023	7.77939167455003e-06	\\
3662.60209517045	5.58712043698536e-06	\\
3663.58087713068	6.12531949998876e-06	\\
3664.55965909091	3.86748561937351e-06	\\
3665.53844105114	6.11418399035535e-06	\\
3666.51722301136	6.52136226211795e-06	\\
3667.49600497159	5.39490944936203e-06	\\
3668.47478693182	6.64737887194797e-06	\\
3669.45356889205	5.84821585264568e-06	\\
3670.43235085227	5.58735033763114e-06	\\
3671.4111328125	5.15332550428465e-06	\\
3672.38991477273	6.75397528384377e-06	\\
3673.36869673295	4.11787227326751e-06	\\
3674.34747869318	6.33391242737011e-06	\\
3675.32626065341	5.06958711753731e-06	\\
3676.30504261364	6.81139890244367e-06	\\
3677.28382457386	7.19125332991926e-06	\\
3678.26260653409	4.86834197120754e-06	\\
3679.24138849432	3.90886532921651e-06	\\
3680.22017045455	5.96340542173075e-06	\\
3681.19895241477	5.66029139252349e-06	\\
3682.177734375	4.59061189445033e-06	\\
3683.15651633523	4.09644727983965e-06	\\
3684.13529829545	3.62014286048305e-06	\\
3685.11408025568	7.12010427111734e-06	\\
3686.09286221591	7.20194412730285e-06	\\
3687.07164417614	5.75163953015437e-06	\\
3688.05042613636	3.90819090327007e-06	\\
3689.02920809659	5.6065707565433e-06	\\
3690.00799005682	3.8067320202121e-06	\\
3690.98677201705	5.06407752635715e-06	\\
3691.96555397727	4.7274957755584e-06	\\
3692.9443359375	4.76847147200634e-06	\\
3693.92311789773	5.86319819679038e-06	\\
3694.90189985795	5.71957391363233e-06	\\
3695.88068181818	7.52492284928144e-06	\\
3696.85946377841	3.81701023934439e-06	\\
3697.83824573864	8.15862764318605e-06	\\
3698.81702769886	5.49574328477703e-06	\\
3699.79580965909	3.92089243203462e-06	\\
3700.77459161932	4.77309933025888e-06	\\
3701.75337357955	5.38666624323178e-06	\\
3702.73215553977	3.65558284974845e-06	\\
3703.7109375	6.69216942003197e-06	\\
3704.68971946023	4.66892048823575e-06	\\
3705.66850142045	7.49996424404987e-06	\\
3706.64728338068	5.13817564166087e-06	\\
3707.62606534091	6.09502166183094e-06	\\
3708.60484730114	4.88668033060319e-06	\\
3709.58362926136	6.15493007413029e-06	\\
3710.56241122159	5.07475277034169e-06	\\
3711.54119318182	6.75473190831321e-06	\\
3712.51997514205	7.62027025344297e-06	\\
3713.49875710227	4.00615039142145e-06	\\
3714.4775390625	4.59406345080459e-06	\\
3715.45632102273	6.74798190642368e-06	\\
3716.43510298295	4.35604739927719e-06	\\
3717.41388494318	9.13093712096936e-06	\\
3718.39266690341	5.30223385313839e-06	\\
3719.37144886364	4.99311258283933e-06	\\
3720.35023082386	4.4551527767481e-06	\\
3721.32901278409	3.91258612541529e-06	\\
3722.30779474432	6.53953915401352e-06	\\
3723.28657670455	6.22838127850222e-06	\\
3724.26535866477	4.84560769546973e-06	\\
3725.244140625	5.33140498760509e-06	\\
3726.22292258523	4.18715429872618e-06	\\
3727.20170454545	4.25195111997762e-06	\\
3728.18048650568	5.53926694232788e-06	\\
3729.15926846591	7.46729790048491e-06	\\
3730.13805042614	5.60130875226364e-06	\\
3731.11683238636	5.42201023002457e-06	\\
3732.09561434659	4.87548579538227e-06	\\
3733.07439630682	4.92031158702522e-06	\\
3734.05317826705	4.75054621273942e-06	\\
3735.03196022727	6.04409324903174e-06	\\
3736.0107421875	6.27101611096258e-06	\\
3736.98952414773	7.17140010532719e-06	\\
3737.96830610795	6.34714736867587e-06	\\
3738.94708806818	5.91309380573564e-06	\\
3739.92587002841	7.18038959765674e-06	\\
3740.90465198864	6.60800169360869e-06	\\
3741.88343394886	4.35385517961982e-06	\\
3742.86221590909	5.5778126083664e-06	\\
3743.84099786932	5.86305685837589e-06	\\
3744.81977982955	6.6160319831123e-06	\\
3745.79856178977	6.08249190252342e-06	\\
3746.77734375	7.3051897894652e-06	\\
3747.75612571023	7.56720891523585e-06	\\
3748.73490767045	7.19916764270522e-06	\\
3749.71368963068	9.08077399513466e-06	\\
3750.69247159091	7.32401762690729e-06	\\
3751.67125355114	6.42329527717843e-06	\\
3752.65003551136	5.33515844392986e-06	\\
3753.62881747159	5.6994733422277e-06	\\
3754.60759943182	8.25283575928271e-06	\\
3755.58638139205	7.80178404470043e-06	\\
3756.56516335227	6.85557930103005e-06	\\
3757.5439453125	8.19469248215412e-06	\\
3758.52272727273	8.60961107909023e-06	\\
3759.50150923295	7.10680009677556e-06	\\
3760.48029119318	6.15354084823305e-06	\\
3761.45907315341	7.1596473690305e-06	\\
3762.43785511364	7.94033811277302e-06	\\
3763.41663707386	5.8654703407583e-06	\\
3764.39541903409	7.63857775116751e-06	\\
3765.37420099432	6.80271255194504e-06	\\
3766.35298295455	9.01616840721988e-06	\\
3767.33176491477	7.56675591010734e-06	\\
3768.310546875	7.53867953585914e-06	\\
3769.28932883523	7.47961240178046e-06	\\
3770.26811079545	8.12152549195699e-06	\\
3771.24689275568	8.14432033252387e-06	\\
3772.22567471591	7.14056788792626e-06	\\
3773.20445667614	6.775951664583e-06	\\
3774.18323863636	7.33012398719955e-06	\\
3775.16202059659	5.85443251347548e-06	\\
3776.14080255682	7.11642562356757e-06	\\
3777.11958451705	7.49956776070063e-06	\\
3778.09836647727	8.84390640009846e-06	\\
3779.0771484375	8.04172728522899e-06	\\
3780.05593039773	9.70850041827951e-06	\\
3781.03471235795	6.56657235374613e-06	\\
3782.01349431818	7.30418108141037e-06	\\
3782.99227627841	7.55988467943172e-06	\\
3783.97105823864	9.33872554869944e-06	\\
3784.94984019886	5.19865395407313e-06	\\
3785.92862215909	9.97753944646779e-06	\\
3786.90740411932	8.16528423118959e-06	\\
3787.88618607955	5.80013307336796e-06	\\
3788.86496803977	8.50093952819187e-06	\\
3789.84375	8.46427396638988e-06	\\
3790.82253196023	1.05088179989209e-05	\\
3791.80131392045	8.94150270743503e-06	\\
3792.78009588068	8.85845552516095e-06	\\
3793.75887784091	8.96466735529499e-06	\\
3794.73765980114	8.25196121966765e-06	\\
3795.71644176136	5.92267210482019e-06	\\
3796.69522372159	8.81989470235938e-06	\\
3797.67400568182	8.52692526746663e-06	\\
3798.65278764205	9.08270262441714e-06	\\
3799.63156960227	8.84279645506618e-06	\\
3800.6103515625	5.86783311547373e-06	\\
3801.58913352273	7.78145102904643e-06	\\
3802.56791548295	9.1002021749213e-06	\\
3803.54669744318	7.62763589872142e-06	\\
3804.52547940341	4.70065848765479e-06	\\
3805.50426136364	8.24816007979119e-06	\\
3806.48304332386	8.57846101076394e-06	\\
3807.46182528409	6.75761637291902e-06	\\
3808.44060724432	7.77233703580377e-06	\\
3809.41938920455	7.68061467755089e-06	\\
3810.39817116477	8.65656331779332e-06	\\
3811.376953125	8.67368376125557e-06	\\
3812.35573508523	8.07101567790571e-06	\\
3813.33451704545	6.64718700994743e-06	\\
3814.31329900568	5.7884619710033e-06	\\
3815.29208096591	5.11486805533566e-06	\\
3816.27086292614	8.35067998315766e-06	\\
3817.24964488636	6.73390842355412e-06	\\
3818.22842684659	5.02528869152769e-06	\\
3819.20720880682	7.24040423380109e-06	\\
3820.18599076705	7.40405679798361e-06	\\
3821.16477272727	7.04180687941928e-06	\\
3822.1435546875	6.88908209704542e-06	\\
3823.12233664773	6.85764581792156e-06	\\
3824.10111860795	5.22918531017707e-06	\\
3825.07990056818	8.20237535616737e-06	\\
3826.05868252841	7.05714581322515e-06	\\
3827.03746448864	1.01102218195791e-05	\\
3828.01624644886	8.09359889222887e-06	\\
3828.99502840909	6.443844659156e-06	\\
3829.97381036932	8.4554522990679e-06	\\
3830.95259232955	6.99177598335416e-06	\\
3831.93137428977	5.16684529421786e-06	\\
3832.91015625	8.2710086621407e-06	\\
3833.88893821023	6.34417980428824e-06	\\
3834.86772017045	7.43693365518207e-06	\\
3835.84650213068	9.10474944346829e-06	\\
3836.82528409091	7.84454196120125e-06	\\
3837.80406605114	7.75858078366535e-06	\\
3838.78284801136	6.40270606706117e-06	\\
3839.76162997159	6.4033595841847e-06	\\
3840.74041193182	8.21812855767188e-06	\\
3841.71919389205	8.47018865624219e-06	\\
3842.69797585227	4.68500405337881e-06	\\
3843.6767578125	5.78163513699088e-06	\\
3844.65553977273	6.05713807714762e-06	\\
3845.63432173295	6.21286318626633e-06	\\
3846.61310369318	6.8854467461958e-06	\\
3847.59188565341	6.965308565192e-06	\\
3848.57066761364	6.7411215237972e-06	\\
3849.54944957386	8.9201928521041e-06	\\
3850.52823153409	6.24148905093687e-06	\\
3851.50701349432	7.40519062976217e-06	\\
3852.48579545455	6.94971957282237e-06	\\
3853.46457741477	7.72593424018709e-06	\\
3854.443359375	9.27437479797837e-06	\\
3855.42214133523	7.35078103408328e-06	\\
3856.40092329545	6.65085788862528e-06	\\
3857.37970525568	5.39256150487343e-06	\\
3858.35848721591	5.89152284029728e-06	\\
3859.33726917614	7.22052342170207e-06	\\
3860.31605113636	8.55974040892545e-06	\\
3861.29483309659	7.319276893695e-06	\\
3862.27361505682	6.9094483842241e-06	\\
3863.25239701705	8.75498248382689e-06	\\
3864.23117897727	5.51012615676596e-06	\\
3865.2099609375	8.3654703420113e-06	\\
3866.18874289773	5.88507260969034e-06	\\
3867.16752485795	8.39905263050578e-06	\\
3868.14630681818	6.91078766357372e-06	\\
3869.12508877841	7.47121091342427e-06	\\
3870.10387073864	8.0679753807878e-06	\\
3871.08265269886	6.88990864420562e-06	\\
3872.06143465909	7.17088350485163e-06	\\
3873.04021661932	8.39084687068638e-06	\\
3874.01899857955	8.93451822919817e-06	\\
3874.99778053977	7.85814896273965e-06	\\
3875.9765625	7.48509845601575e-06	\\
3876.95534446023	9.62683686978885e-06	\\
3877.93412642045	8.95984988144346e-06	\\
3878.91290838068	8.3306252562705e-06	\\
3879.89169034091	8.59219215533931e-06	\\
3880.87047230114	7.86134842458639e-06	\\
3881.84925426136	1.02843845229786e-05	\\
3882.82803622159	7.25938957137229e-06	\\
3883.80681818182	7.83622913301947e-06	\\
3884.78560014205	7.5730150572875e-06	\\
3885.76438210227	9.11674065795103e-06	\\
3886.7431640625	7.53799630336068e-06	\\
3887.72194602273	6.62076767470328e-06	\\
3888.70072798295	7.45031270801566e-06	\\
3889.67950994318	5.79608664530167e-06	\\
3890.65829190341	7.22775779172843e-06	\\
3891.63707386364	6.60379644763585e-06	\\
3892.61585582386	8.22389845314521e-06	\\
3893.59463778409	7.37538689419954e-06	\\
3894.57341974432	6.67614633257298e-06	\\
3895.55220170455	7.80095813483149e-06	\\
3896.53098366477	5.72232590382659e-06	\\
3897.509765625	9.52499878310146e-06	\\
3898.48854758523	8.06903628202201e-06	\\
3899.46732954545	7.87005942221386e-06	\\
3900.44611150568	9.11682255379947e-06	\\
3901.42489346591	7.16153978682968e-06	\\
3902.40367542614	3.99140557224369e-06	\\
3903.38245738636	7.1770146034025e-06	\\
3904.36123934659	6.12408650947004e-06	\\
3905.34002130682	6.45685479606579e-06	\\
3906.31880326705	6.92685376715838e-06	\\
3907.29758522727	8.51197866191796e-06	\\
3908.2763671875	8.4941538356873e-06	\\
3909.25514914773	8.89650135743458e-06	\\
3910.23393110795	8.8920378316292e-06	\\
3911.21271306818	9.75989038802006e-06	\\
3912.19149502841	7.52718260075098e-06	\\
3913.17027698864	8.71212486740328e-06	\\
3914.14905894886	6.93701805232434e-06	\\
3915.12784090909	8.32150091134132e-06	\\
3916.10662286932	1.04606503828543e-05	\\
3917.08540482955	9.0553196361685e-06	\\
3918.06418678977	6.79295597503101e-06	\\
3919.04296875	8.40503129324333e-06	\\
3920.02175071023	9.52941769374174e-06	\\
3921.00053267045	9.52002547864409e-06	\\
3921.97931463068	9.6174055187616e-06	\\
3922.95809659091	1.13283453980459e-05	\\
3923.93687855114	8.03908908924064e-06	\\
3924.91566051136	9.48739426815318e-06	\\
3925.89444247159	8.1827516288427e-06	\\
3926.87322443182	6.53964303345654e-06	\\
3927.85200639205	1.05597369284565e-05	\\
3928.83078835227	8.70486356456537e-06	\\
3929.8095703125	7.51472631717706e-06	\\
3930.78835227273	8.61116182301095e-06	\\
3931.76713423295	1.04082480173022e-05	\\
3932.74591619318	1.09164889536639e-05	\\
3933.72469815341	9.6098342386548e-06	\\
3934.70348011364	6.01012391083549e-06	\\
3935.68226207386	9.42422162956417e-06	\\
3936.66104403409	9.96637091793384e-06	\\
3937.63982599432	7.78673408654932e-06	\\
3938.61860795455	9.93937139144146e-06	\\
3939.59738991477	1.11433008466709e-05	\\
3940.576171875	8.15936566244288e-06	\\
3941.55495383523	1.00230663847045e-05	\\
3942.53373579545	8.45404381876273e-06	\\
3943.51251775568	9.51801601965828e-06	\\
3944.49129971591	5.61522095450719e-06	\\
3945.47008167614	7.01331566528407e-06	\\
3946.44886363636	5.30136176719612e-06	\\
3947.42764559659	7.61330604942051e-06	\\
3948.40642755682	9.47866328853399e-06	\\
3949.38520951705	6.73490992421154e-06	\\
3950.36399147727	9.38661591606604e-06	\\
3951.3427734375	1.08316114926778e-05	\\
3952.32155539773	9.00214686752608e-06	\\
3953.30033735795	1.00577961253319e-05	\\
3954.27911931818	8.70989616150159e-06	\\
3955.25790127841	1.07218115181319e-05	\\
3956.23668323864	1.00317598988915e-05	\\
3957.21546519886	8.06414135273062e-06	\\
3958.19424715909	8.02523297902352e-06	\\
3959.17302911932	6.39968338464726e-06	\\
3960.15181107955	8.56733497604389e-06	\\
3961.13059303977	6.10172819792929e-06	\\
3962.109375	9.39956774681559e-06	\\
3963.08815696023	9.75944796098467e-06	\\
3964.06693892045	8.83136799660985e-06	\\
3965.04572088068	9.02087107722898e-06	\\
3966.02450284091	7.7246002589774e-06	\\
3967.00328480114	9.98918437327275e-06	\\
3967.98206676136	1.0777296194703e-05	\\
3968.96084872159	8.43182268198775e-06	\\
3969.93963068182	9.94385231375369e-06	\\
3970.91841264205	1.06771772431933e-05	\\
3971.89719460227	8.99965090729867e-06	\\
3972.8759765625	8.68978602348306e-06	\\
3973.85475852273	9.12004759594979e-06	\\
3974.83354048295	1.00727680241133e-05	\\
3975.81232244318	1.07206654469847e-05	\\
3976.79110440341	1.2646003268147e-05	\\
3977.76988636364	1.177671650048e-05	\\
3978.74866832386	9.08015093504987e-06	\\
3979.72745028409	7.35387202523408e-06	\\
3980.70623224432	1.08007600881681e-05	\\
3981.68501420455	8.56034056482279e-06	\\
3982.66379616477	7.89022803672603e-06	\\
3983.642578125	8.16984183814062e-06	\\
3984.62136008523	1.11512575975874e-05	\\
3985.60014204545	9.94395968193527e-06	\\
3986.57892400568	1.15824783355002e-05	\\
3987.55770596591	8.90391422826929e-06	\\
3988.53648792614	9.94574859625661e-06	\\
3989.51526988636	1.0247734786807e-05	\\
3990.49405184659	8.60242300611007e-06	\\
3991.47283380682	9.29905156781531e-06	\\
3992.45161576705	1.05452817037713e-05	\\
3993.43039772727	9.12259928957269e-06	\\
3994.4091796875	1.03339732154899e-05	\\
3995.38796164773	1.27528488794303e-05	\\
3996.36674360795	1.22475459425331e-05	\\
3997.34552556818	1.1562265544328e-05	\\
3998.32430752841	1.12299628328105e-05	\\
3999.30308948864	1.14941370340387e-05	\\
4000.28187144886	9.92335105908427e-06	\\
4001.26065340909	1.15430363753465e-05	\\
4002.23943536932	9.50262607140787e-06	\\
4003.21821732955	1.13601488519502e-05	\\
4004.19699928977	9.25768994848809e-06	\\
4005.17578125	8.20972953857546e-06	\\
4006.15456321023	1.05280442224277e-05	\\
4007.13334517045	1.06863691103344e-05	\\
4008.11212713068	1.12125823711796e-05	\\
4009.09090909091	9.64732318838002e-06	\\
4010.06969105114	1.27057713505833e-05	\\
4011.04847301136	1.04694983230607e-05	\\
4012.02725497159	8.94603084466913e-06	\\
4013.00603693182	1.125603534379e-05	\\
4013.98481889205	1.14636299423217e-05	\\
4014.96360085227	1.02160626857584e-05	\\
4015.9423828125	9.01394529945753e-06	\\
4016.92116477273	8.36376652839303e-06	\\
4017.89994673295	1.00645063460259e-05	\\
4018.87872869318	9.8723158586036e-06	\\
4019.85751065341	1.17590881415268e-05	\\
4020.83629261364	8.7277999746391e-06	\\
4021.81507457386	1.14807036800381e-05	\\
4022.79385653409	9.29858415502736e-06	\\
4023.77263849432	1.16140590515971e-05	\\
4024.75142045455	1.04937082284295e-05	\\
4025.73020241477	9.43754362348227e-06	\\
4026.708984375	9.44809073761858e-06	\\
4027.68776633523	8.21506298264876e-06	\\
4028.66654829545	1.07147569955245e-05	\\
4029.64533025568	8.23350106508396e-06	\\
4030.62411221591	9.67481867884016e-06	\\
4031.60289417614	1.03312632466407e-05	\\
4032.58167613636	1.14913405097279e-05	\\
4033.56045809659	1.17820884917585e-05	\\
4034.53924005682	1.32647747773481e-05	\\
4035.51802201705	1.07471654914406e-05	\\
4036.49680397727	1.12555575835861e-05	\\
4037.4755859375	1.08103483322091e-05	\\
4038.45436789773	1.20013978245303e-05	\\
4039.43314985795	9.30214367614368e-06	\\
4040.41193181818	1.10572636738499e-05	\\
4041.39071377841	1.08680839187282e-05	\\
4042.36949573864	1.10880309184718e-05	\\
4043.34827769886	1.03101540928406e-05	\\
4044.32705965909	1.02524918990524e-05	\\
4045.30584161932	1.20084980383104e-05	\\
4046.28462357955	1.04541673662591e-05	\\
4047.26340553977	1.09471496444522e-05	\\
4048.2421875	9.91518324562202e-06	\\
4049.22096946023	1.45390729321661e-05	\\
4050.19975142045	1.1127234428636e-05	\\
4051.17853338068	1.07568740057213e-05	\\
4052.15731534091	1.12773514435456e-05	\\
4053.13609730114	1.2933323354412e-05	\\
4054.11487926136	1.24270564090067e-05	\\
4055.09366122159	1.17243630609351e-05	\\
4056.07244318182	1.13518486136589e-05	\\
4057.05122514205	1.18959466518834e-05	\\
4058.03000710227	1.31016891098849e-05	\\
4059.0087890625	1.34322156221587e-05	\\
4059.98757102273	1.17504474362056e-05	\\
4060.96635298295	1.21767838824387e-05	\\
4061.94513494318	1.20138327810634e-05	\\
4062.92391690341	1.03332142790475e-05	\\
4063.90269886364	1.30909492257425e-05	\\
4064.88148082386	8.4693975260634e-06	\\
4065.86026278409	7.9134922870763e-06	\\
4066.83904474432	1.28937522703447e-05	\\
4067.81782670455	1.13983461314451e-05	\\
4068.79660866477	1.11634155826021e-05	\\
4069.775390625	1.26702043880495e-05	\\
4070.75417258523	8.37289271574246e-06	\\
4071.73295454545	1.26777763874439e-05	\\
4072.71173650568	1.16001491601473e-05	\\
4073.69051846591	1.0544103149984e-05	\\
4074.66930042614	8.90803897045945e-06	\\
4075.64808238636	1.0543431237054e-05	\\
4076.62686434659	8.25319018861143e-06	\\
4077.60564630682	1.12066791840553e-05	\\
4078.58442826705	9.60179398453324e-06	\\
4079.56321022727	1.08061190698275e-05	\\
4080.5419921875	1.26607554335241e-05	\\
4081.52077414773	9.78341245881712e-06	\\
4082.49955610795	1.10860896801584e-05	\\
4083.47833806818	1.05669926506852e-05	\\
4084.45712002841	1.08322707483025e-05	\\
4085.43590198864	1.04694896379022e-05	\\
4086.41468394886	1.25416814093437e-05	\\
4087.39346590909	1.30481047570285e-05	\\
4088.37224786932	1.10843177612133e-05	\\
4089.35102982955	1.09088624578118e-05	\\
4090.32981178977	1.02020168269994e-05	\\
4091.30859375	8.84074522250661e-06	\\
4092.28737571023	1.00934344108589e-05	\\
4093.26615767045	1.1372544370678e-05	\\
4094.24493963068	1.40315979125516e-05	\\
4095.22372159091	1.36890116401201e-05	\\
4096.20250355114	1.08974118103334e-05	\\
4097.18128551136	1.10155977052829e-05	\\
4098.16006747159	1.10838969166808e-05	\\
4099.13884943182	1.04555163723671e-05	\\
4100.11763139205	1.14368446292876e-05	\\
4101.09641335227	1.09068804544851e-05	\\
4102.0751953125	9.07345879996485e-06	\\
4103.05397727273	1.14790526107521e-05	\\
4104.03275923295	1.16934890609912e-05	\\
4105.01154119318	1.31514687920957e-05	\\
4105.99032315341	1.19235766930969e-05	\\
4106.96910511364	9.89490029265186e-06	\\
4107.94788707386	1.14160107561145e-05	\\
4108.92666903409	1.08525571896947e-05	\\
4109.90545099432	1.12495251540513e-05	\\
4110.88423295455	1.08342294252425e-05	\\
4111.86301491477	1.03990078034365e-05	\\
4112.841796875	1.09455897370838e-05	\\
4113.82057883523	8.9176105023947e-06	\\
4114.79936079545	1.09039478835185e-05	\\
4115.77814275568	1.02948771997733e-05	\\
4116.75692471591	1.10945983672383e-05	\\
4117.73570667614	1.06648882877419e-05	\\
4118.71448863636	1.1021688294297e-05	\\
4119.69327059659	1.08063666575099e-05	\\
4120.67205255682	1.16363053061724e-05	\\
4121.65083451705	1.13749391157078e-05	\\
4122.62961647727	9.668670728909e-06	\\
4123.6083984375	7.68380010720509e-06	\\
4124.58718039773	1.15672893975283e-05	\\
4125.56596235795	9.29390186755175e-06	\\
4126.54474431818	1.02804085182505e-05	\\
4127.52352627841	1.16934786535151e-05	\\
4128.50230823864	8.71166830250762e-06	\\
4129.48109019886	1.0974639773991e-05	\\
4130.45987215909	1.24758584067877e-05	\\
4131.43865411932	1.14427031803659e-05	\\
4132.41743607955	9.80721189088986e-06	\\
4133.39621803977	1.16947672497275e-05	\\
4134.375	8.94449036491734e-06	\\
4135.35378196023	9.19783072486974e-06	\\
4136.33256392045	8.47442539654971e-06	\\
4137.31134588068	9.41892882935086e-06	\\
4138.29012784091	1.08448168063588e-05	\\
4139.26890980114	8.51858338733507e-06	\\
4140.24769176136	1.27127612831837e-05	\\
4141.22647372159	9.40697643708478e-06	\\
4142.20525568182	1.17405770376332e-05	\\
4143.18403764205	1.19156136283258e-05	\\
4144.16281960227	1.06710309805444e-05	\\
4145.1416015625	1.05397921328088e-05	\\
4146.12038352273	9.09998962612039e-06	\\
4147.09916548295	1.186392395498e-05	\\
4148.07794744318	1.33377046654394e-05	\\
4149.05672940341	1.08691931018388e-05	\\
4150.03551136364	1.1742406087558e-05	\\
4151.01429332386	1.25895087817326e-05	\\
4151.99307528409	1.0125698405106e-05	\\
4152.97185724432	1.21040979738762e-05	\\
4153.95063920455	1.23180699580101e-05	\\
4154.92942116477	9.80306346587507e-06	\\
4155.908203125	9.34759427699646e-06	\\
4156.88698508523	1.04753387405038e-05	\\
4157.86576704545	1.24703118674464e-05	\\
4158.84454900568	1.19460322648927e-05	\\
4159.82333096591	1.14703338170725e-05	\\
4160.80211292614	1.412569659052e-05	\\
4161.78089488636	1.32900938706933e-05	\\
4162.75967684659	1.1590257875549e-05	\\
4163.73845880682	1.21805594007441e-05	\\
4164.71724076705	1.20952654002389e-05	\\
4165.69602272727	1.14904804066785e-05	\\
4166.6748046875	1.25087513433873e-05	\\
4167.65358664773	1.18732659030451e-05	\\
4168.63236860795	1.17460656410257e-05	\\
4169.61115056818	1.08648343702586e-05	\\
4170.58993252841	1.43791564378082e-05	\\
4171.56871448864	1.06686928848553e-05	\\
4172.54749644886	1.26840151474693e-05	\\
4173.52627840909	1.08655001824134e-05	\\
4174.50506036932	8.55057993904553e-06	\\
4175.48384232955	1.18117088736481e-05	\\
4176.46262428977	1.19025424010018e-05	\\
4177.44140625	1.14026867260127e-05	\\
4178.42018821023	9.29075982669993e-06	\\
4179.39897017045	1.32819980037166e-05	\\
4180.37775213068	9.97445680456294e-06	\\
4181.35653409091	1.19414208214525e-05	\\
4182.33531605114	1.2219467266745e-05	\\
4183.31409801136	1.35924560598968e-05	\\
4184.29287997159	1.22671986836403e-05	\\
4185.27166193182	1.19250862424839e-05	\\
4186.25044389205	1.08657075123939e-05	\\
4187.22922585227	1.01134640049407e-05	\\
4188.2080078125	1.32096553235174e-05	\\
4189.18678977273	1.11002929057443e-05	\\
4190.16557173295	1.43202058356213e-05	\\
4191.14435369318	1.18289928145772e-05	\\
4192.12313565341	1.32874614363111e-05	\\
4193.10191761364	1.22388140734921e-05	\\
4194.08069957386	1.0392153031901e-05	\\
4195.05948153409	1.2838390771041e-05	\\
4196.03826349432	1.11188893169775e-05	\\
4197.01704545455	1.06494413822352e-05	\\
4197.99582741477	1.06193570727773e-05	\\
4198.974609375	1.24144725485177e-05	\\
4199.95339133523	1.33156790302718e-05	\\
4200.93217329545	1.19687288378638e-05	\\
4201.91095525568	1.19556614973088e-05	\\
4202.88973721591	1.14729608501828e-05	\\
4203.86851917614	1.14994153272405e-05	\\
4204.84730113636	1.0766279364051e-05	\\
4205.82608309659	1.26179251537947e-05	\\
4206.80486505682	1.18985329215013e-05	\\
4207.78364701705	1.07453253417655e-05	\\
4208.76242897727	1.1457484157346e-05	\\
4209.7412109375	1.0504666282009e-05	\\
4210.71999289773	1.07371645170878e-05	\\
4211.69877485795	1.15349841631701e-05	\\
4212.67755681818	1.24912346831755e-05	\\
4213.65633877841	1.31482729540747e-05	\\
4214.63512073864	1.35668215187657e-05	\\
4215.61390269886	1.36560061278686e-05	\\
4216.59268465909	1.23544872080358e-05	\\
4217.57146661932	1.37849459045128e-05	\\
4218.55024857955	1.29814590539764e-05	\\
4219.52903053977	1.17234396053416e-05	\\
4220.5078125	1.18212306509521e-05	\\
4221.48659446023	1.25542502392762e-05	\\
4222.46537642045	1.31841919247081e-05	\\
4223.44415838068	1.370843903221e-05	\\
4224.42294034091	1.31072778007384e-05	\\
4225.40172230114	1.3529956393443e-05	\\
4226.38050426136	1.1762272668596e-05	\\
4227.35928622159	1.13684870063481e-05	\\
4228.33806818182	1.07513757938244e-05	\\
4229.31685014205	1.45750433753446e-05	\\
4230.29563210227	1.21988355281096e-05	\\
4231.2744140625	1.28872964358222e-05	\\
4232.25319602273	9.72221854442556e-06	\\
4233.23197798295	1.08424459218431e-05	\\
4234.21075994318	1.44296327230277e-05	\\
4235.18954190341	1.19547251047829e-05	\\
4236.16832386364	1.63813914957792e-05	\\
4237.14710582386	1.28292634888181e-05	\\
4238.12588778409	1.23319883949703e-05	\\
4239.10466974432	1.36719755808748e-05	\\
4240.08345170455	1.335303992634e-05	\\
4241.06223366477	1.19503548607894e-05	\\
4242.041015625	1.36235023702166e-05	\\
4243.01979758523	1.17498565348344e-05	\\
4243.99857954545	1.31276182429458e-05	\\
4244.97736150568	1.18660576085027e-05	\\
4245.95614346591	1.3643036617237e-05	\\
4246.93492542614	1.28456536687676e-05	\\
4247.91370738636	1.31742117886756e-05	\\
4248.89248934659	1.40436634593796e-05	\\
4249.87127130682	1.22185211248293e-05	\\
4250.85005326705	1.36951974877354e-05	\\
4251.82883522727	1.47092977832458e-05	\\
4252.8076171875	1.39131866401437e-05	\\
4253.78639914773	1.19561305884976e-05	\\
4254.76518110795	1.31511264090908e-05	\\
4255.74396306818	1.35187666318313e-05	\\
4256.72274502841	1.2531896270357e-05	\\
4257.70152698864	1.2165558000099e-05	\\
4258.68030894886	1.4283729792454e-05	\\
4259.65909090909	1.70407716699945e-05	\\
4260.63787286932	1.62237288953798e-05	\\
4261.61665482955	1.49116459215433e-05	\\
4262.59543678977	1.35230497282459e-05	\\
4263.57421875	1.36490699202063e-05	\\
4264.55300071023	1.16753782091548e-05	\\
4265.53178267045	1.41150831660057e-05	\\
4266.51056463068	1.66510976411546e-05	\\
4267.48934659091	1.32672056960965e-05	\\
4268.46812855114	1.56859029355873e-05	\\
4269.44691051136	1.53397173673237e-05	\\
4270.42569247159	1.4750011846666e-05	\\
4271.40447443182	1.28574660870952e-05	\\
4272.38325639205	1.55265402921814e-05	\\
4273.36203835227	1.30897332616294e-05	\\
4274.3408203125	1.40065144523935e-05	\\
4275.31960227273	1.5091913839565e-05	\\
4276.29838423295	1.46477875520175e-05	\\
4277.27716619318	1.30257328670778e-05	\\
4278.25594815341	1.65897813082253e-05	\\
4279.23473011364	1.19337207919821e-05	\\
4280.21351207386	1.24089026850587e-05	\\
4281.19229403409	1.36363315157979e-05	\\
4282.17107599432	1.46705876271801e-05	\\
4283.14985795455	1.48519912066101e-05	\\
4284.12863991477	1.47700925679563e-05	\\
4285.107421875	1.1494655095981e-05	\\
4286.08620383523	1.49369620665036e-05	\\
4287.06498579545	1.35053682117847e-05	\\
4288.04376775568	1.26375099924106e-05	\\
4289.02254971591	1.44529289144466e-05	\\
4290.00133167614	1.40205779215077e-05	\\
4290.98011363636	1.2775481960122e-05	\\
4291.95889559659	1.44705073782445e-05	\\
4292.93767755682	1.22472828879235e-05	\\
4293.91645951705	1.06046256489932e-05	\\
4294.89524147727	1.54590526505366e-05	\\
4295.8740234375	1.59948309106453e-05	\\
4296.85280539773	1.50770217847123e-05	\\
4297.83158735795	1.46315437229106e-05	\\
4298.81036931818	1.17833855839009e-05	\\
4299.78915127841	1.02325422504729e-05	\\
4300.76793323864	1.39653432162333e-05	\\
4301.74671519886	1.47180626228427e-05	\\
4302.72549715909	1.29243280653397e-05	\\
4303.70427911932	1.30261981585436e-05	\\
4304.68306107955	1.32364651371891e-05	\\
4305.66184303977	1.35780023154777e-05	\\
4306.640625	1.17224714274689e-05	\\
4307.61940696023	1.41482281601716e-05	\\
4308.59818892045	1.20121313208458e-05	\\
4309.57697088068	1.32245338539851e-05	\\
4310.55575284091	1.58373383511507e-05	\\
4311.53453480114	1.40742420782767e-05	\\
4312.51331676136	1.4617525899897e-05	\\
4313.49209872159	1.40701277151751e-05	\\
4314.47088068182	1.49667791455175e-05	\\
4315.44966264205	1.12870184375547e-05	\\
4316.42844460227	1.40176471352453e-05	\\
4317.4072265625	1.17511926001574e-05	\\
4318.38600852273	1.4510674319956e-05	\\
4319.36479048295	1.30772966411557e-05	\\
4320.34357244318	1.3862859818126e-05	\\
4321.32235440341	1.34252233072938e-05	\\
4322.30113636364	1.43020730643358e-05	\\
4323.27991832386	1.36702847559731e-05	\\
4324.25870028409	1.90941317251519e-05	\\
4325.23748224432	1.32270708035396e-05	\\
4326.21626420455	1.39525310627767e-05	\\
4327.19504616477	1.53713605636459e-05	\\
4328.173828125	1.41362493845656e-05	\\
4329.15261008523	1.16699498083748e-05	\\
4330.13139204545	1.24859472466491e-05	\\
4331.11017400568	1.36795682768847e-05	\\
4332.08895596591	1.39036113380999e-05	\\
4333.06773792614	1.29767558490533e-05	\\
4334.04651988636	1.25377168846452e-05	\\
4335.02530184659	1.51043321844459e-05	\\
4336.00408380682	1.38494309388004e-05	\\
4336.98286576705	1.4138242657227e-05	\\
4337.96164772727	1.49475368926684e-05	\\
4338.9404296875	1.23863918536904e-05	\\
4339.91921164773	1.22287196020843e-05	\\
4340.89799360795	1.52071267720625e-05	\\
4341.87677556818	1.50619317450472e-05	\\
4342.85555752841	1.51929996127179e-05	\\
4343.83433948864	1.48165852552907e-05	\\
4344.81312144886	1.34384217805246e-05	\\
4345.79190340909	1.49020442409913e-05	\\
4346.77068536932	1.4436564759492e-05	\\
4347.74946732955	1.53553910839086e-05	\\
4348.72824928977	1.40678910916191e-05	\\
4349.70703125	1.60228309792272e-05	\\
4350.68581321023	1.61709973973551e-05	\\
4351.66459517045	1.26758084337496e-05	\\
4352.64337713068	1.43384706261157e-05	\\
4353.62215909091	1.37527241765393e-05	\\
4354.60094105114	1.56032971284287e-05	\\
4355.57972301136	1.388884457937e-05	\\
4356.55850497159	1.49769444512159e-05	\\
4357.53728693182	1.22036525167831e-05	\\
4358.51606889205	1.53658919333444e-05	\\
4359.49485085227	1.47041435710996e-05	\\
4360.4736328125	1.64584321523647e-05	\\
4361.45241477273	1.4449729234825e-05	\\
4362.43119673295	1.35917195698443e-05	\\
4363.40997869318	1.45735643851033e-05	\\
4364.38876065341	1.47145063827137e-05	\\
4365.36754261364	1.42518024056544e-05	\\
4366.34632457386	1.42451837060182e-05	\\
4367.32510653409	1.41129086266143e-05	\\
4368.30388849432	1.45471990026112e-05	\\
4369.28267045455	1.61825577650816e-05	\\
4370.26145241477	1.34056870474243e-05	\\
4371.240234375	1.4744145185843e-05	\\
4372.21901633523	1.44035433546429e-05	\\
4373.19779829545	1.570365475987e-05	\\
4374.17658025568	1.39028785867173e-05	\\
4375.15536221591	1.6172116922316e-05	\\
4376.13414417614	1.55095590070767e-05	\\
4377.11292613636	1.37478992634179e-05	\\
4378.09170809659	1.3587118736194e-05	\\
4379.07049005682	1.81557229921118e-05	\\
4380.04927201705	1.63217986953598e-05	\\
4381.02805397727	1.4719497328459e-05	\\
4382.0068359375	1.67034063880698e-05	\\
4382.98561789773	1.56495554676382e-05	\\
4383.96439985795	1.53999099201294e-05	\\
4384.94318181818	1.65064735279101e-05	\\
4385.92196377841	1.4724894218708e-05	\\
4386.90074573864	1.43890176975976e-05	\\
4387.87952769886	1.66854287213355e-05	\\
4388.85830965909	1.47136815070584e-05	\\
4389.83709161932	1.84772891454723e-05	\\
4390.81587357955	1.83034610066342e-05	\\
4391.79465553977	1.66555504808113e-05	\\
4392.7734375	1.49445564316832e-05	\\
4393.75221946023	1.55951505286584e-05	\\
4394.73100142045	1.73155946723708e-05	\\
4395.70978338068	1.73804917423205e-05	\\
4396.68856534091	1.58652430563737e-05	\\
4397.66734730114	1.20490500315092e-05	\\
4398.64612926136	1.5951229219214e-05	\\
4399.62491122159	1.60643770092334e-05	\\
4400.60369318182	1.63851786239871e-05	\\
4401.58247514205	1.85692258406789e-05	\\
4402.56125710227	1.56982204193047e-05	\\
4403.5400390625	1.4611186808441e-05	\\
4404.51882102273	1.57120455753236e-05	\\
4405.49760298295	1.79110846128345e-05	\\
4406.47638494318	1.38680006540444e-05	\\
4407.45516690341	1.70829121114927e-05	\\
4408.43394886364	1.73227179381713e-05	\\
4409.41273082386	1.78711996968529e-05	\\
4410.39151278409	1.65498935549474e-05	\\
4411.37029474432	1.70450324441997e-05	\\
4412.34907670455	1.59539135663402e-05	\\
4413.32785866477	1.66825094047072e-05	\\
4414.306640625	1.5148533842561e-05	\\
4415.28542258523	1.75848393796476e-05	\\
4416.26420454545	1.63081981760412e-05	\\
4417.24298650568	1.60551009156427e-05	\\
4418.22176846591	1.57412794676157e-05	\\
4419.20055042614	1.41533048830349e-05	\\
4420.17933238636	1.70243682963574e-05	\\
4421.15811434659	1.41705695578491e-05	\\
4422.13689630682	1.67067473231358e-05	\\
4423.11567826705	1.49369126953509e-05	\\
4424.09446022727	1.41353735740395e-05	\\
4425.0732421875	1.78170367574247e-05	\\
4426.05202414773	1.54684494188454e-05	\\
4427.03080610795	1.73607711346004e-05	\\
4428.00958806818	1.59473255426041e-05	\\
4428.98837002841	1.77222315639283e-05	\\
4429.96715198864	1.47641117882876e-05	\\
4430.94593394886	1.28844933442869e-05	\\
4431.92471590909	1.57076433054888e-05	\\
4432.90349786932	1.45907026814488e-05	\\
4433.88227982955	1.62628121785321e-05	\\
4434.86106178977	1.48797666272328e-05	\\
4435.83984375	1.3826371574318e-05	\\
4436.81862571023	1.60796668255039e-05	\\
4437.79740767045	1.4489881573226e-05	\\
4438.77618963068	1.54750313039777e-05	\\
4439.75497159091	1.54472480027238e-05	\\
4440.73375355114	1.60669234746963e-05	\\
4441.71253551136	1.54762145078829e-05	\\
4442.69131747159	1.45477223449352e-05	\\
4443.67009943182	1.77976910592092e-05	\\
4444.64888139205	1.59633506288687e-05	\\
4445.62766335227	1.72458054276327e-05	\\
4446.6064453125	1.69052219075642e-05	\\
4447.58522727273	1.68233815845374e-05	\\
4448.56400923295	1.732989186706e-05	\\
4449.54279119318	1.81221914341796e-05	\\
4450.52157315341	1.65290455408773e-05	\\
4451.50035511364	1.42987892532779e-05	\\
4452.47913707386	1.66036133568989e-05	\\
4453.45791903409	1.54259666583399e-05	\\
4454.43670099432	1.5455473742385e-05	\\
4455.41548295455	1.6329398365522e-05	\\
4456.39426491477	1.56773947776381e-05	\\
4457.373046875	1.52257439660771e-05	\\
4458.35182883523	1.52816632563168e-05	\\
4459.33061079545	1.34972341548035e-05	\\
4460.30939275568	1.77555278278804e-05	\\
4461.28817471591	1.49618037584951e-05	\\
4462.26695667614	1.37047686785706e-05	\\
4463.24573863636	1.6775239224629e-05	\\
4464.22452059659	1.58823291096624e-05	\\
4465.20330255682	1.50180751057702e-05	\\
4466.18208451705	1.50933689555145e-05	\\
4467.16086647727	1.3787430881827e-05	\\
4468.1396484375	1.77096016218172e-05	\\
4469.11843039773	1.70455754316298e-05	\\
4470.09721235795	1.61647663078227e-05	\\
4471.07599431818	1.63840663770239e-05	\\
4472.05477627841	1.67060451665483e-05	\\
4473.03355823864	1.53809134263647e-05	\\
4474.01234019886	1.6368675529536e-05	\\
4474.99112215909	1.47238202120624e-05	\\
4475.96990411932	1.65939350634601e-05	\\
4476.94868607955	1.62067391140196e-05	\\
4477.92746803977	1.59433977256241e-05	\\
4478.90625	1.62906242066105e-05	\\
4479.88503196023	1.88324239749026e-05	\\
4480.86381392045	1.92079577988525e-05	\\
4481.84259588068	1.95056635659587e-05	\\
4482.82137784091	1.5255982181362e-05	\\
4483.80015980114	1.7891425559367e-05	\\
4484.77894176136	1.50373447862124e-05	\\
4485.75772372159	1.60371304447972e-05	\\
4486.73650568182	1.60296479605872e-05	\\
4487.71528764205	1.36823841787485e-05	\\
4488.69406960227	1.75087617763441e-05	\\
4489.6728515625	1.75402990247154e-05	\\
4490.65163352273	1.60902855087754e-05	\\
4491.63041548295	1.68028918203611e-05	\\
4492.60919744318	1.69652116748495e-05	\\
4493.58797940341	1.74079360861591e-05	\\
4494.56676136364	1.69544876618744e-05	\\
4495.54554332386	1.73669907398786e-05	\\
4496.52432528409	1.4116624069237e-05	\\
4497.50310724432	1.63949141333622e-05	\\
4498.48188920455	1.37460346961066e-05	\\
4499.46067116477	1.3653739039242e-05	\\
4500.439453125	1.37267181778652e-05	\\
4501.41823508523	1.5313748207907e-05	\\
4502.39701704545	1.57395633977633e-05	\\
4503.37579900568	1.70329833661226e-05	\\
4504.35458096591	1.69300258811506e-05	\\
4505.33336292614	1.78315540691403e-05	\\
4506.31214488636	1.79414673791525e-05	\\
4507.29092684659	1.71279750653605e-05	\\
4508.26970880682	1.63775558800677e-05	\\
4509.24849076705	1.657557411154e-05	\\
4510.22727272727	1.54743435410433e-05	\\
4511.2060546875	1.68872131245196e-05	\\
4512.18483664773	1.48639681398667e-05	\\
4513.16361860795	1.62993403441132e-05	\\
4514.14240056818	1.23872703902068e-05	\\
4515.12118252841	1.78979566380407e-05	\\
4516.09996448864	1.66788453438626e-05	\\
4517.07874644886	1.47045696193241e-05	\\
4518.05752840909	1.57994582492039e-05	\\
4519.03631036932	1.4655575160649e-05	\\
4520.01509232955	1.71248935688981e-05	\\
4520.99387428977	1.57912254419988e-05	\\
4521.97265625	1.29696733916577e-05	\\
4522.95143821023	1.53559501799424e-05	\\
4523.93022017045	1.48734784443299e-05	\\
4524.90900213068	1.78339841925699e-05	\\
4525.88778409091	1.72807655312826e-05	\\
4526.86656605114	1.79072485313828e-05	\\
4527.84534801136	1.51368489747039e-05	\\
4528.82412997159	1.41255700878464e-05	\\
4529.80291193182	1.55620032476466e-05	\\
4530.78169389205	1.60144627237845e-05	\\
4531.76047585227	1.59516157921121e-05	\\
4532.7392578125	1.36778045025369e-05	\\
4533.71803977273	1.49562620323065e-05	\\
4534.69682173295	1.53247750510492e-05	\\
4535.67560369318	1.37848352192047e-05	\\
4536.65438565341	1.48769089238086e-05	\\
4537.63316761364	1.4870122466199e-05	\\
4538.61194957386	1.47632141201282e-05	\\
4539.59073153409	1.52423369618258e-05	\\
4540.56951349432	1.65058267832908e-05	\\
4541.54829545455	1.66199314729079e-05	\\
4542.52707741477	1.49512780452707e-05	\\
4543.505859375	1.759817649251e-05	\\
4544.48464133523	1.57738393070406e-05	\\
4545.46342329545	1.51409098775556e-05	\\
4546.44220525568	1.65070426586688e-05	\\
4547.42098721591	1.42259149635356e-05	\\
4548.39976917614	1.56480304932285e-05	\\
4549.37855113636	1.51784669509958e-05	\\
4550.35733309659	1.38963071638314e-05	\\
4551.33611505682	1.65273525093455e-05	\\
4552.31489701705	1.51087846471714e-05	\\
4553.29367897727	1.60623886922391e-05	\\
4554.2724609375	1.60772031351588e-05	\\
4555.25124289773	1.40196289391925e-05	\\
4556.23002485795	1.58406540948202e-05	\\
4557.20880681818	1.60969563017066e-05	\\
4558.18758877841	1.67297854421431e-05	\\
4559.16637073864	1.60576282931187e-05	\\
4560.14515269886	1.30392592472319e-05	\\
4561.12393465909	1.18010836443579e-05	\\
4562.10271661932	1.52978305805066e-05	\\
4563.08149857955	1.48174410072558e-05	\\
4564.06028053977	1.60385325479085e-05	\\
4565.0390625	1.56166989552795e-05	\\
4566.01784446023	1.29447006020406e-05	\\
4566.99662642045	1.52005391995863e-05	\\
4567.97540838068	1.39134861872788e-05	\\
4568.95419034091	1.44598183737345e-05	\\
4569.93297230114	1.5549653974196e-05	\\
4570.91175426136	1.37047349658784e-05	\\
4571.89053622159	1.49005197011432e-05	\\
4572.86931818182	1.40007047180244e-05	\\
4573.84810014205	1.45277059335038e-05	\\
4574.82688210227	1.29997269495931e-05	\\
4575.8056640625	1.23652744099735e-05	\\
4576.78444602273	1.44091194150583e-05	\\
4577.76322798295	1.36671030989828e-05	\\
4578.74200994318	1.3427554986411e-05	\\
4579.72079190341	1.15860589265107e-05	\\
4580.69957386364	1.36877228371749e-05	\\
4581.67835582386	1.42148597873465e-05	\\
4582.65713778409	1.574264975214e-05	\\
4583.63591974432	1.30505881848682e-05	\\
4584.61470170455	1.60352883284118e-05	\\
4585.59348366477	1.58184787163659e-05	\\
4586.572265625	1.69153396679121e-05	\\
4587.55104758523	1.55636066701422e-05	\\
4588.52982954545	1.47960387732771e-05	\\
4589.50861150568	1.43088878117981e-05	\\
4590.48739346591	1.0801770992063e-05	\\
4591.46617542614	1.28679292768547e-05	\\
4592.44495738636	1.1821004636407e-05	\\
4593.42373934659	1.35763206082845e-05	\\
4594.40252130682	1.42757856934393e-05	\\
4595.38130326705	1.53970018594342e-05	\\
4596.36008522727	1.68365162464609e-05	\\
4597.3388671875	1.58207654706145e-05	\\
4598.31764914773	1.5272837679639e-05	\\
4599.29643110795	1.22502353929052e-05	\\
4600.27521306818	9.84835666239027e-06	\\
4601.25399502841	1.02721114386681e-05	\\
4602.23277698864	1.16516618826375e-05	\\
4603.21155894886	1.14252371026593e-05	\\
4604.19034090909	1.33229011761999e-05	\\
4605.16912286932	1.46052908989918e-05	\\
4606.14790482955	1.38193710695843e-05	\\
4607.12668678977	1.38028567454699e-05	\\
4608.10546875	1.41301627381916e-05	\\
4609.08425071023	1.44336330800106e-05	\\
4610.06303267045	1.42010733041818e-05	\\
4611.04181463068	1.27103702635781e-05	\\
4612.02059659091	1.1678908497518e-05	\\
4612.99937855114	1.11324507456174e-05	\\
4613.97816051136	1.07013810153099e-05	\\
4614.95694247159	1.23782716090755e-05	\\
4615.93572443182	1.17866935245835e-05	\\
4616.91450639205	1.10220558976533e-05	\\
4617.89328835227	1.18266127217874e-05	\\
4618.8720703125	1.26012171063315e-05	\\
4619.85085227273	1.07312704399129e-05	\\
4620.82963423295	1.21055286351159e-05	\\
4621.80841619318	1.3363192312115e-05	\\
4622.78719815341	1.27597526033213e-05	\\
4623.76598011364	1.3529225388728e-05	\\
4624.74476207386	1.12312636278293e-05	\\
4625.72354403409	1.15218187373091e-05	\\
4626.70232599432	1.01131713780693e-05	\\
4627.68110795455	1.21537165415916e-05	\\
4628.65988991477	1.02172887162331e-05	\\
4629.638671875	1.02188550302004e-05	\\
4630.61745383523	1.22101958213903e-05	\\
4631.59623579545	1.22455875948801e-05	\\
4632.57501775568	1.36759074348775e-05	\\
4633.55379971591	1.04468001215063e-05	\\
4634.53258167614	1.03517130549524e-05	\\
4635.51136363636	1.16016703584717e-05	\\
};
\addplot [color=blue,solid,forget plot]
  table[row sep=crcr]{
4635.51136363636	1.16016703584717e-05	\\
4636.49014559659	1.00567691624594e-05	\\
4637.46892755682	1.17239233994675e-05	\\
4638.44770951705	1.29762489024275e-05	\\
4639.42649147727	9.79599350848733e-06	\\
4640.4052734375	1.01273224679945e-05	\\
4641.38405539773	9.11261644765067e-06	\\
4642.36283735795	9.84287688536614e-06	\\
4643.34161931818	1.13537384459511e-05	\\
4644.32040127841	1.02691583445439e-05	\\
4645.29918323864	1.29197380649567e-05	\\
4646.27796519886	1.10811276080848e-05	\\
4647.25674715909	1.08267905609028e-05	\\
4648.23552911932	1.2818099977866e-05	\\
4649.21431107955	1.12232098721855e-05	\\
4650.19309303977	1.09387413147018e-05	\\
4651.171875	7.01680503362248e-06	\\
4652.15065696023	6.69736819938051e-06	\\
4653.12943892045	9.40383841954944e-06	\\
4654.10822088068	1.0111667558442e-05	\\
4655.08700284091	1.28795912441028e-05	\\
4656.06578480114	8.77352397271198e-06	\\
4657.04456676136	1.12200299565403e-05	\\
4658.02334872159	1.01930645795204e-05	\\
4659.00213068182	1.182848813933e-05	\\
4659.98091264205	1.17043116897066e-05	\\
4660.95969460227	1.10520440441331e-05	\\
4661.9384765625	9.93329704530005e-06	\\
4662.91725852273	9.46056446002972e-06	\\
4663.89604048295	1.00594519441445e-05	\\
4664.87482244318	9.72992543151683e-06	\\
4665.85360440341	1.03497360387337e-05	\\
4666.83238636364	1.05034526711286e-05	\\
4667.81116832386	1.07043800430847e-05	\\
4668.78995028409	9.65537950036649e-06	\\
4669.76873224432	1.19097217000777e-05	\\
4670.74751420455	8.90525546780543e-06	\\
4671.72629616477	9.30564431663402e-06	\\
4672.705078125	9.21354868387791e-06	\\
4673.68386008523	8.84412252511396e-06	\\
4674.66264204545	9.175512587855e-06	\\
4675.64142400568	9.31121827026855e-06	\\
4676.62020596591	8.69535592092008e-06	\\
4677.59898792614	9.39067446326479e-06	\\
4678.57776988636	1.03890304655039e-05	\\
4679.55655184659	1.12938696973068e-05	\\
4680.53533380682	9.7824111196514e-06	\\
4681.51411576705	9.62133424293075e-06	\\
4682.49289772727	1.07194226272422e-05	\\
4683.4716796875	1.13396062671045e-05	\\
4684.45046164773	1.04649585811095e-05	\\
4685.42924360795	1.28659563855421e-05	\\
4686.40802556818	1.00377316207757e-05	\\
4687.38680752841	1.02960363675484e-05	\\
4688.36558948864	9.0768880080296e-06	\\
4689.34437144886	1.12482678017824e-05	\\
4690.32315340909	1.04314970525828e-05	\\
4691.30193536932	8.64068014971036e-06	\\
4692.28071732955	1.158714229717e-05	\\
4693.25949928977	1.0018566775943e-05	\\
4694.23828125	9.15382685371652e-06	\\
4695.21706321023	1.14524057342088e-05	\\
4696.19584517045	9.83387858060986e-06	\\
4697.17462713068	1.04078648642201e-05	\\
4698.15340909091	9.60968069560253e-06	\\
4699.13219105114	1.04086579656771e-05	\\
4700.11097301136	1.04320959160327e-05	\\
4701.08975497159	1.11262197392301e-05	\\
4702.06853693182	1.38550607633288e-05	\\
4703.04731889205	1.24898404374073e-05	\\
4704.02610085227	1.20103646336524e-05	\\
4705.0048828125	9.91536706058716e-06	\\
4705.98366477273	1.04610921018053e-05	\\
4706.96244673295	9.65850318926734e-06	\\
4707.94122869318	1.191118672893e-05	\\
4708.92001065341	1.13614854922779e-05	\\
4709.89879261364	9.02569191070446e-06	\\
4710.87757457386	1.21195505092015e-05	\\
4711.85635653409	1.00561523082711e-05	\\
4712.83513849432	9.2083377570879e-06	\\
4713.81392045455	9.61240152172669e-06	\\
4714.79270241477	1.03186637098628e-05	\\
4715.771484375	1.00294503736709e-05	\\
4716.75026633523	9.09492028421486e-06	\\
4717.72904829545	9.04857957856564e-06	\\
4718.70783025568	1.07738367407501e-05	\\
4719.68661221591	9.24899383479051e-06	\\
4720.66539417614	7.79742325054341e-06	\\
4721.64417613636	9.80420665420544e-06	\\
4722.62295809659	1.21323270822959e-05	\\
4723.60174005682	9.51810100338754e-06	\\
4724.58052201705	1.07392754877583e-05	\\
4725.55930397727	1.21667163357676e-05	\\
4726.5380859375	9.98925936180515e-06	\\
4727.51686789773	1.17402666373427e-05	\\
4728.49564985795	9.91246182203302e-06	\\
4729.47443181818	8.24645103891086e-06	\\
4730.45321377841	8.11162652781963e-06	\\
4731.43199573864	9.43321008409998e-06	\\
4732.41077769886	9.50519955523024e-06	\\
4733.38955965909	9.59532679802444e-06	\\
4734.36834161932	1.05681028244615e-05	\\
4735.34712357955	1.23782483054906e-05	\\
4736.32590553977	1.22726712389289e-05	\\
4737.3046875	1.3982163813786e-05	\\
4738.28346946023	1.10798078120563e-05	\\
4739.26225142045	1.03134086987951e-05	\\
4740.24103338068	9.91017890608127e-06	\\
4741.21981534091	9.55223410265336e-06	\\
4742.19859730114	6.93044796433632e-06	\\
4743.17737926136	7.01004157103132e-06	\\
4744.15616122159	8.62212682363643e-06	\\
4745.13494318182	9.57629968119853e-06	\\
4746.11372514205	9.40101613075242e-06	\\
4747.09250710227	1.10268038550139e-05	\\
4748.0712890625	1.10946367846244e-05	\\
4749.05007102273	1.31074677436304e-05	\\
4750.02885298295	1.45395999379348e-05	\\
4751.00763494318	1.41621088642122e-05	\\
4751.98641690341	1.4132465736522e-05	\\
4752.96519886364	1.02158016210396e-05	\\
4753.94398082386	1.20965649110057e-05	\\
4754.92276278409	9.83178044977691e-06	\\
4755.90154474432	9.21071728089462e-06	\\
4756.88032670455	1.11075178790162e-05	\\
4757.85910866477	1.17866097629698e-05	\\
4758.837890625	1.12701980157286e-05	\\
4759.81667258523	1.10194974621012e-05	\\
4760.79545454545	1.22438359877588e-05	\\
4761.77423650568	9.91873733666837e-06	\\
4762.75301846591	1.17452228976147e-05	\\
4763.73180042614	1.21667982123073e-05	\\
4764.71058238636	1.09031553434492e-05	\\
4765.68936434659	1.2305679030704e-05	\\
4766.66814630682	9.71825236608622e-06	\\
4767.64692826705	1.20177049616716e-05	\\
4768.62571022727	1.21244082138877e-05	\\
4769.6044921875	1.03230713617182e-05	\\
4770.58327414773	1.08809525043841e-05	\\
4771.56205610795	8.28632370082625e-06	\\
4772.54083806818	1.09993344668004e-05	\\
4773.51962002841	9.66162256469933e-06	\\
4774.49840198864	1.05041475568944e-05	\\
4775.47718394886	1.23872889763587e-05	\\
4776.45596590909	1.09369771946297e-05	\\
4777.43474786932	1.03418645724996e-05	\\
4778.41352982955	1.35845140926982e-05	\\
4779.39231178977	1.30996199651408e-05	\\
4780.37109375	1.10024209958498e-05	\\
4781.34987571023	1.12191195178717e-05	\\
4782.32865767045	1.15982744313325e-05	\\
4783.30743963068	1.09205808505396e-05	\\
4784.28622159091	8.91487180517925e-06	\\
4785.26500355114	9.76345089212971e-06	\\
4786.24378551136	1.03473568609947e-05	\\
4787.22256747159	1.10389552582046e-05	\\
4788.20134943182	1.03198618506469e-05	\\
4789.18013139205	1.0854192160534e-05	\\
4790.15891335227	9.20212819244525e-06	\\
4791.1376953125	1.0569000375177e-05	\\
4792.11647727273	1.13117315053345e-05	\\
4793.09525923295	1.07424833140186e-05	\\
4794.07404119318	1.10618865491124e-05	\\
4795.05282315341	1.08244209257709e-05	\\
4796.03160511364	1.05187759879279e-05	\\
4797.01038707386	1.08633390189706e-05	\\
4797.98916903409	1.00129340196054e-05	\\
4798.96795099432	1.10880570925187e-05	\\
4799.94673295455	1.29487283535644e-05	\\
4800.92551491477	8.35353141977128e-06	\\
4801.904296875	9.32165461373686e-06	\\
4802.88307883523	1.13097129733085e-05	\\
4803.86186079545	9.68229368046728e-06	\\
4804.84064275568	1.17213944453319e-05	\\
4805.81942471591	1.05882380501137e-05	\\
4806.79820667614	1.14332902812258e-05	\\
4807.77698863636	1.10203244876839e-05	\\
4808.75577059659	8.75683655415047e-06	\\
4809.73455255682	8.97275415638386e-06	\\
4810.71333451705	1.03741877452897e-05	\\
4811.69211647727	1.00669266741402e-05	\\
4812.6708984375	9.68661759812314e-06	\\
4813.64968039773	1.060739157072e-05	\\
4814.62846235795	1.10117700607381e-05	\\
4815.60724431818	1.24918296327377e-05	\\
4816.58602627841	1.02618759859223e-05	\\
4817.56480823864	1.45602417340911e-05	\\
4818.54359019886	8.69964991594862e-06	\\
4819.52237215909	1.13655371872356e-05	\\
4820.50115411932	1.11286792786862e-05	\\
4821.47993607955	1.04622551773851e-05	\\
4822.45871803977	1.15598039645713e-05	\\
4823.4375	1.17391838923043e-05	\\
4824.41628196023	1.10356683237411e-05	\\
4825.39506392045	1.03559910196801e-05	\\
4826.37384588068	1.11220621361489e-05	\\
4827.35262784091	1.15204166114314e-05	\\
4828.33140980114	1.24267595118362e-05	\\
4829.31019176136	1.01643830959776e-05	\\
4830.28897372159	1.15461846707481e-05	\\
4831.26775568182	8.96544964765991e-06	\\
4832.24653764205	1.16646223487682e-05	\\
4833.22531960227	1.08696766674155e-05	\\
4834.2041015625	1.23017798537662e-05	\\
4835.18288352273	1.17256556564389e-05	\\
4836.16166548295	1.3527920060554e-05	\\
4837.14044744318	1.22880139570411e-05	\\
4838.11922940341	1.06631743447384e-05	\\
4839.09801136364	1.21732337384922e-05	\\
4840.07679332386	9.3260400207255e-06	\\
4841.05557528409	1.12346130225655e-05	\\
4842.03435724432	1.03669055016429e-05	\\
4843.01313920455	1.20307209851029e-05	\\
4843.99192116477	1.15153771234104e-05	\\
4844.970703125	1.06879858481987e-05	\\
4845.94948508523	9.91231286861612e-06	\\
4846.92826704545	1.03519540294863e-05	\\
4847.90704900568	9.7865178250415e-06	\\
4848.88583096591	1.20064387624106e-05	\\
4849.86461292614	1.38807895047779e-05	\\
4850.84339488636	9.10054843246285e-06	\\
4851.82217684659	1.04500739254578e-05	\\
4852.80095880682	1.26945590594675e-05	\\
4853.77974076705	1.13547060984898e-05	\\
4854.75852272727	8.92946961025094e-06	\\
4855.7373046875	8.14805620117427e-06	\\
4856.71608664773	1.23970564336567e-05	\\
4857.69486860795	1.01617211105331e-05	\\
4858.67365056818	1.13564935000181e-05	\\
4859.65243252841	8.98889117447824e-06	\\
4860.63121448864	9.12490803927217e-06	\\
4861.60999644886	1.27682048825806e-05	\\
4862.58877840909	1.27803617761564e-05	\\
4863.56756036932	1.09565641848061e-05	\\
4864.54634232955	1.14976834707083e-05	\\
4865.52512428977	1.13795554704073e-05	\\
4866.50390625	8.5635309014844e-06	\\
4867.48268821023	9.66340036402348e-06	\\
4868.46147017045	1.03532833207527e-05	\\
4869.44025213068	9.04190417707405e-06	\\
4870.41903409091	1.06500703332869e-05	\\
4871.39781605114	9.81048954618568e-06	\\
4872.37659801136	1.1513386507245e-05	\\
4873.35537997159	1.12042320581002e-05	\\
4874.33416193182	1.26112268998373e-05	\\
4875.31294389205	1.2279658487588e-05	\\
4876.29172585227	1.00056392677533e-05	\\
4877.2705078125	1.11935094566775e-05	\\
4878.24928977273	1.19095612715543e-05	\\
4879.22807173295	8.5869393268662e-06	\\
4880.20685369318	1.06873483915012e-05	\\
4881.18563565341	1.08835298526768e-05	\\
4882.16441761364	1.10003290896562e-05	\\
4883.14319957386	9.1954623648048e-06	\\
4884.12198153409	1.13336179436171e-05	\\
4885.10076349432	1.14396237600639e-05	\\
4886.07954545455	9.45919155856541e-06	\\
4887.05832741477	1.06439957297945e-05	\\
4888.037109375	9.33366949619457e-06	\\
4889.01589133523	9.35283717434455e-06	\\
4889.99467329545	1.07396655954383e-05	\\
4890.97345525568	7.76899831967786e-06	\\
4891.95223721591	1.24706027936533e-05	\\
4892.93101917614	1.12637930066908e-05	\\
4893.90980113636	1.19824365057878e-05	\\
4894.88858309659	9.52660526143369e-06	\\
4895.86736505682	1.02311044769428e-05	\\
4896.84614701705	9.47719813960598e-06	\\
4897.82492897727	9.85573410254756e-06	\\
4898.8037109375	1.16347398398049e-05	\\
4899.78249289773	9.08058913816619e-06	\\
4900.76127485795	1.21722497827136e-05	\\
4901.74005681818	9.48222589878468e-06	\\
4902.71883877841	9.52140483424217e-06	\\
4903.69762073864	9.38754994361692e-06	\\
4904.67640269886	9.26701815150679e-06	\\
4905.65518465909	7.46455022890146e-06	\\
4906.63396661932	1.00429700202881e-05	\\
4907.61274857955	9.58012163020546e-06	\\
4908.59153053977	9.36291469743398e-06	\\
4909.5703125	1.0095013556751e-05	\\
4910.54909446023	9.35459196422183e-06	\\
4911.52787642045	1.02450587109124e-05	\\
4912.50665838068	1.03935952104458e-05	\\
4913.48544034091	9.9699619130871e-06	\\
4914.46422230114	1.02350403350423e-05	\\
4915.44300426136	1.01335341425838e-05	\\
4916.42178622159	7.69121067746188e-06	\\
4917.40056818182	8.52945075248364e-06	\\
4918.37935014205	9.38115206518869e-06	\\
4919.35813210227	9.7563198315933e-06	\\
4920.3369140625	8.27392188096918e-06	\\
4921.31569602273	9.4237185818296e-06	\\
4922.29447798295	9.43154432989536e-06	\\
4923.27325994318	8.42558072749034e-06	\\
4924.25204190341	9.8740123996761e-06	\\
4925.23082386364	7.86588333379331e-06	\\
4926.20960582386	9.73329659527453e-06	\\
4927.18838778409	6.77983082506897e-06	\\
4928.16716974432	9.89872825326578e-06	\\
4929.14595170455	9.88879828959616e-06	\\
4930.12473366477	7.57232487659181e-06	\\
4931.103515625	7.88436123924725e-06	\\
4932.08229758523	1.08276996859015e-05	\\
4933.06107954545	9.34100345639326e-06	\\
4934.03986150568	7.51002514953506e-06	\\
4935.01864346591	9.66778584066736e-06	\\
4935.99742542614	8.28473723427957e-06	\\
4936.97620738636	9.63929153944974e-06	\\
4937.95498934659	9.81083246565566e-06	\\
4938.93377130682	9.86838752617485e-06	\\
4939.91255326705	8.16457186308285e-06	\\
4940.89133522727	9.44889642615394e-06	\\
4941.8701171875	8.94885812544027e-06	\\
4942.84889914773	8.1116978200214e-06	\\
4943.82768110795	1.04136860826686e-05	\\
4944.80646306818	6.2848032898334e-06	\\
4945.78524502841	8.20730692507965e-06	\\
4946.76402698864	9.91858282803304e-06	\\
4947.74280894886	9.64711697494598e-06	\\
4948.72159090909	8.72074610505896e-06	\\
4949.70037286932	5.87922198131815e-06	\\
4950.67915482955	7.21098664852924e-06	\\
4951.65793678977	8.46596689857982e-06	\\
4952.63671875	7.87331167247765e-06	\\
4953.61550071023	8.05940572143585e-06	\\
4954.59428267045	9.19664868362562e-06	\\
4955.57306463068	9.83760715296547e-06	\\
4956.55184659091	9.19785303142549e-06	\\
4957.53062855114	8.67518858016521e-06	\\
4958.50941051136	9.54625057219927e-06	\\
4959.48819247159	9.16433184968245e-06	\\
4960.46697443182	6.28398523171834e-06	\\
4961.44575639205	7.8553053842271e-06	\\
4962.42453835227	7.07356559429033e-06	\\
4963.4033203125	7.32931854565119e-06	\\
4964.38210227273	4.86200681376202e-06	\\
4965.36088423295	7.67370375371776e-06	\\
4966.33966619318	6.45262017300932e-06	\\
4967.31844815341	7.48137441927013e-06	\\
4968.29723011364	1.00697878709285e-05	\\
4969.27601207386	8.81617687465272e-06	\\
4970.25479403409	8.90195560440437e-06	\\
4971.23357599432	7.41551417493428e-06	\\
4972.21235795455	9.36222436421921e-06	\\
4973.19113991477	7.72307267066434e-06	\\
4974.169921875	6.83579874401634e-06	\\
4975.14870383523	6.94351122969324e-06	\\
4976.12748579545	9.57152728174064e-06	\\
4977.10626775568	6.11887215601011e-06	\\
4978.08504971591	6.70967265740833e-06	\\
4979.06383167614	7.35074413824726e-06	\\
4980.04261363636	6.36913798696135e-06	\\
4981.02139559659	8.18796424239227e-06	\\
4982.00017755682	9.5958711165143e-06	\\
4982.97895951705	8.89464623297542e-06	\\
4983.95774147727	5.80869309359245e-06	\\
4984.9365234375	8.11396251052671e-06	\\
4985.91530539773	7.07926841440945e-06	\\
4986.89408735795	6.46067491723915e-06	\\
4987.87286931818	8.35615572151019e-06	\\
4988.85165127841	7.50450920888504e-06	\\
4989.83043323864	7.59405938986297e-06	\\
4990.80921519886	7.00611685284259e-06	\\
4991.78799715909	6.65329071920974e-06	\\
4992.76677911932	7.02031502049983e-06	\\
4993.74556107955	6.75951569541622e-06	\\
4994.72434303977	7.65543473544417e-06	\\
4995.703125	6.99476151549582e-06	\\
4996.68190696023	7.41928803340327e-06	\\
4997.66068892045	9.05157468224637e-06	\\
4998.63947088068	7.60021169317726e-06	\\
4999.61825284091	7.75487304096141e-06	\\
5000.59703480114	6.78942588010952e-06	\\
5001.57581676136	6.99246664809339e-06	\\
5002.55459872159	6.96271197114398e-06	\\
5003.53338068182	9.45883058408736e-06	\\
5004.51216264205	8.39072490937106e-06	\\
5005.49094460227	8.7598201737315e-06	\\
5006.4697265625	6.82258678879628e-06	\\
5007.44850852273	8.03949508358043e-06	\\
5008.42729048295	7.04178643914382e-06	\\
5009.40607244318	8.45416791743408e-06	\\
5010.38485440341	7.71384378444748e-06	\\
5011.36363636364	7.30881903908967e-06	\\
5012.34241832386	5.73301825160849e-06	\\
5013.32120028409	8.44501335984426e-06	\\
5014.29998224432	6.44973714048103e-06	\\
5015.27876420455	7.63010698836545e-06	\\
5016.25754616477	6.44527688078739e-06	\\
5017.236328125	6.26930068725712e-06	\\
5018.21511008523	6.38741520775821e-06	\\
5019.19389204545	8.62618077641628e-06	\\
5020.17267400568	6.6221740768135e-06	\\
5021.15145596591	6.35672999970194e-06	\\
5022.13023792614	8.94250121856011e-06	\\
5023.10901988636	6.18697721002216e-06	\\
5024.08780184659	8.53597644343846e-06	\\
5025.06658380682	8.4125922907876e-06	\\
5026.04536576705	8.15931209834887e-06	\\
5027.02414772727	6.48527354518384e-06	\\
5028.0029296875	5.81302101507151e-06	\\
5028.98171164773	8.7260499621603e-06	\\
5029.96049360795	6.12363618444592e-06	\\
5030.93927556818	7.98093954096452e-06	\\
5031.91805752841	8.69046804805974e-06	\\
5032.89683948864	4.99525580434888e-06	\\
5033.87562144886	5.43072317442514e-06	\\
5034.85440340909	7.79498847021578e-06	\\
5035.83318536932	7.13651794708593e-06	\\
5036.81196732955	8.37707922416941e-06	\\
5037.79074928977	7.0976606210466e-06	\\
5038.76953125	9.36999841682469e-06	\\
5039.74831321023	8.08675681199535e-06	\\
5040.72709517045	6.43717456176772e-06	\\
5041.70587713068	7.27890315388062e-06	\\
5042.68465909091	9.10714927031413e-06	\\
5043.66344105114	8.72035998070684e-06	\\
5044.64222301136	7.38602215510775e-06	\\
5045.62100497159	6.4349641808221e-06	\\
5046.59978693182	7.58175182386609e-06	\\
5047.57856889205	7.00104821801918e-06	\\
5048.55735085227	6.32322270928378e-06	\\
5049.5361328125	8.10682395156649e-06	\\
5050.51491477273	7.15759065670867e-06	\\
5051.49369673295	7.36705850485745e-06	\\
5052.47247869318	6.39894943516352e-06	\\
5053.45126065341	7.86761999039693e-06	\\
5054.43004261364	6.43645103031238e-06	\\
5055.40882457386	5.28469280211515e-06	\\
5056.38760653409	6.48833276295214e-06	\\
5057.36638849432	6.25448742077713e-06	\\
5058.34517045455	5.85478942389507e-06	\\
5059.32395241477	6.92305948635725e-06	\\
5060.302734375	5.90455848837134e-06	\\
5061.28151633523	7.45379944114043e-06	\\
5062.26029829545	7.62120519216944e-06	\\
5063.23908025568	9.43833544133005e-06	\\
5064.21786221591	8.99376818428724e-06	\\
5065.19664417614	7.06119664809912e-06	\\
5066.17542613636	5.6294083965258e-06	\\
5067.15420809659	7.13609018277036e-06	\\
5068.13299005682	7.61260678609996e-06	\\
5069.11177201705	6.36949424583164e-06	\\
5070.09055397727	6.25644615365221e-06	\\
5071.0693359375	6.85361817584858e-06	\\
5072.04811789773	7.30953576752008e-06	\\
5073.02689985795	7.5326845496402e-06	\\
5074.00568181818	7.03297980644123e-06	\\
5074.98446377841	5.61540696061174e-06	\\
5075.96324573864	8.05924817966051e-06	\\
5076.94202769886	7.40955309534496e-06	\\
5077.92080965909	7.4096739153834e-06	\\
5078.89959161932	7.14030000551392e-06	\\
5079.87837357955	8.79439725220502e-06	\\
5080.85715553977	9.76321272514862e-06	\\
5081.8359375	9.70527654593624e-06	\\
5082.81471946023	6.38605688721591e-06	\\
5083.79350142045	7.16544175152532e-06	\\
5084.77228338068	5.83904210652652e-06	\\
5085.75106534091	7.52003221867027e-06	\\
5086.72984730114	8.00349087543183e-06	\\
5087.70862926136	7.44057148310603e-06	\\
5088.68741122159	8.02661433077731e-06	\\
5089.66619318182	7.45275229593342e-06	\\
5090.64497514205	6.7557733834411e-06	\\
5091.62375710227	1.03277845074154e-05	\\
5092.6025390625	6.68227616754616e-06	\\
5093.58132102273	8.48025989415704e-06	\\
5094.56010298295	6.58433300361638e-06	\\
5095.53888494318	7.88447945329059e-06	\\
5096.51766690341	6.18978060956487e-06	\\
5097.49644886364	6.98826317850596e-06	\\
5098.47523082386	7.10195978625448e-06	\\
5099.45401278409	6.57678026531586e-06	\\
5100.43279474432	7.72494596911493e-06	\\
5101.41157670455	5.85096325103247e-06	\\
5102.39035866477	8.69678361476378e-06	\\
5103.369140625	6.0644740569963e-06	\\
5104.34792258523	7.13659909296607e-06	\\
5105.32670454545	8.90325897914384e-06	\\
5106.30548650568	6.05417493909132e-06	\\
5107.28426846591	8.18730056832272e-06	\\
5108.26305042614	7.85139432437269e-06	\\
5109.24183238636	9.22171325588871e-06	\\
5110.22061434659	8.25747413591213e-06	\\
5111.19939630682	8.90270270366266e-06	\\
5112.17817826705	7.99515472786609e-06	\\
5113.15696022727	9.05208325306562e-06	\\
5114.1357421875	7.43864868677814e-06	\\
5115.11452414773	7.29726772774885e-06	\\
5116.09330610795	7.06025236837951e-06	\\
5117.07208806818	7.16385761578261e-06	\\
5118.05087002841	9.48950924828005e-06	\\
5119.02965198864	7.0491092606139e-06	\\
5120.00843394886	6.73433451811126e-06	\\
5120.98721590909	8.83274926577621e-06	\\
5121.96599786932	7.14199286840638e-06	\\
5122.94477982955	6.64674334669412e-06	\\
5123.92356178977	7.16473259999578e-06	\\
5124.90234375	8.94006739102329e-06	\\
5125.88112571023	6.2092862540075e-06	\\
5126.85990767045	9.26160526304607e-06	\\
5127.83868963068	8.69572910704127e-06	\\
5128.81747159091	6.77047462111056e-06	\\
5129.79625355114	8.17915695444466e-06	\\
5130.77503551136	8.16761735781307e-06	\\
5131.75381747159	5.95583233800254e-06	\\
5132.73259943182	7.49746766815215e-06	\\
5133.71138139205	5.98073789588582e-06	\\
5134.69016335227	6.32478230512791e-06	\\
5135.6689453125	7.1146102525537e-06	\\
5136.64772727273	5.59327440885298e-06	\\
5137.62650923295	8.64183368207665e-06	\\
5138.60529119318	6.13305184830366e-06	\\
5139.58407315341	8.79058918413729e-06	\\
5140.56285511364	9.80369694808362e-06	\\
5141.54163707386	7.28187671748626e-06	\\
5142.52041903409	7.50333038995521e-06	\\
5143.49920099432	7.60520593581658e-06	\\
5144.47798295455	7.72139675224822e-06	\\
5145.45676491477	6.35042367280139e-06	\\
5146.435546875	7.68688466916858e-06	\\
5147.41432883523	6.51308206667024e-06	\\
5148.39311079545	7.84399409593909e-06	\\
5149.37189275568	4.43476659273457e-06	\\
5150.35067471591	8.07359825019843e-06	\\
5151.32945667614	6.97348365791416e-06	\\
5152.30823863636	8.18790112091477e-06	\\
5153.28702059659	6.78567828042357e-06	\\
5154.26580255682	6.86408835868438e-06	\\
5155.24458451705	7.14345469158154e-06	\\
5156.22336647727	7.31963648839347e-06	\\
5157.2021484375	7.02045608197433e-06	\\
5158.18093039773	9.47464030437439e-06	\\
5159.15971235795	8.21838534217875e-06	\\
5160.13849431818	7.1617213109117e-06	\\
5161.11727627841	8.5017625587375e-06	\\
5162.09605823864	8.51538758082722e-06	\\
5163.07484019886	7.41239127118088e-06	\\
5164.05362215909	5.72635243792534e-06	\\
5165.03240411932	6.09919946059081e-06	\\
5166.01118607955	9.87714616734625e-06	\\
5166.98996803977	6.95200675660646e-06	\\
5167.96875	6.5459249954962e-06	\\
5168.94753196023	8.92988565423054e-06	\\
5169.92631392045	7.5766181735827e-06	\\
5170.90509588068	8.0803418898787e-06	\\
5171.88387784091	8.60148369860295e-06	\\
5172.86265980114	7.86978576798707e-06	\\
5173.84144176136	8.05324434823216e-06	\\
5174.82022372159	1.02168275402686e-05	\\
5175.79900568182	8.80825246974273e-06	\\
5176.77778764205	6.41045120205035e-06	\\
5177.75656960227	9.37498750464157e-06	\\
5178.7353515625	6.99038800669629e-06	\\
5179.71413352273	8.84632659642221e-06	\\
5180.69291548295	8.06774577596298e-06	\\
5181.67169744318	7.09897347227096e-06	\\
5182.65047940341	8.52176875377532e-06	\\
5183.62926136364	6.9776192086247e-06	\\
5184.60804332386	6.4740294085731e-06	\\
5185.58682528409	7.92641283091972e-06	\\
5186.56560724432	8.45373363955598e-06	\\
5187.54438920455	6.55202347844403e-06	\\
5188.52317116477	8.78673655478624e-06	\\
5189.501953125	5.67220818935436e-06	\\
5190.48073508523	6.05339145538711e-06	\\
5191.45951704545	9.37632518599404e-06	\\
5192.43829900568	8.35135781305735e-06	\\
5193.41708096591	7.99847461496359e-06	\\
5194.39586292614	6.04829712331949e-06	\\
5195.37464488636	9.46666839989384e-06	\\
5196.35342684659	8.71271010066179e-06	\\
5197.33220880682	6.83495078962145e-06	\\
5198.31099076705	8.43089154920028e-06	\\
5199.28977272727	7.11093980302971e-06	\\
5200.2685546875	6.8183547342159e-06	\\
5201.24733664773	8.94553856067469e-06	\\
5202.22611860795	8.14922741489244e-06	\\
5203.20490056818	6.88867783368647e-06	\\
5204.18368252841	7.44140391036538e-06	\\
5205.16246448864	8.24913967082515e-06	\\
5206.14124644886	7.99620658824171e-06	\\
5207.12002840909	8.02495513820324e-06	\\
5208.09881036932	7.99281492283255e-06	\\
5209.07759232955	9.43807624941167e-06	\\
5210.05637428977	6.59622782501871e-06	\\
5211.03515625	6.77751623282918e-06	\\
5212.01393821023	7.46542130630605e-06	\\
5212.99272017045	8.00683232749394e-06	\\
5213.97150213068	8.24643341837701e-06	\\
5214.95028409091	7.64760756370114e-06	\\
5215.92906605114	8.31918392959238e-06	\\
5216.90784801136	1.06676120818242e-05	\\
5217.88662997159	7.84438182207977e-06	\\
5218.86541193182	9.29435178389021e-06	\\
5219.84419389205	6.98318344755452e-06	\\
5220.82297585227	8.60210225797341e-06	\\
5221.8017578125	6.87564519831841e-06	\\
5222.78053977273	6.86422887421272e-06	\\
5223.75932173295	7.6069645698206e-06	\\
5224.73810369318	5.53271803540207e-06	\\
5225.71688565341	7.59557908739271e-06	\\
5226.69566761364	8.58617669204845e-06	\\
5227.67444957386	6.16669517410933e-06	\\
5228.65323153409	7.08076194751497e-06	\\
5229.63201349432	7.83840031078195e-06	\\
5230.61079545455	8.7146992802948e-06	\\
5231.58957741477	6.64072982877077e-06	\\
5232.568359375	9.69848292162057e-06	\\
5233.54714133523	8.36146766524242e-06	\\
5234.52592329545	8.23171021146475e-06	\\
5235.50470525568	7.71667790722711e-06	\\
5236.48348721591	7.49810646063605e-06	\\
5237.46226917614	1.05313931937313e-05	\\
5238.44105113636	6.57879471438677e-06	\\
5239.41983309659	1.03348052791812e-05	\\
5240.39861505682	9.08987948239351e-06	\\
5241.37739701705	1.00324748321117e-05	\\
5242.35617897727	8.9276005872219e-06	\\
5243.3349609375	6.55671556877101e-06	\\
5244.31374289773	7.33251860899111e-06	\\
5245.29252485795	6.88392362381353e-06	\\
5246.27130681818	9.97333227636139e-06	\\
5247.25008877841	8.63243713828865e-06	\\
5248.22887073864	7.87915361246961e-06	\\
5249.20765269886	9.79735903078226e-06	\\
5250.18643465909	7.71114567778052e-06	\\
5251.16521661932	8.94254947097428e-06	\\
5252.14399857955	8.67194945403376e-06	\\
5253.12278053977	8.94889793387531e-06	\\
5254.1015625	7.20239739426943e-06	\\
5255.08034446023	1.11372455224845e-05	\\
5256.05912642045	8.2438140966367e-06	\\
5257.03790838068	8.30259949039176e-06	\\
5258.01669034091	8.51818935705612e-06	\\
5258.99547230114	7.57133681739243e-06	\\
5259.97425426136	5.67428016007129e-06	\\
5260.95303622159	7.37907102521138e-06	\\
5261.93181818182	7.4924317062673e-06	\\
5262.91060014205	9.65643802343528e-06	\\
5263.88938210227	7.63416837030354e-06	\\
5264.8681640625	8.80582104456351e-06	\\
5265.84694602273	5.9138624910447e-06	\\
5266.82572798295	8.82505195381551e-06	\\
5267.80450994318	7.71884673960921e-06	\\
5268.78329190341	8.65082356817021e-06	\\
5269.76207386364	8.75162979636575e-06	\\
5270.74085582386	8.07557793351492e-06	\\
5271.71963778409	8.59862151510115e-06	\\
5272.69841974432	7.25489097125691e-06	\\
5273.67720170455	8.46919352897768e-06	\\
5274.65598366477	8.77896123945593e-06	\\
5275.634765625	8.47167219283816e-06	\\
5276.61354758523	7.30347624565518e-06	\\
5277.59232954545	8.62548715207373e-06	\\
5278.57111150568	9.67043116086462e-06	\\
5279.54989346591	8.34519103620906e-06	\\
5280.52867542614	7.9909028772505e-06	\\
5281.50745738636	7.32370742115438e-06	\\
5282.48623934659	7.99616797467859e-06	\\
5283.46502130682	8.18434510042745e-06	\\
5284.44380326705	8.35083440997113e-06	\\
5285.42258522727	8.74774649682182e-06	\\
5286.4013671875	7.93195883434156e-06	\\
5287.38014914773	8.76252454925919e-06	\\
5288.35893110795	9.42504430108827e-06	\\
5289.33771306818	9.58031035990339e-06	\\
5290.31649502841	9.74684560558207e-06	\\
5291.29527698864	5.99834983608545e-06	\\
5292.27405894886	7.66912905992299e-06	\\
5293.25284090909	8.7610542700465e-06	\\
5294.23162286932	8.48446990037615e-06	\\
5295.21040482955	7.522915999259e-06	\\
5296.18918678977	9.10953801860528e-06	\\
5297.16796875	8.61707983147454e-06	\\
5298.14675071023	7.80096497321993e-06	\\
5299.12553267045	9.40805228635244e-06	\\
5300.10431463068	8.90543552678781e-06	\\
5301.08309659091	9.07115139308193e-06	\\
5302.06187855114	9.2699165906606e-06	\\
5303.04066051136	7.25493199933045e-06	\\
5304.01944247159	7.59391264577895e-06	\\
5304.99822443182	8.29095760064946e-06	\\
5305.97700639205	8.60820227199834e-06	\\
5306.95578835227	8.60596260559423e-06	\\
5307.9345703125	9.99208509846551e-06	\\
5308.91335227273	7.60713976776061e-06	\\
5309.89213423295	7.53664446403221e-06	\\
5310.87091619318	8.89544288357068e-06	\\
5311.84969815341	9.44569941141897e-06	\\
5312.82848011364	7.63869908877315e-06	\\
5313.80726207386	8.91534749951764e-06	\\
5314.78604403409	8.20645126795558e-06	\\
5315.76482599432	7.46162387711074e-06	\\
5316.74360795455	6.57173677037214e-06	\\
5317.72238991477	8.0782752501772e-06	\\
5318.701171875	9.49524270642114e-06	\\
5319.67995383523	7.99910949082449e-06	\\
5320.65873579545	8.17595052761708e-06	\\
5321.63751775568	7.64141644926917e-06	\\
5322.61629971591	8.56046743503807e-06	\\
5323.59508167614	6.72906746928045e-06	\\
5324.57386363636	9.27322145780399e-06	\\
5325.55264559659	8.88653077854695e-06	\\
5326.53142755682	7.00722838146655e-06	\\
5327.51020951705	6.91054493265729e-06	\\
5328.48899147727	7.84761332079441e-06	\\
5329.4677734375	8.29368185752396e-06	\\
5330.44655539773	8.6101651271752e-06	\\
5331.42533735795	8.5756781562037e-06	\\
5332.40411931818	9.76991786357191e-06	\\
5333.38290127841	9.54728584565103e-06	\\
5334.36168323864	8.13102510799038e-06	\\
5335.34046519886	9.27931367052755e-06	\\
5336.31924715909	1.09290059649985e-05	\\
5337.29802911932	6.22684722422725e-06	\\
5338.27681107955	9.74131815101024e-06	\\
5339.25559303977	6.90838803351859e-06	\\
5340.234375	7.20437881209281e-06	\\
5341.21315696023	9.11296756747923e-06	\\
5342.19193892045	1.04167880631718e-05	\\
5343.17072088068	7.82508512626973e-06	\\
5344.14950284091	7.11212387540659e-06	\\
5345.12828480114	9.7112493171124e-06	\\
5346.10706676136	9.67849944785975e-06	\\
5347.08584872159	5.88432648517859e-06	\\
5348.06463068182	9.32194097087498e-06	\\
5349.04341264205	7.76368834492415e-06	\\
5350.02219460227	8.08432604576649e-06	\\
5351.0009765625	6.71894436828662e-06	\\
5351.97975852273	7.6191619143393e-06	\\
5352.95854048295	8.07721961416958e-06	\\
5353.93732244318	8.20293178670255e-06	\\
5354.91610440341	8.51355720589259e-06	\\
5355.89488636364	7.90013211971153e-06	\\
5356.87366832386	9.52158121763384e-06	\\
5357.85245028409	8.16243661854647e-06	\\
5358.83123224432	7.80065981895321e-06	\\
5359.81001420455	1.03311915742875e-05	\\
5360.78879616477	6.61756364512733e-06	\\
5361.767578125	7.87645475393572e-06	\\
5362.74636008523	9.16479833162918e-06	\\
5363.72514204545	8.14088890589276e-06	\\
5364.70392400568	9.09105883762514e-06	\\
5365.68270596591	7.82609171916827e-06	\\
5366.66148792614	9.64078726755401e-06	\\
5367.64026988636	9.10229509126717e-06	\\
5368.61905184659	7.11778865964146e-06	\\
5369.59783380682	9.62539652163878e-06	\\
5370.57661576705	1.00714599959039e-05	\\
5371.55539772727	9.89738547634236e-06	\\
5372.5341796875	1.00422895874485e-05	\\
5373.51296164773	1.11437327306681e-05	\\
5374.49174360795	8.40728534512886e-06	\\
5375.47052556818	8.12241303153375e-06	\\
5376.44930752841	7.6340816224029e-06	\\
5377.42808948864	9.13575875280591e-06	\\
5378.40687144886	8.44882028484989e-06	\\
5379.38565340909	8.82036359758389e-06	\\
5380.36443536932	8.47140673561575e-06	\\
5381.34321732955	1.03448732349504e-05	\\
5382.32199928977	7.66303304977202e-06	\\
5383.30078125	9.10170447475577e-06	\\
5384.27956321023	9.15453063430921e-06	\\
5385.25834517045	1.01695459351716e-05	\\
5386.23712713068	8.74725605523066e-06	\\
5387.21590909091	8.5542188597003e-06	\\
5388.19469105114	8.87094971849551e-06	\\
5389.17347301136	9.05526998683944e-06	\\
5390.15225497159	8.84436619457893e-06	\\
5391.13103693182	1.11107720745187e-05	\\
5392.10981889205	8.80863148965092e-06	\\
5393.08860085227	8.35981084719891e-06	\\
5394.0673828125	8.12808857707864e-06	\\
5395.04616477273	9.36798200331122e-06	\\
5396.02494673295	7.1992525292381e-06	\\
5397.00372869318	9.97325213410763e-06	\\
5397.98251065341	7.8697748354563e-06	\\
5398.96129261364	9.8121903465167e-06	\\
5399.94007457386	8.32170493214466e-06	\\
5400.91885653409	1.14757906704561e-05	\\
5401.89763849432	9.52524860266844e-06	\\
5402.87642045455	7.96185071801296e-06	\\
5403.85520241477	8.97608191295236e-06	\\
5404.833984375	7.95173646582813e-06	\\
5405.81276633523	8.15706454596066e-06	\\
5406.79154829545	1.04106657290248e-05	\\
5407.77033025568	1.02467696260799e-05	\\
5408.74911221591	8.81401250772707e-06	\\
5409.72789417614	7.86504973787992e-06	\\
5410.70667613636	9.48974404060667e-06	\\
5411.68545809659	8.85126299861911e-06	\\
5412.66424005682	9.04744004755447e-06	\\
5413.64302201705	8.496859975637e-06	\\
5414.62180397727	8.14842905044829e-06	\\
5415.6005859375	1.02265912041731e-05	\\
5416.57936789773	9.03476700189894e-06	\\
5417.55814985795	9.9626245453791e-06	\\
5418.53693181818	8.1110769895159e-06	\\
5419.51571377841	1.00820148505994e-05	\\
5420.49449573864	1.10394760911994e-05	\\
5421.47327769886	8.41506770315747e-06	\\
5422.45205965909	9.24495451965917e-06	\\
5423.43084161932	8.43358550629655e-06	\\
5424.40962357955	7.93546619021292e-06	\\
5425.38840553977	9.76966787317158e-06	\\
5426.3671875	8.87146754114212e-06	\\
5427.34596946023	7.76608586724797e-06	\\
5428.32475142045	8.77704825201701e-06	\\
5429.30353338068	7.34581798921106e-06	\\
5430.28231534091	8.85957612717574e-06	\\
5431.26109730114	8.46511876967273e-06	\\
5432.23987926136	7.93425826285661e-06	\\
5433.21866122159	1.06918034777652e-05	\\
5434.19744318182	8.4756503150417e-06	\\
5435.17622514205	1.07037867599477e-05	\\
5436.15500710227	7.55603901143683e-06	\\
5437.1337890625	1.03537510137318e-05	\\
5438.11257102273	9.56269435612711e-06	\\
5439.09135298295	1.19807984298201e-05	\\
5440.07013494318	8.83872661976548e-06	\\
5441.04891690341	9.19491059133349e-06	\\
5442.02769886364	9.29465417939093e-06	\\
5443.00648082386	9.87064741833496e-06	\\
5443.98526278409	8.23069582401685e-06	\\
5444.96404474432	9.64209920242776e-06	\\
5445.94282670455	8.21612184377132e-06	\\
5446.92160866477	9.17090271255702e-06	\\
5447.900390625	7.68396887927032e-06	\\
5448.87917258523	8.8742353070695e-06	\\
5449.85795454545	9.38925579469578e-06	\\
5450.83673650568	8.95909802739424e-06	\\
5451.81551846591	8.82603215550732e-06	\\
5452.79430042614	9.67531014124929e-06	\\
5453.77308238636	9.35933129311176e-06	\\
5454.75186434659	1.02236121252297e-05	\\
5455.73064630682	1.13789594490745e-05	\\
5456.70942826705	1.06080317724758e-05	\\
5457.68821022727	7.58714675579026e-06	\\
5458.6669921875	9.99325160543463e-06	\\
5459.64577414773	8.57911026681932e-06	\\
5460.62455610795	9.31639241232702e-06	\\
5461.60333806818	7.77030203512502e-06	\\
5462.58212002841	7.23741946713068e-06	\\
5463.56090198864	9.02727389502509e-06	\\
5464.53968394886	9.58248933549989e-06	\\
5465.51846590909	7.6735345856728e-06	\\
5466.49724786932	1.00428596896594e-05	\\
5467.47602982955	8.45974125528905e-06	\\
5468.45481178977	9.02020980424895e-06	\\
5469.43359375	1.06156610052932e-05	\\
5470.41237571023	9.41275990503388e-06	\\
5471.39115767045	1.12639416900065e-05	\\
5472.36993963068	8.50416208427903e-06	\\
5473.34872159091	9.49072105924636e-06	\\
5474.32750355114	9.59917316117451e-06	\\
5475.30628551136	9.29208868295674e-06	\\
5476.28506747159	1.01426264265235e-05	\\
5477.26384943182	1.11014769993393e-05	\\
5478.24263139205	9.1825682927204e-06	\\
5479.22141335227	7.42822793238309e-06	\\
5480.2001953125	9.23664340868887e-06	\\
5481.17897727273	7.47108663558744e-06	\\
5482.15775923295	8.06859871233594e-06	\\
5483.13654119318	9.04063635021089e-06	\\
5484.11532315341	1.031268200291e-05	\\
5485.09410511364	9.14276682674347e-06	\\
5486.07288707386	8.63330276575073e-06	\\
5487.05166903409	9.20208262898142e-06	\\
5488.03045099432	8.11159763799632e-06	\\
5489.00923295455	9.28890193641709e-06	\\
5489.98801491477	1.02063759884717e-05	\\
5490.966796875	7.78861213923931e-06	\\
5491.94557883523	8.71349756613204e-06	\\
5492.92436079545	8.65399116996356e-06	\\
5493.90314275568	8.08164636827185e-06	\\
5494.88192471591	7.89288522242321e-06	\\
5495.86070667614	1.02822476134776e-05	\\
5496.83948863636	9.40121240721641e-06	\\
5497.81827059659	8.77770342024246e-06	\\
5498.79705255682	9.19411982989202e-06	\\
5499.77583451705	7.8315560974154e-06	\\
5500.75461647727	9.16910815452703e-06	\\
5501.7333984375	9.04637093697869e-06	\\
5502.71218039773	6.5713297045241e-06	\\
5503.69096235795	9.79320457978944e-06	\\
5504.66974431818	6.22695563603441e-06	\\
5505.64852627841	9.00887775323198e-06	\\
5506.62730823864	9.63465537060783e-06	\\
5507.60609019886	9.29463104064888e-06	\\
5508.58487215909	7.52746687601049e-06	\\
5509.56365411932	1.14094006508225e-05	\\
5510.54243607955	8.31769244552777e-06	\\
5511.52121803977	1.08475251575137e-05	\\
5512.5	7.75942270045401e-06	\\
5513.47878196023	9.80113697556251e-06	\\
5514.45756392045	8.22077662526047e-06	\\
5515.43634588068	9.44940741593479e-06	\\
5516.41512784091	8.14536055543572e-06	\\
5517.39390980114	8.19959233693752e-06	\\
5518.37269176136	7.57640666823337e-06	\\
5519.35147372159	8.85334129066177e-06	\\
5520.33025568182	1.07928373694824e-05	\\
5521.30903764205	8.69785137037465e-06	\\
5522.28781960227	8.99363564481373e-06	\\
5523.2666015625	8.64544108140292e-06	\\
5524.24538352273	9.87843924048649e-06	\\
5525.22416548295	1.07644191984448e-05	\\
5526.20294744318	9.89404283770028e-06	\\
5527.18172940341	8.56635728720405e-06	\\
5528.16051136364	9.41571820809184e-06	\\
5529.13929332386	6.07941876508506e-06	\\
5530.11807528409	6.58332485348684e-06	\\
5531.09685724432	7.47242723067611e-06	\\
5532.07563920455	7.86593290889638e-06	\\
5533.05442116477	8.44520846845817e-06	\\
5534.033203125	9.23936347664788e-06	\\
5535.01198508523	8.68943661907151e-06	\\
5535.99076704545	9.49572331527772e-06	\\
5536.96954900568	1.06698657811484e-05	\\
5537.94833096591	7.15403506106862e-06	\\
5538.92711292614	8.43137900336932e-06	\\
5539.90589488636	1.08071707499693e-05	\\
5540.88467684659	6.20182008820301e-06	\\
5541.86345880682	7.73761200097007e-06	\\
5542.84224076705	9.11319429691396e-06	\\
5543.82102272727	7.25624505203638e-06	\\
5544.7998046875	8.02556489975785e-06	\\
5545.77858664773	8.66180230841894e-06	\\
5546.75736860795	8.36835023327698e-06	\\
5547.73615056818	6.59370527061258e-06	\\
5548.71493252841	9.04445360965193e-06	\\
5549.69371448864	7.00104253637419e-06	\\
5550.67249644886	9.078488352773e-06	\\
5551.65127840909	8.0701733184336e-06	\\
5552.63006036932	8.23871881850799e-06	\\
5553.60884232955	8.70584954856209e-06	\\
5554.58762428977	1.03389287912508e-05	\\
5555.56640625	1.00636991495326e-05	\\
5556.54518821023	9.79789103970882e-06	\\
5557.52397017045	6.78026480255864e-06	\\
5558.50275213068	8.60678802391948e-06	\\
5559.48153409091	7.58043719400288e-06	\\
5560.46031605114	9.09076003040722e-06	\\
5561.43909801136	8.26918519680624e-06	\\
5562.41787997159	9.8156873257035e-06	\\
5563.39666193182	6.65047912274907e-06	\\
5564.37544389205	6.17927947379914e-06	\\
5565.35422585227	8.45160362700437e-06	\\
5566.3330078125	8.2170020881666e-06	\\
5567.31178977273	8.18725274713226e-06	\\
5568.29057173295	1.00003006458355e-05	\\
5569.26935369318	8.8403265328971e-06	\\
5570.24813565341	9.69069508146333e-06	\\
5571.22691761364	7.97451698145147e-06	\\
5572.20569957386	8.82442825523268e-06	\\
5573.18448153409	6.84504879283887e-06	\\
5574.16326349432	8.65118779217404e-06	\\
5575.14204545455	7.98355488065266e-06	\\
5576.12082741477	9.86073948570604e-06	\\
5577.099609375	7.63307651040654e-06	\\
5578.07839133523	6.77859241014654e-06	\\
5579.05717329545	9.31141209002375e-06	\\
5580.03595525568	6.45216060050777e-06	\\
5581.01473721591	6.90192351927307e-06	\\
5581.99351917614	6.36044884756137e-06	\\
5582.97230113636	8.31956691981693e-06	\\
5583.95108309659	7.26680420834436e-06	\\
5584.92986505682	9.05448695458014e-06	\\
5585.90864701705	9.1622506998711e-06	\\
5586.88742897727	8.12192319117511e-06	\\
5587.8662109375	7.86135868619664e-06	\\
5588.84499289773	8.29150417567127e-06	\\
5589.82377485795	8.00590104126811e-06	\\
5590.80255681818	7.98752366920742e-06	\\
5591.78133877841	8.3366113710208e-06	\\
5592.76012073864	7.71245088462782e-06	\\
5593.73890269886	8.37406016141994e-06	\\
5594.71768465909	6.84577409771674e-06	\\
5595.69646661932	8.28620926003911e-06	\\
5596.67524857955	9.42118524834807e-06	\\
5597.65403053977	7.89093577562777e-06	\\
5598.6328125	6.414730998855e-06	\\
5599.61159446023	8.64418626326104e-06	\\
5600.59037642045	8.19542849827406e-06	\\
5601.56915838068	6.29765957818814e-06	\\
5602.54794034091	9.08656255579249e-06	\\
5603.52672230114	8.26352331768177e-06	\\
5604.50550426136	8.0423199691125e-06	\\
5605.48428622159	7.3997168458223e-06	\\
5606.46306818182	8.31478074754317e-06	\\
5607.44185014205	8.91929899829567e-06	\\
5608.42063210227	8.48779770808253e-06	\\
5609.3994140625	1.13047870183638e-05	\\
5610.37819602273	9.27281594268045e-06	\\
5611.35697798295	9.25244724402367e-06	\\
5612.33575994318	7.81547961631669e-06	\\
5613.31454190341	8.47674040029166e-06	\\
5614.29332386364	9.54435025422976e-06	\\
5615.27210582386	7.80280915136069e-06	\\
5616.25088778409	9.20806851990519e-06	\\
5617.22966974432	7.81293579948754e-06	\\
5618.20845170455	8.54636834177695e-06	\\
5619.18723366477	8.26200406121262e-06	\\
5620.166015625	9.81384569702205e-06	\\
5621.14479758523	9.51493795334429e-06	\\
5622.12357954545	9.18246219057293e-06	\\
5623.10236150568	8.6824214822886e-06	\\
5624.08114346591	9.54456689373816e-06	\\
5625.05992542614	8.57623832254547e-06	\\
5626.03870738636	7.23065567238822e-06	\\
5627.01748934659	8.99662418015358e-06	\\
5627.99627130682	7.14604795508384e-06	\\
5628.97505326705	6.81924070835676e-06	\\
5629.95383522727	7.73783981331255e-06	\\
5630.9326171875	8.89382748408087e-06	\\
5631.91139914773	7.10466662388755e-06	\\
5632.89018110795	7.02396183701151e-06	\\
5633.86896306818	7.13411787806073e-06	\\
5634.84774502841	7.15301319398137e-06	\\
5635.82652698864	7.66606374389693e-06	\\
5636.80530894886	6.24997257485397e-06	\\
5637.78409090909	8.08930832902346e-06	\\
5638.76287286932	7.81661095561205e-06	\\
5639.74165482955	6.34918652297286e-06	\\
5640.72043678977	7.09941885884782e-06	\\
5641.69921875	9.12288986441363e-06	\\
5642.67800071023	7.03388422124048e-06	\\
5643.65678267045	7.71585338807403e-06	\\
5644.63556463068	7.77942812742356e-06	\\
5645.61434659091	8.80721893493021e-06	\\
5646.59312855114	7.80235922497764e-06	\\
5647.57191051136	7.96896218790281e-06	\\
5648.55069247159	7.06463270241948e-06	\\
5649.52947443182	5.72251343873859e-06	\\
5650.50825639205	8.08753807381241e-06	\\
5651.48703835227	9.55084784448268e-06	\\
5652.4658203125	7.15815809772967e-06	\\
5653.44460227273	9.50397674000963e-06	\\
5654.42338423295	7.34749448496465e-06	\\
5655.40216619318	8.89812939407896e-06	\\
5656.38094815341	7.59088826197896e-06	\\
5657.35973011364	7.68738607505099e-06	\\
5658.33851207386	7.36802038592535e-06	\\
5659.31729403409	7.45112264382493e-06	\\
5660.29607599432	6.07705800334933e-06	\\
5661.27485795455	7.12212432109611e-06	\\
5662.25363991477	9.39400558344155e-06	\\
5663.232421875	6.97410235981594e-06	\\
5664.21120383523	9.92867537790441e-06	\\
5665.18998579545	7.11924660993444e-06	\\
5666.16876775568	6.94307851717903e-06	\\
5667.14754971591	6.80379572830267e-06	\\
5668.12633167614	7.0533963430917e-06	\\
5669.10511363636	7.18636369664249e-06	\\
5670.08389559659	9.28005651919318e-06	\\
5671.06267755682	9.69376567514404e-06	\\
5672.04145951705	8.0902869773856e-06	\\
5673.02024147727	7.79722720204006e-06	\\
5673.9990234375	8.80852024320202e-06	\\
5674.97780539773	6.13823592206911e-06	\\
5675.95658735795	7.23600336996619e-06	\\
5676.93536931818	6.66854392622281e-06	\\
5677.91415127841	5.69189517612339e-06	\\
5678.89293323864	6.75875313759106e-06	\\
5679.87171519886	7.80355119865047e-06	\\
5680.85049715909	5.12808183504838e-06	\\
5681.82927911932	5.24167646716185e-06	\\
5682.80806107955	6.15398805154446e-06	\\
5683.78684303977	6.83860725753578e-06	\\
5684.765625	8.62740184082476e-06	\\
5685.74440696023	7.78885632348897e-06	\\
5686.72318892045	5.12311615524387e-06	\\
5687.70197088068	7.94932623325868e-06	\\
5688.68075284091	7.06626028556874e-06	\\
5689.65953480114	6.34352514324264e-06	\\
5690.63831676136	6.64128266972902e-06	\\
5691.61709872159	7.44366872814727e-06	\\
5692.59588068182	6.06703796875342e-06	\\
5693.57466264205	6.72239429291951e-06	\\
5694.55344460227	7.99270081002528e-06	\\
5695.5322265625	9.49544237348205e-06	\\
5696.51100852273	7.73719045249874e-06	\\
5697.48979048295	6.10775706871781e-06	\\
5698.46857244318	5.248820577781e-06	\\
5699.44735440341	9.55859613679268e-06	\\
5700.42613636364	6.05253239108354e-06	\\
5701.40491832386	7.31059734839621e-06	\\
5702.38370028409	6.68846875475659e-06	\\
5703.36248224432	7.36475713541003e-06	\\
5704.34126420455	7.42351526993449e-06	\\
5705.32004616477	5.6686504135105e-06	\\
5706.298828125	8.70811547770537e-06	\\
5707.27761008523	8.21461034171284e-06	\\
5708.25639204545	7.33825395229834e-06	\\
5709.23517400568	8.11030198616268e-06	\\
5710.21395596591	6.18271663968278e-06	\\
5711.19273792614	6.54423946671202e-06	\\
5712.17151988636	7.54297600825087e-06	\\
5713.15030184659	7.40658595670469e-06	\\
5714.12908380682	9.19517430333152e-06	\\
5715.10786576705	8.14398934446144e-06	\\
5716.08664772727	7.03440661809105e-06	\\
5717.0654296875	9.80472618720502e-06	\\
5718.04421164773	6.78476638560728e-06	\\
5719.02299360795	7.49282624327731e-06	\\
5720.00177556818	6.84304396031522e-06	\\
5720.98055752841	6.82157227854224e-06	\\
5721.95933948864	7.23504847772072e-06	\\
5722.93812144886	9.21229344209647e-06	\\
5723.91690340909	3.28836147801541e-06	\\
5724.89568536932	7.24022309251325e-06	\\
5725.87446732955	6.94693082977962e-06	\\
5726.85324928977	5.99298834752869e-06	\\
5727.83203125	5.69519122830688e-06	\\
5728.81081321023	5.84492832759377e-06	\\
5729.78959517045	7.18798367431306e-06	\\
5730.76837713068	8.00734963156176e-06	\\
5731.74715909091	5.71817405357124e-06	\\
5732.72594105114	7.71473189600009e-06	\\
5733.70472301136	7.30253204251474e-06	\\
5734.68350497159	5.8179554570477e-06	\\
5735.66228693182	6.05098525220246e-06	\\
5736.64106889205	8.58023472243378e-06	\\
5737.61985085227	6.56375519019292e-06	\\
5738.5986328125	7.30381442491134e-06	\\
5739.57741477273	7.92232499234992e-06	\\
5740.55619673295	5.65342297640459e-06	\\
5741.53497869318	7.78628062841043e-06	\\
5742.51376065341	5.02065098010585e-06	\\
5743.49254261364	4.41459205443628e-06	\\
5744.47132457386	7.50039538173247e-06	\\
5745.45010653409	4.84699398323089e-06	\\
5746.42888849432	7.48287451080451e-06	\\
5747.40767045455	6.39683597981397e-06	\\
5748.38645241477	5.33121588606663e-06	\\
5749.365234375	6.56367732280706e-06	\\
5750.34401633523	8.56231726023723e-06	\\
5751.32279829545	8.95036609449744e-06	\\
5752.30158025568	7.14470235915506e-06	\\
5753.28036221591	7.81312276295781e-06	\\
5754.25914417614	5.37931561114956e-06	\\
5755.23792613636	8.37974153257094e-06	\\
5756.21670809659	7.72262497871762e-06	\\
5757.19549005682	5.96234929089903e-06	\\
5758.17427201705	6.58521799657093e-06	\\
5759.15305397727	7.40692763683647e-06	\\
5760.1318359375	8.08327860395295e-06	\\
5761.11061789773	7.59485626132942e-06	\\
5762.08939985795	7.15915346381103e-06	\\
5763.06818181818	6.6705132411376e-06	\\
5764.04696377841	7.64159432973649e-06	\\
5765.02574573864	5.12157596524673e-06	\\
5766.00452769886	7.6187594389346e-06	\\
5766.98330965909	6.22948802669485e-06	\\
5767.96209161932	5.38537776990823e-06	\\
5768.94087357955	8.5448072104836e-06	\\
5769.91965553977	6.3796179235442e-06	\\
5770.8984375	5.68921624911577e-06	\\
5771.87721946023	7.20926128637086e-06	\\
5772.85600142045	6.44407588429885e-06	\\
5773.83478338068	6.7318787465837e-06	\\
5774.81356534091	7.69014399640762e-06	\\
5775.79234730114	9.60796347999967e-06	\\
5776.77112926136	8.21233632880374e-06	\\
5777.74991122159	7.15336044599255e-06	\\
5778.72869318182	6.05855605428744e-06	\\
5779.70747514205	7.26227893083497e-06	\\
5780.68625710227	7.51577779610769e-06	\\
5781.6650390625	6.72123050148982e-06	\\
5782.64382102273	9.65329260217544e-06	\\
5783.62260298295	7.14000129821797e-06	\\
5784.60138494318	7.4264408564825e-06	\\
5785.58016690341	8.96236939236433e-06	\\
5786.55894886364	8.4820006258918e-06	\\
5787.53773082386	6.91788764238639e-06	\\
5788.51651278409	8.56833612896587e-06	\\
5789.49529474432	7.89989429305534e-06	\\
5790.47407670455	8.43309375893771e-06	\\
5791.45285866477	6.37529777490405e-06	\\
5792.431640625	9.31501786907083e-06	\\
5793.41042258523	9.12209730582729e-06	\\
5794.38920454545	5.70035449756405e-06	\\
5795.36798650568	8.16810616076204e-06	\\
5796.34676846591	6.21758556061013e-06	\\
5797.32555042614	6.84670324351555e-06	\\
5798.30433238636	6.42366259478178e-06	\\
5799.28311434659	5.48578482294834e-06	\\
5800.26189630682	8.30627900868918e-06	\\
5801.24067826705	5.38040120092793e-06	\\
5802.21946022727	8.39626905225873e-06	\\
5803.1982421875	6.60649785553575e-06	\\
5804.17702414773	5.14600203849905e-06	\\
5805.15580610795	6.93112673404598e-06	\\
5806.13458806818	7.19907084766314e-06	\\
5807.11337002841	6.56439257662582e-06	\\
5808.09215198864	7.19381941730821e-06	\\
5809.07093394886	8.41686529190938e-06	\\
5810.04971590909	5.30669195193211e-06	\\
5811.02849786932	8.387846524259e-06	\\
5812.00727982955	8.29333735975066e-06	\\
5812.98606178977	5.23415182775223e-06	\\
5813.96484375	5.15731966784553e-06	\\
5814.94362571023	6.19683313716727e-06	\\
5815.92240767045	7.9471430109666e-06	\\
5816.90118963068	8.02443786032651e-06	\\
5817.87997159091	7.28214922193027e-06	\\
5818.85875355114	7.11706363590764e-06	\\
5819.83753551136	5.74904165518301e-06	\\
5820.81631747159	6.2820719332643e-06	\\
5821.79509943182	7.87168234079382e-06	\\
5822.77388139205	9.45051897538155e-06	\\
5823.75266335227	8.18631472228047e-06	\\
5824.7314453125	8.28223698978867e-06	\\
5825.71022727273	8.84675905835662e-06	\\
5826.68900923295	8.09316400247747e-06	\\
5827.66779119318	8.25313287894406e-06	\\
5828.64657315341	1.0246394656993e-05	\\
5829.62535511364	8.86272357766044e-06	\\
5830.60413707386	8.07069214112344e-06	\\
5831.58291903409	6.5447372438199e-06	\\
5832.56170099432	7.99722730718995e-06	\\
5833.54048295455	5.77252342837141e-06	\\
5834.51926491477	7.68747035043833e-06	\\
5835.498046875	6.73968702419865e-06	\\
5836.47682883523	5.88696100304937e-06	\\
5837.45561079545	6.47784632289774e-06	\\
5838.43439275568	5.41154437414517e-06	\\
5839.41317471591	6.56318214460837e-06	\\
5840.39195667614	5.35390850481462e-06	\\
5841.37073863636	6.00267263930268e-06	\\
5842.34952059659	5.05642593844648e-06	\\
5843.32830255682	7.43249930121091e-06	\\
5844.30708451705	6.83065684462893e-06	\\
5845.28586647727	8.63517526382443e-06	\\
5846.2646484375	9.33852464116683e-06	\\
5847.24343039773	8.42323670140813e-06	\\
5848.22221235795	8.27400187044176e-06	\\
5849.20099431818	7.04564509728525e-06	\\
5850.17977627841	8.14730136782997e-06	\\
5851.15855823864	6.00777589473901e-06	\\
5852.13734019886	5.78814709675924e-06	\\
5853.11612215909	5.32830917383758e-06	\\
5854.09490411932	7.42046869835998e-06	\\
5855.07368607955	7.5104895004824e-06	\\
5856.05246803977	8.30278979037571e-06	\\
5857.03125	8.50222282913738e-06	\\
5858.01003196023	8.20417769968665e-06	\\
5858.98881392045	1.06241765509692e-05	\\
5859.96759588068	9.90040626309306e-06	\\
5860.94637784091	9.88546236745001e-06	\\
5861.92515980114	8.2879112064763e-06	\\
5862.90394176136	7.62593735064642e-06	\\
5863.88272372159	8.68722956920943e-06	\\
5864.86150568182	5.69604180733096e-06	\\
5865.84028764205	8.08504395707541e-06	\\
5866.81906960227	6.39425417703703e-06	\\
5867.7978515625	8.92983578501621e-06	\\
5868.77663352273	5.40084688843036e-06	\\
5869.75541548295	6.15651099082327e-06	\\
5870.73419744318	6.9297229080762e-06	\\
5871.71297940341	9.05509132290311e-06	\\
5872.69176136364	6.32471471514557e-06	\\
5873.67054332386	7.90128726611043e-06	\\
5874.64932528409	8.42866380567546e-06	\\
5875.62810724432	8.33674553331096e-06	\\
5876.60688920455	8.15784784517758e-06	\\
5877.58567116477	8.51620885559246e-06	\\
5878.564453125	6.3105097263178e-06	\\
5879.54323508523	9.94143733320656e-06	\\
5880.52201704545	9.59321681315643e-06	\\
5881.50079900568	8.88873421010924e-06	\\
5882.47958096591	6.54279789214261e-06	\\
5883.45836292614	8.32204148398853e-06	\\
5884.43714488636	1.01082170692385e-05	\\
5885.41592684659	7.606613518173e-06	\\
5886.39470880682	7.13437800027722e-06	\\
5887.37349076705	8.36139754586691e-06	\\
5888.35227272727	6.14165329290167e-06	\\
5889.3310546875	7.06319653082843e-06	\\
5890.30983664773	8.03667624977494e-06	\\
5891.28861860795	6.92820199233132e-06	\\
5892.26740056818	7.56542868364299e-06	\\
5893.24618252841	1.0766755213951e-05	\\
5894.22496448864	8.2356347289575e-06	\\
5895.20374644886	9.44655503893726e-06	\\
5896.18252840909	8.92722198639029e-06	\\
5897.16131036932	8.87318301596282e-06	\\
5898.14009232955	8.09920608497249e-06	\\
5899.11887428977	7.05619457236488e-06	\\
5900.09765625	9.10075178623528e-06	\\
5901.07643821023	8.39062456618262e-06	\\
5902.05522017045	6.83782122291177e-06	\\
5903.03400213068	7.99909339132907e-06	\\
5904.01278409091	9.87879940017653e-06	\\
5904.99156605114	7.1749074931365e-06	\\
5905.97034801136	8.83706612861185e-06	\\
5906.94912997159	8.64986404352258e-06	\\
5907.92791193182	8.88741887744968e-06	\\
5908.90669389205	1.04660665955869e-05	\\
5909.88547585227	1.035572126399e-05	\\
5910.8642578125	9.97455007853833e-06	\\
5911.84303977273	8.42937405179508e-06	\\
5912.82182173295	8.76513156089719e-06	\\
5913.80060369318	7.35919587432865e-06	\\
5914.77938565341	8.10069642963233e-06	\\
5915.75816761364	7.85008771010231e-06	\\
5916.73694957386	9.56151715125615e-06	\\
5917.71573153409	9.72503293364263e-06	\\
5918.69451349432	7.98732607047629e-06	\\
5919.67329545455	8.07443414022325e-06	\\
5920.65207741477	6.93706096267481e-06	\\
5921.630859375	8.59589958438164e-06	\\
5922.60964133523	7.32840771519959e-06	\\
5923.58842329545	8.3111865316939e-06	\\
5924.56720525568	7.38714131315417e-06	\\
5925.54598721591	7.88660438889262e-06	\\
5926.52476917614	8.82275407395023e-06	\\
5927.50355113636	9.13531781941093e-06	\\
5928.48233309659	8.81629307269889e-06	\\
5929.46111505682	8.59939891454508e-06	\\
5930.43989701705	1.00994066404377e-05	\\
5931.41867897727	9.98114941497434e-06	\\
5932.3974609375	9.7983152029166e-06	\\
5933.37624289773	7.37093607079582e-06	\\
5934.35502485795	6.96312037534146e-06	\\
5935.33380681818	8.15118958571981e-06	\\
5936.31258877841	6.24459510557829e-06	\\
5937.29137073864	6.89260184673044e-06	\\
5938.27015269886	7.60368685462455e-06	\\
5939.24893465909	7.73442101838954e-06	\\
5940.22771661932	6.00483346254427e-06	\\
5941.20649857955	9.5116549950652e-06	\\
5942.18528053977	1.10333104459396e-05	\\
5943.1640625	6.71521299229814e-06	\\
5944.14284446023	8.53425702902362e-06	\\
5945.12162642045	8.04597575642265e-06	\\
5946.10040838068	7.93350311055876e-06	\\
5947.07919034091	8.8436693337199e-06	\\
5948.05797230114	9.13350400598846e-06	\\
5949.03675426136	8.56690422502817e-06	\\
5950.01553622159	7.97159765414216e-06	\\
5950.99431818182	8.82271657713419e-06	\\
5951.97310014205	9.94784431012099e-06	\\
5952.95188210227	9.69283392241095e-06	\\
5953.9306640625	8.56815130946995e-06	\\
5954.90944602273	9.37766420559098e-06	\\
5955.88822798295	9.12154295370033e-06	\\
5956.86700994318	7.79996212670406e-06	\\
5957.84579190341	9.06785170018081e-06	\\
5958.82457386364	7.27336371334554e-06	\\
5959.80335582386	9.39939990333123e-06	\\
5960.78213778409	8.37343235088499e-06	\\
5961.76091974432	7.81956656828701e-06	\\
5962.73970170455	9.15457584872317e-06	\\
5963.71848366477	8.53582622770392e-06	\\
5964.697265625	8.23451050861988e-06	\\
5965.67604758523	8.92235190531943e-06	\\
5966.65482954545	7.69163978836223e-06	\\
5967.63361150568	8.42152970666712e-06	\\
5968.61239346591	8.69465744938435e-06	\\
5969.59117542614	9.48357896356559e-06	\\
5970.56995738636	1.06332055277957e-05	\\
5971.54873934659	9.41928237191086e-06	\\
5972.52752130682	9.46243812275668e-06	\\
5973.50630326705	9.83440725948509e-06	\\
5974.48508522727	8.71461733180578e-06	\\
5975.4638671875	1.00903207331068e-05	\\
5976.44264914773	8.72489722145053e-06	\\
5977.42143110795	7.31495058551165e-06	\\
5978.40021306818	7.87576101700696e-06	\\
5979.37899502841	9.43754588755134e-06	\\
5980.35777698864	7.40412778406176e-06	\\
5981.33655894886	9.27927010293702e-06	\\
5982.31534090909	8.79477837215145e-06	\\
5983.29412286932	9.32958096868588e-06	\\
5984.27290482955	8.49097562378013e-06	\\
5985.25168678977	1.03016055932953e-05	\\
5986.23046875	1.2140991997882e-05	\\
5987.20925071023	9.81154199770383e-06	\\
5988.18803267045	9.56812619502918e-06	\\
5989.16681463068	6.75579906219729e-06	\\
5990.14559659091	5.26499793599272e-06	\\
5991.12437855114	7.7289416773606e-06	\\
5992.10316051136	7.18881507006961e-06	\\
5993.08194247159	7.02346791960969e-06	\\
5994.06072443182	6.78942056109616e-06	\\
5995.03950639205	8.53508473304187e-06	\\
5996.01828835227	8.80159289830043e-06	\\
5996.9970703125	8.71133997088525e-06	\\
5997.97585227273	8.92460651390156e-06	\\
5998.95463423295	1.03263529134998e-05	\\
5999.93341619318	1.08330461778995e-05	\\
6000.91219815341	8.51420004203077e-06	\\
6001.89098011364	9.71972589688494e-06	\\
6002.86976207386	7.84319341066985e-06	\\
6003.84854403409	9.03542755994775e-06	\\
6004.82732599432	7.4643911515606e-06	\\
6005.80610795455	8.43707757121326e-06	\\
6006.78488991477	8.30826952686472e-06	\\
6007.763671875	8.77747595502606e-06	\\
6008.74245383523	9.72979128704791e-06	\\
6009.72123579545	9.1890184811648e-06	\\
6010.70001775568	9.10940529871812e-06	\\
6011.67879971591	9.87252811925522e-06	\\
6012.65758167614	9.55064220525834e-06	\\
6013.63636363636	7.78845076519534e-06	\\
6014.61514559659	9.00684604390887e-06	\\
6015.59392755682	7.42948876427766e-06	\\
6016.57270951705	8.88169847191417e-06	\\
6017.55149147727	7.66345961556644e-06	\\
6018.5302734375	7.12555528716112e-06	\\
6019.50905539773	1.01989991625977e-05	\\
6020.48783735795	8.28576919770566e-06	\\
6021.46661931818	8.61849481291439e-06	\\
6022.44540127841	7.59401340792951e-06	\\
6023.42418323864	7.32522685519402e-06	\\
6024.40296519886	9.60915933904235e-06	\\
6025.38174715909	9.49750764755093e-06	\\
6026.36052911932	9.01009671837525e-06	\\
6027.33931107955	9.87424860210265e-06	\\
6028.31809303977	7.8582227039866e-06	\\
6029.296875	8.13068662174157e-06	\\
6030.27565696023	8.76204704168679e-06	\\
6031.25443892045	9.22820306560407e-06	\\
6032.23322088068	9.17161481519981e-06	\\
6033.21200284091	8.7018637531526e-06	\\
6034.19078480114	9.10965134910341e-06	\\
6035.16956676136	8.31495455978052e-06	\\
6036.14834872159	8.0893194323516e-06	\\
6037.12713068182	7.48016525938875e-06	\\
6038.10591264205	1.0718507569977e-05	\\
6039.08469460227	5.49835829163957e-06	\\
6040.0634765625	8.78883476408924e-06	\\
6041.04225852273	5.6089694761902e-06	\\
6042.02104048295	9.63152899315457e-06	\\
6042.99982244318	8.1958740875374e-06	\\
6043.97860440341	9.84950278024922e-06	\\
6044.95738636364	1.00818461089743e-05	\\
6045.93616832386	8.10931185894386e-06	\\
6046.91495028409	8.48116170246811e-06	\\
6047.89373224432	8.43895340479614e-06	\\
6048.87251420455	8.54907905348612e-06	\\
6049.85129616477	9.91478684422801e-06	\\
6050.830078125	9.95726380616756e-06	\\
6051.80886008523	8.29509189770029e-06	\\
6052.78764204545	8.68722161665737e-06	\\
6053.76642400568	1.03156963390199e-05	\\
6054.74520596591	7.4714531475871e-06	\\
6055.72398792614	8.11293399847671e-06	\\
6056.70276988636	6.86863437185286e-06	\\
6057.68155184659	6.89879048410289e-06	\\
6058.66033380682	7.23930295443347e-06	\\
6059.63911576705	7.59579445871175e-06	\\
6060.61789772727	8.47262562802268e-06	\\
6061.5966796875	7.7320752600795e-06	\\
6062.57546164773	9.15942743772423e-06	\\
6063.55424360795	9.46496787826227e-06	\\
6064.53302556818	9.83983386681171e-06	\\
6065.51180752841	8.8210439468182e-06	\\
6066.49058948864	8.72020518482622e-06	\\
6067.46937144886	8.85289016650787e-06	\\
6068.44815340909	6.54840471315952e-06	\\
6069.42693536932	7.90530293702816e-06	\\
6070.40571732955	8.7869718317272e-06	\\
6071.38449928977	9.12349465312862e-06	\\
6072.36328125	8.44327676448803e-06	\\
6073.34206321023	1.04486959897592e-05	\\
6074.32084517045	7.44274624741008e-06	\\
6075.29962713068	8.19820167243453e-06	\\
6076.27840909091	9.68383394797998e-06	\\
6077.25719105114	8.88375125294378e-06	\\
6078.23597301136	8.38712785127622e-06	\\
6079.21475497159	7.65583134007616e-06	\\
6080.19353693182	8.23179495412411e-06	\\
6081.17231889205	9.10670165441524e-06	\\
6082.15110085227	7.3484229040308e-06	\\
6083.1298828125	6.52962005284858e-06	\\
6084.10866477273	7.54826283692827e-06	\\
6085.08744673295	8.62052019693221e-06	\\
6086.06622869318	9.6591122487773e-06	\\
6087.04501065341	7.69717402881291e-06	\\
6088.02379261364	8.83116730657439e-06	\\
6089.00257457386	8.00218794211504e-06	\\
6089.98135653409	9.22290871665319e-06	\\
6090.96013849432	7.98411167569872e-06	\\
6091.93892045455	9.08612142662527e-06	\\
6092.91770241477	9.39163697353558e-06	\\
6093.896484375	9.06408841888738e-06	\\
6094.87526633523	9.39665940099445e-06	\\
6095.85404829545	1.01754194764948e-05	\\
6096.83283025568	8.72971835593521e-06	\\
6097.81161221591	6.88730489606175e-06	\\
6098.79039417614	7.97323947225522e-06	\\
6099.76917613636	8.6531816483689e-06	\\
6100.74795809659	9.1255505147828e-06	\\
6101.72674005682	8.62814798896719e-06	\\
6102.70552201705	8.67955302153254e-06	\\
6103.68430397727	8.94503010813554e-06	\\
6104.6630859375	7.18964144820881e-06	\\
6105.64186789773	7.39204597011677e-06	\\
6106.62064985795	9.78064832458424e-06	\\
6107.59943181818	8.06827237222756e-06	\\
6108.57821377841	8.15823318082357e-06	\\
6109.55699573864	8.99085928952513e-06	\\
6110.53577769886	7.07167122198332e-06	\\
6111.51455965909	8.43527899843307e-06	\\
6112.49334161932	8.56153377460774e-06	\\
6113.47212357955	8.45194534890776e-06	\\
6114.45090553977	8.56859151494014e-06	\\
6115.4296875	7.8008384783219e-06	\\
6116.40846946023	9.33293269534537e-06	\\
6117.38725142045	6.79892234742269e-06	\\
6118.36603338068	9.52745087246285e-06	\\
6119.34481534091	9.3141051260966e-06	\\
6120.32359730114	8.38631724014092e-06	\\
6121.30237926136	6.85538218089143e-06	\\
6122.28116122159	8.95541400224902e-06	\\
6123.25994318182	8.35489316611074e-06	\\
6124.23872514205	8.26788331879427e-06	\\
6125.21750710227	8.96655956436552e-06	\\
6126.1962890625	8.87886198645326e-06	\\
6127.17507102273	8.64406055455101e-06	\\
6128.15385298295	7.71862278607849e-06	\\
6129.13263494318	8.68957557210668e-06	\\
6130.11141690341	7.8111655624583e-06	\\
6131.09019886364	7.51745731484965e-06	\\
6132.06898082386	8.75817026261942e-06	\\
6133.04776278409	8.7164269394512e-06	\\
6134.02654474432	5.49992777958933e-06	\\
6135.00532670455	8.35095200428086e-06	\\
6135.98410866477	8.54149027641312e-06	\\
6136.962890625	7.57872972732188e-06	\\
6137.94167258523	1.04816686688478e-05	\\
6138.92045454545	9.20553656506597e-06	\\
6139.89923650568	8.11415023196335e-06	\\
6140.87801846591	8.94624184340623e-06	\\
6141.85680042614	9.67512828639098e-06	\\
6142.83558238636	7.73679427477782e-06	\\
6143.81436434659	8.66466379088035e-06	\\
6144.79314630682	1.00083075262611e-05	\\
6145.77192826705	7.21685866863199e-06	\\
6146.75071022727	8.88380493698267e-06	\\
6147.7294921875	7.89033126910565e-06	\\
6148.70827414773	9.52315130554397e-06	\\
6149.68705610795	7.11815389652692e-06	\\
6150.66583806818	8.83055780365871e-06	\\
6151.64462002841	9.50183314775993e-06	\\
6152.62340198864	8.80962978504366e-06	\\
6153.60218394886	7.71282487844554e-06	\\
6154.58096590909	6.79394080189256e-06	\\
6155.55974786932	7.46631351268818e-06	\\
6156.53852982955	9.55015179790634e-06	\\
6157.51731178977	9.08740694151217e-06	\\
6158.49609375	7.64185273267211e-06	\\
6159.47487571023	9.59233441650178e-06	\\
6160.45365767045	9.17612898553175e-06	\\
6161.43243963068	8.41142390790972e-06	\\
6162.41122159091	6.22935081357252e-06	\\
6163.39000355114	7.21547181031573e-06	\\
6164.36878551136	7.14187310528431e-06	\\
6165.34756747159	8.84643214380653e-06	\\
6166.32634943182	9.90104543623886e-06	\\
6167.30513139205	8.20728205547058e-06	\\
6168.28391335227	1.04501176870062e-05	\\
6169.2626953125	7.61709816219417e-06	\\
6170.24147727273	8.28008615687095e-06	\\
6171.22025923295	6.79067922547555e-06	\\
6172.19904119318	8.73280615877353e-06	\\
6173.17782315341	8.33566136665778e-06	\\
6174.15660511364	7.87098294463293e-06	\\
6175.13538707386	7.47398229798129e-06	\\
6176.11416903409	7.46785866774221e-06	\\
6177.09295099432	1.03258591703767e-05	\\
6178.07173295455	9.03342519623578e-06	\\
6179.05051491477	8.09572495082672e-06	\\
6180.029296875	7.89241477760934e-06	\\
6181.00807883523	8.16339974763274e-06	\\
6181.98686079545	7.78170332997037e-06	\\
6182.96564275568	9.66235729024191e-06	\\
6183.94442471591	8.6653170411902e-06	\\
6184.92320667614	7.71908100710342e-06	\\
6185.90198863636	7.73738224265874e-06	\\
6186.88077059659	7.77871269091838e-06	\\
6187.85955255682	9.50874309305719e-06	\\
6188.83833451705	9.7168873808354e-06	\\
6189.81711647727	8.02124595762338e-06	\\
6190.7958984375	6.69271486423742e-06	\\
6191.77468039773	7.66539511209318e-06	\\
6192.75346235795	7.53327110957291e-06	\\
6193.73224431818	9.72838028164871e-06	\\
6194.71102627841	9.50975837381775e-06	\\
6195.68980823864	7.79115179304732e-06	\\
6196.66859019886	9.93623199852423e-06	\\
6197.64737215909	1.01030573201252e-05	\\
6198.62615411932	8.77667682504028e-06	\\
6199.60493607955	8.63678681296248e-06	\\
6200.58371803977	7.73558188095617e-06	\\
6201.5625	7.94459443527765e-06	\\
6202.54128196023	8.51355153883332e-06	\\
6203.52006392045	1.04974335888085e-05	\\
6204.49884588068	7.67088227919641e-06	\\
6205.47762784091	8.10843830627694e-06	\\
6206.45640980114	7.83868691340482e-06	\\
6207.43519176136	9.18693334154491e-06	\\
6208.41397372159	8.47177814545485e-06	\\
6209.39275568182	6.77852651815311e-06	\\
6210.37153764205	8.54003780767734e-06	\\
6211.35031960227	7.84938090847747e-06	\\
6212.3291015625	6.78475912458131e-06	\\
6213.30788352273	1.01587655814866e-05	\\
6214.28666548295	7.9998673592457e-06	\\
6215.26544744318	8.0030536263224e-06	\\
6216.24422940341	9.61593077802284e-06	\\
6217.22301136364	8.276560466567e-06	\\
6218.20179332386	8.9480758646265e-06	\\
6219.18057528409	8.32373698045655e-06	\\
6220.15935724432	9.42007246778163e-06	\\
6221.13813920455	7.96570543017326e-06	\\
6222.11692116477	9.3716518823986e-06	\\
6223.095703125	6.01311315434339e-06	\\
6224.07448508523	9.44773701519894e-06	\\
6225.05326704545	9.64205141034742e-06	\\
6226.03204900568	8.33110407149913e-06	\\
6227.01083096591	9.07636881513476e-06	\\
6227.98961292614	7.5105998372225e-06	\\
6228.96839488636	1.10966248564076e-05	\\
6229.94717684659	6.75365199502606e-06	\\
6230.92595880682	1.01563745385386e-05	\\
6231.90474076705	8.62026047507746e-06	\\
6232.88352272727	8.71046032856703e-06	\\
6233.8623046875	8.96441714133727e-06	\\
6234.84108664773	7.09869253907384e-06	\\
6235.81986860795	8.26921199802208e-06	\\
6236.79865056818	7.52878573802313e-06	\\
6237.77743252841	9.27984637132758e-06	\\
6238.75621448864	7.21057397765391e-06	\\
6239.73499644886	7.79996062585549e-06	\\
6240.71377840909	8.09391104398845e-06	\\
6241.69256036932	7.42631602500037e-06	\\
6242.67134232955	8.81976207479148e-06	\\
6243.65012428977	6.85675644357349e-06	\\
6244.62890625	8.43067508735705e-06	\\
6245.60768821023	8.43062139885919e-06	\\
6246.58647017045	5.67946654235819e-06	\\
6247.56525213068	9.77172437124157e-06	\\
6248.54403409091	7.4407936718572e-06	\\
6249.52281605114	6.74502492603545e-06	\\
6250.50159801136	6.71670473466694e-06	\\
6251.48037997159	8.40276894865099e-06	\\
6252.45916193182	7.57434015904764e-06	\\
6253.43794389205	8.15404085732937e-06	\\
6254.41672585227	9.02541383274001e-06	\\
6255.3955078125	6.77854431658808e-06	\\
6256.37428977273	1.0229189603271e-05	\\
6257.35307173295	7.90960498724363e-06	\\
6258.33185369318	7.33703589377089e-06	\\
6259.31063565341	8.03264672068604e-06	\\
6260.28941761364	4.91700644796767e-06	\\
6261.26819957386	7.91867629045205e-06	\\
6262.24698153409	8.80545377892153e-06	\\
6263.22576349432	7.66915905409323e-06	\\
6264.20454545455	7.94061845636357e-06	\\
6265.18332741477	6.91805412631722e-06	\\
6266.162109375	7.16860458343325e-06	\\
6267.14089133523	8.72092810050389e-06	\\
6268.11967329545	7.72528129749765e-06	\\
6269.09845525568	6.64993240336296e-06	\\
6270.07723721591	7.89211119945611e-06	\\
6271.05601917614	7.7075336940978e-06	\\
6272.03480113636	8.37174644820524e-06	\\
6273.01358309659	7.69753265171221e-06	\\
6273.99236505682	6.05052738855298e-06	\\
6274.97114701705	7.19385337313961e-06	\\
6275.94992897727	8.22329036549148e-06	\\
6276.9287109375	7.77654791133683e-06	\\
6277.90749289773	5.77668685610293e-06	\\
6278.88627485795	6.91528054971029e-06	\\
6279.86505681818	7.00615233029949e-06	\\
6280.84383877841	6.60935504427577e-06	\\
6281.82262073864	8.42117454860918e-06	\\
6282.80140269886	8.82616761029675e-06	\\
6283.78018465909	6.71747041126272e-06	\\
6284.75896661932	6.93219204029637e-06	\\
6285.73774857955	7.28259758270276e-06	\\
6286.71653053977	6.22862502586917e-06	\\
6287.6953125	8.22245715929146e-06	\\
6288.67409446023	7.24676146823928e-06	\\
6289.65287642045	7.2671490985047e-06	\\
6290.63165838068	6.40921762727842e-06	\\
6291.61044034091	9.11081562555637e-06	\\
6292.58922230114	7.38784513793305e-06	\\
6293.56800426136	6.48831950638278e-06	\\
6294.54678622159	8.2957470615779e-06	\\
6295.52556818182	8.82386252015503e-06	\\
6296.50435014205	7.86200482637019e-06	\\
6297.48313210227	7.74811408901633e-06	\\
6298.4619140625	6.61493523459627e-06	\\
6299.44069602273	7.28212210666105e-06	\\
6300.41947798295	6.55491093843706e-06	\\
6301.39825994318	6.45493305788944e-06	\\
6302.37704190341	7.17601256671316e-06	\\
6303.35582386364	5.940539734588e-06	\\
6304.33460582386	6.77639006759838e-06	\\
6305.31338778409	7.09910118459119e-06	\\
6306.29216974432	7.77361568859704e-06	\\
6307.27095170455	6.75825062631293e-06	\\
6308.24973366477	7.87140514733996e-06	\\
6309.228515625	8.35160813405132e-06	\\
6310.20729758523	6.87199378933206e-06	\\
6311.18607954545	7.75330301976241e-06	\\
6312.16486150568	7.23465067900596e-06	\\
6313.14364346591	7.39021252037268e-06	\\
6314.12242542614	7.81713457259381e-06	\\
6315.10120738636	9.17079271451842e-06	\\
6316.07998934659	6.05912222357082e-06	\\
6317.05877130682	7.80730916990234e-06	\\
6318.03755326705	8.49611690046071e-06	\\
6319.01633522727	7.83940420284915e-06	\\
6319.9951171875	5.01773686829645e-06	\\
6320.97389914773	8.02124733939307e-06	\\
6321.95268110795	6.9239666316266e-06	\\
6322.93146306818	6.69068692484868e-06	\\
6323.91024502841	6.74037532316748e-06	\\
6324.88902698864	7.17809929802723e-06	\\
6325.86780894886	8.52884945271739e-06	\\
6326.84659090909	7.7512105561204e-06	\\
6327.82537286932	8.95598259109087e-06	\\
6328.80415482955	7.01834676790718e-06	\\
6329.78293678977	8.67346290708575e-06	\\
6330.76171875	8.18964953646279e-06	\\
6331.74050071023	7.61146041777013e-06	\\
6332.71928267045	6.1121670838958e-06	\\
6333.69806463068	7.28920970605638e-06	\\
6334.67684659091	6.26226504015065e-06	\\
6335.65562855114	7.79566506975927e-06	\\
6336.63441051136	7.81352692845203e-06	\\
6337.61319247159	8.74576087385979e-06	\\
6338.59197443182	6.47822069521606e-06	\\
6339.57075639205	7.09945352964307e-06	\\
6340.54953835227	8.09962986322696e-06	\\
6341.5283203125	9.55388027158276e-06	\\
6342.50710227273	8.49281418069927e-06	\\
6343.48588423295	8.19790527577625e-06	\\
6344.46466619318	5.47861915017882e-06	\\
6345.44344815341	5.71318595627649e-06	\\
6346.42223011364	7.52041792449378e-06	\\
6347.40101207386	7.73841344705279e-06	\\
6348.37979403409	7.02885152074367e-06	\\
6349.35857599432	4.24525622054517e-06	\\
6350.33735795455	7.4705057473295e-06	\\
6351.31613991477	8.24524045895433e-06	\\
6352.294921875	7.6186784719683e-06	\\
6353.27370383523	8.01928436016067e-06	\\
6354.25248579545	9.69188667736395e-06	\\
6355.23126775568	8.03585769941896e-06	\\
6356.21004971591	6.4316780934188e-06	\\
6357.18883167614	8.36351051280417e-06	\\
6358.16761363636	7.23319474943024e-06	\\
6359.14639559659	7.69919991796736e-06	\\
6360.12517755682	8.49743969658923e-06	\\
6361.10395951705	6.34426167165391e-06	\\
6362.08274147727	8.17143453746824e-06	\\
6363.0615234375	6.5491239638106e-06	\\
6364.04030539773	6.82167470182136e-06	\\
6365.01908735795	6.13092351230047e-06	\\
6365.99786931818	6.50027301944308e-06	\\
6366.97665127841	6.67693321100091e-06	\\
6367.95543323864	6.41381128617023e-06	\\
6368.93421519886	8.1132984388784e-06	\\
6369.91299715909	7.24356589016993e-06	\\
6370.89177911932	7.29401400806477e-06	\\
6371.87056107955	7.11875156770047e-06	\\
6372.84934303977	7.22952629570996e-06	\\
6373.828125	5.78790231498287e-06	\\
6374.80690696023	5.51478064192761e-06	\\
6375.78568892045	6.95088543545446e-06	\\
6376.76447088068	7.1144162729288e-06	\\
6377.74325284091	6.75400713349549e-06	\\
6378.72203480114	5.09750842808369e-06	\\
6379.70081676136	5.38656964936777e-06	\\
6380.67959872159	7.9849743396737e-06	\\
6381.65838068182	5.38029996256066e-06	\\
6382.63716264205	6.0222262280122e-06	\\
6383.61594460227	5.46409450583429e-06	\\
6384.5947265625	8.0303636112127e-06	\\
6385.57350852273	7.15250565176501e-06	\\
6386.55229048295	6.28648159544853e-06	\\
6387.53107244318	5.83630337861044e-06	\\
6388.50985440341	6.8654262149285e-06	\\
6389.48863636364	6.7837419653776e-06	\\
6390.46741832386	7.17287536090897e-06	\\
6391.44620028409	6.06185843121162e-06	\\
6392.42498224432	9.36792898465939e-06	\\
6393.40376420455	7.4278058814387e-06	\\
6394.38254616477	9.00257539835349e-06	\\
6395.361328125	5.98246322048289e-06	\\
6396.34011008523	8.15568264315858e-06	\\
6397.31889204545	8.5373037348107e-06	\\
6398.29767400568	8.76479771141405e-06	\\
6399.27645596591	6.74210748225763e-06	\\
6400.25523792614	7.5097261680601e-06	\\
6401.23401988636	9.31896600113932e-06	\\
6402.21280184659	1.01871006457568e-05	\\
6403.19158380682	7.36905802998727e-06	\\
6404.17036576705	7.12529422432021e-06	\\
6405.14914772727	8.28587237923274e-06	\\
6406.1279296875	8.53245111239864e-06	\\
6407.10671164773	7.2428743469579e-06	\\
6408.08549360795	9.01363077918889e-06	\\
6409.06427556818	8.66855189608797e-06	\\
6410.04305752841	7.13274487534369e-06	\\
6411.02183948864	8.24854901059863e-06	\\
6412.00062144886	1.05621889318567e-05	\\
6412.97940340909	9.76318996230859e-06	\\
6413.95818536932	9.81119728256284e-06	\\
6414.93696732955	9.11365241848967e-06	\\
6415.91574928977	8.28241405692516e-06	\\
6416.89453125	8.02901999354197e-06	\\
6417.87331321023	8.90548302641017e-06	\\
6418.85209517045	7.60955051393078e-06	\\
6419.83087713068	8.72937740805095e-06	\\
6420.80965909091	8.18380669992689e-06	\\
6421.78844105114	7.13283109029337e-06	\\
6422.76722301136	7.94916299660805e-06	\\
6423.74600497159	7.22591735019307e-06	\\
6424.72478693182	7.06767925577544e-06	\\
6425.70356889205	9.1044784212695e-06	\\
6426.68235085227	7.12643627844071e-06	\\
6427.6611328125	8.84143168727729e-06	\\
6428.63991477273	9.14078078422633e-06	\\
6429.61869673295	8.95776911406732e-06	\\
6430.59747869318	9.56050497560511e-06	\\
6431.57626065341	1.06798910095109e-05	\\
6432.55504261364	8.73172233101893e-06	\\
6433.53382457386	6.60992965782825e-06	\\
6434.51260653409	8.89720418957706e-06	\\
6435.49138849432	8.96735958678298e-06	\\
6436.47017045455	9.10607607429404e-06	\\
6437.44895241477	9.06504838858732e-06	\\
6438.427734375	9.15393043895657e-06	\\
6439.40651633523	8.95760169509512e-06	\\
6440.38529829545	9.95485280411251e-06	\\
6441.36408025568	8.44510797762435e-06	\\
6442.34286221591	9.21116196524774e-06	\\
6443.32164417614	9.35583432939379e-06	\\
6444.30042613636	9.03738630849419e-06	\\
6445.27920809659	7.78000656000061e-06	\\
6446.25799005682	1.07527240415624e-05	\\
6447.23677201705	1.03569913836335e-05	\\
6448.21555397727	8.19282398540594e-06	\\
6449.1943359375	8.5334213109348e-06	\\
6450.17311789773	9.15252751253098e-06	\\
6451.15189985795	9.92687210141526e-06	\\
6452.13068181818	6.86096698096596e-06	\\
6453.10946377841	1.0151160758812e-05	\\
6454.08824573864	9.06047051056444e-06	\\
6455.06702769886	9.03920983899803e-06	\\
6456.04580965909	7.72721005143105e-06	\\
6457.02459161932	9.37473325749555e-06	\\
6458.00337357955	7.76041344748866e-06	\\
6458.98215553977	1.05596723133305e-05	\\
6459.9609375	6.89885962848442e-06	\\
6460.93971946023	9.76863719420982e-06	\\
6461.91850142045	9.75135195978501e-06	\\
6462.89728338068	7.55873401822311e-06	\\
6463.87606534091	8.54752504752738e-06	\\
6464.85484730114	7.88007996642692e-06	\\
6465.83362926136	7.51010461908783e-06	\\
6466.81241122159	8.89416264256334e-06	\\
6467.79119318182	7.60870194634005e-06	\\
6468.76997514205	9.28399655498786e-06	\\
6469.74875710227	8.89424640870987e-06	\\
6470.7275390625	8.95934360087105e-06	\\
6471.70632102273	9.81571785258344e-06	\\
6472.68510298295	7.44745021465221e-06	\\
6473.66388494318	1.09302208749721e-05	\\
6474.64266690341	9.40785636001547e-06	\\
6475.62144886364	1.01536174175021e-05	\\
6476.60023082386	8.73243588722933e-06	\\
6477.57901278409	9.01380063522072e-06	\\
6478.55779474432	1.00033993673836e-05	\\
6479.53657670455	8.28654147042861e-06	\\
6480.51535866477	1.09146552900389e-05	\\
6481.494140625	1.07661608045333e-05	\\
6482.47292258523	1.09103714991432e-05	\\
6483.45170454545	9.15410596406701e-06	\\
6484.43048650568	9.28031297823287e-06	\\
6485.40926846591	9.49275321031118e-06	\\
6486.38805042614	8.07963956163226e-06	\\
6487.36683238636	9.93276290196641e-06	\\
6488.34561434659	9.67178127281977e-06	\\
6489.32439630682	1.09221544637319e-05	\\
6490.30317826705	8.0492534977045e-06	\\
6491.28196022727	8.58559591688729e-06	\\
6492.2607421875	1.18864016934386e-05	\\
6493.23952414773	7.22651160392155e-06	\\
6494.21830610795	9.20149045948414e-06	\\
6495.19708806818	9.23487875570879e-06	\\
6496.17587002841	1.03278520339438e-05	\\
6497.15465198864	1.029466696794e-05	\\
6498.13343394886	7.52924891642413e-06	\\
6499.11221590909	1.15768215093137e-05	\\
6500.09099786932	1.22796794014604e-05	\\
6501.06977982955	9.73636805426708e-06	\\
6502.04856178977	9.03356701199011e-06	\\
6503.02734375	8.49964070631768e-06	\\
6504.00612571023	7.78273050888696e-06	\\
6504.98490767045	9.45452183307324e-06	\\
6505.96368963068	9.10576276828587e-06	\\
6506.94247159091	9.12949424815988e-06	\\
6507.92125355114	8.58821603993299e-06	\\
6508.90003551136	9.65315991526928e-06	\\
6509.87881747159	8.28208591800287e-06	\\
6510.85759943182	9.22912203459517e-06	\\
6511.83638139205	1.16211439579469e-05	\\
6512.81516335227	1.09412346494908e-05	\\
6513.7939453125	1.06979393273903e-05	\\
6514.77272727273	8.06764880733132e-06	\\
6515.75150923295	1.02656799121609e-05	\\
6516.73029119318	9.41377421510318e-06	\\
6517.70907315341	9.43861480147141e-06	\\
6518.68785511364	1.07903820501664e-05	\\
6519.66663707386	1.07802365588633e-05	\\
6520.64541903409	8.38683204907569e-06	\\
6521.62420099432	1.04470435675682e-05	\\
6522.60298295455	9.901205259367e-06	\\
6523.58176491477	8.23750065142247e-06	\\
6524.560546875	9.34871731055096e-06	\\
6525.53932883523	1.07930832450268e-05	\\
6526.51811079545	1.05985175622609e-05	\\
6527.49689275568	9.79077582036073e-06	\\
6528.47567471591	9.67018695479775e-06	\\
6529.45445667614	7.48110275956129e-06	\\
6530.43323863636	7.3620096103317e-06	\\
6531.41202059659	9.11231671641368e-06	\\
6532.39080255682	8.64150955898057e-06	\\
6533.36958451705	9.06414100225618e-06	\\
6534.34836647727	9.34865385146934e-06	\\
6535.3271484375	8.64021563711814e-06	\\
6536.30593039773	8.24973122335049e-06	\\
6537.28471235795	1.04484472578761e-05	\\
6538.26349431818	1.16726981774021e-05	\\
6539.24227627841	9.59019654374868e-06	\\
6540.22105823864	8.45184870876614e-06	\\
6541.19984019886	8.21469974300767e-06	\\
6542.17862215909	9.70887426840726e-06	\\
6543.15740411932	9.64909565984739e-06	\\
6544.13618607955	8.45031138595918e-06	\\
6545.11496803977	9.20829129593363e-06	\\
6546.09375	8.79974553011078e-06	\\
6547.07253196023	8.43804176603709e-06	\\
6548.05131392045	1.05192962655264e-05	\\
6549.03009588068	9.41235340636219e-06	\\
6550.00887784091	8.57678352760411e-06	\\
6550.98765980114	1.02897451129021e-05	\\
6551.96644176136	9.59272178712896e-06	\\
6552.94522372159	1.13845633527229e-05	\\
6553.92400568182	9.96281249763447e-06	\\
6554.90278764205	1.0003177898552e-05	\\
6555.88156960227	8.54904579613728e-06	\\
6556.8603515625	1.0175420923132e-05	\\
6557.83913352273	8.80855291260796e-06	\\
6558.81791548295	8.69769414337929e-06	\\
6559.79669744318	8.15359150786684e-06	\\
6560.77547940341	9.49314576026799e-06	\\
6561.75426136364	8.26083172672959e-06	\\
6562.73304332386	8.10573348649552e-06	\\
6563.71182528409	8.52691619065622e-06	\\
6564.69060724432	9.16976158895402e-06	\\
6565.66938920455	9.58427694908057e-06	\\
6566.64817116477	1.02687550282193e-05	\\
6567.626953125	1.0416299244038e-05	\\
6568.60573508523	1.12847712780401e-05	\\
6569.58451704545	1.01213840468923e-05	\\
6570.56329900568	7.22722833449079e-06	\\
6571.54208096591	8.8587783981447e-06	\\
6572.52086292614	7.81449843525046e-06	\\
6573.49964488636	8.75044017543056e-06	\\
6574.47842684659	9.12370079215286e-06	\\
6575.45720880682	8.73122623699127e-06	\\
6576.43599076705	8.89771774861252e-06	\\
6577.41477272727	8.08740786141078e-06	\\
6578.3935546875	8.41752192007868e-06	\\
6579.37233664773	6.68727781334698e-06	\\
6580.35111860795	8.42466626733721e-06	\\
6581.32990056818	1.01283393842752e-05	\\
6582.30868252841	8.2409505120804e-06	\\
6583.28746448864	8.3024960393563e-06	\\
6584.26624644886	9.24470737733824e-06	\\
6585.24502840909	7.75969122227994e-06	\\
6586.22381036932	8.35645381106002e-06	\\
6587.20259232955	1.02482621282389e-05	\\
6588.18137428977	8.82884323829954e-06	\\
6589.16015625	9.00555280217679e-06	\\
6590.13893821023	8.79926311434447e-06	\\
6591.11772017045	9.27863801851659e-06	\\
6592.09650213068	7.59033299021468e-06	\\
6593.07528409091	8.20847021204979e-06	\\
6594.05406605114	7.10114854744351e-06	\\
6595.03284801136	1.14789339636648e-05	\\
6596.01162997159	9.63453627230418e-06	\\
6596.99041193182	8.09183498573784e-06	\\
6597.96919389205	6.91657322637595e-06	\\
6598.94797585227	8.14788150422876e-06	\\
6599.9267578125	5.65903141519818e-06	\\
6600.90553977273	7.95410176061368e-06	\\
6601.88432173295	8.8745005073157e-06	\\
6602.86310369318	1.01087980100824e-05	\\
6603.84188565341	8.85075019182923e-06	\\
6604.82066761364	9.22771843312164e-06	\\
6605.79944957386	8.12163410055809e-06	\\
6606.77823153409	7.09676777452328e-06	\\
6607.75701349432	8.97438428024854e-06	\\
6608.73579545455	7.13131820465463e-06	\\
6609.71457741477	7.35679165065665e-06	\\
6610.693359375	7.48929338727429e-06	\\
6611.67214133523	8.35842317640187e-06	\\
6612.65092329545	1.17643139526075e-05	\\
6613.62970525568	7.78115080433031e-06	\\
6614.60848721591	9.52794152934053e-06	\\
6615.58726917614	9.41341313656236e-06	\\
6616.56605113636	9.92513271534477e-06	\\
6617.54483309659	7.57792674501564e-06	\\
6618.52361505682	8.12614037068377e-06	\\
6619.50239701705	9.04568493133063e-06	\\
6620.48117897727	7.70889018220709e-06	\\
6621.4599609375	8.21160029901711e-06	\\
6622.43874289773	8.29309374561701e-06	\\
6623.41752485795	7.69657575636551e-06	\\
6624.39630681818	8.07429684988261e-06	\\
6625.37508877841	8.70313221660756e-06	\\
6626.35387073864	1.10226612271393e-05	\\
6627.33265269886	1.04374360540913e-05	\\
6628.31143465909	9.98213592108313e-06	\\
6629.29021661932	1.0237007986772e-05	\\
6630.26899857955	8.33022665529308e-06	\\
6631.24778053977	8.83840065427446e-06	\\
6632.2265625	7.77723977103694e-06	\\
6633.20534446023	9.75324129964824e-06	\\
6634.18412642045	1.03021922785626e-05	\\
6635.16290838068	8.14496854481195e-06	\\
6636.14169034091	8.93031469429665e-06	\\
6637.12047230114	8.40698640033759e-06	\\
6638.09925426136	8.24478470264844e-06	\\
6639.07803622159	8.15301604772825e-06	\\
6640.05681818182	7.7393432131257e-06	\\
6641.03560014205	9.04188792731244e-06	\\
6642.01438210227	8.4710847789605e-06	\\
6642.9931640625	9.42905946568649e-06	\\
6643.97194602273	8.35942221762775e-06	\\
6644.95072798295	8.40509803097052e-06	\\
6645.92950994318	9.14660236306365e-06	\\
6646.90829190341	9.66821416685944e-06	\\
6647.88707386364	9.16867215180989e-06	\\
6648.86585582386	6.6163440214099e-06	\\
6649.84463778409	8.48018324896323e-06	\\
6650.82341974432	8.10160158892743e-06	\\
6651.80220170455	8.906199239007e-06	\\
6652.78098366477	7.7467750675083e-06	\\
6653.759765625	8.58278679681963e-06	\\
6654.73854758523	8.6181664051648e-06	\\
6655.71732954545	9.00452994781881e-06	\\
6656.69611150568	9.50994680490533e-06	\\
6657.67489346591	9.53826344719118e-06	\\
6658.65367542614	1.03180391285343e-05	\\
6659.63245738636	9.53907946731429e-06	\\
6660.61123934659	9.27622888415378e-06	\\
6661.59002130682	9.10886248731164e-06	\\
6662.56880326705	9.74627139596369e-06	\\
6663.54758522727	8.93284984438267e-06	\\
6664.5263671875	1.08178504556466e-05	\\
6665.50514914773	1.00939322795894e-05	\\
6666.48393110795	8.17682204161159e-06	\\
6667.46271306818	1.07111681654632e-05	\\
6668.44149502841	9.30252257361185e-06	\\
6669.42027698864	9.56542948509985e-06	\\
6670.39905894886	8.3284056311269e-06	\\
6671.37784090909	8.33879239564638e-06	\\
6672.35662286932	8.63325774640022e-06	\\
6673.33540482955	7.63999271808142e-06	\\
6674.31418678977	7.5399276479446e-06	\\
6675.29296875	8.33235065675407e-06	\\
6676.27175071023	1.01348988518755e-05	\\
6677.25053267045	8.99927179975542e-06	\\
6678.22931463068	9.81352586478657e-06	\\
6679.20809659091	1.03953841852965e-05	\\
6680.18687855114	8.65715636515426e-06	\\
6681.16566051136	8.88705451606145e-06	\\
6682.14444247159	9.5372180173227e-06	\\
6683.12322443182	8.73944900374353e-06	\\
6684.10200639205	9.38021758792692e-06	\\
6685.08078835227	8.20259786646787e-06	\\
6686.0595703125	7.15866206899075e-06	\\
6687.03835227273	8.74743520518248e-06	\\
6688.01713423295	9.94371776018916e-06	\\
6688.99591619318	9.83853821375799e-06	\\
6689.97469815341	9.61386196258446e-06	\\
6690.95348011364	8.6642092885751e-06	\\
6691.93226207386	9.24786299509809e-06	\\
6692.91104403409	1.00030613808072e-05	\\
6693.88982599432	9.84713598775908e-06	\\
6694.86860795455	8.96657795585254e-06	\\
6695.84738991477	8.45258653050449e-06	\\
6696.826171875	1.07537321797601e-05	\\
6697.80495383523	9.45209775696591e-06	\\
6698.78373579545	8.625765164534e-06	\\
6699.76251775568	9.97369715756393e-06	\\
6700.74129971591	7.70571971459989e-06	\\
6701.72008167614	1.01642988624734e-05	\\
6702.69886363636	9.5100819156547e-06	\\
6703.67764559659	9.05353007018672e-06	\\
6704.65642755682	9.29213411559192e-06	\\
6705.63520951705	9.21839600768347e-06	\\
6706.61399147727	9.09152926412029e-06	\\
6707.5927734375	9.06063005085155e-06	\\
6708.57155539773	8.89968500351665e-06	\\
6709.55033735795	7.76108624486727e-06	\\
6710.52911931818	8.13958747268908e-06	\\
6711.50790127841	8.46743482965245e-06	\\
6712.48668323864	7.56136144483642e-06	\\
6713.46546519886	6.50653036636333e-06	\\
6714.44424715909	9.13117657749985e-06	\\
6715.42302911932	7.23796230695458e-06	\\
6716.40181107955	9.39694587538718e-06	\\
6717.38059303977	8.29945171932981e-06	\\
6718.359375	1.00617319266304e-05	\\
6719.33815696023	8.50957696880592e-06	\\
6720.31693892045	8.7009045352236e-06	\\
6721.29572088068	9.35762130780926e-06	\\
6722.27450284091	1.08057861579675e-05	\\
6723.25328480114	8.5482542052135e-06	\\
6724.23206676136	9.05483872578321e-06	\\
6725.21084872159	9.6417326680159e-06	\\
6726.18963068182	1.04567505716889e-05	\\
6727.16841264205	9.40342978836344e-06	\\
6728.14719460227	7.54499121152242e-06	\\
6729.1259765625	7.17542157815719e-06	\\
6730.10475852273	7.69842280518846e-06	\\
6731.08354048295	9.13443321865745e-06	\\
6732.06232244318	9.18535241685117e-06	\\
6733.04110440341	8.91704156237316e-06	\\
6734.01988636364	9.50621818795659e-06	\\
6734.99866832386	9.19378504151034e-06	\\
6735.97745028409	1.12484031208391e-05	\\
6736.95623224432	8.13399361852449e-06	\\
6737.93501420455	8.66054273419572e-06	\\
6738.91379616477	9.31288324620154e-06	\\
6739.892578125	7.58460890703911e-06	\\
6740.87136008523	7.73934785464288e-06	\\
6741.85014204545	1.02500067137294e-05	\\
6742.82892400568	9.53256137231177e-06	\\
6743.80770596591	1.06238466810489e-05	\\
6744.78648792614	9.86657339680656e-06	\\
6745.76526988636	9.03680126875122e-06	\\
6746.74405184659	1.00576068554721e-05	\\
6747.72283380682	7.86436707401889e-06	\\
6748.70161576705	8.57309546027787e-06	\\
6749.68039772727	6.87768519564833e-06	\\
6750.6591796875	7.65950813514811e-06	\\
6751.63796164773	7.62915238233601e-06	\\
6752.61674360795	9.18750976604998e-06	\\
6753.59552556818	8.42317733047447e-06	\\
6754.57430752841	9.26031704656207e-06	\\
6755.55308948864	8.18490101537639e-06	\\
6756.53187144886	8.29703280638344e-06	\\
6757.51065340909	9.41467103202498e-06	\\
6758.48943536932	8.47436560048372e-06	\\
6759.46821732955	8.88656427974344e-06	\\
6760.44699928977	8.0489293150954e-06	\\
6761.42578125	9.6590671751116e-06	\\
6762.40456321023	7.71106444254252e-06	\\
6763.38334517045	8.67787575141346e-06	\\
6764.36212713068	9.29763289987138e-06	\\
6765.34090909091	9.93547755755087e-06	\\
6766.31969105114	9.08105047594054e-06	\\
6767.29847301136	1.02183910058552e-05	\\
6768.27725497159	8.89748738547843e-06	\\
6769.25603693182	8.82763810548976e-06	\\
6770.23481889205	6.99582648074864e-06	\\
6771.21360085227	8.78108609253138e-06	\\
6772.1923828125	8.41405139629978e-06	\\
6773.17116477273	7.89422992773412e-06	\\
6774.14994673295	7.97558379028607e-06	\\
6775.12872869318	8.29598966838552e-06	\\
6776.10751065341	8.01691380189595e-06	\\
6777.08629261364	8.8366680914724e-06	\\
6778.06507457386	8.73247318683444e-06	\\
6779.04385653409	1.01016728761476e-05	\\
6780.02263849432	9.00912788103188e-06	\\
6781.00142045455	8.23047033956252e-06	\\
6781.98020241477	8.87511859053381e-06	\\
6782.958984375	6.68825702621699e-06	\\
6783.93776633523	8.80302846223532e-06	\\
6784.91654829545	9.78295159960482e-06	\\
6785.89533025568	6.97868211902366e-06	\\
6786.87411221591	8.37712367855412e-06	\\
6787.85289417614	8.55487348624896e-06	\\
6788.83167613636	1.03152119649128e-05	\\
6789.81045809659	8.75496524883175e-06	\\
6790.78924005682	6.2709257186579e-06	\\
6791.76802201705	9.44395516150094e-06	\\
6792.74680397727	9.11157167859733e-06	\\
6793.7255859375	7.79376959505928e-06	\\
6794.70436789773	9.46815365964995e-06	\\
6795.68314985795	8.01008294912373e-06	\\
6796.66193181818	8.14040294210955e-06	\\
6797.64071377841	9.4930213600646e-06	\\
6798.61949573864	9.26766059407189e-06	\\
6799.59827769886	6.90892068583812e-06	\\
6800.57705965909	7.16103128485307e-06	\\
6801.55584161932	8.2012626234329e-06	\\
6802.53462357955	8.4733445672691e-06	\\
6803.51340553977	8.94018265151139e-06	\\
6804.4921875	7.97993840959528e-06	\\
6805.47096946023	8.67614935499807e-06	\\
6806.44975142045	8.55255148142962e-06	\\
6807.42853338068	9.48053331178769e-06	\\
6808.40731534091	8.55431391173055e-06	\\
6809.38609730114	7.74565876989236e-06	\\
6810.36487926136	8.12973218164748e-06	\\
6811.34366122159	7.75623358569051e-06	\\
6812.32244318182	1.13313552837764e-05	\\
6813.30122514205	5.53759477217105e-06	\\
6814.28000710227	8.257032016532e-06	\\
6815.2587890625	6.83719069089451e-06	\\
6816.23757102273	7.40424984577076e-06	\\
6817.21635298295	7.98621837334569e-06	\\
6818.19513494318	7.56504181196659e-06	\\
6819.17391690341	9.21756918090596e-06	\\
6820.15269886364	8.54820371171399e-06	\\
6821.13148082386	8.36378054464616e-06	\\
6822.11026278409	7.23394064332118e-06	\\
6823.08904474432	7.61873401293869e-06	\\
6824.06782670455	6.53155173308957e-06	\\
6825.04660866477	7.63673730057234e-06	\\
6826.025390625	8.33431434096703e-06	\\
6827.00417258523	7.89592783391418e-06	\\
6827.98295454545	1.139977376691e-05	\\
6828.96173650568	7.98139380312738e-06	\\
6829.94051846591	8.25112082196884e-06	\\
6830.91930042614	8.15065271848219e-06	\\
6831.89808238636	8.37838188065292e-06	\\
6832.87686434659	7.72843583504009e-06	\\
6833.85564630682	9.59290230276642e-06	\\
6834.83442826705	9.16412524850033e-06	\\
6835.81321022727	8.68843997664126e-06	\\
6836.7919921875	7.2489831472051e-06	\\
6837.77077414773	6.54728051300121e-06	\\
6838.74955610795	9.46429243213288e-06	\\
6839.72833806818	7.13073229860623e-06	\\
6840.70712002841	8.27364608855933e-06	\\
6841.68590198864	8.82386278231353e-06	\\
6842.66468394886	7.92868280509009e-06	\\
6843.64346590909	6.49686546720685e-06	\\
6844.62224786932	7.00589890889143e-06	\\
6845.60102982955	8.21642941630348e-06	\\
6846.57981178977	8.74413538624448e-06	\\
6847.55859375	8.5107140219819e-06	\\
6848.53737571023	7.73003662401777e-06	\\
6849.51615767045	8.27182796917994e-06	\\
6850.49493963068	8.03155212611065e-06	\\
6851.47372159091	7.16420745553849e-06	\\
6852.45250355114	8.40790091639147e-06	\\
6853.43128551136	8.52062813460133e-06	\\
6854.41006747159	6.41207411717823e-06	\\
6855.38884943182	5.71154624174535e-06	\\
6856.36763139205	7.61898562854361e-06	\\
6857.34641335227	9.77933297274445e-06	\\
6858.3251953125	9.41048283972432e-06	\\
6859.30397727273	7.97822454975081e-06	\\
6860.28275923295	6.62493701932556e-06	\\
6861.26154119318	9.21470775375984e-06	\\
6862.24032315341	8.08965907703404e-06	\\
6863.21910511364	7.46045500552109e-06	\\
6864.19788707386	7.03023640242426e-06	\\
6865.17666903409	6.83015485746695e-06	\\
6866.15545099432	7.02344518188825e-06	\\
6867.13423295455	5.55559177016742e-06	\\
6868.11301491477	7.0319611103747e-06	\\
6869.091796875	7.24514952084331e-06	\\
6870.07057883523	7.49977544732995e-06	\\
6871.04936079545	7.7770282013016e-06	\\
6872.02814275568	8.51803823156275e-06	\\
6873.00692471591	8.3121676198323e-06	\\
6873.98570667614	7.7851801529936e-06	\\
6874.96448863636	7.66048680218732e-06	\\
6875.94327059659	7.27755756803895e-06	\\
6876.92205255682	5.87888607465419e-06	\\
6877.90083451705	6.58082581182403e-06	\\
6878.87961647727	6.32946621540681e-06	\\
6879.8583984375	9.59793177986405e-06	\\
6880.83718039773	6.22245618678354e-06	\\
6881.81596235795	6.72203228895045e-06	\\
6882.79474431818	5.7662793406581e-06	\\
6883.77352627841	6.42286432627453e-06	\\
6884.75230823864	8.04125702193286e-06	\\
6885.73109019886	7.44547177700889e-06	\\
6886.70987215909	7.61865858812679e-06	\\
6887.68865411932	8.20401175615727e-06	\\
6888.66743607955	5.95482020938743e-06	\\
6889.64621803977	7.53856431930615e-06	\\
6890.625	5.30430374918714e-06	\\
6891.60378196023	5.88365262489208e-06	\\
6892.58256392045	6.39830919597452e-06	\\
6893.56134588068	6.74505698089759e-06	\\
6894.54012784091	7.44694543667209e-06	\\
6895.51890980114	6.35996821127284e-06	\\
6896.49769176136	6.52433486770997e-06	\\
6897.47647372159	8.28782613681746e-06	\\
6898.45525568182	6.8596943345537e-06	\\
6899.43403764205	4.16091648161686e-06	\\
6900.41281960227	6.76615754161179e-06	\\
6901.3916015625	7.34439448680706e-06	\\
6902.37038352273	6.53290037283445e-06	\\
6903.34916548295	6.5251071387942e-06	\\
6904.32794744318	5.54526537493576e-06	\\
6905.30672940341	6.81628411593292e-06	\\
6906.28551136364	6.47852106222336e-06	\\
6907.26429332386	8.43425351677292e-06	\\
6908.24307528409	6.45318486092633e-06	\\
6909.22185724432	7.90708370099686e-06	\\
6910.20063920455	7.31279931728524e-06	\\
6911.17942116477	6.70554480401016e-06	\\
6912.158203125	8.40438889670031e-06	\\
6913.13698508523	7.52916644661108e-06	\\
6914.11576704545	8.62613494410436e-06	\\
6915.09454900568	8.15001460020523e-06	\\
6916.07333096591	7.77002952489715e-06	\\
6917.05211292614	8.16566985631184e-06	\\
6918.03089488636	7.7162451847681e-06	\\
6919.00967684659	6.17257222450178e-06	\\
6919.98845880682	5.60922736364991e-06	\\
6920.96724076705	8.03908144732432e-06	\\
6921.94602272727	7.821179319207e-06	\\
6922.9248046875	8.65582585680134e-06	\\
6923.90358664773	6.13782858001602e-06	\\
6924.88236860795	8.51119593773321e-06	\\
6925.86115056818	7.06542248822119e-06	\\
6926.83993252841	6.29088657245195e-06	\\
6927.81871448864	6.83976779748324e-06	\\
6928.79749644886	7.06417647686342e-06	\\
6929.77627840909	9.77957160016899e-06	\\
6930.75506036932	8.30173240269811e-06	\\
6931.73384232955	5.70846563500666e-06	\\
6932.71262428977	7.10837381022336e-06	\\
6933.69140625	7.52837088840779e-06	\\
6934.67018821023	6.97510896641872e-06	\\
6935.64897017045	7.50511983145775e-06	\\
6936.62775213068	8.62508669667896e-06	\\
6937.60653409091	7.22310094243999e-06	\\
6938.58531605114	7.41910268711874e-06	\\
6939.56409801136	9.03352394046385e-06	\\
6940.54287997159	1.04984731875081e-05	\\
6941.52166193182	6.03778120984488e-06	\\
6942.50044389205	5.60406993002682e-06	\\
6943.47922585227	7.39859321719608e-06	\\
6944.4580078125	6.13553105589811e-06	\\
6945.43678977273	7.99713108028011e-06	\\
6946.41557173295	8.78807011527112e-06	\\
6947.39435369318	7.16589481424747e-06	\\
6948.37313565341	7.93372788385336e-06	\\
6949.35191761364	7.76176988295512e-06	\\
6950.33069957386	8.63171142741645e-06	\\
6951.30948153409	7.66973923804105e-06	\\
6952.28826349432	8.4685796414059e-06	\\
6953.26704545455	7.24469630639574e-06	\\
6954.24582741477	7.7751319840246e-06	\\
6955.224609375	8.9466101838131e-06	\\
6956.20339133523	6.30653935246946e-06	\\
6957.18217329545	7.91896954526421e-06	\\
6958.16095525568	8.0273446546867e-06	\\
6959.13973721591	7.64717899811418e-06	\\
6960.11851917614	8.49823311642144e-06	\\
6961.09730113636	6.21076420358392e-06	\\
6962.07608309659	7.8025191396427e-06	\\
6963.05486505682	9.18888765655051e-06	\\
6964.03364701705	9.12971745664778e-06	\\
6965.01242897727	7.73409370388112e-06	\\
6965.9912109375	6.89294496954417e-06	\\
6966.96999289773	7.76106205941881e-06	\\
6967.94877485795	8.53669865915874e-06	\\
6968.92755681818	5.06234118134543e-06	\\
6969.90633877841	6.84890769142607e-06	\\
6970.88512073864	7.32664185660932e-06	\\
6971.86390269886	6.81693722130165e-06	\\
6972.84268465909	7.61290419568891e-06	\\
6973.82146661932	6.10307889772207e-06	\\
6974.80024857955	7.85028280955507e-06	\\
6975.77903053977	7.37427594013429e-06	\\
6976.7578125	7.65584481207799e-06	\\
6977.73659446023	9.73129811288489e-06	\\
6978.71537642045	8.07169887827452e-06	\\
6979.69415838068	7.43376903813898e-06	\\
6980.67294034091	7.27849839535736e-06	\\
6981.65172230114	9.66590316837131e-06	\\
6982.63050426136	8.36205725334228e-06	\\
6983.60928622159	9.07180553513619e-06	\\
6984.58806818182	6.9474852251888e-06	\\
6985.56685014205	7.47983014874244e-06	\\
6986.54563210227	6.62195324100094e-06	\\
6987.5244140625	9.23759903888994e-06	\\
6988.50319602273	6.61618535393836e-06	\\
6989.48197798295	7.76193656143744e-06	\\
6990.46075994318	7.69618047315033e-06	\\
6991.43954190341	7.31065204143368e-06	\\
6992.41832386364	5.67321039758199e-06	\\
6993.39710582386	7.7566320937903e-06	\\
6994.37588778409	6.04849683373395e-06	\\
6995.35466974432	6.00070126224224e-06	\\
6996.33345170455	8.25694771021647e-06	\\
6997.31223366477	8.00017641561377e-06	\\
6998.291015625	6.33224357818361e-06	\\
6999.26979758523	7.1751089506814e-06	\\
7000.24857954545	6.02978011259439e-06	\\
7001.22736150568	8.44156895375992e-06	\\
7002.20614346591	7.68052734757269e-06	\\
7003.18492542614	6.26388428265914e-06	\\
7004.16370738636	7.99133800194756e-06	\\
7005.14248934659	7.2483925750742e-06	\\
7006.12127130682	6.88670606235209e-06	\\
7007.10005326705	7.06216201938104e-06	\\
7008.07883522727	8.32579820900407e-06	\\
7009.0576171875	7.56977363421178e-06	\\
7010.03639914773	6.44912145501791e-06	\\
7011.01518110795	8.22758837968694e-06	\\
7011.99396306818	6.31981215526694e-06	\\
7012.97274502841	7.49395796366304e-06	\\
7013.95152698864	8.01637604024937e-06	\\
7014.93030894886	7.10774079856555e-06	\\
7015.90909090909	7.76300719149643e-06	\\
7016.88787286932	6.97428069322005e-06	\\
7017.86665482955	7.97015811829325e-06	\\
7018.84543678977	5.65474131380739e-06	\\
7019.82421875	7.18831410357202e-06	\\
7020.80300071023	6.63929767161254e-06	\\
7021.78178267045	6.42095404251673e-06	\\
7022.76056463068	6.88786878433944e-06	\\
7023.73934659091	6.34474867049268e-06	\\
7024.71812855114	7.54795260979798e-06	\\
7025.69691051136	8.34711457304737e-06	\\
7026.67569247159	7.17231608515409e-06	\\
7027.65447443182	8.80056983473404e-06	\\
7028.63325639205	6.57826624437797e-06	\\
7029.61203835227	7.14848958720797e-06	\\
7030.5908203125	7.08569820323663e-06	\\
7031.56960227273	8.26750953404539e-06	\\
7032.54838423295	7.51295850034926e-06	\\
7033.52716619318	6.97694805701585e-06	\\
7034.50594815341	7.63879514666616e-06	\\
7035.48473011364	6.08244498726991e-06	\\
7036.46351207386	5.72170917820912e-06	\\
7037.44229403409	7.19035577878124e-06	\\
7038.42107599432	8.51368054342633e-06	\\
7039.39985795455	6.5414371994137e-06	\\
7040.37863991477	7.04040470728757e-06	\\
7041.357421875	6.00941322265071e-06	\\
7042.33620383523	6.81766720292981e-06	\\
7043.31498579545	7.97832069114485e-06	\\
7044.29376775568	6.16989112251398e-06	\\
7045.27254971591	9.50087333413767e-06	\\
7046.25133167614	9.02157223651989e-06	\\
7047.23011363636	7.82446875851799e-06	\\
7048.20889559659	8.37172940509821e-06	\\
7049.18767755682	8.44026016725964e-06	\\
7050.16645951705	9.63793717854709e-06	\\
7051.14524147727	6.4306516279798e-06	\\
7052.1240234375	7.19203479357354e-06	\\
7053.10280539773	7.68953160628301e-06	\\
7054.08158735795	8.82626749012948e-06	\\
7055.06036931818	8.14986664002446e-06	\\
7056.03915127841	9.6963491379321e-06	\\
7057.01793323864	6.3397903568603e-06	\\
7057.99671519886	6.50868919350289e-06	\\
7058.97549715909	8.48947484421978e-06	\\
7059.95427911932	6.62007008067286e-06	\\
7060.93306107955	6.08809368363363e-06	\\
7061.91184303977	8.194811718247e-06	\\
7062.890625	9.39609849855319e-06	\\
7063.86940696023	7.17101471776314e-06	\\
7064.84818892045	7.05085328696641e-06	\\
7065.82697088068	7.81921406654561e-06	\\
7066.80575284091	8.49406889325202e-06	\\
7067.78453480114	8.10034391424023e-06	\\
7068.76331676136	6.28991222761221e-06	\\
7069.74209872159	6.33472628198794e-06	\\
7070.72088068182	5.96068517823382e-06	\\
7071.69966264205	7.20631154071716e-06	\\
7072.67844460227	7.40464278935221e-06	\\
7073.6572265625	6.09799558802595e-06	\\
7074.63600852273	6.59291178858422e-06	\\
7075.61479048295	7.19640220700735e-06	\\
7076.59357244318	8.32279177368947e-06	\\
7077.57235440341	7.60427328330904e-06	\\
7078.55113636364	7.64625002286116e-06	\\
7079.52991832386	7.15793310451504e-06	\\
7080.50870028409	6.14273472337957e-06	\\
7081.48748224432	7.6820729209556e-06	\\
7082.46626420455	5.75027259681208e-06	\\
7083.44504616477	7.34413870059563e-06	\\
7084.423828125	8.58172338354708e-06	\\
7085.40261008523	6.0381687668839e-06	\\
7086.38139204545	7.02325016741782e-06	\\
7087.36017400568	7.84481680670191e-06	\\
7088.33895596591	7.60654874615164e-06	\\
7089.31773792614	7.74608390033138e-06	\\
7090.29651988636	5.43662557228974e-06	\\
7091.27530184659	7.48714787871677e-06	\\
7092.25408380682	7.26621739810602e-06	\\
7093.23286576705	7.67880221611285e-06	\\
7094.21164772727	9.24605604418464e-06	\\
7095.1904296875	5.71361407030767e-06	\\
7096.16921164773	8.18788892010978e-06	\\
7097.14799360795	8.0233439362491e-06	\\
7098.12677556818	6.14829869970677e-06	\\
7099.10555752841	6.44996163395307e-06	\\
7100.08433948864	6.81194114296618e-06	\\
7101.06312144886	7.7593460994324e-06	\\
7102.04190340909	6.35794857878241e-06	\\
7103.02068536932	8.05026968050711e-06	\\
7103.99946732955	7.22763062162162e-06	\\
7104.97824928977	6.8778189146414e-06	\\
7105.95703125	8.60331574247562e-06	\\
7106.93581321023	6.36603054493093e-06	\\
7107.91459517045	6.18784265429866e-06	\\
7108.89337713068	7.01798188225225e-06	\\
7109.87215909091	6.74904365828668e-06	\\
7110.85094105114	8.96515683554984e-06	\\
7111.82972301136	8.87170505040418e-06	\\
7112.80850497159	7.45901460621416e-06	\\
7113.78728693182	8.00295941952578e-06	\\
7114.76606889205	6.31767583086869e-06	\\
7115.74485085227	6.89448867793391e-06	\\
7116.7236328125	7.61914334974998e-06	\\
7117.70241477273	8.83339092084226e-06	\\
7118.68119673295	7.39163225169661e-06	\\
7119.65997869318	7.50645387781097e-06	\\
7120.63876065341	8.55700786420301e-06	\\
7121.61754261364	7.2976200577416e-06	\\
7122.59632457386	7.11753049773195e-06	\\
7123.57510653409	8.59802229411554e-06	\\
7124.55388849432	9.83293249660877e-06	\\
7125.53267045455	7.04406111629905e-06	\\
7126.51145241477	8.24538343534834e-06	\\
7127.490234375	7.82296212700184e-06	\\
7128.46901633523	7.94846534690065e-06	\\
7129.44779829545	8.43486020743645e-06	\\
7130.42658025568	8.55926609874135e-06	\\
7131.40536221591	7.71190995978677e-06	\\
7132.38414417614	6.41056212521217e-06	\\
7133.36292613636	6.82899385348623e-06	\\
7134.34170809659	7.0563995019826e-06	\\
7135.32049005682	8.22337848866845e-06	\\
7136.29927201705	7.17494080498194e-06	\\
7137.27805397727	8.39607848231639e-06	\\
7138.2568359375	7.69673787473052e-06	\\
7139.23561789773	7.62331818537281e-06	\\
7140.21439985795	6.75271538040343e-06	\\
7141.19318181818	7.26077280029249e-06	\\
7142.17196377841	7.01727232825603e-06	\\
7143.15074573864	6.55518836088163e-06	\\
7144.12952769886	5.29502879022268e-06	\\
7145.10830965909	6.36263289604195e-06	\\
7146.08709161932	5.86564712572309e-06	\\
7147.06587357955	6.72837885550174e-06	\\
7148.04465553977	8.11849879113746e-06	\\
7149.0234375	7.45880086292809e-06	\\
7150.00221946023	6.56241550955294e-06	\\
7150.98100142045	6.33180808047232e-06	\\
7151.95978338068	8.36803740379342e-06	\\
7152.93856534091	9.67009030487546e-06	\\
7153.91734730114	9.10100552272711e-06	\\
7154.89612926136	5.87987391127324e-06	\\
7155.87491122159	7.75048685508946e-06	\\
7156.85369318182	7.90854877340173e-06	\\
7157.83247514205	7.51062823682084e-06	\\
7158.81125710227	6.32990049114985e-06	\\
7159.7900390625	8.05295850586202e-06	\\
7160.76882102273	6.19649540355815e-06	\\
7161.74760298295	8.12522746791528e-06	\\
7162.72638494318	7.22024483218773e-06	\\
7163.70516690341	6.49536359087921e-06	\\
7164.68394886364	6.77392553126582e-06	\\
7165.66273082386	7.48985471561147e-06	\\
7166.64151278409	7.63607071314866e-06	\\
7167.62029474432	7.32656469873057e-06	\\
7168.59907670455	8.47535756300803e-06	\\
7169.57785866477	6.09735125703818e-06	\\
7170.556640625	9.64352416676435e-06	\\
7171.53542258523	6.57258590548881e-06	\\
7172.51420454545	6.21172359577567e-06	\\
7173.49298650568	7.93466103467323e-06	\\
7174.47176846591	6.190879848825e-06	\\
7175.45055042614	6.45477074006246e-06	\\
7176.42933238636	7.35960066877867e-06	\\
7177.40811434659	8.06839296011227e-06	\\
7178.38689630682	5.67421526169271e-06	\\
7179.36567826705	8.04330205674493e-06	\\
7180.34446022727	8.2947711218696e-06	\\
7181.3232421875	7.89160263472128e-06	\\
7182.30202414773	7.18374785826592e-06	\\
7183.28080610795	8.64537816982061e-06	\\
7184.25958806818	8.74049350198356e-06	\\
7185.23837002841	8.56788938681453e-06	\\
7186.21715198864	7.8721745920631e-06	\\
7187.19593394886	8.89968055793045e-06	\\
7188.17471590909	7.22635251794979e-06	\\
7189.15349786932	4.74846954919541e-06	\\
7190.13227982955	6.10634169223411e-06	\\
7191.11106178977	5.80677128137462e-06	\\
7192.08984375	8.5756250916233e-06	\\
7193.06862571023	6.07342555475604e-06	\\
7194.04740767045	7.65913259313612e-06	\\
7195.02618963068	7.08745690410591e-06	\\
7196.00497159091	7.63316793979957e-06	\\
7196.98375355114	7.14864384531712e-06	\\
7197.96253551136	7.47942912373917e-06	\\
7198.94131747159	6.34473526908624e-06	\\
7199.92009943182	8.39886153463822e-06	\\
7200.89888139205	6.37543476274329e-06	\\
7201.87766335227	6.73516960660537e-06	\\
7202.8564453125	8.87178859962746e-06	\\
7203.83522727273	8.21856729011441e-06	\\
7204.81400923295	6.06405219925214e-06	\\
7205.79279119318	7.24075324393625e-06	\\
7206.77157315341	7.91249854689741e-06	\\
7207.75035511364	8.61539734940119e-06	\\
7208.72913707386	7.90620523685174e-06	\\
7209.70791903409	6.18688171235894e-06	\\
7210.68670099432	7.74116705654962e-06	\\
7211.66548295455	7.42216468194724e-06	\\
7212.64426491477	1.01000522595184e-05	\\
7213.623046875	8.03025107124466e-06	\\
7214.60182883523	8.53183592144357e-06	\\
7215.58061079545	6.59007988005851e-06	\\
7216.55939275568	7.51532622076678e-06	\\
7217.53817471591	7.11178145807949e-06	\\
7218.51695667614	8.9903049899673e-06	\\
7219.49573863636	7.30750514333571e-06	\\
7220.47452059659	7.56754054126153e-06	\\
7221.45330255682	6.22495415137758e-06	\\
7222.43208451705	7.40913880093945e-06	\\
7223.41086647727	6.87395501718881e-06	\\
7224.3896484375	7.2349139305588e-06	\\
7225.36843039773	8.37195928217809e-06	\\
7226.34721235795	8.90637680127492e-06	\\
7227.32599431818	8.69209985734058e-06	\\
7228.30477627841	6.20953723160516e-06	\\
7229.28355823864	7.16703915449778e-06	\\
7230.26234019886	7.06653579893607e-06	\\
7231.24112215909	7.41147527101019e-06	\\
7232.21990411932	7.61687710290394e-06	\\
7233.19868607955	9.48079982406965e-06	\\
7234.17746803977	7.033351804729e-06	\\
7235.15625	8.22177818094696e-06	\\
7236.13503196023	7.65752498840226e-06	\\
7237.11381392045	7.60472975506526e-06	\\
7238.09259588068	5.15008037516752e-06	\\
7239.07137784091	6.22306907302989e-06	\\
7240.05015980114	6.74042362993081e-06	\\
7241.02894176136	9.2676227821999e-06	\\
7242.00772372159	8.89178775029283e-06	\\
7242.98650568182	8.23777139579437e-06	\\
7243.96528764205	7.57190769252675e-06	\\
7244.94406960227	8.92515110135373e-06	\\
7245.9228515625	7.09157429836578e-06	\\
7246.90163352273	7.35629635106711e-06	\\
7247.88041548295	7.86056109171741e-06	\\
7248.85919744318	4.93874623043235e-06	\\
7249.83797940341	7.43470362748739e-06	\\
7250.81676136364	5.66487254058786e-06	\\
7251.79554332386	7.31767394636423e-06	\\
7252.77432528409	7.69696712328209e-06	\\
7253.75310724432	7.53275351728116e-06	\\
7254.73188920455	9.6820104707271e-06	\\
7255.71067116477	8.69812118740948e-06	\\
7256.689453125	6.81724997065809e-06	\\
7257.66823508523	8.25682620390124e-06	\\
7258.64701704545	7.10816346032765e-06	\\
7259.62579900568	6.84758839257821e-06	\\
7260.60458096591	6.26288930075115e-06	\\
7261.58336292614	8.02279938989176e-06	\\
7262.56214488636	8.72495637189335e-06	\\
7263.54092684659	7.77486537280873e-06	\\
7264.51970880682	7.25195907612798e-06	\\
7265.49849076705	7.67273026609362e-06	\\
7266.47727272727	7.29132412412152e-06	\\
7267.4560546875	8.44500139113236e-06	\\
7268.43483664773	7.11178318522132e-06	\\
7269.41361860795	6.44408886417046e-06	\\
7270.39240056818	7.09846144525353e-06	\\
7271.37118252841	7.45181333429815e-06	\\
7272.34996448864	7.3636525272792e-06	\\
7273.32874644886	7.64172742495545e-06	\\
7274.30752840909	8.98902903933722e-06	\\
7275.28631036932	9.4428511525001e-06	\\
7276.26509232955	6.9565522488552e-06	\\
7277.24387428977	7.42986980555274e-06	\\
7278.22265625	7.92283853064536e-06	\\
7279.20143821023	9.4712302371885e-06	\\
7280.18022017045	8.69913359207085e-06	\\
7281.15900213068	7.28647543177025e-06	\\
7282.13778409091	7.2217431541526e-06	\\
7283.11656605114	9.35102670115488e-06	\\
7284.09534801136	7.65582232420925e-06	\\
7285.07412997159	5.79328871512702e-06	\\
7286.05291193182	8.32366829018674e-06	\\
7287.03169389205	7.50411370344459e-06	\\
7288.01047585227	7.95807082278338e-06	\\
7288.9892578125	5.8306283041132e-06	\\
7289.96803977273	9.91154679672859e-06	\\
7290.94682173295	7.47010371955684e-06	\\
7291.92560369318	5.1590980039015e-06	\\
7292.90438565341	6.38325222370823e-06	\\
7293.88316761364	5.77832263952599e-06	\\
7294.86194957386	7.67408460298161e-06	\\
7295.84073153409	8.5404546650854e-06	\\
7296.81951349432	6.30597025088627e-06	\\
7297.79829545455	6.68257998410375e-06	\\
7298.77707741477	8.1442895764761e-06	\\
7299.755859375	9.03179529860268e-06	\\
7300.73464133523	7.58861290579415e-06	\\
7301.71342329545	6.96416108119708e-06	\\
7302.69220525568	6.49489098962554e-06	\\
7303.67098721591	8.19318005178325e-06	\\
7304.64976917614	7.08229289220203e-06	\\
7305.62855113636	9.04104692112966e-06	\\
7306.60733309659	8.93266701196232e-06	\\
7307.58611505682	7.03974686286206e-06	\\
7308.56489701705	6.84857045488597e-06	\\
7309.54367897727	6.7755110860804e-06	\\
7310.5224609375	7.37090817547145e-06	\\
7311.50124289773	6.89364223590414e-06	\\
7312.48002485795	9.1196080518218e-06	\\
7313.45880681818	8.44680292645823e-06	\\
7314.43758877841	7.5669056200834e-06	\\
7315.41637073864	7.89345473434266e-06	\\
7316.39515269886	5.64600274123838e-06	\\
7317.37393465909	7.1297593572239e-06	\\
7318.35271661932	7.12647961954999e-06	\\
7319.33149857955	6.73080050377506e-06	\\
7320.31028053977	9.7004420507217e-06	\\
7321.2890625	6.42594374662987e-06	\\
7322.26784446023	9.20003044805411e-06	\\
7323.24662642045	7.1033244971599e-06	\\
7324.22540838068	7.02733683295656e-06	\\
7325.20419034091	7.55568886626585e-06	\\
7326.18297230114	6.80449365831686e-06	\\
7327.16175426136	7.87237890873382e-06	\\
7328.14053622159	5.83539775362214e-06	\\
7329.11931818182	7.11193318063158e-06	\\
7330.09810014205	8.52640026400574e-06	\\
7331.07688210227	7.74095122925083e-06	\\
7332.0556640625	6.86794538690958e-06	\\
7333.03444602273	8.97680353051118e-06	\\
7334.01322798295	5.795250898385e-06	\\
7334.99200994318	6.65943591401843e-06	\\
7335.97079190341	6.40156298872383e-06	\\
7336.94957386364	6.9439778914668e-06	\\
7337.92835582386	6.78757285384708e-06	\\
7338.90713778409	9.68493799396602e-06	\\
7339.88591974432	8.08276315935173e-06	\\
7340.86470170455	6.98389042424355e-06	\\
7341.84348366477	6.29385114467489e-06	\\
7342.822265625	6.80033072961104e-06	\\
7343.80104758523	8.59326966755238e-06	\\
7344.77982954545	6.27180728712046e-06	\\
7345.75861150568	7.17360500436683e-06	\\
7346.73739346591	5.47688015064364e-06	\\
7347.71617542614	6.72359441571493e-06	\\
7348.69495738636	7.49752919131796e-06	\\
7349.67373934659	7.41482716539655e-06	\\
7350.65252130682	7.17485033284121e-06	\\
7351.63130326705	6.21107058241052e-06	\\
7352.61008522727	5.59877281359956e-06	\\
7353.5888671875	5.75371583787116e-06	\\
7354.56764914773	8.37190644591138e-06	\\
7355.54643110795	9.11106621818492e-06	\\
7356.52521306818	6.99544913784337e-06	\\
7357.50399502841	5.90499242918471e-06	\\
7358.48277698864	6.11220985793961e-06	\\
7359.46155894886	6.56909513478965e-06	\\
7360.44034090909	5.71151725701256e-06	\\
7361.41912286932	6.62432300877102e-06	\\
7362.39790482955	7.09929345711125e-06	\\
7363.37668678977	6.58612170407381e-06	\\
7364.35546875	5.9132173193907e-06	\\
7365.33425071023	7.75755753887573e-06	\\
7366.31303267045	7.6286378450734e-06	\\
7367.29181463068	8.36055301836536e-06	\\
7368.27059659091	5.59598358582301e-06	\\
7369.24937855114	6.52520007337011e-06	\\
7370.22816051136	7.70413806850497e-06	\\
7371.20694247159	5.44990720520275e-06	\\
7372.18572443182	7.08837676473322e-06	\\
7373.16450639205	6.49374502472284e-06	\\
7374.14328835227	6.98686092155367e-06	\\
7375.1220703125	7.15155376893296e-06	\\
7376.10085227273	8.06668304390303e-06	\\
7377.07963423295	6.32593845237242e-06	\\
7378.05841619318	5.94136912027241e-06	\\
7379.03719815341	7.81475247297311e-06	\\
7380.01598011364	6.5427793046986e-06	\\
7380.99476207386	6.99173289586734e-06	\\
7381.97354403409	8.36261260389242e-06	\\
7382.95232599432	7.27433698359921e-06	\\
7383.93110795455	6.34679682091453e-06	\\
7384.90988991477	7.61304159498715e-06	\\
7385.888671875	6.12874931738588e-06	\\
7386.86745383523	6.84515601611667e-06	\\
7387.84623579545	7.56204426849086e-06	\\
7388.82501775568	6.0569791015925e-06	\\
7389.80379971591	7.73902325261901e-06	\\
7390.78258167614	5.88404766395498e-06	\\
7391.76136363636	6.89333769606559e-06	\\
7392.74014559659	6.95886049068948e-06	\\
7393.71892755682	5.76732990509443e-06	\\
7394.69770951705	7.47194473945103e-06	\\
7395.67649147727	6.52540490642452e-06	\\
7396.6552734375	6.50259624754593e-06	\\
7397.63405539773	7.08249707477436e-06	\\
7398.61283735795	5.73434540396136e-06	\\
7399.59161931818	5.91346774263805e-06	\\
7400.57040127841	5.6852271792752e-06	\\
7401.54918323864	6.60572055698279e-06	\\
7402.52796519886	7.02353853487544e-06	\\
7403.50674715909	6.44787705213248e-06	\\
7404.48552911932	5.19404248403118e-06	\\
7405.46431107955	8.0755355688736e-06	\\
7406.44309303977	6.67538302376173e-06	\\
7407.421875	7.54036950044218e-06	\\
7408.40065696023	7.16230640777292e-06	\\
7409.37943892045	7.49806571200374e-06	\\
7410.35822088068	8.13420656528205e-06	\\
7411.33700284091	8.26338940070401e-06	\\
7412.31578480114	8.09847195768214e-06	\\
7413.29456676136	6.50009269197249e-06	\\
7414.27334872159	7.46792079054806e-06	\\
7415.25213068182	5.62439187539682e-06	\\
7416.23091264205	9.7640954389016e-06	\\
7417.20969460227	6.12923205522308e-06	\\
7418.1884765625	7.29561093522968e-06	\\
7419.16725852273	5.77564870148018e-06	\\
7420.14604048295	6.1253085794485e-06	\\
7421.12482244318	7.06225885218698e-06	\\
7422.10360440341	8.19540128758041e-06	\\
7423.08238636364	7.74815816631365e-06	\\
7424.06116832386	6.06335206947496e-06	\\
7425.03995028409	7.59703134936223e-06	\\
7426.01873224432	6.50253078207591e-06	\\
7426.99751420455	7.21864572192426e-06	\\
7427.97629616477	7.13282060601689e-06	\\
7428.955078125	6.01400267504653e-06	\\
7429.93386008523	8.87624115084493e-06	\\
7430.91264204545	7.62817906903421e-06	\\
7431.89142400568	5.50218980892996e-06	\\
7432.87020596591	7.67690099177946e-06	\\
7433.84898792614	7.28663534341646e-06	\\
7434.82776988636	7.33172941246108e-06	\\
7435.80655184659	6.81333977605015e-06	\\
7436.78533380682	5.15752916752832e-06	\\
7437.76411576705	5.80283217417246e-06	\\
7438.74289772727	7.63120737914648e-06	\\
7439.7216796875	6.91509097710764e-06	\\
7440.70046164773	7.75212742891379e-06	\\
7441.67924360795	7.31987094840749e-06	\\
7442.65802556818	7.31563827385396e-06	\\
7443.63680752841	7.14578054382375e-06	\\
7444.61558948864	7.95235727351174e-06	\\
7445.59437144886	5.57290441061874e-06	\\
7446.57315340909	5.80004917709387e-06	\\
7447.55193536932	7.26148440035081e-06	\\
7448.53071732955	7.14707839780892e-06	\\
7449.50949928977	7.12704027649723e-06	\\
7450.48828125	7.6245097043836e-06	\\
7451.46706321023	7.86920504162509e-06	\\
7452.44584517045	7.86044114656058e-06	\\
7453.42462713068	5.07611855330074e-06	\\
7454.40340909091	6.34031783424859e-06	\\
7455.38219105114	6.94536518874204e-06	\\
7456.36097301136	6.55008423377766e-06	\\
7457.33975497159	5.98331073353354e-06	\\
7458.31853693182	7.47936167792316e-06	\\
7459.29731889205	8.24095115533304e-06	\\
7460.27610085227	8.91906039416276e-06	\\
7461.2548828125	8.56528970008593e-06	\\
7462.23366477273	5.4310953277899e-06	\\
7463.21244673295	7.94020862880358e-06	\\
7464.19122869318	6.97414797832275e-06	\\
7465.17001065341	7.0085849600365e-06	\\
7466.14879261364	9.34953809608573e-06	\\
7467.12757457386	6.44465653735513e-06	\\
7468.10635653409	5.04923625150578e-06	\\
7469.08513849432	8.22368575269257e-06	\\
7470.06392045455	6.88528148953993e-06	\\
7471.04270241477	5.88991664024032e-06	\\
7472.021484375	7.2060567617614e-06	\\
7473.00026633523	6.54982869432187e-06	\\
7473.97904829545	7.89889408717336e-06	\\
7474.95783025568	8.0097328767653e-06	\\
7475.93661221591	7.49313136432587e-06	\\
7476.91539417614	7.42273716809142e-06	\\
7477.89417613636	7.06486906252365e-06	\\
7478.87295809659	3.71592858978315e-06	\\
7479.85174005682	7.58949711483941e-06	\\
7480.83052201705	7.19859544304399e-06	\\
7481.80930397727	6.96120796443884e-06	\\
7482.7880859375	6.43065933961594e-06	\\
7483.76686789773	7.09816318086824e-06	\\
7484.74564985795	8.03769981980214e-06	\\
7485.72443181818	6.3277351796544e-06	\\
7486.70321377841	6.1857491522079e-06	\\
7487.68199573864	5.64098570294191e-06	\\
7488.66077769886	6.96275415600699e-06	\\
7489.63955965909	8.34211599271624e-06	\\
7490.61834161932	6.80452592244987e-06	\\
7491.59712357955	6.73117409602761e-06	\\
7492.57590553977	5.25730301398761e-06	\\
7493.5546875	7.82541092349935e-06	\\
7494.53346946023	7.99516790922754e-06	\\
7495.51225142045	6.98979261588375e-06	\\
7496.49103338068	7.17111740411291e-06	\\
7497.46981534091	8.78292037802042e-06	\\
7498.44859730114	7.26306928434436e-06	\\
7499.42737926136	5.23134679130635e-06	\\
7500.40616122159	5.48078151146432e-06	\\
7501.38494318182	8.29441208187783e-06	\\
7502.36372514205	9.8783464073172e-06	\\
7503.34250710227	7.29136048747605e-06	\\
7504.3212890625	5.54492051261883e-06	\\
7505.30007102273	7.40802950003423e-06	\\
7506.27885298295	7.53657110186183e-06	\\
7507.25763494318	5.98754142305147e-06	\\
7508.23641690341	6.94657960894001e-06	\\
7509.21519886364	7.17605520090065e-06	\\
7510.19398082386	6.72450956403874e-06	\\
7511.17276278409	6.56881415758782e-06	\\
7512.15154474432	6.12276568950959e-06	\\
7513.13032670455	4.34336914401322e-06	\\
7514.10910866477	5.31184785757744e-06	\\
7515.087890625	6.89989538774475e-06	\\
7516.06667258523	7.04764066519115e-06	\\
7517.04545454545	7.66370782668082e-06	\\
7518.02423650568	7.776812872464e-06	\\
7519.00301846591	9.36955716265797e-06	\\
7519.98180042614	7.7855715097702e-06	\\
7520.96058238636	6.21516523871878e-06	\\
7521.93936434659	7.09070093795218e-06	\\
7522.91814630682	7.74365811927892e-06	\\
7523.89692826705	8.26929877323565e-06	\\
7524.87571022727	6.43431012781147e-06	\\
7525.8544921875	7.06361096969432e-06	\\
7526.83327414773	7.12353457430497e-06	\\
7527.81205610795	7.30526614339531e-06	\\
7528.79083806818	6.39195272244843e-06	\\
7529.76962002841	7.0319002436185e-06	\\
7530.74840198864	8.37516350096574e-06	\\
7531.72718394886	8.78514732862057e-06	\\
7532.70596590909	6.16065156032711e-06	\\
7533.68474786932	8.75959821563163e-06	\\
7534.66352982955	7.23410396185401e-06	\\
7535.64231178977	7.62301135672183e-06	\\
7536.62109375	8.63983448889906e-06	\\
7537.59987571023	7.41271124781164e-06	\\
7538.57865767045	6.68371595521747e-06	\\
7539.55743963068	7.58761704093343e-06	\\
7540.53622159091	6.86761354552136e-06	\\
7541.51500355114	6.81804845066364e-06	\\
7542.49378551136	6.01423593402618e-06	\\
7543.47256747159	7.34635109593241e-06	\\
7544.45134943182	5.08086456350578e-06	\\
7545.43013139205	7.21386861949282e-06	\\
7546.40891335227	6.47479452342416e-06	\\
7547.3876953125	7.23109190298096e-06	\\
7548.36647727273	5.82575962880233e-06	\\
7549.34525923295	6.46360876159158e-06	\\
7550.32404119318	5.84519310039552e-06	\\
7551.30282315341	5.78305764142678e-06	\\
7552.28160511364	6.38210468842217e-06	\\
7553.26038707386	6.41532061102203e-06	\\
7554.23916903409	5.80883585277148e-06	\\
7555.21795099432	8.67882937963078e-06	\\
7556.19673295455	6.06488922825748e-06	\\
7557.17551491477	7.69385780228501e-06	\\
7558.154296875	6.50506891226993e-06	\\
7559.13307883523	7.88240948043696e-06	\\
7560.11186079545	6.07444570697254e-06	\\
7561.09064275568	5.11042203347127e-06	\\
7562.06942471591	6.83150026015207e-06	\\
7563.04820667614	7.24520383708401e-06	\\
7564.02698863636	4.99090532982548e-06	\\
7565.00577059659	6.9440652959557e-06	\\
7565.98455255682	6.61559803308883e-06	\\
7566.96333451705	6.33666970423971e-06	\\
7567.94211647727	6.04177033023357e-06	\\
7568.9208984375	8.50892011990256e-06	\\
7569.89968039773	5.15525540125987e-06	\\
7570.87846235795	6.94342812787511e-06	\\
7571.85724431818	6.4859435244625e-06	\\
7572.83602627841	6.55899548809531e-06	\\
7573.81480823864	6.15305436187199e-06	\\
7574.79359019886	5.5930216508371e-06	\\
7575.77237215909	7.14348044405348e-06	\\
7576.75115411932	8.14683870347221e-06	\\
7577.72993607955	9.00765608535061e-06	\\
7578.70871803977	6.55636169526542e-06	\\
7579.6875	8.37095730861036e-06	\\
7580.66628196023	6.07437317742088e-06	\\
7581.64506392045	6.42364036202871e-06	\\
7582.62384588068	6.45583405483114e-06	\\
7583.60262784091	6.66025754355871e-06	\\
7584.58140980114	5.24530373430364e-06	\\
7585.56019176136	3.95886496712862e-06	\\
7586.53897372159	6.5610036900658e-06	\\
7587.51775568182	5.6490303555562e-06	\\
7588.49653764205	6.00288088791016e-06	\\
7589.47531960227	6.6515691651733e-06	\\
7590.4541015625	6.5175831603318e-06	\\
7591.43288352273	6.12115904288309e-06	\\
7592.41166548295	5.44463469359055e-06	\\
7593.39044744318	5.41153178703232e-06	\\
7594.36922940341	7.8915995674237e-06	\\
7595.34801136364	5.39398775075865e-06	\\
7596.32679332386	7.53840389060061e-06	\\
7597.30557528409	5.06367642407786e-06	\\
7598.28435724432	4.8261201968168e-06	\\
7599.26313920455	5.31546760515215e-06	\\
7600.24192116477	6.79368416621141e-06	\\
7601.220703125	4.64583061840163e-06	\\
7602.19948508523	5.45887233496927e-06	\\
7603.17826704545	7.60541317773608e-06	\\
7604.15704900568	6.2274355831115e-06	\\
7605.13583096591	8.23515277431503e-06	\\
7606.11461292614	6.72258226749676e-06	\\
7607.09339488636	7.30422306427926e-06	\\
7608.07217684659	7.38435619633634e-06	\\
7609.05095880682	6.46652506540509e-06	\\
7610.02974076705	5.9675104498806e-06	\\
7611.00852272727	5.7356187350788e-06	\\
7611.9873046875	8.33198436191315e-06	\\
7612.96608664773	7.45609864995948e-06	\\
7613.94486860795	7.05821942569202e-06	\\
7614.92365056818	7.17188444250222e-06	\\
7615.90243252841	5.0996100677006e-06	\\
7616.88121448864	6.46856430947506e-06	\\
7617.85999644886	4.91044104550229e-06	\\
7618.83877840909	7.36900500621398e-06	\\
7619.81756036932	5.05265108674038e-06	\\
7620.79634232955	6.9488645783449e-06	\\
7621.77512428977	5.74252293136901e-06	\\
7622.75390625	6.91272685317164e-06	\\
7623.73268821023	7.71198683370993e-06	\\
7624.71147017045	7.79033011588099e-06	\\
7625.69025213068	5.76616718520973e-06	\\
7626.66903409091	6.1694736039856e-06	\\
7627.64781605114	4.76631908100352e-06	\\
7628.62659801136	5.58302221737194e-06	\\
7629.60537997159	3.80472956478782e-06	\\
7630.58416193182	3.69914643173711e-06	\\
7631.56294389205	4.15152324450572e-06	\\
7632.54172585227	5.11556269144475e-06	\\
7633.5205078125	4.95803533747211e-06	\\
7634.49928977273	5.4506809838553e-06	\\
7635.47807173295	6.08657052044428e-06	\\
7636.45685369318	5.67292967274385e-06	\\
7637.43563565341	5.1731523560586e-06	\\
7638.41441761364	5.30765213824625e-06	\\
7639.39319957386	6.07619156424662e-06	\\
7640.37198153409	6.84785210041862e-06	\\
7641.35076349432	6.67481869808479e-06	\\
7642.32954545455	4.03064129981226e-06	\\
7643.30832741477	4.05226632831659e-06	\\
7644.287109375	5.13418603943752e-06	\\
7645.26589133523	5.55730456004123e-06	\\
7646.24467329545	6.09715095441022e-06	\\
7647.22345525568	3.80392029385765e-06	\\
7648.20223721591	4.70113238325432e-06	\\
7649.18101917614	4.83615306558564e-06	\\
7650.15980113636	2.99693898284437e-06	\\
7651.13858309659	4.50765112630874e-06	\\
7652.11736505682	3.24144558285382e-06	\\
7653.09614701705	5.07905774025917e-06	\\
7654.07492897727	4.67996803957967e-06	\\
7655.0537109375	4.17196879598944e-06	\\
7656.03249289773	3.20205969274503e-06	\\
7657.01127485795	4.78770826667975e-06	\\
7657.99005681818	5.41589633193471e-06	\\
7658.96883877841	5.32419712540573e-06	\\
7659.94762073864	5.3098737646332e-06	\\
7660.92640269886	3.87267396056824e-06	\\
7661.90518465909	4.1424313259752e-06	\\
7662.88396661932	4.25088829865226e-06	\\
7663.86274857955	2.73445794681647e-06	\\
7664.84153053977	5.98784057568508e-06	\\
7665.8203125	4.9761983453304e-06	\\
7666.79909446023	4.49397216454742e-06	\\
7667.77787642045	5.22685951511536e-06	\\
7668.75665838068	4.25540280683599e-06	\\
7669.73544034091	5.5855422738997e-06	\\
7670.71422230114	5.98507731111559e-06	\\
7671.69300426136	3.77123256832892e-06	\\
7672.67178622159	4.88160145527746e-06	\\
7673.65056818182	2.69466771968836e-06	\\
7674.62935014205	3.3618182545351e-06	\\
7675.60813210227	3.50548139226347e-06	\\
7676.5869140625	2.54184632294639e-06	\\
7677.56569602273	5.44103099701593e-06	\\
7678.54447798295	3.57798018224165e-06	\\
7679.52325994318	2.82529679524227e-06	\\
7680.50204190341	3.90266432025923e-06	\\
7681.48082386364	4.31115157039585e-06	\\
7682.45960582386	4.05833843468889e-06	\\
7683.43838778409	5.83074279726098e-06	\\
7684.41716974432	4.00129722290943e-06	\\
7685.39595170455	4.44167094152018e-06	\\
7686.37473366477	5.37650239905651e-06	\\
7687.353515625	5.0630719303608e-06	\\
7688.33229758523	4.48984394711966e-06	\\
7689.31107954545	6.01083193093993e-06	\\
7690.28986150568	2.87685507119979e-06	\\
7691.26864346591	3.64608893103212e-06	\\
7692.24742542614	4.36083323385526e-06	\\
7693.22620738636	4.69468622634202e-06	\\
7694.20498934659	5.15388732503472e-06	\\
7695.18377130682	4.09568265218462e-06	\\
7696.16255326705	3.77733567206431e-06	\\
7697.14133522727	5.00128738962619e-06	\\
7698.1201171875	2.81509050362654e-06	\\
7699.09889914773	4.25170077713861e-06	\\
7700.07768110795	3.05350712531e-06	\\
7701.05646306818	3.77492911368548e-06	\\
7702.03524502841	4.31964903516843e-06	\\
7703.01402698864	4.05556700210488e-06	\\
7703.99280894886	3.88009469483656e-06	\\
7704.97159090909	3.19330294082684e-06	\\
7705.95037286932	4.83260376313358e-06	\\
7706.92915482955	5.03070577715373e-06	\\
7707.90793678977	5.24274127801731e-06	\\
7708.88671875	3.48539254480888e-06	\\
7709.86550071023	4.74758004932762e-06	\\
7710.84428267045	5.34358124404989e-06	\\
7711.82306463068	4.08472328620087e-06	\\
7712.80184659091	2.82649215963583e-06	\\
7713.78062855114	3.79628854242176e-06	\\
7714.75941051136	1.51853460006145e-06	\\
7715.73819247159	3.27290128549053e-06	\\
7716.71697443182	2.15665256120763e-06	\\
7717.69575639205	4.05958205053319e-06	\\
7718.67453835227	4.53711491203318e-06	\\
7719.6533203125	4.34787109090943e-06	\\
7720.63210227273	5.15354065603946e-06	\\
7721.61088423295	4.59339410908266e-06	\\
7722.58966619318	4.91708730741669e-06	\\
7723.56844815341	4.33677194827124e-06	\\
7724.54723011364	2.99318940643374e-06	\\
7725.52601207386	3.88340443762949e-06	\\
7726.50479403409	3.63239139613376e-06	\\
7727.48357599432	2.74080503198868e-06	\\
7728.46235795455	3.07387914434218e-06	\\
7729.44113991477	3.66598413096813e-06	\\
7730.419921875	3.92744278322672e-06	\\
7731.39870383523	4.25500761265776e-06	\\
7732.37748579545	3.97678744388561e-06	\\
7733.35626775568	3.18959238393451e-06	\\
7734.33504971591	3.9242667373465e-06	\\
7735.31383167614	3.75273033614892e-06	\\
7736.29261363636	3.96284838567332e-06	\\
7737.27139559659	4.46860274962234e-06	\\
7738.25017755682	4.64006031443928e-06	\\
7739.22895951705	2.63483440818099e-06	\\
7740.20774147727	3.98018656960505e-06	\\
7741.1865234375	3.30603324204554e-06	\\
7742.16530539773	5.5365907795035e-06	\\
7743.14408735795	4.33297072794537e-06	\\
7744.12286931818	4.06163148425977e-06	\\
7745.10165127841	4.48525539055014e-06	\\
7746.08043323864	4.31213948834941e-06	\\
7747.05921519886	3.49712650221152e-06	\\
7748.03799715909	4.59040401872365e-06	\\
7749.01677911932	3.5775152174894e-06	\\
7749.99556107955	1.66053674438838e-06	\\
7750.97434303977	3.97881840481995e-06	\\
7751.953125	3.26667644933801e-06	\\
7752.93190696023	3.68398819148277e-06	\\
7753.91068892045	4.13026916526733e-06	\\
7754.88947088068	3.33742792487266e-06	\\
7755.86825284091	3.3995764936337e-06	\\
7756.84703480114	4.6887340891579e-06	\\
7757.82581676136	3.17475952910244e-06	\\
7758.80459872159	4.53282848491678e-06	\\
7759.78338068182	3.04868549865611e-06	\\
7760.76216264205	3.03592437789862e-06	\\
7761.74094460227	5.23856152643546e-06	\\
7762.7197265625	2.77853237603825e-06	\\
7763.69850852273	4.64673022156849e-06	\\
7764.67729048295	3.15158582434796e-06	\\
7765.65607244318	2.49708146367973e-06	\\
7766.63485440341	4.69175638892565e-06	\\
7767.61363636364	3.24202242351867e-06	\\
7768.59241832386	2.35565160773926e-06	\\
7769.57120028409	3.86639880665644e-06	\\
7770.54998224432	3.09852371282497e-06	\\
7771.52876420455	4.20816074144361e-06	\\
7772.50754616477	5.66876846482382e-06	\\
7773.486328125	2.57541235848789e-06	\\
7774.46511008523	2.55810095062111e-06	\\
7775.44389204545	4.3761211903018e-06	\\
7776.42267400568	2.05207672703529e-06	\\
7777.40145596591	4.704707175092e-06	\\
7778.38023792614	4.30926646980156e-06	\\
7779.35901988636	3.97366702333337e-06	\\
7780.33780184659	2.94660909968651e-06	\\
7781.31658380682	2.89751528064451e-06	\\
7782.29536576705	4.9901739462921e-06	\\
7783.27414772727	4.66829523753721e-06	\\
7784.2529296875	3.03079563551214e-06	\\
7785.23171164773	4.28567461262167e-06	\\
7786.21049360795	5.3716472938838e-06	\\
7787.18927556818	3.50514447604355e-06	\\
7788.16805752841	4.88978986635758e-06	\\
7789.14683948864	4.46538145758194e-06	\\
7790.12562144886	4.58641737830382e-06	\\
7791.10440340909	3.97704616407036e-06	\\
7792.08318536932	4.39474984134586e-06	\\
7793.06196732955	3.42450377412855e-06	\\
7794.04074928977	2.84333530266306e-06	\\
7795.01953125	2.04255888083659e-06	\\
7795.99831321023	3.83507132676822e-06	\\
7796.97709517045	3.47607992783407e-06	\\
7797.95587713068	3.20044495704438e-06	\\
7798.93465909091	2.76062842082488e-06	\\
7799.91344105114	3.26991083691671e-06	\\
7800.89222301136	4.12353504885894e-06	\\
7801.87100497159	2.44060257318552e-06	\\
7802.84978693182	2.30775808645629e-06	\\
7803.82856889205	5.38849390415163e-06	\\
7804.80735085227	3.47558803962718e-06	\\
7805.7861328125	4.85597855965815e-06	\\
7806.76491477273	3.55896971078205e-06	\\
7807.74369673295	4.12638489328942e-06	\\
7808.72247869318	3.54209152927826e-06	\\
7809.70126065341	4.16553571015625e-06	\\
7810.68004261364	1.65058464929906e-06	\\
7811.65882457386	2.57847746203861e-06	\\
7812.63760653409	1.47401371725073e-06	\\
7813.61638849432	1.94888468668369e-06	\\
7814.59517045455	4.33552349069138e-06	\\
7815.57395241477	4.7487039059019e-06	\\
7816.552734375	4.25831187341305e-06	\\
7817.53151633523	4.39084836512184e-06	\\
7818.51029829545	4.46909175986283e-06	\\
7819.48908025568	4.38928145125663e-06	\\
7820.46786221591	3.19316456872522e-06	\\
7821.44664417614	4.40333987367945e-06	\\
7822.42542613636	2.71500590488951e-06	\\
7823.40420809659	4.3848608440929e-06	\\
7824.38299005682	3.25869922050144e-06	\\
7825.36177201705	3.97969257335113e-06	\\
7826.34055397727	4.06057755858525e-06	\\
7827.3193359375	3.15088025771416e-06	\\
7828.29811789773	3.64373679733774e-06	\\
7829.27689985795	4.13753973906491e-06	\\
7830.25568181818	4.40169990800988e-06	\\
7831.23446377841	2.79012366910416e-06	\\
7832.21324573864	2.43059183554301e-06	\\
7833.19202769886	3.55147586555627e-06	\\
7834.17080965909	3.03649126018546e-06	\\
7835.14959161932	2.96773692696266e-06	\\
7836.12837357955	1.76728015249982e-06	\\
7837.10715553977	3.87787776103139e-06	\\
7838.0859375	2.57161779430506e-06	\\
7839.06471946023	3.20886041772986e-06	\\
7840.04350142045	3.99909973376876e-06	\\
7841.02228338068	2.52995785573012e-06	\\
7842.00106534091	3.79524121566402e-06	\\
7842.97984730114	5.66948885032192e-06	\\
7843.95862926136	1.98384281503681e-06	\\
7844.93741122159	2.66271006972323e-06	\\
7845.91619318182	3.63358934971427e-06	\\
7846.89497514205	2.48010505603734e-06	\\
7847.87375710227	4.34235168322207e-06	\\
7848.8525390625	3.26920459740048e-06	\\
7849.83132102273	4.03259463282086e-06	\\
7850.81010298295	3.06795122984076e-06	\\
7851.78888494318	4.53112600564621e-06	\\
7852.76766690341	1.88775827353219e-06	\\
7853.74644886364	3.47038284254136e-06	\\
7854.72523082386	3.17820741199374e-06	\\
7855.70401278409	3.91175731457813e-06	\\
7856.68279474432	2.54772355397379e-06	\\
7857.66157670455	3.77106035178296e-06	\\
7858.64035866477	2.50443179980614e-06	\\
7859.619140625	2.7194689296941e-06	\\
7860.59792258523	4.22430158423687e-06	\\
7861.57670454545	2.34389859331812e-06	\\
7862.55548650568	5.55745873736866e-06	\\
7863.53426846591	3.23209938552023e-06	\\
7864.51305042614	1.55943218386319e-06	\\
7865.49183238636	4.24321070083151e-06	\\
7866.47061434659	3.90423779981432e-06	\\
7867.44939630682	2.27298779656195e-06	\\
7868.42817826705	9.00894019958579e-07	\\
7869.40696022727	4.42252611606238e-06	\\
7870.3857421875	4.4874793305152e-06	\\
7871.36452414773	4.00318615490693e-06	\\
7872.34330610795	5.36544818040026e-06	\\
7873.32208806818	4.01834261960214e-06	\\
7874.30087002841	3.61501837908496e-06	\\
7875.27965198864	2.95018584560756e-06	\\
7876.25843394886	3.42362734064623e-06	\\
7877.23721590909	2.34861420238292e-06	\\
7878.21599786932	3.33183043643221e-06	\\
7879.19477982955	3.12871200641967e-06	\\
7880.17356178977	2.94201200121667e-06	\\
7881.15234375	3.62472023224045e-06	\\
7882.13112571023	3.566388138502e-06	\\
7883.10990767045	4.44494776767233e-06	\\
7884.08868963068	4.95149069240185e-06	\\
7885.06747159091	3.9253518144002e-06	\\
7886.04625355114	3.61696895272724e-06	\\
7887.02503551136	5.35248012635357e-06	\\
7888.00381747159	3.54825962657295e-06	\\
7888.98259943182	4.73815509876598e-06	\\
7889.96138139205	4.18924818985261e-06	\\
7890.94016335227	3.14185413419907e-06	\\
7891.9189453125	3.94092294464532e-06	\\
7892.89772727273	2.28198997097511e-06	\\
7893.87650923295	3.39751900267477e-06	\\
7894.85529119318	3.83968164365047e-06	\\
7895.83407315341	3.69811038467497e-06	\\
7896.81285511364	2.8249270826369e-06	\\
7897.79163707386	3.77907234699388e-06	\\
7898.77041903409	4.21483657116399e-06	\\
7899.74920099432	3.83762576344222e-06	\\
7900.72798295455	2.41304726947166e-06	\\
7901.70676491477	3.96810621142491e-06	\\
7902.685546875	3.93473879321527e-06	\\
7903.66432883523	4.03586311858842e-06	\\
7904.64311079545	3.37383859773537e-06	\\
7905.62189275568	4.72725938749642e-06	\\
7906.60067471591	3.31719430583596e-06	\\
7907.57945667614	4.06706573292002e-06	\\
7908.55823863636	2.92868815753547e-06	\\
7909.53702059659	1.99727705904525e-06	\\
7910.51580255682	3.72211804153845e-06	\\
7911.49458451705	4.0553884477117e-06	\\
7912.47336647727	3.93389779952201e-06	\\
7913.4521484375	3.83611226844203e-06	\\
7914.43093039773	4.27301659472962e-06	\\
7915.40971235795	3.63871344734474e-06	\\
7916.38849431818	5.45164600761575e-06	\\
7917.36727627841	3.51464189631606e-06	\\
7918.34605823864	4.27180015172469e-06	\\
7919.32484019886	3.29653714683274e-06	\\
7920.30362215909	4.7616812062691e-06	\\
7921.28240411932	3.37682674560866e-06	\\
7922.26118607955	2.41866584735107e-06	\\
7923.23996803977	4.16593617190586e-06	\\
7924.21875	4.14547556593873e-06	\\
7925.19753196023	4.23736440734904e-06	\\
7926.17631392045	2.76422947892919e-06	\\
7927.15509588068	5.27833696146433e-06	\\
7928.13387784091	2.51040065857917e-06	\\
7929.11265980114	3.01735762917463e-06	\\
7930.09144176136	1.38587022973972e-06	\\
7931.07022372159	3.00633533766275e-06	\\
7932.04900568182	3.72649975360295e-06	\\
7933.02778764205	2.72870457385e-06	\\
7934.00656960227	3.29858252088545e-06	\\
7934.9853515625	3.41590844738599e-06	\\
7935.96413352273	3.62695131655287e-06	\\
7936.94291548295	4.89955633551966e-06	\\
7937.92169744318	4.18740120321855e-06	\\
7938.90047940341	3.60502321980776e-06	\\
7939.87926136364	3.26767423527912e-06	\\
7940.85804332386	3.78144869436532e-06	\\
7941.83682528409	2.89948089961253e-06	\\
7942.81560724432	2.18615769928759e-06	\\
7943.79438920455	4.00595962304644e-06	\\
7944.77317116477	3.37564478568882e-06	\\
7945.751953125	3.76530445306376e-06	\\
7946.73073508523	4.95699119846174e-06	\\
7947.70951704545	4.71354348507402e-06	\\
7948.68829900568	2.18968994638805e-06	\\
7949.66708096591	2.81433903006974e-06	\\
7950.64586292614	3.41236486741708e-06	\\
7951.62464488636	3.53640550807829e-06	\\
7952.60342684659	4.83030798136125e-06	\\
7953.58220880682	3.94092482755572e-06	\\
7954.56099076705	3.1635335879019e-06	\\
7955.53977272727	3.20390917113015e-06	\\
7956.5185546875	3.37339469444079e-06	\\
7957.49733664773	2.99970901363757e-06	\\
7958.47611860795	4.04091269586761e-06	\\
7959.45490056818	3.65086239113443e-06	\\
7960.43368252841	3.26803614189969e-06	\\
7961.41246448864	2.6538860348127e-06	\\
7962.39124644886	4.3376246187587e-06	\\
7963.37002840909	3.35785389494215e-06	\\
7964.34881036932	4.33154108363576e-06	\\
7965.32759232955	3.92954877267243e-06	\\
7966.30637428977	2.78722333475591e-06	\\
7967.28515625	3.06001826300116e-06	\\
7968.26393821023	4.5888045523696e-06	\\
7969.24272017045	3.67250611898623e-06	\\
7970.22150213068	3.23949498666755e-06	\\
7971.20028409091	3.67103100613394e-06	\\
7972.17906605114	2.44735897684603e-06	\\
7973.15784801136	3.88113569616388e-06	\\
7974.13662997159	4.05320184297331e-06	\\
7975.11541193182	2.46871913365275e-06	\\
7976.09419389205	4.70363848363661e-06	\\
7977.07297585227	3.91230335544477e-06	\\
7978.0517578125	3.86293712220544e-06	\\
7979.03053977273	3.89207343305756e-06	\\
7980.00932173295	3.85876432755136e-06	\\
7980.98810369318	3.19380000916739e-06	\\
7981.96688565341	1.27832928492625e-06	\\
7982.94566761364	3.97143616485623e-06	\\
7983.92444957386	3.72726887055853e-06	\\
7984.90323153409	3.09931560236571e-06	\\
7985.88201349432	4.35196021693426e-06	\\
7986.86079545455	4.47812934144121e-06	\\
7987.83957741477	3.08898547855069e-06	\\
7988.818359375	3.92010950105877e-06	\\
7989.79714133523	4.53588629763509e-06	\\
7990.77592329545	5.58981962938395e-06	\\
7991.75470525568	3.41307558511231e-06	\\
7992.73348721591	3.62876214525578e-06	\\
7993.71226917614	3.39713501680835e-06	\\
7994.69105113636	4.09350292985641e-06	\\
7995.66983309659	3.20527517584608e-06	\\
7996.64861505682	3.35552665201132e-06	\\
7997.62739701705	3.56780564519416e-06	\\
7998.60617897727	5.71794625996286e-06	\\
7999.5849609375	3.03887993544027e-06	\\
8000.56374289773	4.44251039839488e-06	\\
8001.54252485795	4.63126869272127e-06	\\
8002.52130681818	3.59027036660695e-06	\\
8003.50008877841	3.75937494529901e-06	\\
8004.47887073864	1.74769639623138e-06	\\
8005.45765269886	4.50161520981577e-06	\\
8006.43643465909	3.34249834309951e-06	\\
8007.41521661932	3.1044242436884e-06	\\
8008.39399857955	4.19739848384101e-06	\\
8009.37278053977	3.54246017830848e-06	\\
8010.3515625	5.37030305355594e-06	\\
8011.33034446023	3.8244526644694e-06	\\
8012.30912642045	3.49329313027521e-06	\\
8013.28790838068	4.44543444599418e-06	\\
8014.26669034091	3.06594183777083e-06	\\
8015.24547230114	4.24287552883818e-06	\\
8016.22425426136	4.13264651085915e-06	\\
8017.20303622159	4.39326384960259e-06	\\
8018.18181818182	3.16814853760717e-06	\\
8019.16060014205	4.67316395572673e-06	\\
8020.13938210227	4.70172520633138e-06	\\
8021.1181640625	5.02569317775881e-06	\\
8022.09694602273	3.89880488965504e-06	\\
8023.07572798295	2.41129207170674e-06	\\
8024.05450994318	5.15741633155279e-06	\\
8025.03329190341	4.02400776791614e-06	\\
8026.01207386364	3.50196217473219e-06	\\
8026.99085582386	3.29594528763224e-06	\\
8027.96963778409	4.80884208356422e-06	\\
8028.94841974432	4.41292812490894e-06	\\
8029.92720170455	4.4822252692288e-06	\\
8030.90598366477	3.32125475017796e-06	\\
8031.884765625	3.53761179479627e-06	\\
8032.86354758523	4.59809768679915e-06	\\
8033.84232954545	4.53495289147857e-06	\\
8034.82111150568	5.44701496117757e-06	\\
8035.79989346591	4.80026790602997e-06	\\
8036.77867542614	3.2969451898938e-06	\\
8037.75745738636	2.81993339026157e-06	\\
8038.73623934659	5.18356720880103e-06	\\
8039.71502130682	5.35510970919901e-06	\\
8040.69380326705	3.70419671155206e-06	\\
8041.67258522727	4.01214603475224e-06	\\
8042.6513671875	3.58985093067775e-06	\\
8043.63014914773	3.62324525551733e-06	\\
8044.60893110795	5.26557914899808e-06	\\
8045.58771306818	3.94803492170395e-06	\\
8046.56649502841	3.44762873509083e-06	\\
8047.54527698864	3.51191618819054e-06	\\
8048.52405894886	3.53239449395818e-06	\\
8049.50284090909	4.6727719556017e-06	\\
8050.48162286932	2.61150719411345e-06	\\
8051.46040482955	3.41257585029993e-06	\\
8052.43918678977	2.90056641690687e-06	\\
8053.41796875	3.6793893003978e-06	\\
8054.39675071023	2.63740544850867e-06	\\
8055.37553267045	3.71177131681179e-06	\\
8056.35431463068	4.55849378823948e-06	\\
8057.33309659091	4.03855999030906e-06	\\
8058.31187855114	3.53670162524078e-06	\\
8059.29066051136	3.92782771635521e-06	\\
8060.26944247159	2.83166794789411e-06	\\
8061.24822443182	4.22892074948213e-06	\\
8062.22700639205	3.25866309791357e-06	\\
8063.20578835227	3.81050054350117e-06	\\
8064.1845703125	4.25507222279825e-06	\\
8065.16335227273	4.76322497739292e-06	\\
8066.14213423295	3.39706995738639e-06	\\
8067.12091619318	3.42770585157396e-06	\\
8068.09969815341	3.34163777212918e-06	\\
8069.07848011364	3.50905985151349e-06	\\
8070.05726207386	3.071418460503e-06	\\
8071.03604403409	3.65083155313178e-06	\\
8072.01482599432	4.17743552745482e-06	\\
8072.99360795455	3.47072975744409e-06	\\
8073.97238991477	3.77376384692493e-06	\\
8074.951171875	4.81002445435867e-06	\\
8075.92995383523	4.38219509857772e-06	\\
8076.90873579545	4.82984135806849e-06	\\
8077.88751775568	3.58235655856222e-06	\\
8078.86629971591	4.31193027327848e-06	\\
8079.84508167614	4.97408179895941e-06	\\
8080.82386363636	3.00003686822757e-06	\\
8081.80264559659	2.91436285600819e-06	\\
8082.78142755682	4.64748916137918e-06	\\
8083.76020951705	3.39665258603124e-06	\\
8084.73899147727	2.55675481308884e-06	\\
8085.7177734375	2.64568510478175e-06	\\
8086.69655539773	5.41117945616053e-06	\\
8087.67533735795	5.22970750738002e-06	\\
8088.65411931818	2.28499529905001e-06	\\
8089.63290127841	4.29895692180268e-06	\\
8090.61168323864	4.62710852818874e-06	\\
8091.59046519886	4.87929773624579e-06	\\
8092.56924715909	3.82842263088139e-06	\\
8093.54802911932	4.93107066555452e-06	\\
8094.52681107955	5.27771552403493e-06	\\
8095.50559303977	4.46119896526228e-06	\\
8096.484375	2.92466935399949e-06	\\
8097.46315696023	4.57513976720849e-06	\\
8098.44193892045	4.47600796213504e-06	\\
8099.42072088068	4.27990075033163e-06	\\
8100.39950284091	3.28485953670172e-06	\\
8101.37828480114	2.61889989301495e-06	\\
8102.35706676136	4.18548100260882e-06	\\
8103.33584872159	2.9867356783272e-06	\\
8104.31463068182	4.15401309910184e-06	\\
8105.29341264205	2.71320253366768e-06	\\
8106.27219460227	2.91430409575783e-06	\\
8107.2509765625	4.04876050923706e-06	\\
8108.22975852273	4.43012749996012e-06	\\
8109.20854048295	5.38648781439355e-06	\\
8110.18732244318	4.04565055248211e-06	\\
8111.16610440341	4.67572299221883e-06	\\
8112.14488636364	2.56379885768763e-06	\\
8113.12366832386	4.9675432218807e-06	\\
8114.10245028409	4.50869542304615e-06	\\
8115.08123224432	4.22008727048276e-06	\\
8116.06001420455	2.53151967918453e-06	\\
8117.03879616477	4.44421800335101e-06	\\
8118.017578125	3.94635017118025e-06	\\
8118.99636008523	4.82550188661479e-06	\\
8119.97514204545	2.94881444230097e-06	\\
8120.95392400568	4.57766183498732e-06	\\
8121.93270596591	3.71214497289621e-06	\\
8122.91148792614	3.73688423536435e-06	\\
8123.89026988636	4.14386142629164e-06	\\
8124.86905184659	3.74426201857356e-06	\\
8125.84783380682	3.10815758624244e-06	\\
8126.82661576705	4.04008089774567e-06	\\
8127.80539772727	4.37621005517839e-06	\\
8128.7841796875	3.3530402281994e-06	\\
8129.76296164773	3.4639693940226e-06	\\
8130.74174360795	5.2059718577381e-06	\\
8131.72052556818	3.52565095390847e-06	\\
8132.69930752841	4.44036403781381e-06	\\
8133.67808948864	2.81432680046688e-06	\\
8134.65687144886	4.61965636846361e-06	\\
8135.63565340909	4.55466958172613e-06	\\
8136.61443536932	4.26138898630806e-06	\\
8137.59321732955	4.78575146817209e-06	\\
8138.57199928977	4.14728564302583e-06	\\
8139.55078125	5.05986006816006e-06	\\
8140.52956321023	3.85275528416859e-06	\\
8141.50834517045	4.65097412653512e-06	\\
8142.48712713068	4.57264348213947e-06	\\
8143.46590909091	4.51801453935094e-06	\\
8144.44469105114	4.59407688841778e-06	\\
8145.42347301136	3.94507851407682e-06	\\
8146.40225497159	3.12024642836601e-06	\\
8147.38103693182	6.04786290277335e-06	\\
8148.35981889205	5.26880823836059e-06	\\
8149.33860085227	3.55459071708549e-06	\\
8150.3173828125	2.54171135219467e-06	\\
8151.29616477273	4.01327441650646e-06	\\
8152.27494673295	4.39395228551738e-06	\\
8153.25372869318	1.97643582699815e-06	\\
8154.23251065341	4.70873305484945e-06	\\
8155.21129261364	3.70950470298931e-06	\\
8156.19007457386	3.98881072458805e-06	\\
8157.16885653409	3.73231026778724e-06	\\
8158.14763849432	5.51203421847094e-06	\\
8159.12642045455	4.01300942808633e-06	\\
8160.10520241477	3.80856062856947e-06	\\
8161.083984375	3.8459792030523e-06	\\
8162.06276633523	2.69744159948873e-06	\\
8163.04154829545	5.32719960241873e-06	\\
8164.02033025568	3.29675545631273e-06	\\
8164.99911221591	4.20791726152974e-06	\\
8165.97789417614	3.37941557207839e-06	\\
8166.95667613636	4.07913248593109e-06	\\
8167.93545809659	4.59542061885925e-06	\\
8168.91424005682	4.06477924994011e-06	\\
8169.89302201705	3.28323957723685e-06	\\
8170.87180397727	4.51076970493982e-06	\\
8171.8505859375	5.01827096980088e-06	\\
8172.82936789773	4.89945787126506e-06	\\
8173.80814985795	5.45094553671614e-06	\\
8174.78693181818	4.12626364085101e-06	\\
8175.76571377841	3.269150967487e-06	\\
8176.74449573864	2.95694969312595e-06	\\
8177.72327769886	4.03730009258854e-06	\\
8178.70205965909	5.12042797479466e-06	\\
8179.68084161932	4.88327441613139e-06	\\
8180.65962357955	4.53823748333751e-06	\\
8181.63840553977	3.78680419205978e-06	\\
8182.6171875	3.00328751575603e-06	\\
8183.59596946023	4.32508632049794e-06	\\
8184.57475142045	3.88649662950786e-06	\\
8185.55353338068	4.99625677152953e-06	\\
8186.53231534091	4.06058192620992e-06	\\
8187.51109730114	4.35699100094392e-06	\\
8188.48987926136	3.49868437169636e-06	\\
8189.46866122159	4.08565696884722e-06	\\
8190.44744318182	4.03982887391404e-06	\\
8191.42622514205	4.66408325523609e-06	\\
8192.40500710227	4.94553760632364e-06	\\
8193.3837890625	5.3359294832428e-06	\\
8194.36257102273	4.33291881416986e-06	\\
8195.34135298295	2.73684377101808e-06	\\
8196.32013494318	6.13629617764857e-06	\\
8197.29891690341	4.35521568657518e-06	\\
8198.27769886364	3.19109552184629e-06	\\
8199.25648082386	4.6089105765229e-06	\\
8200.23526278409	5.98977565550527e-06	\\
8201.21404474432	5.01440053389218e-06	\\
8202.19282670454	4.04840255411202e-06	\\
8203.17160866477	4.45419803507784e-06	\\
8204.150390625	6.10231312895081e-06	\\
8205.12917258523	5.11358968660484e-06	\\
8206.10795454545	4.59957614470925e-06	\\
8207.08673650568	5.69361923199232e-06	\\
8208.06551846591	6.70005428367048e-06	\\
8209.04430042614	5.04146813892379e-06	\\
8210.02308238636	4.98019878131205e-06	\\
8211.00186434659	2.99504676112736e-06	\\
8211.98064630682	5.45085840661458e-06	\\
8212.95942826704	5.99201036480033e-06	\\
8213.93821022727	4.1845524013746e-06	\\
8214.9169921875	3.46544909457475e-06	\\
8215.89577414773	4.45138910630929e-06	\\
8216.87455610795	3.33902718627006e-06	\\
8217.85333806818	3.55680462968343e-06	\\
8218.83212002841	4.29637060311739e-06	\\
8219.81090198864	3.3327568149657e-06	\\
8220.78968394886	5.14859154588613e-06	\\
8221.76846590909	4.32764519696913e-06	\\
8222.74724786932	3.69351929907987e-06	\\
8223.72602982954	4.00405161102705e-06	\\
8224.70481178977	5.76568478133273e-06	\\
8225.68359375	3.70672697863155e-06	\\
8226.66237571023	5.19723672031051e-06	\\
8227.64115767045	5.57860735111391e-06	\\
8228.61993963068	3.93086044303536e-06	\\
8229.59872159091	5.26480656012172e-06	\\
8230.57750355114	4.52021036362256e-06	\\
8231.55628551136	3.34633319030818e-06	\\
8232.53506747159	4.72185872235623e-06	\\
8233.51384943182	5.89273102922816e-06	\\
8234.49263139204	5.21092310788401e-06	\\
8235.47141335227	2.98634926006608e-06	\\
8236.4501953125	6.46028089540198e-06	\\
8237.42897727273	4.56416517766745e-06	\\
8238.40775923295	3.7112764277803e-06	\\
8239.38654119318	2.14140547906102e-06	\\
8240.36532315341	4.38164213460486e-06	\\
8241.34410511364	4.28846450191472e-06	\\
8242.32288707386	5.90763001564869e-06	\\
8243.30166903409	4.91915032718663e-06	\\
8244.28045099432	4.70287531153924e-06	\\
8245.25923295454	4.10443658521602e-06	\\
8246.23801491477	4.83561883413703e-06	\\
8247.216796875	3.21154614956079e-06	\\
8248.19557883523	5.38936343918496e-06	\\
8249.17436079545	4.97647802892557e-06	\\
8250.15314275568	3.29535735293789e-06	\\
8251.13192471591	4.92860600499585e-06	\\
8252.11070667614	4.6295571611347e-06	\\
8253.08948863636	4.4602743926845e-06	\\
8254.06827059659	5.158252296822e-06	\\
8255.04705255682	4.19587377409342e-06	\\
8256.02583451704	4.59054875870544e-06	\\
8257.00461647727	4.32506128621285e-06	\\
8257.9833984375	5.40524445701223e-06	\\
8258.96218039773	4.3535342147413e-06	\\
8259.94096235795	4.50595227947818e-06	\\
8260.91974431818	2.4602783801082e-06	\\
8261.89852627841	3.90800755115448e-06	\\
8262.87730823864	3.95801028525885e-06	\\
8263.85609019886	6.21949132089841e-06	\\
8264.83487215909	4.8770207509354e-06	\\
8265.81365411932	3.29254783256433e-06	\\
8266.79243607954	4.88995300750996e-06	\\
8267.77121803977	4.92935428453507e-06	\\
8268.75	5.51519542928774e-06	\\
8269.72878196023	4.77261940895389e-06	\\
8270.70756392045	4.93983489051059e-06	\\
8271.68634588068	5.3416852431813e-06	\\
8272.66512784091	4.80510955971343e-06	\\
8273.64390980114	5.23505391155859e-06	\\
8274.62269176136	5.49313766926662e-06	\\
8275.60147372159	4.26034291787368e-06	\\
8276.58025568182	4.65399796362037e-06	\\
8277.55903764204	6.25596840006164e-06	\\
8278.53781960227	4.96982534903962e-06	\\
8279.5166015625	4.68001013128388e-06	\\
8280.49538352273	5.68938713163483e-06	\\
8281.47416548295	4.26307984295083e-06	\\
8282.45294744318	4.01733020602376e-06	\\
8283.43172940341	4.86731017429731e-06	\\
8284.41051136364	6.63630031639827e-06	\\
8285.38929332386	4.86950099006534e-06	\\
8286.36807528409	4.33008847649265e-06	\\
8287.34685724432	4.28721274995559e-06	\\
8288.32563920454	4.16171000988208e-06	\\
8289.30442116477	6.49325180859172e-06	\\
8290.283203125	4.13983357946788e-06	\\
8291.26198508523	5.54182080136792e-06	\\
8292.24076704545	4.0055969806618e-06	\\
8293.21954900568	5.02749863735475e-06	\\
8294.19833096591	5.46342489032508e-06	\\
8295.17711292614	5.12871944020726e-06	\\
8296.15589488636	5.5647556073654e-06	\\
8297.13467684659	4.56422395734335e-06	\\
8298.11345880682	5.58507354374956e-06	\\
8299.09224076704	4.04851669958278e-06	\\
8300.07102272727	5.35599614983905e-06	\\
8301.0498046875	4.18410085934921e-06	\\
8302.02858664773	5.27382520115631e-06	\\
8303.00736860795	4.13939324512993e-06	\\
8303.98615056818	2.92846606502063e-06	\\
8304.96493252841	6.50790491267692e-06	\\
8305.94371448864	5.24157211204743e-06	\\
8306.92249644886	5.57662320196367e-06	\\
8307.90127840909	4.28375400414154e-06	\\
8308.88006036932	5.3557922393624e-06	\\
8309.85884232954	5.11367104458784e-06	\\
8310.83762428977	4.47406729560157e-06	\\
8311.81640625	3.83028373360051e-06	\\
8312.79518821023	5.54496810732774e-06	\\
8313.77397017045	6.1417871215923e-06	\\
8314.75275213068	5.55821962915119e-06	\\
8315.73153409091	3.6570258714325e-06	\\
8316.71031605114	4.20472996398816e-06	\\
8317.68909801136	5.09436088189645e-06	\\
8318.66787997159	4.53525637825842e-06	\\
8319.64666193182	5.26818443569006e-06	\\
8320.62544389204	4.35345112409146e-06	\\
8321.60422585227	4.00520464880153e-06	\\
8322.5830078125	6.27523978074576e-06	\\
8323.56178977273	4.9059195804434e-06	\\
8324.54057173295	5.53391847926754e-06	\\
8325.51935369318	4.33551493676886e-06	\\
8326.49813565341	6.12574848069471e-06	\\
8327.47691761364	5.33946987045495e-06	\\
8328.45569957386	5.26729736189236e-06	\\
8329.43448153409	4.98546545558552e-06	\\
8330.41326349432	5.95869782958224e-06	\\
8331.39204545454	4.61904395668464e-06	\\
8332.37082741477	4.0216033390027e-06	\\
8333.349609375	5.50720817856813e-06	\\
8334.32839133523	4.35298650686651e-06	\\
8335.30717329545	5.72205368613416e-06	\\
8336.28595525568	4.52116865130172e-06	\\
8337.26473721591	4.43734928465653e-06	\\
8338.24351917614	4.98562533590352e-06	\\
8339.22230113636	5.64557343135846e-06	\\
8340.20108309659	4.59440859619521e-06	\\
8341.17986505682	5.46761488494529e-06	\\
8342.15864701704	4.30459062710084e-06	\\
8343.13742897727	5.25507489930529e-06	\\
8344.1162109375	3.89782641704067e-06	\\
8345.09499289773	3.55066011886187e-06	\\
8346.07377485795	5.37435781675378e-06	\\
8347.05255681818	3.88022962111197e-06	\\
8348.03133877841	4.26038397445366e-06	\\
8349.01012073864	4.46022500354725e-06	\\
8349.98890269886	3.73767015398263e-06	\\
8350.96768465909	4.35155129140728e-06	\\
8351.94646661932	4.46183021431782e-06	\\
8352.92524857954	5.66052760144552e-06	\\
8353.90403053977	4.56620214707051e-06	\\
8354.8828125	3.48533376159153e-06	\\
8355.86159446023	3.19743776834304e-06	\\
8356.84037642045	4.61501311123946e-06	\\
8357.81915838068	6.11412414564932e-06	\\
8358.79794034091	4.36594062975103e-06	\\
8359.77672230114	5.68446246553269e-06	\\
8360.75550426136	6.1970334690942e-06	\\
8361.73428622159	5.31178457137664e-06	\\
8362.71306818182	5.03947054921853e-06	\\
8363.69185014204	5.24274021455948e-06	\\
8364.67063210227	5.78424164496744e-06	\\
8365.6494140625	4.15760147170708e-06	\\
8366.62819602273	6.20615523627562e-06	\\
8367.60697798295	5.13962445133426e-06	\\
8368.58575994318	5.40014745575697e-06	\\
8369.56454190341	4.38514889941662e-06	\\
8370.54332386364	4.61488470890837e-06	\\
8371.52210582386	5.79723151435165e-06	\\
8372.50088778409	4.00578228531342e-06	\\
8373.47966974432	5.1298873728259e-06	\\
8374.45845170454	6.87819422460147e-06	\\
8375.43723366477	3.70759440631366e-06	\\
8376.416015625	5.94519429946438e-06	\\
8377.39479758523	4.52038707218062e-06	\\
8378.37357954545	5.26833895416819e-06	\\
8379.35236150568	5.83837213364287e-06	\\
8380.33114346591	4.79318497399853e-06	\\
8381.30992542614	5.71881460259013e-06	\\
8382.28870738636	4.12070132886172e-06	\\
8383.26748934659	5.73243220799325e-06	\\
8384.24627130682	5.78460792492836e-06	\\
8385.22505326704	4.73696026261489e-06	\\
8386.20383522727	5.46028219866802e-06	\\
8387.1826171875	3.91021019125885e-06	\\
8388.16139914773	4.91053117917713e-06	\\
8389.14018110795	5.07864857076395e-06	\\
8390.11896306818	7.53237058579631e-06	\\
8391.09774502841	4.88339179598991e-06	\\
8392.07652698864	5.86613186670447e-06	\\
8393.05530894886	4.45365222434051e-06	\\
8394.03409090909	3.47540676875468e-06	\\
8395.01287286932	4.33363754865213e-06	\\
8395.99165482954	4.33805545511495e-06	\\
8396.97043678977	4.08434408344563e-06	\\
8397.94921875	4.58981367922114e-06	\\
8398.92800071023	5.33492881963164e-06	\\
8399.90678267045	4.66785733674442e-06	\\
8400.88556463068	5.14285072976974e-06	\\
8401.86434659091	5.09419193251939e-06	\\
8402.84312855114	5.28746363249511e-06	\\
8403.82191051136	4.18666441694668e-06	\\
8404.80069247159	4.91611658004466e-06	\\
8405.77947443182	4.67812360587818e-06	\\
8406.75825639204	4.63854987246667e-06	\\
8407.73703835227	6.3892864451178e-06	\\
8408.7158203125	5.32274357518576e-06	\\
8409.69460227273	6.04661905319171e-06	\\
8410.67338423295	6.030402153509e-06	\\
8411.65216619318	5.2551306611612e-06	\\
8412.63094815341	4.33512444730005e-06	\\
8413.60973011364	5.9772980213353e-06	\\
8414.58851207386	4.9461742547238e-06	\\
8415.56729403409	5.55267242011482e-06	\\
8416.54607599432	5.2834346102731e-06	\\
8417.52485795454	5.2653027638298e-06	\\
8418.50363991477	5.4646120591351e-06	\\
8419.482421875	5.72984848851098e-06	\\
8420.46120383523	6.85422033884648e-06	\\
8421.43998579545	5.75755720060261e-06	\\
8422.41876775568	6.51833196973623e-06	\\
8423.39754971591	3.13821344381376e-06	\\
8424.37633167614	6.4859597398862e-06	\\
8425.35511363636	4.45845557464349e-06	\\
8426.33389559659	5.05382316314769e-06	\\
8427.31267755682	4.99814536682856e-06	\\
8428.29145951704	6.6470808609104e-06	\\
8429.27024147727	4.03325065088156e-06	\\
8430.2490234375	4.94806819871318e-06	\\
8431.22780539773	6.1749944141467e-06	\\
8432.20658735795	5.75791001669641e-06	\\
8433.18536931818	5.15037352801527e-06	\\
8434.16415127841	4.86820215407126e-06	\\
8435.14293323864	4.2218070794778e-06	\\
8436.12171519886	4.88343879996402e-06	\\
8437.10049715909	4.37655659580758e-06	\\
8438.07927911932	4.52189756284648e-06	\\
8439.05806107954	3.66289139456254e-06	\\
8440.03684303977	5.13575659112296e-06	\\
8441.015625	5.13540128907095e-06	\\
8441.99440696023	4.80491035384268e-06	\\
8442.97318892045	4.94307606681226e-06	\\
8443.95197088068	6.31473642736915e-06	\\
8444.93075284091	4.49033053295165e-06	\\
8445.90953480114	4.94021832985131e-06	\\
8446.88831676136	4.74254509041988e-06	\\
8447.86709872159	5.38475718965571e-06	\\
8448.84588068182	4.85868759108935e-06	\\
8449.82466264204	5.47897108607348e-06	\\
8450.80344460227	5.86273379173293e-06	\\
8451.7822265625	5.15870430615918e-06	\\
8452.76100852273	5.72475148838072e-06	\\
8453.73979048295	6.03143819716291e-06	\\
8454.71857244318	4.69720578795749e-06	\\
8455.69735440341	6.17761146361442e-06	\\
8456.67613636364	4.0267260533187e-06	\\
8457.65491832386	6.23207960933716e-06	\\
8458.63370028409	5.43907648430145e-06	\\
8459.61248224432	3.27060122007752e-06	\\
8460.59126420454	5.62355730161947e-06	\\
8461.57004616477	4.94462979698626e-06	\\
8462.548828125	5.92160993190788e-06	\\
8463.52761008523	4.81130612171139e-06	\\
8464.50639204545	5.48116167883599e-06	\\
8465.48517400568	3.83255702132187e-06	\\
8466.46395596591	4.21039815951899e-06	\\
8467.44273792614	6.67400669087726e-06	\\
8468.42151988636	5.92643879546657e-06	\\
8469.40030184659	4.60300607914995e-06	\\
8470.37908380682	4.4546652205221e-06	\\
8471.35786576704	5.23138938741361e-06	\\
8472.33664772727	5.26725062995097e-06	\\
8473.3154296875	5.01382667907784e-06	\\
8474.29421164773	3.69109942985802e-06	\\
8475.27299360795	3.9832490143528e-06	\\
8476.25177556818	4.53739716849803e-06	\\
8477.23055752841	4.35654571773099e-06	\\
8478.20933948864	4.67724621118625e-06	\\
8479.18812144886	5.59605733576802e-06	\\
8480.16690340909	5.44311682857402e-06	\\
8481.14568536932	5.36126635102337e-06	\\
8482.12446732954	4.86791357236153e-06	\\
8483.10324928977	5.41806228739826e-06	\\
8484.08203125	4.2106581482528e-06	\\
8485.06081321023	4.69880685167271e-06	\\
8486.03959517045	4.51109492886919e-06	\\
8487.01837713068	5.67011210598624e-06	\\
8487.99715909091	5.12912979200791e-06	\\
8488.97594105114	3.57199920826842e-06	\\
8489.95472301136	4.41115450112621e-06	\\
8490.93350497159	4.08569854404416e-06	\\
8491.91228693182	5.05143410398106e-06	\\
8492.89106889204	5.09449487691581e-06	\\
8493.86985085227	3.87718844942375e-06	\\
8494.8486328125	4.35606719876596e-06	\\
8495.82741477273	6.62717752108849e-06	\\
8496.80619673295	5.84873866128141e-06	\\
8497.78497869318	4.23835230120636e-06	\\
8498.76376065341	4.97365808941426e-06	\\
8499.74254261364	4.31346024328387e-06	\\
8500.72132457386	7.64204559214928e-06	\\
8501.70010653409	5.06159491131062e-06	\\
8502.67888849432	5.46868586779261e-06	\\
8503.65767045454	6.39419039845813e-06	\\
8504.63645241477	4.77548130052502e-06	\\
8505.615234375	4.47839709543295e-06	\\
8506.59401633523	4.12057559373661e-06	\\
8507.57279829545	4.56572642762801e-06	\\
8508.55158025568	5.93925687695452e-06	\\
8509.53036221591	3.78750312889126e-06	\\
8510.50914417614	4.41011312011919e-06	\\
8511.48792613636	4.8893989898377e-06	\\
8512.46670809659	4.77464948061342e-06	\\
8513.44549005682	4.56160219680114e-06	\\
8514.42427201704	4.70835946998762e-06	\\
8515.40305397727	7.38873352518045e-06	\\
8516.3818359375	7.21343480900687e-06	\\
8517.36061789773	5.20360685916014e-06	\\
8518.33939985795	4.14124237680001e-06	\\
8519.31818181818	4.87217498496042e-06	\\
8520.29696377841	5.51325484198603e-06	\\
8521.27574573864	3.81021230319034e-06	\\
8522.25452769886	4.27881578004933e-06	\\
8523.23330965909	6.16772629234693e-06	\\
8524.21209161932	5.00583216029358e-06	\\
8525.19087357954	6.19409879981441e-06	\\
8526.16965553977	5.38936909842444e-06	\\
8527.1484375	5.29830962126851e-06	\\
8528.12721946023	6.08955645717257e-06	\\
8529.10600142045	5.62273513692873e-06	\\
8530.08478338068	4.59577165211168e-06	\\
8531.06356534091	4.15094176514102e-06	\\
8532.04234730114	4.37524429467817e-06	\\
8533.02112926136	4.60152396567058e-06	\\
8533.99991122159	5.15968349372955e-06	\\
8534.97869318182	5.47717440409322e-06	\\
8535.95747514204	6.3233270692388e-06	\\
8536.93625710227	4.99426533117099e-06	\\
8537.9150390625	6.52291502940891e-06	\\
8538.89382102273	6.38324942790593e-06	\\
8539.87260298295	5.81453134303812e-06	\\
8540.85138494318	5.53261517405009e-06	\\
8541.83016690341	5.41408972568649e-06	\\
8542.80894886364	4.23052039442061e-06	\\
8543.78773082386	4.82943343139512e-06	\\
8544.76651278409	5.30093386275348e-06	\\
8545.74529474432	4.67866662841051e-06	\\
8546.72407670454	5.57520764868439e-06	\\
8547.70285866477	5.41549048597735e-06	\\
8548.681640625	5.6694372269357e-06	\\
8549.66042258523	5.54322918208629e-06	\\
8550.63920454545	4.50283360994326e-06	\\
};
\addplot [color=blue,solid,forget plot]
  table[row sep=crcr]{
8550.63920454545	4.50283360994326e-06	\\
8551.61798650568	5.28507864329868e-06	\\
8552.59676846591	6.05568446495179e-06	\\
8553.57555042614	5.46233318689677e-06	\\
8554.55433238636	4.30208102552254e-06	\\
8555.53311434659	4.0471471210066e-06	\\
8556.51189630682	6.79595913016726e-06	\\
8557.49067826704	4.02110630329736e-06	\\
8558.46946022727	5.0109165784718e-06	\\
8559.4482421875	3.76532450912798e-06	\\
8560.42702414773	5.8685553837714e-06	\\
8561.40580610795	5.66537440083665e-06	\\
8562.38458806818	3.11722452615378e-06	\\
8563.36337002841	5.2345505949587e-06	\\
8564.34215198864	7.56619085133061e-06	\\
8565.32093394886	4.84703314234015e-06	\\
8566.29971590909	4.94956182766422e-06	\\
8567.27849786932	4.0099172510908e-06	\\
8568.25727982954	4.47572933769975e-06	\\
8569.23606178977	6.36250317953984e-06	\\
8570.21484375	6.1287294805711e-06	\\
8571.19362571023	6.43349887491649e-06	\\
8572.17240767045	4.92923639289478e-06	\\
8573.15118963068	4.80467634369899e-06	\\
8574.12997159091	5.65568056651491e-06	\\
8575.10875355114	5.47334079978452e-06	\\
8576.08753551136	6.14754927379786e-06	\\
8577.06631747159	6.06681381832313e-06	\\
8578.04509943182	3.61536547142684e-06	\\
8579.02388139204	6.09350381076581e-06	\\
8580.00266335227	5.61530458763069e-06	\\
8580.9814453125	6.5756071457651e-06	\\
8581.96022727273	4.59176188522058e-06	\\
8582.93900923295	4.83187886053071e-06	\\
8583.91779119318	4.3308916300407e-06	\\
8584.89657315341	5.06031799908859e-06	\\
8585.87535511364	4.28306392943268e-06	\\
8586.85413707386	5.24004628589225e-06	\\
8587.83291903409	6.48990575296338e-06	\\
8588.81170099432	5.56557893873912e-06	\\
8589.79048295454	5.38641932295823e-06	\\
8590.76926491477	5.65342730676499e-06	\\
8591.748046875	4.3533289418104e-06	\\
8592.72682883523	5.04667375482436e-06	\\
8593.70561079545	5.44402405860557e-06	\\
8594.68439275568	5.34661361930728e-06	\\
8595.66317471591	5.14151027699135e-06	\\
8596.64195667614	6.88567812377062e-06	\\
8597.62073863636	4.81262197270607e-06	\\
8598.59952059659	5.6343020061413e-06	\\
8599.57830255682	5.07217339406235e-06	\\
8600.55708451704	7.69372510439241e-06	\\
8601.53586647727	5.00841239179353e-06	\\
8602.5146484375	5.60607761104895e-06	\\
8603.49343039773	5.19283213403767e-06	\\
8604.47221235795	6.09289498323208e-06	\\
8605.45099431818	5.38627124861789e-06	\\
8606.42977627841	2.70936899745534e-06	\\
8607.40855823864	5.53624515160678e-06	\\
8608.38734019886	5.17711652730667e-06	\\
8609.36612215909	5.35398590832717e-06	\\
8610.34490411932	4.98658132352625e-06	\\
8611.32368607954	5.01463364868784e-06	\\
8612.30246803977	5.69925103653245e-06	\\
8613.28125	4.97218343252285e-06	\\
8614.26003196023	5.608686139921e-06	\\
8615.23881392045	6.1481338498136e-06	\\
8616.21759588068	5.56134116723208e-06	\\
8617.19637784091	4.55083248024213e-06	\\
8618.17515980114	5.98482556878363e-06	\\
8619.15394176136	5.20633049563682e-06	\\
8620.13272372159	4.64015531382344e-06	\\
8621.11150568182	5.37077068196064e-06	\\
8622.09028764204	6.28934592198935e-06	\\
8623.06906960227	4.87215461920182e-06	\\
8624.0478515625	5.67165742036094e-06	\\
8625.02663352273	5.32531109349911e-06	\\
8626.00541548295	5.31119075739469e-06	\\
8626.98419744318	6.81097180080661e-06	\\
8627.96297940341	4.66593544070034e-06	\\
8628.94176136364	5.06649670309938e-06	\\
8629.92054332386	6.02365369780979e-06	\\
8630.89932528409	5.06356693671996e-06	\\
8631.87810724432	6.22941354960873e-06	\\
8632.85688920454	3.21395484956499e-06	\\
8633.83567116477	4.44645584907217e-06	\\
8634.814453125	6.30244319477694e-06	\\
8635.79323508523	5.05287167175521e-06	\\
8636.77201704545	5.0423604296972e-06	\\
8637.75079900568	5.25645716603245e-06	\\
8638.72958096591	5.6592423839212e-06	\\
8639.70836292614	4.7755595569559e-06	\\
8640.68714488636	4.99387831137429e-06	\\
8641.66592684659	5.72539287005212e-06	\\
8642.64470880682	5.60273496365144e-06	\\
8643.62349076704	5.39011188592242e-06	\\
8644.60227272727	5.41039117885248e-06	\\
8645.5810546875	7.10782938515429e-06	\\
8646.55983664773	4.7315805499235e-06	\\
8647.53861860795	5.71980563591302e-06	\\
8648.51740056818	6.75775305388137e-06	\\
8649.49618252841	6.74203601717758e-06	\\
8650.47496448864	5.60022388051925e-06	\\
8651.45374644886	6.78988785485756e-06	\\
8652.43252840909	5.07974609732137e-06	\\
8653.41131036932	5.7282185244286e-06	\\
8654.39009232954	7.04479996371526e-06	\\
8655.36887428977	5.54365683843613e-06	\\
8656.34765625	5.20040363287387e-06	\\
8657.32643821023	4.80203507459865e-06	\\
8658.30522017045	5.54325401739832e-06	\\
8659.28400213068	5.76635673770719e-06	\\
8660.26278409091	4.13541525146483e-06	\\
8661.24156605114	6.8339347842702e-06	\\
8662.22034801136	5.86799026983104e-06	\\
8663.19912997159	4.46134772704234e-06	\\
8664.17791193182	3.05230265939304e-06	\\
8665.15669389204	6.88344475626499e-06	\\
8666.13547585227	4.68391160599964e-06	\\
8667.1142578125	5.78525675316477e-06	\\
8668.09303977273	6.03414679484918e-06	\\
8669.07182173295	4.31487570606644e-06	\\
8670.05060369318	5.18777488654112e-06	\\
8671.02938565341	5.38627057681604e-06	\\
8672.00816761364	5.61538813349067e-06	\\
8672.98694957386	5.64762134647138e-06	\\
8673.96573153409	5.06909555061175e-06	\\
8674.94451349432	5.11372132011726e-06	\\
8675.92329545454	5.79468047885017e-06	\\
8676.90207741477	4.6771000049633e-06	\\
8677.880859375	5.21439583522837e-06	\\
8678.85964133523	5.15403556971326e-06	\\
8679.83842329545	5.83569840939729e-06	\\
8680.81720525568	4.16180770349086e-06	\\
8681.79598721591	6.15217451244296e-06	\\
8682.77476917614	6.09666635313098e-06	\\
8683.75355113636	4.86265064339824e-06	\\
8684.73233309659	7.86277630933032e-06	\\
8685.71111505682	5.00474722743891e-06	\\
8686.68989701704	5.33280013779725e-06	\\
8687.66867897727	4.48891918318677e-06	\\
8688.6474609375	5.13130804544696e-06	\\
8689.62624289773	6.26971070682613e-06	\\
8690.60502485795	4.45605506176183e-06	\\
8691.58380681818	4.63986941037758e-06	\\
8692.56258877841	6.73241156548067e-06	\\
8693.54137073864	5.7092842727823e-06	\\
8694.52015269886	4.58711964362581e-06	\\
8695.49893465909	4.79934330967686e-06	\\
8696.47771661932	5.24034159383047e-06	\\
8697.45649857954	5.65714365783909e-06	\\
8698.43528053977	4.24824744157032e-06	\\
8699.4140625	6.54741064008378e-06	\\
8700.39284446023	6.0822855566778e-06	\\
8701.37162642045	6.16734444648064e-06	\\
8702.35040838068	4.57039804757716e-06	\\
8703.32919034091	4.95947014794087e-06	\\
8704.30797230114	6.26935509629304e-06	\\
8705.28675426136	6.31352887293708e-06	\\
8706.26553622159	4.82544168338126e-06	\\
8707.24431818182	5.06786032329942e-06	\\
8708.22310014204	7.27586378966641e-06	\\
8709.20188210227	5.82905578914333e-06	\\
8710.1806640625	6.0164907713327e-06	\\
8711.15944602273	5.75404511572662e-06	\\
8712.13822798295	5.17717987769227e-06	\\
8713.11700994318	5.03787828072039e-06	\\
8714.09579190341	5.78164411385372e-06	\\
8715.07457386364	5.89617373226856e-06	\\
8716.05335582386	7.31952788546477e-06	\\
8717.03213778409	5.34080722925346e-06	\\
8718.01091974432	5.23785314041512e-06	\\
8718.98970170454	5.17865200896462e-06	\\
8719.96848366477	5.35510914748584e-06	\\
8720.947265625	5.19915919991196e-06	\\
8721.92604758523	5.52850565392071e-06	\\
8722.90482954545	4.66020653006422e-06	\\
8723.88361150568	5.09283250787385e-06	\\
8724.86239346591	5.53203036070596e-06	\\
8725.84117542614	4.56683764752042e-06	\\
8726.81995738636	4.8538074923795e-06	\\
8727.79873934659	5.96531145223848e-06	\\
8728.77752130682	6.02730505994664e-06	\\
8729.75630326704	6.27684333634683e-06	\\
8730.73508522727	6.12425660958809e-06	\\
8731.7138671875	4.76863302446755e-06	\\
8732.69264914773	7.28861567606394e-06	\\
8733.67143110795	6.54435708462141e-06	\\
8734.65021306818	5.15149678700942e-06	\\
8735.62899502841	6.14628356077525e-06	\\
8736.60777698864	5.75189460228251e-06	\\
8737.58655894886	6.30228403069766e-06	\\
8738.56534090909	5.33382207588166e-06	\\
8739.54412286932	4.78373872804511e-06	\\
8740.52290482954	4.03494831612742e-06	\\
8741.50168678977	4.49137664320842e-06	\\
8742.48046875	6.11233996212533e-06	\\
8743.45925071023	5.301992128602e-06	\\
8744.43803267045	7.26782938298378e-06	\\
8745.41681463068	5.51387179368294e-06	\\
8746.39559659091	5.37366856132545e-06	\\
8747.37437855114	4.57188058752673e-06	\\
8748.35316051136	6.83871180181088e-06	\\
8749.33194247159	5.25986604679302e-06	\\
8750.31072443182	9.66020526937903e-06	\\
8751.28950639204	4.39314045306659e-06	\\
8752.26828835227	6.62443703537445e-06	\\
8753.2470703125	5.30337105256542e-06	\\
8754.22585227273	5.93305348737411e-06	\\
8755.20463423295	5.5330991960484e-06	\\
8756.18341619318	6.01697350332285e-06	\\
8757.16219815341	6.68235210550091e-06	\\
8758.14098011364	3.73657335752306e-06	\\
8759.11976207386	5.50816300088392e-06	\\
8760.09854403409	4.93705842135856e-06	\\
8761.07732599432	3.80822818056109e-06	\\
8762.05610795454	4.7992004382159e-06	\\
8763.03488991477	6.36620268512147e-06	\\
8764.013671875	4.3732786989491e-06	\\
8764.99245383523	5.21005811583703e-06	\\
8765.97123579545	5.68777613029393e-06	\\
8766.95001775568	5.74781145845821e-06	\\
8767.92879971591	6.02311804897406e-06	\\
8768.90758167614	5.79703279902566e-06	\\
8769.88636363636	5.31837005367821e-06	\\
8770.86514559659	5.94620308569072e-06	\\
8771.84392755682	5.16495003719901e-06	\\
8772.82270951704	5.21726150194333e-06	\\
8773.80149147727	6.39667194350365e-06	\\
8774.7802734375	5.42928782397259e-06	\\
8775.75905539773	6.56023397901518e-06	\\
8776.73783735795	5.08805688217413e-06	\\
8777.71661931818	4.22220445123614e-06	\\
8778.69540127841	3.92138996898838e-06	\\
8779.67418323864	5.28895712201528e-06	\\
8780.65296519886	3.36713145117808e-06	\\
8781.63174715909	4.59984542085477e-06	\\
8782.61052911932	5.43036622145142e-06	\\
8783.58931107954	5.44280849843286e-06	\\
8784.56809303977	5.69927582755864e-06	\\
8785.546875	5.96291035681047e-06	\\
8786.52565696023	5.49673834460517e-06	\\
8787.50443892045	5.42234604801913e-06	\\
8788.48322088068	5.95033240267217e-06	\\
8789.46200284091	6.03838886741663e-06	\\
8790.44078480114	4.72284692925792e-06	\\
8791.41956676136	4.42865319625188e-06	\\
8792.39834872159	5.39942746217128e-06	\\
8793.37713068182	5.23355464674796e-06	\\
8794.35591264204	4.35395884849444e-06	\\
8795.33469460227	5.60769505912521e-06	\\
8796.3134765625	5.8241306960424e-06	\\
8797.29225852273	5.51806775450344e-06	\\
8798.27104048295	5.20633363605439e-06	\\
8799.24982244318	5.76675710109506e-06	\\
8800.22860440341	5.78545053589228e-06	\\
8801.20738636364	6.22384114454755e-06	\\
8802.18616832386	6.55931195467283e-06	\\
8803.16495028409	6.2830053327552e-06	\\
8804.14373224432	5.66802528120313e-06	\\
8805.12251420454	4.40649736093813e-06	\\
8806.10129616477	6.05411456145554e-06	\\
8807.080078125	3.58246072088697e-06	\\
8808.05886008523	4.80908889785003e-06	\\
8809.03764204545	4.57371771850079e-06	\\
8810.01642400568	5.27148137030667e-06	\\
8810.99520596591	4.62271426944368e-06	\\
8811.97398792614	4.58006671454262e-06	\\
8812.95276988636	4.65572960032991e-06	\\
8813.93155184659	3.79755801056656e-06	\\
8814.91033380682	4.75145390317275e-06	\\
8815.88911576704	4.96342667249711e-06	\\
8816.86789772727	7.11925275066623e-06	\\
8817.8466796875	4.92448487852394e-06	\\
8818.82546164773	4.10526568691948e-06	\\
8819.80424360795	4.40051140809314e-06	\\
8820.78302556818	3.97332205360606e-06	\\
8821.76180752841	5.43387851463692e-06	\\
8822.74058948864	5.92490895336867e-06	\\
8823.71937144886	5.74849176404554e-06	\\
8824.69815340909	6.47641437411411e-06	\\
8825.67693536932	6.67646200944167e-06	\\
8826.65571732954	4.07738018789593e-06	\\
8827.63449928977	3.94091858358413e-06	\\
8828.61328125	4.62930615575548e-06	\\
8829.59206321023	5.63527860281678e-06	\\
8830.57084517045	4.72100589438894e-06	\\
8831.54962713068	6.70171787796547e-06	\\
8832.52840909091	5.2514904860959e-06	\\
8833.50719105114	6.33809310702661e-06	\\
8834.48597301136	6.36669374575805e-06	\\
8835.46475497159	7.09296716512204e-06	\\
8836.44353693182	5.74645185249564e-06	\\
8837.42231889204	5.73599098705917e-06	\\
8838.40110085227	4.03098194418211e-06	\\
8839.3798828125	4.53451311807459e-06	\\
8840.35866477273	3.42109866151561e-06	\\
8841.33744673295	3.67420826950667e-06	\\
8842.31622869318	4.66177000267773e-06	\\
8843.29501065341	4.36566471212351e-06	\\
8844.27379261364	3.74396756850586e-06	\\
8845.25257457386	4.97412953607617e-06	\\
8846.23135653409	5.15882986698629e-06	\\
8847.21013849432	5.57619634161027e-06	\\
8848.18892045454	4.80532088256886e-06	\\
8849.16770241477	5.80631565293791e-06	\\
8850.146484375	5.57885759354921e-06	\\
8851.12526633523	4.99289041401849e-06	\\
8852.10404829545	4.73965160045619e-06	\\
8853.08283025568	4.05913892480968e-06	\\
8854.06161221591	5.30866625245924e-06	\\
8855.04039417614	4.65286174160619e-06	\\
8856.01917613636	4.47755578171238e-06	\\
8856.99795809659	6.26276566273172e-06	\\
8857.97674005682	4.67439382773905e-06	\\
8858.95552201704	4.31266688722923e-06	\\
8859.93430397727	5.01698166874538e-06	\\
8860.9130859375	5.76840672600965e-06	\\
8861.89186789773	4.98336296870812e-06	\\
8862.87064985795	5.78410812668717e-06	\\
8863.84943181818	5.58708715330326e-06	\\
8864.82821377841	5.58610775064256e-06	\\
8865.80699573864	3.71149675138284e-06	\\
8866.78577769886	4.79189314373915e-06	\\
8867.76455965909	5.30166701746981e-06	\\
8868.74334161932	5.38311867778586e-06	\\
8869.72212357954	5.24172198557914e-06	\\
8870.70090553977	4.61006374986599e-06	\\
8871.6796875	4.26115718262926e-06	\\
8872.65846946023	6.08479514659063e-06	\\
8873.63725142045	3.60553385831958e-06	\\
8874.61603338068	4.87137157553499e-06	\\
8875.59481534091	5.57941044838729e-06	\\
8876.57359730114	5.02456280285274e-06	\\
8877.55237926136	4.26407971734252e-06	\\
8878.53116122159	4.4300348003722e-06	\\
8879.50994318182	3.88474916800515e-06	\\
8880.48872514204	6.08481406574694e-06	\\
8881.46750710227	5.41697590423137e-06	\\
8882.4462890625	3.9844992341782e-06	\\
8883.42507102273	5.70329952583767e-06	\\
8884.40385298295	4.82573240045008e-06	\\
8885.38263494318	4.91646493288184e-06	\\
8886.36141690341	6.13584837299344e-06	\\
8887.34019886364	4.87300793480821e-06	\\
8888.31898082386	4.22484152891355e-06	\\
8889.29776278409	5.54729904633099e-06	\\
8890.27654474432	5.61981573063817e-06	\\
8891.25532670454	6.35624338981941e-06	\\
8892.23410866477	5.73037171940363e-06	\\
8893.212890625	4.14013728174692e-06	\\
8894.19167258523	4.55798357969295e-06	\\
8895.17045454545	4.73202458944053e-06	\\
8896.14923650568	5.04432265640867e-06	\\
8897.12801846591	5.18275902055884e-06	\\
8898.10680042614	3.45200151677818e-06	\\
8899.08558238636	5.9107242003536e-06	\\
8900.06436434659	4.43576995761636e-06	\\
8901.04314630682	5.53147659064713e-06	\\
8902.02192826704	5.28050898252092e-06	\\
8903.00071022727	5.25749618999513e-06	\\
8903.9794921875	5.92917332375804e-06	\\
8904.95827414773	4.65811649688935e-06	\\
8905.93705610795	5.44286626701213e-06	\\
8906.91583806818	5.3832609079008e-06	\\
8907.89462002841	4.34825796213425e-06	\\
8908.87340198864	5.86482875348004e-06	\\
8909.85218394886	5.74347087380894e-06	\\
8910.83096590909	6.20881555112556e-06	\\
8911.80974786932	5.70611088412839e-06	\\
8912.78852982954	4.15886664256827e-06	\\
8913.76731178977	5.75797216057975e-06	\\
8914.74609375	4.73444816112341e-06	\\
8915.72487571023	5.87104203517743e-06	\\
8916.70365767045	5.38331065540017e-06	\\
8917.68243963068	5.16772211995033e-06	\\
8918.66122159091	5.53300746208762e-06	\\
8919.64000355114	5.730969079273e-06	\\
8920.61878551136	4.35107744204901e-06	\\
8921.59756747159	7.24602601388784e-06	\\
8922.57634943182	4.33674509298107e-06	\\
8923.55513139204	4.62393517667597e-06	\\
8924.53391335227	5.21476626778185e-06	\\
8925.5126953125	5.66807528999042e-06	\\
8926.49147727273	5.69978109893986e-06	\\
8927.47025923295	4.8493834437686e-06	\\
8928.44904119318	4.39758798601594e-06	\\
8929.42782315341	3.30830255660645e-06	\\
8930.40660511364	3.50338121986012e-06	\\
8931.38538707386	4.76544980188549e-06	\\
8932.36416903409	4.06555148413832e-06	\\
8933.34295099432	4.98288489327948e-06	\\
8934.32173295454	4.81139371085434e-06	\\
8935.30051491477	5.12398460813016e-06	\\
8936.279296875	5.72708756608524e-06	\\
8937.25807883523	4.36096604854964e-06	\\
8938.23686079545	5.02610077088277e-06	\\
8939.21564275568	5.42467867394399e-06	\\
8940.19442471591	5.42586160834275e-06	\\
8941.17320667614	4.61812529105089e-06	\\
8942.15198863636	5.88497766272743e-06	\\
8943.13077059659	4.73085215644368e-06	\\
8944.10955255682	4.67529951824958e-06	\\
8945.08833451704	4.19356463451881e-06	\\
8946.06711647727	6.5710396256658e-06	\\
8947.0458984375	4.71921253270085e-06	\\
8948.02468039773	5.04109413670563e-06	\\
8949.00346235795	6.42046645925175e-06	\\
8949.98224431818	5.27078276807174e-06	\\
8950.96102627841	5.63349138559112e-06	\\
8951.93980823864	4.02686086731329e-06	\\
8952.91859019886	6.14191204841484e-06	\\
8953.89737215909	4.85895429242867e-06	\\
8954.87615411932	5.51114825404315e-06	\\
8955.85493607954	4.42921656863391e-06	\\
8956.83371803977	5.01458250879674e-06	\\
8957.8125	4.6704058748406e-06	\\
8958.79128196023	4.09185356504115e-06	\\
8959.77006392045	4.55521382435242e-06	\\
8960.74884588068	5.51239183755559e-06	\\
8961.72762784091	5.26911312909375e-06	\\
8962.70640980114	5.57678378720222e-06	\\
8963.68519176136	4.90900844872261e-06	\\
8964.66397372159	4.71715410629877e-06	\\
8965.64275568182	4.71596455242027e-06	\\
8966.62153764204	4.983014072779e-06	\\
8967.60031960227	3.09104323045239e-06	\\
8968.5791015625	4.94645227869997e-06	\\
8969.55788352273	6.45169743799963e-06	\\
8970.53666548295	3.54339589135186e-06	\\
8971.51544744318	5.38067558652558e-06	\\
8972.49422940341	6.04023764558533e-06	\\
8973.47301136364	5.53452016860948e-06	\\
8974.45179332386	5.93178095752483e-06	\\
8975.43057528409	6.47914168792374e-06	\\
8976.40935724432	5.65332412973948e-06	\\
8977.38813920454	6.01838220307463e-06	\\
8978.36692116477	4.95851109493748e-06	\\
8979.345703125	5.59446612615311e-06	\\
8980.32448508523	5.32223737324068e-06	\\
8981.30326704545	5.18766957260431e-06	\\
8982.28204900568	5.93958292794391e-06	\\
8983.26083096591	5.83689397220074e-06	\\
8984.23961292614	5.38959114272744e-06	\\
8985.21839488636	4.75591798655495e-06	\\
8986.19717684659	5.56405546846368e-06	\\
8987.17595880682	4.0633333675413e-06	\\
8988.15474076704	5.76426660794542e-06	\\
8989.13352272727	5.75763282308094e-06	\\
8990.1123046875	3.7421110875351e-06	\\
8991.09108664773	7.25699642185806e-06	\\
8992.06986860795	4.85046698206299e-06	\\
8993.04865056818	4.79360984230599e-06	\\
8994.02743252841	4.8923540051355e-06	\\
8995.00621448864	4.3618565984904e-06	\\
8995.98499644886	3.50033613054182e-06	\\
8996.96377840909	5.6811434009049e-06	\\
8997.94256036932	4.94276732605543e-06	\\
8998.92134232954	5.44295369553428e-06	\\
8999.90012428977	9.20017800101613e-06	\\
9000.87890625	4.98402970333462e-06	\\
9001.85768821023	5.96626905079164e-06	\\
9002.83647017045	6.23205702683961e-06	\\
9003.81525213068	4.59638684631431e-06	\\
9004.79403409091	5.80184934065584e-06	\\
9005.77281605114	4.32887144286913e-06	\\
9006.75159801136	6.06168228438749e-06	\\
9007.73037997159	5.7204020852829e-06	\\
9008.70916193182	3.35917063849214e-06	\\
9009.68794389204	5.84118521509079e-06	\\
9010.66672585227	5.30293762757814e-06	\\
9011.6455078125	5.54671077627446e-06	\\
9012.62428977273	5.94648274984025e-06	\\
9013.60307173295	5.04282600066073e-06	\\
9014.58185369318	5.00907512926972e-06	\\
9015.56063565341	5.27204509792741e-06	\\
9016.53941761364	4.54885243890767e-06	\\
9017.51819957386	4.89731820421174e-06	\\
9018.49698153409	4.53323396194986e-06	\\
9019.47576349432	5.66829719150625e-06	\\
9020.45454545454	3.56919828742374e-06	\\
9021.43332741477	4.38378598651854e-06	\\
9022.412109375	6.02395121566628e-06	\\
9023.39089133523	4.81881585210566e-06	\\
9024.36967329545	5.10947764730024e-06	\\
9025.34845525568	4.80320230255392e-06	\\
9026.32723721591	4.75548499428507e-06	\\
9027.30601917614	5.28515143333155e-06	\\
9028.28480113636	4.49362597980237e-06	\\
9029.26358309659	6.01078990424308e-06	\\
9030.24236505682	5.67888718660043e-06	\\
9031.22114701704	3.35149537776366e-06	\\
9032.19992897727	4.66068550466239e-06	\\
9033.1787109375	3.89773032892667e-06	\\
9034.15749289773	3.06818612200116e-06	\\
9035.13627485795	5.51000115065948e-06	\\
9036.11505681818	4.11779706079963e-06	\\
9037.09383877841	7.09239682540161e-06	\\
9038.07262073864	4.96926665122229e-06	\\
9039.05140269886	4.5195191702634e-06	\\
9040.03018465909	7.12827284134849e-06	\\
9041.00896661932	5.77283595143322e-06	\\
9041.98774857954	5.29116901766464e-06	\\
9042.96653053977	5.24014072121217e-06	\\
9043.9453125	4.41141155068554e-06	\\
9044.92409446023	4.66350589432837e-06	\\
9045.90287642045	6.46259532958803e-06	\\
9046.88165838068	4.798866887177e-06	\\
9047.86044034091	4.69557542300515e-06	\\
9048.83922230114	5.37483762671077e-06	\\
9049.81800426136	4.81927578166854e-06	\\
9050.79678622159	6.02003152923557e-06	\\
9051.77556818182	6.2613836634685e-06	\\
9052.75435014204	4.82438520903565e-06	\\
9053.73313210227	5.38867628286564e-06	\\
9054.7119140625	4.58299317664415e-06	\\
9055.69069602273	3.79058974396643e-06	\\
9056.66947798295	5.05036333314179e-06	\\
9057.64825994318	4.60981114731081e-06	\\
9058.62704190341	3.18886433071147e-06	\\
9059.60582386364	4.47834581368376e-06	\\
9060.58460582386	4.68309393732763e-06	\\
9061.56338778409	4.74374432525448e-06	\\
9062.54216974432	4.7141637300597e-06	\\
9063.52095170454	5.11806134322819e-06	\\
9064.49973366477	6.30244104121468e-06	\\
9065.478515625	5.94069911181038e-06	\\
9066.45729758523	5.08002054745013e-06	\\
9067.43607954545	5.53524390451664e-06	\\
9068.41486150568	4.68713628638695e-06	\\
9069.39364346591	4.76989220736131e-06	\\
9070.37242542614	3.9299828064537e-06	\\
9071.35120738636	5.19685300310223e-06	\\
9072.32998934659	3.32229305918777e-06	\\
9073.30877130682	4.2053949119024e-06	\\
9074.28755326704	4.35419139818033e-06	\\
9075.26633522727	3.74705939418957e-06	\\
9076.2451171875	5.1032968420385e-06	\\
9077.22389914773	5.78796172505333e-06	\\
9078.20268110795	4.49323509784139e-06	\\
9079.18146306818	5.01425040487155e-06	\\
9080.16024502841	4.24803781357596e-06	\\
9081.13902698864	4.84457372618688e-06	\\
9082.11780894886	5.14477382277203e-06	\\
9083.09659090909	6.12956054081735e-06	\\
9084.07537286932	5.01684740037186e-06	\\
9085.05415482954	6.23096017303163e-06	\\
9086.03293678977	5.39860344878225e-06	\\
9087.01171875	4.9725179819891e-06	\\
9087.99050071023	5.57957318374702e-06	\\
9088.96928267045	4.08315174417924e-06	\\
9089.94806463068	5.00771380031694e-06	\\
9090.92684659091	5.72086832811003e-06	\\
9091.90562855114	5.11461425085537e-06	\\
9092.88441051136	5.85614195295112e-06	\\
9093.86319247159	5.86477667649871e-06	\\
9094.84197443182	5.12478084229025e-06	\\
9095.82075639204	4.37034628886602e-06	\\
9096.79953835227	4.61737751210393e-06	\\
9097.7783203125	3.44714178595524e-06	\\
9098.75710227273	5.01136020185328e-06	\\
9099.73588423295	4.24002506567166e-06	\\
9100.71466619318	5.61088783740228e-06	\\
9101.69344815341	5.83451683360434e-06	\\
9102.67223011364	5.34924741999786e-06	\\
9103.65101207386	5.13159247401723e-06	\\
9104.62979403409	5.92981710346055e-06	\\
9105.60857599432	4.30623996423928e-06	\\
9106.58735795454	5.24351360435113e-06	\\
9107.56613991477	4.90472806833066e-06	\\
9108.544921875	5.65111067432142e-06	\\
9109.52370383523	5.37816407680803e-06	\\
9110.50248579545	4.16508997333669e-06	\\
9111.48126775568	4.03334631213694e-06	\\
9112.46004971591	4.39061202993842e-06	\\
9113.43883167614	5.40892356020955e-06	\\
9114.41761363636	4.20900188015859e-06	\\
9115.39639559659	4.80539776724446e-06	\\
9116.37517755682	4.57801068260165e-06	\\
9117.35395951704	4.91378618798168e-06	\\
9118.33274147727	5.07259618692623e-06	\\
9119.3115234375	5.28704468417043e-06	\\
9120.29030539773	4.57132544777601e-06	\\
9121.26908735795	5.87967213532302e-06	\\
9122.24786931818	4.72377866710379e-06	\\
9123.22665127841	4.46654939403946e-06	\\
9124.20543323864	4.39537297361547e-06	\\
9125.18421519886	4.74245072153304e-06	\\
9126.16299715909	4.63552037664371e-06	\\
9127.14177911932	5.38662909413593e-06	\\
9128.12056107954	4.28613262811258e-06	\\
9129.09934303977	6.35248894468795e-06	\\
9130.078125	5.17996721878991e-06	\\
9131.05690696023	4.56863091265958e-06	\\
9132.03568892045	4.12652525513543e-06	\\
9133.01447088068	5.65132174725236e-06	\\
9133.99325284091	6.08041538537382e-06	\\
9134.97203480114	4.77078788285531e-06	\\
9135.95081676136	5.01284738396714e-06	\\
9136.92959872159	5.42060775907465e-06	\\
9137.90838068182	4.75706499241167e-06	\\
9138.88716264204	4.43509829047124e-06	\\
9139.86594460227	4.50665870950913e-06	\\
9140.8447265625	6.95779518260947e-06	\\
9141.82350852273	4.62884582783893e-06	\\
9142.80229048295	4.92014428714672e-06	\\
9143.78107244318	5.23458893135872e-06	\\
9144.75985440341	5.57799347864833e-06	\\
9145.73863636364	4.17506724014637e-06	\\
9146.71741832386	4.43119265111068e-06	\\
9147.69620028409	5.62946873837977e-06	\\
9148.67498224432	4.69513998938121e-06	\\
9149.65376420454	5.12886886043402e-06	\\
9150.63254616477	5.89288616155984e-06	\\
9151.611328125	4.23835171841367e-06	\\
9152.59011008523	5.61175404622053e-06	\\
9153.56889204545	6.46182481496955e-06	\\
9154.54767400568	4.87751211251433e-06	\\
9155.52645596591	5.03281451590845e-06	\\
9156.50523792614	5.04197848923864e-06	\\
9157.48401988636	5.00697848848496e-06	\\
9158.46280184659	5.31526492911218e-06	\\
9159.44158380682	5.88535395447183e-06	\\
9160.42036576704	4.70138261478539e-06	\\
9161.39914772727	3.88537282474509e-06	\\
9162.3779296875	5.35652851698607e-06	\\
9163.35671164773	5.11305881941593e-06	\\
9164.33549360795	4.48888038741811e-06	\\
9165.31427556818	4.37753304863494e-06	\\
9166.29305752841	4.91445831099855e-06	\\
9167.27183948864	5.27355402137619e-06	\\
9168.25062144886	5.2895735656431e-06	\\
9169.22940340909	4.89567122394031e-06	\\
9170.20818536932	5.17012071515821e-06	\\
9171.18696732954	4.84344978715131e-06	\\
9172.16574928977	5.77390824578775e-06	\\
9173.14453125	4.32661829722891e-06	\\
9174.12331321023	4.31827330621758e-06	\\
9175.10209517045	4.45071642439133e-06	\\
9176.08087713068	3.31952616039504e-06	\\
9177.05965909091	5.09902612880914e-06	\\
9178.03844105114	5.91962981074164e-06	\\
9179.01722301136	4.63495669113249e-06	\\
9179.99600497159	4.7787447002274e-06	\\
9180.97478693182	4.5320964861573e-06	\\
9181.95356889204	3.75635483325114e-06	\\
9182.93235085227	5.5572901981581e-06	\\
9183.9111328125	5.34354943679787e-06	\\
9184.88991477273	4.86185895910712e-06	\\
9185.86869673295	6.02074180111082e-06	\\
9186.84747869318	6.69572309770228e-06	\\
9187.82626065341	5.39561187502881e-06	\\
9188.80504261364	4.99991588554653e-06	\\
9189.78382457386	4.54039889883155e-06	\\
9190.76260653409	4.90775939424823e-06	\\
9191.74138849432	4.63017609412563e-06	\\
9192.72017045454	5.20265071538648e-06	\\
9193.69895241477	5.16315944208516e-06	\\
9194.677734375	3.75386999507248e-06	\\
9195.65651633523	5.2392644460839e-06	\\
9196.63529829545	4.05980394722698e-06	\\
9197.61408025568	4.95451480189456e-06	\\
9198.59286221591	4.18501552323819e-06	\\
9199.57164417614	5.15415918090055e-06	\\
9200.55042613636	4.65389345921793e-06	\\
9201.52920809659	4.04559971900927e-06	\\
9202.50799005682	5.28253898565395e-06	\\
9203.48677201704	5.81262608912134e-06	\\
9204.46555397727	5.48104033493509e-06	\\
9205.4443359375	6.17858973874704e-06	\\
9206.42311789773	5.32166198904376e-06	\\
9207.40189985795	5.10898834233942e-06	\\
9208.38068181818	6.11986221217826e-06	\\
9209.35946377841	4.66698034158296e-06	\\
9210.33824573864	4.42746879422842e-06	\\
9211.31702769886	4.93333582081454e-06	\\
9212.29580965909	4.45174463955495e-06	\\
9213.27459161932	5.59308652813364e-06	\\
9214.25337357954	4.65935192139692e-06	\\
9215.23215553977	5.93811610143364e-06	\\
9216.2109375	5.61178075569749e-06	\\
9217.18971946023	5.15302069404285e-06	\\
9218.16850142045	5.35688504108375e-06	\\
9219.14728338068	4.83600468876244e-06	\\
9220.12606534091	4.48018023797615e-06	\\
9221.10484730114	5.64831886813786e-06	\\
9222.08362926136	4.45035171734903e-06	\\
9223.06241122159	4.981109549677e-06	\\
9224.04119318182	4.86978182937699e-06	\\
9225.01997514204	4.99679320382383e-06	\\
9225.99875710227	5.53890849990195e-06	\\
9226.9775390625	4.66857957993096e-06	\\
9227.95632102273	5.35975721639576e-06	\\
9228.93510298295	5.4652681795066e-06	\\
9229.91388494318	3.92038115628484e-06	\\
9230.89266690341	5.56599082228855e-06	\\
9231.87144886364	4.42726376641217e-06	\\
9232.85023082386	5.05026544053505e-06	\\
9233.82901278409	4.82941898189609e-06	\\
9234.80779474432	4.57616313650755e-06	\\
9235.78657670454	5.8974621968583e-06	\\
9236.76535866477	5.29589074879935e-06	\\
9237.744140625	4.30176857911018e-06	\\
9238.72292258523	4.7383367314261e-06	\\
9239.70170454545	3.81777768929471e-06	\\
9240.68048650568	4.24366565446122e-06	\\
9241.65926846591	5.71840351567686e-06	\\
9242.63805042614	3.86871991782643e-06	\\
9243.61683238636	6.33830837476412e-06	\\
9244.59561434659	6.11791890818277e-06	\\
9245.57439630682	5.10107295039556e-06	\\
9246.55317826704	5.26541084195725e-06	\\
9247.53196022727	4.66825074034192e-06	\\
9248.5107421875	4.99050188620963e-06	\\
9249.48952414773	4.75282239632388e-06	\\
9250.46830610795	4.54163523275032e-06	\\
9251.44708806818	5.0950111061877e-06	\\
9252.42587002841	4.59599339359263e-06	\\
9253.40465198864	4.52895233613479e-06	\\
9254.38343394886	4.84854011243215e-06	\\
9255.36221590909	5.07849861343709e-06	\\
9256.34099786932	5.71033232096696e-06	\\
9257.31977982954	5.48219994146444e-06	\\
9258.29856178977	5.00909944465599e-06	\\
9259.27734375	6.02075250388669e-06	\\
9260.25612571023	5.89854085203035e-06	\\
9261.23490767045	5.07665486504217e-06	\\
9262.21368963068	5.61212119602386e-06	\\
9263.19247159091	4.96051660658008e-06	\\
9264.17125355114	4.51264077908372e-06	\\
9265.15003551136	4.51808134565292e-06	\\
9266.12881747159	5.05991744527868e-06	\\
9267.10759943182	4.87853888284753e-06	\\
9268.08638139204	3.27417533724591e-06	\\
9269.06516335227	7.16116886323878e-06	\\
9270.0439453125	4.97339165213942e-06	\\
9271.02272727273	5.75725768186161e-06	\\
9272.00150923295	4.42424464930642e-06	\\
9272.98029119318	5.5538039850751e-06	\\
9273.95907315341	5.0923069601073e-06	\\
9274.93785511364	5.39006466647443e-06	\\
9275.91663707386	6.18602463438984e-06	\\
9276.89541903409	3.57412118482675e-06	\\
9277.87420099432	5.75789695197842e-06	\\
9278.85298295454	4.75532930894563e-06	\\
9279.83176491477	5.29336588371147e-06	\\
9280.810546875	4.63023587068498e-06	\\
9281.78932883523	5.04578310585597e-06	\\
9282.76811079545	5.46774478577175e-06	\\
9283.74689275568	4.42498411692546e-06	\\
9284.72567471591	4.40964292671832e-06	\\
9285.70445667614	4.80732385905253e-06	\\
9286.68323863636	3.70812849191505e-06	\\
9287.66202059659	5.0980584251628e-06	\\
9288.64080255682	3.61504369231233e-06	\\
9289.61958451704	5.38103310397655e-06	\\
9290.59836647727	4.617694666881e-06	\\
9291.5771484375	4.83867749753432e-06	\\
9292.55593039773	4.99906868684343e-06	\\
9293.53471235795	4.83462269730134e-06	\\
9294.51349431818	4.55822420095588e-06	\\
9295.49227627841	4.95591575454857e-06	\\
9296.47105823864	3.25446326845273e-06	\\
9297.44984019886	5.41022908732486e-06	\\
9298.42862215909	4.43678349325298e-06	\\
9299.40740411932	6.07173298788315e-06	\\
9300.38618607954	4.98634251183207e-06	\\
9301.36496803977	5.74853907255074e-06	\\
9302.34375	6.04404645379965e-06	\\
9303.32253196023	5.55984562032017e-06	\\
9304.30131392045	4.72490226984649e-06	\\
9305.28009588068	5.55881750685459e-06	\\
9306.25887784091	3.83458971970542e-06	\\
9307.23765980114	4.77250646789562e-06	\\
9308.21644176136	5.45489388600153e-06	\\
9309.19522372159	4.94273996823947e-06	\\
9310.17400568182	5.29919135552489e-06	\\
9311.15278764204	4.84094276080213e-06	\\
9312.13156960227	5.72803165712207e-06	\\
9313.1103515625	5.80076680557721e-06	\\
9314.08913352273	6.19726811028026e-06	\\
9315.06791548295	5.17536429528334e-06	\\
9316.04669744318	4.58279300586489e-06	\\
9317.02547940341	5.61691398244959e-06	\\
9318.00426136364	5.59322136616128e-06	\\
9318.98304332386	4.62725656593997e-06	\\
9319.96182528409	5.65247606792088e-06	\\
9320.94060724432	5.40792444742352e-06	\\
9321.91938920454	4.70695227710319e-06	\\
9322.89817116477	5.6635613898197e-06	\\
9323.876953125	4.77459934889986e-06	\\
9324.85573508523	5.12253904799755e-06	\\
9325.83451704545	5.13773073976366e-06	\\
9326.81329900568	5.42066090834454e-06	\\
9327.79208096591	5.6412478069223e-06	\\
9328.77086292614	5.0230606356717e-06	\\
9329.74964488636	4.50482105571145e-06	\\
9330.72842684659	5.00515044543286e-06	\\
9331.70720880682	4.79295824871387e-06	\\
9332.68599076704	4.66969837182922e-06	\\
9333.66477272727	5.45236452621173e-06	\\
9334.6435546875	5.62367214895774e-06	\\
9335.62233664773	4.88705400479799e-06	\\
9336.60111860795	4.41787660808266e-06	\\
9337.57990056818	4.20431601992473e-06	\\
9338.55868252841	5.61127699784859e-06	\\
9339.53746448864	4.82227730265636e-06	\\
9340.51624644886	3.64792593464692e-06	\\
9341.49502840909	4.70984345084001e-06	\\
9342.47381036932	4.35172418548703e-06	\\
9343.45259232954	4.70874890129521e-06	\\
9344.43137428977	4.57075049619143e-06	\\
9345.41015625	4.28964714127904e-06	\\
9346.38893821023	4.93904098265032e-06	\\
9347.36772017045	5.49353290371248e-06	\\
9348.34650213068	5.03568248574615e-06	\\
9349.32528409091	4.83969184185899e-06	\\
9350.30406605114	4.70125900944143e-06	\\
9351.28284801136	4.69519957634691e-06	\\
9352.26162997159	5.55088011978654e-06	\\
9353.24041193182	4.76692457602398e-06	\\
9354.21919389204	5.55587428664905e-06	\\
9355.19797585227	5.53032042610767e-06	\\
9356.1767578125	4.98959812811227e-06	\\
9357.15553977273	5.78340787253626e-06	\\
9358.13432173295	5.53037431082698e-06	\\
9359.11310369318	4.75085575880427e-06	\\
9360.09188565341	5.53989473615014e-06	\\
9361.07066761364	5.23976008093915e-06	\\
9362.04944957386	5.59318620350679e-06	\\
9363.02823153409	5.55976203562327e-06	\\
9364.00701349432	4.87612594396416e-06	\\
9364.98579545454	4.11041666940964e-06	\\
9365.96457741477	5.1395935128371e-06	\\
9366.943359375	5.51267704799876e-06	\\
9367.92214133523	5.30189595983296e-06	\\
9368.90092329545	5.45165309053084e-06	\\
9369.87970525568	4.83056503646442e-06	\\
9370.85848721591	5.30913467273389e-06	\\
9371.83726917614	5.77276406110603e-06	\\
9372.81605113636	4.3726581155822e-06	\\
9373.79483309659	4.50299016493033e-06	\\
9374.77361505682	5.36283431093309e-06	\\
9375.75239701704	4.99978686583182e-06	\\
9376.73117897727	5.54836339267696e-06	\\
9377.7099609375	5.75950725456782e-06	\\
9378.68874289773	5.47965777444141e-06	\\
9379.66752485795	5.31333824832826e-06	\\
9380.64630681818	4.58980671268127e-06	\\
9381.62508877841	4.86537369212674e-06	\\
9382.60387073864	4.61397164631848e-06	\\
9383.58265269886	4.6025726796741e-06	\\
9384.56143465909	4.64103330962922e-06	\\
9385.54021661932	5.14826461312831e-06	\\
9386.51899857954	6.49684610923187e-06	\\
9387.49778053977	5.21180193103597e-06	\\
9388.4765625	5.84146776174208e-06	\\
9389.45534446023	4.88666663159616e-06	\\
9390.43412642045	4.52511886097225e-06	\\
9391.41290838068	5.10470240363595e-06	\\
9392.39169034091	5.5052386115806e-06	\\
9393.37047230114	6.02962117679177e-06	\\
9394.34925426136	5.3989175470512e-06	\\
9395.32803622159	5.1987767801174e-06	\\
9396.30681818182	6.21023731828139e-06	\\
9397.28560014204	5.05195490142456e-06	\\
9398.26438210227	5.84685671495795e-06	\\
9399.2431640625	4.31696655851378e-06	\\
9400.22194602273	3.89412814953351e-06	\\
9401.20072798295	3.77265215439584e-06	\\
9402.17950994318	4.55922325408704e-06	\\
9403.15829190341	4.35463613739718e-06	\\
9404.13707386364	5.19171809769303e-06	\\
9405.11585582386	5.34001808096279e-06	\\
9406.09463778409	4.56381266714499e-06	\\
9407.07341974432	4.53888837297448e-06	\\
9408.05220170454	5.68927672902503e-06	\\
9409.03098366477	5.38883421835138e-06	\\
9410.009765625	4.96480612442166e-06	\\
9410.98854758523	5.46403032675779e-06	\\
9411.96732954545	4.71214029326691e-06	\\
9412.94611150568	5.4512421031395e-06	\\
9413.92489346591	5.07266776037243e-06	\\
9414.90367542614	3.63577902511525e-06	\\
9415.88245738636	5.92490553831679e-06	\\
9416.86123934659	6.30335421097674e-06	\\
9417.84002130682	4.4293094458738e-06	\\
9418.81880326704	4.87403116708943e-06	\\
9419.79758522727	4.80770301526102e-06	\\
9420.7763671875	4.39757548076527e-06	\\
9421.75514914773	6.31258235526976e-06	\\
9422.73393110795	6.22174075214624e-06	\\
9423.71271306818	4.64822066153428e-06	\\
9424.69149502841	4.73016750047113e-06	\\
9425.67027698864	6.04958429389608e-06	\\
9426.64905894886	5.20457223734284e-06	\\
9427.62784090909	6.14879083494405e-06	\\
9428.60662286932	5.39215986510256e-06	\\
9429.58540482954	5.36061089073215e-06	\\
9430.56418678977	5.83366493505865e-06	\\
9431.54296875	5.43148104506273e-06	\\
9432.52175071023	5.83472611758136e-06	\\
9433.50053267045	6.34449172238907e-06	\\
9434.47931463068	5.25117038783119e-06	\\
9435.45809659091	4.93019199531174e-06	\\
9436.43687855114	5.4278591264343e-06	\\
9437.41566051136	4.75355225847305e-06	\\
9438.39444247159	4.47217719687301e-06	\\
9439.37322443182	4.59196020798319e-06	\\
9440.35200639204	5.19422785590928e-06	\\
9441.33078835227	4.07769117457979e-06	\\
9442.3095703125	5.01888463818636e-06	\\
9443.28835227273	4.14003018877488e-06	\\
9444.26713423295	5.17859121864971e-06	\\
9445.24591619318	5.3381727495596e-06	\\
9446.22469815341	4.86512134619459e-06	\\
9447.20348011364	3.95087507050266e-06	\\
9448.18226207386	5.00398743898621e-06	\\
9449.16104403409	5.32310568469689e-06	\\
9450.13982599432	4.42945747429855e-06	\\
9451.11860795454	4.52421626490787e-06	\\
9452.09738991477	5.28064574186348e-06	\\
9453.076171875	4.74674622742593e-06	\\
9454.05495383523	4.93576118858457e-06	\\
9455.03373579545	4.08987248091298e-06	\\
9456.01251775568	5.29671503529109e-06	\\
9456.99129971591	4.71016105761761e-06	\\
9457.97008167614	4.98480669973301e-06	\\
9458.94886363636	4.261861876362e-06	\\
9459.92764559659	5.37460575527835e-06	\\
9460.90642755682	4.46369401219398e-06	\\
9461.88520951704	4.4068756762602e-06	\\
9462.86399147727	5.31137739633236e-06	\\
9463.8427734375	4.74295767420234e-06	\\
9464.82155539773	5.18973708638571e-06	\\
9465.80033735795	5.86383876744166e-06	\\
9466.77911931818	5.04325394411596e-06	\\
9467.75790127841	5.4819706584226e-06	\\
9468.73668323864	5.04577771649345e-06	\\
9469.71546519886	4.83431937991181e-06	\\
9470.69424715909	3.58986747475239e-06	\\
9471.67302911932	5.29827673779217e-06	\\
9472.65181107954	3.31868609172422e-06	\\
9473.63059303977	4.71352723208783e-06	\\
9474.609375	3.12975469761305e-06	\\
9475.58815696023	3.91416812305468e-06	\\
9476.56693892045	4.5850716524951e-06	\\
9477.54572088068	5.75362750261626e-06	\\
9478.52450284091	6.02491018085751e-06	\\
9479.50328480114	4.47101555363936e-06	\\
9480.48206676136	4.89465747211898e-06	\\
9481.46084872159	4.53268962459865e-06	\\
9482.43963068182	4.40886895795288e-06	\\
9483.41841264204	5.45931931227674e-06	\\
9484.39719460227	5.01476170911124e-06	\\
9485.3759765625	5.85753706298016e-06	\\
9486.35475852273	4.6873735041858e-06	\\
9487.33354048295	5.44744365407714e-06	\\
9488.31232244318	3.82102503834224e-06	\\
9489.29110440341	5.66386028795969e-06	\\
9490.26988636364	4.47551746627528e-06	\\
9491.24866832386	4.75791140116896e-06	\\
9492.22745028409	4.55916629412818e-06	\\
9493.20623224432	4.38113177913825e-06	\\
9494.18501420454	3.88411225669931e-06	\\
9495.16379616477	4.61431758785104e-06	\\
9496.142578125	3.74421364667994e-06	\\
9497.12136008523	4.19284999577339e-06	\\
9498.10014204545	4.31109446220072e-06	\\
9499.07892400568	5.59688359237308e-06	\\
9500.05770596591	5.87511135812812e-06	\\
9501.03648792614	4.48666085646316e-06	\\
9502.01526988636	4.06866686679507e-06	\\
9502.99405184659	4.55926982672462e-06	\\
9503.97283380682	5.2760020675021e-06	\\
9504.95161576704	5.27172949773912e-06	\\
9505.93039772727	5.59643781134623e-06	\\
9506.9091796875	4.11385221757394e-06	\\
9507.88796164773	4.61605883445888e-06	\\
9508.86674360795	4.68436471127456e-06	\\
9509.84552556818	5.65427494310651e-06	\\
9510.82430752841	4.52347635636284e-06	\\
9511.80308948864	4.3056709584796e-06	\\
9512.78187144886	4.59631031589872e-06	\\
9513.76065340909	4.53967078588929e-06	\\
9514.73943536932	3.80429632708133e-06	\\
9515.71821732954	4.32144683351971e-06	\\
9516.69699928977	4.51826355341214e-06	\\
9517.67578125	4.87045420015379e-06	\\
9518.65456321023	4.50774547595574e-06	\\
9519.63334517045	4.27952412547766e-06	\\
9520.61212713068	3.20537635940116e-06	\\
9521.59090909091	5.00380519078113e-06	\\
9522.56969105114	5.15795962700059e-06	\\
9523.54847301136	2.99986463094884e-06	\\
9524.52725497159	3.7227306194232e-06	\\
9525.50603693182	4.60799475862112e-06	\\
9526.48481889204	5.28692868054551e-06	\\
9527.46360085227	4.36672541946506e-06	\\
9528.4423828125	5.37068162111542e-06	\\
9529.42116477273	4.76414295875074e-06	\\
9530.39994673295	3.78087734764868e-06	\\
9531.37872869318	3.6586280418518e-06	\\
9532.35751065341	4.78483620356422e-06	\\
9533.33629261364	3.47011979721155e-06	\\
9534.31507457386	4.70816128864773e-06	\\
9535.29385653409	3.64731947883319e-06	\\
9536.27263849432	3.44097732543096e-06	\\
9537.25142045454	4.17496984901526e-06	\\
9538.23020241477	3.55873335154326e-06	\\
9539.208984375	4.16147715251491e-06	\\
9540.18776633523	4.4345091102044e-06	\\
9541.16654829545	3.46761182059264e-06	\\
9542.14533025568	4.13757507012853e-06	\\
9543.12411221591	4.58950614384245e-06	\\
9544.10289417614	4.12096552845093e-06	\\
9545.08167613636	3.48681895561831e-06	\\
9546.06045809659	4.1182073788898e-06	\\
9547.03924005682	3.62663217350908e-06	\\
9548.01802201704	4.12680542167389e-06	\\
9548.99680397727	4.87713121866935e-06	\\
9549.9755859375	5.06880901152009e-06	\\
9550.95436789773	4.16697450236014e-06	\\
9551.93314985795	4.53835288910986e-06	\\
9552.91193181818	3.38162282550129e-06	\\
9553.89071377841	4.44616767959791e-06	\\
9554.86949573864	3.72072738655326e-06	\\
9555.84827769886	3.94679766195851e-06	\\
9556.82705965909	3.81719722400868e-06	\\
9557.80584161932	3.96328622294749e-06	\\
9558.78462357954	4.84517330992083e-06	\\
9559.76340553977	4.77193921906823e-06	\\
9560.7421875	4.26024166016978e-06	\\
9561.72096946023	3.02785144163166e-06	\\
9562.69975142045	4.16110234394627e-06	\\
9563.67853338068	3.15390369338836e-06	\\
9564.65731534091	3.85232427967867e-06	\\
9565.63609730114	3.34507724602072e-06	\\
9566.61487926136	3.85542718528089e-06	\\
9567.59366122159	4.18203531052221e-06	\\
9568.57244318182	4.04100853859982e-06	\\
9569.55122514204	4.56006240908397e-06	\\
9570.53000710227	3.94691035539564e-06	\\
9571.5087890625	2.82937915430641e-06	\\
9572.48757102273	3.66079222902092e-06	\\
9573.46635298295	3.51673350040046e-06	\\
9574.44513494318	5.33064515503088e-06	\\
9575.42391690341	3.9730374556894e-06	\\
9576.40269886364	4.14701602338035e-06	\\
9577.38148082386	3.86461331305077e-06	\\
9578.36026278409	3.77421372218696e-06	\\
9579.33904474432	4.62413344891489e-06	\\
9580.31782670454	3.73075125071543e-06	\\
9581.29660866477	4.63987003424412e-06	\\
9582.275390625	3.70866555669843e-06	\\
9583.25417258523	3.7606364951724e-06	\\
9584.23295454545	4.23987389906606e-06	\\
9585.21173650568	4.46078474406468e-06	\\
9586.19051846591	4.12041568481738e-06	\\
9587.16930042614	3.40919922543321e-06	\\
9588.14808238636	3.77359041869273e-06	\\
9589.12686434659	2.7966971297856e-06	\\
9590.10564630682	4.24270676398935e-06	\\
9591.08442826704	3.62047859184796e-06	\\
9592.06321022727	3.73648711312426e-06	\\
9593.0419921875	3.03969285701067e-06	\\
9594.02077414773	3.17020315488004e-06	\\
9594.99955610795	3.00300827915516e-06	\\
9595.97833806818	4.16674710745133e-06	\\
9596.95712002841	3.06670005055334e-06	\\
9597.93590198864	3.21100854414804e-06	\\
9598.91468394886	3.00992478269719e-06	\\
9599.89346590909	3.27609300900731e-06	\\
9600.87224786932	4.08665268258707e-06	\\
9601.85102982954	4.81659640959983e-06	\\
9602.82981178977	4.01093427980045e-06	\\
9603.80859375	3.75135239970392e-06	\\
9604.78737571023	3.08532715145033e-06	\\
9605.76615767045	3.53978318682928e-06	\\
9606.74493963068	3.36435696130086e-06	\\
9607.72372159091	3.00178354466449e-06	\\
9608.70250355114	3.06996015500014e-06	\\
9609.68128551136	2.83869813112839e-06	\\
9610.66006747159	3.43080642815034e-06	\\
9611.63884943182	2.93282841415371e-06	\\
9612.61763139204	3.80905537964432e-06	\\
9613.59641335227	4.03535891506719e-06	\\
9614.5751953125	3.7684103213126e-06	\\
9615.55397727273	3.23280883508697e-06	\\
9616.53275923295	2.91163754086063e-06	\\
9617.51154119318	4.07301891836052e-06	\\
9618.49032315341	3.61584421904221e-06	\\
9619.46910511364	3.17457091230487e-06	\\
9620.44788707386	4.1576393827266e-06	\\
9621.42666903409	2.88445124943637e-06	\\
9622.40545099432	2.82541209126587e-06	\\
9623.38423295454	2.7910419735374e-06	\\
9624.36301491477	2.97349065112043e-06	\\
9625.341796875	2.96519342481835e-06	\\
9626.32057883523	3.88253594797666e-06	\\
9627.29936079545	4.19625752094573e-06	\\
9628.27814275568	4.17472853198279e-06	\\
9629.25692471591	3.43335191374385e-06	\\
9630.23570667614	3.41095103940283e-06	\\
9631.21448863636	3.72838111931455e-06	\\
9632.19327059659	3.00239933099355e-06	\\
9633.17205255682	2.96888907987339e-06	\\
9634.15083451704	3.73328824046485e-06	\\
9635.12961647727	3.00077562519061e-06	\\
9636.1083984375	3.39355660027366e-06	\\
9637.08718039773	2.97153405406957e-06	\\
9638.06596235795	3.73469796968328e-06	\\
9639.04474431818	3.11298376624705e-06	\\
9640.02352627841	3.48611988951843e-06	\\
9641.00230823864	3.45009488460365e-06	\\
9641.98109019886	3.87842538998609e-06	\\
9642.95987215909	3.651286260454e-06	\\
9643.93865411932	3.08015290076757e-06	\\
9644.91743607954	2.9182950119366e-06	\\
9645.89621803977	3.88013039498225e-06	\\
9646.875	2.70549503244824e-06	\\
9647.85378196023	3.90067916464622e-06	\\
9648.83256392045	4.04787121040937e-06	\\
9649.81134588068	3.82231810661807e-06	\\
9650.79012784091	2.97271322258097e-06	\\
9651.76890980114	2.78272759864254e-06	\\
9652.74769176136	3.91361408821884e-06	\\
9653.72647372159	2.57228610759202e-06	\\
9654.70525568182	2.44699572074021e-06	\\
9655.68403764204	2.50336876561691e-06	\\
9656.66281960227	2.76257142172355e-06	\\
9657.6416015625	2.48179526837459e-06	\\
9658.62038352273	3.30348473333363e-06	\\
9659.59916548295	3.04473866063385e-06	\\
9660.57794744318	3.56139554771791e-06	\\
9661.55672940341	4.07843597733521e-06	\\
9662.53551136364	3.73392680324732e-06	\\
9663.51429332386	2.19229447226101e-06	\\
9664.49307528409	2.99825892411335e-06	\\
9665.47185724432	2.57530586294207e-06	\\
9666.45063920454	3.80004277887694e-06	\\
9667.42942116477	2.24752604123741e-06	\\
9668.408203125	3.41562623222865e-06	\\
9669.38698508523	2.55468898211008e-06	\\
9670.36576704545	2.35003177816742e-06	\\
9671.34454900568	3.05497668634349e-06	\\
9672.32333096591	3.35413384003001e-06	\\
9673.30211292614	3.11888372166604e-06	\\
9674.28089488636	3.68899661674434e-06	\\
9675.25967684659	3.14479733634139e-06	\\
9676.23845880682	3.95471765306047e-06	\\
9677.21724076704	3.59267581409738e-06	\\
9678.19602272727	3.03423524118623e-06	\\
9679.1748046875	3.90475327013748e-06	\\
9680.15358664773	3.01738382070053e-06	\\
9681.13236860795	3.05685443943778e-06	\\
9682.11115056818	2.87254444902707e-06	\\
9683.08993252841	3.26425662348743e-06	\\
9684.06871448864	3.35970030894665e-06	\\
9685.04749644886	2.76729380846987e-06	\\
9686.02627840909	2.60560561326318e-06	\\
9687.00506036932	3.76951382693163e-06	\\
9687.98384232954	3.18410484747839e-06	\\
9688.96262428977	3.19622508247135e-06	\\
9689.94140625	3.54406656530696e-06	\\
9690.92018821023	3.04648685492987e-06	\\
9691.89897017045	2.62201055573905e-06	\\
9692.87775213068	2.34462334733271e-06	\\
9693.85653409091	2.53441859733947e-06	\\
9694.83531605114	2.81998614823198e-06	\\
9695.81409801136	2.59400193789516e-06	\\
9696.79287997159	3.12637862548099e-06	\\
9697.77166193182	2.85030869010138e-06	\\
9698.75044389204	3.15121159540536e-06	\\
9699.72922585227	2.78699857944507e-06	\\
9700.7080078125	3.87168329663511e-06	\\
9701.68678977273	2.85386320007567e-06	\\
9702.66557173295	3.3576972146777e-06	\\
9703.64435369318	3.08156392846908e-06	\\
9704.62313565341	3.56315828870147e-06	\\
9705.60191761364	3.72614029288689e-06	\\
9706.58069957386	2.70571989107061e-06	\\
9707.55948153409	3.00239463913765e-06	\\
9708.53826349432	3.44145432256257e-06	\\
9709.51704545454	2.65944793192677e-06	\\
9710.49582741477	2.56458679484096e-06	\\
9711.474609375	3.3303995704966e-06	\\
9712.45339133523	2.78137231592049e-06	\\
9713.43217329545	3.45045087400546e-06	\\
9714.41095525568	3.46964963521605e-06	\\
9715.38973721591	2.29427175293864e-06	\\
9716.36851917614	2.69981898854297e-06	\\
9717.34730113636	2.94221972915371e-06	\\
9718.32608309659	3.73001544806083e-06	\\
9719.30486505682	4.69932658526163e-06	\\
9720.28364701704	3.66978231484694e-06	\\
9721.26242897727	2.96723687328732e-06	\\
9722.2412109375	2.92697834804081e-06	\\
9723.21999289773	2.86740231304191e-06	\\
9724.19877485795	3.65056824856238e-06	\\
9725.17755681818	3.49864694489694e-06	\\
9726.15633877841	3.40022317120256e-06	\\
9727.13512073864	3.85861644437448e-06	\\
9728.11390269886	2.7893439574722e-06	\\
9729.09268465909	2.9093149660539e-06	\\
9730.07146661932	2.85095628982678e-06	\\
9731.05024857954	2.6248275020334e-06	\\
9732.02903053977	3.48398673512055e-06	\\
9733.0078125	3.62932616147058e-06	\\
9733.98659446023	3.76107906235332e-06	\\
9734.96537642045	3.21179838066762e-06	\\
9735.94415838068	2.89548086704648e-06	\\
9736.92294034091	3.52920246561844e-06	\\
9737.90172230114	3.20206524948466e-06	\\
9738.88050426136	2.30593429888925e-06	\\
9739.85928622159	3.18833539516255e-06	\\
9740.83806818182	2.33258510103589e-06	\\
9741.81685014204	3.36401806218765e-06	\\
9742.79563210227	2.93554890265854e-06	\\
9743.7744140625	2.59621046102724e-06	\\
9744.75319602273	2.86942482821341e-06	\\
9745.73197798295	3.10672955599901e-06	\\
9746.71075994318	2.97391609773224e-06	\\
9747.68954190341	2.40506280902511e-06	\\
9748.66832386364	3.52008841895498e-06	\\
9749.64710582386	3.25824903753838e-06	\\
9750.62588778409	3.83331081933299e-06	\\
9751.60466974432	3.12942051103446e-06	\\
9752.58345170454	4.08761397865492e-06	\\
9753.56223366477	2.6758649755234e-06	\\
9754.541015625	2.80367622408203e-06	\\
9755.51979758523	2.45394135869741e-06	\\
9756.49857954545	2.98133849149566e-06	\\
9757.47736150568	3.94669068327888e-06	\\
9758.45614346591	2.03564641180625e-06	\\
9759.43492542614	3.17328143555303e-06	\\
9760.41370738636	2.38582269149341e-06	\\
9761.39248934659	2.18827741469466e-06	\\
9762.37127130682	3.12699137641828e-06	\\
9763.35005326704	3.39808569435838e-06	\\
9764.32883522727	2.8163027249347e-06	\\
9765.3076171875	3.19211419295647e-06	\\
9766.28639914773	2.53332477242513e-06	\\
9767.26518110795	2.59573837128044e-06	\\
9768.24396306818	2.1365605706677e-06	\\
9769.22274502841	2.71047691014902e-06	\\
9770.20152698864	3.04835773480564e-06	\\
9771.18030894886	3.16313550543043e-06	\\
9772.15909090909	2.91932052861172e-06	\\
9773.13787286932	3.47009874799926e-06	\\
9774.11665482954	2.42958933576066e-06	\\
9775.09543678977	3.58105313724534e-06	\\
9776.07421875	2.89357017918107e-06	\\
9777.05300071023	2.56469627588548e-06	\\
9778.03178267045	2.75657542862718e-06	\\
9779.01056463068	2.40974949479344e-06	\\
9779.98934659091	1.74579779012925e-06	\\
9780.96812855114	3.06282525379861e-06	\\
9781.94691051136	3.27668227486702e-06	\\
9782.92569247159	3.45405856582388e-06	\\
9783.90447443182	3.4429546056804e-06	\\
9784.88325639204	3.68146748856411e-06	\\
9785.86203835227	3.20039829344057e-06	\\
9786.8408203125	2.75578990720649e-06	\\
9787.81960227273	2.67935214751936e-06	\\
9788.79838423295	3.23261335062803e-06	\\
9789.77716619318	3.47295038456473e-06	\\
9790.75594815341	2.76767968596101e-06	\\
9791.73473011364	2.26754126502173e-06	\\
9792.71351207386	3.06225378776651e-06	\\
9793.69229403409	3.61432736721851e-06	\\
9794.67107599432	2.51176375098109e-06	\\
9795.64985795454	2.65401054957692e-06	\\
9796.62863991477	2.62365411311334e-06	\\
9797.607421875	3.55499839721868e-06	\\
9798.58620383523	3.36520174543066e-06	\\
9799.56498579545	3.02856990268593e-06	\\
9800.54376775568	2.21736670650404e-06	\\
9801.52254971591	2.51366810017348e-06	\\
9802.50133167614	2.56648771914927e-06	\\
9803.48011363636	2.90269149979534e-06	\\
9804.45889559659	3.38557449043859e-06	\\
9805.43767755682	2.666513820649e-06	\\
9806.41645951704	3.43674978052606e-06	\\
9807.39524147727	3.65913066702103e-06	\\
9808.3740234375	2.95526691147262e-06	\\
9809.35280539773	2.38969418439205e-06	\\
9810.33158735795	3.14935492812077e-06	\\
9811.31036931818	3.55849028919845e-06	\\
9812.28915127841	3.33775363191987e-06	\\
9813.26793323864	2.98136992709899e-06	\\
9814.24671519886	2.88333604621216e-06	\\
9815.22549715909	3.42738327424551e-06	\\
9816.20427911932	3.27158104751861e-06	\\
9817.18306107954	3.05055017998517e-06	\\
9818.16184303977	2.98592174531417e-06	\\
9819.140625	1.72255227702505e-06	\\
9820.11940696023	2.56880630916838e-06	\\
9821.09818892045	3.25070889843274e-06	\\
9822.07697088068	2.5628909754316e-06	\\
9823.05575284091	2.73952336169553e-06	\\
9824.03453480114	3.64324873693837e-06	\\
9825.01331676136	3.18091176267496e-06	\\
9825.99209872159	3.83160273108356e-06	\\
9826.97088068182	2.74070238517121e-06	\\
9827.94966264204	3.15209565622503e-06	\\
9828.92844460227	3.33118017099946e-06	\\
9829.9072265625	3.13014388086702e-06	\\
9830.88600852273	3.3032831455564e-06	\\
9831.86479048295	3.45711334824908e-06	\\
9832.84357244318	3.05701977501812e-06	\\
9833.82235440341	2.51472350335913e-06	\\
9834.80113636364	3.46631067174668e-06	\\
9835.77991832386	2.74126410480431e-06	\\
9836.75870028409	3.42604901551655e-06	\\
9837.73748224432	2.80044659731798e-06	\\
9838.71626420454	3.13727240613111e-06	\\
9839.69504616477	2.85225861073642e-06	\\
9840.673828125	2.83644274652374e-06	\\
9841.65261008523	2.86685137595989e-06	\\
9842.63139204545	2.45225735747442e-06	\\
9843.61017400568	2.67050996023939e-06	\\
9844.58895596591	3.16662158497722e-06	\\
9845.56773792614	3.23492796034129e-06	\\
9846.54651988636	4.19742507656718e-06	\\
9847.52530184659	3.4016976416554e-06	\\
9848.50408380682	3.02130673913117e-06	\\
9849.48286576704	2.43377229889112e-06	\\
9850.46164772727	2.62951694175497e-06	\\
9851.4404296875	2.86393182914036e-06	\\
9852.41921164773	2.40925237335061e-06	\\
9853.39799360795	2.39993957586474e-06	\\
9854.37677556818	3.1441254590843e-06	\\
9855.35555752841	2.96298149006537e-06	\\
9856.33433948864	2.93744678604207e-06	\\
9857.31312144886	1.86937951715461e-06	\\
9858.29190340909	2.82608929705857e-06	\\
9859.27068536932	2.99596851706197e-06	\\
9860.24946732954	2.30835955013129e-06	\\
9861.22824928977	3.40109213512336e-06	\\
9862.20703125	3.53171120546296e-06	\\
9863.18581321023	2.67214725244911e-06	\\
9864.16459517045	3.38094295817555e-06	\\
9865.14337713068	3.92281312479059e-06	\\
9866.12215909091	3.18550231303193e-06	\\
9867.10094105114	2.45732670296062e-06	\\
9868.07972301136	2.68236330147428e-06	\\
9869.05850497159	2.54684886370402e-06	\\
9870.03728693182	2.68030989130199e-06	\\
9871.01606889204	3.44916350780556e-06	\\
9871.99485085227	2.81939547248323e-06	\\
9872.9736328125	2.86492637968834e-06	\\
9873.95241477273	2.57996062860507e-06	\\
9874.93119673295	3.03895880681162e-06	\\
9875.90997869318	2.68497330215358e-06	\\
9876.88876065341	3.10182725859633e-06	\\
9877.86754261364	3.35172642873408e-06	\\
9878.84632457386	2.58991640500877e-06	\\
9879.82510653409	2.48876779676326e-06	\\
9880.80388849432	2.65090261156666e-06	\\
9881.78267045454	2.39471692919413e-06	\\
9882.76145241477	1.75953028733261e-06	\\
9883.740234375	2.76401757683157e-06	\\
9884.71901633523	2.07289854946289e-06	\\
9885.69779829545	3.28563368679534e-06	\\
9886.67658025568	2.97303770074169e-06	\\
9887.65536221591	2.67518786316963e-06	\\
9888.63414417614	2.53553454903247e-06	\\
9889.61292613636	3.43218883503504e-06	\\
9890.59170809659	2.62022473632063e-06	\\
9891.57049005682	4.12252741751594e-06	\\
9892.54927201704	3.24465589858941e-06	\\
9893.52805397727	3.03424957275248e-06	\\
9894.5068359375	4.04947486769978e-06	\\
9895.48561789773	3.34943254291621e-06	\\
9896.46439985795	2.8372768530669e-06	\\
9897.44318181818	3.67552280123812e-06	\\
9898.42196377841	2.96563751417427e-06	\\
9899.40074573864	2.87709288812207e-06	\\
9900.37952769886	2.76288479786596e-06	\\
9901.35830965909	3.33001375007492e-06	\\
9902.33709161932	3.35676282737161e-06	\\
9903.31587357954	3.36668316907564e-06	\\
9904.29465553977	2.47459045571051e-06	\\
9905.2734375	3.51717549190133e-06	\\
9906.25221946023	2.60224738430626e-06	\\
9907.23100142045	2.96433872796851e-06	\\
9908.20978338068	2.45049675324952e-06	\\
9909.18856534091	4.00439385892615e-06	\\
9910.16734730114	3.32119382641368e-06	\\
9911.14612926136	3.30589638744867e-06	\\
9912.12491122159	3.38703974725743e-06	\\
9913.10369318182	2.56907576195779e-06	\\
9914.08247514204	3.41917485769125e-06	\\
9915.06125710227	3.06435733954816e-06	\\
9916.0400390625	2.88751518846321e-06	\\
9917.01882102273	2.6719068397138e-06	\\
9917.99760298295	2.80223064203542e-06	\\
9918.97638494318	3.18314753265995e-06	\\
9919.95516690341	3.25384480786614e-06	\\
9920.93394886364	2.70325411287237e-06	\\
9921.91273082386	2.81047307077155e-06	\\
9922.89151278409	2.59178675521705e-06	\\
9923.87029474432	2.02272197056687e-06	\\
9924.84907670454	3.30902897711535e-06	\\
9925.82785866477	2.73765279757632e-06	\\
9926.806640625	3.04880880779369e-06	\\
9927.78542258523	3.31090539336275e-06	\\
9928.76420454545	2.02960710406292e-06	\\
9929.74298650568	2.52749134885285e-06	\\
9930.72176846591	2.56670829688898e-06	\\
9931.70055042614	2.88882208084087e-06	\\
9932.67933238636	3.02243324507901e-06	\\
9933.65811434659	2.6524570677771e-06	\\
9934.63689630682	2.62371513828321e-06	\\
9935.61567826704	3.18651512374696e-06	\\
9936.59446022727	3.0046674266144e-06	\\
9937.5732421875	3.4559505400846e-06	\\
9938.55202414773	2.35627610790995e-06	\\
9939.53080610795	2.43654535531496e-06	\\
9940.50958806818	3.1266298509823e-06	\\
9941.48837002841	2.39479845300508e-06	\\
9942.46715198864	3.22249142494352e-06	\\
9943.44593394886	2.79968941190061e-06	\\
9944.42471590909	1.95050872005905e-06	\\
9945.40349786932	2.72808811656243e-06	\\
9946.38227982954	3.02531235465939e-06	\\
9947.36106178977	2.68422222252965e-06	\\
9948.33984375	2.56790495944956e-06	\\
9949.31862571023	2.22067114458657e-06	\\
9950.29740767045	2.09650842226432e-06	\\
9951.27618963068	2.37367703879817e-06	\\
9952.25497159091	2.4587505274784e-06	\\
9953.23375355114	2.41528609939469e-06	\\
9954.21253551136	2.32293581013305e-06	\\
9955.19131747159	2.6932151102175e-06	\\
9956.17009943182	2.07638500495093e-06	\\
9957.14888139204	1.43324503326665e-06	\\
9958.12766335227	2.45749564919018e-06	\\
9959.1064453125	2.63144444453386e-06	\\
9960.08522727273	3.24665175348792e-06	\\
9961.06400923295	2.86092338394215e-06	\\
9962.04279119318	2.46802826037257e-06	\\
9963.02157315341	2.98892215059984e-06	\\
9964.00035511364	2.54617162049855e-06	\\
9964.97913707386	1.9818172560796e-06	\\
9965.95791903409	2.31231631338631e-06	\\
9966.93670099432	2.18377049083302e-06	\\
9967.91548295454	2.61632860190979e-06	\\
9968.89426491477	2.41968520261556e-06	\\
9969.873046875	2.5479036968077e-06	\\
9970.85182883523	1.37322681496261e-06	\\
9971.83061079545	1.54312035449903e-06	\\
9972.80939275568	2.50154970640904e-06	\\
9973.78817471591	2.88358320035705e-06	\\
9974.76695667614	2.42869452507859e-06	\\
9975.74573863636	2.63193667503996e-06	\\
9976.72452059659	2.56837061383589e-06	\\
9977.70330255682	2.77139917979205e-06	\\
9978.68208451704	1.43720040954144e-06	\\
9979.66086647727	2.46483699084904e-06	\\
9980.6396484375	2.38839326629676e-06	\\
9981.61843039773	3.03286506398686e-06	\\
9982.59721235795	2.2595096093985e-06	\\
9983.57599431818	1.95959349581596e-06	\\
9984.55477627841	3.13194625534649e-06	\\
9985.53355823864	2.43691903327748e-06	\\
9986.51234019886	3.02114592304271e-06	\\
9987.49112215909	2.62036288683511e-06	\\
9988.46990411932	2.0637896500537e-06	\\
9989.44868607954	2.1016858584882e-06	\\
9990.42746803977	1.97748800855244e-06	\\
9991.40625	2.04424971010734e-06	\\
9992.38503196023	2.42639854450536e-06	\\
9993.36381392045	2.16270917060561e-06	\\
9994.34259588068	2.40992089111654e-06	\\
9995.32137784091	1.60565055055078e-06	\\
9996.30015980114	2.46060519525825e-06	\\
9997.27894176136	2.13851159886855e-06	\\
9998.25772372159	2.56619591642114e-06	\\
9999.23650568182	2.13967485427828e-06	\\
10000.215287642	2.47196692098044e-06	\\
10001.1940696023	1.93336366122183e-06	\\
10002.1728515625	3.3770920477197e-06	\\
10003.1516335227	2.19508530833088e-06	\\
10004.130415483	2.13823827841272e-06	\\
10005.1091974432	2.66770230259368e-06	\\
10006.0879794034	2.33986455986189e-06	\\
10007.0667613636	1.84264445580353e-06	\\
10008.0455433239	2.7802607601345e-06	\\
10009.0243252841	2.84069063289107e-06	\\
10010.0031072443	2.41004175270349e-06	\\
10010.9818892045	2.34224627048861e-06	\\
10011.9606711648	2.61807146272724e-06	\\
10012.939453125	2.77400278611069e-06	\\
10013.9182350852	2.01726618987032e-06	\\
10014.8970170455	3.02689559555739e-06	\\
10015.8757990057	2.09182214281679e-06	\\
10016.8545809659	2.70240548886919e-06	\\
10017.8333629261	2.49791394881612e-06	\\
10018.8121448864	1.15080730732168e-06	\\
10019.7909268466	2.28808030204406e-06	\\
10020.7697088068	2.25707080154917e-06	\\
10021.748490767	1.93685091845707e-06	\\
10022.7272727273	1.9659795190725e-06	\\
10023.7060546875	2.39129260260541e-06	\\
10024.6848366477	2.22118154255679e-06	\\
10025.663618608	1.81254183942795e-06	\\
10026.6424005682	1.51485894693966e-06	\\
10027.6211825284	2.19162187058022e-06	\\
10028.5999644886	1.5110462657492e-06	\\
10029.5787464489	1.79684988419574e-06	\\
10030.5575284091	2.24123075554053e-06	\\
10031.5363103693	2.26029221333974e-06	\\
10032.5150923295	2.2296741829779e-06	\\
10033.4938742898	2.50369125399667e-06	\\
10034.47265625	2.4313255449576e-06	\\
10035.4514382102	2.70567483646493e-06	\\
10036.4302201705	2.28844296290912e-06	\\
10037.4090021307	2.43106524164159e-06	\\
10038.3877840909	2.103308346403e-06	\\
10039.3665660511	2.04689675637194e-06	\\
10040.3453480114	2.25818593202593e-06	\\
10041.3241299716	2.59344567163593e-06	\\
10042.3029119318	1.50094138209941e-06	\\
10043.281693892	1.31392498584515e-06	\\
10044.2604758523	1.97210371189792e-06	\\
10045.2392578125	1.55432187573409e-06	\\
10046.2180397727	1.68174752038896e-06	\\
10047.196821733	2.47184537673521e-06	\\
10048.1756036932	1.69028796429751e-06	\\
10049.1543856534	1.38897053930558e-06	\\
10050.1331676136	1.63891795389273e-06	\\
10051.1119495739	2.24745926112757e-06	\\
10052.0907315341	1.46079151732559e-06	\\
10053.0695134943	2.37247597036506e-06	\\
10054.0482954545	2.76651193596734e-06	\\
10055.0270774148	1.66746803008199e-06	\\
10056.005859375	2.21687568494121e-06	\\
10056.9846413352	1.71770047138869e-06	\\
10057.9634232955	1.68914811447247e-06	\\
10058.9422052557	1.56297986126993e-06	\\
10059.9209872159	2.31276171035064e-06	\\
10060.8997691761	2.501496162101e-06	\\
10061.8785511364	1.93890173659124e-06	\\
10062.8573330966	1.50605128970887e-06	\\
10063.8361150568	1.68464547840305e-06	\\
10064.814897017	1.94586388496108e-06	\\
10065.7936789773	1.41186687649839e-06	\\
10066.7724609375	2.26902750511224e-06	\\
10067.7512428977	2.09816449384367e-06	\\
10068.730024858	2.18554696329388e-06	\\
10069.7088068182	2.1469957428506e-06	\\
10070.6875887784	1.41190654792223e-06	\\
10071.6663707386	1.96906786045056e-06	\\
10072.6451526989	8.43287775617854e-07	\\
10073.6239346591	1.98529080331392e-06	\\
10074.6027166193	1.9849423912467e-06	\\
10075.5814985795	1.8175199845789e-06	\\
10076.5602805398	1.82039439008807e-06	\\
10077.5390625	2.14805298550294e-06	\\
10078.5178444602	2.07869624153411e-06	\\
10079.4966264205	1.57734439075427e-06	\\
10080.4754083807	1.96842670851478e-06	\\
10081.4541903409	1.90740708892435e-06	\\
10082.4329723011	1.58622947365599e-06	\\
10083.4117542614	1.71901571936919e-06	\\
10084.3905362216	1.72118745991447e-06	\\
10085.3693181818	9.15713166630672e-07	\\
10086.348100142	1.93732468631128e-06	\\
10087.3268821023	2.00088182333659e-06	\\
10088.3056640625	1.89895010072753e-06	\\
10089.2844460227	1.88783748785917e-06	\\
10090.263227983	1.68393673312873e-06	\\
10091.2420099432	2.25484274389013e-06	\\
10092.2207919034	1.83980759183877e-06	\\
10093.1995738636	1.66934612970067e-06	\\
10094.1783558239	2.29716404707202e-06	\\
10095.1571377841	1.86426684784634e-06	\\
10096.1359197443	2.33499467806052e-06	\\
10097.1147017045	1.82672396856251e-06	\\
10098.0934836648	2.59478536054512e-06	\\
10099.072265625	1.96867178445553e-06	\\
10100.0510475852	1.92915141297361e-06	\\
10101.0298295455	1.78989762410383e-06	\\
10102.0086115057	2.05965982229192e-06	\\
10102.9873934659	1.97400356037986e-06	\\
10103.9661754261	1.95133104446985e-06	\\
10104.9449573864	1.77343651768415e-06	\\
10105.9237393466	1.63481729957461e-06	\\
10106.9025213068	2.12507243508872e-06	\\
10107.881303267	2.1071991357617e-06	\\
10108.8600852273	2.50066335286689e-06	\\
10109.8388671875	1.69326697198698e-06	\\
10110.8176491477	2.29213946330303e-06	\\
10111.796431108	2.16696596016044e-06	\\
10112.7752130682	1.81899581645945e-06	\\
10113.7539950284	1.81827264759204e-06	\\
10114.7327769886	1.96397778473582e-06	\\
10115.7115589489	2.14650895112384e-06	\\
10116.6903409091	1.61077652784843e-06	\\
10117.6691228693	1.70861676989465e-06	\\
10118.6479048295	2.26546147311843e-06	\\
10119.6266867898	1.99894324793625e-06	\\
10120.60546875	1.72651444407429e-06	\\
10121.5842507102	2.32067273086919e-06	\\
10122.5630326705	2.02081297277709e-06	\\
10123.5418146307	1.7903909290154e-06	\\
10124.5205965909	1.6619509115675e-06	\\
10125.4993785511	1.32100902528561e-06	\\
10126.4781605114	2.30791360805085e-06	\\
10127.4569424716	1.55887539039687e-06	\\
10128.4357244318	1.22890803151218e-06	\\
10129.414506392	2.14132608202344e-06	\\
10130.3932883523	1.29595943280113e-06	\\
10131.3720703125	1.95731245943036e-06	\\
10132.3508522727	2.45041004839601e-06	\\
10133.329634233	2.28896743196643e-06	\\
10134.3084161932	2.81752019459734e-06	\\
10135.2871981534	2.91148542392545e-06	\\
10136.2659801136	1.75878764868041e-06	\\
10137.2447620739	2.01250525696861e-06	\\
10138.2235440341	1.33589525779128e-06	\\
10139.2023259943	1.34238410789601e-06	\\
10140.1811079545	1.50960798719909e-06	\\
10141.1598899148	1.29707774604514e-06	\\
10142.138671875	1.36821237450406e-06	\\
10143.1174538352	1.41478227970936e-06	\\
10144.0962357955	1.15165377831651e-06	\\
10145.0750177557	2.09221990611463e-06	\\
10146.0537997159	1.92325293073136e-06	\\
10147.0325816761	1.64152665276087e-06	\\
10148.0113636364	2.21168795532408e-06	\\
10148.9901455966	1.92307475305495e-06	\\
10149.9689275568	1.72761934243673e-06	\\
10150.947709517	2.32829681279706e-06	\\
10151.9264914773	1.81952490656783e-06	\\
10152.9052734375	1.91482794606022e-06	\\
10153.8840553977	1.98638121105334e-06	\\
10154.862837358	2.4698184171821e-06	\\
10155.8416193182	1.20393838901734e-06	\\
10156.8204012784	2.04490448307847e-06	\\
10157.7991832386	2.01674607411119e-06	\\
10158.7779651989	2.34594182048706e-06	\\
10159.7567471591	2.45214653550708e-06	\\
10160.7355291193	1.95974895821371e-06	\\
10161.7143110795	2.49216799946897e-06	\\
10162.6930930398	1.85693740607407e-06	\\
10163.671875	1.85627691775359e-06	\\
10164.6506569602	2.58699545896512e-06	\\
10165.6294389205	1.54062277973108e-06	\\
10166.6082208807	1.84128169816367e-06	\\
10167.5870028409	2.23715843404199e-06	\\
10168.5657848011	7.78735084290895e-07	\\
10169.5445667614	2.23341955815951e-06	\\
10170.5233487216	1.92711084867354e-06	\\
10171.5021306818	1.37326874081715e-06	\\
10172.480912642	2.0192582819232e-06	\\
10173.4596946023	2.14998400054823e-06	\\
10174.4384765625	1.27319667724923e-06	\\
10175.4172585227	2.33360745838286e-06	\\
10176.396040483	1.73410025522907e-06	\\
10177.3748224432	2.29956307651601e-06	\\
10178.3536044034	1.72866877948653e-06	\\
10179.3323863636	1.81259715773617e-06	\\
10180.3111683239	1.48417227330905e-06	\\
10181.2899502841	1.80247099219049e-06	\\
10182.2687322443	1.39976017670943e-06	\\
10183.2475142045	8.41097468348862e-07	\\
10184.2262961648	1.97615970716247e-06	\\
10185.205078125	1.66150072523535e-06	\\
10186.1838600852	1.80582502347982e-06	\\
10187.1626420455	1.86354568642128e-06	\\
10188.1414240057	2.51921612838506e-06	\\
10189.1202059659	2.04011925777003e-06	\\
10190.0989879261	1.76354364779884e-06	\\
10191.0777698864	1.84964176006202e-06	\\
10192.0565518466	2.28606617728866e-06	\\
10193.0353338068	1.66335333772497e-06	\\
10194.014115767	1.72612481761162e-06	\\
10194.9928977273	1.92177612513862e-06	\\
10195.9716796875	1.53864248617473e-06	\\
10196.9504616477	1.63557279696279e-06	\\
10197.929243608	1.23128508049931e-06	\\
10198.9080255682	1.78461998773043e-06	\\
10199.8868075284	1.51199571617165e-06	\\
10200.8655894886	2.28385215646203e-06	\\
10201.8443714489	1.85451223849816e-06	\\
10202.8231534091	1.7565124587856e-06	\\
10203.8019353693	1.67102027887893e-06	\\
10204.7807173295	1.68331141825129e-06	\\
10205.7594992898	2.18234902871545e-06	\\
10206.73828125	1.02854033911937e-06	\\
10207.7170632102	1.84804570049087e-06	\\
10208.6958451705	2.11379902829619e-06	\\
10209.6746271307	1.56889328159802e-06	\\
10210.6534090909	1.70659272602427e-06	\\
10211.6321910511	1.72724436310112e-06	\\
10212.6109730114	2.29312020753979e-06	\\
10213.5897549716	2.34836246043039e-06	\\
10214.5685369318	2.07526037820571e-06	\\
10215.547318892	1.63915457711768e-06	\\
10216.5261008523	2.26031052364427e-06	\\
10217.5048828125	2.58433504237565e-06	\\
10218.4836647727	2.01126866084897e-06	\\
10219.462446733	2.22300430516815e-06	\\
10220.4412286932	1.93825665398911e-06	\\
10221.4200106534	1.89129659631529e-06	\\
10222.3987926136	1.73438588442216e-06	\\
10223.3775745739	2.11152777057365e-06	\\
10224.3563565341	1.85301444797343e-06	\\
10225.3351384943	2.35820298056017e-06	\\
10226.3139204545	1.36299721958888e-06	\\
10227.2927024148	1.85652778569955e-06	\\
10228.271484375	2.03594736316914e-06	\\
10229.2502663352	2.04138473019821e-06	\\
10230.2290482955	1.58665316934805e-06	\\
10231.2078302557	1.87054596786473e-06	\\
10232.1866122159	2.30013068921341e-06	\\
10233.1653941761	1.26415328316555e-06	\\
10234.1441761364	2.07366487046309e-06	\\
10235.1229580966	2.27838117378514e-06	\\
10236.1017400568	1.89004154997862e-06	\\
10237.080522017	2.02902820860798e-06	\\
10238.0593039773	1.66709760143298e-06	\\
10239.0380859375	2.32771646938912e-06	\\
10240.0168678977	1.90357489104002e-06	\\
10240.995649858	1.98823938067438e-06	\\
10241.9744318182	2.08197212805658e-06	\\
10242.9532137784	1.3136978520724e-06	\\
10243.9319957386	2.01751032524318e-06	\\
10244.9107776989	1.99031432993743e-06	\\
10245.8895596591	1.55465638481799e-06	\\
10246.8683416193	8.25998763249426e-07	\\
10247.8471235795	2.01915217175673e-06	\\
10248.8259055398	9.55429119800128e-07	\\
10249.8046875	1.39541611420097e-06	\\
10250.7834694602	2.08442788504247e-06	\\
10251.7622514205	1.85481797275604e-06	\\
10252.7410333807	2.10546589931067e-06	\\
10253.7198153409	2.20622626115443e-06	\\
10254.6985973011	2.52866665945742e-06	\\
10255.6773792614	1.47808809006571e-06	\\
10256.6561612216	2.55613052094937e-06	\\
10257.6349431818	1.24014354585776e-06	\\
10258.613725142	1.66239200999288e-06	\\
10259.5925071023	1.66951515134229e-06	\\
10260.5712890625	1.89155748723782e-06	\\
10261.5500710227	1.66479762191958e-06	\\
10262.528852983	2.38270285093102e-06	\\
10263.5076349432	1.703084934626e-06	\\
10264.4864169034	1.54940784842982e-06	\\
10265.4651988636	2.56318252937736e-06	\\
10266.4439808239	1.06564524882717e-06	\\
10267.4227627841	1.89161112932676e-06	\\
10268.4015447443	1.93593400318131e-06	\\
10269.3803267045	1.7085063248879e-06	\\
10270.3591086648	1.81829325848833e-06	\\
10271.337890625	1.75484296562711e-06	\\
10272.3166725852	1.73252217121408e-06	\\
10273.2954545455	1.91569566921105e-06	\\
10274.2742365057	1.82311522009118e-06	\\
10275.2530184659	1.64113382143156e-06	\\
10276.2318004261	2.47039676189452e-06	\\
10277.2105823864	1.55836278594057e-06	\\
10278.1893643466	1.79011044895909e-06	\\
10279.1681463068	2.05689371489325e-06	\\
10280.146928267	2.14330146238264e-06	\\
10281.1257102273	1.76100069747544e-06	\\
10282.1044921875	1.96072114906278e-06	\\
10283.0832741477	1.72995425061329e-06	\\
10284.062056108	1.4035937242189e-06	\\
10285.0408380682	1.82283505526796e-06	\\
10286.0196200284	2.29693202755815e-06	\\
10286.9984019886	1.91107426718195e-06	\\
10287.9771839489	1.9876727043776e-06	\\
10288.9559659091	1.75421068692166e-06	\\
10289.9347478693	1.97614025281659e-06	\\
10290.9135298295	2.23905860570596e-06	\\
10291.8923117898	2.77677882286673e-06	\\
10292.87109375	2.04519195380728e-06	\\
10293.8498757102	2.43276683968884e-06	\\
10294.8286576705	1.65357141528115e-06	\\
10295.8074396307	1.81754359677246e-06	\\
10296.7862215909	1.90813137089084e-06	\\
10297.7650035511	1.5225052455562e-06	\\
10298.7437855114	1.99796976995394e-06	\\
10299.7225674716	2.02644549317247e-06	\\
10300.7013494318	1.96405934490764e-06	\\
10301.680131392	1.94636664522068e-06	\\
10302.6589133523	2.09527753373258e-06	\\
10303.6376953125	1.92005233307997e-06	\\
10304.6164772727	1.94253039521926e-06	\\
10305.595259233	1.70437727587565e-06	\\
10306.5740411932	2.27315624471268e-06	\\
10307.5528231534	1.9248716599569e-06	\\
10308.5316051136	1.928621682019e-06	\\
10309.5103870739	1.68253719026782e-06	\\
10310.4891690341	1.82711767214395e-06	\\
10311.4679509943	1.99917947247535e-06	\\
10312.4467329545	1.51508957757401e-06	\\
10313.4255149148	1.85430101873421e-06	\\
10314.404296875	1.75944295892217e-06	\\
10315.3830788352	1.62957970981376e-06	\\
10316.3618607955	2.51947935757138e-06	\\
10317.3406427557	1.52767352892546e-06	\\
10318.3194247159	1.79017506224195e-06	\\
10319.2982066761	1.80748978159559e-06	\\
10320.2769886364	1.94143249081654e-06	\\
10321.2557705966	2.00125688763511e-06	\\
10322.2345525568	2.17923600038963e-06	\\
10323.213334517	2.2951141185652e-06	\\
10324.1921164773	1.68374263880182e-06	\\
10325.1708984375	2.59833225169795e-06	\\
10326.1496803977	2.05157499701558e-06	\\
10327.128462358	1.70382967408191e-06	\\
10328.1072443182	1.97686625726545e-06	\\
10329.0860262784	2.21392618318129e-06	\\
10330.0648082386	1.49896199892016e-06	\\
10331.0435901989	2.27586026636388e-06	\\
10332.0223721591	1.78166470187337e-06	\\
10333.0011541193	1.64195356503353e-06	\\
10333.9799360795	1.71158111036115e-06	\\
10334.9587180398	1.94289020100174e-06	\\
10335.9375	1.52923552970648e-06	\\
10336.9162819602	2.10517278010249e-06	\\
10337.8950639205	1.66623519545092e-06	\\
10338.8738458807	1.84264368825974e-06	\\
10339.8526278409	1.57354134182175e-06	\\
10340.8314098011	1.27757914569393e-06	\\
10341.8101917614	2.27131083060394e-06	\\
10342.7889737216	2.11527458655091e-06	\\
10343.7677556818	1.73502264377882e-06	\\
10344.746537642	1.75930292778029e-06	\\
10345.7253196023	2.37536653450773e-06	\\
10346.7041015625	2.48089937904529e-06	\\
10347.6828835227	1.46677080474936e-06	\\
10348.661665483	1.56904607135669e-06	\\
10349.6404474432	1.78906537951947e-06	\\
10350.6192294034	1.9983199513911e-06	\\
10351.5980113636	1.76814889275941e-06	\\
10352.5767933239	1.96761880183494e-06	\\
10353.5555752841	1.77036879553025e-06	\\
10354.5343572443	1.99938900371112e-06	\\
10355.5131392045	1.72245829029305e-06	\\
10356.4919211648	1.84781969467031e-06	\\
10357.470703125	1.92107139491533e-06	\\
10358.4494850852	1.38064850555395e-06	\\
10359.4282670455	1.82291368978908e-06	\\
10360.4070490057	2.41942862293146e-06	\\
10361.3858309659	1.803002478628e-06	\\
10362.3646129261	2.33766774385575e-06	\\
10363.3433948864	1.82025060536747e-06	\\
10364.3221768466	2.20435657283867e-06	\\
10365.3009588068	2.04482045444518e-06	\\
10366.279740767	1.39091928404822e-06	\\
10367.2585227273	1.58754858768246e-06	\\
10368.2373046875	2.03714346941514e-06	\\
10369.2160866477	1.86451797880794e-06	\\
10370.194868608	2.20379210400479e-06	\\
10371.1736505682	2.03724468041465e-06	\\
10372.1524325284	2.12296748976549e-06	\\
10373.1312144886	1.99886104734561e-06	\\
10374.1099964489	1.8802432519703e-06	\\
10375.0887784091	2.10181906722066e-06	\\
10376.0675603693	1.88389533993972e-06	\\
10377.0463423295	1.95022072570821e-06	\\
10378.0251242898	1.76154592213975e-06	\\
10379.00390625	1.61095371591237e-06	\\
10379.9826882102	1.88927080029905e-06	\\
10380.9614701705	1.86762654335031e-06	\\
10381.9402521307	1.79599472763622e-06	\\
10382.9190340909	2.15329135710369e-06	\\
10383.8978160511	1.82316552308046e-06	\\
10384.8765980114	2.18733436753311e-06	\\
10385.8553799716	1.66494001635356e-06	\\
10386.8341619318	1.22213613505095e-06	\\
10387.812943892	2.91213044357128e-06	\\
10388.7917258523	2.07328811762132e-06	\\
10389.7705078125	2.14069106622479e-06	\\
10390.7492897727	1.74400277322347e-06	\\
10391.728071733	1.71344359920891e-06	\\
10392.7068536932	2.05491850534977e-06	\\
10393.6856356534	2.3368886203502e-06	\\
10394.6644176136	1.92965753490477e-06	\\
10395.6431995739	2.02600872231517e-06	\\
10396.6219815341	2.12439513036623e-06	\\
10397.6007634943	1.8470893127294e-06	\\
10398.5795454545	1.9412879256614e-06	\\
10399.5583274148	2.19805401836381e-06	\\
10400.537109375	1.58946917865114e-06	\\
10401.5158913352	1.60125291743462e-06	\\
10402.4946732955	2.43119679076184e-06	\\
10403.4734552557	2.34674740869461e-06	\\
10404.4522372159	1.88502290376284e-06	\\
10405.4310191761	1.57129077071015e-06	\\
10406.4098011364	2.04933431022064e-06	\\
10407.3885830966	2.29396977876003e-06	\\
10408.3673650568	1.45893871527753e-06	\\
10409.346147017	2.27722900515775e-06	\\
10410.3249289773	1.37604177245095e-06	\\
10411.3037109375	2.37819788627662e-06	\\
10412.2824928977	2.09333752566575e-06	\\
10413.261274858	1.76782517185852e-06	\\
10414.2400568182	2.47427796430381e-06	\\
10415.2188387784	1.81860405341296e-06	\\
10416.1976207386	2.2774767991873e-06	\\
10417.1764026989	2.03870860857843e-06	\\
10418.1551846591	1.95312519471257e-06	\\
10419.1339666193	1.58954047070267e-06	\\
10420.1127485795	2.37752834647565e-06	\\
10421.0915305398	2.1582262285267e-06	\\
10422.0703125	2.20568231122686e-06	\\
10423.0490944602	1.90248044401349e-06	\\
10424.0278764205	2.17089043601485e-06	\\
10425.0066583807	1.89914166182355e-06	\\
10425.9854403409	1.90008038970703e-06	\\
10426.9642223011	1.99459753900232e-06	\\
10427.9430042614	2.17997338741191e-06	\\
10428.9217862216	1.86394204980011e-06	\\
10429.9005681818	2.0706490637826e-06	\\
10430.879350142	1.66934441379518e-06	\\
10431.8581321023	2.01609018014906e-06	\\
10432.8369140625	1.6241775637427e-06	\\
10433.8156960227	1.55629558460762e-06	\\
10434.794477983	1.53687855259184e-06	\\
10435.7732599432	1.39796155747328e-06	\\
10436.7520419034	1.64658982985393e-06	\\
10437.7308238636	1.87074018006343e-06	\\
10438.7096058239	1.93451740011059e-06	\\
10439.6883877841	2.06485033389644e-06	\\
10440.6671697443	1.59382291829971e-06	\\
10441.6459517045	2.71626639648261e-06	\\
10442.6247336648	2.12926770642553e-06	\\
10443.603515625	2.04587828352794e-06	\\
10444.5822975852	2.00872210630403e-06	\\
10445.5610795455	1.6761686897998e-06	\\
10446.5398615057	1.295936828462e-06	\\
10447.5186434659	1.9648980946582e-06	\\
10448.4974254261	1.61749219950834e-06	\\
10449.4762073864	1.75485419171595e-06	\\
10450.4549893466	2.29946075601831e-06	\\
10451.4337713068	1.93702240599655e-06	\\
10452.412553267	2.09644193934759e-06	\\
10453.3913352273	1.9771414483537e-06	\\
10454.3701171875	2.11940271044818e-06	\\
10455.3488991477	2.14264510530933e-06	\\
10456.327681108	2.21657509505075e-06	\\
10457.3064630682	1.83945490693255e-06	\\
10458.2852450284	1.84199829004792e-06	\\
10459.2640269886	1.67806984363951e-06	\\
10460.2428089489	1.68990866628866e-06	\\
10461.2215909091	2.31857855997642e-06	\\
10462.2003728693	2.32649404832251e-06	\\
10463.1791548295	1.48619549125721e-06	\\
10464.1579367898	2.01106067982199e-06	\\
10465.13671875	2.26593446982134e-06	\\
10466.1155007102	2.18737727244655e-06	\\
10467.0942826705	1.75960001751464e-06	\\
10468.0730646307	1.99985772740872e-06	\\
10469.0518465909	1.89126818047171e-06	\\
10470.0306285511	1.72145756952633e-06	\\
10471.0094105114	1.69926125555523e-06	\\
10471.9881924716	1.79300272099752e-06	\\
10472.9669744318	1.9845099047221e-06	\\
10473.945756392	2.11948796926615e-06	\\
10474.9245383523	2.13151777759476e-06	\\
10475.9033203125	1.9077567159746e-06	\\
10476.8821022727	2.41271188441388e-06	\\
10477.860884233	2.13157061002837e-06	\\
10478.8396661932	1.99690379224035e-06	\\
10479.8184481534	1.71847988761578e-06	\\
10480.7972301136	1.86820138560678e-06	\\
10481.7760120739	1.82400454236073e-06	\\
10482.7547940341	1.87682759803976e-06	\\
10483.7335759943	1.72999797190455e-06	\\
10484.7123579545	1.68368110309474e-06	\\
10485.6911399148	2.13449104219217e-06	\\
10486.669921875	2.3908916666765e-06	\\
10487.6487038352	1.80034254667223e-06	\\
10488.6274857955	1.79225035302076e-06	\\
10489.6062677557	1.94982897623267e-06	\\
10490.5850497159	2.36379676318765e-06	\\
10491.5638316761	1.79594048718245e-06	\\
10492.5426136364	1.99168488626207e-06	\\
10493.5213955966	1.50623999577085e-06	\\
10494.5001775568	1.63977085464777e-06	\\
10495.478959517	1.6792255041539e-06	\\
10496.4577414773	2.12382648724224e-06	\\
10497.4365234375	2.02158435195398e-06	\\
10498.4153053977	2.23693199975753e-06	\\
10499.394087358	2.25246304470249e-06	\\
10500.3728693182	2.31090043133152e-06	\\
10501.3516512784	1.87314622120651e-06	\\
10502.3304332386	1.7443824829823e-06	\\
10503.3092151989	1.69508196611449e-06	\\
10504.2879971591	1.90690203538433e-06	\\
10505.2667791193	1.92461915715642e-06	\\
10506.2455610795	1.82219947557196e-06	\\
10507.2243430398	1.66513998234346e-06	\\
10508.203125	1.85251105231136e-06	\\
10509.1819069602	1.88214720945666e-06	\\
10510.1606889205	1.38708737531321e-06	\\
10511.1394708807	1.44005354266562e-06	\\
10512.1182528409	1.85402740865634e-06	\\
10513.0970348011	1.64154917672432e-06	\\
10514.0758167614	1.67781614873954e-06	\\
10515.0545987216	2.41722838076789e-06	\\
10516.0333806818	2.0937641449879e-06	\\
10517.012162642	2.1221974600909e-06	\\
10517.9909446023	2.10684812900786e-06	\\
10518.9697265625	1.90744568001248e-06	\\
10519.9485085227	2.0279131242014e-06	\\
10520.927290483	1.90349769447096e-06	\\
10521.9060724432	1.7890693814349e-06	\\
10522.8848544034	1.95761307584148e-06	\\
10523.8636363636	1.68849861550254e-06	\\
10524.8424183239	2.38250716215462e-06	\\
10525.8212002841	1.96744628516578e-06	\\
10526.7999822443	1.54185105432741e-06	\\
10527.7787642045	1.87567099147689e-06	\\
10528.7575461648	1.59891568811187e-06	\\
10529.736328125	1.77357047291501e-06	\\
10530.7151100852	1.56922587907805e-06	\\
10531.6938920455	2.09081303021845e-06	\\
10532.6726740057	1.90798682365903e-06	\\
10533.6514559659	1.53080710993024e-06	\\
10534.6302379261	1.5325850599811e-06	\\
10535.6090198864	2.06096215584305e-06	\\
10536.5878018466	1.89259335064691e-06	\\
10537.5665838068	2.4937771185121e-06	\\
10538.545365767	2.05114433196914e-06	\\
10539.5241477273	1.7259209035327e-06	\\
10540.5029296875	2.04907273414275e-06	\\
10541.4817116477	1.69297823386156e-06	\\
10542.460493608	1.85380301159261e-06	\\
10543.4392755682	2.00604548448206e-06	\\
10544.4180575284	1.00804812541745e-06	\\
10545.3968394886	1.94617151503851e-06	\\
10546.3756214489	1.89412806325096e-06	\\
10547.3544034091	1.6573231066071e-06	\\
10548.3331853693	1.75239698208185e-06	\\
10549.3119673295	2.27651178028024e-06	\\
10550.2907492898	1.64253696557035e-06	\\
10551.26953125	1.64806835553641e-06	\\
10552.2483132102	1.7539742857666e-06	\\
10553.2270951705	1.63109776282007e-06	\\
10554.2058771307	1.9096135257042e-06	\\
10555.1846590909	2.0593936367061e-06	\\
10556.1634410511	1.73001210881436e-06	\\
10557.1422230114	1.97279421608903e-06	\\
10558.1210049716	1.73973629471459e-06	\\
10559.0997869318	1.32586769240541e-06	\\
10560.078568892	1.54715937334094e-06	\\
10561.0573508523	1.93427893414865e-06	\\
10562.0361328125	1.56911657378034e-06	\\
10563.0149147727	1.7079295999536e-06	\\
10563.993696733	1.73865792310315e-06	\\
10564.9724786932	1.50751787757431e-06	\\
10565.9512606534	1.72066864263081e-06	\\
10566.9300426136	1.95800806752246e-06	\\
10567.9088245739	1.83651665590039e-06	\\
10568.8876065341	2.00151808201127e-06	\\
10569.8663884943	1.79976433661595e-06	\\
10570.8451704545	1.27289277175245e-06	\\
10571.8239524148	1.78755521389096e-06	\\
10572.802734375	1.43826831659107e-06	\\
10573.7815163352	1.60794393269024e-06	\\
10574.7602982955	1.46691679872918e-06	\\
10575.7390802557	1.60353550543598e-06	\\
10576.7178622159	1.61892714126375e-06	\\
10577.6966441761	1.81075638205239e-06	\\
10578.6754261364	2.04451361386174e-06	\\
10579.6542080966	1.7288715544405e-06	\\
10580.6329900568	1.46104833053755e-06	\\
10581.611772017	1.71819033698338e-06	\\
10582.5905539773	2.00278413242613e-06	\\
10583.5693359375	1.48374859686759e-06	\\
10584.5481178977	2.12835687483518e-06	\\
10585.526899858	1.46883951925181e-06	\\
10586.5056818182	1.3117248057685e-06	\\
10587.4844637784	2.30307373238927e-06	\\
10588.4632457386	1.19165064689219e-06	\\
10589.4420276989	1.90082477775857e-06	\\
10590.4208096591	1.90554294916754e-06	\\
10591.3995916193	1.39165076899595e-06	\\
10592.3783735795	1.45089503595251e-06	\\
10593.3571555398	1.86232504300243e-06	\\
10594.3359375	1.14000201910193e-06	\\
10595.3147194602	1.49524356242612e-06	\\
10596.2935014205	1.72248970949588e-06	\\
10597.2722833807	1.64294225257713e-06	\\
10598.2510653409	1.46210526982057e-06	\\
10599.2298473011	1.91574313396376e-06	\\
10600.2086292614	1.47885338727188e-06	\\
10601.1874112216	2.55876260308575e-06	\\
10602.1661931818	1.37931916080208e-06	\\
10603.144975142	1.54051096391297e-06	\\
10604.1237571023	1.57719320930431e-06	\\
10605.1025390625	1.77429271611499e-06	\\
10606.0813210227	2.06441481955153e-06	\\
10607.060102983	1.50586123865788e-06	\\
10608.0388849432	1.61745979601552e-06	\\
10609.0176669034	1.59830242209859e-06	\\
10609.9964488636	1.58169332205749e-06	\\
10610.9752308239	1.30868830222978e-06	\\
10611.9540127841	1.45543331468415e-06	\\
10612.9327947443	1.61209612370618e-06	\\
10613.9115767045	1.76786411640753e-06	\\
10614.8903586648	1.67087806847268e-06	\\
10615.869140625	1.35831363988905e-06	\\
10616.8479225852	1.72223216506588e-06	\\
10617.8267045455	1.84879501770855e-06	\\
10618.8054865057	1.85143342192056e-06	\\
10619.7842684659	1.54629727745514e-06	\\
10620.7630504261	1.81988073883918e-06	\\
10621.7418323864	1.40748208389579e-06	\\
10622.7206143466	1.43116468578192e-06	\\
10623.6993963068	1.65917104633973e-06	\\
10624.678178267	1.66334350744885e-06	\\
10625.6569602273	1.48377893627388e-06	\\
10626.6357421875	1.45845676301864e-06	\\
10627.6145241477	1.73680857112471e-06	\\
10628.593306108	1.75919750075172e-06	\\
10629.5720880682	1.59374127379842e-06	\\
10630.5508700284	1.44027713413677e-06	\\
10631.5296519886	1.42159984404236e-06	\\
10632.5084339489	1.50594866617581e-06	\\
10633.4872159091	1.71720617946389e-06	\\
10634.4659978693	1.64527107556353e-06	\\
10635.4447798295	2.43899219215802e-06	\\
10636.4235617898	1.37782937329565e-06	\\
10637.40234375	1.58298195874135e-06	\\
10638.3811257102	1.67141288859236e-06	\\
10639.3599076705	1.17352013548551e-06	\\
10640.3386896307	1.45013669199126e-06	\\
10641.3174715909	1.50195924231419e-06	\\
10642.2962535511	1.34799907033629e-06	\\
10643.2750355114	1.5822261099024e-06	\\
10644.2538174716	1.63794178017728e-06	\\
10645.2325994318	1.39733523735248e-06	\\
10646.211381392	1.78498748488861e-06	\\
10647.1901633523	1.69944367946981e-06	\\
10648.1689453125	1.10155790472186e-06	\\
10649.1477272727	1.62001105224543e-06	\\
10650.126509233	1.3227965836687e-06	\\
10651.1052911932	1.03176986725912e-06	\\
10652.0840731534	1.13634803490281e-06	\\
10653.0628551136	1.69379324969293e-06	\\
10654.0416370739	1.48145288745097e-06	\\
10655.0204190341	9.04253723720971e-07	\\
10655.9992009943	1.76994153740273e-06	\\
10656.9779829545	1.65564341705581e-06	\\
10657.9567649148	1.09822409293743e-06	\\
10658.935546875	2.02742921219631e-06	\\
10659.9143288352	1.70821490130484e-06	\\
10660.8931107955	1.00225970230419e-06	\\
10661.8718927557	1.63228575335038e-06	\\
10662.8506747159	1.33912049861322e-06	\\
10663.8294566761	1.1704491146475e-06	\\
10664.8082386364	1.73167638854661e-06	\\
10665.7870205966	1.61957233524244e-06	\\
10666.7658025568	8.30211881466399e-07	\\
10667.744584517	1.42157940369882e-06	\\
10668.7233664773	1.14894797155956e-06	\\
10669.7021484375	9.13608177688431e-07	\\
10670.6809303977	1.57159653213359e-06	\\
10671.659712358	1.24067853248755e-06	\\
10672.6384943182	1.32387846833265e-06	\\
10673.6172762784	1.28383629926929e-06	\\
10674.5960582386	1.1902995533346e-06	\\
10675.5748401989	1.37734936430779e-06	\\
10676.5536221591	1.75335145002272e-06	\\
10677.5324041193	1.52866183864089e-06	\\
10678.5111860795	1.48822303604508e-06	\\
10679.4899680398	1.67759378519346e-06	\\
10680.46875	1.73687770777158e-06	\\
10681.4475319602	1.30353660362046e-06	\\
10682.4263139205	1.2417232157558e-06	\\
10683.4050958807	1.26193765566298e-06	\\
10684.3838778409	1.16170991135865e-06	\\
10685.3626598011	1.4055994687876e-06	\\
10686.3414417614	1.42683345294343e-06	\\
10687.3202237216	7.71060806346728e-07	\\
10688.2990056818	1.65850159980157e-06	\\
10689.277787642	1.57649054641036e-06	\\
10690.2565696023	1.19285764255101e-06	\\
10691.2353515625	1.47617688360457e-06	\\
10692.2141335227	1.16942198000866e-06	\\
10693.192915483	8.82110626604122e-07	\\
10694.1716974432	1.50662164067877e-06	\\
10695.1504794034	1.33957335216401e-06	\\
10696.1292613636	1.20052852309726e-06	\\
10697.1080433239	1.69358676703615e-06	\\
10698.0868252841	1.43398048567928e-06	\\
10699.0656072443	1.06781913475083e-06	\\
10700.0443892045	1.67536500901278e-06	\\
10701.0231711648	1.33825834521639e-06	\\
10702.001953125	1.10173874293199e-06	\\
10702.9807350852	1.04588835517016e-06	\\
10703.9595170455	1.19302614795384e-06	\\
10704.9382990057	1.65227972084526e-06	\\
10705.9170809659	9.86821755476455e-07	\\
10706.8958629261	1.30572386954599e-06	\\
10707.8746448864	1.14187862905322e-06	\\
10708.8534268466	1.08877229927846e-06	\\
10709.8322088068	1.73058090228391e-06	\\
10710.810990767	9.16240131189627e-07	\\
10711.7897727273	1.41966596743361e-06	\\
10712.7685546875	1.29442665097337e-06	\\
10713.7473366477	9.77522194505134e-07	\\
10714.726118608	1.2601101267707e-06	\\
10715.7049005682	1.23167654442333e-06	\\
10716.6836825284	1.26965958092721e-06	\\
10717.6624644886	8.14409704676653e-07	\\
10718.6412464489	1.06741651737953e-06	\\
10719.6200284091	1.40502744721593e-06	\\
10720.5988103693	1.07180565794112e-06	\\
10721.5775923295	1.63259095041092e-06	\\
10722.5563742898	9.67659603305513e-07	\\
10723.53515625	1.50026177305607e-06	\\
10724.5139382102	1.03222422547113e-06	\\
10725.4927201705	8.60392844247021e-07	\\
10726.4715021307	1.19721932179289e-06	\\
10727.4502840909	1.4671861958572e-06	\\
10728.4290660511	1.2927941868331e-06	\\
10729.4078480114	1.08725862536098e-06	\\
10730.3866299716	1.04038242996062e-06	\\
10731.3654119318	9.97537204553161e-07	\\
10732.344193892	8.45755299461503e-07	\\
10733.3229758523	1.36715216441716e-06	\\
10734.3017578125	8.48855373767657e-07	\\
10735.2805397727	1.30930716740117e-06	\\
10736.259321733	1.12916416041815e-06	\\
10737.2381036932	9.75337098596015e-07	\\
10738.2168856534	1.22219487599005e-06	\\
10739.1956676136	9.57740969131759e-07	\\
10740.1744495739	9.89732816769167e-07	\\
10741.1532315341	1.10372502340791e-06	\\
10742.1320134943	1.30468772522717e-06	\\
10743.1107954545	6.07183425561055e-07	\\
10744.0895774148	1.15393071181095e-06	\\
10745.068359375	9.82420166418912e-07	\\
10746.0471413352	1.19037927423629e-06	\\
10747.0259232955	9.37489975277976e-07	\\
10748.0047052557	6.16235708044255e-07	\\
10748.9834872159	1.28860895663453e-06	\\
10749.9622691761	3.65427176783356e-07	\\
10750.9410511364	1.12815183605704e-06	\\
10751.9198330966	1.18355389043263e-06	\\
10752.8986150568	1.24573427612309e-06	\\
10753.877397017	6.5244706607216e-07	\\
10754.8561789773	1.23612195886143e-06	\\
10755.8349609375	1.11934745179089e-06	\\
10756.8137428977	9.91631887930021e-07	\\
10757.792524858	9.61227745237156e-07	\\
10758.7713068182	1.2386821277501e-06	\\
10759.7500887784	8.44990221001518e-07	\\
10760.7288707386	1.26095263104637e-06	\\
10761.7076526989	9.3864135657436e-07	\\
10762.6864346591	1.00509560634686e-06	\\
10763.6652166193	1.08975162695273e-06	\\
10764.6439985795	1.11318584689428e-06	\\
10765.6227805398	8.54839848130122e-07	\\
10766.6015625	1.27908398953577e-06	\\
10767.5803444602	8.7895403890997e-07	\\
10768.5591264205	3.75995221469226e-07	\\
10769.5379083807	1.32888618034065e-06	\\
10770.5166903409	5.79041739675611e-07	\\
10771.4954723011	7.58365737155929e-07	\\
10772.4742542614	1.24884504218281e-06	\\
10773.4530362216	9.7054106468102e-07	\\
10774.4318181818	9.98632292006535e-07	\\
10775.410600142	8.39457474046841e-07	\\
10776.3893821023	1.0254048914695e-06	\\
10777.3681640625	7.87145854193836e-07	\\
10778.3469460227	6.62429837006691e-07	\\
10779.325727983	6.0571065717384e-07	\\
10780.3045099432	1.00898331661666e-06	\\
10781.2832919034	1.05558642656358e-06	\\
10782.2620738636	3.69589733666551e-07	\\
10783.2408558239	9.11587911226437e-07	\\
10784.2196377841	7.57683451436334e-07	\\
10785.1984197443	7.43618786478437e-07	\\
10786.1772017045	5.12862090619045e-07	\\
10787.1559836648	8.38853580613185e-07	\\
10788.134765625	5.14305522986097e-07	\\
10789.1135475852	9.1545983557036e-07	\\
10790.0923295455	8.2898047102129e-07	\\
10791.0711115057	4.62837141528411e-07	\\
10792.0498934659	5.45726871474695e-07	\\
10793.0286754261	9.34126328161991e-07	\\
10794.0074573864	2.77467707456456e-07	\\
10794.9862393466	5.48007835111423e-07	\\
10795.9650213068	7.08030233286154e-07	\\
10796.943803267	9.00724782989925e-07	\\
10797.9225852273	6.44058227423301e-07	\\
10798.9013671875	1.01734522230241e-06	\\
10799.8801491477	9.00064849056588e-07	\\
10800.858931108	9.04437895402009e-07	\\
10801.8377130682	1.00312275617065e-06	\\
10802.8164950284	7.79905223922409e-07	\\
10803.7952769886	9.47217945023172e-07	\\
10804.7740589489	6.8418080544613e-07	\\
10805.7528409091	6.37694475557133e-07	\\
10806.7316228693	5.05692172860619e-07	\\
10807.7104048295	1.04405085926928e-06	\\
10808.6891867898	7.59067984190636e-07	\\
10809.66796875	8.15055610375864e-07	\\
10810.6467507102	8.88175003335139e-07	\\
10811.6255326705	9.18531631868213e-07	\\
10812.6043146307	3.91064709042655e-07	\\
10813.5830965909	4.49319212657914e-07	\\
10814.5618785511	6.16404224311963e-07	\\
10815.5406605114	4.93968265294121e-07	\\
10816.5194424716	9.68176054751299e-07	\\
10817.4982244318	5.65283842419274e-07	\\
10818.477006392	5.67679017546585e-07	\\
10819.4557883523	9.85101185056617e-07	\\
10820.4345703125	9.84719777862699e-07	\\
10821.4133522727	2.14744448249433e-07	\\
10822.392134233	4.64780117783069e-07	\\
10823.3709161932	6.36323526818899e-07	\\
10824.3496981534	4.91330536468657e-07	\\
10825.3284801136	6.26533714006038e-07	\\
10826.3072620739	6.07710959482608e-07	\\
10827.2860440341	6.65744402706821e-07	\\
10828.2648259943	6.40165685555542e-07	\\
10829.2436079545	3.91798467704527e-07	\\
10830.2223899148	6.37210441287078e-07	\\
10831.201171875	1.01368905557281e-06	\\
10832.1799538352	5.24986712663484e-07	\\
10833.1587357955	5.63088671267615e-07	\\
10834.1375177557	6.10547384243788e-07	\\
10835.1162997159	1.05103283187549e-06	\\
10836.0950816761	6.94885743559264e-07	\\
10837.0738636364	7.13681714127199e-07	\\
10838.0526455966	7.97791239967988e-07	\\
10839.0314275568	7.36623950824119e-07	\\
10840.010209517	8.18811016847649e-07	\\
10840.9889914773	8.5334829071835e-07	\\
10841.9677734375	4.52116324578669e-07	\\
10842.9465553977	5.22611664638098e-07	\\
10843.925337358	1.51458532260998e-07	\\
10844.9041193182	4.90984803187125e-07	\\
10845.8829012784	8.14705024574643e-07	\\
10846.8616832386	8.1603598247136e-07	\\
10847.8404651989	4.31976853330059e-07	\\
10848.8192471591	5.60211827639894e-07	\\
10849.7980291193	7.64619722206479e-07	\\
10850.7768110795	4.63742977635082e-07	\\
10851.7555930398	6.89993773908024e-07	\\
10852.734375	3.21115611862156e-07	\\
10853.7131569602	4.027936235081e-07	\\
10854.6919389205	8.38073794026177e-07	\\
10855.6707208807	7.43166779237395e-07	\\
10856.6495028409	4.28987198998434e-07	\\
10857.6282848011	2.71982413021743e-07	\\
10858.6070667614	3.44030558553592e-07	\\
10859.5858487216	3.02032101171246e-07	\\
10860.5646306818	3.1372765012197e-07	\\
10861.543412642	1.02544596008071e-06	\\
10862.5221946023	4.16772791784408e-07	\\
10863.5009765625	6.80043900321805e-07	\\
10864.4797585227	7.40011797289545e-07	\\
10865.458540483	2.30633768052115e-07	\\
10866.4373224432	4.19935160752029e-07	\\
10867.4161044034	7.46158012096973e-07	\\
10868.3948863636	7.05773836439776e-07	\\
10869.3736683239	4.33188977100943e-07	\\
10870.3524502841	6.41844635139583e-07	\\
10871.3312322443	5.91916879355285e-07	\\
10872.3100142045	3.29795002794599e-07	\\
10873.2887961648	2.03316264337818e-07	\\
10874.267578125	2.88640285242046e-07	\\
10875.2463600852	7.46532657953056e-07	\\
10876.2251420455	3.96068921274439e-07	\\
10877.2039240057	7.85609066137346e-07	\\
10878.1827059659	8.48252847154293e-07	\\
10879.1614879261	4.93264163967701e-07	\\
10880.1402698864	6.03145984343775e-07	\\
10881.1190518466	2.48676846059071e-07	\\
10882.0978338068	4.19936792273434e-07	\\
10883.076615767	5.58255797821834e-07	\\
10884.0553977273	6.2489734208721e-07	\\
10885.0341796875	7.53055617548117e-07	\\
10886.0129616477	1.68850598689432e-07	\\
10886.991743608	3.28615826678686e-07	\\
10887.9705255682	2.43753081324796e-07	\\
10888.9493075284	4.77692693085698e-07	\\
10889.9280894886	4.42842907738643e-07	\\
10890.9068714489	5.88778256131791e-07	\\
10891.8856534091	5.51075430947492e-07	\\
10892.8644353693	7.30345933734798e-07	\\
10893.8432173295	4.70012836830117e-07	\\
10894.8219992898	2.29044159263299e-07	\\
10895.80078125	6.70845407387737e-07	\\
10896.7795632102	3.00504664130681e-07	\\
10897.7583451705	6.31096313942254e-07	\\
10898.7371271307	1.36189620053965e-07	\\
10899.7159090909	1.26913728752086e-07	\\
10900.6946910511	3.150818198516e-07	\\
10901.6734730114	3.84503011268006e-07	\\
10902.6522549716	9.58814699669128e-08	\\
10903.6310369318	5.87264891644337e-07	\\
10904.609818892	3.03958655208556e-07	\\
10905.5886008523	2.38189625279387e-07	\\
10906.5673828125	6.76054239537419e-07	\\
10907.5461647727	8.04454991260035e-07	\\
10908.524946733	2.78564345437554e-07	\\
10909.5037286932	7.60894441375143e-07	\\
10910.4825106534	1.95105673252692e-07	\\
10911.4612926136	5.24227070451293e-07	\\
10912.4400745739	5.34896746315888e-07	\\
10913.4188565341	3.94948024819285e-07	\\
10914.3976384943	3.4991520457526e-07	\\
10915.3764204545	5.35859938765717e-07	\\
10916.3552024148	6.56072801927349e-07	\\
10917.333984375	6.8649309601978e-07	\\
10918.3127663352	5.50606053662159e-07	\\
10919.2915482955	4.82287304258407e-07	\\
10920.2703302557	2.48217804632954e-07	\\
10921.2491122159	3.48859536625154e-07	\\
10922.2278941761	2.59732049416056e-07	\\
10923.2066761364	4.07724092991029e-07	\\
10924.1854580966	2.26397627486676e-07	\\
10925.1642400568	4.97462545054451e-07	\\
10926.143022017	1.33370385223249e-07	\\
10927.1218039773	1.8953780199444e-07	\\
10928.1005859375	7.0322877463093e-08	\\
10929.0793678977	4.3904419415184e-07	\\
10930.058149858	5.88049385406655e-07	\\
10931.0369318182	5.86867656464031e-07	\\
10932.0157137784	1.34206815890962e-07	\\
10932.9944957386	2.32420893860526e-07	\\
10933.9732776989	4.3680195714707e-07	\\
10934.9520596591	2.37898750687519e-07	\\
10935.9308416193	5.99821574399464e-07	\\
10936.9096235795	5.63561389852634e-07	\\
10937.8884055398	3.87152537930741e-07	\\
10938.8671875	8.27242355275961e-07	\\
10939.8459694602	7.65161402308435e-07	\\
10940.8247514205	2.63173222766562e-07	\\
10941.8035333807	2.9910247133764e-07	\\
10942.7823153409	9.13713992697528e-08	\\
10943.7610973011	5.90865511610815e-07	\\
10944.7398792614	3.99119321171895e-08	\\
10945.7186612216	5.13463792217618e-07	\\
10946.6974431818	6.29323114139498e-07	\\
10947.676225142	5.35361028450959e-07	\\
10948.6550071023	5.47005566094221e-07	\\
10949.6337890625	7.66868575529385e-07	\\
10950.6125710227	6.13701644370175e-07	\\
10951.591352983	6.38419933873003e-07	\\
10952.5701349432	3.89218641022002e-08	\\
10953.5489169034	4.83050489112396e-07	\\
10954.5276988636	4.8568511542049e-07	\\
10955.5064808239	2.97311296107927e-07	\\
10956.4852627841	3.02889149056414e-07	\\
10957.4640447443	1.4699708428986e-07	\\
10958.4428267045	2.69393449974596e-07	\\
10959.4216086648	3.49172638169653e-07	\\
10960.400390625	3.1618288420306e-07	\\
10961.3791725852	3.63516591028026e-07	\\
10962.3579545455	7.86232658267853e-07	\\
10963.3367365057	5.37347074259336e-07	\\
10964.3155184659	5.03716996860143e-07	\\
10965.2943004261	2.19294898741985e-07	\\
10966.2730823864	5.40308800731858e-07	\\
10967.2518643466	2.76131671337772e-07	\\
10968.2306463068	1.94406626403223e-07	\\
10969.209428267	5.88185803391748e-07	\\
10970.1882102273	3.59008223369039e-07	\\
10971.1669921875	5.45190301795267e-07	\\
10972.1457741477	2.25916731747243e-07	\\
10973.124556108	4.51753168754172e-07	\\
10974.1033380682	4.55046982334573e-07	\\
10975.0821200284	6.63015433203995e-07	\\
10976.0609019886	2.8016231071612e-07	\\
10977.0396839489	3.68769151934774e-07	\\
10978.0184659091	4.36734092733169e-07	\\
10978.9972478693	5.92382681262396e-07	\\
10979.9760298295	3.16368330504967e-07	\\
10980.9548117898	4.38144840426988e-07	\\
10981.93359375	4.15286305556761e-07	\\
10982.9123757102	3.64330793654919e-07	\\
10983.8911576705	6.05003077357967e-07	\\
10984.8699396307	9.11216474088275e-08	\\
10985.8487215909	2.51175881719678e-07	\\
10986.8275035511	3.14915091473333e-07	\\
10987.8062855114	7.81840294287668e-08	\\
10988.7850674716	3.7925395990644e-07	\\
10989.7638494318	1.55699688157437e-07	\\
10990.742631392	2.9977458020177e-07	\\
10991.7214133523	2.89554152672428e-07	\\
10992.7001953125	3.2306652927595e-07	\\
10993.6789772727	5.97634328985684e-07	\\
10994.657759233	5.41636748686034e-07	\\
10995.6365411932	3.28363532472297e-07	\\
10996.6153231534	4.32986560159093e-07	\\
10997.5941051136	3.7411185714794e-07	\\
10998.5728870739	2.4764631479494e-07	\\
10999.5516690341	4.88493713635106e-07	\\
11000.5304509943	5.27118194910882e-07	\\
11001.5092329545	4.94982095651224e-07	\\
11002.4880149148	2.80178814408448e-07	\\
11003.466796875	3.78434951901723e-07	\\
11004.4455788352	2.06200203960869e-07	\\
11005.4243607955	7.57859302145074e-07	\\
11006.4031427557	2.51140361592898e-07	\\
11007.3819247159	1.52838968460744e-07	\\
11008.3607066761	2.42362992681971e-07	\\
11009.3394886364	3.44841127045598e-07	\\
11010.3182705966	3.04171828736175e-07	\\
11011.2970525568	8.47734481404522e-07	\\
11012.275834517	2.83200744248436e-07	\\
11013.2546164773	3.70594569581616e-07	\\
11014.2333984375	4.04670548691998e-07	\\
11015.2121803977	2.25089304174186e-07	\\
11016.190962358	2.10027945197449e-07	\\
11017.1697443182	3.64168679122739e-07	\\
11018.1485262784	4.98346804249498e-07	\\
11019.1273082386	2.19229584342878e-07	\\
11020.1060901989	3.79275570135795e-07	\\
11021.0848721591	2.29444384043337e-07	\\
11022.0636541193	9.00510737428282e-08	\\
11023.0424360795	3.28821786877167e-07	\\
11024.0212180398	8.5283295000321e-07	\\
};
\end{axis}
\end{tikzpicture}%
	\caption{Impulse response at Fs\_TX = 176.400 Hz and Fs\_RX = 22.050 Hz.}
	\label{fig:highest_freq}
\end{figure}

\end{document}