%!TEX program = xelatex+makeindex+bibtex
\documentclass[final]{scrreprt} %scrreprt of scrartcl
\input{../../library/preamble.tex}
\input{../../library/style.tex}
\addbibresource{../../library/bibliography.bib}
\begin{document}
\chapter{Introduction}
With renewable energy rapidly becoming more and more important, future cars will see a major overhaul. 
An increasing amount of cars become hybrid or fully electric. 
To become familiar with this development a R/C car is modified to become smart and 'green'.
KITT, as the R/C car is called, features a wireless charging system, an anti-collision system and is able to autonomously drive in a straight line. 
Certain requirements are to be met regarding these properties. 
The aim is to optimize every property and preferably exceed the requirements.
\\\\
For the final competition we have to deal with a set of challenges. 
When KITT leaves the charging station, the station has to switch off automatically (over-current protection).
For the first challenge KITT has to drive to a specific point in the field. 
When KITT reaches the destination it must stop and the PC must give a signal. 
In the second challenge KITT has to drive from the starting point to a specific point in the field over specified way-points.
%Later on KITT 
In the last challenge the car needs to drive to a specific point while avoiding obstacles and other cars. 


%Via the wireless charging system ten \unit[350]{F}-supercapacitors in series should be charged in less than 4 minutes. 
%This system will be implemented with the use of coils seperated by a \unit[6]{cm} distance. 
%The input voltage is \unit[18]{V} and the supercapacitors should be charged to \unit[20]{V}. 
%Furthermore, KITT should be able to bridge a 3 meter distance to a wall within 3 seconds. 
%At less than \unit[10]{cm} of the wall KITT should come to a standstill without hitting the wall or performing a large overshoot.
\end{document}