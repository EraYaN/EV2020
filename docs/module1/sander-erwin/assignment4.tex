%!TEX program=xelatex+makeindex+bibtex
\documentclass[final]{scrreprt} %scrreprt of scrartcl
% Include all project wide packages here.
\usepackage{fullpage}
\usepackage{polyglossia}
\setmainlanguage{english}
\usepackage{csquotes}
\usepackage{graphicx}
\usepackage{epstopdf}
\usepackage{pdfpages}
\usepackage{caption}
\usepackage[list=true]{subcaption}
\usepackage{float}
\usepackage{standalone}
\usepackage{import}
\usepackage{tocloft}
\usepackage{wrapfig}
\usepackage{authblk}
\usepackage{array}
\usepackage{booktabs}
\usepackage[toc,page,title,titletoc]{appendix}
\usepackage{xunicode}
\usepackage{fontspec}
\usepackage{pgfplots}
\usepackage{SIunits}
\usepackage{units}
\pgfplotsset{compat=newest}
\pgfplotsset{plot coordinates/math parser=false}
\newlength\figureheight 
\newlength\figurewidth
\usepackage{amsmath}
\usepackage{mathtools}
\usepackage{unicode-math}
\usepackage[
    backend=bibtexu,
	texencoding=utf8,
bibencoding=utf8,
    style=ieee,
    sortlocale=en_US,
    language=auto
]{biblatex}
\usepackage{listings}
\newcommand{\includecode}[3][c]{\lstinputlisting[caption=#2, escapechar=, style=#1]{#3}}
\newcommand{\superscript}[1]{\ensuremath{^{\textrm{#1}}}}
\newcommand{\subscript}[1]{\ensuremath{_{\textrm{#1}}}}


\newcommand{\chapternumber}{\thechapter}
\renewcommand{\appendixname}{Bijlage}
\renewcommand{\appendixtocname}{Bijlagen}
\renewcommand{\appendixpagename}{Bijlagen}

\usepackage[hidelinks]{hyperref} %<--------ALTIJD ALS LAATSTE

\renewcommand{\familydefault}{\sfdefault}

\setmainfont[Ligatures=TeX]{Myriad Pro}
\setmathfont{Asana Math}
\setmonofont{Lucida Console}

\usepackage{titlesec, blindtext, color}
\definecolor{gray75}{gray}{0.75}
\newcommand{\hsp}{\hspace{20pt}}
\titleformat{\chapter}[hang]{\Huge\bfseries}{\chapternumber\hsp\textcolor{gray75}{|}\hsp}{0pt}{\Huge\bfseries}
\renewcommand{\familydefault}{\sfdefault}
\renewcommand{\arraystretch}{1.2}
\setlength\parindent{0pt}

%For code listings
\definecolor{black}{rgb}{0,0,0}
\definecolor{browntags}{rgb}{0.65,0.1,0.1}
\definecolor{bluestrings}{rgb}{0,0,1}
\definecolor{graycomments}{rgb}{0.4,0.4,0.4}
\definecolor{redkeywords}{rgb}{1,0,0}
\definecolor{bluekeywords}{rgb}{0.13,0.13,0.8}
\definecolor{greencomments}{rgb}{0,0.5,0}
\definecolor{redstrings}{rgb}{0.9,0,0}
\definecolor{purpleidentifiers}{rgb}{0.01,0,0.01}


\lstdefinestyle{csharp}{
language=[Sharp]C,
showspaces=false,
showtabs=false,
breaklines=true,
showstringspaces=false,
breakatwhitespace=true,
escapeinside={(*@}{@*)},
columns=fullflexible,
commentstyle=\color{greencomments},
keywordstyle=\color{bluekeywords}\bfseries,
stringstyle=\color{redstrings},
identifierstyle=\color{purpleidentifiers},
basicstyle=\ttfamily\small}

\lstdefinestyle{c}{
language=C,
showspaces=false,
showtabs=false,
breaklines=true,
showstringspaces=false,
breakatwhitespace=true,
escapeinside={(*@}{@*)},
columns=fullflexible,
commentstyle=\color{greencomments},
keywordstyle=\color{bluekeywords}\bfseries,
stringstyle=\color{redstrings},
identifierstyle=\color{purpleidentifiers},
}

\lstdefinestyle{matlab}{
language=Matlab,
showspaces=false,
showtabs=false,
breaklines=true,
showstringspaces=false,
breakatwhitespace=true,
escapeinside={(*@}{@*)},
columns=fullflexible,
commentstyle=\color{greencomments},
keywordstyle=\color{bluekeywords}\bfseries,
stringstyle=\color{redstrings},
identifierstyle=\color{purpleidentifiers}
}

\lstdefinestyle{vhdl}{
language=VHDL,
showspaces=false,
showtabs=false,
breaklines=true,
showstringspaces=false,
breakatwhitespace=true,
escapeinside={(*@}{@*)},
columns=fullflexible,
commentstyle=\color{greencomments},
keywordstyle=\color{bluekeywords}\bfseries,
stringstyle=\color{redstrings},
identifierstyle=\color{purpleidentifiers}
}

\lstdefinestyle{xaml}{
language=XML,
showspaces=false,
showtabs=false,
breaklines=true,
showstringspaces=false,
breakatwhitespace=true,
escapeinside={(*@}{@*)},
columns=fullflexible,
commentstyle=\color{greencomments},
keywordstyle=\color{redkeywords},
stringstyle=\color{bluestrings},
tagstyle=\color{browntags},
morestring=[b]",
  morecomment=[s]{<?}{?>},
  morekeywords={xmlns,version,typex:AsyncRecords,x:Arguments,x:Boolean,x:Byte,x:Char,x:Class,x:ClassAttributes,x:ClassModifier,x:Code,x:ConnectionId,x:Decimal,x:Double,x:FactoryMethod,x:FieldModifier,x:Int16,x:Int32,x:Int64,x:Key,x:Members,x:Name,x:Object,x:Property,x:Shared,x:Single,x:String,x:Subclass,x:SynchronousMode,x:TimeSpan,x:TypeArguments,x:Uid,x:Uri,x:XData,Grid.Column,Grid.ColumnSpan,Click,ClipToBounds,Content,DropDownOpened,FontSize,Foreground,Header,Height,HorizontalAlignment,HorizontalContentAlignment,IsCancel,IsDefault,IsEnabled,IsSelected,Margin,MinHeight,MinWidth,Padding,SnapsToDevicePixels,Target,TextWrapping,Title,VerticalAlignment,VerticalContentAlignment,Width,WindowStartupLocation,Binding,Mode,OneWay,xmlns:x}
}

%defaults
\lstset{
basicstyle=\ttfamily\small,
extendedchars=false,
numbers=left,
numberstyle=\ttfamily\tiny,
stepnumber=1,
tabsize=4,
numbersep=5pt
}
\addbibresource{../library/bibliography.bib}
\title{Module 1 - Report}
\author{Sander {van Dijk} \and Erwin {de Haan}}
\begin{document}
\chapter{Assignment 4}
\section{Task 1: Nissan Leaf}
The manufactuer's claim is valid, the vehicle reaches 96.5 km/h in about 8 seconds.
The motor torque is the limiting factor.
The small error is the result of the non-infinite gain of the driver in the simulation.
\begin{figure}[H]
	\centering
    	\setlength\figureheight{4cm}
    	\setlength\figurewidth{0.8\linewidth}
    	% This file was created by matlab2tikz v0.4.6 running on MATLAB 8.2.
% Copyright (c) 2008--2014, Nico Schlömer <nico.schloemer@gmail.com>
% All rights reserved.
% Minimal pgfplots version: 1.3
% 
% The latest updates can be retrieved from
%   http://www.mathworks.com/matlabcentral/fileexchange/22022-matlab2tikz
% where you can also make suggestions and rate matlab2tikz.
% 
\begin{tikzpicture}

\begin{axis}[%
width=\figurewidth,
height=\figureheight,
scale only axis,
xmin=0,
xmax=200,
xmajorgrids,
ymin=0,
ymax=200,
ymajorgrids,
name=plot1,
legend style={draw=black,fill=white,legend cell align=left}
]
\addplot [color=blue,solid]
  table[row sep=crcr]{
0	0	\\
0.0322292557860021	157.0581	\\
0.352229255786002	157.0581	\\
0.692229255786002	157.0581	\\
1.042229255786	157.0581	\\
1.382229255786	157.0581	\\
1.732229255786	157.0581	\\
2.072229255786	157.0581	\\
2.42222925578599	157.0581	\\
2.76222925578599	157.0581	\\
3.11222925578598	157.0581	\\
3.45222925578597	157.0581	\\
3.80222925578596	157.0581	\\
4.14222925578596	157.0581	\\
4.49222925578595	157.0581	\\
4.83222925578594	157.0581	\\
5.18222925578594	157.0581	\\
5.52222925578593	157.0581	\\
5.86222925578592	157.0581	\\
6.21222925578591	157.0581	\\
6.55222925578591	157.0581	\\
6.9022292557859	157.0581	\\
7.24222925578589	157.0581	\\
7.59222925578588	157.0581	\\
7.93222925578588	157.0581	\\
8.28222925578587	157.0581	\\
8.62222925578586	157.0581	\\
8.97222925578586	157.0581	\\
9.31222925578585	157.0581	\\
9.66222925578584	157.0581	\\
10.0000000000001	157.0581	\\
10.3500000000001	157.0581	\\
10.6900000000001	157.0581	\\
11.0400000000001	157.0581	\\
11.3800000000001	157.0581	\\
11.7300000000001	157.0581	\\
12.0700000000001	157.0581	\\
12.4200000000001	157.0581	\\
12.7600000000001	157.0581	\\
13.11	157.0581	\\
13.45	157.0581	\\
13.8	157.0581	\\
14.14	157.0581	\\
14.49	157.0581	\\
14.83	157.0581	\\
15.18	157.0581	\\
15.52	157.0581	\\
15.87	157.0581	\\
16.21	157.0581	\\
16.5600000000001	157.0581	\\
16.9000000000001	157.0581	\\
17.2500000000002	157.0581	\\
17.5900000000002	157.0581	\\
17.9400000000003	157.0581	\\
18.2800000000003	157.0581	\\
18.6300000000004	157.0581	\\
18.9700000000004	157.0581	\\
19.3200000000005	157.0581	\\
19.6600000000006	157.0581	\\
20	157.0581	\\
20.3472841771288	157.0581	\\
20.6972841771288	157.0581	\\
21.0372841771289	157.0581	\\
21.3872841771289	157.0581	\\
21.727284177129	157.0581	\\
22.077284177129	157.0581	\\
22.4172841771291	157.0581	\\
22.7672841771291	157.0581	\\
23.1072841771292	157.0581	\\
23.4572841771293	157.0581	\\
23.7972841771293	157.0581	\\
24.1472841771294	157.0581	\\
24.4872841771294	157.0581	\\
24.8372841771295	157.0581	\\
25.1772841771295	157.0581	\\
25.5172841771296	157.0581	\\
25.8672841771296	157.0581	\\
26.2072841771297	157.0581	\\
26.5572841771297	157.0581	\\
26.8972841771298	157.0581	\\
27.2472841771298	157.0581	\\
27.5872841771299	157.0581	\\
27.93728417713	157.0581	\\
28.27728417713	157.0581	\\
28.6272841771301	157.0581	\\
28.9672841771301	157.0581	\\
29.3172841771302	157.0581	\\
29.6572841771302	157.0581	\\
30.0072841771303	157.0581	\\
30.3472841771303	157.0581	\\
30.6972841771304	157.0581	\\
31.0372841771304	157.0581	\\
31.3872841771305	157.0581	\\
31.7272841771305	157.0581	\\
32.0772841771306	157.0581	\\
32.4172841771305	157.0581	\\
32.7672841771304	157.0581	\\
33.1072841771304	157.0581	\\
33.4572841771303	157.0581	\\
33.7972841771302	157.0581	\\
34.1472841771302	157.0581	\\
34.4872841771301	157.0581	\\
34.83728417713	157.0581	\\
35.17728417713	157.0581	\\
35.5172841771299	157.0581	\\
35.8672841771298	157.0581	\\
36.2072841771298	157.0581	\\
36.5572841771297	157.0581	\\
36.8972841771296	157.0581	\\
37.2472841771296	157.0581	\\
37.5872841771295	157.0581	\\
37.9372841771294	157.0581	\\
38.2772841771293	157.0581	\\
38.6272841771293	157.0581	\\
38.9672841771292	157.0581	\\
39.3172841771291	157.0581	\\
39.6572841771291	157.0581	\\
40	157.0581	\\
40.3500000000004	157.0581	\\
40.6900000000003	157.0581	\\
41.0400000000002	157.0581	\\
41.3800000000002	157.0581	\\
41.7300000000001	157.0581	\\
42.07	157.0581	\\
42.42	157.0581	\\
42.7599999999999	157.0581	\\
43.1099999999998	157.0581	\\
43.4499999999998	157.0581	\\
43.7999999999997	157.0581	\\
44.1399999999996	157.0581	\\
44.4899999999996	157.0581	\\
44.8299999999995	157.0581	\\
45.1799999999994	157.0581	\\
45.5199999999994	157.0581	\\
45.8699999999993	157.0581	\\
46.2099999999992	157.0581	\\
46.5599999999992	157.0581	\\
46.8999999999991	157.0581	\\
47.249999999999	157.0581	\\
47.5899999999989	157.0581	\\
47.9399999999989	157.0581	\\
48.2799999999988	157.0581	\\
48.6299999999987	157.0581	\\
48.9664934895418	157.0581	\\
49.3164934895418	157.0581	\\
49.6564934895417	157.0581	\\
50.0064934895416	157.0581	\\
50.3464934895416	157.0581	\\
50.6964934895415	157.0581	\\
51.0364934895414	157.0581	\\
51.3864934895414	157.0581	\\
51.7264934895413	157.0581	\\
52.0764934895412	157.0581	\\
52.4164934895411	157.0581	\\
52.7664934895411	157.0581	\\
53.106493489541	157.0581	\\
53.4564934895409	157.0581	\\
53.7964934895409	157.0581	\\
54.1464934895408	157.0581	\\
54.4864934895407	157.0581	\\
54.8364934895407	157.0581	\\
55.1764934895406	157.0581	\\
55.5264934895405	157.0581	\\
55.8664934895405	157.0581	\\
56.2164934895404	157.0581	\\
56.5564934895403	157.0581	\\
56.9064934895403	157.0581	\\
57.2464934895402	157.0581	\\
57.5864934895401	157.0581	\\
57.93649348954	157.0581	\\
58.27649348954	157.0581	\\
58.6264934895399	157.0581	\\
58.9664934895398	157.0581	\\
59.3164934895398	157.0581	\\
59.6564934895397	157.0581	\\
60	157.0581	\\
60.3500000000004	157.0581	\\
60.6900000000003	157.0581	\\
61.0400000000002	157.0581	\\
61.3800000000002	157.0581	\\
61.7300000000001	157.0581	\\
62.07	157.0581	\\
62.42	157.0581	\\
62.7599999999999	157.0581	\\
63.1099999999998	157.0581	\\
63.4499999999998	157.0581	\\
63.7999999999997	157.0581	\\
64.1399999999997	157.0581	\\
64.4899999999999	157.0581	\\
64.8300000000001	157.0581	\\
65.1800000000003	157.0581	\\
65.5200000000004	157.0581	\\
65.8700000000006	157.0581	\\
66.2100000000008	157.0581	\\
66.560000000001	157.0581	\\
66.9000000000011	157.0581	\\
67.2500000000013	157.0581	\\
67.5900000000015	157.0581	\\
67.9400000000017	157.0581	\\
68.2800000000018	157.0581	\\
68.630000000002	157.0581	\\
68.9700000000022	157.0581	\\
69.3200000000024	157.0581	\\
69.6600000000026	157.0581	\\
70.0000000000027	157.0581	\\
70.3500000000029	157.0581	\\
70.6900000000031	157.0581	\\
71.0400000000033	157.0581	\\
71.3800000000034	157.0581	\\
71.7300000000036	157.0581	\\
72.0700000000038	157.0581	\\
72.420000000004	157.0581	\\
72.7600000000041	157.0581	\\
73.1100000000043	157.0581	\\
73.4500000000045	157.0581	\\
73.8000000000047	157.0581	\\
74.1400000000048	157.0581	\\
74.490000000005	157.0581	\\
74.8300000000052	157.0581	\\
75.1800000000054	157.0581	\\
75.5200000000056	157.0581	\\
75.8700000000057	157.0581	\\
76.2100000000059	157.0581	\\
76.5600000000061	157.0581	\\
76.9000000000063	157.0581	\\
77.2500000000064	157.0581	\\
77.5900000000066	157.0581	\\
77.9400000000068	157.0581	\\
78.280000000007	157.0581	\\
78.6300000000071	157.0581	\\
78.9700000000073	157.0581	\\
79.3200000000075	157.0581	\\
79.6600000000077	157.0581	\\
80	157.0581	\\
80.3470680012668	157.0581	\\
80.6970680012669	157.0581	\\
81.0370680012671	157.0581	\\
81.3870680012673	157.0581	\\
81.7270680012675	157.0581	\\
82.0770680012676	157.0581	\\
82.4170680012678	157.0581	\\
82.767068001268	157.0581	\\
83.1070680012682	157.0581	\\
83.4570680012684	157.0581	\\
83.7970680012685	157.0581	\\
84.1470680012687	157.0581	\\
84.4870680012689	157.0581	\\
84.8370680012691	157.0581	\\
85.1770680012692	157.0581	\\
85.5270680012694	157.0581	\\
85.8670680012696	157.0581	\\
86.2070680012698	157.0581	\\
86.5570680012699	157.0581	\\
86.8970680012701	157.0581	\\
87.2470680012703	157.0581	\\
87.5870680012705	157.0581	\\
87.9370680012706	157.0581	\\
88.2770680012708	157.0581	\\
88.627068001271	157.0581	\\
88.9670680012712	157.0581	\\
89.3170680012714	157.0581	\\
89.6570680012715	157.0581	\\
90.0070680012717	157.0581	\\
90.3470680012719	157.0581	\\
90.6970680012721	157.0581	\\
91.0370680012722	157.0581	\\
91.3870680012724	157.0581	\\
91.7270680012726	157.0581	\\
92.0770680012728	157.0581	\\
92.4170680012729	157.0581	\\
92.7670680012731	157.0581	\\
93.1070680012733	157.0581	\\
93.4570680012735	157.0581	\\
93.7970680012736	157.0581	\\
94.1470680012738	157.0581	\\
94.487068001274	157.0581	\\
94.8370680012742	157.0581	\\
95.1770680012744	157.0581	\\
95.5270680012745	157.0581	\\
95.8670680012747	157.0581	\\
96.2070680012749	157.0581	\\
96.5570680012751	157.0581	\\
96.8970680012752	157.0581	\\
97.2470680012754	157.0581	\\
97.5870680012756	157.0581	\\
97.9370680012758	157.0581	\\
98.2770680012759	157.0581	\\
98.6270680012761	157.0581	\\
98.9670680012763	157.0581	\\
99.3170680012765	157.0581	\\
99.6570680012766	157.0581	\\
100.007068001277	157.0581	\\
100.347068001277	157.0581	\\
100.697068001277	157.0581	\\
101.037068001277	157.0581	\\
101.387068001278	157.0581	\\
101.727068001278	157.0581	\\
102.077068001278	157.0581	\\
102.417068001278	157.0581	\\
102.767068001278	157.0581	\\
103.107068001278	157.0581	\\
103.457068001279	157.0581	\\
103.797068001279	157.0581	\\
104.147068001279	157.0581	\\
104.487068001279	157.0581	\\
104.837068001279	157.0581	\\
105.177068001279	157.0581	\\
105.52706800128	157.0581	\\
105.86706800128	157.0581	\\
106.20706800128	157.0581	\\
106.55706800128	157.0581	\\
106.89706800128	157.0581	\\
107.247068001281	157.0581	\\
107.587068001281	157.0581	\\
107.937068001281	157.0581	\\
108.277068001281	157.0581	\\
108.627068001281	157.0581	\\
108.967068001281	157.0581	\\
109.317068001282	157.0581	\\
109.657068001282	157.0581	\\
110.007068001282	157.0581	\\
110.347068001282	157.0581	\\
110.697068001282	157.0581	\\
111.037068001282	157.0581	\\
111.387068001283	157.0581	\\
111.727068001283	157.0581	\\
112.077068001283	157.0581	\\
112.417068001283	157.0581	\\
112.767068001283	157.0581	\\
113.107068001284	157.0581	\\
113.457068001284	157.0581	\\
113.797068001284	157.0581	\\
114.147068001284	157.0581	\\
114.487068001284	157.0581	\\
114.837068001284	157.0581	\\
115.177068001285	157.0581	\\
115.527068001285	157.0581	\\
115.867068001285	157.0581	\\
116.207068001285	157.0581	\\
116.557068001285	157.0581	\\
116.897068001285	157.0581	\\
117.247068001286	157.0581	\\
117.587068001286	157.0581	\\
117.937068001286	157.0581	\\
118.277068001286	157.0581	\\
118.627068001286	157.0581	\\
118.967068001287	157.0581	\\
119.317068001287	157.0581	\\
119.657068001287	157.0581	\\
120.007068001287	157.0581	\\
120.347068001287	157.0581	\\
120.697068001287	157.0581	\\
121.037068001288	157.0581	\\
121.387068001288	157.0581	\\
121.727068001288	157.0581	\\
122.077068001288	157.0581	\\
122.417068001288	157.0581	\\
122.767068001288	157.0581	\\
123.107068001289	157.0581	\\
123.457068001289	157.0581	\\
123.797068001289	157.0581	\\
124.147068001289	157.0581	\\
124.487068001289	157.0581	\\
124.837474866463	157.0581	\\
125.177474866462	157.0581	\\
125.517474866462	157.0581	\\
125.867474866462	157.0581	\\
126.207474866462	157.0581	\\
126.557474866463	157.0581	\\
126.897474866463	157.0581	\\
127.247474866463	157.0581	\\
127.587474866463	157.0581	\\
127.937474866463	157.0581	\\
128.277474866463	157.0581	\\
128.627474866463	157.0581	\\
128.967474866463	157.0581	\\
129.317474866462	157.0581	\\
129.657474866462	157.0581	\\
130.007474866462	157.0581	\\
130.347474866461	157.0581	\\
130.697474866461	157.0581	\\
131.037474866461	157.0581	\\
131.38747486646	157.0581	\\
131.72747486646	157.0581	\\
132.07747486646	157.0581	\\
132.417474866459	157.0581	\\
132.767474866459	157.0581	\\
133.107474866459	157.0581	\\
133.457474866458	157.0581	\\
133.797474866458	157.0581	\\
134.147474866458	157.0581	\\
134.487474866457	157.0581	\\
134.837474866457	157.0581	\\
135.177474866457	157.0581	\\
135.517474866457	157.0581	\\
135.867474866456	157.0581	\\
136.207474866456	157.0581	\\
136.557474866456	157.0581	\\
136.897474866455	157.0581	\\
137.247474866455	157.0581	\\
137.587474866455	157.0581	\\
137.937474866454	157.0581	\\
138.277474866454	157.0581	\\
138.627474866454	157.0581	\\
138.967474866453	157.0581	\\
139.317474866453	157.0581	\\
139.657474866453	157.0581	\\
140.007474866452	157.0581	\\
140.347474866452	157.0581	\\
140.697474866452	157.0581	\\
141.037474866452	157.0581	\\
141.387474866451	157.0581	\\
141.727474866451	157.0581	\\
142.077474866451	157.0581	\\
142.41747486645	157.0581	\\
142.76747486645	157.0581	\\
143.10747486645	157.0581	\\
143.457474866449	157.0581	\\
143.797474866449	157.0581	\\
144.147474866449	157.0581	\\
144.487474866448	157.0581	\\
144.837474866448	157.0581	\\
145.177474866448	157.0581	\\
145.517474866447	157.0581	\\
145.867474866447	157.0581	\\
146.207474866447	157.0581	\\
146.557474866447	157.0581	\\
146.897474866446	157.0581	\\
147.247474866446	157.0581	\\
147.587474866446	157.0581	\\
147.937474866445	157.0581	\\
148.277474866445	157.0581	\\
148.627474866445	157.0581	\\
148.967474866444	157.0581	\\
149.317474866444	157.0581	\\
149.657474866444	157.0581	\\
150.007474866443	157.0581	\\
150.347474866443	157.0581	\\
150.697474866443	157.0581	\\
151.037474866442	157.0581	\\
151.387474866442	157.0581	\\
151.727474866442	157.0581	\\
152.077474866441	157.0581	\\
152.417474866441	157.0581	\\
152.767474866441	157.0581	\\
153.107474866441	157.0581	\\
153.45747486644	157.0581	\\
153.79747486644	157.0581	\\
154.14747486644	157.0581	\\
154.487474866439	157.0581	\\
154.837474866439	157.0581	\\
155.179479601952	157.0581	\\
155.519479601952	157.0581	\\
155.869479601952	157.0581	\\
156.209479601951	157.0581	\\
156.559479601951	157.0581	\\
156.899479601951	157.0581	\\
157.24947960195	157.0581	\\
157.58947960195	157.0581	\\
157.93947960195	157.0581	\\
158.279479601949	157.0581	\\
158.629479601949	157.0581	\\
158.969479601949	157.0581	\\
159.319479601948	157.0581	\\
159.659479601948	157.0581	\\
160.009479601948	157.0581	\\
160.349479601947	157.0581	\\
160.699479601947	157.0581	\\
161.039479601947	157.0581	\\
161.379479601946	157.0581	\\
161.729479601946	157.0581	\\
162.069479601946	157.0581	\\
162.419479601946	157.0581	\\
162.759479601945	157.0581	\\
163.109479601945	157.0581	\\
163.449479601945	157.0581	\\
163.799479601944	157.0581	\\
164.139479601944	157.0581	\\
164.489479601944	157.0581	\\
164.829479601943	157.0581	\\
165.179479601943	157.0581	\\
165.519479601943	157.0581	\\
165.869479601942	157.0581	\\
166.209479601942	157.0581	\\
166.559479601942	157.0581	\\
166.899479601941	157.0581	\\
167.249479601941	157.0581	\\
167.589479601941	157.0581	\\
167.939479601941	157.0581	\\
168.27947960194	157.0581	\\
168.62947960194	157.0581	\\
168.96947960194	157.0581	\\
169.319479601939	157.0581	\\
169.659479601939	157.0581	\\
170.009479601939	157.0581	\\
170.349479601938	157.0581	\\
170.699479601938	157.0581	\\
171.039479601938	157.0581	\\
171.379479601937	157.0581	\\
171.729479601937	157.0581	\\
172.069479601937	157.0581	\\
172.419479601936	157.0581	\\
172.759479601936	157.0581	\\
173.109479601936	157.0581	\\
173.449479601936	157.0581	\\
173.799479601935	157.0581	\\
174.139479601935	157.0581	\\
174.489479601935	157.0581	\\
174.829479601934	157.0581	\\
175.179479601934	157.0581	\\
175.519479601934	157.0581	\\
175.869479601933	157.0581	\\
176.209479601933	157.0581	\\
176.559479601933	157.0581	\\
176.899479601932	157.0581	\\
177.249479601932	157.0581	\\
177.589479601932	157.0581	\\
177.939479601931	157.0581	\\
178.279479601931	157.0581	\\
178.629479601931	157.0581	\\
178.96947960193	157.0581	\\
179.31947960193	157.0581	\\
179.65947960193	157.0581	\\
180.00947960193	157.0581	\\
180.349479601929	157.0581	\\
180.699479601929	157.0581	\\
181.039479601929	157.0581	\\
181.379479601928	157.0581	\\
181.729479601928	157.0581	\\
182.069479601928	157.0581	\\
182.419479601927	157.0581	\\
182.759479601927	157.0581	\\
183.109479601927	157.0581	\\
183.449479601926	157.0581	\\
183.799479601926	157.0581	\\
184.139479601926	157.0581	\\
184.489479601925	157.0581	\\
184.829479601925	157.0581	\\
185.179479601925	157.0581	\\
185.519479601925	157.0581	\\
185.869479601924	157.0581	\\
186.209479601924	157.0581	\\
186.559479601924	157.0581	\\
186.899479601923	157.0581	\\
187.249479601923	157.0581	\\
187.589479601923	157.0581	\\
187.939479601922	157.0581	\\
188.279479601922	157.0581	\\
188.629479601922	157.0581	\\
188.969479601921	157.0581	\\
189.319479601921	157.0581	\\
189.659479601921	157.0581	\\
190.00947960192	157.0581	\\
190.34947960192	157.0581	\\
190.69947960192	157.0581	\\
191.03947960192	157.0581	\\
191.379479601919	157.0581	\\
191.729479601919	157.0581	\\
192.069479601919	157.0581	\\
192.419479601918	157.0581	\\
192.759479601918	157.0581	\\
193.109479601918	157.0581	\\
193.449479601917	157.0581	\\
193.799479601917	157.0581	\\
194.139479601917	157.0581	\\
194.489479601916	157.0581	\\
194.829479601916	157.0581	\\
195.179479601916	157.0581	\\
195.519479601915	157.0581	\\
195.869479601915	157.0581	\\
196.209479601915	157.0581	\\
196.559479601914	157.0581	\\
196.899479601914	157.0581	\\
197.249479601914	157.0581	\\
197.589479601914	157.0581	\\
197.939479601913	157.0581	\\
198.279479601913	157.0581	\\
198.629479601913	157.0581	\\
198.969479601912	157.0581	\\
199.319479601912	157.0581	\\
200	157.0581	\\
};
\addlegendentry{Rolling resistance F (N)};

\end{axis}

\begin{axis}[%
width=\figurewidth,
height=\figureheight,
scale only axis,
xmin=0,
xmax=200,
xmajorgrids,
ymin=0,
ymax=70,
ymajorgrids,
at=(plot1.below south west),
anchor=above north west,
legend style={draw=black,fill=white,legend cell align=left}
]
\addplot [color=blue,solid]
  table[row sep=crcr]{
0	0	\\
0.352229255786002	0.00265235801360643	\\
0.692229255786002	0.0139851527508855	\\
1.042229255786	0.0350460087648918	\\
1.382229255786	0.0646103122756839	\\
1.732229255786	0.104416395736268	\\
2.072229255786	0.152189553164524	\\
2.42222925578599	0.210739908016939	\\
2.76222925578599	0.276721666045433	\\
3.11222925578598	0.354016045101412	\\
3.45222925578597	0.43820616384441	\\
3.80222925578596	0.534244320169286	\\
4.14222925578596	0.63664255975009	\\
4.49222925578595	0.751424246414678	\\
4.83222925578594	0.872030366961856	\\
5.18222925578594	1.00555533704239	\\
5.52222925578593	1.14436909868978	\\
5.86222925578592	1.29215465324267	\\
6.21222925578591	1.45365826815462	\\
6.55222925578591	1.61965110575643	\\
6.9022292557859	1.79989738858783	\\
7.24222925578589	1.98409726938552	\\
7.59222925578588	2.18308597323154	\\
7.93222925578588	2.38549265737735	\\
8.28222925578587	2.60322353533857	\\
8.62222925578586	2.82383678298999	\\
8.97222925578586	3.06030958817244	\\
9.31222925578585	3.29912915949223	\\
9.66222925578584	3.55434364500734	\\
10.0000000000001	3.80965485974093	\\
10.3500000000001	4.08355022917707	\\
10.6900000000001	4.35872290918838	\\
11.0400000000001	4.65135946599391	\\
11.3800000000001	4.94473775096828	\\
11.7300000000001	5.25611524827088	\\
12.0700000000001	5.54582309444751	\\
12.4200000000001	5.57576478515624	\\
12.7600000000001	5.5761225594391	\\
13.11	5.57612742021457	\\
13.45	5.57612747821714	\\
13.8	5.57612747900516	\\
14.14	5.57612747901456	\\
14.35	5.57612747901468	\\
14.49	5.57612747901468	\\
14.83	5.57612747901468	\\
15.18	5.57612747901468	\\
15.52	5.57612747901468	\\
15.87	5.57612747901468	\\
16.21	5.57612747901468	\\
16.5600000000001	5.57612747901468	\\
16.9000000000001	5.57612747901468	\\
17.2500000000002	5.57612747901468	\\
17.5900000000002	5.57612747901468	\\
17.9400000000003	5.57612747901468	\\
18.2800000000003	5.57612747901468	\\
18.6300000000004	5.57612747901468	\\
18.9700000000004	5.57612747901468	\\
19.3200000000005	5.57612747901468	\\
19.6600000000006	5.57612747901468	\\
19.9999999999998	5.57612747901466	\\
20.3472841771288	5.52221926683331	\\
20.6972841771288	5.46772696428889	\\
21.0372841771289	5.41506762947242	\\
21.3872841771289	5.36114329679347	\\
21.727284177129	5.30903501455762	\\
22.077284177129	5.25567723611642	\\
22.4172841771291	5.20411863916739	\\
22.7672841771291	5.15132601555497	\\
23.1072841771292	5.10031575231567	\\
23.4572841771293	5.04808690026253	\\
23.7972841771293	4.99762363479605	\\
24.1472841771294	4.94595718709383	\\
24.4872841771294	4.89603959902793	\\
24.8372841771295	4.84493420445222	\\
25.1772841771295	4.79556098890471	\\
25.5172841771296	4.7464555396926	\\
25.8672841771296	4.69618517964387	\\
26.2072841771297	4.64762218374313	\\
26.5572841771297	4.5979095776391	\\
26.8972841771298	4.54988776846211	\\
27.2472841771298	4.50073162031427	\\
27.5872841771299	4.45324974651051	\\
27.93728417713	4.40464877597933	\\
28.27728417713	4.35770560136513	\\
28.6272841771301	4.30965854368818	\\
28.9672841771301	4.26325284717735	\\
29.3172841771302	4.21575845309837	\\
29.6572841771302	4.16988902863375	\\
30.0072841771303	4.12294606433267	\\
30.3472841771303	4.07761172081862	\\
30.6972841771304	4.03121896784257	\\
31.0372841771304	3.98641852907838	\\
31.3872841771305	3.94057478427354	\\
31.7272841771305	3.89630708888772	\\
32.0772841771306	3.85101116433222	\\
32.4172841771305	3.80727506571772	\\
32.7672841771304	3.76252578865533	\\
33.1072841771304	3.71932015490564	\\
33.4572841771303	3.67511636768051	\\
33.7972841771302	3.63244008152663	\\
34.1472841771302	3.58878064151886	\\
34.4872841771301	3.5466326002672	\\
34.83728417713	3.50351637982932	\\
35.17728417713	3.4618954953004	\\
35.5172841771299	3.4205339733654	\\
35.8672841771298	3.37822658016289	\\
36.2072841771298	3.33739061744027	\\
36.5572841771297	3.29562369763369	\\
36.8972841771296	3.25531224186315	\\
37.2472841771296	3.21408471961466	\\
37.5872841771295	3.17429673284289	\\
37.9372841771294	3.13360754701277	\\
38.2772841771293	3.09434200553605	\\
38.6272841771293	3.05419010962407	\\
38.9672841771292	3.0154460039317	\\
39.3172841771291	2.97583036601935	\\
39.6572841771291	2.93760670073799	\\
40	2.89933687872061	\\
40.3500000000004	2.86052195617325	\\
40.6900000000003	2.82307539955432	\\
41.0400000000002	2.78479422190631	\\
41.3800000000002	2.74786568191362	\\
41.7300000000001	2.71011726779218	\\
42.07	2.67370579806024	\\
42.42	2.63648918042514	\\
42.7599999999999	2.60059384848555	\\
43.1099999999998	2.56390807457588	\\
43.4499999999998	2.52852796180609	\\
43.7999999999997	2.49237209308804	\\
44.1399999999996	2.45750629466101	\\
44.4899999999996	2.42187940677637	\\
44.8299999999995	2.38752703161098	\\
45.1799999999994	2.35242821432654	\\
45.5199999999994	2.31858838503878	\\
45.8699999999993	2.28401674219652	\\
46.2099999999992	2.25068859505153	\\
46.5599999999992	2.21664324451959	\\
46.8999999999991	2.18382592938444	\\
47.249999999999	2.150306003009	\\
47.5899999999989	2.11799868330626	\\
47.9399999999989	2.08500332686405	\\
48.2799999999988	2.05320517952617	\\
48.6299999999987	2.02073355267799	\\
48.8564934895424	2.00054843020666	\\
48.9664934895418	2.05609551147382	\\
49.3164934895418	2.42037889045658	\\
49.6564934895417	2.82474686577199	\\
50.0064934895416	3.27391849193	\\
50.3464934895416	3.74197942329065	\\
50.6964934895415	4.25645861824641	\\
51.0364934895414	4.78795683096445	\\
51.3864934895414	5.36773825814162	\\
51.7264934895413	5.96267214442989	\\
52.0764934895412	6.607754197431	\\
52.4164934895411	7.26612219748864	\\
52.7664934895411	7.97650327067822	\\
53.106493489541	8.69830382477712	\\
53.4564934895409	9.47398231258562	\\
53.7964934895409	10.2592138610616	\\
54.1464934895408	11.1001881579853	\\
54.4864934895407	11.9488491412382	\\
54.8364934895407	12.855117641839	\\
55.1764934895406	13.7672065003324	\\
55.5264934895405	14.7387675992383	\\
55.8664934895405	15.7142827734997	\\
56.2164934895404	16.7511348654043	\\
56.5564934895403	17.7900747960252	\\
56.9064934895403	18.8922162756879	\\
57.2464934895402	19.9945794033237	\\
57.5864934895401	21.1281939676067	\\
57.93649348954	22.3277934309394	\\
58.27649348954	23.5248288646389	\\
58.6264934895399	24.7897137145467	\\
58.9664934895398	26.0501684584771	\\
59.3164934895398	27.3803370899491	\\
59.6564934895397	28.704209584989	\\
60	30.0735073080414	\\
60.3500000000004	30.3892804853671	\\
60.6900000000003	30.3930567951855	\\
61.0400000000002	30.3931080789954	\\
61.3800000000002	30.3931086906786	\\
61.7300000000001	30.3931086989852	\\
62.07	30.3931086990843	\\
62.33	30.3931086990856	\\
62.42	30.3931086990856	\\
62.7599999999999	30.3931086990856	\\
63.1099999999998	30.3931086990856	\\
63.4499999999998	30.3931086990856	\\
63.7999999999997	30.3931086990856	\\
64.1399999999997	30.3931086990856	\\
64.4899999999999	30.3931086990856	\\
64.8300000000001	30.3931086990856	\\
65.1800000000003	30.3931086990856	\\
65.5200000000004	30.3931086990856	\\
65.8700000000006	30.3931086990856	\\
66.2100000000008	30.3931086990856	\\
66.560000000001	30.3931086990856	\\
66.9000000000011	30.3931086990856	\\
67.2500000000013	30.3931086990856	\\
67.5900000000015	30.3931086990856	\\
67.9400000000017	30.3931086990856	\\
68.2800000000018	30.3931086990856	\\
68.630000000002	30.3931086990856	\\
68.9700000000022	30.3931086990856	\\
69.3200000000024	30.3931086990856	\\
69.6600000000026	30.3931086990856	\\
70.0000000000027	30.3931086990856	\\
70.3500000000029	30.3931086990856	\\
70.6900000000031	30.3931086990856	\\
71.0400000000033	30.3931086990856	\\
71.3800000000034	30.3931086990856	\\
71.7300000000036	30.3931086990856	\\
72.0700000000038	30.3931086990856	\\
72.420000000004	30.3931086990856	\\
72.7600000000041	30.3931086990856	\\
73.1100000000043	30.3931086990856	\\
73.4500000000045	30.3931086990856	\\
73.8000000000047	30.3931086990856	\\
74.1400000000048	30.3931086990856	\\
74.490000000005	30.3931086990856	\\
74.8300000000052	30.3931086990856	\\
75.1800000000054	30.3931086990856	\\
75.5200000000056	30.3931086990856	\\
75.8700000000057	30.3931086990856	\\
76.2100000000059	30.3931086990856	\\
76.5600000000061	30.3931086990856	\\
76.9000000000063	30.3931086990856	\\
77.2500000000064	30.3931086990856	\\
77.5900000000066	30.3931086990856	\\
77.9400000000068	30.3931086990856	\\
78.280000000007	30.3931086990856	\\
78.6300000000071	30.3931086990856	\\
78.9700000000073	30.3931086990856	\\
79.3200000000075	30.3931086990856	\\
79.6600000000077	30.3931086990856	\\
79.9999999999991	30.3931086990848	\\
80.3470680012668	30.2479770828164	\\
80.6970680012669	30.1008952356188	\\
81.0370680012671	29.958469043552	\\
81.3870680012673	29.8123193120207	\\
81.7270680012675	29.6707963206488	\\
82.0770680012676	29.525574014312	\\
82.4170680012678	29.3849496865059	\\
82.767068001268	29.2406501546016	\\
83.1070680012682	29.1009199916429	\\
83.4570680012684	28.9575386227814	\\
83.7970680012685	28.8186981640357	\\
84.1470680012687	28.6762303858639	\\
84.4870680012689	28.5382752084571	\\
84.8370680012691	28.3967164873282	\\
85.1770680012692	28.259642205827	\\
85.5270680012694	28.1189880464728	\\
85.8670680012696	27.9827903125684	\\
86.2070680012698	27.8470229127092	\\
86.5570680012699	27.7077107599711	\\
86.8970680012701	27.5728134851496	\\
87.2470680012703	27.4343948535613	\\
87.5870680012705	27.3003634466564	\\
87.9370680012706	27.1628339724799	\\
88.2770680012708	27.0296642124174	\\
88.627068001271	26.893019568866	\\
88.9670680012712	26.7607072703173	\\
89.3170680012714	26.6249431672475	\\
89.6570680012715	26.493484180332	\\
90.0070680012717	26.3585963639394	\\
90.3470680012719	26.2279865739301	\\
90.6970680012721	26.0939708264471	\\
91.0370680012722	25.9642061534792	\\
91.3870680012724	25.8310582928773	\\
91.7270680012726	25.7021346916602	\\
92.0770680012728	25.5698505713548	\\
92.4170680012729	25.4417640308873	\\
92.7670680012731	25.3103395394462	\\
93.1070680012733	25.1830860827344	\\
93.4570680012735	25.0525171435892	\\
93.7970680012736	24.9260928273679	\\
94.1470680012738	24.7963753985286	\\
94.487068001274	24.6707763129854	\\
94.8370680012742	24.5419063867579	\\
95.1770680012744	24.4171286552609	\\
95.5270680012745	24.2891022579681	\\
95.8670680012747	24.1651420367955	\\
96.2070680012749	24.0415832431397	\\
96.5570680012751	23.9148087065747	\\
96.8970680012752	23.7920617461676	\\
97.2470680012754	23.6661209794309	\\
97.5870680012756	23.5441820894329	\\
97.9370680012758	23.4190712355122	\\
98.2770680012759	23.2979366850748	\\
98.6270680012761	23.1736519197572	\\
98.9670680012763	23.0533180097657	\\
99.3170680012765	22.9298555413745	\\
99.6570680012766	22.8103186041945	\\
100.007068001277	22.687674673329	\\
100.347068001277	22.5689310725545	\\
100.697068001277	22.447101951833	\\
101.037068001277	22.3291480820382	\\
101.387068001278	22.2081300758431	\\
101.727068001278	22.0909623623362	\\
102.077068001278	21.9707518065626	\\
102.417068001278	21.8543667051424	\\
102.767068001278	21.7349599669488	\\
103.107068001278	21.6193539636647	\\
103.457068001279	21.5007474412264	\\
103.797068001279	21.3859170521398	\\
104.147068001279	21.2681071744057	\\
104.487068001279	21.1540489453547	\\
104.837068001279	21.0370321718059	\\
105.177068001279	20.9237426781723	\\
105.52706800128	20.8075154985839	\\
105.86706800128	20.694991345063	\\
106.20706800128	20.582843075828	\\
106.55706800128	20.4677880996403	\\
106.89706800128	20.3564001562101	\\
107.247068001281	20.2421261542024	\\
107.587068001281	20.1314952133565	\\
107.937068001281	20.0179987792782	\\
108.277068001281	19.908121546324	\\
108.627068001281	19.7953993031785	\\
108.967068001281	19.6862725117321	\\
109.317068001282	19.5743211115517	\\
109.657068001282	19.4659415233205	\\
110.007068001282	19.3547576469446	\\
110.347068001282	19.2471220515129	\\
110.697068001282	19.1367024083677	\\
111.037068001282	19.029807622984	\\
111.387068001283	18.9201489508655	\\
111.727068001283	18.8139918202327	\\
112.077068001283	18.705090885091	\\
112.417068001283	18.5996682811583	\\
112.767068001283	18.4915218768854	\\
113.107068001284	18.3868306986427	\\
113.457068001284	18.2794356468615	\\
113.797068001284	18.175472820136	\\
114.147068001284	18.0688259699924	\\
114.487068001284	17.9655884472475	\\
114.837068001284	17.859686675204	\\
115.177068001285	17.7571714353403	\\
115.527068001285	17.6520116449728	\\
115.867068001285	17.5502156931308	\\
116.207068001285	17.4487730795856	\\
116.557068001285	17.3447151822924	\\
116.897068001285	17.2439874345814	\\
117.247068001286	17.1406639170632	\\
117.587068001286	17.0406481048942	\\
117.937068001286	16.9380559638584	\\
118.277068001286	16.8387491825082	\\
118.627068001286	16.7368854408865	\\
118.967068001287	16.6382848110135	\\
119.317068001287	16.5371465177697	\\
119.657068001287	16.4392491852283	\\
120.007068001287	16.338833415169	\\
120.347068001287	16.241636550826	\\
120.697068001287	16.1419404044119	\\
121.037068001288	16.0454412039652	\\
121.387068001288	15.946461807126	\\
121.727068001288	15.850657490925	\\
122.077068001288	15.7523919948749	\\
122.417068001288	15.6572798077428	\\
122.767068001288	15.5597253887993	\\
123.107068001289	15.4653025998574	\\
123.457068001289	15.368456459261	\\
123.797068001289	15.274720361755	\\
124.147068001289	15.1785797254912	\\
124.487068001289	15.0855276366193	\\
124.837474866463	14.9919945895883	\\
125.177474866462	15.811796630308	\\
125.517474866462	16.8916749134798	\\
125.867474866462	18.0440167327928	\\
126.207474866462	19.1998810551918	\\
126.557474866463	20.4272158998425	\\
126.897474866463	21.6558868345363	\\
127.247474866463	22.9581681166876	\\
127.587474866463	24.259643195817	\\
127.937474866463	25.6368689336067	\\
128.277474866463	27.011146239367	\\
128.627474866463	28.463314458723	\\
128.967474866463	29.9103920734898	\\
129.317474866462	31.4375008004321	\\
129.657474866462	32.957376806666	\\
130.007474866462	34.5594240672993	\\
130.347474866461	36.1520965475441	\\
130.697474866461	37.8290803680599	\\
131.037474866461	39.4945474049435	\\
131.38747486646	41.24646581162	\\
131.72747486646	42.9847254878547	\\
132.07747486646	44.8115765070563	\\
132.417474866459	46.6226269054385	\\
132.767474866459	48.5244085636167	\\
133.107474866459	50.4082477670269	\\
133.457474866458	52.3849580907196	\\
133.797474866458	54.3415841821224	\\
134.147474866458	56.3932211979543	\\
134.487474866457	58.4226322603985	\\
134.837474866457	60.5491939950807	\\
135.177474866457	61.9872076908811	\\
135.517474866457	62.0384378455684	\\
135.867474866456	62.0391334667516	\\
136.207474866456	62.0391417609326	\\
136.557474866456	62.0391418735304	\\
136.897474866455	62.0391418748729	\\
137.247474866455	62.0391418748912	\\
137.337474866455	62.0391418748913	\\
137.587474866455	62.0391418748913	\\
137.937474866454	62.0391418748913	\\
138.277474866454	62.0391418748913	\\
138.627474866454	62.0391418748913	\\
138.967474866453	62.0391418748913	\\
139.317474866453	62.0391418748913	\\
139.657474866453	62.0391418748913	\\
140.007474866452	62.0391418748913	\\
140.347474866452	62.0391418748913	\\
140.697474866452	62.0391418748913	\\
141.037474866452	62.0391418748913	\\
141.387474866451	62.0391418748913	\\
141.727474866451	62.0391418748913	\\
142.077474866451	62.0391418748913	\\
142.41747486645	62.0391418748913	\\
142.76747486645	62.0391418748913	\\
143.10747486645	62.0391418748913	\\
143.457474866449	62.0391418748913	\\
143.797474866449	62.0391418748913	\\
144.147474866449	62.0391418748913	\\
144.487474866448	62.0391418748913	\\
144.837474866448	62.0391418748913	\\
145.177474866448	62.0391418748913	\\
145.517474866447	62.0391418748913	\\
145.867474866447	62.0391418748913	\\
146.207474866447	62.0391418748913	\\
146.557474866447	62.0391418748913	\\
146.897474866446	62.0391418748913	\\
147.247474866446	62.0391418748913	\\
147.587474866446	62.0391418748913	\\
147.937474866445	62.0391418748913	\\
148.277474866445	62.0391418748913	\\
148.627474866445	62.0391418748913	\\
148.967474866444	62.0391418748913	\\
149.317474866444	62.0391418748913	\\
149.657474866444	62.0391418748913	\\
150.007474866443	62.0391418748913	\\
150.347474866443	62.0391418748913	\\
150.697474866443	62.0391418748913	\\
151.037474866442	62.0391418748913	\\
151.387474866442	62.0391418748913	\\
151.727474866442	62.0391418748913	\\
152.077474866441	62.0391418748913	\\
152.417474866441	62.0391418748913	\\
152.767474866441	62.0391418748913	\\
153.107474866441	62.0391418748913	\\
153.45747486644	62.0391418748913	\\
153.79747486644	62.0391418748913	\\
154.14747486644	62.0391418748913	\\
154.487474866439	62.0391418748913	\\
154.837474866439	62.0391418748913	\\
155.179479601952	61.9192968863709	\\
155.519479601952	61.6802820595075	\\
155.869479601952	61.4349916115519	\\
156.209479601951	61.1974397892185	\\
156.559479601951	60.9536505239951	\\
156.899479601951	60.7175523018826	\\
157.24947960195	60.4752545821424	\\
157.58947960195	60.2406006405457	\\
157.93947960195	59.9997849156857	\\
158.279479601949	59.7665660186389	\\
158.629479601949	59.5272228238174	\\
158.969479601949	59.29542981824	\\
159.319479601948	59.0575497735031	\\
159.659479601948	58.8271735883563	\\
160.009479601948	58.5907473977746	\\
160.349479601947	58.3617790432284	\\
160.699479601947	58.1267974940443	\\
161.039479601947	57.8992280606538	\\
161.379479601946	57.6723446983649	\\
161.729479601946	57.4395026803323	\\
162.069479601946	57.2140050873937	\\
162.419479601946	56.9825851022303	\\
162.759479601945	56.7584645660591	\\
163.109479601945	56.5284576849648	\\
163.449479601945	56.3057055705663	\\
163.799479601944	56.0771029442085	\\
164.139479601944	55.8557106933958	\\
164.489479601944	55.6285035511119	\\
164.829479601943	55.4084626817367	\\
165.179479601943	55.1826423307461	\\
165.519479601943	54.9639444359378	\\
165.869479601942	54.7395022605637	\\
166.209479601942	54.5221390079779	\\
166.559479601942	54.2990664688783	\\
166.899479601941	54.083029599954	\\
167.249479601941	53.8613182333617	\\
167.589479601941	53.6465995625882	\\
167.939479601941	53.4262409795605	\\
168.27947960194	53.2128323937519	\\
168.62947960194	52.9938182794284	\\
168.96947960194	52.7817117370078	\\
169.319479601939	52.5640338498781	\\
169.659479601939	52.3532213801696	\\
170.009479601939	52.1368715513488	\\
170.349479601938	51.9273452538788	\\
170.699479601938	51.7123153863918	\\
171.039479601938	51.5040674301977	\\
171.379479601937	51.2964465222882	\\
171.729479601937	51.08337212122	\\
172.069479601937	50.8770179927393	\\
172.419479601936	50.6652436776975	\\
172.759479601936	50.460148668002	\\
173.109479601936	50.249666587683	\\
173.449479601936	50.045823103291	\\
173.799479601935	49.8366254751897	\\
174.139479601935	49.6340259891256	\\
174.489479601935	49.4261050988655	\\
174.829479601934	49.2247421500133	\\
175.179479601934	49.0180903506842	\\
175.519479601934	48.8179565431475	\\
175.869479601933	48.6125662546508	\\
176.209479601933	48.413654257121	\\
176.559479601933	48.209517965523	\\
176.899479601932	48.0118205106544	\\
177.249479601932	47.8089307675471	\\
177.589479601932	47.6124406513401	\\
177.939479601931	47.4107900732095	\\
178.279479601931	47.2155001544	\\
178.629479601931	47.0150814220017	\\
178.96947960193	46.8209846214583	\\
179.31947960193	46.6217904792006	\\
179.65947960193	46.4288797793291	\\
180.00947960193	46.230903034663	\\
180.349479601929	46.0391714788168	\\
180.699479601929	45.8424050016334	\\
181.039479601929	45.6518456935321	\\
181.379479601928	45.4618614967349	\\
181.729479601928	45.2668885187203	\\
182.069479601928	45.0780663862884	\\
182.419479601927	44.8842861701046	\\
182.759479601927	44.6966193537447	\\
183.109479601927	44.5040249827413	\\
183.449479601926	44.3175067925456	\\
183.799479601926	44.1260914098901	\\
184.139479601926	43.9407152137835	\\
184.489479601925	43.7504720218955	\\
184.829479601925	43.5662312450903	\\
185.179479601925	43.3771535050827	\\
185.519479601925	43.1940416295385	\\
185.869479601924	43.0061226606655	\\
186.209479601924	42.8241332245558	\\
186.559479601924	42.6373664036667	\\
186.899479601923	42.4564930008519	\\
187.249479601923	42.2708717618517	\\
187.589479601923	42.0911080413576	\\
187.939479601922	41.9066258746731	\\
188.279479601922	41.7279655401761	\\
188.629479601922	41.5446159922291	\\
188.969479601921	41.3670528015463	\\
189.319479601921	41.1848294742319	\\
189.659479601921	41.008357238818	\\
190.00947960192	40.8272537889896	\\
190.34947960192	40.651866373439	\\
190.69947960192	40.4718765123987	\\
191.03947960192	40.2975678339536	\\
191.379479601919	40.1237884859687	\\
191.729479601919	39.9454493550128	\\
192.069479601919	39.7727397609755	\\
192.419479601918	39.5954987763951	\\
192.759479601918	39.4238529826801	\\
193.109479601918	39.2477040420395	\\
193.449479601917	39.0771161459943	\\
193.799479601917	38.9020531990886	\\
194.139479601917	38.7325173485661	\\
194.489479601916	38.5585343969428	\\
194.829479601916	38.3900447898396	\\
195.179479601916	38.2171358863258	\\
195.519479601915	38.049686770125	\\
195.869479601915	37.8778460183597	\\
196.209479601915	37.7114316896796	\\
196.559479601914	37.5406532436518	\\
196.899479601914	37.3752680477991	\\
197.249479601914	37.2055461113903	\\
197.589479601914	37.0411844419189	\\
197.939479601913	36.8725132684514	\\
198.279479601913	36.7091695667257	\\
198.629479601913	36.5415434585161	\\
198.969479601912	36.3792122132797	\\
199.319479601912	36.2126255211966	\\
199.659479601912	36.0513012681454	\\
200	35.8902254331572	\\
};
\addlegendentry{Aerodynamic drag F (N)};

\end{axis}
\end{tikzpicture}%
    	\label{fig:resistance-forces}
    	\caption{The resistance forces working on the car.}
\end{figure}
\begin{figure}[H]
	\centering
    	\setlength\figureheight{4cm}
    	\setlength\figurewidth{0.8\linewidth}
    	% This file was created by matlab2tikz v0.4.6 running on MATLAB 8.2.
% Copyright (c) 2008--2014, Nico Schlömer <nico.schloemer@gmail.com>
% All rights reserved.
% Minimal pgfplots version: 1.3
% 
% The latest updates can be retrieved from
%   http://www.mathworks.com/matlabcentral/fileexchange/22022-matlab2tikz
% where you can also make suggestions and rate matlab2tikz.
% 
\begin{tikzpicture}

\begin{axis}[%
width=\figurewidth,
height=\figureheight,
scale only axis,
xmin=0,
xmax=200,
xmajorgrids,
ymin=0,
ymax=80,
ymajorgrids,
name=plot1,
legend style={draw=black,fill=white,legend cell align=left}
]
\addplot [color=blue,solid]
  table[row sep=crcr]{
0	0	\\
0	0	\\
0.462229255786002	31.8275463130679	\\
0.462229255786002	31.8275463130679	\\
0.922229255786003	31.926613508343	\\
0.922229255786003	31.926613508343	\\
1.392229255786	31.9279310313791	\\
1.392229255786	31.9279310313791	\\
1.852229255786	31.929407020684	\\
1.852229255786	31.929407020684	\\
2.312229255786	31.9313498458603	\\
2.312229255786	31.9313498458603	\\
2.77222925578599	31.9337602664406	\\
2.77222925578599	31.9337602664406	\\
3.23222925578598	31.9366382805886	\\
3.23222925578598	31.9366382805886	\\
3.69222925578597	31.9399838842043	\\
3.69222925578597	31.9399838842043	\\
4.15222925578596	31.9437970731808	\\
4.15222925578596	31.9437970731808	\\
4.61222925578595	31.948077843412	\\
4.61222925578595	31.948077843412	\\
5.07222925578594	31.9528261907909	\\
5.07222925578594	31.9528261907909	\\
5.53222925578593	31.9580421112111	\\
5.53222925578593	31.9580421112111	\\
5.99222925578592	31.9637256005669	\\
5.99222925578592	31.9637256005669	\\
6.45222925578591	31.9698766547507	\\
6.45222925578591	31.9698766547507	\\
6.9222292557859	31.9766443452223	\\
6.9222292557859	31.9766443452223	\\
7.38222925578589	31.9837406809733	\\
7.38222925578589	31.9837406809733	\\
7.84222925578588	31.9913045691438	\\
7.84222925578588	31.9913045691438	\\
8.30222925578587	31.9993360056299	\\
8.30222925578587	31.9993360056299	\\
8.76222925578586	32.0078349863242	\\
8.76222925578586	32.0078349863242	\\
9.22222925578585	32.0168015071203	\\
9.22222925578585	32.0168015071203	\\
9.68222925578584	32.026235563912	\\
9.68222925578584	32.026235563912	\\
10.1400000000001	32.0360880400528	\\
10.1400000000001	32.0360880400528	\\
10.6000000000001	32.0464548907933	\\
10.6000000000001	32.0464548907933	\\
11.0600000000001	32.0572892652318	\\
11.0600000000001	32.0572892652318	\\
11.5300000000001	32.0688420447879	\\
11.5300000000001	32.0688420447879	\\
11.9900000000001	32.0806216176432	\\
12.0000000000001	32.0808828874206	\\
12.4500000000001	4.72303452650369	\\
12.4500000000001	4.72303452650369	\\
12.9100000000001	4.63069081683098	\\
12.9100000000001	4.63069081683098	\\
13.37	4.63041626271821	\\
13.37	4.63041626271821	\\
13.83	4.63041544642184	\\
13.83	4.63041544642184	\\
14.29	4.63041544399534	\\
14.29	4.63041544399534	\\
14.35	4.63041544399001	\\
14.75	4.63041544399001	\\
14.75	4.63041544399001	\\
15.21	4.63041544399001	\\
15.21	4.63041544399001	\\
15.67	4.63041544399001	\\
15.67	4.63041544399001	\\
16.13	4.63041544399001	\\
16.13	4.63041544399001	\\
16.5900000000001	4.63041544399001	\\
16.5900000000001	4.63041544399001	\\
17.0600000000001	4.63041544399001	\\
17.0600000000001	4.63041544399001	\\
17.5200000000002	4.63041544399001	\\
17.5200000000002	4.63041544399001	\\
17.9800000000003	4.63041544399001	\\
17.9800000000003	4.63041544399001	\\
18.4400000000004	4.63041544399001	\\
18.4400000000004	4.63041544399001	\\
18.9000000000004	4.63041544399001	\\
18.9000000000004	4.63041544399001	\\
19.3600000000005	4.63041544399001	\\
19.3600000000005	4.63041544399001	\\
19.9999999999998	4.630415443998	\\
20.0057613923765	0	\\
20.2772841771288	0	\\
20.2772841771288	0	\\
20.7472841771288	0	\\
20.7472841771288	0	\\
21.2072841771289	0	\\
21.2072841771289	0	\\
21.667284177129	0	\\
21.667284177129	0	\\
22.127284177129	0	\\
22.127284177129	0	\\
22.5872841771291	0	\\
22.5872841771291	0	\\
23.0472841771292	0	\\
23.0472841771292	0	\\
23.5072841771293	0	\\
23.5072841771293	0	\\
23.9672841771293	0	\\
23.9672841771293	0	\\
24.4272841771294	0	\\
24.4272841771294	0	\\
24.8872841771295	0	\\
24.8872841771295	0	\\
25.3472841771296	0	\\
25.3472841771296	0	\\
25.8072841771296	0	\\
25.8072841771296	0	\\
26.2672841771297	0	\\
26.2672841771297	0	\\
26.7372841771298	0	\\
26.7372841771298	0	\\
27.1972841771298	0	\\
27.1972841771298	0	\\
27.6572841771299	0	\\
27.6572841771299	0	\\
28.11728417713	0	\\
28.11728417713	0	\\
28.5772841771301	0	\\
28.5772841771301	0	\\
29.0372841771301	0	\\
29.0372841771301	0	\\
29.4972841771302	0	\\
29.4972841771302	0	\\
29.9572841771303	0	\\
29.9572841771303	0	\\
30.4172841771303	0	\\
30.4172841771303	0	\\
30.8772841771304	0	\\
30.8772841771304	0	\\
31.3372841771305	0	\\
31.3372841771305	0	\\
31.7972841771306	0	\\
31.7972841771306	0	\\
32.2672841771305	0	\\
32.2672841771305	0	\\
32.7272841771305	0	\\
32.7272841771305	0	\\
33.1872841771304	0	\\
33.1872841771304	0	\\
33.6472841771303	0	\\
33.6472841771303	0	\\
34.1072841771302	0	\\
34.1072841771302	0	\\
34.5672841771301	0	\\
34.5672841771301	0	\\
35.02728417713	0	\\
35.02728417713	0	\\
35.4872841771299	0	\\
35.4872841771299	0	\\
35.9472841771298	0	\\
35.9472841771298	0	\\
36.4072841771297	0	\\
36.4072841771297	0	\\
36.8672841771296	0	\\
36.8672841771296	0	\\
37.3272841771295	0	\\
37.3272841771295	0	\\
37.7972841771294	0	\\
37.7972841771294	0	\\
38.2572841771294	0	\\
38.2572841771294	0	\\
38.7172841771293	0	\\
38.7172841771293	0	\\
39.1772841771292	0	\\
39.1772841771292	0	\\
39.6372841771291	0	\\
39.6372841771291	0	\\
40.1000000000004	0	\\
40.1000000000004	0	\\
40.5600000000003	0	\\
40.5600000000003	0	\\
41.0200000000003	0	\\
41.0200000000003	0	\\
41.4800000000002	0	\\
41.4800000000002	0	\\
41.9400000000001	0	\\
41.9400000000001	0	\\
42.4	0	\\
42.4	0	\\
42.8599999999999	0	\\
42.8599999999999	0	\\
43.3199999999998	0	\\
43.3199999999998	0	\\
43.7799999999997	0	\\
43.7799999999997	0	\\
44.2399999999996	0	\\
44.2399999999996	0	\\
44.7099999999995	0	\\
44.7099999999995	0	\\
45.1699999999994	0	\\
45.1699999999994	0	\\
45.6299999999993	0	\\
45.6299999999993	0	\\
46.0899999999992	0	\\
46.0899999999992	0	\\
46.5499999999992	0	\\
46.5499999999992	0	\\
47.0099999999991	0	\\
47.0099999999991	0	\\
47.469999999999	0	\\
47.469999999999	0	\\
47.9299999999989	0	\\
47.9299999999989	0	\\
48.3899999999988	0	\\
48.8564934895424	6.62636837364472	\\
48.8564934895424	6.62636837364472	\\
49.3164934895418	55.6405136128539	\\
49.3164934895418	55.6405136128539	\\
49.7764934895417	55.8017428643818	\\
49.7764934895417	55.8017428643818	\\
50.2364934895416	55.819323798294	\\
50.2364934895416	55.819323798294	\\
50.6964934895415	55.8381018298775	\\
50.6964934895415	55.8381018298775	\\
51.1564934895414	55.8585076419296	\\
51.1564934895414	55.8585076419296	\\
51.6164934895413	55.8805424884543	\\
51.6164934895413	55.8805424884543	\\
52.0764934895412	55.9042063465567	\\
52.0764934895412	55.9042063465567	\\
52.5364934895411	55.9294991895429	\\
52.5364934895411	55.9294991895429	\\
52.996493489541	55.9564209907126	\\
52.996493489541	55.9564209907126	\\
53.4564934895409	55.98497172336	\\
53.4564934895409	55.98497172336	\\
53.9264934895408	56.0158255300518	\\
53.9264934895408	56.0158255300518	\\
54.3864934895408	56.0476694556558	\\
54.3864934895408	56.0476694556558	\\
54.8464934895407	56.0811422320562	\\
54.8464934895407	56.0811422320562	\\
55.3064934895406	56.1162438325589	\\
55.3064934895406	56.1162438325589	\\
55.7664934895405	56.1529742304616	\\
55.7664934895405	56.1529742304616	\\
56.2264934895404	56.1913333990685	\\
56.2264934895404	56.1913333990685	\\
56.6864934895403	56.2313213116834	\\
56.6864934895403	56.2313213116834	\\
57.1464934895402	56.2729379416078	\\
57.1464934895402	56.2729379416078	\\
57.6064934895401	56.3161832621493	\\
57.6064934895401	56.3161832621493	\\
58.06649348954	56.3610572466082	\\
58.06649348954	56.3610572466082	\\
58.5264934895399	56.4075598682994	\\
58.5264934895399	56.4075598682994	\\
58.9864934895398	56.4556911005223	\\
58.9864934895398	56.4556911005223	\\
59.4564934895397	56.5065507383427	\\
59.4564934895397	56.5065507383427	\\
59.9164934895397	56.5579745146803	\\
59.9964934895396	56.5670840202266	\\
60.3700000000004	5.81217104616805	\\
60.3700000000004	5.81217104616805	\\
60.8300000000003	5.33840028081123	\\
60.8300000000003	5.33840028081123	\\
61.3000000000002	5.33699201251103	\\
61.3000000000002	5.33699201251103	\\
61.7600000000001	5.33698832647644	\\
61.7600000000001	5.33698832647644	\\
62.22	5.33698831552165	\\
62.22	5.33698831552165	\\
62.33	5.33698831549678	\\
62.6799999999999	5.33698831549678	\\
62.6799999999999	5.33698831549678	\\
63.1399999999998	5.33698831549678	\\
63.1399999999998	5.33698831549678	\\
63.5999999999997	5.33698831549678	\\
63.5999999999997	5.33698831549678	\\
64.0599999999997	5.33698831549678	\\
64.0599999999997	5.33698831549678	\\
64.5199999999999	5.33698831549678	\\
64.5199999999999	5.33698831549678	\\
64.9800000000002	5.33698831549678	\\
64.9800000000002	5.33698831549678	\\
65.4400000000004	5.33698831549678	\\
65.4400000000004	5.33698831549678	\\
65.9000000000006	5.33698831549678	\\
65.9000000000006	5.33698831549678	\\
66.3600000000009	5.33698831549678	\\
66.3600000000009	5.33698831549678	\\
66.8300000000011	5.33698831549678	\\
66.8300000000011	5.33698831549678	\\
67.2900000000013	5.33698831549678	\\
67.2900000000013	5.33698831549678	\\
67.7500000000016	5.33698831549678	\\
67.7500000000016	5.33698831549678	\\
68.2100000000018	5.33698831549678	\\
68.2100000000018	5.33698831549678	\\
68.670000000002	5.33698831549678	\\
68.670000000002	5.33698831549678	\\
69.1300000000023	5.33698831549678	\\
69.1300000000023	5.33698831549678	\\
69.5900000000025	5.33698831549678	\\
69.5900000000025	5.33698831549678	\\
70.0500000000028	5.33698831549678	\\
70.0500000000028	5.33698831549678	\\
70.510000000003	5.33698831549678	\\
70.510000000003	5.33698831549678	\\
70.9700000000032	5.33698831549678	\\
70.9700000000032	5.33698831549678	\\
71.4300000000035	5.33698831549678	\\
71.4300000000035	5.33698831549678	\\
71.8900000000037	5.33698831549678	\\
71.8900000000037	5.33698831549678	\\
72.3600000000039	5.33698831549678	\\
72.3600000000039	5.33698831549678	\\
72.8200000000042	5.33698831549678	\\
72.8200000000042	5.33698831549678	\\
73.2800000000044	5.33698831549678	\\
73.2800000000044	5.33698831549678	\\
73.7400000000046	5.33698831549678	\\
73.7400000000046	5.33698831549678	\\
74.2000000000049	5.33698831549678	\\
74.2000000000049	5.33698831549678	\\
74.6600000000051	5.33698831549678	\\
74.6600000000051	5.33698831549678	\\
75.1200000000053	5.33698831549678	\\
75.1200000000053	5.33698831549678	\\
75.5800000000056	5.33698831549678	\\
75.5800000000056	5.33698831549678	\\
76.0400000000058	5.33698831549678	\\
76.0400000000058	5.33698831549678	\\
76.5000000000061	5.33698831549678	\\
76.5000000000061	5.33698831549678	\\
76.9600000000063	5.33698831549678	\\
76.9600000000063	5.33698831549678	\\
77.4200000000065	5.33698831549678	\\
77.4200000000065	5.33698831549678	\\
77.8900000000068	5.33698831549678	\\
77.8900000000068	5.33698831549678	\\
78.350000000007	5.33698831549678	\\
78.350000000007	5.33698831549678	\\
78.8100000000072	5.33698831549678	\\
78.8100000000072	5.33698831549678	\\
79.2700000000075	5.33698831549678	\\
79.2700000000075	5.33698831549678	\\
79.9999999999991	5.3369883156229	\\
80.0056893337565	0	\\
80.1870680012667	0	\\
80.1870680012667	0	\\
80.6470680012669	0	\\
80.6470680012669	0	\\
81.1070680012672	0	\\
81.1070680012672	0	\\
81.5670680012674	0	\\
81.5670680012674	0	\\
82.0370680012676	0	\\
82.0370680012676	0	\\
82.4970680012679	0	\\
82.4970680012679	0	\\
82.9570680012681	0	\\
82.9570680012681	0	\\
83.4170680012683	0	\\
83.4170680012683	0	\\
83.8770680012686	0	\\
83.8770680012686	0	\\
84.3370680012688	0	\\
84.3370680012688	0	\\
84.797068001269	0	\\
84.797068001269	0	\\
85.2570680012693	0	\\
85.2570680012693	0	\\
85.7170680012695	0	\\
85.7170680012695	0	\\
86.1770680012697	0	\\
86.1770680012697	0	\\
86.63706800127	0	\\
86.63706800127	0	\\
87.0970680012702	0	\\
87.0970680012702	0	\\
87.5670680012705	0	\\
87.5670680012705	0	\\
88.0270680012707	0	\\
88.0270680012707	0	\\
88.4870680012709	0	\\
88.4870680012709	0	\\
88.9470680012712	0	\\
88.9470680012712	0	\\
89.4070680012714	0	\\
89.4070680012714	0	\\
89.8670680012716	0	\\
89.8670680012716	0	\\
90.3270680012719	0	\\
90.3270680012719	0	\\
90.7870680012721	0	\\
90.7870680012721	0	\\
91.2470680012723	0	\\
91.2470680012723	0	\\
91.7070680012726	0	\\
91.7070680012726	0	\\
92.1670680012728	0	\\
92.1670680012728	0	\\
92.627068001273	0	\\
92.627068001273	0	\\
93.0970680012733	0	\\
93.0970680012733	0	\\
93.5570680012735	0	\\
93.5570680012735	0	\\
94.0170680012738	0	\\
94.0170680012738	0	\\
94.477068001274	0	\\
94.477068001274	0	\\
94.9370680012742	0	\\
94.9370680012742	0	\\
95.3970680012745	0	\\
95.3970680012745	0	\\
95.8570680012747	0	\\
95.8570680012747	0	\\
96.3170680012749	0	\\
96.3170680012749	0	\\
96.7770680012752	0	\\
96.7770680012752	0	\\
97.2370680012754	0	\\
97.2370680012754	0	\\
97.6970680012756	0	\\
97.6970680012756	0	\\
98.1570680012759	0	\\
98.1570680012759	0	\\
98.6270680012761	0	\\
98.6270680012761	0	\\
99.0870680012764	0	\\
99.0870680012764	0	\\
99.5470680012766	0	\\
99.5470680012766	0	\\
100.007068001277	0	\\
100.007068001277	0	\\
100.467068001277	0	\\
100.467068001277	0	\\
100.927068001277	0	\\
100.927068001277	0	\\
101.387068001278	0	\\
101.387068001278	0	\\
101.847068001278	0	\\
101.847068001278	0	\\
102.307068001278	0	\\
102.307068001278	0	\\
102.767068001278	0	\\
102.767068001278	0	\\
103.227068001278	0	\\
103.227068001278	0	\\
103.687068001279	0	\\
103.687068001279	0	\\
104.157068001279	0	\\
104.157068001279	0	\\
104.617068001279	0	\\
104.617068001279	0	\\
105.077068001279	0	\\
105.077068001279	0	\\
105.53706800128	0	\\
105.53706800128	0	\\
105.99706800128	0	\\
105.99706800128	0	\\
106.45706800128	0	\\
106.45706800128	0	\\
106.91706800128	0	\\
106.91706800128	0	\\
107.377068001281	0	\\
107.377068001281	0	\\
107.837068001281	0	\\
107.837068001281	0	\\
108.297068001281	0	\\
108.297068001281	0	\\
108.757068001281	0	\\
108.757068001281	0	\\
109.217068001282	0	\\
109.217068001282	0	\\
109.687068001282	0	\\
109.687068001282	0	\\
110.147068001282	0	\\
110.147068001282	0	\\
110.607068001282	0	\\
110.607068001282	0	\\
111.067068001282	0	\\
111.067068001282	0	\\
111.527068001283	0	\\
111.527068001283	0	\\
111.987068001283	0	\\
111.987068001283	0	\\
112.447068001283	0	\\
112.447068001283	0	\\
112.907068001283	0	\\
112.907068001283	0	\\
113.367068001284	0	\\
113.367068001284	0	\\
113.827068001284	0	\\
113.827068001284	0	\\
114.287068001284	0	\\
114.287068001284	0	\\
114.747068001284	0	\\
114.747068001284	0	\\
115.217068001285	0	\\
115.217068001285	0	\\
115.677068001285	0	\\
115.677068001285	0	\\
116.137068001285	0	\\
116.137068001285	0	\\
116.597068001285	0	\\
116.597068001285	0	\\
117.057068001286	0	\\
117.057068001286	0	\\
117.517068001286	0	\\
117.517068001286	0	\\
117.977068001286	0	\\
117.977068001286	0	\\
118.437068001286	0	\\
118.437068001286	0	\\
118.897068001286	0	\\
118.897068001286	0	\\
119.357068001287	0	\\
119.357068001287	0	\\
119.817068001287	0	\\
119.817068001287	0	\\
120.277068001287	0	\\
120.277068001287	0	\\
120.747068001287	0	\\
120.747068001287	0	\\
121.207068001288	0	\\
121.207068001288	0	\\
121.667068001288	0	\\
121.667068001288	0	\\
122.127068001288	0	\\
122.127068001288	0	\\
122.587068001288	0	\\
122.587068001288	0	\\
123.047068001289	0	\\
123.047068001289	0	\\
123.507068001289	0	\\
123.507068001289	0	\\
123.967068001289	0	\\
123.967068001289	0	\\
124.427068001289	0	\\
124.887474866462	31.805115640255	\\
124.887474866462	31.805115640255	\\
125.347474866462	59.7559593220574	\\
125.347474866462	59.7559593220574	\\
125.807474866462	59.8812498654668	\\
125.807474866462	59.8812498654668	\\
126.267474866462	59.92569903015	\\
126.267474866462	59.92569903015	\\
126.737474866463	59.972794373925	\\
126.737474866463	59.972794373925	\\
127.197474866463	60.0207769387495	\\
127.197474866463	60.0207769387495	\\
127.657474866463	60.0706291115731	\\
127.657474866463	60.0706291115731	\\
128.117474866463	60.1223508614162	\\
128.117474866463	60.1223508614162	\\
128.577474866463	60.1759421555546	\\
128.577474866463	60.1759421555546	\\
129.037474866462	60.2314029611382	\\
129.037474866462	60.2314029611382	\\
129.497474866462	60.2887332453204	\\
129.497474866462	60.2887332453204	\\
129.957474866462	60.3479329752705	\\
129.957474866462	60.3479329752705	\\
130.417474866461	60.4090021181651	\\
130.417474866461	60.4090021181651	\\
130.877474866461	60.4719406411771	\\
130.877474866461	60.4719406411771	\\
131.33747486646	60.5367485114794	\\
131.33747486646	60.5367485114794	\\
131.79747486646	60.603425696252	\\
131.79747486646	60.603425696252	\\
132.26747486646	60.6734830629776	\\
132.26747486646	60.6734830629776	\\
132.727474866459	60.7439394137046	\\
132.727474866459	60.7439394137046	\\
133.187474866459	60.8162649797155	\\
133.187474866459	60.8162649797155	\\
133.647474866458	60.8904597281974	\\
133.647474866458	60.8904597281974	\\
134.107474866458	60.9665236263357	\\
134.107474866458	60.9665236263357	\\
134.567474866457	61.0444566413051	\\
134.997474866457	61.1189972827226	\\
135.027474866457	45.0009875659809	\\
135.027474866457	45.0009875659809	\\
135.487474866457	6.35312333735527	\\
135.487474866457	6.35312333735527	\\
135.947474866456	6.23833543818897	\\
135.947474866456	6.23833543818897	\\
136.407474866456	6.23799450713491	\\
136.407474866456	6.23799450713491	\\
136.867474866455	6.23799349453691	\\
136.867474866455	6.23799349453691	\\
137.327474866455	6.23799349152776	\\
137.327474866455	6.23799349152776	\\
137.337474866455	6.23799349152598	\\
137.797474866454	6.23799349152598	\\
137.797474866454	6.23799349152598	\\
138.257474866454	6.23799349152598	\\
138.257474866454	6.23799349152598	\\
138.717474866454	6.23799349152598	\\
138.717474866454	6.23799349152598	\\
139.177474866453	6.23799349152598	\\
139.177474866453	6.23799349152598	\\
139.637474866453	6.23799349152598	\\
139.637474866453	6.23799349152598	\\
140.097474866452	6.23799349152598	\\
140.097474866452	6.23799349152598	\\
140.557474866452	6.23799349152598	\\
140.557474866452	6.23799349152598	\\
141.017474866452	6.23799349152598	\\
141.017474866452	6.23799349152598	\\
141.477474866451	6.23799349152598	\\
141.477474866451	6.23799349152598	\\
141.937474866451	6.23799349152598	\\
141.937474866451	6.23799349152598	\\
142.39747486645	6.23799349152598	\\
142.39747486645	6.23799349152598	\\
142.85747486645	6.23799349152598	\\
142.85747486645	6.23799349152598	\\
143.327474866449	6.23799349152598	\\
143.327474866449	6.23799349152598	\\
143.787474866449	6.23799349152598	\\
143.787474866449	6.23799349152598	\\
144.247474866449	6.23799349152598	\\
144.247474866449	6.23799349152598	\\
144.707474866448	6.23799349152598	\\
144.707474866448	6.23799349152598	\\
145.167474866448	6.23799349152598	\\
145.167474866448	6.23799349152598	\\
145.627474866447	6.23799349152598	\\
145.627474866447	6.23799349152598	\\
146.087474866447	6.23799349152598	\\
146.087474866447	6.23799349152598	\\
146.547474866447	6.23799349152598	\\
146.547474866447	6.23799349152598	\\
147.007474866446	6.23799349152598	\\
147.007474866446	6.23799349152598	\\
147.467474866446	6.23799349152598	\\
147.467474866446	6.23799349152598	\\
147.927474866445	6.23799349152598	\\
147.927474866445	6.23799349152598	\\
148.387474866445	6.23799349152598	\\
148.387474866445	6.23799349152598	\\
148.857474866444	6.23799349152598	\\
148.857474866444	6.23799349152598	\\
149.317474866444	6.23799349152598	\\
149.317474866444	6.23799349152598	\\
149.777474866444	6.23799349152598	\\
149.777474866444	6.23799349152598	\\
150.237474866443	6.23799349152598	\\
150.237474866443	6.23799349152598	\\
150.697474866443	6.23799349152598	\\
150.697474866443	6.23799349152598	\\
151.157474866442	6.23799349152598	\\
151.157474866442	6.23799349152598	\\
151.617474866442	6.23799349152598	\\
151.617474866442	6.23799349152598	\\
152.077474866441	6.23799349152598	\\
152.077474866441	6.23799349152598	\\
152.537474866441	6.23799349152598	\\
152.537474866441	6.23799349152598	\\
152.997474866441	6.23799349152598	\\
152.997474866441	6.23799349152598	\\
153.45747486644	6.23799349152598	\\
153.45747486644	6.23799349152598	\\
153.91747486644	6.23799349152598	\\
153.91747486644	6.23799349152598	\\
154.387474866439	6.23799349152598	\\
154.387474866439	6.23799349152598	\\
154.847474866439	6.23799349152598	\\
155.019479601956	0	\\
155.309479601952	0	\\
155.309479601952	0	\\
155.769479601952	0	\\
155.769479601952	0	\\
156.229479601951	0	\\
156.229479601951	0	\\
156.689479601951	0	\\
156.689479601951	0	\\
157.14947960195	0	\\
157.14947960195	0	\\
157.60947960195	0	\\
157.60947960195	0	\\
158.06947960195	0	\\
158.06947960195	0	\\
158.529479601949	0	\\
158.529479601949	0	\\
158.989479601949	0	\\
158.989479601949	0	\\
159.449479601948	0	\\
159.449479601948	0	\\
159.909479601948	0	\\
159.909479601948	0	\\
160.369479601947	0	\\
160.369479601947	0	\\
160.839479601947	0	\\
160.839479601947	0	\\
161.299479601947	0	\\
161.299479601947	0	\\
161.759479601946	0	\\
161.759479601946	0	\\
162.219479601946	0	\\
162.219479601946	0	\\
162.679479601945	0	\\
162.679479601945	0	\\
163.139479601945	0	\\
163.139479601945	0	\\
163.599479601944	0	\\
163.599479601944	0	\\
164.059479601944	0	\\
164.059479601944	0	\\
164.519479601944	0	\\
164.519479601944	0	\\
164.979479601943	0	\\
164.979479601943	0	\\
165.439479601943	0	\\
165.439479601943	0	\\
165.899479601942	0	\\
165.899479601942	0	\\
166.359479601942	0	\\
166.359479601942	0	\\
166.829479601942	0	\\
166.829479601942	0	\\
167.289479601941	0	\\
167.289479601941	0	\\
167.749479601941	0	\\
167.749479601941	0	\\
168.20947960194	0	\\
168.20947960194	0	\\
168.66947960194	0	\\
168.66947960194	0	\\
169.129479601939	0	\\
169.129479601939	0	\\
169.589479601939	0	\\
169.589479601939	0	\\
170.049479601939	0	\\
170.049479601939	0	\\
170.509479601938	0	\\
170.509479601938	0	\\
170.969479601938	0	\\
170.969479601938	0	\\
171.429479601937	0	\\
171.429479601937	0	\\
171.889479601937	0	\\
171.889479601937	0	\\
172.359479601937	0	\\
172.359479601937	0	\\
172.819479601936	0	\\
172.819479601936	0	\\
173.279479601936	0	\\
173.279479601936	0	\\
173.739479601935	0	\\
173.739479601935	0	\\
174.199479601935	0	\\
174.199479601935	0	\\
174.659479601934	0	\\
174.659479601934	0	\\
175.119479601934	0	\\
175.119479601934	0	\\
175.579479601934	0	\\
175.579479601934	0	\\
176.039479601933	0	\\
176.039479601933	0	\\
176.499479601933	0	\\
176.499479601933	0	\\
176.959479601932	0	\\
176.959479601932	0	\\
177.419479601932	0	\\
177.419479601932	0	\\
177.889479601931	0	\\
177.889479601931	0	\\
178.349479601931	0	\\
178.349479601931	0	\\
178.809479601931	0	\\
178.809479601931	0	\\
179.26947960193	0	\\
179.26947960193	0	\\
179.72947960193	0	\\
179.72947960193	0	\\
180.189479601929	0	\\
180.189479601929	0	\\
180.649479601929	0	\\
180.649479601929	0	\\
181.109479601929	0	\\
181.109479601929	0	\\
181.569479601928	0	\\
181.569479601928	0	\\
182.029479601928	0	\\
182.029479601928	0	\\
182.489479601927	0	\\
182.489479601927	0	\\
182.949479601927	0	\\
182.949479601927	0	\\
183.419479601926	0	\\
183.419479601926	0	\\
183.879479601926	0	\\
183.879479601926	0	\\
184.339479601926	0	\\
184.339479601926	0	\\
184.799479601925	0	\\
184.799479601925	0	\\
185.259479601925	0	\\
185.259479601925	0	\\
185.719479601924	0	\\
185.719479601924	0	\\
186.179479601924	0	\\
186.179479601924	0	\\
186.639479601924	0	\\
186.639479601924	0	\\
187.099479601923	0	\\
187.099479601923	0	\\
187.559479601923	0	\\
187.559479601923	0	\\
188.019479601922	0	\\
188.019479601922	0	\\
188.479479601922	0	\\
188.479479601922	0	\\
188.949479601921	0	\\
188.949479601921	0	\\
189.409479601921	0	\\
189.409479601921	0	\\
189.869479601921	0	\\
189.869479601921	0	\\
190.32947960192	0	\\
190.32947960192	0	\\
190.78947960192	0	\\
190.78947960192	0	\\
191.249479601919	0	\\
191.249479601919	0	\\
191.709479601919	0	\\
191.709479601919	0	\\
192.169479601918	0	\\
192.169479601918	0	\\
192.629479601918	0	\\
192.629479601918	0	\\
193.089479601918	0	\\
193.089479601918	0	\\
193.549479601917	0	\\
193.549479601917	0	\\
194.009479601917	0	\\
194.009479601917	0	\\
194.479479601916	0	\\
194.479479601916	0	\\
194.939479601916	0	\\
194.939479601916	0	\\
195.399479601916	0	\\
195.399479601916	0	\\
195.859479601915	0	\\
195.859479601915	0	\\
196.319479601915	0	\\
196.319479601915	0	\\
196.779479601914	0	\\
196.779479601914	0	\\
197.239479601914	0	\\
197.239479601914	0	\\
197.699479601913	0	\\
197.699479601913	0	\\
198.159479601913	0	\\
198.159479601913	0	\\
198.619479601913	0	\\
198.619479601913	0	\\
199.079479601912	0	\\
199.079479601912	0	\\
200	0	\\
};
\addlegendentry{Motor torque (Nm)};

\end{axis}

\begin{axis}[%
width=\figurewidth,
height=\figureheight,
scale only axis,
xmin=0,
xmax=200,
xmajorgrids,
ymin=0,
ymax=40000,
ymajorgrids,
at=(plot1.below south west),
anchor=above north west,
legend style={draw=black,fill=white,legend cell align=left}
]
\addplot [color=blue,solid]
  table[row sep=crcr]{
0	0	\\
0	0	\\
0.462229255786002	159.817978618052	\\
0.462229255786002	159.817978618052	\\
0.922229255786003	359.198691523719	\\
0.922229255786003	359.198691523719	\\
1.392229255786	562.553189296334	\\
1.392229255786	562.553189296334	\\
1.852229255786	761.601461776818	\\
1.852229255786	761.601461776818	\\
2.312229255786	960.68159793665	\\
2.312229255786	960.68159793665	\\
2.77222925578599	1159.80235392212	\\
2.77222925578599	1159.80235392212	\\
3.23222925578598	1358.97247331631	\\
3.23222925578598	1358.97247331631	\\
3.69222925578597	1558.20069949768	\\
3.69222925578597	1558.20069949768	\\
4.15222925578596	1757.49577569092	\\
4.15222925578596	1757.49577569092	\\
4.61222925578595	1956.86644496718	\\
4.61222925578595	1956.86644496718	\\
5.07222925578594	2156.32145024398	\\
5.07222925578594	2156.32145024398	\\
5.53222925578593	2355.86953428531	\\
5.53222925578593	2355.86953428531	\\
5.99222925578592	2555.51943970163	\\
5.99222925578592	2555.51943970163	\\
6.45222925578591	2755.27990894968	\\
6.45222925578591	2755.27990894968	\\
6.9222292557859	2959.50628726821	\\
6.9222292557859	2959.50628726821	\\
7.38222925578589	3159.51699168244	\\
7.38222925578589	3159.51699168244	\\
7.84222925578588	3359.6646764192	\\
7.84222925578588	3359.6646764192	\\
8.30222925578587	3559.95808331801	\\
8.30222925578587	3559.95808331801	\\
8.76222925578586	3760.40595406447	\\
8.76222925578586	3760.40595406447	\\
9.22222925578585	3961.01703019071	\\
9.22222925578585	3961.01703019071	\\
9.68222925578584	4161.80005307536	\\
9.68222925578584	4161.80005307536	\\
10.1400000000001	4361.78940223226	\\
10.1400000000001	4361.78940223226	\\
10.6000000000001	4562.94160306549	\\
10.6000000000001	4562.94160306549	\\
11.0600000000001	4764.2919315119	\\
11.0600000000001	4764.2919315119	\\
11.5300000000001	4970.23316939827	\\
11.5300000000001	4970.23316939827	\\
11.9900000000001	5172.01075943661	\\
12.0000000000001	5176.39969083224	\\
12.4500000000001	767.126168959863	\\
12.4500000000001	767.126168959863	\\
12.9100000000001	752.144178432165	\\
12.9100000000001	752.144178432165	\\
13.37	752.099633343271	\\
13.37	752.099633343271	\\
13.83	752.09950090309	\\
13.83	752.09950090309	\\
14.29	752.099500509401	\\
14.29	752.099500509401	\\
14.35	752.099500508536	\\
14.75	752.099500508536	\\
14.75	752.099500508536	\\
15.21	752.099500508536	\\
15.21	752.099500508536	\\
15.67	752.099500508536	\\
15.67	752.099500508536	\\
16.13	752.099500508536	\\
16.13	752.099500508536	\\
16.5900000000001	752.099500508536	\\
16.5900000000001	752.099500508536	\\
17.0600000000001	752.099500508536	\\
17.0600000000001	752.099500508536	\\
17.5200000000002	752.099500508536	\\
17.5200000000002	752.099500508536	\\
17.9800000000003	752.099500508536	\\
17.9800000000003	752.099500508536	\\
18.4400000000004	752.099500508536	\\
18.4400000000004	752.099500508536	\\
18.9000000000004	752.099500508536	\\
18.9000000000004	752.099500508536	\\
19.3600000000005	752.099500508536	\\
19.3600000000005	752.099500508536	\\
19.9999999999998	752.099500509833	\\
20.0057613923765	0	\\
20.2772841771288	0	\\
20.2772841771288	0	\\
20.7472841771288	0	\\
20.7472841771288	0	\\
21.2072841771289	0	\\
21.2072841771289	0	\\
21.667284177129	0	\\
21.667284177129	0	\\
22.127284177129	0	\\
22.127284177129	0	\\
22.5872841771291	0	\\
22.5872841771291	0	\\
23.0472841771292	0	\\
23.0472841771292	0	\\
23.5072841771293	0	\\
23.5072841771293	0	\\
23.9672841771293	0	\\
23.9672841771293	0	\\
24.4272841771294	0	\\
24.4272841771294	0	\\
24.8872841771295	0	\\
24.8872841771295	0	\\
25.3472841771296	0	\\
25.3472841771296	0	\\
25.8072841771296	0	\\
25.8072841771296	0	\\
26.2672841771297	0	\\
26.2672841771297	0	\\
26.7372841771298	0	\\
26.7372841771298	0	\\
27.1972841771298	0	\\
27.1972841771298	0	\\
27.6572841771299	0	\\
27.6572841771299	0	\\
28.11728417713	0	\\
28.11728417713	0	\\
28.5772841771301	0	\\
28.5772841771301	0	\\
29.0372841771301	0	\\
29.0372841771301	0	\\
29.4972841771302	0	\\
29.4972841771302	0	\\
29.9572841771303	0	\\
29.9572841771303	0	\\
30.4172841771303	0	\\
30.4172841771303	0	\\
30.8772841771304	0	\\
30.8772841771304	0	\\
31.3372841771305	0	\\
31.3372841771305	0	\\
31.7972841771306	0	\\
31.7972841771306	0	\\
32.2672841771305	0	\\
32.2672841771305	0	\\
32.7272841771305	0	\\
32.7272841771305	0	\\
33.1872841771304	0	\\
33.1872841771304	0	\\
33.6472841771303	0	\\
33.6472841771303	0	\\
34.1072841771302	0	\\
34.1072841771302	0	\\
34.5672841771301	0	\\
34.5672841771301	0	\\
35.02728417713	0	\\
35.02728417713	0	\\
35.4872841771299	0	\\
35.4872841771299	0	\\
35.9472841771298	0	\\
35.9472841771298	0	\\
36.4072841771297	0	\\
36.4072841771297	0	\\
36.8672841771296	0	\\
36.8672841771296	0	\\
37.3272841771295	0	\\
37.3272841771295	0	\\
37.7972841771294	0	\\
37.7972841771294	0	\\
38.2572841771294	0	\\
38.2572841771294	0	\\
38.7172841771293	0	\\
38.7172841771293	0	\\
39.1772841771292	0	\\
39.1772841771292	0	\\
39.6372841771291	0	\\
39.6372841771291	0	\\
40.1000000000004	0	\\
40.1000000000004	0	\\
40.5600000000003	0	\\
40.5600000000003	0	\\
41.0200000000003	0	\\
41.0200000000003	0	\\
41.4800000000002	0	\\
41.4800000000002	0	\\
41.9400000000001	0	\\
41.9400000000001	0	\\
42.4	0	\\
42.4	0	\\
42.8599999999999	0	\\
42.8599999999999	0	\\
43.3199999999998	0	\\
43.3199999999998	0	\\
43.7799999999997	0	\\
43.7799999999997	0	\\
44.2399999999996	0	\\
44.2399999999996	0	\\
44.7099999999995	0	\\
44.7099999999995	0	\\
45.1699999999994	0	\\
45.1699999999994	0	\\
45.6299999999993	0	\\
45.6299999999993	0	\\
46.0899999999992	0	\\
46.0899999999992	0	\\
46.5499999999992	0	\\
46.5499999999992	0	\\
47.0099999999991	0	\\
47.0099999999991	0	\\
47.469999999999	0	\\
47.469999999999	0	\\
47.9299999999989	0	\\
47.9299999999989	0	\\
48.3899999999988	0	\\
48.8564934895424	644.67244191589	\\
48.8564934895424	644.67244191589	\\
49.3164934895418	5954.17786536318	\\
49.3164934895418	5954.17786536318	\\
49.7764934895417	6620.35617034632	\\
49.7764934895417	6620.35617034632	\\
50.2364934895416	7271.88426787498	\\
50.2364934895416	7271.88426787498	\\
50.6964934895415	7923.98873559383	\\
50.6964934895415	7923.98873559383	\\
51.1564934895414	8576.77654460539	\\
51.1564934895414	8576.77654460539	\\
51.6164934895413	9230.30471120016	\\
51.6164934895413	9230.30471120016	\\
52.0764934895412	9884.6300864149	\\
52.0764934895412	9884.6300864149	\\
52.5364934895411	10539.809518892	\\
52.5364934895411	10539.809518892	\\
52.996493489541	11195.8998554091	\\
52.996493489541	11195.8998554091	\\
53.4564934895409	11852.9579408787	\\
53.4564934895409	11852.9579408787	\\
53.9264934895408	12525.3585679143	\\
53.9264934895408	12525.3585679143	\\
54.3864934895408	13184.5468192498	\\
54.3864934895408	13184.5468192498	\\
54.8464934895407	13844.8745787134	\\
54.8464934895407	13844.8745787134	\\
55.3064934895406	14506.3986837248	\\
55.3064934895406	14506.3986837248	\\
55.7664934895405	15169.1759698384	\\
55.7664934895405	15169.1759698384	\\
56.2264934895404	15833.2632707467	\\
56.2264934895404	15833.2632707467	\\
56.6864934895403	16498.7174182789	\\
56.6864934895403	16498.7174182789	\\
57.1464934895402	17165.5952423999	\\
57.1464934895402	17165.5952423999	\\
57.6064934895401	17833.9535712132	\\
57.6064934895401	17833.9535712132	\\
58.06649348954	18503.8492309571	\\
58.06649348954	18503.8492309571	\\
58.5264934895399	19175.3390460111	\\
58.5264934895399	19175.3390460111	\\
58.9864934895398	19848.4798388868	\\
58.9864934895398	19848.4798388868	\\
59.4564934895397	20538.0184436151	\\
59.4564934895397	20538.0184436151	\\
59.9164934895397	21214.6706444803	\\
60	21335.4141480229	\\
60.3700000000004	2203.90923609785	\\
60.3700000000004	2203.90923609785	\\
60.8300000000003	2024.35944644016	\\
60.8300000000003	2024.35944644016	\\
61.3000000000002	2023.82571437361	\\
61.3000000000002	2023.82571437361	\\
61.7600000000001	2023.82431737053	\\
61.7600000000001	2023.82431737053	\\
62.22	2023.82431321868	\\
62.22	2023.82431321868	\\
62.33	2023.82431320925	\\
62.6799999999999	2023.82431320925	\\
62.6799999999999	2023.82431320925	\\
63.1399999999998	2023.82431320925	\\
63.1399999999998	2023.82431320925	\\
63.5999999999997	2023.82431320925	\\
63.5999999999997	2023.82431320925	\\
64.0599999999997	2023.82431320925	\\
64.0599999999997	2023.82431320925	\\
64.5199999999999	2023.82431320925	\\
64.5199999999999	2023.82431320925	\\
64.9800000000002	2023.82431320925	\\
64.9800000000002	2023.82431320925	\\
65.4400000000004	2023.82431320925	\\
65.4400000000004	2023.82431320925	\\
65.9000000000006	2023.82431320925	\\
65.9000000000006	2023.82431320925	\\
66.3600000000009	2023.82431320925	\\
66.3600000000009	2023.82431320925	\\
66.8300000000011	2023.82431320925	\\
66.8300000000011	2023.82431320925	\\
67.2900000000013	2023.82431320925	\\
67.2900000000013	2023.82431320925	\\
67.7500000000016	2023.82431320925	\\
67.7500000000016	2023.82431320925	\\
68.2100000000018	2023.82431320925	\\
68.2100000000018	2023.82431320925	\\
68.670000000002	2023.82431320925	\\
68.670000000002	2023.82431320925	\\
69.1300000000023	2023.82431320925	\\
69.1300000000023	2023.82431320925	\\
69.5900000000025	2023.82431320925	\\
69.5900000000025	2023.82431320925	\\
70.0500000000028	2023.82431320925	\\
70.0500000000028	2023.82431320925	\\
70.510000000003	2023.82431320925	\\
70.510000000003	2023.82431320925	\\
70.9700000000032	2023.82431320925	\\
70.9700000000032	2023.82431320925	\\
71.4300000000035	2023.82431320925	\\
71.4300000000035	2023.82431320925	\\
71.8900000000037	2023.82431320925	\\
71.8900000000037	2023.82431320925	\\
72.3600000000039	2023.82431320925	\\
72.3600000000039	2023.82431320925	\\
72.8200000000042	2023.82431320925	\\
72.8200000000042	2023.82431320925	\\
73.2800000000044	2023.82431320925	\\
73.2800000000044	2023.82431320925	\\
73.7400000000046	2023.82431320925	\\
73.7400000000046	2023.82431320925	\\
74.2000000000049	2023.82431320925	\\
74.2000000000049	2023.82431320925	\\
74.6600000000051	2023.82431320925	\\
74.6600000000051	2023.82431320925	\\
75.1200000000053	2023.82431320925	\\
75.1200000000053	2023.82431320925	\\
75.5800000000056	2023.82431320925	\\
75.5800000000056	2023.82431320925	\\
76.0400000000058	2023.82431320925	\\
76.0400000000058	2023.82431320925	\\
76.5000000000061	2023.82431320925	\\
76.5000000000061	2023.82431320925	\\
76.9600000000063	2023.82431320925	\\
76.9600000000063	2023.82431320925	\\
77.4200000000065	2023.82431320925	\\
77.4200000000065	2023.82431320925	\\
77.8900000000068	2023.82431320925	\\
77.8900000000068	2023.82431320925	\\
78.350000000007	2023.82431320925	\\
78.350000000007	2023.82431320925	\\
78.8100000000072	2023.82431320925	\\
78.8100000000072	2023.82431320925	\\
79.2700000000075	2023.82431320925	\\
79.2700000000075	2023.82431320925	\\
79.9999999999991	2023.82431325705	\\
80.0056893337565	0	\\
80.1870680012667	0	\\
80.1870680012667	0	\\
80.6470680012669	0	\\
80.6470680012669	0	\\
81.1070680012672	0	\\
81.1070680012672	0	\\
81.5670680012674	0	\\
81.5670680012674	0	\\
82.0370680012676	0	\\
82.0370680012676	0	\\
82.4970680012679	0	\\
82.4970680012679	0	\\
82.9570680012681	0	\\
82.9570680012681	0	\\
83.4170680012683	0	\\
83.4170680012683	0	\\
83.8770680012686	0	\\
83.8770680012686	0	\\
84.3370680012688	0	\\
84.3370680012688	0	\\
84.797068001269	0	\\
84.797068001269	0	\\
85.2570680012693	0	\\
85.2570680012693	0	\\
85.7170680012695	0	\\
85.7170680012695	0	\\
86.1770680012697	0	\\
86.1770680012697	0	\\
86.63706800127	0	\\
86.63706800127	0	\\
87.0970680012702	0	\\
87.0970680012702	0	\\
87.5670680012705	0	\\
87.5670680012705	0	\\
88.0270680012707	0	\\
88.0270680012707	0	\\
88.4870680012709	0	\\
88.4870680012709	0	\\
88.9470680012712	0	\\
88.9470680012712	0	\\
89.4070680012714	0	\\
89.4070680012714	0	\\
89.8670680012716	0	\\
89.8670680012716	0	\\
90.3270680012719	0	\\
90.3270680012719	0	\\
90.7870680012721	0	\\
90.7870680012721	0	\\
91.2470680012723	0	\\
91.2470680012723	0	\\
91.7070680012726	0	\\
91.7070680012726	0	\\
92.1670680012728	0	\\
92.1670680012728	0	\\
92.627068001273	0	\\
92.627068001273	0	\\
93.0970680012733	0	\\
93.0970680012733	0	\\
93.5570680012735	0	\\
93.5570680012735	0	\\
94.0170680012738	0	\\
94.0170680012738	0	\\
94.477068001274	0	\\
94.477068001274	0	\\
94.9370680012742	0	\\
94.9370680012742	0	\\
95.3970680012745	0	\\
95.3970680012745	0	\\
95.8570680012747	0	\\
95.8570680012747	0	\\
96.3170680012749	0	\\
96.3170680012749	0	\\
96.7770680012752	0	\\
96.7770680012752	0	\\
97.2370680012754	0	\\
97.2370680012754	0	\\
97.6970680012756	0	\\
97.6970680012756	0	\\
98.1570680012759	0	\\
98.1570680012759	0	\\
98.6270680012761	0	\\
98.6270680012761	0	\\
99.0870680012764	0	\\
99.0870680012764	0	\\
99.5470680012766	0	\\
99.5470680012766	0	\\
100.007068001277	0	\\
100.007068001277	0	\\
100.467068001277	0	\\
100.467068001277	0	\\
100.927068001277	0	\\
100.927068001277	0	\\
101.387068001278	0	\\
101.387068001278	0	\\
101.847068001278	0	\\
101.847068001278	0	\\
102.307068001278	0	\\
102.307068001278	0	\\
102.767068001278	0	\\
102.767068001278	0	\\
103.227068001278	0	\\
103.227068001278	0	\\
103.687068001279	0	\\
103.687068001279	0	\\
104.157068001279	0	\\
104.157068001279	0	\\
104.617068001279	0	\\
104.617068001279	0	\\
105.077068001279	0	\\
105.077068001279	0	\\
105.53706800128	0	\\
105.53706800128	0	\\
105.99706800128	0	\\
105.99706800128	0	\\
106.45706800128	0	\\
106.45706800128	0	\\
106.91706800128	0	\\
106.91706800128	0	\\
107.377068001281	0	\\
107.377068001281	0	\\
107.837068001281	0	\\
107.837068001281	0	\\
108.297068001281	0	\\
108.297068001281	0	\\
108.757068001281	0	\\
108.757068001281	0	\\
109.217068001282	0	\\
109.217068001282	0	\\
109.687068001282	0	\\
109.687068001282	0	\\
110.147068001282	0	\\
110.147068001282	0	\\
110.607068001282	0	\\
110.607068001282	0	\\
111.067068001282	0	\\
111.067068001282	0	\\
111.527068001283	0	\\
111.527068001283	0	\\
111.987068001283	0	\\
111.987068001283	0	\\
112.447068001283	0	\\
112.447068001283	0	\\
112.907068001283	0	\\
112.907068001283	0	\\
113.367068001284	0	\\
113.367068001284	0	\\
113.827068001284	0	\\
113.827068001284	0	\\
114.287068001284	0	\\
114.287068001284	0	\\
114.747068001284	0	\\
114.747068001284	0	\\
115.217068001285	0	\\
115.217068001285	0	\\
115.677068001285	0	\\
115.677068001285	0	\\
116.137068001285	0	\\
116.137068001285	0	\\
116.597068001285	0	\\
116.597068001285	0	\\
117.057068001286	0	\\
117.057068001286	0	\\
117.517068001286	0	\\
117.517068001286	0	\\
117.977068001286	0	\\
117.977068001286	0	\\
118.437068001286	0	\\
118.437068001286	0	\\
118.897068001286	0	\\
118.897068001286	0	\\
119.357068001287	0	\\
119.357068001287	0	\\
119.817068001287	0	\\
119.817068001287	0	\\
120.277068001287	0	\\
120.277068001287	0	\\
120.747068001287	0	\\
120.747068001287	0	\\
121.207068001288	0	\\
121.207068001288	0	\\
121.667068001288	0	\\
121.667068001288	0	\\
122.127068001288	0	\\
122.127068001288	0	\\
122.587068001288	0	\\
122.587068001288	0	\\
123.047068001289	0	\\
123.047068001289	0	\\
123.507068001289	0	\\
123.507068001289	0	\\
123.967068001289	0	\\
123.967068001289	0	\\
124.427068001289	0	\\
124.887474866462	8483.06924294532	\\
124.887474866462	8483.06924294532	\\
125.347474866462	16617.8924615289	\\
125.347474866462	16617.8924615289	\\
125.807474866462	17398.952553245	\\
125.807474866462	17398.952553245	\\
126.267474866462	18158.8278576007	\\
126.267474866462	18158.8278576007	\\
126.737474866463	18936.8931913882	\\
126.737474866463	18936.8931913882	\\
127.197474866463	19700.181182619	\\
127.197474866463	19700.181182619	\\
127.657474866463	20465.2979130681	\\
127.657474866463	20465.2979130681	\\
128.117474866463	21232.3132724933	\\
128.117474866463	21232.3132724933	\\
128.577474866463	22001.2971476553	\\
128.577474866463	22001.2971476553	\\
129.037474866462	22772.3194228194	\\
129.037474866462	22772.3194228194	\\
129.497474866462	23545.4499797953	\\
129.497474866462	23545.4499797953	\\
129.957474866462	24320.7586979434	\\
129.957474866462	24320.7586979434	\\
130.417474866461	25098.3154541723	\\
130.417474866461	25098.3154541723	\\
130.877474866461	25878.1901229346	\\
130.877474866461	25878.1901229346	\\
131.33747486646	26660.4525762282	\\
131.33747486646	26660.4525762282	\\
131.79747486646	27445.1726835999	\\
131.79747486646	27445.1726835999	\\
132.26747486646	28249.5629842151	\\
132.26747486646	28249.5629842151	\\
132.727474866459	29039.4652393607	\\
132.727474866459	29039.4652393607	\\
133.187474866459	29832.0362612088	\\
133.187474866459	29832.0362612088	\\
133.647474866458	30627.3459094398	\\
133.647474866458	30627.3459094398	\\
134.107474866458	31425.464041279	\\
134.107474866458	31425.464041279	\\
134.567474866457	32226.460511493	\\
134.997474866457	32977.8827588198	\\
135.027474866457	24312.4990875309	\\
135.027474866457	24312.4990875309	\\
135.487474866457	3441.95803786216	\\
135.487474866457	3441.95803786216	\\
135.947474866456	3379.79686625999	\\
135.947474866456	3379.79686625999	\\
136.407474866456	3379.61224009521	\\
136.407474866456	3379.61224009521	\\
136.867474866455	3379.61169173774	\\
136.867474866455	3379.61169173774	\\
137.327474866455	3379.61169010818	\\
137.327474866455	3379.61169010818	\\
137.337474866455	3379.61169010721	\\
137.797474866454	3379.61169010721	\\
137.797474866454	3379.61169010721	\\
138.257474866454	3379.61169010721	\\
138.257474866454	3379.61169010721	\\
138.717474866454	3379.61169010721	\\
138.717474866454	3379.61169010721	\\
139.177474866453	3379.61169010721	\\
139.177474866453	3379.61169010721	\\
139.637474866453	3379.61169010721	\\
139.637474866453	3379.61169010721	\\
140.097474866452	3379.61169010721	\\
140.097474866452	3379.61169010721	\\
140.557474866452	3379.61169010721	\\
140.557474866452	3379.61169010721	\\
141.017474866452	3379.61169010721	\\
141.017474866452	3379.61169010721	\\
141.477474866451	3379.61169010721	\\
141.477474866451	3379.61169010721	\\
141.937474866451	3379.61169010721	\\
141.937474866451	3379.61169010721	\\
142.39747486645	3379.61169010721	\\
142.39747486645	3379.61169010721	\\
142.85747486645	3379.61169010721	\\
142.85747486645	3379.61169010721	\\
143.327474866449	3379.61169010721	\\
143.327474866449	3379.61169010721	\\
143.787474866449	3379.61169010721	\\
143.787474866449	3379.61169010721	\\
144.247474866449	3379.61169010721	\\
144.247474866449	3379.61169010721	\\
144.707474866448	3379.61169010721	\\
144.707474866448	3379.61169010721	\\
145.167474866448	3379.61169010721	\\
145.167474866448	3379.61169010721	\\
145.627474866447	3379.61169010721	\\
145.627474866447	3379.61169010721	\\
146.087474866447	3379.61169010721	\\
146.087474866447	3379.61169010721	\\
146.547474866447	3379.61169010721	\\
146.547474866447	3379.61169010721	\\
147.007474866446	3379.61169010721	\\
147.007474866446	3379.61169010721	\\
147.467474866446	3379.61169010721	\\
147.467474866446	3379.61169010721	\\
147.927474866445	3379.61169010721	\\
147.927474866445	3379.61169010721	\\
148.387474866445	3379.61169010721	\\
148.387474866445	3379.61169010721	\\
148.857474866444	3379.61169010721	\\
148.857474866444	3379.61169010721	\\
149.317474866444	3379.61169010721	\\
149.317474866444	3379.61169010721	\\
149.777474866444	3379.61169010721	\\
149.777474866444	3379.61169010721	\\
150.237474866443	3379.61169010721	\\
150.237474866443	3379.61169010721	\\
150.697474866443	3379.61169010721	\\
150.697474866443	3379.61169010721	\\
151.157474866442	3379.61169010721	\\
151.157474866442	3379.61169010721	\\
151.617474866442	3379.61169010721	\\
151.617474866442	3379.61169010721	\\
152.077474866441	3379.61169010721	\\
152.077474866441	3379.61169010721	\\
152.537474866441	3379.61169010721	\\
152.537474866441	3379.61169010721	\\
152.997474866441	3379.61169010721	\\
152.997474866441	3379.61169010721	\\
153.45747486644	3379.61169010721	\\
153.45747486644	3379.61169010721	\\
153.91747486644	3379.61169010721	\\
153.91747486644	3379.61169010721	\\
154.387474866439	3379.61169010721	\\
154.387474866439	3379.61169010721	\\
154.847474866439	3379.61169010721	\\
155.019479601956	0	\\
155.309479601952	0	\\
155.309479601952	0	\\
155.769479601952	0	\\
155.769479601952	0	\\
156.229479601951	0	\\
156.229479601951	0	\\
156.689479601951	0	\\
156.689479601951	0	\\
157.14947960195	0	\\
157.14947960195	0	\\
157.60947960195	0	\\
157.60947960195	0	\\
158.06947960195	0	\\
158.06947960195	0	\\
158.529479601949	0	\\
158.529479601949	0	\\
158.989479601949	0	\\
158.989479601949	0	\\
159.449479601948	0	\\
159.449479601948	0	\\
159.909479601948	0	\\
159.909479601948	0	\\
160.369479601947	0	\\
160.369479601947	0	\\
160.839479601947	0	\\
160.839479601947	0	\\
161.299479601947	0	\\
161.299479601947	0	\\
161.759479601946	0	\\
161.759479601946	0	\\
162.219479601946	0	\\
162.219479601946	0	\\
162.679479601945	0	\\
162.679479601945	0	\\
163.139479601945	0	\\
163.139479601945	0	\\
163.599479601944	0	\\
163.599479601944	0	\\
164.059479601944	0	\\
164.059479601944	0	\\
164.519479601944	0	\\
164.519479601944	0	\\
164.979479601943	0	\\
164.979479601943	0	\\
165.439479601943	0	\\
165.439479601943	0	\\
165.899479601942	0	\\
165.899479601942	0	\\
166.359479601942	0	\\
166.359479601942	0	\\
166.829479601942	0	\\
166.829479601942	0	\\
167.289479601941	0	\\
167.289479601941	0	\\
167.749479601941	0	\\
167.749479601941	0	\\
168.20947960194	0	\\
168.20947960194	0	\\
168.66947960194	0	\\
168.66947960194	0	\\
169.129479601939	0	\\
169.129479601939	0	\\
169.589479601939	0	\\
169.589479601939	0	\\
170.049479601939	0	\\
170.049479601939	0	\\
170.509479601938	0	\\
170.509479601938	0	\\
170.969479601938	0	\\
170.969479601938	0	\\
171.429479601937	0	\\
171.429479601937	0	\\
171.889479601937	0	\\
171.889479601937	0	\\
172.359479601937	0	\\
172.359479601937	0	\\
172.819479601936	0	\\
172.819479601936	0	\\
173.279479601936	0	\\
173.279479601936	0	\\
173.739479601935	0	\\
173.739479601935	0	\\
174.199479601935	0	\\
174.199479601935	0	\\
174.659479601934	0	\\
174.659479601934	0	\\
175.119479601934	0	\\
175.119479601934	0	\\
175.579479601934	0	\\
175.579479601934	0	\\
176.039479601933	0	\\
176.039479601933	0	\\
176.499479601933	0	\\
176.499479601933	0	\\
176.959479601932	0	\\
176.959479601932	0	\\
177.419479601932	0	\\
177.419479601932	0	\\
177.889479601931	0	\\
177.889479601931	0	\\
178.349479601931	0	\\
178.349479601931	0	\\
178.809479601931	0	\\
178.809479601931	0	\\
179.26947960193	0	\\
179.26947960193	0	\\
179.72947960193	0	\\
179.72947960193	0	\\
180.189479601929	0	\\
180.189479601929	0	\\
180.649479601929	0	\\
180.649479601929	0	\\
181.109479601929	0	\\
181.109479601929	0	\\
181.569479601928	0	\\
181.569479601928	0	\\
182.029479601928	0	\\
182.029479601928	0	\\
182.489479601927	0	\\
182.489479601927	0	\\
182.949479601927	0	\\
182.949479601927	0	\\
183.419479601926	0	\\
183.419479601926	0	\\
183.879479601926	0	\\
183.879479601926	0	\\
184.339479601926	0	\\
184.339479601926	0	\\
184.799479601925	0	\\
184.799479601925	0	\\
185.259479601925	0	\\
185.259479601925	0	\\
185.719479601924	0	\\
185.719479601924	0	\\
186.179479601924	0	\\
186.179479601924	0	\\
186.639479601924	0	\\
186.639479601924	0	\\
187.099479601923	0	\\
187.099479601923	0	\\
187.559479601923	0	\\
187.559479601923	0	\\
188.019479601922	0	\\
188.019479601922	0	\\
188.479479601922	0	\\
188.479479601922	0	\\
188.949479601921	0	\\
188.949479601921	0	\\
189.409479601921	0	\\
189.409479601921	0	\\
189.869479601921	0	\\
189.869479601921	0	\\
190.32947960192	0	\\
190.32947960192	0	\\
190.78947960192	0	\\
190.78947960192	0	\\
191.249479601919	0	\\
191.249479601919	0	\\
191.709479601919	0	\\
191.709479601919	0	\\
192.169479601918	0	\\
192.169479601918	0	\\
192.629479601918	0	\\
192.629479601918	0	\\
193.089479601918	0	\\
193.089479601918	0	\\
193.549479601917	0	\\
193.549479601917	0	\\
194.009479601917	0	\\
194.009479601917	0	\\
194.479479601916	0	\\
194.479479601916	0	\\
194.939479601916	0	\\
194.939479601916	0	\\
195.399479601916	0	\\
195.399479601916	0	\\
195.859479601915	0	\\
195.859479601915	0	\\
196.319479601915	0	\\
196.319479601915	0	\\
196.779479601914	0	\\
196.779479601914	0	\\
197.239479601914	0	\\
197.239479601914	0	\\
197.699479601913	0	\\
197.699479601913	0	\\
198.159479601913	0	\\
198.159479601913	0	\\
198.619479601913	0	\\
198.619479601913	0	\\
199.079479601912	0	\\
199.079479601912	0	\\
200	0	\\
};
\addlegendentry{Motor power (W)};

\end{axis}
\end{tikzpicture}%
    	\label{fig:motor-torque-power}
    	\caption{The torque and power output of the motor.}
\end{figure}
\begin{figure}[H]
	\centering
    	\setlength\figureheight{4cm}
    	\setlength\figurewidth{0.8\linewidth}
    	% This file was created by matlab2tikz v0.4.6 running on MATLAB 8.2.
% Copyright (c) 2008--2014, Nico Schlömer <nico.schloemer@gmail.com>
% All rights reserved.
% Minimal pgfplots version: 1.3
% 
% The latest updates can be retrieved from
%   http://www.mathworks.com/matlabcentral/fileexchange/22022-matlab2tikz
% where you can also make suggestions and rate matlab2tikz.
% 
\begin{tikzpicture}

\begin{axis}[%
width=\figurewidth,
height=\figureheight,
scale only axis,
xmin=0,
xmax=200,
xmajorgrids,
ymin=0,
ymax=450,
ymajorgrids,
legend style={draw=black,fill=white,legend cell align=left}
]
\addplot [color=blue,solid]
  table[row sep=crcr]{
0	0	\\
0	0	\\
0.462229255786002	227.153198036366	\\
0.462229255786002	227.153198036366	\\
0.922229255786003	227.860240609044	\\
0.922229255786003	227.860240609044	\\
1.392229255786	227.869643770953	\\
1.392229255786	227.869643770953	\\
1.852229255786	227.880177906622	\\
1.852229255786	227.880177906622	\\
2.312229255786	227.894043849905	\\
2.312229255786	227.894043849905	\\
2.77222925578599	227.911247021586	\\
2.77222925578599	227.911247021586	\\
3.23222925578598	227.931787408561	\\
3.23222925578598	227.931787408561	\\
3.69222925578597	227.955664981566	\\
3.69222925578597	227.955664981566	\\
4.15222925578596	227.982879711292	\\
4.15222925578596	227.982879711292	\\
4.61222925578595	228.013431568432	\\
4.61222925578595	228.013431568432	\\
5.07222925578594	228.047320523674	\\
5.07222925578594	228.047320523674	\\
5.53222925578593	228.084546547714	\\
5.53222925578593	228.084546547714	\\
5.99222925578592	228.125109611246	\\
5.99222925578592	228.125109611246	\\
6.45222925578591	228.169009684956	\\
6.45222925578591	228.169009684956	\\
6.9222292557859	228.217310691852	\\
6.9222292557859	228.217310691852	\\
7.38222925578589	228.267957240107	\\
7.38222925578589	228.267957240107	\\
7.84222925578588	228.321940709979	\\
7.84222925578588	228.321940709979	\\
8.30222925578587	228.37926107218	\\
8.30222925578587	228.37926107218	\\
8.76222925578586	228.439918297396	\\
8.76222925578586	228.439918297396	\\
9.22222925578585	228.503912356317	\\
9.22222925578585	228.503912356317	\\
9.68222925578584	228.57124321964	\\
9.68222925578584	228.57124321964	\\
10.1400000000001	228.641560341857	\\
10.1400000000001	228.641560341857	\\
10.6000000000001	228.715548555592	\\
10.6000000000001	228.715548555592	\\
11.0600000000001	228.792873485959	\\
11.0600000000001	228.792873485959	\\
11.5300000000001	228.875325673651	\\
11.5300000000001	228.875325673651	\\
11.9900000000001	228.95939648512	\\
12.0000000000001	228.961261167521	\\
12.4500000000001	33.7082974156568	\\
12.4500000000001	33.7082974156568	\\
12.9100000000001	33.0492403597227	\\
12.9100000000001	33.0492403597227	\\
13.37	33.0472808670198	\\
13.37	33.0472808670198	\\
13.83	33.0472750411127	\\
13.83	33.0472750411127	\\
14.29	33.0472750237947	\\
14.29	33.0472750237947	\\
14.35	33.0472750237567	\\
14.75	33.0472750237567	\\
14.75	33.0472750237567	\\
15.21	33.0472750237567	\\
15.21	33.0472750237567	\\
15.67	33.0472750237567	\\
15.67	33.0472750237567	\\
16.13	33.0472750237567	\\
16.13	33.0472750237567	\\
16.5900000000001	33.0472750237567	\\
16.5900000000001	33.0472750237567	\\
17.0600000000001	33.0472750237567	\\
17.0600000000001	33.0472750237567	\\
17.5200000000002	33.0472750237567	\\
17.5200000000002	33.0472750237567	\\
17.9800000000003	33.0472750237567	\\
17.9800000000003	33.0472750237567	\\
18.4400000000004	33.0472750237567	\\
18.4400000000004	33.0472750237567	\\
18.9000000000004	33.0472750237567	\\
18.9000000000004	33.0472750237567	\\
19.3600000000005	33.0472750237567	\\
19.3600000000005	33.0472750237567	\\
19.9999999999998	33.0472750238138	\\
20.0057613923765	0	\\
20.2772841771288	0	\\
20.2772841771288	0	\\
20.7472841771288	0	\\
20.7472841771288	0	\\
21.2072841771289	0	\\
21.2072841771289	0	\\
21.667284177129	0	\\
21.667284177129	0	\\
22.127284177129	0	\\
22.127284177129	0	\\
22.5872841771291	0	\\
22.5872841771291	0	\\
23.0472841771292	0	\\
23.0472841771292	0	\\
23.5072841771293	0	\\
23.5072841771293	0	\\
23.9672841771293	0	\\
23.9672841771293	0	\\
24.4272841771294	0	\\
24.4272841771294	0	\\
24.8872841771295	0	\\
24.8872841771295	0	\\
25.3472841771296	0	\\
25.3472841771296	0	\\
25.8072841771296	0	\\
25.8072841771296	0	\\
26.2672841771297	0	\\
26.2672841771297	0	\\
26.7372841771298	0	\\
26.7372841771298	0	\\
27.1972841771298	0	\\
27.1972841771298	0	\\
27.6572841771299	0	\\
27.6572841771299	0	\\
28.11728417713	0	\\
28.11728417713	0	\\
28.5772841771301	0	\\
28.5772841771301	0	\\
29.0372841771301	0	\\
29.0372841771301	0	\\
29.4972841771302	0	\\
29.4972841771302	0	\\
29.9572841771303	0	\\
29.9572841771303	0	\\
30.4172841771303	0	\\
30.4172841771303	0	\\
30.8772841771304	0	\\
30.8772841771304	0	\\
31.3372841771305	0	\\
31.3372841771305	0	\\
31.7972841771306	0	\\
31.7972841771306	0	\\
32.2672841771305	0	\\
32.2672841771305	0	\\
32.7272841771305	0	\\
32.7272841771305	0	\\
33.1872841771304	0	\\
33.1872841771304	0	\\
33.6472841771303	0	\\
33.6472841771303	0	\\
34.1072841771302	0	\\
34.1072841771302	0	\\
34.5672841771301	0	\\
34.5672841771301	0	\\
35.02728417713	0	\\
35.02728417713	0	\\
35.4872841771299	0	\\
35.4872841771299	0	\\
35.9472841771298	0	\\
35.9472841771298	0	\\
36.4072841771297	0	\\
36.4072841771297	0	\\
36.8672841771296	0	\\
36.8672841771296	0	\\
37.3272841771295	0	\\
37.3272841771295	0	\\
37.7972841771294	0	\\
37.7972841771294	0	\\
38.2572841771294	0	\\
38.2572841771294	0	\\
38.7172841771293	0	\\
38.7172841771293	0	\\
39.1772841771292	0	\\
39.1772841771292	0	\\
39.6372841771291	0	\\
39.6372841771291	0	\\
40.1000000000004	0	\\
40.1000000000004	0	\\
40.5600000000003	0	\\
40.5600000000003	0	\\
41.0200000000003	0	\\
41.0200000000003	0	\\
41.4800000000002	0	\\
41.4800000000002	0	\\
41.9400000000001	0	\\
41.9400000000001	0	\\
42.4	0	\\
42.4	0	\\
42.8599999999999	0	\\
42.8599999999999	0	\\
43.3199999999998	0	\\
43.3199999999998	0	\\
43.7799999999997	0	\\
43.7799999999997	0	\\
44.2399999999996	0	\\
44.2399999999996	0	\\
44.7099999999995	0	\\
44.7099999999995	0	\\
45.1699999999994	0	\\
45.1699999999994	0	\\
45.6299999999993	0	\\
45.6299999999993	0	\\
46.0899999999992	0	\\
46.0899999999992	0	\\
46.5499999999992	0	\\
46.5499999999992	0	\\
47.0099999999991	0	\\
47.0099999999991	0	\\
47.469999999999	0	\\
47.469999999999	0	\\
47.9299999999989	0	\\
47.9299999999989	0	\\
48.3899999999988	0	\\
48.8564934895424	47.2923910827024	\\
48.8564934895424	47.2923910827024	\\
49.3164934895418	397.106345654939	\\
49.3164934895418	397.106345654939	\\
49.7764934895417	398.257038823093	\\
49.7764934895417	398.257038823093	\\
50.2364934895416	398.382513948424	\\
50.2364934895416	398.382513948424	\\
50.6964934895415	398.516532759836	\\
50.6964934895415	398.516532759836	\\
51.1564934895414	398.662169040451	\\
51.1564934895414	398.662169040451	\\
51.6164934895413	398.819431740098	\\
51.6164934895413	398.819431740098	\\
52.0764934895412	398.988320695375	\\
52.0764934895412	398.988320695375	\\
52.5364934895411	399.168835715768	\\
52.5364934895411	399.168835715768	\\
52.996493489541	399.360976610716	\\
52.996493489541	399.360976610716	\\
53.4564934895409	399.564743189621	\\
53.4564934895409	399.564743189621	\\
53.9264934895408	399.78494680798	\\
53.9264934895408	399.78494680798	\\
54.3864934895408	400.012216905015	\\
54.3864934895408	400.012216905015	\\
54.8464934895407	400.251112110185	\\
54.8464934895407	400.251112110185	\\
55.3064934895406	400.501632232973	\\
55.3064934895406	400.501632232973	\\
55.7664934895405	400.763777082805	\\
55.7664934895405	400.763777082805	\\
56.2264934895404	401.037546469152	\\
56.2264934895404	401.037546469152	\\
56.6864934895403	401.322940201484	\\
56.6864934895403	401.322940201484	\\
57.1464934895402	401.619958089255	\\
57.1464934895402	401.619958089255	\\
57.6064934895401	401.928599941959	\\
57.6064934895401	401.928599941959	\\
58.06649348954	402.248865569043	\\
58.06649348954	402.248865569043	\\
58.5264934895399	402.580754780053	\\
58.5264934895399	402.580754780053	\\
58.9864934895398	402.924267384428	\\
58.9864934895398	402.924267384428	\\
59.4564934895397	403.287252619552	\\
59.4564934895397	403.287252619552	\\
59.9164934895397	403.654264111273	\\
59.9964934895396	403.719278652357	\\
60.3700000000004	41.4814647565013	\\
60.3700000000004	41.4814647565013	\\
60.8300000000003	38.1001628041497	\\
60.8300000000003	38.1001628041497	\\
61.3000000000002	38.0901119932912	\\
61.3000000000002	38.0901119932912	\\
61.7600000000001	38.0900856860624	\\
61.7600000000001	38.0900856860624	\\
62.22	38.090085607878	\\
62.22	38.090085607878	\\
62.33	38.0900856077005	\\
62.6799999999999	38.0900856077005	\\
62.6799999999999	38.0900856077005	\\
63.1399999999998	38.0900856077005	\\
63.1399999999998	38.0900856077005	\\
63.5999999999997	38.0900856077005	\\
63.5999999999997	38.0900856077005	\\
64.0599999999997	38.0900856077005	\\
64.0599999999997	38.0900856077005	\\
64.5199999999999	38.0900856077005	\\
64.5199999999999	38.0900856077005	\\
64.9800000000002	38.0900856077005	\\
64.9800000000002	38.0900856077005	\\
65.4400000000004	38.0900856077005	\\
65.4400000000004	38.0900856077005	\\
65.9000000000006	38.0900856077005	\\
65.9000000000006	38.0900856077005	\\
66.3600000000009	38.0900856077005	\\
66.3600000000009	38.0900856077005	\\
66.8300000000011	38.0900856077005	\\
66.8300000000011	38.0900856077005	\\
67.2900000000013	38.0900856077005	\\
67.2900000000013	38.0900856077005	\\
67.7500000000016	38.0900856077005	\\
67.7500000000016	38.0900856077005	\\
68.2100000000018	38.0900856077005	\\
68.2100000000018	38.0900856077005	\\
68.670000000002	38.0900856077005	\\
68.670000000002	38.0900856077005	\\
69.1300000000023	38.0900856077005	\\
69.1300000000023	38.0900856077005	\\
69.5900000000025	38.0900856077005	\\
69.5900000000025	38.0900856077005	\\
70.0500000000028	38.0900856077005	\\
70.0500000000028	38.0900856077005	\\
70.510000000003	38.0900856077005	\\
70.510000000003	38.0900856077005	\\
70.9700000000032	38.0900856077005	\\
70.9700000000032	38.0900856077005	\\
71.4300000000035	38.0900856077005	\\
71.4300000000035	38.0900856077005	\\
71.8900000000037	38.0900856077005	\\
71.8900000000037	38.0900856077005	\\
72.3600000000039	38.0900856077005	\\
72.3600000000039	38.0900856077005	\\
72.8200000000042	38.0900856077005	\\
72.8200000000042	38.0900856077005	\\
73.2800000000044	38.0900856077005	\\
73.2800000000044	38.0900856077005	\\
73.7400000000046	38.0900856077005	\\
73.7400000000046	38.0900856077005	\\
74.2000000000049	38.0900856077005	\\
74.2000000000049	38.0900856077005	\\
74.6600000000051	38.0900856077005	\\
74.6600000000051	38.0900856077005	\\
75.1200000000053	38.0900856077005	\\
75.1200000000053	38.0900856077005	\\
75.5800000000056	38.0900856077005	\\
75.5800000000056	38.0900856077005	\\
76.0400000000058	38.0900856077005	\\
76.0400000000058	38.0900856077005	\\
76.5000000000061	38.0900856077005	\\
76.5000000000061	38.0900856077005	\\
76.9600000000063	38.0900856077005	\\
76.9600000000063	38.0900856077005	\\
77.4200000000065	38.0900856077005	\\
77.4200000000065	38.0900856077005	\\
77.8900000000068	38.0900856077005	\\
77.8900000000068	38.0900856077005	\\
78.350000000007	38.0900856077005	\\
78.350000000007	38.0900856077005	\\
78.8100000000072	38.0900856077005	\\
78.8100000000072	38.0900856077005	\\
79.2700000000075	38.0900856077005	\\
79.2700000000075	38.0900856077005	\\
79.9999999999991	38.0900856086007	\\
80.0056893337565	0	\\
80.1870680012667	0	\\
80.1870680012667	0	\\
80.6470680012669	0	\\
80.6470680012669	0	\\
81.1070680012672	0	\\
81.1070680012672	0	\\
81.5670680012674	0	\\
81.5670680012674	0	\\
82.0370680012676	0	\\
82.0370680012676	0	\\
82.4970680012679	0	\\
82.4970680012679	0	\\
82.9570680012681	0	\\
82.9570680012681	0	\\
83.4170680012683	0	\\
83.4170680012683	0	\\
83.8770680012686	0	\\
83.8770680012686	0	\\
84.3370680012688	0	\\
84.3370680012688	0	\\
84.797068001269	0	\\
84.797068001269	0	\\
85.2570680012693	0	\\
85.2570680012693	0	\\
85.7170680012695	0	\\
85.7170680012695	0	\\
86.1770680012697	0	\\
86.1770680012697	0	\\
86.63706800127	0	\\
86.63706800127	0	\\
87.0970680012702	0	\\
87.0970680012702	0	\\
87.5670680012705	0	\\
87.5670680012705	0	\\
88.0270680012707	0	\\
88.0270680012707	0	\\
88.4870680012709	0	\\
88.4870680012709	0	\\
88.9470680012712	0	\\
88.9470680012712	0	\\
89.4070680012714	0	\\
89.4070680012714	0	\\
89.8670680012716	0	\\
89.8670680012716	0	\\
90.3270680012719	0	\\
90.3270680012719	0	\\
90.7870680012721	0	\\
90.7870680012721	0	\\
91.2470680012723	0	\\
91.2470680012723	0	\\
91.7070680012726	0	\\
91.7070680012726	0	\\
92.1670680012728	0	\\
92.1670680012728	0	\\
92.627068001273	0	\\
92.627068001273	0	\\
93.0970680012733	0	\\
93.0970680012733	0	\\
93.5570680012735	0	\\
93.5570680012735	0	\\
94.0170680012738	0	\\
94.0170680012738	0	\\
94.477068001274	0	\\
94.477068001274	0	\\
94.9370680012742	0	\\
94.9370680012742	0	\\
95.3970680012745	0	\\
95.3970680012745	0	\\
95.8570680012747	0	\\
95.8570680012747	0	\\
96.3170680012749	0	\\
96.3170680012749	0	\\
96.7770680012752	0	\\
96.7770680012752	0	\\
97.2370680012754	0	\\
97.2370680012754	0	\\
97.6970680012756	0	\\
97.6970680012756	0	\\
98.1570680012759	0	\\
98.1570680012759	0	\\
98.6270680012761	0	\\
98.6270680012761	0	\\
99.0870680012764	0	\\
99.0870680012764	0	\\
99.5470680012766	0	\\
99.5470680012766	0	\\
100.007068001277	0	\\
100.007068001277	0	\\
100.467068001277	0	\\
100.467068001277	0	\\
100.927068001277	0	\\
100.927068001277	0	\\
101.387068001278	0	\\
101.387068001278	0	\\
101.847068001278	0	\\
101.847068001278	0	\\
102.307068001278	0	\\
102.307068001278	0	\\
102.767068001278	0	\\
102.767068001278	0	\\
103.227068001278	0	\\
103.227068001278	0	\\
103.687068001279	0	\\
103.687068001279	0	\\
104.157068001279	0	\\
104.157068001279	0	\\
104.617068001279	0	\\
104.617068001279	0	\\
105.077068001279	0	\\
105.077068001279	0	\\
105.53706800128	0	\\
105.53706800128	0	\\
105.99706800128	0	\\
105.99706800128	0	\\
106.45706800128	0	\\
106.45706800128	0	\\
106.91706800128	0	\\
106.91706800128	0	\\
107.377068001281	0	\\
107.377068001281	0	\\
107.837068001281	0	\\
107.837068001281	0	\\
108.297068001281	0	\\
108.297068001281	0	\\
108.757068001281	0	\\
108.757068001281	0	\\
109.217068001282	0	\\
109.217068001282	0	\\
109.687068001282	0	\\
109.687068001282	0	\\
110.147068001282	0	\\
110.147068001282	0	\\
110.607068001282	0	\\
110.607068001282	0	\\
111.067068001282	0	\\
111.067068001282	0	\\
111.527068001283	0	\\
111.527068001283	0	\\
111.987068001283	0	\\
111.987068001283	0	\\
112.447068001283	0	\\
112.447068001283	0	\\
112.907068001283	0	\\
112.907068001283	0	\\
113.367068001284	0	\\
113.367068001284	0	\\
113.827068001284	0	\\
113.827068001284	0	\\
114.287068001284	0	\\
114.287068001284	0	\\
114.747068001284	0	\\
114.747068001284	0	\\
115.217068001285	0	\\
115.217068001285	0	\\
115.677068001285	0	\\
115.677068001285	0	\\
116.137068001285	0	\\
116.137068001285	0	\\
116.597068001285	0	\\
116.597068001285	0	\\
117.057068001286	0	\\
117.057068001286	0	\\
117.517068001286	0	\\
117.517068001286	0	\\
117.977068001286	0	\\
117.977068001286	0	\\
118.437068001286	0	\\
118.437068001286	0	\\
118.897068001286	0	\\
118.897068001286	0	\\
119.357068001287	0	\\
119.357068001287	0	\\
119.817068001287	0	\\
119.817068001287	0	\\
120.277068001287	0	\\
120.277068001287	0	\\
120.747068001287	0	\\
120.747068001287	0	\\
121.207068001288	0	\\
121.207068001288	0	\\
121.667068001288	0	\\
121.667068001288	0	\\
122.127068001288	0	\\
122.127068001288	0	\\
122.587068001288	0	\\
122.587068001288	0	\\
123.047068001289	0	\\
123.047068001289	0	\\
123.507068001289	0	\\
123.507068001289	0	\\
123.967068001289	0	\\
123.967068001289	0	\\
124.427068001289	0	\\
124.887474866462	226.9931103245	\\
124.887474866462	226.9931103245	\\
125.347474866462	426.478281681523	\\
125.347474866462	426.478281681523	\\
125.807474866462	427.372480289837	\\
125.807474866462	427.372480289837	\\
126.267474866462	427.68971397818	\\
126.267474866462	427.68971397818	\\
126.737474866463	428.025833446703	\\
126.737474866463	428.025833446703	\\
127.197474866463	428.368285011855	\\
127.197474866463	428.368285011855	\\
127.657474866463	428.724079969297	\\
127.657474866463	428.724079969297	\\
128.117474866463	429.093218097927	\\
128.117474866463	429.093218097927	\\
128.577474866463	429.475699164193	\\
128.577474866463	429.475699164193	\\
129.037474866462	429.871522933643	\\
129.037474866462	429.871522933643	\\
129.497474866462	430.280689171851	\\
129.497474866462	430.280689171851	\\
129.957474866462	430.703197644506	\\
129.957474866462	430.703197644506	\\
130.417474866461	431.139048117345	\\
130.417474866461	431.139048117345	\\
130.877474866461	431.588240356081	\\
130.877474866461	431.588240356081	\\
131.33747486646	432.050774126428	\\
131.33747486646	432.050774126428	\\
131.79747486646	432.52664919415	\\
131.79747486646	432.52664919415	\\
132.26747486646	433.026648620471	\\
132.26747486646	433.026648620471	\\
132.727474866459	433.52949559561	\\
132.727474866459	433.52949559561	\\
133.187474866459	434.045683160229	\\
133.187474866459	434.045683160229	\\
133.647474866458	434.575211080145	\\
133.647474866458	434.575211080145	\\
134.107474866458	435.118079121158	\\
134.107474866458	435.118079121158	\\
134.567474866457	435.674287048994	\\
134.997474866457	436.206283606791	\\
135.027474866457	321.172048258406	\\
135.027474866457	321.172048258406	\\
135.487474866457	45.3422412587046	\\
135.487474866457	45.3422412587046	\\
135.947474866456	44.5230000223547	\\
135.947474866456	44.5230000223547	\\
136.407474866456	44.5205667974219	\\
136.407474866456	44.5205667974219	\\
136.867474866455	44.5205595705099	\\
136.867474866455	44.5205595705099	\\
137.327474866455	44.5205595490336	\\
137.327474866455	44.5205595490336	\\
137.337474866455	44.5205595490209	\\
137.797474866454	44.5205595490209	\\
137.797474866454	44.5205595490209	\\
138.257474866454	44.5205595490209	\\
138.257474866454	44.5205595490209	\\
138.717474866454	44.5205595490209	\\
138.717474866454	44.5205595490209	\\
139.177474866453	44.5205595490209	\\
139.177474866453	44.5205595490209	\\
139.637474866453	44.5205595490209	\\
139.637474866453	44.5205595490209	\\
140.097474866452	44.5205595490209	\\
140.097474866452	44.5205595490209	\\
140.557474866452	44.5205595490209	\\
140.557474866452	44.5205595490209	\\
141.017474866452	44.5205595490209	\\
141.017474866452	44.5205595490209	\\
141.477474866451	44.5205595490209	\\
141.477474866451	44.5205595490209	\\
141.937474866451	44.5205595490209	\\
141.937474866451	44.5205595490209	\\
142.39747486645	44.5205595490209	\\
142.39747486645	44.5205595490209	\\
142.85747486645	44.5205595490209	\\
142.85747486645	44.5205595490209	\\
143.327474866449	44.5205595490209	\\
143.327474866449	44.5205595490209	\\
143.787474866449	44.5205595490209	\\
143.787474866449	44.5205595490209	\\
144.247474866449	44.5205595490209	\\
144.247474866449	44.5205595490209	\\
144.707474866448	44.5205595490209	\\
144.707474866448	44.5205595490209	\\
145.167474866448	44.5205595490209	\\
145.167474866448	44.5205595490209	\\
145.627474866447	44.5205595490209	\\
145.627474866447	44.5205595490209	\\
146.087474866447	44.5205595490209	\\
146.087474866447	44.5205595490209	\\
146.547474866447	44.5205595490209	\\
146.547474866447	44.5205595490209	\\
147.007474866446	44.5205595490209	\\
147.007474866446	44.5205595490209	\\
147.467474866446	44.5205595490209	\\
147.467474866446	44.5205595490209	\\
147.927474866445	44.5205595490209	\\
147.927474866445	44.5205595490209	\\
148.387474866445	44.5205595490209	\\
148.387474866445	44.5205595490209	\\
148.857474866444	44.5205595490209	\\
148.857474866444	44.5205595490209	\\
149.317474866444	44.5205595490209	\\
149.317474866444	44.5205595490209	\\
149.777474866444	44.5205595490209	\\
149.777474866444	44.5205595490209	\\
150.237474866443	44.5205595490209	\\
150.237474866443	44.5205595490209	\\
150.697474866443	44.5205595490209	\\
150.697474866443	44.5205595490209	\\
151.157474866442	44.5205595490209	\\
151.157474866442	44.5205595490209	\\
151.617474866442	44.5205595490209	\\
151.617474866442	44.5205595490209	\\
152.077474866441	44.5205595490209	\\
152.077474866441	44.5205595490209	\\
152.537474866441	44.5205595490209	\\
152.537474866441	44.5205595490209	\\
152.997474866441	44.5205595490209	\\
152.997474866441	44.5205595490209	\\
153.45747486644	44.5205595490209	\\
153.45747486644	44.5205595490209	\\
153.91747486644	44.5205595490209	\\
153.91747486644	44.5205595490209	\\
154.387474866439	44.5205595490209	\\
154.387474866439	44.5205595490209	\\
154.847474866439	44.5205595490209	\\
155.019479601956	0	\\
155.309479601952	0	\\
155.309479601952	0	\\
155.769479601952	0	\\
155.769479601952	0	\\
156.229479601951	0	\\
156.229479601951	0	\\
156.689479601951	0	\\
156.689479601951	0	\\
157.14947960195	0	\\
157.14947960195	0	\\
157.60947960195	0	\\
157.60947960195	0	\\
158.06947960195	0	\\
158.06947960195	0	\\
158.529479601949	0	\\
158.529479601949	0	\\
158.989479601949	0	\\
158.989479601949	0	\\
159.449479601948	0	\\
159.449479601948	0	\\
159.909479601948	0	\\
159.909479601948	0	\\
160.369479601947	0	\\
160.369479601947	0	\\
160.839479601947	0	\\
160.839479601947	0	\\
161.299479601947	0	\\
161.299479601947	0	\\
161.759479601946	0	\\
161.759479601946	0	\\
162.219479601946	0	\\
162.219479601946	0	\\
162.679479601945	0	\\
162.679479601945	0	\\
163.139479601945	0	\\
163.139479601945	0	\\
163.599479601944	0	\\
163.599479601944	0	\\
164.059479601944	0	\\
164.059479601944	0	\\
164.519479601944	0	\\
164.519479601944	0	\\
164.979479601943	0	\\
164.979479601943	0	\\
165.439479601943	0	\\
165.439479601943	0	\\
165.899479601942	0	\\
165.899479601942	0	\\
166.359479601942	0	\\
166.359479601942	0	\\
166.829479601942	0	\\
166.829479601942	0	\\
167.289479601941	0	\\
167.289479601941	0	\\
167.749479601941	0	\\
167.749479601941	0	\\
168.20947960194	0	\\
168.20947960194	0	\\
168.66947960194	0	\\
168.66947960194	0	\\
169.129479601939	0	\\
169.129479601939	0	\\
169.589479601939	0	\\
169.589479601939	0	\\
170.049479601939	0	\\
170.049479601939	0	\\
170.509479601938	0	\\
170.509479601938	0	\\
170.969479601938	0	\\
170.969479601938	0	\\
171.429479601937	0	\\
171.429479601937	0	\\
171.889479601937	0	\\
171.889479601937	0	\\
172.359479601937	0	\\
172.359479601937	0	\\
172.819479601936	0	\\
172.819479601936	0	\\
173.279479601936	0	\\
173.279479601936	0	\\
173.739479601935	0	\\
173.739479601935	0	\\
174.199479601935	0	\\
174.199479601935	0	\\
174.659479601934	0	\\
174.659479601934	0	\\
175.119479601934	0	\\
175.119479601934	0	\\
175.579479601934	0	\\
175.579479601934	0	\\
176.039479601933	0	\\
176.039479601933	0	\\
176.499479601933	0	\\
176.499479601933	0	\\
176.959479601932	0	\\
176.959479601932	0	\\
177.419479601932	0	\\
177.419479601932	0	\\
177.889479601931	0	\\
177.889479601931	0	\\
178.349479601931	0	\\
178.349479601931	0	\\
178.809479601931	0	\\
178.809479601931	0	\\
179.26947960193	0	\\
179.26947960193	0	\\
179.72947960193	0	\\
179.72947960193	0	\\
180.189479601929	0	\\
180.189479601929	0	\\
180.649479601929	0	\\
180.649479601929	0	\\
181.109479601929	0	\\
181.109479601929	0	\\
181.569479601928	0	\\
181.569479601928	0	\\
182.029479601928	0	\\
182.029479601928	0	\\
182.489479601927	0	\\
182.489479601927	0	\\
182.949479601927	0	\\
182.949479601927	0	\\
183.419479601926	0	\\
183.419479601926	0	\\
183.879479601926	0	\\
183.879479601926	0	\\
184.339479601926	0	\\
184.339479601926	0	\\
184.799479601925	0	\\
184.799479601925	0	\\
185.259479601925	0	\\
185.259479601925	0	\\
185.719479601924	0	\\
185.719479601924	0	\\
186.179479601924	0	\\
186.179479601924	0	\\
186.639479601924	0	\\
186.639479601924	0	\\
187.099479601923	0	\\
187.099479601923	0	\\
187.559479601923	0	\\
187.559479601923	0	\\
188.019479601922	0	\\
188.019479601922	0	\\
188.479479601922	0	\\
188.479479601922	0	\\
188.949479601921	0	\\
188.949479601921	0	\\
189.409479601921	0	\\
189.409479601921	0	\\
189.869479601921	0	\\
189.869479601921	0	\\
190.32947960192	0	\\
190.32947960192	0	\\
190.78947960192	0	\\
190.78947960192	0	\\
191.249479601919	0	\\
191.249479601919	0	\\
191.709479601919	0	\\
191.709479601919	0	\\
192.169479601918	0	\\
192.169479601918	0	\\
192.629479601918	0	\\
192.629479601918	0	\\
193.089479601918	0	\\
193.089479601918	0	\\
193.549479601917	0	\\
193.549479601917	0	\\
194.009479601917	0	\\
194.009479601917	0	\\
194.479479601916	0	\\
194.479479601916	0	\\
194.939479601916	0	\\
194.939479601916	0	\\
195.399479601916	0	\\
195.399479601916	0	\\
195.859479601915	0	\\
195.859479601915	0	\\
196.319479601915	0	\\
196.319479601915	0	\\
196.779479601914	0	\\
196.779479601914	0	\\
197.239479601914	0	\\
197.239479601914	0	\\
197.699479601913	0	\\
197.699479601913	0	\\
198.159479601913	0	\\
198.159479601913	0	\\
198.619479601913	0	\\
198.619479601913	0	\\
199.079479601912	0	\\
199.079479601912	0	\\
200	0	\\
};
\addlegendentry{Torque on the wheels (Nm)};

\end{axis}
\end{tikzpicture}%
    	\label{fig:wheel-torque}
    	\caption{The torque on the wheels.}
\end{figure}
\end{document}