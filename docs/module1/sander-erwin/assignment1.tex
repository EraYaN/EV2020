%!TEX program = xelatex+makeindex+bibtex

\documentclass[final]{scrreprt} %scrreprt of scrartcl
\input{../../library/preamble.tex}
\input{../../library/style.tex}
\addbibresource{../library/bibliography.bib}
\title{Module 1 - Report}
\author{Sander {van Dijk} \and Erwin {de Haan}}
\begin{document}

\chapter{Assignment 1}
Underneath you can find the subcircuits we designed for the DC-DC converter, splitted in the DC-AC and AC-DC part.

\begin{figure}[h]
	\label{fig:DC-AC}
	\includegraphics[width=\linewidth]{resources/DC-AC-rc.pdf}
	\caption{DC-AC subcircuit}
\end{figure}

\begin{figure}[h]
	\label{fig:AC-DC}
	\includegraphics[width=\linewidth]{resources/AC-DC-rc.pdf}
	\caption{AC-DC subcircuit}
\end{figure}

As described in Figure \ref{fig:DC-AC}, we used the IPP028N08N3G (OptiMOS 3) transistor instead of the IPP50CN10N (OptiMOS 2).
We had two reasons for this descision.
First, the drain-source on-state resistance was approximally ten times lower on the OptiMOS 3, causing the transistor to generate less heat and gain better efficiency.
Secondly, by looking at the current-voltage characteristics per frequency (\cite{OptiMOS2} \emph{:3 Safe operating area} and \cite{OptiMOS3} \emph{:3 Safe operating area}) we found out that the OptiMOS 3 performs better at higher frequencies than the OptiMOS 2.
\\ \\
We also had to choose the diodes used on the secondary side of the tranformer. The diodes will form a full bridge rectifier, which in combination with a capacitor, will form a AC-DC converter.
The candidates are the SB540 (datasheet: \ref{SB540}) and the SF61 (datasheet: \ref{SF61}).
The forward voltage of the two diodes differ, the SB540's is lower. This is favorable since there will be more voltage left on the output terminals.


AUTO:
500g - 13 cm per band
auto: 6kg

opladen supercap 4:30 min

\end{document}