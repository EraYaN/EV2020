\documentclass[final]{scrreprt} %scrreprt of scrartcl
\usepackage{amsmath}
% Include all project wide packages here.
\usepackage{fullpage}
\usepackage{polyglossia}
\setmainlanguage{english}
\usepackage{csquotes}
\usepackage{graphicx}
\usepackage{epstopdf}
\usepackage{pdfpages}
\usepackage{caption}
\usepackage[list=true]{subcaption}
\usepackage{float}
\usepackage{standalone}
\usepackage{import}
\usepackage{tocloft}
\usepackage{wrapfig}
\usepackage{authblk}
\usepackage{array}
\usepackage{booktabs}
\usepackage[toc,page,title,titletoc]{appendix}
\usepackage{xunicode}
\usepackage{fontspec}
\usepackage{pgfplots}
\usepackage{SIunits}
\usepackage{units}
\pgfplotsset{compat=newest}
\pgfplotsset{plot coordinates/math parser=false}
\newlength\figureheight 
\newlength\figurewidth
\usepackage{amsmath}
\usepackage{mathtools}
\usepackage{unicode-math}
\usepackage[
    backend=bibtexu,
	texencoding=utf8,
bibencoding=utf8,
    style=ieee,
    sortlocale=en_US,
    language=auto
]{biblatex}
\usepackage{listings}
\newcommand{\includecode}[3][c]{\lstinputlisting[caption=#2, escapechar=, style=#1]{#3}}
\newcommand{\superscript}[1]{\ensuremath{^{\textrm{#1}}}}
\newcommand{\subscript}[1]{\ensuremath{_{\textrm{#1}}}}


\newcommand{\chapternumber}{\thechapter}
\renewcommand{\appendixname}{Bijlage}
\renewcommand{\appendixtocname}{Bijlagen}
\renewcommand{\appendixpagename}{Bijlagen}

\usepackage[hidelinks]{hyperref} %<--------ALTIJD ALS LAATSTE

\renewcommand{\familydefault}{\sfdefault}

\setmainfont[Ligatures=TeX]{Myriad Pro}
\setmathfont{Asana Math}
\setmonofont{Lucida Console}

\usepackage{titlesec, blindtext, color}
\definecolor{gray75}{gray}{0.75}
\newcommand{\hsp}{\hspace{20pt}}
\titleformat{\chapter}[hang]{\Huge\bfseries}{\chapternumber\hsp\textcolor{gray75}{|}\hsp}{0pt}{\Huge\bfseries}
\renewcommand{\familydefault}{\sfdefault}
\renewcommand{\arraystretch}{1.2}
\setlength\parindent{0pt}

%For code listings
\definecolor{black}{rgb}{0,0,0}
\definecolor{browntags}{rgb}{0.65,0.1,0.1}
\definecolor{bluestrings}{rgb}{0,0,1}
\definecolor{graycomments}{rgb}{0.4,0.4,0.4}
\definecolor{redkeywords}{rgb}{1,0,0}
\definecolor{bluekeywords}{rgb}{0.13,0.13,0.8}
\definecolor{greencomments}{rgb}{0,0.5,0}
\definecolor{redstrings}{rgb}{0.9,0,0}
\definecolor{purpleidentifiers}{rgb}{0.01,0,0.01}


\lstdefinestyle{csharp}{
language=[Sharp]C,
showspaces=false,
showtabs=false,
breaklines=true,
showstringspaces=false,
breakatwhitespace=true,
escapeinside={(*@}{@*)},
columns=fullflexible,
commentstyle=\color{greencomments},
keywordstyle=\color{bluekeywords}\bfseries,
stringstyle=\color{redstrings},
identifierstyle=\color{purpleidentifiers},
basicstyle=\ttfamily\small}

\lstdefinestyle{c}{
language=C,
showspaces=false,
showtabs=false,
breaklines=true,
showstringspaces=false,
breakatwhitespace=true,
escapeinside={(*@}{@*)},
columns=fullflexible,
commentstyle=\color{greencomments},
keywordstyle=\color{bluekeywords}\bfseries,
stringstyle=\color{redstrings},
identifierstyle=\color{purpleidentifiers},
}

\lstdefinestyle{matlab}{
language=Matlab,
showspaces=false,
showtabs=false,
breaklines=true,
showstringspaces=false,
breakatwhitespace=true,
escapeinside={(*@}{@*)},
columns=fullflexible,
commentstyle=\color{greencomments},
keywordstyle=\color{bluekeywords}\bfseries,
stringstyle=\color{redstrings},
identifierstyle=\color{purpleidentifiers}
}

\lstdefinestyle{vhdl}{
language=VHDL,
showspaces=false,
showtabs=false,
breaklines=true,
showstringspaces=false,
breakatwhitespace=true,
escapeinside={(*@}{@*)},
columns=fullflexible,
commentstyle=\color{greencomments},
keywordstyle=\color{bluekeywords}\bfseries,
stringstyle=\color{redstrings},
identifierstyle=\color{purpleidentifiers}
}

\lstdefinestyle{xaml}{
language=XML,
showspaces=false,
showtabs=false,
breaklines=true,
showstringspaces=false,
breakatwhitespace=true,
escapeinside={(*@}{@*)},
columns=fullflexible,
commentstyle=\color{greencomments},
keywordstyle=\color{redkeywords},
stringstyle=\color{bluestrings},
tagstyle=\color{browntags},
morestring=[b]",
  morecomment=[s]{<?}{?>},
  morekeywords={xmlns,version,typex:AsyncRecords,x:Arguments,x:Boolean,x:Byte,x:Char,x:Class,x:ClassAttributes,x:ClassModifier,x:Code,x:ConnectionId,x:Decimal,x:Double,x:FactoryMethod,x:FieldModifier,x:Int16,x:Int32,x:Int64,x:Key,x:Members,x:Name,x:Object,x:Property,x:Shared,x:Single,x:String,x:Subclass,x:SynchronousMode,x:TimeSpan,x:TypeArguments,x:Uid,x:Uri,x:XData,Grid.Column,Grid.ColumnSpan,Click,ClipToBounds,Content,DropDownOpened,FontSize,Foreground,Header,Height,HorizontalAlignment,HorizontalContentAlignment,IsCancel,IsDefault,IsEnabled,IsSelected,Margin,MinHeight,MinWidth,Padding,SnapsToDevicePixels,Target,TextWrapping,Title,VerticalAlignment,VerticalContentAlignment,Width,WindowStartupLocation,Binding,Mode,OneWay,xmlns:x}
}

%defaults
\lstset{
basicstyle=\ttfamily\small,
extendedchars=false,
numbers=left,
numberstyle=\ttfamily\tiny,
stepnumber=1,
tabsize=4,
numbersep=5pt
}
\addbibresource{../library/bibliography.bib}
\title{Module 1 - Report}
\author{Xenia {Wesdijk}}
\begin{document}

\chapter{Assignment 3}
For assignment 3 we had to design a circuit of capacitors to compensate the leakage inductances.

To minimize the effect of the leakage inductances we had to find the values of $C_1$ and $C_2$. We used equations \ref{equation1} and \ref{equation2} to find these values.\\
\begin{equation}\
\label{equation1}
\begin{array}{l}
$$ 0 = j\omega L_{1} + \frac{1}{j\omega C_{1}}$$
$$\Leftrightarrow C_{1} =  \frac{1}{\omega^2 L_{1}}$$
\end{array}\
\end{equation}
\begin{equation}\
\label{equation2}
\begin{array}{l}
$$ 0 = j\omega L_{2} + \frac{1}{j\omega C_{2}}$$
$$\Leftrightarrow C_{2} =  \frac{1}{\omega^2 L_{2}}$$
\end{array}\
\end{equation}
\\
Using these equations we found the following values for $C_1$ and $C_2$:
\begin{equation}\
\begin{array}{l}
$$C_{1} =  \frac{1}{(\pi \cdot (100\cdot 10^3)) ^2 110.8 \cdot 10^-6} = 23\nano F$$
\end{array}\
\end{equation}
\begin{equation}\
\begin{array}{l}
$$C_{1} =  \frac{1}{(\pi \cdot (100\cdot 10^3)) ^2 19.8 \cdot 10^-6} = 128\nano F$$
\end{array}\
\end{equation}
\\

Since our calculated values for $C_1$ (23\nano F) and $C_2$ (128\nano F) weren't available. We had to choose between the following values. On the primary side we only needed 23 \nano F, so we used the 22 \nano F that was available. For $C_2$ however, we needed 128\nano F. On the PCB that was given to us we could place a total of 4 capacitors. The problem was that there were 2 sets of 2 capacitors who could be placed in parallel, but those 2 sets were in serie. So we had to place 2 capacitors in parallel and we got an value of 136 \nano F instead of 128 \nano F. This was the best solution for us and we thought we could always later change our coils if needed.
\begin{itemize}
\item 68 nF
\item 47 nF
\item 33 nF
\item 22 nF
\item 15 nF
\end{itemize}
\\


When using capacitors in our circuit the matrix equation (as described in assignment 2) will change. The new matrix equations will be as follows: \\
\[\left[ {\begin{array}{*{20}{c}}
{{{\boldsymbol{\mathrm{V_1}}}}}\\
{{\boldsymbol{\mathrm{V_2}}}}

\end{array}} \right] = \left[ {\begin{array}{*{20}{c}}
{j\omega {L_{1}} + {\frac{1}{j\omega C_{1}}} + R_{1}}&{ - j\omega M}\\
{j\omega M}&{ - \left( {j\omega {L_{2}} + {\frac{1}{j\omega C_{2}}} + R_{2} + R} \right)}
\end{array}} \right]\left[ {\begin{array}{*{20}{c}}

{{\boldsymbol{\mathrm{I_1}}}}\\
{{\boldsymbol{\mathrm{I_2}}}}
\end{array}} \right]\]
\\

Since we have now compensated our circuit our values for the efficiency and power factor will change. The new values can be found in table \ref{table3}. \\

\begin{table}[h]
\begin{center}
\begin{tabular}{ l | l | l | l }
    
    \textbf{\textit{d} (cm)}            & \textbf{pf}              & \textbf{effici\"{e}ntie}  &  \textbf{P load (W)}\\	\hline
    6                           & 0.5182                     & 0.7509                & 23.6001  \\
    4                           &0.7746                  & 0.8722                     & 28.8824\\
    2                           & 0.9373                     & 0.9368                    &  18.2794 \\
\end{tabular}
\caption{Power factor and efficiency for the different distances when the circuit is compensated.}
\label{table3}
\end{center}
\end{table}
\\

The last thing we had to do for this assignment was test our circuit and charge the capacitor bank to 20V. \\ 
The results of the testing can be found in table \ref{measure1}.\\

It took our circuit, with a distance of 2cm between the coils, 6 minutes and 40 seconds to charge the bank to 20V 

\begin{table} [h]
\begin{center}
	\begin{tabular}{ l | l | l | l | l }
	Primary current (A) & Secondary voltage (V) & Output current & Power delivered to 10 $\Omega$ load \\ \hline
  	0.1537 & 12.34 & 1.234 &15.23 \\
	
	\end{tabular}
	\caption{Measurements on the air coupled transformer when compensated.}
	\label{measure1}
\end{center}
\end{table}



\end{document}