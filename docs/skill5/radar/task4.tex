%!TEX program = xelatex+makeindex+bibtex
\documentclass[final]{scrreprt} %scrreprt of scrartcl
% Include all project wide packages here.
\usepackage{fullpage}
\usepackage{polyglossia}
\setmainlanguage{english}
\usepackage{csquotes}
\usepackage{graphicx}
\usepackage{epstopdf}
\usepackage{pdfpages}
\usepackage{caption}
\usepackage[list=true]{subcaption}
\usepackage{float}
\usepackage{standalone}
\usepackage{import}
\usepackage{tocloft}
\usepackage{wrapfig}
\usepackage{authblk}
\usepackage{array}
\usepackage{booktabs}
\usepackage[toc,page,title,titletoc]{appendix}
\usepackage{xunicode}
\usepackage{fontspec}
\usepackage{pgfplots}
\usepackage{SIunits}
\usepackage{units}
\pgfplotsset{compat=newest}
\pgfplotsset{plot coordinates/math parser=false}
\newlength\figureheight 
\newlength\figurewidth
\usepackage{amsmath}
\usepackage{mathtools}
\usepackage{unicode-math}
\usepackage[
    backend=bibtexu,
	texencoding=utf8,
bibencoding=utf8,
    style=ieee,
    sortlocale=en_US,
    language=auto
]{biblatex}
\usepackage{listings}
\newcommand{\includecode}[3][c]{\lstinputlisting[caption=#2, escapechar=, style=#1]{#3}}
\newcommand{\superscript}[1]{\ensuremath{^{\textrm{#1}}}}
\newcommand{\subscript}[1]{\ensuremath{_{\textrm{#1}}}}


\newcommand{\chapternumber}{\thechapter}
\renewcommand{\appendixname}{Bijlage}
\renewcommand{\appendixtocname}{Bijlagen}
\renewcommand{\appendixpagename}{Bijlagen}

\usepackage[hidelinks]{hyperref} %<--------ALTIJD ALS LAATSTE

\renewcommand{\familydefault}{\sfdefault}

\setmainfont[Ligatures=TeX]{Myriad Pro}
\setmathfont{Asana Math}
\setmonofont{Lucida Console}

\usepackage{titlesec, blindtext, color}
\definecolor{gray75}{gray}{0.75}
\newcommand{\hsp}{\hspace{20pt}}
\titleformat{\chapter}[hang]{\Huge\bfseries}{\chapternumber\hsp\textcolor{gray75}{|}\hsp}{0pt}{\Huge\bfseries}
\renewcommand{\familydefault}{\sfdefault}
\renewcommand{\arraystretch}{1.2}
\setlength\parindent{0pt}

%For code listings
\definecolor{black}{rgb}{0,0,0}
\definecolor{browntags}{rgb}{0.65,0.1,0.1}
\definecolor{bluestrings}{rgb}{0,0,1}
\definecolor{graycomments}{rgb}{0.4,0.4,0.4}
\definecolor{redkeywords}{rgb}{1,0,0}
\definecolor{bluekeywords}{rgb}{0.13,0.13,0.8}
\definecolor{greencomments}{rgb}{0,0.5,0}
\definecolor{redstrings}{rgb}{0.9,0,0}
\definecolor{purpleidentifiers}{rgb}{0.01,0,0.01}


\lstdefinestyle{csharp}{
language=[Sharp]C,
showspaces=false,
showtabs=false,
breaklines=true,
showstringspaces=false,
breakatwhitespace=true,
escapeinside={(*@}{@*)},
columns=fullflexible,
commentstyle=\color{greencomments},
keywordstyle=\color{bluekeywords}\bfseries,
stringstyle=\color{redstrings},
identifierstyle=\color{purpleidentifiers},
basicstyle=\ttfamily\small}

\lstdefinestyle{c}{
language=C,
showspaces=false,
showtabs=false,
breaklines=true,
showstringspaces=false,
breakatwhitespace=true,
escapeinside={(*@}{@*)},
columns=fullflexible,
commentstyle=\color{greencomments},
keywordstyle=\color{bluekeywords}\bfseries,
stringstyle=\color{redstrings},
identifierstyle=\color{purpleidentifiers},
}

\lstdefinestyle{matlab}{
language=Matlab,
showspaces=false,
showtabs=false,
breaklines=true,
showstringspaces=false,
breakatwhitespace=true,
escapeinside={(*@}{@*)},
columns=fullflexible,
commentstyle=\color{greencomments},
keywordstyle=\color{bluekeywords}\bfseries,
stringstyle=\color{redstrings},
identifierstyle=\color{purpleidentifiers}
}

\lstdefinestyle{vhdl}{
language=VHDL,
showspaces=false,
showtabs=false,
breaklines=true,
showstringspaces=false,
breakatwhitespace=true,
escapeinside={(*@}{@*)},
columns=fullflexible,
commentstyle=\color{greencomments},
keywordstyle=\color{bluekeywords}\bfseries,
stringstyle=\color{redstrings},
identifierstyle=\color{purpleidentifiers}
}

\lstdefinestyle{xaml}{
language=XML,
showspaces=false,
showtabs=false,
breaklines=true,
showstringspaces=false,
breakatwhitespace=true,
escapeinside={(*@}{@*)},
columns=fullflexible,
commentstyle=\color{greencomments},
keywordstyle=\color{redkeywords},
stringstyle=\color{bluestrings},
tagstyle=\color{browntags},
morestring=[b]",
  morecomment=[s]{<?}{?>},
  morekeywords={xmlns,version,typex:AsyncRecords,x:Arguments,x:Boolean,x:Byte,x:Char,x:Class,x:ClassAttributes,x:ClassModifier,x:Code,x:ConnectionId,x:Decimal,x:Double,x:FactoryMethod,x:FieldModifier,x:Int16,x:Int32,x:Int64,x:Key,x:Members,x:Name,x:Object,x:Property,x:Shared,x:Single,x:String,x:Subclass,x:SynchronousMode,x:TimeSpan,x:TypeArguments,x:Uid,x:Uri,x:XData,Grid.Column,Grid.ColumnSpan,Click,ClipToBounds,Content,DropDownOpened,FontSize,Foreground,Header,Height,HorizontalAlignment,HorizontalContentAlignment,IsCancel,IsDefault,IsEnabled,IsSelected,Margin,MinHeight,MinWidth,Padding,SnapsToDevicePixels,Target,TextWrapping,Title,VerticalAlignment,VerticalContentAlignment,Width,WindowStartupLocation,Binding,Mode,OneWay,xmlns:x}
}

%defaults
\lstset{
basicstyle=\ttfamily\small,
extendedchars=false,
numbers=left,
numberstyle=\ttfamily\tiny,
stepnumber=1,
tabsize=4,
numbersep=5pt
}
\addbibresource{../../library/bibliography.bib}
\begin{document}

\chapter{Task 4}
The half power beam width is determined in this experiment. This is done by finding the -3 dB tangential distance at several distances from the radar. The two horizontal axes - the center of the beam and the tangent of this beam, parallel to the floor - are moved separately. First the target is placed at a distance R from the radar, at a place where the beam is strongest. The strength of this beam is measured on the computer screen, which is displayed in Volts. This beam strength must be converted to decibels, subtracted 3 dB's and then convert this value back to Volts to find the -3 dB voltage. The following conversion is used:

\begin{equation}
	V_{dB} = 20 \cdot \log_{10} (V)
\end{equation}

\begin{equation}
	V_{-3 dB} = 10^{\frac{V_{dB} - 3}{20}} = 10^{\frac{20 \cdot \log_{10}(V) - 3}{20}} = 10^{\log_{10}(V)} \cdot 10^{\frac{-3}{20}} = 10^{\frac{-3}{20}} V \approx V~/~\sqrt{2}
\end{equation}

The -3 dB does not exactly match the half power location, which would be exactly $20 \cdot \log_{10} (\frac{1}{\sqrt{2}}) \approx -3.0103~dB $. When this exact number is used instead of the rounded -3 dB, the last approximately equal sign will change into an exactly equally sign.

\section{Experimental procedure}
The metal target was placed at a measured distance in front of the radar. The exact center of the beam was found by maximizing the output voltage from the radar. This output voltage was found by determining the peak to peak voltage that corresponds with the target. To check whether a peak corresponds to the target, the target must be moved a bit to the front and back to see the peak move horizontally (the delay time changes) or be panned a bit which must result on a changing peak to peak voltage on the screen.

When the corresponding peak is found, it must be maximized by panning and moving the target tangentially. The found maximum peak to peak voltage is now used to find a -3 dB voltage. The target must now be moved tangentially from the beam center to get a result equal to this -3 dB voltage. The moved tangential distance is measured and used to calculate the beam width $\beta$.

\section{Results}
The data was measured at three distances from the radar: 2.18 \unit{m}, 3.50 \unit{m} and 5.00 \unit{m}. The beam width $\beta$ - the angle of the beam at the radar from a -3 dB to the opposite -3 dB beam strength - is calculated using the following equation:

\begin{equation}
	\beta = 2 \arctan(L / R)
\end{equation}

 The results are shown in Table \ref{tab:measure}.

\begin{table}[h]
\begin{center}
\begin{tabular}{ | l | l | l | }
    \hline
    Beam center distance R (m) & Tangential distance L (cm) & Beamwidth $\beta$ (\textdegree) \\\hline
    2.18 & 15.8 & 3.00 \\\hline
    3.50 & 16.3 & 3.05 \\\hline
    5.00 & 28.5 & 3.03 \\\hline
\end{tabular}
\caption{Measure data plus results.}
\end{center}
\end{table}
\label{tab:measure}

With these three measurements at different distances R from the radar, three different values of tangential distances L are found at which the beam power is halved. The three resulting beam widths $\beta$ are very close to each other, which makes the found beam width assumable. The value of this beam width 3.0\textdegree ~ is rather small, which indicates a slim beam with very focused beam strength.

\end{document}