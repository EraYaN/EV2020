%!TEX program = xelatex+makeindex+bibtex
\documentclass[final]{scrreprt} %scrreprt of scrartcl
\input{../../library/preamble.tex}
\input{../../library/style.tex}
\addbibresource{../../library/bibliography.bib}
\title{Skills 5 - Report}
\author{Alex {Misdorp} \and Sander {van Dijk}}
\begin{document}

\chapter{Task 5}
\section{Theory}
The following experiment will determine the range dependence of the received power. 
To determine this we use a movable reference target that's positioned straight in front of the radar. 
Subsequently we move the reference target further away multiple times and acquire the corresponding signal amplitude from the radar's display.
We assume the signal amplitude of the signal will decrease because as we move the reference target further away, a smaller portion of the signal will be reflected.

\section{Experimental procedure}
First off the target is placed in such a way to receive maximum power at the radar.\\
This means shoveling the target reference around a bit until this goal is reached.\\
Ideally, the mirror side of the reference target should be parallel to the radar.\\
Subsequently the reference target is moved backwards along a line taped on to the ground.
The distance is measured using a tapeline.

\section{Results}
\begin{table}[H]
\begin{center}
\begin{tabular}{|c | c | c | c|}
\hline
Distance (m) & Voltage (mV) & Power (dB) & Increase in power (dB)\\
%\hline
%2 & 85,389 & -21,372 & 0\\
%\hline
%2,5 & 84,080 & -21.506 & \\
\hline
3 & 74,904 & -22,510 & 0\\
\hline
3,5 & 59,965 & -24.442 & -1,932\\
\hline
4 & 50,458 &  -25.941 & - 1,500\\
\hline
4,5 & 42,414 & -27.450 & - 1,508\\
\hline
5 & 35,624 & -28.965 & -1,515\\
\hline
5,5 & 30,191 & -30,403 & -1,437\\
\hline
6 & 25,595 & -31.837 & -1,434\\
\hline
6,5 & 21,312 & -33.428 & -1,591\\
\hline
\end{tabular}
\caption{Measured voltages at different ranges and the corresponding power.}
\label{tab:receivedpower}
\end{center}
\end{table}

Judging from table \ref{tab:receivedpower}, the signal amplitude decreases as we move further away from the radar which is in line with our expectations.
The power is calculated using \\ $20 * $log(voltage).
In figure \ref{fig: Fitteddata} the experiment data is fitted with help of the provided 'radar\textunderscore 4 \textunderscore epo.m' file (see appendix).

\begin{figure}[H]
\begin{center}
\includegraphics[scale = 1]{data/Fitteddata.png}
\caption{Fitted data of the experiment.}
\label{fig: Fitteddata}
\end{center}
\end{figure}

The parameter $a$ is equal to approximately 3,2 as opposed to the theoretically \\predicted 4. The higher parameter $a$ is, the higher the decrease in signal amplitude for distances further away from the radar is. 
Because our parameter $a$ is 3,2 instead of the predicted 4 it seems likely that parts of the signal are reflected by objects other than the reference target. 
As a result of this the parameter $a$ will be lower than 4.

\end{document}