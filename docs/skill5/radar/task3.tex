%!TEX program = xelatex+makeindex+bibtex
\documentclass[final]{scrreprt} %scrreprt of scrartcl
\input{../../library/preamble.tex}
\input{../../library/style.tex}
\addbibresource{../../library/bibliography.bib}

\begin{document}

\chapter{Task 3}
\section{Experimental procedure}
In the first experiment we had to determine the range resolution of the radar demonstrator. 
We got two reference targets that we had to place in the radar's line of sight.
If one target is fixed we change the position of the second target until a minimum distance is reached for which the radar's indicator still identifies two targets.
This minimum distance is called $\Delta R$ and can be found using equation \ref{eq:1}. 
In this equation $c_{0}$ is the speed of light and $B$ is the bandwidth of the system.
In our experiment the bandwidth was between 0 and 26 $\giga Hz$. 
\begin{equation} 
\label{eq:1}
\Delta R= \frac{c_{0}}{2B} 
\end{equation}
The setup of our experiment can be seen in figure \ref{fig:range}
\begin{figure}[h]
	\begin{center}
		\includegraphics[width=\linewidth/2]{resources/meet3.png}
	\end{center}
	\caption{Range resolution experiment}
	\label{fig:range}
\end{figure}


\section{Results}
We did the experiment at several distances from the radar: 2.00 \unit{m}, 2.50 \unit{m}, 3.50 \unit{m} and 5.00 \unit{m}. 
With the results found for $\Delta R$ we can calculate the Bandwidth $B$ from equation \ref{eq:1}.
The results from the experiment are shown in Table \ref{tab:2}
\begin{table}[h]
\begin{center}
\begin{tabular}{ | l | l | l | }
    \hline
    Distance between radar and object R (m) & Minimum distance $\Delta R$ (cm) & Bandwidth ($\giga Hz$)  \\\hline
    2.00 & 0.50 & 30.0 \\\hline
    2.50 & 0.50 & 30.0 \\\hline
    3.50 & 1.00 & 20.0  \\\hline
    5.00 & 0.75 & 15.0  \\\hline
\end{tabular}
\caption{Measure data range resolution experiment.}
\label{tab:2}
\end{center}
\end{table}
\end{document}