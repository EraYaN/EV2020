%!TEX program = xelatex+makeindex+bibtex
\documentclass[final]{scrreprt} %scrreprt of scrartcl
\input{../../library/preamble.tex}
\input{../../library/style.tex}
\addbibresource{../../library/bibliography.bib}
\begin{document}

\chapter{Assignment 2: Basic radar concepts}
\label{ch:sk5-ass2}
\section{Task 1}
\label{sec:sk5-tsk1}
\begin{equation}
\omega_{1}=2f_1\pi
\end{equation}
\begin{equation}
\beta_{1}=\frac{\omega_{1}}{c_{med}}
\end{equation}
\begin{equation}
\Gamma=\frac{Z_L-Z_0}{Z_L+Z_0}
\end{equation}
\begin{equation}
{Z}_{in}(-L)=Z_0\frac{Z_L+iZ_0tan(\beta L)}{Z_0+iZ_Ltan(\beta L)}
\end{equation}
\begin{equation}
tan(\beta L) (Z_0 Z_L-Z_0^2) i=Z_0 Z_{in}-Z_L Z_{in}
\end{equation}
\begin{equation}
tan(\beta L)=i\frac{Z_0 Z_{in} - Z_L Z_{in}}{Z_0Z_L - {Z_0}^2}
\end{equation}
\begin{equation}
L=\frac{1}{\beta}\left[arctan\left(i\frac{Z_0 Z_{in} - Z_L Z_{in}}{Z_0 Z_L - {Z_0}^2}\right)+k\pi\right]
\end{equation}

When you only measure the impedance in one frequency you can not determine the length of the cable.
The unambiguous range of distances is $-\pi/\beta$ to $\pi/\beta$.

When you measure the impedance with two frequencies you can determine the length of the cable.
The lines (L-k) intersect in a point, this is the length of the cable.

\section{Task 2}
\label{sec:sk5-tsk2}


\end{document}
