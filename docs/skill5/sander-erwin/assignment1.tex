%!TEX program = xelatex+makeindex+bibtex
\documentclass[final]{scrreprt} %scrreprt of scrartcl
% Include all project wide packages here.
\usepackage{fullpage}
\usepackage{polyglossia}
\setmainlanguage{english}
\usepackage{csquotes}
\usepackage{graphicx}
\usepackage{epstopdf}
\usepackage{pdfpages}
\usepackage{caption}
\usepackage[list=true]{subcaption}
\usepackage{float}
\usepackage{standalone}
\usepackage{import}
\usepackage{tocloft}
\usepackage{wrapfig}
\usepackage{authblk}
\usepackage{array}
\usepackage{booktabs}
\usepackage[toc,page,title,titletoc]{appendix}
\usepackage{xunicode}
\usepackage{fontspec}
\usepackage{pgfplots}
\usepackage{SIunits}
\usepackage{units}
\pgfplotsset{compat=newest}
\pgfplotsset{plot coordinates/math parser=false}
\newlength\figureheight 
\newlength\figurewidth
\usepackage{amsmath}
\usepackage{mathtools}
\usepackage{unicode-math}
\usepackage[
    backend=bibtexu,
	texencoding=utf8,
bibencoding=utf8,
    style=ieee,
    sortlocale=en_US,
    language=auto
]{biblatex}
\usepackage{listings}
\newcommand{\includecode}[3][c]{\lstinputlisting[caption=#2, escapechar=, style=#1]{#3}}
\newcommand{\superscript}[1]{\ensuremath{^{\textrm{#1}}}}
\newcommand{\subscript}[1]{\ensuremath{_{\textrm{#1}}}}


\newcommand{\chapternumber}{\thechapter}
\renewcommand{\appendixname}{Bijlage}
\renewcommand{\appendixtocname}{Bijlagen}
\renewcommand{\appendixpagename}{Bijlagen}

\usepackage[hidelinks]{hyperref} %<--------ALTIJD ALS LAATSTE

\renewcommand{\familydefault}{\sfdefault}

\setmainfont[Ligatures=TeX]{Myriad Pro}
\setmathfont{Asana Math}
\setmonofont{Lucida Console}

\usepackage{titlesec, blindtext, color}
\definecolor{gray75}{gray}{0.75}
\newcommand{\hsp}{\hspace{20pt}}
\titleformat{\chapter}[hang]{\Huge\bfseries}{\chapternumber\hsp\textcolor{gray75}{|}\hsp}{0pt}{\Huge\bfseries}
\renewcommand{\familydefault}{\sfdefault}
\renewcommand{\arraystretch}{1.2}
\setlength\parindent{0pt}

%For code listings
\definecolor{black}{rgb}{0,0,0}
\definecolor{browntags}{rgb}{0.65,0.1,0.1}
\definecolor{bluestrings}{rgb}{0,0,1}
\definecolor{graycomments}{rgb}{0.4,0.4,0.4}
\definecolor{redkeywords}{rgb}{1,0,0}
\definecolor{bluekeywords}{rgb}{0.13,0.13,0.8}
\definecolor{greencomments}{rgb}{0,0.5,0}
\definecolor{redstrings}{rgb}{0.9,0,0}
\definecolor{purpleidentifiers}{rgb}{0.01,0,0.01}


\lstdefinestyle{csharp}{
language=[Sharp]C,
showspaces=false,
showtabs=false,
breaklines=true,
showstringspaces=false,
breakatwhitespace=true,
escapeinside={(*@}{@*)},
columns=fullflexible,
commentstyle=\color{greencomments},
keywordstyle=\color{bluekeywords}\bfseries,
stringstyle=\color{redstrings},
identifierstyle=\color{purpleidentifiers},
basicstyle=\ttfamily\small}

\lstdefinestyle{c}{
language=C,
showspaces=false,
showtabs=false,
breaklines=true,
showstringspaces=false,
breakatwhitespace=true,
escapeinside={(*@}{@*)},
columns=fullflexible,
commentstyle=\color{greencomments},
keywordstyle=\color{bluekeywords}\bfseries,
stringstyle=\color{redstrings},
identifierstyle=\color{purpleidentifiers},
}

\lstdefinestyle{matlab}{
language=Matlab,
showspaces=false,
showtabs=false,
breaklines=true,
showstringspaces=false,
breakatwhitespace=true,
escapeinside={(*@}{@*)},
columns=fullflexible,
commentstyle=\color{greencomments},
keywordstyle=\color{bluekeywords}\bfseries,
stringstyle=\color{redstrings},
identifierstyle=\color{purpleidentifiers}
}

\lstdefinestyle{vhdl}{
language=VHDL,
showspaces=false,
showtabs=false,
breaklines=true,
showstringspaces=false,
breakatwhitespace=true,
escapeinside={(*@}{@*)},
columns=fullflexible,
commentstyle=\color{greencomments},
keywordstyle=\color{bluekeywords}\bfseries,
stringstyle=\color{redstrings},
identifierstyle=\color{purpleidentifiers}
}

\lstdefinestyle{xaml}{
language=XML,
showspaces=false,
showtabs=false,
breaklines=true,
showstringspaces=false,
breakatwhitespace=true,
escapeinside={(*@}{@*)},
columns=fullflexible,
commentstyle=\color{greencomments},
keywordstyle=\color{redkeywords},
stringstyle=\color{bluestrings},
tagstyle=\color{browntags},
morestring=[b]",
  morecomment=[s]{<?}{?>},
  morekeywords={xmlns,version,typex:AsyncRecords,x:Arguments,x:Boolean,x:Byte,x:Char,x:Class,x:ClassAttributes,x:ClassModifier,x:Code,x:ConnectionId,x:Decimal,x:Double,x:FactoryMethod,x:FieldModifier,x:Int16,x:Int32,x:Int64,x:Key,x:Members,x:Name,x:Object,x:Property,x:Shared,x:Single,x:String,x:Subclass,x:SynchronousMode,x:TimeSpan,x:TypeArguments,x:Uid,x:Uri,x:XData,Grid.Column,Grid.ColumnSpan,Click,ClipToBounds,Content,DropDownOpened,FontSize,Foreground,Header,Height,HorizontalAlignment,HorizontalContentAlignment,IsCancel,IsDefault,IsEnabled,IsSelected,Margin,MinHeight,MinWidth,Padding,SnapsToDevicePixels,Target,TextWrapping,Title,VerticalAlignment,VerticalContentAlignment,Width,WindowStartupLocation,Binding,Mode,OneWay,xmlns:x}
}

%defaults
\lstset{
basicstyle=\ttfamily\small,
extendedchars=false,
numbers=left,
numberstyle=\ttfamily\tiny,
stepnumber=1,
tabsize=4,
numbersep=5pt
}
\addbibresource{../../library/bibliography.bib}
\usepackage[europeanresistors,americaninductors]{circuitikz}
\begin{document}

\chapter{Assignment 1: One dimensional wave propagation - the transmission channel model}
\label{ch:sk5-ass1}
\section{Task 1}

\begin{figure} [H]
\center
\begin{circuitikz}[american voltages]
\draw
	(0,0) to [short, o-] (3,0)
	to [L, l=$Z_p:  j \omega L$] (3,2)
	to [C, -o, l_=$Z_s: \frac{1}{j \omega C}$] (0,2)

	(3,0) to [short, *-] (4,0)
	to [R, l_=$Z_0$] (4,2)
	to [short, -*] (3,2);
\end{circuitikz}
\caption{Section of the transmission channel.}
\label{fig:circuit}
\end{figure}

The transmission channel used in this task has a series impedance $Z_s$ of $\frac{1}{j \omega C}$ and a parallel impedance $Z_p$ of $j \omega L$. A cuircuit that fits that parameters is displayed in Figure \ref{fig:circuit}. The equivalent impedance of the circuit in Figure \ref{fig:circuit} must be $Z_0$:

\begin{equation}
	Z_0 = Z_s + \frac{Z_p Z_0}{Z_p + Z_0}
\end{equation}

\begin{equation}
	Z_0 = \frac{1}{2j \omega C} + j \sqrt{\frac{1}{4 \omega^2 C^2} - \frac{L}{C}}
\end{equation}

The first term, $1 / (2j \omega C)$, is set to be in the power source at the start of the transmission channel and is thus not present when not looking at the ends of the transmission channel. Hence the first term is removed when not looking at the very start of the transmission line:

\begin{equation}
	Z_0 = j \sqrt{\frac{1}{4 \omega^2 C^2} - \frac{L}{C}}
\end{equation}

When $\omega^2 < \frac{1}{4CL}$, the characterisic impedance $Z_0$ is entirely real, otherwise it is entirely imaginary. When $Z_0$ is equal to zero, the cutoff frequency is reached, which is at $\omega = \sqrt{\frac{1}{4CL}}$.

The voltage drop across one section of the transmission line will be $I_n Z_s$ with $I_n$ the current through the transmission channel which equals $V_n / Z_0$ with $Z_n$ the voltage across all previous sections of the transmision channel section. According to this, the voltage before and after one section of the transmision line are proportional to each other anywhere on the transmission channel. This proportional factor is:

%\begin{equation}
%	\gamma = 1 - \frac{Z_s}{Z_0} = \frac{\sqrt{\frac{1}{4 \omega^2 C^2} - \frac{L}{C}} + \frac{1}{2 \omega C}}{\sqrt{\frac{1}{4 \omega^2 C^2} - \frac{L}{C}} - \frac{1}{2 \omega C}}
%\label{eq:equivalent_impedence}
%\end{equation}

\begin{equation}
	\gamma = 1 - \frac{Z_s}{Z_0} = \frac{\sqrt{\frac{1}{4} - LC \omega^2} - \frac{1}{2}}{\sqrt{\frac{1}{4} - LC \omega^2} + \frac{1}{2}}
\end{equation}

This network behaves like a high-pass filter since a capacitor is used as $Z_s$, which will not conduct at all when a DC signal is supplied.

\section{Task 2}
The reflection coefficient is given by:

\begin{equation}
	\Gamma = \frac{Z_l - Z_0}{Z_l + Z_0}
\label{eq:rc}
\end{equation}

Which is used in the calculation of the voltage on a certain location on the channel $z$ at time $t$:

\begin{equation}
	V(z,t) = U(t - \gamma z) + \Gamma U(t + \gamma(z - 2l)) - \Gamma U(t - \gamma(z - 2l)) - \Gamma^2 U(t + \gamma(z - 4l)) ...
\label{eq:v}
\end{equation}

When the channel terminal is an open circuit, $Z_l$ will be infinitily high. Using this in Eq. \ref{eq:rc} and using l'Hopital's rule:

\begin{equation}
	\Gamma = \lim_{Z_l \to \infty} \frac{Z_l - Z_0}{Z_l + Z_0} = \lim_{Z_l \to \infty} \frac{1}{1} = 1
\end{equation}

Using this $\Gamma$ in Eq. \ref{eq:v} of 1 at $z = l$ gives:

\begin{equation}
	V(z,t) = 2U(t - \gamma z) + 2U(t + \gamma(z - 3l)) - 2U(t - \gamma(z + 5l)) ...
\label{eq:v}
\end{equation}

When the terminal is short circuited, $Z_l$ will be zero. Using this in Eq. {eq:rc}:

\begin{equation}
	\Gamma = \frac{0 - Z_0}{0 + Z_0} = -1
\end{equation}

Using this $\Gamma$ of -1 in Eq. \ref{eq:v} at $z = l$ gives:

\begin{equation}
	V(l,t) = 0
\end{equation}

Which makes sense since no voltage could exist over a resistance of exactly zero Ohms.

\section{Task 3}
A lossless tranmission line with $L_0 = 0.5 \mu H$ in series, $C_0 = 2 pF$ in parallel and $Z_l = 100 \Omega$ as load has a characteristic impedance of:

\begin{equation}
	Z_0 = \sqrt{\frac{L_0}{C_0}} = 500 \Omega
\end{equation}

The propagation of a rectangular signal U(t) through this transmission line can be calculated by using the Matlab script task-3.m, which can be found in Appendix A. The Matlab plotting script Bscan\_plot.m was provided, which can be found in Appendix A as well.

The Matlab script task-3.m generates a V(x,t) and I(x,t) 3D plot with x $\epsilon$ [0,1.2] m and t $\epsilon$ [0,5] ns as shown below.

\iffalse
\begin{figure}[H]
	\centering
	\setlength\figureheight{0.8\linewidth}
    	\setlength\figurewidth{0.8\linewidth}
	% This file was created by matlab2tikz v0.4.6 running on MATLAB 8.2.
% Copyright (c) 2008--2014, Nico Schlömer <nico.schloemer@gmail.com>
% All rights reserved.
% Minimal pgfplots version: 1.3
% 
% The latest updates can be retrieved from
%   http://www.mathworks.com/matlabcentral/fileexchange/22022-matlab2tikz
% where you can also make suggestions and rate matlab2tikz.
% 
\begin{tikzpicture}

\begin{axis}[%
width=\figurewidth,
height=\figureheight,
colormap={mymap}{[1pt] rgb(0pt)=(0.063,0.545,0.851); rgb(100pt)=(0.6,0.6,0.6); rgb(200pt)=(0.969,0.035,0.243)},
view={20}{45},
scale only axis,
xmin=0,
xmax=5e-09,
xlabel={t (s)},
xmajorgrids,
ymin=0,
ymax=1.2,
ylabel={x (m)},
ymajorgrids,
zmin=-1.05,
zmax=1.05,
zlabel={U (V)},
zmajorgrids
]

\addplot3[area legend,patch,forget plot]
 table[row sep=crcr, point meta=\thisrow{c}] {
x	y	z	c\\
0	0	-0.666666666666667	nan	\\
0	0	0.5	0.166666666666667	\\
0	0	0.5	0.5	\\
5e-11	0	1	1	\\
1e-10	0	1	1	\\
1.5e-10	0	1	1	\\
2e-10	0	1	1	\\
2.5e-10	0	1	1	\\
3e-10	0	1	1	\\
3.5e-10	0	1	1	\\
4e-10	0	1	1	\\
4.5e-10	0	1	1	\\
5e-10	0	0.5	0.5	\\
5.5e-10	0	0	0	\\
6e-10	0	0	0	\\
6.5e-10	0	0	0	\\
7e-10	0	0	0	\\
7.5e-10	0	0	0	\\
8e-10	0	0	0	\\
8.5e-10	0	0	0	\\
9e-10	0	0	0	\\
9.5e-10	0	0	0	\\
1e-09	0	0	0	\\
1.05e-09	0	0	0	\\
1.1e-09	0	0	0	\\
1.15e-09	0	0	0	\\
1.2e-09	0	0	0	\\
1.25e-09	0	0	0	\\
1.3e-09	0	0	0	\\
1.35e-09	0	0	0	\\
1.4e-09	0	0	0	\\
1.45e-09	0	0	0	\\
1.5e-09	0	0	0	\\
1.55e-09	0	0	0	\\
1.6e-09	0	0	0	\\
1.65e-09	0	0	0	\\
1.7e-09	0	0	0	\\
1.75e-09	0	0	0	\\
1.8e-09	0	0	0	\\
1.85e-09	0	0	0	\\
1.9e-09	0	0	0	\\
1.95e-09	0	0	0	\\
2e-09	0	0	0	\\
2.05e-09	0	0	0	\\
2.1e-09	0	0	0	\\
2.15e-09	0	0	0	\\
2.2e-09	0	0	0	\\
2.25e-09	0	0	0	\\
2.3e-09	0	0	0	\\
2.35e-09	0	0	0	\\
2.4e-09	0	0	0	\\
2.45e-09	0	0	0	\\
2.5e-09	0	0	0	\\
2.55e-09	0	0	0	\\
2.6e-09	0	0	0	\\
2.65e-09	0	0	0	\\
2.7e-09	0	0	0	\\
2.75e-09	0	0	0	\\
2.8e-09	0	0	0	\\
2.85e-09	0	0	0	\\
2.9e-09	0	0	0	\\
2.95e-09	0	0	0	\\
3e-09	0	0	0	\\
3.05e-09	0	0	0	\\
3.1e-09	0	0	0	\\
3.15e-09	0	0	0	\\
3.2e-09	0	0	0	\\
3.25e-09	0	0	0	\\
3.3e-09	0	0	0	\\
3.35e-09	0	0	0	\\
3.4e-09	0	0	0	\\
3.45e-09	0	0	0	\\
3.5e-09	0	0	0	\\
3.55e-09	0	0	0	\\
3.6e-09	0	0	0	\\
3.65e-09	0	0	0	\\
3.7e-09	0	0	0	\\
3.75e-09	0	0	0	\\
3.8e-09	0	0	0	\\
3.85e-09	0	0	0	\\
3.9e-09	0	0	0	\\
3.95e-09	0	0	0	\\
4e-09	0	0	0	\\
4.05e-09	0	0	0	\\
4.1e-09	0	0	0	\\
4.15e-09	0	0	0	\\
4.2e-09	0	0	0	\\
4.25e-09	0	0	0	\\
4.3e-09	0	0	0	\\
4.35e-09	0	0	0	\\
4.4e-09	0	0	0	\\
4.45e-09	0	0	0	\\
4.5e-09	0	0	0	\\
4.55e-09	0	0	0	\\
4.6e-09	0	0	0	\\
4.65e-09	0	0	0	\\
4.7e-09	0	0	0	\\
4.75e-09	0	0	0	\\
4.8e-09	0	0	0	\\
4.85e-09	0	0	0	\\
4.9e-09	0	0	0	\\
4.95e-09	0	0	0	\\
5e-09	0	0	0	\\
5e-09	0	0	nan	\\
5e-09	0	-0.666666666666667	0.166666666666667	\\
5e-09	0	-0.666666666666667	nan	\\
0	0.012	-0.666666666666667	nan	\\
0	0.012	0	0.166666666666667	\\
0	0.012	0	0	\\
5e-11	0.012	1	1	\\
1e-10	0.012	1	1	\\
1.5e-10	0.012	1	1	\\
2e-10	0.012	1	1	\\
2.5e-10	0.012	1	1	\\
3e-10	0.012	1	1	\\
3.5e-10	0.012	1	1	\\
4e-10	0.012	1	1	\\
4.5e-10	0.012	1	1	\\
5e-10	0.012	1	1	\\
5.5e-10	0.012	0	0	\\
6e-10	0.012	0	0	\\
6.5e-10	0.012	0	0	\\
7e-10	0.012	0	0	\\
7.5e-10	0.012	0	0	\\
8e-10	0.012	0	0	\\
8.5e-10	0.012	0	0	\\
9e-10	0.012	0	0	\\
9.5e-10	0.012	0	0	\\
1e-09	0.012	0	0	\\
1.05e-09	0.012	0	0	\\
1.1e-09	0.012	0	0	\\
1.15e-09	0.012	0	0	\\
1.2e-09	0.012	0	0	\\
1.25e-09	0.012	0	0	\\
1.3e-09	0.012	0	0	\\
1.35e-09	0.012	0	0	\\
1.4e-09	0.012	0	0	\\
1.45e-09	0.012	0	0	\\
1.5e-09	0.012	0	0	\\
1.55e-09	0.012	0	0	\\
1.6e-09	0.012	0	0	\\
1.65e-09	0.012	0	0	\\
1.7e-09	0.012	0	0	\\
1.75e-09	0.012	0	0	\\
1.8e-09	0.012	0	0	\\
1.85e-09	0.012	0	0	\\
1.9e-09	0.012	0	0	\\
1.95e-09	0.012	0	0	\\
2e-09	0.012	0	0	\\
2.05e-09	0.012	0	0	\\
2.1e-09	0.012	0	0	\\
2.15e-09	0.012	0	0	\\
2.2e-09	0.012	0	0	\\
2.25e-09	0.012	0	0	\\
2.3e-09	0.012	0	0	\\
2.35e-09	0.012	0	0	\\
2.4e-09	0.012	-0.666666666666667	-0.666666666666667	\\
2.45e-09	0.012	0	0	\\
2.5e-09	0.012	0	0	\\
2.55e-09	0.012	0	0	\\
2.6e-09	0.012	0	0	\\
2.65e-09	0.012	0	0	\\
2.7e-09	0.012	0	0	\\
2.75e-09	0.012	0	0	\\
2.8e-09	0.012	0	0	\\
2.85e-09	0.012	0	0	\\
2.9e-09	0.012	0.666666666666667	0.666666666666667	\\
2.95e-09	0.012	0	0	\\
3e-09	0.012	0	0	\\
3.05e-09	0.012	0	0	\\
3.1e-09	0.012	0	0	\\
3.15e-09	0.012	0	0	\\
3.2e-09	0.012	0	0	\\
3.25e-09	0.012	0	0	\\
3.3e-09	0.012	0	0	\\
3.35e-09	0.012	0	0	\\
3.4e-09	0.012	0	0	\\
3.45e-09	0.012	0	0	\\
3.5e-09	0.012	0	0	\\
3.55e-09	0.012	0	0	\\
3.6e-09	0.012	0	0	\\
3.65e-09	0.012	0	0	\\
3.7e-09	0.012	0	0	\\
3.75e-09	0.012	0	0	\\
3.8e-09	0.012	0	0	\\
3.85e-09	0.012	0	0	\\
3.9e-09	0.012	0	0	\\
3.95e-09	0.012	0	0	\\
4e-09	0.012	0	0	\\
4.05e-09	0.012	0	0	\\
4.1e-09	0.012	0	0	\\
4.15e-09	0.012	0	0	\\
4.2e-09	0.012	0	0	\\
4.25e-09	0.012	0	0	\\
4.3e-09	0.012	0	0	\\
4.35e-09	0.012	0	0	\\
4.4e-09	0.012	0	0	\\
4.45e-09	0.012	0	0	\\
4.5e-09	0.012	0	0	\\
4.55e-09	0.012	0	0	\\
4.6e-09	0.012	0	0	\\
4.65e-09	0.012	0	0	\\
4.7e-09	0.012	0	0	\\
4.75e-09	0.012	0	0	\\
4.8e-09	0.012	-0.444444444444444	-0.444444444444444	\\
4.85e-09	0.012	0	0	\\
4.9e-09	0.012	0	0	\\
4.95e-09	0.012	0	0	\\
5e-09	0.012	0	0	\\
5e-09	0.012	0	nan	\\
5e-09	0.012	-0.666666666666667	0.166666666666667	\\
5e-09	0.012	-0.666666666666667	nan	\\
0	0.024	-0.666666666666667	nan	\\
0	0.024	0	0.166666666666667	\\
0	0.024	0	0	\\
5e-11	0.024	1	1	\\
1e-10	0.024	1	1	\\
1.5e-10	0.024	1	1	\\
2e-10	0.024	1	1	\\
2.5e-10	0.024	1	1	\\
3e-10	0.024	1	1	\\
3.5e-10	0.024	1	1	\\
4e-10	0.024	1	1	\\
4.5e-10	0.024	1	1	\\
5e-10	0.024	1	1	\\
5.5e-10	0.024	0	0	\\
6e-10	0.024	0	0	\\
6.5e-10	0.024	0	0	\\
7e-10	0.024	0	0	\\
7.5e-10	0.024	0	0	\\
8e-10	0.024	0	0	\\
8.5e-10	0.024	0	0	\\
9e-10	0.024	0	0	\\
9.5e-10	0.024	0	0	\\
1e-09	0.024	0	0	\\
1.05e-09	0.024	0	0	\\
1.1e-09	0.024	0	0	\\
1.15e-09	0.024	0	0	\\
1.2e-09	0.024	0	0	\\
1.25e-09	0.024	0	0	\\
1.3e-09	0.024	0	0	\\
1.35e-09	0.024	0	0	\\
1.4e-09	0.024	0	0	\\
1.45e-09	0.024	0	0	\\
1.5e-09	0.024	0	0	\\
1.55e-09	0.024	0	0	\\
1.6e-09	0.024	0	0	\\
1.65e-09	0.024	0	0	\\
1.7e-09	0.024	0	0	\\
1.75e-09	0.024	0	0	\\
1.8e-09	0.024	0	0	\\
1.85e-09	0.024	0	0	\\
1.9e-09	0.024	0	0	\\
1.95e-09	0.024	0	0	\\
2e-09	0.024	0	0	\\
2.05e-09	0.024	0	0	\\
2.1e-09	0.024	0	0	\\
2.15e-09	0.024	0	0	\\
2.2e-09	0.024	0	0	\\
2.25e-09	0.024	0	0	\\
2.3e-09	0.024	0	0	\\
2.35e-09	0.024	0	0	\\
2.4e-09	0.024	-0.666666666666667	-0.666666666666667	\\
2.45e-09	0.024	0	0	\\
2.5e-09	0.024	0	0	\\
2.55e-09	0.024	0	0	\\
2.6e-09	0.024	0	0	\\
2.65e-09	0.024	0	0	\\
2.7e-09	0.024	0	0	\\
2.75e-09	0.024	0	0	\\
2.8e-09	0.024	0	0	\\
2.85e-09	0.024	0	0	\\
2.9e-09	0.024	0.666666666666667	0.666666666666667	\\
2.95e-09	0.024	0	0	\\
3e-09	0.024	0	0	\\
3.05e-09	0.024	0	0	\\
3.1e-09	0.024	0	0	\\
3.15e-09	0.024	0	0	\\
3.2e-09	0.024	0	0	\\
3.25e-09	0.024	0	0	\\
3.3e-09	0.024	0	0	\\
3.35e-09	0.024	0	0	\\
3.4e-09	0.024	0	0	\\
3.45e-09	0.024	0	0	\\
3.5e-09	0.024	0	0	\\
3.55e-09	0.024	0	0	\\
3.6e-09	0.024	0	0	\\
3.65e-09	0.024	0	0	\\
3.7e-09	0.024	0	0	\\
3.75e-09	0.024	0	0	\\
3.8e-09	0.024	0	0	\\
3.85e-09	0.024	0	0	\\
3.9e-09	0.024	0	0	\\
3.95e-09	0.024	0	0	\\
4e-09	0.024	0	0	\\
4.05e-09	0.024	0	0	\\
4.1e-09	0.024	0	0	\\
4.15e-09	0.024	0	0	\\
4.2e-09	0.024	0	0	\\
4.25e-09	0.024	0	0	\\
4.3e-09	0.024	0	0	\\
4.35e-09	0.024	0	0	\\
4.4e-09	0.024	0	0	\\
4.45e-09	0.024	0	0	\\
4.5e-09	0.024	0	0	\\
4.55e-09	0.024	0	0	\\
4.6e-09	0.024	0	0	\\
4.65e-09	0.024	0	0	\\
4.7e-09	0.024	0	0	\\
4.75e-09	0.024	0	0	\\
4.8e-09	0.024	-0.444444444444444	-0.444444444444444	\\
4.85e-09	0.024	0	0	\\
4.9e-09	0.024	0	0	\\
4.95e-09	0.024	0	0	\\
5e-09	0.024	0	0	\\
5e-09	0.024	0	nan	\\
5e-09	0.024	-0.666666666666667	0.166666666666667	\\
5e-09	0.024	-0.666666666666667	nan	\\
0	0.036	-0.666666666666667	nan	\\
0	0.036	0	0.166666666666667	\\
0	0.036	0	0	\\
5e-11	0.036	1	1	\\
1e-10	0.036	1	1	\\
1.5e-10	0.036	1	1	\\
2e-10	0.036	1	1	\\
2.5e-10	0.036	1	1	\\
3e-10	0.036	1	1	\\
3.5e-10	0.036	1	1	\\
4e-10	0.036	1	1	\\
4.5e-10	0.036	1	1	\\
5e-10	0.036	1	1	\\
5.5e-10	0.036	0	0	\\
6e-10	0.036	0	0	\\
6.5e-10	0.036	0	0	\\
7e-10	0.036	0	0	\\
7.5e-10	0.036	0	0	\\
8e-10	0.036	0	0	\\
8.5e-10	0.036	0	0	\\
9e-10	0.036	0	0	\\
9.5e-10	0.036	0	0	\\
1e-09	0.036	0	0	\\
1.05e-09	0.036	0	0	\\
1.1e-09	0.036	0	0	\\
1.15e-09	0.036	0	0	\\
1.2e-09	0.036	0	0	\\
1.25e-09	0.036	0	0	\\
1.3e-09	0.036	0	0	\\
1.35e-09	0.036	0	0	\\
1.4e-09	0.036	0	0	\\
1.45e-09	0.036	0	0	\\
1.5e-09	0.036	0	0	\\
1.55e-09	0.036	0	0	\\
1.6e-09	0.036	0	0	\\
1.65e-09	0.036	0	0	\\
1.7e-09	0.036	0	0	\\
1.75e-09	0.036	0	0	\\
1.8e-09	0.036	0	0	\\
1.85e-09	0.036	0	0	\\
1.9e-09	0.036	0	0	\\
1.95e-09	0.036	0	0	\\
2e-09	0.036	0	0	\\
2.05e-09	0.036	0	0	\\
2.1e-09	0.036	0	0	\\
2.15e-09	0.036	0	0	\\
2.2e-09	0.036	0	0	\\
2.25e-09	0.036	0	0	\\
2.3e-09	0.036	0	0	\\
2.35e-09	0.036	0	0	\\
2.4e-09	0.036	-0.666666666666667	-0.666666666666667	\\
2.45e-09	0.036	0	0	\\
2.5e-09	0.036	0	0	\\
2.55e-09	0.036	0	0	\\
2.6e-09	0.036	0	0	\\
2.65e-09	0.036	0	0	\\
2.7e-09	0.036	0	0	\\
2.75e-09	0.036	0	0	\\
2.8e-09	0.036	0	0	\\
2.85e-09	0.036	0	0	\\
2.9e-09	0.036	0.666666666666667	0.666666666666667	\\
2.95e-09	0.036	0	0	\\
3e-09	0.036	0	0	\\
3.05e-09	0.036	0	0	\\
3.1e-09	0.036	0	0	\\
3.15e-09	0.036	0	0	\\
3.2e-09	0.036	0	0	\\
3.25e-09	0.036	0	0	\\
3.3e-09	0.036	0	0	\\
3.35e-09	0.036	0	0	\\
3.4e-09	0.036	0	0	\\
3.45e-09	0.036	0	0	\\
3.5e-09	0.036	0	0	\\
3.55e-09	0.036	0	0	\\
3.6e-09	0.036	0	0	\\
3.65e-09	0.036	0	0	\\
3.7e-09	0.036	0	0	\\
3.75e-09	0.036	0	0	\\
3.8e-09	0.036	0	0	\\
3.85e-09	0.036	0	0	\\
3.9e-09	0.036	0	0	\\
3.95e-09	0.036	0	0	\\
4e-09	0.036	0	0	\\
4.05e-09	0.036	0	0	\\
4.1e-09	0.036	0	0	\\
4.15e-09	0.036	0	0	\\
4.2e-09	0.036	0	0	\\
4.25e-09	0.036	0	0	\\
4.3e-09	0.036	0	0	\\
4.35e-09	0.036	0	0	\\
4.4e-09	0.036	0	0	\\
4.45e-09	0.036	0	0	\\
4.5e-09	0.036	0	0	\\
4.55e-09	0.036	0	0	\\
4.6e-09	0.036	0	0	\\
4.65e-09	0.036	0	0	\\
4.7e-09	0.036	0	0	\\
4.75e-09	0.036	0	0	\\
4.8e-09	0.036	-0.444444444444444	-0.444444444444444	\\
4.85e-09	0.036	0	0	\\
4.9e-09	0.036	0	0	\\
4.95e-09	0.036	0	0	\\
5e-09	0.036	0	0	\\
5e-09	0.036	0	nan	\\
5e-09	0.036	-0.666666666666667	0.166666666666667	\\
5e-09	0.036	-0.666666666666667	nan	\\
0	0.048	-0.666666666666667	nan	\\
0	0.048	0	0.166666666666667	\\
0	0.048	0	0	\\
5e-11	0.048	1	1	\\
1e-10	0.048	1	1	\\
1.5e-10	0.048	1	1	\\
2e-10	0.048	1	1	\\
2.5e-10	0.048	1	1	\\
3e-10	0.048	1	1	\\
3.5e-10	0.048	1	1	\\
4e-10	0.048	1	1	\\
4.5e-10	0.048	1	1	\\
5e-10	0.048	1	1	\\
5.5e-10	0.048	0	0	\\
6e-10	0.048	0	0	\\
6.5e-10	0.048	0	0	\\
7e-10	0.048	0	0	\\
7.5e-10	0.048	0	0	\\
8e-10	0.048	0	0	\\
8.5e-10	0.048	0	0	\\
9e-10	0.048	0	0	\\
9.5e-10	0.048	0	0	\\
1e-09	0.048	0	0	\\
1.05e-09	0.048	0	0	\\
1.1e-09	0.048	0	0	\\
1.15e-09	0.048	0	0	\\
1.2e-09	0.048	0	0	\\
1.25e-09	0.048	0	0	\\
1.3e-09	0.048	0	0	\\
1.35e-09	0.048	0	0	\\
1.4e-09	0.048	0	0	\\
1.45e-09	0.048	0	0	\\
1.5e-09	0.048	0	0	\\
1.55e-09	0.048	0	0	\\
1.6e-09	0.048	0	0	\\
1.65e-09	0.048	0	0	\\
1.7e-09	0.048	0	0	\\
1.75e-09	0.048	0	0	\\
1.8e-09	0.048	0	0	\\
1.85e-09	0.048	0	0	\\
1.9e-09	0.048	0	0	\\
1.95e-09	0.048	0	0	\\
2e-09	0.048	0	0	\\
2.05e-09	0.048	0	0	\\
2.1e-09	0.048	0	0	\\
2.15e-09	0.048	0	0	\\
2.2e-09	0.048	0	0	\\
2.25e-09	0.048	0	0	\\
2.3e-09	0.048	0	0	\\
2.35e-09	0.048	0	0	\\
2.4e-09	0.048	-0.666666666666667	-0.666666666666667	\\
2.45e-09	0.048	0	0	\\
2.5e-09	0.048	0	0	\\
2.55e-09	0.048	0	0	\\
2.6e-09	0.048	0	0	\\
2.65e-09	0.048	0	0	\\
2.7e-09	0.048	0	0	\\
2.75e-09	0.048	0	0	\\
2.8e-09	0.048	0	0	\\
2.85e-09	0.048	0	0	\\
2.9e-09	0.048	0.666666666666667	0.666666666666667	\\
2.95e-09	0.048	0	0	\\
3e-09	0.048	0	0	\\
3.05e-09	0.048	0	0	\\
3.1e-09	0.048	0	0	\\
3.15e-09	0.048	0	0	\\
3.2e-09	0.048	0	0	\\
3.25e-09	0.048	0	0	\\
3.3e-09	0.048	0	0	\\
3.35e-09	0.048	0	0	\\
3.4e-09	0.048	0	0	\\
3.45e-09	0.048	0	0	\\
3.5e-09	0.048	0	0	\\
3.55e-09	0.048	0	0	\\
3.6e-09	0.048	0	0	\\
3.65e-09	0.048	0	0	\\
3.7e-09	0.048	0	0	\\
3.75e-09	0.048	0	0	\\
3.8e-09	0.048	0	0	\\
3.85e-09	0.048	0	0	\\
3.9e-09	0.048	0	0	\\
3.95e-09	0.048	0	0	\\
4e-09	0.048	0	0	\\
4.05e-09	0.048	0	0	\\
4.1e-09	0.048	0	0	\\
4.15e-09	0.048	0	0	\\
4.2e-09	0.048	0	0	\\
4.25e-09	0.048	0	0	\\
4.3e-09	0.048	0	0	\\
4.35e-09	0.048	0	0	\\
4.4e-09	0.048	0	0	\\
4.45e-09	0.048	0	0	\\
4.5e-09	0.048	0	0	\\
4.55e-09	0.048	0	0	\\
4.6e-09	0.048	0	0	\\
4.65e-09	0.048	0	0	\\
4.7e-09	0.048	0	0	\\
4.75e-09	0.048	0	0	\\
4.8e-09	0.048	-0.444444444444444	-0.444444444444444	\\
4.85e-09	0.048	0	0	\\
4.9e-09	0.048	0	0	\\
4.95e-09	0.048	0	0	\\
5e-09	0.048	0	0	\\
5e-09	0.048	0	nan	\\
5e-09	0.048	-0.666666666666667	0.166666666666667	\\
5e-09	0.048	-0.666666666666667	nan	\\
0	0.06	-0.666666666666667	nan	\\
0	0.06	0	0.166666666666667	\\
0	0.06	0	0	\\
5e-11	0.06	0	0	\\
1e-10	0.06	1	1	\\
1.5e-10	0.06	1	1	\\
2e-10	0.06	1	1	\\
2.5e-10	0.06	1	1	\\
3e-10	0.06	1	1	\\
3.5e-10	0.06	1	1	\\
4e-10	0.06	1	1	\\
4.5e-10	0.06	1	1	\\
5e-10	0.06	1	1	\\
5.5e-10	0.06	1	1	\\
6e-10	0.06	0	0	\\
6.5e-10	0.06	0	0	\\
7e-10	0.06	0	0	\\
7.5e-10	0.06	0	0	\\
8e-10	0.06	0	0	\\
8.5e-10	0.06	0	0	\\
9e-10	0.06	0	0	\\
9.5e-10	0.06	0	0	\\
1e-09	0.06	0	0	\\
1.05e-09	0.06	0	0	\\
1.1e-09	0.06	0	0	\\
1.15e-09	0.06	0	0	\\
1.2e-09	0.06	0	0	\\
1.25e-09	0.06	0	0	\\
1.3e-09	0.06	0	0	\\
1.35e-09	0.06	0	0	\\
1.4e-09	0.06	0	0	\\
1.45e-09	0.06	0	0	\\
1.5e-09	0.06	0	0	\\
1.55e-09	0.06	0	0	\\
1.6e-09	0.06	0	0	\\
1.65e-09	0.06	0	0	\\
1.7e-09	0.06	0	0	\\
1.75e-09	0.06	0	0	\\
1.8e-09	0.06	0	0	\\
1.85e-09	0.06	0	0	\\
1.9e-09	0.06	0	0	\\
1.95e-09	0.06	0	0	\\
2e-09	0.06	0	0	\\
2.05e-09	0.06	0	0	\\
2.1e-09	0.06	0	0	\\
2.15e-09	0.06	0	0	\\
2.2e-09	0.06	0	0	\\
2.25e-09	0.06	0	0	\\
2.3e-09	0.06	0	0	\\
2.35e-09	0.06	-0.666666666666667	-0.666666666666667	\\
2.4e-09	0.06	-0.666666666666667	-0.666666666666667	\\
2.45e-09	0.06	-0.666666666666667	-0.666666666666667	\\
2.5e-09	0.06	0	0	\\
2.55e-09	0.06	0	0	\\
2.6e-09	0.06	0	0	\\
2.65e-09	0.06	0	0	\\
2.7e-09	0.06	0	0	\\
2.75e-09	0.06	0	0	\\
2.8e-09	0.06	0	0	\\
2.85e-09	0.06	0.666666666666667	0.666666666666667	\\
2.9e-09	0.06	0.666666666666667	0.666666666666667	\\
2.95e-09	0.06	0.666666666666667	0.666666666666667	\\
3e-09	0.06	0	0	\\
3.05e-09	0.06	0	0	\\
3.1e-09	0.06	0	0	\\
3.15e-09	0.06	0	0	\\
3.2e-09	0.06	0	0	\\
3.25e-09	0.06	0	0	\\
3.3e-09	0.06	0	0	\\
3.35e-09	0.06	0	0	\\
3.4e-09	0.06	0	0	\\
3.45e-09	0.06	0	0	\\
3.5e-09	0.06	0	0	\\
3.55e-09	0.06	0	0	\\
3.6e-09	0.06	0	0	\\
3.65e-09	0.06	0	0	\\
3.7e-09	0.06	0	0	\\
3.75e-09	0.06	0	0	\\
3.8e-09	0.06	0	0	\\
3.85e-09	0.06	0	0	\\
3.9e-09	0.06	0	0	\\
3.95e-09	0.06	0	0	\\
4e-09	0.06	0	0	\\
4.05e-09	0.06	0	0	\\
4.1e-09	0.06	0	0	\\
4.15e-09	0.06	0	0	\\
4.2e-09	0.06	0	0	\\
4.25e-09	0.06	0	0	\\
4.3e-09	0.06	0	0	\\
4.35e-09	0.06	0	0	\\
4.4e-09	0.06	0	0	\\
4.45e-09	0.06	0	0	\\
4.5e-09	0.06	0	0	\\
4.55e-09	0.06	0	0	\\
4.6e-09	0.06	0	0	\\
4.65e-09	0.06	0	0	\\
4.7e-09	0.06	0	0	\\
4.75e-09	0.06	-0.444444444444444	-0.444444444444444	\\
4.8e-09	0.06	-0.444444444444444	-0.444444444444444	\\
4.85e-09	0.06	-0.444444444444444	-0.444444444444444	\\
4.9e-09	0.06	0	0	\\
4.95e-09	0.06	0	0	\\
5e-09	0.06	0	0	\\
5e-09	0.06	0	nan	\\
5e-09	0.06	-0.666666666666667	0.166666666666667	\\
5e-09	0.06	-0.666666666666667	nan	\\
0	0.072	-0.666666666666667	nan	\\
0	0.072	0	0.166666666666667	\\
0	0.072	0	0	\\
5e-11	0.072	0	0	\\
1e-10	0.072	1	1	\\
1.5e-10	0.072	1	1	\\
2e-10	0.072	1	1	\\
2.5e-10	0.072	1	1	\\
3e-10	0.072	1	1	\\
3.5e-10	0.072	1	1	\\
4e-10	0.072	1	1	\\
4.5e-10	0.072	1	1	\\
5e-10	0.072	1	1	\\
5.5e-10	0.072	1	1	\\
6e-10	0.072	0	0	\\
6.5e-10	0.072	0	0	\\
7e-10	0.072	0	0	\\
7.5e-10	0.072	0	0	\\
8e-10	0.072	0	0	\\
8.5e-10	0.072	0	0	\\
9e-10	0.072	0	0	\\
9.5e-10	0.072	0	0	\\
1e-09	0.072	0	0	\\
1.05e-09	0.072	0	0	\\
1.1e-09	0.072	0	0	\\
1.15e-09	0.072	0	0	\\
1.2e-09	0.072	0	0	\\
1.25e-09	0.072	0	0	\\
1.3e-09	0.072	0	0	\\
1.35e-09	0.072	0	0	\\
1.4e-09	0.072	0	0	\\
1.45e-09	0.072	0	0	\\
1.5e-09	0.072	0	0	\\
1.55e-09	0.072	0	0	\\
1.6e-09	0.072	0	0	\\
1.65e-09	0.072	0	0	\\
1.7e-09	0.072	0	0	\\
1.75e-09	0.072	0	0	\\
1.8e-09	0.072	0	0	\\
1.85e-09	0.072	0	0	\\
1.9e-09	0.072	0	0	\\
1.95e-09	0.072	0	0	\\
2e-09	0.072	0	0	\\
2.05e-09	0.072	0	0	\\
2.1e-09	0.072	0	0	\\
2.15e-09	0.072	0	0	\\
2.2e-09	0.072	0	0	\\
2.25e-09	0.072	0	0	\\
2.3e-09	0.072	0	0	\\
2.35e-09	0.072	-0.666666666666667	-0.666666666666667	\\
2.4e-09	0.072	-0.666666666666667	-0.666666666666667	\\
2.45e-09	0.072	-0.666666666666667	-0.666666666666667	\\
2.5e-09	0.072	0	0	\\
2.55e-09	0.072	0	0	\\
2.6e-09	0.072	0	0	\\
2.65e-09	0.072	0	0	\\
2.7e-09	0.072	0	0	\\
2.75e-09	0.072	0	0	\\
2.8e-09	0.072	0	0	\\
2.85e-09	0.072	0.666666666666667	0.666666666666667	\\
2.9e-09	0.072	0.666666666666667	0.666666666666667	\\
2.95e-09	0.072	0.666666666666667	0.666666666666667	\\
3e-09	0.072	0	0	\\
3.05e-09	0.072	0	0	\\
3.1e-09	0.072	0	0	\\
3.15e-09	0.072	0	0	\\
3.2e-09	0.072	0	0	\\
3.25e-09	0.072	0	0	\\
3.3e-09	0.072	0	0	\\
3.35e-09	0.072	0	0	\\
3.4e-09	0.072	0	0	\\
3.45e-09	0.072	0	0	\\
3.5e-09	0.072	0	0	\\
3.55e-09	0.072	0	0	\\
3.6e-09	0.072	0	0	\\
3.65e-09	0.072	0	0	\\
3.7e-09	0.072	0	0	\\
3.75e-09	0.072	0	0	\\
3.8e-09	0.072	0	0	\\
3.85e-09	0.072	0	0	\\
3.9e-09	0.072	0	0	\\
3.95e-09	0.072	0	0	\\
4e-09	0.072	0	0	\\
4.05e-09	0.072	0	0	\\
4.1e-09	0.072	0	0	\\
4.15e-09	0.072	0	0	\\
4.2e-09	0.072	0	0	\\
4.25e-09	0.072	0	0	\\
4.3e-09	0.072	0	0	\\
4.35e-09	0.072	0	0	\\
4.4e-09	0.072	0	0	\\
4.45e-09	0.072	0	0	\\
4.5e-09	0.072	0	0	\\
4.55e-09	0.072	0	0	\\
4.6e-09	0.072	0	0	\\
4.65e-09	0.072	0	0	\\
4.7e-09	0.072	0	0	\\
4.75e-09	0.072	-0.444444444444444	-0.444444444444444	\\
4.8e-09	0.072	-0.444444444444444	-0.444444444444444	\\
4.85e-09	0.072	-0.444444444444444	-0.444444444444444	\\
4.9e-09	0.072	0	0	\\
4.95e-09	0.072	0	0	\\
5e-09	0.072	0	0	\\
5e-09	0.072	0	nan	\\
5e-09	0.072	-0.666666666666667	0.166666666666667	\\
5e-09	0.072	-0.666666666666667	nan	\\
0	0.084	-0.666666666666667	nan	\\
0	0.084	0	0.166666666666667	\\
0	0.084	0	0	\\
5e-11	0.084	0	0	\\
1e-10	0.084	1	1	\\
1.5e-10	0.084	1	1	\\
2e-10	0.084	1	1	\\
2.5e-10	0.084	1	1	\\
3e-10	0.084	1	1	\\
3.5e-10	0.084	1	1	\\
4e-10	0.084	1	1	\\
4.5e-10	0.084	1	1	\\
5e-10	0.084	1	1	\\
5.5e-10	0.084	1	1	\\
6e-10	0.084	0	0	\\
6.5e-10	0.084	0	0	\\
7e-10	0.084	0	0	\\
7.5e-10	0.084	0	0	\\
8e-10	0.084	0	0	\\
8.5e-10	0.084	0	0	\\
9e-10	0.084	0	0	\\
9.5e-10	0.084	0	0	\\
1e-09	0.084	0	0	\\
1.05e-09	0.084	0	0	\\
1.1e-09	0.084	0	0	\\
1.15e-09	0.084	0	0	\\
1.2e-09	0.084	0	0	\\
1.25e-09	0.084	0	0	\\
1.3e-09	0.084	0	0	\\
1.35e-09	0.084	0	0	\\
1.4e-09	0.084	0	0	\\
1.45e-09	0.084	0	0	\\
1.5e-09	0.084	0	0	\\
1.55e-09	0.084	0	0	\\
1.6e-09	0.084	0	0	\\
1.65e-09	0.084	0	0	\\
1.7e-09	0.084	0	0	\\
1.75e-09	0.084	0	0	\\
1.8e-09	0.084	0	0	\\
1.85e-09	0.084	0	0	\\
1.9e-09	0.084	0	0	\\
1.95e-09	0.084	0	0	\\
2e-09	0.084	0	0	\\
2.05e-09	0.084	0	0	\\
2.1e-09	0.084	0	0	\\
2.15e-09	0.084	0	0	\\
2.2e-09	0.084	0	0	\\
2.25e-09	0.084	0	0	\\
2.3e-09	0.084	0	0	\\
2.35e-09	0.084	-0.666666666666667	-0.666666666666667	\\
2.4e-09	0.084	-0.666666666666667	-0.666666666666667	\\
2.45e-09	0.084	-0.666666666666667	-0.666666666666667	\\
2.5e-09	0.084	0	0	\\
2.55e-09	0.084	0	0	\\
2.6e-09	0.084	0	0	\\
2.65e-09	0.084	0	0	\\
2.7e-09	0.084	0	0	\\
2.75e-09	0.084	0	0	\\
2.8e-09	0.084	0	0	\\
2.85e-09	0.084	0.666666666666667	0.666666666666667	\\
2.9e-09	0.084	0.666666666666667	0.666666666666667	\\
2.95e-09	0.084	0.666666666666667	0.666666666666667	\\
3e-09	0.084	0	0	\\
3.05e-09	0.084	0	0	\\
3.1e-09	0.084	0	0	\\
3.15e-09	0.084	0	0	\\
3.2e-09	0.084	0	0	\\
3.25e-09	0.084	0	0	\\
3.3e-09	0.084	0	0	\\
3.35e-09	0.084	0	0	\\
3.4e-09	0.084	0	0	\\
3.45e-09	0.084	0	0	\\
3.5e-09	0.084	0	0	\\
3.55e-09	0.084	0	0	\\
3.6e-09	0.084	0	0	\\
3.65e-09	0.084	0	0	\\
3.7e-09	0.084	0	0	\\
3.75e-09	0.084	0	0	\\
3.8e-09	0.084	0	0	\\
3.85e-09	0.084	0	0	\\
3.9e-09	0.084	0	0	\\
3.95e-09	0.084	0	0	\\
4e-09	0.084	0	0	\\
4.05e-09	0.084	0	0	\\
4.1e-09	0.084	0	0	\\
4.15e-09	0.084	0	0	\\
4.2e-09	0.084	0	0	\\
4.25e-09	0.084	0	0	\\
4.3e-09	0.084	0	0	\\
4.35e-09	0.084	0	0	\\
4.4e-09	0.084	0	0	\\
4.45e-09	0.084	0	0	\\
4.5e-09	0.084	0	0	\\
4.55e-09	0.084	0	0	\\
4.6e-09	0.084	0	0	\\
4.65e-09	0.084	0	0	\\
4.7e-09	0.084	0	0	\\
4.75e-09	0.084	-0.444444444444444	-0.444444444444444	\\
4.8e-09	0.084	-0.444444444444444	-0.444444444444444	\\
4.85e-09	0.084	-0.444444444444444	-0.444444444444444	\\
4.9e-09	0.084	0	0	\\
4.95e-09	0.084	0	0	\\
5e-09	0.084	0	0	\\
5e-09	0.084	0	nan	\\
5e-09	0.084	-0.666666666666667	0.166666666666667	\\
5e-09	0.084	-0.666666666666667	nan	\\
0	0.096	-0.666666666666667	nan	\\
0	0.096	0	0.166666666666667	\\
0	0.096	0	0	\\
5e-11	0.096	0	0	\\
1e-10	0.096	1	1	\\
1.5e-10	0.096	1	1	\\
2e-10	0.096	1	1	\\
2.5e-10	0.096	1	1	\\
3e-10	0.096	1	1	\\
3.5e-10	0.096	1	1	\\
4e-10	0.096	1	1	\\
4.5e-10	0.096	1	1	\\
5e-10	0.096	1	1	\\
5.5e-10	0.096	1	1	\\
6e-10	0.096	0	0	\\
6.5e-10	0.096	0	0	\\
7e-10	0.096	0	0	\\
7.5e-10	0.096	0	0	\\
8e-10	0.096	0	0	\\
8.5e-10	0.096	0	0	\\
9e-10	0.096	0	0	\\
9.5e-10	0.096	0	0	\\
1e-09	0.096	0	0	\\
1.05e-09	0.096	0	0	\\
1.1e-09	0.096	0	0	\\
1.15e-09	0.096	0	0	\\
1.2e-09	0.096	0	0	\\
1.25e-09	0.096	0	0	\\
1.3e-09	0.096	0	0	\\
1.35e-09	0.096	0	0	\\
1.4e-09	0.096	0	0	\\
1.45e-09	0.096	0	0	\\
1.5e-09	0.096	0	0	\\
1.55e-09	0.096	0	0	\\
1.6e-09	0.096	0	0	\\
1.65e-09	0.096	0	0	\\
1.7e-09	0.096	0	0	\\
1.75e-09	0.096	0	0	\\
1.8e-09	0.096	0	0	\\
1.85e-09	0.096	0	0	\\
1.9e-09	0.096	0	0	\\
1.95e-09	0.096	0	0	\\
2e-09	0.096	0	0	\\
2.05e-09	0.096	0	0	\\
2.1e-09	0.096	0	0	\\
2.15e-09	0.096	0	0	\\
2.2e-09	0.096	0	0	\\
2.25e-09	0.096	0	0	\\
2.3e-09	0.096	0	0	\\
2.35e-09	0.096	-0.666666666666667	-0.666666666666667	\\
2.4e-09	0.096	-0.666666666666667	-0.666666666666667	\\
2.45e-09	0.096	-0.666666666666667	-0.666666666666667	\\
2.5e-09	0.096	0	0	\\
2.55e-09	0.096	0	0	\\
2.6e-09	0.096	0	0	\\
2.65e-09	0.096	0	0	\\
2.7e-09	0.096	0	0	\\
2.75e-09	0.096	0	0	\\
2.8e-09	0.096	0	0	\\
2.85e-09	0.096	0.666666666666667	0.666666666666667	\\
2.9e-09	0.096	0.666666666666667	0.666666666666667	\\
2.95e-09	0.096	0.666666666666667	0.666666666666667	\\
3e-09	0.096	0	0	\\
3.05e-09	0.096	0	0	\\
3.1e-09	0.096	0	0	\\
3.15e-09	0.096	0	0	\\
3.2e-09	0.096	0	0	\\
3.25e-09	0.096	0	0	\\
3.3e-09	0.096	0	0	\\
3.35e-09	0.096	0	0	\\
3.4e-09	0.096	0	0	\\
3.45e-09	0.096	0	0	\\
3.5e-09	0.096	0	0	\\
3.55e-09	0.096	0	0	\\
3.6e-09	0.096	0	0	\\
3.65e-09	0.096	0	0	\\
3.7e-09	0.096	0	0	\\
3.75e-09	0.096	0	0	\\
3.8e-09	0.096	0	0	\\
3.85e-09	0.096	0	0	\\
3.9e-09	0.096	0	0	\\
3.95e-09	0.096	0	0	\\
4e-09	0.096	0	0	\\
4.05e-09	0.096	0	0	\\
4.1e-09	0.096	0	0	\\
4.15e-09	0.096	0	0	\\
4.2e-09	0.096	0	0	\\
4.25e-09	0.096	0	0	\\
4.3e-09	0.096	0	0	\\
4.35e-09	0.096	0	0	\\
4.4e-09	0.096	0	0	\\
4.45e-09	0.096	0	0	\\
4.5e-09	0.096	0	0	\\
4.55e-09	0.096	0	0	\\
4.6e-09	0.096	0	0	\\
4.65e-09	0.096	0	0	\\
4.7e-09	0.096	0	0	\\
4.75e-09	0.096	-0.444444444444444	-0.444444444444444	\\
4.8e-09	0.096	-0.444444444444444	-0.444444444444444	\\
4.85e-09	0.096	-0.444444444444444	-0.444444444444444	\\
4.9e-09	0.096	0	0	\\
4.95e-09	0.096	0	0	\\
5e-09	0.096	0	0	\\
5e-09	0.096	0	nan	\\
5e-09	0.096	-0.666666666666667	0.166666666666667	\\
5e-09	0.096	-0.666666666666667	nan	\\
0	0.108	-0.666666666666667	nan	\\
0	0.108	0	0.166666666666667	\\
0	0.108	0	0	\\
5e-11	0.108	0	0	\\
1e-10	0.108	0	0	\\
1.5e-10	0.108	1	1	\\
2e-10	0.108	1	1	\\
2.5e-10	0.108	1	1	\\
3e-10	0.108	1	1	\\
3.5e-10	0.108	1	1	\\
4e-10	0.108	1	1	\\
4.5e-10	0.108	1	1	\\
5e-10	0.108	1	1	\\
5.5e-10	0.108	1	1	\\
6e-10	0.108	1	1	\\
6.5e-10	0.108	0	0	\\
7e-10	0.108	0	0	\\
7.5e-10	0.108	0	0	\\
8e-10	0.108	0	0	\\
8.5e-10	0.108	0	0	\\
9e-10	0.108	0	0	\\
9.5e-10	0.108	0	0	\\
1e-09	0.108	0	0	\\
1.05e-09	0.108	0	0	\\
1.1e-09	0.108	0	0	\\
1.15e-09	0.108	0	0	\\
1.2e-09	0.108	0	0	\\
1.25e-09	0.108	0	0	\\
1.3e-09	0.108	0	0	\\
1.35e-09	0.108	0	0	\\
1.4e-09	0.108	0	0	\\
1.45e-09	0.108	0	0	\\
1.5e-09	0.108	0	0	\\
1.55e-09	0.108	0	0	\\
1.6e-09	0.108	0	0	\\
1.65e-09	0.108	0	0	\\
1.7e-09	0.108	0	0	\\
1.75e-09	0.108	0	0	\\
1.8e-09	0.108	0	0	\\
1.85e-09	0.108	0	0	\\
1.9e-09	0.108	0	0	\\
1.95e-09	0.108	0	0	\\
2e-09	0.108	0	0	\\
2.05e-09	0.108	0	0	\\
2.1e-09	0.108	0	0	\\
2.15e-09	0.108	0	0	\\
2.2e-09	0.108	0	0	\\
2.25e-09	0.108	0	0	\\
2.3e-09	0.108	-0.666666666666667	-0.666666666666667	\\
2.35e-09	0.108	-0.666666666666667	-0.666666666666667	\\
2.4e-09	0.108	-0.666666666666667	-0.666666666666667	\\
2.45e-09	0.108	-0.666666666666667	-0.666666666666667	\\
2.5e-09	0.108	-0.666666666666667	-0.666666666666667	\\
2.55e-09	0.108	0	0	\\
2.6e-09	0.108	0	0	\\
2.65e-09	0.108	0	0	\\
2.7e-09	0.108	0	0	\\
2.75e-09	0.108	0	0	\\
2.8e-09	0.108	0.666666666666667	0.666666666666667	\\
2.85e-09	0.108	0.666666666666667	0.666666666666667	\\
2.9e-09	0.108	0.666666666666667	0.666666666666667	\\
2.95e-09	0.108	0.666666666666667	0.666666666666667	\\
3e-09	0.108	0.666666666666667	0.666666666666667	\\
3.05e-09	0.108	0	0	\\
3.1e-09	0.108	0	0	\\
3.15e-09	0.108	0	0	\\
3.2e-09	0.108	0	0	\\
3.25e-09	0.108	0	0	\\
3.3e-09	0.108	0	0	\\
3.35e-09	0.108	0	0	\\
3.4e-09	0.108	0	0	\\
3.45e-09	0.108	0	0	\\
3.5e-09	0.108	0	0	\\
3.55e-09	0.108	0	0	\\
3.6e-09	0.108	0	0	\\
3.65e-09	0.108	0	0	\\
3.7e-09	0.108	0	0	\\
3.75e-09	0.108	0	0	\\
3.8e-09	0.108	0	0	\\
3.85e-09	0.108	0	0	\\
3.9e-09	0.108	0	0	\\
3.95e-09	0.108	0	0	\\
4e-09	0.108	0	0	\\
4.05e-09	0.108	0	0	\\
4.1e-09	0.108	0	0	\\
4.15e-09	0.108	0	0	\\
4.2e-09	0.108	0	0	\\
4.25e-09	0.108	0	0	\\
4.3e-09	0.108	0	0	\\
4.35e-09	0.108	0	0	\\
4.4e-09	0.108	0	0	\\
4.45e-09	0.108	0	0	\\
4.5e-09	0.108	0	0	\\
4.55e-09	0.108	0	0	\\
4.6e-09	0.108	0	0	\\
4.65e-09	0.108	0	0	\\
4.7e-09	0.108	-0.444444444444444	-0.444444444444444	\\
4.75e-09	0.108	-0.444444444444444	-0.444444444444444	\\
4.8e-09	0.108	-0.444444444444444	-0.444444444444444	\\
4.85e-09	0.108	-0.444444444444444	-0.444444444444444	\\
4.9e-09	0.108	-0.444444444444444	-0.444444444444444	\\
4.95e-09	0.108	0	0	\\
5e-09	0.108	0	0	\\
5e-09	0.108	0	nan	\\
5e-09	0.108	-0.666666666666667	0.166666666666667	\\
5e-09	0.108	-0.666666666666667	nan	\\
0	0.12	-0.666666666666667	nan	\\
0	0.12	0	0.166666666666667	\\
0	0.12	0	0	\\
5e-11	0.12	0	0	\\
1e-10	0.12	0	0	\\
1.5e-10	0.12	1	1	\\
2e-10	0.12	1	1	\\
2.5e-10	0.12	1	1	\\
3e-10	0.12	1	1	\\
3.5e-10	0.12	1	1	\\
4e-10	0.12	1	1	\\
4.5e-10	0.12	1	1	\\
5e-10	0.12	1	1	\\
5.5e-10	0.12	1	1	\\
6e-10	0.12	1	1	\\
6.5e-10	0.12	0	0	\\
7e-10	0.12	0	0	\\
7.5e-10	0.12	0	0	\\
8e-10	0.12	0	0	\\
8.5e-10	0.12	0	0	\\
9e-10	0.12	0	0	\\
9.5e-10	0.12	0	0	\\
1e-09	0.12	0	0	\\
1.05e-09	0.12	0	0	\\
1.1e-09	0.12	0	0	\\
1.15e-09	0.12	0	0	\\
1.2e-09	0.12	0	0	\\
1.25e-09	0.12	0	0	\\
1.3e-09	0.12	0	0	\\
1.35e-09	0.12	0	0	\\
1.4e-09	0.12	0	0	\\
1.45e-09	0.12	0	0	\\
1.5e-09	0.12	0	0	\\
1.55e-09	0.12	0	0	\\
1.6e-09	0.12	0	0	\\
1.65e-09	0.12	0	0	\\
1.7e-09	0.12	0	0	\\
1.75e-09	0.12	0	0	\\
1.8e-09	0.12	0	0	\\
1.85e-09	0.12	0	0	\\
1.9e-09	0.12	0	0	\\
1.95e-09	0.12	0	0	\\
2e-09	0.12	0	0	\\
2.05e-09	0.12	0	0	\\
2.1e-09	0.12	0	0	\\
2.15e-09	0.12	0	0	\\
2.2e-09	0.12	0	0	\\
2.25e-09	0.12	0	0	\\
2.3e-09	0.12	-0.666666666666667	-0.666666666666667	\\
2.35e-09	0.12	-0.666666666666667	-0.666666666666667	\\
2.4e-09	0.12	-0.666666666666667	-0.666666666666667	\\
2.45e-09	0.12	-0.666666666666667	-0.666666666666667	\\
2.5e-09	0.12	-0.666666666666667	-0.666666666666667	\\
2.55e-09	0.12	0	0	\\
2.6e-09	0.12	0	0	\\
2.65e-09	0.12	0	0	\\
2.7e-09	0.12	0	0	\\
2.75e-09	0.12	0	0	\\
2.8e-09	0.12	0.666666666666667	0.666666666666667	\\
2.85e-09	0.12	0.666666666666667	0.666666666666667	\\
2.9e-09	0.12	0.666666666666667	0.666666666666667	\\
2.95e-09	0.12	0.666666666666667	0.666666666666667	\\
3e-09	0.12	0.666666666666667	0.666666666666667	\\
3.05e-09	0.12	0	0	\\
3.1e-09	0.12	0	0	\\
3.15e-09	0.12	0	0	\\
3.2e-09	0.12	0	0	\\
3.25e-09	0.12	0	0	\\
3.3e-09	0.12	0	0	\\
3.35e-09	0.12	0	0	\\
3.4e-09	0.12	0	0	\\
3.45e-09	0.12	0	0	\\
3.5e-09	0.12	0	0	\\
3.55e-09	0.12	0	0	\\
3.6e-09	0.12	0	0	\\
3.65e-09	0.12	0	0	\\
3.7e-09	0.12	0	0	\\
3.75e-09	0.12	0	0	\\
3.8e-09	0.12	0	0	\\
3.85e-09	0.12	0	0	\\
3.9e-09	0.12	0	0	\\
3.95e-09	0.12	0	0	\\
4e-09	0.12	0	0	\\
4.05e-09	0.12	0	0	\\
4.1e-09	0.12	0	0	\\
4.15e-09	0.12	0	0	\\
4.2e-09	0.12	0	0	\\
4.25e-09	0.12	0	0	\\
4.3e-09	0.12	0	0	\\
4.35e-09	0.12	0	0	\\
4.4e-09	0.12	0	0	\\
4.45e-09	0.12	0	0	\\
4.5e-09	0.12	0	0	\\
4.55e-09	0.12	0	0	\\
4.6e-09	0.12	0	0	\\
4.65e-09	0.12	0	0	\\
4.7e-09	0.12	-0.444444444444444	-0.444444444444444	\\
4.75e-09	0.12	-0.444444444444444	-0.444444444444444	\\
4.8e-09	0.12	-0.444444444444444	-0.444444444444444	\\
4.85e-09	0.12	-0.444444444444444	-0.444444444444444	\\
4.9e-09	0.12	-0.444444444444444	-0.444444444444444	\\
4.95e-09	0.12	0	0	\\
5e-09	0.12	0	0	\\
5e-09	0.12	0	nan	\\
5e-09	0.12	-0.666666666666667	0.166666666666667	\\
5e-09	0.12	-0.666666666666667	nan	\\
0	0.132	-0.666666666666667	nan	\\
0	0.132	0	0.166666666666667	\\
0	0.132	0	0	\\
5e-11	0.132	0	0	\\
1e-10	0.132	0	0	\\
1.5e-10	0.132	1	1	\\
2e-10	0.132	1	1	\\
2.5e-10	0.132	1	1	\\
3e-10	0.132	1	1	\\
3.5e-10	0.132	1	1	\\
4e-10	0.132	1	1	\\
4.5e-10	0.132	1	1	\\
5e-10	0.132	1	1	\\
5.5e-10	0.132	1	1	\\
6e-10	0.132	1	1	\\
6.5e-10	0.132	0	0	\\
7e-10	0.132	0	0	\\
7.5e-10	0.132	0	0	\\
8e-10	0.132	0	0	\\
8.5e-10	0.132	0	0	\\
9e-10	0.132	0	0	\\
9.5e-10	0.132	0	0	\\
1e-09	0.132	0	0	\\
1.05e-09	0.132	0	0	\\
1.1e-09	0.132	0	0	\\
1.15e-09	0.132	0	0	\\
1.2e-09	0.132	0	0	\\
1.25e-09	0.132	0	0	\\
1.3e-09	0.132	0	0	\\
1.35e-09	0.132	0	0	\\
1.4e-09	0.132	0	0	\\
1.45e-09	0.132	0	0	\\
1.5e-09	0.132	0	0	\\
1.55e-09	0.132	0	0	\\
1.6e-09	0.132	0	0	\\
1.65e-09	0.132	0	0	\\
1.7e-09	0.132	0	0	\\
1.75e-09	0.132	0	0	\\
1.8e-09	0.132	0	0	\\
1.85e-09	0.132	0	0	\\
1.9e-09	0.132	0	0	\\
1.95e-09	0.132	0	0	\\
2e-09	0.132	0	0	\\
2.05e-09	0.132	0	0	\\
2.1e-09	0.132	0	0	\\
2.15e-09	0.132	0	0	\\
2.2e-09	0.132	0	0	\\
2.25e-09	0.132	0	0	\\
2.3e-09	0.132	-0.666666666666667	-0.666666666666667	\\
2.35e-09	0.132	-0.666666666666667	-0.666666666666667	\\
2.4e-09	0.132	-0.666666666666667	-0.666666666666667	\\
2.45e-09	0.132	-0.666666666666667	-0.666666666666667	\\
2.5e-09	0.132	-0.666666666666667	-0.666666666666667	\\
2.55e-09	0.132	0	0	\\
2.6e-09	0.132	0	0	\\
2.65e-09	0.132	0	0	\\
2.7e-09	0.132	0	0	\\
2.75e-09	0.132	0	0	\\
2.8e-09	0.132	0.666666666666667	0.666666666666667	\\
2.85e-09	0.132	0.666666666666667	0.666666666666667	\\
2.9e-09	0.132	0.666666666666667	0.666666666666667	\\
2.95e-09	0.132	0.666666666666667	0.666666666666667	\\
3e-09	0.132	0.666666666666667	0.666666666666667	\\
3.05e-09	0.132	0	0	\\
3.1e-09	0.132	0	0	\\
3.15e-09	0.132	0	0	\\
3.2e-09	0.132	0	0	\\
3.25e-09	0.132	0	0	\\
3.3e-09	0.132	0	0	\\
3.35e-09	0.132	0	0	\\
3.4e-09	0.132	0	0	\\
3.45e-09	0.132	0	0	\\
3.5e-09	0.132	0	0	\\
3.55e-09	0.132	0	0	\\
3.6e-09	0.132	0	0	\\
3.65e-09	0.132	0	0	\\
3.7e-09	0.132	0	0	\\
3.75e-09	0.132	0	0	\\
3.8e-09	0.132	0	0	\\
3.85e-09	0.132	0	0	\\
3.9e-09	0.132	0	0	\\
3.95e-09	0.132	0	0	\\
4e-09	0.132	0	0	\\
4.05e-09	0.132	0	0	\\
4.1e-09	0.132	0	0	\\
4.15e-09	0.132	0	0	\\
4.2e-09	0.132	0	0	\\
4.25e-09	0.132	0	0	\\
4.3e-09	0.132	0	0	\\
4.35e-09	0.132	0	0	\\
4.4e-09	0.132	0	0	\\
4.45e-09	0.132	0	0	\\
4.5e-09	0.132	0	0	\\
4.55e-09	0.132	0	0	\\
4.6e-09	0.132	0	0	\\
4.65e-09	0.132	0	0	\\
4.7e-09	0.132	-0.444444444444444	-0.444444444444444	\\
4.75e-09	0.132	-0.444444444444444	-0.444444444444444	\\
4.8e-09	0.132	-0.444444444444444	-0.444444444444444	\\
4.85e-09	0.132	-0.444444444444444	-0.444444444444444	\\
4.9e-09	0.132	-0.444444444444444	-0.444444444444444	\\
4.95e-09	0.132	0	0	\\
5e-09	0.132	0	0	\\
5e-09	0.132	0	nan	\\
5e-09	0.132	-0.666666666666667	0.166666666666667	\\
5e-09	0.132	-0.666666666666667	nan	\\
0	0.144	-0.666666666666667	nan	\\
0	0.144	0	0.166666666666667	\\
0	0.144	0	0	\\
5e-11	0.144	0	0	\\
1e-10	0.144	0	0	\\
1.5e-10	0.144	1	1	\\
2e-10	0.144	1	1	\\
2.5e-10	0.144	1	1	\\
3e-10	0.144	1	1	\\
3.5e-10	0.144	1	1	\\
4e-10	0.144	1	1	\\
4.5e-10	0.144	1	1	\\
5e-10	0.144	1	1	\\
5.5e-10	0.144	1	1	\\
6e-10	0.144	1	1	\\
6.5e-10	0.144	0	0	\\
7e-10	0.144	0	0	\\
7.5e-10	0.144	0	0	\\
8e-10	0.144	0	0	\\
8.5e-10	0.144	0	0	\\
9e-10	0.144	0	0	\\
9.5e-10	0.144	0	0	\\
1e-09	0.144	0	0	\\
1.05e-09	0.144	0	0	\\
1.1e-09	0.144	0	0	\\
1.15e-09	0.144	0	0	\\
1.2e-09	0.144	0	0	\\
1.25e-09	0.144	0	0	\\
1.3e-09	0.144	0	0	\\
1.35e-09	0.144	0	0	\\
1.4e-09	0.144	0	0	\\
1.45e-09	0.144	0	0	\\
1.5e-09	0.144	0	0	\\
1.55e-09	0.144	0	0	\\
1.6e-09	0.144	0	0	\\
1.65e-09	0.144	0	0	\\
1.7e-09	0.144	0	0	\\
1.75e-09	0.144	0	0	\\
1.8e-09	0.144	0	0	\\
1.85e-09	0.144	0	0	\\
1.9e-09	0.144	0	0	\\
1.95e-09	0.144	0	0	\\
2e-09	0.144	0	0	\\
2.05e-09	0.144	0	0	\\
2.1e-09	0.144	0	0	\\
2.15e-09	0.144	0	0	\\
2.2e-09	0.144	0	0	\\
2.25e-09	0.144	0	0	\\
2.3e-09	0.144	-0.666666666666667	-0.666666666666667	\\
2.35e-09	0.144	-0.666666666666667	-0.666666666666667	\\
2.4e-09	0.144	-0.666666666666667	-0.666666666666667	\\
2.45e-09	0.144	-0.666666666666667	-0.666666666666667	\\
2.5e-09	0.144	-0.666666666666667	-0.666666666666667	\\
2.55e-09	0.144	0	0	\\
2.6e-09	0.144	0	0	\\
2.65e-09	0.144	0	0	\\
2.7e-09	0.144	0	0	\\
2.75e-09	0.144	0	0	\\
2.8e-09	0.144	0.666666666666667	0.666666666666667	\\
2.85e-09	0.144	0.666666666666667	0.666666666666667	\\
2.9e-09	0.144	0.666666666666667	0.666666666666667	\\
2.95e-09	0.144	0.666666666666667	0.666666666666667	\\
3e-09	0.144	0.666666666666667	0.666666666666667	\\
3.05e-09	0.144	0	0	\\
3.1e-09	0.144	0	0	\\
3.15e-09	0.144	0	0	\\
3.2e-09	0.144	0	0	\\
3.25e-09	0.144	0	0	\\
3.3e-09	0.144	0	0	\\
3.35e-09	0.144	0	0	\\
3.4e-09	0.144	0	0	\\
3.45e-09	0.144	0	0	\\
3.5e-09	0.144	0	0	\\
3.55e-09	0.144	0	0	\\
3.6e-09	0.144	0	0	\\
3.65e-09	0.144	0	0	\\
3.7e-09	0.144	0	0	\\
3.75e-09	0.144	0	0	\\
3.8e-09	0.144	0	0	\\
3.85e-09	0.144	0	0	\\
3.9e-09	0.144	0	0	\\
3.95e-09	0.144	0	0	\\
4e-09	0.144	0	0	\\
4.05e-09	0.144	0	0	\\
4.1e-09	0.144	0	0	\\
4.15e-09	0.144	0	0	\\
4.2e-09	0.144	0	0	\\
4.25e-09	0.144	0	0	\\
4.3e-09	0.144	0	0	\\
4.35e-09	0.144	0	0	\\
4.4e-09	0.144	0	0	\\
4.45e-09	0.144	0	0	\\
4.5e-09	0.144	0	0	\\
4.55e-09	0.144	0	0	\\
4.6e-09	0.144	0	0	\\
4.65e-09	0.144	0	0	\\
4.7e-09	0.144	-0.444444444444444	-0.444444444444444	\\
4.75e-09	0.144	-0.444444444444444	-0.444444444444444	\\
4.8e-09	0.144	-0.444444444444444	-0.444444444444444	\\
4.85e-09	0.144	-0.444444444444444	-0.444444444444444	\\
4.9e-09	0.144	-0.444444444444444	-0.444444444444444	\\
4.95e-09	0.144	0	0	\\
5e-09	0.144	0	0	\\
5e-09	0.144	0	nan	\\
5e-09	0.144	-0.666666666666667	0.166666666666667	\\
5e-09	0.144	-0.666666666666667	nan	\\
0	0.156	-0.666666666666667	nan	\\
0	0.156	0	0.166666666666667	\\
0	0.156	0	0	\\
5e-11	0.156	0	0	\\
1e-10	0.156	0	0	\\
1.5e-10	0.156	0	0	\\
2e-10	0.156	1	1	\\
2.5e-10	0.156	1	1	\\
3e-10	0.156	1	1	\\
3.5e-10	0.156	1	1	\\
4e-10	0.156	1	1	\\
4.5e-10	0.156	1	1	\\
5e-10	0.156	1	1	\\
5.5e-10	0.156	1	1	\\
6e-10	0.156	1	1	\\
6.5e-10	0.156	1	1	\\
7e-10	0.156	0	0	\\
7.5e-10	0.156	0	0	\\
8e-10	0.156	0	0	\\
8.5e-10	0.156	0	0	\\
9e-10	0.156	0	0	\\
9.5e-10	0.156	0	0	\\
1e-09	0.156	0	0	\\
1.05e-09	0.156	0	0	\\
1.1e-09	0.156	0	0	\\
1.15e-09	0.156	0	0	\\
1.2e-09	0.156	0	0	\\
1.25e-09	0.156	0	0	\\
1.3e-09	0.156	0	0	\\
1.35e-09	0.156	0	0	\\
1.4e-09	0.156	0	0	\\
1.45e-09	0.156	0	0	\\
1.5e-09	0.156	0	0	\\
1.55e-09	0.156	0	0	\\
1.6e-09	0.156	0	0	\\
1.65e-09	0.156	0	0	\\
1.7e-09	0.156	0	0	\\
1.75e-09	0.156	0	0	\\
1.8e-09	0.156	0	0	\\
1.85e-09	0.156	0	0	\\
1.9e-09	0.156	0	0	\\
1.95e-09	0.156	0	0	\\
2e-09	0.156	0	0	\\
2.05e-09	0.156	0	0	\\
2.1e-09	0.156	0	0	\\
2.15e-09	0.156	0	0	\\
2.2e-09	0.156	0	0	\\
2.25e-09	0.156	-0.666666666666667	-0.666666666666667	\\
2.3e-09	0.156	-0.666666666666667	-0.666666666666667	\\
2.35e-09	0.156	-0.666666666666667	-0.666666666666667	\\
2.4e-09	0.156	-0.666666666666667	-0.666666666666667	\\
2.45e-09	0.156	-0.666666666666667	-0.666666666666667	\\
2.5e-09	0.156	-0.666666666666667	-0.666666666666667	\\
2.55e-09	0.156	-0.666666666666667	-0.666666666666667	\\
2.6e-09	0.156	0	0	\\
2.65e-09	0.156	0	0	\\
2.7e-09	0.156	0	0	\\
2.75e-09	0.156	0.666666666666667	0.666666666666667	\\
2.8e-09	0.156	0.666666666666667	0.666666666666667	\\
2.85e-09	0.156	0.666666666666667	0.666666666666667	\\
2.9e-09	0.156	0.666666666666667	0.666666666666667	\\
2.95e-09	0.156	0.666666666666667	0.666666666666667	\\
3e-09	0.156	0.666666666666667	0.666666666666667	\\
3.05e-09	0.156	0.666666666666667	0.666666666666667	\\
3.1e-09	0.156	0	0	\\
3.15e-09	0.156	0	0	\\
3.2e-09	0.156	0	0	\\
3.25e-09	0.156	0	0	\\
3.3e-09	0.156	0	0	\\
3.35e-09	0.156	0	0	\\
3.4e-09	0.156	0	0	\\
3.45e-09	0.156	0	0	\\
3.5e-09	0.156	0	0	\\
3.55e-09	0.156	0	0	\\
3.6e-09	0.156	0	0	\\
3.65e-09	0.156	0	0	\\
3.7e-09	0.156	0	0	\\
3.75e-09	0.156	0	0	\\
3.8e-09	0.156	0	0	\\
3.85e-09	0.156	0	0	\\
3.9e-09	0.156	0	0	\\
3.95e-09	0.156	0	0	\\
4e-09	0.156	0	0	\\
4.05e-09	0.156	0	0	\\
4.1e-09	0.156	0	0	\\
4.15e-09	0.156	0	0	\\
4.2e-09	0.156	0	0	\\
4.25e-09	0.156	0	0	\\
4.3e-09	0.156	0	0	\\
4.35e-09	0.156	0	0	\\
4.4e-09	0.156	0	0	\\
4.45e-09	0.156	0	0	\\
4.5e-09	0.156	0	0	\\
4.55e-09	0.156	0	0	\\
4.6e-09	0.156	0	0	\\
4.65e-09	0.156	-0.444444444444444	-0.444444444444444	\\
4.7e-09	0.156	-0.444444444444444	-0.444444444444444	\\
4.75e-09	0.156	-0.444444444444444	-0.444444444444444	\\
4.8e-09	0.156	-0.444444444444444	-0.444444444444444	\\
4.85e-09	0.156	-0.444444444444444	-0.444444444444444	\\
4.9e-09	0.156	-0.444444444444444	-0.444444444444444	\\
4.95e-09	0.156	-0.444444444444444	-0.444444444444444	\\
5e-09	0.156	0	0	\\
5e-09	0.156	0	nan	\\
5e-09	0.156	-0.666666666666667	0.166666666666667	\\
5e-09	0.156	-0.666666666666667	nan	\\
0	0.168	-0.666666666666667	nan	\\
0	0.168	0	0.166666666666667	\\
0	0.168	0	0	\\
5e-11	0.168	0	0	\\
1e-10	0.168	0	0	\\
1.5e-10	0.168	0	0	\\
2e-10	0.168	1	1	\\
2.5e-10	0.168	1	1	\\
3e-10	0.168	1	1	\\
3.5e-10	0.168	1	1	\\
4e-10	0.168	1	1	\\
4.5e-10	0.168	1	1	\\
5e-10	0.168	1	1	\\
5.5e-10	0.168	1	1	\\
6e-10	0.168	1	1	\\
6.5e-10	0.168	1	1	\\
7e-10	0.168	0	0	\\
7.5e-10	0.168	0	0	\\
8e-10	0.168	0	0	\\
8.5e-10	0.168	0	0	\\
9e-10	0.168	0	0	\\
9.5e-10	0.168	0	0	\\
1e-09	0.168	0	0	\\
1.05e-09	0.168	0	0	\\
1.1e-09	0.168	0	0	\\
1.15e-09	0.168	0	0	\\
1.2e-09	0.168	0	0	\\
1.25e-09	0.168	0	0	\\
1.3e-09	0.168	0	0	\\
1.35e-09	0.168	0	0	\\
1.4e-09	0.168	0	0	\\
1.45e-09	0.168	0	0	\\
1.5e-09	0.168	0	0	\\
1.55e-09	0.168	0	0	\\
1.6e-09	0.168	0	0	\\
1.65e-09	0.168	0	0	\\
1.7e-09	0.168	0	0	\\
1.75e-09	0.168	0	0	\\
1.8e-09	0.168	0	0	\\
1.85e-09	0.168	0	0	\\
1.9e-09	0.168	0	0	\\
1.95e-09	0.168	0	0	\\
2e-09	0.168	0	0	\\
2.05e-09	0.168	0	0	\\
2.1e-09	0.168	0	0	\\
2.15e-09	0.168	0	0	\\
2.2e-09	0.168	0	0	\\
2.25e-09	0.168	-0.666666666666667	-0.666666666666667	\\
2.3e-09	0.168	-0.666666666666667	-0.666666666666667	\\
2.35e-09	0.168	-0.666666666666667	-0.666666666666667	\\
2.4e-09	0.168	-0.666666666666667	-0.666666666666667	\\
2.45e-09	0.168	-0.666666666666667	-0.666666666666667	\\
2.5e-09	0.168	-0.666666666666667	-0.666666666666667	\\
2.55e-09	0.168	-0.666666666666667	-0.666666666666667	\\
2.6e-09	0.168	0	0	\\
2.65e-09	0.168	0	0	\\
2.7e-09	0.168	0	0	\\
2.75e-09	0.168	0.666666666666667	0.666666666666667	\\
2.8e-09	0.168	0.666666666666667	0.666666666666667	\\
2.85e-09	0.168	0.666666666666667	0.666666666666667	\\
2.9e-09	0.168	0.666666666666667	0.666666666666667	\\
2.95e-09	0.168	0.666666666666667	0.666666666666667	\\
3e-09	0.168	0.666666666666667	0.666666666666667	\\
3.05e-09	0.168	0.666666666666667	0.666666666666667	\\
3.1e-09	0.168	0	0	\\
3.15e-09	0.168	0	0	\\
3.2e-09	0.168	0	0	\\
3.25e-09	0.168	0	0	\\
3.3e-09	0.168	0	0	\\
3.35e-09	0.168	0	0	\\
3.4e-09	0.168	0	0	\\
3.45e-09	0.168	0	0	\\
3.5e-09	0.168	0	0	\\
3.55e-09	0.168	0	0	\\
3.6e-09	0.168	0	0	\\
3.65e-09	0.168	0	0	\\
3.7e-09	0.168	0	0	\\
3.75e-09	0.168	0	0	\\
3.8e-09	0.168	0	0	\\
3.85e-09	0.168	0	0	\\
3.9e-09	0.168	0	0	\\
3.95e-09	0.168	0	0	\\
4e-09	0.168	0	0	\\
4.05e-09	0.168	0	0	\\
4.1e-09	0.168	0	0	\\
4.15e-09	0.168	0	0	\\
4.2e-09	0.168	0	0	\\
4.25e-09	0.168	0	0	\\
4.3e-09	0.168	0	0	\\
4.35e-09	0.168	0	0	\\
4.4e-09	0.168	0	0	\\
4.45e-09	0.168	0	0	\\
4.5e-09	0.168	0	0	\\
4.55e-09	0.168	0	0	\\
4.6e-09	0.168	0	0	\\
4.65e-09	0.168	-0.444444444444444	-0.444444444444444	\\
4.7e-09	0.168	-0.444444444444444	-0.444444444444444	\\
4.75e-09	0.168	-0.444444444444444	-0.444444444444444	\\
4.8e-09	0.168	-0.444444444444444	-0.444444444444444	\\
4.85e-09	0.168	-0.444444444444444	-0.444444444444444	\\
4.9e-09	0.168	-0.444444444444444	-0.444444444444444	\\
4.95e-09	0.168	-0.444444444444444	-0.444444444444444	\\
5e-09	0.168	0	0	\\
5e-09	0.168	0	nan	\\
5e-09	0.168	-0.666666666666667	0.166666666666667	\\
5e-09	0.168	-0.666666666666667	nan	\\
0	0.18	-0.666666666666667	nan	\\
0	0.18	0	0.166666666666667	\\
0	0.18	0	0	\\
5e-11	0.18	0	0	\\
1e-10	0.18	0	0	\\
1.5e-10	0.18	0	0	\\
2e-10	0.18	1	1	\\
2.5e-10	0.18	1	1	\\
3e-10	0.18	1	1	\\
3.5e-10	0.18	1	1	\\
4e-10	0.18	1	1	\\
4.5e-10	0.18	1	1	\\
5e-10	0.18	1	1	\\
5.5e-10	0.18	1	1	\\
6e-10	0.18	1	1	\\
6.5e-10	0.18	1	1	\\
7e-10	0.18	0	0	\\
7.5e-10	0.18	0	0	\\
8e-10	0.18	0	0	\\
8.5e-10	0.18	0	0	\\
9e-10	0.18	0	0	\\
9.5e-10	0.18	0	0	\\
1e-09	0.18	0	0	\\
1.05e-09	0.18	0	0	\\
1.1e-09	0.18	0	0	\\
1.15e-09	0.18	0	0	\\
1.2e-09	0.18	0	0	\\
1.25e-09	0.18	0	0	\\
1.3e-09	0.18	0	0	\\
1.35e-09	0.18	0	0	\\
1.4e-09	0.18	0	0	\\
1.45e-09	0.18	0	0	\\
1.5e-09	0.18	0	0	\\
1.55e-09	0.18	0	0	\\
1.6e-09	0.18	0	0	\\
1.65e-09	0.18	0	0	\\
1.7e-09	0.18	0	0	\\
1.75e-09	0.18	0	0	\\
1.8e-09	0.18	0	0	\\
1.85e-09	0.18	0	0	\\
1.9e-09	0.18	0	0	\\
1.95e-09	0.18	0	0	\\
2e-09	0.18	0	0	\\
2.05e-09	0.18	0	0	\\
2.1e-09	0.18	0	0	\\
2.15e-09	0.18	0	0	\\
2.2e-09	0.18	0	0	\\
2.25e-09	0.18	-0.666666666666667	-0.666666666666667	\\
2.3e-09	0.18	-0.666666666666667	-0.666666666666667	\\
2.35e-09	0.18	-0.666666666666667	-0.666666666666667	\\
2.4e-09	0.18	-0.666666666666667	-0.666666666666667	\\
2.45e-09	0.18	-0.666666666666667	-0.666666666666667	\\
2.5e-09	0.18	-0.666666666666667	-0.666666666666667	\\
2.55e-09	0.18	-0.666666666666667	-0.666666666666667	\\
2.6e-09	0.18	0	0	\\
2.65e-09	0.18	0	0	\\
2.7e-09	0.18	0	0	\\
2.75e-09	0.18	0.666666666666667	0.666666666666667	\\
2.8e-09	0.18	0.666666666666667	0.666666666666667	\\
2.85e-09	0.18	0.666666666666667	0.666666666666667	\\
2.9e-09	0.18	0.666666666666667	0.666666666666667	\\
2.95e-09	0.18	0.666666666666667	0.666666666666667	\\
3e-09	0.18	0.666666666666667	0.666666666666667	\\
3.05e-09	0.18	0.666666666666667	0.666666666666667	\\
3.1e-09	0.18	0	0	\\
3.15e-09	0.18	0	0	\\
3.2e-09	0.18	0	0	\\
3.25e-09	0.18	0	0	\\
3.3e-09	0.18	0	0	\\
3.35e-09	0.18	0	0	\\
3.4e-09	0.18	0	0	\\
3.45e-09	0.18	0	0	\\
3.5e-09	0.18	0	0	\\
3.55e-09	0.18	0	0	\\
3.6e-09	0.18	0	0	\\
3.65e-09	0.18	0	0	\\
3.7e-09	0.18	0	0	\\
3.75e-09	0.18	0	0	\\
3.8e-09	0.18	0	0	\\
3.85e-09	0.18	0	0	\\
3.9e-09	0.18	0	0	\\
3.95e-09	0.18	0	0	\\
4e-09	0.18	0	0	\\
4.05e-09	0.18	0	0	\\
4.1e-09	0.18	0	0	\\
4.15e-09	0.18	0	0	\\
4.2e-09	0.18	0	0	\\
4.25e-09	0.18	0	0	\\
4.3e-09	0.18	0	0	\\
4.35e-09	0.18	0	0	\\
4.4e-09	0.18	0	0	\\
4.45e-09	0.18	0	0	\\
4.5e-09	0.18	0	0	\\
4.55e-09	0.18	0	0	\\
4.6e-09	0.18	0	0	\\
4.65e-09	0.18	-0.444444444444444	-0.444444444444444	\\
4.7e-09	0.18	-0.444444444444444	-0.444444444444444	\\
4.75e-09	0.18	-0.444444444444444	-0.444444444444444	\\
4.8e-09	0.18	-0.444444444444444	-0.444444444444444	\\
4.85e-09	0.18	-0.444444444444444	-0.444444444444444	\\
4.9e-09	0.18	-0.444444444444444	-0.444444444444444	\\
4.95e-09	0.18	-0.444444444444444	-0.444444444444444	\\
5e-09	0.18	0	0	\\
5e-09	0.18	0	nan	\\
5e-09	0.18	-0.666666666666667	0.166666666666667	\\
5e-09	0.18	-0.666666666666667	nan	\\
0	0.192	-0.666666666666667	nan	\\
0	0.192	0	0.166666666666667	\\
0	0.192	0	0	\\
5e-11	0.192	0	0	\\
1e-10	0.192	0	0	\\
1.5e-10	0.192	0	0	\\
2e-10	0.192	1	1	\\
2.5e-10	0.192	1	1	\\
3e-10	0.192	1	1	\\
3.5e-10	0.192	1	1	\\
4e-10	0.192	1	1	\\
4.5e-10	0.192	1	1	\\
5e-10	0.192	1	1	\\
5.5e-10	0.192	1	1	\\
6e-10	0.192	1	1	\\
6.5e-10	0.192	1	1	\\
7e-10	0.192	0	0	\\
7.5e-10	0.192	0	0	\\
8e-10	0.192	0	0	\\
8.5e-10	0.192	0	0	\\
9e-10	0.192	0	0	\\
9.5e-10	0.192	0	0	\\
1e-09	0.192	0	0	\\
1.05e-09	0.192	0	0	\\
1.1e-09	0.192	0	0	\\
1.15e-09	0.192	0	0	\\
1.2e-09	0.192	0	0	\\
1.25e-09	0.192	0	0	\\
1.3e-09	0.192	0	0	\\
1.35e-09	0.192	0	0	\\
1.4e-09	0.192	0	0	\\
1.45e-09	0.192	0	0	\\
1.5e-09	0.192	0	0	\\
1.55e-09	0.192	0	0	\\
1.6e-09	0.192	0	0	\\
1.65e-09	0.192	0	0	\\
1.7e-09	0.192	0	0	\\
1.75e-09	0.192	0	0	\\
1.8e-09	0.192	0	0	\\
1.85e-09	0.192	0	0	\\
1.9e-09	0.192	0	0	\\
1.95e-09	0.192	0	0	\\
2e-09	0.192	0	0	\\
2.05e-09	0.192	0	0	\\
2.1e-09	0.192	0	0	\\
2.15e-09	0.192	0	0	\\
2.2e-09	0.192	0	0	\\
2.25e-09	0.192	-0.666666666666667	-0.666666666666667	\\
2.3e-09	0.192	-0.666666666666667	-0.666666666666667	\\
2.35e-09	0.192	-0.666666666666667	-0.666666666666667	\\
2.4e-09	0.192	-0.666666666666667	-0.666666666666667	\\
2.45e-09	0.192	-0.666666666666667	-0.666666666666667	\\
2.5e-09	0.192	-0.666666666666667	-0.666666666666667	\\
2.55e-09	0.192	-0.666666666666667	-0.666666666666667	\\
2.6e-09	0.192	0	0	\\
2.65e-09	0.192	0	0	\\
2.7e-09	0.192	0	0	\\
2.75e-09	0.192	0.666666666666667	0.666666666666667	\\
2.8e-09	0.192	0.666666666666667	0.666666666666667	\\
2.85e-09	0.192	0.666666666666667	0.666666666666667	\\
2.9e-09	0.192	0.666666666666667	0.666666666666667	\\
2.95e-09	0.192	0.666666666666667	0.666666666666667	\\
3e-09	0.192	0.666666666666667	0.666666666666667	\\
3.05e-09	0.192	0.666666666666667	0.666666666666667	\\
3.1e-09	0.192	0	0	\\
3.15e-09	0.192	0	0	\\
3.2e-09	0.192	0	0	\\
3.25e-09	0.192	0	0	\\
3.3e-09	0.192	0	0	\\
3.35e-09	0.192	0	0	\\
3.4e-09	0.192	0	0	\\
3.45e-09	0.192	0	0	\\
3.5e-09	0.192	0	0	\\
3.55e-09	0.192	0	0	\\
3.6e-09	0.192	0	0	\\
3.65e-09	0.192	0	0	\\
3.7e-09	0.192	0	0	\\
3.75e-09	0.192	0	0	\\
3.8e-09	0.192	0	0	\\
3.85e-09	0.192	0	0	\\
3.9e-09	0.192	0	0	\\
3.95e-09	0.192	0	0	\\
4e-09	0.192	0	0	\\
4.05e-09	0.192	0	0	\\
4.1e-09	0.192	0	0	\\
4.15e-09	0.192	0	0	\\
4.2e-09	0.192	0	0	\\
4.25e-09	0.192	0	0	\\
4.3e-09	0.192	0	0	\\
4.35e-09	0.192	0	0	\\
4.4e-09	0.192	0	0	\\
4.45e-09	0.192	0	0	\\
4.5e-09	0.192	0	0	\\
4.55e-09	0.192	0	0	\\
4.6e-09	0.192	0	0	\\
4.65e-09	0.192	-0.444444444444444	-0.444444444444444	\\
4.7e-09	0.192	-0.444444444444444	-0.444444444444444	\\
4.75e-09	0.192	-0.444444444444444	-0.444444444444444	\\
4.8e-09	0.192	-0.444444444444444	-0.444444444444444	\\
4.85e-09	0.192	-0.444444444444444	-0.444444444444444	\\
4.9e-09	0.192	-0.444444444444444	-0.444444444444444	\\
4.95e-09	0.192	-0.444444444444444	-0.444444444444444	\\
5e-09	0.192	0	0	\\
5e-09	0.192	0	nan	\\
5e-09	0.192	-0.666666666666667	0.166666666666667	\\
5e-09	0.192	-0.666666666666667	nan	\\
0	0.204	-0.666666666666667	nan	\\
0	0.204	0	0.166666666666667	\\
0	0.204	0	0	\\
5e-11	0.204	0	0	\\
1e-10	0.204	0	0	\\
1.5e-10	0.204	0	0	\\
2e-10	0.204	0	0	\\
2.5e-10	0.204	1	1	\\
3e-10	0.204	1	1	\\
3.5e-10	0.204	1	1	\\
4e-10	0.204	1	1	\\
4.5e-10	0.204	1	1	\\
5e-10	0.204	1	1	\\
5.5e-10	0.204	1	1	\\
6e-10	0.204	1	1	\\
6.5e-10	0.204	1	1	\\
7e-10	0.204	1	1	\\
7.5e-10	0.204	0	0	\\
8e-10	0.204	0	0	\\
8.5e-10	0.204	0	0	\\
9e-10	0.204	0	0	\\
9.5e-10	0.204	0	0	\\
1e-09	0.204	0	0	\\
1.05e-09	0.204	0	0	\\
1.1e-09	0.204	0	0	\\
1.15e-09	0.204	0	0	\\
1.2e-09	0.204	0	0	\\
1.25e-09	0.204	0	0	\\
1.3e-09	0.204	0	0	\\
1.35e-09	0.204	0	0	\\
1.4e-09	0.204	0	0	\\
1.45e-09	0.204	0	0	\\
1.5e-09	0.204	0	0	\\
1.55e-09	0.204	0	0	\\
1.6e-09	0.204	0	0	\\
1.65e-09	0.204	0	0	\\
1.7e-09	0.204	0	0	\\
1.75e-09	0.204	0	0	\\
1.8e-09	0.204	0	0	\\
1.85e-09	0.204	0	0	\\
1.9e-09	0.204	0	0	\\
1.95e-09	0.204	0	0	\\
2e-09	0.204	0	0	\\
2.05e-09	0.204	0	0	\\
2.1e-09	0.204	0	0	\\
2.15e-09	0.204	0	0	\\
2.2e-09	0.204	-0.666666666666667	-0.666666666666667	\\
2.25e-09	0.204	-0.666666666666667	-0.666666666666667	\\
2.3e-09	0.204	-0.666666666666667	-0.666666666666667	\\
2.35e-09	0.204	-0.666666666666667	-0.666666666666667	\\
2.4e-09	0.204	-0.666666666666667	-0.666666666666667	\\
2.45e-09	0.204	-0.666666666666667	-0.666666666666667	\\
2.5e-09	0.204	-0.666666666666667	-0.666666666666667	\\
2.55e-09	0.204	-0.666666666666667	-0.666666666666667	\\
2.6e-09	0.204	-0.666666666666667	-0.666666666666667	\\
2.65e-09	0.204	0	0	\\
2.7e-09	0.204	0.666666666666667	0.666666666666667	\\
2.75e-09	0.204	0.666666666666667	0.666666666666667	\\
2.8e-09	0.204	0.666666666666667	0.666666666666667	\\
2.85e-09	0.204	0.666666666666667	0.666666666666667	\\
2.9e-09	0.204	0.666666666666667	0.666666666666667	\\
2.95e-09	0.204	0.666666666666667	0.666666666666667	\\
3e-09	0.204	0.666666666666667	0.666666666666667	\\
3.05e-09	0.204	0.666666666666667	0.666666666666667	\\
3.1e-09	0.204	0.666666666666667	0.666666666666667	\\
3.15e-09	0.204	0	0	\\
3.2e-09	0.204	0	0	\\
3.25e-09	0.204	0	0	\\
3.3e-09	0.204	0	0	\\
3.35e-09	0.204	0	0	\\
3.4e-09	0.204	0	0	\\
3.45e-09	0.204	0	0	\\
3.5e-09	0.204	0	0	\\
3.55e-09	0.204	0	0	\\
3.6e-09	0.204	0	0	\\
3.65e-09	0.204	0	0	\\
3.7e-09	0.204	0	0	\\
3.75e-09	0.204	0	0	\\
3.8e-09	0.204	0	0	\\
3.85e-09	0.204	0	0	\\
3.9e-09	0.204	0	0	\\
3.95e-09	0.204	0	0	\\
4e-09	0.204	0	0	\\
4.05e-09	0.204	0	0	\\
4.1e-09	0.204	0	0	\\
4.15e-09	0.204	0	0	\\
4.2e-09	0.204	0	0	\\
4.25e-09	0.204	0	0	\\
4.3e-09	0.204	0	0	\\
4.35e-09	0.204	0	0	\\
4.4e-09	0.204	0	0	\\
4.45e-09	0.204	0	0	\\
4.5e-09	0.204	0	0	\\
4.55e-09	0.204	0	0	\\
4.6e-09	0.204	-0.444444444444444	-0.444444444444444	\\
4.65e-09	0.204	-0.444444444444444	-0.444444444444444	\\
4.7e-09	0.204	-0.444444444444444	-0.444444444444444	\\
4.75e-09	0.204	-0.444444444444444	-0.444444444444444	\\
4.8e-09	0.204	-0.444444444444444	-0.444444444444444	\\
4.85e-09	0.204	-0.444444444444444	-0.444444444444444	\\
4.9e-09	0.204	-0.444444444444444	-0.444444444444444	\\
4.95e-09	0.204	-0.444444444444444	-0.444444444444444	\\
5e-09	0.204	-0.444444444444444	-0.444444444444444	\\
5e-09	0.204	-0.444444444444444	nan	\\
5e-09	0.204	-0.666666666666667	0.166666666666667	\\
5e-09	0.204	-0.666666666666667	nan	\\
0	0.216	-0.666666666666667	nan	\\
0	0.216	0	0.166666666666667	\\
0	0.216	0	0	\\
5e-11	0.216	0	0	\\
1e-10	0.216	0	0	\\
1.5e-10	0.216	0	0	\\
2e-10	0.216	0	0	\\
2.5e-10	0.216	1	1	\\
3e-10	0.216	1	1	\\
3.5e-10	0.216	1	1	\\
4e-10	0.216	1	1	\\
4.5e-10	0.216	1	1	\\
5e-10	0.216	1	1	\\
5.5e-10	0.216	1	1	\\
6e-10	0.216	1	1	\\
6.5e-10	0.216	1	1	\\
7e-10	0.216	1	1	\\
7.5e-10	0.216	0	0	\\
8e-10	0.216	0	0	\\
8.5e-10	0.216	0	0	\\
9e-10	0.216	0	0	\\
9.5e-10	0.216	0	0	\\
1e-09	0.216	0	0	\\
1.05e-09	0.216	0	0	\\
1.1e-09	0.216	0	0	\\
1.15e-09	0.216	0	0	\\
1.2e-09	0.216	0	0	\\
1.25e-09	0.216	0	0	\\
1.3e-09	0.216	0	0	\\
1.35e-09	0.216	0	0	\\
1.4e-09	0.216	0	0	\\
1.45e-09	0.216	0	0	\\
1.5e-09	0.216	0	0	\\
1.55e-09	0.216	0	0	\\
1.6e-09	0.216	0	0	\\
1.65e-09	0.216	0	0	\\
1.7e-09	0.216	0	0	\\
1.75e-09	0.216	0	0	\\
1.8e-09	0.216	0	0	\\
1.85e-09	0.216	0	0	\\
1.9e-09	0.216	0	0	\\
1.95e-09	0.216	0	0	\\
2e-09	0.216	0	0	\\
2.05e-09	0.216	0	0	\\
2.1e-09	0.216	0	0	\\
2.15e-09	0.216	0	0	\\
2.2e-09	0.216	-0.666666666666667	-0.666666666666667	\\
2.25e-09	0.216	-0.666666666666667	-0.666666666666667	\\
2.3e-09	0.216	-0.666666666666667	-0.666666666666667	\\
2.35e-09	0.216	-0.666666666666667	-0.666666666666667	\\
2.4e-09	0.216	-0.666666666666667	-0.666666666666667	\\
2.45e-09	0.216	-0.666666666666667	-0.666666666666667	\\
2.5e-09	0.216	-0.666666666666667	-0.666666666666667	\\
2.55e-09	0.216	-0.666666666666667	-0.666666666666667	\\
2.6e-09	0.216	-0.666666666666667	-0.666666666666667	\\
2.65e-09	0.216	0	0	\\
2.7e-09	0.216	0.666666666666667	0.666666666666667	\\
2.75e-09	0.216	0.666666666666667	0.666666666666667	\\
2.8e-09	0.216	0.666666666666667	0.666666666666667	\\
2.85e-09	0.216	0.666666666666667	0.666666666666667	\\
2.9e-09	0.216	0.666666666666667	0.666666666666667	\\
2.95e-09	0.216	0.666666666666667	0.666666666666667	\\
3e-09	0.216	0.666666666666667	0.666666666666667	\\
3.05e-09	0.216	0.666666666666667	0.666666666666667	\\
3.1e-09	0.216	0.666666666666667	0.666666666666667	\\
3.15e-09	0.216	0	0	\\
3.2e-09	0.216	0	0	\\
3.25e-09	0.216	0	0	\\
3.3e-09	0.216	0	0	\\
3.35e-09	0.216	0	0	\\
3.4e-09	0.216	0	0	\\
3.45e-09	0.216	0	0	\\
3.5e-09	0.216	0	0	\\
3.55e-09	0.216	0	0	\\
3.6e-09	0.216	0	0	\\
3.65e-09	0.216	0	0	\\
3.7e-09	0.216	0	0	\\
3.75e-09	0.216	0	0	\\
3.8e-09	0.216	0	0	\\
3.85e-09	0.216	0	0	\\
3.9e-09	0.216	0	0	\\
3.95e-09	0.216	0	0	\\
4e-09	0.216	0	0	\\
4.05e-09	0.216	0	0	\\
4.1e-09	0.216	0	0	\\
4.15e-09	0.216	0	0	\\
4.2e-09	0.216	0	0	\\
4.25e-09	0.216	0	0	\\
4.3e-09	0.216	0	0	\\
4.35e-09	0.216	0	0	\\
4.4e-09	0.216	0	0	\\
4.45e-09	0.216	0	0	\\
4.5e-09	0.216	0	0	\\
4.55e-09	0.216	0	0	\\
4.6e-09	0.216	-0.444444444444444	-0.444444444444444	\\
4.65e-09	0.216	-0.444444444444444	-0.444444444444444	\\
4.7e-09	0.216	-0.444444444444444	-0.444444444444444	\\
4.75e-09	0.216	-0.444444444444444	-0.444444444444444	\\
4.8e-09	0.216	-0.444444444444444	-0.444444444444444	\\
4.85e-09	0.216	-0.444444444444444	-0.444444444444444	\\
4.9e-09	0.216	-0.444444444444444	-0.444444444444444	\\
4.95e-09	0.216	-0.444444444444444	-0.444444444444444	\\
5e-09	0.216	-0.444444444444444	-0.444444444444444	\\
5e-09	0.216	-0.444444444444444	nan	\\
5e-09	0.216	-0.666666666666667	0.166666666666667	\\
5e-09	0.216	-0.666666666666667	nan	\\
0	0.228	-0.666666666666667	nan	\\
0	0.228	0	0.166666666666667	\\
0	0.228	0	0	\\
5e-11	0.228	0	0	\\
1e-10	0.228	0	0	\\
1.5e-10	0.228	0	0	\\
2e-10	0.228	0	0	\\
2.5e-10	0.228	1	1	\\
3e-10	0.228	1	1	\\
3.5e-10	0.228	1	1	\\
4e-10	0.228	1	1	\\
4.5e-10	0.228	1	1	\\
5e-10	0.228	1	1	\\
5.5e-10	0.228	1	1	\\
6e-10	0.228	1	1	\\
6.5e-10	0.228	1	1	\\
7e-10	0.228	1	1	\\
7.5e-10	0.228	0	0	\\
8e-10	0.228	0	0	\\
8.5e-10	0.228	0	0	\\
9e-10	0.228	0	0	\\
9.5e-10	0.228	0	0	\\
1e-09	0.228	0	0	\\
1.05e-09	0.228	0	0	\\
1.1e-09	0.228	0	0	\\
1.15e-09	0.228	0	0	\\
1.2e-09	0.228	0	0	\\
1.25e-09	0.228	0	0	\\
1.3e-09	0.228	0	0	\\
1.35e-09	0.228	0	0	\\
1.4e-09	0.228	0	0	\\
1.45e-09	0.228	0	0	\\
1.5e-09	0.228	0	0	\\
1.55e-09	0.228	0	0	\\
1.6e-09	0.228	0	0	\\
1.65e-09	0.228	0	0	\\
1.7e-09	0.228	0	0	\\
1.75e-09	0.228	0	0	\\
1.8e-09	0.228	0	0	\\
1.85e-09	0.228	0	0	\\
1.9e-09	0.228	0	0	\\
1.95e-09	0.228	0	0	\\
2e-09	0.228	0	0	\\
2.05e-09	0.228	0	0	\\
2.1e-09	0.228	0	0	\\
2.15e-09	0.228	0	0	\\
2.2e-09	0.228	-0.666666666666667	-0.666666666666667	\\
2.25e-09	0.228	-0.666666666666667	-0.666666666666667	\\
2.3e-09	0.228	-0.666666666666667	-0.666666666666667	\\
2.35e-09	0.228	-0.666666666666667	-0.666666666666667	\\
2.4e-09	0.228	-0.666666666666667	-0.666666666666667	\\
2.45e-09	0.228	-0.666666666666667	-0.666666666666667	\\
2.5e-09	0.228	-0.666666666666667	-0.666666666666667	\\
2.55e-09	0.228	-0.666666666666667	-0.666666666666667	\\
2.6e-09	0.228	-0.666666666666667	-0.666666666666667	\\
2.65e-09	0.228	0	0	\\
2.7e-09	0.228	0.666666666666667	0.666666666666667	\\
2.75e-09	0.228	0.666666666666667	0.666666666666667	\\
2.8e-09	0.228	0.666666666666667	0.666666666666667	\\
2.85e-09	0.228	0.666666666666667	0.666666666666667	\\
2.9e-09	0.228	0.666666666666667	0.666666666666667	\\
2.95e-09	0.228	0.666666666666667	0.666666666666667	\\
3e-09	0.228	0.666666666666667	0.666666666666667	\\
3.05e-09	0.228	0.666666666666667	0.666666666666667	\\
3.1e-09	0.228	0.666666666666667	0.666666666666667	\\
3.15e-09	0.228	0	0	\\
3.2e-09	0.228	0	0	\\
3.25e-09	0.228	0	0	\\
3.3e-09	0.228	0	0	\\
3.35e-09	0.228	0	0	\\
3.4e-09	0.228	0	0	\\
3.45e-09	0.228	0	0	\\
3.5e-09	0.228	0	0	\\
3.55e-09	0.228	0	0	\\
3.6e-09	0.228	0	0	\\
3.65e-09	0.228	0	0	\\
3.7e-09	0.228	0	0	\\
3.75e-09	0.228	0	0	\\
3.8e-09	0.228	0	0	\\
3.85e-09	0.228	0	0	\\
3.9e-09	0.228	0	0	\\
3.95e-09	0.228	0	0	\\
4e-09	0.228	0	0	\\
4.05e-09	0.228	0	0	\\
4.1e-09	0.228	0	0	\\
4.15e-09	0.228	0	0	\\
4.2e-09	0.228	0	0	\\
4.25e-09	0.228	0	0	\\
4.3e-09	0.228	0	0	\\
4.35e-09	0.228	0	0	\\
4.4e-09	0.228	0	0	\\
4.45e-09	0.228	0	0	\\
4.5e-09	0.228	0	0	\\
4.55e-09	0.228	0	0	\\
4.6e-09	0.228	-0.444444444444444	-0.444444444444444	\\
4.65e-09	0.228	-0.444444444444444	-0.444444444444444	\\
4.7e-09	0.228	-0.444444444444444	-0.444444444444444	\\
4.75e-09	0.228	-0.444444444444444	-0.444444444444444	\\
4.8e-09	0.228	-0.444444444444444	-0.444444444444444	\\
4.85e-09	0.228	-0.444444444444444	-0.444444444444444	\\
4.9e-09	0.228	-0.444444444444444	-0.444444444444444	\\
4.95e-09	0.228	-0.444444444444444	-0.444444444444444	\\
5e-09	0.228	-0.444444444444444	-0.444444444444444	\\
5e-09	0.228	-0.444444444444444	nan	\\
5e-09	0.228	-0.666666666666667	0.166666666666667	\\
5e-09	0.228	-0.666666666666667	nan	\\
0	0.24	-0.666666666666667	nan	\\
0	0.24	0	0.166666666666667	\\
0	0.24	0	0	\\
5e-11	0.24	0	0	\\
1e-10	0.24	0	0	\\
1.5e-10	0.24	0	0	\\
2e-10	0.24	0	0	\\
2.5e-10	0.24	1	1	\\
3e-10	0.24	1	1	\\
3.5e-10	0.24	1	1	\\
4e-10	0.24	1	1	\\
4.5e-10	0.24	1	1	\\
5e-10	0.24	1	1	\\
5.5e-10	0.24	1	1	\\
6e-10	0.24	1	1	\\
6.5e-10	0.24	1	1	\\
7e-10	0.24	1	1	\\
7.5e-10	0.24	0	0	\\
8e-10	0.24	0	0	\\
8.5e-10	0.24	0	0	\\
9e-10	0.24	0	0	\\
9.5e-10	0.24	0	0	\\
1e-09	0.24	0	0	\\
1.05e-09	0.24	0	0	\\
1.1e-09	0.24	0	0	\\
1.15e-09	0.24	0	0	\\
1.2e-09	0.24	0	0	\\
1.25e-09	0.24	0	0	\\
1.3e-09	0.24	0	0	\\
1.35e-09	0.24	0	0	\\
1.4e-09	0.24	0	0	\\
1.45e-09	0.24	0	0	\\
1.5e-09	0.24	0	0	\\
1.55e-09	0.24	0	0	\\
1.6e-09	0.24	0	0	\\
1.65e-09	0.24	0	0	\\
1.7e-09	0.24	0	0	\\
1.75e-09	0.24	0	0	\\
1.8e-09	0.24	0	0	\\
1.85e-09	0.24	0	0	\\
1.9e-09	0.24	0	0	\\
1.95e-09	0.24	0	0	\\
2e-09	0.24	0	0	\\
2.05e-09	0.24	0	0	\\
2.1e-09	0.24	0	0	\\
2.15e-09	0.24	0	0	\\
2.2e-09	0.24	-0.666666666666667	-0.666666666666667	\\
2.25e-09	0.24	-0.666666666666667	-0.666666666666667	\\
2.3e-09	0.24	-0.666666666666667	-0.666666666666667	\\
2.35e-09	0.24	-0.666666666666667	-0.666666666666667	\\
2.4e-09	0.24	-0.666666666666667	-0.666666666666667	\\
2.45e-09	0.24	-0.666666666666667	-0.666666666666667	\\
2.5e-09	0.24	-0.666666666666667	-0.666666666666667	\\
2.55e-09	0.24	-0.666666666666667	-0.666666666666667	\\
2.6e-09	0.24	-0.666666666666667	-0.666666666666667	\\
2.65e-09	0.24	0	0	\\
2.7e-09	0.24	0.666666666666667	0.666666666666667	\\
2.75e-09	0.24	0.666666666666667	0.666666666666667	\\
2.8e-09	0.24	0.666666666666667	0.666666666666667	\\
2.85e-09	0.24	0.666666666666667	0.666666666666667	\\
2.9e-09	0.24	0.666666666666667	0.666666666666667	\\
2.95e-09	0.24	0.666666666666667	0.666666666666667	\\
3e-09	0.24	0.666666666666667	0.666666666666667	\\
3.05e-09	0.24	0.666666666666667	0.666666666666667	\\
3.1e-09	0.24	0.666666666666667	0.666666666666667	\\
3.15e-09	0.24	0	0	\\
3.2e-09	0.24	0	0	\\
3.25e-09	0.24	0	0	\\
3.3e-09	0.24	0	0	\\
3.35e-09	0.24	0	0	\\
3.4e-09	0.24	0	0	\\
3.45e-09	0.24	0	0	\\
3.5e-09	0.24	0	0	\\
3.55e-09	0.24	0	0	\\
3.6e-09	0.24	0	0	\\
3.65e-09	0.24	0	0	\\
3.7e-09	0.24	0	0	\\
3.75e-09	0.24	0	0	\\
3.8e-09	0.24	0	0	\\
3.85e-09	0.24	0	0	\\
3.9e-09	0.24	0	0	\\
3.95e-09	0.24	0	0	\\
4e-09	0.24	0	0	\\
4.05e-09	0.24	0	0	\\
4.1e-09	0.24	0	0	\\
4.15e-09	0.24	0	0	\\
4.2e-09	0.24	0	0	\\
4.25e-09	0.24	0	0	\\
4.3e-09	0.24	0	0	\\
4.35e-09	0.24	0	0	\\
4.4e-09	0.24	0	0	\\
4.45e-09	0.24	0	0	\\
4.5e-09	0.24	0	0	\\
4.55e-09	0.24	0	0	\\
4.6e-09	0.24	-0.444444444444444	-0.444444444444444	\\
4.65e-09	0.24	-0.444444444444444	-0.444444444444444	\\
4.7e-09	0.24	-0.444444444444444	-0.444444444444444	\\
4.75e-09	0.24	-0.444444444444444	-0.444444444444444	\\
4.8e-09	0.24	-0.444444444444444	-0.444444444444444	\\
4.85e-09	0.24	-0.444444444444444	-0.444444444444444	\\
4.9e-09	0.24	-0.444444444444444	-0.444444444444444	\\
4.95e-09	0.24	-0.444444444444444	-0.444444444444444	\\
5e-09	0.24	-0.444444444444444	-0.444444444444444	\\
5e-09	0.24	-0.444444444444444	nan	\\
5e-09	0.24	-0.666666666666667	0.166666666666667	\\
5e-09	0.24	-0.666666666666667	nan	\\
0	0.252	-0.666666666666667	nan	\\
0	0.252	0	0.166666666666667	\\
0	0.252	0	0	\\
5e-11	0.252	0	0	\\
1e-10	0.252	0	0	\\
1.5e-10	0.252	0	0	\\
2e-10	0.252	0	0	\\
2.5e-10	0.252	0	0	\\
3e-10	0.252	1	1	\\
3.5e-10	0.252	1	1	\\
4e-10	0.252	1	1	\\
4.5e-10	0.252	1	1	\\
5e-10	0.252	1	1	\\
5.5e-10	0.252	1	1	\\
6e-10	0.252	1	1	\\
6.5e-10	0.252	1	1	\\
7e-10	0.252	1	1	\\
7.5e-10	0.252	1	1	\\
8e-10	0.252	0	0	\\
8.5e-10	0.252	0	0	\\
9e-10	0.252	0	0	\\
9.5e-10	0.252	0	0	\\
1e-09	0.252	0	0	\\
1.05e-09	0.252	0	0	\\
1.1e-09	0.252	0	0	\\
1.15e-09	0.252	0	0	\\
1.2e-09	0.252	0	0	\\
1.25e-09	0.252	0	0	\\
1.3e-09	0.252	0	0	\\
1.35e-09	0.252	0	0	\\
1.4e-09	0.252	0	0	\\
1.45e-09	0.252	0	0	\\
1.5e-09	0.252	0	0	\\
1.55e-09	0.252	0	0	\\
1.6e-09	0.252	0	0	\\
1.65e-09	0.252	0	0	\\
1.7e-09	0.252	0	0	\\
1.75e-09	0.252	0	0	\\
1.8e-09	0.252	0	0	\\
1.85e-09	0.252	0	0	\\
1.9e-09	0.252	0	0	\\
1.95e-09	0.252	0	0	\\
2e-09	0.252	0	0	\\
2.05e-09	0.252	0	0	\\
2.1e-09	0.252	0	0	\\
2.15e-09	0.252	-0.666666666666667	-0.666666666666667	\\
2.2e-09	0.252	-0.666666666666667	-0.666666666666667	\\
2.25e-09	0.252	-0.666666666666667	-0.666666666666667	\\
2.3e-09	0.252	-0.666666666666667	-0.666666666666667	\\
2.35e-09	0.252	-0.666666666666667	-0.666666666666667	\\
2.4e-09	0.252	-0.666666666666667	-0.666666666666667	\\
2.45e-09	0.252	-0.666666666666667	-0.666666666666667	\\
2.5e-09	0.252	-0.666666666666667	-0.666666666666667	\\
2.55e-09	0.252	-0.666666666666667	-0.666666666666667	\\
2.6e-09	0.252	-0.666666666666667	-0.666666666666667	\\
2.65e-09	0.252	0	0	\\
2.7e-09	0.252	0.666666666666667	0.666666666666667	\\
2.75e-09	0.252	0.666666666666667	0.666666666666667	\\
2.8e-09	0.252	0.666666666666667	0.666666666666667	\\
2.85e-09	0.252	0.666666666666667	0.666666666666667	\\
2.9e-09	0.252	0.666666666666667	0.666666666666667	\\
2.95e-09	0.252	0.666666666666667	0.666666666666667	\\
3e-09	0.252	0.666666666666667	0.666666666666667	\\
3.05e-09	0.252	0.666666666666667	0.666666666666667	\\
3.1e-09	0.252	0.666666666666667	0.666666666666667	\\
3.15e-09	0.252	0.666666666666667	0.666666666666667	\\
3.2e-09	0.252	0	0	\\
3.25e-09	0.252	0	0	\\
3.3e-09	0.252	0	0	\\
3.35e-09	0.252	0	0	\\
3.4e-09	0.252	0	0	\\
3.45e-09	0.252	0	0	\\
3.5e-09	0.252	0	0	\\
3.55e-09	0.252	0	0	\\
3.6e-09	0.252	0	0	\\
3.65e-09	0.252	0	0	\\
3.7e-09	0.252	0	0	\\
3.75e-09	0.252	0	0	\\
3.8e-09	0.252	0	0	\\
3.85e-09	0.252	0	0	\\
3.9e-09	0.252	0	0	\\
3.95e-09	0.252	0	0	\\
4e-09	0.252	0	0	\\
4.05e-09	0.252	0	0	\\
4.1e-09	0.252	0	0	\\
4.15e-09	0.252	0	0	\\
4.2e-09	0.252	0	0	\\
4.25e-09	0.252	0	0	\\
4.3e-09	0.252	0	0	\\
4.35e-09	0.252	0	0	\\
4.4e-09	0.252	0	0	\\
4.45e-09	0.252	0	0	\\
4.5e-09	0.252	0	0	\\
4.55e-09	0.252	-0.444444444444444	-0.444444444444444	\\
4.6e-09	0.252	-0.444444444444444	-0.444444444444444	\\
4.65e-09	0.252	-0.444444444444444	-0.444444444444444	\\
4.7e-09	0.252	-0.444444444444444	-0.444444444444444	\\
4.75e-09	0.252	-0.444444444444444	-0.444444444444444	\\
4.8e-09	0.252	-0.444444444444444	-0.444444444444444	\\
4.85e-09	0.252	-0.444444444444444	-0.444444444444444	\\
4.9e-09	0.252	-0.444444444444444	-0.444444444444444	\\
4.95e-09	0.252	-0.444444444444444	-0.444444444444444	\\
5e-09	0.252	-0.444444444444444	-0.444444444444444	\\
5e-09	0.252	-0.444444444444444	nan	\\
5e-09	0.252	-0.666666666666667	0.166666666666667	\\
5e-09	0.252	-0.666666666666667	nan	\\
0	0.264	-0.666666666666667	nan	\\
0	0.264	0	0.166666666666667	\\
0	0.264	0	0	\\
5e-11	0.264	0	0	\\
1e-10	0.264	0	0	\\
1.5e-10	0.264	0	0	\\
2e-10	0.264	0	0	\\
2.5e-10	0.264	0	0	\\
3e-10	0.264	1	1	\\
3.5e-10	0.264	1	1	\\
4e-10	0.264	1	1	\\
4.5e-10	0.264	1	1	\\
5e-10	0.264	1	1	\\
5.5e-10	0.264	1	1	\\
6e-10	0.264	1	1	\\
6.5e-10	0.264	1	1	\\
7e-10	0.264	1	1	\\
7.5e-10	0.264	1	1	\\
8e-10	0.264	0	0	\\
8.5e-10	0.264	0	0	\\
9e-10	0.264	0	0	\\
9.5e-10	0.264	0	0	\\
1e-09	0.264	0	0	\\
1.05e-09	0.264	0	0	\\
1.1e-09	0.264	0	0	\\
1.15e-09	0.264	0	0	\\
1.2e-09	0.264	0	0	\\
1.25e-09	0.264	0	0	\\
1.3e-09	0.264	0	0	\\
1.35e-09	0.264	0	0	\\
1.4e-09	0.264	0	0	\\
1.45e-09	0.264	0	0	\\
1.5e-09	0.264	0	0	\\
1.55e-09	0.264	0	0	\\
1.6e-09	0.264	0	0	\\
1.65e-09	0.264	0	0	\\
1.7e-09	0.264	0	0	\\
1.75e-09	0.264	0	0	\\
1.8e-09	0.264	0	0	\\
1.85e-09	0.264	0	0	\\
1.9e-09	0.264	0	0	\\
1.95e-09	0.264	0	0	\\
2e-09	0.264	0	0	\\
2.05e-09	0.264	0	0	\\
2.1e-09	0.264	0	0	\\
2.15e-09	0.264	-0.666666666666667	-0.666666666666667	\\
2.2e-09	0.264	-0.666666666666667	-0.666666666666667	\\
2.25e-09	0.264	-0.666666666666667	-0.666666666666667	\\
2.3e-09	0.264	-0.666666666666667	-0.666666666666667	\\
2.35e-09	0.264	-0.666666666666667	-0.666666666666667	\\
2.4e-09	0.264	-0.666666666666667	-0.666666666666667	\\
2.45e-09	0.264	-0.666666666666667	-0.666666666666667	\\
2.5e-09	0.264	-0.666666666666667	-0.666666666666667	\\
2.55e-09	0.264	-0.666666666666667	-0.666666666666667	\\
2.6e-09	0.264	-0.666666666666667	-0.666666666666667	\\
2.65e-09	0.264	0	0	\\
2.7e-09	0.264	0.666666666666667	0.666666666666667	\\
2.75e-09	0.264	0.666666666666667	0.666666666666667	\\
2.8e-09	0.264	0.666666666666667	0.666666666666667	\\
2.85e-09	0.264	0.666666666666667	0.666666666666667	\\
2.9e-09	0.264	0.666666666666667	0.666666666666667	\\
2.95e-09	0.264	0.666666666666667	0.666666666666667	\\
3e-09	0.264	0.666666666666667	0.666666666666667	\\
3.05e-09	0.264	0.666666666666667	0.666666666666667	\\
3.1e-09	0.264	0.666666666666667	0.666666666666667	\\
3.15e-09	0.264	0.666666666666667	0.666666666666667	\\
3.2e-09	0.264	0	0	\\
3.25e-09	0.264	0	0	\\
3.3e-09	0.264	0	0	\\
3.35e-09	0.264	0	0	\\
3.4e-09	0.264	0	0	\\
3.45e-09	0.264	0	0	\\
3.5e-09	0.264	0	0	\\
3.55e-09	0.264	0	0	\\
3.6e-09	0.264	0	0	\\
3.65e-09	0.264	0	0	\\
3.7e-09	0.264	0	0	\\
3.75e-09	0.264	0	0	\\
3.8e-09	0.264	0	0	\\
3.85e-09	0.264	0	0	\\
3.9e-09	0.264	0	0	\\
3.95e-09	0.264	0	0	\\
4e-09	0.264	0	0	\\
4.05e-09	0.264	0	0	\\
4.1e-09	0.264	0	0	\\
4.15e-09	0.264	0	0	\\
4.2e-09	0.264	0	0	\\
4.25e-09	0.264	0	0	\\
4.3e-09	0.264	0	0	\\
4.35e-09	0.264	0	0	\\
4.4e-09	0.264	0	0	\\
4.45e-09	0.264	0	0	\\
4.5e-09	0.264	0	0	\\
4.55e-09	0.264	-0.444444444444444	-0.444444444444444	\\
4.6e-09	0.264	-0.444444444444444	-0.444444444444444	\\
4.65e-09	0.264	-0.444444444444444	-0.444444444444444	\\
4.7e-09	0.264	-0.444444444444444	-0.444444444444444	\\
4.75e-09	0.264	-0.444444444444444	-0.444444444444444	\\
4.8e-09	0.264	-0.444444444444444	-0.444444444444444	\\
4.85e-09	0.264	-0.444444444444444	-0.444444444444444	\\
4.9e-09	0.264	-0.444444444444444	-0.444444444444444	\\
4.95e-09	0.264	-0.444444444444444	-0.444444444444444	\\
5e-09	0.264	-0.444444444444444	-0.444444444444444	\\
5e-09	0.264	-0.444444444444444	nan	\\
5e-09	0.264	-0.666666666666667	0.166666666666667	\\
5e-09	0.264	-0.666666666666667	nan	\\
0	0.276	-0.666666666666667	nan	\\
0	0.276	0	0.166666666666667	\\
0	0.276	0	0	\\
5e-11	0.276	0	0	\\
1e-10	0.276	0	0	\\
1.5e-10	0.276	0	0	\\
2e-10	0.276	0	0	\\
2.5e-10	0.276	0	0	\\
3e-10	0.276	1	1	\\
3.5e-10	0.276	1	1	\\
4e-10	0.276	1	1	\\
4.5e-10	0.276	1	1	\\
5e-10	0.276	1	1	\\
5.5e-10	0.276	1	1	\\
6e-10	0.276	1	1	\\
6.5e-10	0.276	1	1	\\
7e-10	0.276	1	1	\\
7.5e-10	0.276	1	1	\\
8e-10	0.276	0	0	\\
8.5e-10	0.276	0	0	\\
9e-10	0.276	0	0	\\
9.5e-10	0.276	0	0	\\
1e-09	0.276	0	0	\\
1.05e-09	0.276	0	0	\\
1.1e-09	0.276	0	0	\\
1.15e-09	0.276	0	0	\\
1.2e-09	0.276	0	0	\\
1.25e-09	0.276	0	0	\\
1.3e-09	0.276	0	0	\\
1.35e-09	0.276	0	0	\\
1.4e-09	0.276	0	0	\\
1.45e-09	0.276	0	0	\\
1.5e-09	0.276	0	0	\\
1.55e-09	0.276	0	0	\\
1.6e-09	0.276	0	0	\\
1.65e-09	0.276	0	0	\\
1.7e-09	0.276	0	0	\\
1.75e-09	0.276	0	0	\\
1.8e-09	0.276	0	0	\\
1.85e-09	0.276	0	0	\\
1.9e-09	0.276	0	0	\\
1.95e-09	0.276	0	0	\\
2e-09	0.276	0	0	\\
2.05e-09	0.276	0	0	\\
2.1e-09	0.276	0	0	\\
2.15e-09	0.276	-0.666666666666667	-0.666666666666667	\\
2.2e-09	0.276	-0.666666666666667	-0.666666666666667	\\
2.25e-09	0.276	-0.666666666666667	-0.666666666666667	\\
2.3e-09	0.276	-0.666666666666667	-0.666666666666667	\\
2.35e-09	0.276	-0.666666666666667	-0.666666666666667	\\
2.4e-09	0.276	-0.666666666666667	-0.666666666666667	\\
2.45e-09	0.276	-0.666666666666667	-0.666666666666667	\\
2.5e-09	0.276	-0.666666666666667	-0.666666666666667	\\
2.55e-09	0.276	-0.666666666666667	-0.666666666666667	\\
2.6e-09	0.276	-0.666666666666667	-0.666666666666667	\\
2.65e-09	0.276	0	0	\\
2.7e-09	0.276	0.666666666666667	0.666666666666667	\\
2.75e-09	0.276	0.666666666666667	0.666666666666667	\\
2.8e-09	0.276	0.666666666666667	0.666666666666667	\\
2.85e-09	0.276	0.666666666666667	0.666666666666667	\\
2.9e-09	0.276	0.666666666666667	0.666666666666667	\\
2.95e-09	0.276	0.666666666666667	0.666666666666667	\\
3e-09	0.276	0.666666666666667	0.666666666666667	\\
3.05e-09	0.276	0.666666666666667	0.666666666666667	\\
3.1e-09	0.276	0.666666666666667	0.666666666666667	\\
3.15e-09	0.276	0.666666666666667	0.666666666666667	\\
3.2e-09	0.276	0	0	\\
3.25e-09	0.276	0	0	\\
3.3e-09	0.276	0	0	\\
3.35e-09	0.276	0	0	\\
3.4e-09	0.276	0	0	\\
3.45e-09	0.276	0	0	\\
3.5e-09	0.276	0	0	\\
3.55e-09	0.276	0	0	\\
3.6e-09	0.276	0	0	\\
3.65e-09	0.276	0	0	\\
3.7e-09	0.276	0	0	\\
3.75e-09	0.276	0	0	\\
3.8e-09	0.276	0	0	\\
3.85e-09	0.276	0	0	\\
3.9e-09	0.276	0	0	\\
3.95e-09	0.276	0	0	\\
4e-09	0.276	0	0	\\
4.05e-09	0.276	0	0	\\
4.1e-09	0.276	0	0	\\
4.15e-09	0.276	0	0	\\
4.2e-09	0.276	0	0	\\
4.25e-09	0.276	0	0	\\
4.3e-09	0.276	0	0	\\
4.35e-09	0.276	0	0	\\
4.4e-09	0.276	0	0	\\
4.45e-09	0.276	0	0	\\
4.5e-09	0.276	0	0	\\
4.55e-09	0.276	-0.444444444444444	-0.444444444444444	\\
4.6e-09	0.276	-0.444444444444444	-0.444444444444444	\\
4.65e-09	0.276	-0.444444444444444	-0.444444444444444	\\
4.7e-09	0.276	-0.444444444444444	-0.444444444444444	\\
4.75e-09	0.276	-0.444444444444444	-0.444444444444444	\\
4.8e-09	0.276	-0.444444444444444	-0.444444444444444	\\
4.85e-09	0.276	-0.444444444444444	-0.444444444444444	\\
4.9e-09	0.276	-0.444444444444444	-0.444444444444444	\\
4.95e-09	0.276	-0.444444444444444	-0.444444444444444	\\
5e-09	0.276	-0.444444444444444	-0.444444444444444	\\
5e-09	0.276	-0.444444444444444	nan	\\
5e-09	0.276	-0.666666666666667	0.166666666666667	\\
5e-09	0.276	-0.666666666666667	nan	\\
0	0.288	-0.666666666666667	nan	\\
0	0.288	0	0.166666666666667	\\
0	0.288	0	0	\\
5e-11	0.288	0	0	\\
1e-10	0.288	0	0	\\
1.5e-10	0.288	0	0	\\
2e-10	0.288	0	0	\\
2.5e-10	0.288	0	0	\\
3e-10	0.288	1	1	\\
3.5e-10	0.288	1	1	\\
4e-10	0.288	1	1	\\
4.5e-10	0.288	1	1	\\
5e-10	0.288	1	1	\\
5.5e-10	0.288	1	1	\\
6e-10	0.288	1	1	\\
6.5e-10	0.288	1	1	\\
7e-10	0.288	1	1	\\
7.5e-10	0.288	1	1	\\
8e-10	0.288	0	0	\\
8.5e-10	0.288	0	0	\\
9e-10	0.288	0	0	\\
9.5e-10	0.288	0	0	\\
1e-09	0.288	0	0	\\
1.05e-09	0.288	0	0	\\
1.1e-09	0.288	0	0	\\
1.15e-09	0.288	0	0	\\
1.2e-09	0.288	0	0	\\
1.25e-09	0.288	0	0	\\
1.3e-09	0.288	0	0	\\
1.35e-09	0.288	0	0	\\
1.4e-09	0.288	0	0	\\
1.45e-09	0.288	0	0	\\
1.5e-09	0.288	0	0	\\
1.55e-09	0.288	0	0	\\
1.6e-09	0.288	0	0	\\
1.65e-09	0.288	0	0	\\
1.7e-09	0.288	0	0	\\
1.75e-09	0.288	0	0	\\
1.8e-09	0.288	0	0	\\
1.85e-09	0.288	0	0	\\
1.9e-09	0.288	0	0	\\
1.95e-09	0.288	0	0	\\
2e-09	0.288	0	0	\\
2.05e-09	0.288	0	0	\\
2.1e-09	0.288	0	0	\\
2.15e-09	0.288	-0.666666666666667	-0.666666666666667	\\
2.2e-09	0.288	-0.666666666666667	-0.666666666666667	\\
2.25e-09	0.288	-0.666666666666667	-0.666666666666667	\\
2.3e-09	0.288	-0.666666666666667	-0.666666666666667	\\
2.35e-09	0.288	-0.666666666666667	-0.666666666666667	\\
2.4e-09	0.288	-0.666666666666667	-0.666666666666667	\\
2.45e-09	0.288	-0.666666666666667	-0.666666666666667	\\
2.5e-09	0.288	-0.666666666666667	-0.666666666666667	\\
2.55e-09	0.288	-0.666666666666667	-0.666666666666667	\\
2.6e-09	0.288	-0.666666666666667	-0.666666666666667	\\
2.65e-09	0.288	0	0	\\
2.7e-09	0.288	0.666666666666667	0.666666666666667	\\
2.75e-09	0.288	0.666666666666667	0.666666666666667	\\
2.8e-09	0.288	0.666666666666667	0.666666666666667	\\
2.85e-09	0.288	0.666666666666667	0.666666666666667	\\
2.9e-09	0.288	0.666666666666667	0.666666666666667	\\
2.95e-09	0.288	0.666666666666667	0.666666666666667	\\
3e-09	0.288	0.666666666666667	0.666666666666667	\\
3.05e-09	0.288	0.666666666666667	0.666666666666667	\\
3.1e-09	0.288	0.666666666666667	0.666666666666667	\\
3.15e-09	0.288	0.666666666666667	0.666666666666667	\\
3.2e-09	0.288	0	0	\\
3.25e-09	0.288	0	0	\\
3.3e-09	0.288	0	0	\\
3.35e-09	0.288	0	0	\\
3.4e-09	0.288	0	0	\\
3.45e-09	0.288	0	0	\\
3.5e-09	0.288	0	0	\\
3.55e-09	0.288	0	0	\\
3.6e-09	0.288	0	0	\\
3.65e-09	0.288	0	0	\\
3.7e-09	0.288	0	0	\\
3.75e-09	0.288	0	0	\\
3.8e-09	0.288	0	0	\\
3.85e-09	0.288	0	0	\\
3.9e-09	0.288	0	0	\\
3.95e-09	0.288	0	0	\\
4e-09	0.288	0	0	\\
4.05e-09	0.288	0	0	\\
4.1e-09	0.288	0	0	\\
4.15e-09	0.288	0	0	\\
4.2e-09	0.288	0	0	\\
4.25e-09	0.288	0	0	\\
4.3e-09	0.288	0	0	\\
4.35e-09	0.288	0	0	\\
4.4e-09	0.288	0	0	\\
4.45e-09	0.288	0	0	\\
4.5e-09	0.288	0	0	\\
4.55e-09	0.288	-0.444444444444444	-0.444444444444444	\\
4.6e-09	0.288	-0.444444444444444	-0.444444444444444	\\
4.65e-09	0.288	-0.444444444444444	-0.444444444444444	\\
4.7e-09	0.288	-0.444444444444444	-0.444444444444444	\\
4.75e-09	0.288	-0.444444444444444	-0.444444444444444	\\
4.8e-09	0.288	-0.444444444444444	-0.444444444444444	\\
4.85e-09	0.288	-0.444444444444444	-0.444444444444444	\\
4.9e-09	0.288	-0.444444444444444	-0.444444444444444	\\
4.95e-09	0.288	-0.444444444444444	-0.444444444444444	\\
5e-09	0.288	-0.444444444444444	-0.444444444444444	\\
5e-09	0.288	-0.444444444444444	nan	\\
5e-09	0.288	-0.666666666666667	0.166666666666667	\\
5e-09	0.288	-0.666666666666667	nan	\\
0	0.3	-0.666666666666667	nan	\\
0	0.3	0	0.166666666666667	\\
0	0.3	0	0	\\
5e-11	0.3	0	0	\\
1e-10	0.3	0	0	\\
1.5e-10	0.3	0	0	\\
2e-10	0.3	0	0	\\
2.5e-10	0.3	0	0	\\
3e-10	0.3	1	1	\\
3.5e-10	0.3	1	1	\\
4e-10	0.3	1	1	\\
4.5e-10	0.3	1	1	\\
5e-10	0.3	1	1	\\
5.5e-10	0.3	1	1	\\
6e-10	0.3	1	1	\\
6.5e-10	0.3	1	1	\\
7e-10	0.3	1	1	\\
7.5e-10	0.3	1	1	\\
8e-10	0.3	0	0	\\
8.5e-10	0.3	0	0	\\
9e-10	0.3	0	0	\\
9.5e-10	0.3	0	0	\\
1e-09	0.3	0	0	\\
1.05e-09	0.3	0	0	\\
1.1e-09	0.3	0	0	\\
1.15e-09	0.3	0	0	\\
1.2e-09	0.3	0	0	\\
1.25e-09	0.3	0	0	\\
1.3e-09	0.3	0	0	\\
1.35e-09	0.3	0	0	\\
1.4e-09	0.3	0	0	\\
1.45e-09	0.3	0	0	\\
1.5e-09	0.3	0	0	\\
1.55e-09	0.3	0	0	\\
1.6e-09	0.3	0	0	\\
1.65e-09	0.3	0	0	\\
1.7e-09	0.3	0	0	\\
1.75e-09	0.3	0	0	\\
1.8e-09	0.3	0	0	\\
1.85e-09	0.3	0	0	\\
1.9e-09	0.3	0	0	\\
1.95e-09	0.3	0	0	\\
2e-09	0.3	0	0	\\
2.05e-09	0.3	0	0	\\
2.1e-09	0.3	-0.666666666666667	-0.666666666666667	\\
2.15e-09	0.3	-0.666666666666667	-0.666666666666667	\\
2.2e-09	0.3	-0.666666666666667	-0.666666666666667	\\
2.25e-09	0.3	-0.666666666666667	-0.666666666666667	\\
2.3e-09	0.3	-0.666666666666667	-0.666666666666667	\\
2.35e-09	0.3	-0.666666666666667	-0.666666666666667	\\
2.4e-09	0.3	-0.666666666666667	-0.666666666666667	\\
2.45e-09	0.3	-0.666666666666667	-0.666666666666667	\\
2.5e-09	0.3	-0.666666666666667	-0.666666666666667	\\
2.55e-09	0.3	-0.666666666666667	-0.666666666666667	\\
2.6e-09	0.3	0	0	\\
2.65e-09	0.3	0	0	\\
2.7e-09	0.3	0.666666666666667	0.666666666666667	\\
2.75e-09	0.3	0.666666666666667	0.666666666666667	\\
2.8e-09	0.3	0.666666666666667	0.666666666666667	\\
2.85e-09	0.3	0.666666666666667	0.666666666666667	\\
2.9e-09	0.3	0.666666666666667	0.666666666666667	\\
2.95e-09	0.3	0.666666666666667	0.666666666666667	\\
3e-09	0.3	0.666666666666667	0.666666666666667	\\
3.05e-09	0.3	0.666666666666667	0.666666666666667	\\
3.1e-09	0.3	0.666666666666667	0.666666666666667	\\
3.15e-09	0.3	0.666666666666667	0.666666666666667	\\
3.2e-09	0.3	0	0	\\
3.25e-09	0.3	0	0	\\
3.3e-09	0.3	0	0	\\
3.35e-09	0.3	0	0	\\
3.4e-09	0.3	0	0	\\
3.45e-09	0.3	0	0	\\
3.5e-09	0.3	0	0	\\
3.55e-09	0.3	0	0	\\
3.6e-09	0.3	0	0	\\
3.65e-09	0.3	0	0	\\
3.7e-09	0.3	0	0	\\
3.75e-09	0.3	0	0	\\
3.8e-09	0.3	0	0	\\
3.85e-09	0.3	0	0	\\
3.9e-09	0.3	0	0	\\
3.95e-09	0.3	0	0	\\
4e-09	0.3	0	0	\\
4.05e-09	0.3	0	0	\\
4.1e-09	0.3	0	0	\\
4.15e-09	0.3	0	0	\\
4.2e-09	0.3	0	0	\\
4.25e-09	0.3	0	0	\\
4.3e-09	0.3	0	0	\\
4.35e-09	0.3	0	0	\\
4.4e-09	0.3	0	0	\\
4.45e-09	0.3	0	0	\\
4.5e-09	0.3	-0.444444444444444	-0.444444444444444	\\
4.55e-09	0.3	-0.444444444444444	-0.444444444444444	\\
4.6e-09	0.3	-0.444444444444444	-0.444444444444444	\\
4.65e-09	0.3	-0.444444444444444	-0.444444444444444	\\
4.7e-09	0.3	-0.444444444444444	-0.444444444444444	\\
4.75e-09	0.3	-0.444444444444444	-0.444444444444444	\\
4.8e-09	0.3	-0.444444444444444	-0.444444444444444	\\
4.85e-09	0.3	-0.444444444444444	-0.444444444444444	\\
4.9e-09	0.3	-0.444444444444444	-0.444444444444444	\\
4.95e-09	0.3	-0.444444444444444	-0.444444444444444	\\
5e-09	0.3	0	0	\\
5e-09	0.3	0	nan	\\
5e-09	0.3	-0.666666666666667	0.166666666666667	\\
5e-09	0.3	-0.666666666666667	nan	\\
0	0.312	-0.666666666666667	nan	\\
0	0.312	0	0.166666666666667	\\
0	0.312	0	0	\\
5e-11	0.312	0	0	\\
1e-10	0.312	0	0	\\
1.5e-10	0.312	0	0	\\
2e-10	0.312	0	0	\\
2.5e-10	0.312	0	0	\\
3e-10	0.312	0	0	\\
3.5e-10	0.312	1	1	\\
4e-10	0.312	1	1	\\
4.5e-10	0.312	1	1	\\
5e-10	0.312	1	1	\\
5.5e-10	0.312	1	1	\\
6e-10	0.312	1	1	\\
6.5e-10	0.312	1	1	\\
7e-10	0.312	1	1	\\
7.5e-10	0.312	1	1	\\
8e-10	0.312	1	1	\\
8.5e-10	0.312	0	0	\\
9e-10	0.312	0	0	\\
9.5e-10	0.312	0	0	\\
1e-09	0.312	0	0	\\
1.05e-09	0.312	0	0	\\
1.1e-09	0.312	0	0	\\
1.15e-09	0.312	0	0	\\
1.2e-09	0.312	0	0	\\
1.25e-09	0.312	0	0	\\
1.3e-09	0.312	0	0	\\
1.35e-09	0.312	0	0	\\
1.4e-09	0.312	0	0	\\
1.45e-09	0.312	0	0	\\
1.5e-09	0.312	0	0	\\
1.55e-09	0.312	0	0	\\
1.6e-09	0.312	0	0	\\
1.65e-09	0.312	0	0	\\
1.7e-09	0.312	0	0	\\
1.75e-09	0.312	0	0	\\
1.8e-09	0.312	0	0	\\
1.85e-09	0.312	0	0	\\
1.9e-09	0.312	0	0	\\
1.95e-09	0.312	0	0	\\
2e-09	0.312	0	0	\\
2.05e-09	0.312	0	0	\\
2.1e-09	0.312	-0.666666666666667	-0.666666666666667	\\
2.15e-09	0.312	-0.666666666666667	-0.666666666666667	\\
2.2e-09	0.312	-0.666666666666667	-0.666666666666667	\\
2.25e-09	0.312	-0.666666666666667	-0.666666666666667	\\
2.3e-09	0.312	-0.666666666666667	-0.666666666666667	\\
2.35e-09	0.312	-0.666666666666667	-0.666666666666667	\\
2.4e-09	0.312	-0.666666666666667	-0.666666666666667	\\
2.45e-09	0.312	-0.666666666666667	-0.666666666666667	\\
2.5e-09	0.312	-0.666666666666667	-0.666666666666667	\\
2.55e-09	0.312	-0.666666666666667	-0.666666666666667	\\
2.6e-09	0.312	0	0	\\
2.65e-09	0.312	0	0	\\
2.7e-09	0.312	0	0	\\
2.75e-09	0.312	0.666666666666667	0.666666666666667	\\
2.8e-09	0.312	0.666666666666667	0.666666666666667	\\
2.85e-09	0.312	0.666666666666667	0.666666666666667	\\
2.9e-09	0.312	0.666666666666667	0.666666666666667	\\
2.95e-09	0.312	0.666666666666667	0.666666666666667	\\
3e-09	0.312	0.666666666666667	0.666666666666667	\\
3.05e-09	0.312	0.666666666666667	0.666666666666667	\\
3.1e-09	0.312	0.666666666666667	0.666666666666667	\\
3.15e-09	0.312	0.666666666666667	0.666666666666667	\\
3.2e-09	0.312	0.666666666666667	0.666666666666667	\\
3.25e-09	0.312	0	0	\\
3.3e-09	0.312	0	0	\\
3.35e-09	0.312	0	0	\\
3.4e-09	0.312	0	0	\\
3.45e-09	0.312	0	0	\\
3.5e-09	0.312	0	0	\\
3.55e-09	0.312	0	0	\\
3.6e-09	0.312	0	0	\\
3.65e-09	0.312	0	0	\\
3.7e-09	0.312	0	0	\\
3.75e-09	0.312	0	0	\\
3.8e-09	0.312	0	0	\\
3.85e-09	0.312	0	0	\\
3.9e-09	0.312	0	0	\\
3.95e-09	0.312	0	0	\\
4e-09	0.312	0	0	\\
4.05e-09	0.312	0	0	\\
4.1e-09	0.312	0	0	\\
4.15e-09	0.312	0	0	\\
4.2e-09	0.312	0	0	\\
4.25e-09	0.312	0	0	\\
4.3e-09	0.312	0	0	\\
4.35e-09	0.312	0	0	\\
4.4e-09	0.312	0	0	\\
4.45e-09	0.312	0	0	\\
4.5e-09	0.312	-0.444444444444444	-0.444444444444444	\\
4.55e-09	0.312	-0.444444444444444	-0.444444444444444	\\
4.6e-09	0.312	-0.444444444444444	-0.444444444444444	\\
4.65e-09	0.312	-0.444444444444444	-0.444444444444444	\\
4.7e-09	0.312	-0.444444444444444	-0.444444444444444	\\
4.75e-09	0.312	-0.444444444444444	-0.444444444444444	\\
4.8e-09	0.312	-0.444444444444444	-0.444444444444444	\\
4.85e-09	0.312	-0.444444444444444	-0.444444444444444	\\
4.9e-09	0.312	-0.444444444444444	-0.444444444444444	\\
4.95e-09	0.312	-0.444444444444444	-0.444444444444444	\\
5e-09	0.312	0	0	\\
5e-09	0.312	0	nan	\\
5e-09	0.312	-0.666666666666667	0.166666666666667	\\
5e-09	0.312	-0.666666666666667	nan	\\
0	0.324	-0.666666666666667	nan	\\
0	0.324	0	0.166666666666667	\\
0	0.324	0	0	\\
5e-11	0.324	0	0	\\
1e-10	0.324	0	0	\\
1.5e-10	0.324	0	0	\\
2e-10	0.324	0	0	\\
2.5e-10	0.324	0	0	\\
3e-10	0.324	0	0	\\
3.5e-10	0.324	1	1	\\
4e-10	0.324	1	1	\\
4.5e-10	0.324	1	1	\\
5e-10	0.324	1	1	\\
5.5e-10	0.324	1	1	\\
6e-10	0.324	1	1	\\
6.5e-10	0.324	1	1	\\
7e-10	0.324	1	1	\\
7.5e-10	0.324	1	1	\\
8e-10	0.324	1	1	\\
8.5e-10	0.324	0	0	\\
9e-10	0.324	0	0	\\
9.5e-10	0.324	0	0	\\
1e-09	0.324	0	0	\\
1.05e-09	0.324	0	0	\\
1.1e-09	0.324	0	0	\\
1.15e-09	0.324	0	0	\\
1.2e-09	0.324	0	0	\\
1.25e-09	0.324	0	0	\\
1.3e-09	0.324	0	0	\\
1.35e-09	0.324	0	0	\\
1.4e-09	0.324	0	0	\\
1.45e-09	0.324	0	0	\\
1.5e-09	0.324	0	0	\\
1.55e-09	0.324	0	0	\\
1.6e-09	0.324	0	0	\\
1.65e-09	0.324	0	0	\\
1.7e-09	0.324	0	0	\\
1.75e-09	0.324	0	0	\\
1.8e-09	0.324	0	0	\\
1.85e-09	0.324	0	0	\\
1.9e-09	0.324	0	0	\\
1.95e-09	0.324	0	0	\\
2e-09	0.324	0	0	\\
2.05e-09	0.324	0	0	\\
2.1e-09	0.324	-0.666666666666667	-0.666666666666667	\\
2.15e-09	0.324	-0.666666666666667	-0.666666666666667	\\
2.2e-09	0.324	-0.666666666666667	-0.666666666666667	\\
2.25e-09	0.324	-0.666666666666667	-0.666666666666667	\\
2.3e-09	0.324	-0.666666666666667	-0.666666666666667	\\
2.35e-09	0.324	-0.666666666666667	-0.666666666666667	\\
2.4e-09	0.324	-0.666666666666667	-0.666666666666667	\\
2.45e-09	0.324	-0.666666666666667	-0.666666666666667	\\
2.5e-09	0.324	-0.666666666666667	-0.666666666666667	\\
2.55e-09	0.324	-0.666666666666667	-0.666666666666667	\\
2.6e-09	0.324	0	0	\\
2.65e-09	0.324	0	0	\\
2.7e-09	0.324	0	0	\\
2.75e-09	0.324	0.666666666666667	0.666666666666667	\\
2.8e-09	0.324	0.666666666666667	0.666666666666667	\\
2.85e-09	0.324	0.666666666666667	0.666666666666667	\\
2.9e-09	0.324	0.666666666666667	0.666666666666667	\\
2.95e-09	0.324	0.666666666666667	0.666666666666667	\\
3e-09	0.324	0.666666666666667	0.666666666666667	\\
3.05e-09	0.324	0.666666666666667	0.666666666666667	\\
3.1e-09	0.324	0.666666666666667	0.666666666666667	\\
3.15e-09	0.324	0.666666666666667	0.666666666666667	\\
3.2e-09	0.324	0.666666666666667	0.666666666666667	\\
3.25e-09	0.324	0	0	\\
3.3e-09	0.324	0	0	\\
3.35e-09	0.324	0	0	\\
3.4e-09	0.324	0	0	\\
3.45e-09	0.324	0	0	\\
3.5e-09	0.324	0	0	\\
3.55e-09	0.324	0	0	\\
3.6e-09	0.324	0	0	\\
3.65e-09	0.324	0	0	\\
3.7e-09	0.324	0	0	\\
3.75e-09	0.324	0	0	\\
3.8e-09	0.324	0	0	\\
3.85e-09	0.324	0	0	\\
3.9e-09	0.324	0	0	\\
3.95e-09	0.324	0	0	\\
4e-09	0.324	0	0	\\
4.05e-09	0.324	0	0	\\
4.1e-09	0.324	0	0	\\
4.15e-09	0.324	0	0	\\
4.2e-09	0.324	0	0	\\
4.25e-09	0.324	0	0	\\
4.3e-09	0.324	0	0	\\
4.35e-09	0.324	0	0	\\
4.4e-09	0.324	0	0	\\
4.45e-09	0.324	0	0	\\
4.5e-09	0.324	-0.444444444444444	-0.444444444444444	\\
4.55e-09	0.324	-0.444444444444444	-0.444444444444444	\\
4.6e-09	0.324	-0.444444444444444	-0.444444444444444	\\
4.65e-09	0.324	-0.444444444444444	-0.444444444444444	\\
4.7e-09	0.324	-0.444444444444444	-0.444444444444444	\\
4.75e-09	0.324	-0.444444444444444	-0.444444444444444	\\
4.8e-09	0.324	-0.444444444444444	-0.444444444444444	\\
4.85e-09	0.324	-0.444444444444444	-0.444444444444444	\\
4.9e-09	0.324	-0.444444444444444	-0.444444444444444	\\
4.95e-09	0.324	-0.444444444444444	-0.444444444444444	\\
5e-09	0.324	0	0	\\
5e-09	0.324	0	nan	\\
5e-09	0.324	-0.666666666666667	0.166666666666667	\\
5e-09	0.324	-0.666666666666667	nan	\\
0	0.336	-0.666666666666667	nan	\\
0	0.336	0	0.166666666666667	\\
0	0.336	0	0	\\
5e-11	0.336	0	0	\\
1e-10	0.336	0	0	\\
1.5e-10	0.336	0	0	\\
2e-10	0.336	0	0	\\
2.5e-10	0.336	0	0	\\
3e-10	0.336	0	0	\\
3.5e-10	0.336	1	1	\\
4e-10	0.336	1	1	\\
4.5e-10	0.336	1	1	\\
5e-10	0.336	1	1	\\
5.5e-10	0.336	1	1	\\
6e-10	0.336	1	1	\\
6.5e-10	0.336	1	1	\\
7e-10	0.336	1	1	\\
7.5e-10	0.336	1	1	\\
8e-10	0.336	1	1	\\
8.5e-10	0.336	0	0	\\
9e-10	0.336	0	0	\\
9.5e-10	0.336	0	0	\\
1e-09	0.336	0	0	\\
1.05e-09	0.336	0	0	\\
1.1e-09	0.336	0	0	\\
1.15e-09	0.336	0	0	\\
1.2e-09	0.336	0	0	\\
1.25e-09	0.336	0	0	\\
1.3e-09	0.336	0	0	\\
1.35e-09	0.336	0	0	\\
1.4e-09	0.336	0	0	\\
1.45e-09	0.336	0	0	\\
1.5e-09	0.336	0	0	\\
1.55e-09	0.336	0	0	\\
1.6e-09	0.336	0	0	\\
1.65e-09	0.336	0	0	\\
1.7e-09	0.336	0	0	\\
1.75e-09	0.336	0	0	\\
1.8e-09	0.336	0	0	\\
1.85e-09	0.336	0	0	\\
1.9e-09	0.336	0	0	\\
1.95e-09	0.336	0	0	\\
2e-09	0.336	0	0	\\
2.05e-09	0.336	0	0	\\
2.1e-09	0.336	-0.666666666666667	-0.666666666666667	\\
2.15e-09	0.336	-0.666666666666667	-0.666666666666667	\\
2.2e-09	0.336	-0.666666666666667	-0.666666666666667	\\
2.25e-09	0.336	-0.666666666666667	-0.666666666666667	\\
2.3e-09	0.336	-0.666666666666667	-0.666666666666667	\\
2.35e-09	0.336	-0.666666666666667	-0.666666666666667	\\
2.4e-09	0.336	-0.666666666666667	-0.666666666666667	\\
2.45e-09	0.336	-0.666666666666667	-0.666666666666667	\\
2.5e-09	0.336	-0.666666666666667	-0.666666666666667	\\
2.55e-09	0.336	-0.666666666666667	-0.666666666666667	\\
2.6e-09	0.336	0	0	\\
2.65e-09	0.336	0	0	\\
2.7e-09	0.336	0	0	\\
2.75e-09	0.336	0.666666666666667	0.666666666666667	\\
2.8e-09	0.336	0.666666666666667	0.666666666666667	\\
2.85e-09	0.336	0.666666666666667	0.666666666666667	\\
2.9e-09	0.336	0.666666666666667	0.666666666666667	\\
2.95e-09	0.336	0.666666666666667	0.666666666666667	\\
3e-09	0.336	0.666666666666667	0.666666666666667	\\
3.05e-09	0.336	0.666666666666667	0.666666666666667	\\
3.1e-09	0.336	0.666666666666667	0.666666666666667	\\
3.15e-09	0.336	0.666666666666667	0.666666666666667	\\
3.2e-09	0.336	0.666666666666667	0.666666666666667	\\
3.25e-09	0.336	0	0	\\
3.3e-09	0.336	0	0	\\
3.35e-09	0.336	0	0	\\
3.4e-09	0.336	0	0	\\
3.45e-09	0.336	0	0	\\
3.5e-09	0.336	0	0	\\
3.55e-09	0.336	0	0	\\
3.6e-09	0.336	0	0	\\
3.65e-09	0.336	0	0	\\
3.7e-09	0.336	0	0	\\
3.75e-09	0.336	0	0	\\
3.8e-09	0.336	0	0	\\
3.85e-09	0.336	0	0	\\
3.9e-09	0.336	0	0	\\
3.95e-09	0.336	0	0	\\
4e-09	0.336	0	0	\\
4.05e-09	0.336	0	0	\\
4.1e-09	0.336	0	0	\\
4.15e-09	0.336	0	0	\\
4.2e-09	0.336	0	0	\\
4.25e-09	0.336	0	0	\\
4.3e-09	0.336	0	0	\\
4.35e-09	0.336	0	0	\\
4.4e-09	0.336	0	0	\\
4.45e-09	0.336	0	0	\\
4.5e-09	0.336	-0.444444444444444	-0.444444444444444	\\
4.55e-09	0.336	-0.444444444444444	-0.444444444444444	\\
4.6e-09	0.336	-0.444444444444444	-0.444444444444444	\\
4.65e-09	0.336	-0.444444444444444	-0.444444444444444	\\
4.7e-09	0.336	-0.444444444444444	-0.444444444444444	\\
4.75e-09	0.336	-0.444444444444444	-0.444444444444444	\\
4.8e-09	0.336	-0.444444444444444	-0.444444444444444	\\
4.85e-09	0.336	-0.444444444444444	-0.444444444444444	\\
4.9e-09	0.336	-0.444444444444444	-0.444444444444444	\\
4.95e-09	0.336	-0.444444444444444	-0.444444444444444	\\
5e-09	0.336	0	0	\\
5e-09	0.336	0	nan	\\
5e-09	0.336	-0.666666666666667	0.166666666666667	\\
5e-09	0.336	-0.666666666666667	nan	\\
0	0.348	-0.666666666666667	nan	\\
0	0.348	0	0.166666666666667	\\
0	0.348	0	0	\\
5e-11	0.348	0	0	\\
1e-10	0.348	0	0	\\
1.5e-10	0.348	0	0	\\
2e-10	0.348	0	0	\\
2.5e-10	0.348	0	0	\\
3e-10	0.348	0	0	\\
3.5e-10	0.348	1	1	\\
4e-10	0.348	1	1	\\
4.5e-10	0.348	1	1	\\
5e-10	0.348	1	1	\\
5.5e-10	0.348	1	1	\\
6e-10	0.348	1	1	\\
6.5e-10	0.348	1	1	\\
7e-10	0.348	1	1	\\
7.5e-10	0.348	1	1	\\
8e-10	0.348	1	1	\\
8.5e-10	0.348	0	0	\\
9e-10	0.348	0	0	\\
9.5e-10	0.348	0	0	\\
1e-09	0.348	0	0	\\
1.05e-09	0.348	0	0	\\
1.1e-09	0.348	0	0	\\
1.15e-09	0.348	0	0	\\
1.2e-09	0.348	0	0	\\
1.25e-09	0.348	0	0	\\
1.3e-09	0.348	0	0	\\
1.35e-09	0.348	0	0	\\
1.4e-09	0.348	0	0	\\
1.45e-09	0.348	0	0	\\
1.5e-09	0.348	0	0	\\
1.55e-09	0.348	0	0	\\
1.6e-09	0.348	0	0	\\
1.65e-09	0.348	0	0	\\
1.7e-09	0.348	0	0	\\
1.75e-09	0.348	0	0	\\
1.8e-09	0.348	0	0	\\
1.85e-09	0.348	0	0	\\
1.9e-09	0.348	0	0	\\
1.95e-09	0.348	0	0	\\
2e-09	0.348	0	0	\\
2.05e-09	0.348	0	0	\\
2.1e-09	0.348	-0.666666666666667	-0.666666666666667	\\
2.15e-09	0.348	-0.666666666666667	-0.666666666666667	\\
2.2e-09	0.348	-0.666666666666667	-0.666666666666667	\\
2.25e-09	0.348	-0.666666666666667	-0.666666666666667	\\
2.3e-09	0.348	-0.666666666666667	-0.666666666666667	\\
2.35e-09	0.348	-0.666666666666667	-0.666666666666667	\\
2.4e-09	0.348	-0.666666666666667	-0.666666666666667	\\
2.45e-09	0.348	-0.666666666666667	-0.666666666666667	\\
2.5e-09	0.348	-0.666666666666667	-0.666666666666667	\\
2.55e-09	0.348	-0.666666666666667	-0.666666666666667	\\
2.6e-09	0.348	0	0	\\
2.65e-09	0.348	0	0	\\
2.7e-09	0.348	0	0	\\
2.75e-09	0.348	0.666666666666667	0.666666666666667	\\
2.8e-09	0.348	0.666666666666667	0.666666666666667	\\
2.85e-09	0.348	0.666666666666667	0.666666666666667	\\
2.9e-09	0.348	0.666666666666667	0.666666666666667	\\
2.95e-09	0.348	0.666666666666667	0.666666666666667	\\
3e-09	0.348	0.666666666666667	0.666666666666667	\\
3.05e-09	0.348	0.666666666666667	0.666666666666667	\\
3.1e-09	0.348	0.666666666666667	0.666666666666667	\\
3.15e-09	0.348	0.666666666666667	0.666666666666667	\\
3.2e-09	0.348	0.666666666666667	0.666666666666667	\\
3.25e-09	0.348	0	0	\\
3.3e-09	0.348	0	0	\\
3.35e-09	0.348	0	0	\\
3.4e-09	0.348	0	0	\\
3.45e-09	0.348	0	0	\\
3.5e-09	0.348	0	0	\\
3.55e-09	0.348	0	0	\\
3.6e-09	0.348	0	0	\\
3.65e-09	0.348	0	0	\\
3.7e-09	0.348	0	0	\\
3.75e-09	0.348	0	0	\\
3.8e-09	0.348	0	0	\\
3.85e-09	0.348	0	0	\\
3.9e-09	0.348	0	0	\\
3.95e-09	0.348	0	0	\\
4e-09	0.348	0	0	\\
4.05e-09	0.348	0	0	\\
4.1e-09	0.348	0	0	\\
4.15e-09	0.348	0	0	\\
4.2e-09	0.348	0	0	\\
4.25e-09	0.348	0	0	\\
4.3e-09	0.348	0	0	\\
4.35e-09	0.348	0	0	\\
4.4e-09	0.348	0	0	\\
4.45e-09	0.348	0	0	\\
4.5e-09	0.348	-0.444444444444444	-0.444444444444444	\\
4.55e-09	0.348	-0.444444444444444	-0.444444444444444	\\
4.6e-09	0.348	-0.444444444444444	-0.444444444444444	\\
4.65e-09	0.348	-0.444444444444444	-0.444444444444444	\\
4.7e-09	0.348	-0.444444444444444	-0.444444444444444	\\
4.75e-09	0.348	-0.444444444444444	-0.444444444444444	\\
4.8e-09	0.348	-0.444444444444444	-0.444444444444444	\\
4.85e-09	0.348	-0.444444444444444	-0.444444444444444	\\
4.9e-09	0.348	-0.444444444444444	-0.444444444444444	\\
4.95e-09	0.348	-0.444444444444444	-0.444444444444444	\\
5e-09	0.348	0	0	\\
5e-09	0.348	0	nan	\\
5e-09	0.348	-0.666666666666667	0.166666666666667	\\
5e-09	0.348	-0.666666666666667	nan	\\
0	0.36	-0.666666666666667	nan	\\
0	0.36	0	0.166666666666667	\\
0	0.36	0	0	\\
5e-11	0.36	0	0	\\
1e-10	0.36	0	0	\\
1.5e-10	0.36	0	0	\\
2e-10	0.36	0	0	\\
2.5e-10	0.36	0	0	\\
3e-10	0.36	0	0	\\
3.5e-10	0.36	0	0	\\
4e-10	0.36	1	1	\\
4.5e-10	0.36	1	1	\\
5e-10	0.36	1	1	\\
5.5e-10	0.36	1	1	\\
6e-10	0.36	1	1	\\
6.5e-10	0.36	1	1	\\
7e-10	0.36	1	1	\\
7.5e-10	0.36	1	1	\\
8e-10	0.36	1	1	\\
8.5e-10	0.36	1	1	\\
9e-10	0.36	0	0	\\
9.5e-10	0.36	0	0	\\
1e-09	0.36	0	0	\\
1.05e-09	0.36	0	0	\\
1.1e-09	0.36	0	0	\\
1.15e-09	0.36	0	0	\\
1.2e-09	0.36	0	0	\\
1.25e-09	0.36	0	0	\\
1.3e-09	0.36	0	0	\\
1.35e-09	0.36	0	0	\\
1.4e-09	0.36	0	0	\\
1.45e-09	0.36	0	0	\\
1.5e-09	0.36	0	0	\\
1.55e-09	0.36	0	0	\\
1.6e-09	0.36	0	0	\\
1.65e-09	0.36	0	0	\\
1.7e-09	0.36	0	0	\\
1.75e-09	0.36	0	0	\\
1.8e-09	0.36	0	0	\\
1.85e-09	0.36	0	0	\\
1.9e-09	0.36	0	0	\\
1.95e-09	0.36	0	0	\\
2e-09	0.36	0	0	\\
2.05e-09	0.36	-0.666666666666667	-0.666666666666667	\\
2.1e-09	0.36	-0.666666666666667	-0.666666666666667	\\
2.15e-09	0.36	-0.666666666666667	-0.666666666666667	\\
2.2e-09	0.36	-0.666666666666667	-0.666666666666667	\\
2.25e-09	0.36	-0.666666666666667	-0.666666666666667	\\
2.3e-09	0.36	-0.666666666666667	-0.666666666666667	\\
2.35e-09	0.36	-0.666666666666667	-0.666666666666667	\\
2.4e-09	0.36	-0.666666666666667	-0.666666666666667	\\
2.45e-09	0.36	-0.666666666666667	-0.666666666666667	\\
2.5e-09	0.36	-0.666666666666667	-0.666666666666667	\\
2.55e-09	0.36	0	0	\\
2.6e-09	0.36	0	0	\\
2.65e-09	0.36	0	0	\\
2.7e-09	0.36	0	0	\\
2.75e-09	0.36	0	0	\\
2.8e-09	0.36	0.666666666666667	0.666666666666667	\\
2.85e-09	0.36	0.666666666666667	0.666666666666667	\\
2.9e-09	0.36	0.666666666666667	0.666666666666667	\\
2.95e-09	0.36	0.666666666666667	0.666666666666667	\\
3e-09	0.36	0.666666666666667	0.666666666666667	\\
3.05e-09	0.36	0.666666666666667	0.666666666666667	\\
3.1e-09	0.36	0.666666666666667	0.666666666666667	\\
3.15e-09	0.36	0.666666666666667	0.666666666666667	\\
3.2e-09	0.36	0.666666666666667	0.666666666666667	\\
3.25e-09	0.36	0.666666666666667	0.666666666666667	\\
3.3e-09	0.36	0	0	\\
3.35e-09	0.36	0	0	\\
3.4e-09	0.36	0	0	\\
3.45e-09	0.36	0	0	\\
3.5e-09	0.36	0	0	\\
3.55e-09	0.36	0	0	\\
3.6e-09	0.36	0	0	\\
3.65e-09	0.36	0	0	\\
3.7e-09	0.36	0	0	\\
3.75e-09	0.36	0	0	\\
3.8e-09	0.36	0	0	\\
3.85e-09	0.36	0	0	\\
3.9e-09	0.36	0	0	\\
3.95e-09	0.36	0	0	\\
4e-09	0.36	0	0	\\
4.05e-09	0.36	0	0	\\
4.1e-09	0.36	0	0	\\
4.15e-09	0.36	0	0	\\
4.2e-09	0.36	0	0	\\
4.25e-09	0.36	0	0	\\
4.3e-09	0.36	0	0	\\
4.35e-09	0.36	0	0	\\
4.4e-09	0.36	0	0	\\
4.45e-09	0.36	-0.444444444444444	-0.444444444444444	\\
4.5e-09	0.36	-0.444444444444444	-0.444444444444444	\\
4.55e-09	0.36	-0.444444444444444	-0.444444444444444	\\
4.6e-09	0.36	-0.444444444444444	-0.444444444444444	\\
4.65e-09	0.36	-0.444444444444444	-0.444444444444444	\\
4.7e-09	0.36	-0.444444444444444	-0.444444444444444	\\
4.75e-09	0.36	-0.444444444444444	-0.444444444444444	\\
4.8e-09	0.36	-0.444444444444444	-0.444444444444444	\\
4.85e-09	0.36	-0.444444444444444	-0.444444444444444	\\
4.9e-09	0.36	-0.444444444444444	-0.444444444444444	\\
4.95e-09	0.36	0	0	\\
5e-09	0.36	0	0	\\
5e-09	0.36	0	nan	\\
5e-09	0.36	-0.666666666666667	0.166666666666667	\\
5e-09	0.36	-0.666666666666667	nan	\\
0	0.372	-0.666666666666667	nan	\\
0	0.372	0	0.166666666666667	\\
0	0.372	0	0	\\
5e-11	0.372	0	0	\\
1e-10	0.372	0	0	\\
1.5e-10	0.372	0	0	\\
2e-10	0.372	0	0	\\
2.5e-10	0.372	0	0	\\
3e-10	0.372	0	0	\\
3.5e-10	0.372	0	0	\\
4e-10	0.372	1	1	\\
4.5e-10	0.372	1	1	\\
5e-10	0.372	1	1	\\
5.5e-10	0.372	1	1	\\
6e-10	0.372	1	1	\\
6.5e-10	0.372	1	1	\\
7e-10	0.372	1	1	\\
7.5e-10	0.372	1	1	\\
8e-10	0.372	1	1	\\
8.5e-10	0.372	1	1	\\
9e-10	0.372	0	0	\\
9.5e-10	0.372	0	0	\\
1e-09	0.372	0	0	\\
1.05e-09	0.372	0	0	\\
1.1e-09	0.372	0	0	\\
1.15e-09	0.372	0	0	\\
1.2e-09	0.372	0	0	\\
1.25e-09	0.372	0	0	\\
1.3e-09	0.372	0	0	\\
1.35e-09	0.372	0	0	\\
1.4e-09	0.372	0	0	\\
1.45e-09	0.372	0	0	\\
1.5e-09	0.372	0	0	\\
1.55e-09	0.372	0	0	\\
1.6e-09	0.372	0	0	\\
1.65e-09	0.372	0	0	\\
1.7e-09	0.372	0	0	\\
1.75e-09	0.372	0	0	\\
1.8e-09	0.372	0	0	\\
1.85e-09	0.372	0	0	\\
1.9e-09	0.372	0	0	\\
1.95e-09	0.372	0	0	\\
2e-09	0.372	0	0	\\
2.05e-09	0.372	-0.666666666666667	-0.666666666666667	\\
2.1e-09	0.372	-0.666666666666667	-0.666666666666667	\\
2.15e-09	0.372	-0.666666666666667	-0.666666666666667	\\
2.2e-09	0.372	-0.666666666666667	-0.666666666666667	\\
2.25e-09	0.372	-0.666666666666667	-0.666666666666667	\\
2.3e-09	0.372	-0.666666666666667	-0.666666666666667	\\
2.35e-09	0.372	-0.666666666666667	-0.666666666666667	\\
2.4e-09	0.372	-0.666666666666667	-0.666666666666667	\\
2.45e-09	0.372	-0.666666666666667	-0.666666666666667	\\
2.5e-09	0.372	-0.666666666666667	-0.666666666666667	\\
2.55e-09	0.372	0	0	\\
2.6e-09	0.372	0	0	\\
2.65e-09	0.372	0	0	\\
2.7e-09	0.372	0	0	\\
2.75e-09	0.372	0	0	\\
2.8e-09	0.372	0.666666666666667	0.666666666666667	\\
2.85e-09	0.372	0.666666666666667	0.666666666666667	\\
2.9e-09	0.372	0.666666666666667	0.666666666666667	\\
2.95e-09	0.372	0.666666666666667	0.666666666666667	\\
3e-09	0.372	0.666666666666667	0.666666666666667	\\
3.05e-09	0.372	0.666666666666667	0.666666666666667	\\
3.1e-09	0.372	0.666666666666667	0.666666666666667	\\
3.15e-09	0.372	0.666666666666667	0.666666666666667	\\
3.2e-09	0.372	0.666666666666667	0.666666666666667	\\
3.25e-09	0.372	0.666666666666667	0.666666666666667	\\
3.3e-09	0.372	0	0	\\
3.35e-09	0.372	0	0	\\
3.4e-09	0.372	0	0	\\
3.45e-09	0.372	0	0	\\
3.5e-09	0.372	0	0	\\
3.55e-09	0.372	0	0	\\
3.6e-09	0.372	0	0	\\
3.65e-09	0.372	0	0	\\
3.7e-09	0.372	0	0	\\
3.75e-09	0.372	0	0	\\
3.8e-09	0.372	0	0	\\
3.85e-09	0.372	0	0	\\
3.9e-09	0.372	0	0	\\
3.95e-09	0.372	0	0	\\
4e-09	0.372	0	0	\\
4.05e-09	0.372	0	0	\\
4.1e-09	0.372	0	0	\\
4.15e-09	0.372	0	0	\\
4.2e-09	0.372	0	0	\\
4.25e-09	0.372	0	0	\\
4.3e-09	0.372	0	0	\\
4.35e-09	0.372	0	0	\\
4.4e-09	0.372	0	0	\\
4.45e-09	0.372	-0.444444444444444	-0.444444444444444	\\
4.5e-09	0.372	-0.444444444444444	-0.444444444444444	\\
4.55e-09	0.372	-0.444444444444444	-0.444444444444444	\\
4.6e-09	0.372	-0.444444444444444	-0.444444444444444	\\
4.65e-09	0.372	-0.444444444444444	-0.444444444444444	\\
4.7e-09	0.372	-0.444444444444444	-0.444444444444444	\\
4.75e-09	0.372	-0.444444444444444	-0.444444444444444	\\
4.8e-09	0.372	-0.444444444444444	-0.444444444444444	\\
4.85e-09	0.372	-0.444444444444444	-0.444444444444444	\\
4.9e-09	0.372	-0.444444444444444	-0.444444444444444	\\
4.95e-09	0.372	0	0	\\
5e-09	0.372	0	0	\\
5e-09	0.372	0	nan	\\
5e-09	0.372	-0.666666666666667	0.166666666666667	\\
5e-09	0.372	-0.666666666666667	nan	\\
0	0.384	-0.666666666666667	nan	\\
0	0.384	0	0.166666666666667	\\
0	0.384	0	0	\\
5e-11	0.384	0	0	\\
1e-10	0.384	0	0	\\
1.5e-10	0.384	0	0	\\
2e-10	0.384	0	0	\\
2.5e-10	0.384	0	0	\\
3e-10	0.384	0	0	\\
3.5e-10	0.384	0	0	\\
4e-10	0.384	1	1	\\
4.5e-10	0.384	1	1	\\
5e-10	0.384	1	1	\\
5.5e-10	0.384	1	1	\\
6e-10	0.384	1	1	\\
6.5e-10	0.384	1	1	\\
7e-10	0.384	1	1	\\
7.5e-10	0.384	1	1	\\
8e-10	0.384	1	1	\\
8.5e-10	0.384	1	1	\\
9e-10	0.384	0	0	\\
9.5e-10	0.384	0	0	\\
1e-09	0.384	0	0	\\
1.05e-09	0.384	0	0	\\
1.1e-09	0.384	0	0	\\
1.15e-09	0.384	0	0	\\
1.2e-09	0.384	0	0	\\
1.25e-09	0.384	0	0	\\
1.3e-09	0.384	0	0	\\
1.35e-09	0.384	0	0	\\
1.4e-09	0.384	0	0	\\
1.45e-09	0.384	0	0	\\
1.5e-09	0.384	0	0	\\
1.55e-09	0.384	0	0	\\
1.6e-09	0.384	0	0	\\
1.65e-09	0.384	0	0	\\
1.7e-09	0.384	0	0	\\
1.75e-09	0.384	0	0	\\
1.8e-09	0.384	0	0	\\
1.85e-09	0.384	0	0	\\
1.9e-09	0.384	0	0	\\
1.95e-09	0.384	0	0	\\
2e-09	0.384	0	0	\\
2.05e-09	0.384	-0.666666666666667	-0.666666666666667	\\
2.1e-09	0.384	-0.666666666666667	-0.666666666666667	\\
2.15e-09	0.384	-0.666666666666667	-0.666666666666667	\\
2.2e-09	0.384	-0.666666666666667	-0.666666666666667	\\
2.25e-09	0.384	-0.666666666666667	-0.666666666666667	\\
2.3e-09	0.384	-0.666666666666667	-0.666666666666667	\\
2.35e-09	0.384	-0.666666666666667	-0.666666666666667	\\
2.4e-09	0.384	-0.666666666666667	-0.666666666666667	\\
2.45e-09	0.384	-0.666666666666667	-0.666666666666667	\\
2.5e-09	0.384	-0.666666666666667	-0.666666666666667	\\
2.55e-09	0.384	0	0	\\
2.6e-09	0.384	0	0	\\
2.65e-09	0.384	0	0	\\
2.7e-09	0.384	0	0	\\
2.75e-09	0.384	0	0	\\
2.8e-09	0.384	0.666666666666667	0.666666666666667	\\
2.85e-09	0.384	0.666666666666667	0.666666666666667	\\
2.9e-09	0.384	0.666666666666667	0.666666666666667	\\
2.95e-09	0.384	0.666666666666667	0.666666666666667	\\
3e-09	0.384	0.666666666666667	0.666666666666667	\\
3.05e-09	0.384	0.666666666666667	0.666666666666667	\\
3.1e-09	0.384	0.666666666666667	0.666666666666667	\\
3.15e-09	0.384	0.666666666666667	0.666666666666667	\\
3.2e-09	0.384	0.666666666666667	0.666666666666667	\\
3.25e-09	0.384	0.666666666666667	0.666666666666667	\\
3.3e-09	0.384	0	0	\\
3.35e-09	0.384	0	0	\\
3.4e-09	0.384	0	0	\\
3.45e-09	0.384	0	0	\\
3.5e-09	0.384	0	0	\\
3.55e-09	0.384	0	0	\\
3.6e-09	0.384	0	0	\\
3.65e-09	0.384	0	0	\\
3.7e-09	0.384	0	0	\\
3.75e-09	0.384	0	0	\\
3.8e-09	0.384	0	0	\\
3.85e-09	0.384	0	0	\\
3.9e-09	0.384	0	0	\\
3.95e-09	0.384	0	0	\\
4e-09	0.384	0	0	\\
4.05e-09	0.384	0	0	\\
4.1e-09	0.384	0	0	\\
4.15e-09	0.384	0	0	\\
4.2e-09	0.384	0	0	\\
4.25e-09	0.384	0	0	\\
4.3e-09	0.384	0	0	\\
4.35e-09	0.384	0	0	\\
4.4e-09	0.384	0	0	\\
4.45e-09	0.384	-0.444444444444444	-0.444444444444444	\\
4.5e-09	0.384	-0.444444444444444	-0.444444444444444	\\
4.55e-09	0.384	-0.444444444444444	-0.444444444444444	\\
4.6e-09	0.384	-0.444444444444444	-0.444444444444444	\\
4.65e-09	0.384	-0.444444444444444	-0.444444444444444	\\
4.7e-09	0.384	-0.444444444444444	-0.444444444444444	\\
4.75e-09	0.384	-0.444444444444444	-0.444444444444444	\\
4.8e-09	0.384	-0.444444444444444	-0.444444444444444	\\
4.85e-09	0.384	-0.444444444444444	-0.444444444444444	\\
4.9e-09	0.384	-0.444444444444444	-0.444444444444444	\\
4.95e-09	0.384	0	0	\\
5e-09	0.384	0	0	\\
5e-09	0.384	0	nan	\\
5e-09	0.384	-0.666666666666667	0.166666666666667	\\
5e-09	0.384	-0.666666666666667	nan	\\
0	0.396	-0.666666666666667	nan	\\
0	0.396	0	0.166666666666667	\\
0	0.396	0	0	\\
5e-11	0.396	0	0	\\
1e-10	0.396	0	0	\\
1.5e-10	0.396	0	0	\\
2e-10	0.396	0	0	\\
2.5e-10	0.396	0	0	\\
3e-10	0.396	0	0	\\
3.5e-10	0.396	0	0	\\
4e-10	0.396	1	1	\\
4.5e-10	0.396	1	1	\\
5e-10	0.396	1	1	\\
5.5e-10	0.396	1	1	\\
6e-10	0.396	1	1	\\
6.5e-10	0.396	1	1	\\
7e-10	0.396	1	1	\\
7.5e-10	0.396	1	1	\\
8e-10	0.396	1	1	\\
8.5e-10	0.396	1	1	\\
9e-10	0.396	0	0	\\
9.5e-10	0.396	0	0	\\
1e-09	0.396	0	0	\\
1.05e-09	0.396	0	0	\\
1.1e-09	0.396	0	0	\\
1.15e-09	0.396	0	0	\\
1.2e-09	0.396	0	0	\\
1.25e-09	0.396	0	0	\\
1.3e-09	0.396	0	0	\\
1.35e-09	0.396	0	0	\\
1.4e-09	0.396	0	0	\\
1.45e-09	0.396	0	0	\\
1.5e-09	0.396	0	0	\\
1.55e-09	0.396	0	0	\\
1.6e-09	0.396	0	0	\\
1.65e-09	0.396	0	0	\\
1.7e-09	0.396	0	0	\\
1.75e-09	0.396	0	0	\\
1.8e-09	0.396	0	0	\\
1.85e-09	0.396	0	0	\\
1.9e-09	0.396	0	0	\\
1.95e-09	0.396	0	0	\\
2e-09	0.396	0	0	\\
2.05e-09	0.396	-0.666666666666667	-0.666666666666667	\\
2.1e-09	0.396	-0.666666666666667	-0.666666666666667	\\
2.15e-09	0.396	-0.666666666666667	-0.666666666666667	\\
2.2e-09	0.396	-0.666666666666667	-0.666666666666667	\\
2.25e-09	0.396	-0.666666666666667	-0.666666666666667	\\
2.3e-09	0.396	-0.666666666666667	-0.666666666666667	\\
2.35e-09	0.396	-0.666666666666667	-0.666666666666667	\\
2.4e-09	0.396	-0.666666666666667	-0.666666666666667	\\
2.45e-09	0.396	-0.666666666666667	-0.666666666666667	\\
2.5e-09	0.396	-0.666666666666667	-0.666666666666667	\\
2.55e-09	0.396	0	0	\\
2.6e-09	0.396	0	0	\\
2.65e-09	0.396	0	0	\\
2.7e-09	0.396	0	0	\\
2.75e-09	0.396	0	0	\\
2.8e-09	0.396	0.666666666666667	0.666666666666667	\\
2.85e-09	0.396	0.666666666666667	0.666666666666667	\\
2.9e-09	0.396	0.666666666666667	0.666666666666667	\\
2.95e-09	0.396	0.666666666666667	0.666666666666667	\\
3e-09	0.396	0.666666666666667	0.666666666666667	\\
3.05e-09	0.396	0.666666666666667	0.666666666666667	\\
3.1e-09	0.396	0.666666666666667	0.666666666666667	\\
3.15e-09	0.396	0.666666666666667	0.666666666666667	\\
3.2e-09	0.396	0.666666666666667	0.666666666666667	\\
3.25e-09	0.396	0.666666666666667	0.666666666666667	\\
3.3e-09	0.396	0	0	\\
3.35e-09	0.396	0	0	\\
3.4e-09	0.396	0	0	\\
3.45e-09	0.396	0	0	\\
3.5e-09	0.396	0	0	\\
3.55e-09	0.396	0	0	\\
3.6e-09	0.396	0	0	\\
3.65e-09	0.396	0	0	\\
3.7e-09	0.396	0	0	\\
3.75e-09	0.396	0	0	\\
3.8e-09	0.396	0	0	\\
3.85e-09	0.396	0	0	\\
3.9e-09	0.396	0	0	\\
3.95e-09	0.396	0	0	\\
4e-09	0.396	0	0	\\
4.05e-09	0.396	0	0	\\
4.1e-09	0.396	0	0	\\
4.15e-09	0.396	0	0	\\
4.2e-09	0.396	0	0	\\
4.25e-09	0.396	0	0	\\
4.3e-09	0.396	0	0	\\
4.35e-09	0.396	0	0	\\
4.4e-09	0.396	0	0	\\
4.45e-09	0.396	-0.444444444444444	-0.444444444444444	\\
4.5e-09	0.396	-0.444444444444444	-0.444444444444444	\\
4.55e-09	0.396	-0.444444444444444	-0.444444444444444	\\
4.6e-09	0.396	-0.444444444444444	-0.444444444444444	\\
4.65e-09	0.396	-0.444444444444444	-0.444444444444444	\\
4.7e-09	0.396	-0.444444444444444	-0.444444444444444	\\
4.75e-09	0.396	-0.444444444444444	-0.444444444444444	\\
4.8e-09	0.396	-0.444444444444444	-0.444444444444444	\\
4.85e-09	0.396	-0.444444444444444	-0.444444444444444	\\
4.9e-09	0.396	-0.444444444444444	-0.444444444444444	\\
4.95e-09	0.396	0	0	\\
5e-09	0.396	0	0	\\
5e-09	0.396	0	nan	\\
5e-09	0.396	-0.666666666666667	0.166666666666667	\\
5e-09	0.396	-0.666666666666667	nan	\\
0	0.408	-0.666666666666667	nan	\\
0	0.408	0	0.166666666666667	\\
0	0.408	0	0	\\
5e-11	0.408	0	0	\\
1e-10	0.408	0	0	\\
1.5e-10	0.408	0	0	\\
2e-10	0.408	0	0	\\
2.5e-10	0.408	0	0	\\
3e-10	0.408	0	0	\\
3.5e-10	0.408	0	0	\\
4e-10	0.408	0	0	\\
4.5e-10	0.408	1	1	\\
5e-10	0.408	1	1	\\
5.5e-10	0.408	1	1	\\
6e-10	0.408	1	1	\\
6.5e-10	0.408	1	1	\\
7e-10	0.408	1	1	\\
7.5e-10	0.408	1	1	\\
8e-10	0.408	1	1	\\
8.5e-10	0.408	1	1	\\
9e-10	0.408	1	1	\\
9.5e-10	0.408	0	0	\\
1e-09	0.408	0	0	\\
1.05e-09	0.408	0	0	\\
1.1e-09	0.408	0	0	\\
1.15e-09	0.408	0	0	\\
1.2e-09	0.408	0	0	\\
1.25e-09	0.408	0	0	\\
1.3e-09	0.408	0	0	\\
1.35e-09	0.408	0	0	\\
1.4e-09	0.408	0	0	\\
1.45e-09	0.408	0	0	\\
1.5e-09	0.408	0	0	\\
1.55e-09	0.408	0	0	\\
1.6e-09	0.408	0	0	\\
1.65e-09	0.408	0	0	\\
1.7e-09	0.408	0	0	\\
1.75e-09	0.408	0	0	\\
1.8e-09	0.408	0	0	\\
1.85e-09	0.408	0	0	\\
1.9e-09	0.408	0	0	\\
1.95e-09	0.408	0	0	\\
2e-09	0.408	-0.666666666666667	-0.666666666666667	\\
2.05e-09	0.408	-0.666666666666667	-0.666666666666667	\\
2.1e-09	0.408	-0.666666666666667	-0.666666666666667	\\
2.15e-09	0.408	-0.666666666666667	-0.666666666666667	\\
2.2e-09	0.408	-0.666666666666667	-0.666666666666667	\\
2.25e-09	0.408	-0.666666666666667	-0.666666666666667	\\
2.3e-09	0.408	-0.666666666666667	-0.666666666666667	\\
2.35e-09	0.408	-0.666666666666667	-0.666666666666667	\\
2.4e-09	0.408	-0.666666666666667	-0.666666666666667	\\
2.45e-09	0.408	-0.666666666666667	-0.666666666666667	\\
2.5e-09	0.408	0	0	\\
2.55e-09	0.408	0	0	\\
2.6e-09	0.408	0	0	\\
2.65e-09	0.408	0	0	\\
2.7e-09	0.408	0	0	\\
2.75e-09	0.408	0	0	\\
2.8e-09	0.408	0	0	\\
2.85e-09	0.408	0.666666666666667	0.666666666666667	\\
2.9e-09	0.408	0.666666666666667	0.666666666666667	\\
2.95e-09	0.408	0.666666666666667	0.666666666666667	\\
3e-09	0.408	0.666666666666667	0.666666666666667	\\
3.05e-09	0.408	0.666666666666667	0.666666666666667	\\
3.1e-09	0.408	0.666666666666667	0.666666666666667	\\
3.15e-09	0.408	0.666666666666667	0.666666666666667	\\
3.2e-09	0.408	0.666666666666667	0.666666666666667	\\
3.25e-09	0.408	0.666666666666667	0.666666666666667	\\
3.3e-09	0.408	0.666666666666667	0.666666666666667	\\
3.35e-09	0.408	0	0	\\
3.4e-09	0.408	0	0	\\
3.45e-09	0.408	0	0	\\
3.5e-09	0.408	0	0	\\
3.55e-09	0.408	0	0	\\
3.6e-09	0.408	0	0	\\
3.65e-09	0.408	0	0	\\
3.7e-09	0.408	0	0	\\
3.75e-09	0.408	0	0	\\
3.8e-09	0.408	0	0	\\
3.85e-09	0.408	0	0	\\
3.9e-09	0.408	0	0	\\
3.95e-09	0.408	0	0	\\
4e-09	0.408	0	0	\\
4.05e-09	0.408	0	0	\\
4.1e-09	0.408	0	0	\\
4.15e-09	0.408	0	0	\\
4.2e-09	0.408	0	0	\\
4.25e-09	0.408	0	0	\\
4.3e-09	0.408	0	0	\\
4.35e-09	0.408	0	0	\\
4.4e-09	0.408	-0.444444444444444	-0.444444444444444	\\
4.45e-09	0.408	-0.444444444444444	-0.444444444444444	\\
4.5e-09	0.408	-0.444444444444444	-0.444444444444444	\\
4.55e-09	0.408	-0.444444444444444	-0.444444444444444	\\
4.6e-09	0.408	-0.444444444444444	-0.444444444444444	\\
4.65e-09	0.408	-0.444444444444444	-0.444444444444444	\\
4.7e-09	0.408	-0.444444444444444	-0.444444444444444	\\
4.75e-09	0.408	-0.444444444444444	-0.444444444444444	\\
4.8e-09	0.408	-0.444444444444444	-0.444444444444444	\\
4.85e-09	0.408	-0.444444444444444	-0.444444444444444	\\
4.9e-09	0.408	0	0	\\
4.95e-09	0.408	0	0	\\
5e-09	0.408	0	0	\\
5e-09	0.408	0	nan	\\
5e-09	0.408	-0.666666666666667	0.166666666666667	\\
5e-09	0.408	-0.666666666666667	nan	\\
0	0.42	-0.666666666666667	nan	\\
0	0.42	0	0.166666666666667	\\
0	0.42	0	0	\\
5e-11	0.42	0	0	\\
1e-10	0.42	0	0	\\
1.5e-10	0.42	0	0	\\
2e-10	0.42	0	0	\\
2.5e-10	0.42	0	0	\\
3e-10	0.42	0	0	\\
3.5e-10	0.42	0	0	\\
4e-10	0.42	0	0	\\
4.5e-10	0.42	1	1	\\
5e-10	0.42	1	1	\\
5.5e-10	0.42	1	1	\\
6e-10	0.42	1	1	\\
6.5e-10	0.42	1	1	\\
7e-10	0.42	1	1	\\
7.5e-10	0.42	1	1	\\
8e-10	0.42	1	1	\\
8.5e-10	0.42	1	1	\\
9e-10	0.42	1	1	\\
9.5e-10	0.42	0	0	\\
1e-09	0.42	0	0	\\
1.05e-09	0.42	0	0	\\
1.1e-09	0.42	0	0	\\
1.15e-09	0.42	0	0	\\
1.2e-09	0.42	0	0	\\
1.25e-09	0.42	0	0	\\
1.3e-09	0.42	0	0	\\
1.35e-09	0.42	0	0	\\
1.4e-09	0.42	0	0	\\
1.45e-09	0.42	0	0	\\
1.5e-09	0.42	0	0	\\
1.55e-09	0.42	0	0	\\
1.6e-09	0.42	0	0	\\
1.65e-09	0.42	0	0	\\
1.7e-09	0.42	0	0	\\
1.75e-09	0.42	0	0	\\
1.8e-09	0.42	0	0	\\
1.85e-09	0.42	0	0	\\
1.9e-09	0.42	0	0	\\
1.95e-09	0.42	0	0	\\
2e-09	0.42	-0.666666666666667	-0.666666666666667	\\
2.05e-09	0.42	-0.666666666666667	-0.666666666666667	\\
2.1e-09	0.42	-0.666666666666667	-0.666666666666667	\\
2.15e-09	0.42	-0.666666666666667	-0.666666666666667	\\
2.2e-09	0.42	-0.666666666666667	-0.666666666666667	\\
2.25e-09	0.42	-0.666666666666667	-0.666666666666667	\\
2.3e-09	0.42	-0.666666666666667	-0.666666666666667	\\
2.35e-09	0.42	-0.666666666666667	-0.666666666666667	\\
2.4e-09	0.42	-0.666666666666667	-0.666666666666667	\\
2.45e-09	0.42	-0.666666666666667	-0.666666666666667	\\
2.5e-09	0.42	0	0	\\
2.55e-09	0.42	0	0	\\
2.6e-09	0.42	0	0	\\
2.65e-09	0.42	0	0	\\
2.7e-09	0.42	0	0	\\
2.75e-09	0.42	0	0	\\
2.8e-09	0.42	0	0	\\
2.85e-09	0.42	0.666666666666667	0.666666666666667	\\
2.9e-09	0.42	0.666666666666667	0.666666666666667	\\
2.95e-09	0.42	0.666666666666667	0.666666666666667	\\
3e-09	0.42	0.666666666666667	0.666666666666667	\\
3.05e-09	0.42	0.666666666666667	0.666666666666667	\\
3.1e-09	0.42	0.666666666666667	0.666666666666667	\\
3.15e-09	0.42	0.666666666666667	0.666666666666667	\\
3.2e-09	0.42	0.666666666666667	0.666666666666667	\\
3.25e-09	0.42	0.666666666666667	0.666666666666667	\\
3.3e-09	0.42	0.666666666666667	0.666666666666667	\\
3.35e-09	0.42	0	0	\\
3.4e-09	0.42	0	0	\\
3.45e-09	0.42	0	0	\\
3.5e-09	0.42	0	0	\\
3.55e-09	0.42	0	0	\\
3.6e-09	0.42	0	0	\\
3.65e-09	0.42	0	0	\\
3.7e-09	0.42	0	0	\\
3.75e-09	0.42	0	0	\\
3.8e-09	0.42	0	0	\\
3.85e-09	0.42	0	0	\\
3.9e-09	0.42	0	0	\\
3.95e-09	0.42	0	0	\\
4e-09	0.42	0	0	\\
4.05e-09	0.42	0	0	\\
4.1e-09	0.42	0	0	\\
4.15e-09	0.42	0	0	\\
4.2e-09	0.42	0	0	\\
4.25e-09	0.42	0	0	\\
4.3e-09	0.42	0	0	\\
4.35e-09	0.42	0	0	\\
4.4e-09	0.42	-0.444444444444444	-0.444444444444444	\\
4.45e-09	0.42	-0.444444444444444	-0.444444444444444	\\
4.5e-09	0.42	-0.444444444444444	-0.444444444444444	\\
4.55e-09	0.42	-0.444444444444444	-0.444444444444444	\\
4.6e-09	0.42	-0.444444444444444	-0.444444444444444	\\
4.65e-09	0.42	-0.444444444444444	-0.444444444444444	\\
4.7e-09	0.42	-0.444444444444444	-0.444444444444444	\\
4.75e-09	0.42	-0.444444444444444	-0.444444444444444	\\
4.8e-09	0.42	-0.444444444444444	-0.444444444444444	\\
4.85e-09	0.42	-0.444444444444444	-0.444444444444444	\\
4.9e-09	0.42	0	0	\\
4.95e-09	0.42	0	0	\\
5e-09	0.42	0	0	\\
5e-09	0.42	0	nan	\\
5e-09	0.42	-0.666666666666667	0.166666666666667	\\
5e-09	0.42	-0.666666666666667	nan	\\
0	0.432	-0.666666666666667	nan	\\
0	0.432	0	0.166666666666667	\\
0	0.432	0	0	\\
5e-11	0.432	0	0	\\
1e-10	0.432	0	0	\\
1.5e-10	0.432	0	0	\\
2e-10	0.432	0	0	\\
2.5e-10	0.432	0	0	\\
3e-10	0.432	0	0	\\
3.5e-10	0.432	0	0	\\
4e-10	0.432	0	0	\\
4.5e-10	0.432	1	1	\\
5e-10	0.432	1	1	\\
5.5e-10	0.432	1	1	\\
6e-10	0.432	1	1	\\
6.5e-10	0.432	1	1	\\
7e-10	0.432	1	1	\\
7.5e-10	0.432	1	1	\\
8e-10	0.432	1	1	\\
8.5e-10	0.432	1	1	\\
9e-10	0.432	1	1	\\
9.5e-10	0.432	0	0	\\
1e-09	0.432	0	0	\\
1.05e-09	0.432	0	0	\\
1.1e-09	0.432	0	0	\\
1.15e-09	0.432	0	0	\\
1.2e-09	0.432	0	0	\\
1.25e-09	0.432	0	0	\\
1.3e-09	0.432	0	0	\\
1.35e-09	0.432	0	0	\\
1.4e-09	0.432	0	0	\\
1.45e-09	0.432	0	0	\\
1.5e-09	0.432	0	0	\\
1.55e-09	0.432	0	0	\\
1.6e-09	0.432	0	0	\\
1.65e-09	0.432	0	0	\\
1.7e-09	0.432	0	0	\\
1.75e-09	0.432	0	0	\\
1.8e-09	0.432	0	0	\\
1.85e-09	0.432	0	0	\\
1.9e-09	0.432	0	0	\\
1.95e-09	0.432	0	0	\\
2e-09	0.432	-0.666666666666667	-0.666666666666667	\\
2.05e-09	0.432	-0.666666666666667	-0.666666666666667	\\
2.1e-09	0.432	-0.666666666666667	-0.666666666666667	\\
2.15e-09	0.432	-0.666666666666667	-0.666666666666667	\\
2.2e-09	0.432	-0.666666666666667	-0.666666666666667	\\
2.25e-09	0.432	-0.666666666666667	-0.666666666666667	\\
2.3e-09	0.432	-0.666666666666667	-0.666666666666667	\\
2.35e-09	0.432	-0.666666666666667	-0.666666666666667	\\
2.4e-09	0.432	-0.666666666666667	-0.666666666666667	\\
2.45e-09	0.432	-0.666666666666667	-0.666666666666667	\\
2.5e-09	0.432	0	0	\\
2.55e-09	0.432	0	0	\\
2.6e-09	0.432	0	0	\\
2.65e-09	0.432	0	0	\\
2.7e-09	0.432	0	0	\\
2.75e-09	0.432	0	0	\\
2.8e-09	0.432	0	0	\\
2.85e-09	0.432	0.666666666666667	0.666666666666667	\\
2.9e-09	0.432	0.666666666666667	0.666666666666667	\\
2.95e-09	0.432	0.666666666666667	0.666666666666667	\\
3e-09	0.432	0.666666666666667	0.666666666666667	\\
3.05e-09	0.432	0.666666666666667	0.666666666666667	\\
3.1e-09	0.432	0.666666666666667	0.666666666666667	\\
3.15e-09	0.432	0.666666666666667	0.666666666666667	\\
3.2e-09	0.432	0.666666666666667	0.666666666666667	\\
3.25e-09	0.432	0.666666666666667	0.666666666666667	\\
3.3e-09	0.432	0.666666666666667	0.666666666666667	\\
3.35e-09	0.432	0	0	\\
3.4e-09	0.432	0	0	\\
3.45e-09	0.432	0	0	\\
3.5e-09	0.432	0	0	\\
3.55e-09	0.432	0	0	\\
3.6e-09	0.432	0	0	\\
3.65e-09	0.432	0	0	\\
3.7e-09	0.432	0	0	\\
3.75e-09	0.432	0	0	\\
3.8e-09	0.432	0	0	\\
3.85e-09	0.432	0	0	\\
3.9e-09	0.432	0	0	\\
3.95e-09	0.432	0	0	\\
4e-09	0.432	0	0	\\
4.05e-09	0.432	0	0	\\
4.1e-09	0.432	0	0	\\
4.15e-09	0.432	0	0	\\
4.2e-09	0.432	0	0	\\
4.25e-09	0.432	0	0	\\
4.3e-09	0.432	0	0	\\
4.35e-09	0.432	0	0	\\
4.4e-09	0.432	-0.444444444444444	-0.444444444444444	\\
4.45e-09	0.432	-0.444444444444444	-0.444444444444444	\\
4.5e-09	0.432	-0.444444444444444	-0.444444444444444	\\
4.55e-09	0.432	-0.444444444444444	-0.444444444444444	\\
4.6e-09	0.432	-0.444444444444444	-0.444444444444444	\\
4.65e-09	0.432	-0.444444444444444	-0.444444444444444	\\
4.7e-09	0.432	-0.444444444444444	-0.444444444444444	\\
4.75e-09	0.432	-0.444444444444444	-0.444444444444444	\\
4.8e-09	0.432	-0.444444444444444	-0.444444444444444	\\
4.85e-09	0.432	-0.444444444444444	-0.444444444444444	\\
4.9e-09	0.432	0	0	\\
4.95e-09	0.432	0	0	\\
5e-09	0.432	0	0	\\
5e-09	0.432	0	nan	\\
5e-09	0.432	-0.666666666666667	0.166666666666667	\\
5e-09	0.432	-0.666666666666667	nan	\\
0	0.444	-0.666666666666667	nan	\\
0	0.444	0	0.166666666666667	\\
0	0.444	0	0	\\
5e-11	0.444	0	0	\\
1e-10	0.444	0	0	\\
1.5e-10	0.444	0	0	\\
2e-10	0.444	0	0	\\
2.5e-10	0.444	0	0	\\
3e-10	0.444	0	0	\\
3.5e-10	0.444	0	0	\\
4e-10	0.444	0	0	\\
4.5e-10	0.444	1	1	\\
5e-10	0.444	1	1	\\
5.5e-10	0.444	1	1	\\
6e-10	0.444	1	1	\\
6.5e-10	0.444	1	1	\\
7e-10	0.444	1	1	\\
7.5e-10	0.444	1	1	\\
8e-10	0.444	1	1	\\
8.5e-10	0.444	1	1	\\
9e-10	0.444	1	1	\\
9.5e-10	0.444	0	0	\\
1e-09	0.444	0	0	\\
1.05e-09	0.444	0	0	\\
1.1e-09	0.444	0	0	\\
1.15e-09	0.444	0	0	\\
1.2e-09	0.444	0	0	\\
1.25e-09	0.444	0	0	\\
1.3e-09	0.444	0	0	\\
1.35e-09	0.444	0	0	\\
1.4e-09	0.444	0	0	\\
1.45e-09	0.444	0	0	\\
1.5e-09	0.444	0	0	\\
1.55e-09	0.444	0	0	\\
1.6e-09	0.444	0	0	\\
1.65e-09	0.444	0	0	\\
1.7e-09	0.444	0	0	\\
1.75e-09	0.444	0	0	\\
1.8e-09	0.444	0	0	\\
1.85e-09	0.444	0	0	\\
1.9e-09	0.444	0	0	\\
1.95e-09	0.444	0	0	\\
2e-09	0.444	-0.666666666666667	-0.666666666666667	\\
2.05e-09	0.444	-0.666666666666667	-0.666666666666667	\\
2.1e-09	0.444	-0.666666666666667	-0.666666666666667	\\
2.15e-09	0.444	-0.666666666666667	-0.666666666666667	\\
2.2e-09	0.444	-0.666666666666667	-0.666666666666667	\\
2.25e-09	0.444	-0.666666666666667	-0.666666666666667	\\
2.3e-09	0.444	-0.666666666666667	-0.666666666666667	\\
2.35e-09	0.444	-0.666666666666667	-0.666666666666667	\\
2.4e-09	0.444	-0.666666666666667	-0.666666666666667	\\
2.45e-09	0.444	-0.666666666666667	-0.666666666666667	\\
2.5e-09	0.444	0	0	\\
2.55e-09	0.444	0	0	\\
2.6e-09	0.444	0	0	\\
2.65e-09	0.444	0	0	\\
2.7e-09	0.444	0	0	\\
2.75e-09	0.444	0	0	\\
2.8e-09	0.444	0	0	\\
2.85e-09	0.444	0.666666666666667	0.666666666666667	\\
2.9e-09	0.444	0.666666666666667	0.666666666666667	\\
2.95e-09	0.444	0.666666666666667	0.666666666666667	\\
3e-09	0.444	0.666666666666667	0.666666666666667	\\
3.05e-09	0.444	0.666666666666667	0.666666666666667	\\
3.1e-09	0.444	0.666666666666667	0.666666666666667	\\
3.15e-09	0.444	0.666666666666667	0.666666666666667	\\
3.2e-09	0.444	0.666666666666667	0.666666666666667	\\
3.25e-09	0.444	0.666666666666667	0.666666666666667	\\
3.3e-09	0.444	0.666666666666667	0.666666666666667	\\
3.35e-09	0.444	0	0	\\
3.4e-09	0.444	0	0	\\
3.45e-09	0.444	0	0	\\
3.5e-09	0.444	0	0	\\
3.55e-09	0.444	0	0	\\
3.6e-09	0.444	0	0	\\
3.65e-09	0.444	0	0	\\
3.7e-09	0.444	0	0	\\
3.75e-09	0.444	0	0	\\
3.8e-09	0.444	0	0	\\
3.85e-09	0.444	0	0	\\
3.9e-09	0.444	0	0	\\
3.95e-09	0.444	0	0	\\
4e-09	0.444	0	0	\\
4.05e-09	0.444	0	0	\\
4.1e-09	0.444	0	0	\\
4.15e-09	0.444	0	0	\\
4.2e-09	0.444	0	0	\\
4.25e-09	0.444	0	0	\\
4.3e-09	0.444	0	0	\\
4.35e-09	0.444	0	0	\\
4.4e-09	0.444	-0.444444444444444	-0.444444444444444	\\
4.45e-09	0.444	-0.444444444444444	-0.444444444444444	\\
4.5e-09	0.444	-0.444444444444444	-0.444444444444444	\\
4.55e-09	0.444	-0.444444444444444	-0.444444444444444	\\
4.6e-09	0.444	-0.444444444444444	-0.444444444444444	\\
4.65e-09	0.444	-0.444444444444444	-0.444444444444444	\\
4.7e-09	0.444	-0.444444444444444	-0.444444444444444	\\
4.75e-09	0.444	-0.444444444444444	-0.444444444444444	\\
4.8e-09	0.444	-0.444444444444444	-0.444444444444444	\\
4.85e-09	0.444	-0.444444444444444	-0.444444444444444	\\
4.9e-09	0.444	0	0	\\
4.95e-09	0.444	0	0	\\
5e-09	0.444	0	0	\\
5e-09	0.444	0	nan	\\
5e-09	0.444	-0.666666666666667	0.166666666666667	\\
5e-09	0.444	-0.666666666666667	nan	\\
0	0.456	-0.666666666666667	nan	\\
0	0.456	0	0.166666666666667	\\
0	0.456	0	0	\\
5e-11	0.456	0	0	\\
1e-10	0.456	0	0	\\
1.5e-10	0.456	0	0	\\
2e-10	0.456	0	0	\\
2.5e-10	0.456	0	0	\\
3e-10	0.456	0	0	\\
3.5e-10	0.456	0	0	\\
4e-10	0.456	0	0	\\
4.5e-10	0.456	0	0	\\
5e-10	0.456	1	1	\\
5.5e-10	0.456	1	1	\\
6e-10	0.456	1	1	\\
6.5e-10	0.456	1	1	\\
7e-10	0.456	1	1	\\
7.5e-10	0.456	1	1	\\
8e-10	0.456	1	1	\\
8.5e-10	0.456	1	1	\\
9e-10	0.456	1	1	\\
9.5e-10	0.456	1	1	\\
1e-09	0.456	0	0	\\
1.05e-09	0.456	0	0	\\
1.1e-09	0.456	0	0	\\
1.15e-09	0.456	0	0	\\
1.2e-09	0.456	0	0	\\
1.25e-09	0.456	0	0	\\
1.3e-09	0.456	0	0	\\
1.35e-09	0.456	0	0	\\
1.4e-09	0.456	0	0	\\
1.45e-09	0.456	0	0	\\
1.5e-09	0.456	0	0	\\
1.55e-09	0.456	0	0	\\
1.6e-09	0.456	0	0	\\
1.65e-09	0.456	0	0	\\
1.7e-09	0.456	0	0	\\
1.75e-09	0.456	0	0	\\
1.8e-09	0.456	0	0	\\
1.85e-09	0.456	0	0	\\
1.9e-09	0.456	0	0	\\
1.95e-09	0.456	-0.666666666666667	-0.666666666666667	\\
2e-09	0.456	-0.666666666666667	-0.666666666666667	\\
2.05e-09	0.456	-0.666666666666667	-0.666666666666667	\\
2.1e-09	0.456	-0.666666666666667	-0.666666666666667	\\
2.15e-09	0.456	-0.666666666666667	-0.666666666666667	\\
2.2e-09	0.456	-0.666666666666667	-0.666666666666667	\\
2.25e-09	0.456	-0.666666666666667	-0.666666666666667	\\
2.3e-09	0.456	-0.666666666666667	-0.666666666666667	\\
2.35e-09	0.456	-0.666666666666667	-0.666666666666667	\\
2.4e-09	0.456	-0.666666666666667	-0.666666666666667	\\
2.45e-09	0.456	0	0	\\
2.5e-09	0.456	0	0	\\
2.55e-09	0.456	0	0	\\
2.6e-09	0.456	0	0	\\
2.65e-09	0.456	0	0	\\
2.7e-09	0.456	0	0	\\
2.75e-09	0.456	0	0	\\
2.8e-09	0.456	0	0	\\
2.85e-09	0.456	0	0	\\
2.9e-09	0.456	0.666666666666667	0.666666666666667	\\
2.95e-09	0.456	0.666666666666667	0.666666666666667	\\
3e-09	0.456	0.666666666666667	0.666666666666667	\\
3.05e-09	0.456	0.666666666666667	0.666666666666667	\\
3.1e-09	0.456	0.666666666666667	0.666666666666667	\\
3.15e-09	0.456	0.666666666666667	0.666666666666667	\\
3.2e-09	0.456	0.666666666666667	0.666666666666667	\\
3.25e-09	0.456	0.666666666666667	0.666666666666667	\\
3.3e-09	0.456	0.666666666666667	0.666666666666667	\\
3.35e-09	0.456	0.666666666666667	0.666666666666667	\\
3.4e-09	0.456	0	0	\\
3.45e-09	0.456	0	0	\\
3.5e-09	0.456	0	0	\\
3.55e-09	0.456	0	0	\\
3.6e-09	0.456	0	0	\\
3.65e-09	0.456	0	0	\\
3.7e-09	0.456	0	0	\\
3.75e-09	0.456	0	0	\\
3.8e-09	0.456	0	0	\\
3.85e-09	0.456	0	0	\\
3.9e-09	0.456	0	0	\\
3.95e-09	0.456	0	0	\\
4e-09	0.456	0	0	\\
4.05e-09	0.456	0	0	\\
4.1e-09	0.456	0	0	\\
4.15e-09	0.456	0	0	\\
4.2e-09	0.456	0	0	\\
4.25e-09	0.456	0	0	\\
4.3e-09	0.456	0	0	\\
4.35e-09	0.456	-0.444444444444444	-0.444444444444444	\\
4.4e-09	0.456	-0.444444444444444	-0.444444444444444	\\
4.45e-09	0.456	-0.444444444444444	-0.444444444444444	\\
4.5e-09	0.456	-0.444444444444444	-0.444444444444444	\\
4.55e-09	0.456	-0.444444444444444	-0.444444444444444	\\
4.6e-09	0.456	-0.444444444444444	-0.444444444444444	\\
4.65e-09	0.456	-0.444444444444444	-0.444444444444444	\\
4.7e-09	0.456	-0.444444444444444	-0.444444444444444	\\
4.75e-09	0.456	-0.444444444444444	-0.444444444444444	\\
4.8e-09	0.456	-0.444444444444444	-0.444444444444444	\\
4.85e-09	0.456	0	0	\\
4.9e-09	0.456	0	0	\\
4.95e-09	0.456	0	0	\\
5e-09	0.456	0	0	\\
5e-09	0.456	0	nan	\\
5e-09	0.456	-0.666666666666667	0.166666666666667	\\
5e-09	0.456	-0.666666666666667	nan	\\
0	0.468	-0.666666666666667	nan	\\
0	0.468	0	0.166666666666667	\\
0	0.468	0	0	\\
5e-11	0.468	0	0	\\
1e-10	0.468	0	0	\\
1.5e-10	0.468	0	0	\\
2e-10	0.468	0	0	\\
2.5e-10	0.468	0	0	\\
3e-10	0.468	0	0	\\
3.5e-10	0.468	0	0	\\
4e-10	0.468	0	0	\\
4.5e-10	0.468	0	0	\\
5e-10	0.468	1	1	\\
5.5e-10	0.468	1	1	\\
6e-10	0.468	1	1	\\
6.5e-10	0.468	1	1	\\
7e-10	0.468	1	1	\\
7.5e-10	0.468	1	1	\\
8e-10	0.468	1	1	\\
8.5e-10	0.468	1	1	\\
9e-10	0.468	1	1	\\
9.5e-10	0.468	1	1	\\
1e-09	0.468	0	0	\\
1.05e-09	0.468	0	0	\\
1.1e-09	0.468	0	0	\\
1.15e-09	0.468	0	0	\\
1.2e-09	0.468	0	0	\\
1.25e-09	0.468	0	0	\\
1.3e-09	0.468	0	0	\\
1.35e-09	0.468	0	0	\\
1.4e-09	0.468	0	0	\\
1.45e-09	0.468	0	0	\\
1.5e-09	0.468	0	0	\\
1.55e-09	0.468	0	0	\\
1.6e-09	0.468	0	0	\\
1.65e-09	0.468	0	0	\\
1.7e-09	0.468	0	0	\\
1.75e-09	0.468	0	0	\\
1.8e-09	0.468	0	0	\\
1.85e-09	0.468	0	0	\\
1.9e-09	0.468	0	0	\\
1.95e-09	0.468	-0.666666666666667	-0.666666666666667	\\
2e-09	0.468	-0.666666666666667	-0.666666666666667	\\
2.05e-09	0.468	-0.666666666666667	-0.666666666666667	\\
2.1e-09	0.468	-0.666666666666667	-0.666666666666667	\\
2.15e-09	0.468	-0.666666666666667	-0.666666666666667	\\
2.2e-09	0.468	-0.666666666666667	-0.666666666666667	\\
2.25e-09	0.468	-0.666666666666667	-0.666666666666667	\\
2.3e-09	0.468	-0.666666666666667	-0.666666666666667	\\
2.35e-09	0.468	-0.666666666666667	-0.666666666666667	\\
2.4e-09	0.468	-0.666666666666667	-0.666666666666667	\\
2.45e-09	0.468	0	0	\\
2.5e-09	0.468	0	0	\\
2.55e-09	0.468	0	0	\\
2.6e-09	0.468	0	0	\\
2.65e-09	0.468	0	0	\\
2.7e-09	0.468	0	0	\\
2.75e-09	0.468	0	0	\\
2.8e-09	0.468	0	0	\\
2.85e-09	0.468	0	0	\\
2.9e-09	0.468	0.666666666666667	0.666666666666667	\\
2.95e-09	0.468	0.666666666666667	0.666666666666667	\\
3e-09	0.468	0.666666666666667	0.666666666666667	\\
3.05e-09	0.468	0.666666666666667	0.666666666666667	\\
3.1e-09	0.468	0.666666666666667	0.666666666666667	\\
3.15e-09	0.468	0.666666666666667	0.666666666666667	\\
3.2e-09	0.468	0.666666666666667	0.666666666666667	\\
3.25e-09	0.468	0.666666666666667	0.666666666666667	\\
3.3e-09	0.468	0.666666666666667	0.666666666666667	\\
3.35e-09	0.468	0.666666666666667	0.666666666666667	\\
3.4e-09	0.468	0	0	\\
3.45e-09	0.468	0	0	\\
3.5e-09	0.468	0	0	\\
3.55e-09	0.468	0	0	\\
3.6e-09	0.468	0	0	\\
3.65e-09	0.468	0	0	\\
3.7e-09	0.468	0	0	\\
3.75e-09	0.468	0	0	\\
3.8e-09	0.468	0	0	\\
3.85e-09	0.468	0	0	\\
3.9e-09	0.468	0	0	\\
3.95e-09	0.468	0	0	\\
4e-09	0.468	0	0	\\
4.05e-09	0.468	0	0	\\
4.1e-09	0.468	0	0	\\
4.15e-09	0.468	0	0	\\
4.2e-09	0.468	0	0	\\
4.25e-09	0.468	0	0	\\
4.3e-09	0.468	0	0	\\
4.35e-09	0.468	-0.444444444444444	-0.444444444444444	\\
4.4e-09	0.468	-0.444444444444444	-0.444444444444444	\\
4.45e-09	0.468	-0.444444444444444	-0.444444444444444	\\
4.5e-09	0.468	-0.444444444444444	-0.444444444444444	\\
4.55e-09	0.468	-0.444444444444444	-0.444444444444444	\\
4.6e-09	0.468	-0.444444444444444	-0.444444444444444	\\
4.65e-09	0.468	-0.444444444444444	-0.444444444444444	\\
4.7e-09	0.468	-0.444444444444444	-0.444444444444444	\\
4.75e-09	0.468	-0.444444444444444	-0.444444444444444	\\
4.8e-09	0.468	-0.444444444444444	-0.444444444444444	\\
4.85e-09	0.468	0	0	\\
4.9e-09	0.468	0	0	\\
4.95e-09	0.468	0	0	\\
5e-09	0.468	0	0	\\
5e-09	0.468	0	nan	\\
5e-09	0.468	-0.666666666666667	0.166666666666667	\\
5e-09	0.468	-0.666666666666667	nan	\\
0	0.48	-0.666666666666667	nan	\\
0	0.48	0	0.166666666666667	\\
0	0.48	0	0	\\
5e-11	0.48	0	0	\\
1e-10	0.48	0	0	\\
1.5e-10	0.48	0	0	\\
2e-10	0.48	0	0	\\
2.5e-10	0.48	0	0	\\
3e-10	0.48	0	0	\\
3.5e-10	0.48	0	0	\\
4e-10	0.48	0	0	\\
4.5e-10	0.48	0	0	\\
5e-10	0.48	1	1	\\
5.5e-10	0.48	1	1	\\
6e-10	0.48	1	1	\\
6.5e-10	0.48	1	1	\\
7e-10	0.48	1	1	\\
7.5e-10	0.48	1	1	\\
8e-10	0.48	1	1	\\
8.5e-10	0.48	1	1	\\
9e-10	0.48	1	1	\\
9.5e-10	0.48	1	1	\\
1e-09	0.48	0	0	\\
1.05e-09	0.48	0	0	\\
1.1e-09	0.48	0	0	\\
1.15e-09	0.48	0	0	\\
1.2e-09	0.48	0	0	\\
1.25e-09	0.48	0	0	\\
1.3e-09	0.48	0	0	\\
1.35e-09	0.48	0	0	\\
1.4e-09	0.48	0	0	\\
1.45e-09	0.48	0	0	\\
1.5e-09	0.48	0	0	\\
1.55e-09	0.48	0	0	\\
1.6e-09	0.48	0	0	\\
1.65e-09	0.48	0	0	\\
1.7e-09	0.48	0	0	\\
1.75e-09	0.48	0	0	\\
1.8e-09	0.48	0	0	\\
1.85e-09	0.48	0	0	\\
1.9e-09	0.48	0	0	\\
1.95e-09	0.48	-0.666666666666667	-0.666666666666667	\\
2e-09	0.48	-0.666666666666667	-0.666666666666667	\\
2.05e-09	0.48	-0.666666666666667	-0.666666666666667	\\
2.1e-09	0.48	-0.666666666666667	-0.666666666666667	\\
2.15e-09	0.48	-0.666666666666667	-0.666666666666667	\\
2.2e-09	0.48	-0.666666666666667	-0.666666666666667	\\
2.25e-09	0.48	-0.666666666666667	-0.666666666666667	\\
2.3e-09	0.48	-0.666666666666667	-0.666666666666667	\\
2.35e-09	0.48	-0.666666666666667	-0.666666666666667	\\
2.4e-09	0.48	-0.666666666666667	-0.666666666666667	\\
2.45e-09	0.48	0	0	\\
2.5e-09	0.48	0	0	\\
2.55e-09	0.48	0	0	\\
2.6e-09	0.48	0	0	\\
2.65e-09	0.48	0	0	\\
2.7e-09	0.48	0	0	\\
2.75e-09	0.48	0	0	\\
2.8e-09	0.48	0	0	\\
2.85e-09	0.48	0	0	\\
2.9e-09	0.48	0.666666666666667	0.666666666666667	\\
2.95e-09	0.48	0.666666666666667	0.666666666666667	\\
3e-09	0.48	0.666666666666667	0.666666666666667	\\
3.05e-09	0.48	0.666666666666667	0.666666666666667	\\
3.1e-09	0.48	0.666666666666667	0.666666666666667	\\
3.15e-09	0.48	0.666666666666667	0.666666666666667	\\
3.2e-09	0.48	0.666666666666667	0.666666666666667	\\
3.25e-09	0.48	0.666666666666667	0.666666666666667	\\
3.3e-09	0.48	0.666666666666667	0.666666666666667	\\
3.35e-09	0.48	0.666666666666667	0.666666666666667	\\
3.4e-09	0.48	0	0	\\
3.45e-09	0.48	0	0	\\
3.5e-09	0.48	0	0	\\
3.55e-09	0.48	0	0	\\
3.6e-09	0.48	0	0	\\
3.65e-09	0.48	0	0	\\
3.7e-09	0.48	0	0	\\
3.75e-09	0.48	0	0	\\
3.8e-09	0.48	0	0	\\
3.85e-09	0.48	0	0	\\
3.9e-09	0.48	0	0	\\
3.95e-09	0.48	0	0	\\
4e-09	0.48	0	0	\\
4.05e-09	0.48	0	0	\\
4.1e-09	0.48	0	0	\\
4.15e-09	0.48	0	0	\\
4.2e-09	0.48	0	0	\\
4.25e-09	0.48	0	0	\\
4.3e-09	0.48	0	0	\\
4.35e-09	0.48	-0.444444444444444	-0.444444444444444	\\
4.4e-09	0.48	-0.444444444444444	-0.444444444444444	\\
4.45e-09	0.48	-0.444444444444444	-0.444444444444444	\\
4.5e-09	0.48	-0.444444444444444	-0.444444444444444	\\
4.55e-09	0.48	-0.444444444444444	-0.444444444444444	\\
4.6e-09	0.48	-0.444444444444444	-0.444444444444444	\\
4.65e-09	0.48	-0.444444444444444	-0.444444444444444	\\
4.7e-09	0.48	-0.444444444444444	-0.444444444444444	\\
4.75e-09	0.48	-0.444444444444444	-0.444444444444444	\\
4.8e-09	0.48	-0.444444444444444	-0.444444444444444	\\
4.85e-09	0.48	0	0	\\
4.9e-09	0.48	0	0	\\
4.95e-09	0.48	0	0	\\
5e-09	0.48	0	0	\\
5e-09	0.48	0	nan	\\
5e-09	0.48	-0.666666666666667	0.166666666666667	\\
5e-09	0.48	-0.666666666666667	nan	\\
0	0.492	-0.666666666666667	nan	\\
0	0.492	0	0.166666666666667	\\
0	0.492	0	0	\\
5e-11	0.492	0	0	\\
1e-10	0.492	0	0	\\
1.5e-10	0.492	0	0	\\
2e-10	0.492	0	0	\\
2.5e-10	0.492	0	0	\\
3e-10	0.492	0	0	\\
3.5e-10	0.492	0	0	\\
4e-10	0.492	0	0	\\
4.5e-10	0.492	0	0	\\
5e-10	0.492	1	1	\\
5.5e-10	0.492	1	1	\\
6e-10	0.492	1	1	\\
6.5e-10	0.492	1	1	\\
7e-10	0.492	1	1	\\
7.5e-10	0.492	1	1	\\
8e-10	0.492	1	1	\\
8.5e-10	0.492	1	1	\\
9e-10	0.492	1	1	\\
9.5e-10	0.492	1	1	\\
1e-09	0.492	0	0	\\
1.05e-09	0.492	0	0	\\
1.1e-09	0.492	0	0	\\
1.15e-09	0.492	0	0	\\
1.2e-09	0.492	0	0	\\
1.25e-09	0.492	0	0	\\
1.3e-09	0.492	0	0	\\
1.35e-09	0.492	0	0	\\
1.4e-09	0.492	0	0	\\
1.45e-09	0.492	0	0	\\
1.5e-09	0.492	0	0	\\
1.55e-09	0.492	0	0	\\
1.6e-09	0.492	0	0	\\
1.65e-09	0.492	0	0	\\
1.7e-09	0.492	0	0	\\
1.75e-09	0.492	0	0	\\
1.8e-09	0.492	0	0	\\
1.85e-09	0.492	0	0	\\
1.9e-09	0.492	0	0	\\
1.95e-09	0.492	-0.666666666666667	-0.666666666666667	\\
2e-09	0.492	-0.666666666666667	-0.666666666666667	\\
2.05e-09	0.492	-0.666666666666667	-0.666666666666667	\\
2.1e-09	0.492	-0.666666666666667	-0.666666666666667	\\
2.15e-09	0.492	-0.666666666666667	-0.666666666666667	\\
2.2e-09	0.492	-0.666666666666667	-0.666666666666667	\\
2.25e-09	0.492	-0.666666666666667	-0.666666666666667	\\
2.3e-09	0.492	-0.666666666666667	-0.666666666666667	\\
2.35e-09	0.492	-0.666666666666667	-0.666666666666667	\\
2.4e-09	0.492	-0.666666666666667	-0.666666666666667	\\
2.45e-09	0.492	0	0	\\
2.5e-09	0.492	0	0	\\
2.55e-09	0.492	0	0	\\
2.6e-09	0.492	0	0	\\
2.65e-09	0.492	0	0	\\
2.7e-09	0.492	0	0	\\
2.75e-09	0.492	0	0	\\
2.8e-09	0.492	0	0	\\
2.85e-09	0.492	0	0	\\
2.9e-09	0.492	0.666666666666667	0.666666666666667	\\
2.95e-09	0.492	0.666666666666667	0.666666666666667	\\
3e-09	0.492	0.666666666666667	0.666666666666667	\\
3.05e-09	0.492	0.666666666666667	0.666666666666667	\\
3.1e-09	0.492	0.666666666666667	0.666666666666667	\\
3.15e-09	0.492	0.666666666666667	0.666666666666667	\\
3.2e-09	0.492	0.666666666666667	0.666666666666667	\\
3.25e-09	0.492	0.666666666666667	0.666666666666667	\\
3.3e-09	0.492	0.666666666666667	0.666666666666667	\\
3.35e-09	0.492	0.666666666666667	0.666666666666667	\\
3.4e-09	0.492	0	0	\\
3.45e-09	0.492	0	0	\\
3.5e-09	0.492	0	0	\\
3.55e-09	0.492	0	0	\\
3.6e-09	0.492	0	0	\\
3.65e-09	0.492	0	0	\\
3.7e-09	0.492	0	0	\\
3.75e-09	0.492	0	0	\\
3.8e-09	0.492	0	0	\\
3.85e-09	0.492	0	0	\\
3.9e-09	0.492	0	0	\\
3.95e-09	0.492	0	0	\\
4e-09	0.492	0	0	\\
4.05e-09	0.492	0	0	\\
4.1e-09	0.492	0	0	\\
4.15e-09	0.492	0	0	\\
4.2e-09	0.492	0	0	\\
4.25e-09	0.492	0	0	\\
4.3e-09	0.492	0	0	\\
4.35e-09	0.492	-0.444444444444444	-0.444444444444444	\\
4.4e-09	0.492	-0.444444444444444	-0.444444444444444	\\
4.45e-09	0.492	-0.444444444444444	-0.444444444444444	\\
4.5e-09	0.492	-0.444444444444444	-0.444444444444444	\\
4.55e-09	0.492	-0.444444444444444	-0.444444444444444	\\
4.6e-09	0.492	-0.444444444444444	-0.444444444444444	\\
4.65e-09	0.492	-0.444444444444444	-0.444444444444444	\\
4.7e-09	0.492	-0.444444444444444	-0.444444444444444	\\
4.75e-09	0.492	-0.444444444444444	-0.444444444444444	\\
4.8e-09	0.492	-0.444444444444444	-0.444444444444444	\\
4.85e-09	0.492	0	0	\\
4.9e-09	0.492	0	0	\\
4.95e-09	0.492	0	0	\\
5e-09	0.492	0	0	\\
5e-09	0.492	0	nan	\\
5e-09	0.492	-0.666666666666667	0.166666666666667	\\
5e-09	0.492	-0.666666666666667	nan	\\
0	0.504	-0.666666666666667	nan	\\
0	0.504	0	0.166666666666667	\\
0	0.504	0	0	\\
5e-11	0.504	0	0	\\
1e-10	0.504	0	0	\\
1.5e-10	0.504	0	0	\\
2e-10	0.504	0	0	\\
2.5e-10	0.504	0	0	\\
3e-10	0.504	0	0	\\
3.5e-10	0.504	0	0	\\
4e-10	0.504	0	0	\\
4.5e-10	0.504	0	0	\\
5e-10	0.504	0	0	\\
5.5e-10	0.504	1	1	\\
6e-10	0.504	1	1	\\
6.5e-10	0.504	1	1	\\
7e-10	0.504	1	1	\\
7.5e-10	0.504	1	1	\\
8e-10	0.504	1	1	\\
8.5e-10	0.504	1	1	\\
9e-10	0.504	1	1	\\
9.5e-10	0.504	1	1	\\
1e-09	0.504	1	1	\\
1.05e-09	0.504	0	0	\\
1.1e-09	0.504	0	0	\\
1.15e-09	0.504	0	0	\\
1.2e-09	0.504	0	0	\\
1.25e-09	0.504	0	0	\\
1.3e-09	0.504	0	0	\\
1.35e-09	0.504	0	0	\\
1.4e-09	0.504	0	0	\\
1.45e-09	0.504	0	0	\\
1.5e-09	0.504	0	0	\\
1.55e-09	0.504	0	0	\\
1.6e-09	0.504	0	0	\\
1.65e-09	0.504	0	0	\\
1.7e-09	0.504	0	0	\\
1.75e-09	0.504	0	0	\\
1.8e-09	0.504	0	0	\\
1.85e-09	0.504	0	0	\\
1.9e-09	0.504	-0.666666666666667	-0.666666666666667	\\
1.95e-09	0.504	-0.666666666666667	-0.666666666666667	\\
2e-09	0.504	-0.666666666666667	-0.666666666666667	\\
2.05e-09	0.504	-0.666666666666667	-0.666666666666667	\\
2.1e-09	0.504	-0.666666666666667	-0.666666666666667	\\
2.15e-09	0.504	-0.666666666666667	-0.666666666666667	\\
2.2e-09	0.504	-0.666666666666667	-0.666666666666667	\\
2.25e-09	0.504	-0.666666666666667	-0.666666666666667	\\
2.3e-09	0.504	-0.666666666666667	-0.666666666666667	\\
2.35e-09	0.504	-0.666666666666667	-0.666666666666667	\\
2.4e-09	0.504	0	0	\\
2.45e-09	0.504	0	0	\\
2.5e-09	0.504	0	0	\\
2.55e-09	0.504	0	0	\\
2.6e-09	0.504	0	0	\\
2.65e-09	0.504	0	0	\\
2.7e-09	0.504	0	0	\\
2.75e-09	0.504	0	0	\\
2.8e-09	0.504	0	0	\\
2.85e-09	0.504	0	0	\\
2.9e-09	0.504	0	0	\\
2.95e-09	0.504	0.666666666666667	0.666666666666667	\\
3e-09	0.504	0.666666666666667	0.666666666666667	\\
3.05e-09	0.504	0.666666666666667	0.666666666666667	\\
3.1e-09	0.504	0.666666666666667	0.666666666666667	\\
3.15e-09	0.504	0.666666666666667	0.666666666666667	\\
3.2e-09	0.504	0.666666666666667	0.666666666666667	\\
3.25e-09	0.504	0.666666666666667	0.666666666666667	\\
3.3e-09	0.504	0.666666666666667	0.666666666666667	\\
3.35e-09	0.504	0.666666666666667	0.666666666666667	\\
3.4e-09	0.504	0.666666666666667	0.666666666666667	\\
3.45e-09	0.504	0	0	\\
3.5e-09	0.504	0	0	\\
3.55e-09	0.504	0	0	\\
3.6e-09	0.504	0	0	\\
3.65e-09	0.504	0	0	\\
3.7e-09	0.504	0	0	\\
3.75e-09	0.504	0	0	\\
3.8e-09	0.504	0	0	\\
3.85e-09	0.504	0	0	\\
3.9e-09	0.504	0	0	\\
3.95e-09	0.504	0	0	\\
4e-09	0.504	0	0	\\
4.05e-09	0.504	0	0	\\
4.1e-09	0.504	0	0	\\
4.15e-09	0.504	0	0	\\
4.2e-09	0.504	0	0	\\
4.25e-09	0.504	0	0	\\
4.3e-09	0.504	-0.444444444444444	-0.444444444444444	\\
4.35e-09	0.504	-0.444444444444444	-0.444444444444444	\\
4.4e-09	0.504	-0.444444444444444	-0.444444444444444	\\
4.45e-09	0.504	-0.444444444444444	-0.444444444444444	\\
4.5e-09	0.504	-0.444444444444444	-0.444444444444444	\\
4.55e-09	0.504	-0.444444444444444	-0.444444444444444	\\
4.6e-09	0.504	-0.444444444444444	-0.444444444444444	\\
4.65e-09	0.504	-0.444444444444444	-0.444444444444444	\\
4.7e-09	0.504	-0.444444444444444	-0.444444444444444	\\
4.75e-09	0.504	-0.444444444444444	-0.444444444444444	\\
4.8e-09	0.504	0	0	\\
4.85e-09	0.504	0	0	\\
4.9e-09	0.504	0	0	\\
4.95e-09	0.504	0	0	\\
5e-09	0.504	0	0	\\
5e-09	0.504	0	nan	\\
5e-09	0.504	-0.666666666666667	0.166666666666667	\\
5e-09	0.504	-0.666666666666667	nan	\\
0	0.516	-0.666666666666667	nan	\\
0	0.516	0	0.166666666666667	\\
0	0.516	0	0	\\
5e-11	0.516	0	0	\\
1e-10	0.516	0	0	\\
1.5e-10	0.516	0	0	\\
2e-10	0.516	0	0	\\
2.5e-10	0.516	0	0	\\
3e-10	0.516	0	0	\\
3.5e-10	0.516	0	0	\\
4e-10	0.516	0	0	\\
4.5e-10	0.516	0	0	\\
5e-10	0.516	0	0	\\
5.5e-10	0.516	1	1	\\
6e-10	0.516	1	1	\\
6.5e-10	0.516	1	1	\\
7e-10	0.516	1	1	\\
7.5e-10	0.516	1	1	\\
8e-10	0.516	1	1	\\
8.5e-10	0.516	1	1	\\
9e-10	0.516	1	1	\\
9.5e-10	0.516	1	1	\\
1e-09	0.516	1	1	\\
1.05e-09	0.516	0	0	\\
1.1e-09	0.516	0	0	\\
1.15e-09	0.516	0	0	\\
1.2e-09	0.516	0	0	\\
1.25e-09	0.516	0	0	\\
1.3e-09	0.516	0	0	\\
1.35e-09	0.516	0	0	\\
1.4e-09	0.516	0	0	\\
1.45e-09	0.516	0	0	\\
1.5e-09	0.516	0	0	\\
1.55e-09	0.516	0	0	\\
1.6e-09	0.516	0	0	\\
1.65e-09	0.516	0	0	\\
1.7e-09	0.516	0	0	\\
1.75e-09	0.516	0	0	\\
1.8e-09	0.516	0	0	\\
1.85e-09	0.516	0	0	\\
1.9e-09	0.516	-0.666666666666667	-0.666666666666667	\\
1.95e-09	0.516	-0.666666666666667	-0.666666666666667	\\
2e-09	0.516	-0.666666666666667	-0.666666666666667	\\
2.05e-09	0.516	-0.666666666666667	-0.666666666666667	\\
2.1e-09	0.516	-0.666666666666667	-0.666666666666667	\\
2.15e-09	0.516	-0.666666666666667	-0.666666666666667	\\
2.2e-09	0.516	-0.666666666666667	-0.666666666666667	\\
2.25e-09	0.516	-0.666666666666667	-0.666666666666667	\\
2.3e-09	0.516	-0.666666666666667	-0.666666666666667	\\
2.35e-09	0.516	-0.666666666666667	-0.666666666666667	\\
2.4e-09	0.516	0	0	\\
2.45e-09	0.516	0	0	\\
2.5e-09	0.516	0	0	\\
2.55e-09	0.516	0	0	\\
2.6e-09	0.516	0	0	\\
2.65e-09	0.516	0	0	\\
2.7e-09	0.516	0	0	\\
2.75e-09	0.516	0	0	\\
2.8e-09	0.516	0	0	\\
2.85e-09	0.516	0	0	\\
2.9e-09	0.516	0	0	\\
2.95e-09	0.516	0.666666666666667	0.666666666666667	\\
3e-09	0.516	0.666666666666667	0.666666666666667	\\
3.05e-09	0.516	0.666666666666667	0.666666666666667	\\
3.1e-09	0.516	0.666666666666667	0.666666666666667	\\
3.15e-09	0.516	0.666666666666667	0.666666666666667	\\
3.2e-09	0.516	0.666666666666667	0.666666666666667	\\
3.25e-09	0.516	0.666666666666667	0.666666666666667	\\
3.3e-09	0.516	0.666666666666667	0.666666666666667	\\
3.35e-09	0.516	0.666666666666667	0.666666666666667	\\
3.4e-09	0.516	0.666666666666667	0.666666666666667	\\
3.45e-09	0.516	0	0	\\
3.5e-09	0.516	0	0	\\
3.55e-09	0.516	0	0	\\
3.6e-09	0.516	0	0	\\
3.65e-09	0.516	0	0	\\
3.7e-09	0.516	0	0	\\
3.75e-09	0.516	0	0	\\
3.8e-09	0.516	0	0	\\
3.85e-09	0.516	0	0	\\
3.9e-09	0.516	0	0	\\
3.95e-09	0.516	0	0	\\
4e-09	0.516	0	0	\\
4.05e-09	0.516	0	0	\\
4.1e-09	0.516	0	0	\\
4.15e-09	0.516	0	0	\\
4.2e-09	0.516	0	0	\\
4.25e-09	0.516	0	0	\\
4.3e-09	0.516	-0.444444444444444	-0.444444444444444	\\
4.35e-09	0.516	-0.444444444444444	-0.444444444444444	\\
4.4e-09	0.516	-0.444444444444444	-0.444444444444444	\\
4.45e-09	0.516	-0.444444444444444	-0.444444444444444	\\
4.5e-09	0.516	-0.444444444444444	-0.444444444444444	\\
4.55e-09	0.516	-0.444444444444444	-0.444444444444444	\\
4.6e-09	0.516	-0.444444444444444	-0.444444444444444	\\
4.65e-09	0.516	-0.444444444444444	-0.444444444444444	\\
4.7e-09	0.516	-0.444444444444444	-0.444444444444444	\\
4.75e-09	0.516	-0.444444444444444	-0.444444444444444	\\
4.8e-09	0.516	0	0	\\
4.85e-09	0.516	0	0	\\
4.9e-09	0.516	0	0	\\
4.95e-09	0.516	0	0	\\
5e-09	0.516	0	0	\\
5e-09	0.516	0	nan	\\
5e-09	0.516	-0.666666666666667	0.166666666666667	\\
5e-09	0.516	-0.666666666666667	nan	\\
0	0.528	-0.666666666666667	nan	\\
0	0.528	0	0.166666666666667	\\
0	0.528	0	0	\\
5e-11	0.528	0	0	\\
1e-10	0.528	0	0	\\
1.5e-10	0.528	0	0	\\
2e-10	0.528	0	0	\\
2.5e-10	0.528	0	0	\\
3e-10	0.528	0	0	\\
3.5e-10	0.528	0	0	\\
4e-10	0.528	0	0	\\
4.5e-10	0.528	0	0	\\
5e-10	0.528	0	0	\\
5.5e-10	0.528	1	1	\\
6e-10	0.528	1	1	\\
6.5e-10	0.528	1	1	\\
7e-10	0.528	1	1	\\
7.5e-10	0.528	1	1	\\
8e-10	0.528	1	1	\\
8.5e-10	0.528	1	1	\\
9e-10	0.528	1	1	\\
9.5e-10	0.528	1	1	\\
1e-09	0.528	1	1	\\
1.05e-09	0.528	0	0	\\
1.1e-09	0.528	0	0	\\
1.15e-09	0.528	0	0	\\
1.2e-09	0.528	0	0	\\
1.25e-09	0.528	0	0	\\
1.3e-09	0.528	0	0	\\
1.35e-09	0.528	0	0	\\
1.4e-09	0.528	0	0	\\
1.45e-09	0.528	0	0	\\
1.5e-09	0.528	0	0	\\
1.55e-09	0.528	0	0	\\
1.6e-09	0.528	0	0	\\
1.65e-09	0.528	0	0	\\
1.7e-09	0.528	0	0	\\
1.75e-09	0.528	0	0	\\
1.8e-09	0.528	0	0	\\
1.85e-09	0.528	0	0	\\
1.9e-09	0.528	-0.666666666666667	-0.666666666666667	\\
1.95e-09	0.528	-0.666666666666667	-0.666666666666667	\\
2e-09	0.528	-0.666666666666667	-0.666666666666667	\\
2.05e-09	0.528	-0.666666666666667	-0.666666666666667	\\
2.1e-09	0.528	-0.666666666666667	-0.666666666666667	\\
2.15e-09	0.528	-0.666666666666667	-0.666666666666667	\\
2.2e-09	0.528	-0.666666666666667	-0.666666666666667	\\
2.25e-09	0.528	-0.666666666666667	-0.666666666666667	\\
2.3e-09	0.528	-0.666666666666667	-0.666666666666667	\\
2.35e-09	0.528	-0.666666666666667	-0.666666666666667	\\
2.4e-09	0.528	0	0	\\
2.45e-09	0.528	0	0	\\
2.5e-09	0.528	0	0	\\
2.55e-09	0.528	0	0	\\
2.6e-09	0.528	0	0	\\
2.65e-09	0.528	0	0	\\
2.7e-09	0.528	0	0	\\
2.75e-09	0.528	0	0	\\
2.8e-09	0.528	0	0	\\
2.85e-09	0.528	0	0	\\
2.9e-09	0.528	0	0	\\
2.95e-09	0.528	0.666666666666667	0.666666666666667	\\
3e-09	0.528	0.666666666666667	0.666666666666667	\\
3.05e-09	0.528	0.666666666666667	0.666666666666667	\\
3.1e-09	0.528	0.666666666666667	0.666666666666667	\\
3.15e-09	0.528	0.666666666666667	0.666666666666667	\\
3.2e-09	0.528	0.666666666666667	0.666666666666667	\\
3.25e-09	0.528	0.666666666666667	0.666666666666667	\\
3.3e-09	0.528	0.666666666666667	0.666666666666667	\\
3.35e-09	0.528	0.666666666666667	0.666666666666667	\\
3.4e-09	0.528	0.666666666666667	0.666666666666667	\\
3.45e-09	0.528	0	0	\\
3.5e-09	0.528	0	0	\\
3.55e-09	0.528	0	0	\\
3.6e-09	0.528	0	0	\\
3.65e-09	0.528	0	0	\\
3.7e-09	0.528	0	0	\\
3.75e-09	0.528	0	0	\\
3.8e-09	0.528	0	0	\\
3.85e-09	0.528	0	0	\\
3.9e-09	0.528	0	0	\\
3.95e-09	0.528	0	0	\\
4e-09	0.528	0	0	\\
4.05e-09	0.528	0	0	\\
4.1e-09	0.528	0	0	\\
4.15e-09	0.528	0	0	\\
4.2e-09	0.528	0	0	\\
4.25e-09	0.528	0	0	\\
4.3e-09	0.528	-0.444444444444444	-0.444444444444444	\\
4.35e-09	0.528	-0.444444444444444	-0.444444444444444	\\
4.4e-09	0.528	-0.444444444444444	-0.444444444444444	\\
4.45e-09	0.528	-0.444444444444444	-0.444444444444444	\\
4.5e-09	0.528	-0.444444444444444	-0.444444444444444	\\
4.55e-09	0.528	-0.444444444444444	-0.444444444444444	\\
4.6e-09	0.528	-0.444444444444444	-0.444444444444444	\\
4.65e-09	0.528	-0.444444444444444	-0.444444444444444	\\
4.7e-09	0.528	-0.444444444444444	-0.444444444444444	\\
4.75e-09	0.528	-0.444444444444444	-0.444444444444444	\\
4.8e-09	0.528	0	0	\\
4.85e-09	0.528	0	0	\\
4.9e-09	0.528	0	0	\\
4.95e-09	0.528	0	0	\\
5e-09	0.528	0	0	\\
5e-09	0.528	0	nan	\\
5e-09	0.528	-0.666666666666667	0.166666666666667	\\
5e-09	0.528	-0.666666666666667	nan	\\
0	0.54	-0.666666666666667	nan	\\
0	0.54	0	0.166666666666667	\\
0	0.54	0	0	\\
5e-11	0.54	0	0	\\
1e-10	0.54	0	0	\\
1.5e-10	0.54	0	0	\\
2e-10	0.54	0	0	\\
2.5e-10	0.54	0	0	\\
3e-10	0.54	0	0	\\
3.5e-10	0.54	0	0	\\
4e-10	0.54	0	0	\\
4.5e-10	0.54	0	0	\\
5e-10	0.54	0	0	\\
5.5e-10	0.54	1	1	\\
6e-10	0.54	1	1	\\
6.5e-10	0.54	1	1	\\
7e-10	0.54	1	1	\\
7.5e-10	0.54	1	1	\\
8e-10	0.54	1	1	\\
8.5e-10	0.54	1	1	\\
9e-10	0.54	1	1	\\
9.5e-10	0.54	1	1	\\
1e-09	0.54	1	1	\\
1.05e-09	0.54	0	0	\\
1.1e-09	0.54	0	0	\\
1.15e-09	0.54	0	0	\\
1.2e-09	0.54	0	0	\\
1.25e-09	0.54	0	0	\\
1.3e-09	0.54	0	0	\\
1.35e-09	0.54	0	0	\\
1.4e-09	0.54	0	0	\\
1.45e-09	0.54	0	0	\\
1.5e-09	0.54	0	0	\\
1.55e-09	0.54	0	0	\\
1.6e-09	0.54	0	0	\\
1.65e-09	0.54	0	0	\\
1.7e-09	0.54	0	0	\\
1.75e-09	0.54	0	0	\\
1.8e-09	0.54	0	0	\\
1.85e-09	0.54	0	0	\\
1.9e-09	0.54	-0.666666666666667	-0.666666666666667	\\
1.95e-09	0.54	-0.666666666666667	-0.666666666666667	\\
2e-09	0.54	-0.666666666666667	-0.666666666666667	\\
2.05e-09	0.54	-0.666666666666667	-0.666666666666667	\\
2.1e-09	0.54	-0.666666666666667	-0.666666666666667	\\
2.15e-09	0.54	-0.666666666666667	-0.666666666666667	\\
2.2e-09	0.54	-0.666666666666667	-0.666666666666667	\\
2.25e-09	0.54	-0.666666666666667	-0.666666666666667	\\
2.3e-09	0.54	-0.666666666666667	-0.666666666666667	\\
2.35e-09	0.54	-0.666666666666667	-0.666666666666667	\\
2.4e-09	0.54	0	0	\\
2.45e-09	0.54	0	0	\\
2.5e-09	0.54	0	0	\\
2.55e-09	0.54	0	0	\\
2.6e-09	0.54	0	0	\\
2.65e-09	0.54	0	0	\\
2.7e-09	0.54	0	0	\\
2.75e-09	0.54	0	0	\\
2.8e-09	0.54	0	0	\\
2.85e-09	0.54	0	0	\\
2.9e-09	0.54	0	0	\\
2.95e-09	0.54	0.666666666666667	0.666666666666667	\\
3e-09	0.54	0.666666666666667	0.666666666666667	\\
3.05e-09	0.54	0.666666666666667	0.666666666666667	\\
3.1e-09	0.54	0.666666666666667	0.666666666666667	\\
3.15e-09	0.54	0.666666666666667	0.666666666666667	\\
3.2e-09	0.54	0.666666666666667	0.666666666666667	\\
3.25e-09	0.54	0.666666666666667	0.666666666666667	\\
3.3e-09	0.54	0.666666666666667	0.666666666666667	\\
3.35e-09	0.54	0.666666666666667	0.666666666666667	\\
3.4e-09	0.54	0.666666666666667	0.666666666666667	\\
3.45e-09	0.54	0	0	\\
3.5e-09	0.54	0	0	\\
3.55e-09	0.54	0	0	\\
3.6e-09	0.54	0	0	\\
3.65e-09	0.54	0	0	\\
3.7e-09	0.54	0	0	\\
3.75e-09	0.54	0	0	\\
3.8e-09	0.54	0	0	\\
3.85e-09	0.54	0	0	\\
3.9e-09	0.54	0	0	\\
3.95e-09	0.54	0	0	\\
4e-09	0.54	0	0	\\
4.05e-09	0.54	0	0	\\
4.1e-09	0.54	0	0	\\
4.15e-09	0.54	0	0	\\
4.2e-09	0.54	0	0	\\
4.25e-09	0.54	0	0	\\
4.3e-09	0.54	-0.444444444444444	-0.444444444444444	\\
4.35e-09	0.54	-0.444444444444444	-0.444444444444444	\\
4.4e-09	0.54	-0.444444444444444	-0.444444444444444	\\
4.45e-09	0.54	-0.444444444444444	-0.444444444444444	\\
4.5e-09	0.54	-0.444444444444444	-0.444444444444444	\\
4.55e-09	0.54	-0.444444444444444	-0.444444444444444	\\
4.6e-09	0.54	-0.444444444444444	-0.444444444444444	\\
4.65e-09	0.54	-0.444444444444444	-0.444444444444444	\\
4.7e-09	0.54	-0.444444444444444	-0.444444444444444	\\
4.75e-09	0.54	-0.444444444444444	-0.444444444444444	\\
4.8e-09	0.54	0	0	\\
4.85e-09	0.54	0	0	\\
4.9e-09	0.54	0	0	\\
4.95e-09	0.54	0	0	\\
5e-09	0.54	0	0	\\
5e-09	0.54	0	nan	\\
5e-09	0.54	-0.666666666666667	0.166666666666667	\\
5e-09	0.54	-0.666666666666667	nan	\\
0	0.552	-0.666666666666667	nan	\\
0	0.552	0	0.166666666666667	\\
0	0.552	0	0	\\
5e-11	0.552	0	0	\\
1e-10	0.552	0	0	\\
1.5e-10	0.552	0	0	\\
2e-10	0.552	0	0	\\
2.5e-10	0.552	0	0	\\
3e-10	0.552	0	0	\\
3.5e-10	0.552	0	0	\\
4e-10	0.552	0	0	\\
4.5e-10	0.552	0	0	\\
5e-10	0.552	0	0	\\
5.5e-10	0.552	0	0	\\
6e-10	0.552	1	1	\\
6.5e-10	0.552	1	1	\\
7e-10	0.552	1	1	\\
7.5e-10	0.552	1	1	\\
8e-10	0.552	1	1	\\
8.5e-10	0.552	1	1	\\
9e-10	0.552	1	1	\\
9.5e-10	0.552	1	1	\\
1e-09	0.552	1	1	\\
1.05e-09	0.552	1	1	\\
1.1e-09	0.552	0	0	\\
1.15e-09	0.552	0	0	\\
1.2e-09	0.552	0	0	\\
1.25e-09	0.552	0	0	\\
1.3e-09	0.552	0	0	\\
1.35e-09	0.552	0	0	\\
1.4e-09	0.552	0	0	\\
1.45e-09	0.552	0	0	\\
1.5e-09	0.552	0	0	\\
1.55e-09	0.552	0	0	\\
1.6e-09	0.552	0	0	\\
1.65e-09	0.552	0	0	\\
1.7e-09	0.552	0	0	\\
1.75e-09	0.552	0	0	\\
1.8e-09	0.552	0	0	\\
1.85e-09	0.552	-0.666666666666667	-0.666666666666667	\\
1.9e-09	0.552	-0.666666666666667	-0.666666666666667	\\
1.95e-09	0.552	-0.666666666666667	-0.666666666666667	\\
2e-09	0.552	-0.666666666666667	-0.666666666666667	\\
2.05e-09	0.552	-0.666666666666667	-0.666666666666667	\\
2.1e-09	0.552	-0.666666666666667	-0.666666666666667	\\
2.15e-09	0.552	-0.666666666666667	-0.666666666666667	\\
2.2e-09	0.552	-0.666666666666667	-0.666666666666667	\\
2.25e-09	0.552	-0.666666666666667	-0.666666666666667	\\
2.3e-09	0.552	-0.666666666666667	-0.666666666666667	\\
2.35e-09	0.552	0	0	\\
2.4e-09	0.552	0	0	\\
2.45e-09	0.552	0	0	\\
2.5e-09	0.552	0	0	\\
2.55e-09	0.552	0	0	\\
2.6e-09	0.552	0	0	\\
2.65e-09	0.552	0	0	\\
2.7e-09	0.552	0	0	\\
2.75e-09	0.552	0	0	\\
2.8e-09	0.552	0	0	\\
2.85e-09	0.552	0	0	\\
2.9e-09	0.552	0	0	\\
2.95e-09	0.552	0	0	\\
3e-09	0.552	0.666666666666667	0.666666666666667	\\
3.05e-09	0.552	0.666666666666667	0.666666666666667	\\
3.1e-09	0.552	0.666666666666667	0.666666666666667	\\
3.15e-09	0.552	0.666666666666667	0.666666666666667	\\
3.2e-09	0.552	0.666666666666667	0.666666666666667	\\
3.25e-09	0.552	0.666666666666667	0.666666666666667	\\
3.3e-09	0.552	0.666666666666667	0.666666666666667	\\
3.35e-09	0.552	0.666666666666667	0.666666666666667	\\
3.4e-09	0.552	0.666666666666667	0.666666666666667	\\
3.45e-09	0.552	0.666666666666667	0.666666666666667	\\
3.5e-09	0.552	0	0	\\
3.55e-09	0.552	0	0	\\
3.6e-09	0.552	0	0	\\
3.65e-09	0.552	0	0	\\
3.7e-09	0.552	0	0	\\
3.75e-09	0.552	0	0	\\
3.8e-09	0.552	0	0	\\
3.85e-09	0.552	0	0	\\
3.9e-09	0.552	0	0	\\
3.95e-09	0.552	0	0	\\
4e-09	0.552	0	0	\\
4.05e-09	0.552	0	0	\\
4.1e-09	0.552	0	0	\\
4.15e-09	0.552	0	0	\\
4.2e-09	0.552	0	0	\\
4.25e-09	0.552	-0.444444444444444	-0.444444444444444	\\
4.3e-09	0.552	-0.444444444444444	-0.444444444444444	\\
4.35e-09	0.552	-0.444444444444444	-0.444444444444444	\\
4.4e-09	0.552	-0.444444444444444	-0.444444444444444	\\
4.45e-09	0.552	-0.444444444444444	-0.444444444444444	\\
4.5e-09	0.552	-0.444444444444444	-0.444444444444444	\\
4.55e-09	0.552	-0.444444444444444	-0.444444444444444	\\
4.6e-09	0.552	-0.444444444444444	-0.444444444444444	\\
4.65e-09	0.552	-0.444444444444444	-0.444444444444444	\\
4.7e-09	0.552	-0.444444444444444	-0.444444444444444	\\
4.75e-09	0.552	0	0	\\
4.8e-09	0.552	0	0	\\
4.85e-09	0.552	0	0	\\
4.9e-09	0.552	0	0	\\
4.95e-09	0.552	0	0	\\
5e-09	0.552	0	0	\\
5e-09	0.552	0	nan	\\
5e-09	0.552	-0.666666666666667	0.166666666666667	\\
5e-09	0.552	-0.666666666666667	nan	\\
0	0.564	-0.666666666666667	nan	\\
0	0.564	0	0.166666666666667	\\
0	0.564	0	0	\\
5e-11	0.564	0	0	\\
1e-10	0.564	0	0	\\
1.5e-10	0.564	0	0	\\
2e-10	0.564	0	0	\\
2.5e-10	0.564	0	0	\\
3e-10	0.564	0	0	\\
3.5e-10	0.564	0	0	\\
4e-10	0.564	0	0	\\
4.5e-10	0.564	0	0	\\
5e-10	0.564	0	0	\\
5.5e-10	0.564	0	0	\\
6e-10	0.564	1	1	\\
6.5e-10	0.564	1	1	\\
7e-10	0.564	1	1	\\
7.5e-10	0.564	1	1	\\
8e-10	0.564	1	1	\\
8.5e-10	0.564	1	1	\\
9e-10	0.564	1	1	\\
9.5e-10	0.564	1	1	\\
1e-09	0.564	1	1	\\
1.05e-09	0.564	1	1	\\
1.1e-09	0.564	0	0	\\
1.15e-09	0.564	0	0	\\
1.2e-09	0.564	0	0	\\
1.25e-09	0.564	0	0	\\
1.3e-09	0.564	0	0	\\
1.35e-09	0.564	0	0	\\
1.4e-09	0.564	0	0	\\
1.45e-09	0.564	0	0	\\
1.5e-09	0.564	0	0	\\
1.55e-09	0.564	0	0	\\
1.6e-09	0.564	0	0	\\
1.65e-09	0.564	0	0	\\
1.7e-09	0.564	0	0	\\
1.75e-09	0.564	0	0	\\
1.8e-09	0.564	0	0	\\
1.85e-09	0.564	-0.666666666666667	-0.666666666666667	\\
1.9e-09	0.564	-0.666666666666667	-0.666666666666667	\\
1.95e-09	0.564	-0.666666666666667	-0.666666666666667	\\
2e-09	0.564	-0.666666666666667	-0.666666666666667	\\
2.05e-09	0.564	-0.666666666666667	-0.666666666666667	\\
2.1e-09	0.564	-0.666666666666667	-0.666666666666667	\\
2.15e-09	0.564	-0.666666666666667	-0.666666666666667	\\
2.2e-09	0.564	-0.666666666666667	-0.666666666666667	\\
2.25e-09	0.564	-0.666666666666667	-0.666666666666667	\\
2.3e-09	0.564	-0.666666666666667	-0.666666666666667	\\
2.35e-09	0.564	0	0	\\
2.4e-09	0.564	0	0	\\
2.45e-09	0.564	0	0	\\
2.5e-09	0.564	0	0	\\
2.55e-09	0.564	0	0	\\
2.6e-09	0.564	0	0	\\
2.65e-09	0.564	0	0	\\
2.7e-09	0.564	0	0	\\
2.75e-09	0.564	0	0	\\
2.8e-09	0.564	0	0	\\
2.85e-09	0.564	0	0	\\
2.9e-09	0.564	0	0	\\
2.95e-09	0.564	0	0	\\
3e-09	0.564	0.666666666666667	0.666666666666667	\\
3.05e-09	0.564	0.666666666666667	0.666666666666667	\\
3.1e-09	0.564	0.666666666666667	0.666666666666667	\\
3.15e-09	0.564	0.666666666666667	0.666666666666667	\\
3.2e-09	0.564	0.666666666666667	0.666666666666667	\\
3.25e-09	0.564	0.666666666666667	0.666666666666667	\\
3.3e-09	0.564	0.666666666666667	0.666666666666667	\\
3.35e-09	0.564	0.666666666666667	0.666666666666667	\\
3.4e-09	0.564	0.666666666666667	0.666666666666667	\\
3.45e-09	0.564	0.666666666666667	0.666666666666667	\\
3.5e-09	0.564	0	0	\\
3.55e-09	0.564	0	0	\\
3.6e-09	0.564	0	0	\\
3.65e-09	0.564	0	0	\\
3.7e-09	0.564	0	0	\\
3.75e-09	0.564	0	0	\\
3.8e-09	0.564	0	0	\\
3.85e-09	0.564	0	0	\\
3.9e-09	0.564	0	0	\\
3.95e-09	0.564	0	0	\\
4e-09	0.564	0	0	\\
4.05e-09	0.564	0	0	\\
4.1e-09	0.564	0	0	\\
4.15e-09	0.564	0	0	\\
4.2e-09	0.564	0	0	\\
4.25e-09	0.564	-0.444444444444444	-0.444444444444444	\\
4.3e-09	0.564	-0.444444444444444	-0.444444444444444	\\
4.35e-09	0.564	-0.444444444444444	-0.444444444444444	\\
4.4e-09	0.564	-0.444444444444444	-0.444444444444444	\\
4.45e-09	0.564	-0.444444444444444	-0.444444444444444	\\
4.5e-09	0.564	-0.444444444444444	-0.444444444444444	\\
4.55e-09	0.564	-0.444444444444444	-0.444444444444444	\\
4.6e-09	0.564	-0.444444444444444	-0.444444444444444	\\
4.65e-09	0.564	-0.444444444444444	-0.444444444444444	\\
4.7e-09	0.564	-0.444444444444444	-0.444444444444444	\\
4.75e-09	0.564	0	0	\\
4.8e-09	0.564	0	0	\\
4.85e-09	0.564	0	0	\\
4.9e-09	0.564	0	0	\\
4.95e-09	0.564	0	0	\\
5e-09	0.564	0	0	\\
5e-09	0.564	0	nan	\\
5e-09	0.564	-0.666666666666667	0.166666666666667	\\
5e-09	0.564	-0.666666666666667	nan	\\
0	0.576	-0.666666666666667	nan	\\
0	0.576	0	0.166666666666667	\\
0	0.576	0	0	\\
5e-11	0.576	0	0	\\
1e-10	0.576	0	0	\\
1.5e-10	0.576	0	0	\\
2e-10	0.576	0	0	\\
2.5e-10	0.576	0	0	\\
3e-10	0.576	0	0	\\
3.5e-10	0.576	0	0	\\
4e-10	0.576	0	0	\\
4.5e-10	0.576	0	0	\\
5e-10	0.576	0	0	\\
5.5e-10	0.576	0	0	\\
6e-10	0.576	1	1	\\
6.5e-10	0.576	1	1	\\
7e-10	0.576	1	1	\\
7.5e-10	0.576	1	1	\\
8e-10	0.576	1	1	\\
8.5e-10	0.576	1	1	\\
9e-10	0.576	1	1	\\
9.5e-10	0.576	1	1	\\
1e-09	0.576	1	1	\\
1.05e-09	0.576	1	1	\\
1.1e-09	0.576	0	0	\\
1.15e-09	0.576	0	0	\\
1.2e-09	0.576	0	0	\\
1.25e-09	0.576	0	0	\\
1.3e-09	0.576	0	0	\\
1.35e-09	0.576	0	0	\\
1.4e-09	0.576	0	0	\\
1.45e-09	0.576	0	0	\\
1.5e-09	0.576	0	0	\\
1.55e-09	0.576	0	0	\\
1.6e-09	0.576	0	0	\\
1.65e-09	0.576	0	0	\\
1.7e-09	0.576	0	0	\\
1.75e-09	0.576	0	0	\\
1.8e-09	0.576	0	0	\\
1.85e-09	0.576	-0.666666666666667	-0.666666666666667	\\
1.9e-09	0.576	-0.666666666666667	-0.666666666666667	\\
1.95e-09	0.576	-0.666666666666667	-0.666666666666667	\\
2e-09	0.576	-0.666666666666667	-0.666666666666667	\\
2.05e-09	0.576	-0.666666666666667	-0.666666666666667	\\
2.1e-09	0.576	-0.666666666666667	-0.666666666666667	\\
2.15e-09	0.576	-0.666666666666667	-0.666666666666667	\\
2.2e-09	0.576	-0.666666666666667	-0.666666666666667	\\
2.25e-09	0.576	-0.666666666666667	-0.666666666666667	\\
2.3e-09	0.576	-0.666666666666667	-0.666666666666667	\\
2.35e-09	0.576	0	0	\\
2.4e-09	0.576	0	0	\\
2.45e-09	0.576	0	0	\\
2.5e-09	0.576	0	0	\\
2.55e-09	0.576	0	0	\\
2.6e-09	0.576	0	0	\\
2.65e-09	0.576	0	0	\\
2.7e-09	0.576	0	0	\\
2.75e-09	0.576	0	0	\\
2.8e-09	0.576	0	0	\\
2.85e-09	0.576	0	0	\\
2.9e-09	0.576	0	0	\\
2.95e-09	0.576	0	0	\\
3e-09	0.576	0.666666666666667	0.666666666666667	\\
3.05e-09	0.576	0.666666666666667	0.666666666666667	\\
3.1e-09	0.576	0.666666666666667	0.666666666666667	\\
3.15e-09	0.576	0.666666666666667	0.666666666666667	\\
3.2e-09	0.576	0.666666666666667	0.666666666666667	\\
3.25e-09	0.576	0.666666666666667	0.666666666666667	\\
3.3e-09	0.576	0.666666666666667	0.666666666666667	\\
3.35e-09	0.576	0.666666666666667	0.666666666666667	\\
3.4e-09	0.576	0.666666666666667	0.666666666666667	\\
3.45e-09	0.576	0.666666666666667	0.666666666666667	\\
3.5e-09	0.576	0	0	\\
3.55e-09	0.576	0	0	\\
3.6e-09	0.576	0	0	\\
3.65e-09	0.576	0	0	\\
3.7e-09	0.576	0	0	\\
3.75e-09	0.576	0	0	\\
3.8e-09	0.576	0	0	\\
3.85e-09	0.576	0	0	\\
3.9e-09	0.576	0	0	\\
3.95e-09	0.576	0	0	\\
4e-09	0.576	0	0	\\
4.05e-09	0.576	0	0	\\
4.1e-09	0.576	0	0	\\
4.15e-09	0.576	0	0	\\
4.2e-09	0.576	0	0	\\
4.25e-09	0.576	-0.444444444444444	-0.444444444444444	\\
4.3e-09	0.576	-0.444444444444444	-0.444444444444444	\\
4.35e-09	0.576	-0.444444444444444	-0.444444444444444	\\
4.4e-09	0.576	-0.444444444444444	-0.444444444444444	\\
4.45e-09	0.576	-0.444444444444444	-0.444444444444444	\\
4.5e-09	0.576	-0.444444444444444	-0.444444444444444	\\
4.55e-09	0.576	-0.444444444444444	-0.444444444444444	\\
4.6e-09	0.576	-0.444444444444444	-0.444444444444444	\\
4.65e-09	0.576	-0.444444444444444	-0.444444444444444	\\
4.7e-09	0.576	-0.444444444444444	-0.444444444444444	\\
4.75e-09	0.576	0	0	\\
4.8e-09	0.576	0	0	\\
4.85e-09	0.576	0	0	\\
4.9e-09	0.576	0	0	\\
4.95e-09	0.576	0	0	\\
5e-09	0.576	0	0	\\
5e-09	0.576	0	nan	\\
5e-09	0.576	-0.666666666666667	0.166666666666667	\\
5e-09	0.576	-0.666666666666667	nan	\\
0	0.588	-0.666666666666667	nan	\\
0	0.588	0	0.166666666666667	\\
0	0.588	0	0	\\
5e-11	0.588	0	0	\\
1e-10	0.588	0	0	\\
1.5e-10	0.588	0	0	\\
2e-10	0.588	0	0	\\
2.5e-10	0.588	0	0	\\
3e-10	0.588	0	0	\\
3.5e-10	0.588	0	0	\\
4e-10	0.588	0	0	\\
4.5e-10	0.588	0	0	\\
5e-10	0.588	0	0	\\
5.5e-10	0.588	0	0	\\
6e-10	0.588	1	1	\\
6.5e-10	0.588	1	1	\\
7e-10	0.588	1	1	\\
7.5e-10	0.588	1	1	\\
8e-10	0.588	1	1	\\
8.5e-10	0.588	1	1	\\
9e-10	0.588	1	1	\\
9.5e-10	0.588	1	1	\\
1e-09	0.588	1	1	\\
1.05e-09	0.588	1	1	\\
1.1e-09	0.588	0	0	\\
1.15e-09	0.588	0	0	\\
1.2e-09	0.588	0	0	\\
1.25e-09	0.588	0	0	\\
1.3e-09	0.588	0	0	\\
1.35e-09	0.588	0	0	\\
1.4e-09	0.588	0	0	\\
1.45e-09	0.588	0	0	\\
1.5e-09	0.588	0	0	\\
1.55e-09	0.588	0	0	\\
1.6e-09	0.588	0	0	\\
1.65e-09	0.588	0	0	\\
1.7e-09	0.588	0	0	\\
1.75e-09	0.588	0	0	\\
1.8e-09	0.588	0	0	\\
1.85e-09	0.588	-0.666666666666667	-0.666666666666667	\\
1.9e-09	0.588	-0.666666666666667	-0.666666666666667	\\
1.95e-09	0.588	-0.666666666666667	-0.666666666666667	\\
2e-09	0.588	-0.666666666666667	-0.666666666666667	\\
2.05e-09	0.588	-0.666666666666667	-0.666666666666667	\\
2.1e-09	0.588	-0.666666666666667	-0.666666666666667	\\
2.15e-09	0.588	-0.666666666666667	-0.666666666666667	\\
2.2e-09	0.588	-0.666666666666667	-0.666666666666667	\\
2.25e-09	0.588	-0.666666666666667	-0.666666666666667	\\
2.3e-09	0.588	-0.666666666666667	-0.666666666666667	\\
2.35e-09	0.588	0	0	\\
2.4e-09	0.588	0	0	\\
2.45e-09	0.588	0	0	\\
2.5e-09	0.588	0	0	\\
2.55e-09	0.588	0	0	\\
2.6e-09	0.588	0	0	\\
2.65e-09	0.588	0	0	\\
2.7e-09	0.588	0	0	\\
2.75e-09	0.588	0	0	\\
2.8e-09	0.588	0	0	\\
2.85e-09	0.588	0	0	\\
2.9e-09	0.588	0	0	\\
2.95e-09	0.588	0	0	\\
3e-09	0.588	0.666666666666667	0.666666666666667	\\
3.05e-09	0.588	0.666666666666667	0.666666666666667	\\
3.1e-09	0.588	0.666666666666667	0.666666666666667	\\
3.15e-09	0.588	0.666666666666667	0.666666666666667	\\
3.2e-09	0.588	0.666666666666667	0.666666666666667	\\
3.25e-09	0.588	0.666666666666667	0.666666666666667	\\
3.3e-09	0.588	0.666666666666667	0.666666666666667	\\
3.35e-09	0.588	0.666666666666667	0.666666666666667	\\
3.4e-09	0.588	0.666666666666667	0.666666666666667	\\
3.45e-09	0.588	0.666666666666667	0.666666666666667	\\
3.5e-09	0.588	0	0	\\
3.55e-09	0.588	0	0	\\
3.6e-09	0.588	0	0	\\
3.65e-09	0.588	0	0	\\
3.7e-09	0.588	0	0	\\
3.75e-09	0.588	0	0	\\
3.8e-09	0.588	0	0	\\
3.85e-09	0.588	0	0	\\
3.9e-09	0.588	0	0	\\
3.95e-09	0.588	0	0	\\
4e-09	0.588	0	0	\\
4.05e-09	0.588	0	0	\\
4.1e-09	0.588	0	0	\\
4.15e-09	0.588	0	0	\\
4.2e-09	0.588	0	0	\\
4.25e-09	0.588	-0.444444444444444	-0.444444444444444	\\
4.3e-09	0.588	-0.444444444444444	-0.444444444444444	\\
4.35e-09	0.588	-0.444444444444444	-0.444444444444444	\\
4.4e-09	0.588	-0.444444444444444	-0.444444444444444	\\
4.45e-09	0.588	-0.444444444444444	-0.444444444444444	\\
4.5e-09	0.588	-0.444444444444444	-0.444444444444444	\\
4.55e-09	0.588	-0.444444444444444	-0.444444444444444	\\
4.6e-09	0.588	-0.444444444444444	-0.444444444444444	\\
4.65e-09	0.588	-0.444444444444444	-0.444444444444444	\\
4.7e-09	0.588	-0.444444444444444	-0.444444444444444	\\
4.75e-09	0.588	0	0	\\
4.8e-09	0.588	0	0	\\
4.85e-09	0.588	0	0	\\
4.9e-09	0.588	0	0	\\
4.95e-09	0.588	0	0	\\
5e-09	0.588	0	0	\\
5e-09	0.588	0	nan	\\
5e-09	0.588	-0.666666666666667	0.166666666666667	\\
5e-09	0.588	-0.666666666666667	nan	\\
0	0.6	-0.666666666666667	nan	\\
0	0.6	0	0.166666666666667	\\
0	0.6	0	0	\\
5e-11	0.6	0	0	\\
1e-10	0.6	0	0	\\
1.5e-10	0.6	0	0	\\
2e-10	0.6	0	0	\\
2.5e-10	0.6	0	0	\\
3e-10	0.6	0	0	\\
3.5e-10	0.6	0	0	\\
4e-10	0.6	0	0	\\
4.5e-10	0.6	0	0	\\
5e-10	0.6	0	0	\\
5.5e-10	0.6	0	0	\\
6e-10	0.6	1	1	\\
6.5e-10	0.6	1	1	\\
7e-10	0.6	1	1	\\
7.5e-10	0.6	1	1	\\
8e-10	0.6	1	1	\\
8.5e-10	0.6	1	1	\\
9e-10	0.6	1	1	\\
9.5e-10	0.6	1	1	\\
1e-09	0.6	1	1	\\
1.05e-09	0.6	1	1	\\
1.1e-09	0.6	0	0	\\
1.15e-09	0.6	0	0	\\
1.2e-09	0.6	0	0	\\
1.25e-09	0.6	0	0	\\
1.3e-09	0.6	0	0	\\
1.35e-09	0.6	0	0	\\
1.4e-09	0.6	0	0	\\
1.45e-09	0.6	0	0	\\
1.5e-09	0.6	0	0	\\
1.55e-09	0.6	0	0	\\
1.6e-09	0.6	0	0	\\
1.65e-09	0.6	0	0	\\
1.7e-09	0.6	0	0	\\
1.75e-09	0.6	0	0	\\
1.8e-09	0.6	-0.666666666666667	-0.666666666666667	\\
1.85e-09	0.6	-0.666666666666667	-0.666666666666667	\\
1.9e-09	0.6	-0.666666666666667	-0.666666666666667	\\
1.95e-09	0.6	-0.666666666666667	-0.666666666666667	\\
2e-09	0.6	-0.666666666666667	-0.666666666666667	\\
2.05e-09	0.6	-0.666666666666667	-0.666666666666667	\\
2.1e-09	0.6	-0.666666666666667	-0.666666666666667	\\
2.15e-09	0.6	-0.666666666666667	-0.666666666666667	\\
2.2e-09	0.6	-0.666666666666667	-0.666666666666667	\\
2.25e-09	0.6	-0.666666666666667	-0.666666666666667	\\
2.3e-09	0.6	0	0	\\
2.35e-09	0.6	0	0	\\
2.4e-09	0.6	0	0	\\
2.45e-09	0.6	0	0	\\
2.5e-09	0.6	0	0	\\
2.55e-09	0.6	0	0	\\
2.6e-09	0.6	0	0	\\
2.65e-09	0.6	0	0	\\
2.7e-09	0.6	0	0	\\
2.75e-09	0.6	0	0	\\
2.8e-09	0.6	0	0	\\
2.85e-09	0.6	0	0	\\
2.9e-09	0.6	0	0	\\
2.95e-09	0.6	0	0	\\
3e-09	0.6	0.666666666666667	0.666666666666667	\\
3.05e-09	0.6	0.666666666666667	0.666666666666667	\\
3.1e-09	0.6	0.666666666666667	0.666666666666667	\\
3.15e-09	0.6	0.666666666666667	0.666666666666667	\\
3.2e-09	0.6	0.666666666666667	0.666666666666667	\\
3.25e-09	0.6	0.666666666666667	0.666666666666667	\\
3.3e-09	0.6	0.666666666666667	0.666666666666667	\\
3.35e-09	0.6	0.666666666666667	0.666666666666667	\\
3.4e-09	0.6	0.666666666666667	0.666666666666667	\\
3.45e-09	0.6	0.666666666666667	0.666666666666667	\\
3.5e-09	0.6	0	0	\\
3.55e-09	0.6	0	0	\\
3.6e-09	0.6	0	0	\\
3.65e-09	0.6	0	0	\\
3.7e-09	0.6	0	0	\\
3.75e-09	0.6	0	0	\\
3.8e-09	0.6	0	0	\\
3.85e-09	0.6	0	0	\\
3.9e-09	0.6	0	0	\\
3.95e-09	0.6	0	0	\\
4e-09	0.6	0	0	\\
4.05e-09	0.6	0	0	\\
4.1e-09	0.6	0	0	\\
4.15e-09	0.6	0	0	\\
4.2e-09	0.6	-0.444444444444444	-0.444444444444444	\\
4.25e-09	0.6	-0.444444444444444	-0.444444444444444	\\
4.3e-09	0.6	-0.444444444444444	-0.444444444444444	\\
4.35e-09	0.6	-0.444444444444444	-0.444444444444444	\\
4.4e-09	0.6	-0.444444444444444	-0.444444444444444	\\
4.45e-09	0.6	-0.444444444444444	-0.444444444444444	\\
4.5e-09	0.6	-0.444444444444444	-0.444444444444444	\\
4.55e-09	0.6	-0.444444444444444	-0.444444444444444	\\
4.6e-09	0.6	-0.444444444444444	-0.444444444444444	\\
4.65e-09	0.6	-0.444444444444444	-0.444444444444444	\\
4.7e-09	0.6	0	0	\\
4.75e-09	0.6	0	0	\\
4.8e-09	0.6	0	0	\\
4.85e-09	0.6	0	0	\\
4.9e-09	0.6	0	0	\\
4.95e-09	0.6	0	0	\\
5e-09	0.6	0	0	\\
5e-09	0.6	0	nan	\\
5e-09	0.6	-0.666666666666667	0.166666666666667	\\
5e-09	0.6	-0.666666666666667	nan	\\
0	0.612	-0.666666666666667	nan	\\
0	0.612	0	0.166666666666667	\\
0	0.612	0	0	\\
5e-11	0.612	0	0	\\
1e-10	0.612	0	0	\\
1.5e-10	0.612	0	0	\\
2e-10	0.612	0	0	\\
2.5e-10	0.612	0	0	\\
3e-10	0.612	0	0	\\
3.5e-10	0.612	0	0	\\
4e-10	0.612	0	0	\\
4.5e-10	0.612	0	0	\\
5e-10	0.612	0	0	\\
5.5e-10	0.612	0	0	\\
6e-10	0.612	0	0	\\
6.5e-10	0.612	1	1	\\
7e-10	0.612	1	1	\\
7.5e-10	0.612	1	1	\\
8e-10	0.612	1	1	\\
8.5e-10	0.612	1	1	\\
9e-10	0.612	1	1	\\
9.5e-10	0.612	1	1	\\
1e-09	0.612	1	1	\\
1.05e-09	0.612	1	1	\\
1.1e-09	0.612	1	1	\\
1.15e-09	0.612	0	0	\\
1.2e-09	0.612	0	0	\\
1.25e-09	0.612	0	0	\\
1.3e-09	0.612	0	0	\\
1.35e-09	0.612	0	0	\\
1.4e-09	0.612	0	0	\\
1.45e-09	0.612	0	0	\\
1.5e-09	0.612	0	0	\\
1.55e-09	0.612	0	0	\\
1.6e-09	0.612	0	0	\\
1.65e-09	0.612	0	0	\\
1.7e-09	0.612	0	0	\\
1.75e-09	0.612	0	0	\\
1.8e-09	0.612	-0.666666666666667	-0.666666666666667	\\
1.85e-09	0.612	-0.666666666666667	-0.666666666666667	\\
1.9e-09	0.612	-0.666666666666667	-0.666666666666667	\\
1.95e-09	0.612	-0.666666666666667	-0.666666666666667	\\
2e-09	0.612	-0.666666666666667	-0.666666666666667	\\
2.05e-09	0.612	-0.666666666666667	-0.666666666666667	\\
2.1e-09	0.612	-0.666666666666667	-0.666666666666667	\\
2.15e-09	0.612	-0.666666666666667	-0.666666666666667	\\
2.2e-09	0.612	-0.666666666666667	-0.666666666666667	\\
2.25e-09	0.612	-0.666666666666667	-0.666666666666667	\\
2.3e-09	0.612	0	0	\\
2.35e-09	0.612	0	0	\\
2.4e-09	0.612	0	0	\\
2.45e-09	0.612	0	0	\\
2.5e-09	0.612	0	0	\\
2.55e-09	0.612	0	0	\\
2.6e-09	0.612	0	0	\\
2.65e-09	0.612	0	0	\\
2.7e-09	0.612	0	0	\\
2.75e-09	0.612	0	0	\\
2.8e-09	0.612	0	0	\\
2.85e-09	0.612	0	0	\\
2.9e-09	0.612	0	0	\\
2.95e-09	0.612	0	0	\\
3e-09	0.612	0	0	\\
3.05e-09	0.612	0.666666666666667	0.666666666666667	\\
3.1e-09	0.612	0.666666666666667	0.666666666666667	\\
3.15e-09	0.612	0.666666666666667	0.666666666666667	\\
3.2e-09	0.612	0.666666666666667	0.666666666666667	\\
3.25e-09	0.612	0.666666666666667	0.666666666666667	\\
3.3e-09	0.612	0.666666666666667	0.666666666666667	\\
3.35e-09	0.612	0.666666666666667	0.666666666666667	\\
3.4e-09	0.612	0.666666666666667	0.666666666666667	\\
3.45e-09	0.612	0.666666666666667	0.666666666666667	\\
3.5e-09	0.612	0.666666666666667	0.666666666666667	\\
3.55e-09	0.612	0	0	\\
3.6e-09	0.612	0	0	\\
3.65e-09	0.612	0	0	\\
3.7e-09	0.612	0	0	\\
3.75e-09	0.612	0	0	\\
3.8e-09	0.612	0	0	\\
3.85e-09	0.612	0	0	\\
3.9e-09	0.612	0	0	\\
3.95e-09	0.612	0	0	\\
4e-09	0.612	0	0	\\
4.05e-09	0.612	0	0	\\
4.1e-09	0.612	0	0	\\
4.15e-09	0.612	0	0	\\
4.2e-09	0.612	-0.444444444444444	-0.444444444444444	\\
4.25e-09	0.612	-0.444444444444444	-0.444444444444444	\\
4.3e-09	0.612	-0.444444444444444	-0.444444444444444	\\
4.35e-09	0.612	-0.444444444444444	-0.444444444444444	\\
4.4e-09	0.612	-0.444444444444444	-0.444444444444444	\\
4.45e-09	0.612	-0.444444444444444	-0.444444444444444	\\
4.5e-09	0.612	-0.444444444444444	-0.444444444444444	\\
4.55e-09	0.612	-0.444444444444444	-0.444444444444444	\\
4.6e-09	0.612	-0.444444444444444	-0.444444444444444	\\
4.65e-09	0.612	-0.444444444444444	-0.444444444444444	\\
4.7e-09	0.612	0	0	\\
4.75e-09	0.612	0	0	\\
4.8e-09	0.612	0	0	\\
4.85e-09	0.612	0	0	\\
4.9e-09	0.612	0	0	\\
4.95e-09	0.612	0	0	\\
5e-09	0.612	0	0	\\
5e-09	0.612	0	nan	\\
5e-09	0.612	-0.666666666666667	0.166666666666667	\\
5e-09	0.612	-0.666666666666667	nan	\\
0	0.624	-0.666666666666667	nan	\\
0	0.624	0	0.166666666666667	\\
0	0.624	0	0	\\
5e-11	0.624	0	0	\\
1e-10	0.624	0	0	\\
1.5e-10	0.624	0	0	\\
2e-10	0.624	0	0	\\
2.5e-10	0.624	0	0	\\
3e-10	0.624	0	0	\\
3.5e-10	0.624	0	0	\\
4e-10	0.624	0	0	\\
4.5e-10	0.624	0	0	\\
5e-10	0.624	0	0	\\
5.5e-10	0.624	0	0	\\
6e-10	0.624	0	0	\\
6.5e-10	0.624	1	1	\\
7e-10	0.624	1	1	\\
7.5e-10	0.624	1	1	\\
8e-10	0.624	1	1	\\
8.5e-10	0.624	1	1	\\
9e-10	0.624	1	1	\\
9.5e-10	0.624	1	1	\\
1e-09	0.624	1	1	\\
1.05e-09	0.624	1	1	\\
1.1e-09	0.624	1	1	\\
1.15e-09	0.624	0	0	\\
1.2e-09	0.624	0	0	\\
1.25e-09	0.624	0	0	\\
1.3e-09	0.624	0	0	\\
1.35e-09	0.624	0	0	\\
1.4e-09	0.624	0	0	\\
1.45e-09	0.624	0	0	\\
1.5e-09	0.624	0	0	\\
1.55e-09	0.624	0	0	\\
1.6e-09	0.624	0	0	\\
1.65e-09	0.624	0	0	\\
1.7e-09	0.624	0	0	\\
1.75e-09	0.624	0	0	\\
1.8e-09	0.624	-0.666666666666667	-0.666666666666667	\\
1.85e-09	0.624	-0.666666666666667	-0.666666666666667	\\
1.9e-09	0.624	-0.666666666666667	-0.666666666666667	\\
1.95e-09	0.624	-0.666666666666667	-0.666666666666667	\\
2e-09	0.624	-0.666666666666667	-0.666666666666667	\\
2.05e-09	0.624	-0.666666666666667	-0.666666666666667	\\
2.1e-09	0.624	-0.666666666666667	-0.666666666666667	\\
2.15e-09	0.624	-0.666666666666667	-0.666666666666667	\\
2.2e-09	0.624	-0.666666666666667	-0.666666666666667	\\
2.25e-09	0.624	-0.666666666666667	-0.666666666666667	\\
2.3e-09	0.624	0	0	\\
2.35e-09	0.624	0	0	\\
2.4e-09	0.624	0	0	\\
2.45e-09	0.624	0	0	\\
2.5e-09	0.624	0	0	\\
2.55e-09	0.624	0	0	\\
2.6e-09	0.624	0	0	\\
2.65e-09	0.624	0	0	\\
2.7e-09	0.624	0	0	\\
2.75e-09	0.624	0	0	\\
2.8e-09	0.624	0	0	\\
2.85e-09	0.624	0	0	\\
2.9e-09	0.624	0	0	\\
2.95e-09	0.624	0	0	\\
3e-09	0.624	0	0	\\
3.05e-09	0.624	0.666666666666667	0.666666666666667	\\
3.1e-09	0.624	0.666666666666667	0.666666666666667	\\
3.15e-09	0.624	0.666666666666667	0.666666666666667	\\
3.2e-09	0.624	0.666666666666667	0.666666666666667	\\
3.25e-09	0.624	0.666666666666667	0.666666666666667	\\
3.3e-09	0.624	0.666666666666667	0.666666666666667	\\
3.35e-09	0.624	0.666666666666667	0.666666666666667	\\
3.4e-09	0.624	0.666666666666667	0.666666666666667	\\
3.45e-09	0.624	0.666666666666667	0.666666666666667	\\
3.5e-09	0.624	0.666666666666667	0.666666666666667	\\
3.55e-09	0.624	0	0	\\
3.6e-09	0.624	0	0	\\
3.65e-09	0.624	0	0	\\
3.7e-09	0.624	0	0	\\
3.75e-09	0.624	0	0	\\
3.8e-09	0.624	0	0	\\
3.85e-09	0.624	0	0	\\
3.9e-09	0.624	0	0	\\
3.95e-09	0.624	0	0	\\
4e-09	0.624	0	0	\\
4.05e-09	0.624	0	0	\\
4.1e-09	0.624	0	0	\\
4.15e-09	0.624	0	0	\\
4.2e-09	0.624	-0.444444444444444	-0.444444444444444	\\
4.25e-09	0.624	-0.444444444444444	-0.444444444444444	\\
4.3e-09	0.624	-0.444444444444444	-0.444444444444444	\\
4.35e-09	0.624	-0.444444444444444	-0.444444444444444	\\
4.4e-09	0.624	-0.444444444444444	-0.444444444444444	\\
4.45e-09	0.624	-0.444444444444444	-0.444444444444444	\\
4.5e-09	0.624	-0.444444444444444	-0.444444444444444	\\
4.55e-09	0.624	-0.444444444444444	-0.444444444444444	\\
4.6e-09	0.624	-0.444444444444444	-0.444444444444444	\\
4.65e-09	0.624	-0.444444444444444	-0.444444444444444	\\
4.7e-09	0.624	0	0	\\
4.75e-09	0.624	0	0	\\
4.8e-09	0.624	0	0	\\
4.85e-09	0.624	0	0	\\
4.9e-09	0.624	0	0	\\
4.95e-09	0.624	0	0	\\
5e-09	0.624	0	0	\\
5e-09	0.624	0	nan	\\
5e-09	0.624	-0.666666666666667	0.166666666666667	\\
5e-09	0.624	-0.666666666666667	nan	\\
0	0.636	-0.666666666666667	nan	\\
0	0.636	0	0.166666666666667	\\
0	0.636	0	0	\\
5e-11	0.636	0	0	\\
1e-10	0.636	0	0	\\
1.5e-10	0.636	0	0	\\
2e-10	0.636	0	0	\\
2.5e-10	0.636	0	0	\\
3e-10	0.636	0	0	\\
3.5e-10	0.636	0	0	\\
4e-10	0.636	0	0	\\
4.5e-10	0.636	0	0	\\
5e-10	0.636	0	0	\\
5.5e-10	0.636	0	0	\\
6e-10	0.636	0	0	\\
6.5e-10	0.636	1	1	\\
7e-10	0.636	1	1	\\
7.5e-10	0.636	1	1	\\
8e-10	0.636	1	1	\\
8.5e-10	0.636	1	1	\\
9e-10	0.636	1	1	\\
9.5e-10	0.636	1	1	\\
1e-09	0.636	1	1	\\
1.05e-09	0.636	1	1	\\
1.1e-09	0.636	1	1	\\
1.15e-09	0.636	0	0	\\
1.2e-09	0.636	0	0	\\
1.25e-09	0.636	0	0	\\
1.3e-09	0.636	0	0	\\
1.35e-09	0.636	0	0	\\
1.4e-09	0.636	0	0	\\
1.45e-09	0.636	0	0	\\
1.5e-09	0.636	0	0	\\
1.55e-09	0.636	0	0	\\
1.6e-09	0.636	0	0	\\
1.65e-09	0.636	0	0	\\
1.7e-09	0.636	0	0	\\
1.75e-09	0.636	0	0	\\
1.8e-09	0.636	-0.666666666666667	-0.666666666666667	\\
1.85e-09	0.636	-0.666666666666667	-0.666666666666667	\\
1.9e-09	0.636	-0.666666666666667	-0.666666666666667	\\
1.95e-09	0.636	-0.666666666666667	-0.666666666666667	\\
2e-09	0.636	-0.666666666666667	-0.666666666666667	\\
2.05e-09	0.636	-0.666666666666667	-0.666666666666667	\\
2.1e-09	0.636	-0.666666666666667	-0.666666666666667	\\
2.15e-09	0.636	-0.666666666666667	-0.666666666666667	\\
2.2e-09	0.636	-0.666666666666667	-0.666666666666667	\\
2.25e-09	0.636	-0.666666666666667	-0.666666666666667	\\
2.3e-09	0.636	0	0	\\
2.35e-09	0.636	0	0	\\
2.4e-09	0.636	0	0	\\
2.45e-09	0.636	0	0	\\
2.5e-09	0.636	0	0	\\
2.55e-09	0.636	0	0	\\
2.6e-09	0.636	0	0	\\
2.65e-09	0.636	0	0	\\
2.7e-09	0.636	0	0	\\
2.75e-09	0.636	0	0	\\
2.8e-09	0.636	0	0	\\
2.85e-09	0.636	0	0	\\
2.9e-09	0.636	0	0	\\
2.95e-09	0.636	0	0	\\
3e-09	0.636	0	0	\\
3.05e-09	0.636	0.666666666666667	0.666666666666667	\\
3.1e-09	0.636	0.666666666666667	0.666666666666667	\\
3.15e-09	0.636	0.666666666666667	0.666666666666667	\\
3.2e-09	0.636	0.666666666666667	0.666666666666667	\\
3.25e-09	0.636	0.666666666666667	0.666666666666667	\\
3.3e-09	0.636	0.666666666666667	0.666666666666667	\\
3.35e-09	0.636	0.666666666666667	0.666666666666667	\\
3.4e-09	0.636	0.666666666666667	0.666666666666667	\\
3.45e-09	0.636	0.666666666666667	0.666666666666667	\\
3.5e-09	0.636	0.666666666666667	0.666666666666667	\\
3.55e-09	0.636	0	0	\\
3.6e-09	0.636	0	0	\\
3.65e-09	0.636	0	0	\\
3.7e-09	0.636	0	0	\\
3.75e-09	0.636	0	0	\\
3.8e-09	0.636	0	0	\\
3.85e-09	0.636	0	0	\\
3.9e-09	0.636	0	0	\\
3.95e-09	0.636	0	0	\\
4e-09	0.636	0	0	\\
4.05e-09	0.636	0	0	\\
4.1e-09	0.636	0	0	\\
4.15e-09	0.636	0	0	\\
4.2e-09	0.636	-0.444444444444444	-0.444444444444444	\\
4.25e-09	0.636	-0.444444444444444	-0.444444444444444	\\
4.3e-09	0.636	-0.444444444444444	-0.444444444444444	\\
4.35e-09	0.636	-0.444444444444444	-0.444444444444444	\\
4.4e-09	0.636	-0.444444444444444	-0.444444444444444	\\
4.45e-09	0.636	-0.444444444444444	-0.444444444444444	\\
4.5e-09	0.636	-0.444444444444444	-0.444444444444444	\\
4.55e-09	0.636	-0.444444444444444	-0.444444444444444	\\
4.6e-09	0.636	-0.444444444444444	-0.444444444444444	\\
4.65e-09	0.636	-0.444444444444444	-0.444444444444444	\\
4.7e-09	0.636	0	0	\\
4.75e-09	0.636	0	0	\\
4.8e-09	0.636	0	0	\\
4.85e-09	0.636	0	0	\\
4.9e-09	0.636	0	0	\\
4.95e-09	0.636	0	0	\\
5e-09	0.636	0	0	\\
5e-09	0.636	0	nan	\\
5e-09	0.636	-0.666666666666667	0.166666666666667	\\
5e-09	0.636	-0.666666666666667	nan	\\
0	0.648	-0.666666666666667	nan	\\
0	0.648	0	0.166666666666667	\\
0	0.648	0	0	\\
5e-11	0.648	0	0	\\
1e-10	0.648	0	0	\\
1.5e-10	0.648	0	0	\\
2e-10	0.648	0	0	\\
2.5e-10	0.648	0	0	\\
3e-10	0.648	0	0	\\
3.5e-10	0.648	0	0	\\
4e-10	0.648	0	0	\\
4.5e-10	0.648	0	0	\\
5e-10	0.648	0	0	\\
5.5e-10	0.648	0	0	\\
6e-10	0.648	0	0	\\
6.5e-10	0.648	1	1	\\
7e-10	0.648	1	1	\\
7.5e-10	0.648	1	1	\\
8e-10	0.648	1	1	\\
8.5e-10	0.648	1	1	\\
9e-10	0.648	1	1	\\
9.5e-10	0.648	1	1	\\
1e-09	0.648	1	1	\\
1.05e-09	0.648	1	1	\\
1.1e-09	0.648	1	1	\\
1.15e-09	0.648	0	0	\\
1.2e-09	0.648	0	0	\\
1.25e-09	0.648	0	0	\\
1.3e-09	0.648	0	0	\\
1.35e-09	0.648	0	0	\\
1.4e-09	0.648	0	0	\\
1.45e-09	0.648	0	0	\\
1.5e-09	0.648	0	0	\\
1.55e-09	0.648	0	0	\\
1.6e-09	0.648	0	0	\\
1.65e-09	0.648	0	0	\\
1.7e-09	0.648	0	0	\\
1.75e-09	0.648	0	0	\\
1.8e-09	0.648	-0.666666666666667	-0.666666666666667	\\
1.85e-09	0.648	-0.666666666666667	-0.666666666666667	\\
1.9e-09	0.648	-0.666666666666667	-0.666666666666667	\\
1.95e-09	0.648	-0.666666666666667	-0.666666666666667	\\
2e-09	0.648	-0.666666666666667	-0.666666666666667	\\
2.05e-09	0.648	-0.666666666666667	-0.666666666666667	\\
2.1e-09	0.648	-0.666666666666667	-0.666666666666667	\\
2.15e-09	0.648	-0.666666666666667	-0.666666666666667	\\
2.2e-09	0.648	-0.666666666666667	-0.666666666666667	\\
2.25e-09	0.648	-0.666666666666667	-0.666666666666667	\\
2.3e-09	0.648	0	0	\\
2.35e-09	0.648	0	0	\\
2.4e-09	0.648	0	0	\\
2.45e-09	0.648	0	0	\\
2.5e-09	0.648	0	0	\\
2.55e-09	0.648	0	0	\\
2.6e-09	0.648	0	0	\\
2.65e-09	0.648	0	0	\\
2.7e-09	0.648	0	0	\\
2.75e-09	0.648	0	0	\\
2.8e-09	0.648	0	0	\\
2.85e-09	0.648	0	0	\\
2.9e-09	0.648	0	0	\\
2.95e-09	0.648	0	0	\\
3e-09	0.648	0	0	\\
3.05e-09	0.648	0.666666666666667	0.666666666666667	\\
3.1e-09	0.648	0.666666666666667	0.666666666666667	\\
3.15e-09	0.648	0.666666666666667	0.666666666666667	\\
3.2e-09	0.648	0.666666666666667	0.666666666666667	\\
3.25e-09	0.648	0.666666666666667	0.666666666666667	\\
3.3e-09	0.648	0.666666666666667	0.666666666666667	\\
3.35e-09	0.648	0.666666666666667	0.666666666666667	\\
3.4e-09	0.648	0.666666666666667	0.666666666666667	\\
3.45e-09	0.648	0.666666666666667	0.666666666666667	\\
3.5e-09	0.648	0.666666666666667	0.666666666666667	\\
3.55e-09	0.648	0	0	\\
3.6e-09	0.648	0	0	\\
3.65e-09	0.648	0	0	\\
3.7e-09	0.648	0	0	\\
3.75e-09	0.648	0	0	\\
3.8e-09	0.648	0	0	\\
3.85e-09	0.648	0	0	\\
3.9e-09	0.648	0	0	\\
3.95e-09	0.648	0	0	\\
4e-09	0.648	0	0	\\
4.05e-09	0.648	0	0	\\
4.1e-09	0.648	0	0	\\
4.15e-09	0.648	0	0	\\
4.2e-09	0.648	-0.444444444444444	-0.444444444444444	\\
4.25e-09	0.648	-0.444444444444444	-0.444444444444444	\\
4.3e-09	0.648	-0.444444444444444	-0.444444444444444	\\
4.35e-09	0.648	-0.444444444444444	-0.444444444444444	\\
4.4e-09	0.648	-0.444444444444444	-0.444444444444444	\\
4.45e-09	0.648	-0.444444444444444	-0.444444444444444	\\
4.5e-09	0.648	-0.444444444444444	-0.444444444444444	\\
4.55e-09	0.648	-0.444444444444444	-0.444444444444444	\\
4.6e-09	0.648	-0.444444444444444	-0.444444444444444	\\
4.65e-09	0.648	-0.444444444444444	-0.444444444444444	\\
4.7e-09	0.648	0	0	\\
4.75e-09	0.648	0	0	\\
4.8e-09	0.648	0	0	\\
4.85e-09	0.648	0	0	\\
4.9e-09	0.648	0	0	\\
4.95e-09	0.648	0	0	\\
5e-09	0.648	0	0	\\
5e-09	0.648	0	nan	\\
5e-09	0.648	-0.666666666666667	0.166666666666667	\\
5e-09	0.648	-0.666666666666667	nan	\\
0	0.66	-0.666666666666667	nan	\\
0	0.66	0	0.166666666666667	\\
0	0.66	0	0	\\
5e-11	0.66	0	0	\\
1e-10	0.66	0	0	\\
1.5e-10	0.66	0	0	\\
2e-10	0.66	0	0	\\
2.5e-10	0.66	0	0	\\
3e-10	0.66	0	0	\\
3.5e-10	0.66	0	0	\\
4e-10	0.66	0	0	\\
4.5e-10	0.66	0	0	\\
5e-10	0.66	0	0	\\
5.5e-10	0.66	0	0	\\
6e-10	0.66	0	0	\\
6.5e-10	0.66	0	0	\\
7e-10	0.66	1	1	\\
7.5e-10	0.66	1	1	\\
8e-10	0.66	1	1	\\
8.5e-10	0.66	1	1	\\
9e-10	0.66	1	1	\\
9.5e-10	0.66	1	1	\\
1e-09	0.66	1	1	\\
1.05e-09	0.66	1	1	\\
1.1e-09	0.66	1	1	\\
1.15e-09	0.66	1	1	\\
1.2e-09	0.66	0	0	\\
1.25e-09	0.66	0	0	\\
1.3e-09	0.66	0	0	\\
1.35e-09	0.66	0	0	\\
1.4e-09	0.66	0	0	\\
1.45e-09	0.66	0	0	\\
1.5e-09	0.66	0	0	\\
1.55e-09	0.66	0	0	\\
1.6e-09	0.66	0	0	\\
1.65e-09	0.66	0	0	\\
1.7e-09	0.66	0	0	\\
1.75e-09	0.66	-0.666666666666667	-0.666666666666667	\\
1.8e-09	0.66	-0.666666666666667	-0.666666666666667	\\
1.85e-09	0.66	-0.666666666666667	-0.666666666666667	\\
1.9e-09	0.66	-0.666666666666667	-0.666666666666667	\\
1.95e-09	0.66	-0.666666666666667	-0.666666666666667	\\
2e-09	0.66	-0.666666666666667	-0.666666666666667	\\
2.05e-09	0.66	-0.666666666666667	-0.666666666666667	\\
2.1e-09	0.66	-0.666666666666667	-0.666666666666667	\\
2.15e-09	0.66	-0.666666666666667	-0.666666666666667	\\
2.2e-09	0.66	-0.666666666666667	-0.666666666666667	\\
2.25e-09	0.66	0	0	\\
2.3e-09	0.66	0	0	\\
2.35e-09	0.66	0	0	\\
2.4e-09	0.66	0	0	\\
2.45e-09	0.66	0	0	\\
2.5e-09	0.66	0	0	\\
2.55e-09	0.66	0	0	\\
2.6e-09	0.66	0	0	\\
2.65e-09	0.66	0	0	\\
2.7e-09	0.66	0	0	\\
2.75e-09	0.66	0	0	\\
2.8e-09	0.66	0	0	\\
2.85e-09	0.66	0	0	\\
2.9e-09	0.66	0	0	\\
2.95e-09	0.66	0	0	\\
3e-09	0.66	0	0	\\
3.05e-09	0.66	0	0	\\
3.1e-09	0.66	0.666666666666667	0.666666666666667	\\
3.15e-09	0.66	0.666666666666667	0.666666666666667	\\
3.2e-09	0.66	0.666666666666667	0.666666666666667	\\
3.25e-09	0.66	0.666666666666667	0.666666666666667	\\
3.3e-09	0.66	0.666666666666667	0.666666666666667	\\
3.35e-09	0.66	0.666666666666667	0.666666666666667	\\
3.4e-09	0.66	0.666666666666667	0.666666666666667	\\
3.45e-09	0.66	0.666666666666667	0.666666666666667	\\
3.5e-09	0.66	0.666666666666667	0.666666666666667	\\
3.55e-09	0.66	0.666666666666667	0.666666666666667	\\
3.6e-09	0.66	0	0	\\
3.65e-09	0.66	0	0	\\
3.7e-09	0.66	0	0	\\
3.75e-09	0.66	0	0	\\
3.8e-09	0.66	0	0	\\
3.85e-09	0.66	0	0	\\
3.9e-09	0.66	0	0	\\
3.95e-09	0.66	0	0	\\
4e-09	0.66	0	0	\\
4.05e-09	0.66	0	0	\\
4.1e-09	0.66	0	0	\\
4.15e-09	0.66	-0.444444444444444	-0.444444444444444	\\
4.2e-09	0.66	-0.444444444444444	-0.444444444444444	\\
4.25e-09	0.66	-0.444444444444444	-0.444444444444444	\\
4.3e-09	0.66	-0.444444444444444	-0.444444444444444	\\
4.35e-09	0.66	-0.444444444444444	-0.444444444444444	\\
4.4e-09	0.66	-0.444444444444444	-0.444444444444444	\\
4.45e-09	0.66	-0.444444444444444	-0.444444444444444	\\
4.5e-09	0.66	-0.444444444444444	-0.444444444444444	\\
4.55e-09	0.66	-0.444444444444444	-0.444444444444444	\\
4.6e-09	0.66	-0.444444444444444	-0.444444444444444	\\
4.65e-09	0.66	0	0	\\
4.7e-09	0.66	0	0	\\
4.75e-09	0.66	0	0	\\
4.8e-09	0.66	0	0	\\
4.85e-09	0.66	0	0	\\
4.9e-09	0.66	0	0	\\
4.95e-09	0.66	0	0	\\
5e-09	0.66	0	0	\\
5e-09	0.66	0	nan	\\
5e-09	0.66	-0.666666666666667	0.166666666666667	\\
5e-09	0.66	-0.666666666666667	nan	\\
0	0.672	-0.666666666666667	nan	\\
0	0.672	0	0.166666666666667	\\
0	0.672	0	0	\\
5e-11	0.672	0	0	\\
1e-10	0.672	0	0	\\
1.5e-10	0.672	0	0	\\
2e-10	0.672	0	0	\\
2.5e-10	0.672	0	0	\\
3e-10	0.672	0	0	\\
3.5e-10	0.672	0	0	\\
4e-10	0.672	0	0	\\
4.5e-10	0.672	0	0	\\
5e-10	0.672	0	0	\\
5.5e-10	0.672	0	0	\\
6e-10	0.672	0	0	\\
6.5e-10	0.672	0	0	\\
7e-10	0.672	1	1	\\
7.5e-10	0.672	1	1	\\
8e-10	0.672	1	1	\\
8.5e-10	0.672	1	1	\\
9e-10	0.672	1	1	\\
9.5e-10	0.672	1	1	\\
1e-09	0.672	1	1	\\
1.05e-09	0.672	1	1	\\
1.1e-09	0.672	1	1	\\
1.15e-09	0.672	1	1	\\
1.2e-09	0.672	0	0	\\
1.25e-09	0.672	0	0	\\
1.3e-09	0.672	0	0	\\
1.35e-09	0.672	0	0	\\
1.4e-09	0.672	0	0	\\
1.45e-09	0.672	0	0	\\
1.5e-09	0.672	0	0	\\
1.55e-09	0.672	0	0	\\
1.6e-09	0.672	0	0	\\
1.65e-09	0.672	0	0	\\
1.7e-09	0.672	0	0	\\
1.75e-09	0.672	-0.666666666666667	-0.666666666666667	\\
1.8e-09	0.672	-0.666666666666667	-0.666666666666667	\\
1.85e-09	0.672	-0.666666666666667	-0.666666666666667	\\
1.9e-09	0.672	-0.666666666666667	-0.666666666666667	\\
1.95e-09	0.672	-0.666666666666667	-0.666666666666667	\\
2e-09	0.672	-0.666666666666667	-0.666666666666667	\\
2.05e-09	0.672	-0.666666666666667	-0.666666666666667	\\
2.1e-09	0.672	-0.666666666666667	-0.666666666666667	\\
2.15e-09	0.672	-0.666666666666667	-0.666666666666667	\\
2.2e-09	0.672	-0.666666666666667	-0.666666666666667	\\
2.25e-09	0.672	0	0	\\
2.3e-09	0.672	0	0	\\
2.35e-09	0.672	0	0	\\
2.4e-09	0.672	0	0	\\
2.45e-09	0.672	0	0	\\
2.5e-09	0.672	0	0	\\
2.55e-09	0.672	0	0	\\
2.6e-09	0.672	0	0	\\
2.65e-09	0.672	0	0	\\
2.7e-09	0.672	0	0	\\
2.75e-09	0.672	0	0	\\
2.8e-09	0.672	0	0	\\
2.85e-09	0.672	0	0	\\
2.9e-09	0.672	0	0	\\
2.95e-09	0.672	0	0	\\
3e-09	0.672	0	0	\\
3.05e-09	0.672	0	0	\\
3.1e-09	0.672	0.666666666666667	0.666666666666667	\\
3.15e-09	0.672	0.666666666666667	0.666666666666667	\\
3.2e-09	0.672	0.666666666666667	0.666666666666667	\\
3.25e-09	0.672	0.666666666666667	0.666666666666667	\\
3.3e-09	0.672	0.666666666666667	0.666666666666667	\\
3.35e-09	0.672	0.666666666666667	0.666666666666667	\\
3.4e-09	0.672	0.666666666666667	0.666666666666667	\\
3.45e-09	0.672	0.666666666666667	0.666666666666667	\\
3.5e-09	0.672	0.666666666666667	0.666666666666667	\\
3.55e-09	0.672	0.666666666666667	0.666666666666667	\\
3.6e-09	0.672	0	0	\\
3.65e-09	0.672	0	0	\\
3.7e-09	0.672	0	0	\\
3.75e-09	0.672	0	0	\\
3.8e-09	0.672	0	0	\\
3.85e-09	0.672	0	0	\\
3.9e-09	0.672	0	0	\\
3.95e-09	0.672	0	0	\\
4e-09	0.672	0	0	\\
4.05e-09	0.672	0	0	\\
4.1e-09	0.672	0	0	\\
4.15e-09	0.672	-0.444444444444444	-0.444444444444444	\\
4.2e-09	0.672	-0.444444444444444	-0.444444444444444	\\
4.25e-09	0.672	-0.444444444444444	-0.444444444444444	\\
4.3e-09	0.672	-0.444444444444444	-0.444444444444444	\\
4.35e-09	0.672	-0.444444444444444	-0.444444444444444	\\
4.4e-09	0.672	-0.444444444444444	-0.444444444444444	\\
4.45e-09	0.672	-0.444444444444444	-0.444444444444444	\\
4.5e-09	0.672	-0.444444444444444	-0.444444444444444	\\
4.55e-09	0.672	-0.444444444444444	-0.444444444444444	\\
4.6e-09	0.672	-0.444444444444444	-0.444444444444444	\\
4.65e-09	0.672	0	0	\\
4.7e-09	0.672	0	0	\\
4.75e-09	0.672	0	0	\\
4.8e-09	0.672	0	0	\\
4.85e-09	0.672	0	0	\\
4.9e-09	0.672	0	0	\\
4.95e-09	0.672	0	0	\\
5e-09	0.672	0	0	\\
5e-09	0.672	0	nan	\\
5e-09	0.672	-0.666666666666667	0.166666666666667	\\
5e-09	0.672	-0.666666666666667	nan	\\
0	0.684	-0.666666666666667	nan	\\
0	0.684	0	0.166666666666667	\\
0	0.684	0	0	\\
5e-11	0.684	0	0	\\
1e-10	0.684	0	0	\\
1.5e-10	0.684	0	0	\\
2e-10	0.684	0	0	\\
2.5e-10	0.684	0	0	\\
3e-10	0.684	0	0	\\
3.5e-10	0.684	0	0	\\
4e-10	0.684	0	0	\\
4.5e-10	0.684	0	0	\\
5e-10	0.684	0	0	\\
5.5e-10	0.684	0	0	\\
6e-10	0.684	0	0	\\
6.5e-10	0.684	0	0	\\
7e-10	0.684	1	1	\\
7.5e-10	0.684	1	1	\\
8e-10	0.684	1	1	\\
8.5e-10	0.684	1	1	\\
9e-10	0.684	1	1	\\
9.5e-10	0.684	1	1	\\
1e-09	0.684	1	1	\\
1.05e-09	0.684	1	1	\\
1.1e-09	0.684	1	1	\\
1.15e-09	0.684	1	1	\\
1.2e-09	0.684	0	0	\\
1.25e-09	0.684	0	0	\\
1.3e-09	0.684	0	0	\\
1.35e-09	0.684	0	0	\\
1.4e-09	0.684	0	0	\\
1.45e-09	0.684	0	0	\\
1.5e-09	0.684	0	0	\\
1.55e-09	0.684	0	0	\\
1.6e-09	0.684	0	0	\\
1.65e-09	0.684	0	0	\\
1.7e-09	0.684	0	0	\\
1.75e-09	0.684	-0.666666666666667	-0.666666666666667	\\
1.8e-09	0.684	-0.666666666666667	-0.666666666666667	\\
1.85e-09	0.684	-0.666666666666667	-0.666666666666667	\\
1.9e-09	0.684	-0.666666666666667	-0.666666666666667	\\
1.95e-09	0.684	-0.666666666666667	-0.666666666666667	\\
2e-09	0.684	-0.666666666666667	-0.666666666666667	\\
2.05e-09	0.684	-0.666666666666667	-0.666666666666667	\\
2.1e-09	0.684	-0.666666666666667	-0.666666666666667	\\
2.15e-09	0.684	-0.666666666666667	-0.666666666666667	\\
2.2e-09	0.684	-0.666666666666667	-0.666666666666667	\\
2.25e-09	0.684	0	0	\\
2.3e-09	0.684	0	0	\\
2.35e-09	0.684	0	0	\\
2.4e-09	0.684	0	0	\\
2.45e-09	0.684	0	0	\\
2.5e-09	0.684	0	0	\\
2.55e-09	0.684	0	0	\\
2.6e-09	0.684	0	0	\\
2.65e-09	0.684	0	0	\\
2.7e-09	0.684	0	0	\\
2.75e-09	0.684	0	0	\\
2.8e-09	0.684	0	0	\\
2.85e-09	0.684	0	0	\\
2.9e-09	0.684	0	0	\\
2.95e-09	0.684	0	0	\\
3e-09	0.684	0	0	\\
3.05e-09	0.684	0	0	\\
3.1e-09	0.684	0.666666666666667	0.666666666666667	\\
3.15e-09	0.684	0.666666666666667	0.666666666666667	\\
3.2e-09	0.684	0.666666666666667	0.666666666666667	\\
3.25e-09	0.684	0.666666666666667	0.666666666666667	\\
3.3e-09	0.684	0.666666666666667	0.666666666666667	\\
3.35e-09	0.684	0.666666666666667	0.666666666666667	\\
3.4e-09	0.684	0.666666666666667	0.666666666666667	\\
3.45e-09	0.684	0.666666666666667	0.666666666666667	\\
3.5e-09	0.684	0.666666666666667	0.666666666666667	\\
3.55e-09	0.684	0.666666666666667	0.666666666666667	\\
3.6e-09	0.684	0	0	\\
3.65e-09	0.684	0	0	\\
3.7e-09	0.684	0	0	\\
3.75e-09	0.684	0	0	\\
3.8e-09	0.684	0	0	\\
3.85e-09	0.684	0	0	\\
3.9e-09	0.684	0	0	\\
3.95e-09	0.684	0	0	\\
4e-09	0.684	0	0	\\
4.05e-09	0.684	0	0	\\
4.1e-09	0.684	0	0	\\
4.15e-09	0.684	-0.444444444444444	-0.444444444444444	\\
4.2e-09	0.684	-0.444444444444444	-0.444444444444444	\\
4.25e-09	0.684	-0.444444444444444	-0.444444444444444	\\
4.3e-09	0.684	-0.444444444444444	-0.444444444444444	\\
4.35e-09	0.684	-0.444444444444444	-0.444444444444444	\\
4.4e-09	0.684	-0.444444444444444	-0.444444444444444	\\
4.45e-09	0.684	-0.444444444444444	-0.444444444444444	\\
4.5e-09	0.684	-0.444444444444444	-0.444444444444444	\\
4.55e-09	0.684	-0.444444444444444	-0.444444444444444	\\
4.6e-09	0.684	-0.444444444444444	-0.444444444444444	\\
4.65e-09	0.684	0	0	\\
4.7e-09	0.684	0	0	\\
4.75e-09	0.684	0	0	\\
4.8e-09	0.684	0	0	\\
4.85e-09	0.684	0	0	\\
4.9e-09	0.684	0	0	\\
4.95e-09	0.684	0	0	\\
5e-09	0.684	0	0	\\
5e-09	0.684	0	nan	\\
5e-09	0.684	-0.666666666666667	0.166666666666667	\\
5e-09	0.684	-0.666666666666667	nan	\\
0	0.696	-0.666666666666667	nan	\\
0	0.696	0	0.166666666666667	\\
0	0.696	0	0	\\
5e-11	0.696	0	0	\\
1e-10	0.696	0	0	\\
1.5e-10	0.696	0	0	\\
2e-10	0.696	0	0	\\
2.5e-10	0.696	0	0	\\
3e-10	0.696	0	0	\\
3.5e-10	0.696	0	0	\\
4e-10	0.696	0	0	\\
4.5e-10	0.696	0	0	\\
5e-10	0.696	0	0	\\
5.5e-10	0.696	0	0	\\
6e-10	0.696	0	0	\\
6.5e-10	0.696	0	0	\\
7e-10	0.696	1	1	\\
7.5e-10	0.696	1	1	\\
8e-10	0.696	1	1	\\
8.5e-10	0.696	1	1	\\
9e-10	0.696	1	1	\\
9.5e-10	0.696	1	1	\\
1e-09	0.696	1	1	\\
1.05e-09	0.696	1	1	\\
1.1e-09	0.696	1	1	\\
1.15e-09	0.696	1	1	\\
1.2e-09	0.696	0	0	\\
1.25e-09	0.696	0	0	\\
1.3e-09	0.696	0	0	\\
1.35e-09	0.696	0	0	\\
1.4e-09	0.696	0	0	\\
1.45e-09	0.696	0	0	\\
1.5e-09	0.696	0	0	\\
1.55e-09	0.696	0	0	\\
1.6e-09	0.696	0	0	\\
1.65e-09	0.696	0	0	\\
1.7e-09	0.696	0	0	\\
1.75e-09	0.696	-0.666666666666667	-0.666666666666667	\\
1.8e-09	0.696	-0.666666666666667	-0.666666666666667	\\
1.85e-09	0.696	-0.666666666666667	-0.666666666666667	\\
1.9e-09	0.696	-0.666666666666667	-0.666666666666667	\\
1.95e-09	0.696	-0.666666666666667	-0.666666666666667	\\
2e-09	0.696	-0.666666666666667	-0.666666666666667	\\
2.05e-09	0.696	-0.666666666666667	-0.666666666666667	\\
2.1e-09	0.696	-0.666666666666667	-0.666666666666667	\\
2.15e-09	0.696	-0.666666666666667	-0.666666666666667	\\
2.2e-09	0.696	-0.666666666666667	-0.666666666666667	\\
2.25e-09	0.696	0	0	\\
2.3e-09	0.696	0	0	\\
2.35e-09	0.696	0	0	\\
2.4e-09	0.696	0	0	\\
2.45e-09	0.696	0	0	\\
2.5e-09	0.696	0	0	\\
2.55e-09	0.696	0	0	\\
2.6e-09	0.696	0	0	\\
2.65e-09	0.696	0	0	\\
2.7e-09	0.696	0	0	\\
2.75e-09	0.696	0	0	\\
2.8e-09	0.696	0	0	\\
2.85e-09	0.696	0	0	\\
2.9e-09	0.696	0	0	\\
2.95e-09	0.696	0	0	\\
3e-09	0.696	0	0	\\
3.05e-09	0.696	0	0	\\
3.1e-09	0.696	0.666666666666667	0.666666666666667	\\
3.15e-09	0.696	0.666666666666667	0.666666666666667	\\
3.2e-09	0.696	0.666666666666667	0.666666666666667	\\
3.25e-09	0.696	0.666666666666667	0.666666666666667	\\
3.3e-09	0.696	0.666666666666667	0.666666666666667	\\
3.35e-09	0.696	0.666666666666667	0.666666666666667	\\
3.4e-09	0.696	0.666666666666667	0.666666666666667	\\
3.45e-09	0.696	0.666666666666667	0.666666666666667	\\
3.5e-09	0.696	0.666666666666667	0.666666666666667	\\
3.55e-09	0.696	0.666666666666667	0.666666666666667	\\
3.6e-09	0.696	0	0	\\
3.65e-09	0.696	0	0	\\
3.7e-09	0.696	0	0	\\
3.75e-09	0.696	0	0	\\
3.8e-09	0.696	0	0	\\
3.85e-09	0.696	0	0	\\
3.9e-09	0.696	0	0	\\
3.95e-09	0.696	0	0	\\
4e-09	0.696	0	0	\\
4.05e-09	0.696	0	0	\\
4.1e-09	0.696	0	0	\\
4.15e-09	0.696	-0.444444444444444	-0.444444444444444	\\
4.2e-09	0.696	-0.444444444444444	-0.444444444444444	\\
4.25e-09	0.696	-0.444444444444444	-0.444444444444444	\\
4.3e-09	0.696	-0.444444444444444	-0.444444444444444	\\
4.35e-09	0.696	-0.444444444444444	-0.444444444444444	\\
4.4e-09	0.696	-0.444444444444444	-0.444444444444444	\\
4.45e-09	0.696	-0.444444444444444	-0.444444444444444	\\
4.5e-09	0.696	-0.444444444444444	-0.444444444444444	\\
4.55e-09	0.696	-0.444444444444444	-0.444444444444444	\\
4.6e-09	0.696	-0.444444444444444	-0.444444444444444	\\
4.65e-09	0.696	0	0	\\
4.7e-09	0.696	0	0	\\
4.75e-09	0.696	0	0	\\
4.8e-09	0.696	0	0	\\
4.85e-09	0.696	0	0	\\
4.9e-09	0.696	0	0	\\
4.95e-09	0.696	0	0	\\
5e-09	0.696	0	0	\\
5e-09	0.696	0	nan	\\
5e-09	0.696	-0.666666666666667	0.166666666666667	\\
5e-09	0.696	-0.666666666666667	nan	\\
0	0.708	-0.666666666666667	nan	\\
0	0.708	0	0.166666666666667	\\
0	0.708	0	0	\\
5e-11	0.708	0	0	\\
1e-10	0.708	0	0	\\
1.5e-10	0.708	0	0	\\
2e-10	0.708	0	0	\\
2.5e-10	0.708	0	0	\\
3e-10	0.708	0	0	\\
3.5e-10	0.708	0	0	\\
4e-10	0.708	0	0	\\
4.5e-10	0.708	0	0	\\
5e-10	0.708	0	0	\\
5.5e-10	0.708	0	0	\\
6e-10	0.708	0	0	\\
6.5e-10	0.708	0	0	\\
7e-10	0.708	0	0	\\
7.5e-10	0.708	1	1	\\
8e-10	0.708	1	1	\\
8.5e-10	0.708	1	1	\\
9e-10	0.708	1	1	\\
9.5e-10	0.708	1	1	\\
1e-09	0.708	1	1	\\
1.05e-09	0.708	1	1	\\
1.1e-09	0.708	1	1	\\
1.15e-09	0.708	1	1	\\
1.2e-09	0.708	1	1	\\
1.25e-09	0.708	0	0	\\
1.3e-09	0.708	0	0	\\
1.35e-09	0.708	0	0	\\
1.4e-09	0.708	0	0	\\
1.45e-09	0.708	0	0	\\
1.5e-09	0.708	0	0	\\
1.55e-09	0.708	0	0	\\
1.6e-09	0.708	0	0	\\
1.65e-09	0.708	0	0	\\
1.7e-09	0.708	-0.666666666666667	-0.666666666666667	\\
1.75e-09	0.708	-0.666666666666667	-0.666666666666667	\\
1.8e-09	0.708	-0.666666666666667	-0.666666666666667	\\
1.85e-09	0.708	-0.666666666666667	-0.666666666666667	\\
1.9e-09	0.708	-0.666666666666667	-0.666666666666667	\\
1.95e-09	0.708	-0.666666666666667	-0.666666666666667	\\
2e-09	0.708	-0.666666666666667	-0.666666666666667	\\
2.05e-09	0.708	-0.666666666666667	-0.666666666666667	\\
2.1e-09	0.708	-0.666666666666667	-0.666666666666667	\\
2.15e-09	0.708	-0.666666666666667	-0.666666666666667	\\
2.2e-09	0.708	0	0	\\
2.25e-09	0.708	0	0	\\
2.3e-09	0.708	0	0	\\
2.35e-09	0.708	0	0	\\
2.4e-09	0.708	0	0	\\
2.45e-09	0.708	0	0	\\
2.5e-09	0.708	0	0	\\
2.55e-09	0.708	0	0	\\
2.6e-09	0.708	0	0	\\
2.65e-09	0.708	0	0	\\
2.7e-09	0.708	0	0	\\
2.75e-09	0.708	0	0	\\
2.8e-09	0.708	0	0	\\
2.85e-09	0.708	0	0	\\
2.9e-09	0.708	0	0	\\
2.95e-09	0.708	0	0	\\
3e-09	0.708	0	0	\\
3.05e-09	0.708	0	0	\\
3.1e-09	0.708	0	0	\\
3.15e-09	0.708	0.666666666666667	0.666666666666667	\\
3.2e-09	0.708	0.666666666666667	0.666666666666667	\\
3.25e-09	0.708	0.666666666666667	0.666666666666667	\\
3.3e-09	0.708	0.666666666666667	0.666666666666667	\\
3.35e-09	0.708	0.666666666666667	0.666666666666667	\\
3.4e-09	0.708	0.666666666666667	0.666666666666667	\\
3.45e-09	0.708	0.666666666666667	0.666666666666667	\\
3.5e-09	0.708	0.666666666666667	0.666666666666667	\\
3.55e-09	0.708	0.666666666666667	0.666666666666667	\\
3.6e-09	0.708	0.666666666666667	0.666666666666667	\\
3.65e-09	0.708	0	0	\\
3.7e-09	0.708	0	0	\\
3.75e-09	0.708	0	0	\\
3.8e-09	0.708	0	0	\\
3.85e-09	0.708	0	0	\\
3.9e-09	0.708	0	0	\\
3.95e-09	0.708	0	0	\\
4e-09	0.708	0	0	\\
4.05e-09	0.708	0	0	\\
4.1e-09	0.708	-0.444444444444444	-0.444444444444444	\\
4.15e-09	0.708	-0.444444444444444	-0.444444444444444	\\
4.2e-09	0.708	-0.444444444444444	-0.444444444444444	\\
4.25e-09	0.708	-0.444444444444444	-0.444444444444444	\\
4.3e-09	0.708	-0.444444444444444	-0.444444444444444	\\
4.35e-09	0.708	-0.444444444444444	-0.444444444444444	\\
4.4e-09	0.708	-0.444444444444444	-0.444444444444444	\\
4.45e-09	0.708	-0.444444444444444	-0.444444444444444	\\
4.5e-09	0.708	-0.444444444444444	-0.444444444444444	\\
4.55e-09	0.708	-0.444444444444444	-0.444444444444444	\\
4.6e-09	0.708	0	0	\\
4.65e-09	0.708	0	0	\\
4.7e-09	0.708	0	0	\\
4.75e-09	0.708	0	0	\\
4.8e-09	0.708	0	0	\\
4.85e-09	0.708	0	0	\\
4.9e-09	0.708	0	0	\\
4.95e-09	0.708	0	0	\\
5e-09	0.708	0	0	\\
5e-09	0.708	0	nan	\\
5e-09	0.708	-0.666666666666667	0.166666666666667	\\
5e-09	0.708	-0.666666666666667	nan	\\
0	0.72	-0.666666666666667	nan	\\
0	0.72	0	0.166666666666667	\\
0	0.72	0	0	\\
5e-11	0.72	0	0	\\
1e-10	0.72	0	0	\\
1.5e-10	0.72	0	0	\\
2e-10	0.72	0	0	\\
2.5e-10	0.72	0	0	\\
3e-10	0.72	0	0	\\
3.5e-10	0.72	0	0	\\
4e-10	0.72	0	0	\\
4.5e-10	0.72	0	0	\\
5e-10	0.72	0	0	\\
5.5e-10	0.72	0	0	\\
6e-10	0.72	0	0	\\
6.5e-10	0.72	0	0	\\
7e-10	0.72	0	0	\\
7.5e-10	0.72	1	1	\\
8e-10	0.72	1	1	\\
8.5e-10	0.72	1	1	\\
9e-10	0.72	1	1	\\
9.5e-10	0.72	1	1	\\
1e-09	0.72	1	1	\\
1.05e-09	0.72	1	1	\\
1.1e-09	0.72	1	1	\\
1.15e-09	0.72	1	1	\\
1.2e-09	0.72	1	1	\\
1.25e-09	0.72	0	0	\\
1.3e-09	0.72	0	0	\\
1.35e-09	0.72	0	0	\\
1.4e-09	0.72	0	0	\\
1.45e-09	0.72	0	0	\\
1.5e-09	0.72	0	0	\\
1.55e-09	0.72	0	0	\\
1.6e-09	0.72	0	0	\\
1.65e-09	0.72	0	0	\\
1.7e-09	0.72	-0.666666666666667	-0.666666666666667	\\
1.75e-09	0.72	-0.666666666666667	-0.666666666666667	\\
1.8e-09	0.72	-0.666666666666667	-0.666666666666667	\\
1.85e-09	0.72	-0.666666666666667	-0.666666666666667	\\
1.9e-09	0.72	-0.666666666666667	-0.666666666666667	\\
1.95e-09	0.72	-0.666666666666667	-0.666666666666667	\\
2e-09	0.72	-0.666666666666667	-0.666666666666667	\\
2.05e-09	0.72	-0.666666666666667	-0.666666666666667	\\
2.1e-09	0.72	-0.666666666666667	-0.666666666666667	\\
2.15e-09	0.72	-0.666666666666667	-0.666666666666667	\\
2.2e-09	0.72	0	0	\\
2.25e-09	0.72	0	0	\\
2.3e-09	0.72	0	0	\\
2.35e-09	0.72	0	0	\\
2.4e-09	0.72	0	0	\\
2.45e-09	0.72	0	0	\\
2.5e-09	0.72	0	0	\\
2.55e-09	0.72	0	0	\\
2.6e-09	0.72	0	0	\\
2.65e-09	0.72	0	0	\\
2.7e-09	0.72	0	0	\\
2.75e-09	0.72	0	0	\\
2.8e-09	0.72	0	0	\\
2.85e-09	0.72	0	0	\\
2.9e-09	0.72	0	0	\\
2.95e-09	0.72	0	0	\\
3e-09	0.72	0	0	\\
3.05e-09	0.72	0	0	\\
3.1e-09	0.72	0	0	\\
3.15e-09	0.72	0.666666666666667	0.666666666666667	\\
3.2e-09	0.72	0.666666666666667	0.666666666666667	\\
3.25e-09	0.72	0.666666666666667	0.666666666666667	\\
3.3e-09	0.72	0.666666666666667	0.666666666666667	\\
3.35e-09	0.72	0.666666666666667	0.666666666666667	\\
3.4e-09	0.72	0.666666666666667	0.666666666666667	\\
3.45e-09	0.72	0.666666666666667	0.666666666666667	\\
3.5e-09	0.72	0.666666666666667	0.666666666666667	\\
3.55e-09	0.72	0.666666666666667	0.666666666666667	\\
3.6e-09	0.72	0.666666666666667	0.666666666666667	\\
3.65e-09	0.72	0	0	\\
3.7e-09	0.72	0	0	\\
3.75e-09	0.72	0	0	\\
3.8e-09	0.72	0	0	\\
3.85e-09	0.72	0	0	\\
3.9e-09	0.72	0	0	\\
3.95e-09	0.72	0	0	\\
4e-09	0.72	0	0	\\
4.05e-09	0.72	0	0	\\
4.1e-09	0.72	-0.444444444444444	-0.444444444444444	\\
4.15e-09	0.72	-0.444444444444444	-0.444444444444444	\\
4.2e-09	0.72	-0.444444444444444	-0.444444444444444	\\
4.25e-09	0.72	-0.444444444444444	-0.444444444444444	\\
4.3e-09	0.72	-0.444444444444444	-0.444444444444444	\\
4.35e-09	0.72	-0.444444444444444	-0.444444444444444	\\
4.4e-09	0.72	-0.444444444444444	-0.444444444444444	\\
4.45e-09	0.72	-0.444444444444444	-0.444444444444444	\\
4.5e-09	0.72	-0.444444444444444	-0.444444444444444	\\
4.55e-09	0.72	-0.444444444444444	-0.444444444444444	\\
4.6e-09	0.72	0	0	\\
4.65e-09	0.72	0	0	\\
4.7e-09	0.72	0	0	\\
4.75e-09	0.72	0	0	\\
4.8e-09	0.72	0	0	\\
4.85e-09	0.72	0	0	\\
4.9e-09	0.72	0	0	\\
4.95e-09	0.72	0	0	\\
5e-09	0.72	0	0	\\
5e-09	0.72	0	nan	\\
5e-09	0.72	-0.666666666666667	0.166666666666667	\\
5e-09	0.72	-0.666666666666667	nan	\\
0	0.732	-0.666666666666667	nan	\\
0	0.732	0	0.166666666666667	\\
0	0.732	0	0	\\
5e-11	0.732	0	0	\\
1e-10	0.732	0	0	\\
1.5e-10	0.732	0	0	\\
2e-10	0.732	0	0	\\
2.5e-10	0.732	0	0	\\
3e-10	0.732	0	0	\\
3.5e-10	0.732	0	0	\\
4e-10	0.732	0	0	\\
4.5e-10	0.732	0	0	\\
5e-10	0.732	0	0	\\
5.5e-10	0.732	0	0	\\
6e-10	0.732	0	0	\\
6.5e-10	0.732	0	0	\\
7e-10	0.732	0	0	\\
7.5e-10	0.732	1	1	\\
8e-10	0.732	1	1	\\
8.5e-10	0.732	1	1	\\
9e-10	0.732	1	1	\\
9.5e-10	0.732	1	1	\\
1e-09	0.732	1	1	\\
1.05e-09	0.732	1	1	\\
1.1e-09	0.732	1	1	\\
1.15e-09	0.732	1	1	\\
1.2e-09	0.732	1	1	\\
1.25e-09	0.732	0	0	\\
1.3e-09	0.732	0	0	\\
1.35e-09	0.732	0	0	\\
1.4e-09	0.732	0	0	\\
1.45e-09	0.732	0	0	\\
1.5e-09	0.732	0	0	\\
1.55e-09	0.732	0	0	\\
1.6e-09	0.732	0	0	\\
1.65e-09	0.732	0	0	\\
1.7e-09	0.732	-0.666666666666667	-0.666666666666667	\\
1.75e-09	0.732	-0.666666666666667	-0.666666666666667	\\
1.8e-09	0.732	-0.666666666666667	-0.666666666666667	\\
1.85e-09	0.732	-0.666666666666667	-0.666666666666667	\\
1.9e-09	0.732	-0.666666666666667	-0.666666666666667	\\
1.95e-09	0.732	-0.666666666666667	-0.666666666666667	\\
2e-09	0.732	-0.666666666666667	-0.666666666666667	\\
2.05e-09	0.732	-0.666666666666667	-0.666666666666667	\\
2.1e-09	0.732	-0.666666666666667	-0.666666666666667	\\
2.15e-09	0.732	-0.666666666666667	-0.666666666666667	\\
2.2e-09	0.732	0	0	\\
2.25e-09	0.732	0	0	\\
2.3e-09	0.732	0	0	\\
2.35e-09	0.732	0	0	\\
2.4e-09	0.732	0	0	\\
2.45e-09	0.732	0	0	\\
2.5e-09	0.732	0	0	\\
2.55e-09	0.732	0	0	\\
2.6e-09	0.732	0	0	\\
2.65e-09	0.732	0	0	\\
2.7e-09	0.732	0	0	\\
2.75e-09	0.732	0	0	\\
2.8e-09	0.732	0	0	\\
2.85e-09	0.732	0	0	\\
2.9e-09	0.732	0	0	\\
2.95e-09	0.732	0	0	\\
3e-09	0.732	0	0	\\
3.05e-09	0.732	0	0	\\
3.1e-09	0.732	0	0	\\
3.15e-09	0.732	0.666666666666667	0.666666666666667	\\
3.2e-09	0.732	0.666666666666667	0.666666666666667	\\
3.25e-09	0.732	0.666666666666667	0.666666666666667	\\
3.3e-09	0.732	0.666666666666667	0.666666666666667	\\
3.35e-09	0.732	0.666666666666667	0.666666666666667	\\
3.4e-09	0.732	0.666666666666667	0.666666666666667	\\
3.45e-09	0.732	0.666666666666667	0.666666666666667	\\
3.5e-09	0.732	0.666666666666667	0.666666666666667	\\
3.55e-09	0.732	0.666666666666667	0.666666666666667	\\
3.6e-09	0.732	0.666666666666667	0.666666666666667	\\
3.65e-09	0.732	0	0	\\
3.7e-09	0.732	0	0	\\
3.75e-09	0.732	0	0	\\
3.8e-09	0.732	0	0	\\
3.85e-09	0.732	0	0	\\
3.9e-09	0.732	0	0	\\
3.95e-09	0.732	0	0	\\
4e-09	0.732	0	0	\\
4.05e-09	0.732	0	0	\\
4.1e-09	0.732	-0.444444444444444	-0.444444444444444	\\
4.15e-09	0.732	-0.444444444444444	-0.444444444444444	\\
4.2e-09	0.732	-0.444444444444444	-0.444444444444444	\\
4.25e-09	0.732	-0.444444444444444	-0.444444444444444	\\
4.3e-09	0.732	-0.444444444444444	-0.444444444444444	\\
4.35e-09	0.732	-0.444444444444444	-0.444444444444444	\\
4.4e-09	0.732	-0.444444444444444	-0.444444444444444	\\
4.45e-09	0.732	-0.444444444444444	-0.444444444444444	\\
4.5e-09	0.732	-0.444444444444444	-0.444444444444444	\\
4.55e-09	0.732	-0.444444444444444	-0.444444444444444	\\
4.6e-09	0.732	0	0	\\
4.65e-09	0.732	0	0	\\
4.7e-09	0.732	0	0	\\
4.75e-09	0.732	0	0	\\
4.8e-09	0.732	0	0	\\
4.85e-09	0.732	0	0	\\
4.9e-09	0.732	0	0	\\
4.95e-09	0.732	0	0	\\
5e-09	0.732	0	0	\\
5e-09	0.732	0	nan	\\
5e-09	0.732	-0.666666666666667	0.166666666666667	\\
5e-09	0.732	-0.666666666666667	nan	\\
0	0.744	-0.666666666666667	nan	\\
0	0.744	0	0.166666666666667	\\
0	0.744	0	0	\\
5e-11	0.744	0	0	\\
1e-10	0.744	0	0	\\
1.5e-10	0.744	0	0	\\
2e-10	0.744	0	0	\\
2.5e-10	0.744	0	0	\\
3e-10	0.744	0	0	\\
3.5e-10	0.744	0	0	\\
4e-10	0.744	0	0	\\
4.5e-10	0.744	0	0	\\
5e-10	0.744	0	0	\\
5.5e-10	0.744	0	0	\\
6e-10	0.744	0	0	\\
6.5e-10	0.744	0	0	\\
7e-10	0.744	0	0	\\
7.5e-10	0.744	1	1	\\
8e-10	0.744	1	1	\\
8.5e-10	0.744	1	1	\\
9e-10	0.744	1	1	\\
9.5e-10	0.744	1	1	\\
1e-09	0.744	1	1	\\
1.05e-09	0.744	1	1	\\
1.1e-09	0.744	1	1	\\
1.15e-09	0.744	1	1	\\
1.2e-09	0.744	1	1	\\
1.25e-09	0.744	0	0	\\
1.3e-09	0.744	0	0	\\
1.35e-09	0.744	0	0	\\
1.4e-09	0.744	0	0	\\
1.45e-09	0.744	0	0	\\
1.5e-09	0.744	0	0	\\
1.55e-09	0.744	0	0	\\
1.6e-09	0.744	0	0	\\
1.65e-09	0.744	0	0	\\
1.7e-09	0.744	-0.666666666666667	-0.666666666666667	\\
1.75e-09	0.744	-0.666666666666667	-0.666666666666667	\\
1.8e-09	0.744	-0.666666666666667	-0.666666666666667	\\
1.85e-09	0.744	-0.666666666666667	-0.666666666666667	\\
1.9e-09	0.744	-0.666666666666667	-0.666666666666667	\\
1.95e-09	0.744	-0.666666666666667	-0.666666666666667	\\
2e-09	0.744	-0.666666666666667	-0.666666666666667	\\
2.05e-09	0.744	-0.666666666666667	-0.666666666666667	\\
2.1e-09	0.744	-0.666666666666667	-0.666666666666667	\\
2.15e-09	0.744	-0.666666666666667	-0.666666666666667	\\
2.2e-09	0.744	0	0	\\
2.25e-09	0.744	0	0	\\
2.3e-09	0.744	0	0	\\
2.35e-09	0.744	0	0	\\
2.4e-09	0.744	0	0	\\
2.45e-09	0.744	0	0	\\
2.5e-09	0.744	0	0	\\
2.55e-09	0.744	0	0	\\
2.6e-09	0.744	0	0	\\
2.65e-09	0.744	0	0	\\
2.7e-09	0.744	0	0	\\
2.75e-09	0.744	0	0	\\
2.8e-09	0.744	0	0	\\
2.85e-09	0.744	0	0	\\
2.9e-09	0.744	0	0	\\
2.95e-09	0.744	0	0	\\
3e-09	0.744	0	0	\\
3.05e-09	0.744	0	0	\\
3.1e-09	0.744	0	0	\\
3.15e-09	0.744	0.666666666666667	0.666666666666667	\\
3.2e-09	0.744	0.666666666666667	0.666666666666667	\\
3.25e-09	0.744	0.666666666666667	0.666666666666667	\\
3.3e-09	0.744	0.666666666666667	0.666666666666667	\\
3.35e-09	0.744	0.666666666666667	0.666666666666667	\\
3.4e-09	0.744	0.666666666666667	0.666666666666667	\\
3.45e-09	0.744	0.666666666666667	0.666666666666667	\\
3.5e-09	0.744	0.666666666666667	0.666666666666667	\\
3.55e-09	0.744	0.666666666666667	0.666666666666667	\\
3.6e-09	0.744	0.666666666666667	0.666666666666667	\\
3.65e-09	0.744	0	0	\\
3.7e-09	0.744	0	0	\\
3.75e-09	0.744	0	0	\\
3.8e-09	0.744	0	0	\\
3.85e-09	0.744	0	0	\\
3.9e-09	0.744	0	0	\\
3.95e-09	0.744	0	0	\\
4e-09	0.744	0	0	\\
4.05e-09	0.744	0	0	\\
4.1e-09	0.744	-0.444444444444444	-0.444444444444444	\\
4.15e-09	0.744	-0.444444444444444	-0.444444444444444	\\
4.2e-09	0.744	-0.444444444444444	-0.444444444444444	\\
4.25e-09	0.744	-0.444444444444444	-0.444444444444444	\\
4.3e-09	0.744	-0.444444444444444	-0.444444444444444	\\
4.35e-09	0.744	-0.444444444444444	-0.444444444444444	\\
4.4e-09	0.744	-0.444444444444444	-0.444444444444444	\\
4.45e-09	0.744	-0.444444444444444	-0.444444444444444	\\
4.5e-09	0.744	-0.444444444444444	-0.444444444444444	\\
4.55e-09	0.744	-0.444444444444444	-0.444444444444444	\\
4.6e-09	0.744	0	0	\\
4.65e-09	0.744	0	0	\\
4.7e-09	0.744	0	0	\\
4.75e-09	0.744	0	0	\\
4.8e-09	0.744	0	0	\\
4.85e-09	0.744	0	0	\\
4.9e-09	0.744	0	0	\\
4.95e-09	0.744	0	0	\\
5e-09	0.744	0	0	\\
5e-09	0.744	0	nan	\\
5e-09	0.744	-0.666666666666667	0.166666666666667	\\
5e-09	0.744	-0.666666666666667	nan	\\
0	0.756	-0.666666666666667	nan	\\
0	0.756	0	0.166666666666667	\\
0	0.756	0	0	\\
5e-11	0.756	0	0	\\
1e-10	0.756	0	0	\\
1.5e-10	0.756	0	0	\\
2e-10	0.756	0	0	\\
2.5e-10	0.756	0	0	\\
3e-10	0.756	0	0	\\
3.5e-10	0.756	0	0	\\
4e-10	0.756	0	0	\\
4.5e-10	0.756	0	0	\\
5e-10	0.756	0	0	\\
5.5e-10	0.756	0	0	\\
6e-10	0.756	0	0	\\
6.5e-10	0.756	0	0	\\
7e-10	0.756	0	0	\\
7.5e-10	0.756	0	0	\\
8e-10	0.756	1	1	\\
8.5e-10	0.756	1	1	\\
9e-10	0.756	1	1	\\
9.5e-10	0.756	1	1	\\
1e-09	0.756	1	1	\\
1.05e-09	0.756	1	1	\\
1.1e-09	0.756	1	1	\\
1.15e-09	0.756	1	1	\\
1.2e-09	0.756	1	1	\\
1.25e-09	0.756	1	1	\\
1.3e-09	0.756	0	0	\\
1.35e-09	0.756	0	0	\\
1.4e-09	0.756	0	0	\\
1.45e-09	0.756	0	0	\\
1.5e-09	0.756	0	0	\\
1.55e-09	0.756	0	0	\\
1.6e-09	0.756	0	0	\\
1.65e-09	0.756	-0.666666666666667	-0.666666666666667	\\
1.7e-09	0.756	-0.666666666666667	-0.666666666666667	\\
1.75e-09	0.756	-0.666666666666667	-0.666666666666667	\\
1.8e-09	0.756	-0.666666666666667	-0.666666666666667	\\
1.85e-09	0.756	-0.666666666666667	-0.666666666666667	\\
1.9e-09	0.756	-0.666666666666667	-0.666666666666667	\\
1.95e-09	0.756	-0.666666666666667	-0.666666666666667	\\
2e-09	0.756	-0.666666666666667	-0.666666666666667	\\
2.05e-09	0.756	-0.666666666666667	-0.666666666666667	\\
2.1e-09	0.756	-0.666666666666667	-0.666666666666667	\\
2.15e-09	0.756	0	0	\\
2.2e-09	0.756	0	0	\\
2.25e-09	0.756	0	0	\\
2.3e-09	0.756	0	0	\\
2.35e-09	0.756	0	0	\\
2.4e-09	0.756	0	0	\\
2.45e-09	0.756	0	0	\\
2.5e-09	0.756	0	0	\\
2.55e-09	0.756	0	0	\\
2.6e-09	0.756	0	0	\\
2.65e-09	0.756	0	0	\\
2.7e-09	0.756	0	0	\\
2.75e-09	0.756	0	0	\\
2.8e-09	0.756	0	0	\\
2.85e-09	0.756	0	0	\\
2.9e-09	0.756	0	0	\\
2.95e-09	0.756	0	0	\\
3e-09	0.756	0	0	\\
3.05e-09	0.756	0	0	\\
3.1e-09	0.756	0	0	\\
3.15e-09	0.756	0	0	\\
3.2e-09	0.756	0.666666666666667	0.666666666666667	\\
3.25e-09	0.756	0.666666666666667	0.666666666666667	\\
3.3e-09	0.756	0.666666666666667	0.666666666666667	\\
3.35e-09	0.756	0.666666666666667	0.666666666666667	\\
3.4e-09	0.756	0.666666666666667	0.666666666666667	\\
3.45e-09	0.756	0.666666666666667	0.666666666666667	\\
3.5e-09	0.756	0.666666666666667	0.666666666666667	\\
3.55e-09	0.756	0.666666666666667	0.666666666666667	\\
3.6e-09	0.756	0.666666666666667	0.666666666666667	\\
3.65e-09	0.756	0.666666666666667	0.666666666666667	\\
3.7e-09	0.756	0	0	\\
3.75e-09	0.756	0	0	\\
3.8e-09	0.756	0	0	\\
3.85e-09	0.756	0	0	\\
3.9e-09	0.756	0	0	\\
3.95e-09	0.756	0	0	\\
4e-09	0.756	0	0	\\
4.05e-09	0.756	-0.444444444444444	-0.444444444444444	\\
4.1e-09	0.756	-0.444444444444444	-0.444444444444444	\\
4.15e-09	0.756	-0.444444444444444	-0.444444444444444	\\
4.2e-09	0.756	-0.444444444444444	-0.444444444444444	\\
4.25e-09	0.756	-0.444444444444444	-0.444444444444444	\\
4.3e-09	0.756	-0.444444444444444	-0.444444444444444	\\
4.35e-09	0.756	-0.444444444444444	-0.444444444444444	\\
4.4e-09	0.756	-0.444444444444444	-0.444444444444444	\\
4.45e-09	0.756	-0.444444444444444	-0.444444444444444	\\
4.5e-09	0.756	-0.444444444444444	-0.444444444444444	\\
4.55e-09	0.756	0	0	\\
4.6e-09	0.756	0	0	\\
4.65e-09	0.756	0	0	\\
4.7e-09	0.756	0	0	\\
4.75e-09	0.756	0	0	\\
4.8e-09	0.756	0	0	\\
4.85e-09	0.756	0	0	\\
4.9e-09	0.756	0	0	\\
4.95e-09	0.756	0	0	\\
5e-09	0.756	0	0	\\
5e-09	0.756	0	nan	\\
5e-09	0.756	-0.666666666666667	0.166666666666667	\\
5e-09	0.756	-0.666666666666667	nan	\\
0	0.768	-0.666666666666667	nan	\\
0	0.768	0	0.166666666666667	\\
0	0.768	0	0	\\
5e-11	0.768	0	0	\\
1e-10	0.768	0	0	\\
1.5e-10	0.768	0	0	\\
2e-10	0.768	0	0	\\
2.5e-10	0.768	0	0	\\
3e-10	0.768	0	0	\\
3.5e-10	0.768	0	0	\\
4e-10	0.768	0	0	\\
4.5e-10	0.768	0	0	\\
5e-10	0.768	0	0	\\
5.5e-10	0.768	0	0	\\
6e-10	0.768	0	0	\\
6.5e-10	0.768	0	0	\\
7e-10	0.768	0	0	\\
7.5e-10	0.768	0	0	\\
8e-10	0.768	1	1	\\
8.5e-10	0.768	1	1	\\
9e-10	0.768	1	1	\\
9.5e-10	0.768	1	1	\\
1e-09	0.768	1	1	\\
1.05e-09	0.768	1	1	\\
1.1e-09	0.768	1	1	\\
1.15e-09	0.768	1	1	\\
1.2e-09	0.768	1	1	\\
1.25e-09	0.768	1	1	\\
1.3e-09	0.768	0	0	\\
1.35e-09	0.768	0	0	\\
1.4e-09	0.768	0	0	\\
1.45e-09	0.768	0	0	\\
1.5e-09	0.768	0	0	\\
1.55e-09	0.768	0	0	\\
1.6e-09	0.768	0	0	\\
1.65e-09	0.768	-0.666666666666667	-0.666666666666667	\\
1.7e-09	0.768	-0.666666666666667	-0.666666666666667	\\
1.75e-09	0.768	-0.666666666666667	-0.666666666666667	\\
1.8e-09	0.768	-0.666666666666667	-0.666666666666667	\\
1.85e-09	0.768	-0.666666666666667	-0.666666666666667	\\
1.9e-09	0.768	-0.666666666666667	-0.666666666666667	\\
1.95e-09	0.768	-0.666666666666667	-0.666666666666667	\\
2e-09	0.768	-0.666666666666667	-0.666666666666667	\\
2.05e-09	0.768	-0.666666666666667	-0.666666666666667	\\
2.1e-09	0.768	-0.666666666666667	-0.666666666666667	\\
2.15e-09	0.768	0	0	\\
2.2e-09	0.768	0	0	\\
2.25e-09	0.768	0	0	\\
2.3e-09	0.768	0	0	\\
2.35e-09	0.768	0	0	\\
2.4e-09	0.768	0	0	\\
2.45e-09	0.768	0	0	\\
2.5e-09	0.768	0	0	\\
2.55e-09	0.768	0	0	\\
2.6e-09	0.768	0	0	\\
2.65e-09	0.768	0	0	\\
2.7e-09	0.768	0	0	\\
2.75e-09	0.768	0	0	\\
2.8e-09	0.768	0	0	\\
2.85e-09	0.768	0	0	\\
2.9e-09	0.768	0	0	\\
2.95e-09	0.768	0	0	\\
3e-09	0.768	0	0	\\
3.05e-09	0.768	0	0	\\
3.1e-09	0.768	0	0	\\
3.15e-09	0.768	0	0	\\
3.2e-09	0.768	0.666666666666667	0.666666666666667	\\
3.25e-09	0.768	0.666666666666667	0.666666666666667	\\
3.3e-09	0.768	0.666666666666667	0.666666666666667	\\
3.35e-09	0.768	0.666666666666667	0.666666666666667	\\
3.4e-09	0.768	0.666666666666667	0.666666666666667	\\
3.45e-09	0.768	0.666666666666667	0.666666666666667	\\
3.5e-09	0.768	0.666666666666667	0.666666666666667	\\
3.55e-09	0.768	0.666666666666667	0.666666666666667	\\
3.6e-09	0.768	0.666666666666667	0.666666666666667	\\
3.65e-09	0.768	0.666666666666667	0.666666666666667	\\
3.7e-09	0.768	0	0	\\
3.75e-09	0.768	0	0	\\
3.8e-09	0.768	0	0	\\
3.85e-09	0.768	0	0	\\
3.9e-09	0.768	0	0	\\
3.95e-09	0.768	0	0	\\
4e-09	0.768	0	0	\\
4.05e-09	0.768	-0.444444444444444	-0.444444444444444	\\
4.1e-09	0.768	-0.444444444444444	-0.444444444444444	\\
4.15e-09	0.768	-0.444444444444444	-0.444444444444444	\\
4.2e-09	0.768	-0.444444444444444	-0.444444444444444	\\
4.25e-09	0.768	-0.444444444444444	-0.444444444444444	\\
4.3e-09	0.768	-0.444444444444444	-0.444444444444444	\\
4.35e-09	0.768	-0.444444444444444	-0.444444444444444	\\
4.4e-09	0.768	-0.444444444444444	-0.444444444444444	\\
4.45e-09	0.768	-0.444444444444444	-0.444444444444444	\\
4.5e-09	0.768	-0.444444444444444	-0.444444444444444	\\
4.55e-09	0.768	0	0	\\
4.6e-09	0.768	0	0	\\
4.65e-09	0.768	0	0	\\
4.7e-09	0.768	0	0	\\
4.75e-09	0.768	0	0	\\
4.8e-09	0.768	0	0	\\
4.85e-09	0.768	0	0	\\
4.9e-09	0.768	0	0	\\
4.95e-09	0.768	0	0	\\
5e-09	0.768	0	0	\\
5e-09	0.768	0	nan	\\
5e-09	0.768	-0.666666666666667	0.166666666666667	\\
5e-09	0.768	-0.666666666666667	nan	\\
0	0.78	-0.666666666666667	nan	\\
0	0.78	0	0.166666666666667	\\
0	0.78	0	0	\\
5e-11	0.78	0	0	\\
1e-10	0.78	0	0	\\
1.5e-10	0.78	0	0	\\
2e-10	0.78	0	0	\\
2.5e-10	0.78	0	0	\\
3e-10	0.78	0	0	\\
3.5e-10	0.78	0	0	\\
4e-10	0.78	0	0	\\
4.5e-10	0.78	0	0	\\
5e-10	0.78	0	0	\\
5.5e-10	0.78	0	0	\\
6e-10	0.78	0	0	\\
6.5e-10	0.78	0	0	\\
7e-10	0.78	0	0	\\
7.5e-10	0.78	0	0	\\
8e-10	0.78	1	1	\\
8.5e-10	0.78	1	1	\\
9e-10	0.78	1	1	\\
9.5e-10	0.78	1	1	\\
1e-09	0.78	1	1	\\
1.05e-09	0.78	1	1	\\
1.1e-09	0.78	1	1	\\
1.15e-09	0.78	1	1	\\
1.2e-09	0.78	1	1	\\
1.25e-09	0.78	1	1	\\
1.3e-09	0.78	0	0	\\
1.35e-09	0.78	0	0	\\
1.4e-09	0.78	0	0	\\
1.45e-09	0.78	0	0	\\
1.5e-09	0.78	0	0	\\
1.55e-09	0.78	0	0	\\
1.6e-09	0.78	0	0	\\
1.65e-09	0.78	-0.666666666666667	-0.666666666666667	\\
1.7e-09	0.78	-0.666666666666667	-0.666666666666667	\\
1.75e-09	0.78	-0.666666666666667	-0.666666666666667	\\
1.8e-09	0.78	-0.666666666666667	-0.666666666666667	\\
1.85e-09	0.78	-0.666666666666667	-0.666666666666667	\\
1.9e-09	0.78	-0.666666666666667	-0.666666666666667	\\
1.95e-09	0.78	-0.666666666666667	-0.666666666666667	\\
2e-09	0.78	-0.666666666666667	-0.666666666666667	\\
2.05e-09	0.78	-0.666666666666667	-0.666666666666667	\\
2.1e-09	0.78	-0.666666666666667	-0.666666666666667	\\
2.15e-09	0.78	0	0	\\
2.2e-09	0.78	0	0	\\
2.25e-09	0.78	0	0	\\
2.3e-09	0.78	0	0	\\
2.35e-09	0.78	0	0	\\
2.4e-09	0.78	0	0	\\
2.45e-09	0.78	0	0	\\
2.5e-09	0.78	0	0	\\
2.55e-09	0.78	0	0	\\
2.6e-09	0.78	0	0	\\
2.65e-09	0.78	0	0	\\
2.7e-09	0.78	0	0	\\
2.75e-09	0.78	0	0	\\
2.8e-09	0.78	0	0	\\
2.85e-09	0.78	0	0	\\
2.9e-09	0.78	0	0	\\
2.95e-09	0.78	0	0	\\
3e-09	0.78	0	0	\\
3.05e-09	0.78	0	0	\\
3.1e-09	0.78	0	0	\\
3.15e-09	0.78	0	0	\\
3.2e-09	0.78	0.666666666666667	0.666666666666667	\\
3.25e-09	0.78	0.666666666666667	0.666666666666667	\\
3.3e-09	0.78	0.666666666666667	0.666666666666667	\\
3.35e-09	0.78	0.666666666666667	0.666666666666667	\\
3.4e-09	0.78	0.666666666666667	0.666666666666667	\\
3.45e-09	0.78	0.666666666666667	0.666666666666667	\\
3.5e-09	0.78	0.666666666666667	0.666666666666667	\\
3.55e-09	0.78	0.666666666666667	0.666666666666667	\\
3.6e-09	0.78	0.666666666666667	0.666666666666667	\\
3.65e-09	0.78	0.666666666666667	0.666666666666667	\\
3.7e-09	0.78	0	0	\\
3.75e-09	0.78	0	0	\\
3.8e-09	0.78	0	0	\\
3.85e-09	0.78	0	0	\\
3.9e-09	0.78	0	0	\\
3.95e-09	0.78	0	0	\\
4e-09	0.78	0	0	\\
4.05e-09	0.78	-0.444444444444444	-0.444444444444444	\\
4.1e-09	0.78	-0.444444444444444	-0.444444444444444	\\
4.15e-09	0.78	-0.444444444444444	-0.444444444444444	\\
4.2e-09	0.78	-0.444444444444444	-0.444444444444444	\\
4.25e-09	0.78	-0.444444444444444	-0.444444444444444	\\
4.3e-09	0.78	-0.444444444444444	-0.444444444444444	\\
4.35e-09	0.78	-0.444444444444444	-0.444444444444444	\\
4.4e-09	0.78	-0.444444444444444	-0.444444444444444	\\
4.45e-09	0.78	-0.444444444444444	-0.444444444444444	\\
4.5e-09	0.78	-0.444444444444444	-0.444444444444444	\\
4.55e-09	0.78	0	0	\\
4.6e-09	0.78	0	0	\\
4.65e-09	0.78	0	0	\\
4.7e-09	0.78	0	0	\\
4.75e-09	0.78	0	0	\\
4.8e-09	0.78	0	0	\\
4.85e-09	0.78	0	0	\\
4.9e-09	0.78	0	0	\\
4.95e-09	0.78	0	0	\\
5e-09	0.78	0	0	\\
5e-09	0.78	0	nan	\\
5e-09	0.78	-0.666666666666667	0.166666666666667	\\
5e-09	0.78	-0.666666666666667	nan	\\
0	0.792	-0.666666666666667	nan	\\
0	0.792	0	0.166666666666667	\\
0	0.792	0	0	\\
5e-11	0.792	0	0	\\
1e-10	0.792	0	0	\\
1.5e-10	0.792	0	0	\\
2e-10	0.792	0	0	\\
2.5e-10	0.792	0	0	\\
3e-10	0.792	0	0	\\
3.5e-10	0.792	0	0	\\
4e-10	0.792	0	0	\\
4.5e-10	0.792	0	0	\\
5e-10	0.792	0	0	\\
5.5e-10	0.792	0	0	\\
6e-10	0.792	0	0	\\
6.5e-10	0.792	0	0	\\
7e-10	0.792	0	0	\\
7.5e-10	0.792	0	0	\\
8e-10	0.792	1	1	\\
8.5e-10	0.792	1	1	\\
9e-10	0.792	1	1	\\
9.5e-10	0.792	1	1	\\
1e-09	0.792	1	1	\\
1.05e-09	0.792	1	1	\\
1.1e-09	0.792	1	1	\\
1.15e-09	0.792	1	1	\\
1.2e-09	0.792	1	1	\\
1.25e-09	0.792	1	1	\\
1.3e-09	0.792	0	0	\\
1.35e-09	0.792	0	0	\\
1.4e-09	0.792	0	0	\\
1.45e-09	0.792	0	0	\\
1.5e-09	0.792	0	0	\\
1.55e-09	0.792	0	0	\\
1.6e-09	0.792	0	0	\\
1.65e-09	0.792	-0.666666666666667	-0.666666666666667	\\
1.7e-09	0.792	-0.666666666666667	-0.666666666666667	\\
1.75e-09	0.792	-0.666666666666667	-0.666666666666667	\\
1.8e-09	0.792	-0.666666666666667	-0.666666666666667	\\
1.85e-09	0.792	-0.666666666666667	-0.666666666666667	\\
1.9e-09	0.792	-0.666666666666667	-0.666666666666667	\\
1.95e-09	0.792	-0.666666666666667	-0.666666666666667	\\
2e-09	0.792	-0.666666666666667	-0.666666666666667	\\
2.05e-09	0.792	-0.666666666666667	-0.666666666666667	\\
2.1e-09	0.792	-0.666666666666667	-0.666666666666667	\\
2.15e-09	0.792	0	0	\\
2.2e-09	0.792	0	0	\\
2.25e-09	0.792	0	0	\\
2.3e-09	0.792	0	0	\\
2.35e-09	0.792	0	0	\\
2.4e-09	0.792	0	0	\\
2.45e-09	0.792	0	0	\\
2.5e-09	0.792	0	0	\\
2.55e-09	0.792	0	0	\\
2.6e-09	0.792	0	0	\\
2.65e-09	0.792	0	0	\\
2.7e-09	0.792	0	0	\\
2.75e-09	0.792	0	0	\\
2.8e-09	0.792	0	0	\\
2.85e-09	0.792	0	0	\\
2.9e-09	0.792	0	0	\\
2.95e-09	0.792	0	0	\\
3e-09	0.792	0	0	\\
3.05e-09	0.792	0	0	\\
3.1e-09	0.792	0	0	\\
3.15e-09	0.792	0	0	\\
3.2e-09	0.792	0.666666666666667	0.666666666666667	\\
3.25e-09	0.792	0.666666666666667	0.666666666666667	\\
3.3e-09	0.792	0.666666666666667	0.666666666666667	\\
3.35e-09	0.792	0.666666666666667	0.666666666666667	\\
3.4e-09	0.792	0.666666666666667	0.666666666666667	\\
3.45e-09	0.792	0.666666666666667	0.666666666666667	\\
3.5e-09	0.792	0.666666666666667	0.666666666666667	\\
3.55e-09	0.792	0.666666666666667	0.666666666666667	\\
3.6e-09	0.792	0.666666666666667	0.666666666666667	\\
3.65e-09	0.792	0.666666666666667	0.666666666666667	\\
3.7e-09	0.792	0	0	\\
3.75e-09	0.792	0	0	\\
3.8e-09	0.792	0	0	\\
3.85e-09	0.792	0	0	\\
3.9e-09	0.792	0	0	\\
3.95e-09	0.792	0	0	\\
4e-09	0.792	0	0	\\
4.05e-09	0.792	-0.444444444444444	-0.444444444444444	\\
4.1e-09	0.792	-0.444444444444444	-0.444444444444444	\\
4.15e-09	0.792	-0.444444444444444	-0.444444444444444	\\
4.2e-09	0.792	-0.444444444444444	-0.444444444444444	\\
4.25e-09	0.792	-0.444444444444444	-0.444444444444444	\\
4.3e-09	0.792	-0.444444444444444	-0.444444444444444	\\
4.35e-09	0.792	-0.444444444444444	-0.444444444444444	\\
4.4e-09	0.792	-0.444444444444444	-0.444444444444444	\\
4.45e-09	0.792	-0.444444444444444	-0.444444444444444	\\
4.5e-09	0.792	-0.444444444444444	-0.444444444444444	\\
4.55e-09	0.792	0	0	\\
4.6e-09	0.792	0	0	\\
4.65e-09	0.792	0	0	\\
4.7e-09	0.792	0	0	\\
4.75e-09	0.792	0	0	\\
4.8e-09	0.792	0	0	\\
4.85e-09	0.792	0	0	\\
4.9e-09	0.792	0	0	\\
4.95e-09	0.792	0	0	\\
5e-09	0.792	0	0	\\
5e-09	0.792	0	nan	\\
5e-09	0.792	-0.666666666666667	0.166666666666667	\\
5e-09	0.792	-0.666666666666667	nan	\\
0	0.804	-0.666666666666667	nan	\\
0	0.804	0	0.166666666666667	\\
0	0.804	0	0	\\
5e-11	0.804	0	0	\\
1e-10	0.804	0	0	\\
1.5e-10	0.804	0	0	\\
2e-10	0.804	0	0	\\
2.5e-10	0.804	0	0	\\
3e-10	0.804	0	0	\\
3.5e-10	0.804	0	0	\\
4e-10	0.804	0	0	\\
4.5e-10	0.804	0	0	\\
5e-10	0.804	0	0	\\
5.5e-10	0.804	0	0	\\
6e-10	0.804	0	0	\\
6.5e-10	0.804	0	0	\\
7e-10	0.804	0	0	\\
7.5e-10	0.804	0	0	\\
8e-10	0.804	0	0	\\
8.5e-10	0.804	1	1	\\
9e-10	0.804	1	1	\\
9.5e-10	0.804	1	1	\\
1e-09	0.804	1	1	\\
1.05e-09	0.804	1	1	\\
1.1e-09	0.804	1	1	\\
1.15e-09	0.804	1	1	\\
1.2e-09	0.804	1	1	\\
1.25e-09	0.804	1	1	\\
1.3e-09	0.804	1	1	\\
1.35e-09	0.804	0	0	\\
1.4e-09	0.804	0	0	\\
1.45e-09	0.804	0	0	\\
1.5e-09	0.804	0	0	\\
1.55e-09	0.804	0	0	\\
1.6e-09	0.804	-0.666666666666667	-0.666666666666667	\\
1.65e-09	0.804	-0.666666666666667	-0.666666666666667	\\
1.7e-09	0.804	-0.666666666666667	-0.666666666666667	\\
1.75e-09	0.804	-0.666666666666667	-0.666666666666667	\\
1.8e-09	0.804	-0.666666666666667	-0.666666666666667	\\
1.85e-09	0.804	-0.666666666666667	-0.666666666666667	\\
1.9e-09	0.804	-0.666666666666667	-0.666666666666667	\\
1.95e-09	0.804	-0.666666666666667	-0.666666666666667	\\
2e-09	0.804	-0.666666666666667	-0.666666666666667	\\
2.05e-09	0.804	-0.666666666666667	-0.666666666666667	\\
2.1e-09	0.804	0	0	\\
2.15e-09	0.804	0	0	\\
2.2e-09	0.804	0	0	\\
2.25e-09	0.804	0	0	\\
2.3e-09	0.804	0	0	\\
2.35e-09	0.804	0	0	\\
2.4e-09	0.804	0	0	\\
2.45e-09	0.804	0	0	\\
2.5e-09	0.804	0	0	\\
2.55e-09	0.804	0	0	\\
2.6e-09	0.804	0	0	\\
2.65e-09	0.804	0	0	\\
2.7e-09	0.804	0	0	\\
2.75e-09	0.804	0	0	\\
2.8e-09	0.804	0	0	\\
2.85e-09	0.804	0	0	\\
2.9e-09	0.804	0	0	\\
2.95e-09	0.804	0	0	\\
3e-09	0.804	0	0	\\
3.05e-09	0.804	0	0	\\
3.1e-09	0.804	0	0	\\
3.15e-09	0.804	0	0	\\
3.2e-09	0.804	0	0	\\
3.25e-09	0.804	0.666666666666667	0.666666666666667	\\
3.3e-09	0.804	0.666666666666667	0.666666666666667	\\
3.35e-09	0.804	0.666666666666667	0.666666666666667	\\
3.4e-09	0.804	0.666666666666667	0.666666666666667	\\
3.45e-09	0.804	0.666666666666667	0.666666666666667	\\
3.5e-09	0.804	0.666666666666667	0.666666666666667	\\
3.55e-09	0.804	0.666666666666667	0.666666666666667	\\
3.6e-09	0.804	0.666666666666667	0.666666666666667	\\
3.65e-09	0.804	0.666666666666667	0.666666666666667	\\
3.7e-09	0.804	0.666666666666667	0.666666666666667	\\
3.75e-09	0.804	0	0	\\
3.8e-09	0.804	0	0	\\
3.85e-09	0.804	0	0	\\
3.9e-09	0.804	0	0	\\
3.95e-09	0.804	0	0	\\
4e-09	0.804	-0.444444444444444	-0.444444444444444	\\
4.05e-09	0.804	-0.444444444444444	-0.444444444444444	\\
4.1e-09	0.804	-0.444444444444444	-0.444444444444444	\\
4.15e-09	0.804	-0.444444444444444	-0.444444444444444	\\
4.2e-09	0.804	-0.444444444444444	-0.444444444444444	\\
4.25e-09	0.804	-0.444444444444444	-0.444444444444444	\\
4.3e-09	0.804	-0.444444444444444	-0.444444444444444	\\
4.35e-09	0.804	-0.444444444444444	-0.444444444444444	\\
4.4e-09	0.804	-0.444444444444444	-0.444444444444444	\\
4.45e-09	0.804	-0.444444444444444	-0.444444444444444	\\
4.5e-09	0.804	0	0	\\
4.55e-09	0.804	0	0	\\
4.6e-09	0.804	0	0	\\
4.65e-09	0.804	0	0	\\
4.7e-09	0.804	0	0	\\
4.75e-09	0.804	0	0	\\
4.8e-09	0.804	0	0	\\
4.85e-09	0.804	0	0	\\
4.9e-09	0.804	0	0	\\
4.95e-09	0.804	0	0	\\
5e-09	0.804	0	0	\\
5e-09	0.804	0	nan	\\
5e-09	0.804	-0.666666666666667	0.166666666666667	\\
5e-09	0.804	-0.666666666666667	nan	\\
0	0.816	-0.666666666666667	nan	\\
0	0.816	0	0.166666666666667	\\
0	0.816	0	0	\\
5e-11	0.816	0	0	\\
1e-10	0.816	0	0	\\
1.5e-10	0.816	0	0	\\
2e-10	0.816	0	0	\\
2.5e-10	0.816	0	0	\\
3e-10	0.816	0	0	\\
3.5e-10	0.816	0	0	\\
4e-10	0.816	0	0	\\
4.5e-10	0.816	0	0	\\
5e-10	0.816	0	0	\\
5.5e-10	0.816	0	0	\\
6e-10	0.816	0	0	\\
6.5e-10	0.816	0	0	\\
7e-10	0.816	0	0	\\
7.5e-10	0.816	0	0	\\
8e-10	0.816	0	0	\\
8.5e-10	0.816	1	1	\\
9e-10	0.816	1	1	\\
9.5e-10	0.816	1	1	\\
1e-09	0.816	1	1	\\
1.05e-09	0.816	1	1	\\
1.1e-09	0.816	1	1	\\
1.15e-09	0.816	1	1	\\
1.2e-09	0.816	1	1	\\
1.25e-09	0.816	1	1	\\
1.3e-09	0.816	1	1	\\
1.35e-09	0.816	0	0	\\
1.4e-09	0.816	0	0	\\
1.45e-09	0.816	0	0	\\
1.5e-09	0.816	0	0	\\
1.55e-09	0.816	0	0	\\
1.6e-09	0.816	-0.666666666666667	-0.666666666666667	\\
1.65e-09	0.816	-0.666666666666667	-0.666666666666667	\\
1.7e-09	0.816	-0.666666666666667	-0.666666666666667	\\
1.75e-09	0.816	-0.666666666666667	-0.666666666666667	\\
1.8e-09	0.816	-0.666666666666667	-0.666666666666667	\\
1.85e-09	0.816	-0.666666666666667	-0.666666666666667	\\
1.9e-09	0.816	-0.666666666666667	-0.666666666666667	\\
1.95e-09	0.816	-0.666666666666667	-0.666666666666667	\\
2e-09	0.816	-0.666666666666667	-0.666666666666667	\\
2.05e-09	0.816	-0.666666666666667	-0.666666666666667	\\
2.1e-09	0.816	0	0	\\
2.15e-09	0.816	0	0	\\
2.2e-09	0.816	0	0	\\
2.25e-09	0.816	0	0	\\
2.3e-09	0.816	0	0	\\
2.35e-09	0.816	0	0	\\
2.4e-09	0.816	0	0	\\
2.45e-09	0.816	0	0	\\
2.5e-09	0.816	0	0	\\
2.55e-09	0.816	0	0	\\
2.6e-09	0.816	0	0	\\
2.65e-09	0.816	0	0	\\
2.7e-09	0.816	0	0	\\
2.75e-09	0.816	0	0	\\
2.8e-09	0.816	0	0	\\
2.85e-09	0.816	0	0	\\
2.9e-09	0.816	0	0	\\
2.95e-09	0.816	0	0	\\
3e-09	0.816	0	0	\\
3.05e-09	0.816	0	0	\\
3.1e-09	0.816	0	0	\\
3.15e-09	0.816	0	0	\\
3.2e-09	0.816	0	0	\\
3.25e-09	0.816	0.666666666666667	0.666666666666667	\\
3.3e-09	0.816	0.666666666666667	0.666666666666667	\\
3.35e-09	0.816	0.666666666666667	0.666666666666667	\\
3.4e-09	0.816	0.666666666666667	0.666666666666667	\\
3.45e-09	0.816	0.666666666666667	0.666666666666667	\\
3.5e-09	0.816	0.666666666666667	0.666666666666667	\\
3.55e-09	0.816	0.666666666666667	0.666666666666667	\\
3.6e-09	0.816	0.666666666666667	0.666666666666667	\\
3.65e-09	0.816	0.666666666666667	0.666666666666667	\\
3.7e-09	0.816	0.666666666666667	0.666666666666667	\\
3.75e-09	0.816	0	0	\\
3.8e-09	0.816	0	0	\\
3.85e-09	0.816	0	0	\\
3.9e-09	0.816	0	0	\\
3.95e-09	0.816	0	0	\\
4e-09	0.816	-0.444444444444444	-0.444444444444444	\\
4.05e-09	0.816	-0.444444444444444	-0.444444444444444	\\
4.1e-09	0.816	-0.444444444444444	-0.444444444444444	\\
4.15e-09	0.816	-0.444444444444444	-0.444444444444444	\\
4.2e-09	0.816	-0.444444444444444	-0.444444444444444	\\
4.25e-09	0.816	-0.444444444444444	-0.444444444444444	\\
4.3e-09	0.816	-0.444444444444444	-0.444444444444444	\\
4.35e-09	0.816	-0.444444444444444	-0.444444444444444	\\
4.4e-09	0.816	-0.444444444444444	-0.444444444444444	\\
4.45e-09	0.816	-0.444444444444444	-0.444444444444444	\\
4.5e-09	0.816	0	0	\\
4.55e-09	0.816	0	0	\\
4.6e-09	0.816	0	0	\\
4.65e-09	0.816	0	0	\\
4.7e-09	0.816	0	0	\\
4.75e-09	0.816	0	0	\\
4.8e-09	0.816	0	0	\\
4.85e-09	0.816	0	0	\\
4.9e-09	0.816	0	0	\\
4.95e-09	0.816	0	0	\\
5e-09	0.816	0	0	\\
5e-09	0.816	0	nan	\\
5e-09	0.816	-0.666666666666667	0.166666666666667	\\
5e-09	0.816	-0.666666666666667	nan	\\
0	0.828	-0.666666666666667	nan	\\
0	0.828	0	0.166666666666667	\\
0	0.828	0	0	\\
5e-11	0.828	0	0	\\
1e-10	0.828	0	0	\\
1.5e-10	0.828	0	0	\\
2e-10	0.828	0	0	\\
2.5e-10	0.828	0	0	\\
3e-10	0.828	0	0	\\
3.5e-10	0.828	0	0	\\
4e-10	0.828	0	0	\\
4.5e-10	0.828	0	0	\\
5e-10	0.828	0	0	\\
5.5e-10	0.828	0	0	\\
6e-10	0.828	0	0	\\
6.5e-10	0.828	0	0	\\
7e-10	0.828	0	0	\\
7.5e-10	0.828	0	0	\\
8e-10	0.828	0	0	\\
8.5e-10	0.828	1	1	\\
9e-10	0.828	1	1	\\
9.5e-10	0.828	1	1	\\
1e-09	0.828	1	1	\\
1.05e-09	0.828	1	1	\\
1.1e-09	0.828	1	1	\\
1.15e-09	0.828	1	1	\\
1.2e-09	0.828	1	1	\\
1.25e-09	0.828	1	1	\\
1.3e-09	0.828	1	1	\\
1.35e-09	0.828	0	0	\\
1.4e-09	0.828	0	0	\\
1.45e-09	0.828	0	0	\\
1.5e-09	0.828	0	0	\\
1.55e-09	0.828	0	0	\\
1.6e-09	0.828	-0.666666666666667	-0.666666666666667	\\
1.65e-09	0.828	-0.666666666666667	-0.666666666666667	\\
1.7e-09	0.828	-0.666666666666667	-0.666666666666667	\\
1.75e-09	0.828	-0.666666666666667	-0.666666666666667	\\
1.8e-09	0.828	-0.666666666666667	-0.666666666666667	\\
1.85e-09	0.828	-0.666666666666667	-0.666666666666667	\\
1.9e-09	0.828	-0.666666666666667	-0.666666666666667	\\
1.95e-09	0.828	-0.666666666666667	-0.666666666666667	\\
2e-09	0.828	-0.666666666666667	-0.666666666666667	\\
2.05e-09	0.828	-0.666666666666667	-0.666666666666667	\\
2.1e-09	0.828	0	0	\\
2.15e-09	0.828	0	0	\\
2.2e-09	0.828	0	0	\\
2.25e-09	0.828	0	0	\\
2.3e-09	0.828	0	0	\\
2.35e-09	0.828	0	0	\\
2.4e-09	0.828	0	0	\\
2.45e-09	0.828	0	0	\\
2.5e-09	0.828	0	0	\\
2.55e-09	0.828	0	0	\\
2.6e-09	0.828	0	0	\\
2.65e-09	0.828	0	0	\\
2.7e-09	0.828	0	0	\\
2.75e-09	0.828	0	0	\\
2.8e-09	0.828	0	0	\\
2.85e-09	0.828	0	0	\\
2.9e-09	0.828	0	0	\\
2.95e-09	0.828	0	0	\\
3e-09	0.828	0	0	\\
3.05e-09	0.828	0	0	\\
3.1e-09	0.828	0	0	\\
3.15e-09	0.828	0	0	\\
3.2e-09	0.828	0	0	\\
3.25e-09	0.828	0.666666666666667	0.666666666666667	\\
3.3e-09	0.828	0.666666666666667	0.666666666666667	\\
3.35e-09	0.828	0.666666666666667	0.666666666666667	\\
3.4e-09	0.828	0.666666666666667	0.666666666666667	\\
3.45e-09	0.828	0.666666666666667	0.666666666666667	\\
3.5e-09	0.828	0.666666666666667	0.666666666666667	\\
3.55e-09	0.828	0.666666666666667	0.666666666666667	\\
3.6e-09	0.828	0.666666666666667	0.666666666666667	\\
3.65e-09	0.828	0.666666666666667	0.666666666666667	\\
3.7e-09	0.828	0.666666666666667	0.666666666666667	\\
3.75e-09	0.828	0	0	\\
3.8e-09	0.828	0	0	\\
3.85e-09	0.828	0	0	\\
3.9e-09	0.828	0	0	\\
3.95e-09	0.828	0	0	\\
4e-09	0.828	-0.444444444444444	-0.444444444444444	\\
4.05e-09	0.828	-0.444444444444444	-0.444444444444444	\\
4.1e-09	0.828	-0.444444444444444	-0.444444444444444	\\
4.15e-09	0.828	-0.444444444444444	-0.444444444444444	\\
4.2e-09	0.828	-0.444444444444444	-0.444444444444444	\\
4.25e-09	0.828	-0.444444444444444	-0.444444444444444	\\
4.3e-09	0.828	-0.444444444444444	-0.444444444444444	\\
4.35e-09	0.828	-0.444444444444444	-0.444444444444444	\\
4.4e-09	0.828	-0.444444444444444	-0.444444444444444	\\
4.45e-09	0.828	-0.444444444444444	-0.444444444444444	\\
4.5e-09	0.828	0	0	\\
4.55e-09	0.828	0	0	\\
4.6e-09	0.828	0	0	\\
4.65e-09	0.828	0	0	\\
4.7e-09	0.828	0	0	\\
4.75e-09	0.828	0	0	\\
4.8e-09	0.828	0	0	\\
4.85e-09	0.828	0	0	\\
4.9e-09	0.828	0	0	\\
4.95e-09	0.828	0	0	\\
5e-09	0.828	0	0	\\
5e-09	0.828	0	nan	\\
5e-09	0.828	-0.666666666666667	0.166666666666667	\\
5e-09	0.828	-0.666666666666667	nan	\\
0	0.84	-0.666666666666667	nan	\\
0	0.84	0	0.166666666666667	\\
0	0.84	0	0	\\
5e-11	0.84	0	0	\\
1e-10	0.84	0	0	\\
1.5e-10	0.84	0	0	\\
2e-10	0.84	0	0	\\
2.5e-10	0.84	0	0	\\
3e-10	0.84	0	0	\\
3.5e-10	0.84	0	0	\\
4e-10	0.84	0	0	\\
4.5e-10	0.84	0	0	\\
5e-10	0.84	0	0	\\
5.5e-10	0.84	0	0	\\
6e-10	0.84	0	0	\\
6.5e-10	0.84	0	0	\\
7e-10	0.84	0	0	\\
7.5e-10	0.84	0	0	\\
8e-10	0.84	0	0	\\
8.5e-10	0.84	1	1	\\
9e-10	0.84	1	1	\\
9.5e-10	0.84	1	1	\\
1e-09	0.84	1	1	\\
1.05e-09	0.84	1	1	\\
1.1e-09	0.84	1	1	\\
1.15e-09	0.84	1	1	\\
1.2e-09	0.84	1	1	\\
1.25e-09	0.84	1	1	\\
1.3e-09	0.84	1	1	\\
1.35e-09	0.84	0	0	\\
1.4e-09	0.84	0	0	\\
1.45e-09	0.84	0	0	\\
1.5e-09	0.84	0	0	\\
1.55e-09	0.84	0	0	\\
1.6e-09	0.84	-0.666666666666667	-0.666666666666667	\\
1.65e-09	0.84	-0.666666666666667	-0.666666666666667	\\
1.7e-09	0.84	-0.666666666666667	-0.666666666666667	\\
1.75e-09	0.84	-0.666666666666667	-0.666666666666667	\\
1.8e-09	0.84	-0.666666666666667	-0.666666666666667	\\
1.85e-09	0.84	-0.666666666666667	-0.666666666666667	\\
1.9e-09	0.84	-0.666666666666667	-0.666666666666667	\\
1.95e-09	0.84	-0.666666666666667	-0.666666666666667	\\
2e-09	0.84	-0.666666666666667	-0.666666666666667	\\
2.05e-09	0.84	-0.666666666666667	-0.666666666666667	\\
2.1e-09	0.84	0	0	\\
2.15e-09	0.84	0	0	\\
2.2e-09	0.84	0	0	\\
2.25e-09	0.84	0	0	\\
2.3e-09	0.84	0	0	\\
2.35e-09	0.84	0	0	\\
2.4e-09	0.84	0	0	\\
2.45e-09	0.84	0	0	\\
2.5e-09	0.84	0	0	\\
2.55e-09	0.84	0	0	\\
2.6e-09	0.84	0	0	\\
2.65e-09	0.84	0	0	\\
2.7e-09	0.84	0	0	\\
2.75e-09	0.84	0	0	\\
2.8e-09	0.84	0	0	\\
2.85e-09	0.84	0	0	\\
2.9e-09	0.84	0	0	\\
2.95e-09	0.84	0	0	\\
3e-09	0.84	0	0	\\
3.05e-09	0.84	0	0	\\
3.1e-09	0.84	0	0	\\
3.15e-09	0.84	0	0	\\
3.2e-09	0.84	0	0	\\
3.25e-09	0.84	0.666666666666667	0.666666666666667	\\
3.3e-09	0.84	0.666666666666667	0.666666666666667	\\
3.35e-09	0.84	0.666666666666667	0.666666666666667	\\
3.4e-09	0.84	0.666666666666667	0.666666666666667	\\
3.45e-09	0.84	0.666666666666667	0.666666666666667	\\
3.5e-09	0.84	0.666666666666667	0.666666666666667	\\
3.55e-09	0.84	0.666666666666667	0.666666666666667	\\
3.6e-09	0.84	0.666666666666667	0.666666666666667	\\
3.65e-09	0.84	0.666666666666667	0.666666666666667	\\
3.7e-09	0.84	0.666666666666667	0.666666666666667	\\
3.75e-09	0.84	0	0	\\
3.8e-09	0.84	0	0	\\
3.85e-09	0.84	0	0	\\
3.9e-09	0.84	0	0	\\
3.95e-09	0.84	0	0	\\
4e-09	0.84	-0.444444444444444	-0.444444444444444	\\
4.05e-09	0.84	-0.444444444444444	-0.444444444444444	\\
4.1e-09	0.84	-0.444444444444444	-0.444444444444444	\\
4.15e-09	0.84	-0.444444444444444	-0.444444444444444	\\
4.2e-09	0.84	-0.444444444444444	-0.444444444444444	\\
4.25e-09	0.84	-0.444444444444444	-0.444444444444444	\\
4.3e-09	0.84	-0.444444444444444	-0.444444444444444	\\
4.35e-09	0.84	-0.444444444444444	-0.444444444444444	\\
4.4e-09	0.84	-0.444444444444444	-0.444444444444444	\\
4.45e-09	0.84	-0.444444444444444	-0.444444444444444	\\
4.5e-09	0.84	0	0	\\
4.55e-09	0.84	0	0	\\
4.6e-09	0.84	0	0	\\
4.65e-09	0.84	0	0	\\
4.7e-09	0.84	0	0	\\
4.75e-09	0.84	0	0	\\
4.8e-09	0.84	0	0	\\
4.85e-09	0.84	0	0	\\
4.9e-09	0.84	0	0	\\
4.95e-09	0.84	0	0	\\
5e-09	0.84	0	0	\\
5e-09	0.84	0	nan	\\
5e-09	0.84	-0.666666666666667	0.166666666666667	\\
5e-09	0.84	-0.666666666666667	nan	\\
0	0.852	-0.666666666666667	nan	\\
0	0.852	0	0.166666666666667	\\
0	0.852	0	0	\\
5e-11	0.852	0	0	\\
1e-10	0.852	0	0	\\
1.5e-10	0.852	0	0	\\
2e-10	0.852	0	0	\\
2.5e-10	0.852	0	0	\\
3e-10	0.852	0	0	\\
3.5e-10	0.852	0	0	\\
4e-10	0.852	0	0	\\
4.5e-10	0.852	0	0	\\
5e-10	0.852	0	0	\\
5.5e-10	0.852	0	0	\\
6e-10	0.852	0	0	\\
6.5e-10	0.852	0	0	\\
7e-10	0.852	0	0	\\
7.5e-10	0.852	0	0	\\
8e-10	0.852	0	0	\\
8.5e-10	0.852	0	0	\\
9e-10	0.852	1	1	\\
9.5e-10	0.852	1	1	\\
1e-09	0.852	1	1	\\
1.05e-09	0.852	1	1	\\
1.1e-09	0.852	1	1	\\
1.15e-09	0.852	1	1	\\
1.2e-09	0.852	1	1	\\
1.25e-09	0.852	1	1	\\
1.3e-09	0.852	1	1	\\
1.35e-09	0.852	1	1	\\
1.4e-09	0.852	0	0	\\
1.45e-09	0.852	0	0	\\
1.5e-09	0.852	0	0	\\
1.55e-09	0.852	-0.666666666666667	-0.666666666666667	\\
1.6e-09	0.852	-0.666666666666667	-0.666666666666667	\\
1.65e-09	0.852	-0.666666666666667	-0.666666666666667	\\
1.7e-09	0.852	-0.666666666666667	-0.666666666666667	\\
1.75e-09	0.852	-0.666666666666667	-0.666666666666667	\\
1.8e-09	0.852	-0.666666666666667	-0.666666666666667	\\
1.85e-09	0.852	-0.666666666666667	-0.666666666666667	\\
1.9e-09	0.852	-0.666666666666667	-0.666666666666667	\\
1.95e-09	0.852	-0.666666666666667	-0.666666666666667	\\
2e-09	0.852	-0.666666666666667	-0.666666666666667	\\
2.05e-09	0.852	0	0	\\
2.1e-09	0.852	0	0	\\
2.15e-09	0.852	0	0	\\
2.2e-09	0.852	0	0	\\
2.25e-09	0.852	0	0	\\
2.3e-09	0.852	0	0	\\
2.35e-09	0.852	0	0	\\
2.4e-09	0.852	0	0	\\
2.45e-09	0.852	0	0	\\
2.5e-09	0.852	0	0	\\
2.55e-09	0.852	0	0	\\
2.6e-09	0.852	0	0	\\
2.65e-09	0.852	0	0	\\
2.7e-09	0.852	0	0	\\
2.75e-09	0.852	0	0	\\
2.8e-09	0.852	0	0	\\
2.85e-09	0.852	0	0	\\
2.9e-09	0.852	0	0	\\
2.95e-09	0.852	0	0	\\
3e-09	0.852	0	0	\\
3.05e-09	0.852	0	0	\\
3.1e-09	0.852	0	0	\\
3.15e-09	0.852	0	0	\\
3.2e-09	0.852	0	0	\\
3.25e-09	0.852	0	0	\\
3.3e-09	0.852	0.666666666666667	0.666666666666667	\\
3.35e-09	0.852	0.666666666666667	0.666666666666667	\\
3.4e-09	0.852	0.666666666666667	0.666666666666667	\\
3.45e-09	0.852	0.666666666666667	0.666666666666667	\\
3.5e-09	0.852	0.666666666666667	0.666666666666667	\\
3.55e-09	0.852	0.666666666666667	0.666666666666667	\\
3.6e-09	0.852	0.666666666666667	0.666666666666667	\\
3.65e-09	0.852	0.666666666666667	0.666666666666667	\\
3.7e-09	0.852	0.666666666666667	0.666666666666667	\\
3.75e-09	0.852	0.666666666666667	0.666666666666667	\\
3.8e-09	0.852	0	0	\\
3.85e-09	0.852	0	0	\\
3.9e-09	0.852	0	0	\\
3.95e-09	0.852	-0.444444444444444	-0.444444444444444	\\
4e-09	0.852	-0.444444444444444	-0.444444444444444	\\
4.05e-09	0.852	-0.444444444444444	-0.444444444444444	\\
4.1e-09	0.852	-0.444444444444444	-0.444444444444444	\\
4.15e-09	0.852	-0.444444444444444	-0.444444444444444	\\
4.2e-09	0.852	-0.444444444444444	-0.444444444444444	\\
4.25e-09	0.852	-0.444444444444444	-0.444444444444444	\\
4.3e-09	0.852	-0.444444444444444	-0.444444444444444	\\
4.35e-09	0.852	-0.444444444444444	-0.444444444444444	\\
4.4e-09	0.852	-0.444444444444444	-0.444444444444444	\\
4.45e-09	0.852	0	0	\\
4.5e-09	0.852	0	0	\\
4.55e-09	0.852	0	0	\\
4.6e-09	0.852	0	0	\\
4.65e-09	0.852	0	0	\\
4.7e-09	0.852	0	0	\\
4.75e-09	0.852	0	0	\\
4.8e-09	0.852	0	0	\\
4.85e-09	0.852	0	0	\\
4.9e-09	0.852	0	0	\\
4.95e-09	0.852	0	0	\\
5e-09	0.852	0	0	\\
5e-09	0.852	0	nan	\\
5e-09	0.852	-0.666666666666667	0.166666666666667	\\
5e-09	0.852	-0.666666666666667	nan	\\
0	0.864	-0.666666666666667	nan	\\
0	0.864	0	0.166666666666667	\\
0	0.864	0	0	\\
5e-11	0.864	0	0	\\
1e-10	0.864	0	0	\\
1.5e-10	0.864	0	0	\\
2e-10	0.864	0	0	\\
2.5e-10	0.864	0	0	\\
3e-10	0.864	0	0	\\
3.5e-10	0.864	0	0	\\
4e-10	0.864	0	0	\\
4.5e-10	0.864	0	0	\\
5e-10	0.864	0	0	\\
5.5e-10	0.864	0	0	\\
6e-10	0.864	0	0	\\
6.5e-10	0.864	0	0	\\
7e-10	0.864	0	0	\\
7.5e-10	0.864	0	0	\\
8e-10	0.864	0	0	\\
8.5e-10	0.864	0	0	\\
9e-10	0.864	1	1	\\
9.5e-10	0.864	1	1	\\
1e-09	0.864	1	1	\\
1.05e-09	0.864	1	1	\\
1.1e-09	0.864	1	1	\\
1.15e-09	0.864	1	1	\\
1.2e-09	0.864	1	1	\\
1.25e-09	0.864	1	1	\\
1.3e-09	0.864	1	1	\\
1.35e-09	0.864	1	1	\\
1.4e-09	0.864	0	0	\\
1.45e-09	0.864	0	0	\\
1.5e-09	0.864	0	0	\\
1.55e-09	0.864	-0.666666666666667	-0.666666666666667	\\
1.6e-09	0.864	-0.666666666666667	-0.666666666666667	\\
1.65e-09	0.864	-0.666666666666667	-0.666666666666667	\\
1.7e-09	0.864	-0.666666666666667	-0.666666666666667	\\
1.75e-09	0.864	-0.666666666666667	-0.666666666666667	\\
1.8e-09	0.864	-0.666666666666667	-0.666666666666667	\\
1.85e-09	0.864	-0.666666666666667	-0.666666666666667	\\
1.9e-09	0.864	-0.666666666666667	-0.666666666666667	\\
1.95e-09	0.864	-0.666666666666667	-0.666666666666667	\\
2e-09	0.864	-0.666666666666667	-0.666666666666667	\\
2.05e-09	0.864	0	0	\\
2.1e-09	0.864	0	0	\\
2.15e-09	0.864	0	0	\\
2.2e-09	0.864	0	0	\\
2.25e-09	0.864	0	0	\\
2.3e-09	0.864	0	0	\\
2.35e-09	0.864	0	0	\\
2.4e-09	0.864	0	0	\\
2.45e-09	0.864	0	0	\\
2.5e-09	0.864	0	0	\\
2.55e-09	0.864	0	0	\\
2.6e-09	0.864	0	0	\\
2.65e-09	0.864	0	0	\\
2.7e-09	0.864	0	0	\\
2.75e-09	0.864	0	0	\\
2.8e-09	0.864	0	0	\\
2.85e-09	0.864	0	0	\\
2.9e-09	0.864	0	0	\\
2.95e-09	0.864	0	0	\\
3e-09	0.864	0	0	\\
3.05e-09	0.864	0	0	\\
3.1e-09	0.864	0	0	\\
3.15e-09	0.864	0	0	\\
3.2e-09	0.864	0	0	\\
3.25e-09	0.864	0	0	\\
3.3e-09	0.864	0.666666666666667	0.666666666666667	\\
3.35e-09	0.864	0.666666666666667	0.666666666666667	\\
3.4e-09	0.864	0.666666666666667	0.666666666666667	\\
3.45e-09	0.864	0.666666666666667	0.666666666666667	\\
3.5e-09	0.864	0.666666666666667	0.666666666666667	\\
3.55e-09	0.864	0.666666666666667	0.666666666666667	\\
3.6e-09	0.864	0.666666666666667	0.666666666666667	\\
3.65e-09	0.864	0.666666666666667	0.666666666666667	\\
3.7e-09	0.864	0.666666666666667	0.666666666666667	\\
3.75e-09	0.864	0.666666666666667	0.666666666666667	\\
3.8e-09	0.864	0	0	\\
3.85e-09	0.864	0	0	\\
3.9e-09	0.864	0	0	\\
3.95e-09	0.864	-0.444444444444444	-0.444444444444444	\\
4e-09	0.864	-0.444444444444444	-0.444444444444444	\\
4.05e-09	0.864	-0.444444444444444	-0.444444444444444	\\
4.1e-09	0.864	-0.444444444444444	-0.444444444444444	\\
4.15e-09	0.864	-0.444444444444444	-0.444444444444444	\\
4.2e-09	0.864	-0.444444444444444	-0.444444444444444	\\
4.25e-09	0.864	-0.444444444444444	-0.444444444444444	\\
4.3e-09	0.864	-0.444444444444444	-0.444444444444444	\\
4.35e-09	0.864	-0.444444444444444	-0.444444444444444	\\
4.4e-09	0.864	-0.444444444444444	-0.444444444444444	\\
4.45e-09	0.864	0	0	\\
4.5e-09	0.864	0	0	\\
4.55e-09	0.864	0	0	\\
4.6e-09	0.864	0	0	\\
4.65e-09	0.864	0	0	\\
4.7e-09	0.864	0	0	\\
4.75e-09	0.864	0	0	\\
4.8e-09	0.864	0	0	\\
4.85e-09	0.864	0	0	\\
4.9e-09	0.864	0	0	\\
4.95e-09	0.864	0	0	\\
5e-09	0.864	0	0	\\
5e-09	0.864	0	nan	\\
5e-09	0.864	-0.666666666666667	0.166666666666667	\\
5e-09	0.864	-0.666666666666667	nan	\\
0	0.876	-0.666666666666667	nan	\\
0	0.876	0	0.166666666666667	\\
0	0.876	0	0	\\
5e-11	0.876	0	0	\\
1e-10	0.876	0	0	\\
1.5e-10	0.876	0	0	\\
2e-10	0.876	0	0	\\
2.5e-10	0.876	0	0	\\
3e-10	0.876	0	0	\\
3.5e-10	0.876	0	0	\\
4e-10	0.876	0	0	\\
4.5e-10	0.876	0	0	\\
5e-10	0.876	0	0	\\
5.5e-10	0.876	0	0	\\
6e-10	0.876	0	0	\\
6.5e-10	0.876	0	0	\\
7e-10	0.876	0	0	\\
7.5e-10	0.876	0	0	\\
8e-10	0.876	0	0	\\
8.5e-10	0.876	0	0	\\
9e-10	0.876	1	1	\\
9.5e-10	0.876	1	1	\\
1e-09	0.876	1	1	\\
1.05e-09	0.876	1	1	\\
1.1e-09	0.876	1	1	\\
1.15e-09	0.876	1	1	\\
1.2e-09	0.876	1	1	\\
1.25e-09	0.876	1	1	\\
1.3e-09	0.876	1	1	\\
1.35e-09	0.876	1	1	\\
1.4e-09	0.876	0	0	\\
1.45e-09	0.876	0	0	\\
1.5e-09	0.876	0	0	\\
1.55e-09	0.876	-0.666666666666667	-0.666666666666667	\\
1.6e-09	0.876	-0.666666666666667	-0.666666666666667	\\
1.65e-09	0.876	-0.666666666666667	-0.666666666666667	\\
1.7e-09	0.876	-0.666666666666667	-0.666666666666667	\\
1.75e-09	0.876	-0.666666666666667	-0.666666666666667	\\
1.8e-09	0.876	-0.666666666666667	-0.666666666666667	\\
1.85e-09	0.876	-0.666666666666667	-0.666666666666667	\\
1.9e-09	0.876	-0.666666666666667	-0.666666666666667	\\
1.95e-09	0.876	-0.666666666666667	-0.666666666666667	\\
2e-09	0.876	-0.666666666666667	-0.666666666666667	\\
2.05e-09	0.876	0	0	\\
2.1e-09	0.876	0	0	\\
2.15e-09	0.876	0	0	\\
2.2e-09	0.876	0	0	\\
2.25e-09	0.876	0	0	\\
2.3e-09	0.876	0	0	\\
2.35e-09	0.876	0	0	\\
2.4e-09	0.876	0	0	\\
2.45e-09	0.876	0	0	\\
2.5e-09	0.876	0	0	\\
2.55e-09	0.876	0	0	\\
2.6e-09	0.876	0	0	\\
2.65e-09	0.876	0	0	\\
2.7e-09	0.876	0	0	\\
2.75e-09	0.876	0	0	\\
2.8e-09	0.876	0	0	\\
2.85e-09	0.876	0	0	\\
2.9e-09	0.876	0	0	\\
2.95e-09	0.876	0	0	\\
3e-09	0.876	0	0	\\
3.05e-09	0.876	0	0	\\
3.1e-09	0.876	0	0	\\
3.15e-09	0.876	0	0	\\
3.2e-09	0.876	0	0	\\
3.25e-09	0.876	0	0	\\
3.3e-09	0.876	0.666666666666667	0.666666666666667	\\
3.35e-09	0.876	0.666666666666667	0.666666666666667	\\
3.4e-09	0.876	0.666666666666667	0.666666666666667	\\
3.45e-09	0.876	0.666666666666667	0.666666666666667	\\
3.5e-09	0.876	0.666666666666667	0.666666666666667	\\
3.55e-09	0.876	0.666666666666667	0.666666666666667	\\
3.6e-09	0.876	0.666666666666667	0.666666666666667	\\
3.65e-09	0.876	0.666666666666667	0.666666666666667	\\
3.7e-09	0.876	0.666666666666667	0.666666666666667	\\
3.75e-09	0.876	0.666666666666667	0.666666666666667	\\
3.8e-09	0.876	0	0	\\
3.85e-09	0.876	0	0	\\
3.9e-09	0.876	0	0	\\
3.95e-09	0.876	-0.444444444444444	-0.444444444444444	\\
4e-09	0.876	-0.444444444444444	-0.444444444444444	\\
4.05e-09	0.876	-0.444444444444444	-0.444444444444444	\\
4.1e-09	0.876	-0.444444444444444	-0.444444444444444	\\
4.15e-09	0.876	-0.444444444444444	-0.444444444444444	\\
4.2e-09	0.876	-0.444444444444444	-0.444444444444444	\\
4.25e-09	0.876	-0.444444444444444	-0.444444444444444	\\
4.3e-09	0.876	-0.444444444444444	-0.444444444444444	\\
4.35e-09	0.876	-0.444444444444444	-0.444444444444444	\\
4.4e-09	0.876	-0.444444444444444	-0.444444444444444	\\
4.45e-09	0.876	0	0	\\
4.5e-09	0.876	0	0	\\
4.55e-09	0.876	0	0	\\
4.6e-09	0.876	0	0	\\
4.65e-09	0.876	0	0	\\
4.7e-09	0.876	0	0	\\
4.75e-09	0.876	0	0	\\
4.8e-09	0.876	0	0	\\
4.85e-09	0.876	0	0	\\
4.9e-09	0.876	0	0	\\
4.95e-09	0.876	0	0	\\
5e-09	0.876	0	0	\\
5e-09	0.876	0	nan	\\
5e-09	0.876	-0.666666666666667	0.166666666666667	\\
5e-09	0.876	-0.666666666666667	nan	\\
0	0.888	-0.666666666666667	nan	\\
0	0.888	0	0.166666666666667	\\
0	0.888	0	0	\\
5e-11	0.888	0	0	\\
1e-10	0.888	0	0	\\
1.5e-10	0.888	0	0	\\
2e-10	0.888	0	0	\\
2.5e-10	0.888	0	0	\\
3e-10	0.888	0	0	\\
3.5e-10	0.888	0	0	\\
4e-10	0.888	0	0	\\
4.5e-10	0.888	0	0	\\
5e-10	0.888	0	0	\\
5.5e-10	0.888	0	0	\\
6e-10	0.888	0	0	\\
6.5e-10	0.888	0	0	\\
7e-10	0.888	0	0	\\
7.5e-10	0.888	0	0	\\
8e-10	0.888	0	0	\\
8.5e-10	0.888	0	0	\\
9e-10	0.888	1	1	\\
9.5e-10	0.888	1	1	\\
1e-09	0.888	1	1	\\
1.05e-09	0.888	1	1	\\
1.1e-09	0.888	1	1	\\
1.15e-09	0.888	1	1	\\
1.2e-09	0.888	1	1	\\
1.25e-09	0.888	1	1	\\
1.3e-09	0.888	1	1	\\
1.35e-09	0.888	1	1	\\
1.4e-09	0.888	0	0	\\
1.45e-09	0.888	0	0	\\
1.5e-09	0.888	0	0	\\
1.55e-09	0.888	-0.666666666666667	-0.666666666666667	\\
1.6e-09	0.888	-0.666666666666667	-0.666666666666667	\\
1.65e-09	0.888	-0.666666666666667	-0.666666666666667	\\
1.7e-09	0.888	-0.666666666666667	-0.666666666666667	\\
1.75e-09	0.888	-0.666666666666667	-0.666666666666667	\\
1.8e-09	0.888	-0.666666666666667	-0.666666666666667	\\
1.85e-09	0.888	-0.666666666666667	-0.666666666666667	\\
1.9e-09	0.888	-0.666666666666667	-0.666666666666667	\\
1.95e-09	0.888	-0.666666666666667	-0.666666666666667	\\
2e-09	0.888	-0.666666666666667	-0.666666666666667	\\
2.05e-09	0.888	0	0	\\
2.1e-09	0.888	0	0	\\
2.15e-09	0.888	0	0	\\
2.2e-09	0.888	0	0	\\
2.25e-09	0.888	0	0	\\
2.3e-09	0.888	0	0	\\
2.35e-09	0.888	0	0	\\
2.4e-09	0.888	0	0	\\
2.45e-09	0.888	0	0	\\
2.5e-09	0.888	0	0	\\
2.55e-09	0.888	0	0	\\
2.6e-09	0.888	0	0	\\
2.65e-09	0.888	0	0	\\
2.7e-09	0.888	0	0	\\
2.75e-09	0.888	0	0	\\
2.8e-09	0.888	0	0	\\
2.85e-09	0.888	0	0	\\
2.9e-09	0.888	0	0	\\
2.95e-09	0.888	0	0	\\
3e-09	0.888	0	0	\\
3.05e-09	0.888	0	0	\\
3.1e-09	0.888	0	0	\\
3.15e-09	0.888	0	0	\\
3.2e-09	0.888	0	0	\\
3.25e-09	0.888	0	0	\\
3.3e-09	0.888	0.666666666666667	0.666666666666667	\\
3.35e-09	0.888	0.666666666666667	0.666666666666667	\\
3.4e-09	0.888	0.666666666666667	0.666666666666667	\\
3.45e-09	0.888	0.666666666666667	0.666666666666667	\\
3.5e-09	0.888	0.666666666666667	0.666666666666667	\\
3.55e-09	0.888	0.666666666666667	0.666666666666667	\\
3.6e-09	0.888	0.666666666666667	0.666666666666667	\\
3.65e-09	0.888	0.666666666666667	0.666666666666667	\\
3.7e-09	0.888	0.666666666666667	0.666666666666667	\\
3.75e-09	0.888	0.666666666666667	0.666666666666667	\\
3.8e-09	0.888	0	0	\\
3.85e-09	0.888	0	0	\\
3.9e-09	0.888	0	0	\\
3.95e-09	0.888	-0.444444444444444	-0.444444444444444	\\
4e-09	0.888	-0.444444444444444	-0.444444444444444	\\
4.05e-09	0.888	-0.444444444444444	-0.444444444444444	\\
4.1e-09	0.888	-0.444444444444444	-0.444444444444444	\\
4.15e-09	0.888	-0.444444444444444	-0.444444444444444	\\
4.2e-09	0.888	-0.444444444444444	-0.444444444444444	\\
4.25e-09	0.888	-0.444444444444444	-0.444444444444444	\\
4.3e-09	0.888	-0.444444444444444	-0.444444444444444	\\
4.35e-09	0.888	-0.444444444444444	-0.444444444444444	\\
4.4e-09	0.888	-0.444444444444444	-0.444444444444444	\\
4.45e-09	0.888	0	0	\\
4.5e-09	0.888	0	0	\\
4.55e-09	0.888	0	0	\\
4.6e-09	0.888	0	0	\\
4.65e-09	0.888	0	0	\\
4.7e-09	0.888	0	0	\\
4.75e-09	0.888	0	0	\\
4.8e-09	0.888	0	0	\\
4.85e-09	0.888	0	0	\\
4.9e-09	0.888	0	0	\\
4.95e-09	0.888	0	0	\\
5e-09	0.888	0	0	\\
5e-09	0.888	0	nan	\\
5e-09	0.888	-0.666666666666667	0.166666666666667	\\
5e-09	0.888	-0.666666666666667	nan	\\
0	0.9	-0.666666666666667	nan	\\
0	0.9	0	0.166666666666667	\\
0	0.9	0	0	\\
5e-11	0.9	0	0	\\
1e-10	0.9	0	0	\\
1.5e-10	0.9	0	0	\\
2e-10	0.9	0	0	\\
2.5e-10	0.9	0	0	\\
3e-10	0.9	0	0	\\
3.5e-10	0.9	0	0	\\
4e-10	0.9	0	0	\\
4.5e-10	0.9	0	0	\\
5e-10	0.9	0	0	\\
5.5e-10	0.9	0	0	\\
6e-10	0.9	0	0	\\
6.5e-10	0.9	0	0	\\
7e-10	0.9	0	0	\\
7.5e-10	0.9	0	0	\\
8e-10	0.9	0	0	\\
8.5e-10	0.9	0	0	\\
9e-10	0.9	1	1	\\
9.5e-10	0.9	1	1	\\
1e-09	0.9	1	1	\\
1.05e-09	0.9	1	1	\\
1.1e-09	0.9	1	1	\\
1.15e-09	0.9	1	1	\\
1.2e-09	0.9	1	1	\\
1.25e-09	0.9	1	1	\\
1.3e-09	0.9	1	1	\\
1.35e-09	0.9	1	1	\\
1.4e-09	0.9	0	0	\\
1.45e-09	0.9	0	0	\\
1.5e-09	0.9	-0.666666666666667	-0.666666666666667	\\
1.55e-09	0.9	-0.666666666666667	-0.666666666666667	\\
1.6e-09	0.9	-0.666666666666667	-0.666666666666667	\\
1.65e-09	0.9	-0.666666666666667	-0.666666666666667	\\
1.7e-09	0.9	-0.666666666666667	-0.666666666666667	\\
1.75e-09	0.9	-0.666666666666667	-0.666666666666667	\\
1.8e-09	0.9	-0.666666666666667	-0.666666666666667	\\
1.85e-09	0.9	-0.666666666666667	-0.666666666666667	\\
1.9e-09	0.9	-0.666666666666667	-0.666666666666667	\\
1.95e-09	0.9	-0.666666666666667	-0.666666666666667	\\
2e-09	0.9	0	0	\\
2.05e-09	0.9	0	0	\\
2.1e-09	0.9	0	0	\\
2.15e-09	0.9	0	0	\\
2.2e-09	0.9	0	0	\\
2.25e-09	0.9	0	0	\\
2.3e-09	0.9	0	0	\\
2.35e-09	0.9	0	0	\\
2.4e-09	0.9	0	0	\\
2.45e-09	0.9	0	0	\\
2.5e-09	0.9	0	0	\\
2.55e-09	0.9	0	0	\\
2.6e-09	0.9	0	0	\\
2.65e-09	0.9	0	0	\\
2.7e-09	0.9	0	0	\\
2.75e-09	0.9	0	0	\\
2.8e-09	0.9	0	0	\\
2.85e-09	0.9	0	0	\\
2.9e-09	0.9	0	0	\\
2.95e-09	0.9	0	0	\\
3e-09	0.9	0	0	\\
3.05e-09	0.9	0	0	\\
3.1e-09	0.9	0	0	\\
3.15e-09	0.9	0	0	\\
3.2e-09	0.9	0	0	\\
3.25e-09	0.9	0	0	\\
3.3e-09	0.9	0.666666666666667	0.666666666666667	\\
3.35e-09	0.9	0.666666666666667	0.666666666666667	\\
3.4e-09	0.9	0.666666666666667	0.666666666666667	\\
3.45e-09	0.9	0.666666666666667	0.666666666666667	\\
3.5e-09	0.9	0.666666666666667	0.666666666666667	\\
3.55e-09	0.9	0.666666666666667	0.666666666666667	\\
3.6e-09	0.9	0.666666666666667	0.666666666666667	\\
3.65e-09	0.9	0.666666666666667	0.666666666666667	\\
3.7e-09	0.9	0.666666666666667	0.666666666666667	\\
3.75e-09	0.9	0.666666666666667	0.666666666666667	\\
3.8e-09	0.9	0	0	\\
3.85e-09	0.9	0	0	\\
3.9e-09	0.9	-0.444444444444444	-0.444444444444444	\\
3.95e-09	0.9	-0.444444444444444	-0.444444444444444	\\
4e-09	0.9	-0.444444444444444	-0.444444444444444	\\
4.05e-09	0.9	-0.444444444444444	-0.444444444444444	\\
4.1e-09	0.9	-0.444444444444444	-0.444444444444444	\\
4.15e-09	0.9	-0.444444444444444	-0.444444444444444	\\
4.2e-09	0.9	-0.444444444444444	-0.444444444444444	\\
4.25e-09	0.9	-0.444444444444444	-0.444444444444444	\\
4.3e-09	0.9	-0.444444444444444	-0.444444444444444	\\
4.35e-09	0.9	-0.444444444444444	-0.444444444444444	\\
4.4e-09	0.9	0	0	\\
4.45e-09	0.9	0	0	\\
4.5e-09	0.9	0	0	\\
4.55e-09	0.9	0	0	\\
4.6e-09	0.9	0	0	\\
4.65e-09	0.9	0	0	\\
4.7e-09	0.9	0	0	\\
4.75e-09	0.9	0	0	\\
4.8e-09	0.9	0	0	\\
4.85e-09	0.9	0	0	\\
4.9e-09	0.9	0	0	\\
4.95e-09	0.9	0	0	\\
5e-09	0.9	0	0	\\
5e-09	0.9	0	nan	\\
5e-09	0.9	-0.666666666666667	0.166666666666667	\\
5e-09	0.9	-0.666666666666667	nan	\\
0	0.912	-0.666666666666667	nan	\\
0	0.912	0	0.166666666666667	\\
0	0.912	0	0	\\
5e-11	0.912	0	0	\\
1e-10	0.912	0	0	\\
1.5e-10	0.912	0	0	\\
2e-10	0.912	0	0	\\
2.5e-10	0.912	0	0	\\
3e-10	0.912	0	0	\\
3.5e-10	0.912	0	0	\\
4e-10	0.912	0	0	\\
4.5e-10	0.912	0	0	\\
5e-10	0.912	0	0	\\
5.5e-10	0.912	0	0	\\
6e-10	0.912	0	0	\\
6.5e-10	0.912	0	0	\\
7e-10	0.912	0	0	\\
7.5e-10	0.912	0	0	\\
8e-10	0.912	0	0	\\
8.5e-10	0.912	0	0	\\
9e-10	0.912	0	0	\\
9.5e-10	0.912	1	1	\\
1e-09	0.912	1	1	\\
1.05e-09	0.912	1	1	\\
1.1e-09	0.912	1	1	\\
1.15e-09	0.912	1	1	\\
1.2e-09	0.912	1	1	\\
1.25e-09	0.912	1	1	\\
1.3e-09	0.912	1	1	\\
1.35e-09	0.912	1	1	\\
1.4e-09	0.912	1	1	\\
1.45e-09	0.912	0	0	\\
1.5e-09	0.912	-0.666666666666667	-0.666666666666667	\\
1.55e-09	0.912	-0.666666666666667	-0.666666666666667	\\
1.6e-09	0.912	-0.666666666666667	-0.666666666666667	\\
1.65e-09	0.912	-0.666666666666667	-0.666666666666667	\\
1.7e-09	0.912	-0.666666666666667	-0.666666666666667	\\
1.75e-09	0.912	-0.666666666666667	-0.666666666666667	\\
1.8e-09	0.912	-0.666666666666667	-0.666666666666667	\\
1.85e-09	0.912	-0.666666666666667	-0.666666666666667	\\
1.9e-09	0.912	-0.666666666666667	-0.666666666666667	\\
1.95e-09	0.912	-0.666666666666667	-0.666666666666667	\\
2e-09	0.912	0	0	\\
2.05e-09	0.912	0	0	\\
2.1e-09	0.912	0	0	\\
2.15e-09	0.912	0	0	\\
2.2e-09	0.912	0	0	\\
2.25e-09	0.912	0	0	\\
2.3e-09	0.912	0	0	\\
2.35e-09	0.912	0	0	\\
2.4e-09	0.912	0	0	\\
2.45e-09	0.912	0	0	\\
2.5e-09	0.912	0	0	\\
2.55e-09	0.912	0	0	\\
2.6e-09	0.912	0	0	\\
2.65e-09	0.912	0	0	\\
2.7e-09	0.912	0	0	\\
2.75e-09	0.912	0	0	\\
2.8e-09	0.912	0	0	\\
2.85e-09	0.912	0	0	\\
2.9e-09	0.912	0	0	\\
2.95e-09	0.912	0	0	\\
3e-09	0.912	0	0	\\
3.05e-09	0.912	0	0	\\
3.1e-09	0.912	0	0	\\
3.15e-09	0.912	0	0	\\
3.2e-09	0.912	0	0	\\
3.25e-09	0.912	0	0	\\
3.3e-09	0.912	0	0	\\
3.35e-09	0.912	0.666666666666667	0.666666666666667	\\
3.4e-09	0.912	0.666666666666667	0.666666666666667	\\
3.45e-09	0.912	0.666666666666667	0.666666666666667	\\
3.5e-09	0.912	0.666666666666667	0.666666666666667	\\
3.55e-09	0.912	0.666666666666667	0.666666666666667	\\
3.6e-09	0.912	0.666666666666667	0.666666666666667	\\
3.65e-09	0.912	0.666666666666667	0.666666666666667	\\
3.7e-09	0.912	0.666666666666667	0.666666666666667	\\
3.75e-09	0.912	0.666666666666667	0.666666666666667	\\
3.8e-09	0.912	0.666666666666667	0.666666666666667	\\
3.85e-09	0.912	0	0	\\
3.9e-09	0.912	-0.444444444444444	-0.444444444444444	\\
3.95e-09	0.912	-0.444444444444444	-0.444444444444444	\\
4e-09	0.912	-0.444444444444444	-0.444444444444444	\\
4.05e-09	0.912	-0.444444444444444	-0.444444444444444	\\
4.1e-09	0.912	-0.444444444444444	-0.444444444444444	\\
4.15e-09	0.912	-0.444444444444444	-0.444444444444444	\\
4.2e-09	0.912	-0.444444444444444	-0.444444444444444	\\
4.25e-09	0.912	-0.444444444444444	-0.444444444444444	\\
4.3e-09	0.912	-0.444444444444444	-0.444444444444444	\\
4.35e-09	0.912	-0.444444444444444	-0.444444444444444	\\
4.4e-09	0.912	0	0	\\
4.45e-09	0.912	0	0	\\
4.5e-09	0.912	0	0	\\
4.55e-09	0.912	0	0	\\
4.6e-09	0.912	0	0	\\
4.65e-09	0.912	0	0	\\
4.7e-09	0.912	0	0	\\
4.75e-09	0.912	0	0	\\
4.8e-09	0.912	0	0	\\
4.85e-09	0.912	0	0	\\
4.9e-09	0.912	0	0	\\
4.95e-09	0.912	0	0	\\
5e-09	0.912	0	0	\\
5e-09	0.912	0	nan	\\
5e-09	0.912	-0.666666666666667	0.166666666666667	\\
5e-09	0.912	-0.666666666666667	nan	\\
0	0.924	-0.666666666666667	nan	\\
0	0.924	0	0.166666666666667	\\
0	0.924	0	0	\\
5e-11	0.924	0	0	\\
1e-10	0.924	0	0	\\
1.5e-10	0.924	0	0	\\
2e-10	0.924	0	0	\\
2.5e-10	0.924	0	0	\\
3e-10	0.924	0	0	\\
3.5e-10	0.924	0	0	\\
4e-10	0.924	0	0	\\
4.5e-10	0.924	0	0	\\
5e-10	0.924	0	0	\\
5.5e-10	0.924	0	0	\\
6e-10	0.924	0	0	\\
6.5e-10	0.924	0	0	\\
7e-10	0.924	0	0	\\
7.5e-10	0.924	0	0	\\
8e-10	0.924	0	0	\\
8.5e-10	0.924	0	0	\\
9e-10	0.924	0	0	\\
9.5e-10	0.924	1	1	\\
1e-09	0.924	1	1	\\
1.05e-09	0.924	1	1	\\
1.1e-09	0.924	1	1	\\
1.15e-09	0.924	1	1	\\
1.2e-09	0.924	1	1	\\
1.25e-09	0.924	1	1	\\
1.3e-09	0.924	1	1	\\
1.35e-09	0.924	1	1	\\
1.4e-09	0.924	1	1	\\
1.45e-09	0.924	0	0	\\
1.5e-09	0.924	-0.666666666666667	-0.666666666666667	\\
1.55e-09	0.924	-0.666666666666667	-0.666666666666667	\\
1.6e-09	0.924	-0.666666666666667	-0.666666666666667	\\
1.65e-09	0.924	-0.666666666666667	-0.666666666666667	\\
1.7e-09	0.924	-0.666666666666667	-0.666666666666667	\\
1.75e-09	0.924	-0.666666666666667	-0.666666666666667	\\
1.8e-09	0.924	-0.666666666666667	-0.666666666666667	\\
1.85e-09	0.924	-0.666666666666667	-0.666666666666667	\\
1.9e-09	0.924	-0.666666666666667	-0.666666666666667	\\
1.95e-09	0.924	-0.666666666666667	-0.666666666666667	\\
2e-09	0.924	0	0	\\
2.05e-09	0.924	0	0	\\
2.1e-09	0.924	0	0	\\
2.15e-09	0.924	0	0	\\
2.2e-09	0.924	0	0	\\
2.25e-09	0.924	0	0	\\
2.3e-09	0.924	0	0	\\
2.35e-09	0.924	0	0	\\
2.4e-09	0.924	0	0	\\
2.45e-09	0.924	0	0	\\
2.5e-09	0.924	0	0	\\
2.55e-09	0.924	0	0	\\
2.6e-09	0.924	0	0	\\
2.65e-09	0.924	0	0	\\
2.7e-09	0.924	0	0	\\
2.75e-09	0.924	0	0	\\
2.8e-09	0.924	0	0	\\
2.85e-09	0.924	0	0	\\
2.9e-09	0.924	0	0	\\
2.95e-09	0.924	0	0	\\
3e-09	0.924	0	0	\\
3.05e-09	0.924	0	0	\\
3.1e-09	0.924	0	0	\\
3.15e-09	0.924	0	0	\\
3.2e-09	0.924	0	0	\\
3.25e-09	0.924	0	0	\\
3.3e-09	0.924	0	0	\\
3.35e-09	0.924	0.666666666666667	0.666666666666667	\\
3.4e-09	0.924	0.666666666666667	0.666666666666667	\\
3.45e-09	0.924	0.666666666666667	0.666666666666667	\\
3.5e-09	0.924	0.666666666666667	0.666666666666667	\\
3.55e-09	0.924	0.666666666666667	0.666666666666667	\\
3.6e-09	0.924	0.666666666666667	0.666666666666667	\\
3.65e-09	0.924	0.666666666666667	0.666666666666667	\\
3.7e-09	0.924	0.666666666666667	0.666666666666667	\\
3.75e-09	0.924	0.666666666666667	0.666666666666667	\\
3.8e-09	0.924	0.666666666666667	0.666666666666667	\\
3.85e-09	0.924	0	0	\\
3.9e-09	0.924	-0.444444444444444	-0.444444444444444	\\
3.95e-09	0.924	-0.444444444444444	-0.444444444444444	\\
4e-09	0.924	-0.444444444444444	-0.444444444444444	\\
4.05e-09	0.924	-0.444444444444444	-0.444444444444444	\\
4.1e-09	0.924	-0.444444444444444	-0.444444444444444	\\
4.15e-09	0.924	-0.444444444444444	-0.444444444444444	\\
4.2e-09	0.924	-0.444444444444444	-0.444444444444444	\\
4.25e-09	0.924	-0.444444444444444	-0.444444444444444	\\
4.3e-09	0.924	-0.444444444444444	-0.444444444444444	\\
4.35e-09	0.924	-0.444444444444444	-0.444444444444444	\\
4.4e-09	0.924	0	0	\\
4.45e-09	0.924	0	0	\\
4.5e-09	0.924	0	0	\\
4.55e-09	0.924	0	0	\\
4.6e-09	0.924	0	0	\\
4.65e-09	0.924	0	0	\\
4.7e-09	0.924	0	0	\\
4.75e-09	0.924	0	0	\\
4.8e-09	0.924	0	0	\\
4.85e-09	0.924	0	0	\\
4.9e-09	0.924	0	0	\\
4.95e-09	0.924	0	0	\\
5e-09	0.924	0	0	\\
5e-09	0.924	0	nan	\\
5e-09	0.924	-0.666666666666667	0.166666666666667	\\
5e-09	0.924	-0.666666666666667	nan	\\
0	0.936	-0.666666666666667	nan	\\
0	0.936	0	0.166666666666667	\\
0	0.936	0	0	\\
5e-11	0.936	0	0	\\
1e-10	0.936	0	0	\\
1.5e-10	0.936	0	0	\\
2e-10	0.936	0	0	\\
2.5e-10	0.936	0	0	\\
3e-10	0.936	0	0	\\
3.5e-10	0.936	0	0	\\
4e-10	0.936	0	0	\\
4.5e-10	0.936	0	0	\\
5e-10	0.936	0	0	\\
5.5e-10	0.936	0	0	\\
6e-10	0.936	0	0	\\
6.5e-10	0.936	0	0	\\
7e-10	0.936	0	0	\\
7.5e-10	0.936	0	0	\\
8e-10	0.936	0	0	\\
8.5e-10	0.936	0	0	\\
9e-10	0.936	0	0	\\
9.5e-10	0.936	1	1	\\
1e-09	0.936	1	1	\\
1.05e-09	0.936	1	1	\\
1.1e-09	0.936	1	1	\\
1.15e-09	0.936	1	1	\\
1.2e-09	0.936	1	1	\\
1.25e-09	0.936	1	1	\\
1.3e-09	0.936	1	1	\\
1.35e-09	0.936	1	1	\\
1.4e-09	0.936	1	1	\\
1.45e-09	0.936	0	0	\\
1.5e-09	0.936	-0.666666666666667	-0.666666666666667	\\
1.55e-09	0.936	-0.666666666666667	-0.666666666666667	\\
1.6e-09	0.936	-0.666666666666667	-0.666666666666667	\\
1.65e-09	0.936	-0.666666666666667	-0.666666666666667	\\
1.7e-09	0.936	-0.666666666666667	-0.666666666666667	\\
1.75e-09	0.936	-0.666666666666667	-0.666666666666667	\\
1.8e-09	0.936	-0.666666666666667	-0.666666666666667	\\
1.85e-09	0.936	-0.666666666666667	-0.666666666666667	\\
1.9e-09	0.936	-0.666666666666667	-0.666666666666667	\\
1.95e-09	0.936	-0.666666666666667	-0.666666666666667	\\
2e-09	0.936	0	0	\\
2.05e-09	0.936	0	0	\\
2.1e-09	0.936	0	0	\\
2.15e-09	0.936	0	0	\\
2.2e-09	0.936	0	0	\\
2.25e-09	0.936	0	0	\\
2.3e-09	0.936	0	0	\\
2.35e-09	0.936	0	0	\\
2.4e-09	0.936	0	0	\\
2.45e-09	0.936	0	0	\\
2.5e-09	0.936	0	0	\\
2.55e-09	0.936	0	0	\\
2.6e-09	0.936	0	0	\\
2.65e-09	0.936	0	0	\\
2.7e-09	0.936	0	0	\\
2.75e-09	0.936	0	0	\\
2.8e-09	0.936	0	0	\\
2.85e-09	0.936	0	0	\\
2.9e-09	0.936	0	0	\\
2.95e-09	0.936	0	0	\\
3e-09	0.936	0	0	\\
3.05e-09	0.936	0	0	\\
3.1e-09	0.936	0	0	\\
3.15e-09	0.936	0	0	\\
3.2e-09	0.936	0	0	\\
3.25e-09	0.936	0	0	\\
3.3e-09	0.936	0	0	\\
3.35e-09	0.936	0.666666666666667	0.666666666666667	\\
3.4e-09	0.936	0.666666666666667	0.666666666666667	\\
3.45e-09	0.936	0.666666666666667	0.666666666666667	\\
3.5e-09	0.936	0.666666666666667	0.666666666666667	\\
3.55e-09	0.936	0.666666666666667	0.666666666666667	\\
3.6e-09	0.936	0.666666666666667	0.666666666666667	\\
3.65e-09	0.936	0.666666666666667	0.666666666666667	\\
3.7e-09	0.936	0.666666666666667	0.666666666666667	\\
3.75e-09	0.936	0.666666666666667	0.666666666666667	\\
3.8e-09	0.936	0.666666666666667	0.666666666666667	\\
3.85e-09	0.936	0	0	\\
3.9e-09	0.936	-0.444444444444444	-0.444444444444444	\\
3.95e-09	0.936	-0.444444444444444	-0.444444444444444	\\
4e-09	0.936	-0.444444444444444	-0.444444444444444	\\
4.05e-09	0.936	-0.444444444444444	-0.444444444444444	\\
4.1e-09	0.936	-0.444444444444444	-0.444444444444444	\\
4.15e-09	0.936	-0.444444444444444	-0.444444444444444	\\
4.2e-09	0.936	-0.444444444444444	-0.444444444444444	\\
4.25e-09	0.936	-0.444444444444444	-0.444444444444444	\\
4.3e-09	0.936	-0.444444444444444	-0.444444444444444	\\
4.35e-09	0.936	-0.444444444444444	-0.444444444444444	\\
4.4e-09	0.936	0	0	\\
4.45e-09	0.936	0	0	\\
4.5e-09	0.936	0	0	\\
4.55e-09	0.936	0	0	\\
4.6e-09	0.936	0	0	\\
4.65e-09	0.936	0	0	\\
4.7e-09	0.936	0	0	\\
4.75e-09	0.936	0	0	\\
4.8e-09	0.936	0	0	\\
4.85e-09	0.936	0	0	\\
4.9e-09	0.936	0	0	\\
4.95e-09	0.936	0	0	\\
5e-09	0.936	0	0	\\
5e-09	0.936	0	nan	\\
5e-09	0.936	-0.666666666666667	0.166666666666667	\\
5e-09	0.936	-0.666666666666667	nan	\\
0	0.948	-0.666666666666667	nan	\\
0	0.948	0	0.166666666666667	\\
0	0.948	0	0	\\
5e-11	0.948	0	0	\\
1e-10	0.948	0	0	\\
1.5e-10	0.948	0	0	\\
2e-10	0.948	0	0	\\
2.5e-10	0.948	0	0	\\
3e-10	0.948	0	0	\\
3.5e-10	0.948	0	0	\\
4e-10	0.948	0	0	\\
4.5e-10	0.948	0	0	\\
5e-10	0.948	0	0	\\
5.5e-10	0.948	0	0	\\
6e-10	0.948	0	0	\\
6.5e-10	0.948	0	0	\\
7e-10	0.948	0	0	\\
7.5e-10	0.948	0	0	\\
8e-10	0.948	0	0	\\
8.5e-10	0.948	0	0	\\
9e-10	0.948	0	0	\\
9.5e-10	0.948	1	1	\\
1e-09	0.948	1	1	\\
1.05e-09	0.948	1	1	\\
1.1e-09	0.948	1	1	\\
1.15e-09	0.948	1	1	\\
1.2e-09	0.948	1	1	\\
1.25e-09	0.948	1	1	\\
1.3e-09	0.948	1	1	\\
1.35e-09	0.948	1	1	\\
1.4e-09	0.948	1	1	\\
1.45e-09	0.948	0	0	\\
1.5e-09	0.948	-0.666666666666667	-0.666666666666667	\\
1.55e-09	0.948	-0.666666666666667	-0.666666666666667	\\
1.6e-09	0.948	-0.666666666666667	-0.666666666666667	\\
1.65e-09	0.948	-0.666666666666667	-0.666666666666667	\\
1.7e-09	0.948	-0.666666666666667	-0.666666666666667	\\
1.75e-09	0.948	-0.666666666666667	-0.666666666666667	\\
1.8e-09	0.948	-0.666666666666667	-0.666666666666667	\\
1.85e-09	0.948	-0.666666666666667	-0.666666666666667	\\
1.9e-09	0.948	-0.666666666666667	-0.666666666666667	\\
1.95e-09	0.948	-0.666666666666667	-0.666666666666667	\\
2e-09	0.948	0	0	\\
2.05e-09	0.948	0	0	\\
2.1e-09	0.948	0	0	\\
2.15e-09	0.948	0	0	\\
2.2e-09	0.948	0	0	\\
2.25e-09	0.948	0	0	\\
2.3e-09	0.948	0	0	\\
2.35e-09	0.948	0	0	\\
2.4e-09	0.948	0	0	\\
2.45e-09	0.948	0	0	\\
2.5e-09	0.948	0	0	\\
2.55e-09	0.948	0	0	\\
2.6e-09	0.948	0	0	\\
2.65e-09	0.948	0	0	\\
2.7e-09	0.948	0	0	\\
2.75e-09	0.948	0	0	\\
2.8e-09	0.948	0	0	\\
2.85e-09	0.948	0	0	\\
2.9e-09	0.948	0	0	\\
2.95e-09	0.948	0	0	\\
3e-09	0.948	0	0	\\
3.05e-09	0.948	0	0	\\
3.1e-09	0.948	0	0	\\
3.15e-09	0.948	0	0	\\
3.2e-09	0.948	0	0	\\
3.25e-09	0.948	0	0	\\
3.3e-09	0.948	0	0	\\
3.35e-09	0.948	0.666666666666667	0.666666666666667	\\
3.4e-09	0.948	0.666666666666667	0.666666666666667	\\
3.45e-09	0.948	0.666666666666667	0.666666666666667	\\
3.5e-09	0.948	0.666666666666667	0.666666666666667	\\
3.55e-09	0.948	0.666666666666667	0.666666666666667	\\
3.6e-09	0.948	0.666666666666667	0.666666666666667	\\
3.65e-09	0.948	0.666666666666667	0.666666666666667	\\
3.7e-09	0.948	0.666666666666667	0.666666666666667	\\
3.75e-09	0.948	0.666666666666667	0.666666666666667	\\
3.8e-09	0.948	0.666666666666667	0.666666666666667	\\
3.85e-09	0.948	0	0	\\
3.9e-09	0.948	-0.444444444444444	-0.444444444444444	\\
3.95e-09	0.948	-0.444444444444444	-0.444444444444444	\\
4e-09	0.948	-0.444444444444444	-0.444444444444444	\\
4.05e-09	0.948	-0.444444444444444	-0.444444444444444	\\
4.1e-09	0.948	-0.444444444444444	-0.444444444444444	\\
4.15e-09	0.948	-0.444444444444444	-0.444444444444444	\\
4.2e-09	0.948	-0.444444444444444	-0.444444444444444	\\
4.25e-09	0.948	-0.444444444444444	-0.444444444444444	\\
4.3e-09	0.948	-0.444444444444444	-0.444444444444444	\\
4.35e-09	0.948	-0.444444444444444	-0.444444444444444	\\
4.4e-09	0.948	0	0	\\
4.45e-09	0.948	0	0	\\
4.5e-09	0.948	0	0	\\
4.55e-09	0.948	0	0	\\
4.6e-09	0.948	0	0	\\
4.65e-09	0.948	0	0	\\
4.7e-09	0.948	0	0	\\
4.75e-09	0.948	0	0	\\
4.8e-09	0.948	0	0	\\
4.85e-09	0.948	0	0	\\
4.9e-09	0.948	0	0	\\
4.95e-09	0.948	0	0	\\
5e-09	0.948	0	0	\\
5e-09	0.948	0	nan	\\
5e-09	0.948	-0.666666666666667	0.166666666666667	\\
5e-09	0.948	-0.666666666666667	nan	\\
0	0.96	-0.666666666666667	nan	\\
0	0.96	0	0.166666666666667	\\
0	0.96	0	0	\\
5e-11	0.96	0	0	\\
1e-10	0.96	0	0	\\
1.5e-10	0.96	0	0	\\
2e-10	0.96	0	0	\\
2.5e-10	0.96	0	0	\\
3e-10	0.96	0	0	\\
3.5e-10	0.96	0	0	\\
4e-10	0.96	0	0	\\
4.5e-10	0.96	0	0	\\
5e-10	0.96	0	0	\\
5.5e-10	0.96	0	0	\\
6e-10	0.96	0	0	\\
6.5e-10	0.96	0	0	\\
7e-10	0.96	0	0	\\
7.5e-10	0.96	0	0	\\
8e-10	0.96	0	0	\\
8.5e-10	0.96	0	0	\\
9e-10	0.96	0	0	\\
9.5e-10	0.96	0	0	\\
1e-09	0.96	1	1	\\
1.05e-09	0.96	1	1	\\
1.1e-09	0.96	1	1	\\
1.15e-09	0.96	1	1	\\
1.2e-09	0.96	1	1	\\
1.25e-09	0.96	1	1	\\
1.3e-09	0.96	1	1	\\
1.35e-09	0.96	1	1	\\
1.4e-09	0.96	1	1	\\
1.45e-09	0.96	0.333333333333333	0.333333333333333	\\
1.5e-09	0.96	-0.666666666666667	-0.666666666666667	\\
1.55e-09	0.96	-0.666666666666667	-0.666666666666667	\\
1.6e-09	0.96	-0.666666666666667	-0.666666666666667	\\
1.65e-09	0.96	-0.666666666666667	-0.666666666666667	\\
1.7e-09	0.96	-0.666666666666667	-0.666666666666667	\\
1.75e-09	0.96	-0.666666666666667	-0.666666666666667	\\
1.8e-09	0.96	-0.666666666666667	-0.666666666666667	\\
1.85e-09	0.96	-0.666666666666667	-0.666666666666667	\\
1.9e-09	0.96	-0.666666666666667	-0.666666666666667	\\
1.95e-09	0.96	0	0	\\
2e-09	0.96	0	0	\\
2.05e-09	0.96	0	0	\\
2.1e-09	0.96	0	0	\\
2.15e-09	0.96	0	0	\\
2.2e-09	0.96	0	0	\\
2.25e-09	0.96	0	0	\\
2.3e-09	0.96	0	0	\\
2.35e-09	0.96	0	0	\\
2.4e-09	0.96	0	0	\\
2.45e-09	0.96	0	0	\\
2.5e-09	0.96	0	0	\\
2.55e-09	0.96	0	0	\\
2.6e-09	0.96	0	0	\\
2.65e-09	0.96	0	0	\\
2.7e-09	0.96	0	0	\\
2.75e-09	0.96	0	0	\\
2.8e-09	0.96	0	0	\\
2.85e-09	0.96	0	0	\\
2.9e-09	0.96	0	0	\\
2.95e-09	0.96	0	0	\\
3e-09	0.96	0	0	\\
3.05e-09	0.96	0	0	\\
3.1e-09	0.96	0	0	\\
3.15e-09	0.96	0	0	\\
3.2e-09	0.96	0	0	\\
3.25e-09	0.96	0	0	\\
3.3e-09	0.96	0	0	\\
3.35e-09	0.96	0	0	\\
3.4e-09	0.96	0.666666666666667	0.666666666666667	\\
3.45e-09	0.96	0.666666666666667	0.666666666666667	\\
3.5e-09	0.96	0.666666666666667	0.666666666666667	\\
3.55e-09	0.96	0.666666666666667	0.666666666666667	\\
3.6e-09	0.96	0.666666666666667	0.666666666666667	\\
3.65e-09	0.96	0.666666666666667	0.666666666666667	\\
3.7e-09	0.96	0.666666666666667	0.666666666666667	\\
3.75e-09	0.96	0.666666666666667	0.666666666666667	\\
3.8e-09	0.96	0.666666666666667	0.666666666666667	\\
3.85e-09	0.96	0.222222222222222	0.222222222222222	\\
3.9e-09	0.96	-0.444444444444444	-0.444444444444444	\\
3.95e-09	0.96	-0.444444444444444	-0.444444444444444	\\
4e-09	0.96	-0.444444444444444	-0.444444444444444	\\
4.05e-09	0.96	-0.444444444444444	-0.444444444444444	\\
4.1e-09	0.96	-0.444444444444444	-0.444444444444444	\\
4.15e-09	0.96	-0.444444444444444	-0.444444444444444	\\
4.2e-09	0.96	-0.444444444444444	-0.444444444444444	\\
4.25e-09	0.96	-0.444444444444444	-0.444444444444444	\\
4.3e-09	0.96	-0.444444444444444	-0.444444444444444	\\
4.35e-09	0.96	0	0	\\
4.4e-09	0.96	0	0	\\
4.45e-09	0.96	0	0	\\
4.5e-09	0.96	0	0	\\
4.55e-09	0.96	0	0	\\
4.6e-09	0.96	0	0	\\
4.65e-09	0.96	0	0	\\
4.7e-09	0.96	0	0	\\
4.75e-09	0.96	0	0	\\
4.8e-09	0.96	0	0	\\
4.85e-09	0.96	0	0	\\
4.9e-09	0.96	0	0	\\
4.95e-09	0.96	0	0	\\
5e-09	0.96	0	0	\\
5e-09	0.96	0	nan	\\
5e-09	0.96	-0.666666666666667	0.166666666666667	\\
5e-09	0.96	-0.666666666666667	nan	\\
0	0.972	-0.666666666666667	nan	\\
0	0.972	0	0.166666666666667	\\
0	0.972	0	0	\\
5e-11	0.972	0	0	\\
1e-10	0.972	0	0	\\
1.5e-10	0.972	0	0	\\
2e-10	0.972	0	0	\\
2.5e-10	0.972	0	0	\\
3e-10	0.972	0	0	\\
3.5e-10	0.972	0	0	\\
4e-10	0.972	0	0	\\
4.5e-10	0.972	0	0	\\
5e-10	0.972	0	0	\\
5.5e-10	0.972	0	0	\\
6e-10	0.972	0	0	\\
6.5e-10	0.972	0	0	\\
7e-10	0.972	0	0	\\
7.5e-10	0.972	0	0	\\
8e-10	0.972	0	0	\\
8.5e-10	0.972	0	0	\\
9e-10	0.972	0	0	\\
9.5e-10	0.972	0	0	\\
1e-09	0.972	1	1	\\
1.05e-09	0.972	1	1	\\
1.1e-09	0.972	1	1	\\
1.15e-09	0.972	1	1	\\
1.2e-09	0.972	1	1	\\
1.25e-09	0.972	1	1	\\
1.3e-09	0.972	1	1	\\
1.35e-09	0.972	1	1	\\
1.4e-09	0.972	1	1	\\
1.45e-09	0.972	0.333333333333333	0.333333333333333	\\
1.5e-09	0.972	-0.666666666666667	-0.666666666666667	\\
1.55e-09	0.972	-0.666666666666667	-0.666666666666667	\\
1.6e-09	0.972	-0.666666666666667	-0.666666666666667	\\
1.65e-09	0.972	-0.666666666666667	-0.666666666666667	\\
1.7e-09	0.972	-0.666666666666667	-0.666666666666667	\\
1.75e-09	0.972	-0.666666666666667	-0.666666666666667	\\
1.8e-09	0.972	-0.666666666666667	-0.666666666666667	\\
1.85e-09	0.972	-0.666666666666667	-0.666666666666667	\\
1.9e-09	0.972	-0.666666666666667	-0.666666666666667	\\
1.95e-09	0.972	0	0	\\
2e-09	0.972	0	0	\\
2.05e-09	0.972	0	0	\\
2.1e-09	0.972	0	0	\\
2.15e-09	0.972	0	0	\\
2.2e-09	0.972	0	0	\\
2.25e-09	0.972	0	0	\\
2.3e-09	0.972	0	0	\\
2.35e-09	0.972	0	0	\\
2.4e-09	0.972	0	0	\\
2.45e-09	0.972	0	0	\\
2.5e-09	0.972	0	0	\\
2.55e-09	0.972	0	0	\\
2.6e-09	0.972	0	0	\\
2.65e-09	0.972	0	0	\\
2.7e-09	0.972	0	0	\\
2.75e-09	0.972	0	0	\\
2.8e-09	0.972	0	0	\\
2.85e-09	0.972	0	0	\\
2.9e-09	0.972	0	0	\\
2.95e-09	0.972	0	0	\\
3e-09	0.972	0	0	\\
3.05e-09	0.972	0	0	\\
3.1e-09	0.972	0	0	\\
3.15e-09	0.972	0	0	\\
3.2e-09	0.972	0	0	\\
3.25e-09	0.972	0	0	\\
3.3e-09	0.972	0	0	\\
3.35e-09	0.972	0	0	\\
3.4e-09	0.972	0.666666666666667	0.666666666666667	\\
3.45e-09	0.972	0.666666666666667	0.666666666666667	\\
3.5e-09	0.972	0.666666666666667	0.666666666666667	\\
3.55e-09	0.972	0.666666666666667	0.666666666666667	\\
3.6e-09	0.972	0.666666666666667	0.666666666666667	\\
3.65e-09	0.972	0.666666666666667	0.666666666666667	\\
3.7e-09	0.972	0.666666666666667	0.666666666666667	\\
3.75e-09	0.972	0.666666666666667	0.666666666666667	\\
3.8e-09	0.972	0.666666666666667	0.666666666666667	\\
3.85e-09	0.972	0.222222222222222	0.222222222222222	\\
3.9e-09	0.972	-0.444444444444444	-0.444444444444444	\\
3.95e-09	0.972	-0.444444444444444	-0.444444444444444	\\
4e-09	0.972	-0.444444444444444	-0.444444444444444	\\
4.05e-09	0.972	-0.444444444444444	-0.444444444444444	\\
4.1e-09	0.972	-0.444444444444444	-0.444444444444444	\\
4.15e-09	0.972	-0.444444444444444	-0.444444444444444	\\
4.2e-09	0.972	-0.444444444444444	-0.444444444444444	\\
4.25e-09	0.972	-0.444444444444444	-0.444444444444444	\\
4.3e-09	0.972	-0.444444444444444	-0.444444444444444	\\
4.35e-09	0.972	0	0	\\
4.4e-09	0.972	0	0	\\
4.45e-09	0.972	0	0	\\
4.5e-09	0.972	0	0	\\
4.55e-09	0.972	0	0	\\
4.6e-09	0.972	0	0	\\
4.65e-09	0.972	0	0	\\
4.7e-09	0.972	0	0	\\
4.75e-09	0.972	0	0	\\
4.8e-09	0.972	0	0	\\
4.85e-09	0.972	0	0	\\
4.9e-09	0.972	0	0	\\
4.95e-09	0.972	0	0	\\
5e-09	0.972	0	0	\\
5e-09	0.972	0	nan	\\
5e-09	0.972	-0.666666666666667	0.166666666666667	\\
5e-09	0.972	-0.666666666666667	nan	\\
0	0.984	-0.666666666666667	nan	\\
0	0.984	0	0.166666666666667	\\
0	0.984	0	0	\\
5e-11	0.984	0	0	\\
1e-10	0.984	0	0	\\
1.5e-10	0.984	0	0	\\
2e-10	0.984	0	0	\\
2.5e-10	0.984	0	0	\\
3e-10	0.984	0	0	\\
3.5e-10	0.984	0	0	\\
4e-10	0.984	0	0	\\
4.5e-10	0.984	0	0	\\
5e-10	0.984	0	0	\\
5.5e-10	0.984	0	0	\\
6e-10	0.984	0	0	\\
6.5e-10	0.984	0	0	\\
7e-10	0.984	0	0	\\
7.5e-10	0.984	0	0	\\
8e-10	0.984	0	0	\\
8.5e-10	0.984	0	0	\\
9e-10	0.984	0	0	\\
9.5e-10	0.984	0	0	\\
1e-09	0.984	1	1	\\
1.05e-09	0.984	1	1	\\
1.1e-09	0.984	1	1	\\
1.15e-09	0.984	1	1	\\
1.2e-09	0.984	1	1	\\
1.25e-09	0.984	1	1	\\
1.3e-09	0.984	1	1	\\
1.35e-09	0.984	1	1	\\
1.4e-09	0.984	1	1	\\
1.45e-09	0.984	0.333333333333333	0.333333333333333	\\
1.5e-09	0.984	-0.666666666666667	-0.666666666666667	\\
1.55e-09	0.984	-0.666666666666667	-0.666666666666667	\\
1.6e-09	0.984	-0.666666666666667	-0.666666666666667	\\
1.65e-09	0.984	-0.666666666666667	-0.666666666666667	\\
1.7e-09	0.984	-0.666666666666667	-0.666666666666667	\\
1.75e-09	0.984	-0.666666666666667	-0.666666666666667	\\
1.8e-09	0.984	-0.666666666666667	-0.666666666666667	\\
1.85e-09	0.984	-0.666666666666667	-0.666666666666667	\\
1.9e-09	0.984	-0.666666666666667	-0.666666666666667	\\
1.95e-09	0.984	0	0	\\
2e-09	0.984	0	0	\\
2.05e-09	0.984	0	0	\\
2.1e-09	0.984	0	0	\\
2.15e-09	0.984	0	0	\\
2.2e-09	0.984	0	0	\\
2.25e-09	0.984	0	0	\\
2.3e-09	0.984	0	0	\\
2.35e-09	0.984	0	0	\\
2.4e-09	0.984	0	0	\\
2.45e-09	0.984	0	0	\\
2.5e-09	0.984	0	0	\\
2.55e-09	0.984	0	0	\\
2.6e-09	0.984	0	0	\\
2.65e-09	0.984	0	0	\\
2.7e-09	0.984	0	0	\\
2.75e-09	0.984	0	0	\\
2.8e-09	0.984	0	0	\\
2.85e-09	0.984	0	0	\\
2.9e-09	0.984	0	0	\\
2.95e-09	0.984	0	0	\\
3e-09	0.984	0	0	\\
3.05e-09	0.984	0	0	\\
3.1e-09	0.984	0	0	\\
3.15e-09	0.984	0	0	\\
3.2e-09	0.984	0	0	\\
3.25e-09	0.984	0	0	\\
3.3e-09	0.984	0	0	\\
3.35e-09	0.984	0	0	\\
3.4e-09	0.984	0.666666666666667	0.666666666666667	\\
3.45e-09	0.984	0.666666666666667	0.666666666666667	\\
3.5e-09	0.984	0.666666666666667	0.666666666666667	\\
3.55e-09	0.984	0.666666666666667	0.666666666666667	\\
3.6e-09	0.984	0.666666666666667	0.666666666666667	\\
3.65e-09	0.984	0.666666666666667	0.666666666666667	\\
3.7e-09	0.984	0.666666666666667	0.666666666666667	\\
3.75e-09	0.984	0.666666666666667	0.666666666666667	\\
3.8e-09	0.984	0.666666666666667	0.666666666666667	\\
3.85e-09	0.984	0.222222222222222	0.222222222222222	\\
3.9e-09	0.984	-0.444444444444444	-0.444444444444444	\\
3.95e-09	0.984	-0.444444444444444	-0.444444444444444	\\
4e-09	0.984	-0.444444444444444	-0.444444444444444	\\
4.05e-09	0.984	-0.444444444444444	-0.444444444444444	\\
4.1e-09	0.984	-0.444444444444444	-0.444444444444444	\\
4.15e-09	0.984	-0.444444444444444	-0.444444444444444	\\
4.2e-09	0.984	-0.444444444444444	-0.444444444444444	\\
4.25e-09	0.984	-0.444444444444444	-0.444444444444444	\\
4.3e-09	0.984	-0.444444444444444	-0.444444444444444	\\
4.35e-09	0.984	0	0	\\
4.4e-09	0.984	0	0	\\
4.45e-09	0.984	0	0	\\
4.5e-09	0.984	0	0	\\
4.55e-09	0.984	0	0	\\
4.6e-09	0.984	0	0	\\
4.65e-09	0.984	0	0	\\
4.7e-09	0.984	0	0	\\
4.75e-09	0.984	0	0	\\
4.8e-09	0.984	0	0	\\
4.85e-09	0.984	0	0	\\
4.9e-09	0.984	0	0	\\
4.95e-09	0.984	0	0	\\
5e-09	0.984	0	0	\\
5e-09	0.984	0	nan	\\
5e-09	0.984	-0.666666666666667	0.166666666666667	\\
5e-09	0.984	-0.666666666666667	nan	\\
0	0.996	-0.666666666666667	nan	\\
0	0.996	0	0.166666666666667	\\
0	0.996	0	0	\\
5e-11	0.996	0	0	\\
1e-10	0.996	0	0	\\
1.5e-10	0.996	0	0	\\
2e-10	0.996	0	0	\\
2.5e-10	0.996	0	0	\\
3e-10	0.996	0	0	\\
3.5e-10	0.996	0	0	\\
4e-10	0.996	0	0	\\
4.5e-10	0.996	0	0	\\
5e-10	0.996	0	0	\\
5.5e-10	0.996	0	0	\\
6e-10	0.996	0	0	\\
6.5e-10	0.996	0	0	\\
7e-10	0.996	0	0	\\
7.5e-10	0.996	0	0	\\
8e-10	0.996	0	0	\\
8.5e-10	0.996	0	0	\\
9e-10	0.996	0	0	\\
9.5e-10	0.996	0	0	\\
1e-09	0.996	1	1	\\
1.05e-09	0.996	1	1	\\
1.1e-09	0.996	1	1	\\
1.15e-09	0.996	1	1	\\
1.2e-09	0.996	1	1	\\
1.25e-09	0.996	1	1	\\
1.3e-09	0.996	1	1	\\
1.35e-09	0.996	1	1	\\
1.4e-09	0.996	1	1	\\
1.45e-09	0.996	0.333333333333333	0.333333333333333	\\
1.5e-09	0.996	-0.666666666666667	-0.666666666666667	\\
1.55e-09	0.996	-0.666666666666667	-0.666666666666667	\\
1.6e-09	0.996	-0.666666666666667	-0.666666666666667	\\
1.65e-09	0.996	-0.666666666666667	-0.666666666666667	\\
1.7e-09	0.996	-0.666666666666667	-0.666666666666667	\\
1.75e-09	0.996	-0.666666666666667	-0.666666666666667	\\
1.8e-09	0.996	-0.666666666666667	-0.666666666666667	\\
1.85e-09	0.996	-0.666666666666667	-0.666666666666667	\\
1.9e-09	0.996	-0.666666666666667	-0.666666666666667	\\
1.95e-09	0.996	0	0	\\
2e-09	0.996	0	0	\\
2.05e-09	0.996	0	0	\\
2.1e-09	0.996	0	0	\\
2.15e-09	0.996	0	0	\\
2.2e-09	0.996	0	0	\\
2.25e-09	0.996	0	0	\\
2.3e-09	0.996	0	0	\\
2.35e-09	0.996	0	0	\\
2.4e-09	0.996	0	0	\\
2.45e-09	0.996	0	0	\\
2.5e-09	0.996	0	0	\\
2.55e-09	0.996	0	0	\\
2.6e-09	0.996	0	0	\\
2.65e-09	0.996	0	0	\\
2.7e-09	0.996	0	0	\\
2.75e-09	0.996	0	0	\\
2.8e-09	0.996	0	0	\\
2.85e-09	0.996	0	0	\\
2.9e-09	0.996	0	0	\\
2.95e-09	0.996	0	0	\\
3e-09	0.996	0	0	\\
3.05e-09	0.996	0	0	\\
3.1e-09	0.996	0	0	\\
3.15e-09	0.996	0	0	\\
3.2e-09	0.996	0	0	\\
3.25e-09	0.996	0	0	\\
3.3e-09	0.996	0	0	\\
3.35e-09	0.996	0	0	\\
3.4e-09	0.996	0.666666666666667	0.666666666666667	\\
3.45e-09	0.996	0.666666666666667	0.666666666666667	\\
3.5e-09	0.996	0.666666666666667	0.666666666666667	\\
3.55e-09	0.996	0.666666666666667	0.666666666666667	\\
3.6e-09	0.996	0.666666666666667	0.666666666666667	\\
3.65e-09	0.996	0.666666666666667	0.666666666666667	\\
3.7e-09	0.996	0.666666666666667	0.666666666666667	\\
3.75e-09	0.996	0.666666666666667	0.666666666666667	\\
3.8e-09	0.996	0.666666666666667	0.666666666666667	\\
3.85e-09	0.996	0.222222222222222	0.222222222222222	\\
3.9e-09	0.996	-0.444444444444444	-0.444444444444444	\\
3.95e-09	0.996	-0.444444444444444	-0.444444444444444	\\
4e-09	0.996	-0.444444444444444	-0.444444444444444	\\
4.05e-09	0.996	-0.444444444444444	-0.444444444444444	\\
4.1e-09	0.996	-0.444444444444444	-0.444444444444444	\\
4.15e-09	0.996	-0.444444444444444	-0.444444444444444	\\
4.2e-09	0.996	-0.444444444444444	-0.444444444444444	\\
4.25e-09	0.996	-0.444444444444444	-0.444444444444444	\\
4.3e-09	0.996	-0.444444444444444	-0.444444444444444	\\
4.35e-09	0.996	0	0	\\
4.4e-09	0.996	0	0	\\
4.45e-09	0.996	0	0	\\
4.5e-09	0.996	0	0	\\
4.55e-09	0.996	0	0	\\
4.6e-09	0.996	0	0	\\
4.65e-09	0.996	0	0	\\
4.7e-09	0.996	0	0	\\
4.75e-09	0.996	0	0	\\
4.8e-09	0.996	0	0	\\
4.85e-09	0.996	0	0	\\
4.9e-09	0.996	0	0	\\
4.95e-09	0.996	0	0	\\
5e-09	0.996	0	0	\\
5e-09	0.996	0	nan	\\
5e-09	0.996	-0.666666666666667	0.166666666666667	\\
5e-09	0.996	-0.666666666666667	nan	\\
0	1.008	-0.666666666666667	nan	\\
0	1.008	0	0.166666666666667	\\
0	1.008	0	0	\\
5e-11	1.008	0	0	\\
1e-10	1.008	0	0	\\
1.5e-10	1.008	0	0	\\
2e-10	1.008	0	0	\\
2.5e-10	1.008	0	0	\\
3e-10	1.008	0	0	\\
3.5e-10	1.008	0	0	\\
4e-10	1.008	0	0	\\
4.5e-10	1.008	0	0	\\
5e-10	1.008	0	0	\\
5.5e-10	1.008	0	0	\\
6e-10	1.008	0	0	\\
6.5e-10	1.008	0	0	\\
7e-10	1.008	0	0	\\
7.5e-10	1.008	0	0	\\
8e-10	1.008	0	0	\\
8.5e-10	1.008	0	0	\\
9e-10	1.008	0	0	\\
9.5e-10	1.008	0	0	\\
1e-09	1.008	0	0	\\
1.05e-09	1.008	1	1	\\
1.1e-09	1.008	1	1	\\
1.15e-09	1.008	1	1	\\
1.2e-09	1.008	1	1	\\
1.25e-09	1.008	1	1	\\
1.3e-09	1.008	1	1	\\
1.35e-09	1.008	1	1	\\
1.4e-09	1.008	0.333333333333333	0.333333333333333	\\
1.45e-09	1.008	0.333333333333333	0.333333333333333	\\
1.5e-09	1.008	0.333333333333333	0.333333333333333	\\
1.55e-09	1.008	-0.666666666666667	-0.666666666666667	\\
1.6e-09	1.008	-0.666666666666667	-0.666666666666667	\\
1.65e-09	1.008	-0.666666666666667	-0.666666666666667	\\
1.7e-09	1.008	-0.666666666666667	-0.666666666666667	\\
1.75e-09	1.008	-0.666666666666667	-0.666666666666667	\\
1.8e-09	1.008	-0.666666666666667	-0.666666666666667	\\
1.85e-09	1.008	-0.666666666666667	-0.666666666666667	\\
1.9e-09	1.008	0	0	\\
1.95e-09	1.008	0	0	\\
2e-09	1.008	0	0	\\
2.05e-09	1.008	0	0	\\
2.1e-09	1.008	0	0	\\
2.15e-09	1.008	0	0	\\
2.2e-09	1.008	0	0	\\
2.25e-09	1.008	0	0	\\
2.3e-09	1.008	0	0	\\
2.35e-09	1.008	0	0	\\
2.4e-09	1.008	0	0	\\
2.45e-09	1.008	0	0	\\
2.5e-09	1.008	0	0	\\
2.55e-09	1.008	0	0	\\
2.6e-09	1.008	0	0	\\
2.65e-09	1.008	0	0	\\
2.7e-09	1.008	0	0	\\
2.75e-09	1.008	0	0	\\
2.8e-09	1.008	0	0	\\
2.85e-09	1.008	0	0	\\
2.9e-09	1.008	0	0	\\
2.95e-09	1.008	0	0	\\
3e-09	1.008	0	0	\\
3.05e-09	1.008	0	0	\\
3.1e-09	1.008	0	0	\\
3.15e-09	1.008	0	0	\\
3.2e-09	1.008	0	0	\\
3.25e-09	1.008	0	0	\\
3.3e-09	1.008	0	0	\\
3.35e-09	1.008	0	0	\\
3.4e-09	1.008	0	0	\\
3.45e-09	1.008	0.666666666666667	0.666666666666667	\\
3.5e-09	1.008	0.666666666666667	0.666666666666667	\\
3.55e-09	1.008	0.666666666666667	0.666666666666667	\\
3.6e-09	1.008	0.666666666666667	0.666666666666667	\\
3.65e-09	1.008	0.666666666666667	0.666666666666667	\\
3.7e-09	1.008	0.666666666666667	0.666666666666667	\\
3.75e-09	1.008	0.666666666666667	0.666666666666667	\\
3.8e-09	1.008	0.222222222222222	0.222222222222222	\\
3.85e-09	1.008	0.222222222222222	0.222222222222222	\\
3.9e-09	1.008	0.222222222222222	0.222222222222222	\\
3.95e-09	1.008	-0.444444444444444	-0.444444444444444	\\
4e-09	1.008	-0.444444444444444	-0.444444444444444	\\
4.05e-09	1.008	-0.444444444444444	-0.444444444444444	\\
4.1e-09	1.008	-0.444444444444444	-0.444444444444444	\\
4.15e-09	1.008	-0.444444444444444	-0.444444444444444	\\
4.2e-09	1.008	-0.444444444444444	-0.444444444444444	\\
4.25e-09	1.008	-0.444444444444444	-0.444444444444444	\\
4.3e-09	1.008	0	0	\\
4.35e-09	1.008	0	0	\\
4.4e-09	1.008	0	0	\\
4.45e-09	1.008	0	0	\\
4.5e-09	1.008	0	0	\\
4.55e-09	1.008	0	0	\\
4.6e-09	1.008	0	0	\\
4.65e-09	1.008	0	0	\\
4.7e-09	1.008	0	0	\\
4.75e-09	1.008	0	0	\\
4.8e-09	1.008	0	0	\\
4.85e-09	1.008	0	0	\\
4.9e-09	1.008	0	0	\\
4.95e-09	1.008	0	0	\\
5e-09	1.008	0	0	\\
5e-09	1.008	0	nan	\\
5e-09	1.008	-0.666666666666667	0.166666666666667	\\
5e-09	1.008	-0.666666666666667	nan	\\
0	1.02	-0.666666666666667	nan	\\
0	1.02	0	0.166666666666667	\\
0	1.02	0	0	\\
5e-11	1.02	0	0	\\
1e-10	1.02	0	0	\\
1.5e-10	1.02	0	0	\\
2e-10	1.02	0	0	\\
2.5e-10	1.02	0	0	\\
3e-10	1.02	0	0	\\
3.5e-10	1.02	0	0	\\
4e-10	1.02	0	0	\\
4.5e-10	1.02	0	0	\\
5e-10	1.02	0	0	\\
5.5e-10	1.02	0	0	\\
6e-10	1.02	0	0	\\
6.5e-10	1.02	0	0	\\
7e-10	1.02	0	0	\\
7.5e-10	1.02	0	0	\\
8e-10	1.02	0	0	\\
8.5e-10	1.02	0	0	\\
9e-10	1.02	0	0	\\
9.5e-10	1.02	0	0	\\
1e-09	1.02	0	0	\\
1.05e-09	1.02	1	1	\\
1.1e-09	1.02	1	1	\\
1.15e-09	1.02	1	1	\\
1.2e-09	1.02	1	1	\\
1.25e-09	1.02	1	1	\\
1.3e-09	1.02	1	1	\\
1.35e-09	1.02	1	1	\\
1.4e-09	1.02	0.333333333333333	0.333333333333333	\\
1.45e-09	1.02	0.333333333333333	0.333333333333333	\\
1.5e-09	1.02	0.333333333333333	0.333333333333333	\\
1.55e-09	1.02	-0.666666666666667	-0.666666666666667	\\
1.6e-09	1.02	-0.666666666666667	-0.666666666666667	\\
1.65e-09	1.02	-0.666666666666667	-0.666666666666667	\\
1.7e-09	1.02	-0.666666666666667	-0.666666666666667	\\
1.75e-09	1.02	-0.666666666666667	-0.666666666666667	\\
1.8e-09	1.02	-0.666666666666667	-0.666666666666667	\\
1.85e-09	1.02	-0.666666666666667	-0.666666666666667	\\
1.9e-09	1.02	0	0	\\
1.95e-09	1.02	0	0	\\
2e-09	1.02	0	0	\\
2.05e-09	1.02	0	0	\\
2.1e-09	1.02	0	0	\\
2.15e-09	1.02	0	0	\\
2.2e-09	1.02	0	0	\\
2.25e-09	1.02	0	0	\\
2.3e-09	1.02	0	0	\\
2.35e-09	1.02	0	0	\\
2.4e-09	1.02	0	0	\\
2.45e-09	1.02	0	0	\\
2.5e-09	1.02	0	0	\\
2.55e-09	1.02	0	0	\\
2.6e-09	1.02	0	0	\\
2.65e-09	1.02	0	0	\\
2.7e-09	1.02	0	0	\\
2.75e-09	1.02	0	0	\\
2.8e-09	1.02	0	0	\\
2.85e-09	1.02	0	0	\\
2.9e-09	1.02	0	0	\\
2.95e-09	1.02	0	0	\\
3e-09	1.02	0	0	\\
3.05e-09	1.02	0	0	\\
3.1e-09	1.02	0	0	\\
3.15e-09	1.02	0	0	\\
3.2e-09	1.02	0	0	\\
3.25e-09	1.02	0	0	\\
3.3e-09	1.02	0	0	\\
3.35e-09	1.02	0	0	\\
3.4e-09	1.02	0	0	\\
3.45e-09	1.02	0.666666666666667	0.666666666666667	\\
3.5e-09	1.02	0.666666666666667	0.666666666666667	\\
3.55e-09	1.02	0.666666666666667	0.666666666666667	\\
3.6e-09	1.02	0.666666666666667	0.666666666666667	\\
3.65e-09	1.02	0.666666666666667	0.666666666666667	\\
3.7e-09	1.02	0.666666666666667	0.666666666666667	\\
3.75e-09	1.02	0.666666666666667	0.666666666666667	\\
3.8e-09	1.02	0.222222222222222	0.222222222222222	\\
3.85e-09	1.02	0.222222222222222	0.222222222222222	\\
3.9e-09	1.02	0.222222222222222	0.222222222222222	\\
3.95e-09	1.02	-0.444444444444444	-0.444444444444444	\\
4e-09	1.02	-0.444444444444444	-0.444444444444444	\\
4.05e-09	1.02	-0.444444444444444	-0.444444444444444	\\
4.1e-09	1.02	-0.444444444444444	-0.444444444444444	\\
4.15e-09	1.02	-0.444444444444444	-0.444444444444444	\\
4.2e-09	1.02	-0.444444444444444	-0.444444444444444	\\
4.25e-09	1.02	-0.444444444444444	-0.444444444444444	\\
4.3e-09	1.02	0	0	\\
4.35e-09	1.02	0	0	\\
4.4e-09	1.02	0	0	\\
4.45e-09	1.02	0	0	\\
4.5e-09	1.02	0	0	\\
4.55e-09	1.02	0	0	\\
4.6e-09	1.02	0	0	\\
4.65e-09	1.02	0	0	\\
4.7e-09	1.02	0	0	\\
4.75e-09	1.02	0	0	\\
4.8e-09	1.02	0	0	\\
4.85e-09	1.02	0	0	\\
4.9e-09	1.02	0	0	\\
4.95e-09	1.02	0	0	\\
5e-09	1.02	0	0	\\
5e-09	1.02	0	nan	\\
5e-09	1.02	-0.666666666666667	0.166666666666667	\\
5e-09	1.02	-0.666666666666667	nan	\\
0	1.032	-0.666666666666667	nan	\\
0	1.032	0	0.166666666666667	\\
0	1.032	0	0	\\
5e-11	1.032	0	0	\\
1e-10	1.032	0	0	\\
1.5e-10	1.032	0	0	\\
2e-10	1.032	0	0	\\
2.5e-10	1.032	0	0	\\
3e-10	1.032	0	0	\\
3.5e-10	1.032	0	0	\\
4e-10	1.032	0	0	\\
4.5e-10	1.032	0	0	\\
5e-10	1.032	0	0	\\
5.5e-10	1.032	0	0	\\
6e-10	1.032	0	0	\\
6.5e-10	1.032	0	0	\\
7e-10	1.032	0	0	\\
7.5e-10	1.032	0	0	\\
8e-10	1.032	0	0	\\
8.5e-10	1.032	0	0	\\
9e-10	1.032	0	0	\\
9.5e-10	1.032	0	0	\\
1e-09	1.032	0	0	\\
1.05e-09	1.032	1	1	\\
1.1e-09	1.032	1	1	\\
1.15e-09	1.032	1	1	\\
1.2e-09	1.032	1	1	\\
1.25e-09	1.032	1	1	\\
1.3e-09	1.032	1	1	\\
1.35e-09	1.032	1	1	\\
1.4e-09	1.032	0.333333333333333	0.333333333333333	\\
1.45e-09	1.032	0.333333333333333	0.333333333333333	\\
1.5e-09	1.032	0.333333333333333	0.333333333333333	\\
1.55e-09	1.032	-0.666666666666667	-0.666666666666667	\\
1.6e-09	1.032	-0.666666666666667	-0.666666666666667	\\
1.65e-09	1.032	-0.666666666666667	-0.666666666666667	\\
1.7e-09	1.032	-0.666666666666667	-0.666666666666667	\\
1.75e-09	1.032	-0.666666666666667	-0.666666666666667	\\
1.8e-09	1.032	-0.666666666666667	-0.666666666666667	\\
1.85e-09	1.032	-0.666666666666667	-0.666666666666667	\\
1.9e-09	1.032	0	0	\\
1.95e-09	1.032	0	0	\\
2e-09	1.032	0	0	\\
2.05e-09	1.032	0	0	\\
2.1e-09	1.032	0	0	\\
2.15e-09	1.032	0	0	\\
2.2e-09	1.032	0	0	\\
2.25e-09	1.032	0	0	\\
2.3e-09	1.032	0	0	\\
2.35e-09	1.032	0	0	\\
2.4e-09	1.032	0	0	\\
2.45e-09	1.032	0	0	\\
2.5e-09	1.032	0	0	\\
2.55e-09	1.032	0	0	\\
2.6e-09	1.032	0	0	\\
2.65e-09	1.032	0	0	\\
2.7e-09	1.032	0	0	\\
2.75e-09	1.032	0	0	\\
2.8e-09	1.032	0	0	\\
2.85e-09	1.032	0	0	\\
2.9e-09	1.032	0	0	\\
2.95e-09	1.032	0	0	\\
3e-09	1.032	0	0	\\
3.05e-09	1.032	0	0	\\
3.1e-09	1.032	0	0	\\
3.15e-09	1.032	0	0	\\
3.2e-09	1.032	0	0	\\
3.25e-09	1.032	0	0	\\
3.3e-09	1.032	0	0	\\
3.35e-09	1.032	0	0	\\
3.4e-09	1.032	0	0	\\
3.45e-09	1.032	0.666666666666667	0.666666666666667	\\
3.5e-09	1.032	0.666666666666667	0.666666666666667	\\
3.55e-09	1.032	0.666666666666667	0.666666666666667	\\
3.6e-09	1.032	0.666666666666667	0.666666666666667	\\
3.65e-09	1.032	0.666666666666667	0.666666666666667	\\
3.7e-09	1.032	0.666666666666667	0.666666666666667	\\
3.75e-09	1.032	0.666666666666667	0.666666666666667	\\
3.8e-09	1.032	0.222222222222222	0.222222222222222	\\
3.85e-09	1.032	0.222222222222222	0.222222222222222	\\
3.9e-09	1.032	0.222222222222222	0.222222222222222	\\
3.95e-09	1.032	-0.444444444444444	-0.444444444444444	\\
4e-09	1.032	-0.444444444444444	-0.444444444444444	\\
4.05e-09	1.032	-0.444444444444444	-0.444444444444444	\\
4.1e-09	1.032	-0.444444444444444	-0.444444444444444	\\
4.15e-09	1.032	-0.444444444444444	-0.444444444444444	\\
4.2e-09	1.032	-0.444444444444444	-0.444444444444444	\\
4.25e-09	1.032	-0.444444444444444	-0.444444444444444	\\
4.3e-09	1.032	0	0	\\
4.35e-09	1.032	0	0	\\
4.4e-09	1.032	0	0	\\
4.45e-09	1.032	0	0	\\
4.5e-09	1.032	0	0	\\
4.55e-09	1.032	0	0	\\
4.6e-09	1.032	0	0	\\
4.65e-09	1.032	0	0	\\
4.7e-09	1.032	0	0	\\
4.75e-09	1.032	0	0	\\
4.8e-09	1.032	0	0	\\
4.85e-09	1.032	0	0	\\
4.9e-09	1.032	0	0	\\
4.95e-09	1.032	0	0	\\
5e-09	1.032	0	0	\\
5e-09	1.032	0	nan	\\
5e-09	1.032	-0.666666666666667	0.166666666666667	\\
5e-09	1.032	-0.666666666666667	nan	\\
0	1.044	-0.666666666666667	nan	\\
0	1.044	0	0.166666666666667	\\
0	1.044	0	0	\\
5e-11	1.044	0	0	\\
1e-10	1.044	0	0	\\
1.5e-10	1.044	0	0	\\
2e-10	1.044	0	0	\\
2.5e-10	1.044	0	0	\\
3e-10	1.044	0	0	\\
3.5e-10	1.044	0	0	\\
4e-10	1.044	0	0	\\
4.5e-10	1.044	0	0	\\
5e-10	1.044	0	0	\\
5.5e-10	1.044	0	0	\\
6e-10	1.044	0	0	\\
6.5e-10	1.044	0	0	\\
7e-10	1.044	0	0	\\
7.5e-10	1.044	0	0	\\
8e-10	1.044	0	0	\\
8.5e-10	1.044	0	0	\\
9e-10	1.044	0	0	\\
9.5e-10	1.044	0	0	\\
1e-09	1.044	0	0	\\
1.05e-09	1.044	1	1	\\
1.1e-09	1.044	1	1	\\
1.15e-09	1.044	1	1	\\
1.2e-09	1.044	1	1	\\
1.25e-09	1.044	1	1	\\
1.3e-09	1.044	1	1	\\
1.35e-09	1.044	1	1	\\
1.4e-09	1.044	0.333333333333333	0.333333333333333	\\
1.45e-09	1.044	0.333333333333333	0.333333333333333	\\
1.5e-09	1.044	0.333333333333333	0.333333333333333	\\
1.55e-09	1.044	-0.666666666666667	-0.666666666666667	\\
1.6e-09	1.044	-0.666666666666667	-0.666666666666667	\\
1.65e-09	1.044	-0.666666666666667	-0.666666666666667	\\
1.7e-09	1.044	-0.666666666666667	-0.666666666666667	\\
1.75e-09	1.044	-0.666666666666667	-0.666666666666667	\\
1.8e-09	1.044	-0.666666666666667	-0.666666666666667	\\
1.85e-09	1.044	-0.666666666666667	-0.666666666666667	\\
1.9e-09	1.044	0	0	\\
1.95e-09	1.044	0	0	\\
2e-09	1.044	0	0	\\
2.05e-09	1.044	0	0	\\
2.1e-09	1.044	0	0	\\
2.15e-09	1.044	0	0	\\
2.2e-09	1.044	0	0	\\
2.25e-09	1.044	0	0	\\
2.3e-09	1.044	0	0	\\
2.35e-09	1.044	0	0	\\
2.4e-09	1.044	0	0	\\
2.45e-09	1.044	0	0	\\
2.5e-09	1.044	0	0	\\
2.55e-09	1.044	0	0	\\
2.6e-09	1.044	0	0	\\
2.65e-09	1.044	0	0	\\
2.7e-09	1.044	0	0	\\
2.75e-09	1.044	0	0	\\
2.8e-09	1.044	0	0	\\
2.85e-09	1.044	0	0	\\
2.9e-09	1.044	0	0	\\
2.95e-09	1.044	0	0	\\
3e-09	1.044	0	0	\\
3.05e-09	1.044	0	0	\\
3.1e-09	1.044	0	0	\\
3.15e-09	1.044	0	0	\\
3.2e-09	1.044	0	0	\\
3.25e-09	1.044	0	0	\\
3.3e-09	1.044	0	0	\\
3.35e-09	1.044	0	0	\\
3.4e-09	1.044	0	0	\\
3.45e-09	1.044	0.666666666666667	0.666666666666667	\\
3.5e-09	1.044	0.666666666666667	0.666666666666667	\\
3.55e-09	1.044	0.666666666666667	0.666666666666667	\\
3.6e-09	1.044	0.666666666666667	0.666666666666667	\\
3.65e-09	1.044	0.666666666666667	0.666666666666667	\\
3.7e-09	1.044	0.666666666666667	0.666666666666667	\\
3.75e-09	1.044	0.666666666666667	0.666666666666667	\\
3.8e-09	1.044	0.222222222222222	0.222222222222222	\\
3.85e-09	1.044	0.222222222222222	0.222222222222222	\\
3.9e-09	1.044	0.222222222222222	0.222222222222222	\\
3.95e-09	1.044	-0.444444444444444	-0.444444444444444	\\
4e-09	1.044	-0.444444444444444	-0.444444444444444	\\
4.05e-09	1.044	-0.444444444444444	-0.444444444444444	\\
4.1e-09	1.044	-0.444444444444444	-0.444444444444444	\\
4.15e-09	1.044	-0.444444444444444	-0.444444444444444	\\
4.2e-09	1.044	-0.444444444444444	-0.444444444444444	\\
4.25e-09	1.044	-0.444444444444444	-0.444444444444444	\\
4.3e-09	1.044	0	0	\\
4.35e-09	1.044	0	0	\\
4.4e-09	1.044	0	0	\\
4.45e-09	1.044	0	0	\\
4.5e-09	1.044	0	0	\\
4.55e-09	1.044	0	0	\\
4.6e-09	1.044	0	0	\\
4.65e-09	1.044	0	0	\\
4.7e-09	1.044	0	0	\\
4.75e-09	1.044	0	0	\\
4.8e-09	1.044	0	0	\\
4.85e-09	1.044	0	0	\\
4.9e-09	1.044	0	0	\\
4.95e-09	1.044	0	0	\\
5e-09	1.044	0	0	\\
5e-09	1.044	0	nan	\\
5e-09	1.044	-0.666666666666667	0.166666666666667	\\
5e-09	1.044	-0.666666666666667	nan	\\
0	1.056	-0.666666666666667	nan	\\
0	1.056	0	0.166666666666667	\\
0	1.056	0	0	\\
5e-11	1.056	0	0	\\
1e-10	1.056	0	0	\\
1.5e-10	1.056	0	0	\\
2e-10	1.056	0	0	\\
2.5e-10	1.056	0	0	\\
3e-10	1.056	0	0	\\
3.5e-10	1.056	0	0	\\
4e-10	1.056	0	0	\\
4.5e-10	1.056	0	0	\\
5e-10	1.056	0	0	\\
5.5e-10	1.056	0	0	\\
6e-10	1.056	0	0	\\
6.5e-10	1.056	0	0	\\
7e-10	1.056	0	0	\\
7.5e-10	1.056	0	0	\\
8e-10	1.056	0	0	\\
8.5e-10	1.056	0	0	\\
9e-10	1.056	0	0	\\
9.5e-10	1.056	0	0	\\
1e-09	1.056	0	0	\\
1.05e-09	1.056	0	0	\\
1.1e-09	1.056	1	1	\\
1.15e-09	1.056	1	1	\\
1.2e-09	1.056	1	1	\\
1.25e-09	1.056	1	1	\\
1.3e-09	1.056	1	1	\\
1.35e-09	1.056	0.333333333333333	0.333333333333333	\\
1.4e-09	1.056	0.333333333333333	0.333333333333333	\\
1.45e-09	1.056	0.333333333333333	0.333333333333333	\\
1.5e-09	1.056	0.333333333333333	0.333333333333333	\\
1.55e-09	1.056	0.333333333333333	0.333333333333333	\\
1.6e-09	1.056	-0.666666666666667	-0.666666666666667	\\
1.65e-09	1.056	-0.666666666666667	-0.666666666666667	\\
1.7e-09	1.056	-0.666666666666667	-0.666666666666667	\\
1.75e-09	1.056	-0.666666666666667	-0.666666666666667	\\
1.8e-09	1.056	-0.666666666666667	-0.666666666666667	\\
1.85e-09	1.056	0	0	\\
1.9e-09	1.056	0	0	\\
1.95e-09	1.056	0	0	\\
2e-09	1.056	0	0	\\
2.05e-09	1.056	0	0	\\
2.1e-09	1.056	0	0	\\
2.15e-09	1.056	0	0	\\
2.2e-09	1.056	0	0	\\
2.25e-09	1.056	0	0	\\
2.3e-09	1.056	0	0	\\
2.35e-09	1.056	0	0	\\
2.4e-09	1.056	0	0	\\
2.45e-09	1.056	0	0	\\
2.5e-09	1.056	0	0	\\
2.55e-09	1.056	0	0	\\
2.6e-09	1.056	0	0	\\
2.65e-09	1.056	0	0	\\
2.7e-09	1.056	0	0	\\
2.75e-09	1.056	0	0	\\
2.8e-09	1.056	0	0	\\
2.85e-09	1.056	0	0	\\
2.9e-09	1.056	0	0	\\
2.95e-09	1.056	0	0	\\
3e-09	1.056	0	0	\\
3.05e-09	1.056	0	0	\\
3.1e-09	1.056	0	0	\\
3.15e-09	1.056	0	0	\\
3.2e-09	1.056	0	0	\\
3.25e-09	1.056	0	0	\\
3.3e-09	1.056	0	0	\\
3.35e-09	1.056	0	0	\\
3.4e-09	1.056	0	0	\\
3.45e-09	1.056	0	0	\\
3.5e-09	1.056	0.666666666666667	0.666666666666667	\\
3.55e-09	1.056	0.666666666666667	0.666666666666667	\\
3.6e-09	1.056	0.666666666666667	0.666666666666667	\\
3.65e-09	1.056	0.666666666666667	0.666666666666667	\\
3.7e-09	1.056	0.666666666666667	0.666666666666667	\\
3.75e-09	1.056	0.222222222222222	0.222222222222222	\\
3.8e-09	1.056	0.222222222222222	0.222222222222222	\\
3.85e-09	1.056	0.222222222222222	0.222222222222222	\\
3.9e-09	1.056	0.222222222222222	0.222222222222222	\\
3.95e-09	1.056	0.222222222222222	0.222222222222222	\\
4e-09	1.056	-0.444444444444444	-0.444444444444444	\\
4.05e-09	1.056	-0.444444444444444	-0.444444444444444	\\
4.1e-09	1.056	-0.444444444444444	-0.444444444444444	\\
4.15e-09	1.056	-0.444444444444444	-0.444444444444444	\\
4.2e-09	1.056	-0.444444444444444	-0.444444444444444	\\
4.25e-09	1.056	0	0	\\
4.3e-09	1.056	0	0	\\
4.35e-09	1.056	0	0	\\
4.4e-09	1.056	0	0	\\
4.45e-09	1.056	0	0	\\
4.5e-09	1.056	0	0	\\
4.55e-09	1.056	0	0	\\
4.6e-09	1.056	0	0	\\
4.65e-09	1.056	0	0	\\
4.7e-09	1.056	0	0	\\
4.75e-09	1.056	0	0	\\
4.8e-09	1.056	0	0	\\
4.85e-09	1.056	0	0	\\
4.9e-09	1.056	0	0	\\
4.95e-09	1.056	0	0	\\
5e-09	1.056	0	0	\\
5e-09	1.056	0	nan	\\
5e-09	1.056	-0.666666666666667	0.166666666666667	\\
5e-09	1.056	-0.666666666666667	nan	\\
0	1.068	-0.666666666666667	nan	\\
0	1.068	0	0.166666666666667	\\
0	1.068	0	0	\\
5e-11	1.068	0	0	\\
1e-10	1.068	0	0	\\
1.5e-10	1.068	0	0	\\
2e-10	1.068	0	0	\\
2.5e-10	1.068	0	0	\\
3e-10	1.068	0	0	\\
3.5e-10	1.068	0	0	\\
4e-10	1.068	0	0	\\
4.5e-10	1.068	0	0	\\
5e-10	1.068	0	0	\\
5.5e-10	1.068	0	0	\\
6e-10	1.068	0	0	\\
6.5e-10	1.068	0	0	\\
7e-10	1.068	0	0	\\
7.5e-10	1.068	0	0	\\
8e-10	1.068	0	0	\\
8.5e-10	1.068	0	0	\\
9e-10	1.068	0	0	\\
9.5e-10	1.068	0	0	\\
1e-09	1.068	0	0	\\
1.05e-09	1.068	0	0	\\
1.1e-09	1.068	1	1	\\
1.15e-09	1.068	1	1	\\
1.2e-09	1.068	1	1	\\
1.25e-09	1.068	1	1	\\
1.3e-09	1.068	1	1	\\
1.35e-09	1.068	0.333333333333333	0.333333333333333	\\
1.4e-09	1.068	0.333333333333333	0.333333333333333	\\
1.45e-09	1.068	0.333333333333333	0.333333333333333	\\
1.5e-09	1.068	0.333333333333333	0.333333333333333	\\
1.55e-09	1.068	0.333333333333333	0.333333333333333	\\
1.6e-09	1.068	-0.666666666666667	-0.666666666666667	\\
1.65e-09	1.068	-0.666666666666667	-0.666666666666667	\\
1.7e-09	1.068	-0.666666666666667	-0.666666666666667	\\
1.75e-09	1.068	-0.666666666666667	-0.666666666666667	\\
1.8e-09	1.068	-0.666666666666667	-0.666666666666667	\\
1.85e-09	1.068	0	0	\\
1.9e-09	1.068	0	0	\\
1.95e-09	1.068	0	0	\\
2e-09	1.068	0	0	\\
2.05e-09	1.068	0	0	\\
2.1e-09	1.068	0	0	\\
2.15e-09	1.068	0	0	\\
2.2e-09	1.068	0	0	\\
2.25e-09	1.068	0	0	\\
2.3e-09	1.068	0	0	\\
2.35e-09	1.068	0	0	\\
2.4e-09	1.068	0	0	\\
2.45e-09	1.068	0	0	\\
2.5e-09	1.068	0	0	\\
2.55e-09	1.068	0	0	\\
2.6e-09	1.068	0	0	\\
2.65e-09	1.068	0	0	\\
2.7e-09	1.068	0	0	\\
2.75e-09	1.068	0	0	\\
2.8e-09	1.068	0	0	\\
2.85e-09	1.068	0	0	\\
2.9e-09	1.068	0	0	\\
2.95e-09	1.068	0	0	\\
3e-09	1.068	0	0	\\
3.05e-09	1.068	0	0	\\
3.1e-09	1.068	0	0	\\
3.15e-09	1.068	0	0	\\
3.2e-09	1.068	0	0	\\
3.25e-09	1.068	0	0	\\
3.3e-09	1.068	0	0	\\
3.35e-09	1.068	0	0	\\
3.4e-09	1.068	0	0	\\
3.45e-09	1.068	0	0	\\
3.5e-09	1.068	0.666666666666667	0.666666666666667	\\
3.55e-09	1.068	0.666666666666667	0.666666666666667	\\
3.6e-09	1.068	0.666666666666667	0.666666666666667	\\
3.65e-09	1.068	0.666666666666667	0.666666666666667	\\
3.7e-09	1.068	0.666666666666667	0.666666666666667	\\
3.75e-09	1.068	0.222222222222222	0.222222222222222	\\
3.8e-09	1.068	0.222222222222222	0.222222222222222	\\
3.85e-09	1.068	0.222222222222222	0.222222222222222	\\
3.9e-09	1.068	0.222222222222222	0.222222222222222	\\
3.95e-09	1.068	0.222222222222222	0.222222222222222	\\
4e-09	1.068	-0.444444444444444	-0.444444444444444	\\
4.05e-09	1.068	-0.444444444444444	-0.444444444444444	\\
4.1e-09	1.068	-0.444444444444444	-0.444444444444444	\\
4.15e-09	1.068	-0.444444444444444	-0.444444444444444	\\
4.2e-09	1.068	-0.444444444444444	-0.444444444444444	\\
4.25e-09	1.068	0	0	\\
4.3e-09	1.068	0	0	\\
4.35e-09	1.068	0	0	\\
4.4e-09	1.068	0	0	\\
4.45e-09	1.068	0	0	\\
4.5e-09	1.068	0	0	\\
4.55e-09	1.068	0	0	\\
4.6e-09	1.068	0	0	\\
4.65e-09	1.068	0	0	\\
4.7e-09	1.068	0	0	\\
4.75e-09	1.068	0	0	\\
4.8e-09	1.068	0	0	\\
4.85e-09	1.068	0	0	\\
4.9e-09	1.068	0	0	\\
4.95e-09	1.068	0	0	\\
5e-09	1.068	0	0	\\
5e-09	1.068	0	nan	\\
5e-09	1.068	-0.666666666666667	0.166666666666667	\\
5e-09	1.068	-0.666666666666667	nan	\\
0	1.08	-0.666666666666667	nan	\\
0	1.08	0	0.166666666666667	\\
0	1.08	0	0	\\
5e-11	1.08	0	0	\\
1e-10	1.08	0	0	\\
1.5e-10	1.08	0	0	\\
2e-10	1.08	0	0	\\
2.5e-10	1.08	0	0	\\
3e-10	1.08	0	0	\\
3.5e-10	1.08	0	0	\\
4e-10	1.08	0	0	\\
4.5e-10	1.08	0	0	\\
5e-10	1.08	0	0	\\
5.5e-10	1.08	0	0	\\
6e-10	1.08	0	0	\\
6.5e-10	1.08	0	0	\\
7e-10	1.08	0	0	\\
7.5e-10	1.08	0	0	\\
8e-10	1.08	0	0	\\
8.5e-10	1.08	0	0	\\
9e-10	1.08	0	0	\\
9.5e-10	1.08	0	0	\\
1e-09	1.08	0	0	\\
1.05e-09	1.08	0	0	\\
1.1e-09	1.08	1	1	\\
1.15e-09	1.08	1	1	\\
1.2e-09	1.08	1	1	\\
1.25e-09	1.08	1	1	\\
1.3e-09	1.08	1	1	\\
1.35e-09	1.08	0.333333333333333	0.333333333333333	\\
1.4e-09	1.08	0.333333333333333	0.333333333333333	\\
1.45e-09	1.08	0.333333333333333	0.333333333333333	\\
1.5e-09	1.08	0.333333333333333	0.333333333333333	\\
1.55e-09	1.08	0.333333333333333	0.333333333333333	\\
1.6e-09	1.08	-0.666666666666667	-0.666666666666667	\\
1.65e-09	1.08	-0.666666666666667	-0.666666666666667	\\
1.7e-09	1.08	-0.666666666666667	-0.666666666666667	\\
1.75e-09	1.08	-0.666666666666667	-0.666666666666667	\\
1.8e-09	1.08	-0.666666666666667	-0.666666666666667	\\
1.85e-09	1.08	0	0	\\
1.9e-09	1.08	0	0	\\
1.95e-09	1.08	0	0	\\
2e-09	1.08	0	0	\\
2.05e-09	1.08	0	0	\\
2.1e-09	1.08	0	0	\\
2.15e-09	1.08	0	0	\\
2.2e-09	1.08	0	0	\\
2.25e-09	1.08	0	0	\\
2.3e-09	1.08	0	0	\\
2.35e-09	1.08	0	0	\\
2.4e-09	1.08	0	0	\\
2.45e-09	1.08	0	0	\\
2.5e-09	1.08	0	0	\\
2.55e-09	1.08	0	0	\\
2.6e-09	1.08	0	0	\\
2.65e-09	1.08	0	0	\\
2.7e-09	1.08	0	0	\\
2.75e-09	1.08	0	0	\\
2.8e-09	1.08	0	0	\\
2.85e-09	1.08	0	0	\\
2.9e-09	1.08	0	0	\\
2.95e-09	1.08	0	0	\\
3e-09	1.08	0	0	\\
3.05e-09	1.08	0	0	\\
3.1e-09	1.08	0	0	\\
3.15e-09	1.08	0	0	\\
3.2e-09	1.08	0	0	\\
3.25e-09	1.08	0	0	\\
3.3e-09	1.08	0	0	\\
3.35e-09	1.08	0	0	\\
3.4e-09	1.08	0	0	\\
3.45e-09	1.08	0	0	\\
3.5e-09	1.08	0.666666666666667	0.666666666666667	\\
3.55e-09	1.08	0.666666666666667	0.666666666666667	\\
3.6e-09	1.08	0.666666666666667	0.666666666666667	\\
3.65e-09	1.08	0.666666666666667	0.666666666666667	\\
3.7e-09	1.08	0.666666666666667	0.666666666666667	\\
3.75e-09	1.08	0.222222222222222	0.222222222222222	\\
3.8e-09	1.08	0.222222222222222	0.222222222222222	\\
3.85e-09	1.08	0.222222222222222	0.222222222222222	\\
3.9e-09	1.08	0.222222222222222	0.222222222222222	\\
3.95e-09	1.08	0.222222222222222	0.222222222222222	\\
4e-09	1.08	-0.444444444444444	-0.444444444444444	\\
4.05e-09	1.08	-0.444444444444444	-0.444444444444444	\\
4.1e-09	1.08	-0.444444444444444	-0.444444444444444	\\
4.15e-09	1.08	-0.444444444444444	-0.444444444444444	\\
4.2e-09	1.08	-0.444444444444444	-0.444444444444444	\\
4.25e-09	1.08	0	0	\\
4.3e-09	1.08	0	0	\\
4.35e-09	1.08	0	0	\\
4.4e-09	1.08	0	0	\\
4.45e-09	1.08	0	0	\\
4.5e-09	1.08	0	0	\\
4.55e-09	1.08	0	0	\\
4.6e-09	1.08	0	0	\\
4.65e-09	1.08	0	0	\\
4.7e-09	1.08	0	0	\\
4.75e-09	1.08	0	0	\\
4.8e-09	1.08	0	0	\\
4.85e-09	1.08	0	0	\\
4.9e-09	1.08	0	0	\\
4.95e-09	1.08	0	0	\\
5e-09	1.08	0	0	\\
5e-09	1.08	0	nan	\\
5e-09	1.08	-0.666666666666667	0.166666666666667	\\
5e-09	1.08	-0.666666666666667	nan	\\
0	1.092	-0.666666666666667	nan	\\
0	1.092	0	0.166666666666667	\\
0	1.092	0	0	\\
5e-11	1.092	0	0	\\
1e-10	1.092	0	0	\\
1.5e-10	1.092	0	0	\\
2e-10	1.092	0	0	\\
2.5e-10	1.092	0	0	\\
3e-10	1.092	0	0	\\
3.5e-10	1.092	0	0	\\
4e-10	1.092	0	0	\\
4.5e-10	1.092	0	0	\\
5e-10	1.092	0	0	\\
5.5e-10	1.092	0	0	\\
6e-10	1.092	0	0	\\
6.5e-10	1.092	0	0	\\
7e-10	1.092	0	0	\\
7.5e-10	1.092	0	0	\\
8e-10	1.092	0	0	\\
8.5e-10	1.092	0	0	\\
9e-10	1.092	0	0	\\
9.5e-10	1.092	0	0	\\
1e-09	1.092	0	0	\\
1.05e-09	1.092	0	0	\\
1.1e-09	1.092	1	1	\\
1.15e-09	1.092	1	1	\\
1.2e-09	1.092	1	1	\\
1.25e-09	1.092	1	1	\\
1.3e-09	1.092	1	1	\\
1.35e-09	1.092	0.333333333333333	0.333333333333333	\\
1.4e-09	1.092	0.333333333333333	0.333333333333333	\\
1.45e-09	1.092	0.333333333333333	0.333333333333333	\\
1.5e-09	1.092	0.333333333333333	0.333333333333333	\\
1.55e-09	1.092	0.333333333333333	0.333333333333333	\\
1.6e-09	1.092	-0.666666666666667	-0.666666666666667	\\
1.65e-09	1.092	-0.666666666666667	-0.666666666666667	\\
1.7e-09	1.092	-0.666666666666667	-0.666666666666667	\\
1.75e-09	1.092	-0.666666666666667	-0.666666666666667	\\
1.8e-09	1.092	-0.666666666666667	-0.666666666666667	\\
1.85e-09	1.092	0	0	\\
1.9e-09	1.092	0	0	\\
1.95e-09	1.092	0	0	\\
2e-09	1.092	0	0	\\
2.05e-09	1.092	0	0	\\
2.1e-09	1.092	0	0	\\
2.15e-09	1.092	0	0	\\
2.2e-09	1.092	0	0	\\
2.25e-09	1.092	0	0	\\
2.3e-09	1.092	0	0	\\
2.35e-09	1.092	0	0	\\
2.4e-09	1.092	0	0	\\
2.45e-09	1.092	0	0	\\
2.5e-09	1.092	0	0	\\
2.55e-09	1.092	0	0	\\
2.6e-09	1.092	0	0	\\
2.65e-09	1.092	0	0	\\
2.7e-09	1.092	0	0	\\
2.75e-09	1.092	0	0	\\
2.8e-09	1.092	0	0	\\
2.85e-09	1.092	0	0	\\
2.9e-09	1.092	0	0	\\
2.95e-09	1.092	0	0	\\
3e-09	1.092	0	0	\\
3.05e-09	1.092	0	0	\\
3.1e-09	1.092	0	0	\\
3.15e-09	1.092	0	0	\\
3.2e-09	1.092	0	0	\\
3.25e-09	1.092	0	0	\\
3.3e-09	1.092	0	0	\\
3.35e-09	1.092	0	0	\\
3.4e-09	1.092	0	0	\\
3.45e-09	1.092	0	0	\\
3.5e-09	1.092	0.666666666666667	0.666666666666667	\\
3.55e-09	1.092	0.666666666666667	0.666666666666667	\\
3.6e-09	1.092	0.666666666666667	0.666666666666667	\\
3.65e-09	1.092	0.666666666666667	0.666666666666667	\\
3.7e-09	1.092	0.666666666666667	0.666666666666667	\\
3.75e-09	1.092	0.222222222222222	0.222222222222222	\\
3.8e-09	1.092	0.222222222222222	0.222222222222222	\\
3.85e-09	1.092	0.222222222222222	0.222222222222222	\\
3.9e-09	1.092	0.222222222222222	0.222222222222222	\\
3.95e-09	1.092	0.222222222222222	0.222222222222222	\\
4e-09	1.092	-0.444444444444444	-0.444444444444444	\\
4.05e-09	1.092	-0.444444444444444	-0.444444444444444	\\
4.1e-09	1.092	-0.444444444444444	-0.444444444444444	\\
4.15e-09	1.092	-0.444444444444444	-0.444444444444444	\\
4.2e-09	1.092	-0.444444444444444	-0.444444444444444	\\
4.25e-09	1.092	0	0	\\
4.3e-09	1.092	0	0	\\
4.35e-09	1.092	0	0	\\
4.4e-09	1.092	0	0	\\
4.45e-09	1.092	0	0	\\
4.5e-09	1.092	0	0	\\
4.55e-09	1.092	0	0	\\
4.6e-09	1.092	0	0	\\
4.65e-09	1.092	0	0	\\
4.7e-09	1.092	0	0	\\
4.75e-09	1.092	0	0	\\
4.8e-09	1.092	0	0	\\
4.85e-09	1.092	0	0	\\
4.9e-09	1.092	0	0	\\
4.95e-09	1.092	0	0	\\
5e-09	1.092	0	0	\\
5e-09	1.092	0	nan	\\
5e-09	1.092	-0.666666666666667	0.166666666666667	\\
5e-09	1.092	-0.666666666666667	nan	\\
0	1.104	-0.666666666666667	nan	\\
0	1.104	0	0.166666666666667	\\
0	1.104	0	0	\\
5e-11	1.104	0	0	\\
1e-10	1.104	0	0	\\
1.5e-10	1.104	0	0	\\
2e-10	1.104	0	0	\\
2.5e-10	1.104	0	0	\\
3e-10	1.104	0	0	\\
3.5e-10	1.104	0	0	\\
4e-10	1.104	0	0	\\
4.5e-10	1.104	0	0	\\
5e-10	1.104	0	0	\\
5.5e-10	1.104	0	0	\\
6e-10	1.104	0	0	\\
6.5e-10	1.104	0	0	\\
7e-10	1.104	0	0	\\
7.5e-10	1.104	0	0	\\
8e-10	1.104	0	0	\\
8.5e-10	1.104	0	0	\\
9e-10	1.104	0	0	\\
9.5e-10	1.104	0	0	\\
1e-09	1.104	0	0	\\
1.05e-09	1.104	0	0	\\
1.1e-09	1.104	0	0	\\
1.15e-09	1.104	1	1	\\
1.2e-09	1.104	1	1	\\
1.25e-09	1.104	1	1	\\
1.3e-09	1.104	0.333333333333333	0.333333333333333	\\
1.35e-09	1.104	0.333333333333333	0.333333333333333	\\
1.4e-09	1.104	0.333333333333333	0.333333333333333	\\
1.45e-09	1.104	0.333333333333333	0.333333333333333	\\
1.5e-09	1.104	0.333333333333333	0.333333333333333	\\
1.55e-09	1.104	0.333333333333333	0.333333333333333	\\
1.6e-09	1.104	0.333333333333333	0.333333333333333	\\
1.65e-09	1.104	-0.666666666666667	-0.666666666666667	\\
1.7e-09	1.104	-0.666666666666667	-0.666666666666667	\\
1.75e-09	1.104	-0.666666666666667	-0.666666666666667	\\
1.8e-09	1.104	0	0	\\
1.85e-09	1.104	0	0	\\
1.9e-09	1.104	0	0	\\
1.95e-09	1.104	0	0	\\
2e-09	1.104	0	0	\\
2.05e-09	1.104	0	0	\\
2.1e-09	1.104	0	0	\\
2.15e-09	1.104	0	0	\\
2.2e-09	1.104	0	0	\\
2.25e-09	1.104	0	0	\\
2.3e-09	1.104	0	0	\\
2.35e-09	1.104	0	0	\\
2.4e-09	1.104	0	0	\\
2.45e-09	1.104	0	0	\\
2.5e-09	1.104	0	0	\\
2.55e-09	1.104	0	0	\\
2.6e-09	1.104	0	0	\\
2.65e-09	1.104	0	0	\\
2.7e-09	1.104	0	0	\\
2.75e-09	1.104	0	0	\\
2.8e-09	1.104	0	0	\\
2.85e-09	1.104	0	0	\\
2.9e-09	1.104	0	0	\\
2.95e-09	1.104	0	0	\\
3e-09	1.104	0	0	\\
3.05e-09	1.104	0	0	\\
3.1e-09	1.104	0	0	\\
3.15e-09	1.104	0	0	\\
3.2e-09	1.104	0	0	\\
3.25e-09	1.104	0	0	\\
3.3e-09	1.104	0	0	\\
3.35e-09	1.104	0	0	\\
3.4e-09	1.104	0	0	\\
3.45e-09	1.104	0	0	\\
3.5e-09	1.104	0	0	\\
3.55e-09	1.104	0.666666666666667	0.666666666666667	\\
3.6e-09	1.104	0.666666666666667	0.666666666666667	\\
3.65e-09	1.104	0.666666666666667	0.666666666666667	\\
3.7e-09	1.104	0.222222222222222	0.222222222222222	\\
3.75e-09	1.104	0.222222222222222	0.222222222222222	\\
3.8e-09	1.104	0.222222222222222	0.222222222222222	\\
3.85e-09	1.104	0.222222222222222	0.222222222222222	\\
3.9e-09	1.104	0.222222222222222	0.222222222222222	\\
3.95e-09	1.104	0.222222222222222	0.222222222222222	\\
4e-09	1.104	0.222222222222222	0.222222222222222	\\
4.05e-09	1.104	-0.444444444444444	-0.444444444444444	\\
4.1e-09	1.104	-0.444444444444444	-0.444444444444444	\\
4.15e-09	1.104	-0.444444444444444	-0.444444444444444	\\
4.2e-09	1.104	0	0	\\
4.25e-09	1.104	0	0	\\
4.3e-09	1.104	0	0	\\
4.35e-09	1.104	0	0	\\
4.4e-09	1.104	0	0	\\
4.45e-09	1.104	0	0	\\
4.5e-09	1.104	0	0	\\
4.55e-09	1.104	0	0	\\
4.6e-09	1.104	0	0	\\
4.65e-09	1.104	0	0	\\
4.7e-09	1.104	0	0	\\
4.75e-09	1.104	0	0	\\
4.8e-09	1.104	0	0	\\
4.85e-09	1.104	0	0	\\
4.9e-09	1.104	0	0	\\
4.95e-09	1.104	0	0	\\
5e-09	1.104	0	0	\\
5e-09	1.104	0	nan	\\
5e-09	1.104	-0.666666666666667	0.166666666666667	\\
5e-09	1.104	-0.666666666666667	nan	\\
0	1.116	-0.666666666666667	nan	\\
0	1.116	0	0.166666666666667	\\
0	1.116	0	0	\\
5e-11	1.116	0	0	\\
1e-10	1.116	0	0	\\
1.5e-10	1.116	0	0	\\
2e-10	1.116	0	0	\\
2.5e-10	1.116	0	0	\\
3e-10	1.116	0	0	\\
3.5e-10	1.116	0	0	\\
4e-10	1.116	0	0	\\
4.5e-10	1.116	0	0	\\
5e-10	1.116	0	0	\\
5.5e-10	1.116	0	0	\\
6e-10	1.116	0	0	\\
6.5e-10	1.116	0	0	\\
7e-10	1.116	0	0	\\
7.5e-10	1.116	0	0	\\
8e-10	1.116	0	0	\\
8.5e-10	1.116	0	0	\\
9e-10	1.116	0	0	\\
9.5e-10	1.116	0	0	\\
1e-09	1.116	0	0	\\
1.05e-09	1.116	0	0	\\
1.1e-09	1.116	0	0	\\
1.15e-09	1.116	1	1	\\
1.2e-09	1.116	1	1	\\
1.25e-09	1.116	1	1	\\
1.3e-09	1.116	0.333333333333333	0.333333333333333	\\
1.35e-09	1.116	0.333333333333333	0.333333333333333	\\
1.4e-09	1.116	0.333333333333333	0.333333333333333	\\
1.45e-09	1.116	0.333333333333333	0.333333333333333	\\
1.5e-09	1.116	0.333333333333333	0.333333333333333	\\
1.55e-09	1.116	0.333333333333333	0.333333333333333	\\
1.6e-09	1.116	0.333333333333333	0.333333333333333	\\
1.65e-09	1.116	-0.666666666666667	-0.666666666666667	\\
1.7e-09	1.116	-0.666666666666667	-0.666666666666667	\\
1.75e-09	1.116	-0.666666666666667	-0.666666666666667	\\
1.8e-09	1.116	0	0	\\
1.85e-09	1.116	0	0	\\
1.9e-09	1.116	0	0	\\
1.95e-09	1.116	0	0	\\
2e-09	1.116	0	0	\\
2.05e-09	1.116	0	0	\\
2.1e-09	1.116	0	0	\\
2.15e-09	1.116	0	0	\\
2.2e-09	1.116	0	0	\\
2.25e-09	1.116	0	0	\\
2.3e-09	1.116	0	0	\\
2.35e-09	1.116	0	0	\\
2.4e-09	1.116	0	0	\\
2.45e-09	1.116	0	0	\\
2.5e-09	1.116	0	0	\\
2.55e-09	1.116	0	0	\\
2.6e-09	1.116	0	0	\\
2.65e-09	1.116	0	0	\\
2.7e-09	1.116	0	0	\\
2.75e-09	1.116	0	0	\\
2.8e-09	1.116	0	0	\\
2.85e-09	1.116	0	0	\\
2.9e-09	1.116	0	0	\\
2.95e-09	1.116	0	0	\\
3e-09	1.116	0	0	\\
3.05e-09	1.116	0	0	\\
3.1e-09	1.116	0	0	\\
3.15e-09	1.116	0	0	\\
3.2e-09	1.116	0	0	\\
3.25e-09	1.116	0	0	\\
3.3e-09	1.116	0	0	\\
3.35e-09	1.116	0	0	\\
3.4e-09	1.116	0	0	\\
3.45e-09	1.116	0	0	\\
3.5e-09	1.116	0	0	\\
3.55e-09	1.116	0.666666666666667	0.666666666666667	\\
3.6e-09	1.116	0.666666666666667	0.666666666666667	\\
3.65e-09	1.116	0.666666666666667	0.666666666666667	\\
3.7e-09	1.116	0.222222222222222	0.222222222222222	\\
3.75e-09	1.116	0.222222222222222	0.222222222222222	\\
3.8e-09	1.116	0.222222222222222	0.222222222222222	\\
3.85e-09	1.116	0.222222222222222	0.222222222222222	\\
3.9e-09	1.116	0.222222222222222	0.222222222222222	\\
3.95e-09	1.116	0.222222222222222	0.222222222222222	\\
4e-09	1.116	0.222222222222222	0.222222222222222	\\
4.05e-09	1.116	-0.444444444444444	-0.444444444444444	\\
4.1e-09	1.116	-0.444444444444444	-0.444444444444444	\\
4.15e-09	1.116	-0.444444444444444	-0.444444444444444	\\
4.2e-09	1.116	0	0	\\
4.25e-09	1.116	0	0	\\
4.3e-09	1.116	0	0	\\
4.35e-09	1.116	0	0	\\
4.4e-09	1.116	0	0	\\
4.45e-09	1.116	0	0	\\
4.5e-09	1.116	0	0	\\
4.55e-09	1.116	0	0	\\
4.6e-09	1.116	0	0	\\
4.65e-09	1.116	0	0	\\
4.7e-09	1.116	0	0	\\
4.75e-09	1.116	0	0	\\
4.8e-09	1.116	0	0	\\
4.85e-09	1.116	0	0	\\
4.9e-09	1.116	0	0	\\
4.95e-09	1.116	0	0	\\
5e-09	1.116	0	0	\\
5e-09	1.116	0	nan	\\
5e-09	1.116	-0.666666666666667	0.166666666666667	\\
5e-09	1.116	-0.666666666666667	nan	\\
0	1.128	-0.666666666666667	nan	\\
0	1.128	0	0.166666666666667	\\
0	1.128	0	0	\\
5e-11	1.128	0	0	\\
1e-10	1.128	0	0	\\
1.5e-10	1.128	0	0	\\
2e-10	1.128	0	0	\\
2.5e-10	1.128	0	0	\\
3e-10	1.128	0	0	\\
3.5e-10	1.128	0	0	\\
4e-10	1.128	0	0	\\
4.5e-10	1.128	0	0	\\
5e-10	1.128	0	0	\\
5.5e-10	1.128	0	0	\\
6e-10	1.128	0	0	\\
6.5e-10	1.128	0	0	\\
7e-10	1.128	0	0	\\
7.5e-10	1.128	0	0	\\
8e-10	1.128	0	0	\\
8.5e-10	1.128	0	0	\\
9e-10	1.128	0	0	\\
9.5e-10	1.128	0	0	\\
1e-09	1.128	0	0	\\
1.05e-09	1.128	0	0	\\
1.1e-09	1.128	0	0	\\
1.15e-09	1.128	1	1	\\
1.2e-09	1.128	1	1	\\
1.25e-09	1.128	1	1	\\
1.3e-09	1.128	0.333333333333333	0.333333333333333	\\
1.35e-09	1.128	0.333333333333333	0.333333333333333	\\
1.4e-09	1.128	0.333333333333333	0.333333333333333	\\
1.45e-09	1.128	0.333333333333333	0.333333333333333	\\
1.5e-09	1.128	0.333333333333333	0.333333333333333	\\
1.55e-09	1.128	0.333333333333333	0.333333333333333	\\
1.6e-09	1.128	0.333333333333333	0.333333333333333	\\
1.65e-09	1.128	-0.666666666666667	-0.666666666666667	\\
1.7e-09	1.128	-0.666666666666667	-0.666666666666667	\\
1.75e-09	1.128	-0.666666666666667	-0.666666666666667	\\
1.8e-09	1.128	0	0	\\
1.85e-09	1.128	0	0	\\
1.9e-09	1.128	0	0	\\
1.95e-09	1.128	0	0	\\
2e-09	1.128	0	0	\\
2.05e-09	1.128	0	0	\\
2.1e-09	1.128	0	0	\\
2.15e-09	1.128	0	0	\\
2.2e-09	1.128	0	0	\\
2.25e-09	1.128	0	0	\\
2.3e-09	1.128	0	0	\\
2.35e-09	1.128	0	0	\\
2.4e-09	1.128	0	0	\\
2.45e-09	1.128	0	0	\\
2.5e-09	1.128	0	0	\\
2.55e-09	1.128	0	0	\\
2.6e-09	1.128	0	0	\\
2.65e-09	1.128	0	0	\\
2.7e-09	1.128	0	0	\\
2.75e-09	1.128	0	0	\\
2.8e-09	1.128	0	0	\\
2.85e-09	1.128	0	0	\\
2.9e-09	1.128	0	0	\\
2.95e-09	1.128	0	0	\\
3e-09	1.128	0	0	\\
3.05e-09	1.128	0	0	\\
3.1e-09	1.128	0	0	\\
3.15e-09	1.128	0	0	\\
3.2e-09	1.128	0	0	\\
3.25e-09	1.128	0	0	\\
3.3e-09	1.128	0	0	\\
3.35e-09	1.128	0	0	\\
3.4e-09	1.128	0	0	\\
3.45e-09	1.128	0	0	\\
3.5e-09	1.128	0	0	\\
3.55e-09	1.128	0.666666666666667	0.666666666666667	\\
3.6e-09	1.128	0.666666666666667	0.666666666666667	\\
3.65e-09	1.128	0.666666666666667	0.666666666666667	\\
3.7e-09	1.128	0.222222222222222	0.222222222222222	\\
3.75e-09	1.128	0.222222222222222	0.222222222222222	\\
3.8e-09	1.128	0.222222222222222	0.222222222222222	\\
3.85e-09	1.128	0.222222222222222	0.222222222222222	\\
3.9e-09	1.128	0.222222222222222	0.222222222222222	\\
3.95e-09	1.128	0.222222222222222	0.222222222222222	\\
4e-09	1.128	0.222222222222222	0.222222222222222	\\
4.05e-09	1.128	-0.444444444444444	-0.444444444444444	\\
4.1e-09	1.128	-0.444444444444444	-0.444444444444444	\\
4.15e-09	1.128	-0.444444444444444	-0.444444444444444	\\
4.2e-09	1.128	0	0	\\
4.25e-09	1.128	0	0	\\
4.3e-09	1.128	0	0	\\
4.35e-09	1.128	0	0	\\
4.4e-09	1.128	0	0	\\
4.45e-09	1.128	0	0	\\
4.5e-09	1.128	0	0	\\
4.55e-09	1.128	0	0	\\
4.6e-09	1.128	0	0	\\
4.65e-09	1.128	0	0	\\
4.7e-09	1.128	0	0	\\
4.75e-09	1.128	0	0	\\
4.8e-09	1.128	0	0	\\
4.85e-09	1.128	0	0	\\
4.9e-09	1.128	0	0	\\
4.95e-09	1.128	0	0	\\
5e-09	1.128	0	0	\\
5e-09	1.128	0	nan	\\
5e-09	1.128	-0.666666666666667	0.166666666666667	\\
5e-09	1.128	-0.666666666666667	nan	\\
0	1.14	-0.666666666666667	nan	\\
0	1.14	0	0.166666666666667	\\
0	1.14	0	0	\\
5e-11	1.14	0	0	\\
1e-10	1.14	0	0	\\
1.5e-10	1.14	0	0	\\
2e-10	1.14	0	0	\\
2.5e-10	1.14	0	0	\\
3e-10	1.14	0	0	\\
3.5e-10	1.14	0	0	\\
4e-10	1.14	0	0	\\
4.5e-10	1.14	0	0	\\
5e-10	1.14	0	0	\\
5.5e-10	1.14	0	0	\\
6e-10	1.14	0	0	\\
6.5e-10	1.14	0	0	\\
7e-10	1.14	0	0	\\
7.5e-10	1.14	0	0	\\
8e-10	1.14	0	0	\\
8.5e-10	1.14	0	0	\\
9e-10	1.14	0	0	\\
9.5e-10	1.14	0	0	\\
1e-09	1.14	0	0	\\
1.05e-09	1.14	0	0	\\
1.1e-09	1.14	0	0	\\
1.15e-09	1.14	1	1	\\
1.2e-09	1.14	1	1	\\
1.25e-09	1.14	1	1	\\
1.3e-09	1.14	0.333333333333333	0.333333333333333	\\
1.35e-09	1.14	0.333333333333333	0.333333333333333	\\
1.4e-09	1.14	0.333333333333333	0.333333333333333	\\
1.45e-09	1.14	0.333333333333333	0.333333333333333	\\
1.5e-09	1.14	0.333333333333333	0.333333333333333	\\
1.55e-09	1.14	0.333333333333333	0.333333333333333	\\
1.6e-09	1.14	0.333333333333333	0.333333333333333	\\
1.65e-09	1.14	-0.666666666666667	-0.666666666666667	\\
1.7e-09	1.14	-0.666666666666667	-0.666666666666667	\\
1.75e-09	1.14	-0.666666666666667	-0.666666666666667	\\
1.8e-09	1.14	0	0	\\
1.85e-09	1.14	0	0	\\
1.9e-09	1.14	0	0	\\
1.95e-09	1.14	0	0	\\
2e-09	1.14	0	0	\\
2.05e-09	1.14	0	0	\\
2.1e-09	1.14	0	0	\\
2.15e-09	1.14	0	0	\\
2.2e-09	1.14	0	0	\\
2.25e-09	1.14	0	0	\\
2.3e-09	1.14	0	0	\\
2.35e-09	1.14	0	0	\\
2.4e-09	1.14	0	0	\\
2.45e-09	1.14	0	0	\\
2.5e-09	1.14	0	0	\\
2.55e-09	1.14	0	0	\\
2.6e-09	1.14	0	0	\\
2.65e-09	1.14	0	0	\\
2.7e-09	1.14	0	0	\\
2.75e-09	1.14	0	0	\\
2.8e-09	1.14	0	0	\\
2.85e-09	1.14	0	0	\\
2.9e-09	1.14	0	0	\\
2.95e-09	1.14	0	0	\\
3e-09	1.14	0	0	\\
3.05e-09	1.14	0	0	\\
3.1e-09	1.14	0	0	\\
3.15e-09	1.14	0	0	\\
3.2e-09	1.14	0	0	\\
3.25e-09	1.14	0	0	\\
3.3e-09	1.14	0	0	\\
3.35e-09	1.14	0	0	\\
3.4e-09	1.14	0	0	\\
3.45e-09	1.14	0	0	\\
3.5e-09	1.14	0	0	\\
3.55e-09	1.14	0.666666666666667	0.666666666666667	\\
3.6e-09	1.14	0.666666666666667	0.666666666666667	\\
3.65e-09	1.14	0.666666666666667	0.666666666666667	\\
3.7e-09	1.14	0.222222222222222	0.222222222222222	\\
3.75e-09	1.14	0.222222222222222	0.222222222222222	\\
3.8e-09	1.14	0.222222222222222	0.222222222222222	\\
3.85e-09	1.14	0.222222222222222	0.222222222222222	\\
3.9e-09	1.14	0.222222222222222	0.222222222222222	\\
3.95e-09	1.14	0.222222222222222	0.222222222222222	\\
4e-09	1.14	0.222222222222222	0.222222222222222	\\
4.05e-09	1.14	-0.444444444444444	-0.444444444444444	\\
4.1e-09	1.14	-0.444444444444444	-0.444444444444444	\\
4.15e-09	1.14	-0.444444444444444	-0.444444444444444	\\
4.2e-09	1.14	0	0	\\
4.25e-09	1.14	0	0	\\
4.3e-09	1.14	0	0	\\
4.35e-09	1.14	0	0	\\
4.4e-09	1.14	0	0	\\
4.45e-09	1.14	0	0	\\
4.5e-09	1.14	0	0	\\
4.55e-09	1.14	0	0	\\
4.6e-09	1.14	0	0	\\
4.65e-09	1.14	0	0	\\
4.7e-09	1.14	0	0	\\
4.75e-09	1.14	0	0	\\
4.8e-09	1.14	0	0	\\
4.85e-09	1.14	0	0	\\
4.9e-09	1.14	0	0	\\
4.95e-09	1.14	0	0	\\
5e-09	1.14	0	0	\\
5e-09	1.14	0	nan	\\
5e-09	1.14	-0.666666666666667	0.166666666666667	\\
5e-09	1.14	-0.666666666666667	nan	\\
0	1.152	-0.666666666666667	nan	\\
0	1.152	0	0.166666666666667	\\
0	1.152	0	0	\\
5e-11	1.152	0	0	\\
1e-10	1.152	0	0	\\
1.5e-10	1.152	0	0	\\
2e-10	1.152	0	0	\\
2.5e-10	1.152	0	0	\\
3e-10	1.152	0	0	\\
3.5e-10	1.152	0	0	\\
4e-10	1.152	0	0	\\
4.5e-10	1.152	0	0	\\
5e-10	1.152	0	0	\\
5.5e-10	1.152	0	0	\\
6e-10	1.152	0	0	\\
6.5e-10	1.152	0	0	\\
7e-10	1.152	0	0	\\
7.5e-10	1.152	0	0	\\
8e-10	1.152	0	0	\\
8.5e-10	1.152	0	0	\\
9e-10	1.152	0	0	\\
9.5e-10	1.152	0	0	\\
1e-09	1.152	0	0	\\
1.05e-09	1.152	0	0	\\
1.1e-09	1.152	0	0	\\
1.15e-09	1.152	0	0	\\
1.2e-09	1.152	1	1	\\
1.25e-09	1.152	0.333333333333333	0.333333333333333	\\
1.3e-09	1.152	0.333333333333333	0.333333333333333	\\
1.35e-09	1.152	0.333333333333333	0.333333333333333	\\
1.4e-09	1.152	0.333333333333333	0.333333333333333	\\
1.45e-09	1.152	0.333333333333333	0.333333333333333	\\
1.5e-09	1.152	0.333333333333333	0.333333333333333	\\
1.55e-09	1.152	0.333333333333333	0.333333333333333	\\
1.6e-09	1.152	0.333333333333333	0.333333333333333	\\
1.65e-09	1.152	0.333333333333333	0.333333333333333	\\
1.7e-09	1.152	-0.666666666666667	-0.666666666666667	\\
1.75e-09	1.152	0	0	\\
1.8e-09	1.152	0	0	\\
1.85e-09	1.152	0	0	\\
1.9e-09	1.152	0	0	\\
1.95e-09	1.152	0	0	\\
2e-09	1.152	0	0	\\
2.05e-09	1.152	0	0	\\
2.1e-09	1.152	0	0	\\
2.15e-09	1.152	0	0	\\
2.2e-09	1.152	0	0	\\
2.25e-09	1.152	0	0	\\
2.3e-09	1.152	0	0	\\
2.35e-09	1.152	0	0	\\
2.4e-09	1.152	0	0	\\
2.45e-09	1.152	0	0	\\
2.5e-09	1.152	0	0	\\
2.55e-09	1.152	0	0	\\
2.6e-09	1.152	0	0	\\
2.65e-09	1.152	0	0	\\
2.7e-09	1.152	0	0	\\
2.75e-09	1.152	0	0	\\
2.8e-09	1.152	0	0	\\
2.85e-09	1.152	0	0	\\
2.9e-09	1.152	0	0	\\
2.95e-09	1.152	0	0	\\
3e-09	1.152	0	0	\\
3.05e-09	1.152	0	0	\\
3.1e-09	1.152	0	0	\\
3.15e-09	1.152	0	0	\\
3.2e-09	1.152	0	0	\\
3.25e-09	1.152	0	0	\\
3.3e-09	1.152	0	0	\\
3.35e-09	1.152	0	0	\\
3.4e-09	1.152	0	0	\\
3.45e-09	1.152	0	0	\\
3.5e-09	1.152	0	0	\\
3.55e-09	1.152	0	0	\\
3.6e-09	1.152	0.666666666666667	0.666666666666667	\\
3.65e-09	1.152	0.222222222222222	0.222222222222222	\\
3.7e-09	1.152	0.222222222222222	0.222222222222222	\\
3.75e-09	1.152	0.222222222222222	0.222222222222222	\\
3.8e-09	1.152	0.222222222222222	0.222222222222222	\\
3.85e-09	1.152	0.222222222222222	0.222222222222222	\\
3.9e-09	1.152	0.222222222222222	0.222222222222222	\\
3.95e-09	1.152	0.222222222222222	0.222222222222222	\\
4e-09	1.152	0.222222222222222	0.222222222222222	\\
4.05e-09	1.152	0.222222222222222	0.222222222222222	\\
4.1e-09	1.152	-0.444444444444444	-0.444444444444444	\\
4.15e-09	1.152	0	0	\\
4.2e-09	1.152	0	0	\\
4.25e-09	1.152	0	0	\\
4.3e-09	1.152	0	0	\\
4.35e-09	1.152	0	0	\\
4.4e-09	1.152	0	0	\\
4.45e-09	1.152	0	0	\\
4.5e-09	1.152	0	0	\\
4.55e-09	1.152	0	0	\\
4.6e-09	1.152	0	0	\\
4.65e-09	1.152	0	0	\\
4.7e-09	1.152	0	0	\\
4.75e-09	1.152	0	0	\\
4.8e-09	1.152	0	0	\\
4.85e-09	1.152	0	0	\\
4.9e-09	1.152	0	0	\\
4.95e-09	1.152	0	0	\\
5e-09	1.152	0	0	\\
5e-09	1.152	0	nan	\\
5e-09	1.152	-0.666666666666667	0.166666666666667	\\
5e-09	1.152	-0.666666666666667	nan	\\
0	1.164	-0.666666666666667	nan	\\
0	1.164	0	0.166666666666667	\\
0	1.164	0	0	\\
5e-11	1.164	0	0	\\
1e-10	1.164	0	0	\\
1.5e-10	1.164	0	0	\\
2e-10	1.164	0	0	\\
2.5e-10	1.164	0	0	\\
3e-10	1.164	0	0	\\
3.5e-10	1.164	0	0	\\
4e-10	1.164	0	0	\\
4.5e-10	1.164	0	0	\\
5e-10	1.164	0	0	\\
5.5e-10	1.164	0	0	\\
6e-10	1.164	0	0	\\
6.5e-10	1.164	0	0	\\
7e-10	1.164	0	0	\\
7.5e-10	1.164	0	0	\\
8e-10	1.164	0	0	\\
8.5e-10	1.164	0	0	\\
9e-10	1.164	0	0	\\
9.5e-10	1.164	0	0	\\
1e-09	1.164	0	0	\\
1.05e-09	1.164	0	0	\\
1.1e-09	1.164	0	0	\\
1.15e-09	1.164	0	0	\\
1.2e-09	1.164	1	1	\\
1.25e-09	1.164	0.333333333333333	0.333333333333333	\\
1.3e-09	1.164	0.333333333333333	0.333333333333333	\\
1.35e-09	1.164	0.333333333333333	0.333333333333333	\\
1.4e-09	1.164	0.333333333333333	0.333333333333333	\\
1.45e-09	1.164	0.333333333333333	0.333333333333333	\\
1.5e-09	1.164	0.333333333333333	0.333333333333333	\\
1.55e-09	1.164	0.333333333333333	0.333333333333333	\\
1.6e-09	1.164	0.333333333333333	0.333333333333333	\\
1.65e-09	1.164	0.333333333333333	0.333333333333333	\\
1.7e-09	1.164	-0.666666666666667	-0.666666666666667	\\
1.75e-09	1.164	0	0	\\
1.8e-09	1.164	0	0	\\
1.85e-09	1.164	0	0	\\
1.9e-09	1.164	0	0	\\
1.95e-09	1.164	0	0	\\
2e-09	1.164	0	0	\\
2.05e-09	1.164	0	0	\\
2.1e-09	1.164	0	0	\\
2.15e-09	1.164	0	0	\\
2.2e-09	1.164	0	0	\\
2.25e-09	1.164	0	0	\\
2.3e-09	1.164	0	0	\\
2.35e-09	1.164	0	0	\\
2.4e-09	1.164	0	0	\\
2.45e-09	1.164	0	0	\\
2.5e-09	1.164	0	0	\\
2.55e-09	1.164	0	0	\\
2.6e-09	1.164	0	0	\\
2.65e-09	1.164	0	0	\\
2.7e-09	1.164	0	0	\\
2.75e-09	1.164	0	0	\\
2.8e-09	1.164	0	0	\\
2.85e-09	1.164	0	0	\\
2.9e-09	1.164	0	0	\\
2.95e-09	1.164	0	0	\\
3e-09	1.164	0	0	\\
3.05e-09	1.164	0	0	\\
3.1e-09	1.164	0	0	\\
3.15e-09	1.164	0	0	\\
3.2e-09	1.164	0	0	\\
3.25e-09	1.164	0	0	\\
3.3e-09	1.164	0	0	\\
3.35e-09	1.164	0	0	\\
3.4e-09	1.164	0	0	\\
3.45e-09	1.164	0	0	\\
3.5e-09	1.164	0	0	\\
3.55e-09	1.164	0	0	\\
3.6e-09	1.164	0.666666666666667	0.666666666666667	\\
3.65e-09	1.164	0.222222222222222	0.222222222222222	\\
3.7e-09	1.164	0.222222222222222	0.222222222222222	\\
3.75e-09	1.164	0.222222222222222	0.222222222222222	\\
3.8e-09	1.164	0.222222222222222	0.222222222222222	\\
3.85e-09	1.164	0.222222222222222	0.222222222222222	\\
3.9e-09	1.164	0.222222222222222	0.222222222222222	\\
3.95e-09	1.164	0.222222222222222	0.222222222222222	\\
4e-09	1.164	0.222222222222222	0.222222222222222	\\
4.05e-09	1.164	0.222222222222222	0.222222222222222	\\
4.1e-09	1.164	-0.444444444444444	-0.444444444444444	\\
4.15e-09	1.164	0	0	\\
4.2e-09	1.164	0	0	\\
4.25e-09	1.164	0	0	\\
4.3e-09	1.164	0	0	\\
4.35e-09	1.164	0	0	\\
4.4e-09	1.164	0	0	\\
4.45e-09	1.164	0	0	\\
4.5e-09	1.164	0	0	\\
4.55e-09	1.164	0	0	\\
4.6e-09	1.164	0	0	\\
4.65e-09	1.164	0	0	\\
4.7e-09	1.164	0	0	\\
4.75e-09	1.164	0	0	\\
4.8e-09	1.164	0	0	\\
4.85e-09	1.164	0	0	\\
4.9e-09	1.164	0	0	\\
4.95e-09	1.164	0	0	\\
5e-09	1.164	0	0	\\
5e-09	1.164	0	nan	\\
5e-09	1.164	-0.666666666666667	0.166666666666667	\\
5e-09	1.164	-0.666666666666667	nan	\\
0	1.176	-0.666666666666667	nan	\\
0	1.176	0	0.166666666666667	\\
0	1.176	0	0	\\
5e-11	1.176	0	0	\\
1e-10	1.176	0	0	\\
1.5e-10	1.176	0	0	\\
2e-10	1.176	0	0	\\
2.5e-10	1.176	0	0	\\
3e-10	1.176	0	0	\\
3.5e-10	1.176	0	0	\\
4e-10	1.176	0	0	\\
4.5e-10	1.176	0	0	\\
5e-10	1.176	0	0	\\
5.5e-10	1.176	0	0	\\
6e-10	1.176	0	0	\\
6.5e-10	1.176	0	0	\\
7e-10	1.176	0	0	\\
7.5e-10	1.176	0	0	\\
8e-10	1.176	0	0	\\
8.5e-10	1.176	0	0	\\
9e-10	1.176	0	0	\\
9.5e-10	1.176	0	0	\\
1e-09	1.176	0	0	\\
1.05e-09	1.176	0	0	\\
1.1e-09	1.176	0	0	\\
1.15e-09	1.176	0	0	\\
1.2e-09	1.176	1	1	\\
1.25e-09	1.176	0.333333333333333	0.333333333333333	\\
1.3e-09	1.176	0.333333333333333	0.333333333333333	\\
1.35e-09	1.176	0.333333333333333	0.333333333333333	\\
1.4e-09	1.176	0.333333333333333	0.333333333333333	\\
1.45e-09	1.176	0.333333333333333	0.333333333333333	\\
1.5e-09	1.176	0.333333333333333	0.333333333333333	\\
1.55e-09	1.176	0.333333333333333	0.333333333333333	\\
1.6e-09	1.176	0.333333333333333	0.333333333333333	\\
1.65e-09	1.176	0.333333333333333	0.333333333333333	\\
1.7e-09	1.176	-0.666666666666667	-0.666666666666667	\\
1.75e-09	1.176	0	0	\\
1.8e-09	1.176	0	0	\\
1.85e-09	1.176	0	0	\\
1.9e-09	1.176	0	0	\\
1.95e-09	1.176	0	0	\\
2e-09	1.176	0	0	\\
2.05e-09	1.176	0	0	\\
2.1e-09	1.176	0	0	\\
2.15e-09	1.176	0	0	\\
2.2e-09	1.176	0	0	\\
2.25e-09	1.176	0	0	\\
2.3e-09	1.176	0	0	\\
2.35e-09	1.176	0	0	\\
2.4e-09	1.176	0	0	\\
2.45e-09	1.176	0	0	\\
2.5e-09	1.176	0	0	\\
2.55e-09	1.176	0	0	\\
2.6e-09	1.176	0	0	\\
2.65e-09	1.176	0	0	\\
2.7e-09	1.176	0	0	\\
2.75e-09	1.176	0	0	\\
2.8e-09	1.176	0	0	\\
2.85e-09	1.176	0	0	\\
2.9e-09	1.176	0	0	\\
2.95e-09	1.176	0	0	\\
3e-09	1.176	0	0	\\
3.05e-09	1.176	0	0	\\
3.1e-09	1.176	0	0	\\
3.15e-09	1.176	0	0	\\
3.2e-09	1.176	0	0	\\
3.25e-09	1.176	0	0	\\
3.3e-09	1.176	0	0	\\
3.35e-09	1.176	0	0	\\
3.4e-09	1.176	0	0	\\
3.45e-09	1.176	0	0	\\
3.5e-09	1.176	0	0	\\
3.55e-09	1.176	0	0	\\
3.6e-09	1.176	0.666666666666667	0.666666666666667	\\
3.65e-09	1.176	0.222222222222222	0.222222222222222	\\
3.7e-09	1.176	0.222222222222222	0.222222222222222	\\
3.75e-09	1.176	0.222222222222222	0.222222222222222	\\
3.8e-09	1.176	0.222222222222222	0.222222222222222	\\
3.85e-09	1.176	0.222222222222222	0.222222222222222	\\
3.9e-09	1.176	0.222222222222222	0.222222222222222	\\
3.95e-09	1.176	0.222222222222222	0.222222222222222	\\
4e-09	1.176	0.222222222222222	0.222222222222222	\\
4.05e-09	1.176	0.222222222222222	0.222222222222222	\\
4.1e-09	1.176	-0.444444444444444	-0.444444444444444	\\
4.15e-09	1.176	0	0	\\
4.2e-09	1.176	0	0	\\
4.25e-09	1.176	0	0	\\
4.3e-09	1.176	0	0	\\
4.35e-09	1.176	0	0	\\
4.4e-09	1.176	0	0	\\
4.45e-09	1.176	0	0	\\
4.5e-09	1.176	0	0	\\
4.55e-09	1.176	0	0	\\
4.6e-09	1.176	0	0	\\
4.65e-09	1.176	0	0	\\
4.7e-09	1.176	0	0	\\
4.75e-09	1.176	0	0	\\
4.8e-09	1.176	0	0	\\
4.85e-09	1.176	0	0	\\
4.9e-09	1.176	0	0	\\
4.95e-09	1.176	0	0	\\
5e-09	1.176	0	0	\\
5e-09	1.176	0	nan	\\
5e-09	1.176	-0.666666666666667	0.166666666666667	\\
5e-09	1.176	-0.666666666666667	nan	\\
0	1.188	-0.666666666666667	nan	\\
0	1.188	0	0.166666666666667	\\
0	1.188	0	0	\\
5e-11	1.188	0	0	\\
1e-10	1.188	0	0	\\
1.5e-10	1.188	0	0	\\
2e-10	1.188	0	0	\\
2.5e-10	1.188	0	0	\\
3e-10	1.188	0	0	\\
3.5e-10	1.188	0	0	\\
4e-10	1.188	0	0	\\
4.5e-10	1.188	0	0	\\
5e-10	1.188	0	0	\\
5.5e-10	1.188	0	0	\\
6e-10	1.188	0	0	\\
6.5e-10	1.188	0	0	\\
7e-10	1.188	0	0	\\
7.5e-10	1.188	0	0	\\
8e-10	1.188	0	0	\\
8.5e-10	1.188	0	0	\\
9e-10	1.188	0	0	\\
9.5e-10	1.188	0	0	\\
1e-09	1.188	0	0	\\
1.05e-09	1.188	0	0	\\
1.1e-09	1.188	0	0	\\
1.15e-09	1.188	0	0	\\
1.2e-09	1.188	1	1	\\
1.25e-09	1.188	0.333333333333333	0.333333333333333	\\
1.3e-09	1.188	0.333333333333333	0.333333333333333	\\
1.35e-09	1.188	0.333333333333333	0.333333333333333	\\
1.4e-09	1.188	0.333333333333333	0.333333333333333	\\
1.45e-09	1.188	0.333333333333333	0.333333333333333	\\
1.5e-09	1.188	0.333333333333333	0.333333333333333	\\
1.55e-09	1.188	0.333333333333333	0.333333333333333	\\
1.6e-09	1.188	0.333333333333333	0.333333333333333	\\
1.65e-09	1.188	0.333333333333333	0.333333333333333	\\
1.7e-09	1.188	-0.666666666666667	-0.666666666666667	\\
1.75e-09	1.188	0	0	\\
1.8e-09	1.188	0	0	\\
1.85e-09	1.188	0	0	\\
1.9e-09	1.188	0	0	\\
1.95e-09	1.188	0	0	\\
2e-09	1.188	0	0	\\
2.05e-09	1.188	0	0	\\
2.1e-09	1.188	0	0	\\
2.15e-09	1.188	0	0	\\
2.2e-09	1.188	0	0	\\
2.25e-09	1.188	0	0	\\
2.3e-09	1.188	0	0	\\
2.35e-09	1.188	0	0	\\
2.4e-09	1.188	0	0	\\
2.45e-09	1.188	0	0	\\
2.5e-09	1.188	0	0	\\
2.55e-09	1.188	0	0	\\
2.6e-09	1.188	0	0	\\
2.65e-09	1.188	0	0	\\
2.7e-09	1.188	0	0	\\
2.75e-09	1.188	0	0	\\
2.8e-09	1.188	0	0	\\
2.85e-09	1.188	0	0	\\
2.9e-09	1.188	0	0	\\
2.95e-09	1.188	0	0	\\
3e-09	1.188	0	0	\\
3.05e-09	1.188	0	0	\\
3.1e-09	1.188	0	0	\\
3.15e-09	1.188	0	0	\\
3.2e-09	1.188	0	0	\\
3.25e-09	1.188	0	0	\\
3.3e-09	1.188	0	0	\\
3.35e-09	1.188	0	0	\\
3.4e-09	1.188	0	0	\\
3.45e-09	1.188	0	0	\\
3.5e-09	1.188	0	0	\\
3.55e-09	1.188	0	0	\\
3.6e-09	1.188	0.666666666666667	0.666666666666667	\\
3.65e-09	1.188	0.222222222222222	0.222222222222222	\\
3.7e-09	1.188	0.222222222222222	0.222222222222222	\\
3.75e-09	1.188	0.222222222222222	0.222222222222222	\\
3.8e-09	1.188	0.222222222222222	0.222222222222222	\\
3.85e-09	1.188	0.222222222222222	0.222222222222222	\\
3.9e-09	1.188	0.222222222222222	0.222222222222222	\\
3.95e-09	1.188	0.222222222222222	0.222222222222222	\\
4e-09	1.188	0.222222222222222	0.222222222222222	\\
4.05e-09	1.188	0.222222222222222	0.222222222222222	\\
4.1e-09	1.188	-0.444444444444444	-0.444444444444444	\\
4.15e-09	1.188	0	0	\\
4.2e-09	1.188	0	0	\\
4.25e-09	1.188	0	0	\\
4.3e-09	1.188	0	0	\\
4.35e-09	1.188	0	0	\\
4.4e-09	1.188	0	0	\\
4.45e-09	1.188	0	0	\\
4.5e-09	1.188	0	0	\\
4.55e-09	1.188	0	0	\\
4.6e-09	1.188	0	0	\\
4.65e-09	1.188	0	0	\\
4.7e-09	1.188	0	0	\\
4.75e-09	1.188	0	0	\\
4.8e-09	1.188	0	0	\\
4.85e-09	1.188	0	0	\\
4.9e-09	1.188	0	0	\\
4.95e-09	1.188	0	0	\\
5e-09	1.188	0	0	\\
5e-09	1.188	0	nan	\\
5e-09	1.188	-0.666666666666667	0.166666666666667	\\
5e-09	1.188	-0.666666666666667	nan	\\
0	1.2	-0.666666666666667	nan	\\
0	1.2	0	0.166666666666667	\\
0	1.2	0	0	\\
5e-11	1.2	0	0	\\
1e-10	1.2	0	0	\\
1.5e-10	1.2	0	0	\\
2e-10	1.2	0	0	\\
2.5e-10	1.2	0	0	\\
3e-10	1.2	0	0	\\
3.5e-10	1.2	0	0	\\
4e-10	1.2	0	0	\\
4.5e-10	1.2	0	0	\\
5e-10	1.2	0	0	\\
5.5e-10	1.2	0	0	\\
6e-10	1.2	0	0	\\
6.5e-10	1.2	0	0	\\
7e-10	1.2	0	0	\\
7.5e-10	1.2	0	0	\\
8e-10	1.2	0	0	\\
8.5e-10	1.2	0	0	\\
9e-10	1.2	0	0	\\
9.5e-10	1.2	0	0	\\
1e-09	1.2	0	0	\\
1.05e-09	1.2	0	0	\\
1.1e-09	1.2	0	0	\\
1.15e-09	1.2	0	0	\\
1.2e-09	1.2	0.333333333333333	0.333333333333333	\\
1.25e-09	1.2	0.333333333333333	0.333333333333333	\\
1.3e-09	1.2	0.333333333333333	0.333333333333333	\\
1.35e-09	1.2	0.333333333333333	0.333333333333333	\\
1.4e-09	1.2	0.333333333333333	0.333333333333333	\\
1.45e-09	1.2	0.333333333333333	0.333333333333333	\\
1.5e-09	1.2	0.333333333333333	0.333333333333333	\\
1.55e-09	1.2	0.333333333333333	0.333333333333333	\\
1.6e-09	1.2	0.333333333333333	0.333333333333333	\\
1.65e-09	1.2	0.333333333333333	0.333333333333333	\\
1.7e-09	1.2	0	0	\\
1.75e-09	1.2	0	0	\\
1.8e-09	1.2	0	0	\\
1.85e-09	1.2	0	0	\\
1.9e-09	1.2	0	0	\\
1.95e-09	1.2	0	0	\\
2e-09	1.2	0	0	\\
2.05e-09	1.2	0	0	\\
2.1e-09	1.2	0	0	\\
2.15e-09	1.2	0	0	\\
2.2e-09	1.2	0	0	\\
2.25e-09	1.2	0	0	\\
2.3e-09	1.2	0	0	\\
2.35e-09	1.2	0	0	\\
2.4e-09	1.2	0	0	\\
2.45e-09	1.2	0	0	\\
2.5e-09	1.2	0	0	\\
2.55e-09	1.2	0	0	\\
2.6e-09	1.2	0	0	\\
2.65e-09	1.2	0	0	\\
2.7e-09	1.2	0	0	\\
2.75e-09	1.2	0	0	\\
2.8e-09	1.2	0	0	\\
2.85e-09	1.2	0	0	\\
2.9e-09	1.2	0	0	\\
2.95e-09	1.2	0	0	\\
3e-09	1.2	0	0	\\
3.05e-09	1.2	0	0	\\
3.1e-09	1.2	0	0	\\
3.15e-09	1.2	0	0	\\
3.2e-09	1.2	0	0	\\
3.25e-09	1.2	0	0	\\
3.3e-09	1.2	0	0	\\
3.35e-09	1.2	0	0	\\
3.4e-09	1.2	0	0	\\
3.45e-09	1.2	0	0	\\
3.5e-09	1.2	0	0	\\
3.55e-09	1.2	0	0	\\
3.6e-09	1.2	0.222222222222222	0.222222222222222	\\
3.65e-09	1.2	0.222222222222222	0.222222222222222	\\
3.7e-09	1.2	0.222222222222222	0.222222222222222	\\
3.75e-09	1.2	0.222222222222222	0.222222222222222	\\
3.8e-09	1.2	0.222222222222222	0.222222222222222	\\
3.85e-09	1.2	0.222222222222222	0.222222222222222	\\
3.9e-09	1.2	0.222222222222222	0.222222222222222	\\
3.95e-09	1.2	0.222222222222222	0.222222222222222	\\
4e-09	1.2	0.222222222222222	0.222222222222222	\\
4.05e-09	1.2	0.222222222222222	0.222222222222222	\\
4.1e-09	1.2	0	0	\\
4.15e-09	1.2	0	0	\\
4.2e-09	1.2	0	0	\\
4.25e-09	1.2	0	0	\\
4.3e-09	1.2	0	0	\\
4.35e-09	1.2	0	0	\\
4.4e-09	1.2	0	0	\\
4.45e-09	1.2	0	0	\\
4.5e-09	1.2	0	0	\\
4.55e-09	1.2	0	0	\\
4.6e-09	1.2	0	0	\\
4.65e-09	1.2	0	0	\\
4.7e-09	1.2	0	0	\\
4.75e-09	1.2	0	0	\\
4.8e-09	1.2	0	0	\\
4.85e-09	1.2	0	0	\\
4.9e-09	1.2	0	0	\\
4.95e-09	1.2	0	0	\\
5e-09	1.2	0	0	\\
5e-09	1.2	0	nan	\\
5e-09	1.2	-0.666666666666667	0.166666666666667	\\
5e-09	1.2	-0.666666666666667	nan	\\
};

\end{axis}
\end{tikzpicture}%
	\label{fig:wheel-torque-labcar}
	\caption{Voltage on the transmission channel with rectangular pulse.}
\end{figure}

\begin{figure}[H]
	\centering
	\setlength\figureheight{0.8\linewidth}
    	\setlength\figurewidth{0.8\linewidth}
	% This file was created by matlab2tikz v0.4.6 running on MATLAB 8.2.
% Copyright (c) 2008--2014, Nico Schlömer <nico.schloemer@gmail.com>
% All rights reserved.
% Minimal pgfplots version: 1.3
% 
% The latest updates can be retrieved from
%   http://www.mathworks.com/matlabcentral/fileexchange/22022-matlab2tikz
% where you can also make suggestions and rate matlab2tikz.
% 
\begin{tikzpicture}

\begin{axis}[%
width=\figurewidth,
height=\figureheight,
colormap={mymap}{[1pt] rgb(0pt)=(0.063,0.545,0.851); rgb(100pt)=(0.6,0.6,0.6); rgb(200pt)=(0.969,0.035,0.243)},
view={20}{45},
scale only axis,
xmin=0,
xmax=5e-09,
xlabel={t (s)},
xmajorgrids,
ymin=0,
ymax=1.2,
ylabel={x (m)},
ymajorgrids,
zmin=-0.0105,
zmax=0.0105,
zlabel={I (A)},
zmajorgrids
]

\addplot3[area legend,patch,forget plot]
 table[row sep=crcr, point meta=\thisrow{c}] {
x	y	z	c\\
0	0	-0.00666666666666667	nan	\\
0	0	0.005	0.00166666666666667	\\
0	0	0.005	0.005	\\
5e-11	0	0.01	0.01	\\
1e-10	0	0.01	0.01	\\
1.5e-10	0	0.01	0.01	\\
2e-10	0	0.01	0.01	\\
2.5e-10	0	0.01	0.01	\\
3e-10	0	0.01	0.01	\\
3.5e-10	0	0.01	0.01	\\
4e-10	0	0.01	0.01	\\
4.5e-10	0	0.01	0.01	\\
5e-10	0	0.005	0.005	\\
5.5e-10	0	0	0	\\
6e-10	0	0	0	\\
6.5e-10	0	0	0	\\
7e-10	0	0	0	\\
7.5e-10	0	0	0	\\
8e-10	0	0	0	\\
8.5e-10	0	0	0	\\
9e-10	0	0	0	\\
9.5e-10	0	0	0	\\
1e-09	0	0	0	\\
1.05e-09	0	0	0	\\
1.1e-09	0	0	0	\\
1.15e-09	0	0	0	\\
1.2e-09	0	0	0	\\
1.25e-09	0	0	0	\\
1.3e-09	0	0	0	\\
1.35e-09	0	0	0	\\
1.4e-09	0	0	0	\\
1.45e-09	0	0	0	\\
1.5e-09	0	0	0	\\
1.55e-09	0	0	0	\\
1.6e-09	0	0	0	\\
1.65e-09	0	0	0	\\
1.7e-09	0	0	0	\\
1.75e-09	0	0	0	\\
1.8e-09	0	0	0	\\
1.85e-09	0	0	0	\\
1.9e-09	0	0	0	\\
1.95e-09	0	0	0	\\
2e-09	0	0	0	\\
2.05e-09	0	0	0	\\
2.1e-09	0	0	0	\\
2.15e-09	0	0	0	\\
2.2e-09	0	0	0	\\
2.25e-09	0	0	0	\\
2.3e-09	0	0	0	\\
2.35e-09	0	0	0	\\
2.4e-09	0	0	0	\\
2.45e-09	0	0	0	\\
2.5e-09	0	0	0	\\
2.55e-09	0	0	0	\\
2.6e-09	0	0	0	\\
2.65e-09	0	0	0	\\
2.7e-09	0	0	0	\\
2.75e-09	0	0	0	\\
2.8e-09	0	0	0	\\
2.85e-09	0	0	0	\\
2.9e-09	0	0	0	\\
2.95e-09	0	0	0	\\
3e-09	0	0	0	\\
3.05e-09	0	0	0	\\
3.1e-09	0	0	0	\\
3.15e-09	0	0	0	\\
3.2e-09	0	0	0	\\
3.25e-09	0	0	0	\\
3.3e-09	0	0	0	\\
3.35e-09	0	0	0	\\
3.4e-09	0	0	0	\\
3.45e-09	0	0	0	\\
3.5e-09	0	0	0	\\
3.55e-09	0	0	0	\\
3.6e-09	0	0	0	\\
3.65e-09	0	0	0	\\
3.7e-09	0	0	0	\\
3.75e-09	0	0	0	\\
3.8e-09	0	0	0	\\
3.85e-09	0	0	0	\\
3.9e-09	0	0	0	\\
3.95e-09	0	0	0	\\
4e-09	0	0	0	\\
4.05e-09	0	0	0	\\
4.1e-09	0	0	0	\\
4.15e-09	0	0	0	\\
4.2e-09	0	0	0	\\
4.25e-09	0	0	0	\\
4.3e-09	0	0	0	\\
4.35e-09	0	0	0	\\
4.4e-09	0	0	0	\\
4.45e-09	0	0	0	\\
4.5e-09	0	0	0	\\
4.55e-09	0	0	0	\\
4.6e-09	0	0	0	\\
4.65e-09	0	0	0	\\
4.7e-09	0	0	0	\\
4.75e-09	0	0	0	\\
4.8e-09	0	0	0	\\
4.85e-09	0	0	0	\\
4.9e-09	0	0	0	\\
4.95e-09	0	0	0	\\
5e-09	0	0	0	\\
5e-09	0	0	nan	\\
5e-09	0	-0.00666666666666667	0.00166666666666667	\\
5e-09	0	-0.00666666666666667	nan	\\
0	0.012	-0.00666666666666667	nan	\\
0	0.012	0	0.00166666666666667	\\
0	0.012	0	0	\\
5e-11	0.012	0.01	0.01	\\
1e-10	0.012	0.01	0.01	\\
1.5e-10	0.012	0.01	0.01	\\
2e-10	0.012	0.01	0.01	\\
2.5e-10	0.012	0.01	0.01	\\
3e-10	0.012	0.01	0.01	\\
3.5e-10	0.012	0.01	0.01	\\
4e-10	0.012	0.01	0.01	\\
4.5e-10	0.012	0.01	0.01	\\
5e-10	0.012	0.01	0.01	\\
5.5e-10	0.012	0	0	\\
6e-10	0.012	0	0	\\
6.5e-10	0.012	0	0	\\
7e-10	0.012	0	0	\\
7.5e-10	0.012	0	0	\\
8e-10	0.012	0	0	\\
8.5e-10	0.012	0	0	\\
9e-10	0.012	0	0	\\
9.5e-10	0.012	0	0	\\
1e-09	0.012	0	0	\\
1.05e-09	0.012	0	0	\\
1.1e-09	0.012	0	0	\\
1.15e-09	0.012	0	0	\\
1.2e-09	0.012	0	0	\\
1.25e-09	0.012	0	0	\\
1.3e-09	0.012	0	0	\\
1.35e-09	0.012	0	0	\\
1.4e-09	0.012	0	0	\\
1.45e-09	0.012	0	0	\\
1.5e-09	0.012	0	0	\\
1.55e-09	0.012	0	0	\\
1.6e-09	0.012	0	0	\\
1.65e-09	0.012	0	0	\\
1.7e-09	0.012	0	0	\\
1.75e-09	0.012	0	0	\\
1.8e-09	0.012	0	0	\\
1.85e-09	0.012	0	0	\\
1.9e-09	0.012	0	0	\\
1.95e-09	0.012	0	0	\\
2e-09	0.012	0	0	\\
2.05e-09	0.012	0	0	\\
2.1e-09	0.012	0	0	\\
2.15e-09	0.012	0	0	\\
2.2e-09	0.012	0	0	\\
2.25e-09	0.012	0	0	\\
2.3e-09	0.012	0	0	\\
2.35e-09	0.012	0	0	\\
2.4e-09	0.012	-0.00666666666666667	-0.00666666666666667	\\
2.45e-09	0.012	0	0	\\
2.5e-09	0.012	0	0	\\
2.55e-09	0.012	0	0	\\
2.6e-09	0.012	0	0	\\
2.65e-09	0.012	0	0	\\
2.7e-09	0.012	0	0	\\
2.75e-09	0.012	0	0	\\
2.8e-09	0.012	0	0	\\
2.85e-09	0.012	0	0	\\
2.9e-09	0.012	0.00666666666666667	0.00666666666666667	\\
2.95e-09	0.012	0	0	\\
3e-09	0.012	0	0	\\
3.05e-09	0.012	0	0	\\
3.1e-09	0.012	0	0	\\
3.15e-09	0.012	0	0	\\
3.2e-09	0.012	0	0	\\
3.25e-09	0.012	0	0	\\
3.3e-09	0.012	0	0	\\
3.35e-09	0.012	0	0	\\
3.4e-09	0.012	0	0	\\
3.45e-09	0.012	0	0	\\
3.5e-09	0.012	0	0	\\
3.55e-09	0.012	0	0	\\
3.6e-09	0.012	0	0	\\
3.65e-09	0.012	0	0	\\
3.7e-09	0.012	0	0	\\
3.75e-09	0.012	0	0	\\
3.8e-09	0.012	0	0	\\
3.85e-09	0.012	0	0	\\
3.9e-09	0.012	0	0	\\
3.95e-09	0.012	0	0	\\
4e-09	0.012	0	0	\\
4.05e-09	0.012	0	0	\\
4.1e-09	0.012	0	0	\\
4.15e-09	0.012	0	0	\\
4.2e-09	0.012	0	0	\\
4.25e-09	0.012	0	0	\\
4.3e-09	0.012	0	0	\\
4.35e-09	0.012	0	0	\\
4.4e-09	0.012	0	0	\\
4.45e-09	0.012	0	0	\\
4.5e-09	0.012	0	0	\\
4.55e-09	0.012	0	0	\\
4.6e-09	0.012	0	0	\\
4.65e-09	0.012	0	0	\\
4.7e-09	0.012	0	0	\\
4.75e-09	0.012	0	0	\\
4.8e-09	0.012	-0.00444444444444444	-0.00444444444444444	\\
4.85e-09	0.012	0	0	\\
4.9e-09	0.012	0	0	\\
4.95e-09	0.012	0	0	\\
5e-09	0.012	0	0	\\
5e-09	0.012	0	nan	\\
5e-09	0.012	-0.00666666666666667	0.00166666666666667	\\
5e-09	0.012	-0.00666666666666667	nan	\\
0	0.024	-0.00666666666666667	nan	\\
0	0.024	0	0.00166666666666667	\\
0	0.024	0	0	\\
5e-11	0.024	0.01	0.01	\\
1e-10	0.024	0.01	0.01	\\
1.5e-10	0.024	0.01	0.01	\\
2e-10	0.024	0.01	0.01	\\
2.5e-10	0.024	0.01	0.01	\\
3e-10	0.024	0.01	0.01	\\
3.5e-10	0.024	0.01	0.01	\\
4e-10	0.024	0.01	0.01	\\
4.5e-10	0.024	0.01	0.01	\\
5e-10	0.024	0.01	0.01	\\
5.5e-10	0.024	0	0	\\
6e-10	0.024	0	0	\\
6.5e-10	0.024	0	0	\\
7e-10	0.024	0	0	\\
7.5e-10	0.024	0	0	\\
8e-10	0.024	0	0	\\
8.5e-10	0.024	0	0	\\
9e-10	0.024	0	0	\\
9.5e-10	0.024	0	0	\\
1e-09	0.024	0	0	\\
1.05e-09	0.024	0	0	\\
1.1e-09	0.024	0	0	\\
1.15e-09	0.024	0	0	\\
1.2e-09	0.024	0	0	\\
1.25e-09	0.024	0	0	\\
1.3e-09	0.024	0	0	\\
1.35e-09	0.024	0	0	\\
1.4e-09	0.024	0	0	\\
1.45e-09	0.024	0	0	\\
1.5e-09	0.024	0	0	\\
1.55e-09	0.024	0	0	\\
1.6e-09	0.024	0	0	\\
1.65e-09	0.024	0	0	\\
1.7e-09	0.024	0	0	\\
1.75e-09	0.024	0	0	\\
1.8e-09	0.024	0	0	\\
1.85e-09	0.024	0	0	\\
1.9e-09	0.024	0	0	\\
1.95e-09	0.024	0	0	\\
2e-09	0.024	0	0	\\
2.05e-09	0.024	0	0	\\
2.1e-09	0.024	0	0	\\
2.15e-09	0.024	0	0	\\
2.2e-09	0.024	0	0	\\
2.25e-09	0.024	0	0	\\
2.3e-09	0.024	0	0	\\
2.35e-09	0.024	0	0	\\
2.4e-09	0.024	-0.00666666666666667	-0.00666666666666667	\\
2.45e-09	0.024	0	0	\\
2.5e-09	0.024	0	0	\\
2.55e-09	0.024	0	0	\\
2.6e-09	0.024	0	0	\\
2.65e-09	0.024	0	0	\\
2.7e-09	0.024	0	0	\\
2.75e-09	0.024	0	0	\\
2.8e-09	0.024	0	0	\\
2.85e-09	0.024	0	0	\\
2.9e-09	0.024	0.00666666666666667	0.00666666666666667	\\
2.95e-09	0.024	0	0	\\
3e-09	0.024	0	0	\\
3.05e-09	0.024	0	0	\\
3.1e-09	0.024	0	0	\\
3.15e-09	0.024	0	0	\\
3.2e-09	0.024	0	0	\\
3.25e-09	0.024	0	0	\\
3.3e-09	0.024	0	0	\\
3.35e-09	0.024	0	0	\\
3.4e-09	0.024	0	0	\\
3.45e-09	0.024	0	0	\\
3.5e-09	0.024	0	0	\\
3.55e-09	0.024	0	0	\\
3.6e-09	0.024	0	0	\\
3.65e-09	0.024	0	0	\\
3.7e-09	0.024	0	0	\\
3.75e-09	0.024	0	0	\\
3.8e-09	0.024	0	0	\\
3.85e-09	0.024	0	0	\\
3.9e-09	0.024	0	0	\\
3.95e-09	0.024	0	0	\\
4e-09	0.024	0	0	\\
4.05e-09	0.024	0	0	\\
4.1e-09	0.024	0	0	\\
4.15e-09	0.024	0	0	\\
4.2e-09	0.024	0	0	\\
4.25e-09	0.024	0	0	\\
4.3e-09	0.024	0	0	\\
4.35e-09	0.024	0	0	\\
4.4e-09	0.024	0	0	\\
4.45e-09	0.024	0	0	\\
4.5e-09	0.024	0	0	\\
4.55e-09	0.024	0	0	\\
4.6e-09	0.024	0	0	\\
4.65e-09	0.024	0	0	\\
4.7e-09	0.024	0	0	\\
4.75e-09	0.024	0	0	\\
4.8e-09	0.024	-0.00444444444444444	-0.00444444444444444	\\
4.85e-09	0.024	0	0	\\
4.9e-09	0.024	0	0	\\
4.95e-09	0.024	0	0	\\
5e-09	0.024	0	0	\\
5e-09	0.024	0	nan	\\
5e-09	0.024	-0.00666666666666667	0.00166666666666667	\\
5e-09	0.024	-0.00666666666666667	nan	\\
0	0.036	-0.00666666666666667	nan	\\
0	0.036	0	0.00166666666666667	\\
0	0.036	0	0	\\
5e-11	0.036	0.01	0.01	\\
1e-10	0.036	0.01	0.01	\\
1.5e-10	0.036	0.01	0.01	\\
2e-10	0.036	0.01	0.01	\\
2.5e-10	0.036	0.01	0.01	\\
3e-10	0.036	0.01	0.01	\\
3.5e-10	0.036	0.01	0.01	\\
4e-10	0.036	0.01	0.01	\\
4.5e-10	0.036	0.01	0.01	\\
5e-10	0.036	0.01	0.01	\\
5.5e-10	0.036	0	0	\\
6e-10	0.036	0	0	\\
6.5e-10	0.036	0	0	\\
7e-10	0.036	0	0	\\
7.5e-10	0.036	0	0	\\
8e-10	0.036	0	0	\\
8.5e-10	0.036	0	0	\\
9e-10	0.036	0	0	\\
9.5e-10	0.036	0	0	\\
1e-09	0.036	0	0	\\
1.05e-09	0.036	0	0	\\
1.1e-09	0.036	0	0	\\
1.15e-09	0.036	0	0	\\
1.2e-09	0.036	0	0	\\
1.25e-09	0.036	0	0	\\
1.3e-09	0.036	0	0	\\
1.35e-09	0.036	0	0	\\
1.4e-09	0.036	0	0	\\
1.45e-09	0.036	0	0	\\
1.5e-09	0.036	0	0	\\
1.55e-09	0.036	0	0	\\
1.6e-09	0.036	0	0	\\
1.65e-09	0.036	0	0	\\
1.7e-09	0.036	0	0	\\
1.75e-09	0.036	0	0	\\
1.8e-09	0.036	0	0	\\
1.85e-09	0.036	0	0	\\
1.9e-09	0.036	0	0	\\
1.95e-09	0.036	0	0	\\
2e-09	0.036	0	0	\\
2.05e-09	0.036	0	0	\\
2.1e-09	0.036	0	0	\\
2.15e-09	0.036	0	0	\\
2.2e-09	0.036	0	0	\\
2.25e-09	0.036	0	0	\\
2.3e-09	0.036	0	0	\\
2.35e-09	0.036	0	0	\\
2.4e-09	0.036	-0.00666666666666667	-0.00666666666666667	\\
2.45e-09	0.036	0	0	\\
2.5e-09	0.036	0	0	\\
2.55e-09	0.036	0	0	\\
2.6e-09	0.036	0	0	\\
2.65e-09	0.036	0	0	\\
2.7e-09	0.036	0	0	\\
2.75e-09	0.036	0	0	\\
2.8e-09	0.036	0	0	\\
2.85e-09	0.036	0	0	\\
2.9e-09	0.036	0.00666666666666667	0.00666666666666667	\\
2.95e-09	0.036	0	0	\\
3e-09	0.036	0	0	\\
3.05e-09	0.036	0	0	\\
3.1e-09	0.036	0	0	\\
3.15e-09	0.036	0	0	\\
3.2e-09	0.036	0	0	\\
3.25e-09	0.036	0	0	\\
3.3e-09	0.036	0	0	\\
3.35e-09	0.036	0	0	\\
3.4e-09	0.036	0	0	\\
3.45e-09	0.036	0	0	\\
3.5e-09	0.036	0	0	\\
3.55e-09	0.036	0	0	\\
3.6e-09	0.036	0	0	\\
3.65e-09	0.036	0	0	\\
3.7e-09	0.036	0	0	\\
3.75e-09	0.036	0	0	\\
3.8e-09	0.036	0	0	\\
3.85e-09	0.036	0	0	\\
3.9e-09	0.036	0	0	\\
3.95e-09	0.036	0	0	\\
4e-09	0.036	0	0	\\
4.05e-09	0.036	0	0	\\
4.1e-09	0.036	0	0	\\
4.15e-09	0.036	0	0	\\
4.2e-09	0.036	0	0	\\
4.25e-09	0.036	0	0	\\
4.3e-09	0.036	0	0	\\
4.35e-09	0.036	0	0	\\
4.4e-09	0.036	0	0	\\
4.45e-09	0.036	0	0	\\
4.5e-09	0.036	0	0	\\
4.55e-09	0.036	0	0	\\
4.6e-09	0.036	0	0	\\
4.65e-09	0.036	0	0	\\
4.7e-09	0.036	0	0	\\
4.75e-09	0.036	0	0	\\
4.8e-09	0.036	-0.00444444444444444	-0.00444444444444444	\\
4.85e-09	0.036	0	0	\\
4.9e-09	0.036	0	0	\\
4.95e-09	0.036	0	0	\\
5e-09	0.036	0	0	\\
5e-09	0.036	0	nan	\\
5e-09	0.036	-0.00666666666666667	0.00166666666666667	\\
5e-09	0.036	-0.00666666666666667	nan	\\
0	0.048	-0.00666666666666667	nan	\\
0	0.048	0	0.00166666666666667	\\
0	0.048	0	0	\\
5e-11	0.048	0.01	0.01	\\
1e-10	0.048	0.01	0.01	\\
1.5e-10	0.048	0.01	0.01	\\
2e-10	0.048	0.01	0.01	\\
2.5e-10	0.048	0.01	0.01	\\
3e-10	0.048	0.01	0.01	\\
3.5e-10	0.048	0.01	0.01	\\
4e-10	0.048	0.01	0.01	\\
4.5e-10	0.048	0.01	0.01	\\
5e-10	0.048	0.01	0.01	\\
5.5e-10	0.048	0	0	\\
6e-10	0.048	0	0	\\
6.5e-10	0.048	0	0	\\
7e-10	0.048	0	0	\\
7.5e-10	0.048	0	0	\\
8e-10	0.048	0	0	\\
8.5e-10	0.048	0	0	\\
9e-10	0.048	0	0	\\
9.5e-10	0.048	0	0	\\
1e-09	0.048	0	0	\\
1.05e-09	0.048	0	0	\\
1.1e-09	0.048	0	0	\\
1.15e-09	0.048	0	0	\\
1.2e-09	0.048	0	0	\\
1.25e-09	0.048	0	0	\\
1.3e-09	0.048	0	0	\\
1.35e-09	0.048	0	0	\\
1.4e-09	0.048	0	0	\\
1.45e-09	0.048	0	0	\\
1.5e-09	0.048	0	0	\\
1.55e-09	0.048	0	0	\\
1.6e-09	0.048	0	0	\\
1.65e-09	0.048	0	0	\\
1.7e-09	0.048	0	0	\\
1.75e-09	0.048	0	0	\\
1.8e-09	0.048	0	0	\\
1.85e-09	0.048	0	0	\\
1.9e-09	0.048	0	0	\\
1.95e-09	0.048	0	0	\\
2e-09	0.048	0	0	\\
2.05e-09	0.048	0	0	\\
2.1e-09	0.048	0	0	\\
2.15e-09	0.048	0	0	\\
2.2e-09	0.048	0	0	\\
2.25e-09	0.048	0	0	\\
2.3e-09	0.048	0	0	\\
2.35e-09	0.048	0	0	\\
2.4e-09	0.048	-0.00666666666666667	-0.00666666666666667	\\
2.45e-09	0.048	0	0	\\
2.5e-09	0.048	0	0	\\
2.55e-09	0.048	0	0	\\
2.6e-09	0.048	0	0	\\
2.65e-09	0.048	0	0	\\
2.7e-09	0.048	0	0	\\
2.75e-09	0.048	0	0	\\
2.8e-09	0.048	0	0	\\
2.85e-09	0.048	0	0	\\
2.9e-09	0.048	0.00666666666666667	0.00666666666666667	\\
2.95e-09	0.048	0	0	\\
3e-09	0.048	0	0	\\
3.05e-09	0.048	0	0	\\
3.1e-09	0.048	0	0	\\
3.15e-09	0.048	0	0	\\
3.2e-09	0.048	0	0	\\
3.25e-09	0.048	0	0	\\
3.3e-09	0.048	0	0	\\
3.35e-09	0.048	0	0	\\
3.4e-09	0.048	0	0	\\
3.45e-09	0.048	0	0	\\
3.5e-09	0.048	0	0	\\
3.55e-09	0.048	0	0	\\
3.6e-09	0.048	0	0	\\
3.65e-09	0.048	0	0	\\
3.7e-09	0.048	0	0	\\
3.75e-09	0.048	0	0	\\
3.8e-09	0.048	0	0	\\
3.85e-09	0.048	0	0	\\
3.9e-09	0.048	0	0	\\
3.95e-09	0.048	0	0	\\
4e-09	0.048	0	0	\\
4.05e-09	0.048	0	0	\\
4.1e-09	0.048	0	0	\\
4.15e-09	0.048	0	0	\\
4.2e-09	0.048	0	0	\\
4.25e-09	0.048	0	0	\\
4.3e-09	0.048	0	0	\\
4.35e-09	0.048	0	0	\\
4.4e-09	0.048	0	0	\\
4.45e-09	0.048	0	0	\\
4.5e-09	0.048	0	0	\\
4.55e-09	0.048	0	0	\\
4.6e-09	0.048	0	0	\\
4.65e-09	0.048	0	0	\\
4.7e-09	0.048	0	0	\\
4.75e-09	0.048	0	0	\\
4.8e-09	0.048	-0.00444444444444444	-0.00444444444444444	\\
4.85e-09	0.048	0	0	\\
4.9e-09	0.048	0	0	\\
4.95e-09	0.048	0	0	\\
5e-09	0.048	0	0	\\
5e-09	0.048	0	nan	\\
5e-09	0.048	-0.00666666666666667	0.00166666666666667	\\
5e-09	0.048	-0.00666666666666667	nan	\\
0	0.06	-0.00666666666666667	nan	\\
0	0.06	0	0.00166666666666667	\\
0	0.06	0	0	\\
5e-11	0.06	0	0	\\
1e-10	0.06	0.01	0.01	\\
1.5e-10	0.06	0.01	0.01	\\
2e-10	0.06	0.01	0.01	\\
2.5e-10	0.06	0.01	0.01	\\
3e-10	0.06	0.01	0.01	\\
3.5e-10	0.06	0.01	0.01	\\
4e-10	0.06	0.01	0.01	\\
4.5e-10	0.06	0.01	0.01	\\
5e-10	0.06	0.01	0.01	\\
5.5e-10	0.06	0.01	0.01	\\
6e-10	0.06	0	0	\\
6.5e-10	0.06	0	0	\\
7e-10	0.06	0	0	\\
7.5e-10	0.06	0	0	\\
8e-10	0.06	0	0	\\
8.5e-10	0.06	0	0	\\
9e-10	0.06	0	0	\\
9.5e-10	0.06	0	0	\\
1e-09	0.06	0	0	\\
1.05e-09	0.06	0	0	\\
1.1e-09	0.06	0	0	\\
1.15e-09	0.06	0	0	\\
1.2e-09	0.06	0	0	\\
1.25e-09	0.06	0	0	\\
1.3e-09	0.06	0	0	\\
1.35e-09	0.06	0	0	\\
1.4e-09	0.06	0	0	\\
1.45e-09	0.06	0	0	\\
1.5e-09	0.06	0	0	\\
1.55e-09	0.06	0	0	\\
1.6e-09	0.06	0	0	\\
1.65e-09	0.06	0	0	\\
1.7e-09	0.06	0	0	\\
1.75e-09	0.06	0	0	\\
1.8e-09	0.06	0	0	\\
1.85e-09	0.06	0	0	\\
1.9e-09	0.06	0	0	\\
1.95e-09	0.06	0	0	\\
2e-09	0.06	0	0	\\
2.05e-09	0.06	0	0	\\
2.1e-09	0.06	0	0	\\
2.15e-09	0.06	0	0	\\
2.2e-09	0.06	0	0	\\
2.25e-09	0.06	0	0	\\
2.3e-09	0.06	0	0	\\
2.35e-09	0.06	-0.00666666666666667	-0.00666666666666667	\\
2.4e-09	0.06	-0.00666666666666667	-0.00666666666666667	\\
2.45e-09	0.06	-0.00666666666666667	-0.00666666666666667	\\
2.5e-09	0.06	0	0	\\
2.55e-09	0.06	0	0	\\
2.6e-09	0.06	0	0	\\
2.65e-09	0.06	0	0	\\
2.7e-09	0.06	0	0	\\
2.75e-09	0.06	0	0	\\
2.8e-09	0.06	0	0	\\
2.85e-09	0.06	0.00666666666666667	0.00666666666666667	\\
2.9e-09	0.06	0.00666666666666667	0.00666666666666667	\\
2.95e-09	0.06	0.00666666666666667	0.00666666666666667	\\
3e-09	0.06	0	0	\\
3.05e-09	0.06	0	0	\\
3.1e-09	0.06	0	0	\\
3.15e-09	0.06	0	0	\\
3.2e-09	0.06	0	0	\\
3.25e-09	0.06	0	0	\\
3.3e-09	0.06	0	0	\\
3.35e-09	0.06	0	0	\\
3.4e-09	0.06	0	0	\\
3.45e-09	0.06	0	0	\\
3.5e-09	0.06	0	0	\\
3.55e-09	0.06	0	0	\\
3.6e-09	0.06	0	0	\\
3.65e-09	0.06	0	0	\\
3.7e-09	0.06	0	0	\\
3.75e-09	0.06	0	0	\\
3.8e-09	0.06	0	0	\\
3.85e-09	0.06	0	0	\\
3.9e-09	0.06	0	0	\\
3.95e-09	0.06	0	0	\\
4e-09	0.06	0	0	\\
4.05e-09	0.06	0	0	\\
4.1e-09	0.06	0	0	\\
4.15e-09	0.06	0	0	\\
4.2e-09	0.06	0	0	\\
4.25e-09	0.06	0	0	\\
4.3e-09	0.06	0	0	\\
4.35e-09	0.06	0	0	\\
4.4e-09	0.06	0	0	\\
4.45e-09	0.06	0	0	\\
4.5e-09	0.06	0	0	\\
4.55e-09	0.06	0	0	\\
4.6e-09	0.06	0	0	\\
4.65e-09	0.06	0	0	\\
4.7e-09	0.06	0	0	\\
4.75e-09	0.06	-0.00444444444444444	-0.00444444444444444	\\
4.8e-09	0.06	-0.00444444444444444	-0.00444444444444444	\\
4.85e-09	0.06	-0.00444444444444444	-0.00444444444444444	\\
4.9e-09	0.06	0	0	\\
4.95e-09	0.06	0	0	\\
5e-09	0.06	0	0	\\
5e-09	0.06	0	nan	\\
5e-09	0.06	-0.00666666666666667	0.00166666666666667	\\
5e-09	0.06	-0.00666666666666667	nan	\\
0	0.072	-0.00666666666666667	nan	\\
0	0.072	0	0.00166666666666667	\\
0	0.072	0	0	\\
5e-11	0.072	0	0	\\
1e-10	0.072	0.01	0.01	\\
1.5e-10	0.072	0.01	0.01	\\
2e-10	0.072	0.01	0.01	\\
2.5e-10	0.072	0.01	0.01	\\
3e-10	0.072	0.01	0.01	\\
3.5e-10	0.072	0.01	0.01	\\
4e-10	0.072	0.01	0.01	\\
4.5e-10	0.072	0.01	0.01	\\
5e-10	0.072	0.01	0.01	\\
5.5e-10	0.072	0.01	0.01	\\
6e-10	0.072	0	0	\\
6.5e-10	0.072	0	0	\\
7e-10	0.072	0	0	\\
7.5e-10	0.072	0	0	\\
8e-10	0.072	0	0	\\
8.5e-10	0.072	0	0	\\
9e-10	0.072	0	0	\\
9.5e-10	0.072	0	0	\\
1e-09	0.072	0	0	\\
1.05e-09	0.072	0	0	\\
1.1e-09	0.072	0	0	\\
1.15e-09	0.072	0	0	\\
1.2e-09	0.072	0	0	\\
1.25e-09	0.072	0	0	\\
1.3e-09	0.072	0	0	\\
1.35e-09	0.072	0	0	\\
1.4e-09	0.072	0	0	\\
1.45e-09	0.072	0	0	\\
1.5e-09	0.072	0	0	\\
1.55e-09	0.072	0	0	\\
1.6e-09	0.072	0	0	\\
1.65e-09	0.072	0	0	\\
1.7e-09	0.072	0	0	\\
1.75e-09	0.072	0	0	\\
1.8e-09	0.072	0	0	\\
1.85e-09	0.072	0	0	\\
1.9e-09	0.072	0	0	\\
1.95e-09	0.072	0	0	\\
2e-09	0.072	0	0	\\
2.05e-09	0.072	0	0	\\
2.1e-09	0.072	0	0	\\
2.15e-09	0.072	0	0	\\
2.2e-09	0.072	0	0	\\
2.25e-09	0.072	0	0	\\
2.3e-09	0.072	0	0	\\
2.35e-09	0.072	-0.00666666666666667	-0.00666666666666667	\\
2.4e-09	0.072	-0.00666666666666667	-0.00666666666666667	\\
2.45e-09	0.072	-0.00666666666666667	-0.00666666666666667	\\
2.5e-09	0.072	0	0	\\
2.55e-09	0.072	0	0	\\
2.6e-09	0.072	0	0	\\
2.65e-09	0.072	0	0	\\
2.7e-09	0.072	0	0	\\
2.75e-09	0.072	0	0	\\
2.8e-09	0.072	0	0	\\
2.85e-09	0.072	0.00666666666666667	0.00666666666666667	\\
2.9e-09	0.072	0.00666666666666667	0.00666666666666667	\\
2.95e-09	0.072	0.00666666666666667	0.00666666666666667	\\
3e-09	0.072	0	0	\\
3.05e-09	0.072	0	0	\\
3.1e-09	0.072	0	0	\\
3.15e-09	0.072	0	0	\\
3.2e-09	0.072	0	0	\\
3.25e-09	0.072	0	0	\\
3.3e-09	0.072	0	0	\\
3.35e-09	0.072	0	0	\\
3.4e-09	0.072	0	0	\\
3.45e-09	0.072	0	0	\\
3.5e-09	0.072	0	0	\\
3.55e-09	0.072	0	0	\\
3.6e-09	0.072	0	0	\\
3.65e-09	0.072	0	0	\\
3.7e-09	0.072	0	0	\\
3.75e-09	0.072	0	0	\\
3.8e-09	0.072	0	0	\\
3.85e-09	0.072	0	0	\\
3.9e-09	0.072	0	0	\\
3.95e-09	0.072	0	0	\\
4e-09	0.072	0	0	\\
4.05e-09	0.072	0	0	\\
4.1e-09	0.072	0	0	\\
4.15e-09	0.072	0	0	\\
4.2e-09	0.072	0	0	\\
4.25e-09	0.072	0	0	\\
4.3e-09	0.072	0	0	\\
4.35e-09	0.072	0	0	\\
4.4e-09	0.072	0	0	\\
4.45e-09	0.072	0	0	\\
4.5e-09	0.072	0	0	\\
4.55e-09	0.072	0	0	\\
4.6e-09	0.072	0	0	\\
4.65e-09	0.072	0	0	\\
4.7e-09	0.072	0	0	\\
4.75e-09	0.072	-0.00444444444444444	-0.00444444444444444	\\
4.8e-09	0.072	-0.00444444444444444	-0.00444444444444444	\\
4.85e-09	0.072	-0.00444444444444444	-0.00444444444444444	\\
4.9e-09	0.072	0	0	\\
4.95e-09	0.072	0	0	\\
5e-09	0.072	0	0	\\
5e-09	0.072	0	nan	\\
5e-09	0.072	-0.00666666666666667	0.00166666666666667	\\
5e-09	0.072	-0.00666666666666667	nan	\\
0	0.084	-0.00666666666666667	nan	\\
0	0.084	0	0.00166666666666667	\\
0	0.084	0	0	\\
5e-11	0.084	0	0	\\
1e-10	0.084	0.01	0.01	\\
1.5e-10	0.084	0.01	0.01	\\
2e-10	0.084	0.01	0.01	\\
2.5e-10	0.084	0.01	0.01	\\
3e-10	0.084	0.01	0.01	\\
3.5e-10	0.084	0.01	0.01	\\
4e-10	0.084	0.01	0.01	\\
4.5e-10	0.084	0.01	0.01	\\
5e-10	0.084	0.01	0.01	\\
5.5e-10	0.084	0.01	0.01	\\
6e-10	0.084	0	0	\\
6.5e-10	0.084	0	0	\\
7e-10	0.084	0	0	\\
7.5e-10	0.084	0	0	\\
8e-10	0.084	0	0	\\
8.5e-10	0.084	0	0	\\
9e-10	0.084	0	0	\\
9.5e-10	0.084	0	0	\\
1e-09	0.084	0	0	\\
1.05e-09	0.084	0	0	\\
1.1e-09	0.084	0	0	\\
1.15e-09	0.084	0	0	\\
1.2e-09	0.084	0	0	\\
1.25e-09	0.084	0	0	\\
1.3e-09	0.084	0	0	\\
1.35e-09	0.084	0	0	\\
1.4e-09	0.084	0	0	\\
1.45e-09	0.084	0	0	\\
1.5e-09	0.084	0	0	\\
1.55e-09	0.084	0	0	\\
1.6e-09	0.084	0	0	\\
1.65e-09	0.084	0	0	\\
1.7e-09	0.084	0	0	\\
1.75e-09	0.084	0	0	\\
1.8e-09	0.084	0	0	\\
1.85e-09	0.084	0	0	\\
1.9e-09	0.084	0	0	\\
1.95e-09	0.084	0	0	\\
2e-09	0.084	0	0	\\
2.05e-09	0.084	0	0	\\
2.1e-09	0.084	0	0	\\
2.15e-09	0.084	0	0	\\
2.2e-09	0.084	0	0	\\
2.25e-09	0.084	0	0	\\
2.3e-09	0.084	0	0	\\
2.35e-09	0.084	-0.00666666666666667	-0.00666666666666667	\\
2.4e-09	0.084	-0.00666666666666667	-0.00666666666666667	\\
2.45e-09	0.084	-0.00666666666666667	-0.00666666666666667	\\
2.5e-09	0.084	0	0	\\
2.55e-09	0.084	0	0	\\
2.6e-09	0.084	0	0	\\
2.65e-09	0.084	0	0	\\
2.7e-09	0.084	0	0	\\
2.75e-09	0.084	0	0	\\
2.8e-09	0.084	0	0	\\
2.85e-09	0.084	0.00666666666666667	0.00666666666666667	\\
2.9e-09	0.084	0.00666666666666667	0.00666666666666667	\\
2.95e-09	0.084	0.00666666666666667	0.00666666666666667	\\
3e-09	0.084	0	0	\\
3.05e-09	0.084	0	0	\\
3.1e-09	0.084	0	0	\\
3.15e-09	0.084	0	0	\\
3.2e-09	0.084	0	0	\\
3.25e-09	0.084	0	0	\\
3.3e-09	0.084	0	0	\\
3.35e-09	0.084	0	0	\\
3.4e-09	0.084	0	0	\\
3.45e-09	0.084	0	0	\\
3.5e-09	0.084	0	0	\\
3.55e-09	0.084	0	0	\\
3.6e-09	0.084	0	0	\\
3.65e-09	0.084	0	0	\\
3.7e-09	0.084	0	0	\\
3.75e-09	0.084	0	0	\\
3.8e-09	0.084	0	0	\\
3.85e-09	0.084	0	0	\\
3.9e-09	0.084	0	0	\\
3.95e-09	0.084	0	0	\\
4e-09	0.084	0	0	\\
4.05e-09	0.084	0	0	\\
4.1e-09	0.084	0	0	\\
4.15e-09	0.084	0	0	\\
4.2e-09	0.084	0	0	\\
4.25e-09	0.084	0	0	\\
4.3e-09	0.084	0	0	\\
4.35e-09	0.084	0	0	\\
4.4e-09	0.084	0	0	\\
4.45e-09	0.084	0	0	\\
4.5e-09	0.084	0	0	\\
4.55e-09	0.084	0	0	\\
4.6e-09	0.084	0	0	\\
4.65e-09	0.084	0	0	\\
4.7e-09	0.084	0	0	\\
4.75e-09	0.084	-0.00444444444444444	-0.00444444444444444	\\
4.8e-09	0.084	-0.00444444444444444	-0.00444444444444444	\\
4.85e-09	0.084	-0.00444444444444444	-0.00444444444444444	\\
4.9e-09	0.084	0	0	\\
4.95e-09	0.084	0	0	\\
5e-09	0.084	0	0	\\
5e-09	0.084	0	nan	\\
5e-09	0.084	-0.00666666666666667	0.00166666666666667	\\
5e-09	0.084	-0.00666666666666667	nan	\\
0	0.096	-0.00666666666666667	nan	\\
0	0.096	0	0.00166666666666667	\\
0	0.096	0	0	\\
5e-11	0.096	0	0	\\
1e-10	0.096	0.01	0.01	\\
1.5e-10	0.096	0.01	0.01	\\
2e-10	0.096	0.01	0.01	\\
2.5e-10	0.096	0.01	0.01	\\
3e-10	0.096	0.01	0.01	\\
3.5e-10	0.096	0.01	0.01	\\
4e-10	0.096	0.01	0.01	\\
4.5e-10	0.096	0.01	0.01	\\
5e-10	0.096	0.01	0.01	\\
5.5e-10	0.096	0.01	0.01	\\
6e-10	0.096	0	0	\\
6.5e-10	0.096	0	0	\\
7e-10	0.096	0	0	\\
7.5e-10	0.096	0	0	\\
8e-10	0.096	0	0	\\
8.5e-10	0.096	0	0	\\
9e-10	0.096	0	0	\\
9.5e-10	0.096	0	0	\\
1e-09	0.096	0	0	\\
1.05e-09	0.096	0	0	\\
1.1e-09	0.096	0	0	\\
1.15e-09	0.096	0	0	\\
1.2e-09	0.096	0	0	\\
1.25e-09	0.096	0	0	\\
1.3e-09	0.096	0	0	\\
1.35e-09	0.096	0	0	\\
1.4e-09	0.096	0	0	\\
1.45e-09	0.096	0	0	\\
1.5e-09	0.096	0	0	\\
1.55e-09	0.096	0	0	\\
1.6e-09	0.096	0	0	\\
1.65e-09	0.096	0	0	\\
1.7e-09	0.096	0	0	\\
1.75e-09	0.096	0	0	\\
1.8e-09	0.096	0	0	\\
1.85e-09	0.096	0	0	\\
1.9e-09	0.096	0	0	\\
1.95e-09	0.096	0	0	\\
2e-09	0.096	0	0	\\
2.05e-09	0.096	0	0	\\
2.1e-09	0.096	0	0	\\
2.15e-09	0.096	0	0	\\
2.2e-09	0.096	0	0	\\
2.25e-09	0.096	0	0	\\
2.3e-09	0.096	0	0	\\
2.35e-09	0.096	-0.00666666666666667	-0.00666666666666667	\\
2.4e-09	0.096	-0.00666666666666667	-0.00666666666666667	\\
2.45e-09	0.096	-0.00666666666666667	-0.00666666666666667	\\
2.5e-09	0.096	0	0	\\
2.55e-09	0.096	0	0	\\
2.6e-09	0.096	0	0	\\
2.65e-09	0.096	0	0	\\
2.7e-09	0.096	0	0	\\
2.75e-09	0.096	0	0	\\
2.8e-09	0.096	0	0	\\
2.85e-09	0.096	0.00666666666666667	0.00666666666666667	\\
2.9e-09	0.096	0.00666666666666667	0.00666666666666667	\\
2.95e-09	0.096	0.00666666666666667	0.00666666666666667	\\
3e-09	0.096	0	0	\\
3.05e-09	0.096	0	0	\\
3.1e-09	0.096	0	0	\\
3.15e-09	0.096	0	0	\\
3.2e-09	0.096	0	0	\\
3.25e-09	0.096	0	0	\\
3.3e-09	0.096	0	0	\\
3.35e-09	0.096	0	0	\\
3.4e-09	0.096	0	0	\\
3.45e-09	0.096	0	0	\\
3.5e-09	0.096	0	0	\\
3.55e-09	0.096	0	0	\\
3.6e-09	0.096	0	0	\\
3.65e-09	0.096	0	0	\\
3.7e-09	0.096	0	0	\\
3.75e-09	0.096	0	0	\\
3.8e-09	0.096	0	0	\\
3.85e-09	0.096	0	0	\\
3.9e-09	0.096	0	0	\\
3.95e-09	0.096	0	0	\\
4e-09	0.096	0	0	\\
4.05e-09	0.096	0	0	\\
4.1e-09	0.096	0	0	\\
4.15e-09	0.096	0	0	\\
4.2e-09	0.096	0	0	\\
4.25e-09	0.096	0	0	\\
4.3e-09	0.096	0	0	\\
4.35e-09	0.096	0	0	\\
4.4e-09	0.096	0	0	\\
4.45e-09	0.096	0	0	\\
4.5e-09	0.096	0	0	\\
4.55e-09	0.096	0	0	\\
4.6e-09	0.096	0	0	\\
4.65e-09	0.096	0	0	\\
4.7e-09	0.096	0	0	\\
4.75e-09	0.096	-0.00444444444444444	-0.00444444444444444	\\
4.8e-09	0.096	-0.00444444444444444	-0.00444444444444444	\\
4.85e-09	0.096	-0.00444444444444444	-0.00444444444444444	\\
4.9e-09	0.096	0	0	\\
4.95e-09	0.096	0	0	\\
5e-09	0.096	0	0	\\
5e-09	0.096	0	nan	\\
5e-09	0.096	-0.00666666666666667	0.00166666666666667	\\
5e-09	0.096	-0.00666666666666667	nan	\\
0	0.108	-0.00666666666666667	nan	\\
0	0.108	0	0.00166666666666667	\\
0	0.108	0	0	\\
5e-11	0.108	0	0	\\
1e-10	0.108	0	0	\\
1.5e-10	0.108	0.01	0.01	\\
2e-10	0.108	0.01	0.01	\\
2.5e-10	0.108	0.01	0.01	\\
3e-10	0.108	0.01	0.01	\\
3.5e-10	0.108	0.01	0.01	\\
4e-10	0.108	0.01	0.01	\\
4.5e-10	0.108	0.01	0.01	\\
5e-10	0.108	0.01	0.01	\\
5.5e-10	0.108	0.01	0.01	\\
6e-10	0.108	0.01	0.01	\\
6.5e-10	0.108	0	0	\\
7e-10	0.108	0	0	\\
7.5e-10	0.108	0	0	\\
8e-10	0.108	0	0	\\
8.5e-10	0.108	0	0	\\
9e-10	0.108	0	0	\\
9.5e-10	0.108	0	0	\\
1e-09	0.108	0	0	\\
1.05e-09	0.108	0	0	\\
1.1e-09	0.108	0	0	\\
1.15e-09	0.108	0	0	\\
1.2e-09	0.108	0	0	\\
1.25e-09	0.108	0	0	\\
1.3e-09	0.108	0	0	\\
1.35e-09	0.108	0	0	\\
1.4e-09	0.108	0	0	\\
1.45e-09	0.108	0	0	\\
1.5e-09	0.108	0	0	\\
1.55e-09	0.108	0	0	\\
1.6e-09	0.108	0	0	\\
1.65e-09	0.108	0	0	\\
1.7e-09	0.108	0	0	\\
1.75e-09	0.108	0	0	\\
1.8e-09	0.108	0	0	\\
1.85e-09	0.108	0	0	\\
1.9e-09	0.108	0	0	\\
1.95e-09	0.108	0	0	\\
2e-09	0.108	0	0	\\
2.05e-09	0.108	0	0	\\
2.1e-09	0.108	0	0	\\
2.15e-09	0.108	0	0	\\
2.2e-09	0.108	0	0	\\
2.25e-09	0.108	0	0	\\
2.3e-09	0.108	-0.00666666666666667	-0.00666666666666667	\\
2.35e-09	0.108	-0.00666666666666667	-0.00666666666666667	\\
2.4e-09	0.108	-0.00666666666666667	-0.00666666666666667	\\
2.45e-09	0.108	-0.00666666666666667	-0.00666666666666667	\\
2.5e-09	0.108	-0.00666666666666667	-0.00666666666666667	\\
2.55e-09	0.108	0	0	\\
2.6e-09	0.108	0	0	\\
2.65e-09	0.108	0	0	\\
2.7e-09	0.108	0	0	\\
2.75e-09	0.108	0	0	\\
2.8e-09	0.108	0.00666666666666667	0.00666666666666667	\\
2.85e-09	0.108	0.00666666666666667	0.00666666666666667	\\
2.9e-09	0.108	0.00666666666666667	0.00666666666666667	\\
2.95e-09	0.108	0.00666666666666667	0.00666666666666667	\\
3e-09	0.108	0.00666666666666667	0.00666666666666667	\\
3.05e-09	0.108	0	0	\\
3.1e-09	0.108	0	0	\\
3.15e-09	0.108	0	0	\\
3.2e-09	0.108	0	0	\\
3.25e-09	0.108	0	0	\\
3.3e-09	0.108	0	0	\\
3.35e-09	0.108	0	0	\\
3.4e-09	0.108	0	0	\\
3.45e-09	0.108	0	0	\\
3.5e-09	0.108	0	0	\\
3.55e-09	0.108	0	0	\\
3.6e-09	0.108	0	0	\\
3.65e-09	0.108	0	0	\\
3.7e-09	0.108	0	0	\\
3.75e-09	0.108	0	0	\\
3.8e-09	0.108	0	0	\\
3.85e-09	0.108	0	0	\\
3.9e-09	0.108	0	0	\\
3.95e-09	0.108	0	0	\\
4e-09	0.108	0	0	\\
4.05e-09	0.108	0	0	\\
4.1e-09	0.108	0	0	\\
4.15e-09	0.108	0	0	\\
4.2e-09	0.108	0	0	\\
4.25e-09	0.108	0	0	\\
4.3e-09	0.108	0	0	\\
4.35e-09	0.108	0	0	\\
4.4e-09	0.108	0	0	\\
4.45e-09	0.108	0	0	\\
4.5e-09	0.108	0	0	\\
4.55e-09	0.108	0	0	\\
4.6e-09	0.108	0	0	\\
4.65e-09	0.108	0	0	\\
4.7e-09	0.108	-0.00444444444444444	-0.00444444444444444	\\
4.75e-09	0.108	-0.00444444444444444	-0.00444444444444444	\\
4.8e-09	0.108	-0.00444444444444444	-0.00444444444444444	\\
4.85e-09	0.108	-0.00444444444444444	-0.00444444444444444	\\
4.9e-09	0.108	-0.00444444444444444	-0.00444444444444444	\\
4.95e-09	0.108	0	0	\\
5e-09	0.108	0	0	\\
5e-09	0.108	0	nan	\\
5e-09	0.108	-0.00666666666666667	0.00166666666666667	\\
5e-09	0.108	-0.00666666666666667	nan	\\
0	0.12	-0.00666666666666667	nan	\\
0	0.12	0	0.00166666666666667	\\
0	0.12	0	0	\\
5e-11	0.12	0	0	\\
1e-10	0.12	0	0	\\
1.5e-10	0.12	0.01	0.01	\\
2e-10	0.12	0.01	0.01	\\
2.5e-10	0.12	0.01	0.01	\\
3e-10	0.12	0.01	0.01	\\
3.5e-10	0.12	0.01	0.01	\\
4e-10	0.12	0.01	0.01	\\
4.5e-10	0.12	0.01	0.01	\\
5e-10	0.12	0.01	0.01	\\
5.5e-10	0.12	0.01	0.01	\\
6e-10	0.12	0.01	0.01	\\
6.5e-10	0.12	0	0	\\
7e-10	0.12	0	0	\\
7.5e-10	0.12	0	0	\\
8e-10	0.12	0	0	\\
8.5e-10	0.12	0	0	\\
9e-10	0.12	0	0	\\
9.5e-10	0.12	0	0	\\
1e-09	0.12	0	0	\\
1.05e-09	0.12	0	0	\\
1.1e-09	0.12	0	0	\\
1.15e-09	0.12	0	0	\\
1.2e-09	0.12	0	0	\\
1.25e-09	0.12	0	0	\\
1.3e-09	0.12	0	0	\\
1.35e-09	0.12	0	0	\\
1.4e-09	0.12	0	0	\\
1.45e-09	0.12	0	0	\\
1.5e-09	0.12	0	0	\\
1.55e-09	0.12	0	0	\\
1.6e-09	0.12	0	0	\\
1.65e-09	0.12	0	0	\\
1.7e-09	0.12	0	0	\\
1.75e-09	0.12	0	0	\\
1.8e-09	0.12	0	0	\\
1.85e-09	0.12	0	0	\\
1.9e-09	0.12	0	0	\\
1.95e-09	0.12	0	0	\\
2e-09	0.12	0	0	\\
2.05e-09	0.12	0	0	\\
2.1e-09	0.12	0	0	\\
2.15e-09	0.12	0	0	\\
2.2e-09	0.12	0	0	\\
2.25e-09	0.12	0	0	\\
2.3e-09	0.12	-0.00666666666666667	-0.00666666666666667	\\
2.35e-09	0.12	-0.00666666666666667	-0.00666666666666667	\\
2.4e-09	0.12	-0.00666666666666667	-0.00666666666666667	\\
2.45e-09	0.12	-0.00666666666666667	-0.00666666666666667	\\
2.5e-09	0.12	-0.00666666666666667	-0.00666666666666667	\\
2.55e-09	0.12	0	0	\\
2.6e-09	0.12	0	0	\\
2.65e-09	0.12	0	0	\\
2.7e-09	0.12	0	0	\\
2.75e-09	0.12	0	0	\\
2.8e-09	0.12	0.00666666666666667	0.00666666666666667	\\
2.85e-09	0.12	0.00666666666666667	0.00666666666666667	\\
2.9e-09	0.12	0.00666666666666667	0.00666666666666667	\\
2.95e-09	0.12	0.00666666666666667	0.00666666666666667	\\
3e-09	0.12	0.00666666666666667	0.00666666666666667	\\
3.05e-09	0.12	0	0	\\
3.1e-09	0.12	0	0	\\
3.15e-09	0.12	0	0	\\
3.2e-09	0.12	0	0	\\
3.25e-09	0.12	0	0	\\
3.3e-09	0.12	0	0	\\
3.35e-09	0.12	0	0	\\
3.4e-09	0.12	0	0	\\
3.45e-09	0.12	0	0	\\
3.5e-09	0.12	0	0	\\
3.55e-09	0.12	0	0	\\
3.6e-09	0.12	0	0	\\
3.65e-09	0.12	0	0	\\
3.7e-09	0.12	0	0	\\
3.75e-09	0.12	0	0	\\
3.8e-09	0.12	0	0	\\
3.85e-09	0.12	0	0	\\
3.9e-09	0.12	0	0	\\
3.95e-09	0.12	0	0	\\
4e-09	0.12	0	0	\\
4.05e-09	0.12	0	0	\\
4.1e-09	0.12	0	0	\\
4.15e-09	0.12	0	0	\\
4.2e-09	0.12	0	0	\\
4.25e-09	0.12	0	0	\\
4.3e-09	0.12	0	0	\\
4.35e-09	0.12	0	0	\\
4.4e-09	0.12	0	0	\\
4.45e-09	0.12	0	0	\\
4.5e-09	0.12	0	0	\\
4.55e-09	0.12	0	0	\\
4.6e-09	0.12	0	0	\\
4.65e-09	0.12	0	0	\\
4.7e-09	0.12	-0.00444444444444444	-0.00444444444444444	\\
4.75e-09	0.12	-0.00444444444444444	-0.00444444444444444	\\
4.8e-09	0.12	-0.00444444444444444	-0.00444444444444444	\\
4.85e-09	0.12	-0.00444444444444444	-0.00444444444444444	\\
4.9e-09	0.12	-0.00444444444444444	-0.00444444444444444	\\
4.95e-09	0.12	0	0	\\
5e-09	0.12	0	0	\\
5e-09	0.12	0	nan	\\
5e-09	0.12	-0.00666666666666667	0.00166666666666667	\\
5e-09	0.12	-0.00666666666666667	nan	\\
0	0.132	-0.00666666666666667	nan	\\
0	0.132	0	0.00166666666666667	\\
0	0.132	0	0	\\
5e-11	0.132	0	0	\\
1e-10	0.132	0	0	\\
1.5e-10	0.132	0.01	0.01	\\
2e-10	0.132	0.01	0.01	\\
2.5e-10	0.132	0.01	0.01	\\
3e-10	0.132	0.01	0.01	\\
3.5e-10	0.132	0.01	0.01	\\
4e-10	0.132	0.01	0.01	\\
4.5e-10	0.132	0.01	0.01	\\
5e-10	0.132	0.01	0.01	\\
5.5e-10	0.132	0.01	0.01	\\
6e-10	0.132	0.01	0.01	\\
6.5e-10	0.132	0	0	\\
7e-10	0.132	0	0	\\
7.5e-10	0.132	0	0	\\
8e-10	0.132	0	0	\\
8.5e-10	0.132	0	0	\\
9e-10	0.132	0	0	\\
9.5e-10	0.132	0	0	\\
1e-09	0.132	0	0	\\
1.05e-09	0.132	0	0	\\
1.1e-09	0.132	0	0	\\
1.15e-09	0.132	0	0	\\
1.2e-09	0.132	0	0	\\
1.25e-09	0.132	0	0	\\
1.3e-09	0.132	0	0	\\
1.35e-09	0.132	0	0	\\
1.4e-09	0.132	0	0	\\
1.45e-09	0.132	0	0	\\
1.5e-09	0.132	0	0	\\
1.55e-09	0.132	0	0	\\
1.6e-09	0.132	0	0	\\
1.65e-09	0.132	0	0	\\
1.7e-09	0.132	0	0	\\
1.75e-09	0.132	0	0	\\
1.8e-09	0.132	0	0	\\
1.85e-09	0.132	0	0	\\
1.9e-09	0.132	0	0	\\
1.95e-09	0.132	0	0	\\
2e-09	0.132	0	0	\\
2.05e-09	0.132	0	0	\\
2.1e-09	0.132	0	0	\\
2.15e-09	0.132	0	0	\\
2.2e-09	0.132	0	0	\\
2.25e-09	0.132	0	0	\\
2.3e-09	0.132	-0.00666666666666667	-0.00666666666666667	\\
2.35e-09	0.132	-0.00666666666666667	-0.00666666666666667	\\
2.4e-09	0.132	-0.00666666666666667	-0.00666666666666667	\\
2.45e-09	0.132	-0.00666666666666667	-0.00666666666666667	\\
2.5e-09	0.132	-0.00666666666666667	-0.00666666666666667	\\
2.55e-09	0.132	0	0	\\
2.6e-09	0.132	0	0	\\
2.65e-09	0.132	0	0	\\
2.7e-09	0.132	0	0	\\
2.75e-09	0.132	0	0	\\
2.8e-09	0.132	0.00666666666666667	0.00666666666666667	\\
2.85e-09	0.132	0.00666666666666667	0.00666666666666667	\\
2.9e-09	0.132	0.00666666666666667	0.00666666666666667	\\
2.95e-09	0.132	0.00666666666666667	0.00666666666666667	\\
3e-09	0.132	0.00666666666666667	0.00666666666666667	\\
3.05e-09	0.132	0	0	\\
3.1e-09	0.132	0	0	\\
3.15e-09	0.132	0	0	\\
3.2e-09	0.132	0	0	\\
3.25e-09	0.132	0	0	\\
3.3e-09	0.132	0	0	\\
3.35e-09	0.132	0	0	\\
3.4e-09	0.132	0	0	\\
3.45e-09	0.132	0	0	\\
3.5e-09	0.132	0	0	\\
3.55e-09	0.132	0	0	\\
3.6e-09	0.132	0	0	\\
3.65e-09	0.132	0	0	\\
3.7e-09	0.132	0	0	\\
3.75e-09	0.132	0	0	\\
3.8e-09	0.132	0	0	\\
3.85e-09	0.132	0	0	\\
3.9e-09	0.132	0	0	\\
3.95e-09	0.132	0	0	\\
4e-09	0.132	0	0	\\
4.05e-09	0.132	0	0	\\
4.1e-09	0.132	0	0	\\
4.15e-09	0.132	0	0	\\
4.2e-09	0.132	0	0	\\
4.25e-09	0.132	0	0	\\
4.3e-09	0.132	0	0	\\
4.35e-09	0.132	0	0	\\
4.4e-09	0.132	0	0	\\
4.45e-09	0.132	0	0	\\
4.5e-09	0.132	0	0	\\
4.55e-09	0.132	0	0	\\
4.6e-09	0.132	0	0	\\
4.65e-09	0.132	0	0	\\
4.7e-09	0.132	-0.00444444444444444	-0.00444444444444444	\\
4.75e-09	0.132	-0.00444444444444444	-0.00444444444444444	\\
4.8e-09	0.132	-0.00444444444444444	-0.00444444444444444	\\
4.85e-09	0.132	-0.00444444444444444	-0.00444444444444444	\\
4.9e-09	0.132	-0.00444444444444444	-0.00444444444444444	\\
4.95e-09	0.132	0	0	\\
5e-09	0.132	0	0	\\
5e-09	0.132	0	nan	\\
5e-09	0.132	-0.00666666666666667	0.00166666666666667	\\
5e-09	0.132	-0.00666666666666667	nan	\\
0	0.144	-0.00666666666666667	nan	\\
0	0.144	0	0.00166666666666667	\\
0	0.144	0	0	\\
5e-11	0.144	0	0	\\
1e-10	0.144	0	0	\\
1.5e-10	0.144	0.01	0.01	\\
2e-10	0.144	0.01	0.01	\\
2.5e-10	0.144	0.01	0.01	\\
3e-10	0.144	0.01	0.01	\\
3.5e-10	0.144	0.01	0.01	\\
4e-10	0.144	0.01	0.01	\\
4.5e-10	0.144	0.01	0.01	\\
5e-10	0.144	0.01	0.01	\\
5.5e-10	0.144	0.01	0.01	\\
6e-10	0.144	0.01	0.01	\\
6.5e-10	0.144	0	0	\\
7e-10	0.144	0	0	\\
7.5e-10	0.144	0	0	\\
8e-10	0.144	0	0	\\
8.5e-10	0.144	0	0	\\
9e-10	0.144	0	0	\\
9.5e-10	0.144	0	0	\\
1e-09	0.144	0	0	\\
1.05e-09	0.144	0	0	\\
1.1e-09	0.144	0	0	\\
1.15e-09	0.144	0	0	\\
1.2e-09	0.144	0	0	\\
1.25e-09	0.144	0	0	\\
1.3e-09	0.144	0	0	\\
1.35e-09	0.144	0	0	\\
1.4e-09	0.144	0	0	\\
1.45e-09	0.144	0	0	\\
1.5e-09	0.144	0	0	\\
1.55e-09	0.144	0	0	\\
1.6e-09	0.144	0	0	\\
1.65e-09	0.144	0	0	\\
1.7e-09	0.144	0	0	\\
1.75e-09	0.144	0	0	\\
1.8e-09	0.144	0	0	\\
1.85e-09	0.144	0	0	\\
1.9e-09	0.144	0	0	\\
1.95e-09	0.144	0	0	\\
2e-09	0.144	0	0	\\
2.05e-09	0.144	0	0	\\
2.1e-09	0.144	0	0	\\
2.15e-09	0.144	0	0	\\
2.2e-09	0.144	0	0	\\
2.25e-09	0.144	0	0	\\
2.3e-09	0.144	-0.00666666666666667	-0.00666666666666667	\\
2.35e-09	0.144	-0.00666666666666667	-0.00666666666666667	\\
2.4e-09	0.144	-0.00666666666666667	-0.00666666666666667	\\
2.45e-09	0.144	-0.00666666666666667	-0.00666666666666667	\\
2.5e-09	0.144	-0.00666666666666667	-0.00666666666666667	\\
2.55e-09	0.144	0	0	\\
2.6e-09	0.144	0	0	\\
2.65e-09	0.144	0	0	\\
2.7e-09	0.144	0	0	\\
2.75e-09	0.144	0	0	\\
2.8e-09	0.144	0.00666666666666667	0.00666666666666667	\\
2.85e-09	0.144	0.00666666666666667	0.00666666666666667	\\
2.9e-09	0.144	0.00666666666666667	0.00666666666666667	\\
2.95e-09	0.144	0.00666666666666667	0.00666666666666667	\\
3e-09	0.144	0.00666666666666667	0.00666666666666667	\\
3.05e-09	0.144	0	0	\\
3.1e-09	0.144	0	0	\\
3.15e-09	0.144	0	0	\\
3.2e-09	0.144	0	0	\\
3.25e-09	0.144	0	0	\\
3.3e-09	0.144	0	0	\\
3.35e-09	0.144	0	0	\\
3.4e-09	0.144	0	0	\\
3.45e-09	0.144	0	0	\\
3.5e-09	0.144	0	0	\\
3.55e-09	0.144	0	0	\\
3.6e-09	0.144	0	0	\\
3.65e-09	0.144	0	0	\\
3.7e-09	0.144	0	0	\\
3.75e-09	0.144	0	0	\\
3.8e-09	0.144	0	0	\\
3.85e-09	0.144	0	0	\\
3.9e-09	0.144	0	0	\\
3.95e-09	0.144	0	0	\\
4e-09	0.144	0	0	\\
4.05e-09	0.144	0	0	\\
4.1e-09	0.144	0	0	\\
4.15e-09	0.144	0	0	\\
4.2e-09	0.144	0	0	\\
4.25e-09	0.144	0	0	\\
4.3e-09	0.144	0	0	\\
4.35e-09	0.144	0	0	\\
4.4e-09	0.144	0	0	\\
4.45e-09	0.144	0	0	\\
4.5e-09	0.144	0	0	\\
4.55e-09	0.144	0	0	\\
4.6e-09	0.144	0	0	\\
4.65e-09	0.144	0	0	\\
4.7e-09	0.144	-0.00444444444444444	-0.00444444444444444	\\
4.75e-09	0.144	-0.00444444444444444	-0.00444444444444444	\\
4.8e-09	0.144	-0.00444444444444444	-0.00444444444444444	\\
4.85e-09	0.144	-0.00444444444444444	-0.00444444444444444	\\
4.9e-09	0.144	-0.00444444444444444	-0.00444444444444444	\\
4.95e-09	0.144	0	0	\\
5e-09	0.144	0	0	\\
5e-09	0.144	0	nan	\\
5e-09	0.144	-0.00666666666666667	0.00166666666666667	\\
5e-09	0.144	-0.00666666666666667	nan	\\
0	0.156	-0.00666666666666667	nan	\\
0	0.156	0	0.00166666666666667	\\
0	0.156	0	0	\\
5e-11	0.156	0	0	\\
1e-10	0.156	0	0	\\
1.5e-10	0.156	0	0	\\
2e-10	0.156	0.01	0.01	\\
2.5e-10	0.156	0.01	0.01	\\
3e-10	0.156	0.01	0.01	\\
3.5e-10	0.156	0.01	0.01	\\
4e-10	0.156	0.01	0.01	\\
4.5e-10	0.156	0.01	0.01	\\
5e-10	0.156	0.01	0.01	\\
5.5e-10	0.156	0.01	0.01	\\
6e-10	0.156	0.01	0.01	\\
6.5e-10	0.156	0.01	0.01	\\
7e-10	0.156	0	0	\\
7.5e-10	0.156	0	0	\\
8e-10	0.156	0	0	\\
8.5e-10	0.156	0	0	\\
9e-10	0.156	0	0	\\
9.5e-10	0.156	0	0	\\
1e-09	0.156	0	0	\\
1.05e-09	0.156	0	0	\\
1.1e-09	0.156	0	0	\\
1.15e-09	0.156	0	0	\\
1.2e-09	0.156	0	0	\\
1.25e-09	0.156	0	0	\\
1.3e-09	0.156	0	0	\\
1.35e-09	0.156	0	0	\\
1.4e-09	0.156	0	0	\\
1.45e-09	0.156	0	0	\\
1.5e-09	0.156	0	0	\\
1.55e-09	0.156	0	0	\\
1.6e-09	0.156	0	0	\\
1.65e-09	0.156	0	0	\\
1.7e-09	0.156	0	0	\\
1.75e-09	0.156	0	0	\\
1.8e-09	0.156	0	0	\\
1.85e-09	0.156	0	0	\\
1.9e-09	0.156	0	0	\\
1.95e-09	0.156	0	0	\\
2e-09	0.156	0	0	\\
2.05e-09	0.156	0	0	\\
2.1e-09	0.156	0	0	\\
2.15e-09	0.156	0	0	\\
2.2e-09	0.156	0	0	\\
2.25e-09	0.156	-0.00666666666666667	-0.00666666666666667	\\
2.3e-09	0.156	-0.00666666666666667	-0.00666666666666667	\\
2.35e-09	0.156	-0.00666666666666667	-0.00666666666666667	\\
2.4e-09	0.156	-0.00666666666666667	-0.00666666666666667	\\
2.45e-09	0.156	-0.00666666666666667	-0.00666666666666667	\\
2.5e-09	0.156	-0.00666666666666667	-0.00666666666666667	\\
2.55e-09	0.156	-0.00666666666666667	-0.00666666666666667	\\
2.6e-09	0.156	0	0	\\
2.65e-09	0.156	0	0	\\
2.7e-09	0.156	0	0	\\
2.75e-09	0.156	0.00666666666666667	0.00666666666666667	\\
2.8e-09	0.156	0.00666666666666667	0.00666666666666667	\\
2.85e-09	0.156	0.00666666666666667	0.00666666666666667	\\
2.9e-09	0.156	0.00666666666666667	0.00666666666666667	\\
2.95e-09	0.156	0.00666666666666667	0.00666666666666667	\\
3e-09	0.156	0.00666666666666667	0.00666666666666667	\\
3.05e-09	0.156	0.00666666666666667	0.00666666666666667	\\
3.1e-09	0.156	0	0	\\
3.15e-09	0.156	0	0	\\
3.2e-09	0.156	0	0	\\
3.25e-09	0.156	0	0	\\
3.3e-09	0.156	0	0	\\
3.35e-09	0.156	0	0	\\
3.4e-09	0.156	0	0	\\
3.45e-09	0.156	0	0	\\
3.5e-09	0.156	0	0	\\
3.55e-09	0.156	0	0	\\
3.6e-09	0.156	0	0	\\
3.65e-09	0.156	0	0	\\
3.7e-09	0.156	0	0	\\
3.75e-09	0.156	0	0	\\
3.8e-09	0.156	0	0	\\
3.85e-09	0.156	0	0	\\
3.9e-09	0.156	0	0	\\
3.95e-09	0.156	0	0	\\
4e-09	0.156	0	0	\\
4.05e-09	0.156	0	0	\\
4.1e-09	0.156	0	0	\\
4.15e-09	0.156	0	0	\\
4.2e-09	0.156	0	0	\\
4.25e-09	0.156	0	0	\\
4.3e-09	0.156	0	0	\\
4.35e-09	0.156	0	0	\\
4.4e-09	0.156	0	0	\\
4.45e-09	0.156	0	0	\\
4.5e-09	0.156	0	0	\\
4.55e-09	0.156	0	0	\\
4.6e-09	0.156	0	0	\\
4.65e-09	0.156	-0.00444444444444444	-0.00444444444444444	\\
4.7e-09	0.156	-0.00444444444444444	-0.00444444444444444	\\
4.75e-09	0.156	-0.00444444444444444	-0.00444444444444444	\\
4.8e-09	0.156	-0.00444444444444444	-0.00444444444444444	\\
4.85e-09	0.156	-0.00444444444444444	-0.00444444444444444	\\
4.9e-09	0.156	-0.00444444444444444	-0.00444444444444444	\\
4.95e-09	0.156	-0.00444444444444444	-0.00444444444444444	\\
5e-09	0.156	0	0	\\
5e-09	0.156	0	nan	\\
5e-09	0.156	-0.00666666666666667	0.00166666666666667	\\
5e-09	0.156	-0.00666666666666667	nan	\\
0	0.168	-0.00666666666666667	nan	\\
0	0.168	0	0.00166666666666667	\\
0	0.168	0	0	\\
5e-11	0.168	0	0	\\
1e-10	0.168	0	0	\\
1.5e-10	0.168	0	0	\\
2e-10	0.168	0.01	0.01	\\
2.5e-10	0.168	0.01	0.01	\\
3e-10	0.168	0.01	0.01	\\
3.5e-10	0.168	0.01	0.01	\\
4e-10	0.168	0.01	0.01	\\
4.5e-10	0.168	0.01	0.01	\\
5e-10	0.168	0.01	0.01	\\
5.5e-10	0.168	0.01	0.01	\\
6e-10	0.168	0.01	0.01	\\
6.5e-10	0.168	0.01	0.01	\\
7e-10	0.168	0	0	\\
7.5e-10	0.168	0	0	\\
8e-10	0.168	0	0	\\
8.5e-10	0.168	0	0	\\
9e-10	0.168	0	0	\\
9.5e-10	0.168	0	0	\\
1e-09	0.168	0	0	\\
1.05e-09	0.168	0	0	\\
1.1e-09	0.168	0	0	\\
1.15e-09	0.168	0	0	\\
1.2e-09	0.168	0	0	\\
1.25e-09	0.168	0	0	\\
1.3e-09	0.168	0	0	\\
1.35e-09	0.168	0	0	\\
1.4e-09	0.168	0	0	\\
1.45e-09	0.168	0	0	\\
1.5e-09	0.168	0	0	\\
1.55e-09	0.168	0	0	\\
1.6e-09	0.168	0	0	\\
1.65e-09	0.168	0	0	\\
1.7e-09	0.168	0	0	\\
1.75e-09	0.168	0	0	\\
1.8e-09	0.168	0	0	\\
1.85e-09	0.168	0	0	\\
1.9e-09	0.168	0	0	\\
1.95e-09	0.168	0	0	\\
2e-09	0.168	0	0	\\
2.05e-09	0.168	0	0	\\
2.1e-09	0.168	0	0	\\
2.15e-09	0.168	0	0	\\
2.2e-09	0.168	0	0	\\
2.25e-09	0.168	-0.00666666666666667	-0.00666666666666667	\\
2.3e-09	0.168	-0.00666666666666667	-0.00666666666666667	\\
2.35e-09	0.168	-0.00666666666666667	-0.00666666666666667	\\
2.4e-09	0.168	-0.00666666666666667	-0.00666666666666667	\\
2.45e-09	0.168	-0.00666666666666667	-0.00666666666666667	\\
2.5e-09	0.168	-0.00666666666666667	-0.00666666666666667	\\
2.55e-09	0.168	-0.00666666666666667	-0.00666666666666667	\\
2.6e-09	0.168	0	0	\\
2.65e-09	0.168	0	0	\\
2.7e-09	0.168	0	0	\\
2.75e-09	0.168	0.00666666666666667	0.00666666666666667	\\
2.8e-09	0.168	0.00666666666666667	0.00666666666666667	\\
2.85e-09	0.168	0.00666666666666667	0.00666666666666667	\\
2.9e-09	0.168	0.00666666666666667	0.00666666666666667	\\
2.95e-09	0.168	0.00666666666666667	0.00666666666666667	\\
3e-09	0.168	0.00666666666666667	0.00666666666666667	\\
3.05e-09	0.168	0.00666666666666667	0.00666666666666667	\\
3.1e-09	0.168	0	0	\\
3.15e-09	0.168	0	0	\\
3.2e-09	0.168	0	0	\\
3.25e-09	0.168	0	0	\\
3.3e-09	0.168	0	0	\\
3.35e-09	0.168	0	0	\\
3.4e-09	0.168	0	0	\\
3.45e-09	0.168	0	0	\\
3.5e-09	0.168	0	0	\\
3.55e-09	0.168	0	0	\\
3.6e-09	0.168	0	0	\\
3.65e-09	0.168	0	0	\\
3.7e-09	0.168	0	0	\\
3.75e-09	0.168	0	0	\\
3.8e-09	0.168	0	0	\\
3.85e-09	0.168	0	0	\\
3.9e-09	0.168	0	0	\\
3.95e-09	0.168	0	0	\\
4e-09	0.168	0	0	\\
4.05e-09	0.168	0	0	\\
4.1e-09	0.168	0	0	\\
4.15e-09	0.168	0	0	\\
4.2e-09	0.168	0	0	\\
4.25e-09	0.168	0	0	\\
4.3e-09	0.168	0	0	\\
4.35e-09	0.168	0	0	\\
4.4e-09	0.168	0	0	\\
4.45e-09	0.168	0	0	\\
4.5e-09	0.168	0	0	\\
4.55e-09	0.168	0	0	\\
4.6e-09	0.168	0	0	\\
4.65e-09	0.168	-0.00444444444444444	-0.00444444444444444	\\
4.7e-09	0.168	-0.00444444444444444	-0.00444444444444444	\\
4.75e-09	0.168	-0.00444444444444444	-0.00444444444444444	\\
4.8e-09	0.168	-0.00444444444444444	-0.00444444444444444	\\
4.85e-09	0.168	-0.00444444444444444	-0.00444444444444444	\\
4.9e-09	0.168	-0.00444444444444444	-0.00444444444444444	\\
4.95e-09	0.168	-0.00444444444444444	-0.00444444444444444	\\
5e-09	0.168	0	0	\\
5e-09	0.168	0	nan	\\
5e-09	0.168	-0.00666666666666667	0.00166666666666667	\\
5e-09	0.168	-0.00666666666666667	nan	\\
0	0.18	-0.00666666666666667	nan	\\
0	0.18	0	0.00166666666666667	\\
0	0.18	0	0	\\
5e-11	0.18	0	0	\\
1e-10	0.18	0	0	\\
1.5e-10	0.18	0	0	\\
2e-10	0.18	0.01	0.01	\\
2.5e-10	0.18	0.01	0.01	\\
3e-10	0.18	0.01	0.01	\\
3.5e-10	0.18	0.01	0.01	\\
4e-10	0.18	0.01	0.01	\\
4.5e-10	0.18	0.01	0.01	\\
5e-10	0.18	0.01	0.01	\\
5.5e-10	0.18	0.01	0.01	\\
6e-10	0.18	0.01	0.01	\\
6.5e-10	0.18	0.01	0.01	\\
7e-10	0.18	0	0	\\
7.5e-10	0.18	0	0	\\
8e-10	0.18	0	0	\\
8.5e-10	0.18	0	0	\\
9e-10	0.18	0	0	\\
9.5e-10	0.18	0	0	\\
1e-09	0.18	0	0	\\
1.05e-09	0.18	0	0	\\
1.1e-09	0.18	0	0	\\
1.15e-09	0.18	0	0	\\
1.2e-09	0.18	0	0	\\
1.25e-09	0.18	0	0	\\
1.3e-09	0.18	0	0	\\
1.35e-09	0.18	0	0	\\
1.4e-09	0.18	0	0	\\
1.45e-09	0.18	0	0	\\
1.5e-09	0.18	0	0	\\
1.55e-09	0.18	0	0	\\
1.6e-09	0.18	0	0	\\
1.65e-09	0.18	0	0	\\
1.7e-09	0.18	0	0	\\
1.75e-09	0.18	0	0	\\
1.8e-09	0.18	0	0	\\
1.85e-09	0.18	0	0	\\
1.9e-09	0.18	0	0	\\
1.95e-09	0.18	0	0	\\
2e-09	0.18	0	0	\\
2.05e-09	0.18	0	0	\\
2.1e-09	0.18	0	0	\\
2.15e-09	0.18	0	0	\\
2.2e-09	0.18	0	0	\\
2.25e-09	0.18	-0.00666666666666667	-0.00666666666666667	\\
2.3e-09	0.18	-0.00666666666666667	-0.00666666666666667	\\
2.35e-09	0.18	-0.00666666666666667	-0.00666666666666667	\\
2.4e-09	0.18	-0.00666666666666667	-0.00666666666666667	\\
2.45e-09	0.18	-0.00666666666666667	-0.00666666666666667	\\
2.5e-09	0.18	-0.00666666666666667	-0.00666666666666667	\\
2.55e-09	0.18	-0.00666666666666667	-0.00666666666666667	\\
2.6e-09	0.18	0	0	\\
2.65e-09	0.18	0	0	\\
2.7e-09	0.18	0	0	\\
2.75e-09	0.18	0.00666666666666667	0.00666666666666667	\\
2.8e-09	0.18	0.00666666666666667	0.00666666666666667	\\
2.85e-09	0.18	0.00666666666666667	0.00666666666666667	\\
2.9e-09	0.18	0.00666666666666667	0.00666666666666667	\\
2.95e-09	0.18	0.00666666666666667	0.00666666666666667	\\
3e-09	0.18	0.00666666666666667	0.00666666666666667	\\
3.05e-09	0.18	0.00666666666666667	0.00666666666666667	\\
3.1e-09	0.18	0	0	\\
3.15e-09	0.18	0	0	\\
3.2e-09	0.18	0	0	\\
3.25e-09	0.18	0	0	\\
3.3e-09	0.18	0	0	\\
3.35e-09	0.18	0	0	\\
3.4e-09	0.18	0	0	\\
3.45e-09	0.18	0	0	\\
3.5e-09	0.18	0	0	\\
3.55e-09	0.18	0	0	\\
3.6e-09	0.18	0	0	\\
3.65e-09	0.18	0	0	\\
3.7e-09	0.18	0	0	\\
3.75e-09	0.18	0	0	\\
3.8e-09	0.18	0	0	\\
3.85e-09	0.18	0	0	\\
3.9e-09	0.18	0	0	\\
3.95e-09	0.18	0	0	\\
4e-09	0.18	0	0	\\
4.05e-09	0.18	0	0	\\
4.1e-09	0.18	0	0	\\
4.15e-09	0.18	0	0	\\
4.2e-09	0.18	0	0	\\
4.25e-09	0.18	0	0	\\
4.3e-09	0.18	0	0	\\
4.35e-09	0.18	0	0	\\
4.4e-09	0.18	0	0	\\
4.45e-09	0.18	0	0	\\
4.5e-09	0.18	0	0	\\
4.55e-09	0.18	0	0	\\
4.6e-09	0.18	0	0	\\
4.65e-09	0.18	-0.00444444444444444	-0.00444444444444444	\\
4.7e-09	0.18	-0.00444444444444444	-0.00444444444444444	\\
4.75e-09	0.18	-0.00444444444444444	-0.00444444444444444	\\
4.8e-09	0.18	-0.00444444444444444	-0.00444444444444444	\\
4.85e-09	0.18	-0.00444444444444444	-0.00444444444444444	\\
4.9e-09	0.18	-0.00444444444444444	-0.00444444444444444	\\
4.95e-09	0.18	-0.00444444444444444	-0.00444444444444444	\\
5e-09	0.18	0	0	\\
5e-09	0.18	0	nan	\\
5e-09	0.18	-0.00666666666666667	0.00166666666666667	\\
5e-09	0.18	-0.00666666666666667	nan	\\
0	0.192	-0.00666666666666667	nan	\\
0	0.192	0	0.00166666666666667	\\
0	0.192	0	0	\\
5e-11	0.192	0	0	\\
1e-10	0.192	0	0	\\
1.5e-10	0.192	0	0	\\
2e-10	0.192	0.01	0.01	\\
2.5e-10	0.192	0.01	0.01	\\
3e-10	0.192	0.01	0.01	\\
3.5e-10	0.192	0.01	0.01	\\
4e-10	0.192	0.01	0.01	\\
4.5e-10	0.192	0.01	0.01	\\
5e-10	0.192	0.01	0.01	\\
5.5e-10	0.192	0.01	0.01	\\
6e-10	0.192	0.01	0.01	\\
6.5e-10	0.192	0.01	0.01	\\
7e-10	0.192	0	0	\\
7.5e-10	0.192	0	0	\\
8e-10	0.192	0	0	\\
8.5e-10	0.192	0	0	\\
9e-10	0.192	0	0	\\
9.5e-10	0.192	0	0	\\
1e-09	0.192	0	0	\\
1.05e-09	0.192	0	0	\\
1.1e-09	0.192	0	0	\\
1.15e-09	0.192	0	0	\\
1.2e-09	0.192	0	0	\\
1.25e-09	0.192	0	0	\\
1.3e-09	0.192	0	0	\\
1.35e-09	0.192	0	0	\\
1.4e-09	0.192	0	0	\\
1.45e-09	0.192	0	0	\\
1.5e-09	0.192	0	0	\\
1.55e-09	0.192	0	0	\\
1.6e-09	0.192	0	0	\\
1.65e-09	0.192	0	0	\\
1.7e-09	0.192	0	0	\\
1.75e-09	0.192	0	0	\\
1.8e-09	0.192	0	0	\\
1.85e-09	0.192	0	0	\\
1.9e-09	0.192	0	0	\\
1.95e-09	0.192	0	0	\\
2e-09	0.192	0	0	\\
2.05e-09	0.192	0	0	\\
2.1e-09	0.192	0	0	\\
2.15e-09	0.192	0	0	\\
2.2e-09	0.192	0	0	\\
2.25e-09	0.192	-0.00666666666666667	-0.00666666666666667	\\
2.3e-09	0.192	-0.00666666666666667	-0.00666666666666667	\\
2.35e-09	0.192	-0.00666666666666667	-0.00666666666666667	\\
2.4e-09	0.192	-0.00666666666666667	-0.00666666666666667	\\
2.45e-09	0.192	-0.00666666666666667	-0.00666666666666667	\\
2.5e-09	0.192	-0.00666666666666667	-0.00666666666666667	\\
2.55e-09	0.192	-0.00666666666666667	-0.00666666666666667	\\
2.6e-09	0.192	0	0	\\
2.65e-09	0.192	0	0	\\
2.7e-09	0.192	0	0	\\
2.75e-09	0.192	0.00666666666666667	0.00666666666666667	\\
2.8e-09	0.192	0.00666666666666667	0.00666666666666667	\\
2.85e-09	0.192	0.00666666666666667	0.00666666666666667	\\
2.9e-09	0.192	0.00666666666666667	0.00666666666666667	\\
2.95e-09	0.192	0.00666666666666667	0.00666666666666667	\\
3e-09	0.192	0.00666666666666667	0.00666666666666667	\\
3.05e-09	0.192	0.00666666666666667	0.00666666666666667	\\
3.1e-09	0.192	0	0	\\
3.15e-09	0.192	0	0	\\
3.2e-09	0.192	0	0	\\
3.25e-09	0.192	0	0	\\
3.3e-09	0.192	0	0	\\
3.35e-09	0.192	0	0	\\
3.4e-09	0.192	0	0	\\
3.45e-09	0.192	0	0	\\
3.5e-09	0.192	0	0	\\
3.55e-09	0.192	0	0	\\
3.6e-09	0.192	0	0	\\
3.65e-09	0.192	0	0	\\
3.7e-09	0.192	0	0	\\
3.75e-09	0.192	0	0	\\
3.8e-09	0.192	0	0	\\
3.85e-09	0.192	0	0	\\
3.9e-09	0.192	0	0	\\
3.95e-09	0.192	0	0	\\
4e-09	0.192	0	0	\\
4.05e-09	0.192	0	0	\\
4.1e-09	0.192	0	0	\\
4.15e-09	0.192	0	0	\\
4.2e-09	0.192	0	0	\\
4.25e-09	0.192	0	0	\\
4.3e-09	0.192	0	0	\\
4.35e-09	0.192	0	0	\\
4.4e-09	0.192	0	0	\\
4.45e-09	0.192	0	0	\\
4.5e-09	0.192	0	0	\\
4.55e-09	0.192	0	0	\\
4.6e-09	0.192	0	0	\\
4.65e-09	0.192	-0.00444444444444444	-0.00444444444444444	\\
4.7e-09	0.192	-0.00444444444444444	-0.00444444444444444	\\
4.75e-09	0.192	-0.00444444444444444	-0.00444444444444444	\\
4.8e-09	0.192	-0.00444444444444444	-0.00444444444444444	\\
4.85e-09	0.192	-0.00444444444444444	-0.00444444444444444	\\
4.9e-09	0.192	-0.00444444444444444	-0.00444444444444444	\\
4.95e-09	0.192	-0.00444444444444444	-0.00444444444444444	\\
5e-09	0.192	0	0	\\
5e-09	0.192	0	nan	\\
5e-09	0.192	-0.00666666666666667	0.00166666666666667	\\
5e-09	0.192	-0.00666666666666667	nan	\\
0	0.204	-0.00666666666666667	nan	\\
0	0.204	0	0.00166666666666667	\\
0	0.204	0	0	\\
5e-11	0.204	0	0	\\
1e-10	0.204	0	0	\\
1.5e-10	0.204	0	0	\\
2e-10	0.204	0	0	\\
2.5e-10	0.204	0.01	0.01	\\
3e-10	0.204	0.01	0.01	\\
3.5e-10	0.204	0.01	0.01	\\
4e-10	0.204	0.01	0.01	\\
4.5e-10	0.204	0.01	0.01	\\
5e-10	0.204	0.01	0.01	\\
5.5e-10	0.204	0.01	0.01	\\
6e-10	0.204	0.01	0.01	\\
6.5e-10	0.204	0.01	0.01	\\
7e-10	0.204	0.01	0.01	\\
7.5e-10	0.204	0	0	\\
8e-10	0.204	0	0	\\
8.5e-10	0.204	0	0	\\
9e-10	0.204	0	0	\\
9.5e-10	0.204	0	0	\\
1e-09	0.204	0	0	\\
1.05e-09	0.204	0	0	\\
1.1e-09	0.204	0	0	\\
1.15e-09	0.204	0	0	\\
1.2e-09	0.204	0	0	\\
1.25e-09	0.204	0	0	\\
1.3e-09	0.204	0	0	\\
1.35e-09	0.204	0	0	\\
1.4e-09	0.204	0	0	\\
1.45e-09	0.204	0	0	\\
1.5e-09	0.204	0	0	\\
1.55e-09	0.204	0	0	\\
1.6e-09	0.204	0	0	\\
1.65e-09	0.204	0	0	\\
1.7e-09	0.204	0	0	\\
1.75e-09	0.204	0	0	\\
1.8e-09	0.204	0	0	\\
1.85e-09	0.204	0	0	\\
1.9e-09	0.204	0	0	\\
1.95e-09	0.204	0	0	\\
2e-09	0.204	0	0	\\
2.05e-09	0.204	0	0	\\
2.1e-09	0.204	0	0	\\
2.15e-09	0.204	0	0	\\
2.2e-09	0.204	-0.00666666666666667	-0.00666666666666667	\\
2.25e-09	0.204	-0.00666666666666667	-0.00666666666666667	\\
2.3e-09	0.204	-0.00666666666666667	-0.00666666666666667	\\
2.35e-09	0.204	-0.00666666666666667	-0.00666666666666667	\\
2.4e-09	0.204	-0.00666666666666667	-0.00666666666666667	\\
2.45e-09	0.204	-0.00666666666666667	-0.00666666666666667	\\
2.5e-09	0.204	-0.00666666666666667	-0.00666666666666667	\\
2.55e-09	0.204	-0.00666666666666667	-0.00666666666666667	\\
2.6e-09	0.204	-0.00666666666666667	-0.00666666666666667	\\
2.65e-09	0.204	0	0	\\
2.7e-09	0.204	0.00666666666666667	0.00666666666666667	\\
2.75e-09	0.204	0.00666666666666667	0.00666666666666667	\\
2.8e-09	0.204	0.00666666666666667	0.00666666666666667	\\
2.85e-09	0.204	0.00666666666666667	0.00666666666666667	\\
2.9e-09	0.204	0.00666666666666667	0.00666666666666667	\\
2.95e-09	0.204	0.00666666666666667	0.00666666666666667	\\
3e-09	0.204	0.00666666666666667	0.00666666666666667	\\
3.05e-09	0.204	0.00666666666666667	0.00666666666666667	\\
3.1e-09	0.204	0.00666666666666667	0.00666666666666667	\\
3.15e-09	0.204	0	0	\\
3.2e-09	0.204	0	0	\\
3.25e-09	0.204	0	0	\\
3.3e-09	0.204	0	0	\\
3.35e-09	0.204	0	0	\\
3.4e-09	0.204	0	0	\\
3.45e-09	0.204	0	0	\\
3.5e-09	0.204	0	0	\\
3.55e-09	0.204	0	0	\\
3.6e-09	0.204	0	0	\\
3.65e-09	0.204	0	0	\\
3.7e-09	0.204	0	0	\\
3.75e-09	0.204	0	0	\\
3.8e-09	0.204	0	0	\\
3.85e-09	0.204	0	0	\\
3.9e-09	0.204	0	0	\\
3.95e-09	0.204	0	0	\\
4e-09	0.204	0	0	\\
4.05e-09	0.204	0	0	\\
4.1e-09	0.204	0	0	\\
4.15e-09	0.204	0	0	\\
4.2e-09	0.204	0	0	\\
4.25e-09	0.204	0	0	\\
4.3e-09	0.204	0	0	\\
4.35e-09	0.204	0	0	\\
4.4e-09	0.204	0	0	\\
4.45e-09	0.204	0	0	\\
4.5e-09	0.204	0	0	\\
4.55e-09	0.204	0	0	\\
4.6e-09	0.204	-0.00444444444444444	-0.00444444444444444	\\
4.65e-09	0.204	-0.00444444444444444	-0.00444444444444444	\\
4.7e-09	0.204	-0.00444444444444444	-0.00444444444444444	\\
4.75e-09	0.204	-0.00444444444444444	-0.00444444444444444	\\
4.8e-09	0.204	-0.00444444444444444	-0.00444444444444444	\\
4.85e-09	0.204	-0.00444444444444444	-0.00444444444444444	\\
4.9e-09	0.204	-0.00444444444444444	-0.00444444444444444	\\
4.95e-09	0.204	-0.00444444444444444	-0.00444444444444444	\\
5e-09	0.204	-0.00444444444444444	-0.00444444444444444	\\
5e-09	0.204	-0.00444444444444444	nan	\\
5e-09	0.204	-0.00666666666666667	0.00166666666666667	\\
5e-09	0.204	-0.00666666666666667	nan	\\
0	0.216	-0.00666666666666667	nan	\\
0	0.216	0	0.00166666666666667	\\
0	0.216	0	0	\\
5e-11	0.216	0	0	\\
1e-10	0.216	0	0	\\
1.5e-10	0.216	0	0	\\
2e-10	0.216	0	0	\\
2.5e-10	0.216	0.01	0.01	\\
3e-10	0.216	0.01	0.01	\\
3.5e-10	0.216	0.01	0.01	\\
4e-10	0.216	0.01	0.01	\\
4.5e-10	0.216	0.01	0.01	\\
5e-10	0.216	0.01	0.01	\\
5.5e-10	0.216	0.01	0.01	\\
6e-10	0.216	0.01	0.01	\\
6.5e-10	0.216	0.01	0.01	\\
7e-10	0.216	0.01	0.01	\\
7.5e-10	0.216	0	0	\\
8e-10	0.216	0	0	\\
8.5e-10	0.216	0	0	\\
9e-10	0.216	0	0	\\
9.5e-10	0.216	0	0	\\
1e-09	0.216	0	0	\\
1.05e-09	0.216	0	0	\\
1.1e-09	0.216	0	0	\\
1.15e-09	0.216	0	0	\\
1.2e-09	0.216	0	0	\\
1.25e-09	0.216	0	0	\\
1.3e-09	0.216	0	0	\\
1.35e-09	0.216	0	0	\\
1.4e-09	0.216	0	0	\\
1.45e-09	0.216	0	0	\\
1.5e-09	0.216	0	0	\\
1.55e-09	0.216	0	0	\\
1.6e-09	0.216	0	0	\\
1.65e-09	0.216	0	0	\\
1.7e-09	0.216	0	0	\\
1.75e-09	0.216	0	0	\\
1.8e-09	0.216	0	0	\\
1.85e-09	0.216	0	0	\\
1.9e-09	0.216	0	0	\\
1.95e-09	0.216	0	0	\\
2e-09	0.216	0	0	\\
2.05e-09	0.216	0	0	\\
2.1e-09	0.216	0	0	\\
2.15e-09	0.216	0	0	\\
2.2e-09	0.216	-0.00666666666666667	-0.00666666666666667	\\
2.25e-09	0.216	-0.00666666666666667	-0.00666666666666667	\\
2.3e-09	0.216	-0.00666666666666667	-0.00666666666666667	\\
2.35e-09	0.216	-0.00666666666666667	-0.00666666666666667	\\
2.4e-09	0.216	-0.00666666666666667	-0.00666666666666667	\\
2.45e-09	0.216	-0.00666666666666667	-0.00666666666666667	\\
2.5e-09	0.216	-0.00666666666666667	-0.00666666666666667	\\
2.55e-09	0.216	-0.00666666666666667	-0.00666666666666667	\\
2.6e-09	0.216	-0.00666666666666667	-0.00666666666666667	\\
2.65e-09	0.216	0	0	\\
2.7e-09	0.216	0.00666666666666667	0.00666666666666667	\\
2.75e-09	0.216	0.00666666666666667	0.00666666666666667	\\
2.8e-09	0.216	0.00666666666666667	0.00666666666666667	\\
2.85e-09	0.216	0.00666666666666667	0.00666666666666667	\\
2.9e-09	0.216	0.00666666666666667	0.00666666666666667	\\
2.95e-09	0.216	0.00666666666666667	0.00666666666666667	\\
3e-09	0.216	0.00666666666666667	0.00666666666666667	\\
3.05e-09	0.216	0.00666666666666667	0.00666666666666667	\\
3.1e-09	0.216	0.00666666666666667	0.00666666666666667	\\
3.15e-09	0.216	0	0	\\
3.2e-09	0.216	0	0	\\
3.25e-09	0.216	0	0	\\
3.3e-09	0.216	0	0	\\
3.35e-09	0.216	0	0	\\
3.4e-09	0.216	0	0	\\
3.45e-09	0.216	0	0	\\
3.5e-09	0.216	0	0	\\
3.55e-09	0.216	0	0	\\
3.6e-09	0.216	0	0	\\
3.65e-09	0.216	0	0	\\
3.7e-09	0.216	0	0	\\
3.75e-09	0.216	0	0	\\
3.8e-09	0.216	0	0	\\
3.85e-09	0.216	0	0	\\
3.9e-09	0.216	0	0	\\
3.95e-09	0.216	0	0	\\
4e-09	0.216	0	0	\\
4.05e-09	0.216	0	0	\\
4.1e-09	0.216	0	0	\\
4.15e-09	0.216	0	0	\\
4.2e-09	0.216	0	0	\\
4.25e-09	0.216	0	0	\\
4.3e-09	0.216	0	0	\\
4.35e-09	0.216	0	0	\\
4.4e-09	0.216	0	0	\\
4.45e-09	0.216	0	0	\\
4.5e-09	0.216	0	0	\\
4.55e-09	0.216	0	0	\\
4.6e-09	0.216	-0.00444444444444444	-0.00444444444444444	\\
4.65e-09	0.216	-0.00444444444444444	-0.00444444444444444	\\
4.7e-09	0.216	-0.00444444444444444	-0.00444444444444444	\\
4.75e-09	0.216	-0.00444444444444444	-0.00444444444444444	\\
4.8e-09	0.216	-0.00444444444444444	-0.00444444444444444	\\
4.85e-09	0.216	-0.00444444444444444	-0.00444444444444444	\\
4.9e-09	0.216	-0.00444444444444444	-0.00444444444444444	\\
4.95e-09	0.216	-0.00444444444444444	-0.00444444444444444	\\
5e-09	0.216	-0.00444444444444444	-0.00444444444444444	\\
5e-09	0.216	-0.00444444444444444	nan	\\
5e-09	0.216	-0.00666666666666667	0.00166666666666667	\\
5e-09	0.216	-0.00666666666666667	nan	\\
0	0.228	-0.00666666666666667	nan	\\
0	0.228	0	0.00166666666666667	\\
0	0.228	0	0	\\
5e-11	0.228	0	0	\\
1e-10	0.228	0	0	\\
1.5e-10	0.228	0	0	\\
2e-10	0.228	0	0	\\
2.5e-10	0.228	0.01	0.01	\\
3e-10	0.228	0.01	0.01	\\
3.5e-10	0.228	0.01	0.01	\\
4e-10	0.228	0.01	0.01	\\
4.5e-10	0.228	0.01	0.01	\\
5e-10	0.228	0.01	0.01	\\
5.5e-10	0.228	0.01	0.01	\\
6e-10	0.228	0.01	0.01	\\
6.5e-10	0.228	0.01	0.01	\\
7e-10	0.228	0.01	0.01	\\
7.5e-10	0.228	0	0	\\
8e-10	0.228	0	0	\\
8.5e-10	0.228	0	0	\\
9e-10	0.228	0	0	\\
9.5e-10	0.228	0	0	\\
1e-09	0.228	0	0	\\
1.05e-09	0.228	0	0	\\
1.1e-09	0.228	0	0	\\
1.15e-09	0.228	0	0	\\
1.2e-09	0.228	0	0	\\
1.25e-09	0.228	0	0	\\
1.3e-09	0.228	0	0	\\
1.35e-09	0.228	0	0	\\
1.4e-09	0.228	0	0	\\
1.45e-09	0.228	0	0	\\
1.5e-09	0.228	0	0	\\
1.55e-09	0.228	0	0	\\
1.6e-09	0.228	0	0	\\
1.65e-09	0.228	0	0	\\
1.7e-09	0.228	0	0	\\
1.75e-09	0.228	0	0	\\
1.8e-09	0.228	0	0	\\
1.85e-09	0.228	0	0	\\
1.9e-09	0.228	0	0	\\
1.95e-09	0.228	0	0	\\
2e-09	0.228	0	0	\\
2.05e-09	0.228	0	0	\\
2.1e-09	0.228	0	0	\\
2.15e-09	0.228	0	0	\\
2.2e-09	0.228	-0.00666666666666667	-0.00666666666666667	\\
2.25e-09	0.228	-0.00666666666666667	-0.00666666666666667	\\
2.3e-09	0.228	-0.00666666666666667	-0.00666666666666667	\\
2.35e-09	0.228	-0.00666666666666667	-0.00666666666666667	\\
2.4e-09	0.228	-0.00666666666666667	-0.00666666666666667	\\
2.45e-09	0.228	-0.00666666666666667	-0.00666666666666667	\\
2.5e-09	0.228	-0.00666666666666667	-0.00666666666666667	\\
2.55e-09	0.228	-0.00666666666666667	-0.00666666666666667	\\
2.6e-09	0.228	-0.00666666666666667	-0.00666666666666667	\\
2.65e-09	0.228	0	0	\\
2.7e-09	0.228	0.00666666666666667	0.00666666666666667	\\
2.75e-09	0.228	0.00666666666666667	0.00666666666666667	\\
2.8e-09	0.228	0.00666666666666667	0.00666666666666667	\\
2.85e-09	0.228	0.00666666666666667	0.00666666666666667	\\
2.9e-09	0.228	0.00666666666666667	0.00666666666666667	\\
2.95e-09	0.228	0.00666666666666667	0.00666666666666667	\\
3e-09	0.228	0.00666666666666667	0.00666666666666667	\\
3.05e-09	0.228	0.00666666666666667	0.00666666666666667	\\
3.1e-09	0.228	0.00666666666666667	0.00666666666666667	\\
3.15e-09	0.228	0	0	\\
3.2e-09	0.228	0	0	\\
3.25e-09	0.228	0	0	\\
3.3e-09	0.228	0	0	\\
3.35e-09	0.228	0	0	\\
3.4e-09	0.228	0	0	\\
3.45e-09	0.228	0	0	\\
3.5e-09	0.228	0	0	\\
3.55e-09	0.228	0	0	\\
3.6e-09	0.228	0	0	\\
3.65e-09	0.228	0	0	\\
3.7e-09	0.228	0	0	\\
3.75e-09	0.228	0	0	\\
3.8e-09	0.228	0	0	\\
3.85e-09	0.228	0	0	\\
3.9e-09	0.228	0	0	\\
3.95e-09	0.228	0	0	\\
4e-09	0.228	0	0	\\
4.05e-09	0.228	0	0	\\
4.1e-09	0.228	0	0	\\
4.15e-09	0.228	0	0	\\
4.2e-09	0.228	0	0	\\
4.25e-09	0.228	0	0	\\
4.3e-09	0.228	0	0	\\
4.35e-09	0.228	0	0	\\
4.4e-09	0.228	0	0	\\
4.45e-09	0.228	0	0	\\
4.5e-09	0.228	0	0	\\
4.55e-09	0.228	0	0	\\
4.6e-09	0.228	-0.00444444444444444	-0.00444444444444444	\\
4.65e-09	0.228	-0.00444444444444444	-0.00444444444444444	\\
4.7e-09	0.228	-0.00444444444444444	-0.00444444444444444	\\
4.75e-09	0.228	-0.00444444444444444	-0.00444444444444444	\\
4.8e-09	0.228	-0.00444444444444444	-0.00444444444444444	\\
4.85e-09	0.228	-0.00444444444444444	-0.00444444444444444	\\
4.9e-09	0.228	-0.00444444444444444	-0.00444444444444444	\\
4.95e-09	0.228	-0.00444444444444444	-0.00444444444444444	\\
5e-09	0.228	-0.00444444444444444	-0.00444444444444444	\\
5e-09	0.228	-0.00444444444444444	nan	\\
5e-09	0.228	-0.00666666666666667	0.00166666666666667	\\
5e-09	0.228	-0.00666666666666667	nan	\\
0	0.24	-0.00666666666666667	nan	\\
0	0.24	0	0.00166666666666667	\\
0	0.24	0	0	\\
5e-11	0.24	0	0	\\
1e-10	0.24	0	0	\\
1.5e-10	0.24	0	0	\\
2e-10	0.24	0	0	\\
2.5e-10	0.24	0.01	0.01	\\
3e-10	0.24	0.01	0.01	\\
3.5e-10	0.24	0.01	0.01	\\
4e-10	0.24	0.01	0.01	\\
4.5e-10	0.24	0.01	0.01	\\
5e-10	0.24	0.01	0.01	\\
5.5e-10	0.24	0.01	0.01	\\
6e-10	0.24	0.01	0.01	\\
6.5e-10	0.24	0.01	0.01	\\
7e-10	0.24	0.01	0.01	\\
7.5e-10	0.24	0	0	\\
8e-10	0.24	0	0	\\
8.5e-10	0.24	0	0	\\
9e-10	0.24	0	0	\\
9.5e-10	0.24	0	0	\\
1e-09	0.24	0	0	\\
1.05e-09	0.24	0	0	\\
1.1e-09	0.24	0	0	\\
1.15e-09	0.24	0	0	\\
1.2e-09	0.24	0	0	\\
1.25e-09	0.24	0	0	\\
1.3e-09	0.24	0	0	\\
1.35e-09	0.24	0	0	\\
1.4e-09	0.24	0	0	\\
1.45e-09	0.24	0	0	\\
1.5e-09	0.24	0	0	\\
1.55e-09	0.24	0	0	\\
1.6e-09	0.24	0	0	\\
1.65e-09	0.24	0	0	\\
1.7e-09	0.24	0	0	\\
1.75e-09	0.24	0	0	\\
1.8e-09	0.24	0	0	\\
1.85e-09	0.24	0	0	\\
1.9e-09	0.24	0	0	\\
1.95e-09	0.24	0	0	\\
2e-09	0.24	0	0	\\
2.05e-09	0.24	0	0	\\
2.1e-09	0.24	0	0	\\
2.15e-09	0.24	0	0	\\
2.2e-09	0.24	-0.00666666666666667	-0.00666666666666667	\\
2.25e-09	0.24	-0.00666666666666667	-0.00666666666666667	\\
2.3e-09	0.24	-0.00666666666666667	-0.00666666666666667	\\
2.35e-09	0.24	-0.00666666666666667	-0.00666666666666667	\\
2.4e-09	0.24	-0.00666666666666667	-0.00666666666666667	\\
2.45e-09	0.24	-0.00666666666666667	-0.00666666666666667	\\
2.5e-09	0.24	-0.00666666666666667	-0.00666666666666667	\\
2.55e-09	0.24	-0.00666666666666667	-0.00666666666666667	\\
2.6e-09	0.24	-0.00666666666666667	-0.00666666666666667	\\
2.65e-09	0.24	0	0	\\
2.7e-09	0.24	0.00666666666666667	0.00666666666666667	\\
2.75e-09	0.24	0.00666666666666667	0.00666666666666667	\\
2.8e-09	0.24	0.00666666666666667	0.00666666666666667	\\
2.85e-09	0.24	0.00666666666666667	0.00666666666666667	\\
2.9e-09	0.24	0.00666666666666667	0.00666666666666667	\\
2.95e-09	0.24	0.00666666666666667	0.00666666666666667	\\
3e-09	0.24	0.00666666666666667	0.00666666666666667	\\
3.05e-09	0.24	0.00666666666666667	0.00666666666666667	\\
3.1e-09	0.24	0.00666666666666667	0.00666666666666667	\\
3.15e-09	0.24	0	0	\\
3.2e-09	0.24	0	0	\\
3.25e-09	0.24	0	0	\\
3.3e-09	0.24	0	0	\\
3.35e-09	0.24	0	0	\\
3.4e-09	0.24	0	0	\\
3.45e-09	0.24	0	0	\\
3.5e-09	0.24	0	0	\\
3.55e-09	0.24	0	0	\\
3.6e-09	0.24	0	0	\\
3.65e-09	0.24	0	0	\\
3.7e-09	0.24	0	0	\\
3.75e-09	0.24	0	0	\\
3.8e-09	0.24	0	0	\\
3.85e-09	0.24	0	0	\\
3.9e-09	0.24	0	0	\\
3.95e-09	0.24	0	0	\\
4e-09	0.24	0	0	\\
4.05e-09	0.24	0	0	\\
4.1e-09	0.24	0	0	\\
4.15e-09	0.24	0	0	\\
4.2e-09	0.24	0	0	\\
4.25e-09	0.24	0	0	\\
4.3e-09	0.24	0	0	\\
4.35e-09	0.24	0	0	\\
4.4e-09	0.24	0	0	\\
4.45e-09	0.24	0	0	\\
4.5e-09	0.24	0	0	\\
4.55e-09	0.24	0	0	\\
4.6e-09	0.24	-0.00444444444444444	-0.00444444444444444	\\
4.65e-09	0.24	-0.00444444444444444	-0.00444444444444444	\\
4.7e-09	0.24	-0.00444444444444444	-0.00444444444444444	\\
4.75e-09	0.24	-0.00444444444444444	-0.00444444444444444	\\
4.8e-09	0.24	-0.00444444444444444	-0.00444444444444444	\\
4.85e-09	0.24	-0.00444444444444444	-0.00444444444444444	\\
4.9e-09	0.24	-0.00444444444444444	-0.00444444444444444	\\
4.95e-09	0.24	-0.00444444444444444	-0.00444444444444444	\\
5e-09	0.24	-0.00444444444444444	-0.00444444444444444	\\
5e-09	0.24	-0.00444444444444444	nan	\\
5e-09	0.24	-0.00666666666666667	0.00166666666666667	\\
5e-09	0.24	-0.00666666666666667	nan	\\
0	0.252	-0.00666666666666667	nan	\\
0	0.252	0	0.00166666666666667	\\
0	0.252	0	0	\\
5e-11	0.252	0	0	\\
1e-10	0.252	0	0	\\
1.5e-10	0.252	0	0	\\
2e-10	0.252	0	0	\\
2.5e-10	0.252	0	0	\\
3e-10	0.252	0.01	0.01	\\
3.5e-10	0.252	0.01	0.01	\\
4e-10	0.252	0.01	0.01	\\
4.5e-10	0.252	0.01	0.01	\\
5e-10	0.252	0.01	0.01	\\
5.5e-10	0.252	0.01	0.01	\\
6e-10	0.252	0.01	0.01	\\
6.5e-10	0.252	0.01	0.01	\\
7e-10	0.252	0.01	0.01	\\
7.5e-10	0.252	0.01	0.01	\\
8e-10	0.252	0	0	\\
8.5e-10	0.252	0	0	\\
9e-10	0.252	0	0	\\
9.5e-10	0.252	0	0	\\
1e-09	0.252	0	0	\\
1.05e-09	0.252	0	0	\\
1.1e-09	0.252	0	0	\\
1.15e-09	0.252	0	0	\\
1.2e-09	0.252	0	0	\\
1.25e-09	0.252	0	0	\\
1.3e-09	0.252	0	0	\\
1.35e-09	0.252	0	0	\\
1.4e-09	0.252	0	0	\\
1.45e-09	0.252	0	0	\\
1.5e-09	0.252	0	0	\\
1.55e-09	0.252	0	0	\\
1.6e-09	0.252	0	0	\\
1.65e-09	0.252	0	0	\\
1.7e-09	0.252	0	0	\\
1.75e-09	0.252	0	0	\\
1.8e-09	0.252	0	0	\\
1.85e-09	0.252	0	0	\\
1.9e-09	0.252	0	0	\\
1.95e-09	0.252	0	0	\\
2e-09	0.252	0	0	\\
2.05e-09	0.252	0	0	\\
2.1e-09	0.252	0	0	\\
2.15e-09	0.252	-0.00666666666666667	-0.00666666666666667	\\
2.2e-09	0.252	-0.00666666666666667	-0.00666666666666667	\\
2.25e-09	0.252	-0.00666666666666667	-0.00666666666666667	\\
2.3e-09	0.252	-0.00666666666666667	-0.00666666666666667	\\
2.35e-09	0.252	-0.00666666666666667	-0.00666666666666667	\\
2.4e-09	0.252	-0.00666666666666667	-0.00666666666666667	\\
2.45e-09	0.252	-0.00666666666666667	-0.00666666666666667	\\
2.5e-09	0.252	-0.00666666666666667	-0.00666666666666667	\\
2.55e-09	0.252	-0.00666666666666667	-0.00666666666666667	\\
2.6e-09	0.252	-0.00666666666666667	-0.00666666666666667	\\
2.65e-09	0.252	0	0	\\
2.7e-09	0.252	0.00666666666666667	0.00666666666666667	\\
2.75e-09	0.252	0.00666666666666667	0.00666666666666667	\\
2.8e-09	0.252	0.00666666666666667	0.00666666666666667	\\
2.85e-09	0.252	0.00666666666666667	0.00666666666666667	\\
2.9e-09	0.252	0.00666666666666667	0.00666666666666667	\\
2.95e-09	0.252	0.00666666666666667	0.00666666666666667	\\
3e-09	0.252	0.00666666666666667	0.00666666666666667	\\
3.05e-09	0.252	0.00666666666666667	0.00666666666666667	\\
3.1e-09	0.252	0.00666666666666667	0.00666666666666667	\\
3.15e-09	0.252	0.00666666666666667	0.00666666666666667	\\
3.2e-09	0.252	0	0	\\
3.25e-09	0.252	0	0	\\
3.3e-09	0.252	0	0	\\
3.35e-09	0.252	0	0	\\
3.4e-09	0.252	0	0	\\
3.45e-09	0.252	0	0	\\
3.5e-09	0.252	0	0	\\
3.55e-09	0.252	0	0	\\
3.6e-09	0.252	0	0	\\
3.65e-09	0.252	0	0	\\
3.7e-09	0.252	0	0	\\
3.75e-09	0.252	0	0	\\
3.8e-09	0.252	0	0	\\
3.85e-09	0.252	0	0	\\
3.9e-09	0.252	0	0	\\
3.95e-09	0.252	0	0	\\
4e-09	0.252	0	0	\\
4.05e-09	0.252	0	0	\\
4.1e-09	0.252	0	0	\\
4.15e-09	0.252	0	0	\\
4.2e-09	0.252	0	0	\\
4.25e-09	0.252	0	0	\\
4.3e-09	0.252	0	0	\\
4.35e-09	0.252	0	0	\\
4.4e-09	0.252	0	0	\\
4.45e-09	0.252	0	0	\\
4.5e-09	0.252	0	0	\\
4.55e-09	0.252	-0.00444444444444444	-0.00444444444444444	\\
4.6e-09	0.252	-0.00444444444444444	-0.00444444444444444	\\
4.65e-09	0.252	-0.00444444444444444	-0.00444444444444444	\\
4.7e-09	0.252	-0.00444444444444444	-0.00444444444444444	\\
4.75e-09	0.252	-0.00444444444444444	-0.00444444444444444	\\
4.8e-09	0.252	-0.00444444444444444	-0.00444444444444444	\\
4.85e-09	0.252	-0.00444444444444444	-0.00444444444444444	\\
4.9e-09	0.252	-0.00444444444444444	-0.00444444444444444	\\
4.95e-09	0.252	-0.00444444444444444	-0.00444444444444444	\\
5e-09	0.252	-0.00444444444444444	-0.00444444444444444	\\
5e-09	0.252	-0.00444444444444444	nan	\\
5e-09	0.252	-0.00666666666666667	0.00166666666666667	\\
5e-09	0.252	-0.00666666666666667	nan	\\
0	0.264	-0.00666666666666667	nan	\\
0	0.264	0	0.00166666666666667	\\
0	0.264	0	0	\\
5e-11	0.264	0	0	\\
1e-10	0.264	0	0	\\
1.5e-10	0.264	0	0	\\
2e-10	0.264	0	0	\\
2.5e-10	0.264	0	0	\\
3e-10	0.264	0.01	0.01	\\
3.5e-10	0.264	0.01	0.01	\\
4e-10	0.264	0.01	0.01	\\
4.5e-10	0.264	0.01	0.01	\\
5e-10	0.264	0.01	0.01	\\
5.5e-10	0.264	0.01	0.01	\\
6e-10	0.264	0.01	0.01	\\
6.5e-10	0.264	0.01	0.01	\\
7e-10	0.264	0.01	0.01	\\
7.5e-10	0.264	0.01	0.01	\\
8e-10	0.264	0	0	\\
8.5e-10	0.264	0	0	\\
9e-10	0.264	0	0	\\
9.5e-10	0.264	0	0	\\
1e-09	0.264	0	0	\\
1.05e-09	0.264	0	0	\\
1.1e-09	0.264	0	0	\\
1.15e-09	0.264	0	0	\\
1.2e-09	0.264	0	0	\\
1.25e-09	0.264	0	0	\\
1.3e-09	0.264	0	0	\\
1.35e-09	0.264	0	0	\\
1.4e-09	0.264	0	0	\\
1.45e-09	0.264	0	0	\\
1.5e-09	0.264	0	0	\\
1.55e-09	0.264	0	0	\\
1.6e-09	0.264	0	0	\\
1.65e-09	0.264	0	0	\\
1.7e-09	0.264	0	0	\\
1.75e-09	0.264	0	0	\\
1.8e-09	0.264	0	0	\\
1.85e-09	0.264	0	0	\\
1.9e-09	0.264	0	0	\\
1.95e-09	0.264	0	0	\\
2e-09	0.264	0	0	\\
2.05e-09	0.264	0	0	\\
2.1e-09	0.264	0	0	\\
2.15e-09	0.264	-0.00666666666666667	-0.00666666666666667	\\
2.2e-09	0.264	-0.00666666666666667	-0.00666666666666667	\\
2.25e-09	0.264	-0.00666666666666667	-0.00666666666666667	\\
2.3e-09	0.264	-0.00666666666666667	-0.00666666666666667	\\
2.35e-09	0.264	-0.00666666666666667	-0.00666666666666667	\\
2.4e-09	0.264	-0.00666666666666667	-0.00666666666666667	\\
2.45e-09	0.264	-0.00666666666666667	-0.00666666666666667	\\
2.5e-09	0.264	-0.00666666666666667	-0.00666666666666667	\\
2.55e-09	0.264	-0.00666666666666667	-0.00666666666666667	\\
2.6e-09	0.264	-0.00666666666666667	-0.00666666666666667	\\
2.65e-09	0.264	0	0	\\
2.7e-09	0.264	0.00666666666666667	0.00666666666666667	\\
2.75e-09	0.264	0.00666666666666667	0.00666666666666667	\\
2.8e-09	0.264	0.00666666666666667	0.00666666666666667	\\
2.85e-09	0.264	0.00666666666666667	0.00666666666666667	\\
2.9e-09	0.264	0.00666666666666667	0.00666666666666667	\\
2.95e-09	0.264	0.00666666666666667	0.00666666666666667	\\
3e-09	0.264	0.00666666666666667	0.00666666666666667	\\
3.05e-09	0.264	0.00666666666666667	0.00666666666666667	\\
3.1e-09	0.264	0.00666666666666667	0.00666666666666667	\\
3.15e-09	0.264	0.00666666666666667	0.00666666666666667	\\
3.2e-09	0.264	0	0	\\
3.25e-09	0.264	0	0	\\
3.3e-09	0.264	0	0	\\
3.35e-09	0.264	0	0	\\
3.4e-09	0.264	0	0	\\
3.45e-09	0.264	0	0	\\
3.5e-09	0.264	0	0	\\
3.55e-09	0.264	0	0	\\
3.6e-09	0.264	0	0	\\
3.65e-09	0.264	0	0	\\
3.7e-09	0.264	0	0	\\
3.75e-09	0.264	0	0	\\
3.8e-09	0.264	0	0	\\
3.85e-09	0.264	0	0	\\
3.9e-09	0.264	0	0	\\
3.95e-09	0.264	0	0	\\
4e-09	0.264	0	0	\\
4.05e-09	0.264	0	0	\\
4.1e-09	0.264	0	0	\\
4.15e-09	0.264	0	0	\\
4.2e-09	0.264	0	0	\\
4.25e-09	0.264	0	0	\\
4.3e-09	0.264	0	0	\\
4.35e-09	0.264	0	0	\\
4.4e-09	0.264	0	0	\\
4.45e-09	0.264	0	0	\\
4.5e-09	0.264	0	0	\\
4.55e-09	0.264	-0.00444444444444444	-0.00444444444444444	\\
4.6e-09	0.264	-0.00444444444444444	-0.00444444444444444	\\
4.65e-09	0.264	-0.00444444444444444	-0.00444444444444444	\\
4.7e-09	0.264	-0.00444444444444444	-0.00444444444444444	\\
4.75e-09	0.264	-0.00444444444444444	-0.00444444444444444	\\
4.8e-09	0.264	-0.00444444444444444	-0.00444444444444444	\\
4.85e-09	0.264	-0.00444444444444444	-0.00444444444444444	\\
4.9e-09	0.264	-0.00444444444444444	-0.00444444444444444	\\
4.95e-09	0.264	-0.00444444444444444	-0.00444444444444444	\\
5e-09	0.264	-0.00444444444444444	-0.00444444444444444	\\
5e-09	0.264	-0.00444444444444444	nan	\\
5e-09	0.264	-0.00666666666666667	0.00166666666666667	\\
5e-09	0.264	-0.00666666666666667	nan	\\
0	0.276	-0.00666666666666667	nan	\\
0	0.276	0	0.00166666666666667	\\
0	0.276	0	0	\\
5e-11	0.276	0	0	\\
1e-10	0.276	0	0	\\
1.5e-10	0.276	0	0	\\
2e-10	0.276	0	0	\\
2.5e-10	0.276	0	0	\\
3e-10	0.276	0.01	0.01	\\
3.5e-10	0.276	0.01	0.01	\\
4e-10	0.276	0.01	0.01	\\
4.5e-10	0.276	0.01	0.01	\\
5e-10	0.276	0.01	0.01	\\
5.5e-10	0.276	0.01	0.01	\\
6e-10	0.276	0.01	0.01	\\
6.5e-10	0.276	0.01	0.01	\\
7e-10	0.276	0.01	0.01	\\
7.5e-10	0.276	0.01	0.01	\\
8e-10	0.276	0	0	\\
8.5e-10	0.276	0	0	\\
9e-10	0.276	0	0	\\
9.5e-10	0.276	0	0	\\
1e-09	0.276	0	0	\\
1.05e-09	0.276	0	0	\\
1.1e-09	0.276	0	0	\\
1.15e-09	0.276	0	0	\\
1.2e-09	0.276	0	0	\\
1.25e-09	0.276	0	0	\\
1.3e-09	0.276	0	0	\\
1.35e-09	0.276	0	0	\\
1.4e-09	0.276	0	0	\\
1.45e-09	0.276	0	0	\\
1.5e-09	0.276	0	0	\\
1.55e-09	0.276	0	0	\\
1.6e-09	0.276	0	0	\\
1.65e-09	0.276	0	0	\\
1.7e-09	0.276	0	0	\\
1.75e-09	0.276	0	0	\\
1.8e-09	0.276	0	0	\\
1.85e-09	0.276	0	0	\\
1.9e-09	0.276	0	0	\\
1.95e-09	0.276	0	0	\\
2e-09	0.276	0	0	\\
2.05e-09	0.276	0	0	\\
2.1e-09	0.276	0	0	\\
2.15e-09	0.276	-0.00666666666666667	-0.00666666666666667	\\
2.2e-09	0.276	-0.00666666666666667	-0.00666666666666667	\\
2.25e-09	0.276	-0.00666666666666667	-0.00666666666666667	\\
2.3e-09	0.276	-0.00666666666666667	-0.00666666666666667	\\
2.35e-09	0.276	-0.00666666666666667	-0.00666666666666667	\\
2.4e-09	0.276	-0.00666666666666667	-0.00666666666666667	\\
2.45e-09	0.276	-0.00666666666666667	-0.00666666666666667	\\
2.5e-09	0.276	-0.00666666666666667	-0.00666666666666667	\\
2.55e-09	0.276	-0.00666666666666667	-0.00666666666666667	\\
2.6e-09	0.276	-0.00666666666666667	-0.00666666666666667	\\
2.65e-09	0.276	0	0	\\
2.7e-09	0.276	0.00666666666666667	0.00666666666666667	\\
2.75e-09	0.276	0.00666666666666667	0.00666666666666667	\\
2.8e-09	0.276	0.00666666666666667	0.00666666666666667	\\
2.85e-09	0.276	0.00666666666666667	0.00666666666666667	\\
2.9e-09	0.276	0.00666666666666667	0.00666666666666667	\\
2.95e-09	0.276	0.00666666666666667	0.00666666666666667	\\
3e-09	0.276	0.00666666666666667	0.00666666666666667	\\
3.05e-09	0.276	0.00666666666666667	0.00666666666666667	\\
3.1e-09	0.276	0.00666666666666667	0.00666666666666667	\\
3.15e-09	0.276	0.00666666666666667	0.00666666666666667	\\
3.2e-09	0.276	0	0	\\
3.25e-09	0.276	0	0	\\
3.3e-09	0.276	0	0	\\
3.35e-09	0.276	0	0	\\
3.4e-09	0.276	0	0	\\
3.45e-09	0.276	0	0	\\
3.5e-09	0.276	0	0	\\
3.55e-09	0.276	0	0	\\
3.6e-09	0.276	0	0	\\
3.65e-09	0.276	0	0	\\
3.7e-09	0.276	0	0	\\
3.75e-09	0.276	0	0	\\
3.8e-09	0.276	0	0	\\
3.85e-09	0.276	0	0	\\
3.9e-09	0.276	0	0	\\
3.95e-09	0.276	0	0	\\
4e-09	0.276	0	0	\\
4.05e-09	0.276	0	0	\\
4.1e-09	0.276	0	0	\\
4.15e-09	0.276	0	0	\\
4.2e-09	0.276	0	0	\\
4.25e-09	0.276	0	0	\\
4.3e-09	0.276	0	0	\\
4.35e-09	0.276	0	0	\\
4.4e-09	0.276	0	0	\\
4.45e-09	0.276	0	0	\\
4.5e-09	0.276	0	0	\\
4.55e-09	0.276	-0.00444444444444444	-0.00444444444444444	\\
4.6e-09	0.276	-0.00444444444444444	-0.00444444444444444	\\
4.65e-09	0.276	-0.00444444444444444	-0.00444444444444444	\\
4.7e-09	0.276	-0.00444444444444444	-0.00444444444444444	\\
4.75e-09	0.276	-0.00444444444444444	-0.00444444444444444	\\
4.8e-09	0.276	-0.00444444444444444	-0.00444444444444444	\\
4.85e-09	0.276	-0.00444444444444444	-0.00444444444444444	\\
4.9e-09	0.276	-0.00444444444444444	-0.00444444444444444	\\
4.95e-09	0.276	-0.00444444444444444	-0.00444444444444444	\\
5e-09	0.276	-0.00444444444444444	-0.00444444444444444	\\
5e-09	0.276	-0.00444444444444444	nan	\\
5e-09	0.276	-0.00666666666666667	0.00166666666666667	\\
5e-09	0.276	-0.00666666666666667	nan	\\
0	0.288	-0.00666666666666667	nan	\\
0	0.288	0	0.00166666666666667	\\
0	0.288	0	0	\\
5e-11	0.288	0	0	\\
1e-10	0.288	0	0	\\
1.5e-10	0.288	0	0	\\
2e-10	0.288	0	0	\\
2.5e-10	0.288	0	0	\\
3e-10	0.288	0.01	0.01	\\
3.5e-10	0.288	0.01	0.01	\\
4e-10	0.288	0.01	0.01	\\
4.5e-10	0.288	0.01	0.01	\\
5e-10	0.288	0.01	0.01	\\
5.5e-10	0.288	0.01	0.01	\\
6e-10	0.288	0.01	0.01	\\
6.5e-10	0.288	0.01	0.01	\\
7e-10	0.288	0.01	0.01	\\
7.5e-10	0.288	0.01	0.01	\\
8e-10	0.288	0	0	\\
8.5e-10	0.288	0	0	\\
9e-10	0.288	0	0	\\
9.5e-10	0.288	0	0	\\
1e-09	0.288	0	0	\\
1.05e-09	0.288	0	0	\\
1.1e-09	0.288	0	0	\\
1.15e-09	0.288	0	0	\\
1.2e-09	0.288	0	0	\\
1.25e-09	0.288	0	0	\\
1.3e-09	0.288	0	0	\\
1.35e-09	0.288	0	0	\\
1.4e-09	0.288	0	0	\\
1.45e-09	0.288	0	0	\\
1.5e-09	0.288	0	0	\\
1.55e-09	0.288	0	0	\\
1.6e-09	0.288	0	0	\\
1.65e-09	0.288	0	0	\\
1.7e-09	0.288	0	0	\\
1.75e-09	0.288	0	0	\\
1.8e-09	0.288	0	0	\\
1.85e-09	0.288	0	0	\\
1.9e-09	0.288	0	0	\\
1.95e-09	0.288	0	0	\\
2e-09	0.288	0	0	\\
2.05e-09	0.288	0	0	\\
2.1e-09	0.288	0	0	\\
2.15e-09	0.288	-0.00666666666666667	-0.00666666666666667	\\
2.2e-09	0.288	-0.00666666666666667	-0.00666666666666667	\\
2.25e-09	0.288	-0.00666666666666667	-0.00666666666666667	\\
2.3e-09	0.288	-0.00666666666666667	-0.00666666666666667	\\
2.35e-09	0.288	-0.00666666666666667	-0.00666666666666667	\\
2.4e-09	0.288	-0.00666666666666667	-0.00666666666666667	\\
2.45e-09	0.288	-0.00666666666666667	-0.00666666666666667	\\
2.5e-09	0.288	-0.00666666666666667	-0.00666666666666667	\\
2.55e-09	0.288	-0.00666666666666667	-0.00666666666666667	\\
2.6e-09	0.288	-0.00666666666666667	-0.00666666666666667	\\
2.65e-09	0.288	0	0	\\
2.7e-09	0.288	0.00666666666666667	0.00666666666666667	\\
2.75e-09	0.288	0.00666666666666667	0.00666666666666667	\\
2.8e-09	0.288	0.00666666666666667	0.00666666666666667	\\
2.85e-09	0.288	0.00666666666666667	0.00666666666666667	\\
2.9e-09	0.288	0.00666666666666667	0.00666666666666667	\\
2.95e-09	0.288	0.00666666666666667	0.00666666666666667	\\
3e-09	0.288	0.00666666666666667	0.00666666666666667	\\
3.05e-09	0.288	0.00666666666666667	0.00666666666666667	\\
3.1e-09	0.288	0.00666666666666667	0.00666666666666667	\\
3.15e-09	0.288	0.00666666666666667	0.00666666666666667	\\
3.2e-09	0.288	0	0	\\
3.25e-09	0.288	0	0	\\
3.3e-09	0.288	0	0	\\
3.35e-09	0.288	0	0	\\
3.4e-09	0.288	0	0	\\
3.45e-09	0.288	0	0	\\
3.5e-09	0.288	0	0	\\
3.55e-09	0.288	0	0	\\
3.6e-09	0.288	0	0	\\
3.65e-09	0.288	0	0	\\
3.7e-09	0.288	0	0	\\
3.75e-09	0.288	0	0	\\
3.8e-09	0.288	0	0	\\
3.85e-09	0.288	0	0	\\
3.9e-09	0.288	0	0	\\
3.95e-09	0.288	0	0	\\
4e-09	0.288	0	0	\\
4.05e-09	0.288	0	0	\\
4.1e-09	0.288	0	0	\\
4.15e-09	0.288	0	0	\\
4.2e-09	0.288	0	0	\\
4.25e-09	0.288	0	0	\\
4.3e-09	0.288	0	0	\\
4.35e-09	0.288	0	0	\\
4.4e-09	0.288	0	0	\\
4.45e-09	0.288	0	0	\\
4.5e-09	0.288	0	0	\\
4.55e-09	0.288	-0.00444444444444444	-0.00444444444444444	\\
4.6e-09	0.288	-0.00444444444444444	-0.00444444444444444	\\
4.65e-09	0.288	-0.00444444444444444	-0.00444444444444444	\\
4.7e-09	0.288	-0.00444444444444444	-0.00444444444444444	\\
4.75e-09	0.288	-0.00444444444444444	-0.00444444444444444	\\
4.8e-09	0.288	-0.00444444444444444	-0.00444444444444444	\\
4.85e-09	0.288	-0.00444444444444444	-0.00444444444444444	\\
4.9e-09	0.288	-0.00444444444444444	-0.00444444444444444	\\
4.95e-09	0.288	-0.00444444444444444	-0.00444444444444444	\\
5e-09	0.288	-0.00444444444444444	-0.00444444444444444	\\
5e-09	0.288	-0.00444444444444444	nan	\\
5e-09	0.288	-0.00666666666666667	0.00166666666666667	\\
5e-09	0.288	-0.00666666666666667	nan	\\
0	0.3	-0.00666666666666667	nan	\\
0	0.3	0	0.00166666666666667	\\
0	0.3	0	0	\\
5e-11	0.3	0	0	\\
1e-10	0.3	0	0	\\
1.5e-10	0.3	0	0	\\
2e-10	0.3	0	0	\\
2.5e-10	0.3	0	0	\\
3e-10	0.3	0.01	0.01	\\
3.5e-10	0.3	0.01	0.01	\\
4e-10	0.3	0.01	0.01	\\
4.5e-10	0.3	0.01	0.01	\\
5e-10	0.3	0.01	0.01	\\
5.5e-10	0.3	0.01	0.01	\\
6e-10	0.3	0.01	0.01	\\
6.5e-10	0.3	0.01	0.01	\\
7e-10	0.3	0.01	0.01	\\
7.5e-10	0.3	0.01	0.01	\\
8e-10	0.3	0	0	\\
8.5e-10	0.3	0	0	\\
9e-10	0.3	0	0	\\
9.5e-10	0.3	0	0	\\
1e-09	0.3	0	0	\\
1.05e-09	0.3	0	0	\\
1.1e-09	0.3	0	0	\\
1.15e-09	0.3	0	0	\\
1.2e-09	0.3	0	0	\\
1.25e-09	0.3	0	0	\\
1.3e-09	0.3	0	0	\\
1.35e-09	0.3	0	0	\\
1.4e-09	0.3	0	0	\\
1.45e-09	0.3	0	0	\\
1.5e-09	0.3	0	0	\\
1.55e-09	0.3	0	0	\\
1.6e-09	0.3	0	0	\\
1.65e-09	0.3	0	0	\\
1.7e-09	0.3	0	0	\\
1.75e-09	0.3	0	0	\\
1.8e-09	0.3	0	0	\\
1.85e-09	0.3	0	0	\\
1.9e-09	0.3	0	0	\\
1.95e-09	0.3	0	0	\\
2e-09	0.3	0	0	\\
2.05e-09	0.3	0	0	\\
2.1e-09	0.3	-0.00666666666666667	-0.00666666666666667	\\
2.15e-09	0.3	-0.00666666666666667	-0.00666666666666667	\\
2.2e-09	0.3	-0.00666666666666667	-0.00666666666666667	\\
2.25e-09	0.3	-0.00666666666666667	-0.00666666666666667	\\
2.3e-09	0.3	-0.00666666666666667	-0.00666666666666667	\\
2.35e-09	0.3	-0.00666666666666667	-0.00666666666666667	\\
2.4e-09	0.3	-0.00666666666666667	-0.00666666666666667	\\
2.45e-09	0.3	-0.00666666666666667	-0.00666666666666667	\\
2.5e-09	0.3	-0.00666666666666667	-0.00666666666666667	\\
2.55e-09	0.3	-0.00666666666666667	-0.00666666666666667	\\
2.6e-09	0.3	0	0	\\
2.65e-09	0.3	0	0	\\
2.7e-09	0.3	0.00666666666666667	0.00666666666666667	\\
2.75e-09	0.3	0.00666666666666667	0.00666666666666667	\\
2.8e-09	0.3	0.00666666666666667	0.00666666666666667	\\
2.85e-09	0.3	0.00666666666666667	0.00666666666666667	\\
2.9e-09	0.3	0.00666666666666667	0.00666666666666667	\\
2.95e-09	0.3	0.00666666666666667	0.00666666666666667	\\
3e-09	0.3	0.00666666666666667	0.00666666666666667	\\
3.05e-09	0.3	0.00666666666666667	0.00666666666666667	\\
3.1e-09	0.3	0.00666666666666667	0.00666666666666667	\\
3.15e-09	0.3	0.00666666666666667	0.00666666666666667	\\
3.2e-09	0.3	0	0	\\
3.25e-09	0.3	0	0	\\
3.3e-09	0.3	0	0	\\
3.35e-09	0.3	0	0	\\
3.4e-09	0.3	0	0	\\
3.45e-09	0.3	0	0	\\
3.5e-09	0.3	0	0	\\
3.55e-09	0.3	0	0	\\
3.6e-09	0.3	0	0	\\
3.65e-09	0.3	0	0	\\
3.7e-09	0.3	0	0	\\
3.75e-09	0.3	0	0	\\
3.8e-09	0.3	0	0	\\
3.85e-09	0.3	0	0	\\
3.9e-09	0.3	0	0	\\
3.95e-09	0.3	0	0	\\
4e-09	0.3	0	0	\\
4.05e-09	0.3	0	0	\\
4.1e-09	0.3	0	0	\\
4.15e-09	0.3	0	0	\\
4.2e-09	0.3	0	0	\\
4.25e-09	0.3	0	0	\\
4.3e-09	0.3	0	0	\\
4.35e-09	0.3	0	0	\\
4.4e-09	0.3	0	0	\\
4.45e-09	0.3	0	0	\\
4.5e-09	0.3	-0.00444444444444444	-0.00444444444444444	\\
4.55e-09	0.3	-0.00444444444444444	-0.00444444444444444	\\
4.6e-09	0.3	-0.00444444444444444	-0.00444444444444444	\\
4.65e-09	0.3	-0.00444444444444444	-0.00444444444444444	\\
4.7e-09	0.3	-0.00444444444444444	-0.00444444444444444	\\
4.75e-09	0.3	-0.00444444444444444	-0.00444444444444444	\\
4.8e-09	0.3	-0.00444444444444444	-0.00444444444444444	\\
4.85e-09	0.3	-0.00444444444444444	-0.00444444444444444	\\
4.9e-09	0.3	-0.00444444444444444	-0.00444444444444444	\\
4.95e-09	0.3	-0.00444444444444444	-0.00444444444444444	\\
5e-09	0.3	0	0	\\
5e-09	0.3	0	nan	\\
5e-09	0.3	-0.00666666666666667	0.00166666666666667	\\
5e-09	0.3	-0.00666666666666667	nan	\\
0	0.312	-0.00666666666666667	nan	\\
0	0.312	0	0.00166666666666667	\\
0	0.312	0	0	\\
5e-11	0.312	0	0	\\
1e-10	0.312	0	0	\\
1.5e-10	0.312	0	0	\\
2e-10	0.312	0	0	\\
2.5e-10	0.312	0	0	\\
3e-10	0.312	0	0	\\
3.5e-10	0.312	0.01	0.01	\\
4e-10	0.312	0.01	0.01	\\
4.5e-10	0.312	0.01	0.01	\\
5e-10	0.312	0.01	0.01	\\
5.5e-10	0.312	0.01	0.01	\\
6e-10	0.312	0.01	0.01	\\
6.5e-10	0.312	0.01	0.01	\\
7e-10	0.312	0.01	0.01	\\
7.5e-10	0.312	0.01	0.01	\\
8e-10	0.312	0.01	0.01	\\
8.5e-10	0.312	0	0	\\
9e-10	0.312	0	0	\\
9.5e-10	0.312	0	0	\\
1e-09	0.312	0	0	\\
1.05e-09	0.312	0	0	\\
1.1e-09	0.312	0	0	\\
1.15e-09	0.312	0	0	\\
1.2e-09	0.312	0	0	\\
1.25e-09	0.312	0	0	\\
1.3e-09	0.312	0	0	\\
1.35e-09	0.312	0	0	\\
1.4e-09	0.312	0	0	\\
1.45e-09	0.312	0	0	\\
1.5e-09	0.312	0	0	\\
1.55e-09	0.312	0	0	\\
1.6e-09	0.312	0	0	\\
1.65e-09	0.312	0	0	\\
1.7e-09	0.312	0	0	\\
1.75e-09	0.312	0	0	\\
1.8e-09	0.312	0	0	\\
1.85e-09	0.312	0	0	\\
1.9e-09	0.312	0	0	\\
1.95e-09	0.312	0	0	\\
2e-09	0.312	0	0	\\
2.05e-09	0.312	0	0	\\
2.1e-09	0.312	-0.00666666666666667	-0.00666666666666667	\\
2.15e-09	0.312	-0.00666666666666667	-0.00666666666666667	\\
2.2e-09	0.312	-0.00666666666666667	-0.00666666666666667	\\
2.25e-09	0.312	-0.00666666666666667	-0.00666666666666667	\\
2.3e-09	0.312	-0.00666666666666667	-0.00666666666666667	\\
2.35e-09	0.312	-0.00666666666666667	-0.00666666666666667	\\
2.4e-09	0.312	-0.00666666666666667	-0.00666666666666667	\\
2.45e-09	0.312	-0.00666666666666667	-0.00666666666666667	\\
2.5e-09	0.312	-0.00666666666666667	-0.00666666666666667	\\
2.55e-09	0.312	-0.00666666666666667	-0.00666666666666667	\\
2.6e-09	0.312	0	0	\\
2.65e-09	0.312	0	0	\\
2.7e-09	0.312	0	0	\\
2.75e-09	0.312	0.00666666666666667	0.00666666666666667	\\
2.8e-09	0.312	0.00666666666666667	0.00666666666666667	\\
2.85e-09	0.312	0.00666666666666667	0.00666666666666667	\\
2.9e-09	0.312	0.00666666666666667	0.00666666666666667	\\
2.95e-09	0.312	0.00666666666666667	0.00666666666666667	\\
3e-09	0.312	0.00666666666666667	0.00666666666666667	\\
3.05e-09	0.312	0.00666666666666667	0.00666666666666667	\\
3.1e-09	0.312	0.00666666666666667	0.00666666666666667	\\
3.15e-09	0.312	0.00666666666666667	0.00666666666666667	\\
3.2e-09	0.312	0.00666666666666667	0.00666666666666667	\\
3.25e-09	0.312	0	0	\\
3.3e-09	0.312	0	0	\\
3.35e-09	0.312	0	0	\\
3.4e-09	0.312	0	0	\\
3.45e-09	0.312	0	0	\\
3.5e-09	0.312	0	0	\\
3.55e-09	0.312	0	0	\\
3.6e-09	0.312	0	0	\\
3.65e-09	0.312	0	0	\\
3.7e-09	0.312	0	0	\\
3.75e-09	0.312	0	0	\\
3.8e-09	0.312	0	0	\\
3.85e-09	0.312	0	0	\\
3.9e-09	0.312	0	0	\\
3.95e-09	0.312	0	0	\\
4e-09	0.312	0	0	\\
4.05e-09	0.312	0	0	\\
4.1e-09	0.312	0	0	\\
4.15e-09	0.312	0	0	\\
4.2e-09	0.312	0	0	\\
4.25e-09	0.312	0	0	\\
4.3e-09	0.312	0	0	\\
4.35e-09	0.312	0	0	\\
4.4e-09	0.312	0	0	\\
4.45e-09	0.312	0	0	\\
4.5e-09	0.312	-0.00444444444444444	-0.00444444444444444	\\
4.55e-09	0.312	-0.00444444444444444	-0.00444444444444444	\\
4.6e-09	0.312	-0.00444444444444444	-0.00444444444444444	\\
4.65e-09	0.312	-0.00444444444444444	-0.00444444444444444	\\
4.7e-09	0.312	-0.00444444444444444	-0.00444444444444444	\\
4.75e-09	0.312	-0.00444444444444444	-0.00444444444444444	\\
4.8e-09	0.312	-0.00444444444444444	-0.00444444444444444	\\
4.85e-09	0.312	-0.00444444444444444	-0.00444444444444444	\\
4.9e-09	0.312	-0.00444444444444444	-0.00444444444444444	\\
4.95e-09	0.312	-0.00444444444444444	-0.00444444444444444	\\
5e-09	0.312	0	0	\\
5e-09	0.312	0	nan	\\
5e-09	0.312	-0.00666666666666667	0.00166666666666667	\\
5e-09	0.312	-0.00666666666666667	nan	\\
0	0.324	-0.00666666666666667	nan	\\
0	0.324	0	0.00166666666666667	\\
0	0.324	0	0	\\
5e-11	0.324	0	0	\\
1e-10	0.324	0	0	\\
1.5e-10	0.324	0	0	\\
2e-10	0.324	0	0	\\
2.5e-10	0.324	0	0	\\
3e-10	0.324	0	0	\\
3.5e-10	0.324	0.01	0.01	\\
4e-10	0.324	0.01	0.01	\\
4.5e-10	0.324	0.01	0.01	\\
5e-10	0.324	0.01	0.01	\\
5.5e-10	0.324	0.01	0.01	\\
6e-10	0.324	0.01	0.01	\\
6.5e-10	0.324	0.01	0.01	\\
7e-10	0.324	0.01	0.01	\\
7.5e-10	0.324	0.01	0.01	\\
8e-10	0.324	0.01	0.01	\\
8.5e-10	0.324	0	0	\\
9e-10	0.324	0	0	\\
9.5e-10	0.324	0	0	\\
1e-09	0.324	0	0	\\
1.05e-09	0.324	0	0	\\
1.1e-09	0.324	0	0	\\
1.15e-09	0.324	0	0	\\
1.2e-09	0.324	0	0	\\
1.25e-09	0.324	0	0	\\
1.3e-09	0.324	0	0	\\
1.35e-09	0.324	0	0	\\
1.4e-09	0.324	0	0	\\
1.45e-09	0.324	0	0	\\
1.5e-09	0.324	0	0	\\
1.55e-09	0.324	0	0	\\
1.6e-09	0.324	0	0	\\
1.65e-09	0.324	0	0	\\
1.7e-09	0.324	0	0	\\
1.75e-09	0.324	0	0	\\
1.8e-09	0.324	0	0	\\
1.85e-09	0.324	0	0	\\
1.9e-09	0.324	0	0	\\
1.95e-09	0.324	0	0	\\
2e-09	0.324	0	0	\\
2.05e-09	0.324	0	0	\\
2.1e-09	0.324	-0.00666666666666667	-0.00666666666666667	\\
2.15e-09	0.324	-0.00666666666666667	-0.00666666666666667	\\
2.2e-09	0.324	-0.00666666666666667	-0.00666666666666667	\\
2.25e-09	0.324	-0.00666666666666667	-0.00666666666666667	\\
2.3e-09	0.324	-0.00666666666666667	-0.00666666666666667	\\
2.35e-09	0.324	-0.00666666666666667	-0.00666666666666667	\\
2.4e-09	0.324	-0.00666666666666667	-0.00666666666666667	\\
2.45e-09	0.324	-0.00666666666666667	-0.00666666666666667	\\
2.5e-09	0.324	-0.00666666666666667	-0.00666666666666667	\\
2.55e-09	0.324	-0.00666666666666667	-0.00666666666666667	\\
2.6e-09	0.324	0	0	\\
2.65e-09	0.324	0	0	\\
2.7e-09	0.324	0	0	\\
2.75e-09	0.324	0.00666666666666667	0.00666666666666667	\\
2.8e-09	0.324	0.00666666666666667	0.00666666666666667	\\
2.85e-09	0.324	0.00666666666666667	0.00666666666666667	\\
2.9e-09	0.324	0.00666666666666667	0.00666666666666667	\\
2.95e-09	0.324	0.00666666666666667	0.00666666666666667	\\
3e-09	0.324	0.00666666666666667	0.00666666666666667	\\
3.05e-09	0.324	0.00666666666666667	0.00666666666666667	\\
3.1e-09	0.324	0.00666666666666667	0.00666666666666667	\\
3.15e-09	0.324	0.00666666666666667	0.00666666666666667	\\
3.2e-09	0.324	0.00666666666666667	0.00666666666666667	\\
3.25e-09	0.324	0	0	\\
3.3e-09	0.324	0	0	\\
3.35e-09	0.324	0	0	\\
3.4e-09	0.324	0	0	\\
3.45e-09	0.324	0	0	\\
3.5e-09	0.324	0	0	\\
3.55e-09	0.324	0	0	\\
3.6e-09	0.324	0	0	\\
3.65e-09	0.324	0	0	\\
3.7e-09	0.324	0	0	\\
3.75e-09	0.324	0	0	\\
3.8e-09	0.324	0	0	\\
3.85e-09	0.324	0	0	\\
3.9e-09	0.324	0	0	\\
3.95e-09	0.324	0	0	\\
4e-09	0.324	0	0	\\
4.05e-09	0.324	0	0	\\
4.1e-09	0.324	0	0	\\
4.15e-09	0.324	0	0	\\
4.2e-09	0.324	0	0	\\
4.25e-09	0.324	0	0	\\
4.3e-09	0.324	0	0	\\
4.35e-09	0.324	0	0	\\
4.4e-09	0.324	0	0	\\
4.45e-09	0.324	0	0	\\
4.5e-09	0.324	-0.00444444444444444	-0.00444444444444444	\\
4.55e-09	0.324	-0.00444444444444444	-0.00444444444444444	\\
4.6e-09	0.324	-0.00444444444444444	-0.00444444444444444	\\
4.65e-09	0.324	-0.00444444444444444	-0.00444444444444444	\\
4.7e-09	0.324	-0.00444444444444444	-0.00444444444444444	\\
4.75e-09	0.324	-0.00444444444444444	-0.00444444444444444	\\
4.8e-09	0.324	-0.00444444444444444	-0.00444444444444444	\\
4.85e-09	0.324	-0.00444444444444444	-0.00444444444444444	\\
4.9e-09	0.324	-0.00444444444444444	-0.00444444444444444	\\
4.95e-09	0.324	-0.00444444444444444	-0.00444444444444444	\\
5e-09	0.324	0	0	\\
5e-09	0.324	0	nan	\\
5e-09	0.324	-0.00666666666666667	0.00166666666666667	\\
5e-09	0.324	-0.00666666666666667	nan	\\
0	0.336	-0.00666666666666667	nan	\\
0	0.336	0	0.00166666666666667	\\
0	0.336	0	0	\\
5e-11	0.336	0	0	\\
1e-10	0.336	0	0	\\
1.5e-10	0.336	0	0	\\
2e-10	0.336	0	0	\\
2.5e-10	0.336	0	0	\\
3e-10	0.336	0	0	\\
3.5e-10	0.336	0.01	0.01	\\
4e-10	0.336	0.01	0.01	\\
4.5e-10	0.336	0.01	0.01	\\
5e-10	0.336	0.01	0.01	\\
5.5e-10	0.336	0.01	0.01	\\
6e-10	0.336	0.01	0.01	\\
6.5e-10	0.336	0.01	0.01	\\
7e-10	0.336	0.01	0.01	\\
7.5e-10	0.336	0.01	0.01	\\
8e-10	0.336	0.01	0.01	\\
8.5e-10	0.336	0	0	\\
9e-10	0.336	0	0	\\
9.5e-10	0.336	0	0	\\
1e-09	0.336	0	0	\\
1.05e-09	0.336	0	0	\\
1.1e-09	0.336	0	0	\\
1.15e-09	0.336	0	0	\\
1.2e-09	0.336	0	0	\\
1.25e-09	0.336	0	0	\\
1.3e-09	0.336	0	0	\\
1.35e-09	0.336	0	0	\\
1.4e-09	0.336	0	0	\\
1.45e-09	0.336	0	0	\\
1.5e-09	0.336	0	0	\\
1.55e-09	0.336	0	0	\\
1.6e-09	0.336	0	0	\\
1.65e-09	0.336	0	0	\\
1.7e-09	0.336	0	0	\\
1.75e-09	0.336	0	0	\\
1.8e-09	0.336	0	0	\\
1.85e-09	0.336	0	0	\\
1.9e-09	0.336	0	0	\\
1.95e-09	0.336	0	0	\\
2e-09	0.336	0	0	\\
2.05e-09	0.336	0	0	\\
2.1e-09	0.336	-0.00666666666666667	-0.00666666666666667	\\
2.15e-09	0.336	-0.00666666666666667	-0.00666666666666667	\\
2.2e-09	0.336	-0.00666666666666667	-0.00666666666666667	\\
2.25e-09	0.336	-0.00666666666666667	-0.00666666666666667	\\
2.3e-09	0.336	-0.00666666666666667	-0.00666666666666667	\\
2.35e-09	0.336	-0.00666666666666667	-0.00666666666666667	\\
2.4e-09	0.336	-0.00666666666666667	-0.00666666666666667	\\
2.45e-09	0.336	-0.00666666666666667	-0.00666666666666667	\\
2.5e-09	0.336	-0.00666666666666667	-0.00666666666666667	\\
2.55e-09	0.336	-0.00666666666666667	-0.00666666666666667	\\
2.6e-09	0.336	0	0	\\
2.65e-09	0.336	0	0	\\
2.7e-09	0.336	0	0	\\
2.75e-09	0.336	0.00666666666666667	0.00666666666666667	\\
2.8e-09	0.336	0.00666666666666667	0.00666666666666667	\\
2.85e-09	0.336	0.00666666666666667	0.00666666666666667	\\
2.9e-09	0.336	0.00666666666666667	0.00666666666666667	\\
2.95e-09	0.336	0.00666666666666667	0.00666666666666667	\\
3e-09	0.336	0.00666666666666667	0.00666666666666667	\\
3.05e-09	0.336	0.00666666666666667	0.00666666666666667	\\
3.1e-09	0.336	0.00666666666666667	0.00666666666666667	\\
3.15e-09	0.336	0.00666666666666667	0.00666666666666667	\\
3.2e-09	0.336	0.00666666666666667	0.00666666666666667	\\
3.25e-09	0.336	0	0	\\
3.3e-09	0.336	0	0	\\
3.35e-09	0.336	0	0	\\
3.4e-09	0.336	0	0	\\
3.45e-09	0.336	0	0	\\
3.5e-09	0.336	0	0	\\
3.55e-09	0.336	0	0	\\
3.6e-09	0.336	0	0	\\
3.65e-09	0.336	0	0	\\
3.7e-09	0.336	0	0	\\
3.75e-09	0.336	0	0	\\
3.8e-09	0.336	0	0	\\
3.85e-09	0.336	0	0	\\
3.9e-09	0.336	0	0	\\
3.95e-09	0.336	0	0	\\
4e-09	0.336	0	0	\\
4.05e-09	0.336	0	0	\\
4.1e-09	0.336	0	0	\\
4.15e-09	0.336	0	0	\\
4.2e-09	0.336	0	0	\\
4.25e-09	0.336	0	0	\\
4.3e-09	0.336	0	0	\\
4.35e-09	0.336	0	0	\\
4.4e-09	0.336	0	0	\\
4.45e-09	0.336	0	0	\\
4.5e-09	0.336	-0.00444444444444444	-0.00444444444444444	\\
4.55e-09	0.336	-0.00444444444444444	-0.00444444444444444	\\
4.6e-09	0.336	-0.00444444444444444	-0.00444444444444444	\\
4.65e-09	0.336	-0.00444444444444444	-0.00444444444444444	\\
4.7e-09	0.336	-0.00444444444444444	-0.00444444444444444	\\
4.75e-09	0.336	-0.00444444444444444	-0.00444444444444444	\\
4.8e-09	0.336	-0.00444444444444444	-0.00444444444444444	\\
4.85e-09	0.336	-0.00444444444444444	-0.00444444444444444	\\
4.9e-09	0.336	-0.00444444444444444	-0.00444444444444444	\\
4.95e-09	0.336	-0.00444444444444444	-0.00444444444444444	\\
5e-09	0.336	0	0	\\
5e-09	0.336	0	nan	\\
5e-09	0.336	-0.00666666666666667	0.00166666666666667	\\
5e-09	0.336	-0.00666666666666667	nan	\\
0	0.348	-0.00666666666666667	nan	\\
0	0.348	0	0.00166666666666667	\\
0	0.348	0	0	\\
5e-11	0.348	0	0	\\
1e-10	0.348	0	0	\\
1.5e-10	0.348	0	0	\\
2e-10	0.348	0	0	\\
2.5e-10	0.348	0	0	\\
3e-10	0.348	0	0	\\
3.5e-10	0.348	0.01	0.01	\\
4e-10	0.348	0.01	0.01	\\
4.5e-10	0.348	0.01	0.01	\\
5e-10	0.348	0.01	0.01	\\
5.5e-10	0.348	0.01	0.01	\\
6e-10	0.348	0.01	0.01	\\
6.5e-10	0.348	0.01	0.01	\\
7e-10	0.348	0.01	0.01	\\
7.5e-10	0.348	0.01	0.01	\\
8e-10	0.348	0.01	0.01	\\
8.5e-10	0.348	0	0	\\
9e-10	0.348	0	0	\\
9.5e-10	0.348	0	0	\\
1e-09	0.348	0	0	\\
1.05e-09	0.348	0	0	\\
1.1e-09	0.348	0	0	\\
1.15e-09	0.348	0	0	\\
1.2e-09	0.348	0	0	\\
1.25e-09	0.348	0	0	\\
1.3e-09	0.348	0	0	\\
1.35e-09	0.348	0	0	\\
1.4e-09	0.348	0	0	\\
1.45e-09	0.348	0	0	\\
1.5e-09	0.348	0	0	\\
1.55e-09	0.348	0	0	\\
1.6e-09	0.348	0	0	\\
1.65e-09	0.348	0	0	\\
1.7e-09	0.348	0	0	\\
1.75e-09	0.348	0	0	\\
1.8e-09	0.348	0	0	\\
1.85e-09	0.348	0	0	\\
1.9e-09	0.348	0	0	\\
1.95e-09	0.348	0	0	\\
2e-09	0.348	0	0	\\
2.05e-09	0.348	0	0	\\
2.1e-09	0.348	-0.00666666666666667	-0.00666666666666667	\\
2.15e-09	0.348	-0.00666666666666667	-0.00666666666666667	\\
2.2e-09	0.348	-0.00666666666666667	-0.00666666666666667	\\
2.25e-09	0.348	-0.00666666666666667	-0.00666666666666667	\\
2.3e-09	0.348	-0.00666666666666667	-0.00666666666666667	\\
2.35e-09	0.348	-0.00666666666666667	-0.00666666666666667	\\
2.4e-09	0.348	-0.00666666666666667	-0.00666666666666667	\\
2.45e-09	0.348	-0.00666666666666667	-0.00666666666666667	\\
2.5e-09	0.348	-0.00666666666666667	-0.00666666666666667	\\
2.55e-09	0.348	-0.00666666666666667	-0.00666666666666667	\\
2.6e-09	0.348	0	0	\\
2.65e-09	0.348	0	0	\\
2.7e-09	0.348	0	0	\\
2.75e-09	0.348	0.00666666666666667	0.00666666666666667	\\
2.8e-09	0.348	0.00666666666666667	0.00666666666666667	\\
2.85e-09	0.348	0.00666666666666667	0.00666666666666667	\\
2.9e-09	0.348	0.00666666666666667	0.00666666666666667	\\
2.95e-09	0.348	0.00666666666666667	0.00666666666666667	\\
3e-09	0.348	0.00666666666666667	0.00666666666666667	\\
3.05e-09	0.348	0.00666666666666667	0.00666666666666667	\\
3.1e-09	0.348	0.00666666666666667	0.00666666666666667	\\
3.15e-09	0.348	0.00666666666666667	0.00666666666666667	\\
3.2e-09	0.348	0.00666666666666667	0.00666666666666667	\\
3.25e-09	0.348	0	0	\\
3.3e-09	0.348	0	0	\\
3.35e-09	0.348	0	0	\\
3.4e-09	0.348	0	0	\\
3.45e-09	0.348	0	0	\\
3.5e-09	0.348	0	0	\\
3.55e-09	0.348	0	0	\\
3.6e-09	0.348	0	0	\\
3.65e-09	0.348	0	0	\\
3.7e-09	0.348	0	0	\\
3.75e-09	0.348	0	0	\\
3.8e-09	0.348	0	0	\\
3.85e-09	0.348	0	0	\\
3.9e-09	0.348	0	0	\\
3.95e-09	0.348	0	0	\\
4e-09	0.348	0	0	\\
4.05e-09	0.348	0	0	\\
4.1e-09	0.348	0	0	\\
4.15e-09	0.348	0	0	\\
4.2e-09	0.348	0	0	\\
4.25e-09	0.348	0	0	\\
4.3e-09	0.348	0	0	\\
4.35e-09	0.348	0	0	\\
4.4e-09	0.348	0	0	\\
4.45e-09	0.348	0	0	\\
4.5e-09	0.348	-0.00444444444444444	-0.00444444444444444	\\
4.55e-09	0.348	-0.00444444444444444	-0.00444444444444444	\\
4.6e-09	0.348	-0.00444444444444444	-0.00444444444444444	\\
4.65e-09	0.348	-0.00444444444444444	-0.00444444444444444	\\
4.7e-09	0.348	-0.00444444444444444	-0.00444444444444444	\\
4.75e-09	0.348	-0.00444444444444444	-0.00444444444444444	\\
4.8e-09	0.348	-0.00444444444444444	-0.00444444444444444	\\
4.85e-09	0.348	-0.00444444444444444	-0.00444444444444444	\\
4.9e-09	0.348	-0.00444444444444444	-0.00444444444444444	\\
4.95e-09	0.348	-0.00444444444444444	-0.00444444444444444	\\
5e-09	0.348	0	0	\\
5e-09	0.348	0	nan	\\
5e-09	0.348	-0.00666666666666667	0.00166666666666667	\\
5e-09	0.348	-0.00666666666666667	nan	\\
0	0.36	-0.00666666666666667	nan	\\
0	0.36	0	0.00166666666666667	\\
0	0.36	0	0	\\
5e-11	0.36	0	0	\\
1e-10	0.36	0	0	\\
1.5e-10	0.36	0	0	\\
2e-10	0.36	0	0	\\
2.5e-10	0.36	0	0	\\
3e-10	0.36	0	0	\\
3.5e-10	0.36	0	0	\\
4e-10	0.36	0.01	0.01	\\
4.5e-10	0.36	0.01	0.01	\\
5e-10	0.36	0.01	0.01	\\
5.5e-10	0.36	0.01	0.01	\\
6e-10	0.36	0.01	0.01	\\
6.5e-10	0.36	0.01	0.01	\\
7e-10	0.36	0.01	0.01	\\
7.5e-10	0.36	0.01	0.01	\\
8e-10	0.36	0.01	0.01	\\
8.5e-10	0.36	0.01	0.01	\\
9e-10	0.36	0	0	\\
9.5e-10	0.36	0	0	\\
1e-09	0.36	0	0	\\
1.05e-09	0.36	0	0	\\
1.1e-09	0.36	0	0	\\
1.15e-09	0.36	0	0	\\
1.2e-09	0.36	0	0	\\
1.25e-09	0.36	0	0	\\
1.3e-09	0.36	0	0	\\
1.35e-09	0.36	0	0	\\
1.4e-09	0.36	0	0	\\
1.45e-09	0.36	0	0	\\
1.5e-09	0.36	0	0	\\
1.55e-09	0.36	0	0	\\
1.6e-09	0.36	0	0	\\
1.65e-09	0.36	0	0	\\
1.7e-09	0.36	0	0	\\
1.75e-09	0.36	0	0	\\
1.8e-09	0.36	0	0	\\
1.85e-09	0.36	0	0	\\
1.9e-09	0.36	0	0	\\
1.95e-09	0.36	0	0	\\
2e-09	0.36	0	0	\\
2.05e-09	0.36	-0.00666666666666667	-0.00666666666666667	\\
2.1e-09	0.36	-0.00666666666666667	-0.00666666666666667	\\
2.15e-09	0.36	-0.00666666666666667	-0.00666666666666667	\\
2.2e-09	0.36	-0.00666666666666667	-0.00666666666666667	\\
2.25e-09	0.36	-0.00666666666666667	-0.00666666666666667	\\
2.3e-09	0.36	-0.00666666666666667	-0.00666666666666667	\\
2.35e-09	0.36	-0.00666666666666667	-0.00666666666666667	\\
2.4e-09	0.36	-0.00666666666666667	-0.00666666666666667	\\
2.45e-09	0.36	-0.00666666666666667	-0.00666666666666667	\\
2.5e-09	0.36	-0.00666666666666667	-0.00666666666666667	\\
2.55e-09	0.36	0	0	\\
2.6e-09	0.36	0	0	\\
2.65e-09	0.36	0	0	\\
2.7e-09	0.36	0	0	\\
2.75e-09	0.36	0	0	\\
2.8e-09	0.36	0.00666666666666667	0.00666666666666667	\\
2.85e-09	0.36	0.00666666666666667	0.00666666666666667	\\
2.9e-09	0.36	0.00666666666666667	0.00666666666666667	\\
2.95e-09	0.36	0.00666666666666667	0.00666666666666667	\\
3e-09	0.36	0.00666666666666667	0.00666666666666667	\\
3.05e-09	0.36	0.00666666666666667	0.00666666666666667	\\
3.1e-09	0.36	0.00666666666666667	0.00666666666666667	\\
3.15e-09	0.36	0.00666666666666667	0.00666666666666667	\\
3.2e-09	0.36	0.00666666666666667	0.00666666666666667	\\
3.25e-09	0.36	0.00666666666666667	0.00666666666666667	\\
3.3e-09	0.36	0	0	\\
3.35e-09	0.36	0	0	\\
3.4e-09	0.36	0	0	\\
3.45e-09	0.36	0	0	\\
3.5e-09	0.36	0	0	\\
3.55e-09	0.36	0	0	\\
3.6e-09	0.36	0	0	\\
3.65e-09	0.36	0	0	\\
3.7e-09	0.36	0	0	\\
3.75e-09	0.36	0	0	\\
3.8e-09	0.36	0	0	\\
3.85e-09	0.36	0	0	\\
3.9e-09	0.36	0	0	\\
3.95e-09	0.36	0	0	\\
4e-09	0.36	0	0	\\
4.05e-09	0.36	0	0	\\
4.1e-09	0.36	0	0	\\
4.15e-09	0.36	0	0	\\
4.2e-09	0.36	0	0	\\
4.25e-09	0.36	0	0	\\
4.3e-09	0.36	0	0	\\
4.35e-09	0.36	0	0	\\
4.4e-09	0.36	0	0	\\
4.45e-09	0.36	-0.00444444444444444	-0.00444444444444444	\\
4.5e-09	0.36	-0.00444444444444444	-0.00444444444444444	\\
4.55e-09	0.36	-0.00444444444444444	-0.00444444444444444	\\
4.6e-09	0.36	-0.00444444444444444	-0.00444444444444444	\\
4.65e-09	0.36	-0.00444444444444444	-0.00444444444444444	\\
4.7e-09	0.36	-0.00444444444444444	-0.00444444444444444	\\
4.75e-09	0.36	-0.00444444444444444	-0.00444444444444444	\\
4.8e-09	0.36	-0.00444444444444444	-0.00444444444444444	\\
4.85e-09	0.36	-0.00444444444444444	-0.00444444444444444	\\
4.9e-09	0.36	-0.00444444444444444	-0.00444444444444444	\\
4.95e-09	0.36	0	0	\\
5e-09	0.36	0	0	\\
5e-09	0.36	0	nan	\\
5e-09	0.36	-0.00666666666666667	0.00166666666666667	\\
5e-09	0.36	-0.00666666666666667	nan	\\
0	0.372	-0.00666666666666667	nan	\\
0	0.372	0	0.00166666666666667	\\
0	0.372	0	0	\\
5e-11	0.372	0	0	\\
1e-10	0.372	0	0	\\
1.5e-10	0.372	0	0	\\
2e-10	0.372	0	0	\\
2.5e-10	0.372	0	0	\\
3e-10	0.372	0	0	\\
3.5e-10	0.372	0	0	\\
4e-10	0.372	0.01	0.01	\\
4.5e-10	0.372	0.01	0.01	\\
5e-10	0.372	0.01	0.01	\\
5.5e-10	0.372	0.01	0.01	\\
6e-10	0.372	0.01	0.01	\\
6.5e-10	0.372	0.01	0.01	\\
7e-10	0.372	0.01	0.01	\\
7.5e-10	0.372	0.01	0.01	\\
8e-10	0.372	0.01	0.01	\\
8.5e-10	0.372	0.01	0.01	\\
9e-10	0.372	0	0	\\
9.5e-10	0.372	0	0	\\
1e-09	0.372	0	0	\\
1.05e-09	0.372	0	0	\\
1.1e-09	0.372	0	0	\\
1.15e-09	0.372	0	0	\\
1.2e-09	0.372	0	0	\\
1.25e-09	0.372	0	0	\\
1.3e-09	0.372	0	0	\\
1.35e-09	0.372	0	0	\\
1.4e-09	0.372	0	0	\\
1.45e-09	0.372	0	0	\\
1.5e-09	0.372	0	0	\\
1.55e-09	0.372	0	0	\\
1.6e-09	0.372	0	0	\\
1.65e-09	0.372	0	0	\\
1.7e-09	0.372	0	0	\\
1.75e-09	0.372	0	0	\\
1.8e-09	0.372	0	0	\\
1.85e-09	0.372	0	0	\\
1.9e-09	0.372	0	0	\\
1.95e-09	0.372	0	0	\\
2e-09	0.372	0	0	\\
2.05e-09	0.372	-0.00666666666666667	-0.00666666666666667	\\
2.1e-09	0.372	-0.00666666666666667	-0.00666666666666667	\\
2.15e-09	0.372	-0.00666666666666667	-0.00666666666666667	\\
2.2e-09	0.372	-0.00666666666666667	-0.00666666666666667	\\
2.25e-09	0.372	-0.00666666666666667	-0.00666666666666667	\\
2.3e-09	0.372	-0.00666666666666667	-0.00666666666666667	\\
2.35e-09	0.372	-0.00666666666666667	-0.00666666666666667	\\
2.4e-09	0.372	-0.00666666666666667	-0.00666666666666667	\\
2.45e-09	0.372	-0.00666666666666667	-0.00666666666666667	\\
2.5e-09	0.372	-0.00666666666666667	-0.00666666666666667	\\
2.55e-09	0.372	0	0	\\
2.6e-09	0.372	0	0	\\
2.65e-09	0.372	0	0	\\
2.7e-09	0.372	0	0	\\
2.75e-09	0.372	0	0	\\
2.8e-09	0.372	0.00666666666666667	0.00666666666666667	\\
2.85e-09	0.372	0.00666666666666667	0.00666666666666667	\\
2.9e-09	0.372	0.00666666666666667	0.00666666666666667	\\
2.95e-09	0.372	0.00666666666666667	0.00666666666666667	\\
3e-09	0.372	0.00666666666666667	0.00666666666666667	\\
3.05e-09	0.372	0.00666666666666667	0.00666666666666667	\\
3.1e-09	0.372	0.00666666666666667	0.00666666666666667	\\
3.15e-09	0.372	0.00666666666666667	0.00666666666666667	\\
3.2e-09	0.372	0.00666666666666667	0.00666666666666667	\\
3.25e-09	0.372	0.00666666666666667	0.00666666666666667	\\
3.3e-09	0.372	0	0	\\
3.35e-09	0.372	0	0	\\
3.4e-09	0.372	0	0	\\
3.45e-09	0.372	0	0	\\
3.5e-09	0.372	0	0	\\
3.55e-09	0.372	0	0	\\
3.6e-09	0.372	0	0	\\
3.65e-09	0.372	0	0	\\
3.7e-09	0.372	0	0	\\
3.75e-09	0.372	0	0	\\
3.8e-09	0.372	0	0	\\
3.85e-09	0.372	0	0	\\
3.9e-09	0.372	0	0	\\
3.95e-09	0.372	0	0	\\
4e-09	0.372	0	0	\\
4.05e-09	0.372	0	0	\\
4.1e-09	0.372	0	0	\\
4.15e-09	0.372	0	0	\\
4.2e-09	0.372	0	0	\\
4.25e-09	0.372	0	0	\\
4.3e-09	0.372	0	0	\\
4.35e-09	0.372	0	0	\\
4.4e-09	0.372	0	0	\\
4.45e-09	0.372	-0.00444444444444444	-0.00444444444444444	\\
4.5e-09	0.372	-0.00444444444444444	-0.00444444444444444	\\
4.55e-09	0.372	-0.00444444444444444	-0.00444444444444444	\\
4.6e-09	0.372	-0.00444444444444444	-0.00444444444444444	\\
4.65e-09	0.372	-0.00444444444444444	-0.00444444444444444	\\
4.7e-09	0.372	-0.00444444444444444	-0.00444444444444444	\\
4.75e-09	0.372	-0.00444444444444444	-0.00444444444444444	\\
4.8e-09	0.372	-0.00444444444444444	-0.00444444444444444	\\
4.85e-09	0.372	-0.00444444444444444	-0.00444444444444444	\\
4.9e-09	0.372	-0.00444444444444444	-0.00444444444444444	\\
4.95e-09	0.372	0	0	\\
5e-09	0.372	0	0	\\
5e-09	0.372	0	nan	\\
5e-09	0.372	-0.00666666666666667	0.00166666666666667	\\
5e-09	0.372	-0.00666666666666667	nan	\\
0	0.384	-0.00666666666666667	nan	\\
0	0.384	0	0.00166666666666667	\\
0	0.384	0	0	\\
5e-11	0.384	0	0	\\
1e-10	0.384	0	0	\\
1.5e-10	0.384	0	0	\\
2e-10	0.384	0	0	\\
2.5e-10	0.384	0	0	\\
3e-10	0.384	0	0	\\
3.5e-10	0.384	0	0	\\
4e-10	0.384	0.01	0.01	\\
4.5e-10	0.384	0.01	0.01	\\
5e-10	0.384	0.01	0.01	\\
5.5e-10	0.384	0.01	0.01	\\
6e-10	0.384	0.01	0.01	\\
6.5e-10	0.384	0.01	0.01	\\
7e-10	0.384	0.01	0.01	\\
7.5e-10	0.384	0.01	0.01	\\
8e-10	0.384	0.01	0.01	\\
8.5e-10	0.384	0.01	0.01	\\
9e-10	0.384	0	0	\\
9.5e-10	0.384	0	0	\\
1e-09	0.384	0	0	\\
1.05e-09	0.384	0	0	\\
1.1e-09	0.384	0	0	\\
1.15e-09	0.384	0	0	\\
1.2e-09	0.384	0	0	\\
1.25e-09	0.384	0	0	\\
1.3e-09	0.384	0	0	\\
1.35e-09	0.384	0	0	\\
1.4e-09	0.384	0	0	\\
1.45e-09	0.384	0	0	\\
1.5e-09	0.384	0	0	\\
1.55e-09	0.384	0	0	\\
1.6e-09	0.384	0	0	\\
1.65e-09	0.384	0	0	\\
1.7e-09	0.384	0	0	\\
1.75e-09	0.384	0	0	\\
1.8e-09	0.384	0	0	\\
1.85e-09	0.384	0	0	\\
1.9e-09	0.384	0	0	\\
1.95e-09	0.384	0	0	\\
2e-09	0.384	0	0	\\
2.05e-09	0.384	-0.00666666666666667	-0.00666666666666667	\\
2.1e-09	0.384	-0.00666666666666667	-0.00666666666666667	\\
2.15e-09	0.384	-0.00666666666666667	-0.00666666666666667	\\
2.2e-09	0.384	-0.00666666666666667	-0.00666666666666667	\\
2.25e-09	0.384	-0.00666666666666667	-0.00666666666666667	\\
2.3e-09	0.384	-0.00666666666666667	-0.00666666666666667	\\
2.35e-09	0.384	-0.00666666666666667	-0.00666666666666667	\\
2.4e-09	0.384	-0.00666666666666667	-0.00666666666666667	\\
2.45e-09	0.384	-0.00666666666666667	-0.00666666666666667	\\
2.5e-09	0.384	-0.00666666666666667	-0.00666666666666667	\\
2.55e-09	0.384	0	0	\\
2.6e-09	0.384	0	0	\\
2.65e-09	0.384	0	0	\\
2.7e-09	0.384	0	0	\\
2.75e-09	0.384	0	0	\\
2.8e-09	0.384	0.00666666666666667	0.00666666666666667	\\
2.85e-09	0.384	0.00666666666666667	0.00666666666666667	\\
2.9e-09	0.384	0.00666666666666667	0.00666666666666667	\\
2.95e-09	0.384	0.00666666666666667	0.00666666666666667	\\
3e-09	0.384	0.00666666666666667	0.00666666666666667	\\
3.05e-09	0.384	0.00666666666666667	0.00666666666666667	\\
3.1e-09	0.384	0.00666666666666667	0.00666666666666667	\\
3.15e-09	0.384	0.00666666666666667	0.00666666666666667	\\
3.2e-09	0.384	0.00666666666666667	0.00666666666666667	\\
3.25e-09	0.384	0.00666666666666667	0.00666666666666667	\\
3.3e-09	0.384	0	0	\\
3.35e-09	0.384	0	0	\\
3.4e-09	0.384	0	0	\\
3.45e-09	0.384	0	0	\\
3.5e-09	0.384	0	0	\\
3.55e-09	0.384	0	0	\\
3.6e-09	0.384	0	0	\\
3.65e-09	0.384	0	0	\\
3.7e-09	0.384	0	0	\\
3.75e-09	0.384	0	0	\\
3.8e-09	0.384	0	0	\\
3.85e-09	0.384	0	0	\\
3.9e-09	0.384	0	0	\\
3.95e-09	0.384	0	0	\\
4e-09	0.384	0	0	\\
4.05e-09	0.384	0	0	\\
4.1e-09	0.384	0	0	\\
4.15e-09	0.384	0	0	\\
4.2e-09	0.384	0	0	\\
4.25e-09	0.384	0	0	\\
4.3e-09	0.384	0	0	\\
4.35e-09	0.384	0	0	\\
4.4e-09	0.384	0	0	\\
4.45e-09	0.384	-0.00444444444444444	-0.00444444444444444	\\
4.5e-09	0.384	-0.00444444444444444	-0.00444444444444444	\\
4.55e-09	0.384	-0.00444444444444444	-0.00444444444444444	\\
4.6e-09	0.384	-0.00444444444444444	-0.00444444444444444	\\
4.65e-09	0.384	-0.00444444444444444	-0.00444444444444444	\\
4.7e-09	0.384	-0.00444444444444444	-0.00444444444444444	\\
4.75e-09	0.384	-0.00444444444444444	-0.00444444444444444	\\
4.8e-09	0.384	-0.00444444444444444	-0.00444444444444444	\\
4.85e-09	0.384	-0.00444444444444444	-0.00444444444444444	\\
4.9e-09	0.384	-0.00444444444444444	-0.00444444444444444	\\
4.95e-09	0.384	0	0	\\
5e-09	0.384	0	0	\\
5e-09	0.384	0	nan	\\
5e-09	0.384	-0.00666666666666667	0.00166666666666667	\\
5e-09	0.384	-0.00666666666666667	nan	\\
0	0.396	-0.00666666666666667	nan	\\
0	0.396	0	0.00166666666666667	\\
0	0.396	0	0	\\
5e-11	0.396	0	0	\\
1e-10	0.396	0	0	\\
1.5e-10	0.396	0	0	\\
2e-10	0.396	0	0	\\
2.5e-10	0.396	0	0	\\
3e-10	0.396	0	0	\\
3.5e-10	0.396	0	0	\\
4e-10	0.396	0.01	0.01	\\
4.5e-10	0.396	0.01	0.01	\\
5e-10	0.396	0.01	0.01	\\
5.5e-10	0.396	0.01	0.01	\\
6e-10	0.396	0.01	0.01	\\
6.5e-10	0.396	0.01	0.01	\\
7e-10	0.396	0.01	0.01	\\
7.5e-10	0.396	0.01	0.01	\\
8e-10	0.396	0.01	0.01	\\
8.5e-10	0.396	0.01	0.01	\\
9e-10	0.396	0	0	\\
9.5e-10	0.396	0	0	\\
1e-09	0.396	0	0	\\
1.05e-09	0.396	0	0	\\
1.1e-09	0.396	0	0	\\
1.15e-09	0.396	0	0	\\
1.2e-09	0.396	0	0	\\
1.25e-09	0.396	0	0	\\
1.3e-09	0.396	0	0	\\
1.35e-09	0.396	0	0	\\
1.4e-09	0.396	0	0	\\
1.45e-09	0.396	0	0	\\
1.5e-09	0.396	0	0	\\
1.55e-09	0.396	0	0	\\
1.6e-09	0.396	0	0	\\
1.65e-09	0.396	0	0	\\
1.7e-09	0.396	0	0	\\
1.75e-09	0.396	0	0	\\
1.8e-09	0.396	0	0	\\
1.85e-09	0.396	0	0	\\
1.9e-09	0.396	0	0	\\
1.95e-09	0.396	0	0	\\
2e-09	0.396	0	0	\\
2.05e-09	0.396	-0.00666666666666667	-0.00666666666666667	\\
2.1e-09	0.396	-0.00666666666666667	-0.00666666666666667	\\
2.15e-09	0.396	-0.00666666666666667	-0.00666666666666667	\\
2.2e-09	0.396	-0.00666666666666667	-0.00666666666666667	\\
2.25e-09	0.396	-0.00666666666666667	-0.00666666666666667	\\
2.3e-09	0.396	-0.00666666666666667	-0.00666666666666667	\\
2.35e-09	0.396	-0.00666666666666667	-0.00666666666666667	\\
2.4e-09	0.396	-0.00666666666666667	-0.00666666666666667	\\
2.45e-09	0.396	-0.00666666666666667	-0.00666666666666667	\\
2.5e-09	0.396	-0.00666666666666667	-0.00666666666666667	\\
2.55e-09	0.396	0	0	\\
2.6e-09	0.396	0	0	\\
2.65e-09	0.396	0	0	\\
2.7e-09	0.396	0	0	\\
2.75e-09	0.396	0	0	\\
2.8e-09	0.396	0.00666666666666667	0.00666666666666667	\\
2.85e-09	0.396	0.00666666666666667	0.00666666666666667	\\
2.9e-09	0.396	0.00666666666666667	0.00666666666666667	\\
2.95e-09	0.396	0.00666666666666667	0.00666666666666667	\\
3e-09	0.396	0.00666666666666667	0.00666666666666667	\\
3.05e-09	0.396	0.00666666666666667	0.00666666666666667	\\
3.1e-09	0.396	0.00666666666666667	0.00666666666666667	\\
3.15e-09	0.396	0.00666666666666667	0.00666666666666667	\\
3.2e-09	0.396	0.00666666666666667	0.00666666666666667	\\
3.25e-09	0.396	0.00666666666666667	0.00666666666666667	\\
3.3e-09	0.396	0	0	\\
3.35e-09	0.396	0	0	\\
3.4e-09	0.396	0	0	\\
3.45e-09	0.396	0	0	\\
3.5e-09	0.396	0	0	\\
3.55e-09	0.396	0	0	\\
3.6e-09	0.396	0	0	\\
3.65e-09	0.396	0	0	\\
3.7e-09	0.396	0	0	\\
3.75e-09	0.396	0	0	\\
3.8e-09	0.396	0	0	\\
3.85e-09	0.396	0	0	\\
3.9e-09	0.396	0	0	\\
3.95e-09	0.396	0	0	\\
4e-09	0.396	0	0	\\
4.05e-09	0.396	0	0	\\
4.1e-09	0.396	0	0	\\
4.15e-09	0.396	0	0	\\
4.2e-09	0.396	0	0	\\
4.25e-09	0.396	0	0	\\
4.3e-09	0.396	0	0	\\
4.35e-09	0.396	0	0	\\
4.4e-09	0.396	0	0	\\
4.45e-09	0.396	-0.00444444444444444	-0.00444444444444444	\\
4.5e-09	0.396	-0.00444444444444444	-0.00444444444444444	\\
4.55e-09	0.396	-0.00444444444444444	-0.00444444444444444	\\
4.6e-09	0.396	-0.00444444444444444	-0.00444444444444444	\\
4.65e-09	0.396	-0.00444444444444444	-0.00444444444444444	\\
4.7e-09	0.396	-0.00444444444444444	-0.00444444444444444	\\
4.75e-09	0.396	-0.00444444444444444	-0.00444444444444444	\\
4.8e-09	0.396	-0.00444444444444444	-0.00444444444444444	\\
4.85e-09	0.396	-0.00444444444444444	-0.00444444444444444	\\
4.9e-09	0.396	-0.00444444444444444	-0.00444444444444444	\\
4.95e-09	0.396	0	0	\\
5e-09	0.396	0	0	\\
5e-09	0.396	0	nan	\\
5e-09	0.396	-0.00666666666666667	0.00166666666666667	\\
5e-09	0.396	-0.00666666666666667	nan	\\
0	0.408	-0.00666666666666667	nan	\\
0	0.408	0	0.00166666666666667	\\
0	0.408	0	0	\\
5e-11	0.408	0	0	\\
1e-10	0.408	0	0	\\
1.5e-10	0.408	0	0	\\
2e-10	0.408	0	0	\\
2.5e-10	0.408	0	0	\\
3e-10	0.408	0	0	\\
3.5e-10	0.408	0	0	\\
4e-10	0.408	0	0	\\
4.5e-10	0.408	0.01	0.01	\\
5e-10	0.408	0.01	0.01	\\
5.5e-10	0.408	0.01	0.01	\\
6e-10	0.408	0.01	0.01	\\
6.5e-10	0.408	0.01	0.01	\\
7e-10	0.408	0.01	0.01	\\
7.5e-10	0.408	0.01	0.01	\\
8e-10	0.408	0.01	0.01	\\
8.5e-10	0.408	0.01	0.01	\\
9e-10	0.408	0.01	0.01	\\
9.5e-10	0.408	0	0	\\
1e-09	0.408	0	0	\\
1.05e-09	0.408	0	0	\\
1.1e-09	0.408	0	0	\\
1.15e-09	0.408	0	0	\\
1.2e-09	0.408	0	0	\\
1.25e-09	0.408	0	0	\\
1.3e-09	0.408	0	0	\\
1.35e-09	0.408	0	0	\\
1.4e-09	0.408	0	0	\\
1.45e-09	0.408	0	0	\\
1.5e-09	0.408	0	0	\\
1.55e-09	0.408	0	0	\\
1.6e-09	0.408	0	0	\\
1.65e-09	0.408	0	0	\\
1.7e-09	0.408	0	0	\\
1.75e-09	0.408	0	0	\\
1.8e-09	0.408	0	0	\\
1.85e-09	0.408	0	0	\\
1.9e-09	0.408	0	0	\\
1.95e-09	0.408	0	0	\\
2e-09	0.408	-0.00666666666666667	-0.00666666666666667	\\
2.05e-09	0.408	-0.00666666666666667	-0.00666666666666667	\\
2.1e-09	0.408	-0.00666666666666667	-0.00666666666666667	\\
2.15e-09	0.408	-0.00666666666666667	-0.00666666666666667	\\
2.2e-09	0.408	-0.00666666666666667	-0.00666666666666667	\\
2.25e-09	0.408	-0.00666666666666667	-0.00666666666666667	\\
2.3e-09	0.408	-0.00666666666666667	-0.00666666666666667	\\
2.35e-09	0.408	-0.00666666666666667	-0.00666666666666667	\\
2.4e-09	0.408	-0.00666666666666667	-0.00666666666666667	\\
2.45e-09	0.408	-0.00666666666666667	-0.00666666666666667	\\
2.5e-09	0.408	0	0	\\
2.55e-09	0.408	0	0	\\
2.6e-09	0.408	0	0	\\
2.65e-09	0.408	0	0	\\
2.7e-09	0.408	0	0	\\
2.75e-09	0.408	0	0	\\
2.8e-09	0.408	0	0	\\
2.85e-09	0.408	0.00666666666666667	0.00666666666666667	\\
2.9e-09	0.408	0.00666666666666667	0.00666666666666667	\\
2.95e-09	0.408	0.00666666666666667	0.00666666666666667	\\
3e-09	0.408	0.00666666666666667	0.00666666666666667	\\
3.05e-09	0.408	0.00666666666666667	0.00666666666666667	\\
3.1e-09	0.408	0.00666666666666667	0.00666666666666667	\\
3.15e-09	0.408	0.00666666666666667	0.00666666666666667	\\
3.2e-09	0.408	0.00666666666666667	0.00666666666666667	\\
3.25e-09	0.408	0.00666666666666667	0.00666666666666667	\\
3.3e-09	0.408	0.00666666666666667	0.00666666666666667	\\
3.35e-09	0.408	0	0	\\
3.4e-09	0.408	0	0	\\
3.45e-09	0.408	0	0	\\
3.5e-09	0.408	0	0	\\
3.55e-09	0.408	0	0	\\
3.6e-09	0.408	0	0	\\
3.65e-09	0.408	0	0	\\
3.7e-09	0.408	0	0	\\
3.75e-09	0.408	0	0	\\
3.8e-09	0.408	0	0	\\
3.85e-09	0.408	0	0	\\
3.9e-09	0.408	0	0	\\
3.95e-09	0.408	0	0	\\
4e-09	0.408	0	0	\\
4.05e-09	0.408	0	0	\\
4.1e-09	0.408	0	0	\\
4.15e-09	0.408	0	0	\\
4.2e-09	0.408	0	0	\\
4.25e-09	0.408	0	0	\\
4.3e-09	0.408	0	0	\\
4.35e-09	0.408	0	0	\\
4.4e-09	0.408	-0.00444444444444444	-0.00444444444444444	\\
4.45e-09	0.408	-0.00444444444444444	-0.00444444444444444	\\
4.5e-09	0.408	-0.00444444444444444	-0.00444444444444444	\\
4.55e-09	0.408	-0.00444444444444444	-0.00444444444444444	\\
4.6e-09	0.408	-0.00444444444444444	-0.00444444444444444	\\
4.65e-09	0.408	-0.00444444444444444	-0.00444444444444444	\\
4.7e-09	0.408	-0.00444444444444444	-0.00444444444444444	\\
4.75e-09	0.408	-0.00444444444444444	-0.00444444444444444	\\
4.8e-09	0.408	-0.00444444444444444	-0.00444444444444444	\\
4.85e-09	0.408	-0.00444444444444444	-0.00444444444444444	\\
4.9e-09	0.408	0	0	\\
4.95e-09	0.408	0	0	\\
5e-09	0.408	0	0	\\
5e-09	0.408	0	nan	\\
5e-09	0.408	-0.00666666666666667	0.00166666666666667	\\
5e-09	0.408	-0.00666666666666667	nan	\\
0	0.42	-0.00666666666666667	nan	\\
0	0.42	0	0.00166666666666667	\\
0	0.42	0	0	\\
5e-11	0.42	0	0	\\
1e-10	0.42	0	0	\\
1.5e-10	0.42	0	0	\\
2e-10	0.42	0	0	\\
2.5e-10	0.42	0	0	\\
3e-10	0.42	0	0	\\
3.5e-10	0.42	0	0	\\
4e-10	0.42	0	0	\\
4.5e-10	0.42	0.01	0.01	\\
5e-10	0.42	0.01	0.01	\\
5.5e-10	0.42	0.01	0.01	\\
6e-10	0.42	0.01	0.01	\\
6.5e-10	0.42	0.01	0.01	\\
7e-10	0.42	0.01	0.01	\\
7.5e-10	0.42	0.01	0.01	\\
8e-10	0.42	0.01	0.01	\\
8.5e-10	0.42	0.01	0.01	\\
9e-10	0.42	0.01	0.01	\\
9.5e-10	0.42	0	0	\\
1e-09	0.42	0	0	\\
1.05e-09	0.42	0	0	\\
1.1e-09	0.42	0	0	\\
1.15e-09	0.42	0	0	\\
1.2e-09	0.42	0	0	\\
1.25e-09	0.42	0	0	\\
1.3e-09	0.42	0	0	\\
1.35e-09	0.42	0	0	\\
1.4e-09	0.42	0	0	\\
1.45e-09	0.42	0	0	\\
1.5e-09	0.42	0	0	\\
1.55e-09	0.42	0	0	\\
1.6e-09	0.42	0	0	\\
1.65e-09	0.42	0	0	\\
1.7e-09	0.42	0	0	\\
1.75e-09	0.42	0	0	\\
1.8e-09	0.42	0	0	\\
1.85e-09	0.42	0	0	\\
1.9e-09	0.42	0	0	\\
1.95e-09	0.42	0	0	\\
2e-09	0.42	-0.00666666666666667	-0.00666666666666667	\\
2.05e-09	0.42	-0.00666666666666667	-0.00666666666666667	\\
2.1e-09	0.42	-0.00666666666666667	-0.00666666666666667	\\
2.15e-09	0.42	-0.00666666666666667	-0.00666666666666667	\\
2.2e-09	0.42	-0.00666666666666667	-0.00666666666666667	\\
2.25e-09	0.42	-0.00666666666666667	-0.00666666666666667	\\
2.3e-09	0.42	-0.00666666666666667	-0.00666666666666667	\\
2.35e-09	0.42	-0.00666666666666667	-0.00666666666666667	\\
2.4e-09	0.42	-0.00666666666666667	-0.00666666666666667	\\
2.45e-09	0.42	-0.00666666666666667	-0.00666666666666667	\\
2.5e-09	0.42	0	0	\\
2.55e-09	0.42	0	0	\\
2.6e-09	0.42	0	0	\\
2.65e-09	0.42	0	0	\\
2.7e-09	0.42	0	0	\\
2.75e-09	0.42	0	0	\\
2.8e-09	0.42	0	0	\\
2.85e-09	0.42	0.00666666666666667	0.00666666666666667	\\
2.9e-09	0.42	0.00666666666666667	0.00666666666666667	\\
2.95e-09	0.42	0.00666666666666667	0.00666666666666667	\\
3e-09	0.42	0.00666666666666667	0.00666666666666667	\\
3.05e-09	0.42	0.00666666666666667	0.00666666666666667	\\
3.1e-09	0.42	0.00666666666666667	0.00666666666666667	\\
3.15e-09	0.42	0.00666666666666667	0.00666666666666667	\\
3.2e-09	0.42	0.00666666666666667	0.00666666666666667	\\
3.25e-09	0.42	0.00666666666666667	0.00666666666666667	\\
3.3e-09	0.42	0.00666666666666667	0.00666666666666667	\\
3.35e-09	0.42	0	0	\\
3.4e-09	0.42	0	0	\\
3.45e-09	0.42	0	0	\\
3.5e-09	0.42	0	0	\\
3.55e-09	0.42	0	0	\\
3.6e-09	0.42	0	0	\\
3.65e-09	0.42	0	0	\\
3.7e-09	0.42	0	0	\\
3.75e-09	0.42	0	0	\\
3.8e-09	0.42	0	0	\\
3.85e-09	0.42	0	0	\\
3.9e-09	0.42	0	0	\\
3.95e-09	0.42	0	0	\\
4e-09	0.42	0	0	\\
4.05e-09	0.42	0	0	\\
4.1e-09	0.42	0	0	\\
4.15e-09	0.42	0	0	\\
4.2e-09	0.42	0	0	\\
4.25e-09	0.42	0	0	\\
4.3e-09	0.42	0	0	\\
4.35e-09	0.42	0	0	\\
4.4e-09	0.42	-0.00444444444444444	-0.00444444444444444	\\
4.45e-09	0.42	-0.00444444444444444	-0.00444444444444444	\\
4.5e-09	0.42	-0.00444444444444444	-0.00444444444444444	\\
4.55e-09	0.42	-0.00444444444444444	-0.00444444444444444	\\
4.6e-09	0.42	-0.00444444444444444	-0.00444444444444444	\\
4.65e-09	0.42	-0.00444444444444444	-0.00444444444444444	\\
4.7e-09	0.42	-0.00444444444444444	-0.00444444444444444	\\
4.75e-09	0.42	-0.00444444444444444	-0.00444444444444444	\\
4.8e-09	0.42	-0.00444444444444444	-0.00444444444444444	\\
4.85e-09	0.42	-0.00444444444444444	-0.00444444444444444	\\
4.9e-09	0.42	0	0	\\
4.95e-09	0.42	0	0	\\
5e-09	0.42	0	0	\\
5e-09	0.42	0	nan	\\
5e-09	0.42	-0.00666666666666667	0.00166666666666667	\\
5e-09	0.42	-0.00666666666666667	nan	\\
0	0.432	-0.00666666666666667	nan	\\
0	0.432	0	0.00166666666666667	\\
0	0.432	0	0	\\
5e-11	0.432	0	0	\\
1e-10	0.432	0	0	\\
1.5e-10	0.432	0	0	\\
2e-10	0.432	0	0	\\
2.5e-10	0.432	0	0	\\
3e-10	0.432	0	0	\\
3.5e-10	0.432	0	0	\\
4e-10	0.432	0	0	\\
4.5e-10	0.432	0.01	0.01	\\
5e-10	0.432	0.01	0.01	\\
5.5e-10	0.432	0.01	0.01	\\
6e-10	0.432	0.01	0.01	\\
6.5e-10	0.432	0.01	0.01	\\
7e-10	0.432	0.01	0.01	\\
7.5e-10	0.432	0.01	0.01	\\
8e-10	0.432	0.01	0.01	\\
8.5e-10	0.432	0.01	0.01	\\
9e-10	0.432	0.01	0.01	\\
9.5e-10	0.432	0	0	\\
1e-09	0.432	0	0	\\
1.05e-09	0.432	0	0	\\
1.1e-09	0.432	0	0	\\
1.15e-09	0.432	0	0	\\
1.2e-09	0.432	0	0	\\
1.25e-09	0.432	0	0	\\
1.3e-09	0.432	0	0	\\
1.35e-09	0.432	0	0	\\
1.4e-09	0.432	0	0	\\
1.45e-09	0.432	0	0	\\
1.5e-09	0.432	0	0	\\
1.55e-09	0.432	0	0	\\
1.6e-09	0.432	0	0	\\
1.65e-09	0.432	0	0	\\
1.7e-09	0.432	0	0	\\
1.75e-09	0.432	0	0	\\
1.8e-09	0.432	0	0	\\
1.85e-09	0.432	0	0	\\
1.9e-09	0.432	0	0	\\
1.95e-09	0.432	0	0	\\
2e-09	0.432	-0.00666666666666667	-0.00666666666666667	\\
2.05e-09	0.432	-0.00666666666666667	-0.00666666666666667	\\
2.1e-09	0.432	-0.00666666666666667	-0.00666666666666667	\\
2.15e-09	0.432	-0.00666666666666667	-0.00666666666666667	\\
2.2e-09	0.432	-0.00666666666666667	-0.00666666666666667	\\
2.25e-09	0.432	-0.00666666666666667	-0.00666666666666667	\\
2.3e-09	0.432	-0.00666666666666667	-0.00666666666666667	\\
2.35e-09	0.432	-0.00666666666666667	-0.00666666666666667	\\
2.4e-09	0.432	-0.00666666666666667	-0.00666666666666667	\\
2.45e-09	0.432	-0.00666666666666667	-0.00666666666666667	\\
2.5e-09	0.432	0	0	\\
2.55e-09	0.432	0	0	\\
2.6e-09	0.432	0	0	\\
2.65e-09	0.432	0	0	\\
2.7e-09	0.432	0	0	\\
2.75e-09	0.432	0	0	\\
2.8e-09	0.432	0	0	\\
2.85e-09	0.432	0.00666666666666667	0.00666666666666667	\\
2.9e-09	0.432	0.00666666666666667	0.00666666666666667	\\
2.95e-09	0.432	0.00666666666666667	0.00666666666666667	\\
3e-09	0.432	0.00666666666666667	0.00666666666666667	\\
3.05e-09	0.432	0.00666666666666667	0.00666666666666667	\\
3.1e-09	0.432	0.00666666666666667	0.00666666666666667	\\
3.15e-09	0.432	0.00666666666666667	0.00666666666666667	\\
3.2e-09	0.432	0.00666666666666667	0.00666666666666667	\\
3.25e-09	0.432	0.00666666666666667	0.00666666666666667	\\
3.3e-09	0.432	0.00666666666666667	0.00666666666666667	\\
3.35e-09	0.432	0	0	\\
3.4e-09	0.432	0	0	\\
3.45e-09	0.432	0	0	\\
3.5e-09	0.432	0	0	\\
3.55e-09	0.432	0	0	\\
3.6e-09	0.432	0	0	\\
3.65e-09	0.432	0	0	\\
3.7e-09	0.432	0	0	\\
3.75e-09	0.432	0	0	\\
3.8e-09	0.432	0	0	\\
3.85e-09	0.432	0	0	\\
3.9e-09	0.432	0	0	\\
3.95e-09	0.432	0	0	\\
4e-09	0.432	0	0	\\
4.05e-09	0.432	0	0	\\
4.1e-09	0.432	0	0	\\
4.15e-09	0.432	0	0	\\
4.2e-09	0.432	0	0	\\
4.25e-09	0.432	0	0	\\
4.3e-09	0.432	0	0	\\
4.35e-09	0.432	0	0	\\
4.4e-09	0.432	-0.00444444444444444	-0.00444444444444444	\\
4.45e-09	0.432	-0.00444444444444444	-0.00444444444444444	\\
4.5e-09	0.432	-0.00444444444444444	-0.00444444444444444	\\
4.55e-09	0.432	-0.00444444444444444	-0.00444444444444444	\\
4.6e-09	0.432	-0.00444444444444444	-0.00444444444444444	\\
4.65e-09	0.432	-0.00444444444444444	-0.00444444444444444	\\
4.7e-09	0.432	-0.00444444444444444	-0.00444444444444444	\\
4.75e-09	0.432	-0.00444444444444444	-0.00444444444444444	\\
4.8e-09	0.432	-0.00444444444444444	-0.00444444444444444	\\
4.85e-09	0.432	-0.00444444444444444	-0.00444444444444444	\\
4.9e-09	0.432	0	0	\\
4.95e-09	0.432	0	0	\\
5e-09	0.432	0	0	\\
5e-09	0.432	0	nan	\\
5e-09	0.432	-0.00666666666666667	0.00166666666666667	\\
5e-09	0.432	-0.00666666666666667	nan	\\
0	0.444	-0.00666666666666667	nan	\\
0	0.444	0	0.00166666666666667	\\
0	0.444	0	0	\\
5e-11	0.444	0	0	\\
1e-10	0.444	0	0	\\
1.5e-10	0.444	0	0	\\
2e-10	0.444	0	0	\\
2.5e-10	0.444	0	0	\\
3e-10	0.444	0	0	\\
3.5e-10	0.444	0	0	\\
4e-10	0.444	0	0	\\
4.5e-10	0.444	0.01	0.01	\\
5e-10	0.444	0.01	0.01	\\
5.5e-10	0.444	0.01	0.01	\\
6e-10	0.444	0.01	0.01	\\
6.5e-10	0.444	0.01	0.01	\\
7e-10	0.444	0.01	0.01	\\
7.5e-10	0.444	0.01	0.01	\\
8e-10	0.444	0.01	0.01	\\
8.5e-10	0.444	0.01	0.01	\\
9e-10	0.444	0.01	0.01	\\
9.5e-10	0.444	0	0	\\
1e-09	0.444	0	0	\\
1.05e-09	0.444	0	0	\\
1.1e-09	0.444	0	0	\\
1.15e-09	0.444	0	0	\\
1.2e-09	0.444	0	0	\\
1.25e-09	0.444	0	0	\\
1.3e-09	0.444	0	0	\\
1.35e-09	0.444	0	0	\\
1.4e-09	0.444	0	0	\\
1.45e-09	0.444	0	0	\\
1.5e-09	0.444	0	0	\\
1.55e-09	0.444	0	0	\\
1.6e-09	0.444	0	0	\\
1.65e-09	0.444	0	0	\\
1.7e-09	0.444	0	0	\\
1.75e-09	0.444	0	0	\\
1.8e-09	0.444	0	0	\\
1.85e-09	0.444	0	0	\\
1.9e-09	0.444	0	0	\\
1.95e-09	0.444	0	0	\\
2e-09	0.444	-0.00666666666666667	-0.00666666666666667	\\
2.05e-09	0.444	-0.00666666666666667	-0.00666666666666667	\\
2.1e-09	0.444	-0.00666666666666667	-0.00666666666666667	\\
2.15e-09	0.444	-0.00666666666666667	-0.00666666666666667	\\
2.2e-09	0.444	-0.00666666666666667	-0.00666666666666667	\\
2.25e-09	0.444	-0.00666666666666667	-0.00666666666666667	\\
2.3e-09	0.444	-0.00666666666666667	-0.00666666666666667	\\
2.35e-09	0.444	-0.00666666666666667	-0.00666666666666667	\\
2.4e-09	0.444	-0.00666666666666667	-0.00666666666666667	\\
2.45e-09	0.444	-0.00666666666666667	-0.00666666666666667	\\
2.5e-09	0.444	0	0	\\
2.55e-09	0.444	0	0	\\
2.6e-09	0.444	0	0	\\
2.65e-09	0.444	0	0	\\
2.7e-09	0.444	0	0	\\
2.75e-09	0.444	0	0	\\
2.8e-09	0.444	0	0	\\
2.85e-09	0.444	0.00666666666666667	0.00666666666666667	\\
2.9e-09	0.444	0.00666666666666667	0.00666666666666667	\\
2.95e-09	0.444	0.00666666666666667	0.00666666666666667	\\
3e-09	0.444	0.00666666666666667	0.00666666666666667	\\
3.05e-09	0.444	0.00666666666666667	0.00666666666666667	\\
3.1e-09	0.444	0.00666666666666667	0.00666666666666667	\\
3.15e-09	0.444	0.00666666666666667	0.00666666666666667	\\
3.2e-09	0.444	0.00666666666666667	0.00666666666666667	\\
3.25e-09	0.444	0.00666666666666667	0.00666666666666667	\\
3.3e-09	0.444	0.00666666666666667	0.00666666666666667	\\
3.35e-09	0.444	0	0	\\
3.4e-09	0.444	0	0	\\
3.45e-09	0.444	0	0	\\
3.5e-09	0.444	0	0	\\
3.55e-09	0.444	0	0	\\
3.6e-09	0.444	0	0	\\
3.65e-09	0.444	0	0	\\
3.7e-09	0.444	0	0	\\
3.75e-09	0.444	0	0	\\
3.8e-09	0.444	0	0	\\
3.85e-09	0.444	0	0	\\
3.9e-09	0.444	0	0	\\
3.95e-09	0.444	0	0	\\
4e-09	0.444	0	0	\\
4.05e-09	0.444	0	0	\\
4.1e-09	0.444	0	0	\\
4.15e-09	0.444	0	0	\\
4.2e-09	0.444	0	0	\\
4.25e-09	0.444	0	0	\\
4.3e-09	0.444	0	0	\\
4.35e-09	0.444	0	0	\\
4.4e-09	0.444	-0.00444444444444444	-0.00444444444444444	\\
4.45e-09	0.444	-0.00444444444444444	-0.00444444444444444	\\
4.5e-09	0.444	-0.00444444444444444	-0.00444444444444444	\\
4.55e-09	0.444	-0.00444444444444444	-0.00444444444444444	\\
4.6e-09	0.444	-0.00444444444444444	-0.00444444444444444	\\
4.65e-09	0.444	-0.00444444444444444	-0.00444444444444444	\\
4.7e-09	0.444	-0.00444444444444444	-0.00444444444444444	\\
4.75e-09	0.444	-0.00444444444444444	-0.00444444444444444	\\
4.8e-09	0.444	-0.00444444444444444	-0.00444444444444444	\\
4.85e-09	0.444	-0.00444444444444444	-0.00444444444444444	\\
4.9e-09	0.444	0	0	\\
4.95e-09	0.444	0	0	\\
5e-09	0.444	0	0	\\
5e-09	0.444	0	nan	\\
5e-09	0.444	-0.00666666666666667	0.00166666666666667	\\
5e-09	0.444	-0.00666666666666667	nan	\\
0	0.456	-0.00666666666666667	nan	\\
0	0.456	0	0.00166666666666667	\\
0	0.456	0	0	\\
5e-11	0.456	0	0	\\
1e-10	0.456	0	0	\\
1.5e-10	0.456	0	0	\\
2e-10	0.456	0	0	\\
2.5e-10	0.456	0	0	\\
3e-10	0.456	0	0	\\
3.5e-10	0.456	0	0	\\
4e-10	0.456	0	0	\\
4.5e-10	0.456	0	0	\\
5e-10	0.456	0.01	0.01	\\
5.5e-10	0.456	0.01	0.01	\\
6e-10	0.456	0.01	0.01	\\
6.5e-10	0.456	0.01	0.01	\\
7e-10	0.456	0.01	0.01	\\
7.5e-10	0.456	0.01	0.01	\\
8e-10	0.456	0.01	0.01	\\
8.5e-10	0.456	0.01	0.01	\\
9e-10	0.456	0.01	0.01	\\
9.5e-10	0.456	0.01	0.01	\\
1e-09	0.456	0	0	\\
1.05e-09	0.456	0	0	\\
1.1e-09	0.456	0	0	\\
1.15e-09	0.456	0	0	\\
1.2e-09	0.456	0	0	\\
1.25e-09	0.456	0	0	\\
1.3e-09	0.456	0	0	\\
1.35e-09	0.456	0	0	\\
1.4e-09	0.456	0	0	\\
1.45e-09	0.456	0	0	\\
1.5e-09	0.456	0	0	\\
1.55e-09	0.456	0	0	\\
1.6e-09	0.456	0	0	\\
1.65e-09	0.456	0	0	\\
1.7e-09	0.456	0	0	\\
1.75e-09	0.456	0	0	\\
1.8e-09	0.456	0	0	\\
1.85e-09	0.456	0	0	\\
1.9e-09	0.456	0	0	\\
1.95e-09	0.456	-0.00666666666666667	-0.00666666666666667	\\
2e-09	0.456	-0.00666666666666667	-0.00666666666666667	\\
2.05e-09	0.456	-0.00666666666666667	-0.00666666666666667	\\
2.1e-09	0.456	-0.00666666666666667	-0.00666666666666667	\\
2.15e-09	0.456	-0.00666666666666667	-0.00666666666666667	\\
2.2e-09	0.456	-0.00666666666666667	-0.00666666666666667	\\
2.25e-09	0.456	-0.00666666666666667	-0.00666666666666667	\\
2.3e-09	0.456	-0.00666666666666667	-0.00666666666666667	\\
2.35e-09	0.456	-0.00666666666666667	-0.00666666666666667	\\
2.4e-09	0.456	-0.00666666666666667	-0.00666666666666667	\\
2.45e-09	0.456	0	0	\\
2.5e-09	0.456	0	0	\\
2.55e-09	0.456	0	0	\\
2.6e-09	0.456	0	0	\\
2.65e-09	0.456	0	0	\\
2.7e-09	0.456	0	0	\\
2.75e-09	0.456	0	0	\\
2.8e-09	0.456	0	0	\\
2.85e-09	0.456	0	0	\\
2.9e-09	0.456	0.00666666666666667	0.00666666666666667	\\
2.95e-09	0.456	0.00666666666666667	0.00666666666666667	\\
3e-09	0.456	0.00666666666666667	0.00666666666666667	\\
3.05e-09	0.456	0.00666666666666667	0.00666666666666667	\\
3.1e-09	0.456	0.00666666666666667	0.00666666666666667	\\
3.15e-09	0.456	0.00666666666666667	0.00666666666666667	\\
3.2e-09	0.456	0.00666666666666667	0.00666666666666667	\\
3.25e-09	0.456	0.00666666666666667	0.00666666666666667	\\
3.3e-09	0.456	0.00666666666666667	0.00666666666666667	\\
3.35e-09	0.456	0.00666666666666667	0.00666666666666667	\\
3.4e-09	0.456	0	0	\\
3.45e-09	0.456	0	0	\\
3.5e-09	0.456	0	0	\\
3.55e-09	0.456	0	0	\\
3.6e-09	0.456	0	0	\\
3.65e-09	0.456	0	0	\\
3.7e-09	0.456	0	0	\\
3.75e-09	0.456	0	0	\\
3.8e-09	0.456	0	0	\\
3.85e-09	0.456	0	0	\\
3.9e-09	0.456	0	0	\\
3.95e-09	0.456	0	0	\\
4e-09	0.456	0	0	\\
4.05e-09	0.456	0	0	\\
4.1e-09	0.456	0	0	\\
4.15e-09	0.456	0	0	\\
4.2e-09	0.456	0	0	\\
4.25e-09	0.456	0	0	\\
4.3e-09	0.456	0	0	\\
4.35e-09	0.456	-0.00444444444444444	-0.00444444444444444	\\
4.4e-09	0.456	-0.00444444444444444	-0.00444444444444444	\\
4.45e-09	0.456	-0.00444444444444444	-0.00444444444444444	\\
4.5e-09	0.456	-0.00444444444444444	-0.00444444444444444	\\
4.55e-09	0.456	-0.00444444444444444	-0.00444444444444444	\\
4.6e-09	0.456	-0.00444444444444444	-0.00444444444444444	\\
4.65e-09	0.456	-0.00444444444444444	-0.00444444444444444	\\
4.7e-09	0.456	-0.00444444444444444	-0.00444444444444444	\\
4.75e-09	0.456	-0.00444444444444444	-0.00444444444444444	\\
4.8e-09	0.456	-0.00444444444444444	-0.00444444444444444	\\
4.85e-09	0.456	0	0	\\
4.9e-09	0.456	0	0	\\
4.95e-09	0.456	0	0	\\
5e-09	0.456	0	0	\\
5e-09	0.456	0	nan	\\
5e-09	0.456	-0.00666666666666667	0.00166666666666667	\\
5e-09	0.456	-0.00666666666666667	nan	\\
0	0.468	-0.00666666666666667	nan	\\
0	0.468	0	0.00166666666666667	\\
0	0.468	0	0	\\
5e-11	0.468	0	0	\\
1e-10	0.468	0	0	\\
1.5e-10	0.468	0	0	\\
2e-10	0.468	0	0	\\
2.5e-10	0.468	0	0	\\
3e-10	0.468	0	0	\\
3.5e-10	0.468	0	0	\\
4e-10	0.468	0	0	\\
4.5e-10	0.468	0	0	\\
5e-10	0.468	0.01	0.01	\\
5.5e-10	0.468	0.01	0.01	\\
6e-10	0.468	0.01	0.01	\\
6.5e-10	0.468	0.01	0.01	\\
7e-10	0.468	0.01	0.01	\\
7.5e-10	0.468	0.01	0.01	\\
8e-10	0.468	0.01	0.01	\\
8.5e-10	0.468	0.01	0.01	\\
9e-10	0.468	0.01	0.01	\\
9.5e-10	0.468	0.01	0.01	\\
1e-09	0.468	0	0	\\
1.05e-09	0.468	0	0	\\
1.1e-09	0.468	0	0	\\
1.15e-09	0.468	0	0	\\
1.2e-09	0.468	0	0	\\
1.25e-09	0.468	0	0	\\
1.3e-09	0.468	0	0	\\
1.35e-09	0.468	0	0	\\
1.4e-09	0.468	0	0	\\
1.45e-09	0.468	0	0	\\
1.5e-09	0.468	0	0	\\
1.55e-09	0.468	0	0	\\
1.6e-09	0.468	0	0	\\
1.65e-09	0.468	0	0	\\
1.7e-09	0.468	0	0	\\
1.75e-09	0.468	0	0	\\
1.8e-09	0.468	0	0	\\
1.85e-09	0.468	0	0	\\
1.9e-09	0.468	0	0	\\
1.95e-09	0.468	-0.00666666666666667	-0.00666666666666667	\\
2e-09	0.468	-0.00666666666666667	-0.00666666666666667	\\
2.05e-09	0.468	-0.00666666666666667	-0.00666666666666667	\\
2.1e-09	0.468	-0.00666666666666667	-0.00666666666666667	\\
2.15e-09	0.468	-0.00666666666666667	-0.00666666666666667	\\
2.2e-09	0.468	-0.00666666666666667	-0.00666666666666667	\\
2.25e-09	0.468	-0.00666666666666667	-0.00666666666666667	\\
2.3e-09	0.468	-0.00666666666666667	-0.00666666666666667	\\
2.35e-09	0.468	-0.00666666666666667	-0.00666666666666667	\\
2.4e-09	0.468	-0.00666666666666667	-0.00666666666666667	\\
2.45e-09	0.468	0	0	\\
2.5e-09	0.468	0	0	\\
2.55e-09	0.468	0	0	\\
2.6e-09	0.468	0	0	\\
2.65e-09	0.468	0	0	\\
2.7e-09	0.468	0	0	\\
2.75e-09	0.468	0	0	\\
2.8e-09	0.468	0	0	\\
2.85e-09	0.468	0	0	\\
2.9e-09	0.468	0.00666666666666667	0.00666666666666667	\\
2.95e-09	0.468	0.00666666666666667	0.00666666666666667	\\
3e-09	0.468	0.00666666666666667	0.00666666666666667	\\
3.05e-09	0.468	0.00666666666666667	0.00666666666666667	\\
3.1e-09	0.468	0.00666666666666667	0.00666666666666667	\\
3.15e-09	0.468	0.00666666666666667	0.00666666666666667	\\
3.2e-09	0.468	0.00666666666666667	0.00666666666666667	\\
3.25e-09	0.468	0.00666666666666667	0.00666666666666667	\\
3.3e-09	0.468	0.00666666666666667	0.00666666666666667	\\
3.35e-09	0.468	0.00666666666666667	0.00666666666666667	\\
3.4e-09	0.468	0	0	\\
3.45e-09	0.468	0	0	\\
3.5e-09	0.468	0	0	\\
3.55e-09	0.468	0	0	\\
3.6e-09	0.468	0	0	\\
3.65e-09	0.468	0	0	\\
3.7e-09	0.468	0	0	\\
3.75e-09	0.468	0	0	\\
3.8e-09	0.468	0	0	\\
3.85e-09	0.468	0	0	\\
3.9e-09	0.468	0	0	\\
3.95e-09	0.468	0	0	\\
4e-09	0.468	0	0	\\
4.05e-09	0.468	0	0	\\
4.1e-09	0.468	0	0	\\
4.15e-09	0.468	0	0	\\
4.2e-09	0.468	0	0	\\
4.25e-09	0.468	0	0	\\
4.3e-09	0.468	0	0	\\
4.35e-09	0.468	-0.00444444444444444	-0.00444444444444444	\\
4.4e-09	0.468	-0.00444444444444444	-0.00444444444444444	\\
4.45e-09	0.468	-0.00444444444444444	-0.00444444444444444	\\
4.5e-09	0.468	-0.00444444444444444	-0.00444444444444444	\\
4.55e-09	0.468	-0.00444444444444444	-0.00444444444444444	\\
4.6e-09	0.468	-0.00444444444444444	-0.00444444444444444	\\
4.65e-09	0.468	-0.00444444444444444	-0.00444444444444444	\\
4.7e-09	0.468	-0.00444444444444444	-0.00444444444444444	\\
4.75e-09	0.468	-0.00444444444444444	-0.00444444444444444	\\
4.8e-09	0.468	-0.00444444444444444	-0.00444444444444444	\\
4.85e-09	0.468	0	0	\\
4.9e-09	0.468	0	0	\\
4.95e-09	0.468	0	0	\\
5e-09	0.468	0	0	\\
5e-09	0.468	0	nan	\\
5e-09	0.468	-0.00666666666666667	0.00166666666666667	\\
5e-09	0.468	-0.00666666666666667	nan	\\
0	0.48	-0.00666666666666667	nan	\\
0	0.48	0	0.00166666666666667	\\
0	0.48	0	0	\\
5e-11	0.48	0	0	\\
1e-10	0.48	0	0	\\
1.5e-10	0.48	0	0	\\
2e-10	0.48	0	0	\\
2.5e-10	0.48	0	0	\\
3e-10	0.48	0	0	\\
3.5e-10	0.48	0	0	\\
4e-10	0.48	0	0	\\
4.5e-10	0.48	0	0	\\
5e-10	0.48	0.01	0.01	\\
5.5e-10	0.48	0.01	0.01	\\
6e-10	0.48	0.01	0.01	\\
6.5e-10	0.48	0.01	0.01	\\
7e-10	0.48	0.01	0.01	\\
7.5e-10	0.48	0.01	0.01	\\
8e-10	0.48	0.01	0.01	\\
8.5e-10	0.48	0.01	0.01	\\
9e-10	0.48	0.01	0.01	\\
9.5e-10	0.48	0.01	0.01	\\
1e-09	0.48	0	0	\\
1.05e-09	0.48	0	0	\\
1.1e-09	0.48	0	0	\\
1.15e-09	0.48	0	0	\\
1.2e-09	0.48	0	0	\\
1.25e-09	0.48	0	0	\\
1.3e-09	0.48	0	0	\\
1.35e-09	0.48	0	0	\\
1.4e-09	0.48	0	0	\\
1.45e-09	0.48	0	0	\\
1.5e-09	0.48	0	0	\\
1.55e-09	0.48	0	0	\\
1.6e-09	0.48	0	0	\\
1.65e-09	0.48	0	0	\\
1.7e-09	0.48	0	0	\\
1.75e-09	0.48	0	0	\\
1.8e-09	0.48	0	0	\\
1.85e-09	0.48	0	0	\\
1.9e-09	0.48	0	0	\\
1.95e-09	0.48	-0.00666666666666667	-0.00666666666666667	\\
2e-09	0.48	-0.00666666666666667	-0.00666666666666667	\\
2.05e-09	0.48	-0.00666666666666667	-0.00666666666666667	\\
2.1e-09	0.48	-0.00666666666666667	-0.00666666666666667	\\
2.15e-09	0.48	-0.00666666666666667	-0.00666666666666667	\\
2.2e-09	0.48	-0.00666666666666667	-0.00666666666666667	\\
2.25e-09	0.48	-0.00666666666666667	-0.00666666666666667	\\
2.3e-09	0.48	-0.00666666666666667	-0.00666666666666667	\\
2.35e-09	0.48	-0.00666666666666667	-0.00666666666666667	\\
2.4e-09	0.48	-0.00666666666666667	-0.00666666666666667	\\
2.45e-09	0.48	0	0	\\
2.5e-09	0.48	0	0	\\
2.55e-09	0.48	0	0	\\
2.6e-09	0.48	0	0	\\
2.65e-09	0.48	0	0	\\
2.7e-09	0.48	0	0	\\
2.75e-09	0.48	0	0	\\
2.8e-09	0.48	0	0	\\
2.85e-09	0.48	0	0	\\
2.9e-09	0.48	0.00666666666666667	0.00666666666666667	\\
2.95e-09	0.48	0.00666666666666667	0.00666666666666667	\\
3e-09	0.48	0.00666666666666667	0.00666666666666667	\\
3.05e-09	0.48	0.00666666666666667	0.00666666666666667	\\
3.1e-09	0.48	0.00666666666666667	0.00666666666666667	\\
3.15e-09	0.48	0.00666666666666667	0.00666666666666667	\\
3.2e-09	0.48	0.00666666666666667	0.00666666666666667	\\
3.25e-09	0.48	0.00666666666666667	0.00666666666666667	\\
3.3e-09	0.48	0.00666666666666667	0.00666666666666667	\\
3.35e-09	0.48	0.00666666666666667	0.00666666666666667	\\
3.4e-09	0.48	0	0	\\
3.45e-09	0.48	0	0	\\
3.5e-09	0.48	0	0	\\
3.55e-09	0.48	0	0	\\
3.6e-09	0.48	0	0	\\
3.65e-09	0.48	0	0	\\
3.7e-09	0.48	0	0	\\
3.75e-09	0.48	0	0	\\
3.8e-09	0.48	0	0	\\
3.85e-09	0.48	0	0	\\
3.9e-09	0.48	0	0	\\
3.95e-09	0.48	0	0	\\
4e-09	0.48	0	0	\\
4.05e-09	0.48	0	0	\\
4.1e-09	0.48	0	0	\\
4.15e-09	0.48	0	0	\\
4.2e-09	0.48	0	0	\\
4.25e-09	0.48	0	0	\\
4.3e-09	0.48	0	0	\\
4.35e-09	0.48	-0.00444444444444444	-0.00444444444444444	\\
4.4e-09	0.48	-0.00444444444444444	-0.00444444444444444	\\
4.45e-09	0.48	-0.00444444444444444	-0.00444444444444444	\\
4.5e-09	0.48	-0.00444444444444444	-0.00444444444444444	\\
4.55e-09	0.48	-0.00444444444444444	-0.00444444444444444	\\
4.6e-09	0.48	-0.00444444444444444	-0.00444444444444444	\\
4.65e-09	0.48	-0.00444444444444444	-0.00444444444444444	\\
4.7e-09	0.48	-0.00444444444444444	-0.00444444444444444	\\
4.75e-09	0.48	-0.00444444444444444	-0.00444444444444444	\\
4.8e-09	0.48	-0.00444444444444444	-0.00444444444444444	\\
4.85e-09	0.48	0	0	\\
4.9e-09	0.48	0	0	\\
4.95e-09	0.48	0	0	\\
5e-09	0.48	0	0	\\
5e-09	0.48	0	nan	\\
5e-09	0.48	-0.00666666666666667	0.00166666666666667	\\
5e-09	0.48	-0.00666666666666667	nan	\\
0	0.492	-0.00666666666666667	nan	\\
0	0.492	0	0.00166666666666667	\\
0	0.492	0	0	\\
5e-11	0.492	0	0	\\
1e-10	0.492	0	0	\\
1.5e-10	0.492	0	0	\\
2e-10	0.492	0	0	\\
2.5e-10	0.492	0	0	\\
3e-10	0.492	0	0	\\
3.5e-10	0.492	0	0	\\
4e-10	0.492	0	0	\\
4.5e-10	0.492	0	0	\\
5e-10	0.492	0.01	0.01	\\
5.5e-10	0.492	0.01	0.01	\\
6e-10	0.492	0.01	0.01	\\
6.5e-10	0.492	0.01	0.01	\\
7e-10	0.492	0.01	0.01	\\
7.5e-10	0.492	0.01	0.01	\\
8e-10	0.492	0.01	0.01	\\
8.5e-10	0.492	0.01	0.01	\\
9e-10	0.492	0.01	0.01	\\
9.5e-10	0.492	0.01	0.01	\\
1e-09	0.492	0	0	\\
1.05e-09	0.492	0	0	\\
1.1e-09	0.492	0	0	\\
1.15e-09	0.492	0	0	\\
1.2e-09	0.492	0	0	\\
1.25e-09	0.492	0	0	\\
1.3e-09	0.492	0	0	\\
1.35e-09	0.492	0	0	\\
1.4e-09	0.492	0	0	\\
1.45e-09	0.492	0	0	\\
1.5e-09	0.492	0	0	\\
1.55e-09	0.492	0	0	\\
1.6e-09	0.492	0	0	\\
1.65e-09	0.492	0	0	\\
1.7e-09	0.492	0	0	\\
1.75e-09	0.492	0	0	\\
1.8e-09	0.492	0	0	\\
1.85e-09	0.492	0	0	\\
1.9e-09	0.492	0	0	\\
1.95e-09	0.492	-0.00666666666666667	-0.00666666666666667	\\
2e-09	0.492	-0.00666666666666667	-0.00666666666666667	\\
2.05e-09	0.492	-0.00666666666666667	-0.00666666666666667	\\
2.1e-09	0.492	-0.00666666666666667	-0.00666666666666667	\\
2.15e-09	0.492	-0.00666666666666667	-0.00666666666666667	\\
2.2e-09	0.492	-0.00666666666666667	-0.00666666666666667	\\
2.25e-09	0.492	-0.00666666666666667	-0.00666666666666667	\\
2.3e-09	0.492	-0.00666666666666667	-0.00666666666666667	\\
2.35e-09	0.492	-0.00666666666666667	-0.00666666666666667	\\
2.4e-09	0.492	-0.00666666666666667	-0.00666666666666667	\\
2.45e-09	0.492	0	0	\\
2.5e-09	0.492	0	0	\\
2.55e-09	0.492	0	0	\\
2.6e-09	0.492	0	0	\\
2.65e-09	0.492	0	0	\\
2.7e-09	0.492	0	0	\\
2.75e-09	0.492	0	0	\\
2.8e-09	0.492	0	0	\\
2.85e-09	0.492	0	0	\\
2.9e-09	0.492	0.00666666666666667	0.00666666666666667	\\
2.95e-09	0.492	0.00666666666666667	0.00666666666666667	\\
3e-09	0.492	0.00666666666666667	0.00666666666666667	\\
3.05e-09	0.492	0.00666666666666667	0.00666666666666667	\\
3.1e-09	0.492	0.00666666666666667	0.00666666666666667	\\
3.15e-09	0.492	0.00666666666666667	0.00666666666666667	\\
3.2e-09	0.492	0.00666666666666667	0.00666666666666667	\\
3.25e-09	0.492	0.00666666666666667	0.00666666666666667	\\
3.3e-09	0.492	0.00666666666666667	0.00666666666666667	\\
3.35e-09	0.492	0.00666666666666667	0.00666666666666667	\\
3.4e-09	0.492	0	0	\\
3.45e-09	0.492	0	0	\\
3.5e-09	0.492	0	0	\\
3.55e-09	0.492	0	0	\\
3.6e-09	0.492	0	0	\\
3.65e-09	0.492	0	0	\\
3.7e-09	0.492	0	0	\\
3.75e-09	0.492	0	0	\\
3.8e-09	0.492	0	0	\\
3.85e-09	0.492	0	0	\\
3.9e-09	0.492	0	0	\\
3.95e-09	0.492	0	0	\\
4e-09	0.492	0	0	\\
4.05e-09	0.492	0	0	\\
4.1e-09	0.492	0	0	\\
4.15e-09	0.492	0	0	\\
4.2e-09	0.492	0	0	\\
4.25e-09	0.492	0	0	\\
4.3e-09	0.492	0	0	\\
4.35e-09	0.492	-0.00444444444444444	-0.00444444444444444	\\
4.4e-09	0.492	-0.00444444444444444	-0.00444444444444444	\\
4.45e-09	0.492	-0.00444444444444444	-0.00444444444444444	\\
4.5e-09	0.492	-0.00444444444444444	-0.00444444444444444	\\
4.55e-09	0.492	-0.00444444444444444	-0.00444444444444444	\\
4.6e-09	0.492	-0.00444444444444444	-0.00444444444444444	\\
4.65e-09	0.492	-0.00444444444444444	-0.00444444444444444	\\
4.7e-09	0.492	-0.00444444444444444	-0.00444444444444444	\\
4.75e-09	0.492	-0.00444444444444444	-0.00444444444444444	\\
4.8e-09	0.492	-0.00444444444444444	-0.00444444444444444	\\
4.85e-09	0.492	0	0	\\
4.9e-09	0.492	0	0	\\
4.95e-09	0.492	0	0	\\
5e-09	0.492	0	0	\\
5e-09	0.492	0	nan	\\
5e-09	0.492	-0.00666666666666667	0.00166666666666667	\\
5e-09	0.492	-0.00666666666666667	nan	\\
0	0.504	-0.00666666666666667	nan	\\
0	0.504	0	0.00166666666666667	\\
0	0.504	0	0	\\
5e-11	0.504	0	0	\\
1e-10	0.504	0	0	\\
1.5e-10	0.504	0	0	\\
2e-10	0.504	0	0	\\
2.5e-10	0.504	0	0	\\
3e-10	0.504	0	0	\\
3.5e-10	0.504	0	0	\\
4e-10	0.504	0	0	\\
4.5e-10	0.504	0	0	\\
5e-10	0.504	0	0	\\
5.5e-10	0.504	0.01	0.01	\\
6e-10	0.504	0.01	0.01	\\
6.5e-10	0.504	0.01	0.01	\\
7e-10	0.504	0.01	0.01	\\
7.5e-10	0.504	0.01	0.01	\\
8e-10	0.504	0.01	0.01	\\
8.5e-10	0.504	0.01	0.01	\\
9e-10	0.504	0.01	0.01	\\
9.5e-10	0.504	0.01	0.01	\\
1e-09	0.504	0.01	0.01	\\
1.05e-09	0.504	0	0	\\
1.1e-09	0.504	0	0	\\
1.15e-09	0.504	0	0	\\
1.2e-09	0.504	0	0	\\
1.25e-09	0.504	0	0	\\
1.3e-09	0.504	0	0	\\
1.35e-09	0.504	0	0	\\
1.4e-09	0.504	0	0	\\
1.45e-09	0.504	0	0	\\
1.5e-09	0.504	0	0	\\
1.55e-09	0.504	0	0	\\
1.6e-09	0.504	0	0	\\
1.65e-09	0.504	0	0	\\
1.7e-09	0.504	0	0	\\
1.75e-09	0.504	0	0	\\
1.8e-09	0.504	0	0	\\
1.85e-09	0.504	0	0	\\
1.9e-09	0.504	-0.00666666666666667	-0.00666666666666667	\\
1.95e-09	0.504	-0.00666666666666667	-0.00666666666666667	\\
2e-09	0.504	-0.00666666666666667	-0.00666666666666667	\\
2.05e-09	0.504	-0.00666666666666667	-0.00666666666666667	\\
2.1e-09	0.504	-0.00666666666666667	-0.00666666666666667	\\
2.15e-09	0.504	-0.00666666666666667	-0.00666666666666667	\\
2.2e-09	0.504	-0.00666666666666667	-0.00666666666666667	\\
2.25e-09	0.504	-0.00666666666666667	-0.00666666666666667	\\
2.3e-09	0.504	-0.00666666666666667	-0.00666666666666667	\\
2.35e-09	0.504	-0.00666666666666667	-0.00666666666666667	\\
2.4e-09	0.504	0	0	\\
2.45e-09	0.504	0	0	\\
2.5e-09	0.504	0	0	\\
2.55e-09	0.504	0	0	\\
2.6e-09	0.504	0	0	\\
2.65e-09	0.504	0	0	\\
2.7e-09	0.504	0	0	\\
2.75e-09	0.504	0	0	\\
2.8e-09	0.504	0	0	\\
2.85e-09	0.504	0	0	\\
2.9e-09	0.504	0	0	\\
2.95e-09	0.504	0.00666666666666667	0.00666666666666667	\\
3e-09	0.504	0.00666666666666667	0.00666666666666667	\\
3.05e-09	0.504	0.00666666666666667	0.00666666666666667	\\
3.1e-09	0.504	0.00666666666666667	0.00666666666666667	\\
3.15e-09	0.504	0.00666666666666667	0.00666666666666667	\\
3.2e-09	0.504	0.00666666666666667	0.00666666666666667	\\
3.25e-09	0.504	0.00666666666666667	0.00666666666666667	\\
3.3e-09	0.504	0.00666666666666667	0.00666666666666667	\\
3.35e-09	0.504	0.00666666666666667	0.00666666666666667	\\
3.4e-09	0.504	0.00666666666666667	0.00666666666666667	\\
3.45e-09	0.504	0	0	\\
3.5e-09	0.504	0	0	\\
3.55e-09	0.504	0	0	\\
3.6e-09	0.504	0	0	\\
3.65e-09	0.504	0	0	\\
3.7e-09	0.504	0	0	\\
3.75e-09	0.504	0	0	\\
3.8e-09	0.504	0	0	\\
3.85e-09	0.504	0	0	\\
3.9e-09	0.504	0	0	\\
3.95e-09	0.504	0	0	\\
4e-09	0.504	0	0	\\
4.05e-09	0.504	0	0	\\
4.1e-09	0.504	0	0	\\
4.15e-09	0.504	0	0	\\
4.2e-09	0.504	0	0	\\
4.25e-09	0.504	0	0	\\
4.3e-09	0.504	-0.00444444444444444	-0.00444444444444444	\\
4.35e-09	0.504	-0.00444444444444444	-0.00444444444444444	\\
4.4e-09	0.504	-0.00444444444444444	-0.00444444444444444	\\
4.45e-09	0.504	-0.00444444444444444	-0.00444444444444444	\\
4.5e-09	0.504	-0.00444444444444444	-0.00444444444444444	\\
4.55e-09	0.504	-0.00444444444444444	-0.00444444444444444	\\
4.6e-09	0.504	-0.00444444444444444	-0.00444444444444444	\\
4.65e-09	0.504	-0.00444444444444444	-0.00444444444444444	\\
4.7e-09	0.504	-0.00444444444444444	-0.00444444444444444	\\
4.75e-09	0.504	-0.00444444444444444	-0.00444444444444444	\\
4.8e-09	0.504	0	0	\\
4.85e-09	0.504	0	0	\\
4.9e-09	0.504	0	0	\\
4.95e-09	0.504	0	0	\\
5e-09	0.504	0	0	\\
5e-09	0.504	0	nan	\\
5e-09	0.504	-0.00666666666666667	0.00166666666666667	\\
5e-09	0.504	-0.00666666666666667	nan	\\
0	0.516	-0.00666666666666667	nan	\\
0	0.516	0	0.00166666666666667	\\
0	0.516	0	0	\\
5e-11	0.516	0	0	\\
1e-10	0.516	0	0	\\
1.5e-10	0.516	0	0	\\
2e-10	0.516	0	0	\\
2.5e-10	0.516	0	0	\\
3e-10	0.516	0	0	\\
3.5e-10	0.516	0	0	\\
4e-10	0.516	0	0	\\
4.5e-10	0.516	0	0	\\
5e-10	0.516	0	0	\\
5.5e-10	0.516	0.01	0.01	\\
6e-10	0.516	0.01	0.01	\\
6.5e-10	0.516	0.01	0.01	\\
7e-10	0.516	0.01	0.01	\\
7.5e-10	0.516	0.01	0.01	\\
8e-10	0.516	0.01	0.01	\\
8.5e-10	0.516	0.01	0.01	\\
9e-10	0.516	0.01	0.01	\\
9.5e-10	0.516	0.01	0.01	\\
1e-09	0.516	0.01	0.01	\\
1.05e-09	0.516	0	0	\\
1.1e-09	0.516	0	0	\\
1.15e-09	0.516	0	0	\\
1.2e-09	0.516	0	0	\\
1.25e-09	0.516	0	0	\\
1.3e-09	0.516	0	0	\\
1.35e-09	0.516	0	0	\\
1.4e-09	0.516	0	0	\\
1.45e-09	0.516	0	0	\\
1.5e-09	0.516	0	0	\\
1.55e-09	0.516	0	0	\\
1.6e-09	0.516	0	0	\\
1.65e-09	0.516	0	0	\\
1.7e-09	0.516	0	0	\\
1.75e-09	0.516	0	0	\\
1.8e-09	0.516	0	0	\\
1.85e-09	0.516	0	0	\\
1.9e-09	0.516	-0.00666666666666667	-0.00666666666666667	\\
1.95e-09	0.516	-0.00666666666666667	-0.00666666666666667	\\
2e-09	0.516	-0.00666666666666667	-0.00666666666666667	\\
2.05e-09	0.516	-0.00666666666666667	-0.00666666666666667	\\
2.1e-09	0.516	-0.00666666666666667	-0.00666666666666667	\\
2.15e-09	0.516	-0.00666666666666667	-0.00666666666666667	\\
2.2e-09	0.516	-0.00666666666666667	-0.00666666666666667	\\
2.25e-09	0.516	-0.00666666666666667	-0.00666666666666667	\\
2.3e-09	0.516	-0.00666666666666667	-0.00666666666666667	\\
2.35e-09	0.516	-0.00666666666666667	-0.00666666666666667	\\
2.4e-09	0.516	0	0	\\
2.45e-09	0.516	0	0	\\
2.5e-09	0.516	0	0	\\
2.55e-09	0.516	0	0	\\
2.6e-09	0.516	0	0	\\
2.65e-09	0.516	0	0	\\
2.7e-09	0.516	0	0	\\
2.75e-09	0.516	0	0	\\
2.8e-09	0.516	0	0	\\
2.85e-09	0.516	0	0	\\
2.9e-09	0.516	0	0	\\
2.95e-09	0.516	0.00666666666666667	0.00666666666666667	\\
3e-09	0.516	0.00666666666666667	0.00666666666666667	\\
3.05e-09	0.516	0.00666666666666667	0.00666666666666667	\\
3.1e-09	0.516	0.00666666666666667	0.00666666666666667	\\
3.15e-09	0.516	0.00666666666666667	0.00666666666666667	\\
3.2e-09	0.516	0.00666666666666667	0.00666666666666667	\\
3.25e-09	0.516	0.00666666666666667	0.00666666666666667	\\
3.3e-09	0.516	0.00666666666666667	0.00666666666666667	\\
3.35e-09	0.516	0.00666666666666667	0.00666666666666667	\\
3.4e-09	0.516	0.00666666666666667	0.00666666666666667	\\
3.45e-09	0.516	0	0	\\
3.5e-09	0.516	0	0	\\
3.55e-09	0.516	0	0	\\
3.6e-09	0.516	0	0	\\
3.65e-09	0.516	0	0	\\
3.7e-09	0.516	0	0	\\
3.75e-09	0.516	0	0	\\
3.8e-09	0.516	0	0	\\
3.85e-09	0.516	0	0	\\
3.9e-09	0.516	0	0	\\
3.95e-09	0.516	0	0	\\
4e-09	0.516	0	0	\\
4.05e-09	0.516	0	0	\\
4.1e-09	0.516	0	0	\\
4.15e-09	0.516	0	0	\\
4.2e-09	0.516	0	0	\\
4.25e-09	0.516	0	0	\\
4.3e-09	0.516	-0.00444444444444444	-0.00444444444444444	\\
4.35e-09	0.516	-0.00444444444444444	-0.00444444444444444	\\
4.4e-09	0.516	-0.00444444444444444	-0.00444444444444444	\\
4.45e-09	0.516	-0.00444444444444444	-0.00444444444444444	\\
4.5e-09	0.516	-0.00444444444444444	-0.00444444444444444	\\
4.55e-09	0.516	-0.00444444444444444	-0.00444444444444444	\\
4.6e-09	0.516	-0.00444444444444444	-0.00444444444444444	\\
4.65e-09	0.516	-0.00444444444444444	-0.00444444444444444	\\
4.7e-09	0.516	-0.00444444444444444	-0.00444444444444444	\\
4.75e-09	0.516	-0.00444444444444444	-0.00444444444444444	\\
4.8e-09	0.516	0	0	\\
4.85e-09	0.516	0	0	\\
4.9e-09	0.516	0	0	\\
4.95e-09	0.516	0	0	\\
5e-09	0.516	0	0	\\
5e-09	0.516	0	nan	\\
5e-09	0.516	-0.00666666666666667	0.00166666666666667	\\
5e-09	0.516	-0.00666666666666667	nan	\\
0	0.528	-0.00666666666666667	nan	\\
0	0.528	0	0.00166666666666667	\\
0	0.528	0	0	\\
5e-11	0.528	0	0	\\
1e-10	0.528	0	0	\\
1.5e-10	0.528	0	0	\\
2e-10	0.528	0	0	\\
2.5e-10	0.528	0	0	\\
3e-10	0.528	0	0	\\
3.5e-10	0.528	0	0	\\
4e-10	0.528	0	0	\\
4.5e-10	0.528	0	0	\\
5e-10	0.528	0	0	\\
5.5e-10	0.528	0.01	0.01	\\
6e-10	0.528	0.01	0.01	\\
6.5e-10	0.528	0.01	0.01	\\
7e-10	0.528	0.01	0.01	\\
7.5e-10	0.528	0.01	0.01	\\
8e-10	0.528	0.01	0.01	\\
8.5e-10	0.528	0.01	0.01	\\
9e-10	0.528	0.01	0.01	\\
9.5e-10	0.528	0.01	0.01	\\
1e-09	0.528	0.01	0.01	\\
1.05e-09	0.528	0	0	\\
1.1e-09	0.528	0	0	\\
1.15e-09	0.528	0	0	\\
1.2e-09	0.528	0	0	\\
1.25e-09	0.528	0	0	\\
1.3e-09	0.528	0	0	\\
1.35e-09	0.528	0	0	\\
1.4e-09	0.528	0	0	\\
1.45e-09	0.528	0	0	\\
1.5e-09	0.528	0	0	\\
1.55e-09	0.528	0	0	\\
1.6e-09	0.528	0	0	\\
1.65e-09	0.528	0	0	\\
1.7e-09	0.528	0	0	\\
1.75e-09	0.528	0	0	\\
1.8e-09	0.528	0	0	\\
1.85e-09	0.528	0	0	\\
1.9e-09	0.528	-0.00666666666666667	-0.00666666666666667	\\
1.95e-09	0.528	-0.00666666666666667	-0.00666666666666667	\\
2e-09	0.528	-0.00666666666666667	-0.00666666666666667	\\
2.05e-09	0.528	-0.00666666666666667	-0.00666666666666667	\\
2.1e-09	0.528	-0.00666666666666667	-0.00666666666666667	\\
2.15e-09	0.528	-0.00666666666666667	-0.00666666666666667	\\
2.2e-09	0.528	-0.00666666666666667	-0.00666666666666667	\\
2.25e-09	0.528	-0.00666666666666667	-0.00666666666666667	\\
2.3e-09	0.528	-0.00666666666666667	-0.00666666666666667	\\
2.35e-09	0.528	-0.00666666666666667	-0.00666666666666667	\\
2.4e-09	0.528	0	0	\\
2.45e-09	0.528	0	0	\\
2.5e-09	0.528	0	0	\\
2.55e-09	0.528	0	0	\\
2.6e-09	0.528	0	0	\\
2.65e-09	0.528	0	0	\\
2.7e-09	0.528	0	0	\\
2.75e-09	0.528	0	0	\\
2.8e-09	0.528	0	0	\\
2.85e-09	0.528	0	0	\\
2.9e-09	0.528	0	0	\\
2.95e-09	0.528	0.00666666666666667	0.00666666666666667	\\
3e-09	0.528	0.00666666666666667	0.00666666666666667	\\
3.05e-09	0.528	0.00666666666666667	0.00666666666666667	\\
3.1e-09	0.528	0.00666666666666667	0.00666666666666667	\\
3.15e-09	0.528	0.00666666666666667	0.00666666666666667	\\
3.2e-09	0.528	0.00666666666666667	0.00666666666666667	\\
3.25e-09	0.528	0.00666666666666667	0.00666666666666667	\\
3.3e-09	0.528	0.00666666666666667	0.00666666666666667	\\
3.35e-09	0.528	0.00666666666666667	0.00666666666666667	\\
3.4e-09	0.528	0.00666666666666667	0.00666666666666667	\\
3.45e-09	0.528	0	0	\\
3.5e-09	0.528	0	0	\\
3.55e-09	0.528	0	0	\\
3.6e-09	0.528	0	0	\\
3.65e-09	0.528	0	0	\\
3.7e-09	0.528	0	0	\\
3.75e-09	0.528	0	0	\\
3.8e-09	0.528	0	0	\\
3.85e-09	0.528	0	0	\\
3.9e-09	0.528	0	0	\\
3.95e-09	0.528	0	0	\\
4e-09	0.528	0	0	\\
4.05e-09	0.528	0	0	\\
4.1e-09	0.528	0	0	\\
4.15e-09	0.528	0	0	\\
4.2e-09	0.528	0	0	\\
4.25e-09	0.528	0	0	\\
4.3e-09	0.528	-0.00444444444444444	-0.00444444444444444	\\
4.35e-09	0.528	-0.00444444444444444	-0.00444444444444444	\\
4.4e-09	0.528	-0.00444444444444444	-0.00444444444444444	\\
4.45e-09	0.528	-0.00444444444444444	-0.00444444444444444	\\
4.5e-09	0.528	-0.00444444444444444	-0.00444444444444444	\\
4.55e-09	0.528	-0.00444444444444444	-0.00444444444444444	\\
4.6e-09	0.528	-0.00444444444444444	-0.00444444444444444	\\
4.65e-09	0.528	-0.00444444444444444	-0.00444444444444444	\\
4.7e-09	0.528	-0.00444444444444444	-0.00444444444444444	\\
4.75e-09	0.528	-0.00444444444444444	-0.00444444444444444	\\
4.8e-09	0.528	0	0	\\
4.85e-09	0.528	0	0	\\
4.9e-09	0.528	0	0	\\
4.95e-09	0.528	0	0	\\
5e-09	0.528	0	0	\\
5e-09	0.528	0	nan	\\
5e-09	0.528	-0.00666666666666667	0.00166666666666667	\\
5e-09	0.528	-0.00666666666666667	nan	\\
0	0.54	-0.00666666666666667	nan	\\
0	0.54	0	0.00166666666666667	\\
0	0.54	0	0	\\
5e-11	0.54	0	0	\\
1e-10	0.54	0	0	\\
1.5e-10	0.54	0	0	\\
2e-10	0.54	0	0	\\
2.5e-10	0.54	0	0	\\
3e-10	0.54	0	0	\\
3.5e-10	0.54	0	0	\\
4e-10	0.54	0	0	\\
4.5e-10	0.54	0	0	\\
5e-10	0.54	0	0	\\
5.5e-10	0.54	0.01	0.01	\\
6e-10	0.54	0.01	0.01	\\
6.5e-10	0.54	0.01	0.01	\\
7e-10	0.54	0.01	0.01	\\
7.5e-10	0.54	0.01	0.01	\\
8e-10	0.54	0.01	0.01	\\
8.5e-10	0.54	0.01	0.01	\\
9e-10	0.54	0.01	0.01	\\
9.5e-10	0.54	0.01	0.01	\\
1e-09	0.54	0.01	0.01	\\
1.05e-09	0.54	0	0	\\
1.1e-09	0.54	0	0	\\
1.15e-09	0.54	0	0	\\
1.2e-09	0.54	0	0	\\
1.25e-09	0.54	0	0	\\
1.3e-09	0.54	0	0	\\
1.35e-09	0.54	0	0	\\
1.4e-09	0.54	0	0	\\
1.45e-09	0.54	0	0	\\
1.5e-09	0.54	0	0	\\
1.55e-09	0.54	0	0	\\
1.6e-09	0.54	0	0	\\
1.65e-09	0.54	0	0	\\
1.7e-09	0.54	0	0	\\
1.75e-09	0.54	0	0	\\
1.8e-09	0.54	0	0	\\
1.85e-09	0.54	0	0	\\
1.9e-09	0.54	-0.00666666666666667	-0.00666666666666667	\\
1.95e-09	0.54	-0.00666666666666667	-0.00666666666666667	\\
2e-09	0.54	-0.00666666666666667	-0.00666666666666667	\\
2.05e-09	0.54	-0.00666666666666667	-0.00666666666666667	\\
2.1e-09	0.54	-0.00666666666666667	-0.00666666666666667	\\
2.15e-09	0.54	-0.00666666666666667	-0.00666666666666667	\\
2.2e-09	0.54	-0.00666666666666667	-0.00666666666666667	\\
2.25e-09	0.54	-0.00666666666666667	-0.00666666666666667	\\
2.3e-09	0.54	-0.00666666666666667	-0.00666666666666667	\\
2.35e-09	0.54	-0.00666666666666667	-0.00666666666666667	\\
2.4e-09	0.54	0	0	\\
2.45e-09	0.54	0	0	\\
2.5e-09	0.54	0	0	\\
2.55e-09	0.54	0	0	\\
2.6e-09	0.54	0	0	\\
2.65e-09	0.54	0	0	\\
2.7e-09	0.54	0	0	\\
2.75e-09	0.54	0	0	\\
2.8e-09	0.54	0	0	\\
2.85e-09	0.54	0	0	\\
2.9e-09	0.54	0	0	\\
2.95e-09	0.54	0.00666666666666667	0.00666666666666667	\\
3e-09	0.54	0.00666666666666667	0.00666666666666667	\\
3.05e-09	0.54	0.00666666666666667	0.00666666666666667	\\
3.1e-09	0.54	0.00666666666666667	0.00666666666666667	\\
3.15e-09	0.54	0.00666666666666667	0.00666666666666667	\\
3.2e-09	0.54	0.00666666666666667	0.00666666666666667	\\
3.25e-09	0.54	0.00666666666666667	0.00666666666666667	\\
3.3e-09	0.54	0.00666666666666667	0.00666666666666667	\\
3.35e-09	0.54	0.00666666666666667	0.00666666666666667	\\
3.4e-09	0.54	0.00666666666666667	0.00666666666666667	\\
3.45e-09	0.54	0	0	\\
3.5e-09	0.54	0	0	\\
3.55e-09	0.54	0	0	\\
3.6e-09	0.54	0	0	\\
3.65e-09	0.54	0	0	\\
3.7e-09	0.54	0	0	\\
3.75e-09	0.54	0	0	\\
3.8e-09	0.54	0	0	\\
3.85e-09	0.54	0	0	\\
3.9e-09	0.54	0	0	\\
3.95e-09	0.54	0	0	\\
4e-09	0.54	0	0	\\
4.05e-09	0.54	0	0	\\
4.1e-09	0.54	0	0	\\
4.15e-09	0.54	0	0	\\
4.2e-09	0.54	0	0	\\
4.25e-09	0.54	0	0	\\
4.3e-09	0.54	-0.00444444444444444	-0.00444444444444444	\\
4.35e-09	0.54	-0.00444444444444444	-0.00444444444444444	\\
4.4e-09	0.54	-0.00444444444444444	-0.00444444444444444	\\
4.45e-09	0.54	-0.00444444444444444	-0.00444444444444444	\\
4.5e-09	0.54	-0.00444444444444444	-0.00444444444444444	\\
4.55e-09	0.54	-0.00444444444444444	-0.00444444444444444	\\
4.6e-09	0.54	-0.00444444444444444	-0.00444444444444444	\\
4.65e-09	0.54	-0.00444444444444444	-0.00444444444444444	\\
4.7e-09	0.54	-0.00444444444444444	-0.00444444444444444	\\
4.75e-09	0.54	-0.00444444444444444	-0.00444444444444444	\\
4.8e-09	0.54	0	0	\\
4.85e-09	0.54	0	0	\\
4.9e-09	0.54	0	0	\\
4.95e-09	0.54	0	0	\\
5e-09	0.54	0	0	\\
5e-09	0.54	0	nan	\\
5e-09	0.54	-0.00666666666666667	0.00166666666666667	\\
5e-09	0.54	-0.00666666666666667	nan	\\
0	0.552	-0.00666666666666667	nan	\\
0	0.552	0	0.00166666666666667	\\
0	0.552	0	0	\\
5e-11	0.552	0	0	\\
1e-10	0.552	0	0	\\
1.5e-10	0.552	0	0	\\
2e-10	0.552	0	0	\\
2.5e-10	0.552	0	0	\\
3e-10	0.552	0	0	\\
3.5e-10	0.552	0	0	\\
4e-10	0.552	0	0	\\
4.5e-10	0.552	0	0	\\
5e-10	0.552	0	0	\\
5.5e-10	0.552	0	0	\\
6e-10	0.552	0.01	0.01	\\
6.5e-10	0.552	0.01	0.01	\\
7e-10	0.552	0.01	0.01	\\
7.5e-10	0.552	0.01	0.01	\\
8e-10	0.552	0.01	0.01	\\
8.5e-10	0.552	0.01	0.01	\\
9e-10	0.552	0.01	0.01	\\
9.5e-10	0.552	0.01	0.01	\\
1e-09	0.552	0.01	0.01	\\
1.05e-09	0.552	0.01	0.01	\\
1.1e-09	0.552	0	0	\\
1.15e-09	0.552	0	0	\\
1.2e-09	0.552	0	0	\\
1.25e-09	0.552	0	0	\\
1.3e-09	0.552	0	0	\\
1.35e-09	0.552	0	0	\\
1.4e-09	0.552	0	0	\\
1.45e-09	0.552	0	0	\\
1.5e-09	0.552	0	0	\\
1.55e-09	0.552	0	0	\\
1.6e-09	0.552	0	0	\\
1.65e-09	0.552	0	0	\\
1.7e-09	0.552	0	0	\\
1.75e-09	0.552	0	0	\\
1.8e-09	0.552	0	0	\\
1.85e-09	0.552	-0.00666666666666667	-0.00666666666666667	\\
1.9e-09	0.552	-0.00666666666666667	-0.00666666666666667	\\
1.95e-09	0.552	-0.00666666666666667	-0.00666666666666667	\\
2e-09	0.552	-0.00666666666666667	-0.00666666666666667	\\
2.05e-09	0.552	-0.00666666666666667	-0.00666666666666667	\\
2.1e-09	0.552	-0.00666666666666667	-0.00666666666666667	\\
2.15e-09	0.552	-0.00666666666666667	-0.00666666666666667	\\
2.2e-09	0.552	-0.00666666666666667	-0.00666666666666667	\\
2.25e-09	0.552	-0.00666666666666667	-0.00666666666666667	\\
2.3e-09	0.552	-0.00666666666666667	-0.00666666666666667	\\
2.35e-09	0.552	0	0	\\
2.4e-09	0.552	0	0	\\
2.45e-09	0.552	0	0	\\
2.5e-09	0.552	0	0	\\
2.55e-09	0.552	0	0	\\
2.6e-09	0.552	0	0	\\
2.65e-09	0.552	0	0	\\
2.7e-09	0.552	0	0	\\
2.75e-09	0.552	0	0	\\
2.8e-09	0.552	0	0	\\
2.85e-09	0.552	0	0	\\
2.9e-09	0.552	0	0	\\
2.95e-09	0.552	0	0	\\
3e-09	0.552	0.00666666666666667	0.00666666666666667	\\
3.05e-09	0.552	0.00666666666666667	0.00666666666666667	\\
3.1e-09	0.552	0.00666666666666667	0.00666666666666667	\\
3.15e-09	0.552	0.00666666666666667	0.00666666666666667	\\
3.2e-09	0.552	0.00666666666666667	0.00666666666666667	\\
3.25e-09	0.552	0.00666666666666667	0.00666666666666667	\\
3.3e-09	0.552	0.00666666666666667	0.00666666666666667	\\
3.35e-09	0.552	0.00666666666666667	0.00666666666666667	\\
3.4e-09	0.552	0.00666666666666667	0.00666666666666667	\\
3.45e-09	0.552	0.00666666666666667	0.00666666666666667	\\
3.5e-09	0.552	0	0	\\
3.55e-09	0.552	0	0	\\
3.6e-09	0.552	0	0	\\
3.65e-09	0.552	0	0	\\
3.7e-09	0.552	0	0	\\
3.75e-09	0.552	0	0	\\
3.8e-09	0.552	0	0	\\
3.85e-09	0.552	0	0	\\
3.9e-09	0.552	0	0	\\
3.95e-09	0.552	0	0	\\
4e-09	0.552	0	0	\\
4.05e-09	0.552	0	0	\\
4.1e-09	0.552	0	0	\\
4.15e-09	0.552	0	0	\\
4.2e-09	0.552	0	0	\\
4.25e-09	0.552	-0.00444444444444444	-0.00444444444444444	\\
4.3e-09	0.552	-0.00444444444444444	-0.00444444444444444	\\
4.35e-09	0.552	-0.00444444444444444	-0.00444444444444444	\\
4.4e-09	0.552	-0.00444444444444444	-0.00444444444444444	\\
4.45e-09	0.552	-0.00444444444444444	-0.00444444444444444	\\
4.5e-09	0.552	-0.00444444444444444	-0.00444444444444444	\\
4.55e-09	0.552	-0.00444444444444444	-0.00444444444444444	\\
4.6e-09	0.552	-0.00444444444444444	-0.00444444444444444	\\
4.65e-09	0.552	-0.00444444444444444	-0.00444444444444444	\\
4.7e-09	0.552	-0.00444444444444444	-0.00444444444444444	\\
4.75e-09	0.552	0	0	\\
4.8e-09	0.552	0	0	\\
4.85e-09	0.552	0	0	\\
4.9e-09	0.552	0	0	\\
4.95e-09	0.552	0	0	\\
5e-09	0.552	0	0	\\
5e-09	0.552	0	nan	\\
5e-09	0.552	-0.00666666666666667	0.00166666666666667	\\
5e-09	0.552	-0.00666666666666667	nan	\\
0	0.564	-0.00666666666666667	nan	\\
0	0.564	0	0.00166666666666667	\\
0	0.564	0	0	\\
5e-11	0.564	0	0	\\
1e-10	0.564	0	0	\\
1.5e-10	0.564	0	0	\\
2e-10	0.564	0	0	\\
2.5e-10	0.564	0	0	\\
3e-10	0.564	0	0	\\
3.5e-10	0.564	0	0	\\
4e-10	0.564	0	0	\\
4.5e-10	0.564	0	0	\\
5e-10	0.564	0	0	\\
5.5e-10	0.564	0	0	\\
6e-10	0.564	0.01	0.01	\\
6.5e-10	0.564	0.01	0.01	\\
7e-10	0.564	0.01	0.01	\\
7.5e-10	0.564	0.01	0.01	\\
8e-10	0.564	0.01	0.01	\\
8.5e-10	0.564	0.01	0.01	\\
9e-10	0.564	0.01	0.01	\\
9.5e-10	0.564	0.01	0.01	\\
1e-09	0.564	0.01	0.01	\\
1.05e-09	0.564	0.01	0.01	\\
1.1e-09	0.564	0	0	\\
1.15e-09	0.564	0	0	\\
1.2e-09	0.564	0	0	\\
1.25e-09	0.564	0	0	\\
1.3e-09	0.564	0	0	\\
1.35e-09	0.564	0	0	\\
1.4e-09	0.564	0	0	\\
1.45e-09	0.564	0	0	\\
1.5e-09	0.564	0	0	\\
1.55e-09	0.564	0	0	\\
1.6e-09	0.564	0	0	\\
1.65e-09	0.564	0	0	\\
1.7e-09	0.564	0	0	\\
1.75e-09	0.564	0	0	\\
1.8e-09	0.564	0	0	\\
1.85e-09	0.564	-0.00666666666666667	-0.00666666666666667	\\
1.9e-09	0.564	-0.00666666666666667	-0.00666666666666667	\\
1.95e-09	0.564	-0.00666666666666667	-0.00666666666666667	\\
2e-09	0.564	-0.00666666666666667	-0.00666666666666667	\\
2.05e-09	0.564	-0.00666666666666667	-0.00666666666666667	\\
2.1e-09	0.564	-0.00666666666666667	-0.00666666666666667	\\
2.15e-09	0.564	-0.00666666666666667	-0.00666666666666667	\\
2.2e-09	0.564	-0.00666666666666667	-0.00666666666666667	\\
2.25e-09	0.564	-0.00666666666666667	-0.00666666666666667	\\
2.3e-09	0.564	-0.00666666666666667	-0.00666666666666667	\\
2.35e-09	0.564	0	0	\\
2.4e-09	0.564	0	0	\\
2.45e-09	0.564	0	0	\\
2.5e-09	0.564	0	0	\\
2.55e-09	0.564	0	0	\\
2.6e-09	0.564	0	0	\\
2.65e-09	0.564	0	0	\\
2.7e-09	0.564	0	0	\\
2.75e-09	0.564	0	0	\\
2.8e-09	0.564	0	0	\\
2.85e-09	0.564	0	0	\\
2.9e-09	0.564	0	0	\\
2.95e-09	0.564	0	0	\\
3e-09	0.564	0.00666666666666667	0.00666666666666667	\\
3.05e-09	0.564	0.00666666666666667	0.00666666666666667	\\
3.1e-09	0.564	0.00666666666666667	0.00666666666666667	\\
3.15e-09	0.564	0.00666666666666667	0.00666666666666667	\\
3.2e-09	0.564	0.00666666666666667	0.00666666666666667	\\
3.25e-09	0.564	0.00666666666666667	0.00666666666666667	\\
3.3e-09	0.564	0.00666666666666667	0.00666666666666667	\\
3.35e-09	0.564	0.00666666666666667	0.00666666666666667	\\
3.4e-09	0.564	0.00666666666666667	0.00666666666666667	\\
3.45e-09	0.564	0.00666666666666667	0.00666666666666667	\\
3.5e-09	0.564	0	0	\\
3.55e-09	0.564	0	0	\\
3.6e-09	0.564	0	0	\\
3.65e-09	0.564	0	0	\\
3.7e-09	0.564	0	0	\\
3.75e-09	0.564	0	0	\\
3.8e-09	0.564	0	0	\\
3.85e-09	0.564	0	0	\\
3.9e-09	0.564	0	0	\\
3.95e-09	0.564	0	0	\\
4e-09	0.564	0	0	\\
4.05e-09	0.564	0	0	\\
4.1e-09	0.564	0	0	\\
4.15e-09	0.564	0	0	\\
4.2e-09	0.564	0	0	\\
4.25e-09	0.564	-0.00444444444444444	-0.00444444444444444	\\
4.3e-09	0.564	-0.00444444444444444	-0.00444444444444444	\\
4.35e-09	0.564	-0.00444444444444444	-0.00444444444444444	\\
4.4e-09	0.564	-0.00444444444444444	-0.00444444444444444	\\
4.45e-09	0.564	-0.00444444444444444	-0.00444444444444444	\\
4.5e-09	0.564	-0.00444444444444444	-0.00444444444444444	\\
4.55e-09	0.564	-0.00444444444444444	-0.00444444444444444	\\
4.6e-09	0.564	-0.00444444444444444	-0.00444444444444444	\\
4.65e-09	0.564	-0.00444444444444444	-0.00444444444444444	\\
4.7e-09	0.564	-0.00444444444444444	-0.00444444444444444	\\
4.75e-09	0.564	0	0	\\
4.8e-09	0.564	0	0	\\
4.85e-09	0.564	0	0	\\
4.9e-09	0.564	0	0	\\
4.95e-09	0.564	0	0	\\
5e-09	0.564	0	0	\\
5e-09	0.564	0	nan	\\
5e-09	0.564	-0.00666666666666667	0.00166666666666667	\\
5e-09	0.564	-0.00666666666666667	nan	\\
0	0.576	-0.00666666666666667	nan	\\
0	0.576	0	0.00166666666666667	\\
0	0.576	0	0	\\
5e-11	0.576	0	0	\\
1e-10	0.576	0	0	\\
1.5e-10	0.576	0	0	\\
2e-10	0.576	0	0	\\
2.5e-10	0.576	0	0	\\
3e-10	0.576	0	0	\\
3.5e-10	0.576	0	0	\\
4e-10	0.576	0	0	\\
4.5e-10	0.576	0	0	\\
5e-10	0.576	0	0	\\
5.5e-10	0.576	0	0	\\
6e-10	0.576	0.01	0.01	\\
6.5e-10	0.576	0.01	0.01	\\
7e-10	0.576	0.01	0.01	\\
7.5e-10	0.576	0.01	0.01	\\
8e-10	0.576	0.01	0.01	\\
8.5e-10	0.576	0.01	0.01	\\
9e-10	0.576	0.01	0.01	\\
9.5e-10	0.576	0.01	0.01	\\
1e-09	0.576	0.01	0.01	\\
1.05e-09	0.576	0.01	0.01	\\
1.1e-09	0.576	0	0	\\
1.15e-09	0.576	0	0	\\
1.2e-09	0.576	0	0	\\
1.25e-09	0.576	0	0	\\
1.3e-09	0.576	0	0	\\
1.35e-09	0.576	0	0	\\
1.4e-09	0.576	0	0	\\
1.45e-09	0.576	0	0	\\
1.5e-09	0.576	0	0	\\
1.55e-09	0.576	0	0	\\
1.6e-09	0.576	0	0	\\
1.65e-09	0.576	0	0	\\
1.7e-09	0.576	0	0	\\
1.75e-09	0.576	0	0	\\
1.8e-09	0.576	0	0	\\
1.85e-09	0.576	-0.00666666666666667	-0.00666666666666667	\\
1.9e-09	0.576	-0.00666666666666667	-0.00666666666666667	\\
1.95e-09	0.576	-0.00666666666666667	-0.00666666666666667	\\
2e-09	0.576	-0.00666666666666667	-0.00666666666666667	\\
2.05e-09	0.576	-0.00666666666666667	-0.00666666666666667	\\
2.1e-09	0.576	-0.00666666666666667	-0.00666666666666667	\\
2.15e-09	0.576	-0.00666666666666667	-0.00666666666666667	\\
2.2e-09	0.576	-0.00666666666666667	-0.00666666666666667	\\
2.25e-09	0.576	-0.00666666666666667	-0.00666666666666667	\\
2.3e-09	0.576	-0.00666666666666667	-0.00666666666666667	\\
2.35e-09	0.576	0	0	\\
2.4e-09	0.576	0	0	\\
2.45e-09	0.576	0	0	\\
2.5e-09	0.576	0	0	\\
2.55e-09	0.576	0	0	\\
2.6e-09	0.576	0	0	\\
2.65e-09	0.576	0	0	\\
2.7e-09	0.576	0	0	\\
2.75e-09	0.576	0	0	\\
2.8e-09	0.576	0	0	\\
2.85e-09	0.576	0	0	\\
2.9e-09	0.576	0	0	\\
2.95e-09	0.576	0	0	\\
3e-09	0.576	0.00666666666666667	0.00666666666666667	\\
3.05e-09	0.576	0.00666666666666667	0.00666666666666667	\\
3.1e-09	0.576	0.00666666666666667	0.00666666666666667	\\
3.15e-09	0.576	0.00666666666666667	0.00666666666666667	\\
3.2e-09	0.576	0.00666666666666667	0.00666666666666667	\\
3.25e-09	0.576	0.00666666666666667	0.00666666666666667	\\
3.3e-09	0.576	0.00666666666666667	0.00666666666666667	\\
3.35e-09	0.576	0.00666666666666667	0.00666666666666667	\\
3.4e-09	0.576	0.00666666666666667	0.00666666666666667	\\
3.45e-09	0.576	0.00666666666666667	0.00666666666666667	\\
3.5e-09	0.576	0	0	\\
3.55e-09	0.576	0	0	\\
3.6e-09	0.576	0	0	\\
3.65e-09	0.576	0	0	\\
3.7e-09	0.576	0	0	\\
3.75e-09	0.576	0	0	\\
3.8e-09	0.576	0	0	\\
3.85e-09	0.576	0	0	\\
3.9e-09	0.576	0	0	\\
3.95e-09	0.576	0	0	\\
4e-09	0.576	0	0	\\
4.05e-09	0.576	0	0	\\
4.1e-09	0.576	0	0	\\
4.15e-09	0.576	0	0	\\
4.2e-09	0.576	0	0	\\
4.25e-09	0.576	-0.00444444444444444	-0.00444444444444444	\\
4.3e-09	0.576	-0.00444444444444444	-0.00444444444444444	\\
4.35e-09	0.576	-0.00444444444444444	-0.00444444444444444	\\
4.4e-09	0.576	-0.00444444444444444	-0.00444444444444444	\\
4.45e-09	0.576	-0.00444444444444444	-0.00444444444444444	\\
4.5e-09	0.576	-0.00444444444444444	-0.00444444444444444	\\
4.55e-09	0.576	-0.00444444444444444	-0.00444444444444444	\\
4.6e-09	0.576	-0.00444444444444444	-0.00444444444444444	\\
4.65e-09	0.576	-0.00444444444444444	-0.00444444444444444	\\
4.7e-09	0.576	-0.00444444444444444	-0.00444444444444444	\\
4.75e-09	0.576	0	0	\\
4.8e-09	0.576	0	0	\\
4.85e-09	0.576	0	0	\\
4.9e-09	0.576	0	0	\\
4.95e-09	0.576	0	0	\\
5e-09	0.576	0	0	\\
5e-09	0.576	0	nan	\\
5e-09	0.576	-0.00666666666666667	0.00166666666666667	\\
5e-09	0.576	-0.00666666666666667	nan	\\
0	0.588	-0.00666666666666667	nan	\\
0	0.588	0	0.00166666666666667	\\
0	0.588	0	0	\\
5e-11	0.588	0	0	\\
1e-10	0.588	0	0	\\
1.5e-10	0.588	0	0	\\
2e-10	0.588	0	0	\\
2.5e-10	0.588	0	0	\\
3e-10	0.588	0	0	\\
3.5e-10	0.588	0	0	\\
4e-10	0.588	0	0	\\
4.5e-10	0.588	0	0	\\
5e-10	0.588	0	0	\\
5.5e-10	0.588	0	0	\\
6e-10	0.588	0.01	0.01	\\
6.5e-10	0.588	0.01	0.01	\\
7e-10	0.588	0.01	0.01	\\
7.5e-10	0.588	0.01	0.01	\\
8e-10	0.588	0.01	0.01	\\
8.5e-10	0.588	0.01	0.01	\\
9e-10	0.588	0.01	0.01	\\
9.5e-10	0.588	0.01	0.01	\\
1e-09	0.588	0.01	0.01	\\
1.05e-09	0.588	0.01	0.01	\\
1.1e-09	0.588	0	0	\\
1.15e-09	0.588	0	0	\\
1.2e-09	0.588	0	0	\\
1.25e-09	0.588	0	0	\\
1.3e-09	0.588	0	0	\\
1.35e-09	0.588	0	0	\\
1.4e-09	0.588	0	0	\\
1.45e-09	0.588	0	0	\\
1.5e-09	0.588	0	0	\\
1.55e-09	0.588	0	0	\\
1.6e-09	0.588	0	0	\\
1.65e-09	0.588	0	0	\\
1.7e-09	0.588	0	0	\\
1.75e-09	0.588	0	0	\\
1.8e-09	0.588	0	0	\\
1.85e-09	0.588	-0.00666666666666667	-0.00666666666666667	\\
1.9e-09	0.588	-0.00666666666666667	-0.00666666666666667	\\
1.95e-09	0.588	-0.00666666666666667	-0.00666666666666667	\\
2e-09	0.588	-0.00666666666666667	-0.00666666666666667	\\
2.05e-09	0.588	-0.00666666666666667	-0.00666666666666667	\\
2.1e-09	0.588	-0.00666666666666667	-0.00666666666666667	\\
2.15e-09	0.588	-0.00666666666666667	-0.00666666666666667	\\
2.2e-09	0.588	-0.00666666666666667	-0.00666666666666667	\\
2.25e-09	0.588	-0.00666666666666667	-0.00666666666666667	\\
2.3e-09	0.588	-0.00666666666666667	-0.00666666666666667	\\
2.35e-09	0.588	0	0	\\
2.4e-09	0.588	0	0	\\
2.45e-09	0.588	0	0	\\
2.5e-09	0.588	0	0	\\
2.55e-09	0.588	0	0	\\
2.6e-09	0.588	0	0	\\
2.65e-09	0.588	0	0	\\
2.7e-09	0.588	0	0	\\
2.75e-09	0.588	0	0	\\
2.8e-09	0.588	0	0	\\
2.85e-09	0.588	0	0	\\
2.9e-09	0.588	0	0	\\
2.95e-09	0.588	0	0	\\
3e-09	0.588	0.00666666666666667	0.00666666666666667	\\
3.05e-09	0.588	0.00666666666666667	0.00666666666666667	\\
3.1e-09	0.588	0.00666666666666667	0.00666666666666667	\\
3.15e-09	0.588	0.00666666666666667	0.00666666666666667	\\
3.2e-09	0.588	0.00666666666666667	0.00666666666666667	\\
3.25e-09	0.588	0.00666666666666667	0.00666666666666667	\\
3.3e-09	0.588	0.00666666666666667	0.00666666666666667	\\
3.35e-09	0.588	0.00666666666666667	0.00666666666666667	\\
3.4e-09	0.588	0.00666666666666667	0.00666666666666667	\\
3.45e-09	0.588	0.00666666666666667	0.00666666666666667	\\
3.5e-09	0.588	0	0	\\
3.55e-09	0.588	0	0	\\
3.6e-09	0.588	0	0	\\
3.65e-09	0.588	0	0	\\
3.7e-09	0.588	0	0	\\
3.75e-09	0.588	0	0	\\
3.8e-09	0.588	0	0	\\
3.85e-09	0.588	0	0	\\
3.9e-09	0.588	0	0	\\
3.95e-09	0.588	0	0	\\
4e-09	0.588	0	0	\\
4.05e-09	0.588	0	0	\\
4.1e-09	0.588	0	0	\\
4.15e-09	0.588	0	0	\\
4.2e-09	0.588	0	0	\\
4.25e-09	0.588	-0.00444444444444444	-0.00444444444444444	\\
4.3e-09	0.588	-0.00444444444444444	-0.00444444444444444	\\
4.35e-09	0.588	-0.00444444444444444	-0.00444444444444444	\\
4.4e-09	0.588	-0.00444444444444444	-0.00444444444444444	\\
4.45e-09	0.588	-0.00444444444444444	-0.00444444444444444	\\
4.5e-09	0.588	-0.00444444444444444	-0.00444444444444444	\\
4.55e-09	0.588	-0.00444444444444444	-0.00444444444444444	\\
4.6e-09	0.588	-0.00444444444444444	-0.00444444444444444	\\
4.65e-09	0.588	-0.00444444444444444	-0.00444444444444444	\\
4.7e-09	0.588	-0.00444444444444444	-0.00444444444444444	\\
4.75e-09	0.588	0	0	\\
4.8e-09	0.588	0	0	\\
4.85e-09	0.588	0	0	\\
4.9e-09	0.588	0	0	\\
4.95e-09	0.588	0	0	\\
5e-09	0.588	0	0	\\
5e-09	0.588	0	nan	\\
5e-09	0.588	-0.00666666666666667	0.00166666666666667	\\
5e-09	0.588	-0.00666666666666667	nan	\\
0	0.6	-0.00666666666666667	nan	\\
0	0.6	0	0.00166666666666667	\\
0	0.6	0	0	\\
5e-11	0.6	0	0	\\
1e-10	0.6	0	0	\\
1.5e-10	0.6	0	0	\\
2e-10	0.6	0	0	\\
2.5e-10	0.6	0	0	\\
3e-10	0.6	0	0	\\
3.5e-10	0.6	0	0	\\
4e-10	0.6	0	0	\\
4.5e-10	0.6	0	0	\\
5e-10	0.6	0	0	\\
5.5e-10	0.6	0	0	\\
6e-10	0.6	0.01	0.01	\\
6.5e-10	0.6	0.01	0.01	\\
7e-10	0.6	0.01	0.01	\\
7.5e-10	0.6	0.01	0.01	\\
8e-10	0.6	0.01	0.01	\\
8.5e-10	0.6	0.01	0.01	\\
9e-10	0.6	0.01	0.01	\\
9.5e-10	0.6	0.01	0.01	\\
1e-09	0.6	0.01	0.01	\\
1.05e-09	0.6	0.01	0.01	\\
1.1e-09	0.6	0	0	\\
1.15e-09	0.6	0	0	\\
1.2e-09	0.6	0	0	\\
1.25e-09	0.6	0	0	\\
1.3e-09	0.6	0	0	\\
1.35e-09	0.6	0	0	\\
1.4e-09	0.6	0	0	\\
1.45e-09	0.6	0	0	\\
1.5e-09	0.6	0	0	\\
1.55e-09	0.6	0	0	\\
1.6e-09	0.6	0	0	\\
1.65e-09	0.6	0	0	\\
1.7e-09	0.6	0	0	\\
1.75e-09	0.6	0	0	\\
1.8e-09	0.6	-0.00666666666666667	-0.00666666666666667	\\
1.85e-09	0.6	-0.00666666666666667	-0.00666666666666667	\\
1.9e-09	0.6	-0.00666666666666667	-0.00666666666666667	\\
1.95e-09	0.6	-0.00666666666666667	-0.00666666666666667	\\
2e-09	0.6	-0.00666666666666667	-0.00666666666666667	\\
2.05e-09	0.6	-0.00666666666666667	-0.00666666666666667	\\
2.1e-09	0.6	-0.00666666666666667	-0.00666666666666667	\\
2.15e-09	0.6	-0.00666666666666667	-0.00666666666666667	\\
2.2e-09	0.6	-0.00666666666666667	-0.00666666666666667	\\
2.25e-09	0.6	-0.00666666666666667	-0.00666666666666667	\\
2.3e-09	0.6	0	0	\\
2.35e-09	0.6	0	0	\\
2.4e-09	0.6	0	0	\\
2.45e-09	0.6	0	0	\\
2.5e-09	0.6	0	0	\\
2.55e-09	0.6	0	0	\\
2.6e-09	0.6	0	0	\\
2.65e-09	0.6	0	0	\\
2.7e-09	0.6	0	0	\\
2.75e-09	0.6	0	0	\\
2.8e-09	0.6	0	0	\\
2.85e-09	0.6	0	0	\\
2.9e-09	0.6	0	0	\\
2.95e-09	0.6	0	0	\\
3e-09	0.6	0.00666666666666667	0.00666666666666667	\\
3.05e-09	0.6	0.00666666666666667	0.00666666666666667	\\
3.1e-09	0.6	0.00666666666666667	0.00666666666666667	\\
3.15e-09	0.6	0.00666666666666667	0.00666666666666667	\\
3.2e-09	0.6	0.00666666666666667	0.00666666666666667	\\
3.25e-09	0.6	0.00666666666666667	0.00666666666666667	\\
3.3e-09	0.6	0.00666666666666667	0.00666666666666667	\\
3.35e-09	0.6	0.00666666666666667	0.00666666666666667	\\
3.4e-09	0.6	0.00666666666666667	0.00666666666666667	\\
3.45e-09	0.6	0.00666666666666667	0.00666666666666667	\\
3.5e-09	0.6	0	0	\\
3.55e-09	0.6	0	0	\\
3.6e-09	0.6	0	0	\\
3.65e-09	0.6	0	0	\\
3.7e-09	0.6	0	0	\\
3.75e-09	0.6	0	0	\\
3.8e-09	0.6	0	0	\\
3.85e-09	0.6	0	0	\\
3.9e-09	0.6	0	0	\\
3.95e-09	0.6	0	0	\\
4e-09	0.6	0	0	\\
4.05e-09	0.6	0	0	\\
4.1e-09	0.6	0	0	\\
4.15e-09	0.6	0	0	\\
4.2e-09	0.6	-0.00444444444444444	-0.00444444444444444	\\
4.25e-09	0.6	-0.00444444444444444	-0.00444444444444444	\\
4.3e-09	0.6	-0.00444444444444444	-0.00444444444444444	\\
4.35e-09	0.6	-0.00444444444444444	-0.00444444444444444	\\
4.4e-09	0.6	-0.00444444444444444	-0.00444444444444444	\\
4.45e-09	0.6	-0.00444444444444444	-0.00444444444444444	\\
4.5e-09	0.6	-0.00444444444444444	-0.00444444444444444	\\
4.55e-09	0.6	-0.00444444444444444	-0.00444444444444444	\\
4.6e-09	0.6	-0.00444444444444444	-0.00444444444444444	\\
4.65e-09	0.6	-0.00444444444444444	-0.00444444444444444	\\
4.7e-09	0.6	0	0	\\
4.75e-09	0.6	0	0	\\
4.8e-09	0.6	0	0	\\
4.85e-09	0.6	0	0	\\
4.9e-09	0.6	0	0	\\
4.95e-09	0.6	0	0	\\
5e-09	0.6	0	0	\\
5e-09	0.6	0	nan	\\
5e-09	0.6	-0.00666666666666667	0.00166666666666667	\\
5e-09	0.6	-0.00666666666666667	nan	\\
0	0.612	-0.00666666666666667	nan	\\
0	0.612	0	0.00166666666666667	\\
0	0.612	0	0	\\
5e-11	0.612	0	0	\\
1e-10	0.612	0	0	\\
1.5e-10	0.612	0	0	\\
2e-10	0.612	0	0	\\
2.5e-10	0.612	0	0	\\
3e-10	0.612	0	0	\\
3.5e-10	0.612	0	0	\\
4e-10	0.612	0	0	\\
4.5e-10	0.612	0	0	\\
5e-10	0.612	0	0	\\
5.5e-10	0.612	0	0	\\
6e-10	0.612	0	0	\\
6.5e-10	0.612	0.01	0.01	\\
7e-10	0.612	0.01	0.01	\\
7.5e-10	0.612	0.01	0.01	\\
8e-10	0.612	0.01	0.01	\\
8.5e-10	0.612	0.01	0.01	\\
9e-10	0.612	0.01	0.01	\\
9.5e-10	0.612	0.01	0.01	\\
1e-09	0.612	0.01	0.01	\\
1.05e-09	0.612	0.01	0.01	\\
1.1e-09	0.612	0.01	0.01	\\
1.15e-09	0.612	0	0	\\
1.2e-09	0.612	0	0	\\
1.25e-09	0.612	0	0	\\
1.3e-09	0.612	0	0	\\
1.35e-09	0.612	0	0	\\
1.4e-09	0.612	0	0	\\
1.45e-09	0.612	0	0	\\
1.5e-09	0.612	0	0	\\
1.55e-09	0.612	0	0	\\
1.6e-09	0.612	0	0	\\
1.65e-09	0.612	0	0	\\
1.7e-09	0.612	0	0	\\
1.75e-09	0.612	0	0	\\
1.8e-09	0.612	-0.00666666666666667	-0.00666666666666667	\\
1.85e-09	0.612	-0.00666666666666667	-0.00666666666666667	\\
1.9e-09	0.612	-0.00666666666666667	-0.00666666666666667	\\
1.95e-09	0.612	-0.00666666666666667	-0.00666666666666667	\\
2e-09	0.612	-0.00666666666666667	-0.00666666666666667	\\
2.05e-09	0.612	-0.00666666666666667	-0.00666666666666667	\\
2.1e-09	0.612	-0.00666666666666667	-0.00666666666666667	\\
2.15e-09	0.612	-0.00666666666666667	-0.00666666666666667	\\
2.2e-09	0.612	-0.00666666666666667	-0.00666666666666667	\\
2.25e-09	0.612	-0.00666666666666667	-0.00666666666666667	\\
2.3e-09	0.612	0	0	\\
2.35e-09	0.612	0	0	\\
2.4e-09	0.612	0	0	\\
2.45e-09	0.612	0	0	\\
2.5e-09	0.612	0	0	\\
2.55e-09	0.612	0	0	\\
2.6e-09	0.612	0	0	\\
2.65e-09	0.612	0	0	\\
2.7e-09	0.612	0	0	\\
2.75e-09	0.612	0	0	\\
2.8e-09	0.612	0	0	\\
2.85e-09	0.612	0	0	\\
2.9e-09	0.612	0	0	\\
2.95e-09	0.612	0	0	\\
3e-09	0.612	0	0	\\
3.05e-09	0.612	0.00666666666666667	0.00666666666666667	\\
3.1e-09	0.612	0.00666666666666667	0.00666666666666667	\\
3.15e-09	0.612	0.00666666666666667	0.00666666666666667	\\
3.2e-09	0.612	0.00666666666666667	0.00666666666666667	\\
3.25e-09	0.612	0.00666666666666667	0.00666666666666667	\\
3.3e-09	0.612	0.00666666666666667	0.00666666666666667	\\
3.35e-09	0.612	0.00666666666666667	0.00666666666666667	\\
3.4e-09	0.612	0.00666666666666667	0.00666666666666667	\\
3.45e-09	0.612	0.00666666666666667	0.00666666666666667	\\
3.5e-09	0.612	0.00666666666666667	0.00666666666666667	\\
3.55e-09	0.612	0	0	\\
3.6e-09	0.612	0	0	\\
3.65e-09	0.612	0	0	\\
3.7e-09	0.612	0	0	\\
3.75e-09	0.612	0	0	\\
3.8e-09	0.612	0	0	\\
3.85e-09	0.612	0	0	\\
3.9e-09	0.612	0	0	\\
3.95e-09	0.612	0	0	\\
4e-09	0.612	0	0	\\
4.05e-09	0.612	0	0	\\
4.1e-09	0.612	0	0	\\
4.15e-09	0.612	0	0	\\
4.2e-09	0.612	-0.00444444444444444	-0.00444444444444444	\\
4.25e-09	0.612	-0.00444444444444444	-0.00444444444444444	\\
4.3e-09	0.612	-0.00444444444444444	-0.00444444444444444	\\
4.35e-09	0.612	-0.00444444444444444	-0.00444444444444444	\\
4.4e-09	0.612	-0.00444444444444444	-0.00444444444444444	\\
4.45e-09	0.612	-0.00444444444444444	-0.00444444444444444	\\
4.5e-09	0.612	-0.00444444444444444	-0.00444444444444444	\\
4.55e-09	0.612	-0.00444444444444444	-0.00444444444444444	\\
4.6e-09	0.612	-0.00444444444444444	-0.00444444444444444	\\
4.65e-09	0.612	-0.00444444444444444	-0.00444444444444444	\\
4.7e-09	0.612	0	0	\\
4.75e-09	0.612	0	0	\\
4.8e-09	0.612	0	0	\\
4.85e-09	0.612	0	0	\\
4.9e-09	0.612	0	0	\\
4.95e-09	0.612	0	0	\\
5e-09	0.612	0	0	\\
5e-09	0.612	0	nan	\\
5e-09	0.612	-0.00666666666666667	0.00166666666666667	\\
5e-09	0.612	-0.00666666666666667	nan	\\
0	0.624	-0.00666666666666667	nan	\\
0	0.624	0	0.00166666666666667	\\
0	0.624	0	0	\\
5e-11	0.624	0	0	\\
1e-10	0.624	0	0	\\
1.5e-10	0.624	0	0	\\
2e-10	0.624	0	0	\\
2.5e-10	0.624	0	0	\\
3e-10	0.624	0	0	\\
3.5e-10	0.624	0	0	\\
4e-10	0.624	0	0	\\
4.5e-10	0.624	0	0	\\
5e-10	0.624	0	0	\\
5.5e-10	0.624	0	0	\\
6e-10	0.624	0	0	\\
6.5e-10	0.624	0.01	0.01	\\
7e-10	0.624	0.01	0.01	\\
7.5e-10	0.624	0.01	0.01	\\
8e-10	0.624	0.01	0.01	\\
8.5e-10	0.624	0.01	0.01	\\
9e-10	0.624	0.01	0.01	\\
9.5e-10	0.624	0.01	0.01	\\
1e-09	0.624	0.01	0.01	\\
1.05e-09	0.624	0.01	0.01	\\
1.1e-09	0.624	0.01	0.01	\\
1.15e-09	0.624	0	0	\\
1.2e-09	0.624	0	0	\\
1.25e-09	0.624	0	0	\\
1.3e-09	0.624	0	0	\\
1.35e-09	0.624	0	0	\\
1.4e-09	0.624	0	0	\\
1.45e-09	0.624	0	0	\\
1.5e-09	0.624	0	0	\\
1.55e-09	0.624	0	0	\\
1.6e-09	0.624	0	0	\\
1.65e-09	0.624	0	0	\\
1.7e-09	0.624	0	0	\\
1.75e-09	0.624	0	0	\\
1.8e-09	0.624	-0.00666666666666667	-0.00666666666666667	\\
1.85e-09	0.624	-0.00666666666666667	-0.00666666666666667	\\
1.9e-09	0.624	-0.00666666666666667	-0.00666666666666667	\\
1.95e-09	0.624	-0.00666666666666667	-0.00666666666666667	\\
2e-09	0.624	-0.00666666666666667	-0.00666666666666667	\\
2.05e-09	0.624	-0.00666666666666667	-0.00666666666666667	\\
2.1e-09	0.624	-0.00666666666666667	-0.00666666666666667	\\
2.15e-09	0.624	-0.00666666666666667	-0.00666666666666667	\\
2.2e-09	0.624	-0.00666666666666667	-0.00666666666666667	\\
2.25e-09	0.624	-0.00666666666666667	-0.00666666666666667	\\
2.3e-09	0.624	0	0	\\
2.35e-09	0.624	0	0	\\
2.4e-09	0.624	0	0	\\
2.45e-09	0.624	0	0	\\
2.5e-09	0.624	0	0	\\
2.55e-09	0.624	0	0	\\
2.6e-09	0.624	0	0	\\
2.65e-09	0.624	0	0	\\
2.7e-09	0.624	0	0	\\
2.75e-09	0.624	0	0	\\
2.8e-09	0.624	0	0	\\
2.85e-09	0.624	0	0	\\
2.9e-09	0.624	0	0	\\
2.95e-09	0.624	0	0	\\
3e-09	0.624	0	0	\\
3.05e-09	0.624	0.00666666666666667	0.00666666666666667	\\
3.1e-09	0.624	0.00666666666666667	0.00666666666666667	\\
3.15e-09	0.624	0.00666666666666667	0.00666666666666667	\\
3.2e-09	0.624	0.00666666666666667	0.00666666666666667	\\
3.25e-09	0.624	0.00666666666666667	0.00666666666666667	\\
3.3e-09	0.624	0.00666666666666667	0.00666666666666667	\\
3.35e-09	0.624	0.00666666666666667	0.00666666666666667	\\
3.4e-09	0.624	0.00666666666666667	0.00666666666666667	\\
3.45e-09	0.624	0.00666666666666667	0.00666666666666667	\\
3.5e-09	0.624	0.00666666666666667	0.00666666666666667	\\
3.55e-09	0.624	0	0	\\
3.6e-09	0.624	0	0	\\
3.65e-09	0.624	0	0	\\
3.7e-09	0.624	0	0	\\
3.75e-09	0.624	0	0	\\
3.8e-09	0.624	0	0	\\
3.85e-09	0.624	0	0	\\
3.9e-09	0.624	0	0	\\
3.95e-09	0.624	0	0	\\
4e-09	0.624	0	0	\\
4.05e-09	0.624	0	0	\\
4.1e-09	0.624	0	0	\\
4.15e-09	0.624	0	0	\\
4.2e-09	0.624	-0.00444444444444444	-0.00444444444444444	\\
4.25e-09	0.624	-0.00444444444444444	-0.00444444444444444	\\
4.3e-09	0.624	-0.00444444444444444	-0.00444444444444444	\\
4.35e-09	0.624	-0.00444444444444444	-0.00444444444444444	\\
4.4e-09	0.624	-0.00444444444444444	-0.00444444444444444	\\
4.45e-09	0.624	-0.00444444444444444	-0.00444444444444444	\\
4.5e-09	0.624	-0.00444444444444444	-0.00444444444444444	\\
4.55e-09	0.624	-0.00444444444444444	-0.00444444444444444	\\
4.6e-09	0.624	-0.00444444444444444	-0.00444444444444444	\\
4.65e-09	0.624	-0.00444444444444444	-0.00444444444444444	\\
4.7e-09	0.624	0	0	\\
4.75e-09	0.624	0	0	\\
4.8e-09	0.624	0	0	\\
4.85e-09	0.624	0	0	\\
4.9e-09	0.624	0	0	\\
4.95e-09	0.624	0	0	\\
5e-09	0.624	0	0	\\
5e-09	0.624	0	nan	\\
5e-09	0.624	-0.00666666666666667	0.00166666666666667	\\
5e-09	0.624	-0.00666666666666667	nan	\\
0	0.636	-0.00666666666666667	nan	\\
0	0.636	0	0.00166666666666667	\\
0	0.636	0	0	\\
5e-11	0.636	0	0	\\
1e-10	0.636	0	0	\\
1.5e-10	0.636	0	0	\\
2e-10	0.636	0	0	\\
2.5e-10	0.636	0	0	\\
3e-10	0.636	0	0	\\
3.5e-10	0.636	0	0	\\
4e-10	0.636	0	0	\\
4.5e-10	0.636	0	0	\\
5e-10	0.636	0	0	\\
5.5e-10	0.636	0	0	\\
6e-10	0.636	0	0	\\
6.5e-10	0.636	0.01	0.01	\\
7e-10	0.636	0.01	0.01	\\
7.5e-10	0.636	0.01	0.01	\\
8e-10	0.636	0.01	0.01	\\
8.5e-10	0.636	0.01	0.01	\\
9e-10	0.636	0.01	0.01	\\
9.5e-10	0.636	0.01	0.01	\\
1e-09	0.636	0.01	0.01	\\
1.05e-09	0.636	0.01	0.01	\\
1.1e-09	0.636	0.01	0.01	\\
1.15e-09	0.636	0	0	\\
1.2e-09	0.636	0	0	\\
1.25e-09	0.636	0	0	\\
1.3e-09	0.636	0	0	\\
1.35e-09	0.636	0	0	\\
1.4e-09	0.636	0	0	\\
1.45e-09	0.636	0	0	\\
1.5e-09	0.636	0	0	\\
1.55e-09	0.636	0	0	\\
1.6e-09	0.636	0	0	\\
1.65e-09	0.636	0	0	\\
1.7e-09	0.636	0	0	\\
1.75e-09	0.636	0	0	\\
1.8e-09	0.636	-0.00666666666666667	-0.00666666666666667	\\
1.85e-09	0.636	-0.00666666666666667	-0.00666666666666667	\\
1.9e-09	0.636	-0.00666666666666667	-0.00666666666666667	\\
1.95e-09	0.636	-0.00666666666666667	-0.00666666666666667	\\
2e-09	0.636	-0.00666666666666667	-0.00666666666666667	\\
2.05e-09	0.636	-0.00666666666666667	-0.00666666666666667	\\
2.1e-09	0.636	-0.00666666666666667	-0.00666666666666667	\\
2.15e-09	0.636	-0.00666666666666667	-0.00666666666666667	\\
2.2e-09	0.636	-0.00666666666666667	-0.00666666666666667	\\
2.25e-09	0.636	-0.00666666666666667	-0.00666666666666667	\\
2.3e-09	0.636	0	0	\\
2.35e-09	0.636	0	0	\\
2.4e-09	0.636	0	0	\\
2.45e-09	0.636	0	0	\\
2.5e-09	0.636	0	0	\\
2.55e-09	0.636	0	0	\\
2.6e-09	0.636	0	0	\\
2.65e-09	0.636	0	0	\\
2.7e-09	0.636	0	0	\\
2.75e-09	0.636	0	0	\\
2.8e-09	0.636	0	0	\\
2.85e-09	0.636	0	0	\\
2.9e-09	0.636	0	0	\\
2.95e-09	0.636	0	0	\\
3e-09	0.636	0	0	\\
3.05e-09	0.636	0.00666666666666667	0.00666666666666667	\\
3.1e-09	0.636	0.00666666666666667	0.00666666666666667	\\
3.15e-09	0.636	0.00666666666666667	0.00666666666666667	\\
3.2e-09	0.636	0.00666666666666667	0.00666666666666667	\\
3.25e-09	0.636	0.00666666666666667	0.00666666666666667	\\
3.3e-09	0.636	0.00666666666666667	0.00666666666666667	\\
3.35e-09	0.636	0.00666666666666667	0.00666666666666667	\\
3.4e-09	0.636	0.00666666666666667	0.00666666666666667	\\
3.45e-09	0.636	0.00666666666666667	0.00666666666666667	\\
3.5e-09	0.636	0.00666666666666667	0.00666666666666667	\\
3.55e-09	0.636	0	0	\\
3.6e-09	0.636	0	0	\\
3.65e-09	0.636	0	0	\\
3.7e-09	0.636	0	0	\\
3.75e-09	0.636	0	0	\\
3.8e-09	0.636	0	0	\\
3.85e-09	0.636	0	0	\\
3.9e-09	0.636	0	0	\\
3.95e-09	0.636	0	0	\\
4e-09	0.636	0	0	\\
4.05e-09	0.636	0	0	\\
4.1e-09	0.636	0	0	\\
4.15e-09	0.636	0	0	\\
4.2e-09	0.636	-0.00444444444444444	-0.00444444444444444	\\
4.25e-09	0.636	-0.00444444444444444	-0.00444444444444444	\\
4.3e-09	0.636	-0.00444444444444444	-0.00444444444444444	\\
4.35e-09	0.636	-0.00444444444444444	-0.00444444444444444	\\
4.4e-09	0.636	-0.00444444444444444	-0.00444444444444444	\\
4.45e-09	0.636	-0.00444444444444444	-0.00444444444444444	\\
4.5e-09	0.636	-0.00444444444444444	-0.00444444444444444	\\
4.55e-09	0.636	-0.00444444444444444	-0.00444444444444444	\\
4.6e-09	0.636	-0.00444444444444444	-0.00444444444444444	\\
4.65e-09	0.636	-0.00444444444444444	-0.00444444444444444	\\
4.7e-09	0.636	0	0	\\
4.75e-09	0.636	0	0	\\
4.8e-09	0.636	0	0	\\
4.85e-09	0.636	0	0	\\
4.9e-09	0.636	0	0	\\
4.95e-09	0.636	0	0	\\
5e-09	0.636	0	0	\\
5e-09	0.636	0	nan	\\
5e-09	0.636	-0.00666666666666667	0.00166666666666667	\\
5e-09	0.636	-0.00666666666666667	nan	\\
0	0.648	-0.00666666666666667	nan	\\
0	0.648	0	0.00166666666666667	\\
0	0.648	0	0	\\
5e-11	0.648	0	0	\\
1e-10	0.648	0	0	\\
1.5e-10	0.648	0	0	\\
2e-10	0.648	0	0	\\
2.5e-10	0.648	0	0	\\
3e-10	0.648	0	0	\\
3.5e-10	0.648	0	0	\\
4e-10	0.648	0	0	\\
4.5e-10	0.648	0	0	\\
5e-10	0.648	0	0	\\
5.5e-10	0.648	0	0	\\
6e-10	0.648	0	0	\\
6.5e-10	0.648	0.01	0.01	\\
7e-10	0.648	0.01	0.01	\\
7.5e-10	0.648	0.01	0.01	\\
8e-10	0.648	0.01	0.01	\\
8.5e-10	0.648	0.01	0.01	\\
9e-10	0.648	0.01	0.01	\\
9.5e-10	0.648	0.01	0.01	\\
1e-09	0.648	0.01	0.01	\\
1.05e-09	0.648	0.01	0.01	\\
1.1e-09	0.648	0.01	0.01	\\
1.15e-09	0.648	0	0	\\
1.2e-09	0.648	0	0	\\
1.25e-09	0.648	0	0	\\
1.3e-09	0.648	0	0	\\
1.35e-09	0.648	0	0	\\
1.4e-09	0.648	0	0	\\
1.45e-09	0.648	0	0	\\
1.5e-09	0.648	0	0	\\
1.55e-09	0.648	0	0	\\
1.6e-09	0.648	0	0	\\
1.65e-09	0.648	0	0	\\
1.7e-09	0.648	0	0	\\
1.75e-09	0.648	0	0	\\
1.8e-09	0.648	-0.00666666666666667	-0.00666666666666667	\\
1.85e-09	0.648	-0.00666666666666667	-0.00666666666666667	\\
1.9e-09	0.648	-0.00666666666666667	-0.00666666666666667	\\
1.95e-09	0.648	-0.00666666666666667	-0.00666666666666667	\\
2e-09	0.648	-0.00666666666666667	-0.00666666666666667	\\
2.05e-09	0.648	-0.00666666666666667	-0.00666666666666667	\\
2.1e-09	0.648	-0.00666666666666667	-0.00666666666666667	\\
2.15e-09	0.648	-0.00666666666666667	-0.00666666666666667	\\
2.2e-09	0.648	-0.00666666666666667	-0.00666666666666667	\\
2.25e-09	0.648	-0.00666666666666667	-0.00666666666666667	\\
2.3e-09	0.648	0	0	\\
2.35e-09	0.648	0	0	\\
2.4e-09	0.648	0	0	\\
2.45e-09	0.648	0	0	\\
2.5e-09	0.648	0	0	\\
2.55e-09	0.648	0	0	\\
2.6e-09	0.648	0	0	\\
2.65e-09	0.648	0	0	\\
2.7e-09	0.648	0	0	\\
2.75e-09	0.648	0	0	\\
2.8e-09	0.648	0	0	\\
2.85e-09	0.648	0	0	\\
2.9e-09	0.648	0	0	\\
2.95e-09	0.648	0	0	\\
3e-09	0.648	0	0	\\
3.05e-09	0.648	0.00666666666666667	0.00666666666666667	\\
3.1e-09	0.648	0.00666666666666667	0.00666666666666667	\\
3.15e-09	0.648	0.00666666666666667	0.00666666666666667	\\
3.2e-09	0.648	0.00666666666666667	0.00666666666666667	\\
3.25e-09	0.648	0.00666666666666667	0.00666666666666667	\\
3.3e-09	0.648	0.00666666666666667	0.00666666666666667	\\
3.35e-09	0.648	0.00666666666666667	0.00666666666666667	\\
3.4e-09	0.648	0.00666666666666667	0.00666666666666667	\\
3.45e-09	0.648	0.00666666666666667	0.00666666666666667	\\
3.5e-09	0.648	0.00666666666666667	0.00666666666666667	\\
3.55e-09	0.648	0	0	\\
3.6e-09	0.648	0	0	\\
3.65e-09	0.648	0	0	\\
3.7e-09	0.648	0	0	\\
3.75e-09	0.648	0	0	\\
3.8e-09	0.648	0	0	\\
3.85e-09	0.648	0	0	\\
3.9e-09	0.648	0	0	\\
3.95e-09	0.648	0	0	\\
4e-09	0.648	0	0	\\
4.05e-09	0.648	0	0	\\
4.1e-09	0.648	0	0	\\
4.15e-09	0.648	0	0	\\
4.2e-09	0.648	-0.00444444444444444	-0.00444444444444444	\\
4.25e-09	0.648	-0.00444444444444444	-0.00444444444444444	\\
4.3e-09	0.648	-0.00444444444444444	-0.00444444444444444	\\
4.35e-09	0.648	-0.00444444444444444	-0.00444444444444444	\\
4.4e-09	0.648	-0.00444444444444444	-0.00444444444444444	\\
4.45e-09	0.648	-0.00444444444444444	-0.00444444444444444	\\
4.5e-09	0.648	-0.00444444444444444	-0.00444444444444444	\\
4.55e-09	0.648	-0.00444444444444444	-0.00444444444444444	\\
4.6e-09	0.648	-0.00444444444444444	-0.00444444444444444	\\
4.65e-09	0.648	-0.00444444444444444	-0.00444444444444444	\\
4.7e-09	0.648	0	0	\\
4.75e-09	0.648	0	0	\\
4.8e-09	0.648	0	0	\\
4.85e-09	0.648	0	0	\\
4.9e-09	0.648	0	0	\\
4.95e-09	0.648	0	0	\\
5e-09	0.648	0	0	\\
5e-09	0.648	0	nan	\\
5e-09	0.648	-0.00666666666666667	0.00166666666666667	\\
5e-09	0.648	-0.00666666666666667	nan	\\
0	0.66	-0.00666666666666667	nan	\\
0	0.66	0	0.00166666666666667	\\
0	0.66	0	0	\\
5e-11	0.66	0	0	\\
1e-10	0.66	0	0	\\
1.5e-10	0.66	0	0	\\
2e-10	0.66	0	0	\\
2.5e-10	0.66	0	0	\\
3e-10	0.66	0	0	\\
3.5e-10	0.66	0	0	\\
4e-10	0.66	0	0	\\
4.5e-10	0.66	0	0	\\
5e-10	0.66	0	0	\\
5.5e-10	0.66	0	0	\\
6e-10	0.66	0	0	\\
6.5e-10	0.66	0	0	\\
7e-10	0.66	0.01	0.01	\\
7.5e-10	0.66	0.01	0.01	\\
8e-10	0.66	0.01	0.01	\\
8.5e-10	0.66	0.01	0.01	\\
9e-10	0.66	0.01	0.01	\\
9.5e-10	0.66	0.01	0.01	\\
1e-09	0.66	0.01	0.01	\\
1.05e-09	0.66	0.01	0.01	\\
1.1e-09	0.66	0.01	0.01	\\
1.15e-09	0.66	0.01	0.01	\\
1.2e-09	0.66	0	0	\\
1.25e-09	0.66	0	0	\\
1.3e-09	0.66	0	0	\\
1.35e-09	0.66	0	0	\\
1.4e-09	0.66	0	0	\\
1.45e-09	0.66	0	0	\\
1.5e-09	0.66	0	0	\\
1.55e-09	0.66	0	0	\\
1.6e-09	0.66	0	0	\\
1.65e-09	0.66	0	0	\\
1.7e-09	0.66	0	0	\\
1.75e-09	0.66	-0.00666666666666667	-0.00666666666666667	\\
1.8e-09	0.66	-0.00666666666666667	-0.00666666666666667	\\
1.85e-09	0.66	-0.00666666666666667	-0.00666666666666667	\\
1.9e-09	0.66	-0.00666666666666667	-0.00666666666666667	\\
1.95e-09	0.66	-0.00666666666666667	-0.00666666666666667	\\
2e-09	0.66	-0.00666666666666667	-0.00666666666666667	\\
2.05e-09	0.66	-0.00666666666666667	-0.00666666666666667	\\
2.1e-09	0.66	-0.00666666666666667	-0.00666666666666667	\\
2.15e-09	0.66	-0.00666666666666667	-0.00666666666666667	\\
2.2e-09	0.66	-0.00666666666666667	-0.00666666666666667	\\
2.25e-09	0.66	0	0	\\
2.3e-09	0.66	0	0	\\
2.35e-09	0.66	0	0	\\
2.4e-09	0.66	0	0	\\
2.45e-09	0.66	0	0	\\
2.5e-09	0.66	0	0	\\
2.55e-09	0.66	0	0	\\
2.6e-09	0.66	0	0	\\
2.65e-09	0.66	0	0	\\
2.7e-09	0.66	0	0	\\
2.75e-09	0.66	0	0	\\
2.8e-09	0.66	0	0	\\
2.85e-09	0.66	0	0	\\
2.9e-09	0.66	0	0	\\
2.95e-09	0.66	0	0	\\
3e-09	0.66	0	0	\\
3.05e-09	0.66	0	0	\\
3.1e-09	0.66	0.00666666666666667	0.00666666666666667	\\
3.15e-09	0.66	0.00666666666666667	0.00666666666666667	\\
3.2e-09	0.66	0.00666666666666667	0.00666666666666667	\\
3.25e-09	0.66	0.00666666666666667	0.00666666666666667	\\
3.3e-09	0.66	0.00666666666666667	0.00666666666666667	\\
3.35e-09	0.66	0.00666666666666667	0.00666666666666667	\\
3.4e-09	0.66	0.00666666666666667	0.00666666666666667	\\
3.45e-09	0.66	0.00666666666666667	0.00666666666666667	\\
3.5e-09	0.66	0.00666666666666667	0.00666666666666667	\\
3.55e-09	0.66	0.00666666666666667	0.00666666666666667	\\
3.6e-09	0.66	0	0	\\
3.65e-09	0.66	0	0	\\
3.7e-09	0.66	0	0	\\
3.75e-09	0.66	0	0	\\
3.8e-09	0.66	0	0	\\
3.85e-09	0.66	0	0	\\
3.9e-09	0.66	0	0	\\
3.95e-09	0.66	0	0	\\
4e-09	0.66	0	0	\\
4.05e-09	0.66	0	0	\\
4.1e-09	0.66	0	0	\\
4.15e-09	0.66	-0.00444444444444444	-0.00444444444444444	\\
4.2e-09	0.66	-0.00444444444444444	-0.00444444444444444	\\
4.25e-09	0.66	-0.00444444444444444	-0.00444444444444444	\\
4.3e-09	0.66	-0.00444444444444444	-0.00444444444444444	\\
4.35e-09	0.66	-0.00444444444444444	-0.00444444444444444	\\
4.4e-09	0.66	-0.00444444444444444	-0.00444444444444444	\\
4.45e-09	0.66	-0.00444444444444444	-0.00444444444444444	\\
4.5e-09	0.66	-0.00444444444444444	-0.00444444444444444	\\
4.55e-09	0.66	-0.00444444444444444	-0.00444444444444444	\\
4.6e-09	0.66	-0.00444444444444444	-0.00444444444444444	\\
4.65e-09	0.66	0	0	\\
4.7e-09	0.66	0	0	\\
4.75e-09	0.66	0	0	\\
4.8e-09	0.66	0	0	\\
4.85e-09	0.66	0	0	\\
4.9e-09	0.66	0	0	\\
4.95e-09	0.66	0	0	\\
5e-09	0.66	0	0	\\
5e-09	0.66	0	nan	\\
5e-09	0.66	-0.00666666666666667	0.00166666666666667	\\
5e-09	0.66	-0.00666666666666667	nan	\\
0	0.672	-0.00666666666666667	nan	\\
0	0.672	0	0.00166666666666667	\\
0	0.672	0	0	\\
5e-11	0.672	0	0	\\
1e-10	0.672	0	0	\\
1.5e-10	0.672	0	0	\\
2e-10	0.672	0	0	\\
2.5e-10	0.672	0	0	\\
3e-10	0.672	0	0	\\
3.5e-10	0.672	0	0	\\
4e-10	0.672	0	0	\\
4.5e-10	0.672	0	0	\\
5e-10	0.672	0	0	\\
5.5e-10	0.672	0	0	\\
6e-10	0.672	0	0	\\
6.5e-10	0.672	0	0	\\
7e-10	0.672	0.01	0.01	\\
7.5e-10	0.672	0.01	0.01	\\
8e-10	0.672	0.01	0.01	\\
8.5e-10	0.672	0.01	0.01	\\
9e-10	0.672	0.01	0.01	\\
9.5e-10	0.672	0.01	0.01	\\
1e-09	0.672	0.01	0.01	\\
1.05e-09	0.672	0.01	0.01	\\
1.1e-09	0.672	0.01	0.01	\\
1.15e-09	0.672	0.01	0.01	\\
1.2e-09	0.672	0	0	\\
1.25e-09	0.672	0	0	\\
1.3e-09	0.672	0	0	\\
1.35e-09	0.672	0	0	\\
1.4e-09	0.672	0	0	\\
1.45e-09	0.672	0	0	\\
1.5e-09	0.672	0	0	\\
1.55e-09	0.672	0	0	\\
1.6e-09	0.672	0	0	\\
1.65e-09	0.672	0	0	\\
1.7e-09	0.672	0	0	\\
1.75e-09	0.672	-0.00666666666666667	-0.00666666666666667	\\
1.8e-09	0.672	-0.00666666666666667	-0.00666666666666667	\\
1.85e-09	0.672	-0.00666666666666667	-0.00666666666666667	\\
1.9e-09	0.672	-0.00666666666666667	-0.00666666666666667	\\
1.95e-09	0.672	-0.00666666666666667	-0.00666666666666667	\\
2e-09	0.672	-0.00666666666666667	-0.00666666666666667	\\
2.05e-09	0.672	-0.00666666666666667	-0.00666666666666667	\\
2.1e-09	0.672	-0.00666666666666667	-0.00666666666666667	\\
2.15e-09	0.672	-0.00666666666666667	-0.00666666666666667	\\
2.2e-09	0.672	-0.00666666666666667	-0.00666666666666667	\\
2.25e-09	0.672	0	0	\\
2.3e-09	0.672	0	0	\\
2.35e-09	0.672	0	0	\\
2.4e-09	0.672	0	0	\\
2.45e-09	0.672	0	0	\\
2.5e-09	0.672	0	0	\\
2.55e-09	0.672	0	0	\\
2.6e-09	0.672	0	0	\\
2.65e-09	0.672	0	0	\\
2.7e-09	0.672	0	0	\\
2.75e-09	0.672	0	0	\\
2.8e-09	0.672	0	0	\\
2.85e-09	0.672	0	0	\\
2.9e-09	0.672	0	0	\\
2.95e-09	0.672	0	0	\\
3e-09	0.672	0	0	\\
3.05e-09	0.672	0	0	\\
3.1e-09	0.672	0.00666666666666667	0.00666666666666667	\\
3.15e-09	0.672	0.00666666666666667	0.00666666666666667	\\
3.2e-09	0.672	0.00666666666666667	0.00666666666666667	\\
3.25e-09	0.672	0.00666666666666667	0.00666666666666667	\\
3.3e-09	0.672	0.00666666666666667	0.00666666666666667	\\
3.35e-09	0.672	0.00666666666666667	0.00666666666666667	\\
3.4e-09	0.672	0.00666666666666667	0.00666666666666667	\\
3.45e-09	0.672	0.00666666666666667	0.00666666666666667	\\
3.5e-09	0.672	0.00666666666666667	0.00666666666666667	\\
3.55e-09	0.672	0.00666666666666667	0.00666666666666667	\\
3.6e-09	0.672	0	0	\\
3.65e-09	0.672	0	0	\\
3.7e-09	0.672	0	0	\\
3.75e-09	0.672	0	0	\\
3.8e-09	0.672	0	0	\\
3.85e-09	0.672	0	0	\\
3.9e-09	0.672	0	0	\\
3.95e-09	0.672	0	0	\\
4e-09	0.672	0	0	\\
4.05e-09	0.672	0	0	\\
4.1e-09	0.672	0	0	\\
4.15e-09	0.672	-0.00444444444444444	-0.00444444444444444	\\
4.2e-09	0.672	-0.00444444444444444	-0.00444444444444444	\\
4.25e-09	0.672	-0.00444444444444444	-0.00444444444444444	\\
4.3e-09	0.672	-0.00444444444444444	-0.00444444444444444	\\
4.35e-09	0.672	-0.00444444444444444	-0.00444444444444444	\\
4.4e-09	0.672	-0.00444444444444444	-0.00444444444444444	\\
4.45e-09	0.672	-0.00444444444444444	-0.00444444444444444	\\
4.5e-09	0.672	-0.00444444444444444	-0.00444444444444444	\\
4.55e-09	0.672	-0.00444444444444444	-0.00444444444444444	\\
4.6e-09	0.672	-0.00444444444444444	-0.00444444444444444	\\
4.65e-09	0.672	0	0	\\
4.7e-09	0.672	0	0	\\
4.75e-09	0.672	0	0	\\
4.8e-09	0.672	0	0	\\
4.85e-09	0.672	0	0	\\
4.9e-09	0.672	0	0	\\
4.95e-09	0.672	0	0	\\
5e-09	0.672	0	0	\\
5e-09	0.672	0	nan	\\
5e-09	0.672	-0.00666666666666667	0.00166666666666667	\\
5e-09	0.672	-0.00666666666666667	nan	\\
0	0.684	-0.00666666666666667	nan	\\
0	0.684	0	0.00166666666666667	\\
0	0.684	0	0	\\
5e-11	0.684	0	0	\\
1e-10	0.684	0	0	\\
1.5e-10	0.684	0	0	\\
2e-10	0.684	0	0	\\
2.5e-10	0.684	0	0	\\
3e-10	0.684	0	0	\\
3.5e-10	0.684	0	0	\\
4e-10	0.684	0	0	\\
4.5e-10	0.684	0	0	\\
5e-10	0.684	0	0	\\
5.5e-10	0.684	0	0	\\
6e-10	0.684	0	0	\\
6.5e-10	0.684	0	0	\\
7e-10	0.684	0.01	0.01	\\
7.5e-10	0.684	0.01	0.01	\\
8e-10	0.684	0.01	0.01	\\
8.5e-10	0.684	0.01	0.01	\\
9e-10	0.684	0.01	0.01	\\
9.5e-10	0.684	0.01	0.01	\\
1e-09	0.684	0.01	0.01	\\
1.05e-09	0.684	0.01	0.01	\\
1.1e-09	0.684	0.01	0.01	\\
1.15e-09	0.684	0.01	0.01	\\
1.2e-09	0.684	0	0	\\
1.25e-09	0.684	0	0	\\
1.3e-09	0.684	0	0	\\
1.35e-09	0.684	0	0	\\
1.4e-09	0.684	0	0	\\
1.45e-09	0.684	0	0	\\
1.5e-09	0.684	0	0	\\
1.55e-09	0.684	0	0	\\
1.6e-09	0.684	0	0	\\
1.65e-09	0.684	0	0	\\
1.7e-09	0.684	0	0	\\
1.75e-09	0.684	-0.00666666666666667	-0.00666666666666667	\\
1.8e-09	0.684	-0.00666666666666667	-0.00666666666666667	\\
1.85e-09	0.684	-0.00666666666666667	-0.00666666666666667	\\
1.9e-09	0.684	-0.00666666666666667	-0.00666666666666667	\\
1.95e-09	0.684	-0.00666666666666667	-0.00666666666666667	\\
2e-09	0.684	-0.00666666666666667	-0.00666666666666667	\\
2.05e-09	0.684	-0.00666666666666667	-0.00666666666666667	\\
2.1e-09	0.684	-0.00666666666666667	-0.00666666666666667	\\
2.15e-09	0.684	-0.00666666666666667	-0.00666666666666667	\\
2.2e-09	0.684	-0.00666666666666667	-0.00666666666666667	\\
2.25e-09	0.684	0	0	\\
2.3e-09	0.684	0	0	\\
2.35e-09	0.684	0	0	\\
2.4e-09	0.684	0	0	\\
2.45e-09	0.684	0	0	\\
2.5e-09	0.684	0	0	\\
2.55e-09	0.684	0	0	\\
2.6e-09	0.684	0	0	\\
2.65e-09	0.684	0	0	\\
2.7e-09	0.684	0	0	\\
2.75e-09	0.684	0	0	\\
2.8e-09	0.684	0	0	\\
2.85e-09	0.684	0	0	\\
2.9e-09	0.684	0	0	\\
2.95e-09	0.684	0	0	\\
3e-09	0.684	0	0	\\
3.05e-09	0.684	0	0	\\
3.1e-09	0.684	0.00666666666666667	0.00666666666666667	\\
3.15e-09	0.684	0.00666666666666667	0.00666666666666667	\\
3.2e-09	0.684	0.00666666666666667	0.00666666666666667	\\
3.25e-09	0.684	0.00666666666666667	0.00666666666666667	\\
3.3e-09	0.684	0.00666666666666667	0.00666666666666667	\\
3.35e-09	0.684	0.00666666666666667	0.00666666666666667	\\
3.4e-09	0.684	0.00666666666666667	0.00666666666666667	\\
3.45e-09	0.684	0.00666666666666667	0.00666666666666667	\\
3.5e-09	0.684	0.00666666666666667	0.00666666666666667	\\
3.55e-09	0.684	0.00666666666666667	0.00666666666666667	\\
3.6e-09	0.684	0	0	\\
3.65e-09	0.684	0	0	\\
3.7e-09	0.684	0	0	\\
3.75e-09	0.684	0	0	\\
3.8e-09	0.684	0	0	\\
3.85e-09	0.684	0	0	\\
3.9e-09	0.684	0	0	\\
3.95e-09	0.684	0	0	\\
4e-09	0.684	0	0	\\
4.05e-09	0.684	0	0	\\
4.1e-09	0.684	0	0	\\
4.15e-09	0.684	-0.00444444444444444	-0.00444444444444444	\\
4.2e-09	0.684	-0.00444444444444444	-0.00444444444444444	\\
4.25e-09	0.684	-0.00444444444444444	-0.00444444444444444	\\
4.3e-09	0.684	-0.00444444444444444	-0.00444444444444444	\\
4.35e-09	0.684	-0.00444444444444444	-0.00444444444444444	\\
4.4e-09	0.684	-0.00444444444444444	-0.00444444444444444	\\
4.45e-09	0.684	-0.00444444444444444	-0.00444444444444444	\\
4.5e-09	0.684	-0.00444444444444444	-0.00444444444444444	\\
4.55e-09	0.684	-0.00444444444444444	-0.00444444444444444	\\
4.6e-09	0.684	-0.00444444444444444	-0.00444444444444444	\\
4.65e-09	0.684	0	0	\\
4.7e-09	0.684	0	0	\\
4.75e-09	0.684	0	0	\\
4.8e-09	0.684	0	0	\\
4.85e-09	0.684	0	0	\\
4.9e-09	0.684	0	0	\\
4.95e-09	0.684	0	0	\\
5e-09	0.684	0	0	\\
5e-09	0.684	0	nan	\\
5e-09	0.684	-0.00666666666666667	0.00166666666666667	\\
5e-09	0.684	-0.00666666666666667	nan	\\
0	0.696	-0.00666666666666667	nan	\\
0	0.696	0	0.00166666666666667	\\
0	0.696	0	0	\\
5e-11	0.696	0	0	\\
1e-10	0.696	0	0	\\
1.5e-10	0.696	0	0	\\
2e-10	0.696	0	0	\\
2.5e-10	0.696	0	0	\\
3e-10	0.696	0	0	\\
3.5e-10	0.696	0	0	\\
4e-10	0.696	0	0	\\
4.5e-10	0.696	0	0	\\
5e-10	0.696	0	0	\\
5.5e-10	0.696	0	0	\\
6e-10	0.696	0	0	\\
6.5e-10	0.696	0	0	\\
7e-10	0.696	0.01	0.01	\\
7.5e-10	0.696	0.01	0.01	\\
8e-10	0.696	0.01	0.01	\\
8.5e-10	0.696	0.01	0.01	\\
9e-10	0.696	0.01	0.01	\\
9.5e-10	0.696	0.01	0.01	\\
1e-09	0.696	0.01	0.01	\\
1.05e-09	0.696	0.01	0.01	\\
1.1e-09	0.696	0.01	0.01	\\
1.15e-09	0.696	0.01	0.01	\\
1.2e-09	0.696	0	0	\\
1.25e-09	0.696	0	0	\\
1.3e-09	0.696	0	0	\\
1.35e-09	0.696	0	0	\\
1.4e-09	0.696	0	0	\\
1.45e-09	0.696	0	0	\\
1.5e-09	0.696	0	0	\\
1.55e-09	0.696	0	0	\\
1.6e-09	0.696	0	0	\\
1.65e-09	0.696	0	0	\\
1.7e-09	0.696	0	0	\\
1.75e-09	0.696	-0.00666666666666667	-0.00666666666666667	\\
1.8e-09	0.696	-0.00666666666666667	-0.00666666666666667	\\
1.85e-09	0.696	-0.00666666666666667	-0.00666666666666667	\\
1.9e-09	0.696	-0.00666666666666667	-0.00666666666666667	\\
1.95e-09	0.696	-0.00666666666666667	-0.00666666666666667	\\
2e-09	0.696	-0.00666666666666667	-0.00666666666666667	\\
2.05e-09	0.696	-0.00666666666666667	-0.00666666666666667	\\
2.1e-09	0.696	-0.00666666666666667	-0.00666666666666667	\\
2.15e-09	0.696	-0.00666666666666667	-0.00666666666666667	\\
2.2e-09	0.696	-0.00666666666666667	-0.00666666666666667	\\
2.25e-09	0.696	0	0	\\
2.3e-09	0.696	0	0	\\
2.35e-09	0.696	0	0	\\
2.4e-09	0.696	0	0	\\
2.45e-09	0.696	0	0	\\
2.5e-09	0.696	0	0	\\
2.55e-09	0.696	0	0	\\
2.6e-09	0.696	0	0	\\
2.65e-09	0.696	0	0	\\
2.7e-09	0.696	0	0	\\
2.75e-09	0.696	0	0	\\
2.8e-09	0.696	0	0	\\
2.85e-09	0.696	0	0	\\
2.9e-09	0.696	0	0	\\
2.95e-09	0.696	0	0	\\
3e-09	0.696	0	0	\\
3.05e-09	0.696	0	0	\\
3.1e-09	0.696	0.00666666666666667	0.00666666666666667	\\
3.15e-09	0.696	0.00666666666666667	0.00666666666666667	\\
3.2e-09	0.696	0.00666666666666667	0.00666666666666667	\\
3.25e-09	0.696	0.00666666666666667	0.00666666666666667	\\
3.3e-09	0.696	0.00666666666666667	0.00666666666666667	\\
3.35e-09	0.696	0.00666666666666667	0.00666666666666667	\\
3.4e-09	0.696	0.00666666666666667	0.00666666666666667	\\
3.45e-09	0.696	0.00666666666666667	0.00666666666666667	\\
3.5e-09	0.696	0.00666666666666667	0.00666666666666667	\\
3.55e-09	0.696	0.00666666666666667	0.00666666666666667	\\
3.6e-09	0.696	0	0	\\
3.65e-09	0.696	0	0	\\
3.7e-09	0.696	0	0	\\
3.75e-09	0.696	0	0	\\
3.8e-09	0.696	0	0	\\
3.85e-09	0.696	0	0	\\
3.9e-09	0.696	0	0	\\
3.95e-09	0.696	0	0	\\
4e-09	0.696	0	0	\\
4.05e-09	0.696	0	0	\\
4.1e-09	0.696	0	0	\\
4.15e-09	0.696	-0.00444444444444444	-0.00444444444444444	\\
4.2e-09	0.696	-0.00444444444444444	-0.00444444444444444	\\
4.25e-09	0.696	-0.00444444444444444	-0.00444444444444444	\\
4.3e-09	0.696	-0.00444444444444444	-0.00444444444444444	\\
4.35e-09	0.696	-0.00444444444444444	-0.00444444444444444	\\
4.4e-09	0.696	-0.00444444444444444	-0.00444444444444444	\\
4.45e-09	0.696	-0.00444444444444444	-0.00444444444444444	\\
4.5e-09	0.696	-0.00444444444444444	-0.00444444444444444	\\
4.55e-09	0.696	-0.00444444444444444	-0.00444444444444444	\\
4.6e-09	0.696	-0.00444444444444444	-0.00444444444444444	\\
4.65e-09	0.696	0	0	\\
4.7e-09	0.696	0	0	\\
4.75e-09	0.696	0	0	\\
4.8e-09	0.696	0	0	\\
4.85e-09	0.696	0	0	\\
4.9e-09	0.696	0	0	\\
4.95e-09	0.696	0	0	\\
5e-09	0.696	0	0	\\
5e-09	0.696	0	nan	\\
5e-09	0.696	-0.00666666666666667	0.00166666666666667	\\
5e-09	0.696	-0.00666666666666667	nan	\\
0	0.708	-0.00666666666666667	nan	\\
0	0.708	0	0.00166666666666667	\\
0	0.708	0	0	\\
5e-11	0.708	0	0	\\
1e-10	0.708	0	0	\\
1.5e-10	0.708	0	0	\\
2e-10	0.708	0	0	\\
2.5e-10	0.708	0	0	\\
3e-10	0.708	0	0	\\
3.5e-10	0.708	0	0	\\
4e-10	0.708	0	0	\\
4.5e-10	0.708	0	0	\\
5e-10	0.708	0	0	\\
5.5e-10	0.708	0	0	\\
6e-10	0.708	0	0	\\
6.5e-10	0.708	0	0	\\
7e-10	0.708	0	0	\\
7.5e-10	0.708	0.01	0.01	\\
8e-10	0.708	0.01	0.01	\\
8.5e-10	0.708	0.01	0.01	\\
9e-10	0.708	0.01	0.01	\\
9.5e-10	0.708	0.01	0.01	\\
1e-09	0.708	0.01	0.01	\\
1.05e-09	0.708	0.01	0.01	\\
1.1e-09	0.708	0.01	0.01	\\
1.15e-09	0.708	0.01	0.01	\\
1.2e-09	0.708	0.01	0.01	\\
1.25e-09	0.708	0	0	\\
1.3e-09	0.708	0	0	\\
1.35e-09	0.708	0	0	\\
1.4e-09	0.708	0	0	\\
1.45e-09	0.708	0	0	\\
1.5e-09	0.708	0	0	\\
1.55e-09	0.708	0	0	\\
1.6e-09	0.708	0	0	\\
1.65e-09	0.708	0	0	\\
1.7e-09	0.708	-0.00666666666666667	-0.00666666666666667	\\
1.75e-09	0.708	-0.00666666666666667	-0.00666666666666667	\\
1.8e-09	0.708	-0.00666666666666667	-0.00666666666666667	\\
1.85e-09	0.708	-0.00666666666666667	-0.00666666666666667	\\
1.9e-09	0.708	-0.00666666666666667	-0.00666666666666667	\\
1.95e-09	0.708	-0.00666666666666667	-0.00666666666666667	\\
2e-09	0.708	-0.00666666666666667	-0.00666666666666667	\\
2.05e-09	0.708	-0.00666666666666667	-0.00666666666666667	\\
2.1e-09	0.708	-0.00666666666666667	-0.00666666666666667	\\
2.15e-09	0.708	-0.00666666666666667	-0.00666666666666667	\\
2.2e-09	0.708	0	0	\\
2.25e-09	0.708	0	0	\\
2.3e-09	0.708	0	0	\\
2.35e-09	0.708	0	0	\\
2.4e-09	0.708	0	0	\\
2.45e-09	0.708	0	0	\\
2.5e-09	0.708	0	0	\\
2.55e-09	0.708	0	0	\\
2.6e-09	0.708	0	0	\\
2.65e-09	0.708	0	0	\\
2.7e-09	0.708	0	0	\\
2.75e-09	0.708	0	0	\\
2.8e-09	0.708	0	0	\\
2.85e-09	0.708	0	0	\\
2.9e-09	0.708	0	0	\\
2.95e-09	0.708	0	0	\\
3e-09	0.708	0	0	\\
3.05e-09	0.708	0	0	\\
3.1e-09	0.708	0	0	\\
3.15e-09	0.708	0.00666666666666667	0.00666666666666667	\\
3.2e-09	0.708	0.00666666666666667	0.00666666666666667	\\
3.25e-09	0.708	0.00666666666666667	0.00666666666666667	\\
3.3e-09	0.708	0.00666666666666667	0.00666666666666667	\\
3.35e-09	0.708	0.00666666666666667	0.00666666666666667	\\
3.4e-09	0.708	0.00666666666666667	0.00666666666666667	\\
3.45e-09	0.708	0.00666666666666667	0.00666666666666667	\\
3.5e-09	0.708	0.00666666666666667	0.00666666666666667	\\
3.55e-09	0.708	0.00666666666666667	0.00666666666666667	\\
3.6e-09	0.708	0.00666666666666667	0.00666666666666667	\\
3.65e-09	0.708	0	0	\\
3.7e-09	0.708	0	0	\\
3.75e-09	0.708	0	0	\\
3.8e-09	0.708	0	0	\\
3.85e-09	0.708	0	0	\\
3.9e-09	0.708	0	0	\\
3.95e-09	0.708	0	0	\\
4e-09	0.708	0	0	\\
4.05e-09	0.708	0	0	\\
4.1e-09	0.708	-0.00444444444444444	-0.00444444444444444	\\
4.15e-09	0.708	-0.00444444444444444	-0.00444444444444444	\\
4.2e-09	0.708	-0.00444444444444444	-0.00444444444444444	\\
4.25e-09	0.708	-0.00444444444444444	-0.00444444444444444	\\
4.3e-09	0.708	-0.00444444444444444	-0.00444444444444444	\\
4.35e-09	0.708	-0.00444444444444444	-0.00444444444444444	\\
4.4e-09	0.708	-0.00444444444444444	-0.00444444444444444	\\
4.45e-09	0.708	-0.00444444444444444	-0.00444444444444444	\\
4.5e-09	0.708	-0.00444444444444444	-0.00444444444444444	\\
4.55e-09	0.708	-0.00444444444444444	-0.00444444444444444	\\
4.6e-09	0.708	0	0	\\
4.65e-09	0.708	0	0	\\
4.7e-09	0.708	0	0	\\
4.75e-09	0.708	0	0	\\
4.8e-09	0.708	0	0	\\
4.85e-09	0.708	0	0	\\
4.9e-09	0.708	0	0	\\
4.95e-09	0.708	0	0	\\
5e-09	0.708	0	0	\\
5e-09	0.708	0	nan	\\
5e-09	0.708	-0.00666666666666667	0.00166666666666667	\\
5e-09	0.708	-0.00666666666666667	nan	\\
0	0.72	-0.00666666666666667	nan	\\
0	0.72	0	0.00166666666666667	\\
0	0.72	0	0	\\
5e-11	0.72	0	0	\\
1e-10	0.72	0	0	\\
1.5e-10	0.72	0	0	\\
2e-10	0.72	0	0	\\
2.5e-10	0.72	0	0	\\
3e-10	0.72	0	0	\\
3.5e-10	0.72	0	0	\\
4e-10	0.72	0	0	\\
4.5e-10	0.72	0	0	\\
5e-10	0.72	0	0	\\
5.5e-10	0.72	0	0	\\
6e-10	0.72	0	0	\\
6.5e-10	0.72	0	0	\\
7e-10	0.72	0	0	\\
7.5e-10	0.72	0.01	0.01	\\
8e-10	0.72	0.01	0.01	\\
8.5e-10	0.72	0.01	0.01	\\
9e-10	0.72	0.01	0.01	\\
9.5e-10	0.72	0.01	0.01	\\
1e-09	0.72	0.01	0.01	\\
1.05e-09	0.72	0.01	0.01	\\
1.1e-09	0.72	0.01	0.01	\\
1.15e-09	0.72	0.01	0.01	\\
1.2e-09	0.72	0.01	0.01	\\
1.25e-09	0.72	0	0	\\
1.3e-09	0.72	0	0	\\
1.35e-09	0.72	0	0	\\
1.4e-09	0.72	0	0	\\
1.45e-09	0.72	0	0	\\
1.5e-09	0.72	0	0	\\
1.55e-09	0.72	0	0	\\
1.6e-09	0.72	0	0	\\
1.65e-09	0.72	0	0	\\
1.7e-09	0.72	-0.00666666666666667	-0.00666666666666667	\\
1.75e-09	0.72	-0.00666666666666667	-0.00666666666666667	\\
1.8e-09	0.72	-0.00666666666666667	-0.00666666666666667	\\
1.85e-09	0.72	-0.00666666666666667	-0.00666666666666667	\\
1.9e-09	0.72	-0.00666666666666667	-0.00666666666666667	\\
1.95e-09	0.72	-0.00666666666666667	-0.00666666666666667	\\
2e-09	0.72	-0.00666666666666667	-0.00666666666666667	\\
2.05e-09	0.72	-0.00666666666666667	-0.00666666666666667	\\
2.1e-09	0.72	-0.00666666666666667	-0.00666666666666667	\\
2.15e-09	0.72	-0.00666666666666667	-0.00666666666666667	\\
2.2e-09	0.72	0	0	\\
2.25e-09	0.72	0	0	\\
2.3e-09	0.72	0	0	\\
2.35e-09	0.72	0	0	\\
2.4e-09	0.72	0	0	\\
2.45e-09	0.72	0	0	\\
2.5e-09	0.72	0	0	\\
2.55e-09	0.72	0	0	\\
2.6e-09	0.72	0	0	\\
2.65e-09	0.72	0	0	\\
2.7e-09	0.72	0	0	\\
2.75e-09	0.72	0	0	\\
2.8e-09	0.72	0	0	\\
2.85e-09	0.72	0	0	\\
2.9e-09	0.72	0	0	\\
2.95e-09	0.72	0	0	\\
3e-09	0.72	0	0	\\
3.05e-09	0.72	0	0	\\
3.1e-09	0.72	0	0	\\
3.15e-09	0.72	0.00666666666666667	0.00666666666666667	\\
3.2e-09	0.72	0.00666666666666667	0.00666666666666667	\\
3.25e-09	0.72	0.00666666666666667	0.00666666666666667	\\
3.3e-09	0.72	0.00666666666666667	0.00666666666666667	\\
3.35e-09	0.72	0.00666666666666667	0.00666666666666667	\\
3.4e-09	0.72	0.00666666666666667	0.00666666666666667	\\
3.45e-09	0.72	0.00666666666666667	0.00666666666666667	\\
3.5e-09	0.72	0.00666666666666667	0.00666666666666667	\\
3.55e-09	0.72	0.00666666666666667	0.00666666666666667	\\
3.6e-09	0.72	0.00666666666666667	0.00666666666666667	\\
3.65e-09	0.72	0	0	\\
3.7e-09	0.72	0	0	\\
3.75e-09	0.72	0	0	\\
3.8e-09	0.72	0	0	\\
3.85e-09	0.72	0	0	\\
3.9e-09	0.72	0	0	\\
3.95e-09	0.72	0	0	\\
4e-09	0.72	0	0	\\
4.05e-09	0.72	0	0	\\
4.1e-09	0.72	-0.00444444444444444	-0.00444444444444444	\\
4.15e-09	0.72	-0.00444444444444444	-0.00444444444444444	\\
4.2e-09	0.72	-0.00444444444444444	-0.00444444444444444	\\
4.25e-09	0.72	-0.00444444444444444	-0.00444444444444444	\\
4.3e-09	0.72	-0.00444444444444444	-0.00444444444444444	\\
4.35e-09	0.72	-0.00444444444444444	-0.00444444444444444	\\
4.4e-09	0.72	-0.00444444444444444	-0.00444444444444444	\\
4.45e-09	0.72	-0.00444444444444444	-0.00444444444444444	\\
4.5e-09	0.72	-0.00444444444444444	-0.00444444444444444	\\
4.55e-09	0.72	-0.00444444444444444	-0.00444444444444444	\\
4.6e-09	0.72	0	0	\\
4.65e-09	0.72	0	0	\\
4.7e-09	0.72	0	0	\\
4.75e-09	0.72	0	0	\\
4.8e-09	0.72	0	0	\\
4.85e-09	0.72	0	0	\\
4.9e-09	0.72	0	0	\\
4.95e-09	0.72	0	0	\\
5e-09	0.72	0	0	\\
5e-09	0.72	0	nan	\\
5e-09	0.72	-0.00666666666666667	0.00166666666666667	\\
5e-09	0.72	-0.00666666666666667	nan	\\
0	0.732	-0.00666666666666667	nan	\\
0	0.732	0	0.00166666666666667	\\
0	0.732	0	0	\\
5e-11	0.732	0	0	\\
1e-10	0.732	0	0	\\
1.5e-10	0.732	0	0	\\
2e-10	0.732	0	0	\\
2.5e-10	0.732	0	0	\\
3e-10	0.732	0	0	\\
3.5e-10	0.732	0	0	\\
4e-10	0.732	0	0	\\
4.5e-10	0.732	0	0	\\
5e-10	0.732	0	0	\\
5.5e-10	0.732	0	0	\\
6e-10	0.732	0	0	\\
6.5e-10	0.732	0	0	\\
7e-10	0.732	0	0	\\
7.5e-10	0.732	0.01	0.01	\\
8e-10	0.732	0.01	0.01	\\
8.5e-10	0.732	0.01	0.01	\\
9e-10	0.732	0.01	0.01	\\
9.5e-10	0.732	0.01	0.01	\\
1e-09	0.732	0.01	0.01	\\
1.05e-09	0.732	0.01	0.01	\\
1.1e-09	0.732	0.01	0.01	\\
1.15e-09	0.732	0.01	0.01	\\
1.2e-09	0.732	0.01	0.01	\\
1.25e-09	0.732	0	0	\\
1.3e-09	0.732	0	0	\\
1.35e-09	0.732	0	0	\\
1.4e-09	0.732	0	0	\\
1.45e-09	0.732	0	0	\\
1.5e-09	0.732	0	0	\\
1.55e-09	0.732	0	0	\\
1.6e-09	0.732	0	0	\\
1.65e-09	0.732	0	0	\\
1.7e-09	0.732	-0.00666666666666667	-0.00666666666666667	\\
1.75e-09	0.732	-0.00666666666666667	-0.00666666666666667	\\
1.8e-09	0.732	-0.00666666666666667	-0.00666666666666667	\\
1.85e-09	0.732	-0.00666666666666667	-0.00666666666666667	\\
1.9e-09	0.732	-0.00666666666666667	-0.00666666666666667	\\
1.95e-09	0.732	-0.00666666666666667	-0.00666666666666667	\\
2e-09	0.732	-0.00666666666666667	-0.00666666666666667	\\
2.05e-09	0.732	-0.00666666666666667	-0.00666666666666667	\\
2.1e-09	0.732	-0.00666666666666667	-0.00666666666666667	\\
2.15e-09	0.732	-0.00666666666666667	-0.00666666666666667	\\
2.2e-09	0.732	0	0	\\
2.25e-09	0.732	0	0	\\
2.3e-09	0.732	0	0	\\
2.35e-09	0.732	0	0	\\
2.4e-09	0.732	0	0	\\
2.45e-09	0.732	0	0	\\
2.5e-09	0.732	0	0	\\
2.55e-09	0.732	0	0	\\
2.6e-09	0.732	0	0	\\
2.65e-09	0.732	0	0	\\
2.7e-09	0.732	0	0	\\
2.75e-09	0.732	0	0	\\
2.8e-09	0.732	0	0	\\
2.85e-09	0.732	0	0	\\
2.9e-09	0.732	0	0	\\
2.95e-09	0.732	0	0	\\
3e-09	0.732	0	0	\\
3.05e-09	0.732	0	0	\\
3.1e-09	0.732	0	0	\\
3.15e-09	0.732	0.00666666666666667	0.00666666666666667	\\
3.2e-09	0.732	0.00666666666666667	0.00666666666666667	\\
3.25e-09	0.732	0.00666666666666667	0.00666666666666667	\\
3.3e-09	0.732	0.00666666666666667	0.00666666666666667	\\
3.35e-09	0.732	0.00666666666666667	0.00666666666666667	\\
3.4e-09	0.732	0.00666666666666667	0.00666666666666667	\\
3.45e-09	0.732	0.00666666666666667	0.00666666666666667	\\
3.5e-09	0.732	0.00666666666666667	0.00666666666666667	\\
3.55e-09	0.732	0.00666666666666667	0.00666666666666667	\\
3.6e-09	0.732	0.00666666666666667	0.00666666666666667	\\
3.65e-09	0.732	0	0	\\
3.7e-09	0.732	0	0	\\
3.75e-09	0.732	0	0	\\
3.8e-09	0.732	0	0	\\
3.85e-09	0.732	0	0	\\
3.9e-09	0.732	0	0	\\
3.95e-09	0.732	0	0	\\
4e-09	0.732	0	0	\\
4.05e-09	0.732	0	0	\\
4.1e-09	0.732	-0.00444444444444444	-0.00444444444444444	\\
4.15e-09	0.732	-0.00444444444444444	-0.00444444444444444	\\
4.2e-09	0.732	-0.00444444444444444	-0.00444444444444444	\\
4.25e-09	0.732	-0.00444444444444444	-0.00444444444444444	\\
4.3e-09	0.732	-0.00444444444444444	-0.00444444444444444	\\
4.35e-09	0.732	-0.00444444444444444	-0.00444444444444444	\\
4.4e-09	0.732	-0.00444444444444444	-0.00444444444444444	\\
4.45e-09	0.732	-0.00444444444444444	-0.00444444444444444	\\
4.5e-09	0.732	-0.00444444444444444	-0.00444444444444444	\\
4.55e-09	0.732	-0.00444444444444444	-0.00444444444444444	\\
4.6e-09	0.732	0	0	\\
4.65e-09	0.732	0	0	\\
4.7e-09	0.732	0	0	\\
4.75e-09	0.732	0	0	\\
4.8e-09	0.732	0	0	\\
4.85e-09	0.732	0	0	\\
4.9e-09	0.732	0	0	\\
4.95e-09	0.732	0	0	\\
5e-09	0.732	0	0	\\
5e-09	0.732	0	nan	\\
5e-09	0.732	-0.00666666666666667	0.00166666666666667	\\
5e-09	0.732	-0.00666666666666667	nan	\\
0	0.744	-0.00666666666666667	nan	\\
0	0.744	0	0.00166666666666667	\\
0	0.744	0	0	\\
5e-11	0.744	0	0	\\
1e-10	0.744	0	0	\\
1.5e-10	0.744	0	0	\\
2e-10	0.744	0	0	\\
2.5e-10	0.744	0	0	\\
3e-10	0.744	0	0	\\
3.5e-10	0.744	0	0	\\
4e-10	0.744	0	0	\\
4.5e-10	0.744	0	0	\\
5e-10	0.744	0	0	\\
5.5e-10	0.744	0	0	\\
6e-10	0.744	0	0	\\
6.5e-10	0.744	0	0	\\
7e-10	0.744	0	0	\\
7.5e-10	0.744	0.01	0.01	\\
8e-10	0.744	0.01	0.01	\\
8.5e-10	0.744	0.01	0.01	\\
9e-10	0.744	0.01	0.01	\\
9.5e-10	0.744	0.01	0.01	\\
1e-09	0.744	0.01	0.01	\\
1.05e-09	0.744	0.01	0.01	\\
1.1e-09	0.744	0.01	0.01	\\
1.15e-09	0.744	0.01	0.01	\\
1.2e-09	0.744	0.01	0.01	\\
1.25e-09	0.744	0	0	\\
1.3e-09	0.744	0	0	\\
1.35e-09	0.744	0	0	\\
1.4e-09	0.744	0	0	\\
1.45e-09	0.744	0	0	\\
1.5e-09	0.744	0	0	\\
1.55e-09	0.744	0	0	\\
1.6e-09	0.744	0	0	\\
1.65e-09	0.744	0	0	\\
1.7e-09	0.744	-0.00666666666666667	-0.00666666666666667	\\
1.75e-09	0.744	-0.00666666666666667	-0.00666666666666667	\\
1.8e-09	0.744	-0.00666666666666667	-0.00666666666666667	\\
1.85e-09	0.744	-0.00666666666666667	-0.00666666666666667	\\
1.9e-09	0.744	-0.00666666666666667	-0.00666666666666667	\\
1.95e-09	0.744	-0.00666666666666667	-0.00666666666666667	\\
2e-09	0.744	-0.00666666666666667	-0.00666666666666667	\\
2.05e-09	0.744	-0.00666666666666667	-0.00666666666666667	\\
2.1e-09	0.744	-0.00666666666666667	-0.00666666666666667	\\
2.15e-09	0.744	-0.00666666666666667	-0.00666666666666667	\\
2.2e-09	0.744	0	0	\\
2.25e-09	0.744	0	0	\\
2.3e-09	0.744	0	0	\\
2.35e-09	0.744	0	0	\\
2.4e-09	0.744	0	0	\\
2.45e-09	0.744	0	0	\\
2.5e-09	0.744	0	0	\\
2.55e-09	0.744	0	0	\\
2.6e-09	0.744	0	0	\\
2.65e-09	0.744	0	0	\\
2.7e-09	0.744	0	0	\\
2.75e-09	0.744	0	0	\\
2.8e-09	0.744	0	0	\\
2.85e-09	0.744	0	0	\\
2.9e-09	0.744	0	0	\\
2.95e-09	0.744	0	0	\\
3e-09	0.744	0	0	\\
3.05e-09	0.744	0	0	\\
3.1e-09	0.744	0	0	\\
3.15e-09	0.744	0.00666666666666667	0.00666666666666667	\\
3.2e-09	0.744	0.00666666666666667	0.00666666666666667	\\
3.25e-09	0.744	0.00666666666666667	0.00666666666666667	\\
3.3e-09	0.744	0.00666666666666667	0.00666666666666667	\\
3.35e-09	0.744	0.00666666666666667	0.00666666666666667	\\
3.4e-09	0.744	0.00666666666666667	0.00666666666666667	\\
3.45e-09	0.744	0.00666666666666667	0.00666666666666667	\\
3.5e-09	0.744	0.00666666666666667	0.00666666666666667	\\
3.55e-09	0.744	0.00666666666666667	0.00666666666666667	\\
3.6e-09	0.744	0.00666666666666667	0.00666666666666667	\\
3.65e-09	0.744	0	0	\\
3.7e-09	0.744	0	0	\\
3.75e-09	0.744	0	0	\\
3.8e-09	0.744	0	0	\\
3.85e-09	0.744	0	0	\\
3.9e-09	0.744	0	0	\\
3.95e-09	0.744	0	0	\\
4e-09	0.744	0	0	\\
4.05e-09	0.744	0	0	\\
4.1e-09	0.744	-0.00444444444444444	-0.00444444444444444	\\
4.15e-09	0.744	-0.00444444444444444	-0.00444444444444444	\\
4.2e-09	0.744	-0.00444444444444444	-0.00444444444444444	\\
4.25e-09	0.744	-0.00444444444444444	-0.00444444444444444	\\
4.3e-09	0.744	-0.00444444444444444	-0.00444444444444444	\\
4.35e-09	0.744	-0.00444444444444444	-0.00444444444444444	\\
4.4e-09	0.744	-0.00444444444444444	-0.00444444444444444	\\
4.45e-09	0.744	-0.00444444444444444	-0.00444444444444444	\\
4.5e-09	0.744	-0.00444444444444444	-0.00444444444444444	\\
4.55e-09	0.744	-0.00444444444444444	-0.00444444444444444	\\
4.6e-09	0.744	0	0	\\
4.65e-09	0.744	0	0	\\
4.7e-09	0.744	0	0	\\
4.75e-09	0.744	0	0	\\
4.8e-09	0.744	0	0	\\
4.85e-09	0.744	0	0	\\
4.9e-09	0.744	0	0	\\
4.95e-09	0.744	0	0	\\
5e-09	0.744	0	0	\\
5e-09	0.744	0	nan	\\
5e-09	0.744	-0.00666666666666667	0.00166666666666667	\\
5e-09	0.744	-0.00666666666666667	nan	\\
0	0.756	-0.00666666666666667	nan	\\
0	0.756	0	0.00166666666666667	\\
0	0.756	0	0	\\
5e-11	0.756	0	0	\\
1e-10	0.756	0	0	\\
1.5e-10	0.756	0	0	\\
2e-10	0.756	0	0	\\
2.5e-10	0.756	0	0	\\
3e-10	0.756	0	0	\\
3.5e-10	0.756	0	0	\\
4e-10	0.756	0	0	\\
4.5e-10	0.756	0	0	\\
5e-10	0.756	0	0	\\
5.5e-10	0.756	0	0	\\
6e-10	0.756	0	0	\\
6.5e-10	0.756	0	0	\\
7e-10	0.756	0	0	\\
7.5e-10	0.756	0	0	\\
8e-10	0.756	0.01	0.01	\\
8.5e-10	0.756	0.01	0.01	\\
9e-10	0.756	0.01	0.01	\\
9.5e-10	0.756	0.01	0.01	\\
1e-09	0.756	0.01	0.01	\\
1.05e-09	0.756	0.01	0.01	\\
1.1e-09	0.756	0.01	0.01	\\
1.15e-09	0.756	0.01	0.01	\\
1.2e-09	0.756	0.01	0.01	\\
1.25e-09	0.756	0.01	0.01	\\
1.3e-09	0.756	0	0	\\
1.35e-09	0.756	0	0	\\
1.4e-09	0.756	0	0	\\
1.45e-09	0.756	0	0	\\
1.5e-09	0.756	0	0	\\
1.55e-09	0.756	0	0	\\
1.6e-09	0.756	0	0	\\
1.65e-09	0.756	-0.00666666666666667	-0.00666666666666667	\\
1.7e-09	0.756	-0.00666666666666667	-0.00666666666666667	\\
1.75e-09	0.756	-0.00666666666666667	-0.00666666666666667	\\
1.8e-09	0.756	-0.00666666666666667	-0.00666666666666667	\\
1.85e-09	0.756	-0.00666666666666667	-0.00666666666666667	\\
1.9e-09	0.756	-0.00666666666666667	-0.00666666666666667	\\
1.95e-09	0.756	-0.00666666666666667	-0.00666666666666667	\\
2e-09	0.756	-0.00666666666666667	-0.00666666666666667	\\
2.05e-09	0.756	-0.00666666666666667	-0.00666666666666667	\\
2.1e-09	0.756	-0.00666666666666667	-0.00666666666666667	\\
2.15e-09	0.756	0	0	\\
2.2e-09	0.756	0	0	\\
2.25e-09	0.756	0	0	\\
2.3e-09	0.756	0	0	\\
2.35e-09	0.756	0	0	\\
2.4e-09	0.756	0	0	\\
2.45e-09	0.756	0	0	\\
2.5e-09	0.756	0	0	\\
2.55e-09	0.756	0	0	\\
2.6e-09	0.756	0	0	\\
2.65e-09	0.756	0	0	\\
2.7e-09	0.756	0	0	\\
2.75e-09	0.756	0	0	\\
2.8e-09	0.756	0	0	\\
2.85e-09	0.756	0	0	\\
2.9e-09	0.756	0	0	\\
2.95e-09	0.756	0	0	\\
3e-09	0.756	0	0	\\
3.05e-09	0.756	0	0	\\
3.1e-09	0.756	0	0	\\
3.15e-09	0.756	0	0	\\
3.2e-09	0.756	0.00666666666666667	0.00666666666666667	\\
3.25e-09	0.756	0.00666666666666667	0.00666666666666667	\\
3.3e-09	0.756	0.00666666666666667	0.00666666666666667	\\
3.35e-09	0.756	0.00666666666666667	0.00666666666666667	\\
3.4e-09	0.756	0.00666666666666667	0.00666666666666667	\\
3.45e-09	0.756	0.00666666666666667	0.00666666666666667	\\
3.5e-09	0.756	0.00666666666666667	0.00666666666666667	\\
3.55e-09	0.756	0.00666666666666667	0.00666666666666667	\\
3.6e-09	0.756	0.00666666666666667	0.00666666666666667	\\
3.65e-09	0.756	0.00666666666666667	0.00666666666666667	\\
3.7e-09	0.756	0	0	\\
3.75e-09	0.756	0	0	\\
3.8e-09	0.756	0	0	\\
3.85e-09	0.756	0	0	\\
3.9e-09	0.756	0	0	\\
3.95e-09	0.756	0	0	\\
4e-09	0.756	0	0	\\
4.05e-09	0.756	-0.00444444444444444	-0.00444444444444444	\\
4.1e-09	0.756	-0.00444444444444444	-0.00444444444444444	\\
4.15e-09	0.756	-0.00444444444444444	-0.00444444444444444	\\
4.2e-09	0.756	-0.00444444444444444	-0.00444444444444444	\\
4.25e-09	0.756	-0.00444444444444444	-0.00444444444444444	\\
4.3e-09	0.756	-0.00444444444444444	-0.00444444444444444	\\
4.35e-09	0.756	-0.00444444444444444	-0.00444444444444444	\\
4.4e-09	0.756	-0.00444444444444444	-0.00444444444444444	\\
4.45e-09	0.756	-0.00444444444444444	-0.00444444444444444	\\
4.5e-09	0.756	-0.00444444444444444	-0.00444444444444444	\\
4.55e-09	0.756	0	0	\\
4.6e-09	0.756	0	0	\\
4.65e-09	0.756	0	0	\\
4.7e-09	0.756	0	0	\\
4.75e-09	0.756	0	0	\\
4.8e-09	0.756	0	0	\\
4.85e-09	0.756	0	0	\\
4.9e-09	0.756	0	0	\\
4.95e-09	0.756	0	0	\\
5e-09	0.756	0	0	\\
5e-09	0.756	0	nan	\\
5e-09	0.756	-0.00666666666666667	0.00166666666666667	\\
5e-09	0.756	-0.00666666666666667	nan	\\
0	0.768	-0.00666666666666667	nan	\\
0	0.768	0	0.00166666666666667	\\
0	0.768	0	0	\\
5e-11	0.768	0	0	\\
1e-10	0.768	0	0	\\
1.5e-10	0.768	0	0	\\
2e-10	0.768	0	0	\\
2.5e-10	0.768	0	0	\\
3e-10	0.768	0	0	\\
3.5e-10	0.768	0	0	\\
4e-10	0.768	0	0	\\
4.5e-10	0.768	0	0	\\
5e-10	0.768	0	0	\\
5.5e-10	0.768	0	0	\\
6e-10	0.768	0	0	\\
6.5e-10	0.768	0	0	\\
7e-10	0.768	0	0	\\
7.5e-10	0.768	0	0	\\
8e-10	0.768	0.01	0.01	\\
8.5e-10	0.768	0.01	0.01	\\
9e-10	0.768	0.01	0.01	\\
9.5e-10	0.768	0.01	0.01	\\
1e-09	0.768	0.01	0.01	\\
1.05e-09	0.768	0.01	0.01	\\
1.1e-09	0.768	0.01	0.01	\\
1.15e-09	0.768	0.01	0.01	\\
1.2e-09	0.768	0.01	0.01	\\
1.25e-09	0.768	0.01	0.01	\\
1.3e-09	0.768	0	0	\\
1.35e-09	0.768	0	0	\\
1.4e-09	0.768	0	0	\\
1.45e-09	0.768	0	0	\\
1.5e-09	0.768	0	0	\\
1.55e-09	0.768	0	0	\\
1.6e-09	0.768	0	0	\\
1.65e-09	0.768	-0.00666666666666667	-0.00666666666666667	\\
1.7e-09	0.768	-0.00666666666666667	-0.00666666666666667	\\
1.75e-09	0.768	-0.00666666666666667	-0.00666666666666667	\\
1.8e-09	0.768	-0.00666666666666667	-0.00666666666666667	\\
1.85e-09	0.768	-0.00666666666666667	-0.00666666666666667	\\
1.9e-09	0.768	-0.00666666666666667	-0.00666666666666667	\\
1.95e-09	0.768	-0.00666666666666667	-0.00666666666666667	\\
2e-09	0.768	-0.00666666666666667	-0.00666666666666667	\\
2.05e-09	0.768	-0.00666666666666667	-0.00666666666666667	\\
2.1e-09	0.768	-0.00666666666666667	-0.00666666666666667	\\
2.15e-09	0.768	0	0	\\
2.2e-09	0.768	0	0	\\
2.25e-09	0.768	0	0	\\
2.3e-09	0.768	0	0	\\
2.35e-09	0.768	0	0	\\
2.4e-09	0.768	0	0	\\
2.45e-09	0.768	0	0	\\
2.5e-09	0.768	0	0	\\
2.55e-09	0.768	0	0	\\
2.6e-09	0.768	0	0	\\
2.65e-09	0.768	0	0	\\
2.7e-09	0.768	0	0	\\
2.75e-09	0.768	0	0	\\
2.8e-09	0.768	0	0	\\
2.85e-09	0.768	0	0	\\
2.9e-09	0.768	0	0	\\
2.95e-09	0.768	0	0	\\
3e-09	0.768	0	0	\\
3.05e-09	0.768	0	0	\\
3.1e-09	0.768	0	0	\\
3.15e-09	0.768	0	0	\\
3.2e-09	0.768	0.00666666666666667	0.00666666666666667	\\
3.25e-09	0.768	0.00666666666666667	0.00666666666666667	\\
3.3e-09	0.768	0.00666666666666667	0.00666666666666667	\\
3.35e-09	0.768	0.00666666666666667	0.00666666666666667	\\
3.4e-09	0.768	0.00666666666666667	0.00666666666666667	\\
3.45e-09	0.768	0.00666666666666667	0.00666666666666667	\\
3.5e-09	0.768	0.00666666666666667	0.00666666666666667	\\
3.55e-09	0.768	0.00666666666666667	0.00666666666666667	\\
3.6e-09	0.768	0.00666666666666667	0.00666666666666667	\\
3.65e-09	0.768	0.00666666666666667	0.00666666666666667	\\
3.7e-09	0.768	0	0	\\
3.75e-09	0.768	0	0	\\
3.8e-09	0.768	0	0	\\
3.85e-09	0.768	0	0	\\
3.9e-09	0.768	0	0	\\
3.95e-09	0.768	0	0	\\
4e-09	0.768	0	0	\\
4.05e-09	0.768	-0.00444444444444444	-0.00444444444444444	\\
4.1e-09	0.768	-0.00444444444444444	-0.00444444444444444	\\
4.15e-09	0.768	-0.00444444444444444	-0.00444444444444444	\\
4.2e-09	0.768	-0.00444444444444444	-0.00444444444444444	\\
4.25e-09	0.768	-0.00444444444444444	-0.00444444444444444	\\
4.3e-09	0.768	-0.00444444444444444	-0.00444444444444444	\\
4.35e-09	0.768	-0.00444444444444444	-0.00444444444444444	\\
4.4e-09	0.768	-0.00444444444444444	-0.00444444444444444	\\
4.45e-09	0.768	-0.00444444444444444	-0.00444444444444444	\\
4.5e-09	0.768	-0.00444444444444444	-0.00444444444444444	\\
4.55e-09	0.768	0	0	\\
4.6e-09	0.768	0	0	\\
4.65e-09	0.768	0	0	\\
4.7e-09	0.768	0	0	\\
4.75e-09	0.768	0	0	\\
4.8e-09	0.768	0	0	\\
4.85e-09	0.768	0	0	\\
4.9e-09	0.768	0	0	\\
4.95e-09	0.768	0	0	\\
5e-09	0.768	0	0	\\
5e-09	0.768	0	nan	\\
5e-09	0.768	-0.00666666666666667	0.00166666666666667	\\
5e-09	0.768	-0.00666666666666667	nan	\\
0	0.78	-0.00666666666666667	nan	\\
0	0.78	0	0.00166666666666667	\\
0	0.78	0	0	\\
5e-11	0.78	0	0	\\
1e-10	0.78	0	0	\\
1.5e-10	0.78	0	0	\\
2e-10	0.78	0	0	\\
2.5e-10	0.78	0	0	\\
3e-10	0.78	0	0	\\
3.5e-10	0.78	0	0	\\
4e-10	0.78	0	0	\\
4.5e-10	0.78	0	0	\\
5e-10	0.78	0	0	\\
5.5e-10	0.78	0	0	\\
6e-10	0.78	0	0	\\
6.5e-10	0.78	0	0	\\
7e-10	0.78	0	0	\\
7.5e-10	0.78	0	0	\\
8e-10	0.78	0.01	0.01	\\
8.5e-10	0.78	0.01	0.01	\\
9e-10	0.78	0.01	0.01	\\
9.5e-10	0.78	0.01	0.01	\\
1e-09	0.78	0.01	0.01	\\
1.05e-09	0.78	0.01	0.01	\\
1.1e-09	0.78	0.01	0.01	\\
1.15e-09	0.78	0.01	0.01	\\
1.2e-09	0.78	0.01	0.01	\\
1.25e-09	0.78	0.01	0.01	\\
1.3e-09	0.78	0	0	\\
1.35e-09	0.78	0	0	\\
1.4e-09	0.78	0	0	\\
1.45e-09	0.78	0	0	\\
1.5e-09	0.78	0	0	\\
1.55e-09	0.78	0	0	\\
1.6e-09	0.78	0	0	\\
1.65e-09	0.78	-0.00666666666666667	-0.00666666666666667	\\
1.7e-09	0.78	-0.00666666666666667	-0.00666666666666667	\\
1.75e-09	0.78	-0.00666666666666667	-0.00666666666666667	\\
1.8e-09	0.78	-0.00666666666666667	-0.00666666666666667	\\
1.85e-09	0.78	-0.00666666666666667	-0.00666666666666667	\\
1.9e-09	0.78	-0.00666666666666667	-0.00666666666666667	\\
1.95e-09	0.78	-0.00666666666666667	-0.00666666666666667	\\
2e-09	0.78	-0.00666666666666667	-0.00666666666666667	\\
2.05e-09	0.78	-0.00666666666666667	-0.00666666666666667	\\
2.1e-09	0.78	-0.00666666666666667	-0.00666666666666667	\\
2.15e-09	0.78	0	0	\\
2.2e-09	0.78	0	0	\\
2.25e-09	0.78	0	0	\\
2.3e-09	0.78	0	0	\\
2.35e-09	0.78	0	0	\\
2.4e-09	0.78	0	0	\\
2.45e-09	0.78	0	0	\\
2.5e-09	0.78	0	0	\\
2.55e-09	0.78	0	0	\\
2.6e-09	0.78	0	0	\\
2.65e-09	0.78	0	0	\\
2.7e-09	0.78	0	0	\\
2.75e-09	0.78	0	0	\\
2.8e-09	0.78	0	0	\\
2.85e-09	0.78	0	0	\\
2.9e-09	0.78	0	0	\\
2.95e-09	0.78	0	0	\\
3e-09	0.78	0	0	\\
3.05e-09	0.78	0	0	\\
3.1e-09	0.78	0	0	\\
3.15e-09	0.78	0	0	\\
3.2e-09	0.78	0.00666666666666667	0.00666666666666667	\\
3.25e-09	0.78	0.00666666666666667	0.00666666666666667	\\
3.3e-09	0.78	0.00666666666666667	0.00666666666666667	\\
3.35e-09	0.78	0.00666666666666667	0.00666666666666667	\\
3.4e-09	0.78	0.00666666666666667	0.00666666666666667	\\
3.45e-09	0.78	0.00666666666666667	0.00666666666666667	\\
3.5e-09	0.78	0.00666666666666667	0.00666666666666667	\\
3.55e-09	0.78	0.00666666666666667	0.00666666666666667	\\
3.6e-09	0.78	0.00666666666666667	0.00666666666666667	\\
3.65e-09	0.78	0.00666666666666667	0.00666666666666667	\\
3.7e-09	0.78	0	0	\\
3.75e-09	0.78	0	0	\\
3.8e-09	0.78	0	0	\\
3.85e-09	0.78	0	0	\\
3.9e-09	0.78	0	0	\\
3.95e-09	0.78	0	0	\\
4e-09	0.78	0	0	\\
4.05e-09	0.78	-0.00444444444444444	-0.00444444444444444	\\
4.1e-09	0.78	-0.00444444444444444	-0.00444444444444444	\\
4.15e-09	0.78	-0.00444444444444444	-0.00444444444444444	\\
4.2e-09	0.78	-0.00444444444444444	-0.00444444444444444	\\
4.25e-09	0.78	-0.00444444444444444	-0.00444444444444444	\\
4.3e-09	0.78	-0.00444444444444444	-0.00444444444444444	\\
4.35e-09	0.78	-0.00444444444444444	-0.00444444444444444	\\
4.4e-09	0.78	-0.00444444444444444	-0.00444444444444444	\\
4.45e-09	0.78	-0.00444444444444444	-0.00444444444444444	\\
4.5e-09	0.78	-0.00444444444444444	-0.00444444444444444	\\
4.55e-09	0.78	0	0	\\
4.6e-09	0.78	0	0	\\
4.65e-09	0.78	0	0	\\
4.7e-09	0.78	0	0	\\
4.75e-09	0.78	0	0	\\
4.8e-09	0.78	0	0	\\
4.85e-09	0.78	0	0	\\
4.9e-09	0.78	0	0	\\
4.95e-09	0.78	0	0	\\
5e-09	0.78	0	0	\\
5e-09	0.78	0	nan	\\
5e-09	0.78	-0.00666666666666667	0.00166666666666667	\\
5e-09	0.78	-0.00666666666666667	nan	\\
0	0.792	-0.00666666666666667	nan	\\
0	0.792	0	0.00166666666666667	\\
0	0.792	0	0	\\
5e-11	0.792	0	0	\\
1e-10	0.792	0	0	\\
1.5e-10	0.792	0	0	\\
2e-10	0.792	0	0	\\
2.5e-10	0.792	0	0	\\
3e-10	0.792	0	0	\\
3.5e-10	0.792	0	0	\\
4e-10	0.792	0	0	\\
4.5e-10	0.792	0	0	\\
5e-10	0.792	0	0	\\
5.5e-10	0.792	0	0	\\
6e-10	0.792	0	0	\\
6.5e-10	0.792	0	0	\\
7e-10	0.792	0	0	\\
7.5e-10	0.792	0	0	\\
8e-10	0.792	0.01	0.01	\\
8.5e-10	0.792	0.01	0.01	\\
9e-10	0.792	0.01	0.01	\\
9.5e-10	0.792	0.01	0.01	\\
1e-09	0.792	0.01	0.01	\\
1.05e-09	0.792	0.01	0.01	\\
1.1e-09	0.792	0.01	0.01	\\
1.15e-09	0.792	0.01	0.01	\\
1.2e-09	0.792	0.01	0.01	\\
1.25e-09	0.792	0.01	0.01	\\
1.3e-09	0.792	0	0	\\
1.35e-09	0.792	0	0	\\
1.4e-09	0.792	0	0	\\
1.45e-09	0.792	0	0	\\
1.5e-09	0.792	0	0	\\
1.55e-09	0.792	0	0	\\
1.6e-09	0.792	0	0	\\
1.65e-09	0.792	-0.00666666666666667	-0.00666666666666667	\\
1.7e-09	0.792	-0.00666666666666667	-0.00666666666666667	\\
1.75e-09	0.792	-0.00666666666666667	-0.00666666666666667	\\
1.8e-09	0.792	-0.00666666666666667	-0.00666666666666667	\\
1.85e-09	0.792	-0.00666666666666667	-0.00666666666666667	\\
1.9e-09	0.792	-0.00666666666666667	-0.00666666666666667	\\
1.95e-09	0.792	-0.00666666666666667	-0.00666666666666667	\\
2e-09	0.792	-0.00666666666666667	-0.00666666666666667	\\
2.05e-09	0.792	-0.00666666666666667	-0.00666666666666667	\\
2.1e-09	0.792	-0.00666666666666667	-0.00666666666666667	\\
2.15e-09	0.792	0	0	\\
2.2e-09	0.792	0	0	\\
2.25e-09	0.792	0	0	\\
2.3e-09	0.792	0	0	\\
2.35e-09	0.792	0	0	\\
2.4e-09	0.792	0	0	\\
2.45e-09	0.792	0	0	\\
2.5e-09	0.792	0	0	\\
2.55e-09	0.792	0	0	\\
2.6e-09	0.792	0	0	\\
2.65e-09	0.792	0	0	\\
2.7e-09	0.792	0	0	\\
2.75e-09	0.792	0	0	\\
2.8e-09	0.792	0	0	\\
2.85e-09	0.792	0	0	\\
2.9e-09	0.792	0	0	\\
2.95e-09	0.792	0	0	\\
3e-09	0.792	0	0	\\
3.05e-09	0.792	0	0	\\
3.1e-09	0.792	0	0	\\
3.15e-09	0.792	0	0	\\
3.2e-09	0.792	0.00666666666666667	0.00666666666666667	\\
3.25e-09	0.792	0.00666666666666667	0.00666666666666667	\\
3.3e-09	0.792	0.00666666666666667	0.00666666666666667	\\
3.35e-09	0.792	0.00666666666666667	0.00666666666666667	\\
3.4e-09	0.792	0.00666666666666667	0.00666666666666667	\\
3.45e-09	0.792	0.00666666666666667	0.00666666666666667	\\
3.5e-09	0.792	0.00666666666666667	0.00666666666666667	\\
3.55e-09	0.792	0.00666666666666667	0.00666666666666667	\\
3.6e-09	0.792	0.00666666666666667	0.00666666666666667	\\
3.65e-09	0.792	0.00666666666666667	0.00666666666666667	\\
3.7e-09	0.792	0	0	\\
3.75e-09	0.792	0	0	\\
3.8e-09	0.792	0	0	\\
3.85e-09	0.792	0	0	\\
3.9e-09	0.792	0	0	\\
3.95e-09	0.792	0	0	\\
4e-09	0.792	0	0	\\
4.05e-09	0.792	-0.00444444444444444	-0.00444444444444444	\\
4.1e-09	0.792	-0.00444444444444444	-0.00444444444444444	\\
4.15e-09	0.792	-0.00444444444444444	-0.00444444444444444	\\
4.2e-09	0.792	-0.00444444444444444	-0.00444444444444444	\\
4.25e-09	0.792	-0.00444444444444444	-0.00444444444444444	\\
4.3e-09	0.792	-0.00444444444444444	-0.00444444444444444	\\
4.35e-09	0.792	-0.00444444444444444	-0.00444444444444444	\\
4.4e-09	0.792	-0.00444444444444444	-0.00444444444444444	\\
4.45e-09	0.792	-0.00444444444444444	-0.00444444444444444	\\
4.5e-09	0.792	-0.00444444444444444	-0.00444444444444444	\\
4.55e-09	0.792	0	0	\\
4.6e-09	0.792	0	0	\\
4.65e-09	0.792	0	0	\\
4.7e-09	0.792	0	0	\\
4.75e-09	0.792	0	0	\\
4.8e-09	0.792	0	0	\\
4.85e-09	0.792	0	0	\\
4.9e-09	0.792	0	0	\\
4.95e-09	0.792	0	0	\\
5e-09	0.792	0	0	\\
5e-09	0.792	0	nan	\\
5e-09	0.792	-0.00666666666666667	0.00166666666666667	\\
5e-09	0.792	-0.00666666666666667	nan	\\
0	0.804	-0.00666666666666667	nan	\\
0	0.804	0	0.00166666666666667	\\
0	0.804	0	0	\\
5e-11	0.804	0	0	\\
1e-10	0.804	0	0	\\
1.5e-10	0.804	0	0	\\
2e-10	0.804	0	0	\\
2.5e-10	0.804	0	0	\\
3e-10	0.804	0	0	\\
3.5e-10	0.804	0	0	\\
4e-10	0.804	0	0	\\
4.5e-10	0.804	0	0	\\
5e-10	0.804	0	0	\\
5.5e-10	0.804	0	0	\\
6e-10	0.804	0	0	\\
6.5e-10	0.804	0	0	\\
7e-10	0.804	0	0	\\
7.5e-10	0.804	0	0	\\
8e-10	0.804	0	0	\\
8.5e-10	0.804	0.01	0.01	\\
9e-10	0.804	0.01	0.01	\\
9.5e-10	0.804	0.01	0.01	\\
1e-09	0.804	0.01	0.01	\\
1.05e-09	0.804	0.01	0.01	\\
1.1e-09	0.804	0.01	0.01	\\
1.15e-09	0.804	0.01	0.01	\\
1.2e-09	0.804	0.01	0.01	\\
1.25e-09	0.804	0.01	0.01	\\
1.3e-09	0.804	0.01	0.01	\\
1.35e-09	0.804	0	0	\\
1.4e-09	0.804	0	0	\\
1.45e-09	0.804	0	0	\\
1.5e-09	0.804	0	0	\\
1.55e-09	0.804	0	0	\\
1.6e-09	0.804	-0.00666666666666667	-0.00666666666666667	\\
1.65e-09	0.804	-0.00666666666666667	-0.00666666666666667	\\
1.7e-09	0.804	-0.00666666666666667	-0.00666666666666667	\\
1.75e-09	0.804	-0.00666666666666667	-0.00666666666666667	\\
1.8e-09	0.804	-0.00666666666666667	-0.00666666666666667	\\
1.85e-09	0.804	-0.00666666666666667	-0.00666666666666667	\\
1.9e-09	0.804	-0.00666666666666667	-0.00666666666666667	\\
1.95e-09	0.804	-0.00666666666666667	-0.00666666666666667	\\
2e-09	0.804	-0.00666666666666667	-0.00666666666666667	\\
2.05e-09	0.804	-0.00666666666666667	-0.00666666666666667	\\
2.1e-09	0.804	0	0	\\
2.15e-09	0.804	0	0	\\
2.2e-09	0.804	0	0	\\
2.25e-09	0.804	0	0	\\
2.3e-09	0.804	0	0	\\
2.35e-09	0.804	0	0	\\
2.4e-09	0.804	0	0	\\
2.45e-09	0.804	0	0	\\
2.5e-09	0.804	0	0	\\
2.55e-09	0.804	0	0	\\
2.6e-09	0.804	0	0	\\
2.65e-09	0.804	0	0	\\
2.7e-09	0.804	0	0	\\
2.75e-09	0.804	0	0	\\
2.8e-09	0.804	0	0	\\
2.85e-09	0.804	0	0	\\
2.9e-09	0.804	0	0	\\
2.95e-09	0.804	0	0	\\
3e-09	0.804	0	0	\\
3.05e-09	0.804	0	0	\\
3.1e-09	0.804	0	0	\\
3.15e-09	0.804	0	0	\\
3.2e-09	0.804	0	0	\\
3.25e-09	0.804	0.00666666666666667	0.00666666666666667	\\
3.3e-09	0.804	0.00666666666666667	0.00666666666666667	\\
3.35e-09	0.804	0.00666666666666667	0.00666666666666667	\\
3.4e-09	0.804	0.00666666666666667	0.00666666666666667	\\
3.45e-09	0.804	0.00666666666666667	0.00666666666666667	\\
3.5e-09	0.804	0.00666666666666667	0.00666666666666667	\\
3.55e-09	0.804	0.00666666666666667	0.00666666666666667	\\
3.6e-09	0.804	0.00666666666666667	0.00666666666666667	\\
3.65e-09	0.804	0.00666666666666667	0.00666666666666667	\\
3.7e-09	0.804	0.00666666666666667	0.00666666666666667	\\
3.75e-09	0.804	0	0	\\
3.8e-09	0.804	0	0	\\
3.85e-09	0.804	0	0	\\
3.9e-09	0.804	0	0	\\
3.95e-09	0.804	0	0	\\
4e-09	0.804	-0.00444444444444444	-0.00444444444444444	\\
4.05e-09	0.804	-0.00444444444444444	-0.00444444444444444	\\
4.1e-09	0.804	-0.00444444444444444	-0.00444444444444444	\\
4.15e-09	0.804	-0.00444444444444444	-0.00444444444444444	\\
4.2e-09	0.804	-0.00444444444444444	-0.00444444444444444	\\
4.25e-09	0.804	-0.00444444444444444	-0.00444444444444444	\\
4.3e-09	0.804	-0.00444444444444444	-0.00444444444444444	\\
4.35e-09	0.804	-0.00444444444444444	-0.00444444444444444	\\
4.4e-09	0.804	-0.00444444444444444	-0.00444444444444444	\\
4.45e-09	0.804	-0.00444444444444444	-0.00444444444444444	\\
4.5e-09	0.804	0	0	\\
4.55e-09	0.804	0	0	\\
4.6e-09	0.804	0	0	\\
4.65e-09	0.804	0	0	\\
4.7e-09	0.804	0	0	\\
4.75e-09	0.804	0	0	\\
4.8e-09	0.804	0	0	\\
4.85e-09	0.804	0	0	\\
4.9e-09	0.804	0	0	\\
4.95e-09	0.804	0	0	\\
5e-09	0.804	0	0	\\
5e-09	0.804	0	nan	\\
5e-09	0.804	-0.00666666666666667	0.00166666666666667	\\
5e-09	0.804	-0.00666666666666667	nan	\\
0	0.816	-0.00666666666666667	nan	\\
0	0.816	0	0.00166666666666667	\\
0	0.816	0	0	\\
5e-11	0.816	0	0	\\
1e-10	0.816	0	0	\\
1.5e-10	0.816	0	0	\\
2e-10	0.816	0	0	\\
2.5e-10	0.816	0	0	\\
3e-10	0.816	0	0	\\
3.5e-10	0.816	0	0	\\
4e-10	0.816	0	0	\\
4.5e-10	0.816	0	0	\\
5e-10	0.816	0	0	\\
5.5e-10	0.816	0	0	\\
6e-10	0.816	0	0	\\
6.5e-10	0.816	0	0	\\
7e-10	0.816	0	0	\\
7.5e-10	0.816	0	0	\\
8e-10	0.816	0	0	\\
8.5e-10	0.816	0.01	0.01	\\
9e-10	0.816	0.01	0.01	\\
9.5e-10	0.816	0.01	0.01	\\
1e-09	0.816	0.01	0.01	\\
1.05e-09	0.816	0.01	0.01	\\
1.1e-09	0.816	0.01	0.01	\\
1.15e-09	0.816	0.01	0.01	\\
1.2e-09	0.816	0.01	0.01	\\
1.25e-09	0.816	0.01	0.01	\\
1.3e-09	0.816	0.01	0.01	\\
1.35e-09	0.816	0	0	\\
1.4e-09	0.816	0	0	\\
1.45e-09	0.816	0	0	\\
1.5e-09	0.816	0	0	\\
1.55e-09	0.816	0	0	\\
1.6e-09	0.816	-0.00666666666666667	-0.00666666666666667	\\
1.65e-09	0.816	-0.00666666666666667	-0.00666666666666667	\\
1.7e-09	0.816	-0.00666666666666667	-0.00666666666666667	\\
1.75e-09	0.816	-0.00666666666666667	-0.00666666666666667	\\
1.8e-09	0.816	-0.00666666666666667	-0.00666666666666667	\\
1.85e-09	0.816	-0.00666666666666667	-0.00666666666666667	\\
1.9e-09	0.816	-0.00666666666666667	-0.00666666666666667	\\
1.95e-09	0.816	-0.00666666666666667	-0.00666666666666667	\\
2e-09	0.816	-0.00666666666666667	-0.00666666666666667	\\
2.05e-09	0.816	-0.00666666666666667	-0.00666666666666667	\\
2.1e-09	0.816	0	0	\\
2.15e-09	0.816	0	0	\\
2.2e-09	0.816	0	0	\\
2.25e-09	0.816	0	0	\\
2.3e-09	0.816	0	0	\\
2.35e-09	0.816	0	0	\\
2.4e-09	0.816	0	0	\\
2.45e-09	0.816	0	0	\\
2.5e-09	0.816	0	0	\\
2.55e-09	0.816	0	0	\\
2.6e-09	0.816	0	0	\\
2.65e-09	0.816	0	0	\\
2.7e-09	0.816	0	0	\\
2.75e-09	0.816	0	0	\\
2.8e-09	0.816	0	0	\\
2.85e-09	0.816	0	0	\\
2.9e-09	0.816	0	0	\\
2.95e-09	0.816	0	0	\\
3e-09	0.816	0	0	\\
3.05e-09	0.816	0	0	\\
3.1e-09	0.816	0	0	\\
3.15e-09	0.816	0	0	\\
3.2e-09	0.816	0	0	\\
3.25e-09	0.816	0.00666666666666667	0.00666666666666667	\\
3.3e-09	0.816	0.00666666666666667	0.00666666666666667	\\
3.35e-09	0.816	0.00666666666666667	0.00666666666666667	\\
3.4e-09	0.816	0.00666666666666667	0.00666666666666667	\\
3.45e-09	0.816	0.00666666666666667	0.00666666666666667	\\
3.5e-09	0.816	0.00666666666666667	0.00666666666666667	\\
3.55e-09	0.816	0.00666666666666667	0.00666666666666667	\\
3.6e-09	0.816	0.00666666666666667	0.00666666666666667	\\
3.65e-09	0.816	0.00666666666666667	0.00666666666666667	\\
3.7e-09	0.816	0.00666666666666667	0.00666666666666667	\\
3.75e-09	0.816	0	0	\\
3.8e-09	0.816	0	0	\\
3.85e-09	0.816	0	0	\\
3.9e-09	0.816	0	0	\\
3.95e-09	0.816	0	0	\\
4e-09	0.816	-0.00444444444444444	-0.00444444444444444	\\
4.05e-09	0.816	-0.00444444444444444	-0.00444444444444444	\\
4.1e-09	0.816	-0.00444444444444444	-0.00444444444444444	\\
4.15e-09	0.816	-0.00444444444444444	-0.00444444444444444	\\
4.2e-09	0.816	-0.00444444444444444	-0.00444444444444444	\\
4.25e-09	0.816	-0.00444444444444444	-0.00444444444444444	\\
4.3e-09	0.816	-0.00444444444444444	-0.00444444444444444	\\
4.35e-09	0.816	-0.00444444444444444	-0.00444444444444444	\\
4.4e-09	0.816	-0.00444444444444444	-0.00444444444444444	\\
4.45e-09	0.816	-0.00444444444444444	-0.00444444444444444	\\
4.5e-09	0.816	0	0	\\
4.55e-09	0.816	0	0	\\
4.6e-09	0.816	0	0	\\
4.65e-09	0.816	0	0	\\
4.7e-09	0.816	0	0	\\
4.75e-09	0.816	0	0	\\
4.8e-09	0.816	0	0	\\
4.85e-09	0.816	0	0	\\
4.9e-09	0.816	0	0	\\
4.95e-09	0.816	0	0	\\
5e-09	0.816	0	0	\\
5e-09	0.816	0	nan	\\
5e-09	0.816	-0.00666666666666667	0.00166666666666667	\\
5e-09	0.816	-0.00666666666666667	nan	\\
0	0.828	-0.00666666666666667	nan	\\
0	0.828	0	0.00166666666666667	\\
0	0.828	0	0	\\
5e-11	0.828	0	0	\\
1e-10	0.828	0	0	\\
1.5e-10	0.828	0	0	\\
2e-10	0.828	0	0	\\
2.5e-10	0.828	0	0	\\
3e-10	0.828	0	0	\\
3.5e-10	0.828	0	0	\\
4e-10	0.828	0	0	\\
4.5e-10	0.828	0	0	\\
5e-10	0.828	0	0	\\
5.5e-10	0.828	0	0	\\
6e-10	0.828	0	0	\\
6.5e-10	0.828	0	0	\\
7e-10	0.828	0	0	\\
7.5e-10	0.828	0	0	\\
8e-10	0.828	0	0	\\
8.5e-10	0.828	0.01	0.01	\\
9e-10	0.828	0.01	0.01	\\
9.5e-10	0.828	0.01	0.01	\\
1e-09	0.828	0.01	0.01	\\
1.05e-09	0.828	0.01	0.01	\\
1.1e-09	0.828	0.01	0.01	\\
1.15e-09	0.828	0.01	0.01	\\
1.2e-09	0.828	0.01	0.01	\\
1.25e-09	0.828	0.01	0.01	\\
1.3e-09	0.828	0.01	0.01	\\
1.35e-09	0.828	0	0	\\
1.4e-09	0.828	0	0	\\
1.45e-09	0.828	0	0	\\
1.5e-09	0.828	0	0	\\
1.55e-09	0.828	0	0	\\
1.6e-09	0.828	-0.00666666666666667	-0.00666666666666667	\\
1.65e-09	0.828	-0.00666666666666667	-0.00666666666666667	\\
1.7e-09	0.828	-0.00666666666666667	-0.00666666666666667	\\
1.75e-09	0.828	-0.00666666666666667	-0.00666666666666667	\\
1.8e-09	0.828	-0.00666666666666667	-0.00666666666666667	\\
1.85e-09	0.828	-0.00666666666666667	-0.00666666666666667	\\
1.9e-09	0.828	-0.00666666666666667	-0.00666666666666667	\\
1.95e-09	0.828	-0.00666666666666667	-0.00666666666666667	\\
2e-09	0.828	-0.00666666666666667	-0.00666666666666667	\\
2.05e-09	0.828	-0.00666666666666667	-0.00666666666666667	\\
2.1e-09	0.828	0	0	\\
2.15e-09	0.828	0	0	\\
2.2e-09	0.828	0	0	\\
2.25e-09	0.828	0	0	\\
2.3e-09	0.828	0	0	\\
2.35e-09	0.828	0	0	\\
2.4e-09	0.828	0	0	\\
2.45e-09	0.828	0	0	\\
2.5e-09	0.828	0	0	\\
2.55e-09	0.828	0	0	\\
2.6e-09	0.828	0	0	\\
2.65e-09	0.828	0	0	\\
2.7e-09	0.828	0	0	\\
2.75e-09	0.828	0	0	\\
2.8e-09	0.828	0	0	\\
2.85e-09	0.828	0	0	\\
2.9e-09	0.828	0	0	\\
2.95e-09	0.828	0	0	\\
3e-09	0.828	0	0	\\
3.05e-09	0.828	0	0	\\
3.1e-09	0.828	0	0	\\
3.15e-09	0.828	0	0	\\
3.2e-09	0.828	0	0	\\
3.25e-09	0.828	0.00666666666666667	0.00666666666666667	\\
3.3e-09	0.828	0.00666666666666667	0.00666666666666667	\\
3.35e-09	0.828	0.00666666666666667	0.00666666666666667	\\
3.4e-09	0.828	0.00666666666666667	0.00666666666666667	\\
3.45e-09	0.828	0.00666666666666667	0.00666666666666667	\\
3.5e-09	0.828	0.00666666666666667	0.00666666666666667	\\
3.55e-09	0.828	0.00666666666666667	0.00666666666666667	\\
3.6e-09	0.828	0.00666666666666667	0.00666666666666667	\\
3.65e-09	0.828	0.00666666666666667	0.00666666666666667	\\
3.7e-09	0.828	0.00666666666666667	0.00666666666666667	\\
3.75e-09	0.828	0	0	\\
3.8e-09	0.828	0	0	\\
3.85e-09	0.828	0	0	\\
3.9e-09	0.828	0	0	\\
3.95e-09	0.828	0	0	\\
4e-09	0.828	-0.00444444444444444	-0.00444444444444444	\\
4.05e-09	0.828	-0.00444444444444444	-0.00444444444444444	\\
4.1e-09	0.828	-0.00444444444444444	-0.00444444444444444	\\
4.15e-09	0.828	-0.00444444444444444	-0.00444444444444444	\\
4.2e-09	0.828	-0.00444444444444444	-0.00444444444444444	\\
4.25e-09	0.828	-0.00444444444444444	-0.00444444444444444	\\
4.3e-09	0.828	-0.00444444444444444	-0.00444444444444444	\\
4.35e-09	0.828	-0.00444444444444444	-0.00444444444444444	\\
4.4e-09	0.828	-0.00444444444444444	-0.00444444444444444	\\
4.45e-09	0.828	-0.00444444444444444	-0.00444444444444444	\\
4.5e-09	0.828	0	0	\\
4.55e-09	0.828	0	0	\\
4.6e-09	0.828	0	0	\\
4.65e-09	0.828	0	0	\\
4.7e-09	0.828	0	0	\\
4.75e-09	0.828	0	0	\\
4.8e-09	0.828	0	0	\\
4.85e-09	0.828	0	0	\\
4.9e-09	0.828	0	0	\\
4.95e-09	0.828	0	0	\\
5e-09	0.828	0	0	\\
5e-09	0.828	0	nan	\\
5e-09	0.828	-0.00666666666666667	0.00166666666666667	\\
5e-09	0.828	-0.00666666666666667	nan	\\
0	0.84	-0.00666666666666667	nan	\\
0	0.84	0	0.00166666666666667	\\
0	0.84	0	0	\\
5e-11	0.84	0	0	\\
1e-10	0.84	0	0	\\
1.5e-10	0.84	0	0	\\
2e-10	0.84	0	0	\\
2.5e-10	0.84	0	0	\\
3e-10	0.84	0	0	\\
3.5e-10	0.84	0	0	\\
4e-10	0.84	0	0	\\
4.5e-10	0.84	0	0	\\
5e-10	0.84	0	0	\\
5.5e-10	0.84	0	0	\\
6e-10	0.84	0	0	\\
6.5e-10	0.84	0	0	\\
7e-10	0.84	0	0	\\
7.5e-10	0.84	0	0	\\
8e-10	0.84	0	0	\\
8.5e-10	0.84	0.01	0.01	\\
9e-10	0.84	0.01	0.01	\\
9.5e-10	0.84	0.01	0.01	\\
1e-09	0.84	0.01	0.01	\\
1.05e-09	0.84	0.01	0.01	\\
1.1e-09	0.84	0.01	0.01	\\
1.15e-09	0.84	0.01	0.01	\\
1.2e-09	0.84	0.01	0.01	\\
1.25e-09	0.84	0.01	0.01	\\
1.3e-09	0.84	0.01	0.01	\\
1.35e-09	0.84	0	0	\\
1.4e-09	0.84	0	0	\\
1.45e-09	0.84	0	0	\\
1.5e-09	0.84	0	0	\\
1.55e-09	0.84	0	0	\\
1.6e-09	0.84	-0.00666666666666667	-0.00666666666666667	\\
1.65e-09	0.84	-0.00666666666666667	-0.00666666666666667	\\
1.7e-09	0.84	-0.00666666666666667	-0.00666666666666667	\\
1.75e-09	0.84	-0.00666666666666667	-0.00666666666666667	\\
1.8e-09	0.84	-0.00666666666666667	-0.00666666666666667	\\
1.85e-09	0.84	-0.00666666666666667	-0.00666666666666667	\\
1.9e-09	0.84	-0.00666666666666667	-0.00666666666666667	\\
1.95e-09	0.84	-0.00666666666666667	-0.00666666666666667	\\
2e-09	0.84	-0.00666666666666667	-0.00666666666666667	\\
2.05e-09	0.84	-0.00666666666666667	-0.00666666666666667	\\
2.1e-09	0.84	0	0	\\
2.15e-09	0.84	0	0	\\
2.2e-09	0.84	0	0	\\
2.25e-09	0.84	0	0	\\
2.3e-09	0.84	0	0	\\
2.35e-09	0.84	0	0	\\
2.4e-09	0.84	0	0	\\
2.45e-09	0.84	0	0	\\
2.5e-09	0.84	0	0	\\
2.55e-09	0.84	0	0	\\
2.6e-09	0.84	0	0	\\
2.65e-09	0.84	0	0	\\
2.7e-09	0.84	0	0	\\
2.75e-09	0.84	0	0	\\
2.8e-09	0.84	0	0	\\
2.85e-09	0.84	0	0	\\
2.9e-09	0.84	0	0	\\
2.95e-09	0.84	0	0	\\
3e-09	0.84	0	0	\\
3.05e-09	0.84	0	0	\\
3.1e-09	0.84	0	0	\\
3.15e-09	0.84	0	0	\\
3.2e-09	0.84	0	0	\\
3.25e-09	0.84	0.00666666666666667	0.00666666666666667	\\
3.3e-09	0.84	0.00666666666666667	0.00666666666666667	\\
3.35e-09	0.84	0.00666666666666667	0.00666666666666667	\\
3.4e-09	0.84	0.00666666666666667	0.00666666666666667	\\
3.45e-09	0.84	0.00666666666666667	0.00666666666666667	\\
3.5e-09	0.84	0.00666666666666667	0.00666666666666667	\\
3.55e-09	0.84	0.00666666666666667	0.00666666666666667	\\
3.6e-09	0.84	0.00666666666666667	0.00666666666666667	\\
3.65e-09	0.84	0.00666666666666667	0.00666666666666667	\\
3.7e-09	0.84	0.00666666666666667	0.00666666666666667	\\
3.75e-09	0.84	0	0	\\
3.8e-09	0.84	0	0	\\
3.85e-09	0.84	0	0	\\
3.9e-09	0.84	0	0	\\
3.95e-09	0.84	0	0	\\
4e-09	0.84	-0.00444444444444444	-0.00444444444444444	\\
4.05e-09	0.84	-0.00444444444444444	-0.00444444444444444	\\
4.1e-09	0.84	-0.00444444444444444	-0.00444444444444444	\\
4.15e-09	0.84	-0.00444444444444444	-0.00444444444444444	\\
4.2e-09	0.84	-0.00444444444444444	-0.00444444444444444	\\
4.25e-09	0.84	-0.00444444444444444	-0.00444444444444444	\\
4.3e-09	0.84	-0.00444444444444444	-0.00444444444444444	\\
4.35e-09	0.84	-0.00444444444444444	-0.00444444444444444	\\
4.4e-09	0.84	-0.00444444444444444	-0.00444444444444444	\\
4.45e-09	0.84	-0.00444444444444444	-0.00444444444444444	\\
4.5e-09	0.84	0	0	\\
4.55e-09	0.84	0	0	\\
4.6e-09	0.84	0	0	\\
4.65e-09	0.84	0	0	\\
4.7e-09	0.84	0	0	\\
4.75e-09	0.84	0	0	\\
4.8e-09	0.84	0	0	\\
4.85e-09	0.84	0	0	\\
4.9e-09	0.84	0	0	\\
4.95e-09	0.84	0	0	\\
5e-09	0.84	0	0	\\
5e-09	0.84	0	nan	\\
5e-09	0.84	-0.00666666666666667	0.00166666666666667	\\
5e-09	0.84	-0.00666666666666667	nan	\\
0	0.852	-0.00666666666666667	nan	\\
0	0.852	0	0.00166666666666667	\\
0	0.852	0	0	\\
5e-11	0.852	0	0	\\
1e-10	0.852	0	0	\\
1.5e-10	0.852	0	0	\\
2e-10	0.852	0	0	\\
2.5e-10	0.852	0	0	\\
3e-10	0.852	0	0	\\
3.5e-10	0.852	0	0	\\
4e-10	0.852	0	0	\\
4.5e-10	0.852	0	0	\\
5e-10	0.852	0	0	\\
5.5e-10	0.852	0	0	\\
6e-10	0.852	0	0	\\
6.5e-10	0.852	0	0	\\
7e-10	0.852	0	0	\\
7.5e-10	0.852	0	0	\\
8e-10	0.852	0	0	\\
8.5e-10	0.852	0	0	\\
9e-10	0.852	0.01	0.01	\\
9.5e-10	0.852	0.01	0.01	\\
1e-09	0.852	0.01	0.01	\\
1.05e-09	0.852	0.01	0.01	\\
1.1e-09	0.852	0.01	0.01	\\
1.15e-09	0.852	0.01	0.01	\\
1.2e-09	0.852	0.01	0.01	\\
1.25e-09	0.852	0.01	0.01	\\
1.3e-09	0.852	0.01	0.01	\\
1.35e-09	0.852	0.01	0.01	\\
1.4e-09	0.852	0	0	\\
1.45e-09	0.852	0	0	\\
1.5e-09	0.852	0	0	\\
1.55e-09	0.852	-0.00666666666666667	-0.00666666666666667	\\
1.6e-09	0.852	-0.00666666666666667	-0.00666666666666667	\\
1.65e-09	0.852	-0.00666666666666667	-0.00666666666666667	\\
1.7e-09	0.852	-0.00666666666666667	-0.00666666666666667	\\
1.75e-09	0.852	-0.00666666666666667	-0.00666666666666667	\\
1.8e-09	0.852	-0.00666666666666667	-0.00666666666666667	\\
1.85e-09	0.852	-0.00666666666666667	-0.00666666666666667	\\
1.9e-09	0.852	-0.00666666666666667	-0.00666666666666667	\\
1.95e-09	0.852	-0.00666666666666667	-0.00666666666666667	\\
2e-09	0.852	-0.00666666666666667	-0.00666666666666667	\\
2.05e-09	0.852	0	0	\\
2.1e-09	0.852	0	0	\\
2.15e-09	0.852	0	0	\\
2.2e-09	0.852	0	0	\\
2.25e-09	0.852	0	0	\\
2.3e-09	0.852	0	0	\\
2.35e-09	0.852	0	0	\\
2.4e-09	0.852	0	0	\\
2.45e-09	0.852	0	0	\\
2.5e-09	0.852	0	0	\\
2.55e-09	0.852	0	0	\\
2.6e-09	0.852	0	0	\\
2.65e-09	0.852	0	0	\\
2.7e-09	0.852	0	0	\\
2.75e-09	0.852	0	0	\\
2.8e-09	0.852	0	0	\\
2.85e-09	0.852	0	0	\\
2.9e-09	0.852	0	0	\\
2.95e-09	0.852	0	0	\\
3e-09	0.852	0	0	\\
3.05e-09	0.852	0	0	\\
3.1e-09	0.852	0	0	\\
3.15e-09	0.852	0	0	\\
3.2e-09	0.852	0	0	\\
3.25e-09	0.852	0	0	\\
3.3e-09	0.852	0.00666666666666667	0.00666666666666667	\\
3.35e-09	0.852	0.00666666666666667	0.00666666666666667	\\
3.4e-09	0.852	0.00666666666666667	0.00666666666666667	\\
3.45e-09	0.852	0.00666666666666667	0.00666666666666667	\\
3.5e-09	0.852	0.00666666666666667	0.00666666666666667	\\
3.55e-09	0.852	0.00666666666666667	0.00666666666666667	\\
3.6e-09	0.852	0.00666666666666667	0.00666666666666667	\\
3.65e-09	0.852	0.00666666666666667	0.00666666666666667	\\
3.7e-09	0.852	0.00666666666666667	0.00666666666666667	\\
3.75e-09	0.852	0.00666666666666667	0.00666666666666667	\\
3.8e-09	0.852	0	0	\\
3.85e-09	0.852	0	0	\\
3.9e-09	0.852	0	0	\\
3.95e-09	0.852	-0.00444444444444444	-0.00444444444444444	\\
4e-09	0.852	-0.00444444444444444	-0.00444444444444444	\\
4.05e-09	0.852	-0.00444444444444444	-0.00444444444444444	\\
4.1e-09	0.852	-0.00444444444444444	-0.00444444444444444	\\
4.15e-09	0.852	-0.00444444444444444	-0.00444444444444444	\\
4.2e-09	0.852	-0.00444444444444444	-0.00444444444444444	\\
4.25e-09	0.852	-0.00444444444444444	-0.00444444444444444	\\
4.3e-09	0.852	-0.00444444444444444	-0.00444444444444444	\\
4.35e-09	0.852	-0.00444444444444444	-0.00444444444444444	\\
4.4e-09	0.852	-0.00444444444444444	-0.00444444444444444	\\
4.45e-09	0.852	0	0	\\
4.5e-09	0.852	0	0	\\
4.55e-09	0.852	0	0	\\
4.6e-09	0.852	0	0	\\
4.65e-09	0.852	0	0	\\
4.7e-09	0.852	0	0	\\
4.75e-09	0.852	0	0	\\
4.8e-09	0.852	0	0	\\
4.85e-09	0.852	0	0	\\
4.9e-09	0.852	0	0	\\
4.95e-09	0.852	0	0	\\
5e-09	0.852	0	0	\\
5e-09	0.852	0	nan	\\
5e-09	0.852	-0.00666666666666667	0.00166666666666667	\\
5e-09	0.852	-0.00666666666666667	nan	\\
0	0.864	-0.00666666666666667	nan	\\
0	0.864	0	0.00166666666666667	\\
0	0.864	0	0	\\
5e-11	0.864	0	0	\\
1e-10	0.864	0	0	\\
1.5e-10	0.864	0	0	\\
2e-10	0.864	0	0	\\
2.5e-10	0.864	0	0	\\
3e-10	0.864	0	0	\\
3.5e-10	0.864	0	0	\\
4e-10	0.864	0	0	\\
4.5e-10	0.864	0	0	\\
5e-10	0.864	0	0	\\
5.5e-10	0.864	0	0	\\
6e-10	0.864	0	0	\\
6.5e-10	0.864	0	0	\\
7e-10	0.864	0	0	\\
7.5e-10	0.864	0	0	\\
8e-10	0.864	0	0	\\
8.5e-10	0.864	0	0	\\
9e-10	0.864	0.01	0.01	\\
9.5e-10	0.864	0.01	0.01	\\
1e-09	0.864	0.01	0.01	\\
1.05e-09	0.864	0.01	0.01	\\
1.1e-09	0.864	0.01	0.01	\\
1.15e-09	0.864	0.01	0.01	\\
1.2e-09	0.864	0.01	0.01	\\
1.25e-09	0.864	0.01	0.01	\\
1.3e-09	0.864	0.01	0.01	\\
1.35e-09	0.864	0.01	0.01	\\
1.4e-09	0.864	0	0	\\
1.45e-09	0.864	0	0	\\
1.5e-09	0.864	0	0	\\
1.55e-09	0.864	-0.00666666666666667	-0.00666666666666667	\\
1.6e-09	0.864	-0.00666666666666667	-0.00666666666666667	\\
1.65e-09	0.864	-0.00666666666666667	-0.00666666666666667	\\
1.7e-09	0.864	-0.00666666666666667	-0.00666666666666667	\\
1.75e-09	0.864	-0.00666666666666667	-0.00666666666666667	\\
1.8e-09	0.864	-0.00666666666666667	-0.00666666666666667	\\
1.85e-09	0.864	-0.00666666666666667	-0.00666666666666667	\\
1.9e-09	0.864	-0.00666666666666667	-0.00666666666666667	\\
1.95e-09	0.864	-0.00666666666666667	-0.00666666666666667	\\
2e-09	0.864	-0.00666666666666667	-0.00666666666666667	\\
2.05e-09	0.864	0	0	\\
2.1e-09	0.864	0	0	\\
2.15e-09	0.864	0	0	\\
2.2e-09	0.864	0	0	\\
2.25e-09	0.864	0	0	\\
2.3e-09	0.864	0	0	\\
2.35e-09	0.864	0	0	\\
2.4e-09	0.864	0	0	\\
2.45e-09	0.864	0	0	\\
2.5e-09	0.864	0	0	\\
2.55e-09	0.864	0	0	\\
2.6e-09	0.864	0	0	\\
2.65e-09	0.864	0	0	\\
2.7e-09	0.864	0	0	\\
2.75e-09	0.864	0	0	\\
2.8e-09	0.864	0	0	\\
2.85e-09	0.864	0	0	\\
2.9e-09	0.864	0	0	\\
2.95e-09	0.864	0	0	\\
3e-09	0.864	0	0	\\
3.05e-09	0.864	0	0	\\
3.1e-09	0.864	0	0	\\
3.15e-09	0.864	0	0	\\
3.2e-09	0.864	0	0	\\
3.25e-09	0.864	0	0	\\
3.3e-09	0.864	0.00666666666666667	0.00666666666666667	\\
3.35e-09	0.864	0.00666666666666667	0.00666666666666667	\\
3.4e-09	0.864	0.00666666666666667	0.00666666666666667	\\
3.45e-09	0.864	0.00666666666666667	0.00666666666666667	\\
3.5e-09	0.864	0.00666666666666667	0.00666666666666667	\\
3.55e-09	0.864	0.00666666666666667	0.00666666666666667	\\
3.6e-09	0.864	0.00666666666666667	0.00666666666666667	\\
3.65e-09	0.864	0.00666666666666667	0.00666666666666667	\\
3.7e-09	0.864	0.00666666666666667	0.00666666666666667	\\
3.75e-09	0.864	0.00666666666666667	0.00666666666666667	\\
3.8e-09	0.864	0	0	\\
3.85e-09	0.864	0	0	\\
3.9e-09	0.864	0	0	\\
3.95e-09	0.864	-0.00444444444444444	-0.00444444444444444	\\
4e-09	0.864	-0.00444444444444444	-0.00444444444444444	\\
4.05e-09	0.864	-0.00444444444444444	-0.00444444444444444	\\
4.1e-09	0.864	-0.00444444444444444	-0.00444444444444444	\\
4.15e-09	0.864	-0.00444444444444444	-0.00444444444444444	\\
4.2e-09	0.864	-0.00444444444444444	-0.00444444444444444	\\
4.25e-09	0.864	-0.00444444444444444	-0.00444444444444444	\\
4.3e-09	0.864	-0.00444444444444444	-0.00444444444444444	\\
4.35e-09	0.864	-0.00444444444444444	-0.00444444444444444	\\
4.4e-09	0.864	-0.00444444444444444	-0.00444444444444444	\\
4.45e-09	0.864	0	0	\\
4.5e-09	0.864	0	0	\\
4.55e-09	0.864	0	0	\\
4.6e-09	0.864	0	0	\\
4.65e-09	0.864	0	0	\\
4.7e-09	0.864	0	0	\\
4.75e-09	0.864	0	0	\\
4.8e-09	0.864	0	0	\\
4.85e-09	0.864	0	0	\\
4.9e-09	0.864	0	0	\\
4.95e-09	0.864	0	0	\\
5e-09	0.864	0	0	\\
5e-09	0.864	0	nan	\\
5e-09	0.864	-0.00666666666666667	0.00166666666666667	\\
5e-09	0.864	-0.00666666666666667	nan	\\
0	0.876	-0.00666666666666667	nan	\\
0	0.876	0	0.00166666666666667	\\
0	0.876	0	0	\\
5e-11	0.876	0	0	\\
1e-10	0.876	0	0	\\
1.5e-10	0.876	0	0	\\
2e-10	0.876	0	0	\\
2.5e-10	0.876	0	0	\\
3e-10	0.876	0	0	\\
3.5e-10	0.876	0	0	\\
4e-10	0.876	0	0	\\
4.5e-10	0.876	0	0	\\
5e-10	0.876	0	0	\\
5.5e-10	0.876	0	0	\\
6e-10	0.876	0	0	\\
6.5e-10	0.876	0	0	\\
7e-10	0.876	0	0	\\
7.5e-10	0.876	0	0	\\
8e-10	0.876	0	0	\\
8.5e-10	0.876	0	0	\\
9e-10	0.876	0.01	0.01	\\
9.5e-10	0.876	0.01	0.01	\\
1e-09	0.876	0.01	0.01	\\
1.05e-09	0.876	0.01	0.01	\\
1.1e-09	0.876	0.01	0.01	\\
1.15e-09	0.876	0.01	0.01	\\
1.2e-09	0.876	0.01	0.01	\\
1.25e-09	0.876	0.01	0.01	\\
1.3e-09	0.876	0.01	0.01	\\
1.35e-09	0.876	0.01	0.01	\\
1.4e-09	0.876	0	0	\\
1.45e-09	0.876	0	0	\\
1.5e-09	0.876	0	0	\\
1.55e-09	0.876	-0.00666666666666667	-0.00666666666666667	\\
1.6e-09	0.876	-0.00666666666666667	-0.00666666666666667	\\
1.65e-09	0.876	-0.00666666666666667	-0.00666666666666667	\\
1.7e-09	0.876	-0.00666666666666667	-0.00666666666666667	\\
1.75e-09	0.876	-0.00666666666666667	-0.00666666666666667	\\
1.8e-09	0.876	-0.00666666666666667	-0.00666666666666667	\\
1.85e-09	0.876	-0.00666666666666667	-0.00666666666666667	\\
1.9e-09	0.876	-0.00666666666666667	-0.00666666666666667	\\
1.95e-09	0.876	-0.00666666666666667	-0.00666666666666667	\\
2e-09	0.876	-0.00666666666666667	-0.00666666666666667	\\
2.05e-09	0.876	0	0	\\
2.1e-09	0.876	0	0	\\
2.15e-09	0.876	0	0	\\
2.2e-09	0.876	0	0	\\
2.25e-09	0.876	0	0	\\
2.3e-09	0.876	0	0	\\
2.35e-09	0.876	0	0	\\
2.4e-09	0.876	0	0	\\
2.45e-09	0.876	0	0	\\
2.5e-09	0.876	0	0	\\
2.55e-09	0.876	0	0	\\
2.6e-09	0.876	0	0	\\
2.65e-09	0.876	0	0	\\
2.7e-09	0.876	0	0	\\
2.75e-09	0.876	0	0	\\
2.8e-09	0.876	0	0	\\
2.85e-09	0.876	0	0	\\
2.9e-09	0.876	0	0	\\
2.95e-09	0.876	0	0	\\
3e-09	0.876	0	0	\\
3.05e-09	0.876	0	0	\\
3.1e-09	0.876	0	0	\\
3.15e-09	0.876	0	0	\\
3.2e-09	0.876	0	0	\\
3.25e-09	0.876	0	0	\\
3.3e-09	0.876	0.00666666666666667	0.00666666666666667	\\
3.35e-09	0.876	0.00666666666666667	0.00666666666666667	\\
3.4e-09	0.876	0.00666666666666667	0.00666666666666667	\\
3.45e-09	0.876	0.00666666666666667	0.00666666666666667	\\
3.5e-09	0.876	0.00666666666666667	0.00666666666666667	\\
3.55e-09	0.876	0.00666666666666667	0.00666666666666667	\\
3.6e-09	0.876	0.00666666666666667	0.00666666666666667	\\
3.65e-09	0.876	0.00666666666666667	0.00666666666666667	\\
3.7e-09	0.876	0.00666666666666667	0.00666666666666667	\\
3.75e-09	0.876	0.00666666666666667	0.00666666666666667	\\
3.8e-09	0.876	0	0	\\
3.85e-09	0.876	0	0	\\
3.9e-09	0.876	0	0	\\
3.95e-09	0.876	-0.00444444444444444	-0.00444444444444444	\\
4e-09	0.876	-0.00444444444444444	-0.00444444444444444	\\
4.05e-09	0.876	-0.00444444444444444	-0.00444444444444444	\\
4.1e-09	0.876	-0.00444444444444444	-0.00444444444444444	\\
4.15e-09	0.876	-0.00444444444444444	-0.00444444444444444	\\
4.2e-09	0.876	-0.00444444444444444	-0.00444444444444444	\\
4.25e-09	0.876	-0.00444444444444444	-0.00444444444444444	\\
4.3e-09	0.876	-0.00444444444444444	-0.00444444444444444	\\
4.35e-09	0.876	-0.00444444444444444	-0.00444444444444444	\\
4.4e-09	0.876	-0.00444444444444444	-0.00444444444444444	\\
4.45e-09	0.876	0	0	\\
4.5e-09	0.876	0	0	\\
4.55e-09	0.876	0	0	\\
4.6e-09	0.876	0	0	\\
4.65e-09	0.876	0	0	\\
4.7e-09	0.876	0	0	\\
4.75e-09	0.876	0	0	\\
4.8e-09	0.876	0	0	\\
4.85e-09	0.876	0	0	\\
4.9e-09	0.876	0	0	\\
4.95e-09	0.876	0	0	\\
5e-09	0.876	0	0	\\
5e-09	0.876	0	nan	\\
5e-09	0.876	-0.00666666666666667	0.00166666666666667	\\
5e-09	0.876	-0.00666666666666667	nan	\\
0	0.888	-0.00666666666666667	nan	\\
0	0.888	0	0.00166666666666667	\\
0	0.888	0	0	\\
5e-11	0.888	0	0	\\
1e-10	0.888	0	0	\\
1.5e-10	0.888	0	0	\\
2e-10	0.888	0	0	\\
2.5e-10	0.888	0	0	\\
3e-10	0.888	0	0	\\
3.5e-10	0.888	0	0	\\
4e-10	0.888	0	0	\\
4.5e-10	0.888	0	0	\\
5e-10	0.888	0	0	\\
5.5e-10	0.888	0	0	\\
6e-10	0.888	0	0	\\
6.5e-10	0.888	0	0	\\
7e-10	0.888	0	0	\\
7.5e-10	0.888	0	0	\\
8e-10	0.888	0	0	\\
8.5e-10	0.888	0	0	\\
9e-10	0.888	0.01	0.01	\\
9.5e-10	0.888	0.01	0.01	\\
1e-09	0.888	0.01	0.01	\\
1.05e-09	0.888	0.01	0.01	\\
1.1e-09	0.888	0.01	0.01	\\
1.15e-09	0.888	0.01	0.01	\\
1.2e-09	0.888	0.01	0.01	\\
1.25e-09	0.888	0.01	0.01	\\
1.3e-09	0.888	0.01	0.01	\\
1.35e-09	0.888	0.01	0.01	\\
1.4e-09	0.888	0	0	\\
1.45e-09	0.888	0	0	\\
1.5e-09	0.888	0	0	\\
1.55e-09	0.888	-0.00666666666666667	-0.00666666666666667	\\
1.6e-09	0.888	-0.00666666666666667	-0.00666666666666667	\\
1.65e-09	0.888	-0.00666666666666667	-0.00666666666666667	\\
1.7e-09	0.888	-0.00666666666666667	-0.00666666666666667	\\
1.75e-09	0.888	-0.00666666666666667	-0.00666666666666667	\\
1.8e-09	0.888	-0.00666666666666667	-0.00666666666666667	\\
1.85e-09	0.888	-0.00666666666666667	-0.00666666666666667	\\
1.9e-09	0.888	-0.00666666666666667	-0.00666666666666667	\\
1.95e-09	0.888	-0.00666666666666667	-0.00666666666666667	\\
2e-09	0.888	-0.00666666666666667	-0.00666666666666667	\\
2.05e-09	0.888	0	0	\\
2.1e-09	0.888	0	0	\\
2.15e-09	0.888	0	0	\\
2.2e-09	0.888	0	0	\\
2.25e-09	0.888	0	0	\\
2.3e-09	0.888	0	0	\\
2.35e-09	0.888	0	0	\\
2.4e-09	0.888	0	0	\\
2.45e-09	0.888	0	0	\\
2.5e-09	0.888	0	0	\\
2.55e-09	0.888	0	0	\\
2.6e-09	0.888	0	0	\\
2.65e-09	0.888	0	0	\\
2.7e-09	0.888	0	0	\\
2.75e-09	0.888	0	0	\\
2.8e-09	0.888	0	0	\\
2.85e-09	0.888	0	0	\\
2.9e-09	0.888	0	0	\\
2.95e-09	0.888	0	0	\\
3e-09	0.888	0	0	\\
3.05e-09	0.888	0	0	\\
3.1e-09	0.888	0	0	\\
3.15e-09	0.888	0	0	\\
3.2e-09	0.888	0	0	\\
3.25e-09	0.888	0	0	\\
3.3e-09	0.888	0.00666666666666667	0.00666666666666667	\\
3.35e-09	0.888	0.00666666666666667	0.00666666666666667	\\
3.4e-09	0.888	0.00666666666666667	0.00666666666666667	\\
3.45e-09	0.888	0.00666666666666667	0.00666666666666667	\\
3.5e-09	0.888	0.00666666666666667	0.00666666666666667	\\
3.55e-09	0.888	0.00666666666666667	0.00666666666666667	\\
3.6e-09	0.888	0.00666666666666667	0.00666666666666667	\\
3.65e-09	0.888	0.00666666666666667	0.00666666666666667	\\
3.7e-09	0.888	0.00666666666666667	0.00666666666666667	\\
3.75e-09	0.888	0.00666666666666667	0.00666666666666667	\\
3.8e-09	0.888	0	0	\\
3.85e-09	0.888	0	0	\\
3.9e-09	0.888	0	0	\\
3.95e-09	0.888	-0.00444444444444444	-0.00444444444444444	\\
4e-09	0.888	-0.00444444444444444	-0.00444444444444444	\\
4.05e-09	0.888	-0.00444444444444444	-0.00444444444444444	\\
4.1e-09	0.888	-0.00444444444444444	-0.00444444444444444	\\
4.15e-09	0.888	-0.00444444444444444	-0.00444444444444444	\\
4.2e-09	0.888	-0.00444444444444444	-0.00444444444444444	\\
4.25e-09	0.888	-0.00444444444444444	-0.00444444444444444	\\
4.3e-09	0.888	-0.00444444444444444	-0.00444444444444444	\\
4.35e-09	0.888	-0.00444444444444444	-0.00444444444444444	\\
4.4e-09	0.888	-0.00444444444444444	-0.00444444444444444	\\
4.45e-09	0.888	0	0	\\
4.5e-09	0.888	0	0	\\
4.55e-09	0.888	0	0	\\
4.6e-09	0.888	0	0	\\
4.65e-09	0.888	0	0	\\
4.7e-09	0.888	0	0	\\
4.75e-09	0.888	0	0	\\
4.8e-09	0.888	0	0	\\
4.85e-09	0.888	0	0	\\
4.9e-09	0.888	0	0	\\
4.95e-09	0.888	0	0	\\
5e-09	0.888	0	0	\\
5e-09	0.888	0	nan	\\
5e-09	0.888	-0.00666666666666667	0.00166666666666667	\\
5e-09	0.888	-0.00666666666666667	nan	\\
0	0.9	-0.00666666666666667	nan	\\
0	0.9	0	0.00166666666666667	\\
0	0.9	0	0	\\
5e-11	0.9	0	0	\\
1e-10	0.9	0	0	\\
1.5e-10	0.9	0	0	\\
2e-10	0.9	0	0	\\
2.5e-10	0.9	0	0	\\
3e-10	0.9	0	0	\\
3.5e-10	0.9	0	0	\\
4e-10	0.9	0	0	\\
4.5e-10	0.9	0	0	\\
5e-10	0.9	0	0	\\
5.5e-10	0.9	0	0	\\
6e-10	0.9	0	0	\\
6.5e-10	0.9	0	0	\\
7e-10	0.9	0	0	\\
7.5e-10	0.9	0	0	\\
8e-10	0.9	0	0	\\
8.5e-10	0.9	0	0	\\
9e-10	0.9	0.01	0.01	\\
9.5e-10	0.9	0.01	0.01	\\
1e-09	0.9	0.01	0.01	\\
1.05e-09	0.9	0.01	0.01	\\
1.1e-09	0.9	0.01	0.01	\\
1.15e-09	0.9	0.01	0.01	\\
1.2e-09	0.9	0.01	0.01	\\
1.25e-09	0.9	0.01	0.01	\\
1.3e-09	0.9	0.01	0.01	\\
1.35e-09	0.9	0.01	0.01	\\
1.4e-09	0.9	0	0	\\
1.45e-09	0.9	0	0	\\
1.5e-09	0.9	-0.00666666666666667	-0.00666666666666667	\\
1.55e-09	0.9	-0.00666666666666667	-0.00666666666666667	\\
1.6e-09	0.9	-0.00666666666666667	-0.00666666666666667	\\
1.65e-09	0.9	-0.00666666666666667	-0.00666666666666667	\\
1.7e-09	0.9	-0.00666666666666667	-0.00666666666666667	\\
1.75e-09	0.9	-0.00666666666666667	-0.00666666666666667	\\
1.8e-09	0.9	-0.00666666666666667	-0.00666666666666667	\\
1.85e-09	0.9	-0.00666666666666667	-0.00666666666666667	\\
1.9e-09	0.9	-0.00666666666666667	-0.00666666666666667	\\
1.95e-09	0.9	-0.00666666666666667	-0.00666666666666667	\\
2e-09	0.9	0	0	\\
2.05e-09	0.9	0	0	\\
2.1e-09	0.9	0	0	\\
2.15e-09	0.9	0	0	\\
2.2e-09	0.9	0	0	\\
2.25e-09	0.9	0	0	\\
2.3e-09	0.9	0	0	\\
2.35e-09	0.9	0	0	\\
2.4e-09	0.9	0	0	\\
2.45e-09	0.9	0	0	\\
2.5e-09	0.9	0	0	\\
2.55e-09	0.9	0	0	\\
2.6e-09	0.9	0	0	\\
2.65e-09	0.9	0	0	\\
2.7e-09	0.9	0	0	\\
2.75e-09	0.9	0	0	\\
2.8e-09	0.9	0	0	\\
2.85e-09	0.9	0	0	\\
2.9e-09	0.9	0	0	\\
2.95e-09	0.9	0	0	\\
3e-09	0.9	0	0	\\
3.05e-09	0.9	0	0	\\
3.1e-09	0.9	0	0	\\
3.15e-09	0.9	0	0	\\
3.2e-09	0.9	0	0	\\
3.25e-09	0.9	0	0	\\
3.3e-09	0.9	0.00666666666666667	0.00666666666666667	\\
3.35e-09	0.9	0.00666666666666667	0.00666666666666667	\\
3.4e-09	0.9	0.00666666666666667	0.00666666666666667	\\
3.45e-09	0.9	0.00666666666666667	0.00666666666666667	\\
3.5e-09	0.9	0.00666666666666667	0.00666666666666667	\\
3.55e-09	0.9	0.00666666666666667	0.00666666666666667	\\
3.6e-09	0.9	0.00666666666666667	0.00666666666666667	\\
3.65e-09	0.9	0.00666666666666667	0.00666666666666667	\\
3.7e-09	0.9	0.00666666666666667	0.00666666666666667	\\
3.75e-09	0.9	0.00666666666666667	0.00666666666666667	\\
3.8e-09	0.9	0	0	\\
3.85e-09	0.9	0	0	\\
3.9e-09	0.9	-0.00444444444444444	-0.00444444444444444	\\
3.95e-09	0.9	-0.00444444444444444	-0.00444444444444444	\\
4e-09	0.9	-0.00444444444444444	-0.00444444444444444	\\
4.05e-09	0.9	-0.00444444444444444	-0.00444444444444444	\\
4.1e-09	0.9	-0.00444444444444444	-0.00444444444444444	\\
4.15e-09	0.9	-0.00444444444444444	-0.00444444444444444	\\
4.2e-09	0.9	-0.00444444444444444	-0.00444444444444444	\\
4.25e-09	0.9	-0.00444444444444444	-0.00444444444444444	\\
4.3e-09	0.9	-0.00444444444444444	-0.00444444444444444	\\
4.35e-09	0.9	-0.00444444444444444	-0.00444444444444444	\\
4.4e-09	0.9	0	0	\\
4.45e-09	0.9	0	0	\\
4.5e-09	0.9	0	0	\\
4.55e-09	0.9	0	0	\\
4.6e-09	0.9	0	0	\\
4.65e-09	0.9	0	0	\\
4.7e-09	0.9	0	0	\\
4.75e-09	0.9	0	0	\\
4.8e-09	0.9	0	0	\\
4.85e-09	0.9	0	0	\\
4.9e-09	0.9	0	0	\\
4.95e-09	0.9	0	0	\\
5e-09	0.9	0	0	\\
5e-09	0.9	0	nan	\\
5e-09	0.9	-0.00666666666666667	0.00166666666666667	\\
5e-09	0.9	-0.00666666666666667	nan	\\
0	0.912	-0.00666666666666667	nan	\\
0	0.912	0	0.00166666666666667	\\
0	0.912	0	0	\\
5e-11	0.912	0	0	\\
1e-10	0.912	0	0	\\
1.5e-10	0.912	0	0	\\
2e-10	0.912	0	0	\\
2.5e-10	0.912	0	0	\\
3e-10	0.912	0	0	\\
3.5e-10	0.912	0	0	\\
4e-10	0.912	0	0	\\
4.5e-10	0.912	0	0	\\
5e-10	0.912	0	0	\\
5.5e-10	0.912	0	0	\\
6e-10	0.912	0	0	\\
6.5e-10	0.912	0	0	\\
7e-10	0.912	0	0	\\
7.5e-10	0.912	0	0	\\
8e-10	0.912	0	0	\\
8.5e-10	0.912	0	0	\\
9e-10	0.912	0	0	\\
9.5e-10	0.912	0.01	0.01	\\
1e-09	0.912	0.01	0.01	\\
1.05e-09	0.912	0.01	0.01	\\
1.1e-09	0.912	0.01	0.01	\\
1.15e-09	0.912	0.01	0.01	\\
1.2e-09	0.912	0.01	0.01	\\
1.25e-09	0.912	0.01	0.01	\\
1.3e-09	0.912	0.01	0.01	\\
1.35e-09	0.912	0.01	0.01	\\
1.4e-09	0.912	0.01	0.01	\\
1.45e-09	0.912	0	0	\\
1.5e-09	0.912	-0.00666666666666667	-0.00666666666666667	\\
1.55e-09	0.912	-0.00666666666666667	-0.00666666666666667	\\
1.6e-09	0.912	-0.00666666666666667	-0.00666666666666667	\\
1.65e-09	0.912	-0.00666666666666667	-0.00666666666666667	\\
1.7e-09	0.912	-0.00666666666666667	-0.00666666666666667	\\
1.75e-09	0.912	-0.00666666666666667	-0.00666666666666667	\\
1.8e-09	0.912	-0.00666666666666667	-0.00666666666666667	\\
1.85e-09	0.912	-0.00666666666666667	-0.00666666666666667	\\
1.9e-09	0.912	-0.00666666666666667	-0.00666666666666667	\\
1.95e-09	0.912	-0.00666666666666667	-0.00666666666666667	\\
2e-09	0.912	0	0	\\
2.05e-09	0.912	0	0	\\
2.1e-09	0.912	0	0	\\
2.15e-09	0.912	0	0	\\
2.2e-09	0.912	0	0	\\
2.25e-09	0.912	0	0	\\
2.3e-09	0.912	0	0	\\
2.35e-09	0.912	0	0	\\
2.4e-09	0.912	0	0	\\
2.45e-09	0.912	0	0	\\
2.5e-09	0.912	0	0	\\
2.55e-09	0.912	0	0	\\
2.6e-09	0.912	0	0	\\
2.65e-09	0.912	0	0	\\
2.7e-09	0.912	0	0	\\
2.75e-09	0.912	0	0	\\
2.8e-09	0.912	0	0	\\
2.85e-09	0.912	0	0	\\
2.9e-09	0.912	0	0	\\
2.95e-09	0.912	0	0	\\
3e-09	0.912	0	0	\\
3.05e-09	0.912	0	0	\\
3.1e-09	0.912	0	0	\\
3.15e-09	0.912	0	0	\\
3.2e-09	0.912	0	0	\\
3.25e-09	0.912	0	0	\\
3.3e-09	0.912	0	0	\\
3.35e-09	0.912	0.00666666666666667	0.00666666666666667	\\
3.4e-09	0.912	0.00666666666666667	0.00666666666666667	\\
3.45e-09	0.912	0.00666666666666667	0.00666666666666667	\\
3.5e-09	0.912	0.00666666666666667	0.00666666666666667	\\
3.55e-09	0.912	0.00666666666666667	0.00666666666666667	\\
3.6e-09	0.912	0.00666666666666667	0.00666666666666667	\\
3.65e-09	0.912	0.00666666666666667	0.00666666666666667	\\
3.7e-09	0.912	0.00666666666666667	0.00666666666666667	\\
3.75e-09	0.912	0.00666666666666667	0.00666666666666667	\\
3.8e-09	0.912	0.00666666666666667	0.00666666666666667	\\
3.85e-09	0.912	0	0	\\
3.9e-09	0.912	-0.00444444444444444	-0.00444444444444444	\\
3.95e-09	0.912	-0.00444444444444444	-0.00444444444444444	\\
4e-09	0.912	-0.00444444444444444	-0.00444444444444444	\\
4.05e-09	0.912	-0.00444444444444444	-0.00444444444444444	\\
4.1e-09	0.912	-0.00444444444444444	-0.00444444444444444	\\
4.15e-09	0.912	-0.00444444444444444	-0.00444444444444444	\\
4.2e-09	0.912	-0.00444444444444444	-0.00444444444444444	\\
4.25e-09	0.912	-0.00444444444444444	-0.00444444444444444	\\
4.3e-09	0.912	-0.00444444444444444	-0.00444444444444444	\\
4.35e-09	0.912	-0.00444444444444444	-0.00444444444444444	\\
4.4e-09	0.912	0	0	\\
4.45e-09	0.912	0	0	\\
4.5e-09	0.912	0	0	\\
4.55e-09	0.912	0	0	\\
4.6e-09	0.912	0	0	\\
4.65e-09	0.912	0	0	\\
4.7e-09	0.912	0	0	\\
4.75e-09	0.912	0	0	\\
4.8e-09	0.912	0	0	\\
4.85e-09	0.912	0	0	\\
4.9e-09	0.912	0	0	\\
4.95e-09	0.912	0	0	\\
5e-09	0.912	0	0	\\
5e-09	0.912	0	nan	\\
5e-09	0.912	-0.00666666666666667	0.00166666666666667	\\
5e-09	0.912	-0.00666666666666667	nan	\\
0	0.924	-0.00666666666666667	nan	\\
0	0.924	0	0.00166666666666667	\\
0	0.924	0	0	\\
5e-11	0.924	0	0	\\
1e-10	0.924	0	0	\\
1.5e-10	0.924	0	0	\\
2e-10	0.924	0	0	\\
2.5e-10	0.924	0	0	\\
3e-10	0.924	0	0	\\
3.5e-10	0.924	0	0	\\
4e-10	0.924	0	0	\\
4.5e-10	0.924	0	0	\\
5e-10	0.924	0	0	\\
5.5e-10	0.924	0	0	\\
6e-10	0.924	0	0	\\
6.5e-10	0.924	0	0	\\
7e-10	0.924	0	0	\\
7.5e-10	0.924	0	0	\\
8e-10	0.924	0	0	\\
8.5e-10	0.924	0	0	\\
9e-10	0.924	0	0	\\
9.5e-10	0.924	0.01	0.01	\\
1e-09	0.924	0.01	0.01	\\
1.05e-09	0.924	0.01	0.01	\\
1.1e-09	0.924	0.01	0.01	\\
1.15e-09	0.924	0.01	0.01	\\
1.2e-09	0.924	0.01	0.01	\\
1.25e-09	0.924	0.01	0.01	\\
1.3e-09	0.924	0.01	0.01	\\
1.35e-09	0.924	0.01	0.01	\\
1.4e-09	0.924	0.01	0.01	\\
1.45e-09	0.924	0	0	\\
1.5e-09	0.924	-0.00666666666666667	-0.00666666666666667	\\
1.55e-09	0.924	-0.00666666666666667	-0.00666666666666667	\\
1.6e-09	0.924	-0.00666666666666667	-0.00666666666666667	\\
1.65e-09	0.924	-0.00666666666666667	-0.00666666666666667	\\
1.7e-09	0.924	-0.00666666666666667	-0.00666666666666667	\\
1.75e-09	0.924	-0.00666666666666667	-0.00666666666666667	\\
1.8e-09	0.924	-0.00666666666666667	-0.00666666666666667	\\
1.85e-09	0.924	-0.00666666666666667	-0.00666666666666667	\\
1.9e-09	0.924	-0.00666666666666667	-0.00666666666666667	\\
1.95e-09	0.924	-0.00666666666666667	-0.00666666666666667	\\
2e-09	0.924	0	0	\\
2.05e-09	0.924	0	0	\\
2.1e-09	0.924	0	0	\\
2.15e-09	0.924	0	0	\\
2.2e-09	0.924	0	0	\\
2.25e-09	0.924	0	0	\\
2.3e-09	0.924	0	0	\\
2.35e-09	0.924	0	0	\\
2.4e-09	0.924	0	0	\\
2.45e-09	0.924	0	0	\\
2.5e-09	0.924	0	0	\\
2.55e-09	0.924	0	0	\\
2.6e-09	0.924	0	0	\\
2.65e-09	0.924	0	0	\\
2.7e-09	0.924	0	0	\\
2.75e-09	0.924	0	0	\\
2.8e-09	0.924	0	0	\\
2.85e-09	0.924	0	0	\\
2.9e-09	0.924	0	0	\\
2.95e-09	0.924	0	0	\\
3e-09	0.924	0	0	\\
3.05e-09	0.924	0	0	\\
3.1e-09	0.924	0	0	\\
3.15e-09	0.924	0	0	\\
3.2e-09	0.924	0	0	\\
3.25e-09	0.924	0	0	\\
3.3e-09	0.924	0	0	\\
3.35e-09	0.924	0.00666666666666667	0.00666666666666667	\\
3.4e-09	0.924	0.00666666666666667	0.00666666666666667	\\
3.45e-09	0.924	0.00666666666666667	0.00666666666666667	\\
3.5e-09	0.924	0.00666666666666667	0.00666666666666667	\\
3.55e-09	0.924	0.00666666666666667	0.00666666666666667	\\
3.6e-09	0.924	0.00666666666666667	0.00666666666666667	\\
3.65e-09	0.924	0.00666666666666667	0.00666666666666667	\\
3.7e-09	0.924	0.00666666666666667	0.00666666666666667	\\
3.75e-09	0.924	0.00666666666666667	0.00666666666666667	\\
3.8e-09	0.924	0.00666666666666667	0.00666666666666667	\\
3.85e-09	0.924	0	0	\\
3.9e-09	0.924	-0.00444444444444444	-0.00444444444444444	\\
3.95e-09	0.924	-0.00444444444444444	-0.00444444444444444	\\
4e-09	0.924	-0.00444444444444444	-0.00444444444444444	\\
4.05e-09	0.924	-0.00444444444444444	-0.00444444444444444	\\
4.1e-09	0.924	-0.00444444444444444	-0.00444444444444444	\\
4.15e-09	0.924	-0.00444444444444444	-0.00444444444444444	\\
4.2e-09	0.924	-0.00444444444444444	-0.00444444444444444	\\
4.25e-09	0.924	-0.00444444444444444	-0.00444444444444444	\\
4.3e-09	0.924	-0.00444444444444444	-0.00444444444444444	\\
4.35e-09	0.924	-0.00444444444444444	-0.00444444444444444	\\
4.4e-09	0.924	0	0	\\
4.45e-09	0.924	0	0	\\
4.5e-09	0.924	0	0	\\
4.55e-09	0.924	0	0	\\
4.6e-09	0.924	0	0	\\
4.65e-09	0.924	0	0	\\
4.7e-09	0.924	0	0	\\
4.75e-09	0.924	0	0	\\
4.8e-09	0.924	0	0	\\
4.85e-09	0.924	0	0	\\
4.9e-09	0.924	0	0	\\
4.95e-09	0.924	0	0	\\
5e-09	0.924	0	0	\\
5e-09	0.924	0	nan	\\
5e-09	0.924	-0.00666666666666667	0.00166666666666667	\\
5e-09	0.924	-0.00666666666666667	nan	\\
0	0.936	-0.00666666666666667	nan	\\
0	0.936	0	0.00166666666666667	\\
0	0.936	0	0	\\
5e-11	0.936	0	0	\\
1e-10	0.936	0	0	\\
1.5e-10	0.936	0	0	\\
2e-10	0.936	0	0	\\
2.5e-10	0.936	0	0	\\
3e-10	0.936	0	0	\\
3.5e-10	0.936	0	0	\\
4e-10	0.936	0	0	\\
4.5e-10	0.936	0	0	\\
5e-10	0.936	0	0	\\
5.5e-10	0.936	0	0	\\
6e-10	0.936	0	0	\\
6.5e-10	0.936	0	0	\\
7e-10	0.936	0	0	\\
7.5e-10	0.936	0	0	\\
8e-10	0.936	0	0	\\
8.5e-10	0.936	0	0	\\
9e-10	0.936	0	0	\\
9.5e-10	0.936	0.01	0.01	\\
1e-09	0.936	0.01	0.01	\\
1.05e-09	0.936	0.01	0.01	\\
1.1e-09	0.936	0.01	0.01	\\
1.15e-09	0.936	0.01	0.01	\\
1.2e-09	0.936	0.01	0.01	\\
1.25e-09	0.936	0.01	0.01	\\
1.3e-09	0.936	0.01	0.01	\\
1.35e-09	0.936	0.01	0.01	\\
1.4e-09	0.936	0.01	0.01	\\
1.45e-09	0.936	0	0	\\
1.5e-09	0.936	-0.00666666666666667	-0.00666666666666667	\\
1.55e-09	0.936	-0.00666666666666667	-0.00666666666666667	\\
1.6e-09	0.936	-0.00666666666666667	-0.00666666666666667	\\
1.65e-09	0.936	-0.00666666666666667	-0.00666666666666667	\\
1.7e-09	0.936	-0.00666666666666667	-0.00666666666666667	\\
1.75e-09	0.936	-0.00666666666666667	-0.00666666666666667	\\
1.8e-09	0.936	-0.00666666666666667	-0.00666666666666667	\\
1.85e-09	0.936	-0.00666666666666667	-0.00666666666666667	\\
1.9e-09	0.936	-0.00666666666666667	-0.00666666666666667	\\
1.95e-09	0.936	-0.00666666666666667	-0.00666666666666667	\\
2e-09	0.936	0	0	\\
2.05e-09	0.936	0	0	\\
2.1e-09	0.936	0	0	\\
2.15e-09	0.936	0	0	\\
2.2e-09	0.936	0	0	\\
2.25e-09	0.936	0	0	\\
2.3e-09	0.936	0	0	\\
2.35e-09	0.936	0	0	\\
2.4e-09	0.936	0	0	\\
2.45e-09	0.936	0	0	\\
2.5e-09	0.936	0	0	\\
2.55e-09	0.936	0	0	\\
2.6e-09	0.936	0	0	\\
2.65e-09	0.936	0	0	\\
2.7e-09	0.936	0	0	\\
2.75e-09	0.936	0	0	\\
2.8e-09	0.936	0	0	\\
2.85e-09	0.936	0	0	\\
2.9e-09	0.936	0	0	\\
2.95e-09	0.936	0	0	\\
3e-09	0.936	0	0	\\
3.05e-09	0.936	0	0	\\
3.1e-09	0.936	0	0	\\
3.15e-09	0.936	0	0	\\
3.2e-09	0.936	0	0	\\
3.25e-09	0.936	0	0	\\
3.3e-09	0.936	0	0	\\
3.35e-09	0.936	0.00666666666666667	0.00666666666666667	\\
3.4e-09	0.936	0.00666666666666667	0.00666666666666667	\\
3.45e-09	0.936	0.00666666666666667	0.00666666666666667	\\
3.5e-09	0.936	0.00666666666666667	0.00666666666666667	\\
3.55e-09	0.936	0.00666666666666667	0.00666666666666667	\\
3.6e-09	0.936	0.00666666666666667	0.00666666666666667	\\
3.65e-09	0.936	0.00666666666666667	0.00666666666666667	\\
3.7e-09	0.936	0.00666666666666667	0.00666666666666667	\\
3.75e-09	0.936	0.00666666666666667	0.00666666666666667	\\
3.8e-09	0.936	0.00666666666666667	0.00666666666666667	\\
3.85e-09	0.936	0	0	\\
3.9e-09	0.936	-0.00444444444444444	-0.00444444444444444	\\
3.95e-09	0.936	-0.00444444444444444	-0.00444444444444444	\\
4e-09	0.936	-0.00444444444444444	-0.00444444444444444	\\
4.05e-09	0.936	-0.00444444444444444	-0.00444444444444444	\\
4.1e-09	0.936	-0.00444444444444444	-0.00444444444444444	\\
4.15e-09	0.936	-0.00444444444444444	-0.00444444444444444	\\
4.2e-09	0.936	-0.00444444444444444	-0.00444444444444444	\\
4.25e-09	0.936	-0.00444444444444444	-0.00444444444444444	\\
4.3e-09	0.936	-0.00444444444444444	-0.00444444444444444	\\
4.35e-09	0.936	-0.00444444444444444	-0.00444444444444444	\\
4.4e-09	0.936	0	0	\\
4.45e-09	0.936	0	0	\\
4.5e-09	0.936	0	0	\\
4.55e-09	0.936	0	0	\\
4.6e-09	0.936	0	0	\\
4.65e-09	0.936	0	0	\\
4.7e-09	0.936	0	0	\\
4.75e-09	0.936	0	0	\\
4.8e-09	0.936	0	0	\\
4.85e-09	0.936	0	0	\\
4.9e-09	0.936	0	0	\\
4.95e-09	0.936	0	0	\\
5e-09	0.936	0	0	\\
5e-09	0.936	0	nan	\\
5e-09	0.936	-0.00666666666666667	0.00166666666666667	\\
5e-09	0.936	-0.00666666666666667	nan	\\
0	0.948	-0.00666666666666667	nan	\\
0	0.948	0	0.00166666666666667	\\
0	0.948	0	0	\\
5e-11	0.948	0	0	\\
1e-10	0.948	0	0	\\
1.5e-10	0.948	0	0	\\
2e-10	0.948	0	0	\\
2.5e-10	0.948	0	0	\\
3e-10	0.948	0	0	\\
3.5e-10	0.948	0	0	\\
4e-10	0.948	0	0	\\
4.5e-10	0.948	0	0	\\
5e-10	0.948	0	0	\\
5.5e-10	0.948	0	0	\\
6e-10	0.948	0	0	\\
6.5e-10	0.948	0	0	\\
7e-10	0.948	0	0	\\
7.5e-10	0.948	0	0	\\
8e-10	0.948	0	0	\\
8.5e-10	0.948	0	0	\\
9e-10	0.948	0	0	\\
9.5e-10	0.948	0.01	0.01	\\
1e-09	0.948	0.01	0.01	\\
1.05e-09	0.948	0.01	0.01	\\
1.1e-09	0.948	0.01	0.01	\\
1.15e-09	0.948	0.01	0.01	\\
1.2e-09	0.948	0.01	0.01	\\
1.25e-09	0.948	0.01	0.01	\\
1.3e-09	0.948	0.01	0.01	\\
1.35e-09	0.948	0.01	0.01	\\
1.4e-09	0.948	0.01	0.01	\\
1.45e-09	0.948	0	0	\\
1.5e-09	0.948	-0.00666666666666667	-0.00666666666666667	\\
1.55e-09	0.948	-0.00666666666666667	-0.00666666666666667	\\
1.6e-09	0.948	-0.00666666666666667	-0.00666666666666667	\\
1.65e-09	0.948	-0.00666666666666667	-0.00666666666666667	\\
1.7e-09	0.948	-0.00666666666666667	-0.00666666666666667	\\
1.75e-09	0.948	-0.00666666666666667	-0.00666666666666667	\\
1.8e-09	0.948	-0.00666666666666667	-0.00666666666666667	\\
1.85e-09	0.948	-0.00666666666666667	-0.00666666666666667	\\
1.9e-09	0.948	-0.00666666666666667	-0.00666666666666667	\\
1.95e-09	0.948	-0.00666666666666667	-0.00666666666666667	\\
2e-09	0.948	0	0	\\
2.05e-09	0.948	0	0	\\
2.1e-09	0.948	0	0	\\
2.15e-09	0.948	0	0	\\
2.2e-09	0.948	0	0	\\
2.25e-09	0.948	0	0	\\
2.3e-09	0.948	0	0	\\
2.35e-09	0.948	0	0	\\
2.4e-09	0.948	0	0	\\
2.45e-09	0.948	0	0	\\
2.5e-09	0.948	0	0	\\
2.55e-09	0.948	0	0	\\
2.6e-09	0.948	0	0	\\
2.65e-09	0.948	0	0	\\
2.7e-09	0.948	0	0	\\
2.75e-09	0.948	0	0	\\
2.8e-09	0.948	0	0	\\
2.85e-09	0.948	0	0	\\
2.9e-09	0.948	0	0	\\
2.95e-09	0.948	0	0	\\
3e-09	0.948	0	0	\\
3.05e-09	0.948	0	0	\\
3.1e-09	0.948	0	0	\\
3.15e-09	0.948	0	0	\\
3.2e-09	0.948	0	0	\\
3.25e-09	0.948	0	0	\\
3.3e-09	0.948	0	0	\\
3.35e-09	0.948	0.00666666666666667	0.00666666666666667	\\
3.4e-09	0.948	0.00666666666666667	0.00666666666666667	\\
3.45e-09	0.948	0.00666666666666667	0.00666666666666667	\\
3.5e-09	0.948	0.00666666666666667	0.00666666666666667	\\
3.55e-09	0.948	0.00666666666666667	0.00666666666666667	\\
3.6e-09	0.948	0.00666666666666667	0.00666666666666667	\\
3.65e-09	0.948	0.00666666666666667	0.00666666666666667	\\
3.7e-09	0.948	0.00666666666666667	0.00666666666666667	\\
3.75e-09	0.948	0.00666666666666667	0.00666666666666667	\\
3.8e-09	0.948	0.00666666666666667	0.00666666666666667	\\
3.85e-09	0.948	0	0	\\
3.9e-09	0.948	-0.00444444444444444	-0.00444444444444444	\\
3.95e-09	0.948	-0.00444444444444444	-0.00444444444444444	\\
4e-09	0.948	-0.00444444444444444	-0.00444444444444444	\\
4.05e-09	0.948	-0.00444444444444444	-0.00444444444444444	\\
4.1e-09	0.948	-0.00444444444444444	-0.00444444444444444	\\
4.15e-09	0.948	-0.00444444444444444	-0.00444444444444444	\\
4.2e-09	0.948	-0.00444444444444444	-0.00444444444444444	\\
4.25e-09	0.948	-0.00444444444444444	-0.00444444444444444	\\
4.3e-09	0.948	-0.00444444444444444	-0.00444444444444444	\\
4.35e-09	0.948	-0.00444444444444444	-0.00444444444444444	\\
4.4e-09	0.948	0	0	\\
4.45e-09	0.948	0	0	\\
4.5e-09	0.948	0	0	\\
4.55e-09	0.948	0	0	\\
4.6e-09	0.948	0	0	\\
4.65e-09	0.948	0	0	\\
4.7e-09	0.948	0	0	\\
4.75e-09	0.948	0	0	\\
4.8e-09	0.948	0	0	\\
4.85e-09	0.948	0	0	\\
4.9e-09	0.948	0	0	\\
4.95e-09	0.948	0	0	\\
5e-09	0.948	0	0	\\
5e-09	0.948	0	nan	\\
5e-09	0.948	-0.00666666666666667	0.00166666666666667	\\
5e-09	0.948	-0.00666666666666667	nan	\\
0	0.96	-0.00666666666666667	nan	\\
0	0.96	0	0.00166666666666667	\\
0	0.96	0	0	\\
5e-11	0.96	0	0	\\
1e-10	0.96	0	0	\\
1.5e-10	0.96	0	0	\\
2e-10	0.96	0	0	\\
2.5e-10	0.96	0	0	\\
3e-10	0.96	0	0	\\
3.5e-10	0.96	0	0	\\
4e-10	0.96	0	0	\\
4.5e-10	0.96	0	0	\\
5e-10	0.96	0	0	\\
5.5e-10	0.96	0	0	\\
6e-10	0.96	0	0	\\
6.5e-10	0.96	0	0	\\
7e-10	0.96	0	0	\\
7.5e-10	0.96	0	0	\\
8e-10	0.96	0	0	\\
8.5e-10	0.96	0	0	\\
9e-10	0.96	0	0	\\
9.5e-10	0.96	0	0	\\
1e-09	0.96	0.01	0.01	\\
1.05e-09	0.96	0.01	0.01	\\
1.1e-09	0.96	0.01	0.01	\\
1.15e-09	0.96	0.01	0.01	\\
1.2e-09	0.96	0.01	0.01	\\
1.25e-09	0.96	0.01	0.01	\\
1.3e-09	0.96	0.01	0.01	\\
1.35e-09	0.96	0.01	0.01	\\
1.4e-09	0.96	0.01	0.01	\\
1.45e-09	0.96	0.00333333333333333	0.00333333333333333	\\
1.5e-09	0.96	-0.00666666666666667	-0.00666666666666667	\\
1.55e-09	0.96	-0.00666666666666667	-0.00666666666666667	\\
1.6e-09	0.96	-0.00666666666666667	-0.00666666666666667	\\
1.65e-09	0.96	-0.00666666666666667	-0.00666666666666667	\\
1.7e-09	0.96	-0.00666666666666667	-0.00666666666666667	\\
1.75e-09	0.96	-0.00666666666666667	-0.00666666666666667	\\
1.8e-09	0.96	-0.00666666666666667	-0.00666666666666667	\\
1.85e-09	0.96	-0.00666666666666667	-0.00666666666666667	\\
1.9e-09	0.96	-0.00666666666666667	-0.00666666666666667	\\
1.95e-09	0.96	0	0	\\
2e-09	0.96	0	0	\\
2.05e-09	0.96	0	0	\\
2.1e-09	0.96	0	0	\\
2.15e-09	0.96	0	0	\\
2.2e-09	0.96	0	0	\\
2.25e-09	0.96	0	0	\\
2.3e-09	0.96	0	0	\\
2.35e-09	0.96	0	0	\\
2.4e-09	0.96	0	0	\\
2.45e-09	0.96	0	0	\\
2.5e-09	0.96	0	0	\\
2.55e-09	0.96	0	0	\\
2.6e-09	0.96	0	0	\\
2.65e-09	0.96	0	0	\\
2.7e-09	0.96	0	0	\\
2.75e-09	0.96	0	0	\\
2.8e-09	0.96	0	0	\\
2.85e-09	0.96	0	0	\\
2.9e-09	0.96	0	0	\\
2.95e-09	0.96	0	0	\\
3e-09	0.96	0	0	\\
3.05e-09	0.96	0	0	\\
3.1e-09	0.96	0	0	\\
3.15e-09	0.96	0	0	\\
3.2e-09	0.96	0	0	\\
3.25e-09	0.96	0	0	\\
3.3e-09	0.96	0	0	\\
3.35e-09	0.96	0	0	\\
3.4e-09	0.96	0.00666666666666667	0.00666666666666667	\\
3.45e-09	0.96	0.00666666666666667	0.00666666666666667	\\
3.5e-09	0.96	0.00666666666666667	0.00666666666666667	\\
3.55e-09	0.96	0.00666666666666667	0.00666666666666667	\\
3.6e-09	0.96	0.00666666666666667	0.00666666666666667	\\
3.65e-09	0.96	0.00666666666666667	0.00666666666666667	\\
3.7e-09	0.96	0.00666666666666667	0.00666666666666667	\\
3.75e-09	0.96	0.00666666666666667	0.00666666666666667	\\
3.8e-09	0.96	0.00666666666666667	0.00666666666666667	\\
3.85e-09	0.96	0.00222222222222222	0.00222222222222222	\\
3.9e-09	0.96	-0.00444444444444444	-0.00444444444444444	\\
3.95e-09	0.96	-0.00444444444444444	-0.00444444444444444	\\
4e-09	0.96	-0.00444444444444444	-0.00444444444444444	\\
4.05e-09	0.96	-0.00444444444444444	-0.00444444444444444	\\
4.1e-09	0.96	-0.00444444444444444	-0.00444444444444444	\\
4.15e-09	0.96	-0.00444444444444444	-0.00444444444444444	\\
4.2e-09	0.96	-0.00444444444444444	-0.00444444444444444	\\
4.25e-09	0.96	-0.00444444444444444	-0.00444444444444444	\\
4.3e-09	0.96	-0.00444444444444444	-0.00444444444444444	\\
4.35e-09	0.96	0	0	\\
4.4e-09	0.96	0	0	\\
4.45e-09	0.96	0	0	\\
4.5e-09	0.96	0	0	\\
4.55e-09	0.96	0	0	\\
4.6e-09	0.96	0	0	\\
4.65e-09	0.96	0	0	\\
4.7e-09	0.96	0	0	\\
4.75e-09	0.96	0	0	\\
4.8e-09	0.96	0	0	\\
4.85e-09	0.96	0	0	\\
4.9e-09	0.96	0	0	\\
4.95e-09	0.96	0	0	\\
5e-09	0.96	0	0	\\
5e-09	0.96	0	nan	\\
5e-09	0.96	-0.00666666666666667	0.00166666666666667	\\
5e-09	0.96	-0.00666666666666667	nan	\\
0	0.972	-0.00666666666666667	nan	\\
0	0.972	0	0.00166666666666667	\\
0	0.972	0	0	\\
5e-11	0.972	0	0	\\
1e-10	0.972	0	0	\\
1.5e-10	0.972	0	0	\\
2e-10	0.972	0	0	\\
2.5e-10	0.972	0	0	\\
3e-10	0.972	0	0	\\
3.5e-10	0.972	0	0	\\
4e-10	0.972	0	0	\\
4.5e-10	0.972	0	0	\\
5e-10	0.972	0	0	\\
5.5e-10	0.972	0	0	\\
6e-10	0.972	0	0	\\
6.5e-10	0.972	0	0	\\
7e-10	0.972	0	0	\\
7.5e-10	0.972	0	0	\\
8e-10	0.972	0	0	\\
8.5e-10	0.972	0	0	\\
9e-10	0.972	0	0	\\
9.5e-10	0.972	0	0	\\
1e-09	0.972	0.01	0.01	\\
1.05e-09	0.972	0.01	0.01	\\
1.1e-09	0.972	0.01	0.01	\\
1.15e-09	0.972	0.01	0.01	\\
1.2e-09	0.972	0.01	0.01	\\
1.25e-09	0.972	0.01	0.01	\\
1.3e-09	0.972	0.01	0.01	\\
1.35e-09	0.972	0.01	0.01	\\
1.4e-09	0.972	0.01	0.01	\\
1.45e-09	0.972	0.00333333333333333	0.00333333333333333	\\
1.5e-09	0.972	-0.00666666666666667	-0.00666666666666667	\\
1.55e-09	0.972	-0.00666666666666667	-0.00666666666666667	\\
1.6e-09	0.972	-0.00666666666666667	-0.00666666666666667	\\
1.65e-09	0.972	-0.00666666666666667	-0.00666666666666667	\\
1.7e-09	0.972	-0.00666666666666667	-0.00666666666666667	\\
1.75e-09	0.972	-0.00666666666666667	-0.00666666666666667	\\
1.8e-09	0.972	-0.00666666666666667	-0.00666666666666667	\\
1.85e-09	0.972	-0.00666666666666667	-0.00666666666666667	\\
1.9e-09	0.972	-0.00666666666666667	-0.00666666666666667	\\
1.95e-09	0.972	0	0	\\
2e-09	0.972	0	0	\\
2.05e-09	0.972	0	0	\\
2.1e-09	0.972	0	0	\\
2.15e-09	0.972	0	0	\\
2.2e-09	0.972	0	0	\\
2.25e-09	0.972	0	0	\\
2.3e-09	0.972	0	0	\\
2.35e-09	0.972	0	0	\\
2.4e-09	0.972	0	0	\\
2.45e-09	0.972	0	0	\\
2.5e-09	0.972	0	0	\\
2.55e-09	0.972	0	0	\\
2.6e-09	0.972	0	0	\\
2.65e-09	0.972	0	0	\\
2.7e-09	0.972	0	0	\\
2.75e-09	0.972	0	0	\\
2.8e-09	0.972	0	0	\\
2.85e-09	0.972	0	0	\\
2.9e-09	0.972	0	0	\\
2.95e-09	0.972	0	0	\\
3e-09	0.972	0	0	\\
3.05e-09	0.972	0	0	\\
3.1e-09	0.972	0	0	\\
3.15e-09	0.972	0	0	\\
3.2e-09	0.972	0	0	\\
3.25e-09	0.972	0	0	\\
3.3e-09	0.972	0	0	\\
3.35e-09	0.972	0	0	\\
3.4e-09	0.972	0.00666666666666667	0.00666666666666667	\\
3.45e-09	0.972	0.00666666666666667	0.00666666666666667	\\
3.5e-09	0.972	0.00666666666666667	0.00666666666666667	\\
3.55e-09	0.972	0.00666666666666667	0.00666666666666667	\\
3.6e-09	0.972	0.00666666666666667	0.00666666666666667	\\
3.65e-09	0.972	0.00666666666666667	0.00666666666666667	\\
3.7e-09	0.972	0.00666666666666667	0.00666666666666667	\\
3.75e-09	0.972	0.00666666666666667	0.00666666666666667	\\
3.8e-09	0.972	0.00666666666666667	0.00666666666666667	\\
3.85e-09	0.972	0.00222222222222222	0.00222222222222222	\\
3.9e-09	0.972	-0.00444444444444444	-0.00444444444444444	\\
3.95e-09	0.972	-0.00444444444444444	-0.00444444444444444	\\
4e-09	0.972	-0.00444444444444444	-0.00444444444444444	\\
4.05e-09	0.972	-0.00444444444444444	-0.00444444444444444	\\
4.1e-09	0.972	-0.00444444444444444	-0.00444444444444444	\\
4.15e-09	0.972	-0.00444444444444444	-0.00444444444444444	\\
4.2e-09	0.972	-0.00444444444444444	-0.00444444444444444	\\
4.25e-09	0.972	-0.00444444444444444	-0.00444444444444444	\\
4.3e-09	0.972	-0.00444444444444444	-0.00444444444444444	\\
4.35e-09	0.972	0	0	\\
4.4e-09	0.972	0	0	\\
4.45e-09	0.972	0	0	\\
4.5e-09	0.972	0	0	\\
4.55e-09	0.972	0	0	\\
4.6e-09	0.972	0	0	\\
4.65e-09	0.972	0	0	\\
4.7e-09	0.972	0	0	\\
4.75e-09	0.972	0	0	\\
4.8e-09	0.972	0	0	\\
4.85e-09	0.972	0	0	\\
4.9e-09	0.972	0	0	\\
4.95e-09	0.972	0	0	\\
5e-09	0.972	0	0	\\
5e-09	0.972	0	nan	\\
5e-09	0.972	-0.00666666666666667	0.00166666666666667	\\
5e-09	0.972	-0.00666666666666667	nan	\\
0	0.984	-0.00666666666666667	nan	\\
0	0.984	0	0.00166666666666667	\\
0	0.984	0	0	\\
5e-11	0.984	0	0	\\
1e-10	0.984	0	0	\\
1.5e-10	0.984	0	0	\\
2e-10	0.984	0	0	\\
2.5e-10	0.984	0	0	\\
3e-10	0.984	0	0	\\
3.5e-10	0.984	0	0	\\
4e-10	0.984	0	0	\\
4.5e-10	0.984	0	0	\\
5e-10	0.984	0	0	\\
5.5e-10	0.984	0	0	\\
6e-10	0.984	0	0	\\
6.5e-10	0.984	0	0	\\
7e-10	0.984	0	0	\\
7.5e-10	0.984	0	0	\\
8e-10	0.984	0	0	\\
8.5e-10	0.984	0	0	\\
9e-10	0.984	0	0	\\
9.5e-10	0.984	0	0	\\
1e-09	0.984	0.01	0.01	\\
1.05e-09	0.984	0.01	0.01	\\
1.1e-09	0.984	0.01	0.01	\\
1.15e-09	0.984	0.01	0.01	\\
1.2e-09	0.984	0.01	0.01	\\
1.25e-09	0.984	0.01	0.01	\\
1.3e-09	0.984	0.01	0.01	\\
1.35e-09	0.984	0.01	0.01	\\
1.4e-09	0.984	0.01	0.01	\\
1.45e-09	0.984	0.00333333333333333	0.00333333333333333	\\
1.5e-09	0.984	-0.00666666666666667	-0.00666666666666667	\\
1.55e-09	0.984	-0.00666666666666667	-0.00666666666666667	\\
1.6e-09	0.984	-0.00666666666666667	-0.00666666666666667	\\
1.65e-09	0.984	-0.00666666666666667	-0.00666666666666667	\\
1.7e-09	0.984	-0.00666666666666667	-0.00666666666666667	\\
1.75e-09	0.984	-0.00666666666666667	-0.00666666666666667	\\
1.8e-09	0.984	-0.00666666666666667	-0.00666666666666667	\\
1.85e-09	0.984	-0.00666666666666667	-0.00666666666666667	\\
1.9e-09	0.984	-0.00666666666666667	-0.00666666666666667	\\
1.95e-09	0.984	0	0	\\
2e-09	0.984	0	0	\\
2.05e-09	0.984	0	0	\\
2.1e-09	0.984	0	0	\\
2.15e-09	0.984	0	0	\\
2.2e-09	0.984	0	0	\\
2.25e-09	0.984	0	0	\\
2.3e-09	0.984	0	0	\\
2.35e-09	0.984	0	0	\\
2.4e-09	0.984	0	0	\\
2.45e-09	0.984	0	0	\\
2.5e-09	0.984	0	0	\\
2.55e-09	0.984	0	0	\\
2.6e-09	0.984	0	0	\\
2.65e-09	0.984	0	0	\\
2.7e-09	0.984	0	0	\\
2.75e-09	0.984	0	0	\\
2.8e-09	0.984	0	0	\\
2.85e-09	0.984	0	0	\\
2.9e-09	0.984	0	0	\\
2.95e-09	0.984	0	0	\\
3e-09	0.984	0	0	\\
3.05e-09	0.984	0	0	\\
3.1e-09	0.984	0	0	\\
3.15e-09	0.984	0	0	\\
3.2e-09	0.984	0	0	\\
3.25e-09	0.984	0	0	\\
3.3e-09	0.984	0	0	\\
3.35e-09	0.984	0	0	\\
3.4e-09	0.984	0.00666666666666667	0.00666666666666667	\\
3.45e-09	0.984	0.00666666666666667	0.00666666666666667	\\
3.5e-09	0.984	0.00666666666666667	0.00666666666666667	\\
3.55e-09	0.984	0.00666666666666667	0.00666666666666667	\\
3.6e-09	0.984	0.00666666666666667	0.00666666666666667	\\
3.65e-09	0.984	0.00666666666666667	0.00666666666666667	\\
3.7e-09	0.984	0.00666666666666667	0.00666666666666667	\\
3.75e-09	0.984	0.00666666666666667	0.00666666666666667	\\
3.8e-09	0.984	0.00666666666666667	0.00666666666666667	\\
3.85e-09	0.984	0.00222222222222222	0.00222222222222222	\\
3.9e-09	0.984	-0.00444444444444444	-0.00444444444444444	\\
3.95e-09	0.984	-0.00444444444444444	-0.00444444444444444	\\
4e-09	0.984	-0.00444444444444444	-0.00444444444444444	\\
4.05e-09	0.984	-0.00444444444444444	-0.00444444444444444	\\
4.1e-09	0.984	-0.00444444444444444	-0.00444444444444444	\\
4.15e-09	0.984	-0.00444444444444444	-0.00444444444444444	\\
4.2e-09	0.984	-0.00444444444444444	-0.00444444444444444	\\
4.25e-09	0.984	-0.00444444444444444	-0.00444444444444444	\\
4.3e-09	0.984	-0.00444444444444444	-0.00444444444444444	\\
4.35e-09	0.984	0	0	\\
4.4e-09	0.984	0	0	\\
4.45e-09	0.984	0	0	\\
4.5e-09	0.984	0	0	\\
4.55e-09	0.984	0	0	\\
4.6e-09	0.984	0	0	\\
4.65e-09	0.984	0	0	\\
4.7e-09	0.984	0	0	\\
4.75e-09	0.984	0	0	\\
4.8e-09	0.984	0	0	\\
4.85e-09	0.984	0	0	\\
4.9e-09	0.984	0	0	\\
4.95e-09	0.984	0	0	\\
5e-09	0.984	0	0	\\
5e-09	0.984	0	nan	\\
5e-09	0.984	-0.00666666666666667	0.00166666666666667	\\
5e-09	0.984	-0.00666666666666667	nan	\\
0	0.996	-0.00666666666666667	nan	\\
0	0.996	0	0.00166666666666667	\\
0	0.996	0	0	\\
5e-11	0.996	0	0	\\
1e-10	0.996	0	0	\\
1.5e-10	0.996	0	0	\\
2e-10	0.996	0	0	\\
2.5e-10	0.996	0	0	\\
3e-10	0.996	0	0	\\
3.5e-10	0.996	0	0	\\
4e-10	0.996	0	0	\\
4.5e-10	0.996	0	0	\\
5e-10	0.996	0	0	\\
5.5e-10	0.996	0	0	\\
6e-10	0.996	0	0	\\
6.5e-10	0.996	0	0	\\
7e-10	0.996	0	0	\\
7.5e-10	0.996	0	0	\\
8e-10	0.996	0	0	\\
8.5e-10	0.996	0	0	\\
9e-10	0.996	0	0	\\
9.5e-10	0.996	0	0	\\
1e-09	0.996	0.01	0.01	\\
1.05e-09	0.996	0.01	0.01	\\
1.1e-09	0.996	0.01	0.01	\\
1.15e-09	0.996	0.01	0.01	\\
1.2e-09	0.996	0.01	0.01	\\
1.25e-09	0.996	0.01	0.01	\\
1.3e-09	0.996	0.01	0.01	\\
1.35e-09	0.996	0.01	0.01	\\
1.4e-09	0.996	0.01	0.01	\\
1.45e-09	0.996	0.00333333333333333	0.00333333333333333	\\
1.5e-09	0.996	-0.00666666666666667	-0.00666666666666667	\\
1.55e-09	0.996	-0.00666666666666667	-0.00666666666666667	\\
1.6e-09	0.996	-0.00666666666666667	-0.00666666666666667	\\
1.65e-09	0.996	-0.00666666666666667	-0.00666666666666667	\\
1.7e-09	0.996	-0.00666666666666667	-0.00666666666666667	\\
1.75e-09	0.996	-0.00666666666666667	-0.00666666666666667	\\
1.8e-09	0.996	-0.00666666666666667	-0.00666666666666667	\\
1.85e-09	0.996	-0.00666666666666667	-0.00666666666666667	\\
1.9e-09	0.996	-0.00666666666666667	-0.00666666666666667	\\
1.95e-09	0.996	0	0	\\
2e-09	0.996	0	0	\\
2.05e-09	0.996	0	0	\\
2.1e-09	0.996	0	0	\\
2.15e-09	0.996	0	0	\\
2.2e-09	0.996	0	0	\\
2.25e-09	0.996	0	0	\\
2.3e-09	0.996	0	0	\\
2.35e-09	0.996	0	0	\\
2.4e-09	0.996	0	0	\\
2.45e-09	0.996	0	0	\\
2.5e-09	0.996	0	0	\\
2.55e-09	0.996	0	0	\\
2.6e-09	0.996	0	0	\\
2.65e-09	0.996	0	0	\\
2.7e-09	0.996	0	0	\\
2.75e-09	0.996	0	0	\\
2.8e-09	0.996	0	0	\\
2.85e-09	0.996	0	0	\\
2.9e-09	0.996	0	0	\\
2.95e-09	0.996	0	0	\\
3e-09	0.996	0	0	\\
3.05e-09	0.996	0	0	\\
3.1e-09	0.996	0	0	\\
3.15e-09	0.996	0	0	\\
3.2e-09	0.996	0	0	\\
3.25e-09	0.996	0	0	\\
3.3e-09	0.996	0	0	\\
3.35e-09	0.996	0	0	\\
3.4e-09	0.996	0.00666666666666667	0.00666666666666667	\\
3.45e-09	0.996	0.00666666666666667	0.00666666666666667	\\
3.5e-09	0.996	0.00666666666666667	0.00666666666666667	\\
3.55e-09	0.996	0.00666666666666667	0.00666666666666667	\\
3.6e-09	0.996	0.00666666666666667	0.00666666666666667	\\
3.65e-09	0.996	0.00666666666666667	0.00666666666666667	\\
3.7e-09	0.996	0.00666666666666667	0.00666666666666667	\\
3.75e-09	0.996	0.00666666666666667	0.00666666666666667	\\
3.8e-09	0.996	0.00666666666666667	0.00666666666666667	\\
3.85e-09	0.996	0.00222222222222222	0.00222222222222222	\\
3.9e-09	0.996	-0.00444444444444444	-0.00444444444444444	\\
3.95e-09	0.996	-0.00444444444444444	-0.00444444444444444	\\
4e-09	0.996	-0.00444444444444444	-0.00444444444444444	\\
4.05e-09	0.996	-0.00444444444444444	-0.00444444444444444	\\
4.1e-09	0.996	-0.00444444444444444	-0.00444444444444444	\\
4.15e-09	0.996	-0.00444444444444444	-0.00444444444444444	\\
4.2e-09	0.996	-0.00444444444444444	-0.00444444444444444	\\
4.25e-09	0.996	-0.00444444444444444	-0.00444444444444444	\\
4.3e-09	0.996	-0.00444444444444444	-0.00444444444444444	\\
4.35e-09	0.996	0	0	\\
4.4e-09	0.996	0	0	\\
4.45e-09	0.996	0	0	\\
4.5e-09	0.996	0	0	\\
4.55e-09	0.996	0	0	\\
4.6e-09	0.996	0	0	\\
4.65e-09	0.996	0	0	\\
4.7e-09	0.996	0	0	\\
4.75e-09	0.996	0	0	\\
4.8e-09	0.996	0	0	\\
4.85e-09	0.996	0	0	\\
4.9e-09	0.996	0	0	\\
4.95e-09	0.996	0	0	\\
5e-09	0.996	0	0	\\
5e-09	0.996	0	nan	\\
5e-09	0.996	-0.00666666666666667	0.00166666666666667	\\
5e-09	0.996	-0.00666666666666667	nan	\\
0	1.008	-0.00666666666666667	nan	\\
0	1.008	0	0.00166666666666667	\\
0	1.008	0	0	\\
5e-11	1.008	0	0	\\
1e-10	1.008	0	0	\\
1.5e-10	1.008	0	0	\\
2e-10	1.008	0	0	\\
2.5e-10	1.008	0	0	\\
3e-10	1.008	0	0	\\
3.5e-10	1.008	0	0	\\
4e-10	1.008	0	0	\\
4.5e-10	1.008	0	0	\\
5e-10	1.008	0	0	\\
5.5e-10	1.008	0	0	\\
6e-10	1.008	0	0	\\
6.5e-10	1.008	0	0	\\
7e-10	1.008	0	0	\\
7.5e-10	1.008	0	0	\\
8e-10	1.008	0	0	\\
8.5e-10	1.008	0	0	\\
9e-10	1.008	0	0	\\
9.5e-10	1.008	0	0	\\
1e-09	1.008	0	0	\\
1.05e-09	1.008	0.01	0.01	\\
1.1e-09	1.008	0.01	0.01	\\
1.15e-09	1.008	0.01	0.01	\\
1.2e-09	1.008	0.01	0.01	\\
1.25e-09	1.008	0.01	0.01	\\
1.3e-09	1.008	0.01	0.01	\\
1.35e-09	1.008	0.01	0.01	\\
1.4e-09	1.008	0.00333333333333333	0.00333333333333333	\\
1.45e-09	1.008	0.00333333333333333	0.00333333333333333	\\
1.5e-09	1.008	0.00333333333333333	0.00333333333333333	\\
1.55e-09	1.008	-0.00666666666666667	-0.00666666666666667	\\
1.6e-09	1.008	-0.00666666666666667	-0.00666666666666667	\\
1.65e-09	1.008	-0.00666666666666667	-0.00666666666666667	\\
1.7e-09	1.008	-0.00666666666666667	-0.00666666666666667	\\
1.75e-09	1.008	-0.00666666666666667	-0.00666666666666667	\\
1.8e-09	1.008	-0.00666666666666667	-0.00666666666666667	\\
1.85e-09	1.008	-0.00666666666666667	-0.00666666666666667	\\
1.9e-09	1.008	0	0	\\
1.95e-09	1.008	0	0	\\
2e-09	1.008	0	0	\\
2.05e-09	1.008	0	0	\\
2.1e-09	1.008	0	0	\\
2.15e-09	1.008	0	0	\\
2.2e-09	1.008	0	0	\\
2.25e-09	1.008	0	0	\\
2.3e-09	1.008	0	0	\\
2.35e-09	1.008	0	0	\\
2.4e-09	1.008	0	0	\\
2.45e-09	1.008	0	0	\\
2.5e-09	1.008	0	0	\\
2.55e-09	1.008	0	0	\\
2.6e-09	1.008	0	0	\\
2.65e-09	1.008	0	0	\\
2.7e-09	1.008	0	0	\\
2.75e-09	1.008	0	0	\\
2.8e-09	1.008	0	0	\\
2.85e-09	1.008	0	0	\\
2.9e-09	1.008	0	0	\\
2.95e-09	1.008	0	0	\\
3e-09	1.008	0	0	\\
3.05e-09	1.008	0	0	\\
3.1e-09	1.008	0	0	\\
3.15e-09	1.008	0	0	\\
3.2e-09	1.008	0	0	\\
3.25e-09	1.008	0	0	\\
3.3e-09	1.008	0	0	\\
3.35e-09	1.008	0	0	\\
3.4e-09	1.008	0	0	\\
3.45e-09	1.008	0.00666666666666667	0.00666666666666667	\\
3.5e-09	1.008	0.00666666666666667	0.00666666666666667	\\
3.55e-09	1.008	0.00666666666666667	0.00666666666666667	\\
3.6e-09	1.008	0.00666666666666667	0.00666666666666667	\\
3.65e-09	1.008	0.00666666666666667	0.00666666666666667	\\
3.7e-09	1.008	0.00666666666666667	0.00666666666666667	\\
3.75e-09	1.008	0.00666666666666667	0.00666666666666667	\\
3.8e-09	1.008	0.00222222222222222	0.00222222222222222	\\
3.85e-09	1.008	0.00222222222222222	0.00222222222222222	\\
3.9e-09	1.008	0.00222222222222222	0.00222222222222222	\\
3.95e-09	1.008	-0.00444444444444444	-0.00444444444444444	\\
4e-09	1.008	-0.00444444444444444	-0.00444444444444444	\\
4.05e-09	1.008	-0.00444444444444444	-0.00444444444444444	\\
4.1e-09	1.008	-0.00444444444444444	-0.00444444444444444	\\
4.15e-09	1.008	-0.00444444444444444	-0.00444444444444444	\\
4.2e-09	1.008	-0.00444444444444444	-0.00444444444444444	\\
4.25e-09	1.008	-0.00444444444444444	-0.00444444444444444	\\
4.3e-09	1.008	0	0	\\
4.35e-09	1.008	0	0	\\
4.4e-09	1.008	0	0	\\
4.45e-09	1.008	0	0	\\
4.5e-09	1.008	0	0	\\
4.55e-09	1.008	0	0	\\
4.6e-09	1.008	0	0	\\
4.65e-09	1.008	0	0	\\
4.7e-09	1.008	0	0	\\
4.75e-09	1.008	0	0	\\
4.8e-09	1.008	0	0	\\
4.85e-09	1.008	0	0	\\
4.9e-09	1.008	0	0	\\
4.95e-09	1.008	0	0	\\
5e-09	1.008	0	0	\\
5e-09	1.008	0	nan	\\
5e-09	1.008	-0.00666666666666667	0.00166666666666667	\\
5e-09	1.008	-0.00666666666666667	nan	\\
0	1.02	-0.00666666666666667	nan	\\
0	1.02	0	0.00166666666666667	\\
0	1.02	0	0	\\
5e-11	1.02	0	0	\\
1e-10	1.02	0	0	\\
1.5e-10	1.02	0	0	\\
2e-10	1.02	0	0	\\
2.5e-10	1.02	0	0	\\
3e-10	1.02	0	0	\\
3.5e-10	1.02	0	0	\\
4e-10	1.02	0	0	\\
4.5e-10	1.02	0	0	\\
5e-10	1.02	0	0	\\
5.5e-10	1.02	0	0	\\
6e-10	1.02	0	0	\\
6.5e-10	1.02	0	0	\\
7e-10	1.02	0	0	\\
7.5e-10	1.02	0	0	\\
8e-10	1.02	0	0	\\
8.5e-10	1.02	0	0	\\
9e-10	1.02	0	0	\\
9.5e-10	1.02	0	0	\\
1e-09	1.02	0	0	\\
1.05e-09	1.02	0.01	0.01	\\
1.1e-09	1.02	0.01	0.01	\\
1.15e-09	1.02	0.01	0.01	\\
1.2e-09	1.02	0.01	0.01	\\
1.25e-09	1.02	0.01	0.01	\\
1.3e-09	1.02	0.01	0.01	\\
1.35e-09	1.02	0.01	0.01	\\
1.4e-09	1.02	0.00333333333333333	0.00333333333333333	\\
1.45e-09	1.02	0.00333333333333333	0.00333333333333333	\\
1.5e-09	1.02	0.00333333333333333	0.00333333333333333	\\
1.55e-09	1.02	-0.00666666666666667	-0.00666666666666667	\\
1.6e-09	1.02	-0.00666666666666667	-0.00666666666666667	\\
1.65e-09	1.02	-0.00666666666666667	-0.00666666666666667	\\
1.7e-09	1.02	-0.00666666666666667	-0.00666666666666667	\\
1.75e-09	1.02	-0.00666666666666667	-0.00666666666666667	\\
1.8e-09	1.02	-0.00666666666666667	-0.00666666666666667	\\
1.85e-09	1.02	-0.00666666666666667	-0.00666666666666667	\\
1.9e-09	1.02	0	0	\\
1.95e-09	1.02	0	0	\\
2e-09	1.02	0	0	\\
2.05e-09	1.02	0	0	\\
2.1e-09	1.02	0	0	\\
2.15e-09	1.02	0	0	\\
2.2e-09	1.02	0	0	\\
2.25e-09	1.02	0	0	\\
2.3e-09	1.02	0	0	\\
2.35e-09	1.02	0	0	\\
2.4e-09	1.02	0	0	\\
2.45e-09	1.02	0	0	\\
2.5e-09	1.02	0	0	\\
2.55e-09	1.02	0	0	\\
2.6e-09	1.02	0	0	\\
2.65e-09	1.02	0	0	\\
2.7e-09	1.02	0	0	\\
2.75e-09	1.02	0	0	\\
2.8e-09	1.02	0	0	\\
2.85e-09	1.02	0	0	\\
2.9e-09	1.02	0	0	\\
2.95e-09	1.02	0	0	\\
3e-09	1.02	0	0	\\
3.05e-09	1.02	0	0	\\
3.1e-09	1.02	0	0	\\
3.15e-09	1.02	0	0	\\
3.2e-09	1.02	0	0	\\
3.25e-09	1.02	0	0	\\
3.3e-09	1.02	0	0	\\
3.35e-09	1.02	0	0	\\
3.4e-09	1.02	0	0	\\
3.45e-09	1.02	0.00666666666666667	0.00666666666666667	\\
3.5e-09	1.02	0.00666666666666667	0.00666666666666667	\\
3.55e-09	1.02	0.00666666666666667	0.00666666666666667	\\
3.6e-09	1.02	0.00666666666666667	0.00666666666666667	\\
3.65e-09	1.02	0.00666666666666667	0.00666666666666667	\\
3.7e-09	1.02	0.00666666666666667	0.00666666666666667	\\
3.75e-09	1.02	0.00666666666666667	0.00666666666666667	\\
3.8e-09	1.02	0.00222222222222222	0.00222222222222222	\\
3.85e-09	1.02	0.00222222222222222	0.00222222222222222	\\
3.9e-09	1.02	0.00222222222222222	0.00222222222222222	\\
3.95e-09	1.02	-0.00444444444444444	-0.00444444444444444	\\
4e-09	1.02	-0.00444444444444444	-0.00444444444444444	\\
4.05e-09	1.02	-0.00444444444444444	-0.00444444444444444	\\
4.1e-09	1.02	-0.00444444444444444	-0.00444444444444444	\\
4.15e-09	1.02	-0.00444444444444444	-0.00444444444444444	\\
4.2e-09	1.02	-0.00444444444444444	-0.00444444444444444	\\
4.25e-09	1.02	-0.00444444444444444	-0.00444444444444444	\\
4.3e-09	1.02	0	0	\\
4.35e-09	1.02	0	0	\\
4.4e-09	1.02	0	0	\\
4.45e-09	1.02	0	0	\\
4.5e-09	1.02	0	0	\\
4.55e-09	1.02	0	0	\\
4.6e-09	1.02	0	0	\\
4.65e-09	1.02	0	0	\\
4.7e-09	1.02	0	0	\\
4.75e-09	1.02	0	0	\\
4.8e-09	1.02	0	0	\\
4.85e-09	1.02	0	0	\\
4.9e-09	1.02	0	0	\\
4.95e-09	1.02	0	0	\\
5e-09	1.02	0	0	\\
5e-09	1.02	0	nan	\\
5e-09	1.02	-0.00666666666666667	0.00166666666666667	\\
5e-09	1.02	-0.00666666666666667	nan	\\
0	1.032	-0.00666666666666667	nan	\\
0	1.032	0	0.00166666666666667	\\
0	1.032	0	0	\\
5e-11	1.032	0	0	\\
1e-10	1.032	0	0	\\
1.5e-10	1.032	0	0	\\
2e-10	1.032	0	0	\\
2.5e-10	1.032	0	0	\\
3e-10	1.032	0	0	\\
3.5e-10	1.032	0	0	\\
4e-10	1.032	0	0	\\
4.5e-10	1.032	0	0	\\
5e-10	1.032	0	0	\\
5.5e-10	1.032	0	0	\\
6e-10	1.032	0	0	\\
6.5e-10	1.032	0	0	\\
7e-10	1.032	0	0	\\
7.5e-10	1.032	0	0	\\
8e-10	1.032	0	0	\\
8.5e-10	1.032	0	0	\\
9e-10	1.032	0	0	\\
9.5e-10	1.032	0	0	\\
1e-09	1.032	0	0	\\
1.05e-09	1.032	0.01	0.01	\\
1.1e-09	1.032	0.01	0.01	\\
1.15e-09	1.032	0.01	0.01	\\
1.2e-09	1.032	0.01	0.01	\\
1.25e-09	1.032	0.01	0.01	\\
1.3e-09	1.032	0.01	0.01	\\
1.35e-09	1.032	0.01	0.01	\\
1.4e-09	1.032	0.00333333333333333	0.00333333333333333	\\
1.45e-09	1.032	0.00333333333333333	0.00333333333333333	\\
1.5e-09	1.032	0.00333333333333333	0.00333333333333333	\\
1.55e-09	1.032	-0.00666666666666667	-0.00666666666666667	\\
1.6e-09	1.032	-0.00666666666666667	-0.00666666666666667	\\
1.65e-09	1.032	-0.00666666666666667	-0.00666666666666667	\\
1.7e-09	1.032	-0.00666666666666667	-0.00666666666666667	\\
1.75e-09	1.032	-0.00666666666666667	-0.00666666666666667	\\
1.8e-09	1.032	-0.00666666666666667	-0.00666666666666667	\\
1.85e-09	1.032	-0.00666666666666667	-0.00666666666666667	\\
1.9e-09	1.032	0	0	\\
1.95e-09	1.032	0	0	\\
2e-09	1.032	0	0	\\
2.05e-09	1.032	0	0	\\
2.1e-09	1.032	0	0	\\
2.15e-09	1.032	0	0	\\
2.2e-09	1.032	0	0	\\
2.25e-09	1.032	0	0	\\
2.3e-09	1.032	0	0	\\
2.35e-09	1.032	0	0	\\
2.4e-09	1.032	0	0	\\
2.45e-09	1.032	0	0	\\
2.5e-09	1.032	0	0	\\
2.55e-09	1.032	0	0	\\
2.6e-09	1.032	0	0	\\
2.65e-09	1.032	0	0	\\
2.7e-09	1.032	0	0	\\
2.75e-09	1.032	0	0	\\
2.8e-09	1.032	0	0	\\
2.85e-09	1.032	0	0	\\
2.9e-09	1.032	0	0	\\
2.95e-09	1.032	0	0	\\
3e-09	1.032	0	0	\\
3.05e-09	1.032	0	0	\\
3.1e-09	1.032	0	0	\\
3.15e-09	1.032	0	0	\\
3.2e-09	1.032	0	0	\\
3.25e-09	1.032	0	0	\\
3.3e-09	1.032	0	0	\\
3.35e-09	1.032	0	0	\\
3.4e-09	1.032	0	0	\\
3.45e-09	1.032	0.00666666666666667	0.00666666666666667	\\
3.5e-09	1.032	0.00666666666666667	0.00666666666666667	\\
3.55e-09	1.032	0.00666666666666667	0.00666666666666667	\\
3.6e-09	1.032	0.00666666666666667	0.00666666666666667	\\
3.65e-09	1.032	0.00666666666666667	0.00666666666666667	\\
3.7e-09	1.032	0.00666666666666667	0.00666666666666667	\\
3.75e-09	1.032	0.00666666666666667	0.00666666666666667	\\
3.8e-09	1.032	0.00222222222222222	0.00222222222222222	\\
3.85e-09	1.032	0.00222222222222222	0.00222222222222222	\\
3.9e-09	1.032	0.00222222222222222	0.00222222222222222	\\
3.95e-09	1.032	-0.00444444444444444	-0.00444444444444444	\\
4e-09	1.032	-0.00444444444444444	-0.00444444444444444	\\
4.05e-09	1.032	-0.00444444444444444	-0.00444444444444444	\\
4.1e-09	1.032	-0.00444444444444444	-0.00444444444444444	\\
4.15e-09	1.032	-0.00444444444444444	-0.00444444444444444	\\
4.2e-09	1.032	-0.00444444444444444	-0.00444444444444444	\\
4.25e-09	1.032	-0.00444444444444444	-0.00444444444444444	\\
4.3e-09	1.032	0	0	\\
4.35e-09	1.032	0	0	\\
4.4e-09	1.032	0	0	\\
4.45e-09	1.032	0	0	\\
4.5e-09	1.032	0	0	\\
4.55e-09	1.032	0	0	\\
4.6e-09	1.032	0	0	\\
4.65e-09	1.032	0	0	\\
4.7e-09	1.032	0	0	\\
4.75e-09	1.032	0	0	\\
4.8e-09	1.032	0	0	\\
4.85e-09	1.032	0	0	\\
4.9e-09	1.032	0	0	\\
4.95e-09	1.032	0	0	\\
5e-09	1.032	0	0	\\
5e-09	1.032	0	nan	\\
5e-09	1.032	-0.00666666666666667	0.00166666666666667	\\
5e-09	1.032	-0.00666666666666667	nan	\\
0	1.044	-0.00666666666666667	nan	\\
0	1.044	0	0.00166666666666667	\\
0	1.044	0	0	\\
5e-11	1.044	0	0	\\
1e-10	1.044	0	0	\\
1.5e-10	1.044	0	0	\\
2e-10	1.044	0	0	\\
2.5e-10	1.044	0	0	\\
3e-10	1.044	0	0	\\
3.5e-10	1.044	0	0	\\
4e-10	1.044	0	0	\\
4.5e-10	1.044	0	0	\\
5e-10	1.044	0	0	\\
5.5e-10	1.044	0	0	\\
6e-10	1.044	0	0	\\
6.5e-10	1.044	0	0	\\
7e-10	1.044	0	0	\\
7.5e-10	1.044	0	0	\\
8e-10	1.044	0	0	\\
8.5e-10	1.044	0	0	\\
9e-10	1.044	0	0	\\
9.5e-10	1.044	0	0	\\
1e-09	1.044	0	0	\\
1.05e-09	1.044	0.01	0.01	\\
1.1e-09	1.044	0.01	0.01	\\
1.15e-09	1.044	0.01	0.01	\\
1.2e-09	1.044	0.01	0.01	\\
1.25e-09	1.044	0.01	0.01	\\
1.3e-09	1.044	0.01	0.01	\\
1.35e-09	1.044	0.01	0.01	\\
1.4e-09	1.044	0.00333333333333333	0.00333333333333333	\\
1.45e-09	1.044	0.00333333333333333	0.00333333333333333	\\
1.5e-09	1.044	0.00333333333333333	0.00333333333333333	\\
1.55e-09	1.044	-0.00666666666666667	-0.00666666666666667	\\
1.6e-09	1.044	-0.00666666666666667	-0.00666666666666667	\\
1.65e-09	1.044	-0.00666666666666667	-0.00666666666666667	\\
1.7e-09	1.044	-0.00666666666666667	-0.00666666666666667	\\
1.75e-09	1.044	-0.00666666666666667	-0.00666666666666667	\\
1.8e-09	1.044	-0.00666666666666667	-0.00666666666666667	\\
1.85e-09	1.044	-0.00666666666666667	-0.00666666666666667	\\
1.9e-09	1.044	0	0	\\
1.95e-09	1.044	0	0	\\
2e-09	1.044	0	0	\\
2.05e-09	1.044	0	0	\\
2.1e-09	1.044	0	0	\\
2.15e-09	1.044	0	0	\\
2.2e-09	1.044	0	0	\\
2.25e-09	1.044	0	0	\\
2.3e-09	1.044	0	0	\\
2.35e-09	1.044	0	0	\\
2.4e-09	1.044	0	0	\\
2.45e-09	1.044	0	0	\\
2.5e-09	1.044	0	0	\\
2.55e-09	1.044	0	0	\\
2.6e-09	1.044	0	0	\\
2.65e-09	1.044	0	0	\\
2.7e-09	1.044	0	0	\\
2.75e-09	1.044	0	0	\\
2.8e-09	1.044	0	0	\\
2.85e-09	1.044	0	0	\\
2.9e-09	1.044	0	0	\\
2.95e-09	1.044	0	0	\\
3e-09	1.044	0	0	\\
3.05e-09	1.044	0	0	\\
3.1e-09	1.044	0	0	\\
3.15e-09	1.044	0	0	\\
3.2e-09	1.044	0	0	\\
3.25e-09	1.044	0	0	\\
3.3e-09	1.044	0	0	\\
3.35e-09	1.044	0	0	\\
3.4e-09	1.044	0	0	\\
3.45e-09	1.044	0.00666666666666667	0.00666666666666667	\\
3.5e-09	1.044	0.00666666666666667	0.00666666666666667	\\
3.55e-09	1.044	0.00666666666666667	0.00666666666666667	\\
3.6e-09	1.044	0.00666666666666667	0.00666666666666667	\\
3.65e-09	1.044	0.00666666666666667	0.00666666666666667	\\
3.7e-09	1.044	0.00666666666666667	0.00666666666666667	\\
3.75e-09	1.044	0.00666666666666667	0.00666666666666667	\\
3.8e-09	1.044	0.00222222222222222	0.00222222222222222	\\
3.85e-09	1.044	0.00222222222222222	0.00222222222222222	\\
3.9e-09	1.044	0.00222222222222222	0.00222222222222222	\\
3.95e-09	1.044	-0.00444444444444444	-0.00444444444444444	\\
4e-09	1.044	-0.00444444444444444	-0.00444444444444444	\\
4.05e-09	1.044	-0.00444444444444444	-0.00444444444444444	\\
4.1e-09	1.044	-0.00444444444444444	-0.00444444444444444	\\
4.15e-09	1.044	-0.00444444444444444	-0.00444444444444444	\\
4.2e-09	1.044	-0.00444444444444444	-0.00444444444444444	\\
4.25e-09	1.044	-0.00444444444444444	-0.00444444444444444	\\
4.3e-09	1.044	0	0	\\
4.35e-09	1.044	0	0	\\
4.4e-09	1.044	0	0	\\
4.45e-09	1.044	0	0	\\
4.5e-09	1.044	0	0	\\
4.55e-09	1.044	0	0	\\
4.6e-09	1.044	0	0	\\
4.65e-09	1.044	0	0	\\
4.7e-09	1.044	0	0	\\
4.75e-09	1.044	0	0	\\
4.8e-09	1.044	0	0	\\
4.85e-09	1.044	0	0	\\
4.9e-09	1.044	0	0	\\
4.95e-09	1.044	0	0	\\
5e-09	1.044	0	0	\\
5e-09	1.044	0	nan	\\
5e-09	1.044	-0.00666666666666667	0.00166666666666667	\\
5e-09	1.044	-0.00666666666666667	nan	\\
0	1.056	-0.00666666666666667	nan	\\
0	1.056	0	0.00166666666666667	\\
0	1.056	0	0	\\
5e-11	1.056	0	0	\\
1e-10	1.056	0	0	\\
1.5e-10	1.056	0	0	\\
2e-10	1.056	0	0	\\
2.5e-10	1.056	0	0	\\
3e-10	1.056	0	0	\\
3.5e-10	1.056	0	0	\\
4e-10	1.056	0	0	\\
4.5e-10	1.056	0	0	\\
5e-10	1.056	0	0	\\
5.5e-10	1.056	0	0	\\
6e-10	1.056	0	0	\\
6.5e-10	1.056	0	0	\\
7e-10	1.056	0	0	\\
7.5e-10	1.056	0	0	\\
8e-10	1.056	0	0	\\
8.5e-10	1.056	0	0	\\
9e-10	1.056	0	0	\\
9.5e-10	1.056	0	0	\\
1e-09	1.056	0	0	\\
1.05e-09	1.056	0	0	\\
1.1e-09	1.056	0.01	0.01	\\
1.15e-09	1.056	0.01	0.01	\\
1.2e-09	1.056	0.01	0.01	\\
1.25e-09	1.056	0.01	0.01	\\
1.3e-09	1.056	0.01	0.01	\\
1.35e-09	1.056	0.00333333333333333	0.00333333333333333	\\
1.4e-09	1.056	0.00333333333333333	0.00333333333333333	\\
1.45e-09	1.056	0.00333333333333333	0.00333333333333333	\\
1.5e-09	1.056	0.00333333333333333	0.00333333333333333	\\
1.55e-09	1.056	0.00333333333333333	0.00333333333333333	\\
1.6e-09	1.056	-0.00666666666666667	-0.00666666666666667	\\
1.65e-09	1.056	-0.00666666666666667	-0.00666666666666667	\\
1.7e-09	1.056	-0.00666666666666667	-0.00666666666666667	\\
1.75e-09	1.056	-0.00666666666666667	-0.00666666666666667	\\
1.8e-09	1.056	-0.00666666666666667	-0.00666666666666667	\\
1.85e-09	1.056	0	0	\\
1.9e-09	1.056	0	0	\\
1.95e-09	1.056	0	0	\\
2e-09	1.056	0	0	\\
2.05e-09	1.056	0	0	\\
2.1e-09	1.056	0	0	\\
2.15e-09	1.056	0	0	\\
2.2e-09	1.056	0	0	\\
2.25e-09	1.056	0	0	\\
2.3e-09	1.056	0	0	\\
2.35e-09	1.056	0	0	\\
2.4e-09	1.056	0	0	\\
2.45e-09	1.056	0	0	\\
2.5e-09	1.056	0	0	\\
2.55e-09	1.056	0	0	\\
2.6e-09	1.056	0	0	\\
2.65e-09	1.056	0	0	\\
2.7e-09	1.056	0	0	\\
2.75e-09	1.056	0	0	\\
2.8e-09	1.056	0	0	\\
2.85e-09	1.056	0	0	\\
2.9e-09	1.056	0	0	\\
2.95e-09	1.056	0	0	\\
3e-09	1.056	0	0	\\
3.05e-09	1.056	0	0	\\
3.1e-09	1.056	0	0	\\
3.15e-09	1.056	0	0	\\
3.2e-09	1.056	0	0	\\
3.25e-09	1.056	0	0	\\
3.3e-09	1.056	0	0	\\
3.35e-09	1.056	0	0	\\
3.4e-09	1.056	0	0	\\
3.45e-09	1.056	0	0	\\
3.5e-09	1.056	0.00666666666666667	0.00666666666666667	\\
3.55e-09	1.056	0.00666666666666667	0.00666666666666667	\\
3.6e-09	1.056	0.00666666666666667	0.00666666666666667	\\
3.65e-09	1.056	0.00666666666666667	0.00666666666666667	\\
3.7e-09	1.056	0.00666666666666667	0.00666666666666667	\\
3.75e-09	1.056	0.00222222222222222	0.00222222222222222	\\
3.8e-09	1.056	0.00222222222222222	0.00222222222222222	\\
3.85e-09	1.056	0.00222222222222222	0.00222222222222222	\\
3.9e-09	1.056	0.00222222222222222	0.00222222222222222	\\
3.95e-09	1.056	0.00222222222222222	0.00222222222222222	\\
4e-09	1.056	-0.00444444444444444	-0.00444444444444444	\\
4.05e-09	1.056	-0.00444444444444444	-0.00444444444444444	\\
4.1e-09	1.056	-0.00444444444444444	-0.00444444444444444	\\
4.15e-09	1.056	-0.00444444444444444	-0.00444444444444444	\\
4.2e-09	1.056	-0.00444444444444444	-0.00444444444444444	\\
4.25e-09	1.056	0	0	\\
4.3e-09	1.056	0	0	\\
4.35e-09	1.056	0	0	\\
4.4e-09	1.056	0	0	\\
4.45e-09	1.056	0	0	\\
4.5e-09	1.056	0	0	\\
4.55e-09	1.056	0	0	\\
4.6e-09	1.056	0	0	\\
4.65e-09	1.056	0	0	\\
4.7e-09	1.056	0	0	\\
4.75e-09	1.056	0	0	\\
4.8e-09	1.056	0	0	\\
4.85e-09	1.056	0	0	\\
4.9e-09	1.056	0	0	\\
4.95e-09	1.056	0	0	\\
5e-09	1.056	0	0	\\
5e-09	1.056	0	nan	\\
5e-09	1.056	-0.00666666666666667	0.00166666666666667	\\
5e-09	1.056	-0.00666666666666667	nan	\\
0	1.068	-0.00666666666666667	nan	\\
0	1.068	0	0.00166666666666667	\\
0	1.068	0	0	\\
5e-11	1.068	0	0	\\
1e-10	1.068	0	0	\\
1.5e-10	1.068	0	0	\\
2e-10	1.068	0	0	\\
2.5e-10	1.068	0	0	\\
3e-10	1.068	0	0	\\
3.5e-10	1.068	0	0	\\
4e-10	1.068	0	0	\\
4.5e-10	1.068	0	0	\\
5e-10	1.068	0	0	\\
5.5e-10	1.068	0	0	\\
6e-10	1.068	0	0	\\
6.5e-10	1.068	0	0	\\
7e-10	1.068	0	0	\\
7.5e-10	1.068	0	0	\\
8e-10	1.068	0	0	\\
8.5e-10	1.068	0	0	\\
9e-10	1.068	0	0	\\
9.5e-10	1.068	0	0	\\
1e-09	1.068	0	0	\\
1.05e-09	1.068	0	0	\\
1.1e-09	1.068	0.01	0.01	\\
1.15e-09	1.068	0.01	0.01	\\
1.2e-09	1.068	0.01	0.01	\\
1.25e-09	1.068	0.01	0.01	\\
1.3e-09	1.068	0.01	0.01	\\
1.35e-09	1.068	0.00333333333333333	0.00333333333333333	\\
1.4e-09	1.068	0.00333333333333333	0.00333333333333333	\\
1.45e-09	1.068	0.00333333333333333	0.00333333333333333	\\
1.5e-09	1.068	0.00333333333333333	0.00333333333333333	\\
1.55e-09	1.068	0.00333333333333333	0.00333333333333333	\\
1.6e-09	1.068	-0.00666666666666667	-0.00666666666666667	\\
1.65e-09	1.068	-0.00666666666666667	-0.00666666666666667	\\
1.7e-09	1.068	-0.00666666666666667	-0.00666666666666667	\\
1.75e-09	1.068	-0.00666666666666667	-0.00666666666666667	\\
1.8e-09	1.068	-0.00666666666666667	-0.00666666666666667	\\
1.85e-09	1.068	0	0	\\
1.9e-09	1.068	0	0	\\
1.95e-09	1.068	0	0	\\
2e-09	1.068	0	0	\\
2.05e-09	1.068	0	0	\\
2.1e-09	1.068	0	0	\\
2.15e-09	1.068	0	0	\\
2.2e-09	1.068	0	0	\\
2.25e-09	1.068	0	0	\\
2.3e-09	1.068	0	0	\\
2.35e-09	1.068	0	0	\\
2.4e-09	1.068	0	0	\\
2.45e-09	1.068	0	0	\\
2.5e-09	1.068	0	0	\\
2.55e-09	1.068	0	0	\\
2.6e-09	1.068	0	0	\\
2.65e-09	1.068	0	0	\\
2.7e-09	1.068	0	0	\\
2.75e-09	1.068	0	0	\\
2.8e-09	1.068	0	0	\\
2.85e-09	1.068	0	0	\\
2.9e-09	1.068	0	0	\\
2.95e-09	1.068	0	0	\\
3e-09	1.068	0	0	\\
3.05e-09	1.068	0	0	\\
3.1e-09	1.068	0	0	\\
3.15e-09	1.068	0	0	\\
3.2e-09	1.068	0	0	\\
3.25e-09	1.068	0	0	\\
3.3e-09	1.068	0	0	\\
3.35e-09	1.068	0	0	\\
3.4e-09	1.068	0	0	\\
3.45e-09	1.068	0	0	\\
3.5e-09	1.068	0.00666666666666667	0.00666666666666667	\\
3.55e-09	1.068	0.00666666666666667	0.00666666666666667	\\
3.6e-09	1.068	0.00666666666666667	0.00666666666666667	\\
3.65e-09	1.068	0.00666666666666667	0.00666666666666667	\\
3.7e-09	1.068	0.00666666666666667	0.00666666666666667	\\
3.75e-09	1.068	0.00222222222222222	0.00222222222222222	\\
3.8e-09	1.068	0.00222222222222222	0.00222222222222222	\\
3.85e-09	1.068	0.00222222222222222	0.00222222222222222	\\
3.9e-09	1.068	0.00222222222222222	0.00222222222222222	\\
3.95e-09	1.068	0.00222222222222222	0.00222222222222222	\\
4e-09	1.068	-0.00444444444444444	-0.00444444444444444	\\
4.05e-09	1.068	-0.00444444444444444	-0.00444444444444444	\\
4.1e-09	1.068	-0.00444444444444444	-0.00444444444444444	\\
4.15e-09	1.068	-0.00444444444444444	-0.00444444444444444	\\
4.2e-09	1.068	-0.00444444444444444	-0.00444444444444444	\\
4.25e-09	1.068	0	0	\\
4.3e-09	1.068	0	0	\\
4.35e-09	1.068	0	0	\\
4.4e-09	1.068	0	0	\\
4.45e-09	1.068	0	0	\\
4.5e-09	1.068	0	0	\\
4.55e-09	1.068	0	0	\\
4.6e-09	1.068	0	0	\\
4.65e-09	1.068	0	0	\\
4.7e-09	1.068	0	0	\\
4.75e-09	1.068	0	0	\\
4.8e-09	1.068	0	0	\\
4.85e-09	1.068	0	0	\\
4.9e-09	1.068	0	0	\\
4.95e-09	1.068	0	0	\\
5e-09	1.068	0	0	\\
5e-09	1.068	0	nan	\\
5e-09	1.068	-0.00666666666666667	0.00166666666666667	\\
5e-09	1.068	-0.00666666666666667	nan	\\
0	1.08	-0.00666666666666667	nan	\\
0	1.08	0	0.00166666666666667	\\
0	1.08	0	0	\\
5e-11	1.08	0	0	\\
1e-10	1.08	0	0	\\
1.5e-10	1.08	0	0	\\
2e-10	1.08	0	0	\\
2.5e-10	1.08	0	0	\\
3e-10	1.08	0	0	\\
3.5e-10	1.08	0	0	\\
4e-10	1.08	0	0	\\
4.5e-10	1.08	0	0	\\
5e-10	1.08	0	0	\\
5.5e-10	1.08	0	0	\\
6e-10	1.08	0	0	\\
6.5e-10	1.08	0	0	\\
7e-10	1.08	0	0	\\
7.5e-10	1.08	0	0	\\
8e-10	1.08	0	0	\\
8.5e-10	1.08	0	0	\\
9e-10	1.08	0	0	\\
9.5e-10	1.08	0	0	\\
1e-09	1.08	0	0	\\
1.05e-09	1.08	0	0	\\
1.1e-09	1.08	0.01	0.01	\\
1.15e-09	1.08	0.01	0.01	\\
1.2e-09	1.08	0.01	0.01	\\
1.25e-09	1.08	0.01	0.01	\\
1.3e-09	1.08	0.01	0.01	\\
1.35e-09	1.08	0.00333333333333333	0.00333333333333333	\\
1.4e-09	1.08	0.00333333333333333	0.00333333333333333	\\
1.45e-09	1.08	0.00333333333333333	0.00333333333333333	\\
1.5e-09	1.08	0.00333333333333333	0.00333333333333333	\\
1.55e-09	1.08	0.00333333333333333	0.00333333333333333	\\
1.6e-09	1.08	-0.00666666666666667	-0.00666666666666667	\\
1.65e-09	1.08	-0.00666666666666667	-0.00666666666666667	\\
1.7e-09	1.08	-0.00666666666666667	-0.00666666666666667	\\
1.75e-09	1.08	-0.00666666666666667	-0.00666666666666667	\\
1.8e-09	1.08	-0.00666666666666667	-0.00666666666666667	\\
1.85e-09	1.08	0	0	\\
1.9e-09	1.08	0	0	\\
1.95e-09	1.08	0	0	\\
2e-09	1.08	0	0	\\
2.05e-09	1.08	0	0	\\
2.1e-09	1.08	0	0	\\
2.15e-09	1.08	0	0	\\
2.2e-09	1.08	0	0	\\
2.25e-09	1.08	0	0	\\
2.3e-09	1.08	0	0	\\
2.35e-09	1.08	0	0	\\
2.4e-09	1.08	0	0	\\
2.45e-09	1.08	0	0	\\
2.5e-09	1.08	0	0	\\
2.55e-09	1.08	0	0	\\
2.6e-09	1.08	0	0	\\
2.65e-09	1.08	0	0	\\
2.7e-09	1.08	0	0	\\
2.75e-09	1.08	0	0	\\
2.8e-09	1.08	0	0	\\
2.85e-09	1.08	0	0	\\
2.9e-09	1.08	0	0	\\
2.95e-09	1.08	0	0	\\
3e-09	1.08	0	0	\\
3.05e-09	1.08	0	0	\\
3.1e-09	1.08	0	0	\\
3.15e-09	1.08	0	0	\\
3.2e-09	1.08	0	0	\\
3.25e-09	1.08	0	0	\\
3.3e-09	1.08	0	0	\\
3.35e-09	1.08	0	0	\\
3.4e-09	1.08	0	0	\\
3.45e-09	1.08	0	0	\\
3.5e-09	1.08	0.00666666666666667	0.00666666666666667	\\
3.55e-09	1.08	0.00666666666666667	0.00666666666666667	\\
3.6e-09	1.08	0.00666666666666667	0.00666666666666667	\\
3.65e-09	1.08	0.00666666666666667	0.00666666666666667	\\
3.7e-09	1.08	0.00666666666666667	0.00666666666666667	\\
3.75e-09	1.08	0.00222222222222222	0.00222222222222222	\\
3.8e-09	1.08	0.00222222222222222	0.00222222222222222	\\
3.85e-09	1.08	0.00222222222222222	0.00222222222222222	\\
3.9e-09	1.08	0.00222222222222222	0.00222222222222222	\\
3.95e-09	1.08	0.00222222222222222	0.00222222222222222	\\
4e-09	1.08	-0.00444444444444444	-0.00444444444444444	\\
4.05e-09	1.08	-0.00444444444444444	-0.00444444444444444	\\
4.1e-09	1.08	-0.00444444444444444	-0.00444444444444444	\\
4.15e-09	1.08	-0.00444444444444444	-0.00444444444444444	\\
4.2e-09	1.08	-0.00444444444444444	-0.00444444444444444	\\
4.25e-09	1.08	0	0	\\
4.3e-09	1.08	0	0	\\
4.35e-09	1.08	0	0	\\
4.4e-09	1.08	0	0	\\
4.45e-09	1.08	0	0	\\
4.5e-09	1.08	0	0	\\
4.55e-09	1.08	0	0	\\
4.6e-09	1.08	0	0	\\
4.65e-09	1.08	0	0	\\
4.7e-09	1.08	0	0	\\
4.75e-09	1.08	0	0	\\
4.8e-09	1.08	0	0	\\
4.85e-09	1.08	0	0	\\
4.9e-09	1.08	0	0	\\
4.95e-09	1.08	0	0	\\
5e-09	1.08	0	0	\\
5e-09	1.08	0	nan	\\
5e-09	1.08	-0.00666666666666667	0.00166666666666667	\\
5e-09	1.08	-0.00666666666666667	nan	\\
0	1.092	-0.00666666666666667	nan	\\
0	1.092	0	0.00166666666666667	\\
0	1.092	0	0	\\
5e-11	1.092	0	0	\\
1e-10	1.092	0	0	\\
1.5e-10	1.092	0	0	\\
2e-10	1.092	0	0	\\
2.5e-10	1.092	0	0	\\
3e-10	1.092	0	0	\\
3.5e-10	1.092	0	0	\\
4e-10	1.092	0	0	\\
4.5e-10	1.092	0	0	\\
5e-10	1.092	0	0	\\
5.5e-10	1.092	0	0	\\
6e-10	1.092	0	0	\\
6.5e-10	1.092	0	0	\\
7e-10	1.092	0	0	\\
7.5e-10	1.092	0	0	\\
8e-10	1.092	0	0	\\
8.5e-10	1.092	0	0	\\
9e-10	1.092	0	0	\\
9.5e-10	1.092	0	0	\\
1e-09	1.092	0	0	\\
1.05e-09	1.092	0	0	\\
1.1e-09	1.092	0.01	0.01	\\
1.15e-09	1.092	0.01	0.01	\\
1.2e-09	1.092	0.01	0.01	\\
1.25e-09	1.092	0.01	0.01	\\
1.3e-09	1.092	0.01	0.01	\\
1.35e-09	1.092	0.00333333333333333	0.00333333333333333	\\
1.4e-09	1.092	0.00333333333333333	0.00333333333333333	\\
1.45e-09	1.092	0.00333333333333333	0.00333333333333333	\\
1.5e-09	1.092	0.00333333333333333	0.00333333333333333	\\
1.55e-09	1.092	0.00333333333333333	0.00333333333333333	\\
1.6e-09	1.092	-0.00666666666666667	-0.00666666666666667	\\
1.65e-09	1.092	-0.00666666666666667	-0.00666666666666667	\\
1.7e-09	1.092	-0.00666666666666667	-0.00666666666666667	\\
1.75e-09	1.092	-0.00666666666666667	-0.00666666666666667	\\
1.8e-09	1.092	-0.00666666666666667	-0.00666666666666667	\\
1.85e-09	1.092	0	0	\\
1.9e-09	1.092	0	0	\\
1.95e-09	1.092	0	0	\\
2e-09	1.092	0	0	\\
2.05e-09	1.092	0	0	\\
2.1e-09	1.092	0	0	\\
2.15e-09	1.092	0	0	\\
2.2e-09	1.092	0	0	\\
2.25e-09	1.092	0	0	\\
2.3e-09	1.092	0	0	\\
2.35e-09	1.092	0	0	\\
2.4e-09	1.092	0	0	\\
2.45e-09	1.092	0	0	\\
2.5e-09	1.092	0	0	\\
2.55e-09	1.092	0	0	\\
2.6e-09	1.092	0	0	\\
2.65e-09	1.092	0	0	\\
2.7e-09	1.092	0	0	\\
2.75e-09	1.092	0	0	\\
2.8e-09	1.092	0	0	\\
2.85e-09	1.092	0	0	\\
2.9e-09	1.092	0	0	\\
2.95e-09	1.092	0	0	\\
3e-09	1.092	0	0	\\
3.05e-09	1.092	0	0	\\
3.1e-09	1.092	0	0	\\
3.15e-09	1.092	0	0	\\
3.2e-09	1.092	0	0	\\
3.25e-09	1.092	0	0	\\
3.3e-09	1.092	0	0	\\
3.35e-09	1.092	0	0	\\
3.4e-09	1.092	0	0	\\
3.45e-09	1.092	0	0	\\
3.5e-09	1.092	0.00666666666666667	0.00666666666666667	\\
3.55e-09	1.092	0.00666666666666667	0.00666666666666667	\\
3.6e-09	1.092	0.00666666666666667	0.00666666666666667	\\
3.65e-09	1.092	0.00666666666666667	0.00666666666666667	\\
3.7e-09	1.092	0.00666666666666667	0.00666666666666667	\\
3.75e-09	1.092	0.00222222222222222	0.00222222222222222	\\
3.8e-09	1.092	0.00222222222222222	0.00222222222222222	\\
3.85e-09	1.092	0.00222222222222222	0.00222222222222222	\\
3.9e-09	1.092	0.00222222222222222	0.00222222222222222	\\
3.95e-09	1.092	0.00222222222222222	0.00222222222222222	\\
4e-09	1.092	-0.00444444444444444	-0.00444444444444444	\\
4.05e-09	1.092	-0.00444444444444444	-0.00444444444444444	\\
4.1e-09	1.092	-0.00444444444444444	-0.00444444444444444	\\
4.15e-09	1.092	-0.00444444444444444	-0.00444444444444444	\\
4.2e-09	1.092	-0.00444444444444444	-0.00444444444444444	\\
4.25e-09	1.092	0	0	\\
4.3e-09	1.092	0	0	\\
4.35e-09	1.092	0	0	\\
4.4e-09	1.092	0	0	\\
4.45e-09	1.092	0	0	\\
4.5e-09	1.092	0	0	\\
4.55e-09	1.092	0	0	\\
4.6e-09	1.092	0	0	\\
4.65e-09	1.092	0	0	\\
4.7e-09	1.092	0	0	\\
4.75e-09	1.092	0	0	\\
4.8e-09	1.092	0	0	\\
4.85e-09	1.092	0	0	\\
4.9e-09	1.092	0	0	\\
4.95e-09	1.092	0	0	\\
5e-09	1.092	0	0	\\
5e-09	1.092	0	nan	\\
5e-09	1.092	-0.00666666666666667	0.00166666666666667	\\
5e-09	1.092	-0.00666666666666667	nan	\\
0	1.104	-0.00666666666666667	nan	\\
0	1.104	0	0.00166666666666667	\\
0	1.104	0	0	\\
5e-11	1.104	0	0	\\
1e-10	1.104	0	0	\\
1.5e-10	1.104	0	0	\\
2e-10	1.104	0	0	\\
2.5e-10	1.104	0	0	\\
3e-10	1.104	0	0	\\
3.5e-10	1.104	0	0	\\
4e-10	1.104	0	0	\\
4.5e-10	1.104	0	0	\\
5e-10	1.104	0	0	\\
5.5e-10	1.104	0	0	\\
6e-10	1.104	0	0	\\
6.5e-10	1.104	0	0	\\
7e-10	1.104	0	0	\\
7.5e-10	1.104	0	0	\\
8e-10	1.104	0	0	\\
8.5e-10	1.104	0	0	\\
9e-10	1.104	0	0	\\
9.5e-10	1.104	0	0	\\
1e-09	1.104	0	0	\\
1.05e-09	1.104	0	0	\\
1.1e-09	1.104	0	0	\\
1.15e-09	1.104	0.01	0.01	\\
1.2e-09	1.104	0.01	0.01	\\
1.25e-09	1.104	0.01	0.01	\\
1.3e-09	1.104	0.00333333333333333	0.00333333333333333	\\
1.35e-09	1.104	0.00333333333333333	0.00333333333333333	\\
1.4e-09	1.104	0.00333333333333333	0.00333333333333333	\\
1.45e-09	1.104	0.00333333333333333	0.00333333333333333	\\
1.5e-09	1.104	0.00333333333333333	0.00333333333333333	\\
1.55e-09	1.104	0.00333333333333333	0.00333333333333333	\\
1.6e-09	1.104	0.00333333333333333	0.00333333333333333	\\
1.65e-09	1.104	-0.00666666666666667	-0.00666666666666667	\\
1.7e-09	1.104	-0.00666666666666667	-0.00666666666666667	\\
1.75e-09	1.104	-0.00666666666666667	-0.00666666666666667	\\
1.8e-09	1.104	0	0	\\
1.85e-09	1.104	0	0	\\
1.9e-09	1.104	0	0	\\
1.95e-09	1.104	0	0	\\
2e-09	1.104	0	0	\\
2.05e-09	1.104	0	0	\\
2.1e-09	1.104	0	0	\\
2.15e-09	1.104	0	0	\\
2.2e-09	1.104	0	0	\\
2.25e-09	1.104	0	0	\\
2.3e-09	1.104	0	0	\\
2.35e-09	1.104	0	0	\\
2.4e-09	1.104	0	0	\\
2.45e-09	1.104	0	0	\\
2.5e-09	1.104	0	0	\\
2.55e-09	1.104	0	0	\\
2.6e-09	1.104	0	0	\\
2.65e-09	1.104	0	0	\\
2.7e-09	1.104	0	0	\\
2.75e-09	1.104	0	0	\\
2.8e-09	1.104	0	0	\\
2.85e-09	1.104	0	0	\\
2.9e-09	1.104	0	0	\\
2.95e-09	1.104	0	0	\\
3e-09	1.104	0	0	\\
3.05e-09	1.104	0	0	\\
3.1e-09	1.104	0	0	\\
3.15e-09	1.104	0	0	\\
3.2e-09	1.104	0	0	\\
3.25e-09	1.104	0	0	\\
3.3e-09	1.104	0	0	\\
3.35e-09	1.104	0	0	\\
3.4e-09	1.104	0	0	\\
3.45e-09	1.104	0	0	\\
3.5e-09	1.104	0	0	\\
3.55e-09	1.104	0.00666666666666667	0.00666666666666667	\\
3.6e-09	1.104	0.00666666666666667	0.00666666666666667	\\
3.65e-09	1.104	0.00666666666666667	0.00666666666666667	\\
3.7e-09	1.104	0.00222222222222222	0.00222222222222222	\\
3.75e-09	1.104	0.00222222222222222	0.00222222222222222	\\
3.8e-09	1.104	0.00222222222222222	0.00222222222222222	\\
3.85e-09	1.104	0.00222222222222222	0.00222222222222222	\\
3.9e-09	1.104	0.00222222222222222	0.00222222222222222	\\
3.95e-09	1.104	0.00222222222222222	0.00222222222222222	\\
4e-09	1.104	0.00222222222222222	0.00222222222222222	\\
4.05e-09	1.104	-0.00444444444444444	-0.00444444444444444	\\
4.1e-09	1.104	-0.00444444444444444	-0.00444444444444444	\\
4.15e-09	1.104	-0.00444444444444444	-0.00444444444444444	\\
4.2e-09	1.104	0	0	\\
4.25e-09	1.104	0	0	\\
4.3e-09	1.104	0	0	\\
4.35e-09	1.104	0	0	\\
4.4e-09	1.104	0	0	\\
4.45e-09	1.104	0	0	\\
4.5e-09	1.104	0	0	\\
4.55e-09	1.104	0	0	\\
4.6e-09	1.104	0	0	\\
4.65e-09	1.104	0	0	\\
4.7e-09	1.104	0	0	\\
4.75e-09	1.104	0	0	\\
4.8e-09	1.104	0	0	\\
4.85e-09	1.104	0	0	\\
4.9e-09	1.104	0	0	\\
4.95e-09	1.104	0	0	\\
5e-09	1.104	0	0	\\
5e-09	1.104	0	nan	\\
5e-09	1.104	-0.00666666666666667	0.00166666666666667	\\
5e-09	1.104	-0.00666666666666667	nan	\\
0	1.116	-0.00666666666666667	nan	\\
0	1.116	0	0.00166666666666667	\\
0	1.116	0	0	\\
5e-11	1.116	0	0	\\
1e-10	1.116	0	0	\\
1.5e-10	1.116	0	0	\\
2e-10	1.116	0	0	\\
2.5e-10	1.116	0	0	\\
3e-10	1.116	0	0	\\
3.5e-10	1.116	0	0	\\
4e-10	1.116	0	0	\\
4.5e-10	1.116	0	0	\\
5e-10	1.116	0	0	\\
5.5e-10	1.116	0	0	\\
6e-10	1.116	0	0	\\
6.5e-10	1.116	0	0	\\
7e-10	1.116	0	0	\\
7.5e-10	1.116	0	0	\\
8e-10	1.116	0	0	\\
8.5e-10	1.116	0	0	\\
9e-10	1.116	0	0	\\
9.5e-10	1.116	0	0	\\
1e-09	1.116	0	0	\\
1.05e-09	1.116	0	0	\\
1.1e-09	1.116	0	0	\\
1.15e-09	1.116	0.01	0.01	\\
1.2e-09	1.116	0.01	0.01	\\
1.25e-09	1.116	0.01	0.01	\\
1.3e-09	1.116	0.00333333333333333	0.00333333333333333	\\
1.35e-09	1.116	0.00333333333333333	0.00333333333333333	\\
1.4e-09	1.116	0.00333333333333333	0.00333333333333333	\\
1.45e-09	1.116	0.00333333333333333	0.00333333333333333	\\
1.5e-09	1.116	0.00333333333333333	0.00333333333333333	\\
1.55e-09	1.116	0.00333333333333333	0.00333333333333333	\\
1.6e-09	1.116	0.00333333333333333	0.00333333333333333	\\
1.65e-09	1.116	-0.00666666666666667	-0.00666666666666667	\\
1.7e-09	1.116	-0.00666666666666667	-0.00666666666666667	\\
1.75e-09	1.116	-0.00666666666666667	-0.00666666666666667	\\
1.8e-09	1.116	0	0	\\
1.85e-09	1.116	0	0	\\
1.9e-09	1.116	0	0	\\
1.95e-09	1.116	0	0	\\
2e-09	1.116	0	0	\\
2.05e-09	1.116	0	0	\\
2.1e-09	1.116	0	0	\\
2.15e-09	1.116	0	0	\\
2.2e-09	1.116	0	0	\\
2.25e-09	1.116	0	0	\\
2.3e-09	1.116	0	0	\\
2.35e-09	1.116	0	0	\\
2.4e-09	1.116	0	0	\\
2.45e-09	1.116	0	0	\\
2.5e-09	1.116	0	0	\\
2.55e-09	1.116	0	0	\\
2.6e-09	1.116	0	0	\\
2.65e-09	1.116	0	0	\\
2.7e-09	1.116	0	0	\\
2.75e-09	1.116	0	0	\\
2.8e-09	1.116	0	0	\\
2.85e-09	1.116	0	0	\\
2.9e-09	1.116	0	0	\\
2.95e-09	1.116	0	0	\\
3e-09	1.116	0	0	\\
3.05e-09	1.116	0	0	\\
3.1e-09	1.116	0	0	\\
3.15e-09	1.116	0	0	\\
3.2e-09	1.116	0	0	\\
3.25e-09	1.116	0	0	\\
3.3e-09	1.116	0	0	\\
3.35e-09	1.116	0	0	\\
3.4e-09	1.116	0	0	\\
3.45e-09	1.116	0	0	\\
3.5e-09	1.116	0	0	\\
3.55e-09	1.116	0.00666666666666667	0.00666666666666667	\\
3.6e-09	1.116	0.00666666666666667	0.00666666666666667	\\
3.65e-09	1.116	0.00666666666666667	0.00666666666666667	\\
3.7e-09	1.116	0.00222222222222222	0.00222222222222222	\\
3.75e-09	1.116	0.00222222222222222	0.00222222222222222	\\
3.8e-09	1.116	0.00222222222222222	0.00222222222222222	\\
3.85e-09	1.116	0.00222222222222222	0.00222222222222222	\\
3.9e-09	1.116	0.00222222222222222	0.00222222222222222	\\
3.95e-09	1.116	0.00222222222222222	0.00222222222222222	\\
4e-09	1.116	0.00222222222222222	0.00222222222222222	\\
4.05e-09	1.116	-0.00444444444444444	-0.00444444444444444	\\
4.1e-09	1.116	-0.00444444444444444	-0.00444444444444444	\\
4.15e-09	1.116	-0.00444444444444444	-0.00444444444444444	\\
4.2e-09	1.116	0	0	\\
4.25e-09	1.116	0	0	\\
4.3e-09	1.116	0	0	\\
4.35e-09	1.116	0	0	\\
4.4e-09	1.116	0	0	\\
4.45e-09	1.116	0	0	\\
4.5e-09	1.116	0	0	\\
4.55e-09	1.116	0	0	\\
4.6e-09	1.116	0	0	\\
4.65e-09	1.116	0	0	\\
4.7e-09	1.116	0	0	\\
4.75e-09	1.116	0	0	\\
4.8e-09	1.116	0	0	\\
4.85e-09	1.116	0	0	\\
4.9e-09	1.116	0	0	\\
4.95e-09	1.116	0	0	\\
5e-09	1.116	0	0	\\
5e-09	1.116	0	nan	\\
5e-09	1.116	-0.00666666666666667	0.00166666666666667	\\
5e-09	1.116	-0.00666666666666667	nan	\\
0	1.128	-0.00666666666666667	nan	\\
0	1.128	0	0.00166666666666667	\\
0	1.128	0	0	\\
5e-11	1.128	0	0	\\
1e-10	1.128	0	0	\\
1.5e-10	1.128	0	0	\\
2e-10	1.128	0	0	\\
2.5e-10	1.128	0	0	\\
3e-10	1.128	0	0	\\
3.5e-10	1.128	0	0	\\
4e-10	1.128	0	0	\\
4.5e-10	1.128	0	0	\\
5e-10	1.128	0	0	\\
5.5e-10	1.128	0	0	\\
6e-10	1.128	0	0	\\
6.5e-10	1.128	0	0	\\
7e-10	1.128	0	0	\\
7.5e-10	1.128	0	0	\\
8e-10	1.128	0	0	\\
8.5e-10	1.128	0	0	\\
9e-10	1.128	0	0	\\
9.5e-10	1.128	0	0	\\
1e-09	1.128	0	0	\\
1.05e-09	1.128	0	0	\\
1.1e-09	1.128	0	0	\\
1.15e-09	1.128	0.01	0.01	\\
1.2e-09	1.128	0.01	0.01	\\
1.25e-09	1.128	0.01	0.01	\\
1.3e-09	1.128	0.00333333333333333	0.00333333333333333	\\
1.35e-09	1.128	0.00333333333333333	0.00333333333333333	\\
1.4e-09	1.128	0.00333333333333333	0.00333333333333333	\\
1.45e-09	1.128	0.00333333333333333	0.00333333333333333	\\
1.5e-09	1.128	0.00333333333333333	0.00333333333333333	\\
1.55e-09	1.128	0.00333333333333333	0.00333333333333333	\\
1.6e-09	1.128	0.00333333333333333	0.00333333333333333	\\
1.65e-09	1.128	-0.00666666666666667	-0.00666666666666667	\\
1.7e-09	1.128	-0.00666666666666667	-0.00666666666666667	\\
1.75e-09	1.128	-0.00666666666666667	-0.00666666666666667	\\
1.8e-09	1.128	0	0	\\
1.85e-09	1.128	0	0	\\
1.9e-09	1.128	0	0	\\
1.95e-09	1.128	0	0	\\
2e-09	1.128	0	0	\\
2.05e-09	1.128	0	0	\\
2.1e-09	1.128	0	0	\\
2.15e-09	1.128	0	0	\\
2.2e-09	1.128	0	0	\\
2.25e-09	1.128	0	0	\\
2.3e-09	1.128	0	0	\\
2.35e-09	1.128	0	0	\\
2.4e-09	1.128	0	0	\\
2.45e-09	1.128	0	0	\\
2.5e-09	1.128	0	0	\\
2.55e-09	1.128	0	0	\\
2.6e-09	1.128	0	0	\\
2.65e-09	1.128	0	0	\\
2.7e-09	1.128	0	0	\\
2.75e-09	1.128	0	0	\\
2.8e-09	1.128	0	0	\\
2.85e-09	1.128	0	0	\\
2.9e-09	1.128	0	0	\\
2.95e-09	1.128	0	0	\\
3e-09	1.128	0	0	\\
3.05e-09	1.128	0	0	\\
3.1e-09	1.128	0	0	\\
3.15e-09	1.128	0	0	\\
3.2e-09	1.128	0	0	\\
3.25e-09	1.128	0	0	\\
3.3e-09	1.128	0	0	\\
3.35e-09	1.128	0	0	\\
3.4e-09	1.128	0	0	\\
3.45e-09	1.128	0	0	\\
3.5e-09	1.128	0	0	\\
3.55e-09	1.128	0.00666666666666667	0.00666666666666667	\\
3.6e-09	1.128	0.00666666666666667	0.00666666666666667	\\
3.65e-09	1.128	0.00666666666666667	0.00666666666666667	\\
3.7e-09	1.128	0.00222222222222222	0.00222222222222222	\\
3.75e-09	1.128	0.00222222222222222	0.00222222222222222	\\
3.8e-09	1.128	0.00222222222222222	0.00222222222222222	\\
3.85e-09	1.128	0.00222222222222222	0.00222222222222222	\\
3.9e-09	1.128	0.00222222222222222	0.00222222222222222	\\
3.95e-09	1.128	0.00222222222222222	0.00222222222222222	\\
4e-09	1.128	0.00222222222222222	0.00222222222222222	\\
4.05e-09	1.128	-0.00444444444444444	-0.00444444444444444	\\
4.1e-09	1.128	-0.00444444444444444	-0.00444444444444444	\\
4.15e-09	1.128	-0.00444444444444444	-0.00444444444444444	\\
4.2e-09	1.128	0	0	\\
4.25e-09	1.128	0	0	\\
4.3e-09	1.128	0	0	\\
4.35e-09	1.128	0	0	\\
4.4e-09	1.128	0	0	\\
4.45e-09	1.128	0	0	\\
4.5e-09	1.128	0	0	\\
4.55e-09	1.128	0	0	\\
4.6e-09	1.128	0	0	\\
4.65e-09	1.128	0	0	\\
4.7e-09	1.128	0	0	\\
4.75e-09	1.128	0	0	\\
4.8e-09	1.128	0	0	\\
4.85e-09	1.128	0	0	\\
4.9e-09	1.128	0	0	\\
4.95e-09	1.128	0	0	\\
5e-09	1.128	0	0	\\
5e-09	1.128	0	nan	\\
5e-09	1.128	-0.00666666666666667	0.00166666666666667	\\
5e-09	1.128	-0.00666666666666667	nan	\\
0	1.14	-0.00666666666666667	nan	\\
0	1.14	0	0.00166666666666667	\\
0	1.14	0	0	\\
5e-11	1.14	0	0	\\
1e-10	1.14	0	0	\\
1.5e-10	1.14	0	0	\\
2e-10	1.14	0	0	\\
2.5e-10	1.14	0	0	\\
3e-10	1.14	0	0	\\
3.5e-10	1.14	0	0	\\
4e-10	1.14	0	0	\\
4.5e-10	1.14	0	0	\\
5e-10	1.14	0	0	\\
5.5e-10	1.14	0	0	\\
6e-10	1.14	0	0	\\
6.5e-10	1.14	0	0	\\
7e-10	1.14	0	0	\\
7.5e-10	1.14	0	0	\\
8e-10	1.14	0	0	\\
8.5e-10	1.14	0	0	\\
9e-10	1.14	0	0	\\
9.5e-10	1.14	0	0	\\
1e-09	1.14	0	0	\\
1.05e-09	1.14	0	0	\\
1.1e-09	1.14	0	0	\\
1.15e-09	1.14	0.01	0.01	\\
1.2e-09	1.14	0.01	0.01	\\
1.25e-09	1.14	0.01	0.01	\\
1.3e-09	1.14	0.00333333333333333	0.00333333333333333	\\
1.35e-09	1.14	0.00333333333333333	0.00333333333333333	\\
1.4e-09	1.14	0.00333333333333333	0.00333333333333333	\\
1.45e-09	1.14	0.00333333333333333	0.00333333333333333	\\
1.5e-09	1.14	0.00333333333333333	0.00333333333333333	\\
1.55e-09	1.14	0.00333333333333333	0.00333333333333333	\\
1.6e-09	1.14	0.00333333333333333	0.00333333333333333	\\
1.65e-09	1.14	-0.00666666666666667	-0.00666666666666667	\\
1.7e-09	1.14	-0.00666666666666667	-0.00666666666666667	\\
1.75e-09	1.14	-0.00666666666666667	-0.00666666666666667	\\
1.8e-09	1.14	0	0	\\
1.85e-09	1.14	0	0	\\
1.9e-09	1.14	0	0	\\
1.95e-09	1.14	0	0	\\
2e-09	1.14	0	0	\\
2.05e-09	1.14	0	0	\\
2.1e-09	1.14	0	0	\\
2.15e-09	1.14	0	0	\\
2.2e-09	1.14	0	0	\\
2.25e-09	1.14	0	0	\\
2.3e-09	1.14	0	0	\\
2.35e-09	1.14	0	0	\\
2.4e-09	1.14	0	0	\\
2.45e-09	1.14	0	0	\\
2.5e-09	1.14	0	0	\\
2.55e-09	1.14	0	0	\\
2.6e-09	1.14	0	0	\\
2.65e-09	1.14	0	0	\\
2.7e-09	1.14	0	0	\\
2.75e-09	1.14	0	0	\\
2.8e-09	1.14	0	0	\\
2.85e-09	1.14	0	0	\\
2.9e-09	1.14	0	0	\\
2.95e-09	1.14	0	0	\\
3e-09	1.14	0	0	\\
3.05e-09	1.14	0	0	\\
3.1e-09	1.14	0	0	\\
3.15e-09	1.14	0	0	\\
3.2e-09	1.14	0	0	\\
3.25e-09	1.14	0	0	\\
3.3e-09	1.14	0	0	\\
3.35e-09	1.14	0	0	\\
3.4e-09	1.14	0	0	\\
3.45e-09	1.14	0	0	\\
3.5e-09	1.14	0	0	\\
3.55e-09	1.14	0.00666666666666667	0.00666666666666667	\\
3.6e-09	1.14	0.00666666666666667	0.00666666666666667	\\
3.65e-09	1.14	0.00666666666666667	0.00666666666666667	\\
3.7e-09	1.14	0.00222222222222222	0.00222222222222222	\\
3.75e-09	1.14	0.00222222222222222	0.00222222222222222	\\
3.8e-09	1.14	0.00222222222222222	0.00222222222222222	\\
3.85e-09	1.14	0.00222222222222222	0.00222222222222222	\\
3.9e-09	1.14	0.00222222222222222	0.00222222222222222	\\
3.95e-09	1.14	0.00222222222222222	0.00222222222222222	\\
4e-09	1.14	0.00222222222222222	0.00222222222222222	\\
4.05e-09	1.14	-0.00444444444444444	-0.00444444444444444	\\
4.1e-09	1.14	-0.00444444444444444	-0.00444444444444444	\\
4.15e-09	1.14	-0.00444444444444444	-0.00444444444444444	\\
4.2e-09	1.14	0	0	\\
4.25e-09	1.14	0	0	\\
4.3e-09	1.14	0	0	\\
4.35e-09	1.14	0	0	\\
4.4e-09	1.14	0	0	\\
4.45e-09	1.14	0	0	\\
4.5e-09	1.14	0	0	\\
4.55e-09	1.14	0	0	\\
4.6e-09	1.14	0	0	\\
4.65e-09	1.14	0	0	\\
4.7e-09	1.14	0	0	\\
4.75e-09	1.14	0	0	\\
4.8e-09	1.14	0	0	\\
4.85e-09	1.14	0	0	\\
4.9e-09	1.14	0	0	\\
4.95e-09	1.14	0	0	\\
5e-09	1.14	0	0	\\
5e-09	1.14	0	nan	\\
5e-09	1.14	-0.00666666666666667	0.00166666666666667	\\
5e-09	1.14	-0.00666666666666667	nan	\\
0	1.152	-0.00666666666666667	nan	\\
0	1.152	0	0.00166666666666667	\\
0	1.152	0	0	\\
5e-11	1.152	0	0	\\
1e-10	1.152	0	0	\\
1.5e-10	1.152	0	0	\\
2e-10	1.152	0	0	\\
2.5e-10	1.152	0	0	\\
3e-10	1.152	0	0	\\
3.5e-10	1.152	0	0	\\
4e-10	1.152	0	0	\\
4.5e-10	1.152	0	0	\\
5e-10	1.152	0	0	\\
5.5e-10	1.152	0	0	\\
6e-10	1.152	0	0	\\
6.5e-10	1.152	0	0	\\
7e-10	1.152	0	0	\\
7.5e-10	1.152	0	0	\\
8e-10	1.152	0	0	\\
8.5e-10	1.152	0	0	\\
9e-10	1.152	0	0	\\
9.5e-10	1.152	0	0	\\
1e-09	1.152	0	0	\\
1.05e-09	1.152	0	0	\\
1.1e-09	1.152	0	0	\\
1.15e-09	1.152	0	0	\\
1.2e-09	1.152	0.01	0.01	\\
1.25e-09	1.152	0.00333333333333333	0.00333333333333333	\\
1.3e-09	1.152	0.00333333333333333	0.00333333333333333	\\
1.35e-09	1.152	0.00333333333333333	0.00333333333333333	\\
1.4e-09	1.152	0.00333333333333333	0.00333333333333333	\\
1.45e-09	1.152	0.00333333333333333	0.00333333333333333	\\
1.5e-09	1.152	0.00333333333333333	0.00333333333333333	\\
1.55e-09	1.152	0.00333333333333333	0.00333333333333333	\\
1.6e-09	1.152	0.00333333333333333	0.00333333333333333	\\
1.65e-09	1.152	0.00333333333333333	0.00333333333333333	\\
1.7e-09	1.152	-0.00666666666666667	-0.00666666666666667	\\
1.75e-09	1.152	0	0	\\
1.8e-09	1.152	0	0	\\
1.85e-09	1.152	0	0	\\
1.9e-09	1.152	0	0	\\
1.95e-09	1.152	0	0	\\
2e-09	1.152	0	0	\\
2.05e-09	1.152	0	0	\\
2.1e-09	1.152	0	0	\\
2.15e-09	1.152	0	0	\\
2.2e-09	1.152	0	0	\\
2.25e-09	1.152	0	0	\\
2.3e-09	1.152	0	0	\\
2.35e-09	1.152	0	0	\\
2.4e-09	1.152	0	0	\\
2.45e-09	1.152	0	0	\\
2.5e-09	1.152	0	0	\\
2.55e-09	1.152	0	0	\\
2.6e-09	1.152	0	0	\\
2.65e-09	1.152	0	0	\\
2.7e-09	1.152	0	0	\\
2.75e-09	1.152	0	0	\\
2.8e-09	1.152	0	0	\\
2.85e-09	1.152	0	0	\\
2.9e-09	1.152	0	0	\\
2.95e-09	1.152	0	0	\\
3e-09	1.152	0	0	\\
3.05e-09	1.152	0	0	\\
3.1e-09	1.152	0	0	\\
3.15e-09	1.152	0	0	\\
3.2e-09	1.152	0	0	\\
3.25e-09	1.152	0	0	\\
3.3e-09	1.152	0	0	\\
3.35e-09	1.152	0	0	\\
3.4e-09	1.152	0	0	\\
3.45e-09	1.152	0	0	\\
3.5e-09	1.152	0	0	\\
3.55e-09	1.152	0	0	\\
3.6e-09	1.152	0.00666666666666667	0.00666666666666667	\\
3.65e-09	1.152	0.00222222222222222	0.00222222222222222	\\
3.7e-09	1.152	0.00222222222222222	0.00222222222222222	\\
3.75e-09	1.152	0.00222222222222222	0.00222222222222222	\\
3.8e-09	1.152	0.00222222222222222	0.00222222222222222	\\
3.85e-09	1.152	0.00222222222222222	0.00222222222222222	\\
3.9e-09	1.152	0.00222222222222222	0.00222222222222222	\\
3.95e-09	1.152	0.00222222222222222	0.00222222222222222	\\
4e-09	1.152	0.00222222222222222	0.00222222222222222	\\
4.05e-09	1.152	0.00222222222222222	0.00222222222222222	\\
4.1e-09	1.152	-0.00444444444444444	-0.00444444444444444	\\
4.15e-09	1.152	0	0	\\
4.2e-09	1.152	0	0	\\
4.25e-09	1.152	0	0	\\
4.3e-09	1.152	0	0	\\
4.35e-09	1.152	0	0	\\
4.4e-09	1.152	0	0	\\
4.45e-09	1.152	0	0	\\
4.5e-09	1.152	0	0	\\
4.55e-09	1.152	0	0	\\
4.6e-09	1.152	0	0	\\
4.65e-09	1.152	0	0	\\
4.7e-09	1.152	0	0	\\
4.75e-09	1.152	0	0	\\
4.8e-09	1.152	0	0	\\
4.85e-09	1.152	0	0	\\
4.9e-09	1.152	0	0	\\
4.95e-09	1.152	0	0	\\
5e-09	1.152	0	0	\\
5e-09	1.152	0	nan	\\
5e-09	1.152	-0.00666666666666667	0.00166666666666667	\\
5e-09	1.152	-0.00666666666666667	nan	\\
0	1.164	-0.00666666666666667	nan	\\
0	1.164	0	0.00166666666666667	\\
0	1.164	0	0	\\
5e-11	1.164	0	0	\\
1e-10	1.164	0	0	\\
1.5e-10	1.164	0	0	\\
2e-10	1.164	0	0	\\
2.5e-10	1.164	0	0	\\
3e-10	1.164	0	0	\\
3.5e-10	1.164	0	0	\\
4e-10	1.164	0	0	\\
4.5e-10	1.164	0	0	\\
5e-10	1.164	0	0	\\
5.5e-10	1.164	0	0	\\
6e-10	1.164	0	0	\\
6.5e-10	1.164	0	0	\\
7e-10	1.164	0	0	\\
7.5e-10	1.164	0	0	\\
8e-10	1.164	0	0	\\
8.5e-10	1.164	0	0	\\
9e-10	1.164	0	0	\\
9.5e-10	1.164	0	0	\\
1e-09	1.164	0	0	\\
1.05e-09	1.164	0	0	\\
1.1e-09	1.164	0	0	\\
1.15e-09	1.164	0	0	\\
1.2e-09	1.164	0.01	0.01	\\
1.25e-09	1.164	0.00333333333333333	0.00333333333333333	\\
1.3e-09	1.164	0.00333333333333333	0.00333333333333333	\\
1.35e-09	1.164	0.00333333333333333	0.00333333333333333	\\
1.4e-09	1.164	0.00333333333333333	0.00333333333333333	\\
1.45e-09	1.164	0.00333333333333333	0.00333333333333333	\\
1.5e-09	1.164	0.00333333333333333	0.00333333333333333	\\
1.55e-09	1.164	0.00333333333333333	0.00333333333333333	\\
1.6e-09	1.164	0.00333333333333333	0.00333333333333333	\\
1.65e-09	1.164	0.00333333333333333	0.00333333333333333	\\
1.7e-09	1.164	-0.00666666666666667	-0.00666666666666667	\\
1.75e-09	1.164	0	0	\\
1.8e-09	1.164	0	0	\\
1.85e-09	1.164	0	0	\\
1.9e-09	1.164	0	0	\\
1.95e-09	1.164	0	0	\\
2e-09	1.164	0	0	\\
2.05e-09	1.164	0	0	\\
2.1e-09	1.164	0	0	\\
2.15e-09	1.164	0	0	\\
2.2e-09	1.164	0	0	\\
2.25e-09	1.164	0	0	\\
2.3e-09	1.164	0	0	\\
2.35e-09	1.164	0	0	\\
2.4e-09	1.164	0	0	\\
2.45e-09	1.164	0	0	\\
2.5e-09	1.164	0	0	\\
2.55e-09	1.164	0	0	\\
2.6e-09	1.164	0	0	\\
2.65e-09	1.164	0	0	\\
2.7e-09	1.164	0	0	\\
2.75e-09	1.164	0	0	\\
2.8e-09	1.164	0	0	\\
2.85e-09	1.164	0	0	\\
2.9e-09	1.164	0	0	\\
2.95e-09	1.164	0	0	\\
3e-09	1.164	0	0	\\
3.05e-09	1.164	0	0	\\
3.1e-09	1.164	0	0	\\
3.15e-09	1.164	0	0	\\
3.2e-09	1.164	0	0	\\
3.25e-09	1.164	0	0	\\
3.3e-09	1.164	0	0	\\
3.35e-09	1.164	0	0	\\
3.4e-09	1.164	0	0	\\
3.45e-09	1.164	0	0	\\
3.5e-09	1.164	0	0	\\
3.55e-09	1.164	0	0	\\
3.6e-09	1.164	0.00666666666666667	0.00666666666666667	\\
3.65e-09	1.164	0.00222222222222222	0.00222222222222222	\\
3.7e-09	1.164	0.00222222222222222	0.00222222222222222	\\
3.75e-09	1.164	0.00222222222222222	0.00222222222222222	\\
3.8e-09	1.164	0.00222222222222222	0.00222222222222222	\\
3.85e-09	1.164	0.00222222222222222	0.00222222222222222	\\
3.9e-09	1.164	0.00222222222222222	0.00222222222222222	\\
3.95e-09	1.164	0.00222222222222222	0.00222222222222222	\\
4e-09	1.164	0.00222222222222222	0.00222222222222222	\\
4.05e-09	1.164	0.00222222222222222	0.00222222222222222	\\
4.1e-09	1.164	-0.00444444444444444	-0.00444444444444444	\\
4.15e-09	1.164	0	0	\\
4.2e-09	1.164	0	0	\\
4.25e-09	1.164	0	0	\\
4.3e-09	1.164	0	0	\\
4.35e-09	1.164	0	0	\\
4.4e-09	1.164	0	0	\\
4.45e-09	1.164	0	0	\\
4.5e-09	1.164	0	0	\\
4.55e-09	1.164	0	0	\\
4.6e-09	1.164	0	0	\\
4.65e-09	1.164	0	0	\\
4.7e-09	1.164	0	0	\\
4.75e-09	1.164	0	0	\\
4.8e-09	1.164	0	0	\\
4.85e-09	1.164	0	0	\\
4.9e-09	1.164	0	0	\\
4.95e-09	1.164	0	0	\\
5e-09	1.164	0	0	\\
5e-09	1.164	0	nan	\\
5e-09	1.164	-0.00666666666666667	0.00166666666666667	\\
5e-09	1.164	-0.00666666666666667	nan	\\
0	1.176	-0.00666666666666667	nan	\\
0	1.176	0	0.00166666666666667	\\
0	1.176	0	0	\\
5e-11	1.176	0	0	\\
1e-10	1.176	0	0	\\
1.5e-10	1.176	0	0	\\
2e-10	1.176	0	0	\\
2.5e-10	1.176	0	0	\\
3e-10	1.176	0	0	\\
3.5e-10	1.176	0	0	\\
4e-10	1.176	0	0	\\
4.5e-10	1.176	0	0	\\
5e-10	1.176	0	0	\\
5.5e-10	1.176	0	0	\\
6e-10	1.176	0	0	\\
6.5e-10	1.176	0	0	\\
7e-10	1.176	0	0	\\
7.5e-10	1.176	0	0	\\
8e-10	1.176	0	0	\\
8.5e-10	1.176	0	0	\\
9e-10	1.176	0	0	\\
9.5e-10	1.176	0	0	\\
1e-09	1.176	0	0	\\
1.05e-09	1.176	0	0	\\
1.1e-09	1.176	0	0	\\
1.15e-09	1.176	0	0	\\
1.2e-09	1.176	0.01	0.01	\\
1.25e-09	1.176	0.00333333333333333	0.00333333333333333	\\
1.3e-09	1.176	0.00333333333333333	0.00333333333333333	\\
1.35e-09	1.176	0.00333333333333333	0.00333333333333333	\\
1.4e-09	1.176	0.00333333333333333	0.00333333333333333	\\
1.45e-09	1.176	0.00333333333333333	0.00333333333333333	\\
1.5e-09	1.176	0.00333333333333333	0.00333333333333333	\\
1.55e-09	1.176	0.00333333333333333	0.00333333333333333	\\
1.6e-09	1.176	0.00333333333333333	0.00333333333333333	\\
1.65e-09	1.176	0.00333333333333333	0.00333333333333333	\\
1.7e-09	1.176	-0.00666666666666667	-0.00666666666666667	\\
1.75e-09	1.176	0	0	\\
1.8e-09	1.176	0	0	\\
1.85e-09	1.176	0	0	\\
1.9e-09	1.176	0	0	\\
1.95e-09	1.176	0	0	\\
2e-09	1.176	0	0	\\
2.05e-09	1.176	0	0	\\
2.1e-09	1.176	0	0	\\
2.15e-09	1.176	0	0	\\
2.2e-09	1.176	0	0	\\
2.25e-09	1.176	0	0	\\
2.3e-09	1.176	0	0	\\
2.35e-09	1.176	0	0	\\
2.4e-09	1.176	0	0	\\
2.45e-09	1.176	0	0	\\
2.5e-09	1.176	0	0	\\
2.55e-09	1.176	0	0	\\
2.6e-09	1.176	0	0	\\
2.65e-09	1.176	0	0	\\
2.7e-09	1.176	0	0	\\
2.75e-09	1.176	0	0	\\
2.8e-09	1.176	0	0	\\
2.85e-09	1.176	0	0	\\
2.9e-09	1.176	0	0	\\
2.95e-09	1.176	0	0	\\
3e-09	1.176	0	0	\\
3.05e-09	1.176	0	0	\\
3.1e-09	1.176	0	0	\\
3.15e-09	1.176	0	0	\\
3.2e-09	1.176	0	0	\\
3.25e-09	1.176	0	0	\\
3.3e-09	1.176	0	0	\\
3.35e-09	1.176	0	0	\\
3.4e-09	1.176	0	0	\\
3.45e-09	1.176	0	0	\\
3.5e-09	1.176	0	0	\\
3.55e-09	1.176	0	0	\\
3.6e-09	1.176	0.00666666666666667	0.00666666666666667	\\
3.65e-09	1.176	0.00222222222222222	0.00222222222222222	\\
3.7e-09	1.176	0.00222222222222222	0.00222222222222222	\\
3.75e-09	1.176	0.00222222222222222	0.00222222222222222	\\
3.8e-09	1.176	0.00222222222222222	0.00222222222222222	\\
3.85e-09	1.176	0.00222222222222222	0.00222222222222222	\\
3.9e-09	1.176	0.00222222222222222	0.00222222222222222	\\
3.95e-09	1.176	0.00222222222222222	0.00222222222222222	\\
4e-09	1.176	0.00222222222222222	0.00222222222222222	\\
4.05e-09	1.176	0.00222222222222222	0.00222222222222222	\\
4.1e-09	1.176	-0.00444444444444444	-0.00444444444444444	\\
4.15e-09	1.176	0	0	\\
4.2e-09	1.176	0	0	\\
4.25e-09	1.176	0	0	\\
4.3e-09	1.176	0	0	\\
4.35e-09	1.176	0	0	\\
4.4e-09	1.176	0	0	\\
4.45e-09	1.176	0	0	\\
4.5e-09	1.176	0	0	\\
4.55e-09	1.176	0	0	\\
4.6e-09	1.176	0	0	\\
4.65e-09	1.176	0	0	\\
4.7e-09	1.176	0	0	\\
4.75e-09	1.176	0	0	\\
4.8e-09	1.176	0	0	\\
4.85e-09	1.176	0	0	\\
4.9e-09	1.176	0	0	\\
4.95e-09	1.176	0	0	\\
5e-09	1.176	0	0	\\
5e-09	1.176	0	nan	\\
5e-09	1.176	-0.00666666666666667	0.00166666666666667	\\
5e-09	1.176	-0.00666666666666667	nan	\\
0	1.188	-0.00666666666666667	nan	\\
0	1.188	0	0.00166666666666667	\\
0	1.188	0	0	\\
5e-11	1.188	0	0	\\
1e-10	1.188	0	0	\\
1.5e-10	1.188	0	0	\\
2e-10	1.188	0	0	\\
2.5e-10	1.188	0	0	\\
3e-10	1.188	0	0	\\
3.5e-10	1.188	0	0	\\
4e-10	1.188	0	0	\\
4.5e-10	1.188	0	0	\\
5e-10	1.188	0	0	\\
5.5e-10	1.188	0	0	\\
6e-10	1.188	0	0	\\
6.5e-10	1.188	0	0	\\
7e-10	1.188	0	0	\\
7.5e-10	1.188	0	0	\\
8e-10	1.188	0	0	\\
8.5e-10	1.188	0	0	\\
9e-10	1.188	0	0	\\
9.5e-10	1.188	0	0	\\
1e-09	1.188	0	0	\\
1.05e-09	1.188	0	0	\\
1.1e-09	1.188	0	0	\\
1.15e-09	1.188	0	0	\\
1.2e-09	1.188	0.01	0.01	\\
1.25e-09	1.188	0.00333333333333333	0.00333333333333333	\\
1.3e-09	1.188	0.00333333333333333	0.00333333333333333	\\
1.35e-09	1.188	0.00333333333333333	0.00333333333333333	\\
1.4e-09	1.188	0.00333333333333333	0.00333333333333333	\\
1.45e-09	1.188	0.00333333333333333	0.00333333333333333	\\
1.5e-09	1.188	0.00333333333333333	0.00333333333333333	\\
1.55e-09	1.188	0.00333333333333333	0.00333333333333333	\\
1.6e-09	1.188	0.00333333333333333	0.00333333333333333	\\
1.65e-09	1.188	0.00333333333333333	0.00333333333333333	\\
1.7e-09	1.188	-0.00666666666666667	-0.00666666666666667	\\
1.75e-09	1.188	0	0	\\
1.8e-09	1.188	0	0	\\
1.85e-09	1.188	0	0	\\
1.9e-09	1.188	0	0	\\
1.95e-09	1.188	0	0	\\
2e-09	1.188	0	0	\\
2.05e-09	1.188	0	0	\\
2.1e-09	1.188	0	0	\\
2.15e-09	1.188	0	0	\\
2.2e-09	1.188	0	0	\\
2.25e-09	1.188	0	0	\\
2.3e-09	1.188	0	0	\\
2.35e-09	1.188	0	0	\\
2.4e-09	1.188	0	0	\\
2.45e-09	1.188	0	0	\\
2.5e-09	1.188	0	0	\\
2.55e-09	1.188	0	0	\\
2.6e-09	1.188	0	0	\\
2.65e-09	1.188	0	0	\\
2.7e-09	1.188	0	0	\\
2.75e-09	1.188	0	0	\\
2.8e-09	1.188	0	0	\\
2.85e-09	1.188	0	0	\\
2.9e-09	1.188	0	0	\\
2.95e-09	1.188	0	0	\\
3e-09	1.188	0	0	\\
3.05e-09	1.188	0	0	\\
3.1e-09	1.188	0	0	\\
3.15e-09	1.188	0	0	\\
3.2e-09	1.188	0	0	\\
3.25e-09	1.188	0	0	\\
3.3e-09	1.188	0	0	\\
3.35e-09	1.188	0	0	\\
3.4e-09	1.188	0	0	\\
3.45e-09	1.188	0	0	\\
3.5e-09	1.188	0	0	\\
3.55e-09	1.188	0	0	\\
3.6e-09	1.188	0.00666666666666667	0.00666666666666667	\\
3.65e-09	1.188	0.00222222222222222	0.00222222222222222	\\
3.7e-09	1.188	0.00222222222222222	0.00222222222222222	\\
3.75e-09	1.188	0.00222222222222222	0.00222222222222222	\\
3.8e-09	1.188	0.00222222222222222	0.00222222222222222	\\
3.85e-09	1.188	0.00222222222222222	0.00222222222222222	\\
3.9e-09	1.188	0.00222222222222222	0.00222222222222222	\\
3.95e-09	1.188	0.00222222222222222	0.00222222222222222	\\
4e-09	1.188	0.00222222222222222	0.00222222222222222	\\
4.05e-09	1.188	0.00222222222222222	0.00222222222222222	\\
4.1e-09	1.188	-0.00444444444444444	-0.00444444444444444	\\
4.15e-09	1.188	0	0	\\
4.2e-09	1.188	0	0	\\
4.25e-09	1.188	0	0	\\
4.3e-09	1.188	0	0	\\
4.35e-09	1.188	0	0	\\
4.4e-09	1.188	0	0	\\
4.45e-09	1.188	0	0	\\
4.5e-09	1.188	0	0	\\
4.55e-09	1.188	0	0	\\
4.6e-09	1.188	0	0	\\
4.65e-09	1.188	0	0	\\
4.7e-09	1.188	0	0	\\
4.75e-09	1.188	0	0	\\
4.8e-09	1.188	0	0	\\
4.85e-09	1.188	0	0	\\
4.9e-09	1.188	0	0	\\
4.95e-09	1.188	0	0	\\
5e-09	1.188	0	0	\\
5e-09	1.188	0	nan	\\
5e-09	1.188	-0.00666666666666667	0.00166666666666667	\\
5e-09	1.188	-0.00666666666666667	nan	\\
0	1.2	-0.00666666666666667	nan	\\
0	1.2	0	0.00166666666666667	\\
0	1.2	0	0	\\
5e-11	1.2	0	0	\\
1e-10	1.2	0	0	\\
1.5e-10	1.2	0	0	\\
2e-10	1.2	0	0	\\
2.5e-10	1.2	0	0	\\
3e-10	1.2	0	0	\\
3.5e-10	1.2	0	0	\\
4e-10	1.2	0	0	\\
4.5e-10	1.2	0	0	\\
5e-10	1.2	0	0	\\
5.5e-10	1.2	0	0	\\
6e-10	1.2	0	0	\\
6.5e-10	1.2	0	0	\\
7e-10	1.2	0	0	\\
7.5e-10	1.2	0	0	\\
8e-10	1.2	0	0	\\
8.5e-10	1.2	0	0	\\
9e-10	1.2	0	0	\\
9.5e-10	1.2	0	0	\\
1e-09	1.2	0	0	\\
1.05e-09	1.2	0	0	\\
1.1e-09	1.2	0	0	\\
1.15e-09	1.2	0	0	\\
1.2e-09	1.2	0.00333333333333333	0.00333333333333333	\\
1.25e-09	1.2	0.00333333333333333	0.00333333333333333	\\
1.3e-09	1.2	0.00333333333333333	0.00333333333333333	\\
1.35e-09	1.2	0.00333333333333333	0.00333333333333333	\\
1.4e-09	1.2	0.00333333333333333	0.00333333333333333	\\
1.45e-09	1.2	0.00333333333333333	0.00333333333333333	\\
1.5e-09	1.2	0.00333333333333333	0.00333333333333333	\\
1.55e-09	1.2	0.00333333333333333	0.00333333333333333	\\
1.6e-09	1.2	0.00333333333333333	0.00333333333333333	\\
1.65e-09	1.2	0.00333333333333333	0.00333333333333333	\\
1.7e-09	1.2	0	0	\\
1.75e-09	1.2	0	0	\\
1.8e-09	1.2	0	0	\\
1.85e-09	1.2	0	0	\\
1.9e-09	1.2	0	0	\\
1.95e-09	1.2	0	0	\\
2e-09	1.2	0	0	\\
2.05e-09	1.2	0	0	\\
2.1e-09	1.2	0	0	\\
2.15e-09	1.2	0	0	\\
2.2e-09	1.2	0	0	\\
2.25e-09	1.2	0	0	\\
2.3e-09	1.2	0	0	\\
2.35e-09	1.2	0	0	\\
2.4e-09	1.2	0	0	\\
2.45e-09	1.2	0	0	\\
2.5e-09	1.2	0	0	\\
2.55e-09	1.2	0	0	\\
2.6e-09	1.2	0	0	\\
2.65e-09	1.2	0	0	\\
2.7e-09	1.2	0	0	\\
2.75e-09	1.2	0	0	\\
2.8e-09	1.2	0	0	\\
2.85e-09	1.2	0	0	\\
2.9e-09	1.2	0	0	\\
2.95e-09	1.2	0	0	\\
3e-09	1.2	0	0	\\
3.05e-09	1.2	0	0	\\
3.1e-09	1.2	0	0	\\
3.15e-09	1.2	0	0	\\
3.2e-09	1.2	0	0	\\
3.25e-09	1.2	0	0	\\
3.3e-09	1.2	0	0	\\
3.35e-09	1.2	0	0	\\
3.4e-09	1.2	0	0	\\
3.45e-09	1.2	0	0	\\
3.5e-09	1.2	0	0	\\
3.55e-09	1.2	0	0	\\
3.6e-09	1.2	0.00222222222222222	0.00222222222222222	\\
3.65e-09	1.2	0.00222222222222222	0.00222222222222222	\\
3.7e-09	1.2	0.00222222222222222	0.00222222222222222	\\
3.75e-09	1.2	0.00222222222222222	0.00222222222222222	\\
3.8e-09	1.2	0.00222222222222222	0.00222222222222222	\\
3.85e-09	1.2	0.00222222222222222	0.00222222222222222	\\
3.9e-09	1.2	0.00222222222222222	0.00222222222222222	\\
3.95e-09	1.2	0.00222222222222222	0.00222222222222222	\\
4e-09	1.2	0.00222222222222222	0.00222222222222222	\\
4.05e-09	1.2	0.00222222222222222	0.00222222222222222	\\
4.1e-09	1.2	0	0	\\
4.15e-09	1.2	0	0	\\
4.2e-09	1.2	0	0	\\
4.25e-09	1.2	0	0	\\
4.3e-09	1.2	0	0	\\
4.35e-09	1.2	0	0	\\
4.4e-09	1.2	0	0	\\
4.45e-09	1.2	0	0	\\
4.5e-09	1.2	0	0	\\
4.55e-09	1.2	0	0	\\
4.6e-09	1.2	0	0	\\
4.65e-09	1.2	0	0	\\
4.7e-09	1.2	0	0	\\
4.75e-09	1.2	0	0	\\
4.8e-09	1.2	0	0	\\
4.85e-09	1.2	0	0	\\
4.9e-09	1.2	0	0	\\
4.95e-09	1.2	0	0	\\
5e-09	1.2	0	0	\\
5e-09	1.2	0	nan	\\
5e-09	1.2	-0.00666666666666667	0.00166666666666667	\\
5e-09	1.2	-0.00666666666666667	nan	\\
};

\end{axis}
\end{tikzpicture}%
	\label{fig:wheel-torque-labcar}
	\caption{Current through the transmission channel with rectangular pulse.}
\end{figure}
\fi

The characteristic impedance of 

\section{Task 4}
The load impedance $Z_l$ is changed to $50 + j50 \Omega$, while keeping the configuration of the transmission channel as in Task 3. The delivered complex power to the load will be divided into real and imaginary power, with which the power efficiency can be calculated as:

\begin{equation}
	\eta = \frac{P}{S} \cdot 100\% = \frac{P}{P + Q} \cdot 100\%
\end{equation}

\iffalse
\begin{figure}[H]
	\centering
	\setlength\figureheight{4cm}
    	\setlength\figurewidth{0.8\linewidth}
	% This file was created by matlab2tikz v0.4.6 running on MATLAB 8.2.
% Copyright (c) 2008--2014, Nico Schlömer <nico.schloemer@gmail.com>
% All rights reserved.
% Minimal pgfplots version: 1.3
% 
% The latest updates can be retrieved from
%   http://www.mathworks.com/matlabcentral/fileexchange/22022-matlab2tikz
% where you can also make suggestions and rate matlab2tikz.
% 
%
% defining custom colors
\definecolor{mycolor1}{rgb}{0.00000,0.75000,0.75000}%
\definecolor{mycolor2}{rgb}{0.75000,0.00000,0.75000}%
\definecolor{mycolor3}{rgb}{0.75000,0.75000,0.00000}%
%
\begin{tikzpicture}

\begin{axis}[%
width=\figurewidth,
height=\figureheight,
scale only axis,
xmin=0,
xmax=5e-09,
ymin=0,
ymax=0.001
]
\addplot [color=blue,solid,forget plot]
  table[row sep=crcr]{
0	0	\\
1.1e-10	0	\\
2.2e-10	0	\\
3.3e-10	0	\\
4.4e-10	0	\\
5.4e-10	0	\\
6.5e-10	0	\\
7.5e-10	0	\\
8.6e-10	0	\\
9.6e-10	0	\\
1.07e-09	0	\\
1.18e-09	0	\\
1.2e-09	0.000763235689330825	\\
1.31e-09	0.000763235689330825	\\
1.41e-09	0.000763235689330825	\\
1.51e-09	0.000763235689330825	\\
1.62e-09	0.000763235689330825	\\
1.7e-09	0	\\
1.8e-09	0	\\
1.9e-09	0	\\
2.01e-09	0	\\
2.11e-09	0	\\
2.21e-09	0	\\
2.32e-09	0	\\
2.42e-09	0	\\
2.52e-09	0	\\
2.63e-09	0	\\
2.73e-09	0	\\
2.83e-09	0	\\
2.93e-09	0	\\
3.04e-09	0	\\
3.14e-09	0	\\
3.24e-09	0	\\
3.34e-09	0	\\
3.45e-09	0	\\
3.55e-09	0	\\
3.6e-09	0.000610698909513694	\\
3.7e-09	0.000610698909513694	\\
3.8e-09	0.000610698909513694	\\
3.9e-09	0.000610698909513694	\\
4e-09	0.000610698909513694	\\
4.1e-09	0	\\
4.2e-09	0	\\
4.3e-09	0	\\
4.41e-09	0	\\
4.51e-09	0	\\
4.61e-09	0	\\
4.71e-09	0	\\
4.82e-09	0	\\
4.92e-09	0	\\
5e-09	0	\\
};
\addplot [color=black!50!green,solid,forget plot]
  table[row sep=crcr]{
0	0	\\
1.1e-10	0	\\
2.2e-10	0	\\
3.3e-10	0	\\
4.4e-10	0	\\
5.4e-10	0	\\
6.5e-10	0	\\
7.5e-10	0	\\
8.6e-10	0	\\
9.6e-10	0	\\
1.07e-09	0	\\
1.18e-09	0	\\
1.28e-09	0	\\
1.38e-09	0	\\
1.49e-09	0	\\
1.59e-09	0	\\
1.69e-09	0	\\
1.8e-09	0	\\
1.9e-09	0	\\
2.01e-09	0	\\
2.11e-09	0	\\
2.21e-09	0	\\
2.32e-09	0	\\
2.42e-09	0	\\
2.52e-09	0	\\
2.63e-09	0	\\
2.73e-09	0	\\
2.83e-09	0	\\
2.93e-09	0	\\
3.04e-09	0	\\
3.14e-09	0	\\
3.24e-09	0	\\
3.34e-09	0	\\
3.45e-09	0	\\
3.55e-09	0	\\
3.65e-09	0	\\
3.75e-09	0	\\
3.86e-09	0	\\
3.96e-09	0	\\
4.06e-09	0	\\
4.16e-09	0	\\
4.27e-09	0	\\
4.37e-09	0	\\
4.47e-09	0	\\
4.57e-09	0	\\
4.68e-09	0	\\
4.78e-09	0	\\
4.89e-09	0	\\
4.99e-09	0	\\
5e-09	0	\\
};
\addplot [color=red,solid,forget plot]
  table[row sep=crcr]{
0	0	\\
1.1e-10	0	\\
2.2e-10	0	\\
3.3e-10	0	\\
4.4e-10	0	\\
5.4e-10	0	\\
6.5e-10	0	\\
7.5e-10	0	\\
8.6e-10	0	\\
9.6e-10	0	\\
1.07e-09	0	\\
1.18e-09	0	\\
1.28e-09	0	\\
1.38e-09	0	\\
1.49e-09	0	\\
1.59e-09	0	\\
1.69e-09	0	\\
1.8e-09	0	\\
1.9e-09	0	\\
2.01e-09	0	\\
2.11e-09	0	\\
2.21e-09	0	\\
2.32e-09	0	\\
2.42e-09	0	\\
2.52e-09	0	\\
2.63e-09	0	\\
2.73e-09	0	\\
2.83e-09	0	\\
2.93e-09	0	\\
3.04e-09	0	\\
3.14e-09	0	\\
3.24e-09	0	\\
3.34e-09	0	\\
3.45e-09	0	\\
3.55e-09	0	\\
3.65e-09	0	\\
3.75e-09	0	\\
3.86e-09	0	\\
3.96e-09	0	\\
4.06e-09	0	\\
4.16e-09	0	\\
4.27e-09	0	\\
4.37e-09	0	\\
4.47e-09	0	\\
4.57e-09	0	\\
4.68e-09	0	\\
4.78e-09	0	\\
4.89e-09	0	\\
4.99e-09	0	\\
5e-09	0	\\
};
\addplot [color=mycolor1,solid,forget plot]
  table[row sep=crcr]{
0	0	\\
1.1e-10	0	\\
2.2e-10	0	\\
3.3e-10	0	\\
4.4e-10	0	\\
5.4e-10	0	\\
6.5e-10	0	\\
7.5e-10	0	\\
8.6e-10	0	\\
9.6e-10	0	\\
1.07e-09	0	\\
1.18e-09	0	\\
1.28e-09	0	\\
1.38e-09	0	\\
1.49e-09	0	\\
1.59e-09	0	\\
1.69e-09	0	\\
1.8e-09	0	\\
1.9e-09	0	\\
2.01e-09	0	\\
2.11e-09	0	\\
2.21e-09	0	\\
2.32e-09	0	\\
2.42e-09	0	\\
2.52e-09	0	\\
2.63e-09	0	\\
2.73e-09	0	\\
2.83e-09	0	\\
2.93e-09	0	\\
3.04e-09	0	\\
3.14e-09	0	\\
3.24e-09	0	\\
3.34e-09	0	\\
3.45e-09	0	\\
3.55e-09	0	\\
3.65e-09	0	\\
3.75e-09	0	\\
3.86e-09	0	\\
3.96e-09	0	\\
4.06e-09	0	\\
4.16e-09	0	\\
4.27e-09	0	\\
4.37e-09	0	\\
4.47e-09	0	\\
4.57e-09	0	\\
4.68e-09	0	\\
4.78e-09	0	\\
4.89e-09	0	\\
4.99e-09	0	\\
5e-09	0	\\
};
\addplot [color=mycolor2,solid,forget plot]
  table[row sep=crcr]{
0	0	\\
1.1e-10	0	\\
2.2e-10	0	\\
3.3e-10	0	\\
4.4e-10	0	\\
5.4e-10	0	\\
6.5e-10	0	\\
7.5e-10	0	\\
8.6e-10	0	\\
9.6e-10	0	\\
1.07e-09	0	\\
1.18e-09	0	\\
1.28e-09	0	\\
1.38e-09	0	\\
1.49e-09	0	\\
1.59e-09	0	\\
1.69e-09	0	\\
1.8e-09	0	\\
1.9e-09	0	\\
2.01e-09	0	\\
2.11e-09	0	\\
2.21e-09	0	\\
2.32e-09	0	\\
2.42e-09	0	\\
2.52e-09	0	\\
2.63e-09	0	\\
2.73e-09	0	\\
2.83e-09	0	\\
2.93e-09	0	\\
3.04e-09	0	\\
3.14e-09	0	\\
3.24e-09	0	\\
3.34e-09	0	\\
3.45e-09	0	\\
3.55e-09	0	\\
3.65e-09	0	\\
3.75e-09	0	\\
3.86e-09	0	\\
3.96e-09	0	\\
4.06e-09	0	\\
4.16e-09	0	\\
4.27e-09	0	\\
4.37e-09	0	\\
4.47e-09	0	\\
4.57e-09	0	\\
4.68e-09	0	\\
4.78e-09	0	\\
4.89e-09	0	\\
4.99e-09	0	\\
5e-09	0	\\
};
\addplot [color=mycolor3,solid,forget plot]
  table[row sep=crcr]{
0	0	\\
1.1e-10	0	\\
2.2e-10	0	\\
3.3e-10	0	\\
4.4e-10	0	\\
5.4e-10	0	\\
6.5e-10	0	\\
7.5e-10	0	\\
8.6e-10	0	\\
9.6e-10	0	\\
1.07e-09	0	\\
1.18e-09	0	\\
1.28e-09	0	\\
1.38e-09	0	\\
1.49e-09	0	\\
1.59e-09	0	\\
1.69e-09	0	\\
1.8e-09	0	\\
1.9e-09	0	\\
2.01e-09	0	\\
2.11e-09	0	\\
2.21e-09	0	\\
2.32e-09	0	\\
2.42e-09	0	\\
2.52e-09	0	\\
2.63e-09	0	\\
2.73e-09	0	\\
2.83e-09	0	\\
2.93e-09	0	\\
3.04e-09	0	\\
3.14e-09	0	\\
3.24e-09	0	\\
3.34e-09	0	\\
3.45e-09	0	\\
3.55e-09	0	\\
3.65e-09	0	\\
3.75e-09	0	\\
3.86e-09	0	\\
3.96e-09	0	\\
4.06e-09	0	\\
4.16e-09	0	\\
4.27e-09	0	\\
4.37e-09	0	\\
4.47e-09	0	\\
4.57e-09	0	\\
4.68e-09	0	\\
4.78e-09	0	\\
4.89e-09	0	\\
4.99e-09	0	\\
5e-09	0	\\
};
\addplot [color=darkgray,solid,forget plot]
  table[row sep=crcr]{
0	0	\\
1.1e-10	0	\\
2.2e-10	0	\\
3.3e-10	0	\\
4.4e-10	0	\\
5.4e-10	0	\\
6.5e-10	0	\\
7.5e-10	0	\\
8.6e-10	0	\\
9.6e-10	0	\\
1.07e-09	0	\\
1.18e-09	0	\\
1.28e-09	0	\\
1.38e-09	0	\\
1.49e-09	0	\\
1.59e-09	0	\\
1.69e-09	0	\\
1.8e-09	0	\\
1.9e-09	0	\\
2.01e-09	0	\\
2.11e-09	0	\\
2.21e-09	0	\\
2.32e-09	0	\\
2.42e-09	0	\\
2.52e-09	0	\\
2.63e-09	0	\\
2.73e-09	0	\\
2.83e-09	0	\\
2.93e-09	0	\\
3.04e-09	0	\\
3.14e-09	0	\\
3.24e-09	0	\\
3.34e-09	0	\\
3.45e-09	0	\\
3.55e-09	0	\\
3.65e-09	0	\\
3.75e-09	0	\\
3.86e-09	0	\\
3.96e-09	0	\\
4.06e-09	0	\\
4.16e-09	0	\\
4.27e-09	0	\\
4.37e-09	0	\\
4.47e-09	0	\\
4.57e-09	0	\\
4.68e-09	0	\\
4.78e-09	0	\\
4.89e-09	0	\\
4.99e-09	0	\\
5e-09	0	\\
};
\addplot [color=blue,solid,forget plot]
  table[row sep=crcr]{
0	0	\\
1.1e-10	0	\\
2.2e-10	0	\\
3.3e-10	0	\\
4.4e-10	0	\\
5.4e-10	0	\\
6.5e-10	0	\\
7.5e-10	0	\\
8.6e-10	0	\\
9.6e-10	0	\\
1.07e-09	0	\\
1.18e-09	0	\\
1.28e-09	0	\\
1.38e-09	0	\\
1.49e-09	0	\\
1.59e-09	0	\\
1.69e-09	0	\\
1.8e-09	0	\\
1.9e-09	0	\\
2.01e-09	0	\\
2.11e-09	0	\\
2.21e-09	0	\\
2.32e-09	0	\\
2.42e-09	0	\\
2.52e-09	0	\\
2.63e-09	0	\\
2.73e-09	0	\\
2.83e-09	0	\\
2.93e-09	0	\\
3.04e-09	0	\\
3.14e-09	0	\\
3.24e-09	0	\\
3.34e-09	0	\\
3.45e-09	0	\\
3.55e-09	0	\\
3.65e-09	0	\\
3.75e-09	0	\\
3.86e-09	0	\\
3.96e-09	0	\\
4.06e-09	0	\\
4.16e-09	0	\\
4.27e-09	0	\\
4.37e-09	0	\\
4.47e-09	0	\\
4.57e-09	0	\\
4.68e-09	0	\\
4.78e-09	0	\\
4.89e-09	0	\\
4.99e-09	0	\\
5e-09	0	\\
};
\addplot [color=black!50!green,solid,forget plot]
  table[row sep=crcr]{
0	0	\\
1.1e-10	0	\\
2.2e-10	0	\\
3.3e-10	0	\\
4.4e-10	0	\\
5.4e-10	0	\\
6.5e-10	0	\\
7.5e-10	0	\\
8.6e-10	0	\\
9.6e-10	0	\\
1.07e-09	0	\\
1.18e-09	0	\\
1.28e-09	0	\\
1.38e-09	0	\\
1.49e-09	0	\\
1.59e-09	0	\\
1.69e-09	0	\\
1.8e-09	0	\\
1.9e-09	0	\\
2.01e-09	0	\\
2.11e-09	0	\\
2.21e-09	0	\\
2.32e-09	0	\\
2.42e-09	0	\\
2.52e-09	0	\\
2.63e-09	0	\\
2.73e-09	0	\\
2.83e-09	0	\\
2.93e-09	0	\\
3.04e-09	0	\\
3.14e-09	0	\\
3.24e-09	0	\\
3.34e-09	0	\\
3.45e-09	0	\\
3.55e-09	0	\\
3.65e-09	0	\\
3.75e-09	0	\\
3.86e-09	0	\\
3.96e-09	0	\\
4.06e-09	0	\\
4.16e-09	0	\\
4.27e-09	0	\\
4.37e-09	0	\\
4.47e-09	0	\\
4.57e-09	0	\\
4.68e-09	0	\\
4.78e-09	0	\\
4.89e-09	0	\\
4.99e-09	0	\\
5e-09	0	\\
};
\addplot [color=red,solid,forget plot]
  table[row sep=crcr]{
0	0	\\
1.1e-10	0	\\
2.2e-10	0	\\
3.3e-10	0	\\
4.4e-10	0	\\
5.4e-10	0	\\
6.5e-10	0	\\
7.5e-10	0	\\
8.6e-10	0	\\
9.6e-10	0	\\
1.07e-09	0	\\
1.18e-09	0	\\
1.28e-09	0	\\
1.38e-09	0	\\
1.49e-09	0	\\
1.59e-09	0	\\
1.69e-09	0	\\
1.8e-09	0	\\
1.9e-09	0	\\
2.01e-09	0	\\
2.11e-09	0	\\
2.21e-09	0	\\
2.32e-09	0	\\
2.42e-09	0	\\
2.52e-09	0	\\
2.63e-09	0	\\
2.73e-09	0	\\
2.83e-09	0	\\
2.93e-09	0	\\
3.04e-09	0	\\
3.14e-09	0	\\
3.24e-09	0	\\
3.34e-09	0	\\
3.45e-09	0	\\
3.55e-09	0	\\
3.65e-09	0	\\
3.75e-09	0	\\
3.86e-09	0	\\
3.96e-09	0	\\
4.06e-09	0	\\
4.16e-09	0	\\
4.27e-09	0	\\
4.37e-09	0	\\
4.47e-09	0	\\
4.57e-09	0	\\
4.68e-09	0	\\
4.78e-09	0	\\
4.89e-09	0	\\
4.99e-09	0	\\
5e-09	0	\\
};
\addplot [color=mycolor1,solid,forget plot]
  table[row sep=crcr]{
0	0	\\
1.1e-10	0	\\
2.2e-10	0	\\
3.3e-10	0	\\
4.4e-10	0	\\
5.4e-10	0	\\
6.5e-10	0	\\
7.5e-10	0	\\
8.6e-10	0	\\
9.6e-10	0	\\
1.07e-09	0	\\
1.18e-09	0	\\
1.28e-09	0	\\
1.38e-09	0	\\
1.49e-09	0	\\
1.59e-09	0	\\
1.69e-09	0	\\
1.8e-09	0	\\
1.9e-09	0	\\
2.01e-09	0	\\
2.11e-09	0	\\
2.21e-09	0	\\
2.32e-09	0	\\
2.42e-09	0	\\
2.52e-09	0	\\
2.63e-09	0	\\
2.73e-09	0	\\
2.83e-09	0	\\
2.93e-09	0	\\
3.04e-09	0	\\
3.14e-09	0	\\
3.24e-09	0	\\
3.34e-09	0	\\
3.45e-09	0	\\
3.55e-09	0	\\
3.65e-09	0	\\
3.75e-09	0	\\
3.86e-09	0	\\
3.96e-09	0	\\
4.06e-09	0	\\
4.16e-09	0	\\
4.27e-09	0	\\
4.37e-09	0	\\
4.47e-09	0	\\
4.57e-09	0	\\
4.68e-09	0	\\
4.78e-09	0	\\
4.89e-09	0	\\
4.99e-09	0	\\
5e-09	0	\\
};
\addplot [color=mycolor2,solid,forget plot]
  table[row sep=crcr]{
0	0	\\
1.1e-10	0	\\
2.2e-10	0	\\
3.3e-10	0	\\
4.4e-10	0	\\
5.4e-10	0	\\
6.5e-10	0	\\
7.5e-10	0	\\
8.6e-10	0	\\
9.6e-10	0	\\
1.07e-09	0	\\
1.18e-09	0	\\
1.28e-09	0	\\
1.38e-09	0	\\
1.49e-09	0	\\
1.59e-09	0	\\
1.69e-09	0	\\
1.8e-09	0	\\
1.9e-09	0	\\
2.01e-09	0	\\
2.11e-09	0	\\
2.21e-09	0	\\
2.32e-09	0	\\
2.42e-09	0	\\
2.52e-09	0	\\
2.63e-09	0	\\
2.73e-09	0	\\
2.83e-09	0	\\
2.93e-09	0	\\
3.04e-09	0	\\
3.14e-09	0	\\
3.24e-09	0	\\
3.34e-09	0	\\
3.45e-09	0	\\
3.55e-09	0	\\
3.65e-09	0	\\
3.75e-09	0	\\
3.86e-09	0	\\
3.96e-09	0	\\
4.06e-09	0	\\
4.16e-09	0	\\
4.27e-09	0	\\
4.37e-09	0	\\
4.47e-09	0	\\
4.57e-09	0	\\
4.68e-09	0	\\
4.78e-09	0	\\
4.89e-09	0	\\
4.99e-09	0	\\
5e-09	0	\\
};
\addplot [color=mycolor3,solid,forget plot]
  table[row sep=crcr]{
0	0	\\
1.1e-10	0	\\
2.2e-10	0	\\
3.3e-10	0	\\
4.4e-10	0	\\
5.4e-10	0	\\
6.5e-10	0	\\
7.5e-10	0	\\
8.6e-10	0	\\
9.6e-10	0	\\
1.07e-09	0	\\
1.18e-09	0	\\
1.28e-09	0	\\
1.38e-09	0	\\
1.49e-09	0	\\
1.59e-09	0	\\
1.69e-09	0	\\
1.8e-09	0	\\
1.9e-09	0	\\
2.01e-09	0	\\
2.11e-09	0	\\
2.21e-09	0	\\
2.32e-09	0	\\
2.42e-09	0	\\
2.52e-09	0	\\
2.63e-09	0	\\
2.73e-09	0	\\
2.83e-09	0	\\
2.93e-09	0	\\
3.04e-09	0	\\
3.14e-09	0	\\
3.24e-09	0	\\
3.34e-09	0	\\
3.45e-09	0	\\
3.55e-09	0	\\
3.65e-09	0	\\
3.75e-09	0	\\
3.86e-09	0	\\
3.96e-09	0	\\
4.06e-09	0	\\
4.16e-09	0	\\
4.27e-09	0	\\
4.37e-09	0	\\
4.47e-09	0	\\
4.57e-09	0	\\
4.68e-09	0	\\
4.78e-09	0	\\
4.89e-09	0	\\
4.99e-09	0	\\
5e-09	0	\\
};
\addplot [color=darkgray,solid,forget plot]
  table[row sep=crcr]{
0	0	\\
1.1e-10	0	\\
2.2e-10	0	\\
3.3e-10	0	\\
4.4e-10	0	\\
5.4e-10	0	\\
6.5e-10	0	\\
7.5e-10	0	\\
8.6e-10	0	\\
9.6e-10	0	\\
1.07e-09	0	\\
1.18e-09	0	\\
1.28e-09	0	\\
1.38e-09	0	\\
1.49e-09	0	\\
1.59e-09	0	\\
1.69e-09	0	\\
1.8e-09	0	\\
1.9e-09	0	\\
2.01e-09	0	\\
2.11e-09	0	\\
2.21e-09	0	\\
2.32e-09	0	\\
2.42e-09	0	\\
2.52e-09	0	\\
2.63e-09	0	\\
2.73e-09	0	\\
2.83e-09	0	\\
2.93e-09	0	\\
3.04e-09	0	\\
3.14e-09	0	\\
3.24e-09	0	\\
3.34e-09	0	\\
3.45e-09	0	\\
3.55e-09	0	\\
3.65e-09	0	\\
3.75e-09	0	\\
3.86e-09	0	\\
3.96e-09	0	\\
4.06e-09	0	\\
4.16e-09	0	\\
4.27e-09	0	\\
4.37e-09	0	\\
4.47e-09	0	\\
4.57e-09	0	\\
4.68e-09	0	\\
4.78e-09	0	\\
4.89e-09	0	\\
4.99e-09	0	\\
5e-09	0	\\
};
\addplot [color=blue,solid,forget plot]
  table[row sep=crcr]{
0	0	\\
1.1e-10	0	\\
2.2e-10	0	\\
3.3e-10	0	\\
4.4e-10	0	\\
5.4e-10	0	\\
6.5e-10	0	\\
7.5e-10	0	\\
8.6e-10	0	\\
9.6e-10	0	\\
1.07e-09	0	\\
1.18e-09	0	\\
1.28e-09	0	\\
1.38e-09	0	\\
1.49e-09	0	\\
1.59e-09	0	\\
1.69e-09	0	\\
1.8e-09	0	\\
1.9e-09	0	\\
2.01e-09	0	\\
2.11e-09	0	\\
2.21e-09	0	\\
2.32e-09	0	\\
2.42e-09	0	\\
2.52e-09	0	\\
2.63e-09	0	\\
2.73e-09	0	\\
2.83e-09	0	\\
2.93e-09	0	\\
3.04e-09	0	\\
3.14e-09	0	\\
3.24e-09	0	\\
3.34e-09	0	\\
3.45e-09	0	\\
3.55e-09	0	\\
3.65e-09	0	\\
3.75e-09	0	\\
3.86e-09	0	\\
3.96e-09	0	\\
4.06e-09	0	\\
4.16e-09	0	\\
4.27e-09	0	\\
4.37e-09	0	\\
4.47e-09	0	\\
4.57e-09	0	\\
4.68e-09	0	\\
4.78e-09	0	\\
4.89e-09	0	\\
4.99e-09	0	\\
5e-09	0	\\
};
\addplot [color=black!50!green,solid,forget plot]
  table[row sep=crcr]{
0	0	\\
1.1e-10	0	\\
2.2e-10	0	\\
3.3e-10	0	\\
4.4e-10	0	\\
5.4e-10	0	\\
6.5e-10	0	\\
7.5e-10	0	\\
8.6e-10	0	\\
9.6e-10	0	\\
1.07e-09	0	\\
1.18e-09	0	\\
1.28e-09	0	\\
1.38e-09	0	\\
1.49e-09	0	\\
1.59e-09	0	\\
1.69e-09	0	\\
1.8e-09	0	\\
1.9e-09	0	\\
2.01e-09	0	\\
2.11e-09	0	\\
2.21e-09	0	\\
2.32e-09	0	\\
2.42e-09	0	\\
2.52e-09	0	\\
2.63e-09	0	\\
2.73e-09	0	\\
2.83e-09	0	\\
2.93e-09	0	\\
3.04e-09	0	\\
3.14e-09	0	\\
3.24e-09	0	\\
3.34e-09	0	\\
3.45e-09	0	\\
3.55e-09	0	\\
3.65e-09	0	\\
3.75e-09	0	\\
3.86e-09	0	\\
3.96e-09	0	\\
4.06e-09	0	\\
4.16e-09	0	\\
4.27e-09	0	\\
4.37e-09	0	\\
4.47e-09	0	\\
4.57e-09	0	\\
4.68e-09	0	\\
4.78e-09	0	\\
4.89e-09	0	\\
4.99e-09	0	\\
5e-09	0	\\
};
\addplot [color=red,solid,forget plot]
  table[row sep=crcr]{
0	0	\\
1.1e-10	0	\\
2.2e-10	0	\\
3.3e-10	0	\\
4.4e-10	0	\\
5.4e-10	0	\\
6.5e-10	0	\\
7.5e-10	0	\\
8.6e-10	0	\\
9.6e-10	0	\\
1.07e-09	0	\\
1.18e-09	0	\\
1.28e-09	0	\\
1.38e-09	0	\\
1.49e-09	0	\\
1.59e-09	0	\\
1.69e-09	0	\\
1.8e-09	0	\\
1.9e-09	0	\\
2.01e-09	0	\\
2.11e-09	0	\\
2.21e-09	0	\\
2.32e-09	0	\\
2.42e-09	0	\\
2.52e-09	0	\\
2.63e-09	0	\\
2.73e-09	0	\\
2.83e-09	0	\\
2.93e-09	0	\\
3.04e-09	0	\\
3.14e-09	0	\\
3.24e-09	0	\\
3.34e-09	0	\\
3.45e-09	0	\\
3.55e-09	0	\\
3.65e-09	0	\\
3.75e-09	0	\\
3.86e-09	0	\\
3.96e-09	0	\\
4.06e-09	0	\\
4.16e-09	0	\\
4.27e-09	0	\\
4.37e-09	0	\\
4.47e-09	0	\\
4.57e-09	0	\\
4.68e-09	0	\\
4.78e-09	0	\\
4.89e-09	0	\\
4.99e-09	0	\\
5e-09	0	\\
};
\addplot [color=mycolor1,solid,forget plot]
  table[row sep=crcr]{
0	0	\\
1.1e-10	0	\\
2.2e-10	0	\\
3.3e-10	0	\\
4.4e-10	0	\\
5.4e-10	0	\\
6.5e-10	0	\\
7.5e-10	0	\\
8.6e-10	0	\\
9.6e-10	0	\\
1.07e-09	0	\\
1.18e-09	0	\\
1.28e-09	0	\\
1.38e-09	0	\\
1.49e-09	0	\\
1.59e-09	0	\\
1.69e-09	0	\\
1.8e-09	0	\\
1.9e-09	0	\\
2.01e-09	0	\\
2.11e-09	0	\\
2.21e-09	0	\\
2.32e-09	0	\\
2.42e-09	0	\\
2.52e-09	0	\\
2.63e-09	0	\\
2.73e-09	0	\\
2.83e-09	0	\\
2.93e-09	0	\\
3.04e-09	0	\\
3.14e-09	0	\\
3.24e-09	0	\\
3.34e-09	0	\\
3.45e-09	0	\\
3.55e-09	0	\\
3.65e-09	0	\\
3.75e-09	0	\\
3.86e-09	0	\\
3.96e-09	0	\\
4.06e-09	0	\\
4.16e-09	0	\\
4.27e-09	0	\\
4.37e-09	0	\\
4.47e-09	0	\\
4.57e-09	0	\\
4.68e-09	0	\\
4.78e-09	0	\\
4.89e-09	0	\\
4.99e-09	0	\\
5e-09	0	\\
};
\addplot [color=mycolor2,solid,forget plot]
  table[row sep=crcr]{
0	0	\\
1.1e-10	0	\\
2.2e-10	0	\\
3.3e-10	0	\\
4.4e-10	0	\\
5.4e-10	0	\\
6.5e-10	0	\\
7.5e-10	0	\\
8.6e-10	0	\\
9.6e-10	0	\\
1.07e-09	0	\\
1.18e-09	0	\\
1.28e-09	0	\\
1.38e-09	0	\\
1.49e-09	0	\\
1.59e-09	0	\\
1.69e-09	0	\\
1.8e-09	0	\\
1.9e-09	0	\\
2.01e-09	0	\\
2.11e-09	0	\\
2.21e-09	0	\\
2.32e-09	0	\\
2.42e-09	0	\\
2.52e-09	0	\\
2.63e-09	0	\\
2.73e-09	0	\\
2.83e-09	0	\\
2.93e-09	0	\\
3.04e-09	0	\\
3.14e-09	0	\\
3.24e-09	0	\\
3.34e-09	0	\\
3.45e-09	0	\\
3.55e-09	0	\\
3.65e-09	0	\\
3.75e-09	0	\\
3.86e-09	0	\\
3.96e-09	0	\\
4.06e-09	0	\\
4.16e-09	0	\\
4.27e-09	0	\\
4.37e-09	0	\\
4.47e-09	0	\\
4.57e-09	0	\\
4.68e-09	0	\\
4.78e-09	0	\\
4.89e-09	0	\\
4.99e-09	0	\\
5e-09	0	\\
};
\addplot [color=mycolor3,solid,forget plot]
  table[row sep=crcr]{
0	0	\\
1.1e-10	0	\\
2.2e-10	0	\\
3.3e-10	0	\\
4.4e-10	0	\\
5.4e-10	0	\\
6.5e-10	0	\\
7.5e-10	0	\\
8.6e-10	0	\\
9.6e-10	0	\\
1.07e-09	0	\\
1.18e-09	0	\\
1.28e-09	0	\\
1.38e-09	0	\\
1.49e-09	0	\\
1.59e-09	0	\\
1.69e-09	0	\\
1.8e-09	0	\\
1.9e-09	0	\\
2.01e-09	0	\\
2.11e-09	0	\\
2.21e-09	0	\\
2.32e-09	0	\\
2.42e-09	0	\\
2.52e-09	0	\\
2.63e-09	0	\\
2.73e-09	0	\\
2.83e-09	0	\\
2.93e-09	0	\\
3.04e-09	0	\\
3.14e-09	0	\\
3.24e-09	0	\\
3.34e-09	0	\\
3.45e-09	0	\\
3.55e-09	0	\\
3.65e-09	0	\\
3.75e-09	0	\\
3.86e-09	0	\\
3.96e-09	0	\\
4.06e-09	0	\\
4.16e-09	0	\\
4.27e-09	0	\\
4.37e-09	0	\\
4.47e-09	0	\\
4.57e-09	0	\\
4.68e-09	0	\\
4.78e-09	0	\\
4.89e-09	0	\\
4.99e-09	0	\\
5e-09	0	\\
};
\addplot [color=darkgray,solid,forget plot]
  table[row sep=crcr]{
0	0	\\
1.1e-10	0	\\
2.2e-10	0	\\
3.3e-10	0	\\
4.4e-10	0	\\
5.4e-10	0	\\
6.5e-10	0	\\
7.5e-10	0	\\
8.6e-10	0	\\
9.6e-10	0	\\
1.07e-09	0	\\
1.18e-09	0	\\
1.28e-09	0	\\
1.38e-09	0	\\
1.49e-09	0	\\
1.59e-09	0	\\
1.69e-09	0	\\
1.8e-09	0	\\
1.9e-09	0	\\
2.01e-09	0	\\
2.11e-09	0	\\
2.21e-09	0	\\
2.32e-09	0	\\
2.42e-09	0	\\
2.52e-09	0	\\
2.63e-09	0	\\
2.73e-09	0	\\
2.83e-09	0	\\
2.93e-09	0	\\
3.04e-09	0	\\
3.14e-09	0	\\
3.24e-09	0	\\
3.34e-09	0	\\
3.45e-09	0	\\
3.55e-09	0	\\
3.65e-09	0	\\
3.75e-09	0	\\
3.86e-09	0	\\
3.96e-09	0	\\
4.06e-09	0	\\
4.16e-09	0	\\
4.27e-09	0	\\
4.37e-09	0	\\
4.47e-09	0	\\
4.57e-09	0	\\
4.68e-09	0	\\
4.78e-09	0	\\
4.89e-09	0	\\
4.99e-09	0	\\
5e-09	0	\\
};
\addplot [color=blue,solid,forget plot]
  table[row sep=crcr]{
0	0	\\
1.1e-10	0	\\
2.2e-10	0	\\
3.3e-10	0	\\
4.4e-10	0	\\
5.4e-10	0	\\
6.5e-10	0	\\
7.5e-10	0	\\
8.6e-10	0	\\
9.6e-10	0	\\
1.07e-09	0	\\
1.18e-09	0	\\
1.28e-09	0	\\
1.38e-09	0	\\
1.49e-09	0	\\
1.59e-09	0	\\
1.69e-09	0	\\
1.8e-09	0	\\
1.9e-09	0	\\
2.01e-09	0	\\
2.11e-09	0	\\
2.21e-09	0	\\
2.32e-09	0	\\
2.42e-09	0	\\
2.52e-09	0	\\
2.63e-09	0	\\
2.73e-09	0	\\
2.83e-09	0	\\
2.93e-09	0	\\
3.04e-09	0	\\
3.14e-09	0	\\
3.24e-09	0	\\
3.34e-09	0	\\
3.45e-09	0	\\
3.55e-09	0	\\
3.65e-09	0	\\
3.75e-09	0	\\
3.86e-09	0	\\
3.96e-09	0	\\
4.06e-09	0	\\
4.16e-09	0	\\
4.27e-09	0	\\
4.37e-09	0	\\
4.47e-09	0	\\
4.57e-09	0	\\
4.68e-09	0	\\
4.78e-09	0	\\
4.89e-09	0	\\
4.99e-09	0	\\
5e-09	0	\\
};
\addplot [color=black!50!green,solid,forget plot]
  table[row sep=crcr]{
0	0	\\
1.1e-10	0	\\
2.2e-10	0	\\
3.3e-10	0	\\
4.4e-10	0	\\
5.4e-10	0	\\
6.5e-10	0	\\
7.5e-10	0	\\
8.6e-10	0	\\
9.6e-10	0	\\
1.07e-09	0	\\
1.18e-09	0	\\
1.28e-09	0	\\
1.38e-09	0	\\
1.49e-09	0	\\
1.59e-09	0	\\
1.69e-09	0	\\
1.8e-09	0	\\
1.9e-09	0	\\
2.01e-09	0	\\
2.11e-09	0	\\
2.21e-09	0	\\
2.32e-09	0	\\
2.42e-09	0	\\
2.52e-09	0	\\
2.63e-09	0	\\
2.73e-09	0	\\
2.83e-09	0	\\
2.93e-09	0	\\
3.04e-09	0	\\
3.14e-09	0	\\
3.24e-09	0	\\
3.34e-09	0	\\
3.45e-09	0	\\
3.55e-09	0	\\
3.65e-09	0	\\
3.75e-09	0	\\
3.86e-09	0	\\
3.96e-09	0	\\
4.06e-09	0	\\
4.16e-09	0	\\
4.27e-09	0	\\
4.37e-09	0	\\
4.47e-09	0	\\
4.57e-09	0	\\
4.68e-09	0	\\
4.78e-09	0	\\
4.89e-09	0	\\
4.99e-09	0	\\
5e-09	0	\\
};
\addplot [color=red,solid,forget plot]
  table[row sep=crcr]{
0	0	\\
1.1e-10	0	\\
2.2e-10	0	\\
3.3e-10	0	\\
4.4e-10	0	\\
5.4e-10	0	\\
6.5e-10	0	\\
7.5e-10	0	\\
8.6e-10	0	\\
9.6e-10	0	\\
1.07e-09	0	\\
1.18e-09	0	\\
1.28e-09	0	\\
1.38e-09	0	\\
1.49e-09	0	\\
1.59e-09	0	\\
1.69e-09	0	\\
1.8e-09	0	\\
1.9e-09	0	\\
2.01e-09	0	\\
2.11e-09	0	\\
2.21e-09	0	\\
2.32e-09	0	\\
2.42e-09	0	\\
2.52e-09	0	\\
2.63e-09	0	\\
2.73e-09	0	\\
2.83e-09	0	\\
2.93e-09	0	\\
3.04e-09	0	\\
3.14e-09	0	\\
3.24e-09	0	\\
3.34e-09	0	\\
3.45e-09	0	\\
3.55e-09	0	\\
3.65e-09	0	\\
3.75e-09	0	\\
3.86e-09	0	\\
3.96e-09	0	\\
4.06e-09	0	\\
4.16e-09	0	\\
4.27e-09	0	\\
4.37e-09	0	\\
4.47e-09	0	\\
4.57e-09	0	\\
4.68e-09	0	\\
4.78e-09	0	\\
4.89e-09	0	\\
4.99e-09	0	\\
5e-09	0	\\
};
\addplot [color=mycolor1,solid,forget plot]
  table[row sep=crcr]{
0	0	\\
1.1e-10	0	\\
2.2e-10	0	\\
3.3e-10	0	\\
4.4e-10	0	\\
5.4e-10	0	\\
6.5e-10	0	\\
7.5e-10	0	\\
8.6e-10	0	\\
9.6e-10	0	\\
1.07e-09	0	\\
1.18e-09	0	\\
1.28e-09	0	\\
1.38e-09	0	\\
1.49e-09	0	\\
1.59e-09	0	\\
1.69e-09	0	\\
1.8e-09	0	\\
1.9e-09	0	\\
2.01e-09	0	\\
2.11e-09	0	\\
2.21e-09	0	\\
2.32e-09	0	\\
2.42e-09	0	\\
2.52e-09	0	\\
2.63e-09	0	\\
2.73e-09	0	\\
2.83e-09	0	\\
2.93e-09	0	\\
3.04e-09	0	\\
3.14e-09	0	\\
3.24e-09	0	\\
3.34e-09	0	\\
3.45e-09	0	\\
3.55e-09	0	\\
3.65e-09	0	\\
3.75e-09	0	\\
3.86e-09	0	\\
3.96e-09	0	\\
4.06e-09	0	\\
4.16e-09	0	\\
4.27e-09	0	\\
4.37e-09	0	\\
4.47e-09	0	\\
4.57e-09	0	\\
4.68e-09	0	\\
4.78e-09	0	\\
4.89e-09	0	\\
4.99e-09	0	\\
5e-09	0	\\
};
\addplot [color=mycolor2,solid,forget plot]
  table[row sep=crcr]{
0	0	\\
1.1e-10	0	\\
2.2e-10	0	\\
3.3e-10	0	\\
4.4e-10	0	\\
5.4e-10	0	\\
6.5e-10	0	\\
7.5e-10	0	\\
8.6e-10	0	\\
9.6e-10	0	\\
1.07e-09	0	\\
1.18e-09	0	\\
1.28e-09	0	\\
1.38e-09	0	\\
1.49e-09	0	\\
1.59e-09	0	\\
1.69e-09	0	\\
1.8e-09	0	\\
1.9e-09	0	\\
2.01e-09	0	\\
2.11e-09	0	\\
2.21e-09	0	\\
2.32e-09	0	\\
2.42e-09	0	\\
2.52e-09	0	\\
2.63e-09	0	\\
2.73e-09	0	\\
2.83e-09	0	\\
2.93e-09	0	\\
3.04e-09	0	\\
3.14e-09	0	\\
3.24e-09	0	\\
3.34e-09	0	\\
3.45e-09	0	\\
3.55e-09	0	\\
3.65e-09	0	\\
3.75e-09	0	\\
3.86e-09	0	\\
3.96e-09	0	\\
4.06e-09	0	\\
4.16e-09	0	\\
4.27e-09	0	\\
4.37e-09	0	\\
4.47e-09	0	\\
4.57e-09	0	\\
4.68e-09	0	\\
4.78e-09	0	\\
4.89e-09	0	\\
4.99e-09	0	\\
5e-09	0	\\
};
\addplot [color=mycolor3,solid,forget plot]
  table[row sep=crcr]{
0	0	\\
1.1e-10	0	\\
2.2e-10	0	\\
3.3e-10	0	\\
4.4e-10	0	\\
5.4e-10	0	\\
6.5e-10	0	\\
7.5e-10	0	\\
8.6e-10	0	\\
9.6e-10	0	\\
1.07e-09	0	\\
1.18e-09	0	\\
1.28e-09	0	\\
1.38e-09	0	\\
1.49e-09	0	\\
1.59e-09	0	\\
1.69e-09	0	\\
1.8e-09	0	\\
1.9e-09	0	\\
2.01e-09	0	\\
2.11e-09	0	\\
2.21e-09	0	\\
2.32e-09	0	\\
2.42e-09	0	\\
2.52e-09	0	\\
2.63e-09	0	\\
2.73e-09	0	\\
2.83e-09	0	\\
2.93e-09	0	\\
3.04e-09	0	\\
3.14e-09	0	\\
3.24e-09	0	\\
3.34e-09	0	\\
3.45e-09	0	\\
3.55e-09	0	\\
3.65e-09	0	\\
3.75e-09	0	\\
3.86e-09	0	\\
3.96e-09	0	\\
4.06e-09	0	\\
4.16e-09	0	\\
4.27e-09	0	\\
4.37e-09	0	\\
4.47e-09	0	\\
4.57e-09	0	\\
4.68e-09	0	\\
4.78e-09	0	\\
4.89e-09	0	\\
4.99e-09	0	\\
5e-09	0	\\
};
\addplot [color=darkgray,solid,forget plot]
  table[row sep=crcr]{
0	0	\\
1.1e-10	0	\\
2.2e-10	0	\\
3.3e-10	0	\\
4.4e-10	0	\\
5.4e-10	0	\\
6.5e-10	0	\\
7.5e-10	0	\\
8.6e-10	0	\\
9.6e-10	0	\\
1.07e-09	0	\\
1.18e-09	0	\\
1.28e-09	0	\\
1.38e-09	0	\\
1.49e-09	0	\\
1.59e-09	0	\\
1.69e-09	0	\\
1.8e-09	0	\\
1.9e-09	0	\\
2.01e-09	0	\\
2.11e-09	0	\\
2.21e-09	0	\\
2.32e-09	0	\\
2.42e-09	0	\\
2.52e-09	0	\\
2.63e-09	0	\\
2.73e-09	0	\\
2.83e-09	0	\\
2.93e-09	0	\\
3.04e-09	0	\\
3.14e-09	0	\\
3.24e-09	0	\\
3.34e-09	0	\\
3.45e-09	0	\\
3.55e-09	0	\\
3.65e-09	0	\\
3.75e-09	0	\\
3.86e-09	0	\\
3.96e-09	0	\\
4.06e-09	0	\\
4.16e-09	0	\\
4.27e-09	0	\\
4.37e-09	0	\\
4.47e-09	0	\\
4.57e-09	0	\\
4.68e-09	0	\\
4.78e-09	0	\\
4.89e-09	0	\\
4.99e-09	0	\\
5e-09	0	\\
};
\addplot [color=blue,solid,forget plot]
  table[row sep=crcr]{
0	0	\\
1.1e-10	0	\\
2.2e-10	0	\\
3.3e-10	0	\\
4.4e-10	0	\\
5.4e-10	0	\\
6.5e-10	0	\\
7.5e-10	0	\\
8.6e-10	0	\\
9.6e-10	0	\\
1.07e-09	0	\\
1.18e-09	0	\\
1.28e-09	0	\\
1.38e-09	0	\\
1.49e-09	0	\\
1.59e-09	0	\\
1.69e-09	0	\\
1.8e-09	0	\\
1.9e-09	0	\\
2.01e-09	0	\\
2.11e-09	0	\\
2.21e-09	0	\\
2.32e-09	0	\\
2.42e-09	0	\\
2.52e-09	0	\\
2.63e-09	0	\\
2.73e-09	0	\\
2.83e-09	0	\\
2.93e-09	0	\\
3.04e-09	0	\\
3.14e-09	0	\\
3.24e-09	0	\\
3.34e-09	0	\\
3.45e-09	0	\\
3.55e-09	0	\\
3.65e-09	0	\\
3.75e-09	0	\\
3.86e-09	0	\\
3.96e-09	0	\\
4.06e-09	0	\\
4.16e-09	0	\\
4.27e-09	0	\\
4.37e-09	0	\\
4.47e-09	0	\\
4.57e-09	0	\\
4.68e-09	0	\\
4.78e-09	0	\\
4.89e-09	0	\\
4.99e-09	0	\\
5e-09	0	\\
};
\addplot [color=black!50!green,solid,forget plot]
  table[row sep=crcr]{
0	0	\\
1.1e-10	0	\\
2.2e-10	0	\\
3.3e-10	0	\\
4.4e-10	0	\\
5.4e-10	0	\\
6.5e-10	0	\\
7.5e-10	0	\\
8.6e-10	0	\\
9.6e-10	0	\\
1.07e-09	0	\\
1.18e-09	0	\\
1.28e-09	0	\\
1.38e-09	0	\\
1.49e-09	0	\\
1.59e-09	0	\\
1.69e-09	0	\\
1.8e-09	0	\\
1.9e-09	0	\\
2.01e-09	0	\\
2.11e-09	0	\\
2.21e-09	0	\\
2.32e-09	0	\\
2.42e-09	0	\\
2.52e-09	0	\\
2.63e-09	0	\\
2.73e-09	0	\\
2.83e-09	0	\\
2.93e-09	0	\\
3.04e-09	0	\\
3.14e-09	0	\\
3.24e-09	0	\\
3.34e-09	0	\\
3.45e-09	0	\\
3.55e-09	0	\\
3.65e-09	0	\\
3.75e-09	0	\\
3.86e-09	0	\\
3.96e-09	0	\\
4.06e-09	0	\\
4.16e-09	0	\\
4.27e-09	0	\\
4.37e-09	0	\\
4.47e-09	0	\\
4.57e-09	0	\\
4.68e-09	0	\\
4.78e-09	0	\\
4.89e-09	0	\\
4.99e-09	0	\\
5e-09	0	\\
};
\addplot [color=red,solid,forget plot]
  table[row sep=crcr]{
0	0	\\
1.1e-10	0	\\
2.2e-10	0	\\
3.3e-10	0	\\
4.4e-10	0	\\
5.4e-10	0	\\
6.5e-10	0	\\
7.5e-10	0	\\
8.6e-10	0	\\
9.6e-10	0	\\
1.07e-09	0	\\
1.18e-09	0	\\
1.28e-09	0	\\
1.38e-09	0	\\
1.49e-09	0	\\
1.59e-09	0	\\
1.69e-09	0	\\
1.8e-09	0	\\
1.9e-09	0	\\
2.01e-09	0	\\
2.11e-09	0	\\
2.21e-09	0	\\
2.32e-09	0	\\
2.42e-09	0	\\
2.52e-09	0	\\
2.63e-09	0	\\
2.73e-09	0	\\
2.83e-09	0	\\
2.93e-09	0	\\
3.04e-09	0	\\
3.14e-09	0	\\
3.24e-09	0	\\
3.34e-09	0	\\
3.45e-09	0	\\
3.55e-09	0	\\
3.65e-09	0	\\
3.75e-09	0	\\
3.86e-09	0	\\
3.96e-09	0	\\
4.06e-09	0	\\
4.16e-09	0	\\
4.27e-09	0	\\
4.37e-09	0	\\
4.47e-09	0	\\
4.57e-09	0	\\
4.68e-09	0	\\
4.78e-09	0	\\
4.89e-09	0	\\
4.99e-09	0	\\
5e-09	0	\\
};
\addplot [color=mycolor1,solid,forget plot]
  table[row sep=crcr]{
0	0	\\
1.1e-10	0	\\
2.2e-10	0	\\
3.3e-10	0	\\
4.4e-10	0	\\
5.4e-10	0	\\
6.5e-10	0	\\
7.5e-10	0	\\
8.6e-10	0	\\
9.6e-10	0	\\
1.07e-09	0	\\
1.18e-09	0	\\
1.28e-09	0	\\
1.38e-09	0	\\
1.49e-09	0	\\
1.59e-09	0	\\
1.69e-09	0	\\
1.8e-09	0	\\
1.9e-09	0	\\
2.01e-09	0	\\
2.11e-09	0	\\
2.21e-09	0	\\
2.32e-09	0	\\
2.42e-09	0	\\
2.52e-09	0	\\
2.63e-09	0	\\
2.73e-09	0	\\
2.83e-09	0	\\
2.93e-09	0	\\
3.04e-09	0	\\
3.14e-09	0	\\
3.24e-09	0	\\
3.34e-09	0	\\
3.45e-09	0	\\
3.55e-09	0	\\
3.65e-09	0	\\
3.75e-09	0	\\
3.86e-09	0	\\
3.96e-09	0	\\
4.06e-09	0	\\
4.16e-09	0	\\
4.27e-09	0	\\
4.37e-09	0	\\
4.47e-09	0	\\
4.57e-09	0	\\
4.68e-09	0	\\
4.78e-09	0	\\
4.89e-09	0	\\
4.99e-09	0	\\
5e-09	0	\\
};
\addplot [color=mycolor2,solid,forget plot]
  table[row sep=crcr]{
0	0	\\
1.1e-10	0	\\
2.2e-10	0	\\
3.3e-10	0	\\
4.4e-10	0	\\
5.4e-10	0	\\
6.5e-10	0	\\
7.5e-10	0	\\
8.6e-10	0	\\
9.6e-10	0	\\
1.07e-09	0	\\
1.18e-09	0	\\
1.28e-09	0	\\
1.38e-09	0	\\
1.49e-09	0	\\
1.59e-09	0	\\
1.69e-09	0	\\
1.8e-09	0	\\
1.9e-09	0	\\
2.01e-09	0	\\
2.11e-09	0	\\
2.21e-09	0	\\
2.32e-09	0	\\
2.42e-09	0	\\
2.52e-09	0	\\
2.63e-09	0	\\
2.73e-09	0	\\
2.83e-09	0	\\
2.93e-09	0	\\
3.04e-09	0	\\
3.14e-09	0	\\
3.24e-09	0	\\
3.34e-09	0	\\
3.45e-09	0	\\
3.55e-09	0	\\
3.65e-09	0	\\
3.75e-09	0	\\
3.86e-09	0	\\
3.96e-09	0	\\
4.06e-09	0	\\
4.16e-09	0	\\
4.27e-09	0	\\
4.37e-09	0	\\
4.47e-09	0	\\
4.57e-09	0	\\
4.68e-09	0	\\
4.78e-09	0	\\
4.89e-09	0	\\
4.99e-09	0	\\
5e-09	0	\\
};
\addplot [color=mycolor3,solid,forget plot]
  table[row sep=crcr]{
0	0	\\
1.1e-10	0	\\
2.2e-10	0	\\
3.3e-10	0	\\
4.4e-10	0	\\
5.4e-10	0	\\
6.5e-10	0	\\
7.5e-10	0	\\
8.6e-10	0	\\
9.6e-10	0	\\
1.07e-09	0	\\
1.18e-09	0	\\
1.28e-09	0	\\
1.38e-09	0	\\
1.49e-09	0	\\
1.59e-09	0	\\
1.69e-09	0	\\
1.8e-09	0	\\
1.9e-09	0	\\
2.01e-09	0	\\
2.11e-09	0	\\
2.21e-09	0	\\
2.32e-09	0	\\
2.42e-09	0	\\
2.52e-09	0	\\
2.63e-09	0	\\
2.73e-09	0	\\
2.83e-09	0	\\
2.93e-09	0	\\
3.04e-09	0	\\
3.14e-09	0	\\
3.24e-09	0	\\
3.34e-09	0	\\
3.45e-09	0	\\
3.55e-09	0	\\
3.65e-09	0	\\
3.75e-09	0	\\
3.86e-09	0	\\
3.96e-09	0	\\
4.06e-09	0	\\
4.16e-09	0	\\
4.27e-09	0	\\
4.37e-09	0	\\
4.47e-09	0	\\
4.57e-09	0	\\
4.68e-09	0	\\
4.78e-09	0	\\
4.89e-09	0	\\
4.99e-09	0	\\
5e-09	0	\\
};
\addplot [color=darkgray,solid,forget plot]
  table[row sep=crcr]{
0	0	\\
1.1e-10	0	\\
2.2e-10	0	\\
3.3e-10	0	\\
4.4e-10	0	\\
5.4e-10	0	\\
6.5e-10	0	\\
7.5e-10	0	\\
8.6e-10	0	\\
9.6e-10	0	\\
1.07e-09	0	\\
1.18e-09	0	\\
1.28e-09	0	\\
1.38e-09	0	\\
1.49e-09	0	\\
1.59e-09	0	\\
1.69e-09	0	\\
1.8e-09	0	\\
1.9e-09	0	\\
2.01e-09	0	\\
2.11e-09	0	\\
2.21e-09	0	\\
2.32e-09	0	\\
2.42e-09	0	\\
2.52e-09	0	\\
2.63e-09	0	\\
2.73e-09	0	\\
2.83e-09	0	\\
2.93e-09	0	\\
3.04e-09	0	\\
3.14e-09	0	\\
3.24e-09	0	\\
3.34e-09	0	\\
3.45e-09	0	\\
3.55e-09	0	\\
3.65e-09	0	\\
3.75e-09	0	\\
3.86e-09	0	\\
3.96e-09	0	\\
4.06e-09	0	\\
4.16e-09	0	\\
4.27e-09	0	\\
4.37e-09	0	\\
4.47e-09	0	\\
4.57e-09	0	\\
4.68e-09	0	\\
4.78e-09	0	\\
4.89e-09	0	\\
4.99e-09	0	\\
5e-09	0	\\
};
\addplot [color=blue,solid,forget plot]
  table[row sep=crcr]{
0	0	\\
1.1e-10	0	\\
2.2e-10	0	\\
3.3e-10	0	\\
4.4e-10	0	\\
5.4e-10	0	\\
6.5e-10	0	\\
7.5e-10	0	\\
8.6e-10	0	\\
9.6e-10	0	\\
1.07e-09	0	\\
1.18e-09	0	\\
1.28e-09	0	\\
1.38e-09	0	\\
1.49e-09	0	\\
1.59e-09	0	\\
1.69e-09	0	\\
1.8e-09	0	\\
1.9e-09	0	\\
2.01e-09	0	\\
2.11e-09	0	\\
2.21e-09	0	\\
2.32e-09	0	\\
2.42e-09	0	\\
2.52e-09	0	\\
2.63e-09	0	\\
2.73e-09	0	\\
2.83e-09	0	\\
2.93e-09	0	\\
3.04e-09	0	\\
3.14e-09	0	\\
3.24e-09	0	\\
3.34e-09	0	\\
3.45e-09	0	\\
3.55e-09	0	\\
3.65e-09	0	\\
3.75e-09	0	\\
3.86e-09	0	\\
3.96e-09	0	\\
4.06e-09	0	\\
4.16e-09	0	\\
4.27e-09	0	\\
4.37e-09	0	\\
4.47e-09	0	\\
4.57e-09	0	\\
4.68e-09	0	\\
4.78e-09	0	\\
4.89e-09	0	\\
4.99e-09	0	\\
5e-09	0	\\
};
\addplot [color=black!50!green,solid,forget plot]
  table[row sep=crcr]{
0	0	\\
1.1e-10	0	\\
2.2e-10	0	\\
3.3e-10	0	\\
4.4e-10	0	\\
5.4e-10	0	\\
6.5e-10	0	\\
7.5e-10	0	\\
8.6e-10	0	\\
9.6e-10	0	\\
1.07e-09	0	\\
1.18e-09	0	\\
1.28e-09	0	\\
1.38e-09	0	\\
1.49e-09	0	\\
1.59e-09	0	\\
1.69e-09	0	\\
1.8e-09	0	\\
1.9e-09	0	\\
2.01e-09	0	\\
2.11e-09	0	\\
2.21e-09	0	\\
2.32e-09	0	\\
2.42e-09	0	\\
2.52e-09	0	\\
2.63e-09	0	\\
2.73e-09	0	\\
2.83e-09	0	\\
2.93e-09	0	\\
3.04e-09	0	\\
3.14e-09	0	\\
3.24e-09	0	\\
3.34e-09	0	\\
3.45e-09	0	\\
3.55e-09	0	\\
3.65e-09	0	\\
3.75e-09	0	\\
3.86e-09	0	\\
3.96e-09	0	\\
4.06e-09	0	\\
4.16e-09	0	\\
4.27e-09	0	\\
4.37e-09	0	\\
4.47e-09	0	\\
4.57e-09	0	\\
4.68e-09	0	\\
4.78e-09	0	\\
4.89e-09	0	\\
4.99e-09	0	\\
5e-09	0	\\
};
\addplot [color=red,solid,forget plot]
  table[row sep=crcr]{
0	0	\\
1.1e-10	0	\\
2.2e-10	0	\\
3.3e-10	0	\\
4.4e-10	0	\\
5.4e-10	0	\\
6.5e-10	0	\\
7.5e-10	0	\\
8.6e-10	0	\\
9.6e-10	0	\\
1.07e-09	0	\\
1.18e-09	0	\\
1.28e-09	0	\\
1.38e-09	0	\\
1.49e-09	0	\\
1.59e-09	0	\\
1.69e-09	0	\\
1.8e-09	0	\\
1.9e-09	0	\\
2.01e-09	0	\\
2.11e-09	0	\\
2.21e-09	0	\\
2.32e-09	0	\\
2.42e-09	0	\\
2.52e-09	0	\\
2.63e-09	0	\\
2.73e-09	0	\\
2.83e-09	0	\\
2.93e-09	0	\\
3.04e-09	0	\\
3.14e-09	0	\\
3.24e-09	0	\\
3.34e-09	0	\\
3.45e-09	0	\\
3.55e-09	0	\\
3.65e-09	0	\\
3.75e-09	0	\\
3.86e-09	0	\\
3.96e-09	0	\\
4.06e-09	0	\\
4.16e-09	0	\\
4.27e-09	0	\\
4.37e-09	0	\\
4.47e-09	0	\\
4.57e-09	0	\\
4.68e-09	0	\\
4.78e-09	0	\\
4.89e-09	0	\\
4.99e-09	0	\\
5e-09	0	\\
};
\addplot [color=mycolor1,solid,forget plot]
  table[row sep=crcr]{
0	0	\\
1.1e-10	0	\\
2.2e-10	0	\\
3.3e-10	0	\\
4.4e-10	0	\\
5.4e-10	0	\\
6.5e-10	0	\\
7.5e-10	0	\\
8.6e-10	0	\\
9.6e-10	0	\\
1.07e-09	0	\\
1.18e-09	0	\\
1.28e-09	0	\\
1.38e-09	0	\\
1.49e-09	0	\\
1.59e-09	0	\\
1.69e-09	0	\\
1.8e-09	0	\\
1.9e-09	0	\\
2.01e-09	0	\\
2.11e-09	0	\\
2.21e-09	0	\\
2.32e-09	0	\\
2.42e-09	0	\\
2.52e-09	0	\\
2.63e-09	0	\\
2.73e-09	0	\\
2.83e-09	0	\\
2.93e-09	0	\\
3.04e-09	0	\\
3.14e-09	0	\\
3.24e-09	0	\\
3.34e-09	0	\\
3.45e-09	0	\\
3.55e-09	0	\\
3.65e-09	0	\\
3.75e-09	0	\\
3.86e-09	0	\\
3.96e-09	0	\\
4.06e-09	0	\\
4.16e-09	0	\\
4.27e-09	0	\\
4.37e-09	0	\\
4.47e-09	0	\\
4.57e-09	0	\\
4.68e-09	0	\\
4.78e-09	0	\\
4.89e-09	0	\\
4.99e-09	0	\\
5e-09	0	\\
};
\addplot [color=mycolor2,solid,forget plot]
  table[row sep=crcr]{
0	0	\\
1.1e-10	0	\\
2.2e-10	0	\\
3.3e-10	0	\\
4.4e-10	0	\\
5.4e-10	0	\\
6.5e-10	0	\\
7.5e-10	0	\\
8.6e-10	0	\\
9.6e-10	0	\\
1.07e-09	0	\\
1.18e-09	0	\\
1.28e-09	0	\\
1.38e-09	0	\\
1.49e-09	0	\\
1.59e-09	0	\\
1.69e-09	0	\\
1.8e-09	0	\\
1.9e-09	0	\\
2.01e-09	0	\\
2.11e-09	0	\\
2.21e-09	0	\\
2.32e-09	0	\\
2.42e-09	0	\\
2.52e-09	0	\\
2.63e-09	0	\\
2.73e-09	0	\\
2.83e-09	0	\\
2.93e-09	0	\\
3.04e-09	0	\\
3.14e-09	0	\\
3.24e-09	0	\\
3.34e-09	0	\\
3.45e-09	0	\\
3.55e-09	0	\\
3.65e-09	0	\\
3.75e-09	0	\\
3.86e-09	0	\\
3.96e-09	0	\\
4.06e-09	0	\\
4.16e-09	0	\\
4.27e-09	0	\\
4.37e-09	0	\\
4.47e-09	0	\\
4.57e-09	0	\\
4.68e-09	0	\\
4.78e-09	0	\\
4.89e-09	0	\\
4.99e-09	0	\\
5e-09	0	\\
};
\addplot [color=mycolor3,solid,forget plot]
  table[row sep=crcr]{
0	0	\\
1.1e-10	0	\\
2.2e-10	0	\\
3.3e-10	0	\\
4.4e-10	0	\\
5.4e-10	0	\\
6.5e-10	0	\\
7.5e-10	0	\\
8.6e-10	0	\\
9.6e-10	0	\\
1.07e-09	0	\\
1.18e-09	0	\\
1.28e-09	0	\\
1.38e-09	0	\\
1.49e-09	0	\\
1.59e-09	0	\\
1.69e-09	0	\\
1.8e-09	0	\\
1.9e-09	0	\\
2.01e-09	0	\\
2.11e-09	0	\\
2.21e-09	0	\\
2.32e-09	0	\\
2.42e-09	0	\\
2.52e-09	0	\\
2.63e-09	0	\\
2.73e-09	0	\\
2.83e-09	0	\\
2.93e-09	0	\\
3.04e-09	0	\\
3.14e-09	0	\\
3.24e-09	0	\\
3.34e-09	0	\\
3.45e-09	0	\\
3.55e-09	0	\\
3.65e-09	0	\\
3.75e-09	0	\\
3.86e-09	0	\\
3.96e-09	0	\\
4.06e-09	0	\\
4.16e-09	0	\\
4.27e-09	0	\\
4.37e-09	0	\\
4.47e-09	0	\\
4.57e-09	0	\\
4.68e-09	0	\\
4.78e-09	0	\\
4.89e-09	0	\\
4.99e-09	0	\\
5e-09	0	\\
};
\addplot [color=darkgray,solid,forget plot]
  table[row sep=crcr]{
0	0	\\
1.1e-10	0	\\
2.2e-10	0	\\
3.3e-10	0	\\
4.4e-10	0	\\
5.4e-10	0	\\
6.5e-10	0	\\
7.5e-10	0	\\
8.6e-10	0	\\
9.6e-10	0	\\
1.07e-09	0	\\
1.18e-09	0	\\
1.28e-09	0	\\
1.38e-09	0	\\
1.49e-09	0	\\
1.59e-09	0	\\
1.69e-09	0	\\
1.8e-09	0	\\
1.9e-09	0	\\
2.01e-09	0	\\
2.11e-09	0	\\
2.21e-09	0	\\
2.32e-09	0	\\
2.42e-09	0	\\
2.52e-09	0	\\
2.63e-09	0	\\
2.73e-09	0	\\
2.83e-09	0	\\
2.93e-09	0	\\
3.04e-09	0	\\
3.14e-09	0	\\
3.24e-09	0	\\
3.34e-09	0	\\
3.45e-09	0	\\
3.55e-09	0	\\
3.65e-09	0	\\
3.75e-09	0	\\
3.86e-09	0	\\
3.96e-09	0	\\
4.06e-09	0	\\
4.16e-09	0	\\
4.27e-09	0	\\
4.37e-09	0	\\
4.47e-09	0	\\
4.57e-09	0	\\
4.68e-09	0	\\
4.78e-09	0	\\
4.89e-09	0	\\
4.99e-09	0	\\
5e-09	0	\\
};
\addplot [color=blue,solid,forget plot]
  table[row sep=crcr]{
0	0	\\
1.1e-10	0	\\
2.2e-10	0	\\
3.3e-10	0	\\
4.4e-10	0	\\
5.4e-10	0	\\
6.5e-10	0	\\
7.5e-10	0	\\
8.6e-10	0	\\
9.6e-10	0	\\
1.07e-09	0	\\
1.18e-09	0	\\
1.28e-09	0	\\
1.38e-09	0	\\
1.49e-09	0	\\
1.59e-09	0	\\
1.69e-09	0	\\
1.8e-09	0	\\
1.9e-09	0	\\
2.01e-09	0	\\
2.11e-09	0	\\
2.21e-09	0	\\
2.32e-09	0	\\
2.42e-09	0	\\
2.52e-09	0	\\
2.63e-09	0	\\
2.73e-09	0	\\
2.83e-09	0	\\
2.93e-09	0	\\
3.04e-09	0	\\
3.14e-09	0	\\
3.24e-09	0	\\
3.34e-09	0	\\
3.45e-09	0	\\
3.55e-09	0	\\
3.65e-09	0	\\
3.75e-09	0	\\
3.86e-09	0	\\
3.96e-09	0	\\
4.06e-09	0	\\
4.16e-09	0	\\
4.27e-09	0	\\
4.37e-09	0	\\
4.47e-09	0	\\
4.57e-09	0	\\
4.68e-09	0	\\
4.78e-09	0	\\
4.89e-09	0	\\
4.99e-09	0	\\
5e-09	0	\\
};
\addplot [color=black!50!green,solid,forget plot]
  table[row sep=crcr]{
0	0	\\
1.1e-10	0	\\
2.2e-10	0	\\
3.3e-10	0	\\
4.4e-10	0	\\
5.4e-10	0	\\
6.5e-10	0	\\
7.5e-10	0	\\
8.6e-10	0	\\
9.6e-10	0	\\
1.07e-09	0	\\
1.18e-09	0	\\
1.28e-09	0	\\
1.38e-09	0	\\
1.49e-09	0	\\
1.59e-09	0	\\
1.69e-09	0	\\
1.8e-09	0	\\
1.9e-09	0	\\
2.01e-09	0	\\
2.11e-09	0	\\
2.21e-09	0	\\
2.32e-09	0	\\
2.42e-09	0	\\
2.52e-09	0	\\
2.63e-09	0	\\
2.73e-09	0	\\
2.83e-09	0	\\
2.93e-09	0	\\
3.04e-09	0	\\
3.14e-09	0	\\
3.24e-09	0	\\
3.34e-09	0	\\
3.45e-09	0	\\
3.55e-09	0	\\
3.65e-09	0	\\
3.75e-09	0	\\
3.86e-09	0	\\
3.96e-09	0	\\
4.06e-09	0	\\
4.16e-09	0	\\
4.27e-09	0	\\
4.37e-09	0	\\
4.47e-09	0	\\
4.57e-09	0	\\
4.68e-09	0	\\
4.78e-09	0	\\
4.89e-09	0	\\
4.99e-09	0	\\
5e-09	0	\\
};
\addplot [color=red,solid,forget plot]
  table[row sep=crcr]{
0	0	\\
1.1e-10	0	\\
2.2e-10	0	\\
3.3e-10	0	\\
4.4e-10	0	\\
5.4e-10	0	\\
6.5e-10	0	\\
7.5e-10	0	\\
8.6e-10	0	\\
9.6e-10	0	\\
1.07e-09	0	\\
1.18e-09	0	\\
1.28e-09	0	\\
1.38e-09	0	\\
1.49e-09	0	\\
1.59e-09	0	\\
1.69e-09	0	\\
1.8e-09	0	\\
1.9e-09	0	\\
2.01e-09	0	\\
2.11e-09	0	\\
2.21e-09	0	\\
2.32e-09	0	\\
2.42e-09	0	\\
2.52e-09	0	\\
2.63e-09	0	\\
2.73e-09	0	\\
2.83e-09	0	\\
2.93e-09	0	\\
3.04e-09	0	\\
3.14e-09	0	\\
3.24e-09	0	\\
3.34e-09	0	\\
3.45e-09	0	\\
3.55e-09	0	\\
3.65e-09	0	\\
3.75e-09	0	\\
3.86e-09	0	\\
3.96e-09	0	\\
4.06e-09	0	\\
4.16e-09	0	\\
4.27e-09	0	\\
4.37e-09	0	\\
4.47e-09	0	\\
4.57e-09	0	\\
4.68e-09	0	\\
4.78e-09	0	\\
4.89e-09	0	\\
4.99e-09	0	\\
5e-09	0	\\
};
\addplot [color=mycolor1,solid,forget plot]
  table[row sep=crcr]{
0	0	\\
1.1e-10	0	\\
2.2e-10	0	\\
3.3e-10	0	\\
4.4e-10	0	\\
5.4e-10	0	\\
6.5e-10	0	\\
7.5e-10	0	\\
8.6e-10	0	\\
9.6e-10	0	\\
1.07e-09	0	\\
1.18e-09	0	\\
1.28e-09	0	\\
1.38e-09	0	\\
1.49e-09	0	\\
1.59e-09	0	\\
1.69e-09	0	\\
1.8e-09	0	\\
1.9e-09	0	\\
2.01e-09	0	\\
2.11e-09	0	\\
2.21e-09	0	\\
2.32e-09	0	\\
2.42e-09	0	\\
2.52e-09	0	\\
2.63e-09	0	\\
2.73e-09	0	\\
2.83e-09	0	\\
2.93e-09	0	\\
3.04e-09	0	\\
3.14e-09	0	\\
3.24e-09	0	\\
3.34e-09	0	\\
3.45e-09	0	\\
3.55e-09	0	\\
3.65e-09	0	\\
3.75e-09	0	\\
3.86e-09	0	\\
3.96e-09	0	\\
4.06e-09	0	\\
4.16e-09	0	\\
4.27e-09	0	\\
4.37e-09	0	\\
4.47e-09	0	\\
4.57e-09	0	\\
4.68e-09	0	\\
4.78e-09	0	\\
4.89e-09	0	\\
4.99e-09	0	\\
5e-09	0	\\
};
\addplot [color=mycolor2,solid,forget plot]
  table[row sep=crcr]{
0	0	\\
1.1e-10	0	\\
2.2e-10	0	\\
3.3e-10	0	\\
4.4e-10	0	\\
5.4e-10	0	\\
6.5e-10	0	\\
7.5e-10	0	\\
8.6e-10	0	\\
9.6e-10	0	\\
1.07e-09	0	\\
1.18e-09	0	\\
1.28e-09	0	\\
1.38e-09	0	\\
1.49e-09	0	\\
1.59e-09	0	\\
1.69e-09	0	\\
1.8e-09	0	\\
1.9e-09	0	\\
2.01e-09	0	\\
2.11e-09	0	\\
2.21e-09	0	\\
2.32e-09	0	\\
2.42e-09	0	\\
2.52e-09	0	\\
2.63e-09	0	\\
2.73e-09	0	\\
2.83e-09	0	\\
2.93e-09	0	\\
3.04e-09	0	\\
3.14e-09	0	\\
3.24e-09	0	\\
3.34e-09	0	\\
3.45e-09	0	\\
3.55e-09	0	\\
3.65e-09	0	\\
3.75e-09	0	\\
3.86e-09	0	\\
3.96e-09	0	\\
4.06e-09	0	\\
4.16e-09	0	\\
4.27e-09	0	\\
4.37e-09	0	\\
4.47e-09	0	\\
4.57e-09	0	\\
4.68e-09	0	\\
4.78e-09	0	\\
4.89e-09	0	\\
4.99e-09	0	\\
5e-09	0	\\
};
\addplot [color=mycolor3,solid,forget plot]
  table[row sep=crcr]{
0	0	\\
1.1e-10	0	\\
2.2e-10	0	\\
3.3e-10	0	\\
4.4e-10	0	\\
5.4e-10	0	\\
6.5e-10	0	\\
7.5e-10	0	\\
8.6e-10	0	\\
9.6e-10	0	\\
1.07e-09	0	\\
1.18e-09	0	\\
1.28e-09	0	\\
1.38e-09	0	\\
1.49e-09	0	\\
1.59e-09	0	\\
1.69e-09	0	\\
1.8e-09	0	\\
1.9e-09	0	\\
2.01e-09	0	\\
2.11e-09	0	\\
2.21e-09	0	\\
2.32e-09	0	\\
2.42e-09	0	\\
2.52e-09	0	\\
2.63e-09	0	\\
2.73e-09	0	\\
2.83e-09	0	\\
2.93e-09	0	\\
3.04e-09	0	\\
3.14e-09	0	\\
3.24e-09	0	\\
3.34e-09	0	\\
3.45e-09	0	\\
3.55e-09	0	\\
3.65e-09	0	\\
3.75e-09	0	\\
3.86e-09	0	\\
3.96e-09	0	\\
4.06e-09	0	\\
4.16e-09	0	\\
4.27e-09	0	\\
4.37e-09	0	\\
4.47e-09	0	\\
4.57e-09	0	\\
4.68e-09	0	\\
4.78e-09	0	\\
4.89e-09	0	\\
4.99e-09	0	\\
5e-09	0	\\
};
\addplot [color=darkgray,solid,forget plot]
  table[row sep=crcr]{
0	0	\\
1.1e-10	0	\\
2.2e-10	0	\\
3.3e-10	0	\\
4.4e-10	0	\\
5.4e-10	0	\\
6.5e-10	0	\\
7.5e-10	0	\\
8.6e-10	0	\\
9.6e-10	0	\\
1.07e-09	0	\\
1.18e-09	0	\\
1.28e-09	0	\\
1.38e-09	0	\\
1.49e-09	0	\\
1.59e-09	0	\\
1.69e-09	0	\\
1.8e-09	0	\\
1.9e-09	0	\\
2.01e-09	0	\\
2.11e-09	0	\\
2.21e-09	0	\\
2.32e-09	0	\\
2.42e-09	0	\\
2.52e-09	0	\\
2.63e-09	0	\\
2.73e-09	0	\\
2.83e-09	0	\\
2.93e-09	0	\\
3.04e-09	0	\\
3.14e-09	0	\\
3.24e-09	0	\\
3.34e-09	0	\\
3.45e-09	0	\\
3.55e-09	0	\\
3.65e-09	0	\\
3.75e-09	0	\\
3.86e-09	0	\\
3.96e-09	0	\\
4.06e-09	0	\\
4.16e-09	0	\\
4.27e-09	0	\\
4.37e-09	0	\\
4.47e-09	0	\\
4.57e-09	0	\\
4.68e-09	0	\\
4.78e-09	0	\\
4.89e-09	0	\\
4.99e-09	0	\\
5e-09	0	\\
};
\addplot [color=blue,solid,forget plot]
  table[row sep=crcr]{
0	0	\\
1.1e-10	0	\\
2.2e-10	0	\\
3.3e-10	0	\\
4.4e-10	0	\\
5.4e-10	0	\\
6.5e-10	0	\\
7.5e-10	0	\\
8.6e-10	0	\\
9.6e-10	0	\\
1.07e-09	0	\\
1.18e-09	0	\\
1.28e-09	0	\\
1.38e-09	0	\\
1.49e-09	0	\\
1.59e-09	0	\\
1.69e-09	0	\\
1.8e-09	0	\\
1.9e-09	0	\\
2.01e-09	0	\\
2.11e-09	0	\\
2.21e-09	0	\\
2.32e-09	0	\\
2.42e-09	0	\\
2.52e-09	0	\\
2.63e-09	0	\\
2.73e-09	0	\\
2.83e-09	0	\\
2.93e-09	0	\\
3.04e-09	0	\\
3.14e-09	0	\\
3.24e-09	0	\\
3.34e-09	0	\\
3.45e-09	0	\\
3.55e-09	0	\\
3.65e-09	0	\\
3.75e-09	0	\\
3.86e-09	0	\\
3.96e-09	0	\\
4.06e-09	0	\\
4.16e-09	0	\\
4.27e-09	0	\\
4.37e-09	0	\\
4.47e-09	0	\\
4.57e-09	0	\\
4.68e-09	0	\\
4.78e-09	0	\\
4.89e-09	0	\\
4.99e-09	0	\\
5e-09	0	\\
};
\addplot [color=black!50!green,solid,forget plot]
  table[row sep=crcr]{
0	0	\\
1.1e-10	0	\\
2.2e-10	0	\\
3.3e-10	0	\\
4.4e-10	0	\\
5.4e-10	0	\\
6.5e-10	0	\\
7.5e-10	0	\\
8.6e-10	0	\\
9.6e-10	0	\\
1.07e-09	0	\\
1.18e-09	0	\\
1.28e-09	0	\\
1.38e-09	0	\\
1.49e-09	0	\\
1.59e-09	0	\\
1.69e-09	0	\\
1.8e-09	0	\\
1.9e-09	0	\\
2.01e-09	0	\\
2.11e-09	0	\\
2.21e-09	0	\\
2.32e-09	0	\\
2.42e-09	0	\\
2.52e-09	0	\\
2.63e-09	0	\\
2.73e-09	0	\\
2.83e-09	0	\\
2.93e-09	0	\\
3.04e-09	0	\\
3.14e-09	0	\\
3.24e-09	0	\\
3.34e-09	0	\\
3.45e-09	0	\\
3.55e-09	0	\\
3.65e-09	0	\\
3.75e-09	0	\\
3.86e-09	0	\\
3.96e-09	0	\\
4.06e-09	0	\\
4.16e-09	0	\\
4.27e-09	0	\\
4.37e-09	0	\\
4.47e-09	0	\\
4.57e-09	0	\\
4.68e-09	0	\\
4.78e-09	0	\\
4.89e-09	0	\\
4.99e-09	0	\\
5e-09	0	\\
};
\addplot [color=red,solid,forget plot]
  table[row sep=crcr]{
0	0	\\
1.1e-10	0	\\
2.2e-10	0	\\
3.3e-10	0	\\
4.4e-10	0	\\
5.4e-10	0	\\
6.5e-10	0	\\
7.5e-10	0	\\
8.6e-10	0	\\
9.6e-10	0	\\
1.07e-09	0	\\
1.18e-09	0	\\
1.28e-09	0	\\
1.38e-09	0	\\
1.49e-09	0	\\
1.59e-09	0	\\
1.69e-09	0	\\
1.8e-09	0	\\
1.9e-09	0	\\
2.01e-09	0	\\
2.11e-09	0	\\
2.21e-09	0	\\
2.32e-09	0	\\
2.42e-09	0	\\
2.52e-09	0	\\
2.63e-09	0	\\
2.73e-09	0	\\
2.83e-09	0	\\
2.93e-09	0	\\
3.04e-09	0	\\
3.14e-09	0	\\
3.24e-09	0	\\
3.34e-09	0	\\
3.45e-09	0	\\
3.55e-09	0	\\
3.65e-09	0	\\
3.75e-09	0	\\
3.86e-09	0	\\
3.96e-09	0	\\
4.06e-09	0	\\
4.16e-09	0	\\
4.27e-09	0	\\
4.37e-09	0	\\
4.47e-09	0	\\
4.57e-09	0	\\
4.68e-09	0	\\
4.78e-09	0	\\
4.89e-09	0	\\
4.99e-09	0	\\
5e-09	0	\\
};
\addplot [color=mycolor1,solid,forget plot]
  table[row sep=crcr]{
0	0	\\
1.1e-10	0	\\
2.2e-10	0	\\
3.3e-10	0	\\
4.4e-10	0	\\
5.4e-10	0	\\
6.5e-10	0	\\
7.5e-10	0	\\
8.6e-10	0	\\
9.6e-10	0	\\
1.07e-09	0	\\
1.18e-09	0	\\
1.28e-09	0	\\
1.38e-09	0	\\
1.49e-09	0	\\
1.59e-09	0	\\
1.69e-09	0	\\
1.8e-09	0	\\
1.9e-09	0	\\
2.01e-09	0	\\
2.11e-09	0	\\
2.21e-09	0	\\
2.32e-09	0	\\
2.42e-09	0	\\
2.52e-09	0	\\
2.63e-09	0	\\
2.73e-09	0	\\
2.83e-09	0	\\
2.93e-09	0	\\
3.04e-09	0	\\
3.14e-09	0	\\
3.24e-09	0	\\
3.34e-09	0	\\
3.45e-09	0	\\
3.55e-09	0	\\
3.65e-09	0	\\
3.75e-09	0	\\
3.86e-09	0	\\
3.96e-09	0	\\
4.06e-09	0	\\
4.16e-09	0	\\
4.27e-09	0	\\
4.37e-09	0	\\
4.47e-09	0	\\
4.57e-09	0	\\
4.68e-09	0	\\
4.78e-09	0	\\
4.89e-09	0	\\
4.99e-09	0	\\
5e-09	0	\\
};
\addplot [color=mycolor2,solid,forget plot]
  table[row sep=crcr]{
0	0	\\
1.1e-10	0	\\
2.2e-10	0	\\
3.3e-10	0	\\
4.4e-10	0	\\
5.4e-10	0	\\
6.5e-10	0	\\
7.5e-10	0	\\
8.6e-10	0	\\
9.6e-10	0	\\
1.07e-09	0	\\
1.18e-09	0	\\
1.28e-09	0	\\
1.38e-09	0	\\
1.49e-09	0	\\
1.59e-09	0	\\
1.69e-09	0	\\
1.8e-09	0	\\
1.9e-09	0	\\
2.01e-09	0	\\
2.11e-09	0	\\
2.21e-09	0	\\
2.32e-09	0	\\
2.42e-09	0	\\
2.52e-09	0	\\
2.63e-09	0	\\
2.73e-09	0	\\
2.83e-09	0	\\
2.93e-09	0	\\
3.04e-09	0	\\
3.14e-09	0	\\
3.24e-09	0	\\
3.34e-09	0	\\
3.45e-09	0	\\
3.55e-09	0	\\
3.65e-09	0	\\
3.75e-09	0	\\
3.86e-09	0	\\
3.96e-09	0	\\
4.06e-09	0	\\
4.16e-09	0	\\
4.27e-09	0	\\
4.37e-09	0	\\
4.47e-09	0	\\
4.57e-09	0	\\
4.68e-09	0	\\
4.78e-09	0	\\
4.89e-09	0	\\
4.99e-09	0	\\
5e-09	0	\\
};
\addplot [color=mycolor3,solid,forget plot]
  table[row sep=crcr]{
0	0	\\
1.1e-10	0	\\
2.2e-10	0	\\
3.3e-10	0	\\
4.4e-10	0	\\
5.4e-10	0	\\
6.5e-10	0	\\
7.5e-10	0	\\
8.6e-10	0	\\
9.6e-10	0	\\
1.07e-09	0	\\
1.18e-09	0	\\
1.28e-09	0	\\
1.38e-09	0	\\
1.49e-09	0	\\
1.59e-09	0	\\
1.69e-09	0	\\
1.8e-09	0	\\
1.9e-09	0	\\
2.01e-09	0	\\
2.11e-09	0	\\
2.21e-09	0	\\
2.32e-09	0	\\
2.42e-09	0	\\
2.52e-09	0	\\
2.63e-09	0	\\
2.73e-09	0	\\
2.83e-09	0	\\
2.93e-09	0	\\
3.04e-09	0	\\
3.14e-09	0	\\
3.24e-09	0	\\
3.34e-09	0	\\
3.45e-09	0	\\
3.55e-09	0	\\
3.65e-09	0	\\
3.75e-09	0	\\
3.86e-09	0	\\
3.96e-09	0	\\
4.06e-09	0	\\
4.16e-09	0	\\
4.27e-09	0	\\
4.37e-09	0	\\
4.47e-09	0	\\
4.57e-09	0	\\
4.68e-09	0	\\
4.78e-09	0	\\
4.89e-09	0	\\
4.99e-09	0	\\
5e-09	0	\\
};
\addplot [color=darkgray,solid,forget plot]
  table[row sep=crcr]{
0	0	\\
1.1e-10	0	\\
2.2e-10	0	\\
3.3e-10	0	\\
4.4e-10	0	\\
5.4e-10	0	\\
6.5e-10	0	\\
7.5e-10	0	\\
8.6e-10	0	\\
9.6e-10	0	\\
1.07e-09	0	\\
1.18e-09	0	\\
1.28e-09	0	\\
1.38e-09	0	\\
1.49e-09	0	\\
1.59e-09	0	\\
1.69e-09	0	\\
1.8e-09	0	\\
1.9e-09	0	\\
2.01e-09	0	\\
2.11e-09	0	\\
2.21e-09	0	\\
2.32e-09	0	\\
2.42e-09	0	\\
2.52e-09	0	\\
2.63e-09	0	\\
2.73e-09	0	\\
2.83e-09	0	\\
2.93e-09	0	\\
3.04e-09	0	\\
3.14e-09	0	\\
3.24e-09	0	\\
3.34e-09	0	\\
3.45e-09	0	\\
3.55e-09	0	\\
3.65e-09	0	\\
3.75e-09	0	\\
3.86e-09	0	\\
3.96e-09	0	\\
4.06e-09	0	\\
4.16e-09	0	\\
4.27e-09	0	\\
4.37e-09	0	\\
4.47e-09	0	\\
4.57e-09	0	\\
4.68e-09	0	\\
4.78e-09	0	\\
4.89e-09	0	\\
4.99e-09	0	\\
5e-09	0	\\
};
\addplot [color=blue,solid,forget plot]
  table[row sep=crcr]{
0	0	\\
1.1e-10	0	\\
2.2e-10	0	\\
3.3e-10	0	\\
4.4e-10	0	\\
5.4e-10	0	\\
6.5e-10	0	\\
7.5e-10	0	\\
8.6e-10	0	\\
9.6e-10	0	\\
1.07e-09	0	\\
1.18e-09	0	\\
1.28e-09	0	\\
1.38e-09	0	\\
1.49e-09	0	\\
1.59e-09	0	\\
1.69e-09	0	\\
1.8e-09	0	\\
1.9e-09	0	\\
2.01e-09	0	\\
2.11e-09	0	\\
2.21e-09	0	\\
2.32e-09	0	\\
2.42e-09	0	\\
2.52e-09	0	\\
2.63e-09	0	\\
2.73e-09	0	\\
2.83e-09	0	\\
2.93e-09	0	\\
3.04e-09	0	\\
3.14e-09	0	\\
3.24e-09	0	\\
3.34e-09	0	\\
3.45e-09	0	\\
3.55e-09	0	\\
3.65e-09	0	\\
3.75e-09	0	\\
3.86e-09	0	\\
3.96e-09	0	\\
4.06e-09	0	\\
4.16e-09	0	\\
4.27e-09	0	\\
4.37e-09	0	\\
4.47e-09	0	\\
4.57e-09	0	\\
4.68e-09	0	\\
4.78e-09	0	\\
4.89e-09	0	\\
4.99e-09	0	\\
5e-09	0	\\
};
\addplot [color=black!50!green,solid,forget plot]
  table[row sep=crcr]{
0	0	\\
1.1e-10	0	\\
2.2e-10	0	\\
3.3e-10	0	\\
4.4e-10	0	\\
5.4e-10	0	\\
6.5e-10	0	\\
7.5e-10	0	\\
8.6e-10	0	\\
9.6e-10	0	\\
1.07e-09	0	\\
1.18e-09	0	\\
1.28e-09	0	\\
1.38e-09	0	\\
1.49e-09	0	\\
1.59e-09	0	\\
1.69e-09	0	\\
1.8e-09	0	\\
1.9e-09	0	\\
2.01e-09	0	\\
2.11e-09	0	\\
2.21e-09	0	\\
2.32e-09	0	\\
2.42e-09	0	\\
2.52e-09	0	\\
2.63e-09	0	\\
2.73e-09	0	\\
2.83e-09	0	\\
2.93e-09	0	\\
3.04e-09	0	\\
3.14e-09	0	\\
3.24e-09	0	\\
3.34e-09	0	\\
3.45e-09	0	\\
3.55e-09	0	\\
3.65e-09	0	\\
3.75e-09	0	\\
3.86e-09	0	\\
3.96e-09	0	\\
4.06e-09	0	\\
4.16e-09	0	\\
4.27e-09	0	\\
4.37e-09	0	\\
4.47e-09	0	\\
4.57e-09	0	\\
4.68e-09	0	\\
4.78e-09	0	\\
4.89e-09	0	\\
4.99e-09	0	\\
5e-09	0	\\
};
\addplot [color=red,solid,forget plot]
  table[row sep=crcr]{
0	0	\\
1.1e-10	0	\\
2.2e-10	0	\\
3.3e-10	0	\\
4.4e-10	0	\\
5.4e-10	0	\\
6.5e-10	0	\\
7.5e-10	0	\\
8.6e-10	0	\\
9.6e-10	0	\\
1.07e-09	0	\\
1.18e-09	0	\\
1.28e-09	0	\\
1.38e-09	0	\\
1.49e-09	0	\\
1.59e-09	0	\\
1.69e-09	0	\\
1.8e-09	0	\\
1.9e-09	0	\\
2.01e-09	0	\\
2.11e-09	0	\\
2.21e-09	0	\\
2.32e-09	0	\\
2.42e-09	0	\\
2.52e-09	0	\\
2.63e-09	0	\\
2.73e-09	0	\\
2.83e-09	0	\\
2.93e-09	0	\\
3.04e-09	0	\\
3.14e-09	0	\\
3.24e-09	0	\\
3.34e-09	0	\\
3.45e-09	0	\\
3.55e-09	0	\\
3.65e-09	0	\\
3.75e-09	0	\\
3.86e-09	0	\\
3.96e-09	0	\\
4.06e-09	0	\\
4.16e-09	0	\\
4.27e-09	0	\\
4.37e-09	0	\\
4.47e-09	0	\\
4.57e-09	0	\\
4.68e-09	0	\\
4.78e-09	0	\\
4.89e-09	0	\\
4.99e-09	0	\\
5e-09	0	\\
};
\addplot [color=mycolor1,solid,forget plot]
  table[row sep=crcr]{
0	0	\\
1.1e-10	0	\\
2.2e-10	0	\\
3.3e-10	0	\\
4.4e-10	0	\\
5.4e-10	0	\\
6.5e-10	0	\\
7.5e-10	0	\\
8.6e-10	0	\\
9.6e-10	0	\\
1.07e-09	0	\\
1.18e-09	0	\\
1.28e-09	0	\\
1.38e-09	0	\\
1.49e-09	0	\\
1.59e-09	0	\\
1.69e-09	0	\\
1.8e-09	0	\\
1.9e-09	0	\\
2.01e-09	0	\\
2.11e-09	0	\\
2.21e-09	0	\\
2.32e-09	0	\\
2.42e-09	0	\\
2.52e-09	0	\\
2.63e-09	0	\\
2.73e-09	0	\\
2.83e-09	0	\\
2.93e-09	0	\\
3.04e-09	0	\\
3.14e-09	0	\\
3.24e-09	0	\\
3.34e-09	0	\\
3.45e-09	0	\\
3.55e-09	0	\\
3.65e-09	0	\\
3.75e-09	0	\\
3.86e-09	0	\\
3.96e-09	0	\\
4.06e-09	0	\\
4.16e-09	0	\\
4.27e-09	0	\\
4.37e-09	0	\\
4.47e-09	0	\\
4.57e-09	0	\\
4.68e-09	0	\\
4.78e-09	0	\\
4.89e-09	0	\\
4.99e-09	0	\\
5e-09	0	\\
};
\addplot [color=mycolor2,solid,forget plot]
  table[row sep=crcr]{
0	0	\\
1.1e-10	0	\\
2.2e-10	0	\\
3.3e-10	0	\\
4.4e-10	0	\\
5.4e-10	0	\\
6.5e-10	0	\\
7.5e-10	0	\\
8.6e-10	0	\\
9.6e-10	0	\\
1.07e-09	0	\\
1.18e-09	0	\\
1.28e-09	0	\\
1.38e-09	0	\\
1.49e-09	0	\\
1.59e-09	0	\\
1.69e-09	0	\\
1.8e-09	0	\\
1.9e-09	0	\\
2.01e-09	0	\\
2.11e-09	0	\\
2.21e-09	0	\\
2.32e-09	0	\\
2.42e-09	0	\\
2.52e-09	0	\\
2.63e-09	0	\\
2.73e-09	0	\\
2.83e-09	0	\\
2.93e-09	0	\\
3.04e-09	0	\\
3.14e-09	0	\\
3.24e-09	0	\\
3.34e-09	0	\\
3.45e-09	0	\\
3.55e-09	0	\\
3.65e-09	0	\\
3.75e-09	0	\\
3.86e-09	0	\\
3.96e-09	0	\\
4.06e-09	0	\\
4.16e-09	0	\\
4.27e-09	0	\\
4.37e-09	0	\\
4.47e-09	0	\\
4.57e-09	0	\\
4.68e-09	0	\\
4.78e-09	0	\\
4.89e-09	0	\\
4.99e-09	0	\\
5e-09	0	\\
};
\addplot [color=mycolor3,solid,forget plot]
  table[row sep=crcr]{
0	0	\\
1.1e-10	0	\\
2.2e-10	0	\\
3.3e-10	0	\\
4.4e-10	0	\\
5.4e-10	0	\\
6.5e-10	0	\\
7.5e-10	0	\\
8.6e-10	0	\\
9.6e-10	0	\\
1.07e-09	0	\\
1.18e-09	0	\\
1.28e-09	0	\\
1.38e-09	0	\\
1.49e-09	0	\\
1.59e-09	0	\\
1.69e-09	0	\\
1.8e-09	0	\\
1.9e-09	0	\\
2.01e-09	0	\\
2.11e-09	0	\\
2.21e-09	0	\\
2.32e-09	0	\\
2.42e-09	0	\\
2.52e-09	0	\\
2.63e-09	0	\\
2.73e-09	0	\\
2.83e-09	0	\\
2.93e-09	0	\\
3.04e-09	0	\\
3.14e-09	0	\\
3.24e-09	0	\\
3.34e-09	0	\\
3.45e-09	0	\\
3.55e-09	0	\\
3.65e-09	0	\\
3.75e-09	0	\\
3.86e-09	0	\\
3.96e-09	0	\\
4.06e-09	0	\\
4.16e-09	0	\\
4.27e-09	0	\\
4.37e-09	0	\\
4.47e-09	0	\\
4.57e-09	0	\\
4.68e-09	0	\\
4.78e-09	0	\\
4.89e-09	0	\\
4.99e-09	0	\\
5e-09	0	\\
};
\addplot [color=darkgray,solid,forget plot]
  table[row sep=crcr]{
0	0	\\
1.1e-10	0	\\
2.2e-10	0	\\
3.3e-10	0	\\
4.4e-10	0	\\
5.4e-10	0	\\
6.5e-10	0	\\
7.5e-10	0	\\
8.6e-10	0	\\
9.6e-10	0	\\
1.07e-09	0	\\
1.18e-09	0	\\
1.28e-09	0	\\
1.38e-09	0	\\
1.49e-09	0	\\
1.59e-09	0	\\
1.69e-09	0	\\
1.8e-09	0	\\
1.9e-09	0	\\
2.01e-09	0	\\
2.11e-09	0	\\
2.21e-09	0	\\
2.32e-09	0	\\
2.42e-09	0	\\
2.52e-09	0	\\
2.63e-09	0	\\
2.73e-09	0	\\
2.83e-09	0	\\
2.93e-09	0	\\
3.04e-09	0	\\
3.14e-09	0	\\
3.24e-09	0	\\
3.34e-09	0	\\
3.45e-09	0	\\
3.55e-09	0	\\
3.65e-09	0	\\
3.75e-09	0	\\
3.86e-09	0	\\
3.96e-09	0	\\
4.06e-09	0	\\
4.16e-09	0	\\
4.27e-09	0	\\
4.37e-09	0	\\
4.47e-09	0	\\
4.57e-09	0	\\
4.68e-09	0	\\
4.78e-09	0	\\
4.89e-09	0	\\
4.99e-09	0	\\
5e-09	0	\\
};
\addplot [color=blue,solid,forget plot]
  table[row sep=crcr]{
0	0	\\
1.1e-10	0	\\
2.2e-10	0	\\
3.3e-10	0	\\
4.4e-10	0	\\
5.4e-10	0	\\
6.5e-10	0	\\
7.5e-10	0	\\
8.6e-10	0	\\
9.6e-10	0	\\
1.07e-09	0	\\
1.18e-09	0	\\
1.28e-09	0	\\
1.38e-09	0	\\
1.49e-09	0	\\
1.59e-09	0	\\
1.69e-09	0	\\
1.8e-09	0	\\
1.9e-09	0	\\
2.01e-09	0	\\
2.11e-09	0	\\
2.21e-09	0	\\
2.32e-09	0	\\
2.42e-09	0	\\
2.52e-09	0	\\
2.63e-09	0	\\
2.73e-09	0	\\
2.83e-09	0	\\
2.93e-09	0	\\
3.04e-09	0	\\
3.14e-09	0	\\
3.24e-09	0	\\
3.34e-09	0	\\
3.45e-09	0	\\
3.55e-09	0	\\
3.65e-09	0	\\
3.75e-09	0	\\
3.86e-09	0	\\
3.96e-09	0	\\
4.06e-09	0	\\
4.16e-09	0	\\
4.27e-09	0	\\
4.37e-09	0	\\
4.47e-09	0	\\
4.57e-09	0	\\
4.68e-09	0	\\
4.78e-09	0	\\
4.89e-09	0	\\
4.99e-09	0	\\
5e-09	0	\\
};
\addplot [color=black!50!green,solid,forget plot]
  table[row sep=crcr]{
0	0	\\
1.1e-10	0	\\
2.2e-10	0	\\
3.3e-10	0	\\
4.4e-10	0	\\
5.4e-10	0	\\
6.5e-10	0	\\
7.5e-10	0	\\
8.6e-10	0	\\
9.6e-10	0	\\
1.07e-09	0	\\
1.18e-09	0	\\
1.28e-09	0	\\
1.38e-09	0	\\
1.49e-09	0	\\
1.59e-09	0	\\
1.69e-09	0	\\
1.8e-09	0	\\
1.9e-09	0	\\
2.01e-09	0	\\
2.11e-09	0	\\
2.21e-09	0	\\
2.32e-09	0	\\
2.42e-09	0	\\
2.52e-09	0	\\
2.63e-09	0	\\
2.73e-09	0	\\
2.83e-09	0	\\
2.93e-09	0	\\
3.04e-09	0	\\
3.14e-09	0	\\
3.24e-09	0	\\
3.34e-09	0	\\
3.45e-09	0	\\
3.55e-09	0	\\
3.65e-09	0	\\
3.75e-09	0	\\
3.86e-09	0	\\
3.96e-09	0	\\
4.06e-09	0	\\
4.16e-09	0	\\
4.27e-09	0	\\
4.37e-09	0	\\
4.47e-09	0	\\
4.57e-09	0	\\
4.68e-09	0	\\
4.78e-09	0	\\
4.89e-09	0	\\
4.99e-09	0	\\
5e-09	0	\\
};
\addplot [color=red,solid,forget plot]
  table[row sep=crcr]{
0	0	\\
1.1e-10	0	\\
2.2e-10	0	\\
3.3e-10	0	\\
4.4e-10	0	\\
5.4e-10	0	\\
6.5e-10	0	\\
7.5e-10	0	\\
8.6e-10	0	\\
9.6e-10	0	\\
1.07e-09	0	\\
1.18e-09	0	\\
1.28e-09	0	\\
1.38e-09	0	\\
1.49e-09	0	\\
1.59e-09	0	\\
1.69e-09	0	\\
1.8e-09	0	\\
1.9e-09	0	\\
2.01e-09	0	\\
2.11e-09	0	\\
2.21e-09	0	\\
2.32e-09	0	\\
2.42e-09	0	\\
2.52e-09	0	\\
2.63e-09	0	\\
2.73e-09	0	\\
2.83e-09	0	\\
2.93e-09	0	\\
3.04e-09	0	\\
3.14e-09	0	\\
3.24e-09	0	\\
3.34e-09	0	\\
3.45e-09	0	\\
3.55e-09	0	\\
3.65e-09	0	\\
3.75e-09	0	\\
3.86e-09	0	\\
3.96e-09	0	\\
4.06e-09	0	\\
4.16e-09	0	\\
4.27e-09	0	\\
4.37e-09	0	\\
4.47e-09	0	\\
4.57e-09	0	\\
4.68e-09	0	\\
4.78e-09	0	\\
4.89e-09	0	\\
4.99e-09	0	\\
5e-09	0	\\
};
\addplot [color=mycolor1,solid,forget plot]
  table[row sep=crcr]{
0	0	\\
1.1e-10	0	\\
2.2e-10	0	\\
3.3e-10	0	\\
4.4e-10	0	\\
5.4e-10	0	\\
6.5e-10	0	\\
7.5e-10	0	\\
8.6e-10	0	\\
9.6e-10	0	\\
1.07e-09	0	\\
1.18e-09	0	\\
1.28e-09	0	\\
1.38e-09	0	\\
1.49e-09	0	\\
1.59e-09	0	\\
1.69e-09	0	\\
1.8e-09	0	\\
1.9e-09	0	\\
2.01e-09	0	\\
2.11e-09	0	\\
2.21e-09	0	\\
2.32e-09	0	\\
2.42e-09	0	\\
2.52e-09	0	\\
2.63e-09	0	\\
2.73e-09	0	\\
2.83e-09	0	\\
2.93e-09	0	\\
3.04e-09	0	\\
3.14e-09	0	\\
3.24e-09	0	\\
3.34e-09	0	\\
3.45e-09	0	\\
3.55e-09	0	\\
3.65e-09	0	\\
3.75e-09	0	\\
3.86e-09	0	\\
3.96e-09	0	\\
4.06e-09	0	\\
4.16e-09	0	\\
4.27e-09	0	\\
4.37e-09	0	\\
4.47e-09	0	\\
4.57e-09	0	\\
4.68e-09	0	\\
4.78e-09	0	\\
4.89e-09	0	\\
4.99e-09	0	\\
5e-09	0	\\
};
\addplot [color=mycolor2,solid,forget plot]
  table[row sep=crcr]{
0	0	\\
1.1e-10	0	\\
2.2e-10	0	\\
3.3e-10	0	\\
4.4e-10	0	\\
5.4e-10	0	\\
6.5e-10	0	\\
7.5e-10	0	\\
8.6e-10	0	\\
9.6e-10	0	\\
1.07e-09	0	\\
1.18e-09	0	\\
1.28e-09	0	\\
1.38e-09	0	\\
1.49e-09	0	\\
1.59e-09	0	\\
1.69e-09	0	\\
1.8e-09	0	\\
1.9e-09	0	\\
2.01e-09	0	\\
2.11e-09	0	\\
2.21e-09	0	\\
2.32e-09	0	\\
2.42e-09	0	\\
2.52e-09	0	\\
2.63e-09	0	\\
2.73e-09	0	\\
2.83e-09	0	\\
2.93e-09	0	\\
3.04e-09	0	\\
3.14e-09	0	\\
3.24e-09	0	\\
3.34e-09	0	\\
3.45e-09	0	\\
3.55e-09	0	\\
3.65e-09	0	\\
3.75e-09	0	\\
3.86e-09	0	\\
3.96e-09	0	\\
4.06e-09	0	\\
4.16e-09	0	\\
4.27e-09	0	\\
4.37e-09	0	\\
4.47e-09	0	\\
4.57e-09	0	\\
4.68e-09	0	\\
4.78e-09	0	\\
4.89e-09	0	\\
4.99e-09	0	\\
5e-09	0	\\
};
\addplot [color=mycolor3,solid,forget plot]
  table[row sep=crcr]{
0	0	\\
1.1e-10	0	\\
2.2e-10	0	\\
3.3e-10	0	\\
4.4e-10	0	\\
5.4e-10	0	\\
6.5e-10	0	\\
7.5e-10	0	\\
8.6e-10	0	\\
9.6e-10	0	\\
1.07e-09	0	\\
1.18e-09	0	\\
1.28e-09	0	\\
1.38e-09	0	\\
1.49e-09	0	\\
1.59e-09	0	\\
1.69e-09	0	\\
1.8e-09	0	\\
1.9e-09	0	\\
2.01e-09	0	\\
2.11e-09	0	\\
2.21e-09	0	\\
2.32e-09	0	\\
2.42e-09	0	\\
2.52e-09	0	\\
2.63e-09	0	\\
2.73e-09	0	\\
2.83e-09	0	\\
2.93e-09	0	\\
3.04e-09	0	\\
3.14e-09	0	\\
3.24e-09	0	\\
3.34e-09	0	\\
3.45e-09	0	\\
3.55e-09	0	\\
3.65e-09	0	\\
3.75e-09	0	\\
3.86e-09	0	\\
3.96e-09	0	\\
4.06e-09	0	\\
4.16e-09	0	\\
4.27e-09	0	\\
4.37e-09	0	\\
4.47e-09	0	\\
4.57e-09	0	\\
4.68e-09	0	\\
4.78e-09	0	\\
4.89e-09	0	\\
4.99e-09	0	\\
5e-09	0	\\
};
\addplot [color=darkgray,solid,forget plot]
  table[row sep=crcr]{
0	0	\\
1.1e-10	0	\\
2.2e-10	0	\\
3.3e-10	0	\\
4.4e-10	0	\\
5.4e-10	0	\\
6.5e-10	0	\\
7.5e-10	0	\\
8.6e-10	0	\\
9.6e-10	0	\\
1.07e-09	0	\\
1.18e-09	0	\\
1.28e-09	0	\\
1.38e-09	0	\\
1.49e-09	0	\\
1.59e-09	0	\\
1.69e-09	0	\\
1.8e-09	0	\\
1.9e-09	0	\\
2.01e-09	0	\\
2.11e-09	0	\\
2.21e-09	0	\\
2.32e-09	0	\\
2.42e-09	0	\\
2.52e-09	0	\\
2.63e-09	0	\\
2.73e-09	0	\\
2.83e-09	0	\\
2.93e-09	0	\\
3.04e-09	0	\\
3.14e-09	0	\\
3.24e-09	0	\\
3.34e-09	0	\\
3.45e-09	0	\\
3.55e-09	0	\\
3.65e-09	0	\\
3.75e-09	0	\\
3.86e-09	0	\\
3.96e-09	0	\\
4.06e-09	0	\\
4.16e-09	0	\\
4.27e-09	0	\\
4.37e-09	0	\\
4.47e-09	0	\\
4.57e-09	0	\\
4.68e-09	0	\\
4.78e-09	0	\\
4.89e-09	0	\\
4.99e-09	0	\\
5e-09	0	\\
};
\addplot [color=blue,solid,forget plot]
  table[row sep=crcr]{
0	0	\\
1.1e-10	0	\\
2.2e-10	0	\\
3.3e-10	0	\\
4.4e-10	0	\\
5.4e-10	0	\\
6.5e-10	0	\\
7.5e-10	0	\\
8.6e-10	0	\\
9.6e-10	0	\\
1.07e-09	0	\\
1.18e-09	0	\\
1.28e-09	0	\\
1.38e-09	0	\\
1.49e-09	0	\\
1.59e-09	0	\\
1.69e-09	0	\\
1.8e-09	0	\\
1.9e-09	0	\\
2.01e-09	0	\\
2.11e-09	0	\\
2.21e-09	0	\\
2.32e-09	0	\\
2.42e-09	0	\\
2.52e-09	0	\\
2.63e-09	0	\\
2.73e-09	0	\\
2.83e-09	0	\\
2.93e-09	0	\\
3.04e-09	0	\\
3.14e-09	0	\\
3.24e-09	0	\\
3.34e-09	0	\\
3.45e-09	0	\\
3.55e-09	0	\\
3.65e-09	0	\\
3.75e-09	0	\\
3.86e-09	0	\\
3.96e-09	0	\\
4.06e-09	0	\\
4.16e-09	0	\\
4.27e-09	0	\\
4.37e-09	0	\\
4.47e-09	0	\\
4.57e-09	0	\\
4.68e-09	0	\\
4.78e-09	0	\\
4.89e-09	0	\\
4.99e-09	0	\\
5e-09	0	\\
};
\addplot [color=black!50!green,solid,forget plot]
  table[row sep=crcr]{
0	0	\\
1.1e-10	0	\\
2.2e-10	0	\\
3.3e-10	0	\\
4.4e-10	0	\\
5.4e-10	0	\\
6.5e-10	0	\\
7.5e-10	0	\\
8.6e-10	0	\\
9.6e-10	0	\\
1.07e-09	0	\\
1.18e-09	0	\\
1.28e-09	0	\\
1.38e-09	0	\\
1.49e-09	0	\\
1.59e-09	0	\\
1.69e-09	0	\\
1.8e-09	0	\\
1.9e-09	0	\\
2.01e-09	0	\\
2.11e-09	0	\\
2.21e-09	0	\\
2.32e-09	0	\\
2.42e-09	0	\\
2.52e-09	0	\\
2.63e-09	0	\\
2.73e-09	0	\\
2.83e-09	0	\\
2.93e-09	0	\\
3.04e-09	0	\\
3.14e-09	0	\\
3.24e-09	0	\\
3.34e-09	0	\\
3.45e-09	0	\\
3.55e-09	0	\\
3.65e-09	0	\\
3.75e-09	0	\\
3.86e-09	0	\\
3.96e-09	0	\\
4.06e-09	0	\\
4.16e-09	0	\\
4.27e-09	0	\\
4.37e-09	0	\\
4.47e-09	0	\\
4.57e-09	0	\\
4.68e-09	0	\\
4.78e-09	0	\\
4.89e-09	0	\\
4.99e-09	0	\\
5e-09	0	\\
};
\addplot [color=red,solid,forget plot]
  table[row sep=crcr]{
0	0	\\
1.1e-10	0	\\
2.2e-10	0	\\
3.3e-10	0	\\
4.4e-10	0	\\
5.4e-10	0	\\
6.5e-10	0	\\
7.5e-10	0	\\
8.6e-10	0	\\
9.6e-10	0	\\
1.07e-09	0	\\
1.18e-09	0	\\
1.28e-09	0	\\
1.38e-09	0	\\
1.49e-09	0	\\
1.59e-09	0	\\
1.69e-09	0	\\
1.8e-09	0	\\
1.9e-09	0	\\
2.01e-09	0	\\
2.11e-09	0	\\
2.21e-09	0	\\
2.32e-09	0	\\
2.42e-09	0	\\
2.52e-09	0	\\
2.63e-09	0	\\
2.73e-09	0	\\
2.83e-09	0	\\
2.93e-09	0	\\
3.04e-09	0	\\
3.14e-09	0	\\
3.24e-09	0	\\
3.34e-09	0	\\
3.45e-09	0	\\
3.55e-09	0	\\
3.65e-09	0	\\
3.75e-09	0	\\
3.86e-09	0	\\
3.96e-09	0	\\
4.06e-09	0	\\
4.16e-09	0	\\
4.27e-09	0	\\
4.37e-09	0	\\
4.47e-09	0	\\
4.57e-09	0	\\
4.68e-09	0	\\
4.78e-09	0	\\
4.89e-09	0	\\
4.99e-09	0	\\
5e-09	0	\\
};
\addplot [color=mycolor1,solid,forget plot]
  table[row sep=crcr]{
0	0	\\
1.1e-10	0	\\
2.2e-10	0	\\
3.3e-10	0	\\
4.4e-10	0	\\
5.4e-10	0	\\
6.5e-10	0	\\
7.5e-10	0	\\
8.6e-10	0	\\
9.6e-10	0	\\
1.07e-09	0	\\
1.18e-09	0	\\
1.28e-09	0	\\
1.38e-09	0	\\
1.49e-09	0	\\
1.59e-09	0	\\
1.69e-09	0	\\
1.8e-09	0	\\
1.9e-09	0	\\
2.01e-09	0	\\
2.11e-09	0	\\
2.21e-09	0	\\
2.32e-09	0	\\
2.42e-09	0	\\
2.52e-09	0	\\
2.63e-09	0	\\
2.73e-09	0	\\
2.83e-09	0	\\
2.93e-09	0	\\
3.04e-09	0	\\
3.14e-09	0	\\
3.24e-09	0	\\
3.34e-09	0	\\
3.45e-09	0	\\
3.55e-09	0	\\
3.65e-09	0	\\
3.75e-09	0	\\
3.86e-09	0	\\
3.96e-09	0	\\
4.06e-09	0	\\
4.16e-09	0	\\
4.27e-09	0	\\
4.37e-09	0	\\
4.47e-09	0	\\
4.57e-09	0	\\
4.68e-09	0	\\
4.78e-09	0	\\
4.89e-09	0	\\
4.99e-09	0	\\
5e-09	0	\\
};
\addplot [color=mycolor2,solid,forget plot]
  table[row sep=crcr]{
0	0	\\
1.1e-10	0	\\
2.2e-10	0	\\
3.3e-10	0	\\
4.4e-10	0	\\
5.4e-10	0	\\
6.5e-10	0	\\
7.5e-10	0	\\
8.6e-10	0	\\
9.6e-10	0	\\
1.07e-09	0	\\
1.18e-09	0	\\
1.28e-09	0	\\
1.38e-09	0	\\
1.49e-09	0	\\
1.59e-09	0	\\
1.69e-09	0	\\
1.8e-09	0	\\
1.9e-09	0	\\
2.01e-09	0	\\
2.11e-09	0	\\
2.21e-09	0	\\
2.32e-09	0	\\
2.42e-09	0	\\
2.52e-09	0	\\
2.63e-09	0	\\
2.73e-09	0	\\
2.83e-09	0	\\
2.93e-09	0	\\
3.04e-09	0	\\
3.14e-09	0	\\
3.24e-09	0	\\
3.34e-09	0	\\
3.45e-09	0	\\
3.55e-09	0	\\
3.65e-09	0	\\
3.75e-09	0	\\
3.86e-09	0	\\
3.96e-09	0	\\
4.06e-09	0	\\
4.16e-09	0	\\
4.27e-09	0	\\
4.37e-09	0	\\
4.47e-09	0	\\
4.57e-09	0	\\
4.68e-09	0	\\
4.78e-09	0	\\
4.89e-09	0	\\
4.99e-09	0	\\
5e-09	0	\\
};
\addplot [color=mycolor3,solid,forget plot]
  table[row sep=crcr]{
0	0	\\
1.1e-10	0	\\
2.2e-10	0	\\
3.3e-10	0	\\
4.4e-10	0	\\
5.4e-10	0	\\
6.5e-10	0	\\
7.5e-10	0	\\
8.6e-10	0	\\
9.6e-10	0	\\
1.07e-09	0	\\
1.18e-09	0	\\
1.28e-09	0	\\
1.38e-09	0	\\
1.49e-09	0	\\
1.59e-09	0	\\
1.69e-09	0	\\
1.8e-09	0	\\
1.9e-09	0	\\
2.01e-09	0	\\
2.11e-09	0	\\
2.21e-09	0	\\
2.32e-09	0	\\
2.42e-09	0	\\
2.52e-09	0	\\
2.63e-09	0	\\
2.73e-09	0	\\
2.83e-09	0	\\
2.93e-09	0	\\
3.04e-09	0	\\
3.14e-09	0	\\
3.24e-09	0	\\
3.34e-09	0	\\
3.45e-09	0	\\
3.55e-09	0	\\
3.65e-09	0	\\
3.75e-09	0	\\
3.86e-09	0	\\
3.96e-09	0	\\
4.06e-09	0	\\
4.16e-09	0	\\
4.27e-09	0	\\
4.37e-09	0	\\
4.47e-09	0	\\
4.57e-09	0	\\
4.68e-09	0	\\
4.78e-09	0	\\
4.89e-09	0	\\
4.99e-09	0	\\
5e-09	0	\\
};
\addplot [color=darkgray,solid,forget plot]
  table[row sep=crcr]{
0	0	\\
1.1e-10	0	\\
2.2e-10	0	\\
3.3e-10	0	\\
4.4e-10	0	\\
5.4e-10	0	\\
6.5e-10	0	\\
7.5e-10	0	\\
8.6e-10	0	\\
9.6e-10	0	\\
1.07e-09	0	\\
1.18e-09	0	\\
1.28e-09	0	\\
1.38e-09	0	\\
1.49e-09	0	\\
1.59e-09	0	\\
1.69e-09	0	\\
1.8e-09	0	\\
1.9e-09	0	\\
2.01e-09	0	\\
2.11e-09	0	\\
2.21e-09	0	\\
2.32e-09	0	\\
2.42e-09	0	\\
2.52e-09	0	\\
2.63e-09	0	\\
2.73e-09	0	\\
2.83e-09	0	\\
2.93e-09	0	\\
3.04e-09	0	\\
3.14e-09	0	\\
3.24e-09	0	\\
3.34e-09	0	\\
3.45e-09	0	\\
3.55e-09	0	\\
3.65e-09	0	\\
3.75e-09	0	\\
3.86e-09	0	\\
3.96e-09	0	\\
4.06e-09	0	\\
4.16e-09	0	\\
4.27e-09	0	\\
4.37e-09	0	\\
4.47e-09	0	\\
4.57e-09	0	\\
4.68e-09	0	\\
4.78e-09	0	\\
4.89e-09	0	\\
4.99e-09	0	\\
5e-09	0	\\
};
\addplot [color=blue,solid,forget plot]
  table[row sep=crcr]{
0	0	\\
1.1e-10	0	\\
2.2e-10	0	\\
3.3e-10	0	\\
4.4e-10	0	\\
5.4e-10	0	\\
6.5e-10	0	\\
7.5e-10	0	\\
8.6e-10	0	\\
9.6e-10	0	\\
1.07e-09	0	\\
1.18e-09	0	\\
1.28e-09	0	\\
1.38e-09	0	\\
1.49e-09	0	\\
1.59e-09	0	\\
1.69e-09	0	\\
1.8e-09	0	\\
1.9e-09	0	\\
2.01e-09	0	\\
2.11e-09	0	\\
2.21e-09	0	\\
2.32e-09	0	\\
2.42e-09	0	\\
2.52e-09	0	\\
2.63e-09	0	\\
2.73e-09	0	\\
2.83e-09	0	\\
2.93e-09	0	\\
3.04e-09	0	\\
3.14e-09	0	\\
3.24e-09	0	\\
3.34e-09	0	\\
3.45e-09	0	\\
3.55e-09	0	\\
3.65e-09	0	\\
3.75e-09	0	\\
3.86e-09	0	\\
3.96e-09	0	\\
4.06e-09	0	\\
4.16e-09	0	\\
4.27e-09	0	\\
4.37e-09	0	\\
4.47e-09	0	\\
4.57e-09	0	\\
4.68e-09	0	\\
4.78e-09	0	\\
4.89e-09	0	\\
4.99e-09	0	\\
5e-09	0	\\
};
\addplot [color=black!50!green,solid,forget plot]
  table[row sep=crcr]{
0	0	\\
1.1e-10	0	\\
2.2e-10	0	\\
3.3e-10	0	\\
4.4e-10	0	\\
5.4e-10	0	\\
6.5e-10	0	\\
7.5e-10	0	\\
8.6e-10	0	\\
9.6e-10	0	\\
1.07e-09	0	\\
1.18e-09	0	\\
1.28e-09	0	\\
1.38e-09	0	\\
1.49e-09	0	\\
1.59e-09	0	\\
1.69e-09	0	\\
1.8e-09	0	\\
1.9e-09	0	\\
2.01e-09	0	\\
2.11e-09	0	\\
2.21e-09	0	\\
2.32e-09	0	\\
2.42e-09	0	\\
2.52e-09	0	\\
2.63e-09	0	\\
2.73e-09	0	\\
2.83e-09	0	\\
2.93e-09	0	\\
3.04e-09	0	\\
3.14e-09	0	\\
3.24e-09	0	\\
3.34e-09	0	\\
3.45e-09	0	\\
3.55e-09	0	\\
3.65e-09	0	\\
3.75e-09	0	\\
3.86e-09	0	\\
3.96e-09	0	\\
4.06e-09	0	\\
4.16e-09	0	\\
4.27e-09	0	\\
4.37e-09	0	\\
4.47e-09	0	\\
4.57e-09	0	\\
4.68e-09	0	\\
4.78e-09	0	\\
4.89e-09	0	\\
4.99e-09	0	\\
5e-09	0	\\
};
\addplot [color=red,solid,forget plot]
  table[row sep=crcr]{
0	0	\\
1.1e-10	0	\\
2.2e-10	0	\\
3.3e-10	0	\\
4.4e-10	0	\\
5.4e-10	0	\\
6.5e-10	0	\\
7.5e-10	0	\\
8.6e-10	0	\\
9.6e-10	0	\\
1.07e-09	0	\\
1.18e-09	0	\\
1.28e-09	0	\\
1.38e-09	0	\\
1.49e-09	0	\\
1.59e-09	0	\\
1.69e-09	0	\\
1.8e-09	0	\\
1.9e-09	0	\\
2.01e-09	0	\\
2.11e-09	0	\\
2.21e-09	0	\\
2.32e-09	0	\\
2.42e-09	0	\\
2.52e-09	0	\\
2.63e-09	0	\\
2.73e-09	0	\\
2.83e-09	0	\\
2.93e-09	0	\\
3.04e-09	0	\\
3.14e-09	0	\\
3.24e-09	0	\\
3.34e-09	0	\\
3.45e-09	0	\\
3.55e-09	0	\\
3.65e-09	0	\\
3.75e-09	0	\\
3.86e-09	0	\\
3.96e-09	0	\\
4.06e-09	0	\\
4.16e-09	0	\\
4.27e-09	0	\\
4.37e-09	0	\\
4.47e-09	0	\\
4.57e-09	0	\\
4.68e-09	0	\\
4.78e-09	0	\\
4.89e-09	0	\\
4.99e-09	0	\\
5e-09	0	\\
};
\addplot [color=mycolor1,solid,forget plot]
  table[row sep=crcr]{
0	0	\\
1.1e-10	0	\\
2.2e-10	0	\\
3.3e-10	0	\\
4.4e-10	0	\\
5.4e-10	0	\\
6.5e-10	0	\\
7.5e-10	0	\\
8.6e-10	0	\\
9.6e-10	0	\\
1.07e-09	0	\\
1.18e-09	0	\\
1.28e-09	0	\\
1.38e-09	0	\\
1.49e-09	0	\\
1.59e-09	0	\\
1.69e-09	0	\\
1.8e-09	0	\\
1.9e-09	0	\\
2.01e-09	0	\\
2.11e-09	0	\\
2.21e-09	0	\\
2.32e-09	0	\\
2.42e-09	0	\\
2.52e-09	0	\\
2.63e-09	0	\\
2.73e-09	0	\\
2.83e-09	0	\\
2.93e-09	0	\\
3.04e-09	0	\\
3.14e-09	0	\\
3.24e-09	0	\\
3.34e-09	0	\\
3.45e-09	0	\\
3.55e-09	0	\\
3.65e-09	0	\\
3.75e-09	0	\\
3.86e-09	0	\\
3.96e-09	0	\\
4.06e-09	0	\\
4.16e-09	0	\\
4.27e-09	0	\\
4.37e-09	0	\\
4.47e-09	0	\\
4.57e-09	0	\\
4.68e-09	0	\\
4.78e-09	0	\\
4.89e-09	0	\\
4.99e-09	0	\\
5e-09	0	\\
};
\addplot [color=mycolor2,solid,forget plot]
  table[row sep=crcr]{
0	0	\\
1.1e-10	0	\\
2.2e-10	0	\\
3.3e-10	0	\\
4.4e-10	0	\\
5.4e-10	0	\\
6.5e-10	0	\\
7.5e-10	0	\\
8.6e-10	0	\\
9.6e-10	0	\\
1.07e-09	0	\\
1.18e-09	0	\\
1.28e-09	0	\\
1.38e-09	0	\\
1.49e-09	0	\\
1.59e-09	0	\\
1.69e-09	0	\\
1.8e-09	0	\\
1.9e-09	0	\\
2.01e-09	0	\\
2.11e-09	0	\\
2.21e-09	0	\\
2.32e-09	0	\\
2.42e-09	0	\\
2.52e-09	0	\\
2.63e-09	0	\\
2.73e-09	0	\\
2.83e-09	0	\\
2.93e-09	0	\\
3.04e-09	0	\\
3.14e-09	0	\\
3.24e-09	0	\\
3.34e-09	0	\\
3.45e-09	0	\\
3.55e-09	0	\\
3.65e-09	0	\\
3.75e-09	0	\\
3.86e-09	0	\\
3.96e-09	0	\\
4.06e-09	0	\\
4.16e-09	0	\\
4.27e-09	0	\\
4.37e-09	0	\\
4.47e-09	0	\\
4.57e-09	0	\\
4.68e-09	0	\\
4.78e-09	0	\\
4.89e-09	0	\\
4.99e-09	0	\\
5e-09	0	\\
};
\addplot [color=mycolor3,solid,forget plot]
  table[row sep=crcr]{
0	0	\\
1.1e-10	0	\\
2.2e-10	0	\\
3.3e-10	0	\\
4.4e-10	0	\\
5.4e-10	0	\\
6.5e-10	0	\\
7.5e-10	0	\\
8.6e-10	0	\\
9.6e-10	0	\\
1.07e-09	0	\\
1.18e-09	0	\\
1.28e-09	0	\\
1.38e-09	0	\\
1.49e-09	0	\\
1.59e-09	0	\\
1.69e-09	0	\\
1.8e-09	0	\\
1.9e-09	0	\\
2.01e-09	0	\\
2.11e-09	0	\\
2.21e-09	0	\\
2.32e-09	0	\\
2.42e-09	0	\\
2.52e-09	0	\\
2.63e-09	0	\\
2.73e-09	0	\\
2.83e-09	0	\\
2.93e-09	0	\\
3.04e-09	0	\\
3.14e-09	0	\\
3.24e-09	0	\\
3.34e-09	0	\\
3.45e-09	0	\\
3.55e-09	0	\\
3.65e-09	0	\\
3.75e-09	0	\\
3.86e-09	0	\\
3.96e-09	0	\\
4.06e-09	0	\\
4.16e-09	0	\\
4.27e-09	0	\\
4.37e-09	0	\\
4.47e-09	0	\\
4.57e-09	0	\\
4.68e-09	0	\\
4.78e-09	0	\\
4.89e-09	0	\\
4.99e-09	0	\\
5e-09	0	\\
};
\addplot [color=darkgray,solid,forget plot]
  table[row sep=crcr]{
0	0	\\
1.1e-10	0	\\
2.2e-10	0	\\
3.3e-10	0	\\
4.4e-10	0	\\
5.4e-10	0	\\
6.5e-10	0	\\
7.5e-10	0	\\
8.6e-10	0	\\
9.6e-10	0	\\
1.07e-09	0	\\
1.18e-09	0	\\
1.28e-09	0	\\
1.38e-09	0	\\
1.49e-09	0	\\
1.59e-09	0	\\
1.69e-09	0	\\
1.8e-09	0	\\
1.9e-09	0	\\
2.01e-09	0	\\
2.11e-09	0	\\
2.21e-09	0	\\
2.32e-09	0	\\
2.42e-09	0	\\
2.52e-09	0	\\
2.63e-09	0	\\
2.73e-09	0	\\
2.83e-09	0	\\
2.93e-09	0	\\
3.04e-09	0	\\
3.14e-09	0	\\
3.24e-09	0	\\
3.34e-09	0	\\
3.45e-09	0	\\
3.55e-09	0	\\
3.65e-09	0	\\
3.75e-09	0	\\
3.86e-09	0	\\
3.96e-09	0	\\
4.06e-09	0	\\
4.16e-09	0	\\
4.27e-09	0	\\
4.37e-09	0	\\
4.47e-09	0	\\
4.57e-09	0	\\
4.68e-09	0	\\
4.78e-09	0	\\
4.89e-09	0	\\
4.99e-09	0	\\
5e-09	0	\\
};
\addplot [color=blue,solid,forget plot]
  table[row sep=crcr]{
0	0	\\
1.1e-10	0	\\
2.2e-10	0	\\
3.3e-10	0	\\
4.4e-10	0	\\
5.4e-10	0	\\
6.5e-10	0	\\
7.5e-10	0	\\
8.6e-10	0	\\
9.6e-10	0	\\
1.07e-09	0	\\
1.18e-09	0	\\
1.28e-09	0	\\
1.38e-09	0	\\
1.49e-09	0	\\
1.59e-09	0	\\
1.69e-09	0	\\
1.8e-09	0	\\
1.9e-09	0	\\
2.01e-09	0	\\
2.11e-09	0	\\
2.21e-09	0	\\
2.32e-09	0	\\
2.42e-09	0	\\
2.52e-09	0	\\
2.63e-09	0	\\
2.73e-09	0	\\
2.83e-09	0	\\
2.93e-09	0	\\
3.04e-09	0	\\
3.14e-09	0	\\
3.24e-09	0	\\
3.34e-09	0	\\
3.45e-09	0	\\
3.55e-09	0	\\
3.65e-09	0	\\
3.75e-09	0	\\
3.86e-09	0	\\
3.96e-09	0	\\
4.06e-09	0	\\
4.16e-09	0	\\
4.27e-09	0	\\
4.37e-09	0	\\
4.47e-09	0	\\
4.57e-09	0	\\
4.68e-09	0	\\
4.78e-09	0	\\
4.89e-09	0	\\
4.99e-09	0	\\
5e-09	0	\\
};
\addplot [color=black!50!green,solid,forget plot]
  table[row sep=crcr]{
0	0	\\
1.1e-10	0	\\
2.2e-10	0	\\
3.3e-10	0	\\
4.4e-10	0	\\
5.4e-10	0	\\
6.5e-10	0	\\
7.5e-10	0	\\
8.6e-10	0	\\
9.6e-10	0	\\
1.07e-09	0	\\
1.18e-09	0	\\
1.28e-09	0	\\
1.38e-09	0	\\
1.49e-09	0	\\
1.59e-09	0	\\
1.69e-09	0	\\
1.8e-09	0	\\
1.9e-09	0	\\
2.01e-09	0	\\
2.11e-09	0	\\
2.21e-09	0	\\
2.32e-09	0	\\
2.42e-09	0	\\
2.52e-09	0	\\
2.63e-09	0	\\
2.73e-09	0	\\
2.83e-09	0	\\
2.93e-09	0	\\
3.04e-09	0	\\
3.14e-09	0	\\
3.24e-09	0	\\
3.34e-09	0	\\
3.45e-09	0	\\
3.55e-09	0	\\
3.65e-09	0	\\
3.75e-09	0	\\
3.86e-09	0	\\
3.96e-09	0	\\
4.06e-09	0	\\
4.16e-09	0	\\
4.27e-09	0	\\
4.37e-09	0	\\
4.47e-09	0	\\
4.57e-09	0	\\
4.68e-09	0	\\
4.78e-09	0	\\
4.89e-09	0	\\
4.99e-09	0	\\
5e-09	0	\\
};
\addplot [color=red,solid,forget plot]
  table[row sep=crcr]{
0	0	\\
1.1e-10	0	\\
2.2e-10	0	\\
3.3e-10	0	\\
4.4e-10	0	\\
5.4e-10	0	\\
6.5e-10	0	\\
7.5e-10	0	\\
8.6e-10	0	\\
9.6e-10	0	\\
1.07e-09	0	\\
1.18e-09	0	\\
1.28e-09	0	\\
1.38e-09	0	\\
1.49e-09	0	\\
1.59e-09	0	\\
1.69e-09	0	\\
1.8e-09	0	\\
1.9e-09	0	\\
2.01e-09	0	\\
2.11e-09	0	\\
2.21e-09	0	\\
2.32e-09	0	\\
2.42e-09	0	\\
2.52e-09	0	\\
2.63e-09	0	\\
2.73e-09	0	\\
2.83e-09	0	\\
2.93e-09	0	\\
3.04e-09	0	\\
3.14e-09	0	\\
3.24e-09	0	\\
3.34e-09	0	\\
3.45e-09	0	\\
3.55e-09	0	\\
3.65e-09	0	\\
3.75e-09	0	\\
3.86e-09	0	\\
3.96e-09	0	\\
4.06e-09	0	\\
4.16e-09	0	\\
4.27e-09	0	\\
4.37e-09	0	\\
4.47e-09	0	\\
4.57e-09	0	\\
4.68e-09	0	\\
4.78e-09	0	\\
4.89e-09	0	\\
4.99e-09	0	\\
5e-09	0	\\
};
\addplot [color=mycolor1,solid,forget plot]
  table[row sep=crcr]{
0	0	\\
1.1e-10	0	\\
2.2e-10	0	\\
3.3e-10	0	\\
4.4e-10	0	\\
5.4e-10	0	\\
6.5e-10	0	\\
7.5e-10	0	\\
8.6e-10	0	\\
9.6e-10	0	\\
1.07e-09	0	\\
1.18e-09	0	\\
1.28e-09	0	\\
1.38e-09	0	\\
1.49e-09	0	\\
1.59e-09	0	\\
1.69e-09	0	\\
1.8e-09	0	\\
1.9e-09	0	\\
2.01e-09	0	\\
2.11e-09	0	\\
2.21e-09	0	\\
2.32e-09	0	\\
2.42e-09	0	\\
2.52e-09	0	\\
2.63e-09	0	\\
2.73e-09	0	\\
2.83e-09	0	\\
2.93e-09	0	\\
3.04e-09	0	\\
3.14e-09	0	\\
3.24e-09	0	\\
3.34e-09	0	\\
3.45e-09	0	\\
3.55e-09	0	\\
3.65e-09	0	\\
3.75e-09	0	\\
3.86e-09	0	\\
3.96e-09	0	\\
4.06e-09	0	\\
4.16e-09	0	\\
4.27e-09	0	\\
4.37e-09	0	\\
4.47e-09	0	\\
4.57e-09	0	\\
4.68e-09	0	\\
4.78e-09	0	\\
4.89e-09	0	\\
4.99e-09	0	\\
5e-09	0	\\
};
\addplot [color=mycolor2,solid,forget plot]
  table[row sep=crcr]{
0	0	\\
1.1e-10	0	\\
2.2e-10	0	\\
3.3e-10	0	\\
4.4e-10	0	\\
5.4e-10	0	\\
6.5e-10	0	\\
7.5e-10	0	\\
8.6e-10	0	\\
9.6e-10	0	\\
1.07e-09	0	\\
1.18e-09	0	\\
1.28e-09	0	\\
1.38e-09	0	\\
1.49e-09	0	\\
1.59e-09	0	\\
1.69e-09	0	\\
1.8e-09	0	\\
1.9e-09	0	\\
2.01e-09	0	\\
2.11e-09	0	\\
2.21e-09	0	\\
2.32e-09	0	\\
2.42e-09	0	\\
2.52e-09	0	\\
2.63e-09	0	\\
2.73e-09	0	\\
2.83e-09	0	\\
2.93e-09	0	\\
3.04e-09	0	\\
3.14e-09	0	\\
3.24e-09	0	\\
3.34e-09	0	\\
3.45e-09	0	\\
3.55e-09	0	\\
3.65e-09	0	\\
3.75e-09	0	\\
3.86e-09	0	\\
3.96e-09	0	\\
4.06e-09	0	\\
4.16e-09	0	\\
4.27e-09	0	\\
4.37e-09	0	\\
4.47e-09	0	\\
4.57e-09	0	\\
4.68e-09	0	\\
4.78e-09	0	\\
4.89e-09	0	\\
4.99e-09	0	\\
5e-09	0	\\
};
\addplot [color=mycolor3,solid,forget plot]
  table[row sep=crcr]{
0	0	\\
1.1e-10	0	\\
2.2e-10	0	\\
3.3e-10	0	\\
4.4e-10	0	\\
5.4e-10	0	\\
6.5e-10	0	\\
7.5e-10	0	\\
8.6e-10	0	\\
9.6e-10	0	\\
1.07e-09	0	\\
1.18e-09	0	\\
1.28e-09	0	\\
1.38e-09	0	\\
1.49e-09	0	\\
1.59e-09	0	\\
1.69e-09	0	\\
1.8e-09	0	\\
1.9e-09	0	\\
2.01e-09	0	\\
2.11e-09	0	\\
2.21e-09	0	\\
2.32e-09	0	\\
2.42e-09	0	\\
2.52e-09	0	\\
2.63e-09	0	\\
2.73e-09	0	\\
2.83e-09	0	\\
2.93e-09	0	\\
3.04e-09	0	\\
3.14e-09	0	\\
3.24e-09	0	\\
3.34e-09	0	\\
3.45e-09	0	\\
3.55e-09	0	\\
3.65e-09	0	\\
3.75e-09	0	\\
3.86e-09	0	\\
3.96e-09	0	\\
4.06e-09	0	\\
4.16e-09	0	\\
4.27e-09	0	\\
4.37e-09	0	\\
4.47e-09	0	\\
4.57e-09	0	\\
4.68e-09	0	\\
4.78e-09	0	\\
4.89e-09	0	\\
4.99e-09	0	\\
5e-09	0	\\
};
\addplot [color=darkgray,solid,forget plot]
  table[row sep=crcr]{
0	0	\\
1.1e-10	0	\\
2.2e-10	0	\\
3.3e-10	0	\\
4.4e-10	0	\\
5.4e-10	0	\\
6.5e-10	0	\\
7.5e-10	0	\\
8.6e-10	0	\\
9.6e-10	0	\\
1.07e-09	0	\\
1.18e-09	0	\\
1.28e-09	0	\\
1.38e-09	0	\\
1.49e-09	0	\\
1.59e-09	0	\\
1.69e-09	0	\\
1.8e-09	0	\\
1.9e-09	0	\\
2.01e-09	0	\\
2.11e-09	0	\\
2.21e-09	0	\\
2.32e-09	0	\\
2.42e-09	0	\\
2.52e-09	0	\\
2.63e-09	0	\\
2.73e-09	0	\\
2.83e-09	0	\\
2.93e-09	0	\\
3.04e-09	0	\\
3.14e-09	0	\\
3.24e-09	0	\\
3.34e-09	0	\\
3.45e-09	0	\\
3.55e-09	0	\\
3.65e-09	0	\\
3.75e-09	0	\\
3.86e-09	0	\\
3.96e-09	0	\\
4.06e-09	0	\\
4.16e-09	0	\\
4.27e-09	0	\\
4.37e-09	0	\\
4.47e-09	0	\\
4.57e-09	0	\\
4.68e-09	0	\\
4.78e-09	0	\\
4.89e-09	0	\\
4.99e-09	0	\\
5e-09	0	\\
};
\addplot [color=blue,solid,forget plot]
  table[row sep=crcr]{
0	0	\\
1.1e-10	0	\\
2.2e-10	0	\\
3.3e-10	0	\\
4.4e-10	0	\\
5.4e-10	0	\\
6.5e-10	0	\\
7.5e-10	0	\\
8.6e-10	0	\\
9.6e-10	0	\\
1.07e-09	0	\\
1.18e-09	0	\\
1.28e-09	0	\\
1.38e-09	0	\\
1.49e-09	0	\\
1.59e-09	0	\\
1.69e-09	0	\\
1.8e-09	0	\\
1.9e-09	0	\\
2.01e-09	0	\\
2.11e-09	0	\\
2.21e-09	0	\\
2.32e-09	0	\\
2.42e-09	0	\\
2.52e-09	0	\\
2.63e-09	0	\\
2.73e-09	0	\\
2.83e-09	0	\\
2.93e-09	0	\\
3.04e-09	0	\\
3.14e-09	0	\\
3.24e-09	0	\\
3.34e-09	0	\\
3.45e-09	0	\\
3.55e-09	0	\\
3.65e-09	0	\\
3.75e-09	0	\\
3.86e-09	0	\\
3.96e-09	0	\\
4.06e-09	0	\\
4.16e-09	0	\\
4.27e-09	0	\\
4.37e-09	0	\\
4.47e-09	0	\\
4.57e-09	0	\\
4.68e-09	0	\\
4.78e-09	0	\\
4.89e-09	0	\\
4.99e-09	0	\\
5e-09	0	\\
};
\addplot [color=black!50!green,solid,forget plot]
  table[row sep=crcr]{
0	0	\\
1.1e-10	0	\\
2.2e-10	0	\\
3.3e-10	0	\\
4.4e-10	0	\\
5.4e-10	0	\\
6.5e-10	0	\\
7.5e-10	0	\\
8.6e-10	0	\\
9.6e-10	0	\\
1.07e-09	0	\\
1.18e-09	0	\\
1.28e-09	0	\\
1.38e-09	0	\\
1.49e-09	0	\\
1.59e-09	0	\\
1.69e-09	0	\\
1.8e-09	0	\\
1.9e-09	0	\\
2.01e-09	0	\\
2.11e-09	0	\\
2.21e-09	0	\\
2.32e-09	0	\\
2.42e-09	0	\\
2.52e-09	0	\\
2.63e-09	0	\\
2.73e-09	0	\\
2.83e-09	0	\\
2.93e-09	0	\\
3.04e-09	0	\\
3.14e-09	0	\\
3.24e-09	0	\\
3.34e-09	0	\\
3.45e-09	0	\\
3.55e-09	0	\\
3.65e-09	0	\\
3.75e-09	0	\\
3.86e-09	0	\\
3.96e-09	0	\\
4.06e-09	0	\\
4.16e-09	0	\\
4.27e-09	0	\\
4.37e-09	0	\\
4.47e-09	0	\\
4.57e-09	0	\\
4.68e-09	0	\\
4.78e-09	0	\\
4.89e-09	0	\\
4.99e-09	0	\\
5e-09	0	\\
};
\addplot [color=red,solid,forget plot]
  table[row sep=crcr]{
0	0	\\
1.1e-10	0	\\
2.2e-10	0	\\
3.3e-10	0	\\
4.4e-10	0	\\
5.4e-10	0	\\
6.5e-10	0	\\
7.5e-10	0	\\
8.6e-10	0	\\
9.6e-10	0	\\
1.07e-09	0	\\
1.18e-09	0	\\
1.28e-09	0	\\
1.38e-09	0	\\
1.49e-09	0	\\
1.59e-09	0	\\
1.69e-09	0	\\
1.8e-09	0	\\
1.9e-09	0	\\
2.01e-09	0	\\
2.11e-09	0	\\
2.21e-09	0	\\
2.32e-09	0	\\
2.42e-09	0	\\
2.52e-09	0	\\
2.63e-09	0	\\
2.73e-09	0	\\
2.83e-09	0	\\
2.93e-09	0	\\
3.04e-09	0	\\
3.14e-09	0	\\
3.24e-09	0	\\
3.34e-09	0	\\
3.45e-09	0	\\
3.55e-09	0	\\
3.65e-09	0	\\
3.75e-09	0	\\
3.86e-09	0	\\
3.96e-09	0	\\
4.06e-09	0	\\
4.16e-09	0	\\
4.27e-09	0	\\
4.37e-09	0	\\
4.47e-09	0	\\
4.57e-09	0	\\
4.68e-09	0	\\
4.78e-09	0	\\
4.89e-09	0	\\
4.99e-09	0	\\
5e-09	0	\\
};
\addplot [color=mycolor1,solid,forget plot]
  table[row sep=crcr]{
0	0	\\
1.1e-10	0	\\
2.2e-10	0	\\
3.3e-10	0	\\
4.4e-10	0	\\
5.4e-10	0	\\
6.5e-10	0	\\
7.5e-10	0	\\
8.6e-10	0	\\
9.6e-10	0	\\
1.07e-09	0	\\
1.18e-09	0	\\
1.28e-09	0	\\
1.38e-09	0	\\
1.49e-09	0	\\
1.59e-09	0	\\
1.69e-09	0	\\
1.8e-09	0	\\
1.9e-09	0	\\
2.01e-09	0	\\
2.11e-09	0	\\
2.21e-09	0	\\
2.32e-09	0	\\
2.42e-09	0	\\
2.52e-09	0	\\
2.63e-09	0	\\
2.73e-09	0	\\
2.83e-09	0	\\
2.93e-09	0	\\
3.04e-09	0	\\
3.14e-09	0	\\
3.24e-09	0	\\
3.34e-09	0	\\
3.45e-09	0	\\
3.55e-09	0	\\
3.65e-09	0	\\
3.75e-09	0	\\
3.86e-09	0	\\
3.96e-09	0	\\
4.06e-09	0	\\
4.16e-09	0	\\
4.27e-09	0	\\
4.37e-09	0	\\
4.47e-09	0	\\
4.57e-09	0	\\
4.68e-09	0	\\
4.78e-09	0	\\
4.89e-09	0	\\
4.99e-09	0	\\
5e-09	0	\\
};
\addplot [color=mycolor2,solid,forget plot]
  table[row sep=crcr]{
0	0	\\
1.1e-10	0	\\
2.2e-10	0	\\
3.3e-10	0	\\
4.4e-10	0	\\
5.4e-10	0	\\
6.5e-10	0	\\
7.5e-10	0	\\
8.6e-10	0	\\
9.6e-10	0	\\
1.07e-09	0	\\
1.18e-09	0	\\
1.28e-09	0	\\
1.38e-09	0	\\
1.49e-09	0	\\
1.59e-09	0	\\
1.69e-09	0	\\
1.8e-09	0	\\
1.9e-09	0	\\
2.01e-09	0	\\
2.11e-09	0	\\
2.21e-09	0	\\
2.32e-09	0	\\
2.42e-09	0	\\
2.52e-09	0	\\
2.63e-09	0	\\
2.73e-09	0	\\
2.83e-09	0	\\
2.93e-09	0	\\
3.04e-09	0	\\
3.14e-09	0	\\
3.24e-09	0	\\
3.34e-09	0	\\
3.45e-09	0	\\
3.55e-09	0	\\
3.65e-09	0	\\
3.75e-09	0	\\
3.86e-09	0	\\
3.96e-09	0	\\
4.06e-09	0	\\
4.16e-09	0	\\
4.27e-09	0	\\
4.37e-09	0	\\
4.47e-09	0	\\
4.57e-09	0	\\
4.68e-09	0	\\
4.78e-09	0	\\
4.89e-09	0	\\
4.99e-09	0	\\
5e-09	0	\\
};
\addplot [color=mycolor3,solid,forget plot]
  table[row sep=crcr]{
0	0	\\
1.1e-10	0	\\
2.2e-10	0	\\
3.3e-10	0	\\
4.4e-10	0	\\
5.4e-10	0	\\
6.5e-10	0	\\
7.5e-10	0	\\
8.6e-10	0	\\
9.6e-10	0	\\
1.07e-09	0	\\
1.18e-09	0	\\
1.28e-09	0	\\
1.38e-09	0	\\
1.49e-09	0	\\
1.59e-09	0	\\
1.69e-09	0	\\
1.8e-09	0	\\
1.9e-09	0	\\
2.01e-09	0	\\
2.11e-09	0	\\
2.21e-09	0	\\
2.32e-09	0	\\
2.42e-09	0	\\
2.52e-09	0	\\
2.63e-09	0	\\
2.73e-09	0	\\
2.83e-09	0	\\
2.93e-09	0	\\
3.04e-09	0	\\
3.14e-09	0	\\
3.24e-09	0	\\
3.34e-09	0	\\
3.45e-09	0	\\
3.55e-09	0	\\
3.65e-09	0	\\
3.75e-09	0	\\
3.86e-09	0	\\
3.96e-09	0	\\
4.06e-09	0	\\
4.16e-09	0	\\
4.27e-09	0	\\
4.37e-09	0	\\
4.47e-09	0	\\
4.57e-09	0	\\
4.68e-09	0	\\
4.78e-09	0	\\
4.89e-09	0	\\
4.99e-09	0	\\
5e-09	0	\\
};
\addplot [color=darkgray,solid,forget plot]
  table[row sep=crcr]{
0	0	\\
1.1e-10	0	\\
2.2e-10	0	\\
3.3e-10	0	\\
4.4e-10	0	\\
5.4e-10	0	\\
6.5e-10	0	\\
7.5e-10	0	\\
8.6e-10	0	\\
9.6e-10	0	\\
1.07e-09	0	\\
1.18e-09	0	\\
1.28e-09	0	\\
1.38e-09	0	\\
1.49e-09	0	\\
1.59e-09	0	\\
1.69e-09	0	\\
1.8e-09	0	\\
1.9e-09	0	\\
2.01e-09	0	\\
2.11e-09	0	\\
2.21e-09	0	\\
2.32e-09	0	\\
2.42e-09	0	\\
2.52e-09	0	\\
2.63e-09	0	\\
2.73e-09	0	\\
2.83e-09	0	\\
2.93e-09	0	\\
3.04e-09	0	\\
3.14e-09	0	\\
3.24e-09	0	\\
3.34e-09	0	\\
3.45e-09	0	\\
3.55e-09	0	\\
3.65e-09	0	\\
3.75e-09	0	\\
3.86e-09	0	\\
3.96e-09	0	\\
4.06e-09	0	\\
4.16e-09	0	\\
4.27e-09	0	\\
4.37e-09	0	\\
4.47e-09	0	\\
4.57e-09	0	\\
4.68e-09	0	\\
4.78e-09	0	\\
4.89e-09	0	\\
4.99e-09	0	\\
5e-09	0	\\
};
\addplot [color=blue,solid,forget plot]
  table[row sep=crcr]{
0	0	\\
1.1e-10	0	\\
2.2e-10	0	\\
3.3e-10	0	\\
4.4e-10	0	\\
5.4e-10	0	\\
6.5e-10	0	\\
7.5e-10	0	\\
8.6e-10	0	\\
9.6e-10	0	\\
1.07e-09	0	\\
1.18e-09	0	\\
1.28e-09	0	\\
1.38e-09	0	\\
1.49e-09	0	\\
1.59e-09	0	\\
1.69e-09	0	\\
1.8e-09	0	\\
1.9e-09	0	\\
2.01e-09	0	\\
2.11e-09	0	\\
2.21e-09	0	\\
2.32e-09	0	\\
2.42e-09	0	\\
2.52e-09	0	\\
2.63e-09	0	\\
2.73e-09	0	\\
2.83e-09	0	\\
2.93e-09	0	\\
3.04e-09	0	\\
3.14e-09	0	\\
3.24e-09	0	\\
3.34e-09	0	\\
3.45e-09	0	\\
3.55e-09	0	\\
3.65e-09	0	\\
3.75e-09	0	\\
3.86e-09	0	\\
3.96e-09	0	\\
4.06e-09	0	\\
4.16e-09	0	\\
4.27e-09	0	\\
4.37e-09	0	\\
4.47e-09	0	\\
4.57e-09	0	\\
4.68e-09	0	\\
4.78e-09	0	\\
4.89e-09	0	\\
4.99e-09	0	\\
5e-09	0	\\
};
\addplot [color=black!50!green,solid,forget plot]
  table[row sep=crcr]{
0	0	\\
1.1e-10	0	\\
2.2e-10	0	\\
3.3e-10	0	\\
4.4e-10	0	\\
5.4e-10	0	\\
6.5e-10	0	\\
7.5e-10	0	\\
8.6e-10	0	\\
9.6e-10	0	\\
1.07e-09	0	\\
1.18e-09	0	\\
1.28e-09	0	\\
1.38e-09	0	\\
1.49e-09	0	\\
1.59e-09	0	\\
1.69e-09	0	\\
1.8e-09	0	\\
1.9e-09	0	\\
2.01e-09	0	\\
2.11e-09	0	\\
2.21e-09	0	\\
2.32e-09	0	\\
2.42e-09	0	\\
2.52e-09	0	\\
2.63e-09	0	\\
2.73e-09	0	\\
2.83e-09	0	\\
2.93e-09	0	\\
3.04e-09	0	\\
3.14e-09	0	\\
3.24e-09	0	\\
3.34e-09	0	\\
3.45e-09	0	\\
3.55e-09	0	\\
3.65e-09	0	\\
3.75e-09	0	\\
3.86e-09	0	\\
3.96e-09	0	\\
4.06e-09	0	\\
4.16e-09	0	\\
4.27e-09	0	\\
4.37e-09	0	\\
4.47e-09	0	\\
4.57e-09	0	\\
4.68e-09	0	\\
4.78e-09	0	\\
4.89e-09	0	\\
4.99e-09	0	\\
5e-09	0	\\
};
\addplot [color=red,solid,forget plot]
  table[row sep=crcr]{
0	0	\\
1.1e-10	0	\\
2.2e-10	0	\\
3.3e-10	0	\\
4.4e-10	0	\\
5.4e-10	0	\\
6.5e-10	0	\\
7.5e-10	0	\\
8.6e-10	0	\\
9.6e-10	0	\\
1.07e-09	0	\\
1.18e-09	0	\\
1.28e-09	0	\\
1.38e-09	0	\\
1.49e-09	0	\\
1.59e-09	0	\\
1.69e-09	0	\\
1.8e-09	0	\\
1.9e-09	0	\\
2.01e-09	0	\\
2.11e-09	0	\\
2.21e-09	0	\\
2.32e-09	0	\\
2.42e-09	0	\\
2.52e-09	0	\\
2.63e-09	0	\\
2.73e-09	0	\\
2.83e-09	0	\\
2.93e-09	0	\\
3.04e-09	0	\\
3.14e-09	0	\\
3.24e-09	0	\\
3.34e-09	0	\\
3.45e-09	0	\\
3.55e-09	0	\\
3.65e-09	0	\\
3.75e-09	0	\\
3.86e-09	0	\\
3.96e-09	0	\\
4.06e-09	0	\\
4.16e-09	0	\\
4.27e-09	0	\\
4.37e-09	0	\\
4.47e-09	0	\\
4.57e-09	0	\\
4.68e-09	0	\\
4.78e-09	0	\\
4.89e-09	0	\\
4.99e-09	0	\\
5e-09	0	\\
};
\addplot [color=mycolor1,solid,forget plot]
  table[row sep=crcr]{
0	0	\\
1.1e-10	0	\\
2.2e-10	0	\\
3.3e-10	0	\\
4.4e-10	0	\\
5.4e-10	0	\\
6.5e-10	0	\\
7.5e-10	0	\\
8.6e-10	0	\\
9.6e-10	0	\\
1.07e-09	0	\\
1.18e-09	0	\\
1.28e-09	0	\\
1.38e-09	0	\\
1.49e-09	0	\\
1.59e-09	0	\\
1.69e-09	0	\\
1.8e-09	0	\\
1.9e-09	0	\\
2.01e-09	0	\\
2.11e-09	0	\\
2.21e-09	0	\\
2.32e-09	0	\\
2.42e-09	0	\\
2.52e-09	0	\\
2.63e-09	0	\\
2.73e-09	0	\\
2.83e-09	0	\\
2.93e-09	0	\\
3.04e-09	0	\\
3.14e-09	0	\\
3.24e-09	0	\\
3.34e-09	0	\\
3.45e-09	0	\\
3.55e-09	0	\\
3.65e-09	0	\\
3.75e-09	0	\\
3.86e-09	0	\\
3.96e-09	0	\\
4.06e-09	0	\\
4.16e-09	0	\\
4.27e-09	0	\\
4.37e-09	0	\\
4.47e-09	0	\\
4.57e-09	0	\\
4.68e-09	0	\\
4.78e-09	0	\\
4.89e-09	0	\\
4.99e-09	0	\\
5e-09	0	\\
};
\addplot [color=mycolor2,solid,forget plot]
  table[row sep=crcr]{
0	0	\\
1.1e-10	0	\\
2.2e-10	0	\\
3.3e-10	0	\\
4.4e-10	0	\\
5.4e-10	0	\\
6.5e-10	0	\\
7.5e-10	0	\\
8.6e-10	0	\\
9.6e-10	0	\\
1.07e-09	0	\\
1.18e-09	0	\\
1.28e-09	0	\\
1.38e-09	0	\\
1.49e-09	0	\\
1.59e-09	0	\\
1.69e-09	0	\\
1.8e-09	0	\\
1.9e-09	0	\\
2.01e-09	0	\\
2.11e-09	0	\\
2.21e-09	0	\\
2.32e-09	0	\\
2.42e-09	0	\\
2.52e-09	0	\\
2.63e-09	0	\\
2.73e-09	0	\\
2.83e-09	0	\\
2.93e-09	0	\\
3.04e-09	0	\\
3.14e-09	0	\\
3.24e-09	0	\\
3.34e-09	0	\\
3.45e-09	0	\\
3.55e-09	0	\\
3.65e-09	0	\\
3.75e-09	0	\\
3.86e-09	0	\\
3.96e-09	0	\\
4.06e-09	0	\\
4.16e-09	0	\\
4.27e-09	0	\\
4.37e-09	0	\\
4.47e-09	0	\\
4.57e-09	0	\\
4.68e-09	0	\\
4.78e-09	0	\\
4.89e-09	0	\\
4.99e-09	0	\\
5e-09	0	\\
};
\addplot [color=mycolor3,solid,forget plot]
  table[row sep=crcr]{
0	0	\\
1.1e-10	0	\\
2.2e-10	0	\\
3.3e-10	0	\\
4.4e-10	0	\\
5.4e-10	0	\\
6.5e-10	0	\\
7.5e-10	0	\\
8.6e-10	0	\\
9.6e-10	0	\\
1.07e-09	0	\\
1.18e-09	0	\\
1.28e-09	0	\\
1.38e-09	0	\\
1.49e-09	0	\\
1.59e-09	0	\\
1.69e-09	0	\\
1.8e-09	0	\\
1.9e-09	0	\\
2.01e-09	0	\\
2.11e-09	0	\\
2.21e-09	0	\\
2.32e-09	0	\\
2.42e-09	0	\\
2.52e-09	0	\\
2.63e-09	0	\\
2.73e-09	0	\\
2.83e-09	0	\\
2.93e-09	0	\\
3.04e-09	0	\\
3.14e-09	0	\\
3.24e-09	0	\\
3.34e-09	0	\\
3.45e-09	0	\\
3.55e-09	0	\\
3.65e-09	0	\\
3.75e-09	0	\\
3.86e-09	0	\\
3.96e-09	0	\\
4.06e-09	0	\\
4.16e-09	0	\\
4.27e-09	0	\\
4.37e-09	0	\\
4.47e-09	0	\\
4.57e-09	0	\\
4.68e-09	0	\\
4.78e-09	0	\\
4.89e-09	0	\\
4.99e-09	0	\\
5e-09	0	\\
};
\addplot [color=darkgray,solid,forget plot]
  table[row sep=crcr]{
0	0	\\
1.1e-10	0	\\
2.2e-10	0	\\
3.3e-10	0	\\
4.4e-10	0	\\
5.4e-10	0	\\
6.5e-10	0	\\
7.5e-10	0	\\
8.6e-10	0	\\
9.6e-10	0	\\
1.07e-09	0	\\
1.18e-09	0	\\
1.28e-09	0	\\
1.38e-09	0	\\
1.49e-09	0	\\
1.59e-09	0	\\
1.69e-09	0	\\
1.8e-09	0	\\
1.9e-09	0	\\
2.01e-09	0	\\
2.11e-09	0	\\
2.21e-09	0	\\
2.32e-09	0	\\
2.42e-09	0	\\
2.52e-09	0	\\
2.63e-09	0	\\
2.73e-09	0	\\
2.83e-09	0	\\
2.93e-09	0	\\
3.04e-09	0	\\
3.14e-09	0	\\
3.24e-09	0	\\
3.34e-09	0	\\
3.45e-09	0	\\
3.55e-09	0	\\
3.65e-09	0	\\
3.75e-09	0	\\
3.86e-09	0	\\
3.96e-09	0	\\
4.06e-09	0	\\
4.16e-09	0	\\
4.27e-09	0	\\
4.37e-09	0	\\
4.47e-09	0	\\
4.57e-09	0	\\
4.68e-09	0	\\
4.78e-09	0	\\
4.89e-09	0	\\
4.99e-09	0	\\
5e-09	0	\\
};
\addplot [color=blue,solid,forget plot]
  table[row sep=crcr]{
0	0	\\
1.1e-10	0	\\
2.2e-10	0	\\
3.3e-10	0	\\
4.4e-10	0	\\
5.4e-10	0	\\
6.5e-10	0	\\
7.5e-10	0	\\
8.6e-10	0	\\
9.6e-10	0	\\
1.07e-09	0	\\
1.18e-09	0	\\
1.28e-09	0	\\
1.38e-09	0	\\
1.49e-09	0	\\
1.59e-09	0	\\
1.69e-09	0	\\
1.8e-09	0	\\
1.9e-09	0	\\
2.01e-09	0	\\
2.11e-09	0	\\
2.21e-09	0	\\
2.32e-09	0	\\
2.42e-09	0	\\
2.52e-09	0	\\
2.63e-09	0	\\
2.73e-09	0	\\
2.83e-09	0	\\
2.93e-09	0	\\
3.04e-09	0	\\
3.14e-09	0	\\
3.24e-09	0	\\
3.34e-09	0	\\
3.45e-09	0	\\
3.55e-09	0	\\
3.65e-09	0	\\
3.75e-09	0	\\
3.86e-09	0	\\
3.96e-09	0	\\
4.06e-09	0	\\
4.16e-09	0	\\
4.27e-09	0	\\
4.37e-09	0	\\
4.47e-09	0	\\
4.57e-09	0	\\
4.68e-09	0	\\
4.78e-09	0	\\
4.89e-09	0	\\
4.99e-09	0	\\
5e-09	0	\\
};
\addplot [color=black!50!green,solid,forget plot]
  table[row sep=crcr]{
0	0	\\
1.1e-10	0	\\
2.2e-10	0	\\
3.3e-10	0	\\
4.4e-10	0	\\
5.4e-10	0	\\
6.5e-10	0	\\
7.5e-10	0	\\
8.6e-10	0	\\
9.6e-10	0	\\
1.07e-09	0	\\
1.18e-09	0	\\
1.28e-09	0	\\
1.38e-09	0	\\
1.49e-09	0	\\
1.59e-09	0	\\
1.69e-09	0	\\
1.8e-09	0	\\
1.9e-09	0	\\
2.01e-09	0	\\
2.11e-09	0	\\
2.21e-09	0	\\
2.32e-09	0	\\
2.42e-09	0	\\
2.52e-09	0	\\
2.63e-09	0	\\
2.73e-09	0	\\
2.83e-09	0	\\
2.93e-09	0	\\
3.04e-09	0	\\
3.14e-09	0	\\
3.24e-09	0	\\
3.34e-09	0	\\
3.45e-09	0	\\
3.55e-09	0	\\
3.65e-09	0	\\
3.75e-09	0	\\
3.86e-09	0	\\
3.96e-09	0	\\
4.06e-09	0	\\
4.16e-09	0	\\
4.27e-09	0	\\
4.37e-09	0	\\
4.47e-09	0	\\
4.57e-09	0	\\
4.68e-09	0	\\
4.78e-09	0	\\
4.89e-09	0	\\
4.99e-09	0	\\
5e-09	0	\\
};
\addplot [color=red,solid,forget plot]
  table[row sep=crcr]{
0	0	\\
1.1e-10	0	\\
2.2e-10	0	\\
3.3e-10	0	\\
4.4e-10	0	\\
5.4e-10	0	\\
6.5e-10	0	\\
7.5e-10	0	\\
8.6e-10	0	\\
9.6e-10	0	\\
1.07e-09	0	\\
1.18e-09	0	\\
1.28e-09	0	\\
1.38e-09	0	\\
1.49e-09	0	\\
1.59e-09	0	\\
1.69e-09	0	\\
1.8e-09	0	\\
1.9e-09	0	\\
2.01e-09	0	\\
2.11e-09	0	\\
2.21e-09	0	\\
2.32e-09	0	\\
2.42e-09	0	\\
2.52e-09	0	\\
2.63e-09	0	\\
2.73e-09	0	\\
2.83e-09	0	\\
2.93e-09	0	\\
3.04e-09	0	\\
3.14e-09	0	\\
3.24e-09	0	\\
3.34e-09	0	\\
3.45e-09	0	\\
3.55e-09	0	\\
3.65e-09	0	\\
3.75e-09	0	\\
3.86e-09	0	\\
3.96e-09	0	\\
4.06e-09	0	\\
4.16e-09	0	\\
4.27e-09	0	\\
4.37e-09	0	\\
4.47e-09	0	\\
4.57e-09	0	\\
4.68e-09	0	\\
4.78e-09	0	\\
4.89e-09	0	\\
4.99e-09	0	\\
5e-09	0	\\
};
\addplot [color=mycolor1,solid,forget plot]
  table[row sep=crcr]{
0	0	\\
1.1e-10	0	\\
2.2e-10	0	\\
3.3e-10	0	\\
4.4e-10	0	\\
5.4e-10	0	\\
6.5e-10	0	\\
7.5e-10	0	\\
8.6e-10	0	\\
9.6e-10	0	\\
1.07e-09	0	\\
1.18e-09	0	\\
1.28e-09	0	\\
1.38e-09	0	\\
1.49e-09	0	\\
1.59e-09	0	\\
1.69e-09	0	\\
1.8e-09	0	\\
1.9e-09	0	\\
2.01e-09	0	\\
2.11e-09	0	\\
2.21e-09	0	\\
2.32e-09	0	\\
2.42e-09	0	\\
2.52e-09	0	\\
2.63e-09	0	\\
2.73e-09	0	\\
2.83e-09	0	\\
2.93e-09	0	\\
3.04e-09	0	\\
3.14e-09	0	\\
3.24e-09	0	\\
3.34e-09	0	\\
3.45e-09	0	\\
3.55e-09	0	\\
3.65e-09	0	\\
3.75e-09	0	\\
3.86e-09	0	\\
3.96e-09	0	\\
4.06e-09	0	\\
4.16e-09	0	\\
4.27e-09	0	\\
4.37e-09	0	\\
4.47e-09	0	\\
4.57e-09	0	\\
4.68e-09	0	\\
4.78e-09	0	\\
4.89e-09	0	\\
4.99e-09	0	\\
5e-09	0	\\
};
\addplot [color=mycolor2,solid,forget plot]
  table[row sep=crcr]{
0	0	\\
1.1e-10	0	\\
2.2e-10	0	\\
3.3e-10	0	\\
4.4e-10	0	\\
5.4e-10	0	\\
6.5e-10	0	\\
7.5e-10	0	\\
8.6e-10	0	\\
9.6e-10	0	\\
1.07e-09	0	\\
1.18e-09	0	\\
1.28e-09	0	\\
1.38e-09	0	\\
1.49e-09	0	\\
1.59e-09	0	\\
1.69e-09	0	\\
1.8e-09	0	\\
1.9e-09	0	\\
2.01e-09	0	\\
2.11e-09	0	\\
2.21e-09	0	\\
2.32e-09	0	\\
2.42e-09	0	\\
2.52e-09	0	\\
2.63e-09	0	\\
2.73e-09	0	\\
2.83e-09	0	\\
2.93e-09	0	\\
3.04e-09	0	\\
3.14e-09	0	\\
3.24e-09	0	\\
3.34e-09	0	\\
3.45e-09	0	\\
3.55e-09	0	\\
3.65e-09	0	\\
3.75e-09	0	\\
3.86e-09	0	\\
3.96e-09	0	\\
4.06e-09	0	\\
4.16e-09	0	\\
4.27e-09	0	\\
4.37e-09	0	\\
4.47e-09	0	\\
4.57e-09	0	\\
4.68e-09	0	\\
4.78e-09	0	\\
4.89e-09	0	\\
4.99e-09	0	\\
5e-09	0	\\
};
\addplot [color=mycolor3,solid,forget plot]
  table[row sep=crcr]{
0	0	\\
1.1e-10	0	\\
2.2e-10	0	\\
3.3e-10	0	\\
4.4e-10	0	\\
5.4e-10	0	\\
6.5e-10	0	\\
7.5e-10	0	\\
8.6e-10	0	\\
9.6e-10	0	\\
1.07e-09	0	\\
1.18e-09	0	\\
1.28e-09	0	\\
1.38e-09	0	\\
1.49e-09	0	\\
1.59e-09	0	\\
1.69e-09	0	\\
1.8e-09	0	\\
1.9e-09	0	\\
2.01e-09	0	\\
2.11e-09	0	\\
2.21e-09	0	\\
2.32e-09	0	\\
2.42e-09	0	\\
2.52e-09	0	\\
2.63e-09	0	\\
2.73e-09	0	\\
2.83e-09	0	\\
2.93e-09	0	\\
3.04e-09	0	\\
3.14e-09	0	\\
3.24e-09	0	\\
3.34e-09	0	\\
3.45e-09	0	\\
3.55e-09	0	\\
3.65e-09	0	\\
3.75e-09	0	\\
3.86e-09	0	\\
3.96e-09	0	\\
4.06e-09	0	\\
4.16e-09	0	\\
4.27e-09	0	\\
4.37e-09	0	\\
4.47e-09	0	\\
4.57e-09	0	\\
4.68e-09	0	\\
4.78e-09	0	\\
4.89e-09	0	\\
4.99e-09	0	\\
5e-09	0	\\
};
\addplot [color=darkgray,solid,forget plot]
  table[row sep=crcr]{
0	0	\\
1.1e-10	0	\\
2.2e-10	0	\\
3.3e-10	0	\\
4.4e-10	0	\\
5.4e-10	0	\\
6.5e-10	0	\\
7.5e-10	0	\\
8.6e-10	0	\\
9.6e-10	0	\\
1.07e-09	0	\\
1.18e-09	0	\\
1.28e-09	0	\\
1.38e-09	0	\\
1.49e-09	0	\\
1.59e-09	0	\\
1.69e-09	0	\\
1.8e-09	0	\\
1.9e-09	0	\\
2.01e-09	0	\\
2.11e-09	0	\\
2.21e-09	0	\\
2.32e-09	0	\\
2.42e-09	0	\\
2.52e-09	0	\\
2.63e-09	0	\\
2.73e-09	0	\\
2.83e-09	0	\\
2.93e-09	0	\\
3.04e-09	0	\\
3.14e-09	0	\\
3.24e-09	0	\\
3.34e-09	0	\\
3.45e-09	0	\\
3.55e-09	0	\\
3.65e-09	0	\\
3.75e-09	0	\\
3.86e-09	0	\\
3.96e-09	0	\\
4.06e-09	0	\\
4.16e-09	0	\\
4.27e-09	0	\\
4.37e-09	0	\\
4.47e-09	0	\\
4.57e-09	0	\\
4.68e-09	0	\\
4.78e-09	0	\\
4.89e-09	0	\\
4.99e-09	0	\\
5e-09	0	\\
};
\addplot [color=blue,solid,forget plot]
  table[row sep=crcr]{
0	0	\\
1.1e-10	0	\\
2.2e-10	0	\\
3.3e-10	0	\\
4.4e-10	0	\\
5.4e-10	0	\\
6.5e-10	0	\\
7.5e-10	0	\\
8.6e-10	0	\\
9.6e-10	0	\\
1.07e-09	0	\\
1.18e-09	0	\\
1.28e-09	0	\\
1.38e-09	0	\\
1.49e-09	0	\\
1.59e-09	0	\\
1.69e-09	0	\\
1.8e-09	0	\\
1.9e-09	0	\\
2.01e-09	0	\\
2.11e-09	0	\\
2.21e-09	0	\\
2.32e-09	0	\\
2.42e-09	0	\\
2.52e-09	0	\\
2.63e-09	0	\\
2.73e-09	0	\\
2.83e-09	0	\\
2.93e-09	0	\\
3.04e-09	0	\\
3.14e-09	0	\\
3.24e-09	0	\\
3.34e-09	0	\\
3.45e-09	0	\\
3.55e-09	0	\\
3.65e-09	0	\\
3.75e-09	0	\\
3.86e-09	0	\\
3.96e-09	0	\\
4.06e-09	0	\\
4.16e-09	0	\\
4.27e-09	0	\\
4.37e-09	0	\\
4.47e-09	0	\\
4.57e-09	0	\\
4.68e-09	0	\\
4.78e-09	0	\\
4.89e-09	0	\\
4.99e-09	0	\\
5e-09	0	\\
};
\addplot [color=black!50!green,solid,forget plot]
  table[row sep=crcr]{
0	0	\\
1.1e-10	0	\\
2.2e-10	0	\\
3.3e-10	0	\\
4.4e-10	0	\\
5.4e-10	0	\\
6.5e-10	0	\\
7.5e-10	0	\\
8.6e-10	0	\\
9.6e-10	0	\\
1.07e-09	0	\\
1.18e-09	0	\\
1.28e-09	0	\\
1.38e-09	0	\\
1.49e-09	0	\\
1.59e-09	0	\\
1.69e-09	0	\\
1.8e-09	0	\\
1.9e-09	0	\\
2.01e-09	0	\\
2.11e-09	0	\\
2.21e-09	0	\\
2.32e-09	0	\\
2.42e-09	0	\\
2.52e-09	0	\\
2.63e-09	0	\\
2.73e-09	0	\\
2.83e-09	0	\\
2.93e-09	0	\\
3.04e-09	0	\\
3.14e-09	0	\\
3.24e-09	0	\\
3.34e-09	0	\\
3.45e-09	0	\\
3.55e-09	0	\\
3.65e-09	0	\\
3.75e-09	0	\\
3.86e-09	0	\\
3.96e-09	0	\\
4.06e-09	0	\\
4.16e-09	0	\\
4.27e-09	0	\\
4.37e-09	0	\\
4.47e-09	0	\\
4.57e-09	0	\\
4.68e-09	0	\\
4.78e-09	0	\\
4.89e-09	0	\\
4.99e-09	0	\\
5e-09	0	\\
};
\addplot [color=red,solid,forget plot]
  table[row sep=crcr]{
0	0	\\
1.1e-10	0	\\
2.2e-10	0	\\
3.3e-10	0	\\
4.4e-10	0	\\
5.4e-10	0	\\
6.5e-10	0	\\
7.5e-10	0	\\
8.6e-10	0	\\
9.6e-10	0	\\
1.07e-09	0	\\
1.18e-09	0	\\
1.28e-09	0	\\
1.38e-09	0	\\
1.49e-09	0	\\
1.59e-09	0	\\
1.69e-09	0	\\
1.8e-09	0	\\
1.9e-09	0	\\
2.01e-09	0	\\
2.11e-09	0	\\
2.21e-09	0	\\
2.32e-09	0	\\
2.42e-09	0	\\
2.52e-09	0	\\
2.63e-09	0	\\
2.73e-09	0	\\
2.83e-09	0	\\
2.93e-09	0	\\
3.04e-09	0	\\
3.14e-09	0	\\
3.24e-09	0	\\
3.34e-09	0	\\
3.45e-09	0	\\
3.55e-09	0	\\
3.65e-09	0	\\
3.75e-09	0	\\
3.86e-09	0	\\
3.96e-09	0	\\
4.06e-09	0	\\
4.16e-09	0	\\
4.27e-09	0	\\
4.37e-09	0	\\
4.47e-09	0	\\
4.57e-09	0	\\
4.68e-09	0	\\
4.78e-09	0	\\
4.89e-09	0	\\
4.99e-09	0	\\
5e-09	0	\\
};
\addplot [color=mycolor1,solid,forget plot]
  table[row sep=crcr]{
0	0	\\
1.1e-10	0	\\
2.2e-10	0	\\
3.3e-10	0	\\
4.4e-10	0	\\
5.4e-10	0	\\
6.5e-10	0	\\
7.5e-10	0	\\
8.6e-10	0	\\
9.6e-10	0	\\
1.07e-09	0	\\
1.18e-09	0	\\
1.28e-09	0	\\
1.38e-09	0	\\
1.49e-09	0	\\
1.59e-09	0	\\
1.69e-09	0	\\
1.8e-09	0	\\
1.9e-09	0	\\
2.01e-09	0	\\
2.11e-09	0	\\
2.21e-09	0	\\
2.32e-09	0	\\
2.42e-09	0	\\
2.52e-09	0	\\
2.63e-09	0	\\
2.73e-09	0	\\
2.83e-09	0	\\
2.93e-09	0	\\
3.04e-09	0	\\
3.14e-09	0	\\
3.24e-09	0	\\
3.34e-09	0	\\
3.45e-09	0	\\
3.55e-09	0	\\
3.65e-09	0	\\
3.75e-09	0	\\
3.86e-09	0	\\
3.96e-09	0	\\
4.06e-09	0	\\
4.16e-09	0	\\
4.27e-09	0	\\
4.37e-09	0	\\
4.47e-09	0	\\
4.57e-09	0	\\
4.68e-09	0	\\
4.78e-09	0	\\
4.89e-09	0	\\
4.99e-09	0	\\
5e-09	0	\\
};
\addplot [color=mycolor2,solid,forget plot]
  table[row sep=crcr]{
0	0	\\
1.1e-10	0	\\
2.2e-10	0	\\
3.3e-10	0	\\
4.4e-10	0	\\
5.4e-10	0	\\
6.5e-10	0	\\
7.5e-10	0	\\
8.6e-10	0	\\
9.6e-10	0	\\
1.07e-09	0	\\
1.18e-09	0	\\
1.28e-09	0	\\
1.38e-09	0	\\
1.49e-09	0	\\
1.59e-09	0	\\
1.69e-09	0	\\
1.8e-09	0	\\
1.9e-09	0	\\
2.01e-09	0	\\
2.11e-09	0	\\
2.21e-09	0	\\
2.32e-09	0	\\
2.42e-09	0	\\
2.52e-09	0	\\
2.63e-09	0	\\
2.73e-09	0	\\
2.83e-09	0	\\
2.93e-09	0	\\
3.04e-09	0	\\
3.14e-09	0	\\
3.24e-09	0	\\
3.34e-09	0	\\
3.45e-09	0	\\
3.55e-09	0	\\
3.65e-09	0	\\
3.75e-09	0	\\
3.86e-09	0	\\
3.96e-09	0	\\
4.06e-09	0	\\
4.16e-09	0	\\
4.27e-09	0	\\
4.37e-09	0	\\
4.47e-09	0	\\
4.57e-09	0	\\
4.68e-09	0	\\
4.78e-09	0	\\
4.89e-09	0	\\
4.99e-09	0	\\
5e-09	0	\\
};
\addplot [color=mycolor3,solid,forget plot]
  table[row sep=crcr]{
0	0	\\
1.1e-10	0	\\
2.2e-10	0	\\
3.3e-10	0	\\
4.4e-10	0	\\
5.4e-10	0	\\
6.5e-10	0	\\
7.5e-10	0	\\
8.6e-10	0	\\
9.6e-10	0	\\
1.07e-09	0	\\
1.18e-09	0	\\
1.28e-09	0	\\
1.38e-09	0	\\
1.49e-09	0	\\
1.59e-09	0	\\
1.69e-09	0	\\
1.8e-09	0	\\
1.9e-09	0	\\
2.01e-09	0	\\
2.11e-09	0	\\
2.21e-09	0	\\
2.32e-09	0	\\
2.42e-09	0	\\
2.52e-09	0	\\
2.63e-09	0	\\
2.73e-09	0	\\
2.83e-09	0	\\
2.93e-09	0	\\
3.04e-09	0	\\
3.14e-09	0	\\
3.24e-09	0	\\
3.34e-09	0	\\
3.45e-09	0	\\
3.55e-09	0	\\
3.65e-09	0	\\
3.75e-09	0	\\
3.86e-09	0	\\
3.96e-09	0	\\
4.06e-09	0	\\
4.16e-09	0	\\
4.27e-09	0	\\
4.37e-09	0	\\
4.47e-09	0	\\
4.57e-09	0	\\
4.68e-09	0	\\
4.78e-09	0	\\
4.89e-09	0	\\
4.99e-09	0	\\
5e-09	0	\\
};
\addplot [color=darkgray,solid,forget plot]
  table[row sep=crcr]{
0	0	\\
1.1e-10	0	\\
2.2e-10	0	\\
3.3e-10	0	\\
4.4e-10	0	\\
5.4e-10	0	\\
6.5e-10	0	\\
7.5e-10	0	\\
8.6e-10	0	\\
9.6e-10	0	\\
1.07e-09	0	\\
1.18e-09	0	\\
1.28e-09	0	\\
1.38e-09	0	\\
1.49e-09	0	\\
1.59e-09	0	\\
1.69e-09	0	\\
1.8e-09	0	\\
1.9e-09	0	\\
2.01e-09	0	\\
2.11e-09	0	\\
2.21e-09	0	\\
2.32e-09	0	\\
2.42e-09	0	\\
2.52e-09	0	\\
2.63e-09	0	\\
2.73e-09	0	\\
2.83e-09	0	\\
2.93e-09	0	\\
3.04e-09	0	\\
3.14e-09	0	\\
3.24e-09	0	\\
3.34e-09	0	\\
3.45e-09	0	\\
3.55e-09	0	\\
3.65e-09	0	\\
3.75e-09	0	\\
3.86e-09	0	\\
3.96e-09	0	\\
4.06e-09	0	\\
4.16e-09	0	\\
4.27e-09	0	\\
4.37e-09	0	\\
4.47e-09	0	\\
4.57e-09	0	\\
4.68e-09	0	\\
4.78e-09	0	\\
4.89e-09	0	\\
4.99e-09	0	\\
5e-09	0	\\
};
\addplot [color=blue,solid,forget plot]
  table[row sep=crcr]{
0	0	\\
1.1e-10	0	\\
2.2e-10	0	\\
3.3e-10	0	\\
4.4e-10	0	\\
5.4e-10	0	\\
6.5e-10	0	\\
7.5e-10	0	\\
8.6e-10	0	\\
9.6e-10	0	\\
1.07e-09	0	\\
1.18e-09	0	\\
1.28e-09	0	\\
1.38e-09	0	\\
1.49e-09	0	\\
1.59e-09	0	\\
1.69e-09	0	\\
1.8e-09	0	\\
1.9e-09	0	\\
2.01e-09	0	\\
2.11e-09	0	\\
2.21e-09	0	\\
2.32e-09	0	\\
2.42e-09	0	\\
2.52e-09	0	\\
2.63e-09	0	\\
2.73e-09	0	\\
2.83e-09	0	\\
2.93e-09	0	\\
3.04e-09	0	\\
3.14e-09	0	\\
3.24e-09	0	\\
3.34e-09	0	\\
3.45e-09	0	\\
3.55e-09	0	\\
3.65e-09	0	\\
3.75e-09	0	\\
3.86e-09	0	\\
3.96e-09	0	\\
4.06e-09	0	\\
4.16e-09	0	\\
4.27e-09	0	\\
4.37e-09	0	\\
4.47e-09	0	\\
4.57e-09	0	\\
4.68e-09	0	\\
4.78e-09	0	\\
4.89e-09	0	\\
4.99e-09	0	\\
5e-09	0	\\
};
\addplot [color=black!50!green,solid,forget plot]
  table[row sep=crcr]{
0	0	\\
1.1e-10	0	\\
2.2e-10	0	\\
3.3e-10	0	\\
4.4e-10	0	\\
5.4e-10	0	\\
6.5e-10	0	\\
7.5e-10	0	\\
8.6e-10	0	\\
9.6e-10	0	\\
1.07e-09	0	\\
1.18e-09	0	\\
1.28e-09	0	\\
1.38e-09	0	\\
1.49e-09	0	\\
1.59e-09	0	\\
1.69e-09	0	\\
1.8e-09	0	\\
1.9e-09	0	\\
2.01e-09	0	\\
2.11e-09	0	\\
2.21e-09	0	\\
2.32e-09	0	\\
2.42e-09	0	\\
2.52e-09	0	\\
2.63e-09	0	\\
2.73e-09	0	\\
2.83e-09	0	\\
2.93e-09	0	\\
3.04e-09	0	\\
3.14e-09	0	\\
3.24e-09	0	\\
3.34e-09	0	\\
3.45e-09	0	\\
3.55e-09	0	\\
3.65e-09	0	\\
3.75e-09	0	\\
3.86e-09	0	\\
3.96e-09	0	\\
4.06e-09	0	\\
4.16e-09	0	\\
4.27e-09	0	\\
4.37e-09	0	\\
4.47e-09	0	\\
4.57e-09	0	\\
4.68e-09	0	\\
4.78e-09	0	\\
4.89e-09	0	\\
4.99e-09	0	\\
5e-09	0	\\
};
\addplot [color=red,solid,forget plot]
  table[row sep=crcr]{
0	0	\\
1.1e-10	0	\\
2.2e-10	0	\\
3.3e-10	0	\\
4.4e-10	0	\\
5.4e-10	0	\\
6.5e-10	0	\\
7.5e-10	0	\\
8.6e-10	0	\\
9.6e-10	0	\\
1.07e-09	0	\\
1.18e-09	0	\\
1.28e-09	0	\\
1.38e-09	0	\\
1.49e-09	0	\\
1.59e-09	0	\\
1.69e-09	0	\\
1.8e-09	0	\\
1.9e-09	0	\\
2.01e-09	0	\\
2.11e-09	0	\\
2.21e-09	0	\\
2.32e-09	0	\\
2.42e-09	0	\\
2.52e-09	0	\\
2.63e-09	0	\\
2.73e-09	0	\\
2.83e-09	0	\\
2.93e-09	0	\\
3.04e-09	0	\\
3.14e-09	0	\\
3.24e-09	0	\\
3.34e-09	0	\\
3.45e-09	0	\\
3.55e-09	0	\\
3.65e-09	0	\\
3.75e-09	0	\\
3.86e-09	0	\\
3.96e-09	0	\\
4.06e-09	0	\\
4.16e-09	0	\\
4.27e-09	0	\\
4.37e-09	0	\\
4.47e-09	0	\\
4.57e-09	0	\\
4.68e-09	0	\\
4.78e-09	0	\\
4.89e-09	0	\\
4.99e-09	0	\\
5e-09	0	\\
};
\addplot [color=mycolor1,solid,forget plot]
  table[row sep=crcr]{
0	0	\\
1.1e-10	0	\\
2.2e-10	0	\\
3.3e-10	0	\\
4.4e-10	0	\\
5.4e-10	0	\\
6.5e-10	0	\\
7.5e-10	0	\\
8.6e-10	0	\\
9.6e-10	0	\\
1.07e-09	0	\\
1.18e-09	0	\\
1.28e-09	0	\\
1.38e-09	0	\\
1.49e-09	0	\\
1.59e-09	0	\\
1.69e-09	0	\\
1.8e-09	0	\\
1.9e-09	0	\\
2.01e-09	0	\\
2.11e-09	0	\\
2.21e-09	0	\\
2.32e-09	0	\\
2.42e-09	0	\\
2.52e-09	0	\\
2.63e-09	0	\\
2.73e-09	0	\\
2.83e-09	0	\\
2.93e-09	0	\\
3.04e-09	0	\\
3.14e-09	0	\\
3.24e-09	0	\\
3.34e-09	0	\\
3.45e-09	0	\\
3.55e-09	0	\\
3.65e-09	0	\\
3.75e-09	0	\\
3.86e-09	0	\\
3.96e-09	0	\\
4.06e-09	0	\\
4.16e-09	0	\\
4.27e-09	0	\\
4.37e-09	0	\\
4.47e-09	0	\\
4.57e-09	0	\\
4.68e-09	0	\\
4.78e-09	0	\\
4.89e-09	0	\\
4.99e-09	0	\\
5e-09	0	\\
};
\addplot [color=mycolor2,solid,forget plot]
  table[row sep=crcr]{
0	0	\\
1.1e-10	0	\\
2.2e-10	0	\\
3.3e-10	0	\\
4.4e-10	0	\\
5.4e-10	0	\\
6.5e-10	0	\\
7.5e-10	0	\\
8.6e-10	0	\\
9.6e-10	0	\\
1.07e-09	0	\\
1.18e-09	0	\\
1.28e-09	0	\\
1.38e-09	0	\\
1.49e-09	0	\\
1.59e-09	0	\\
1.69e-09	0	\\
1.8e-09	0	\\
1.9e-09	0	\\
2.01e-09	0	\\
2.11e-09	0	\\
2.21e-09	0	\\
2.32e-09	0	\\
2.42e-09	0	\\
2.52e-09	0	\\
2.63e-09	0	\\
2.73e-09	0	\\
2.83e-09	0	\\
2.93e-09	0	\\
3.04e-09	0	\\
3.14e-09	0	\\
3.24e-09	0	\\
3.34e-09	0	\\
3.45e-09	0	\\
3.55e-09	0	\\
3.65e-09	0	\\
3.75e-09	0	\\
3.86e-09	0	\\
3.96e-09	0	\\
4.06e-09	0	\\
4.16e-09	0	\\
4.27e-09	0	\\
4.37e-09	0	\\
4.47e-09	0	\\
4.57e-09	0	\\
4.68e-09	0	\\
4.78e-09	0	\\
4.89e-09	0	\\
4.99e-09	0	\\
5e-09	0	\\
};
\addplot [color=mycolor3,solid,forget plot]
  table[row sep=crcr]{
0	0	\\
1.1e-10	0	\\
2.2e-10	0	\\
3.3e-10	0	\\
4.4e-10	0	\\
5.4e-10	0	\\
6.5e-10	0	\\
7.5e-10	0	\\
8.6e-10	0	\\
9.6e-10	0	\\
1.07e-09	0	\\
1.18e-09	0	\\
1.28e-09	0	\\
1.38e-09	0	\\
1.49e-09	0	\\
1.59e-09	0	\\
1.69e-09	0	\\
1.8e-09	0	\\
1.9e-09	0	\\
2.01e-09	0	\\
2.11e-09	0	\\
2.21e-09	0	\\
2.32e-09	0	\\
2.42e-09	0	\\
2.52e-09	0	\\
2.63e-09	0	\\
2.73e-09	0	\\
2.83e-09	0	\\
2.93e-09	0	\\
3.04e-09	0	\\
3.14e-09	0	\\
3.24e-09	0	\\
3.34e-09	0	\\
3.45e-09	0	\\
3.55e-09	0	\\
3.65e-09	0	\\
3.75e-09	0	\\
3.86e-09	0	\\
3.96e-09	0	\\
4.06e-09	0	\\
4.16e-09	0	\\
4.27e-09	0	\\
4.37e-09	0	\\
4.47e-09	0	\\
4.57e-09	0	\\
4.68e-09	0	\\
4.78e-09	0	\\
4.89e-09	0	\\
4.99e-09	0	\\
5e-09	0	\\
};
\addplot [color=darkgray,solid,forget plot]
  table[row sep=crcr]{
0	0	\\
1.1e-10	0	\\
2.2e-10	0	\\
3.3e-10	0	\\
4.4e-10	0	\\
5.4e-10	0	\\
6.5e-10	0	\\
7.5e-10	0	\\
8.6e-10	0	\\
9.6e-10	0	\\
1.07e-09	0	\\
1.18e-09	0	\\
1.28e-09	0	\\
1.38e-09	0	\\
1.49e-09	0	\\
1.59e-09	0	\\
1.69e-09	0	\\
1.8e-09	0	\\
1.9e-09	0	\\
2.01e-09	0	\\
2.11e-09	0	\\
2.21e-09	0	\\
2.32e-09	0	\\
2.42e-09	0	\\
2.52e-09	0	\\
2.63e-09	0	\\
2.73e-09	0	\\
2.83e-09	0	\\
2.93e-09	0	\\
3.04e-09	0	\\
3.14e-09	0	\\
3.24e-09	0	\\
3.34e-09	0	\\
3.45e-09	0	\\
3.55e-09	0	\\
3.65e-09	0	\\
3.75e-09	0	\\
3.86e-09	0	\\
3.96e-09	0	\\
4.06e-09	0	\\
4.16e-09	0	\\
4.27e-09	0	\\
4.37e-09	0	\\
4.47e-09	0	\\
4.57e-09	0	\\
4.68e-09	0	\\
4.78e-09	0	\\
4.89e-09	0	\\
4.99e-09	0	\\
5e-09	0	\\
};
\addplot [color=blue,solid,forget plot]
  table[row sep=crcr]{
0	0	\\
1.1e-10	0	\\
2.2e-10	0	\\
3.3e-10	0	\\
4.4e-10	0	\\
5.4e-10	0	\\
6.5e-10	0	\\
7.5e-10	0	\\
8.6e-10	0	\\
9.6e-10	0	\\
1.07e-09	0	\\
1.18e-09	0	\\
1.28e-09	0	\\
1.38e-09	0	\\
1.49e-09	0	\\
1.59e-09	0	\\
1.69e-09	0	\\
1.8e-09	0	\\
1.9e-09	0	\\
2.01e-09	0	\\
2.11e-09	0	\\
2.21e-09	0	\\
2.32e-09	0	\\
2.42e-09	0	\\
2.52e-09	0	\\
2.63e-09	0	\\
2.73e-09	0	\\
2.83e-09	0	\\
2.93e-09	0	\\
3.04e-09	0	\\
3.14e-09	0	\\
3.24e-09	0	\\
3.34e-09	0	\\
3.45e-09	0	\\
3.55e-09	0	\\
3.65e-09	0	\\
3.75e-09	0	\\
3.86e-09	0	\\
3.96e-09	0	\\
4.06e-09	0	\\
4.16e-09	0	\\
4.27e-09	0	\\
4.37e-09	0	\\
4.47e-09	0	\\
4.57e-09	0	\\
4.68e-09	0	\\
4.78e-09	0	\\
4.89e-09	0	\\
4.99e-09	0	\\
5e-09	0	\\
};
\addplot [color=black!50!green,solid,forget plot]
  table[row sep=crcr]{
0	0	\\
1.1e-10	0	\\
2.2e-10	0	\\
3.3e-10	0	\\
4.4e-10	0	\\
5.4e-10	0	\\
6.5e-10	0	\\
7.5e-10	0	\\
8.6e-10	0	\\
9.6e-10	0	\\
1.07e-09	0	\\
1.18e-09	0	\\
1.28e-09	0	\\
1.38e-09	0	\\
1.49e-09	0	\\
1.59e-09	0	\\
1.69e-09	0	\\
1.8e-09	0	\\
1.9e-09	0	\\
2.01e-09	0	\\
2.11e-09	0	\\
2.21e-09	0	\\
2.32e-09	0	\\
2.42e-09	0	\\
2.52e-09	0	\\
2.63e-09	0	\\
2.73e-09	0	\\
2.83e-09	0	\\
2.93e-09	0	\\
3.04e-09	0	\\
3.14e-09	0	\\
3.24e-09	0	\\
3.34e-09	0	\\
3.45e-09	0	\\
3.55e-09	0	\\
3.65e-09	0	\\
3.75e-09	0	\\
3.86e-09	0	\\
3.96e-09	0	\\
4.06e-09	0	\\
4.16e-09	0	\\
4.27e-09	0	\\
4.37e-09	0	\\
4.47e-09	0	\\
4.57e-09	0	\\
4.68e-09	0	\\
4.78e-09	0	\\
4.89e-09	0	\\
4.99e-09	0	\\
5e-09	0	\\
};
\addplot [color=red,solid,forget plot]
  table[row sep=crcr]{
0	0	\\
1.1e-10	0	\\
2.2e-10	0	\\
3.3e-10	0	\\
4.4e-10	0	\\
5.4e-10	0	\\
6.5e-10	0	\\
7.5e-10	0	\\
8.6e-10	0	\\
9.6e-10	0	\\
1.07e-09	0	\\
1.18e-09	0	\\
1.28e-09	0	\\
1.38e-09	0	\\
1.49e-09	0	\\
1.59e-09	0	\\
1.69e-09	0	\\
1.8e-09	0	\\
1.9e-09	0	\\
2.01e-09	0	\\
2.11e-09	0	\\
2.21e-09	0	\\
2.32e-09	0	\\
2.42e-09	0	\\
2.52e-09	0	\\
2.63e-09	0	\\
2.73e-09	0	\\
2.83e-09	0	\\
2.93e-09	0	\\
3.04e-09	0	\\
3.14e-09	0	\\
3.24e-09	0	\\
3.34e-09	0	\\
3.45e-09	0	\\
3.55e-09	0	\\
3.65e-09	0	\\
3.75e-09	0	\\
3.86e-09	0	\\
3.96e-09	0	\\
4.06e-09	0	\\
4.16e-09	0	\\
4.27e-09	0	\\
4.37e-09	0	\\
4.47e-09	0	\\
4.57e-09	0	\\
4.68e-09	0	\\
4.78e-09	0	\\
4.89e-09	0	\\
4.99e-09	0	\\
5e-09	0	\\
};
\addplot [color=mycolor1,solid,forget plot]
  table[row sep=crcr]{
0	0	\\
1.1e-10	0	\\
2.2e-10	0	\\
3.3e-10	0	\\
4.4e-10	0	\\
5.4e-10	0	\\
6.5e-10	0	\\
7.5e-10	0	\\
8.6e-10	0	\\
9.6e-10	0	\\
1.07e-09	0	\\
1.18e-09	0	\\
1.28e-09	0	\\
1.38e-09	0	\\
1.49e-09	0	\\
1.59e-09	0	\\
1.69e-09	0	\\
1.8e-09	0	\\
1.9e-09	0	\\
2.01e-09	0	\\
2.11e-09	0	\\
2.21e-09	0	\\
2.32e-09	0	\\
2.42e-09	0	\\
2.52e-09	0	\\
2.63e-09	0	\\
2.73e-09	0	\\
2.83e-09	0	\\
2.93e-09	0	\\
3.04e-09	0	\\
3.14e-09	0	\\
3.24e-09	0	\\
3.34e-09	0	\\
3.45e-09	0	\\
3.55e-09	0	\\
3.65e-09	0	\\
3.75e-09	0	\\
3.86e-09	0	\\
3.96e-09	0	\\
4.06e-09	0	\\
4.16e-09	0	\\
4.27e-09	0	\\
4.37e-09	0	\\
4.47e-09	0	\\
4.57e-09	0	\\
4.68e-09	0	\\
4.78e-09	0	\\
4.89e-09	0	\\
4.99e-09	0	\\
5e-09	0	\\
};
\addplot [color=mycolor2,solid,forget plot]
  table[row sep=crcr]{
0	0	\\
1.1e-10	0	\\
2.2e-10	0	\\
3.3e-10	0	\\
4.4e-10	0	\\
5.4e-10	0	\\
6.5e-10	0	\\
7.5e-10	0	\\
8.6e-10	0	\\
9.6e-10	0	\\
1.07e-09	0	\\
1.18e-09	0	\\
1.28e-09	0	\\
1.38e-09	0	\\
1.49e-09	0	\\
1.59e-09	0	\\
1.69e-09	0	\\
1.8e-09	0	\\
1.9e-09	0	\\
2.01e-09	0	\\
2.11e-09	0	\\
2.21e-09	0	\\
2.32e-09	0	\\
2.42e-09	0	\\
2.52e-09	0	\\
2.63e-09	0	\\
2.73e-09	0	\\
2.83e-09	0	\\
2.93e-09	0	\\
3.04e-09	0	\\
3.14e-09	0	\\
3.24e-09	0	\\
3.34e-09	0	\\
3.45e-09	0	\\
3.55e-09	0	\\
3.65e-09	0	\\
3.75e-09	0	\\
3.86e-09	0	\\
3.96e-09	0	\\
4.06e-09	0	\\
4.16e-09	0	\\
4.27e-09	0	\\
4.37e-09	0	\\
4.47e-09	0	\\
4.57e-09	0	\\
4.68e-09	0	\\
4.78e-09	0	\\
4.89e-09	0	\\
4.99e-09	0	\\
5e-09	0	\\
};
\addplot [color=mycolor3,solid,forget plot]
  table[row sep=crcr]{
0	0	\\
1.1e-10	0	\\
2.2e-10	0	\\
3.3e-10	0	\\
4.4e-10	0	\\
5.4e-10	0	\\
6.5e-10	0	\\
7.5e-10	0	\\
8.6e-10	0	\\
9.6e-10	0	\\
1.07e-09	0	\\
1.18e-09	0	\\
1.28e-09	0	\\
1.38e-09	0	\\
1.49e-09	0	\\
1.59e-09	0	\\
1.69e-09	0	\\
1.8e-09	0	\\
1.9e-09	0	\\
2.01e-09	0	\\
2.11e-09	0	\\
2.21e-09	0	\\
2.32e-09	0	\\
2.42e-09	0	\\
2.52e-09	0	\\
2.63e-09	0	\\
2.73e-09	0	\\
2.83e-09	0	\\
2.93e-09	0	\\
3.04e-09	0	\\
3.14e-09	0	\\
3.24e-09	0	\\
3.34e-09	0	\\
3.45e-09	0	\\
3.55e-09	0	\\
3.65e-09	0	\\
3.75e-09	0	\\
3.86e-09	0	\\
3.96e-09	0	\\
4.06e-09	0	\\
4.16e-09	0	\\
4.27e-09	0	\\
4.37e-09	0	\\
4.47e-09	0	\\
4.57e-09	0	\\
4.68e-09	0	\\
4.78e-09	0	\\
4.89e-09	0	\\
4.99e-09	0	\\
5e-09	0	\\
};
\addplot [color=darkgray,solid,forget plot]
  table[row sep=crcr]{
0	0	\\
1.1e-10	0	\\
2.2e-10	0	\\
3.3e-10	0	\\
4.4e-10	0	\\
5.4e-10	0	\\
6.5e-10	0	\\
7.5e-10	0	\\
8.6e-10	0	\\
9.6e-10	0	\\
1.07e-09	0	\\
1.18e-09	0	\\
1.28e-09	0	\\
1.38e-09	0	\\
1.49e-09	0	\\
1.59e-09	0	\\
1.69e-09	0	\\
1.8e-09	0	\\
1.9e-09	0	\\
2.01e-09	0	\\
2.11e-09	0	\\
2.21e-09	0	\\
2.32e-09	0	\\
2.42e-09	0	\\
2.52e-09	0	\\
2.63e-09	0	\\
2.73e-09	0	\\
2.83e-09	0	\\
2.93e-09	0	\\
3.04e-09	0	\\
3.14e-09	0	\\
3.24e-09	0	\\
3.34e-09	0	\\
3.45e-09	0	\\
3.55e-09	0	\\
3.65e-09	0	\\
3.75e-09	0	\\
3.86e-09	0	\\
3.96e-09	0	\\
4.06e-09	0	\\
4.16e-09	0	\\
4.27e-09	0	\\
4.37e-09	0	\\
4.47e-09	0	\\
4.57e-09	0	\\
4.68e-09	0	\\
4.78e-09	0	\\
4.89e-09	0	\\
4.99e-09	0	\\
5e-09	0	\\
};
\addplot [color=blue,solid,forget plot]
  table[row sep=crcr]{
0	0	\\
1.1e-10	0	\\
2.2e-10	0	\\
3.3e-10	0	\\
4.4e-10	0	\\
5.4e-10	0	\\
6.5e-10	0	\\
7.5e-10	0	\\
8.6e-10	0	\\
9.6e-10	0	\\
1.07e-09	0	\\
1.18e-09	0	\\
1.28e-09	0	\\
1.38e-09	0	\\
1.49e-09	0	\\
1.59e-09	0	\\
1.69e-09	0	\\
1.8e-09	0	\\
1.9e-09	0	\\
2.01e-09	0	\\
2.11e-09	0	\\
2.21e-09	0	\\
2.32e-09	0	\\
2.42e-09	0	\\
2.52e-09	0	\\
2.63e-09	0	\\
2.73e-09	0	\\
2.83e-09	0	\\
2.93e-09	0	\\
3.04e-09	0	\\
3.14e-09	0	\\
3.24e-09	0	\\
3.34e-09	0	\\
3.45e-09	0	\\
3.55e-09	0	\\
3.65e-09	0	\\
3.75e-09	0	\\
3.86e-09	0	\\
3.96e-09	0	\\
4.06e-09	0	\\
4.16e-09	0	\\
4.27e-09	0	\\
4.37e-09	0	\\
4.47e-09	0	\\
4.57e-09	0	\\
4.68e-09	0	\\
4.78e-09	0	\\
4.89e-09	0	\\
4.99e-09	0	\\
5e-09	0	\\
};
\addplot [color=black!50!green,solid,forget plot]
  table[row sep=crcr]{
0	0	\\
1.1e-10	0	\\
2.2e-10	0	\\
3.3e-10	0	\\
4.4e-10	0	\\
5.4e-10	0	\\
6.5e-10	0	\\
7.5e-10	0	\\
8.6e-10	0	\\
9.6e-10	0	\\
1.07e-09	0	\\
1.18e-09	0	\\
1.28e-09	0	\\
1.38e-09	0	\\
1.49e-09	0	\\
1.59e-09	0	\\
1.69e-09	0	\\
1.8e-09	0	\\
1.9e-09	0	\\
2.01e-09	0	\\
2.11e-09	0	\\
2.21e-09	0	\\
2.32e-09	0	\\
2.42e-09	0	\\
2.52e-09	0	\\
2.63e-09	0	\\
2.73e-09	0	\\
2.83e-09	0	\\
2.93e-09	0	\\
3.04e-09	0	\\
3.14e-09	0	\\
3.24e-09	0	\\
3.34e-09	0	\\
3.45e-09	0	\\
3.55e-09	0	\\
3.65e-09	0	\\
3.75e-09	0	\\
3.86e-09	0	\\
3.96e-09	0	\\
4.06e-09	0	\\
4.16e-09	0	\\
4.27e-09	0	\\
4.37e-09	0	\\
4.47e-09	0	\\
4.57e-09	0	\\
4.68e-09	0	\\
4.78e-09	0	\\
4.89e-09	0	\\
4.99e-09	0	\\
5e-09	0	\\
};
\addplot [color=red,solid,forget plot]
  table[row sep=crcr]{
0	0	\\
1.1e-10	0	\\
2.2e-10	0	\\
3.3e-10	0	\\
4.4e-10	0	\\
5.4e-10	0	\\
6.5e-10	0	\\
7.5e-10	0	\\
8.6e-10	0	\\
9.6e-10	0	\\
1.07e-09	0	\\
1.18e-09	0	\\
1.28e-09	0	\\
1.38e-09	0	\\
1.49e-09	0	\\
1.59e-09	0	\\
1.69e-09	0	\\
1.8e-09	0	\\
1.9e-09	0	\\
2.01e-09	0	\\
2.11e-09	0	\\
2.21e-09	0	\\
2.32e-09	0	\\
2.42e-09	0	\\
2.52e-09	0	\\
2.63e-09	0	\\
2.73e-09	0	\\
2.83e-09	0	\\
2.93e-09	0	\\
3.04e-09	0	\\
3.14e-09	0	\\
3.24e-09	0	\\
3.34e-09	0	\\
3.45e-09	0	\\
3.55e-09	0	\\
3.65e-09	0	\\
3.75e-09	0	\\
3.86e-09	0	\\
3.96e-09	0	\\
4.06e-09	0	\\
4.16e-09	0	\\
4.27e-09	0	\\
4.37e-09	0	\\
4.47e-09	0	\\
4.57e-09	0	\\
4.68e-09	0	\\
4.78e-09	0	\\
4.89e-09	0	\\
4.99e-09	0	\\
5e-09	0	\\
};
\addplot [color=mycolor1,solid,forget plot]
  table[row sep=crcr]{
0	0	\\
1.1e-10	0	\\
2.2e-10	0	\\
3.3e-10	0	\\
4.4e-10	0	\\
5.4e-10	0	\\
6.5e-10	0	\\
7.5e-10	0	\\
8.6e-10	0	\\
9.6e-10	0	\\
1.07e-09	0	\\
1.18e-09	0	\\
1.28e-09	0	\\
1.38e-09	0	\\
1.49e-09	0	\\
1.59e-09	0	\\
1.69e-09	0	\\
1.8e-09	0	\\
1.9e-09	0	\\
2.01e-09	0	\\
2.11e-09	0	\\
2.21e-09	0	\\
2.32e-09	0	\\
2.42e-09	0	\\
2.52e-09	0	\\
2.63e-09	0	\\
2.73e-09	0	\\
2.83e-09	0	\\
2.93e-09	0	\\
3.04e-09	0	\\
3.14e-09	0	\\
3.24e-09	0	\\
3.34e-09	0	\\
3.45e-09	0	\\
3.55e-09	0	\\
3.65e-09	0	\\
3.75e-09	0	\\
3.86e-09	0	\\
3.96e-09	0	\\
4.06e-09	0	\\
4.16e-09	0	\\
4.27e-09	0	\\
4.37e-09	0	\\
4.47e-09	0	\\
4.57e-09	0	\\
4.68e-09	0	\\
4.78e-09	0	\\
4.89e-09	0	\\
4.99e-09	0	\\
5e-09	0	\\
};
\addplot [color=mycolor2,solid,forget plot]
  table[row sep=crcr]{
0	0	\\
1.1e-10	0	\\
2.2e-10	0	\\
3.3e-10	0	\\
4.4e-10	0	\\
5.4e-10	0	\\
6.5e-10	0	\\
7.5e-10	0	\\
8.6e-10	0	\\
9.6e-10	0	\\
1.07e-09	0	\\
1.18e-09	0	\\
1.28e-09	0	\\
1.38e-09	0	\\
1.49e-09	0	\\
1.59e-09	0	\\
1.69e-09	0	\\
1.8e-09	0	\\
1.9e-09	0	\\
2.01e-09	0	\\
2.11e-09	0	\\
2.21e-09	0	\\
2.32e-09	0	\\
2.42e-09	0	\\
2.52e-09	0	\\
2.63e-09	0	\\
2.73e-09	0	\\
2.83e-09	0	\\
2.93e-09	0	\\
3.04e-09	0	\\
3.14e-09	0	\\
3.24e-09	0	\\
3.34e-09	0	\\
3.45e-09	0	\\
3.55e-09	0	\\
3.65e-09	0	\\
3.75e-09	0	\\
3.86e-09	0	\\
3.96e-09	0	\\
4.06e-09	0	\\
4.16e-09	0	\\
4.27e-09	0	\\
4.37e-09	0	\\
4.47e-09	0	\\
4.57e-09	0	\\
4.68e-09	0	\\
4.78e-09	0	\\
4.89e-09	0	\\
4.99e-09	0	\\
5e-09	0	\\
};
\addplot [color=mycolor3,solid,forget plot]
  table[row sep=crcr]{
0	0	\\
1.1e-10	0	\\
2.2e-10	0	\\
3.3e-10	0	\\
4.4e-10	0	\\
5.4e-10	0	\\
6.5e-10	0	\\
7.5e-10	0	\\
8.6e-10	0	\\
9.6e-10	0	\\
1.07e-09	0	\\
1.18e-09	0	\\
1.28e-09	0	\\
1.38e-09	0	\\
1.49e-09	0	\\
1.59e-09	0	\\
1.69e-09	0	\\
1.8e-09	0	\\
1.9e-09	0	\\
2.01e-09	0	\\
2.11e-09	0	\\
2.21e-09	0	\\
2.32e-09	0	\\
2.42e-09	0	\\
2.52e-09	0	\\
2.63e-09	0	\\
2.73e-09	0	\\
2.83e-09	0	\\
2.93e-09	0	\\
3.04e-09	0	\\
3.14e-09	0	\\
3.24e-09	0	\\
3.34e-09	0	\\
3.45e-09	0	\\
3.55e-09	0	\\
3.65e-09	0	\\
3.75e-09	0	\\
3.86e-09	0	\\
3.96e-09	0	\\
4.06e-09	0	\\
4.16e-09	0	\\
4.27e-09	0	\\
4.37e-09	0	\\
4.47e-09	0	\\
4.57e-09	0	\\
4.68e-09	0	\\
4.78e-09	0	\\
4.89e-09	0	\\
4.99e-09	0	\\
5e-09	0	\\
};
\addplot [color=darkgray,solid,forget plot]
  table[row sep=crcr]{
0	0	\\
1.1e-10	0	\\
2.2e-10	0	\\
3.3e-10	0	\\
4.4e-10	0	\\
5.4e-10	0	\\
6.5e-10	0	\\
7.5e-10	0	\\
8.6e-10	0	\\
9.6e-10	0	\\
1.07e-09	0	\\
1.18e-09	0	\\
1.28e-09	0	\\
1.38e-09	0	\\
1.49e-09	0	\\
1.59e-09	0	\\
1.69e-09	0	\\
1.8e-09	0	\\
1.9e-09	0	\\
2.01e-09	0	\\
2.11e-09	0	\\
2.21e-09	0	\\
2.32e-09	0	\\
2.42e-09	0	\\
2.52e-09	0	\\
2.63e-09	0	\\
2.73e-09	0	\\
2.83e-09	0	\\
2.93e-09	0	\\
3.04e-09	0	\\
3.14e-09	0	\\
3.24e-09	0	\\
3.34e-09	0	\\
3.45e-09	0	\\
3.55e-09	0	\\
3.65e-09	0	\\
3.75e-09	0	\\
3.86e-09	0	\\
3.96e-09	0	\\
4.06e-09	0	\\
4.16e-09	0	\\
4.27e-09	0	\\
4.37e-09	0	\\
4.47e-09	0	\\
4.57e-09	0	\\
4.68e-09	0	\\
4.78e-09	0	\\
4.89e-09	0	\\
4.99e-09	0	\\
5e-09	0	\\
};
\addplot [color=blue,solid,forget plot]
  table[row sep=crcr]{
0	0	\\
1.1e-10	0	\\
2.2e-10	0	\\
3.3e-10	0	\\
4.4e-10	0	\\
5.4e-10	0	\\
6.5e-10	0	\\
7.5e-10	0	\\
8.6e-10	0	\\
9.6e-10	0	\\
1.07e-09	0	\\
1.18e-09	0	\\
1.28e-09	0	\\
1.38e-09	0	\\
1.49e-09	0	\\
1.59e-09	0	\\
1.69e-09	0	\\
1.8e-09	0	\\
1.9e-09	0	\\
2.01e-09	0	\\
2.11e-09	0	\\
2.21e-09	0	\\
2.32e-09	0	\\
2.42e-09	0	\\
2.52e-09	0	\\
2.63e-09	0	\\
2.73e-09	0	\\
2.83e-09	0	\\
2.93e-09	0	\\
3.04e-09	0	\\
3.14e-09	0	\\
3.24e-09	0	\\
3.34e-09	0	\\
3.45e-09	0	\\
3.55e-09	0	\\
3.65e-09	0	\\
3.75e-09	0	\\
3.86e-09	0	\\
3.96e-09	0	\\
4.06e-09	0	\\
4.16e-09	0	\\
4.27e-09	0	\\
4.37e-09	0	\\
4.47e-09	0	\\
4.57e-09	0	\\
4.68e-09	0	\\
4.78e-09	0	\\
4.89e-09	0	\\
4.99e-09	0	\\
5e-09	0	\\
};
\addplot [color=black!50!green,solid,forget plot]
  table[row sep=crcr]{
0	0	\\
1.1e-10	0	\\
2.2e-10	0	\\
3.3e-10	0	\\
4.4e-10	0	\\
5.4e-10	0	\\
6.5e-10	0	\\
7.5e-10	0	\\
8.6e-10	0	\\
9.6e-10	0	\\
1.07e-09	0	\\
1.18e-09	0	\\
1.28e-09	0	\\
1.38e-09	0	\\
1.49e-09	0	\\
1.59e-09	0	\\
1.69e-09	0	\\
1.8e-09	0	\\
1.9e-09	0	\\
2.01e-09	0	\\
2.11e-09	0	\\
2.21e-09	0	\\
2.32e-09	0	\\
2.42e-09	0	\\
2.52e-09	0	\\
2.63e-09	0	\\
2.73e-09	0	\\
2.83e-09	0	\\
2.93e-09	0	\\
3.04e-09	0	\\
3.14e-09	0	\\
3.24e-09	0	\\
3.34e-09	0	\\
3.45e-09	0	\\
3.55e-09	0	\\
3.65e-09	0	\\
3.75e-09	0	\\
3.86e-09	0	\\
3.96e-09	0	\\
4.06e-09	0	\\
4.16e-09	0	\\
4.27e-09	0	\\
4.37e-09	0	\\
4.47e-09	0	\\
4.57e-09	0	\\
4.68e-09	0	\\
4.78e-09	0	\\
4.89e-09	0	\\
4.99e-09	0	\\
5e-09	0	\\
};
\addplot [color=red,solid,forget plot]
  table[row sep=crcr]{
0	0	\\
1.1e-10	0	\\
2.2e-10	0	\\
3.3e-10	0	\\
4.4e-10	0	\\
5.4e-10	0	\\
6.5e-10	0	\\
7.5e-10	0	\\
8.6e-10	0	\\
9.6e-10	0	\\
1.07e-09	0	\\
1.18e-09	0	\\
1.28e-09	0	\\
1.38e-09	0	\\
1.49e-09	0	\\
1.59e-09	0	\\
1.69e-09	0	\\
1.8e-09	0	\\
1.9e-09	0	\\
2.01e-09	0	\\
2.11e-09	0	\\
2.21e-09	0	\\
2.32e-09	0	\\
2.42e-09	0	\\
2.52e-09	0	\\
2.63e-09	0	\\
2.73e-09	0	\\
2.83e-09	0	\\
2.93e-09	0	\\
3.04e-09	0	\\
3.14e-09	0	\\
3.24e-09	0	\\
3.34e-09	0	\\
3.45e-09	0	\\
3.55e-09	0	\\
3.65e-09	0	\\
3.75e-09	0	\\
3.86e-09	0	\\
3.96e-09	0	\\
4.06e-09	0	\\
4.16e-09	0	\\
4.27e-09	0	\\
4.37e-09	0	\\
4.47e-09	0	\\
4.57e-09	0	\\
4.68e-09	0	\\
4.78e-09	0	\\
4.89e-09	0	\\
4.99e-09	0	\\
5e-09	0	\\
};
\addplot [color=mycolor1,solid,forget plot]
  table[row sep=crcr]{
0	0	\\
1.1e-10	0	\\
2.2e-10	0	\\
3.3e-10	0	\\
4.4e-10	0	\\
5.4e-10	0	\\
6.5e-10	0	\\
7.5e-10	0	\\
8.6e-10	0	\\
9.6e-10	0	\\
1.07e-09	0	\\
1.18e-09	0	\\
1.28e-09	0	\\
1.38e-09	0	\\
1.49e-09	0	\\
1.59e-09	0	\\
1.69e-09	0	\\
1.8e-09	0	\\
1.9e-09	0	\\
2.01e-09	0	\\
2.11e-09	0	\\
2.21e-09	0	\\
2.32e-09	0	\\
2.42e-09	0	\\
2.52e-09	0	\\
2.63e-09	0	\\
2.73e-09	0	\\
2.83e-09	0	\\
2.93e-09	0	\\
3.04e-09	0	\\
3.14e-09	0	\\
3.24e-09	0	\\
3.34e-09	0	\\
3.45e-09	0	\\
3.55e-09	0	\\
3.65e-09	0	\\
3.75e-09	0	\\
3.86e-09	0	\\
3.96e-09	0	\\
4.06e-09	0	\\
4.16e-09	0	\\
4.27e-09	0	\\
4.37e-09	0	\\
4.47e-09	0	\\
4.57e-09	0	\\
4.68e-09	0	\\
4.78e-09	0	\\
4.89e-09	0	\\
4.99e-09	0	\\
5e-09	0	\\
};
\addplot [color=mycolor2,solid,forget plot]
  table[row sep=crcr]{
0	0	\\
1.1e-10	0	\\
2.2e-10	0	\\
3.3e-10	0	\\
4.4e-10	0	\\
5.4e-10	0	\\
6.5e-10	0	\\
7.5e-10	0	\\
8.6e-10	0	\\
9.6e-10	0	\\
1.07e-09	0	\\
1.18e-09	0	\\
1.28e-09	0	\\
1.38e-09	0	\\
1.49e-09	0	\\
1.59e-09	0	\\
1.69e-09	0	\\
1.8e-09	0	\\
1.9e-09	0	\\
2.01e-09	0	\\
2.11e-09	0	\\
2.21e-09	0	\\
2.32e-09	0	\\
2.42e-09	0	\\
2.52e-09	0	\\
2.63e-09	0	\\
2.73e-09	0	\\
2.83e-09	0	\\
2.93e-09	0	\\
3.04e-09	0	\\
3.14e-09	0	\\
3.24e-09	0	\\
3.34e-09	0	\\
3.45e-09	0	\\
3.55e-09	0	\\
3.65e-09	0	\\
3.75e-09	0	\\
3.86e-09	0	\\
3.96e-09	0	\\
4.06e-09	0	\\
4.16e-09	0	\\
4.27e-09	0	\\
4.37e-09	0	\\
4.47e-09	0	\\
4.57e-09	0	\\
4.68e-09	0	\\
4.78e-09	0	\\
4.89e-09	0	\\
4.99e-09	0	\\
5e-09	0	\\
};
\addplot [color=mycolor3,solid,forget plot]
  table[row sep=crcr]{
0	0	\\
1.1e-10	0	\\
2.2e-10	0	\\
3.3e-10	0	\\
4.4e-10	0	\\
5.4e-10	0	\\
6.5e-10	0	\\
7.5e-10	0	\\
8.6e-10	0	\\
9.6e-10	0	\\
1.07e-09	0	\\
1.18e-09	0	\\
1.28e-09	0	\\
1.38e-09	0	\\
1.49e-09	0	\\
1.59e-09	0	\\
1.69e-09	0	\\
1.8e-09	0	\\
1.9e-09	0	\\
2.01e-09	0	\\
2.11e-09	0	\\
2.21e-09	0	\\
2.32e-09	0	\\
2.42e-09	0	\\
2.52e-09	0	\\
2.63e-09	0	\\
2.73e-09	0	\\
2.83e-09	0	\\
2.93e-09	0	\\
3.04e-09	0	\\
3.14e-09	0	\\
3.24e-09	0	\\
3.34e-09	0	\\
3.45e-09	0	\\
3.55e-09	0	\\
3.65e-09	0	\\
3.75e-09	0	\\
3.86e-09	0	\\
3.96e-09	0	\\
4.06e-09	0	\\
4.16e-09	0	\\
4.27e-09	0	\\
4.37e-09	0	\\
4.47e-09	0	\\
4.57e-09	0	\\
4.68e-09	0	\\
4.78e-09	0	\\
4.89e-09	0	\\
4.99e-09	0	\\
5e-09	0	\\
};
\addplot [color=darkgray,solid,forget plot]
  table[row sep=crcr]{
0	0	\\
1.1e-10	0	\\
2.2e-10	0	\\
3.3e-10	0	\\
4.4e-10	0	\\
5.4e-10	0	\\
6.5e-10	0	\\
7.5e-10	0	\\
8.6e-10	0	\\
9.6e-10	0	\\
1.07e-09	0	\\
1.18e-09	0	\\
1.28e-09	0	\\
1.38e-09	0	\\
1.49e-09	0	\\
1.59e-09	0	\\
1.69e-09	0	\\
1.8e-09	0	\\
1.9e-09	0	\\
2.01e-09	0	\\
2.11e-09	0	\\
2.21e-09	0	\\
2.32e-09	0	\\
2.42e-09	0	\\
2.52e-09	0	\\
2.63e-09	0	\\
2.73e-09	0	\\
2.83e-09	0	\\
2.93e-09	0	\\
3.04e-09	0	\\
3.14e-09	0	\\
3.24e-09	0	\\
3.34e-09	0	\\
3.45e-09	0	\\
3.55e-09	0	\\
3.65e-09	0	\\
3.75e-09	0	\\
3.86e-09	0	\\
3.96e-09	0	\\
4.06e-09	0	\\
4.16e-09	0	\\
4.27e-09	0	\\
4.37e-09	0	\\
4.47e-09	0	\\
4.57e-09	0	\\
4.68e-09	0	\\
4.78e-09	0	\\
4.89e-09	0	\\
4.99e-09	0	\\
5e-09	0	\\
};
\addplot [color=blue,solid,forget plot]
  table[row sep=crcr]{
0	0	\\
1.1e-10	0	\\
2.2e-10	0	\\
3.3e-10	0	\\
4.4e-10	0	\\
5.4e-10	0	\\
6.5e-10	0	\\
7.5e-10	0	\\
8.6e-10	0	\\
9.6e-10	0	\\
1.07e-09	0	\\
1.18e-09	0	\\
1.28e-09	0	\\
1.38e-09	0	\\
1.49e-09	0	\\
1.59e-09	0	\\
1.69e-09	0	\\
1.8e-09	0	\\
1.9e-09	0	\\
2.01e-09	0	\\
2.11e-09	0	\\
2.21e-09	0	\\
2.32e-09	0	\\
2.42e-09	0	\\
2.52e-09	0	\\
2.63e-09	0	\\
2.73e-09	0	\\
2.83e-09	0	\\
2.93e-09	0	\\
3.04e-09	0	\\
3.14e-09	0	\\
3.24e-09	0	\\
3.34e-09	0	\\
3.45e-09	0	\\
3.55e-09	0	\\
3.65e-09	0	\\
3.75e-09	0	\\
3.86e-09	0	\\
3.96e-09	0	\\
4.06e-09	0	\\
4.16e-09	0	\\
4.27e-09	0	\\
4.37e-09	0	\\
4.47e-09	0	\\
4.57e-09	0	\\
4.68e-09	0	\\
4.78e-09	0	\\
4.89e-09	0	\\
4.99e-09	0	\\
5e-09	0	\\
};
\addplot [color=black!50!green,solid,forget plot]
  table[row sep=crcr]{
0	0	\\
1.1e-10	0	\\
2.2e-10	0	\\
3.3e-10	0	\\
4.4e-10	0	\\
5.4e-10	0	\\
6.5e-10	0	\\
7.5e-10	0	\\
8.6e-10	0	\\
9.6e-10	0	\\
1.07e-09	0	\\
1.18e-09	0	\\
1.28e-09	0	\\
1.38e-09	0	\\
1.49e-09	0	\\
1.59e-09	0	\\
1.69e-09	0	\\
1.8e-09	0	\\
1.9e-09	0	\\
2.01e-09	0	\\
2.11e-09	0	\\
2.21e-09	0	\\
2.32e-09	0	\\
2.42e-09	0	\\
2.52e-09	0	\\
2.63e-09	0	\\
2.73e-09	0	\\
2.83e-09	0	\\
2.93e-09	0	\\
3.04e-09	0	\\
3.14e-09	0	\\
3.24e-09	0	\\
3.34e-09	0	\\
3.45e-09	0	\\
3.55e-09	0	\\
3.65e-09	0	\\
3.75e-09	0	\\
3.86e-09	0	\\
3.96e-09	0	\\
4.06e-09	0	\\
4.16e-09	0	\\
4.27e-09	0	\\
4.37e-09	0	\\
4.47e-09	0	\\
4.57e-09	0	\\
4.68e-09	0	\\
4.78e-09	0	\\
4.89e-09	0	\\
4.99e-09	0	\\
5e-09	0	\\
};
\addplot [color=red,solid,forget plot]
  table[row sep=crcr]{
0	0	\\
1.1e-10	0	\\
2.2e-10	0	\\
3.3e-10	0	\\
4.4e-10	0	\\
5.4e-10	0	\\
6.5e-10	0	\\
7.5e-10	0	\\
8.6e-10	0	\\
9.6e-10	0	\\
1.07e-09	0	\\
1.18e-09	0	\\
1.28e-09	0	\\
1.38e-09	0	\\
1.49e-09	0	\\
1.59e-09	0	\\
1.69e-09	0	\\
1.8e-09	0	\\
1.9e-09	0	\\
2.01e-09	0	\\
2.11e-09	0	\\
2.21e-09	0	\\
2.32e-09	0	\\
2.42e-09	0	\\
2.52e-09	0	\\
2.63e-09	0	\\
2.73e-09	0	\\
2.83e-09	0	\\
2.93e-09	0	\\
3.04e-09	0	\\
3.14e-09	0	\\
3.24e-09	0	\\
3.34e-09	0	\\
3.45e-09	0	\\
3.55e-09	0	\\
3.65e-09	0	\\
3.75e-09	0	\\
3.86e-09	0	\\
3.96e-09	0	\\
4.06e-09	0	\\
4.16e-09	0	\\
4.27e-09	0	\\
4.37e-09	0	\\
4.47e-09	0	\\
4.57e-09	0	\\
4.68e-09	0	\\
4.78e-09	0	\\
4.89e-09	0	\\
4.99e-09	0	\\
5e-09	0	\\
};
\addplot [color=mycolor1,solid,forget plot]
  table[row sep=crcr]{
0	0	\\
1.1e-10	0	\\
2.2e-10	0	\\
3.3e-10	0	\\
4.4e-10	0	\\
5.4e-10	0	\\
6.5e-10	0	\\
7.5e-10	0	\\
8.6e-10	0	\\
9.6e-10	0	\\
1.07e-09	0	\\
1.18e-09	0	\\
1.28e-09	0	\\
1.38e-09	0	\\
1.49e-09	0	\\
1.59e-09	0	\\
1.69e-09	0	\\
1.8e-09	0	\\
1.9e-09	0	\\
2.01e-09	0	\\
2.11e-09	0	\\
2.21e-09	0	\\
2.32e-09	0	\\
2.42e-09	0	\\
2.52e-09	0	\\
2.63e-09	0	\\
2.73e-09	0	\\
2.83e-09	0	\\
2.93e-09	0	\\
3.04e-09	0	\\
3.14e-09	0	\\
3.24e-09	0	\\
3.34e-09	0	\\
3.45e-09	0	\\
3.55e-09	0	\\
3.65e-09	0	\\
3.75e-09	0	\\
3.86e-09	0	\\
3.96e-09	0	\\
4.06e-09	0	\\
4.16e-09	0	\\
4.27e-09	0	\\
4.37e-09	0	\\
4.47e-09	0	\\
4.57e-09	0	\\
4.68e-09	0	\\
4.78e-09	0	\\
4.89e-09	0	\\
4.99e-09	0	\\
5e-09	0	\\
};
\addplot [color=mycolor2,solid,forget plot]
  table[row sep=crcr]{
0	0	\\
1.1e-10	0	\\
2.2e-10	0	\\
3.3e-10	0	\\
4.4e-10	0	\\
5.4e-10	0	\\
6.5e-10	0	\\
7.5e-10	0	\\
8.6e-10	0	\\
9.6e-10	0	\\
1.07e-09	0	\\
1.18e-09	0	\\
1.28e-09	0	\\
1.38e-09	0	\\
1.49e-09	0	\\
1.59e-09	0	\\
1.69e-09	0	\\
1.8e-09	0	\\
1.9e-09	0	\\
2.01e-09	0	\\
2.11e-09	0	\\
2.21e-09	0	\\
2.32e-09	0	\\
2.42e-09	0	\\
2.52e-09	0	\\
2.63e-09	0	\\
2.73e-09	0	\\
2.83e-09	0	\\
2.93e-09	0	\\
3.04e-09	0	\\
3.14e-09	0	\\
3.24e-09	0	\\
3.34e-09	0	\\
3.45e-09	0	\\
3.55e-09	0	\\
3.65e-09	0	\\
3.75e-09	0	\\
3.86e-09	0	\\
3.96e-09	0	\\
4.06e-09	0	\\
4.16e-09	0	\\
4.27e-09	0	\\
4.37e-09	0	\\
4.47e-09	0	\\
4.57e-09	0	\\
4.68e-09	0	\\
4.78e-09	0	\\
4.89e-09	0	\\
4.99e-09	0	\\
5e-09	0	\\
};
\addplot [color=mycolor3,solid,forget plot]
  table[row sep=crcr]{
0	0	\\
1.1e-10	0	\\
2.2e-10	0	\\
3.3e-10	0	\\
4.4e-10	0	\\
5.4e-10	0	\\
6.5e-10	0	\\
7.5e-10	0	\\
8.6e-10	0	\\
9.6e-10	0	\\
1.07e-09	0	\\
1.18e-09	0	\\
1.28e-09	0	\\
1.38e-09	0	\\
1.49e-09	0	\\
1.59e-09	0	\\
1.69e-09	0	\\
1.8e-09	0	\\
1.9e-09	0	\\
2.01e-09	0	\\
2.11e-09	0	\\
2.21e-09	0	\\
2.32e-09	0	\\
2.42e-09	0	\\
2.52e-09	0	\\
2.63e-09	0	\\
2.73e-09	0	\\
2.83e-09	0	\\
2.93e-09	0	\\
3.04e-09	0	\\
3.14e-09	0	\\
3.24e-09	0	\\
3.34e-09	0	\\
3.45e-09	0	\\
3.55e-09	0	\\
3.65e-09	0	\\
3.75e-09	0	\\
3.86e-09	0	\\
3.96e-09	0	\\
4.06e-09	0	\\
4.16e-09	0	\\
4.27e-09	0	\\
4.37e-09	0	\\
4.47e-09	0	\\
4.57e-09	0	\\
4.68e-09	0	\\
4.78e-09	0	\\
4.89e-09	0	\\
4.99e-09	0	\\
5e-09	0	\\
};
\addplot [color=darkgray,solid,forget plot]
  table[row sep=crcr]{
0	0	\\
1.1e-10	0	\\
2.2e-10	0	\\
3.3e-10	0	\\
4.4e-10	0	\\
5.4e-10	0	\\
6.5e-10	0	\\
7.5e-10	0	\\
8.6e-10	0	\\
9.6e-10	0	\\
1.07e-09	0	\\
1.18e-09	0	\\
1.28e-09	0	\\
1.38e-09	0	\\
1.49e-09	0	\\
1.59e-09	0	\\
1.69e-09	0	\\
1.8e-09	0	\\
1.9e-09	0	\\
2.01e-09	0	\\
2.11e-09	0	\\
2.21e-09	0	\\
2.32e-09	0	\\
2.42e-09	0	\\
2.52e-09	0	\\
2.63e-09	0	\\
2.73e-09	0	\\
2.83e-09	0	\\
2.93e-09	0	\\
3.04e-09	0	\\
3.14e-09	0	\\
3.24e-09	0	\\
3.34e-09	0	\\
3.45e-09	0	\\
3.55e-09	0	\\
3.65e-09	0	\\
3.75e-09	0	\\
3.86e-09	0	\\
3.96e-09	0	\\
4.06e-09	0	\\
4.16e-09	0	\\
4.27e-09	0	\\
4.37e-09	0	\\
4.47e-09	0	\\
4.57e-09	0	\\
4.68e-09	0	\\
4.78e-09	0	\\
4.89e-09	0	\\
4.99e-09	0	\\
5e-09	0	\\
};
\addplot [color=blue,solid,forget plot]
  table[row sep=crcr]{
0	0	\\
1.1e-10	0	\\
2.2e-10	0	\\
3.3e-10	0	\\
4.4e-10	0	\\
5.4e-10	0	\\
6.5e-10	0	\\
7.5e-10	0	\\
8.6e-10	0	\\
9.6e-10	0	\\
1.07e-09	0	\\
1.18e-09	0	\\
1.28e-09	0	\\
1.38e-09	0	\\
1.49e-09	0	\\
1.59e-09	0	\\
1.69e-09	0	\\
1.8e-09	0	\\
1.9e-09	0	\\
2.01e-09	0	\\
2.11e-09	0	\\
2.21e-09	0	\\
2.32e-09	0	\\
2.42e-09	0	\\
2.52e-09	0	\\
2.63e-09	0	\\
2.73e-09	0	\\
2.83e-09	0	\\
2.93e-09	0	\\
3.04e-09	0	\\
3.14e-09	0	\\
3.24e-09	0	\\
3.34e-09	0	\\
3.45e-09	0	\\
3.55e-09	0	\\
3.65e-09	0	\\
3.75e-09	0	\\
3.86e-09	0	\\
3.96e-09	0	\\
4.06e-09	0	\\
4.16e-09	0	\\
4.27e-09	0	\\
4.37e-09	0	\\
4.47e-09	0	\\
4.57e-09	0	\\
4.68e-09	0	\\
4.78e-09	0	\\
4.89e-09	0	\\
4.99e-09	0	\\
5e-09	0	\\
};
\addplot [color=black!50!green,solid,forget plot]
  table[row sep=crcr]{
0	0	\\
1.1e-10	0	\\
2.2e-10	0	\\
3.3e-10	0	\\
4.4e-10	0	\\
5.4e-10	0	\\
6.5e-10	0	\\
7.5e-10	0	\\
8.6e-10	0	\\
9.6e-10	0	\\
1.07e-09	0	\\
1.18e-09	0	\\
1.28e-09	0	\\
1.38e-09	0	\\
1.49e-09	0	\\
1.59e-09	0	\\
1.69e-09	0	\\
1.8e-09	0	\\
1.9e-09	0	\\
2.01e-09	0	\\
2.11e-09	0	\\
2.21e-09	0	\\
2.32e-09	0	\\
2.42e-09	0	\\
2.52e-09	0	\\
2.63e-09	0	\\
2.73e-09	0	\\
2.83e-09	0	\\
2.93e-09	0	\\
3.04e-09	0	\\
3.14e-09	0	\\
3.24e-09	0	\\
3.34e-09	0	\\
3.45e-09	0	\\
3.55e-09	0	\\
3.65e-09	0	\\
3.75e-09	0	\\
3.86e-09	0	\\
3.96e-09	0	\\
4.06e-09	0	\\
4.16e-09	0	\\
4.27e-09	0	\\
4.37e-09	0	\\
4.47e-09	0	\\
4.57e-09	0	\\
4.68e-09	0	\\
4.78e-09	0	\\
4.89e-09	0	\\
4.99e-09	0	\\
5e-09	0	\\
};
\addplot [color=red,solid,forget plot]
  table[row sep=crcr]{
0	0	\\
1.1e-10	0	\\
2.2e-10	0	\\
3.3e-10	0	\\
4.4e-10	0	\\
5.4e-10	0	\\
6.5e-10	0	\\
7.5e-10	0	\\
8.6e-10	0	\\
9.6e-10	0	\\
1.07e-09	0	\\
1.18e-09	0	\\
1.28e-09	0	\\
1.38e-09	0	\\
1.49e-09	0	\\
1.59e-09	0	\\
1.69e-09	0	\\
1.8e-09	0	\\
1.9e-09	0	\\
2.01e-09	0	\\
2.11e-09	0	\\
2.21e-09	0	\\
2.32e-09	0	\\
2.42e-09	0	\\
2.52e-09	0	\\
2.63e-09	0	\\
2.73e-09	0	\\
2.83e-09	0	\\
2.93e-09	0	\\
3.04e-09	0	\\
3.14e-09	0	\\
3.24e-09	0	\\
3.34e-09	0	\\
3.45e-09	0	\\
3.55e-09	0	\\
3.65e-09	0	\\
3.75e-09	0	\\
3.86e-09	0	\\
3.96e-09	0	\\
4.06e-09	0	\\
4.16e-09	0	\\
4.27e-09	0	\\
4.37e-09	0	\\
4.47e-09	0	\\
4.57e-09	0	\\
4.68e-09	0	\\
4.78e-09	0	\\
4.89e-09	0	\\
4.99e-09	0	\\
5e-09	0	\\
};
\addplot [color=mycolor1,solid,forget plot]
  table[row sep=crcr]{
0	0	\\
1.1e-10	0	\\
2.2e-10	0	\\
3.3e-10	0	\\
4.4e-10	0	\\
5.4e-10	0	\\
6.5e-10	0	\\
7.5e-10	0	\\
8.6e-10	0	\\
9.6e-10	0	\\
1.07e-09	0	\\
1.18e-09	0	\\
1.28e-09	0	\\
1.38e-09	0	\\
1.49e-09	0	\\
1.59e-09	0	\\
1.69e-09	0	\\
1.8e-09	0	\\
1.9e-09	0	\\
2.01e-09	0	\\
2.11e-09	0	\\
2.21e-09	0	\\
2.32e-09	0	\\
2.42e-09	0	\\
2.52e-09	0	\\
2.63e-09	0	\\
2.73e-09	0	\\
2.83e-09	0	\\
2.93e-09	0	\\
3.04e-09	0	\\
3.14e-09	0	\\
3.24e-09	0	\\
3.34e-09	0	\\
3.45e-09	0	\\
3.55e-09	0	\\
3.65e-09	0	\\
3.75e-09	0	\\
3.86e-09	0	\\
3.96e-09	0	\\
4.06e-09	0	\\
4.16e-09	0	\\
4.27e-09	0	\\
4.37e-09	0	\\
4.47e-09	0	\\
4.57e-09	0	\\
4.68e-09	0	\\
4.78e-09	0	\\
4.89e-09	0	\\
4.99e-09	0	\\
5e-09	0	\\
};
\addplot [color=mycolor2,solid,forget plot]
  table[row sep=crcr]{
0	0	\\
1.1e-10	0	\\
2.2e-10	0	\\
3.3e-10	0	\\
4.4e-10	0	\\
5.4e-10	0	\\
6.5e-10	0	\\
7.5e-10	0	\\
8.6e-10	0	\\
9.6e-10	0	\\
1.07e-09	0	\\
1.18e-09	0	\\
1.28e-09	0	\\
1.38e-09	0	\\
1.49e-09	0	\\
1.59e-09	0	\\
1.69e-09	0	\\
1.8e-09	0	\\
1.9e-09	0	\\
2.01e-09	0	\\
2.11e-09	0	\\
2.21e-09	0	\\
2.32e-09	0	\\
2.42e-09	0	\\
2.52e-09	0	\\
2.63e-09	0	\\
2.73e-09	0	\\
2.83e-09	0	\\
2.93e-09	0	\\
3.04e-09	0	\\
3.14e-09	0	\\
3.24e-09	0	\\
3.34e-09	0	\\
3.45e-09	0	\\
3.55e-09	0	\\
3.65e-09	0	\\
3.75e-09	0	\\
3.86e-09	0	\\
3.96e-09	0	\\
4.06e-09	0	\\
4.16e-09	0	\\
4.27e-09	0	\\
4.37e-09	0	\\
4.47e-09	0	\\
4.57e-09	0	\\
4.68e-09	0	\\
4.78e-09	0	\\
4.89e-09	0	\\
4.99e-09	0	\\
5e-09	0	\\
};
\addplot [color=mycolor3,solid,forget plot]
  table[row sep=crcr]{
0	0	\\
1.1e-10	0	\\
2.2e-10	0	\\
3.3e-10	0	\\
4.4e-10	0	\\
5.4e-10	0	\\
6.5e-10	0	\\
7.5e-10	0	\\
8.6e-10	0	\\
9.6e-10	0	\\
1.07e-09	0	\\
1.18e-09	0	\\
1.28e-09	0	\\
1.38e-09	0	\\
1.49e-09	0	\\
1.59e-09	0	\\
1.69e-09	0	\\
1.8e-09	0	\\
1.9e-09	0	\\
2.01e-09	0	\\
2.11e-09	0	\\
2.21e-09	0	\\
2.32e-09	0	\\
2.42e-09	0	\\
2.52e-09	0	\\
2.63e-09	0	\\
2.73e-09	0	\\
2.83e-09	0	\\
2.93e-09	0	\\
3.04e-09	0	\\
3.14e-09	0	\\
3.24e-09	0	\\
3.34e-09	0	\\
3.45e-09	0	\\
3.55e-09	0	\\
3.65e-09	0	\\
3.75e-09	0	\\
3.86e-09	0	\\
3.96e-09	0	\\
4.06e-09	0	\\
4.16e-09	0	\\
4.27e-09	0	\\
4.37e-09	0	\\
4.47e-09	0	\\
4.57e-09	0	\\
4.68e-09	0	\\
4.78e-09	0	\\
4.89e-09	0	\\
4.99e-09	0	\\
5e-09	0	\\
};
\addplot [color=darkgray,solid,forget plot]
  table[row sep=crcr]{
0	0	\\
1.1e-10	0	\\
2.2e-10	0	\\
3.3e-10	0	\\
4.4e-10	0	\\
5.4e-10	0	\\
6.5e-10	0	\\
7.5e-10	0	\\
8.6e-10	0	\\
9.6e-10	0	\\
1.07e-09	0	\\
1.18e-09	0	\\
1.28e-09	0	\\
1.38e-09	0	\\
1.49e-09	0	\\
1.59e-09	0	\\
1.69e-09	0	\\
1.8e-09	0	\\
1.9e-09	0	\\
2.01e-09	0	\\
2.11e-09	0	\\
2.21e-09	0	\\
2.32e-09	0	\\
2.42e-09	0	\\
2.52e-09	0	\\
2.63e-09	0	\\
2.73e-09	0	\\
2.83e-09	0	\\
2.93e-09	0	\\
3.04e-09	0	\\
3.14e-09	0	\\
3.24e-09	0	\\
3.34e-09	0	\\
3.45e-09	0	\\
3.55e-09	0	\\
3.65e-09	0	\\
3.75e-09	0	\\
3.86e-09	0	\\
3.96e-09	0	\\
4.06e-09	0	\\
4.16e-09	0	\\
4.27e-09	0	\\
4.37e-09	0	\\
4.47e-09	0	\\
4.57e-09	0	\\
4.68e-09	0	\\
4.78e-09	0	\\
4.89e-09	0	\\
4.99e-09	0	\\
5e-09	0	\\
};
\addplot [color=blue,solid,forget plot]
  table[row sep=crcr]{
0	0	\\
1.1e-10	0	\\
2.2e-10	0	\\
3.3e-10	0	\\
4.4e-10	0	\\
5.4e-10	0	\\
6.5e-10	0	\\
7.5e-10	0	\\
8.6e-10	0	\\
9.6e-10	0	\\
1.07e-09	0	\\
1.18e-09	0	\\
1.28e-09	0	\\
1.38e-09	0	\\
1.49e-09	0	\\
1.59e-09	0	\\
1.69e-09	0	\\
1.8e-09	0	\\
1.9e-09	0	\\
2.01e-09	0	\\
2.11e-09	0	\\
2.21e-09	0	\\
2.32e-09	0	\\
2.42e-09	0	\\
2.52e-09	0	\\
2.63e-09	0	\\
2.73e-09	0	\\
2.83e-09	0	\\
2.93e-09	0	\\
3.04e-09	0	\\
3.14e-09	0	\\
3.24e-09	0	\\
3.34e-09	0	\\
3.45e-09	0	\\
3.55e-09	0	\\
3.65e-09	0	\\
3.75e-09	0	\\
3.86e-09	0	\\
3.96e-09	0	\\
4.06e-09	0	\\
4.16e-09	0	\\
4.27e-09	0	\\
4.37e-09	0	\\
4.47e-09	0	\\
4.57e-09	0	\\
4.68e-09	0	\\
4.78e-09	0	\\
4.89e-09	0	\\
4.99e-09	0	\\
5e-09	0	\\
};
\addplot [color=black!50!green,solid,forget plot]
  table[row sep=crcr]{
0	0	\\
1.1e-10	0	\\
2.2e-10	0	\\
3.3e-10	0	\\
4.4e-10	0	\\
5.4e-10	0	\\
6.5e-10	0	\\
7.5e-10	0	\\
8.6e-10	0	\\
9.6e-10	0	\\
1.07e-09	0	\\
1.18e-09	0	\\
1.28e-09	0	\\
1.38e-09	0	\\
1.49e-09	0	\\
1.59e-09	0	\\
1.69e-09	0	\\
1.8e-09	0	\\
1.9e-09	0	\\
2.01e-09	0	\\
2.11e-09	0	\\
2.21e-09	0	\\
2.32e-09	0	\\
2.42e-09	0	\\
2.52e-09	0	\\
2.63e-09	0	\\
2.73e-09	0	\\
2.83e-09	0	\\
2.93e-09	0	\\
3.04e-09	0	\\
3.14e-09	0	\\
3.24e-09	0	\\
3.34e-09	0	\\
3.45e-09	0	\\
3.55e-09	0	\\
3.65e-09	0	\\
3.75e-09	0	\\
3.86e-09	0	\\
3.96e-09	0	\\
4.06e-09	0	\\
4.16e-09	0	\\
4.27e-09	0	\\
4.37e-09	0	\\
4.47e-09	0	\\
4.57e-09	0	\\
4.68e-09	0	\\
4.78e-09	0	\\
4.89e-09	0	\\
4.99e-09	0	\\
5e-09	0	\\
};
\addplot [color=red,solid,forget plot]
  table[row sep=crcr]{
0	0	\\
1.1e-10	0	\\
2.2e-10	0	\\
3.3e-10	0	\\
4.4e-10	0	\\
5.4e-10	0	\\
6.5e-10	0	\\
7.5e-10	0	\\
8.6e-10	0	\\
9.6e-10	0	\\
1.07e-09	0	\\
1.18e-09	0	\\
1.28e-09	0	\\
1.38e-09	0	\\
1.49e-09	0	\\
1.59e-09	0	\\
1.69e-09	0	\\
1.8e-09	0	\\
1.9e-09	0	\\
2.01e-09	0	\\
2.11e-09	0	\\
2.21e-09	0	\\
2.32e-09	0	\\
2.42e-09	0	\\
2.52e-09	0	\\
2.63e-09	0	\\
2.73e-09	0	\\
2.83e-09	0	\\
2.93e-09	0	\\
3.04e-09	0	\\
3.14e-09	0	\\
3.24e-09	0	\\
3.34e-09	0	\\
3.45e-09	0	\\
3.55e-09	0	\\
3.65e-09	0	\\
3.75e-09	0	\\
3.86e-09	0	\\
3.96e-09	0	\\
4.06e-09	0	\\
4.16e-09	0	\\
4.27e-09	0	\\
4.37e-09	0	\\
4.47e-09	0	\\
4.57e-09	0	\\
4.68e-09	0	\\
4.78e-09	0	\\
4.89e-09	0	\\
4.99e-09	0	\\
5e-09	0	\\
};
\addplot [color=mycolor1,solid,forget plot]
  table[row sep=crcr]{
0	0	\\
1.1e-10	0	\\
2.2e-10	0	\\
3.3e-10	0	\\
4.4e-10	0	\\
5.4e-10	0	\\
6.5e-10	0	\\
7.5e-10	0	\\
8.6e-10	0	\\
9.6e-10	0	\\
1.07e-09	0	\\
1.18e-09	0	\\
1.28e-09	0	\\
1.38e-09	0	\\
1.49e-09	0	\\
1.59e-09	0	\\
1.69e-09	0	\\
1.8e-09	0	\\
1.9e-09	0	\\
2.01e-09	0	\\
2.11e-09	0	\\
2.21e-09	0	\\
2.32e-09	0	\\
2.42e-09	0	\\
2.52e-09	0	\\
2.63e-09	0	\\
2.73e-09	0	\\
2.83e-09	0	\\
2.93e-09	0	\\
3.04e-09	0	\\
3.14e-09	0	\\
3.24e-09	0	\\
3.34e-09	0	\\
3.45e-09	0	\\
3.55e-09	0	\\
3.65e-09	0	\\
3.75e-09	0	\\
3.86e-09	0	\\
3.96e-09	0	\\
4.06e-09	0	\\
4.16e-09	0	\\
4.27e-09	0	\\
4.37e-09	0	\\
4.47e-09	0	\\
4.57e-09	0	\\
4.68e-09	0	\\
4.78e-09	0	\\
4.89e-09	0	\\
4.99e-09	0	\\
5e-09	0	\\
};
\addplot [color=mycolor2,solid,forget plot]
  table[row sep=crcr]{
0	0	\\
1.1e-10	0	\\
2.2e-10	0	\\
3.3e-10	0	\\
4.4e-10	0	\\
5.4e-10	0	\\
6.5e-10	0	\\
7.5e-10	0	\\
8.6e-10	0	\\
9.6e-10	0	\\
1.07e-09	0	\\
1.18e-09	0	\\
1.28e-09	0	\\
1.38e-09	0	\\
1.49e-09	0	\\
1.59e-09	0	\\
1.69e-09	0	\\
1.8e-09	0	\\
1.9e-09	0	\\
2.01e-09	0	\\
2.11e-09	0	\\
2.21e-09	0	\\
2.32e-09	0	\\
2.42e-09	0	\\
2.52e-09	0	\\
2.63e-09	0	\\
2.73e-09	0	\\
2.83e-09	0	\\
2.93e-09	0	\\
3.04e-09	0	\\
3.14e-09	0	\\
3.24e-09	0	\\
3.34e-09	0	\\
3.45e-09	0	\\
3.55e-09	0	\\
3.65e-09	0	\\
3.75e-09	0	\\
3.86e-09	0	\\
3.96e-09	0	\\
4.06e-09	0	\\
4.16e-09	0	\\
4.27e-09	0	\\
4.37e-09	0	\\
4.47e-09	0	\\
4.57e-09	0	\\
4.68e-09	0	\\
4.78e-09	0	\\
4.89e-09	0	\\
4.99e-09	0	\\
5e-09	0	\\
};
\addplot [color=mycolor3,solid,forget plot]
  table[row sep=crcr]{
0	0	\\
1.1e-10	0	\\
2.2e-10	0	\\
3.3e-10	0	\\
4.4e-10	0	\\
5.4e-10	0	\\
6.5e-10	0	\\
7.5e-10	0	\\
8.6e-10	0	\\
9.6e-10	0	\\
1.07e-09	0	\\
1.18e-09	0	\\
1.28e-09	0	\\
1.38e-09	0	\\
1.49e-09	0	\\
1.59e-09	0	\\
1.69e-09	0	\\
1.8e-09	0	\\
1.9e-09	0	\\
2.01e-09	0	\\
2.11e-09	0	\\
2.21e-09	0	\\
2.32e-09	0	\\
2.42e-09	0	\\
2.52e-09	0	\\
2.63e-09	0	\\
2.73e-09	0	\\
2.83e-09	0	\\
2.93e-09	0	\\
3.04e-09	0	\\
3.14e-09	0	\\
3.24e-09	0	\\
3.34e-09	0	\\
3.45e-09	0	\\
3.55e-09	0	\\
3.65e-09	0	\\
3.75e-09	0	\\
3.86e-09	0	\\
3.96e-09	0	\\
4.06e-09	0	\\
4.16e-09	0	\\
4.27e-09	0	\\
4.37e-09	0	\\
4.47e-09	0	\\
4.57e-09	0	\\
4.68e-09	0	\\
4.78e-09	0	\\
4.89e-09	0	\\
4.99e-09	0	\\
5e-09	0	\\
};
\addplot [color=darkgray,solid,forget plot]
  table[row sep=crcr]{
0	0	\\
1.1e-10	0	\\
2.2e-10	0	\\
3.3e-10	0	\\
4.4e-10	0	\\
5.4e-10	0	\\
6.5e-10	0	\\
7.5e-10	0	\\
8.6e-10	0	\\
9.6e-10	0	\\
1.07e-09	0	\\
1.18e-09	0	\\
1.28e-09	0	\\
1.38e-09	0	\\
1.49e-09	0	\\
1.59e-09	0	\\
1.69e-09	0	\\
1.8e-09	0	\\
1.9e-09	0	\\
2.01e-09	0	\\
2.11e-09	0	\\
2.21e-09	0	\\
2.32e-09	0	\\
2.42e-09	0	\\
2.52e-09	0	\\
2.63e-09	0	\\
2.73e-09	0	\\
2.83e-09	0	\\
2.93e-09	0	\\
3.04e-09	0	\\
3.14e-09	0	\\
3.24e-09	0	\\
3.34e-09	0	\\
3.45e-09	0	\\
3.55e-09	0	\\
3.65e-09	0	\\
3.75e-09	0	\\
3.86e-09	0	\\
3.96e-09	0	\\
4.06e-09	0	\\
4.16e-09	0	\\
4.27e-09	0	\\
4.37e-09	0	\\
4.47e-09	0	\\
4.57e-09	0	\\
4.68e-09	0	\\
4.78e-09	0	\\
4.89e-09	0	\\
4.99e-09	0	\\
5e-09	0	\\
};
\addplot [color=blue,solid,forget plot]
  table[row sep=crcr]{
0	0	\\
1.1e-10	0	\\
2.2e-10	0	\\
3.3e-10	0	\\
4.4e-10	0	\\
5.4e-10	0	\\
6.5e-10	0	\\
7.5e-10	0	\\
8.6e-10	0	\\
9.6e-10	0	\\
1.07e-09	0	\\
1.18e-09	0	\\
1.28e-09	0	\\
1.38e-09	0	\\
1.49e-09	0	\\
1.59e-09	0	\\
1.69e-09	0	\\
1.8e-09	0	\\
1.9e-09	0	\\
2.01e-09	0	\\
2.11e-09	0	\\
2.21e-09	0	\\
2.32e-09	0	\\
2.42e-09	0	\\
2.52e-09	0	\\
2.63e-09	0	\\
2.73e-09	0	\\
2.83e-09	0	\\
2.93e-09	0	\\
3.04e-09	0	\\
3.14e-09	0	\\
3.24e-09	0	\\
3.34e-09	0	\\
3.45e-09	0	\\
3.55e-09	0	\\
3.65e-09	0	\\
3.75e-09	0	\\
3.86e-09	0	\\
3.96e-09	0	\\
4.06e-09	0	\\
4.16e-09	0	\\
4.27e-09	0	\\
4.37e-09	0	\\
4.47e-09	0	\\
4.57e-09	0	\\
4.68e-09	0	\\
4.78e-09	0	\\
4.89e-09	0	\\
4.99e-09	0	\\
5e-09	0	\\
};
\addplot [color=black!50!green,solid,forget plot]
  table[row sep=crcr]{
0	0	\\
1.1e-10	0	\\
2.2e-10	0	\\
3.3e-10	0	\\
4.4e-10	0	\\
5.4e-10	0	\\
6.5e-10	0	\\
7.5e-10	0	\\
8.6e-10	0	\\
9.6e-10	0	\\
1.07e-09	0	\\
1.18e-09	0	\\
1.28e-09	0	\\
1.38e-09	0	\\
1.49e-09	0	\\
1.59e-09	0	\\
1.69e-09	0	\\
1.8e-09	0	\\
1.9e-09	0	\\
2.01e-09	0	\\
2.11e-09	0	\\
2.21e-09	0	\\
2.32e-09	0	\\
2.42e-09	0	\\
2.52e-09	0	\\
2.63e-09	0	\\
2.73e-09	0	\\
2.83e-09	0	\\
2.93e-09	0	\\
3.04e-09	0	\\
3.14e-09	0	\\
3.24e-09	0	\\
3.34e-09	0	\\
3.45e-09	0	\\
3.55e-09	0	\\
3.65e-09	0	\\
3.75e-09	0	\\
3.86e-09	0	\\
3.96e-09	0	\\
4.06e-09	0	\\
4.16e-09	0	\\
4.27e-09	0	\\
4.37e-09	0	\\
4.47e-09	0	\\
4.57e-09	0	\\
4.68e-09	0	\\
4.78e-09	0	\\
4.89e-09	0	\\
4.99e-09	0	\\
5e-09	0	\\
};
\addplot [color=red,solid,forget plot]
  table[row sep=crcr]{
0	0	\\
1.1e-10	0	\\
2.2e-10	0	\\
3.3e-10	0	\\
4.4e-10	0	\\
5.4e-10	0	\\
6.5e-10	0	\\
7.5e-10	0	\\
8.6e-10	0	\\
9.6e-10	0	\\
1.07e-09	0	\\
1.18e-09	0	\\
1.28e-09	0	\\
1.38e-09	0	\\
1.49e-09	0	\\
1.59e-09	0	\\
1.69e-09	0	\\
1.8e-09	0	\\
1.9e-09	0	\\
2.01e-09	0	\\
2.11e-09	0	\\
2.21e-09	0	\\
2.32e-09	0	\\
2.42e-09	0	\\
2.52e-09	0	\\
2.63e-09	0	\\
2.73e-09	0	\\
2.83e-09	0	\\
2.93e-09	0	\\
3.04e-09	0	\\
3.14e-09	0	\\
3.24e-09	0	\\
3.34e-09	0	\\
3.45e-09	0	\\
3.55e-09	0	\\
3.65e-09	0	\\
3.75e-09	0	\\
3.86e-09	0	\\
3.96e-09	0	\\
4.06e-09	0	\\
4.16e-09	0	\\
4.27e-09	0	\\
4.37e-09	0	\\
4.47e-09	0	\\
4.57e-09	0	\\
4.68e-09	0	\\
4.78e-09	0	\\
4.89e-09	0	\\
4.99e-09	0	\\
5e-09	0	\\
};
\addplot [color=mycolor1,solid,forget plot]
  table[row sep=crcr]{
0	0	\\
1.1e-10	0	\\
2.2e-10	0	\\
3.3e-10	0	\\
4.4e-10	0	\\
5.4e-10	0	\\
6.5e-10	0	\\
7.5e-10	0	\\
8.6e-10	0	\\
9.6e-10	0	\\
1.07e-09	0	\\
1.18e-09	0	\\
1.28e-09	0	\\
1.38e-09	0	\\
1.49e-09	0	\\
1.59e-09	0	\\
1.69e-09	0	\\
1.8e-09	0	\\
1.9e-09	0	\\
2.01e-09	0	\\
2.11e-09	0	\\
2.21e-09	0	\\
2.32e-09	0	\\
2.42e-09	0	\\
2.52e-09	0	\\
2.63e-09	0	\\
2.73e-09	0	\\
2.83e-09	0	\\
2.93e-09	0	\\
3.04e-09	0	\\
3.14e-09	0	\\
3.24e-09	0	\\
3.34e-09	0	\\
3.45e-09	0	\\
3.55e-09	0	\\
3.65e-09	0	\\
3.75e-09	0	\\
3.86e-09	0	\\
3.96e-09	0	\\
4.06e-09	0	\\
4.16e-09	0	\\
4.27e-09	0	\\
4.37e-09	0	\\
4.47e-09	0	\\
4.57e-09	0	\\
4.68e-09	0	\\
4.78e-09	0	\\
4.89e-09	0	\\
4.99e-09	0	\\
5e-09	0	\\
};
\addplot [color=mycolor2,solid,forget plot]
  table[row sep=crcr]{
0	0	\\
1.1e-10	0	\\
2.2e-10	0	\\
3.3e-10	0	\\
4.4e-10	0	\\
5.4e-10	0	\\
6.5e-10	0	\\
7.5e-10	0	\\
8.6e-10	0	\\
9.6e-10	0	\\
1.07e-09	0	\\
1.18e-09	0	\\
1.28e-09	0	\\
1.38e-09	0	\\
1.49e-09	0	\\
1.59e-09	0	\\
1.69e-09	0	\\
1.8e-09	0	\\
1.9e-09	0	\\
2.01e-09	0	\\
2.11e-09	0	\\
2.21e-09	0	\\
2.32e-09	0	\\
2.42e-09	0	\\
2.52e-09	0	\\
2.63e-09	0	\\
2.73e-09	0	\\
2.83e-09	0	\\
2.93e-09	0	\\
3.04e-09	0	\\
3.14e-09	0	\\
3.24e-09	0	\\
3.34e-09	0	\\
3.45e-09	0	\\
3.55e-09	0	\\
3.65e-09	0	\\
3.75e-09	0	\\
3.86e-09	0	\\
3.96e-09	0	\\
4.06e-09	0	\\
4.16e-09	0	\\
4.27e-09	0	\\
4.37e-09	0	\\
4.47e-09	0	\\
4.57e-09	0	\\
4.68e-09	0	\\
4.78e-09	0	\\
4.89e-09	0	\\
4.99e-09	0	\\
5e-09	0	\\
};
\addplot [color=mycolor3,solid,forget plot]
  table[row sep=crcr]{
0	0	\\
1.1e-10	0	\\
2.2e-10	0	\\
3.3e-10	0	\\
4.4e-10	0	\\
5.4e-10	0	\\
6.5e-10	0	\\
7.5e-10	0	\\
8.6e-10	0	\\
9.6e-10	0	\\
1.07e-09	0	\\
1.18e-09	0	\\
1.28e-09	0	\\
1.38e-09	0	\\
1.49e-09	0	\\
1.59e-09	0	\\
1.69e-09	0	\\
1.8e-09	0	\\
1.9e-09	0	\\
2.01e-09	0	\\
2.11e-09	0	\\
2.21e-09	0	\\
2.32e-09	0	\\
2.42e-09	0	\\
2.52e-09	0	\\
2.63e-09	0	\\
2.73e-09	0	\\
2.83e-09	0	\\
2.93e-09	0	\\
3.04e-09	0	\\
3.14e-09	0	\\
3.24e-09	0	\\
3.34e-09	0	\\
3.45e-09	0	\\
3.55e-09	0	\\
3.65e-09	0	\\
3.75e-09	0	\\
3.86e-09	0	\\
3.96e-09	0	\\
4.06e-09	0	\\
4.16e-09	0	\\
4.27e-09	0	\\
4.37e-09	0	\\
4.47e-09	0	\\
4.57e-09	0	\\
4.68e-09	0	\\
4.78e-09	0	\\
4.89e-09	0	\\
4.99e-09	0	\\
5e-09	0	\\
};
\addplot [color=darkgray,solid,forget plot]
  table[row sep=crcr]{
0	0	\\
1.1e-10	0	\\
2.2e-10	0	\\
3.3e-10	0	\\
4.4e-10	0	\\
5.4e-10	0	\\
6.5e-10	0	\\
7.5e-10	0	\\
8.6e-10	0	\\
9.6e-10	0	\\
1.07e-09	0	\\
1.18e-09	0	\\
1.28e-09	0	\\
1.38e-09	0	\\
1.49e-09	0	\\
1.59e-09	0	\\
1.69e-09	0	\\
1.8e-09	0	\\
1.9e-09	0	\\
2.01e-09	0	\\
2.11e-09	0	\\
2.21e-09	0	\\
2.32e-09	0	\\
2.42e-09	0	\\
2.52e-09	0	\\
2.63e-09	0	\\
2.73e-09	0	\\
2.83e-09	0	\\
2.93e-09	0	\\
3.04e-09	0	\\
3.14e-09	0	\\
3.24e-09	0	\\
3.34e-09	0	\\
3.45e-09	0	\\
3.55e-09	0	\\
3.65e-09	0	\\
3.75e-09	0	\\
3.86e-09	0	\\
3.96e-09	0	\\
4.06e-09	0	\\
4.16e-09	0	\\
4.27e-09	0	\\
4.37e-09	0	\\
4.47e-09	0	\\
4.57e-09	0	\\
4.68e-09	0	\\
4.78e-09	0	\\
4.89e-09	0	\\
4.99e-09	0	\\
5e-09	0	\\
};
\addplot [color=blue,solid,forget plot]
  table[row sep=crcr]{
0	0	\\
1.1e-10	0	\\
2.2e-10	0	\\
3.3e-10	0	\\
4.4e-10	0	\\
5.4e-10	0	\\
6.5e-10	0	\\
7.5e-10	0	\\
8.6e-10	0	\\
9.6e-10	0	\\
1.07e-09	0	\\
1.18e-09	0	\\
1.28e-09	0	\\
1.38e-09	0	\\
1.49e-09	0	\\
1.59e-09	0	\\
1.69e-09	0	\\
1.8e-09	0	\\
1.9e-09	0	\\
2.01e-09	0	\\
2.11e-09	0	\\
2.21e-09	0	\\
2.32e-09	0	\\
2.42e-09	0	\\
2.52e-09	0	\\
2.63e-09	0	\\
2.73e-09	0	\\
2.83e-09	0	\\
2.93e-09	0	\\
3.04e-09	0	\\
3.14e-09	0	\\
3.24e-09	0	\\
3.34e-09	0	\\
3.45e-09	0	\\
3.55e-09	0	\\
3.65e-09	0	\\
3.75e-09	0	\\
3.86e-09	0	\\
3.96e-09	0	\\
4.06e-09	0	\\
4.16e-09	0	\\
4.27e-09	0	\\
4.37e-09	0	\\
4.47e-09	0	\\
4.57e-09	0	\\
4.68e-09	0	\\
4.78e-09	0	\\
4.89e-09	0	\\
4.99e-09	0	\\
5e-09	0	\\
};
\addplot [color=black!50!green,solid,forget plot]
  table[row sep=crcr]{
0	0	\\
1.1e-10	0	\\
2.2e-10	0	\\
3.3e-10	0	\\
4.4e-10	0	\\
5.4e-10	0	\\
6.5e-10	0	\\
7.5e-10	0	\\
8.6e-10	0	\\
9.6e-10	0	\\
1.07e-09	0	\\
1.18e-09	0	\\
1.28e-09	0	\\
1.38e-09	0	\\
1.49e-09	0	\\
1.59e-09	0	\\
1.69e-09	0	\\
1.8e-09	0	\\
1.9e-09	0	\\
2.01e-09	0	\\
2.11e-09	0	\\
2.21e-09	0	\\
2.32e-09	0	\\
2.42e-09	0	\\
2.52e-09	0	\\
2.63e-09	0	\\
2.73e-09	0	\\
2.83e-09	0	\\
2.93e-09	0	\\
3.04e-09	0	\\
3.14e-09	0	\\
3.24e-09	0	\\
3.34e-09	0	\\
3.45e-09	0	\\
3.55e-09	0	\\
3.65e-09	0	\\
3.75e-09	0	\\
3.86e-09	0	\\
3.96e-09	0	\\
4.06e-09	0	\\
4.16e-09	0	\\
4.27e-09	0	\\
4.37e-09	0	\\
4.47e-09	0	\\
4.57e-09	0	\\
4.68e-09	0	\\
4.78e-09	0	\\
4.89e-09	0	\\
4.99e-09	0	\\
5e-09	0	\\
};
\addplot [color=red,solid,forget plot]
  table[row sep=crcr]{
0	0	\\
1.1e-10	0	\\
2.2e-10	0	\\
3.3e-10	0	\\
4.4e-10	0	\\
5.4e-10	0	\\
6.5e-10	0	\\
7.5e-10	0	\\
8.6e-10	0	\\
9.6e-10	0	\\
1.07e-09	0	\\
1.18e-09	0	\\
1.28e-09	0	\\
1.38e-09	0	\\
1.49e-09	0	\\
1.59e-09	0	\\
1.69e-09	0	\\
1.8e-09	0	\\
1.9e-09	0	\\
2.01e-09	0	\\
2.11e-09	0	\\
2.21e-09	0	\\
2.32e-09	0	\\
2.42e-09	0	\\
2.52e-09	0	\\
2.63e-09	0	\\
2.73e-09	0	\\
2.83e-09	0	\\
2.93e-09	0	\\
3.04e-09	0	\\
3.14e-09	0	\\
3.24e-09	0	\\
3.34e-09	0	\\
3.45e-09	0	\\
3.55e-09	0	\\
3.65e-09	0	\\
3.75e-09	0	\\
3.86e-09	0	\\
3.96e-09	0	\\
4.06e-09	0	\\
4.16e-09	0	\\
4.27e-09	0	\\
4.37e-09	0	\\
4.47e-09	0	\\
4.57e-09	0	\\
4.68e-09	0	\\
4.78e-09	0	\\
4.89e-09	0	\\
4.99e-09	0	\\
5e-09	0	\\
};
\addplot [color=mycolor1,solid,forget plot]
  table[row sep=crcr]{
0	0	\\
1.1e-10	0	\\
2.2e-10	0	\\
3.3e-10	0	\\
4.4e-10	0	\\
5.4e-10	0	\\
6.5e-10	0	\\
7.5e-10	0	\\
8.6e-10	0	\\
9.6e-10	0	\\
1.07e-09	0	\\
1.18e-09	0	\\
1.28e-09	0	\\
1.38e-09	0	\\
1.49e-09	0	\\
1.59e-09	0	\\
1.69e-09	0	\\
1.8e-09	0	\\
1.9e-09	0	\\
2.01e-09	0	\\
2.11e-09	0	\\
2.21e-09	0	\\
2.32e-09	0	\\
2.42e-09	0	\\
2.52e-09	0	\\
2.63e-09	0	\\
2.73e-09	0	\\
2.83e-09	0	\\
2.93e-09	0	\\
3.04e-09	0	\\
3.14e-09	0	\\
3.24e-09	0	\\
3.34e-09	0	\\
3.45e-09	0	\\
3.55e-09	0	\\
3.65e-09	0	\\
3.75e-09	0	\\
3.86e-09	0	\\
3.96e-09	0	\\
4.06e-09	0	\\
4.16e-09	0	\\
4.27e-09	0	\\
4.37e-09	0	\\
4.47e-09	0	\\
4.57e-09	0	\\
4.68e-09	0	\\
4.78e-09	0	\\
4.89e-09	0	\\
4.99e-09	0	\\
5e-09	0	\\
};
\addplot [color=mycolor2,solid,forget plot]
  table[row sep=crcr]{
0	0	\\
1.1e-10	0	\\
2.2e-10	0	\\
3.3e-10	0	\\
4.4e-10	0	\\
5.4e-10	0	\\
6.5e-10	0	\\
7.5e-10	0	\\
8.6e-10	0	\\
9.6e-10	0	\\
1.07e-09	0	\\
1.18e-09	0	\\
1.28e-09	0	\\
1.38e-09	0	\\
1.49e-09	0	\\
1.59e-09	0	\\
1.69e-09	0	\\
1.8e-09	0	\\
1.9e-09	0	\\
2.01e-09	0	\\
2.11e-09	0	\\
2.21e-09	0	\\
2.32e-09	0	\\
2.42e-09	0	\\
2.52e-09	0	\\
2.63e-09	0	\\
2.73e-09	0	\\
2.83e-09	0	\\
2.93e-09	0	\\
3.04e-09	0	\\
3.14e-09	0	\\
3.24e-09	0	\\
3.34e-09	0	\\
3.45e-09	0	\\
3.55e-09	0	\\
3.65e-09	0	\\
3.75e-09	0	\\
3.86e-09	0	\\
3.96e-09	0	\\
4.06e-09	0	\\
4.16e-09	0	\\
4.27e-09	0	\\
4.37e-09	0	\\
4.47e-09	0	\\
4.57e-09	0	\\
4.68e-09	0	\\
4.78e-09	0	\\
4.89e-09	0	\\
4.99e-09	0	\\
5e-09	0	\\
};
\addplot [color=mycolor3,solid,forget plot]
  table[row sep=crcr]{
0	0	\\
1.1e-10	0	\\
2.2e-10	0	\\
3.3e-10	0	\\
4.4e-10	0	\\
5.4e-10	0	\\
6.5e-10	0	\\
7.5e-10	0	\\
8.6e-10	0	\\
9.6e-10	0	\\
1.07e-09	0	\\
1.18e-09	0	\\
1.28e-09	0	\\
1.38e-09	0	\\
1.49e-09	0	\\
1.59e-09	0	\\
1.69e-09	0	\\
1.8e-09	0	\\
1.9e-09	0	\\
2.01e-09	0	\\
2.11e-09	0	\\
2.21e-09	0	\\
2.32e-09	0	\\
2.42e-09	0	\\
2.52e-09	0	\\
2.63e-09	0	\\
2.73e-09	0	\\
2.83e-09	0	\\
2.93e-09	0	\\
3.04e-09	0	\\
3.14e-09	0	\\
3.24e-09	0	\\
3.34e-09	0	\\
3.45e-09	0	\\
3.55e-09	0	\\
3.65e-09	0	\\
3.75e-09	0	\\
3.86e-09	0	\\
3.96e-09	0	\\
4.06e-09	0	\\
4.16e-09	0	\\
4.27e-09	0	\\
4.37e-09	0	\\
4.47e-09	0	\\
4.57e-09	0	\\
4.68e-09	0	\\
4.78e-09	0	\\
4.89e-09	0	\\
4.99e-09	0	\\
5e-09	0	\\
};
\addplot [color=darkgray,solid,forget plot]
  table[row sep=crcr]{
0	0	\\
1.1e-10	0	\\
2.2e-10	0	\\
3.3e-10	0	\\
4.4e-10	0	\\
5.4e-10	0	\\
6.5e-10	0	\\
7.5e-10	0	\\
8.6e-10	0	\\
9.6e-10	0	\\
1.07e-09	0	\\
1.18e-09	0	\\
1.28e-09	0	\\
1.38e-09	0	\\
1.49e-09	0	\\
1.59e-09	0	\\
1.69e-09	0	\\
1.8e-09	0	\\
1.9e-09	0	\\
2.01e-09	0	\\
2.11e-09	0	\\
2.21e-09	0	\\
2.32e-09	0	\\
2.42e-09	0	\\
2.52e-09	0	\\
2.63e-09	0	\\
2.73e-09	0	\\
2.83e-09	0	\\
2.93e-09	0	\\
3.04e-09	0	\\
3.14e-09	0	\\
3.24e-09	0	\\
3.34e-09	0	\\
3.45e-09	0	\\
3.55e-09	0	\\
3.65e-09	0	\\
3.75e-09	0	\\
3.86e-09	0	\\
3.96e-09	0	\\
4.06e-09	0	\\
4.16e-09	0	\\
4.27e-09	0	\\
4.37e-09	0	\\
4.47e-09	0	\\
4.57e-09	0	\\
4.68e-09	0	\\
4.78e-09	0	\\
4.89e-09	0	\\
4.99e-09	0	\\
5e-09	0	\\
};
\addplot [color=blue,solid,forget plot]
  table[row sep=crcr]{
0	0	\\
1.1e-10	0	\\
2.2e-10	0	\\
3.3e-10	0	\\
4.4e-10	0	\\
5.4e-10	0	\\
6.5e-10	0	\\
7.5e-10	0	\\
8.6e-10	0	\\
9.6e-10	0	\\
1.07e-09	0	\\
1.18e-09	0	\\
1.28e-09	0	\\
1.38e-09	0	\\
1.49e-09	0	\\
1.59e-09	0	\\
1.69e-09	0	\\
1.8e-09	0	\\
1.9e-09	0	\\
2.01e-09	0	\\
2.11e-09	0	\\
2.21e-09	0	\\
2.32e-09	0	\\
2.42e-09	0	\\
2.52e-09	0	\\
2.63e-09	0	\\
2.73e-09	0	\\
2.83e-09	0	\\
2.93e-09	0	\\
3.04e-09	0	\\
3.14e-09	0	\\
3.24e-09	0	\\
3.34e-09	0	\\
3.45e-09	0	\\
3.55e-09	0	\\
3.65e-09	0	\\
3.75e-09	0	\\
3.86e-09	0	\\
3.96e-09	0	\\
4.06e-09	0	\\
4.16e-09	0	\\
4.27e-09	0	\\
4.37e-09	0	\\
4.47e-09	0	\\
4.57e-09	0	\\
4.68e-09	0	\\
4.78e-09	0	\\
4.89e-09	0	\\
4.99e-09	0	\\
5e-09	0	\\
};
\addplot [color=black!50!green,solid,forget plot]
  table[row sep=crcr]{
0	0	\\
1.1e-10	0	\\
2.2e-10	0	\\
3.3e-10	0	\\
4.4e-10	0	\\
5.4e-10	0	\\
6.5e-10	0	\\
7.5e-10	0	\\
8.6e-10	0	\\
9.6e-10	0	\\
1.07e-09	0	\\
1.18e-09	0	\\
1.28e-09	0	\\
1.38e-09	0	\\
1.49e-09	0	\\
1.59e-09	0	\\
1.69e-09	0	\\
1.8e-09	0	\\
1.9e-09	0	\\
2.01e-09	0	\\
2.11e-09	0	\\
2.21e-09	0	\\
2.32e-09	0	\\
2.42e-09	0	\\
2.52e-09	0	\\
2.63e-09	0	\\
2.73e-09	0	\\
2.83e-09	0	\\
2.93e-09	0	\\
3.04e-09	0	\\
3.14e-09	0	\\
3.24e-09	0	\\
3.34e-09	0	\\
3.45e-09	0	\\
3.55e-09	0	\\
3.65e-09	0	\\
3.75e-09	0	\\
3.86e-09	0	\\
3.96e-09	0	\\
4.06e-09	0	\\
4.16e-09	0	\\
4.27e-09	0	\\
4.37e-09	0	\\
4.47e-09	0	\\
4.57e-09	0	\\
4.68e-09	0	\\
4.78e-09	0	\\
4.89e-09	0	\\
4.99e-09	0	\\
5e-09	0	\\
};
\addplot [color=red,solid,forget plot]
  table[row sep=crcr]{
0	0	\\
1.1e-10	0	\\
2.2e-10	0	\\
3.3e-10	0	\\
4.4e-10	0	\\
5.4e-10	0	\\
6.5e-10	0	\\
7.5e-10	0	\\
8.6e-10	0	\\
9.6e-10	0	\\
1.07e-09	0	\\
1.18e-09	0	\\
1.28e-09	0	\\
1.38e-09	0	\\
1.49e-09	0	\\
1.59e-09	0	\\
1.69e-09	0	\\
1.8e-09	0	\\
1.9e-09	0	\\
2.01e-09	0	\\
2.11e-09	0	\\
2.21e-09	0	\\
2.32e-09	0	\\
2.42e-09	0	\\
2.52e-09	0	\\
2.63e-09	0	\\
2.73e-09	0	\\
2.83e-09	0	\\
2.93e-09	0	\\
3.04e-09	0	\\
3.14e-09	0	\\
3.24e-09	0	\\
3.34e-09	0	\\
3.45e-09	0	\\
3.55e-09	0	\\
3.65e-09	0	\\
3.75e-09	0	\\
3.86e-09	0	\\
3.96e-09	0	\\
4.06e-09	0	\\
4.16e-09	0	\\
4.27e-09	0	\\
4.37e-09	0	\\
4.47e-09	0	\\
4.57e-09	0	\\
4.68e-09	0	\\
4.78e-09	0	\\
4.89e-09	0	\\
4.99e-09	0	\\
5e-09	0	\\
};
\addplot [color=mycolor1,solid,forget plot]
  table[row sep=crcr]{
0	0	\\
1.1e-10	0	\\
2.2e-10	0	\\
3.3e-10	0	\\
4.4e-10	0	\\
5.4e-10	0	\\
6.5e-10	0	\\
7.5e-10	0	\\
8.6e-10	0	\\
9.6e-10	0	\\
1.07e-09	0	\\
1.18e-09	0	\\
1.28e-09	0	\\
1.38e-09	0	\\
1.49e-09	0	\\
1.59e-09	0	\\
1.69e-09	0	\\
1.8e-09	0	\\
1.9e-09	0	\\
2.01e-09	0	\\
2.11e-09	0	\\
2.21e-09	0	\\
2.32e-09	0	\\
2.42e-09	0	\\
2.52e-09	0	\\
2.63e-09	0	\\
2.73e-09	0	\\
2.83e-09	0	\\
2.93e-09	0	\\
3.04e-09	0	\\
3.14e-09	0	\\
3.24e-09	0	\\
3.34e-09	0	\\
3.45e-09	0	\\
3.55e-09	0	\\
3.65e-09	0	\\
3.75e-09	0	\\
3.86e-09	0	\\
3.96e-09	0	\\
4.06e-09	0	\\
4.16e-09	0	\\
4.27e-09	0	\\
4.37e-09	0	\\
4.47e-09	0	\\
4.57e-09	0	\\
4.68e-09	0	\\
4.78e-09	0	\\
4.89e-09	0	\\
4.99e-09	0	\\
5e-09	0	\\
};
\addplot [color=mycolor2,solid,forget plot]
  table[row sep=crcr]{
0	0	\\
1.1e-10	0	\\
2.2e-10	0	\\
3.3e-10	0	\\
4.4e-10	0	\\
5.4e-10	0	\\
6.5e-10	0	\\
7.5e-10	0	\\
8.6e-10	0	\\
9.6e-10	0	\\
1.07e-09	0	\\
1.18e-09	0	\\
1.28e-09	0	\\
1.38e-09	0	\\
1.49e-09	0	\\
1.59e-09	0	\\
1.69e-09	0	\\
1.8e-09	0	\\
1.9e-09	0	\\
2.01e-09	0	\\
2.11e-09	0	\\
2.21e-09	0	\\
2.32e-09	0	\\
2.42e-09	0	\\
2.52e-09	0	\\
2.63e-09	0	\\
2.73e-09	0	\\
2.83e-09	0	\\
2.93e-09	0	\\
3.04e-09	0	\\
3.14e-09	0	\\
3.24e-09	0	\\
3.34e-09	0	\\
3.45e-09	0	\\
3.55e-09	0	\\
3.65e-09	0	\\
3.75e-09	0	\\
3.86e-09	0	\\
3.96e-09	0	\\
4.06e-09	0	\\
4.16e-09	0	\\
4.27e-09	0	\\
4.37e-09	0	\\
4.47e-09	0	\\
4.57e-09	0	\\
4.68e-09	0	\\
4.78e-09	0	\\
4.89e-09	0	\\
4.99e-09	0	\\
5e-09	0	\\
};
\addplot [color=mycolor3,solid,forget plot]
  table[row sep=crcr]{
0	0	\\
1.1e-10	0	\\
2.2e-10	0	\\
3.3e-10	0	\\
4.4e-10	0	\\
5.4e-10	0	\\
6.5e-10	0	\\
7.5e-10	0	\\
8.6e-10	0	\\
9.6e-10	0	\\
1.07e-09	0	\\
1.18e-09	0	\\
1.28e-09	0	\\
1.38e-09	0	\\
1.49e-09	0	\\
1.59e-09	0	\\
1.69e-09	0	\\
1.8e-09	0	\\
1.9e-09	0	\\
2.01e-09	0	\\
2.11e-09	0	\\
2.21e-09	0	\\
2.32e-09	0	\\
2.42e-09	0	\\
2.52e-09	0	\\
2.63e-09	0	\\
2.73e-09	0	\\
2.83e-09	0	\\
2.93e-09	0	\\
3.04e-09	0	\\
3.14e-09	0	\\
3.24e-09	0	\\
3.34e-09	0	\\
3.45e-09	0	\\
3.55e-09	0	\\
3.65e-09	0	\\
3.75e-09	0	\\
3.86e-09	0	\\
3.96e-09	0	\\
4.06e-09	0	\\
4.16e-09	0	\\
4.27e-09	0	\\
4.37e-09	0	\\
4.47e-09	0	\\
4.57e-09	0	\\
4.68e-09	0	\\
4.78e-09	0	\\
4.89e-09	0	\\
4.99e-09	0	\\
5e-09	0	\\
};
\addplot [color=darkgray,solid,forget plot]
  table[row sep=crcr]{
0	0	\\
1.1e-10	0	\\
2.2e-10	0	\\
3.3e-10	0	\\
4.4e-10	0	\\
5.4e-10	0	\\
6.5e-10	0	\\
7.5e-10	0	\\
8.6e-10	0	\\
9.6e-10	0	\\
1.07e-09	0	\\
1.18e-09	0	\\
1.28e-09	0	\\
1.38e-09	0	\\
1.49e-09	0	\\
1.59e-09	0	\\
1.69e-09	0	\\
1.8e-09	0	\\
1.9e-09	0	\\
2.01e-09	0	\\
2.11e-09	0	\\
2.21e-09	0	\\
2.32e-09	0	\\
2.42e-09	0	\\
2.52e-09	0	\\
2.63e-09	0	\\
2.73e-09	0	\\
2.83e-09	0	\\
2.93e-09	0	\\
3.04e-09	0	\\
3.14e-09	0	\\
3.24e-09	0	\\
3.34e-09	0	\\
3.45e-09	0	\\
3.55e-09	0	\\
3.65e-09	0	\\
3.75e-09	0	\\
3.86e-09	0	\\
3.96e-09	0	\\
4.06e-09	0	\\
4.16e-09	0	\\
4.27e-09	0	\\
4.37e-09	0	\\
4.47e-09	0	\\
4.57e-09	0	\\
4.68e-09	0	\\
4.78e-09	0	\\
4.89e-09	0	\\
4.99e-09	0	\\
5e-09	0	\\
};
\addplot [color=blue,solid,forget plot]
  table[row sep=crcr]{
0	0	\\
1.1e-10	0	\\
2.2e-10	0	\\
3.3e-10	0	\\
4.4e-10	0	\\
5.4e-10	0	\\
6.5e-10	0	\\
7.5e-10	0	\\
8.6e-10	0	\\
9.6e-10	0	\\
1.07e-09	0	\\
1.18e-09	0	\\
1.28e-09	0	\\
1.38e-09	0	\\
1.49e-09	0	\\
1.59e-09	0	\\
1.69e-09	0	\\
1.8e-09	0	\\
1.9e-09	0	\\
2.01e-09	0	\\
2.11e-09	0	\\
2.21e-09	0	\\
2.32e-09	0	\\
2.42e-09	0	\\
2.52e-09	0	\\
2.63e-09	0	\\
2.73e-09	0	\\
2.83e-09	0	\\
2.93e-09	0	\\
3.04e-09	0	\\
3.14e-09	0	\\
3.24e-09	0	\\
3.34e-09	0	\\
3.45e-09	0	\\
3.55e-09	0	\\
3.65e-09	0	\\
3.75e-09	0	\\
3.86e-09	0	\\
3.96e-09	0	\\
4.06e-09	0	\\
4.16e-09	0	\\
4.27e-09	0	\\
4.37e-09	0	\\
4.47e-09	0	\\
4.57e-09	0	\\
4.68e-09	0	\\
4.78e-09	0	\\
4.89e-09	0	\\
4.99e-09	0	\\
5e-09	0	\\
};
\addplot [color=black!50!green,solid,forget plot]
  table[row sep=crcr]{
0	0	\\
1.1e-10	0	\\
2.2e-10	0	\\
3.3e-10	0	\\
4.4e-10	0	\\
5.4e-10	0	\\
6.5e-10	0	\\
7.5e-10	0	\\
8.6e-10	0	\\
9.6e-10	0	\\
1.07e-09	0	\\
1.18e-09	0	\\
1.28e-09	0	\\
1.38e-09	0	\\
1.49e-09	0	\\
1.59e-09	0	\\
1.69e-09	0	\\
1.8e-09	0	\\
1.9e-09	0	\\
2.01e-09	0	\\
2.11e-09	0	\\
2.21e-09	0	\\
2.32e-09	0	\\
2.42e-09	0	\\
2.52e-09	0	\\
2.63e-09	0	\\
2.73e-09	0	\\
2.83e-09	0	\\
2.93e-09	0	\\
3.04e-09	0	\\
3.14e-09	0	\\
3.24e-09	0	\\
3.34e-09	0	\\
3.45e-09	0	\\
3.55e-09	0	\\
3.65e-09	0	\\
3.75e-09	0	\\
3.86e-09	0	\\
3.96e-09	0	\\
4.06e-09	0	\\
4.16e-09	0	\\
4.27e-09	0	\\
4.37e-09	0	\\
4.47e-09	0	\\
4.57e-09	0	\\
4.68e-09	0	\\
4.78e-09	0	\\
4.89e-09	0	\\
4.99e-09	0	\\
5e-09	0	\\
};
\addplot [color=red,solid,forget plot]
  table[row sep=crcr]{
0	0	\\
1.1e-10	0	\\
2.2e-10	0	\\
3.3e-10	0	\\
4.4e-10	0	\\
5.4e-10	0	\\
6.5e-10	0	\\
7.5e-10	0	\\
8.6e-10	0	\\
9.6e-10	0	\\
1.07e-09	0	\\
1.18e-09	0	\\
1.28e-09	0	\\
1.38e-09	0	\\
1.49e-09	0	\\
1.59e-09	0	\\
1.69e-09	0	\\
1.8e-09	0	\\
1.9e-09	0	\\
2.01e-09	0	\\
2.11e-09	0	\\
2.21e-09	0	\\
2.32e-09	0	\\
2.42e-09	0	\\
2.52e-09	0	\\
2.63e-09	0	\\
2.73e-09	0	\\
2.83e-09	0	\\
2.93e-09	0	\\
3.04e-09	0	\\
3.14e-09	0	\\
3.24e-09	0	\\
3.34e-09	0	\\
3.45e-09	0	\\
3.55e-09	0	\\
3.65e-09	0	\\
3.75e-09	0	\\
3.86e-09	0	\\
3.96e-09	0	\\
4.06e-09	0	\\
4.16e-09	0	\\
4.27e-09	0	\\
4.37e-09	0	\\
4.47e-09	0	\\
4.57e-09	0	\\
4.68e-09	0	\\
4.78e-09	0	\\
4.89e-09	0	\\
4.99e-09	0	\\
5e-09	0	\\
};
\addplot [color=mycolor1,solid,forget plot]
  table[row sep=crcr]{
0	0	\\
1.1e-10	0	\\
2.2e-10	0	\\
3.3e-10	0	\\
4.4e-10	0	\\
5.4e-10	0	\\
6.5e-10	0	\\
7.5e-10	0	\\
8.6e-10	0	\\
9.6e-10	0	\\
1.07e-09	0	\\
1.18e-09	0	\\
1.28e-09	0	\\
1.38e-09	0	\\
1.49e-09	0	\\
1.59e-09	0	\\
1.69e-09	0	\\
1.8e-09	0	\\
1.9e-09	0	\\
2.01e-09	0	\\
2.11e-09	0	\\
2.21e-09	0	\\
2.32e-09	0	\\
2.42e-09	0	\\
2.52e-09	0	\\
2.63e-09	0	\\
2.73e-09	0	\\
2.83e-09	0	\\
2.93e-09	0	\\
3.04e-09	0	\\
3.14e-09	0	\\
3.24e-09	0	\\
3.34e-09	0	\\
3.45e-09	0	\\
3.55e-09	0	\\
3.65e-09	0	\\
3.75e-09	0	\\
3.86e-09	0	\\
3.96e-09	0	\\
4.06e-09	0	\\
4.16e-09	0	\\
4.27e-09	0	\\
4.37e-09	0	\\
4.47e-09	0	\\
4.57e-09	0	\\
4.68e-09	0	\\
4.78e-09	0	\\
4.89e-09	0	\\
4.99e-09	0	\\
5e-09	0	\\
};
\addplot [color=mycolor2,solid,forget plot]
  table[row sep=crcr]{
0	0	\\
1.1e-10	0	\\
2.2e-10	0	\\
3.3e-10	0	\\
4.4e-10	0	\\
5.4e-10	0	\\
6.5e-10	0	\\
7.5e-10	0	\\
8.6e-10	0	\\
9.6e-10	0	\\
1.07e-09	0	\\
1.18e-09	0	\\
1.28e-09	0	\\
1.38e-09	0	\\
1.49e-09	0	\\
1.59e-09	0	\\
1.69e-09	0	\\
1.8e-09	0	\\
1.9e-09	0	\\
2.01e-09	0	\\
2.11e-09	0	\\
2.21e-09	0	\\
2.32e-09	0	\\
2.42e-09	0	\\
2.52e-09	0	\\
2.63e-09	0	\\
2.73e-09	0	\\
2.83e-09	0	\\
2.93e-09	0	\\
3.04e-09	0	\\
3.14e-09	0	\\
3.24e-09	0	\\
3.34e-09	0	\\
3.45e-09	0	\\
3.55e-09	0	\\
3.65e-09	0	\\
3.75e-09	0	\\
3.86e-09	0	\\
3.96e-09	0	\\
4.06e-09	0	\\
4.16e-09	0	\\
4.27e-09	0	\\
4.37e-09	0	\\
4.47e-09	0	\\
4.57e-09	0	\\
4.68e-09	0	\\
4.78e-09	0	\\
4.89e-09	0	\\
4.99e-09	0	\\
5e-09	0	\\
};
\addplot [color=mycolor3,solid,forget plot]
  table[row sep=crcr]{
0	0	\\
1.1e-10	0	\\
2.2e-10	0	\\
3.3e-10	0	\\
4.4e-10	0	\\
5.4e-10	0	\\
6.5e-10	0	\\
7.5e-10	0	\\
8.6e-10	0	\\
9.6e-10	0	\\
1.07e-09	0	\\
1.18e-09	0	\\
1.28e-09	0	\\
1.38e-09	0	\\
1.49e-09	0	\\
1.59e-09	0	\\
1.69e-09	0	\\
1.8e-09	0	\\
1.9e-09	0	\\
2.01e-09	0	\\
2.11e-09	0	\\
2.21e-09	0	\\
2.32e-09	0	\\
2.42e-09	0	\\
2.52e-09	0	\\
2.63e-09	0	\\
2.73e-09	0	\\
2.83e-09	0	\\
2.93e-09	0	\\
3.04e-09	0	\\
3.14e-09	0	\\
3.24e-09	0	\\
3.34e-09	0	\\
3.45e-09	0	\\
3.55e-09	0	\\
3.65e-09	0	\\
3.75e-09	0	\\
3.86e-09	0	\\
3.96e-09	0	\\
4.06e-09	0	\\
4.16e-09	0	\\
4.27e-09	0	\\
4.37e-09	0	\\
4.47e-09	0	\\
4.57e-09	0	\\
4.68e-09	0	\\
4.78e-09	0	\\
4.89e-09	0	\\
4.99e-09	0	\\
5e-09	0	\\
};
\addplot [color=darkgray,solid,forget plot]
  table[row sep=crcr]{
0	0	\\
1.1e-10	0	\\
2.2e-10	0	\\
3.3e-10	0	\\
4.4e-10	0	\\
5.4e-10	0	\\
6.5e-10	0	\\
7.5e-10	0	\\
8.6e-10	0	\\
9.6e-10	0	\\
1.07e-09	0	\\
1.18e-09	0	\\
1.28e-09	0	\\
1.38e-09	0	\\
1.49e-09	0	\\
1.59e-09	0	\\
1.69e-09	0	\\
1.8e-09	0	\\
1.9e-09	0	\\
2.01e-09	0	\\
2.11e-09	0	\\
2.21e-09	0	\\
2.32e-09	0	\\
2.42e-09	0	\\
2.52e-09	0	\\
2.63e-09	0	\\
2.73e-09	0	\\
2.83e-09	0	\\
2.93e-09	0	\\
3.04e-09	0	\\
3.14e-09	0	\\
3.24e-09	0	\\
3.34e-09	0	\\
3.45e-09	0	\\
3.55e-09	0	\\
3.65e-09	0	\\
3.75e-09	0	\\
3.86e-09	0	\\
3.96e-09	0	\\
4.06e-09	0	\\
4.16e-09	0	\\
4.27e-09	0	\\
4.37e-09	0	\\
4.47e-09	0	\\
4.57e-09	0	\\
4.68e-09	0	\\
4.78e-09	0	\\
4.89e-09	0	\\
4.99e-09	0	\\
5e-09	0	\\
};
\addplot [color=blue,solid,forget plot]
  table[row sep=crcr]{
0	0	\\
1.1e-10	0	\\
2.2e-10	0	\\
3.3e-10	0	\\
4.4e-10	0	\\
5.4e-10	0	\\
6.5e-10	0	\\
7.5e-10	0	\\
8.6e-10	0	\\
9.6e-10	0	\\
1.07e-09	0	\\
1.18e-09	0	\\
1.28e-09	0	\\
1.38e-09	0	\\
1.49e-09	0	\\
1.59e-09	0	\\
1.69e-09	0	\\
1.8e-09	0	\\
1.9e-09	0	\\
2.01e-09	0	\\
2.11e-09	0	\\
2.21e-09	0	\\
2.32e-09	0	\\
2.42e-09	0	\\
2.52e-09	0	\\
2.63e-09	0	\\
2.73e-09	0	\\
2.83e-09	0	\\
2.93e-09	0	\\
3.04e-09	0	\\
3.14e-09	0	\\
3.24e-09	0	\\
3.34e-09	0	\\
3.45e-09	0	\\
3.55e-09	0	\\
3.65e-09	0	\\
3.75e-09	0	\\
3.86e-09	0	\\
3.96e-09	0	\\
4.06e-09	0	\\
4.16e-09	0	\\
4.27e-09	0	\\
4.37e-09	0	\\
4.47e-09	0	\\
4.57e-09	0	\\
4.68e-09	0	\\
4.78e-09	0	\\
4.89e-09	0	\\
4.99e-09	0	\\
5e-09	0	\\
};
\addplot [color=black!50!green,solid,forget plot]
  table[row sep=crcr]{
0	0	\\
1.1e-10	0	\\
2.2e-10	0	\\
3.3e-10	0	\\
4.4e-10	0	\\
5.4e-10	0	\\
6.5e-10	0	\\
7.5e-10	0	\\
8.6e-10	0	\\
9.6e-10	0	\\
1.07e-09	0	\\
1.18e-09	0	\\
1.28e-09	0	\\
1.38e-09	0	\\
1.49e-09	0	\\
1.59e-09	0	\\
1.69e-09	0	\\
1.8e-09	0	\\
1.9e-09	0	\\
2.01e-09	0	\\
2.11e-09	0	\\
2.21e-09	0	\\
2.32e-09	0	\\
2.42e-09	0	\\
2.52e-09	0	\\
2.63e-09	0	\\
2.73e-09	0	\\
2.83e-09	0	\\
2.93e-09	0	\\
3.04e-09	0	\\
3.14e-09	0	\\
3.24e-09	0	\\
3.34e-09	0	\\
3.45e-09	0	\\
3.55e-09	0	\\
3.65e-09	0	\\
3.75e-09	0	\\
3.86e-09	0	\\
3.96e-09	0	\\
4.06e-09	0	\\
4.16e-09	0	\\
4.27e-09	0	\\
4.37e-09	0	\\
4.47e-09	0	\\
4.57e-09	0	\\
4.68e-09	0	\\
4.78e-09	0	\\
4.89e-09	0	\\
4.99e-09	0	\\
5e-09	0	\\
};
\addplot [color=red,solid,forget plot]
  table[row sep=crcr]{
0	0	\\
1.1e-10	0	\\
2.2e-10	0	\\
3.3e-10	0	\\
4.4e-10	0	\\
5.4e-10	0	\\
6.5e-10	0	\\
7.5e-10	0	\\
8.6e-10	0	\\
9.6e-10	0	\\
1.07e-09	0	\\
1.18e-09	0	\\
1.28e-09	0	\\
1.38e-09	0	\\
1.49e-09	0	\\
1.59e-09	0	\\
1.69e-09	0	\\
1.8e-09	0	\\
1.9e-09	0	\\
2.01e-09	0	\\
2.11e-09	0	\\
2.21e-09	0	\\
2.32e-09	0	\\
2.42e-09	0	\\
2.52e-09	0	\\
2.63e-09	0	\\
2.73e-09	0	\\
2.83e-09	0	\\
2.93e-09	0	\\
3.04e-09	0	\\
3.14e-09	0	\\
3.24e-09	0	\\
3.34e-09	0	\\
3.45e-09	0	\\
3.55e-09	0	\\
3.65e-09	0	\\
3.75e-09	0	\\
3.86e-09	0	\\
3.96e-09	0	\\
4.06e-09	0	\\
4.16e-09	0	\\
4.27e-09	0	\\
4.37e-09	0	\\
4.47e-09	0	\\
4.57e-09	0	\\
4.68e-09	0	\\
4.78e-09	0	\\
4.89e-09	0	\\
4.99e-09	0	\\
5e-09	0	\\
};
\addplot [color=mycolor1,solid,forget plot]
  table[row sep=crcr]{
0	0	\\
1.1e-10	0	\\
2.2e-10	0	\\
3.3e-10	0	\\
4.4e-10	0	\\
5.4e-10	0	\\
6.5e-10	0	\\
7.5e-10	0	\\
8.6e-10	0	\\
9.6e-10	0	\\
1.07e-09	0	\\
1.18e-09	0	\\
1.28e-09	0	\\
1.38e-09	0	\\
1.49e-09	0	\\
1.59e-09	0	\\
1.69e-09	0	\\
1.8e-09	0	\\
1.9e-09	0	\\
2.01e-09	0	\\
2.11e-09	0	\\
2.21e-09	0	\\
2.32e-09	0	\\
2.42e-09	0	\\
2.52e-09	0	\\
2.63e-09	0	\\
2.73e-09	0	\\
2.83e-09	0	\\
2.93e-09	0	\\
3.04e-09	0	\\
3.14e-09	0	\\
3.24e-09	0	\\
3.34e-09	0	\\
3.45e-09	0	\\
3.55e-09	0	\\
3.65e-09	0	\\
3.75e-09	0	\\
3.86e-09	0	\\
3.96e-09	0	\\
4.06e-09	0	\\
4.16e-09	0	\\
4.27e-09	0	\\
4.37e-09	0	\\
4.47e-09	0	\\
4.57e-09	0	\\
4.68e-09	0	\\
4.78e-09	0	\\
4.89e-09	0	\\
4.99e-09	0	\\
5e-09	0	\\
};
\addplot [color=mycolor2,solid,forget plot]
  table[row sep=crcr]{
0	0	\\
1.1e-10	0	\\
2.2e-10	0	\\
3.3e-10	0	\\
4.4e-10	0	\\
5.4e-10	0	\\
6.5e-10	0	\\
7.5e-10	0	\\
8.6e-10	0	\\
9.6e-10	0	\\
1.07e-09	0	\\
1.18e-09	0	\\
1.28e-09	0	\\
1.38e-09	0	\\
1.49e-09	0	\\
1.59e-09	0	\\
1.69e-09	0	\\
1.8e-09	0	\\
1.9e-09	0	\\
2.01e-09	0	\\
2.11e-09	0	\\
2.21e-09	0	\\
2.32e-09	0	\\
2.42e-09	0	\\
2.52e-09	0	\\
2.63e-09	0	\\
2.73e-09	0	\\
2.83e-09	0	\\
2.93e-09	0	\\
3.04e-09	0	\\
3.14e-09	0	\\
3.24e-09	0	\\
3.34e-09	0	\\
3.45e-09	0	\\
3.55e-09	0	\\
3.65e-09	0	\\
3.75e-09	0	\\
3.86e-09	0	\\
3.96e-09	0	\\
4.06e-09	0	\\
4.16e-09	0	\\
4.27e-09	0	\\
4.37e-09	0	\\
4.47e-09	0	\\
4.57e-09	0	\\
4.68e-09	0	\\
4.78e-09	0	\\
4.89e-09	0	\\
4.99e-09	0	\\
5e-09	0	\\
};
\addplot [color=mycolor3,solid,forget plot]
  table[row sep=crcr]{
0	0	\\
1.1e-10	0	\\
2.2e-10	0	\\
3.3e-10	0	\\
4.4e-10	0	\\
5.4e-10	0	\\
6.5e-10	0	\\
7.5e-10	0	\\
8.6e-10	0	\\
9.6e-10	0	\\
1.07e-09	0	\\
1.18e-09	0	\\
1.28e-09	0	\\
1.38e-09	0	\\
1.49e-09	0	\\
1.59e-09	0	\\
1.69e-09	0	\\
1.8e-09	0	\\
1.9e-09	0	\\
2.01e-09	0	\\
2.11e-09	0	\\
2.21e-09	0	\\
2.32e-09	0	\\
2.42e-09	0	\\
2.52e-09	0	\\
2.63e-09	0	\\
2.73e-09	0	\\
2.83e-09	0	\\
2.93e-09	0	\\
3.04e-09	0	\\
3.14e-09	0	\\
3.24e-09	0	\\
3.34e-09	0	\\
3.45e-09	0	\\
3.55e-09	0	\\
3.65e-09	0	\\
3.75e-09	0	\\
3.86e-09	0	\\
3.96e-09	0	\\
4.06e-09	0	\\
4.16e-09	0	\\
4.27e-09	0	\\
4.37e-09	0	\\
4.47e-09	0	\\
4.57e-09	0	\\
4.68e-09	0	\\
4.78e-09	0	\\
4.89e-09	0	\\
4.99e-09	0	\\
5e-09	0	\\
};
\addplot [color=darkgray,solid,forget plot]
  table[row sep=crcr]{
0	0	\\
1.1e-10	0	\\
2.2e-10	0	\\
3.3e-10	0	\\
4.4e-10	0	\\
5.4e-10	0	\\
6.5e-10	0	\\
7.5e-10	0	\\
8.6e-10	0	\\
9.6e-10	0	\\
1.07e-09	0	\\
1.18e-09	0	\\
1.28e-09	0	\\
1.38e-09	0	\\
1.49e-09	0	\\
1.59e-09	0	\\
1.69e-09	0	\\
1.8e-09	0	\\
1.9e-09	0	\\
2.01e-09	0	\\
2.11e-09	0	\\
2.21e-09	0	\\
2.32e-09	0	\\
2.42e-09	0	\\
2.52e-09	0	\\
2.63e-09	0	\\
2.73e-09	0	\\
2.83e-09	0	\\
2.93e-09	0	\\
3.04e-09	0	\\
3.14e-09	0	\\
3.24e-09	0	\\
3.34e-09	0	\\
3.45e-09	0	\\
3.55e-09	0	\\
3.65e-09	0	\\
3.75e-09	0	\\
3.86e-09	0	\\
3.96e-09	0	\\
4.06e-09	0	\\
4.16e-09	0	\\
4.27e-09	0	\\
4.37e-09	0	\\
4.47e-09	0	\\
4.57e-09	0	\\
4.68e-09	0	\\
4.78e-09	0	\\
4.89e-09	0	\\
4.99e-09	0	\\
5e-09	0	\\
};
\addplot [color=blue,solid,forget plot]
  table[row sep=crcr]{
0	0	\\
1.1e-10	0	\\
2.2e-10	0	\\
3.3e-10	0	\\
4.4e-10	0	\\
5.4e-10	0	\\
6.5e-10	0	\\
7.5e-10	0	\\
8.6e-10	0	\\
9.6e-10	0	\\
1.07e-09	0	\\
1.18e-09	0	\\
1.28e-09	0	\\
1.38e-09	0	\\
1.49e-09	0	\\
1.59e-09	0	\\
1.69e-09	0	\\
1.8e-09	0	\\
1.9e-09	0	\\
2.01e-09	0	\\
2.11e-09	0	\\
2.21e-09	0	\\
2.32e-09	0	\\
2.42e-09	0	\\
2.52e-09	0	\\
2.63e-09	0	\\
2.73e-09	0	\\
2.83e-09	0	\\
2.93e-09	0	\\
3.04e-09	0	\\
3.14e-09	0	\\
3.24e-09	0	\\
3.34e-09	0	\\
3.45e-09	0	\\
3.55e-09	0	\\
3.65e-09	0	\\
3.75e-09	0	\\
3.86e-09	0	\\
3.96e-09	0	\\
4.06e-09	0	\\
4.16e-09	0	\\
4.27e-09	0	\\
4.37e-09	0	\\
4.47e-09	0	\\
4.57e-09	0	\\
4.68e-09	0	\\
4.78e-09	0	\\
4.89e-09	0	\\
4.99e-09	0	\\
5e-09	0	\\
};
\addplot [color=black!50!green,solid,forget plot]
  table[row sep=crcr]{
0	0	\\
1.1e-10	0	\\
2.2e-10	0	\\
3.3e-10	0	\\
4.4e-10	0	\\
5.4e-10	0	\\
6.5e-10	0	\\
7.5e-10	0	\\
8.6e-10	0	\\
9.6e-10	0	\\
1.07e-09	0	\\
1.18e-09	0	\\
1.28e-09	0	\\
1.38e-09	0	\\
1.49e-09	0	\\
1.59e-09	0	\\
1.69e-09	0	\\
1.8e-09	0	\\
1.9e-09	0	\\
2.01e-09	0	\\
2.11e-09	0	\\
2.21e-09	0	\\
2.32e-09	0	\\
2.42e-09	0	\\
2.52e-09	0	\\
2.63e-09	0	\\
2.73e-09	0	\\
2.83e-09	0	\\
2.93e-09	0	\\
3.04e-09	0	\\
3.14e-09	0	\\
3.24e-09	0	\\
3.34e-09	0	\\
3.45e-09	0	\\
3.55e-09	0	\\
3.65e-09	0	\\
3.75e-09	0	\\
3.86e-09	0	\\
3.96e-09	0	\\
4.06e-09	0	\\
4.16e-09	0	\\
4.27e-09	0	\\
4.37e-09	0	\\
4.47e-09	0	\\
4.57e-09	0	\\
4.68e-09	0	\\
4.78e-09	0	\\
4.89e-09	0	\\
4.99e-09	0	\\
5e-09	0	\\
};
\addplot [color=red,solid,forget plot]
  table[row sep=crcr]{
0	0	\\
1.1e-10	0	\\
2.2e-10	0	\\
3.3e-10	0	\\
4.4e-10	0	\\
5.4e-10	0	\\
6.5e-10	0	\\
7.5e-10	0	\\
8.6e-10	0	\\
9.6e-10	0	\\
1.07e-09	0	\\
1.18e-09	0	\\
1.28e-09	0	\\
1.38e-09	0	\\
1.49e-09	0	\\
1.59e-09	0	\\
1.69e-09	0	\\
1.8e-09	0	\\
1.9e-09	0	\\
2.01e-09	0	\\
2.11e-09	0	\\
2.21e-09	0	\\
2.32e-09	0	\\
2.42e-09	0	\\
2.52e-09	0	\\
2.63e-09	0	\\
2.73e-09	0	\\
2.83e-09	0	\\
2.93e-09	0	\\
3.04e-09	0	\\
3.14e-09	0	\\
3.24e-09	0	\\
3.34e-09	0	\\
3.45e-09	0	\\
3.55e-09	0	\\
3.65e-09	0	\\
3.75e-09	0	\\
3.86e-09	0	\\
3.96e-09	0	\\
4.06e-09	0	\\
4.16e-09	0	\\
4.27e-09	0	\\
4.37e-09	0	\\
4.47e-09	0	\\
4.57e-09	0	\\
4.68e-09	0	\\
4.78e-09	0	\\
4.89e-09	0	\\
4.99e-09	0	\\
5e-09	0	\\
};
\addplot [color=mycolor1,solid,forget plot]
  table[row sep=crcr]{
0	0	\\
1.1e-10	0	\\
2.2e-10	0	\\
3.3e-10	0	\\
4.4e-10	0	\\
5.4e-10	0	\\
6.5e-10	0	\\
7.5e-10	0	\\
8.6e-10	0	\\
9.6e-10	0	\\
1.07e-09	0	\\
1.18e-09	0	\\
1.28e-09	0	\\
1.38e-09	0	\\
1.49e-09	0	\\
1.59e-09	0	\\
1.69e-09	0	\\
1.8e-09	0	\\
1.9e-09	0	\\
2.01e-09	0	\\
2.11e-09	0	\\
2.21e-09	0	\\
2.32e-09	0	\\
2.42e-09	0	\\
2.52e-09	0	\\
2.63e-09	0	\\
2.73e-09	0	\\
2.83e-09	0	\\
2.93e-09	0	\\
3.04e-09	0	\\
3.14e-09	0	\\
3.24e-09	0	\\
3.34e-09	0	\\
3.45e-09	0	\\
3.55e-09	0	\\
3.65e-09	0	\\
3.75e-09	0	\\
3.86e-09	0	\\
3.96e-09	0	\\
4.06e-09	0	\\
4.16e-09	0	\\
4.27e-09	0	\\
4.37e-09	0	\\
4.47e-09	0	\\
4.57e-09	0	\\
4.68e-09	0	\\
4.78e-09	0	\\
4.89e-09	0	\\
4.99e-09	0	\\
5e-09	0	\\
};
\addplot [color=mycolor2,solid,forget plot]
  table[row sep=crcr]{
0	0	\\
1.1e-10	0	\\
2.2e-10	0	\\
3.3e-10	0	\\
4.4e-10	0	\\
5.4e-10	0	\\
6.5e-10	0	\\
7.5e-10	0	\\
8.6e-10	0	\\
9.6e-10	0	\\
1.07e-09	0	\\
1.18e-09	0	\\
1.28e-09	0	\\
1.38e-09	0	\\
1.49e-09	0	\\
1.59e-09	0	\\
1.69e-09	0	\\
1.8e-09	0	\\
1.9e-09	0	\\
2.01e-09	0	\\
2.11e-09	0	\\
2.21e-09	0	\\
2.32e-09	0	\\
2.42e-09	0	\\
2.52e-09	0	\\
2.63e-09	0	\\
2.73e-09	0	\\
2.83e-09	0	\\
2.93e-09	0	\\
3.04e-09	0	\\
3.14e-09	0	\\
3.24e-09	0	\\
3.34e-09	0	\\
3.45e-09	0	\\
3.55e-09	0	\\
3.65e-09	0	\\
3.75e-09	0	\\
3.86e-09	0	\\
3.96e-09	0	\\
4.06e-09	0	\\
4.16e-09	0	\\
4.27e-09	0	\\
4.37e-09	0	\\
4.47e-09	0	\\
4.57e-09	0	\\
4.68e-09	0	\\
4.78e-09	0	\\
4.89e-09	0	\\
4.99e-09	0	\\
5e-09	0	\\
};
\addplot [color=mycolor3,solid,forget plot]
  table[row sep=crcr]{
0	0	\\
1.1e-10	0	\\
2.2e-10	0	\\
3.3e-10	0	\\
4.4e-10	0	\\
5.4e-10	0	\\
6.5e-10	0	\\
7.5e-10	0	\\
8.6e-10	0	\\
9.6e-10	0	\\
1.07e-09	0	\\
1.18e-09	0	\\
1.28e-09	0	\\
1.38e-09	0	\\
1.49e-09	0	\\
1.59e-09	0	\\
1.69e-09	0	\\
1.8e-09	0	\\
1.9e-09	0	\\
2.01e-09	0	\\
2.11e-09	0	\\
2.21e-09	0	\\
2.32e-09	0	\\
2.42e-09	0	\\
2.52e-09	0	\\
2.63e-09	0	\\
2.73e-09	0	\\
2.83e-09	0	\\
2.93e-09	0	\\
3.04e-09	0	\\
3.14e-09	0	\\
3.24e-09	0	\\
3.34e-09	0	\\
3.45e-09	0	\\
3.55e-09	0	\\
3.65e-09	0	\\
3.75e-09	0	\\
3.86e-09	0	\\
3.96e-09	0	\\
4.06e-09	0	\\
4.16e-09	0	\\
4.27e-09	0	\\
4.37e-09	0	\\
4.47e-09	0	\\
4.57e-09	0	\\
4.68e-09	0	\\
4.78e-09	0	\\
4.89e-09	0	\\
4.99e-09	0	\\
5e-09	0	\\
};
\addplot [color=darkgray,solid,forget plot]
  table[row sep=crcr]{
0	0	\\
1.1e-10	0	\\
2.2e-10	0	\\
3.3e-10	0	\\
4.4e-10	0	\\
5.4e-10	0	\\
6.5e-10	0	\\
7.5e-10	0	\\
8.6e-10	0	\\
9.6e-10	0	\\
1.07e-09	0	\\
1.18e-09	0	\\
1.28e-09	0	\\
1.38e-09	0	\\
1.49e-09	0	\\
1.59e-09	0	\\
1.69e-09	0	\\
1.8e-09	0	\\
1.9e-09	0	\\
2.01e-09	0	\\
2.11e-09	0	\\
2.21e-09	0	\\
2.32e-09	0	\\
2.42e-09	0	\\
2.52e-09	0	\\
2.63e-09	0	\\
2.73e-09	0	\\
2.83e-09	0	\\
2.93e-09	0	\\
3.04e-09	0	\\
3.14e-09	0	\\
3.24e-09	0	\\
3.34e-09	0	\\
3.45e-09	0	\\
3.55e-09	0	\\
3.65e-09	0	\\
3.75e-09	0	\\
3.86e-09	0	\\
3.96e-09	0	\\
4.06e-09	0	\\
4.16e-09	0	\\
4.27e-09	0	\\
4.37e-09	0	\\
4.47e-09	0	\\
4.57e-09	0	\\
4.68e-09	0	\\
4.78e-09	0	\\
4.89e-09	0	\\
4.99e-09	0	\\
5e-09	0	\\
};
\addplot [color=blue,solid,forget plot]
  table[row sep=crcr]{
0	0	\\
1.1e-10	0	\\
2.2e-10	0	\\
3.3e-10	0	\\
4.4e-10	0	\\
5.4e-10	0	\\
6.5e-10	0	\\
7.5e-10	0	\\
8.6e-10	0	\\
9.6e-10	0	\\
1.07e-09	0	\\
1.18e-09	0	\\
1.28e-09	0	\\
1.38e-09	0	\\
1.49e-09	0	\\
1.59e-09	0	\\
1.69e-09	0	\\
1.8e-09	0	\\
1.9e-09	0	\\
2.01e-09	0	\\
2.11e-09	0	\\
2.21e-09	0	\\
2.32e-09	0	\\
2.42e-09	0	\\
2.52e-09	0	\\
2.63e-09	0	\\
2.73e-09	0	\\
2.83e-09	0	\\
2.93e-09	0	\\
3.04e-09	0	\\
3.14e-09	0	\\
3.24e-09	0	\\
3.34e-09	0	\\
3.45e-09	0	\\
3.55e-09	0	\\
3.65e-09	0	\\
3.75e-09	0	\\
3.86e-09	0	\\
3.96e-09	0	\\
4.06e-09	0	\\
4.16e-09	0	\\
4.27e-09	0	\\
4.37e-09	0	\\
4.47e-09	0	\\
4.57e-09	0	\\
4.68e-09	0	\\
4.78e-09	0	\\
4.89e-09	0	\\
4.99e-09	0	\\
5e-09	0	\\
};
\addplot [color=black!50!green,solid,forget plot]
  table[row sep=crcr]{
0	0	\\
1.1e-10	0	\\
2.2e-10	0	\\
3.3e-10	0	\\
4.4e-10	0	\\
5.4e-10	0	\\
6.5e-10	0	\\
7.5e-10	0	\\
8.6e-10	0	\\
9.6e-10	0	\\
1.07e-09	0	\\
1.18e-09	0	\\
1.28e-09	0	\\
1.38e-09	0	\\
1.49e-09	0	\\
1.59e-09	0	\\
1.69e-09	0	\\
1.8e-09	0	\\
1.9e-09	0	\\
2.01e-09	0	\\
2.11e-09	0	\\
2.21e-09	0	\\
2.32e-09	0	\\
2.42e-09	0	\\
2.52e-09	0	\\
2.63e-09	0	\\
2.73e-09	0	\\
2.83e-09	0	\\
2.93e-09	0	\\
3.04e-09	0	\\
3.14e-09	0	\\
3.24e-09	0	\\
3.34e-09	0	\\
3.45e-09	0	\\
3.55e-09	0	\\
3.65e-09	0	\\
3.75e-09	0	\\
3.86e-09	0	\\
3.96e-09	0	\\
4.06e-09	0	\\
4.16e-09	0	\\
4.27e-09	0	\\
4.37e-09	0	\\
4.47e-09	0	\\
4.57e-09	0	\\
4.68e-09	0	\\
4.78e-09	0	\\
4.89e-09	0	\\
4.99e-09	0	\\
5e-09	0	\\
};
\addplot [color=red,solid,forget plot]
  table[row sep=crcr]{
0	0	\\
1.1e-10	0	\\
2.2e-10	0	\\
3.3e-10	0	\\
4.4e-10	0	\\
5.4e-10	0	\\
6.5e-10	0	\\
7.5e-10	0	\\
8.6e-10	0	\\
9.6e-10	0	\\
1.07e-09	0	\\
1.18e-09	0	\\
1.28e-09	0	\\
1.38e-09	0	\\
1.49e-09	0	\\
1.59e-09	0	\\
1.69e-09	0	\\
1.8e-09	0	\\
1.9e-09	0	\\
2.01e-09	0	\\
2.11e-09	0	\\
2.21e-09	0	\\
2.32e-09	0	\\
2.42e-09	0	\\
2.52e-09	0	\\
2.63e-09	0	\\
2.73e-09	0	\\
2.83e-09	0	\\
2.93e-09	0	\\
3.04e-09	0	\\
3.14e-09	0	\\
3.24e-09	0	\\
3.34e-09	0	\\
3.45e-09	0	\\
3.55e-09	0	\\
3.65e-09	0	\\
3.75e-09	0	\\
3.86e-09	0	\\
3.96e-09	0	\\
4.06e-09	0	\\
4.16e-09	0	\\
4.27e-09	0	\\
4.37e-09	0	\\
4.47e-09	0	\\
4.57e-09	0	\\
4.68e-09	0	\\
4.78e-09	0	\\
4.89e-09	0	\\
4.99e-09	0	\\
5e-09	0	\\
};
\addplot [color=mycolor1,solid,forget plot]
  table[row sep=crcr]{
0	0	\\
1.1e-10	0	\\
2.2e-10	0	\\
3.3e-10	0	\\
4.4e-10	0	\\
5.4e-10	0	\\
6.5e-10	0	\\
7.5e-10	0	\\
8.6e-10	0	\\
9.6e-10	0	\\
1.07e-09	0	\\
1.18e-09	0	\\
1.28e-09	0	\\
1.38e-09	0	\\
1.49e-09	0	\\
1.59e-09	0	\\
1.69e-09	0	\\
1.8e-09	0	\\
1.9e-09	0	\\
2.01e-09	0	\\
2.11e-09	0	\\
2.21e-09	0	\\
2.32e-09	0	\\
2.42e-09	0	\\
2.52e-09	0	\\
2.63e-09	0	\\
2.73e-09	0	\\
2.83e-09	0	\\
2.93e-09	0	\\
3.04e-09	0	\\
3.14e-09	0	\\
3.24e-09	0	\\
3.34e-09	0	\\
3.45e-09	0	\\
3.55e-09	0	\\
3.65e-09	0	\\
3.75e-09	0	\\
3.86e-09	0	\\
3.96e-09	0	\\
4.06e-09	0	\\
4.16e-09	0	\\
4.27e-09	0	\\
4.37e-09	0	\\
4.47e-09	0	\\
4.57e-09	0	\\
4.68e-09	0	\\
4.78e-09	0	\\
4.89e-09	0	\\
4.99e-09	0	\\
5e-09	0	\\
};
\addplot [color=mycolor2,solid,forget plot]
  table[row sep=crcr]{
0	0	\\
1.1e-10	0	\\
2.2e-10	0	\\
3.3e-10	0	\\
4.4e-10	0	\\
5.4e-10	0	\\
6.5e-10	0	\\
7.5e-10	0	\\
8.6e-10	0	\\
9.6e-10	0	\\
1.07e-09	0	\\
1.18e-09	0	\\
1.28e-09	0	\\
1.38e-09	0	\\
1.49e-09	0	\\
1.59e-09	0	\\
1.69e-09	0	\\
1.8e-09	0	\\
1.9e-09	0	\\
2.01e-09	0	\\
2.11e-09	0	\\
2.21e-09	0	\\
2.32e-09	0	\\
2.42e-09	0	\\
2.52e-09	0	\\
2.63e-09	0	\\
2.73e-09	0	\\
2.83e-09	0	\\
2.93e-09	0	\\
3.04e-09	0	\\
3.14e-09	0	\\
3.24e-09	0	\\
3.34e-09	0	\\
3.45e-09	0	\\
3.55e-09	0	\\
3.65e-09	0	\\
3.75e-09	0	\\
3.86e-09	0	\\
3.96e-09	0	\\
4.06e-09	0	\\
4.16e-09	0	\\
4.27e-09	0	\\
4.37e-09	0	\\
4.47e-09	0	\\
4.57e-09	0	\\
4.68e-09	0	\\
4.78e-09	0	\\
4.89e-09	0	\\
4.99e-09	0	\\
5e-09	0	\\
};
\addplot [color=mycolor3,solid,forget plot]
  table[row sep=crcr]{
0	0	\\
1.1e-10	0	\\
2.2e-10	0	\\
3.3e-10	0	\\
4.4e-10	0	\\
5.4e-10	0	\\
6.5e-10	0	\\
7.5e-10	0	\\
8.6e-10	0	\\
9.6e-10	0	\\
1.07e-09	0	\\
1.18e-09	0	\\
1.28e-09	0	\\
1.38e-09	0	\\
1.49e-09	0	\\
1.59e-09	0	\\
1.69e-09	0	\\
1.8e-09	0	\\
1.9e-09	0	\\
2.01e-09	0	\\
2.11e-09	0	\\
2.21e-09	0	\\
2.32e-09	0	\\
2.42e-09	0	\\
2.52e-09	0	\\
2.63e-09	0	\\
2.73e-09	0	\\
2.83e-09	0	\\
2.93e-09	0	\\
3.04e-09	0	\\
3.14e-09	0	\\
3.24e-09	0	\\
3.34e-09	0	\\
3.45e-09	0	\\
3.55e-09	0	\\
3.65e-09	0	\\
3.75e-09	0	\\
3.86e-09	0	\\
3.96e-09	0	\\
4.06e-09	0	\\
4.16e-09	0	\\
4.27e-09	0	\\
4.37e-09	0	\\
4.47e-09	0	\\
4.57e-09	0	\\
4.68e-09	0	\\
4.78e-09	0	\\
4.89e-09	0	\\
4.99e-09	0	\\
5e-09	0	\\
};
\addplot [color=darkgray,solid,forget plot]
  table[row sep=crcr]{
0	0	\\
1.1e-10	0	\\
2.2e-10	0	\\
3.3e-10	0	\\
4.4e-10	0	\\
5.4e-10	0	\\
6.5e-10	0	\\
7.5e-10	0	\\
8.6e-10	0	\\
9.6e-10	0	\\
1.07e-09	0	\\
1.18e-09	0	\\
1.28e-09	0	\\
1.38e-09	0	\\
1.49e-09	0	\\
1.59e-09	0	\\
1.69e-09	0	\\
1.8e-09	0	\\
1.9e-09	0	\\
2.01e-09	0	\\
2.11e-09	0	\\
2.21e-09	0	\\
2.32e-09	0	\\
2.42e-09	0	\\
2.52e-09	0	\\
2.63e-09	0	\\
2.73e-09	0	\\
2.83e-09	0	\\
2.93e-09	0	\\
3.04e-09	0	\\
3.14e-09	0	\\
3.24e-09	0	\\
3.34e-09	0	\\
3.45e-09	0	\\
3.55e-09	0	\\
3.65e-09	0	\\
3.75e-09	0	\\
3.86e-09	0	\\
3.96e-09	0	\\
4.06e-09	0	\\
4.16e-09	0	\\
4.27e-09	0	\\
4.37e-09	0	\\
4.47e-09	0	\\
4.57e-09	0	\\
4.68e-09	0	\\
4.78e-09	0	\\
4.89e-09	0	\\
4.99e-09	0	\\
5e-09	0	\\
};
\addplot [color=blue,solid,forget plot]
  table[row sep=crcr]{
0	0	\\
1.1e-10	0	\\
2.2e-10	0	\\
3.3e-10	0	\\
4.4e-10	0	\\
5.4e-10	0	\\
6.5e-10	0	\\
7.5e-10	0	\\
8.6e-10	0	\\
9.6e-10	0	\\
1.07e-09	0	\\
1.18e-09	0	\\
1.28e-09	0	\\
1.38e-09	0	\\
1.49e-09	0	\\
1.59e-09	0	\\
1.69e-09	0	\\
1.8e-09	0	\\
1.9e-09	0	\\
2.01e-09	0	\\
2.11e-09	0	\\
2.21e-09	0	\\
2.32e-09	0	\\
2.42e-09	0	\\
2.52e-09	0	\\
2.63e-09	0	\\
2.73e-09	0	\\
2.83e-09	0	\\
2.93e-09	0	\\
3.04e-09	0	\\
3.14e-09	0	\\
3.24e-09	0	\\
3.34e-09	0	\\
3.45e-09	0	\\
3.55e-09	0	\\
3.65e-09	0	\\
3.75e-09	0	\\
3.86e-09	0	\\
3.96e-09	0	\\
4.06e-09	0	\\
4.16e-09	0	\\
4.27e-09	0	\\
4.37e-09	0	\\
4.47e-09	0	\\
4.57e-09	0	\\
4.68e-09	0	\\
4.78e-09	0	\\
4.89e-09	0	\\
4.99e-09	0	\\
5e-09	0	\\
};
\addplot [color=black!50!green,solid,forget plot]
  table[row sep=crcr]{
0	0	\\
1.1e-10	0	\\
2.2e-10	0	\\
3.3e-10	0	\\
4.4e-10	0	\\
5.4e-10	0	\\
6.5e-10	0	\\
7.5e-10	0	\\
8.6e-10	0	\\
9.6e-10	0	\\
1.07e-09	0	\\
1.18e-09	0	\\
1.28e-09	0	\\
1.38e-09	0	\\
1.49e-09	0	\\
1.59e-09	0	\\
1.69e-09	0	\\
1.8e-09	0	\\
1.9e-09	0	\\
2.01e-09	0	\\
2.11e-09	0	\\
2.21e-09	0	\\
2.32e-09	0	\\
2.42e-09	0	\\
2.52e-09	0	\\
2.63e-09	0	\\
2.73e-09	0	\\
2.83e-09	0	\\
2.93e-09	0	\\
3.04e-09	0	\\
3.14e-09	0	\\
3.24e-09	0	\\
3.34e-09	0	\\
3.45e-09	0	\\
3.55e-09	0	\\
3.65e-09	0	\\
3.75e-09	0	\\
3.86e-09	0	\\
3.96e-09	0	\\
4.06e-09	0	\\
4.16e-09	0	\\
4.27e-09	0	\\
4.37e-09	0	\\
4.47e-09	0	\\
4.57e-09	0	\\
4.68e-09	0	\\
4.78e-09	0	\\
4.89e-09	0	\\
4.99e-09	0	\\
5e-09	0	\\
};
\addplot [color=red,solid,forget plot]
  table[row sep=crcr]{
0	0	\\
1.1e-10	0	\\
2.2e-10	0	\\
3.3e-10	0	\\
4.4e-10	0	\\
5.4e-10	0	\\
6.5e-10	0	\\
7.5e-10	0	\\
8.6e-10	0	\\
9.6e-10	0	\\
1.07e-09	0	\\
1.18e-09	0	\\
1.28e-09	0	\\
1.38e-09	0	\\
1.49e-09	0	\\
1.59e-09	0	\\
1.69e-09	0	\\
1.8e-09	0	\\
1.9e-09	0	\\
2.01e-09	0	\\
2.11e-09	0	\\
2.21e-09	0	\\
2.32e-09	0	\\
2.42e-09	0	\\
2.52e-09	0	\\
2.63e-09	0	\\
2.73e-09	0	\\
2.83e-09	0	\\
2.93e-09	0	\\
3.04e-09	0	\\
3.14e-09	0	\\
3.24e-09	0	\\
3.34e-09	0	\\
3.45e-09	0	\\
3.55e-09	0	\\
3.65e-09	0	\\
3.75e-09	0	\\
3.86e-09	0	\\
3.96e-09	0	\\
4.06e-09	0	\\
4.16e-09	0	\\
4.27e-09	0	\\
4.37e-09	0	\\
4.47e-09	0	\\
4.57e-09	0	\\
4.68e-09	0	\\
4.78e-09	0	\\
4.89e-09	0	\\
4.99e-09	0	\\
5e-09	0	\\
};
\addplot [color=mycolor1,solid,forget plot]
  table[row sep=crcr]{
0	0	\\
1.1e-10	0	\\
2.2e-10	0	\\
3.3e-10	0	\\
4.4e-10	0	\\
5.4e-10	0	\\
6.5e-10	0	\\
7.5e-10	0	\\
8.6e-10	0	\\
9.6e-10	0	\\
1.07e-09	0	\\
1.18e-09	0	\\
1.28e-09	0	\\
1.38e-09	0	\\
1.49e-09	0	\\
1.59e-09	0	\\
1.69e-09	0	\\
1.8e-09	0	\\
1.9e-09	0	\\
2.01e-09	0	\\
2.11e-09	0	\\
2.21e-09	0	\\
2.32e-09	0	\\
2.42e-09	0	\\
2.52e-09	0	\\
2.63e-09	0	\\
2.73e-09	0	\\
2.83e-09	0	\\
2.93e-09	0	\\
3.04e-09	0	\\
3.14e-09	0	\\
3.24e-09	0	\\
3.34e-09	0	\\
3.45e-09	0	\\
3.55e-09	0	\\
3.65e-09	0	\\
3.75e-09	0	\\
3.86e-09	0	\\
3.96e-09	0	\\
4.06e-09	0	\\
4.16e-09	0	\\
4.27e-09	0	\\
4.37e-09	0	\\
4.47e-09	0	\\
4.57e-09	0	\\
4.68e-09	0	\\
4.78e-09	0	\\
4.89e-09	0	\\
4.99e-09	0	\\
5e-09	0	\\
};
\addplot [color=mycolor2,solid,forget plot]
  table[row sep=crcr]{
0	0	\\
1.1e-10	0	\\
2.2e-10	0	\\
3.3e-10	0	\\
4.4e-10	0	\\
5.4e-10	0	\\
6.5e-10	0	\\
7.5e-10	0	\\
8.6e-10	0	\\
9.6e-10	0	\\
1.07e-09	0	\\
1.18e-09	0	\\
1.28e-09	0	\\
1.38e-09	0	\\
1.49e-09	0	\\
1.59e-09	0	\\
1.69e-09	0	\\
1.8e-09	0	\\
1.9e-09	0	\\
2.01e-09	0	\\
2.11e-09	0	\\
2.21e-09	0	\\
2.32e-09	0	\\
2.42e-09	0	\\
2.52e-09	0	\\
2.63e-09	0	\\
2.73e-09	0	\\
2.83e-09	0	\\
2.93e-09	0	\\
3.04e-09	0	\\
3.14e-09	0	\\
3.24e-09	0	\\
3.34e-09	0	\\
3.45e-09	0	\\
3.55e-09	0	\\
3.65e-09	0	\\
3.75e-09	0	\\
3.86e-09	0	\\
3.96e-09	0	\\
4.06e-09	0	\\
4.16e-09	0	\\
4.27e-09	0	\\
4.37e-09	0	\\
4.47e-09	0	\\
4.57e-09	0	\\
4.68e-09	0	\\
4.78e-09	0	\\
4.89e-09	0	\\
4.99e-09	0	\\
5e-09	0	\\
};
\addplot [color=mycolor3,solid,forget plot]
  table[row sep=crcr]{
0	0	\\
1.1e-10	0	\\
2.2e-10	0	\\
3.3e-10	0	\\
4.4e-10	0	\\
5.4e-10	0	\\
6.5e-10	0	\\
7.5e-10	0	\\
8.6e-10	0	\\
9.6e-10	0	\\
1.07e-09	0	\\
1.18e-09	0	\\
1.28e-09	0	\\
1.38e-09	0	\\
1.49e-09	0	\\
1.59e-09	0	\\
1.69e-09	0	\\
1.8e-09	0	\\
1.9e-09	0	\\
2.01e-09	0	\\
2.11e-09	0	\\
2.21e-09	0	\\
2.32e-09	0	\\
2.42e-09	0	\\
2.52e-09	0	\\
2.63e-09	0	\\
2.73e-09	0	\\
2.83e-09	0	\\
2.93e-09	0	\\
3.04e-09	0	\\
3.14e-09	0	\\
3.24e-09	0	\\
3.34e-09	0	\\
3.45e-09	0	\\
3.55e-09	0	\\
3.65e-09	0	\\
3.75e-09	0	\\
3.86e-09	0	\\
3.96e-09	0	\\
4.06e-09	0	\\
4.16e-09	0	\\
4.27e-09	0	\\
4.37e-09	0	\\
4.47e-09	0	\\
4.57e-09	0	\\
4.68e-09	0	\\
4.78e-09	0	\\
4.89e-09	0	\\
4.99e-09	0	\\
5e-09	0	\\
};
\addplot [color=darkgray,solid,forget plot]
  table[row sep=crcr]{
0	0	\\
1.1e-10	0	\\
2.2e-10	0	\\
3.3e-10	0	\\
4.4e-10	0	\\
5.4e-10	0	\\
6.5e-10	0	\\
7.5e-10	0	\\
8.6e-10	0	\\
9.6e-10	0	\\
1.07e-09	0	\\
1.18e-09	0	\\
1.28e-09	0	\\
1.38e-09	0	\\
1.49e-09	0	\\
1.59e-09	0	\\
1.69e-09	0	\\
1.8e-09	0	\\
1.9e-09	0	\\
2.01e-09	0	\\
2.11e-09	0	\\
2.21e-09	0	\\
2.32e-09	0	\\
2.42e-09	0	\\
2.52e-09	0	\\
2.63e-09	0	\\
2.73e-09	0	\\
2.83e-09	0	\\
2.93e-09	0	\\
3.04e-09	0	\\
3.14e-09	0	\\
3.24e-09	0	\\
3.34e-09	0	\\
3.45e-09	0	\\
3.55e-09	0	\\
3.65e-09	0	\\
3.75e-09	0	\\
3.86e-09	0	\\
3.96e-09	0	\\
4.06e-09	0	\\
4.16e-09	0	\\
4.27e-09	0	\\
4.37e-09	0	\\
4.47e-09	0	\\
4.57e-09	0	\\
4.68e-09	0	\\
4.78e-09	0	\\
4.89e-09	0	\\
4.99e-09	0	\\
5e-09	0	\\
};
\addplot [color=blue,solid,forget plot]
  table[row sep=crcr]{
0	0	\\
1.1e-10	0	\\
2.2e-10	0	\\
3.3e-10	0	\\
4.4e-10	0	\\
5.4e-10	0	\\
6.5e-10	0	\\
7.5e-10	0	\\
8.6e-10	0	\\
9.6e-10	0	\\
1.07e-09	0	\\
1.18e-09	0	\\
1.28e-09	0	\\
1.38e-09	0	\\
1.49e-09	0	\\
1.59e-09	0	\\
1.69e-09	0	\\
1.8e-09	0	\\
1.9e-09	0	\\
2.01e-09	0	\\
2.11e-09	0	\\
2.21e-09	0	\\
2.32e-09	0	\\
2.42e-09	0	\\
2.52e-09	0	\\
2.63e-09	0	\\
2.73e-09	0	\\
2.83e-09	0	\\
2.93e-09	0	\\
3.04e-09	0	\\
3.14e-09	0	\\
3.24e-09	0	\\
3.34e-09	0	\\
3.45e-09	0	\\
3.55e-09	0	\\
3.65e-09	0	\\
3.75e-09	0	\\
3.86e-09	0	\\
3.96e-09	0	\\
4.06e-09	0	\\
4.16e-09	0	\\
4.27e-09	0	\\
4.37e-09	0	\\
4.47e-09	0	\\
4.57e-09	0	\\
4.68e-09	0	\\
4.78e-09	0	\\
4.89e-09	0	\\
4.99e-09	0	\\
5e-09	0	\\
};
\addplot [color=black!50!green,solid,forget plot]
  table[row sep=crcr]{
0	0	\\
1.1e-10	0	\\
2.2e-10	0	\\
3.3e-10	0	\\
4.4e-10	0	\\
5.4e-10	0	\\
6.5e-10	0	\\
7.5e-10	0	\\
8.6e-10	0	\\
9.6e-10	0	\\
1.07e-09	0	\\
1.18e-09	0	\\
1.28e-09	0	\\
1.38e-09	0	\\
1.49e-09	0	\\
1.59e-09	0	\\
1.69e-09	0	\\
1.8e-09	0	\\
1.9e-09	0	\\
2.01e-09	0	\\
2.11e-09	0	\\
2.21e-09	0	\\
2.32e-09	0	\\
2.42e-09	0	\\
2.52e-09	0	\\
2.63e-09	0	\\
2.73e-09	0	\\
2.83e-09	0	\\
2.93e-09	0	\\
3.04e-09	0	\\
3.14e-09	0	\\
3.24e-09	0	\\
3.34e-09	0	\\
3.45e-09	0	\\
3.55e-09	0	\\
3.65e-09	0	\\
3.75e-09	0	\\
3.86e-09	0	\\
3.96e-09	0	\\
4.06e-09	0	\\
4.16e-09	0	\\
4.27e-09	0	\\
4.37e-09	0	\\
4.47e-09	0	\\
4.57e-09	0	\\
4.68e-09	0	\\
4.78e-09	0	\\
4.89e-09	0	\\
4.99e-09	0	\\
5e-09	0	\\
};
\addplot [color=red,solid,forget plot]
  table[row sep=crcr]{
0	0	\\
1.1e-10	0	\\
2.2e-10	0	\\
3.3e-10	0	\\
4.4e-10	0	\\
5.4e-10	0	\\
6.5e-10	0	\\
7.5e-10	0	\\
8.6e-10	0	\\
9.6e-10	0	\\
1.07e-09	0	\\
1.18e-09	0	\\
1.28e-09	0	\\
1.38e-09	0	\\
1.49e-09	0	\\
1.59e-09	0	\\
1.69e-09	0	\\
1.8e-09	0	\\
1.9e-09	0	\\
2.01e-09	0	\\
2.11e-09	0	\\
2.21e-09	0	\\
2.32e-09	0	\\
2.42e-09	0	\\
2.52e-09	0	\\
2.63e-09	0	\\
2.73e-09	0	\\
2.83e-09	0	\\
2.93e-09	0	\\
3.04e-09	0	\\
3.14e-09	0	\\
3.24e-09	0	\\
3.34e-09	0	\\
3.45e-09	0	\\
3.55e-09	0	\\
3.65e-09	0	\\
3.75e-09	0	\\
3.86e-09	0	\\
3.96e-09	0	\\
4.06e-09	0	\\
4.16e-09	0	\\
4.27e-09	0	\\
4.37e-09	0	\\
4.47e-09	0	\\
4.57e-09	0	\\
4.68e-09	0	\\
4.78e-09	0	\\
4.89e-09	0	\\
4.99e-09	0	\\
5e-09	0	\\
};
\addplot [color=mycolor1,solid,forget plot]
  table[row sep=crcr]{
0	0	\\
1.1e-10	0	\\
2.2e-10	0	\\
3.3e-10	0	\\
4.4e-10	0	\\
5.4e-10	0	\\
6.5e-10	0	\\
7.5e-10	0	\\
8.6e-10	0	\\
9.6e-10	0	\\
1.07e-09	0	\\
1.18e-09	0	\\
1.28e-09	0	\\
1.38e-09	0	\\
1.49e-09	0	\\
1.59e-09	0	\\
1.69e-09	0	\\
1.8e-09	0	\\
1.9e-09	0	\\
2.01e-09	0	\\
2.11e-09	0	\\
2.21e-09	0	\\
2.32e-09	0	\\
2.42e-09	0	\\
2.52e-09	0	\\
2.63e-09	0	\\
2.73e-09	0	\\
2.83e-09	0	\\
2.93e-09	0	\\
3.04e-09	0	\\
3.14e-09	0	\\
3.24e-09	0	\\
3.34e-09	0	\\
3.45e-09	0	\\
3.55e-09	0	\\
3.65e-09	0	\\
3.75e-09	0	\\
3.86e-09	0	\\
3.96e-09	0	\\
4.06e-09	0	\\
4.16e-09	0	\\
4.27e-09	0	\\
4.37e-09	0	\\
4.47e-09	0	\\
4.57e-09	0	\\
4.68e-09	0	\\
4.78e-09	0	\\
4.89e-09	0	\\
4.99e-09	0	\\
5e-09	0	\\
};
\addplot [color=mycolor2,solid,forget plot]
  table[row sep=crcr]{
0	0	\\
1.1e-10	0	\\
2.2e-10	0	\\
3.3e-10	0	\\
4.4e-10	0	\\
5.4e-10	0	\\
6.5e-10	0	\\
7.5e-10	0	\\
8.6e-10	0	\\
9.6e-10	0	\\
1.07e-09	0	\\
1.18e-09	0	\\
1.28e-09	0	\\
1.38e-09	0	\\
1.49e-09	0	\\
1.59e-09	0	\\
1.69e-09	0	\\
1.8e-09	0	\\
1.9e-09	0	\\
2.01e-09	0	\\
2.11e-09	0	\\
2.21e-09	0	\\
2.32e-09	0	\\
2.42e-09	0	\\
2.52e-09	0	\\
2.63e-09	0	\\
2.73e-09	0	\\
2.83e-09	0	\\
2.93e-09	0	\\
3.04e-09	0	\\
3.14e-09	0	\\
3.24e-09	0	\\
3.34e-09	0	\\
3.45e-09	0	\\
3.55e-09	0	\\
3.65e-09	0	\\
3.75e-09	0	\\
3.86e-09	0	\\
3.96e-09	0	\\
4.06e-09	0	\\
4.16e-09	0	\\
4.27e-09	0	\\
4.37e-09	0	\\
4.47e-09	0	\\
4.57e-09	0	\\
4.68e-09	0	\\
4.78e-09	0	\\
4.89e-09	0	\\
4.99e-09	0	\\
5e-09	0	\\
};
\addplot [color=mycolor3,solid,forget plot]
  table[row sep=crcr]{
0	0	\\
1.1e-10	0	\\
2.2e-10	0	\\
3.3e-10	0	\\
4.4e-10	0	\\
5.4e-10	0	\\
6.5e-10	0	\\
7.5e-10	0	\\
8.6e-10	0	\\
9.6e-10	0	\\
1.07e-09	0	\\
1.18e-09	0	\\
1.28e-09	0	\\
1.38e-09	0	\\
1.49e-09	0	\\
1.59e-09	0	\\
1.69e-09	0	\\
1.8e-09	0	\\
1.9e-09	0	\\
2.01e-09	0	\\
2.11e-09	0	\\
2.21e-09	0	\\
2.32e-09	0	\\
2.42e-09	0	\\
2.52e-09	0	\\
2.63e-09	0	\\
2.73e-09	0	\\
2.83e-09	0	\\
2.93e-09	0	\\
3.04e-09	0	\\
3.14e-09	0	\\
3.24e-09	0	\\
3.34e-09	0	\\
3.45e-09	0	\\
3.55e-09	0	\\
3.65e-09	0	\\
3.75e-09	0	\\
3.86e-09	0	\\
3.96e-09	0	\\
4.06e-09	0	\\
4.16e-09	0	\\
4.27e-09	0	\\
4.37e-09	0	\\
4.47e-09	0	\\
4.57e-09	0	\\
4.68e-09	0	\\
4.78e-09	0	\\
4.89e-09	0	\\
4.99e-09	0	\\
5e-09	0	\\
};
\addplot [color=darkgray,solid,forget plot]
  table[row sep=crcr]{
0	0	\\
1.1e-10	0	\\
2.2e-10	0	\\
3.3e-10	0	\\
4.4e-10	0	\\
5.4e-10	0	\\
6.5e-10	0	\\
7.5e-10	0	\\
8.6e-10	0	\\
9.6e-10	0	\\
1.07e-09	0	\\
1.18e-09	0	\\
1.28e-09	0	\\
1.38e-09	0	\\
1.49e-09	0	\\
1.59e-09	0	\\
1.69e-09	0	\\
1.8e-09	0	\\
1.9e-09	0	\\
2.01e-09	0	\\
2.11e-09	0	\\
2.21e-09	0	\\
2.32e-09	0	\\
2.42e-09	0	\\
2.52e-09	0	\\
2.63e-09	0	\\
2.73e-09	0	\\
2.83e-09	0	\\
2.93e-09	0	\\
3.04e-09	0	\\
3.14e-09	0	\\
3.24e-09	0	\\
3.34e-09	0	\\
3.45e-09	0	\\
3.55e-09	0	\\
3.65e-09	0	\\
3.75e-09	0	\\
3.86e-09	0	\\
3.96e-09	0	\\
4.06e-09	0	\\
4.16e-09	0	\\
4.27e-09	0	\\
4.37e-09	0	\\
4.47e-09	0	\\
4.57e-09	0	\\
4.68e-09	0	\\
4.78e-09	0	\\
4.89e-09	0	\\
4.99e-09	0	\\
5e-09	0	\\
};
\addplot [color=blue,solid,forget plot]
  table[row sep=crcr]{
0	0	\\
1.1e-10	0	\\
2.2e-10	0	\\
3.3e-10	0	\\
4.4e-10	0	\\
5.4e-10	0	\\
6.5e-10	0	\\
7.5e-10	0	\\
8.6e-10	0	\\
9.6e-10	0	\\
1.07e-09	0	\\
1.18e-09	0	\\
1.28e-09	0	\\
1.38e-09	0	\\
1.49e-09	0	\\
1.59e-09	0	\\
1.69e-09	0	\\
1.8e-09	0	\\
1.9e-09	0	\\
2.01e-09	0	\\
2.11e-09	0	\\
2.21e-09	0	\\
2.32e-09	0	\\
2.42e-09	0	\\
2.52e-09	0	\\
2.63e-09	0	\\
2.73e-09	0	\\
2.83e-09	0	\\
2.93e-09	0	\\
3.04e-09	0	\\
3.14e-09	0	\\
3.24e-09	0	\\
3.34e-09	0	\\
3.45e-09	0	\\
3.55e-09	0	\\
3.65e-09	0	\\
3.75e-09	0	\\
3.86e-09	0	\\
3.96e-09	0	\\
4.06e-09	0	\\
4.16e-09	0	\\
4.27e-09	0	\\
4.37e-09	0	\\
4.47e-09	0	\\
4.57e-09	0	\\
4.68e-09	0	\\
4.78e-09	0	\\
4.89e-09	0	\\
4.99e-09	0	\\
5e-09	0	\\
};
\addplot [color=black!50!green,solid,forget plot]
  table[row sep=crcr]{
0	0	\\
1.1e-10	0	\\
2.2e-10	0	\\
3.3e-10	0	\\
4.4e-10	0	\\
5.4e-10	0	\\
6.5e-10	0	\\
7.5e-10	0	\\
8.6e-10	0	\\
9.6e-10	0	\\
1.07e-09	0	\\
1.18e-09	0	\\
1.28e-09	0	\\
1.38e-09	0	\\
1.49e-09	0	\\
1.59e-09	0	\\
1.69e-09	0	\\
1.8e-09	0	\\
1.9e-09	0	\\
2.01e-09	0	\\
2.11e-09	0	\\
2.21e-09	0	\\
2.32e-09	0	\\
2.42e-09	0	\\
2.52e-09	0	\\
2.63e-09	0	\\
2.73e-09	0	\\
2.83e-09	0	\\
2.93e-09	0	\\
3.04e-09	0	\\
3.14e-09	0	\\
3.24e-09	0	\\
3.34e-09	0	\\
3.45e-09	0	\\
3.55e-09	0	\\
3.65e-09	0	\\
3.75e-09	0	\\
3.86e-09	0	\\
3.96e-09	0	\\
4.06e-09	0	\\
4.16e-09	0	\\
4.27e-09	0	\\
4.37e-09	0	\\
4.47e-09	0	\\
4.57e-09	0	\\
4.68e-09	0	\\
4.78e-09	0	\\
4.89e-09	0	\\
4.99e-09	0	\\
5e-09	0	\\
};
\addplot [color=red,solid,forget plot]
  table[row sep=crcr]{
0	0	\\
1.1e-10	0	\\
2.2e-10	0	\\
3.3e-10	0	\\
4.4e-10	0	\\
5.4e-10	0	\\
6.5e-10	0	\\
7.5e-10	0	\\
8.6e-10	0	\\
9.6e-10	0	\\
1.07e-09	0	\\
1.18e-09	0	\\
1.28e-09	0	\\
1.38e-09	0	\\
1.49e-09	0	\\
1.59e-09	0	\\
1.69e-09	0	\\
1.8e-09	0	\\
1.9e-09	0	\\
2.01e-09	0	\\
2.11e-09	0	\\
2.21e-09	0	\\
2.32e-09	0	\\
2.42e-09	0	\\
2.52e-09	0	\\
2.63e-09	0	\\
2.73e-09	0	\\
2.83e-09	0	\\
2.93e-09	0	\\
3.04e-09	0	\\
3.14e-09	0	\\
3.24e-09	0	\\
3.34e-09	0	\\
3.45e-09	0	\\
3.55e-09	0	\\
3.65e-09	0	\\
3.75e-09	0	\\
3.86e-09	0	\\
3.96e-09	0	\\
4.06e-09	0	\\
4.16e-09	0	\\
4.27e-09	0	\\
4.37e-09	0	\\
4.47e-09	0	\\
4.57e-09	0	\\
4.68e-09	0	\\
4.78e-09	0	\\
4.89e-09	0	\\
4.99e-09	0	\\
5e-09	0	\\
};
\addplot [color=mycolor1,solid,forget plot]
  table[row sep=crcr]{
0	0	\\
1.1e-10	0	\\
2.2e-10	0	\\
3.3e-10	0	\\
4.4e-10	0	\\
5.4e-10	0	\\
6.5e-10	0	\\
7.5e-10	0	\\
8.6e-10	0	\\
9.6e-10	0	\\
1.07e-09	0	\\
1.18e-09	0	\\
1.28e-09	0	\\
1.38e-09	0	\\
1.49e-09	0	\\
1.59e-09	0	\\
1.69e-09	0	\\
1.8e-09	0	\\
1.9e-09	0	\\
2.01e-09	0	\\
2.11e-09	0	\\
2.21e-09	0	\\
2.32e-09	0	\\
2.42e-09	0	\\
2.52e-09	0	\\
2.63e-09	0	\\
2.73e-09	0	\\
2.83e-09	0	\\
2.93e-09	0	\\
3.04e-09	0	\\
3.14e-09	0	\\
3.24e-09	0	\\
3.34e-09	0	\\
3.45e-09	0	\\
3.55e-09	0	\\
3.65e-09	0	\\
3.75e-09	0	\\
3.86e-09	0	\\
3.96e-09	0	\\
4.06e-09	0	\\
4.16e-09	0	\\
4.27e-09	0	\\
4.37e-09	0	\\
4.47e-09	0	\\
4.57e-09	0	\\
4.68e-09	0	\\
4.78e-09	0	\\
4.89e-09	0	\\
4.99e-09	0	\\
5e-09	0	\\
};
\addplot [color=mycolor2,solid,forget plot]
  table[row sep=crcr]{
0	0	\\
1.1e-10	0	\\
2.2e-10	0	\\
3.3e-10	0	\\
4.4e-10	0	\\
5.4e-10	0	\\
6.5e-10	0	\\
7.5e-10	0	\\
8.6e-10	0	\\
9.6e-10	0	\\
1.07e-09	0	\\
1.18e-09	0	\\
1.28e-09	0	\\
1.38e-09	0	\\
1.49e-09	0	\\
1.59e-09	0	\\
1.69e-09	0	\\
1.8e-09	0	\\
1.9e-09	0	\\
2.01e-09	0	\\
2.11e-09	0	\\
2.21e-09	0	\\
2.32e-09	0	\\
2.42e-09	0	\\
2.52e-09	0	\\
2.63e-09	0	\\
2.73e-09	0	\\
2.83e-09	0	\\
2.93e-09	0	\\
3.04e-09	0	\\
3.14e-09	0	\\
3.24e-09	0	\\
3.34e-09	0	\\
3.45e-09	0	\\
3.55e-09	0	\\
3.65e-09	0	\\
3.75e-09	0	\\
3.86e-09	0	\\
3.96e-09	0	\\
4.06e-09	0	\\
4.16e-09	0	\\
4.27e-09	0	\\
4.37e-09	0	\\
4.47e-09	0	\\
4.57e-09	0	\\
4.68e-09	0	\\
4.78e-09	0	\\
4.89e-09	0	\\
4.99e-09	0	\\
5e-09	0	\\
};
\addplot [color=mycolor3,solid,forget plot]
  table[row sep=crcr]{
0	0	\\
1.1e-10	0	\\
2.2e-10	0	\\
3.3e-10	0	\\
4.4e-10	0	\\
5.4e-10	0	\\
6.5e-10	0	\\
7.5e-10	0	\\
8.6e-10	0	\\
9.6e-10	0	\\
1.07e-09	0	\\
1.18e-09	0	\\
1.28e-09	0	\\
1.38e-09	0	\\
1.49e-09	0	\\
1.59e-09	0	\\
1.69e-09	0	\\
1.8e-09	0	\\
1.9e-09	0	\\
2.01e-09	0	\\
2.11e-09	0	\\
2.21e-09	0	\\
2.32e-09	0	\\
2.42e-09	0	\\
2.52e-09	0	\\
2.63e-09	0	\\
2.73e-09	0	\\
2.83e-09	0	\\
2.93e-09	0	\\
3.04e-09	0	\\
3.14e-09	0	\\
3.24e-09	0	\\
3.34e-09	0	\\
3.45e-09	0	\\
3.55e-09	0	\\
3.65e-09	0	\\
3.75e-09	0	\\
3.86e-09	0	\\
3.96e-09	0	\\
4.06e-09	0	\\
4.16e-09	0	\\
4.27e-09	0	\\
4.37e-09	0	\\
4.47e-09	0	\\
4.57e-09	0	\\
4.68e-09	0	\\
4.78e-09	0	\\
4.89e-09	0	\\
4.99e-09	0	\\
5e-09	0	\\
};
\addplot [color=darkgray,solid,forget plot]
  table[row sep=crcr]{
0	0	\\
1.1e-10	0	\\
2.2e-10	0	\\
3.3e-10	0	\\
4.4e-10	0	\\
5.4e-10	0	\\
6.5e-10	0	\\
7.5e-10	0	\\
8.6e-10	0	\\
9.6e-10	0	\\
1.07e-09	0	\\
1.18e-09	0	\\
1.28e-09	0	\\
1.38e-09	0	\\
1.49e-09	0	\\
1.59e-09	0	\\
1.69e-09	0	\\
1.8e-09	0	\\
1.9e-09	0	\\
2.01e-09	0	\\
2.11e-09	0	\\
2.21e-09	0	\\
2.32e-09	0	\\
2.42e-09	0	\\
2.52e-09	0	\\
2.63e-09	0	\\
2.73e-09	0	\\
2.83e-09	0	\\
2.93e-09	0	\\
3.04e-09	0	\\
3.14e-09	0	\\
3.24e-09	0	\\
3.34e-09	0	\\
3.45e-09	0	\\
3.55e-09	0	\\
3.65e-09	0	\\
3.75e-09	0	\\
3.86e-09	0	\\
3.96e-09	0	\\
4.06e-09	0	\\
4.16e-09	0	\\
4.27e-09	0	\\
4.37e-09	0	\\
4.47e-09	0	\\
4.57e-09	0	\\
4.68e-09	0	\\
4.78e-09	0	\\
4.89e-09	0	\\
4.99e-09	0	\\
5e-09	0	\\
};
\addplot [color=blue,solid,forget plot]
  table[row sep=crcr]{
0	0	\\
1.1e-10	0	\\
2.2e-10	0	\\
3.3e-10	0	\\
4.4e-10	0	\\
5.4e-10	0	\\
6.5e-10	0	\\
7.5e-10	0	\\
8.6e-10	0	\\
9.6e-10	0	\\
1.07e-09	0	\\
1.18e-09	0	\\
1.28e-09	0	\\
1.38e-09	0	\\
1.49e-09	0	\\
1.59e-09	0	\\
1.69e-09	0	\\
1.8e-09	0	\\
1.9e-09	0	\\
2.01e-09	0	\\
2.11e-09	0	\\
2.21e-09	0	\\
2.32e-09	0	\\
2.42e-09	0	\\
2.52e-09	0	\\
2.63e-09	0	\\
2.73e-09	0	\\
2.83e-09	0	\\
2.93e-09	0	\\
3.04e-09	0	\\
3.14e-09	0	\\
3.24e-09	0	\\
3.34e-09	0	\\
3.45e-09	0	\\
3.55e-09	0	\\
3.65e-09	0	\\
3.75e-09	0	\\
3.86e-09	0	\\
3.96e-09	0	\\
4.06e-09	0	\\
4.16e-09	0	\\
4.27e-09	0	\\
4.37e-09	0	\\
4.47e-09	0	\\
4.57e-09	0	\\
4.68e-09	0	\\
4.78e-09	0	\\
4.89e-09	0	\\
4.99e-09	0	\\
5e-09	0	\\
};
\addplot [color=black!50!green,solid,forget plot]
  table[row sep=crcr]{
0	0	\\
1.1e-10	0	\\
2.2e-10	0	\\
3.3e-10	0	\\
4.4e-10	0	\\
5.4e-10	0	\\
6.5e-10	0	\\
7.5e-10	0	\\
8.6e-10	0	\\
9.6e-10	0	\\
1.07e-09	0	\\
1.18e-09	0	\\
1.28e-09	0	\\
1.38e-09	0	\\
1.49e-09	0	\\
1.59e-09	0	\\
1.69e-09	0	\\
1.8e-09	0	\\
1.9e-09	0	\\
2.01e-09	0	\\
2.11e-09	0	\\
2.21e-09	0	\\
2.32e-09	0	\\
2.42e-09	0	\\
2.52e-09	0	\\
2.63e-09	0	\\
2.73e-09	0	\\
2.83e-09	0	\\
2.93e-09	0	\\
3.04e-09	0	\\
3.14e-09	0	\\
3.24e-09	0	\\
3.34e-09	0	\\
3.45e-09	0	\\
3.55e-09	0	\\
3.65e-09	0	\\
3.75e-09	0	\\
3.86e-09	0	\\
3.96e-09	0	\\
4.06e-09	0	\\
4.16e-09	0	\\
4.27e-09	0	\\
4.37e-09	0	\\
4.47e-09	0	\\
4.57e-09	0	\\
4.68e-09	0	\\
4.78e-09	0	\\
4.89e-09	0	\\
4.99e-09	0	\\
5e-09	0	\\
};
\addplot [color=red,solid,forget plot]
  table[row sep=crcr]{
0	0	\\
1.1e-10	0	\\
2.2e-10	0	\\
3.3e-10	0	\\
4.4e-10	0	\\
5.4e-10	0	\\
6.5e-10	0	\\
7.5e-10	0	\\
8.6e-10	0	\\
9.6e-10	0	\\
1.07e-09	0	\\
1.18e-09	0	\\
1.28e-09	0	\\
1.38e-09	0	\\
1.49e-09	0	\\
1.59e-09	0	\\
1.69e-09	0	\\
1.8e-09	0	\\
1.9e-09	0	\\
2.01e-09	0	\\
2.11e-09	0	\\
2.21e-09	0	\\
2.32e-09	0	\\
2.42e-09	0	\\
2.52e-09	0	\\
2.63e-09	0	\\
2.73e-09	0	\\
2.83e-09	0	\\
2.93e-09	0	\\
3.04e-09	0	\\
3.14e-09	0	\\
3.24e-09	0	\\
3.34e-09	0	\\
3.45e-09	0	\\
3.55e-09	0	\\
3.65e-09	0	\\
3.75e-09	0	\\
3.86e-09	0	\\
3.96e-09	0	\\
4.06e-09	0	\\
4.16e-09	0	\\
4.27e-09	0	\\
4.37e-09	0	\\
4.47e-09	0	\\
4.57e-09	0	\\
4.68e-09	0	\\
4.78e-09	0	\\
4.89e-09	0	\\
4.99e-09	0	\\
5e-09	0	\\
};
\addplot [color=mycolor1,solid,forget plot]
  table[row sep=crcr]{
0	0	\\
1.1e-10	0	\\
2.2e-10	0	\\
3.3e-10	0	\\
4.4e-10	0	\\
5.4e-10	0	\\
6.5e-10	0	\\
7.5e-10	0	\\
8.6e-10	0	\\
9.6e-10	0	\\
1.07e-09	0	\\
1.18e-09	0	\\
1.28e-09	0	\\
1.38e-09	0	\\
1.49e-09	0	\\
1.59e-09	0	\\
1.69e-09	0	\\
1.8e-09	0	\\
1.9e-09	0	\\
2.01e-09	0	\\
2.11e-09	0	\\
2.21e-09	0	\\
2.32e-09	0	\\
2.42e-09	0	\\
2.52e-09	0	\\
2.63e-09	0	\\
2.73e-09	0	\\
2.83e-09	0	\\
2.93e-09	0	\\
3.04e-09	0	\\
3.14e-09	0	\\
3.24e-09	0	\\
3.34e-09	0	\\
3.45e-09	0	\\
3.55e-09	0	\\
3.65e-09	0	\\
3.75e-09	0	\\
3.86e-09	0	\\
3.96e-09	0	\\
4.06e-09	0	\\
4.16e-09	0	\\
4.27e-09	0	\\
4.37e-09	0	\\
4.47e-09	0	\\
4.57e-09	0	\\
4.68e-09	0	\\
4.78e-09	0	\\
4.89e-09	0	\\
4.99e-09	0	\\
5e-09	0	\\
};
\addplot [color=mycolor2,solid,forget plot]
  table[row sep=crcr]{
0	0	\\
1.1e-10	0	\\
2.2e-10	0	\\
3.3e-10	0	\\
4.4e-10	0	\\
5.4e-10	0	\\
6.5e-10	0	\\
7.5e-10	0	\\
8.6e-10	0	\\
9.6e-10	0	\\
1.07e-09	0	\\
1.18e-09	0	\\
1.28e-09	0	\\
1.38e-09	0	\\
1.49e-09	0	\\
1.59e-09	0	\\
1.69e-09	0	\\
1.8e-09	0	\\
1.9e-09	0	\\
2.01e-09	0	\\
2.11e-09	0	\\
2.21e-09	0	\\
2.32e-09	0	\\
2.42e-09	0	\\
2.52e-09	0	\\
2.63e-09	0	\\
2.73e-09	0	\\
2.83e-09	0	\\
2.93e-09	0	\\
3.04e-09	0	\\
3.14e-09	0	\\
3.24e-09	0	\\
3.34e-09	0	\\
3.45e-09	0	\\
3.55e-09	0	\\
3.65e-09	0	\\
3.75e-09	0	\\
3.86e-09	0	\\
3.96e-09	0	\\
4.06e-09	0	\\
4.16e-09	0	\\
4.27e-09	0	\\
4.37e-09	0	\\
4.47e-09	0	\\
4.57e-09	0	\\
4.68e-09	0	\\
4.78e-09	0	\\
4.89e-09	0	\\
4.99e-09	0	\\
5e-09	0	\\
};
\addplot [color=mycolor3,solid,forget plot]
  table[row sep=crcr]{
0	0	\\
1.1e-10	0	\\
2.2e-10	0	\\
3.3e-10	0	\\
4.4e-10	0	\\
5.4e-10	0	\\
6.5e-10	0	\\
7.5e-10	0	\\
8.6e-10	0	\\
9.6e-10	0	\\
1.07e-09	0	\\
1.18e-09	0	\\
1.28e-09	0	\\
1.38e-09	0	\\
1.49e-09	0	\\
1.59e-09	0	\\
1.69e-09	0	\\
1.8e-09	0	\\
1.9e-09	0	\\
2.01e-09	0	\\
2.11e-09	0	\\
2.21e-09	0	\\
2.32e-09	0	\\
2.42e-09	0	\\
2.52e-09	0	\\
2.63e-09	0	\\
2.73e-09	0	\\
2.83e-09	0	\\
2.93e-09	0	\\
3.04e-09	0	\\
3.14e-09	0	\\
3.24e-09	0	\\
3.34e-09	0	\\
3.45e-09	0	\\
3.55e-09	0	\\
3.65e-09	0	\\
3.75e-09	0	\\
3.86e-09	0	\\
3.96e-09	0	\\
4.06e-09	0	\\
4.16e-09	0	\\
4.27e-09	0	\\
4.37e-09	0	\\
4.47e-09	0	\\
4.57e-09	0	\\
4.68e-09	0	\\
4.78e-09	0	\\
4.89e-09	0	\\
4.99e-09	0	\\
5e-09	0	\\
};
\addplot [color=darkgray,solid,forget plot]
  table[row sep=crcr]{
0	0	\\
1.1e-10	0	\\
2.2e-10	0	\\
3.3e-10	0	\\
4.4e-10	0	\\
5.4e-10	0	\\
6.5e-10	0	\\
7.5e-10	0	\\
8.6e-10	0	\\
9.6e-10	0	\\
1.07e-09	0	\\
1.18e-09	0	\\
1.28e-09	0	\\
1.38e-09	0	\\
1.49e-09	0	\\
1.59e-09	0	\\
1.69e-09	0	\\
1.8e-09	0	\\
1.9e-09	0	\\
2.01e-09	0	\\
2.11e-09	0	\\
2.21e-09	0	\\
2.32e-09	0	\\
2.42e-09	0	\\
2.52e-09	0	\\
2.63e-09	0	\\
2.73e-09	0	\\
2.83e-09	0	\\
2.93e-09	0	\\
3.04e-09	0	\\
3.14e-09	0	\\
3.24e-09	0	\\
3.34e-09	0	\\
3.45e-09	0	\\
3.55e-09	0	\\
3.65e-09	0	\\
3.75e-09	0	\\
3.86e-09	0	\\
3.96e-09	0	\\
4.06e-09	0	\\
4.16e-09	0	\\
4.27e-09	0	\\
4.37e-09	0	\\
4.47e-09	0	\\
4.57e-09	0	\\
4.68e-09	0	\\
4.78e-09	0	\\
4.89e-09	0	\\
4.99e-09	0	\\
5e-09	0	\\
};
\addplot [color=blue,solid,forget plot]
  table[row sep=crcr]{
0	0	\\
1.1e-10	0	\\
2.2e-10	0	\\
3.3e-10	0	\\
4.4e-10	0	\\
5.4e-10	0	\\
6.5e-10	0	\\
7.5e-10	0	\\
8.6e-10	0	\\
9.6e-10	0	\\
1.07e-09	0	\\
1.18e-09	0	\\
1.28e-09	0	\\
1.38e-09	0	\\
1.49e-09	0	\\
1.59e-09	0	\\
1.69e-09	0	\\
1.8e-09	0	\\
1.9e-09	0	\\
2.01e-09	0	\\
2.11e-09	0	\\
2.21e-09	0	\\
2.32e-09	0	\\
2.42e-09	0	\\
2.52e-09	0	\\
2.63e-09	0	\\
2.73e-09	0	\\
2.83e-09	0	\\
2.93e-09	0	\\
3.04e-09	0	\\
3.14e-09	0	\\
3.24e-09	0	\\
3.34e-09	0	\\
3.45e-09	0	\\
3.55e-09	0	\\
3.65e-09	0	\\
3.75e-09	0	\\
3.86e-09	0	\\
3.96e-09	0	\\
4.06e-09	0	\\
4.16e-09	0	\\
4.27e-09	0	\\
4.37e-09	0	\\
4.47e-09	0	\\
4.57e-09	0	\\
4.68e-09	0	\\
4.78e-09	0	\\
4.89e-09	0	\\
4.99e-09	0	\\
5e-09	0	\\
};
\addplot [color=black!50!green,solid,forget plot]
  table[row sep=crcr]{
0	0	\\
1.1e-10	0	\\
2.2e-10	0	\\
3.3e-10	0	\\
4.4e-10	0	\\
5.4e-10	0	\\
6.5e-10	0	\\
7.5e-10	0	\\
8.6e-10	0	\\
9.6e-10	0	\\
1.07e-09	0	\\
1.18e-09	0	\\
1.28e-09	0	\\
1.38e-09	0	\\
1.49e-09	0	\\
1.59e-09	0	\\
1.69e-09	0	\\
1.8e-09	0	\\
1.9e-09	0	\\
2.01e-09	0	\\
2.11e-09	0	\\
2.21e-09	0	\\
2.32e-09	0	\\
2.42e-09	0	\\
2.52e-09	0	\\
2.63e-09	0	\\
2.73e-09	0	\\
2.83e-09	0	\\
2.93e-09	0	\\
3.04e-09	0	\\
3.14e-09	0	\\
3.24e-09	0	\\
3.34e-09	0	\\
3.45e-09	0	\\
3.55e-09	0	\\
3.65e-09	0	\\
3.75e-09	0	\\
3.86e-09	0	\\
3.96e-09	0	\\
4.06e-09	0	\\
4.16e-09	0	\\
4.27e-09	0	\\
4.37e-09	0	\\
4.47e-09	0	\\
4.57e-09	0	\\
4.68e-09	0	\\
4.78e-09	0	\\
4.89e-09	0	\\
4.99e-09	0	\\
5e-09	0	\\
};
\addplot [color=red,solid,forget plot]
  table[row sep=crcr]{
0	0	\\
1.1e-10	0	\\
2.2e-10	0	\\
3.3e-10	0	\\
4.4e-10	0	\\
5.4e-10	0	\\
6.5e-10	0	\\
7.5e-10	0	\\
8.6e-10	0	\\
9.6e-10	0	\\
1.07e-09	0	\\
1.18e-09	0	\\
1.28e-09	0	\\
1.38e-09	0	\\
1.49e-09	0	\\
1.59e-09	0	\\
1.69e-09	0	\\
1.8e-09	0	\\
1.9e-09	0	\\
2.01e-09	0	\\
2.11e-09	0	\\
2.21e-09	0	\\
2.32e-09	0	\\
2.42e-09	0	\\
2.52e-09	0	\\
2.63e-09	0	\\
2.73e-09	0	\\
2.83e-09	0	\\
2.93e-09	0	\\
3.04e-09	0	\\
3.14e-09	0	\\
3.24e-09	0	\\
3.34e-09	0	\\
3.45e-09	0	\\
3.55e-09	0	\\
3.65e-09	0	\\
3.75e-09	0	\\
3.86e-09	0	\\
3.96e-09	0	\\
4.06e-09	0	\\
4.16e-09	0	\\
4.27e-09	0	\\
4.37e-09	0	\\
4.47e-09	0	\\
4.57e-09	0	\\
4.68e-09	0	\\
4.78e-09	0	\\
4.89e-09	0	\\
4.99e-09	0	\\
5e-09	0	\\
};
\addplot [color=mycolor1,solid,forget plot]
  table[row sep=crcr]{
0	0	\\
1.1e-10	0	\\
2.2e-10	0	\\
3.3e-10	0	\\
4.4e-10	0	\\
5.4e-10	0	\\
6.5e-10	0	\\
7.5e-10	0	\\
8.6e-10	0	\\
9.6e-10	0	\\
1.07e-09	0	\\
1.18e-09	0	\\
1.28e-09	0	\\
1.38e-09	0	\\
1.49e-09	0	\\
1.59e-09	0	\\
1.69e-09	0	\\
1.8e-09	0	\\
1.9e-09	0	\\
2.01e-09	0	\\
2.11e-09	0	\\
2.21e-09	0	\\
2.32e-09	0	\\
2.42e-09	0	\\
2.52e-09	0	\\
2.63e-09	0	\\
2.73e-09	0	\\
2.83e-09	0	\\
2.93e-09	0	\\
3.04e-09	0	\\
3.14e-09	0	\\
3.24e-09	0	\\
3.34e-09	0	\\
3.45e-09	0	\\
3.55e-09	0	\\
3.65e-09	0	\\
3.75e-09	0	\\
3.86e-09	0	\\
3.96e-09	0	\\
4.06e-09	0	\\
4.16e-09	0	\\
4.27e-09	0	\\
4.37e-09	0	\\
4.47e-09	0	\\
4.57e-09	0	\\
4.68e-09	0	\\
4.78e-09	0	\\
4.89e-09	0	\\
4.99e-09	0	\\
5e-09	0	\\
};
\addplot [color=mycolor2,solid,forget plot]
  table[row sep=crcr]{
0	0	\\
1.1e-10	0	\\
2.2e-10	0	\\
3.3e-10	0	\\
4.4e-10	0	\\
5.4e-10	0	\\
6.5e-10	0	\\
7.5e-10	0	\\
8.6e-10	0	\\
9.6e-10	0	\\
1.07e-09	0	\\
1.18e-09	0	\\
1.28e-09	0	\\
1.38e-09	0	\\
1.49e-09	0	\\
1.59e-09	0	\\
1.69e-09	0	\\
1.8e-09	0	\\
1.9e-09	0	\\
2.01e-09	0	\\
2.11e-09	0	\\
2.21e-09	0	\\
2.32e-09	0	\\
2.42e-09	0	\\
2.52e-09	0	\\
2.63e-09	0	\\
2.73e-09	0	\\
2.83e-09	0	\\
2.93e-09	0	\\
3.04e-09	0	\\
3.14e-09	0	\\
3.24e-09	0	\\
3.34e-09	0	\\
3.45e-09	0	\\
3.55e-09	0	\\
3.65e-09	0	\\
3.75e-09	0	\\
3.86e-09	0	\\
3.96e-09	0	\\
4.06e-09	0	\\
4.16e-09	0	\\
4.27e-09	0	\\
4.37e-09	0	\\
4.47e-09	0	\\
4.57e-09	0	\\
4.68e-09	0	\\
4.78e-09	0	\\
4.89e-09	0	\\
4.99e-09	0	\\
5e-09	0	\\
};
\addplot [color=mycolor3,solid,forget plot]
  table[row sep=crcr]{
0	0	\\
1.1e-10	0	\\
2.2e-10	0	\\
3.3e-10	0	\\
4.4e-10	0	\\
5.4e-10	0	\\
6.5e-10	0	\\
7.5e-10	0	\\
8.6e-10	0	\\
9.6e-10	0	\\
1.07e-09	0	\\
1.18e-09	0	\\
1.28e-09	0	\\
1.38e-09	0	\\
1.49e-09	0	\\
1.59e-09	0	\\
1.69e-09	0	\\
1.8e-09	0	\\
1.9e-09	0	\\
2.01e-09	0	\\
2.11e-09	0	\\
2.21e-09	0	\\
2.32e-09	0	\\
2.42e-09	0	\\
2.52e-09	0	\\
2.63e-09	0	\\
2.73e-09	0	\\
2.83e-09	0	\\
2.93e-09	0	\\
3.04e-09	0	\\
3.14e-09	0	\\
3.24e-09	0	\\
3.34e-09	0	\\
3.45e-09	0	\\
3.55e-09	0	\\
3.65e-09	0	\\
3.75e-09	0	\\
3.86e-09	0	\\
3.96e-09	0	\\
4.06e-09	0	\\
4.16e-09	0	\\
4.27e-09	0	\\
4.37e-09	0	\\
4.47e-09	0	\\
4.57e-09	0	\\
4.68e-09	0	\\
4.78e-09	0	\\
4.89e-09	0	\\
4.99e-09	0	\\
5e-09	0	\\
};
\addplot [color=darkgray,solid,forget plot]
  table[row sep=crcr]{
0	0	\\
1.1e-10	0	\\
2.2e-10	0	\\
3.3e-10	0	\\
4.4e-10	0	\\
5.4e-10	0	\\
6.5e-10	0	\\
7.5e-10	0	\\
8.6e-10	0	\\
9.6e-10	0	\\
1.07e-09	0	\\
1.18e-09	0	\\
1.28e-09	0	\\
1.38e-09	0	\\
1.49e-09	0	\\
1.59e-09	0	\\
1.69e-09	0	\\
1.8e-09	0	\\
1.9e-09	0	\\
2.01e-09	0	\\
2.11e-09	0	\\
2.21e-09	0	\\
2.32e-09	0	\\
2.42e-09	0	\\
2.52e-09	0	\\
2.63e-09	0	\\
2.73e-09	0	\\
2.83e-09	0	\\
2.93e-09	0	\\
3.04e-09	0	\\
3.14e-09	0	\\
3.24e-09	0	\\
3.34e-09	0	\\
3.45e-09	0	\\
3.55e-09	0	\\
3.65e-09	0	\\
3.75e-09	0	\\
3.86e-09	0	\\
3.96e-09	0	\\
4.06e-09	0	\\
4.16e-09	0	\\
4.27e-09	0	\\
4.37e-09	0	\\
4.47e-09	0	\\
4.57e-09	0	\\
4.68e-09	0	\\
4.78e-09	0	\\
4.89e-09	0	\\
4.99e-09	0	\\
5e-09	0	\\
};
\addplot [color=blue,solid,forget plot]
  table[row sep=crcr]{
0	0	\\
1.1e-10	0	\\
2.2e-10	0	\\
3.3e-10	0	\\
4.4e-10	0	\\
5.4e-10	0	\\
6.5e-10	0	\\
7.5e-10	0	\\
8.6e-10	0	\\
9.6e-10	0	\\
1.07e-09	0	\\
1.18e-09	0	\\
1.28e-09	0	\\
1.38e-09	0	\\
1.49e-09	0	\\
1.59e-09	0	\\
1.69e-09	0	\\
1.8e-09	0	\\
1.9e-09	0	\\
2.01e-09	0	\\
2.11e-09	0	\\
2.21e-09	0	\\
2.32e-09	0	\\
2.42e-09	0	\\
2.52e-09	0	\\
2.63e-09	0	\\
2.73e-09	0	\\
2.83e-09	0	\\
2.93e-09	0	\\
3.04e-09	0	\\
3.14e-09	0	\\
3.24e-09	0	\\
3.34e-09	0	\\
3.45e-09	0	\\
3.55e-09	0	\\
3.65e-09	0	\\
3.75e-09	0	\\
3.86e-09	0	\\
3.96e-09	0	\\
4.06e-09	0	\\
4.16e-09	0	\\
4.27e-09	0	\\
4.37e-09	0	\\
4.47e-09	0	\\
4.57e-09	0	\\
4.68e-09	0	\\
4.78e-09	0	\\
4.89e-09	0	\\
4.99e-09	0	\\
5e-09	0	\\
};
\addplot [color=black!50!green,solid,forget plot]
  table[row sep=crcr]{
0	0	\\
1.1e-10	0	\\
2.2e-10	0	\\
3.3e-10	0	\\
4.4e-10	0	\\
5.4e-10	0	\\
6.5e-10	0	\\
7.5e-10	0	\\
8.6e-10	0	\\
9.6e-10	0	\\
1.07e-09	0	\\
1.18e-09	0	\\
1.28e-09	0	\\
1.38e-09	0	\\
1.49e-09	0	\\
1.59e-09	0	\\
1.69e-09	0	\\
1.8e-09	0	\\
1.9e-09	0	\\
2.01e-09	0	\\
2.11e-09	0	\\
2.21e-09	0	\\
2.32e-09	0	\\
2.42e-09	0	\\
2.52e-09	0	\\
2.63e-09	0	\\
2.73e-09	0	\\
2.83e-09	0	\\
2.93e-09	0	\\
3.04e-09	0	\\
3.14e-09	0	\\
3.24e-09	0	\\
3.34e-09	0	\\
3.45e-09	0	\\
3.55e-09	0	\\
3.65e-09	0	\\
3.75e-09	0	\\
3.86e-09	0	\\
3.96e-09	0	\\
4.06e-09	0	\\
4.16e-09	0	\\
4.27e-09	0	\\
4.37e-09	0	\\
4.47e-09	0	\\
4.57e-09	0	\\
4.68e-09	0	\\
4.78e-09	0	\\
4.89e-09	0	\\
4.99e-09	0	\\
5e-09	0	\\
};
\addplot [color=red,solid,forget plot]
  table[row sep=crcr]{
0	0	\\
1.1e-10	0	\\
2.2e-10	0	\\
3.3e-10	0	\\
4.4e-10	0	\\
5.4e-10	0	\\
6.5e-10	0	\\
7.5e-10	0	\\
8.6e-10	0	\\
9.6e-10	0	\\
1.07e-09	0	\\
1.18e-09	0	\\
1.28e-09	0	\\
1.38e-09	0	\\
1.49e-09	0	\\
1.59e-09	0	\\
1.69e-09	0	\\
1.8e-09	0	\\
1.9e-09	0	\\
2.01e-09	0	\\
2.11e-09	0	\\
2.21e-09	0	\\
2.32e-09	0	\\
2.42e-09	0	\\
2.52e-09	0	\\
2.63e-09	0	\\
2.73e-09	0	\\
2.83e-09	0	\\
2.93e-09	0	\\
3.04e-09	0	\\
3.14e-09	0	\\
3.24e-09	0	\\
3.34e-09	0	\\
3.45e-09	0	\\
3.55e-09	0	\\
3.65e-09	0	\\
3.75e-09	0	\\
3.86e-09	0	\\
3.96e-09	0	\\
4.06e-09	0	\\
4.16e-09	0	\\
4.27e-09	0	\\
4.37e-09	0	\\
4.47e-09	0	\\
4.57e-09	0	\\
4.68e-09	0	\\
4.78e-09	0	\\
4.89e-09	0	\\
4.99e-09	0	\\
5e-09	0	\\
};
\addplot [color=mycolor1,solid,forget plot]
  table[row sep=crcr]{
0	0	\\
1.1e-10	0	\\
2.2e-10	0	\\
3.3e-10	0	\\
4.4e-10	0	\\
5.4e-10	0	\\
6.5e-10	0	\\
7.5e-10	0	\\
8.6e-10	0	\\
9.6e-10	0	\\
1.07e-09	0	\\
1.18e-09	0	\\
1.28e-09	0	\\
1.38e-09	0	\\
1.49e-09	0	\\
1.59e-09	0	\\
1.69e-09	0	\\
1.8e-09	0	\\
1.9e-09	0	\\
2.01e-09	0	\\
2.11e-09	0	\\
2.21e-09	0	\\
2.32e-09	0	\\
2.42e-09	0	\\
2.52e-09	0	\\
2.63e-09	0	\\
2.73e-09	0	\\
2.83e-09	0	\\
2.93e-09	0	\\
3.04e-09	0	\\
3.14e-09	0	\\
3.24e-09	0	\\
3.34e-09	0	\\
3.45e-09	0	\\
3.55e-09	0	\\
3.65e-09	0	\\
3.75e-09	0	\\
3.86e-09	0	\\
3.96e-09	0	\\
4.06e-09	0	\\
4.16e-09	0	\\
4.27e-09	0	\\
4.37e-09	0	\\
4.47e-09	0	\\
4.57e-09	0	\\
4.68e-09	0	\\
4.78e-09	0	\\
4.89e-09	0	\\
4.99e-09	0	\\
5e-09	0	\\
};
\addplot [color=mycolor2,solid,forget plot]
  table[row sep=crcr]{
0	0	\\
1.1e-10	0	\\
2.2e-10	0	\\
3.3e-10	0	\\
4.4e-10	0	\\
5.4e-10	0	\\
6.5e-10	0	\\
7.5e-10	0	\\
8.6e-10	0	\\
9.6e-10	0	\\
1.07e-09	0	\\
1.18e-09	0	\\
1.28e-09	0	\\
1.38e-09	0	\\
1.49e-09	0	\\
1.59e-09	0	\\
1.69e-09	0	\\
1.8e-09	0	\\
1.9e-09	0	\\
2.01e-09	0	\\
2.11e-09	0	\\
2.21e-09	0	\\
2.32e-09	0	\\
2.42e-09	0	\\
2.52e-09	0	\\
2.63e-09	0	\\
2.73e-09	0	\\
2.83e-09	0	\\
2.93e-09	0	\\
3.04e-09	0	\\
3.14e-09	0	\\
3.24e-09	0	\\
3.34e-09	0	\\
3.45e-09	0	\\
3.55e-09	0	\\
3.65e-09	0	\\
3.75e-09	0	\\
3.86e-09	0	\\
3.96e-09	0	\\
4.06e-09	0	\\
4.16e-09	0	\\
4.27e-09	0	\\
4.37e-09	0	\\
4.47e-09	0	\\
4.57e-09	0	\\
4.68e-09	0	\\
4.78e-09	0	\\
4.89e-09	0	\\
4.99e-09	0	\\
5e-09	0	\\
};
\addplot [color=mycolor3,solid,forget plot]
  table[row sep=crcr]{
0	0	\\
1.1e-10	0	\\
2.2e-10	0	\\
3.3e-10	0	\\
4.4e-10	0	\\
5.4e-10	0	\\
6.5e-10	0	\\
7.5e-10	0	\\
8.6e-10	0	\\
9.6e-10	0	\\
1.07e-09	0	\\
1.18e-09	0	\\
1.28e-09	0	\\
1.38e-09	0	\\
1.49e-09	0	\\
1.59e-09	0	\\
1.69e-09	0	\\
1.8e-09	0	\\
1.9e-09	0	\\
2.01e-09	0	\\
2.11e-09	0	\\
2.21e-09	0	\\
2.32e-09	0	\\
2.42e-09	0	\\
2.52e-09	0	\\
2.63e-09	0	\\
2.73e-09	0	\\
2.83e-09	0	\\
2.93e-09	0	\\
3.04e-09	0	\\
3.14e-09	0	\\
3.24e-09	0	\\
3.34e-09	0	\\
3.45e-09	0	\\
3.55e-09	0	\\
3.65e-09	0	\\
3.75e-09	0	\\
3.86e-09	0	\\
3.96e-09	0	\\
4.06e-09	0	\\
4.16e-09	0	\\
4.27e-09	0	\\
4.37e-09	0	\\
4.47e-09	0	\\
4.57e-09	0	\\
4.68e-09	0	\\
4.78e-09	0	\\
4.89e-09	0	\\
4.99e-09	0	\\
5e-09	0	\\
};
\addplot [color=darkgray,solid,forget plot]
  table[row sep=crcr]{
0	0	\\
1.1e-10	0	\\
2.2e-10	0	\\
3.3e-10	0	\\
4.4e-10	0	\\
5.4e-10	0	\\
6.5e-10	0	\\
7.5e-10	0	\\
8.6e-10	0	\\
9.6e-10	0	\\
1.07e-09	0	\\
1.18e-09	0	\\
1.28e-09	0	\\
1.38e-09	0	\\
1.49e-09	0	\\
1.59e-09	0	\\
1.69e-09	0	\\
1.8e-09	0	\\
1.9e-09	0	\\
2.01e-09	0	\\
2.11e-09	0	\\
2.21e-09	0	\\
2.32e-09	0	\\
2.42e-09	0	\\
2.52e-09	0	\\
2.63e-09	0	\\
2.73e-09	0	\\
2.83e-09	0	\\
2.93e-09	0	\\
3.04e-09	0	\\
3.14e-09	0	\\
3.24e-09	0	\\
3.34e-09	0	\\
3.45e-09	0	\\
3.55e-09	0	\\
3.65e-09	0	\\
3.75e-09	0	\\
3.86e-09	0	\\
3.96e-09	0	\\
4.06e-09	0	\\
4.16e-09	0	\\
4.27e-09	0	\\
4.37e-09	0	\\
4.47e-09	0	\\
4.57e-09	0	\\
4.68e-09	0	\\
4.78e-09	0	\\
4.89e-09	0	\\
4.99e-09	0	\\
5e-09	0	\\
};
\addplot [color=blue,solid,forget plot]
  table[row sep=crcr]{
0	0	\\
1.1e-10	0	\\
2.2e-10	0	\\
3.3e-10	0	\\
4.4e-10	0	\\
5.4e-10	0	\\
6.5e-10	0	\\
7.5e-10	0	\\
8.6e-10	0	\\
9.6e-10	0	\\
1.07e-09	0	\\
1.18e-09	0	\\
1.28e-09	0	\\
1.38e-09	0	\\
1.49e-09	0	\\
1.59e-09	0	\\
1.69e-09	0	\\
1.8e-09	0	\\
1.9e-09	0	\\
2.01e-09	0	\\
2.11e-09	0	\\
2.21e-09	0	\\
2.32e-09	0	\\
2.42e-09	0	\\
2.52e-09	0	\\
2.63e-09	0	\\
2.73e-09	0	\\
2.83e-09	0	\\
2.93e-09	0	\\
3.04e-09	0	\\
3.14e-09	0	\\
3.24e-09	0	\\
3.34e-09	0	\\
3.45e-09	0	\\
3.55e-09	0	\\
3.65e-09	0	\\
3.75e-09	0	\\
3.86e-09	0	\\
3.96e-09	0	\\
4.06e-09	0	\\
4.16e-09	0	\\
4.27e-09	0	\\
4.37e-09	0	\\
4.47e-09	0	\\
4.57e-09	0	\\
4.68e-09	0	\\
4.78e-09	0	\\
4.89e-09	0	\\
4.99e-09	0	\\
5e-09	0	\\
};
\addplot [color=black!50!green,solid,forget plot]
  table[row sep=crcr]{
0	0	\\
1.1e-10	0	\\
2.2e-10	0	\\
3.3e-10	0	\\
4.4e-10	0	\\
5.4e-10	0	\\
6.5e-10	0	\\
7.5e-10	0	\\
8.6e-10	0	\\
9.6e-10	0	\\
1.07e-09	0	\\
1.18e-09	0	\\
1.28e-09	0	\\
1.38e-09	0	\\
1.49e-09	0	\\
1.59e-09	0	\\
1.69e-09	0	\\
1.8e-09	0	\\
1.9e-09	0	\\
2.01e-09	0	\\
2.11e-09	0	\\
2.21e-09	0	\\
2.32e-09	0	\\
2.42e-09	0	\\
2.52e-09	0	\\
2.63e-09	0	\\
2.73e-09	0	\\
2.83e-09	0	\\
2.93e-09	0	\\
3.04e-09	0	\\
3.14e-09	0	\\
3.24e-09	0	\\
3.34e-09	0	\\
3.45e-09	0	\\
3.55e-09	0	\\
3.65e-09	0	\\
3.75e-09	0	\\
3.86e-09	0	\\
3.96e-09	0	\\
4.06e-09	0	\\
4.16e-09	0	\\
4.27e-09	0	\\
4.37e-09	0	\\
4.47e-09	0	\\
4.57e-09	0	\\
4.68e-09	0	\\
4.78e-09	0	\\
4.89e-09	0	\\
4.99e-09	0	\\
5e-09	0	\\
};
\addplot [color=red,solid,forget plot]
  table[row sep=crcr]{
0	0	\\
1.1e-10	0	\\
2.2e-10	0	\\
3.3e-10	0	\\
4.4e-10	0	\\
5.4e-10	0	\\
6.5e-10	0	\\
7.5e-10	0	\\
8.6e-10	0	\\
9.6e-10	0	\\
1.07e-09	0	\\
1.18e-09	0	\\
1.28e-09	0	\\
1.38e-09	0	\\
1.49e-09	0	\\
1.59e-09	0	\\
1.69e-09	0	\\
1.8e-09	0	\\
1.9e-09	0	\\
2.01e-09	0	\\
2.11e-09	0	\\
2.21e-09	0	\\
2.32e-09	0	\\
2.42e-09	0	\\
2.52e-09	0	\\
2.63e-09	0	\\
2.73e-09	0	\\
2.83e-09	0	\\
2.93e-09	0	\\
3.04e-09	0	\\
3.14e-09	0	\\
3.24e-09	0	\\
3.34e-09	0	\\
3.45e-09	0	\\
3.55e-09	0	\\
3.65e-09	0	\\
3.75e-09	0	\\
3.86e-09	0	\\
3.96e-09	0	\\
4.06e-09	0	\\
4.16e-09	0	\\
4.27e-09	0	\\
4.37e-09	0	\\
4.47e-09	0	\\
4.57e-09	0	\\
4.68e-09	0	\\
4.78e-09	0	\\
4.89e-09	0	\\
4.99e-09	0	\\
5e-09	0	\\
};
\addplot [color=mycolor1,solid,forget plot]
  table[row sep=crcr]{
0	0	\\
1.1e-10	0	\\
2.2e-10	0	\\
3.3e-10	0	\\
4.4e-10	0	\\
5.4e-10	0	\\
6.5e-10	0	\\
7.5e-10	0	\\
8.6e-10	0	\\
9.6e-10	0	\\
1.07e-09	0	\\
1.18e-09	0	\\
1.28e-09	0	\\
1.38e-09	0	\\
1.49e-09	0	\\
1.59e-09	0	\\
1.69e-09	0	\\
1.8e-09	0	\\
1.9e-09	0	\\
2.01e-09	0	\\
2.11e-09	0	\\
2.21e-09	0	\\
2.32e-09	0	\\
2.42e-09	0	\\
2.52e-09	0	\\
2.63e-09	0	\\
2.73e-09	0	\\
2.83e-09	0	\\
2.93e-09	0	\\
3.04e-09	0	\\
3.14e-09	0	\\
3.24e-09	0	\\
3.34e-09	0	\\
3.45e-09	0	\\
3.55e-09	0	\\
3.65e-09	0	\\
3.75e-09	0	\\
3.86e-09	0	\\
3.96e-09	0	\\
4.06e-09	0	\\
4.16e-09	0	\\
4.27e-09	0	\\
4.37e-09	0	\\
4.47e-09	0	\\
4.57e-09	0	\\
4.68e-09	0	\\
4.78e-09	0	\\
4.89e-09	0	\\
4.99e-09	0	\\
5e-09	0	\\
};
\addplot [color=mycolor2,solid,forget plot]
  table[row sep=crcr]{
0	0	\\
1.1e-10	0	\\
2.2e-10	0	\\
3.3e-10	0	\\
4.4e-10	0	\\
5.4e-10	0	\\
6.5e-10	0	\\
7.5e-10	0	\\
8.6e-10	0	\\
9.6e-10	0	\\
1.07e-09	0	\\
1.18e-09	0	\\
1.28e-09	0	\\
1.38e-09	0	\\
1.49e-09	0	\\
1.59e-09	0	\\
1.69e-09	0	\\
1.8e-09	0	\\
1.9e-09	0	\\
2.01e-09	0	\\
2.11e-09	0	\\
2.21e-09	0	\\
2.32e-09	0	\\
2.42e-09	0	\\
2.52e-09	0	\\
2.63e-09	0	\\
2.73e-09	0	\\
2.83e-09	0	\\
2.93e-09	0	\\
3.04e-09	0	\\
3.14e-09	0	\\
3.24e-09	0	\\
3.34e-09	0	\\
3.45e-09	0	\\
3.55e-09	0	\\
3.65e-09	0	\\
3.75e-09	0	\\
3.86e-09	0	\\
3.96e-09	0	\\
4.06e-09	0	\\
4.16e-09	0	\\
4.27e-09	0	\\
4.37e-09	0	\\
4.47e-09	0	\\
4.57e-09	0	\\
4.68e-09	0	\\
4.78e-09	0	\\
4.89e-09	0	\\
4.99e-09	0	\\
5e-09	0	\\
};
\addplot [color=mycolor3,solid,forget plot]
  table[row sep=crcr]{
0	0	\\
1.1e-10	0	\\
2.2e-10	0	\\
3.3e-10	0	\\
4.4e-10	0	\\
5.4e-10	0	\\
6.5e-10	0	\\
7.5e-10	0	\\
8.6e-10	0	\\
9.6e-10	0	\\
1.07e-09	0	\\
1.18e-09	0	\\
1.28e-09	0	\\
1.38e-09	0	\\
1.49e-09	0	\\
1.59e-09	0	\\
1.69e-09	0	\\
1.8e-09	0	\\
1.9e-09	0	\\
2.01e-09	0	\\
2.11e-09	0	\\
2.21e-09	0	\\
2.32e-09	0	\\
2.42e-09	0	\\
2.52e-09	0	\\
2.63e-09	0	\\
2.73e-09	0	\\
2.83e-09	0	\\
2.93e-09	0	\\
3.04e-09	0	\\
3.14e-09	0	\\
3.24e-09	0	\\
3.34e-09	0	\\
3.45e-09	0	\\
3.55e-09	0	\\
3.65e-09	0	\\
3.75e-09	0	\\
3.86e-09	0	\\
3.96e-09	0	\\
4.06e-09	0	\\
4.16e-09	0	\\
4.27e-09	0	\\
4.37e-09	0	\\
4.47e-09	0	\\
4.57e-09	0	\\
4.68e-09	0	\\
4.78e-09	0	\\
4.89e-09	0	\\
4.99e-09	0	\\
5e-09	0	\\
};
\addplot [color=darkgray,solid,forget plot]
  table[row sep=crcr]{
0	0	\\
1.1e-10	0	\\
2.2e-10	0	\\
3.3e-10	0	\\
4.4e-10	0	\\
5.4e-10	0	\\
6.5e-10	0	\\
7.5e-10	0	\\
8.6e-10	0	\\
9.6e-10	0	\\
1.07e-09	0	\\
1.18e-09	0	\\
1.28e-09	0	\\
1.38e-09	0	\\
1.49e-09	0	\\
1.59e-09	0	\\
1.69e-09	0	\\
1.8e-09	0	\\
1.9e-09	0	\\
2.01e-09	0	\\
2.11e-09	0	\\
2.21e-09	0	\\
2.32e-09	0	\\
2.42e-09	0	\\
2.52e-09	0	\\
2.63e-09	0	\\
2.73e-09	0	\\
2.83e-09	0	\\
2.93e-09	0	\\
3.04e-09	0	\\
3.14e-09	0	\\
3.24e-09	0	\\
3.34e-09	0	\\
3.45e-09	0	\\
3.55e-09	0	\\
3.65e-09	0	\\
3.75e-09	0	\\
3.86e-09	0	\\
3.96e-09	0	\\
4.06e-09	0	\\
4.16e-09	0	\\
4.27e-09	0	\\
4.37e-09	0	\\
4.47e-09	0	\\
4.57e-09	0	\\
4.68e-09	0	\\
4.78e-09	0	\\
4.89e-09	0	\\
4.99e-09	0	\\
5e-09	0	\\
};
\addplot [color=blue,solid,forget plot]
  table[row sep=crcr]{
0	0	\\
1.1e-10	0	\\
2.2e-10	0	\\
3.3e-10	0	\\
4.4e-10	0	\\
5.4e-10	0	\\
6.5e-10	0	\\
7.5e-10	0	\\
8.6e-10	0	\\
9.6e-10	0	\\
1.07e-09	0	\\
1.18e-09	0	\\
1.28e-09	0	\\
1.38e-09	0	\\
1.49e-09	0	\\
1.59e-09	0	\\
1.69e-09	0	\\
1.8e-09	0	\\
1.9e-09	0	\\
2.01e-09	0	\\
2.11e-09	0	\\
2.21e-09	0	\\
2.32e-09	0	\\
2.42e-09	0	\\
2.52e-09	0	\\
2.63e-09	0	\\
2.73e-09	0	\\
2.83e-09	0	\\
2.93e-09	0	\\
3.04e-09	0	\\
3.14e-09	0	\\
3.24e-09	0	\\
3.34e-09	0	\\
3.45e-09	0	\\
3.55e-09	0	\\
3.65e-09	0	\\
3.75e-09	0	\\
3.86e-09	0	\\
3.96e-09	0	\\
4.06e-09	0	\\
4.16e-09	0	\\
4.27e-09	0	\\
4.37e-09	0	\\
4.47e-09	0	\\
4.57e-09	0	\\
4.68e-09	0	\\
4.78e-09	0	\\
4.89e-09	0	\\
4.99e-09	0	\\
5e-09	0	\\
};
\addplot [color=black!50!green,solid,forget plot]
  table[row sep=crcr]{
0	0	\\
1.1e-10	0	\\
2.2e-10	0	\\
3.3e-10	0	\\
4.4e-10	0	\\
5.4e-10	0	\\
6.5e-10	0	\\
7.5e-10	0	\\
8.6e-10	0	\\
9.6e-10	0	\\
1.07e-09	0	\\
1.18e-09	0	\\
1.28e-09	0	\\
1.38e-09	0	\\
1.49e-09	0	\\
1.59e-09	0	\\
1.69e-09	0	\\
1.8e-09	0	\\
1.9e-09	0	\\
2.01e-09	0	\\
2.11e-09	0	\\
2.21e-09	0	\\
2.32e-09	0	\\
2.42e-09	0	\\
2.52e-09	0	\\
2.63e-09	0	\\
2.73e-09	0	\\
2.83e-09	0	\\
2.93e-09	0	\\
3.04e-09	0	\\
3.14e-09	0	\\
3.24e-09	0	\\
3.34e-09	0	\\
3.45e-09	0	\\
3.55e-09	0	\\
3.65e-09	0	\\
3.75e-09	0	\\
3.86e-09	0	\\
3.96e-09	0	\\
4.06e-09	0	\\
4.16e-09	0	\\
4.27e-09	0	\\
4.37e-09	0	\\
4.47e-09	0	\\
4.57e-09	0	\\
4.68e-09	0	\\
4.78e-09	0	\\
4.89e-09	0	\\
4.99e-09	0	\\
5e-09	0	\\
};
\addplot [color=red,solid,forget plot]
  table[row sep=crcr]{
0	0	\\
1.1e-10	0	\\
2.2e-10	0	\\
3.3e-10	0	\\
4.4e-10	0	\\
5.4e-10	0	\\
6.5e-10	0	\\
7.5e-10	0	\\
8.6e-10	0	\\
9.6e-10	0	\\
1.07e-09	0	\\
1.18e-09	0	\\
1.28e-09	0	\\
1.38e-09	0	\\
1.49e-09	0	\\
1.59e-09	0	\\
1.69e-09	0	\\
1.8e-09	0	\\
1.9e-09	0	\\
2.01e-09	0	\\
2.11e-09	0	\\
2.21e-09	0	\\
2.32e-09	0	\\
2.42e-09	0	\\
2.52e-09	0	\\
2.63e-09	0	\\
2.73e-09	0	\\
2.83e-09	0	\\
2.93e-09	0	\\
3.04e-09	0	\\
3.14e-09	0	\\
3.24e-09	0	\\
3.34e-09	0	\\
3.45e-09	0	\\
3.55e-09	0	\\
3.65e-09	0	\\
3.75e-09	0	\\
3.86e-09	0	\\
3.96e-09	0	\\
4.06e-09	0	\\
4.16e-09	0	\\
4.27e-09	0	\\
4.37e-09	0	\\
4.47e-09	0	\\
4.57e-09	0	\\
4.68e-09	0	\\
4.78e-09	0	\\
4.89e-09	0	\\
4.99e-09	0	\\
5e-09	0	\\
};
\addplot [color=mycolor1,solid,forget plot]
  table[row sep=crcr]{
0	0	\\
1.1e-10	0	\\
2.2e-10	0	\\
3.3e-10	0	\\
4.4e-10	0	\\
5.4e-10	0	\\
6.5e-10	0	\\
7.5e-10	0	\\
8.6e-10	0	\\
9.6e-10	0	\\
1.07e-09	0	\\
1.18e-09	0	\\
1.28e-09	0	\\
1.38e-09	0	\\
1.49e-09	0	\\
1.59e-09	0	\\
1.69e-09	0	\\
1.8e-09	0	\\
1.9e-09	0	\\
2.01e-09	0	\\
2.11e-09	0	\\
2.21e-09	0	\\
2.32e-09	0	\\
2.42e-09	0	\\
2.52e-09	0	\\
2.63e-09	0	\\
2.73e-09	0	\\
2.83e-09	0	\\
2.93e-09	0	\\
3.04e-09	0	\\
3.14e-09	0	\\
3.24e-09	0	\\
3.34e-09	0	\\
3.45e-09	0	\\
3.55e-09	0	\\
3.65e-09	0	\\
3.75e-09	0	\\
3.86e-09	0	\\
3.96e-09	0	\\
4.06e-09	0	\\
4.16e-09	0	\\
4.27e-09	0	\\
4.37e-09	0	\\
4.47e-09	0	\\
4.57e-09	0	\\
4.68e-09	0	\\
4.78e-09	0	\\
4.89e-09	0	\\
4.99e-09	0	\\
5e-09	0	\\
};
\addplot [color=mycolor2,solid,forget plot]
  table[row sep=crcr]{
0	0	\\
1.1e-10	0	\\
2.2e-10	0	\\
3.3e-10	0	\\
4.4e-10	0	\\
5.4e-10	0	\\
6.5e-10	0	\\
7.5e-10	0	\\
8.6e-10	0	\\
9.6e-10	0	\\
1.07e-09	0	\\
1.18e-09	0	\\
1.28e-09	0	\\
1.38e-09	0	\\
1.49e-09	0	\\
1.59e-09	0	\\
1.69e-09	0	\\
1.8e-09	0	\\
1.9e-09	0	\\
2.01e-09	0	\\
2.11e-09	0	\\
2.21e-09	0	\\
2.32e-09	0	\\
2.42e-09	0	\\
2.52e-09	0	\\
2.63e-09	0	\\
2.73e-09	0	\\
2.83e-09	0	\\
2.93e-09	0	\\
3.04e-09	0	\\
3.14e-09	0	\\
3.24e-09	0	\\
3.34e-09	0	\\
3.45e-09	0	\\
3.55e-09	0	\\
3.65e-09	0	\\
3.75e-09	0	\\
3.86e-09	0	\\
3.96e-09	0	\\
4.06e-09	0	\\
4.16e-09	0	\\
4.27e-09	0	\\
4.37e-09	0	\\
4.47e-09	0	\\
4.57e-09	0	\\
4.68e-09	0	\\
4.78e-09	0	\\
4.89e-09	0	\\
4.99e-09	0	\\
5e-09	0	\\
};
\addplot [color=mycolor3,solid,forget plot]
  table[row sep=crcr]{
0	0	\\
1.1e-10	0	\\
2.2e-10	0	\\
3.3e-10	0	\\
4.4e-10	0	\\
5.4e-10	0	\\
6.5e-10	0	\\
7.5e-10	0	\\
8.6e-10	0	\\
9.6e-10	0	\\
1.07e-09	0	\\
1.18e-09	0	\\
1.28e-09	0	\\
1.38e-09	0	\\
1.49e-09	0	\\
1.59e-09	0	\\
1.69e-09	0	\\
1.8e-09	0	\\
1.9e-09	0	\\
2.01e-09	0	\\
2.11e-09	0	\\
2.21e-09	0	\\
2.32e-09	0	\\
2.42e-09	0	\\
2.52e-09	0	\\
2.63e-09	0	\\
2.73e-09	0	\\
2.83e-09	0	\\
2.93e-09	0	\\
3.04e-09	0	\\
3.14e-09	0	\\
3.24e-09	0	\\
3.34e-09	0	\\
3.45e-09	0	\\
3.55e-09	0	\\
3.65e-09	0	\\
3.75e-09	0	\\
3.86e-09	0	\\
3.96e-09	0	\\
4.06e-09	0	\\
4.16e-09	0	\\
4.27e-09	0	\\
4.37e-09	0	\\
4.47e-09	0	\\
4.57e-09	0	\\
4.68e-09	0	\\
4.78e-09	0	\\
4.89e-09	0	\\
4.99e-09	0	\\
5e-09	0	\\
};
\addplot [color=darkgray,solid,forget plot]
  table[row sep=crcr]{
0	0	\\
1.1e-10	0	\\
2.2e-10	0	\\
3.3e-10	0	\\
4.4e-10	0	\\
5.4e-10	0	\\
6.5e-10	0	\\
7.5e-10	0	\\
8.6e-10	0	\\
9.6e-10	0	\\
1.07e-09	0	\\
1.18e-09	0	\\
1.28e-09	0	\\
1.38e-09	0	\\
1.49e-09	0	\\
1.59e-09	0	\\
1.69e-09	0	\\
1.8e-09	0	\\
1.9e-09	0	\\
2.01e-09	0	\\
2.11e-09	0	\\
2.21e-09	0	\\
2.32e-09	0	\\
2.42e-09	0	\\
2.52e-09	0	\\
2.63e-09	0	\\
2.73e-09	0	\\
2.83e-09	0	\\
2.93e-09	0	\\
3.04e-09	0	\\
3.14e-09	0	\\
3.24e-09	0	\\
3.34e-09	0	\\
3.45e-09	0	\\
3.55e-09	0	\\
3.65e-09	0	\\
3.75e-09	0	\\
3.86e-09	0	\\
3.96e-09	0	\\
4.06e-09	0	\\
4.16e-09	0	\\
4.27e-09	0	\\
4.37e-09	0	\\
4.47e-09	0	\\
4.57e-09	0	\\
4.68e-09	0	\\
4.78e-09	0	\\
4.89e-09	0	\\
4.99e-09	0	\\
5e-09	0	\\
};
\addplot [color=blue,solid,forget plot]
  table[row sep=crcr]{
0	0	\\
1.1e-10	0	\\
2.2e-10	0	\\
3.3e-10	0	\\
4.4e-10	0	\\
5.4e-10	0	\\
6.5e-10	0	\\
7.5e-10	0	\\
8.6e-10	0	\\
9.6e-10	0	\\
1.07e-09	0	\\
1.18e-09	0	\\
1.28e-09	0	\\
1.38e-09	0	\\
1.49e-09	0	\\
1.59e-09	0	\\
1.69e-09	0	\\
1.8e-09	0	\\
1.9e-09	0	\\
2.01e-09	0	\\
2.11e-09	0	\\
2.21e-09	0	\\
2.32e-09	0	\\
2.42e-09	0	\\
2.52e-09	0	\\
2.63e-09	0	\\
2.73e-09	0	\\
2.83e-09	0	\\
2.93e-09	0	\\
3.04e-09	0	\\
3.14e-09	0	\\
3.24e-09	0	\\
3.34e-09	0	\\
3.45e-09	0	\\
3.55e-09	0	\\
3.65e-09	0	\\
3.75e-09	0	\\
3.86e-09	0	\\
3.96e-09	0	\\
4.06e-09	0	\\
4.16e-09	0	\\
4.27e-09	0	\\
4.37e-09	0	\\
4.47e-09	0	\\
4.57e-09	0	\\
4.68e-09	0	\\
4.78e-09	0	\\
4.89e-09	0	\\
4.99e-09	0	\\
5e-09	0	\\
};
\addplot [color=black!50!green,solid,forget plot]
  table[row sep=crcr]{
0	0	\\
1.1e-10	0	\\
2.2e-10	0	\\
3.3e-10	0	\\
4.4e-10	0	\\
5.4e-10	0	\\
6.5e-10	0	\\
7.5e-10	0	\\
8.6e-10	0	\\
9.6e-10	0	\\
1.07e-09	0	\\
1.18e-09	0	\\
1.28e-09	0	\\
1.38e-09	0	\\
1.49e-09	0	\\
1.59e-09	0	\\
1.69e-09	0	\\
1.8e-09	0	\\
1.9e-09	0	\\
2.01e-09	0	\\
2.11e-09	0	\\
2.21e-09	0	\\
2.32e-09	0	\\
2.42e-09	0	\\
2.52e-09	0	\\
2.63e-09	0	\\
2.73e-09	0	\\
2.83e-09	0	\\
2.93e-09	0	\\
3.04e-09	0	\\
3.14e-09	0	\\
3.24e-09	0	\\
3.34e-09	0	\\
3.45e-09	0	\\
3.55e-09	0	\\
3.65e-09	0	\\
3.75e-09	0	\\
3.86e-09	0	\\
3.96e-09	0	\\
4.06e-09	0	\\
4.16e-09	0	\\
4.27e-09	0	\\
4.37e-09	0	\\
4.47e-09	0	\\
4.57e-09	0	\\
4.68e-09	0	\\
4.78e-09	0	\\
4.89e-09	0	\\
4.99e-09	0	\\
5e-09	0	\\
};
\addplot [color=red,solid,forget plot]
  table[row sep=crcr]{
0	0	\\
1.1e-10	0	\\
2.2e-10	0	\\
3.3e-10	0	\\
4.4e-10	0	\\
5.4e-10	0	\\
6.5e-10	0	\\
7.5e-10	0	\\
8.6e-10	0	\\
9.6e-10	0	\\
1.07e-09	0	\\
1.18e-09	0	\\
1.28e-09	0	\\
1.38e-09	0	\\
1.49e-09	0	\\
1.59e-09	0	\\
1.69e-09	0	\\
1.8e-09	0	\\
1.9e-09	0	\\
2.01e-09	0	\\
2.11e-09	0	\\
2.21e-09	0	\\
2.32e-09	0	\\
2.42e-09	0	\\
2.52e-09	0	\\
2.63e-09	0	\\
2.73e-09	0	\\
2.83e-09	0	\\
2.93e-09	0	\\
3.04e-09	0	\\
3.14e-09	0	\\
3.24e-09	0	\\
3.34e-09	0	\\
3.45e-09	0	\\
3.55e-09	0	\\
3.65e-09	0	\\
3.75e-09	0	\\
3.86e-09	0	\\
3.96e-09	0	\\
4.06e-09	0	\\
4.16e-09	0	\\
4.27e-09	0	\\
4.37e-09	0	\\
4.47e-09	0	\\
4.57e-09	0	\\
4.68e-09	0	\\
4.78e-09	0	\\
4.89e-09	0	\\
4.99e-09	0	\\
5e-09	0	\\
};
\addplot [color=mycolor1,solid,forget plot]
  table[row sep=crcr]{
0	0	\\
1.1e-10	0	\\
2.2e-10	0	\\
3.3e-10	0	\\
4.4e-10	0	\\
5.4e-10	0	\\
6.5e-10	0	\\
7.5e-10	0	\\
8.6e-10	0	\\
9.6e-10	0	\\
1.07e-09	0	\\
1.18e-09	0	\\
1.28e-09	0	\\
1.38e-09	0	\\
1.49e-09	0	\\
1.59e-09	0	\\
1.69e-09	0	\\
1.8e-09	0	\\
1.9e-09	0	\\
2.01e-09	0	\\
2.11e-09	0	\\
2.21e-09	0	\\
2.32e-09	0	\\
2.42e-09	0	\\
2.52e-09	0	\\
2.63e-09	0	\\
2.73e-09	0	\\
2.83e-09	0	\\
2.93e-09	0	\\
3.04e-09	0	\\
3.14e-09	0	\\
3.24e-09	0	\\
3.34e-09	0	\\
3.45e-09	0	\\
3.55e-09	0	\\
3.65e-09	0	\\
3.75e-09	0	\\
3.86e-09	0	\\
3.96e-09	0	\\
4.06e-09	0	\\
4.16e-09	0	\\
4.27e-09	0	\\
4.37e-09	0	\\
4.47e-09	0	\\
4.57e-09	0	\\
4.68e-09	0	\\
4.78e-09	0	\\
4.89e-09	0	\\
4.99e-09	0	\\
5e-09	0	\\
};
\addplot [color=mycolor2,solid,forget plot]
  table[row sep=crcr]{
0	0	\\
1.1e-10	0	\\
2.2e-10	0	\\
3.3e-10	0	\\
4.4e-10	0	\\
5.4e-10	0	\\
6.5e-10	0	\\
7.5e-10	0	\\
8.6e-10	0	\\
9.6e-10	0	\\
1.07e-09	0	\\
1.18e-09	0	\\
1.28e-09	0	\\
1.38e-09	0	\\
1.49e-09	0	\\
1.59e-09	0	\\
1.69e-09	0	\\
1.8e-09	0	\\
1.9e-09	0	\\
2.01e-09	0	\\
2.11e-09	0	\\
2.21e-09	0	\\
2.32e-09	0	\\
2.42e-09	0	\\
2.52e-09	0	\\
2.63e-09	0	\\
2.73e-09	0	\\
2.83e-09	0	\\
2.93e-09	0	\\
3.04e-09	0	\\
3.14e-09	0	\\
3.24e-09	0	\\
3.34e-09	0	\\
3.45e-09	0	\\
3.55e-09	0	\\
3.65e-09	0	\\
3.75e-09	0	\\
3.86e-09	0	\\
3.96e-09	0	\\
4.06e-09	0	\\
4.16e-09	0	\\
4.27e-09	0	\\
4.37e-09	0	\\
4.47e-09	0	\\
4.57e-09	0	\\
4.68e-09	0	\\
4.78e-09	0	\\
4.89e-09	0	\\
4.99e-09	0	\\
5e-09	0	\\
};
\addplot [color=mycolor3,solid,forget plot]
  table[row sep=crcr]{
0	0	\\
1.1e-10	0	\\
2.2e-10	0	\\
3.3e-10	0	\\
4.4e-10	0	\\
5.4e-10	0	\\
6.5e-10	0	\\
7.5e-10	0	\\
8.6e-10	0	\\
9.6e-10	0	\\
1.07e-09	0	\\
1.18e-09	0	\\
1.28e-09	0	\\
1.38e-09	0	\\
1.49e-09	0	\\
1.59e-09	0	\\
1.69e-09	0	\\
1.8e-09	0	\\
1.9e-09	0	\\
2.01e-09	0	\\
2.11e-09	0	\\
2.21e-09	0	\\
2.32e-09	0	\\
2.42e-09	0	\\
2.52e-09	0	\\
2.63e-09	0	\\
2.73e-09	0	\\
2.83e-09	0	\\
2.93e-09	0	\\
3.04e-09	0	\\
3.14e-09	0	\\
3.24e-09	0	\\
3.34e-09	0	\\
3.45e-09	0	\\
3.55e-09	0	\\
3.65e-09	0	\\
3.75e-09	0	\\
3.86e-09	0	\\
3.96e-09	0	\\
4.06e-09	0	\\
4.16e-09	0	\\
4.27e-09	0	\\
4.37e-09	0	\\
4.47e-09	0	\\
4.57e-09	0	\\
4.68e-09	0	\\
4.78e-09	0	\\
4.89e-09	0	\\
4.99e-09	0	\\
5e-09	0	\\
};
\addplot [color=darkgray,solid,forget plot]
  table[row sep=crcr]{
0	0	\\
1.1e-10	0	\\
2.2e-10	0	\\
3.3e-10	0	\\
4.4e-10	0	\\
5.4e-10	0	\\
6.5e-10	0	\\
7.5e-10	0	\\
8.6e-10	0	\\
9.6e-10	0	\\
1.07e-09	0	\\
1.18e-09	0	\\
1.28e-09	0	\\
1.38e-09	0	\\
1.49e-09	0	\\
1.59e-09	0	\\
1.69e-09	0	\\
1.8e-09	0	\\
1.9e-09	0	\\
2.01e-09	0	\\
2.11e-09	0	\\
2.21e-09	0	\\
2.32e-09	0	\\
2.42e-09	0	\\
2.52e-09	0	\\
2.63e-09	0	\\
2.73e-09	0	\\
2.83e-09	0	\\
2.93e-09	0	\\
3.04e-09	0	\\
3.14e-09	0	\\
3.24e-09	0	\\
3.34e-09	0	\\
3.45e-09	0	\\
3.55e-09	0	\\
3.65e-09	0	\\
3.75e-09	0	\\
3.86e-09	0	\\
3.96e-09	0	\\
4.06e-09	0	\\
4.16e-09	0	\\
4.27e-09	0	\\
4.37e-09	0	\\
4.47e-09	0	\\
4.57e-09	0	\\
4.68e-09	0	\\
4.78e-09	0	\\
4.89e-09	0	\\
4.99e-09	0	\\
5e-09	0	\\
};
\addplot [color=blue,solid,forget plot]
  table[row sep=crcr]{
0	0	\\
1.1e-10	0	\\
2.2e-10	0	\\
3.3e-10	0	\\
4.4e-10	0	\\
5.4e-10	0	\\
6.5e-10	0	\\
7.5e-10	0	\\
8.6e-10	0	\\
9.6e-10	0	\\
1.07e-09	0	\\
1.18e-09	0	\\
1.28e-09	0	\\
1.38e-09	0	\\
1.49e-09	0	\\
1.59e-09	0	\\
1.69e-09	0	\\
1.8e-09	0	\\
1.9e-09	0	\\
2.01e-09	0	\\
2.11e-09	0	\\
2.21e-09	0	\\
2.32e-09	0	\\
2.42e-09	0	\\
2.52e-09	0	\\
2.63e-09	0	\\
2.73e-09	0	\\
2.83e-09	0	\\
2.93e-09	0	\\
3.04e-09	0	\\
3.14e-09	0	\\
3.24e-09	0	\\
3.34e-09	0	\\
3.45e-09	0	\\
3.55e-09	0	\\
3.65e-09	0	\\
3.75e-09	0	\\
3.86e-09	0	\\
3.96e-09	0	\\
4.06e-09	0	\\
4.16e-09	0	\\
4.27e-09	0	\\
4.37e-09	0	\\
4.47e-09	0	\\
4.57e-09	0	\\
4.68e-09	0	\\
4.78e-09	0	\\
4.89e-09	0	\\
4.99e-09	0	\\
5e-09	0	\\
};
\addplot [color=black!50!green,solid,forget plot]
  table[row sep=crcr]{
0	0	\\
1.1e-10	0	\\
2.2e-10	0	\\
3.3e-10	0	\\
4.4e-10	0	\\
5.4e-10	0	\\
6.5e-10	0	\\
7.5e-10	0	\\
8.6e-10	0	\\
9.6e-10	0	\\
1.07e-09	0	\\
1.18e-09	0	\\
1.28e-09	0	\\
1.38e-09	0	\\
1.49e-09	0	\\
1.59e-09	0	\\
1.69e-09	0	\\
1.8e-09	0	\\
1.9e-09	0	\\
2.01e-09	0	\\
2.11e-09	0	\\
2.21e-09	0	\\
2.32e-09	0	\\
2.42e-09	0	\\
2.52e-09	0	\\
2.63e-09	0	\\
2.73e-09	0	\\
2.83e-09	0	\\
2.93e-09	0	\\
3.04e-09	0	\\
3.14e-09	0	\\
3.24e-09	0	\\
3.34e-09	0	\\
3.45e-09	0	\\
3.55e-09	0	\\
3.65e-09	0	\\
3.75e-09	0	\\
3.86e-09	0	\\
3.96e-09	0	\\
4.06e-09	0	\\
4.16e-09	0	\\
4.27e-09	0	\\
4.37e-09	0	\\
4.47e-09	0	\\
4.57e-09	0	\\
4.68e-09	0	\\
4.78e-09	0	\\
4.89e-09	0	\\
4.99e-09	0	\\
5e-09	0	\\
};
\addplot [color=red,solid,forget plot]
  table[row sep=crcr]{
0	0	\\
1.1e-10	0	\\
2.2e-10	0	\\
3.3e-10	0	\\
4.4e-10	0	\\
5.4e-10	0	\\
6.5e-10	0	\\
7.5e-10	0	\\
8.6e-10	0	\\
9.6e-10	0	\\
1.07e-09	0	\\
1.18e-09	0	\\
1.28e-09	0	\\
1.38e-09	0	\\
1.49e-09	0	\\
1.59e-09	0	\\
1.69e-09	0	\\
1.8e-09	0	\\
1.9e-09	0	\\
2.01e-09	0	\\
2.11e-09	0	\\
2.21e-09	0	\\
2.32e-09	0	\\
2.42e-09	0	\\
2.52e-09	0	\\
2.63e-09	0	\\
2.73e-09	0	\\
2.83e-09	0	\\
2.93e-09	0	\\
3.04e-09	0	\\
3.14e-09	0	\\
3.24e-09	0	\\
3.34e-09	0	\\
3.45e-09	0	\\
3.55e-09	0	\\
3.65e-09	0	\\
3.75e-09	0	\\
3.86e-09	0	\\
3.96e-09	0	\\
4.06e-09	0	\\
4.16e-09	0	\\
4.27e-09	0	\\
4.37e-09	0	\\
4.47e-09	0	\\
4.57e-09	0	\\
4.68e-09	0	\\
4.78e-09	0	\\
4.89e-09	0	\\
4.99e-09	0	\\
5e-09	0	\\
};
\addplot [color=mycolor1,solid,forget plot]
  table[row sep=crcr]{
0	0	\\
1.1e-10	0	\\
2.2e-10	0	\\
3.3e-10	0	\\
4.4e-10	0	\\
5.4e-10	0	\\
6.5e-10	0	\\
7.5e-10	0	\\
8.6e-10	0	\\
9.6e-10	0	\\
1.07e-09	0	\\
1.18e-09	0	\\
1.28e-09	0	\\
1.38e-09	0	\\
1.49e-09	0	\\
1.59e-09	0	\\
1.69e-09	0	\\
1.8e-09	0	\\
1.9e-09	0	\\
2.01e-09	0	\\
2.11e-09	0	\\
2.21e-09	0	\\
2.32e-09	0	\\
2.42e-09	0	\\
2.52e-09	0	\\
2.63e-09	0	\\
2.73e-09	0	\\
2.83e-09	0	\\
2.93e-09	0	\\
3.04e-09	0	\\
3.14e-09	0	\\
3.24e-09	0	\\
3.34e-09	0	\\
3.45e-09	0	\\
3.55e-09	0	\\
3.65e-09	0	\\
3.75e-09	0	\\
3.86e-09	0	\\
3.96e-09	0	\\
4.06e-09	0	\\
4.16e-09	0	\\
4.27e-09	0	\\
4.37e-09	0	\\
4.47e-09	0	\\
4.57e-09	0	\\
4.68e-09	0	\\
4.78e-09	0	\\
4.89e-09	0	\\
4.99e-09	0	\\
5e-09	0	\\
};
\addplot [color=mycolor2,solid,forget plot]
  table[row sep=crcr]{
0	0	\\
1.1e-10	0	\\
2.2e-10	0	\\
3.3e-10	0	\\
4.4e-10	0	\\
5.4e-10	0	\\
6.5e-10	0	\\
7.5e-10	0	\\
8.6e-10	0	\\
9.6e-10	0	\\
1.07e-09	0	\\
1.18e-09	0	\\
1.28e-09	0	\\
1.38e-09	0	\\
1.49e-09	0	\\
1.59e-09	0	\\
1.69e-09	0	\\
1.8e-09	0	\\
1.9e-09	0	\\
2.01e-09	0	\\
2.11e-09	0	\\
2.21e-09	0	\\
2.32e-09	0	\\
2.42e-09	0	\\
2.52e-09	0	\\
2.63e-09	0	\\
2.73e-09	0	\\
2.83e-09	0	\\
2.93e-09	0	\\
3.04e-09	0	\\
3.14e-09	0	\\
3.24e-09	0	\\
3.34e-09	0	\\
3.45e-09	0	\\
3.55e-09	0	\\
3.65e-09	0	\\
3.75e-09	0	\\
3.86e-09	0	\\
3.96e-09	0	\\
4.06e-09	0	\\
4.16e-09	0	\\
4.27e-09	0	\\
4.37e-09	0	\\
4.47e-09	0	\\
4.57e-09	0	\\
4.68e-09	0	\\
4.78e-09	0	\\
4.89e-09	0	\\
4.99e-09	0	\\
5e-09	0	\\
};
\addplot [color=mycolor3,solid,forget plot]
  table[row sep=crcr]{
0	0	\\
1.1e-10	0	\\
2.2e-10	0	\\
3.3e-10	0	\\
4.4e-10	0	\\
5.4e-10	0	\\
6.5e-10	0	\\
7.5e-10	0	\\
8.6e-10	0	\\
9.6e-10	0	\\
1.07e-09	0	\\
1.18e-09	0	\\
1.28e-09	0	\\
1.38e-09	0	\\
1.49e-09	0	\\
1.59e-09	0	\\
1.69e-09	0	\\
1.8e-09	0	\\
1.9e-09	0	\\
2.01e-09	0	\\
2.11e-09	0	\\
2.21e-09	0	\\
2.32e-09	0	\\
2.42e-09	0	\\
2.52e-09	0	\\
2.63e-09	0	\\
2.73e-09	0	\\
2.83e-09	0	\\
2.93e-09	0	\\
3.04e-09	0	\\
3.14e-09	0	\\
3.24e-09	0	\\
3.34e-09	0	\\
3.45e-09	0	\\
3.55e-09	0	\\
3.65e-09	0	\\
3.75e-09	0	\\
3.86e-09	0	\\
3.96e-09	0	\\
4.06e-09	0	\\
4.16e-09	0	\\
4.27e-09	0	\\
4.37e-09	0	\\
4.47e-09	0	\\
4.57e-09	0	\\
4.68e-09	0	\\
4.78e-09	0	\\
4.89e-09	0	\\
4.99e-09	0	\\
5e-09	0	\\
};
\addplot [color=darkgray,solid,forget plot]
  table[row sep=crcr]{
0	0	\\
1.1e-10	0	\\
2.2e-10	0	\\
3.3e-10	0	\\
4.4e-10	0	\\
5.4e-10	0	\\
6.5e-10	0	\\
7.5e-10	0	\\
8.6e-10	0	\\
9.6e-10	0	\\
1.07e-09	0	\\
1.18e-09	0	\\
1.28e-09	0	\\
1.38e-09	0	\\
1.49e-09	0	\\
1.59e-09	0	\\
1.69e-09	0	\\
1.8e-09	0	\\
1.9e-09	0	\\
2.01e-09	0	\\
2.11e-09	0	\\
2.21e-09	0	\\
2.32e-09	0	\\
2.42e-09	0	\\
2.52e-09	0	\\
2.63e-09	0	\\
2.73e-09	0	\\
2.83e-09	0	\\
2.93e-09	0	\\
3.04e-09	0	\\
3.14e-09	0	\\
3.24e-09	0	\\
3.34e-09	0	\\
3.45e-09	0	\\
3.55e-09	0	\\
3.65e-09	0	\\
3.75e-09	0	\\
3.86e-09	0	\\
3.96e-09	0	\\
4.06e-09	0	\\
4.16e-09	0	\\
4.27e-09	0	\\
4.37e-09	0	\\
4.47e-09	0	\\
4.57e-09	0	\\
4.68e-09	0	\\
4.78e-09	0	\\
4.89e-09	0	\\
4.99e-09	0	\\
5e-09	0	\\
};
\addplot [color=blue,solid,forget plot]
  table[row sep=crcr]{
0	0	\\
1.1e-10	0	\\
2.2e-10	0	\\
3.3e-10	0	\\
4.4e-10	0	\\
5.4e-10	0	\\
6.5e-10	0	\\
7.5e-10	0	\\
8.6e-10	0	\\
9.6e-10	0	\\
1.07e-09	0	\\
1.18e-09	0	\\
1.28e-09	0	\\
1.38e-09	0	\\
1.49e-09	0	\\
1.59e-09	0	\\
1.69e-09	0	\\
1.8e-09	0	\\
1.9e-09	0	\\
2.01e-09	0	\\
2.11e-09	0	\\
2.21e-09	0	\\
2.32e-09	0	\\
2.42e-09	0	\\
2.52e-09	0	\\
2.63e-09	0	\\
2.73e-09	0	\\
2.83e-09	0	\\
2.93e-09	0	\\
3.04e-09	0	\\
3.14e-09	0	\\
3.24e-09	0	\\
3.34e-09	0	\\
3.45e-09	0	\\
3.55e-09	0	\\
3.65e-09	0	\\
3.75e-09	0	\\
3.86e-09	0	\\
3.96e-09	0	\\
4.06e-09	0	\\
4.16e-09	0	\\
4.27e-09	0	\\
4.37e-09	0	\\
4.47e-09	0	\\
4.57e-09	0	\\
4.68e-09	0	\\
4.78e-09	0	\\
4.89e-09	0	\\
4.99e-09	0	\\
5e-09	0	\\
};
\addplot [color=black!50!green,solid,forget plot]
  table[row sep=crcr]{
0	0	\\
1.1e-10	0	\\
2.2e-10	0	\\
3.3e-10	0	\\
4.4e-10	0	\\
5.4e-10	0	\\
6.5e-10	0	\\
7.5e-10	0	\\
8.6e-10	0	\\
9.6e-10	0	\\
1.07e-09	0	\\
1.18e-09	0	\\
1.28e-09	0	\\
1.38e-09	0	\\
1.49e-09	0	\\
1.59e-09	0	\\
1.69e-09	0	\\
1.8e-09	0	\\
1.9e-09	0	\\
2.01e-09	0	\\
2.11e-09	0	\\
2.21e-09	0	\\
2.32e-09	0	\\
2.42e-09	0	\\
2.52e-09	0	\\
2.63e-09	0	\\
2.73e-09	0	\\
2.83e-09	0	\\
2.93e-09	0	\\
3.04e-09	0	\\
3.14e-09	0	\\
3.24e-09	0	\\
3.34e-09	0	\\
3.45e-09	0	\\
3.55e-09	0	\\
3.65e-09	0	\\
3.75e-09	0	\\
3.86e-09	0	\\
3.96e-09	0	\\
4.06e-09	0	\\
4.16e-09	0	\\
4.27e-09	0	\\
4.37e-09	0	\\
4.47e-09	0	\\
4.57e-09	0	\\
4.68e-09	0	\\
4.78e-09	0	\\
4.89e-09	0	\\
4.99e-09	0	\\
5e-09	0	\\
};
\addplot [color=red,solid,forget plot]
  table[row sep=crcr]{
0	0	\\
1.1e-10	0	\\
2.2e-10	0	\\
3.3e-10	0	\\
4.4e-10	0	\\
5.4e-10	0	\\
6.5e-10	0	\\
7.5e-10	0	\\
8.6e-10	0	\\
9.6e-10	0	\\
1.07e-09	0	\\
1.18e-09	0	\\
1.28e-09	0	\\
1.38e-09	0	\\
1.49e-09	0	\\
1.59e-09	0	\\
1.69e-09	0	\\
1.8e-09	0	\\
1.9e-09	0	\\
2.01e-09	0	\\
2.11e-09	0	\\
2.21e-09	0	\\
2.32e-09	0	\\
2.42e-09	0	\\
2.52e-09	0	\\
2.63e-09	0	\\
2.73e-09	0	\\
2.83e-09	0	\\
2.93e-09	0	\\
3.04e-09	0	\\
3.14e-09	0	\\
3.24e-09	0	\\
3.34e-09	0	\\
3.45e-09	0	\\
3.55e-09	0	\\
3.65e-09	0	\\
3.75e-09	0	\\
3.86e-09	0	\\
3.96e-09	0	\\
4.06e-09	0	\\
4.16e-09	0	\\
4.27e-09	0	\\
4.37e-09	0	\\
4.47e-09	0	\\
4.57e-09	0	\\
4.68e-09	0	\\
4.78e-09	0	\\
4.89e-09	0	\\
4.99e-09	0	\\
5e-09	0	\\
};
\addplot [color=mycolor1,solid,forget plot]
  table[row sep=crcr]{
0	0	\\
1.1e-10	0	\\
2.2e-10	0	\\
3.3e-10	0	\\
4.4e-10	0	\\
5.4e-10	0	\\
6.5e-10	0	\\
7.5e-10	0	\\
8.6e-10	0	\\
9.6e-10	0	\\
1.07e-09	0	\\
1.18e-09	0	\\
1.28e-09	0	\\
1.38e-09	0	\\
1.49e-09	0	\\
1.59e-09	0	\\
1.69e-09	0	\\
1.8e-09	0	\\
1.9e-09	0	\\
2.01e-09	0	\\
2.11e-09	0	\\
2.21e-09	0	\\
2.32e-09	0	\\
2.42e-09	0	\\
2.52e-09	0	\\
2.63e-09	0	\\
2.73e-09	0	\\
2.83e-09	0	\\
2.93e-09	0	\\
3.04e-09	0	\\
3.14e-09	0	\\
3.24e-09	0	\\
3.34e-09	0	\\
3.45e-09	0	\\
3.55e-09	0	\\
3.65e-09	0	\\
3.75e-09	0	\\
3.86e-09	0	\\
3.96e-09	0	\\
4.06e-09	0	\\
4.16e-09	0	\\
4.27e-09	0	\\
4.37e-09	0	\\
4.47e-09	0	\\
4.57e-09	0	\\
4.68e-09	0	\\
4.78e-09	0	\\
4.89e-09	0	\\
4.99e-09	0	\\
5e-09	0	\\
};
\addplot [color=mycolor2,solid,forget plot]
  table[row sep=crcr]{
0	0	\\
1.1e-10	0	\\
2.2e-10	0	\\
3.3e-10	0	\\
4.4e-10	0	\\
5.4e-10	0	\\
6.5e-10	0	\\
7.5e-10	0	\\
8.6e-10	0	\\
9.6e-10	0	\\
1.07e-09	0	\\
1.18e-09	0	\\
1.28e-09	0	\\
1.38e-09	0	\\
1.49e-09	0	\\
1.59e-09	0	\\
1.69e-09	0	\\
1.8e-09	0	\\
1.9e-09	0	\\
2.01e-09	0	\\
2.11e-09	0	\\
2.21e-09	0	\\
2.32e-09	0	\\
2.42e-09	0	\\
2.52e-09	0	\\
2.63e-09	0	\\
2.73e-09	0	\\
2.83e-09	0	\\
2.93e-09	0	\\
3.04e-09	0	\\
3.14e-09	0	\\
3.24e-09	0	\\
3.34e-09	0	\\
3.45e-09	0	\\
3.55e-09	0	\\
3.65e-09	0	\\
3.75e-09	0	\\
3.86e-09	0	\\
3.96e-09	0	\\
4.06e-09	0	\\
4.16e-09	0	\\
4.27e-09	0	\\
4.37e-09	0	\\
4.47e-09	0	\\
4.57e-09	0	\\
4.68e-09	0	\\
4.78e-09	0	\\
4.89e-09	0	\\
4.99e-09	0	\\
5e-09	0	\\
};
\addplot [color=mycolor3,solid,forget plot]
  table[row sep=crcr]{
0	0	\\
1.1e-10	0	\\
2.2e-10	0	\\
3.3e-10	0	\\
4.4e-10	0	\\
5.4e-10	0	\\
6.5e-10	0	\\
7.5e-10	0	\\
8.6e-10	0	\\
9.6e-10	0	\\
1.07e-09	0	\\
1.18e-09	0	\\
1.28e-09	0	\\
1.38e-09	0	\\
1.49e-09	0	\\
1.59e-09	0	\\
1.69e-09	0	\\
1.8e-09	0	\\
1.9e-09	0	\\
2.01e-09	0	\\
2.11e-09	0	\\
2.21e-09	0	\\
2.32e-09	0	\\
2.42e-09	0	\\
2.52e-09	0	\\
2.63e-09	0	\\
2.73e-09	0	\\
2.83e-09	0	\\
2.93e-09	0	\\
3.04e-09	0	\\
3.14e-09	0	\\
3.24e-09	0	\\
3.34e-09	0	\\
3.45e-09	0	\\
3.55e-09	0	\\
3.65e-09	0	\\
3.75e-09	0	\\
3.86e-09	0	\\
3.96e-09	0	\\
4.06e-09	0	\\
4.16e-09	0	\\
4.27e-09	0	\\
4.37e-09	0	\\
4.47e-09	0	\\
4.57e-09	0	\\
4.68e-09	0	\\
4.78e-09	0	\\
4.89e-09	0	\\
4.99e-09	0	\\
5e-09	0	\\
};
\addplot [color=darkgray,solid,forget plot]
  table[row sep=crcr]{
0	0	\\
1.1e-10	0	\\
2.2e-10	0	\\
3.3e-10	0	\\
4.4e-10	0	\\
5.4e-10	0	\\
6.5e-10	0	\\
7.5e-10	0	\\
8.6e-10	0	\\
9.6e-10	0	\\
1.07e-09	0	\\
1.18e-09	0	\\
1.28e-09	0	\\
1.38e-09	0	\\
1.49e-09	0	\\
1.59e-09	0	\\
1.69e-09	0	\\
1.8e-09	0	\\
1.9e-09	0	\\
2.01e-09	0	\\
2.11e-09	0	\\
2.21e-09	0	\\
2.32e-09	0	\\
2.42e-09	0	\\
2.52e-09	0	\\
2.63e-09	0	\\
2.73e-09	0	\\
2.83e-09	0	\\
2.93e-09	0	\\
3.04e-09	0	\\
3.14e-09	0	\\
3.24e-09	0	\\
3.34e-09	0	\\
3.45e-09	0	\\
3.55e-09	0	\\
3.65e-09	0	\\
3.75e-09	0	\\
3.86e-09	0	\\
3.96e-09	0	\\
4.06e-09	0	\\
4.16e-09	0	\\
4.27e-09	0	\\
4.37e-09	0	\\
4.47e-09	0	\\
4.57e-09	0	\\
4.68e-09	0	\\
4.78e-09	0	\\
4.89e-09	0	\\
4.99e-09	0	\\
5e-09	0	\\
};
\addplot [color=blue,solid,forget plot]
  table[row sep=crcr]{
0	0	\\
1.1e-10	0	\\
2.2e-10	0	\\
3.3e-10	0	\\
4.4e-10	0	\\
5.4e-10	0	\\
6.5e-10	0	\\
7.5e-10	0	\\
8.6e-10	0	\\
9.6e-10	0	\\
1.07e-09	0	\\
1.18e-09	0	\\
1.28e-09	0	\\
1.38e-09	0	\\
1.49e-09	0	\\
1.59e-09	0	\\
1.69e-09	0	\\
1.8e-09	0	\\
1.9e-09	0	\\
2.01e-09	0	\\
2.11e-09	0	\\
2.21e-09	0	\\
2.32e-09	0	\\
2.42e-09	0	\\
2.52e-09	0	\\
2.63e-09	0	\\
2.73e-09	0	\\
2.83e-09	0	\\
2.93e-09	0	\\
3.04e-09	0	\\
3.14e-09	0	\\
3.24e-09	0	\\
3.34e-09	0	\\
3.45e-09	0	\\
3.55e-09	0	\\
3.65e-09	0	\\
3.75e-09	0	\\
3.86e-09	0	\\
3.96e-09	0	\\
4.06e-09	0	\\
4.16e-09	0	\\
4.27e-09	0	\\
4.37e-09	0	\\
4.47e-09	0	\\
4.57e-09	0	\\
4.68e-09	0	\\
4.78e-09	0	\\
4.89e-09	0	\\
4.99e-09	0	\\
5e-09	0	\\
};
\addplot [color=black!50!green,solid,forget plot]
  table[row sep=crcr]{
0	0	\\
1.1e-10	0	\\
2.2e-10	0	\\
3.3e-10	0	\\
4.4e-10	0	\\
5.4e-10	0	\\
6.5e-10	0	\\
7.5e-10	0	\\
8.6e-10	0	\\
9.6e-10	0	\\
1.07e-09	0	\\
1.18e-09	0	\\
1.28e-09	0	\\
1.38e-09	0	\\
1.49e-09	0	\\
1.59e-09	0	\\
1.69e-09	0	\\
1.8e-09	0	\\
1.9e-09	0	\\
2.01e-09	0	\\
2.11e-09	0	\\
2.21e-09	0	\\
2.32e-09	0	\\
2.42e-09	0	\\
2.52e-09	0	\\
2.63e-09	0	\\
2.73e-09	0	\\
2.83e-09	0	\\
2.93e-09	0	\\
3.04e-09	0	\\
3.14e-09	0	\\
3.24e-09	0	\\
3.34e-09	0	\\
3.45e-09	0	\\
3.55e-09	0	\\
3.65e-09	0	\\
3.75e-09	0	\\
3.86e-09	0	\\
3.96e-09	0	\\
4.06e-09	0	\\
4.16e-09	0	\\
4.27e-09	0	\\
4.37e-09	0	\\
4.47e-09	0	\\
4.57e-09	0	\\
4.68e-09	0	\\
4.78e-09	0	\\
4.89e-09	0	\\
4.99e-09	0	\\
5e-09	0	\\
};
\addplot [color=red,solid,forget plot]
  table[row sep=crcr]{
0	0	\\
1.1e-10	0	\\
2.2e-10	0	\\
3.3e-10	0	\\
4.4e-10	0	\\
5.4e-10	0	\\
6.5e-10	0	\\
7.5e-10	0	\\
8.6e-10	0	\\
9.6e-10	0	\\
1.07e-09	0	\\
1.18e-09	0	\\
1.28e-09	0	\\
1.38e-09	0	\\
1.49e-09	0	\\
1.59e-09	0	\\
1.69e-09	0	\\
1.8e-09	0	\\
1.9e-09	0	\\
2.01e-09	0	\\
2.11e-09	0	\\
2.21e-09	0	\\
2.32e-09	0	\\
2.42e-09	0	\\
2.52e-09	0	\\
2.63e-09	0	\\
2.73e-09	0	\\
2.83e-09	0	\\
2.93e-09	0	\\
3.04e-09	0	\\
3.14e-09	0	\\
3.24e-09	0	\\
3.34e-09	0	\\
3.45e-09	0	\\
3.55e-09	0	\\
3.65e-09	0	\\
3.75e-09	0	\\
3.86e-09	0	\\
3.96e-09	0	\\
4.06e-09	0	\\
4.16e-09	0	\\
4.27e-09	0	\\
4.37e-09	0	\\
4.47e-09	0	\\
4.57e-09	0	\\
4.68e-09	0	\\
4.78e-09	0	\\
4.89e-09	0	\\
4.99e-09	0	\\
5e-09	0	\\
};
\addplot [color=mycolor1,solid,forget plot]
  table[row sep=crcr]{
0	0	\\
1.1e-10	0	\\
2.2e-10	0	\\
3.3e-10	0	\\
4.4e-10	0	\\
5.4e-10	0	\\
6.5e-10	0	\\
7.5e-10	0	\\
8.6e-10	0	\\
9.6e-10	0	\\
1.07e-09	0	\\
1.18e-09	0	\\
1.28e-09	0	\\
1.38e-09	0	\\
1.49e-09	0	\\
1.59e-09	0	\\
1.69e-09	0	\\
1.8e-09	0	\\
1.9e-09	0	\\
2.01e-09	0	\\
2.11e-09	0	\\
2.21e-09	0	\\
2.32e-09	0	\\
2.42e-09	0	\\
2.52e-09	0	\\
2.63e-09	0	\\
2.73e-09	0	\\
2.83e-09	0	\\
2.93e-09	0	\\
3.04e-09	0	\\
3.14e-09	0	\\
3.24e-09	0	\\
3.34e-09	0	\\
3.45e-09	0	\\
3.55e-09	0	\\
3.65e-09	0	\\
3.75e-09	0	\\
3.86e-09	0	\\
3.96e-09	0	\\
4.06e-09	0	\\
4.16e-09	0	\\
4.27e-09	0	\\
4.37e-09	0	\\
4.47e-09	0	\\
4.57e-09	0	\\
4.68e-09	0	\\
4.78e-09	0	\\
4.89e-09	0	\\
4.99e-09	0	\\
5e-09	0	\\
};
\addplot [color=mycolor2,solid,forget plot]
  table[row sep=crcr]{
0	0	\\
1.1e-10	0	\\
2.2e-10	0	\\
3.3e-10	0	\\
4.4e-10	0	\\
5.4e-10	0	\\
6.5e-10	0	\\
7.5e-10	0	\\
8.6e-10	0	\\
9.6e-10	0	\\
1.07e-09	0	\\
1.18e-09	0	\\
1.28e-09	0	\\
1.38e-09	0	\\
1.49e-09	0	\\
1.59e-09	0	\\
1.69e-09	0	\\
1.8e-09	0	\\
1.9e-09	0	\\
2.01e-09	0	\\
2.11e-09	0	\\
2.21e-09	0	\\
2.32e-09	0	\\
2.42e-09	0	\\
2.52e-09	0	\\
2.63e-09	0	\\
2.73e-09	0	\\
2.83e-09	0	\\
2.93e-09	0	\\
3.04e-09	0	\\
3.14e-09	0	\\
3.24e-09	0	\\
3.34e-09	0	\\
3.45e-09	0	\\
3.55e-09	0	\\
3.65e-09	0	\\
3.75e-09	0	\\
3.86e-09	0	\\
3.96e-09	0	\\
4.06e-09	0	\\
4.16e-09	0	\\
4.27e-09	0	\\
4.37e-09	0	\\
4.47e-09	0	\\
4.57e-09	0	\\
4.68e-09	0	\\
4.78e-09	0	\\
4.89e-09	0	\\
4.99e-09	0	\\
5e-09	0	\\
};
\addplot [color=mycolor3,solid,forget plot]
  table[row sep=crcr]{
0	0	\\
1.1e-10	0	\\
2.2e-10	0	\\
3.3e-10	0	\\
4.4e-10	0	\\
5.4e-10	0	\\
6.5e-10	0	\\
7.5e-10	0	\\
8.6e-10	0	\\
9.6e-10	0	\\
1.07e-09	0	\\
1.18e-09	0	\\
1.28e-09	0	\\
1.38e-09	0	\\
1.49e-09	0	\\
1.59e-09	0	\\
1.69e-09	0	\\
1.8e-09	0	\\
1.9e-09	0	\\
2.01e-09	0	\\
2.11e-09	0	\\
2.21e-09	0	\\
2.32e-09	0	\\
2.42e-09	0	\\
2.52e-09	0	\\
2.63e-09	0	\\
2.73e-09	0	\\
2.83e-09	0	\\
2.93e-09	0	\\
3.04e-09	0	\\
3.14e-09	0	\\
3.24e-09	0	\\
3.34e-09	0	\\
3.45e-09	0	\\
3.55e-09	0	\\
3.65e-09	0	\\
3.75e-09	0	\\
3.86e-09	0	\\
3.96e-09	0	\\
4.06e-09	0	\\
4.16e-09	0	\\
4.27e-09	0	\\
4.37e-09	0	\\
4.47e-09	0	\\
4.57e-09	0	\\
4.68e-09	0	\\
4.78e-09	0	\\
4.89e-09	0	\\
4.99e-09	0	\\
5e-09	0	\\
};
\addplot [color=darkgray,solid,forget plot]
  table[row sep=crcr]{
0	0	\\
1.1e-10	0	\\
2.2e-10	0	\\
3.3e-10	0	\\
4.4e-10	0	\\
5.4e-10	0	\\
6.5e-10	0	\\
7.5e-10	0	\\
8.6e-10	0	\\
9.6e-10	0	\\
1.07e-09	0	\\
1.18e-09	0	\\
1.28e-09	0	\\
1.38e-09	0	\\
1.49e-09	0	\\
1.59e-09	0	\\
1.69e-09	0	\\
1.8e-09	0	\\
1.9e-09	0	\\
2.01e-09	0	\\
2.11e-09	0	\\
2.21e-09	0	\\
2.32e-09	0	\\
2.42e-09	0	\\
2.52e-09	0	\\
2.63e-09	0	\\
2.73e-09	0	\\
2.83e-09	0	\\
2.93e-09	0	\\
3.04e-09	0	\\
3.14e-09	0	\\
3.24e-09	0	\\
3.34e-09	0	\\
3.45e-09	0	\\
3.55e-09	0	\\
3.65e-09	0	\\
3.75e-09	0	\\
3.86e-09	0	\\
3.96e-09	0	\\
4.06e-09	0	\\
4.16e-09	0	\\
4.27e-09	0	\\
4.37e-09	0	\\
4.47e-09	0	\\
4.57e-09	0	\\
4.68e-09	0	\\
4.78e-09	0	\\
4.89e-09	0	\\
4.99e-09	0	\\
5e-09	0	\\
};
\addplot [color=blue,solid,forget plot]
  table[row sep=crcr]{
0	0	\\
1.1e-10	0	\\
2.2e-10	0	\\
3.3e-10	0	\\
4.4e-10	0	\\
5.4e-10	0	\\
6.5e-10	0	\\
7.5e-10	0	\\
8.6e-10	0	\\
9.6e-10	0	\\
1.07e-09	0	\\
1.18e-09	0	\\
1.28e-09	0	\\
1.38e-09	0	\\
1.49e-09	0	\\
1.59e-09	0	\\
1.69e-09	0	\\
1.8e-09	0	\\
1.9e-09	0	\\
2.01e-09	0	\\
2.11e-09	0	\\
2.21e-09	0	\\
2.32e-09	0	\\
2.42e-09	0	\\
2.52e-09	0	\\
2.63e-09	0	\\
2.73e-09	0	\\
2.83e-09	0	\\
2.93e-09	0	\\
3.04e-09	0	\\
3.14e-09	0	\\
3.24e-09	0	\\
3.34e-09	0	\\
3.45e-09	0	\\
3.55e-09	0	\\
3.65e-09	0	\\
3.75e-09	0	\\
3.86e-09	0	\\
3.96e-09	0	\\
4.06e-09	0	\\
4.16e-09	0	\\
4.27e-09	0	\\
4.37e-09	0	\\
4.47e-09	0	\\
4.57e-09	0	\\
4.68e-09	0	\\
4.78e-09	0	\\
4.89e-09	0	\\
4.99e-09	0	\\
5e-09	0	\\
};
\addplot [color=black!50!green,solid,forget plot]
  table[row sep=crcr]{
0	0	\\
1.1e-10	0	\\
2.2e-10	0	\\
3.3e-10	0	\\
4.4e-10	0	\\
5.4e-10	0	\\
6.5e-10	0	\\
7.5e-10	0	\\
8.6e-10	0	\\
9.6e-10	0	\\
1.07e-09	0	\\
1.18e-09	0	\\
1.28e-09	0	\\
1.38e-09	0	\\
1.49e-09	0	\\
1.59e-09	0	\\
1.69e-09	0	\\
1.8e-09	0	\\
1.9e-09	0	\\
2.01e-09	0	\\
2.11e-09	0	\\
2.21e-09	0	\\
2.32e-09	0	\\
2.42e-09	0	\\
2.52e-09	0	\\
2.63e-09	0	\\
2.73e-09	0	\\
2.83e-09	0	\\
2.93e-09	0	\\
3.04e-09	0	\\
3.14e-09	0	\\
3.24e-09	0	\\
3.34e-09	0	\\
3.45e-09	0	\\
3.55e-09	0	\\
3.65e-09	0	\\
3.75e-09	0	\\
3.86e-09	0	\\
3.96e-09	0	\\
4.06e-09	0	\\
4.16e-09	0	\\
4.27e-09	0	\\
4.37e-09	0	\\
4.47e-09	0	\\
4.57e-09	0	\\
4.68e-09	0	\\
4.78e-09	0	\\
4.89e-09	0	\\
4.99e-09	0	\\
5e-09	0	\\
};
\addplot [color=red,solid,forget plot]
  table[row sep=crcr]{
0	0	\\
1.1e-10	0	\\
2.2e-10	0	\\
3.3e-10	0	\\
4.4e-10	0	\\
5.4e-10	0	\\
6.5e-10	0	\\
7.5e-10	0	\\
8.6e-10	0	\\
9.6e-10	0	\\
1.07e-09	0	\\
1.18e-09	0	\\
1.28e-09	0	\\
1.38e-09	0	\\
1.49e-09	0	\\
1.59e-09	0	\\
1.69e-09	0	\\
1.8e-09	0	\\
1.9e-09	0	\\
2.01e-09	0	\\
2.11e-09	0	\\
2.21e-09	0	\\
2.32e-09	0	\\
2.42e-09	0	\\
2.52e-09	0	\\
2.63e-09	0	\\
2.73e-09	0	\\
2.83e-09	0	\\
2.93e-09	0	\\
3.04e-09	0	\\
3.14e-09	0	\\
3.24e-09	0	\\
3.34e-09	0	\\
3.45e-09	0	\\
3.55e-09	0	\\
3.65e-09	0	\\
3.75e-09	0	\\
3.86e-09	0	\\
3.96e-09	0	\\
4.06e-09	0	\\
4.16e-09	0	\\
4.27e-09	0	\\
4.37e-09	0	\\
4.47e-09	0	\\
4.57e-09	0	\\
4.68e-09	0	\\
4.78e-09	0	\\
4.89e-09	0	\\
4.99e-09	0	\\
5e-09	0	\\
};
\addplot [color=mycolor1,solid,forget plot]
  table[row sep=crcr]{
0	0	\\
1.1e-10	0	\\
2.2e-10	0	\\
3.3e-10	0	\\
4.4e-10	0	\\
5.4e-10	0	\\
6.5e-10	0	\\
7.5e-10	0	\\
8.6e-10	0	\\
9.6e-10	0	\\
1.07e-09	0	\\
1.18e-09	0	\\
1.28e-09	0	\\
1.38e-09	0	\\
1.49e-09	0	\\
1.59e-09	0	\\
1.69e-09	0	\\
1.8e-09	0	\\
1.9e-09	0	\\
2.01e-09	0	\\
2.11e-09	0	\\
2.21e-09	0	\\
2.32e-09	0	\\
2.42e-09	0	\\
2.52e-09	0	\\
2.63e-09	0	\\
2.73e-09	0	\\
2.83e-09	0	\\
2.93e-09	0	\\
3.04e-09	0	\\
3.14e-09	0	\\
3.24e-09	0	\\
3.34e-09	0	\\
3.45e-09	0	\\
3.55e-09	0	\\
3.65e-09	0	\\
3.75e-09	0	\\
3.86e-09	0	\\
3.96e-09	0	\\
4.06e-09	0	\\
4.16e-09	0	\\
4.27e-09	0	\\
4.37e-09	0	\\
4.47e-09	0	\\
4.57e-09	0	\\
4.68e-09	0	\\
4.78e-09	0	\\
4.89e-09	0	\\
4.99e-09	0	\\
5e-09	0	\\
};
\addplot [color=mycolor2,solid,forget plot]
  table[row sep=crcr]{
0	0	\\
1.1e-10	0	\\
2.2e-10	0	\\
3.3e-10	0	\\
4.4e-10	0	\\
5.4e-10	0	\\
6.5e-10	0	\\
7.5e-10	0	\\
8.6e-10	0	\\
9.6e-10	0	\\
1.07e-09	0	\\
1.18e-09	0	\\
1.28e-09	0	\\
1.38e-09	0	\\
1.49e-09	0	\\
1.59e-09	0	\\
1.69e-09	0	\\
1.8e-09	0	\\
1.9e-09	0	\\
2.01e-09	0	\\
2.11e-09	0	\\
2.21e-09	0	\\
2.32e-09	0	\\
2.42e-09	0	\\
2.52e-09	0	\\
2.63e-09	0	\\
2.73e-09	0	\\
2.83e-09	0	\\
2.93e-09	0	\\
3.04e-09	0	\\
3.14e-09	0	\\
3.24e-09	0	\\
3.34e-09	0	\\
3.45e-09	0	\\
3.55e-09	0	\\
3.65e-09	0	\\
3.75e-09	0	\\
3.86e-09	0	\\
3.96e-09	0	\\
4.06e-09	0	\\
4.16e-09	0	\\
4.27e-09	0	\\
4.37e-09	0	\\
4.47e-09	0	\\
4.57e-09	0	\\
4.68e-09	0	\\
4.78e-09	0	\\
4.89e-09	0	\\
4.99e-09	0	\\
5e-09	0	\\
};
\addplot [color=mycolor3,solid,forget plot]
  table[row sep=crcr]{
0	0	\\
1.1e-10	0	\\
2.2e-10	0	\\
3.3e-10	0	\\
4.4e-10	0	\\
5.4e-10	0	\\
6.5e-10	0	\\
7.5e-10	0	\\
8.6e-10	0	\\
9.6e-10	0	\\
1.07e-09	0	\\
1.18e-09	0	\\
1.28e-09	0	\\
1.38e-09	0	\\
1.49e-09	0	\\
1.59e-09	0	\\
1.69e-09	0	\\
1.8e-09	0	\\
1.9e-09	0	\\
2.01e-09	0	\\
2.11e-09	0	\\
2.21e-09	0	\\
2.32e-09	0	\\
2.42e-09	0	\\
2.52e-09	0	\\
2.63e-09	0	\\
2.73e-09	0	\\
2.83e-09	0	\\
2.93e-09	0	\\
3.04e-09	0	\\
3.14e-09	0	\\
3.24e-09	0	\\
3.34e-09	0	\\
3.45e-09	0	\\
3.55e-09	0	\\
3.65e-09	0	\\
3.75e-09	0	\\
3.86e-09	0	\\
3.96e-09	0	\\
4.06e-09	0	\\
4.16e-09	0	\\
4.27e-09	0	\\
4.37e-09	0	\\
4.47e-09	0	\\
4.57e-09	0	\\
4.68e-09	0	\\
4.78e-09	0	\\
4.89e-09	0	\\
4.99e-09	0	\\
5e-09	0	\\
};
\addplot [color=darkgray,solid,forget plot]
  table[row sep=crcr]{
0	0	\\
1.1e-10	0	\\
2.2e-10	0	\\
3.3e-10	0	\\
4.4e-10	0	\\
5.4e-10	0	\\
6.5e-10	0	\\
7.5e-10	0	\\
8.6e-10	0	\\
9.6e-10	0	\\
1.07e-09	0	\\
1.18e-09	0	\\
1.28e-09	0	\\
1.38e-09	0	\\
1.49e-09	0	\\
1.59e-09	0	\\
1.69e-09	0	\\
1.8e-09	0	\\
1.9e-09	0	\\
2.01e-09	0	\\
2.11e-09	0	\\
2.21e-09	0	\\
2.32e-09	0	\\
2.42e-09	0	\\
2.52e-09	0	\\
2.63e-09	0	\\
2.73e-09	0	\\
2.83e-09	0	\\
2.93e-09	0	\\
3.04e-09	0	\\
3.14e-09	0	\\
3.24e-09	0	\\
3.34e-09	0	\\
3.45e-09	0	\\
3.55e-09	0	\\
3.65e-09	0	\\
3.75e-09	0	\\
3.86e-09	0	\\
3.96e-09	0	\\
4.06e-09	0	\\
4.16e-09	0	\\
4.27e-09	0	\\
4.37e-09	0	\\
4.47e-09	0	\\
4.57e-09	0	\\
4.68e-09	0	\\
4.78e-09	0	\\
4.89e-09	0	\\
4.99e-09	0	\\
5e-09	0	\\
};
\addplot [color=blue,solid,forget plot]
  table[row sep=crcr]{
0	0	\\
1.1e-10	0	\\
2.2e-10	0	\\
3.3e-10	0	\\
4.4e-10	0	\\
5.4e-10	0	\\
6.5e-10	0	\\
7.5e-10	0	\\
8.6e-10	0	\\
9.6e-10	0	\\
1.07e-09	0	\\
1.18e-09	0	\\
1.28e-09	0	\\
1.38e-09	0	\\
1.49e-09	0	\\
1.59e-09	0	\\
1.69e-09	0	\\
1.8e-09	0	\\
1.9e-09	0	\\
2.01e-09	0	\\
2.11e-09	0	\\
2.21e-09	0	\\
2.32e-09	0	\\
2.42e-09	0	\\
2.52e-09	0	\\
2.63e-09	0	\\
2.73e-09	0	\\
2.83e-09	0	\\
2.93e-09	0	\\
3.04e-09	0	\\
3.14e-09	0	\\
3.24e-09	0	\\
3.34e-09	0	\\
3.45e-09	0	\\
3.55e-09	0	\\
3.65e-09	0	\\
3.75e-09	0	\\
3.86e-09	0	\\
3.96e-09	0	\\
4.06e-09	0	\\
4.16e-09	0	\\
4.27e-09	0	\\
4.37e-09	0	\\
4.47e-09	0	\\
4.57e-09	0	\\
4.68e-09	0	\\
4.78e-09	0	\\
4.89e-09	0	\\
4.99e-09	0	\\
5e-09	0	\\
};
\addplot [color=black!50!green,solid,forget plot]
  table[row sep=crcr]{
0	0	\\
1.1e-10	0	\\
2.2e-10	0	\\
3.3e-10	0	\\
4.4e-10	0	\\
5.4e-10	0	\\
6.5e-10	0	\\
7.5e-10	0	\\
8.6e-10	0	\\
9.6e-10	0	\\
1.07e-09	0	\\
1.18e-09	0	\\
1.28e-09	0	\\
1.38e-09	0	\\
1.49e-09	0	\\
1.59e-09	0	\\
1.69e-09	0	\\
1.8e-09	0	\\
1.9e-09	0	\\
2.01e-09	0	\\
2.11e-09	0	\\
2.21e-09	0	\\
2.32e-09	0	\\
2.42e-09	0	\\
2.52e-09	0	\\
2.63e-09	0	\\
2.73e-09	0	\\
2.83e-09	0	\\
2.93e-09	0	\\
3.04e-09	0	\\
3.14e-09	0	\\
3.24e-09	0	\\
3.34e-09	0	\\
3.45e-09	0	\\
3.55e-09	0	\\
3.65e-09	0	\\
3.75e-09	0	\\
3.86e-09	0	\\
3.96e-09	0	\\
4.06e-09	0	\\
4.16e-09	0	\\
4.27e-09	0	\\
4.37e-09	0	\\
4.47e-09	0	\\
4.57e-09	0	\\
4.68e-09	0	\\
4.78e-09	0	\\
4.89e-09	0	\\
4.99e-09	0	\\
5e-09	0	\\
};
\addplot [color=red,solid,forget plot]
  table[row sep=crcr]{
0	0	\\
1.1e-10	0	\\
2.2e-10	0	\\
3.3e-10	0	\\
4.4e-10	0	\\
5.4e-10	0	\\
6.5e-10	0	\\
7.5e-10	0	\\
8.6e-10	0	\\
9.6e-10	0	\\
1.07e-09	0	\\
1.18e-09	0	\\
1.28e-09	0	\\
1.38e-09	0	\\
1.49e-09	0	\\
1.59e-09	0	\\
1.69e-09	0	\\
1.8e-09	0	\\
1.9e-09	0	\\
2.01e-09	0	\\
2.11e-09	0	\\
2.21e-09	0	\\
2.32e-09	0	\\
2.42e-09	0	\\
2.52e-09	0	\\
2.63e-09	0	\\
2.73e-09	0	\\
2.83e-09	0	\\
2.93e-09	0	\\
3.04e-09	0	\\
3.14e-09	0	\\
3.24e-09	0	\\
3.34e-09	0	\\
3.45e-09	0	\\
3.55e-09	0	\\
3.65e-09	0	\\
3.75e-09	0	\\
3.86e-09	0	\\
3.96e-09	0	\\
4.06e-09	0	\\
4.16e-09	0	\\
4.27e-09	0	\\
4.37e-09	0	\\
4.47e-09	0	\\
4.57e-09	0	\\
4.68e-09	0	\\
4.78e-09	0	\\
4.89e-09	0	\\
4.99e-09	0	\\
5e-09	0	\\
};
\addplot [color=mycolor1,solid,forget plot]
  table[row sep=crcr]{
0	0	\\
1.1e-10	0	\\
2.2e-10	0	\\
3.3e-10	0	\\
4.4e-10	0	\\
5.4e-10	0	\\
6.5e-10	0	\\
7.5e-10	0	\\
8.6e-10	0	\\
9.6e-10	0	\\
1.07e-09	0	\\
1.18e-09	0	\\
1.28e-09	0	\\
1.38e-09	0	\\
1.49e-09	0	\\
1.59e-09	0	\\
1.69e-09	0	\\
1.8e-09	0	\\
1.9e-09	0	\\
2.01e-09	0	\\
2.11e-09	0	\\
2.21e-09	0	\\
2.32e-09	0	\\
2.42e-09	0	\\
2.52e-09	0	\\
2.63e-09	0	\\
2.73e-09	0	\\
2.83e-09	0	\\
2.93e-09	0	\\
3.04e-09	0	\\
3.14e-09	0	\\
3.24e-09	0	\\
3.34e-09	0	\\
3.45e-09	0	\\
3.55e-09	0	\\
3.65e-09	0	\\
3.75e-09	0	\\
3.86e-09	0	\\
3.96e-09	0	\\
4.06e-09	0	\\
4.16e-09	0	\\
4.27e-09	0	\\
4.37e-09	0	\\
4.47e-09	0	\\
4.57e-09	0	\\
4.68e-09	0	\\
4.78e-09	0	\\
4.89e-09	0	\\
4.99e-09	0	\\
5e-09	0	\\
};
\addplot [color=mycolor2,solid,forget plot]
  table[row sep=crcr]{
0	0	\\
1.1e-10	0	\\
2.2e-10	0	\\
3.3e-10	0	\\
4.4e-10	0	\\
5.4e-10	0	\\
6.5e-10	0	\\
7.5e-10	0	\\
8.6e-10	0	\\
9.6e-10	0	\\
1.07e-09	0	\\
1.18e-09	0	\\
1.28e-09	0	\\
1.38e-09	0	\\
1.49e-09	0	\\
1.59e-09	0	\\
1.69e-09	0	\\
1.8e-09	0	\\
1.9e-09	0	\\
2.01e-09	0	\\
2.11e-09	0	\\
2.21e-09	0	\\
2.32e-09	0	\\
2.42e-09	0	\\
2.52e-09	0	\\
2.63e-09	0	\\
2.73e-09	0	\\
2.83e-09	0	\\
2.93e-09	0	\\
3.04e-09	0	\\
3.14e-09	0	\\
3.24e-09	0	\\
3.34e-09	0	\\
3.45e-09	0	\\
3.55e-09	0	\\
3.65e-09	0	\\
3.75e-09	0	\\
3.86e-09	0	\\
3.96e-09	0	\\
4.06e-09	0	\\
4.16e-09	0	\\
4.27e-09	0	\\
4.37e-09	0	\\
4.47e-09	0	\\
4.57e-09	0	\\
4.68e-09	0	\\
4.78e-09	0	\\
4.89e-09	0	\\
4.99e-09	0	\\
5e-09	0	\\
};
\addplot [color=mycolor3,solid,forget plot]
  table[row sep=crcr]{
0	0	\\
1.1e-10	0	\\
2.2e-10	0	\\
3.3e-10	0	\\
4.4e-10	0	\\
5.4e-10	0	\\
6.5e-10	0	\\
7.5e-10	0	\\
8.6e-10	0	\\
9.6e-10	0	\\
1.07e-09	0	\\
1.18e-09	0	\\
1.28e-09	0	\\
1.38e-09	0	\\
1.49e-09	0	\\
1.59e-09	0	\\
1.69e-09	0	\\
1.8e-09	0	\\
1.9e-09	0	\\
2.01e-09	0	\\
2.11e-09	0	\\
2.21e-09	0	\\
2.32e-09	0	\\
2.42e-09	0	\\
2.52e-09	0	\\
2.63e-09	0	\\
2.73e-09	0	\\
2.83e-09	0	\\
2.93e-09	0	\\
3.04e-09	0	\\
3.14e-09	0	\\
3.24e-09	0	\\
3.34e-09	0	\\
3.45e-09	0	\\
3.55e-09	0	\\
3.65e-09	0	\\
3.75e-09	0	\\
3.86e-09	0	\\
3.96e-09	0	\\
4.06e-09	0	\\
4.16e-09	0	\\
4.27e-09	0	\\
4.37e-09	0	\\
4.47e-09	0	\\
4.57e-09	0	\\
4.68e-09	0	\\
4.78e-09	0	\\
4.89e-09	0	\\
4.99e-09	0	\\
5e-09	0	\\
};
\addplot [color=darkgray,solid,forget plot]
  table[row sep=crcr]{
0	0	\\
1.1e-10	0	\\
2.2e-10	0	\\
3.3e-10	0	\\
4.4e-10	0	\\
5.4e-10	0	\\
6.5e-10	0	\\
7.5e-10	0	\\
8.6e-10	0	\\
9.6e-10	0	\\
1.07e-09	0	\\
1.18e-09	0	\\
1.28e-09	0	\\
1.38e-09	0	\\
1.49e-09	0	\\
1.59e-09	0	\\
1.69e-09	0	\\
1.8e-09	0	\\
1.9e-09	0	\\
2.01e-09	0	\\
2.11e-09	0	\\
2.21e-09	0	\\
2.32e-09	0	\\
2.42e-09	0	\\
2.52e-09	0	\\
2.63e-09	0	\\
2.73e-09	0	\\
2.83e-09	0	\\
2.93e-09	0	\\
3.04e-09	0	\\
3.14e-09	0	\\
3.24e-09	0	\\
3.34e-09	0	\\
3.45e-09	0	\\
3.55e-09	0	\\
3.65e-09	0	\\
3.75e-09	0	\\
3.86e-09	0	\\
3.96e-09	0	\\
4.06e-09	0	\\
4.16e-09	0	\\
4.27e-09	0	\\
4.37e-09	0	\\
4.47e-09	0	\\
4.57e-09	0	\\
4.68e-09	0	\\
4.78e-09	0	\\
4.89e-09	0	\\
4.99e-09	0	\\
5e-09	0	\\
};
\addplot [color=blue,solid,forget plot]
  table[row sep=crcr]{
0	0	\\
1.1e-10	0	\\
2.2e-10	0	\\
3.3e-10	0	\\
4.4e-10	0	\\
5.4e-10	0	\\
6.5e-10	0	\\
7.5e-10	0	\\
8.6e-10	0	\\
9.6e-10	0	\\
1.07e-09	0	\\
1.18e-09	0	\\
1.28e-09	0	\\
1.38e-09	0	\\
1.49e-09	0	\\
1.59e-09	0	\\
1.69e-09	0	\\
1.8e-09	0	\\
1.9e-09	0	\\
2.01e-09	0	\\
2.11e-09	0	\\
2.21e-09	0	\\
2.32e-09	0	\\
2.42e-09	0	\\
2.52e-09	0	\\
2.63e-09	0	\\
2.73e-09	0	\\
2.83e-09	0	\\
2.93e-09	0	\\
3.04e-09	0	\\
3.14e-09	0	\\
3.24e-09	0	\\
3.34e-09	0	\\
3.45e-09	0	\\
3.55e-09	0	\\
3.65e-09	0	\\
3.75e-09	0	\\
3.86e-09	0	\\
3.96e-09	0	\\
4.06e-09	0	\\
4.16e-09	0	\\
4.27e-09	0	\\
4.37e-09	0	\\
4.47e-09	0	\\
4.57e-09	0	\\
4.68e-09	0	\\
4.78e-09	0	\\
4.89e-09	0	\\
4.99e-09	0	\\
5e-09	0	\\
};
\addplot [color=black!50!green,solid,forget plot]
  table[row sep=crcr]{
0	0	\\
1.1e-10	0	\\
2.2e-10	0	\\
3.3e-10	0	\\
4.4e-10	0	\\
5.4e-10	0	\\
6.5e-10	0	\\
7.5e-10	0	\\
8.6e-10	0	\\
9.6e-10	0	\\
1.07e-09	0	\\
1.18e-09	0	\\
1.28e-09	0	\\
1.38e-09	0	\\
1.49e-09	0	\\
1.59e-09	0	\\
1.69e-09	0	\\
1.8e-09	0	\\
1.9e-09	0	\\
2.01e-09	0	\\
2.11e-09	0	\\
2.21e-09	0	\\
2.32e-09	0	\\
2.42e-09	0	\\
2.52e-09	0	\\
2.63e-09	0	\\
2.73e-09	0	\\
2.83e-09	0	\\
2.93e-09	0	\\
3.04e-09	0	\\
3.14e-09	0	\\
3.24e-09	0	\\
3.34e-09	0	\\
3.45e-09	0	\\
3.55e-09	0	\\
3.65e-09	0	\\
3.75e-09	0	\\
3.86e-09	0	\\
3.96e-09	0	\\
4.06e-09	0	\\
4.16e-09	0	\\
4.27e-09	0	\\
4.37e-09	0	\\
4.47e-09	0	\\
4.57e-09	0	\\
4.68e-09	0	\\
4.78e-09	0	\\
4.89e-09	0	\\
4.99e-09	0	\\
5e-09	0	\\
};
\addplot [color=red,solid,forget plot]
  table[row sep=crcr]{
0	0	\\
1.1e-10	0	\\
2.2e-10	0	\\
3.3e-10	0	\\
4.4e-10	0	\\
5.4e-10	0	\\
6.5e-10	0	\\
7.5e-10	0	\\
8.6e-10	0	\\
9.6e-10	0	\\
1.07e-09	0	\\
1.18e-09	0	\\
1.28e-09	0	\\
1.38e-09	0	\\
1.49e-09	0	\\
1.59e-09	0	\\
1.69e-09	0	\\
1.8e-09	0	\\
1.9e-09	0	\\
2.01e-09	0	\\
2.11e-09	0	\\
2.21e-09	0	\\
2.32e-09	0	\\
2.42e-09	0	\\
2.52e-09	0	\\
2.63e-09	0	\\
2.73e-09	0	\\
2.83e-09	0	\\
2.93e-09	0	\\
3.04e-09	0	\\
3.14e-09	0	\\
3.24e-09	0	\\
3.34e-09	0	\\
3.45e-09	0	\\
3.55e-09	0	\\
3.65e-09	0	\\
3.75e-09	0	\\
3.86e-09	0	\\
3.96e-09	0	\\
4.06e-09	0	\\
4.16e-09	0	\\
4.27e-09	0	\\
4.37e-09	0	\\
4.47e-09	0	\\
4.57e-09	0	\\
4.68e-09	0	\\
4.78e-09	0	\\
4.89e-09	0	\\
4.99e-09	0	\\
5e-09	0	\\
};
\addplot [color=mycolor1,solid,forget plot]
  table[row sep=crcr]{
0	0	\\
1.1e-10	0	\\
2.2e-10	0	\\
3.3e-10	0	\\
4.4e-10	0	\\
5.4e-10	0	\\
6.5e-10	0	\\
7.5e-10	0	\\
8.6e-10	0	\\
9.6e-10	0	\\
1.07e-09	0	\\
1.18e-09	0	\\
1.28e-09	0	\\
1.38e-09	0	\\
1.49e-09	0	\\
1.59e-09	0	\\
1.69e-09	0	\\
1.8e-09	0	\\
1.9e-09	0	\\
2.01e-09	0	\\
2.11e-09	0	\\
2.21e-09	0	\\
2.32e-09	0	\\
2.42e-09	0	\\
2.52e-09	0	\\
2.63e-09	0	\\
2.73e-09	0	\\
2.83e-09	0	\\
2.93e-09	0	\\
3.04e-09	0	\\
3.14e-09	0	\\
3.24e-09	0	\\
3.34e-09	0	\\
3.45e-09	0	\\
3.55e-09	0	\\
3.65e-09	0	\\
3.75e-09	0	\\
3.86e-09	0	\\
3.96e-09	0	\\
4.06e-09	0	\\
4.16e-09	0	\\
4.27e-09	0	\\
4.37e-09	0	\\
4.47e-09	0	\\
4.57e-09	0	\\
4.68e-09	0	\\
4.78e-09	0	\\
4.89e-09	0	\\
4.99e-09	0	\\
5e-09	0	\\
};
\addplot [color=mycolor2,solid,forget plot]
  table[row sep=crcr]{
0	0	\\
1.1e-10	0	\\
2.2e-10	0	\\
3.3e-10	0	\\
4.4e-10	0	\\
5.4e-10	0	\\
6.5e-10	0	\\
7.5e-10	0	\\
8.6e-10	0	\\
9.6e-10	0	\\
1.07e-09	0	\\
1.18e-09	0	\\
1.28e-09	0	\\
1.38e-09	0	\\
1.49e-09	0	\\
1.59e-09	0	\\
1.69e-09	0	\\
1.8e-09	0	\\
1.9e-09	0	\\
2.01e-09	0	\\
2.11e-09	0	\\
2.21e-09	0	\\
2.32e-09	0	\\
2.42e-09	0	\\
2.52e-09	0	\\
2.63e-09	0	\\
2.73e-09	0	\\
2.83e-09	0	\\
2.93e-09	0	\\
3.04e-09	0	\\
3.14e-09	0	\\
3.24e-09	0	\\
3.34e-09	0	\\
3.45e-09	0	\\
3.55e-09	0	\\
3.65e-09	0	\\
3.75e-09	0	\\
3.86e-09	0	\\
3.96e-09	0	\\
4.06e-09	0	\\
4.16e-09	0	\\
4.27e-09	0	\\
4.37e-09	0	\\
4.47e-09	0	\\
4.57e-09	0	\\
4.68e-09	0	\\
4.78e-09	0	\\
4.89e-09	0	\\
4.99e-09	0	\\
5e-09	0	\\
};
\addplot [color=mycolor3,solid,forget plot]
  table[row sep=crcr]{
0	0	\\
1.1e-10	0	\\
2.2e-10	0	\\
3.3e-10	0	\\
4.4e-10	0	\\
5.4e-10	0	\\
6.5e-10	0	\\
7.5e-10	0	\\
8.6e-10	0	\\
9.6e-10	0	\\
1.07e-09	0	\\
1.18e-09	0	\\
1.28e-09	0	\\
1.38e-09	0	\\
1.49e-09	0	\\
1.59e-09	0	\\
1.69e-09	0	\\
1.8e-09	0	\\
1.9e-09	0	\\
2.01e-09	0	\\
2.11e-09	0	\\
2.21e-09	0	\\
2.32e-09	0	\\
2.42e-09	0	\\
2.52e-09	0	\\
2.63e-09	0	\\
2.73e-09	0	\\
2.83e-09	0	\\
2.93e-09	0	\\
3.04e-09	0	\\
3.14e-09	0	\\
3.24e-09	0	\\
3.34e-09	0	\\
3.45e-09	0	\\
3.55e-09	0	\\
3.65e-09	0	\\
3.75e-09	0	\\
3.86e-09	0	\\
3.96e-09	0	\\
4.06e-09	0	\\
4.16e-09	0	\\
4.27e-09	0	\\
4.37e-09	0	\\
4.47e-09	0	\\
4.57e-09	0	\\
4.68e-09	0	\\
4.78e-09	0	\\
4.89e-09	0	\\
4.99e-09	0	\\
5e-09	0	\\
};
\addplot [color=darkgray,solid,forget plot]
  table[row sep=crcr]{
0	0	\\
1.1e-10	0	\\
2.2e-10	0	\\
3.3e-10	0	\\
4.4e-10	0	\\
5.4e-10	0	\\
6.5e-10	0	\\
7.5e-10	0	\\
8.6e-10	0	\\
9.6e-10	0	\\
1.07e-09	0	\\
1.18e-09	0	\\
1.28e-09	0	\\
1.38e-09	0	\\
1.49e-09	0	\\
1.59e-09	0	\\
1.69e-09	0	\\
1.8e-09	0	\\
1.9e-09	0	\\
2.01e-09	0	\\
2.11e-09	0	\\
2.21e-09	0	\\
2.32e-09	0	\\
2.42e-09	0	\\
2.52e-09	0	\\
2.63e-09	0	\\
2.73e-09	0	\\
2.83e-09	0	\\
2.93e-09	0	\\
3.04e-09	0	\\
3.14e-09	0	\\
3.24e-09	0	\\
3.34e-09	0	\\
3.45e-09	0	\\
3.55e-09	0	\\
3.65e-09	0	\\
3.75e-09	0	\\
3.86e-09	0	\\
3.96e-09	0	\\
4.06e-09	0	\\
4.16e-09	0	\\
4.27e-09	0	\\
4.37e-09	0	\\
4.47e-09	0	\\
4.57e-09	0	\\
4.68e-09	0	\\
4.78e-09	0	\\
4.89e-09	0	\\
4.99e-09	0	\\
5e-09	0	\\
};
\addplot [color=blue,solid,forget plot]
  table[row sep=crcr]{
0	0	\\
1.1e-10	0	\\
2.2e-10	0	\\
3.3e-10	0	\\
4.4e-10	0	\\
5.4e-10	0	\\
6.5e-10	0	\\
7.5e-10	0	\\
8.6e-10	0	\\
9.6e-10	0	\\
1.07e-09	0	\\
1.18e-09	0	\\
1.28e-09	0	\\
1.38e-09	0	\\
1.49e-09	0	\\
1.59e-09	0	\\
1.69e-09	0	\\
1.8e-09	0	\\
1.9e-09	0	\\
2.01e-09	0	\\
2.11e-09	0	\\
2.21e-09	0	\\
2.32e-09	0	\\
2.42e-09	0	\\
2.52e-09	0	\\
2.63e-09	0	\\
2.73e-09	0	\\
2.83e-09	0	\\
2.93e-09	0	\\
3.04e-09	0	\\
3.14e-09	0	\\
3.24e-09	0	\\
3.34e-09	0	\\
3.45e-09	0	\\
3.55e-09	0	\\
3.65e-09	0	\\
3.75e-09	0	\\
3.86e-09	0	\\
3.96e-09	0	\\
4.06e-09	0	\\
4.16e-09	0	\\
4.27e-09	0	\\
4.37e-09	0	\\
4.47e-09	0	\\
4.57e-09	0	\\
4.68e-09	0	\\
4.78e-09	0	\\
4.89e-09	0	\\
4.99e-09	0	\\
5e-09	0	\\
};
\addplot [color=black!50!green,solid,forget plot]
  table[row sep=crcr]{
0	0	\\
1.1e-10	0	\\
2.2e-10	0	\\
3.3e-10	0	\\
4.4e-10	0	\\
5.4e-10	0	\\
6.5e-10	0	\\
7.5e-10	0	\\
8.6e-10	0	\\
9.6e-10	0	\\
1.07e-09	0	\\
1.18e-09	0	\\
1.28e-09	0	\\
1.38e-09	0	\\
1.49e-09	0	\\
1.59e-09	0	\\
1.69e-09	0	\\
1.8e-09	0	\\
1.9e-09	0	\\
2.01e-09	0	\\
2.11e-09	0	\\
2.21e-09	0	\\
2.32e-09	0	\\
2.42e-09	0	\\
2.52e-09	0	\\
2.63e-09	0	\\
2.73e-09	0	\\
2.83e-09	0	\\
2.93e-09	0	\\
3.04e-09	0	\\
3.14e-09	0	\\
3.24e-09	0	\\
3.34e-09	0	\\
3.45e-09	0	\\
3.55e-09	0	\\
3.65e-09	0	\\
3.75e-09	0	\\
3.86e-09	0	\\
3.96e-09	0	\\
4.06e-09	0	\\
4.16e-09	0	\\
4.27e-09	0	\\
4.37e-09	0	\\
4.47e-09	0	\\
4.57e-09	0	\\
4.68e-09	0	\\
4.78e-09	0	\\
4.89e-09	0	\\
4.99e-09	0	\\
5e-09	0	\\
};
\addplot [color=red,solid,forget plot]
  table[row sep=crcr]{
0	0	\\
1.1e-10	0	\\
2.2e-10	0	\\
3.3e-10	0	\\
4.4e-10	0	\\
5.4e-10	0	\\
6.5e-10	0	\\
7.5e-10	0	\\
8.6e-10	0	\\
9.6e-10	0	\\
1.07e-09	0	\\
1.18e-09	0	\\
1.28e-09	0	\\
1.38e-09	0	\\
1.49e-09	0	\\
1.59e-09	0	\\
1.69e-09	0	\\
1.8e-09	0	\\
1.9e-09	0	\\
2.01e-09	0	\\
2.11e-09	0	\\
2.21e-09	0	\\
2.32e-09	0	\\
2.42e-09	0	\\
2.52e-09	0	\\
2.63e-09	0	\\
2.73e-09	0	\\
2.83e-09	0	\\
2.93e-09	0	\\
3.04e-09	0	\\
3.14e-09	0	\\
3.24e-09	0	\\
3.34e-09	0	\\
3.45e-09	0	\\
3.55e-09	0	\\
3.65e-09	0	\\
3.75e-09	0	\\
3.86e-09	0	\\
3.96e-09	0	\\
4.06e-09	0	\\
4.16e-09	0	\\
4.27e-09	0	\\
4.37e-09	0	\\
4.47e-09	0	\\
4.57e-09	0	\\
4.68e-09	0	\\
4.78e-09	0	\\
4.89e-09	0	\\
4.99e-09	0	\\
5e-09	0	\\
};
\addplot [color=mycolor1,solid,forget plot]
  table[row sep=crcr]{
0	0	\\
1.1e-10	0	\\
2.2e-10	0	\\
3.3e-10	0	\\
4.4e-10	0	\\
5.4e-10	0	\\
6.5e-10	0	\\
7.5e-10	0	\\
8.6e-10	0	\\
9.6e-10	0	\\
1.07e-09	0	\\
1.18e-09	0	\\
1.28e-09	0	\\
1.38e-09	0	\\
1.49e-09	0	\\
1.59e-09	0	\\
1.69e-09	0	\\
1.8e-09	0	\\
1.9e-09	0	\\
2.01e-09	0	\\
2.11e-09	0	\\
2.21e-09	0	\\
2.32e-09	0	\\
2.42e-09	0	\\
2.52e-09	0	\\
2.63e-09	0	\\
2.73e-09	0	\\
2.83e-09	0	\\
2.93e-09	0	\\
3.04e-09	0	\\
3.14e-09	0	\\
3.24e-09	0	\\
3.34e-09	0	\\
3.45e-09	0	\\
3.55e-09	0	\\
3.65e-09	0	\\
3.75e-09	0	\\
3.86e-09	0	\\
3.96e-09	0	\\
4.06e-09	0	\\
4.16e-09	0	\\
4.27e-09	0	\\
4.37e-09	0	\\
4.47e-09	0	\\
4.57e-09	0	\\
4.68e-09	0	\\
4.78e-09	0	\\
4.89e-09	0	\\
4.99e-09	0	\\
5e-09	0	\\
};
\addplot [color=mycolor2,solid,forget plot]
  table[row sep=crcr]{
0	0	\\
1.1e-10	0	\\
2.2e-10	0	\\
3.3e-10	0	\\
4.4e-10	0	\\
5.4e-10	0	\\
6.5e-10	0	\\
7.5e-10	0	\\
8.6e-10	0	\\
9.6e-10	0	\\
1.07e-09	0	\\
1.18e-09	0	\\
1.28e-09	0	\\
1.38e-09	0	\\
1.49e-09	0	\\
1.59e-09	0	\\
1.69e-09	0	\\
1.8e-09	0	\\
1.9e-09	0	\\
2.01e-09	0	\\
2.11e-09	0	\\
2.21e-09	0	\\
2.32e-09	0	\\
2.42e-09	0	\\
2.52e-09	0	\\
2.63e-09	0	\\
2.73e-09	0	\\
2.83e-09	0	\\
2.93e-09	0	\\
3.04e-09	0	\\
3.14e-09	0	\\
3.24e-09	0	\\
3.34e-09	0	\\
3.45e-09	0	\\
3.55e-09	0	\\
3.65e-09	0	\\
3.75e-09	0	\\
3.86e-09	0	\\
3.96e-09	0	\\
4.06e-09	0	\\
4.16e-09	0	\\
4.27e-09	0	\\
4.37e-09	0	\\
4.47e-09	0	\\
4.57e-09	0	\\
4.68e-09	0	\\
4.78e-09	0	\\
4.89e-09	0	\\
4.99e-09	0	\\
5e-09	0	\\
};
\addplot [color=mycolor3,solid,forget plot]
  table[row sep=crcr]{
0	0	\\
1.1e-10	0	\\
2.2e-10	0	\\
3.3e-10	0	\\
4.4e-10	0	\\
5.4e-10	0	\\
6.5e-10	0	\\
7.5e-10	0	\\
8.6e-10	0	\\
9.6e-10	0	\\
1.07e-09	0	\\
1.18e-09	0	\\
1.28e-09	0	\\
1.38e-09	0	\\
1.49e-09	0	\\
1.59e-09	0	\\
1.69e-09	0	\\
1.8e-09	0	\\
1.9e-09	0	\\
2.01e-09	0	\\
2.11e-09	0	\\
2.21e-09	0	\\
2.32e-09	0	\\
2.42e-09	0	\\
2.52e-09	0	\\
2.63e-09	0	\\
2.73e-09	0	\\
2.83e-09	0	\\
2.93e-09	0	\\
3.04e-09	0	\\
3.14e-09	0	\\
3.24e-09	0	\\
3.34e-09	0	\\
3.45e-09	0	\\
3.55e-09	0	\\
3.65e-09	0	\\
3.75e-09	0	\\
3.86e-09	0	\\
3.96e-09	0	\\
4.06e-09	0	\\
4.16e-09	0	\\
4.27e-09	0	\\
4.37e-09	0	\\
4.47e-09	0	\\
4.57e-09	0	\\
4.68e-09	0	\\
4.78e-09	0	\\
4.89e-09	0	\\
4.99e-09	0	\\
5e-09	0	\\
};
\addplot [color=darkgray,solid,forget plot]
  table[row sep=crcr]{
0	0	\\
1.1e-10	0	\\
2.2e-10	0	\\
3.3e-10	0	\\
4.4e-10	0	\\
5.4e-10	0	\\
6.5e-10	0	\\
7.5e-10	0	\\
8.6e-10	0	\\
9.6e-10	0	\\
1.07e-09	0	\\
1.18e-09	0	\\
1.28e-09	0	\\
1.38e-09	0	\\
1.49e-09	0	\\
1.59e-09	0	\\
1.69e-09	0	\\
1.8e-09	0	\\
1.9e-09	0	\\
2.01e-09	0	\\
2.11e-09	0	\\
2.21e-09	0	\\
2.32e-09	0	\\
2.42e-09	0	\\
2.52e-09	0	\\
2.63e-09	0	\\
2.73e-09	0	\\
2.83e-09	0	\\
2.93e-09	0	\\
3.04e-09	0	\\
3.14e-09	0	\\
3.24e-09	0	\\
3.34e-09	0	\\
3.45e-09	0	\\
3.55e-09	0	\\
3.65e-09	0	\\
3.75e-09	0	\\
3.86e-09	0	\\
3.96e-09	0	\\
4.06e-09	0	\\
4.16e-09	0	\\
4.27e-09	0	\\
4.37e-09	0	\\
4.47e-09	0	\\
4.57e-09	0	\\
4.68e-09	0	\\
4.78e-09	0	\\
4.89e-09	0	\\
4.99e-09	0	\\
5e-09	0	\\
};
\addplot [color=blue,solid,forget plot]
  table[row sep=crcr]{
0	0	\\
1.1e-10	0	\\
2.2e-10	0	\\
3.3e-10	0	\\
4.4e-10	0	\\
5.4e-10	0	\\
6.5e-10	0	\\
7.5e-10	0	\\
8.6e-10	0	\\
9.6e-10	0	\\
1.07e-09	0	\\
1.18e-09	0	\\
1.28e-09	0	\\
1.38e-09	0	\\
1.49e-09	0	\\
1.59e-09	0	\\
1.69e-09	0	\\
1.8e-09	0	\\
1.9e-09	0	\\
2.01e-09	0	\\
2.11e-09	0	\\
2.21e-09	0	\\
2.32e-09	0	\\
2.42e-09	0	\\
2.52e-09	0	\\
2.63e-09	0	\\
2.73e-09	0	\\
2.83e-09	0	\\
2.93e-09	0	\\
3.04e-09	0	\\
3.14e-09	0	\\
3.24e-09	0	\\
3.34e-09	0	\\
3.45e-09	0	\\
3.55e-09	0	\\
3.65e-09	0	\\
3.75e-09	0	\\
3.86e-09	0	\\
3.96e-09	0	\\
4.06e-09	0	\\
4.16e-09	0	\\
4.27e-09	0	\\
4.37e-09	0	\\
4.47e-09	0	\\
4.57e-09	0	\\
4.68e-09	0	\\
4.78e-09	0	\\
4.89e-09	0	\\
4.99e-09	0	\\
5e-09	0	\\
};
\addplot [color=black!50!green,solid,forget plot]
  table[row sep=crcr]{
0	0	\\
1.1e-10	0	\\
2.2e-10	0	\\
3.3e-10	0	\\
4.4e-10	0	\\
5.4e-10	0	\\
6.5e-10	0	\\
7.5e-10	0	\\
8.6e-10	0	\\
9.6e-10	0	\\
1.07e-09	0	\\
1.18e-09	0	\\
1.28e-09	0	\\
1.38e-09	0	\\
1.49e-09	0	\\
1.59e-09	0	\\
1.69e-09	0	\\
1.8e-09	0	\\
1.9e-09	0	\\
2.01e-09	0	\\
2.11e-09	0	\\
2.21e-09	0	\\
2.32e-09	0	\\
2.42e-09	0	\\
2.52e-09	0	\\
2.63e-09	0	\\
2.73e-09	0	\\
2.83e-09	0	\\
2.93e-09	0	\\
3.04e-09	0	\\
3.14e-09	0	\\
3.24e-09	0	\\
3.34e-09	0	\\
3.45e-09	0	\\
3.55e-09	0	\\
3.65e-09	0	\\
3.75e-09	0	\\
3.86e-09	0	\\
3.96e-09	0	\\
4.06e-09	0	\\
4.16e-09	0	\\
4.27e-09	0	\\
4.37e-09	0	\\
4.47e-09	0	\\
4.57e-09	0	\\
4.68e-09	0	\\
4.78e-09	0	\\
4.89e-09	0	\\
4.99e-09	0	\\
5e-09	0	\\
};
\addplot [color=red,solid,forget plot]
  table[row sep=crcr]{
0	0	\\
1.1e-10	0	\\
2.2e-10	0	\\
3.3e-10	0	\\
4.4e-10	0	\\
5.4e-10	0	\\
6.5e-10	0	\\
7.5e-10	0	\\
8.6e-10	0	\\
9.6e-10	0	\\
1.07e-09	0	\\
1.18e-09	0	\\
1.28e-09	0	\\
1.38e-09	0	\\
1.49e-09	0	\\
1.59e-09	0	\\
1.69e-09	0	\\
1.8e-09	0	\\
1.9e-09	0	\\
2.01e-09	0	\\
2.11e-09	0	\\
2.21e-09	0	\\
2.32e-09	0	\\
2.42e-09	0	\\
2.52e-09	0	\\
2.63e-09	0	\\
2.73e-09	0	\\
2.83e-09	0	\\
2.93e-09	0	\\
3.04e-09	0	\\
3.14e-09	0	\\
3.24e-09	0	\\
3.34e-09	0	\\
3.45e-09	0	\\
3.55e-09	0	\\
3.65e-09	0	\\
3.75e-09	0	\\
3.86e-09	0	\\
3.96e-09	0	\\
4.06e-09	0	\\
4.16e-09	0	\\
4.27e-09	0	\\
4.37e-09	0	\\
4.47e-09	0	\\
4.57e-09	0	\\
4.68e-09	0	\\
4.78e-09	0	\\
4.89e-09	0	\\
4.99e-09	0	\\
5e-09	0	\\
};
\addplot [color=mycolor1,solid,forget plot]
  table[row sep=crcr]{
0	0	\\
1.1e-10	0	\\
2.2e-10	0	\\
3.3e-10	0	\\
4.4e-10	0	\\
5.4e-10	0	\\
6.5e-10	0	\\
7.5e-10	0	\\
8.6e-10	0	\\
9.6e-10	0	\\
1.07e-09	0	\\
1.18e-09	0	\\
1.28e-09	0	\\
1.38e-09	0	\\
1.49e-09	0	\\
1.59e-09	0	\\
1.69e-09	0	\\
1.8e-09	0	\\
1.9e-09	0	\\
2.01e-09	0	\\
2.11e-09	0	\\
2.21e-09	0	\\
2.32e-09	0	\\
2.42e-09	0	\\
2.52e-09	0	\\
2.63e-09	0	\\
2.73e-09	0	\\
2.83e-09	0	\\
2.93e-09	0	\\
3.04e-09	0	\\
3.14e-09	0	\\
3.24e-09	0	\\
3.34e-09	0	\\
3.45e-09	0	\\
3.55e-09	0	\\
3.65e-09	0	\\
3.75e-09	0	\\
3.86e-09	0	\\
3.96e-09	0	\\
4.06e-09	0	\\
4.16e-09	0	\\
4.27e-09	0	\\
4.37e-09	0	\\
4.47e-09	0	\\
4.57e-09	0	\\
4.68e-09	0	\\
4.78e-09	0	\\
4.89e-09	0	\\
4.99e-09	0	\\
5e-09	0	\\
};
\addplot [color=mycolor2,solid,forget plot]
  table[row sep=crcr]{
0	0	\\
1.1e-10	0	\\
2.2e-10	0	\\
3.3e-10	0	\\
4.4e-10	0	\\
5.4e-10	0	\\
6.5e-10	0	\\
7.5e-10	0	\\
8.6e-10	0	\\
9.6e-10	0	\\
1.07e-09	0	\\
1.18e-09	0	\\
1.28e-09	0	\\
1.38e-09	0	\\
1.49e-09	0	\\
1.59e-09	0	\\
1.69e-09	0	\\
1.8e-09	0	\\
1.9e-09	0	\\
2.01e-09	0	\\
2.11e-09	0	\\
2.21e-09	0	\\
2.32e-09	0	\\
2.42e-09	0	\\
2.52e-09	0	\\
2.63e-09	0	\\
2.73e-09	0	\\
2.83e-09	0	\\
2.93e-09	0	\\
3.04e-09	0	\\
3.14e-09	0	\\
3.24e-09	0	\\
3.34e-09	0	\\
3.45e-09	0	\\
3.55e-09	0	\\
3.65e-09	0	\\
3.75e-09	0	\\
3.86e-09	0	\\
3.96e-09	0	\\
4.06e-09	0	\\
4.16e-09	0	\\
4.27e-09	0	\\
4.37e-09	0	\\
4.47e-09	0	\\
4.57e-09	0	\\
4.68e-09	0	\\
4.78e-09	0	\\
4.89e-09	0	\\
4.99e-09	0	\\
5e-09	0	\\
};
\addplot [color=mycolor3,solid,forget plot]
  table[row sep=crcr]{
0	0	\\
1.1e-10	0	\\
2.2e-10	0	\\
3.3e-10	0	\\
4.4e-10	0	\\
5.4e-10	0	\\
6.5e-10	0	\\
7.5e-10	0	\\
8.6e-10	0	\\
9.6e-10	0	\\
1.07e-09	0	\\
1.18e-09	0	\\
1.28e-09	0	\\
1.38e-09	0	\\
1.49e-09	0	\\
1.59e-09	0	\\
1.69e-09	0	\\
1.8e-09	0	\\
1.9e-09	0	\\
2.01e-09	0	\\
2.11e-09	0	\\
2.21e-09	0	\\
2.32e-09	0	\\
2.42e-09	0	\\
2.52e-09	0	\\
2.63e-09	0	\\
2.73e-09	0	\\
2.83e-09	0	\\
2.93e-09	0	\\
3.04e-09	0	\\
3.14e-09	0	\\
3.24e-09	0	\\
3.34e-09	0	\\
3.45e-09	0	\\
3.55e-09	0	\\
3.65e-09	0	\\
3.75e-09	0	\\
3.86e-09	0	\\
3.96e-09	0	\\
4.06e-09	0	\\
4.16e-09	0	\\
4.27e-09	0	\\
4.37e-09	0	\\
4.47e-09	0	\\
4.57e-09	0	\\
4.68e-09	0	\\
4.78e-09	0	\\
4.89e-09	0	\\
4.99e-09	0	\\
5e-09	0	\\
};
\addplot [color=darkgray,solid,forget plot]
  table[row sep=crcr]{
0	0	\\
1.1e-10	0	\\
2.2e-10	0	\\
3.3e-10	0	\\
4.4e-10	0	\\
5.4e-10	0	\\
6.5e-10	0	\\
7.5e-10	0	\\
8.6e-10	0	\\
9.6e-10	0	\\
1.07e-09	0	\\
1.18e-09	0	\\
1.28e-09	0	\\
1.38e-09	0	\\
1.49e-09	0	\\
1.59e-09	0	\\
1.69e-09	0	\\
1.8e-09	0	\\
1.9e-09	0	\\
2.01e-09	0	\\
2.11e-09	0	\\
2.21e-09	0	\\
2.32e-09	0	\\
2.42e-09	0	\\
2.52e-09	0	\\
2.63e-09	0	\\
2.73e-09	0	\\
2.83e-09	0	\\
2.93e-09	0	\\
3.04e-09	0	\\
3.14e-09	0	\\
3.24e-09	0	\\
3.34e-09	0	\\
3.45e-09	0	\\
3.55e-09	0	\\
3.65e-09	0	\\
3.75e-09	0	\\
3.86e-09	0	\\
3.96e-09	0	\\
4.06e-09	0	\\
4.16e-09	0	\\
4.27e-09	0	\\
4.37e-09	0	\\
4.47e-09	0	\\
4.57e-09	0	\\
4.68e-09	0	\\
4.78e-09	0	\\
4.89e-09	0	\\
4.99e-09	0	\\
5e-09	0	\\
};
\addplot [color=blue,solid,forget plot]
  table[row sep=crcr]{
0	0	\\
1.1e-10	0	\\
2.2e-10	0	\\
3.3e-10	0	\\
4.4e-10	0	\\
5.4e-10	0	\\
6.5e-10	0	\\
7.5e-10	0	\\
8.6e-10	0	\\
9.6e-10	0	\\
1.07e-09	0	\\
1.18e-09	0	\\
1.28e-09	0	\\
1.38e-09	0	\\
1.49e-09	0	\\
1.59e-09	0	\\
1.69e-09	0	\\
1.8e-09	0	\\
1.9e-09	0	\\
2.01e-09	0	\\
2.11e-09	0	\\
2.21e-09	0	\\
2.32e-09	0	\\
2.42e-09	0	\\
2.52e-09	0	\\
2.63e-09	0	\\
2.73e-09	0	\\
2.83e-09	0	\\
2.93e-09	0	\\
3.04e-09	0	\\
3.14e-09	0	\\
3.24e-09	0	\\
3.34e-09	0	\\
3.45e-09	0	\\
3.55e-09	0	\\
3.65e-09	0	\\
3.75e-09	0	\\
3.86e-09	0	\\
3.96e-09	0	\\
4.06e-09	0	\\
4.16e-09	0	\\
4.27e-09	0	\\
4.37e-09	0	\\
4.47e-09	0	\\
4.57e-09	0	\\
4.68e-09	0	\\
4.78e-09	0	\\
4.89e-09	0	\\
4.99e-09	0	\\
5e-09	0	\\
};
\addplot [color=black!50!green,solid,forget plot]
  table[row sep=crcr]{
0	0	\\
1.1e-10	0	\\
2.2e-10	0	\\
3.3e-10	0	\\
4.4e-10	0	\\
5.4e-10	0	\\
6.5e-10	0	\\
7.5e-10	0	\\
8.6e-10	0	\\
9.6e-10	0	\\
1.07e-09	0	\\
1.18e-09	0	\\
1.28e-09	0	\\
1.38e-09	0	\\
1.49e-09	0	\\
1.59e-09	0	\\
1.69e-09	0	\\
1.8e-09	0	\\
1.9e-09	0	\\
2.01e-09	0	\\
2.11e-09	0	\\
2.21e-09	0	\\
2.32e-09	0	\\
2.42e-09	0	\\
2.52e-09	0	\\
2.63e-09	0	\\
2.73e-09	0	\\
2.83e-09	0	\\
2.93e-09	0	\\
3.04e-09	0	\\
3.14e-09	0	\\
3.24e-09	0	\\
3.34e-09	0	\\
3.45e-09	0	\\
3.55e-09	0	\\
3.65e-09	0	\\
3.75e-09	0	\\
3.86e-09	0	\\
3.96e-09	0	\\
4.06e-09	0	\\
4.16e-09	0	\\
4.27e-09	0	\\
4.37e-09	0	\\
4.47e-09	0	\\
4.57e-09	0	\\
4.68e-09	0	\\
4.78e-09	0	\\
4.89e-09	0	\\
4.99e-09	0	\\
5e-09	0	\\
};
\addplot [color=red,solid,forget plot]
  table[row sep=crcr]{
0	0	\\
1.1e-10	0	\\
2.2e-10	0	\\
3.3e-10	0	\\
4.4e-10	0	\\
5.4e-10	0	\\
6.5e-10	0	\\
7.5e-10	0	\\
8.6e-10	0	\\
9.6e-10	0	\\
1.07e-09	0	\\
1.18e-09	0	\\
1.28e-09	0	\\
1.38e-09	0	\\
1.49e-09	0	\\
1.59e-09	0	\\
1.69e-09	0	\\
1.8e-09	0	\\
1.9e-09	0	\\
2.01e-09	0	\\
2.11e-09	0	\\
2.21e-09	0	\\
2.32e-09	0	\\
2.42e-09	0	\\
2.52e-09	0	\\
2.63e-09	0	\\
2.73e-09	0	\\
2.83e-09	0	\\
2.93e-09	0	\\
3.04e-09	0	\\
3.14e-09	0	\\
3.24e-09	0	\\
3.34e-09	0	\\
3.45e-09	0	\\
3.55e-09	0	\\
3.65e-09	0	\\
3.75e-09	0	\\
3.86e-09	0	\\
3.96e-09	0	\\
4.06e-09	0	\\
4.16e-09	0	\\
4.27e-09	0	\\
4.37e-09	0	\\
4.47e-09	0	\\
4.57e-09	0	\\
4.68e-09	0	\\
4.78e-09	0	\\
4.89e-09	0	\\
4.99e-09	0	\\
5e-09	0	\\
};
\addplot [color=mycolor1,solid,forget plot]
  table[row sep=crcr]{
0	0	\\
1.1e-10	0	\\
2.2e-10	0	\\
3.3e-10	0	\\
4.4e-10	0	\\
5.4e-10	0	\\
6.5e-10	0	\\
7.5e-10	0	\\
8.6e-10	0	\\
9.6e-10	0	\\
1.07e-09	0	\\
1.18e-09	0	\\
1.28e-09	0	\\
1.38e-09	0	\\
1.49e-09	0	\\
1.59e-09	0	\\
1.69e-09	0	\\
1.8e-09	0	\\
1.9e-09	0	\\
2.01e-09	0	\\
2.11e-09	0	\\
2.21e-09	0	\\
2.32e-09	0	\\
2.42e-09	0	\\
2.52e-09	0	\\
2.63e-09	0	\\
2.73e-09	0	\\
2.83e-09	0	\\
2.93e-09	0	\\
3.04e-09	0	\\
3.14e-09	0	\\
3.24e-09	0	\\
3.34e-09	0	\\
3.45e-09	0	\\
3.55e-09	0	\\
3.65e-09	0	\\
3.75e-09	0	\\
3.86e-09	0	\\
3.96e-09	0	\\
4.06e-09	0	\\
4.16e-09	0	\\
4.27e-09	0	\\
4.37e-09	0	\\
4.47e-09	0	\\
4.57e-09	0	\\
4.68e-09	0	\\
4.78e-09	0	\\
4.89e-09	0	\\
4.99e-09	0	\\
5e-09	0	\\
};
\addplot [color=mycolor2,solid,forget plot]
  table[row sep=crcr]{
0	0	\\
1.1e-10	0	\\
2.2e-10	0	\\
3.3e-10	0	\\
4.4e-10	0	\\
5.4e-10	0	\\
6.5e-10	0	\\
7.5e-10	0	\\
8.6e-10	0	\\
9.6e-10	0	\\
1.07e-09	0	\\
1.18e-09	0	\\
1.28e-09	0	\\
1.38e-09	0	\\
1.49e-09	0	\\
1.59e-09	0	\\
1.69e-09	0	\\
1.8e-09	0	\\
1.9e-09	0	\\
2.01e-09	0	\\
2.11e-09	0	\\
2.21e-09	0	\\
2.32e-09	0	\\
2.42e-09	0	\\
2.52e-09	0	\\
2.63e-09	0	\\
2.73e-09	0	\\
2.83e-09	0	\\
2.93e-09	0	\\
3.04e-09	0	\\
3.14e-09	0	\\
3.24e-09	0	\\
3.34e-09	0	\\
3.45e-09	0	\\
3.55e-09	0	\\
3.65e-09	0	\\
3.75e-09	0	\\
3.86e-09	0	\\
3.96e-09	0	\\
4.06e-09	0	\\
4.16e-09	0	\\
4.27e-09	0	\\
4.37e-09	0	\\
4.47e-09	0	\\
4.57e-09	0	\\
4.68e-09	0	\\
4.78e-09	0	\\
4.89e-09	0	\\
4.99e-09	0	\\
5e-09	0	\\
};
\addplot [color=mycolor3,solid,forget plot]
  table[row sep=crcr]{
0	0	\\
1.1e-10	0	\\
2.2e-10	0	\\
3.3e-10	0	\\
4.4e-10	0	\\
5.4e-10	0	\\
6.5e-10	0	\\
7.5e-10	0	\\
8.6e-10	0	\\
9.6e-10	0	\\
1.07e-09	0	\\
1.18e-09	0	\\
1.28e-09	0	\\
1.38e-09	0	\\
1.49e-09	0	\\
1.59e-09	0	\\
1.69e-09	0	\\
1.8e-09	0	\\
1.9e-09	0	\\
2.01e-09	0	\\
2.11e-09	0	\\
2.21e-09	0	\\
2.32e-09	0	\\
2.42e-09	0	\\
2.52e-09	0	\\
2.63e-09	0	\\
2.73e-09	0	\\
2.83e-09	0	\\
2.93e-09	0	\\
3.04e-09	0	\\
3.14e-09	0	\\
3.24e-09	0	\\
3.34e-09	0	\\
3.45e-09	0	\\
3.55e-09	0	\\
3.65e-09	0	\\
3.75e-09	0	\\
3.86e-09	0	\\
3.96e-09	0	\\
4.06e-09	0	\\
4.16e-09	0	\\
4.27e-09	0	\\
4.37e-09	0	\\
4.47e-09	0	\\
4.57e-09	0	\\
4.68e-09	0	\\
4.78e-09	0	\\
4.89e-09	0	\\
4.99e-09	0	\\
5e-09	0	\\
};
\addplot [color=darkgray,solid,forget plot]
  table[row sep=crcr]{
0	0	\\
1.1e-10	0	\\
2.2e-10	0	\\
3.3e-10	0	\\
4.4e-10	0	\\
5.4e-10	0	\\
6.5e-10	0	\\
7.5e-10	0	\\
8.6e-10	0	\\
9.6e-10	0	\\
1.07e-09	0	\\
1.18e-09	0	\\
1.28e-09	0	\\
1.38e-09	0	\\
1.49e-09	0	\\
1.59e-09	0	\\
1.69e-09	0	\\
1.8e-09	0	\\
1.9e-09	0	\\
2.01e-09	0	\\
2.11e-09	0	\\
2.21e-09	0	\\
2.32e-09	0	\\
2.42e-09	0	\\
2.52e-09	0	\\
2.63e-09	0	\\
2.73e-09	0	\\
2.83e-09	0	\\
2.93e-09	0	\\
3.04e-09	0	\\
3.14e-09	0	\\
3.24e-09	0	\\
3.34e-09	0	\\
3.45e-09	0	\\
3.55e-09	0	\\
3.65e-09	0	\\
3.75e-09	0	\\
3.86e-09	0	\\
3.96e-09	0	\\
4.06e-09	0	\\
4.16e-09	0	\\
4.27e-09	0	\\
4.37e-09	0	\\
4.47e-09	0	\\
4.57e-09	0	\\
4.68e-09	0	\\
4.78e-09	0	\\
4.89e-09	0	\\
4.99e-09	0	\\
5e-09	0	\\
};
\addplot [color=blue,solid,forget plot]
  table[row sep=crcr]{
0	0	\\
1.1e-10	0	\\
2.2e-10	0	\\
3.3e-10	0	\\
4.4e-10	0	\\
5.4e-10	0	\\
6.5e-10	0	\\
7.5e-10	0	\\
8.6e-10	0	\\
9.6e-10	0	\\
1.07e-09	0	\\
1.18e-09	0	\\
1.28e-09	0	\\
1.38e-09	0	\\
1.49e-09	0	\\
1.59e-09	0	\\
1.69e-09	0	\\
1.8e-09	0	\\
1.9e-09	0	\\
2.01e-09	0	\\
2.11e-09	0	\\
2.21e-09	0	\\
2.32e-09	0	\\
2.42e-09	0	\\
2.52e-09	0	\\
2.63e-09	0	\\
2.73e-09	0	\\
2.83e-09	0	\\
2.93e-09	0	\\
3.04e-09	0	\\
3.14e-09	0	\\
3.24e-09	0	\\
3.34e-09	0	\\
3.45e-09	0	\\
3.55e-09	0	\\
3.65e-09	0	\\
3.75e-09	0	\\
3.86e-09	0	\\
3.96e-09	0	\\
4.06e-09	0	\\
4.16e-09	0	\\
4.27e-09	0	\\
4.37e-09	0	\\
4.47e-09	0	\\
4.57e-09	0	\\
4.68e-09	0	\\
4.78e-09	0	\\
4.89e-09	0	\\
4.99e-09	0	\\
5e-09	0	\\
};
\addplot [color=black!50!green,solid,forget plot]
  table[row sep=crcr]{
0	0	\\
1.1e-10	0	\\
2.2e-10	0	\\
3.3e-10	0	\\
4.4e-10	0	\\
5.4e-10	0	\\
6.5e-10	0	\\
7.5e-10	0	\\
8.6e-10	0	\\
9.6e-10	0	\\
1.07e-09	0	\\
1.18e-09	0	\\
1.28e-09	0	\\
1.38e-09	0	\\
1.49e-09	0	\\
1.59e-09	0	\\
1.69e-09	0	\\
1.8e-09	0	\\
1.9e-09	0	\\
2.01e-09	0	\\
2.11e-09	0	\\
2.21e-09	0	\\
2.32e-09	0	\\
2.42e-09	0	\\
2.52e-09	0	\\
2.63e-09	0	\\
2.73e-09	0	\\
2.83e-09	0	\\
2.93e-09	0	\\
3.04e-09	0	\\
3.14e-09	0	\\
3.24e-09	0	\\
3.34e-09	0	\\
3.45e-09	0	\\
3.55e-09	0	\\
3.65e-09	0	\\
3.75e-09	0	\\
3.86e-09	0	\\
3.96e-09	0	\\
4.06e-09	0	\\
4.16e-09	0	\\
4.27e-09	0	\\
4.37e-09	0	\\
4.47e-09	0	\\
4.57e-09	0	\\
4.68e-09	0	\\
4.78e-09	0	\\
4.89e-09	0	\\
4.99e-09	0	\\
5e-09	0	\\
};
\addplot [color=red,solid,forget plot]
  table[row sep=crcr]{
0	0	\\
1.1e-10	0	\\
2.2e-10	0	\\
3.3e-10	0	\\
4.4e-10	0	\\
5.4e-10	0	\\
6.5e-10	0	\\
7.5e-10	0	\\
8.6e-10	0	\\
9.6e-10	0	\\
1.07e-09	0	\\
1.18e-09	0	\\
1.28e-09	0	\\
1.38e-09	0	\\
1.49e-09	0	\\
1.59e-09	0	\\
1.69e-09	0	\\
1.8e-09	0	\\
1.9e-09	0	\\
2.01e-09	0	\\
2.11e-09	0	\\
2.21e-09	0	\\
2.32e-09	0	\\
2.42e-09	0	\\
2.52e-09	0	\\
2.63e-09	0	\\
2.73e-09	0	\\
2.83e-09	0	\\
2.93e-09	0	\\
3.04e-09	0	\\
3.14e-09	0	\\
3.24e-09	0	\\
3.34e-09	0	\\
3.45e-09	0	\\
3.55e-09	0	\\
3.65e-09	0	\\
3.75e-09	0	\\
3.86e-09	0	\\
3.96e-09	0	\\
4.06e-09	0	\\
4.16e-09	0	\\
4.27e-09	0	\\
4.37e-09	0	\\
4.47e-09	0	\\
4.57e-09	0	\\
4.68e-09	0	\\
4.78e-09	0	\\
4.89e-09	0	\\
4.99e-09	0	\\
5e-09	0	\\
};
\addplot [color=mycolor1,solid,forget plot]
  table[row sep=crcr]{
0	0	\\
1.1e-10	0	\\
2.2e-10	0	\\
3.3e-10	0	\\
4.4e-10	0	\\
5.4e-10	0	\\
6.5e-10	0	\\
7.5e-10	0	\\
8.6e-10	0	\\
9.6e-10	0	\\
1.07e-09	0	\\
1.18e-09	0	\\
1.28e-09	0	\\
1.38e-09	0	\\
1.49e-09	0	\\
1.59e-09	0	\\
1.69e-09	0	\\
1.8e-09	0	\\
1.9e-09	0	\\
2.01e-09	0	\\
2.11e-09	0	\\
2.21e-09	0	\\
2.32e-09	0	\\
2.42e-09	0	\\
2.52e-09	0	\\
2.63e-09	0	\\
2.73e-09	0	\\
2.83e-09	0	\\
2.93e-09	0	\\
3.04e-09	0	\\
3.14e-09	0	\\
3.24e-09	0	\\
3.34e-09	0	\\
3.45e-09	0	\\
3.55e-09	0	\\
3.65e-09	0	\\
3.75e-09	0	\\
3.86e-09	0	\\
3.96e-09	0	\\
4.06e-09	0	\\
4.16e-09	0	\\
4.27e-09	0	\\
4.37e-09	0	\\
4.47e-09	0	\\
4.57e-09	0	\\
4.68e-09	0	\\
4.78e-09	0	\\
4.89e-09	0	\\
4.99e-09	0	\\
5e-09	0	\\
};
\addplot [color=mycolor2,solid,forget plot]
  table[row sep=crcr]{
0	0	\\
1.1e-10	0	\\
2.2e-10	0	\\
3.3e-10	0	\\
4.4e-10	0	\\
5.4e-10	0	\\
6.5e-10	0	\\
7.5e-10	0	\\
8.6e-10	0	\\
9.6e-10	0	\\
1.07e-09	0	\\
1.18e-09	0	\\
1.28e-09	0	\\
1.38e-09	0	\\
1.49e-09	0	\\
1.59e-09	0	\\
1.69e-09	0	\\
1.8e-09	0	\\
1.9e-09	0	\\
2.01e-09	0	\\
2.11e-09	0	\\
2.21e-09	0	\\
2.32e-09	0	\\
2.42e-09	0	\\
2.52e-09	0	\\
2.63e-09	0	\\
2.73e-09	0	\\
2.83e-09	0	\\
2.93e-09	0	\\
3.04e-09	0	\\
3.14e-09	0	\\
3.24e-09	0	\\
3.34e-09	0	\\
3.45e-09	0	\\
3.55e-09	0	\\
3.65e-09	0	\\
3.75e-09	0	\\
3.86e-09	0	\\
3.96e-09	0	\\
4.06e-09	0	\\
4.16e-09	0	\\
4.27e-09	0	\\
4.37e-09	0	\\
4.47e-09	0	\\
4.57e-09	0	\\
4.68e-09	0	\\
4.78e-09	0	\\
4.89e-09	0	\\
4.99e-09	0	\\
5e-09	0	\\
};
\addplot [color=mycolor3,solid,forget plot]
  table[row sep=crcr]{
0	0	\\
1.1e-10	0	\\
2.2e-10	0	\\
3.3e-10	0	\\
4.4e-10	0	\\
5.4e-10	0	\\
6.5e-10	0	\\
7.5e-10	0	\\
8.6e-10	0	\\
9.6e-10	0	\\
1.07e-09	0	\\
1.18e-09	0	\\
1.28e-09	0	\\
1.38e-09	0	\\
1.49e-09	0	\\
1.59e-09	0	\\
1.69e-09	0	\\
1.8e-09	0	\\
1.9e-09	0	\\
2.01e-09	0	\\
2.11e-09	0	\\
2.21e-09	0	\\
2.32e-09	0	\\
2.42e-09	0	\\
2.52e-09	0	\\
2.63e-09	0	\\
2.73e-09	0	\\
2.83e-09	0	\\
2.93e-09	0	\\
3.04e-09	0	\\
3.14e-09	0	\\
3.24e-09	0	\\
3.34e-09	0	\\
3.45e-09	0	\\
3.55e-09	0	\\
3.65e-09	0	\\
3.75e-09	0	\\
3.86e-09	0	\\
3.96e-09	0	\\
4.06e-09	0	\\
4.16e-09	0	\\
4.27e-09	0	\\
4.37e-09	0	\\
4.47e-09	0	\\
4.57e-09	0	\\
4.68e-09	0	\\
4.78e-09	0	\\
4.89e-09	0	\\
4.99e-09	0	\\
5e-09	0	\\
};
\addplot [color=darkgray,solid,forget plot]
  table[row sep=crcr]{
0	0	\\
1.1e-10	0	\\
2.2e-10	0	\\
3.3e-10	0	\\
4.4e-10	0	\\
5.4e-10	0	\\
6.5e-10	0	\\
7.5e-10	0	\\
8.6e-10	0	\\
9.6e-10	0	\\
1.07e-09	0	\\
1.18e-09	0	\\
1.28e-09	0	\\
1.38e-09	0	\\
1.49e-09	0	\\
1.59e-09	0	\\
1.69e-09	0	\\
1.8e-09	0	\\
1.9e-09	0	\\
2.01e-09	0	\\
2.11e-09	0	\\
2.21e-09	0	\\
2.32e-09	0	\\
2.42e-09	0	\\
2.52e-09	0	\\
2.63e-09	0	\\
2.73e-09	0	\\
2.83e-09	0	\\
2.93e-09	0	\\
3.04e-09	0	\\
3.14e-09	0	\\
3.24e-09	0	\\
3.34e-09	0	\\
3.45e-09	0	\\
3.55e-09	0	\\
3.65e-09	0	\\
3.75e-09	0	\\
3.86e-09	0	\\
3.96e-09	0	\\
4.06e-09	0	\\
4.16e-09	0	\\
4.27e-09	0	\\
4.37e-09	0	\\
4.47e-09	0	\\
4.57e-09	0	\\
4.68e-09	0	\\
4.78e-09	0	\\
4.89e-09	0	\\
4.99e-09	0	\\
5e-09	0	\\
};
\addplot [color=blue,solid,forget plot]
  table[row sep=crcr]{
0	0	\\
1.1e-10	0	\\
2.2e-10	0	\\
3.3e-10	0	\\
4.4e-10	0	\\
5.4e-10	0	\\
6.5e-10	0	\\
7.5e-10	0	\\
8.6e-10	0	\\
9.6e-10	0	\\
1.07e-09	0	\\
1.18e-09	0	\\
1.28e-09	0	\\
1.38e-09	0	\\
1.49e-09	0	\\
1.59e-09	0	\\
1.69e-09	0	\\
1.8e-09	0	\\
1.9e-09	0	\\
2.01e-09	0	\\
2.11e-09	0	\\
2.21e-09	0	\\
2.32e-09	0	\\
2.42e-09	0	\\
2.52e-09	0	\\
2.63e-09	0	\\
2.73e-09	0	\\
2.83e-09	0	\\
2.93e-09	0	\\
3.04e-09	0	\\
3.14e-09	0	\\
3.24e-09	0	\\
3.34e-09	0	\\
3.45e-09	0	\\
3.55e-09	0	\\
3.65e-09	0	\\
3.75e-09	0	\\
3.86e-09	0	\\
3.96e-09	0	\\
4.06e-09	0	\\
4.16e-09	0	\\
4.27e-09	0	\\
4.37e-09	0	\\
4.47e-09	0	\\
4.57e-09	0	\\
4.68e-09	0	\\
4.78e-09	0	\\
4.89e-09	0	\\
4.99e-09	0	\\
5e-09	0	\\
};
\addplot [color=black!50!green,solid,forget plot]
  table[row sep=crcr]{
0	0	\\
1.1e-10	0	\\
2.2e-10	0	\\
3.3e-10	0	\\
4.4e-10	0	\\
5.4e-10	0	\\
6.5e-10	0	\\
7.5e-10	0	\\
8.6e-10	0	\\
9.6e-10	0	\\
1.07e-09	0	\\
1.18e-09	0	\\
1.28e-09	0	\\
1.38e-09	0	\\
1.49e-09	0	\\
1.59e-09	0	\\
1.69e-09	0	\\
1.8e-09	0	\\
1.9e-09	0	\\
2.01e-09	0	\\
2.11e-09	0	\\
2.21e-09	0	\\
2.32e-09	0	\\
2.42e-09	0	\\
2.52e-09	0	\\
2.63e-09	0	\\
2.73e-09	0	\\
2.83e-09	0	\\
2.93e-09	0	\\
3.04e-09	0	\\
3.14e-09	0	\\
3.24e-09	0	\\
3.34e-09	0	\\
3.45e-09	0	\\
3.55e-09	0	\\
3.65e-09	0	\\
3.75e-09	0	\\
3.86e-09	0	\\
3.96e-09	0	\\
4.06e-09	0	\\
4.16e-09	0	\\
4.27e-09	0	\\
4.37e-09	0	\\
4.47e-09	0	\\
4.57e-09	0	\\
4.68e-09	0	\\
4.78e-09	0	\\
4.89e-09	0	\\
4.99e-09	0	\\
5e-09	0	\\
};
\addplot [color=red,solid,forget plot]
  table[row sep=crcr]{
0	0	\\
1.1e-10	0	\\
2.2e-10	0	\\
3.3e-10	0	\\
4.4e-10	0	\\
5.4e-10	0	\\
6.5e-10	0	\\
7.5e-10	0	\\
8.6e-10	0	\\
9.6e-10	0	\\
1.07e-09	0	\\
1.18e-09	0	\\
1.28e-09	0	\\
1.38e-09	0	\\
1.49e-09	0	\\
1.59e-09	0	\\
1.69e-09	0	\\
1.8e-09	0	\\
1.9e-09	0	\\
2.01e-09	0	\\
2.11e-09	0	\\
2.21e-09	0	\\
2.32e-09	0	\\
2.42e-09	0	\\
2.52e-09	0	\\
2.63e-09	0	\\
2.73e-09	0	\\
2.83e-09	0	\\
2.93e-09	0	\\
3.04e-09	0	\\
3.14e-09	0	\\
3.24e-09	0	\\
3.34e-09	0	\\
3.45e-09	0	\\
3.55e-09	0	\\
3.65e-09	0	\\
3.75e-09	0	\\
3.86e-09	0	\\
3.96e-09	0	\\
4.06e-09	0	\\
4.16e-09	0	\\
4.27e-09	0	\\
4.37e-09	0	\\
4.47e-09	0	\\
4.57e-09	0	\\
4.68e-09	0	\\
4.78e-09	0	\\
4.89e-09	0	\\
4.99e-09	0	\\
5e-09	0	\\
};
\addplot [color=mycolor1,solid,forget plot]
  table[row sep=crcr]{
0	0	\\
1.1e-10	0	\\
2.2e-10	0	\\
3.3e-10	0	\\
4.4e-10	0	\\
5.4e-10	0	\\
6.5e-10	0	\\
7.5e-10	0	\\
8.6e-10	0	\\
9.6e-10	0	\\
1.07e-09	0	\\
1.18e-09	0	\\
1.28e-09	0	\\
1.38e-09	0	\\
1.49e-09	0	\\
1.59e-09	0	\\
1.69e-09	0	\\
1.8e-09	0	\\
1.9e-09	0	\\
2.01e-09	0	\\
2.11e-09	0	\\
2.21e-09	0	\\
2.32e-09	0	\\
2.42e-09	0	\\
2.52e-09	0	\\
2.63e-09	0	\\
2.73e-09	0	\\
2.83e-09	0	\\
2.93e-09	0	\\
3.04e-09	0	\\
3.14e-09	0	\\
3.24e-09	0	\\
3.34e-09	0	\\
3.45e-09	0	\\
3.55e-09	0	\\
3.65e-09	0	\\
3.75e-09	0	\\
3.86e-09	0	\\
3.96e-09	0	\\
4.06e-09	0	\\
4.16e-09	0	\\
4.27e-09	0	\\
4.37e-09	0	\\
4.47e-09	0	\\
4.57e-09	0	\\
4.68e-09	0	\\
4.78e-09	0	\\
4.89e-09	0	\\
4.99e-09	0	\\
5e-09	0	\\
};
\addplot [color=mycolor2,solid,forget plot]
  table[row sep=crcr]{
0	0	\\
1.1e-10	0	\\
2.2e-10	0	\\
3.3e-10	0	\\
4.4e-10	0	\\
5.4e-10	0	\\
6.5e-10	0	\\
7.5e-10	0	\\
8.6e-10	0	\\
9.6e-10	0	\\
1.07e-09	0	\\
1.18e-09	0	\\
1.28e-09	0	\\
1.38e-09	0	\\
1.49e-09	0	\\
1.59e-09	0	\\
1.69e-09	0	\\
1.8e-09	0	\\
1.9e-09	0	\\
2.01e-09	0	\\
2.11e-09	0	\\
2.21e-09	0	\\
2.32e-09	0	\\
2.42e-09	0	\\
2.52e-09	0	\\
2.63e-09	0	\\
2.73e-09	0	\\
2.83e-09	0	\\
2.93e-09	0	\\
3.04e-09	0	\\
3.14e-09	0	\\
3.24e-09	0	\\
3.34e-09	0	\\
3.45e-09	0	\\
3.55e-09	0	\\
3.65e-09	0	\\
3.75e-09	0	\\
3.86e-09	0	\\
3.96e-09	0	\\
4.06e-09	0	\\
4.16e-09	0	\\
4.27e-09	0	\\
4.37e-09	0	\\
4.47e-09	0	\\
4.57e-09	0	\\
4.68e-09	0	\\
4.78e-09	0	\\
4.89e-09	0	\\
4.99e-09	0	\\
5e-09	0	\\
};
\addplot [color=mycolor3,solid,forget plot]
  table[row sep=crcr]{
0	0	\\
1.1e-10	0	\\
2.2e-10	0	\\
3.3e-10	0	\\
4.4e-10	0	\\
5.4e-10	0	\\
6.5e-10	0	\\
7.5e-10	0	\\
8.6e-10	0	\\
9.6e-10	0	\\
1.07e-09	0	\\
1.18e-09	0	\\
1.28e-09	0	\\
1.38e-09	0	\\
1.49e-09	0	\\
1.59e-09	0	\\
1.69e-09	0	\\
1.8e-09	0	\\
1.9e-09	0	\\
2.01e-09	0	\\
2.11e-09	0	\\
2.21e-09	0	\\
2.32e-09	0	\\
2.42e-09	0	\\
2.52e-09	0	\\
2.63e-09	0	\\
2.73e-09	0	\\
2.83e-09	0	\\
2.93e-09	0	\\
3.04e-09	0	\\
3.14e-09	0	\\
3.24e-09	0	\\
3.34e-09	0	\\
3.45e-09	0	\\
3.55e-09	0	\\
3.65e-09	0	\\
3.75e-09	0	\\
3.86e-09	0	\\
3.96e-09	0	\\
4.06e-09	0	\\
4.16e-09	0	\\
4.27e-09	0	\\
4.37e-09	0	\\
4.47e-09	0	\\
4.57e-09	0	\\
4.68e-09	0	\\
4.78e-09	0	\\
4.89e-09	0	\\
4.99e-09	0	\\
5e-09	0	\\
};
\addplot [color=darkgray,solid,forget plot]
  table[row sep=crcr]{
0	0	\\
1.1e-10	0	\\
2.2e-10	0	\\
3.3e-10	0	\\
4.4e-10	0	\\
5.4e-10	0	\\
6.5e-10	0	\\
7.5e-10	0	\\
8.6e-10	0	\\
9.6e-10	0	\\
1.07e-09	0	\\
1.18e-09	0	\\
1.28e-09	0	\\
1.38e-09	0	\\
1.49e-09	0	\\
1.59e-09	0	\\
1.69e-09	0	\\
1.8e-09	0	\\
1.9e-09	0	\\
2.01e-09	0	\\
2.11e-09	0	\\
2.21e-09	0	\\
2.32e-09	0	\\
2.42e-09	0	\\
2.52e-09	0	\\
2.63e-09	0	\\
2.73e-09	0	\\
2.83e-09	0	\\
2.93e-09	0	\\
3.04e-09	0	\\
3.14e-09	0	\\
3.24e-09	0	\\
3.34e-09	0	\\
3.45e-09	0	\\
3.55e-09	0	\\
3.65e-09	0	\\
3.75e-09	0	\\
3.86e-09	0	\\
3.96e-09	0	\\
4.06e-09	0	\\
4.16e-09	0	\\
4.27e-09	0	\\
4.37e-09	0	\\
4.47e-09	0	\\
4.57e-09	0	\\
4.68e-09	0	\\
4.78e-09	0	\\
4.89e-09	0	\\
4.99e-09	0	\\
5e-09	0	\\
};
\addplot [color=blue,solid,forget plot]
  table[row sep=crcr]{
0	0	\\
1.1e-10	0	\\
2.2e-10	0	\\
3.3e-10	0	\\
4.4e-10	0	\\
5.4e-10	0	\\
6.5e-10	0	\\
7.5e-10	0	\\
8.6e-10	0	\\
9.6e-10	0	\\
1.07e-09	0	\\
1.18e-09	0	\\
1.28e-09	0	\\
1.38e-09	0	\\
1.49e-09	0	\\
1.59e-09	0	\\
1.69e-09	0	\\
1.8e-09	0	\\
1.9e-09	0	\\
2.01e-09	0	\\
2.11e-09	0	\\
2.21e-09	0	\\
2.32e-09	0	\\
2.42e-09	0	\\
2.52e-09	0	\\
2.63e-09	0	\\
2.73e-09	0	\\
2.83e-09	0	\\
2.93e-09	0	\\
3.04e-09	0	\\
3.14e-09	0	\\
3.24e-09	0	\\
3.34e-09	0	\\
3.45e-09	0	\\
3.55e-09	0	\\
3.65e-09	0	\\
3.75e-09	0	\\
3.86e-09	0	\\
3.96e-09	0	\\
4.06e-09	0	\\
4.16e-09	0	\\
4.27e-09	0	\\
4.37e-09	0	\\
4.47e-09	0	\\
4.57e-09	0	\\
4.68e-09	0	\\
4.78e-09	0	\\
4.89e-09	0	\\
4.99e-09	0	\\
5e-09	0	\\
};
\addplot [color=black!50!green,solid,forget plot]
  table[row sep=crcr]{
0	0	\\
1.1e-10	0	\\
2.2e-10	0	\\
3.3e-10	0	\\
4.4e-10	0	\\
5.4e-10	0	\\
6.5e-10	0	\\
7.5e-10	0	\\
8.6e-10	0	\\
9.6e-10	0	\\
1.07e-09	0	\\
1.18e-09	0	\\
1.28e-09	0	\\
1.38e-09	0	\\
1.49e-09	0	\\
1.59e-09	0	\\
1.69e-09	0	\\
1.8e-09	0	\\
1.9e-09	0	\\
2.01e-09	0	\\
2.11e-09	0	\\
2.21e-09	0	\\
2.32e-09	0	\\
2.42e-09	0	\\
2.52e-09	0	\\
2.63e-09	0	\\
2.73e-09	0	\\
2.83e-09	0	\\
2.93e-09	0	\\
3.04e-09	0	\\
3.14e-09	0	\\
3.24e-09	0	\\
3.34e-09	0	\\
3.45e-09	0	\\
3.55e-09	0	\\
3.65e-09	0	\\
3.75e-09	0	\\
3.86e-09	0	\\
3.96e-09	0	\\
4.06e-09	0	\\
4.16e-09	0	\\
4.27e-09	0	\\
4.37e-09	0	\\
4.47e-09	0	\\
4.57e-09	0	\\
4.68e-09	0	\\
4.78e-09	0	\\
4.89e-09	0	\\
4.99e-09	0	\\
5e-09	0	\\
};
\addplot [color=red,solid,forget plot]
  table[row sep=crcr]{
0	0	\\
1.1e-10	0	\\
2.2e-10	0	\\
3.3e-10	0	\\
4.4e-10	0	\\
5.4e-10	0	\\
6.5e-10	0	\\
7.5e-10	0	\\
8.6e-10	0	\\
9.6e-10	0	\\
1.07e-09	0	\\
1.18e-09	0	\\
1.28e-09	0	\\
1.38e-09	0	\\
1.49e-09	0	\\
1.59e-09	0	\\
1.69e-09	0	\\
1.8e-09	0	\\
1.9e-09	0	\\
2.01e-09	0	\\
2.11e-09	0	\\
2.21e-09	0	\\
2.32e-09	0	\\
2.42e-09	0	\\
2.52e-09	0	\\
2.63e-09	0	\\
2.73e-09	0	\\
2.83e-09	0	\\
2.93e-09	0	\\
3.04e-09	0	\\
3.14e-09	0	\\
3.24e-09	0	\\
3.34e-09	0	\\
3.45e-09	0	\\
3.55e-09	0	\\
3.65e-09	0	\\
3.75e-09	0	\\
3.86e-09	0	\\
3.96e-09	0	\\
4.06e-09	0	\\
4.16e-09	0	\\
4.27e-09	0	\\
4.37e-09	0	\\
4.47e-09	0	\\
4.57e-09	0	\\
4.68e-09	0	\\
4.78e-09	0	\\
4.89e-09	0	\\
4.99e-09	0	\\
5e-09	0	\\
};
\addplot [color=mycolor1,solid,forget plot]
  table[row sep=crcr]{
0	0	\\
1.1e-10	0	\\
2.2e-10	0	\\
3.3e-10	0	\\
4.4e-10	0	\\
5.4e-10	0	\\
6.5e-10	0	\\
7.5e-10	0	\\
8.6e-10	0	\\
9.6e-10	0	\\
1.07e-09	0	\\
1.18e-09	0	\\
1.28e-09	0	\\
1.38e-09	0	\\
1.49e-09	0	\\
1.59e-09	0	\\
1.69e-09	0	\\
1.8e-09	0	\\
1.9e-09	0	\\
2.01e-09	0	\\
2.11e-09	0	\\
2.21e-09	0	\\
2.32e-09	0	\\
2.42e-09	0	\\
2.52e-09	0	\\
2.63e-09	0	\\
2.73e-09	0	\\
2.83e-09	0	\\
2.93e-09	0	\\
3.04e-09	0	\\
3.14e-09	0	\\
3.24e-09	0	\\
3.34e-09	0	\\
3.45e-09	0	\\
3.55e-09	0	\\
3.65e-09	0	\\
3.75e-09	0	\\
3.86e-09	0	\\
3.96e-09	0	\\
4.06e-09	0	\\
4.16e-09	0	\\
4.27e-09	0	\\
4.37e-09	0	\\
4.47e-09	0	\\
4.57e-09	0	\\
4.68e-09	0	\\
4.78e-09	0	\\
4.89e-09	0	\\
4.99e-09	0	\\
5e-09	0	\\
};
\addplot [color=mycolor2,solid,forget plot]
  table[row sep=crcr]{
0	0	\\
1.1e-10	0	\\
2.2e-10	0	\\
3.3e-10	0	\\
4.4e-10	0	\\
5.4e-10	0	\\
6.5e-10	0	\\
7.5e-10	0	\\
8.6e-10	0	\\
9.6e-10	0	\\
1.07e-09	0	\\
1.18e-09	0	\\
1.28e-09	0	\\
1.38e-09	0	\\
1.49e-09	0	\\
1.59e-09	0	\\
1.69e-09	0	\\
1.8e-09	0	\\
1.9e-09	0	\\
2.01e-09	0	\\
2.11e-09	0	\\
2.21e-09	0	\\
2.32e-09	0	\\
2.42e-09	0	\\
2.52e-09	0	\\
2.63e-09	0	\\
2.73e-09	0	\\
2.83e-09	0	\\
2.93e-09	0	\\
3.04e-09	0	\\
3.14e-09	0	\\
3.24e-09	0	\\
3.34e-09	0	\\
3.45e-09	0	\\
3.55e-09	0	\\
3.65e-09	0	\\
3.75e-09	0	\\
3.86e-09	0	\\
3.96e-09	0	\\
4.06e-09	0	\\
4.16e-09	0	\\
4.27e-09	0	\\
4.37e-09	0	\\
4.47e-09	0	\\
4.57e-09	0	\\
4.68e-09	0	\\
4.78e-09	0	\\
4.89e-09	0	\\
4.99e-09	0	\\
5e-09	0	\\
};
\addplot [color=mycolor3,solid,forget plot]
  table[row sep=crcr]{
0	0	\\
1.1e-10	0	\\
2.2e-10	0	\\
3.3e-10	0	\\
4.4e-10	0	\\
5.4e-10	0	\\
6.5e-10	0	\\
7.5e-10	0	\\
8.6e-10	0	\\
9.6e-10	0	\\
1.07e-09	0	\\
1.18e-09	0	\\
1.28e-09	0	\\
1.38e-09	0	\\
1.49e-09	0	\\
1.59e-09	0	\\
1.69e-09	0	\\
1.8e-09	0	\\
1.9e-09	0	\\
2.01e-09	0	\\
2.11e-09	0	\\
2.21e-09	0	\\
2.32e-09	0	\\
2.42e-09	0	\\
2.52e-09	0	\\
2.63e-09	0	\\
2.73e-09	0	\\
2.83e-09	0	\\
2.93e-09	0	\\
3.04e-09	0	\\
3.14e-09	0	\\
3.24e-09	0	\\
3.34e-09	0	\\
3.45e-09	0	\\
3.55e-09	0	\\
3.65e-09	0	\\
3.75e-09	0	\\
3.86e-09	0	\\
3.96e-09	0	\\
4.06e-09	0	\\
4.16e-09	0	\\
4.27e-09	0	\\
4.37e-09	0	\\
4.47e-09	0	\\
4.57e-09	0	\\
4.68e-09	0	\\
4.78e-09	0	\\
4.89e-09	0	\\
4.99e-09	0	\\
5e-09	0	\\
};
\addplot [color=darkgray,solid,forget plot]
  table[row sep=crcr]{
0	0	\\
1.1e-10	0	\\
2.2e-10	0	\\
3.3e-10	0	\\
4.4e-10	0	\\
5.4e-10	0	\\
6.5e-10	0	\\
7.5e-10	0	\\
8.6e-10	0	\\
9.6e-10	0	\\
1.07e-09	0	\\
1.18e-09	0	\\
1.28e-09	0	\\
1.38e-09	0	\\
1.49e-09	0	\\
1.59e-09	0	\\
1.69e-09	0	\\
1.8e-09	0	\\
1.9e-09	0	\\
2.01e-09	0	\\
2.11e-09	0	\\
2.21e-09	0	\\
2.32e-09	0	\\
2.42e-09	0	\\
2.52e-09	0	\\
2.63e-09	0	\\
2.73e-09	0	\\
2.83e-09	0	\\
2.93e-09	0	\\
3.04e-09	0	\\
3.14e-09	0	\\
3.24e-09	0	\\
3.34e-09	0	\\
3.45e-09	0	\\
3.55e-09	0	\\
3.65e-09	0	\\
3.75e-09	0	\\
3.86e-09	0	\\
3.96e-09	0	\\
4.06e-09	0	\\
4.16e-09	0	\\
4.27e-09	0	\\
4.37e-09	0	\\
4.47e-09	0	\\
4.57e-09	0	\\
4.68e-09	0	\\
4.78e-09	0	\\
4.89e-09	0	\\
4.99e-09	0	\\
5e-09	0	\\
};
\addplot [color=blue,solid,forget plot]
  table[row sep=crcr]{
0	0	\\
1.1e-10	0	\\
2.2e-10	0	\\
3.3e-10	0	\\
4.4e-10	0	\\
5.4e-10	0	\\
6.5e-10	0	\\
7.5e-10	0	\\
8.6e-10	0	\\
9.6e-10	0	\\
1.07e-09	0	\\
1.18e-09	0	\\
1.28e-09	0	\\
1.38e-09	0	\\
1.49e-09	0	\\
1.59e-09	0	\\
1.69e-09	0	\\
1.8e-09	0	\\
1.9e-09	0	\\
2.01e-09	0	\\
2.11e-09	0	\\
2.21e-09	0	\\
2.32e-09	0	\\
2.42e-09	0	\\
2.52e-09	0	\\
2.63e-09	0	\\
2.73e-09	0	\\
2.83e-09	0	\\
2.93e-09	0	\\
3.04e-09	0	\\
3.14e-09	0	\\
3.24e-09	0	\\
3.34e-09	0	\\
3.45e-09	0	\\
3.55e-09	0	\\
3.65e-09	0	\\
3.75e-09	0	\\
3.86e-09	0	\\
3.96e-09	0	\\
4.06e-09	0	\\
4.16e-09	0	\\
4.27e-09	0	\\
4.37e-09	0	\\
4.47e-09	0	\\
4.57e-09	0	\\
4.68e-09	0	\\
4.78e-09	0	\\
4.89e-09	0	\\
4.99e-09	0	\\
5e-09	0	\\
};
\addplot [color=black!50!green,solid,forget plot]
  table[row sep=crcr]{
0	0	\\
1.1e-10	0	\\
2.2e-10	0	\\
3.3e-10	0	\\
4.4e-10	0	\\
5.4e-10	0	\\
6.5e-10	0	\\
7.5e-10	0	\\
8.6e-10	0	\\
9.6e-10	0	\\
1.07e-09	0	\\
1.18e-09	0	\\
1.28e-09	0	\\
1.38e-09	0	\\
1.49e-09	0	\\
1.59e-09	0	\\
1.69e-09	0	\\
1.8e-09	0	\\
1.9e-09	0	\\
2.01e-09	0	\\
2.11e-09	0	\\
2.21e-09	0	\\
2.32e-09	0	\\
2.42e-09	0	\\
2.52e-09	0	\\
2.63e-09	0	\\
2.73e-09	0	\\
2.83e-09	0	\\
2.93e-09	0	\\
3.04e-09	0	\\
3.14e-09	0	\\
3.24e-09	0	\\
3.34e-09	0	\\
3.45e-09	0	\\
3.55e-09	0	\\
3.65e-09	0	\\
3.75e-09	0	\\
3.86e-09	0	\\
3.96e-09	0	\\
4.06e-09	0	\\
4.16e-09	0	\\
4.27e-09	0	\\
4.37e-09	0	\\
4.47e-09	0	\\
4.57e-09	0	\\
4.68e-09	0	\\
4.78e-09	0	\\
4.89e-09	0	\\
4.99e-09	0	\\
5e-09	0	\\
};
\addplot [color=red,solid,forget plot]
  table[row sep=crcr]{
0	0	\\
1.1e-10	0	\\
2.2e-10	0	\\
3.3e-10	0	\\
4.4e-10	0	\\
5.4e-10	0	\\
6.5e-10	0	\\
7.5e-10	0	\\
8.6e-10	0	\\
9.6e-10	0	\\
1.07e-09	0	\\
1.18e-09	0	\\
1.28e-09	0	\\
1.38e-09	0	\\
1.49e-09	0	\\
1.59e-09	0	\\
1.69e-09	0	\\
1.8e-09	0	\\
1.9e-09	0	\\
2.01e-09	0	\\
2.11e-09	0	\\
2.21e-09	0	\\
2.32e-09	0	\\
2.42e-09	0	\\
2.52e-09	0	\\
2.63e-09	0	\\
2.73e-09	0	\\
2.83e-09	0	\\
2.93e-09	0	\\
3.04e-09	0	\\
3.14e-09	0	\\
3.24e-09	0	\\
3.34e-09	0	\\
3.45e-09	0	\\
3.55e-09	0	\\
3.65e-09	0	\\
3.75e-09	0	\\
3.86e-09	0	\\
3.96e-09	0	\\
4.06e-09	0	\\
4.16e-09	0	\\
4.27e-09	0	\\
4.37e-09	0	\\
4.47e-09	0	\\
4.57e-09	0	\\
4.68e-09	0	\\
4.78e-09	0	\\
4.89e-09	0	\\
4.99e-09	0	\\
5e-09	0	\\
};
\addplot [color=mycolor1,solid,forget plot]
  table[row sep=crcr]{
0	0	\\
1.1e-10	0	\\
2.2e-10	0	\\
3.3e-10	0	\\
4.4e-10	0	\\
5.4e-10	0	\\
6.5e-10	0	\\
7.5e-10	0	\\
8.6e-10	0	\\
9.6e-10	0	\\
1.07e-09	0	\\
1.18e-09	0	\\
1.28e-09	0	\\
1.38e-09	0	\\
1.49e-09	0	\\
1.59e-09	0	\\
1.69e-09	0	\\
1.8e-09	0	\\
1.9e-09	0	\\
2.01e-09	0	\\
2.11e-09	0	\\
2.21e-09	0	\\
2.32e-09	0	\\
2.42e-09	0	\\
2.52e-09	0	\\
2.63e-09	0	\\
2.73e-09	0	\\
2.83e-09	0	\\
2.93e-09	0	\\
3.04e-09	0	\\
3.14e-09	0	\\
3.24e-09	0	\\
3.34e-09	0	\\
3.45e-09	0	\\
3.55e-09	0	\\
3.65e-09	0	\\
3.75e-09	0	\\
3.86e-09	0	\\
3.96e-09	0	\\
4.06e-09	0	\\
4.16e-09	0	\\
4.27e-09	0	\\
4.37e-09	0	\\
4.47e-09	0	\\
4.57e-09	0	\\
4.68e-09	0	\\
4.78e-09	0	\\
4.89e-09	0	\\
4.99e-09	0	\\
5e-09	0	\\
};
\addplot [color=mycolor2,solid,forget plot]
  table[row sep=crcr]{
0	0	\\
1.1e-10	0	\\
2.2e-10	0	\\
3.3e-10	0	\\
4.4e-10	0	\\
5.4e-10	0	\\
6.5e-10	0	\\
7.5e-10	0	\\
8.6e-10	0	\\
9.6e-10	0	\\
1.07e-09	0	\\
1.18e-09	0	\\
1.28e-09	0	\\
1.38e-09	0	\\
1.49e-09	0	\\
1.59e-09	0	\\
1.69e-09	0	\\
1.8e-09	0	\\
1.9e-09	0	\\
2.01e-09	0	\\
2.11e-09	0	\\
2.21e-09	0	\\
2.32e-09	0	\\
2.42e-09	0	\\
2.52e-09	0	\\
2.63e-09	0	\\
2.73e-09	0	\\
2.83e-09	0	\\
2.93e-09	0	\\
3.04e-09	0	\\
3.14e-09	0	\\
3.24e-09	0	\\
3.34e-09	0	\\
3.45e-09	0	\\
3.55e-09	0	\\
3.65e-09	0	\\
3.75e-09	0	\\
3.86e-09	0	\\
3.96e-09	0	\\
4.06e-09	0	\\
4.16e-09	0	\\
4.27e-09	0	\\
4.37e-09	0	\\
4.47e-09	0	\\
4.57e-09	0	\\
4.68e-09	0	\\
4.78e-09	0	\\
4.89e-09	0	\\
4.99e-09	0	\\
5e-09	0	\\
};
\addplot [color=mycolor3,solid,forget plot]
  table[row sep=crcr]{
0	0	\\
1.1e-10	0	\\
2.2e-10	0	\\
3.3e-10	0	\\
4.4e-10	0	\\
5.4e-10	0	\\
6.5e-10	0	\\
7.5e-10	0	\\
8.6e-10	0	\\
9.6e-10	0	\\
1.07e-09	0	\\
1.18e-09	0	\\
1.28e-09	0	\\
1.38e-09	0	\\
1.49e-09	0	\\
1.59e-09	0	\\
1.69e-09	0	\\
1.8e-09	0	\\
1.9e-09	0	\\
2.01e-09	0	\\
2.11e-09	0	\\
2.21e-09	0	\\
2.32e-09	0	\\
2.42e-09	0	\\
2.52e-09	0	\\
2.63e-09	0	\\
2.73e-09	0	\\
2.83e-09	0	\\
2.93e-09	0	\\
3.04e-09	0	\\
3.14e-09	0	\\
3.24e-09	0	\\
3.34e-09	0	\\
3.45e-09	0	\\
3.55e-09	0	\\
3.65e-09	0	\\
3.75e-09	0	\\
3.86e-09	0	\\
3.96e-09	0	\\
4.06e-09	0	\\
4.16e-09	0	\\
4.27e-09	0	\\
4.37e-09	0	\\
4.47e-09	0	\\
4.57e-09	0	\\
4.68e-09	0	\\
4.78e-09	0	\\
4.89e-09	0	\\
4.99e-09	0	\\
5e-09	0	\\
};
\addplot [color=darkgray,solid,forget plot]
  table[row sep=crcr]{
0	0	\\
1.1e-10	0	\\
2.2e-10	0	\\
3.3e-10	0	\\
4.4e-10	0	\\
5.4e-10	0	\\
6.5e-10	0	\\
7.5e-10	0	\\
8.6e-10	0	\\
9.6e-10	0	\\
1.07e-09	0	\\
1.18e-09	0	\\
1.28e-09	0	\\
1.38e-09	0	\\
1.49e-09	0	\\
1.59e-09	0	\\
1.69e-09	0	\\
1.8e-09	0	\\
1.9e-09	0	\\
2.01e-09	0	\\
2.11e-09	0	\\
2.21e-09	0	\\
2.32e-09	0	\\
2.42e-09	0	\\
2.52e-09	0	\\
2.63e-09	0	\\
2.73e-09	0	\\
2.83e-09	0	\\
2.93e-09	0	\\
3.04e-09	0	\\
3.14e-09	0	\\
3.24e-09	0	\\
3.34e-09	0	\\
3.45e-09	0	\\
3.55e-09	0	\\
3.65e-09	0	\\
3.75e-09	0	\\
3.86e-09	0	\\
3.96e-09	0	\\
4.06e-09	0	\\
4.16e-09	0	\\
4.27e-09	0	\\
4.37e-09	0	\\
4.47e-09	0	\\
4.57e-09	0	\\
4.68e-09	0	\\
4.78e-09	0	\\
4.89e-09	0	\\
4.99e-09	0	\\
5e-09	0	\\
};
\addplot [color=blue,solid,forget plot]
  table[row sep=crcr]{
0	0	\\
1.1e-10	0	\\
2.2e-10	0	\\
3.3e-10	0	\\
4.4e-10	0	\\
5.4e-10	0	\\
6.5e-10	0	\\
7.5e-10	0	\\
8.6e-10	0	\\
9.6e-10	0	\\
1.07e-09	0	\\
1.18e-09	0	\\
1.28e-09	0	\\
1.38e-09	0	\\
1.49e-09	0	\\
1.59e-09	0	\\
1.69e-09	0	\\
1.8e-09	0	\\
1.9e-09	0	\\
2.01e-09	0	\\
2.11e-09	0	\\
2.21e-09	0	\\
2.32e-09	0	\\
2.42e-09	0	\\
2.52e-09	0	\\
2.63e-09	0	\\
2.73e-09	0	\\
2.83e-09	0	\\
2.93e-09	0	\\
3.04e-09	0	\\
3.14e-09	0	\\
3.24e-09	0	\\
3.34e-09	0	\\
3.45e-09	0	\\
3.55e-09	0	\\
3.65e-09	0	\\
3.75e-09	0	\\
3.86e-09	0	\\
3.96e-09	0	\\
4.06e-09	0	\\
4.16e-09	0	\\
4.27e-09	0	\\
4.37e-09	0	\\
4.47e-09	0	\\
4.57e-09	0	\\
4.68e-09	0	\\
4.78e-09	0	\\
4.89e-09	0	\\
4.99e-09	0	\\
5e-09	0	\\
};
\addplot [color=black!50!green,solid,forget plot]
  table[row sep=crcr]{
0	0	\\
1.1e-10	0	\\
2.2e-10	0	\\
3.3e-10	0	\\
4.4e-10	0	\\
5.4e-10	0	\\
6.5e-10	0	\\
7.5e-10	0	\\
8.6e-10	0	\\
9.6e-10	0	\\
1.07e-09	0	\\
1.18e-09	0	\\
1.28e-09	0	\\
1.38e-09	0	\\
1.49e-09	0	\\
1.59e-09	0	\\
1.69e-09	0	\\
1.8e-09	0	\\
1.9e-09	0	\\
2.01e-09	0	\\
2.11e-09	0	\\
2.21e-09	0	\\
2.32e-09	0	\\
2.42e-09	0	\\
2.52e-09	0	\\
2.63e-09	0	\\
2.73e-09	0	\\
2.83e-09	0	\\
2.93e-09	0	\\
3.04e-09	0	\\
3.14e-09	0	\\
3.24e-09	0	\\
3.34e-09	0	\\
3.45e-09	0	\\
3.55e-09	0	\\
3.65e-09	0	\\
3.75e-09	0	\\
3.86e-09	0	\\
3.96e-09	0	\\
4.06e-09	0	\\
4.16e-09	0	\\
4.27e-09	0	\\
4.37e-09	0	\\
4.47e-09	0	\\
4.57e-09	0	\\
4.68e-09	0	\\
4.78e-09	0	\\
4.89e-09	0	\\
4.99e-09	0	\\
5e-09	0	\\
};
\addplot [color=red,solid,forget plot]
  table[row sep=crcr]{
0	0	\\
1.1e-10	0	\\
2.2e-10	0	\\
3.3e-10	0	\\
4.4e-10	0	\\
5.4e-10	0	\\
6.5e-10	0	\\
7.5e-10	0	\\
8.6e-10	0	\\
9.6e-10	0	\\
1.07e-09	0	\\
1.18e-09	0	\\
1.28e-09	0	\\
1.38e-09	0	\\
1.49e-09	0	\\
1.59e-09	0	\\
1.69e-09	0	\\
1.8e-09	0	\\
1.9e-09	0	\\
2.01e-09	0	\\
2.11e-09	0	\\
2.21e-09	0	\\
2.32e-09	0	\\
2.42e-09	0	\\
2.52e-09	0	\\
2.63e-09	0	\\
2.73e-09	0	\\
2.83e-09	0	\\
2.93e-09	0	\\
3.04e-09	0	\\
3.14e-09	0	\\
3.24e-09	0	\\
3.34e-09	0	\\
3.45e-09	0	\\
3.55e-09	0	\\
3.65e-09	0	\\
3.75e-09	0	\\
3.86e-09	0	\\
3.96e-09	0	\\
4.06e-09	0	\\
4.16e-09	0	\\
4.27e-09	0	\\
4.37e-09	0	\\
4.47e-09	0	\\
4.57e-09	0	\\
4.68e-09	0	\\
4.78e-09	0	\\
4.89e-09	0	\\
4.99e-09	0	\\
5e-09	0	\\
};
\addplot [color=mycolor1,solid,forget plot]
  table[row sep=crcr]{
0	0	\\
1.1e-10	0	\\
2.2e-10	0	\\
3.3e-10	0	\\
4.4e-10	0	\\
5.4e-10	0	\\
6.5e-10	0	\\
7.5e-10	0	\\
8.6e-10	0	\\
9.6e-10	0	\\
1.07e-09	0	\\
1.18e-09	0	\\
1.28e-09	0	\\
1.38e-09	0	\\
1.49e-09	0	\\
1.59e-09	0	\\
1.69e-09	0	\\
1.8e-09	0	\\
1.9e-09	0	\\
2.01e-09	0	\\
2.11e-09	0	\\
2.21e-09	0	\\
2.32e-09	0	\\
2.42e-09	0	\\
2.52e-09	0	\\
2.63e-09	0	\\
2.73e-09	0	\\
2.83e-09	0	\\
2.93e-09	0	\\
3.04e-09	0	\\
3.14e-09	0	\\
3.24e-09	0	\\
3.34e-09	0	\\
3.45e-09	0	\\
3.55e-09	0	\\
3.65e-09	0	\\
3.75e-09	0	\\
3.86e-09	0	\\
3.96e-09	0	\\
4.06e-09	0	\\
4.16e-09	0	\\
4.27e-09	0	\\
4.37e-09	0	\\
4.47e-09	0	\\
4.57e-09	0	\\
4.68e-09	0	\\
4.78e-09	0	\\
4.89e-09	0	\\
4.99e-09	0	\\
5e-09	0	\\
};
\addplot [color=mycolor2,solid,forget plot]
  table[row sep=crcr]{
0	0	\\
1.1e-10	0	\\
2.2e-10	0	\\
3.3e-10	0	\\
4.4e-10	0	\\
5.4e-10	0	\\
6.5e-10	0	\\
7.5e-10	0	\\
8.6e-10	0	\\
9.6e-10	0	\\
1.07e-09	0	\\
1.18e-09	0	\\
1.28e-09	0	\\
1.38e-09	0	\\
1.49e-09	0	\\
1.59e-09	0	\\
1.69e-09	0	\\
1.8e-09	0	\\
1.9e-09	0	\\
2.01e-09	0	\\
2.11e-09	0	\\
2.21e-09	0	\\
2.32e-09	0	\\
2.42e-09	0	\\
2.52e-09	0	\\
2.63e-09	0	\\
2.73e-09	0	\\
2.83e-09	0	\\
2.93e-09	0	\\
3.04e-09	0	\\
3.14e-09	0	\\
3.24e-09	0	\\
3.34e-09	0	\\
3.45e-09	0	\\
3.55e-09	0	\\
3.65e-09	0	\\
3.75e-09	0	\\
3.86e-09	0	\\
3.96e-09	0	\\
4.06e-09	0	\\
4.16e-09	0	\\
4.27e-09	0	\\
4.37e-09	0	\\
4.47e-09	0	\\
4.57e-09	0	\\
4.68e-09	0	\\
4.78e-09	0	\\
4.89e-09	0	\\
4.99e-09	0	\\
5e-09	0	\\
};
\addplot [color=mycolor3,solid,forget plot]
  table[row sep=crcr]{
0	0	\\
1.1e-10	0	\\
2.2e-10	0	\\
3.3e-10	0	\\
4.4e-10	0	\\
5.4e-10	0	\\
6.5e-10	0	\\
7.5e-10	0	\\
8.6e-10	0	\\
9.6e-10	0	\\
1.07e-09	0	\\
1.18e-09	0	\\
1.28e-09	0	\\
1.38e-09	0	\\
1.49e-09	0	\\
1.59e-09	0	\\
1.69e-09	0	\\
1.8e-09	0	\\
1.9e-09	0	\\
2.01e-09	0	\\
2.11e-09	0	\\
2.21e-09	0	\\
2.32e-09	0	\\
2.42e-09	0	\\
2.52e-09	0	\\
2.63e-09	0	\\
2.73e-09	0	\\
2.83e-09	0	\\
2.93e-09	0	\\
3.04e-09	0	\\
3.14e-09	0	\\
3.24e-09	0	\\
3.34e-09	0	\\
3.45e-09	0	\\
3.55e-09	0	\\
3.65e-09	0	\\
3.75e-09	0	\\
3.86e-09	0	\\
3.96e-09	0	\\
4.06e-09	0	\\
4.16e-09	0	\\
4.27e-09	0	\\
4.37e-09	0	\\
4.47e-09	0	\\
4.57e-09	0	\\
4.68e-09	0	\\
4.78e-09	0	\\
4.89e-09	0	\\
4.99e-09	0	\\
5e-09	0	\\
};
\addplot [color=darkgray,solid,forget plot]
  table[row sep=crcr]{
0	0	\\
1.1e-10	0	\\
2.2e-10	0	\\
3.3e-10	0	\\
4.4e-10	0	\\
5.4e-10	0	\\
6.5e-10	0	\\
7.5e-10	0	\\
8.6e-10	0	\\
9.6e-10	0	\\
1.07e-09	0	\\
1.18e-09	0	\\
1.28e-09	0	\\
1.38e-09	0	\\
1.49e-09	0	\\
1.59e-09	0	\\
1.69e-09	0	\\
1.8e-09	0	\\
1.9e-09	0	\\
2.01e-09	0	\\
2.11e-09	0	\\
2.21e-09	0	\\
2.32e-09	0	\\
2.42e-09	0	\\
2.52e-09	0	\\
2.63e-09	0	\\
2.73e-09	0	\\
2.83e-09	0	\\
2.93e-09	0	\\
3.04e-09	0	\\
3.14e-09	0	\\
3.24e-09	0	\\
3.34e-09	0	\\
3.45e-09	0	\\
3.55e-09	0	\\
3.65e-09	0	\\
3.75e-09	0	\\
3.86e-09	0	\\
3.96e-09	0	\\
4.06e-09	0	\\
4.16e-09	0	\\
4.27e-09	0	\\
4.37e-09	0	\\
4.47e-09	0	\\
4.57e-09	0	\\
4.68e-09	0	\\
4.78e-09	0	\\
4.89e-09	0	\\
4.99e-09	0	\\
5e-09	0	\\
};
\addplot [color=blue,solid,forget plot]
  table[row sep=crcr]{
0	0	\\
1.1e-10	0	\\
2.2e-10	0	\\
3.3e-10	0	\\
4.4e-10	0	\\
5.4e-10	0	\\
6.5e-10	0	\\
7.5e-10	0	\\
8.6e-10	0	\\
9.6e-10	0	\\
1.07e-09	0	\\
1.18e-09	0	\\
1.28e-09	0	\\
1.38e-09	0	\\
1.49e-09	0	\\
1.59e-09	0	\\
1.69e-09	0	\\
1.8e-09	0	\\
1.9e-09	0	\\
2.01e-09	0	\\
2.11e-09	0	\\
2.21e-09	0	\\
2.32e-09	0	\\
2.42e-09	0	\\
2.52e-09	0	\\
2.63e-09	0	\\
2.73e-09	0	\\
2.83e-09	0	\\
2.93e-09	0	\\
3.04e-09	0	\\
3.14e-09	0	\\
3.24e-09	0	\\
3.34e-09	0	\\
3.45e-09	0	\\
3.55e-09	0	\\
3.65e-09	0	\\
3.75e-09	0	\\
3.86e-09	0	\\
3.96e-09	0	\\
4.06e-09	0	\\
4.16e-09	0	\\
4.27e-09	0	\\
4.37e-09	0	\\
4.47e-09	0	\\
4.57e-09	0	\\
4.68e-09	0	\\
4.78e-09	0	\\
4.89e-09	0	\\
4.99e-09	0	\\
5e-09	0	\\
};
\addplot [color=black!50!green,solid,forget plot]
  table[row sep=crcr]{
0	0	\\
1.1e-10	0	\\
2.2e-10	0	\\
3.3e-10	0	\\
4.4e-10	0	\\
5.4e-10	0	\\
6.5e-10	0	\\
7.5e-10	0	\\
8.6e-10	0	\\
9.6e-10	0	\\
1.07e-09	0	\\
1.18e-09	0	\\
1.28e-09	0	\\
1.38e-09	0	\\
1.49e-09	0	\\
1.59e-09	0	\\
1.69e-09	0	\\
1.8e-09	0	\\
1.9e-09	0	\\
2.01e-09	0	\\
2.11e-09	0	\\
2.21e-09	0	\\
2.32e-09	0	\\
2.42e-09	0	\\
2.52e-09	0	\\
2.63e-09	0	\\
2.73e-09	0	\\
2.83e-09	0	\\
2.93e-09	0	\\
3.04e-09	0	\\
3.14e-09	0	\\
3.24e-09	0	\\
3.34e-09	0	\\
3.45e-09	0	\\
3.55e-09	0	\\
3.65e-09	0	\\
3.75e-09	0	\\
3.86e-09	0	\\
3.96e-09	0	\\
4.06e-09	0	\\
4.16e-09	0	\\
4.27e-09	0	\\
4.37e-09	0	\\
4.47e-09	0	\\
4.57e-09	0	\\
4.68e-09	0	\\
4.78e-09	0	\\
4.89e-09	0	\\
4.99e-09	0	\\
5e-09	0	\\
};
\addplot [color=red,solid,forget plot]
  table[row sep=crcr]{
0	0	\\
1.1e-10	0	\\
2.2e-10	0	\\
3.3e-10	0	\\
4.4e-10	0	\\
5.4e-10	0	\\
6.5e-10	0	\\
7.5e-10	0	\\
8.6e-10	0	\\
9.6e-10	0	\\
1.07e-09	0	\\
1.18e-09	0	\\
1.28e-09	0	\\
1.38e-09	0	\\
1.49e-09	0	\\
1.59e-09	0	\\
1.69e-09	0	\\
1.8e-09	0	\\
1.9e-09	0	\\
2.01e-09	0	\\
2.11e-09	0	\\
2.21e-09	0	\\
2.32e-09	0	\\
2.42e-09	0	\\
2.52e-09	0	\\
2.63e-09	0	\\
2.73e-09	0	\\
2.83e-09	0	\\
2.93e-09	0	\\
3.04e-09	0	\\
3.14e-09	0	\\
3.24e-09	0	\\
3.34e-09	0	\\
3.45e-09	0	\\
3.55e-09	0	\\
3.65e-09	0	\\
3.75e-09	0	\\
3.86e-09	0	\\
3.96e-09	0	\\
4.06e-09	0	\\
4.16e-09	0	\\
4.27e-09	0	\\
4.37e-09	0	\\
4.47e-09	0	\\
4.57e-09	0	\\
4.68e-09	0	\\
4.78e-09	0	\\
4.89e-09	0	\\
4.99e-09	0	\\
5e-09	0	\\
};
\addplot [color=mycolor1,solid,forget plot]
  table[row sep=crcr]{
0	0	\\
1.1e-10	0	\\
2.2e-10	0	\\
3.3e-10	0	\\
4.4e-10	0	\\
5.4e-10	0	\\
6.5e-10	0	\\
7.5e-10	0	\\
8.6e-10	0	\\
9.6e-10	0	\\
1.07e-09	0	\\
1.18e-09	0	\\
1.28e-09	0	\\
1.38e-09	0	\\
1.49e-09	0	\\
1.59e-09	0	\\
1.69e-09	0	\\
1.8e-09	0	\\
1.9e-09	0	\\
2.01e-09	0	\\
2.11e-09	0	\\
2.21e-09	0	\\
2.32e-09	0	\\
2.42e-09	0	\\
2.52e-09	0	\\
2.63e-09	0	\\
2.73e-09	0	\\
2.83e-09	0	\\
2.93e-09	0	\\
3.04e-09	0	\\
3.14e-09	0	\\
3.24e-09	0	\\
3.34e-09	0	\\
3.45e-09	0	\\
3.55e-09	0	\\
3.65e-09	0	\\
3.75e-09	0	\\
3.86e-09	0	\\
3.96e-09	0	\\
4.06e-09	0	\\
4.16e-09	0	\\
4.27e-09	0	\\
4.37e-09	0	\\
4.47e-09	0	\\
4.57e-09	0	\\
4.68e-09	0	\\
4.78e-09	0	\\
4.89e-09	0	\\
4.99e-09	0	\\
5e-09	0	\\
};
\addplot [color=mycolor2,solid,forget plot]
  table[row sep=crcr]{
0	0	\\
1.1e-10	0	\\
2.2e-10	0	\\
3.3e-10	0	\\
4.4e-10	0	\\
5.4e-10	0	\\
6.5e-10	0	\\
7.5e-10	0	\\
8.6e-10	0	\\
9.6e-10	0	\\
1.07e-09	0	\\
1.18e-09	0	\\
1.28e-09	0	\\
1.38e-09	0	\\
1.49e-09	0	\\
1.59e-09	0	\\
1.69e-09	0	\\
1.8e-09	0	\\
1.9e-09	0	\\
2.01e-09	0	\\
2.11e-09	0	\\
2.21e-09	0	\\
2.32e-09	0	\\
2.42e-09	0	\\
2.52e-09	0	\\
2.63e-09	0	\\
2.73e-09	0	\\
2.83e-09	0	\\
2.93e-09	0	\\
3.04e-09	0	\\
3.14e-09	0	\\
3.24e-09	0	\\
3.34e-09	0	\\
3.45e-09	0	\\
3.55e-09	0	\\
3.65e-09	0	\\
3.75e-09	0	\\
3.86e-09	0	\\
3.96e-09	0	\\
4.06e-09	0	\\
4.16e-09	0	\\
4.27e-09	0	\\
4.37e-09	0	\\
4.47e-09	0	\\
4.57e-09	0	\\
4.68e-09	0	\\
4.78e-09	0	\\
4.89e-09	0	\\
4.99e-09	0	\\
5e-09	0	\\
};
\addplot [color=mycolor3,solid,forget plot]
  table[row sep=crcr]{
0	0	\\
1.1e-10	0	\\
2.2e-10	0	\\
3.3e-10	0	\\
4.4e-10	0	\\
5.4e-10	0	\\
6.5e-10	0	\\
7.5e-10	0	\\
8.6e-10	0	\\
9.6e-10	0	\\
1.07e-09	0	\\
1.18e-09	0	\\
1.28e-09	0	\\
1.38e-09	0	\\
1.49e-09	0	\\
1.59e-09	0	\\
1.69e-09	0	\\
1.8e-09	0	\\
1.9e-09	0	\\
2.01e-09	0	\\
2.11e-09	0	\\
2.21e-09	0	\\
2.32e-09	0	\\
2.42e-09	0	\\
2.52e-09	0	\\
2.63e-09	0	\\
2.73e-09	0	\\
2.83e-09	0	\\
2.93e-09	0	\\
3.04e-09	0	\\
3.14e-09	0	\\
3.24e-09	0	\\
3.34e-09	0	\\
3.45e-09	0	\\
3.55e-09	0	\\
3.65e-09	0	\\
3.75e-09	0	\\
3.86e-09	0	\\
3.96e-09	0	\\
4.06e-09	0	\\
4.16e-09	0	\\
4.27e-09	0	\\
4.37e-09	0	\\
4.47e-09	0	\\
4.57e-09	0	\\
4.68e-09	0	\\
4.78e-09	0	\\
4.89e-09	0	\\
4.99e-09	0	\\
5e-09	0	\\
};
\addplot [color=darkgray,solid,forget plot]
  table[row sep=crcr]{
0	0	\\
1.1e-10	0	\\
2.2e-10	0	\\
3.3e-10	0	\\
4.4e-10	0	\\
5.4e-10	0	\\
6.5e-10	0	\\
7.5e-10	0	\\
8.6e-10	0	\\
9.6e-10	0	\\
1.07e-09	0	\\
1.18e-09	0	\\
1.28e-09	0	\\
1.38e-09	0	\\
1.49e-09	0	\\
1.59e-09	0	\\
1.69e-09	0	\\
1.8e-09	0	\\
1.9e-09	0	\\
2.01e-09	0	\\
2.11e-09	0	\\
2.21e-09	0	\\
2.32e-09	0	\\
2.42e-09	0	\\
2.52e-09	0	\\
2.63e-09	0	\\
2.73e-09	0	\\
2.83e-09	0	\\
2.93e-09	0	\\
3.04e-09	0	\\
3.14e-09	0	\\
3.24e-09	0	\\
3.34e-09	0	\\
3.45e-09	0	\\
3.55e-09	0	\\
3.65e-09	0	\\
3.75e-09	0	\\
3.86e-09	0	\\
3.96e-09	0	\\
4.06e-09	0	\\
4.16e-09	0	\\
4.27e-09	0	\\
4.37e-09	0	\\
4.47e-09	0	\\
4.57e-09	0	\\
4.68e-09	0	\\
4.78e-09	0	\\
4.89e-09	0	\\
4.99e-09	0	\\
5e-09	0	\\
};
\addplot [color=blue,solid,forget plot]
  table[row sep=crcr]{
0	0	\\
1.1e-10	0	\\
2.2e-10	0	\\
3.3e-10	0	\\
4.4e-10	0	\\
5.4e-10	0	\\
6.5e-10	0	\\
7.5e-10	0	\\
8.6e-10	0	\\
9.6e-10	0	\\
1.07e-09	0	\\
1.18e-09	0	\\
1.28e-09	0	\\
1.38e-09	0	\\
1.49e-09	0	\\
1.59e-09	0	\\
1.69e-09	0	\\
1.8e-09	0	\\
1.9e-09	0	\\
2.01e-09	0	\\
2.11e-09	0	\\
2.21e-09	0	\\
2.32e-09	0	\\
2.42e-09	0	\\
2.52e-09	0	\\
2.63e-09	0	\\
2.73e-09	0	\\
2.83e-09	0	\\
2.93e-09	0	\\
3.04e-09	0	\\
3.14e-09	0	\\
3.24e-09	0	\\
3.34e-09	0	\\
3.45e-09	0	\\
3.55e-09	0	\\
3.65e-09	0	\\
3.75e-09	0	\\
3.86e-09	0	\\
3.96e-09	0	\\
4.06e-09	0	\\
4.16e-09	0	\\
4.27e-09	0	\\
4.37e-09	0	\\
4.47e-09	0	\\
4.57e-09	0	\\
4.68e-09	0	\\
4.78e-09	0	\\
4.89e-09	0	\\
4.99e-09	0	\\
5e-09	0	\\
};
\addplot [color=black!50!green,solid,forget plot]
  table[row sep=crcr]{
0	0	\\
1.1e-10	0	\\
2.2e-10	0	\\
3.3e-10	0	\\
4.4e-10	0	\\
5.4e-10	0	\\
6.5e-10	0	\\
7.5e-10	0	\\
8.6e-10	0	\\
9.6e-10	0	\\
1.07e-09	0	\\
1.18e-09	0	\\
1.28e-09	0	\\
1.38e-09	0	\\
1.49e-09	0	\\
1.59e-09	0	\\
1.69e-09	0	\\
1.8e-09	0	\\
1.9e-09	0	\\
2.01e-09	0	\\
2.11e-09	0	\\
2.21e-09	0	\\
2.32e-09	0	\\
2.42e-09	0	\\
2.52e-09	0	\\
2.63e-09	0	\\
2.73e-09	0	\\
2.83e-09	0	\\
2.93e-09	0	\\
3.04e-09	0	\\
3.14e-09	0	\\
3.24e-09	0	\\
3.34e-09	0	\\
3.45e-09	0	\\
3.55e-09	0	\\
3.65e-09	0	\\
3.75e-09	0	\\
3.86e-09	0	\\
3.96e-09	0	\\
4.06e-09	0	\\
4.16e-09	0	\\
4.27e-09	0	\\
4.37e-09	0	\\
4.47e-09	0	\\
4.57e-09	0	\\
4.68e-09	0	\\
4.78e-09	0	\\
4.89e-09	0	\\
4.99e-09	0	\\
5e-09	0	\\
};
\addplot [color=red,solid,forget plot]
  table[row sep=crcr]{
0	0	\\
1.1e-10	0	\\
2.2e-10	0	\\
3.3e-10	0	\\
4.4e-10	0	\\
5.4e-10	0	\\
6.5e-10	0	\\
7.5e-10	0	\\
8.6e-10	0	\\
9.6e-10	0	\\
1.07e-09	0	\\
1.18e-09	0	\\
1.28e-09	0	\\
1.38e-09	0	\\
1.49e-09	0	\\
1.59e-09	0	\\
1.69e-09	0	\\
1.8e-09	0	\\
1.9e-09	0	\\
2.01e-09	0	\\
2.11e-09	0	\\
2.21e-09	0	\\
2.32e-09	0	\\
2.42e-09	0	\\
2.52e-09	0	\\
2.63e-09	0	\\
2.73e-09	0	\\
2.83e-09	0	\\
2.93e-09	0	\\
3.04e-09	0	\\
3.14e-09	0	\\
3.24e-09	0	\\
3.34e-09	0	\\
3.45e-09	0	\\
3.55e-09	0	\\
3.65e-09	0	\\
3.75e-09	0	\\
3.86e-09	0	\\
3.96e-09	0	\\
4.06e-09	0	\\
4.16e-09	0	\\
4.27e-09	0	\\
4.37e-09	0	\\
4.47e-09	0	\\
4.57e-09	0	\\
4.68e-09	0	\\
4.78e-09	0	\\
4.89e-09	0	\\
4.99e-09	0	\\
5e-09	0	\\
};
\addplot [color=mycolor1,solid,forget plot]
  table[row sep=crcr]{
0	0	\\
1.1e-10	0	\\
2.2e-10	0	\\
3.3e-10	0	\\
4.4e-10	0	\\
5.4e-10	0	\\
6.5e-10	0	\\
7.5e-10	0	\\
8.6e-10	0	\\
9.6e-10	0	\\
1.07e-09	0	\\
1.18e-09	0	\\
1.28e-09	0	\\
1.38e-09	0	\\
1.49e-09	0	\\
1.59e-09	0	\\
1.69e-09	0	\\
1.8e-09	0	\\
1.9e-09	0	\\
2.01e-09	0	\\
2.11e-09	0	\\
2.21e-09	0	\\
2.32e-09	0	\\
2.42e-09	0	\\
2.52e-09	0	\\
2.63e-09	0	\\
2.73e-09	0	\\
2.83e-09	0	\\
2.93e-09	0	\\
3.04e-09	0	\\
3.14e-09	0	\\
3.24e-09	0	\\
3.34e-09	0	\\
3.45e-09	0	\\
3.55e-09	0	\\
3.65e-09	0	\\
3.75e-09	0	\\
3.86e-09	0	\\
3.96e-09	0	\\
4.06e-09	0	\\
4.16e-09	0	\\
4.27e-09	0	\\
4.37e-09	0	\\
4.47e-09	0	\\
4.57e-09	0	\\
4.68e-09	0	\\
4.78e-09	0	\\
4.89e-09	0	\\
4.99e-09	0	\\
5e-09	0	\\
};
\addplot [color=mycolor2,solid,forget plot]
  table[row sep=crcr]{
0	0	\\
1.1e-10	0	\\
2.2e-10	0	\\
3.3e-10	0	\\
4.4e-10	0	\\
5.4e-10	0	\\
6.5e-10	0	\\
7.5e-10	0	\\
8.6e-10	0	\\
9.6e-10	0	\\
1.07e-09	0	\\
1.18e-09	0	\\
1.28e-09	0	\\
1.38e-09	0	\\
1.49e-09	0	\\
1.59e-09	0	\\
1.69e-09	0	\\
1.8e-09	0	\\
1.9e-09	0	\\
2.01e-09	0	\\
2.11e-09	0	\\
2.21e-09	0	\\
2.32e-09	0	\\
2.42e-09	0	\\
2.52e-09	0	\\
2.63e-09	0	\\
2.73e-09	0	\\
2.83e-09	0	\\
2.93e-09	0	\\
3.04e-09	0	\\
3.14e-09	0	\\
3.24e-09	0	\\
3.34e-09	0	\\
3.45e-09	0	\\
3.55e-09	0	\\
3.65e-09	0	\\
3.75e-09	0	\\
3.86e-09	0	\\
3.96e-09	0	\\
4.06e-09	0	\\
4.16e-09	0	\\
4.27e-09	0	\\
4.37e-09	0	\\
4.47e-09	0	\\
4.57e-09	0	\\
4.68e-09	0	\\
4.78e-09	0	\\
4.89e-09	0	\\
4.99e-09	0	\\
5e-09	0	\\
};
\addplot [color=mycolor3,solid,forget plot]
  table[row sep=crcr]{
0	0	\\
1.1e-10	0	\\
2.2e-10	0	\\
3.3e-10	0	\\
4.4e-10	0	\\
5.4e-10	0	\\
6.5e-10	0	\\
7.5e-10	0	\\
8.6e-10	0	\\
9.6e-10	0	\\
1.07e-09	0	\\
1.18e-09	0	\\
1.28e-09	0	\\
1.38e-09	0	\\
1.49e-09	0	\\
1.59e-09	0	\\
1.69e-09	0	\\
1.8e-09	0	\\
1.9e-09	0	\\
2.01e-09	0	\\
2.11e-09	0	\\
2.21e-09	0	\\
2.32e-09	0	\\
2.42e-09	0	\\
2.52e-09	0	\\
2.63e-09	0	\\
2.73e-09	0	\\
2.83e-09	0	\\
2.93e-09	0	\\
3.04e-09	0	\\
3.14e-09	0	\\
3.24e-09	0	\\
3.34e-09	0	\\
3.45e-09	0	\\
3.55e-09	0	\\
3.65e-09	0	\\
3.75e-09	0	\\
3.86e-09	0	\\
3.96e-09	0	\\
4.06e-09	0	\\
4.16e-09	0	\\
4.27e-09	0	\\
4.37e-09	0	\\
4.47e-09	0	\\
4.57e-09	0	\\
4.68e-09	0	\\
4.78e-09	0	\\
4.89e-09	0	\\
4.99e-09	0	\\
5e-09	0	\\
};
\addplot [color=darkgray,solid,forget plot]
  table[row sep=crcr]{
0	0	\\
1.1e-10	0	\\
2.2e-10	0	\\
3.3e-10	0	\\
4.4e-10	0	\\
5.4e-10	0	\\
6.5e-10	0	\\
7.5e-10	0	\\
8.6e-10	0	\\
9.6e-10	0	\\
1.07e-09	0	\\
1.18e-09	0	\\
1.28e-09	0	\\
1.38e-09	0	\\
1.49e-09	0	\\
1.59e-09	0	\\
1.69e-09	0	\\
1.8e-09	0	\\
1.9e-09	0	\\
2.01e-09	0	\\
2.11e-09	0	\\
2.21e-09	0	\\
2.32e-09	0	\\
2.42e-09	0	\\
2.52e-09	0	\\
2.63e-09	0	\\
2.73e-09	0	\\
2.83e-09	0	\\
2.93e-09	0	\\
3.04e-09	0	\\
3.14e-09	0	\\
3.24e-09	0	\\
3.34e-09	0	\\
3.45e-09	0	\\
3.55e-09	0	\\
3.65e-09	0	\\
3.75e-09	0	\\
3.86e-09	0	\\
3.96e-09	0	\\
4.06e-09	0	\\
4.16e-09	0	\\
4.27e-09	0	\\
4.37e-09	0	\\
4.47e-09	0	\\
4.57e-09	0	\\
4.68e-09	0	\\
4.78e-09	0	\\
4.89e-09	0	\\
4.99e-09	0	\\
5e-09	0	\\
};
\addplot [color=blue,solid,forget plot]
  table[row sep=crcr]{
0	0	\\
1.1e-10	0	\\
2.2e-10	0	\\
3.3e-10	0	\\
4.4e-10	0	\\
5.4e-10	0	\\
6.5e-10	0	\\
7.5e-10	0	\\
8.6e-10	0	\\
9.6e-10	0	\\
1.07e-09	0	\\
1.18e-09	0	\\
1.28e-09	0	\\
1.38e-09	0	\\
1.49e-09	0	\\
1.59e-09	0	\\
1.69e-09	0	\\
1.8e-09	0	\\
1.9e-09	0	\\
2.01e-09	0	\\
2.11e-09	0	\\
2.21e-09	0	\\
2.32e-09	0	\\
2.42e-09	0	\\
2.52e-09	0	\\
2.63e-09	0	\\
2.73e-09	0	\\
2.83e-09	0	\\
2.93e-09	0	\\
3.04e-09	0	\\
3.14e-09	0	\\
3.24e-09	0	\\
3.34e-09	0	\\
3.45e-09	0	\\
3.55e-09	0	\\
3.65e-09	0	\\
3.75e-09	0	\\
3.86e-09	0	\\
3.96e-09	0	\\
4.06e-09	0	\\
4.16e-09	0	\\
4.27e-09	0	\\
4.37e-09	0	\\
4.47e-09	0	\\
4.57e-09	0	\\
4.68e-09	0	\\
4.78e-09	0	\\
4.89e-09	0	\\
4.99e-09	0	\\
5e-09	0	\\
};
\addplot [color=black!50!green,solid,forget plot]
  table[row sep=crcr]{
0	0	\\
1.1e-10	0	\\
2.2e-10	0	\\
3.3e-10	0	\\
4.4e-10	0	\\
5.4e-10	0	\\
6.5e-10	0	\\
7.5e-10	0	\\
8.6e-10	0	\\
9.6e-10	0	\\
1.07e-09	0	\\
1.18e-09	0	\\
1.28e-09	0	\\
1.38e-09	0	\\
1.49e-09	0	\\
1.59e-09	0	\\
1.69e-09	0	\\
1.8e-09	0	\\
1.9e-09	0	\\
2.01e-09	0	\\
2.11e-09	0	\\
2.21e-09	0	\\
2.32e-09	0	\\
2.42e-09	0	\\
2.52e-09	0	\\
2.63e-09	0	\\
2.73e-09	0	\\
2.83e-09	0	\\
2.93e-09	0	\\
3.04e-09	0	\\
3.14e-09	0	\\
3.24e-09	0	\\
3.34e-09	0	\\
3.45e-09	0	\\
3.55e-09	0	\\
3.65e-09	0	\\
3.75e-09	0	\\
3.86e-09	0	\\
3.96e-09	0	\\
4.06e-09	0	\\
4.16e-09	0	\\
4.27e-09	0	\\
4.37e-09	0	\\
4.47e-09	0	\\
4.57e-09	0	\\
4.68e-09	0	\\
4.78e-09	0	\\
4.89e-09	0	\\
4.99e-09	0	\\
5e-09	0	\\
};
\addplot [color=red,solid,forget plot]
  table[row sep=crcr]{
0	0	\\
1.1e-10	0	\\
2.2e-10	0	\\
3.3e-10	0	\\
4.4e-10	0	\\
5.4e-10	0	\\
6.5e-10	0	\\
7.5e-10	0	\\
8.6e-10	0	\\
9.6e-10	0	\\
1.07e-09	0	\\
1.18e-09	0	\\
1.28e-09	0	\\
1.38e-09	0	\\
1.49e-09	0	\\
1.59e-09	0	\\
1.69e-09	0	\\
1.8e-09	0	\\
1.9e-09	0	\\
2.01e-09	0	\\
2.11e-09	0	\\
2.21e-09	0	\\
2.32e-09	0	\\
2.42e-09	0	\\
2.52e-09	0	\\
2.63e-09	0	\\
2.73e-09	0	\\
2.83e-09	0	\\
2.93e-09	0	\\
3.04e-09	0	\\
3.14e-09	0	\\
3.24e-09	0	\\
3.34e-09	0	\\
3.45e-09	0	\\
3.55e-09	0	\\
3.65e-09	0	\\
3.75e-09	0	\\
3.86e-09	0	\\
3.96e-09	0	\\
4.06e-09	0	\\
4.16e-09	0	\\
4.27e-09	0	\\
4.37e-09	0	\\
4.47e-09	0	\\
4.57e-09	0	\\
4.68e-09	0	\\
4.78e-09	0	\\
4.89e-09	0	\\
4.99e-09	0	\\
5e-09	0	\\
};
\addplot [color=mycolor1,solid,forget plot]
  table[row sep=crcr]{
0	0	\\
1.1e-10	0	\\
2.2e-10	0	\\
3.3e-10	0	\\
4.4e-10	0	\\
5.4e-10	0	\\
6.5e-10	0	\\
7.5e-10	0	\\
8.6e-10	0	\\
9.6e-10	0	\\
1.07e-09	0	\\
1.18e-09	0	\\
1.28e-09	0	\\
1.38e-09	0	\\
1.49e-09	0	\\
1.59e-09	0	\\
1.69e-09	0	\\
1.8e-09	0	\\
1.9e-09	0	\\
2.01e-09	0	\\
2.11e-09	0	\\
2.21e-09	0	\\
2.32e-09	0	\\
2.42e-09	0	\\
2.52e-09	0	\\
2.63e-09	0	\\
2.73e-09	0	\\
2.83e-09	0	\\
2.93e-09	0	\\
3.04e-09	0	\\
3.14e-09	0	\\
3.24e-09	0	\\
3.34e-09	0	\\
3.45e-09	0	\\
3.55e-09	0	\\
3.65e-09	0	\\
3.75e-09	0	\\
3.86e-09	0	\\
3.96e-09	0	\\
4.06e-09	0	\\
4.16e-09	0	\\
4.27e-09	0	\\
4.37e-09	0	\\
4.47e-09	0	\\
4.57e-09	0	\\
4.68e-09	0	\\
4.78e-09	0	\\
4.89e-09	0	\\
4.99e-09	0	\\
5e-09	0	\\
};
\addplot [color=mycolor2,solid,forget plot]
  table[row sep=crcr]{
0	0	\\
1.1e-10	0	\\
2.2e-10	0	\\
3.3e-10	0	\\
4.4e-10	0	\\
5.4e-10	0	\\
6.5e-10	0	\\
7.5e-10	0	\\
8.6e-10	0	\\
9.6e-10	0	\\
1.07e-09	0	\\
1.18e-09	0	\\
1.28e-09	0	\\
1.38e-09	0	\\
1.49e-09	0	\\
1.59e-09	0	\\
1.69e-09	0	\\
1.8e-09	0	\\
1.9e-09	0	\\
2.01e-09	0	\\
2.11e-09	0	\\
2.21e-09	0	\\
2.32e-09	0	\\
2.42e-09	0	\\
2.52e-09	0	\\
2.63e-09	0	\\
2.73e-09	0	\\
2.83e-09	0	\\
2.93e-09	0	\\
3.04e-09	0	\\
3.14e-09	0	\\
3.24e-09	0	\\
3.34e-09	0	\\
3.45e-09	0	\\
3.55e-09	0	\\
3.65e-09	0	\\
3.75e-09	0	\\
3.86e-09	0	\\
3.96e-09	0	\\
4.06e-09	0	\\
4.16e-09	0	\\
4.27e-09	0	\\
4.37e-09	0	\\
4.47e-09	0	\\
4.57e-09	0	\\
4.68e-09	0	\\
4.78e-09	0	\\
4.89e-09	0	\\
4.99e-09	0	\\
5e-09	0	\\
};
\addplot [color=mycolor3,solid,forget plot]
  table[row sep=crcr]{
0	0	\\
1.1e-10	0	\\
2.2e-10	0	\\
3.3e-10	0	\\
4.4e-10	0	\\
5.4e-10	0	\\
6.5e-10	0	\\
7.5e-10	0	\\
8.6e-10	0	\\
9.6e-10	0	\\
1.07e-09	0	\\
1.18e-09	0	\\
1.28e-09	0	\\
1.38e-09	0	\\
1.49e-09	0	\\
1.59e-09	0	\\
1.69e-09	0	\\
1.8e-09	0	\\
1.9e-09	0	\\
2.01e-09	0	\\
2.11e-09	0	\\
2.21e-09	0	\\
2.32e-09	0	\\
2.42e-09	0	\\
2.52e-09	0	\\
2.63e-09	0	\\
2.73e-09	0	\\
2.83e-09	0	\\
2.93e-09	0	\\
3.04e-09	0	\\
3.14e-09	0	\\
3.24e-09	0	\\
3.34e-09	0	\\
3.45e-09	0	\\
3.55e-09	0	\\
3.65e-09	0	\\
3.75e-09	0	\\
3.86e-09	0	\\
3.96e-09	0	\\
4.06e-09	0	\\
4.16e-09	0	\\
4.27e-09	0	\\
4.37e-09	0	\\
4.47e-09	0	\\
4.57e-09	0	\\
4.68e-09	0	\\
4.78e-09	0	\\
4.89e-09	0	\\
4.99e-09	0	\\
5e-09	0	\\
};
\addplot [color=darkgray,solid,forget plot]
  table[row sep=crcr]{
0	0	\\
1.1e-10	0	\\
2.2e-10	0	\\
3.3e-10	0	\\
4.4e-10	0	\\
5.4e-10	0	\\
6.5e-10	0	\\
7.5e-10	0	\\
8.6e-10	0	\\
9.6e-10	0	\\
1.07e-09	0	\\
1.18e-09	0	\\
1.28e-09	0	\\
1.38e-09	0	\\
1.49e-09	0	\\
1.59e-09	0	\\
1.69e-09	0	\\
1.8e-09	0	\\
1.9e-09	0	\\
2.01e-09	0	\\
2.11e-09	0	\\
2.21e-09	0	\\
2.32e-09	0	\\
2.42e-09	0	\\
2.52e-09	0	\\
2.63e-09	0	\\
2.73e-09	0	\\
2.83e-09	0	\\
2.93e-09	0	\\
3.04e-09	0	\\
3.14e-09	0	\\
3.24e-09	0	\\
3.34e-09	0	\\
3.45e-09	0	\\
3.55e-09	0	\\
3.65e-09	0	\\
3.75e-09	0	\\
3.86e-09	0	\\
3.96e-09	0	\\
4.06e-09	0	\\
4.16e-09	0	\\
4.27e-09	0	\\
4.37e-09	0	\\
4.47e-09	0	\\
4.57e-09	0	\\
4.68e-09	0	\\
4.78e-09	0	\\
4.89e-09	0	\\
4.99e-09	0	\\
5e-09	0	\\
};
\addplot [color=blue,solid,forget plot]
  table[row sep=crcr]{
0	0	\\
1.1e-10	0	\\
2.2e-10	0	\\
3.3e-10	0	\\
4.4e-10	0	\\
5.4e-10	0	\\
6.5e-10	0	\\
7.5e-10	0	\\
8.6e-10	0	\\
9.6e-10	0	\\
1.07e-09	0	\\
1.18e-09	0	\\
1.28e-09	0	\\
1.38e-09	0	\\
1.49e-09	0	\\
1.59e-09	0	\\
1.69e-09	0	\\
1.8e-09	0	\\
1.9e-09	0	\\
2.01e-09	0	\\
2.11e-09	0	\\
2.21e-09	0	\\
2.32e-09	0	\\
2.42e-09	0	\\
2.52e-09	0	\\
2.63e-09	0	\\
2.73e-09	0	\\
2.83e-09	0	\\
2.93e-09	0	\\
3.04e-09	0	\\
3.14e-09	0	\\
3.24e-09	0	\\
3.34e-09	0	\\
3.45e-09	0	\\
3.55e-09	0	\\
3.65e-09	0	\\
3.75e-09	0	\\
3.86e-09	0	\\
3.96e-09	0	\\
4.06e-09	0	\\
4.16e-09	0	\\
4.27e-09	0	\\
4.37e-09	0	\\
4.47e-09	0	\\
4.57e-09	0	\\
4.68e-09	0	\\
4.78e-09	0	\\
4.89e-09	0	\\
4.99e-09	0	\\
5e-09	0	\\
};
\addplot [color=black!50!green,solid,forget plot]
  table[row sep=crcr]{
0	0	\\
1.1e-10	0	\\
2.2e-10	0	\\
3.3e-10	0	\\
4.4e-10	0	\\
5.4e-10	0	\\
6.5e-10	0	\\
7.5e-10	0	\\
8.6e-10	0	\\
9.6e-10	0	\\
1.07e-09	0	\\
1.18e-09	0	\\
1.28e-09	0	\\
1.38e-09	0	\\
1.49e-09	0	\\
1.59e-09	0	\\
1.69e-09	0	\\
1.8e-09	0	\\
1.9e-09	0	\\
2.01e-09	0	\\
2.11e-09	0	\\
2.21e-09	0	\\
2.32e-09	0	\\
2.42e-09	0	\\
2.52e-09	0	\\
2.63e-09	0	\\
2.73e-09	0	\\
2.83e-09	0	\\
2.93e-09	0	\\
3.04e-09	0	\\
3.14e-09	0	\\
3.24e-09	0	\\
3.34e-09	0	\\
3.45e-09	0	\\
3.55e-09	0	\\
3.65e-09	0	\\
3.75e-09	0	\\
3.86e-09	0	\\
3.96e-09	0	\\
4.06e-09	0	\\
4.16e-09	0	\\
4.27e-09	0	\\
4.37e-09	0	\\
4.47e-09	0	\\
4.57e-09	0	\\
4.68e-09	0	\\
4.78e-09	0	\\
4.89e-09	0	\\
4.99e-09	0	\\
5e-09	0	\\
};
\addplot [color=red,solid,forget plot]
  table[row sep=crcr]{
0	0	\\
1.1e-10	0	\\
2.2e-10	0	\\
3.3e-10	0	\\
4.4e-10	0	\\
5.4e-10	0	\\
6.5e-10	0	\\
7.5e-10	0	\\
8.6e-10	0	\\
9.6e-10	0	\\
1.07e-09	0	\\
1.18e-09	0	\\
1.28e-09	0	\\
1.38e-09	0	\\
1.49e-09	0	\\
1.59e-09	0	\\
1.69e-09	0	\\
1.8e-09	0	\\
1.9e-09	0	\\
2.01e-09	0	\\
2.11e-09	0	\\
2.21e-09	0	\\
2.32e-09	0	\\
2.42e-09	0	\\
2.52e-09	0	\\
2.63e-09	0	\\
2.73e-09	0	\\
2.83e-09	0	\\
2.93e-09	0	\\
3.04e-09	0	\\
3.14e-09	0	\\
3.24e-09	0	\\
3.34e-09	0	\\
3.45e-09	0	\\
3.55e-09	0	\\
3.65e-09	0	\\
3.75e-09	0	\\
3.86e-09	0	\\
3.96e-09	0	\\
4.06e-09	0	\\
4.16e-09	0	\\
4.27e-09	0	\\
4.37e-09	0	\\
4.47e-09	0	\\
4.57e-09	0	\\
4.68e-09	0	\\
4.78e-09	0	\\
4.89e-09	0	\\
4.99e-09	0	\\
5e-09	0	\\
};
\addplot [color=mycolor1,solid,forget plot]
  table[row sep=crcr]{
0	0	\\
1.1e-10	0	\\
2.2e-10	0	\\
3.3e-10	0	\\
4.4e-10	0	\\
5.4e-10	0	\\
6.5e-10	0	\\
7.5e-10	0	\\
8.6e-10	0	\\
9.6e-10	0	\\
1.07e-09	0	\\
1.18e-09	0	\\
1.28e-09	0	\\
1.38e-09	0	\\
1.49e-09	0	\\
1.59e-09	0	\\
1.69e-09	0	\\
1.8e-09	0	\\
1.9e-09	0	\\
2.01e-09	0	\\
2.11e-09	0	\\
2.21e-09	0	\\
2.32e-09	0	\\
2.42e-09	0	\\
2.52e-09	0	\\
2.63e-09	0	\\
2.73e-09	0	\\
2.83e-09	0	\\
2.93e-09	0	\\
3.04e-09	0	\\
3.14e-09	0	\\
3.24e-09	0	\\
3.34e-09	0	\\
3.45e-09	0	\\
3.55e-09	0	\\
3.65e-09	0	\\
3.75e-09	0	\\
3.86e-09	0	\\
3.96e-09	0	\\
4.06e-09	0	\\
4.16e-09	0	\\
4.27e-09	0	\\
4.37e-09	0	\\
4.47e-09	0	\\
4.57e-09	0	\\
4.68e-09	0	\\
4.78e-09	0	\\
4.89e-09	0	\\
4.99e-09	0	\\
5e-09	0	\\
};
\addplot [color=mycolor2,solid,forget plot]
  table[row sep=crcr]{
0	0	\\
1.1e-10	0	\\
2.2e-10	0	\\
3.3e-10	0	\\
4.4e-10	0	\\
5.4e-10	0	\\
6.5e-10	0	\\
7.5e-10	0	\\
8.6e-10	0	\\
9.6e-10	0	\\
1.07e-09	0	\\
1.18e-09	0	\\
1.28e-09	0	\\
1.38e-09	0	\\
1.49e-09	0	\\
1.59e-09	0	\\
1.69e-09	0	\\
1.8e-09	0	\\
1.9e-09	0	\\
2.01e-09	0	\\
2.11e-09	0	\\
2.21e-09	0	\\
2.32e-09	0	\\
2.42e-09	0	\\
2.52e-09	0	\\
2.63e-09	0	\\
2.73e-09	0	\\
2.83e-09	0	\\
2.93e-09	0	\\
3.04e-09	0	\\
3.14e-09	0	\\
3.24e-09	0	\\
3.34e-09	0	\\
3.45e-09	0	\\
3.55e-09	0	\\
3.65e-09	0	\\
3.75e-09	0	\\
3.86e-09	0	\\
3.96e-09	0	\\
4.06e-09	0	\\
4.16e-09	0	\\
4.27e-09	0	\\
4.37e-09	0	\\
4.47e-09	0	\\
4.57e-09	0	\\
4.68e-09	0	\\
4.78e-09	0	\\
4.89e-09	0	\\
4.99e-09	0	\\
5e-09	0	\\
};
\addplot [color=mycolor3,solid,forget plot]
  table[row sep=crcr]{
0	0	\\
1.1e-10	0	\\
2.2e-10	0	\\
3.3e-10	0	\\
4.4e-10	0	\\
5.4e-10	0	\\
6.5e-10	0	\\
7.5e-10	0	\\
8.6e-10	0	\\
9.6e-10	0	\\
1.07e-09	0	\\
1.18e-09	0	\\
1.28e-09	0	\\
1.38e-09	0	\\
1.49e-09	0	\\
1.59e-09	0	\\
1.69e-09	0	\\
1.8e-09	0	\\
1.9e-09	0	\\
2.01e-09	0	\\
2.11e-09	0	\\
2.21e-09	0	\\
2.32e-09	0	\\
2.42e-09	0	\\
2.52e-09	0	\\
2.63e-09	0	\\
2.73e-09	0	\\
2.83e-09	0	\\
2.93e-09	0	\\
3.04e-09	0	\\
3.14e-09	0	\\
3.24e-09	0	\\
3.34e-09	0	\\
3.45e-09	0	\\
3.55e-09	0	\\
3.65e-09	0	\\
3.75e-09	0	\\
3.86e-09	0	\\
3.96e-09	0	\\
4.06e-09	0	\\
4.16e-09	0	\\
4.27e-09	0	\\
4.37e-09	0	\\
4.47e-09	0	\\
4.57e-09	0	\\
4.68e-09	0	\\
4.78e-09	0	\\
4.89e-09	0	\\
4.99e-09	0	\\
5e-09	0	\\
};
\addplot [color=darkgray,solid,forget plot]
  table[row sep=crcr]{
0	0	\\
1.1e-10	0	\\
2.2e-10	0	\\
3.3e-10	0	\\
4.4e-10	0	\\
5.4e-10	0	\\
6.5e-10	0	\\
7.5e-10	0	\\
8.6e-10	0	\\
9.6e-10	0	\\
1.07e-09	0	\\
1.18e-09	0	\\
1.28e-09	0	\\
1.38e-09	0	\\
1.49e-09	0	\\
1.59e-09	0	\\
1.69e-09	0	\\
1.8e-09	0	\\
1.9e-09	0	\\
2.01e-09	0	\\
2.11e-09	0	\\
2.21e-09	0	\\
2.32e-09	0	\\
2.42e-09	0	\\
2.52e-09	0	\\
2.63e-09	0	\\
2.73e-09	0	\\
2.83e-09	0	\\
2.93e-09	0	\\
3.04e-09	0	\\
3.14e-09	0	\\
3.24e-09	0	\\
3.34e-09	0	\\
3.45e-09	0	\\
3.55e-09	0	\\
3.65e-09	0	\\
3.75e-09	0	\\
3.86e-09	0	\\
3.96e-09	0	\\
4.06e-09	0	\\
4.16e-09	0	\\
4.27e-09	0	\\
4.37e-09	0	\\
4.47e-09	0	\\
4.57e-09	0	\\
4.68e-09	0	\\
4.78e-09	0	\\
4.89e-09	0	\\
4.99e-09	0	\\
5e-09	0	\\
};
\addplot [color=blue,solid,forget plot]
  table[row sep=crcr]{
0	0	\\
1.1e-10	0	\\
2.2e-10	0	\\
3.3e-10	0	\\
4.4e-10	0	\\
5.4e-10	0	\\
6.5e-10	0	\\
7.5e-10	0	\\
8.6e-10	0	\\
9.6e-10	0	\\
1.07e-09	0	\\
1.18e-09	0	\\
1.28e-09	0	\\
1.38e-09	0	\\
1.49e-09	0	\\
1.59e-09	0	\\
1.69e-09	0	\\
1.8e-09	0	\\
1.9e-09	0	\\
2.01e-09	0	\\
2.11e-09	0	\\
2.21e-09	0	\\
2.32e-09	0	\\
2.42e-09	0	\\
2.52e-09	0	\\
2.63e-09	0	\\
2.73e-09	0	\\
2.83e-09	0	\\
2.93e-09	0	\\
3.04e-09	0	\\
3.14e-09	0	\\
3.24e-09	0	\\
3.34e-09	0	\\
3.45e-09	0	\\
3.55e-09	0	\\
3.65e-09	0	\\
3.75e-09	0	\\
3.86e-09	0	\\
3.96e-09	0	\\
4.06e-09	0	\\
4.16e-09	0	\\
4.27e-09	0	\\
4.37e-09	0	\\
4.47e-09	0	\\
4.57e-09	0	\\
4.68e-09	0	\\
4.78e-09	0	\\
4.89e-09	0	\\
4.99e-09	0	\\
5e-09	0	\\
};
\addplot [color=black!50!green,solid,forget plot]
  table[row sep=crcr]{
0	0	\\
1.1e-10	0	\\
2.2e-10	0	\\
3.3e-10	0	\\
4.4e-10	0	\\
5.4e-10	0	\\
6.5e-10	0	\\
7.5e-10	0	\\
8.6e-10	0	\\
9.6e-10	0	\\
1.07e-09	0	\\
1.18e-09	0	\\
1.28e-09	0	\\
1.38e-09	0	\\
1.49e-09	0	\\
1.59e-09	0	\\
1.69e-09	0	\\
1.8e-09	0	\\
1.9e-09	0	\\
2.01e-09	0	\\
2.11e-09	0	\\
2.21e-09	0	\\
2.32e-09	0	\\
2.42e-09	0	\\
2.52e-09	0	\\
2.63e-09	0	\\
2.73e-09	0	\\
2.83e-09	0	\\
2.93e-09	0	\\
3.04e-09	0	\\
3.14e-09	0	\\
3.24e-09	0	\\
3.34e-09	0	\\
3.45e-09	0	\\
3.55e-09	0	\\
3.65e-09	0	\\
3.75e-09	0	\\
3.86e-09	0	\\
3.96e-09	0	\\
4.06e-09	0	\\
4.16e-09	0	\\
4.27e-09	0	\\
4.37e-09	0	\\
4.47e-09	0	\\
4.57e-09	0	\\
4.68e-09	0	\\
4.78e-09	0	\\
4.89e-09	0	\\
4.99e-09	0	\\
5e-09	0	\\
};
\addplot [color=red,solid,forget plot]
  table[row sep=crcr]{
0	0	\\
1.1e-10	0	\\
2.2e-10	0	\\
3.3e-10	0	\\
4.4e-10	0	\\
5.4e-10	0	\\
6.5e-10	0	\\
7.5e-10	0	\\
8.6e-10	0	\\
9.6e-10	0	\\
1.07e-09	0	\\
1.18e-09	0	\\
1.28e-09	0	\\
1.38e-09	0	\\
1.49e-09	0	\\
1.59e-09	0	\\
1.69e-09	0	\\
1.8e-09	0	\\
1.9e-09	0	\\
2.01e-09	0	\\
2.11e-09	0	\\
2.21e-09	0	\\
2.32e-09	0	\\
2.42e-09	0	\\
2.52e-09	0	\\
2.63e-09	0	\\
2.73e-09	0	\\
2.83e-09	0	\\
2.93e-09	0	\\
3.04e-09	0	\\
3.14e-09	0	\\
3.24e-09	0	\\
3.34e-09	0	\\
3.45e-09	0	\\
3.55e-09	0	\\
3.65e-09	0	\\
3.75e-09	0	\\
3.86e-09	0	\\
3.96e-09	0	\\
4.06e-09	0	\\
4.16e-09	0	\\
4.27e-09	0	\\
4.37e-09	0	\\
4.47e-09	0	\\
4.57e-09	0	\\
4.68e-09	0	\\
4.78e-09	0	\\
4.89e-09	0	\\
4.99e-09	0	\\
5e-09	0	\\
};
\addplot [color=mycolor1,solid,forget plot]
  table[row sep=crcr]{
0	0	\\
1.1e-10	0	\\
2.2e-10	0	\\
3.3e-10	0	\\
4.4e-10	0	\\
5.4e-10	0	\\
6.5e-10	0	\\
7.5e-10	0	\\
8.6e-10	0	\\
9.6e-10	0	\\
1.07e-09	0	\\
1.18e-09	0	\\
1.28e-09	0	\\
1.38e-09	0	\\
1.49e-09	0	\\
1.59e-09	0	\\
1.69e-09	0	\\
1.8e-09	0	\\
1.9e-09	0	\\
2.01e-09	0	\\
2.11e-09	0	\\
2.21e-09	0	\\
2.32e-09	0	\\
2.42e-09	0	\\
2.52e-09	0	\\
2.63e-09	0	\\
2.73e-09	0	\\
2.83e-09	0	\\
2.93e-09	0	\\
3.04e-09	0	\\
3.14e-09	0	\\
3.24e-09	0	\\
3.34e-09	0	\\
3.45e-09	0	\\
3.55e-09	0	\\
3.65e-09	0	\\
3.75e-09	0	\\
3.86e-09	0	\\
3.96e-09	0	\\
4.06e-09	0	\\
4.16e-09	0	\\
4.27e-09	0	\\
4.37e-09	0	\\
4.47e-09	0	\\
4.57e-09	0	\\
4.68e-09	0	\\
4.78e-09	0	\\
4.89e-09	0	\\
4.99e-09	0	\\
5e-09	0	\\
};
\addplot [color=mycolor2,solid,forget plot]
  table[row sep=crcr]{
0	0	\\
1.1e-10	0	\\
2.2e-10	0	\\
3.3e-10	0	\\
4.4e-10	0	\\
5.4e-10	0	\\
6.5e-10	0	\\
7.5e-10	0	\\
8.6e-10	0	\\
9.6e-10	0	\\
1.07e-09	0	\\
1.18e-09	0	\\
1.28e-09	0	\\
1.38e-09	0	\\
1.49e-09	0	\\
1.59e-09	0	\\
1.69e-09	0	\\
1.8e-09	0	\\
1.9e-09	0	\\
2.01e-09	0	\\
2.11e-09	0	\\
2.21e-09	0	\\
2.32e-09	0	\\
2.42e-09	0	\\
2.52e-09	0	\\
2.63e-09	0	\\
2.73e-09	0	\\
2.83e-09	0	\\
2.93e-09	0	\\
3.04e-09	0	\\
3.14e-09	0	\\
3.24e-09	0	\\
3.34e-09	0	\\
3.45e-09	0	\\
3.55e-09	0	\\
3.65e-09	0	\\
3.75e-09	0	\\
3.86e-09	0	\\
3.96e-09	0	\\
4.06e-09	0	\\
4.16e-09	0	\\
4.27e-09	0	\\
4.37e-09	0	\\
4.47e-09	0	\\
4.57e-09	0	\\
4.68e-09	0	\\
4.78e-09	0	\\
4.89e-09	0	\\
4.99e-09	0	\\
5e-09	0	\\
};
\addplot [color=mycolor3,solid,forget plot]
  table[row sep=crcr]{
0	0	\\
1.1e-10	0	\\
2.2e-10	0	\\
3.3e-10	0	\\
4.4e-10	0	\\
5.4e-10	0	\\
6.5e-10	0	\\
7.5e-10	0	\\
8.6e-10	0	\\
9.6e-10	0	\\
1.07e-09	0	\\
1.18e-09	0	\\
1.28e-09	0	\\
1.38e-09	0	\\
1.49e-09	0	\\
1.59e-09	0	\\
1.69e-09	0	\\
1.8e-09	0	\\
1.9e-09	0	\\
2.01e-09	0	\\
2.11e-09	0	\\
2.21e-09	0	\\
2.32e-09	0	\\
2.42e-09	0	\\
2.52e-09	0	\\
2.63e-09	0	\\
2.73e-09	0	\\
2.83e-09	0	\\
2.93e-09	0	\\
3.04e-09	0	\\
3.14e-09	0	\\
3.24e-09	0	\\
3.34e-09	0	\\
3.45e-09	0	\\
3.55e-09	0	\\
3.65e-09	0	\\
3.75e-09	0	\\
3.86e-09	0	\\
3.96e-09	0	\\
4.06e-09	0	\\
4.16e-09	0	\\
4.27e-09	0	\\
4.37e-09	0	\\
4.47e-09	0	\\
4.57e-09	0	\\
4.68e-09	0	\\
4.78e-09	0	\\
4.89e-09	0	\\
4.99e-09	0	\\
5e-09	0	\\
};
\addplot [color=darkgray,solid,forget plot]
  table[row sep=crcr]{
0	0	\\
1.1e-10	0	\\
2.2e-10	0	\\
3.3e-10	0	\\
4.4e-10	0	\\
5.4e-10	0	\\
6.5e-10	0	\\
7.5e-10	0	\\
8.6e-10	0	\\
9.6e-10	0	\\
1.07e-09	0	\\
1.18e-09	0	\\
1.28e-09	0	\\
1.38e-09	0	\\
1.49e-09	0	\\
1.59e-09	0	\\
1.69e-09	0	\\
1.8e-09	0	\\
1.9e-09	0	\\
2.01e-09	0	\\
2.11e-09	0	\\
2.21e-09	0	\\
2.32e-09	0	\\
2.42e-09	0	\\
2.52e-09	0	\\
2.63e-09	0	\\
2.73e-09	0	\\
2.83e-09	0	\\
2.93e-09	0	\\
3.04e-09	0	\\
3.14e-09	0	\\
3.24e-09	0	\\
3.34e-09	0	\\
3.45e-09	0	\\
3.55e-09	0	\\
3.65e-09	0	\\
3.75e-09	0	\\
3.86e-09	0	\\
3.96e-09	0	\\
4.06e-09	0	\\
4.16e-09	0	\\
4.27e-09	0	\\
4.37e-09	0	\\
4.47e-09	0	\\
4.57e-09	0	\\
4.68e-09	0	\\
4.78e-09	0	\\
4.89e-09	0	\\
4.99e-09	0	\\
5e-09	0	\\
};
\addplot [color=blue,solid,forget plot]
  table[row sep=crcr]{
0	0	\\
1.1e-10	0	\\
2.2e-10	0	\\
3.3e-10	0	\\
4.4e-10	0	\\
5.4e-10	0	\\
6.5e-10	0	\\
7.5e-10	0	\\
8.6e-10	0	\\
9.6e-10	0	\\
1.07e-09	0	\\
1.18e-09	0	\\
1.28e-09	0	\\
1.38e-09	0	\\
1.49e-09	0	\\
1.59e-09	0	\\
1.69e-09	0	\\
1.8e-09	0	\\
1.9e-09	0	\\
2.01e-09	0	\\
2.11e-09	0	\\
2.21e-09	0	\\
2.32e-09	0	\\
2.42e-09	0	\\
2.52e-09	0	\\
2.63e-09	0	\\
2.73e-09	0	\\
2.83e-09	0	\\
2.93e-09	0	\\
3.04e-09	0	\\
3.14e-09	0	\\
3.24e-09	0	\\
3.34e-09	0	\\
3.45e-09	0	\\
3.55e-09	0	\\
3.65e-09	0	\\
3.75e-09	0	\\
3.86e-09	0	\\
3.96e-09	0	\\
4.06e-09	0	\\
4.16e-09	0	\\
4.27e-09	0	\\
4.37e-09	0	\\
4.47e-09	0	\\
4.57e-09	0	\\
4.68e-09	0	\\
4.78e-09	0	\\
4.89e-09	0	\\
4.99e-09	0	\\
5e-09	0	\\
};
\addplot [color=black!50!green,solid,forget plot]
  table[row sep=crcr]{
0	0	\\
1.1e-10	0	\\
2.2e-10	0	\\
3.3e-10	0	\\
4.4e-10	0	\\
5.4e-10	0	\\
6.5e-10	0	\\
7.5e-10	0	\\
8.6e-10	0	\\
9.6e-10	0	\\
1.07e-09	0	\\
1.18e-09	0	\\
1.28e-09	0	\\
1.38e-09	0	\\
1.49e-09	0	\\
1.59e-09	0	\\
1.69e-09	0	\\
1.8e-09	0	\\
1.9e-09	0	\\
2.01e-09	0	\\
2.11e-09	0	\\
2.21e-09	0	\\
2.32e-09	0	\\
2.42e-09	0	\\
2.52e-09	0	\\
2.63e-09	0	\\
2.73e-09	0	\\
2.83e-09	0	\\
2.93e-09	0	\\
3.04e-09	0	\\
3.14e-09	0	\\
3.24e-09	0	\\
3.34e-09	0	\\
3.45e-09	0	\\
3.55e-09	0	\\
3.65e-09	0	\\
3.75e-09	0	\\
3.86e-09	0	\\
3.96e-09	0	\\
4.06e-09	0	\\
4.16e-09	0	\\
4.27e-09	0	\\
4.37e-09	0	\\
4.47e-09	0	\\
4.57e-09	0	\\
4.68e-09	0	\\
4.78e-09	0	\\
4.89e-09	0	\\
4.99e-09	0	\\
5e-09	0	\\
};
\addplot [color=red,solid,forget plot]
  table[row sep=crcr]{
0	0	\\
1.1e-10	0	\\
2.2e-10	0	\\
3.3e-10	0	\\
4.4e-10	0	\\
5.4e-10	0	\\
6.5e-10	0	\\
7.5e-10	0	\\
8.6e-10	0	\\
9.6e-10	0	\\
1.07e-09	0	\\
1.18e-09	0	\\
1.28e-09	0	\\
1.38e-09	0	\\
1.49e-09	0	\\
1.59e-09	0	\\
1.69e-09	0	\\
1.8e-09	0	\\
1.9e-09	0	\\
2.01e-09	0	\\
2.11e-09	0	\\
2.21e-09	0	\\
2.32e-09	0	\\
2.42e-09	0	\\
2.52e-09	0	\\
2.63e-09	0	\\
2.73e-09	0	\\
2.83e-09	0	\\
2.93e-09	0	\\
3.04e-09	0	\\
3.14e-09	0	\\
3.24e-09	0	\\
3.34e-09	0	\\
3.45e-09	0	\\
3.55e-09	0	\\
3.65e-09	0	\\
3.75e-09	0	\\
3.86e-09	0	\\
3.96e-09	0	\\
4.06e-09	0	\\
4.16e-09	0	\\
4.27e-09	0	\\
4.37e-09	0	\\
4.47e-09	0	\\
4.57e-09	0	\\
4.68e-09	0	\\
4.78e-09	0	\\
4.89e-09	0	\\
4.99e-09	0	\\
5e-09	0	\\
};
\addplot [color=mycolor1,solid,forget plot]
  table[row sep=crcr]{
0	0	\\
1.1e-10	0	\\
2.2e-10	0	\\
3.3e-10	0	\\
4.4e-10	0	\\
5.4e-10	0	\\
6.5e-10	0	\\
7.5e-10	0	\\
8.6e-10	0	\\
9.6e-10	0	\\
1.07e-09	0	\\
1.18e-09	0	\\
1.28e-09	0	\\
1.38e-09	0	\\
1.49e-09	0	\\
1.59e-09	0	\\
1.69e-09	0	\\
1.8e-09	0	\\
1.9e-09	0	\\
2.01e-09	0	\\
2.11e-09	0	\\
2.21e-09	0	\\
2.32e-09	0	\\
2.42e-09	0	\\
2.52e-09	0	\\
2.63e-09	0	\\
2.73e-09	0	\\
2.83e-09	0	\\
2.93e-09	0	\\
3.04e-09	0	\\
3.14e-09	0	\\
3.24e-09	0	\\
3.34e-09	0	\\
3.45e-09	0	\\
3.55e-09	0	\\
3.65e-09	0	\\
3.75e-09	0	\\
3.86e-09	0	\\
3.96e-09	0	\\
4.06e-09	0	\\
4.16e-09	0	\\
4.27e-09	0	\\
4.37e-09	0	\\
4.47e-09	0	\\
4.57e-09	0	\\
4.68e-09	0	\\
4.78e-09	0	\\
4.89e-09	0	\\
4.99e-09	0	\\
5e-09	0	\\
};
\addplot [color=mycolor2,solid,forget plot]
  table[row sep=crcr]{
0	0	\\
1.1e-10	0	\\
2.2e-10	0	\\
3.3e-10	0	\\
4.4e-10	0	\\
5.4e-10	0	\\
6.5e-10	0	\\
7.5e-10	0	\\
8.6e-10	0	\\
9.6e-10	0	\\
1.07e-09	0	\\
1.18e-09	0	\\
1.28e-09	0	\\
1.38e-09	0	\\
1.49e-09	0	\\
1.59e-09	0	\\
1.69e-09	0	\\
1.8e-09	0	\\
1.9e-09	0	\\
2.01e-09	0	\\
2.11e-09	0	\\
2.21e-09	0	\\
2.32e-09	0	\\
2.42e-09	0	\\
2.52e-09	0	\\
2.63e-09	0	\\
2.73e-09	0	\\
2.83e-09	0	\\
2.93e-09	0	\\
3.04e-09	0	\\
3.14e-09	0	\\
3.24e-09	0	\\
3.34e-09	0	\\
3.45e-09	0	\\
3.55e-09	0	\\
3.65e-09	0	\\
3.75e-09	0	\\
3.86e-09	0	\\
3.96e-09	0	\\
4.06e-09	0	\\
4.16e-09	0	\\
4.27e-09	0	\\
4.37e-09	0	\\
4.47e-09	0	\\
4.57e-09	0	\\
4.68e-09	0	\\
4.78e-09	0	\\
4.89e-09	0	\\
4.99e-09	0	\\
5e-09	0	\\
};
\addplot [color=mycolor3,solid,forget plot]
  table[row sep=crcr]{
0	0	\\
1.1e-10	0	\\
2.2e-10	0	\\
3.3e-10	0	\\
4.4e-10	0	\\
5.4e-10	0	\\
6.5e-10	0	\\
7.5e-10	0	\\
8.6e-10	0	\\
9.6e-10	0	\\
1.07e-09	0	\\
1.18e-09	0	\\
1.28e-09	0	\\
1.38e-09	0	\\
1.49e-09	0	\\
1.59e-09	0	\\
1.69e-09	0	\\
1.8e-09	0	\\
1.9e-09	0	\\
2.01e-09	0	\\
2.11e-09	0	\\
2.21e-09	0	\\
2.32e-09	0	\\
2.42e-09	0	\\
2.52e-09	0	\\
2.63e-09	0	\\
2.73e-09	0	\\
2.83e-09	0	\\
2.93e-09	0	\\
3.04e-09	0	\\
3.14e-09	0	\\
3.24e-09	0	\\
3.34e-09	0	\\
3.45e-09	0	\\
3.55e-09	0	\\
3.65e-09	0	\\
3.75e-09	0	\\
3.86e-09	0	\\
3.96e-09	0	\\
4.06e-09	0	\\
4.16e-09	0	\\
4.27e-09	0	\\
4.37e-09	0	\\
4.47e-09	0	\\
4.57e-09	0	\\
4.68e-09	0	\\
4.78e-09	0	\\
4.89e-09	0	\\
4.99e-09	0	\\
5e-09	0	\\
};
\addplot [color=darkgray,solid,forget plot]
  table[row sep=crcr]{
0	0	\\
1.1e-10	0	\\
2.2e-10	0	\\
3.3e-10	0	\\
4.4e-10	0	\\
5.4e-10	0	\\
6.5e-10	0	\\
7.5e-10	0	\\
8.6e-10	0	\\
9.6e-10	0	\\
1.07e-09	0	\\
1.18e-09	0	\\
1.28e-09	0	\\
1.38e-09	0	\\
1.49e-09	0	\\
1.59e-09	0	\\
1.69e-09	0	\\
1.8e-09	0	\\
1.9e-09	0	\\
2.01e-09	0	\\
2.11e-09	0	\\
2.21e-09	0	\\
2.32e-09	0	\\
2.42e-09	0	\\
2.52e-09	0	\\
2.63e-09	0	\\
2.73e-09	0	\\
2.83e-09	0	\\
2.93e-09	0	\\
3.04e-09	0	\\
3.14e-09	0	\\
3.24e-09	0	\\
3.34e-09	0	\\
3.45e-09	0	\\
3.55e-09	0	\\
3.65e-09	0	\\
3.75e-09	0	\\
3.86e-09	0	\\
3.96e-09	0	\\
4.06e-09	0	\\
4.16e-09	0	\\
4.27e-09	0	\\
4.37e-09	0	\\
4.47e-09	0	\\
4.57e-09	0	\\
4.68e-09	0	\\
4.78e-09	0	\\
4.89e-09	0	\\
4.99e-09	0	\\
5e-09	0	\\
};
\addplot [color=blue,solid,forget plot]
  table[row sep=crcr]{
0	0	\\
1.1e-10	0	\\
2.2e-10	0	\\
3.3e-10	0	\\
4.4e-10	0	\\
5.4e-10	0	\\
6.5e-10	0	\\
7.5e-10	0	\\
8.6e-10	0	\\
9.6e-10	0	\\
1.07e-09	0	\\
1.18e-09	0	\\
1.28e-09	0	\\
1.38e-09	0	\\
1.49e-09	0	\\
1.59e-09	0	\\
1.69e-09	0	\\
1.8e-09	0	\\
1.9e-09	0	\\
2.01e-09	0	\\
2.11e-09	0	\\
2.21e-09	0	\\
2.32e-09	0	\\
2.42e-09	0	\\
2.52e-09	0	\\
2.63e-09	0	\\
2.73e-09	0	\\
2.83e-09	0	\\
2.93e-09	0	\\
3.04e-09	0	\\
3.14e-09	0	\\
3.24e-09	0	\\
3.34e-09	0	\\
3.45e-09	0	\\
3.55e-09	0	\\
3.65e-09	0	\\
3.75e-09	0	\\
3.86e-09	0	\\
3.96e-09	0	\\
4.06e-09	0	\\
4.16e-09	0	\\
4.27e-09	0	\\
4.37e-09	0	\\
4.47e-09	0	\\
4.57e-09	0	\\
4.68e-09	0	\\
4.78e-09	0	\\
4.89e-09	0	\\
4.99e-09	0	\\
5e-09	0	\\
};
\addplot [color=black!50!green,solid,forget plot]
  table[row sep=crcr]{
0	0	\\
1.1e-10	0	\\
2.2e-10	0	\\
3.3e-10	0	\\
4.4e-10	0	\\
5.4e-10	0	\\
6.5e-10	0	\\
7.5e-10	0	\\
8.6e-10	0	\\
9.6e-10	0	\\
1.07e-09	0	\\
1.18e-09	0	\\
1.28e-09	0	\\
1.38e-09	0	\\
1.49e-09	0	\\
1.59e-09	0	\\
1.69e-09	0	\\
1.8e-09	0	\\
1.9e-09	0	\\
2.01e-09	0	\\
2.11e-09	0	\\
2.21e-09	0	\\
2.32e-09	0	\\
2.42e-09	0	\\
2.52e-09	0	\\
2.63e-09	0	\\
2.73e-09	0	\\
2.83e-09	0	\\
2.93e-09	0	\\
3.04e-09	0	\\
3.14e-09	0	\\
3.24e-09	0	\\
3.34e-09	0	\\
3.45e-09	0	\\
3.55e-09	0	\\
3.65e-09	0	\\
3.75e-09	0	\\
3.86e-09	0	\\
3.96e-09	0	\\
4.06e-09	0	\\
4.16e-09	0	\\
4.27e-09	0	\\
4.37e-09	0	\\
4.47e-09	0	\\
4.57e-09	0	\\
4.68e-09	0	\\
4.78e-09	0	\\
4.89e-09	0	\\
4.99e-09	0	\\
5e-09	0	\\
};
\addplot [color=red,solid,forget plot]
  table[row sep=crcr]{
0	0	\\
1.1e-10	0	\\
2.2e-10	0	\\
3.3e-10	0	\\
4.4e-10	0	\\
5.4e-10	0	\\
6.5e-10	0	\\
7.5e-10	0	\\
8.6e-10	0	\\
9.6e-10	0	\\
1.07e-09	0	\\
1.18e-09	0	\\
1.28e-09	0	\\
1.38e-09	0	\\
1.49e-09	0	\\
1.59e-09	0	\\
1.69e-09	0	\\
1.8e-09	0	\\
1.9e-09	0	\\
2.01e-09	0	\\
2.11e-09	0	\\
2.21e-09	0	\\
2.32e-09	0	\\
2.42e-09	0	\\
2.52e-09	0	\\
2.63e-09	0	\\
2.73e-09	0	\\
2.83e-09	0	\\
2.93e-09	0	\\
3.04e-09	0	\\
3.14e-09	0	\\
3.24e-09	0	\\
3.34e-09	0	\\
3.45e-09	0	\\
3.55e-09	0	\\
3.65e-09	0	\\
3.75e-09	0	\\
3.86e-09	0	\\
3.96e-09	0	\\
4.06e-09	0	\\
4.16e-09	0	\\
4.27e-09	0	\\
4.37e-09	0	\\
4.47e-09	0	\\
4.57e-09	0	\\
4.68e-09	0	\\
4.78e-09	0	\\
4.89e-09	0	\\
4.99e-09	0	\\
5e-09	0	\\
};
\addplot [color=mycolor1,solid,forget plot]
  table[row sep=crcr]{
0	0	\\
1.1e-10	0	\\
2.2e-10	0	\\
3.3e-10	0	\\
4.4e-10	0	\\
5.4e-10	0	\\
6.5e-10	0	\\
7.5e-10	0	\\
8.6e-10	0	\\
9.6e-10	0	\\
1.07e-09	0	\\
1.18e-09	0	\\
1.28e-09	0	\\
1.38e-09	0	\\
1.49e-09	0	\\
1.59e-09	0	\\
1.69e-09	0	\\
1.8e-09	0	\\
1.9e-09	0	\\
2.01e-09	0	\\
2.11e-09	0	\\
2.21e-09	0	\\
2.32e-09	0	\\
2.42e-09	0	\\
2.52e-09	0	\\
2.63e-09	0	\\
2.73e-09	0	\\
2.83e-09	0	\\
2.93e-09	0	\\
3.04e-09	0	\\
3.14e-09	0	\\
3.24e-09	0	\\
3.34e-09	0	\\
3.45e-09	0	\\
3.55e-09	0	\\
3.65e-09	0	\\
3.75e-09	0	\\
3.86e-09	0	\\
3.96e-09	0	\\
4.06e-09	0	\\
4.16e-09	0	\\
4.27e-09	0	\\
4.37e-09	0	\\
4.47e-09	0	\\
4.57e-09	0	\\
4.68e-09	0	\\
4.78e-09	0	\\
4.89e-09	0	\\
4.99e-09	0	\\
5e-09	0	\\
};
\addplot [color=mycolor2,solid,forget plot]
  table[row sep=crcr]{
0	0	\\
1.1e-10	0	\\
2.2e-10	0	\\
3.3e-10	0	\\
4.4e-10	0	\\
5.4e-10	0	\\
6.5e-10	0	\\
7.5e-10	0	\\
8.6e-10	0	\\
9.6e-10	0	\\
1.07e-09	0	\\
1.18e-09	0	\\
1.28e-09	0	\\
1.38e-09	0	\\
1.49e-09	0	\\
1.59e-09	0	\\
1.69e-09	0	\\
1.8e-09	0	\\
1.9e-09	0	\\
2.01e-09	0	\\
2.11e-09	0	\\
2.21e-09	0	\\
2.32e-09	0	\\
2.42e-09	0	\\
2.52e-09	0	\\
2.63e-09	0	\\
2.73e-09	0	\\
2.83e-09	0	\\
2.93e-09	0	\\
3.04e-09	0	\\
3.14e-09	0	\\
3.24e-09	0	\\
3.34e-09	0	\\
3.45e-09	0	\\
3.55e-09	0	\\
3.65e-09	0	\\
3.75e-09	0	\\
3.86e-09	0	\\
3.96e-09	0	\\
4.06e-09	0	\\
4.16e-09	0	\\
4.27e-09	0	\\
4.37e-09	0	\\
4.47e-09	0	\\
4.57e-09	0	\\
4.68e-09	0	\\
4.78e-09	0	\\
4.89e-09	0	\\
4.99e-09	0	\\
5e-09	0	\\
};
\addplot [color=mycolor3,solid,forget plot]
  table[row sep=crcr]{
0	0	\\
1.1e-10	0	\\
2.2e-10	0	\\
3.3e-10	0	\\
4.4e-10	0	\\
5.4e-10	0	\\
6.5e-10	0	\\
7.5e-10	0	\\
8.6e-10	0	\\
9.6e-10	0	\\
1.07e-09	0	\\
1.18e-09	0	\\
1.28e-09	0	\\
1.38e-09	0	\\
1.49e-09	0	\\
1.59e-09	0	\\
1.69e-09	0	\\
1.8e-09	0	\\
1.9e-09	0	\\
2.01e-09	0	\\
2.11e-09	0	\\
2.21e-09	0	\\
2.32e-09	0	\\
2.42e-09	0	\\
2.52e-09	0	\\
2.63e-09	0	\\
2.73e-09	0	\\
2.83e-09	0	\\
2.93e-09	0	\\
3.04e-09	0	\\
3.14e-09	0	\\
3.24e-09	0	\\
3.34e-09	0	\\
3.45e-09	0	\\
3.55e-09	0	\\
3.65e-09	0	\\
3.75e-09	0	\\
3.86e-09	0	\\
3.96e-09	0	\\
4.06e-09	0	\\
4.16e-09	0	\\
4.27e-09	0	\\
4.37e-09	0	\\
4.47e-09	0	\\
4.57e-09	0	\\
4.68e-09	0	\\
4.78e-09	0	\\
4.89e-09	0	\\
4.99e-09	0	\\
5e-09	0	\\
};
\addplot [color=darkgray,solid,forget plot]
  table[row sep=crcr]{
0	0	\\
1.1e-10	0	\\
2.2e-10	0	\\
3.3e-10	0	\\
4.4e-10	0	\\
5.4e-10	0	\\
6.5e-10	0	\\
7.5e-10	0	\\
8.6e-10	0	\\
9.6e-10	0	\\
1.07e-09	0	\\
1.18e-09	0	\\
1.28e-09	0	\\
1.38e-09	0	\\
1.49e-09	0	\\
1.59e-09	0	\\
1.69e-09	0	\\
1.8e-09	0	\\
1.9e-09	0	\\
2.01e-09	0	\\
2.11e-09	0	\\
2.21e-09	0	\\
2.32e-09	0	\\
2.42e-09	0	\\
2.52e-09	0	\\
2.63e-09	0	\\
2.73e-09	0	\\
2.83e-09	0	\\
2.93e-09	0	\\
3.04e-09	0	\\
3.14e-09	0	\\
3.24e-09	0	\\
3.34e-09	0	\\
3.45e-09	0	\\
3.55e-09	0	\\
3.65e-09	0	\\
3.75e-09	0	\\
3.86e-09	0	\\
3.96e-09	0	\\
4.06e-09	0	\\
4.16e-09	0	\\
4.27e-09	0	\\
4.37e-09	0	\\
4.47e-09	0	\\
4.57e-09	0	\\
4.68e-09	0	\\
4.78e-09	0	\\
4.89e-09	0	\\
4.99e-09	0	\\
5e-09	0	\\
};
\addplot [color=blue,solid,forget plot]
  table[row sep=crcr]{
0	0	\\
1.1e-10	0	\\
2.2e-10	0	\\
3.3e-10	0	\\
4.4e-10	0	\\
5.4e-10	0	\\
6.5e-10	0	\\
7.5e-10	0	\\
8.6e-10	0	\\
9.6e-10	0	\\
1.07e-09	0	\\
1.18e-09	0	\\
1.28e-09	0	\\
1.38e-09	0	\\
1.49e-09	0	\\
1.59e-09	0	\\
1.69e-09	0	\\
1.8e-09	0	\\
1.9e-09	0	\\
2.01e-09	0	\\
2.11e-09	0	\\
2.21e-09	0	\\
2.32e-09	0	\\
2.42e-09	0	\\
2.52e-09	0	\\
2.63e-09	0	\\
2.73e-09	0	\\
2.83e-09	0	\\
2.93e-09	0	\\
3.04e-09	0	\\
3.14e-09	0	\\
3.24e-09	0	\\
3.34e-09	0	\\
3.45e-09	0	\\
3.55e-09	0	\\
3.65e-09	0	\\
3.75e-09	0	\\
3.86e-09	0	\\
3.96e-09	0	\\
4.06e-09	0	\\
4.16e-09	0	\\
4.27e-09	0	\\
4.37e-09	0	\\
4.47e-09	0	\\
4.57e-09	0	\\
4.68e-09	0	\\
4.78e-09	0	\\
4.89e-09	0	\\
4.99e-09	0	\\
5e-09	0	\\
};
\addplot [color=black!50!green,solid,forget plot]
  table[row sep=crcr]{
0	0	\\
1.1e-10	0	\\
2.2e-10	0	\\
3.3e-10	0	\\
4.4e-10	0	\\
5.4e-10	0	\\
6.5e-10	0	\\
7.5e-10	0	\\
8.6e-10	0	\\
9.6e-10	0	\\
1.07e-09	0	\\
1.18e-09	0	\\
1.28e-09	0	\\
1.38e-09	0	\\
1.49e-09	0	\\
1.59e-09	0	\\
1.69e-09	0	\\
1.8e-09	0	\\
1.9e-09	0	\\
2.01e-09	0	\\
2.11e-09	0	\\
2.21e-09	0	\\
2.32e-09	0	\\
2.42e-09	0	\\
2.52e-09	0	\\
2.63e-09	0	\\
2.73e-09	0	\\
2.83e-09	0	\\
2.93e-09	0	\\
3.04e-09	0	\\
3.14e-09	0	\\
3.24e-09	0	\\
3.34e-09	0	\\
3.45e-09	0	\\
3.55e-09	0	\\
3.65e-09	0	\\
3.75e-09	0	\\
3.86e-09	0	\\
3.96e-09	0	\\
4.06e-09	0	\\
4.16e-09	0	\\
4.27e-09	0	\\
4.37e-09	0	\\
4.47e-09	0	\\
4.57e-09	0	\\
4.68e-09	0	\\
4.78e-09	0	\\
4.89e-09	0	\\
4.99e-09	0	\\
5e-09	0	\\
};
\addplot [color=red,solid,forget plot]
  table[row sep=crcr]{
0	0	\\
1.1e-10	0	\\
2.2e-10	0	\\
3.3e-10	0	\\
4.4e-10	0	\\
5.4e-10	0	\\
6.5e-10	0	\\
7.5e-10	0	\\
8.6e-10	0	\\
9.6e-10	0	\\
1.07e-09	0	\\
1.18e-09	0	\\
1.28e-09	0	\\
1.38e-09	0	\\
1.49e-09	0	\\
1.59e-09	0	\\
1.69e-09	0	\\
1.8e-09	0	\\
1.9e-09	0	\\
2.01e-09	0	\\
2.11e-09	0	\\
2.21e-09	0	\\
2.32e-09	0	\\
2.42e-09	0	\\
2.52e-09	0	\\
2.63e-09	0	\\
2.73e-09	0	\\
2.83e-09	0	\\
2.93e-09	0	\\
3.04e-09	0	\\
3.14e-09	0	\\
3.24e-09	0	\\
3.34e-09	0	\\
3.45e-09	0	\\
3.55e-09	0	\\
3.65e-09	0	\\
3.75e-09	0	\\
3.86e-09	0	\\
3.96e-09	0	\\
4.06e-09	0	\\
4.16e-09	0	\\
4.27e-09	0	\\
4.37e-09	0	\\
4.47e-09	0	\\
4.57e-09	0	\\
4.68e-09	0	\\
4.78e-09	0	\\
4.89e-09	0	\\
4.99e-09	0	\\
5e-09	0	\\
};
\addplot [color=mycolor1,solid,forget plot]
  table[row sep=crcr]{
0	0	\\
1.1e-10	0	\\
2.2e-10	0	\\
3.3e-10	0	\\
4.4e-10	0	\\
5.4e-10	0	\\
6.5e-10	0	\\
7.5e-10	0	\\
8.6e-10	0	\\
9.6e-10	0	\\
1.07e-09	0	\\
1.18e-09	0	\\
1.28e-09	0	\\
1.38e-09	0	\\
1.49e-09	0	\\
1.59e-09	0	\\
1.69e-09	0	\\
1.8e-09	0	\\
1.9e-09	0	\\
2.01e-09	0	\\
2.11e-09	0	\\
2.21e-09	0	\\
2.32e-09	0	\\
2.42e-09	0	\\
2.52e-09	0	\\
2.63e-09	0	\\
2.73e-09	0	\\
2.83e-09	0	\\
2.93e-09	0	\\
3.04e-09	0	\\
3.14e-09	0	\\
3.24e-09	0	\\
3.34e-09	0	\\
3.45e-09	0	\\
3.55e-09	0	\\
3.65e-09	0	\\
3.75e-09	0	\\
3.86e-09	0	\\
3.96e-09	0	\\
4.06e-09	0	\\
4.16e-09	0	\\
4.27e-09	0	\\
4.37e-09	0	\\
4.47e-09	0	\\
4.57e-09	0	\\
4.68e-09	0	\\
4.78e-09	0	\\
4.89e-09	0	\\
4.99e-09	0	\\
5e-09	0	\\
};
\addplot [color=mycolor2,solid,forget plot]
  table[row sep=crcr]{
0	0	\\
1.1e-10	0	\\
2.2e-10	0	\\
3.3e-10	0	\\
4.4e-10	0	\\
5.4e-10	0	\\
6.5e-10	0	\\
7.5e-10	0	\\
8.6e-10	0	\\
9.6e-10	0	\\
1.07e-09	0	\\
1.18e-09	0	\\
1.28e-09	0	\\
1.38e-09	0	\\
1.49e-09	0	\\
1.59e-09	0	\\
1.69e-09	0	\\
1.8e-09	0	\\
1.9e-09	0	\\
2.01e-09	0	\\
2.11e-09	0	\\
2.21e-09	0	\\
2.32e-09	0	\\
2.42e-09	0	\\
2.52e-09	0	\\
2.63e-09	0	\\
2.73e-09	0	\\
2.83e-09	0	\\
2.93e-09	0	\\
3.04e-09	0	\\
3.14e-09	0	\\
3.24e-09	0	\\
3.34e-09	0	\\
3.45e-09	0	\\
3.55e-09	0	\\
3.65e-09	0	\\
3.75e-09	0	\\
3.86e-09	0	\\
3.96e-09	0	\\
4.06e-09	0	\\
4.16e-09	0	\\
4.27e-09	0	\\
4.37e-09	0	\\
4.47e-09	0	\\
4.57e-09	0	\\
4.68e-09	0	\\
4.78e-09	0	\\
4.89e-09	0	\\
4.99e-09	0	\\
5e-09	0	\\
};
\addplot [color=mycolor3,solid,forget plot]
  table[row sep=crcr]{
0	0	\\
1.1e-10	0	\\
2.2e-10	0	\\
3.3e-10	0	\\
4.4e-10	0	\\
5.4e-10	0	\\
6.5e-10	0	\\
7.5e-10	0	\\
8.6e-10	0	\\
9.6e-10	0	\\
1.07e-09	0	\\
1.18e-09	0	\\
1.28e-09	0	\\
1.38e-09	0	\\
1.49e-09	0	\\
1.59e-09	0	\\
1.69e-09	0	\\
1.8e-09	0	\\
1.9e-09	0	\\
2.01e-09	0	\\
2.11e-09	0	\\
2.21e-09	0	\\
2.32e-09	0	\\
2.42e-09	0	\\
2.52e-09	0	\\
2.63e-09	0	\\
2.73e-09	0	\\
2.83e-09	0	\\
2.93e-09	0	\\
3.04e-09	0	\\
3.14e-09	0	\\
3.24e-09	0	\\
3.34e-09	0	\\
3.45e-09	0	\\
3.55e-09	0	\\
3.65e-09	0	\\
3.75e-09	0	\\
3.86e-09	0	\\
3.96e-09	0	\\
4.06e-09	0	\\
4.16e-09	0	\\
4.27e-09	0	\\
4.37e-09	0	\\
4.47e-09	0	\\
4.57e-09	0	\\
4.68e-09	0	\\
4.78e-09	0	\\
4.89e-09	0	\\
4.99e-09	0	\\
5e-09	0	\\
};
\addplot [color=darkgray,solid,forget plot]
  table[row sep=crcr]{
0	0	\\
1.1e-10	0	\\
2.2e-10	0	\\
3.3e-10	0	\\
4.4e-10	0	\\
5.4e-10	0	\\
6.5e-10	0	\\
7.5e-10	0	\\
8.6e-10	0	\\
9.6e-10	0	\\
1.07e-09	0	\\
1.18e-09	0	\\
1.28e-09	0	\\
1.38e-09	0	\\
1.49e-09	0	\\
1.59e-09	0	\\
1.69e-09	0	\\
1.8e-09	0	\\
1.9e-09	0	\\
2.01e-09	0	\\
2.11e-09	0	\\
2.21e-09	0	\\
2.32e-09	0	\\
2.42e-09	0	\\
2.52e-09	0	\\
2.63e-09	0	\\
2.73e-09	0	\\
2.83e-09	0	\\
2.93e-09	0	\\
3.04e-09	0	\\
3.14e-09	0	\\
3.24e-09	0	\\
3.34e-09	0	\\
3.45e-09	0	\\
3.55e-09	0	\\
3.65e-09	0	\\
3.75e-09	0	\\
3.86e-09	0	\\
3.96e-09	0	\\
4.06e-09	0	\\
4.16e-09	0	\\
4.27e-09	0	\\
4.37e-09	0	\\
4.47e-09	0	\\
4.57e-09	0	\\
4.68e-09	0	\\
4.78e-09	0	\\
4.89e-09	0	\\
4.99e-09	0	\\
5e-09	0	\\
};
\addplot [color=blue,solid,forget plot]
  table[row sep=crcr]{
0	0	\\
1.1e-10	0	\\
2.2e-10	0	\\
3.3e-10	0	\\
4.4e-10	0	\\
5.4e-10	0	\\
6.5e-10	0	\\
7.5e-10	0	\\
8.6e-10	0	\\
9.6e-10	0	\\
1.07e-09	0	\\
1.18e-09	0	\\
1.28e-09	0	\\
1.38e-09	0	\\
1.49e-09	0	\\
1.59e-09	0	\\
1.69e-09	0	\\
1.8e-09	0	\\
1.9e-09	0	\\
2.01e-09	0	\\
2.11e-09	0	\\
2.21e-09	0	\\
2.32e-09	0	\\
2.42e-09	0	\\
2.52e-09	0	\\
2.63e-09	0	\\
2.73e-09	0	\\
2.83e-09	0	\\
2.93e-09	0	\\
3.04e-09	0	\\
3.14e-09	0	\\
3.24e-09	0	\\
3.34e-09	0	\\
3.45e-09	0	\\
3.55e-09	0	\\
3.65e-09	0	\\
3.75e-09	0	\\
3.86e-09	0	\\
3.96e-09	0	\\
4.06e-09	0	\\
4.16e-09	0	\\
4.27e-09	0	\\
4.37e-09	0	\\
4.47e-09	0	\\
4.57e-09	0	\\
4.68e-09	0	\\
4.78e-09	0	\\
4.89e-09	0	\\
4.99e-09	0	\\
5e-09	0	\\
};
\addplot [color=black!50!green,solid,forget plot]
  table[row sep=crcr]{
0	0	\\
1.1e-10	0	\\
2.2e-10	0	\\
3.3e-10	0	\\
4.4e-10	0	\\
5.4e-10	0	\\
6.5e-10	0	\\
7.5e-10	0	\\
8.6e-10	0	\\
9.6e-10	0	\\
1.07e-09	0	\\
1.18e-09	0	\\
1.28e-09	0	\\
1.38e-09	0	\\
1.49e-09	0	\\
1.59e-09	0	\\
1.69e-09	0	\\
1.8e-09	0	\\
1.9e-09	0	\\
2.01e-09	0	\\
2.11e-09	0	\\
2.21e-09	0	\\
2.32e-09	0	\\
2.42e-09	0	\\
2.52e-09	0	\\
2.63e-09	0	\\
2.73e-09	0	\\
2.83e-09	0	\\
2.93e-09	0	\\
3.04e-09	0	\\
3.14e-09	0	\\
3.24e-09	0	\\
3.34e-09	0	\\
3.45e-09	0	\\
3.55e-09	0	\\
3.65e-09	0	\\
3.75e-09	0	\\
3.86e-09	0	\\
3.96e-09	0	\\
4.06e-09	0	\\
4.16e-09	0	\\
4.27e-09	0	\\
4.37e-09	0	\\
4.47e-09	0	\\
4.57e-09	0	\\
4.68e-09	0	\\
4.78e-09	0	\\
4.89e-09	0	\\
4.99e-09	0	\\
5e-09	0	\\
};
\addplot [color=red,solid,forget plot]
  table[row sep=crcr]{
0	0	\\
1.1e-10	0	\\
2.2e-10	0	\\
3.3e-10	0	\\
4.4e-10	0	\\
5.4e-10	0	\\
6.5e-10	0	\\
7.5e-10	0	\\
8.6e-10	0	\\
9.6e-10	0	\\
1.07e-09	0	\\
1.18e-09	0	\\
1.28e-09	0	\\
1.38e-09	0	\\
1.49e-09	0	\\
1.59e-09	0	\\
1.69e-09	0	\\
1.8e-09	0	\\
1.9e-09	0	\\
2.01e-09	0	\\
2.11e-09	0	\\
2.21e-09	0	\\
2.32e-09	0	\\
2.42e-09	0	\\
2.52e-09	0	\\
2.63e-09	0	\\
2.73e-09	0	\\
2.83e-09	0	\\
2.93e-09	0	\\
3.04e-09	0	\\
3.14e-09	0	\\
3.24e-09	0	\\
3.34e-09	0	\\
3.45e-09	0	\\
3.55e-09	0	\\
3.65e-09	0	\\
3.75e-09	0	\\
3.86e-09	0	\\
3.96e-09	0	\\
4.06e-09	0	\\
4.16e-09	0	\\
4.27e-09	0	\\
4.37e-09	0	\\
4.47e-09	0	\\
4.57e-09	0	\\
4.68e-09	0	\\
4.78e-09	0	\\
4.89e-09	0	\\
4.99e-09	0	\\
5e-09	0	\\
};
\addplot [color=mycolor1,solid,forget plot]
  table[row sep=crcr]{
0	0	\\
1.1e-10	0	\\
2.2e-10	0	\\
3.3e-10	0	\\
4.4e-10	0	\\
5.4e-10	0	\\
6.5e-10	0	\\
7.5e-10	0	\\
8.6e-10	0	\\
9.6e-10	0	\\
1.07e-09	0	\\
1.18e-09	0	\\
1.28e-09	0	\\
1.38e-09	0	\\
1.49e-09	0	\\
1.59e-09	0	\\
1.69e-09	0	\\
1.8e-09	0	\\
1.9e-09	0	\\
2.01e-09	0	\\
2.11e-09	0	\\
2.21e-09	0	\\
2.32e-09	0	\\
2.42e-09	0	\\
2.52e-09	0	\\
2.63e-09	0	\\
2.73e-09	0	\\
2.83e-09	0	\\
2.93e-09	0	\\
3.04e-09	0	\\
3.14e-09	0	\\
3.24e-09	0	\\
3.34e-09	0	\\
3.45e-09	0	\\
3.55e-09	0	\\
3.65e-09	0	\\
3.75e-09	0	\\
3.86e-09	0	\\
3.96e-09	0	\\
4.06e-09	0	\\
4.16e-09	0	\\
4.27e-09	0	\\
4.37e-09	0	\\
4.47e-09	0	\\
4.57e-09	0	\\
4.68e-09	0	\\
4.78e-09	0	\\
4.89e-09	0	\\
4.99e-09	0	\\
5e-09	0	\\
};
\addplot [color=mycolor2,solid,forget plot]
  table[row sep=crcr]{
0	0	\\
1.1e-10	0	\\
2.2e-10	0	\\
3.3e-10	0	\\
4.4e-10	0	\\
5.4e-10	0	\\
6.5e-10	0	\\
7.5e-10	0	\\
8.6e-10	0	\\
9.6e-10	0	\\
1.07e-09	0	\\
1.18e-09	0	\\
1.28e-09	0	\\
1.38e-09	0	\\
1.49e-09	0	\\
1.59e-09	0	\\
1.69e-09	0	\\
1.8e-09	0	\\
1.9e-09	0	\\
2.01e-09	0	\\
2.11e-09	0	\\
2.21e-09	0	\\
2.32e-09	0	\\
2.42e-09	0	\\
2.52e-09	0	\\
2.63e-09	0	\\
2.73e-09	0	\\
2.83e-09	0	\\
2.93e-09	0	\\
3.04e-09	0	\\
3.14e-09	0	\\
3.24e-09	0	\\
3.34e-09	0	\\
3.45e-09	0	\\
3.55e-09	0	\\
3.65e-09	0	\\
3.75e-09	0	\\
3.86e-09	0	\\
3.96e-09	0	\\
4.06e-09	0	\\
4.16e-09	0	\\
4.27e-09	0	\\
4.37e-09	0	\\
4.47e-09	0	\\
4.57e-09	0	\\
4.68e-09	0	\\
4.78e-09	0	\\
4.89e-09	0	\\
4.99e-09	0	\\
5e-09	0	\\
};
\addplot [color=mycolor3,solid,forget plot]
  table[row sep=crcr]{
0	0	\\
1.1e-10	0	\\
2.2e-10	0	\\
3.3e-10	0	\\
4.4e-10	0	\\
5.4e-10	0	\\
6.5e-10	0	\\
7.5e-10	0	\\
8.6e-10	0	\\
9.6e-10	0	\\
1.07e-09	0	\\
1.18e-09	0	\\
1.28e-09	0	\\
1.38e-09	0	\\
1.49e-09	0	\\
1.59e-09	0	\\
1.69e-09	0	\\
1.8e-09	0	\\
1.9e-09	0	\\
2.01e-09	0	\\
2.11e-09	0	\\
2.21e-09	0	\\
2.32e-09	0	\\
2.42e-09	0	\\
2.52e-09	0	\\
2.63e-09	0	\\
2.73e-09	0	\\
2.83e-09	0	\\
2.93e-09	0	\\
3.04e-09	0	\\
3.14e-09	0	\\
3.24e-09	0	\\
3.34e-09	0	\\
3.45e-09	0	\\
3.55e-09	0	\\
3.65e-09	0	\\
3.75e-09	0	\\
3.86e-09	0	\\
3.96e-09	0	\\
4.06e-09	0	\\
4.16e-09	0	\\
4.27e-09	0	\\
4.37e-09	0	\\
4.47e-09	0	\\
4.57e-09	0	\\
4.68e-09	0	\\
4.78e-09	0	\\
4.89e-09	0	\\
4.99e-09	0	\\
5e-09	0	\\
};
\addplot [color=darkgray,solid,forget plot]
  table[row sep=crcr]{
0	0	\\
1.1e-10	0	\\
2.2e-10	0	\\
3.3e-10	0	\\
4.4e-10	0	\\
5.4e-10	0	\\
6.5e-10	0	\\
7.5e-10	0	\\
8.6e-10	0	\\
9.6e-10	0	\\
1.07e-09	0	\\
1.18e-09	0	\\
1.28e-09	0	\\
1.38e-09	0	\\
1.49e-09	0	\\
1.59e-09	0	\\
1.69e-09	0	\\
1.8e-09	0	\\
1.9e-09	0	\\
2.01e-09	0	\\
2.11e-09	0	\\
2.21e-09	0	\\
2.32e-09	0	\\
2.42e-09	0	\\
2.52e-09	0	\\
2.63e-09	0	\\
2.73e-09	0	\\
2.83e-09	0	\\
2.93e-09	0	\\
3.04e-09	0	\\
3.14e-09	0	\\
3.24e-09	0	\\
3.34e-09	0	\\
3.45e-09	0	\\
3.55e-09	0	\\
3.65e-09	0	\\
3.75e-09	0	\\
3.86e-09	0	\\
3.96e-09	0	\\
4.06e-09	0	\\
4.16e-09	0	\\
4.27e-09	0	\\
4.37e-09	0	\\
4.47e-09	0	\\
4.57e-09	0	\\
4.68e-09	0	\\
4.78e-09	0	\\
4.89e-09	0	\\
4.99e-09	0	\\
5e-09	0	\\
};
\addplot [color=blue,solid,forget plot]
  table[row sep=crcr]{
0	0	\\
1.1e-10	0	\\
2.2e-10	0	\\
3.3e-10	0	\\
4.4e-10	0	\\
5.4e-10	0	\\
6.5e-10	0	\\
7.5e-10	0	\\
8.6e-10	0	\\
9.6e-10	0	\\
1.07e-09	0	\\
1.18e-09	0	\\
1.28e-09	0	\\
1.38e-09	0	\\
1.49e-09	0	\\
1.59e-09	0	\\
1.69e-09	0	\\
1.8e-09	0	\\
1.9e-09	0	\\
2.01e-09	0	\\
2.11e-09	0	\\
2.21e-09	0	\\
2.32e-09	0	\\
2.42e-09	0	\\
2.52e-09	0	\\
2.63e-09	0	\\
2.73e-09	0	\\
2.83e-09	0	\\
2.93e-09	0	\\
3.04e-09	0	\\
3.14e-09	0	\\
3.24e-09	0	\\
3.34e-09	0	\\
3.45e-09	0	\\
3.55e-09	0	\\
3.65e-09	0	\\
3.75e-09	0	\\
3.86e-09	0	\\
3.96e-09	0	\\
4.06e-09	0	\\
4.16e-09	0	\\
4.27e-09	0	\\
4.37e-09	0	\\
4.47e-09	0	\\
4.57e-09	0	\\
4.68e-09	0	\\
4.78e-09	0	\\
4.89e-09	0	\\
4.99e-09	0	\\
5e-09	0	\\
};
\addplot [color=black!50!green,solid,forget plot]
  table[row sep=crcr]{
0	0	\\
1.1e-10	0	\\
2.2e-10	0	\\
3.3e-10	0	\\
4.4e-10	0	\\
5.4e-10	0	\\
6.5e-10	0	\\
7.5e-10	0	\\
8.6e-10	0	\\
9.6e-10	0	\\
1.07e-09	0	\\
1.18e-09	0	\\
1.28e-09	0	\\
1.38e-09	0	\\
1.49e-09	0	\\
1.59e-09	0	\\
1.69e-09	0	\\
1.8e-09	0	\\
1.9e-09	0	\\
2.01e-09	0	\\
2.11e-09	0	\\
2.21e-09	0	\\
2.32e-09	0	\\
2.42e-09	0	\\
2.52e-09	0	\\
2.63e-09	0	\\
2.73e-09	0	\\
2.83e-09	0	\\
2.93e-09	0	\\
3.04e-09	0	\\
3.14e-09	0	\\
3.24e-09	0	\\
3.34e-09	0	\\
3.45e-09	0	\\
3.55e-09	0	\\
3.65e-09	0	\\
3.75e-09	0	\\
3.86e-09	0	\\
3.96e-09	0	\\
4.06e-09	0	\\
4.16e-09	0	\\
4.27e-09	0	\\
4.37e-09	0	\\
4.47e-09	0	\\
4.57e-09	0	\\
4.68e-09	0	\\
4.78e-09	0	\\
4.89e-09	0	\\
4.99e-09	0	\\
5e-09	0	\\
};
\addplot [color=red,solid,forget plot]
  table[row sep=crcr]{
0	0	\\
1.1e-10	0	\\
2.2e-10	0	\\
3.3e-10	0	\\
4.4e-10	0	\\
5.4e-10	0	\\
6.5e-10	0	\\
7.5e-10	0	\\
8.6e-10	0	\\
9.6e-10	0	\\
1.07e-09	0	\\
1.18e-09	0	\\
1.28e-09	0	\\
1.38e-09	0	\\
1.49e-09	0	\\
1.59e-09	0	\\
1.69e-09	0	\\
1.8e-09	0	\\
1.9e-09	0	\\
2.01e-09	0	\\
2.11e-09	0	\\
2.21e-09	0	\\
2.32e-09	0	\\
2.42e-09	0	\\
2.52e-09	0	\\
2.63e-09	0	\\
2.73e-09	0	\\
2.83e-09	0	\\
2.93e-09	0	\\
3.04e-09	0	\\
3.14e-09	0	\\
3.24e-09	0	\\
3.34e-09	0	\\
3.45e-09	0	\\
3.55e-09	0	\\
3.65e-09	0	\\
3.75e-09	0	\\
3.86e-09	0	\\
3.96e-09	0	\\
4.06e-09	0	\\
4.16e-09	0	\\
4.27e-09	0	\\
4.37e-09	0	\\
4.47e-09	0	\\
4.57e-09	0	\\
4.68e-09	0	\\
4.78e-09	0	\\
4.89e-09	0	\\
4.99e-09	0	\\
5e-09	0	\\
};
\addplot [color=mycolor1,solid,forget plot]
  table[row sep=crcr]{
0	0	\\
1.1e-10	0	\\
2.2e-10	0	\\
3.3e-10	0	\\
4.4e-10	0	\\
5.4e-10	0	\\
6.5e-10	0	\\
7.5e-10	0	\\
8.6e-10	0	\\
9.6e-10	0	\\
1.07e-09	0	\\
1.18e-09	0	\\
1.28e-09	0	\\
1.38e-09	0	\\
1.49e-09	0	\\
1.59e-09	0	\\
1.69e-09	0	\\
1.8e-09	0	\\
1.9e-09	0	\\
2.01e-09	0	\\
2.11e-09	0	\\
2.21e-09	0	\\
2.32e-09	0	\\
2.42e-09	0	\\
2.52e-09	0	\\
2.63e-09	0	\\
2.73e-09	0	\\
2.83e-09	0	\\
2.93e-09	0	\\
3.04e-09	0	\\
3.14e-09	0	\\
3.24e-09	0	\\
3.34e-09	0	\\
3.45e-09	0	\\
3.55e-09	0	\\
3.65e-09	0	\\
3.75e-09	0	\\
3.86e-09	0	\\
3.96e-09	0	\\
4.06e-09	0	\\
4.16e-09	0	\\
4.27e-09	0	\\
4.37e-09	0	\\
4.47e-09	0	\\
4.57e-09	0	\\
4.68e-09	0	\\
4.78e-09	0	\\
4.89e-09	0	\\
4.99e-09	0	\\
5e-09	0	\\
};
\addplot [color=mycolor2,solid,forget plot]
  table[row sep=crcr]{
0	0	\\
1.1e-10	0	\\
2.2e-10	0	\\
3.3e-10	0	\\
4.4e-10	0	\\
5.4e-10	0	\\
6.5e-10	0	\\
7.5e-10	0	\\
8.6e-10	0	\\
9.6e-10	0	\\
1.07e-09	0	\\
1.18e-09	0	\\
1.28e-09	0	\\
1.38e-09	0	\\
1.49e-09	0	\\
1.59e-09	0	\\
1.69e-09	0	\\
1.8e-09	0	\\
1.9e-09	0	\\
2.01e-09	0	\\
2.11e-09	0	\\
2.21e-09	0	\\
2.32e-09	0	\\
2.42e-09	0	\\
2.52e-09	0	\\
2.63e-09	0	\\
2.73e-09	0	\\
2.83e-09	0	\\
2.93e-09	0	\\
3.04e-09	0	\\
3.14e-09	0	\\
3.24e-09	0	\\
3.34e-09	0	\\
3.45e-09	0	\\
3.55e-09	0	\\
3.65e-09	0	\\
3.75e-09	0	\\
3.86e-09	0	\\
3.96e-09	0	\\
4.06e-09	0	\\
4.16e-09	0	\\
4.27e-09	0	\\
4.37e-09	0	\\
4.47e-09	0	\\
4.57e-09	0	\\
4.68e-09	0	\\
4.78e-09	0	\\
4.89e-09	0	\\
4.99e-09	0	\\
5e-09	0	\\
};
\addplot [color=mycolor3,solid,forget plot]
  table[row sep=crcr]{
0	0	\\
1.1e-10	0	\\
2.2e-10	0	\\
3.3e-10	0	\\
4.4e-10	0	\\
5.4e-10	0	\\
6.5e-10	0	\\
7.5e-10	0	\\
8.6e-10	0	\\
9.6e-10	0	\\
1.07e-09	0	\\
1.18e-09	0	\\
1.28e-09	0	\\
1.38e-09	0	\\
1.49e-09	0	\\
1.59e-09	0	\\
1.69e-09	0	\\
1.8e-09	0	\\
1.9e-09	0	\\
2.01e-09	0	\\
2.11e-09	0	\\
2.21e-09	0	\\
2.32e-09	0	\\
2.42e-09	0	\\
2.52e-09	0	\\
2.63e-09	0	\\
2.73e-09	0	\\
2.83e-09	0	\\
2.93e-09	0	\\
3.04e-09	0	\\
3.14e-09	0	\\
3.24e-09	0	\\
3.34e-09	0	\\
3.45e-09	0	\\
3.55e-09	0	\\
3.65e-09	0	\\
3.75e-09	0	\\
3.86e-09	0	\\
3.96e-09	0	\\
4.06e-09	0	\\
4.16e-09	0	\\
4.27e-09	0	\\
4.37e-09	0	\\
4.47e-09	0	\\
4.57e-09	0	\\
4.68e-09	0	\\
4.78e-09	0	\\
4.89e-09	0	\\
4.99e-09	0	\\
5e-09	0	\\
};
\addplot [color=darkgray,solid,forget plot]
  table[row sep=crcr]{
0	0	\\
1.1e-10	0	\\
2.2e-10	0	\\
3.3e-10	0	\\
4.4e-10	0	\\
5.4e-10	0	\\
6.5e-10	0	\\
7.5e-10	0	\\
8.6e-10	0	\\
9.6e-10	0	\\
1.07e-09	0	\\
1.18e-09	0	\\
1.28e-09	0	\\
1.38e-09	0	\\
1.49e-09	0	\\
1.59e-09	0	\\
1.69e-09	0	\\
1.8e-09	0	\\
1.9e-09	0	\\
2.01e-09	0	\\
2.11e-09	0	\\
2.21e-09	0	\\
2.32e-09	0	\\
2.42e-09	0	\\
2.52e-09	0	\\
2.63e-09	0	\\
2.73e-09	0	\\
2.83e-09	0	\\
2.93e-09	0	\\
3.04e-09	0	\\
3.14e-09	0	\\
3.24e-09	0	\\
3.34e-09	0	\\
3.45e-09	0	\\
3.55e-09	0	\\
3.65e-09	0	\\
3.75e-09	0	\\
3.86e-09	0	\\
3.96e-09	0	\\
4.06e-09	0	\\
4.16e-09	0	\\
4.27e-09	0	\\
4.37e-09	0	\\
4.47e-09	0	\\
4.57e-09	0	\\
4.68e-09	0	\\
4.78e-09	0	\\
4.89e-09	0	\\
4.99e-09	0	\\
5e-09	0	\\
};
\addplot [color=blue,solid,forget plot]
  table[row sep=crcr]{
0	0	\\
1.1e-10	0	\\
2.2e-10	0	\\
3.3e-10	0	\\
4.4e-10	0	\\
5.4e-10	0	\\
6.5e-10	0	\\
7.5e-10	0	\\
8.6e-10	0	\\
9.6e-10	0	\\
1.07e-09	0	\\
1.18e-09	0	\\
1.28e-09	0	\\
1.38e-09	0	\\
1.49e-09	0	\\
1.59e-09	0	\\
1.69e-09	0	\\
1.8e-09	0	\\
1.9e-09	0	\\
2.01e-09	0	\\
2.11e-09	0	\\
2.21e-09	0	\\
2.32e-09	0	\\
2.42e-09	0	\\
2.52e-09	0	\\
2.63e-09	0	\\
2.73e-09	0	\\
2.83e-09	0	\\
2.93e-09	0	\\
3.04e-09	0	\\
3.14e-09	0	\\
3.24e-09	0	\\
3.34e-09	0	\\
3.45e-09	0	\\
3.55e-09	0	\\
3.65e-09	0	\\
3.75e-09	0	\\
3.86e-09	0	\\
3.96e-09	0	\\
4.06e-09	0	\\
4.16e-09	0	\\
4.27e-09	0	\\
4.37e-09	0	\\
4.47e-09	0	\\
4.57e-09	0	\\
4.68e-09	0	\\
4.78e-09	0	\\
4.89e-09	0	\\
4.99e-09	0	\\
5e-09	0	\\
};
\addplot [color=black!50!green,solid,forget plot]
  table[row sep=crcr]{
0	0	\\
1.1e-10	0	\\
2.2e-10	0	\\
3.3e-10	0	\\
4.4e-10	0	\\
5.4e-10	0	\\
6.5e-10	0	\\
7.5e-10	0	\\
8.6e-10	0	\\
9.6e-10	0	\\
1.07e-09	0	\\
1.18e-09	0	\\
1.28e-09	0	\\
1.38e-09	0	\\
1.49e-09	0	\\
1.59e-09	0	\\
1.69e-09	0	\\
1.8e-09	0	\\
1.9e-09	0	\\
2.01e-09	0	\\
2.11e-09	0	\\
2.21e-09	0	\\
2.32e-09	0	\\
2.42e-09	0	\\
2.52e-09	0	\\
2.63e-09	0	\\
2.73e-09	0	\\
2.83e-09	0	\\
2.93e-09	0	\\
3.04e-09	0	\\
3.14e-09	0	\\
3.24e-09	0	\\
3.34e-09	0	\\
3.45e-09	0	\\
3.55e-09	0	\\
3.65e-09	0	\\
3.75e-09	0	\\
3.86e-09	0	\\
3.96e-09	0	\\
4.06e-09	0	\\
4.16e-09	0	\\
4.27e-09	0	\\
4.37e-09	0	\\
4.47e-09	0	\\
4.57e-09	0	\\
4.68e-09	0	\\
4.78e-09	0	\\
4.89e-09	0	\\
4.99e-09	0	\\
5e-09	0	\\
};
\addplot [color=red,solid,forget plot]
  table[row sep=crcr]{
0	0	\\
1.1e-10	0	\\
2.2e-10	0	\\
3.3e-10	0	\\
4.4e-10	0	\\
5.4e-10	0	\\
6.5e-10	0	\\
7.5e-10	0	\\
8.6e-10	0	\\
9.6e-10	0	\\
1.07e-09	0	\\
1.18e-09	0	\\
1.28e-09	0	\\
1.38e-09	0	\\
1.49e-09	0	\\
1.59e-09	0	\\
1.69e-09	0	\\
1.8e-09	0	\\
1.9e-09	0	\\
2.01e-09	0	\\
2.11e-09	0	\\
2.21e-09	0	\\
2.32e-09	0	\\
2.42e-09	0	\\
2.52e-09	0	\\
2.63e-09	0	\\
2.73e-09	0	\\
2.83e-09	0	\\
2.93e-09	0	\\
3.04e-09	0	\\
3.14e-09	0	\\
3.24e-09	0	\\
3.34e-09	0	\\
3.45e-09	0	\\
3.55e-09	0	\\
3.65e-09	0	\\
3.75e-09	0	\\
3.86e-09	0	\\
3.96e-09	0	\\
4.06e-09	0	\\
4.16e-09	0	\\
4.27e-09	0	\\
4.37e-09	0	\\
4.47e-09	0	\\
4.57e-09	0	\\
4.68e-09	0	\\
4.78e-09	0	\\
4.89e-09	0	\\
4.99e-09	0	\\
5e-09	0	\\
};
\addplot [color=mycolor1,solid,forget plot]
  table[row sep=crcr]{
0	0	\\
1.1e-10	0	\\
2.2e-10	0	\\
3.3e-10	0	\\
4.4e-10	0	\\
5.4e-10	0	\\
6.5e-10	0	\\
7.5e-10	0	\\
8.6e-10	0	\\
9.6e-10	0	\\
1.07e-09	0	\\
1.18e-09	0	\\
1.28e-09	0	\\
1.38e-09	0	\\
1.49e-09	0	\\
1.59e-09	0	\\
1.69e-09	0	\\
1.8e-09	0	\\
1.9e-09	0	\\
2.01e-09	0	\\
2.11e-09	0	\\
2.21e-09	0	\\
2.32e-09	0	\\
2.42e-09	0	\\
2.52e-09	0	\\
2.63e-09	0	\\
2.73e-09	0	\\
2.83e-09	0	\\
2.93e-09	0	\\
3.04e-09	0	\\
3.14e-09	0	\\
3.24e-09	0	\\
3.34e-09	0	\\
3.45e-09	0	\\
3.55e-09	0	\\
3.65e-09	0	\\
3.75e-09	0	\\
3.86e-09	0	\\
3.96e-09	0	\\
4.06e-09	0	\\
4.16e-09	0	\\
4.27e-09	0	\\
4.37e-09	0	\\
4.47e-09	0	\\
4.57e-09	0	\\
4.68e-09	0	\\
4.78e-09	0	\\
4.89e-09	0	\\
4.99e-09	0	\\
5e-09	0	\\
};
\addplot [color=mycolor2,solid,forget plot]
  table[row sep=crcr]{
0	0	\\
1.1e-10	0	\\
2.2e-10	0	\\
3.3e-10	0	\\
4.4e-10	0	\\
5.4e-10	0	\\
6.5e-10	0	\\
7.5e-10	0	\\
8.6e-10	0	\\
9.6e-10	0	\\
1.07e-09	0	\\
1.18e-09	0	\\
1.28e-09	0	\\
1.38e-09	0	\\
1.49e-09	0	\\
1.59e-09	0	\\
1.69e-09	0	\\
1.8e-09	0	\\
1.9e-09	0	\\
2.01e-09	0	\\
2.11e-09	0	\\
2.21e-09	0	\\
2.32e-09	0	\\
2.42e-09	0	\\
2.52e-09	0	\\
2.63e-09	0	\\
2.73e-09	0	\\
2.83e-09	0	\\
2.93e-09	0	\\
3.04e-09	0	\\
3.14e-09	0	\\
3.24e-09	0	\\
3.34e-09	0	\\
3.45e-09	0	\\
3.55e-09	0	\\
3.65e-09	0	\\
3.75e-09	0	\\
3.86e-09	0	\\
3.96e-09	0	\\
4.06e-09	0	\\
4.16e-09	0	\\
4.27e-09	0	\\
4.37e-09	0	\\
4.47e-09	0	\\
4.57e-09	0	\\
4.68e-09	0	\\
4.78e-09	0	\\
4.89e-09	0	\\
4.99e-09	0	\\
5e-09	0	\\
};
\addplot [color=mycolor3,solid,forget plot]
  table[row sep=crcr]{
0	0	\\
1.1e-10	0	\\
2.2e-10	0	\\
3.3e-10	0	\\
4.4e-10	0	\\
5.4e-10	0	\\
6.5e-10	0	\\
7.5e-10	0	\\
8.6e-10	0	\\
9.6e-10	0	\\
1.07e-09	0	\\
1.18e-09	0	\\
1.28e-09	0	\\
1.38e-09	0	\\
1.49e-09	0	\\
1.59e-09	0	\\
1.69e-09	0	\\
1.8e-09	0	\\
1.9e-09	0	\\
2.01e-09	0	\\
2.11e-09	0	\\
2.21e-09	0	\\
2.32e-09	0	\\
2.42e-09	0	\\
2.52e-09	0	\\
2.63e-09	0	\\
2.73e-09	0	\\
2.83e-09	0	\\
2.93e-09	0	\\
3.04e-09	0	\\
3.14e-09	0	\\
3.24e-09	0	\\
3.34e-09	0	\\
3.45e-09	0	\\
3.55e-09	0	\\
3.65e-09	0	\\
3.75e-09	0	\\
3.86e-09	0	\\
3.96e-09	0	\\
4.06e-09	0	\\
4.16e-09	0	\\
4.27e-09	0	\\
4.37e-09	0	\\
4.47e-09	0	\\
4.57e-09	0	\\
4.68e-09	0	\\
4.78e-09	0	\\
4.89e-09	0	\\
4.99e-09	0	\\
5e-09	0	\\
};
\addplot [color=darkgray,solid,forget plot]
  table[row sep=crcr]{
0	0	\\
1.1e-10	0	\\
2.2e-10	0	\\
3.3e-10	0	\\
4.4e-10	0	\\
5.4e-10	0	\\
6.5e-10	0	\\
7.5e-10	0	\\
8.6e-10	0	\\
9.6e-10	0	\\
1.07e-09	0	\\
1.18e-09	0	\\
1.28e-09	0	\\
1.38e-09	0	\\
1.49e-09	0	\\
1.59e-09	0	\\
1.69e-09	0	\\
1.8e-09	0	\\
1.9e-09	0	\\
2.01e-09	0	\\
2.11e-09	0	\\
2.21e-09	0	\\
2.32e-09	0	\\
2.42e-09	0	\\
2.52e-09	0	\\
2.63e-09	0	\\
2.73e-09	0	\\
2.83e-09	0	\\
2.93e-09	0	\\
3.04e-09	0	\\
3.14e-09	0	\\
3.24e-09	0	\\
3.34e-09	0	\\
3.45e-09	0	\\
3.55e-09	0	\\
3.65e-09	0	\\
3.75e-09	0	\\
3.86e-09	0	\\
3.96e-09	0	\\
4.06e-09	0	\\
4.16e-09	0	\\
4.27e-09	0	\\
4.37e-09	0	\\
4.47e-09	0	\\
4.57e-09	0	\\
4.68e-09	0	\\
4.78e-09	0	\\
4.89e-09	0	\\
4.99e-09	0	\\
5e-09	0	\\
};
\addplot [color=blue,solid,forget plot]
  table[row sep=crcr]{
0	0	\\
1.1e-10	0	\\
2.2e-10	0	\\
3.3e-10	0	\\
4.4e-10	0	\\
5.4e-10	0	\\
6.5e-10	0	\\
7.5e-10	0	\\
8.6e-10	0	\\
9.6e-10	0	\\
1.07e-09	0	\\
1.18e-09	0	\\
1.28e-09	0	\\
1.38e-09	0	\\
1.49e-09	0	\\
1.59e-09	0	\\
1.69e-09	0	\\
1.8e-09	0	\\
1.9e-09	0	\\
2.01e-09	0	\\
2.11e-09	0	\\
2.21e-09	0	\\
2.32e-09	0	\\
2.42e-09	0	\\
2.52e-09	0	\\
2.63e-09	0	\\
2.73e-09	0	\\
2.83e-09	0	\\
2.93e-09	0	\\
3.04e-09	0	\\
3.14e-09	0	\\
3.24e-09	0	\\
3.34e-09	0	\\
3.45e-09	0	\\
3.55e-09	0	\\
3.65e-09	0	\\
3.75e-09	0	\\
3.86e-09	0	\\
3.96e-09	0	\\
4.06e-09	0	\\
4.16e-09	0	\\
4.27e-09	0	\\
4.37e-09	0	\\
4.47e-09	0	\\
4.57e-09	0	\\
4.68e-09	0	\\
4.78e-09	0	\\
4.89e-09	0	\\
4.99e-09	0	\\
5e-09	0	\\
};
\addplot [color=black!50!green,solid,forget plot]
  table[row sep=crcr]{
0	0	\\
1.1e-10	0	\\
2.2e-10	0	\\
3.3e-10	0	\\
4.4e-10	0	\\
5.4e-10	0	\\
6.5e-10	0	\\
7.5e-10	0	\\
8.6e-10	0	\\
9.6e-10	0	\\
1.07e-09	0	\\
1.18e-09	0	\\
1.28e-09	0	\\
1.38e-09	0	\\
1.49e-09	0	\\
1.59e-09	0	\\
1.69e-09	0	\\
1.8e-09	0	\\
1.9e-09	0	\\
2.01e-09	0	\\
2.11e-09	0	\\
2.21e-09	0	\\
2.32e-09	0	\\
2.42e-09	0	\\
2.52e-09	0	\\
2.63e-09	0	\\
2.73e-09	0	\\
2.83e-09	0	\\
2.93e-09	0	\\
3.04e-09	0	\\
3.14e-09	0	\\
3.24e-09	0	\\
3.34e-09	0	\\
3.45e-09	0	\\
3.55e-09	0	\\
3.65e-09	0	\\
3.75e-09	0	\\
3.86e-09	0	\\
3.96e-09	0	\\
4.06e-09	0	\\
4.16e-09	0	\\
4.27e-09	0	\\
4.37e-09	0	\\
4.47e-09	0	\\
4.57e-09	0	\\
4.68e-09	0	\\
4.78e-09	0	\\
4.89e-09	0	\\
4.99e-09	0	\\
5e-09	0	\\
};
\addplot [color=red,solid,forget plot]
  table[row sep=crcr]{
0	0	\\
1.1e-10	0	\\
2.2e-10	0	\\
3.3e-10	0	\\
4.4e-10	0	\\
5.4e-10	0	\\
6.5e-10	0	\\
7.5e-10	0	\\
8.6e-10	0	\\
9.6e-10	0	\\
1.07e-09	0	\\
1.18e-09	0	\\
1.28e-09	0	\\
1.38e-09	0	\\
1.49e-09	0	\\
1.59e-09	0	\\
1.69e-09	0	\\
1.8e-09	0	\\
1.9e-09	0	\\
2.01e-09	0	\\
2.11e-09	0	\\
2.21e-09	0	\\
2.32e-09	0	\\
2.42e-09	0	\\
2.52e-09	0	\\
2.63e-09	0	\\
2.73e-09	0	\\
2.83e-09	0	\\
2.93e-09	0	\\
3.04e-09	0	\\
3.14e-09	0	\\
3.24e-09	0	\\
3.34e-09	0	\\
3.45e-09	0	\\
3.55e-09	0	\\
3.65e-09	0	\\
3.75e-09	0	\\
3.86e-09	0	\\
3.96e-09	0	\\
4.06e-09	0	\\
4.16e-09	0	\\
4.27e-09	0	\\
4.37e-09	0	\\
4.47e-09	0	\\
4.57e-09	0	\\
4.68e-09	0	\\
4.78e-09	0	\\
4.89e-09	0	\\
4.99e-09	0	\\
5e-09	0	\\
};
\addplot [color=mycolor1,solid,forget plot]
  table[row sep=crcr]{
0	0	\\
1.1e-10	0	\\
2.2e-10	0	\\
3.3e-10	0	\\
4.4e-10	0	\\
5.4e-10	0	\\
6.5e-10	0	\\
7.5e-10	0	\\
8.6e-10	0	\\
9.6e-10	0	\\
1.07e-09	0	\\
1.18e-09	0	\\
1.28e-09	0	\\
1.38e-09	0	\\
1.49e-09	0	\\
1.59e-09	0	\\
1.69e-09	0	\\
1.8e-09	0	\\
1.9e-09	0	\\
2.01e-09	0	\\
2.11e-09	0	\\
2.21e-09	0	\\
2.32e-09	0	\\
2.42e-09	0	\\
2.52e-09	0	\\
2.63e-09	0	\\
2.73e-09	0	\\
2.83e-09	0	\\
2.93e-09	0	\\
3.04e-09	0	\\
3.14e-09	0	\\
3.24e-09	0	\\
3.34e-09	0	\\
3.45e-09	0	\\
3.55e-09	0	\\
3.65e-09	0	\\
3.75e-09	0	\\
3.86e-09	0	\\
3.96e-09	0	\\
4.06e-09	0	\\
4.16e-09	0	\\
4.27e-09	0	\\
4.37e-09	0	\\
4.47e-09	0	\\
4.57e-09	0	\\
4.68e-09	0	\\
4.78e-09	0	\\
4.89e-09	0	\\
4.99e-09	0	\\
5e-09	0	\\
};
\addplot [color=mycolor2,solid,forget plot]
  table[row sep=crcr]{
0	0	\\
1.1e-10	0	\\
2.2e-10	0	\\
3.3e-10	0	\\
4.4e-10	0	\\
5.4e-10	0	\\
6.5e-10	0	\\
7.5e-10	0	\\
8.6e-10	0	\\
9.6e-10	0	\\
1.07e-09	0	\\
1.18e-09	0	\\
1.28e-09	0	\\
1.38e-09	0	\\
1.49e-09	0	\\
1.59e-09	0	\\
1.69e-09	0	\\
1.8e-09	0	\\
1.9e-09	0	\\
2.01e-09	0	\\
2.11e-09	0	\\
2.21e-09	0	\\
2.32e-09	0	\\
2.42e-09	0	\\
2.52e-09	0	\\
2.63e-09	0	\\
2.73e-09	0	\\
2.83e-09	0	\\
2.93e-09	0	\\
3.04e-09	0	\\
3.14e-09	0	\\
3.24e-09	0	\\
3.34e-09	0	\\
3.45e-09	0	\\
3.55e-09	0	\\
3.65e-09	0	\\
3.75e-09	0	\\
3.86e-09	0	\\
3.96e-09	0	\\
4.06e-09	0	\\
4.16e-09	0	\\
4.27e-09	0	\\
4.37e-09	0	\\
4.47e-09	0	\\
4.57e-09	0	\\
4.68e-09	0	\\
4.78e-09	0	\\
4.89e-09	0	\\
4.99e-09	0	\\
5e-09	0	\\
};
\addplot [color=mycolor3,solid,forget plot]
  table[row sep=crcr]{
0	0	\\
1.1e-10	0	\\
2.2e-10	0	\\
3.3e-10	0	\\
4.4e-10	0	\\
5.4e-10	0	\\
6.5e-10	0	\\
7.5e-10	0	\\
8.6e-10	0	\\
9.6e-10	0	\\
1.07e-09	0	\\
1.18e-09	0	\\
1.28e-09	0	\\
1.38e-09	0	\\
1.49e-09	0	\\
1.59e-09	0	\\
1.69e-09	0	\\
1.8e-09	0	\\
1.9e-09	0	\\
2.01e-09	0	\\
2.11e-09	0	\\
2.21e-09	0	\\
2.32e-09	0	\\
2.42e-09	0	\\
2.52e-09	0	\\
2.63e-09	0	\\
2.73e-09	0	\\
2.83e-09	0	\\
2.93e-09	0	\\
3.04e-09	0	\\
3.14e-09	0	\\
3.24e-09	0	\\
3.34e-09	0	\\
3.45e-09	0	\\
3.55e-09	0	\\
3.65e-09	0	\\
3.75e-09	0	\\
3.86e-09	0	\\
3.96e-09	0	\\
4.06e-09	0	\\
4.16e-09	0	\\
4.27e-09	0	\\
4.37e-09	0	\\
4.47e-09	0	\\
4.57e-09	0	\\
4.68e-09	0	\\
4.78e-09	0	\\
4.89e-09	0	\\
4.99e-09	0	\\
5e-09	0	\\
};
\addplot [color=darkgray,solid,forget plot]
  table[row sep=crcr]{
0	0	\\
1.1e-10	0	\\
2.2e-10	0	\\
3.3e-10	0	\\
4.4e-10	0	\\
5.4e-10	0	\\
6.5e-10	0	\\
7.5e-10	0	\\
8.6e-10	0	\\
9.6e-10	0	\\
1.07e-09	0	\\
1.18e-09	0	\\
1.28e-09	0	\\
1.38e-09	0	\\
1.49e-09	0	\\
1.59e-09	0	\\
1.69e-09	0	\\
1.8e-09	0	\\
1.9e-09	0	\\
2.01e-09	0	\\
2.11e-09	0	\\
2.21e-09	0	\\
2.32e-09	0	\\
2.42e-09	0	\\
2.52e-09	0	\\
2.63e-09	0	\\
2.73e-09	0	\\
2.83e-09	0	\\
2.93e-09	0	\\
3.04e-09	0	\\
3.14e-09	0	\\
3.24e-09	0	\\
3.34e-09	0	\\
3.45e-09	0	\\
3.55e-09	0	\\
3.65e-09	0	\\
3.75e-09	0	\\
3.86e-09	0	\\
3.96e-09	0	\\
4.06e-09	0	\\
4.16e-09	0	\\
4.27e-09	0	\\
4.37e-09	0	\\
4.47e-09	0	\\
4.57e-09	0	\\
4.68e-09	0	\\
4.78e-09	0	\\
4.89e-09	0	\\
4.99e-09	0	\\
5e-09	0	\\
};
\addplot [color=blue,solid,forget plot]
  table[row sep=crcr]{
0	0	\\
1.1e-10	0	\\
2.2e-10	0	\\
3.3e-10	0	\\
4.4e-10	0	\\
5.4e-10	0	\\
6.5e-10	0	\\
7.5e-10	0	\\
8.6e-10	0	\\
9.6e-10	0	\\
1.07e-09	0	\\
1.18e-09	0	\\
1.28e-09	0	\\
1.38e-09	0	\\
1.49e-09	0	\\
1.59e-09	0	\\
1.69e-09	0	\\
1.8e-09	0	\\
1.9e-09	0	\\
2.01e-09	0	\\
2.11e-09	0	\\
2.21e-09	0	\\
2.32e-09	0	\\
2.42e-09	0	\\
2.52e-09	0	\\
2.63e-09	0	\\
2.73e-09	0	\\
2.83e-09	0	\\
2.93e-09	0	\\
3.04e-09	0	\\
3.14e-09	0	\\
3.24e-09	0	\\
3.34e-09	0	\\
3.45e-09	0	\\
3.55e-09	0	\\
3.65e-09	0	\\
3.75e-09	0	\\
3.86e-09	0	\\
3.96e-09	0	\\
4.06e-09	0	\\
4.16e-09	0	\\
4.27e-09	0	\\
4.37e-09	0	\\
4.47e-09	0	\\
4.57e-09	0	\\
4.68e-09	0	\\
4.78e-09	0	\\
4.89e-09	0	\\
4.99e-09	0	\\
5e-09	0	\\
};
\addplot [color=black!50!green,solid,forget plot]
  table[row sep=crcr]{
0	0	\\
1.1e-10	0	\\
2.2e-10	0	\\
3.3e-10	0	\\
4.4e-10	0	\\
5.4e-10	0	\\
6.5e-10	0	\\
7.5e-10	0	\\
8.6e-10	0	\\
9.6e-10	0	\\
1.07e-09	0	\\
1.18e-09	0	\\
1.28e-09	0	\\
1.38e-09	0	\\
1.49e-09	0	\\
1.59e-09	0	\\
1.69e-09	0	\\
1.8e-09	0	\\
1.9e-09	0	\\
2.01e-09	0	\\
2.11e-09	0	\\
2.21e-09	0	\\
2.32e-09	0	\\
2.42e-09	0	\\
2.52e-09	0	\\
2.63e-09	0	\\
2.73e-09	0	\\
2.83e-09	0	\\
2.93e-09	0	\\
3.04e-09	0	\\
3.14e-09	0	\\
3.24e-09	0	\\
3.34e-09	0	\\
3.45e-09	0	\\
3.55e-09	0	\\
3.65e-09	0	\\
3.75e-09	0	\\
3.86e-09	0	\\
3.96e-09	0	\\
4.06e-09	0	\\
4.16e-09	0	\\
4.27e-09	0	\\
4.37e-09	0	\\
4.47e-09	0	\\
4.57e-09	0	\\
4.68e-09	0	\\
4.78e-09	0	\\
4.89e-09	0	\\
4.99e-09	0	\\
5e-09	0	\\
};
\addplot [color=red,solid,forget plot]
  table[row sep=crcr]{
0	0	\\
1.1e-10	0	\\
2.2e-10	0	\\
3.3e-10	0	\\
4.4e-10	0	\\
5.4e-10	0	\\
6.5e-10	0	\\
7.5e-10	0	\\
8.6e-10	0	\\
9.6e-10	0	\\
1.07e-09	0	\\
1.18e-09	0	\\
1.28e-09	0	\\
1.38e-09	0	\\
1.49e-09	0	\\
1.59e-09	0	\\
1.69e-09	0	\\
1.8e-09	0	\\
1.9e-09	0	\\
2.01e-09	0	\\
2.11e-09	0	\\
2.21e-09	0	\\
2.32e-09	0	\\
2.42e-09	0	\\
2.52e-09	0	\\
2.63e-09	0	\\
2.73e-09	0	\\
2.83e-09	0	\\
2.93e-09	0	\\
3.04e-09	0	\\
3.14e-09	0	\\
3.24e-09	0	\\
3.34e-09	0	\\
3.45e-09	0	\\
3.55e-09	0	\\
3.65e-09	0	\\
3.75e-09	0	\\
3.86e-09	0	\\
3.96e-09	0	\\
4.06e-09	0	\\
4.16e-09	0	\\
4.27e-09	0	\\
4.37e-09	0	\\
4.47e-09	0	\\
4.57e-09	0	\\
4.68e-09	0	\\
4.78e-09	0	\\
4.89e-09	0	\\
4.99e-09	0	\\
5e-09	0	\\
};
\addplot [color=mycolor1,solid,forget plot]
  table[row sep=crcr]{
0	0	\\
1.1e-10	0	\\
2.2e-10	0	\\
3.3e-10	0	\\
4.4e-10	0	\\
5.4e-10	0	\\
6.5e-10	0	\\
7.5e-10	0	\\
8.6e-10	0	\\
9.6e-10	0	\\
1.07e-09	0	\\
1.18e-09	0	\\
1.28e-09	0	\\
1.38e-09	0	\\
1.49e-09	0	\\
1.59e-09	0	\\
1.69e-09	0	\\
1.8e-09	0	\\
1.9e-09	0	\\
2.01e-09	0	\\
2.11e-09	0	\\
2.21e-09	0	\\
2.32e-09	0	\\
2.42e-09	0	\\
2.52e-09	0	\\
2.63e-09	0	\\
2.73e-09	0	\\
2.83e-09	0	\\
2.93e-09	0	\\
3.04e-09	0	\\
3.14e-09	0	\\
3.24e-09	0	\\
3.34e-09	0	\\
3.45e-09	0	\\
3.55e-09	0	\\
3.65e-09	0	\\
3.75e-09	0	\\
3.86e-09	0	\\
3.96e-09	0	\\
4.06e-09	0	\\
4.16e-09	0	\\
4.27e-09	0	\\
4.37e-09	0	\\
4.47e-09	0	\\
4.57e-09	0	\\
4.68e-09	0	\\
4.78e-09	0	\\
4.89e-09	0	\\
4.99e-09	0	\\
5e-09	0	\\
};
\addplot [color=mycolor2,solid,forget plot]
  table[row sep=crcr]{
0	0	\\
1.1e-10	0	\\
2.2e-10	0	\\
3.3e-10	0	\\
4.4e-10	0	\\
5.4e-10	0	\\
6.5e-10	0	\\
7.5e-10	0	\\
8.6e-10	0	\\
9.6e-10	0	\\
1.07e-09	0	\\
1.18e-09	0	\\
1.28e-09	0	\\
1.38e-09	0	\\
1.49e-09	0	\\
1.59e-09	0	\\
1.69e-09	0	\\
1.8e-09	0	\\
1.9e-09	0	\\
2.01e-09	0	\\
2.11e-09	0	\\
2.21e-09	0	\\
2.32e-09	0	\\
2.42e-09	0	\\
2.52e-09	0	\\
2.63e-09	0	\\
2.73e-09	0	\\
2.83e-09	0	\\
2.93e-09	0	\\
3.04e-09	0	\\
3.14e-09	0	\\
3.24e-09	0	\\
3.34e-09	0	\\
3.45e-09	0	\\
3.55e-09	0	\\
3.65e-09	0	\\
3.75e-09	0	\\
3.86e-09	0	\\
3.96e-09	0	\\
4.06e-09	0	\\
4.16e-09	0	\\
4.27e-09	0	\\
4.37e-09	0	\\
4.47e-09	0	\\
4.57e-09	0	\\
4.68e-09	0	\\
4.78e-09	0	\\
4.89e-09	0	\\
4.99e-09	0	\\
5e-09	0	\\
};
\addplot [color=mycolor3,solid,forget plot]
  table[row sep=crcr]{
0	0	\\
1.1e-10	0	\\
2.2e-10	0	\\
3.3e-10	0	\\
4.4e-10	0	\\
5.4e-10	0	\\
6.5e-10	0	\\
7.5e-10	0	\\
8.6e-10	0	\\
9.6e-10	0	\\
1.07e-09	0	\\
1.18e-09	0	\\
1.28e-09	0	\\
1.38e-09	0	\\
1.49e-09	0	\\
1.59e-09	0	\\
1.69e-09	0	\\
1.8e-09	0	\\
1.9e-09	0	\\
2.01e-09	0	\\
2.11e-09	0	\\
2.21e-09	0	\\
2.32e-09	0	\\
2.42e-09	0	\\
2.52e-09	0	\\
2.63e-09	0	\\
2.73e-09	0	\\
2.83e-09	0	\\
2.93e-09	0	\\
3.04e-09	0	\\
3.14e-09	0	\\
3.24e-09	0	\\
3.34e-09	0	\\
3.45e-09	0	\\
3.55e-09	0	\\
3.65e-09	0	\\
3.75e-09	0	\\
3.86e-09	0	\\
3.96e-09	0	\\
4.06e-09	0	\\
4.16e-09	0	\\
4.27e-09	0	\\
4.37e-09	0	\\
4.47e-09	0	\\
4.57e-09	0	\\
4.68e-09	0	\\
4.78e-09	0	\\
4.89e-09	0	\\
4.99e-09	0	\\
5e-09	0	\\
};
\addplot [color=darkgray,solid,forget plot]
  table[row sep=crcr]{
0	0	\\
1.1e-10	0	\\
2.2e-10	0	\\
3.3e-10	0	\\
4.4e-10	0	\\
5.4e-10	0	\\
6.5e-10	0	\\
7.5e-10	0	\\
8.6e-10	0	\\
9.6e-10	0	\\
1.07e-09	0	\\
1.18e-09	0	\\
1.28e-09	0	\\
1.38e-09	0	\\
1.49e-09	0	\\
1.59e-09	0	\\
1.69e-09	0	\\
1.8e-09	0	\\
1.9e-09	0	\\
2.01e-09	0	\\
2.11e-09	0	\\
2.21e-09	0	\\
2.32e-09	0	\\
2.42e-09	0	\\
2.52e-09	0	\\
2.63e-09	0	\\
2.73e-09	0	\\
2.83e-09	0	\\
2.93e-09	0	\\
3.04e-09	0	\\
3.14e-09	0	\\
3.24e-09	0	\\
3.34e-09	0	\\
3.45e-09	0	\\
3.55e-09	0	\\
3.65e-09	0	\\
3.75e-09	0	\\
3.86e-09	0	\\
3.96e-09	0	\\
4.06e-09	0	\\
4.16e-09	0	\\
4.27e-09	0	\\
4.37e-09	0	\\
4.47e-09	0	\\
4.57e-09	0	\\
4.68e-09	0	\\
4.78e-09	0	\\
4.89e-09	0	\\
4.99e-09	0	\\
5e-09	0	\\
};
\addplot [color=blue,solid,forget plot]
  table[row sep=crcr]{
0	0	\\
1.1e-10	0	\\
2.2e-10	0	\\
3.3e-10	0	\\
4.4e-10	0	\\
5.4e-10	0	\\
6.5e-10	0	\\
7.5e-10	0	\\
8.6e-10	0	\\
9.6e-10	0	\\
1.07e-09	0	\\
1.18e-09	0	\\
1.28e-09	0	\\
1.38e-09	0	\\
1.49e-09	0	\\
1.59e-09	0	\\
1.69e-09	0	\\
1.8e-09	0	\\
1.9e-09	0	\\
2.01e-09	0	\\
2.11e-09	0	\\
2.21e-09	0	\\
2.32e-09	0	\\
2.42e-09	0	\\
2.52e-09	0	\\
2.63e-09	0	\\
2.73e-09	0	\\
2.83e-09	0	\\
2.93e-09	0	\\
3.04e-09	0	\\
3.14e-09	0	\\
3.24e-09	0	\\
3.34e-09	0	\\
3.45e-09	0	\\
3.55e-09	0	\\
3.65e-09	0	\\
3.75e-09	0	\\
3.86e-09	0	\\
3.96e-09	0	\\
4.06e-09	0	\\
4.16e-09	0	\\
4.27e-09	0	\\
4.37e-09	0	\\
4.47e-09	0	\\
4.57e-09	0	\\
4.68e-09	0	\\
4.78e-09	0	\\
4.89e-09	0	\\
4.99e-09	0	\\
5e-09	0	\\
};
\addplot [color=black!50!green,solid,forget plot]
  table[row sep=crcr]{
0	0	\\
1.1e-10	0	\\
2.2e-10	0	\\
3.3e-10	0	\\
4.4e-10	0	\\
5.4e-10	0	\\
6.5e-10	0	\\
7.5e-10	0	\\
8.6e-10	0	\\
9.6e-10	0	\\
1.07e-09	0	\\
1.18e-09	0	\\
1.28e-09	0	\\
1.38e-09	0	\\
1.49e-09	0	\\
1.59e-09	0	\\
1.69e-09	0	\\
1.8e-09	0	\\
1.9e-09	0	\\
2.01e-09	0	\\
2.11e-09	0	\\
2.21e-09	0	\\
2.32e-09	0	\\
2.42e-09	0	\\
2.52e-09	0	\\
2.63e-09	0	\\
2.73e-09	0	\\
2.83e-09	0	\\
2.93e-09	0	\\
3.04e-09	0	\\
3.14e-09	0	\\
3.24e-09	0	\\
3.34e-09	0	\\
3.45e-09	0	\\
3.55e-09	0	\\
3.65e-09	0	\\
3.75e-09	0	\\
3.86e-09	0	\\
3.96e-09	0	\\
4.06e-09	0	\\
4.16e-09	0	\\
4.27e-09	0	\\
4.37e-09	0	\\
4.47e-09	0	\\
4.57e-09	0	\\
4.68e-09	0	\\
4.78e-09	0	\\
4.89e-09	0	\\
4.99e-09	0	\\
5e-09	0	\\
};
\addplot [color=red,solid,forget plot]
  table[row sep=crcr]{
0	0	\\
1.1e-10	0	\\
2.2e-10	0	\\
3.3e-10	0	\\
4.4e-10	0	\\
5.4e-10	0	\\
6.5e-10	0	\\
7.5e-10	0	\\
8.6e-10	0	\\
9.6e-10	0	\\
1.07e-09	0	\\
1.18e-09	0	\\
1.28e-09	0	\\
1.38e-09	0	\\
1.49e-09	0	\\
1.59e-09	0	\\
1.69e-09	0	\\
1.8e-09	0	\\
1.9e-09	0	\\
2.01e-09	0	\\
2.11e-09	0	\\
2.21e-09	0	\\
2.32e-09	0	\\
2.42e-09	0	\\
2.52e-09	0	\\
2.63e-09	0	\\
2.73e-09	0	\\
2.83e-09	0	\\
2.93e-09	0	\\
3.04e-09	0	\\
3.14e-09	0	\\
3.24e-09	0	\\
3.34e-09	0	\\
3.45e-09	0	\\
3.55e-09	0	\\
3.65e-09	0	\\
3.75e-09	0	\\
3.86e-09	0	\\
3.96e-09	0	\\
4.06e-09	0	\\
4.16e-09	0	\\
4.27e-09	0	\\
4.37e-09	0	\\
4.47e-09	0	\\
4.57e-09	0	\\
4.68e-09	0	\\
4.78e-09	0	\\
4.89e-09	0	\\
4.99e-09	0	\\
5e-09	0	\\
};
\addplot [color=mycolor1,solid,forget plot]
  table[row sep=crcr]{
0	0	\\
1.1e-10	0	\\
2.2e-10	0	\\
3.3e-10	0	\\
4.4e-10	0	\\
5.4e-10	0	\\
6.5e-10	0	\\
7.5e-10	0	\\
8.6e-10	0	\\
9.6e-10	0	\\
1.07e-09	0	\\
1.18e-09	0	\\
1.28e-09	0	\\
1.38e-09	0	\\
1.49e-09	0	\\
1.59e-09	0	\\
1.69e-09	0	\\
1.8e-09	0	\\
1.9e-09	0	\\
2.01e-09	0	\\
2.11e-09	0	\\
2.21e-09	0	\\
2.32e-09	0	\\
2.42e-09	0	\\
2.52e-09	0	\\
2.63e-09	0	\\
2.73e-09	0	\\
2.83e-09	0	\\
2.93e-09	0	\\
3.04e-09	0	\\
3.14e-09	0	\\
3.24e-09	0	\\
3.34e-09	0	\\
3.45e-09	0	\\
3.55e-09	0	\\
3.65e-09	0	\\
3.75e-09	0	\\
3.86e-09	0	\\
3.96e-09	0	\\
4.06e-09	0	\\
4.16e-09	0	\\
4.27e-09	0	\\
4.37e-09	0	\\
4.47e-09	0	\\
4.57e-09	0	\\
4.68e-09	0	\\
4.78e-09	0	\\
4.89e-09	0	\\
4.99e-09	0	\\
5e-09	0	\\
};
\addplot [color=mycolor2,solid,forget plot]
  table[row sep=crcr]{
0	0	\\
1.1e-10	0	\\
2.2e-10	0	\\
3.3e-10	0	\\
4.4e-10	0	\\
5.4e-10	0	\\
6.5e-10	0	\\
7.5e-10	0	\\
8.6e-10	0	\\
9.6e-10	0	\\
1.07e-09	0	\\
1.18e-09	0	\\
1.28e-09	0	\\
1.38e-09	0	\\
1.49e-09	0	\\
1.59e-09	0	\\
1.69e-09	0	\\
1.8e-09	0	\\
1.9e-09	0	\\
2.01e-09	0	\\
2.11e-09	0	\\
2.21e-09	0	\\
2.32e-09	0	\\
2.42e-09	0	\\
2.52e-09	0	\\
2.63e-09	0	\\
2.73e-09	0	\\
2.83e-09	0	\\
2.93e-09	0	\\
3.04e-09	0	\\
3.14e-09	0	\\
3.24e-09	0	\\
3.34e-09	0	\\
3.45e-09	0	\\
3.55e-09	0	\\
3.65e-09	0	\\
3.75e-09	0	\\
3.86e-09	0	\\
3.96e-09	0	\\
4.06e-09	0	\\
4.16e-09	0	\\
4.27e-09	0	\\
4.37e-09	0	\\
4.47e-09	0	\\
4.57e-09	0	\\
4.68e-09	0	\\
4.78e-09	0	\\
4.89e-09	0	\\
4.99e-09	0	\\
5e-09	0	\\
};
\addplot [color=mycolor3,solid,forget plot]
  table[row sep=crcr]{
0	0	\\
1.1e-10	0	\\
2.2e-10	0	\\
3.3e-10	0	\\
4.4e-10	0	\\
5.4e-10	0	\\
6.5e-10	0	\\
7.5e-10	0	\\
8.6e-10	0	\\
9.6e-10	0	\\
1.07e-09	0	\\
1.18e-09	0	\\
1.28e-09	0	\\
1.38e-09	0	\\
1.49e-09	0	\\
1.59e-09	0	\\
1.69e-09	0	\\
1.8e-09	0	\\
1.9e-09	0	\\
2.01e-09	0	\\
2.11e-09	0	\\
2.21e-09	0	\\
2.32e-09	0	\\
2.42e-09	0	\\
2.52e-09	0	\\
2.63e-09	0	\\
2.73e-09	0	\\
2.83e-09	0	\\
2.93e-09	0	\\
3.04e-09	0	\\
3.14e-09	0	\\
3.24e-09	0	\\
3.34e-09	0	\\
3.45e-09	0	\\
3.55e-09	0	\\
3.65e-09	0	\\
3.75e-09	0	\\
3.86e-09	0	\\
3.96e-09	0	\\
4.06e-09	0	\\
4.16e-09	0	\\
4.27e-09	0	\\
4.37e-09	0	\\
4.47e-09	0	\\
4.57e-09	0	\\
4.68e-09	0	\\
4.78e-09	0	\\
4.89e-09	0	\\
4.99e-09	0	\\
5e-09	0	\\
};
\addplot [color=darkgray,solid,forget plot]
  table[row sep=crcr]{
0	0	\\
1.1e-10	0	\\
2.2e-10	0	\\
3.3e-10	0	\\
4.4e-10	0	\\
5.4e-10	0	\\
6.5e-10	0	\\
7.5e-10	0	\\
8.6e-10	0	\\
9.6e-10	0	\\
1.07e-09	0	\\
1.18e-09	0	\\
1.28e-09	0	\\
1.38e-09	0	\\
1.49e-09	0	\\
1.59e-09	0	\\
1.69e-09	0	\\
1.8e-09	0	\\
1.9e-09	0	\\
2.01e-09	0	\\
2.11e-09	0	\\
2.21e-09	0	\\
2.32e-09	0	\\
2.42e-09	0	\\
2.52e-09	0	\\
2.63e-09	0	\\
2.73e-09	0	\\
2.83e-09	0	\\
2.93e-09	0	\\
3.04e-09	0	\\
3.14e-09	0	\\
3.24e-09	0	\\
3.34e-09	0	\\
3.45e-09	0	\\
3.55e-09	0	\\
3.65e-09	0	\\
3.75e-09	0	\\
3.86e-09	0	\\
3.96e-09	0	\\
4.06e-09	0	\\
4.16e-09	0	\\
4.27e-09	0	\\
4.37e-09	0	\\
4.47e-09	0	\\
4.57e-09	0	\\
4.68e-09	0	\\
4.78e-09	0	\\
4.89e-09	0	\\
4.99e-09	0	\\
5e-09	0	\\
};
\addplot [color=blue,solid,forget plot]
  table[row sep=crcr]{
0	0	\\
1.1e-10	0	\\
2.2e-10	0	\\
3.3e-10	0	\\
4.4e-10	0	\\
5.4e-10	0	\\
6.5e-10	0	\\
7.5e-10	0	\\
8.6e-10	0	\\
9.6e-10	0	\\
1.07e-09	0	\\
1.18e-09	0	\\
1.28e-09	0	\\
1.38e-09	0	\\
1.49e-09	0	\\
1.59e-09	0	\\
1.69e-09	0	\\
1.8e-09	0	\\
1.9e-09	0	\\
2.01e-09	0	\\
2.11e-09	0	\\
2.21e-09	0	\\
2.32e-09	0	\\
2.42e-09	0	\\
2.52e-09	0	\\
2.63e-09	0	\\
2.73e-09	0	\\
2.83e-09	0	\\
2.93e-09	0	\\
3.04e-09	0	\\
3.14e-09	0	\\
3.24e-09	0	\\
3.34e-09	0	\\
3.45e-09	0	\\
3.55e-09	0	\\
3.65e-09	0	\\
3.75e-09	0	\\
3.86e-09	0	\\
3.96e-09	0	\\
4.06e-09	0	\\
4.16e-09	0	\\
4.27e-09	0	\\
4.37e-09	0	\\
4.47e-09	0	\\
4.57e-09	0	\\
4.68e-09	0	\\
4.78e-09	0	\\
4.89e-09	0	\\
4.99e-09	0	\\
5e-09	0	\\
};
\addplot [color=black!50!green,solid,forget plot]
  table[row sep=crcr]{
0	0	\\
1.1e-10	0	\\
2.2e-10	0	\\
3.3e-10	0	\\
4.4e-10	0	\\
5.4e-10	0	\\
6.5e-10	0	\\
7.5e-10	0	\\
8.6e-10	0	\\
9.6e-10	0	\\
1.07e-09	0	\\
1.18e-09	0	\\
1.28e-09	0	\\
1.38e-09	0	\\
1.49e-09	0	\\
1.59e-09	0	\\
1.69e-09	0	\\
1.8e-09	0	\\
1.9e-09	0	\\
2.01e-09	0	\\
2.11e-09	0	\\
2.21e-09	0	\\
2.32e-09	0	\\
2.42e-09	0	\\
2.52e-09	0	\\
2.63e-09	0	\\
2.73e-09	0	\\
2.83e-09	0	\\
2.93e-09	0	\\
3.04e-09	0	\\
3.14e-09	0	\\
3.24e-09	0	\\
3.34e-09	0	\\
3.45e-09	0	\\
3.55e-09	0	\\
3.65e-09	0	\\
3.75e-09	0	\\
3.86e-09	0	\\
3.96e-09	0	\\
4.06e-09	0	\\
4.16e-09	0	\\
4.27e-09	0	\\
4.37e-09	0	\\
4.47e-09	0	\\
4.57e-09	0	\\
4.68e-09	0	\\
4.78e-09	0	\\
4.89e-09	0	\\
4.99e-09	0	\\
5e-09	0	\\
};
\addplot [color=red,solid,forget plot]
  table[row sep=crcr]{
0	0	\\
1.1e-10	0	\\
2.2e-10	0	\\
3.3e-10	0	\\
4.4e-10	0	\\
5.4e-10	0	\\
6.5e-10	0	\\
7.5e-10	0	\\
8.6e-10	0	\\
9.6e-10	0	\\
1.07e-09	0	\\
1.18e-09	0	\\
1.28e-09	0	\\
1.38e-09	0	\\
1.49e-09	0	\\
1.59e-09	0	\\
1.69e-09	0	\\
1.8e-09	0	\\
1.9e-09	0	\\
2.01e-09	0	\\
2.11e-09	0	\\
2.21e-09	0	\\
2.32e-09	0	\\
2.42e-09	0	\\
2.52e-09	0	\\
2.63e-09	0	\\
2.73e-09	0	\\
2.83e-09	0	\\
2.93e-09	0	\\
3.04e-09	0	\\
3.14e-09	0	\\
3.24e-09	0	\\
3.34e-09	0	\\
3.45e-09	0	\\
3.55e-09	0	\\
3.65e-09	0	\\
3.75e-09	0	\\
3.86e-09	0	\\
3.96e-09	0	\\
4.06e-09	0	\\
4.16e-09	0	\\
4.27e-09	0	\\
4.37e-09	0	\\
4.47e-09	0	\\
4.57e-09	0	\\
4.68e-09	0	\\
4.78e-09	0	\\
4.89e-09	0	\\
4.99e-09	0	\\
5e-09	0	\\
};
\addplot [color=mycolor1,solid,forget plot]
  table[row sep=crcr]{
0	0	\\
1.1e-10	0	\\
2.2e-10	0	\\
3.3e-10	0	\\
4.4e-10	0	\\
5.4e-10	0	\\
6.5e-10	0	\\
7.5e-10	0	\\
8.6e-10	0	\\
9.6e-10	0	\\
1.07e-09	0	\\
1.18e-09	0	\\
1.28e-09	0	\\
1.38e-09	0	\\
1.49e-09	0	\\
1.59e-09	0	\\
1.69e-09	0	\\
1.8e-09	0	\\
1.9e-09	0	\\
2.01e-09	0	\\
2.11e-09	0	\\
2.21e-09	0	\\
2.32e-09	0	\\
2.42e-09	0	\\
2.52e-09	0	\\
2.63e-09	0	\\
2.73e-09	0	\\
2.83e-09	0	\\
2.93e-09	0	\\
3.04e-09	0	\\
3.14e-09	0	\\
3.24e-09	0	\\
3.34e-09	0	\\
3.45e-09	0	\\
3.55e-09	0	\\
3.65e-09	0	\\
3.75e-09	0	\\
3.86e-09	0	\\
3.96e-09	0	\\
4.06e-09	0	\\
4.16e-09	0	\\
4.27e-09	0	\\
4.37e-09	0	\\
4.47e-09	0	\\
4.57e-09	0	\\
4.68e-09	0	\\
4.78e-09	0	\\
4.89e-09	0	\\
4.99e-09	0	\\
5e-09	0	\\
};
\addplot [color=mycolor2,solid,forget plot]
  table[row sep=crcr]{
0	0	\\
1.1e-10	0	\\
2.2e-10	0	\\
3.3e-10	0	\\
4.4e-10	0	\\
5.4e-10	0	\\
6.5e-10	0	\\
7.5e-10	0	\\
8.6e-10	0	\\
9.6e-10	0	\\
1.07e-09	0	\\
1.18e-09	0	\\
1.28e-09	0	\\
1.38e-09	0	\\
1.49e-09	0	\\
1.59e-09	0	\\
1.69e-09	0	\\
1.8e-09	0	\\
1.9e-09	0	\\
2.01e-09	0	\\
2.11e-09	0	\\
2.21e-09	0	\\
2.32e-09	0	\\
2.42e-09	0	\\
2.52e-09	0	\\
2.63e-09	0	\\
2.73e-09	0	\\
2.83e-09	0	\\
2.93e-09	0	\\
3.04e-09	0	\\
3.14e-09	0	\\
3.24e-09	0	\\
3.34e-09	0	\\
3.45e-09	0	\\
3.55e-09	0	\\
3.65e-09	0	\\
3.75e-09	0	\\
3.86e-09	0	\\
3.96e-09	0	\\
4.06e-09	0	\\
4.16e-09	0	\\
4.27e-09	0	\\
4.37e-09	0	\\
4.47e-09	0	\\
4.57e-09	0	\\
4.68e-09	0	\\
4.78e-09	0	\\
4.89e-09	0	\\
4.99e-09	0	\\
5e-09	0	\\
};
\addplot [color=mycolor3,solid,forget plot]
  table[row sep=crcr]{
0	0	\\
1.1e-10	0	\\
2.2e-10	0	\\
3.3e-10	0	\\
4.4e-10	0	\\
5.4e-10	0	\\
6.5e-10	0	\\
7.5e-10	0	\\
8.6e-10	0	\\
9.6e-10	0	\\
1.07e-09	0	\\
1.18e-09	0	\\
1.28e-09	0	\\
1.38e-09	0	\\
1.49e-09	0	\\
1.59e-09	0	\\
1.69e-09	0	\\
1.8e-09	0	\\
1.9e-09	0	\\
2.01e-09	0	\\
2.11e-09	0	\\
2.21e-09	0	\\
2.32e-09	0	\\
2.42e-09	0	\\
2.52e-09	0	\\
2.63e-09	0	\\
2.73e-09	0	\\
2.83e-09	0	\\
2.93e-09	0	\\
3.04e-09	0	\\
3.14e-09	0	\\
3.24e-09	0	\\
3.34e-09	0	\\
3.45e-09	0	\\
3.55e-09	0	\\
3.65e-09	0	\\
3.75e-09	0	\\
3.86e-09	0	\\
3.96e-09	0	\\
4.06e-09	0	\\
4.16e-09	0	\\
4.27e-09	0	\\
4.37e-09	0	\\
4.47e-09	0	\\
4.57e-09	0	\\
4.68e-09	0	\\
4.78e-09	0	\\
4.89e-09	0	\\
4.99e-09	0	\\
5e-09	0	\\
};
\addplot [color=darkgray,solid,forget plot]
  table[row sep=crcr]{
0	0	\\
1.1e-10	0	\\
2.2e-10	0	\\
3.3e-10	0	\\
4.4e-10	0	\\
5.4e-10	0	\\
6.5e-10	0	\\
7.5e-10	0	\\
8.6e-10	0	\\
9.6e-10	0	\\
1.07e-09	0	\\
1.18e-09	0	\\
1.28e-09	0	\\
1.38e-09	0	\\
1.49e-09	0	\\
1.59e-09	0	\\
1.69e-09	0	\\
1.8e-09	0	\\
1.9e-09	0	\\
2.01e-09	0	\\
2.11e-09	0	\\
2.21e-09	0	\\
2.32e-09	0	\\
2.42e-09	0	\\
2.52e-09	0	\\
2.63e-09	0	\\
2.73e-09	0	\\
2.83e-09	0	\\
2.93e-09	0	\\
3.04e-09	0	\\
3.14e-09	0	\\
3.24e-09	0	\\
3.34e-09	0	\\
3.45e-09	0	\\
3.55e-09	0	\\
3.65e-09	0	\\
3.75e-09	0	\\
3.86e-09	0	\\
3.96e-09	0	\\
4.06e-09	0	\\
4.16e-09	0	\\
4.27e-09	0	\\
4.37e-09	0	\\
4.47e-09	0	\\
4.57e-09	0	\\
4.68e-09	0	\\
4.78e-09	0	\\
4.89e-09	0	\\
4.99e-09	0	\\
5e-09	0	\\
};
\addplot [color=blue,solid,forget plot]
  table[row sep=crcr]{
0	0	\\
1.1e-10	0	\\
2.2e-10	0	\\
3.3e-10	0	\\
4.4e-10	0	\\
5.4e-10	0	\\
6.5e-10	0	\\
7.5e-10	0	\\
8.6e-10	0	\\
9.6e-10	0	\\
1.07e-09	0	\\
1.18e-09	0	\\
1.28e-09	0	\\
1.38e-09	0	\\
1.49e-09	0	\\
1.59e-09	0	\\
1.69e-09	0	\\
1.8e-09	0	\\
1.9e-09	0	\\
2.01e-09	0	\\
2.11e-09	0	\\
2.21e-09	0	\\
2.32e-09	0	\\
2.42e-09	0	\\
2.52e-09	0	\\
2.63e-09	0	\\
2.73e-09	0	\\
2.83e-09	0	\\
2.93e-09	0	\\
3.04e-09	0	\\
3.14e-09	0	\\
3.24e-09	0	\\
3.34e-09	0	\\
3.45e-09	0	\\
3.55e-09	0	\\
3.65e-09	0	\\
3.75e-09	0	\\
3.86e-09	0	\\
3.96e-09	0	\\
4.06e-09	0	\\
4.16e-09	0	\\
4.27e-09	0	\\
4.37e-09	0	\\
4.47e-09	0	\\
4.57e-09	0	\\
4.68e-09	0	\\
4.78e-09	0	\\
4.89e-09	0	\\
4.99e-09	0	\\
5e-09	0	\\
};
\addplot [color=black!50!green,solid,forget plot]
  table[row sep=crcr]{
0	0	\\
1.1e-10	0	\\
2.2e-10	0	\\
3.3e-10	0	\\
4.4e-10	0	\\
5.4e-10	0	\\
6.5e-10	0	\\
7.5e-10	0	\\
8.6e-10	0	\\
9.6e-10	0	\\
1.07e-09	0	\\
1.18e-09	0	\\
1.28e-09	0	\\
1.38e-09	0	\\
1.49e-09	0	\\
1.59e-09	0	\\
1.69e-09	0	\\
1.8e-09	0	\\
1.9e-09	0	\\
2.01e-09	0	\\
2.11e-09	0	\\
2.21e-09	0	\\
2.32e-09	0	\\
2.42e-09	0	\\
2.52e-09	0	\\
2.63e-09	0	\\
2.73e-09	0	\\
2.83e-09	0	\\
2.93e-09	0	\\
3.04e-09	0	\\
3.14e-09	0	\\
3.24e-09	0	\\
3.34e-09	0	\\
3.45e-09	0	\\
3.55e-09	0	\\
3.65e-09	0	\\
3.75e-09	0	\\
3.86e-09	0	\\
3.96e-09	0	\\
4.06e-09	0	\\
4.16e-09	0	\\
4.27e-09	0	\\
4.37e-09	0	\\
4.47e-09	0	\\
4.57e-09	0	\\
4.68e-09	0	\\
4.78e-09	0	\\
4.89e-09	0	\\
4.99e-09	0	\\
5e-09	0	\\
};
\addplot [color=red,solid,forget plot]
  table[row sep=crcr]{
0	0	\\
1.1e-10	0	\\
2.2e-10	0	\\
3.3e-10	0	\\
4.4e-10	0	\\
5.4e-10	0	\\
6.5e-10	0	\\
7.5e-10	0	\\
8.6e-10	0	\\
9.6e-10	0	\\
1.07e-09	0	\\
1.18e-09	0	\\
1.28e-09	0	\\
1.38e-09	0	\\
1.49e-09	0	\\
1.59e-09	0	\\
1.69e-09	0	\\
1.8e-09	0	\\
1.9e-09	0	\\
2.01e-09	0	\\
2.11e-09	0	\\
2.21e-09	0	\\
2.32e-09	0	\\
2.42e-09	0	\\
2.52e-09	0	\\
2.63e-09	0	\\
2.73e-09	0	\\
2.83e-09	0	\\
2.93e-09	0	\\
3.04e-09	0	\\
3.14e-09	0	\\
3.24e-09	0	\\
3.34e-09	0	\\
3.45e-09	0	\\
3.55e-09	0	\\
3.65e-09	0	\\
3.75e-09	0	\\
3.86e-09	0	\\
3.96e-09	0	\\
4.06e-09	0	\\
4.16e-09	0	\\
4.27e-09	0	\\
4.37e-09	0	\\
4.47e-09	0	\\
4.57e-09	0	\\
4.68e-09	0	\\
4.78e-09	0	\\
4.89e-09	0	\\
4.99e-09	0	\\
5e-09	0	\\
};
\addplot [color=mycolor1,solid,forget plot]
  table[row sep=crcr]{
0	0	\\
1.1e-10	0	\\
2.2e-10	0	\\
3.3e-10	0	\\
4.4e-10	0	\\
5.4e-10	0	\\
6.5e-10	0	\\
7.5e-10	0	\\
8.6e-10	0	\\
9.6e-10	0	\\
1.07e-09	0	\\
1.18e-09	0	\\
1.28e-09	0	\\
1.38e-09	0	\\
1.49e-09	0	\\
1.59e-09	0	\\
1.69e-09	0	\\
1.8e-09	0	\\
1.9e-09	0	\\
2.01e-09	0	\\
2.11e-09	0	\\
2.21e-09	0	\\
2.32e-09	0	\\
2.42e-09	0	\\
2.52e-09	0	\\
2.63e-09	0	\\
2.73e-09	0	\\
2.83e-09	0	\\
2.93e-09	0	\\
3.04e-09	0	\\
3.14e-09	0	\\
3.24e-09	0	\\
3.34e-09	0	\\
3.45e-09	0	\\
3.55e-09	0	\\
3.65e-09	0	\\
3.75e-09	0	\\
3.86e-09	0	\\
3.96e-09	0	\\
4.06e-09	0	\\
4.16e-09	0	\\
4.27e-09	0	\\
4.37e-09	0	\\
4.47e-09	0	\\
4.57e-09	0	\\
4.68e-09	0	\\
4.78e-09	0	\\
4.89e-09	0	\\
4.99e-09	0	\\
5e-09	0	\\
};
\addplot [color=mycolor2,solid,forget plot]
  table[row sep=crcr]{
0	0	\\
1.1e-10	0	\\
2.2e-10	0	\\
3.3e-10	0	\\
4.4e-10	0	\\
5.4e-10	0	\\
6.5e-10	0	\\
7.5e-10	0	\\
8.6e-10	0	\\
9.6e-10	0	\\
1.07e-09	0	\\
1.18e-09	0	\\
1.28e-09	0	\\
1.38e-09	0	\\
1.49e-09	0	\\
1.59e-09	0	\\
1.69e-09	0	\\
1.8e-09	0	\\
1.9e-09	0	\\
2.01e-09	0	\\
2.11e-09	0	\\
2.21e-09	0	\\
2.32e-09	0	\\
2.42e-09	0	\\
2.52e-09	0	\\
2.63e-09	0	\\
2.73e-09	0	\\
2.83e-09	0	\\
2.93e-09	0	\\
3.04e-09	0	\\
3.14e-09	0	\\
3.24e-09	0	\\
3.34e-09	0	\\
3.45e-09	0	\\
3.55e-09	0	\\
3.65e-09	0	\\
3.75e-09	0	\\
3.86e-09	0	\\
3.96e-09	0	\\
4.06e-09	0	\\
4.16e-09	0	\\
4.27e-09	0	\\
4.37e-09	0	\\
4.47e-09	0	\\
4.57e-09	0	\\
4.68e-09	0	\\
4.78e-09	0	\\
4.89e-09	0	\\
4.99e-09	0	\\
5e-09	0	\\
};
\addplot [color=mycolor3,solid,forget plot]
  table[row sep=crcr]{
0	0	\\
1.1e-10	0	\\
2.2e-10	0	\\
3.3e-10	0	\\
4.4e-10	0	\\
5.4e-10	0	\\
6.5e-10	0	\\
7.5e-10	0	\\
8.6e-10	0	\\
9.6e-10	0	\\
1.07e-09	0	\\
1.18e-09	0	\\
1.28e-09	0	\\
1.38e-09	0	\\
1.49e-09	0	\\
1.59e-09	0	\\
1.69e-09	0	\\
1.8e-09	0	\\
1.9e-09	0	\\
2.01e-09	0	\\
2.11e-09	0	\\
2.21e-09	0	\\
2.32e-09	0	\\
2.42e-09	0	\\
2.52e-09	0	\\
2.63e-09	0	\\
2.73e-09	0	\\
2.83e-09	0	\\
2.93e-09	0	\\
3.04e-09	0	\\
3.14e-09	0	\\
3.24e-09	0	\\
3.34e-09	0	\\
3.45e-09	0	\\
3.55e-09	0	\\
3.65e-09	0	\\
3.75e-09	0	\\
3.86e-09	0	\\
3.96e-09	0	\\
4.06e-09	0	\\
4.16e-09	0	\\
4.27e-09	0	\\
4.37e-09	0	\\
4.47e-09	0	\\
4.57e-09	0	\\
4.68e-09	0	\\
4.78e-09	0	\\
4.89e-09	0	\\
4.99e-09	0	\\
5e-09	0	\\
};
\addplot [color=darkgray,solid,forget plot]
  table[row sep=crcr]{
0	0	\\
1.1e-10	0	\\
2.2e-10	0	\\
3.3e-10	0	\\
4.4e-10	0	\\
5.4e-10	0	\\
6.5e-10	0	\\
7.5e-10	0	\\
8.6e-10	0	\\
9.6e-10	0	\\
1.07e-09	0	\\
1.18e-09	0	\\
1.28e-09	0	\\
1.38e-09	0	\\
1.49e-09	0	\\
1.59e-09	0	\\
1.69e-09	0	\\
1.8e-09	0	\\
1.9e-09	0	\\
2.01e-09	0	\\
2.11e-09	0	\\
2.21e-09	0	\\
2.32e-09	0	\\
2.42e-09	0	\\
2.52e-09	0	\\
2.63e-09	0	\\
2.73e-09	0	\\
2.83e-09	0	\\
2.93e-09	0	\\
3.04e-09	0	\\
3.14e-09	0	\\
3.24e-09	0	\\
3.34e-09	0	\\
3.45e-09	0	\\
3.55e-09	0	\\
3.65e-09	0	\\
3.75e-09	0	\\
3.86e-09	0	\\
3.96e-09	0	\\
4.06e-09	0	\\
4.16e-09	0	\\
4.27e-09	0	\\
4.37e-09	0	\\
4.47e-09	0	\\
4.57e-09	0	\\
4.68e-09	0	\\
4.78e-09	0	\\
4.89e-09	0	\\
4.99e-09	0	\\
5e-09	0	\\
};
\addplot [color=blue,solid,forget plot]
  table[row sep=crcr]{
0	0	\\
1.1e-10	0	\\
2.2e-10	0	\\
3.3e-10	0	\\
4.4e-10	0	\\
5.4e-10	0	\\
6.5e-10	0	\\
7.5e-10	0	\\
8.6e-10	0	\\
9.6e-10	0	\\
1.07e-09	0	\\
1.18e-09	0	\\
1.28e-09	0	\\
1.38e-09	0	\\
1.49e-09	0	\\
1.59e-09	0	\\
1.69e-09	0	\\
1.8e-09	0	\\
1.9e-09	0	\\
2.01e-09	0	\\
2.11e-09	0	\\
2.21e-09	0	\\
2.32e-09	0	\\
2.42e-09	0	\\
2.52e-09	0	\\
2.63e-09	0	\\
2.73e-09	0	\\
2.83e-09	0	\\
2.93e-09	0	\\
3.04e-09	0	\\
3.14e-09	0	\\
3.24e-09	0	\\
3.34e-09	0	\\
3.45e-09	0	\\
3.55e-09	0	\\
3.65e-09	0	\\
3.75e-09	0	\\
3.86e-09	0	\\
3.96e-09	0	\\
4.06e-09	0	\\
4.16e-09	0	\\
4.27e-09	0	\\
4.37e-09	0	\\
4.47e-09	0	\\
4.57e-09	0	\\
4.68e-09	0	\\
4.78e-09	0	\\
4.89e-09	0	\\
4.99e-09	0	\\
5e-09	0	\\
};
\addplot [color=black!50!green,solid,forget plot]
  table[row sep=crcr]{
0	0	\\
1.1e-10	0	\\
2.2e-10	0	\\
3.3e-10	0	\\
4.4e-10	0	\\
5.4e-10	0	\\
6.5e-10	0	\\
7.5e-10	0	\\
8.6e-10	0	\\
9.6e-10	0	\\
1.07e-09	0	\\
1.18e-09	0	\\
1.28e-09	0	\\
1.38e-09	0	\\
1.49e-09	0	\\
1.59e-09	0	\\
1.69e-09	0	\\
1.8e-09	0	\\
1.9e-09	0	\\
2.01e-09	0	\\
2.11e-09	0	\\
2.21e-09	0	\\
2.32e-09	0	\\
2.42e-09	0	\\
2.52e-09	0	\\
2.63e-09	0	\\
2.73e-09	0	\\
2.83e-09	0	\\
2.93e-09	0	\\
3.04e-09	0	\\
3.14e-09	0	\\
3.24e-09	0	\\
3.34e-09	0	\\
3.45e-09	0	\\
3.55e-09	0	\\
3.65e-09	0	\\
3.75e-09	0	\\
3.86e-09	0	\\
3.96e-09	0	\\
4.06e-09	0	\\
4.16e-09	0	\\
4.27e-09	0	\\
4.37e-09	0	\\
4.47e-09	0	\\
4.57e-09	0	\\
4.68e-09	0	\\
4.78e-09	0	\\
4.89e-09	0	\\
4.99e-09	0	\\
5e-09	0	\\
};
\addplot [color=red,solid,forget plot]
  table[row sep=crcr]{
0	0	\\
1.1e-10	0	\\
2.2e-10	0	\\
3.3e-10	0	\\
4.4e-10	0	\\
5.4e-10	0	\\
6.5e-10	0	\\
7.5e-10	0	\\
8.6e-10	0	\\
9.6e-10	0	\\
1.07e-09	0	\\
1.18e-09	0	\\
1.28e-09	0	\\
1.38e-09	0	\\
1.49e-09	0	\\
1.59e-09	0	\\
1.69e-09	0	\\
1.8e-09	0	\\
1.9e-09	0	\\
2.01e-09	0	\\
2.11e-09	0	\\
2.21e-09	0	\\
2.32e-09	0	\\
2.42e-09	0	\\
2.52e-09	0	\\
2.63e-09	0	\\
2.73e-09	0	\\
2.83e-09	0	\\
2.93e-09	0	\\
3.04e-09	0	\\
3.14e-09	0	\\
3.24e-09	0	\\
3.34e-09	0	\\
3.45e-09	0	\\
3.55e-09	0	\\
3.65e-09	0	\\
3.75e-09	0	\\
3.86e-09	0	\\
3.96e-09	0	\\
4.06e-09	0	\\
4.16e-09	0	\\
4.27e-09	0	\\
4.37e-09	0	\\
4.47e-09	0	\\
4.57e-09	0	\\
4.68e-09	0	\\
4.78e-09	0	\\
4.89e-09	0	\\
4.99e-09	0	\\
5e-09	0	\\
};
\addplot [color=mycolor1,solid,forget plot]
  table[row sep=crcr]{
0	0	\\
1.1e-10	0	\\
2.2e-10	0	\\
3.3e-10	0	\\
4.4e-10	0	\\
5.4e-10	0	\\
6.5e-10	0	\\
7.5e-10	0	\\
8.6e-10	0	\\
9.6e-10	0	\\
1.07e-09	0	\\
1.18e-09	0	\\
1.28e-09	0	\\
1.38e-09	0	\\
1.49e-09	0	\\
1.59e-09	0	\\
1.69e-09	0	\\
1.8e-09	0	\\
1.9e-09	0	\\
2.01e-09	0	\\
2.11e-09	0	\\
2.21e-09	0	\\
2.32e-09	0	\\
2.42e-09	0	\\
2.52e-09	0	\\
2.63e-09	0	\\
2.73e-09	0	\\
2.83e-09	0	\\
2.93e-09	0	\\
3.04e-09	0	\\
3.14e-09	0	\\
3.24e-09	0	\\
3.34e-09	0	\\
3.45e-09	0	\\
3.55e-09	0	\\
3.65e-09	0	\\
3.75e-09	0	\\
3.86e-09	0	\\
3.96e-09	0	\\
4.06e-09	0	\\
4.16e-09	0	\\
4.27e-09	0	\\
4.37e-09	0	\\
4.47e-09	0	\\
4.57e-09	0	\\
4.68e-09	0	\\
4.78e-09	0	\\
4.89e-09	0	\\
4.99e-09	0	\\
5e-09	0	\\
};
\addplot [color=mycolor2,solid,forget plot]
  table[row sep=crcr]{
0	0	\\
1.1e-10	0	\\
2.2e-10	0	\\
3.3e-10	0	\\
4.4e-10	0	\\
5.4e-10	0	\\
6.5e-10	0	\\
7.5e-10	0	\\
8.6e-10	0	\\
9.6e-10	0	\\
1.07e-09	0	\\
1.18e-09	0	\\
1.28e-09	0	\\
1.38e-09	0	\\
1.49e-09	0	\\
1.59e-09	0	\\
1.69e-09	0	\\
1.8e-09	0	\\
1.9e-09	0	\\
2.01e-09	0	\\
2.11e-09	0	\\
2.21e-09	0	\\
2.32e-09	0	\\
2.42e-09	0	\\
2.52e-09	0	\\
2.63e-09	0	\\
2.73e-09	0	\\
2.83e-09	0	\\
2.93e-09	0	\\
3.04e-09	0	\\
3.14e-09	0	\\
3.24e-09	0	\\
3.34e-09	0	\\
3.45e-09	0	\\
3.55e-09	0	\\
3.65e-09	0	\\
3.75e-09	0	\\
3.86e-09	0	\\
3.96e-09	0	\\
4.06e-09	0	\\
4.16e-09	0	\\
4.27e-09	0	\\
4.37e-09	0	\\
4.47e-09	0	\\
4.57e-09	0	\\
4.68e-09	0	\\
4.78e-09	0	\\
4.89e-09	0	\\
4.99e-09	0	\\
5e-09	0	\\
};
\addplot [color=mycolor3,solid,forget plot]
  table[row sep=crcr]{
0	0	\\
1.1e-10	0	\\
2.2e-10	0	\\
3.3e-10	0	\\
4.4e-10	0	\\
5.4e-10	0	\\
6.5e-10	0	\\
7.5e-10	0	\\
8.6e-10	0	\\
9.6e-10	0	\\
1.07e-09	0	\\
1.18e-09	0	\\
1.28e-09	0	\\
1.38e-09	0	\\
1.49e-09	0	\\
1.59e-09	0	\\
1.69e-09	0	\\
1.8e-09	0	\\
1.9e-09	0	\\
2.01e-09	0	\\
2.11e-09	0	\\
2.21e-09	0	\\
2.32e-09	0	\\
2.42e-09	0	\\
2.52e-09	0	\\
2.63e-09	0	\\
2.73e-09	0	\\
2.83e-09	0	\\
2.93e-09	0	\\
3.04e-09	0	\\
3.14e-09	0	\\
3.24e-09	0	\\
3.34e-09	0	\\
3.45e-09	0	\\
3.55e-09	0	\\
3.65e-09	0	\\
3.75e-09	0	\\
3.86e-09	0	\\
3.96e-09	0	\\
4.06e-09	0	\\
4.16e-09	0	\\
4.27e-09	0	\\
4.37e-09	0	\\
4.47e-09	0	\\
4.57e-09	0	\\
4.68e-09	0	\\
4.78e-09	0	\\
4.89e-09	0	\\
4.99e-09	0	\\
5e-09	0	\\
};
\addplot [color=darkgray,solid,forget plot]
  table[row sep=crcr]{
0	0	\\
1.1e-10	0	\\
2.2e-10	0	\\
3.3e-10	0	\\
4.4e-10	0	\\
5.4e-10	0	\\
6.5e-10	0	\\
7.5e-10	0	\\
8.6e-10	0	\\
9.6e-10	0	\\
1.07e-09	0	\\
1.18e-09	0	\\
1.28e-09	0	\\
1.38e-09	0	\\
1.49e-09	0	\\
1.59e-09	0	\\
1.69e-09	0	\\
1.8e-09	0	\\
1.9e-09	0	\\
2.01e-09	0	\\
2.11e-09	0	\\
2.21e-09	0	\\
2.32e-09	0	\\
2.42e-09	0	\\
2.52e-09	0	\\
2.63e-09	0	\\
2.73e-09	0	\\
2.83e-09	0	\\
2.93e-09	0	\\
3.04e-09	0	\\
3.14e-09	0	\\
3.24e-09	0	\\
3.34e-09	0	\\
3.45e-09	0	\\
3.55e-09	0	\\
3.65e-09	0	\\
3.75e-09	0	\\
3.86e-09	0	\\
3.96e-09	0	\\
4.06e-09	0	\\
4.16e-09	0	\\
4.27e-09	0	\\
4.37e-09	0	\\
4.47e-09	0	\\
4.57e-09	0	\\
4.68e-09	0	\\
4.78e-09	0	\\
4.89e-09	0	\\
4.99e-09	0	\\
5e-09	0	\\
};
\addplot [color=blue,solid,forget plot]
  table[row sep=crcr]{
0	0	\\
1.1e-10	0	\\
2.2e-10	0	\\
3.3e-10	0	\\
4.4e-10	0	\\
5.4e-10	0	\\
6.5e-10	0	\\
7.5e-10	0	\\
8.6e-10	0	\\
9.6e-10	0	\\
1.07e-09	0	\\
1.18e-09	0	\\
1.28e-09	0	\\
1.38e-09	0	\\
1.49e-09	0	\\
1.59e-09	0	\\
1.69e-09	0	\\
1.8e-09	0	\\
1.9e-09	0	\\
2.01e-09	0	\\
2.11e-09	0	\\
2.21e-09	0	\\
2.32e-09	0	\\
2.42e-09	0	\\
2.52e-09	0	\\
2.63e-09	0	\\
2.73e-09	0	\\
2.83e-09	0	\\
2.93e-09	0	\\
3.04e-09	0	\\
3.14e-09	0	\\
3.24e-09	0	\\
3.34e-09	0	\\
3.45e-09	0	\\
3.55e-09	0	\\
3.65e-09	0	\\
3.75e-09	0	\\
3.86e-09	0	\\
3.96e-09	0	\\
4.06e-09	0	\\
4.16e-09	0	\\
4.27e-09	0	\\
4.37e-09	0	\\
4.47e-09	0	\\
4.57e-09	0	\\
4.68e-09	0	\\
4.78e-09	0	\\
4.89e-09	0	\\
4.99e-09	0	\\
5e-09	0	\\
};
\addplot [color=black!50!green,solid,forget plot]
  table[row sep=crcr]{
0	0	\\
1.1e-10	0	\\
2.2e-10	0	\\
3.3e-10	0	\\
4.4e-10	0	\\
5.4e-10	0	\\
6.5e-10	0	\\
7.5e-10	0	\\
8.6e-10	0	\\
9.6e-10	0	\\
1.07e-09	0	\\
1.18e-09	0	\\
1.28e-09	0	\\
1.38e-09	0	\\
1.49e-09	0	\\
1.59e-09	0	\\
1.69e-09	0	\\
1.8e-09	0	\\
1.9e-09	0	\\
2.01e-09	0	\\
2.11e-09	0	\\
2.21e-09	0	\\
2.32e-09	0	\\
2.42e-09	0	\\
2.52e-09	0	\\
2.63e-09	0	\\
2.73e-09	0	\\
2.83e-09	0	\\
2.93e-09	0	\\
3.04e-09	0	\\
3.14e-09	0	\\
3.24e-09	0	\\
3.34e-09	0	\\
3.45e-09	0	\\
3.55e-09	0	\\
3.65e-09	0	\\
3.75e-09	0	\\
3.86e-09	0	\\
3.96e-09	0	\\
4.06e-09	0	\\
4.16e-09	0	\\
4.27e-09	0	\\
4.37e-09	0	\\
4.47e-09	0	\\
4.57e-09	0	\\
4.68e-09	0	\\
4.78e-09	0	\\
4.89e-09	0	\\
4.99e-09	0	\\
5e-09	0	\\
};
\addplot [color=red,solid,forget plot]
  table[row sep=crcr]{
0	0	\\
1.1e-10	0	\\
2.2e-10	0	\\
3.3e-10	0	\\
4.4e-10	0	\\
5.4e-10	0	\\
6.5e-10	0	\\
7.5e-10	0	\\
8.6e-10	0	\\
9.6e-10	0	\\
1.07e-09	0	\\
1.18e-09	0	\\
1.28e-09	0	\\
1.38e-09	0	\\
1.49e-09	0	\\
1.59e-09	0	\\
1.69e-09	0	\\
1.8e-09	0	\\
1.9e-09	0	\\
2.01e-09	0	\\
2.11e-09	0	\\
2.21e-09	0	\\
2.32e-09	0	\\
2.42e-09	0	\\
2.52e-09	0	\\
2.63e-09	0	\\
2.73e-09	0	\\
2.83e-09	0	\\
2.93e-09	0	\\
3.04e-09	0	\\
3.14e-09	0	\\
3.24e-09	0	\\
3.34e-09	0	\\
3.45e-09	0	\\
3.55e-09	0	\\
3.65e-09	0	\\
3.75e-09	0	\\
3.86e-09	0	\\
3.96e-09	0	\\
4.06e-09	0	\\
4.16e-09	0	\\
4.27e-09	0	\\
4.37e-09	0	\\
4.47e-09	0	\\
4.57e-09	0	\\
4.68e-09	0	\\
4.78e-09	0	\\
4.89e-09	0	\\
4.99e-09	0	\\
5e-09	0	\\
};
\addplot [color=mycolor1,solid,forget plot]
  table[row sep=crcr]{
0	0	\\
1.1e-10	0	\\
2.2e-10	0	\\
3.3e-10	0	\\
4.4e-10	0	\\
5.4e-10	0	\\
6.5e-10	0	\\
7.5e-10	0	\\
8.6e-10	0	\\
9.6e-10	0	\\
1.07e-09	0	\\
1.18e-09	0	\\
1.28e-09	0	\\
1.38e-09	0	\\
1.49e-09	0	\\
1.59e-09	0	\\
1.69e-09	0	\\
1.8e-09	0	\\
1.9e-09	0	\\
2.01e-09	0	\\
2.11e-09	0	\\
2.21e-09	0	\\
2.32e-09	0	\\
2.42e-09	0	\\
2.52e-09	0	\\
2.63e-09	0	\\
2.73e-09	0	\\
2.83e-09	0	\\
2.93e-09	0	\\
3.04e-09	0	\\
3.14e-09	0	\\
3.24e-09	0	\\
3.34e-09	0	\\
3.45e-09	0	\\
3.55e-09	0	\\
3.65e-09	0	\\
3.75e-09	0	\\
3.86e-09	0	\\
3.96e-09	0	\\
4.06e-09	0	\\
4.16e-09	0	\\
4.27e-09	0	\\
4.37e-09	0	\\
4.47e-09	0	\\
4.57e-09	0	\\
4.68e-09	0	\\
4.78e-09	0	\\
4.89e-09	0	\\
4.99e-09	0	\\
5e-09	0	\\
};
\addplot [color=mycolor2,solid,forget plot]
  table[row sep=crcr]{
0	0	\\
1.1e-10	0	\\
2.2e-10	0	\\
3.3e-10	0	\\
4.4e-10	0	\\
5.4e-10	0	\\
6.5e-10	0	\\
7.5e-10	0	\\
8.6e-10	0	\\
9.6e-10	0	\\
1.07e-09	0	\\
1.18e-09	0	\\
1.2e-09	0.000526740123622682	\\
1.31e-09	0.000526740123622682	\\
1.41e-09	0.000526740123622682	\\
1.51e-09	0.000526740123622682	\\
1.62e-09	0.000526740123622682	\\
1.7e-09	0	\\
1.8e-09	0	\\
1.9e-09	0	\\
2.01e-09	0	\\
2.11e-09	0	\\
2.21e-09	0	\\
2.32e-09	0	\\
2.42e-09	0	\\
2.52e-09	0	\\
2.63e-09	0	\\
2.73e-09	0	\\
2.83e-09	0	\\
2.93e-09	0	\\
3.04e-09	0	\\
3.14e-09	0	\\
3.24e-09	0	\\
3.34e-09	0	\\
3.45e-09	0	\\
3.55e-09	0	\\
3.6e-09	0.00012471326227132	\\
3.7e-09	0.00012471326227132	\\
3.8e-09	0.00012471326227132	\\
3.9e-09	0.00012471326227132	\\
4e-09	0.00012471326227132	\\
4.1e-09	0	\\
4.2e-09	0	\\
4.3e-09	0	\\
4.41e-09	0	\\
4.51e-09	0	\\
4.61e-09	0	\\
4.71e-09	0	\\
4.82e-09	0	\\
4.92e-09	0	\\
5e-09	0	\\
};
\addplot [color=mycolor3,solid,forget plot]
  table[row sep=crcr]{
0	0	\\
1.1e-10	0	\\
2.2e-10	0	\\
3.3e-10	0	\\
4.4e-10	0	\\
5.4e-10	0	\\
6.5e-10	0	\\
7.5e-10	0	\\
8.6e-10	0	\\
9.6e-10	0	\\
1.07e-09	0	\\
1.18e-09	0	\\
1.28e-09	0	\\
1.38e-09	0	\\
1.49e-09	0	\\
1.59e-09	0	\\
1.69e-09	0	\\
1.8e-09	0	\\
1.9e-09	0	\\
2.01e-09	0	\\
2.11e-09	0	\\
2.21e-09	0	\\
2.32e-09	0	\\
2.42e-09	0	\\
2.52e-09	0	\\
2.63e-09	0	\\
2.73e-09	0	\\
2.83e-09	0	\\
2.93e-09	0	\\
3.04e-09	0	\\
3.14e-09	0	\\
3.24e-09	0	\\
3.34e-09	0	\\
3.45e-09	0	\\
3.55e-09	0	\\
3.65e-09	0	\\
3.75e-09	0	\\
3.86e-09	0	\\
3.96e-09	0	\\
4.06e-09	0	\\
4.16e-09	0	\\
4.27e-09	0	\\
4.37e-09	0	\\
4.47e-09	0	\\
4.57e-09	0	\\
4.68e-09	0	\\
4.78e-09	0	\\
4.89e-09	0	\\
4.99e-09	0	\\
5e-09	0	\\
};
\addplot [color=darkgray,solid,forget plot]
  table[row sep=crcr]{
0	0	\\
1.1e-10	0	\\
2.2e-10	0	\\
3.3e-10	0	\\
4.4e-10	0	\\
5.4e-10	0	\\
6.5e-10	0	\\
7.5e-10	0	\\
8.6e-10	0	\\
9.6e-10	0	\\
1.07e-09	0	\\
1.18e-09	0	\\
1.28e-09	0	\\
1.38e-09	0	\\
1.49e-09	0	\\
1.59e-09	0	\\
1.69e-09	0	\\
1.8e-09	0	\\
1.9e-09	0	\\
2.01e-09	0	\\
2.11e-09	0	\\
2.21e-09	0	\\
2.32e-09	0	\\
2.42e-09	0	\\
2.52e-09	0	\\
2.63e-09	0	\\
2.73e-09	0	\\
2.83e-09	0	\\
2.93e-09	0	\\
3.04e-09	0	\\
3.14e-09	0	\\
3.24e-09	0	\\
3.34e-09	0	\\
3.45e-09	0	\\
3.55e-09	0	\\
3.65e-09	0	\\
3.75e-09	0	\\
3.86e-09	0	\\
3.96e-09	0	\\
4.06e-09	0	\\
4.16e-09	0	\\
4.27e-09	0	\\
4.37e-09	0	\\
4.47e-09	0	\\
4.57e-09	0	\\
4.68e-09	0	\\
4.78e-09	0	\\
4.89e-09	0	\\
4.99e-09	0	\\
5e-09	0	\\
};
\addplot [color=blue,solid,forget plot]
  table[row sep=crcr]{
0	0	\\
1.1e-10	0	\\
2.2e-10	0	\\
3.3e-10	0	\\
4.4e-10	0	\\
5.4e-10	0	\\
6.5e-10	0	\\
7.5e-10	0	\\
8.6e-10	0	\\
9.6e-10	0	\\
1.07e-09	0	\\
1.18e-09	0	\\
1.28e-09	0	\\
1.38e-09	0	\\
1.49e-09	0	\\
1.59e-09	0	\\
1.69e-09	0	\\
1.8e-09	0	\\
1.9e-09	0	\\
2.01e-09	0	\\
2.11e-09	0	\\
2.21e-09	0	\\
2.32e-09	0	\\
2.42e-09	0	\\
2.52e-09	0	\\
2.63e-09	0	\\
2.73e-09	0	\\
2.83e-09	0	\\
2.93e-09	0	\\
3.04e-09	0	\\
3.14e-09	0	\\
3.24e-09	0	\\
3.34e-09	0	\\
3.45e-09	0	\\
3.55e-09	0	\\
3.65e-09	0	\\
3.75e-09	0	\\
3.86e-09	0	\\
3.96e-09	0	\\
4.06e-09	0	\\
4.16e-09	0	\\
4.27e-09	0	\\
4.37e-09	0	\\
4.47e-09	0	\\
4.57e-09	0	\\
4.68e-09	0	\\
4.78e-09	0	\\
4.89e-09	0	\\
4.99e-09	0	\\
5e-09	0	\\
};
\addplot [color=black!50!green,solid,forget plot]
  table[row sep=crcr]{
0	0	\\
1.1e-10	0	\\
2.2e-10	0	\\
3.3e-10	0	\\
4.4e-10	0	\\
5.4e-10	0	\\
6.5e-10	0	\\
7.5e-10	0	\\
8.6e-10	0	\\
9.6e-10	0	\\
1.07e-09	0	\\
1.18e-09	0	\\
1.28e-09	0	\\
1.38e-09	0	\\
1.49e-09	0	\\
1.59e-09	0	\\
1.69e-09	0	\\
1.8e-09	0	\\
1.9e-09	0	\\
2.01e-09	0	\\
2.11e-09	0	\\
2.21e-09	0	\\
2.32e-09	0	\\
2.42e-09	0	\\
2.52e-09	0	\\
2.63e-09	0	\\
2.73e-09	0	\\
2.83e-09	0	\\
2.93e-09	0	\\
3.04e-09	0	\\
3.14e-09	0	\\
3.24e-09	0	\\
3.34e-09	0	\\
3.45e-09	0	\\
3.55e-09	0	\\
3.65e-09	0	\\
3.75e-09	0	\\
3.86e-09	0	\\
3.96e-09	0	\\
4.06e-09	0	\\
4.16e-09	0	\\
4.27e-09	0	\\
4.37e-09	0	\\
4.47e-09	0	\\
4.57e-09	0	\\
4.68e-09	0	\\
4.78e-09	0	\\
4.89e-09	0	\\
4.99e-09	0	\\
5e-09	0	\\
};
\addplot [color=red,solid,forget plot]
  table[row sep=crcr]{
0	0	\\
1.1e-10	0	\\
2.2e-10	0	\\
3.3e-10	0	\\
4.4e-10	0	\\
5.4e-10	0	\\
6.5e-10	0	\\
7.5e-10	0	\\
8.6e-10	0	\\
9.6e-10	0	\\
1.07e-09	0	\\
1.18e-09	0	\\
1.28e-09	0	\\
1.38e-09	0	\\
1.49e-09	0	\\
1.59e-09	0	\\
1.69e-09	0	\\
1.8e-09	0	\\
1.9e-09	0	\\
2.01e-09	0	\\
2.11e-09	0	\\
2.21e-09	0	\\
2.32e-09	0	\\
2.42e-09	0	\\
2.52e-09	0	\\
2.63e-09	0	\\
2.73e-09	0	\\
2.83e-09	0	\\
2.93e-09	0	\\
3.04e-09	0	\\
3.14e-09	0	\\
3.24e-09	0	\\
3.34e-09	0	\\
3.45e-09	0	\\
3.55e-09	0	\\
3.65e-09	0	\\
3.75e-09	0	\\
3.86e-09	0	\\
3.96e-09	0	\\
4.06e-09	0	\\
4.16e-09	0	\\
4.27e-09	0	\\
4.37e-09	0	\\
4.47e-09	0	\\
4.57e-09	0	\\
4.68e-09	0	\\
4.78e-09	0	\\
4.89e-09	0	\\
4.99e-09	0	\\
5e-09	0	\\
};
\addplot [color=mycolor1,solid,forget plot]
  table[row sep=crcr]{
0	0	\\
1.1e-10	0	\\
2.2e-10	0	\\
3.3e-10	0	\\
4.4e-10	0	\\
5.4e-10	0	\\
6.5e-10	0	\\
7.5e-10	0	\\
8.6e-10	0	\\
9.6e-10	0	\\
1.07e-09	0	\\
1.18e-09	0	\\
1.28e-09	0	\\
1.38e-09	0	\\
1.49e-09	0	\\
1.59e-09	0	\\
1.69e-09	0	\\
1.8e-09	0	\\
1.9e-09	0	\\
2.01e-09	0	\\
2.11e-09	0	\\
2.21e-09	0	\\
2.32e-09	0	\\
2.42e-09	0	\\
2.52e-09	0	\\
2.63e-09	0	\\
2.73e-09	0	\\
2.83e-09	0	\\
2.93e-09	0	\\
3.04e-09	0	\\
3.14e-09	0	\\
3.24e-09	0	\\
3.34e-09	0	\\
3.45e-09	0	\\
3.55e-09	0	\\
3.65e-09	0	\\
3.75e-09	0	\\
3.86e-09	0	\\
3.96e-09	0	\\
4.06e-09	0	\\
4.16e-09	0	\\
4.27e-09	0	\\
4.37e-09	0	\\
4.47e-09	0	\\
4.57e-09	0	\\
4.68e-09	0	\\
4.78e-09	0	\\
4.89e-09	0	\\
4.99e-09	0	\\
5e-09	0	\\
};
\addplot [color=mycolor2,solid,forget plot]
  table[row sep=crcr]{
0	0	\\
1.1e-10	0	\\
2.2e-10	0	\\
3.3e-10	0	\\
4.4e-10	0	\\
5.4e-10	0	\\
6.5e-10	0	\\
7.5e-10	0	\\
8.6e-10	0	\\
9.6e-10	0	\\
1.07e-09	0	\\
1.18e-09	0	\\
1.28e-09	0	\\
1.38e-09	0	\\
1.49e-09	0	\\
1.59e-09	0	\\
1.69e-09	0	\\
1.8e-09	0	\\
1.9e-09	0	\\
2.01e-09	0	\\
2.11e-09	0	\\
2.21e-09	0	\\
2.32e-09	0	\\
2.42e-09	0	\\
2.52e-09	0	\\
2.63e-09	0	\\
2.73e-09	0	\\
2.83e-09	0	\\
2.93e-09	0	\\
3.04e-09	0	\\
3.14e-09	0	\\
3.24e-09	0	\\
3.34e-09	0	\\
3.45e-09	0	\\
3.55e-09	0	\\
3.65e-09	0	\\
3.75e-09	0	\\
3.86e-09	0	\\
3.96e-09	0	\\
4.06e-09	0	\\
4.16e-09	0	\\
4.27e-09	0	\\
4.37e-09	0	\\
4.47e-09	0	\\
4.57e-09	0	\\
4.68e-09	0	\\
4.78e-09	0	\\
4.89e-09	0	\\
4.99e-09	0	\\
5e-09	0	\\
};
\addplot [color=mycolor3,solid,forget plot]
  table[row sep=crcr]{
0	0	\\
1.1e-10	0	\\
2.2e-10	0	\\
3.3e-10	0	\\
4.4e-10	0	\\
5.4e-10	0	\\
6.5e-10	0	\\
7.5e-10	0	\\
8.6e-10	0	\\
9.6e-10	0	\\
1.07e-09	0	\\
1.18e-09	0	\\
1.28e-09	0	\\
1.38e-09	0	\\
1.49e-09	0	\\
1.59e-09	0	\\
1.69e-09	0	\\
1.8e-09	0	\\
1.9e-09	0	\\
2.01e-09	0	\\
2.11e-09	0	\\
2.21e-09	0	\\
2.32e-09	0	\\
2.42e-09	0	\\
2.52e-09	0	\\
2.63e-09	0	\\
2.73e-09	0	\\
2.83e-09	0	\\
2.93e-09	0	\\
3.04e-09	0	\\
3.14e-09	0	\\
3.24e-09	0	\\
3.34e-09	0	\\
3.45e-09	0	\\
3.55e-09	0	\\
3.65e-09	0	\\
3.75e-09	0	\\
3.86e-09	0	\\
3.96e-09	0	\\
4.06e-09	0	\\
4.16e-09	0	\\
4.27e-09	0	\\
4.37e-09	0	\\
4.47e-09	0	\\
4.57e-09	0	\\
4.68e-09	0	\\
4.78e-09	0	\\
4.89e-09	0	\\
4.99e-09	0	\\
5e-09	0	\\
};
\addplot [color=darkgray,solid,forget plot]
  table[row sep=crcr]{
0	0	\\
1.1e-10	0	\\
2.2e-10	0	\\
3.3e-10	0	\\
4.4e-10	0	\\
5.4e-10	0	\\
6.5e-10	0	\\
7.5e-10	0	\\
8.6e-10	0	\\
9.6e-10	0	\\
1.07e-09	0	\\
1.18e-09	0	\\
1.28e-09	0	\\
1.38e-09	0	\\
1.49e-09	0	\\
1.59e-09	0	\\
1.69e-09	0	\\
1.8e-09	0	\\
1.9e-09	0	\\
2.01e-09	0	\\
2.11e-09	0	\\
2.21e-09	0	\\
2.32e-09	0	\\
2.42e-09	0	\\
2.52e-09	0	\\
2.63e-09	0	\\
2.73e-09	0	\\
2.83e-09	0	\\
2.93e-09	0	\\
3.04e-09	0	\\
3.14e-09	0	\\
3.24e-09	0	\\
3.34e-09	0	\\
3.45e-09	0	\\
3.55e-09	0	\\
3.65e-09	0	\\
3.75e-09	0	\\
3.86e-09	0	\\
3.96e-09	0	\\
4.06e-09	0	\\
4.16e-09	0	\\
4.27e-09	0	\\
4.37e-09	0	\\
4.47e-09	0	\\
4.57e-09	0	\\
4.68e-09	0	\\
4.78e-09	0	\\
4.89e-09	0	\\
4.99e-09	0	\\
5e-09	0	\\
};
\addplot [color=blue,solid,forget plot]
  table[row sep=crcr]{
0	0	\\
1.1e-10	0	\\
2.2e-10	0	\\
3.3e-10	0	\\
4.4e-10	0	\\
5.4e-10	0	\\
6.5e-10	0	\\
7.5e-10	0	\\
8.6e-10	0	\\
9.6e-10	0	\\
1.07e-09	0	\\
1.18e-09	0	\\
1.28e-09	0	\\
1.38e-09	0	\\
1.49e-09	0	\\
1.59e-09	0	\\
1.69e-09	0	\\
1.8e-09	0	\\
1.9e-09	0	\\
2.01e-09	0	\\
2.11e-09	0	\\
2.21e-09	0	\\
2.32e-09	0	\\
2.42e-09	0	\\
2.52e-09	0	\\
2.63e-09	0	\\
2.73e-09	0	\\
2.83e-09	0	\\
2.93e-09	0	\\
3.04e-09	0	\\
3.14e-09	0	\\
3.24e-09	0	\\
3.34e-09	0	\\
3.45e-09	0	\\
3.55e-09	0	\\
3.65e-09	0	\\
3.75e-09	0	\\
3.86e-09	0	\\
3.96e-09	0	\\
4.06e-09	0	\\
4.16e-09	0	\\
4.27e-09	0	\\
4.37e-09	0	\\
4.47e-09	0	\\
4.57e-09	0	\\
4.68e-09	0	\\
4.78e-09	0	\\
4.89e-09	0	\\
4.99e-09	0	\\
5e-09	0	\\
};
\addplot [color=black!50!green,solid,forget plot]
  table[row sep=crcr]{
0	0	\\
1.1e-10	0	\\
2.2e-10	0	\\
3.3e-10	0	\\
4.4e-10	0	\\
5.4e-10	0	\\
6.5e-10	0	\\
7.5e-10	0	\\
8.6e-10	0	\\
9.6e-10	0	\\
1.07e-09	0	\\
1.18e-09	0	\\
1.28e-09	0	\\
1.38e-09	0	\\
1.49e-09	0	\\
1.59e-09	0	\\
1.69e-09	0	\\
1.8e-09	0	\\
1.9e-09	0	\\
2.01e-09	0	\\
2.11e-09	0	\\
2.21e-09	0	\\
2.32e-09	0	\\
2.42e-09	0	\\
2.52e-09	0	\\
2.63e-09	0	\\
2.73e-09	0	\\
2.83e-09	0	\\
2.93e-09	0	\\
3.04e-09	0	\\
3.14e-09	0	\\
3.24e-09	0	\\
3.34e-09	0	\\
3.45e-09	0	\\
3.55e-09	0	\\
3.65e-09	0	\\
3.75e-09	0	\\
3.86e-09	0	\\
3.96e-09	0	\\
4.06e-09	0	\\
4.16e-09	0	\\
4.27e-09	0	\\
4.37e-09	0	\\
4.47e-09	0	\\
4.57e-09	0	\\
4.68e-09	0	\\
4.78e-09	0	\\
4.89e-09	0	\\
4.99e-09	0	\\
5e-09	0	\\
};
\addplot [color=red,solid,forget plot]
  table[row sep=crcr]{
0	0	\\
1.1e-10	0	\\
2.2e-10	0	\\
3.3e-10	0	\\
4.4e-10	0	\\
5.4e-10	0	\\
6.5e-10	0	\\
7.5e-10	0	\\
8.6e-10	0	\\
9.6e-10	0	\\
1.07e-09	0	\\
1.18e-09	0	\\
1.28e-09	0	\\
1.38e-09	0	\\
1.49e-09	0	\\
1.59e-09	0	\\
1.69e-09	0	\\
1.8e-09	0	\\
1.9e-09	0	\\
2.01e-09	0	\\
2.11e-09	0	\\
2.21e-09	0	\\
2.32e-09	0	\\
2.42e-09	0	\\
2.52e-09	0	\\
2.63e-09	0	\\
2.73e-09	0	\\
2.83e-09	0	\\
2.93e-09	0	\\
3.04e-09	0	\\
3.14e-09	0	\\
3.24e-09	0	\\
3.34e-09	0	\\
3.45e-09	0	\\
3.55e-09	0	\\
3.65e-09	0	\\
3.75e-09	0	\\
3.86e-09	0	\\
3.96e-09	0	\\
4.06e-09	0	\\
4.16e-09	0	\\
4.27e-09	0	\\
4.37e-09	0	\\
4.47e-09	0	\\
4.57e-09	0	\\
4.68e-09	0	\\
4.78e-09	0	\\
4.89e-09	0	\\
4.99e-09	0	\\
5e-09	0	\\
};
\addplot [color=mycolor1,solid,forget plot]
  table[row sep=crcr]{
0	0	\\
1.1e-10	0	\\
2.2e-10	0	\\
3.3e-10	0	\\
4.4e-10	0	\\
5.4e-10	0	\\
6.5e-10	0	\\
7.5e-10	0	\\
8.6e-10	0	\\
9.6e-10	0	\\
1.07e-09	0	\\
1.18e-09	0	\\
1.28e-09	0	\\
1.38e-09	0	\\
1.49e-09	0	\\
1.59e-09	0	\\
1.69e-09	0	\\
1.8e-09	0	\\
1.9e-09	0	\\
2.01e-09	0	\\
2.11e-09	0	\\
2.21e-09	0	\\
2.32e-09	0	\\
2.42e-09	0	\\
2.52e-09	0	\\
2.63e-09	0	\\
2.73e-09	0	\\
2.83e-09	0	\\
2.93e-09	0	\\
3.04e-09	0	\\
3.14e-09	0	\\
3.24e-09	0	\\
3.34e-09	0	\\
3.45e-09	0	\\
3.55e-09	0	\\
3.65e-09	0	\\
3.75e-09	0	\\
3.86e-09	0	\\
3.96e-09	0	\\
4.06e-09	0	\\
4.16e-09	0	\\
4.27e-09	0	\\
4.37e-09	0	\\
4.47e-09	0	\\
4.57e-09	0	\\
4.68e-09	0	\\
4.78e-09	0	\\
4.89e-09	0	\\
4.99e-09	0	\\
5e-09	0	\\
};
\addplot [color=mycolor2,solid,forget plot]
  table[row sep=crcr]{
0	0	\\
1.1e-10	0	\\
2.2e-10	0	\\
3.3e-10	0	\\
4.4e-10	0	\\
5.4e-10	0	\\
6.5e-10	0	\\
7.5e-10	0	\\
8.6e-10	0	\\
9.6e-10	0	\\
1.07e-09	0	\\
1.18e-09	0	\\
1.28e-09	0	\\
1.38e-09	0	\\
1.49e-09	0	\\
1.59e-09	0	\\
1.69e-09	0	\\
1.8e-09	0	\\
1.9e-09	0	\\
2.01e-09	0	\\
2.11e-09	0	\\
2.21e-09	0	\\
2.32e-09	0	\\
2.42e-09	0	\\
2.52e-09	0	\\
2.63e-09	0	\\
2.73e-09	0	\\
2.83e-09	0	\\
2.93e-09	0	\\
3.04e-09	0	\\
3.14e-09	0	\\
3.24e-09	0	\\
3.34e-09	0	\\
3.45e-09	0	\\
3.55e-09	0	\\
3.65e-09	0	\\
3.75e-09	0	\\
3.86e-09	0	\\
3.96e-09	0	\\
4.06e-09	0	\\
4.16e-09	0	\\
4.27e-09	0	\\
4.37e-09	0	\\
4.47e-09	0	\\
4.57e-09	0	\\
4.68e-09	0	\\
4.78e-09	0	\\
4.89e-09	0	\\
4.99e-09	0	\\
5e-09	0	\\
};
\addplot [color=mycolor3,solid,forget plot]
  table[row sep=crcr]{
0	0	\\
1.1e-10	0	\\
2.2e-10	0	\\
3.3e-10	0	\\
4.4e-10	0	\\
5.4e-10	0	\\
6.5e-10	0	\\
7.5e-10	0	\\
8.6e-10	0	\\
9.6e-10	0	\\
1.07e-09	0	\\
1.18e-09	0	\\
1.28e-09	0	\\
1.38e-09	0	\\
1.49e-09	0	\\
1.59e-09	0	\\
1.69e-09	0	\\
1.8e-09	0	\\
1.9e-09	0	\\
2.01e-09	0	\\
2.11e-09	0	\\
2.21e-09	0	\\
2.32e-09	0	\\
2.42e-09	0	\\
2.52e-09	0	\\
2.63e-09	0	\\
2.73e-09	0	\\
2.83e-09	0	\\
2.93e-09	0	\\
3.04e-09	0	\\
3.14e-09	0	\\
3.24e-09	0	\\
3.34e-09	0	\\
3.45e-09	0	\\
3.55e-09	0	\\
3.65e-09	0	\\
3.75e-09	0	\\
3.86e-09	0	\\
3.96e-09	0	\\
4.06e-09	0	\\
4.16e-09	0	\\
4.27e-09	0	\\
4.37e-09	0	\\
4.47e-09	0	\\
4.57e-09	0	\\
4.68e-09	0	\\
4.78e-09	0	\\
4.89e-09	0	\\
4.99e-09	0	\\
5e-09	0	\\
};
\addplot [color=darkgray,solid,forget plot]
  table[row sep=crcr]{
0	0	\\
1.1e-10	0	\\
2.2e-10	0	\\
3.3e-10	0	\\
4.4e-10	0	\\
5.4e-10	0	\\
6.5e-10	0	\\
7.5e-10	0	\\
8.6e-10	0	\\
9.6e-10	0	\\
1.07e-09	0	\\
1.18e-09	0	\\
1.28e-09	0	\\
1.38e-09	0	\\
1.49e-09	0	\\
1.59e-09	0	\\
1.69e-09	0	\\
1.8e-09	0	\\
1.9e-09	0	\\
2.01e-09	0	\\
2.11e-09	0	\\
2.21e-09	0	\\
2.32e-09	0	\\
2.42e-09	0	\\
2.52e-09	0	\\
2.63e-09	0	\\
2.73e-09	0	\\
2.83e-09	0	\\
2.93e-09	0	\\
3.04e-09	0	\\
3.14e-09	0	\\
3.24e-09	0	\\
3.34e-09	0	\\
3.45e-09	0	\\
3.55e-09	0	\\
3.65e-09	0	\\
3.75e-09	0	\\
3.86e-09	0	\\
3.96e-09	0	\\
4.06e-09	0	\\
4.16e-09	0	\\
4.27e-09	0	\\
4.37e-09	0	\\
4.47e-09	0	\\
4.57e-09	0	\\
4.68e-09	0	\\
4.78e-09	0	\\
4.89e-09	0	\\
4.99e-09	0	\\
5e-09	0	\\
};
\addplot [color=blue,solid,forget plot]
  table[row sep=crcr]{
0	0	\\
1.1e-10	0	\\
2.2e-10	0	\\
3.3e-10	0	\\
4.4e-10	0	\\
5.4e-10	0	\\
6.5e-10	0	\\
7.5e-10	0	\\
8.6e-10	0	\\
9.6e-10	0	\\
1.07e-09	0	\\
1.18e-09	0	\\
1.28e-09	0	\\
1.38e-09	0	\\
1.49e-09	0	\\
1.59e-09	0	\\
1.69e-09	0	\\
1.8e-09	0	\\
1.9e-09	0	\\
2.01e-09	0	\\
2.11e-09	0	\\
2.21e-09	0	\\
2.32e-09	0	\\
2.42e-09	0	\\
2.52e-09	0	\\
2.63e-09	0	\\
2.73e-09	0	\\
2.83e-09	0	\\
2.93e-09	0	\\
3.04e-09	0	\\
3.14e-09	0	\\
3.24e-09	0	\\
3.34e-09	0	\\
3.45e-09	0	\\
3.55e-09	0	\\
3.65e-09	0	\\
3.75e-09	0	\\
3.86e-09	0	\\
3.96e-09	0	\\
4.06e-09	0	\\
4.16e-09	0	\\
4.27e-09	0	\\
4.37e-09	0	\\
4.47e-09	0	\\
4.57e-09	0	\\
4.68e-09	0	\\
4.78e-09	0	\\
4.89e-09	0	\\
4.99e-09	0	\\
5e-09	0	\\
};
\addplot [color=black!50!green,solid,forget plot]
  table[row sep=crcr]{
0	0	\\
1.1e-10	0	\\
2.2e-10	0	\\
3.3e-10	0	\\
4.4e-10	0	\\
5.4e-10	0	\\
6.5e-10	0	\\
7.5e-10	0	\\
8.6e-10	0	\\
9.6e-10	0	\\
1.07e-09	0	\\
1.18e-09	0	\\
1.28e-09	0	\\
1.38e-09	0	\\
1.49e-09	0	\\
1.59e-09	0	\\
1.69e-09	0	\\
1.8e-09	0	\\
1.9e-09	0	\\
2.01e-09	0	\\
2.11e-09	0	\\
2.21e-09	0	\\
2.32e-09	0	\\
2.42e-09	0	\\
2.52e-09	0	\\
2.63e-09	0	\\
2.73e-09	0	\\
2.83e-09	0	\\
2.93e-09	0	\\
3.04e-09	0	\\
3.14e-09	0	\\
3.24e-09	0	\\
3.34e-09	0	\\
3.45e-09	0	\\
3.55e-09	0	\\
3.65e-09	0	\\
3.75e-09	0	\\
3.86e-09	0	\\
3.96e-09	0	\\
4.06e-09	0	\\
4.16e-09	0	\\
4.27e-09	0	\\
4.37e-09	0	\\
4.47e-09	0	\\
4.57e-09	0	\\
4.68e-09	0	\\
4.78e-09	0	\\
4.89e-09	0	\\
4.99e-09	0	\\
5e-09	0	\\
};
\addplot [color=red,solid,forget plot]
  table[row sep=crcr]{
0	0	\\
1.1e-10	0	\\
2.2e-10	0	\\
3.3e-10	0	\\
4.4e-10	0	\\
5.4e-10	0	\\
6.5e-10	0	\\
7.5e-10	0	\\
8.6e-10	0	\\
9.6e-10	0	\\
1.07e-09	0	\\
1.18e-09	0	\\
1.28e-09	0	\\
1.38e-09	0	\\
1.49e-09	0	\\
1.59e-09	0	\\
1.69e-09	0	\\
1.8e-09	0	\\
1.9e-09	0	\\
2.01e-09	0	\\
2.11e-09	0	\\
2.21e-09	0	\\
2.32e-09	0	\\
2.42e-09	0	\\
2.52e-09	0	\\
2.63e-09	0	\\
2.73e-09	0	\\
2.83e-09	0	\\
2.93e-09	0	\\
3.04e-09	0	\\
3.14e-09	0	\\
3.24e-09	0	\\
3.34e-09	0	\\
3.45e-09	0	\\
3.55e-09	0	\\
3.65e-09	0	\\
3.75e-09	0	\\
3.86e-09	0	\\
3.96e-09	0	\\
4.06e-09	0	\\
4.16e-09	0	\\
4.27e-09	0	\\
4.37e-09	0	\\
4.47e-09	0	\\
4.57e-09	0	\\
4.68e-09	0	\\
4.78e-09	0	\\
4.89e-09	0	\\
4.99e-09	0	\\
5e-09	0	\\
};
\addplot [color=mycolor1,solid,forget plot]
  table[row sep=crcr]{
0	0	\\
1.1e-10	0	\\
2.2e-10	0	\\
3.3e-10	0	\\
4.4e-10	0	\\
5.4e-10	0	\\
6.5e-10	0	\\
7.5e-10	0	\\
8.6e-10	0	\\
9.6e-10	0	\\
1.07e-09	0	\\
1.18e-09	0	\\
1.28e-09	0	\\
1.38e-09	0	\\
1.49e-09	0	\\
1.59e-09	0	\\
1.69e-09	0	\\
1.8e-09	0	\\
1.9e-09	0	\\
2.01e-09	0	\\
2.11e-09	0	\\
2.21e-09	0	\\
2.32e-09	0	\\
2.42e-09	0	\\
2.52e-09	0	\\
2.63e-09	0	\\
2.73e-09	0	\\
2.83e-09	0	\\
2.93e-09	0	\\
3.04e-09	0	\\
3.14e-09	0	\\
3.24e-09	0	\\
3.34e-09	0	\\
3.45e-09	0	\\
3.55e-09	0	\\
3.65e-09	0	\\
3.75e-09	0	\\
3.86e-09	0	\\
3.96e-09	0	\\
4.06e-09	0	\\
4.16e-09	0	\\
4.27e-09	0	\\
4.37e-09	0	\\
4.47e-09	0	\\
4.57e-09	0	\\
4.68e-09	0	\\
4.78e-09	0	\\
4.89e-09	0	\\
4.99e-09	0	\\
5e-09	0	\\
};
\addplot [color=mycolor2,solid,forget plot]
  table[row sep=crcr]{
0	0	\\
1.1e-10	0	\\
2.2e-10	0	\\
3.3e-10	0	\\
4.4e-10	0	\\
5.4e-10	0	\\
6.5e-10	0	\\
7.5e-10	0	\\
8.6e-10	0	\\
9.6e-10	0	\\
1.07e-09	0	\\
1.18e-09	0	\\
1.28e-09	0	\\
1.38e-09	0	\\
1.49e-09	0	\\
1.59e-09	0	\\
1.69e-09	0	\\
1.8e-09	0	\\
1.9e-09	0	\\
2.01e-09	0	\\
2.11e-09	0	\\
2.21e-09	0	\\
2.32e-09	0	\\
2.42e-09	0	\\
2.52e-09	0	\\
2.63e-09	0	\\
2.73e-09	0	\\
2.83e-09	0	\\
2.93e-09	0	\\
3.04e-09	0	\\
3.14e-09	0	\\
3.24e-09	0	\\
3.34e-09	0	\\
3.45e-09	0	\\
3.55e-09	0	\\
3.65e-09	0	\\
3.75e-09	0	\\
3.86e-09	0	\\
3.96e-09	0	\\
4.06e-09	0	\\
4.16e-09	0	\\
4.27e-09	0	\\
4.37e-09	0	\\
4.47e-09	0	\\
4.57e-09	0	\\
4.68e-09	0	\\
4.78e-09	0	\\
4.89e-09	0	\\
4.99e-09	0	\\
5e-09	0	\\
};
\addplot [color=mycolor3,solid,forget plot]
  table[row sep=crcr]{
0	0	\\
1.1e-10	0	\\
2.2e-10	0	\\
3.3e-10	0	\\
4.4e-10	0	\\
5.4e-10	0	\\
6.5e-10	0	\\
7.5e-10	0	\\
8.6e-10	0	\\
9.6e-10	0	\\
1.07e-09	0	\\
1.18e-09	0	\\
1.28e-09	0	\\
1.38e-09	0	\\
1.49e-09	0	\\
1.59e-09	0	\\
1.69e-09	0	\\
1.8e-09	0	\\
1.9e-09	0	\\
2.01e-09	0	\\
2.11e-09	0	\\
2.21e-09	0	\\
2.32e-09	0	\\
2.42e-09	0	\\
2.52e-09	0	\\
2.63e-09	0	\\
2.73e-09	0	\\
2.83e-09	0	\\
2.93e-09	0	\\
3.04e-09	0	\\
3.14e-09	0	\\
3.24e-09	0	\\
3.34e-09	0	\\
3.45e-09	0	\\
3.55e-09	0	\\
3.65e-09	0	\\
3.75e-09	0	\\
3.86e-09	0	\\
3.96e-09	0	\\
4.06e-09	0	\\
4.16e-09	0	\\
4.27e-09	0	\\
4.37e-09	0	\\
4.47e-09	0	\\
4.57e-09	0	\\
4.68e-09	0	\\
4.78e-09	0	\\
4.89e-09	0	\\
4.99e-09	0	\\
5e-09	0	\\
};
\addplot [color=darkgray,solid,forget plot]
  table[row sep=crcr]{
0	0	\\
1.1e-10	0	\\
2.2e-10	0	\\
3.3e-10	0	\\
4.4e-10	0	\\
5.4e-10	0	\\
6.5e-10	0	\\
7.5e-10	0	\\
8.6e-10	0	\\
9.6e-10	0	\\
1.07e-09	0	\\
1.18e-09	0	\\
1.28e-09	0	\\
1.38e-09	0	\\
1.49e-09	0	\\
1.59e-09	0	\\
1.69e-09	0	\\
1.8e-09	0	\\
1.9e-09	0	\\
2.01e-09	0	\\
2.11e-09	0	\\
2.21e-09	0	\\
2.32e-09	0	\\
2.42e-09	0	\\
2.52e-09	0	\\
2.63e-09	0	\\
2.73e-09	0	\\
2.83e-09	0	\\
2.93e-09	0	\\
3.04e-09	0	\\
3.14e-09	0	\\
3.24e-09	0	\\
3.34e-09	0	\\
3.45e-09	0	\\
3.55e-09	0	\\
3.65e-09	0	\\
3.75e-09	0	\\
3.86e-09	0	\\
3.96e-09	0	\\
4.06e-09	0	\\
4.16e-09	0	\\
4.27e-09	0	\\
4.37e-09	0	\\
4.47e-09	0	\\
4.57e-09	0	\\
4.68e-09	0	\\
4.78e-09	0	\\
4.89e-09	0	\\
4.99e-09	0	\\
5e-09	0	\\
};
\addplot [color=blue,solid,forget plot]
  table[row sep=crcr]{
0	0	\\
1.1e-10	0	\\
2.2e-10	0	\\
3.3e-10	0	\\
4.4e-10	0	\\
5.4e-10	0	\\
6.5e-10	0	\\
7.5e-10	0	\\
8.6e-10	0	\\
9.6e-10	0	\\
1.07e-09	0	\\
1.18e-09	0	\\
1.28e-09	0	\\
1.38e-09	0	\\
1.49e-09	0	\\
1.59e-09	0	\\
1.69e-09	0	\\
1.8e-09	0	\\
1.9e-09	0	\\
2.01e-09	0	\\
2.11e-09	0	\\
2.21e-09	0	\\
2.32e-09	0	\\
2.42e-09	0	\\
2.52e-09	0	\\
2.63e-09	0	\\
2.73e-09	0	\\
2.83e-09	0	\\
2.93e-09	0	\\
3.04e-09	0	\\
3.14e-09	0	\\
3.24e-09	0	\\
3.34e-09	0	\\
3.45e-09	0	\\
3.55e-09	0	\\
3.65e-09	0	\\
3.75e-09	0	\\
3.86e-09	0	\\
3.96e-09	0	\\
4.06e-09	0	\\
4.16e-09	0	\\
4.27e-09	0	\\
4.37e-09	0	\\
4.47e-09	0	\\
4.57e-09	0	\\
4.68e-09	0	\\
4.78e-09	0	\\
4.89e-09	0	\\
4.99e-09	0	\\
5e-09	0	\\
};
\addplot [color=black!50!green,solid,forget plot]
  table[row sep=crcr]{
0	0	\\
1.1e-10	0	\\
2.2e-10	0	\\
3.3e-10	0	\\
4.4e-10	0	\\
5.4e-10	0	\\
6.5e-10	0	\\
7.5e-10	0	\\
8.6e-10	0	\\
9.6e-10	0	\\
1.07e-09	0	\\
1.18e-09	0	\\
1.28e-09	0	\\
1.38e-09	0	\\
1.49e-09	0	\\
1.59e-09	0	\\
1.69e-09	0	\\
1.8e-09	0	\\
1.9e-09	0	\\
2.01e-09	0	\\
2.11e-09	0	\\
2.21e-09	0	\\
2.32e-09	0	\\
2.42e-09	0	\\
2.52e-09	0	\\
2.63e-09	0	\\
2.73e-09	0	\\
2.83e-09	0	\\
2.93e-09	0	\\
3.04e-09	0	\\
3.14e-09	0	\\
3.24e-09	0	\\
3.34e-09	0	\\
3.45e-09	0	\\
3.55e-09	0	\\
3.65e-09	0	\\
3.75e-09	0	\\
3.86e-09	0	\\
3.96e-09	0	\\
4.06e-09	0	\\
4.16e-09	0	\\
4.27e-09	0	\\
4.37e-09	0	\\
4.47e-09	0	\\
4.57e-09	0	\\
4.68e-09	0	\\
4.78e-09	0	\\
4.89e-09	0	\\
4.99e-09	0	\\
5e-09	0	\\
};
\addplot [color=red,solid,forget plot]
  table[row sep=crcr]{
0	0	\\
1.1e-10	0	\\
2.2e-10	0	\\
3.3e-10	0	\\
4.4e-10	0	\\
5.4e-10	0	\\
6.5e-10	0	\\
7.5e-10	0	\\
8.6e-10	0	\\
9.6e-10	0	\\
1.07e-09	0	\\
1.18e-09	0	\\
1.28e-09	0	\\
1.38e-09	0	\\
1.49e-09	0	\\
1.59e-09	0	\\
1.69e-09	0	\\
1.8e-09	0	\\
1.9e-09	0	\\
2.01e-09	0	\\
2.11e-09	0	\\
2.21e-09	0	\\
2.32e-09	0	\\
2.42e-09	0	\\
2.52e-09	0	\\
2.63e-09	0	\\
2.73e-09	0	\\
2.83e-09	0	\\
2.93e-09	0	\\
3.04e-09	0	\\
3.14e-09	0	\\
3.24e-09	0	\\
3.34e-09	0	\\
3.45e-09	0	\\
3.55e-09	0	\\
3.65e-09	0	\\
3.75e-09	0	\\
3.86e-09	0	\\
3.96e-09	0	\\
4.06e-09	0	\\
4.16e-09	0	\\
4.27e-09	0	\\
4.37e-09	0	\\
4.47e-09	0	\\
4.57e-09	0	\\
4.68e-09	0	\\
4.78e-09	0	\\
4.89e-09	0	\\
4.99e-09	0	\\
5e-09	0	\\
};
\addplot [color=mycolor1,solid,forget plot]
  table[row sep=crcr]{
0	0	\\
1.1e-10	0	\\
2.2e-10	0	\\
3.3e-10	0	\\
4.4e-10	0	\\
5.4e-10	0	\\
6.5e-10	0	\\
7.5e-10	0	\\
8.6e-10	0	\\
9.6e-10	0	\\
1.07e-09	0	\\
1.18e-09	0	\\
1.28e-09	0	\\
1.38e-09	0	\\
1.49e-09	0	\\
1.59e-09	0	\\
1.69e-09	0	\\
1.8e-09	0	\\
1.9e-09	0	\\
2.01e-09	0	\\
2.11e-09	0	\\
2.21e-09	0	\\
2.32e-09	0	\\
2.42e-09	0	\\
2.52e-09	0	\\
2.63e-09	0	\\
2.73e-09	0	\\
2.83e-09	0	\\
2.93e-09	0	\\
3.04e-09	0	\\
3.14e-09	0	\\
3.24e-09	0	\\
3.34e-09	0	\\
3.45e-09	0	\\
3.55e-09	0	\\
3.65e-09	0	\\
3.75e-09	0	\\
3.86e-09	0	\\
3.96e-09	0	\\
4.06e-09	0	\\
4.16e-09	0	\\
4.27e-09	0	\\
4.37e-09	0	\\
4.47e-09	0	\\
4.57e-09	0	\\
4.68e-09	0	\\
4.78e-09	0	\\
4.89e-09	0	\\
4.99e-09	0	\\
5e-09	0	\\
};
\addplot [color=mycolor2,solid,forget plot]
  table[row sep=crcr]{
0	0	\\
1.1e-10	0	\\
2.2e-10	0	\\
3.3e-10	0	\\
4.4e-10	0	\\
5.4e-10	0	\\
6.5e-10	0	\\
7.5e-10	0	\\
8.6e-10	0	\\
9.6e-10	0	\\
1.07e-09	0	\\
1.18e-09	0	\\
1.28e-09	0	\\
1.38e-09	0	\\
1.49e-09	0	\\
1.59e-09	0	\\
1.69e-09	0	\\
1.8e-09	0	\\
1.9e-09	0	\\
2.01e-09	0	\\
2.11e-09	0	\\
2.21e-09	0	\\
2.32e-09	0	\\
2.42e-09	0	\\
2.52e-09	0	\\
2.63e-09	0	\\
2.73e-09	0	\\
2.83e-09	0	\\
2.93e-09	0	\\
3.04e-09	0	\\
3.14e-09	0	\\
3.24e-09	0	\\
3.34e-09	0	\\
3.45e-09	0	\\
3.55e-09	0	\\
3.65e-09	0	\\
3.75e-09	0	\\
3.86e-09	0	\\
3.96e-09	0	\\
4.06e-09	0	\\
4.16e-09	0	\\
4.27e-09	0	\\
4.37e-09	0	\\
4.47e-09	0	\\
4.57e-09	0	\\
4.68e-09	0	\\
4.78e-09	0	\\
4.89e-09	0	\\
4.99e-09	0	\\
5e-09	0	\\
};
\addplot [color=mycolor3,solid,forget plot]
  table[row sep=crcr]{
0	0	\\
1.1e-10	0	\\
2.2e-10	0	\\
3.3e-10	0	\\
4.4e-10	0	\\
5.4e-10	0	\\
6.5e-10	0	\\
7.5e-10	0	\\
8.6e-10	0	\\
9.6e-10	0	\\
1.07e-09	0	\\
1.18e-09	0	\\
1.28e-09	0	\\
1.38e-09	0	\\
1.49e-09	0	\\
1.59e-09	0	\\
1.69e-09	0	\\
1.8e-09	0	\\
1.9e-09	0	\\
2.01e-09	0	\\
2.11e-09	0	\\
2.21e-09	0	\\
2.32e-09	0	\\
2.42e-09	0	\\
2.52e-09	0	\\
2.63e-09	0	\\
2.73e-09	0	\\
2.83e-09	0	\\
2.93e-09	0	\\
3.04e-09	0	\\
3.14e-09	0	\\
3.24e-09	0	\\
3.34e-09	0	\\
3.45e-09	0	\\
3.55e-09	0	\\
3.65e-09	0	\\
3.75e-09	0	\\
3.86e-09	0	\\
3.96e-09	0	\\
4.06e-09	0	\\
4.16e-09	0	\\
4.27e-09	0	\\
4.37e-09	0	\\
4.47e-09	0	\\
4.57e-09	0	\\
4.68e-09	0	\\
4.78e-09	0	\\
4.89e-09	0	\\
4.99e-09	0	\\
5e-09	0	\\
};
\addplot [color=darkgray,solid,forget plot]
  table[row sep=crcr]{
0	0	\\
1.1e-10	0	\\
2.2e-10	0	\\
3.3e-10	0	\\
4.4e-10	0	\\
5.4e-10	0	\\
6.5e-10	0	\\
7.5e-10	0	\\
8.6e-10	0	\\
9.6e-10	0	\\
1.07e-09	0	\\
1.18e-09	0	\\
1.28e-09	0	\\
1.38e-09	0	\\
1.49e-09	0	\\
1.59e-09	0	\\
1.69e-09	0	\\
1.8e-09	0	\\
1.9e-09	0	\\
2.01e-09	0	\\
2.11e-09	0	\\
2.21e-09	0	\\
2.32e-09	0	\\
2.42e-09	0	\\
2.52e-09	0	\\
2.63e-09	0	\\
2.73e-09	0	\\
2.83e-09	0	\\
2.93e-09	0	\\
3.04e-09	0	\\
3.14e-09	0	\\
3.24e-09	0	\\
3.34e-09	0	\\
3.45e-09	0	\\
3.55e-09	0	\\
3.65e-09	0	\\
3.75e-09	0	\\
3.86e-09	0	\\
3.96e-09	0	\\
4.06e-09	0	\\
4.16e-09	0	\\
4.27e-09	0	\\
4.37e-09	0	\\
4.47e-09	0	\\
4.57e-09	0	\\
4.68e-09	0	\\
4.78e-09	0	\\
4.89e-09	0	\\
4.99e-09	0	\\
5e-09	0	\\
};
\addplot [color=blue,solid,forget plot]
  table[row sep=crcr]{
0	0	\\
1.1e-10	0	\\
2.2e-10	0	\\
3.3e-10	0	\\
4.4e-10	0	\\
5.4e-10	0	\\
6.5e-10	0	\\
7.5e-10	0	\\
8.6e-10	0	\\
9.6e-10	0	\\
1.07e-09	0	\\
1.18e-09	0	\\
1.28e-09	0	\\
1.38e-09	0	\\
1.49e-09	0	\\
1.59e-09	0	\\
1.69e-09	0	\\
1.8e-09	0	\\
1.9e-09	0	\\
2.01e-09	0	\\
2.11e-09	0	\\
2.21e-09	0	\\
2.32e-09	0	\\
2.42e-09	0	\\
2.52e-09	0	\\
2.63e-09	0	\\
2.73e-09	0	\\
2.83e-09	0	\\
2.93e-09	0	\\
3.04e-09	0	\\
3.14e-09	0	\\
3.24e-09	0	\\
3.34e-09	0	\\
3.45e-09	0	\\
3.55e-09	0	\\
3.65e-09	0	\\
3.75e-09	0	\\
3.86e-09	0	\\
3.96e-09	0	\\
4.06e-09	0	\\
4.16e-09	0	\\
4.27e-09	0	\\
4.37e-09	0	\\
4.47e-09	0	\\
4.57e-09	0	\\
4.68e-09	0	\\
4.78e-09	0	\\
4.89e-09	0	\\
4.99e-09	0	\\
5e-09	0	\\
};
\addplot [color=black!50!green,solid,forget plot]
  table[row sep=crcr]{
0	0	\\
1.1e-10	0	\\
2.2e-10	0	\\
3.3e-10	0	\\
4.4e-10	0	\\
5.4e-10	0	\\
6.5e-10	0	\\
7.5e-10	0	\\
8.6e-10	0	\\
9.6e-10	0	\\
1.07e-09	0	\\
1.18e-09	0	\\
1.28e-09	0	\\
1.38e-09	0	\\
1.49e-09	0	\\
1.59e-09	0	\\
1.69e-09	0	\\
1.8e-09	0	\\
1.9e-09	0	\\
2.01e-09	0	\\
2.11e-09	0	\\
2.21e-09	0	\\
2.32e-09	0	\\
2.42e-09	0	\\
2.52e-09	0	\\
2.63e-09	0	\\
2.73e-09	0	\\
2.83e-09	0	\\
2.93e-09	0	\\
3.04e-09	0	\\
3.14e-09	0	\\
3.24e-09	0	\\
3.34e-09	0	\\
3.45e-09	0	\\
3.55e-09	0	\\
3.65e-09	0	\\
3.75e-09	0	\\
3.86e-09	0	\\
3.96e-09	0	\\
4.06e-09	0	\\
4.16e-09	0	\\
4.27e-09	0	\\
4.37e-09	0	\\
4.47e-09	0	\\
4.57e-09	0	\\
4.68e-09	0	\\
4.78e-09	0	\\
4.89e-09	0	\\
4.99e-09	0	\\
5e-09	0	\\
};
\addplot [color=red,solid,forget plot]
  table[row sep=crcr]{
0	0	\\
1.1e-10	0	\\
2.2e-10	0	\\
3.3e-10	0	\\
4.4e-10	0	\\
5.4e-10	0	\\
6.5e-10	0	\\
7.5e-10	0	\\
8.6e-10	0	\\
9.6e-10	0	\\
1.07e-09	0	\\
1.18e-09	0	\\
1.28e-09	0	\\
1.38e-09	0	\\
1.49e-09	0	\\
1.59e-09	0	\\
1.69e-09	0	\\
1.8e-09	0	\\
1.9e-09	0	\\
2.01e-09	0	\\
2.11e-09	0	\\
2.21e-09	0	\\
2.32e-09	0	\\
2.42e-09	0	\\
2.52e-09	0	\\
2.63e-09	0	\\
2.73e-09	0	\\
2.83e-09	0	\\
2.93e-09	0	\\
3.04e-09	0	\\
3.14e-09	0	\\
3.24e-09	0	\\
3.34e-09	0	\\
3.45e-09	0	\\
3.55e-09	0	\\
3.65e-09	0	\\
3.75e-09	0	\\
3.86e-09	0	\\
3.96e-09	0	\\
4.06e-09	0	\\
4.16e-09	0	\\
4.27e-09	0	\\
4.37e-09	0	\\
4.47e-09	0	\\
4.57e-09	0	\\
4.68e-09	0	\\
4.78e-09	0	\\
4.89e-09	0	\\
4.99e-09	0	\\
5e-09	0	\\
};
\addplot [color=mycolor1,solid,forget plot]
  table[row sep=crcr]{
0	0	\\
1.1e-10	0	\\
2.2e-10	0	\\
3.3e-10	0	\\
4.4e-10	0	\\
5.4e-10	0	\\
6.5e-10	0	\\
7.5e-10	0	\\
8.6e-10	0	\\
9.6e-10	0	\\
1.07e-09	0	\\
1.18e-09	0	\\
1.28e-09	0	\\
1.38e-09	0	\\
1.49e-09	0	\\
1.59e-09	0	\\
1.69e-09	0	\\
1.8e-09	0	\\
1.9e-09	0	\\
2.01e-09	0	\\
2.11e-09	0	\\
2.21e-09	0	\\
2.32e-09	0	\\
2.42e-09	0	\\
2.52e-09	0	\\
2.63e-09	0	\\
2.73e-09	0	\\
2.83e-09	0	\\
2.93e-09	0	\\
3.04e-09	0	\\
3.14e-09	0	\\
3.24e-09	0	\\
3.34e-09	0	\\
3.45e-09	0	\\
3.55e-09	0	\\
3.65e-09	0	\\
3.75e-09	0	\\
3.86e-09	0	\\
3.96e-09	0	\\
4.06e-09	0	\\
4.16e-09	0	\\
4.27e-09	0	\\
4.37e-09	0	\\
4.47e-09	0	\\
4.57e-09	0	\\
4.68e-09	0	\\
4.78e-09	0	\\
4.89e-09	0	\\
4.99e-09	0	\\
5e-09	0	\\
};
\addplot [color=mycolor2,solid,forget plot]
  table[row sep=crcr]{
0	0	\\
1.1e-10	0	\\
2.2e-10	0	\\
3.3e-10	0	\\
4.4e-10	0	\\
5.4e-10	0	\\
6.5e-10	0	\\
7.5e-10	0	\\
8.6e-10	0	\\
9.6e-10	0	\\
1.07e-09	0	\\
1.18e-09	0	\\
1.28e-09	0	\\
1.38e-09	0	\\
1.49e-09	0	\\
1.59e-09	0	\\
1.69e-09	0	\\
1.8e-09	0	\\
1.9e-09	0	\\
2.01e-09	0	\\
2.11e-09	0	\\
2.21e-09	0	\\
2.32e-09	0	\\
2.42e-09	0	\\
2.52e-09	0	\\
2.63e-09	0	\\
2.73e-09	0	\\
2.83e-09	0	\\
2.93e-09	0	\\
3.04e-09	0	\\
3.14e-09	0	\\
3.24e-09	0	\\
3.34e-09	0	\\
3.45e-09	0	\\
3.55e-09	0	\\
3.65e-09	0	\\
3.75e-09	0	\\
3.86e-09	0	\\
3.96e-09	0	\\
4.06e-09	0	\\
4.16e-09	0	\\
4.27e-09	0	\\
4.37e-09	0	\\
4.47e-09	0	\\
4.57e-09	0	\\
4.68e-09	0	\\
4.78e-09	0	\\
4.89e-09	0	\\
4.99e-09	0	\\
5e-09	0	\\
};
\addplot [color=mycolor3,solid,forget plot]
  table[row sep=crcr]{
0	0	\\
1.1e-10	0	\\
2.2e-10	0	\\
3.3e-10	0	\\
4.4e-10	0	\\
5.4e-10	0	\\
6.5e-10	0	\\
7.5e-10	0	\\
8.6e-10	0	\\
9.6e-10	0	\\
1.07e-09	0	\\
1.18e-09	0	\\
1.28e-09	0	\\
1.38e-09	0	\\
1.49e-09	0	\\
1.59e-09	0	\\
1.69e-09	0	\\
1.8e-09	0	\\
1.9e-09	0	\\
2.01e-09	0	\\
2.11e-09	0	\\
2.21e-09	0	\\
2.32e-09	0	\\
2.42e-09	0	\\
2.52e-09	0	\\
2.63e-09	0	\\
2.73e-09	0	\\
2.83e-09	0	\\
2.93e-09	0	\\
3.04e-09	0	\\
3.14e-09	0	\\
3.24e-09	0	\\
3.34e-09	0	\\
3.45e-09	0	\\
3.55e-09	0	\\
3.65e-09	0	\\
3.75e-09	0	\\
3.86e-09	0	\\
3.96e-09	0	\\
4.06e-09	0	\\
4.16e-09	0	\\
4.27e-09	0	\\
4.37e-09	0	\\
4.47e-09	0	\\
4.57e-09	0	\\
4.68e-09	0	\\
4.78e-09	0	\\
4.89e-09	0	\\
4.99e-09	0	\\
5e-09	0	\\
};
\addplot [color=darkgray,solid,forget plot]
  table[row sep=crcr]{
0	0	\\
1.1e-10	0	\\
2.2e-10	0	\\
3.3e-10	0	\\
4.4e-10	0	\\
5.4e-10	0	\\
6.5e-10	0	\\
7.5e-10	0	\\
8.6e-10	0	\\
9.6e-10	0	\\
1.07e-09	0	\\
1.18e-09	0	\\
1.28e-09	0	\\
1.38e-09	0	\\
1.49e-09	0	\\
1.59e-09	0	\\
1.69e-09	0	\\
1.8e-09	0	\\
1.9e-09	0	\\
2.01e-09	0	\\
2.11e-09	0	\\
2.21e-09	0	\\
2.32e-09	0	\\
2.42e-09	0	\\
2.52e-09	0	\\
2.63e-09	0	\\
2.73e-09	0	\\
2.83e-09	0	\\
2.93e-09	0	\\
3.04e-09	0	\\
3.14e-09	0	\\
3.24e-09	0	\\
3.34e-09	0	\\
3.45e-09	0	\\
3.55e-09	0	\\
3.65e-09	0	\\
3.75e-09	0	\\
3.86e-09	0	\\
3.96e-09	0	\\
4.06e-09	0	\\
4.16e-09	0	\\
4.27e-09	0	\\
4.37e-09	0	\\
4.47e-09	0	\\
4.57e-09	0	\\
4.68e-09	0	\\
4.78e-09	0	\\
4.89e-09	0	\\
4.99e-09	0	\\
5e-09	0	\\
};
\addplot [color=blue,solid,forget plot]
  table[row sep=crcr]{
0	0	\\
1.1e-10	0	\\
2.2e-10	0	\\
3.3e-10	0	\\
4.4e-10	0	\\
5.4e-10	0	\\
6.5e-10	0	\\
7.5e-10	0	\\
8.6e-10	0	\\
9.6e-10	0	\\
1.07e-09	0	\\
1.18e-09	0	\\
1.28e-09	0	\\
1.38e-09	0	\\
1.49e-09	0	\\
1.59e-09	0	\\
1.69e-09	0	\\
1.8e-09	0	\\
1.9e-09	0	\\
2.01e-09	0	\\
2.11e-09	0	\\
2.21e-09	0	\\
2.32e-09	0	\\
2.42e-09	0	\\
2.52e-09	0	\\
2.63e-09	0	\\
2.73e-09	0	\\
2.83e-09	0	\\
2.93e-09	0	\\
3.04e-09	0	\\
3.14e-09	0	\\
3.24e-09	0	\\
3.34e-09	0	\\
3.45e-09	0	\\
3.55e-09	0	\\
3.65e-09	0	\\
3.75e-09	0	\\
3.86e-09	0	\\
3.96e-09	0	\\
4.06e-09	0	\\
4.16e-09	0	\\
4.27e-09	0	\\
4.37e-09	0	\\
4.47e-09	0	\\
4.57e-09	0	\\
4.68e-09	0	\\
4.78e-09	0	\\
4.89e-09	0	\\
4.99e-09	0	\\
5e-09	0	\\
};
\addplot [color=black!50!green,solid,forget plot]
  table[row sep=crcr]{
0	0	\\
1.1e-10	0	\\
2.2e-10	0	\\
3.3e-10	0	\\
4.4e-10	0	\\
5.4e-10	0	\\
6.5e-10	0	\\
7.5e-10	0	\\
8.6e-10	0	\\
9.6e-10	0	\\
1.07e-09	0	\\
1.18e-09	0	\\
1.28e-09	0	\\
1.38e-09	0	\\
1.49e-09	0	\\
1.59e-09	0	\\
1.69e-09	0	\\
1.8e-09	0	\\
1.9e-09	0	\\
2.01e-09	0	\\
2.11e-09	0	\\
2.21e-09	0	\\
2.32e-09	0	\\
2.42e-09	0	\\
2.52e-09	0	\\
2.63e-09	0	\\
2.73e-09	0	\\
2.83e-09	0	\\
2.93e-09	0	\\
3.04e-09	0	\\
3.14e-09	0	\\
3.24e-09	0	\\
3.34e-09	0	\\
3.45e-09	0	\\
3.55e-09	0	\\
3.65e-09	0	\\
3.75e-09	0	\\
3.86e-09	0	\\
3.96e-09	0	\\
4.06e-09	0	\\
4.16e-09	0	\\
4.27e-09	0	\\
4.37e-09	0	\\
4.47e-09	0	\\
4.57e-09	0	\\
4.68e-09	0	\\
4.78e-09	0	\\
4.89e-09	0	\\
4.99e-09	0	\\
5e-09	0	\\
};
\addplot [color=red,solid,forget plot]
  table[row sep=crcr]{
0	0	\\
1.1e-10	0	\\
2.2e-10	0	\\
3.3e-10	0	\\
4.4e-10	0	\\
5.4e-10	0	\\
6.5e-10	0	\\
7.5e-10	0	\\
8.6e-10	0	\\
9.6e-10	0	\\
1.07e-09	0	\\
1.18e-09	0	\\
1.28e-09	0	\\
1.38e-09	0	\\
1.49e-09	0	\\
1.59e-09	0	\\
1.69e-09	0	\\
1.8e-09	0	\\
1.9e-09	0	\\
2.01e-09	0	\\
2.11e-09	0	\\
2.21e-09	0	\\
2.32e-09	0	\\
2.42e-09	0	\\
2.52e-09	0	\\
2.63e-09	0	\\
2.73e-09	0	\\
2.83e-09	0	\\
2.93e-09	0	\\
3.04e-09	0	\\
3.14e-09	0	\\
3.24e-09	0	\\
3.34e-09	0	\\
3.45e-09	0	\\
3.55e-09	0	\\
3.65e-09	0	\\
3.75e-09	0	\\
3.86e-09	0	\\
3.96e-09	0	\\
4.06e-09	0	\\
4.16e-09	0	\\
4.27e-09	0	\\
4.37e-09	0	\\
4.47e-09	0	\\
4.57e-09	0	\\
4.68e-09	0	\\
4.78e-09	0	\\
4.89e-09	0	\\
4.99e-09	0	\\
5e-09	0	\\
};
\addplot [color=mycolor1,solid,forget plot]
  table[row sep=crcr]{
0	0	\\
1.1e-10	0	\\
2.2e-10	0	\\
3.3e-10	0	\\
4.4e-10	0	\\
5.4e-10	0	\\
6.5e-10	0	\\
7.5e-10	0	\\
8.6e-10	0	\\
9.6e-10	0	\\
1.07e-09	0	\\
1.18e-09	0	\\
1.28e-09	0	\\
1.38e-09	0	\\
1.49e-09	0	\\
1.59e-09	0	\\
1.69e-09	0	\\
1.8e-09	0	\\
1.9e-09	0	\\
2.01e-09	0	\\
2.11e-09	0	\\
2.21e-09	0	\\
2.32e-09	0	\\
2.42e-09	0	\\
2.52e-09	0	\\
2.63e-09	0	\\
2.73e-09	0	\\
2.83e-09	0	\\
2.93e-09	0	\\
3.04e-09	0	\\
3.14e-09	0	\\
3.24e-09	0	\\
3.34e-09	0	\\
3.45e-09	0	\\
3.55e-09	0	\\
3.65e-09	0	\\
3.75e-09	0	\\
3.86e-09	0	\\
3.96e-09	0	\\
4.06e-09	0	\\
4.16e-09	0	\\
4.27e-09	0	\\
4.37e-09	0	\\
4.47e-09	0	\\
4.57e-09	0	\\
4.68e-09	0	\\
4.78e-09	0	\\
4.89e-09	0	\\
4.99e-09	0	\\
5e-09	0	\\
};
\addplot [color=mycolor2,solid,forget plot]
  table[row sep=crcr]{
0	0	\\
1.1e-10	0	\\
2.2e-10	0	\\
3.3e-10	0	\\
4.4e-10	0	\\
5.4e-10	0	\\
6.5e-10	0	\\
7.5e-10	0	\\
8.6e-10	0	\\
9.6e-10	0	\\
1.07e-09	0	\\
1.18e-09	0	\\
1.28e-09	0	\\
1.38e-09	0	\\
1.49e-09	0	\\
1.59e-09	0	\\
1.69e-09	0	\\
1.8e-09	0	\\
1.9e-09	0	\\
2.01e-09	0	\\
2.11e-09	0	\\
2.21e-09	0	\\
2.32e-09	0	\\
2.42e-09	0	\\
2.52e-09	0	\\
2.63e-09	0	\\
2.73e-09	0	\\
2.83e-09	0	\\
2.93e-09	0	\\
3.04e-09	0	\\
3.14e-09	0	\\
3.24e-09	0	\\
3.34e-09	0	\\
3.45e-09	0	\\
3.55e-09	0	\\
3.65e-09	0	\\
3.75e-09	0	\\
3.86e-09	0	\\
3.96e-09	0	\\
4.06e-09	0	\\
4.16e-09	0	\\
4.27e-09	0	\\
4.37e-09	0	\\
4.47e-09	0	\\
4.57e-09	0	\\
4.68e-09	0	\\
4.78e-09	0	\\
4.89e-09	0	\\
4.99e-09	0	\\
5e-09	0	\\
};
\addplot [color=mycolor3,solid,forget plot]
  table[row sep=crcr]{
0	0	\\
1.1e-10	0	\\
2.2e-10	0	\\
3.3e-10	0	\\
4.4e-10	0	\\
5.4e-10	0	\\
6.5e-10	0	\\
7.5e-10	0	\\
8.6e-10	0	\\
9.6e-10	0	\\
1.07e-09	0	\\
1.18e-09	0	\\
1.28e-09	0	\\
1.38e-09	0	\\
1.49e-09	0	\\
1.59e-09	0	\\
1.69e-09	0	\\
1.8e-09	0	\\
1.9e-09	0	\\
2.01e-09	0	\\
2.11e-09	0	\\
2.21e-09	0	\\
2.32e-09	0	\\
2.42e-09	0	\\
2.52e-09	0	\\
2.63e-09	0	\\
2.73e-09	0	\\
2.83e-09	0	\\
2.93e-09	0	\\
3.04e-09	0	\\
3.14e-09	0	\\
3.24e-09	0	\\
3.34e-09	0	\\
3.45e-09	0	\\
3.55e-09	0	\\
3.65e-09	0	\\
3.75e-09	0	\\
3.86e-09	0	\\
3.96e-09	0	\\
4.06e-09	0	\\
4.16e-09	0	\\
4.27e-09	0	\\
4.37e-09	0	\\
4.47e-09	0	\\
4.57e-09	0	\\
4.68e-09	0	\\
4.78e-09	0	\\
4.89e-09	0	\\
4.99e-09	0	\\
5e-09	0	\\
};
\addplot [color=darkgray,solid,forget plot]
  table[row sep=crcr]{
0	0	\\
1.1e-10	0	\\
2.2e-10	0	\\
3.3e-10	0	\\
4.4e-10	0	\\
5.4e-10	0	\\
6.5e-10	0	\\
7.5e-10	0	\\
8.6e-10	0	\\
9.6e-10	0	\\
1.07e-09	0	\\
1.18e-09	0	\\
1.28e-09	0	\\
1.38e-09	0	\\
1.49e-09	0	\\
1.59e-09	0	\\
1.69e-09	0	\\
1.8e-09	0	\\
1.9e-09	0	\\
2.01e-09	0	\\
2.11e-09	0	\\
2.21e-09	0	\\
2.32e-09	0	\\
2.42e-09	0	\\
2.52e-09	0	\\
2.63e-09	0	\\
2.73e-09	0	\\
2.83e-09	0	\\
2.93e-09	0	\\
3.04e-09	0	\\
3.14e-09	0	\\
3.24e-09	0	\\
3.34e-09	0	\\
3.45e-09	0	\\
3.55e-09	0	\\
3.65e-09	0	\\
3.75e-09	0	\\
3.86e-09	0	\\
3.96e-09	0	\\
4.06e-09	0	\\
4.16e-09	0	\\
4.27e-09	0	\\
4.37e-09	0	\\
4.47e-09	0	\\
4.57e-09	0	\\
4.68e-09	0	\\
4.78e-09	0	\\
4.89e-09	0	\\
4.99e-09	0	\\
5e-09	0	\\
};
\addplot [color=blue,solid,forget plot]
  table[row sep=crcr]{
0	0	\\
1.1e-10	0	\\
2.2e-10	0	\\
3.3e-10	0	\\
4.4e-10	0	\\
5.4e-10	0	\\
6.5e-10	0	\\
7.5e-10	0	\\
8.6e-10	0	\\
9.6e-10	0	\\
1.07e-09	0	\\
1.18e-09	0	\\
1.28e-09	0	\\
1.38e-09	0	\\
1.49e-09	0	\\
1.59e-09	0	\\
1.69e-09	0	\\
1.8e-09	0	\\
1.9e-09	0	\\
2.01e-09	0	\\
2.11e-09	0	\\
2.21e-09	0	\\
2.32e-09	0	\\
2.42e-09	0	\\
2.52e-09	0	\\
2.63e-09	0	\\
2.73e-09	0	\\
2.83e-09	0	\\
2.93e-09	0	\\
3.04e-09	0	\\
3.14e-09	0	\\
3.24e-09	0	\\
3.34e-09	0	\\
3.45e-09	0	\\
3.55e-09	0	\\
3.65e-09	0	\\
3.75e-09	0	\\
3.86e-09	0	\\
3.96e-09	0	\\
4.06e-09	0	\\
4.16e-09	0	\\
4.27e-09	0	\\
4.37e-09	0	\\
4.47e-09	0	\\
4.57e-09	0	\\
4.68e-09	0	\\
4.78e-09	0	\\
4.89e-09	0	\\
4.99e-09	0	\\
5e-09	0	\\
};
\addplot [color=black!50!green,solid,forget plot]
  table[row sep=crcr]{
0	0	\\
1.1e-10	0	\\
2.2e-10	0	\\
3.3e-10	0	\\
4.4e-10	0	\\
5.4e-10	0	\\
6.5e-10	0	\\
7.5e-10	0	\\
8.6e-10	0	\\
9.6e-10	0	\\
1.07e-09	0	\\
1.18e-09	0	\\
1.28e-09	0	\\
1.38e-09	0	\\
1.49e-09	0	\\
1.59e-09	0	\\
1.69e-09	0	\\
1.8e-09	0	\\
1.9e-09	0	\\
2.01e-09	0	\\
2.11e-09	0	\\
2.21e-09	0	\\
2.32e-09	0	\\
2.42e-09	0	\\
2.52e-09	0	\\
2.63e-09	0	\\
2.73e-09	0	\\
2.83e-09	0	\\
2.93e-09	0	\\
3.04e-09	0	\\
3.14e-09	0	\\
3.24e-09	0	\\
3.34e-09	0	\\
3.45e-09	0	\\
3.55e-09	0	\\
3.65e-09	0	\\
3.75e-09	0	\\
3.86e-09	0	\\
3.96e-09	0	\\
4.06e-09	0	\\
4.16e-09	0	\\
4.27e-09	0	\\
4.37e-09	0	\\
4.47e-09	0	\\
4.57e-09	0	\\
4.68e-09	0	\\
4.78e-09	0	\\
4.89e-09	0	\\
4.99e-09	0	\\
5e-09	0	\\
};
\addplot [color=red,solid,forget plot]
  table[row sep=crcr]{
0	0	\\
1.1e-10	0	\\
2.2e-10	0	\\
3.3e-10	0	\\
4.4e-10	0	\\
5.4e-10	0	\\
6.5e-10	0	\\
7.5e-10	0	\\
8.6e-10	0	\\
9.6e-10	0	\\
1.07e-09	0	\\
1.18e-09	0	\\
1.28e-09	0	\\
1.38e-09	0	\\
1.49e-09	0	\\
1.59e-09	0	\\
1.69e-09	0	\\
1.8e-09	0	\\
1.9e-09	0	\\
2.01e-09	0	\\
2.11e-09	0	\\
2.21e-09	0	\\
2.32e-09	0	\\
2.42e-09	0	\\
2.52e-09	0	\\
2.63e-09	0	\\
2.73e-09	0	\\
2.83e-09	0	\\
2.93e-09	0	\\
3.04e-09	0	\\
3.14e-09	0	\\
3.24e-09	0	\\
3.34e-09	0	\\
3.45e-09	0	\\
3.55e-09	0	\\
3.65e-09	0	\\
3.75e-09	0	\\
3.86e-09	0	\\
3.96e-09	0	\\
4.06e-09	0	\\
4.16e-09	0	\\
4.27e-09	0	\\
4.37e-09	0	\\
4.47e-09	0	\\
4.57e-09	0	\\
4.68e-09	0	\\
4.78e-09	0	\\
4.89e-09	0	\\
4.99e-09	0	\\
5e-09	0	\\
};
\addplot [color=mycolor1,solid,forget plot]
  table[row sep=crcr]{
0	0	\\
1.1e-10	0	\\
2.2e-10	0	\\
3.3e-10	0	\\
4.4e-10	0	\\
5.4e-10	0	\\
6.5e-10	0	\\
7.5e-10	0	\\
8.6e-10	0	\\
9.6e-10	0	\\
1.07e-09	0	\\
1.18e-09	0	\\
1.28e-09	0	\\
1.38e-09	0	\\
1.49e-09	0	\\
1.59e-09	0	\\
1.69e-09	0	\\
1.8e-09	0	\\
1.9e-09	0	\\
2.01e-09	0	\\
2.11e-09	0	\\
2.21e-09	0	\\
2.32e-09	0	\\
2.42e-09	0	\\
2.52e-09	0	\\
2.63e-09	0	\\
2.73e-09	0	\\
2.83e-09	0	\\
2.93e-09	0	\\
3.04e-09	0	\\
3.14e-09	0	\\
3.24e-09	0	\\
3.34e-09	0	\\
3.45e-09	0	\\
3.55e-09	0	\\
3.65e-09	0	\\
3.75e-09	0	\\
3.86e-09	0	\\
3.96e-09	0	\\
4.06e-09	0	\\
4.16e-09	0	\\
4.27e-09	0	\\
4.37e-09	0	\\
4.47e-09	0	\\
4.57e-09	0	\\
4.68e-09	0	\\
4.78e-09	0	\\
4.89e-09	0	\\
4.99e-09	0	\\
5e-09	0	\\
};
\addplot [color=mycolor2,solid,forget plot]
  table[row sep=crcr]{
0	0	\\
1.1e-10	0	\\
2.2e-10	0	\\
3.3e-10	0	\\
4.4e-10	0	\\
5.4e-10	0	\\
6.5e-10	0	\\
7.5e-10	0	\\
8.6e-10	0	\\
9.6e-10	0	\\
1.07e-09	0	\\
1.18e-09	0	\\
1.28e-09	0	\\
1.38e-09	0	\\
1.49e-09	0	\\
1.59e-09	0	\\
1.69e-09	0	\\
1.8e-09	0	\\
1.9e-09	0	\\
2.01e-09	0	\\
2.11e-09	0	\\
2.21e-09	0	\\
2.32e-09	0	\\
2.42e-09	0	\\
2.52e-09	0	\\
2.63e-09	0	\\
2.73e-09	0	\\
2.83e-09	0	\\
2.93e-09	0	\\
3.04e-09	0	\\
3.14e-09	0	\\
3.24e-09	0	\\
3.34e-09	0	\\
3.45e-09	0	\\
3.55e-09	0	\\
3.65e-09	0	\\
3.75e-09	0	\\
3.86e-09	0	\\
3.96e-09	0	\\
4.06e-09	0	\\
4.16e-09	0	\\
4.27e-09	0	\\
4.37e-09	0	\\
4.47e-09	0	\\
4.57e-09	0	\\
4.68e-09	0	\\
4.78e-09	0	\\
4.89e-09	0	\\
4.99e-09	0	\\
5e-09	0	\\
};
\addplot [color=mycolor3,solid,forget plot]
  table[row sep=crcr]{
0	0	\\
1.1e-10	0	\\
2.2e-10	0	\\
3.3e-10	0	\\
4.4e-10	0	\\
5.4e-10	0	\\
6.5e-10	0	\\
7.5e-10	0	\\
8.6e-10	0	\\
9.6e-10	0	\\
1.07e-09	0	\\
1.18e-09	0	\\
1.28e-09	0	\\
1.38e-09	0	\\
1.49e-09	0	\\
1.59e-09	0	\\
1.69e-09	0	\\
1.8e-09	0	\\
1.9e-09	0	\\
2.01e-09	0	\\
2.11e-09	0	\\
2.21e-09	0	\\
2.32e-09	0	\\
2.42e-09	0	\\
2.52e-09	0	\\
2.63e-09	0	\\
2.73e-09	0	\\
2.83e-09	0	\\
2.93e-09	0	\\
3.04e-09	0	\\
3.14e-09	0	\\
3.24e-09	0	\\
3.34e-09	0	\\
3.45e-09	0	\\
3.55e-09	0	\\
3.65e-09	0	\\
3.75e-09	0	\\
3.86e-09	0	\\
3.96e-09	0	\\
4.06e-09	0	\\
4.16e-09	0	\\
4.27e-09	0	\\
4.37e-09	0	\\
4.47e-09	0	\\
4.57e-09	0	\\
4.68e-09	0	\\
4.78e-09	0	\\
4.89e-09	0	\\
4.99e-09	0	\\
5e-09	0	\\
};
\addplot [color=darkgray,solid,forget plot]
  table[row sep=crcr]{
0	0	\\
1.1e-10	0	\\
2.2e-10	0	\\
3.3e-10	0	\\
4.4e-10	0	\\
5.4e-10	0	\\
6.5e-10	0	\\
7.5e-10	0	\\
8.6e-10	0	\\
9.6e-10	0	\\
1.07e-09	0	\\
1.18e-09	0	\\
1.28e-09	0	\\
1.38e-09	0	\\
1.49e-09	0	\\
1.59e-09	0	\\
1.69e-09	0	\\
1.8e-09	0	\\
1.9e-09	0	\\
2.01e-09	0	\\
2.11e-09	0	\\
2.21e-09	0	\\
2.32e-09	0	\\
2.42e-09	0	\\
2.52e-09	0	\\
2.63e-09	0	\\
2.73e-09	0	\\
2.83e-09	0	\\
2.93e-09	0	\\
3.04e-09	0	\\
3.14e-09	0	\\
3.24e-09	0	\\
3.34e-09	0	\\
3.45e-09	0	\\
3.55e-09	0	\\
3.65e-09	0	\\
3.75e-09	0	\\
3.86e-09	0	\\
3.96e-09	0	\\
4.06e-09	0	\\
4.16e-09	0	\\
4.27e-09	0	\\
4.37e-09	0	\\
4.47e-09	0	\\
4.57e-09	0	\\
4.68e-09	0	\\
4.78e-09	0	\\
4.89e-09	0	\\
4.99e-09	0	\\
5e-09	0	\\
};
\addplot [color=blue,solid,forget plot]
  table[row sep=crcr]{
0	0	\\
1.1e-10	0	\\
2.2e-10	0	\\
3.3e-10	0	\\
4.4e-10	0	\\
5.4e-10	0	\\
6.5e-10	0	\\
7.5e-10	0	\\
8.6e-10	0	\\
9.6e-10	0	\\
1.07e-09	0	\\
1.18e-09	0	\\
1.28e-09	0	\\
1.38e-09	0	\\
1.49e-09	0	\\
1.59e-09	0	\\
1.69e-09	0	\\
1.8e-09	0	\\
1.9e-09	0	\\
2.01e-09	0	\\
2.11e-09	0	\\
2.21e-09	0	\\
2.32e-09	0	\\
2.42e-09	0	\\
2.52e-09	0	\\
2.63e-09	0	\\
2.73e-09	0	\\
2.83e-09	0	\\
2.93e-09	0	\\
3.04e-09	0	\\
3.14e-09	0	\\
3.24e-09	0	\\
3.34e-09	0	\\
3.45e-09	0	\\
3.55e-09	0	\\
3.65e-09	0	\\
3.75e-09	0	\\
3.86e-09	0	\\
3.96e-09	0	\\
4.06e-09	0	\\
4.16e-09	0	\\
4.27e-09	0	\\
4.37e-09	0	\\
4.47e-09	0	\\
4.57e-09	0	\\
4.68e-09	0	\\
4.78e-09	0	\\
4.89e-09	0	\\
4.99e-09	0	\\
5e-09	0	\\
};
\addplot [color=black!50!green,solid,forget plot]
  table[row sep=crcr]{
0	0	\\
1.1e-10	0	\\
2.2e-10	0	\\
3.3e-10	0	\\
4.4e-10	0	\\
5.4e-10	0	\\
6.5e-10	0	\\
7.5e-10	0	\\
8.6e-10	0	\\
9.6e-10	0	\\
1.07e-09	0	\\
1.18e-09	0	\\
1.28e-09	0	\\
1.38e-09	0	\\
1.49e-09	0	\\
1.59e-09	0	\\
1.69e-09	0	\\
1.8e-09	0	\\
1.9e-09	0	\\
2.01e-09	0	\\
2.11e-09	0	\\
2.21e-09	0	\\
2.32e-09	0	\\
2.42e-09	0	\\
2.52e-09	0	\\
2.63e-09	0	\\
2.73e-09	0	\\
2.83e-09	0	\\
2.93e-09	0	\\
3.04e-09	0	\\
3.14e-09	0	\\
3.24e-09	0	\\
3.34e-09	0	\\
3.45e-09	0	\\
3.55e-09	0	\\
3.65e-09	0	\\
3.75e-09	0	\\
3.86e-09	0	\\
3.96e-09	0	\\
4.06e-09	0	\\
4.16e-09	0	\\
4.27e-09	0	\\
4.37e-09	0	\\
4.47e-09	0	\\
4.57e-09	0	\\
4.68e-09	0	\\
4.78e-09	0	\\
4.89e-09	0	\\
4.99e-09	0	\\
5e-09	0	\\
};
\addplot [color=red,solid,forget plot]
  table[row sep=crcr]{
0	0	\\
1.1e-10	0	\\
2.2e-10	0	\\
3.3e-10	0	\\
4.4e-10	0	\\
5.4e-10	0	\\
6.5e-10	0	\\
7.5e-10	0	\\
8.6e-10	0	\\
9.6e-10	0	\\
1.07e-09	0	\\
1.18e-09	0	\\
1.28e-09	0	\\
1.38e-09	0	\\
1.49e-09	0	\\
1.59e-09	0	\\
1.69e-09	0	\\
1.8e-09	0	\\
1.9e-09	0	\\
2.01e-09	0	\\
2.11e-09	0	\\
2.21e-09	0	\\
2.32e-09	0	\\
2.42e-09	0	\\
2.52e-09	0	\\
2.63e-09	0	\\
2.73e-09	0	\\
2.83e-09	0	\\
2.93e-09	0	\\
3.04e-09	0	\\
3.14e-09	0	\\
3.24e-09	0	\\
3.34e-09	0	\\
3.45e-09	0	\\
3.55e-09	0	\\
3.65e-09	0	\\
3.75e-09	0	\\
3.86e-09	0	\\
3.96e-09	0	\\
4.06e-09	0	\\
4.16e-09	0	\\
4.27e-09	0	\\
4.37e-09	0	\\
4.47e-09	0	\\
4.57e-09	0	\\
4.68e-09	0	\\
4.78e-09	0	\\
4.89e-09	0	\\
4.99e-09	0	\\
5e-09	0	\\
};
\addplot [color=mycolor1,solid,forget plot]
  table[row sep=crcr]{
0	0	\\
1.1e-10	0	\\
2.2e-10	0	\\
3.3e-10	0	\\
4.4e-10	0	\\
5.4e-10	0	\\
6.5e-10	0	\\
7.5e-10	0	\\
8.6e-10	0	\\
9.6e-10	0	\\
1.07e-09	0	\\
1.18e-09	0	\\
1.28e-09	0	\\
1.38e-09	0	\\
1.49e-09	0	\\
1.59e-09	0	\\
1.69e-09	0	\\
1.8e-09	0	\\
1.9e-09	0	\\
2.01e-09	0	\\
2.11e-09	0	\\
2.21e-09	0	\\
2.32e-09	0	\\
2.42e-09	0	\\
2.52e-09	0	\\
2.63e-09	0	\\
2.73e-09	0	\\
2.83e-09	0	\\
2.93e-09	0	\\
3.04e-09	0	\\
3.14e-09	0	\\
3.24e-09	0	\\
3.34e-09	0	\\
3.45e-09	0	\\
3.55e-09	0	\\
3.65e-09	0	\\
3.75e-09	0	\\
3.86e-09	0	\\
3.96e-09	0	\\
4.06e-09	0	\\
4.16e-09	0	\\
4.27e-09	0	\\
4.37e-09	0	\\
4.47e-09	0	\\
4.57e-09	0	\\
4.68e-09	0	\\
4.78e-09	0	\\
4.89e-09	0	\\
4.99e-09	0	\\
5e-09	0	\\
};
\addplot [color=mycolor2,solid,forget plot]
  table[row sep=crcr]{
0	0	\\
1.1e-10	0	\\
2.2e-10	0	\\
3.3e-10	0	\\
4.4e-10	0	\\
5.4e-10	0	\\
6.5e-10	0	\\
7.5e-10	0	\\
8.6e-10	0	\\
9.6e-10	0	\\
1.07e-09	0	\\
1.18e-09	0	\\
1.28e-09	0	\\
1.38e-09	0	\\
1.49e-09	0	\\
1.59e-09	0	\\
1.69e-09	0	\\
1.8e-09	0	\\
1.9e-09	0	\\
2.01e-09	0	\\
2.11e-09	0	\\
2.21e-09	0	\\
2.32e-09	0	\\
2.42e-09	0	\\
2.52e-09	0	\\
2.63e-09	0	\\
2.73e-09	0	\\
2.83e-09	0	\\
2.93e-09	0	\\
3.04e-09	0	\\
3.14e-09	0	\\
3.24e-09	0	\\
3.34e-09	0	\\
3.45e-09	0	\\
3.55e-09	0	\\
3.65e-09	0	\\
3.75e-09	0	\\
3.86e-09	0	\\
3.96e-09	0	\\
4.06e-09	0	\\
4.16e-09	0	\\
4.27e-09	0	\\
4.37e-09	0	\\
4.47e-09	0	\\
4.57e-09	0	\\
4.68e-09	0	\\
4.78e-09	0	\\
4.89e-09	0	\\
4.99e-09	0	\\
5e-09	0	\\
};
\addplot [color=mycolor3,solid,forget plot]
  table[row sep=crcr]{
0	0	\\
1.1e-10	0	\\
2.2e-10	0	\\
3.3e-10	0	\\
4.4e-10	0	\\
5.4e-10	0	\\
6.5e-10	0	\\
7.5e-10	0	\\
8.6e-10	0	\\
9.6e-10	0	\\
1.07e-09	0	\\
1.18e-09	0	\\
1.28e-09	0	\\
1.38e-09	0	\\
1.49e-09	0	\\
1.59e-09	0	\\
1.69e-09	0	\\
1.8e-09	0	\\
1.9e-09	0	\\
2.01e-09	0	\\
2.11e-09	0	\\
2.21e-09	0	\\
2.32e-09	0	\\
2.42e-09	0	\\
2.52e-09	0	\\
2.63e-09	0	\\
2.73e-09	0	\\
2.83e-09	0	\\
2.93e-09	0	\\
3.04e-09	0	\\
3.14e-09	0	\\
3.24e-09	0	\\
3.34e-09	0	\\
3.45e-09	0	\\
3.55e-09	0	\\
3.65e-09	0	\\
3.75e-09	0	\\
3.86e-09	0	\\
3.96e-09	0	\\
4.06e-09	0	\\
4.16e-09	0	\\
4.27e-09	0	\\
4.37e-09	0	\\
4.47e-09	0	\\
4.57e-09	0	\\
4.68e-09	0	\\
4.78e-09	0	\\
4.89e-09	0	\\
4.99e-09	0	\\
5e-09	0	\\
};
\addplot [color=darkgray,solid,forget plot]
  table[row sep=crcr]{
0	0	\\
1.1e-10	0	\\
2.2e-10	0	\\
3.3e-10	0	\\
4.4e-10	0	\\
5.4e-10	0	\\
6.5e-10	0	\\
7.5e-10	0	\\
8.6e-10	0	\\
9.6e-10	0	\\
1.07e-09	0	\\
1.18e-09	0	\\
1.28e-09	0	\\
1.38e-09	0	\\
1.49e-09	0	\\
1.59e-09	0	\\
1.69e-09	0	\\
1.8e-09	0	\\
1.9e-09	0	\\
2.01e-09	0	\\
2.11e-09	0	\\
2.21e-09	0	\\
2.32e-09	0	\\
2.42e-09	0	\\
2.52e-09	0	\\
2.63e-09	0	\\
2.73e-09	0	\\
2.83e-09	0	\\
2.93e-09	0	\\
3.04e-09	0	\\
3.14e-09	0	\\
3.24e-09	0	\\
3.34e-09	0	\\
3.45e-09	0	\\
3.55e-09	0	\\
3.65e-09	0	\\
3.75e-09	0	\\
3.86e-09	0	\\
3.96e-09	0	\\
4.06e-09	0	\\
4.16e-09	0	\\
4.27e-09	0	\\
4.37e-09	0	\\
4.47e-09	0	\\
4.57e-09	0	\\
4.68e-09	0	\\
4.78e-09	0	\\
4.89e-09	0	\\
4.99e-09	0	\\
5e-09	0	\\
};
\addplot [color=blue,solid,forget plot]
  table[row sep=crcr]{
0	0	\\
1.1e-10	0	\\
2.2e-10	0	\\
3.3e-10	0	\\
4.4e-10	0	\\
5.4e-10	0	\\
6.5e-10	0	\\
7.5e-10	0	\\
8.6e-10	0	\\
9.6e-10	0	\\
1.07e-09	0	\\
1.18e-09	0	\\
1.28e-09	0	\\
1.38e-09	0	\\
1.49e-09	0	\\
1.59e-09	0	\\
1.69e-09	0	\\
1.8e-09	0	\\
1.9e-09	0	\\
2.01e-09	0	\\
2.11e-09	0	\\
2.21e-09	0	\\
2.32e-09	0	\\
2.42e-09	0	\\
2.52e-09	0	\\
2.63e-09	0	\\
2.73e-09	0	\\
2.83e-09	0	\\
2.93e-09	0	\\
3.04e-09	0	\\
3.14e-09	0	\\
3.24e-09	0	\\
3.34e-09	0	\\
3.45e-09	0	\\
3.55e-09	0	\\
3.65e-09	0	\\
3.75e-09	0	\\
3.86e-09	0	\\
3.96e-09	0	\\
4.06e-09	0	\\
4.16e-09	0	\\
4.27e-09	0	\\
4.37e-09	0	\\
4.47e-09	0	\\
4.57e-09	0	\\
4.68e-09	0	\\
4.78e-09	0	\\
4.89e-09	0	\\
4.99e-09	0	\\
5e-09	0	\\
};
\addplot [color=black!50!green,solid,forget plot]
  table[row sep=crcr]{
0	0	\\
1.1e-10	0	\\
2.2e-10	0	\\
3.3e-10	0	\\
4.4e-10	0	\\
5.4e-10	0	\\
6.5e-10	0	\\
7.5e-10	0	\\
8.6e-10	0	\\
9.6e-10	0	\\
1.07e-09	0	\\
1.18e-09	0	\\
1.28e-09	0	\\
1.38e-09	0	\\
1.49e-09	0	\\
1.59e-09	0	\\
1.69e-09	0	\\
1.8e-09	0	\\
1.9e-09	0	\\
2.01e-09	0	\\
2.11e-09	0	\\
2.21e-09	0	\\
2.32e-09	0	\\
2.42e-09	0	\\
2.52e-09	0	\\
2.63e-09	0	\\
2.73e-09	0	\\
2.83e-09	0	\\
2.93e-09	0	\\
3.04e-09	0	\\
3.14e-09	0	\\
3.24e-09	0	\\
3.34e-09	0	\\
3.45e-09	0	\\
3.55e-09	0	\\
3.65e-09	0	\\
3.75e-09	0	\\
3.86e-09	0	\\
3.96e-09	0	\\
4.06e-09	0	\\
4.16e-09	0	\\
4.27e-09	0	\\
4.37e-09	0	\\
4.47e-09	0	\\
4.57e-09	0	\\
4.68e-09	0	\\
4.78e-09	0	\\
4.89e-09	0	\\
4.99e-09	0	\\
5e-09	0	\\
};
\addplot [color=red,solid,forget plot]
  table[row sep=crcr]{
0	0	\\
1.1e-10	0	\\
2.2e-10	0	\\
3.3e-10	0	\\
4.4e-10	0	\\
5.4e-10	0	\\
6.5e-10	0	\\
7.5e-10	0	\\
8.6e-10	0	\\
9.6e-10	0	\\
1.07e-09	0	\\
1.18e-09	0	\\
1.28e-09	0	\\
1.38e-09	0	\\
1.49e-09	0	\\
1.59e-09	0	\\
1.69e-09	0	\\
1.8e-09	0	\\
1.9e-09	0	\\
2.01e-09	0	\\
2.11e-09	0	\\
2.21e-09	0	\\
2.32e-09	0	\\
2.42e-09	0	\\
2.52e-09	0	\\
2.63e-09	0	\\
2.73e-09	0	\\
2.83e-09	0	\\
2.93e-09	0	\\
3.04e-09	0	\\
3.14e-09	0	\\
3.24e-09	0	\\
3.34e-09	0	\\
3.45e-09	0	\\
3.55e-09	0	\\
3.65e-09	0	\\
3.75e-09	0	\\
3.86e-09	0	\\
3.96e-09	0	\\
4.06e-09	0	\\
4.16e-09	0	\\
4.27e-09	0	\\
4.37e-09	0	\\
4.47e-09	0	\\
4.57e-09	0	\\
4.68e-09	0	\\
4.78e-09	0	\\
4.89e-09	0	\\
4.99e-09	0	\\
5e-09	0	\\
};
\addplot [color=mycolor1,solid,forget plot]
  table[row sep=crcr]{
0	0	\\
1.1e-10	0	\\
2.2e-10	0	\\
3.3e-10	0	\\
4.4e-10	0	\\
5.4e-10	0	\\
6.5e-10	0	\\
7.5e-10	0	\\
8.6e-10	0	\\
9.6e-10	0	\\
1.07e-09	0	\\
1.18e-09	0	\\
1.28e-09	0	\\
1.38e-09	0	\\
1.49e-09	0	\\
1.59e-09	0	\\
1.69e-09	0	\\
1.8e-09	0	\\
1.9e-09	0	\\
2.01e-09	0	\\
2.11e-09	0	\\
2.21e-09	0	\\
2.32e-09	0	\\
2.42e-09	0	\\
2.52e-09	0	\\
2.63e-09	0	\\
2.73e-09	0	\\
2.83e-09	0	\\
2.93e-09	0	\\
3.04e-09	0	\\
3.14e-09	0	\\
3.24e-09	0	\\
3.34e-09	0	\\
3.45e-09	0	\\
3.55e-09	0	\\
3.65e-09	0	\\
3.75e-09	0	\\
3.86e-09	0	\\
3.96e-09	0	\\
4.06e-09	0	\\
4.16e-09	0	\\
4.27e-09	0	\\
4.37e-09	0	\\
4.47e-09	0	\\
4.57e-09	0	\\
4.68e-09	0	\\
4.78e-09	0	\\
4.89e-09	0	\\
4.99e-09	0	\\
5e-09	0	\\
};
\addplot [color=mycolor2,solid,forget plot]
  table[row sep=crcr]{
0	0	\\
1.1e-10	0	\\
2.2e-10	0	\\
3.3e-10	0	\\
4.4e-10	0	\\
5.4e-10	0	\\
6.5e-10	0	\\
7.5e-10	0	\\
8.6e-10	0	\\
9.6e-10	0	\\
1.07e-09	0	\\
1.18e-09	0	\\
1.28e-09	0	\\
1.38e-09	0	\\
1.49e-09	0	\\
1.59e-09	0	\\
1.69e-09	0	\\
1.8e-09	0	\\
1.9e-09	0	\\
2.01e-09	0	\\
2.11e-09	0	\\
2.21e-09	0	\\
2.32e-09	0	\\
2.42e-09	0	\\
2.52e-09	0	\\
2.63e-09	0	\\
2.73e-09	0	\\
2.83e-09	0	\\
2.93e-09	0	\\
3.04e-09	0	\\
3.14e-09	0	\\
3.24e-09	0	\\
3.34e-09	0	\\
3.45e-09	0	\\
3.55e-09	0	\\
3.65e-09	0	\\
3.75e-09	0	\\
3.86e-09	0	\\
3.96e-09	0	\\
4.06e-09	0	\\
4.16e-09	0	\\
4.27e-09	0	\\
4.37e-09	0	\\
4.47e-09	0	\\
4.57e-09	0	\\
4.68e-09	0	\\
4.78e-09	0	\\
4.89e-09	0	\\
4.99e-09	0	\\
5e-09	0	\\
};
\addplot [color=mycolor3,solid,forget plot]
  table[row sep=crcr]{
0	0	\\
1.1e-10	0	\\
2.2e-10	0	\\
3.3e-10	0	\\
4.4e-10	0	\\
5.4e-10	0	\\
6.5e-10	0	\\
7.5e-10	0	\\
8.6e-10	0	\\
9.6e-10	0	\\
1.07e-09	0	\\
1.18e-09	0	\\
1.28e-09	0	\\
1.38e-09	0	\\
1.49e-09	0	\\
1.59e-09	0	\\
1.69e-09	0	\\
1.8e-09	0	\\
1.9e-09	0	\\
2.01e-09	0	\\
2.11e-09	0	\\
2.21e-09	0	\\
2.32e-09	0	\\
2.42e-09	0	\\
2.52e-09	0	\\
2.63e-09	0	\\
2.73e-09	0	\\
2.83e-09	0	\\
2.93e-09	0	\\
3.04e-09	0	\\
3.14e-09	0	\\
3.24e-09	0	\\
3.34e-09	0	\\
3.45e-09	0	\\
3.55e-09	0	\\
3.65e-09	0	\\
3.75e-09	0	\\
3.86e-09	0	\\
3.96e-09	0	\\
4.06e-09	0	\\
4.16e-09	0	\\
4.27e-09	0	\\
4.37e-09	0	\\
4.47e-09	0	\\
4.57e-09	0	\\
4.68e-09	0	\\
4.78e-09	0	\\
4.89e-09	0	\\
4.99e-09	0	\\
5e-09	0	\\
};
\addplot [color=darkgray,solid,forget plot]
  table[row sep=crcr]{
0	0	\\
1.1e-10	0	\\
2.2e-10	0	\\
3.3e-10	0	\\
4.4e-10	0	\\
5.4e-10	0	\\
6.5e-10	0	\\
7.5e-10	0	\\
8.6e-10	0	\\
9.6e-10	0	\\
1.07e-09	0	\\
1.18e-09	0	\\
1.28e-09	0	\\
1.38e-09	0	\\
1.49e-09	0	\\
1.59e-09	0	\\
1.69e-09	0	\\
1.8e-09	0	\\
1.9e-09	0	\\
2.01e-09	0	\\
2.11e-09	0	\\
2.21e-09	0	\\
2.32e-09	0	\\
2.42e-09	0	\\
2.52e-09	0	\\
2.63e-09	0	\\
2.73e-09	0	\\
2.83e-09	0	\\
2.93e-09	0	\\
3.04e-09	0	\\
3.14e-09	0	\\
3.24e-09	0	\\
3.34e-09	0	\\
3.45e-09	0	\\
3.55e-09	0	\\
3.65e-09	0	\\
3.75e-09	0	\\
3.86e-09	0	\\
3.96e-09	0	\\
4.06e-09	0	\\
4.16e-09	0	\\
4.27e-09	0	\\
4.37e-09	0	\\
4.47e-09	0	\\
4.57e-09	0	\\
4.68e-09	0	\\
4.78e-09	0	\\
4.89e-09	0	\\
4.99e-09	0	\\
5e-09	0	\\
};
\addplot [color=blue,solid,forget plot]
  table[row sep=crcr]{
0	0	\\
1.1e-10	0	\\
2.2e-10	0	\\
3.3e-10	0	\\
4.4e-10	0	\\
5.4e-10	0	\\
6.5e-10	0	\\
7.5e-10	0	\\
8.6e-10	0	\\
9.6e-10	0	\\
1.07e-09	0	\\
1.18e-09	0	\\
1.28e-09	0	\\
1.38e-09	0	\\
1.49e-09	0	\\
1.59e-09	0	\\
1.69e-09	0	\\
1.8e-09	0	\\
1.9e-09	0	\\
2.01e-09	0	\\
2.11e-09	0	\\
2.21e-09	0	\\
2.32e-09	0	\\
2.42e-09	0	\\
2.52e-09	0	\\
2.63e-09	0	\\
2.73e-09	0	\\
2.83e-09	0	\\
2.93e-09	0	\\
3.04e-09	0	\\
3.14e-09	0	\\
3.24e-09	0	\\
3.34e-09	0	\\
3.45e-09	0	\\
3.55e-09	0	\\
3.65e-09	0	\\
3.75e-09	0	\\
3.86e-09	0	\\
3.96e-09	0	\\
4.06e-09	0	\\
4.16e-09	0	\\
4.27e-09	0	\\
4.37e-09	0	\\
4.47e-09	0	\\
4.57e-09	0	\\
4.68e-09	0	\\
4.78e-09	0	\\
4.89e-09	0	\\
4.99e-09	0	\\
5e-09	0	\\
};
\addplot [color=black!50!green,solid,forget plot]
  table[row sep=crcr]{
0	0	\\
1.1e-10	0	\\
2.2e-10	0	\\
3.3e-10	0	\\
4.4e-10	0	\\
5.4e-10	0	\\
6.5e-10	0	\\
7.5e-10	0	\\
8.6e-10	0	\\
9.6e-10	0	\\
1.07e-09	0	\\
1.18e-09	0	\\
1.28e-09	0	\\
1.38e-09	0	\\
1.49e-09	0	\\
1.59e-09	0	\\
1.69e-09	0	\\
1.8e-09	0	\\
1.9e-09	0	\\
2.01e-09	0	\\
2.11e-09	0	\\
2.21e-09	0	\\
2.32e-09	0	\\
2.42e-09	0	\\
2.52e-09	0	\\
2.63e-09	0	\\
2.73e-09	0	\\
2.83e-09	0	\\
2.93e-09	0	\\
3.04e-09	0	\\
3.14e-09	0	\\
3.24e-09	0	\\
3.34e-09	0	\\
3.45e-09	0	\\
3.55e-09	0	\\
3.65e-09	0	\\
3.75e-09	0	\\
3.86e-09	0	\\
3.96e-09	0	\\
4.06e-09	0	\\
4.16e-09	0	\\
4.27e-09	0	\\
4.37e-09	0	\\
4.47e-09	0	\\
4.57e-09	0	\\
4.68e-09	0	\\
4.78e-09	0	\\
4.89e-09	0	\\
4.99e-09	0	\\
5e-09	0	\\
};
\addplot [color=red,solid,forget plot]
  table[row sep=crcr]{
0	0	\\
1.1e-10	0	\\
2.2e-10	0	\\
3.3e-10	0	\\
4.4e-10	0	\\
5.4e-10	0	\\
6.5e-10	0	\\
7.5e-10	0	\\
8.6e-10	0	\\
9.6e-10	0	\\
1.07e-09	0	\\
1.18e-09	0	\\
1.28e-09	0	\\
1.38e-09	0	\\
1.49e-09	0	\\
1.59e-09	0	\\
1.69e-09	0	\\
1.8e-09	0	\\
1.9e-09	0	\\
2.01e-09	0	\\
2.11e-09	0	\\
2.21e-09	0	\\
2.32e-09	0	\\
2.42e-09	0	\\
2.52e-09	0	\\
2.63e-09	0	\\
2.73e-09	0	\\
2.83e-09	0	\\
2.93e-09	0	\\
3.04e-09	0	\\
3.14e-09	0	\\
3.24e-09	0	\\
3.34e-09	0	\\
3.45e-09	0	\\
3.55e-09	0	\\
3.65e-09	0	\\
3.75e-09	0	\\
3.86e-09	0	\\
3.96e-09	0	\\
4.06e-09	0	\\
4.16e-09	0	\\
4.27e-09	0	\\
4.37e-09	0	\\
4.47e-09	0	\\
4.57e-09	0	\\
4.68e-09	0	\\
4.78e-09	0	\\
4.89e-09	0	\\
4.99e-09	0	\\
5e-09	0	\\
};
\addplot [color=mycolor1,solid,forget plot]
  table[row sep=crcr]{
0	0	\\
1.1e-10	0	\\
2.2e-10	0	\\
3.3e-10	0	\\
4.4e-10	0	\\
5.4e-10	0	\\
6.5e-10	0	\\
7.5e-10	0	\\
8.6e-10	0	\\
9.6e-10	0	\\
1.07e-09	0	\\
1.18e-09	0	\\
1.28e-09	0	\\
1.38e-09	0	\\
1.49e-09	0	\\
1.59e-09	0	\\
1.69e-09	0	\\
1.8e-09	0	\\
1.9e-09	0	\\
2.01e-09	0	\\
2.11e-09	0	\\
2.21e-09	0	\\
2.32e-09	0	\\
2.42e-09	0	\\
2.52e-09	0	\\
2.63e-09	0	\\
2.73e-09	0	\\
2.83e-09	0	\\
2.93e-09	0	\\
3.04e-09	0	\\
3.14e-09	0	\\
3.24e-09	0	\\
3.34e-09	0	\\
3.45e-09	0	\\
3.55e-09	0	\\
3.65e-09	0	\\
3.75e-09	0	\\
3.86e-09	0	\\
3.96e-09	0	\\
4.06e-09	0	\\
4.16e-09	0	\\
4.27e-09	0	\\
4.37e-09	0	\\
4.47e-09	0	\\
4.57e-09	0	\\
4.68e-09	0	\\
4.78e-09	0	\\
4.89e-09	0	\\
4.99e-09	0	\\
5e-09	0	\\
};
\addplot [color=mycolor2,solid,forget plot]
  table[row sep=crcr]{
0	0	\\
1.1e-10	0	\\
2.2e-10	0	\\
3.3e-10	0	\\
4.4e-10	0	\\
5.4e-10	0	\\
6.5e-10	0	\\
7.5e-10	0	\\
8.6e-10	0	\\
9.6e-10	0	\\
1.07e-09	0	\\
1.18e-09	0	\\
1.28e-09	0	\\
1.38e-09	0	\\
1.49e-09	0	\\
1.59e-09	0	\\
1.69e-09	0	\\
1.8e-09	0	\\
1.9e-09	0	\\
2.01e-09	0	\\
2.11e-09	0	\\
2.21e-09	0	\\
2.32e-09	0	\\
2.42e-09	0	\\
2.52e-09	0	\\
2.63e-09	0	\\
2.73e-09	0	\\
2.83e-09	0	\\
2.93e-09	0	\\
3.04e-09	0	\\
3.14e-09	0	\\
3.24e-09	0	\\
3.34e-09	0	\\
3.45e-09	0	\\
3.55e-09	0	\\
3.65e-09	0	\\
3.75e-09	0	\\
3.86e-09	0	\\
3.96e-09	0	\\
4.06e-09	0	\\
4.16e-09	0	\\
4.27e-09	0	\\
4.37e-09	0	\\
4.47e-09	0	\\
4.57e-09	0	\\
4.68e-09	0	\\
4.78e-09	0	\\
4.89e-09	0	\\
4.99e-09	0	\\
5e-09	0	\\
};
\addplot [color=mycolor3,solid,forget plot]
  table[row sep=crcr]{
0	0	\\
1.1e-10	0	\\
2.2e-10	0	\\
3.3e-10	0	\\
4.4e-10	0	\\
5.4e-10	0	\\
6.5e-10	0	\\
7.5e-10	0	\\
8.6e-10	0	\\
9.6e-10	0	\\
1.07e-09	0	\\
1.18e-09	0	\\
1.28e-09	0	\\
1.38e-09	0	\\
1.49e-09	0	\\
1.59e-09	0	\\
1.69e-09	0	\\
1.8e-09	0	\\
1.9e-09	0	\\
2.01e-09	0	\\
2.11e-09	0	\\
2.21e-09	0	\\
2.32e-09	0	\\
2.42e-09	0	\\
2.52e-09	0	\\
2.63e-09	0	\\
2.73e-09	0	\\
2.83e-09	0	\\
2.93e-09	0	\\
3.04e-09	0	\\
3.14e-09	0	\\
3.24e-09	0	\\
3.34e-09	0	\\
3.45e-09	0	\\
3.55e-09	0	\\
3.65e-09	0	\\
3.75e-09	0	\\
3.86e-09	0	\\
3.96e-09	0	\\
4.06e-09	0	\\
4.16e-09	0	\\
4.27e-09	0	\\
4.37e-09	0	\\
4.47e-09	0	\\
4.57e-09	0	\\
4.68e-09	0	\\
4.78e-09	0	\\
4.89e-09	0	\\
4.99e-09	0	\\
5e-09	0	\\
};
\addplot [color=darkgray,solid,forget plot]
  table[row sep=crcr]{
0	0	\\
1.1e-10	0	\\
2.2e-10	0	\\
3.3e-10	0	\\
4.4e-10	0	\\
5.4e-10	0	\\
6.5e-10	0	\\
7.5e-10	0	\\
8.6e-10	0	\\
9.6e-10	0	\\
1.07e-09	0	\\
1.18e-09	0	\\
1.28e-09	0	\\
1.38e-09	0	\\
1.49e-09	0	\\
1.59e-09	0	\\
1.69e-09	0	\\
1.8e-09	0	\\
1.9e-09	0	\\
2.01e-09	0	\\
2.11e-09	0	\\
2.21e-09	0	\\
2.32e-09	0	\\
2.42e-09	0	\\
2.52e-09	0	\\
2.63e-09	0	\\
2.73e-09	0	\\
2.83e-09	0	\\
2.93e-09	0	\\
3.04e-09	0	\\
3.14e-09	0	\\
3.24e-09	0	\\
3.34e-09	0	\\
3.45e-09	0	\\
3.55e-09	0	\\
3.65e-09	0	\\
3.75e-09	0	\\
3.86e-09	0	\\
3.96e-09	0	\\
4.06e-09	0	\\
4.16e-09	0	\\
4.27e-09	0	\\
4.37e-09	0	\\
4.47e-09	0	\\
4.57e-09	0	\\
4.68e-09	0	\\
4.78e-09	0	\\
4.89e-09	0	\\
4.99e-09	0	\\
5e-09	0	\\
};
\addplot [color=blue,solid,forget plot]
  table[row sep=crcr]{
0	0	\\
1.1e-10	0	\\
2.2e-10	0	\\
3.3e-10	0	\\
4.4e-10	0	\\
5.4e-10	0	\\
6.5e-10	0	\\
7.5e-10	0	\\
8.6e-10	0	\\
9.6e-10	0	\\
1.07e-09	0	\\
1.18e-09	0	\\
1.28e-09	0	\\
1.38e-09	0	\\
1.49e-09	0	\\
1.59e-09	0	\\
1.69e-09	0	\\
1.8e-09	0	\\
1.9e-09	0	\\
2.01e-09	0	\\
2.11e-09	0	\\
2.21e-09	0	\\
2.32e-09	0	\\
2.42e-09	0	\\
2.52e-09	0	\\
2.63e-09	0	\\
2.73e-09	0	\\
2.83e-09	0	\\
2.93e-09	0	\\
3.04e-09	0	\\
3.14e-09	0	\\
3.24e-09	0	\\
3.34e-09	0	\\
3.45e-09	0	\\
3.55e-09	0	\\
3.65e-09	0	\\
3.75e-09	0	\\
3.86e-09	0	\\
3.96e-09	0	\\
4.06e-09	0	\\
4.16e-09	0	\\
4.27e-09	0	\\
4.37e-09	0	\\
4.47e-09	0	\\
4.57e-09	0	\\
4.68e-09	0	\\
4.78e-09	0	\\
4.89e-09	0	\\
4.99e-09	0	\\
5e-09	0	\\
};
\addplot [color=black!50!green,solid,forget plot]
  table[row sep=crcr]{
0	0	\\
1.1e-10	0	\\
2.2e-10	0	\\
3.3e-10	0	\\
4.4e-10	0	\\
5.4e-10	0	\\
6.5e-10	0	\\
7.5e-10	0	\\
8.6e-10	0	\\
9.6e-10	0	\\
1.07e-09	0	\\
1.18e-09	0	\\
1.28e-09	0	\\
1.38e-09	0	\\
1.49e-09	0	\\
1.59e-09	0	\\
1.69e-09	0	\\
1.8e-09	0	\\
1.9e-09	0	\\
2.01e-09	0	\\
2.11e-09	0	\\
2.21e-09	0	\\
2.32e-09	0	\\
2.42e-09	0	\\
2.52e-09	0	\\
2.63e-09	0	\\
2.73e-09	0	\\
2.83e-09	0	\\
2.93e-09	0	\\
3.04e-09	0	\\
3.14e-09	0	\\
3.24e-09	0	\\
3.34e-09	0	\\
3.45e-09	0	\\
3.55e-09	0	\\
3.65e-09	0	\\
3.75e-09	0	\\
3.86e-09	0	\\
3.96e-09	0	\\
4.06e-09	0	\\
4.16e-09	0	\\
4.27e-09	0	\\
4.37e-09	0	\\
4.47e-09	0	\\
4.57e-09	0	\\
4.68e-09	0	\\
4.78e-09	0	\\
4.89e-09	0	\\
4.99e-09	0	\\
5e-09	0	\\
};
\addplot [color=red,solid,forget plot]
  table[row sep=crcr]{
0	0	\\
1.1e-10	0	\\
2.2e-10	0	\\
3.3e-10	0	\\
4.4e-10	0	\\
5.4e-10	0	\\
6.5e-10	0	\\
7.5e-10	0	\\
8.6e-10	0	\\
9.6e-10	0	\\
1.07e-09	0	\\
1.18e-09	0	\\
1.28e-09	0	\\
1.38e-09	0	\\
1.49e-09	0	\\
1.59e-09	0	\\
1.69e-09	0	\\
1.8e-09	0	\\
1.9e-09	0	\\
2.01e-09	0	\\
2.11e-09	0	\\
2.21e-09	0	\\
2.32e-09	0	\\
2.42e-09	0	\\
2.52e-09	0	\\
2.63e-09	0	\\
2.73e-09	0	\\
2.83e-09	0	\\
2.93e-09	0	\\
3.04e-09	0	\\
3.14e-09	0	\\
3.24e-09	0	\\
3.34e-09	0	\\
3.45e-09	0	\\
3.55e-09	0	\\
3.65e-09	0	\\
3.75e-09	0	\\
3.86e-09	0	\\
3.96e-09	0	\\
4.06e-09	0	\\
4.16e-09	0	\\
4.27e-09	0	\\
4.37e-09	0	\\
4.47e-09	0	\\
4.57e-09	0	\\
4.68e-09	0	\\
4.78e-09	0	\\
4.89e-09	0	\\
4.99e-09	0	\\
5e-09	0	\\
};
\addplot [color=mycolor1,solid,forget plot]
  table[row sep=crcr]{
0	0	\\
1.1e-10	0	\\
2.2e-10	0	\\
3.3e-10	0	\\
4.4e-10	0	\\
5.4e-10	0	\\
6.5e-10	0	\\
7.5e-10	0	\\
8.6e-10	0	\\
9.6e-10	0	\\
1.07e-09	0	\\
1.18e-09	0	\\
1.28e-09	0	\\
1.38e-09	0	\\
1.49e-09	0	\\
1.59e-09	0	\\
1.69e-09	0	\\
1.8e-09	0	\\
1.9e-09	0	\\
2.01e-09	0	\\
2.11e-09	0	\\
2.21e-09	0	\\
2.32e-09	0	\\
2.42e-09	0	\\
2.52e-09	0	\\
2.63e-09	0	\\
2.73e-09	0	\\
2.83e-09	0	\\
2.93e-09	0	\\
3.04e-09	0	\\
3.14e-09	0	\\
3.24e-09	0	\\
3.34e-09	0	\\
3.45e-09	0	\\
3.55e-09	0	\\
3.65e-09	0	\\
3.75e-09	0	\\
3.86e-09	0	\\
3.96e-09	0	\\
4.06e-09	0	\\
4.16e-09	0	\\
4.27e-09	0	\\
4.37e-09	0	\\
4.47e-09	0	\\
4.57e-09	0	\\
4.68e-09	0	\\
4.78e-09	0	\\
4.89e-09	0	\\
4.99e-09	0	\\
5e-09	0	\\
};
\addplot [color=mycolor2,solid,forget plot]
  table[row sep=crcr]{
0	0	\\
1.1e-10	0	\\
2.2e-10	0	\\
3.3e-10	0	\\
4.4e-10	0	\\
5.4e-10	0	\\
6.5e-10	0	\\
7.5e-10	0	\\
8.6e-10	0	\\
9.6e-10	0	\\
1.07e-09	0	\\
1.18e-09	0	\\
1.28e-09	0	\\
1.38e-09	0	\\
1.49e-09	0	\\
1.59e-09	0	\\
1.69e-09	0	\\
1.8e-09	0	\\
1.9e-09	0	\\
2.01e-09	0	\\
2.11e-09	0	\\
2.21e-09	0	\\
2.32e-09	0	\\
2.42e-09	0	\\
2.52e-09	0	\\
2.63e-09	0	\\
2.73e-09	0	\\
2.83e-09	0	\\
2.93e-09	0	\\
3.04e-09	0	\\
3.14e-09	0	\\
3.24e-09	0	\\
3.34e-09	0	\\
3.45e-09	0	\\
3.55e-09	0	\\
3.65e-09	0	\\
3.75e-09	0	\\
3.86e-09	0	\\
3.96e-09	0	\\
4.06e-09	0	\\
4.16e-09	0	\\
4.27e-09	0	\\
4.37e-09	0	\\
4.47e-09	0	\\
4.57e-09	0	\\
4.68e-09	0	\\
4.78e-09	0	\\
4.89e-09	0	\\
4.99e-09	0	\\
5e-09	0	\\
};
\addplot [color=mycolor3,solid,forget plot]
  table[row sep=crcr]{
0	0	\\
1.1e-10	0	\\
2.2e-10	0	\\
3.3e-10	0	\\
4.4e-10	0	\\
5.4e-10	0	\\
6.5e-10	0	\\
7.5e-10	0	\\
8.6e-10	0	\\
9.6e-10	0	\\
1.07e-09	0	\\
1.18e-09	0	\\
1.28e-09	0	\\
1.38e-09	0	\\
1.49e-09	0	\\
1.59e-09	0	\\
1.69e-09	0	\\
1.8e-09	0	\\
1.9e-09	0	\\
2.01e-09	0	\\
2.11e-09	0	\\
2.21e-09	0	\\
2.32e-09	0	\\
2.42e-09	0	\\
2.52e-09	0	\\
2.63e-09	0	\\
2.73e-09	0	\\
2.83e-09	0	\\
2.93e-09	0	\\
3.04e-09	0	\\
3.14e-09	0	\\
3.24e-09	0	\\
3.34e-09	0	\\
3.45e-09	0	\\
3.55e-09	0	\\
3.65e-09	0	\\
3.75e-09	0	\\
3.86e-09	0	\\
3.96e-09	0	\\
4.06e-09	0	\\
4.16e-09	0	\\
4.27e-09	0	\\
4.37e-09	0	\\
4.47e-09	0	\\
4.57e-09	0	\\
4.68e-09	0	\\
4.78e-09	0	\\
4.89e-09	0	\\
4.99e-09	0	\\
5e-09	0	\\
};
\addplot [color=darkgray,solid,forget plot]
  table[row sep=crcr]{
0	0	\\
1.1e-10	0	\\
2.2e-10	0	\\
3.3e-10	0	\\
4.4e-10	0	\\
5.4e-10	0	\\
6.5e-10	0	\\
7.5e-10	0	\\
8.6e-10	0	\\
9.6e-10	0	\\
1.07e-09	0	\\
1.18e-09	0	\\
1.28e-09	0	\\
1.38e-09	0	\\
1.49e-09	0	\\
1.59e-09	0	\\
1.69e-09	0	\\
1.8e-09	0	\\
1.9e-09	0	\\
2.01e-09	0	\\
2.11e-09	0	\\
2.21e-09	0	\\
2.32e-09	0	\\
2.42e-09	0	\\
2.52e-09	0	\\
2.63e-09	0	\\
2.73e-09	0	\\
2.83e-09	0	\\
2.93e-09	0	\\
3.04e-09	0	\\
3.14e-09	0	\\
3.24e-09	0	\\
3.34e-09	0	\\
3.45e-09	0	\\
3.55e-09	0	\\
3.65e-09	0	\\
3.75e-09	0	\\
3.86e-09	0	\\
3.96e-09	0	\\
4.06e-09	0	\\
4.16e-09	0	\\
4.27e-09	0	\\
4.37e-09	0	\\
4.47e-09	0	\\
4.57e-09	0	\\
4.68e-09	0	\\
4.78e-09	0	\\
4.89e-09	0	\\
4.99e-09	0	\\
5e-09	0	\\
};
\addplot [color=blue,solid,forget plot]
  table[row sep=crcr]{
0	0	\\
1.1e-10	0	\\
2.2e-10	0	\\
3.3e-10	0	\\
4.4e-10	0	\\
5.4e-10	0	\\
6.5e-10	0	\\
7.5e-10	0	\\
8.6e-10	0	\\
9.6e-10	0	\\
1.07e-09	0	\\
1.18e-09	0	\\
1.28e-09	0	\\
1.38e-09	0	\\
1.49e-09	0	\\
1.59e-09	0	\\
1.69e-09	0	\\
1.8e-09	0	\\
1.9e-09	0	\\
2.01e-09	0	\\
2.11e-09	0	\\
2.21e-09	0	\\
2.32e-09	0	\\
2.42e-09	0	\\
2.52e-09	0	\\
2.63e-09	0	\\
2.73e-09	0	\\
2.83e-09	0	\\
2.93e-09	0	\\
3.04e-09	0	\\
3.14e-09	0	\\
3.24e-09	0	\\
3.34e-09	0	\\
3.45e-09	0	\\
3.55e-09	0	\\
3.65e-09	0	\\
3.75e-09	0	\\
3.86e-09	0	\\
3.96e-09	0	\\
4.06e-09	0	\\
4.16e-09	0	\\
4.27e-09	0	\\
4.37e-09	0	\\
4.47e-09	0	\\
4.57e-09	0	\\
4.68e-09	0	\\
4.78e-09	0	\\
4.89e-09	0	\\
4.99e-09	0	\\
5e-09	0	\\
};
\addplot [color=black!50!green,solid,forget plot]
  table[row sep=crcr]{
0	0	\\
1.1e-10	0	\\
2.2e-10	0	\\
3.3e-10	0	\\
4.4e-10	0	\\
5.4e-10	0	\\
6.5e-10	0	\\
7.5e-10	0	\\
8.6e-10	0	\\
9.6e-10	0	\\
1.07e-09	0	\\
1.18e-09	0	\\
1.28e-09	0	\\
1.38e-09	0	\\
1.49e-09	0	\\
1.59e-09	0	\\
1.69e-09	0	\\
1.8e-09	0	\\
1.9e-09	0	\\
2.01e-09	0	\\
2.11e-09	0	\\
2.21e-09	0	\\
2.32e-09	0	\\
2.42e-09	0	\\
2.52e-09	0	\\
2.63e-09	0	\\
2.73e-09	0	\\
2.83e-09	0	\\
2.93e-09	0	\\
3.04e-09	0	\\
3.14e-09	0	\\
3.24e-09	0	\\
3.34e-09	0	\\
3.45e-09	0	\\
3.55e-09	0	\\
3.65e-09	0	\\
3.75e-09	0	\\
3.86e-09	0	\\
3.96e-09	0	\\
4.06e-09	0	\\
4.16e-09	0	\\
4.27e-09	0	\\
4.37e-09	0	\\
4.47e-09	0	\\
4.57e-09	0	\\
4.68e-09	0	\\
4.78e-09	0	\\
4.89e-09	0	\\
4.99e-09	0	\\
5e-09	0	\\
};
\addplot [color=red,solid,forget plot]
  table[row sep=crcr]{
0	0	\\
1.1e-10	0	\\
2.2e-10	0	\\
3.3e-10	0	\\
4.4e-10	0	\\
5.4e-10	0	\\
6.5e-10	0	\\
7.5e-10	0	\\
8.6e-10	0	\\
9.6e-10	0	\\
1.07e-09	0	\\
1.18e-09	0	\\
1.28e-09	0	\\
1.38e-09	0	\\
1.49e-09	0	\\
1.59e-09	0	\\
1.69e-09	0	\\
1.8e-09	0	\\
1.9e-09	0	\\
2.01e-09	0	\\
2.11e-09	0	\\
2.21e-09	0	\\
2.32e-09	0	\\
2.42e-09	0	\\
2.52e-09	0	\\
2.63e-09	0	\\
2.73e-09	0	\\
2.83e-09	0	\\
2.93e-09	0	\\
3.04e-09	0	\\
3.14e-09	0	\\
3.24e-09	0	\\
3.34e-09	0	\\
3.45e-09	0	\\
3.55e-09	0	\\
3.65e-09	0	\\
3.75e-09	0	\\
3.86e-09	0	\\
3.96e-09	0	\\
4.06e-09	0	\\
4.16e-09	0	\\
4.27e-09	0	\\
4.37e-09	0	\\
4.47e-09	0	\\
4.57e-09	0	\\
4.68e-09	0	\\
4.78e-09	0	\\
4.89e-09	0	\\
4.99e-09	0	\\
5e-09	0	\\
};
\addplot [color=mycolor1,solid,forget plot]
  table[row sep=crcr]{
0	0	\\
1.1e-10	0	\\
2.2e-10	0	\\
3.3e-10	0	\\
4.4e-10	0	\\
5.4e-10	0	\\
6.5e-10	0	\\
7.5e-10	0	\\
8.6e-10	0	\\
9.6e-10	0	\\
1.07e-09	0	\\
1.18e-09	0	\\
1.28e-09	0	\\
1.38e-09	0	\\
1.49e-09	0	\\
1.59e-09	0	\\
1.69e-09	0	\\
1.8e-09	0	\\
1.9e-09	0	\\
2.01e-09	0	\\
2.11e-09	0	\\
2.21e-09	0	\\
2.32e-09	0	\\
2.42e-09	0	\\
2.52e-09	0	\\
2.63e-09	0	\\
2.73e-09	0	\\
2.83e-09	0	\\
2.93e-09	0	\\
3.04e-09	0	\\
3.14e-09	0	\\
3.24e-09	0	\\
3.34e-09	0	\\
3.45e-09	0	\\
3.55e-09	0	\\
3.65e-09	0	\\
3.75e-09	0	\\
3.86e-09	0	\\
3.96e-09	0	\\
4.06e-09	0	\\
4.16e-09	0	\\
4.27e-09	0	\\
4.37e-09	0	\\
4.47e-09	0	\\
4.57e-09	0	\\
4.68e-09	0	\\
4.78e-09	0	\\
4.89e-09	0	\\
4.99e-09	0	\\
5e-09	0	\\
};
\addplot [color=mycolor2,solid,forget plot]
  table[row sep=crcr]{
0	0	\\
1.1e-10	0	\\
2.2e-10	0	\\
3.3e-10	0	\\
4.4e-10	0	\\
5.4e-10	0	\\
6.5e-10	0	\\
7.5e-10	0	\\
8.6e-10	0	\\
9.6e-10	0	\\
1.07e-09	0	\\
1.18e-09	0	\\
1.28e-09	0	\\
1.38e-09	0	\\
1.49e-09	0	\\
1.59e-09	0	\\
1.69e-09	0	\\
1.8e-09	0	\\
1.9e-09	0	\\
2.01e-09	0	\\
2.11e-09	0	\\
2.21e-09	0	\\
2.32e-09	0	\\
2.42e-09	0	\\
2.52e-09	0	\\
2.63e-09	0	\\
2.73e-09	0	\\
2.83e-09	0	\\
2.93e-09	0	\\
3.04e-09	0	\\
3.14e-09	0	\\
3.24e-09	0	\\
3.34e-09	0	\\
3.45e-09	0	\\
3.55e-09	0	\\
3.65e-09	0	\\
3.75e-09	0	\\
3.86e-09	0	\\
3.96e-09	0	\\
4.06e-09	0	\\
4.16e-09	0	\\
4.27e-09	0	\\
4.37e-09	0	\\
4.47e-09	0	\\
4.57e-09	0	\\
4.68e-09	0	\\
4.78e-09	0	\\
4.89e-09	0	\\
4.99e-09	0	\\
5e-09	0	\\
};
\addplot [color=mycolor3,solid,forget plot]
  table[row sep=crcr]{
0	0	\\
1.1e-10	0	\\
2.2e-10	0	\\
3.3e-10	0	\\
4.4e-10	0	\\
5.4e-10	0	\\
6.5e-10	0	\\
7.5e-10	0	\\
8.6e-10	0	\\
9.6e-10	0	\\
1.07e-09	0	\\
1.18e-09	0	\\
1.28e-09	0	\\
1.38e-09	0	\\
1.49e-09	0	\\
1.59e-09	0	\\
1.69e-09	0	\\
1.8e-09	0	\\
1.9e-09	0	\\
2.01e-09	0	\\
2.11e-09	0	\\
2.21e-09	0	\\
2.32e-09	0	\\
2.42e-09	0	\\
2.52e-09	0	\\
2.63e-09	0	\\
2.73e-09	0	\\
2.83e-09	0	\\
2.93e-09	0	\\
3.04e-09	0	\\
3.14e-09	0	\\
3.24e-09	0	\\
3.34e-09	0	\\
3.45e-09	0	\\
3.55e-09	0	\\
3.65e-09	0	\\
3.75e-09	0	\\
3.86e-09	0	\\
3.96e-09	0	\\
4.06e-09	0	\\
4.16e-09	0	\\
4.27e-09	0	\\
4.37e-09	0	\\
4.47e-09	0	\\
4.57e-09	0	\\
4.68e-09	0	\\
4.78e-09	0	\\
4.89e-09	0	\\
4.99e-09	0	\\
5e-09	0	\\
};
\addplot [color=darkgray,solid,forget plot]
  table[row sep=crcr]{
0	0	\\
1.1e-10	0	\\
2.2e-10	0	\\
3.3e-10	0	\\
4.4e-10	0	\\
5.4e-10	0	\\
6.5e-10	0	\\
7.5e-10	0	\\
8.6e-10	0	\\
9.6e-10	0	\\
1.07e-09	0	\\
1.18e-09	0	\\
1.28e-09	0	\\
1.38e-09	0	\\
1.49e-09	0	\\
1.59e-09	0	\\
1.69e-09	0	\\
1.8e-09	0	\\
1.9e-09	0	\\
2.01e-09	0	\\
2.11e-09	0	\\
2.21e-09	0	\\
2.32e-09	0	\\
2.42e-09	0	\\
2.52e-09	0	\\
2.63e-09	0	\\
2.73e-09	0	\\
2.83e-09	0	\\
2.93e-09	0	\\
3.04e-09	0	\\
3.14e-09	0	\\
3.24e-09	0	\\
3.34e-09	0	\\
3.45e-09	0	\\
3.55e-09	0	\\
3.65e-09	0	\\
3.75e-09	0	\\
3.86e-09	0	\\
3.96e-09	0	\\
4.06e-09	0	\\
4.16e-09	0	\\
4.27e-09	0	\\
4.37e-09	0	\\
4.47e-09	0	\\
4.57e-09	0	\\
4.68e-09	0	\\
4.78e-09	0	\\
4.89e-09	0	\\
4.99e-09	0	\\
5e-09	0	\\
};
\addplot [color=blue,solid,forget plot]
  table[row sep=crcr]{
0	0	\\
1.1e-10	0	\\
2.2e-10	0	\\
3.3e-10	0	\\
4.4e-10	0	\\
5.4e-10	0	\\
6.5e-10	0	\\
7.5e-10	0	\\
8.6e-10	0	\\
9.6e-10	0	\\
1.07e-09	0	\\
1.18e-09	0	\\
1.28e-09	0	\\
1.38e-09	0	\\
1.49e-09	0	\\
1.59e-09	0	\\
1.69e-09	0	\\
1.8e-09	0	\\
1.9e-09	0	\\
2.01e-09	0	\\
2.11e-09	0	\\
2.21e-09	0	\\
2.32e-09	0	\\
2.42e-09	0	\\
2.52e-09	0	\\
2.63e-09	0	\\
2.73e-09	0	\\
2.83e-09	0	\\
2.93e-09	0	\\
3.04e-09	0	\\
3.14e-09	0	\\
3.24e-09	0	\\
3.34e-09	0	\\
3.45e-09	0	\\
3.55e-09	0	\\
3.65e-09	0	\\
3.75e-09	0	\\
3.86e-09	0	\\
3.96e-09	0	\\
4.06e-09	0	\\
4.16e-09	0	\\
4.27e-09	0	\\
4.37e-09	0	\\
4.47e-09	0	\\
4.57e-09	0	\\
4.68e-09	0	\\
4.78e-09	0	\\
4.89e-09	0	\\
4.99e-09	0	\\
5e-09	0	\\
};
\addplot [color=black!50!green,solid,forget plot]
  table[row sep=crcr]{
0	0	\\
1.1e-10	0	\\
2.2e-10	0	\\
3.3e-10	0	\\
4.4e-10	0	\\
5.4e-10	0	\\
6.5e-10	0	\\
7.5e-10	0	\\
8.6e-10	0	\\
9.6e-10	0	\\
1.07e-09	0	\\
1.18e-09	0	\\
1.28e-09	0	\\
1.38e-09	0	\\
1.49e-09	0	\\
1.59e-09	0	\\
1.69e-09	0	\\
1.8e-09	0	\\
1.9e-09	0	\\
2.01e-09	0	\\
2.11e-09	0	\\
2.21e-09	0	\\
2.32e-09	0	\\
2.42e-09	0	\\
2.52e-09	0	\\
2.63e-09	0	\\
2.73e-09	0	\\
2.83e-09	0	\\
2.93e-09	0	\\
3.04e-09	0	\\
3.14e-09	0	\\
3.24e-09	0	\\
3.34e-09	0	\\
3.45e-09	0	\\
3.55e-09	0	\\
3.65e-09	0	\\
3.75e-09	0	\\
3.86e-09	0	\\
3.96e-09	0	\\
4.06e-09	0	\\
4.16e-09	0	\\
4.27e-09	0	\\
4.37e-09	0	\\
4.47e-09	0	\\
4.57e-09	0	\\
4.68e-09	0	\\
4.78e-09	0	\\
4.89e-09	0	\\
4.99e-09	0	\\
5e-09	0	\\
};
\addplot [color=red,solid,forget plot]
  table[row sep=crcr]{
0	0	\\
1.1e-10	0	\\
2.2e-10	0	\\
3.3e-10	0	\\
4.4e-10	0	\\
5.4e-10	0	\\
6.5e-10	0	\\
7.5e-10	0	\\
8.6e-10	0	\\
9.6e-10	0	\\
1.07e-09	0	\\
1.18e-09	0	\\
1.28e-09	0	\\
1.38e-09	0	\\
1.49e-09	0	\\
1.59e-09	0	\\
1.69e-09	0	\\
1.8e-09	0	\\
1.9e-09	0	\\
2.01e-09	0	\\
2.11e-09	0	\\
2.21e-09	0	\\
2.32e-09	0	\\
2.42e-09	0	\\
2.52e-09	0	\\
2.63e-09	0	\\
2.73e-09	0	\\
2.83e-09	0	\\
2.93e-09	0	\\
3.04e-09	0	\\
3.14e-09	0	\\
3.24e-09	0	\\
3.34e-09	0	\\
3.45e-09	0	\\
3.55e-09	0	\\
3.65e-09	0	\\
3.75e-09	0	\\
3.86e-09	0	\\
3.96e-09	0	\\
4.06e-09	0	\\
4.16e-09	0	\\
4.27e-09	0	\\
4.37e-09	0	\\
4.47e-09	0	\\
4.57e-09	0	\\
4.68e-09	0	\\
4.78e-09	0	\\
4.89e-09	0	\\
4.99e-09	0	\\
5e-09	0	\\
};
\addplot [color=mycolor1,solid,forget plot]
  table[row sep=crcr]{
0	0	\\
1.1e-10	0	\\
2.2e-10	0	\\
3.3e-10	0	\\
4.4e-10	0	\\
5.4e-10	0	\\
6.5e-10	0	\\
7.5e-10	0	\\
8.6e-10	0	\\
9.6e-10	0	\\
1.07e-09	0	\\
1.18e-09	0	\\
1.28e-09	0	\\
1.38e-09	0	\\
1.49e-09	0	\\
1.59e-09	0	\\
1.69e-09	0	\\
1.8e-09	0	\\
1.9e-09	0	\\
2.01e-09	0	\\
2.11e-09	0	\\
2.21e-09	0	\\
2.32e-09	0	\\
2.42e-09	0	\\
2.52e-09	0	\\
2.63e-09	0	\\
2.73e-09	0	\\
2.83e-09	0	\\
2.93e-09	0	\\
3.04e-09	0	\\
3.14e-09	0	\\
3.24e-09	0	\\
3.34e-09	0	\\
3.45e-09	0	\\
3.55e-09	0	\\
3.65e-09	0	\\
3.75e-09	0	\\
3.86e-09	0	\\
3.96e-09	0	\\
4.06e-09	0	\\
4.16e-09	0	\\
4.27e-09	0	\\
4.37e-09	0	\\
4.47e-09	0	\\
4.57e-09	0	\\
4.68e-09	0	\\
4.78e-09	0	\\
4.89e-09	0	\\
4.99e-09	0	\\
5e-09	0	\\
};
\addplot [color=mycolor2,solid,forget plot]
  table[row sep=crcr]{
0	0	\\
1.1e-10	0	\\
2.2e-10	0	\\
3.3e-10	0	\\
4.4e-10	0	\\
5.4e-10	0	\\
6.5e-10	0	\\
7.5e-10	0	\\
8.6e-10	0	\\
9.6e-10	0	\\
1.07e-09	0	\\
1.18e-09	0	\\
1.28e-09	0	\\
1.38e-09	0	\\
1.49e-09	0	\\
1.59e-09	0	\\
1.69e-09	0	\\
1.8e-09	0	\\
1.9e-09	0	\\
2.01e-09	0	\\
2.11e-09	0	\\
2.21e-09	0	\\
2.32e-09	0	\\
2.42e-09	0	\\
2.52e-09	0	\\
2.63e-09	0	\\
2.73e-09	0	\\
2.83e-09	0	\\
2.93e-09	0	\\
3.04e-09	0	\\
3.14e-09	0	\\
3.24e-09	0	\\
3.34e-09	0	\\
3.45e-09	0	\\
3.55e-09	0	\\
3.65e-09	0	\\
3.75e-09	0	\\
3.86e-09	0	\\
3.96e-09	0	\\
4.06e-09	0	\\
4.16e-09	0	\\
4.27e-09	0	\\
4.37e-09	0	\\
4.47e-09	0	\\
4.57e-09	0	\\
4.68e-09	0	\\
4.78e-09	0	\\
4.89e-09	0	\\
4.99e-09	0	\\
5e-09	0	\\
};
\addplot [color=mycolor3,solid,forget plot]
  table[row sep=crcr]{
0	0	\\
1.1e-10	0	\\
2.2e-10	0	\\
3.3e-10	0	\\
4.4e-10	0	\\
5.4e-10	0	\\
6.5e-10	0	\\
7.5e-10	0	\\
8.6e-10	0	\\
9.6e-10	0	\\
1.07e-09	0	\\
1.18e-09	0	\\
1.28e-09	0	\\
1.38e-09	0	\\
1.49e-09	0	\\
1.59e-09	0	\\
1.69e-09	0	\\
1.8e-09	0	\\
1.9e-09	0	\\
2.01e-09	0	\\
2.11e-09	0	\\
2.21e-09	0	\\
2.32e-09	0	\\
2.42e-09	0	\\
2.52e-09	0	\\
2.63e-09	0	\\
2.73e-09	0	\\
2.83e-09	0	\\
2.93e-09	0	\\
3.04e-09	0	\\
3.14e-09	0	\\
3.24e-09	0	\\
3.34e-09	0	\\
3.45e-09	0	\\
3.55e-09	0	\\
3.65e-09	0	\\
3.75e-09	0	\\
3.86e-09	0	\\
3.96e-09	0	\\
4.06e-09	0	\\
4.16e-09	0	\\
4.27e-09	0	\\
4.37e-09	0	\\
4.47e-09	0	\\
4.57e-09	0	\\
4.68e-09	0	\\
4.78e-09	0	\\
4.89e-09	0	\\
4.99e-09	0	\\
5e-09	0	\\
};
\addplot [color=darkgray,solid,forget plot]
  table[row sep=crcr]{
0	0	\\
1.1e-10	0	\\
2.2e-10	0	\\
3.3e-10	0	\\
4.4e-10	0	\\
5.4e-10	0	\\
6.5e-10	0	\\
7.5e-10	0	\\
8.6e-10	0	\\
9.6e-10	0	\\
1.07e-09	0	\\
1.18e-09	0	\\
1.28e-09	0	\\
1.38e-09	0	\\
1.49e-09	0	\\
1.59e-09	0	\\
1.69e-09	0	\\
1.8e-09	0	\\
1.9e-09	0	\\
2.01e-09	0	\\
2.11e-09	0	\\
2.21e-09	0	\\
2.32e-09	0	\\
2.42e-09	0	\\
2.52e-09	0	\\
2.63e-09	0	\\
2.73e-09	0	\\
2.83e-09	0	\\
2.93e-09	0	\\
3.04e-09	0	\\
3.14e-09	0	\\
3.24e-09	0	\\
3.34e-09	0	\\
3.45e-09	0	\\
3.55e-09	0	\\
3.65e-09	0	\\
3.75e-09	0	\\
3.86e-09	0	\\
3.96e-09	0	\\
4.06e-09	0	\\
4.16e-09	0	\\
4.27e-09	0	\\
4.37e-09	0	\\
4.47e-09	0	\\
4.57e-09	0	\\
4.68e-09	0	\\
4.78e-09	0	\\
4.89e-09	0	\\
4.99e-09	0	\\
5e-09	0	\\
};
\addplot [color=blue,solid,forget plot]
  table[row sep=crcr]{
0	0	\\
1.1e-10	0	\\
2.2e-10	0	\\
3.3e-10	0	\\
4.4e-10	0	\\
5.4e-10	0	\\
6.5e-10	0	\\
7.5e-10	0	\\
8.6e-10	0	\\
9.6e-10	0	\\
1.07e-09	0	\\
1.18e-09	0	\\
1.28e-09	0	\\
1.38e-09	0	\\
1.49e-09	0	\\
1.59e-09	0	\\
1.69e-09	0	\\
1.8e-09	0	\\
1.9e-09	0	\\
2.01e-09	0	\\
2.11e-09	0	\\
2.21e-09	0	\\
2.32e-09	0	\\
2.42e-09	0	\\
2.52e-09	0	\\
2.63e-09	0	\\
2.73e-09	0	\\
2.83e-09	0	\\
2.93e-09	0	\\
3.04e-09	0	\\
3.14e-09	0	\\
3.24e-09	0	\\
3.34e-09	0	\\
3.45e-09	0	\\
3.55e-09	0	\\
3.65e-09	0	\\
3.75e-09	0	\\
3.86e-09	0	\\
3.96e-09	0	\\
4.06e-09	0	\\
4.16e-09	0	\\
4.27e-09	0	\\
4.37e-09	0	\\
4.47e-09	0	\\
4.57e-09	0	\\
4.68e-09	0	\\
4.78e-09	0	\\
4.89e-09	0	\\
4.99e-09	0	\\
5e-09	0	\\
};
\addplot [color=black!50!green,solid,forget plot]
  table[row sep=crcr]{
0	0	\\
1.1e-10	0	\\
2.2e-10	0	\\
3.3e-10	0	\\
4.4e-10	0	\\
5.4e-10	0	\\
6.5e-10	0	\\
7.5e-10	0	\\
8.6e-10	0	\\
9.6e-10	0	\\
1.07e-09	0	\\
1.18e-09	0	\\
1.28e-09	0	\\
1.38e-09	0	\\
1.49e-09	0	\\
1.59e-09	0	\\
1.69e-09	0	\\
1.8e-09	0	\\
1.9e-09	0	\\
2.01e-09	0	\\
2.11e-09	0	\\
2.21e-09	0	\\
2.32e-09	0	\\
2.42e-09	0	\\
2.52e-09	0	\\
2.63e-09	0	\\
2.73e-09	0	\\
2.83e-09	0	\\
2.93e-09	0	\\
3.04e-09	0	\\
3.14e-09	0	\\
3.24e-09	0	\\
3.34e-09	0	\\
3.45e-09	0	\\
3.55e-09	0	\\
3.65e-09	0	\\
3.75e-09	0	\\
3.86e-09	0	\\
3.96e-09	0	\\
4.06e-09	0	\\
4.16e-09	0	\\
4.27e-09	0	\\
4.37e-09	0	\\
4.47e-09	0	\\
4.57e-09	0	\\
4.68e-09	0	\\
4.78e-09	0	\\
4.89e-09	0	\\
4.99e-09	0	\\
5e-09	0	\\
};
\addplot [color=red,solid,forget plot]
  table[row sep=crcr]{
0	0	\\
1.1e-10	0	\\
2.2e-10	0	\\
3.3e-10	0	\\
4.4e-10	0	\\
5.4e-10	0	\\
6.5e-10	0	\\
7.5e-10	0	\\
8.6e-10	0	\\
9.6e-10	0	\\
1.07e-09	0	\\
1.18e-09	0	\\
1.28e-09	0	\\
1.38e-09	0	\\
1.49e-09	0	\\
1.59e-09	0	\\
1.69e-09	0	\\
1.8e-09	0	\\
1.9e-09	0	\\
2.01e-09	0	\\
2.11e-09	0	\\
2.21e-09	0	\\
2.32e-09	0	\\
2.42e-09	0	\\
2.52e-09	0	\\
2.63e-09	0	\\
2.73e-09	0	\\
2.83e-09	0	\\
2.93e-09	0	\\
3.04e-09	0	\\
3.14e-09	0	\\
3.24e-09	0	\\
3.34e-09	0	\\
3.45e-09	0	\\
3.55e-09	0	\\
3.65e-09	0	\\
3.75e-09	0	\\
3.86e-09	0	\\
3.96e-09	0	\\
4.06e-09	0	\\
4.16e-09	0	\\
4.27e-09	0	\\
4.37e-09	0	\\
4.47e-09	0	\\
4.57e-09	0	\\
4.68e-09	0	\\
4.78e-09	0	\\
4.89e-09	0	\\
4.99e-09	0	\\
5e-09	0	\\
};
\addplot [color=mycolor1,solid,forget plot]
  table[row sep=crcr]{
0	0	\\
1.1e-10	0	\\
2.2e-10	0	\\
3.3e-10	0	\\
4.4e-10	0	\\
5.4e-10	0	\\
6.5e-10	0	\\
7.5e-10	0	\\
8.6e-10	0	\\
9.6e-10	0	\\
1.07e-09	0	\\
1.18e-09	0	\\
1.28e-09	0	\\
1.38e-09	0	\\
1.49e-09	0	\\
1.59e-09	0	\\
1.69e-09	0	\\
1.8e-09	0	\\
1.9e-09	0	\\
2.01e-09	0	\\
2.11e-09	0	\\
2.21e-09	0	\\
2.32e-09	0	\\
2.42e-09	0	\\
2.52e-09	0	\\
2.63e-09	0	\\
2.73e-09	0	\\
2.83e-09	0	\\
2.93e-09	0	\\
3.04e-09	0	\\
3.14e-09	0	\\
3.24e-09	0	\\
3.34e-09	0	\\
3.45e-09	0	\\
3.55e-09	0	\\
3.65e-09	0	\\
3.75e-09	0	\\
3.86e-09	0	\\
3.96e-09	0	\\
4.06e-09	0	\\
4.16e-09	0	\\
4.27e-09	0	\\
4.37e-09	0	\\
4.47e-09	0	\\
4.57e-09	0	\\
4.68e-09	0	\\
4.78e-09	0	\\
4.89e-09	0	\\
4.99e-09	0	\\
5e-09	0	\\
};
\addplot [color=mycolor2,solid,forget plot]
  table[row sep=crcr]{
0	0	\\
1.1e-10	0	\\
2.2e-10	0	\\
3.3e-10	0	\\
4.4e-10	0	\\
5.4e-10	0	\\
6.5e-10	0	\\
7.5e-10	0	\\
8.6e-10	0	\\
9.6e-10	0	\\
1.07e-09	0	\\
1.18e-09	0	\\
1.28e-09	0	\\
1.38e-09	0	\\
1.49e-09	0	\\
1.59e-09	0	\\
1.69e-09	0	\\
1.8e-09	0	\\
1.9e-09	0	\\
2.01e-09	0	\\
2.11e-09	0	\\
2.21e-09	0	\\
2.32e-09	0	\\
2.42e-09	0	\\
2.52e-09	0	\\
2.63e-09	0	\\
2.73e-09	0	\\
2.83e-09	0	\\
2.93e-09	0	\\
3.04e-09	0	\\
3.14e-09	0	\\
3.24e-09	0	\\
3.34e-09	0	\\
3.45e-09	0	\\
3.55e-09	0	\\
3.65e-09	0	\\
3.75e-09	0	\\
3.86e-09	0	\\
3.96e-09	0	\\
4.06e-09	0	\\
4.16e-09	0	\\
4.27e-09	0	\\
4.37e-09	0	\\
4.47e-09	0	\\
4.57e-09	0	\\
4.68e-09	0	\\
4.78e-09	0	\\
4.89e-09	0	\\
4.99e-09	0	\\
5e-09	0	\\
};
\addplot [color=mycolor3,solid,forget plot]
  table[row sep=crcr]{
0	0	\\
1.1e-10	0	\\
2.2e-10	0	\\
3.3e-10	0	\\
4.4e-10	0	\\
5.4e-10	0	\\
6.5e-10	0	\\
7.5e-10	0	\\
8.6e-10	0	\\
9.6e-10	0	\\
1.07e-09	0	\\
1.18e-09	0	\\
1.28e-09	0	\\
1.38e-09	0	\\
1.49e-09	0	\\
1.59e-09	0	\\
1.69e-09	0	\\
1.8e-09	0	\\
1.9e-09	0	\\
2.01e-09	0	\\
2.11e-09	0	\\
2.21e-09	0	\\
2.32e-09	0	\\
2.42e-09	0	\\
2.52e-09	0	\\
2.63e-09	0	\\
2.73e-09	0	\\
2.83e-09	0	\\
2.93e-09	0	\\
3.04e-09	0	\\
3.14e-09	0	\\
3.24e-09	0	\\
3.34e-09	0	\\
3.45e-09	0	\\
3.55e-09	0	\\
3.65e-09	0	\\
3.75e-09	0	\\
3.86e-09	0	\\
3.96e-09	0	\\
4.06e-09	0	\\
4.16e-09	0	\\
4.27e-09	0	\\
4.37e-09	0	\\
4.47e-09	0	\\
4.57e-09	0	\\
4.68e-09	0	\\
4.78e-09	0	\\
4.89e-09	0	\\
4.99e-09	0	\\
5e-09	0	\\
};
\addplot [color=darkgray,solid,forget plot]
  table[row sep=crcr]{
0	0	\\
1.1e-10	0	\\
2.2e-10	0	\\
3.3e-10	0	\\
4.4e-10	0	\\
5.4e-10	0	\\
6.5e-10	0	\\
7.5e-10	0	\\
8.6e-10	0	\\
9.6e-10	0	\\
1.07e-09	0	\\
1.18e-09	0	\\
1.28e-09	0	\\
1.38e-09	0	\\
1.49e-09	0	\\
1.59e-09	0	\\
1.69e-09	0	\\
1.8e-09	0	\\
1.9e-09	0	\\
2.01e-09	0	\\
2.11e-09	0	\\
2.21e-09	0	\\
2.32e-09	0	\\
2.42e-09	0	\\
2.52e-09	0	\\
2.63e-09	0	\\
2.73e-09	0	\\
2.83e-09	0	\\
2.93e-09	0	\\
3.04e-09	0	\\
3.14e-09	0	\\
3.24e-09	0	\\
3.34e-09	0	\\
3.45e-09	0	\\
3.55e-09	0	\\
3.65e-09	0	\\
3.75e-09	0	\\
3.86e-09	0	\\
3.96e-09	0	\\
4.06e-09	0	\\
4.16e-09	0	\\
4.27e-09	0	\\
4.37e-09	0	\\
4.47e-09	0	\\
4.57e-09	0	\\
4.68e-09	0	\\
4.78e-09	0	\\
4.89e-09	0	\\
4.99e-09	0	\\
5e-09	0	\\
};
\addplot [color=blue,solid,forget plot]
  table[row sep=crcr]{
0	0	\\
1.1e-10	0	\\
2.2e-10	0	\\
3.3e-10	0	\\
4.4e-10	0	\\
5.4e-10	0	\\
6.5e-10	0	\\
7.5e-10	0	\\
8.6e-10	0	\\
9.6e-10	0	\\
1.07e-09	0	\\
1.18e-09	0	\\
1.28e-09	0	\\
1.38e-09	0	\\
1.49e-09	0	\\
1.59e-09	0	\\
1.69e-09	0	\\
1.8e-09	0	\\
1.9e-09	0	\\
2.01e-09	0	\\
2.11e-09	0	\\
2.21e-09	0	\\
2.32e-09	0	\\
2.42e-09	0	\\
2.52e-09	0	\\
2.63e-09	0	\\
2.73e-09	0	\\
2.83e-09	0	\\
2.93e-09	0	\\
3.04e-09	0	\\
3.14e-09	0	\\
3.24e-09	0	\\
3.34e-09	0	\\
3.45e-09	0	\\
3.55e-09	0	\\
3.65e-09	0	\\
3.75e-09	0	\\
3.86e-09	0	\\
3.96e-09	0	\\
4.06e-09	0	\\
4.16e-09	0	\\
4.27e-09	0	\\
4.37e-09	0	\\
4.47e-09	0	\\
4.57e-09	0	\\
4.68e-09	0	\\
4.78e-09	0	\\
4.89e-09	0	\\
4.99e-09	0	\\
5e-09	0	\\
};
\addplot [color=black!50!green,solid,forget plot]
  table[row sep=crcr]{
0	0	\\
1.1e-10	0	\\
2.2e-10	0	\\
3.3e-10	0	\\
4.4e-10	0	\\
5.4e-10	0	\\
6.5e-10	0	\\
7.5e-10	0	\\
8.6e-10	0	\\
9.6e-10	0	\\
1.07e-09	0	\\
1.18e-09	0	\\
1.28e-09	0	\\
1.38e-09	0	\\
1.49e-09	0	\\
1.59e-09	0	\\
1.69e-09	0	\\
1.8e-09	0	\\
1.9e-09	0	\\
2.01e-09	0	\\
2.11e-09	0	\\
2.21e-09	0	\\
2.32e-09	0	\\
2.42e-09	0	\\
2.52e-09	0	\\
2.63e-09	0	\\
2.73e-09	0	\\
2.83e-09	0	\\
2.93e-09	0	\\
3.04e-09	0	\\
3.14e-09	0	\\
3.24e-09	0	\\
3.34e-09	0	\\
3.45e-09	0	\\
3.55e-09	0	\\
3.65e-09	0	\\
3.75e-09	0	\\
3.86e-09	0	\\
3.96e-09	0	\\
4.06e-09	0	\\
4.16e-09	0	\\
4.27e-09	0	\\
4.37e-09	0	\\
4.47e-09	0	\\
4.57e-09	0	\\
4.68e-09	0	\\
4.78e-09	0	\\
4.89e-09	0	\\
4.99e-09	0	\\
5e-09	0	\\
};
\addplot [color=red,solid,forget plot]
  table[row sep=crcr]{
0	0	\\
1.1e-10	0	\\
2.2e-10	0	\\
3.3e-10	0	\\
4.4e-10	0	\\
5.4e-10	0	\\
6.5e-10	0	\\
7.5e-10	0	\\
8.6e-10	0	\\
9.6e-10	0	\\
1.07e-09	0	\\
1.18e-09	0	\\
1.28e-09	0	\\
1.38e-09	0	\\
1.49e-09	0	\\
1.59e-09	0	\\
1.69e-09	0	\\
1.8e-09	0	\\
1.9e-09	0	\\
2.01e-09	0	\\
2.11e-09	0	\\
2.21e-09	0	\\
2.32e-09	0	\\
2.42e-09	0	\\
2.52e-09	0	\\
2.63e-09	0	\\
2.73e-09	0	\\
2.83e-09	0	\\
2.93e-09	0	\\
3.04e-09	0	\\
3.14e-09	0	\\
3.24e-09	0	\\
3.34e-09	0	\\
3.45e-09	0	\\
3.55e-09	0	\\
3.65e-09	0	\\
3.75e-09	0	\\
3.86e-09	0	\\
3.96e-09	0	\\
4.06e-09	0	\\
4.16e-09	0	\\
4.27e-09	0	\\
4.37e-09	0	\\
4.47e-09	0	\\
4.57e-09	0	\\
4.68e-09	0	\\
4.78e-09	0	\\
4.89e-09	0	\\
4.99e-09	0	\\
5e-09	0	\\
};
\addplot [color=mycolor1,solid,forget plot]
  table[row sep=crcr]{
0	0	\\
1.1e-10	0	\\
2.2e-10	0	\\
3.3e-10	0	\\
4.4e-10	0	\\
5.4e-10	0	\\
6.5e-10	0	\\
7.5e-10	0	\\
8.6e-10	0	\\
9.6e-10	0	\\
1.07e-09	0	\\
1.18e-09	0	\\
1.28e-09	0	\\
1.38e-09	0	\\
1.49e-09	0	\\
1.59e-09	0	\\
1.69e-09	0	\\
1.8e-09	0	\\
1.9e-09	0	\\
2.01e-09	0	\\
2.11e-09	0	\\
2.21e-09	0	\\
2.32e-09	0	\\
2.42e-09	0	\\
2.52e-09	0	\\
2.63e-09	0	\\
2.73e-09	0	\\
2.83e-09	0	\\
2.93e-09	0	\\
3.04e-09	0	\\
3.14e-09	0	\\
3.24e-09	0	\\
3.34e-09	0	\\
3.45e-09	0	\\
3.55e-09	0	\\
3.65e-09	0	\\
3.75e-09	0	\\
3.86e-09	0	\\
3.96e-09	0	\\
4.06e-09	0	\\
4.16e-09	0	\\
4.27e-09	0	\\
4.37e-09	0	\\
4.47e-09	0	\\
4.57e-09	0	\\
4.68e-09	0	\\
4.78e-09	0	\\
4.89e-09	0	\\
4.99e-09	0	\\
5e-09	0	\\
};
\addplot [color=mycolor2,solid,forget plot]
  table[row sep=crcr]{
0	0	\\
1.1e-10	0	\\
2.2e-10	0	\\
3.3e-10	0	\\
4.4e-10	0	\\
5.4e-10	0	\\
6.5e-10	0	\\
7.5e-10	0	\\
8.6e-10	0	\\
9.6e-10	0	\\
1.07e-09	0	\\
1.18e-09	0	\\
1.28e-09	0	\\
1.38e-09	0	\\
1.49e-09	0	\\
1.59e-09	0	\\
1.69e-09	0	\\
1.8e-09	0	\\
1.9e-09	0	\\
2.01e-09	0	\\
2.11e-09	0	\\
2.21e-09	0	\\
2.32e-09	0	\\
2.42e-09	0	\\
2.52e-09	0	\\
2.63e-09	0	\\
2.73e-09	0	\\
2.83e-09	0	\\
2.93e-09	0	\\
3.04e-09	0	\\
3.14e-09	0	\\
3.24e-09	0	\\
3.34e-09	0	\\
3.45e-09	0	\\
3.55e-09	0	\\
3.65e-09	0	\\
3.75e-09	0	\\
3.86e-09	0	\\
3.96e-09	0	\\
4.06e-09	0	\\
4.16e-09	0	\\
4.27e-09	0	\\
4.37e-09	0	\\
4.47e-09	0	\\
4.57e-09	0	\\
4.68e-09	0	\\
4.78e-09	0	\\
4.89e-09	0	\\
4.99e-09	0	\\
5e-09	0	\\
};
\addplot [color=mycolor3,solid,forget plot]
  table[row sep=crcr]{
0	0	\\
1.1e-10	0	\\
2.2e-10	0	\\
3.3e-10	0	\\
4.4e-10	0	\\
5.4e-10	0	\\
6.5e-10	0	\\
7.5e-10	0	\\
8.6e-10	0	\\
9.6e-10	0	\\
1.07e-09	0	\\
1.18e-09	0	\\
1.28e-09	0	\\
1.38e-09	0	\\
1.49e-09	0	\\
1.59e-09	0	\\
1.69e-09	0	\\
1.8e-09	0	\\
1.9e-09	0	\\
2.01e-09	0	\\
2.11e-09	0	\\
2.21e-09	0	\\
2.32e-09	0	\\
2.42e-09	0	\\
2.52e-09	0	\\
2.63e-09	0	\\
2.73e-09	0	\\
2.83e-09	0	\\
2.93e-09	0	\\
3.04e-09	0	\\
3.14e-09	0	\\
3.24e-09	0	\\
3.34e-09	0	\\
3.45e-09	0	\\
3.55e-09	0	\\
3.65e-09	0	\\
3.75e-09	0	\\
3.86e-09	0	\\
3.96e-09	0	\\
4.06e-09	0	\\
4.16e-09	0	\\
4.27e-09	0	\\
4.37e-09	0	\\
4.47e-09	0	\\
4.57e-09	0	\\
4.68e-09	0	\\
4.78e-09	0	\\
4.89e-09	0	\\
4.99e-09	0	\\
5e-09	0	\\
};
\addplot [color=darkgray,solid,forget plot]
  table[row sep=crcr]{
0	0	\\
1.1e-10	0	\\
2.2e-10	0	\\
3.3e-10	0	\\
4.4e-10	0	\\
5.4e-10	0	\\
6.5e-10	0	\\
7.5e-10	0	\\
8.6e-10	0	\\
9.6e-10	0	\\
1.07e-09	0	\\
1.18e-09	0	\\
1.28e-09	0	\\
1.38e-09	0	\\
1.49e-09	0	\\
1.59e-09	0	\\
1.69e-09	0	\\
1.8e-09	0	\\
1.9e-09	0	\\
2.01e-09	0	\\
2.11e-09	0	\\
2.21e-09	0	\\
2.32e-09	0	\\
2.42e-09	0	\\
2.52e-09	0	\\
2.63e-09	0	\\
2.73e-09	0	\\
2.83e-09	0	\\
2.93e-09	0	\\
3.04e-09	0	\\
3.14e-09	0	\\
3.24e-09	0	\\
3.34e-09	0	\\
3.45e-09	0	\\
3.55e-09	0	\\
3.65e-09	0	\\
3.75e-09	0	\\
3.86e-09	0	\\
3.96e-09	0	\\
4.06e-09	0	\\
4.16e-09	0	\\
4.27e-09	0	\\
4.37e-09	0	\\
4.47e-09	0	\\
4.57e-09	0	\\
4.68e-09	0	\\
4.78e-09	0	\\
4.89e-09	0	\\
4.99e-09	0	\\
5e-09	0	\\
};
\addplot [color=blue,solid,forget plot]
  table[row sep=crcr]{
0	0	\\
1.1e-10	0	\\
2.2e-10	0	\\
3.3e-10	0	\\
4.4e-10	0	\\
5.4e-10	0	\\
6.5e-10	0	\\
7.5e-10	0	\\
8.6e-10	0	\\
9.6e-10	0	\\
1.07e-09	0	\\
1.18e-09	0	\\
1.28e-09	0	\\
1.38e-09	0	\\
1.49e-09	0	\\
1.59e-09	0	\\
1.69e-09	0	\\
1.8e-09	0	\\
1.9e-09	0	\\
2.01e-09	0	\\
2.11e-09	0	\\
2.21e-09	0	\\
2.32e-09	0	\\
2.42e-09	0	\\
2.52e-09	0	\\
2.63e-09	0	\\
2.73e-09	0	\\
2.83e-09	0	\\
2.93e-09	0	\\
3.04e-09	0	\\
3.14e-09	0	\\
3.24e-09	0	\\
3.34e-09	0	\\
3.45e-09	0	\\
3.55e-09	0	\\
3.65e-09	0	\\
3.75e-09	0	\\
3.86e-09	0	\\
3.96e-09	0	\\
4.06e-09	0	\\
4.16e-09	0	\\
4.27e-09	0	\\
4.37e-09	0	\\
4.47e-09	0	\\
4.57e-09	0	\\
4.68e-09	0	\\
4.78e-09	0	\\
4.89e-09	0	\\
4.99e-09	0	\\
5e-09	0	\\
};
\addplot [color=black!50!green,solid,forget plot]
  table[row sep=crcr]{
0	0	\\
1.1e-10	0	\\
2.2e-10	0	\\
3.3e-10	0	\\
4.4e-10	0	\\
5.4e-10	0	\\
6.5e-10	0	\\
7.5e-10	0	\\
8.6e-10	0	\\
9.6e-10	0	\\
1.07e-09	0	\\
1.18e-09	0	\\
1.28e-09	0	\\
1.38e-09	0	\\
1.49e-09	0	\\
1.59e-09	0	\\
1.69e-09	0	\\
1.8e-09	0	\\
1.9e-09	0	\\
2.01e-09	0	\\
2.11e-09	0	\\
2.21e-09	0	\\
2.32e-09	0	\\
2.42e-09	0	\\
2.52e-09	0	\\
2.63e-09	0	\\
2.73e-09	0	\\
2.83e-09	0	\\
2.93e-09	0	\\
3.04e-09	0	\\
3.14e-09	0	\\
3.24e-09	0	\\
3.34e-09	0	\\
3.45e-09	0	\\
3.55e-09	0	\\
3.65e-09	0	\\
3.75e-09	0	\\
3.86e-09	0	\\
3.96e-09	0	\\
4.06e-09	0	\\
4.16e-09	0	\\
4.27e-09	0	\\
4.37e-09	0	\\
4.47e-09	0	\\
4.57e-09	0	\\
4.68e-09	0	\\
4.78e-09	0	\\
4.89e-09	0	\\
4.99e-09	0	\\
5e-09	0	\\
};
\addplot [color=red,solid,forget plot]
  table[row sep=crcr]{
0	0	\\
1.1e-10	0	\\
2.2e-10	0	\\
3.3e-10	0	\\
4.4e-10	0	\\
5.4e-10	0	\\
6.5e-10	0	\\
7.5e-10	0	\\
8.6e-10	0	\\
9.6e-10	0	\\
1.07e-09	0	\\
1.18e-09	0	\\
1.28e-09	0	\\
1.38e-09	0	\\
1.49e-09	0	\\
1.59e-09	0	\\
1.69e-09	0	\\
1.8e-09	0	\\
1.9e-09	0	\\
2.01e-09	0	\\
2.11e-09	0	\\
2.21e-09	0	\\
2.32e-09	0	\\
2.42e-09	0	\\
2.52e-09	0	\\
2.63e-09	0	\\
2.73e-09	0	\\
2.83e-09	0	\\
2.93e-09	0	\\
3.04e-09	0	\\
3.14e-09	0	\\
3.24e-09	0	\\
3.34e-09	0	\\
3.45e-09	0	\\
3.55e-09	0	\\
3.65e-09	0	\\
3.75e-09	0	\\
3.86e-09	0	\\
3.96e-09	0	\\
4.06e-09	0	\\
4.16e-09	0	\\
4.27e-09	0	\\
4.37e-09	0	\\
4.47e-09	0	\\
4.57e-09	0	\\
4.68e-09	0	\\
4.78e-09	0	\\
4.89e-09	0	\\
4.99e-09	0	\\
5e-09	0	\\
};
\addplot [color=mycolor1,solid,forget plot]
  table[row sep=crcr]{
0	0	\\
1.1e-10	0	\\
2.2e-10	0	\\
3.3e-10	0	\\
4.4e-10	0	\\
5.4e-10	0	\\
6.5e-10	0	\\
7.5e-10	0	\\
8.6e-10	0	\\
9.6e-10	0	\\
1.07e-09	0	\\
1.18e-09	0	\\
1.28e-09	0	\\
1.38e-09	0	\\
1.49e-09	0	\\
1.59e-09	0	\\
1.69e-09	0	\\
1.8e-09	0	\\
1.9e-09	0	\\
2.01e-09	0	\\
2.11e-09	0	\\
2.21e-09	0	\\
2.32e-09	0	\\
2.42e-09	0	\\
2.52e-09	0	\\
2.63e-09	0	\\
2.73e-09	0	\\
2.83e-09	0	\\
2.93e-09	0	\\
3.04e-09	0	\\
3.14e-09	0	\\
3.24e-09	0	\\
3.34e-09	0	\\
3.45e-09	0	\\
3.55e-09	0	\\
3.65e-09	0	\\
3.75e-09	0	\\
3.86e-09	0	\\
3.96e-09	0	\\
4.06e-09	0	\\
4.16e-09	0	\\
4.27e-09	0	\\
4.37e-09	0	\\
4.47e-09	0	\\
4.57e-09	0	\\
4.68e-09	0	\\
4.78e-09	0	\\
4.89e-09	0	\\
4.99e-09	0	\\
5e-09	0	\\
};
\addplot [color=mycolor2,solid,forget plot]
  table[row sep=crcr]{
0	0	\\
1.1e-10	0	\\
2.2e-10	0	\\
3.3e-10	0	\\
4.4e-10	0	\\
5.4e-10	0	\\
6.5e-10	0	\\
7.5e-10	0	\\
8.6e-10	0	\\
9.6e-10	0	\\
1.07e-09	0	\\
1.18e-09	0	\\
1.28e-09	0	\\
1.38e-09	0	\\
1.49e-09	0	\\
1.59e-09	0	\\
1.69e-09	0	\\
1.8e-09	0	\\
1.9e-09	0	\\
2.01e-09	0	\\
2.11e-09	0	\\
2.21e-09	0	\\
2.32e-09	0	\\
2.42e-09	0	\\
2.52e-09	0	\\
2.63e-09	0	\\
2.73e-09	0	\\
2.83e-09	0	\\
2.93e-09	0	\\
3.04e-09	0	\\
3.14e-09	0	\\
3.24e-09	0	\\
3.34e-09	0	\\
3.45e-09	0	\\
3.55e-09	0	\\
3.65e-09	0	\\
3.75e-09	0	\\
3.86e-09	0	\\
3.96e-09	0	\\
4.06e-09	0	\\
4.16e-09	0	\\
4.27e-09	0	\\
4.37e-09	0	\\
4.47e-09	0	\\
4.57e-09	0	\\
4.68e-09	0	\\
4.78e-09	0	\\
4.89e-09	0	\\
4.99e-09	0	\\
5e-09	0	\\
};
\addplot [color=mycolor3,solid,forget plot]
  table[row sep=crcr]{
0	0	\\
1.1e-10	0	\\
2.2e-10	0	\\
3.3e-10	0	\\
4.4e-10	0	\\
5.4e-10	0	\\
6.5e-10	0	\\
7.5e-10	0	\\
8.6e-10	0	\\
9.6e-10	0	\\
1.07e-09	0	\\
1.18e-09	0	\\
1.28e-09	0	\\
1.38e-09	0	\\
1.49e-09	0	\\
1.59e-09	0	\\
1.69e-09	0	\\
1.8e-09	0	\\
1.9e-09	0	\\
2.01e-09	0	\\
2.11e-09	0	\\
2.21e-09	0	\\
2.32e-09	0	\\
2.42e-09	0	\\
2.52e-09	0	\\
2.63e-09	0	\\
2.73e-09	0	\\
2.83e-09	0	\\
2.93e-09	0	\\
3.04e-09	0	\\
3.14e-09	0	\\
3.24e-09	0	\\
3.34e-09	0	\\
3.45e-09	0	\\
3.55e-09	0	\\
3.65e-09	0	\\
3.75e-09	0	\\
3.86e-09	0	\\
3.96e-09	0	\\
4.06e-09	0	\\
4.16e-09	0	\\
4.27e-09	0	\\
4.37e-09	0	\\
4.47e-09	0	\\
4.57e-09	0	\\
4.68e-09	0	\\
4.78e-09	0	\\
4.89e-09	0	\\
4.99e-09	0	\\
5e-09	0	\\
};
\addplot [color=darkgray,solid,forget plot]
  table[row sep=crcr]{
0	0	\\
1.1e-10	0	\\
2.2e-10	0	\\
3.3e-10	0	\\
4.4e-10	0	\\
5.4e-10	0	\\
6.5e-10	0	\\
7.5e-10	0	\\
8.6e-10	0	\\
9.6e-10	0	\\
1.07e-09	0	\\
1.18e-09	0	\\
1.28e-09	0	\\
1.38e-09	0	\\
1.49e-09	0	\\
1.59e-09	0	\\
1.69e-09	0	\\
1.8e-09	0	\\
1.9e-09	0	\\
2.01e-09	0	\\
2.11e-09	0	\\
2.21e-09	0	\\
2.32e-09	0	\\
2.42e-09	0	\\
2.52e-09	0	\\
2.63e-09	0	\\
2.73e-09	0	\\
2.83e-09	0	\\
2.93e-09	0	\\
3.04e-09	0	\\
3.14e-09	0	\\
3.24e-09	0	\\
3.34e-09	0	\\
3.45e-09	0	\\
3.55e-09	0	\\
3.65e-09	0	\\
3.75e-09	0	\\
3.86e-09	0	\\
3.96e-09	0	\\
4.06e-09	0	\\
4.16e-09	0	\\
4.27e-09	0	\\
4.37e-09	0	\\
4.47e-09	0	\\
4.57e-09	0	\\
4.68e-09	0	\\
4.78e-09	0	\\
4.89e-09	0	\\
4.99e-09	0	\\
5e-09	0	\\
};
\addplot [color=blue,solid,forget plot]
  table[row sep=crcr]{
0	0	\\
1.1e-10	0	\\
2.2e-10	0	\\
3.3e-10	0	\\
4.4e-10	0	\\
5.4e-10	0	\\
6.5e-10	0	\\
7.5e-10	0	\\
8.6e-10	0	\\
9.6e-10	0	\\
1.07e-09	0	\\
1.18e-09	0	\\
1.28e-09	0	\\
1.38e-09	0	\\
1.49e-09	0	\\
1.59e-09	0	\\
1.69e-09	0	\\
1.8e-09	0	\\
1.9e-09	0	\\
2.01e-09	0	\\
2.11e-09	0	\\
2.21e-09	0	\\
2.32e-09	0	\\
2.42e-09	0	\\
2.52e-09	0	\\
2.63e-09	0	\\
2.73e-09	0	\\
2.83e-09	0	\\
2.93e-09	0	\\
3.04e-09	0	\\
3.14e-09	0	\\
3.24e-09	0	\\
3.34e-09	0	\\
3.45e-09	0	\\
3.55e-09	0	\\
3.65e-09	0	\\
3.75e-09	0	\\
3.86e-09	0	\\
3.96e-09	0	\\
4.06e-09	0	\\
4.16e-09	0	\\
4.27e-09	0	\\
4.37e-09	0	\\
4.47e-09	0	\\
4.57e-09	0	\\
4.68e-09	0	\\
4.78e-09	0	\\
4.89e-09	0	\\
4.99e-09	0	\\
5e-09	0	\\
};
\addplot [color=black!50!green,solid,forget plot]
  table[row sep=crcr]{
0	0	\\
1.1e-10	0	\\
2.2e-10	0	\\
3.3e-10	0	\\
4.4e-10	0	\\
5.4e-10	0	\\
6.5e-10	0	\\
7.5e-10	0	\\
8.6e-10	0	\\
9.6e-10	0	\\
1.07e-09	0	\\
1.18e-09	0	\\
1.28e-09	0	\\
1.38e-09	0	\\
1.49e-09	0	\\
1.59e-09	0	\\
1.69e-09	0	\\
1.8e-09	0	\\
1.9e-09	0	\\
2.01e-09	0	\\
2.11e-09	0	\\
2.21e-09	0	\\
2.32e-09	0	\\
2.42e-09	0	\\
2.52e-09	0	\\
2.63e-09	0	\\
2.73e-09	0	\\
2.83e-09	0	\\
2.93e-09	0	\\
3.04e-09	0	\\
3.14e-09	0	\\
3.24e-09	0	\\
3.34e-09	0	\\
3.45e-09	0	\\
3.55e-09	0	\\
3.65e-09	0	\\
3.75e-09	0	\\
3.86e-09	0	\\
3.96e-09	0	\\
4.06e-09	0	\\
4.16e-09	0	\\
4.27e-09	0	\\
4.37e-09	0	\\
4.47e-09	0	\\
4.57e-09	0	\\
4.68e-09	0	\\
4.78e-09	0	\\
4.89e-09	0	\\
4.99e-09	0	\\
5e-09	0	\\
};
\addplot [color=red,solid,forget plot]
  table[row sep=crcr]{
0	0	\\
1.1e-10	0	\\
2.2e-10	0	\\
3.3e-10	0	\\
4.4e-10	0	\\
5.4e-10	0	\\
6.5e-10	0	\\
7.5e-10	0	\\
8.6e-10	0	\\
9.6e-10	0	\\
1.07e-09	0	\\
1.18e-09	0	\\
1.28e-09	0	\\
1.38e-09	0	\\
1.49e-09	0	\\
1.59e-09	0	\\
1.69e-09	0	\\
1.8e-09	0	\\
1.9e-09	0	\\
2.01e-09	0	\\
2.11e-09	0	\\
2.21e-09	0	\\
2.32e-09	0	\\
2.42e-09	0	\\
2.52e-09	0	\\
2.63e-09	0	\\
2.73e-09	0	\\
2.83e-09	0	\\
2.93e-09	0	\\
3.04e-09	0	\\
3.14e-09	0	\\
3.24e-09	0	\\
3.34e-09	0	\\
3.45e-09	0	\\
3.55e-09	0	\\
3.65e-09	0	\\
3.75e-09	0	\\
3.86e-09	0	\\
3.96e-09	0	\\
4.06e-09	0	\\
4.16e-09	0	\\
4.27e-09	0	\\
4.37e-09	0	\\
4.47e-09	0	\\
4.57e-09	0	\\
4.68e-09	0	\\
4.78e-09	0	\\
4.89e-09	0	\\
4.99e-09	0	\\
5e-09	0	\\
};
\addplot [color=mycolor1,solid,forget plot]
  table[row sep=crcr]{
0	0	\\
1.1e-10	0	\\
2.2e-10	0	\\
3.3e-10	0	\\
4.4e-10	0	\\
5.4e-10	0	\\
6.5e-10	0	\\
7.5e-10	0	\\
8.6e-10	0	\\
9.6e-10	0	\\
1.07e-09	0	\\
1.18e-09	0	\\
1.28e-09	0	\\
1.38e-09	0	\\
1.49e-09	0	\\
1.59e-09	0	\\
1.69e-09	0	\\
1.8e-09	0	\\
1.9e-09	0	\\
2.01e-09	0	\\
2.11e-09	0	\\
2.21e-09	0	\\
2.32e-09	0	\\
2.42e-09	0	\\
2.52e-09	0	\\
2.63e-09	0	\\
2.73e-09	0	\\
2.83e-09	0	\\
2.93e-09	0	\\
3.04e-09	0	\\
3.14e-09	0	\\
3.24e-09	0	\\
3.34e-09	0	\\
3.45e-09	0	\\
3.55e-09	0	\\
3.65e-09	0	\\
3.75e-09	0	\\
3.86e-09	0	\\
3.96e-09	0	\\
4.06e-09	0	\\
4.16e-09	0	\\
4.27e-09	0	\\
4.37e-09	0	\\
4.47e-09	0	\\
4.57e-09	0	\\
4.68e-09	0	\\
4.78e-09	0	\\
4.89e-09	0	\\
4.99e-09	0	\\
5e-09	0	\\
};
\addplot [color=mycolor2,solid,forget plot]
  table[row sep=crcr]{
0	0	\\
1.1e-10	0	\\
2.2e-10	0	\\
3.3e-10	0	\\
4.4e-10	0	\\
5.4e-10	0	\\
6.5e-10	0	\\
7.5e-10	0	\\
8.6e-10	0	\\
9.6e-10	0	\\
1.07e-09	0	\\
1.18e-09	0	\\
1.28e-09	0	\\
1.38e-09	0	\\
1.49e-09	0	\\
1.59e-09	0	\\
1.69e-09	0	\\
1.8e-09	0	\\
1.9e-09	0	\\
2.01e-09	0	\\
2.11e-09	0	\\
2.21e-09	0	\\
2.32e-09	0	\\
2.42e-09	0	\\
2.52e-09	0	\\
2.63e-09	0	\\
2.73e-09	0	\\
2.83e-09	0	\\
2.93e-09	0	\\
3.04e-09	0	\\
3.14e-09	0	\\
3.24e-09	0	\\
3.34e-09	0	\\
3.45e-09	0	\\
3.55e-09	0	\\
3.65e-09	0	\\
3.75e-09	0	\\
3.86e-09	0	\\
3.96e-09	0	\\
4.06e-09	0	\\
4.16e-09	0	\\
4.27e-09	0	\\
4.37e-09	0	\\
4.47e-09	0	\\
4.57e-09	0	\\
4.68e-09	0	\\
4.78e-09	0	\\
4.89e-09	0	\\
4.99e-09	0	\\
5e-09	0	\\
};
\addplot [color=mycolor3,solid,forget plot]
  table[row sep=crcr]{
0	0	\\
1.1e-10	0	\\
2.2e-10	0	\\
3.3e-10	0	\\
4.4e-10	0	\\
5.4e-10	0	\\
6.5e-10	0	\\
7.5e-10	0	\\
8.6e-10	0	\\
9.6e-10	0	\\
1.07e-09	0	\\
1.18e-09	0	\\
1.28e-09	0	\\
1.38e-09	0	\\
1.49e-09	0	\\
1.59e-09	0	\\
1.69e-09	0	\\
1.8e-09	0	\\
1.9e-09	0	\\
2.01e-09	0	\\
2.11e-09	0	\\
2.21e-09	0	\\
2.32e-09	0	\\
2.42e-09	0	\\
2.52e-09	0	\\
2.63e-09	0	\\
2.73e-09	0	\\
2.83e-09	0	\\
2.93e-09	0	\\
3.04e-09	0	\\
3.14e-09	0	\\
3.24e-09	0	\\
3.34e-09	0	\\
3.45e-09	0	\\
3.55e-09	0	\\
3.65e-09	0	\\
3.75e-09	0	\\
3.86e-09	0	\\
3.96e-09	0	\\
4.06e-09	0	\\
4.16e-09	0	\\
4.27e-09	0	\\
4.37e-09	0	\\
4.47e-09	0	\\
4.57e-09	0	\\
4.68e-09	0	\\
4.78e-09	0	\\
4.89e-09	0	\\
4.99e-09	0	\\
5e-09	0	\\
};
\addplot [color=darkgray,solid,forget plot]
  table[row sep=crcr]{
0	0	\\
1.1e-10	0	\\
2.2e-10	0	\\
3.3e-10	0	\\
4.4e-10	0	\\
5.4e-10	0	\\
6.5e-10	0	\\
7.5e-10	0	\\
8.6e-10	0	\\
9.6e-10	0	\\
1.07e-09	0	\\
1.18e-09	0	\\
1.28e-09	0	\\
1.38e-09	0	\\
1.49e-09	0	\\
1.59e-09	0	\\
1.69e-09	0	\\
1.8e-09	0	\\
1.9e-09	0	\\
2.01e-09	0	\\
2.11e-09	0	\\
2.21e-09	0	\\
2.32e-09	0	\\
2.42e-09	0	\\
2.52e-09	0	\\
2.63e-09	0	\\
2.73e-09	0	\\
2.83e-09	0	\\
2.93e-09	0	\\
3.04e-09	0	\\
3.14e-09	0	\\
3.24e-09	0	\\
3.34e-09	0	\\
3.45e-09	0	\\
3.55e-09	0	\\
3.65e-09	0	\\
3.75e-09	0	\\
3.86e-09	0	\\
3.96e-09	0	\\
4.06e-09	0	\\
4.16e-09	0	\\
4.27e-09	0	\\
4.37e-09	0	\\
4.47e-09	0	\\
4.57e-09	0	\\
4.68e-09	0	\\
4.78e-09	0	\\
4.89e-09	0	\\
4.99e-09	0	\\
5e-09	0	\\
};
\addplot [color=blue,solid,forget plot]
  table[row sep=crcr]{
0	0	\\
1.1e-10	0	\\
2.2e-10	0	\\
3.3e-10	0	\\
4.4e-10	0	\\
5.4e-10	0	\\
6.5e-10	0	\\
7.5e-10	0	\\
8.6e-10	0	\\
9.6e-10	0	\\
1.07e-09	0	\\
1.18e-09	0	\\
1.28e-09	0	\\
1.38e-09	0	\\
1.49e-09	0	\\
1.59e-09	0	\\
1.69e-09	0	\\
1.8e-09	0	\\
1.9e-09	0	\\
2.01e-09	0	\\
2.11e-09	0	\\
2.21e-09	0	\\
2.32e-09	0	\\
2.42e-09	0	\\
2.52e-09	0	\\
2.63e-09	0	\\
2.73e-09	0	\\
2.83e-09	0	\\
2.93e-09	0	\\
3.04e-09	0	\\
3.14e-09	0	\\
3.24e-09	0	\\
3.34e-09	0	\\
3.45e-09	0	\\
3.55e-09	0	\\
3.65e-09	0	\\
3.75e-09	0	\\
3.86e-09	0	\\
3.96e-09	0	\\
4.06e-09	0	\\
4.16e-09	0	\\
4.27e-09	0	\\
4.37e-09	0	\\
4.47e-09	0	\\
4.57e-09	0	\\
4.68e-09	0	\\
4.78e-09	0	\\
4.89e-09	0	\\
4.99e-09	0	\\
5e-09	0	\\
};
\addplot [color=black!50!green,solid,forget plot]
  table[row sep=crcr]{
0	0	\\
1.1e-10	0	\\
2.2e-10	0	\\
3.3e-10	0	\\
4.4e-10	0	\\
5.4e-10	0	\\
6.5e-10	0	\\
7.5e-10	0	\\
8.6e-10	0	\\
9.6e-10	0	\\
1.07e-09	0	\\
1.18e-09	0	\\
1.28e-09	0	\\
1.38e-09	0	\\
1.49e-09	0	\\
1.59e-09	0	\\
1.69e-09	0	\\
1.8e-09	0	\\
1.9e-09	0	\\
2.01e-09	0	\\
2.11e-09	0	\\
2.21e-09	0	\\
2.32e-09	0	\\
2.42e-09	0	\\
2.52e-09	0	\\
2.63e-09	0	\\
2.73e-09	0	\\
2.83e-09	0	\\
2.93e-09	0	\\
3.04e-09	0	\\
3.14e-09	0	\\
3.24e-09	0	\\
3.34e-09	0	\\
3.45e-09	0	\\
3.55e-09	0	\\
3.65e-09	0	\\
3.75e-09	0	\\
3.86e-09	0	\\
3.96e-09	0	\\
4.06e-09	0	\\
4.16e-09	0	\\
4.27e-09	0	\\
4.37e-09	0	\\
4.47e-09	0	\\
4.57e-09	0	\\
4.68e-09	0	\\
4.78e-09	0	\\
4.89e-09	0	\\
4.99e-09	0	\\
5e-09	0	\\
};
\addplot [color=red,solid,forget plot]
  table[row sep=crcr]{
0	0	\\
1.1e-10	0	\\
2.2e-10	0	\\
3.3e-10	0	\\
4.4e-10	0	\\
5.4e-10	0	\\
6.5e-10	0	\\
7.5e-10	0	\\
8.6e-10	0	\\
9.6e-10	0	\\
1.07e-09	0	\\
1.18e-09	0	\\
1.28e-09	0	\\
1.38e-09	0	\\
1.49e-09	0	\\
1.59e-09	0	\\
1.69e-09	0	\\
1.8e-09	0	\\
1.9e-09	0	\\
2.01e-09	0	\\
2.11e-09	0	\\
2.21e-09	0	\\
2.32e-09	0	\\
2.42e-09	0	\\
2.52e-09	0	\\
2.63e-09	0	\\
2.73e-09	0	\\
2.83e-09	0	\\
2.93e-09	0	\\
3.04e-09	0	\\
3.14e-09	0	\\
3.24e-09	0	\\
3.34e-09	0	\\
3.45e-09	0	\\
3.55e-09	0	\\
3.65e-09	0	\\
3.75e-09	0	\\
3.86e-09	0	\\
3.96e-09	0	\\
4.06e-09	0	\\
4.16e-09	0	\\
4.27e-09	0	\\
4.37e-09	0	\\
4.47e-09	0	\\
4.57e-09	0	\\
4.68e-09	0	\\
4.78e-09	0	\\
4.89e-09	0	\\
4.99e-09	0	\\
5e-09	0	\\
};
\addplot [color=mycolor1,solid,forget plot]
  table[row sep=crcr]{
0	0	\\
1.1e-10	0	\\
2.2e-10	0	\\
3.3e-10	0	\\
4.4e-10	0	\\
5.4e-10	0	\\
6.5e-10	0	\\
7.5e-10	0	\\
8.6e-10	0	\\
9.6e-10	0	\\
1.07e-09	0	\\
1.18e-09	0	\\
1.28e-09	0	\\
1.38e-09	0	\\
1.49e-09	0	\\
1.59e-09	0	\\
1.69e-09	0	\\
1.8e-09	0	\\
1.9e-09	0	\\
2.01e-09	0	\\
2.11e-09	0	\\
2.21e-09	0	\\
2.32e-09	0	\\
2.42e-09	0	\\
2.52e-09	0	\\
2.63e-09	0	\\
2.73e-09	0	\\
2.83e-09	0	\\
2.93e-09	0	\\
3.04e-09	0	\\
3.14e-09	0	\\
3.24e-09	0	\\
3.34e-09	0	\\
3.45e-09	0	\\
3.55e-09	0	\\
3.65e-09	0	\\
3.75e-09	0	\\
3.86e-09	0	\\
3.96e-09	0	\\
4.06e-09	0	\\
4.16e-09	0	\\
4.27e-09	0	\\
4.37e-09	0	\\
4.47e-09	0	\\
4.57e-09	0	\\
4.68e-09	0	\\
4.78e-09	0	\\
4.89e-09	0	\\
4.99e-09	0	\\
5e-09	0	\\
};
\addplot [color=mycolor2,solid,forget plot]
  table[row sep=crcr]{
0	0	\\
1.1e-10	0	\\
2.2e-10	0	\\
3.3e-10	0	\\
4.4e-10	0	\\
5.4e-10	0	\\
6.5e-10	0	\\
7.5e-10	0	\\
8.6e-10	0	\\
9.6e-10	0	\\
1.07e-09	0	\\
1.18e-09	0	\\
1.28e-09	0	\\
1.38e-09	0	\\
1.49e-09	0	\\
1.59e-09	0	\\
1.69e-09	0	\\
1.8e-09	0	\\
1.9e-09	0	\\
2.01e-09	0	\\
2.11e-09	0	\\
2.21e-09	0	\\
2.32e-09	0	\\
2.42e-09	0	\\
2.52e-09	0	\\
2.63e-09	0	\\
2.73e-09	0	\\
2.83e-09	0	\\
2.93e-09	0	\\
3.04e-09	0	\\
3.14e-09	0	\\
3.24e-09	0	\\
3.34e-09	0	\\
3.45e-09	0	\\
3.55e-09	0	\\
3.65e-09	0	\\
3.75e-09	0	\\
3.86e-09	0	\\
3.96e-09	0	\\
4.06e-09	0	\\
4.16e-09	0	\\
4.27e-09	0	\\
4.37e-09	0	\\
4.47e-09	0	\\
4.57e-09	0	\\
4.68e-09	0	\\
4.78e-09	0	\\
4.89e-09	0	\\
4.99e-09	0	\\
5e-09	0	\\
};
\addplot [color=mycolor3,solid,forget plot]
  table[row sep=crcr]{
0	0	\\
1.1e-10	0	\\
2.2e-10	0	\\
3.3e-10	0	\\
4.4e-10	0	\\
5.4e-10	0	\\
6.5e-10	0	\\
7.5e-10	0	\\
8.6e-10	0	\\
9.6e-10	0	\\
1.07e-09	0	\\
1.18e-09	0	\\
1.28e-09	0	\\
1.38e-09	0	\\
1.49e-09	0	\\
1.59e-09	0	\\
1.69e-09	0	\\
1.8e-09	0	\\
1.9e-09	0	\\
2.01e-09	0	\\
2.11e-09	0	\\
2.21e-09	0	\\
2.32e-09	0	\\
2.42e-09	0	\\
2.52e-09	0	\\
2.63e-09	0	\\
2.73e-09	0	\\
2.83e-09	0	\\
2.93e-09	0	\\
3.04e-09	0	\\
3.14e-09	0	\\
3.24e-09	0	\\
3.34e-09	0	\\
3.45e-09	0	\\
3.55e-09	0	\\
3.65e-09	0	\\
3.75e-09	0	\\
3.86e-09	0	\\
3.96e-09	0	\\
4.06e-09	0	\\
4.16e-09	0	\\
4.27e-09	0	\\
4.37e-09	0	\\
4.47e-09	0	\\
4.57e-09	0	\\
4.68e-09	0	\\
4.78e-09	0	\\
4.89e-09	0	\\
4.99e-09	0	\\
5e-09	0	\\
};
\addplot [color=darkgray,solid,forget plot]
  table[row sep=crcr]{
0	0	\\
1.1e-10	0	\\
2.2e-10	0	\\
3.3e-10	0	\\
4.4e-10	0	\\
5.4e-10	0	\\
6.5e-10	0	\\
7.5e-10	0	\\
8.6e-10	0	\\
9.6e-10	0	\\
1.07e-09	0	\\
1.18e-09	0	\\
1.28e-09	0	\\
1.38e-09	0	\\
1.49e-09	0	\\
1.59e-09	0	\\
1.69e-09	0	\\
1.8e-09	0	\\
1.9e-09	0	\\
2.01e-09	0	\\
2.11e-09	0	\\
2.21e-09	0	\\
2.32e-09	0	\\
2.42e-09	0	\\
2.52e-09	0	\\
2.63e-09	0	\\
2.73e-09	0	\\
2.83e-09	0	\\
2.93e-09	0	\\
3.04e-09	0	\\
3.14e-09	0	\\
3.24e-09	0	\\
3.34e-09	0	\\
3.45e-09	0	\\
3.55e-09	0	\\
3.65e-09	0	\\
3.75e-09	0	\\
3.86e-09	0	\\
3.96e-09	0	\\
4.06e-09	0	\\
4.16e-09	0	\\
4.27e-09	0	\\
4.37e-09	0	\\
4.47e-09	0	\\
4.57e-09	0	\\
4.68e-09	0	\\
4.78e-09	0	\\
4.89e-09	0	\\
4.99e-09	0	\\
5e-09	0	\\
};
\addplot [color=blue,solid,forget plot]
  table[row sep=crcr]{
0	0	\\
1.1e-10	0	\\
2.2e-10	0	\\
3.3e-10	0	\\
4.4e-10	0	\\
5.4e-10	0	\\
6.5e-10	0	\\
7.5e-10	0	\\
8.6e-10	0	\\
9.6e-10	0	\\
1.07e-09	0	\\
1.18e-09	0	\\
1.28e-09	0	\\
1.38e-09	0	\\
1.49e-09	0	\\
1.59e-09	0	\\
1.69e-09	0	\\
1.8e-09	0	\\
1.9e-09	0	\\
2.01e-09	0	\\
2.11e-09	0	\\
2.21e-09	0	\\
2.32e-09	0	\\
2.42e-09	0	\\
2.52e-09	0	\\
2.63e-09	0	\\
2.73e-09	0	\\
2.83e-09	0	\\
2.93e-09	0	\\
3.04e-09	0	\\
3.14e-09	0	\\
3.24e-09	0	\\
3.34e-09	0	\\
3.45e-09	0	\\
3.55e-09	0	\\
3.65e-09	0	\\
3.75e-09	0	\\
3.86e-09	0	\\
3.96e-09	0	\\
4.06e-09	0	\\
4.16e-09	0	\\
4.27e-09	0	\\
4.37e-09	0	\\
4.47e-09	0	\\
4.57e-09	0	\\
4.68e-09	0	\\
4.78e-09	0	\\
4.89e-09	0	\\
4.99e-09	0	\\
5e-09	0	\\
};
\addplot [color=black!50!green,solid,forget plot]
  table[row sep=crcr]{
0	0	\\
1.1e-10	0	\\
2.2e-10	0	\\
3.3e-10	0	\\
4.4e-10	0	\\
5.4e-10	0	\\
6.5e-10	0	\\
7.5e-10	0	\\
8.6e-10	0	\\
9.6e-10	0	\\
1.07e-09	0	\\
1.18e-09	0	\\
1.28e-09	0	\\
1.38e-09	0	\\
1.49e-09	0	\\
1.59e-09	0	\\
1.69e-09	0	\\
1.8e-09	0	\\
1.9e-09	0	\\
2.01e-09	0	\\
2.11e-09	0	\\
2.21e-09	0	\\
2.32e-09	0	\\
2.42e-09	0	\\
2.52e-09	0	\\
2.63e-09	0	\\
2.73e-09	0	\\
2.83e-09	0	\\
2.93e-09	0	\\
3.04e-09	0	\\
3.14e-09	0	\\
3.24e-09	0	\\
3.34e-09	0	\\
3.45e-09	0	\\
3.55e-09	0	\\
3.65e-09	0	\\
3.75e-09	0	\\
3.86e-09	0	\\
3.96e-09	0	\\
4.06e-09	0	\\
4.16e-09	0	\\
4.27e-09	0	\\
4.37e-09	0	\\
4.47e-09	0	\\
4.57e-09	0	\\
4.68e-09	0	\\
4.78e-09	0	\\
4.89e-09	0	\\
4.99e-09	0	\\
5e-09	0	\\
};
\addplot [color=red,solid,forget plot]
  table[row sep=crcr]{
0	0	\\
1.1e-10	0	\\
2.2e-10	0	\\
3.3e-10	0	\\
4.4e-10	0	\\
5.4e-10	0	\\
6.5e-10	0	\\
7.5e-10	0	\\
8.6e-10	0	\\
9.6e-10	0	\\
1.07e-09	0	\\
1.18e-09	0	\\
1.28e-09	0	\\
1.38e-09	0	\\
1.49e-09	0	\\
1.59e-09	0	\\
1.69e-09	0	\\
1.8e-09	0	\\
1.9e-09	0	\\
2.01e-09	0	\\
2.11e-09	0	\\
2.21e-09	0	\\
2.32e-09	0	\\
2.42e-09	0	\\
2.52e-09	0	\\
2.63e-09	0	\\
2.73e-09	0	\\
2.83e-09	0	\\
2.93e-09	0	\\
3.04e-09	0	\\
3.14e-09	0	\\
3.24e-09	0	\\
3.34e-09	0	\\
3.45e-09	0	\\
3.55e-09	0	\\
3.65e-09	0	\\
3.75e-09	0	\\
3.86e-09	0	\\
3.96e-09	0	\\
4.06e-09	0	\\
4.16e-09	0	\\
4.27e-09	0	\\
4.37e-09	0	\\
4.47e-09	0	\\
4.57e-09	0	\\
4.68e-09	0	\\
4.78e-09	0	\\
4.89e-09	0	\\
4.99e-09	0	\\
5e-09	0	\\
};
\addplot [color=mycolor1,solid,forget plot]
  table[row sep=crcr]{
0	0	\\
1.1e-10	0	\\
2.2e-10	0	\\
3.3e-10	0	\\
4.4e-10	0	\\
5.4e-10	0	\\
6.5e-10	0	\\
7.5e-10	0	\\
8.6e-10	0	\\
9.6e-10	0	\\
1.07e-09	0	\\
1.18e-09	0	\\
1.28e-09	0	\\
1.38e-09	0	\\
1.49e-09	0	\\
1.59e-09	0	\\
1.69e-09	0	\\
1.8e-09	0	\\
1.9e-09	0	\\
2.01e-09	0	\\
2.11e-09	0	\\
2.21e-09	0	\\
2.32e-09	0	\\
2.42e-09	0	\\
2.52e-09	0	\\
2.63e-09	0	\\
2.73e-09	0	\\
2.83e-09	0	\\
2.93e-09	0	\\
3.04e-09	0	\\
3.14e-09	0	\\
3.24e-09	0	\\
3.34e-09	0	\\
3.45e-09	0	\\
3.55e-09	0	\\
3.65e-09	0	\\
3.75e-09	0	\\
3.86e-09	0	\\
3.96e-09	0	\\
4.06e-09	0	\\
4.16e-09	0	\\
4.27e-09	0	\\
4.37e-09	0	\\
4.47e-09	0	\\
4.57e-09	0	\\
4.68e-09	0	\\
4.78e-09	0	\\
4.89e-09	0	\\
4.99e-09	0	\\
5e-09	0	\\
};
\addplot [color=mycolor2,solid,forget plot]
  table[row sep=crcr]{
0	0	\\
1.1e-10	0	\\
2.2e-10	0	\\
3.3e-10	0	\\
4.4e-10	0	\\
5.4e-10	0	\\
6.5e-10	0	\\
7.5e-10	0	\\
8.6e-10	0	\\
9.6e-10	0	\\
1.07e-09	0	\\
1.18e-09	0	\\
1.28e-09	0	\\
1.38e-09	0	\\
1.49e-09	0	\\
1.59e-09	0	\\
1.69e-09	0	\\
1.8e-09	0	\\
1.9e-09	0	\\
2.01e-09	0	\\
2.11e-09	0	\\
2.21e-09	0	\\
2.32e-09	0	\\
2.42e-09	0	\\
2.52e-09	0	\\
2.63e-09	0	\\
2.73e-09	0	\\
2.83e-09	0	\\
2.93e-09	0	\\
3.04e-09	0	\\
3.14e-09	0	\\
3.24e-09	0	\\
3.34e-09	0	\\
3.45e-09	0	\\
3.55e-09	0	\\
3.65e-09	0	\\
3.75e-09	0	\\
3.86e-09	0	\\
3.96e-09	0	\\
4.06e-09	0	\\
4.16e-09	0	\\
4.27e-09	0	\\
4.37e-09	0	\\
4.47e-09	0	\\
4.57e-09	0	\\
4.68e-09	0	\\
4.78e-09	0	\\
4.89e-09	0	\\
4.99e-09	0	\\
5e-09	0	\\
};
\addplot [color=mycolor3,solid,forget plot]
  table[row sep=crcr]{
0	0	\\
1.1e-10	0	\\
2.2e-10	0	\\
3.3e-10	0	\\
4.4e-10	0	\\
5.4e-10	0	\\
6.5e-10	0	\\
7.5e-10	0	\\
8.6e-10	0	\\
9.6e-10	0	\\
1.07e-09	0	\\
1.18e-09	0	\\
1.28e-09	0	\\
1.38e-09	0	\\
1.49e-09	0	\\
1.59e-09	0	\\
1.69e-09	0	\\
1.8e-09	0	\\
1.9e-09	0	\\
2.01e-09	0	\\
2.11e-09	0	\\
2.21e-09	0	\\
2.32e-09	0	\\
2.42e-09	0	\\
2.52e-09	0	\\
2.63e-09	0	\\
2.73e-09	0	\\
2.83e-09	0	\\
2.93e-09	0	\\
3.04e-09	0	\\
3.14e-09	0	\\
3.24e-09	0	\\
3.34e-09	0	\\
3.45e-09	0	\\
3.55e-09	0	\\
3.65e-09	0	\\
3.75e-09	0	\\
3.86e-09	0	\\
3.96e-09	0	\\
4.06e-09	0	\\
4.16e-09	0	\\
4.27e-09	0	\\
4.37e-09	0	\\
4.47e-09	0	\\
4.57e-09	0	\\
4.68e-09	0	\\
4.78e-09	0	\\
4.89e-09	0	\\
4.99e-09	0	\\
5e-09	0	\\
};
\addplot [color=darkgray,solid,forget plot]
  table[row sep=crcr]{
0	0	\\
1.1e-10	0	\\
2.2e-10	0	\\
3.3e-10	0	\\
4.4e-10	0	\\
5.4e-10	0	\\
6.5e-10	0	\\
7.5e-10	0	\\
8.6e-10	0	\\
9.6e-10	0	\\
1.07e-09	0	\\
1.18e-09	0	\\
1.28e-09	0	\\
1.38e-09	0	\\
1.49e-09	0	\\
1.59e-09	0	\\
1.69e-09	0	\\
1.8e-09	0	\\
1.9e-09	0	\\
2.01e-09	0	\\
2.11e-09	0	\\
2.21e-09	0	\\
2.32e-09	0	\\
2.42e-09	0	\\
2.52e-09	0	\\
2.63e-09	0	\\
2.73e-09	0	\\
2.83e-09	0	\\
2.93e-09	0	\\
3.04e-09	0	\\
3.14e-09	0	\\
3.24e-09	0	\\
3.34e-09	0	\\
3.45e-09	0	\\
3.55e-09	0	\\
3.65e-09	0	\\
3.75e-09	0	\\
3.86e-09	0	\\
3.96e-09	0	\\
4.06e-09	0	\\
4.16e-09	0	\\
4.27e-09	0	\\
4.37e-09	0	\\
4.47e-09	0	\\
4.57e-09	0	\\
4.68e-09	0	\\
4.78e-09	0	\\
4.89e-09	0	\\
4.99e-09	0	\\
5e-09	0	\\
};
\addplot [color=blue,solid,forget plot]
  table[row sep=crcr]{
0	0	\\
1.1e-10	0	\\
2.2e-10	0	\\
3.3e-10	0	\\
4.4e-10	0	\\
5.4e-10	0	\\
6.5e-10	0	\\
7.5e-10	0	\\
8.6e-10	0	\\
9.6e-10	0	\\
1.07e-09	0	\\
1.18e-09	0	\\
1.28e-09	0	\\
1.38e-09	0	\\
1.49e-09	0	\\
1.59e-09	0	\\
1.69e-09	0	\\
1.8e-09	0	\\
1.9e-09	0	\\
2.01e-09	0	\\
2.11e-09	0	\\
2.21e-09	0	\\
2.32e-09	0	\\
2.42e-09	0	\\
2.52e-09	0	\\
2.63e-09	0	\\
2.73e-09	0	\\
2.83e-09	0	\\
2.93e-09	0	\\
3.04e-09	0	\\
3.14e-09	0	\\
3.24e-09	0	\\
3.34e-09	0	\\
3.45e-09	0	\\
3.55e-09	0	\\
3.65e-09	0	\\
3.75e-09	0	\\
3.86e-09	0	\\
3.96e-09	0	\\
4.06e-09	0	\\
4.16e-09	0	\\
4.27e-09	0	\\
4.37e-09	0	\\
4.47e-09	0	\\
4.57e-09	0	\\
4.68e-09	0	\\
4.78e-09	0	\\
4.89e-09	0	\\
4.99e-09	0	\\
5e-09	0	\\
};
\addplot [color=black!50!green,solid,forget plot]
  table[row sep=crcr]{
0	0	\\
1.1e-10	0	\\
2.2e-10	0	\\
3.3e-10	0	\\
4.4e-10	0	\\
5.4e-10	0	\\
6.5e-10	0	\\
7.5e-10	0	\\
8.6e-10	0	\\
9.6e-10	0	\\
1.07e-09	0	\\
1.18e-09	0	\\
1.28e-09	0	\\
1.38e-09	0	\\
1.49e-09	0	\\
1.59e-09	0	\\
1.69e-09	0	\\
1.8e-09	0	\\
1.9e-09	0	\\
2.01e-09	0	\\
2.11e-09	0	\\
2.21e-09	0	\\
2.32e-09	0	\\
2.42e-09	0	\\
2.52e-09	0	\\
2.63e-09	0	\\
2.73e-09	0	\\
2.83e-09	0	\\
2.93e-09	0	\\
3.04e-09	0	\\
3.14e-09	0	\\
3.24e-09	0	\\
3.34e-09	0	\\
3.45e-09	0	\\
3.55e-09	0	\\
3.65e-09	0	\\
3.75e-09	0	\\
3.86e-09	0	\\
3.96e-09	0	\\
4.06e-09	0	\\
4.16e-09	0	\\
4.27e-09	0	\\
4.37e-09	0	\\
4.47e-09	0	\\
4.57e-09	0	\\
4.68e-09	0	\\
4.78e-09	0	\\
4.89e-09	0	\\
4.99e-09	0	\\
5e-09	0	\\
};
\addplot [color=red,solid,forget plot]
  table[row sep=crcr]{
0	0	\\
1.1e-10	0	\\
2.2e-10	0	\\
3.3e-10	0	\\
4.4e-10	0	\\
5.4e-10	0	\\
6.5e-10	0	\\
7.5e-10	0	\\
8.6e-10	0	\\
9.6e-10	0	\\
1.07e-09	0	\\
1.18e-09	0	\\
1.28e-09	0	\\
1.38e-09	0	\\
1.49e-09	0	\\
1.59e-09	0	\\
1.69e-09	0	\\
1.8e-09	0	\\
1.9e-09	0	\\
2.01e-09	0	\\
2.11e-09	0	\\
2.21e-09	0	\\
2.32e-09	0	\\
2.42e-09	0	\\
2.52e-09	0	\\
2.63e-09	0	\\
2.73e-09	0	\\
2.83e-09	0	\\
2.93e-09	0	\\
3.04e-09	0	\\
3.14e-09	0	\\
3.24e-09	0	\\
3.34e-09	0	\\
3.45e-09	0	\\
3.55e-09	0	\\
3.65e-09	0	\\
3.75e-09	0	\\
3.86e-09	0	\\
3.96e-09	0	\\
4.06e-09	0	\\
4.16e-09	0	\\
4.27e-09	0	\\
4.37e-09	0	\\
4.47e-09	0	\\
4.57e-09	0	\\
4.68e-09	0	\\
4.78e-09	0	\\
4.89e-09	0	\\
4.99e-09	0	\\
5e-09	0	\\
};
\addplot [color=mycolor1,solid,forget plot]
  table[row sep=crcr]{
0	0	\\
1.1e-10	0	\\
2.2e-10	0	\\
3.3e-10	0	\\
4.4e-10	0	\\
5.4e-10	0	\\
6.5e-10	0	\\
7.5e-10	0	\\
8.6e-10	0	\\
9.6e-10	0	\\
1.07e-09	0	\\
1.18e-09	0	\\
1.28e-09	0	\\
1.38e-09	0	\\
1.49e-09	0	\\
1.59e-09	0	\\
1.69e-09	0	\\
1.8e-09	0	\\
1.9e-09	0	\\
2.01e-09	0	\\
2.11e-09	0	\\
2.21e-09	0	\\
2.32e-09	0	\\
2.42e-09	0	\\
2.52e-09	0	\\
2.63e-09	0	\\
2.73e-09	0	\\
2.83e-09	0	\\
2.93e-09	0	\\
3.04e-09	0	\\
3.14e-09	0	\\
3.24e-09	0	\\
3.34e-09	0	\\
3.45e-09	0	\\
3.55e-09	0	\\
3.65e-09	0	\\
3.75e-09	0	\\
3.86e-09	0	\\
3.96e-09	0	\\
4.06e-09	0	\\
4.16e-09	0	\\
4.27e-09	0	\\
4.37e-09	0	\\
4.47e-09	0	\\
4.57e-09	0	\\
4.68e-09	0	\\
4.78e-09	0	\\
4.89e-09	0	\\
4.99e-09	0	\\
5e-09	0	\\
};
\addplot [color=mycolor2,solid,forget plot]
  table[row sep=crcr]{
0	0	\\
1.1e-10	0	\\
2.2e-10	0	\\
3.3e-10	0	\\
4.4e-10	0	\\
5.4e-10	0	\\
6.5e-10	0	\\
7.5e-10	0	\\
8.6e-10	0	\\
9.6e-10	0	\\
1.07e-09	0	\\
1.18e-09	0	\\
1.28e-09	0	\\
1.38e-09	0	\\
1.49e-09	0	\\
1.59e-09	0	\\
1.69e-09	0	\\
1.8e-09	0	\\
1.9e-09	0	\\
2.01e-09	0	\\
2.11e-09	0	\\
2.21e-09	0	\\
2.32e-09	0	\\
2.42e-09	0	\\
2.52e-09	0	\\
2.63e-09	0	\\
2.73e-09	0	\\
2.83e-09	0	\\
2.93e-09	0	\\
3.04e-09	0	\\
3.14e-09	0	\\
3.24e-09	0	\\
3.34e-09	0	\\
3.45e-09	0	\\
3.55e-09	0	\\
3.65e-09	0	\\
3.75e-09	0	\\
3.86e-09	0	\\
3.96e-09	0	\\
4.06e-09	0	\\
4.16e-09	0	\\
4.27e-09	0	\\
4.37e-09	0	\\
4.47e-09	0	\\
4.57e-09	0	\\
4.68e-09	0	\\
4.78e-09	0	\\
4.89e-09	0	\\
4.99e-09	0	\\
5e-09	0	\\
};
\addplot [color=mycolor3,solid,forget plot]
  table[row sep=crcr]{
0	0	\\
1.1e-10	0	\\
2.2e-10	0	\\
3.3e-10	0	\\
4.4e-10	0	\\
5.4e-10	0	\\
6.5e-10	0	\\
7.5e-10	0	\\
8.6e-10	0	\\
9.6e-10	0	\\
1.07e-09	0	\\
1.18e-09	0	\\
1.28e-09	0	\\
1.38e-09	0	\\
1.49e-09	0	\\
1.59e-09	0	\\
1.69e-09	0	\\
1.8e-09	0	\\
1.9e-09	0	\\
2.01e-09	0	\\
2.11e-09	0	\\
2.21e-09	0	\\
2.32e-09	0	\\
2.42e-09	0	\\
2.52e-09	0	\\
2.63e-09	0	\\
2.73e-09	0	\\
2.83e-09	0	\\
2.93e-09	0	\\
3.04e-09	0	\\
3.14e-09	0	\\
3.24e-09	0	\\
3.34e-09	0	\\
3.45e-09	0	\\
3.55e-09	0	\\
3.65e-09	0	\\
3.75e-09	0	\\
3.86e-09	0	\\
3.96e-09	0	\\
4.06e-09	0	\\
4.16e-09	0	\\
4.27e-09	0	\\
4.37e-09	0	\\
4.47e-09	0	\\
4.57e-09	0	\\
4.68e-09	0	\\
4.78e-09	0	\\
4.89e-09	0	\\
4.99e-09	0	\\
5e-09	0	\\
};
\addplot [color=darkgray,solid,forget plot]
  table[row sep=crcr]{
0	0	\\
1.1e-10	0	\\
2.2e-10	0	\\
3.3e-10	0	\\
4.4e-10	0	\\
5.4e-10	0	\\
6.5e-10	0	\\
7.5e-10	0	\\
8.6e-10	0	\\
9.6e-10	0	\\
1.07e-09	0	\\
1.18e-09	0	\\
1.28e-09	0	\\
1.38e-09	0	\\
1.49e-09	0	\\
1.59e-09	0	\\
1.69e-09	0	\\
1.8e-09	0	\\
1.9e-09	0	\\
2.01e-09	0	\\
2.11e-09	0	\\
2.21e-09	0	\\
2.32e-09	0	\\
2.42e-09	0	\\
2.52e-09	0	\\
2.63e-09	0	\\
2.73e-09	0	\\
2.83e-09	0	\\
2.93e-09	0	\\
3.04e-09	0	\\
3.14e-09	0	\\
3.24e-09	0	\\
3.34e-09	0	\\
3.45e-09	0	\\
3.55e-09	0	\\
3.65e-09	0	\\
3.75e-09	0	\\
3.86e-09	0	\\
3.96e-09	0	\\
4.06e-09	0	\\
4.16e-09	0	\\
4.27e-09	0	\\
4.37e-09	0	\\
4.47e-09	0	\\
4.57e-09	0	\\
4.68e-09	0	\\
4.78e-09	0	\\
4.89e-09	0	\\
4.99e-09	0	\\
5e-09	0	\\
};
\addplot [color=blue,solid,forget plot]
  table[row sep=crcr]{
0	0	\\
1.1e-10	0	\\
2.2e-10	0	\\
3.3e-10	0	\\
4.4e-10	0	\\
5.4e-10	0	\\
6.5e-10	0	\\
7.5e-10	0	\\
8.6e-10	0	\\
9.6e-10	0	\\
1.07e-09	0	\\
1.18e-09	0	\\
1.28e-09	0	\\
1.38e-09	0	\\
1.49e-09	0	\\
1.59e-09	0	\\
1.69e-09	0	\\
1.8e-09	0	\\
1.9e-09	0	\\
2.01e-09	0	\\
2.11e-09	0	\\
2.21e-09	0	\\
2.32e-09	0	\\
2.42e-09	0	\\
2.52e-09	0	\\
2.63e-09	0	\\
2.73e-09	0	\\
2.83e-09	0	\\
2.93e-09	0	\\
3.04e-09	0	\\
3.14e-09	0	\\
3.24e-09	0	\\
3.34e-09	0	\\
3.45e-09	0	\\
3.55e-09	0	\\
3.65e-09	0	\\
3.75e-09	0	\\
3.86e-09	0	\\
3.96e-09	0	\\
4.06e-09	0	\\
4.16e-09	0	\\
4.27e-09	0	\\
4.37e-09	0	\\
4.47e-09	0	\\
4.57e-09	0	\\
4.68e-09	0	\\
4.78e-09	0	\\
4.89e-09	0	\\
4.99e-09	0	\\
5e-09	0	\\
};
\addplot [color=black!50!green,solid,forget plot]
  table[row sep=crcr]{
0	0	\\
1.1e-10	0	\\
2.2e-10	0	\\
3.3e-10	0	\\
4.4e-10	0	\\
5.4e-10	0	\\
6.5e-10	0	\\
7.5e-10	0	\\
8.6e-10	0	\\
9.6e-10	0	\\
1.07e-09	0	\\
1.18e-09	0	\\
1.28e-09	0	\\
1.38e-09	0	\\
1.49e-09	0	\\
1.59e-09	0	\\
1.69e-09	0	\\
1.8e-09	0	\\
1.9e-09	0	\\
2.01e-09	0	\\
2.11e-09	0	\\
2.21e-09	0	\\
2.32e-09	0	\\
2.42e-09	0	\\
2.52e-09	0	\\
2.63e-09	0	\\
2.73e-09	0	\\
2.83e-09	0	\\
2.93e-09	0	\\
3.04e-09	0	\\
3.14e-09	0	\\
3.24e-09	0	\\
3.34e-09	0	\\
3.45e-09	0	\\
3.55e-09	0	\\
3.65e-09	0	\\
3.75e-09	0	\\
3.86e-09	0	\\
3.96e-09	0	\\
4.06e-09	0	\\
4.16e-09	0	\\
4.27e-09	0	\\
4.37e-09	0	\\
4.47e-09	0	\\
4.57e-09	0	\\
4.68e-09	0	\\
4.78e-09	0	\\
4.89e-09	0	\\
4.99e-09	0	\\
5e-09	0	\\
};
\addplot [color=red,solid,forget plot]
  table[row sep=crcr]{
0	0	\\
1.1e-10	0	\\
2.2e-10	0	\\
3.3e-10	0	\\
4.4e-10	0	\\
5.4e-10	0	\\
6.5e-10	0	\\
7.5e-10	0	\\
8.6e-10	0	\\
9.6e-10	0	\\
1.07e-09	0	\\
1.18e-09	0	\\
1.28e-09	0	\\
1.38e-09	0	\\
1.49e-09	0	\\
1.59e-09	0	\\
1.69e-09	0	\\
1.8e-09	0	\\
1.9e-09	0	\\
2.01e-09	0	\\
2.11e-09	0	\\
2.21e-09	0	\\
2.32e-09	0	\\
2.42e-09	0	\\
2.52e-09	0	\\
2.63e-09	0	\\
2.73e-09	0	\\
2.83e-09	0	\\
2.93e-09	0	\\
3.04e-09	0	\\
3.14e-09	0	\\
3.24e-09	0	\\
3.34e-09	0	\\
3.45e-09	0	\\
3.55e-09	0	\\
3.65e-09	0	\\
3.75e-09	0	\\
3.86e-09	0	\\
3.96e-09	0	\\
4.06e-09	0	\\
4.16e-09	0	\\
4.27e-09	0	\\
4.37e-09	0	\\
4.47e-09	0	\\
4.57e-09	0	\\
4.68e-09	0	\\
4.78e-09	0	\\
4.89e-09	0	\\
4.99e-09	0	\\
5e-09	0	\\
};
\addplot [color=mycolor1,solid,forget plot]
  table[row sep=crcr]{
0	0	\\
1.1e-10	0	\\
2.2e-10	0	\\
3.3e-10	0	\\
4.4e-10	0	\\
5.4e-10	0	\\
6.5e-10	0	\\
7.5e-10	0	\\
8.6e-10	0	\\
9.6e-10	0	\\
1.07e-09	0	\\
1.18e-09	0	\\
1.28e-09	0	\\
1.38e-09	0	\\
1.49e-09	0	\\
1.59e-09	0	\\
1.69e-09	0	\\
1.8e-09	0	\\
1.9e-09	0	\\
2.01e-09	0	\\
2.11e-09	0	\\
2.21e-09	0	\\
2.32e-09	0	\\
2.42e-09	0	\\
2.52e-09	0	\\
2.63e-09	0	\\
2.73e-09	0	\\
2.83e-09	0	\\
2.93e-09	0	\\
3.04e-09	0	\\
3.14e-09	0	\\
3.24e-09	0	\\
3.34e-09	0	\\
3.45e-09	0	\\
3.55e-09	0	\\
3.65e-09	0	\\
3.75e-09	0	\\
3.86e-09	0	\\
3.96e-09	0	\\
4.06e-09	0	\\
4.16e-09	0	\\
4.27e-09	0	\\
4.37e-09	0	\\
4.47e-09	0	\\
4.57e-09	0	\\
4.68e-09	0	\\
4.78e-09	0	\\
4.89e-09	0	\\
4.99e-09	0	\\
5e-09	0	\\
};
\addplot [color=mycolor2,solid,forget plot]
  table[row sep=crcr]{
0	0	\\
1.1e-10	0	\\
2.2e-10	0	\\
3.3e-10	0	\\
4.4e-10	0	\\
5.4e-10	0	\\
6.5e-10	0	\\
7.5e-10	0	\\
8.6e-10	0	\\
9.6e-10	0	\\
1.07e-09	0	\\
1.18e-09	0	\\
1.28e-09	0	\\
1.38e-09	0	\\
1.49e-09	0	\\
1.59e-09	0	\\
1.69e-09	0	\\
1.8e-09	0	\\
1.9e-09	0	\\
2.01e-09	0	\\
2.11e-09	0	\\
2.21e-09	0	\\
2.32e-09	0	\\
2.42e-09	0	\\
2.52e-09	0	\\
2.63e-09	0	\\
2.73e-09	0	\\
2.83e-09	0	\\
2.93e-09	0	\\
3.04e-09	0	\\
3.14e-09	0	\\
3.24e-09	0	\\
3.34e-09	0	\\
3.45e-09	0	\\
3.55e-09	0	\\
3.65e-09	0	\\
3.75e-09	0	\\
3.86e-09	0	\\
3.96e-09	0	\\
4.06e-09	0	\\
4.16e-09	0	\\
4.27e-09	0	\\
4.37e-09	0	\\
4.47e-09	0	\\
4.57e-09	0	\\
4.68e-09	0	\\
4.78e-09	0	\\
4.89e-09	0	\\
4.99e-09	0	\\
5e-09	0	\\
};
\addplot [color=mycolor3,solid,forget plot]
  table[row sep=crcr]{
0	0	\\
1.1e-10	0	\\
2.2e-10	0	\\
3.3e-10	0	\\
4.4e-10	0	\\
5.4e-10	0	\\
6.5e-10	0	\\
7.5e-10	0	\\
8.6e-10	0	\\
9.6e-10	0	\\
1.07e-09	0	\\
1.18e-09	0	\\
1.28e-09	0	\\
1.38e-09	0	\\
1.49e-09	0	\\
1.59e-09	0	\\
1.69e-09	0	\\
1.8e-09	0	\\
1.9e-09	0	\\
2.01e-09	0	\\
2.11e-09	0	\\
2.21e-09	0	\\
2.32e-09	0	\\
2.42e-09	0	\\
2.52e-09	0	\\
2.63e-09	0	\\
2.73e-09	0	\\
2.83e-09	0	\\
2.93e-09	0	\\
3.04e-09	0	\\
3.14e-09	0	\\
3.24e-09	0	\\
3.34e-09	0	\\
3.45e-09	0	\\
3.55e-09	0	\\
3.65e-09	0	\\
3.75e-09	0	\\
3.86e-09	0	\\
3.96e-09	0	\\
4.06e-09	0	\\
4.16e-09	0	\\
4.27e-09	0	\\
4.37e-09	0	\\
4.47e-09	0	\\
4.57e-09	0	\\
4.68e-09	0	\\
4.78e-09	0	\\
4.89e-09	0	\\
4.99e-09	0	\\
5e-09	0	\\
};
\addplot [color=darkgray,solid,forget plot]
  table[row sep=crcr]{
0	0	\\
1.1e-10	0	\\
2.2e-10	0	\\
3.3e-10	0	\\
4.4e-10	0	\\
5.4e-10	0	\\
6.5e-10	0	\\
7.5e-10	0	\\
8.6e-10	0	\\
9.6e-10	0	\\
1.07e-09	0	\\
1.18e-09	0	\\
1.28e-09	0	\\
1.38e-09	0	\\
1.49e-09	0	\\
1.59e-09	0	\\
1.69e-09	0	\\
1.8e-09	0	\\
1.9e-09	0	\\
2.01e-09	0	\\
2.11e-09	0	\\
2.21e-09	0	\\
2.32e-09	0	\\
2.42e-09	0	\\
2.52e-09	0	\\
2.63e-09	0	\\
2.73e-09	0	\\
2.83e-09	0	\\
2.93e-09	0	\\
3.04e-09	0	\\
3.14e-09	0	\\
3.24e-09	0	\\
3.34e-09	0	\\
3.45e-09	0	\\
3.55e-09	0	\\
3.65e-09	0	\\
3.75e-09	0	\\
3.86e-09	0	\\
3.96e-09	0	\\
4.06e-09	0	\\
4.16e-09	0	\\
4.27e-09	0	\\
4.37e-09	0	\\
4.47e-09	0	\\
4.57e-09	0	\\
4.68e-09	0	\\
4.78e-09	0	\\
4.89e-09	0	\\
4.99e-09	0	\\
5e-09	0	\\
};
\addplot [color=blue,solid,forget plot]
  table[row sep=crcr]{
0	0	\\
1.1e-10	0	\\
2.2e-10	0	\\
3.3e-10	0	\\
4.4e-10	0	\\
5.4e-10	0	\\
6.5e-10	0	\\
7.5e-10	0	\\
8.6e-10	0	\\
9.6e-10	0	\\
1.07e-09	0	\\
1.18e-09	0	\\
1.28e-09	0	\\
1.38e-09	0	\\
1.49e-09	0	\\
1.59e-09	0	\\
1.69e-09	0	\\
1.8e-09	0	\\
1.9e-09	0	\\
2.01e-09	0	\\
2.11e-09	0	\\
2.21e-09	0	\\
2.32e-09	0	\\
2.42e-09	0	\\
2.52e-09	0	\\
2.63e-09	0	\\
2.73e-09	0	\\
2.83e-09	0	\\
2.93e-09	0	\\
3.04e-09	0	\\
3.14e-09	0	\\
3.24e-09	0	\\
3.34e-09	0	\\
3.45e-09	0	\\
3.55e-09	0	\\
3.65e-09	0	\\
3.75e-09	0	\\
3.86e-09	0	\\
3.96e-09	0	\\
4.06e-09	0	\\
4.16e-09	0	\\
4.27e-09	0	\\
4.37e-09	0	\\
4.47e-09	0	\\
4.57e-09	0	\\
4.68e-09	0	\\
4.78e-09	0	\\
4.89e-09	0	\\
4.99e-09	0	\\
5e-09	0	\\
};
\addplot [color=black!50!green,solid,forget plot]
  table[row sep=crcr]{
0	0	\\
1.1e-10	0	\\
2.2e-10	0	\\
3.3e-10	0	\\
4.4e-10	0	\\
5.4e-10	0	\\
6.5e-10	0	\\
7.5e-10	0	\\
8.6e-10	0	\\
9.6e-10	0	\\
1.07e-09	0	\\
1.18e-09	0	\\
1.28e-09	0	\\
1.38e-09	0	\\
1.49e-09	0	\\
1.59e-09	0	\\
1.69e-09	0	\\
1.8e-09	0	\\
1.9e-09	0	\\
2.01e-09	0	\\
2.11e-09	0	\\
2.21e-09	0	\\
2.32e-09	0	\\
2.42e-09	0	\\
2.52e-09	0	\\
2.63e-09	0	\\
2.73e-09	0	\\
2.83e-09	0	\\
2.93e-09	0	\\
3.04e-09	0	\\
3.14e-09	0	\\
3.24e-09	0	\\
3.34e-09	0	\\
3.45e-09	0	\\
3.55e-09	0	\\
3.65e-09	0	\\
3.75e-09	0	\\
3.86e-09	0	\\
3.96e-09	0	\\
4.06e-09	0	\\
4.16e-09	0	\\
4.27e-09	0	\\
4.37e-09	0	\\
4.47e-09	0	\\
4.57e-09	0	\\
4.68e-09	0	\\
4.78e-09	0	\\
4.89e-09	0	\\
4.99e-09	0	\\
5e-09	0	\\
};
\addplot [color=red,solid,forget plot]
  table[row sep=crcr]{
0	0	\\
1.1e-10	0	\\
2.2e-10	0	\\
3.3e-10	0	\\
4.4e-10	0	\\
5.4e-10	0	\\
6.5e-10	0	\\
7.5e-10	0	\\
8.6e-10	0	\\
9.6e-10	0	\\
1.07e-09	0	\\
1.18e-09	0	\\
1.28e-09	0	\\
1.38e-09	0	\\
1.49e-09	0	\\
1.59e-09	0	\\
1.69e-09	0	\\
1.8e-09	0	\\
1.9e-09	0	\\
2.01e-09	0	\\
2.11e-09	0	\\
2.21e-09	0	\\
2.32e-09	0	\\
2.42e-09	0	\\
2.52e-09	0	\\
2.63e-09	0	\\
2.73e-09	0	\\
2.83e-09	0	\\
2.93e-09	0	\\
3.04e-09	0	\\
3.14e-09	0	\\
3.24e-09	0	\\
3.34e-09	0	\\
3.45e-09	0	\\
3.55e-09	0	\\
3.65e-09	0	\\
3.75e-09	0	\\
3.86e-09	0	\\
3.96e-09	0	\\
4.06e-09	0	\\
4.16e-09	0	\\
4.27e-09	0	\\
4.37e-09	0	\\
4.47e-09	0	\\
4.57e-09	0	\\
4.68e-09	0	\\
4.78e-09	0	\\
4.89e-09	0	\\
4.99e-09	0	\\
5e-09	0	\\
};
\addplot [color=mycolor1,solid,forget plot]
  table[row sep=crcr]{
0	0	\\
1.1e-10	0	\\
2.2e-10	0	\\
3.3e-10	0	\\
4.4e-10	0	\\
5.4e-10	0	\\
6.5e-10	0	\\
7.5e-10	0	\\
8.6e-10	0	\\
9.6e-10	0	\\
1.07e-09	0	\\
1.18e-09	0	\\
1.28e-09	0	\\
1.38e-09	0	\\
1.49e-09	0	\\
1.59e-09	0	\\
1.69e-09	0	\\
1.8e-09	0	\\
1.9e-09	0	\\
2.01e-09	0	\\
2.11e-09	0	\\
2.21e-09	0	\\
2.32e-09	0	\\
2.42e-09	0	\\
2.52e-09	0	\\
2.63e-09	0	\\
2.73e-09	0	\\
2.83e-09	0	\\
2.93e-09	0	\\
3.04e-09	0	\\
3.14e-09	0	\\
3.24e-09	0	\\
3.34e-09	0	\\
3.45e-09	0	\\
3.55e-09	0	\\
3.65e-09	0	\\
3.75e-09	0	\\
3.86e-09	0	\\
3.96e-09	0	\\
4.06e-09	0	\\
4.16e-09	0	\\
4.27e-09	0	\\
4.37e-09	0	\\
4.47e-09	0	\\
4.57e-09	0	\\
4.68e-09	0	\\
4.78e-09	0	\\
4.89e-09	0	\\
4.99e-09	0	\\
5e-09	0	\\
};
\addplot [color=mycolor2,solid,forget plot]
  table[row sep=crcr]{
0	0	\\
1.1e-10	0	\\
2.2e-10	0	\\
3.3e-10	0	\\
4.4e-10	0	\\
5.4e-10	0	\\
6.5e-10	0	\\
7.5e-10	0	\\
8.6e-10	0	\\
9.6e-10	0	\\
1.07e-09	0	\\
1.18e-09	0	\\
1.28e-09	0	\\
1.38e-09	0	\\
1.49e-09	0	\\
1.59e-09	0	\\
1.69e-09	0	\\
1.8e-09	0	\\
1.9e-09	0	\\
2.01e-09	0	\\
2.11e-09	0	\\
2.21e-09	0	\\
2.32e-09	0	\\
2.42e-09	0	\\
2.52e-09	0	\\
2.63e-09	0	\\
2.73e-09	0	\\
2.83e-09	0	\\
2.93e-09	0	\\
3.04e-09	0	\\
3.14e-09	0	\\
3.24e-09	0	\\
3.34e-09	0	\\
3.45e-09	0	\\
3.55e-09	0	\\
3.65e-09	0	\\
3.75e-09	0	\\
3.86e-09	0	\\
3.96e-09	0	\\
4.06e-09	0	\\
4.16e-09	0	\\
4.27e-09	0	\\
4.37e-09	0	\\
4.47e-09	0	\\
4.57e-09	0	\\
4.68e-09	0	\\
4.78e-09	0	\\
4.89e-09	0	\\
4.99e-09	0	\\
5e-09	0	\\
};
\addplot [color=mycolor3,solid,forget plot]
  table[row sep=crcr]{
0	0	\\
1.1e-10	0	\\
2.2e-10	0	\\
3.3e-10	0	\\
4.4e-10	0	\\
5.4e-10	0	\\
6.5e-10	0	\\
7.5e-10	0	\\
8.6e-10	0	\\
9.6e-10	0	\\
1.07e-09	0	\\
1.18e-09	0	\\
1.28e-09	0	\\
1.38e-09	0	\\
1.49e-09	0	\\
1.59e-09	0	\\
1.69e-09	0	\\
1.8e-09	0	\\
1.9e-09	0	\\
2.01e-09	0	\\
2.11e-09	0	\\
2.21e-09	0	\\
2.32e-09	0	\\
2.42e-09	0	\\
2.52e-09	0	\\
2.63e-09	0	\\
2.73e-09	0	\\
2.83e-09	0	\\
2.93e-09	0	\\
3.04e-09	0	\\
3.14e-09	0	\\
3.24e-09	0	\\
3.34e-09	0	\\
3.45e-09	0	\\
3.55e-09	0	\\
3.65e-09	0	\\
3.75e-09	0	\\
3.86e-09	0	\\
3.96e-09	0	\\
4.06e-09	0	\\
4.16e-09	0	\\
4.27e-09	0	\\
4.37e-09	0	\\
4.47e-09	0	\\
4.57e-09	0	\\
4.68e-09	0	\\
4.78e-09	0	\\
4.89e-09	0	\\
4.99e-09	0	\\
5e-09	0	\\
};
\addplot [color=darkgray,solid,forget plot]
  table[row sep=crcr]{
0	0	\\
1.1e-10	0	\\
2.2e-10	0	\\
3.3e-10	0	\\
4.4e-10	0	\\
5.4e-10	0	\\
6.5e-10	0	\\
7.5e-10	0	\\
8.6e-10	0	\\
9.6e-10	0	\\
1.07e-09	0	\\
1.18e-09	0	\\
1.28e-09	0	\\
1.38e-09	0	\\
1.49e-09	0	\\
1.59e-09	0	\\
1.69e-09	0	\\
1.8e-09	0	\\
1.9e-09	0	\\
2.01e-09	0	\\
2.11e-09	0	\\
2.21e-09	0	\\
2.32e-09	0	\\
2.42e-09	0	\\
2.52e-09	0	\\
2.63e-09	0	\\
2.73e-09	0	\\
2.83e-09	0	\\
2.93e-09	0	\\
3.04e-09	0	\\
3.14e-09	0	\\
3.24e-09	0	\\
3.34e-09	0	\\
3.45e-09	0	\\
3.55e-09	0	\\
3.65e-09	0	\\
3.75e-09	0	\\
3.86e-09	0	\\
3.96e-09	0	\\
4.06e-09	0	\\
4.16e-09	0	\\
4.27e-09	0	\\
4.37e-09	0	\\
4.47e-09	0	\\
4.57e-09	0	\\
4.68e-09	0	\\
4.78e-09	0	\\
4.89e-09	0	\\
4.99e-09	0	\\
5e-09	0	\\
};
\addplot [color=blue,solid,forget plot]
  table[row sep=crcr]{
0	0	\\
1.1e-10	0	\\
2.2e-10	0	\\
3.3e-10	0	\\
4.4e-10	0	\\
5.4e-10	0	\\
6.5e-10	0	\\
7.5e-10	0	\\
8.6e-10	0	\\
9.6e-10	0	\\
1.07e-09	0	\\
1.18e-09	0	\\
1.28e-09	0	\\
1.38e-09	0	\\
1.49e-09	0	\\
1.59e-09	0	\\
1.69e-09	0	\\
1.8e-09	0	\\
1.9e-09	0	\\
2.01e-09	0	\\
2.11e-09	0	\\
2.21e-09	0	\\
2.32e-09	0	\\
2.42e-09	0	\\
2.52e-09	0	\\
2.63e-09	0	\\
2.73e-09	0	\\
2.83e-09	0	\\
2.93e-09	0	\\
3.04e-09	0	\\
3.14e-09	0	\\
3.24e-09	0	\\
3.34e-09	0	\\
3.45e-09	0	\\
3.55e-09	0	\\
3.65e-09	0	\\
3.75e-09	0	\\
3.86e-09	0	\\
3.96e-09	0	\\
4.06e-09	0	\\
4.16e-09	0	\\
4.27e-09	0	\\
4.37e-09	0	\\
4.47e-09	0	\\
4.57e-09	0	\\
4.68e-09	0	\\
4.78e-09	0	\\
4.89e-09	0	\\
4.99e-09	0	\\
5e-09	0	\\
};
\addplot [color=black!50!green,solid,forget plot]
  table[row sep=crcr]{
0	0	\\
1.1e-10	0	\\
2.2e-10	0	\\
3.3e-10	0	\\
4.4e-10	0	\\
5.4e-10	0	\\
6.5e-10	0	\\
7.5e-10	0	\\
8.6e-10	0	\\
9.6e-10	0	\\
1.07e-09	0	\\
1.18e-09	0	\\
1.28e-09	0	\\
1.38e-09	0	\\
1.49e-09	0	\\
1.59e-09	0	\\
1.69e-09	0	\\
1.8e-09	0	\\
1.9e-09	0	\\
2.01e-09	0	\\
2.11e-09	0	\\
2.21e-09	0	\\
2.32e-09	0	\\
2.42e-09	0	\\
2.52e-09	0	\\
2.63e-09	0	\\
2.73e-09	0	\\
2.83e-09	0	\\
2.93e-09	0	\\
3.04e-09	0	\\
3.14e-09	0	\\
3.24e-09	0	\\
3.34e-09	0	\\
3.45e-09	0	\\
3.55e-09	0	\\
3.65e-09	0	\\
3.75e-09	0	\\
3.86e-09	0	\\
3.96e-09	0	\\
4.06e-09	0	\\
4.16e-09	0	\\
4.27e-09	0	\\
4.37e-09	0	\\
4.47e-09	0	\\
4.57e-09	0	\\
4.68e-09	0	\\
4.78e-09	0	\\
4.89e-09	0	\\
4.99e-09	0	\\
5e-09	0	\\
};
\addplot [color=red,solid,forget plot]
  table[row sep=crcr]{
0	0	\\
1.1e-10	0	\\
2.2e-10	0	\\
3.3e-10	0	\\
4.4e-10	0	\\
5.4e-10	0	\\
6.5e-10	0	\\
7.5e-10	0	\\
8.6e-10	0	\\
9.6e-10	0	\\
1.07e-09	0	\\
1.18e-09	0	\\
1.28e-09	0	\\
1.38e-09	0	\\
1.49e-09	0	\\
1.59e-09	0	\\
1.69e-09	0	\\
1.8e-09	0	\\
1.9e-09	0	\\
2.01e-09	0	\\
2.11e-09	0	\\
2.21e-09	0	\\
2.32e-09	0	\\
2.42e-09	0	\\
2.52e-09	0	\\
2.63e-09	0	\\
2.73e-09	0	\\
2.83e-09	0	\\
2.93e-09	0	\\
3.04e-09	0	\\
3.14e-09	0	\\
3.24e-09	0	\\
3.34e-09	0	\\
3.45e-09	0	\\
3.55e-09	0	\\
3.65e-09	0	\\
3.75e-09	0	\\
3.86e-09	0	\\
3.96e-09	0	\\
4.06e-09	0	\\
4.16e-09	0	\\
4.27e-09	0	\\
4.37e-09	0	\\
4.47e-09	0	\\
4.57e-09	0	\\
4.68e-09	0	\\
4.78e-09	0	\\
4.89e-09	0	\\
4.99e-09	0	\\
5e-09	0	\\
};
\addplot [color=mycolor1,solid,forget plot]
  table[row sep=crcr]{
0	0	\\
1.1e-10	0	\\
2.2e-10	0	\\
3.3e-10	0	\\
4.4e-10	0	\\
5.4e-10	0	\\
6.5e-10	0	\\
7.5e-10	0	\\
8.6e-10	0	\\
9.6e-10	0	\\
1.07e-09	0	\\
1.18e-09	0	\\
1.28e-09	0	\\
1.38e-09	0	\\
1.49e-09	0	\\
1.59e-09	0	\\
1.69e-09	0	\\
1.8e-09	0	\\
1.9e-09	0	\\
2.01e-09	0	\\
2.11e-09	0	\\
2.21e-09	0	\\
2.32e-09	0	\\
2.42e-09	0	\\
2.52e-09	0	\\
2.63e-09	0	\\
2.73e-09	0	\\
2.83e-09	0	\\
2.93e-09	0	\\
3.04e-09	0	\\
3.14e-09	0	\\
3.24e-09	0	\\
3.34e-09	0	\\
3.45e-09	0	\\
3.55e-09	0	\\
3.65e-09	0	\\
3.75e-09	0	\\
3.86e-09	0	\\
3.96e-09	0	\\
4.06e-09	0	\\
4.16e-09	0	\\
4.27e-09	0	\\
4.37e-09	0	\\
4.47e-09	0	\\
4.57e-09	0	\\
4.68e-09	0	\\
4.78e-09	0	\\
4.89e-09	0	\\
4.99e-09	0	\\
5e-09	0	\\
};
\addplot [color=mycolor2,solid,forget plot]
  table[row sep=crcr]{
0	0	\\
1.1e-10	0	\\
2.2e-10	0	\\
3.3e-10	0	\\
4.4e-10	0	\\
5.4e-10	0	\\
6.5e-10	0	\\
7.5e-10	0	\\
8.6e-10	0	\\
9.6e-10	0	\\
1.07e-09	0	\\
1.18e-09	0	\\
1.28e-09	0	\\
1.38e-09	0	\\
1.49e-09	0	\\
1.59e-09	0	\\
1.69e-09	0	\\
1.8e-09	0	\\
1.9e-09	0	\\
2.01e-09	0	\\
2.11e-09	0	\\
2.21e-09	0	\\
2.32e-09	0	\\
2.42e-09	0	\\
2.52e-09	0	\\
2.63e-09	0	\\
2.73e-09	0	\\
2.83e-09	0	\\
2.93e-09	0	\\
3.04e-09	0	\\
3.14e-09	0	\\
3.24e-09	0	\\
3.34e-09	0	\\
3.45e-09	0	\\
3.55e-09	0	\\
3.65e-09	0	\\
3.75e-09	0	\\
3.86e-09	0	\\
3.96e-09	0	\\
4.06e-09	0	\\
4.16e-09	0	\\
4.27e-09	0	\\
4.37e-09	0	\\
4.47e-09	0	\\
4.57e-09	0	\\
4.68e-09	0	\\
4.78e-09	0	\\
4.89e-09	0	\\
4.99e-09	0	\\
5e-09	0	\\
};
\addplot [color=mycolor3,solid,forget plot]
  table[row sep=crcr]{
0	0	\\
1.1e-10	0	\\
2.2e-10	0	\\
3.3e-10	0	\\
4.4e-10	0	\\
5.4e-10	0	\\
6.5e-10	0	\\
7.5e-10	0	\\
8.6e-10	0	\\
9.6e-10	0	\\
1.07e-09	0	\\
1.18e-09	0	\\
1.28e-09	0	\\
1.38e-09	0	\\
1.49e-09	0	\\
1.59e-09	0	\\
1.69e-09	0	\\
1.8e-09	0	\\
1.9e-09	0	\\
2.01e-09	0	\\
2.11e-09	0	\\
2.21e-09	0	\\
2.32e-09	0	\\
2.42e-09	0	\\
2.52e-09	0	\\
2.63e-09	0	\\
2.73e-09	0	\\
2.83e-09	0	\\
2.93e-09	0	\\
3.04e-09	0	\\
3.14e-09	0	\\
3.24e-09	0	\\
3.34e-09	0	\\
3.45e-09	0	\\
3.55e-09	0	\\
3.65e-09	0	\\
3.75e-09	0	\\
3.86e-09	0	\\
3.96e-09	0	\\
4.06e-09	0	\\
4.16e-09	0	\\
4.27e-09	0	\\
4.37e-09	0	\\
4.47e-09	0	\\
4.57e-09	0	\\
4.68e-09	0	\\
4.78e-09	0	\\
4.89e-09	0	\\
4.99e-09	0	\\
5e-09	0	\\
};
\addplot [color=darkgray,solid,forget plot]
  table[row sep=crcr]{
0	0	\\
1.1e-10	0	\\
2.2e-10	0	\\
3.3e-10	0	\\
4.4e-10	0	\\
5.4e-10	0	\\
6.5e-10	0	\\
7.5e-10	0	\\
8.6e-10	0	\\
9.6e-10	0	\\
1.07e-09	0	\\
1.18e-09	0	\\
1.28e-09	0	\\
1.38e-09	0	\\
1.49e-09	0	\\
1.59e-09	0	\\
1.69e-09	0	\\
1.8e-09	0	\\
1.9e-09	0	\\
2.01e-09	0	\\
2.11e-09	0	\\
2.21e-09	0	\\
2.32e-09	0	\\
2.42e-09	0	\\
2.52e-09	0	\\
2.63e-09	0	\\
2.73e-09	0	\\
2.83e-09	0	\\
2.93e-09	0	\\
3.04e-09	0	\\
3.14e-09	0	\\
3.24e-09	0	\\
3.34e-09	0	\\
3.45e-09	0	\\
3.55e-09	0	\\
3.65e-09	0	\\
3.75e-09	0	\\
3.86e-09	0	\\
3.96e-09	0	\\
4.06e-09	0	\\
4.16e-09	0	\\
4.27e-09	0	\\
4.37e-09	0	\\
4.47e-09	0	\\
4.57e-09	0	\\
4.68e-09	0	\\
4.78e-09	0	\\
4.89e-09	0	\\
4.99e-09	0	\\
5e-09	0	\\
};
\addplot [color=blue,solid,forget plot]
  table[row sep=crcr]{
0	0	\\
1.1e-10	0	\\
2.2e-10	0	\\
3.3e-10	0	\\
4.4e-10	0	\\
5.4e-10	0	\\
6.5e-10	0	\\
7.5e-10	0	\\
8.6e-10	0	\\
9.6e-10	0	\\
1.07e-09	0	\\
1.18e-09	0	\\
1.28e-09	0	\\
1.38e-09	0	\\
1.49e-09	0	\\
1.59e-09	0	\\
1.69e-09	0	\\
1.8e-09	0	\\
1.9e-09	0	\\
2.01e-09	0	\\
2.11e-09	0	\\
2.21e-09	0	\\
2.32e-09	0	\\
2.42e-09	0	\\
2.52e-09	0	\\
2.63e-09	0	\\
2.73e-09	0	\\
2.83e-09	0	\\
2.93e-09	0	\\
3.04e-09	0	\\
3.14e-09	0	\\
3.24e-09	0	\\
3.34e-09	0	\\
3.45e-09	0	\\
3.55e-09	0	\\
3.65e-09	0	\\
3.75e-09	0	\\
3.86e-09	0	\\
3.96e-09	0	\\
4.06e-09	0	\\
4.16e-09	0	\\
4.27e-09	0	\\
4.37e-09	0	\\
4.47e-09	0	\\
4.57e-09	0	\\
4.68e-09	0	\\
4.78e-09	0	\\
4.89e-09	0	\\
4.99e-09	0	\\
5e-09	0	\\
};
\addplot [color=black!50!green,solid,forget plot]
  table[row sep=crcr]{
0	0	\\
1.1e-10	0	\\
2.2e-10	0	\\
3.3e-10	0	\\
4.4e-10	0	\\
5.4e-10	0	\\
6.5e-10	0	\\
7.5e-10	0	\\
8.6e-10	0	\\
9.6e-10	0	\\
1.07e-09	0	\\
1.18e-09	0	\\
1.28e-09	0	\\
1.38e-09	0	\\
1.49e-09	0	\\
1.59e-09	0	\\
1.69e-09	0	\\
1.8e-09	0	\\
1.9e-09	0	\\
2.01e-09	0	\\
2.11e-09	0	\\
2.21e-09	0	\\
2.32e-09	0	\\
2.42e-09	0	\\
2.52e-09	0	\\
2.63e-09	0	\\
2.73e-09	0	\\
2.83e-09	0	\\
2.93e-09	0	\\
3.04e-09	0	\\
3.14e-09	0	\\
3.24e-09	0	\\
3.34e-09	0	\\
3.45e-09	0	\\
3.55e-09	0	\\
3.65e-09	0	\\
3.75e-09	0	\\
3.86e-09	0	\\
3.96e-09	0	\\
4.06e-09	0	\\
4.16e-09	0	\\
4.27e-09	0	\\
4.37e-09	0	\\
4.47e-09	0	\\
4.57e-09	0	\\
4.68e-09	0	\\
4.78e-09	0	\\
4.89e-09	0	\\
4.99e-09	0	\\
5e-09	0	\\
};
\addplot [color=red,solid,forget plot]
  table[row sep=crcr]{
0	0	\\
1.1e-10	0	\\
2.2e-10	0	\\
3.3e-10	0	\\
4.4e-10	0	\\
5.4e-10	0	\\
6.5e-10	0	\\
7.5e-10	0	\\
8.6e-10	0	\\
9.6e-10	0	\\
1.07e-09	0	\\
1.18e-09	0	\\
1.28e-09	0	\\
1.38e-09	0	\\
1.49e-09	0	\\
1.59e-09	0	\\
1.69e-09	0	\\
1.8e-09	0	\\
1.9e-09	0	\\
2.01e-09	0	\\
2.11e-09	0	\\
2.21e-09	0	\\
2.32e-09	0	\\
2.42e-09	0	\\
2.52e-09	0	\\
2.63e-09	0	\\
2.73e-09	0	\\
2.83e-09	0	\\
2.93e-09	0	\\
3.04e-09	0	\\
3.14e-09	0	\\
3.24e-09	0	\\
3.34e-09	0	\\
3.45e-09	0	\\
3.55e-09	0	\\
3.65e-09	0	\\
3.75e-09	0	\\
3.86e-09	0	\\
3.96e-09	0	\\
4.06e-09	0	\\
4.16e-09	0	\\
4.27e-09	0	\\
4.37e-09	0	\\
4.47e-09	0	\\
4.57e-09	0	\\
4.68e-09	0	\\
4.78e-09	0	\\
4.89e-09	0	\\
4.99e-09	0	\\
5e-09	0	\\
};
\addplot [color=mycolor1,solid,forget plot]
  table[row sep=crcr]{
0	0	\\
1.1e-10	0	\\
2.2e-10	0	\\
3.3e-10	0	\\
4.4e-10	0	\\
5.4e-10	0	\\
6.5e-10	0	\\
7.5e-10	0	\\
8.6e-10	0	\\
9.6e-10	0	\\
1.07e-09	0	\\
1.18e-09	0	\\
1.28e-09	0	\\
1.38e-09	0	\\
1.49e-09	0	\\
1.59e-09	0	\\
1.69e-09	0	\\
1.8e-09	0	\\
1.9e-09	0	\\
2.01e-09	0	\\
2.11e-09	0	\\
2.21e-09	0	\\
2.32e-09	0	\\
2.42e-09	0	\\
2.52e-09	0	\\
2.63e-09	0	\\
2.73e-09	0	\\
2.83e-09	0	\\
2.93e-09	0	\\
3.04e-09	0	\\
3.14e-09	0	\\
3.24e-09	0	\\
3.34e-09	0	\\
3.45e-09	0	\\
3.55e-09	0	\\
3.65e-09	0	\\
3.75e-09	0	\\
3.86e-09	0	\\
3.96e-09	0	\\
4.06e-09	0	\\
4.16e-09	0	\\
4.27e-09	0	\\
4.37e-09	0	\\
4.47e-09	0	\\
4.57e-09	0	\\
4.68e-09	0	\\
4.78e-09	0	\\
4.89e-09	0	\\
4.99e-09	0	\\
5e-09	0	\\
};
\addplot [color=mycolor2,solid,forget plot]
  table[row sep=crcr]{
0	0	\\
1.1e-10	0	\\
2.2e-10	0	\\
3.3e-10	0	\\
4.4e-10	0	\\
5.4e-10	0	\\
6.5e-10	0	\\
7.5e-10	0	\\
8.6e-10	0	\\
9.6e-10	0	\\
1.07e-09	0	\\
1.18e-09	0	\\
1.28e-09	0	\\
1.38e-09	0	\\
1.49e-09	0	\\
1.59e-09	0	\\
1.69e-09	0	\\
1.8e-09	0	\\
1.9e-09	0	\\
2.01e-09	0	\\
2.11e-09	0	\\
2.21e-09	0	\\
2.32e-09	0	\\
2.42e-09	0	\\
2.52e-09	0	\\
2.63e-09	0	\\
2.73e-09	0	\\
2.83e-09	0	\\
2.93e-09	0	\\
3.04e-09	0	\\
3.14e-09	0	\\
3.24e-09	0	\\
3.34e-09	0	\\
3.45e-09	0	\\
3.55e-09	0	\\
3.65e-09	0	\\
3.75e-09	0	\\
3.86e-09	0	\\
3.96e-09	0	\\
4.06e-09	0	\\
4.16e-09	0	\\
4.27e-09	0	\\
4.37e-09	0	\\
4.47e-09	0	\\
4.57e-09	0	\\
4.68e-09	0	\\
4.78e-09	0	\\
4.89e-09	0	\\
4.99e-09	0	\\
5e-09	0	\\
};
\addplot [color=mycolor3,solid,forget plot]
  table[row sep=crcr]{
0	0	\\
1.1e-10	0	\\
2.2e-10	0	\\
3.3e-10	0	\\
4.4e-10	0	\\
5.4e-10	0	\\
6.5e-10	0	\\
7.5e-10	0	\\
8.6e-10	0	\\
9.6e-10	0	\\
1.07e-09	0	\\
1.18e-09	0	\\
1.28e-09	0	\\
1.38e-09	0	\\
1.49e-09	0	\\
1.59e-09	0	\\
1.69e-09	0	\\
1.8e-09	0	\\
1.9e-09	0	\\
2.01e-09	0	\\
2.11e-09	0	\\
2.21e-09	0	\\
2.32e-09	0	\\
2.42e-09	0	\\
2.52e-09	0	\\
2.63e-09	0	\\
2.73e-09	0	\\
2.83e-09	0	\\
2.93e-09	0	\\
3.04e-09	0	\\
3.14e-09	0	\\
3.24e-09	0	\\
3.34e-09	0	\\
3.45e-09	0	\\
3.55e-09	0	\\
3.65e-09	0	\\
3.75e-09	0	\\
3.86e-09	0	\\
3.96e-09	0	\\
4.06e-09	0	\\
4.16e-09	0	\\
4.27e-09	0	\\
4.37e-09	0	\\
4.47e-09	0	\\
4.57e-09	0	\\
4.68e-09	0	\\
4.78e-09	0	\\
4.89e-09	0	\\
4.99e-09	0	\\
5e-09	0	\\
};
\addplot [color=darkgray,solid,forget plot]
  table[row sep=crcr]{
0	0	\\
1.1e-10	0	\\
2.2e-10	0	\\
3.3e-10	0	\\
4.4e-10	0	\\
5.4e-10	0	\\
6.5e-10	0	\\
7.5e-10	0	\\
8.6e-10	0	\\
9.6e-10	0	\\
1.07e-09	0	\\
1.18e-09	0	\\
1.28e-09	0	\\
1.38e-09	0	\\
1.49e-09	0	\\
1.59e-09	0	\\
1.69e-09	0	\\
1.8e-09	0	\\
1.9e-09	0	\\
2.01e-09	0	\\
2.11e-09	0	\\
2.21e-09	0	\\
2.32e-09	0	\\
2.42e-09	0	\\
2.52e-09	0	\\
2.63e-09	0	\\
2.73e-09	0	\\
2.83e-09	0	\\
2.93e-09	0	\\
3.04e-09	0	\\
3.14e-09	0	\\
3.24e-09	0	\\
3.34e-09	0	\\
3.45e-09	0	\\
3.55e-09	0	\\
3.65e-09	0	\\
3.75e-09	0	\\
3.86e-09	0	\\
3.96e-09	0	\\
4.06e-09	0	\\
4.16e-09	0	\\
4.27e-09	0	\\
4.37e-09	0	\\
4.47e-09	0	\\
4.57e-09	0	\\
4.68e-09	0	\\
4.78e-09	0	\\
4.89e-09	0	\\
4.99e-09	0	\\
5e-09	0	\\
};
\addplot [color=blue,solid,forget plot]
  table[row sep=crcr]{
0	0	\\
1.1e-10	0	\\
2.2e-10	0	\\
3.3e-10	0	\\
4.4e-10	0	\\
5.4e-10	0	\\
6.5e-10	0	\\
7.5e-10	0	\\
8.6e-10	0	\\
9.6e-10	0	\\
1.07e-09	0	\\
1.18e-09	0	\\
1.28e-09	0	\\
1.38e-09	0	\\
1.49e-09	0	\\
1.59e-09	0	\\
1.69e-09	0	\\
1.8e-09	0	\\
1.9e-09	0	\\
2.01e-09	0	\\
2.11e-09	0	\\
2.21e-09	0	\\
2.32e-09	0	\\
2.42e-09	0	\\
2.52e-09	0	\\
2.63e-09	0	\\
2.73e-09	0	\\
2.83e-09	0	\\
2.93e-09	0	\\
3.04e-09	0	\\
3.14e-09	0	\\
3.24e-09	0	\\
3.34e-09	0	\\
3.45e-09	0	\\
3.55e-09	0	\\
3.65e-09	0	\\
3.75e-09	0	\\
3.86e-09	0	\\
3.96e-09	0	\\
4.06e-09	0	\\
4.16e-09	0	\\
4.27e-09	0	\\
4.37e-09	0	\\
4.47e-09	0	\\
4.57e-09	0	\\
4.68e-09	0	\\
4.78e-09	0	\\
4.89e-09	0	\\
4.99e-09	0	\\
5e-09	0	\\
};
\addplot [color=black!50!green,solid,forget plot]
  table[row sep=crcr]{
0	0	\\
1.1e-10	0	\\
2.2e-10	0	\\
3.3e-10	0	\\
4.4e-10	0	\\
5.4e-10	0	\\
6.5e-10	0	\\
7.5e-10	0	\\
8.6e-10	0	\\
9.6e-10	0	\\
1.07e-09	0	\\
1.18e-09	0	\\
1.28e-09	0	\\
1.38e-09	0	\\
1.49e-09	0	\\
1.59e-09	0	\\
1.69e-09	0	\\
1.8e-09	0	\\
1.9e-09	0	\\
2.01e-09	0	\\
2.11e-09	0	\\
2.21e-09	0	\\
2.32e-09	0	\\
2.42e-09	0	\\
2.52e-09	0	\\
2.63e-09	0	\\
2.73e-09	0	\\
2.83e-09	0	\\
2.93e-09	0	\\
3.04e-09	0	\\
3.14e-09	0	\\
3.24e-09	0	\\
3.34e-09	0	\\
3.45e-09	0	\\
3.55e-09	0	\\
3.65e-09	0	\\
3.75e-09	0	\\
3.86e-09	0	\\
3.96e-09	0	\\
4.06e-09	0	\\
4.16e-09	0	\\
4.27e-09	0	\\
4.37e-09	0	\\
4.47e-09	0	\\
4.57e-09	0	\\
4.68e-09	0	\\
4.78e-09	0	\\
4.89e-09	0	\\
4.99e-09	0	\\
5e-09	0	\\
};
\addplot [color=red,solid,forget plot]
  table[row sep=crcr]{
0	0	\\
1.1e-10	0	\\
2.2e-10	0	\\
3.3e-10	0	\\
4.4e-10	0	\\
5.4e-10	0	\\
6.5e-10	0	\\
7.5e-10	0	\\
8.6e-10	0	\\
9.6e-10	0	\\
1.07e-09	0	\\
1.18e-09	0	\\
1.28e-09	0	\\
1.38e-09	0	\\
1.49e-09	0	\\
1.59e-09	0	\\
1.69e-09	0	\\
1.8e-09	0	\\
1.9e-09	0	\\
2.01e-09	0	\\
2.11e-09	0	\\
2.21e-09	0	\\
2.32e-09	0	\\
2.42e-09	0	\\
2.52e-09	0	\\
2.63e-09	0	\\
2.73e-09	0	\\
2.83e-09	0	\\
2.93e-09	0	\\
3.04e-09	0	\\
3.14e-09	0	\\
3.24e-09	0	\\
3.34e-09	0	\\
3.45e-09	0	\\
3.55e-09	0	\\
3.65e-09	0	\\
3.75e-09	0	\\
3.86e-09	0	\\
3.96e-09	0	\\
4.06e-09	0	\\
4.16e-09	0	\\
4.27e-09	0	\\
4.37e-09	0	\\
4.47e-09	0	\\
4.57e-09	0	\\
4.68e-09	0	\\
4.78e-09	0	\\
4.89e-09	0	\\
4.99e-09	0	\\
5e-09	0	\\
};
\addplot [color=mycolor1,solid,forget plot]
  table[row sep=crcr]{
0	0	\\
1.1e-10	0	\\
2.2e-10	0	\\
3.3e-10	0	\\
4.4e-10	0	\\
5.4e-10	0	\\
6.5e-10	0	\\
7.5e-10	0	\\
8.6e-10	0	\\
9.6e-10	0	\\
1.07e-09	0	\\
1.18e-09	0	\\
1.28e-09	0	\\
1.38e-09	0	\\
1.49e-09	0	\\
1.59e-09	0	\\
1.69e-09	0	\\
1.8e-09	0	\\
1.9e-09	0	\\
2.01e-09	0	\\
2.11e-09	0	\\
2.21e-09	0	\\
2.32e-09	0	\\
2.42e-09	0	\\
2.52e-09	0	\\
2.63e-09	0	\\
2.73e-09	0	\\
2.83e-09	0	\\
2.93e-09	0	\\
3.04e-09	0	\\
3.14e-09	0	\\
3.24e-09	0	\\
3.34e-09	0	\\
3.45e-09	0	\\
3.55e-09	0	\\
3.65e-09	0	\\
3.75e-09	0	\\
3.86e-09	0	\\
3.96e-09	0	\\
4.06e-09	0	\\
4.16e-09	0	\\
4.27e-09	0	\\
4.37e-09	0	\\
4.47e-09	0	\\
4.57e-09	0	\\
4.68e-09	0	\\
4.78e-09	0	\\
4.89e-09	0	\\
4.99e-09	0	\\
5e-09	0	\\
};
\addplot [color=mycolor2,solid,forget plot]
  table[row sep=crcr]{
0	0	\\
1.1e-10	0	\\
2.2e-10	0	\\
3.3e-10	0	\\
4.4e-10	0	\\
5.4e-10	0	\\
6.5e-10	0	\\
7.5e-10	0	\\
8.6e-10	0	\\
9.6e-10	0	\\
1.07e-09	0	\\
1.18e-09	0	\\
1.28e-09	0	\\
1.38e-09	0	\\
1.49e-09	0	\\
1.59e-09	0	\\
1.69e-09	0	\\
1.8e-09	0	\\
1.9e-09	0	\\
2.01e-09	0	\\
2.11e-09	0	\\
2.21e-09	0	\\
2.32e-09	0	\\
2.42e-09	0	\\
2.52e-09	0	\\
2.63e-09	0	\\
2.73e-09	0	\\
2.83e-09	0	\\
2.93e-09	0	\\
3.04e-09	0	\\
3.14e-09	0	\\
3.24e-09	0	\\
3.34e-09	0	\\
3.45e-09	0	\\
3.55e-09	0	\\
3.65e-09	0	\\
3.75e-09	0	\\
3.86e-09	0	\\
3.96e-09	0	\\
4.06e-09	0	\\
4.16e-09	0	\\
4.27e-09	0	\\
4.37e-09	0	\\
4.47e-09	0	\\
4.57e-09	0	\\
4.68e-09	0	\\
4.78e-09	0	\\
4.89e-09	0	\\
4.99e-09	0	\\
5e-09	0	\\
};
\addplot [color=mycolor3,solid,forget plot]
  table[row sep=crcr]{
0	0	\\
1.1e-10	0	\\
2.2e-10	0	\\
3.3e-10	0	\\
4.4e-10	0	\\
5.4e-10	0	\\
6.5e-10	0	\\
7.5e-10	0	\\
8.6e-10	0	\\
9.6e-10	0	\\
1.07e-09	0	\\
1.18e-09	0	\\
1.28e-09	0	\\
1.38e-09	0	\\
1.49e-09	0	\\
1.59e-09	0	\\
1.69e-09	0	\\
1.8e-09	0	\\
1.9e-09	0	\\
2.01e-09	0	\\
2.11e-09	0	\\
2.21e-09	0	\\
2.32e-09	0	\\
2.42e-09	0	\\
2.52e-09	0	\\
2.63e-09	0	\\
2.73e-09	0	\\
2.83e-09	0	\\
2.93e-09	0	\\
3.04e-09	0	\\
3.14e-09	0	\\
3.24e-09	0	\\
3.34e-09	0	\\
3.45e-09	0	\\
3.55e-09	0	\\
3.65e-09	0	\\
3.75e-09	0	\\
3.86e-09	0	\\
3.96e-09	0	\\
4.06e-09	0	\\
4.16e-09	0	\\
4.27e-09	0	\\
4.37e-09	0	\\
4.47e-09	0	\\
4.57e-09	0	\\
4.68e-09	0	\\
4.78e-09	0	\\
4.89e-09	0	\\
4.99e-09	0	\\
5e-09	0	\\
};
\addplot [color=darkgray,solid,forget plot]
  table[row sep=crcr]{
0	0	\\
1.1e-10	0	\\
2.2e-10	0	\\
3.3e-10	0	\\
4.4e-10	0	\\
5.4e-10	0	\\
6.5e-10	0	\\
7.5e-10	0	\\
8.6e-10	0	\\
9.6e-10	0	\\
1.07e-09	0	\\
1.18e-09	0	\\
1.28e-09	0	\\
1.38e-09	0	\\
1.49e-09	0	\\
1.59e-09	0	\\
1.69e-09	0	\\
1.8e-09	0	\\
1.9e-09	0	\\
2.01e-09	0	\\
2.11e-09	0	\\
2.21e-09	0	\\
2.32e-09	0	\\
2.42e-09	0	\\
2.52e-09	0	\\
2.63e-09	0	\\
2.73e-09	0	\\
2.83e-09	0	\\
2.93e-09	0	\\
3.04e-09	0	\\
3.14e-09	0	\\
3.24e-09	0	\\
3.34e-09	0	\\
3.45e-09	0	\\
3.55e-09	0	\\
3.65e-09	0	\\
3.75e-09	0	\\
3.86e-09	0	\\
3.96e-09	0	\\
4.06e-09	0	\\
4.16e-09	0	\\
4.27e-09	0	\\
4.37e-09	0	\\
4.47e-09	0	\\
4.57e-09	0	\\
4.68e-09	0	\\
4.78e-09	0	\\
4.89e-09	0	\\
4.99e-09	0	\\
5e-09	0	\\
};
\addplot [color=blue,solid,forget plot]
  table[row sep=crcr]{
0	0	\\
1.1e-10	0	\\
2.2e-10	0	\\
3.3e-10	0	\\
4.4e-10	0	\\
5.4e-10	0	\\
6.5e-10	0	\\
7.5e-10	0	\\
8.6e-10	0	\\
9.6e-10	0	\\
1.07e-09	0	\\
1.18e-09	0	\\
1.28e-09	0	\\
1.38e-09	0	\\
1.49e-09	0	\\
1.59e-09	0	\\
1.69e-09	0	\\
1.8e-09	0	\\
1.9e-09	0	\\
2.01e-09	0	\\
2.11e-09	0	\\
2.21e-09	0	\\
2.32e-09	0	\\
2.42e-09	0	\\
2.52e-09	0	\\
2.63e-09	0	\\
2.73e-09	0	\\
2.83e-09	0	\\
2.93e-09	0	\\
3.04e-09	0	\\
3.14e-09	0	\\
3.24e-09	0	\\
3.34e-09	0	\\
3.45e-09	0	\\
3.55e-09	0	\\
3.65e-09	0	\\
3.75e-09	0	\\
3.86e-09	0	\\
3.96e-09	0	\\
4.06e-09	0	\\
4.16e-09	0	\\
4.27e-09	0	\\
4.37e-09	0	\\
4.47e-09	0	\\
4.57e-09	0	\\
4.68e-09	0	\\
4.78e-09	0	\\
4.89e-09	0	\\
4.99e-09	0	\\
5e-09	0	\\
};
\addplot [color=black!50!green,solid,forget plot]
  table[row sep=crcr]{
0	0	\\
1.1e-10	0	\\
2.2e-10	0	\\
3.3e-10	0	\\
4.4e-10	0	\\
5.4e-10	0	\\
6.5e-10	0	\\
7.5e-10	0	\\
8.6e-10	0	\\
9.6e-10	0	\\
1.07e-09	0	\\
1.18e-09	0	\\
1.28e-09	0	\\
1.38e-09	0	\\
1.49e-09	0	\\
1.59e-09	0	\\
1.69e-09	0	\\
1.8e-09	0	\\
1.9e-09	0	\\
2.01e-09	0	\\
2.11e-09	0	\\
2.21e-09	0	\\
2.32e-09	0	\\
2.42e-09	0	\\
2.52e-09	0	\\
2.63e-09	0	\\
2.73e-09	0	\\
2.83e-09	0	\\
2.93e-09	0	\\
3.04e-09	0	\\
3.14e-09	0	\\
3.24e-09	0	\\
3.34e-09	0	\\
3.45e-09	0	\\
3.55e-09	0	\\
3.65e-09	0	\\
3.75e-09	0	\\
3.86e-09	0	\\
3.96e-09	0	\\
4.06e-09	0	\\
4.16e-09	0	\\
4.27e-09	0	\\
4.37e-09	0	\\
4.47e-09	0	\\
4.57e-09	0	\\
4.68e-09	0	\\
4.78e-09	0	\\
4.89e-09	0	\\
4.99e-09	0	\\
5e-09	0	\\
};
\addplot [color=red,solid,forget plot]
  table[row sep=crcr]{
0	0	\\
1.1e-10	0	\\
2.2e-10	0	\\
3.3e-10	0	\\
4.4e-10	0	\\
5.4e-10	0	\\
6.5e-10	0	\\
7.5e-10	0	\\
8.6e-10	0	\\
9.6e-10	0	\\
1.07e-09	0	\\
1.18e-09	0	\\
1.28e-09	0	\\
1.38e-09	0	\\
1.49e-09	0	\\
1.59e-09	0	\\
1.69e-09	0	\\
1.8e-09	0	\\
1.9e-09	0	\\
2.01e-09	0	\\
2.11e-09	0	\\
2.21e-09	0	\\
2.32e-09	0	\\
2.42e-09	0	\\
2.52e-09	0	\\
2.63e-09	0	\\
2.73e-09	0	\\
2.83e-09	0	\\
2.93e-09	0	\\
3.04e-09	0	\\
3.14e-09	0	\\
3.24e-09	0	\\
3.34e-09	0	\\
3.45e-09	0	\\
3.55e-09	0	\\
3.65e-09	0	\\
3.75e-09	0	\\
3.86e-09	0	\\
3.96e-09	0	\\
4.06e-09	0	\\
4.16e-09	0	\\
4.27e-09	0	\\
4.37e-09	0	\\
4.47e-09	0	\\
4.57e-09	0	\\
4.68e-09	0	\\
4.78e-09	0	\\
4.89e-09	0	\\
4.99e-09	0	\\
5e-09	0	\\
};
\addplot [color=mycolor1,solid,forget plot]
  table[row sep=crcr]{
0	0	\\
1.1e-10	0	\\
2.2e-10	0	\\
3.3e-10	0	\\
4.4e-10	0	\\
5.4e-10	0	\\
6.5e-10	0	\\
7.5e-10	0	\\
8.6e-10	0	\\
9.6e-10	0	\\
1.07e-09	0	\\
1.18e-09	0	\\
1.28e-09	0	\\
1.38e-09	0	\\
1.49e-09	0	\\
1.59e-09	0	\\
1.69e-09	0	\\
1.8e-09	0	\\
1.9e-09	0	\\
2.01e-09	0	\\
2.11e-09	0	\\
2.21e-09	0	\\
2.32e-09	0	\\
2.42e-09	0	\\
2.52e-09	0	\\
2.63e-09	0	\\
2.73e-09	0	\\
2.83e-09	0	\\
2.93e-09	0	\\
3.04e-09	0	\\
3.14e-09	0	\\
3.24e-09	0	\\
3.34e-09	0	\\
3.45e-09	0	\\
3.55e-09	0	\\
3.65e-09	0	\\
3.75e-09	0	\\
3.86e-09	0	\\
3.96e-09	0	\\
4.06e-09	0	\\
4.16e-09	0	\\
4.27e-09	0	\\
4.37e-09	0	\\
4.47e-09	0	\\
4.57e-09	0	\\
4.68e-09	0	\\
4.78e-09	0	\\
4.89e-09	0	\\
4.99e-09	0	\\
5e-09	0	\\
};
\addplot [color=mycolor2,solid,forget plot]
  table[row sep=crcr]{
0	0	\\
1.1e-10	0	\\
2.2e-10	0	\\
3.3e-10	0	\\
4.4e-10	0	\\
5.4e-10	0	\\
6.5e-10	0	\\
7.5e-10	0	\\
8.6e-10	0	\\
9.6e-10	0	\\
1.07e-09	0	\\
1.18e-09	0	\\
1.28e-09	0	\\
1.38e-09	0	\\
1.49e-09	0	\\
1.59e-09	0	\\
1.69e-09	0	\\
1.8e-09	0	\\
1.9e-09	0	\\
2.01e-09	0	\\
2.11e-09	0	\\
2.21e-09	0	\\
2.32e-09	0	\\
2.42e-09	0	\\
2.52e-09	0	\\
2.63e-09	0	\\
2.73e-09	0	\\
2.83e-09	0	\\
2.93e-09	0	\\
3.04e-09	0	\\
3.14e-09	0	\\
3.24e-09	0	\\
3.34e-09	0	\\
3.45e-09	0	\\
3.55e-09	0	\\
3.65e-09	0	\\
3.75e-09	0	\\
3.86e-09	0	\\
3.96e-09	0	\\
4.06e-09	0	\\
4.16e-09	0	\\
4.27e-09	0	\\
4.37e-09	0	\\
4.47e-09	0	\\
4.57e-09	0	\\
4.68e-09	0	\\
4.78e-09	0	\\
4.89e-09	0	\\
4.99e-09	0	\\
5e-09	0	\\
};
\addplot [color=mycolor3,solid,forget plot]
  table[row sep=crcr]{
0	0	\\
1.1e-10	0	\\
2.2e-10	0	\\
3.3e-10	0	\\
4.4e-10	0	\\
5.4e-10	0	\\
6.5e-10	0	\\
7.5e-10	0	\\
8.6e-10	0	\\
9.6e-10	0	\\
1.07e-09	0	\\
1.18e-09	0	\\
1.28e-09	0	\\
1.38e-09	0	\\
1.49e-09	0	\\
1.59e-09	0	\\
1.69e-09	0	\\
1.8e-09	0	\\
1.9e-09	0	\\
2.01e-09	0	\\
2.11e-09	0	\\
2.21e-09	0	\\
2.32e-09	0	\\
2.42e-09	0	\\
2.52e-09	0	\\
2.63e-09	0	\\
2.73e-09	0	\\
2.83e-09	0	\\
2.93e-09	0	\\
3.04e-09	0	\\
3.14e-09	0	\\
3.24e-09	0	\\
3.34e-09	0	\\
3.45e-09	0	\\
3.55e-09	0	\\
3.65e-09	0	\\
3.75e-09	0	\\
3.86e-09	0	\\
3.96e-09	0	\\
4.06e-09	0	\\
4.16e-09	0	\\
4.27e-09	0	\\
4.37e-09	0	\\
4.47e-09	0	\\
4.57e-09	0	\\
4.68e-09	0	\\
4.78e-09	0	\\
4.89e-09	0	\\
4.99e-09	0	\\
5e-09	0	\\
};
\addplot [color=darkgray,solid,forget plot]
  table[row sep=crcr]{
0	0	\\
1.1e-10	0	\\
2.2e-10	0	\\
3.3e-10	0	\\
4.4e-10	0	\\
5.4e-10	0	\\
6.5e-10	0	\\
7.5e-10	0	\\
8.6e-10	0	\\
9.6e-10	0	\\
1.07e-09	0	\\
1.18e-09	0	\\
1.28e-09	0	\\
1.38e-09	0	\\
1.49e-09	0	\\
1.59e-09	0	\\
1.69e-09	0	\\
1.8e-09	0	\\
1.9e-09	0	\\
2.01e-09	0	\\
2.11e-09	0	\\
2.21e-09	0	\\
2.32e-09	0	\\
2.42e-09	0	\\
2.52e-09	0	\\
2.63e-09	0	\\
2.73e-09	0	\\
2.83e-09	0	\\
2.93e-09	0	\\
3.04e-09	0	\\
3.14e-09	0	\\
3.24e-09	0	\\
3.34e-09	0	\\
3.45e-09	0	\\
3.55e-09	0	\\
3.65e-09	0	\\
3.75e-09	0	\\
3.86e-09	0	\\
3.96e-09	0	\\
4.06e-09	0	\\
4.16e-09	0	\\
4.27e-09	0	\\
4.37e-09	0	\\
4.47e-09	0	\\
4.57e-09	0	\\
4.68e-09	0	\\
4.78e-09	0	\\
4.89e-09	0	\\
4.99e-09	0	\\
5e-09	0	\\
};
\addplot [color=blue,solid,forget plot]
  table[row sep=crcr]{
0	0	\\
1.1e-10	0	\\
2.2e-10	0	\\
3.3e-10	0	\\
4.4e-10	0	\\
5.4e-10	0	\\
6.5e-10	0	\\
7.5e-10	0	\\
8.6e-10	0	\\
9.6e-10	0	\\
1.07e-09	0	\\
1.18e-09	0	\\
1.28e-09	0	\\
1.38e-09	0	\\
1.49e-09	0	\\
1.59e-09	0	\\
1.69e-09	0	\\
1.8e-09	0	\\
1.9e-09	0	\\
2.01e-09	0	\\
2.11e-09	0	\\
2.21e-09	0	\\
2.32e-09	0	\\
2.42e-09	0	\\
2.52e-09	0	\\
2.63e-09	0	\\
2.73e-09	0	\\
2.83e-09	0	\\
2.93e-09	0	\\
3.04e-09	0	\\
3.14e-09	0	\\
3.24e-09	0	\\
3.34e-09	0	\\
3.45e-09	0	\\
3.55e-09	0	\\
3.65e-09	0	\\
3.75e-09	0	\\
3.86e-09	0	\\
3.96e-09	0	\\
4.06e-09	0	\\
4.16e-09	0	\\
4.27e-09	0	\\
4.37e-09	0	\\
4.47e-09	0	\\
4.57e-09	0	\\
4.68e-09	0	\\
4.78e-09	0	\\
4.89e-09	0	\\
4.99e-09	0	\\
5e-09	0	\\
};
\addplot [color=black!50!green,solid,forget plot]
  table[row sep=crcr]{
0	0	\\
1.1e-10	0	\\
2.2e-10	0	\\
3.3e-10	0	\\
4.4e-10	0	\\
5.4e-10	0	\\
6.5e-10	0	\\
7.5e-10	0	\\
8.6e-10	0	\\
9.6e-10	0	\\
1.07e-09	0	\\
1.18e-09	0	\\
1.28e-09	0	\\
1.38e-09	0	\\
1.49e-09	0	\\
1.59e-09	0	\\
1.69e-09	0	\\
1.8e-09	0	\\
1.9e-09	0	\\
2.01e-09	0	\\
2.11e-09	0	\\
2.21e-09	0	\\
2.32e-09	0	\\
2.42e-09	0	\\
2.52e-09	0	\\
2.63e-09	0	\\
2.73e-09	0	\\
2.83e-09	0	\\
2.93e-09	0	\\
3.04e-09	0	\\
3.14e-09	0	\\
3.24e-09	0	\\
3.34e-09	0	\\
3.45e-09	0	\\
3.55e-09	0	\\
3.65e-09	0	\\
3.75e-09	0	\\
3.86e-09	0	\\
3.96e-09	0	\\
4.06e-09	0	\\
4.16e-09	0	\\
4.27e-09	0	\\
4.37e-09	0	\\
4.47e-09	0	\\
4.57e-09	0	\\
4.68e-09	0	\\
4.78e-09	0	\\
4.89e-09	0	\\
4.99e-09	0	\\
5e-09	0	\\
};
\addplot [color=red,solid,forget plot]
  table[row sep=crcr]{
0	0	\\
1.1e-10	0	\\
2.2e-10	0	\\
3.3e-10	0	\\
4.4e-10	0	\\
5.4e-10	0	\\
6.5e-10	0	\\
7.5e-10	0	\\
8.6e-10	0	\\
9.6e-10	0	\\
1.07e-09	0	\\
1.18e-09	0	\\
1.28e-09	0	\\
1.38e-09	0	\\
1.49e-09	0	\\
1.59e-09	0	\\
1.69e-09	0	\\
1.8e-09	0	\\
1.9e-09	0	\\
2.01e-09	0	\\
2.11e-09	0	\\
2.21e-09	0	\\
2.32e-09	0	\\
2.42e-09	0	\\
2.52e-09	0	\\
2.63e-09	0	\\
2.73e-09	0	\\
2.83e-09	0	\\
2.93e-09	0	\\
3.04e-09	0	\\
3.14e-09	0	\\
3.24e-09	0	\\
3.34e-09	0	\\
3.45e-09	0	\\
3.55e-09	0	\\
3.65e-09	0	\\
3.75e-09	0	\\
3.86e-09	0	\\
3.96e-09	0	\\
4.06e-09	0	\\
4.16e-09	0	\\
4.27e-09	0	\\
4.37e-09	0	\\
4.47e-09	0	\\
4.57e-09	0	\\
4.68e-09	0	\\
4.78e-09	0	\\
4.89e-09	0	\\
4.99e-09	0	\\
5e-09	0	\\
};
\addplot [color=mycolor1,solid,forget plot]
  table[row sep=crcr]{
0	0	\\
1.1e-10	0	\\
2.2e-10	0	\\
3.3e-10	0	\\
4.4e-10	0	\\
5.4e-10	0	\\
6.5e-10	0	\\
7.5e-10	0	\\
8.6e-10	0	\\
9.6e-10	0	\\
1.07e-09	0	\\
1.18e-09	0	\\
1.28e-09	0	\\
1.38e-09	0	\\
1.49e-09	0	\\
1.59e-09	0	\\
1.69e-09	0	\\
1.8e-09	0	\\
1.9e-09	0	\\
2.01e-09	0	\\
2.11e-09	0	\\
2.21e-09	0	\\
2.32e-09	0	\\
2.42e-09	0	\\
2.52e-09	0	\\
2.63e-09	0	\\
2.73e-09	0	\\
2.83e-09	0	\\
2.93e-09	0	\\
3.04e-09	0	\\
3.14e-09	0	\\
3.24e-09	0	\\
3.34e-09	0	\\
3.45e-09	0	\\
3.55e-09	0	\\
3.65e-09	0	\\
3.75e-09	0	\\
3.86e-09	0	\\
3.96e-09	0	\\
4.06e-09	0	\\
4.16e-09	0	\\
4.27e-09	0	\\
4.37e-09	0	\\
4.47e-09	0	\\
4.57e-09	0	\\
4.68e-09	0	\\
4.78e-09	0	\\
4.89e-09	0	\\
4.99e-09	0	\\
5e-09	0	\\
};
\addplot [color=mycolor2,solid,forget plot]
  table[row sep=crcr]{
0	0	\\
1.1e-10	0	\\
2.2e-10	0	\\
3.3e-10	0	\\
4.4e-10	0	\\
5.4e-10	0	\\
6.5e-10	0	\\
7.5e-10	0	\\
8.6e-10	0	\\
9.6e-10	0	\\
1.07e-09	0	\\
1.18e-09	0	\\
1.28e-09	0	\\
1.38e-09	0	\\
1.49e-09	0	\\
1.59e-09	0	\\
1.69e-09	0	\\
1.8e-09	0	\\
1.9e-09	0	\\
2.01e-09	0	\\
2.11e-09	0	\\
2.21e-09	0	\\
2.32e-09	0	\\
2.42e-09	0	\\
2.52e-09	0	\\
2.63e-09	0	\\
2.73e-09	0	\\
2.83e-09	0	\\
2.93e-09	0	\\
3.04e-09	0	\\
3.14e-09	0	\\
3.24e-09	0	\\
3.34e-09	0	\\
3.45e-09	0	\\
3.55e-09	0	\\
3.65e-09	0	\\
3.75e-09	0	\\
3.86e-09	0	\\
3.96e-09	0	\\
4.06e-09	0	\\
4.16e-09	0	\\
4.27e-09	0	\\
4.37e-09	0	\\
4.47e-09	0	\\
4.57e-09	0	\\
4.68e-09	0	\\
4.78e-09	0	\\
4.89e-09	0	\\
4.99e-09	0	\\
5e-09	0	\\
};
\addplot [color=mycolor3,solid,forget plot]
  table[row sep=crcr]{
0	0	\\
1.1e-10	0	\\
2.2e-10	0	\\
3.3e-10	0	\\
4.4e-10	0	\\
5.4e-10	0	\\
6.5e-10	0	\\
7.5e-10	0	\\
8.6e-10	0	\\
9.6e-10	0	\\
1.07e-09	0	\\
1.18e-09	0	\\
1.28e-09	0	\\
1.38e-09	0	\\
1.49e-09	0	\\
1.59e-09	0	\\
1.69e-09	0	\\
1.8e-09	0	\\
1.9e-09	0	\\
2.01e-09	0	\\
2.11e-09	0	\\
2.21e-09	0	\\
2.32e-09	0	\\
2.42e-09	0	\\
2.52e-09	0	\\
2.63e-09	0	\\
2.73e-09	0	\\
2.83e-09	0	\\
2.93e-09	0	\\
3.04e-09	0	\\
3.14e-09	0	\\
3.24e-09	0	\\
3.34e-09	0	\\
3.45e-09	0	\\
3.55e-09	0	\\
3.65e-09	0	\\
3.75e-09	0	\\
3.86e-09	0	\\
3.96e-09	0	\\
4.06e-09	0	\\
4.16e-09	0	\\
4.27e-09	0	\\
4.37e-09	0	\\
4.47e-09	0	\\
4.57e-09	0	\\
4.68e-09	0	\\
4.78e-09	0	\\
4.89e-09	0	\\
4.99e-09	0	\\
5e-09	0	\\
};
\addplot [color=darkgray,solid,forget plot]
  table[row sep=crcr]{
0	0	\\
1.1e-10	0	\\
2.2e-10	0	\\
3.3e-10	0	\\
4.4e-10	0	\\
5.4e-10	0	\\
6.5e-10	0	\\
7.5e-10	0	\\
8.6e-10	0	\\
9.6e-10	0	\\
1.07e-09	0	\\
1.18e-09	0	\\
1.28e-09	0	\\
1.38e-09	0	\\
1.49e-09	0	\\
1.59e-09	0	\\
1.69e-09	0	\\
1.8e-09	0	\\
1.9e-09	0	\\
2.01e-09	0	\\
2.11e-09	0	\\
2.21e-09	0	\\
2.32e-09	0	\\
2.42e-09	0	\\
2.52e-09	0	\\
2.63e-09	0	\\
2.73e-09	0	\\
2.83e-09	0	\\
2.93e-09	0	\\
3.04e-09	0	\\
3.14e-09	0	\\
3.24e-09	0	\\
3.34e-09	0	\\
3.45e-09	0	\\
3.55e-09	0	\\
3.65e-09	0	\\
3.75e-09	0	\\
3.86e-09	0	\\
3.96e-09	0	\\
4.06e-09	0	\\
4.16e-09	0	\\
4.27e-09	0	\\
4.37e-09	0	\\
4.47e-09	0	\\
4.57e-09	0	\\
4.68e-09	0	\\
4.78e-09	0	\\
4.89e-09	0	\\
4.99e-09	0	\\
5e-09	0	\\
};
\addplot [color=blue,solid,forget plot]
  table[row sep=crcr]{
0	0	\\
1.1e-10	0	\\
2.2e-10	0	\\
3.3e-10	0	\\
4.4e-10	0	\\
5.4e-10	0	\\
6.5e-10	0	\\
7.5e-10	0	\\
8.6e-10	0	\\
9.6e-10	0	\\
1.07e-09	0	\\
1.18e-09	0	\\
1.28e-09	0	\\
1.38e-09	0	\\
1.49e-09	0	\\
1.59e-09	0	\\
1.69e-09	0	\\
1.8e-09	0	\\
1.9e-09	0	\\
2.01e-09	0	\\
2.11e-09	0	\\
2.21e-09	0	\\
2.32e-09	0	\\
2.42e-09	0	\\
2.52e-09	0	\\
2.63e-09	0	\\
2.73e-09	0	\\
2.83e-09	0	\\
2.93e-09	0	\\
3.04e-09	0	\\
3.14e-09	0	\\
3.24e-09	0	\\
3.34e-09	0	\\
3.45e-09	0	\\
3.55e-09	0	\\
3.65e-09	0	\\
3.75e-09	0	\\
3.86e-09	0	\\
3.96e-09	0	\\
4.06e-09	0	\\
4.16e-09	0	\\
4.27e-09	0	\\
4.37e-09	0	\\
4.47e-09	0	\\
4.57e-09	0	\\
4.68e-09	0	\\
4.78e-09	0	\\
4.89e-09	0	\\
4.99e-09	0	\\
5e-09	0	\\
};
\addplot [color=black!50!green,solid,forget plot]
  table[row sep=crcr]{
0	0	\\
1.1e-10	0	\\
2.2e-10	0	\\
3.3e-10	0	\\
4.4e-10	0	\\
5.4e-10	0	\\
6.5e-10	0	\\
7.5e-10	0	\\
8.6e-10	0	\\
9.6e-10	0	\\
1.07e-09	0	\\
1.18e-09	0	\\
1.28e-09	0	\\
1.38e-09	0	\\
1.49e-09	0	\\
1.59e-09	0	\\
1.69e-09	0	\\
1.8e-09	0	\\
1.9e-09	0	\\
2.01e-09	0	\\
2.11e-09	0	\\
2.21e-09	0	\\
2.32e-09	0	\\
2.42e-09	0	\\
2.52e-09	0	\\
2.63e-09	0	\\
2.73e-09	0	\\
2.83e-09	0	\\
2.93e-09	0	\\
3.04e-09	0	\\
3.14e-09	0	\\
3.24e-09	0	\\
3.34e-09	0	\\
3.45e-09	0	\\
3.55e-09	0	\\
3.65e-09	0	\\
3.75e-09	0	\\
3.86e-09	0	\\
3.96e-09	0	\\
4.06e-09	0	\\
4.16e-09	0	\\
4.27e-09	0	\\
4.37e-09	0	\\
4.47e-09	0	\\
4.57e-09	0	\\
4.68e-09	0	\\
4.78e-09	0	\\
4.89e-09	0	\\
4.99e-09	0	\\
5e-09	0	\\
};
\addplot [color=red,solid,forget plot]
  table[row sep=crcr]{
0	0	\\
1.1e-10	0	\\
2.2e-10	0	\\
3.3e-10	0	\\
4.4e-10	0	\\
5.4e-10	0	\\
6.5e-10	0	\\
7.5e-10	0	\\
8.6e-10	0	\\
9.6e-10	0	\\
1.07e-09	0	\\
1.18e-09	0	\\
1.28e-09	0	\\
1.38e-09	0	\\
1.49e-09	0	\\
1.59e-09	0	\\
1.69e-09	0	\\
1.8e-09	0	\\
1.9e-09	0	\\
2.01e-09	0	\\
2.11e-09	0	\\
2.21e-09	0	\\
2.32e-09	0	\\
2.42e-09	0	\\
2.52e-09	0	\\
2.63e-09	0	\\
2.73e-09	0	\\
2.83e-09	0	\\
2.93e-09	0	\\
3.04e-09	0	\\
3.14e-09	0	\\
3.24e-09	0	\\
3.34e-09	0	\\
3.45e-09	0	\\
3.55e-09	0	\\
3.65e-09	0	\\
3.75e-09	0	\\
3.86e-09	0	\\
3.96e-09	0	\\
4.06e-09	0	\\
4.16e-09	0	\\
4.27e-09	0	\\
4.37e-09	0	\\
4.47e-09	0	\\
4.57e-09	0	\\
4.68e-09	0	\\
4.78e-09	0	\\
4.89e-09	0	\\
4.99e-09	0	\\
5e-09	0	\\
};
\addplot [color=mycolor1,solid,forget plot]
  table[row sep=crcr]{
0	0	\\
1.1e-10	0	\\
2.2e-10	0	\\
3.3e-10	0	\\
4.4e-10	0	\\
5.4e-10	0	\\
6.5e-10	0	\\
7.5e-10	0	\\
8.6e-10	0	\\
9.6e-10	0	\\
1.07e-09	0	\\
1.18e-09	0	\\
1.28e-09	0	\\
1.38e-09	0	\\
1.49e-09	0	\\
1.59e-09	0	\\
1.69e-09	0	\\
1.8e-09	0	\\
1.9e-09	0	\\
2.01e-09	0	\\
2.11e-09	0	\\
2.21e-09	0	\\
2.32e-09	0	\\
2.42e-09	0	\\
2.52e-09	0	\\
2.63e-09	0	\\
2.73e-09	0	\\
2.83e-09	0	\\
2.93e-09	0	\\
3.04e-09	0	\\
3.14e-09	0	\\
3.24e-09	0	\\
3.34e-09	0	\\
3.45e-09	0	\\
3.55e-09	0	\\
3.65e-09	0	\\
3.75e-09	0	\\
3.86e-09	0	\\
3.96e-09	0	\\
4.06e-09	0	\\
4.16e-09	0	\\
4.27e-09	0	\\
4.37e-09	0	\\
4.47e-09	0	\\
4.57e-09	0	\\
4.68e-09	0	\\
4.78e-09	0	\\
4.89e-09	0	\\
4.99e-09	0	\\
5e-09	0	\\
};
\addplot [color=mycolor2,solid,forget plot]
  table[row sep=crcr]{
0	0	\\
1.1e-10	0	\\
2.2e-10	0	\\
3.3e-10	0	\\
4.4e-10	0	\\
5.4e-10	0	\\
6.5e-10	0	\\
7.5e-10	0	\\
8.6e-10	0	\\
9.6e-10	0	\\
1.07e-09	0	\\
1.18e-09	0	\\
1.28e-09	0	\\
1.38e-09	0	\\
1.49e-09	0	\\
1.59e-09	0	\\
1.69e-09	0	\\
1.8e-09	0	\\
1.9e-09	0	\\
2.01e-09	0	\\
2.11e-09	0	\\
2.21e-09	0	\\
2.32e-09	0	\\
2.42e-09	0	\\
2.52e-09	0	\\
2.63e-09	0	\\
2.73e-09	0	\\
2.83e-09	0	\\
2.93e-09	0	\\
3.04e-09	0	\\
3.14e-09	0	\\
3.24e-09	0	\\
3.34e-09	0	\\
3.45e-09	0	\\
3.55e-09	0	\\
3.65e-09	0	\\
3.75e-09	0	\\
3.86e-09	0	\\
3.96e-09	0	\\
4.06e-09	0	\\
4.16e-09	0	\\
4.27e-09	0	\\
4.37e-09	0	\\
4.47e-09	0	\\
4.57e-09	0	\\
4.68e-09	0	\\
4.78e-09	0	\\
4.89e-09	0	\\
4.99e-09	0	\\
5e-09	0	\\
};
\addplot [color=mycolor3,solid,forget plot]
  table[row sep=crcr]{
0	0	\\
1.1e-10	0	\\
2.2e-10	0	\\
3.3e-10	0	\\
4.4e-10	0	\\
5.4e-10	0	\\
6.5e-10	0	\\
7.5e-10	0	\\
8.6e-10	0	\\
9.6e-10	0	\\
1.07e-09	0	\\
1.18e-09	0	\\
1.28e-09	0	\\
1.38e-09	0	\\
1.49e-09	0	\\
1.59e-09	0	\\
1.69e-09	0	\\
1.8e-09	0	\\
1.9e-09	0	\\
2.01e-09	0	\\
2.11e-09	0	\\
2.21e-09	0	\\
2.32e-09	0	\\
2.42e-09	0	\\
2.52e-09	0	\\
2.63e-09	0	\\
2.73e-09	0	\\
2.83e-09	0	\\
2.93e-09	0	\\
3.04e-09	0	\\
3.14e-09	0	\\
3.24e-09	0	\\
3.34e-09	0	\\
3.45e-09	0	\\
3.55e-09	0	\\
3.65e-09	0	\\
3.75e-09	0	\\
3.86e-09	0	\\
3.96e-09	0	\\
4.06e-09	0	\\
4.16e-09	0	\\
4.27e-09	0	\\
4.37e-09	0	\\
4.47e-09	0	\\
4.57e-09	0	\\
4.68e-09	0	\\
4.78e-09	0	\\
4.89e-09	0	\\
4.99e-09	0	\\
5e-09	0	\\
};
\addplot [color=darkgray,solid,forget plot]
  table[row sep=crcr]{
0	0	\\
1.1e-10	0	\\
2.2e-10	0	\\
3.3e-10	0	\\
4.4e-10	0	\\
5.4e-10	0	\\
6.5e-10	0	\\
7.5e-10	0	\\
8.6e-10	0	\\
9.6e-10	0	\\
1.07e-09	0	\\
1.18e-09	0	\\
1.28e-09	0	\\
1.38e-09	0	\\
1.49e-09	0	\\
1.59e-09	0	\\
1.69e-09	0	\\
1.8e-09	0	\\
1.9e-09	0	\\
2.01e-09	0	\\
2.11e-09	0	\\
2.21e-09	0	\\
2.32e-09	0	\\
2.42e-09	0	\\
2.52e-09	0	\\
2.63e-09	0	\\
2.73e-09	0	\\
2.83e-09	0	\\
2.93e-09	0	\\
3.04e-09	0	\\
3.14e-09	0	\\
3.24e-09	0	\\
3.34e-09	0	\\
3.45e-09	0	\\
3.55e-09	0	\\
3.65e-09	0	\\
3.75e-09	0	\\
3.86e-09	0	\\
3.96e-09	0	\\
4.06e-09	0	\\
4.16e-09	0	\\
4.27e-09	0	\\
4.37e-09	0	\\
4.47e-09	0	\\
4.57e-09	0	\\
4.68e-09	0	\\
4.78e-09	0	\\
4.89e-09	0	\\
4.99e-09	0	\\
5e-09	0	\\
};
\addplot [color=blue,solid,forget plot]
  table[row sep=crcr]{
0	0	\\
1.1e-10	0	\\
2.2e-10	0	\\
3.3e-10	0	\\
4.4e-10	0	\\
5.4e-10	0	\\
6.5e-10	0	\\
7.5e-10	0	\\
8.6e-10	0	\\
9.6e-10	0	\\
1.07e-09	0	\\
1.18e-09	0	\\
1.28e-09	0	\\
1.38e-09	0	\\
1.49e-09	0	\\
1.59e-09	0	\\
1.69e-09	0	\\
1.8e-09	0	\\
1.9e-09	0	\\
2.01e-09	0	\\
2.11e-09	0	\\
2.21e-09	0	\\
2.32e-09	0	\\
2.42e-09	0	\\
2.52e-09	0	\\
2.63e-09	0	\\
2.73e-09	0	\\
2.83e-09	0	\\
2.93e-09	0	\\
3.04e-09	0	\\
3.14e-09	0	\\
3.24e-09	0	\\
3.34e-09	0	\\
3.45e-09	0	\\
3.55e-09	0	\\
3.65e-09	0	\\
3.75e-09	0	\\
3.86e-09	0	\\
3.96e-09	0	\\
4.06e-09	0	\\
4.16e-09	0	\\
4.27e-09	0	\\
4.37e-09	0	\\
4.47e-09	0	\\
4.57e-09	0	\\
4.68e-09	0	\\
4.78e-09	0	\\
4.89e-09	0	\\
4.99e-09	0	\\
5e-09	0	\\
};
\addplot [color=black!50!green,solid,forget plot]
  table[row sep=crcr]{
0	0	\\
1.1e-10	0	\\
2.2e-10	0	\\
3.3e-10	0	\\
4.4e-10	0	\\
5.4e-10	0	\\
6.5e-10	0	\\
7.5e-10	0	\\
8.6e-10	0	\\
9.6e-10	0	\\
1.07e-09	0	\\
1.18e-09	0	\\
1.28e-09	0	\\
1.38e-09	0	\\
1.49e-09	0	\\
1.59e-09	0	\\
1.69e-09	0	\\
1.8e-09	0	\\
1.9e-09	0	\\
2.01e-09	0	\\
2.11e-09	0	\\
2.21e-09	0	\\
2.32e-09	0	\\
2.42e-09	0	\\
2.52e-09	0	\\
2.63e-09	0	\\
2.73e-09	0	\\
2.83e-09	0	\\
2.93e-09	0	\\
3.04e-09	0	\\
3.14e-09	0	\\
3.24e-09	0	\\
3.34e-09	0	\\
3.45e-09	0	\\
3.55e-09	0	\\
3.65e-09	0	\\
3.75e-09	0	\\
3.86e-09	0	\\
3.96e-09	0	\\
4.06e-09	0	\\
4.16e-09	0	\\
4.27e-09	0	\\
4.37e-09	0	\\
4.47e-09	0	\\
4.57e-09	0	\\
4.68e-09	0	\\
4.78e-09	0	\\
4.89e-09	0	\\
4.99e-09	0	\\
5e-09	0	\\
};
\addplot [color=red,solid,forget plot]
  table[row sep=crcr]{
0	0	\\
1.1e-10	0	\\
2.2e-10	0	\\
3.3e-10	0	\\
4.4e-10	0	\\
5.4e-10	0	\\
6.5e-10	0	\\
7.5e-10	0	\\
8.6e-10	0	\\
9.6e-10	0	\\
1.07e-09	0	\\
1.18e-09	0	\\
1.28e-09	0	\\
1.38e-09	0	\\
1.49e-09	0	\\
1.59e-09	0	\\
1.69e-09	0	\\
1.8e-09	0	\\
1.9e-09	0	\\
2.01e-09	0	\\
2.11e-09	0	\\
2.21e-09	0	\\
2.32e-09	0	\\
2.42e-09	0	\\
2.52e-09	0	\\
2.63e-09	0	\\
2.73e-09	0	\\
2.83e-09	0	\\
2.93e-09	0	\\
3.04e-09	0	\\
3.14e-09	0	\\
3.24e-09	0	\\
3.34e-09	0	\\
3.45e-09	0	\\
3.55e-09	0	\\
3.65e-09	0	\\
3.75e-09	0	\\
3.86e-09	0	\\
3.96e-09	0	\\
4.06e-09	0	\\
4.16e-09	0	\\
4.27e-09	0	\\
4.37e-09	0	\\
4.47e-09	0	\\
4.57e-09	0	\\
4.68e-09	0	\\
4.78e-09	0	\\
4.89e-09	0	\\
4.99e-09	0	\\
5e-09	0	\\
};
\addplot [color=mycolor1,solid,forget plot]
  table[row sep=crcr]{
0	0	\\
1.1e-10	0	\\
2.2e-10	0	\\
3.3e-10	0	\\
4.4e-10	0	\\
5.4e-10	0	\\
6.5e-10	0	\\
7.5e-10	0	\\
8.6e-10	0	\\
9.6e-10	0	\\
1.07e-09	0	\\
1.18e-09	0	\\
1.28e-09	0	\\
1.38e-09	0	\\
1.49e-09	0	\\
1.59e-09	0	\\
1.69e-09	0	\\
1.8e-09	0	\\
1.9e-09	0	\\
2.01e-09	0	\\
2.11e-09	0	\\
2.21e-09	0	\\
2.32e-09	0	\\
2.42e-09	0	\\
2.52e-09	0	\\
2.63e-09	0	\\
2.73e-09	0	\\
2.83e-09	0	\\
2.93e-09	0	\\
3.04e-09	0	\\
3.14e-09	0	\\
3.24e-09	0	\\
3.34e-09	0	\\
3.45e-09	0	\\
3.55e-09	0	\\
3.65e-09	0	\\
3.75e-09	0	\\
3.86e-09	0	\\
3.96e-09	0	\\
4.06e-09	0	\\
4.16e-09	0	\\
4.27e-09	0	\\
4.37e-09	0	\\
4.47e-09	0	\\
4.57e-09	0	\\
4.68e-09	0	\\
4.78e-09	0	\\
4.89e-09	0	\\
4.99e-09	0	\\
5e-09	0	\\
};
\addplot [color=mycolor2,solid,forget plot]
  table[row sep=crcr]{
0	0	\\
1.1e-10	0	\\
2.2e-10	0	\\
3.3e-10	0	\\
4.4e-10	0	\\
5.4e-10	0	\\
6.5e-10	0	\\
7.5e-10	0	\\
8.6e-10	0	\\
9.6e-10	0	\\
1.07e-09	0	\\
1.18e-09	0	\\
1.28e-09	0	\\
1.38e-09	0	\\
1.49e-09	0	\\
1.59e-09	0	\\
1.69e-09	0	\\
1.8e-09	0	\\
1.9e-09	0	\\
2.01e-09	0	\\
2.11e-09	0	\\
2.21e-09	0	\\
2.32e-09	0	\\
2.42e-09	0	\\
2.52e-09	0	\\
2.63e-09	0	\\
2.73e-09	0	\\
2.83e-09	0	\\
2.93e-09	0	\\
3.04e-09	0	\\
3.14e-09	0	\\
3.24e-09	0	\\
3.34e-09	0	\\
3.45e-09	0	\\
3.55e-09	0	\\
3.65e-09	0	\\
3.75e-09	0	\\
3.86e-09	0	\\
3.96e-09	0	\\
4.06e-09	0	\\
4.16e-09	0	\\
4.27e-09	0	\\
4.37e-09	0	\\
4.47e-09	0	\\
4.57e-09	0	\\
4.68e-09	0	\\
4.78e-09	0	\\
4.89e-09	0	\\
4.99e-09	0	\\
5e-09	0	\\
};
\addplot [color=mycolor3,solid,forget plot]
  table[row sep=crcr]{
0	0	\\
1.1e-10	0	\\
2.2e-10	0	\\
3.3e-10	0	\\
4.4e-10	0	\\
5.4e-10	0	\\
6.5e-10	0	\\
7.5e-10	0	\\
8.6e-10	0	\\
9.6e-10	0	\\
1.07e-09	0	\\
1.18e-09	0	\\
1.28e-09	0	\\
1.38e-09	0	\\
1.49e-09	0	\\
1.59e-09	0	\\
1.69e-09	0	\\
1.8e-09	0	\\
1.9e-09	0	\\
2.01e-09	0	\\
2.11e-09	0	\\
2.21e-09	0	\\
2.32e-09	0	\\
2.42e-09	0	\\
2.52e-09	0	\\
2.63e-09	0	\\
2.73e-09	0	\\
2.83e-09	0	\\
2.93e-09	0	\\
3.04e-09	0	\\
3.14e-09	0	\\
3.24e-09	0	\\
3.34e-09	0	\\
3.45e-09	0	\\
3.55e-09	0	\\
3.65e-09	0	\\
3.75e-09	0	\\
3.86e-09	0	\\
3.96e-09	0	\\
4.06e-09	0	\\
4.16e-09	0	\\
4.27e-09	0	\\
4.37e-09	0	\\
4.47e-09	0	\\
4.57e-09	0	\\
4.68e-09	0	\\
4.78e-09	0	\\
4.89e-09	0	\\
4.99e-09	0	\\
5e-09	0	\\
};
\addplot [color=darkgray,solid,forget plot]
  table[row sep=crcr]{
0	0	\\
1.1e-10	0	\\
2.2e-10	0	\\
3.3e-10	0	\\
4.4e-10	0	\\
5.4e-10	0	\\
6.5e-10	0	\\
7.5e-10	0	\\
8.6e-10	0	\\
9.6e-10	0	\\
1.07e-09	0	\\
1.18e-09	0	\\
1.28e-09	0	\\
1.38e-09	0	\\
1.49e-09	0	\\
1.59e-09	0	\\
1.69e-09	0	\\
1.8e-09	0	\\
1.9e-09	0	\\
2.01e-09	0	\\
2.11e-09	0	\\
2.21e-09	0	\\
2.32e-09	0	\\
2.42e-09	0	\\
2.52e-09	0	\\
2.63e-09	0	\\
2.73e-09	0	\\
2.83e-09	0	\\
2.93e-09	0	\\
3.04e-09	0	\\
3.14e-09	0	\\
3.24e-09	0	\\
3.34e-09	0	\\
3.45e-09	0	\\
3.55e-09	0	\\
3.65e-09	0	\\
3.75e-09	0	\\
3.86e-09	0	\\
3.96e-09	0	\\
4.06e-09	0	\\
4.16e-09	0	\\
4.27e-09	0	\\
4.37e-09	0	\\
4.47e-09	0	\\
4.57e-09	0	\\
4.68e-09	0	\\
4.78e-09	0	\\
4.89e-09	0	\\
4.99e-09	0	\\
5e-09	0	\\
};
\addplot [color=blue,solid,forget plot]
  table[row sep=crcr]{
0	0	\\
1.1e-10	0	\\
2.2e-10	0	\\
3.3e-10	0	\\
4.4e-10	0	\\
5.4e-10	0	\\
6.5e-10	0	\\
7.5e-10	0	\\
8.6e-10	0	\\
9.6e-10	0	\\
1.07e-09	0	\\
1.18e-09	0	\\
1.28e-09	0	\\
1.38e-09	0	\\
1.49e-09	0	\\
1.59e-09	0	\\
1.69e-09	0	\\
1.8e-09	0	\\
1.9e-09	0	\\
2.01e-09	0	\\
2.11e-09	0	\\
2.21e-09	0	\\
2.32e-09	0	\\
2.42e-09	0	\\
2.52e-09	0	\\
2.63e-09	0	\\
2.73e-09	0	\\
2.83e-09	0	\\
2.93e-09	0	\\
3.04e-09	0	\\
3.14e-09	0	\\
3.24e-09	0	\\
3.34e-09	0	\\
3.45e-09	0	\\
3.55e-09	0	\\
3.65e-09	0	\\
3.75e-09	0	\\
3.86e-09	0	\\
3.96e-09	0	\\
4.06e-09	0	\\
4.16e-09	0	\\
4.27e-09	0	\\
4.37e-09	0	\\
4.47e-09	0	\\
4.57e-09	0	\\
4.68e-09	0	\\
4.78e-09	0	\\
4.89e-09	0	\\
4.99e-09	0	\\
5e-09	0	\\
};
\addplot [color=black!50!green,solid,forget plot]
  table[row sep=crcr]{
0	0	\\
1.1e-10	0	\\
2.2e-10	0	\\
3.3e-10	0	\\
4.4e-10	0	\\
5.4e-10	0	\\
6.5e-10	0	\\
7.5e-10	0	\\
8.6e-10	0	\\
9.6e-10	0	\\
1.07e-09	0	\\
1.18e-09	0	\\
1.28e-09	0	\\
1.38e-09	0	\\
1.49e-09	0	\\
1.59e-09	0	\\
1.69e-09	0	\\
1.8e-09	0	\\
1.9e-09	0	\\
2.01e-09	0	\\
2.11e-09	0	\\
2.21e-09	0	\\
2.32e-09	0	\\
2.42e-09	0	\\
2.52e-09	0	\\
2.63e-09	0	\\
2.73e-09	0	\\
2.83e-09	0	\\
2.93e-09	0	\\
3.04e-09	0	\\
3.14e-09	0	\\
3.24e-09	0	\\
3.34e-09	0	\\
3.45e-09	0	\\
3.55e-09	0	\\
3.65e-09	0	\\
3.75e-09	0	\\
3.86e-09	0	\\
3.96e-09	0	\\
4.06e-09	0	\\
4.16e-09	0	\\
4.27e-09	0	\\
4.37e-09	0	\\
4.47e-09	0	\\
4.57e-09	0	\\
4.68e-09	0	\\
4.78e-09	0	\\
4.89e-09	0	\\
4.99e-09	0	\\
5e-09	0	\\
};
\addplot [color=red,solid,forget plot]
  table[row sep=crcr]{
0	0	\\
1.1e-10	0	\\
2.2e-10	0	\\
3.3e-10	0	\\
4.4e-10	0	\\
5.4e-10	0	\\
6.5e-10	0	\\
7.5e-10	0	\\
8.6e-10	0	\\
9.6e-10	0	\\
1.07e-09	0	\\
1.18e-09	0	\\
1.28e-09	0	\\
1.38e-09	0	\\
1.49e-09	0	\\
1.59e-09	0	\\
1.69e-09	0	\\
1.8e-09	0	\\
1.9e-09	0	\\
2.01e-09	0	\\
2.11e-09	0	\\
2.21e-09	0	\\
2.32e-09	0	\\
2.42e-09	0	\\
2.52e-09	0	\\
2.63e-09	0	\\
2.73e-09	0	\\
2.83e-09	0	\\
2.93e-09	0	\\
3.04e-09	0	\\
3.14e-09	0	\\
3.24e-09	0	\\
3.34e-09	0	\\
3.45e-09	0	\\
3.55e-09	0	\\
3.65e-09	0	\\
3.75e-09	0	\\
3.86e-09	0	\\
3.96e-09	0	\\
4.06e-09	0	\\
4.16e-09	0	\\
4.27e-09	0	\\
4.37e-09	0	\\
4.47e-09	0	\\
4.57e-09	0	\\
4.68e-09	0	\\
4.78e-09	0	\\
4.89e-09	0	\\
4.99e-09	0	\\
5e-09	0	\\
};
\addplot [color=mycolor1,solid,forget plot]
  table[row sep=crcr]{
0	0	\\
1.1e-10	0	\\
2.2e-10	0	\\
3.3e-10	0	\\
4.4e-10	0	\\
5.4e-10	0	\\
6.5e-10	0	\\
7.5e-10	0	\\
8.6e-10	0	\\
9.6e-10	0	\\
1.07e-09	0	\\
1.18e-09	0	\\
1.28e-09	0	\\
1.38e-09	0	\\
1.49e-09	0	\\
1.59e-09	0	\\
1.69e-09	0	\\
1.8e-09	0	\\
1.9e-09	0	\\
2.01e-09	0	\\
2.11e-09	0	\\
2.21e-09	0	\\
2.32e-09	0	\\
2.42e-09	0	\\
2.52e-09	0	\\
2.63e-09	0	\\
2.73e-09	0	\\
2.83e-09	0	\\
2.93e-09	0	\\
3.04e-09	0	\\
3.14e-09	0	\\
3.24e-09	0	\\
3.34e-09	0	\\
3.45e-09	0	\\
3.55e-09	0	\\
3.65e-09	0	\\
3.75e-09	0	\\
3.86e-09	0	\\
3.96e-09	0	\\
4.06e-09	0	\\
4.16e-09	0	\\
4.27e-09	0	\\
4.37e-09	0	\\
4.47e-09	0	\\
4.57e-09	0	\\
4.68e-09	0	\\
4.78e-09	0	\\
4.89e-09	0	\\
4.99e-09	0	\\
5e-09	0	\\
};
\addplot [color=mycolor2,solid,forget plot]
  table[row sep=crcr]{
0	0	\\
1.1e-10	0	\\
2.2e-10	0	\\
3.3e-10	0	\\
4.4e-10	0	\\
5.4e-10	0	\\
6.5e-10	0	\\
7.5e-10	0	\\
8.6e-10	0	\\
9.6e-10	0	\\
1.07e-09	0	\\
1.18e-09	0	\\
1.28e-09	0	\\
1.38e-09	0	\\
1.49e-09	0	\\
1.59e-09	0	\\
1.69e-09	0	\\
1.8e-09	0	\\
1.9e-09	0	\\
2.01e-09	0	\\
2.11e-09	0	\\
2.21e-09	0	\\
2.32e-09	0	\\
2.42e-09	0	\\
2.52e-09	0	\\
2.63e-09	0	\\
2.73e-09	0	\\
2.83e-09	0	\\
2.93e-09	0	\\
3.04e-09	0	\\
3.14e-09	0	\\
3.24e-09	0	\\
3.34e-09	0	\\
3.45e-09	0	\\
3.55e-09	0	\\
3.65e-09	0	\\
3.75e-09	0	\\
3.86e-09	0	\\
3.96e-09	0	\\
4.06e-09	0	\\
4.16e-09	0	\\
4.27e-09	0	\\
4.37e-09	0	\\
4.47e-09	0	\\
4.57e-09	0	\\
4.68e-09	0	\\
4.78e-09	0	\\
4.89e-09	0	\\
4.99e-09	0	\\
5e-09	0	\\
};
\addplot [color=mycolor3,solid,forget plot]
  table[row sep=crcr]{
0	0	\\
1.1e-10	0	\\
2.2e-10	0	\\
3.3e-10	0	\\
4.4e-10	0	\\
5.4e-10	0	\\
6.5e-10	0	\\
7.5e-10	0	\\
8.6e-10	0	\\
9.6e-10	0	\\
1.07e-09	0	\\
1.18e-09	0	\\
1.28e-09	0	\\
1.38e-09	0	\\
1.49e-09	0	\\
1.59e-09	0	\\
1.69e-09	0	\\
1.8e-09	0	\\
1.9e-09	0	\\
2.01e-09	0	\\
2.11e-09	0	\\
2.21e-09	0	\\
2.32e-09	0	\\
2.42e-09	0	\\
2.52e-09	0	\\
2.63e-09	0	\\
2.73e-09	0	\\
2.83e-09	0	\\
2.93e-09	0	\\
3.04e-09	0	\\
3.14e-09	0	\\
3.24e-09	0	\\
3.34e-09	0	\\
3.45e-09	0	\\
3.55e-09	0	\\
3.65e-09	0	\\
3.75e-09	0	\\
3.86e-09	0	\\
3.96e-09	0	\\
4.06e-09	0	\\
4.16e-09	0	\\
4.27e-09	0	\\
4.37e-09	0	\\
4.47e-09	0	\\
4.57e-09	0	\\
4.68e-09	0	\\
4.78e-09	0	\\
4.89e-09	0	\\
4.99e-09	0	\\
5e-09	0	\\
};
\addplot [color=darkgray,solid,forget plot]
  table[row sep=crcr]{
0	0	\\
1.1e-10	0	\\
2.2e-10	0	\\
3.3e-10	0	\\
4.4e-10	0	\\
5.4e-10	0	\\
6.5e-10	0	\\
7.5e-10	0	\\
8.6e-10	0	\\
9.6e-10	0	\\
1.07e-09	0	\\
1.18e-09	0	\\
1.28e-09	0	\\
1.38e-09	0	\\
1.49e-09	0	\\
1.59e-09	0	\\
1.69e-09	0	\\
1.8e-09	0	\\
1.9e-09	0	\\
2.01e-09	0	\\
2.11e-09	0	\\
2.21e-09	0	\\
2.32e-09	0	\\
2.42e-09	0	\\
2.52e-09	0	\\
2.63e-09	0	\\
2.73e-09	0	\\
2.83e-09	0	\\
2.93e-09	0	\\
3.04e-09	0	\\
3.14e-09	0	\\
3.24e-09	0	\\
3.34e-09	0	\\
3.45e-09	0	\\
3.55e-09	0	\\
3.65e-09	0	\\
3.75e-09	0	\\
3.86e-09	0	\\
3.96e-09	0	\\
4.06e-09	0	\\
4.16e-09	0	\\
4.27e-09	0	\\
4.37e-09	0	\\
4.47e-09	0	\\
4.57e-09	0	\\
4.68e-09	0	\\
4.78e-09	0	\\
4.89e-09	0	\\
4.99e-09	0	\\
5e-09	0	\\
};
\addplot [color=blue,solid,forget plot]
  table[row sep=crcr]{
0	0	\\
1.1e-10	0	\\
2.2e-10	0	\\
3.3e-10	0	\\
4.4e-10	0	\\
5.4e-10	0	\\
6.5e-10	0	\\
7.5e-10	0	\\
8.6e-10	0	\\
9.6e-10	0	\\
1.07e-09	0	\\
1.18e-09	0	\\
1.28e-09	0	\\
1.38e-09	0	\\
1.49e-09	0	\\
1.59e-09	0	\\
1.69e-09	0	\\
1.8e-09	0	\\
1.9e-09	0	\\
2.01e-09	0	\\
2.11e-09	0	\\
2.21e-09	0	\\
2.32e-09	0	\\
2.42e-09	0	\\
2.52e-09	0	\\
2.63e-09	0	\\
2.73e-09	0	\\
2.83e-09	0	\\
2.93e-09	0	\\
3.04e-09	0	\\
3.14e-09	0	\\
3.24e-09	0	\\
3.34e-09	0	\\
3.45e-09	0	\\
3.55e-09	0	\\
3.65e-09	0	\\
3.75e-09	0	\\
3.86e-09	0	\\
3.96e-09	0	\\
4.06e-09	0	\\
4.16e-09	0	\\
4.27e-09	0	\\
4.37e-09	0	\\
4.47e-09	0	\\
4.57e-09	0	\\
4.68e-09	0	\\
4.78e-09	0	\\
4.89e-09	0	\\
4.99e-09	0	\\
5e-09	0	\\
};
\addplot [color=black!50!green,solid,forget plot]
  table[row sep=crcr]{
0	0	\\
1.1e-10	0	\\
2.2e-10	0	\\
3.3e-10	0	\\
4.4e-10	0	\\
5.4e-10	0	\\
6.5e-10	0	\\
7.5e-10	0	\\
8.6e-10	0	\\
9.6e-10	0	\\
1.07e-09	0	\\
1.18e-09	0	\\
1.28e-09	0	\\
1.38e-09	0	\\
1.49e-09	0	\\
1.59e-09	0	\\
1.69e-09	0	\\
1.8e-09	0	\\
1.9e-09	0	\\
2.01e-09	0	\\
2.11e-09	0	\\
2.21e-09	0	\\
2.32e-09	0	\\
2.42e-09	0	\\
2.52e-09	0	\\
2.63e-09	0	\\
2.73e-09	0	\\
2.83e-09	0	\\
2.93e-09	0	\\
3.04e-09	0	\\
3.14e-09	0	\\
3.24e-09	0	\\
3.34e-09	0	\\
3.45e-09	0	\\
3.55e-09	0	\\
3.65e-09	0	\\
3.75e-09	0	\\
3.86e-09	0	\\
3.96e-09	0	\\
4.06e-09	0	\\
4.16e-09	0	\\
4.27e-09	0	\\
4.37e-09	0	\\
4.47e-09	0	\\
4.57e-09	0	\\
4.68e-09	0	\\
4.78e-09	0	\\
4.89e-09	0	\\
4.99e-09	0	\\
5e-09	0	\\
};
\addplot [color=red,solid,forget plot]
  table[row sep=crcr]{
0	0	\\
1.1e-10	0	\\
2.2e-10	0	\\
3.3e-10	0	\\
4.4e-10	0	\\
5.4e-10	0	\\
6.5e-10	0	\\
7.5e-10	0	\\
8.6e-10	0	\\
9.6e-10	0	\\
1.07e-09	0	\\
1.18e-09	0	\\
1.28e-09	0	\\
1.38e-09	0	\\
1.49e-09	0	\\
1.59e-09	0	\\
1.69e-09	0	\\
1.8e-09	0	\\
1.9e-09	0	\\
2.01e-09	0	\\
2.11e-09	0	\\
2.21e-09	0	\\
2.32e-09	0	\\
2.42e-09	0	\\
2.52e-09	0	\\
2.63e-09	0	\\
2.73e-09	0	\\
2.83e-09	0	\\
2.93e-09	0	\\
3.04e-09	0	\\
3.14e-09	0	\\
3.24e-09	0	\\
3.34e-09	0	\\
3.45e-09	0	\\
3.55e-09	0	\\
3.65e-09	0	\\
3.75e-09	0	\\
3.86e-09	0	\\
3.96e-09	0	\\
4.06e-09	0	\\
4.16e-09	0	\\
4.27e-09	0	\\
4.37e-09	0	\\
4.47e-09	0	\\
4.57e-09	0	\\
4.68e-09	0	\\
4.78e-09	0	\\
4.89e-09	0	\\
4.99e-09	0	\\
5e-09	0	\\
};
\addplot [color=mycolor1,solid,forget plot]
  table[row sep=crcr]{
0	0	\\
1.1e-10	0	\\
2.2e-10	0	\\
3.3e-10	0	\\
4.4e-10	0	\\
5.4e-10	0	\\
6.5e-10	0	\\
7.5e-10	0	\\
8.6e-10	0	\\
9.6e-10	0	\\
1.07e-09	0	\\
1.18e-09	0	\\
1.28e-09	0	\\
1.38e-09	0	\\
1.49e-09	0	\\
1.59e-09	0	\\
1.69e-09	0	\\
1.8e-09	0	\\
1.9e-09	0	\\
2.01e-09	0	\\
2.11e-09	0	\\
2.21e-09	0	\\
2.32e-09	0	\\
2.42e-09	0	\\
2.52e-09	0	\\
2.63e-09	0	\\
2.73e-09	0	\\
2.83e-09	0	\\
2.93e-09	0	\\
3.04e-09	0	\\
3.14e-09	0	\\
3.24e-09	0	\\
3.34e-09	0	\\
3.45e-09	0	\\
3.55e-09	0	\\
3.65e-09	0	\\
3.75e-09	0	\\
3.86e-09	0	\\
3.96e-09	0	\\
4.06e-09	0	\\
4.16e-09	0	\\
4.27e-09	0	\\
4.37e-09	0	\\
4.47e-09	0	\\
4.57e-09	0	\\
4.68e-09	0	\\
4.78e-09	0	\\
4.89e-09	0	\\
4.99e-09	0	\\
5e-09	0	\\
};
\addplot [color=mycolor2,solid,forget plot]
  table[row sep=crcr]{
0	0	\\
1.1e-10	0	\\
2.2e-10	0	\\
3.3e-10	0	\\
4.4e-10	0	\\
5.4e-10	0	\\
6.5e-10	0	\\
7.5e-10	0	\\
8.6e-10	0	\\
9.6e-10	0	\\
1.07e-09	0	\\
1.18e-09	0	\\
1.28e-09	0	\\
1.38e-09	0	\\
1.49e-09	0	\\
1.59e-09	0	\\
1.69e-09	0	\\
1.8e-09	0	\\
1.9e-09	0	\\
2.01e-09	0	\\
2.11e-09	0	\\
2.21e-09	0	\\
2.32e-09	0	\\
2.42e-09	0	\\
2.52e-09	0	\\
2.63e-09	0	\\
2.73e-09	0	\\
2.83e-09	0	\\
2.93e-09	0	\\
3.04e-09	0	\\
3.14e-09	0	\\
3.24e-09	0	\\
3.34e-09	0	\\
3.45e-09	0	\\
3.55e-09	0	\\
3.65e-09	0	\\
3.75e-09	0	\\
3.86e-09	0	\\
3.96e-09	0	\\
4.06e-09	0	\\
4.16e-09	0	\\
4.27e-09	0	\\
4.37e-09	0	\\
4.47e-09	0	\\
4.57e-09	0	\\
4.68e-09	0	\\
4.78e-09	0	\\
4.89e-09	0	\\
4.99e-09	0	\\
5e-09	0	\\
};
\addplot [color=mycolor3,solid,forget plot]
  table[row sep=crcr]{
0	0	\\
1.1e-10	0	\\
2.2e-10	0	\\
3.3e-10	0	\\
4.4e-10	0	\\
5.4e-10	0	\\
6.5e-10	0	\\
7.5e-10	0	\\
8.6e-10	0	\\
9.6e-10	0	\\
1.07e-09	0	\\
1.18e-09	0	\\
1.28e-09	0	\\
1.38e-09	0	\\
1.49e-09	0	\\
1.59e-09	0	\\
1.69e-09	0	\\
1.8e-09	0	\\
1.9e-09	0	\\
2.01e-09	0	\\
2.11e-09	0	\\
2.21e-09	0	\\
2.32e-09	0	\\
2.42e-09	0	\\
2.52e-09	0	\\
2.63e-09	0	\\
2.73e-09	0	\\
2.83e-09	0	\\
2.93e-09	0	\\
3.04e-09	0	\\
3.14e-09	0	\\
3.24e-09	0	\\
3.34e-09	0	\\
3.45e-09	0	\\
3.55e-09	0	\\
3.65e-09	0	\\
3.75e-09	0	\\
3.86e-09	0	\\
3.96e-09	0	\\
4.06e-09	0	\\
4.16e-09	0	\\
4.27e-09	0	\\
4.37e-09	0	\\
4.47e-09	0	\\
4.57e-09	0	\\
4.68e-09	0	\\
4.78e-09	0	\\
4.89e-09	0	\\
4.99e-09	0	\\
5e-09	0	\\
};
\addplot [color=darkgray,solid,forget plot]
  table[row sep=crcr]{
0	0	\\
1.1e-10	0	\\
2.2e-10	0	\\
3.3e-10	0	\\
4.4e-10	0	\\
5.4e-10	0	\\
6.5e-10	0	\\
7.5e-10	0	\\
8.6e-10	0	\\
9.6e-10	0	\\
1.07e-09	0	\\
1.18e-09	0	\\
1.28e-09	0	\\
1.38e-09	0	\\
1.49e-09	0	\\
1.59e-09	0	\\
1.69e-09	0	\\
1.8e-09	0	\\
1.9e-09	0	\\
2.01e-09	0	\\
2.11e-09	0	\\
2.21e-09	0	\\
2.32e-09	0	\\
2.42e-09	0	\\
2.52e-09	0	\\
2.63e-09	0	\\
2.73e-09	0	\\
2.83e-09	0	\\
2.93e-09	0	\\
3.04e-09	0	\\
3.14e-09	0	\\
3.24e-09	0	\\
3.34e-09	0	\\
3.45e-09	0	\\
3.55e-09	0	\\
3.65e-09	0	\\
3.75e-09	0	\\
3.86e-09	0	\\
3.96e-09	0	\\
4.06e-09	0	\\
4.16e-09	0	\\
4.27e-09	0	\\
4.37e-09	0	\\
4.47e-09	0	\\
4.57e-09	0	\\
4.68e-09	0	\\
4.78e-09	0	\\
4.89e-09	0	\\
4.99e-09	0	\\
5e-09	0	\\
};
\addplot [color=blue,solid,forget plot]
  table[row sep=crcr]{
0	0	\\
1.1e-10	0	\\
2.2e-10	0	\\
3.3e-10	0	\\
4.4e-10	0	\\
5.4e-10	0	\\
6.5e-10	0	\\
7.5e-10	0	\\
8.6e-10	0	\\
9.6e-10	0	\\
1.07e-09	0	\\
1.18e-09	0	\\
1.28e-09	0	\\
1.38e-09	0	\\
1.49e-09	0	\\
1.59e-09	0	\\
1.69e-09	0	\\
1.8e-09	0	\\
1.9e-09	0	\\
2.01e-09	0	\\
2.11e-09	0	\\
2.21e-09	0	\\
2.32e-09	0	\\
2.42e-09	0	\\
2.52e-09	0	\\
2.63e-09	0	\\
2.73e-09	0	\\
2.83e-09	0	\\
2.93e-09	0	\\
3.04e-09	0	\\
3.14e-09	0	\\
3.24e-09	0	\\
3.34e-09	0	\\
3.45e-09	0	\\
3.55e-09	0	\\
3.65e-09	0	\\
3.75e-09	0	\\
3.86e-09	0	\\
3.96e-09	0	\\
4.06e-09	0	\\
4.16e-09	0	\\
4.27e-09	0	\\
4.37e-09	0	\\
4.47e-09	0	\\
4.57e-09	0	\\
4.68e-09	0	\\
4.78e-09	0	\\
4.89e-09	0	\\
4.99e-09	0	\\
5e-09	0	\\
};
\addplot [color=black!50!green,solid,forget plot]
  table[row sep=crcr]{
0	0	\\
1.1e-10	0	\\
2.2e-10	0	\\
3.3e-10	0	\\
4.4e-10	0	\\
5.4e-10	0	\\
6.5e-10	0	\\
7.5e-10	0	\\
8.6e-10	0	\\
9.6e-10	0	\\
1.07e-09	0	\\
1.18e-09	0	\\
1.28e-09	0	\\
1.38e-09	0	\\
1.49e-09	0	\\
1.59e-09	0	\\
1.69e-09	0	\\
1.8e-09	0	\\
1.9e-09	0	\\
2.01e-09	0	\\
2.11e-09	0	\\
2.21e-09	0	\\
2.32e-09	0	\\
2.42e-09	0	\\
2.52e-09	0	\\
2.63e-09	0	\\
2.73e-09	0	\\
2.83e-09	0	\\
2.93e-09	0	\\
3.04e-09	0	\\
3.14e-09	0	\\
3.24e-09	0	\\
3.34e-09	0	\\
3.45e-09	0	\\
3.55e-09	0	\\
3.65e-09	0	\\
3.75e-09	0	\\
3.86e-09	0	\\
3.96e-09	0	\\
4.06e-09	0	\\
4.16e-09	0	\\
4.27e-09	0	\\
4.37e-09	0	\\
4.47e-09	0	\\
4.57e-09	0	\\
4.68e-09	0	\\
4.78e-09	0	\\
4.89e-09	0	\\
4.99e-09	0	\\
5e-09	0	\\
};
\addplot [color=red,solid,forget plot]
  table[row sep=crcr]{
0	0	\\
1.1e-10	0	\\
2.2e-10	0	\\
3.3e-10	0	\\
4.4e-10	0	\\
5.4e-10	0	\\
6.5e-10	0	\\
7.5e-10	0	\\
8.6e-10	0	\\
9.6e-10	0	\\
1.07e-09	0	\\
1.18e-09	0	\\
1.28e-09	0	\\
1.38e-09	0	\\
1.49e-09	0	\\
1.59e-09	0	\\
1.69e-09	0	\\
1.8e-09	0	\\
1.9e-09	0	\\
2.01e-09	0	\\
2.11e-09	0	\\
2.21e-09	0	\\
2.32e-09	0	\\
2.42e-09	0	\\
2.52e-09	0	\\
2.63e-09	0	\\
2.73e-09	0	\\
2.83e-09	0	\\
2.93e-09	0	\\
3.04e-09	0	\\
3.14e-09	0	\\
3.24e-09	0	\\
3.34e-09	0	\\
3.45e-09	0	\\
3.55e-09	0	\\
3.65e-09	0	\\
3.75e-09	0	\\
3.86e-09	0	\\
3.96e-09	0	\\
4.06e-09	0	\\
4.16e-09	0	\\
4.27e-09	0	\\
4.37e-09	0	\\
4.47e-09	0	\\
4.57e-09	0	\\
4.68e-09	0	\\
4.78e-09	0	\\
4.89e-09	0	\\
4.99e-09	0	\\
5e-09	0	\\
};
\addplot [color=mycolor1,solid,forget plot]
  table[row sep=crcr]{
0	0	\\
1.1e-10	0	\\
2.2e-10	0	\\
3.3e-10	0	\\
4.4e-10	0	\\
5.4e-10	0	\\
6.5e-10	0	\\
7.5e-10	0	\\
8.6e-10	0	\\
9.6e-10	0	\\
1.07e-09	0	\\
1.18e-09	0	\\
1.28e-09	0	\\
1.38e-09	0	\\
1.49e-09	0	\\
1.59e-09	0	\\
1.69e-09	0	\\
1.8e-09	0	\\
1.9e-09	0	\\
2.01e-09	0	\\
2.11e-09	0	\\
2.21e-09	0	\\
2.32e-09	0	\\
2.42e-09	0	\\
2.52e-09	0	\\
2.63e-09	0	\\
2.73e-09	0	\\
2.83e-09	0	\\
2.93e-09	0	\\
3.04e-09	0	\\
3.14e-09	0	\\
3.24e-09	0	\\
3.34e-09	0	\\
3.45e-09	0	\\
3.55e-09	0	\\
3.65e-09	0	\\
3.75e-09	0	\\
3.86e-09	0	\\
3.96e-09	0	\\
4.06e-09	0	\\
4.16e-09	0	\\
4.27e-09	0	\\
4.37e-09	0	\\
4.47e-09	0	\\
4.57e-09	0	\\
4.68e-09	0	\\
4.78e-09	0	\\
4.89e-09	0	\\
4.99e-09	0	\\
5e-09	0	\\
};
\addplot [color=mycolor2,solid,forget plot]
  table[row sep=crcr]{
0	0	\\
1.1e-10	0	\\
2.2e-10	0	\\
3.3e-10	0	\\
4.4e-10	0	\\
5.4e-10	0	\\
6.5e-10	0	\\
7.5e-10	0	\\
8.6e-10	0	\\
9.6e-10	0	\\
1.07e-09	0	\\
1.18e-09	0	\\
1.28e-09	0	\\
1.38e-09	0	\\
1.49e-09	0	\\
1.59e-09	0	\\
1.69e-09	0	\\
1.8e-09	0	\\
1.9e-09	0	\\
2.01e-09	0	\\
2.11e-09	0	\\
2.21e-09	0	\\
2.32e-09	0	\\
2.42e-09	0	\\
2.52e-09	0	\\
2.63e-09	0	\\
2.73e-09	0	\\
2.83e-09	0	\\
2.93e-09	0	\\
3.04e-09	0	\\
3.14e-09	0	\\
3.24e-09	0	\\
3.34e-09	0	\\
3.45e-09	0	\\
3.55e-09	0	\\
3.65e-09	0	\\
3.75e-09	0	\\
3.86e-09	0	\\
3.96e-09	0	\\
4.06e-09	0	\\
4.16e-09	0	\\
4.27e-09	0	\\
4.37e-09	0	\\
4.47e-09	0	\\
4.57e-09	0	\\
4.68e-09	0	\\
4.78e-09	0	\\
4.89e-09	0	\\
4.99e-09	0	\\
5e-09	0	\\
};
\addplot [color=mycolor3,solid,forget plot]
  table[row sep=crcr]{
0	0	\\
1.1e-10	0	\\
2.2e-10	0	\\
3.3e-10	0	\\
4.4e-10	0	\\
5.4e-10	0	\\
6.5e-10	0	\\
7.5e-10	0	\\
8.6e-10	0	\\
9.6e-10	0	\\
1.07e-09	0	\\
1.18e-09	0	\\
1.28e-09	0	\\
1.38e-09	0	\\
1.49e-09	0	\\
1.59e-09	0	\\
1.69e-09	0	\\
1.8e-09	0	\\
1.9e-09	0	\\
2.01e-09	0	\\
2.11e-09	0	\\
2.21e-09	0	\\
2.32e-09	0	\\
2.42e-09	0	\\
2.52e-09	0	\\
2.63e-09	0	\\
2.73e-09	0	\\
2.83e-09	0	\\
2.93e-09	0	\\
3.04e-09	0	\\
3.14e-09	0	\\
3.24e-09	0	\\
3.34e-09	0	\\
3.45e-09	0	\\
3.55e-09	0	\\
3.65e-09	0	\\
3.75e-09	0	\\
3.86e-09	0	\\
3.96e-09	0	\\
4.06e-09	0	\\
4.16e-09	0	\\
4.27e-09	0	\\
4.37e-09	0	\\
4.47e-09	0	\\
4.57e-09	0	\\
4.68e-09	0	\\
4.78e-09	0	\\
4.89e-09	0	\\
4.99e-09	0	\\
5e-09	0	\\
};
\addplot [color=darkgray,solid,forget plot]
  table[row sep=crcr]{
0	0	\\
1.1e-10	0	\\
2.2e-10	0	\\
3.3e-10	0	\\
4.4e-10	0	\\
5.4e-10	0	\\
6.5e-10	0	\\
7.5e-10	0	\\
8.6e-10	0	\\
9.6e-10	0	\\
1.07e-09	0	\\
1.18e-09	0	\\
1.28e-09	0	\\
1.38e-09	0	\\
1.49e-09	0	\\
1.59e-09	0	\\
1.69e-09	0	\\
1.8e-09	0	\\
1.9e-09	0	\\
2.01e-09	0	\\
2.11e-09	0	\\
2.21e-09	0	\\
2.32e-09	0	\\
2.42e-09	0	\\
2.52e-09	0	\\
2.63e-09	0	\\
2.73e-09	0	\\
2.83e-09	0	\\
2.93e-09	0	\\
3.04e-09	0	\\
3.14e-09	0	\\
3.24e-09	0	\\
3.34e-09	0	\\
3.45e-09	0	\\
3.55e-09	0	\\
3.65e-09	0	\\
3.75e-09	0	\\
3.86e-09	0	\\
3.96e-09	0	\\
4.06e-09	0	\\
4.16e-09	0	\\
4.27e-09	0	\\
4.37e-09	0	\\
4.47e-09	0	\\
4.57e-09	0	\\
4.68e-09	0	\\
4.78e-09	0	\\
4.89e-09	0	\\
4.99e-09	0	\\
5e-09	0	\\
};
\addplot [color=blue,solid,forget plot]
  table[row sep=crcr]{
0	0	\\
1.1e-10	0	\\
2.2e-10	0	\\
3.3e-10	0	\\
4.4e-10	0	\\
5.4e-10	0	\\
6.5e-10	0	\\
7.5e-10	0	\\
8.6e-10	0	\\
9.6e-10	0	\\
1.07e-09	0	\\
1.18e-09	0	\\
1.28e-09	0	\\
1.38e-09	0	\\
1.49e-09	0	\\
1.59e-09	0	\\
1.69e-09	0	\\
1.8e-09	0	\\
1.9e-09	0	\\
2.01e-09	0	\\
2.11e-09	0	\\
2.21e-09	0	\\
2.32e-09	0	\\
2.42e-09	0	\\
2.52e-09	0	\\
2.63e-09	0	\\
2.73e-09	0	\\
2.83e-09	0	\\
2.93e-09	0	\\
3.04e-09	0	\\
3.14e-09	0	\\
3.24e-09	0	\\
3.34e-09	0	\\
3.45e-09	0	\\
3.55e-09	0	\\
3.65e-09	0	\\
3.75e-09	0	\\
3.86e-09	0	\\
3.96e-09	0	\\
4.06e-09	0	\\
4.16e-09	0	\\
4.27e-09	0	\\
4.37e-09	0	\\
4.47e-09	0	\\
4.57e-09	0	\\
4.68e-09	0	\\
4.78e-09	0	\\
4.89e-09	0	\\
4.99e-09	0	\\
5e-09	0	\\
};
\addplot [color=black!50!green,solid,forget plot]
  table[row sep=crcr]{
0	0	\\
1.1e-10	0	\\
2.2e-10	0	\\
3.3e-10	0	\\
4.4e-10	0	\\
5.4e-10	0	\\
6.5e-10	0	\\
7.5e-10	0	\\
8.6e-10	0	\\
9.6e-10	0	\\
1.07e-09	0	\\
1.18e-09	0	\\
1.28e-09	0	\\
1.38e-09	0	\\
1.49e-09	0	\\
1.59e-09	0	\\
1.69e-09	0	\\
1.8e-09	0	\\
1.9e-09	0	\\
2.01e-09	0	\\
2.11e-09	0	\\
2.21e-09	0	\\
2.32e-09	0	\\
2.42e-09	0	\\
2.52e-09	0	\\
2.63e-09	0	\\
2.73e-09	0	\\
2.83e-09	0	\\
2.93e-09	0	\\
3.04e-09	0	\\
3.14e-09	0	\\
3.24e-09	0	\\
3.34e-09	0	\\
3.45e-09	0	\\
3.55e-09	0	\\
3.65e-09	0	\\
3.75e-09	0	\\
3.86e-09	0	\\
3.96e-09	0	\\
4.06e-09	0	\\
4.16e-09	0	\\
4.27e-09	0	\\
4.37e-09	0	\\
4.47e-09	0	\\
4.57e-09	0	\\
4.68e-09	0	\\
4.78e-09	0	\\
4.89e-09	0	\\
4.99e-09	0	\\
5e-09	0	\\
};
\addplot [color=red,solid,forget plot]
  table[row sep=crcr]{
0	0	\\
1.1e-10	0	\\
2.2e-10	0	\\
3.3e-10	0	\\
4.4e-10	0	\\
5.4e-10	0	\\
6.5e-10	0	\\
7.5e-10	0	\\
8.6e-10	0	\\
9.6e-10	0	\\
1.07e-09	0	\\
1.18e-09	0	\\
1.28e-09	0	\\
1.38e-09	0	\\
1.49e-09	0	\\
1.59e-09	0	\\
1.69e-09	0	\\
1.8e-09	0	\\
1.9e-09	0	\\
2.01e-09	0	\\
2.11e-09	0	\\
2.21e-09	0	\\
2.32e-09	0	\\
2.42e-09	0	\\
2.52e-09	0	\\
2.63e-09	0	\\
2.73e-09	0	\\
2.83e-09	0	\\
2.93e-09	0	\\
3.04e-09	0	\\
3.14e-09	0	\\
3.24e-09	0	\\
3.34e-09	0	\\
3.45e-09	0	\\
3.55e-09	0	\\
3.65e-09	0	\\
3.75e-09	0	\\
3.86e-09	0	\\
3.96e-09	0	\\
4.06e-09	0	\\
4.16e-09	0	\\
4.27e-09	0	\\
4.37e-09	0	\\
4.47e-09	0	\\
4.57e-09	0	\\
4.68e-09	0	\\
4.78e-09	0	\\
4.89e-09	0	\\
4.99e-09	0	\\
5e-09	0	\\
};
\addplot [color=mycolor1,solid,forget plot]
  table[row sep=crcr]{
0	0	\\
1.1e-10	0	\\
2.2e-10	0	\\
3.3e-10	0	\\
4.4e-10	0	\\
5.4e-10	0	\\
6.5e-10	0	\\
7.5e-10	0	\\
8.6e-10	0	\\
9.6e-10	0	\\
1.07e-09	0	\\
1.18e-09	0	\\
1.28e-09	0	\\
1.38e-09	0	\\
1.49e-09	0	\\
1.59e-09	0	\\
1.69e-09	0	\\
1.8e-09	0	\\
1.9e-09	0	\\
2.01e-09	0	\\
2.11e-09	0	\\
2.21e-09	0	\\
2.32e-09	0	\\
2.42e-09	0	\\
2.52e-09	0	\\
2.63e-09	0	\\
2.73e-09	0	\\
2.83e-09	0	\\
2.93e-09	0	\\
3.04e-09	0	\\
3.14e-09	0	\\
3.24e-09	0	\\
3.34e-09	0	\\
3.45e-09	0	\\
3.55e-09	0	\\
3.65e-09	0	\\
3.75e-09	0	\\
3.86e-09	0	\\
3.96e-09	0	\\
4.06e-09	0	\\
4.16e-09	0	\\
4.27e-09	0	\\
4.37e-09	0	\\
4.47e-09	0	\\
4.57e-09	0	\\
4.68e-09	0	\\
4.78e-09	0	\\
4.89e-09	0	\\
4.99e-09	0	\\
5e-09	0	\\
};
\addplot [color=mycolor2,solid,forget plot]
  table[row sep=crcr]{
0	0	\\
1.1e-10	0	\\
2.2e-10	0	\\
3.3e-10	0	\\
4.4e-10	0	\\
5.4e-10	0	\\
6.5e-10	0	\\
7.5e-10	0	\\
8.6e-10	0	\\
9.6e-10	0	\\
1.07e-09	0	\\
1.18e-09	0	\\
1.28e-09	0	\\
1.38e-09	0	\\
1.49e-09	0	\\
1.59e-09	0	\\
1.69e-09	0	\\
1.8e-09	0	\\
1.9e-09	0	\\
2.01e-09	0	\\
2.11e-09	0	\\
2.21e-09	0	\\
2.32e-09	0	\\
2.42e-09	0	\\
2.52e-09	0	\\
2.63e-09	0	\\
2.73e-09	0	\\
2.83e-09	0	\\
2.93e-09	0	\\
3.04e-09	0	\\
3.14e-09	0	\\
3.24e-09	0	\\
3.34e-09	0	\\
3.45e-09	0	\\
3.55e-09	0	\\
3.65e-09	0	\\
3.75e-09	0	\\
3.86e-09	0	\\
3.96e-09	0	\\
4.06e-09	0	\\
4.16e-09	0	\\
4.27e-09	0	\\
4.37e-09	0	\\
4.47e-09	0	\\
4.57e-09	0	\\
4.68e-09	0	\\
4.78e-09	0	\\
4.89e-09	0	\\
4.99e-09	0	\\
5e-09	0	\\
};
\addplot [color=mycolor3,solid,forget plot]
  table[row sep=crcr]{
0	0	\\
1.1e-10	0	\\
2.2e-10	0	\\
3.3e-10	0	\\
4.4e-10	0	\\
5.4e-10	0	\\
6.5e-10	0	\\
7.5e-10	0	\\
8.6e-10	0	\\
9.6e-10	0	\\
1.07e-09	0	\\
1.18e-09	0	\\
1.28e-09	0	\\
1.38e-09	0	\\
1.49e-09	0	\\
1.59e-09	0	\\
1.69e-09	0	\\
1.8e-09	0	\\
1.9e-09	0	\\
2.01e-09	0	\\
2.11e-09	0	\\
2.21e-09	0	\\
2.32e-09	0	\\
2.42e-09	0	\\
2.52e-09	0	\\
2.63e-09	0	\\
2.73e-09	0	\\
2.83e-09	0	\\
2.93e-09	0	\\
3.04e-09	0	\\
3.14e-09	0	\\
3.24e-09	0	\\
3.34e-09	0	\\
3.45e-09	0	\\
3.55e-09	0	\\
3.65e-09	0	\\
3.75e-09	0	\\
3.86e-09	0	\\
3.96e-09	0	\\
4.06e-09	0	\\
4.16e-09	0	\\
4.27e-09	0	\\
4.37e-09	0	\\
4.47e-09	0	\\
4.57e-09	0	\\
4.68e-09	0	\\
4.78e-09	0	\\
4.89e-09	0	\\
4.99e-09	0	\\
5e-09	0	\\
};
\addplot [color=darkgray,solid,forget plot]
  table[row sep=crcr]{
0	0	\\
1.1e-10	0	\\
2.2e-10	0	\\
3.3e-10	0	\\
4.4e-10	0	\\
5.4e-10	0	\\
6.5e-10	0	\\
7.5e-10	0	\\
8.6e-10	0	\\
9.6e-10	0	\\
1.07e-09	0	\\
1.18e-09	0	\\
1.28e-09	0	\\
1.38e-09	0	\\
1.49e-09	0	\\
1.59e-09	0	\\
1.69e-09	0	\\
1.8e-09	0	\\
1.9e-09	0	\\
2.01e-09	0	\\
2.11e-09	0	\\
2.21e-09	0	\\
2.32e-09	0	\\
2.42e-09	0	\\
2.52e-09	0	\\
2.63e-09	0	\\
2.73e-09	0	\\
2.83e-09	0	\\
2.93e-09	0	\\
3.04e-09	0	\\
3.14e-09	0	\\
3.24e-09	0	\\
3.34e-09	0	\\
3.45e-09	0	\\
3.55e-09	0	\\
3.65e-09	0	\\
3.75e-09	0	\\
3.86e-09	0	\\
3.96e-09	0	\\
4.06e-09	0	\\
4.16e-09	0	\\
4.27e-09	0	\\
4.37e-09	0	\\
4.47e-09	0	\\
4.57e-09	0	\\
4.68e-09	0	\\
4.78e-09	0	\\
4.89e-09	0	\\
4.99e-09	0	\\
5e-09	0	\\
};
\addplot [color=blue,solid,forget plot]
  table[row sep=crcr]{
0	0	\\
1.1e-10	0	\\
2.2e-10	0	\\
3.3e-10	0	\\
4.4e-10	0	\\
5.4e-10	0	\\
6.5e-10	0	\\
7.5e-10	0	\\
8.6e-10	0	\\
9.6e-10	0	\\
1.07e-09	0	\\
1.18e-09	0	\\
1.28e-09	0	\\
1.38e-09	0	\\
1.49e-09	0	\\
1.59e-09	0	\\
1.69e-09	0	\\
1.8e-09	0	\\
1.9e-09	0	\\
2.01e-09	0	\\
2.11e-09	0	\\
2.21e-09	0	\\
2.32e-09	0	\\
2.42e-09	0	\\
2.52e-09	0	\\
2.63e-09	0	\\
2.73e-09	0	\\
2.83e-09	0	\\
2.93e-09	0	\\
3.04e-09	0	\\
3.14e-09	0	\\
3.24e-09	0	\\
3.34e-09	0	\\
3.45e-09	0	\\
3.55e-09	0	\\
3.65e-09	0	\\
3.75e-09	0	\\
3.86e-09	0	\\
3.96e-09	0	\\
4.06e-09	0	\\
4.16e-09	0	\\
4.27e-09	0	\\
4.37e-09	0	\\
4.47e-09	0	\\
4.57e-09	0	\\
4.68e-09	0	\\
4.78e-09	0	\\
4.89e-09	0	\\
4.99e-09	0	\\
5e-09	0	\\
};
\addplot [color=black!50!green,solid,forget plot]
  table[row sep=crcr]{
0	0	\\
1.1e-10	0	\\
2.2e-10	0	\\
3.3e-10	0	\\
4.4e-10	0	\\
5.4e-10	0	\\
6.5e-10	0	\\
7.5e-10	0	\\
8.6e-10	0	\\
9.6e-10	0	\\
1.07e-09	0	\\
1.18e-09	0	\\
1.28e-09	0	\\
1.38e-09	0	\\
1.49e-09	0	\\
1.59e-09	0	\\
1.69e-09	0	\\
1.8e-09	0	\\
1.9e-09	0	\\
2.01e-09	0	\\
2.11e-09	0	\\
2.21e-09	0	\\
2.32e-09	0	\\
2.42e-09	0	\\
2.52e-09	0	\\
2.63e-09	0	\\
2.73e-09	0	\\
2.83e-09	0	\\
2.93e-09	0	\\
3.04e-09	0	\\
3.14e-09	0	\\
3.24e-09	0	\\
3.34e-09	0	\\
3.45e-09	0	\\
3.55e-09	0	\\
3.65e-09	0	\\
3.75e-09	0	\\
3.86e-09	0	\\
3.96e-09	0	\\
4.06e-09	0	\\
4.16e-09	0	\\
4.27e-09	0	\\
4.37e-09	0	\\
4.47e-09	0	\\
4.57e-09	0	\\
4.68e-09	0	\\
4.78e-09	0	\\
4.89e-09	0	\\
4.99e-09	0	\\
5e-09	0	\\
};
\addplot [color=red,solid,forget plot]
  table[row sep=crcr]{
0	0	\\
1.1e-10	0	\\
2.2e-10	0	\\
3.3e-10	0	\\
4.4e-10	0	\\
5.4e-10	0	\\
6.5e-10	0	\\
7.5e-10	0	\\
8.6e-10	0	\\
9.6e-10	0	\\
1.07e-09	0	\\
1.18e-09	0	\\
1.28e-09	0	\\
1.38e-09	0	\\
1.49e-09	0	\\
1.59e-09	0	\\
1.69e-09	0	\\
1.8e-09	0	\\
1.9e-09	0	\\
2.01e-09	0	\\
2.11e-09	0	\\
2.21e-09	0	\\
2.32e-09	0	\\
2.42e-09	0	\\
2.52e-09	0	\\
2.63e-09	0	\\
2.73e-09	0	\\
2.83e-09	0	\\
2.93e-09	0	\\
3.04e-09	0	\\
3.14e-09	0	\\
3.24e-09	0	\\
3.34e-09	0	\\
3.45e-09	0	\\
3.55e-09	0	\\
3.65e-09	0	\\
3.75e-09	0	\\
3.86e-09	0	\\
3.96e-09	0	\\
4.06e-09	0	\\
4.16e-09	0	\\
4.27e-09	0	\\
4.37e-09	0	\\
4.47e-09	0	\\
4.57e-09	0	\\
4.68e-09	0	\\
4.78e-09	0	\\
4.89e-09	0	\\
4.99e-09	0	\\
5e-09	0	\\
};
\addplot [color=mycolor1,solid,forget plot]
  table[row sep=crcr]{
0	0	\\
1.1e-10	0	\\
2.2e-10	0	\\
3.3e-10	0	\\
4.4e-10	0	\\
5.4e-10	0	\\
6.5e-10	0	\\
7.5e-10	0	\\
8.6e-10	0	\\
9.6e-10	0	\\
1.07e-09	0	\\
1.18e-09	0	\\
1.28e-09	0	\\
1.38e-09	0	\\
1.49e-09	0	\\
1.59e-09	0	\\
1.69e-09	0	\\
1.8e-09	0	\\
1.9e-09	0	\\
2.01e-09	0	\\
2.11e-09	0	\\
2.21e-09	0	\\
2.32e-09	0	\\
2.42e-09	0	\\
2.52e-09	0	\\
2.63e-09	0	\\
2.73e-09	0	\\
2.83e-09	0	\\
2.93e-09	0	\\
3.04e-09	0	\\
3.14e-09	0	\\
3.24e-09	0	\\
3.34e-09	0	\\
3.45e-09	0	\\
3.55e-09	0	\\
3.65e-09	0	\\
3.75e-09	0	\\
3.86e-09	0	\\
3.96e-09	0	\\
4.06e-09	0	\\
4.16e-09	0	\\
4.27e-09	0	\\
4.37e-09	0	\\
4.47e-09	0	\\
4.57e-09	0	\\
4.68e-09	0	\\
4.78e-09	0	\\
4.89e-09	0	\\
4.99e-09	0	\\
5e-09	0	\\
};
\addplot [color=mycolor2,solid,forget plot]
  table[row sep=crcr]{
0	0	\\
1.1e-10	0	\\
2.2e-10	0	\\
3.3e-10	0	\\
4.4e-10	0	\\
5.4e-10	0	\\
6.5e-10	0	\\
7.5e-10	0	\\
8.6e-10	0	\\
9.6e-10	0	\\
1.07e-09	0	\\
1.18e-09	0	\\
1.28e-09	0	\\
1.38e-09	0	\\
1.49e-09	0	\\
1.59e-09	0	\\
1.69e-09	0	\\
1.8e-09	0	\\
1.9e-09	0	\\
2.01e-09	0	\\
2.11e-09	0	\\
2.21e-09	0	\\
2.32e-09	0	\\
2.42e-09	0	\\
2.52e-09	0	\\
2.63e-09	0	\\
2.73e-09	0	\\
2.83e-09	0	\\
2.93e-09	0	\\
3.04e-09	0	\\
3.14e-09	0	\\
3.24e-09	0	\\
3.34e-09	0	\\
3.45e-09	0	\\
3.55e-09	0	\\
3.65e-09	0	\\
3.75e-09	0	\\
3.86e-09	0	\\
3.96e-09	0	\\
4.06e-09	0	\\
4.16e-09	0	\\
4.27e-09	0	\\
4.37e-09	0	\\
4.47e-09	0	\\
4.57e-09	0	\\
4.68e-09	0	\\
4.78e-09	0	\\
4.89e-09	0	\\
4.99e-09	0	\\
5e-09	0	\\
};
\addplot [color=mycolor3,solid,forget plot]
  table[row sep=crcr]{
0	0	\\
1.1e-10	0	\\
2.2e-10	0	\\
3.3e-10	0	\\
4.4e-10	0	\\
5.4e-10	0	\\
6.5e-10	0	\\
7.5e-10	0	\\
8.6e-10	0	\\
9.6e-10	0	\\
1.07e-09	0	\\
1.18e-09	0	\\
1.28e-09	0	\\
1.38e-09	0	\\
1.49e-09	0	\\
1.59e-09	0	\\
1.69e-09	0	\\
1.8e-09	0	\\
1.9e-09	0	\\
2.01e-09	0	\\
2.11e-09	0	\\
2.21e-09	0	\\
2.32e-09	0	\\
2.42e-09	0	\\
2.52e-09	0	\\
2.63e-09	0	\\
2.73e-09	0	\\
2.83e-09	0	\\
2.93e-09	0	\\
3.04e-09	0	\\
3.14e-09	0	\\
3.24e-09	0	\\
3.34e-09	0	\\
3.45e-09	0	\\
3.55e-09	0	\\
3.65e-09	0	\\
3.75e-09	0	\\
3.86e-09	0	\\
3.96e-09	0	\\
4.06e-09	0	\\
4.16e-09	0	\\
4.27e-09	0	\\
4.37e-09	0	\\
4.47e-09	0	\\
4.57e-09	0	\\
4.68e-09	0	\\
4.78e-09	0	\\
4.89e-09	0	\\
4.99e-09	0	\\
5e-09	0	\\
};
\addplot [color=darkgray,solid,forget plot]
  table[row sep=crcr]{
0	0	\\
1.1e-10	0	\\
2.2e-10	0	\\
3.3e-10	0	\\
4.4e-10	0	\\
5.4e-10	0	\\
6.5e-10	0	\\
7.5e-10	0	\\
8.6e-10	0	\\
9.6e-10	0	\\
1.07e-09	0	\\
1.18e-09	0	\\
1.28e-09	0	\\
1.38e-09	0	\\
1.49e-09	0	\\
1.59e-09	0	\\
1.69e-09	0	\\
1.8e-09	0	\\
1.9e-09	0	\\
2.01e-09	0	\\
2.11e-09	0	\\
2.21e-09	0	\\
2.32e-09	0	\\
2.42e-09	0	\\
2.52e-09	0	\\
2.63e-09	0	\\
2.73e-09	0	\\
2.83e-09	0	\\
2.93e-09	0	\\
3.04e-09	0	\\
3.14e-09	0	\\
3.24e-09	0	\\
3.34e-09	0	\\
3.45e-09	0	\\
3.55e-09	0	\\
3.65e-09	0	\\
3.75e-09	0	\\
3.86e-09	0	\\
3.96e-09	0	\\
4.06e-09	0	\\
4.16e-09	0	\\
4.27e-09	0	\\
4.37e-09	0	\\
4.47e-09	0	\\
4.57e-09	0	\\
4.68e-09	0	\\
4.78e-09	0	\\
4.89e-09	0	\\
4.99e-09	0	\\
5e-09	0	\\
};
\addplot [color=blue,solid,forget plot]
  table[row sep=crcr]{
0	0	\\
1.1e-10	0	\\
2.2e-10	0	\\
3.3e-10	0	\\
4.4e-10	0	\\
5.4e-10	0	\\
6.5e-10	0	\\
7.5e-10	0	\\
8.6e-10	0	\\
9.6e-10	0	\\
1.07e-09	0	\\
1.18e-09	0	\\
1.28e-09	0	\\
1.38e-09	0	\\
1.49e-09	0	\\
1.59e-09	0	\\
1.69e-09	0	\\
1.8e-09	0	\\
1.9e-09	0	\\
2.01e-09	0	\\
2.11e-09	0	\\
2.21e-09	0	\\
2.32e-09	0	\\
2.42e-09	0	\\
2.52e-09	0	\\
2.63e-09	0	\\
2.73e-09	0	\\
2.83e-09	0	\\
2.93e-09	0	\\
3.04e-09	0	\\
3.14e-09	0	\\
3.24e-09	0	\\
3.34e-09	0	\\
3.45e-09	0	\\
3.55e-09	0	\\
3.65e-09	0	\\
3.75e-09	0	\\
3.86e-09	0	\\
3.96e-09	0	\\
4.06e-09	0	\\
4.16e-09	0	\\
4.27e-09	0	\\
4.37e-09	0	\\
4.47e-09	0	\\
4.57e-09	0	\\
4.68e-09	0	\\
4.78e-09	0	\\
4.89e-09	0	\\
4.99e-09	0	\\
5e-09	0	\\
};
\addplot [color=black!50!green,solid,forget plot]
  table[row sep=crcr]{
0	0	\\
1.1e-10	0	\\
2.2e-10	0	\\
3.3e-10	0	\\
4.4e-10	0	\\
5.4e-10	0	\\
6.5e-10	0	\\
7.5e-10	0	\\
8.6e-10	0	\\
9.6e-10	0	\\
1.07e-09	0	\\
1.18e-09	0	\\
1.28e-09	0	\\
1.38e-09	0	\\
1.49e-09	0	\\
1.59e-09	0	\\
1.69e-09	0	\\
1.8e-09	0	\\
1.9e-09	0	\\
2.01e-09	0	\\
2.11e-09	0	\\
2.21e-09	0	\\
2.32e-09	0	\\
2.42e-09	0	\\
2.52e-09	0	\\
2.63e-09	0	\\
2.73e-09	0	\\
2.83e-09	0	\\
2.93e-09	0	\\
3.04e-09	0	\\
3.14e-09	0	\\
3.24e-09	0	\\
3.34e-09	0	\\
3.45e-09	0	\\
3.55e-09	0	\\
3.65e-09	0	\\
3.75e-09	0	\\
3.86e-09	0	\\
3.96e-09	0	\\
4.06e-09	0	\\
4.16e-09	0	\\
4.27e-09	0	\\
4.37e-09	0	\\
4.47e-09	0	\\
4.57e-09	0	\\
4.68e-09	0	\\
4.78e-09	0	\\
4.89e-09	0	\\
4.99e-09	0	\\
5e-09	0	\\
};
\addplot [color=red,solid,forget plot]
  table[row sep=crcr]{
0	0	\\
1.1e-10	0	\\
2.2e-10	0	\\
3.3e-10	0	\\
4.4e-10	0	\\
5.4e-10	0	\\
6.5e-10	0	\\
7.5e-10	0	\\
8.6e-10	0	\\
9.6e-10	0	\\
1.07e-09	0	\\
1.18e-09	0	\\
1.28e-09	0	\\
1.38e-09	0	\\
1.49e-09	0	\\
1.59e-09	0	\\
1.69e-09	0	\\
1.8e-09	0	\\
1.9e-09	0	\\
2.01e-09	0	\\
2.11e-09	0	\\
2.21e-09	0	\\
2.32e-09	0	\\
2.42e-09	0	\\
2.52e-09	0	\\
2.63e-09	0	\\
2.73e-09	0	\\
2.83e-09	0	\\
2.93e-09	0	\\
3.04e-09	0	\\
3.14e-09	0	\\
3.24e-09	0	\\
3.34e-09	0	\\
3.45e-09	0	\\
3.55e-09	0	\\
3.65e-09	0	\\
3.75e-09	0	\\
3.86e-09	0	\\
3.96e-09	0	\\
4.06e-09	0	\\
4.16e-09	0	\\
4.27e-09	0	\\
4.37e-09	0	\\
4.47e-09	0	\\
4.57e-09	0	\\
4.68e-09	0	\\
4.78e-09	0	\\
4.89e-09	0	\\
4.99e-09	0	\\
5e-09	0	\\
};
\addplot [color=mycolor1,solid,forget plot]
  table[row sep=crcr]{
0	0	\\
1.1e-10	0	\\
2.2e-10	0	\\
3.3e-10	0	\\
4.4e-10	0	\\
5.4e-10	0	\\
6.5e-10	0	\\
7.5e-10	0	\\
8.6e-10	0	\\
9.6e-10	0	\\
1.07e-09	0	\\
1.18e-09	0	\\
1.28e-09	0	\\
1.38e-09	0	\\
1.49e-09	0	\\
1.59e-09	0	\\
1.69e-09	0	\\
1.8e-09	0	\\
1.9e-09	0	\\
2.01e-09	0	\\
2.11e-09	0	\\
2.21e-09	0	\\
2.32e-09	0	\\
2.42e-09	0	\\
2.52e-09	0	\\
2.63e-09	0	\\
2.73e-09	0	\\
2.83e-09	0	\\
2.93e-09	0	\\
3.04e-09	0	\\
3.14e-09	0	\\
3.24e-09	0	\\
3.34e-09	0	\\
3.45e-09	0	\\
3.55e-09	0	\\
3.65e-09	0	\\
3.75e-09	0	\\
3.86e-09	0	\\
3.96e-09	0	\\
4.06e-09	0	\\
4.16e-09	0	\\
4.27e-09	0	\\
4.37e-09	0	\\
4.47e-09	0	\\
4.57e-09	0	\\
4.68e-09	0	\\
4.78e-09	0	\\
4.89e-09	0	\\
4.99e-09	0	\\
5e-09	0	\\
};
\addplot [color=mycolor2,solid,forget plot]
  table[row sep=crcr]{
0	0	\\
1.1e-10	0	\\
2.2e-10	0	\\
3.3e-10	0	\\
4.4e-10	0	\\
5.4e-10	0	\\
6.5e-10	0	\\
7.5e-10	0	\\
8.6e-10	0	\\
9.6e-10	0	\\
1.07e-09	0	\\
1.18e-09	0	\\
1.28e-09	0	\\
1.38e-09	0	\\
1.49e-09	0	\\
1.59e-09	0	\\
1.69e-09	0	\\
1.8e-09	0	\\
1.9e-09	0	\\
2.01e-09	0	\\
2.11e-09	0	\\
2.21e-09	0	\\
2.32e-09	0	\\
2.42e-09	0	\\
2.52e-09	0	\\
2.63e-09	0	\\
2.73e-09	0	\\
2.83e-09	0	\\
2.93e-09	0	\\
3.04e-09	0	\\
3.14e-09	0	\\
3.24e-09	0	\\
3.34e-09	0	\\
3.45e-09	0	\\
3.55e-09	0	\\
3.65e-09	0	\\
3.75e-09	0	\\
3.86e-09	0	\\
3.96e-09	0	\\
4.06e-09	0	\\
4.16e-09	0	\\
4.27e-09	0	\\
4.37e-09	0	\\
4.47e-09	0	\\
4.57e-09	0	\\
4.68e-09	0	\\
4.78e-09	0	\\
4.89e-09	0	\\
4.99e-09	0	\\
5e-09	0	\\
};
\addplot [color=mycolor3,solid,forget plot]
  table[row sep=crcr]{
0	0	\\
1.1e-10	0	\\
2.2e-10	0	\\
3.3e-10	0	\\
4.4e-10	0	\\
5.4e-10	0	\\
6.5e-10	0	\\
7.5e-10	0	\\
8.6e-10	0	\\
9.6e-10	0	\\
1.07e-09	0	\\
1.18e-09	0	\\
1.28e-09	0	\\
1.38e-09	0	\\
1.49e-09	0	\\
1.59e-09	0	\\
1.69e-09	0	\\
1.8e-09	0	\\
1.9e-09	0	\\
2.01e-09	0	\\
2.11e-09	0	\\
2.21e-09	0	\\
2.32e-09	0	\\
2.42e-09	0	\\
2.52e-09	0	\\
2.63e-09	0	\\
2.73e-09	0	\\
2.83e-09	0	\\
2.93e-09	0	\\
3.04e-09	0	\\
3.14e-09	0	\\
3.24e-09	0	\\
3.34e-09	0	\\
3.45e-09	0	\\
3.55e-09	0	\\
3.65e-09	0	\\
3.75e-09	0	\\
3.86e-09	0	\\
3.96e-09	0	\\
4.06e-09	0	\\
4.16e-09	0	\\
4.27e-09	0	\\
4.37e-09	0	\\
4.47e-09	0	\\
4.57e-09	0	\\
4.68e-09	0	\\
4.78e-09	0	\\
4.89e-09	0	\\
4.99e-09	0	\\
5e-09	0	\\
};
\addplot [color=darkgray,solid,forget plot]
  table[row sep=crcr]{
0	0	\\
1.1e-10	0	\\
2.2e-10	0	\\
3.3e-10	0	\\
4.4e-10	0	\\
5.4e-10	0	\\
6.5e-10	0	\\
7.5e-10	0	\\
8.6e-10	0	\\
9.6e-10	0	\\
1.07e-09	0	\\
1.18e-09	0	\\
1.28e-09	0	\\
1.38e-09	0	\\
1.49e-09	0	\\
1.59e-09	0	\\
1.69e-09	0	\\
1.8e-09	0	\\
1.9e-09	0	\\
2.01e-09	0	\\
2.11e-09	0	\\
2.21e-09	0	\\
2.32e-09	0	\\
2.42e-09	0	\\
2.52e-09	0	\\
2.63e-09	0	\\
2.73e-09	0	\\
2.83e-09	0	\\
2.93e-09	0	\\
3.04e-09	0	\\
3.14e-09	0	\\
3.24e-09	0	\\
3.34e-09	0	\\
3.45e-09	0	\\
3.55e-09	0	\\
3.65e-09	0	\\
3.75e-09	0	\\
3.86e-09	0	\\
3.96e-09	0	\\
4.06e-09	0	\\
4.16e-09	0	\\
4.27e-09	0	\\
4.37e-09	0	\\
4.47e-09	0	\\
4.57e-09	0	\\
4.68e-09	0	\\
4.78e-09	0	\\
4.89e-09	0	\\
4.99e-09	0	\\
5e-09	0	\\
};
\addplot [color=blue,solid,forget plot]
  table[row sep=crcr]{
0	0	\\
1.1e-10	0	\\
2.2e-10	0	\\
3.3e-10	0	\\
4.4e-10	0	\\
5.4e-10	0	\\
6.5e-10	0	\\
7.5e-10	0	\\
8.6e-10	0	\\
9.6e-10	0	\\
1.07e-09	0	\\
1.18e-09	0	\\
1.28e-09	0	\\
1.38e-09	0	\\
1.49e-09	0	\\
1.59e-09	0	\\
1.69e-09	0	\\
1.8e-09	0	\\
1.9e-09	0	\\
2.01e-09	0	\\
2.11e-09	0	\\
2.21e-09	0	\\
2.32e-09	0	\\
2.42e-09	0	\\
2.52e-09	0	\\
2.63e-09	0	\\
2.73e-09	0	\\
2.83e-09	0	\\
2.93e-09	0	\\
3.04e-09	0	\\
3.14e-09	0	\\
3.24e-09	0	\\
3.34e-09	0	\\
3.45e-09	0	\\
3.55e-09	0	\\
3.65e-09	0	\\
3.75e-09	0	\\
3.86e-09	0	\\
3.96e-09	0	\\
4.06e-09	0	\\
4.16e-09	0	\\
4.27e-09	0	\\
4.37e-09	0	\\
4.47e-09	0	\\
4.57e-09	0	\\
4.68e-09	0	\\
4.78e-09	0	\\
4.89e-09	0	\\
4.99e-09	0	\\
5e-09	0	\\
};
\addplot [color=black!50!green,solid,forget plot]
  table[row sep=crcr]{
0	0	\\
1.1e-10	0	\\
2.2e-10	0	\\
3.3e-10	0	\\
4.4e-10	0	\\
5.4e-10	0	\\
6.5e-10	0	\\
7.5e-10	0	\\
8.6e-10	0	\\
9.6e-10	0	\\
1.07e-09	0	\\
1.18e-09	0	\\
1.28e-09	0	\\
1.38e-09	0	\\
1.49e-09	0	\\
1.59e-09	0	\\
1.69e-09	0	\\
1.8e-09	0	\\
1.9e-09	0	\\
2.01e-09	0	\\
2.11e-09	0	\\
2.21e-09	0	\\
2.32e-09	0	\\
2.42e-09	0	\\
2.52e-09	0	\\
2.63e-09	0	\\
2.73e-09	0	\\
2.83e-09	0	\\
2.93e-09	0	\\
3.04e-09	0	\\
3.14e-09	0	\\
3.24e-09	0	\\
3.34e-09	0	\\
3.45e-09	0	\\
3.55e-09	0	\\
3.65e-09	0	\\
3.75e-09	0	\\
3.86e-09	0	\\
3.96e-09	0	\\
4.06e-09	0	\\
4.16e-09	0	\\
4.27e-09	0	\\
4.37e-09	0	\\
4.47e-09	0	\\
4.57e-09	0	\\
4.68e-09	0	\\
4.78e-09	0	\\
4.89e-09	0	\\
4.99e-09	0	\\
5e-09	0	\\
};
\addplot [color=red,solid,forget plot]
  table[row sep=crcr]{
0	0	\\
1.1e-10	0	\\
2.2e-10	0	\\
3.3e-10	0	\\
4.4e-10	0	\\
5.4e-10	0	\\
6.5e-10	0	\\
7.5e-10	0	\\
8.6e-10	0	\\
9.6e-10	0	\\
1.07e-09	0	\\
1.18e-09	0	\\
1.28e-09	0	\\
1.38e-09	0	\\
1.49e-09	0	\\
1.59e-09	0	\\
1.69e-09	0	\\
1.8e-09	0	\\
1.9e-09	0	\\
2.01e-09	0	\\
2.11e-09	0	\\
2.21e-09	0	\\
2.32e-09	0	\\
2.42e-09	0	\\
2.52e-09	0	\\
2.63e-09	0	\\
2.73e-09	0	\\
2.83e-09	0	\\
2.93e-09	0	\\
3.04e-09	0	\\
3.14e-09	0	\\
3.24e-09	0	\\
3.34e-09	0	\\
3.45e-09	0	\\
3.55e-09	0	\\
3.65e-09	0	\\
3.75e-09	0	\\
3.86e-09	0	\\
3.96e-09	0	\\
4.06e-09	0	\\
4.16e-09	0	\\
4.27e-09	0	\\
4.37e-09	0	\\
4.47e-09	0	\\
4.57e-09	0	\\
4.68e-09	0	\\
4.78e-09	0	\\
4.89e-09	0	\\
4.99e-09	0	\\
5e-09	0	\\
};
\addplot [color=mycolor1,solid,forget plot]
  table[row sep=crcr]{
0	0	\\
1.1e-10	0	\\
2.2e-10	0	\\
3.3e-10	0	\\
4.4e-10	0	\\
5.4e-10	0	\\
6.5e-10	0	\\
7.5e-10	0	\\
8.6e-10	0	\\
9.6e-10	0	\\
1.07e-09	0	\\
1.18e-09	0	\\
1.28e-09	0	\\
1.38e-09	0	\\
1.49e-09	0	\\
1.59e-09	0	\\
1.69e-09	0	\\
1.8e-09	0	\\
1.9e-09	0	\\
2.01e-09	0	\\
2.11e-09	0	\\
2.21e-09	0	\\
2.32e-09	0	\\
2.42e-09	0	\\
2.52e-09	0	\\
2.63e-09	0	\\
2.73e-09	0	\\
2.83e-09	0	\\
2.93e-09	0	\\
3.04e-09	0	\\
3.14e-09	0	\\
3.24e-09	0	\\
3.34e-09	0	\\
3.45e-09	0	\\
3.55e-09	0	\\
3.65e-09	0	\\
3.75e-09	0	\\
3.86e-09	0	\\
3.96e-09	0	\\
4.06e-09	0	\\
4.16e-09	0	\\
4.27e-09	0	\\
4.37e-09	0	\\
4.47e-09	0	\\
4.57e-09	0	\\
4.68e-09	0	\\
4.78e-09	0	\\
4.89e-09	0	\\
4.99e-09	0	\\
5e-09	0	\\
};
\addplot [color=mycolor2,solid,forget plot]
  table[row sep=crcr]{
0	0	\\
1.1e-10	0	\\
2.2e-10	0	\\
3.3e-10	0	\\
4.4e-10	0	\\
5.4e-10	0	\\
6.5e-10	0	\\
7.5e-10	0	\\
8.6e-10	0	\\
9.6e-10	0	\\
1.07e-09	0	\\
1.18e-09	0	\\
1.28e-09	0	\\
1.38e-09	0	\\
1.49e-09	0	\\
1.59e-09	0	\\
1.69e-09	0	\\
1.8e-09	0	\\
1.9e-09	0	\\
2.01e-09	0	\\
2.11e-09	0	\\
2.21e-09	0	\\
2.32e-09	0	\\
2.42e-09	0	\\
2.52e-09	0	\\
2.63e-09	0	\\
2.73e-09	0	\\
2.83e-09	0	\\
2.93e-09	0	\\
3.04e-09	0	\\
3.14e-09	0	\\
3.24e-09	0	\\
3.34e-09	0	\\
3.45e-09	0	\\
3.55e-09	0	\\
3.65e-09	0	\\
3.75e-09	0	\\
3.86e-09	0	\\
3.96e-09	0	\\
4.06e-09	0	\\
4.16e-09	0	\\
4.27e-09	0	\\
4.37e-09	0	\\
4.47e-09	0	\\
4.57e-09	0	\\
4.68e-09	0	\\
4.78e-09	0	\\
4.89e-09	0	\\
4.99e-09	0	\\
5e-09	0	\\
};
\addplot [color=mycolor3,solid,forget plot]
  table[row sep=crcr]{
0	0	\\
1.1e-10	0	\\
2.2e-10	0	\\
3.3e-10	0	\\
4.4e-10	0	\\
5.4e-10	0	\\
6.5e-10	0	\\
7.5e-10	0	\\
8.6e-10	0	\\
9.6e-10	0	\\
1.07e-09	0	\\
1.18e-09	0	\\
1.28e-09	0	\\
1.38e-09	0	\\
1.49e-09	0	\\
1.59e-09	0	\\
1.69e-09	0	\\
1.8e-09	0	\\
1.9e-09	0	\\
2.01e-09	0	\\
2.11e-09	0	\\
2.21e-09	0	\\
2.32e-09	0	\\
2.42e-09	0	\\
2.52e-09	0	\\
2.63e-09	0	\\
2.73e-09	0	\\
2.83e-09	0	\\
2.93e-09	0	\\
3.04e-09	0	\\
3.14e-09	0	\\
3.24e-09	0	\\
3.34e-09	0	\\
3.45e-09	0	\\
3.55e-09	0	\\
3.65e-09	0	\\
3.75e-09	0	\\
3.86e-09	0	\\
3.96e-09	0	\\
4.06e-09	0	\\
4.16e-09	0	\\
4.27e-09	0	\\
4.37e-09	0	\\
4.47e-09	0	\\
4.57e-09	0	\\
4.68e-09	0	\\
4.78e-09	0	\\
4.89e-09	0	\\
4.99e-09	0	\\
5e-09	0	\\
};
\addplot [color=darkgray,solid,forget plot]
  table[row sep=crcr]{
0	0	\\
1.1e-10	0	\\
2.2e-10	0	\\
3.3e-10	0	\\
4.4e-10	0	\\
5.4e-10	0	\\
6.5e-10	0	\\
7.5e-10	0	\\
8.6e-10	0	\\
9.6e-10	0	\\
1.07e-09	0	\\
1.18e-09	0	\\
1.28e-09	0	\\
1.38e-09	0	\\
1.49e-09	0	\\
1.59e-09	0	\\
1.69e-09	0	\\
1.8e-09	0	\\
1.9e-09	0	\\
2.01e-09	0	\\
2.11e-09	0	\\
2.21e-09	0	\\
2.32e-09	0	\\
2.42e-09	0	\\
2.52e-09	0	\\
2.63e-09	0	\\
2.73e-09	0	\\
2.83e-09	0	\\
2.93e-09	0	\\
3.04e-09	0	\\
3.14e-09	0	\\
3.24e-09	0	\\
3.34e-09	0	\\
3.45e-09	0	\\
3.55e-09	0	\\
3.65e-09	0	\\
3.75e-09	0	\\
3.86e-09	0	\\
3.96e-09	0	\\
4.06e-09	0	\\
4.16e-09	0	\\
4.27e-09	0	\\
4.37e-09	0	\\
4.47e-09	0	\\
4.57e-09	0	\\
4.68e-09	0	\\
4.78e-09	0	\\
4.89e-09	0	\\
4.99e-09	0	\\
5e-09	0	\\
};
\addplot [color=blue,solid,forget plot]
  table[row sep=crcr]{
0	0	\\
1.1e-10	0	\\
2.2e-10	0	\\
3.3e-10	0	\\
4.4e-10	0	\\
5.4e-10	0	\\
6.5e-10	0	\\
7.5e-10	0	\\
8.6e-10	0	\\
9.6e-10	0	\\
1.07e-09	0	\\
1.18e-09	0	\\
1.28e-09	0	\\
1.38e-09	0	\\
1.49e-09	0	\\
1.59e-09	0	\\
1.69e-09	0	\\
1.8e-09	0	\\
1.9e-09	0	\\
2.01e-09	0	\\
2.11e-09	0	\\
2.21e-09	0	\\
2.32e-09	0	\\
2.42e-09	0	\\
2.52e-09	0	\\
2.63e-09	0	\\
2.73e-09	0	\\
2.83e-09	0	\\
2.93e-09	0	\\
3.04e-09	0	\\
3.14e-09	0	\\
3.24e-09	0	\\
3.34e-09	0	\\
3.45e-09	0	\\
3.55e-09	0	\\
3.65e-09	0	\\
3.75e-09	0	\\
3.86e-09	0	\\
3.96e-09	0	\\
4.06e-09	0	\\
4.16e-09	0	\\
4.27e-09	0	\\
4.37e-09	0	\\
4.47e-09	0	\\
4.57e-09	0	\\
4.68e-09	0	\\
4.78e-09	0	\\
4.89e-09	0	\\
4.99e-09	0	\\
5e-09	0	\\
};
\addplot [color=black!50!green,solid,forget plot]
  table[row sep=crcr]{
0	0	\\
1.1e-10	0	\\
2.2e-10	0	\\
3.3e-10	0	\\
4.4e-10	0	\\
5.4e-10	0	\\
6.5e-10	0	\\
7.5e-10	0	\\
8.6e-10	0	\\
9.6e-10	0	\\
1.07e-09	0	\\
1.18e-09	0	\\
1.28e-09	0	\\
1.38e-09	0	\\
1.49e-09	0	\\
1.59e-09	0	\\
1.69e-09	0	\\
1.8e-09	0	\\
1.9e-09	0	\\
2.01e-09	0	\\
2.11e-09	0	\\
2.21e-09	0	\\
2.32e-09	0	\\
2.42e-09	0	\\
2.52e-09	0	\\
2.63e-09	0	\\
2.73e-09	0	\\
2.83e-09	0	\\
2.93e-09	0	\\
3.04e-09	0	\\
3.14e-09	0	\\
3.24e-09	0	\\
3.34e-09	0	\\
3.45e-09	0	\\
3.55e-09	0	\\
3.65e-09	0	\\
3.75e-09	0	\\
3.86e-09	0	\\
3.96e-09	0	\\
4.06e-09	0	\\
4.16e-09	0	\\
4.27e-09	0	\\
4.37e-09	0	\\
4.47e-09	0	\\
4.57e-09	0	\\
4.68e-09	0	\\
4.78e-09	0	\\
4.89e-09	0	\\
4.99e-09	0	\\
5e-09	0	\\
};
\addplot [color=red,solid,forget plot]
  table[row sep=crcr]{
0	0	\\
1.1e-10	0	\\
2.2e-10	0	\\
3.3e-10	0	\\
4.4e-10	0	\\
5.4e-10	0	\\
6.5e-10	0	\\
7.5e-10	0	\\
8.6e-10	0	\\
9.6e-10	0	\\
1.07e-09	0	\\
1.18e-09	0	\\
1.28e-09	0	\\
1.38e-09	0	\\
1.49e-09	0	\\
1.59e-09	0	\\
1.69e-09	0	\\
1.8e-09	0	\\
1.9e-09	0	\\
2.01e-09	0	\\
2.11e-09	0	\\
2.21e-09	0	\\
2.32e-09	0	\\
2.42e-09	0	\\
2.52e-09	0	\\
2.63e-09	0	\\
2.73e-09	0	\\
2.83e-09	0	\\
2.93e-09	0	\\
3.04e-09	0	\\
3.14e-09	0	\\
3.24e-09	0	\\
3.34e-09	0	\\
3.45e-09	0	\\
3.55e-09	0	\\
3.65e-09	0	\\
3.75e-09	0	\\
3.86e-09	0	\\
3.96e-09	0	\\
4.06e-09	0	\\
4.16e-09	0	\\
4.27e-09	0	\\
4.37e-09	0	\\
4.47e-09	0	\\
4.57e-09	0	\\
4.68e-09	0	\\
4.78e-09	0	\\
4.89e-09	0	\\
4.99e-09	0	\\
5e-09	0	\\
};
\addplot [color=mycolor1,solid,forget plot]
  table[row sep=crcr]{
0	0	\\
1.1e-10	0	\\
2.2e-10	0	\\
3.3e-10	0	\\
4.4e-10	0	\\
5.4e-10	0	\\
6.5e-10	0	\\
7.5e-10	0	\\
8.6e-10	0	\\
9.6e-10	0	\\
1.07e-09	0	\\
1.18e-09	0	\\
1.28e-09	0	\\
1.38e-09	0	\\
1.49e-09	0	\\
1.59e-09	0	\\
1.69e-09	0	\\
1.8e-09	0	\\
1.9e-09	0	\\
2.01e-09	0	\\
2.11e-09	0	\\
2.21e-09	0	\\
2.32e-09	0	\\
2.42e-09	0	\\
2.52e-09	0	\\
2.63e-09	0	\\
2.73e-09	0	\\
2.83e-09	0	\\
2.93e-09	0	\\
3.04e-09	0	\\
3.14e-09	0	\\
3.24e-09	0	\\
3.34e-09	0	\\
3.45e-09	0	\\
3.55e-09	0	\\
3.65e-09	0	\\
3.75e-09	0	\\
3.86e-09	0	\\
3.96e-09	0	\\
4.06e-09	0	\\
4.16e-09	0	\\
4.27e-09	0	\\
4.37e-09	0	\\
4.47e-09	0	\\
4.57e-09	0	\\
4.68e-09	0	\\
4.78e-09	0	\\
4.89e-09	0	\\
4.99e-09	0	\\
5e-09	0	\\
};
\addplot [color=mycolor2,solid,forget plot]
  table[row sep=crcr]{
0	0	\\
1.1e-10	0	\\
2.2e-10	0	\\
3.3e-10	0	\\
4.4e-10	0	\\
5.4e-10	0	\\
6.5e-10	0	\\
7.5e-10	0	\\
8.6e-10	0	\\
9.6e-10	0	\\
1.07e-09	0	\\
1.18e-09	0	\\
1.28e-09	0	\\
1.38e-09	0	\\
1.49e-09	0	\\
1.59e-09	0	\\
1.69e-09	0	\\
1.8e-09	0	\\
1.9e-09	0	\\
2.01e-09	0	\\
2.11e-09	0	\\
2.21e-09	0	\\
2.32e-09	0	\\
2.42e-09	0	\\
2.52e-09	0	\\
2.63e-09	0	\\
2.73e-09	0	\\
2.83e-09	0	\\
2.93e-09	0	\\
3.04e-09	0	\\
3.14e-09	0	\\
3.24e-09	0	\\
3.34e-09	0	\\
3.45e-09	0	\\
3.55e-09	0	\\
3.65e-09	0	\\
3.75e-09	0	\\
3.86e-09	0	\\
3.96e-09	0	\\
4.06e-09	0	\\
4.16e-09	0	\\
4.27e-09	0	\\
4.37e-09	0	\\
4.47e-09	0	\\
4.57e-09	0	\\
4.68e-09	0	\\
4.78e-09	0	\\
4.89e-09	0	\\
4.99e-09	0	\\
5e-09	0	\\
};
\addplot [color=mycolor3,solid,forget plot]
  table[row sep=crcr]{
0	0	\\
1.1e-10	0	\\
2.2e-10	0	\\
3.3e-10	0	\\
4.4e-10	0	\\
5.4e-10	0	\\
6.5e-10	0	\\
7.5e-10	0	\\
8.6e-10	0	\\
9.6e-10	0	\\
1.07e-09	0	\\
1.18e-09	0	\\
1.28e-09	0	\\
1.38e-09	0	\\
1.49e-09	0	\\
1.59e-09	0	\\
1.69e-09	0	\\
1.8e-09	0	\\
1.9e-09	0	\\
2.01e-09	0	\\
2.11e-09	0	\\
2.21e-09	0	\\
2.32e-09	0	\\
2.42e-09	0	\\
2.52e-09	0	\\
2.63e-09	0	\\
2.73e-09	0	\\
2.83e-09	0	\\
2.93e-09	0	\\
3.04e-09	0	\\
3.14e-09	0	\\
3.24e-09	0	\\
3.34e-09	0	\\
3.45e-09	0	\\
3.55e-09	0	\\
3.65e-09	0	\\
3.75e-09	0	\\
3.86e-09	0	\\
3.96e-09	0	\\
4.06e-09	0	\\
4.16e-09	0	\\
4.27e-09	0	\\
4.37e-09	0	\\
4.47e-09	0	\\
4.57e-09	0	\\
4.68e-09	0	\\
4.78e-09	0	\\
4.89e-09	0	\\
4.99e-09	0	\\
5e-09	0	\\
};
\addplot [color=darkgray,solid,forget plot]
  table[row sep=crcr]{
0	0	\\
1.1e-10	0	\\
2.2e-10	0	\\
3.3e-10	0	\\
4.4e-10	0	\\
5.4e-10	0	\\
6.5e-10	0	\\
7.5e-10	0	\\
8.6e-10	0	\\
9.6e-10	0	\\
1.07e-09	0	\\
1.18e-09	0	\\
1.28e-09	0	\\
1.38e-09	0	\\
1.49e-09	0	\\
1.59e-09	0	\\
1.69e-09	0	\\
1.8e-09	0	\\
1.9e-09	0	\\
2.01e-09	0	\\
2.11e-09	0	\\
2.21e-09	0	\\
2.32e-09	0	\\
2.42e-09	0	\\
2.52e-09	0	\\
2.63e-09	0	\\
2.73e-09	0	\\
2.83e-09	0	\\
2.93e-09	0	\\
3.04e-09	0	\\
3.14e-09	0	\\
3.24e-09	0	\\
3.34e-09	0	\\
3.45e-09	0	\\
3.55e-09	0	\\
3.65e-09	0	\\
3.75e-09	0	\\
3.86e-09	0	\\
3.96e-09	0	\\
4.06e-09	0	\\
4.16e-09	0	\\
4.27e-09	0	\\
4.37e-09	0	\\
4.47e-09	0	\\
4.57e-09	0	\\
4.68e-09	0	\\
4.78e-09	0	\\
4.89e-09	0	\\
4.99e-09	0	\\
5e-09	0	\\
};
\addplot [color=blue,solid,forget plot]
  table[row sep=crcr]{
0	0	\\
1.1e-10	0	\\
2.2e-10	0	\\
3.3e-10	0	\\
4.4e-10	0	\\
5.4e-10	0	\\
6.5e-10	0	\\
7.5e-10	0	\\
8.6e-10	0	\\
9.6e-10	0	\\
1.07e-09	0	\\
1.18e-09	0	\\
1.28e-09	0	\\
1.38e-09	0	\\
1.49e-09	0	\\
1.59e-09	0	\\
1.69e-09	0	\\
1.8e-09	0	\\
1.9e-09	0	\\
2.01e-09	0	\\
2.11e-09	0	\\
2.21e-09	0	\\
2.32e-09	0	\\
2.42e-09	0	\\
2.52e-09	0	\\
2.63e-09	0	\\
2.73e-09	0	\\
2.83e-09	0	\\
2.93e-09	0	\\
3.04e-09	0	\\
3.14e-09	0	\\
3.24e-09	0	\\
3.34e-09	0	\\
3.45e-09	0	\\
3.55e-09	0	\\
3.65e-09	0	\\
3.75e-09	0	\\
3.86e-09	0	\\
3.96e-09	0	\\
4.06e-09	0	\\
4.16e-09	0	\\
4.27e-09	0	\\
4.37e-09	0	\\
4.47e-09	0	\\
4.57e-09	0	\\
4.68e-09	0	\\
4.78e-09	0	\\
4.89e-09	0	\\
4.99e-09	0	\\
5e-09	0	\\
};
\addplot [color=black!50!green,solid,forget plot]
  table[row sep=crcr]{
0	0	\\
1.1e-10	0	\\
2.2e-10	0	\\
3.3e-10	0	\\
4.4e-10	0	\\
5.4e-10	0	\\
6.5e-10	0	\\
7.5e-10	0	\\
8.6e-10	0	\\
9.6e-10	0	\\
1.07e-09	0	\\
1.18e-09	0	\\
1.28e-09	0	\\
1.38e-09	0	\\
1.49e-09	0	\\
1.59e-09	0	\\
1.69e-09	0	\\
1.8e-09	0	\\
1.9e-09	0	\\
2.01e-09	0	\\
2.11e-09	0	\\
2.21e-09	0	\\
2.32e-09	0	\\
2.42e-09	0	\\
2.52e-09	0	\\
2.63e-09	0	\\
2.73e-09	0	\\
2.83e-09	0	\\
2.93e-09	0	\\
3.04e-09	0	\\
3.14e-09	0	\\
3.24e-09	0	\\
3.34e-09	0	\\
3.45e-09	0	\\
3.55e-09	0	\\
3.65e-09	0	\\
3.75e-09	0	\\
3.86e-09	0	\\
3.96e-09	0	\\
4.06e-09	0	\\
4.16e-09	0	\\
4.27e-09	0	\\
4.37e-09	0	\\
4.47e-09	0	\\
4.57e-09	0	\\
4.68e-09	0	\\
4.78e-09	0	\\
4.89e-09	0	\\
4.99e-09	0	\\
5e-09	0	\\
};
\addplot [color=red,solid,forget plot]
  table[row sep=crcr]{
0	0	\\
1.1e-10	0	\\
2.2e-10	0	\\
3.3e-10	0	\\
4.4e-10	0	\\
5.4e-10	0	\\
6.5e-10	0	\\
7.5e-10	0	\\
8.6e-10	0	\\
9.6e-10	0	\\
1.07e-09	0	\\
1.18e-09	0	\\
1.28e-09	0	\\
1.38e-09	0	\\
1.49e-09	0	\\
1.59e-09	0	\\
1.69e-09	0	\\
1.8e-09	0	\\
1.9e-09	0	\\
2.01e-09	0	\\
2.11e-09	0	\\
2.21e-09	0	\\
2.32e-09	0	\\
2.42e-09	0	\\
2.52e-09	0	\\
2.63e-09	0	\\
2.73e-09	0	\\
2.83e-09	0	\\
2.93e-09	0	\\
3.04e-09	0	\\
3.14e-09	0	\\
3.24e-09	0	\\
3.34e-09	0	\\
3.45e-09	0	\\
3.55e-09	0	\\
3.65e-09	0	\\
3.75e-09	0	\\
3.86e-09	0	\\
3.96e-09	0	\\
4.06e-09	0	\\
4.16e-09	0	\\
4.27e-09	0	\\
4.37e-09	0	\\
4.47e-09	0	\\
4.57e-09	0	\\
4.68e-09	0	\\
4.78e-09	0	\\
4.89e-09	0	\\
4.99e-09	0	\\
5e-09	0	\\
};
\addplot [color=mycolor1,solid,forget plot]
  table[row sep=crcr]{
0	0	\\
1.1e-10	0	\\
2.2e-10	0	\\
3.3e-10	0	\\
4.4e-10	0	\\
5.4e-10	0	\\
6.5e-10	0	\\
7.5e-10	0	\\
8.6e-10	0	\\
9.6e-10	0	\\
1.07e-09	0	\\
1.18e-09	0	\\
1.28e-09	0	\\
1.38e-09	0	\\
1.49e-09	0	\\
1.59e-09	0	\\
1.69e-09	0	\\
1.8e-09	0	\\
1.9e-09	0	\\
2.01e-09	0	\\
2.11e-09	0	\\
2.21e-09	0	\\
2.32e-09	0	\\
2.42e-09	0	\\
2.52e-09	0	\\
2.63e-09	0	\\
2.73e-09	0	\\
2.83e-09	0	\\
2.93e-09	0	\\
3.04e-09	0	\\
3.14e-09	0	\\
3.24e-09	0	\\
3.34e-09	0	\\
3.45e-09	0	\\
3.55e-09	0	\\
3.65e-09	0	\\
3.75e-09	0	\\
3.86e-09	0	\\
3.96e-09	0	\\
4.06e-09	0	\\
4.16e-09	0	\\
4.27e-09	0	\\
4.37e-09	0	\\
4.47e-09	0	\\
4.57e-09	0	\\
4.68e-09	0	\\
4.78e-09	0	\\
4.89e-09	0	\\
4.99e-09	0	\\
5e-09	0	\\
};
\addplot [color=mycolor2,solid,forget plot]
  table[row sep=crcr]{
0	0	\\
1.1e-10	0	\\
2.2e-10	0	\\
3.3e-10	0	\\
4.4e-10	0	\\
5.4e-10	0	\\
6.5e-10	0	\\
7.5e-10	0	\\
8.6e-10	0	\\
9.6e-10	0	\\
1.07e-09	0	\\
1.18e-09	0	\\
1.28e-09	0	\\
1.38e-09	0	\\
1.49e-09	0	\\
1.59e-09	0	\\
1.69e-09	0	\\
1.8e-09	0	\\
1.9e-09	0	\\
2.01e-09	0	\\
2.11e-09	0	\\
2.21e-09	0	\\
2.32e-09	0	\\
2.42e-09	0	\\
2.52e-09	0	\\
2.63e-09	0	\\
2.73e-09	0	\\
2.83e-09	0	\\
2.93e-09	0	\\
3.04e-09	0	\\
3.14e-09	0	\\
3.24e-09	0	\\
3.34e-09	0	\\
3.45e-09	0	\\
3.55e-09	0	\\
3.65e-09	0	\\
3.75e-09	0	\\
3.86e-09	0	\\
3.96e-09	0	\\
4.06e-09	0	\\
4.16e-09	0	\\
4.27e-09	0	\\
4.37e-09	0	\\
4.47e-09	0	\\
4.57e-09	0	\\
4.68e-09	0	\\
4.78e-09	0	\\
4.89e-09	0	\\
4.99e-09	0	\\
5e-09	0	\\
};
\addplot [color=mycolor3,solid,forget plot]
  table[row sep=crcr]{
0	0	\\
1.1e-10	0	\\
2.2e-10	0	\\
3.3e-10	0	\\
4.4e-10	0	\\
5.4e-10	0	\\
6.5e-10	0	\\
7.5e-10	0	\\
8.6e-10	0	\\
9.6e-10	0	\\
1.07e-09	0	\\
1.18e-09	0	\\
1.28e-09	0	\\
1.38e-09	0	\\
1.49e-09	0	\\
1.59e-09	0	\\
1.69e-09	0	\\
1.8e-09	0	\\
1.9e-09	0	\\
2.01e-09	0	\\
2.11e-09	0	\\
2.21e-09	0	\\
2.32e-09	0	\\
2.42e-09	0	\\
2.52e-09	0	\\
2.63e-09	0	\\
2.73e-09	0	\\
2.83e-09	0	\\
2.93e-09	0	\\
3.04e-09	0	\\
3.14e-09	0	\\
3.24e-09	0	\\
3.34e-09	0	\\
3.45e-09	0	\\
3.55e-09	0	\\
3.65e-09	0	\\
3.75e-09	0	\\
3.86e-09	0	\\
3.96e-09	0	\\
4.06e-09	0	\\
4.16e-09	0	\\
4.27e-09	0	\\
4.37e-09	0	\\
4.47e-09	0	\\
4.57e-09	0	\\
4.68e-09	0	\\
4.78e-09	0	\\
4.89e-09	0	\\
4.99e-09	0	\\
5e-09	0	\\
};
\addplot [color=darkgray,solid,forget plot]
  table[row sep=crcr]{
0	0	\\
1.1e-10	0	\\
2.2e-10	0	\\
3.3e-10	0	\\
4.4e-10	0	\\
5.4e-10	0	\\
6.5e-10	0	\\
7.5e-10	0	\\
8.6e-10	0	\\
9.6e-10	0	\\
1.07e-09	0	\\
1.18e-09	0	\\
1.28e-09	0	\\
1.38e-09	0	\\
1.49e-09	0	\\
1.59e-09	0	\\
1.69e-09	0	\\
1.8e-09	0	\\
1.9e-09	0	\\
2.01e-09	0	\\
2.11e-09	0	\\
2.21e-09	0	\\
2.32e-09	0	\\
2.42e-09	0	\\
2.52e-09	0	\\
2.63e-09	0	\\
2.73e-09	0	\\
2.83e-09	0	\\
2.93e-09	0	\\
3.04e-09	0	\\
3.14e-09	0	\\
3.24e-09	0	\\
3.34e-09	0	\\
3.45e-09	0	\\
3.55e-09	0	\\
3.65e-09	0	\\
3.75e-09	0	\\
3.86e-09	0	\\
3.96e-09	0	\\
4.06e-09	0	\\
4.16e-09	0	\\
4.27e-09	0	\\
4.37e-09	0	\\
4.47e-09	0	\\
4.57e-09	0	\\
4.68e-09	0	\\
4.78e-09	0	\\
4.89e-09	0	\\
4.99e-09	0	\\
5e-09	0	\\
};
\addplot [color=blue,solid,forget plot]
  table[row sep=crcr]{
0	0	\\
1.1e-10	0	\\
2.2e-10	0	\\
3.3e-10	0	\\
4.4e-10	0	\\
5.4e-10	0	\\
6.5e-10	0	\\
7.5e-10	0	\\
8.6e-10	0	\\
9.6e-10	0	\\
1.07e-09	0	\\
1.18e-09	0	\\
1.28e-09	0	\\
1.38e-09	0	\\
1.49e-09	0	\\
1.59e-09	0	\\
1.69e-09	0	\\
1.8e-09	0	\\
1.9e-09	0	\\
2.01e-09	0	\\
2.11e-09	0	\\
2.21e-09	0	\\
2.32e-09	0	\\
2.42e-09	0	\\
2.52e-09	0	\\
2.63e-09	0	\\
2.73e-09	0	\\
2.83e-09	0	\\
2.93e-09	0	\\
3.04e-09	0	\\
3.14e-09	0	\\
3.24e-09	0	\\
3.34e-09	0	\\
3.45e-09	0	\\
3.55e-09	0	\\
3.65e-09	0	\\
3.75e-09	0	\\
3.86e-09	0	\\
3.96e-09	0	\\
4.06e-09	0	\\
4.16e-09	0	\\
4.27e-09	0	\\
4.37e-09	0	\\
4.47e-09	0	\\
4.57e-09	0	\\
4.68e-09	0	\\
4.78e-09	0	\\
4.89e-09	0	\\
4.99e-09	0	\\
5e-09	0	\\
};
\addplot [color=black!50!green,solid,forget plot]
  table[row sep=crcr]{
0	0	\\
1.1e-10	0	\\
2.2e-10	0	\\
3.3e-10	0	\\
4.4e-10	0	\\
5.4e-10	0	\\
6.5e-10	0	\\
7.5e-10	0	\\
8.6e-10	0	\\
9.6e-10	0	\\
1.07e-09	0	\\
1.18e-09	0	\\
1.28e-09	0	\\
1.38e-09	0	\\
1.49e-09	0	\\
1.59e-09	0	\\
1.69e-09	0	\\
1.8e-09	0	\\
1.9e-09	0	\\
2.01e-09	0	\\
2.11e-09	0	\\
2.21e-09	0	\\
2.32e-09	0	\\
2.42e-09	0	\\
2.52e-09	0	\\
2.63e-09	0	\\
2.73e-09	0	\\
2.83e-09	0	\\
2.93e-09	0	\\
3.04e-09	0	\\
3.14e-09	0	\\
3.24e-09	0	\\
3.34e-09	0	\\
3.45e-09	0	\\
3.55e-09	0	\\
3.65e-09	0	\\
3.75e-09	0	\\
3.86e-09	0	\\
3.96e-09	0	\\
4.06e-09	0	\\
4.16e-09	0	\\
4.27e-09	0	\\
4.37e-09	0	\\
4.47e-09	0	\\
4.57e-09	0	\\
4.68e-09	0	\\
4.78e-09	0	\\
4.89e-09	0	\\
4.99e-09	0	\\
5e-09	0	\\
};
\addplot [color=red,solid,forget plot]
  table[row sep=crcr]{
0	0	\\
1.1e-10	0	\\
2.2e-10	0	\\
3.3e-10	0	\\
4.4e-10	0	\\
5.4e-10	0	\\
6.5e-10	0	\\
7.5e-10	0	\\
8.6e-10	0	\\
9.6e-10	0	\\
1.07e-09	0	\\
1.18e-09	0	\\
1.28e-09	0	\\
1.38e-09	0	\\
1.49e-09	0	\\
1.59e-09	0	\\
1.69e-09	0	\\
1.8e-09	0	\\
1.9e-09	0	\\
2.01e-09	0	\\
2.11e-09	0	\\
2.21e-09	0	\\
2.32e-09	0	\\
2.42e-09	0	\\
2.52e-09	0	\\
2.63e-09	0	\\
2.73e-09	0	\\
2.83e-09	0	\\
2.93e-09	0	\\
3.04e-09	0	\\
3.14e-09	0	\\
3.24e-09	0	\\
3.34e-09	0	\\
3.45e-09	0	\\
3.55e-09	0	\\
3.65e-09	0	\\
3.75e-09	0	\\
3.86e-09	0	\\
3.96e-09	0	\\
4.06e-09	0	\\
4.16e-09	0	\\
4.27e-09	0	\\
4.37e-09	0	\\
4.47e-09	0	\\
4.57e-09	0	\\
4.68e-09	0	\\
4.78e-09	0	\\
4.89e-09	0	\\
4.99e-09	0	\\
5e-09	0	\\
};
\addplot [color=mycolor1,solid,forget plot]
  table[row sep=crcr]{
0	0	\\
1.1e-10	0	\\
2.2e-10	0	\\
3.3e-10	0	\\
4.4e-10	0	\\
5.4e-10	0	\\
6.5e-10	0	\\
7.5e-10	0	\\
8.6e-10	0	\\
9.6e-10	0	\\
1.07e-09	0	\\
1.18e-09	0	\\
1.28e-09	0	\\
1.38e-09	0	\\
1.49e-09	0	\\
1.59e-09	0	\\
1.69e-09	0	\\
1.8e-09	0	\\
1.9e-09	0	\\
2.01e-09	0	\\
2.11e-09	0	\\
2.21e-09	0	\\
2.32e-09	0	\\
2.42e-09	0	\\
2.52e-09	0	\\
2.63e-09	0	\\
2.73e-09	0	\\
2.83e-09	0	\\
2.93e-09	0	\\
3.04e-09	0	\\
3.14e-09	0	\\
3.24e-09	0	\\
3.34e-09	0	\\
3.45e-09	0	\\
3.55e-09	0	\\
3.65e-09	0	\\
3.75e-09	0	\\
3.86e-09	0	\\
3.96e-09	0	\\
4.06e-09	0	\\
4.16e-09	0	\\
4.27e-09	0	\\
4.37e-09	0	\\
4.47e-09	0	\\
4.57e-09	0	\\
4.68e-09	0	\\
4.78e-09	0	\\
4.89e-09	0	\\
4.99e-09	0	\\
5e-09	0	\\
};
\addplot [color=mycolor2,solid,forget plot]
  table[row sep=crcr]{
0	0	\\
1.1e-10	0	\\
2.2e-10	0	\\
3.3e-10	0	\\
4.4e-10	0	\\
5.4e-10	0	\\
6.5e-10	0	\\
7.5e-10	0	\\
8.6e-10	0	\\
9.6e-10	0	\\
1.07e-09	0	\\
1.18e-09	0	\\
1.28e-09	0	\\
1.38e-09	0	\\
1.49e-09	0	\\
1.59e-09	0	\\
1.69e-09	0	\\
1.8e-09	0	\\
1.9e-09	0	\\
2.01e-09	0	\\
2.11e-09	0	\\
2.21e-09	0	\\
2.32e-09	0	\\
2.42e-09	0	\\
2.52e-09	0	\\
2.63e-09	0	\\
2.73e-09	0	\\
2.83e-09	0	\\
2.93e-09	0	\\
3.04e-09	0	\\
3.14e-09	0	\\
3.24e-09	0	\\
3.34e-09	0	\\
3.45e-09	0	\\
3.55e-09	0	\\
3.65e-09	0	\\
3.75e-09	0	\\
3.86e-09	0	\\
3.96e-09	0	\\
4.06e-09	0	\\
4.16e-09	0	\\
4.27e-09	0	\\
4.37e-09	0	\\
4.47e-09	0	\\
4.57e-09	0	\\
4.68e-09	0	\\
4.78e-09	0	\\
4.89e-09	0	\\
4.99e-09	0	\\
5e-09	0	\\
};
\addplot [color=mycolor3,solid,forget plot]
  table[row sep=crcr]{
0	0	\\
1.1e-10	0	\\
2.2e-10	0	\\
3.3e-10	0	\\
4.4e-10	0	\\
5.4e-10	0	\\
6.5e-10	0	\\
7.5e-10	0	\\
8.6e-10	0	\\
9.6e-10	0	\\
1.07e-09	0	\\
1.18e-09	0	\\
1.28e-09	0	\\
1.38e-09	0	\\
1.49e-09	0	\\
1.59e-09	0	\\
1.69e-09	0	\\
1.8e-09	0	\\
1.9e-09	0	\\
2.01e-09	0	\\
2.11e-09	0	\\
2.21e-09	0	\\
2.32e-09	0	\\
2.42e-09	0	\\
2.52e-09	0	\\
2.63e-09	0	\\
2.73e-09	0	\\
2.83e-09	0	\\
2.93e-09	0	\\
3.04e-09	0	\\
3.14e-09	0	\\
3.24e-09	0	\\
3.34e-09	0	\\
3.45e-09	0	\\
3.55e-09	0	\\
3.65e-09	0	\\
3.75e-09	0	\\
3.86e-09	0	\\
3.96e-09	0	\\
4.06e-09	0	\\
4.16e-09	0	\\
4.27e-09	0	\\
4.37e-09	0	\\
4.47e-09	0	\\
4.57e-09	0	\\
4.68e-09	0	\\
4.78e-09	0	\\
4.89e-09	0	\\
4.99e-09	0	\\
5e-09	0	\\
};
\addplot [color=darkgray,solid,forget plot]
  table[row sep=crcr]{
0	0	\\
1.1e-10	0	\\
2.2e-10	0	\\
3.3e-10	0	\\
4.4e-10	0	\\
5.4e-10	0	\\
6.5e-10	0	\\
7.5e-10	0	\\
8.6e-10	0	\\
9.6e-10	0	\\
1.07e-09	0	\\
1.18e-09	0	\\
1.28e-09	0	\\
1.38e-09	0	\\
1.49e-09	0	\\
1.59e-09	0	\\
1.69e-09	0	\\
1.8e-09	0	\\
1.9e-09	0	\\
2.01e-09	0	\\
2.11e-09	0	\\
2.21e-09	0	\\
2.32e-09	0	\\
2.42e-09	0	\\
2.52e-09	0	\\
2.63e-09	0	\\
2.73e-09	0	\\
2.83e-09	0	\\
2.93e-09	0	\\
3.04e-09	0	\\
3.14e-09	0	\\
3.24e-09	0	\\
3.34e-09	0	\\
3.45e-09	0	\\
3.55e-09	0	\\
3.65e-09	0	\\
3.75e-09	0	\\
3.86e-09	0	\\
3.96e-09	0	\\
4.06e-09	0	\\
4.16e-09	0	\\
4.27e-09	0	\\
4.37e-09	0	\\
4.47e-09	0	\\
4.57e-09	0	\\
4.68e-09	0	\\
4.78e-09	0	\\
4.89e-09	0	\\
4.99e-09	0	\\
5e-09	0	\\
};
\addplot [color=blue,solid,forget plot]
  table[row sep=crcr]{
0	0	\\
1.1e-10	0	\\
2.2e-10	0	\\
3.3e-10	0	\\
4.4e-10	0	\\
5.4e-10	0	\\
6.5e-10	0	\\
7.5e-10	0	\\
8.6e-10	0	\\
9.6e-10	0	\\
1.07e-09	0	\\
1.18e-09	0	\\
1.28e-09	0	\\
1.38e-09	0	\\
1.49e-09	0	\\
1.59e-09	0	\\
1.69e-09	0	\\
1.8e-09	0	\\
1.9e-09	0	\\
2.01e-09	0	\\
2.11e-09	0	\\
2.21e-09	0	\\
2.32e-09	0	\\
2.42e-09	0	\\
2.52e-09	0	\\
2.63e-09	0	\\
2.73e-09	0	\\
2.83e-09	0	\\
2.93e-09	0	\\
3.04e-09	0	\\
3.14e-09	0	\\
3.24e-09	0	\\
3.34e-09	0	\\
3.45e-09	0	\\
3.55e-09	0	\\
3.65e-09	0	\\
3.75e-09	0	\\
3.86e-09	0	\\
3.96e-09	0	\\
4.06e-09	0	\\
4.16e-09	0	\\
4.27e-09	0	\\
4.37e-09	0	\\
4.47e-09	0	\\
4.57e-09	0	\\
4.68e-09	0	\\
4.78e-09	0	\\
4.89e-09	0	\\
4.99e-09	0	\\
5e-09	0	\\
};
\addplot [color=black!50!green,solid,forget plot]
  table[row sep=crcr]{
0	0	\\
1.1e-10	0	\\
2.2e-10	0	\\
3.3e-10	0	\\
4.4e-10	0	\\
5.4e-10	0	\\
6.5e-10	0	\\
7.5e-10	0	\\
8.6e-10	0	\\
9.6e-10	0	\\
1.07e-09	0	\\
1.18e-09	0	\\
1.28e-09	0	\\
1.38e-09	0	\\
1.49e-09	0	\\
1.59e-09	0	\\
1.69e-09	0	\\
1.8e-09	0	\\
1.9e-09	0	\\
2.01e-09	0	\\
2.11e-09	0	\\
2.21e-09	0	\\
2.32e-09	0	\\
2.42e-09	0	\\
2.52e-09	0	\\
2.63e-09	0	\\
2.73e-09	0	\\
2.83e-09	0	\\
2.93e-09	0	\\
3.04e-09	0	\\
3.14e-09	0	\\
3.24e-09	0	\\
3.34e-09	0	\\
3.45e-09	0	\\
3.55e-09	0	\\
3.65e-09	0	\\
3.75e-09	0	\\
3.86e-09	0	\\
3.96e-09	0	\\
4.06e-09	0	\\
4.16e-09	0	\\
4.27e-09	0	\\
4.37e-09	0	\\
4.47e-09	0	\\
4.57e-09	0	\\
4.68e-09	0	\\
4.78e-09	0	\\
4.89e-09	0	\\
4.99e-09	0	\\
5e-09	0	\\
};
\addplot [color=red,solid,forget plot]
  table[row sep=crcr]{
0	0	\\
1.1e-10	0	\\
2.2e-10	0	\\
3.3e-10	0	\\
4.4e-10	0	\\
5.4e-10	0	\\
6.5e-10	0	\\
7.5e-10	0	\\
8.6e-10	0	\\
9.6e-10	0	\\
1.07e-09	0	\\
1.18e-09	0	\\
1.28e-09	0	\\
1.38e-09	0	\\
1.49e-09	0	\\
1.59e-09	0	\\
1.69e-09	0	\\
1.8e-09	0	\\
1.9e-09	0	\\
2.01e-09	0	\\
2.11e-09	0	\\
2.21e-09	0	\\
2.32e-09	0	\\
2.42e-09	0	\\
2.52e-09	0	\\
2.63e-09	0	\\
2.73e-09	0	\\
2.83e-09	0	\\
2.93e-09	0	\\
3.04e-09	0	\\
3.14e-09	0	\\
3.24e-09	0	\\
3.34e-09	0	\\
3.45e-09	0	\\
3.55e-09	0	\\
3.65e-09	0	\\
3.75e-09	0	\\
3.86e-09	0	\\
3.96e-09	0	\\
4.06e-09	0	\\
4.16e-09	0	\\
4.27e-09	0	\\
4.37e-09	0	\\
4.47e-09	0	\\
4.57e-09	0	\\
4.68e-09	0	\\
4.78e-09	0	\\
4.89e-09	0	\\
4.99e-09	0	\\
5e-09	0	\\
};
\addplot [color=mycolor1,solid,forget plot]
  table[row sep=crcr]{
0	0	\\
1.1e-10	0	\\
2.2e-10	0	\\
3.3e-10	0	\\
4.4e-10	0	\\
5.4e-10	0	\\
6.5e-10	0	\\
7.5e-10	0	\\
8.6e-10	0	\\
9.6e-10	0	\\
1.07e-09	0	\\
1.18e-09	0	\\
1.28e-09	0	\\
1.38e-09	0	\\
1.49e-09	0	\\
1.59e-09	0	\\
1.69e-09	0	\\
1.8e-09	0	\\
1.9e-09	0	\\
2.01e-09	0	\\
2.11e-09	0	\\
2.21e-09	0	\\
2.32e-09	0	\\
2.42e-09	0	\\
2.52e-09	0	\\
2.63e-09	0	\\
2.73e-09	0	\\
2.83e-09	0	\\
2.93e-09	0	\\
3.04e-09	0	\\
3.14e-09	0	\\
3.24e-09	0	\\
3.34e-09	0	\\
3.45e-09	0	\\
3.55e-09	0	\\
3.65e-09	0	\\
3.75e-09	0	\\
3.86e-09	0	\\
3.96e-09	0	\\
4.06e-09	0	\\
4.16e-09	0	\\
4.27e-09	0	\\
4.37e-09	0	\\
4.47e-09	0	\\
4.57e-09	0	\\
4.68e-09	0	\\
4.78e-09	0	\\
4.89e-09	0	\\
4.99e-09	0	\\
5e-09	0	\\
};
\addplot [color=mycolor2,solid,forget plot]
  table[row sep=crcr]{
0	0	\\
1.1e-10	0	\\
2.2e-10	0	\\
3.3e-10	0	\\
4.4e-10	0	\\
5.4e-10	0	\\
6.5e-10	0	\\
7.5e-10	0	\\
8.6e-10	0	\\
9.6e-10	0	\\
1.07e-09	0	\\
1.18e-09	0	\\
1.28e-09	0	\\
1.38e-09	0	\\
1.49e-09	0	\\
1.59e-09	0	\\
1.69e-09	0	\\
1.8e-09	0	\\
1.9e-09	0	\\
2.01e-09	0	\\
2.11e-09	0	\\
2.21e-09	0	\\
2.32e-09	0	\\
2.42e-09	0	\\
2.52e-09	0	\\
2.63e-09	0	\\
2.73e-09	0	\\
2.83e-09	0	\\
2.93e-09	0	\\
3.04e-09	0	\\
3.14e-09	0	\\
3.24e-09	0	\\
3.34e-09	0	\\
3.45e-09	0	\\
3.55e-09	0	\\
3.65e-09	0	\\
3.75e-09	0	\\
3.86e-09	0	\\
3.96e-09	0	\\
4.06e-09	0	\\
4.16e-09	0	\\
4.27e-09	0	\\
4.37e-09	0	\\
4.47e-09	0	\\
4.57e-09	0	\\
4.68e-09	0	\\
4.78e-09	0	\\
4.89e-09	0	\\
4.99e-09	0	\\
5e-09	0	\\
};
\addplot [color=mycolor3,solid,forget plot]
  table[row sep=crcr]{
0	0	\\
1.1e-10	0	\\
2.2e-10	0	\\
3.3e-10	0	\\
4.4e-10	0	\\
5.4e-10	0	\\
6.5e-10	0	\\
7.5e-10	0	\\
8.6e-10	0	\\
9.6e-10	0	\\
1.07e-09	0	\\
1.18e-09	0	\\
1.28e-09	0	\\
1.38e-09	0	\\
1.49e-09	0	\\
1.59e-09	0	\\
1.69e-09	0	\\
1.8e-09	0	\\
1.9e-09	0	\\
2.01e-09	0	\\
2.11e-09	0	\\
2.21e-09	0	\\
2.32e-09	0	\\
2.42e-09	0	\\
2.52e-09	0	\\
2.63e-09	0	\\
2.73e-09	0	\\
2.83e-09	0	\\
2.93e-09	0	\\
3.04e-09	0	\\
3.14e-09	0	\\
3.24e-09	0	\\
3.34e-09	0	\\
3.45e-09	0	\\
3.55e-09	0	\\
3.65e-09	0	\\
3.75e-09	0	\\
3.86e-09	0	\\
3.96e-09	0	\\
4.06e-09	0	\\
4.16e-09	0	\\
4.27e-09	0	\\
4.37e-09	0	\\
4.47e-09	0	\\
4.57e-09	0	\\
4.68e-09	0	\\
4.78e-09	0	\\
4.89e-09	0	\\
4.99e-09	0	\\
5e-09	0	\\
};
\addplot [color=darkgray,solid,forget plot]
  table[row sep=crcr]{
0	0	\\
1.1e-10	0	\\
2.2e-10	0	\\
3.3e-10	0	\\
4.4e-10	0	\\
5.4e-10	0	\\
6.5e-10	0	\\
7.5e-10	0	\\
8.6e-10	0	\\
9.6e-10	0	\\
1.07e-09	0	\\
1.18e-09	0	\\
1.28e-09	0	\\
1.38e-09	0	\\
1.49e-09	0	\\
1.59e-09	0	\\
1.69e-09	0	\\
1.8e-09	0	\\
1.9e-09	0	\\
2.01e-09	0	\\
2.11e-09	0	\\
2.21e-09	0	\\
2.32e-09	0	\\
2.42e-09	0	\\
2.52e-09	0	\\
2.63e-09	0	\\
2.73e-09	0	\\
2.83e-09	0	\\
2.93e-09	0	\\
3.04e-09	0	\\
3.14e-09	0	\\
3.24e-09	0	\\
3.34e-09	0	\\
3.45e-09	0	\\
3.55e-09	0	\\
3.65e-09	0	\\
3.75e-09	0	\\
3.86e-09	0	\\
3.96e-09	0	\\
4.06e-09	0	\\
4.16e-09	0	\\
4.27e-09	0	\\
4.37e-09	0	\\
4.47e-09	0	\\
4.57e-09	0	\\
4.68e-09	0	\\
4.78e-09	0	\\
4.89e-09	0	\\
4.99e-09	0	\\
5e-09	0	\\
};
\addplot [color=blue,solid,forget plot]
  table[row sep=crcr]{
0	0	\\
1.1e-10	0	\\
2.2e-10	0	\\
3.3e-10	0	\\
4.4e-10	0	\\
5.4e-10	0	\\
6.5e-10	0	\\
7.5e-10	0	\\
8.6e-10	0	\\
9.6e-10	0	\\
1.07e-09	0	\\
1.18e-09	0	\\
1.28e-09	0	\\
1.38e-09	0	\\
1.49e-09	0	\\
1.59e-09	0	\\
1.69e-09	0	\\
1.8e-09	0	\\
1.9e-09	0	\\
2.01e-09	0	\\
2.11e-09	0	\\
2.21e-09	0	\\
2.32e-09	0	\\
2.42e-09	0	\\
2.52e-09	0	\\
2.63e-09	0	\\
2.73e-09	0	\\
2.83e-09	0	\\
2.93e-09	0	\\
3.04e-09	0	\\
3.14e-09	0	\\
3.24e-09	0	\\
3.34e-09	0	\\
3.45e-09	0	\\
3.55e-09	0	\\
3.65e-09	0	\\
3.75e-09	0	\\
3.86e-09	0	\\
3.96e-09	0	\\
4.06e-09	0	\\
4.16e-09	0	\\
4.27e-09	0	\\
4.37e-09	0	\\
4.47e-09	0	\\
4.57e-09	0	\\
4.68e-09	0	\\
4.78e-09	0	\\
4.89e-09	0	\\
4.99e-09	0	\\
5e-09	0	\\
};
\addplot [color=black!50!green,solid,forget plot]
  table[row sep=crcr]{
0	0	\\
1.1e-10	0	\\
2.2e-10	0	\\
3.3e-10	0	\\
4.4e-10	0	\\
5.4e-10	0	\\
6.5e-10	0	\\
7.5e-10	0	\\
8.6e-10	0	\\
9.6e-10	0	\\
1.07e-09	0	\\
1.18e-09	0	\\
1.28e-09	0	\\
1.38e-09	0	\\
1.49e-09	0	\\
1.59e-09	0	\\
1.69e-09	0	\\
1.8e-09	0	\\
1.9e-09	0	\\
2.01e-09	0	\\
2.11e-09	0	\\
2.21e-09	0	\\
2.32e-09	0	\\
2.42e-09	0	\\
2.52e-09	0	\\
2.63e-09	0	\\
2.73e-09	0	\\
2.83e-09	0	\\
2.93e-09	0	\\
3.04e-09	0	\\
3.14e-09	0	\\
3.24e-09	0	\\
3.34e-09	0	\\
3.45e-09	0	\\
3.55e-09	0	\\
3.65e-09	0	\\
3.75e-09	0	\\
3.86e-09	0	\\
3.96e-09	0	\\
4.06e-09	0	\\
4.16e-09	0	\\
4.27e-09	0	\\
4.37e-09	0	\\
4.47e-09	0	\\
4.57e-09	0	\\
4.68e-09	0	\\
4.78e-09	0	\\
4.89e-09	0	\\
4.99e-09	0	\\
5e-09	0	\\
};
\addplot [color=red,solid,forget plot]
  table[row sep=crcr]{
0	0	\\
1.1e-10	0	\\
2.2e-10	0	\\
3.3e-10	0	\\
4.4e-10	0	\\
5.4e-10	0	\\
6.5e-10	0	\\
7.5e-10	0	\\
8.6e-10	0	\\
9.6e-10	0	\\
1.07e-09	0	\\
1.18e-09	0	\\
1.28e-09	0	\\
1.38e-09	0	\\
1.49e-09	0	\\
1.59e-09	0	\\
1.69e-09	0	\\
1.8e-09	0	\\
1.9e-09	0	\\
2.01e-09	0	\\
2.11e-09	0	\\
2.21e-09	0	\\
2.32e-09	0	\\
2.42e-09	0	\\
2.52e-09	0	\\
2.63e-09	0	\\
2.73e-09	0	\\
2.83e-09	0	\\
2.93e-09	0	\\
3.04e-09	0	\\
3.14e-09	0	\\
3.24e-09	0	\\
3.34e-09	0	\\
3.45e-09	0	\\
3.55e-09	0	\\
3.65e-09	0	\\
3.75e-09	0	\\
3.86e-09	0	\\
3.96e-09	0	\\
4.06e-09	0	\\
4.16e-09	0	\\
4.27e-09	0	\\
4.37e-09	0	\\
4.47e-09	0	\\
4.57e-09	0	\\
4.68e-09	0	\\
4.78e-09	0	\\
4.89e-09	0	\\
4.99e-09	0	\\
5e-09	0	\\
};
\addplot [color=mycolor1,solid,forget plot]
  table[row sep=crcr]{
0	0	\\
1.1e-10	0	\\
2.2e-10	0	\\
3.3e-10	0	\\
4.4e-10	0	\\
5.4e-10	0	\\
6.5e-10	0	\\
7.5e-10	0	\\
8.6e-10	0	\\
9.6e-10	0	\\
1.07e-09	0	\\
1.18e-09	0	\\
1.28e-09	0	\\
1.38e-09	0	\\
1.49e-09	0	\\
1.59e-09	0	\\
1.69e-09	0	\\
1.8e-09	0	\\
1.9e-09	0	\\
2.01e-09	0	\\
2.11e-09	0	\\
2.21e-09	0	\\
2.32e-09	0	\\
2.42e-09	0	\\
2.52e-09	0	\\
2.63e-09	0	\\
2.73e-09	0	\\
2.83e-09	0	\\
2.93e-09	0	\\
3.04e-09	0	\\
3.14e-09	0	\\
3.24e-09	0	\\
3.34e-09	0	\\
3.45e-09	0	\\
3.55e-09	0	\\
3.65e-09	0	\\
3.75e-09	0	\\
3.86e-09	0	\\
3.96e-09	0	\\
4.06e-09	0	\\
4.16e-09	0	\\
4.27e-09	0	\\
4.37e-09	0	\\
4.47e-09	0	\\
4.57e-09	0	\\
4.68e-09	0	\\
4.78e-09	0	\\
4.89e-09	0	\\
4.99e-09	0	\\
5e-09	0	\\
};
\addplot [color=mycolor2,solid,forget plot]
  table[row sep=crcr]{
0	0	\\
1.1e-10	0	\\
2.2e-10	0	\\
3.3e-10	0	\\
4.4e-10	0	\\
5.4e-10	0	\\
6.5e-10	0	\\
7.5e-10	0	\\
8.6e-10	0	\\
9.6e-10	0	\\
1.07e-09	0	\\
1.18e-09	0	\\
1.28e-09	0	\\
1.38e-09	0	\\
1.49e-09	0	\\
1.59e-09	0	\\
1.69e-09	0	\\
1.8e-09	0	\\
1.9e-09	0	\\
2.01e-09	0	\\
2.11e-09	0	\\
2.21e-09	0	\\
2.32e-09	0	\\
2.42e-09	0	\\
2.52e-09	0	\\
2.63e-09	0	\\
2.73e-09	0	\\
2.83e-09	0	\\
2.93e-09	0	\\
3.04e-09	0	\\
3.14e-09	0	\\
3.24e-09	0	\\
3.34e-09	0	\\
3.45e-09	0	\\
3.55e-09	0	\\
3.65e-09	0	\\
3.75e-09	0	\\
3.86e-09	0	\\
3.96e-09	0	\\
4.06e-09	0	\\
4.16e-09	0	\\
4.27e-09	0	\\
4.37e-09	0	\\
4.47e-09	0	\\
4.57e-09	0	\\
4.68e-09	0	\\
4.78e-09	0	\\
4.89e-09	0	\\
4.99e-09	0	\\
5e-09	0	\\
};
\addplot [color=mycolor3,solid,forget plot]
  table[row sep=crcr]{
0	0	\\
1.1e-10	0	\\
2.2e-10	0	\\
3.3e-10	0	\\
4.4e-10	0	\\
5.4e-10	0	\\
6.5e-10	0	\\
7.5e-10	0	\\
8.6e-10	0	\\
9.6e-10	0	\\
1.07e-09	0	\\
1.18e-09	0	\\
1.28e-09	0	\\
1.38e-09	0	\\
1.49e-09	0	\\
1.59e-09	0	\\
1.69e-09	0	\\
1.8e-09	0	\\
1.9e-09	0	\\
2.01e-09	0	\\
2.11e-09	0	\\
2.21e-09	0	\\
2.32e-09	0	\\
2.42e-09	0	\\
2.52e-09	0	\\
2.63e-09	0	\\
2.73e-09	0	\\
2.83e-09	0	\\
2.93e-09	0	\\
3.04e-09	0	\\
3.14e-09	0	\\
3.24e-09	0	\\
3.34e-09	0	\\
3.45e-09	0	\\
3.55e-09	0	\\
3.65e-09	0	\\
3.75e-09	0	\\
3.86e-09	0	\\
3.96e-09	0	\\
4.06e-09	0	\\
4.16e-09	0	\\
4.27e-09	0	\\
4.37e-09	0	\\
4.47e-09	0	\\
4.57e-09	0	\\
4.68e-09	0	\\
4.78e-09	0	\\
4.89e-09	0	\\
4.99e-09	0	\\
5e-09	0	\\
};
\addplot [color=darkgray,solid,forget plot]
  table[row sep=crcr]{
0	0	\\
1.1e-10	0	\\
2.2e-10	0	\\
3.3e-10	0	\\
4.4e-10	0	\\
5.4e-10	0	\\
6.5e-10	0	\\
7.5e-10	0	\\
8.6e-10	0	\\
9.6e-10	0	\\
1.07e-09	0	\\
1.18e-09	0	\\
1.28e-09	0	\\
1.38e-09	0	\\
1.49e-09	0	\\
1.59e-09	0	\\
1.69e-09	0	\\
1.8e-09	0	\\
1.9e-09	0	\\
2.01e-09	0	\\
2.11e-09	0	\\
2.21e-09	0	\\
2.32e-09	0	\\
2.42e-09	0	\\
2.52e-09	0	\\
2.63e-09	0	\\
2.73e-09	0	\\
2.83e-09	0	\\
2.93e-09	0	\\
3.04e-09	0	\\
3.14e-09	0	\\
3.24e-09	0	\\
3.34e-09	0	\\
3.45e-09	0	\\
3.55e-09	0	\\
3.65e-09	0	\\
3.75e-09	0	\\
3.86e-09	0	\\
3.96e-09	0	\\
4.06e-09	0	\\
4.16e-09	0	\\
4.27e-09	0	\\
4.37e-09	0	\\
4.47e-09	0	\\
4.57e-09	0	\\
4.68e-09	0	\\
4.78e-09	0	\\
4.89e-09	0	\\
4.99e-09	0	\\
5e-09	0	\\
};
\addplot [color=blue,solid,forget plot]
  table[row sep=crcr]{
0	0	\\
1.1e-10	0	\\
2.2e-10	0	\\
3.3e-10	0	\\
4.4e-10	0	\\
5.4e-10	0	\\
6.5e-10	0	\\
7.5e-10	0	\\
8.6e-10	0	\\
9.6e-10	0	\\
1.07e-09	0	\\
1.18e-09	0	\\
1.28e-09	0	\\
1.38e-09	0	\\
1.49e-09	0	\\
1.59e-09	0	\\
1.69e-09	0	\\
1.8e-09	0	\\
1.9e-09	0	\\
2.01e-09	0	\\
2.11e-09	0	\\
2.21e-09	0	\\
2.32e-09	0	\\
2.42e-09	0	\\
2.52e-09	0	\\
2.63e-09	0	\\
2.73e-09	0	\\
2.83e-09	0	\\
2.93e-09	0	\\
3.04e-09	0	\\
3.14e-09	0	\\
3.24e-09	0	\\
3.34e-09	0	\\
3.45e-09	0	\\
3.55e-09	0	\\
3.65e-09	0	\\
3.75e-09	0	\\
3.86e-09	0	\\
3.96e-09	0	\\
4.06e-09	0	\\
4.16e-09	0	\\
4.27e-09	0	\\
4.37e-09	0	\\
4.47e-09	0	\\
4.57e-09	0	\\
4.68e-09	0	\\
4.78e-09	0	\\
4.89e-09	0	\\
4.99e-09	0	\\
5e-09	0	\\
};
\addplot [color=black!50!green,solid,forget plot]
  table[row sep=crcr]{
0	0	\\
1.1e-10	0	\\
2.2e-10	0	\\
3.3e-10	0	\\
4.4e-10	0	\\
5.4e-10	0	\\
6.5e-10	0	\\
7.5e-10	0	\\
8.6e-10	0	\\
9.6e-10	0	\\
1.07e-09	0	\\
1.18e-09	0	\\
1.28e-09	0	\\
1.38e-09	0	\\
1.49e-09	0	\\
1.59e-09	0	\\
1.69e-09	0	\\
1.8e-09	0	\\
1.9e-09	0	\\
2.01e-09	0	\\
2.11e-09	0	\\
2.21e-09	0	\\
2.32e-09	0	\\
2.42e-09	0	\\
2.52e-09	0	\\
2.63e-09	0	\\
2.73e-09	0	\\
2.83e-09	0	\\
2.93e-09	0	\\
3.04e-09	0	\\
3.14e-09	0	\\
3.24e-09	0	\\
3.34e-09	0	\\
3.45e-09	0	\\
3.55e-09	0	\\
3.65e-09	0	\\
3.75e-09	0	\\
3.86e-09	0	\\
3.96e-09	0	\\
4.06e-09	0	\\
4.16e-09	0	\\
4.27e-09	0	\\
4.37e-09	0	\\
4.47e-09	0	\\
4.57e-09	0	\\
4.68e-09	0	\\
4.78e-09	0	\\
4.89e-09	0	\\
4.99e-09	0	\\
5e-09	0	\\
};
\addplot [color=red,solid,forget plot]
  table[row sep=crcr]{
0	0	\\
1.1e-10	0	\\
2.2e-10	0	\\
3.3e-10	0	\\
4.4e-10	0	\\
5.4e-10	0	\\
6.5e-10	0	\\
7.5e-10	0	\\
8.6e-10	0	\\
9.6e-10	0	\\
1.07e-09	0	\\
1.18e-09	0	\\
1.28e-09	0	\\
1.38e-09	0	\\
1.49e-09	0	\\
1.59e-09	0	\\
1.69e-09	0	\\
1.8e-09	0	\\
1.9e-09	0	\\
2.01e-09	0	\\
2.11e-09	0	\\
2.21e-09	0	\\
2.32e-09	0	\\
2.42e-09	0	\\
2.52e-09	0	\\
2.63e-09	0	\\
2.73e-09	0	\\
2.83e-09	0	\\
2.93e-09	0	\\
3.04e-09	0	\\
3.14e-09	0	\\
3.24e-09	0	\\
3.34e-09	0	\\
3.45e-09	0	\\
3.55e-09	0	\\
3.65e-09	0	\\
3.75e-09	0	\\
3.86e-09	0	\\
3.96e-09	0	\\
4.06e-09	0	\\
4.16e-09	0	\\
4.27e-09	0	\\
4.37e-09	0	\\
4.47e-09	0	\\
4.57e-09	0	\\
4.68e-09	0	\\
4.78e-09	0	\\
4.89e-09	0	\\
4.99e-09	0	\\
5e-09	0	\\
};
\addplot [color=mycolor1,solid,forget plot]
  table[row sep=crcr]{
0	0	\\
1.1e-10	0	\\
2.2e-10	0	\\
3.3e-10	0	\\
4.4e-10	0	\\
5.4e-10	0	\\
6.5e-10	0	\\
7.5e-10	0	\\
8.6e-10	0	\\
9.6e-10	0	\\
1.07e-09	0	\\
1.18e-09	0	\\
1.28e-09	0	\\
1.38e-09	0	\\
1.49e-09	0	\\
1.59e-09	0	\\
1.69e-09	0	\\
1.8e-09	0	\\
1.9e-09	0	\\
2.01e-09	0	\\
2.11e-09	0	\\
2.21e-09	0	\\
2.32e-09	0	\\
2.42e-09	0	\\
2.52e-09	0	\\
2.63e-09	0	\\
2.73e-09	0	\\
2.83e-09	0	\\
2.93e-09	0	\\
3.04e-09	0	\\
3.14e-09	0	\\
3.24e-09	0	\\
3.34e-09	0	\\
3.45e-09	0	\\
3.55e-09	0	\\
3.65e-09	0	\\
3.75e-09	0	\\
3.86e-09	0	\\
3.96e-09	0	\\
4.06e-09	0	\\
4.16e-09	0	\\
4.27e-09	0	\\
4.37e-09	0	\\
4.47e-09	0	\\
4.57e-09	0	\\
4.68e-09	0	\\
4.78e-09	0	\\
4.89e-09	0	\\
4.99e-09	0	\\
5e-09	0	\\
};
\addplot [color=mycolor2,solid,forget plot]
  table[row sep=crcr]{
0	0	\\
1.1e-10	0	\\
2.2e-10	0	\\
3.3e-10	0	\\
4.4e-10	0	\\
5.4e-10	0	\\
6.5e-10	0	\\
7.5e-10	0	\\
8.6e-10	0	\\
9.6e-10	0	\\
1.07e-09	0	\\
1.18e-09	0	\\
1.28e-09	0	\\
1.38e-09	0	\\
1.49e-09	0	\\
1.59e-09	0	\\
1.69e-09	0	\\
1.8e-09	0	\\
1.9e-09	0	\\
2.01e-09	0	\\
2.11e-09	0	\\
2.21e-09	0	\\
2.32e-09	0	\\
2.42e-09	0	\\
2.52e-09	0	\\
2.63e-09	0	\\
2.73e-09	0	\\
2.83e-09	0	\\
2.93e-09	0	\\
3.04e-09	0	\\
3.14e-09	0	\\
3.24e-09	0	\\
3.34e-09	0	\\
3.45e-09	0	\\
3.55e-09	0	\\
3.65e-09	0	\\
3.75e-09	0	\\
3.86e-09	0	\\
3.96e-09	0	\\
4.06e-09	0	\\
4.16e-09	0	\\
4.27e-09	0	\\
4.37e-09	0	\\
4.47e-09	0	\\
4.57e-09	0	\\
4.68e-09	0	\\
4.78e-09	0	\\
4.89e-09	0	\\
4.99e-09	0	\\
5e-09	0	\\
};
\addplot [color=mycolor3,solid,forget plot]
  table[row sep=crcr]{
0	0	\\
1.1e-10	0	\\
2.2e-10	0	\\
3.3e-10	0	\\
4.4e-10	0	\\
5.4e-10	0	\\
6.5e-10	0	\\
7.5e-10	0	\\
8.6e-10	0	\\
9.6e-10	0	\\
1.07e-09	0	\\
1.18e-09	0	\\
1.28e-09	0	\\
1.38e-09	0	\\
1.49e-09	0	\\
1.59e-09	0	\\
1.69e-09	0	\\
1.8e-09	0	\\
1.9e-09	0	\\
2.01e-09	0	\\
2.11e-09	0	\\
2.21e-09	0	\\
2.32e-09	0	\\
2.42e-09	0	\\
2.52e-09	0	\\
2.63e-09	0	\\
2.73e-09	0	\\
2.83e-09	0	\\
2.93e-09	0	\\
3.04e-09	0	\\
3.14e-09	0	\\
3.24e-09	0	\\
3.34e-09	0	\\
3.45e-09	0	\\
3.55e-09	0	\\
3.65e-09	0	\\
3.75e-09	0	\\
3.86e-09	0	\\
3.96e-09	0	\\
4.06e-09	0	\\
4.16e-09	0	\\
4.27e-09	0	\\
4.37e-09	0	\\
4.47e-09	0	\\
4.57e-09	0	\\
4.68e-09	0	\\
4.78e-09	0	\\
4.89e-09	0	\\
4.99e-09	0	\\
5e-09	0	\\
};
\addplot [color=darkgray,solid,forget plot]
  table[row sep=crcr]{
0	0	\\
1.1e-10	0	\\
2.2e-10	0	\\
3.3e-10	0	\\
4.4e-10	0	\\
5.4e-10	0	\\
6.5e-10	0	\\
7.5e-10	0	\\
8.6e-10	0	\\
9.6e-10	0	\\
1.07e-09	0	\\
1.18e-09	0	\\
1.28e-09	0	\\
1.38e-09	0	\\
1.49e-09	0	\\
1.59e-09	0	\\
1.69e-09	0	\\
1.8e-09	0	\\
1.9e-09	0	\\
2.01e-09	0	\\
2.11e-09	0	\\
2.21e-09	0	\\
2.32e-09	0	\\
2.42e-09	0	\\
2.52e-09	0	\\
2.63e-09	0	\\
2.73e-09	0	\\
2.83e-09	0	\\
2.93e-09	0	\\
3.04e-09	0	\\
3.14e-09	0	\\
3.24e-09	0	\\
3.34e-09	0	\\
3.45e-09	0	\\
3.55e-09	0	\\
3.65e-09	0	\\
3.75e-09	0	\\
3.86e-09	0	\\
3.96e-09	0	\\
4.06e-09	0	\\
4.16e-09	0	\\
4.27e-09	0	\\
4.37e-09	0	\\
4.47e-09	0	\\
4.57e-09	0	\\
4.68e-09	0	\\
4.78e-09	0	\\
4.89e-09	0	\\
4.99e-09	0	\\
5e-09	0	\\
};
\addplot [color=blue,solid,forget plot]
  table[row sep=crcr]{
0	0	\\
1.1e-10	0	\\
2.2e-10	0	\\
3.3e-10	0	\\
4.4e-10	0	\\
5.4e-10	0	\\
6.5e-10	0	\\
7.5e-10	0	\\
8.6e-10	0	\\
9.6e-10	0	\\
1.07e-09	0	\\
1.18e-09	0	\\
1.28e-09	0	\\
1.38e-09	0	\\
1.49e-09	0	\\
1.59e-09	0	\\
1.69e-09	0	\\
1.8e-09	0	\\
1.9e-09	0	\\
2.01e-09	0	\\
2.11e-09	0	\\
2.21e-09	0	\\
2.32e-09	0	\\
2.42e-09	0	\\
2.52e-09	0	\\
2.63e-09	0	\\
2.73e-09	0	\\
2.83e-09	0	\\
2.93e-09	0	\\
3.04e-09	0	\\
3.14e-09	0	\\
3.24e-09	0	\\
3.34e-09	0	\\
3.45e-09	0	\\
3.55e-09	0	\\
3.65e-09	0	\\
3.75e-09	0	\\
3.86e-09	0	\\
3.96e-09	0	\\
4.06e-09	0	\\
4.16e-09	0	\\
4.27e-09	0	\\
4.37e-09	0	\\
4.47e-09	0	\\
4.57e-09	0	\\
4.68e-09	0	\\
4.78e-09	0	\\
4.89e-09	0	\\
4.99e-09	0	\\
5e-09	0	\\
};
\addplot [color=black!50!green,solid,forget plot]
  table[row sep=crcr]{
0	0	\\
1.1e-10	0	\\
2.2e-10	0	\\
3.3e-10	0	\\
4.4e-10	0	\\
5.4e-10	0	\\
6.5e-10	0	\\
7.5e-10	0	\\
8.6e-10	0	\\
9.6e-10	0	\\
1.07e-09	0	\\
1.18e-09	0	\\
1.28e-09	0	\\
1.38e-09	0	\\
1.49e-09	0	\\
1.59e-09	0	\\
1.69e-09	0	\\
1.8e-09	0	\\
1.9e-09	0	\\
2.01e-09	0	\\
2.11e-09	0	\\
2.21e-09	0	\\
2.32e-09	0	\\
2.42e-09	0	\\
2.52e-09	0	\\
2.63e-09	0	\\
2.73e-09	0	\\
2.83e-09	0	\\
2.93e-09	0	\\
3.04e-09	0	\\
3.14e-09	0	\\
3.24e-09	0	\\
3.34e-09	0	\\
3.45e-09	0	\\
3.55e-09	0	\\
3.65e-09	0	\\
3.75e-09	0	\\
3.86e-09	0	\\
3.96e-09	0	\\
4.06e-09	0	\\
4.16e-09	0	\\
4.27e-09	0	\\
4.37e-09	0	\\
4.47e-09	0	\\
4.57e-09	0	\\
4.68e-09	0	\\
4.78e-09	0	\\
4.89e-09	0	\\
4.99e-09	0	\\
5e-09	0	\\
};
\addplot [color=red,solid,forget plot]
  table[row sep=crcr]{
0	0	\\
1.1e-10	0	\\
2.2e-10	0	\\
3.3e-10	0	\\
4.4e-10	0	\\
5.4e-10	0	\\
6.5e-10	0	\\
7.5e-10	0	\\
8.6e-10	0	\\
9.6e-10	0	\\
1.07e-09	0	\\
1.18e-09	0	\\
1.28e-09	0	\\
1.38e-09	0	\\
1.49e-09	0	\\
1.59e-09	0	\\
1.69e-09	0	\\
1.8e-09	0	\\
1.9e-09	0	\\
2.01e-09	0	\\
2.11e-09	0	\\
2.21e-09	0	\\
2.32e-09	0	\\
2.42e-09	0	\\
2.52e-09	0	\\
2.63e-09	0	\\
2.73e-09	0	\\
2.83e-09	0	\\
2.93e-09	0	\\
3.04e-09	0	\\
3.14e-09	0	\\
3.24e-09	0	\\
3.34e-09	0	\\
3.45e-09	0	\\
3.55e-09	0	\\
3.65e-09	0	\\
3.75e-09	0	\\
3.86e-09	0	\\
3.96e-09	0	\\
4.06e-09	0	\\
4.16e-09	0	\\
4.27e-09	0	\\
4.37e-09	0	\\
4.47e-09	0	\\
4.57e-09	0	\\
4.68e-09	0	\\
4.78e-09	0	\\
4.89e-09	0	\\
4.99e-09	0	\\
5e-09	0	\\
};
\addplot [color=mycolor1,solid,forget plot]
  table[row sep=crcr]{
0	0	\\
1.1e-10	0	\\
2.2e-10	0	\\
3.3e-10	0	\\
4.4e-10	0	\\
5.4e-10	0	\\
6.5e-10	0	\\
7.5e-10	0	\\
8.6e-10	0	\\
9.6e-10	0	\\
1.07e-09	0	\\
1.18e-09	0	\\
1.28e-09	0	\\
1.38e-09	0	\\
1.49e-09	0	\\
1.59e-09	0	\\
1.69e-09	0	\\
1.8e-09	0	\\
1.9e-09	0	\\
2.01e-09	0	\\
2.11e-09	0	\\
2.21e-09	0	\\
2.32e-09	0	\\
2.42e-09	0	\\
2.52e-09	0	\\
2.63e-09	0	\\
2.73e-09	0	\\
2.83e-09	0	\\
2.93e-09	0	\\
3.04e-09	0	\\
3.14e-09	0	\\
3.24e-09	0	\\
3.34e-09	0	\\
3.45e-09	0	\\
3.55e-09	0	\\
3.65e-09	0	\\
3.75e-09	0	\\
3.86e-09	0	\\
3.96e-09	0	\\
4.06e-09	0	\\
4.16e-09	0	\\
4.27e-09	0	\\
4.37e-09	0	\\
4.47e-09	0	\\
4.57e-09	0	\\
4.68e-09	0	\\
4.78e-09	0	\\
4.89e-09	0	\\
4.99e-09	0	\\
5e-09	0	\\
};
\addplot [color=mycolor2,solid,forget plot]
  table[row sep=crcr]{
0	0	\\
1.1e-10	0	\\
2.2e-10	0	\\
3.3e-10	0	\\
4.4e-10	0	\\
5.4e-10	0	\\
6.5e-10	0	\\
7.5e-10	0	\\
8.6e-10	0	\\
9.6e-10	0	\\
1.07e-09	0	\\
1.18e-09	0	\\
1.28e-09	0	\\
1.38e-09	0	\\
1.49e-09	0	\\
1.59e-09	0	\\
1.69e-09	0	\\
1.8e-09	0	\\
1.9e-09	0	\\
2.01e-09	0	\\
2.11e-09	0	\\
2.21e-09	0	\\
2.32e-09	0	\\
2.42e-09	0	\\
2.52e-09	0	\\
2.63e-09	0	\\
2.73e-09	0	\\
2.83e-09	0	\\
2.93e-09	0	\\
3.04e-09	0	\\
3.14e-09	0	\\
3.24e-09	0	\\
3.34e-09	0	\\
3.45e-09	0	\\
3.55e-09	0	\\
3.65e-09	0	\\
3.75e-09	0	\\
3.86e-09	0	\\
3.96e-09	0	\\
4.06e-09	0	\\
4.16e-09	0	\\
4.27e-09	0	\\
4.37e-09	0	\\
4.47e-09	0	\\
4.57e-09	0	\\
4.68e-09	0	\\
4.78e-09	0	\\
4.89e-09	0	\\
4.99e-09	0	\\
5e-09	0	\\
};
\addplot [color=mycolor3,solid,forget plot]
  table[row sep=crcr]{
0	0	\\
1.1e-10	0	\\
2.2e-10	0	\\
3.3e-10	0	\\
4.4e-10	0	\\
5.4e-10	0	\\
6.5e-10	0	\\
7.5e-10	0	\\
8.6e-10	0	\\
9.6e-10	0	\\
1.07e-09	0	\\
1.18e-09	0	\\
1.28e-09	0	\\
1.38e-09	0	\\
1.49e-09	0	\\
1.59e-09	0	\\
1.69e-09	0	\\
1.8e-09	0	\\
1.9e-09	0	\\
2.01e-09	0	\\
2.11e-09	0	\\
2.21e-09	0	\\
2.32e-09	0	\\
2.42e-09	0	\\
2.52e-09	0	\\
2.63e-09	0	\\
2.73e-09	0	\\
2.83e-09	0	\\
2.93e-09	0	\\
3.04e-09	0	\\
3.14e-09	0	\\
3.24e-09	0	\\
3.34e-09	0	\\
3.45e-09	0	\\
3.55e-09	0	\\
3.65e-09	0	\\
3.75e-09	0	\\
3.86e-09	0	\\
3.96e-09	0	\\
4.06e-09	0	\\
4.16e-09	0	\\
4.27e-09	0	\\
4.37e-09	0	\\
4.47e-09	0	\\
4.57e-09	0	\\
4.68e-09	0	\\
4.78e-09	0	\\
4.89e-09	0	\\
4.99e-09	0	\\
5e-09	0	\\
};
\addplot [color=darkgray,solid,forget plot]
  table[row sep=crcr]{
0	0	\\
1.1e-10	0	\\
2.2e-10	0	\\
3.3e-10	0	\\
4.4e-10	0	\\
5.4e-10	0	\\
6.5e-10	0	\\
7.5e-10	0	\\
8.6e-10	0	\\
9.6e-10	0	\\
1.07e-09	0	\\
1.18e-09	0	\\
1.28e-09	0	\\
1.38e-09	0	\\
1.49e-09	0	\\
1.59e-09	0	\\
1.69e-09	0	\\
1.8e-09	0	\\
1.9e-09	0	\\
2.01e-09	0	\\
2.11e-09	0	\\
2.21e-09	0	\\
2.32e-09	0	\\
2.42e-09	0	\\
2.52e-09	0	\\
2.63e-09	0	\\
2.73e-09	0	\\
2.83e-09	0	\\
2.93e-09	0	\\
3.04e-09	0	\\
3.14e-09	0	\\
3.24e-09	0	\\
3.34e-09	0	\\
3.45e-09	0	\\
3.55e-09	0	\\
3.65e-09	0	\\
3.75e-09	0	\\
3.86e-09	0	\\
3.96e-09	0	\\
4.06e-09	0	\\
4.16e-09	0	\\
4.27e-09	0	\\
4.37e-09	0	\\
4.47e-09	0	\\
4.57e-09	0	\\
4.68e-09	0	\\
4.78e-09	0	\\
4.89e-09	0	\\
4.99e-09	0	\\
5e-09	0	\\
};
\addplot [color=blue,solid,forget plot]
  table[row sep=crcr]{
0	0	\\
1.1e-10	0	\\
2.2e-10	0	\\
3.3e-10	0	\\
4.4e-10	0	\\
5.4e-10	0	\\
6.5e-10	0	\\
7.5e-10	0	\\
8.6e-10	0	\\
9.6e-10	0	\\
1.07e-09	0	\\
1.18e-09	0	\\
1.28e-09	0	\\
1.38e-09	0	\\
1.49e-09	0	\\
1.59e-09	0	\\
1.69e-09	0	\\
1.8e-09	0	\\
1.9e-09	0	\\
2.01e-09	0	\\
2.11e-09	0	\\
2.21e-09	0	\\
2.32e-09	0	\\
2.42e-09	0	\\
2.52e-09	0	\\
2.63e-09	0	\\
2.73e-09	0	\\
2.83e-09	0	\\
2.93e-09	0	\\
3.04e-09	0	\\
3.14e-09	0	\\
3.24e-09	0	\\
3.34e-09	0	\\
3.45e-09	0	\\
3.55e-09	0	\\
3.65e-09	0	\\
3.75e-09	0	\\
3.86e-09	0	\\
3.96e-09	0	\\
4.06e-09	0	\\
4.16e-09	0	\\
4.27e-09	0	\\
4.37e-09	0	\\
4.47e-09	0	\\
4.57e-09	0	\\
4.68e-09	0	\\
4.78e-09	0	\\
4.89e-09	0	\\
4.99e-09	0	\\
5e-09	0	\\
};
\addplot [color=black!50!green,solid,forget plot]
  table[row sep=crcr]{
0	0	\\
1.1e-10	0	\\
2.2e-10	0	\\
3.3e-10	0	\\
4.4e-10	0	\\
5.4e-10	0	\\
6.5e-10	0	\\
7.5e-10	0	\\
8.6e-10	0	\\
9.6e-10	0	\\
1.07e-09	0	\\
1.18e-09	0	\\
1.28e-09	0	\\
1.38e-09	0	\\
1.49e-09	0	\\
1.59e-09	0	\\
1.69e-09	0	\\
1.8e-09	0	\\
1.9e-09	0	\\
2.01e-09	0	\\
2.11e-09	0	\\
2.21e-09	0	\\
2.32e-09	0	\\
2.42e-09	0	\\
2.52e-09	0	\\
2.63e-09	0	\\
2.73e-09	0	\\
2.83e-09	0	\\
2.93e-09	0	\\
3.04e-09	0	\\
3.14e-09	0	\\
3.24e-09	0	\\
3.34e-09	0	\\
3.45e-09	0	\\
3.55e-09	0	\\
3.65e-09	0	\\
3.75e-09	0	\\
3.86e-09	0	\\
3.96e-09	0	\\
4.06e-09	0	\\
4.16e-09	0	\\
4.27e-09	0	\\
4.37e-09	0	\\
4.47e-09	0	\\
4.57e-09	0	\\
4.68e-09	0	\\
4.78e-09	0	\\
4.89e-09	0	\\
4.99e-09	0	\\
5e-09	0	\\
};
\addplot [color=red,solid,forget plot]
  table[row sep=crcr]{
0	0	\\
1.1e-10	0	\\
2.2e-10	0	\\
3.3e-10	0	\\
4.4e-10	0	\\
5.4e-10	0	\\
6.5e-10	0	\\
7.5e-10	0	\\
8.6e-10	0	\\
9.6e-10	0	\\
1.07e-09	0	\\
1.18e-09	0	\\
1.28e-09	0	\\
1.38e-09	0	\\
1.49e-09	0	\\
1.59e-09	0	\\
1.69e-09	0	\\
1.8e-09	0	\\
1.9e-09	0	\\
2.01e-09	0	\\
2.11e-09	0	\\
2.21e-09	0	\\
2.32e-09	0	\\
2.42e-09	0	\\
2.52e-09	0	\\
2.63e-09	0	\\
2.73e-09	0	\\
2.83e-09	0	\\
2.93e-09	0	\\
3.04e-09	0	\\
3.14e-09	0	\\
3.24e-09	0	\\
3.34e-09	0	\\
3.45e-09	0	\\
3.55e-09	0	\\
3.65e-09	0	\\
3.75e-09	0	\\
3.86e-09	0	\\
3.96e-09	0	\\
4.06e-09	0	\\
4.16e-09	0	\\
4.27e-09	0	\\
4.37e-09	0	\\
4.47e-09	0	\\
4.57e-09	0	\\
4.68e-09	0	\\
4.78e-09	0	\\
4.89e-09	0	\\
4.99e-09	0	\\
5e-09	0	\\
};
\addplot [color=mycolor1,solid,forget plot]
  table[row sep=crcr]{
0	0	\\
1.1e-10	0	\\
2.2e-10	0	\\
3.3e-10	0	\\
4.4e-10	0	\\
5.4e-10	0	\\
6.5e-10	0	\\
7.5e-10	0	\\
8.6e-10	0	\\
9.6e-10	0	\\
1.07e-09	0	\\
1.18e-09	0	\\
1.28e-09	0	\\
1.38e-09	0	\\
1.49e-09	0	\\
1.59e-09	0	\\
1.69e-09	0	\\
1.8e-09	0	\\
1.9e-09	0	\\
2.01e-09	0	\\
2.11e-09	0	\\
2.21e-09	0	\\
2.32e-09	0	\\
2.42e-09	0	\\
2.52e-09	0	\\
2.63e-09	0	\\
2.73e-09	0	\\
2.83e-09	0	\\
2.93e-09	0	\\
3.04e-09	0	\\
3.14e-09	0	\\
3.24e-09	0	\\
3.34e-09	0	\\
3.45e-09	0	\\
3.55e-09	0	\\
3.65e-09	0	\\
3.75e-09	0	\\
3.86e-09	0	\\
3.96e-09	0	\\
4.06e-09	0	\\
4.16e-09	0	\\
4.27e-09	0	\\
4.37e-09	0	\\
4.47e-09	0	\\
4.57e-09	0	\\
4.68e-09	0	\\
4.78e-09	0	\\
4.89e-09	0	\\
4.99e-09	0	\\
5e-09	0	\\
};
\addplot [color=mycolor2,solid,forget plot]
  table[row sep=crcr]{
0	0	\\
1.1e-10	0	\\
2.2e-10	0	\\
3.3e-10	0	\\
4.4e-10	0	\\
5.4e-10	0	\\
6.5e-10	0	\\
7.5e-10	0	\\
8.6e-10	0	\\
9.6e-10	0	\\
1.07e-09	0	\\
1.18e-09	0	\\
1.28e-09	0	\\
1.38e-09	0	\\
1.49e-09	0	\\
1.59e-09	0	\\
1.69e-09	0	\\
1.8e-09	0	\\
1.9e-09	0	\\
2.01e-09	0	\\
2.11e-09	0	\\
2.21e-09	0	\\
2.32e-09	0	\\
2.42e-09	0	\\
2.52e-09	0	\\
2.63e-09	0	\\
2.73e-09	0	\\
2.83e-09	0	\\
2.93e-09	0	\\
3.04e-09	0	\\
3.14e-09	0	\\
3.24e-09	0	\\
3.34e-09	0	\\
3.45e-09	0	\\
3.55e-09	0	\\
3.65e-09	0	\\
3.75e-09	0	\\
3.86e-09	0	\\
3.96e-09	0	\\
4.06e-09	0	\\
4.16e-09	0	\\
4.27e-09	0	\\
4.37e-09	0	\\
4.47e-09	0	\\
4.57e-09	0	\\
4.68e-09	0	\\
4.78e-09	0	\\
4.89e-09	0	\\
4.99e-09	0	\\
5e-09	0	\\
};
\addplot [color=mycolor3,solid,forget plot]
  table[row sep=crcr]{
0	0	\\
1.1e-10	0	\\
2.2e-10	0	\\
3.3e-10	0	\\
4.4e-10	0	\\
5.4e-10	0	\\
6.5e-10	0	\\
7.5e-10	0	\\
8.6e-10	0	\\
9.6e-10	0	\\
1.07e-09	0	\\
1.18e-09	0	\\
1.28e-09	0	\\
1.38e-09	0	\\
1.49e-09	0	\\
1.59e-09	0	\\
1.69e-09	0	\\
1.8e-09	0	\\
1.9e-09	0	\\
2.01e-09	0	\\
2.11e-09	0	\\
2.21e-09	0	\\
2.32e-09	0	\\
2.42e-09	0	\\
2.52e-09	0	\\
2.63e-09	0	\\
2.73e-09	0	\\
2.83e-09	0	\\
2.93e-09	0	\\
3.04e-09	0	\\
3.14e-09	0	\\
3.24e-09	0	\\
3.34e-09	0	\\
3.45e-09	0	\\
3.55e-09	0	\\
3.65e-09	0	\\
3.75e-09	0	\\
3.86e-09	0	\\
3.96e-09	0	\\
4.06e-09	0	\\
4.16e-09	0	\\
4.27e-09	0	\\
4.37e-09	0	\\
4.47e-09	0	\\
4.57e-09	0	\\
4.68e-09	0	\\
4.78e-09	0	\\
4.89e-09	0	\\
4.99e-09	0	\\
5e-09	0	\\
};
\addplot [color=darkgray,solid,forget plot]
  table[row sep=crcr]{
0	0	\\
1.1e-10	0	\\
2.2e-10	0	\\
3.3e-10	0	\\
4.4e-10	0	\\
5.4e-10	0	\\
6.5e-10	0	\\
7.5e-10	0	\\
8.6e-10	0	\\
9.6e-10	0	\\
1.07e-09	0	\\
1.18e-09	0	\\
1.28e-09	0	\\
1.38e-09	0	\\
1.49e-09	0	\\
1.59e-09	0	\\
1.69e-09	0	\\
1.8e-09	0	\\
1.9e-09	0	\\
2.01e-09	0	\\
2.11e-09	0	\\
2.21e-09	0	\\
2.32e-09	0	\\
2.42e-09	0	\\
2.52e-09	0	\\
2.63e-09	0	\\
2.73e-09	0	\\
2.83e-09	0	\\
2.93e-09	0	\\
3.04e-09	0	\\
3.14e-09	0	\\
3.24e-09	0	\\
3.34e-09	0	\\
3.45e-09	0	\\
3.55e-09	0	\\
3.65e-09	0	\\
3.75e-09	0	\\
3.86e-09	0	\\
3.96e-09	0	\\
4.06e-09	0	\\
4.16e-09	0	\\
4.27e-09	0	\\
4.37e-09	0	\\
4.47e-09	0	\\
4.57e-09	0	\\
4.68e-09	0	\\
4.78e-09	0	\\
4.89e-09	0	\\
4.99e-09	0	\\
5e-09	0	\\
};
\addplot [color=blue,solid,forget plot]
  table[row sep=crcr]{
0	0	\\
1.1e-10	0	\\
2.2e-10	0	\\
3.3e-10	0	\\
4.4e-10	0	\\
5.4e-10	0	\\
6.5e-10	0	\\
7.5e-10	0	\\
8.6e-10	0	\\
9.6e-10	0	\\
1.07e-09	0	\\
1.18e-09	0	\\
1.28e-09	0	\\
1.38e-09	0	\\
1.49e-09	0	\\
1.59e-09	0	\\
1.69e-09	0	\\
1.8e-09	0	\\
1.9e-09	0	\\
2.01e-09	0	\\
2.11e-09	0	\\
2.21e-09	0	\\
2.32e-09	0	\\
2.42e-09	0	\\
2.52e-09	0	\\
2.63e-09	0	\\
2.73e-09	0	\\
2.83e-09	0	\\
2.93e-09	0	\\
3.04e-09	0	\\
3.14e-09	0	\\
3.24e-09	0	\\
3.34e-09	0	\\
3.45e-09	0	\\
3.55e-09	0	\\
3.65e-09	0	\\
3.75e-09	0	\\
3.86e-09	0	\\
3.96e-09	0	\\
4.06e-09	0	\\
4.16e-09	0	\\
4.27e-09	0	\\
4.37e-09	0	\\
4.47e-09	0	\\
4.57e-09	0	\\
4.68e-09	0	\\
4.78e-09	0	\\
4.89e-09	0	\\
4.99e-09	0	\\
5e-09	0	\\
};
\addplot [color=black!50!green,solid,forget plot]
  table[row sep=crcr]{
0	0	\\
1.1e-10	0	\\
2.2e-10	0	\\
3.3e-10	0	\\
4.4e-10	0	\\
5.4e-10	0	\\
6.5e-10	0	\\
7.5e-10	0	\\
8.6e-10	0	\\
9.6e-10	0	\\
1.07e-09	0	\\
1.18e-09	0	\\
1.28e-09	0	\\
1.38e-09	0	\\
1.49e-09	0	\\
1.59e-09	0	\\
1.69e-09	0	\\
1.8e-09	0	\\
1.9e-09	0	\\
2.01e-09	0	\\
2.11e-09	0	\\
2.21e-09	0	\\
2.32e-09	0	\\
2.42e-09	0	\\
2.52e-09	0	\\
2.63e-09	0	\\
2.73e-09	0	\\
2.83e-09	0	\\
2.93e-09	0	\\
3.04e-09	0	\\
3.14e-09	0	\\
3.24e-09	0	\\
3.34e-09	0	\\
3.45e-09	0	\\
3.55e-09	0	\\
3.65e-09	0	\\
3.75e-09	0	\\
3.86e-09	0	\\
3.96e-09	0	\\
4.06e-09	0	\\
4.16e-09	0	\\
4.27e-09	0	\\
4.37e-09	0	\\
4.47e-09	0	\\
4.57e-09	0	\\
4.68e-09	0	\\
4.78e-09	0	\\
4.89e-09	0	\\
4.99e-09	0	\\
5e-09	0	\\
};
\addplot [color=red,solid,forget plot]
  table[row sep=crcr]{
0	0	\\
1.1e-10	0	\\
2.2e-10	0	\\
3.3e-10	0	\\
4.4e-10	0	\\
5.4e-10	0	\\
6.5e-10	0	\\
7.5e-10	0	\\
8.6e-10	0	\\
9.6e-10	0	\\
1.07e-09	0	\\
1.18e-09	0	\\
1.28e-09	0	\\
1.38e-09	0	\\
1.49e-09	0	\\
1.59e-09	0	\\
1.69e-09	0	\\
1.8e-09	0	\\
1.9e-09	0	\\
2.01e-09	0	\\
2.11e-09	0	\\
2.21e-09	0	\\
2.32e-09	0	\\
2.42e-09	0	\\
2.52e-09	0	\\
2.63e-09	0	\\
2.73e-09	0	\\
2.83e-09	0	\\
2.93e-09	0	\\
3.04e-09	0	\\
3.14e-09	0	\\
3.24e-09	0	\\
3.34e-09	0	\\
3.45e-09	0	\\
3.55e-09	0	\\
3.65e-09	0	\\
3.75e-09	0	\\
3.86e-09	0	\\
3.96e-09	0	\\
4.06e-09	0	\\
4.16e-09	0	\\
4.27e-09	0	\\
4.37e-09	0	\\
4.47e-09	0	\\
4.57e-09	0	\\
4.68e-09	0	\\
4.78e-09	0	\\
4.89e-09	0	\\
4.99e-09	0	\\
5e-09	0	\\
};
\addplot [color=mycolor1,solid,forget plot]
  table[row sep=crcr]{
0	0	\\
1.1e-10	0	\\
2.2e-10	0	\\
3.3e-10	0	\\
4.4e-10	0	\\
5.4e-10	0	\\
6.5e-10	0	\\
7.5e-10	0	\\
8.6e-10	0	\\
9.6e-10	0	\\
1.07e-09	0	\\
1.18e-09	0	\\
1.28e-09	0	\\
1.38e-09	0	\\
1.49e-09	0	\\
1.59e-09	0	\\
1.69e-09	0	\\
1.8e-09	0	\\
1.9e-09	0	\\
2.01e-09	0	\\
2.11e-09	0	\\
2.21e-09	0	\\
2.32e-09	0	\\
2.42e-09	0	\\
2.52e-09	0	\\
2.63e-09	0	\\
2.73e-09	0	\\
2.83e-09	0	\\
2.93e-09	0	\\
3.04e-09	0	\\
3.14e-09	0	\\
3.24e-09	0	\\
3.34e-09	0	\\
3.45e-09	0	\\
3.55e-09	0	\\
3.65e-09	0	\\
3.75e-09	0	\\
3.86e-09	0	\\
3.96e-09	0	\\
4.06e-09	0	\\
4.16e-09	0	\\
4.27e-09	0	\\
4.37e-09	0	\\
4.47e-09	0	\\
4.57e-09	0	\\
4.68e-09	0	\\
4.78e-09	0	\\
4.89e-09	0	\\
4.99e-09	0	\\
5e-09	0	\\
};
\addplot [color=mycolor2,solid,forget plot]
  table[row sep=crcr]{
0	0	\\
1.1e-10	0	\\
2.2e-10	0	\\
3.3e-10	0	\\
4.4e-10	0	\\
5.4e-10	0	\\
6.5e-10	0	\\
7.5e-10	0	\\
8.6e-10	0	\\
9.6e-10	0	\\
1.07e-09	0	\\
1.18e-09	0	\\
1.28e-09	0	\\
1.38e-09	0	\\
1.49e-09	0	\\
1.59e-09	0	\\
1.69e-09	0	\\
1.8e-09	0	\\
1.9e-09	0	\\
2.01e-09	0	\\
2.11e-09	0	\\
2.21e-09	0	\\
2.32e-09	0	\\
2.42e-09	0	\\
2.52e-09	0	\\
2.63e-09	0	\\
2.73e-09	0	\\
2.83e-09	0	\\
2.93e-09	0	\\
3.04e-09	0	\\
3.14e-09	0	\\
3.24e-09	0	\\
3.34e-09	0	\\
3.45e-09	0	\\
3.55e-09	0	\\
3.65e-09	0	\\
3.75e-09	0	\\
3.86e-09	0	\\
3.96e-09	0	\\
4.06e-09	0	\\
4.16e-09	0	\\
4.27e-09	0	\\
4.37e-09	0	\\
4.47e-09	0	\\
4.57e-09	0	\\
4.68e-09	0	\\
4.78e-09	0	\\
4.89e-09	0	\\
4.99e-09	0	\\
5e-09	0	\\
};
\addplot [color=mycolor3,solid,forget plot]
  table[row sep=crcr]{
0	0	\\
1.1e-10	0	\\
2.2e-10	0	\\
3.3e-10	0	\\
4.4e-10	0	\\
5.4e-10	0	\\
6.5e-10	0	\\
7.5e-10	0	\\
8.6e-10	0	\\
9.6e-10	0	\\
1.07e-09	0	\\
1.18e-09	0	\\
1.28e-09	0	\\
1.38e-09	0	\\
1.49e-09	0	\\
1.59e-09	0	\\
1.69e-09	0	\\
1.8e-09	0	\\
1.9e-09	0	\\
2.01e-09	0	\\
2.11e-09	0	\\
2.21e-09	0	\\
2.32e-09	0	\\
2.42e-09	0	\\
2.52e-09	0	\\
2.63e-09	0	\\
2.73e-09	0	\\
2.83e-09	0	\\
2.93e-09	0	\\
3.04e-09	0	\\
3.14e-09	0	\\
3.24e-09	0	\\
3.34e-09	0	\\
3.45e-09	0	\\
3.55e-09	0	\\
3.65e-09	0	\\
3.75e-09	0	\\
3.86e-09	0	\\
3.96e-09	0	\\
4.06e-09	0	\\
4.16e-09	0	\\
4.27e-09	0	\\
4.37e-09	0	\\
4.47e-09	0	\\
4.57e-09	0	\\
4.68e-09	0	\\
4.78e-09	0	\\
4.89e-09	0	\\
4.99e-09	0	\\
5e-09	0	\\
};
\addplot [color=darkgray,solid,forget plot]
  table[row sep=crcr]{
0	0	\\
1.1e-10	0	\\
2.2e-10	0	\\
3.3e-10	0	\\
4.4e-10	0	\\
5.4e-10	0	\\
6.5e-10	0	\\
7.5e-10	0	\\
8.6e-10	0	\\
9.6e-10	0	\\
1.07e-09	0	\\
1.18e-09	0	\\
1.28e-09	0	\\
1.38e-09	0	\\
1.49e-09	0	\\
1.59e-09	0	\\
1.69e-09	0	\\
1.8e-09	0	\\
1.9e-09	0	\\
2.01e-09	0	\\
2.11e-09	0	\\
2.21e-09	0	\\
2.32e-09	0	\\
2.42e-09	0	\\
2.52e-09	0	\\
2.63e-09	0	\\
2.73e-09	0	\\
2.83e-09	0	\\
2.93e-09	0	\\
3.04e-09	0	\\
3.14e-09	0	\\
3.24e-09	0	\\
3.34e-09	0	\\
3.45e-09	0	\\
3.55e-09	0	\\
3.65e-09	0	\\
3.75e-09	0	\\
3.86e-09	0	\\
3.96e-09	0	\\
4.06e-09	0	\\
4.16e-09	0	\\
4.27e-09	0	\\
4.37e-09	0	\\
4.47e-09	0	\\
4.57e-09	0	\\
4.68e-09	0	\\
4.78e-09	0	\\
4.89e-09	0	\\
4.99e-09	0	\\
5e-09	0	\\
};
\addplot [color=blue,solid,forget plot]
  table[row sep=crcr]{
0	0	\\
1.1e-10	0	\\
2.2e-10	0	\\
3.3e-10	0	\\
4.4e-10	0	\\
5.4e-10	0	\\
6.5e-10	0	\\
7.5e-10	0	\\
8.6e-10	0	\\
9.6e-10	0	\\
1.07e-09	0	\\
1.18e-09	0	\\
1.28e-09	0	\\
1.38e-09	0	\\
1.49e-09	0	\\
1.59e-09	0	\\
1.69e-09	0	\\
1.8e-09	0	\\
1.9e-09	0	\\
2.01e-09	0	\\
2.11e-09	0	\\
2.21e-09	0	\\
2.32e-09	0	\\
2.42e-09	0	\\
2.52e-09	0	\\
2.63e-09	0	\\
2.73e-09	0	\\
2.83e-09	0	\\
2.93e-09	0	\\
3.04e-09	0	\\
3.14e-09	0	\\
3.24e-09	0	\\
3.34e-09	0	\\
3.45e-09	0	\\
3.55e-09	0	\\
3.65e-09	0	\\
3.75e-09	0	\\
3.86e-09	0	\\
3.96e-09	0	\\
4.06e-09	0	\\
4.16e-09	0	\\
4.27e-09	0	\\
4.37e-09	0	\\
4.47e-09	0	\\
4.57e-09	0	\\
4.68e-09	0	\\
4.78e-09	0	\\
4.89e-09	0	\\
4.99e-09	0	\\
5e-09	0	\\
};
\addplot [color=black!50!green,solid,forget plot]
  table[row sep=crcr]{
0	0	\\
1.1e-10	0	\\
2.2e-10	0	\\
3.3e-10	0	\\
4.4e-10	0	\\
5.4e-10	0	\\
6.5e-10	0	\\
7.5e-10	0	\\
8.6e-10	0	\\
9.6e-10	0	\\
1.07e-09	0	\\
1.18e-09	0	\\
1.28e-09	0	\\
1.38e-09	0	\\
1.49e-09	0	\\
1.59e-09	0	\\
1.69e-09	0	\\
1.8e-09	0	\\
1.9e-09	0	\\
2.01e-09	0	\\
2.11e-09	0	\\
2.21e-09	0	\\
2.32e-09	0	\\
2.42e-09	0	\\
2.52e-09	0	\\
2.63e-09	0	\\
2.73e-09	0	\\
2.83e-09	0	\\
2.93e-09	0	\\
3.04e-09	0	\\
3.14e-09	0	\\
3.24e-09	0	\\
3.34e-09	0	\\
3.45e-09	0	\\
3.55e-09	0	\\
3.65e-09	0	\\
3.75e-09	0	\\
3.86e-09	0	\\
3.96e-09	0	\\
4.06e-09	0	\\
4.16e-09	0	\\
4.27e-09	0	\\
4.37e-09	0	\\
4.47e-09	0	\\
4.57e-09	0	\\
4.68e-09	0	\\
4.78e-09	0	\\
4.89e-09	0	\\
4.99e-09	0	\\
5e-09	0	\\
};
\addplot [color=red,solid,forget plot]
  table[row sep=crcr]{
0	0	\\
1.1e-10	0	\\
2.2e-10	0	\\
3.3e-10	0	\\
4.4e-10	0	\\
5.4e-10	0	\\
6.5e-10	0	\\
7.5e-10	0	\\
8.6e-10	0	\\
9.6e-10	0	\\
1.07e-09	0	\\
1.18e-09	0	\\
1.28e-09	0	\\
1.38e-09	0	\\
1.49e-09	0	\\
1.59e-09	0	\\
1.69e-09	0	\\
1.8e-09	0	\\
1.9e-09	0	\\
2.01e-09	0	\\
2.11e-09	0	\\
2.21e-09	0	\\
2.32e-09	0	\\
2.42e-09	0	\\
2.52e-09	0	\\
2.63e-09	0	\\
2.73e-09	0	\\
2.83e-09	0	\\
2.93e-09	0	\\
3.04e-09	0	\\
3.14e-09	0	\\
3.24e-09	0	\\
3.34e-09	0	\\
3.45e-09	0	\\
3.55e-09	0	\\
3.65e-09	0	\\
3.75e-09	0	\\
3.86e-09	0	\\
3.96e-09	0	\\
4.06e-09	0	\\
4.16e-09	0	\\
4.27e-09	0	\\
4.37e-09	0	\\
4.47e-09	0	\\
4.57e-09	0	\\
4.68e-09	0	\\
4.78e-09	0	\\
4.89e-09	0	\\
4.99e-09	0	\\
5e-09	0	\\
};
\addplot [color=mycolor1,solid,forget plot]
  table[row sep=crcr]{
0	0	\\
1.1e-10	0	\\
2.2e-10	0	\\
3.3e-10	0	\\
4.4e-10	0	\\
5.4e-10	0	\\
6.5e-10	0	\\
7.5e-10	0	\\
8.6e-10	0	\\
9.6e-10	0	\\
1.07e-09	0	\\
1.18e-09	0	\\
1.28e-09	0	\\
1.38e-09	0	\\
1.49e-09	0	\\
1.59e-09	0	\\
1.69e-09	0	\\
1.8e-09	0	\\
1.9e-09	0	\\
2.01e-09	0	\\
2.11e-09	0	\\
2.21e-09	0	\\
2.32e-09	0	\\
2.42e-09	0	\\
2.52e-09	0	\\
2.63e-09	0	\\
2.73e-09	0	\\
2.83e-09	0	\\
2.93e-09	0	\\
3.04e-09	0	\\
3.14e-09	0	\\
3.24e-09	0	\\
3.34e-09	0	\\
3.45e-09	0	\\
3.55e-09	0	\\
3.65e-09	0	\\
3.75e-09	0	\\
3.86e-09	0	\\
3.96e-09	0	\\
4.06e-09	0	\\
4.16e-09	0	\\
4.27e-09	0	\\
4.37e-09	0	\\
4.47e-09	0	\\
4.57e-09	0	\\
4.68e-09	0	\\
4.78e-09	0	\\
4.89e-09	0	\\
4.99e-09	0	\\
5e-09	0	\\
};
\addplot [color=mycolor2,solid,forget plot]
  table[row sep=crcr]{
0	0	\\
1.1e-10	0	\\
2.2e-10	0	\\
3.3e-10	0	\\
4.4e-10	0	\\
5.4e-10	0	\\
6.5e-10	0	\\
7.5e-10	0	\\
8.6e-10	0	\\
9.6e-10	0	\\
1.07e-09	0	\\
1.18e-09	0	\\
1.28e-09	0	\\
1.38e-09	0	\\
1.49e-09	0	\\
1.59e-09	0	\\
1.69e-09	0	\\
1.8e-09	0	\\
1.9e-09	0	\\
2.01e-09	0	\\
2.11e-09	0	\\
2.21e-09	0	\\
2.32e-09	0	\\
2.42e-09	0	\\
2.52e-09	0	\\
2.63e-09	0	\\
2.73e-09	0	\\
2.83e-09	0	\\
2.93e-09	0	\\
3.04e-09	0	\\
3.14e-09	0	\\
3.24e-09	0	\\
3.34e-09	0	\\
3.45e-09	0	\\
3.55e-09	0	\\
3.65e-09	0	\\
3.75e-09	0	\\
3.86e-09	0	\\
3.96e-09	0	\\
4.06e-09	0	\\
4.16e-09	0	\\
4.27e-09	0	\\
4.37e-09	0	\\
4.47e-09	0	\\
4.57e-09	0	\\
4.68e-09	0	\\
4.78e-09	0	\\
4.89e-09	0	\\
4.99e-09	0	\\
5e-09	0	\\
};
\addplot [color=mycolor3,solid,forget plot]
  table[row sep=crcr]{
0	0	\\
1.1e-10	0	\\
2.2e-10	0	\\
3.3e-10	0	\\
4.4e-10	0	\\
5.4e-10	0	\\
6.5e-10	0	\\
7.5e-10	0	\\
8.6e-10	0	\\
9.6e-10	0	\\
1.07e-09	0	\\
1.18e-09	0	\\
1.28e-09	0	\\
1.38e-09	0	\\
1.49e-09	0	\\
1.59e-09	0	\\
1.69e-09	0	\\
1.8e-09	0	\\
1.9e-09	0	\\
2.01e-09	0	\\
2.11e-09	0	\\
2.21e-09	0	\\
2.32e-09	0	\\
2.42e-09	0	\\
2.52e-09	0	\\
2.63e-09	0	\\
2.73e-09	0	\\
2.83e-09	0	\\
2.93e-09	0	\\
3.04e-09	0	\\
3.14e-09	0	\\
3.24e-09	0	\\
3.34e-09	0	\\
3.45e-09	0	\\
3.55e-09	0	\\
3.65e-09	0	\\
3.75e-09	0	\\
3.86e-09	0	\\
3.96e-09	0	\\
4.06e-09	0	\\
4.16e-09	0	\\
4.27e-09	0	\\
4.37e-09	0	\\
4.47e-09	0	\\
4.57e-09	0	\\
4.68e-09	0	\\
4.78e-09	0	\\
4.89e-09	0	\\
4.99e-09	0	\\
5e-09	0	\\
};
\addplot [color=darkgray,solid,forget plot]
  table[row sep=crcr]{
0	0	\\
1.1e-10	0	\\
2.2e-10	0	\\
3.3e-10	0	\\
4.4e-10	0	\\
5.4e-10	0	\\
6.5e-10	0	\\
7.5e-10	0	\\
8.6e-10	0	\\
9.6e-10	0	\\
1.07e-09	0	\\
1.18e-09	0	\\
1.28e-09	0	\\
1.38e-09	0	\\
1.49e-09	0	\\
1.59e-09	0	\\
1.69e-09	0	\\
1.8e-09	0	\\
1.9e-09	0	\\
2.01e-09	0	\\
2.11e-09	0	\\
2.21e-09	0	\\
2.32e-09	0	\\
2.42e-09	0	\\
2.52e-09	0	\\
2.63e-09	0	\\
2.73e-09	0	\\
2.83e-09	0	\\
2.93e-09	0	\\
3.04e-09	0	\\
3.14e-09	0	\\
3.24e-09	0	\\
3.34e-09	0	\\
3.45e-09	0	\\
3.55e-09	0	\\
3.65e-09	0	\\
3.75e-09	0	\\
3.86e-09	0	\\
3.96e-09	0	\\
4.06e-09	0	\\
4.16e-09	0	\\
4.27e-09	0	\\
4.37e-09	0	\\
4.47e-09	0	\\
4.57e-09	0	\\
4.68e-09	0	\\
4.78e-09	0	\\
4.89e-09	0	\\
4.99e-09	0	\\
5e-09	0	\\
};
\addplot [color=blue,solid,forget plot]
  table[row sep=crcr]{
0	0	\\
1.1e-10	0	\\
2.2e-10	0	\\
3.3e-10	0	\\
4.4e-10	0	\\
5.4e-10	0	\\
6.5e-10	0	\\
7.5e-10	0	\\
8.6e-10	0	\\
9.6e-10	0	\\
1.07e-09	0	\\
1.18e-09	0	\\
1.28e-09	0	\\
1.38e-09	0	\\
1.49e-09	0	\\
1.59e-09	0	\\
1.69e-09	0	\\
1.8e-09	0	\\
1.9e-09	0	\\
2.01e-09	0	\\
2.11e-09	0	\\
2.21e-09	0	\\
2.32e-09	0	\\
2.42e-09	0	\\
2.52e-09	0	\\
2.63e-09	0	\\
2.73e-09	0	\\
2.83e-09	0	\\
2.93e-09	0	\\
3.04e-09	0	\\
3.14e-09	0	\\
3.24e-09	0	\\
3.34e-09	0	\\
3.45e-09	0	\\
3.55e-09	0	\\
3.65e-09	0	\\
3.75e-09	0	\\
3.86e-09	0	\\
3.96e-09	0	\\
4.06e-09	0	\\
4.16e-09	0	\\
4.27e-09	0	\\
4.37e-09	0	\\
4.47e-09	0	\\
4.57e-09	0	\\
4.68e-09	0	\\
4.78e-09	0	\\
4.89e-09	0	\\
4.99e-09	0	\\
5e-09	0	\\
};
\addplot [color=black!50!green,solid,forget plot]
  table[row sep=crcr]{
0	0	\\
1.1e-10	0	\\
2.2e-10	0	\\
3.3e-10	0	\\
4.4e-10	0	\\
5.4e-10	0	\\
6.5e-10	0	\\
7.5e-10	0	\\
8.6e-10	0	\\
9.6e-10	0	\\
1.07e-09	0	\\
1.18e-09	0	\\
1.28e-09	0	\\
1.38e-09	0	\\
1.49e-09	0	\\
1.59e-09	0	\\
1.69e-09	0	\\
1.8e-09	0	\\
1.9e-09	0	\\
2.01e-09	0	\\
2.11e-09	0	\\
2.21e-09	0	\\
2.32e-09	0	\\
2.42e-09	0	\\
2.52e-09	0	\\
2.63e-09	0	\\
2.73e-09	0	\\
2.83e-09	0	\\
2.93e-09	0	\\
3.04e-09	0	\\
3.14e-09	0	\\
3.24e-09	0	\\
3.34e-09	0	\\
3.45e-09	0	\\
3.55e-09	0	\\
3.65e-09	0	\\
3.75e-09	0	\\
3.86e-09	0	\\
3.96e-09	0	\\
4.06e-09	0	\\
4.16e-09	0	\\
4.27e-09	0	\\
4.37e-09	0	\\
4.47e-09	0	\\
4.57e-09	0	\\
4.68e-09	0	\\
4.78e-09	0	\\
4.89e-09	0	\\
4.99e-09	0	\\
5e-09	0	\\
};
\addplot [color=red,solid,forget plot]
  table[row sep=crcr]{
0	0	\\
1.1e-10	0	\\
2.2e-10	0	\\
3.3e-10	0	\\
4.4e-10	0	\\
5.4e-10	0	\\
6.5e-10	0	\\
7.5e-10	0	\\
8.6e-10	0	\\
9.6e-10	0	\\
1.07e-09	0	\\
1.18e-09	0	\\
1.28e-09	0	\\
1.38e-09	0	\\
1.49e-09	0	\\
1.59e-09	0	\\
1.69e-09	0	\\
1.8e-09	0	\\
1.9e-09	0	\\
2.01e-09	0	\\
2.11e-09	0	\\
2.21e-09	0	\\
2.32e-09	0	\\
2.42e-09	0	\\
2.52e-09	0	\\
2.63e-09	0	\\
2.73e-09	0	\\
2.83e-09	0	\\
2.93e-09	0	\\
3.04e-09	0	\\
3.14e-09	0	\\
3.24e-09	0	\\
3.34e-09	0	\\
3.45e-09	0	\\
3.55e-09	0	\\
3.65e-09	0	\\
3.75e-09	0	\\
3.86e-09	0	\\
3.96e-09	0	\\
4.06e-09	0	\\
4.16e-09	0	\\
4.27e-09	0	\\
4.37e-09	0	\\
4.47e-09	0	\\
4.57e-09	0	\\
4.68e-09	0	\\
4.78e-09	0	\\
4.89e-09	0	\\
4.99e-09	0	\\
5e-09	0	\\
};
\addplot [color=mycolor1,solid,forget plot]
  table[row sep=crcr]{
0	0	\\
1.1e-10	0	\\
2.2e-10	0	\\
3.3e-10	0	\\
4.4e-10	0	\\
5.4e-10	0	\\
6.5e-10	0	\\
7.5e-10	0	\\
8.6e-10	0	\\
9.6e-10	0	\\
1.07e-09	0	\\
1.18e-09	0	\\
1.28e-09	0	\\
1.38e-09	0	\\
1.49e-09	0	\\
1.59e-09	0	\\
1.69e-09	0	\\
1.8e-09	0	\\
1.9e-09	0	\\
2.01e-09	0	\\
2.11e-09	0	\\
2.21e-09	0	\\
2.32e-09	0	\\
2.42e-09	0	\\
2.52e-09	0	\\
2.63e-09	0	\\
2.73e-09	0	\\
2.83e-09	0	\\
2.93e-09	0	\\
3.04e-09	0	\\
3.14e-09	0	\\
3.24e-09	0	\\
3.34e-09	0	\\
3.45e-09	0	\\
3.55e-09	0	\\
3.65e-09	0	\\
3.75e-09	0	\\
3.86e-09	0	\\
3.96e-09	0	\\
4.06e-09	0	\\
4.16e-09	0	\\
4.27e-09	0	\\
4.37e-09	0	\\
4.47e-09	0	\\
4.57e-09	0	\\
4.68e-09	0	\\
4.78e-09	0	\\
4.89e-09	0	\\
4.99e-09	0	\\
5e-09	0	\\
};
\addplot [color=mycolor2,solid,forget plot]
  table[row sep=crcr]{
0	0	\\
1.1e-10	0	\\
2.2e-10	0	\\
3.3e-10	0	\\
4.4e-10	0	\\
5.4e-10	0	\\
6.5e-10	0	\\
7.5e-10	0	\\
8.6e-10	0	\\
9.6e-10	0	\\
1.07e-09	0	\\
1.18e-09	0	\\
1.28e-09	0	\\
1.38e-09	0	\\
1.49e-09	0	\\
1.59e-09	0	\\
1.69e-09	0	\\
1.8e-09	0	\\
1.9e-09	0	\\
2.01e-09	0	\\
2.11e-09	0	\\
2.21e-09	0	\\
2.32e-09	0	\\
2.42e-09	0	\\
2.52e-09	0	\\
2.63e-09	0	\\
2.73e-09	0	\\
2.83e-09	0	\\
2.93e-09	0	\\
3.04e-09	0	\\
3.14e-09	0	\\
3.24e-09	0	\\
3.34e-09	0	\\
3.45e-09	0	\\
3.55e-09	0	\\
3.65e-09	0	\\
3.75e-09	0	\\
3.86e-09	0	\\
3.96e-09	0	\\
4.06e-09	0	\\
4.16e-09	0	\\
4.27e-09	0	\\
4.37e-09	0	\\
4.47e-09	0	\\
4.57e-09	0	\\
4.68e-09	0	\\
4.78e-09	0	\\
4.89e-09	0	\\
4.99e-09	0	\\
5e-09	0	\\
};
\addplot [color=mycolor3,solid,forget plot]
  table[row sep=crcr]{
0	0	\\
1.1e-10	0	\\
2.2e-10	0	\\
3.3e-10	0	\\
4.4e-10	0	\\
5.4e-10	0	\\
6.5e-10	0	\\
7.5e-10	0	\\
8.6e-10	0	\\
9.6e-10	0	\\
1.07e-09	0	\\
1.18e-09	0	\\
1.28e-09	0	\\
1.38e-09	0	\\
1.49e-09	0	\\
1.59e-09	0	\\
1.69e-09	0	\\
1.8e-09	0	\\
1.9e-09	0	\\
2.01e-09	0	\\
2.11e-09	0	\\
2.21e-09	0	\\
2.32e-09	0	\\
2.42e-09	0	\\
2.52e-09	0	\\
2.63e-09	0	\\
2.73e-09	0	\\
2.83e-09	0	\\
2.93e-09	0	\\
3.04e-09	0	\\
3.14e-09	0	\\
3.24e-09	0	\\
3.34e-09	0	\\
3.45e-09	0	\\
3.55e-09	0	\\
3.65e-09	0	\\
3.75e-09	0	\\
3.86e-09	0	\\
3.96e-09	0	\\
4.06e-09	0	\\
4.16e-09	0	\\
4.27e-09	0	\\
4.37e-09	0	\\
4.47e-09	0	\\
4.57e-09	0	\\
4.68e-09	0	\\
4.78e-09	0	\\
4.89e-09	0	\\
4.99e-09	0	\\
5e-09	0	\\
};
\addplot [color=darkgray,solid,forget plot]
  table[row sep=crcr]{
0	0	\\
1.1e-10	0	\\
2.2e-10	0	\\
3.3e-10	0	\\
4.4e-10	0	\\
5.4e-10	0	\\
6.5e-10	0	\\
7.5e-10	0	\\
8.6e-10	0	\\
9.6e-10	0	\\
1.07e-09	0	\\
1.18e-09	0	\\
1.28e-09	0	\\
1.38e-09	0	\\
1.49e-09	0	\\
1.59e-09	0	\\
1.69e-09	0	\\
1.8e-09	0	\\
1.9e-09	0	\\
2.01e-09	0	\\
2.11e-09	0	\\
2.21e-09	0	\\
2.32e-09	0	\\
2.42e-09	0	\\
2.52e-09	0	\\
2.63e-09	0	\\
2.73e-09	0	\\
2.83e-09	0	\\
2.93e-09	0	\\
3.04e-09	0	\\
3.14e-09	0	\\
3.24e-09	0	\\
3.34e-09	0	\\
3.45e-09	0	\\
3.55e-09	0	\\
3.65e-09	0	\\
3.75e-09	0	\\
3.86e-09	0	\\
3.96e-09	0	\\
4.06e-09	0	\\
4.16e-09	0	\\
4.27e-09	0	\\
4.37e-09	0	\\
4.47e-09	0	\\
4.57e-09	0	\\
4.68e-09	0	\\
4.78e-09	0	\\
4.89e-09	0	\\
4.99e-09	0	\\
5e-09	0	\\
};
\addplot [color=blue,solid,forget plot]
  table[row sep=crcr]{
0	0	\\
1.1e-10	0	\\
2.2e-10	0	\\
3.3e-10	0	\\
4.4e-10	0	\\
5.4e-10	0	\\
6.5e-10	0	\\
7.5e-10	0	\\
8.6e-10	0	\\
9.6e-10	0	\\
1.07e-09	0	\\
1.18e-09	0	\\
1.28e-09	0	\\
1.38e-09	0	\\
1.49e-09	0	\\
1.59e-09	0	\\
1.69e-09	0	\\
1.8e-09	0	\\
1.9e-09	0	\\
2.01e-09	0	\\
2.11e-09	0	\\
2.21e-09	0	\\
2.32e-09	0	\\
2.42e-09	0	\\
2.52e-09	0	\\
2.63e-09	0	\\
2.73e-09	0	\\
2.83e-09	0	\\
2.93e-09	0	\\
3.04e-09	0	\\
3.14e-09	0	\\
3.24e-09	0	\\
3.34e-09	0	\\
3.45e-09	0	\\
3.55e-09	0	\\
3.65e-09	0	\\
3.75e-09	0	\\
3.86e-09	0	\\
3.96e-09	0	\\
4.06e-09	0	\\
4.16e-09	0	\\
4.27e-09	0	\\
4.37e-09	0	\\
4.47e-09	0	\\
4.57e-09	0	\\
4.68e-09	0	\\
4.78e-09	0	\\
4.89e-09	0	\\
4.99e-09	0	\\
5e-09	0	\\
};
\addplot [color=black!50!green,solid,forget plot]
  table[row sep=crcr]{
0	0	\\
1.1e-10	0	\\
2.2e-10	0	\\
3.3e-10	0	\\
4.4e-10	0	\\
5.4e-10	0	\\
6.5e-10	0	\\
7.5e-10	0	\\
8.6e-10	0	\\
9.6e-10	0	\\
1.07e-09	0	\\
1.18e-09	0	\\
1.28e-09	0	\\
1.38e-09	0	\\
1.49e-09	0	\\
1.59e-09	0	\\
1.69e-09	0	\\
1.8e-09	0	\\
1.9e-09	0	\\
2.01e-09	0	\\
2.11e-09	0	\\
2.21e-09	0	\\
2.32e-09	0	\\
2.42e-09	0	\\
2.52e-09	0	\\
2.63e-09	0	\\
2.73e-09	0	\\
2.83e-09	0	\\
2.93e-09	0	\\
3.04e-09	0	\\
3.14e-09	0	\\
3.24e-09	0	\\
3.34e-09	0	\\
3.45e-09	0	\\
3.55e-09	0	\\
3.65e-09	0	\\
3.75e-09	0	\\
3.86e-09	0	\\
3.96e-09	0	\\
4.06e-09	0	\\
4.16e-09	0	\\
4.27e-09	0	\\
4.37e-09	0	\\
4.47e-09	0	\\
4.57e-09	0	\\
4.68e-09	0	\\
4.78e-09	0	\\
4.89e-09	0	\\
4.99e-09	0	\\
5e-09	0	\\
};
\addplot [color=red,solid,forget plot]
  table[row sep=crcr]{
0	0	\\
1.1e-10	0	\\
2.2e-10	0	\\
3.3e-10	0	\\
4.4e-10	0	\\
5.4e-10	0	\\
6.5e-10	0	\\
7.5e-10	0	\\
8.6e-10	0	\\
9.6e-10	0	\\
1.07e-09	0	\\
1.18e-09	0	\\
1.28e-09	0	\\
1.38e-09	0	\\
1.49e-09	0	\\
1.59e-09	0	\\
1.69e-09	0	\\
1.8e-09	0	\\
1.9e-09	0	\\
2.01e-09	0	\\
2.11e-09	0	\\
2.21e-09	0	\\
2.32e-09	0	\\
2.42e-09	0	\\
2.52e-09	0	\\
2.63e-09	0	\\
2.73e-09	0	\\
2.83e-09	0	\\
2.93e-09	0	\\
3.04e-09	0	\\
3.14e-09	0	\\
3.24e-09	0	\\
3.34e-09	0	\\
3.45e-09	0	\\
3.55e-09	0	\\
3.65e-09	0	\\
3.75e-09	0	\\
3.86e-09	0	\\
3.96e-09	0	\\
4.06e-09	0	\\
4.16e-09	0	\\
4.27e-09	0	\\
4.37e-09	0	\\
4.47e-09	0	\\
4.57e-09	0	\\
4.68e-09	0	\\
4.78e-09	0	\\
4.89e-09	0	\\
4.99e-09	0	\\
5e-09	0	\\
};
\addplot [color=mycolor1,solid,forget plot]
  table[row sep=crcr]{
0	0	\\
1.1e-10	0	\\
2.2e-10	0	\\
3.3e-10	0	\\
4.4e-10	0	\\
5.4e-10	0	\\
6.5e-10	0	\\
7.5e-10	0	\\
8.6e-10	0	\\
9.6e-10	0	\\
1.07e-09	0	\\
1.18e-09	0	\\
1.28e-09	0	\\
1.38e-09	0	\\
1.49e-09	0	\\
1.59e-09	0	\\
1.69e-09	0	\\
1.8e-09	0	\\
1.9e-09	0	\\
2.01e-09	0	\\
2.11e-09	0	\\
2.21e-09	0	\\
2.32e-09	0	\\
2.42e-09	0	\\
2.52e-09	0	\\
2.63e-09	0	\\
2.73e-09	0	\\
2.83e-09	0	\\
2.93e-09	0	\\
3.04e-09	0	\\
3.14e-09	0	\\
3.24e-09	0	\\
3.34e-09	0	\\
3.45e-09	0	\\
3.55e-09	0	\\
3.65e-09	0	\\
3.75e-09	0	\\
3.86e-09	0	\\
3.96e-09	0	\\
4.06e-09	0	\\
4.16e-09	0	\\
4.27e-09	0	\\
4.37e-09	0	\\
4.47e-09	0	\\
4.57e-09	0	\\
4.68e-09	0	\\
4.78e-09	0	\\
4.89e-09	0	\\
4.99e-09	0	\\
5e-09	0	\\
};
\addplot [color=mycolor2,solid,forget plot]
  table[row sep=crcr]{
0	0	\\
1.1e-10	0	\\
2.2e-10	0	\\
3.3e-10	0	\\
4.4e-10	0	\\
5.4e-10	0	\\
6.5e-10	0	\\
7.5e-10	0	\\
8.6e-10	0	\\
9.6e-10	0	\\
1.07e-09	0	\\
1.18e-09	0	\\
1.28e-09	0	\\
1.38e-09	0	\\
1.49e-09	0	\\
1.59e-09	0	\\
1.69e-09	0	\\
1.8e-09	0	\\
1.9e-09	0	\\
2.01e-09	0	\\
2.11e-09	0	\\
2.21e-09	0	\\
2.32e-09	0	\\
2.42e-09	0	\\
2.52e-09	0	\\
2.63e-09	0	\\
2.73e-09	0	\\
2.83e-09	0	\\
2.93e-09	0	\\
3.04e-09	0	\\
3.14e-09	0	\\
3.24e-09	0	\\
3.34e-09	0	\\
3.45e-09	0	\\
3.55e-09	0	\\
3.65e-09	0	\\
3.75e-09	0	\\
3.86e-09	0	\\
3.96e-09	0	\\
4.06e-09	0	\\
4.16e-09	0	\\
4.27e-09	0	\\
4.37e-09	0	\\
4.47e-09	0	\\
4.57e-09	0	\\
4.68e-09	0	\\
4.78e-09	0	\\
4.89e-09	0	\\
4.99e-09	0	\\
5e-09	0	\\
};
\addplot [color=mycolor3,solid,forget plot]
  table[row sep=crcr]{
0	0	\\
1.1e-10	0	\\
2.2e-10	0	\\
3.3e-10	0	\\
4.4e-10	0	\\
5.4e-10	0	\\
6.5e-10	0	\\
7.5e-10	0	\\
8.6e-10	0	\\
9.6e-10	0	\\
1.07e-09	0	\\
1.18e-09	0	\\
1.28e-09	0	\\
1.38e-09	0	\\
1.49e-09	0	\\
1.59e-09	0	\\
1.69e-09	0	\\
1.8e-09	0	\\
1.9e-09	0	\\
2.01e-09	0	\\
2.11e-09	0	\\
2.21e-09	0	\\
2.32e-09	0	\\
2.42e-09	0	\\
2.52e-09	0	\\
2.63e-09	0	\\
2.73e-09	0	\\
2.83e-09	0	\\
2.93e-09	0	\\
3.04e-09	0	\\
3.14e-09	0	\\
3.24e-09	0	\\
3.34e-09	0	\\
3.45e-09	0	\\
3.55e-09	0	\\
3.65e-09	0	\\
3.75e-09	0	\\
3.86e-09	0	\\
3.96e-09	0	\\
4.06e-09	0	\\
4.16e-09	0	\\
4.27e-09	0	\\
4.37e-09	0	\\
4.47e-09	0	\\
4.57e-09	0	\\
4.68e-09	0	\\
4.78e-09	0	\\
4.89e-09	0	\\
4.99e-09	0	\\
5e-09	0	\\
};
\addplot [color=darkgray,solid,forget plot]
  table[row sep=crcr]{
0	0	\\
1.1e-10	0	\\
2.2e-10	0	\\
3.3e-10	0	\\
4.4e-10	0	\\
5.4e-10	0	\\
6.5e-10	0	\\
7.5e-10	0	\\
8.6e-10	0	\\
9.6e-10	0	\\
1.07e-09	0	\\
1.18e-09	0	\\
1.28e-09	0	\\
1.38e-09	0	\\
1.49e-09	0	\\
1.59e-09	0	\\
1.69e-09	0	\\
1.8e-09	0	\\
1.9e-09	0	\\
2.01e-09	0	\\
2.11e-09	0	\\
2.21e-09	0	\\
2.32e-09	0	\\
2.42e-09	0	\\
2.52e-09	0	\\
2.63e-09	0	\\
2.73e-09	0	\\
2.83e-09	0	\\
2.93e-09	0	\\
3.04e-09	0	\\
3.14e-09	0	\\
3.24e-09	0	\\
3.34e-09	0	\\
3.45e-09	0	\\
3.55e-09	0	\\
3.65e-09	0	\\
3.75e-09	0	\\
3.86e-09	0	\\
3.96e-09	0	\\
4.06e-09	0	\\
4.16e-09	0	\\
4.27e-09	0	\\
4.37e-09	0	\\
4.47e-09	0	\\
4.57e-09	0	\\
4.68e-09	0	\\
4.78e-09	0	\\
4.89e-09	0	\\
4.99e-09	0	\\
5e-09	0	\\
};
\addplot [color=blue,solid,forget plot]
  table[row sep=crcr]{
0	0	\\
1.1e-10	0	\\
2.2e-10	0	\\
3.3e-10	0	\\
4.4e-10	0	\\
5.4e-10	0	\\
6.5e-10	0	\\
7.5e-10	0	\\
8.6e-10	0	\\
9.6e-10	0	\\
1.07e-09	0	\\
1.18e-09	0	\\
1.28e-09	0	\\
1.38e-09	0	\\
1.49e-09	0	\\
1.59e-09	0	\\
1.69e-09	0	\\
1.8e-09	0	\\
1.9e-09	0	\\
2.01e-09	0	\\
2.11e-09	0	\\
2.21e-09	0	\\
2.32e-09	0	\\
2.42e-09	0	\\
2.52e-09	0	\\
2.63e-09	0	\\
2.73e-09	0	\\
2.83e-09	0	\\
2.93e-09	0	\\
3.04e-09	0	\\
3.14e-09	0	\\
3.24e-09	0	\\
3.34e-09	0	\\
3.45e-09	0	\\
3.55e-09	0	\\
3.65e-09	0	\\
3.75e-09	0	\\
3.86e-09	0	\\
3.96e-09	0	\\
4.06e-09	0	\\
4.16e-09	0	\\
4.27e-09	0	\\
4.37e-09	0	\\
4.47e-09	0	\\
4.57e-09	0	\\
4.68e-09	0	\\
4.78e-09	0	\\
4.89e-09	0	\\
4.99e-09	0	\\
5e-09	0	\\
};
\addplot [color=black!50!green,solid,forget plot]
  table[row sep=crcr]{
0	0	\\
1.1e-10	0	\\
2.2e-10	0	\\
3.3e-10	0	\\
4.4e-10	0	\\
5.4e-10	0	\\
6.5e-10	0	\\
7.5e-10	0	\\
8.6e-10	0	\\
9.6e-10	0	\\
1.07e-09	0	\\
1.18e-09	0	\\
1.28e-09	0	\\
1.38e-09	0	\\
1.49e-09	0	\\
1.59e-09	0	\\
1.69e-09	0	\\
1.8e-09	0	\\
1.9e-09	0	\\
2.01e-09	0	\\
2.11e-09	0	\\
2.21e-09	0	\\
2.32e-09	0	\\
2.42e-09	0	\\
2.52e-09	0	\\
2.63e-09	0	\\
2.73e-09	0	\\
2.83e-09	0	\\
2.93e-09	0	\\
3.04e-09	0	\\
3.14e-09	0	\\
3.24e-09	0	\\
3.34e-09	0	\\
3.45e-09	0	\\
3.55e-09	0	\\
3.65e-09	0	\\
3.75e-09	0	\\
3.86e-09	0	\\
3.96e-09	0	\\
4.06e-09	0	\\
4.16e-09	0	\\
4.27e-09	0	\\
4.37e-09	0	\\
4.47e-09	0	\\
4.57e-09	0	\\
4.68e-09	0	\\
4.78e-09	0	\\
4.89e-09	0	\\
4.99e-09	0	\\
5e-09	0	\\
};
\addplot [color=red,solid,forget plot]
  table[row sep=crcr]{
0	0	\\
1.1e-10	0	\\
2.2e-10	0	\\
3.3e-10	0	\\
4.4e-10	0	\\
5.4e-10	0	\\
6.5e-10	0	\\
7.5e-10	0	\\
8.6e-10	0	\\
9.6e-10	0	\\
1.07e-09	0	\\
1.18e-09	0	\\
1.28e-09	0	\\
1.38e-09	0	\\
1.49e-09	0	\\
1.59e-09	0	\\
1.69e-09	0	\\
1.8e-09	0	\\
1.9e-09	0	\\
2.01e-09	0	\\
2.11e-09	0	\\
2.21e-09	0	\\
2.32e-09	0	\\
2.42e-09	0	\\
2.52e-09	0	\\
2.63e-09	0	\\
2.73e-09	0	\\
2.83e-09	0	\\
2.93e-09	0	\\
3.04e-09	0	\\
3.14e-09	0	\\
3.24e-09	0	\\
3.34e-09	0	\\
3.45e-09	0	\\
3.55e-09	0	\\
3.65e-09	0	\\
3.75e-09	0	\\
3.86e-09	0	\\
3.96e-09	0	\\
4.06e-09	0	\\
4.16e-09	0	\\
4.27e-09	0	\\
4.37e-09	0	\\
4.47e-09	0	\\
4.57e-09	0	\\
4.68e-09	0	\\
4.78e-09	0	\\
4.89e-09	0	\\
4.99e-09	0	\\
5e-09	0	\\
};
\addplot [color=mycolor1,solid,forget plot]
  table[row sep=crcr]{
0	0	\\
1.1e-10	0	\\
2.2e-10	0	\\
3.3e-10	0	\\
4.4e-10	0	\\
5.4e-10	0	\\
6.5e-10	0	\\
7.5e-10	0	\\
8.6e-10	0	\\
9.6e-10	0	\\
1.07e-09	0	\\
1.18e-09	0	\\
1.28e-09	0	\\
1.38e-09	0	\\
1.49e-09	0	\\
1.59e-09	0	\\
1.69e-09	0	\\
1.8e-09	0	\\
1.9e-09	0	\\
2.01e-09	0	\\
2.11e-09	0	\\
2.21e-09	0	\\
2.32e-09	0	\\
2.42e-09	0	\\
2.52e-09	0	\\
2.63e-09	0	\\
2.73e-09	0	\\
2.83e-09	0	\\
2.93e-09	0	\\
3.04e-09	0	\\
3.14e-09	0	\\
3.24e-09	0	\\
3.34e-09	0	\\
3.45e-09	0	\\
3.55e-09	0	\\
3.65e-09	0	\\
3.75e-09	0	\\
3.86e-09	0	\\
3.96e-09	0	\\
4.06e-09	0	\\
4.16e-09	0	\\
4.27e-09	0	\\
4.37e-09	0	\\
4.47e-09	0	\\
4.57e-09	0	\\
4.68e-09	0	\\
4.78e-09	0	\\
4.89e-09	0	\\
4.99e-09	0	\\
5e-09	0	\\
};
\addplot [color=mycolor2,solid,forget plot]
  table[row sep=crcr]{
0	0	\\
1.1e-10	0	\\
2.2e-10	0	\\
3.3e-10	0	\\
4.4e-10	0	\\
5.4e-10	0	\\
6.5e-10	0	\\
7.5e-10	0	\\
8.6e-10	0	\\
9.6e-10	0	\\
1.07e-09	0	\\
1.18e-09	0	\\
1.28e-09	0	\\
1.38e-09	0	\\
1.49e-09	0	\\
1.59e-09	0	\\
1.69e-09	0	\\
1.8e-09	0	\\
1.9e-09	0	\\
2.01e-09	0	\\
2.11e-09	0	\\
2.21e-09	0	\\
2.32e-09	0	\\
2.42e-09	0	\\
2.52e-09	0	\\
2.63e-09	0	\\
2.73e-09	0	\\
2.83e-09	0	\\
2.93e-09	0	\\
3.04e-09	0	\\
3.14e-09	0	\\
3.24e-09	0	\\
3.34e-09	0	\\
3.45e-09	0	\\
3.55e-09	0	\\
3.65e-09	0	\\
3.75e-09	0	\\
3.86e-09	0	\\
3.96e-09	0	\\
4.06e-09	0	\\
4.16e-09	0	\\
4.27e-09	0	\\
4.37e-09	0	\\
4.47e-09	0	\\
4.57e-09	0	\\
4.68e-09	0	\\
4.78e-09	0	\\
4.89e-09	0	\\
4.99e-09	0	\\
5e-09	0	\\
};
\addplot [color=mycolor3,solid,forget plot]
  table[row sep=crcr]{
0	0	\\
1.1e-10	0	\\
2.2e-10	0	\\
3.3e-10	0	\\
4.4e-10	0	\\
5.4e-10	0	\\
6.5e-10	0	\\
7.5e-10	0	\\
8.6e-10	0	\\
9.6e-10	0	\\
1.07e-09	0	\\
1.18e-09	0	\\
1.28e-09	0	\\
1.38e-09	0	\\
1.49e-09	0	\\
1.59e-09	0	\\
1.69e-09	0	\\
1.8e-09	0	\\
1.9e-09	0	\\
2.01e-09	0	\\
2.11e-09	0	\\
2.21e-09	0	\\
2.32e-09	0	\\
2.42e-09	0	\\
2.52e-09	0	\\
2.63e-09	0	\\
2.73e-09	0	\\
2.83e-09	0	\\
2.93e-09	0	\\
3.04e-09	0	\\
3.14e-09	0	\\
3.24e-09	0	\\
3.34e-09	0	\\
3.45e-09	0	\\
3.55e-09	0	\\
3.65e-09	0	\\
3.75e-09	0	\\
3.86e-09	0	\\
3.96e-09	0	\\
4.06e-09	0	\\
4.16e-09	0	\\
4.27e-09	0	\\
4.37e-09	0	\\
4.47e-09	0	\\
4.57e-09	0	\\
4.68e-09	0	\\
4.78e-09	0	\\
4.89e-09	0	\\
4.99e-09	0	\\
5e-09	0	\\
};
\addplot [color=darkgray,solid,forget plot]
  table[row sep=crcr]{
0	0	\\
1.1e-10	0	\\
2.2e-10	0	\\
3.3e-10	0	\\
4.4e-10	0	\\
5.4e-10	0	\\
6.5e-10	0	\\
7.5e-10	0	\\
8.6e-10	0	\\
9.6e-10	0	\\
1.07e-09	0	\\
1.18e-09	0	\\
1.28e-09	0	\\
1.38e-09	0	\\
1.49e-09	0	\\
1.59e-09	0	\\
1.69e-09	0	\\
1.8e-09	0	\\
1.9e-09	0	\\
2.01e-09	0	\\
2.11e-09	0	\\
2.21e-09	0	\\
2.32e-09	0	\\
2.42e-09	0	\\
2.52e-09	0	\\
2.63e-09	0	\\
2.73e-09	0	\\
2.83e-09	0	\\
2.93e-09	0	\\
3.04e-09	0	\\
3.14e-09	0	\\
3.24e-09	0	\\
3.34e-09	0	\\
3.45e-09	0	\\
3.55e-09	0	\\
3.65e-09	0	\\
3.75e-09	0	\\
3.86e-09	0	\\
3.96e-09	0	\\
4.06e-09	0	\\
4.16e-09	0	\\
4.27e-09	0	\\
4.37e-09	0	\\
4.47e-09	0	\\
4.57e-09	0	\\
4.68e-09	0	\\
4.78e-09	0	\\
4.89e-09	0	\\
4.99e-09	0	\\
5e-09	0	\\
};
\addplot [color=blue,solid,forget plot]
  table[row sep=crcr]{
0	0	\\
1.1e-10	0	\\
2.2e-10	0	\\
3.3e-10	0	\\
4.4e-10	0	\\
5.4e-10	0	\\
6.5e-10	0	\\
7.5e-10	0	\\
8.6e-10	0	\\
9.6e-10	0	\\
1.07e-09	0	\\
1.18e-09	0	\\
1.28e-09	0	\\
1.38e-09	0	\\
1.49e-09	0	\\
1.59e-09	0	\\
1.69e-09	0	\\
1.8e-09	0	\\
1.9e-09	0	\\
2.01e-09	0	\\
2.11e-09	0	\\
2.21e-09	0	\\
2.32e-09	0	\\
2.42e-09	0	\\
2.52e-09	0	\\
2.63e-09	0	\\
2.73e-09	0	\\
2.83e-09	0	\\
2.93e-09	0	\\
3.04e-09	0	\\
3.14e-09	0	\\
3.24e-09	0	\\
3.34e-09	0	\\
3.45e-09	0	\\
3.55e-09	0	\\
3.65e-09	0	\\
3.75e-09	0	\\
3.86e-09	0	\\
3.96e-09	0	\\
4.06e-09	0	\\
4.16e-09	0	\\
4.27e-09	0	\\
4.37e-09	0	\\
4.47e-09	0	\\
4.57e-09	0	\\
4.68e-09	0	\\
4.78e-09	0	\\
4.89e-09	0	\\
4.99e-09	0	\\
5e-09	0	\\
};
\addplot [color=black!50!green,solid,forget plot]
  table[row sep=crcr]{
0	0	\\
1.1e-10	0	\\
2.2e-10	0	\\
3.3e-10	0	\\
4.4e-10	0	\\
5.4e-10	0	\\
6.5e-10	0	\\
7.5e-10	0	\\
8.6e-10	0	\\
9.6e-10	0	\\
1.07e-09	0	\\
1.18e-09	0	\\
1.28e-09	0	\\
1.38e-09	0	\\
1.49e-09	0	\\
1.59e-09	0	\\
1.69e-09	0	\\
1.8e-09	0	\\
1.9e-09	0	\\
2.01e-09	0	\\
2.11e-09	0	\\
2.21e-09	0	\\
2.32e-09	0	\\
2.42e-09	0	\\
2.52e-09	0	\\
2.63e-09	0	\\
2.73e-09	0	\\
2.83e-09	0	\\
2.93e-09	0	\\
3.04e-09	0	\\
3.14e-09	0	\\
3.24e-09	0	\\
3.34e-09	0	\\
3.45e-09	0	\\
3.55e-09	0	\\
3.65e-09	0	\\
3.75e-09	0	\\
3.86e-09	0	\\
3.96e-09	0	\\
4.06e-09	0	\\
4.16e-09	0	\\
4.27e-09	0	\\
4.37e-09	0	\\
4.47e-09	0	\\
4.57e-09	0	\\
4.68e-09	0	\\
4.78e-09	0	\\
4.89e-09	0	\\
4.99e-09	0	\\
5e-09	0	\\
};
\addplot [color=red,solid,forget plot]
  table[row sep=crcr]{
0	0	\\
1.1e-10	0	\\
2.2e-10	0	\\
3.3e-10	0	\\
4.4e-10	0	\\
5.4e-10	0	\\
6.5e-10	0	\\
7.5e-10	0	\\
8.6e-10	0	\\
9.6e-10	0	\\
1.07e-09	0	\\
1.18e-09	0	\\
1.28e-09	0	\\
1.38e-09	0	\\
1.49e-09	0	\\
1.59e-09	0	\\
1.69e-09	0	\\
1.8e-09	0	\\
1.9e-09	0	\\
2.01e-09	0	\\
2.11e-09	0	\\
2.21e-09	0	\\
2.32e-09	0	\\
2.42e-09	0	\\
2.52e-09	0	\\
2.63e-09	0	\\
2.73e-09	0	\\
2.83e-09	0	\\
2.93e-09	0	\\
3.04e-09	0	\\
3.14e-09	0	\\
3.24e-09	0	\\
3.34e-09	0	\\
3.45e-09	0	\\
3.55e-09	0	\\
3.65e-09	0	\\
3.75e-09	0	\\
3.86e-09	0	\\
3.96e-09	0	\\
4.06e-09	0	\\
4.16e-09	0	\\
4.27e-09	0	\\
4.37e-09	0	\\
4.47e-09	0	\\
4.57e-09	0	\\
4.68e-09	0	\\
4.78e-09	0	\\
4.89e-09	0	\\
4.99e-09	0	\\
5e-09	0	\\
};
\addplot [color=mycolor1,solid,forget plot]
  table[row sep=crcr]{
0	0	\\
1.1e-10	0	\\
2.2e-10	0	\\
3.3e-10	0	\\
4.4e-10	0	\\
5.4e-10	0	\\
6.5e-10	0	\\
7.5e-10	0	\\
8.6e-10	0	\\
9.6e-10	0	\\
1.07e-09	0	\\
1.18e-09	0	\\
1.28e-09	0	\\
1.38e-09	0	\\
1.49e-09	0	\\
1.59e-09	0	\\
1.69e-09	0	\\
1.8e-09	0	\\
1.9e-09	0	\\
2.01e-09	0	\\
2.11e-09	0	\\
2.21e-09	0	\\
2.32e-09	0	\\
2.42e-09	0	\\
2.52e-09	0	\\
2.63e-09	0	\\
2.73e-09	0	\\
2.83e-09	0	\\
2.93e-09	0	\\
3.04e-09	0	\\
3.14e-09	0	\\
3.24e-09	0	\\
3.34e-09	0	\\
3.45e-09	0	\\
3.55e-09	0	\\
3.65e-09	0	\\
3.75e-09	0	\\
3.86e-09	0	\\
3.96e-09	0	\\
4.06e-09	0	\\
4.16e-09	0	\\
4.27e-09	0	\\
4.37e-09	0	\\
4.47e-09	0	\\
4.57e-09	0	\\
4.68e-09	0	\\
4.78e-09	0	\\
4.89e-09	0	\\
4.99e-09	0	\\
5e-09	0	\\
};
\addplot [color=mycolor2,solid,forget plot]
  table[row sep=crcr]{
0	0	\\
1.1e-10	0	\\
2.2e-10	0	\\
3.3e-10	0	\\
4.4e-10	0	\\
5.4e-10	0	\\
6.5e-10	0	\\
7.5e-10	0	\\
8.6e-10	0	\\
9.6e-10	0	\\
1.07e-09	0	\\
1.18e-09	0	\\
1.28e-09	0	\\
1.38e-09	0	\\
1.49e-09	0	\\
1.59e-09	0	\\
1.69e-09	0	\\
1.8e-09	0	\\
1.9e-09	0	\\
2.01e-09	0	\\
2.11e-09	0	\\
2.21e-09	0	\\
2.32e-09	0	\\
2.42e-09	0	\\
2.52e-09	0	\\
2.63e-09	0	\\
2.73e-09	0	\\
2.83e-09	0	\\
2.93e-09	0	\\
3.04e-09	0	\\
3.14e-09	0	\\
3.24e-09	0	\\
3.34e-09	0	\\
3.45e-09	0	\\
3.55e-09	0	\\
3.65e-09	0	\\
3.75e-09	0	\\
3.86e-09	0	\\
3.96e-09	0	\\
4.06e-09	0	\\
4.16e-09	0	\\
4.27e-09	0	\\
4.37e-09	0	\\
4.47e-09	0	\\
4.57e-09	0	\\
4.68e-09	0	\\
4.78e-09	0	\\
4.89e-09	0	\\
4.99e-09	0	\\
5e-09	0	\\
};
\addplot [color=mycolor3,solid,forget plot]
  table[row sep=crcr]{
0	0	\\
1.1e-10	0	\\
2.2e-10	0	\\
3.3e-10	0	\\
4.4e-10	0	\\
5.4e-10	0	\\
6.5e-10	0	\\
7.5e-10	0	\\
8.6e-10	0	\\
9.6e-10	0	\\
1.07e-09	0	\\
1.18e-09	0	\\
1.28e-09	0	\\
1.38e-09	0	\\
1.49e-09	0	\\
1.59e-09	0	\\
1.69e-09	0	\\
1.8e-09	0	\\
1.9e-09	0	\\
2.01e-09	0	\\
2.11e-09	0	\\
2.21e-09	0	\\
2.32e-09	0	\\
2.42e-09	0	\\
2.52e-09	0	\\
2.63e-09	0	\\
2.73e-09	0	\\
2.83e-09	0	\\
2.93e-09	0	\\
3.04e-09	0	\\
3.14e-09	0	\\
3.24e-09	0	\\
3.34e-09	0	\\
3.45e-09	0	\\
3.55e-09	0	\\
3.65e-09	0	\\
3.75e-09	0	\\
3.86e-09	0	\\
3.96e-09	0	\\
4.06e-09	0	\\
4.16e-09	0	\\
4.27e-09	0	\\
4.37e-09	0	\\
4.47e-09	0	\\
4.57e-09	0	\\
4.68e-09	0	\\
4.78e-09	0	\\
4.89e-09	0	\\
4.99e-09	0	\\
5e-09	0	\\
};
\addplot [color=darkgray,solid,forget plot]
  table[row sep=crcr]{
0	0	\\
1.1e-10	0	\\
2.2e-10	0	\\
3.3e-10	0	\\
4.4e-10	0	\\
5.4e-10	0	\\
6.5e-10	0	\\
7.5e-10	0	\\
8.6e-10	0	\\
9.6e-10	0	\\
1.07e-09	0	\\
1.18e-09	0	\\
1.28e-09	0	\\
1.38e-09	0	\\
1.49e-09	0	\\
1.59e-09	0	\\
1.69e-09	0	\\
1.8e-09	0	\\
1.9e-09	0	\\
2.01e-09	0	\\
2.11e-09	0	\\
2.21e-09	0	\\
2.32e-09	0	\\
2.42e-09	0	\\
2.52e-09	0	\\
2.63e-09	0	\\
2.73e-09	0	\\
2.83e-09	0	\\
2.93e-09	0	\\
3.04e-09	0	\\
3.14e-09	0	\\
3.24e-09	0	\\
3.34e-09	0	\\
3.45e-09	0	\\
3.55e-09	0	\\
3.65e-09	0	\\
3.75e-09	0	\\
3.86e-09	0	\\
3.96e-09	0	\\
4.06e-09	0	\\
4.16e-09	0	\\
4.27e-09	0	\\
4.37e-09	0	\\
4.47e-09	0	\\
4.57e-09	0	\\
4.68e-09	0	\\
4.78e-09	0	\\
4.89e-09	0	\\
4.99e-09	0	\\
5e-09	0	\\
};
\addplot [color=blue,solid,forget plot]
  table[row sep=crcr]{
0	0	\\
1.1e-10	0	\\
2.2e-10	0	\\
3.3e-10	0	\\
4.4e-10	0	\\
5.4e-10	0	\\
6.5e-10	0	\\
7.5e-10	0	\\
8.6e-10	0	\\
9.6e-10	0	\\
1.07e-09	0	\\
1.18e-09	0	\\
1.28e-09	0	\\
1.38e-09	0	\\
1.49e-09	0	\\
1.59e-09	0	\\
1.69e-09	0	\\
1.8e-09	0	\\
1.9e-09	0	\\
2.01e-09	0	\\
2.11e-09	0	\\
2.21e-09	0	\\
2.32e-09	0	\\
2.42e-09	0	\\
2.52e-09	0	\\
2.63e-09	0	\\
2.73e-09	0	\\
2.83e-09	0	\\
2.93e-09	0	\\
3.04e-09	0	\\
3.14e-09	0	\\
3.24e-09	0	\\
3.34e-09	0	\\
3.45e-09	0	\\
3.55e-09	0	\\
3.65e-09	0	\\
3.75e-09	0	\\
3.86e-09	0	\\
3.96e-09	0	\\
4.06e-09	0	\\
4.16e-09	0	\\
4.27e-09	0	\\
4.37e-09	0	\\
4.47e-09	0	\\
4.57e-09	0	\\
4.68e-09	0	\\
4.78e-09	0	\\
4.89e-09	0	\\
4.99e-09	0	\\
5e-09	0	\\
};
\addplot [color=black!50!green,solid,forget plot]
  table[row sep=crcr]{
0	0	\\
1.1e-10	0	\\
2.2e-10	0	\\
3.3e-10	0	\\
4.4e-10	0	\\
5.4e-10	0	\\
6.5e-10	0	\\
7.5e-10	0	\\
8.6e-10	0	\\
9.6e-10	0	\\
1.07e-09	0	\\
1.18e-09	0	\\
1.28e-09	0	\\
1.38e-09	0	\\
1.49e-09	0	\\
1.59e-09	0	\\
1.69e-09	0	\\
1.8e-09	0	\\
1.9e-09	0	\\
2.01e-09	0	\\
2.11e-09	0	\\
2.21e-09	0	\\
2.32e-09	0	\\
2.42e-09	0	\\
2.52e-09	0	\\
2.63e-09	0	\\
2.73e-09	0	\\
2.83e-09	0	\\
2.93e-09	0	\\
3.04e-09	0	\\
3.14e-09	0	\\
3.24e-09	0	\\
3.34e-09	0	\\
3.45e-09	0	\\
3.55e-09	0	\\
3.65e-09	0	\\
3.75e-09	0	\\
3.86e-09	0	\\
3.96e-09	0	\\
4.06e-09	0	\\
4.16e-09	0	\\
4.27e-09	0	\\
4.37e-09	0	\\
4.47e-09	0	\\
4.57e-09	0	\\
4.68e-09	0	\\
4.78e-09	0	\\
4.89e-09	0	\\
4.99e-09	0	\\
5e-09	0	\\
};
\addplot [color=red,solid,forget plot]
  table[row sep=crcr]{
0	0	\\
1.1e-10	0	\\
2.2e-10	0	\\
3.3e-10	0	\\
4.4e-10	0	\\
5.4e-10	0	\\
6.5e-10	0	\\
7.5e-10	0	\\
8.6e-10	0	\\
9.6e-10	0	\\
1.07e-09	0	\\
1.18e-09	0	\\
1.28e-09	0	\\
1.38e-09	0	\\
1.49e-09	0	\\
1.59e-09	0	\\
1.69e-09	0	\\
1.8e-09	0	\\
1.9e-09	0	\\
2.01e-09	0	\\
2.11e-09	0	\\
2.21e-09	0	\\
2.32e-09	0	\\
2.42e-09	0	\\
2.52e-09	0	\\
2.63e-09	0	\\
2.73e-09	0	\\
2.83e-09	0	\\
2.93e-09	0	\\
3.04e-09	0	\\
3.14e-09	0	\\
3.24e-09	0	\\
3.34e-09	0	\\
3.45e-09	0	\\
3.55e-09	0	\\
3.65e-09	0	\\
3.75e-09	0	\\
3.86e-09	0	\\
3.96e-09	0	\\
4.06e-09	0	\\
4.16e-09	0	\\
4.27e-09	0	\\
4.37e-09	0	\\
4.47e-09	0	\\
4.57e-09	0	\\
4.68e-09	0	\\
4.78e-09	0	\\
4.89e-09	0	\\
4.99e-09	0	\\
5e-09	0	\\
};
\addplot [color=mycolor1,solid,forget plot]
  table[row sep=crcr]{
0	0	\\
1.1e-10	0	\\
2.2e-10	0	\\
3.3e-10	0	\\
4.4e-10	0	\\
5.4e-10	0	\\
6.5e-10	0	\\
7.5e-10	0	\\
8.6e-10	0	\\
9.6e-10	0	\\
1.07e-09	0	\\
1.18e-09	0	\\
1.28e-09	0	\\
1.38e-09	0	\\
1.49e-09	0	\\
1.59e-09	0	\\
1.69e-09	0	\\
1.8e-09	0	\\
1.9e-09	0	\\
2.01e-09	0	\\
2.11e-09	0	\\
2.21e-09	0	\\
2.32e-09	0	\\
2.42e-09	0	\\
2.52e-09	0	\\
2.63e-09	0	\\
2.73e-09	0	\\
2.83e-09	0	\\
2.93e-09	0	\\
3.04e-09	0	\\
3.14e-09	0	\\
3.24e-09	0	\\
3.34e-09	0	\\
3.45e-09	0	\\
3.55e-09	0	\\
3.65e-09	0	\\
3.75e-09	0	\\
3.86e-09	0	\\
3.96e-09	0	\\
4.06e-09	0	\\
4.16e-09	0	\\
4.27e-09	0	\\
4.37e-09	0	\\
4.47e-09	0	\\
4.57e-09	0	\\
4.68e-09	0	\\
4.78e-09	0	\\
4.89e-09	0	\\
4.99e-09	0	\\
5e-09	0	\\
};
\addplot [color=mycolor2,solid,forget plot]
  table[row sep=crcr]{
0	0	\\
1.1e-10	0	\\
2.2e-10	0	\\
3.3e-10	0	\\
4.4e-10	0	\\
5.4e-10	0	\\
6.5e-10	0	\\
7.5e-10	0	\\
8.6e-10	0	\\
9.6e-10	0	\\
1.07e-09	0	\\
1.18e-09	0	\\
1.28e-09	0	\\
1.38e-09	0	\\
1.49e-09	0	\\
1.59e-09	0	\\
1.69e-09	0	\\
1.8e-09	0	\\
1.9e-09	0	\\
2.01e-09	0	\\
2.11e-09	0	\\
2.21e-09	0	\\
2.32e-09	0	\\
2.42e-09	0	\\
2.52e-09	0	\\
2.63e-09	0	\\
2.73e-09	0	\\
2.83e-09	0	\\
2.93e-09	0	\\
3.04e-09	0	\\
3.14e-09	0	\\
3.24e-09	0	\\
3.34e-09	0	\\
3.45e-09	0	\\
3.55e-09	0	\\
3.65e-09	0	\\
3.75e-09	0	\\
3.86e-09	0	\\
3.96e-09	0	\\
4.06e-09	0	\\
4.16e-09	0	\\
4.27e-09	0	\\
4.37e-09	0	\\
4.47e-09	0	\\
4.57e-09	0	\\
4.68e-09	0	\\
4.78e-09	0	\\
4.89e-09	0	\\
4.99e-09	0	\\
5e-09	0	\\
};
\addplot [color=mycolor3,solid,forget plot]
  table[row sep=crcr]{
0	0	\\
1.1e-10	0	\\
2.2e-10	0	\\
3.3e-10	0	\\
4.4e-10	0	\\
5.4e-10	0	\\
6.5e-10	0	\\
7.5e-10	0	\\
8.6e-10	0	\\
9.6e-10	0	\\
1.07e-09	0	\\
1.18e-09	0	\\
1.28e-09	0	\\
1.38e-09	0	\\
1.49e-09	0	\\
1.59e-09	0	\\
1.69e-09	0	\\
1.8e-09	0	\\
1.9e-09	0	\\
2.01e-09	0	\\
2.11e-09	0	\\
2.21e-09	0	\\
2.32e-09	0	\\
2.42e-09	0	\\
2.52e-09	0	\\
2.63e-09	0	\\
2.73e-09	0	\\
2.83e-09	0	\\
2.93e-09	0	\\
3.04e-09	0	\\
3.14e-09	0	\\
3.24e-09	0	\\
3.34e-09	0	\\
3.45e-09	0	\\
3.55e-09	0	\\
3.65e-09	0	\\
3.75e-09	0	\\
3.86e-09	0	\\
3.96e-09	0	\\
4.06e-09	0	\\
4.16e-09	0	\\
4.27e-09	0	\\
4.37e-09	0	\\
4.47e-09	0	\\
4.57e-09	0	\\
4.68e-09	0	\\
4.78e-09	0	\\
4.89e-09	0	\\
4.99e-09	0	\\
5e-09	0	\\
};
\addplot [color=darkgray,solid,forget plot]
  table[row sep=crcr]{
0	0	\\
1.1e-10	0	\\
2.2e-10	0	\\
3.3e-10	0	\\
4.4e-10	0	\\
5.4e-10	0	\\
6.5e-10	0	\\
7.5e-10	0	\\
8.6e-10	0	\\
9.6e-10	0	\\
1.07e-09	0	\\
1.18e-09	0	\\
1.28e-09	0	\\
1.38e-09	0	\\
1.49e-09	0	\\
1.59e-09	0	\\
1.69e-09	0	\\
1.8e-09	0	\\
1.9e-09	0	\\
2.01e-09	0	\\
2.11e-09	0	\\
2.21e-09	0	\\
2.32e-09	0	\\
2.42e-09	0	\\
2.52e-09	0	\\
2.63e-09	0	\\
2.73e-09	0	\\
2.83e-09	0	\\
2.93e-09	0	\\
3.04e-09	0	\\
3.14e-09	0	\\
3.24e-09	0	\\
3.34e-09	0	\\
3.45e-09	0	\\
3.55e-09	0	\\
3.65e-09	0	\\
3.75e-09	0	\\
3.86e-09	0	\\
3.96e-09	0	\\
4.06e-09	0	\\
4.16e-09	0	\\
4.27e-09	0	\\
4.37e-09	0	\\
4.47e-09	0	\\
4.57e-09	0	\\
4.68e-09	0	\\
4.78e-09	0	\\
4.89e-09	0	\\
4.99e-09	0	\\
5e-09	0	\\
};
\addplot [color=blue,solid,forget plot]
  table[row sep=crcr]{
0	0	\\
1.1e-10	0	\\
2.2e-10	0	\\
3.3e-10	0	\\
4.4e-10	0	\\
5.4e-10	0	\\
6.5e-10	0	\\
7.5e-10	0	\\
8.6e-10	0	\\
9.6e-10	0	\\
1.07e-09	0	\\
1.18e-09	0	\\
1.28e-09	0	\\
1.38e-09	0	\\
1.49e-09	0	\\
1.59e-09	0	\\
1.69e-09	0	\\
1.8e-09	0	\\
1.9e-09	0	\\
2.01e-09	0	\\
2.11e-09	0	\\
2.21e-09	0	\\
2.32e-09	0	\\
2.42e-09	0	\\
2.52e-09	0	\\
2.63e-09	0	\\
2.73e-09	0	\\
2.83e-09	0	\\
2.93e-09	0	\\
3.04e-09	0	\\
3.14e-09	0	\\
3.24e-09	0	\\
3.34e-09	0	\\
3.45e-09	0	\\
3.55e-09	0	\\
3.65e-09	0	\\
3.75e-09	0	\\
3.86e-09	0	\\
3.96e-09	0	\\
4.06e-09	0	\\
4.16e-09	0	\\
4.27e-09	0	\\
4.37e-09	0	\\
4.47e-09	0	\\
4.57e-09	0	\\
4.68e-09	0	\\
4.78e-09	0	\\
4.89e-09	0	\\
4.99e-09	0	\\
5e-09	0	\\
};
\addplot [color=black!50!green,solid,forget plot]
  table[row sep=crcr]{
0	0	\\
1.1e-10	0	\\
2.2e-10	0	\\
3.3e-10	0	\\
4.4e-10	0	\\
5.4e-10	0	\\
6.5e-10	0	\\
7.5e-10	0	\\
8.6e-10	0	\\
9.6e-10	0	\\
1.07e-09	0	\\
1.18e-09	0	\\
1.28e-09	0	\\
1.38e-09	0	\\
1.49e-09	0	\\
1.59e-09	0	\\
1.69e-09	0	\\
1.8e-09	0	\\
1.9e-09	0	\\
2.01e-09	0	\\
2.11e-09	0	\\
2.21e-09	0	\\
2.32e-09	0	\\
2.42e-09	0	\\
2.52e-09	0	\\
2.63e-09	0	\\
2.73e-09	0	\\
2.83e-09	0	\\
2.93e-09	0	\\
3.04e-09	0	\\
3.14e-09	0	\\
3.24e-09	0	\\
3.34e-09	0	\\
3.45e-09	0	\\
3.55e-09	0	\\
3.65e-09	0	\\
3.75e-09	0	\\
3.86e-09	0	\\
3.96e-09	0	\\
4.06e-09	0	\\
4.16e-09	0	\\
4.27e-09	0	\\
4.37e-09	0	\\
4.47e-09	0	\\
4.57e-09	0	\\
4.68e-09	0	\\
4.78e-09	0	\\
4.89e-09	0	\\
4.99e-09	0	\\
5e-09	0	\\
};
\addplot [color=red,solid,forget plot]
  table[row sep=crcr]{
0	0	\\
1.1e-10	0	\\
2.2e-10	0	\\
3.3e-10	0	\\
4.4e-10	0	\\
5.4e-10	0	\\
6.5e-10	0	\\
7.5e-10	0	\\
8.6e-10	0	\\
9.6e-10	0	\\
1.07e-09	0	\\
1.18e-09	0	\\
1.28e-09	0	\\
1.38e-09	0	\\
1.49e-09	0	\\
1.59e-09	0	\\
1.69e-09	0	\\
1.8e-09	0	\\
1.9e-09	0	\\
2.01e-09	0	\\
2.11e-09	0	\\
2.21e-09	0	\\
2.32e-09	0	\\
2.42e-09	0	\\
2.52e-09	0	\\
2.63e-09	0	\\
2.73e-09	0	\\
2.83e-09	0	\\
2.93e-09	0	\\
3.04e-09	0	\\
3.14e-09	0	\\
3.24e-09	0	\\
3.34e-09	0	\\
3.45e-09	0	\\
3.55e-09	0	\\
3.65e-09	0	\\
3.75e-09	0	\\
3.86e-09	0	\\
3.96e-09	0	\\
4.06e-09	0	\\
4.16e-09	0	\\
4.27e-09	0	\\
4.37e-09	0	\\
4.47e-09	0	\\
4.57e-09	0	\\
4.68e-09	0	\\
4.78e-09	0	\\
4.89e-09	0	\\
4.99e-09	0	\\
5e-09	0	\\
};
\addplot [color=mycolor1,solid,forget plot]
  table[row sep=crcr]{
0	0	\\
1.1e-10	0	\\
2.2e-10	0	\\
3.3e-10	0	\\
4.4e-10	0	\\
5.4e-10	0	\\
6.5e-10	0	\\
7.5e-10	0	\\
8.6e-10	0	\\
9.6e-10	0	\\
1.07e-09	0	\\
1.18e-09	0	\\
1.28e-09	0	\\
1.38e-09	0	\\
1.49e-09	0	\\
1.59e-09	0	\\
1.69e-09	0	\\
1.8e-09	0	\\
1.9e-09	0	\\
2.01e-09	0	\\
2.11e-09	0	\\
2.21e-09	0	\\
2.32e-09	0	\\
2.42e-09	0	\\
2.52e-09	0	\\
2.63e-09	0	\\
2.73e-09	0	\\
2.83e-09	0	\\
2.93e-09	0	\\
3.04e-09	0	\\
3.14e-09	0	\\
3.24e-09	0	\\
3.34e-09	0	\\
3.45e-09	0	\\
3.55e-09	0	\\
3.65e-09	0	\\
3.75e-09	0	\\
3.86e-09	0	\\
3.96e-09	0	\\
4.06e-09	0	\\
4.16e-09	0	\\
4.27e-09	0	\\
4.37e-09	0	\\
4.47e-09	0	\\
4.57e-09	0	\\
4.68e-09	0	\\
4.78e-09	0	\\
4.89e-09	0	\\
4.99e-09	0	\\
5e-09	0	\\
};
\addplot [color=mycolor2,solid,forget plot]
  table[row sep=crcr]{
0	0	\\
1.1e-10	0	\\
2.2e-10	0	\\
3.3e-10	0	\\
4.4e-10	0	\\
5.4e-10	0	\\
6.5e-10	0	\\
7.5e-10	0	\\
8.6e-10	0	\\
9.6e-10	0	\\
1.07e-09	0	\\
1.18e-09	0	\\
1.28e-09	0	\\
1.38e-09	0	\\
1.49e-09	0	\\
1.59e-09	0	\\
1.69e-09	0	\\
1.8e-09	0	\\
1.9e-09	0	\\
2.01e-09	0	\\
2.11e-09	0	\\
2.21e-09	0	\\
2.32e-09	0	\\
2.42e-09	0	\\
2.52e-09	0	\\
2.63e-09	0	\\
2.73e-09	0	\\
2.83e-09	0	\\
2.93e-09	0	\\
3.04e-09	0	\\
3.14e-09	0	\\
3.24e-09	0	\\
3.34e-09	0	\\
3.45e-09	0	\\
3.55e-09	0	\\
3.65e-09	0	\\
3.75e-09	0	\\
3.86e-09	0	\\
3.96e-09	0	\\
4.06e-09	0	\\
4.16e-09	0	\\
4.27e-09	0	\\
4.37e-09	0	\\
4.47e-09	0	\\
4.57e-09	0	\\
4.68e-09	0	\\
4.78e-09	0	\\
4.89e-09	0	\\
4.99e-09	0	\\
5e-09	0	\\
};
\addplot [color=mycolor3,solid,forget plot]
  table[row sep=crcr]{
0	0	\\
1.1e-10	0	\\
2.2e-10	0	\\
3.3e-10	0	\\
4.4e-10	0	\\
5.4e-10	0	\\
6.5e-10	0	\\
7.5e-10	0	\\
8.6e-10	0	\\
9.6e-10	0	\\
1.07e-09	0	\\
1.18e-09	0	\\
1.28e-09	0	\\
1.38e-09	0	\\
1.49e-09	0	\\
1.59e-09	0	\\
1.69e-09	0	\\
1.8e-09	0	\\
1.9e-09	0	\\
2.01e-09	0	\\
2.11e-09	0	\\
2.21e-09	0	\\
2.32e-09	0	\\
2.42e-09	0	\\
2.52e-09	0	\\
2.63e-09	0	\\
2.73e-09	0	\\
2.83e-09	0	\\
2.93e-09	0	\\
3.04e-09	0	\\
3.14e-09	0	\\
3.24e-09	0	\\
3.34e-09	0	\\
3.45e-09	0	\\
3.55e-09	0	\\
3.65e-09	0	\\
3.75e-09	0	\\
3.86e-09	0	\\
3.96e-09	0	\\
4.06e-09	0	\\
4.16e-09	0	\\
4.27e-09	0	\\
4.37e-09	0	\\
4.47e-09	0	\\
4.57e-09	0	\\
4.68e-09	0	\\
4.78e-09	0	\\
4.89e-09	0	\\
4.99e-09	0	\\
5e-09	0	\\
};
\addplot [color=darkgray,solid,forget plot]
  table[row sep=crcr]{
0	0	\\
1.1e-10	0	\\
2.2e-10	0	\\
3.3e-10	0	\\
4.4e-10	0	\\
5.4e-10	0	\\
6.5e-10	0	\\
7.5e-10	0	\\
8.6e-10	0	\\
9.6e-10	0	\\
1.07e-09	0	\\
1.18e-09	0	\\
1.28e-09	0	\\
1.38e-09	0	\\
1.49e-09	0	\\
1.59e-09	0	\\
1.69e-09	0	\\
1.8e-09	0	\\
1.9e-09	0	\\
2.01e-09	0	\\
2.11e-09	0	\\
2.21e-09	0	\\
2.32e-09	0	\\
2.42e-09	0	\\
2.52e-09	0	\\
2.63e-09	0	\\
2.73e-09	0	\\
2.83e-09	0	\\
2.93e-09	0	\\
3.04e-09	0	\\
3.14e-09	0	\\
3.24e-09	0	\\
3.34e-09	0	\\
3.45e-09	0	\\
3.55e-09	0	\\
3.65e-09	0	\\
3.75e-09	0	\\
3.86e-09	0	\\
3.96e-09	0	\\
4.06e-09	0	\\
4.16e-09	0	\\
4.27e-09	0	\\
4.37e-09	0	\\
4.47e-09	0	\\
4.57e-09	0	\\
4.68e-09	0	\\
4.78e-09	0	\\
4.89e-09	0	\\
4.99e-09	0	\\
5e-09	0	\\
};
\addplot [color=blue,solid,forget plot]
  table[row sep=crcr]{
0	0	\\
1.1e-10	0	\\
2.2e-10	0	\\
3.3e-10	0	\\
4.4e-10	0	\\
5.4e-10	0	\\
6.5e-10	0	\\
7.5e-10	0	\\
8.6e-10	0	\\
9.6e-10	0	\\
1.07e-09	0	\\
1.18e-09	0	\\
1.28e-09	0	\\
1.38e-09	0	\\
1.49e-09	0	\\
1.59e-09	0	\\
1.69e-09	0	\\
1.8e-09	0	\\
1.9e-09	0	\\
2.01e-09	0	\\
2.11e-09	0	\\
2.21e-09	0	\\
2.32e-09	0	\\
2.42e-09	0	\\
2.52e-09	0	\\
2.63e-09	0	\\
2.73e-09	0	\\
2.83e-09	0	\\
2.93e-09	0	\\
3.04e-09	0	\\
3.14e-09	0	\\
3.24e-09	0	\\
3.34e-09	0	\\
3.45e-09	0	\\
3.55e-09	0	\\
3.65e-09	0	\\
3.75e-09	0	\\
3.86e-09	0	\\
3.96e-09	0	\\
4.06e-09	0	\\
4.16e-09	0	\\
4.27e-09	0	\\
4.37e-09	0	\\
4.47e-09	0	\\
4.57e-09	0	\\
4.68e-09	0	\\
4.78e-09	0	\\
4.89e-09	0	\\
4.99e-09	0	\\
5e-09	0	\\
};
\addplot [color=black!50!green,solid,forget plot]
  table[row sep=crcr]{
0	0	\\
1.1e-10	0	\\
2.2e-10	0	\\
3.3e-10	0	\\
4.4e-10	0	\\
5.4e-10	0	\\
6.5e-10	0	\\
7.5e-10	0	\\
8.6e-10	0	\\
9.6e-10	0	\\
1.07e-09	0	\\
1.18e-09	0	\\
1.28e-09	0	\\
1.38e-09	0	\\
1.49e-09	0	\\
1.59e-09	0	\\
1.69e-09	0	\\
1.8e-09	0	\\
1.9e-09	0	\\
2.01e-09	0	\\
2.11e-09	0	\\
2.21e-09	0	\\
2.32e-09	0	\\
2.42e-09	0	\\
2.52e-09	0	\\
2.63e-09	0	\\
2.73e-09	0	\\
2.83e-09	0	\\
2.93e-09	0	\\
3.04e-09	0	\\
3.14e-09	0	\\
3.24e-09	0	\\
3.34e-09	0	\\
3.45e-09	0	\\
3.55e-09	0	\\
3.65e-09	0	\\
3.75e-09	0	\\
3.86e-09	0	\\
3.96e-09	0	\\
4.06e-09	0	\\
4.16e-09	0	\\
4.27e-09	0	\\
4.37e-09	0	\\
4.47e-09	0	\\
4.57e-09	0	\\
4.68e-09	0	\\
4.78e-09	0	\\
4.89e-09	0	\\
4.99e-09	0	\\
5e-09	0	\\
};
\addplot [color=red,solid,forget plot]
  table[row sep=crcr]{
0	0	\\
1.1e-10	0	\\
2.2e-10	0	\\
3.3e-10	0	\\
4.4e-10	0	\\
5.4e-10	0	\\
6.5e-10	0	\\
7.5e-10	0	\\
8.6e-10	0	\\
9.6e-10	0	\\
1.07e-09	0	\\
1.18e-09	0	\\
1.28e-09	0	\\
1.38e-09	0	\\
1.49e-09	0	\\
1.59e-09	0	\\
1.69e-09	0	\\
1.8e-09	0	\\
1.9e-09	0	\\
2.01e-09	0	\\
2.11e-09	0	\\
2.21e-09	0	\\
2.32e-09	0	\\
2.42e-09	0	\\
2.52e-09	0	\\
2.63e-09	0	\\
2.73e-09	0	\\
2.83e-09	0	\\
2.93e-09	0	\\
3.04e-09	0	\\
3.14e-09	0	\\
3.24e-09	0	\\
3.34e-09	0	\\
3.45e-09	0	\\
3.55e-09	0	\\
3.65e-09	0	\\
3.75e-09	0	\\
3.86e-09	0	\\
3.96e-09	0	\\
4.06e-09	0	\\
4.16e-09	0	\\
4.27e-09	0	\\
4.37e-09	0	\\
4.47e-09	0	\\
4.57e-09	0	\\
4.68e-09	0	\\
4.78e-09	0	\\
4.89e-09	0	\\
4.99e-09	0	\\
5e-09	0	\\
};
\addplot [color=mycolor1,solid,forget plot]
  table[row sep=crcr]{
0	0	\\
1.1e-10	0	\\
2.2e-10	0	\\
3.3e-10	0	\\
4.4e-10	0	\\
5.4e-10	0	\\
6.5e-10	0	\\
7.5e-10	0	\\
8.6e-10	0	\\
9.6e-10	0	\\
1.07e-09	0	\\
1.18e-09	0	\\
1.28e-09	0	\\
1.38e-09	0	\\
1.49e-09	0	\\
1.59e-09	0	\\
1.69e-09	0	\\
1.8e-09	0	\\
1.9e-09	0	\\
2.01e-09	0	\\
2.11e-09	0	\\
2.21e-09	0	\\
2.32e-09	0	\\
2.42e-09	0	\\
2.52e-09	0	\\
2.63e-09	0	\\
2.73e-09	0	\\
2.83e-09	0	\\
2.93e-09	0	\\
3.04e-09	0	\\
3.14e-09	0	\\
3.24e-09	0	\\
3.34e-09	0	\\
3.45e-09	0	\\
3.55e-09	0	\\
3.65e-09	0	\\
3.75e-09	0	\\
3.86e-09	0	\\
3.96e-09	0	\\
4.06e-09	0	\\
4.16e-09	0	\\
4.27e-09	0	\\
4.37e-09	0	\\
4.47e-09	0	\\
4.57e-09	0	\\
4.68e-09	0	\\
4.78e-09	0	\\
4.89e-09	0	\\
4.99e-09	0	\\
5e-09	0	\\
};
\addplot [color=mycolor2,solid,forget plot]
  table[row sep=crcr]{
0	0	\\
1.1e-10	0	\\
2.2e-10	0	\\
3.3e-10	0	\\
4.4e-10	0	\\
5.4e-10	0	\\
6.5e-10	0	\\
7.5e-10	0	\\
8.6e-10	0	\\
9.6e-10	0	\\
1.07e-09	0	\\
1.18e-09	0	\\
1.28e-09	0	\\
1.38e-09	0	\\
1.49e-09	0	\\
1.59e-09	0	\\
1.69e-09	0	\\
1.8e-09	0	\\
1.9e-09	0	\\
2.01e-09	0	\\
2.11e-09	0	\\
2.21e-09	0	\\
2.32e-09	0	\\
2.42e-09	0	\\
2.52e-09	0	\\
2.63e-09	0	\\
2.73e-09	0	\\
2.83e-09	0	\\
2.93e-09	0	\\
3.04e-09	0	\\
3.14e-09	0	\\
3.24e-09	0	\\
3.34e-09	0	\\
3.45e-09	0	\\
3.55e-09	0	\\
3.65e-09	0	\\
3.75e-09	0	\\
3.86e-09	0	\\
3.96e-09	0	\\
4.06e-09	0	\\
4.16e-09	0	\\
4.27e-09	0	\\
4.37e-09	0	\\
4.47e-09	0	\\
4.57e-09	0	\\
4.68e-09	0	\\
4.78e-09	0	\\
4.89e-09	0	\\
4.99e-09	0	\\
5e-09	0	\\
};
\addplot [color=mycolor3,solid,forget plot]
  table[row sep=crcr]{
0	0	\\
1.1e-10	0	\\
2.2e-10	0	\\
3.3e-10	0	\\
4.4e-10	0	\\
5.4e-10	0	\\
6.5e-10	0	\\
7.5e-10	0	\\
8.6e-10	0	\\
9.6e-10	0	\\
1.07e-09	0	\\
1.18e-09	0	\\
1.28e-09	0	\\
1.38e-09	0	\\
1.49e-09	0	\\
1.59e-09	0	\\
1.69e-09	0	\\
1.8e-09	0	\\
1.9e-09	0	\\
2.01e-09	0	\\
2.11e-09	0	\\
2.21e-09	0	\\
2.32e-09	0	\\
2.42e-09	0	\\
2.52e-09	0	\\
2.63e-09	0	\\
2.73e-09	0	\\
2.83e-09	0	\\
2.93e-09	0	\\
3.04e-09	0	\\
3.14e-09	0	\\
3.24e-09	0	\\
3.34e-09	0	\\
3.45e-09	0	\\
3.55e-09	0	\\
3.65e-09	0	\\
3.75e-09	0	\\
3.86e-09	0	\\
3.96e-09	0	\\
4.06e-09	0	\\
4.16e-09	0	\\
4.27e-09	0	\\
4.37e-09	0	\\
4.47e-09	0	\\
4.57e-09	0	\\
4.68e-09	0	\\
4.78e-09	0	\\
4.89e-09	0	\\
4.99e-09	0	\\
5e-09	0	\\
};
\addplot [color=darkgray,solid,forget plot]
  table[row sep=crcr]{
0	0	\\
1.1e-10	0	\\
2.2e-10	0	\\
3.3e-10	0	\\
4.4e-10	0	\\
5.4e-10	0	\\
6.5e-10	0	\\
7.5e-10	0	\\
8.6e-10	0	\\
9.6e-10	0	\\
1.07e-09	0	\\
1.18e-09	0	\\
1.28e-09	0	\\
1.38e-09	0	\\
1.49e-09	0	\\
1.59e-09	0	\\
1.69e-09	0	\\
1.8e-09	0	\\
1.9e-09	0	\\
2.01e-09	0	\\
2.11e-09	0	\\
2.21e-09	0	\\
2.32e-09	0	\\
2.42e-09	0	\\
2.52e-09	0	\\
2.63e-09	0	\\
2.73e-09	0	\\
2.83e-09	0	\\
2.93e-09	0	\\
3.04e-09	0	\\
3.14e-09	0	\\
3.24e-09	0	\\
3.34e-09	0	\\
3.45e-09	0	\\
3.55e-09	0	\\
3.65e-09	0	\\
3.75e-09	0	\\
3.86e-09	0	\\
3.96e-09	0	\\
4.06e-09	0	\\
4.16e-09	0	\\
4.27e-09	0	\\
4.37e-09	0	\\
4.47e-09	0	\\
4.57e-09	0	\\
4.68e-09	0	\\
4.78e-09	0	\\
4.89e-09	0	\\
4.99e-09	0	\\
5e-09	0	\\
};
\addplot [color=blue,solid,forget plot]
  table[row sep=crcr]{
0	0	\\
1.1e-10	0	\\
2.2e-10	0	\\
3.3e-10	0	\\
4.4e-10	0	\\
5.4e-10	0	\\
6.5e-10	0	\\
7.5e-10	0	\\
8.6e-10	0	\\
9.6e-10	0	\\
1.07e-09	0	\\
1.18e-09	0	\\
1.28e-09	0	\\
1.38e-09	0	\\
1.49e-09	0	\\
1.59e-09	0	\\
1.69e-09	0	\\
1.8e-09	0	\\
1.9e-09	0	\\
2.01e-09	0	\\
2.11e-09	0	\\
2.21e-09	0	\\
2.32e-09	0	\\
2.42e-09	0	\\
2.52e-09	0	\\
2.63e-09	0	\\
2.73e-09	0	\\
2.83e-09	0	\\
2.93e-09	0	\\
3.04e-09	0	\\
3.14e-09	0	\\
3.24e-09	0	\\
3.34e-09	0	\\
3.45e-09	0	\\
3.55e-09	0	\\
3.65e-09	0	\\
3.75e-09	0	\\
3.86e-09	0	\\
3.96e-09	0	\\
4.06e-09	0	\\
4.16e-09	0	\\
4.27e-09	0	\\
4.37e-09	0	\\
4.47e-09	0	\\
4.57e-09	0	\\
4.68e-09	0	\\
4.78e-09	0	\\
4.89e-09	0	\\
4.99e-09	0	\\
5e-09	0	\\
};
\addplot [color=black!50!green,solid,forget plot]
  table[row sep=crcr]{
0	0	\\
1.1e-10	0	\\
2.2e-10	0	\\
3.3e-10	0	\\
4.4e-10	0	\\
5.4e-10	0	\\
6.5e-10	0	\\
7.5e-10	0	\\
8.6e-10	0	\\
9.6e-10	0	\\
1.07e-09	0	\\
1.18e-09	0	\\
1.28e-09	0	\\
1.38e-09	0	\\
1.49e-09	0	\\
1.59e-09	0	\\
1.69e-09	0	\\
1.8e-09	0	\\
1.9e-09	0	\\
2.01e-09	0	\\
2.11e-09	0	\\
2.21e-09	0	\\
2.32e-09	0	\\
2.42e-09	0	\\
2.52e-09	0	\\
2.63e-09	0	\\
2.73e-09	0	\\
2.83e-09	0	\\
2.93e-09	0	\\
3.04e-09	0	\\
3.14e-09	0	\\
3.24e-09	0	\\
3.34e-09	0	\\
3.45e-09	0	\\
3.55e-09	0	\\
3.65e-09	0	\\
3.75e-09	0	\\
3.86e-09	0	\\
3.96e-09	0	\\
4.06e-09	0	\\
4.16e-09	0	\\
4.27e-09	0	\\
4.37e-09	0	\\
4.47e-09	0	\\
4.57e-09	0	\\
4.68e-09	0	\\
4.78e-09	0	\\
4.89e-09	0	\\
4.99e-09	0	\\
5e-09	0	\\
};
\addplot [color=red,solid,forget plot]
  table[row sep=crcr]{
0	0	\\
1.1e-10	0	\\
2.2e-10	0	\\
3.3e-10	0	\\
4.4e-10	0	\\
5.4e-10	0	\\
6.5e-10	0	\\
7.5e-10	0	\\
8.6e-10	0	\\
9.6e-10	0	\\
1.07e-09	0	\\
1.18e-09	0	\\
1.28e-09	0	\\
1.38e-09	0	\\
1.49e-09	0	\\
1.59e-09	0	\\
1.69e-09	0	\\
1.8e-09	0	\\
1.9e-09	0	\\
2.01e-09	0	\\
2.11e-09	0	\\
2.21e-09	0	\\
2.32e-09	0	\\
2.42e-09	0	\\
2.52e-09	0	\\
2.63e-09	0	\\
2.73e-09	0	\\
2.83e-09	0	\\
2.93e-09	0	\\
3.04e-09	0	\\
3.14e-09	0	\\
3.24e-09	0	\\
3.34e-09	0	\\
3.45e-09	0	\\
3.55e-09	0	\\
3.65e-09	0	\\
3.75e-09	0	\\
3.86e-09	0	\\
3.96e-09	0	\\
4.06e-09	0	\\
4.16e-09	0	\\
4.27e-09	0	\\
4.37e-09	0	\\
4.47e-09	0	\\
4.57e-09	0	\\
4.68e-09	0	\\
4.78e-09	0	\\
4.89e-09	0	\\
4.99e-09	0	\\
5e-09	0	\\
};
\addplot [color=mycolor1,solid,forget plot]
  table[row sep=crcr]{
0	0	\\
1.1e-10	0	\\
2.2e-10	0	\\
3.3e-10	0	\\
4.4e-10	0	\\
5.4e-10	0	\\
6.5e-10	0	\\
7.5e-10	0	\\
8.6e-10	0	\\
9.6e-10	0	\\
1.07e-09	0	\\
1.18e-09	0	\\
1.28e-09	0	\\
1.38e-09	0	\\
1.49e-09	0	\\
1.59e-09	0	\\
1.69e-09	0	\\
1.8e-09	0	\\
1.9e-09	0	\\
2.01e-09	0	\\
2.11e-09	0	\\
2.21e-09	0	\\
2.32e-09	0	\\
2.42e-09	0	\\
2.52e-09	0	\\
2.63e-09	0	\\
2.73e-09	0	\\
2.83e-09	0	\\
2.93e-09	0	\\
3.04e-09	0	\\
3.14e-09	0	\\
3.24e-09	0	\\
3.34e-09	0	\\
3.45e-09	0	\\
3.55e-09	0	\\
3.65e-09	0	\\
3.75e-09	0	\\
3.86e-09	0	\\
3.96e-09	0	\\
4.06e-09	0	\\
4.16e-09	0	\\
4.27e-09	0	\\
4.37e-09	0	\\
4.47e-09	0	\\
4.57e-09	0	\\
4.68e-09	0	\\
4.78e-09	0	\\
4.89e-09	0	\\
4.99e-09	0	\\
5e-09	0	\\
};
\addplot [color=mycolor2,solid,forget plot]
  table[row sep=crcr]{
0	0	\\
1.1e-10	0	\\
2.2e-10	0	\\
3.3e-10	0	\\
4.4e-10	0	\\
5.4e-10	0	\\
6.5e-10	0	\\
7.5e-10	0	\\
8.6e-10	0	\\
9.6e-10	0	\\
1.07e-09	0	\\
1.18e-09	0	\\
1.28e-09	0	\\
1.38e-09	0	\\
1.49e-09	0	\\
1.59e-09	0	\\
1.69e-09	0	\\
1.8e-09	0	\\
1.9e-09	0	\\
2.01e-09	0	\\
2.11e-09	0	\\
2.21e-09	0	\\
2.32e-09	0	\\
2.42e-09	0	\\
2.52e-09	0	\\
2.63e-09	0	\\
2.73e-09	0	\\
2.83e-09	0	\\
2.93e-09	0	\\
3.04e-09	0	\\
3.14e-09	0	\\
3.24e-09	0	\\
3.34e-09	0	\\
3.45e-09	0	\\
3.55e-09	0	\\
3.65e-09	0	\\
3.75e-09	0	\\
3.86e-09	0	\\
3.96e-09	0	\\
4.06e-09	0	\\
4.16e-09	0	\\
4.27e-09	0	\\
4.37e-09	0	\\
4.47e-09	0	\\
4.57e-09	0	\\
4.68e-09	0	\\
4.78e-09	0	\\
4.89e-09	0	\\
4.99e-09	0	\\
5e-09	0	\\
};
\addplot [color=mycolor3,solid,forget plot]
  table[row sep=crcr]{
0	0	\\
1.1e-10	0	\\
2.2e-10	0	\\
3.3e-10	0	\\
4.4e-10	0	\\
5.4e-10	0	\\
6.5e-10	0	\\
7.5e-10	0	\\
8.6e-10	0	\\
9.6e-10	0	\\
1.07e-09	0	\\
1.18e-09	0	\\
1.28e-09	0	\\
1.38e-09	0	\\
1.49e-09	0	\\
1.59e-09	0	\\
1.69e-09	0	\\
1.8e-09	0	\\
1.9e-09	0	\\
2.01e-09	0	\\
2.11e-09	0	\\
2.21e-09	0	\\
2.32e-09	0	\\
2.42e-09	0	\\
2.52e-09	0	\\
2.63e-09	0	\\
2.73e-09	0	\\
2.83e-09	0	\\
2.93e-09	0	\\
3.04e-09	0	\\
3.14e-09	0	\\
3.24e-09	0	\\
3.34e-09	0	\\
3.45e-09	0	\\
3.55e-09	0	\\
3.65e-09	0	\\
3.75e-09	0	\\
3.86e-09	0	\\
3.96e-09	0	\\
4.06e-09	0	\\
4.16e-09	0	\\
4.27e-09	0	\\
4.37e-09	0	\\
4.47e-09	0	\\
4.57e-09	0	\\
4.68e-09	0	\\
4.78e-09	0	\\
4.89e-09	0	\\
4.99e-09	0	\\
5e-09	0	\\
};
\addplot [color=darkgray,solid,forget plot]
  table[row sep=crcr]{
0	0	\\
1.1e-10	0	\\
2.2e-10	0	\\
3.3e-10	0	\\
4.4e-10	0	\\
5.4e-10	0	\\
6.5e-10	0	\\
7.5e-10	0	\\
8.6e-10	0	\\
9.6e-10	0	\\
1.07e-09	0	\\
1.18e-09	0	\\
1.28e-09	0	\\
1.38e-09	0	\\
1.49e-09	0	\\
1.59e-09	0	\\
1.69e-09	0	\\
1.8e-09	0	\\
1.9e-09	0	\\
2.01e-09	0	\\
2.11e-09	0	\\
2.21e-09	0	\\
2.32e-09	0	\\
2.42e-09	0	\\
2.52e-09	0	\\
2.63e-09	0	\\
2.73e-09	0	\\
2.83e-09	0	\\
2.93e-09	0	\\
3.04e-09	0	\\
3.14e-09	0	\\
3.24e-09	0	\\
3.34e-09	0	\\
3.45e-09	0	\\
3.55e-09	0	\\
3.65e-09	0	\\
3.75e-09	0	\\
3.86e-09	0	\\
3.96e-09	0	\\
4.06e-09	0	\\
4.16e-09	0	\\
4.27e-09	0	\\
4.37e-09	0	\\
4.47e-09	0	\\
4.57e-09	0	\\
4.68e-09	0	\\
4.78e-09	0	\\
4.89e-09	0	\\
4.99e-09	0	\\
5e-09	0	\\
};
\addplot [color=blue,solid,forget plot]
  table[row sep=crcr]{
0	0	\\
1.1e-10	0	\\
2.2e-10	0	\\
3.3e-10	0	\\
4.4e-10	0	\\
5.4e-10	0	\\
6.5e-10	0	\\
7.5e-10	0	\\
8.6e-10	0	\\
9.6e-10	0	\\
1.07e-09	0	\\
1.18e-09	0	\\
1.28e-09	0	\\
1.38e-09	0	\\
1.49e-09	0	\\
1.59e-09	0	\\
1.69e-09	0	\\
1.8e-09	0	\\
1.9e-09	0	\\
2.01e-09	0	\\
2.11e-09	0	\\
2.21e-09	0	\\
2.32e-09	0	\\
2.42e-09	0	\\
2.52e-09	0	\\
2.63e-09	0	\\
2.73e-09	0	\\
2.83e-09	0	\\
2.93e-09	0	\\
3.04e-09	0	\\
3.14e-09	0	\\
3.24e-09	0	\\
3.34e-09	0	\\
3.45e-09	0	\\
3.55e-09	0	\\
3.65e-09	0	\\
3.75e-09	0	\\
3.86e-09	0	\\
3.96e-09	0	\\
4.06e-09	0	\\
4.16e-09	0	\\
4.27e-09	0	\\
4.37e-09	0	\\
4.47e-09	0	\\
4.57e-09	0	\\
4.68e-09	0	\\
4.78e-09	0	\\
4.89e-09	0	\\
4.99e-09	0	\\
5e-09	0	\\
};
\addplot [color=black!50!green,solid,forget plot]
  table[row sep=crcr]{
0	0	\\
1.1e-10	0	\\
2.2e-10	0	\\
3.3e-10	0	\\
4.4e-10	0	\\
5.4e-10	0	\\
6.5e-10	0	\\
7.5e-10	0	\\
8.6e-10	0	\\
9.6e-10	0	\\
1.07e-09	0	\\
1.18e-09	0	\\
1.28e-09	0	\\
1.38e-09	0	\\
1.49e-09	0	\\
1.59e-09	0	\\
1.69e-09	0	\\
1.8e-09	0	\\
1.9e-09	0	\\
2.01e-09	0	\\
2.11e-09	0	\\
2.21e-09	0	\\
2.32e-09	0	\\
2.42e-09	0	\\
2.52e-09	0	\\
2.63e-09	0	\\
2.73e-09	0	\\
2.83e-09	0	\\
2.93e-09	0	\\
3.04e-09	0	\\
3.14e-09	0	\\
3.24e-09	0	\\
3.34e-09	0	\\
3.45e-09	0	\\
3.55e-09	0	\\
3.65e-09	0	\\
3.75e-09	0	\\
3.86e-09	0	\\
3.96e-09	0	\\
4.06e-09	0	\\
4.16e-09	0	\\
4.27e-09	0	\\
4.37e-09	0	\\
4.47e-09	0	\\
4.57e-09	0	\\
4.68e-09	0	\\
4.78e-09	0	\\
4.89e-09	0	\\
4.99e-09	0	\\
5e-09	0	\\
};
\addplot [color=red,solid,forget plot]
  table[row sep=crcr]{
0	0	\\
1.1e-10	0	\\
2.2e-10	0	\\
3.3e-10	0	\\
4.4e-10	0	\\
5.4e-10	0	\\
6.5e-10	0	\\
7.5e-10	0	\\
8.6e-10	0	\\
9.6e-10	0	\\
1.07e-09	0	\\
1.18e-09	0	\\
1.28e-09	0	\\
1.38e-09	0	\\
1.49e-09	0	\\
1.59e-09	0	\\
1.69e-09	0	\\
1.8e-09	0	\\
1.9e-09	0	\\
2.01e-09	0	\\
2.11e-09	0	\\
2.21e-09	0	\\
2.32e-09	0	\\
2.42e-09	0	\\
2.52e-09	0	\\
2.63e-09	0	\\
2.73e-09	0	\\
2.83e-09	0	\\
2.93e-09	0	\\
3.04e-09	0	\\
3.14e-09	0	\\
3.24e-09	0	\\
3.34e-09	0	\\
3.45e-09	0	\\
3.55e-09	0	\\
3.65e-09	0	\\
3.75e-09	0	\\
3.86e-09	0	\\
3.96e-09	0	\\
4.06e-09	0	\\
4.16e-09	0	\\
4.27e-09	0	\\
4.37e-09	0	\\
4.47e-09	0	\\
4.57e-09	0	\\
4.68e-09	0	\\
4.78e-09	0	\\
4.89e-09	0	\\
4.99e-09	0	\\
5e-09	0	\\
};
\addplot [color=mycolor1,solid,forget plot]
  table[row sep=crcr]{
0	0	\\
1.1e-10	0	\\
2.2e-10	0	\\
3.3e-10	0	\\
4.4e-10	0	\\
5.4e-10	0	\\
6.5e-10	0	\\
7.5e-10	0	\\
8.6e-10	0	\\
9.6e-10	0	\\
1.07e-09	0	\\
1.18e-09	0	\\
1.28e-09	0	\\
1.38e-09	0	\\
1.49e-09	0	\\
1.59e-09	0	\\
1.69e-09	0	\\
1.8e-09	0	\\
1.9e-09	0	\\
2.01e-09	0	\\
2.11e-09	0	\\
2.21e-09	0	\\
2.32e-09	0	\\
2.42e-09	0	\\
2.52e-09	0	\\
2.63e-09	0	\\
2.73e-09	0	\\
2.83e-09	0	\\
2.93e-09	0	\\
3.04e-09	0	\\
3.14e-09	0	\\
3.24e-09	0	\\
3.34e-09	0	\\
3.45e-09	0	\\
3.55e-09	0	\\
3.65e-09	0	\\
3.75e-09	0	\\
3.86e-09	0	\\
3.96e-09	0	\\
4.06e-09	0	\\
4.16e-09	0	\\
4.27e-09	0	\\
4.37e-09	0	\\
4.47e-09	0	\\
4.57e-09	0	\\
4.68e-09	0	\\
4.78e-09	0	\\
4.89e-09	0	\\
4.99e-09	0	\\
5e-09	0	\\
};
\addplot [color=mycolor2,solid,forget plot]
  table[row sep=crcr]{
0	0	\\
1.1e-10	0	\\
2.2e-10	0	\\
3.3e-10	0	\\
4.4e-10	0	\\
5.4e-10	0	\\
6.5e-10	0	\\
7.5e-10	0	\\
8.6e-10	0	\\
9.6e-10	0	\\
1.07e-09	0	\\
1.18e-09	0	\\
1.28e-09	0	\\
1.38e-09	0	\\
1.49e-09	0	\\
1.59e-09	0	\\
1.69e-09	0	\\
1.8e-09	0	\\
1.9e-09	0	\\
2.01e-09	0	\\
2.11e-09	0	\\
2.21e-09	0	\\
2.32e-09	0	\\
2.42e-09	0	\\
2.52e-09	0	\\
2.63e-09	0	\\
2.73e-09	0	\\
2.83e-09	0	\\
2.93e-09	0	\\
3.04e-09	0	\\
3.14e-09	0	\\
3.24e-09	0	\\
3.34e-09	0	\\
3.45e-09	0	\\
3.55e-09	0	\\
3.65e-09	0	\\
3.75e-09	0	\\
3.86e-09	0	\\
3.96e-09	0	\\
4.06e-09	0	\\
4.16e-09	0	\\
4.27e-09	0	\\
4.37e-09	0	\\
4.47e-09	0	\\
4.57e-09	0	\\
4.68e-09	0	\\
4.78e-09	0	\\
4.89e-09	0	\\
4.99e-09	0	\\
5e-09	0	\\
};
\addplot [color=mycolor3,solid,forget plot]
  table[row sep=crcr]{
0	0	\\
1.1e-10	0	\\
2.2e-10	0	\\
3.3e-10	0	\\
4.4e-10	0	\\
5.4e-10	0	\\
6.5e-10	0	\\
7.5e-10	0	\\
8.6e-10	0	\\
9.6e-10	0	\\
1.07e-09	0	\\
1.18e-09	0	\\
1.28e-09	0	\\
1.38e-09	0	\\
1.49e-09	0	\\
1.59e-09	0	\\
1.69e-09	0	\\
1.8e-09	0	\\
1.9e-09	0	\\
2.01e-09	0	\\
2.11e-09	0	\\
2.21e-09	0	\\
2.32e-09	0	\\
2.42e-09	0	\\
2.52e-09	0	\\
2.63e-09	0	\\
2.73e-09	0	\\
2.83e-09	0	\\
2.93e-09	0	\\
3.04e-09	0	\\
3.14e-09	0	\\
3.24e-09	0	\\
3.34e-09	0	\\
3.45e-09	0	\\
3.55e-09	0	\\
3.65e-09	0	\\
3.75e-09	0	\\
3.86e-09	0	\\
3.96e-09	0	\\
4.06e-09	0	\\
4.16e-09	0	\\
4.27e-09	0	\\
4.37e-09	0	\\
4.47e-09	0	\\
4.57e-09	0	\\
4.68e-09	0	\\
4.78e-09	0	\\
4.89e-09	0	\\
4.99e-09	0	\\
5e-09	0	\\
};
\addplot [color=darkgray,solid,forget plot]
  table[row sep=crcr]{
0	0	\\
1.1e-10	0	\\
2.2e-10	0	\\
3.3e-10	0	\\
4.4e-10	0	\\
5.4e-10	0	\\
6.5e-10	0	\\
7.5e-10	0	\\
8.6e-10	0	\\
9.6e-10	0	\\
1.07e-09	0	\\
1.18e-09	0	\\
1.28e-09	0	\\
1.38e-09	0	\\
1.49e-09	0	\\
1.59e-09	0	\\
1.69e-09	0	\\
1.8e-09	0	\\
1.9e-09	0	\\
2.01e-09	0	\\
2.11e-09	0	\\
2.21e-09	0	\\
2.32e-09	0	\\
2.42e-09	0	\\
2.52e-09	0	\\
2.63e-09	0	\\
2.73e-09	0	\\
2.83e-09	0	\\
2.93e-09	0	\\
3.04e-09	0	\\
3.14e-09	0	\\
3.24e-09	0	\\
3.34e-09	0	\\
3.45e-09	0	\\
3.55e-09	0	\\
3.65e-09	0	\\
3.75e-09	0	\\
3.86e-09	0	\\
3.96e-09	0	\\
4.06e-09	0	\\
4.16e-09	0	\\
4.27e-09	0	\\
4.37e-09	0	\\
4.47e-09	0	\\
4.57e-09	0	\\
4.68e-09	0	\\
4.78e-09	0	\\
4.89e-09	0	\\
4.99e-09	0	\\
5e-09	0	\\
};
\addplot [color=blue,solid,forget plot]
  table[row sep=crcr]{
0	0	\\
1.1e-10	0	\\
2.2e-10	0	\\
3.3e-10	0	\\
4.4e-10	0	\\
5.4e-10	0	\\
6.5e-10	0	\\
7.5e-10	0	\\
8.6e-10	0	\\
9.6e-10	0	\\
1.07e-09	0	\\
1.18e-09	0	\\
1.28e-09	0	\\
1.38e-09	0	\\
1.49e-09	0	\\
1.59e-09	0	\\
1.69e-09	0	\\
1.8e-09	0	\\
1.9e-09	0	\\
2.01e-09	0	\\
2.11e-09	0	\\
2.21e-09	0	\\
2.32e-09	0	\\
2.42e-09	0	\\
2.52e-09	0	\\
2.63e-09	0	\\
2.73e-09	0	\\
2.83e-09	0	\\
2.93e-09	0	\\
3.04e-09	0	\\
3.14e-09	0	\\
3.24e-09	0	\\
3.34e-09	0	\\
3.45e-09	0	\\
3.55e-09	0	\\
3.65e-09	0	\\
3.75e-09	0	\\
3.86e-09	0	\\
3.96e-09	0	\\
4.06e-09	0	\\
4.16e-09	0	\\
4.27e-09	0	\\
4.37e-09	0	\\
4.47e-09	0	\\
4.57e-09	0	\\
4.68e-09	0	\\
4.78e-09	0	\\
4.89e-09	0	\\
4.99e-09	0	\\
5e-09	0	\\
};
\addplot [color=black!50!green,solid,forget plot]
  table[row sep=crcr]{
0	0	\\
1.1e-10	0	\\
2.2e-10	0	\\
3.3e-10	0	\\
4.4e-10	0	\\
5.4e-10	0	\\
6.5e-10	0	\\
7.5e-10	0	\\
8.6e-10	0	\\
9.6e-10	0	\\
1.07e-09	0	\\
1.18e-09	0	\\
1.28e-09	0	\\
1.38e-09	0	\\
1.49e-09	0	\\
1.59e-09	0	\\
1.69e-09	0	\\
1.8e-09	0	\\
1.9e-09	0	\\
2.01e-09	0	\\
2.11e-09	0	\\
2.21e-09	0	\\
2.32e-09	0	\\
2.42e-09	0	\\
2.52e-09	0	\\
2.63e-09	0	\\
2.73e-09	0	\\
2.83e-09	0	\\
2.93e-09	0	\\
3.04e-09	0	\\
3.14e-09	0	\\
3.24e-09	0	\\
3.34e-09	0	\\
3.45e-09	0	\\
3.55e-09	0	\\
3.65e-09	0	\\
3.75e-09	0	\\
3.86e-09	0	\\
3.96e-09	0	\\
4.06e-09	0	\\
4.16e-09	0	\\
4.27e-09	0	\\
4.37e-09	0	\\
4.47e-09	0	\\
4.57e-09	0	\\
4.68e-09	0	\\
4.78e-09	0	\\
4.89e-09	0	\\
4.99e-09	0	\\
5e-09	0	\\
};
\addplot [color=red,solid,forget plot]
  table[row sep=crcr]{
0	0	\\
1.1e-10	0	\\
2.2e-10	0	\\
3.3e-10	0	\\
4.4e-10	0	\\
5.4e-10	0	\\
6.5e-10	0	\\
7.5e-10	0	\\
8.6e-10	0	\\
9.6e-10	0	\\
1.07e-09	0	\\
1.18e-09	0	\\
1.28e-09	0	\\
1.38e-09	0	\\
1.49e-09	0	\\
1.59e-09	0	\\
1.69e-09	0	\\
1.8e-09	0	\\
1.9e-09	0	\\
2.01e-09	0	\\
2.11e-09	0	\\
2.21e-09	0	\\
2.32e-09	0	\\
2.42e-09	0	\\
2.52e-09	0	\\
2.63e-09	0	\\
2.73e-09	0	\\
2.83e-09	0	\\
2.93e-09	0	\\
3.04e-09	0	\\
3.14e-09	0	\\
3.24e-09	0	\\
3.34e-09	0	\\
3.45e-09	0	\\
3.55e-09	0	\\
3.65e-09	0	\\
3.75e-09	0	\\
3.86e-09	0	\\
3.96e-09	0	\\
4.06e-09	0	\\
4.16e-09	0	\\
4.27e-09	0	\\
4.37e-09	0	\\
4.47e-09	0	\\
4.57e-09	0	\\
4.68e-09	0	\\
4.78e-09	0	\\
4.89e-09	0	\\
4.99e-09	0	\\
5e-09	0	\\
};
\addplot [color=mycolor1,solid,forget plot]
  table[row sep=crcr]{
0	0	\\
1.1e-10	0	\\
2.2e-10	0	\\
3.3e-10	0	\\
4.4e-10	0	\\
5.4e-10	0	\\
6.5e-10	0	\\
7.5e-10	0	\\
8.6e-10	0	\\
9.6e-10	0	\\
1.07e-09	0	\\
1.18e-09	0	\\
1.28e-09	0	\\
1.38e-09	0	\\
1.49e-09	0	\\
1.59e-09	0	\\
1.69e-09	0	\\
1.8e-09	0	\\
1.9e-09	0	\\
2.01e-09	0	\\
2.11e-09	0	\\
2.21e-09	0	\\
2.32e-09	0	\\
2.42e-09	0	\\
2.52e-09	0	\\
2.63e-09	0	\\
2.73e-09	0	\\
2.83e-09	0	\\
2.93e-09	0	\\
3.04e-09	0	\\
3.14e-09	0	\\
3.24e-09	0	\\
3.34e-09	0	\\
3.45e-09	0	\\
3.55e-09	0	\\
3.65e-09	0	\\
3.75e-09	0	\\
3.86e-09	0	\\
3.96e-09	0	\\
4.06e-09	0	\\
4.16e-09	0	\\
4.27e-09	0	\\
4.37e-09	0	\\
4.47e-09	0	\\
4.57e-09	0	\\
4.68e-09	0	\\
4.78e-09	0	\\
4.89e-09	0	\\
4.99e-09	0	\\
5e-09	0	\\
};
\addplot [color=mycolor2,solid,forget plot]
  table[row sep=crcr]{
0	0	\\
1.1e-10	0	\\
2.2e-10	0	\\
3.3e-10	0	\\
4.4e-10	0	\\
5.4e-10	0	\\
6.5e-10	0	\\
7.5e-10	0	\\
8.6e-10	0	\\
9.6e-10	0	\\
1.07e-09	0	\\
1.18e-09	0	\\
1.28e-09	0	\\
1.38e-09	0	\\
1.49e-09	0	\\
1.59e-09	0	\\
1.69e-09	0	\\
1.8e-09	0	\\
1.9e-09	0	\\
2.01e-09	0	\\
2.11e-09	0	\\
2.21e-09	0	\\
2.32e-09	0	\\
2.42e-09	0	\\
2.52e-09	0	\\
2.63e-09	0	\\
2.73e-09	0	\\
2.83e-09	0	\\
2.93e-09	0	\\
3.04e-09	0	\\
3.14e-09	0	\\
3.24e-09	0	\\
3.34e-09	0	\\
3.45e-09	0	\\
3.55e-09	0	\\
3.65e-09	0	\\
3.75e-09	0	\\
3.86e-09	0	\\
3.96e-09	0	\\
4.06e-09	0	\\
4.16e-09	0	\\
4.27e-09	0	\\
4.37e-09	0	\\
4.47e-09	0	\\
4.57e-09	0	\\
4.68e-09	0	\\
4.78e-09	0	\\
4.89e-09	0	\\
4.99e-09	0	\\
5e-09	0	\\
};
\addplot [color=mycolor3,solid,forget plot]
  table[row sep=crcr]{
0	0	\\
1.1e-10	0	\\
2.2e-10	0	\\
3.3e-10	0	\\
4.4e-10	0	\\
5.4e-10	0	\\
6.5e-10	0	\\
7.5e-10	0	\\
8.6e-10	0	\\
9.6e-10	0	\\
1.07e-09	0	\\
1.18e-09	0	\\
1.28e-09	0	\\
1.38e-09	0	\\
1.49e-09	0	\\
1.59e-09	0	\\
1.69e-09	0	\\
1.8e-09	0	\\
1.9e-09	0	\\
2.01e-09	0	\\
2.11e-09	0	\\
2.21e-09	0	\\
2.32e-09	0	\\
2.42e-09	0	\\
2.52e-09	0	\\
2.63e-09	0	\\
2.73e-09	0	\\
2.83e-09	0	\\
2.93e-09	0	\\
3.04e-09	0	\\
3.14e-09	0	\\
3.24e-09	0	\\
3.34e-09	0	\\
3.45e-09	0	\\
3.55e-09	0	\\
3.65e-09	0	\\
3.75e-09	0	\\
3.86e-09	0	\\
3.96e-09	0	\\
4.06e-09	0	\\
4.16e-09	0	\\
4.27e-09	0	\\
4.37e-09	0	\\
4.47e-09	0	\\
4.57e-09	0	\\
4.68e-09	0	\\
4.78e-09	0	\\
4.89e-09	0	\\
4.99e-09	0	\\
5e-09	0	\\
};
\addplot [color=darkgray,solid,forget plot]
  table[row sep=crcr]{
0	0	\\
1.1e-10	0	\\
2.2e-10	0	\\
3.3e-10	0	\\
4.4e-10	0	\\
5.4e-10	0	\\
6.5e-10	0	\\
7.5e-10	0	\\
8.6e-10	0	\\
9.6e-10	0	\\
1.07e-09	0	\\
1.18e-09	0	\\
1.28e-09	0	\\
1.38e-09	0	\\
1.49e-09	0	\\
1.59e-09	0	\\
1.69e-09	0	\\
1.8e-09	0	\\
1.9e-09	0	\\
2.01e-09	0	\\
2.11e-09	0	\\
2.21e-09	0	\\
2.32e-09	0	\\
2.42e-09	0	\\
2.52e-09	0	\\
2.63e-09	0	\\
2.73e-09	0	\\
2.83e-09	0	\\
2.93e-09	0	\\
3.04e-09	0	\\
3.14e-09	0	\\
3.24e-09	0	\\
3.34e-09	0	\\
3.45e-09	0	\\
3.55e-09	0	\\
3.65e-09	0	\\
3.75e-09	0	\\
3.86e-09	0	\\
3.96e-09	0	\\
4.06e-09	0	\\
4.16e-09	0	\\
4.27e-09	0	\\
4.37e-09	0	\\
4.47e-09	0	\\
4.57e-09	0	\\
4.68e-09	0	\\
4.78e-09	0	\\
4.89e-09	0	\\
4.99e-09	0	\\
5e-09	0	\\
};
\addplot [color=blue,solid,forget plot]
  table[row sep=crcr]{
0	0	\\
1.1e-10	0	\\
2.2e-10	0	\\
3.3e-10	0	\\
4.4e-10	0	\\
5.4e-10	0	\\
6.5e-10	0	\\
7.5e-10	0	\\
8.6e-10	0	\\
9.6e-10	0	\\
1.07e-09	0	\\
1.18e-09	0	\\
1.28e-09	0	\\
1.38e-09	0	\\
1.49e-09	0	\\
1.59e-09	0	\\
1.69e-09	0	\\
1.8e-09	0	\\
1.9e-09	0	\\
2.01e-09	0	\\
2.11e-09	0	\\
2.21e-09	0	\\
2.32e-09	0	\\
2.42e-09	0	\\
2.52e-09	0	\\
2.63e-09	0	\\
2.73e-09	0	\\
2.83e-09	0	\\
2.93e-09	0	\\
3.04e-09	0	\\
3.14e-09	0	\\
3.24e-09	0	\\
3.34e-09	0	\\
3.45e-09	0	\\
3.55e-09	0	\\
3.65e-09	0	\\
3.75e-09	0	\\
3.86e-09	0	\\
3.96e-09	0	\\
4.06e-09	0	\\
4.16e-09	0	\\
4.27e-09	0	\\
4.37e-09	0	\\
4.47e-09	0	\\
4.57e-09	0	\\
4.68e-09	0	\\
4.78e-09	0	\\
4.89e-09	0	\\
4.99e-09	0	\\
5e-09	0	\\
};
\addplot [color=black!50!green,solid,forget plot]
  table[row sep=crcr]{
0	0	\\
1.1e-10	0	\\
2.2e-10	0	\\
3.3e-10	0	\\
4.4e-10	0	\\
5.4e-10	0	\\
6.5e-10	0	\\
7.5e-10	0	\\
8.6e-10	0	\\
9.6e-10	0	\\
1.07e-09	0	\\
1.18e-09	0	\\
1.28e-09	0	\\
1.38e-09	0	\\
1.49e-09	0	\\
1.59e-09	0	\\
1.69e-09	0	\\
1.8e-09	0	\\
1.9e-09	0	\\
2.01e-09	0	\\
2.11e-09	0	\\
2.21e-09	0	\\
2.32e-09	0	\\
2.42e-09	0	\\
2.52e-09	0	\\
2.63e-09	0	\\
2.73e-09	0	\\
2.83e-09	0	\\
2.93e-09	0	\\
3.04e-09	0	\\
3.14e-09	0	\\
3.24e-09	0	\\
3.34e-09	0	\\
3.45e-09	0	\\
3.55e-09	0	\\
3.65e-09	0	\\
3.75e-09	0	\\
3.86e-09	0	\\
3.96e-09	0	\\
4.06e-09	0	\\
4.16e-09	0	\\
4.27e-09	0	\\
4.37e-09	0	\\
4.47e-09	0	\\
4.57e-09	0	\\
4.68e-09	0	\\
4.78e-09	0	\\
4.89e-09	0	\\
4.99e-09	0	\\
5e-09	0	\\
};
\addplot [color=red,solid,forget plot]
  table[row sep=crcr]{
0	0	\\
1.1e-10	0	\\
2.2e-10	0	\\
3.3e-10	0	\\
4.4e-10	0	\\
5.4e-10	0	\\
6.5e-10	0	\\
7.5e-10	0	\\
8.6e-10	0	\\
9.6e-10	0	\\
1.07e-09	0	\\
1.18e-09	0	\\
1.28e-09	0	\\
1.38e-09	0	\\
1.49e-09	0	\\
1.59e-09	0	\\
1.69e-09	0	\\
1.8e-09	0	\\
1.9e-09	0	\\
2.01e-09	0	\\
2.11e-09	0	\\
2.21e-09	0	\\
2.32e-09	0	\\
2.42e-09	0	\\
2.52e-09	0	\\
2.63e-09	0	\\
2.73e-09	0	\\
2.83e-09	0	\\
2.93e-09	0	\\
3.04e-09	0	\\
3.14e-09	0	\\
3.24e-09	0	\\
3.34e-09	0	\\
3.45e-09	0	\\
3.55e-09	0	\\
3.65e-09	0	\\
3.75e-09	0	\\
3.86e-09	0	\\
3.96e-09	0	\\
4.06e-09	0	\\
4.16e-09	0	\\
4.27e-09	0	\\
4.37e-09	0	\\
4.47e-09	0	\\
4.57e-09	0	\\
4.68e-09	0	\\
4.78e-09	0	\\
4.89e-09	0	\\
4.99e-09	0	\\
5e-09	0	\\
};
\addplot [color=mycolor1,solid,forget plot]
  table[row sep=crcr]{
0	0	\\
1.1e-10	0	\\
2.2e-10	0	\\
3.3e-10	0	\\
4.4e-10	0	\\
5.4e-10	0	\\
6.5e-10	0	\\
7.5e-10	0	\\
8.6e-10	0	\\
9.6e-10	0	\\
1.07e-09	0	\\
1.18e-09	0	\\
1.28e-09	0	\\
1.38e-09	0	\\
1.49e-09	0	\\
1.59e-09	0	\\
1.69e-09	0	\\
1.8e-09	0	\\
1.9e-09	0	\\
2.01e-09	0	\\
2.11e-09	0	\\
2.21e-09	0	\\
2.32e-09	0	\\
2.42e-09	0	\\
2.52e-09	0	\\
2.63e-09	0	\\
2.73e-09	0	\\
2.83e-09	0	\\
2.93e-09	0	\\
3.04e-09	0	\\
3.14e-09	0	\\
3.24e-09	0	\\
3.34e-09	0	\\
3.45e-09	0	\\
3.55e-09	0	\\
3.65e-09	0	\\
3.75e-09	0	\\
3.86e-09	0	\\
3.96e-09	0	\\
4.06e-09	0	\\
4.16e-09	0	\\
4.27e-09	0	\\
4.37e-09	0	\\
4.47e-09	0	\\
4.57e-09	0	\\
4.68e-09	0	\\
4.78e-09	0	\\
4.89e-09	0	\\
4.99e-09	0	\\
5e-09	0	\\
};
\addplot [color=mycolor2,solid,forget plot]
  table[row sep=crcr]{
0	0	\\
1.1e-10	0	\\
2.2e-10	0	\\
3.3e-10	0	\\
4.4e-10	0	\\
5.4e-10	0	\\
6.5e-10	0	\\
7.5e-10	0	\\
8.6e-10	0	\\
9.6e-10	0	\\
1.07e-09	0	\\
1.18e-09	0	\\
1.28e-09	0	\\
1.38e-09	0	\\
1.49e-09	0	\\
1.59e-09	0	\\
1.69e-09	0	\\
1.8e-09	0	\\
1.9e-09	0	\\
2.01e-09	0	\\
2.11e-09	0	\\
2.21e-09	0	\\
2.32e-09	0	\\
2.42e-09	0	\\
2.52e-09	0	\\
2.63e-09	0	\\
2.73e-09	0	\\
2.83e-09	0	\\
2.93e-09	0	\\
3.04e-09	0	\\
3.14e-09	0	\\
3.24e-09	0	\\
3.34e-09	0	\\
3.45e-09	0	\\
3.55e-09	0	\\
3.65e-09	0	\\
3.75e-09	0	\\
3.86e-09	0	\\
3.96e-09	0	\\
4.06e-09	0	\\
4.16e-09	0	\\
4.27e-09	0	\\
4.37e-09	0	\\
4.47e-09	0	\\
4.57e-09	0	\\
4.68e-09	0	\\
4.78e-09	0	\\
4.89e-09	0	\\
4.99e-09	0	\\
5e-09	0	\\
};
\addplot [color=mycolor3,solid,forget plot]
  table[row sep=crcr]{
0	0	\\
1.1e-10	0	\\
2.2e-10	0	\\
3.3e-10	0	\\
4.4e-10	0	\\
5.4e-10	0	\\
6.5e-10	0	\\
7.5e-10	0	\\
8.6e-10	0	\\
9.6e-10	0	\\
1.07e-09	0	\\
1.18e-09	0	\\
1.28e-09	0	\\
1.38e-09	0	\\
1.49e-09	0	\\
1.59e-09	0	\\
1.69e-09	0	\\
1.8e-09	0	\\
1.9e-09	0	\\
2.01e-09	0	\\
2.11e-09	0	\\
2.21e-09	0	\\
2.32e-09	0	\\
2.42e-09	0	\\
2.52e-09	0	\\
2.63e-09	0	\\
2.73e-09	0	\\
2.83e-09	0	\\
2.93e-09	0	\\
3.04e-09	0	\\
3.14e-09	0	\\
3.24e-09	0	\\
3.34e-09	0	\\
3.45e-09	0	\\
3.55e-09	0	\\
3.65e-09	0	\\
3.75e-09	0	\\
3.86e-09	0	\\
3.96e-09	0	\\
4.06e-09	0	\\
4.16e-09	0	\\
4.27e-09	0	\\
4.37e-09	0	\\
4.47e-09	0	\\
4.57e-09	0	\\
4.68e-09	0	\\
4.78e-09	0	\\
4.89e-09	0	\\
4.99e-09	0	\\
5e-09	0	\\
};
\addplot [color=darkgray,solid,forget plot]
  table[row sep=crcr]{
0	0	\\
1.1e-10	0	\\
2.2e-10	0	\\
3.3e-10	0	\\
4.4e-10	0	\\
5.4e-10	0	\\
6.5e-10	0	\\
7.5e-10	0	\\
8.6e-10	0	\\
9.6e-10	0	\\
1.07e-09	0	\\
1.18e-09	0	\\
1.28e-09	0	\\
1.38e-09	0	\\
1.49e-09	0	\\
1.59e-09	0	\\
1.69e-09	0	\\
1.8e-09	0	\\
1.9e-09	0	\\
2.01e-09	0	\\
2.11e-09	0	\\
2.21e-09	0	\\
2.32e-09	0	\\
2.42e-09	0	\\
2.52e-09	0	\\
2.63e-09	0	\\
2.73e-09	0	\\
2.83e-09	0	\\
2.93e-09	0	\\
3.04e-09	0	\\
3.14e-09	0	\\
3.24e-09	0	\\
3.34e-09	0	\\
3.45e-09	0	\\
3.55e-09	0	\\
3.65e-09	0	\\
3.75e-09	0	\\
3.86e-09	0	\\
3.96e-09	0	\\
4.06e-09	0	\\
4.16e-09	0	\\
4.27e-09	0	\\
4.37e-09	0	\\
4.47e-09	0	\\
4.57e-09	0	\\
4.68e-09	0	\\
4.78e-09	0	\\
4.89e-09	0	\\
4.99e-09	0	\\
5e-09	0	\\
};
\addplot [color=blue,solid,forget plot]
  table[row sep=crcr]{
0	0	\\
1.1e-10	0	\\
2.2e-10	0	\\
3.3e-10	0	\\
4.4e-10	0	\\
5.4e-10	0	\\
6.5e-10	0	\\
7.5e-10	0	\\
8.6e-10	0	\\
9.6e-10	0	\\
1.07e-09	0	\\
1.18e-09	0	\\
1.28e-09	0	\\
1.38e-09	0	\\
1.49e-09	0	\\
1.59e-09	0	\\
1.69e-09	0	\\
1.8e-09	0	\\
1.9e-09	0	\\
2.01e-09	0	\\
2.11e-09	0	\\
2.21e-09	0	\\
2.32e-09	0	\\
2.42e-09	0	\\
2.52e-09	0	\\
2.63e-09	0	\\
2.73e-09	0	\\
2.83e-09	0	\\
2.93e-09	0	\\
3.04e-09	0	\\
3.14e-09	0	\\
3.24e-09	0	\\
3.34e-09	0	\\
3.45e-09	0	\\
3.55e-09	0	\\
3.65e-09	0	\\
3.75e-09	0	\\
3.86e-09	0	\\
3.96e-09	0	\\
4.06e-09	0	\\
4.16e-09	0	\\
4.27e-09	0	\\
4.37e-09	0	\\
4.47e-09	0	\\
4.57e-09	0	\\
4.68e-09	0	\\
4.78e-09	0	\\
4.89e-09	0	\\
4.99e-09	0	\\
5e-09	0	\\
};
\addplot [color=black!50!green,solid,forget plot]
  table[row sep=crcr]{
0	0	\\
1.1e-10	0	\\
2.2e-10	0	\\
3.3e-10	0	\\
4.4e-10	0	\\
5.4e-10	0	\\
6.5e-10	0	\\
7.5e-10	0	\\
8.6e-10	0	\\
9.6e-10	0	\\
1.07e-09	0	\\
1.18e-09	0	\\
1.28e-09	0	\\
1.38e-09	0	\\
1.49e-09	0	\\
1.59e-09	0	\\
1.69e-09	0	\\
1.8e-09	0	\\
1.9e-09	0	\\
2.01e-09	0	\\
2.11e-09	0	\\
2.21e-09	0	\\
2.32e-09	0	\\
2.42e-09	0	\\
2.52e-09	0	\\
2.63e-09	0	\\
2.73e-09	0	\\
2.83e-09	0	\\
2.93e-09	0	\\
3.04e-09	0	\\
3.14e-09	0	\\
3.24e-09	0	\\
3.34e-09	0	\\
3.45e-09	0	\\
3.55e-09	0	\\
3.65e-09	0	\\
3.75e-09	0	\\
3.86e-09	0	\\
3.96e-09	0	\\
4.06e-09	0	\\
4.16e-09	0	\\
4.27e-09	0	\\
4.37e-09	0	\\
4.47e-09	0	\\
4.57e-09	0	\\
4.68e-09	0	\\
4.78e-09	0	\\
4.89e-09	0	\\
4.99e-09	0	\\
5e-09	0	\\
};
\addplot [color=red,solid,forget plot]
  table[row sep=crcr]{
0	0	\\
1.1e-10	0	\\
2.2e-10	0	\\
3.3e-10	0	\\
4.4e-10	0	\\
5.4e-10	0	\\
6.5e-10	0	\\
7.5e-10	0	\\
8.6e-10	0	\\
9.6e-10	0	\\
1.07e-09	0	\\
1.18e-09	0	\\
1.28e-09	0	\\
1.38e-09	0	\\
1.49e-09	0	\\
1.59e-09	0	\\
1.69e-09	0	\\
1.8e-09	0	\\
1.9e-09	0	\\
2.01e-09	0	\\
2.11e-09	0	\\
2.21e-09	0	\\
2.32e-09	0	\\
2.42e-09	0	\\
2.52e-09	0	\\
2.63e-09	0	\\
2.73e-09	0	\\
2.83e-09	0	\\
2.93e-09	0	\\
3.04e-09	0	\\
3.14e-09	0	\\
3.24e-09	0	\\
3.34e-09	0	\\
3.45e-09	0	\\
3.55e-09	0	\\
3.65e-09	0	\\
3.75e-09	0	\\
3.86e-09	0	\\
3.96e-09	0	\\
4.06e-09	0	\\
4.16e-09	0	\\
4.27e-09	0	\\
4.37e-09	0	\\
4.47e-09	0	\\
4.57e-09	0	\\
4.68e-09	0	\\
4.78e-09	0	\\
4.89e-09	0	\\
4.99e-09	0	\\
5e-09	0	\\
};
\addplot [color=mycolor1,solid,forget plot]
  table[row sep=crcr]{
0	0	\\
1.1e-10	0	\\
2.2e-10	0	\\
3.3e-10	0	\\
4.4e-10	0	\\
5.4e-10	0	\\
6.5e-10	0	\\
7.5e-10	0	\\
8.6e-10	0	\\
9.6e-10	0	\\
1.07e-09	0	\\
1.18e-09	0	\\
1.28e-09	0	\\
1.38e-09	0	\\
1.49e-09	0	\\
1.59e-09	0	\\
1.69e-09	0	\\
1.8e-09	0	\\
1.9e-09	0	\\
2.01e-09	0	\\
2.11e-09	0	\\
2.21e-09	0	\\
2.32e-09	0	\\
2.42e-09	0	\\
2.52e-09	0	\\
2.63e-09	0	\\
2.73e-09	0	\\
2.83e-09	0	\\
2.93e-09	0	\\
3.04e-09	0	\\
3.14e-09	0	\\
3.24e-09	0	\\
3.34e-09	0	\\
3.45e-09	0	\\
3.55e-09	0	\\
3.65e-09	0	\\
3.75e-09	0	\\
3.86e-09	0	\\
3.96e-09	0	\\
4.06e-09	0	\\
4.16e-09	0	\\
4.27e-09	0	\\
4.37e-09	0	\\
4.47e-09	0	\\
4.57e-09	0	\\
4.68e-09	0	\\
4.78e-09	0	\\
4.89e-09	0	\\
4.99e-09	0	\\
5e-09	0	\\
};
\addplot [color=mycolor2,solid,forget plot]
  table[row sep=crcr]{
0	0	\\
1.1e-10	0	\\
2.2e-10	0	\\
3.3e-10	0	\\
4.4e-10	0	\\
5.4e-10	0	\\
6.5e-10	0	\\
7.5e-10	0	\\
8.6e-10	0	\\
9.6e-10	0	\\
1.07e-09	0	\\
1.18e-09	0	\\
1.28e-09	0	\\
1.38e-09	0	\\
1.49e-09	0	\\
1.59e-09	0	\\
1.69e-09	0	\\
1.8e-09	0	\\
1.9e-09	0	\\
2.01e-09	0	\\
2.11e-09	0	\\
2.21e-09	0	\\
2.32e-09	0	\\
2.42e-09	0	\\
2.52e-09	0	\\
2.63e-09	0	\\
2.73e-09	0	\\
2.83e-09	0	\\
2.93e-09	0	\\
3.04e-09	0	\\
3.14e-09	0	\\
3.24e-09	0	\\
3.34e-09	0	\\
3.45e-09	0	\\
3.55e-09	0	\\
3.65e-09	0	\\
3.75e-09	0	\\
3.86e-09	0	\\
3.96e-09	0	\\
4.06e-09	0	\\
4.16e-09	0	\\
4.27e-09	0	\\
4.37e-09	0	\\
4.47e-09	0	\\
4.57e-09	0	\\
4.68e-09	0	\\
4.78e-09	0	\\
4.89e-09	0	\\
4.99e-09	0	\\
5e-09	0	\\
};
\addplot [color=mycolor3,solid,forget plot]
  table[row sep=crcr]{
0	0	\\
1.1e-10	0	\\
2.2e-10	0	\\
3.3e-10	0	\\
4.4e-10	0	\\
5.4e-10	0	\\
6.5e-10	0	\\
7.5e-10	0	\\
8.6e-10	0	\\
9.6e-10	0	\\
1.07e-09	0	\\
1.18e-09	0	\\
1.28e-09	0	\\
1.38e-09	0	\\
1.49e-09	0	\\
1.59e-09	0	\\
1.69e-09	0	\\
1.8e-09	0	\\
1.9e-09	0	\\
2.01e-09	0	\\
2.11e-09	0	\\
2.21e-09	0	\\
2.32e-09	0	\\
2.42e-09	0	\\
2.52e-09	0	\\
2.63e-09	0	\\
2.73e-09	0	\\
2.83e-09	0	\\
2.93e-09	0	\\
3.04e-09	0	\\
3.14e-09	0	\\
3.24e-09	0	\\
3.34e-09	0	\\
3.45e-09	0	\\
3.55e-09	0	\\
3.65e-09	0	\\
3.75e-09	0	\\
3.86e-09	0	\\
3.96e-09	0	\\
4.06e-09	0	\\
4.16e-09	0	\\
4.27e-09	0	\\
4.37e-09	0	\\
4.47e-09	0	\\
4.57e-09	0	\\
4.68e-09	0	\\
4.78e-09	0	\\
4.89e-09	0	\\
4.99e-09	0	\\
5e-09	0	\\
};
\addplot [color=darkgray,solid,forget plot]
  table[row sep=crcr]{
0	0	\\
1.1e-10	0	\\
2.2e-10	0	\\
3.3e-10	0	\\
4.4e-10	0	\\
5.4e-10	0	\\
6.5e-10	0	\\
7.5e-10	0	\\
8.6e-10	0	\\
9.6e-10	0	\\
1.07e-09	0	\\
1.18e-09	0	\\
1.28e-09	0	\\
1.38e-09	0	\\
1.49e-09	0	\\
1.59e-09	0	\\
1.69e-09	0	\\
1.8e-09	0	\\
1.9e-09	0	\\
2.01e-09	0	\\
2.11e-09	0	\\
2.21e-09	0	\\
2.32e-09	0	\\
2.42e-09	0	\\
2.52e-09	0	\\
2.63e-09	0	\\
2.73e-09	0	\\
2.83e-09	0	\\
2.93e-09	0	\\
3.04e-09	0	\\
3.14e-09	0	\\
3.24e-09	0	\\
3.34e-09	0	\\
3.45e-09	0	\\
3.55e-09	0	\\
3.65e-09	0	\\
3.75e-09	0	\\
3.86e-09	0	\\
3.96e-09	0	\\
4.06e-09	0	\\
4.16e-09	0	\\
4.27e-09	0	\\
4.37e-09	0	\\
4.47e-09	0	\\
4.57e-09	0	\\
4.68e-09	0	\\
4.78e-09	0	\\
4.89e-09	0	\\
4.99e-09	0	\\
5e-09	0	\\
};
\addplot [color=blue,solid,forget plot]
  table[row sep=crcr]{
0	0	\\
1.1e-10	0	\\
2.2e-10	0	\\
3.3e-10	0	\\
4.4e-10	0	\\
5.4e-10	0	\\
6.5e-10	0	\\
7.5e-10	0	\\
8.6e-10	0	\\
9.6e-10	0	\\
1.07e-09	0	\\
1.18e-09	0	\\
1.28e-09	0	\\
1.38e-09	0	\\
1.49e-09	0	\\
1.59e-09	0	\\
1.69e-09	0	\\
1.8e-09	0	\\
1.9e-09	0	\\
2.01e-09	0	\\
2.11e-09	0	\\
2.21e-09	0	\\
2.32e-09	0	\\
2.42e-09	0	\\
2.52e-09	0	\\
2.63e-09	0	\\
2.73e-09	0	\\
2.83e-09	0	\\
2.93e-09	0	\\
3.04e-09	0	\\
3.14e-09	0	\\
3.24e-09	0	\\
3.34e-09	0	\\
3.45e-09	0	\\
3.55e-09	0	\\
3.65e-09	0	\\
3.75e-09	0	\\
3.86e-09	0	\\
3.96e-09	0	\\
4.06e-09	0	\\
4.16e-09	0	\\
4.27e-09	0	\\
4.37e-09	0	\\
4.47e-09	0	\\
4.57e-09	0	\\
4.68e-09	0	\\
4.78e-09	0	\\
4.89e-09	0	\\
4.99e-09	0	\\
5e-09	0	\\
};
\addplot [color=black!50!green,solid,forget plot]
  table[row sep=crcr]{
0	0	\\
1.1e-10	0	\\
2.2e-10	0	\\
3.3e-10	0	\\
4.4e-10	0	\\
5.4e-10	0	\\
6.5e-10	0	\\
7.5e-10	0	\\
8.6e-10	0	\\
9.6e-10	0	\\
1.07e-09	0	\\
1.18e-09	0	\\
1.28e-09	0	\\
1.38e-09	0	\\
1.49e-09	0	\\
1.59e-09	0	\\
1.69e-09	0	\\
1.8e-09	0	\\
1.9e-09	0	\\
2.01e-09	0	\\
2.11e-09	0	\\
2.21e-09	0	\\
2.32e-09	0	\\
2.42e-09	0	\\
2.52e-09	0	\\
2.63e-09	0	\\
2.73e-09	0	\\
2.83e-09	0	\\
2.93e-09	0	\\
3.04e-09	0	\\
3.14e-09	0	\\
3.24e-09	0	\\
3.34e-09	0	\\
3.45e-09	0	\\
3.55e-09	0	\\
3.65e-09	0	\\
3.75e-09	0	\\
3.86e-09	0	\\
3.96e-09	0	\\
4.06e-09	0	\\
4.16e-09	0	\\
4.27e-09	0	\\
4.37e-09	0	\\
4.47e-09	0	\\
4.57e-09	0	\\
4.68e-09	0	\\
4.78e-09	0	\\
4.89e-09	0	\\
4.99e-09	0	\\
5e-09	0	\\
};
\addplot [color=red,solid,forget plot]
  table[row sep=crcr]{
0	0	\\
1.1e-10	0	\\
2.2e-10	0	\\
3.3e-10	0	\\
4.4e-10	0	\\
5.4e-10	0	\\
6.5e-10	0	\\
7.5e-10	0	\\
8.6e-10	0	\\
9.6e-10	0	\\
1.07e-09	0	\\
1.18e-09	0	\\
1.28e-09	0	\\
1.38e-09	0	\\
1.49e-09	0	\\
1.59e-09	0	\\
1.69e-09	0	\\
1.8e-09	0	\\
1.9e-09	0	\\
2.01e-09	0	\\
2.11e-09	0	\\
2.21e-09	0	\\
2.32e-09	0	\\
2.42e-09	0	\\
2.52e-09	0	\\
2.63e-09	0	\\
2.73e-09	0	\\
2.83e-09	0	\\
2.93e-09	0	\\
3.04e-09	0	\\
3.14e-09	0	\\
3.24e-09	0	\\
3.34e-09	0	\\
3.45e-09	0	\\
3.55e-09	0	\\
3.65e-09	0	\\
3.75e-09	0	\\
3.86e-09	0	\\
3.96e-09	0	\\
4.06e-09	0	\\
4.16e-09	0	\\
4.27e-09	0	\\
4.37e-09	0	\\
4.47e-09	0	\\
4.57e-09	0	\\
4.68e-09	0	\\
4.78e-09	0	\\
4.89e-09	0	\\
4.99e-09	0	\\
5e-09	0	\\
};
\addplot [color=mycolor1,solid,forget plot]
  table[row sep=crcr]{
0	0	\\
1.1e-10	0	\\
2.2e-10	0	\\
3.3e-10	0	\\
4.4e-10	0	\\
5.4e-10	0	\\
6.5e-10	0	\\
7.5e-10	0	\\
8.6e-10	0	\\
9.6e-10	0	\\
1.07e-09	0	\\
1.18e-09	0	\\
1.28e-09	0	\\
1.38e-09	0	\\
1.49e-09	0	\\
1.59e-09	0	\\
1.69e-09	0	\\
1.8e-09	0	\\
1.9e-09	0	\\
2.01e-09	0	\\
2.11e-09	0	\\
2.21e-09	0	\\
2.32e-09	0	\\
2.42e-09	0	\\
2.52e-09	0	\\
2.63e-09	0	\\
2.73e-09	0	\\
2.83e-09	0	\\
2.93e-09	0	\\
3.04e-09	0	\\
3.14e-09	0	\\
3.24e-09	0	\\
3.34e-09	0	\\
3.45e-09	0	\\
3.55e-09	0	\\
3.65e-09	0	\\
3.75e-09	0	\\
3.86e-09	0	\\
3.96e-09	0	\\
4.06e-09	0	\\
4.16e-09	0	\\
4.27e-09	0	\\
4.37e-09	0	\\
4.47e-09	0	\\
4.57e-09	0	\\
4.68e-09	0	\\
4.78e-09	0	\\
4.89e-09	0	\\
4.99e-09	0	\\
5e-09	0	\\
};
\addplot [color=mycolor2,solid,forget plot]
  table[row sep=crcr]{
0	0	\\
1.1e-10	0	\\
2.2e-10	0	\\
3.3e-10	0	\\
4.4e-10	0	\\
5.4e-10	0	\\
6.5e-10	0	\\
7.5e-10	0	\\
8.6e-10	0	\\
9.6e-10	0	\\
1.07e-09	0	\\
1.18e-09	0	\\
1.28e-09	0	\\
1.38e-09	0	\\
1.49e-09	0	\\
1.59e-09	0	\\
1.69e-09	0	\\
1.8e-09	0	\\
1.9e-09	0	\\
2.01e-09	0	\\
2.11e-09	0	\\
2.21e-09	0	\\
2.32e-09	0	\\
2.42e-09	0	\\
2.52e-09	0	\\
2.63e-09	0	\\
2.73e-09	0	\\
2.83e-09	0	\\
2.93e-09	0	\\
3.04e-09	0	\\
3.14e-09	0	\\
3.24e-09	0	\\
3.34e-09	0	\\
3.45e-09	0	\\
3.55e-09	0	\\
3.65e-09	0	\\
3.75e-09	0	\\
3.86e-09	0	\\
3.96e-09	0	\\
4.06e-09	0	\\
4.16e-09	0	\\
4.27e-09	0	\\
4.37e-09	0	\\
4.47e-09	0	\\
4.57e-09	0	\\
4.68e-09	0	\\
4.78e-09	0	\\
4.89e-09	0	\\
4.99e-09	0	\\
5e-09	0	\\
};
\addplot [color=mycolor3,solid,forget plot]
  table[row sep=crcr]{
0	0	\\
1.1e-10	0	\\
2.2e-10	0	\\
3.3e-10	0	\\
4.4e-10	0	\\
5.4e-10	0	\\
6.5e-10	0	\\
7.5e-10	0	\\
8.6e-10	0	\\
9.6e-10	0	\\
1.07e-09	0	\\
1.18e-09	0	\\
1.28e-09	0	\\
1.38e-09	0	\\
1.49e-09	0	\\
1.59e-09	0	\\
1.69e-09	0	\\
1.8e-09	0	\\
1.9e-09	0	\\
2.01e-09	0	\\
2.11e-09	0	\\
2.21e-09	0	\\
2.32e-09	0	\\
2.42e-09	0	\\
2.52e-09	0	\\
2.63e-09	0	\\
2.73e-09	0	\\
2.83e-09	0	\\
2.93e-09	0	\\
3.04e-09	0	\\
3.14e-09	0	\\
3.24e-09	0	\\
3.34e-09	0	\\
3.45e-09	0	\\
3.55e-09	0	\\
3.65e-09	0	\\
3.75e-09	0	\\
3.86e-09	0	\\
3.96e-09	0	\\
4.06e-09	0	\\
4.16e-09	0	\\
4.27e-09	0	\\
4.37e-09	0	\\
4.47e-09	0	\\
4.57e-09	0	\\
4.68e-09	0	\\
4.78e-09	0	\\
4.89e-09	0	\\
4.99e-09	0	\\
5e-09	0	\\
};
\addplot [color=darkgray,solid,forget plot]
  table[row sep=crcr]{
0	0	\\
1.1e-10	0	\\
2.2e-10	0	\\
3.3e-10	0	\\
4.4e-10	0	\\
5.4e-10	0	\\
6.5e-10	0	\\
7.5e-10	0	\\
8.6e-10	0	\\
9.6e-10	0	\\
1.07e-09	0	\\
1.18e-09	0	\\
1.28e-09	0	\\
1.38e-09	0	\\
1.49e-09	0	\\
1.59e-09	0	\\
1.69e-09	0	\\
1.8e-09	0	\\
1.9e-09	0	\\
2.01e-09	0	\\
2.11e-09	0	\\
2.21e-09	0	\\
2.32e-09	0	\\
2.42e-09	0	\\
2.52e-09	0	\\
2.63e-09	0	\\
2.73e-09	0	\\
2.83e-09	0	\\
2.93e-09	0	\\
3.04e-09	0	\\
3.14e-09	0	\\
3.24e-09	0	\\
3.34e-09	0	\\
3.45e-09	0	\\
3.55e-09	0	\\
3.65e-09	0	\\
3.75e-09	0	\\
3.86e-09	0	\\
3.96e-09	0	\\
4.06e-09	0	\\
4.16e-09	0	\\
4.27e-09	0	\\
4.37e-09	0	\\
4.47e-09	0	\\
4.57e-09	0	\\
4.68e-09	0	\\
4.78e-09	0	\\
4.89e-09	0	\\
4.99e-09	0	\\
5e-09	0	\\
};
\addplot [color=blue,solid,forget plot]
  table[row sep=crcr]{
0	0	\\
1.1e-10	0	\\
2.2e-10	0	\\
3.3e-10	0	\\
4.4e-10	0	\\
5.4e-10	0	\\
6.5e-10	0	\\
7.5e-10	0	\\
8.6e-10	0	\\
9.6e-10	0	\\
1.07e-09	0	\\
1.18e-09	0	\\
1.28e-09	0	\\
1.38e-09	0	\\
1.49e-09	0	\\
1.59e-09	0	\\
1.69e-09	0	\\
1.8e-09	0	\\
1.9e-09	0	\\
2.01e-09	0	\\
2.11e-09	0	\\
2.21e-09	0	\\
2.32e-09	0	\\
2.42e-09	0	\\
2.52e-09	0	\\
2.63e-09	0	\\
2.73e-09	0	\\
2.83e-09	0	\\
2.93e-09	0	\\
3.04e-09	0	\\
3.14e-09	0	\\
3.24e-09	0	\\
3.34e-09	0	\\
3.45e-09	0	\\
3.55e-09	0	\\
3.65e-09	0	\\
3.75e-09	0	\\
3.86e-09	0	\\
3.96e-09	0	\\
4.06e-09	0	\\
4.16e-09	0	\\
4.27e-09	0	\\
4.37e-09	0	\\
4.47e-09	0	\\
4.57e-09	0	\\
4.68e-09	0	\\
4.78e-09	0	\\
4.89e-09	0	\\
4.99e-09	0	\\
5e-09	0	\\
};
\addplot [color=black!50!green,solid,forget plot]
  table[row sep=crcr]{
0	0	\\
1.1e-10	0	\\
2.2e-10	0	\\
3.3e-10	0	\\
4.4e-10	0	\\
5.4e-10	0	\\
6.5e-10	0	\\
7.5e-10	0	\\
8.6e-10	0	\\
9.6e-10	0	\\
1.07e-09	0	\\
1.18e-09	0	\\
1.28e-09	0	\\
1.38e-09	0	\\
1.49e-09	0	\\
1.59e-09	0	\\
1.69e-09	0	\\
1.8e-09	0	\\
1.9e-09	0	\\
2.01e-09	0	\\
2.11e-09	0	\\
2.21e-09	0	\\
2.32e-09	0	\\
2.42e-09	0	\\
2.52e-09	0	\\
2.63e-09	0	\\
2.73e-09	0	\\
2.83e-09	0	\\
2.93e-09	0	\\
3.04e-09	0	\\
3.14e-09	0	\\
3.24e-09	0	\\
3.34e-09	0	\\
3.45e-09	0	\\
3.55e-09	0	\\
3.65e-09	0	\\
3.75e-09	0	\\
3.86e-09	0	\\
3.96e-09	0	\\
4.06e-09	0	\\
4.16e-09	0	\\
4.27e-09	0	\\
4.37e-09	0	\\
4.47e-09	0	\\
4.57e-09	0	\\
4.68e-09	0	\\
4.78e-09	0	\\
4.89e-09	0	\\
4.99e-09	0	\\
5e-09	0	\\
};
\addplot [color=red,solid,forget plot]
  table[row sep=crcr]{
0	0	\\
1.1e-10	0	\\
2.2e-10	0	\\
3.3e-10	0	\\
4.4e-10	0	\\
5.4e-10	0	\\
6.5e-10	0	\\
7.5e-10	0	\\
8.6e-10	0	\\
9.6e-10	0	\\
1.07e-09	0	\\
1.18e-09	0	\\
1.28e-09	0	\\
1.38e-09	0	\\
1.49e-09	0	\\
1.59e-09	0	\\
1.69e-09	0	\\
1.8e-09	0	\\
1.9e-09	0	\\
2.01e-09	0	\\
2.11e-09	0	\\
2.21e-09	0	\\
2.32e-09	0	\\
2.42e-09	0	\\
2.52e-09	0	\\
2.63e-09	0	\\
2.73e-09	0	\\
2.83e-09	0	\\
2.93e-09	0	\\
3.04e-09	0	\\
3.14e-09	0	\\
3.24e-09	0	\\
3.34e-09	0	\\
3.45e-09	0	\\
3.55e-09	0	\\
3.65e-09	0	\\
3.75e-09	0	\\
3.86e-09	0	\\
3.96e-09	0	\\
4.06e-09	0	\\
4.16e-09	0	\\
4.27e-09	0	\\
4.37e-09	0	\\
4.47e-09	0	\\
4.57e-09	0	\\
4.68e-09	0	\\
4.78e-09	0	\\
4.89e-09	0	\\
4.99e-09	0	\\
5e-09	0	\\
};
\addplot [color=mycolor1,solid,forget plot]
  table[row sep=crcr]{
0	0	\\
1.1e-10	0	\\
2.2e-10	0	\\
3.3e-10	0	\\
4.4e-10	0	\\
5.4e-10	0	\\
6.5e-10	0	\\
7.5e-10	0	\\
8.6e-10	0	\\
9.6e-10	0	\\
1.07e-09	0	\\
1.18e-09	0	\\
1.28e-09	0	\\
1.38e-09	0	\\
1.49e-09	0	\\
1.59e-09	0	\\
1.69e-09	0	\\
1.8e-09	0	\\
1.9e-09	0	\\
2.01e-09	0	\\
2.11e-09	0	\\
2.21e-09	0	\\
2.32e-09	0	\\
2.42e-09	0	\\
2.52e-09	0	\\
2.63e-09	0	\\
2.73e-09	0	\\
2.83e-09	0	\\
2.93e-09	0	\\
3.04e-09	0	\\
3.14e-09	0	\\
3.24e-09	0	\\
3.34e-09	0	\\
3.45e-09	0	\\
3.55e-09	0	\\
3.65e-09	0	\\
3.75e-09	0	\\
3.86e-09	0	\\
3.96e-09	0	\\
4.06e-09	0	\\
4.16e-09	0	\\
4.27e-09	0	\\
4.37e-09	0	\\
4.47e-09	0	\\
4.57e-09	0	\\
4.68e-09	0	\\
4.78e-09	0	\\
4.89e-09	0	\\
4.99e-09	0	\\
5e-09	0	\\
};
\addplot [color=mycolor2,solid,forget plot]
  table[row sep=crcr]{
0	0	\\
1.1e-10	0	\\
2.2e-10	0	\\
3.3e-10	0	\\
4.4e-10	0	\\
5.4e-10	0	\\
6.5e-10	0	\\
7.5e-10	0	\\
8.6e-10	0	\\
9.6e-10	0	\\
1.07e-09	0	\\
1.18e-09	0	\\
1.28e-09	0	\\
1.38e-09	0	\\
1.49e-09	0	\\
1.59e-09	0	\\
1.69e-09	0	\\
1.8e-09	0	\\
1.9e-09	0	\\
2.01e-09	0	\\
2.11e-09	0	\\
2.21e-09	0	\\
2.32e-09	0	\\
2.42e-09	0	\\
2.52e-09	0	\\
2.63e-09	0	\\
2.73e-09	0	\\
2.83e-09	0	\\
2.93e-09	0	\\
3.04e-09	0	\\
3.14e-09	0	\\
3.24e-09	0	\\
3.34e-09	0	\\
3.45e-09	0	\\
3.55e-09	0	\\
3.65e-09	0	\\
3.75e-09	0	\\
3.86e-09	0	\\
3.96e-09	0	\\
4.06e-09	0	\\
4.16e-09	0	\\
4.27e-09	0	\\
4.37e-09	0	\\
4.47e-09	0	\\
4.57e-09	0	\\
4.68e-09	0	\\
4.78e-09	0	\\
4.89e-09	0	\\
4.99e-09	0	\\
5e-09	0	\\
};
\addplot [color=mycolor3,solid,forget plot]
  table[row sep=crcr]{
0	0	\\
1.1e-10	0	\\
2.2e-10	0	\\
3.3e-10	0	\\
4.4e-10	0	\\
5.4e-10	0	\\
6.5e-10	0	\\
7.5e-10	0	\\
8.6e-10	0	\\
9.6e-10	0	\\
1.07e-09	0	\\
1.18e-09	0	\\
1.28e-09	0	\\
1.38e-09	0	\\
1.49e-09	0	\\
1.59e-09	0	\\
1.69e-09	0	\\
1.8e-09	0	\\
1.9e-09	0	\\
2.01e-09	0	\\
2.11e-09	0	\\
2.21e-09	0	\\
2.32e-09	0	\\
2.42e-09	0	\\
2.52e-09	0	\\
2.63e-09	0	\\
2.73e-09	0	\\
2.83e-09	0	\\
2.93e-09	0	\\
3.04e-09	0	\\
3.14e-09	0	\\
3.24e-09	0	\\
3.34e-09	0	\\
3.45e-09	0	\\
3.55e-09	0	\\
3.65e-09	0	\\
3.75e-09	0	\\
3.86e-09	0	\\
3.96e-09	0	\\
4.06e-09	0	\\
4.16e-09	0	\\
4.27e-09	0	\\
4.37e-09	0	\\
4.47e-09	0	\\
4.57e-09	0	\\
4.68e-09	0	\\
4.78e-09	0	\\
4.89e-09	0	\\
4.99e-09	0	\\
5e-09	0	\\
};
\addplot [color=darkgray,solid,forget plot]
  table[row sep=crcr]{
0	0	\\
1.1e-10	0	\\
2.2e-10	0	\\
3.3e-10	0	\\
4.4e-10	0	\\
5.4e-10	0	\\
6.5e-10	0	\\
7.5e-10	0	\\
8.6e-10	0	\\
9.6e-10	0	\\
1.07e-09	0	\\
1.18e-09	0	\\
1.28e-09	0	\\
1.38e-09	0	\\
1.49e-09	0	\\
1.59e-09	0	\\
1.69e-09	0	\\
1.8e-09	0	\\
1.9e-09	0	\\
2.01e-09	0	\\
2.11e-09	0	\\
2.21e-09	0	\\
2.32e-09	0	\\
2.42e-09	0	\\
2.52e-09	0	\\
2.63e-09	0	\\
2.73e-09	0	\\
2.83e-09	0	\\
2.93e-09	0	\\
3.04e-09	0	\\
3.14e-09	0	\\
3.24e-09	0	\\
3.34e-09	0	\\
3.45e-09	0	\\
3.55e-09	0	\\
3.65e-09	0	\\
3.75e-09	0	\\
3.86e-09	0	\\
3.96e-09	0	\\
4.06e-09	0	\\
4.16e-09	0	\\
4.27e-09	0	\\
4.37e-09	0	\\
4.47e-09	0	\\
4.57e-09	0	\\
4.68e-09	0	\\
4.78e-09	0	\\
4.89e-09	0	\\
4.99e-09	0	\\
5e-09	0	\\
};
\addplot [color=blue,solid,forget plot]
  table[row sep=crcr]{
0	0	\\
1.1e-10	0	\\
2.2e-10	0	\\
3.3e-10	0	\\
4.4e-10	0	\\
5.4e-10	0	\\
6.5e-10	0	\\
7.5e-10	0	\\
8.6e-10	0	\\
9.6e-10	0	\\
1.07e-09	0	\\
1.18e-09	0	\\
1.28e-09	0	\\
1.38e-09	0	\\
1.49e-09	0	\\
1.59e-09	0	\\
1.69e-09	0	\\
1.8e-09	0	\\
1.9e-09	0	\\
2.01e-09	0	\\
2.11e-09	0	\\
2.21e-09	0	\\
2.32e-09	0	\\
2.42e-09	0	\\
2.52e-09	0	\\
2.63e-09	0	\\
2.73e-09	0	\\
2.83e-09	0	\\
2.93e-09	0	\\
3.04e-09	0	\\
3.14e-09	0	\\
3.24e-09	0	\\
3.34e-09	0	\\
3.45e-09	0	\\
3.55e-09	0	\\
3.65e-09	0	\\
3.75e-09	0	\\
3.86e-09	0	\\
3.96e-09	0	\\
4.06e-09	0	\\
4.16e-09	0	\\
4.27e-09	0	\\
4.37e-09	0	\\
4.47e-09	0	\\
4.57e-09	0	\\
4.68e-09	0	\\
4.78e-09	0	\\
4.89e-09	0	\\
4.99e-09	0	\\
5e-09	0	\\
};
\addplot [color=black!50!green,solid,forget plot]
  table[row sep=crcr]{
0	0	\\
1.1e-10	0	\\
2.2e-10	0	\\
3.3e-10	0	\\
4.4e-10	0	\\
5.4e-10	0	\\
6.5e-10	0	\\
7.5e-10	0	\\
8.6e-10	0	\\
9.6e-10	0	\\
1.07e-09	0	\\
1.18e-09	0	\\
1.28e-09	0	\\
1.38e-09	0	\\
1.49e-09	0	\\
1.59e-09	0	\\
1.69e-09	0	\\
1.8e-09	0	\\
1.9e-09	0	\\
2.01e-09	0	\\
2.11e-09	0	\\
2.21e-09	0	\\
2.32e-09	0	\\
2.42e-09	0	\\
2.52e-09	0	\\
2.63e-09	0	\\
2.73e-09	0	\\
2.83e-09	0	\\
2.93e-09	0	\\
3.04e-09	0	\\
3.14e-09	0	\\
3.24e-09	0	\\
3.34e-09	0	\\
3.45e-09	0	\\
3.55e-09	0	\\
3.65e-09	0	\\
3.75e-09	0	\\
3.86e-09	0	\\
3.96e-09	0	\\
4.06e-09	0	\\
4.16e-09	0	\\
4.27e-09	0	\\
4.37e-09	0	\\
4.47e-09	0	\\
4.57e-09	0	\\
4.68e-09	0	\\
4.78e-09	0	\\
4.89e-09	0	\\
4.99e-09	0	\\
5e-09	0	\\
};
\addplot [color=red,solid,forget plot]
  table[row sep=crcr]{
0	0	\\
1.1e-10	0	\\
2.2e-10	0	\\
3.3e-10	0	\\
4.4e-10	0	\\
5.4e-10	0	\\
6.5e-10	0	\\
7.5e-10	0	\\
8.6e-10	0	\\
9.6e-10	0	\\
1.07e-09	0	\\
1.18e-09	0	\\
1.28e-09	0	\\
1.38e-09	0	\\
1.49e-09	0	\\
1.59e-09	0	\\
1.69e-09	0	\\
1.8e-09	0	\\
1.9e-09	0	\\
2.01e-09	0	\\
2.11e-09	0	\\
2.21e-09	0	\\
2.32e-09	0	\\
2.42e-09	0	\\
2.52e-09	0	\\
2.63e-09	0	\\
2.73e-09	0	\\
2.83e-09	0	\\
2.93e-09	0	\\
3.04e-09	0	\\
3.14e-09	0	\\
3.24e-09	0	\\
3.34e-09	0	\\
3.45e-09	0	\\
3.55e-09	0	\\
3.65e-09	0	\\
3.75e-09	0	\\
3.86e-09	0	\\
3.96e-09	0	\\
4.06e-09	0	\\
4.16e-09	0	\\
4.27e-09	0	\\
4.37e-09	0	\\
4.47e-09	0	\\
4.57e-09	0	\\
4.68e-09	0	\\
4.78e-09	0	\\
4.89e-09	0	\\
4.99e-09	0	\\
5e-09	0	\\
};
\addplot [color=mycolor1,solid,forget plot]
  table[row sep=crcr]{
0	0	\\
1.1e-10	0	\\
2.2e-10	0	\\
3.3e-10	0	\\
4.4e-10	0	\\
5.4e-10	0	\\
6.5e-10	0	\\
7.5e-10	0	\\
8.6e-10	0	\\
9.6e-10	0	\\
1.07e-09	0	\\
1.18e-09	0	\\
1.28e-09	0	\\
1.38e-09	0	\\
1.49e-09	0	\\
1.59e-09	0	\\
1.69e-09	0	\\
1.8e-09	0	\\
1.9e-09	0	\\
2.01e-09	0	\\
2.11e-09	0	\\
2.21e-09	0	\\
2.32e-09	0	\\
2.42e-09	0	\\
2.52e-09	0	\\
2.63e-09	0	\\
2.73e-09	0	\\
2.83e-09	0	\\
2.93e-09	0	\\
3.04e-09	0	\\
3.14e-09	0	\\
3.24e-09	0	\\
3.34e-09	0	\\
3.45e-09	0	\\
3.55e-09	0	\\
3.65e-09	0	\\
3.75e-09	0	\\
3.86e-09	0	\\
3.96e-09	0	\\
4.06e-09	0	\\
4.16e-09	0	\\
4.27e-09	0	\\
4.37e-09	0	\\
4.47e-09	0	\\
4.57e-09	0	\\
4.68e-09	0	\\
4.78e-09	0	\\
4.89e-09	0	\\
4.99e-09	0	\\
5e-09	0	\\
};
\addplot [color=mycolor2,solid,forget plot]
  table[row sep=crcr]{
0	0	\\
1.1e-10	0	\\
2.2e-10	0	\\
3.3e-10	0	\\
4.4e-10	0	\\
5.4e-10	0	\\
6.5e-10	0	\\
7.5e-10	0	\\
8.6e-10	0	\\
9.6e-10	0	\\
1.07e-09	0	\\
1.18e-09	0	\\
1.28e-09	0	\\
1.38e-09	0	\\
1.49e-09	0	\\
1.59e-09	0	\\
1.69e-09	0	\\
1.8e-09	0	\\
1.9e-09	0	\\
2.01e-09	0	\\
2.11e-09	0	\\
2.21e-09	0	\\
2.32e-09	0	\\
2.42e-09	0	\\
2.52e-09	0	\\
2.63e-09	0	\\
2.73e-09	0	\\
2.83e-09	0	\\
2.93e-09	0	\\
3.04e-09	0	\\
3.14e-09	0	\\
3.24e-09	0	\\
3.34e-09	0	\\
3.45e-09	0	\\
3.55e-09	0	\\
3.65e-09	0	\\
3.75e-09	0	\\
3.86e-09	0	\\
3.96e-09	0	\\
4.06e-09	0	\\
4.16e-09	0	\\
4.27e-09	0	\\
4.37e-09	0	\\
4.47e-09	0	\\
4.57e-09	0	\\
4.68e-09	0	\\
4.78e-09	0	\\
4.89e-09	0	\\
4.99e-09	0	\\
5e-09	0	\\
};
\addplot [color=mycolor3,solid,forget plot]
  table[row sep=crcr]{
0	0	\\
1.1e-10	0	\\
2.2e-10	0	\\
3.3e-10	0	\\
4.4e-10	0	\\
5.4e-10	0	\\
6.5e-10	0	\\
7.5e-10	0	\\
8.6e-10	0	\\
9.6e-10	0	\\
1.07e-09	0	\\
1.18e-09	0	\\
1.28e-09	0	\\
1.38e-09	0	\\
1.49e-09	0	\\
1.59e-09	0	\\
1.69e-09	0	\\
1.8e-09	0	\\
1.9e-09	0	\\
2.01e-09	0	\\
2.11e-09	0	\\
2.21e-09	0	\\
2.32e-09	0	\\
2.42e-09	0	\\
2.52e-09	0	\\
2.63e-09	0	\\
2.73e-09	0	\\
2.83e-09	0	\\
2.93e-09	0	\\
3.04e-09	0	\\
3.14e-09	0	\\
3.24e-09	0	\\
3.34e-09	0	\\
3.45e-09	0	\\
3.55e-09	0	\\
3.65e-09	0	\\
3.75e-09	0	\\
3.86e-09	0	\\
3.96e-09	0	\\
4.06e-09	0	\\
4.16e-09	0	\\
4.27e-09	0	\\
4.37e-09	0	\\
4.47e-09	0	\\
4.57e-09	0	\\
4.68e-09	0	\\
4.78e-09	0	\\
4.89e-09	0	\\
4.99e-09	0	\\
5e-09	0	\\
};
\addplot [color=darkgray,solid,forget plot]
  table[row sep=crcr]{
0	0	\\
1.1e-10	0	\\
2.2e-10	0	\\
3.3e-10	0	\\
4.4e-10	0	\\
5.4e-10	0	\\
6.5e-10	0	\\
7.5e-10	0	\\
8.6e-10	0	\\
9.6e-10	0	\\
1.07e-09	0	\\
1.18e-09	0	\\
1.28e-09	0	\\
1.38e-09	0	\\
1.49e-09	0	\\
1.59e-09	0	\\
1.69e-09	0	\\
1.8e-09	0	\\
1.9e-09	0	\\
2.01e-09	0	\\
2.11e-09	0	\\
2.21e-09	0	\\
2.32e-09	0	\\
2.42e-09	0	\\
2.52e-09	0	\\
2.63e-09	0	\\
2.73e-09	0	\\
2.83e-09	0	\\
2.93e-09	0	\\
3.04e-09	0	\\
3.14e-09	0	\\
3.24e-09	0	\\
3.34e-09	0	\\
3.45e-09	0	\\
3.55e-09	0	\\
3.65e-09	0	\\
3.75e-09	0	\\
3.86e-09	0	\\
3.96e-09	0	\\
4.06e-09	0	\\
4.16e-09	0	\\
4.27e-09	0	\\
4.37e-09	0	\\
4.47e-09	0	\\
4.57e-09	0	\\
4.68e-09	0	\\
4.78e-09	0	\\
4.89e-09	0	\\
4.99e-09	0	\\
5e-09	0	\\
};
\addplot [color=blue,solid,forget plot]
  table[row sep=crcr]{
0	0	\\
1.1e-10	0	\\
2.2e-10	0	\\
3.3e-10	0	\\
4.4e-10	0	\\
5.4e-10	0	\\
6.5e-10	0	\\
7.5e-10	0	\\
8.6e-10	0	\\
9.6e-10	0	\\
1.07e-09	0	\\
1.18e-09	0	\\
1.28e-09	0	\\
1.38e-09	0	\\
1.49e-09	0	\\
1.59e-09	0	\\
1.69e-09	0	\\
1.8e-09	0	\\
1.9e-09	0	\\
2.01e-09	0	\\
2.11e-09	0	\\
2.21e-09	0	\\
2.32e-09	0	\\
2.42e-09	0	\\
2.52e-09	0	\\
2.63e-09	0	\\
2.73e-09	0	\\
2.83e-09	0	\\
2.93e-09	0	\\
3.04e-09	0	\\
3.14e-09	0	\\
3.24e-09	0	\\
3.34e-09	0	\\
3.45e-09	0	\\
3.55e-09	0	\\
3.65e-09	0	\\
3.75e-09	0	\\
3.86e-09	0	\\
3.96e-09	0	\\
4.06e-09	0	\\
4.16e-09	0	\\
4.27e-09	0	\\
4.37e-09	0	\\
4.47e-09	0	\\
4.57e-09	0	\\
4.68e-09	0	\\
4.78e-09	0	\\
4.89e-09	0	\\
4.99e-09	0	\\
5e-09	0	\\
};
\addplot [color=black!50!green,solid,forget plot]
  table[row sep=crcr]{
0	0	\\
1.1e-10	0	\\
2.2e-10	0	\\
3.3e-10	0	\\
4.4e-10	0	\\
5.4e-10	0	\\
6.5e-10	0	\\
7.5e-10	0	\\
8.6e-10	0	\\
9.6e-10	0	\\
1.07e-09	0	\\
1.18e-09	0	\\
1.28e-09	0	\\
1.38e-09	0	\\
1.49e-09	0	\\
1.59e-09	0	\\
1.69e-09	0	\\
1.8e-09	0	\\
1.9e-09	0	\\
2.01e-09	0	\\
2.11e-09	0	\\
2.21e-09	0	\\
2.32e-09	0	\\
2.42e-09	0	\\
2.52e-09	0	\\
2.63e-09	0	\\
2.73e-09	0	\\
2.83e-09	0	\\
2.93e-09	0	\\
3.04e-09	0	\\
3.14e-09	0	\\
3.24e-09	0	\\
3.34e-09	0	\\
3.45e-09	0	\\
3.55e-09	0	\\
3.65e-09	0	\\
3.75e-09	0	\\
3.86e-09	0	\\
3.96e-09	0	\\
4.06e-09	0	\\
4.16e-09	0	\\
4.27e-09	0	\\
4.37e-09	0	\\
4.47e-09	0	\\
4.57e-09	0	\\
4.68e-09	0	\\
4.78e-09	0	\\
4.89e-09	0	\\
4.99e-09	0	\\
5e-09	0	\\
};
\addplot [color=red,solid,forget plot]
  table[row sep=crcr]{
0	0	\\
1.1e-10	0	\\
2.2e-10	0	\\
3.3e-10	0	\\
4.4e-10	0	\\
5.4e-10	0	\\
6.5e-10	0	\\
7.5e-10	0	\\
8.6e-10	0	\\
9.6e-10	0	\\
1.07e-09	0	\\
1.18e-09	0	\\
1.28e-09	0	\\
1.38e-09	0	\\
1.49e-09	0	\\
1.59e-09	0	\\
1.69e-09	0	\\
1.8e-09	0	\\
1.9e-09	0	\\
2.01e-09	0	\\
2.11e-09	0	\\
2.21e-09	0	\\
2.32e-09	0	\\
2.42e-09	0	\\
2.52e-09	0	\\
2.63e-09	0	\\
2.73e-09	0	\\
2.83e-09	0	\\
2.93e-09	0	\\
3.04e-09	0	\\
3.14e-09	0	\\
3.24e-09	0	\\
3.34e-09	0	\\
3.45e-09	0	\\
3.55e-09	0	\\
3.65e-09	0	\\
3.75e-09	0	\\
3.86e-09	0	\\
3.96e-09	0	\\
4.06e-09	0	\\
4.16e-09	0	\\
4.27e-09	0	\\
4.37e-09	0	\\
4.47e-09	0	\\
4.57e-09	0	\\
4.68e-09	0	\\
4.78e-09	0	\\
4.89e-09	0	\\
4.99e-09	0	\\
5e-09	0	\\
};
\addplot [color=mycolor1,solid,forget plot]
  table[row sep=crcr]{
0	0	\\
1.1e-10	0	\\
2.2e-10	0	\\
3.3e-10	0	\\
4.4e-10	0	\\
5.4e-10	0	\\
6.5e-10	0	\\
7.5e-10	0	\\
8.6e-10	0	\\
9.6e-10	0	\\
1.07e-09	0	\\
1.18e-09	0	\\
1.28e-09	0	\\
1.38e-09	0	\\
1.49e-09	0	\\
1.59e-09	0	\\
1.69e-09	0	\\
1.8e-09	0	\\
1.9e-09	0	\\
2.01e-09	0	\\
2.11e-09	0	\\
2.21e-09	0	\\
2.32e-09	0	\\
2.42e-09	0	\\
2.52e-09	0	\\
2.63e-09	0	\\
2.73e-09	0	\\
2.83e-09	0	\\
2.93e-09	0	\\
3.04e-09	0	\\
3.14e-09	0	\\
3.24e-09	0	\\
3.34e-09	0	\\
3.45e-09	0	\\
3.55e-09	0	\\
3.65e-09	0	\\
3.75e-09	0	\\
3.86e-09	0	\\
3.96e-09	0	\\
4.06e-09	0	\\
4.16e-09	0	\\
4.27e-09	0	\\
4.37e-09	0	\\
4.47e-09	0	\\
4.57e-09	0	\\
4.68e-09	0	\\
4.78e-09	0	\\
4.89e-09	0	\\
4.99e-09	0	\\
5e-09	0	\\
};
\addplot [color=mycolor2,solid,forget plot]
  table[row sep=crcr]{
0	0	\\
1.1e-10	0	\\
2.2e-10	0	\\
3.3e-10	0	\\
4.4e-10	0	\\
5.4e-10	0	\\
6.5e-10	0	\\
7.5e-10	0	\\
8.6e-10	0	\\
9.6e-10	0	\\
1.07e-09	0	\\
1.18e-09	0	\\
1.28e-09	0	\\
1.38e-09	0	\\
1.49e-09	0	\\
1.59e-09	0	\\
1.69e-09	0	\\
1.8e-09	0	\\
1.9e-09	0	\\
2.01e-09	0	\\
2.11e-09	0	\\
2.21e-09	0	\\
2.32e-09	0	\\
2.42e-09	0	\\
2.52e-09	0	\\
2.63e-09	0	\\
2.73e-09	0	\\
2.83e-09	0	\\
2.93e-09	0	\\
3.04e-09	0	\\
3.14e-09	0	\\
3.24e-09	0	\\
3.34e-09	0	\\
3.45e-09	0	\\
3.55e-09	0	\\
3.65e-09	0	\\
3.75e-09	0	\\
3.86e-09	0	\\
3.96e-09	0	\\
4.06e-09	0	\\
4.16e-09	0	\\
4.27e-09	0	\\
4.37e-09	0	\\
4.47e-09	0	\\
4.57e-09	0	\\
4.68e-09	0	\\
4.78e-09	0	\\
4.89e-09	0	\\
4.99e-09	0	\\
5e-09	0	\\
};
\addplot [color=mycolor3,solid,forget plot]
  table[row sep=crcr]{
0	0	\\
1.1e-10	0	\\
2.2e-10	0	\\
3.3e-10	0	\\
4.4e-10	0	\\
5.4e-10	0	\\
6.5e-10	0	\\
7.5e-10	0	\\
8.6e-10	0	\\
9.6e-10	0	\\
1.07e-09	0	\\
1.18e-09	0	\\
1.28e-09	0	\\
1.38e-09	0	\\
1.49e-09	0	\\
1.59e-09	0	\\
1.69e-09	0	\\
1.8e-09	0	\\
1.9e-09	0	\\
2.01e-09	0	\\
2.11e-09	0	\\
2.21e-09	0	\\
2.32e-09	0	\\
2.42e-09	0	\\
2.52e-09	0	\\
2.63e-09	0	\\
2.73e-09	0	\\
2.83e-09	0	\\
2.93e-09	0	\\
3.04e-09	0	\\
3.14e-09	0	\\
3.24e-09	0	\\
3.34e-09	0	\\
3.45e-09	0	\\
3.55e-09	0	\\
3.65e-09	0	\\
3.75e-09	0	\\
3.86e-09	0	\\
3.96e-09	0	\\
4.06e-09	0	\\
4.16e-09	0	\\
4.27e-09	0	\\
4.37e-09	0	\\
4.47e-09	0	\\
4.57e-09	0	\\
4.68e-09	0	\\
4.78e-09	0	\\
4.89e-09	0	\\
4.99e-09	0	\\
5e-09	0	\\
};
\addplot [color=darkgray,solid,forget plot]
  table[row sep=crcr]{
0	0	\\
1.1e-10	0	\\
2.2e-10	0	\\
3.3e-10	0	\\
4.4e-10	0	\\
5.4e-10	0	\\
6.5e-10	0	\\
7.5e-10	0	\\
8.6e-10	0	\\
9.6e-10	0	\\
1.07e-09	0	\\
1.18e-09	0	\\
1.28e-09	0	\\
1.38e-09	0	\\
1.49e-09	0	\\
1.59e-09	0	\\
1.69e-09	0	\\
1.8e-09	0	\\
1.9e-09	0	\\
2.01e-09	0	\\
2.11e-09	0	\\
2.21e-09	0	\\
2.32e-09	0	\\
2.42e-09	0	\\
2.52e-09	0	\\
2.63e-09	0	\\
2.73e-09	0	\\
2.83e-09	0	\\
2.93e-09	0	\\
3.04e-09	0	\\
3.14e-09	0	\\
3.24e-09	0	\\
3.34e-09	0	\\
3.45e-09	0	\\
3.55e-09	0	\\
3.65e-09	0	\\
3.75e-09	0	\\
3.86e-09	0	\\
3.96e-09	0	\\
4.06e-09	0	\\
4.16e-09	0	\\
4.27e-09	0	\\
4.37e-09	0	\\
4.47e-09	0	\\
4.57e-09	0	\\
4.68e-09	0	\\
4.78e-09	0	\\
4.89e-09	0	\\
4.99e-09	0	\\
5e-09	0	\\
};
\addplot [color=blue,solid,forget plot]
  table[row sep=crcr]{
0	0	\\
1.1e-10	0	\\
2.2e-10	0	\\
3.3e-10	0	\\
4.4e-10	0	\\
5.4e-10	0	\\
6.5e-10	0	\\
7.5e-10	0	\\
8.6e-10	0	\\
9.6e-10	0	\\
1.07e-09	0	\\
1.18e-09	0	\\
1.28e-09	0	\\
1.38e-09	0	\\
1.49e-09	0	\\
1.59e-09	0	\\
1.69e-09	0	\\
1.8e-09	0	\\
1.9e-09	0	\\
2.01e-09	0	\\
2.11e-09	0	\\
2.21e-09	0	\\
2.32e-09	0	\\
2.42e-09	0	\\
2.52e-09	0	\\
2.63e-09	0	\\
2.73e-09	0	\\
2.83e-09	0	\\
2.93e-09	0	\\
3.04e-09	0	\\
3.14e-09	0	\\
3.24e-09	0	\\
3.34e-09	0	\\
3.45e-09	0	\\
3.55e-09	0	\\
3.65e-09	0	\\
3.75e-09	0	\\
3.86e-09	0	\\
3.96e-09	0	\\
4.06e-09	0	\\
4.16e-09	0	\\
4.27e-09	0	\\
4.37e-09	0	\\
4.47e-09	0	\\
4.57e-09	0	\\
4.68e-09	0	\\
4.78e-09	0	\\
4.89e-09	0	\\
4.99e-09	0	\\
5e-09	0	\\
};
\addplot [color=black!50!green,solid,forget plot]
  table[row sep=crcr]{
0	0	\\
1.1e-10	0	\\
2.2e-10	0	\\
3.3e-10	0	\\
4.4e-10	0	\\
5.4e-10	0	\\
6.5e-10	0	\\
7.5e-10	0	\\
8.6e-10	0	\\
9.6e-10	0	\\
1.07e-09	0	\\
1.18e-09	0	\\
1.28e-09	0	\\
1.38e-09	0	\\
1.49e-09	0	\\
1.59e-09	0	\\
1.69e-09	0	\\
1.8e-09	0	\\
1.9e-09	0	\\
2.01e-09	0	\\
2.11e-09	0	\\
2.21e-09	0	\\
2.32e-09	0	\\
2.42e-09	0	\\
2.52e-09	0	\\
2.63e-09	0	\\
2.73e-09	0	\\
2.83e-09	0	\\
2.93e-09	0	\\
3.04e-09	0	\\
3.14e-09	0	\\
3.24e-09	0	\\
3.34e-09	0	\\
3.45e-09	0	\\
3.55e-09	0	\\
3.65e-09	0	\\
3.75e-09	0	\\
3.86e-09	0	\\
3.96e-09	0	\\
4.06e-09	0	\\
4.16e-09	0	\\
4.27e-09	0	\\
4.37e-09	0	\\
4.47e-09	0	\\
4.57e-09	0	\\
4.68e-09	0	\\
4.78e-09	0	\\
4.89e-09	0	\\
4.99e-09	0	\\
5e-09	0	\\
};
\addplot [color=red,solid,forget plot]
  table[row sep=crcr]{
0	0	\\
1.1e-10	0	\\
2.2e-10	0	\\
3.3e-10	0	\\
4.4e-10	0	\\
5.4e-10	0	\\
6.5e-10	0	\\
7.5e-10	0	\\
8.6e-10	0	\\
9.6e-10	0	\\
1.07e-09	0	\\
1.18e-09	0	\\
1.28e-09	0	\\
1.38e-09	0	\\
1.49e-09	0	\\
1.59e-09	0	\\
1.69e-09	0	\\
1.8e-09	0	\\
1.9e-09	0	\\
2.01e-09	0	\\
2.11e-09	0	\\
2.21e-09	0	\\
2.32e-09	0	\\
2.42e-09	0	\\
2.52e-09	0	\\
2.63e-09	0	\\
2.73e-09	0	\\
2.83e-09	0	\\
2.93e-09	0	\\
3.04e-09	0	\\
3.14e-09	0	\\
3.24e-09	0	\\
3.34e-09	0	\\
3.45e-09	0	\\
3.55e-09	0	\\
3.65e-09	0	\\
3.75e-09	0	\\
3.86e-09	0	\\
3.96e-09	0	\\
4.06e-09	0	\\
4.16e-09	0	\\
4.27e-09	0	\\
4.37e-09	0	\\
4.47e-09	0	\\
4.57e-09	0	\\
4.68e-09	0	\\
4.78e-09	0	\\
4.89e-09	0	\\
4.99e-09	0	\\
5e-09	0	\\
};
\addplot [color=mycolor1,solid,forget plot]
  table[row sep=crcr]{
0	0	\\
1.1e-10	0	\\
2.2e-10	0	\\
3.3e-10	0	\\
4.4e-10	0	\\
5.4e-10	0	\\
6.5e-10	0	\\
7.5e-10	0	\\
8.6e-10	0	\\
9.6e-10	0	\\
1.07e-09	0	\\
1.18e-09	0	\\
1.28e-09	0	\\
1.38e-09	0	\\
1.49e-09	0	\\
1.59e-09	0	\\
1.69e-09	0	\\
1.8e-09	0	\\
1.9e-09	0	\\
2.01e-09	0	\\
2.11e-09	0	\\
2.21e-09	0	\\
2.32e-09	0	\\
2.42e-09	0	\\
2.52e-09	0	\\
2.63e-09	0	\\
2.73e-09	0	\\
2.83e-09	0	\\
2.93e-09	0	\\
3.04e-09	0	\\
3.14e-09	0	\\
3.24e-09	0	\\
3.34e-09	0	\\
3.45e-09	0	\\
3.55e-09	0	\\
3.65e-09	0	\\
3.75e-09	0	\\
3.86e-09	0	\\
3.96e-09	0	\\
4.06e-09	0	\\
4.16e-09	0	\\
4.27e-09	0	\\
4.37e-09	0	\\
4.47e-09	0	\\
4.57e-09	0	\\
4.68e-09	0	\\
4.78e-09	0	\\
4.89e-09	0	\\
4.99e-09	0	\\
5e-09	0	\\
};
\addplot [color=mycolor2,solid,forget plot]
  table[row sep=crcr]{
0	0	\\
1.1e-10	0	\\
2.2e-10	0	\\
3.3e-10	0	\\
4.4e-10	0	\\
5.4e-10	0	\\
6.5e-10	0	\\
7.5e-10	0	\\
8.6e-10	0	\\
9.6e-10	0	\\
1.07e-09	0	\\
1.18e-09	0	\\
1.28e-09	0	\\
1.38e-09	0	\\
1.49e-09	0	\\
1.59e-09	0	\\
1.69e-09	0	\\
1.8e-09	0	\\
1.9e-09	0	\\
2.01e-09	0	\\
2.11e-09	0	\\
2.21e-09	0	\\
2.32e-09	0	\\
2.42e-09	0	\\
2.52e-09	0	\\
2.63e-09	0	\\
2.73e-09	0	\\
2.83e-09	0	\\
2.93e-09	0	\\
3.04e-09	0	\\
3.14e-09	0	\\
3.24e-09	0	\\
3.34e-09	0	\\
3.45e-09	0	\\
3.55e-09	0	\\
3.65e-09	0	\\
3.75e-09	0	\\
3.86e-09	0	\\
3.96e-09	0	\\
4.06e-09	0	\\
4.16e-09	0	\\
4.27e-09	0	\\
4.37e-09	0	\\
4.47e-09	0	\\
4.57e-09	0	\\
4.68e-09	0	\\
4.78e-09	0	\\
4.89e-09	0	\\
4.99e-09	0	\\
5e-09	0	\\
};
\addplot [color=mycolor3,solid,forget plot]
  table[row sep=crcr]{
0	0	\\
1.1e-10	0	\\
2.2e-10	0	\\
3.3e-10	0	\\
4.4e-10	0	\\
5.4e-10	0	\\
6.5e-10	0	\\
7.5e-10	0	\\
8.6e-10	0	\\
9.6e-10	0	\\
1.07e-09	0	\\
1.18e-09	0	\\
1.28e-09	0	\\
1.38e-09	0	\\
1.49e-09	0	\\
1.59e-09	0	\\
1.69e-09	0	\\
1.8e-09	0	\\
1.9e-09	0	\\
2.01e-09	0	\\
2.11e-09	0	\\
2.21e-09	0	\\
2.32e-09	0	\\
2.42e-09	0	\\
2.52e-09	0	\\
2.63e-09	0	\\
2.73e-09	0	\\
2.83e-09	0	\\
2.93e-09	0	\\
3.04e-09	0	\\
3.14e-09	0	\\
3.24e-09	0	\\
3.34e-09	0	\\
3.45e-09	0	\\
3.55e-09	0	\\
3.65e-09	0	\\
3.75e-09	0	\\
3.86e-09	0	\\
3.96e-09	0	\\
4.06e-09	0	\\
4.16e-09	0	\\
4.27e-09	0	\\
4.37e-09	0	\\
4.47e-09	0	\\
4.57e-09	0	\\
4.68e-09	0	\\
4.78e-09	0	\\
4.89e-09	0	\\
4.99e-09	0	\\
5e-09	0	\\
};
\addplot [color=darkgray,solid,forget plot]
  table[row sep=crcr]{
0	0	\\
1.1e-10	0	\\
2.2e-10	0	\\
3.3e-10	0	\\
4.4e-10	0	\\
5.4e-10	0	\\
6.5e-10	0	\\
7.5e-10	0	\\
8.6e-10	0	\\
9.6e-10	0	\\
1.07e-09	0	\\
1.18e-09	0	\\
1.28e-09	0	\\
1.38e-09	0	\\
1.49e-09	0	\\
1.59e-09	0	\\
1.69e-09	0	\\
1.8e-09	0	\\
1.9e-09	0	\\
2.01e-09	0	\\
2.11e-09	0	\\
2.21e-09	0	\\
2.32e-09	0	\\
2.42e-09	0	\\
2.52e-09	0	\\
2.63e-09	0	\\
2.73e-09	0	\\
2.83e-09	0	\\
2.93e-09	0	\\
3.04e-09	0	\\
3.14e-09	0	\\
3.24e-09	0	\\
3.34e-09	0	\\
3.45e-09	0	\\
3.55e-09	0	\\
3.65e-09	0	\\
3.75e-09	0	\\
3.86e-09	0	\\
3.96e-09	0	\\
4.06e-09	0	\\
4.16e-09	0	\\
4.27e-09	0	\\
4.37e-09	0	\\
4.47e-09	0	\\
4.57e-09	0	\\
4.68e-09	0	\\
4.78e-09	0	\\
4.89e-09	0	\\
4.99e-09	0	\\
5e-09	0	\\
};
\addplot [color=blue,solid,forget plot]
  table[row sep=crcr]{
0	0	\\
1.1e-10	0	\\
2.2e-10	0	\\
3.3e-10	0	\\
4.4e-10	0	\\
5.4e-10	0	\\
6.5e-10	0	\\
7.5e-10	0	\\
8.6e-10	0	\\
9.6e-10	0	\\
1.07e-09	0	\\
1.18e-09	0	\\
1.28e-09	0	\\
1.38e-09	0	\\
1.49e-09	0	\\
1.59e-09	0	\\
1.69e-09	0	\\
1.8e-09	0	\\
1.9e-09	0	\\
2.01e-09	0	\\
2.11e-09	0	\\
2.21e-09	0	\\
2.32e-09	0	\\
2.42e-09	0	\\
2.52e-09	0	\\
2.63e-09	0	\\
2.73e-09	0	\\
2.83e-09	0	\\
2.93e-09	0	\\
3.04e-09	0	\\
3.14e-09	0	\\
3.24e-09	0	\\
3.34e-09	0	\\
3.45e-09	0	\\
3.55e-09	0	\\
3.65e-09	0	\\
3.75e-09	0	\\
3.86e-09	0	\\
3.96e-09	0	\\
4.06e-09	0	\\
4.16e-09	0	\\
4.27e-09	0	\\
4.37e-09	0	\\
4.47e-09	0	\\
4.57e-09	0	\\
4.68e-09	0	\\
4.78e-09	0	\\
4.89e-09	0	\\
4.99e-09	0	\\
5e-09	0	\\
};
\addplot [color=black!50!green,solid,forget plot]
  table[row sep=crcr]{
0	0	\\
1.1e-10	0	\\
2.2e-10	0	\\
3.3e-10	0	\\
4.4e-10	0	\\
5.4e-10	0	\\
6.5e-10	0	\\
7.5e-10	0	\\
8.6e-10	0	\\
9.6e-10	0	\\
1.07e-09	0	\\
1.18e-09	0	\\
1.28e-09	0	\\
1.38e-09	0	\\
1.49e-09	0	\\
1.59e-09	0	\\
1.69e-09	0	\\
1.8e-09	0	\\
1.9e-09	0	\\
2.01e-09	0	\\
2.11e-09	0	\\
2.21e-09	0	\\
2.32e-09	0	\\
2.42e-09	0	\\
2.52e-09	0	\\
2.63e-09	0	\\
2.73e-09	0	\\
2.83e-09	0	\\
2.93e-09	0	\\
3.04e-09	0	\\
3.14e-09	0	\\
3.24e-09	0	\\
3.34e-09	0	\\
3.45e-09	0	\\
3.55e-09	0	\\
3.65e-09	0	\\
3.75e-09	0	\\
3.86e-09	0	\\
3.96e-09	0	\\
4.06e-09	0	\\
4.16e-09	0	\\
4.27e-09	0	\\
4.37e-09	0	\\
4.47e-09	0	\\
4.57e-09	0	\\
4.68e-09	0	\\
4.78e-09	0	\\
4.89e-09	0	\\
4.99e-09	0	\\
5e-09	0	\\
};
\addplot [color=red,solid,forget plot]
  table[row sep=crcr]{
0	0	\\
1.1e-10	0	\\
2.2e-10	0	\\
3.3e-10	0	\\
4.4e-10	0	\\
5.4e-10	0	\\
6.5e-10	0	\\
7.5e-10	0	\\
8.6e-10	0	\\
9.6e-10	0	\\
1.07e-09	0	\\
1.18e-09	0	\\
1.28e-09	0	\\
1.38e-09	0	\\
1.49e-09	0	\\
1.59e-09	0	\\
1.69e-09	0	\\
1.8e-09	0	\\
1.9e-09	0	\\
2.01e-09	0	\\
2.11e-09	0	\\
2.21e-09	0	\\
2.32e-09	0	\\
2.42e-09	0	\\
2.52e-09	0	\\
2.63e-09	0	\\
2.73e-09	0	\\
2.83e-09	0	\\
2.93e-09	0	\\
3.04e-09	0	\\
3.14e-09	0	\\
3.24e-09	0	\\
3.34e-09	0	\\
3.45e-09	0	\\
3.55e-09	0	\\
3.65e-09	0	\\
3.75e-09	0	\\
3.86e-09	0	\\
3.96e-09	0	\\
4.06e-09	0	\\
4.16e-09	0	\\
4.27e-09	0	\\
4.37e-09	0	\\
4.47e-09	0	\\
4.57e-09	0	\\
4.68e-09	0	\\
4.78e-09	0	\\
4.89e-09	0	\\
4.99e-09	0	\\
5e-09	0	\\
};
\addplot [color=mycolor1,solid,forget plot]
  table[row sep=crcr]{
0	0	\\
1.1e-10	0	\\
2.2e-10	0	\\
3.3e-10	0	\\
4.4e-10	0	\\
5.4e-10	0	\\
6.5e-10	0	\\
7.5e-10	0	\\
8.6e-10	0	\\
9.6e-10	0	\\
1.07e-09	0	\\
1.18e-09	0	\\
1.28e-09	0	\\
1.38e-09	0	\\
1.49e-09	0	\\
1.59e-09	0	\\
1.69e-09	0	\\
1.8e-09	0	\\
1.9e-09	0	\\
2.01e-09	0	\\
2.11e-09	0	\\
2.21e-09	0	\\
2.32e-09	0	\\
2.42e-09	0	\\
2.52e-09	0	\\
2.63e-09	0	\\
2.73e-09	0	\\
2.83e-09	0	\\
2.93e-09	0	\\
3.04e-09	0	\\
3.14e-09	0	\\
3.24e-09	0	\\
3.34e-09	0	\\
3.45e-09	0	\\
3.55e-09	0	\\
3.65e-09	0	\\
3.75e-09	0	\\
3.86e-09	0	\\
3.96e-09	0	\\
4.06e-09	0	\\
4.16e-09	0	\\
4.27e-09	0	\\
4.37e-09	0	\\
4.47e-09	0	\\
4.57e-09	0	\\
4.68e-09	0	\\
4.78e-09	0	\\
4.89e-09	0	\\
4.99e-09	0	\\
5e-09	0	\\
};
\addplot [color=mycolor2,solid,forget plot]
  table[row sep=crcr]{
0	0	\\
1.1e-10	0	\\
2.2e-10	0	\\
3.3e-10	0	\\
4.4e-10	0	\\
5.4e-10	0	\\
6.5e-10	0	\\
7.5e-10	0	\\
8.6e-10	0	\\
9.6e-10	0	\\
1.07e-09	0	\\
1.18e-09	0	\\
1.28e-09	0	\\
1.38e-09	0	\\
1.49e-09	0	\\
1.59e-09	0	\\
1.69e-09	0	\\
1.8e-09	0	\\
1.9e-09	0	\\
2.01e-09	0	\\
2.11e-09	0	\\
2.21e-09	0	\\
2.32e-09	0	\\
2.42e-09	0	\\
2.52e-09	0	\\
2.63e-09	0	\\
2.73e-09	0	\\
2.83e-09	0	\\
2.93e-09	0	\\
3.04e-09	0	\\
3.14e-09	0	\\
3.24e-09	0	\\
3.34e-09	0	\\
3.45e-09	0	\\
3.55e-09	0	\\
3.65e-09	0	\\
3.75e-09	0	\\
3.86e-09	0	\\
3.96e-09	0	\\
4.06e-09	0	\\
4.16e-09	0	\\
4.27e-09	0	\\
4.37e-09	0	\\
4.47e-09	0	\\
4.57e-09	0	\\
4.68e-09	0	\\
4.78e-09	0	\\
4.89e-09	0	\\
4.99e-09	0	\\
5e-09	0	\\
};
\addplot [color=mycolor3,solid,forget plot]
  table[row sep=crcr]{
0	0	\\
1.1e-10	0	\\
2.2e-10	0	\\
3.3e-10	0	\\
4.4e-10	0	\\
5.4e-10	0	\\
6.5e-10	0	\\
7.5e-10	0	\\
8.6e-10	0	\\
9.6e-10	0	\\
1.07e-09	0	\\
1.18e-09	0	\\
1.28e-09	0	\\
1.38e-09	0	\\
1.49e-09	0	\\
1.59e-09	0	\\
1.69e-09	0	\\
1.8e-09	0	\\
1.9e-09	0	\\
2.01e-09	0	\\
2.11e-09	0	\\
2.21e-09	0	\\
2.32e-09	0	\\
2.42e-09	0	\\
2.52e-09	0	\\
2.63e-09	0	\\
2.73e-09	0	\\
2.83e-09	0	\\
2.93e-09	0	\\
3.04e-09	0	\\
3.14e-09	0	\\
3.24e-09	0	\\
3.34e-09	0	\\
3.45e-09	0	\\
3.55e-09	0	\\
3.65e-09	0	\\
3.75e-09	0	\\
3.86e-09	0	\\
3.96e-09	0	\\
4.06e-09	0	\\
4.16e-09	0	\\
4.27e-09	0	\\
4.37e-09	0	\\
4.47e-09	0	\\
4.57e-09	0	\\
4.68e-09	0	\\
4.78e-09	0	\\
4.89e-09	0	\\
4.99e-09	0	\\
5e-09	0	\\
};
\addplot [color=darkgray,solid,forget plot]
  table[row sep=crcr]{
0	0	\\
1.1e-10	0	\\
2.2e-10	0	\\
3.3e-10	0	\\
4.4e-10	0	\\
5.4e-10	0	\\
6.5e-10	0	\\
7.5e-10	0	\\
8.6e-10	0	\\
9.6e-10	0	\\
1.07e-09	0	\\
1.18e-09	0	\\
1.28e-09	0	\\
1.38e-09	0	\\
1.49e-09	0	\\
1.59e-09	0	\\
1.69e-09	0	\\
1.8e-09	0	\\
1.9e-09	0	\\
2.01e-09	0	\\
2.11e-09	0	\\
2.21e-09	0	\\
2.32e-09	0	\\
2.42e-09	0	\\
2.52e-09	0	\\
2.63e-09	0	\\
2.73e-09	0	\\
2.83e-09	0	\\
2.93e-09	0	\\
3.04e-09	0	\\
3.14e-09	0	\\
3.24e-09	0	\\
3.34e-09	0	\\
3.45e-09	0	\\
3.55e-09	0	\\
3.65e-09	0	\\
3.75e-09	0	\\
3.86e-09	0	\\
3.96e-09	0	\\
4.06e-09	0	\\
4.16e-09	0	\\
4.27e-09	0	\\
4.37e-09	0	\\
4.47e-09	0	\\
4.57e-09	0	\\
4.68e-09	0	\\
4.78e-09	0	\\
4.89e-09	0	\\
4.99e-09	0	\\
5e-09	0	\\
};
\addplot [color=blue,solid,forget plot]
  table[row sep=crcr]{
0	0	\\
1.1e-10	0	\\
2.2e-10	0	\\
3.3e-10	0	\\
4.4e-10	0	\\
5.4e-10	0	\\
6.5e-10	0	\\
7.5e-10	0	\\
8.6e-10	0	\\
9.6e-10	0	\\
1.07e-09	0	\\
1.18e-09	0	\\
1.28e-09	0	\\
1.38e-09	0	\\
1.49e-09	0	\\
1.59e-09	0	\\
1.69e-09	0	\\
1.8e-09	0	\\
1.9e-09	0	\\
2.01e-09	0	\\
2.11e-09	0	\\
2.21e-09	0	\\
2.32e-09	0	\\
2.42e-09	0	\\
2.52e-09	0	\\
2.63e-09	0	\\
2.73e-09	0	\\
2.83e-09	0	\\
2.93e-09	0	\\
3.04e-09	0	\\
3.14e-09	0	\\
3.24e-09	0	\\
3.34e-09	0	\\
3.45e-09	0	\\
3.55e-09	0	\\
3.65e-09	0	\\
3.75e-09	0	\\
3.86e-09	0	\\
3.96e-09	0	\\
4.06e-09	0	\\
4.16e-09	0	\\
4.27e-09	0	\\
4.37e-09	0	\\
4.47e-09	0	\\
4.57e-09	0	\\
4.68e-09	0	\\
4.78e-09	0	\\
4.89e-09	0	\\
4.99e-09	0	\\
5e-09	0	\\
};
\addplot [color=black!50!green,solid,forget plot]
  table[row sep=crcr]{
0	0	\\
1.1e-10	0	\\
2.2e-10	0	\\
3.3e-10	0	\\
4.4e-10	0	\\
5.4e-10	0	\\
6.5e-10	0	\\
7.5e-10	0	\\
8.6e-10	0	\\
9.6e-10	0	\\
1.07e-09	0	\\
1.18e-09	0	\\
1.28e-09	0	\\
1.38e-09	0	\\
1.49e-09	0	\\
1.59e-09	0	\\
1.69e-09	0	\\
1.8e-09	0	\\
1.9e-09	0	\\
2.01e-09	0	\\
2.11e-09	0	\\
2.21e-09	0	\\
2.32e-09	0	\\
2.42e-09	0	\\
2.52e-09	0	\\
2.63e-09	0	\\
2.73e-09	0	\\
2.83e-09	0	\\
2.93e-09	0	\\
3.04e-09	0	\\
3.14e-09	0	\\
3.24e-09	0	\\
3.34e-09	0	\\
3.45e-09	0	\\
3.55e-09	0	\\
3.65e-09	0	\\
3.75e-09	0	\\
3.86e-09	0	\\
3.96e-09	0	\\
4.06e-09	0	\\
4.16e-09	0	\\
4.27e-09	0	\\
4.37e-09	0	\\
4.47e-09	0	\\
4.57e-09	0	\\
4.68e-09	0	\\
4.78e-09	0	\\
4.89e-09	0	\\
4.99e-09	0	\\
5e-09	0	\\
};
\addplot [color=red,solid,forget plot]
  table[row sep=crcr]{
0	0	\\
1.1e-10	0	\\
2.2e-10	0	\\
3.3e-10	0	\\
4.4e-10	0	\\
5.4e-10	0	\\
6.5e-10	0	\\
7.5e-10	0	\\
8.6e-10	0	\\
9.6e-10	0	\\
1.07e-09	0	\\
1.18e-09	0	\\
1.28e-09	0	\\
1.38e-09	0	\\
1.49e-09	0	\\
1.59e-09	0	\\
1.69e-09	0	\\
1.8e-09	0	\\
1.9e-09	0	\\
2.01e-09	0	\\
2.11e-09	0	\\
2.21e-09	0	\\
2.32e-09	0	\\
2.42e-09	0	\\
2.52e-09	0	\\
2.63e-09	0	\\
2.73e-09	0	\\
2.83e-09	0	\\
2.93e-09	0	\\
3.04e-09	0	\\
3.14e-09	0	\\
3.24e-09	0	\\
3.34e-09	0	\\
3.45e-09	0	\\
3.55e-09	0	\\
3.65e-09	0	\\
3.75e-09	0	\\
3.86e-09	0	\\
3.96e-09	0	\\
4.06e-09	0	\\
4.16e-09	0	\\
4.27e-09	0	\\
4.37e-09	0	\\
4.47e-09	0	\\
4.57e-09	0	\\
4.68e-09	0	\\
4.78e-09	0	\\
4.89e-09	0	\\
4.99e-09	0	\\
5e-09	0	\\
};
\addplot [color=mycolor1,solid,forget plot]
  table[row sep=crcr]{
0	0	\\
1.1e-10	0	\\
2.2e-10	0	\\
3.3e-10	0	\\
4.4e-10	0	\\
5.4e-10	0	\\
6.5e-10	0	\\
7.5e-10	0	\\
8.6e-10	0	\\
9.6e-10	0	\\
1.07e-09	0	\\
1.18e-09	0	\\
1.28e-09	0	\\
1.38e-09	0	\\
1.49e-09	0	\\
1.59e-09	0	\\
1.69e-09	0	\\
1.8e-09	0	\\
1.9e-09	0	\\
2.01e-09	0	\\
2.11e-09	0	\\
2.21e-09	0	\\
2.32e-09	0	\\
2.42e-09	0	\\
2.52e-09	0	\\
2.63e-09	0	\\
2.73e-09	0	\\
2.83e-09	0	\\
2.93e-09	0	\\
3.04e-09	0	\\
3.14e-09	0	\\
3.24e-09	0	\\
3.34e-09	0	\\
3.45e-09	0	\\
3.55e-09	0	\\
3.65e-09	0	\\
3.75e-09	0	\\
3.86e-09	0	\\
3.96e-09	0	\\
4.06e-09	0	\\
4.16e-09	0	\\
4.27e-09	0	\\
4.37e-09	0	\\
4.47e-09	0	\\
4.57e-09	0	\\
4.68e-09	0	\\
4.78e-09	0	\\
4.89e-09	0	\\
4.99e-09	0	\\
5e-09	0	\\
};
\addplot [color=mycolor2,solid,forget plot]
  table[row sep=crcr]{
0	0	\\
1.1e-10	0	\\
2.2e-10	0	\\
3.3e-10	0	\\
4.4e-10	0	\\
5.4e-10	0	\\
6.5e-10	0	\\
7.5e-10	0	\\
8.6e-10	0	\\
9.6e-10	0	\\
1.07e-09	0	\\
1.18e-09	0	\\
1.28e-09	0	\\
1.38e-09	0	\\
1.49e-09	0	\\
1.59e-09	0	\\
1.69e-09	0	\\
1.8e-09	0	\\
1.9e-09	0	\\
2.01e-09	0	\\
2.11e-09	0	\\
2.21e-09	0	\\
2.32e-09	0	\\
2.42e-09	0	\\
2.52e-09	0	\\
2.63e-09	0	\\
2.73e-09	0	\\
2.83e-09	0	\\
2.93e-09	0	\\
3.04e-09	0	\\
3.14e-09	0	\\
3.24e-09	0	\\
3.34e-09	0	\\
3.45e-09	0	\\
3.55e-09	0	\\
3.65e-09	0	\\
3.75e-09	0	\\
3.86e-09	0	\\
3.96e-09	0	\\
4.06e-09	0	\\
4.16e-09	0	\\
4.27e-09	0	\\
4.37e-09	0	\\
4.47e-09	0	\\
4.57e-09	0	\\
4.68e-09	0	\\
4.78e-09	0	\\
4.89e-09	0	\\
4.99e-09	0	\\
5e-09	0	\\
};
\addplot [color=mycolor3,solid,forget plot]
  table[row sep=crcr]{
0	0	\\
1.1e-10	0	\\
2.2e-10	0	\\
3.3e-10	0	\\
4.4e-10	0	\\
5.4e-10	0	\\
6.5e-10	0	\\
7.5e-10	0	\\
8.6e-10	0	\\
9.6e-10	0	\\
1.07e-09	0	\\
1.18e-09	0	\\
1.28e-09	0	\\
1.38e-09	0	\\
1.49e-09	0	\\
1.59e-09	0	\\
1.69e-09	0	\\
1.8e-09	0	\\
1.9e-09	0	\\
2.01e-09	0	\\
2.11e-09	0	\\
2.21e-09	0	\\
2.32e-09	0	\\
2.42e-09	0	\\
2.52e-09	0	\\
2.63e-09	0	\\
2.73e-09	0	\\
2.83e-09	0	\\
2.93e-09	0	\\
3.04e-09	0	\\
3.14e-09	0	\\
3.24e-09	0	\\
3.34e-09	0	\\
3.45e-09	0	\\
3.55e-09	0	\\
3.65e-09	0	\\
3.75e-09	0	\\
3.86e-09	0	\\
3.96e-09	0	\\
4.06e-09	0	\\
4.16e-09	0	\\
4.27e-09	0	\\
4.37e-09	0	\\
4.47e-09	0	\\
4.57e-09	0	\\
4.68e-09	0	\\
4.78e-09	0	\\
4.89e-09	0	\\
4.99e-09	0	\\
5e-09	0	\\
};
\addplot [color=darkgray,solid,forget plot]
  table[row sep=crcr]{
0	0	\\
1.1e-10	0	\\
2.2e-10	0	\\
3.3e-10	0	\\
4.4e-10	0	\\
5.4e-10	0	\\
6.5e-10	0	\\
7.5e-10	0	\\
8.6e-10	0	\\
9.6e-10	0	\\
1.07e-09	0	\\
1.18e-09	0	\\
1.28e-09	0	\\
1.38e-09	0	\\
1.49e-09	0	\\
1.59e-09	0	\\
1.69e-09	0	\\
1.8e-09	0	\\
1.9e-09	0	\\
2.01e-09	0	\\
2.11e-09	0	\\
2.21e-09	0	\\
2.32e-09	0	\\
2.42e-09	0	\\
2.52e-09	0	\\
2.63e-09	0	\\
2.73e-09	0	\\
2.83e-09	0	\\
2.93e-09	0	\\
3.04e-09	0	\\
3.14e-09	0	\\
3.24e-09	0	\\
3.34e-09	0	\\
3.45e-09	0	\\
3.55e-09	0	\\
3.65e-09	0	\\
3.75e-09	0	\\
3.86e-09	0	\\
3.96e-09	0	\\
4.06e-09	0	\\
4.16e-09	0	\\
4.27e-09	0	\\
4.37e-09	0	\\
4.47e-09	0	\\
4.57e-09	0	\\
4.68e-09	0	\\
4.78e-09	0	\\
4.89e-09	0	\\
4.99e-09	0	\\
5e-09	0	\\
};
\addplot [color=blue,solid,forget plot]
  table[row sep=crcr]{
0	0	\\
1.1e-10	0	\\
2.2e-10	0	\\
3.3e-10	0	\\
4.4e-10	0	\\
5.4e-10	0	\\
6.5e-10	0	\\
7.5e-10	0	\\
8.6e-10	0	\\
9.6e-10	0	\\
1.07e-09	0	\\
1.18e-09	0	\\
1.28e-09	0	\\
1.38e-09	0	\\
1.49e-09	0	\\
1.59e-09	0	\\
1.69e-09	0	\\
1.8e-09	0	\\
1.9e-09	0	\\
2.01e-09	0	\\
2.11e-09	0	\\
2.21e-09	0	\\
2.32e-09	0	\\
2.42e-09	0	\\
2.52e-09	0	\\
2.63e-09	0	\\
2.73e-09	0	\\
2.83e-09	0	\\
2.93e-09	0	\\
3.04e-09	0	\\
3.14e-09	0	\\
3.24e-09	0	\\
3.34e-09	0	\\
3.45e-09	0	\\
3.55e-09	0	\\
3.65e-09	0	\\
3.75e-09	0	\\
3.86e-09	0	\\
3.96e-09	0	\\
4.06e-09	0	\\
4.16e-09	0	\\
4.27e-09	0	\\
4.37e-09	0	\\
4.47e-09	0	\\
4.57e-09	0	\\
4.68e-09	0	\\
4.78e-09	0	\\
4.89e-09	0	\\
4.99e-09	0	\\
5e-09	0	\\
};
\addplot [color=black!50!green,solid,forget plot]
  table[row sep=crcr]{
0	0	\\
1.1e-10	0	\\
2.2e-10	0	\\
3.3e-10	0	\\
4.4e-10	0	\\
5.4e-10	0	\\
6.5e-10	0	\\
7.5e-10	0	\\
8.6e-10	0	\\
9.6e-10	0	\\
1.07e-09	0	\\
1.18e-09	0	\\
1.28e-09	0	\\
1.38e-09	0	\\
1.49e-09	0	\\
1.59e-09	0	\\
1.69e-09	0	\\
1.8e-09	0	\\
1.9e-09	0	\\
2.01e-09	0	\\
2.11e-09	0	\\
2.21e-09	0	\\
2.32e-09	0	\\
2.42e-09	0	\\
2.52e-09	0	\\
2.63e-09	0	\\
2.73e-09	0	\\
2.83e-09	0	\\
2.93e-09	0	\\
3.04e-09	0	\\
3.14e-09	0	\\
3.24e-09	0	\\
3.34e-09	0	\\
3.45e-09	0	\\
3.55e-09	0	\\
3.65e-09	0	\\
3.75e-09	0	\\
3.86e-09	0	\\
3.96e-09	0	\\
4.06e-09	0	\\
4.16e-09	0	\\
4.27e-09	0	\\
4.37e-09	0	\\
4.47e-09	0	\\
4.57e-09	0	\\
4.68e-09	0	\\
4.78e-09	0	\\
4.89e-09	0	\\
4.99e-09	0	\\
5e-09	0	\\
};
\addplot [color=red,solid,forget plot]
  table[row sep=crcr]{
0	0	\\
1.1e-10	0	\\
2.2e-10	0	\\
3.3e-10	0	\\
4.4e-10	0	\\
5.4e-10	0	\\
6.5e-10	0	\\
7.5e-10	0	\\
8.6e-10	0	\\
9.6e-10	0	\\
1.07e-09	0	\\
1.18e-09	0	\\
1.28e-09	0	\\
1.38e-09	0	\\
1.49e-09	0	\\
1.59e-09	0	\\
1.69e-09	0	\\
1.8e-09	0	\\
1.9e-09	0	\\
2.01e-09	0	\\
2.11e-09	0	\\
2.21e-09	0	\\
2.32e-09	0	\\
2.42e-09	0	\\
2.52e-09	0	\\
2.63e-09	0	\\
2.73e-09	0	\\
2.83e-09	0	\\
2.93e-09	0	\\
3.04e-09	0	\\
3.14e-09	0	\\
3.24e-09	0	\\
3.34e-09	0	\\
3.45e-09	0	\\
3.55e-09	0	\\
3.65e-09	0	\\
3.75e-09	0	\\
3.86e-09	0	\\
3.96e-09	0	\\
4.06e-09	0	\\
4.16e-09	0	\\
4.27e-09	0	\\
4.37e-09	0	\\
4.47e-09	0	\\
4.57e-09	0	\\
4.68e-09	0	\\
4.78e-09	0	\\
4.89e-09	0	\\
4.99e-09	0	\\
5e-09	0	\\
};
\addplot [color=mycolor1,solid,forget plot]
  table[row sep=crcr]{
0	0	\\
1.1e-10	0	\\
2.2e-10	0	\\
3.3e-10	0	\\
4.4e-10	0	\\
5.4e-10	0	\\
6.5e-10	0	\\
7.5e-10	0	\\
8.6e-10	0	\\
9.6e-10	0	\\
1.07e-09	0	\\
1.18e-09	0	\\
1.28e-09	0	\\
1.38e-09	0	\\
1.49e-09	0	\\
1.59e-09	0	\\
1.69e-09	0	\\
1.8e-09	0	\\
1.9e-09	0	\\
2.01e-09	0	\\
2.11e-09	0	\\
2.21e-09	0	\\
2.32e-09	0	\\
2.42e-09	0	\\
2.52e-09	0	\\
2.63e-09	0	\\
2.73e-09	0	\\
2.83e-09	0	\\
2.93e-09	0	\\
3.04e-09	0	\\
3.14e-09	0	\\
3.24e-09	0	\\
3.34e-09	0	\\
3.45e-09	0	\\
3.55e-09	0	\\
3.65e-09	0	\\
3.75e-09	0	\\
3.86e-09	0	\\
3.96e-09	0	\\
4.06e-09	0	\\
4.16e-09	0	\\
4.27e-09	0	\\
4.37e-09	0	\\
4.47e-09	0	\\
4.57e-09	0	\\
4.68e-09	0	\\
4.78e-09	0	\\
4.89e-09	0	\\
4.99e-09	0	\\
5e-09	0	\\
};
\addplot [color=mycolor2,solid,forget plot]
  table[row sep=crcr]{
0	0	\\
1.1e-10	0	\\
2.2e-10	0	\\
3.3e-10	0	\\
4.4e-10	0	\\
5.4e-10	0	\\
6.5e-10	0	\\
7.5e-10	0	\\
8.6e-10	0	\\
9.6e-10	0	\\
1.07e-09	0	\\
1.18e-09	0	\\
1.28e-09	0	\\
1.38e-09	0	\\
1.49e-09	0	\\
1.59e-09	0	\\
1.69e-09	0	\\
1.8e-09	0	\\
1.9e-09	0	\\
2.01e-09	0	\\
2.11e-09	0	\\
2.21e-09	0	\\
2.32e-09	0	\\
2.42e-09	0	\\
2.52e-09	0	\\
2.63e-09	0	\\
2.73e-09	0	\\
2.83e-09	0	\\
2.93e-09	0	\\
3.04e-09	0	\\
3.14e-09	0	\\
3.24e-09	0	\\
3.34e-09	0	\\
3.45e-09	0	\\
3.55e-09	0	\\
3.65e-09	0	\\
3.75e-09	0	\\
3.86e-09	0	\\
3.96e-09	0	\\
4.06e-09	0	\\
4.16e-09	0	\\
4.27e-09	0	\\
4.37e-09	0	\\
4.47e-09	0	\\
4.57e-09	0	\\
4.68e-09	0	\\
4.78e-09	0	\\
4.89e-09	0	\\
4.99e-09	0	\\
5e-09	0	\\
};
\addplot [color=mycolor3,solid,forget plot]
  table[row sep=crcr]{
0	0	\\
1.1e-10	0	\\
2.2e-10	0	\\
3.3e-10	0	\\
4.4e-10	0	\\
5.4e-10	0	\\
6.5e-10	0	\\
7.5e-10	0	\\
8.6e-10	0	\\
9.6e-10	0	\\
1.07e-09	0	\\
1.18e-09	0	\\
1.28e-09	0	\\
1.38e-09	0	\\
1.49e-09	0	\\
1.59e-09	0	\\
1.69e-09	0	\\
1.8e-09	0	\\
1.9e-09	0	\\
2.01e-09	0	\\
2.11e-09	0	\\
2.21e-09	0	\\
2.32e-09	0	\\
2.42e-09	0	\\
2.52e-09	0	\\
2.63e-09	0	\\
2.73e-09	0	\\
2.83e-09	0	\\
2.93e-09	0	\\
3.04e-09	0	\\
3.14e-09	0	\\
3.24e-09	0	\\
3.34e-09	0	\\
3.45e-09	0	\\
3.55e-09	0	\\
3.65e-09	0	\\
3.75e-09	0	\\
3.86e-09	0	\\
3.96e-09	0	\\
4.06e-09	0	\\
4.16e-09	0	\\
4.27e-09	0	\\
4.37e-09	0	\\
4.47e-09	0	\\
4.57e-09	0	\\
4.68e-09	0	\\
4.78e-09	0	\\
4.89e-09	0	\\
4.99e-09	0	\\
5e-09	0	\\
};
\addplot [color=darkgray,solid,forget plot]
  table[row sep=crcr]{
0	0	\\
1.1e-10	0	\\
2.2e-10	0	\\
3.3e-10	0	\\
4.4e-10	0	\\
5.4e-10	0	\\
6.5e-10	0	\\
7.5e-10	0	\\
8.6e-10	0	\\
9.6e-10	0	\\
1.07e-09	0	\\
1.18e-09	0	\\
1.28e-09	0	\\
1.38e-09	0	\\
1.49e-09	0	\\
1.59e-09	0	\\
1.69e-09	0	\\
1.8e-09	0	\\
1.9e-09	0	\\
2.01e-09	0	\\
2.11e-09	0	\\
2.21e-09	0	\\
2.32e-09	0	\\
2.42e-09	0	\\
2.52e-09	0	\\
2.63e-09	0	\\
2.73e-09	0	\\
2.83e-09	0	\\
2.93e-09	0	\\
3.04e-09	0	\\
3.14e-09	0	\\
3.24e-09	0	\\
3.34e-09	0	\\
3.45e-09	0	\\
3.55e-09	0	\\
3.65e-09	0	\\
3.75e-09	0	\\
3.86e-09	0	\\
3.96e-09	0	\\
4.06e-09	0	\\
4.16e-09	0	\\
4.27e-09	0	\\
4.37e-09	0	\\
4.47e-09	0	\\
4.57e-09	0	\\
4.68e-09	0	\\
4.78e-09	0	\\
4.89e-09	0	\\
4.99e-09	0	\\
5e-09	0	\\
};
\addplot [color=blue,solid,forget plot]
  table[row sep=crcr]{
0	0	\\
1.1e-10	0	\\
2.2e-10	0	\\
3.3e-10	0	\\
4.4e-10	0	\\
5.4e-10	0	\\
6.5e-10	0	\\
7.5e-10	0	\\
8.6e-10	0	\\
9.6e-10	0	\\
1.07e-09	0	\\
1.18e-09	0	\\
1.28e-09	0	\\
1.38e-09	0	\\
1.49e-09	0	\\
1.59e-09	0	\\
1.69e-09	0	\\
1.8e-09	0	\\
1.9e-09	0	\\
2.01e-09	0	\\
2.11e-09	0	\\
2.21e-09	0	\\
2.32e-09	0	\\
2.42e-09	0	\\
2.52e-09	0	\\
2.63e-09	0	\\
2.73e-09	0	\\
2.83e-09	0	\\
2.93e-09	0	\\
3.04e-09	0	\\
3.14e-09	0	\\
3.24e-09	0	\\
3.34e-09	0	\\
3.45e-09	0	\\
3.55e-09	0	\\
3.65e-09	0	\\
3.75e-09	0	\\
3.86e-09	0	\\
3.96e-09	0	\\
4.06e-09	0	\\
4.16e-09	0	\\
4.27e-09	0	\\
4.37e-09	0	\\
4.47e-09	0	\\
4.57e-09	0	\\
4.68e-09	0	\\
4.78e-09	0	\\
4.89e-09	0	\\
4.99e-09	0	\\
5e-09	0	\\
};
\addplot [color=black!50!green,solid,forget plot]
  table[row sep=crcr]{
0	0	\\
1.1e-10	0	\\
2.2e-10	0	\\
3.3e-10	0	\\
4.4e-10	0	\\
5.4e-10	0	\\
6.5e-10	0	\\
7.5e-10	0	\\
8.6e-10	0	\\
9.6e-10	0	\\
1.07e-09	0	\\
1.18e-09	0	\\
1.28e-09	0	\\
1.38e-09	0	\\
1.49e-09	0	\\
1.59e-09	0	\\
1.69e-09	0	\\
1.8e-09	0	\\
1.9e-09	0	\\
2.01e-09	0	\\
2.11e-09	0	\\
2.21e-09	0	\\
2.32e-09	0	\\
2.42e-09	0	\\
2.52e-09	0	\\
2.63e-09	0	\\
2.73e-09	0	\\
2.83e-09	0	\\
2.93e-09	0	\\
3.04e-09	0	\\
3.14e-09	0	\\
3.24e-09	0	\\
3.34e-09	0	\\
3.45e-09	0	\\
3.55e-09	0	\\
3.65e-09	0	\\
3.75e-09	0	\\
3.86e-09	0	\\
3.96e-09	0	\\
4.06e-09	0	\\
4.16e-09	0	\\
4.27e-09	0	\\
4.37e-09	0	\\
4.47e-09	0	\\
4.57e-09	0	\\
4.68e-09	0	\\
4.78e-09	0	\\
4.89e-09	0	\\
4.99e-09	0	\\
5e-09	0	\\
};
\addplot [color=red,solid,forget plot]
  table[row sep=crcr]{
0	0	\\
1.1e-10	0	\\
2.2e-10	0	\\
3.3e-10	0	\\
4.4e-10	0	\\
5.4e-10	0	\\
6.5e-10	0	\\
7.5e-10	0	\\
8.6e-10	0	\\
9.6e-10	0	\\
1.07e-09	0	\\
1.18e-09	0	\\
1.28e-09	0	\\
1.38e-09	0	\\
1.49e-09	0	\\
1.59e-09	0	\\
1.69e-09	0	\\
1.8e-09	0	\\
1.9e-09	0	\\
2.01e-09	0	\\
2.11e-09	0	\\
2.21e-09	0	\\
2.32e-09	0	\\
2.42e-09	0	\\
2.52e-09	0	\\
2.63e-09	0	\\
2.73e-09	0	\\
2.83e-09	0	\\
2.93e-09	0	\\
3.04e-09	0	\\
3.14e-09	0	\\
3.24e-09	0	\\
3.34e-09	0	\\
3.45e-09	0	\\
3.55e-09	0	\\
3.65e-09	0	\\
3.75e-09	0	\\
3.86e-09	0	\\
3.96e-09	0	\\
4.06e-09	0	\\
4.16e-09	0	\\
4.27e-09	0	\\
4.37e-09	0	\\
4.47e-09	0	\\
4.57e-09	0	\\
4.68e-09	0	\\
4.78e-09	0	\\
4.89e-09	0	\\
4.99e-09	0	\\
5e-09	0	\\
};
\addplot [color=mycolor1,solid,forget plot]
  table[row sep=crcr]{
0	0	\\
1.1e-10	0	\\
2.2e-10	0	\\
3.3e-10	0	\\
4.4e-10	0	\\
5.4e-10	0	\\
6.5e-10	0	\\
7.5e-10	0	\\
8.6e-10	0	\\
9.6e-10	0	\\
1.07e-09	0	\\
1.18e-09	0	\\
1.28e-09	0	\\
1.38e-09	0	\\
1.49e-09	0	\\
1.59e-09	0	\\
1.69e-09	0	\\
1.8e-09	0	\\
1.9e-09	0	\\
2.01e-09	0	\\
2.11e-09	0	\\
2.21e-09	0	\\
2.32e-09	0	\\
2.42e-09	0	\\
2.52e-09	0	\\
2.63e-09	0	\\
2.73e-09	0	\\
2.83e-09	0	\\
2.93e-09	0	\\
3.04e-09	0	\\
3.14e-09	0	\\
3.24e-09	0	\\
3.34e-09	0	\\
3.45e-09	0	\\
3.55e-09	0	\\
3.65e-09	0	\\
3.75e-09	0	\\
3.86e-09	0	\\
3.96e-09	0	\\
4.06e-09	0	\\
4.16e-09	0	\\
4.27e-09	0	\\
4.37e-09	0	\\
4.47e-09	0	\\
4.57e-09	0	\\
4.68e-09	0	\\
4.78e-09	0	\\
4.89e-09	0	\\
4.99e-09	0	\\
5e-09	0	\\
};
\addplot [color=mycolor2,solid,forget plot]
  table[row sep=crcr]{
0	0	\\
1.1e-10	0	\\
2.2e-10	0	\\
3.3e-10	0	\\
4.4e-10	0	\\
5.4e-10	0	\\
6.5e-10	0	\\
7.5e-10	0	\\
8.6e-10	0	\\
9.6e-10	0	\\
1.07e-09	0	\\
1.18e-09	0	\\
1.28e-09	0	\\
1.38e-09	0	\\
1.49e-09	0	\\
1.59e-09	0	\\
1.69e-09	0	\\
1.8e-09	0	\\
1.9e-09	0	\\
2.01e-09	0	\\
2.11e-09	0	\\
2.21e-09	0	\\
2.32e-09	0	\\
2.42e-09	0	\\
2.52e-09	0	\\
2.63e-09	0	\\
2.73e-09	0	\\
2.83e-09	0	\\
2.93e-09	0	\\
3.04e-09	0	\\
3.14e-09	0	\\
3.24e-09	0	\\
3.34e-09	0	\\
3.45e-09	0	\\
3.55e-09	0	\\
3.65e-09	0	\\
3.75e-09	0	\\
3.86e-09	0	\\
3.96e-09	0	\\
4.06e-09	0	\\
4.16e-09	0	\\
4.27e-09	0	\\
4.37e-09	0	\\
4.47e-09	0	\\
4.57e-09	0	\\
4.68e-09	0	\\
4.78e-09	0	\\
4.89e-09	0	\\
4.99e-09	0	\\
5e-09	0	\\
};
\addplot [color=mycolor3,solid,forget plot]
  table[row sep=crcr]{
0	0	\\
1.1e-10	0	\\
2.2e-10	0	\\
3.3e-10	0	\\
4.4e-10	0	\\
5.4e-10	0	\\
6.5e-10	0	\\
7.5e-10	0	\\
8.6e-10	0	\\
9.6e-10	0	\\
1.07e-09	0	\\
1.18e-09	0	\\
1.28e-09	0	\\
1.38e-09	0	\\
1.49e-09	0	\\
1.59e-09	0	\\
1.69e-09	0	\\
1.8e-09	0	\\
1.9e-09	0	\\
2.01e-09	0	\\
2.11e-09	0	\\
2.21e-09	0	\\
2.32e-09	0	\\
2.42e-09	0	\\
2.52e-09	0	\\
2.63e-09	0	\\
2.73e-09	0	\\
2.83e-09	0	\\
2.93e-09	0	\\
3.04e-09	0	\\
3.14e-09	0	\\
3.24e-09	0	\\
3.34e-09	0	\\
3.45e-09	0	\\
3.55e-09	0	\\
3.65e-09	0	\\
3.75e-09	0	\\
3.86e-09	0	\\
3.96e-09	0	\\
4.06e-09	0	\\
4.16e-09	0	\\
4.27e-09	0	\\
4.37e-09	0	\\
4.47e-09	0	\\
4.57e-09	0	\\
4.68e-09	0	\\
4.78e-09	0	\\
4.89e-09	0	\\
4.99e-09	0	\\
5e-09	0	\\
};
\addplot [color=darkgray,solid,forget plot]
  table[row sep=crcr]{
0	0	\\
1.1e-10	0	\\
2.2e-10	0	\\
3.3e-10	0	\\
4.4e-10	0	\\
5.4e-10	0	\\
6.5e-10	0	\\
7.5e-10	0	\\
8.6e-10	0	\\
9.6e-10	0	\\
1.07e-09	0	\\
1.18e-09	0	\\
1.28e-09	0	\\
1.38e-09	0	\\
1.49e-09	0	\\
1.59e-09	0	\\
1.69e-09	0	\\
1.8e-09	0	\\
1.9e-09	0	\\
2.01e-09	0	\\
2.11e-09	0	\\
2.21e-09	0	\\
2.32e-09	0	\\
2.42e-09	0	\\
2.52e-09	0	\\
2.63e-09	0	\\
2.73e-09	0	\\
2.83e-09	0	\\
2.93e-09	0	\\
3.04e-09	0	\\
3.14e-09	0	\\
3.24e-09	0	\\
3.34e-09	0	\\
3.45e-09	0	\\
3.55e-09	0	\\
3.65e-09	0	\\
3.75e-09	0	\\
3.86e-09	0	\\
3.96e-09	0	\\
4.06e-09	0	\\
4.16e-09	0	\\
4.27e-09	0	\\
4.37e-09	0	\\
4.47e-09	0	\\
4.57e-09	0	\\
4.68e-09	0	\\
4.78e-09	0	\\
4.89e-09	0	\\
4.99e-09	0	\\
5e-09	0	\\
};
\addplot [color=blue,solid,forget plot]
  table[row sep=crcr]{
0	0	\\
1.1e-10	0	\\
2.2e-10	0	\\
3.3e-10	0	\\
4.4e-10	0	\\
5.4e-10	0	\\
6.5e-10	0	\\
7.5e-10	0	\\
8.6e-10	0	\\
9.6e-10	0	\\
1.07e-09	0	\\
1.18e-09	0	\\
1.28e-09	0	\\
1.38e-09	0	\\
1.49e-09	0	\\
1.59e-09	0	\\
1.69e-09	0	\\
1.8e-09	0	\\
1.9e-09	0	\\
2.01e-09	0	\\
2.11e-09	0	\\
2.21e-09	0	\\
2.32e-09	0	\\
2.42e-09	0	\\
2.52e-09	0	\\
2.63e-09	0	\\
2.73e-09	0	\\
2.83e-09	0	\\
2.93e-09	0	\\
3.04e-09	0	\\
3.14e-09	0	\\
3.24e-09	0	\\
3.34e-09	0	\\
3.45e-09	0	\\
3.55e-09	0	\\
3.65e-09	0	\\
3.75e-09	0	\\
3.86e-09	0	\\
3.96e-09	0	\\
4.06e-09	0	\\
4.16e-09	0	\\
4.27e-09	0	\\
4.37e-09	0	\\
4.47e-09	0	\\
4.57e-09	0	\\
4.68e-09	0	\\
4.78e-09	0	\\
4.89e-09	0	\\
4.99e-09	0	\\
5e-09	0	\\
};
\addplot [color=black!50!green,solid,forget plot]
  table[row sep=crcr]{
0	0	\\
1.1e-10	0	\\
2.2e-10	0	\\
3.3e-10	0	\\
4.4e-10	0	\\
5.4e-10	0	\\
6.5e-10	0	\\
7.5e-10	0	\\
8.6e-10	0	\\
9.6e-10	0	\\
1.07e-09	0	\\
1.18e-09	0	\\
1.28e-09	0	\\
1.38e-09	0	\\
1.49e-09	0	\\
1.59e-09	0	\\
1.69e-09	0	\\
1.8e-09	0	\\
1.9e-09	0	\\
2.01e-09	0	\\
2.11e-09	0	\\
2.21e-09	0	\\
2.32e-09	0	\\
2.42e-09	0	\\
2.52e-09	0	\\
2.63e-09	0	\\
2.73e-09	0	\\
2.83e-09	0	\\
2.93e-09	0	\\
3.04e-09	0	\\
3.14e-09	0	\\
3.24e-09	0	\\
3.34e-09	0	\\
3.45e-09	0	\\
3.55e-09	0	\\
3.65e-09	0	\\
3.75e-09	0	\\
3.86e-09	0	\\
3.96e-09	0	\\
4.06e-09	0	\\
4.16e-09	0	\\
4.27e-09	0	\\
4.37e-09	0	\\
4.47e-09	0	\\
4.57e-09	0	\\
4.68e-09	0	\\
4.78e-09	0	\\
4.89e-09	0	\\
4.99e-09	0	\\
5e-09	0	\\
};
\addplot [color=red,solid,forget plot]
  table[row sep=crcr]{
0	0	\\
1.1e-10	0	\\
2.2e-10	0	\\
3.3e-10	0	\\
4.4e-10	0	\\
5.4e-10	0	\\
6.5e-10	0	\\
7.5e-10	0	\\
8.6e-10	0	\\
9.6e-10	0	\\
1.07e-09	0	\\
1.18e-09	0	\\
1.28e-09	0	\\
1.38e-09	0	\\
1.49e-09	0	\\
1.59e-09	0	\\
1.69e-09	0	\\
1.8e-09	0	\\
1.9e-09	0	\\
2.01e-09	0	\\
2.11e-09	0	\\
2.21e-09	0	\\
2.32e-09	0	\\
2.42e-09	0	\\
2.52e-09	0	\\
2.63e-09	0	\\
2.73e-09	0	\\
2.83e-09	0	\\
2.93e-09	0	\\
3.04e-09	0	\\
3.14e-09	0	\\
3.24e-09	0	\\
3.34e-09	0	\\
3.45e-09	0	\\
3.55e-09	0	\\
3.65e-09	0	\\
3.75e-09	0	\\
3.86e-09	0	\\
3.96e-09	0	\\
4.06e-09	0	\\
4.16e-09	0	\\
4.27e-09	0	\\
4.37e-09	0	\\
4.47e-09	0	\\
4.57e-09	0	\\
4.68e-09	0	\\
4.78e-09	0	\\
4.89e-09	0	\\
4.99e-09	0	\\
5e-09	0	\\
};
\addplot [color=mycolor1,solid,forget plot]
  table[row sep=crcr]{
0	0	\\
1.1e-10	0	\\
2.2e-10	0	\\
3.3e-10	0	\\
4.4e-10	0	\\
5.4e-10	0	\\
6.5e-10	0	\\
7.5e-10	0	\\
8.6e-10	0	\\
9.6e-10	0	\\
1.07e-09	0	\\
1.18e-09	0	\\
1.28e-09	0	\\
1.38e-09	0	\\
1.49e-09	0	\\
1.59e-09	0	\\
1.69e-09	0	\\
1.8e-09	0	\\
1.9e-09	0	\\
2.01e-09	0	\\
2.11e-09	0	\\
2.21e-09	0	\\
2.32e-09	0	\\
2.42e-09	0	\\
2.52e-09	0	\\
2.63e-09	0	\\
2.73e-09	0	\\
2.83e-09	0	\\
2.93e-09	0	\\
3.04e-09	0	\\
3.14e-09	0	\\
3.24e-09	0	\\
3.34e-09	0	\\
3.45e-09	0	\\
3.55e-09	0	\\
3.65e-09	0	\\
3.75e-09	0	\\
3.86e-09	0	\\
3.96e-09	0	\\
4.06e-09	0	\\
4.16e-09	0	\\
4.27e-09	0	\\
4.37e-09	0	\\
4.47e-09	0	\\
4.57e-09	0	\\
4.68e-09	0	\\
4.78e-09	0	\\
4.89e-09	0	\\
4.99e-09	0	\\
5e-09	0	\\
};
\addplot [color=mycolor2,solid,forget plot]
  table[row sep=crcr]{
0	0	\\
1.1e-10	0	\\
2.2e-10	0	\\
3.3e-10	0	\\
4.4e-10	0	\\
5.4e-10	0	\\
6.5e-10	0	\\
7.5e-10	0	\\
8.6e-10	0	\\
9.6e-10	0	\\
1.07e-09	0	\\
1.18e-09	0	\\
1.28e-09	0	\\
1.38e-09	0	\\
1.49e-09	0	\\
1.59e-09	0	\\
1.69e-09	0	\\
1.8e-09	0	\\
1.9e-09	0	\\
2.01e-09	0	\\
2.11e-09	0	\\
2.21e-09	0	\\
2.32e-09	0	\\
2.42e-09	0	\\
2.52e-09	0	\\
2.63e-09	0	\\
2.73e-09	0	\\
2.83e-09	0	\\
2.93e-09	0	\\
3.04e-09	0	\\
3.14e-09	0	\\
3.24e-09	0	\\
3.34e-09	0	\\
3.45e-09	0	\\
3.55e-09	0	\\
3.65e-09	0	\\
3.75e-09	0	\\
3.86e-09	0	\\
3.96e-09	0	\\
4.06e-09	0	\\
4.16e-09	0	\\
4.27e-09	0	\\
4.37e-09	0	\\
4.47e-09	0	\\
4.57e-09	0	\\
4.68e-09	0	\\
4.78e-09	0	\\
4.89e-09	0	\\
4.99e-09	0	\\
5e-09	0	\\
};
\addplot [color=mycolor3,solid,forget plot]
  table[row sep=crcr]{
0	0	\\
1.1e-10	0	\\
2.2e-10	0	\\
3.3e-10	0	\\
4.4e-10	0	\\
5.4e-10	0	\\
6.5e-10	0	\\
7.5e-10	0	\\
8.6e-10	0	\\
9.6e-10	0	\\
1.07e-09	0	\\
1.18e-09	0	\\
1.28e-09	0	\\
1.38e-09	0	\\
1.49e-09	0	\\
1.59e-09	0	\\
1.69e-09	0	\\
1.8e-09	0	\\
1.9e-09	0	\\
2.01e-09	0	\\
2.11e-09	0	\\
2.21e-09	0	\\
2.32e-09	0	\\
2.42e-09	0	\\
2.52e-09	0	\\
2.63e-09	0	\\
2.73e-09	0	\\
2.83e-09	0	\\
2.93e-09	0	\\
3.04e-09	0	\\
3.14e-09	0	\\
3.24e-09	0	\\
3.34e-09	0	\\
3.45e-09	0	\\
3.55e-09	0	\\
3.65e-09	0	\\
3.75e-09	0	\\
3.86e-09	0	\\
3.96e-09	0	\\
4.06e-09	0	\\
4.16e-09	0	\\
4.27e-09	0	\\
4.37e-09	0	\\
4.47e-09	0	\\
4.57e-09	0	\\
4.68e-09	0	\\
4.78e-09	0	\\
4.89e-09	0	\\
4.99e-09	0	\\
5e-09	0	\\
};
\addplot [color=darkgray,solid,forget plot]
  table[row sep=crcr]{
0	0	\\
1.1e-10	0	\\
2.2e-10	0	\\
3.3e-10	0	\\
4.4e-10	0	\\
5.4e-10	0	\\
6.5e-10	0	\\
7.5e-10	0	\\
8.6e-10	0	\\
9.6e-10	0	\\
1.07e-09	0	\\
1.18e-09	0	\\
1.28e-09	0	\\
1.38e-09	0	\\
1.49e-09	0	\\
1.59e-09	0	\\
1.69e-09	0	\\
1.8e-09	0	\\
1.9e-09	0	\\
2.01e-09	0	\\
2.11e-09	0	\\
2.21e-09	0	\\
2.32e-09	0	\\
2.42e-09	0	\\
2.52e-09	0	\\
2.63e-09	0	\\
2.73e-09	0	\\
2.83e-09	0	\\
2.93e-09	0	\\
3.04e-09	0	\\
3.14e-09	0	\\
3.24e-09	0	\\
3.34e-09	0	\\
3.45e-09	0	\\
3.55e-09	0	\\
3.65e-09	0	\\
3.75e-09	0	\\
3.86e-09	0	\\
3.96e-09	0	\\
4.06e-09	0	\\
4.16e-09	0	\\
4.27e-09	0	\\
4.37e-09	0	\\
4.47e-09	0	\\
4.57e-09	0	\\
4.68e-09	0	\\
4.78e-09	0	\\
4.89e-09	0	\\
4.99e-09	0	\\
5e-09	0	\\
};
\addplot [color=blue,solid,forget plot]
  table[row sep=crcr]{
0	0	\\
1.1e-10	0	\\
2.2e-10	0	\\
3.3e-10	0	\\
4.4e-10	0	\\
5.4e-10	0	\\
6.5e-10	0	\\
7.5e-10	0	\\
8.6e-10	0	\\
9.6e-10	0	\\
1.07e-09	0	\\
1.18e-09	0	\\
1.28e-09	0	\\
1.38e-09	0	\\
1.49e-09	0	\\
1.59e-09	0	\\
1.69e-09	0	\\
1.8e-09	0	\\
1.9e-09	0	\\
2.01e-09	0	\\
2.11e-09	0	\\
2.21e-09	0	\\
2.32e-09	0	\\
2.42e-09	0	\\
2.52e-09	0	\\
2.63e-09	0	\\
2.73e-09	0	\\
2.83e-09	0	\\
2.93e-09	0	\\
3.04e-09	0	\\
3.14e-09	0	\\
3.24e-09	0	\\
3.34e-09	0	\\
3.45e-09	0	\\
3.55e-09	0	\\
3.65e-09	0	\\
3.75e-09	0	\\
3.86e-09	0	\\
3.96e-09	0	\\
4.06e-09	0	\\
4.16e-09	0	\\
4.27e-09	0	\\
4.37e-09	0	\\
4.47e-09	0	\\
4.57e-09	0	\\
4.68e-09	0	\\
4.78e-09	0	\\
4.89e-09	0	\\
4.99e-09	0	\\
5e-09	0	\\
};
\addplot [color=black!50!green,solid,forget plot]
  table[row sep=crcr]{
0	0	\\
1.1e-10	0	\\
2.2e-10	0	\\
3.3e-10	0	\\
4.4e-10	0	\\
5.4e-10	0	\\
6.5e-10	0	\\
7.5e-10	0	\\
8.6e-10	0	\\
9.6e-10	0	\\
1.07e-09	0	\\
1.18e-09	0	\\
1.28e-09	0	\\
1.38e-09	0	\\
1.49e-09	0	\\
1.59e-09	0	\\
1.69e-09	0	\\
1.8e-09	0	\\
1.9e-09	0	\\
2.01e-09	0	\\
2.11e-09	0	\\
2.21e-09	0	\\
2.32e-09	0	\\
2.42e-09	0	\\
2.52e-09	0	\\
2.63e-09	0	\\
2.73e-09	0	\\
2.83e-09	0	\\
2.93e-09	0	\\
3.04e-09	0	\\
3.14e-09	0	\\
3.24e-09	0	\\
3.34e-09	0	\\
3.45e-09	0	\\
3.55e-09	0	\\
3.65e-09	0	\\
3.75e-09	0	\\
3.86e-09	0	\\
3.96e-09	0	\\
4.06e-09	0	\\
4.16e-09	0	\\
4.27e-09	0	\\
4.37e-09	0	\\
4.47e-09	0	\\
4.57e-09	0	\\
4.68e-09	0	\\
4.78e-09	0	\\
4.89e-09	0	\\
4.99e-09	0	\\
5e-09	0	\\
};
\addplot [color=red,solid,forget plot]
  table[row sep=crcr]{
0	0	\\
1.1e-10	0	\\
2.2e-10	0	\\
3.3e-10	0	\\
4.4e-10	0	\\
5.4e-10	0	\\
6.5e-10	0	\\
7.5e-10	0	\\
8.6e-10	0	\\
9.6e-10	0	\\
1.07e-09	0	\\
1.18e-09	0	\\
1.28e-09	0	\\
1.38e-09	0	\\
1.49e-09	0	\\
1.59e-09	0	\\
1.69e-09	0	\\
1.8e-09	0	\\
1.9e-09	0	\\
2.01e-09	0	\\
2.11e-09	0	\\
2.21e-09	0	\\
2.32e-09	0	\\
2.42e-09	0	\\
2.52e-09	0	\\
2.63e-09	0	\\
2.73e-09	0	\\
2.83e-09	0	\\
2.93e-09	0	\\
3.04e-09	0	\\
3.14e-09	0	\\
3.24e-09	0	\\
3.34e-09	0	\\
3.45e-09	0	\\
3.55e-09	0	\\
3.65e-09	0	\\
3.75e-09	0	\\
3.86e-09	0	\\
3.96e-09	0	\\
4.06e-09	0	\\
4.16e-09	0	\\
4.27e-09	0	\\
4.37e-09	0	\\
4.47e-09	0	\\
4.57e-09	0	\\
4.68e-09	0	\\
4.78e-09	0	\\
4.89e-09	0	\\
4.99e-09	0	\\
5e-09	0	\\
};
\addplot [color=mycolor1,solid,forget plot]
  table[row sep=crcr]{
0	0	\\
1.1e-10	0	\\
2.2e-10	0	\\
3.3e-10	0	\\
4.4e-10	0	\\
5.4e-10	0	\\
6.5e-10	0	\\
7.5e-10	0	\\
8.6e-10	0	\\
9.6e-10	0	\\
1.07e-09	0	\\
1.18e-09	0	\\
1.28e-09	0	\\
1.38e-09	0	\\
1.49e-09	0	\\
1.59e-09	0	\\
1.69e-09	0	\\
1.8e-09	0	\\
1.9e-09	0	\\
2.01e-09	0	\\
2.11e-09	0	\\
2.21e-09	0	\\
2.32e-09	0	\\
2.42e-09	0	\\
2.52e-09	0	\\
2.63e-09	0	\\
2.73e-09	0	\\
2.83e-09	0	\\
2.93e-09	0	\\
3.04e-09	0	\\
3.14e-09	0	\\
3.24e-09	0	\\
3.34e-09	0	\\
3.45e-09	0	\\
3.55e-09	0	\\
3.65e-09	0	\\
3.75e-09	0	\\
3.86e-09	0	\\
3.96e-09	0	\\
4.06e-09	0	\\
4.16e-09	0	\\
4.27e-09	0	\\
4.37e-09	0	\\
4.47e-09	0	\\
4.57e-09	0	\\
4.68e-09	0	\\
4.78e-09	0	\\
4.89e-09	0	\\
4.99e-09	0	\\
5e-09	0	\\
};
\addplot [color=mycolor2,solid,forget plot]
  table[row sep=crcr]{
0	0	\\
1.1e-10	0	\\
2.2e-10	0	\\
3.3e-10	0	\\
4.4e-10	0	\\
5.4e-10	0	\\
6.5e-10	0	\\
7.5e-10	0	\\
8.6e-10	0	\\
9.6e-10	0	\\
1.07e-09	0	\\
1.18e-09	0	\\
1.28e-09	0	\\
1.38e-09	0	\\
1.49e-09	0	\\
1.59e-09	0	\\
1.69e-09	0	\\
1.8e-09	0	\\
1.9e-09	0	\\
2.01e-09	0	\\
2.11e-09	0	\\
2.21e-09	0	\\
2.32e-09	0	\\
2.42e-09	0	\\
2.52e-09	0	\\
2.63e-09	0	\\
2.73e-09	0	\\
2.83e-09	0	\\
2.93e-09	0	\\
3.04e-09	0	\\
3.14e-09	0	\\
3.24e-09	0	\\
3.34e-09	0	\\
3.45e-09	0	\\
3.55e-09	0	\\
3.65e-09	0	\\
3.75e-09	0	\\
3.86e-09	0	\\
3.96e-09	0	\\
4.06e-09	0	\\
4.16e-09	0	\\
4.27e-09	0	\\
4.37e-09	0	\\
4.47e-09	0	\\
4.57e-09	0	\\
4.68e-09	0	\\
4.78e-09	0	\\
4.89e-09	0	\\
4.99e-09	0	\\
5e-09	0	\\
};
\addplot [color=mycolor3,solid,forget plot]
  table[row sep=crcr]{
0	0	\\
1.1e-10	0	\\
2.2e-10	0	\\
3.3e-10	0	\\
4.4e-10	0	\\
5.4e-10	0	\\
6.5e-10	0	\\
7.5e-10	0	\\
8.6e-10	0	\\
9.6e-10	0	\\
1.07e-09	0	\\
1.18e-09	0	\\
1.28e-09	0	\\
1.38e-09	0	\\
1.49e-09	0	\\
1.59e-09	0	\\
1.69e-09	0	\\
1.8e-09	0	\\
1.9e-09	0	\\
2.01e-09	0	\\
2.11e-09	0	\\
2.21e-09	0	\\
2.32e-09	0	\\
2.42e-09	0	\\
2.52e-09	0	\\
2.63e-09	0	\\
2.73e-09	0	\\
2.83e-09	0	\\
2.93e-09	0	\\
3.04e-09	0	\\
3.14e-09	0	\\
3.24e-09	0	\\
3.34e-09	0	\\
3.45e-09	0	\\
3.55e-09	0	\\
3.65e-09	0	\\
3.75e-09	0	\\
3.86e-09	0	\\
3.96e-09	0	\\
4.06e-09	0	\\
4.16e-09	0	\\
4.27e-09	0	\\
4.37e-09	0	\\
4.47e-09	0	\\
4.57e-09	0	\\
4.68e-09	0	\\
4.78e-09	0	\\
4.89e-09	0	\\
4.99e-09	0	\\
5e-09	0	\\
};
\addplot [color=darkgray,solid,forget plot]
  table[row sep=crcr]{
0	0	\\
1.1e-10	0	\\
2.2e-10	0	\\
3.3e-10	0	\\
4.4e-10	0	\\
5.4e-10	0	\\
6.5e-10	0	\\
7.5e-10	0	\\
8.6e-10	0	\\
9.6e-10	0	\\
1.07e-09	0	\\
1.18e-09	0	\\
1.28e-09	0	\\
1.38e-09	0	\\
1.49e-09	0	\\
1.59e-09	0	\\
1.69e-09	0	\\
1.8e-09	0	\\
1.9e-09	0	\\
2.01e-09	0	\\
2.11e-09	0	\\
2.21e-09	0	\\
2.32e-09	0	\\
2.42e-09	0	\\
2.52e-09	0	\\
2.63e-09	0	\\
2.73e-09	0	\\
2.83e-09	0	\\
2.93e-09	0	\\
3.04e-09	0	\\
3.14e-09	0	\\
3.24e-09	0	\\
3.34e-09	0	\\
3.45e-09	0	\\
3.55e-09	0	\\
3.65e-09	0	\\
3.75e-09	0	\\
3.86e-09	0	\\
3.96e-09	0	\\
4.06e-09	0	\\
4.16e-09	0	\\
4.27e-09	0	\\
4.37e-09	0	\\
4.47e-09	0	\\
4.57e-09	0	\\
4.68e-09	0	\\
4.78e-09	0	\\
4.89e-09	0	\\
4.99e-09	0	\\
5e-09	0	\\
};
\addplot [color=blue,solid,forget plot]
  table[row sep=crcr]{
0	0	\\
1.1e-10	0	\\
2.2e-10	0	\\
3.3e-10	0	\\
4.4e-10	0	\\
5.4e-10	0	\\
6.5e-10	0	\\
7.5e-10	0	\\
8.6e-10	0	\\
9.6e-10	0	\\
1.07e-09	0	\\
1.18e-09	0	\\
1.28e-09	0	\\
1.38e-09	0	\\
1.49e-09	0	\\
1.59e-09	0	\\
1.69e-09	0	\\
1.8e-09	0	\\
1.9e-09	0	\\
2.01e-09	0	\\
2.11e-09	0	\\
2.21e-09	0	\\
2.32e-09	0	\\
2.42e-09	0	\\
2.52e-09	0	\\
2.63e-09	0	\\
2.73e-09	0	\\
2.83e-09	0	\\
2.93e-09	0	\\
3.04e-09	0	\\
3.14e-09	0	\\
3.24e-09	0	\\
3.34e-09	0	\\
3.45e-09	0	\\
3.55e-09	0	\\
3.65e-09	0	\\
3.75e-09	0	\\
3.86e-09	0	\\
3.96e-09	0	\\
4.06e-09	0	\\
4.16e-09	0	\\
4.27e-09	0	\\
4.37e-09	0	\\
4.47e-09	0	\\
4.57e-09	0	\\
4.68e-09	0	\\
4.78e-09	0	\\
4.89e-09	0	\\
4.99e-09	0	\\
5e-09	0	\\
};
\addplot [color=black!50!green,solid,forget plot]
  table[row sep=crcr]{
0	0	\\
1.1e-10	0	\\
2.2e-10	0	\\
3.3e-10	0	\\
4.4e-10	0	\\
5.4e-10	0	\\
6.5e-10	0	\\
7.5e-10	0	\\
8.6e-10	0	\\
9.6e-10	0	\\
1.07e-09	0	\\
1.18e-09	0	\\
1.28e-09	0	\\
1.38e-09	0	\\
1.49e-09	0	\\
1.59e-09	0	\\
1.69e-09	0	\\
1.8e-09	0	\\
1.9e-09	0	\\
2.01e-09	0	\\
2.11e-09	0	\\
2.21e-09	0	\\
2.32e-09	0	\\
2.42e-09	0	\\
2.52e-09	0	\\
2.63e-09	0	\\
2.73e-09	0	\\
2.83e-09	0	\\
2.93e-09	0	\\
3.04e-09	0	\\
3.14e-09	0	\\
3.24e-09	0	\\
3.34e-09	0	\\
3.45e-09	0	\\
3.55e-09	0	\\
3.65e-09	0	\\
3.75e-09	0	\\
3.86e-09	0	\\
3.96e-09	0	\\
4.06e-09	0	\\
4.16e-09	0	\\
4.27e-09	0	\\
4.37e-09	0	\\
4.47e-09	0	\\
4.57e-09	0	\\
4.68e-09	0	\\
4.78e-09	0	\\
4.89e-09	0	\\
4.99e-09	0	\\
5e-09	0	\\
};
\addplot [color=red,solid,forget plot]
  table[row sep=crcr]{
0	0	\\
1.1e-10	0	\\
2.2e-10	0	\\
3.3e-10	0	\\
4.4e-10	0	\\
5.4e-10	0	\\
6.5e-10	0	\\
7.5e-10	0	\\
8.6e-10	0	\\
9.6e-10	0	\\
1.07e-09	0	\\
1.18e-09	0	\\
1.28e-09	0	\\
1.38e-09	0	\\
1.49e-09	0	\\
1.59e-09	0	\\
1.69e-09	0	\\
1.8e-09	0	\\
1.9e-09	0	\\
2.01e-09	0	\\
2.11e-09	0	\\
2.21e-09	0	\\
2.32e-09	0	\\
2.42e-09	0	\\
2.52e-09	0	\\
2.63e-09	0	\\
2.73e-09	0	\\
2.83e-09	0	\\
2.93e-09	0	\\
3.04e-09	0	\\
3.14e-09	0	\\
3.24e-09	0	\\
3.34e-09	0	\\
3.45e-09	0	\\
3.55e-09	0	\\
3.65e-09	0	\\
3.75e-09	0	\\
3.86e-09	0	\\
3.96e-09	0	\\
4.06e-09	0	\\
4.16e-09	0	\\
4.27e-09	0	\\
4.37e-09	0	\\
4.47e-09	0	\\
4.57e-09	0	\\
4.68e-09	0	\\
4.78e-09	0	\\
4.89e-09	0	\\
4.99e-09	0	\\
5e-09	0	\\
};
\addplot [color=mycolor1,solid,forget plot]
  table[row sep=crcr]{
0	0	\\
1.1e-10	0	\\
2.2e-10	0	\\
3.3e-10	0	\\
4.4e-10	0	\\
5.4e-10	0	\\
6.5e-10	0	\\
7.5e-10	0	\\
8.6e-10	0	\\
9.6e-10	0	\\
1.07e-09	0	\\
1.18e-09	0	\\
1.28e-09	0	\\
1.38e-09	0	\\
1.49e-09	0	\\
1.59e-09	0	\\
1.69e-09	0	\\
1.8e-09	0	\\
1.9e-09	0	\\
2.01e-09	0	\\
2.11e-09	0	\\
2.21e-09	0	\\
2.32e-09	0	\\
2.42e-09	0	\\
2.52e-09	0	\\
2.63e-09	0	\\
2.73e-09	0	\\
2.83e-09	0	\\
2.93e-09	0	\\
3.04e-09	0	\\
3.14e-09	0	\\
3.24e-09	0	\\
3.34e-09	0	\\
3.45e-09	0	\\
3.55e-09	0	\\
3.65e-09	0	\\
3.75e-09	0	\\
3.86e-09	0	\\
3.96e-09	0	\\
4.06e-09	0	\\
4.16e-09	0	\\
4.27e-09	0	\\
4.37e-09	0	\\
4.47e-09	0	\\
4.57e-09	0	\\
4.68e-09	0	\\
4.78e-09	0	\\
4.89e-09	0	\\
4.99e-09	0	\\
5e-09	0	\\
};
\addplot [color=mycolor2,solid,forget plot]
  table[row sep=crcr]{
0	0	\\
1.1e-10	0	\\
2.2e-10	0	\\
3.3e-10	0	\\
4.4e-10	0	\\
5.4e-10	0	\\
6.5e-10	0	\\
7.5e-10	0	\\
8.6e-10	0	\\
9.6e-10	0	\\
1.07e-09	0	\\
1.18e-09	0	\\
1.28e-09	0	\\
1.38e-09	0	\\
1.49e-09	0	\\
1.59e-09	0	\\
1.69e-09	0	\\
1.8e-09	0	\\
1.9e-09	0	\\
2.01e-09	0	\\
2.11e-09	0	\\
2.21e-09	0	\\
2.32e-09	0	\\
2.42e-09	0	\\
2.52e-09	0	\\
2.63e-09	0	\\
2.73e-09	0	\\
2.83e-09	0	\\
2.93e-09	0	\\
3.04e-09	0	\\
3.14e-09	0	\\
3.24e-09	0	\\
3.34e-09	0	\\
3.45e-09	0	\\
3.55e-09	0	\\
3.65e-09	0	\\
3.75e-09	0	\\
3.86e-09	0	\\
3.96e-09	0	\\
4.06e-09	0	\\
4.16e-09	0	\\
4.27e-09	0	\\
4.37e-09	0	\\
4.47e-09	0	\\
4.57e-09	0	\\
4.68e-09	0	\\
4.78e-09	0	\\
4.89e-09	0	\\
4.99e-09	0	\\
5e-09	0	\\
};
\addplot [color=mycolor3,solid,forget plot]
  table[row sep=crcr]{
0	0	\\
1.1e-10	0	\\
2.2e-10	0	\\
3.3e-10	0	\\
4.4e-10	0	\\
5.4e-10	0	\\
6.5e-10	0	\\
7.5e-10	0	\\
8.6e-10	0	\\
9.6e-10	0	\\
1.07e-09	0	\\
1.18e-09	0	\\
1.28e-09	0	\\
1.38e-09	0	\\
1.49e-09	0	\\
1.59e-09	0	\\
1.69e-09	0	\\
1.8e-09	0	\\
1.9e-09	0	\\
2.01e-09	0	\\
2.11e-09	0	\\
2.21e-09	0	\\
2.32e-09	0	\\
2.42e-09	0	\\
2.52e-09	0	\\
2.63e-09	0	\\
2.73e-09	0	\\
2.83e-09	0	\\
2.93e-09	0	\\
3.04e-09	0	\\
3.14e-09	0	\\
3.24e-09	0	\\
3.34e-09	0	\\
3.45e-09	0	\\
3.55e-09	0	\\
3.65e-09	0	\\
3.75e-09	0	\\
3.86e-09	0	\\
3.96e-09	0	\\
4.06e-09	0	\\
4.16e-09	0	\\
4.27e-09	0	\\
4.37e-09	0	\\
4.47e-09	0	\\
4.57e-09	0	\\
4.68e-09	0	\\
4.78e-09	0	\\
4.89e-09	0	\\
4.99e-09	0	\\
5e-09	0	\\
};
\addplot [color=darkgray,solid,forget plot]
  table[row sep=crcr]{
0	0	\\
1.1e-10	0	\\
2.2e-10	0	\\
3.3e-10	0	\\
4.4e-10	0	\\
5.4e-10	0	\\
6.5e-10	0	\\
7.5e-10	0	\\
8.6e-10	0	\\
9.6e-10	0	\\
1.07e-09	0	\\
1.18e-09	0	\\
1.28e-09	0	\\
1.38e-09	0	\\
1.49e-09	0	\\
1.59e-09	0	\\
1.69e-09	0	\\
1.8e-09	0	\\
1.9e-09	0	\\
2.01e-09	0	\\
2.11e-09	0	\\
2.21e-09	0	\\
2.32e-09	0	\\
2.42e-09	0	\\
2.52e-09	0	\\
2.63e-09	0	\\
2.73e-09	0	\\
2.83e-09	0	\\
2.93e-09	0	\\
3.04e-09	0	\\
3.14e-09	0	\\
3.24e-09	0	\\
3.34e-09	0	\\
3.45e-09	0	\\
3.55e-09	0	\\
3.65e-09	0	\\
3.75e-09	0	\\
3.86e-09	0	\\
3.96e-09	0	\\
4.06e-09	0	\\
4.16e-09	0	\\
4.27e-09	0	\\
4.37e-09	0	\\
4.47e-09	0	\\
4.57e-09	0	\\
4.68e-09	0	\\
4.78e-09	0	\\
4.89e-09	0	\\
4.99e-09	0	\\
5e-09	0	\\
};
\addplot [color=blue,solid,forget plot]
  table[row sep=crcr]{
0	0	\\
1.1e-10	0	\\
2.2e-10	0	\\
3.3e-10	0	\\
4.4e-10	0	\\
5.4e-10	0	\\
6.5e-10	0	\\
7.5e-10	0	\\
8.6e-10	0	\\
9.6e-10	0	\\
1.07e-09	0	\\
1.18e-09	0	\\
1.28e-09	0	\\
1.38e-09	0	\\
1.49e-09	0	\\
1.59e-09	0	\\
1.69e-09	0	\\
1.8e-09	0	\\
1.9e-09	0	\\
2.01e-09	0	\\
2.11e-09	0	\\
2.21e-09	0	\\
2.32e-09	0	\\
2.42e-09	0	\\
2.52e-09	0	\\
2.63e-09	0	\\
2.73e-09	0	\\
2.83e-09	0	\\
2.93e-09	0	\\
3.04e-09	0	\\
3.14e-09	0	\\
3.24e-09	0	\\
3.34e-09	0	\\
3.45e-09	0	\\
3.55e-09	0	\\
3.65e-09	0	\\
3.75e-09	0	\\
3.86e-09	0	\\
3.96e-09	0	\\
4.06e-09	0	\\
4.16e-09	0	\\
4.27e-09	0	\\
4.37e-09	0	\\
4.47e-09	0	\\
4.57e-09	0	\\
4.68e-09	0	\\
4.78e-09	0	\\
4.89e-09	0	\\
4.99e-09	0	\\
5e-09	0	\\
};
\addplot [color=black!50!green,solid,forget plot]
  table[row sep=crcr]{
0	0	\\
1.1e-10	0	\\
2.2e-10	0	\\
3.3e-10	0	\\
4.4e-10	0	\\
5.4e-10	0	\\
6.5e-10	0	\\
7.5e-10	0	\\
8.6e-10	0	\\
9.6e-10	0	\\
1.07e-09	0	\\
1.18e-09	0	\\
1.28e-09	0	\\
1.38e-09	0	\\
1.49e-09	0	\\
1.59e-09	0	\\
1.69e-09	0	\\
1.8e-09	0	\\
1.9e-09	0	\\
2.01e-09	0	\\
2.11e-09	0	\\
2.21e-09	0	\\
2.32e-09	0	\\
2.42e-09	0	\\
2.52e-09	0	\\
2.63e-09	0	\\
2.73e-09	0	\\
2.83e-09	0	\\
2.93e-09	0	\\
3.04e-09	0	\\
3.14e-09	0	\\
3.24e-09	0	\\
3.34e-09	0	\\
3.45e-09	0	\\
3.55e-09	0	\\
3.65e-09	0	\\
3.75e-09	0	\\
3.86e-09	0	\\
3.96e-09	0	\\
4.06e-09	0	\\
4.16e-09	0	\\
4.27e-09	0	\\
4.37e-09	0	\\
4.47e-09	0	\\
4.57e-09	0	\\
4.68e-09	0	\\
4.78e-09	0	\\
4.89e-09	0	\\
4.99e-09	0	\\
5e-09	0	\\
};
\addplot [color=red,solid,forget plot]
  table[row sep=crcr]{
0	0	\\
1.1e-10	0	\\
2.2e-10	0	\\
3.3e-10	0	\\
4.4e-10	0	\\
5.4e-10	0	\\
6.5e-10	0	\\
7.5e-10	0	\\
8.6e-10	0	\\
9.6e-10	0	\\
1.07e-09	0	\\
1.18e-09	0	\\
1.28e-09	0	\\
1.38e-09	0	\\
1.49e-09	0	\\
1.59e-09	0	\\
1.69e-09	0	\\
1.8e-09	0	\\
1.9e-09	0	\\
2.01e-09	0	\\
2.11e-09	0	\\
2.21e-09	0	\\
2.32e-09	0	\\
2.42e-09	0	\\
2.52e-09	0	\\
2.63e-09	0	\\
2.73e-09	0	\\
2.83e-09	0	\\
2.93e-09	0	\\
3.04e-09	0	\\
3.14e-09	0	\\
3.24e-09	0	\\
3.34e-09	0	\\
3.45e-09	0	\\
3.55e-09	0	\\
3.65e-09	0	\\
3.75e-09	0	\\
3.86e-09	0	\\
3.96e-09	0	\\
4.06e-09	0	\\
4.16e-09	0	\\
4.27e-09	0	\\
4.37e-09	0	\\
4.47e-09	0	\\
4.57e-09	0	\\
4.68e-09	0	\\
4.78e-09	0	\\
4.89e-09	0	\\
4.99e-09	0	\\
5e-09	0	\\
};
\addplot [color=mycolor1,solid,forget plot]
  table[row sep=crcr]{
0	0	\\
1.1e-10	0	\\
2.2e-10	0	\\
3.3e-10	0	\\
4.4e-10	0	\\
5.4e-10	0	\\
6.5e-10	0	\\
7.5e-10	0	\\
8.6e-10	0	\\
9.6e-10	0	\\
1.07e-09	0	\\
1.18e-09	0	\\
1.28e-09	0	\\
1.38e-09	0	\\
1.49e-09	0	\\
1.59e-09	0	\\
1.69e-09	0	\\
1.8e-09	0	\\
1.9e-09	0	\\
2.01e-09	0	\\
2.11e-09	0	\\
2.21e-09	0	\\
2.32e-09	0	\\
2.42e-09	0	\\
2.52e-09	0	\\
2.63e-09	0	\\
2.73e-09	0	\\
2.83e-09	0	\\
2.93e-09	0	\\
3.04e-09	0	\\
3.14e-09	0	\\
3.24e-09	0	\\
3.34e-09	0	\\
3.45e-09	0	\\
3.55e-09	0	\\
3.65e-09	0	\\
3.75e-09	0	\\
3.86e-09	0	\\
3.96e-09	0	\\
4.06e-09	0	\\
4.16e-09	0	\\
4.27e-09	0	\\
4.37e-09	0	\\
4.47e-09	0	\\
4.57e-09	0	\\
4.68e-09	0	\\
4.78e-09	0	\\
4.89e-09	0	\\
4.99e-09	0	\\
5e-09	0	\\
};
\addplot [color=mycolor2,solid,forget plot]
  table[row sep=crcr]{
0	0	\\
1.1e-10	0	\\
2.2e-10	0	\\
3.3e-10	0	\\
4.4e-10	0	\\
5.4e-10	0	\\
6.5e-10	0	\\
7.5e-10	0	\\
8.6e-10	0	\\
9.6e-10	0	\\
1.07e-09	0	\\
1.18e-09	0	\\
1.28e-09	0	\\
1.38e-09	0	\\
1.49e-09	0	\\
1.59e-09	0	\\
1.69e-09	0	\\
1.8e-09	0	\\
1.9e-09	0	\\
2.01e-09	0	\\
2.11e-09	0	\\
2.21e-09	0	\\
2.32e-09	0	\\
2.42e-09	0	\\
2.52e-09	0	\\
2.63e-09	0	\\
2.73e-09	0	\\
2.83e-09	0	\\
2.93e-09	0	\\
3.04e-09	0	\\
3.14e-09	0	\\
3.24e-09	0	\\
3.34e-09	0	\\
3.45e-09	0	\\
3.55e-09	0	\\
3.65e-09	0	\\
3.75e-09	0	\\
3.86e-09	0	\\
3.96e-09	0	\\
4.06e-09	0	\\
4.16e-09	0	\\
4.27e-09	0	\\
4.37e-09	0	\\
4.47e-09	0	\\
4.57e-09	0	\\
4.68e-09	0	\\
4.78e-09	0	\\
4.89e-09	0	\\
4.99e-09	0	\\
5e-09	0	\\
};
\addplot [color=mycolor3,solid,forget plot]
  table[row sep=crcr]{
0	0	\\
1.1e-10	0	\\
2.2e-10	0	\\
3.3e-10	0	\\
4.4e-10	0	\\
5.4e-10	0	\\
6.5e-10	0	\\
7.5e-10	0	\\
8.6e-10	0	\\
9.6e-10	0	\\
1.07e-09	0	\\
1.18e-09	0	\\
1.28e-09	0	\\
1.38e-09	0	\\
1.49e-09	0	\\
1.59e-09	0	\\
1.69e-09	0	\\
1.8e-09	0	\\
1.9e-09	0	\\
2.01e-09	0	\\
2.11e-09	0	\\
2.21e-09	0	\\
2.32e-09	0	\\
2.42e-09	0	\\
2.52e-09	0	\\
2.63e-09	0	\\
2.73e-09	0	\\
2.83e-09	0	\\
2.93e-09	0	\\
3.04e-09	0	\\
3.14e-09	0	\\
3.24e-09	0	\\
3.34e-09	0	\\
3.45e-09	0	\\
3.55e-09	0	\\
3.65e-09	0	\\
3.75e-09	0	\\
3.86e-09	0	\\
3.96e-09	0	\\
4.06e-09	0	\\
4.16e-09	0	\\
4.27e-09	0	\\
4.37e-09	0	\\
4.47e-09	0	\\
4.57e-09	0	\\
4.68e-09	0	\\
4.78e-09	0	\\
4.89e-09	0	\\
4.99e-09	0	\\
5e-09	0	\\
};
\addplot [color=darkgray,solid,forget plot]
  table[row sep=crcr]{
0	0	\\
1.1e-10	0	\\
2.2e-10	0	\\
3.3e-10	0	\\
4.4e-10	0	\\
5.4e-10	0	\\
6.5e-10	0	\\
7.5e-10	0	\\
8.6e-10	0	\\
9.6e-10	0	\\
1.07e-09	0	\\
1.18e-09	0	\\
1.28e-09	0	\\
1.38e-09	0	\\
1.49e-09	0	\\
1.59e-09	0	\\
1.69e-09	0	\\
1.8e-09	0	\\
1.9e-09	0	\\
2.01e-09	0	\\
2.11e-09	0	\\
2.21e-09	0	\\
2.32e-09	0	\\
2.42e-09	0	\\
2.52e-09	0	\\
2.63e-09	0	\\
2.73e-09	0	\\
2.83e-09	0	\\
2.93e-09	0	\\
3.04e-09	0	\\
3.14e-09	0	\\
3.24e-09	0	\\
3.34e-09	0	\\
3.45e-09	0	\\
3.55e-09	0	\\
3.65e-09	0	\\
3.75e-09	0	\\
3.86e-09	0	\\
3.96e-09	0	\\
4.06e-09	0	\\
4.16e-09	0	\\
4.27e-09	0	\\
4.37e-09	0	\\
4.47e-09	0	\\
4.57e-09	0	\\
4.68e-09	0	\\
4.78e-09	0	\\
4.89e-09	0	\\
4.99e-09	0	\\
5e-09	0	\\
};
\addplot [color=blue,solid,forget plot]
  table[row sep=crcr]{
0	0	\\
1.1e-10	0	\\
2.2e-10	0	\\
3.3e-10	0	\\
4.4e-10	0	\\
5.4e-10	0	\\
6.5e-10	0	\\
7.5e-10	0	\\
8.6e-10	0	\\
9.6e-10	0	\\
1.07e-09	0	\\
1.18e-09	0	\\
1.28e-09	0	\\
1.38e-09	0	\\
1.49e-09	0	\\
1.59e-09	0	\\
1.69e-09	0	\\
1.8e-09	0	\\
1.9e-09	0	\\
2.01e-09	0	\\
2.11e-09	0	\\
2.21e-09	0	\\
2.32e-09	0	\\
2.42e-09	0	\\
2.52e-09	0	\\
2.63e-09	0	\\
2.73e-09	0	\\
2.83e-09	0	\\
2.93e-09	0	\\
3.04e-09	0	\\
3.14e-09	0	\\
3.24e-09	0	\\
3.34e-09	0	\\
3.45e-09	0	\\
3.55e-09	0	\\
3.65e-09	0	\\
3.75e-09	0	\\
3.86e-09	0	\\
3.96e-09	0	\\
4.06e-09	0	\\
4.16e-09	0	\\
4.27e-09	0	\\
4.37e-09	0	\\
4.47e-09	0	\\
4.57e-09	0	\\
4.68e-09	0	\\
4.78e-09	0	\\
4.89e-09	0	\\
4.99e-09	0	\\
5e-09	0	\\
};
\addplot [color=black!50!green,solid,forget plot]
  table[row sep=crcr]{
0	0	\\
1.1e-10	0	\\
2.2e-10	0	\\
3.3e-10	0	\\
4.4e-10	0	\\
5.4e-10	0	\\
6.5e-10	0	\\
7.5e-10	0	\\
8.6e-10	0	\\
9.6e-10	0	\\
1.07e-09	0	\\
1.18e-09	0	\\
1.2e-09	0.000927353155654489	\\
1.31e-09	0.000927353155654489	\\
1.41e-09	0.000927353155654489	\\
1.51e-09	0.000927353155654489	\\
1.62e-09	0.000927353155654489	\\
1.7e-09	0	\\
1.8e-09	0	\\
1.9e-09	0	\\
2.01e-09	0	\\
2.11e-09	0	\\
2.21e-09	0	\\
2.32e-09	0	\\
2.42e-09	0	\\
2.52e-09	0	\\
2.63e-09	0	\\
2.73e-09	0	\\
2.83e-09	0	\\
2.93e-09	0	\\
3.04e-09	0	\\
3.14e-09	0	\\
3.24e-09	0	\\
3.34e-09	0	\\
3.45e-09	0	\\
3.55e-09	0	\\
3.6e-09	0.000623302940685804	\\
3.7e-09	0.000623302940685804	\\
3.8e-09	0.000623302940685804	\\
3.9e-09	0.000623302940685804	\\
4e-09	0.000623302940685804	\\
4.1e-09	0	\\
4.2e-09	0	\\
4.3e-09	0	\\
4.41e-09	0	\\
4.51e-09	0	\\
4.61e-09	0	\\
4.71e-09	0	\\
4.82e-09	0	\\
4.92e-09	0	\\
5e-09	0	\\
};
\addplot [color=red,solid,forget plot]
  table[row sep=crcr]{
0	0	\\
1.1e-10	0	\\
2.2e-10	0	\\
3.3e-10	0	\\
4.4e-10	0	\\
5.4e-10	0	\\
6.5e-10	0	\\
7.5e-10	0	\\
8.6e-10	0	\\
9.6e-10	0	\\
1.07e-09	0	\\
1.18e-09	0	\\
1.28e-09	0	\\
1.38e-09	0	\\
1.49e-09	0	\\
1.59e-09	0	\\
1.69e-09	0	\\
1.8e-09	0	\\
1.9e-09	0	\\
2.01e-09	0	\\
2.11e-09	0	\\
2.21e-09	0	\\
2.32e-09	0	\\
2.42e-09	0	\\
2.52e-09	0	\\
2.63e-09	0	\\
2.73e-09	0	\\
2.83e-09	0	\\
2.93e-09	0	\\
3.04e-09	0	\\
3.14e-09	0	\\
3.24e-09	0	\\
3.34e-09	0	\\
3.45e-09	0	\\
3.55e-09	0	\\
3.65e-09	0	\\
3.75e-09	0	\\
3.86e-09	0	\\
3.96e-09	0	\\
4.06e-09	0	\\
4.16e-09	0	\\
4.27e-09	0	\\
4.37e-09	0	\\
4.47e-09	0	\\
4.57e-09	0	\\
4.68e-09	0	\\
4.78e-09	0	\\
4.89e-09	0	\\
4.99e-09	0	\\
5e-09	0	\\
};
\addplot [color=mycolor1,solid,forget plot]
  table[row sep=crcr]{
0	0	\\
1.1e-10	0	\\
2.2e-10	0	\\
3.3e-10	0	\\
4.4e-10	0	\\
5.4e-10	0	\\
6.5e-10	0	\\
7.5e-10	0	\\
8.6e-10	0	\\
9.6e-10	0	\\
1.07e-09	0	\\
1.18e-09	0	\\
1.28e-09	0	\\
1.38e-09	0	\\
1.49e-09	0	\\
1.59e-09	0	\\
1.69e-09	0	\\
1.8e-09	0	\\
1.9e-09	0	\\
2.01e-09	0	\\
2.11e-09	0	\\
2.21e-09	0	\\
2.32e-09	0	\\
2.42e-09	0	\\
2.52e-09	0	\\
2.63e-09	0	\\
2.73e-09	0	\\
2.83e-09	0	\\
2.93e-09	0	\\
3.04e-09	0	\\
3.14e-09	0	\\
3.24e-09	0	\\
3.34e-09	0	\\
3.45e-09	0	\\
3.55e-09	0	\\
3.65e-09	0	\\
3.75e-09	0	\\
3.86e-09	0	\\
3.96e-09	0	\\
4.06e-09	0	\\
4.16e-09	0	\\
4.27e-09	0	\\
4.37e-09	0	\\
4.47e-09	0	\\
4.57e-09	0	\\
4.68e-09	0	\\
4.78e-09	0	\\
4.89e-09	0	\\
4.99e-09	0	\\
5e-09	0	\\
};
\addplot [color=mycolor2,solid,forget plot]
  table[row sep=crcr]{
0	0	\\
1.1e-10	0	\\
2.2e-10	0	\\
3.3e-10	0	\\
4.4e-10	0	\\
5.4e-10	0	\\
6.5e-10	0	\\
7.5e-10	0	\\
8.6e-10	0	\\
9.6e-10	0	\\
1.07e-09	0	\\
1.18e-09	0	\\
1.28e-09	0	\\
1.38e-09	0	\\
1.49e-09	0	\\
1.59e-09	0	\\
1.69e-09	0	\\
1.8e-09	0	\\
1.9e-09	0	\\
2.01e-09	0	\\
2.11e-09	0	\\
2.21e-09	0	\\
2.32e-09	0	\\
2.42e-09	0	\\
2.52e-09	0	\\
2.63e-09	0	\\
2.73e-09	0	\\
2.83e-09	0	\\
2.93e-09	0	\\
3.04e-09	0	\\
3.14e-09	0	\\
3.24e-09	0	\\
3.34e-09	0	\\
3.45e-09	0	\\
3.55e-09	0	\\
3.65e-09	0	\\
3.75e-09	0	\\
3.86e-09	0	\\
3.96e-09	0	\\
4.06e-09	0	\\
4.16e-09	0	\\
4.27e-09	0	\\
4.37e-09	0	\\
4.47e-09	0	\\
4.57e-09	0	\\
4.68e-09	0	\\
4.78e-09	0	\\
4.89e-09	0	\\
4.99e-09	0	\\
5e-09	0	\\
};
\addplot [color=mycolor3,solid,forget plot]
  table[row sep=crcr]{
0	0	\\
1.1e-10	0	\\
2.2e-10	0	\\
3.3e-10	0	\\
4.4e-10	0	\\
5.4e-10	0	\\
6.5e-10	0	\\
7.5e-10	0	\\
8.6e-10	0	\\
9.6e-10	0	\\
1.07e-09	0	\\
1.18e-09	0	\\
1.28e-09	0	\\
1.38e-09	0	\\
1.49e-09	0	\\
1.59e-09	0	\\
1.69e-09	0	\\
1.8e-09	0	\\
1.9e-09	0	\\
2.01e-09	0	\\
2.11e-09	0	\\
2.21e-09	0	\\
2.32e-09	0	\\
2.42e-09	0	\\
2.52e-09	0	\\
2.63e-09	0	\\
2.73e-09	0	\\
2.83e-09	0	\\
2.93e-09	0	\\
3.04e-09	0	\\
3.14e-09	0	\\
3.24e-09	0	\\
3.34e-09	0	\\
3.45e-09	0	\\
3.55e-09	0	\\
3.65e-09	0	\\
3.75e-09	0	\\
3.86e-09	0	\\
3.96e-09	0	\\
4.06e-09	0	\\
4.16e-09	0	\\
4.27e-09	0	\\
4.37e-09	0	\\
4.47e-09	0	\\
4.57e-09	0	\\
4.68e-09	0	\\
4.78e-09	0	\\
4.89e-09	0	\\
4.99e-09	0	\\
5e-09	0	\\
};
\addplot [color=darkgray,solid,forget plot]
  table[row sep=crcr]{
0	0	\\
1.1e-10	0	\\
2.2e-10	0	\\
3.3e-10	0	\\
4.4e-10	0	\\
5.4e-10	0	\\
6.5e-10	0	\\
7.5e-10	0	\\
8.6e-10	0	\\
9.6e-10	0	\\
1.07e-09	0	\\
1.18e-09	0	\\
1.28e-09	0	\\
1.38e-09	0	\\
1.49e-09	0	\\
1.59e-09	0	\\
1.69e-09	0	\\
1.8e-09	0	\\
1.9e-09	0	\\
2.01e-09	0	\\
2.11e-09	0	\\
2.21e-09	0	\\
2.32e-09	0	\\
2.42e-09	0	\\
2.52e-09	0	\\
2.63e-09	0	\\
2.73e-09	0	\\
2.83e-09	0	\\
2.93e-09	0	\\
3.04e-09	0	\\
3.14e-09	0	\\
3.24e-09	0	\\
3.34e-09	0	\\
3.45e-09	0	\\
3.55e-09	0	\\
3.65e-09	0	\\
3.75e-09	0	\\
3.86e-09	0	\\
3.96e-09	0	\\
4.06e-09	0	\\
4.16e-09	0	\\
4.27e-09	0	\\
4.37e-09	0	\\
4.47e-09	0	\\
4.57e-09	0	\\
4.68e-09	0	\\
4.78e-09	0	\\
4.89e-09	0	\\
4.99e-09	0	\\
5e-09	0	\\
};
\addplot [color=blue,solid,forget plot]
  table[row sep=crcr]{
0	0	\\
1.1e-10	0	\\
2.2e-10	0	\\
3.3e-10	0	\\
4.4e-10	0	\\
5.4e-10	0	\\
6.5e-10	0	\\
7.5e-10	0	\\
8.6e-10	0	\\
9.6e-10	0	\\
1.07e-09	0	\\
1.18e-09	0	\\
1.28e-09	0	\\
1.38e-09	0	\\
1.49e-09	0	\\
1.59e-09	0	\\
1.69e-09	0	\\
1.8e-09	0	\\
1.9e-09	0	\\
2.01e-09	0	\\
2.11e-09	0	\\
2.21e-09	0	\\
2.32e-09	0	\\
2.42e-09	0	\\
2.52e-09	0	\\
2.63e-09	0	\\
2.73e-09	0	\\
2.83e-09	0	\\
2.93e-09	0	\\
3.04e-09	0	\\
3.14e-09	0	\\
3.24e-09	0	\\
3.34e-09	0	\\
3.45e-09	0	\\
3.55e-09	0	\\
3.65e-09	0	\\
3.75e-09	0	\\
3.86e-09	0	\\
3.96e-09	0	\\
4.06e-09	0	\\
4.16e-09	0	\\
4.27e-09	0	\\
4.37e-09	0	\\
4.47e-09	0	\\
4.57e-09	0	\\
4.68e-09	0	\\
4.78e-09	0	\\
4.89e-09	0	\\
4.99e-09	0	\\
5e-09	0	\\
};
\addplot [color=black!50!green,solid,forget plot]
  table[row sep=crcr]{
0	0	\\
1.1e-10	0	\\
2.2e-10	0	\\
3.3e-10	0	\\
4.4e-10	0	\\
5.4e-10	0	\\
6.5e-10	0	\\
7.5e-10	0	\\
8.6e-10	0	\\
9.6e-10	0	\\
1.07e-09	0	\\
1.18e-09	0	\\
1.28e-09	0	\\
1.38e-09	0	\\
1.49e-09	0	\\
1.59e-09	0	\\
1.69e-09	0	\\
1.8e-09	0	\\
1.9e-09	0	\\
2.01e-09	0	\\
2.11e-09	0	\\
2.21e-09	0	\\
2.32e-09	0	\\
2.42e-09	0	\\
2.52e-09	0	\\
2.63e-09	0	\\
2.73e-09	0	\\
2.83e-09	0	\\
2.93e-09	0	\\
3.04e-09	0	\\
3.14e-09	0	\\
3.24e-09	0	\\
3.34e-09	0	\\
3.45e-09	0	\\
3.55e-09	0	\\
3.65e-09	0	\\
3.75e-09	0	\\
3.86e-09	0	\\
3.96e-09	0	\\
4.06e-09	0	\\
4.16e-09	0	\\
4.27e-09	0	\\
4.37e-09	0	\\
4.47e-09	0	\\
4.57e-09	0	\\
4.68e-09	0	\\
4.78e-09	0	\\
4.89e-09	0	\\
4.99e-09	0	\\
5e-09	0	\\
};
\addplot [color=red,solid,forget plot]
  table[row sep=crcr]{
0	0	\\
1.1e-10	0	\\
2.2e-10	0	\\
3.3e-10	0	\\
4.4e-10	0	\\
5.4e-10	0	\\
6.5e-10	0	\\
7.5e-10	0	\\
8.6e-10	0	\\
9.6e-10	0	\\
1.07e-09	0	\\
1.18e-09	0	\\
1.28e-09	0	\\
1.38e-09	0	\\
1.49e-09	0	\\
1.59e-09	0	\\
1.69e-09	0	\\
1.8e-09	0	\\
1.9e-09	0	\\
2.01e-09	0	\\
2.11e-09	0	\\
2.21e-09	0	\\
2.32e-09	0	\\
2.42e-09	0	\\
2.52e-09	0	\\
2.63e-09	0	\\
2.73e-09	0	\\
2.83e-09	0	\\
2.93e-09	0	\\
3.04e-09	0	\\
3.14e-09	0	\\
3.24e-09	0	\\
3.34e-09	0	\\
3.45e-09	0	\\
3.55e-09	0	\\
3.65e-09	0	\\
3.75e-09	0	\\
3.86e-09	0	\\
3.96e-09	0	\\
4.06e-09	0	\\
4.16e-09	0	\\
4.27e-09	0	\\
4.37e-09	0	\\
4.47e-09	0	\\
4.57e-09	0	\\
4.68e-09	0	\\
4.78e-09	0	\\
4.89e-09	0	\\
4.99e-09	0	\\
5e-09	0	\\
};
\addplot [color=mycolor1,solid,forget plot]
  table[row sep=crcr]{
0	0	\\
1.1e-10	0	\\
2.2e-10	0	\\
3.3e-10	0	\\
4.4e-10	0	\\
5.4e-10	0	\\
6.5e-10	0	\\
7.5e-10	0	\\
8.6e-10	0	\\
9.6e-10	0	\\
1.07e-09	0	\\
1.18e-09	0	\\
1.28e-09	0	\\
1.38e-09	0	\\
1.49e-09	0	\\
1.59e-09	0	\\
1.69e-09	0	\\
1.8e-09	0	\\
1.9e-09	0	\\
2.01e-09	0	\\
2.11e-09	0	\\
2.21e-09	0	\\
2.32e-09	0	\\
2.42e-09	0	\\
2.52e-09	0	\\
2.63e-09	0	\\
2.73e-09	0	\\
2.83e-09	0	\\
2.93e-09	0	\\
3.04e-09	0	\\
3.14e-09	0	\\
3.24e-09	0	\\
3.34e-09	0	\\
3.45e-09	0	\\
3.55e-09	0	\\
3.65e-09	0	\\
3.75e-09	0	\\
3.86e-09	0	\\
3.96e-09	0	\\
4.06e-09	0	\\
4.16e-09	0	\\
4.27e-09	0	\\
4.37e-09	0	\\
4.47e-09	0	\\
4.57e-09	0	\\
4.68e-09	0	\\
4.78e-09	0	\\
4.89e-09	0	\\
4.99e-09	0	\\
5e-09	0	\\
};
\addplot [color=mycolor2,solid,forget plot]
  table[row sep=crcr]{
0	0	\\
1.1e-10	0	\\
2.2e-10	0	\\
3.3e-10	0	\\
4.4e-10	0	\\
5.4e-10	0	\\
6.5e-10	0	\\
7.5e-10	0	\\
8.6e-10	0	\\
9.6e-10	0	\\
1.07e-09	0	\\
1.18e-09	0	\\
1.28e-09	0	\\
1.38e-09	0	\\
1.49e-09	0	\\
1.59e-09	0	\\
1.69e-09	0	\\
1.8e-09	0	\\
1.9e-09	0	\\
2.01e-09	0	\\
2.11e-09	0	\\
2.21e-09	0	\\
2.32e-09	0	\\
2.42e-09	0	\\
2.52e-09	0	\\
2.63e-09	0	\\
2.73e-09	0	\\
2.83e-09	0	\\
2.93e-09	0	\\
3.04e-09	0	\\
3.14e-09	0	\\
3.24e-09	0	\\
3.34e-09	0	\\
3.45e-09	0	\\
3.55e-09	0	\\
3.65e-09	0	\\
3.75e-09	0	\\
3.86e-09	0	\\
3.96e-09	0	\\
4.06e-09	0	\\
4.16e-09	0	\\
4.27e-09	0	\\
4.37e-09	0	\\
4.47e-09	0	\\
4.57e-09	0	\\
4.68e-09	0	\\
4.78e-09	0	\\
4.89e-09	0	\\
4.99e-09	0	\\
5e-09	0	\\
};
\addplot [color=mycolor3,solid,forget plot]
  table[row sep=crcr]{
0	0	\\
1.1e-10	0	\\
2.2e-10	0	\\
3.3e-10	0	\\
4.4e-10	0	\\
5.4e-10	0	\\
6.5e-10	0	\\
7.5e-10	0	\\
8.6e-10	0	\\
9.6e-10	0	\\
1.07e-09	0	\\
1.18e-09	0	\\
1.28e-09	0	\\
1.38e-09	0	\\
1.49e-09	0	\\
1.59e-09	0	\\
1.69e-09	0	\\
1.8e-09	0	\\
1.9e-09	0	\\
2.01e-09	0	\\
2.11e-09	0	\\
2.21e-09	0	\\
2.32e-09	0	\\
2.42e-09	0	\\
2.52e-09	0	\\
2.63e-09	0	\\
2.73e-09	0	\\
2.83e-09	0	\\
2.93e-09	0	\\
3.04e-09	0	\\
3.14e-09	0	\\
3.24e-09	0	\\
3.34e-09	0	\\
3.45e-09	0	\\
3.55e-09	0	\\
3.65e-09	0	\\
3.75e-09	0	\\
3.86e-09	0	\\
3.96e-09	0	\\
4.06e-09	0	\\
4.16e-09	0	\\
4.27e-09	0	\\
4.37e-09	0	\\
4.47e-09	0	\\
4.57e-09	0	\\
4.68e-09	0	\\
4.78e-09	0	\\
4.89e-09	0	\\
4.99e-09	0	\\
5e-09	0	\\
};
\addplot [color=darkgray,solid,forget plot]
  table[row sep=crcr]{
0	0	\\
1.1e-10	0	\\
2.2e-10	0	\\
3.3e-10	0	\\
4.4e-10	0	\\
5.4e-10	0	\\
6.5e-10	0	\\
7.5e-10	0	\\
8.6e-10	0	\\
9.6e-10	0	\\
1.07e-09	0	\\
1.18e-09	0	\\
1.28e-09	0	\\
1.38e-09	0	\\
1.49e-09	0	\\
1.59e-09	0	\\
1.69e-09	0	\\
1.8e-09	0	\\
1.9e-09	0	\\
2.01e-09	0	\\
2.11e-09	0	\\
2.21e-09	0	\\
2.32e-09	0	\\
2.42e-09	0	\\
2.52e-09	0	\\
2.63e-09	0	\\
2.73e-09	0	\\
2.83e-09	0	\\
2.93e-09	0	\\
3.04e-09	0	\\
3.14e-09	0	\\
3.24e-09	0	\\
3.34e-09	0	\\
3.45e-09	0	\\
3.55e-09	0	\\
3.65e-09	0	\\
3.75e-09	0	\\
3.86e-09	0	\\
3.96e-09	0	\\
4.06e-09	0	\\
4.16e-09	0	\\
4.27e-09	0	\\
4.37e-09	0	\\
4.47e-09	0	\\
4.57e-09	0	\\
4.68e-09	0	\\
4.78e-09	0	\\
4.89e-09	0	\\
4.99e-09	0	\\
5e-09	0	\\
};
\addplot [color=blue,solid,forget plot]
  table[row sep=crcr]{
0	0	\\
1.1e-10	0	\\
2.2e-10	0	\\
3.3e-10	0	\\
4.4e-10	0	\\
5.4e-10	0	\\
6.5e-10	0	\\
7.5e-10	0	\\
8.6e-10	0	\\
9.6e-10	0	\\
1.07e-09	0	\\
1.18e-09	0	\\
1.28e-09	0	\\
1.38e-09	0	\\
1.49e-09	0	\\
1.59e-09	0	\\
1.69e-09	0	\\
1.8e-09	0	\\
1.9e-09	0	\\
2.01e-09	0	\\
2.11e-09	0	\\
2.21e-09	0	\\
2.32e-09	0	\\
2.42e-09	0	\\
2.52e-09	0	\\
2.63e-09	0	\\
2.73e-09	0	\\
2.83e-09	0	\\
2.93e-09	0	\\
3.04e-09	0	\\
3.14e-09	0	\\
3.24e-09	0	\\
3.34e-09	0	\\
3.45e-09	0	\\
3.55e-09	0	\\
3.65e-09	0	\\
3.75e-09	0	\\
3.86e-09	0	\\
3.96e-09	0	\\
4.06e-09	0	\\
4.16e-09	0	\\
4.27e-09	0	\\
4.37e-09	0	\\
4.47e-09	0	\\
4.57e-09	0	\\
4.68e-09	0	\\
4.78e-09	0	\\
4.89e-09	0	\\
4.99e-09	0	\\
5e-09	0	\\
};
\addplot [color=black!50!green,solid,forget plot]
  table[row sep=crcr]{
0	0	\\
1.1e-10	0	\\
2.2e-10	0	\\
3.3e-10	0	\\
4.4e-10	0	\\
5.4e-10	0	\\
6.5e-10	0	\\
7.5e-10	0	\\
8.6e-10	0	\\
9.6e-10	0	\\
1.07e-09	0	\\
1.18e-09	0	\\
1.28e-09	0	\\
1.38e-09	0	\\
1.49e-09	0	\\
1.59e-09	0	\\
1.69e-09	0	\\
1.8e-09	0	\\
1.9e-09	0	\\
2.01e-09	0	\\
2.11e-09	0	\\
2.21e-09	0	\\
2.32e-09	0	\\
2.42e-09	0	\\
2.52e-09	0	\\
2.63e-09	0	\\
2.73e-09	0	\\
2.83e-09	0	\\
2.93e-09	0	\\
3.04e-09	0	\\
3.14e-09	0	\\
3.24e-09	0	\\
3.34e-09	0	\\
3.45e-09	0	\\
3.55e-09	0	\\
3.65e-09	0	\\
3.75e-09	0	\\
3.86e-09	0	\\
3.96e-09	0	\\
4.06e-09	0	\\
4.16e-09	0	\\
4.27e-09	0	\\
4.37e-09	0	\\
4.47e-09	0	\\
4.57e-09	0	\\
4.68e-09	0	\\
4.78e-09	0	\\
4.89e-09	0	\\
4.99e-09	0	\\
5e-09	0	\\
};
\addplot [color=red,solid,forget plot]
  table[row sep=crcr]{
0	0	\\
1.1e-10	0	\\
2.2e-10	0	\\
3.3e-10	0	\\
4.4e-10	0	\\
5.4e-10	0	\\
6.5e-10	0	\\
7.5e-10	0	\\
8.6e-10	0	\\
9.6e-10	0	\\
1.07e-09	0	\\
1.18e-09	0	\\
1.28e-09	0	\\
1.38e-09	0	\\
1.49e-09	0	\\
1.59e-09	0	\\
1.69e-09	0	\\
1.8e-09	0	\\
1.9e-09	0	\\
2.01e-09	0	\\
2.11e-09	0	\\
2.21e-09	0	\\
2.32e-09	0	\\
2.42e-09	0	\\
2.52e-09	0	\\
2.63e-09	0	\\
2.73e-09	0	\\
2.83e-09	0	\\
2.93e-09	0	\\
3.04e-09	0	\\
3.14e-09	0	\\
3.24e-09	0	\\
3.34e-09	0	\\
3.45e-09	0	\\
3.55e-09	0	\\
3.65e-09	0	\\
3.75e-09	0	\\
3.86e-09	0	\\
3.96e-09	0	\\
4.06e-09	0	\\
4.16e-09	0	\\
4.27e-09	0	\\
4.37e-09	0	\\
4.47e-09	0	\\
4.57e-09	0	\\
4.68e-09	0	\\
4.78e-09	0	\\
4.89e-09	0	\\
4.99e-09	0	\\
5e-09	0	\\
};
\addplot [color=mycolor1,solid,forget plot]
  table[row sep=crcr]{
0	0	\\
1.1e-10	0	\\
2.2e-10	0	\\
3.3e-10	0	\\
4.4e-10	0	\\
5.4e-10	0	\\
6.5e-10	0	\\
7.5e-10	0	\\
8.6e-10	0	\\
9.6e-10	0	\\
1.07e-09	0	\\
1.18e-09	0	\\
1.28e-09	0	\\
1.38e-09	0	\\
1.49e-09	0	\\
1.59e-09	0	\\
1.69e-09	0	\\
1.8e-09	0	\\
1.9e-09	0	\\
2.01e-09	0	\\
2.11e-09	0	\\
2.21e-09	0	\\
2.32e-09	0	\\
2.42e-09	0	\\
2.52e-09	0	\\
2.63e-09	0	\\
2.73e-09	0	\\
2.83e-09	0	\\
2.93e-09	0	\\
3.04e-09	0	\\
3.14e-09	0	\\
3.24e-09	0	\\
3.34e-09	0	\\
3.45e-09	0	\\
3.55e-09	0	\\
3.65e-09	0	\\
3.75e-09	0	\\
3.86e-09	0	\\
3.96e-09	0	\\
4.06e-09	0	\\
4.16e-09	0	\\
4.27e-09	0	\\
4.37e-09	0	\\
4.47e-09	0	\\
4.57e-09	0	\\
4.68e-09	0	\\
4.78e-09	0	\\
4.89e-09	0	\\
4.99e-09	0	\\
5e-09	0	\\
};
\addplot [color=mycolor2,solid,forget plot]
  table[row sep=crcr]{
0	0	\\
1.1e-10	0	\\
2.2e-10	0	\\
3.3e-10	0	\\
4.4e-10	0	\\
5.4e-10	0	\\
6.5e-10	0	\\
7.5e-10	0	\\
8.6e-10	0	\\
9.6e-10	0	\\
1.07e-09	0	\\
1.18e-09	0	\\
1.28e-09	0	\\
1.38e-09	0	\\
1.49e-09	0	\\
1.59e-09	0	\\
1.69e-09	0	\\
1.8e-09	0	\\
1.9e-09	0	\\
2.01e-09	0	\\
2.11e-09	0	\\
2.21e-09	0	\\
2.32e-09	0	\\
2.42e-09	0	\\
2.52e-09	0	\\
2.63e-09	0	\\
2.73e-09	0	\\
2.83e-09	0	\\
2.93e-09	0	\\
3.04e-09	0	\\
3.14e-09	0	\\
3.24e-09	0	\\
3.34e-09	0	\\
3.45e-09	0	\\
3.55e-09	0	\\
3.65e-09	0	\\
3.75e-09	0	\\
3.86e-09	0	\\
3.96e-09	0	\\
4.06e-09	0	\\
4.16e-09	0	\\
4.27e-09	0	\\
4.37e-09	0	\\
4.47e-09	0	\\
4.57e-09	0	\\
4.68e-09	0	\\
4.78e-09	0	\\
4.89e-09	0	\\
4.99e-09	0	\\
5e-09	0	\\
};
\addplot [color=mycolor3,solid,forget plot]
  table[row sep=crcr]{
0	0	\\
1.1e-10	0	\\
2.2e-10	0	\\
3.3e-10	0	\\
4.4e-10	0	\\
5.4e-10	0	\\
6.5e-10	0	\\
7.5e-10	0	\\
8.6e-10	0	\\
9.6e-10	0	\\
1.07e-09	0	\\
1.18e-09	0	\\
1.28e-09	0	\\
1.38e-09	0	\\
1.49e-09	0	\\
1.59e-09	0	\\
1.69e-09	0	\\
1.8e-09	0	\\
1.9e-09	0	\\
2.01e-09	0	\\
2.11e-09	0	\\
2.21e-09	0	\\
2.32e-09	0	\\
2.42e-09	0	\\
2.52e-09	0	\\
2.63e-09	0	\\
2.73e-09	0	\\
2.83e-09	0	\\
2.93e-09	0	\\
3.04e-09	0	\\
3.14e-09	0	\\
3.24e-09	0	\\
3.34e-09	0	\\
3.45e-09	0	\\
3.55e-09	0	\\
3.65e-09	0	\\
3.75e-09	0	\\
3.86e-09	0	\\
3.96e-09	0	\\
4.06e-09	0	\\
4.16e-09	0	\\
4.27e-09	0	\\
4.37e-09	0	\\
4.47e-09	0	\\
4.57e-09	0	\\
4.68e-09	0	\\
4.78e-09	0	\\
4.89e-09	0	\\
4.99e-09	0	\\
5e-09	0	\\
};
\addplot [color=darkgray,solid,forget plot]
  table[row sep=crcr]{
0	0	\\
1.1e-10	0	\\
2.2e-10	0	\\
3.3e-10	0	\\
4.4e-10	0	\\
5.4e-10	0	\\
6.5e-10	0	\\
7.5e-10	0	\\
8.6e-10	0	\\
9.6e-10	0	\\
1.07e-09	0	\\
1.18e-09	0	\\
1.28e-09	0	\\
1.38e-09	0	\\
1.49e-09	0	\\
1.59e-09	0	\\
1.69e-09	0	\\
1.8e-09	0	\\
1.9e-09	0	\\
2.01e-09	0	\\
2.11e-09	0	\\
2.21e-09	0	\\
2.32e-09	0	\\
2.42e-09	0	\\
2.52e-09	0	\\
2.63e-09	0	\\
2.73e-09	0	\\
2.83e-09	0	\\
2.93e-09	0	\\
3.04e-09	0	\\
3.14e-09	0	\\
3.24e-09	0	\\
3.34e-09	0	\\
3.45e-09	0	\\
3.55e-09	0	\\
3.65e-09	0	\\
3.75e-09	0	\\
3.86e-09	0	\\
3.96e-09	0	\\
4.06e-09	0	\\
4.16e-09	0	\\
4.27e-09	0	\\
4.37e-09	0	\\
4.47e-09	0	\\
4.57e-09	0	\\
4.68e-09	0	\\
4.78e-09	0	\\
4.89e-09	0	\\
4.99e-09	0	\\
5e-09	0	\\
};
\addplot [color=blue,solid,forget plot]
  table[row sep=crcr]{
0	0	\\
1.1e-10	0	\\
2.2e-10	0	\\
3.3e-10	0	\\
4.4e-10	0	\\
5.4e-10	0	\\
6.5e-10	0	\\
7.5e-10	0	\\
8.6e-10	0	\\
9.6e-10	0	\\
1.07e-09	0	\\
1.18e-09	0	\\
1.28e-09	0	\\
1.38e-09	0	\\
1.49e-09	0	\\
1.59e-09	0	\\
1.69e-09	0	\\
1.8e-09	0	\\
1.9e-09	0	\\
2.01e-09	0	\\
2.11e-09	0	\\
2.21e-09	0	\\
2.32e-09	0	\\
2.42e-09	0	\\
2.52e-09	0	\\
2.63e-09	0	\\
2.73e-09	0	\\
2.83e-09	0	\\
2.93e-09	0	\\
3.04e-09	0	\\
3.14e-09	0	\\
3.24e-09	0	\\
3.34e-09	0	\\
3.45e-09	0	\\
3.55e-09	0	\\
3.65e-09	0	\\
3.75e-09	0	\\
3.86e-09	0	\\
3.96e-09	0	\\
4.06e-09	0	\\
4.16e-09	0	\\
4.27e-09	0	\\
4.37e-09	0	\\
4.47e-09	0	\\
4.57e-09	0	\\
4.68e-09	0	\\
4.78e-09	0	\\
4.89e-09	0	\\
4.99e-09	0	\\
5e-09	0	\\
};
\addplot [color=black!50!green,solid,forget plot]
  table[row sep=crcr]{
0	0	\\
1.1e-10	0	\\
2.2e-10	0	\\
3.3e-10	0	\\
4.4e-10	0	\\
5.4e-10	0	\\
6.5e-10	0	\\
7.5e-10	0	\\
8.6e-10	0	\\
9.6e-10	0	\\
1.07e-09	0	\\
1.18e-09	0	\\
1.28e-09	0	\\
1.38e-09	0	\\
1.49e-09	0	\\
1.59e-09	0	\\
1.69e-09	0	\\
1.8e-09	0	\\
1.9e-09	0	\\
2.01e-09	0	\\
2.11e-09	0	\\
2.21e-09	0	\\
2.32e-09	0	\\
2.42e-09	0	\\
2.52e-09	0	\\
2.63e-09	0	\\
2.73e-09	0	\\
2.83e-09	0	\\
2.93e-09	0	\\
3.04e-09	0	\\
3.14e-09	0	\\
3.24e-09	0	\\
3.34e-09	0	\\
3.45e-09	0	\\
3.55e-09	0	\\
3.65e-09	0	\\
3.75e-09	0	\\
3.86e-09	0	\\
3.96e-09	0	\\
4.06e-09	0	\\
4.16e-09	0	\\
4.27e-09	0	\\
4.37e-09	0	\\
4.47e-09	0	\\
4.57e-09	0	\\
4.68e-09	0	\\
4.78e-09	0	\\
4.89e-09	0	\\
4.99e-09	0	\\
5e-09	0	\\
};
\addplot [color=red,solid,forget plot]
  table[row sep=crcr]{
0	0	\\
1.1e-10	0	\\
2.2e-10	0	\\
3.3e-10	0	\\
4.4e-10	0	\\
5.4e-10	0	\\
6.5e-10	0	\\
7.5e-10	0	\\
8.6e-10	0	\\
9.6e-10	0	\\
1.07e-09	0	\\
1.18e-09	0	\\
1.28e-09	0	\\
1.38e-09	0	\\
1.49e-09	0	\\
1.59e-09	0	\\
1.69e-09	0	\\
1.8e-09	0	\\
1.9e-09	0	\\
2.01e-09	0	\\
2.11e-09	0	\\
2.21e-09	0	\\
2.32e-09	0	\\
2.42e-09	0	\\
2.52e-09	0	\\
2.63e-09	0	\\
2.73e-09	0	\\
2.83e-09	0	\\
2.93e-09	0	\\
3.04e-09	0	\\
3.14e-09	0	\\
3.24e-09	0	\\
3.34e-09	0	\\
3.45e-09	0	\\
3.55e-09	0	\\
3.65e-09	0	\\
3.75e-09	0	\\
3.86e-09	0	\\
3.96e-09	0	\\
4.06e-09	0	\\
4.16e-09	0	\\
4.27e-09	0	\\
4.37e-09	0	\\
4.47e-09	0	\\
4.57e-09	0	\\
4.68e-09	0	\\
4.78e-09	0	\\
4.89e-09	0	\\
4.99e-09	0	\\
5e-09	0	\\
};
\addplot [color=mycolor1,solid,forget plot]
  table[row sep=crcr]{
0	0	\\
1.1e-10	0	\\
2.2e-10	0	\\
3.3e-10	0	\\
4.4e-10	0	\\
5.4e-10	0	\\
6.5e-10	0	\\
7.5e-10	0	\\
8.6e-10	0	\\
9.6e-10	0	\\
1.07e-09	0	\\
1.18e-09	0	\\
1.28e-09	0	\\
1.38e-09	0	\\
1.49e-09	0	\\
1.59e-09	0	\\
1.69e-09	0	\\
1.8e-09	0	\\
1.9e-09	0	\\
2.01e-09	0	\\
2.11e-09	0	\\
2.21e-09	0	\\
2.32e-09	0	\\
2.42e-09	0	\\
2.52e-09	0	\\
2.63e-09	0	\\
2.73e-09	0	\\
2.83e-09	0	\\
2.93e-09	0	\\
3.04e-09	0	\\
3.14e-09	0	\\
3.24e-09	0	\\
3.34e-09	0	\\
3.45e-09	0	\\
3.55e-09	0	\\
3.65e-09	0	\\
3.75e-09	0	\\
3.86e-09	0	\\
3.96e-09	0	\\
4.06e-09	0	\\
4.16e-09	0	\\
4.27e-09	0	\\
4.37e-09	0	\\
4.47e-09	0	\\
4.57e-09	0	\\
4.68e-09	0	\\
4.78e-09	0	\\
4.89e-09	0	\\
4.99e-09	0	\\
5e-09	0	\\
};
\addplot [color=mycolor2,solid,forget plot]
  table[row sep=crcr]{
0	0	\\
1.1e-10	0	\\
2.2e-10	0	\\
3.3e-10	0	\\
4.4e-10	0	\\
5.4e-10	0	\\
6.5e-10	0	\\
7.5e-10	0	\\
8.6e-10	0	\\
9.6e-10	0	\\
1.07e-09	0	\\
1.18e-09	0	\\
1.28e-09	0	\\
1.38e-09	0	\\
1.49e-09	0	\\
1.59e-09	0	\\
1.69e-09	0	\\
1.8e-09	0	\\
1.9e-09	0	\\
2.01e-09	0	\\
2.11e-09	0	\\
2.21e-09	0	\\
2.32e-09	0	\\
2.42e-09	0	\\
2.52e-09	0	\\
2.63e-09	0	\\
2.73e-09	0	\\
2.83e-09	0	\\
2.93e-09	0	\\
3.04e-09	0	\\
3.14e-09	0	\\
3.24e-09	0	\\
3.34e-09	0	\\
3.45e-09	0	\\
3.55e-09	0	\\
3.65e-09	0	\\
3.75e-09	0	\\
3.86e-09	0	\\
3.96e-09	0	\\
4.06e-09	0	\\
4.16e-09	0	\\
4.27e-09	0	\\
4.37e-09	0	\\
4.47e-09	0	\\
4.57e-09	0	\\
4.68e-09	0	\\
4.78e-09	0	\\
4.89e-09	0	\\
4.99e-09	0	\\
5e-09	0	\\
};
\addplot [color=mycolor3,solid,forget plot]
  table[row sep=crcr]{
0	0	\\
1.1e-10	0	\\
2.2e-10	0	\\
3.3e-10	0	\\
4.4e-10	0	\\
5.4e-10	0	\\
6.5e-10	0	\\
7.5e-10	0	\\
8.6e-10	0	\\
9.6e-10	0	\\
1.07e-09	0	\\
1.18e-09	0	\\
1.28e-09	0	\\
1.38e-09	0	\\
1.49e-09	0	\\
1.59e-09	0	\\
1.69e-09	0	\\
1.8e-09	0	\\
1.9e-09	0	\\
2.01e-09	0	\\
2.11e-09	0	\\
2.21e-09	0	\\
2.32e-09	0	\\
2.42e-09	0	\\
2.52e-09	0	\\
2.63e-09	0	\\
2.73e-09	0	\\
2.83e-09	0	\\
2.93e-09	0	\\
3.04e-09	0	\\
3.14e-09	0	\\
3.24e-09	0	\\
3.34e-09	0	\\
3.45e-09	0	\\
3.55e-09	0	\\
3.65e-09	0	\\
3.75e-09	0	\\
3.86e-09	0	\\
3.96e-09	0	\\
4.06e-09	0	\\
4.16e-09	0	\\
4.27e-09	0	\\
4.37e-09	0	\\
4.47e-09	0	\\
4.57e-09	0	\\
4.68e-09	0	\\
4.78e-09	0	\\
4.89e-09	0	\\
4.99e-09	0	\\
5e-09	0	\\
};
\addplot [color=darkgray,solid,forget plot]
  table[row sep=crcr]{
0	0	\\
1.1e-10	0	\\
2.2e-10	0	\\
3.3e-10	0	\\
4.4e-10	0	\\
5.4e-10	0	\\
6.5e-10	0	\\
7.5e-10	0	\\
8.6e-10	0	\\
9.6e-10	0	\\
1.07e-09	0	\\
1.18e-09	0	\\
1.28e-09	0	\\
1.38e-09	0	\\
1.49e-09	0	\\
1.59e-09	0	\\
1.69e-09	0	\\
1.8e-09	0	\\
1.9e-09	0	\\
2.01e-09	0	\\
2.11e-09	0	\\
2.21e-09	0	\\
2.32e-09	0	\\
2.42e-09	0	\\
2.52e-09	0	\\
2.63e-09	0	\\
2.73e-09	0	\\
2.83e-09	0	\\
2.93e-09	0	\\
3.04e-09	0	\\
3.14e-09	0	\\
3.24e-09	0	\\
3.34e-09	0	\\
3.45e-09	0	\\
3.55e-09	0	\\
3.65e-09	0	\\
3.75e-09	0	\\
3.86e-09	0	\\
3.96e-09	0	\\
4.06e-09	0	\\
4.16e-09	0	\\
4.27e-09	0	\\
4.37e-09	0	\\
4.47e-09	0	\\
4.57e-09	0	\\
4.68e-09	0	\\
4.78e-09	0	\\
4.89e-09	0	\\
4.99e-09	0	\\
5e-09	0	\\
};
\addplot [color=blue,solid,forget plot]
  table[row sep=crcr]{
0	0	\\
1.1e-10	0	\\
2.2e-10	0	\\
3.3e-10	0	\\
4.4e-10	0	\\
5.4e-10	0	\\
6.5e-10	0	\\
7.5e-10	0	\\
8.6e-10	0	\\
9.6e-10	0	\\
1.07e-09	0	\\
1.18e-09	0	\\
1.28e-09	0	\\
1.38e-09	0	\\
1.49e-09	0	\\
1.59e-09	0	\\
1.69e-09	0	\\
1.8e-09	0	\\
1.9e-09	0	\\
2.01e-09	0	\\
2.11e-09	0	\\
2.21e-09	0	\\
2.32e-09	0	\\
2.42e-09	0	\\
2.52e-09	0	\\
2.63e-09	0	\\
2.73e-09	0	\\
2.83e-09	0	\\
2.93e-09	0	\\
3.04e-09	0	\\
3.14e-09	0	\\
3.24e-09	0	\\
3.34e-09	0	\\
3.45e-09	0	\\
3.55e-09	0	\\
3.65e-09	0	\\
3.75e-09	0	\\
3.86e-09	0	\\
3.96e-09	0	\\
4.06e-09	0	\\
4.16e-09	0	\\
4.27e-09	0	\\
4.37e-09	0	\\
4.47e-09	0	\\
4.57e-09	0	\\
4.68e-09	0	\\
4.78e-09	0	\\
4.89e-09	0	\\
4.99e-09	0	\\
5e-09	0	\\
};
\addplot [color=black!50!green,solid,forget plot]
  table[row sep=crcr]{
0	0	\\
1.1e-10	0	\\
2.2e-10	0	\\
3.3e-10	0	\\
4.4e-10	0	\\
5.4e-10	0	\\
6.5e-10	0	\\
7.5e-10	0	\\
8.6e-10	0	\\
9.6e-10	0	\\
1.07e-09	0	\\
1.18e-09	0	\\
1.28e-09	0	\\
1.38e-09	0	\\
1.49e-09	0	\\
1.59e-09	0	\\
1.69e-09	0	\\
1.8e-09	0	\\
1.9e-09	0	\\
2.01e-09	0	\\
2.11e-09	0	\\
2.21e-09	0	\\
2.32e-09	0	\\
2.42e-09	0	\\
2.52e-09	0	\\
2.63e-09	0	\\
2.73e-09	0	\\
2.83e-09	0	\\
2.93e-09	0	\\
3.04e-09	0	\\
3.14e-09	0	\\
3.24e-09	0	\\
3.34e-09	0	\\
3.45e-09	0	\\
3.55e-09	0	\\
3.65e-09	0	\\
3.75e-09	0	\\
3.86e-09	0	\\
3.96e-09	0	\\
4.06e-09	0	\\
4.16e-09	0	\\
4.27e-09	0	\\
4.37e-09	0	\\
4.47e-09	0	\\
4.57e-09	0	\\
4.68e-09	0	\\
4.78e-09	0	\\
4.89e-09	0	\\
4.99e-09	0	\\
5e-09	0	\\
};
\addplot [color=red,solid,forget plot]
  table[row sep=crcr]{
0	0	\\
1.1e-10	0	\\
2.2e-10	0	\\
3.3e-10	0	\\
4.4e-10	0	\\
5.4e-10	0	\\
6.5e-10	0	\\
7.5e-10	0	\\
8.6e-10	0	\\
9.6e-10	0	\\
1.07e-09	0	\\
1.18e-09	0	\\
1.28e-09	0	\\
1.38e-09	0	\\
1.49e-09	0	\\
1.59e-09	0	\\
1.69e-09	0	\\
1.8e-09	0	\\
1.9e-09	0	\\
2.01e-09	0	\\
2.11e-09	0	\\
2.21e-09	0	\\
2.32e-09	0	\\
2.42e-09	0	\\
2.52e-09	0	\\
2.63e-09	0	\\
2.73e-09	0	\\
2.83e-09	0	\\
2.93e-09	0	\\
3.04e-09	0	\\
3.14e-09	0	\\
3.24e-09	0	\\
3.34e-09	0	\\
3.45e-09	0	\\
3.55e-09	0	\\
3.65e-09	0	\\
3.75e-09	0	\\
3.86e-09	0	\\
3.96e-09	0	\\
4.06e-09	0	\\
4.16e-09	0	\\
4.27e-09	0	\\
4.37e-09	0	\\
4.47e-09	0	\\
4.57e-09	0	\\
4.68e-09	0	\\
4.78e-09	0	\\
4.89e-09	0	\\
4.99e-09	0	\\
5e-09	0	\\
};
\addplot [color=mycolor1,solid,forget plot]
  table[row sep=crcr]{
0	0	\\
1.1e-10	0	\\
2.2e-10	0	\\
3.3e-10	0	\\
4.4e-10	0	\\
5.4e-10	0	\\
6.5e-10	0	\\
7.5e-10	0	\\
8.6e-10	0	\\
9.6e-10	0	\\
1.07e-09	0	\\
1.18e-09	0	\\
1.28e-09	0	\\
1.38e-09	0	\\
1.49e-09	0	\\
1.59e-09	0	\\
1.69e-09	0	\\
1.8e-09	0	\\
1.9e-09	0	\\
2.01e-09	0	\\
2.11e-09	0	\\
2.21e-09	0	\\
2.32e-09	0	\\
2.42e-09	0	\\
2.52e-09	0	\\
2.63e-09	0	\\
2.73e-09	0	\\
2.83e-09	0	\\
2.93e-09	0	\\
3.04e-09	0	\\
3.14e-09	0	\\
3.24e-09	0	\\
3.34e-09	0	\\
3.45e-09	0	\\
3.55e-09	0	\\
3.65e-09	0	\\
3.75e-09	0	\\
3.86e-09	0	\\
3.96e-09	0	\\
4.06e-09	0	\\
4.16e-09	0	\\
4.27e-09	0	\\
4.37e-09	0	\\
4.47e-09	0	\\
4.57e-09	0	\\
4.68e-09	0	\\
4.78e-09	0	\\
4.89e-09	0	\\
4.99e-09	0	\\
5e-09	0	\\
};
\addplot [color=mycolor2,solid,forget plot]
  table[row sep=crcr]{
0	0	\\
1.1e-10	0	\\
2.2e-10	0	\\
3.3e-10	0	\\
4.4e-10	0	\\
5.4e-10	0	\\
6.5e-10	0	\\
7.5e-10	0	\\
8.6e-10	0	\\
9.6e-10	0	\\
1.07e-09	0	\\
1.18e-09	0	\\
1.28e-09	0	\\
1.38e-09	0	\\
1.49e-09	0	\\
1.59e-09	0	\\
1.69e-09	0	\\
1.8e-09	0	\\
1.9e-09	0	\\
2.01e-09	0	\\
2.11e-09	0	\\
2.21e-09	0	\\
2.32e-09	0	\\
2.42e-09	0	\\
2.52e-09	0	\\
2.63e-09	0	\\
2.73e-09	0	\\
2.83e-09	0	\\
2.93e-09	0	\\
3.04e-09	0	\\
3.14e-09	0	\\
3.24e-09	0	\\
3.34e-09	0	\\
3.45e-09	0	\\
3.55e-09	0	\\
3.65e-09	0	\\
3.75e-09	0	\\
3.86e-09	0	\\
3.96e-09	0	\\
4.06e-09	0	\\
4.16e-09	0	\\
4.27e-09	0	\\
4.37e-09	0	\\
4.47e-09	0	\\
4.57e-09	0	\\
4.68e-09	0	\\
4.78e-09	0	\\
4.89e-09	0	\\
4.99e-09	0	\\
5e-09	0	\\
};
\addplot [color=mycolor3,solid,forget plot]
  table[row sep=crcr]{
0	0	\\
1.1e-10	0	\\
2.2e-10	0	\\
3.3e-10	0	\\
4.4e-10	0	\\
5.4e-10	0	\\
6.5e-10	0	\\
7.5e-10	0	\\
8.6e-10	0	\\
9.6e-10	0	\\
1.07e-09	0	\\
1.18e-09	0	\\
1.28e-09	0	\\
1.38e-09	0	\\
1.49e-09	0	\\
1.59e-09	0	\\
1.69e-09	0	\\
1.8e-09	0	\\
1.9e-09	0	\\
2.01e-09	0	\\
2.11e-09	0	\\
2.21e-09	0	\\
2.32e-09	0	\\
2.42e-09	0	\\
2.52e-09	0	\\
2.63e-09	0	\\
2.73e-09	0	\\
2.83e-09	0	\\
2.93e-09	0	\\
3.04e-09	0	\\
3.14e-09	0	\\
3.24e-09	0	\\
3.34e-09	0	\\
3.45e-09	0	\\
3.55e-09	0	\\
3.65e-09	0	\\
3.75e-09	0	\\
3.86e-09	0	\\
3.96e-09	0	\\
4.06e-09	0	\\
4.16e-09	0	\\
4.27e-09	0	\\
4.37e-09	0	\\
4.47e-09	0	\\
4.57e-09	0	\\
4.68e-09	0	\\
4.78e-09	0	\\
4.89e-09	0	\\
4.99e-09	0	\\
5e-09	0	\\
};
\addplot [color=darkgray,solid,forget plot]
  table[row sep=crcr]{
0	0	\\
1.1e-10	0	\\
2.2e-10	0	\\
3.3e-10	0	\\
4.4e-10	0	\\
5.4e-10	0	\\
6.5e-10	0	\\
7.5e-10	0	\\
8.6e-10	0	\\
9.6e-10	0	\\
1.07e-09	0	\\
1.18e-09	0	\\
1.28e-09	0	\\
1.38e-09	0	\\
1.49e-09	0	\\
1.59e-09	0	\\
1.69e-09	0	\\
1.8e-09	0	\\
1.9e-09	0	\\
2.01e-09	0	\\
2.11e-09	0	\\
2.21e-09	0	\\
2.32e-09	0	\\
2.42e-09	0	\\
2.52e-09	0	\\
2.63e-09	0	\\
2.73e-09	0	\\
2.83e-09	0	\\
2.93e-09	0	\\
3.04e-09	0	\\
3.14e-09	0	\\
3.24e-09	0	\\
3.34e-09	0	\\
3.45e-09	0	\\
3.55e-09	0	\\
3.65e-09	0	\\
3.75e-09	0	\\
3.86e-09	0	\\
3.96e-09	0	\\
4.06e-09	0	\\
4.16e-09	0	\\
4.27e-09	0	\\
4.37e-09	0	\\
4.47e-09	0	\\
4.57e-09	0	\\
4.68e-09	0	\\
4.78e-09	0	\\
4.89e-09	0	\\
4.99e-09	0	\\
5e-09	0	\\
};
\addplot [color=blue,solid,forget plot]
  table[row sep=crcr]{
0	0	\\
1.1e-10	0	\\
2.2e-10	0	\\
3.3e-10	0	\\
4.4e-10	0	\\
5.4e-10	0	\\
6.5e-10	0	\\
7.5e-10	0	\\
8.6e-10	0	\\
9.6e-10	0	\\
1.07e-09	0	\\
1.18e-09	0	\\
1.28e-09	0	\\
1.38e-09	0	\\
1.49e-09	0	\\
1.59e-09	0	\\
1.69e-09	0	\\
1.8e-09	0	\\
1.9e-09	0	\\
2.01e-09	0	\\
2.11e-09	0	\\
2.21e-09	0	\\
2.32e-09	0	\\
2.42e-09	0	\\
2.52e-09	0	\\
2.63e-09	0	\\
2.73e-09	0	\\
2.83e-09	0	\\
2.93e-09	0	\\
3.04e-09	0	\\
3.14e-09	0	\\
3.24e-09	0	\\
3.34e-09	0	\\
3.45e-09	0	\\
3.55e-09	0	\\
3.65e-09	0	\\
3.75e-09	0	\\
3.86e-09	0	\\
3.96e-09	0	\\
4.06e-09	0	\\
4.16e-09	0	\\
4.27e-09	0	\\
4.37e-09	0	\\
4.47e-09	0	\\
4.57e-09	0	\\
4.68e-09	0	\\
4.78e-09	0	\\
4.89e-09	0	\\
4.99e-09	0	\\
5e-09	0	\\
};
\addplot [color=black!50!green,solid,forget plot]
  table[row sep=crcr]{
0	0	\\
1.1e-10	0	\\
2.2e-10	0	\\
3.3e-10	0	\\
4.4e-10	0	\\
5.4e-10	0	\\
6.5e-10	0	\\
7.5e-10	0	\\
8.6e-10	0	\\
9.6e-10	0	\\
1.07e-09	0	\\
1.18e-09	0	\\
1.28e-09	0	\\
1.38e-09	0	\\
1.49e-09	0	\\
1.59e-09	0	\\
1.69e-09	0	\\
1.8e-09	0	\\
1.9e-09	0	\\
2.01e-09	0	\\
2.11e-09	0	\\
2.21e-09	0	\\
2.32e-09	0	\\
2.42e-09	0	\\
2.52e-09	0	\\
2.63e-09	0	\\
2.73e-09	0	\\
2.83e-09	0	\\
2.93e-09	0	\\
3.04e-09	0	\\
3.14e-09	0	\\
3.24e-09	0	\\
3.34e-09	0	\\
3.45e-09	0	\\
3.55e-09	0	\\
3.65e-09	0	\\
3.75e-09	0	\\
3.86e-09	0	\\
3.96e-09	0	\\
4.06e-09	0	\\
4.16e-09	0	\\
4.27e-09	0	\\
4.37e-09	0	\\
4.47e-09	0	\\
4.57e-09	0	\\
4.68e-09	0	\\
4.78e-09	0	\\
4.89e-09	0	\\
4.99e-09	0	\\
5e-09	0	\\
};
\addplot [color=red,solid,forget plot]
  table[row sep=crcr]{
0	0	\\
1.1e-10	0	\\
2.2e-10	0	\\
3.3e-10	0	\\
4.4e-10	0	\\
5.4e-10	0	\\
6.5e-10	0	\\
7.5e-10	0	\\
8.6e-10	0	\\
9.6e-10	0	\\
1.07e-09	0	\\
1.18e-09	0	\\
1.28e-09	0	\\
1.38e-09	0	\\
1.49e-09	0	\\
1.59e-09	0	\\
1.69e-09	0	\\
1.8e-09	0	\\
1.9e-09	0	\\
2.01e-09	0	\\
2.11e-09	0	\\
2.21e-09	0	\\
2.32e-09	0	\\
2.42e-09	0	\\
2.52e-09	0	\\
2.63e-09	0	\\
2.73e-09	0	\\
2.83e-09	0	\\
2.93e-09	0	\\
3.04e-09	0	\\
3.14e-09	0	\\
3.24e-09	0	\\
3.34e-09	0	\\
3.45e-09	0	\\
3.55e-09	0	\\
3.65e-09	0	\\
3.75e-09	0	\\
3.86e-09	0	\\
3.96e-09	0	\\
4.06e-09	0	\\
4.16e-09	0	\\
4.27e-09	0	\\
4.37e-09	0	\\
4.47e-09	0	\\
4.57e-09	0	\\
4.68e-09	0	\\
4.78e-09	0	\\
4.89e-09	0	\\
4.99e-09	0	\\
5e-09	0	\\
};
\addplot [color=mycolor1,solid,forget plot]
  table[row sep=crcr]{
0	0	\\
1.1e-10	0	\\
2.2e-10	0	\\
3.3e-10	0	\\
4.4e-10	0	\\
5.4e-10	0	\\
6.5e-10	0	\\
7.5e-10	0	\\
8.6e-10	0	\\
9.6e-10	0	\\
1.07e-09	0	\\
1.18e-09	0	\\
1.28e-09	0	\\
1.38e-09	0	\\
1.49e-09	0	\\
1.59e-09	0	\\
1.69e-09	0	\\
1.8e-09	0	\\
1.9e-09	0	\\
2.01e-09	0	\\
2.11e-09	0	\\
2.21e-09	0	\\
2.32e-09	0	\\
2.42e-09	0	\\
2.52e-09	0	\\
2.63e-09	0	\\
2.73e-09	0	\\
2.83e-09	0	\\
2.93e-09	0	\\
3.04e-09	0	\\
3.14e-09	0	\\
3.24e-09	0	\\
3.34e-09	0	\\
3.45e-09	0	\\
3.55e-09	0	\\
3.65e-09	0	\\
3.75e-09	0	\\
3.86e-09	0	\\
3.96e-09	0	\\
4.06e-09	0	\\
4.16e-09	0	\\
4.27e-09	0	\\
4.37e-09	0	\\
4.47e-09	0	\\
4.57e-09	0	\\
4.68e-09	0	\\
4.78e-09	0	\\
4.89e-09	0	\\
4.99e-09	0	\\
5e-09	0	\\
};
\addplot [color=mycolor2,solid,forget plot]
  table[row sep=crcr]{
0	0	\\
1.1e-10	0	\\
2.2e-10	0	\\
3.3e-10	0	\\
4.4e-10	0	\\
5.4e-10	0	\\
6.5e-10	0	\\
7.5e-10	0	\\
8.6e-10	0	\\
9.6e-10	0	\\
1.07e-09	0	\\
1.18e-09	0	\\
1.28e-09	0	\\
1.38e-09	0	\\
1.49e-09	0	\\
1.59e-09	0	\\
1.69e-09	0	\\
1.8e-09	0	\\
1.9e-09	0	\\
2.01e-09	0	\\
2.11e-09	0	\\
2.21e-09	0	\\
2.32e-09	0	\\
2.42e-09	0	\\
2.52e-09	0	\\
2.63e-09	0	\\
2.73e-09	0	\\
2.83e-09	0	\\
2.93e-09	0	\\
3.04e-09	0	\\
3.14e-09	0	\\
3.24e-09	0	\\
3.34e-09	0	\\
3.45e-09	0	\\
3.55e-09	0	\\
3.65e-09	0	\\
3.75e-09	0	\\
3.86e-09	0	\\
3.96e-09	0	\\
4.06e-09	0	\\
4.16e-09	0	\\
4.27e-09	0	\\
4.37e-09	0	\\
4.47e-09	0	\\
4.57e-09	0	\\
4.68e-09	0	\\
4.78e-09	0	\\
4.89e-09	0	\\
4.99e-09	0	\\
5e-09	0	\\
};
\addplot [color=mycolor3,solid,forget plot]
  table[row sep=crcr]{
0	0	\\
1.1e-10	0	\\
2.2e-10	0	\\
3.3e-10	0	\\
4.4e-10	0	\\
5.4e-10	0	\\
6.5e-10	0	\\
7.5e-10	0	\\
8.6e-10	0	\\
9.6e-10	0	\\
1.07e-09	0	\\
1.18e-09	0	\\
1.28e-09	0	\\
1.38e-09	0	\\
1.49e-09	0	\\
1.59e-09	0	\\
1.69e-09	0	\\
1.8e-09	0	\\
1.9e-09	0	\\
2.01e-09	0	\\
2.11e-09	0	\\
2.21e-09	0	\\
2.32e-09	0	\\
2.42e-09	0	\\
2.52e-09	0	\\
2.63e-09	0	\\
2.73e-09	0	\\
2.83e-09	0	\\
2.93e-09	0	\\
3.04e-09	0	\\
3.14e-09	0	\\
3.24e-09	0	\\
3.34e-09	0	\\
3.45e-09	0	\\
3.55e-09	0	\\
3.65e-09	0	\\
3.75e-09	0	\\
3.86e-09	0	\\
3.96e-09	0	\\
4.06e-09	0	\\
4.16e-09	0	\\
4.27e-09	0	\\
4.37e-09	0	\\
4.47e-09	0	\\
4.57e-09	0	\\
4.68e-09	0	\\
4.78e-09	0	\\
4.89e-09	0	\\
4.99e-09	0	\\
5e-09	0	\\
};
\addplot [color=darkgray,solid,forget plot]
  table[row sep=crcr]{
0	0	\\
1.1e-10	0	\\
2.2e-10	0	\\
3.3e-10	0	\\
4.4e-10	0	\\
5.4e-10	0	\\
6.5e-10	0	\\
7.5e-10	0	\\
8.6e-10	0	\\
9.6e-10	0	\\
1.07e-09	0	\\
1.18e-09	0	\\
1.28e-09	0	\\
1.38e-09	0	\\
1.49e-09	0	\\
1.59e-09	0	\\
1.69e-09	0	\\
1.8e-09	0	\\
1.9e-09	0	\\
2.01e-09	0	\\
2.11e-09	0	\\
2.21e-09	0	\\
2.32e-09	0	\\
2.42e-09	0	\\
2.52e-09	0	\\
2.63e-09	0	\\
2.73e-09	0	\\
2.83e-09	0	\\
2.93e-09	0	\\
3.04e-09	0	\\
3.14e-09	0	\\
3.24e-09	0	\\
3.34e-09	0	\\
3.45e-09	0	\\
3.55e-09	0	\\
3.65e-09	0	\\
3.75e-09	0	\\
3.86e-09	0	\\
3.96e-09	0	\\
4.06e-09	0	\\
4.16e-09	0	\\
4.27e-09	0	\\
4.37e-09	0	\\
4.47e-09	0	\\
4.57e-09	0	\\
4.68e-09	0	\\
4.78e-09	0	\\
4.89e-09	0	\\
4.99e-09	0	\\
5e-09	0	\\
};
\addplot [color=blue,solid,forget plot]
  table[row sep=crcr]{
0	0	\\
1.1e-10	0	\\
2.2e-10	0	\\
3.3e-10	0	\\
4.4e-10	0	\\
5.4e-10	0	\\
6.5e-10	0	\\
7.5e-10	0	\\
8.6e-10	0	\\
9.6e-10	0	\\
1.07e-09	0	\\
1.18e-09	0	\\
1.28e-09	0	\\
1.38e-09	0	\\
1.49e-09	0	\\
1.59e-09	0	\\
1.69e-09	0	\\
1.8e-09	0	\\
1.9e-09	0	\\
2.01e-09	0	\\
2.11e-09	0	\\
2.21e-09	0	\\
2.32e-09	0	\\
2.42e-09	0	\\
2.52e-09	0	\\
2.63e-09	0	\\
2.73e-09	0	\\
2.83e-09	0	\\
2.93e-09	0	\\
3.04e-09	0	\\
3.14e-09	0	\\
3.24e-09	0	\\
3.34e-09	0	\\
3.45e-09	0	\\
3.55e-09	0	\\
3.65e-09	0	\\
3.75e-09	0	\\
3.86e-09	0	\\
3.96e-09	0	\\
4.06e-09	0	\\
4.16e-09	0	\\
4.27e-09	0	\\
4.37e-09	0	\\
4.47e-09	0	\\
4.57e-09	0	\\
4.68e-09	0	\\
4.78e-09	0	\\
4.89e-09	0	\\
4.99e-09	0	\\
5e-09	0	\\
};
\addplot [color=black!50!green,solid,forget plot]
  table[row sep=crcr]{
0	0	\\
1.1e-10	0	\\
2.2e-10	0	\\
3.3e-10	0	\\
4.4e-10	0	\\
5.4e-10	0	\\
6.5e-10	0	\\
7.5e-10	0	\\
8.6e-10	0	\\
9.6e-10	0	\\
1.07e-09	0	\\
1.18e-09	0	\\
1.28e-09	0	\\
1.38e-09	0	\\
1.49e-09	0	\\
1.59e-09	0	\\
1.69e-09	0	\\
1.8e-09	0	\\
1.9e-09	0	\\
2.01e-09	0	\\
2.11e-09	0	\\
2.21e-09	0	\\
2.32e-09	0	\\
2.42e-09	0	\\
2.52e-09	0	\\
2.63e-09	0	\\
2.73e-09	0	\\
2.83e-09	0	\\
2.93e-09	0	\\
3.04e-09	0	\\
3.14e-09	0	\\
3.24e-09	0	\\
3.34e-09	0	\\
3.45e-09	0	\\
3.55e-09	0	\\
3.65e-09	0	\\
3.75e-09	0	\\
3.86e-09	0	\\
3.96e-09	0	\\
4.06e-09	0	\\
4.16e-09	0	\\
4.27e-09	0	\\
4.37e-09	0	\\
4.47e-09	0	\\
4.57e-09	0	\\
4.68e-09	0	\\
4.78e-09	0	\\
4.89e-09	0	\\
4.99e-09	0	\\
5e-09	0	\\
};
\addplot [color=red,solid,forget plot]
  table[row sep=crcr]{
0	0	\\
1.1e-10	0	\\
2.2e-10	0	\\
3.3e-10	0	\\
4.4e-10	0	\\
5.4e-10	0	\\
6.5e-10	0	\\
7.5e-10	0	\\
8.6e-10	0	\\
9.6e-10	0	\\
1.07e-09	0	\\
1.18e-09	0	\\
1.28e-09	0	\\
1.38e-09	0	\\
1.49e-09	0	\\
1.59e-09	0	\\
1.69e-09	0	\\
1.8e-09	0	\\
1.9e-09	0	\\
2.01e-09	0	\\
2.11e-09	0	\\
2.21e-09	0	\\
2.32e-09	0	\\
2.42e-09	0	\\
2.52e-09	0	\\
2.63e-09	0	\\
2.73e-09	0	\\
2.83e-09	0	\\
2.93e-09	0	\\
3.04e-09	0	\\
3.14e-09	0	\\
3.24e-09	0	\\
3.34e-09	0	\\
3.45e-09	0	\\
3.55e-09	0	\\
3.65e-09	0	\\
3.75e-09	0	\\
3.86e-09	0	\\
3.96e-09	0	\\
4.06e-09	0	\\
4.16e-09	0	\\
4.27e-09	0	\\
4.37e-09	0	\\
4.47e-09	0	\\
4.57e-09	0	\\
4.68e-09	0	\\
4.78e-09	0	\\
4.89e-09	0	\\
4.99e-09	0	\\
5e-09	0	\\
};
\addplot [color=mycolor1,solid,forget plot]
  table[row sep=crcr]{
0	0	\\
1.1e-10	0	\\
2.2e-10	0	\\
3.3e-10	0	\\
4.4e-10	0	\\
5.4e-10	0	\\
6.5e-10	0	\\
7.5e-10	0	\\
8.6e-10	0	\\
9.6e-10	0	\\
1.07e-09	0	\\
1.18e-09	0	\\
1.28e-09	0	\\
1.38e-09	0	\\
1.49e-09	0	\\
1.59e-09	0	\\
1.69e-09	0	\\
1.8e-09	0	\\
1.9e-09	0	\\
2.01e-09	0	\\
2.11e-09	0	\\
2.21e-09	0	\\
2.32e-09	0	\\
2.42e-09	0	\\
2.52e-09	0	\\
2.63e-09	0	\\
2.73e-09	0	\\
2.83e-09	0	\\
2.93e-09	0	\\
3.04e-09	0	\\
3.14e-09	0	\\
3.24e-09	0	\\
3.34e-09	0	\\
3.45e-09	0	\\
3.55e-09	0	\\
3.65e-09	0	\\
3.75e-09	0	\\
3.86e-09	0	\\
3.96e-09	0	\\
4.06e-09	0	\\
4.16e-09	0	\\
4.27e-09	0	\\
4.37e-09	0	\\
4.47e-09	0	\\
4.57e-09	0	\\
4.68e-09	0	\\
4.78e-09	0	\\
4.89e-09	0	\\
4.99e-09	0	\\
5e-09	0	\\
};
\addplot [color=mycolor2,solid,forget plot]
  table[row sep=crcr]{
0	0	\\
1.1e-10	0	\\
2.2e-10	0	\\
3.3e-10	0	\\
4.4e-10	0	\\
5.4e-10	0	\\
6.5e-10	0	\\
7.5e-10	0	\\
8.6e-10	0	\\
9.6e-10	0	\\
1.07e-09	0	\\
1.18e-09	0	\\
1.28e-09	0	\\
1.38e-09	0	\\
1.49e-09	0	\\
1.59e-09	0	\\
1.69e-09	0	\\
1.8e-09	0	\\
1.9e-09	0	\\
2.01e-09	0	\\
2.11e-09	0	\\
2.21e-09	0	\\
2.32e-09	0	\\
2.42e-09	0	\\
2.52e-09	0	\\
2.63e-09	0	\\
2.73e-09	0	\\
2.83e-09	0	\\
2.93e-09	0	\\
3.04e-09	0	\\
3.14e-09	0	\\
3.24e-09	0	\\
3.34e-09	0	\\
3.45e-09	0	\\
3.55e-09	0	\\
3.65e-09	0	\\
3.75e-09	0	\\
3.86e-09	0	\\
3.96e-09	0	\\
4.06e-09	0	\\
4.16e-09	0	\\
4.27e-09	0	\\
4.37e-09	0	\\
4.47e-09	0	\\
4.57e-09	0	\\
4.68e-09	0	\\
4.78e-09	0	\\
4.89e-09	0	\\
4.99e-09	0	\\
5e-09	0	\\
};
\addplot [color=mycolor3,solid,forget plot]
  table[row sep=crcr]{
0	0	\\
1.1e-10	0	\\
2.2e-10	0	\\
3.3e-10	0	\\
4.4e-10	0	\\
5.4e-10	0	\\
6.5e-10	0	\\
7.5e-10	0	\\
8.6e-10	0	\\
9.6e-10	0	\\
1.07e-09	0	\\
1.18e-09	0	\\
1.28e-09	0	\\
1.38e-09	0	\\
1.49e-09	0	\\
1.59e-09	0	\\
1.69e-09	0	\\
1.8e-09	0	\\
1.9e-09	0	\\
2.01e-09	0	\\
2.11e-09	0	\\
2.21e-09	0	\\
2.32e-09	0	\\
2.42e-09	0	\\
2.52e-09	0	\\
2.63e-09	0	\\
2.73e-09	0	\\
2.83e-09	0	\\
2.93e-09	0	\\
3.04e-09	0	\\
3.14e-09	0	\\
3.24e-09	0	\\
3.34e-09	0	\\
3.45e-09	0	\\
3.55e-09	0	\\
3.65e-09	0	\\
3.75e-09	0	\\
3.86e-09	0	\\
3.96e-09	0	\\
4.06e-09	0	\\
4.16e-09	0	\\
4.27e-09	0	\\
4.37e-09	0	\\
4.47e-09	0	\\
4.57e-09	0	\\
4.68e-09	0	\\
4.78e-09	0	\\
4.89e-09	0	\\
4.99e-09	0	\\
5e-09	0	\\
};
\addplot [color=darkgray,solid,forget plot]
  table[row sep=crcr]{
0	0	\\
1.1e-10	0	\\
2.2e-10	0	\\
3.3e-10	0	\\
4.4e-10	0	\\
5.4e-10	0	\\
6.5e-10	0	\\
7.5e-10	0	\\
8.6e-10	0	\\
9.6e-10	0	\\
1.07e-09	0	\\
1.18e-09	0	\\
1.28e-09	0	\\
1.38e-09	0	\\
1.49e-09	0	\\
1.59e-09	0	\\
1.69e-09	0	\\
1.8e-09	0	\\
1.9e-09	0	\\
2.01e-09	0	\\
2.11e-09	0	\\
2.21e-09	0	\\
2.32e-09	0	\\
2.42e-09	0	\\
2.52e-09	0	\\
2.63e-09	0	\\
2.73e-09	0	\\
2.83e-09	0	\\
2.93e-09	0	\\
3.04e-09	0	\\
3.14e-09	0	\\
3.24e-09	0	\\
3.34e-09	0	\\
3.45e-09	0	\\
3.55e-09	0	\\
3.65e-09	0	\\
3.75e-09	0	\\
3.86e-09	0	\\
3.96e-09	0	\\
4.06e-09	0	\\
4.16e-09	0	\\
4.27e-09	0	\\
4.37e-09	0	\\
4.47e-09	0	\\
4.57e-09	0	\\
4.68e-09	0	\\
4.78e-09	0	\\
4.89e-09	0	\\
4.99e-09	0	\\
5e-09	0	\\
};
\addplot [color=blue,solid,forget plot]
  table[row sep=crcr]{
0	0	\\
1.1e-10	0	\\
2.2e-10	0	\\
3.3e-10	0	\\
4.4e-10	0	\\
5.4e-10	0	\\
6.5e-10	0	\\
7.5e-10	0	\\
8.6e-10	0	\\
9.6e-10	0	\\
1.07e-09	0	\\
1.18e-09	0	\\
1.28e-09	0	\\
1.38e-09	0	\\
1.49e-09	0	\\
1.59e-09	0	\\
1.69e-09	0	\\
1.8e-09	0	\\
1.9e-09	0	\\
2.01e-09	0	\\
2.11e-09	0	\\
2.21e-09	0	\\
2.32e-09	0	\\
2.42e-09	0	\\
2.52e-09	0	\\
2.63e-09	0	\\
2.73e-09	0	\\
2.83e-09	0	\\
2.93e-09	0	\\
3.04e-09	0	\\
3.14e-09	0	\\
3.24e-09	0	\\
3.34e-09	0	\\
3.45e-09	0	\\
3.55e-09	0	\\
3.65e-09	0	\\
3.75e-09	0	\\
3.86e-09	0	\\
3.96e-09	0	\\
4.06e-09	0	\\
4.16e-09	0	\\
4.27e-09	0	\\
4.37e-09	0	\\
4.47e-09	0	\\
4.57e-09	0	\\
4.68e-09	0	\\
4.78e-09	0	\\
4.89e-09	0	\\
4.99e-09	0	\\
5e-09	0	\\
};
\addplot [color=black!50!green,solid,forget plot]
  table[row sep=crcr]{
0	0	\\
1.1e-10	0	\\
2.2e-10	0	\\
3.3e-10	0	\\
4.4e-10	0	\\
5.4e-10	0	\\
6.5e-10	0	\\
7.5e-10	0	\\
8.6e-10	0	\\
9.6e-10	0	\\
1.07e-09	0	\\
1.18e-09	0	\\
1.28e-09	0	\\
1.38e-09	0	\\
1.49e-09	0	\\
1.59e-09	0	\\
1.69e-09	0	\\
1.8e-09	0	\\
1.9e-09	0	\\
2.01e-09	0	\\
2.11e-09	0	\\
2.21e-09	0	\\
2.32e-09	0	\\
2.42e-09	0	\\
2.52e-09	0	\\
2.63e-09	0	\\
2.73e-09	0	\\
2.83e-09	0	\\
2.93e-09	0	\\
3.04e-09	0	\\
3.14e-09	0	\\
3.24e-09	0	\\
3.34e-09	0	\\
3.45e-09	0	\\
3.55e-09	0	\\
3.65e-09	0	\\
3.75e-09	0	\\
3.86e-09	0	\\
3.96e-09	0	\\
4.06e-09	0	\\
4.16e-09	0	\\
4.27e-09	0	\\
4.37e-09	0	\\
4.47e-09	0	\\
4.57e-09	0	\\
4.68e-09	0	\\
4.78e-09	0	\\
4.89e-09	0	\\
4.99e-09	0	\\
5e-09	0	\\
};
\addplot [color=red,solid,forget plot]
  table[row sep=crcr]{
0	0	\\
1.1e-10	0	\\
2.2e-10	0	\\
3.3e-10	0	\\
4.4e-10	0	\\
5.4e-10	0	\\
6.5e-10	0	\\
7.5e-10	0	\\
8.6e-10	0	\\
9.6e-10	0	\\
1.07e-09	0	\\
1.18e-09	0	\\
1.28e-09	0	\\
1.38e-09	0	\\
1.49e-09	0	\\
1.59e-09	0	\\
1.69e-09	0	\\
1.8e-09	0	\\
1.9e-09	0	\\
2.01e-09	0	\\
2.11e-09	0	\\
2.21e-09	0	\\
2.32e-09	0	\\
2.42e-09	0	\\
2.52e-09	0	\\
2.63e-09	0	\\
2.73e-09	0	\\
2.83e-09	0	\\
2.93e-09	0	\\
3.04e-09	0	\\
3.14e-09	0	\\
3.24e-09	0	\\
3.34e-09	0	\\
3.45e-09	0	\\
3.55e-09	0	\\
3.65e-09	0	\\
3.75e-09	0	\\
3.86e-09	0	\\
3.96e-09	0	\\
4.06e-09	0	\\
4.16e-09	0	\\
4.27e-09	0	\\
4.37e-09	0	\\
4.47e-09	0	\\
4.57e-09	0	\\
4.68e-09	0	\\
4.78e-09	0	\\
4.89e-09	0	\\
4.99e-09	0	\\
5e-09	0	\\
};
\addplot [color=mycolor1,solid,forget plot]
  table[row sep=crcr]{
0	0	\\
1.1e-10	0	\\
2.2e-10	0	\\
3.3e-10	0	\\
4.4e-10	0	\\
5.4e-10	0	\\
6.5e-10	0	\\
7.5e-10	0	\\
8.6e-10	0	\\
9.6e-10	0	\\
1.07e-09	0	\\
1.18e-09	0	\\
1.28e-09	0	\\
1.38e-09	0	\\
1.49e-09	0	\\
1.59e-09	0	\\
1.69e-09	0	\\
1.8e-09	0	\\
1.9e-09	0	\\
2.01e-09	0	\\
2.11e-09	0	\\
2.21e-09	0	\\
2.32e-09	0	\\
2.42e-09	0	\\
2.52e-09	0	\\
2.63e-09	0	\\
2.73e-09	0	\\
2.83e-09	0	\\
2.93e-09	0	\\
3.04e-09	0	\\
3.14e-09	0	\\
3.24e-09	0	\\
3.34e-09	0	\\
3.45e-09	0	\\
3.55e-09	0	\\
3.65e-09	0	\\
3.75e-09	0	\\
3.86e-09	0	\\
3.96e-09	0	\\
4.06e-09	0	\\
4.16e-09	0	\\
4.27e-09	0	\\
4.37e-09	0	\\
4.47e-09	0	\\
4.57e-09	0	\\
4.68e-09	0	\\
4.78e-09	0	\\
4.89e-09	0	\\
4.99e-09	0	\\
5e-09	0	\\
};
\addplot [color=mycolor2,solid,forget plot]
  table[row sep=crcr]{
0	0	\\
1.1e-10	0	\\
2.2e-10	0	\\
3.3e-10	0	\\
4.4e-10	0	\\
5.4e-10	0	\\
6.5e-10	0	\\
7.5e-10	0	\\
8.6e-10	0	\\
9.6e-10	0	\\
1.07e-09	0	\\
1.18e-09	0	\\
1.28e-09	0	\\
1.38e-09	0	\\
1.49e-09	0	\\
1.59e-09	0	\\
1.69e-09	0	\\
1.8e-09	0	\\
1.9e-09	0	\\
2.01e-09	0	\\
2.11e-09	0	\\
2.21e-09	0	\\
2.32e-09	0	\\
2.42e-09	0	\\
2.52e-09	0	\\
2.63e-09	0	\\
2.73e-09	0	\\
2.83e-09	0	\\
2.93e-09	0	\\
3.04e-09	0	\\
3.14e-09	0	\\
3.24e-09	0	\\
3.34e-09	0	\\
3.45e-09	0	\\
3.55e-09	0	\\
3.65e-09	0	\\
3.75e-09	0	\\
3.86e-09	0	\\
3.96e-09	0	\\
4.06e-09	0	\\
4.16e-09	0	\\
4.27e-09	0	\\
4.37e-09	0	\\
4.47e-09	0	\\
4.57e-09	0	\\
4.68e-09	0	\\
4.78e-09	0	\\
4.89e-09	0	\\
4.99e-09	0	\\
5e-09	0	\\
};
\addplot [color=mycolor3,solid,forget plot]
  table[row sep=crcr]{
0	0	\\
1.1e-10	0	\\
2.2e-10	0	\\
3.3e-10	0	\\
4.4e-10	0	\\
5.4e-10	0	\\
6.5e-10	0	\\
7.5e-10	0	\\
8.6e-10	0	\\
9.6e-10	0	\\
1.07e-09	0	\\
1.18e-09	0	\\
1.28e-09	0	\\
1.38e-09	0	\\
1.49e-09	0	\\
1.59e-09	0	\\
1.69e-09	0	\\
1.8e-09	0	\\
1.9e-09	0	\\
2.01e-09	0	\\
2.11e-09	0	\\
2.21e-09	0	\\
2.32e-09	0	\\
2.42e-09	0	\\
2.52e-09	0	\\
2.63e-09	0	\\
2.73e-09	0	\\
2.83e-09	0	\\
2.93e-09	0	\\
3.04e-09	0	\\
3.14e-09	0	\\
3.24e-09	0	\\
3.34e-09	0	\\
3.45e-09	0	\\
3.55e-09	0	\\
3.65e-09	0	\\
3.75e-09	0	\\
3.86e-09	0	\\
3.96e-09	0	\\
4.06e-09	0	\\
4.16e-09	0	\\
4.27e-09	0	\\
4.37e-09	0	\\
4.47e-09	0	\\
4.57e-09	0	\\
4.68e-09	0	\\
4.78e-09	0	\\
4.89e-09	0	\\
4.99e-09	0	\\
5e-09	0	\\
};
\addplot [color=darkgray,solid,forget plot]
  table[row sep=crcr]{
0	0	\\
1.1e-10	0	\\
2.2e-10	0	\\
3.3e-10	0	\\
4.4e-10	0	\\
5.4e-10	0	\\
6.5e-10	0	\\
7.5e-10	0	\\
8.6e-10	0	\\
9.6e-10	0	\\
1.07e-09	0	\\
1.18e-09	0	\\
1.28e-09	0	\\
1.38e-09	0	\\
1.49e-09	0	\\
1.59e-09	0	\\
1.69e-09	0	\\
1.8e-09	0	\\
1.9e-09	0	\\
2.01e-09	0	\\
2.11e-09	0	\\
2.21e-09	0	\\
2.32e-09	0	\\
2.42e-09	0	\\
2.52e-09	0	\\
2.63e-09	0	\\
2.73e-09	0	\\
2.83e-09	0	\\
2.93e-09	0	\\
3.04e-09	0	\\
3.14e-09	0	\\
3.24e-09	0	\\
3.34e-09	0	\\
3.45e-09	0	\\
3.55e-09	0	\\
3.65e-09	0	\\
3.75e-09	0	\\
3.86e-09	0	\\
3.96e-09	0	\\
4.06e-09	0	\\
4.16e-09	0	\\
4.27e-09	0	\\
4.37e-09	0	\\
4.47e-09	0	\\
4.57e-09	0	\\
4.68e-09	0	\\
4.78e-09	0	\\
4.89e-09	0	\\
4.99e-09	0	\\
5e-09	0	\\
};
\addplot [color=blue,solid,forget plot]
  table[row sep=crcr]{
0	0	\\
1.1e-10	0	\\
2.2e-10	0	\\
3.3e-10	0	\\
4.4e-10	0	\\
5.4e-10	0	\\
6.5e-10	0	\\
7.5e-10	0	\\
8.6e-10	0	\\
9.6e-10	0	\\
1.07e-09	0	\\
1.18e-09	0	\\
1.28e-09	0	\\
1.38e-09	0	\\
1.49e-09	0	\\
1.59e-09	0	\\
1.69e-09	0	\\
1.8e-09	0	\\
1.9e-09	0	\\
2.01e-09	0	\\
2.11e-09	0	\\
2.21e-09	0	\\
2.32e-09	0	\\
2.42e-09	0	\\
2.52e-09	0	\\
2.63e-09	0	\\
2.73e-09	0	\\
2.83e-09	0	\\
2.93e-09	0	\\
3.04e-09	0	\\
3.14e-09	0	\\
3.24e-09	0	\\
3.34e-09	0	\\
3.45e-09	0	\\
3.55e-09	0	\\
3.65e-09	0	\\
3.75e-09	0	\\
3.86e-09	0	\\
3.96e-09	0	\\
4.06e-09	0	\\
4.16e-09	0	\\
4.27e-09	0	\\
4.37e-09	0	\\
4.47e-09	0	\\
4.57e-09	0	\\
4.68e-09	0	\\
4.78e-09	0	\\
4.89e-09	0	\\
4.99e-09	0	\\
5e-09	0	\\
};
\addplot [color=black!50!green,solid,forget plot]
  table[row sep=crcr]{
0	0	\\
1.1e-10	0	\\
2.2e-10	0	\\
3.3e-10	0	\\
4.4e-10	0	\\
5.4e-10	0	\\
6.5e-10	0	\\
7.5e-10	0	\\
8.6e-10	0	\\
9.6e-10	0	\\
1.07e-09	0	\\
1.18e-09	0	\\
1.28e-09	0	\\
1.38e-09	0	\\
1.49e-09	0	\\
1.59e-09	0	\\
1.69e-09	0	\\
1.8e-09	0	\\
1.9e-09	0	\\
2.01e-09	0	\\
2.11e-09	0	\\
2.21e-09	0	\\
2.32e-09	0	\\
2.42e-09	0	\\
2.52e-09	0	\\
2.63e-09	0	\\
2.73e-09	0	\\
2.83e-09	0	\\
2.93e-09	0	\\
3.04e-09	0	\\
3.14e-09	0	\\
3.24e-09	0	\\
3.34e-09	0	\\
3.45e-09	0	\\
3.55e-09	0	\\
3.65e-09	0	\\
3.75e-09	0	\\
3.86e-09	0	\\
3.96e-09	0	\\
4.06e-09	0	\\
4.16e-09	0	\\
4.27e-09	0	\\
4.37e-09	0	\\
4.47e-09	0	\\
4.57e-09	0	\\
4.68e-09	0	\\
4.78e-09	0	\\
4.89e-09	0	\\
4.99e-09	0	\\
5e-09	0	\\
};
\addplot [color=red,solid,forget plot]
  table[row sep=crcr]{
0	0	\\
1.1e-10	0	\\
2.2e-10	0	\\
3.3e-10	0	\\
4.4e-10	0	\\
5.4e-10	0	\\
6.5e-10	0	\\
7.5e-10	0	\\
8.6e-10	0	\\
9.6e-10	0	\\
1.07e-09	0	\\
1.18e-09	0	\\
1.28e-09	0	\\
1.38e-09	0	\\
1.49e-09	0	\\
1.59e-09	0	\\
1.69e-09	0	\\
1.8e-09	0	\\
1.9e-09	0	\\
2.01e-09	0	\\
2.11e-09	0	\\
2.21e-09	0	\\
2.32e-09	0	\\
2.42e-09	0	\\
2.52e-09	0	\\
2.63e-09	0	\\
2.73e-09	0	\\
2.83e-09	0	\\
2.93e-09	0	\\
3.04e-09	0	\\
3.14e-09	0	\\
3.24e-09	0	\\
3.34e-09	0	\\
3.45e-09	0	\\
3.55e-09	0	\\
3.65e-09	0	\\
3.75e-09	0	\\
3.86e-09	0	\\
3.96e-09	0	\\
4.06e-09	0	\\
4.16e-09	0	\\
4.27e-09	0	\\
4.37e-09	0	\\
4.47e-09	0	\\
4.57e-09	0	\\
4.68e-09	0	\\
4.78e-09	0	\\
4.89e-09	0	\\
4.99e-09	0	\\
5e-09	0	\\
};
\addplot [color=mycolor1,solid,forget plot]
  table[row sep=crcr]{
0	0	\\
1.1e-10	0	\\
2.2e-10	0	\\
3.3e-10	0	\\
4.4e-10	0	\\
5.4e-10	0	\\
6.5e-10	0	\\
7.5e-10	0	\\
8.6e-10	0	\\
9.6e-10	0	\\
1.07e-09	0	\\
1.18e-09	0	\\
1.28e-09	0	\\
1.38e-09	0	\\
1.49e-09	0	\\
1.59e-09	0	\\
1.69e-09	0	\\
1.8e-09	0	\\
1.9e-09	0	\\
2.01e-09	0	\\
2.11e-09	0	\\
2.21e-09	0	\\
2.32e-09	0	\\
2.42e-09	0	\\
2.52e-09	0	\\
2.63e-09	0	\\
2.73e-09	0	\\
2.83e-09	0	\\
2.93e-09	0	\\
3.04e-09	0	\\
3.14e-09	0	\\
3.24e-09	0	\\
3.34e-09	0	\\
3.45e-09	0	\\
3.55e-09	0	\\
3.65e-09	0	\\
3.75e-09	0	\\
3.86e-09	0	\\
3.96e-09	0	\\
4.06e-09	0	\\
4.16e-09	0	\\
4.27e-09	0	\\
4.37e-09	0	\\
4.47e-09	0	\\
4.57e-09	0	\\
4.68e-09	0	\\
4.78e-09	0	\\
4.89e-09	0	\\
4.99e-09	0	\\
5e-09	0	\\
};
\addplot [color=mycolor2,solid,forget plot]
  table[row sep=crcr]{
0	0	\\
1.1e-10	0	\\
2.2e-10	0	\\
3.3e-10	0	\\
4.4e-10	0	\\
5.4e-10	0	\\
6.5e-10	0	\\
7.5e-10	0	\\
8.6e-10	0	\\
9.6e-10	0	\\
1.07e-09	0	\\
1.18e-09	0	\\
1.28e-09	0	\\
1.38e-09	0	\\
1.49e-09	0	\\
1.59e-09	0	\\
1.69e-09	0	\\
1.8e-09	0	\\
1.9e-09	0	\\
2.01e-09	0	\\
2.11e-09	0	\\
2.21e-09	0	\\
2.32e-09	0	\\
2.42e-09	0	\\
2.52e-09	0	\\
2.63e-09	0	\\
2.73e-09	0	\\
2.83e-09	0	\\
2.93e-09	0	\\
3.04e-09	0	\\
3.14e-09	0	\\
3.24e-09	0	\\
3.34e-09	0	\\
3.45e-09	0	\\
3.55e-09	0	\\
3.65e-09	0	\\
3.75e-09	0	\\
3.86e-09	0	\\
3.96e-09	0	\\
4.06e-09	0	\\
4.16e-09	0	\\
4.27e-09	0	\\
4.37e-09	0	\\
4.47e-09	0	\\
4.57e-09	0	\\
4.68e-09	0	\\
4.78e-09	0	\\
4.89e-09	0	\\
4.99e-09	0	\\
5e-09	0	\\
};
\addplot [color=mycolor3,solid,forget plot]
  table[row sep=crcr]{
0	0	\\
1.1e-10	0	\\
2.2e-10	0	\\
3.3e-10	0	\\
4.4e-10	0	\\
5.4e-10	0	\\
6.5e-10	0	\\
7.5e-10	0	\\
8.6e-10	0	\\
9.6e-10	0	\\
1.07e-09	0	\\
1.18e-09	0	\\
1.28e-09	0	\\
1.38e-09	0	\\
1.49e-09	0	\\
1.59e-09	0	\\
1.69e-09	0	\\
1.8e-09	0	\\
1.9e-09	0	\\
2.01e-09	0	\\
2.11e-09	0	\\
2.21e-09	0	\\
2.32e-09	0	\\
2.42e-09	0	\\
2.52e-09	0	\\
2.63e-09	0	\\
2.73e-09	0	\\
2.83e-09	0	\\
2.93e-09	0	\\
3.04e-09	0	\\
3.14e-09	0	\\
3.24e-09	0	\\
3.34e-09	0	\\
3.45e-09	0	\\
3.55e-09	0	\\
3.65e-09	0	\\
3.75e-09	0	\\
3.86e-09	0	\\
3.96e-09	0	\\
4.06e-09	0	\\
4.16e-09	0	\\
4.27e-09	0	\\
4.37e-09	0	\\
4.47e-09	0	\\
4.57e-09	0	\\
4.68e-09	0	\\
4.78e-09	0	\\
4.89e-09	0	\\
4.99e-09	0	\\
5e-09	0	\\
};
\addplot [color=darkgray,solid,forget plot]
  table[row sep=crcr]{
0	0	\\
1.1e-10	0	\\
2.2e-10	0	\\
3.3e-10	0	\\
4.4e-10	0	\\
5.4e-10	0	\\
6.5e-10	0	\\
7.5e-10	0	\\
8.6e-10	0	\\
9.6e-10	0	\\
1.07e-09	0	\\
1.18e-09	0	\\
1.28e-09	0	\\
1.38e-09	0	\\
1.49e-09	0	\\
1.59e-09	0	\\
1.69e-09	0	\\
1.8e-09	0	\\
1.9e-09	0	\\
2.01e-09	0	\\
2.11e-09	0	\\
2.21e-09	0	\\
2.32e-09	0	\\
2.42e-09	0	\\
2.52e-09	0	\\
2.63e-09	0	\\
2.73e-09	0	\\
2.83e-09	0	\\
2.93e-09	0	\\
3.04e-09	0	\\
3.14e-09	0	\\
3.24e-09	0	\\
3.34e-09	0	\\
3.45e-09	0	\\
3.55e-09	0	\\
3.65e-09	0	\\
3.75e-09	0	\\
3.86e-09	0	\\
3.96e-09	0	\\
4.06e-09	0	\\
4.16e-09	0	\\
4.27e-09	0	\\
4.37e-09	0	\\
4.47e-09	0	\\
4.57e-09	0	\\
4.68e-09	0	\\
4.78e-09	0	\\
4.89e-09	0	\\
4.99e-09	0	\\
5e-09	0	\\
};
\addplot [color=blue,solid,forget plot]
  table[row sep=crcr]{
0	0	\\
1.1e-10	0	\\
2.2e-10	0	\\
3.3e-10	0	\\
4.4e-10	0	\\
5.4e-10	0	\\
6.5e-10	0	\\
7.5e-10	0	\\
8.6e-10	0	\\
9.6e-10	0	\\
1.07e-09	0	\\
1.18e-09	0	\\
1.28e-09	0	\\
1.38e-09	0	\\
1.49e-09	0	\\
1.59e-09	0	\\
1.69e-09	0	\\
1.8e-09	0	\\
1.9e-09	0	\\
2.01e-09	0	\\
2.11e-09	0	\\
2.21e-09	0	\\
2.32e-09	0	\\
2.42e-09	0	\\
2.52e-09	0	\\
2.63e-09	0	\\
2.73e-09	0	\\
2.83e-09	0	\\
2.93e-09	0	\\
3.04e-09	0	\\
3.14e-09	0	\\
3.24e-09	0	\\
3.34e-09	0	\\
3.45e-09	0	\\
3.55e-09	0	\\
3.65e-09	0	\\
3.75e-09	0	\\
3.86e-09	0	\\
3.96e-09	0	\\
4.06e-09	0	\\
4.16e-09	0	\\
4.27e-09	0	\\
4.37e-09	0	\\
4.47e-09	0	\\
4.57e-09	0	\\
4.68e-09	0	\\
4.78e-09	0	\\
4.89e-09	0	\\
4.99e-09	0	\\
5e-09	0	\\
};
\addplot [color=black!50!green,solid,forget plot]
  table[row sep=crcr]{
0	0	\\
1.1e-10	0	\\
2.2e-10	0	\\
3.3e-10	0	\\
4.4e-10	0	\\
5.4e-10	0	\\
6.5e-10	0	\\
7.5e-10	0	\\
8.6e-10	0	\\
9.6e-10	0	\\
1.07e-09	0	\\
1.18e-09	0	\\
1.28e-09	0	\\
1.38e-09	0	\\
1.49e-09	0	\\
1.59e-09	0	\\
1.69e-09	0	\\
1.8e-09	0	\\
1.9e-09	0	\\
2.01e-09	0	\\
2.11e-09	0	\\
2.21e-09	0	\\
2.32e-09	0	\\
2.42e-09	0	\\
2.52e-09	0	\\
2.63e-09	0	\\
2.73e-09	0	\\
2.83e-09	0	\\
2.93e-09	0	\\
3.04e-09	0	\\
3.14e-09	0	\\
3.24e-09	0	\\
3.34e-09	0	\\
3.45e-09	0	\\
3.55e-09	0	\\
3.65e-09	0	\\
3.75e-09	0	\\
3.86e-09	0	\\
3.96e-09	0	\\
4.06e-09	0	\\
4.16e-09	0	\\
4.27e-09	0	\\
4.37e-09	0	\\
4.47e-09	0	\\
4.57e-09	0	\\
4.68e-09	0	\\
4.78e-09	0	\\
4.89e-09	0	\\
4.99e-09	0	\\
5e-09	0	\\
};
\addplot [color=red,solid,forget plot]
  table[row sep=crcr]{
0	0	\\
1.1e-10	0	\\
2.2e-10	0	\\
3.3e-10	0	\\
4.4e-10	0	\\
5.4e-10	0	\\
6.5e-10	0	\\
7.5e-10	0	\\
8.6e-10	0	\\
9.6e-10	0	\\
1.07e-09	0	\\
1.18e-09	0	\\
1.28e-09	0	\\
1.38e-09	0	\\
1.49e-09	0	\\
1.59e-09	0	\\
1.69e-09	0	\\
1.8e-09	0	\\
1.9e-09	0	\\
2.01e-09	0	\\
2.11e-09	0	\\
2.21e-09	0	\\
2.32e-09	0	\\
2.42e-09	0	\\
2.52e-09	0	\\
2.63e-09	0	\\
2.73e-09	0	\\
2.83e-09	0	\\
2.93e-09	0	\\
3.04e-09	0	\\
3.14e-09	0	\\
3.24e-09	0	\\
3.34e-09	0	\\
3.45e-09	0	\\
3.55e-09	0	\\
3.65e-09	0	\\
3.75e-09	0	\\
3.86e-09	0	\\
3.96e-09	0	\\
4.06e-09	0	\\
4.16e-09	0	\\
4.27e-09	0	\\
4.37e-09	0	\\
4.47e-09	0	\\
4.57e-09	0	\\
4.68e-09	0	\\
4.78e-09	0	\\
4.89e-09	0	\\
4.99e-09	0	\\
5e-09	0	\\
};
\addplot [color=mycolor1,solid,forget plot]
  table[row sep=crcr]{
0	0	\\
1.1e-10	0	\\
2.2e-10	0	\\
3.3e-10	0	\\
4.4e-10	0	\\
5.4e-10	0	\\
6.5e-10	0	\\
7.5e-10	0	\\
8.6e-10	0	\\
9.6e-10	0	\\
1.07e-09	0	\\
1.18e-09	0	\\
1.28e-09	0	\\
1.38e-09	0	\\
1.49e-09	0	\\
1.59e-09	0	\\
1.69e-09	0	\\
1.8e-09	0	\\
1.9e-09	0	\\
2.01e-09	0	\\
2.11e-09	0	\\
2.21e-09	0	\\
2.32e-09	0	\\
2.42e-09	0	\\
2.52e-09	0	\\
2.63e-09	0	\\
2.73e-09	0	\\
2.83e-09	0	\\
2.93e-09	0	\\
3.04e-09	0	\\
3.14e-09	0	\\
3.24e-09	0	\\
3.34e-09	0	\\
3.45e-09	0	\\
3.55e-09	0	\\
3.65e-09	0	\\
3.75e-09	0	\\
3.86e-09	0	\\
3.96e-09	0	\\
4.06e-09	0	\\
4.16e-09	0	\\
4.27e-09	0	\\
4.37e-09	0	\\
4.47e-09	0	\\
4.57e-09	0	\\
4.68e-09	0	\\
4.78e-09	0	\\
4.89e-09	0	\\
4.99e-09	0	\\
5e-09	0	\\
};
\addplot [color=mycolor2,solid,forget plot]
  table[row sep=crcr]{
0	0	\\
1.1e-10	0	\\
2.2e-10	0	\\
3.3e-10	0	\\
4.4e-10	0	\\
5.4e-10	0	\\
6.5e-10	0	\\
7.5e-10	0	\\
8.6e-10	0	\\
9.6e-10	0	\\
1.07e-09	0	\\
1.18e-09	0	\\
1.28e-09	0	\\
1.38e-09	0	\\
1.49e-09	0	\\
1.59e-09	0	\\
1.69e-09	0	\\
1.8e-09	0	\\
1.9e-09	0	\\
2.01e-09	0	\\
2.11e-09	0	\\
2.21e-09	0	\\
2.32e-09	0	\\
2.42e-09	0	\\
2.52e-09	0	\\
2.63e-09	0	\\
2.73e-09	0	\\
2.83e-09	0	\\
2.93e-09	0	\\
3.04e-09	0	\\
3.14e-09	0	\\
3.24e-09	0	\\
3.34e-09	0	\\
3.45e-09	0	\\
3.55e-09	0	\\
3.65e-09	0	\\
3.75e-09	0	\\
3.86e-09	0	\\
3.96e-09	0	\\
4.06e-09	0	\\
4.16e-09	0	\\
4.27e-09	0	\\
4.37e-09	0	\\
4.47e-09	0	\\
4.57e-09	0	\\
4.68e-09	0	\\
4.78e-09	0	\\
4.89e-09	0	\\
4.99e-09	0	\\
5e-09	0	\\
};
\addplot [color=mycolor3,solid,forget plot]
  table[row sep=crcr]{
0	0	\\
1.1e-10	0	\\
2.2e-10	0	\\
3.3e-10	0	\\
4.4e-10	0	\\
5.4e-10	0	\\
6.5e-10	0	\\
7.5e-10	0	\\
8.6e-10	0	\\
9.6e-10	0	\\
1.07e-09	0	\\
1.18e-09	0	\\
1.28e-09	0	\\
1.38e-09	0	\\
1.49e-09	0	\\
1.59e-09	0	\\
1.69e-09	0	\\
1.8e-09	0	\\
1.9e-09	0	\\
2.01e-09	0	\\
2.11e-09	0	\\
2.21e-09	0	\\
2.32e-09	0	\\
2.42e-09	0	\\
2.52e-09	0	\\
2.63e-09	0	\\
2.73e-09	0	\\
2.83e-09	0	\\
2.93e-09	0	\\
3.04e-09	0	\\
3.14e-09	0	\\
3.24e-09	0	\\
3.34e-09	0	\\
3.45e-09	0	\\
3.55e-09	0	\\
3.65e-09	0	\\
3.75e-09	0	\\
3.86e-09	0	\\
3.96e-09	0	\\
4.06e-09	0	\\
4.16e-09	0	\\
4.27e-09	0	\\
4.37e-09	0	\\
4.47e-09	0	\\
4.57e-09	0	\\
4.68e-09	0	\\
4.78e-09	0	\\
4.89e-09	0	\\
4.99e-09	0	\\
5e-09	0	\\
};
\addplot [color=darkgray,solid,forget plot]
  table[row sep=crcr]{
0	0	\\
1.1e-10	0	\\
2.2e-10	0	\\
3.3e-10	0	\\
4.4e-10	0	\\
5.4e-10	0	\\
6.5e-10	0	\\
7.5e-10	0	\\
8.6e-10	0	\\
9.6e-10	0	\\
1.07e-09	0	\\
1.18e-09	0	\\
1.28e-09	0	\\
1.38e-09	0	\\
1.49e-09	0	\\
1.59e-09	0	\\
1.69e-09	0	\\
1.8e-09	0	\\
1.9e-09	0	\\
2.01e-09	0	\\
2.11e-09	0	\\
2.21e-09	0	\\
2.32e-09	0	\\
2.42e-09	0	\\
2.52e-09	0	\\
2.63e-09	0	\\
2.73e-09	0	\\
2.83e-09	0	\\
2.93e-09	0	\\
3.04e-09	0	\\
3.14e-09	0	\\
3.24e-09	0	\\
3.34e-09	0	\\
3.45e-09	0	\\
3.55e-09	0	\\
3.65e-09	0	\\
3.75e-09	0	\\
3.86e-09	0	\\
3.96e-09	0	\\
4.06e-09	0	\\
4.16e-09	0	\\
4.27e-09	0	\\
4.37e-09	0	\\
4.47e-09	0	\\
4.57e-09	0	\\
4.68e-09	0	\\
4.78e-09	0	\\
4.89e-09	0	\\
4.99e-09	0	\\
5e-09	0	\\
};
\addplot [color=blue,solid,forget plot]
  table[row sep=crcr]{
0	0	\\
1.1e-10	0	\\
2.2e-10	0	\\
3.3e-10	0	\\
4.4e-10	0	\\
5.4e-10	0	\\
6.5e-10	0	\\
7.5e-10	0	\\
8.6e-10	0	\\
9.6e-10	0	\\
1.07e-09	0	\\
1.18e-09	0	\\
1.28e-09	0	\\
1.38e-09	0	\\
1.49e-09	0	\\
1.59e-09	0	\\
1.69e-09	0	\\
1.8e-09	0	\\
1.9e-09	0	\\
2.01e-09	0	\\
2.11e-09	0	\\
2.21e-09	0	\\
2.32e-09	0	\\
2.42e-09	0	\\
2.52e-09	0	\\
2.63e-09	0	\\
2.73e-09	0	\\
2.83e-09	0	\\
2.93e-09	0	\\
3.04e-09	0	\\
3.14e-09	0	\\
3.24e-09	0	\\
3.34e-09	0	\\
3.45e-09	0	\\
3.55e-09	0	\\
3.65e-09	0	\\
3.75e-09	0	\\
3.86e-09	0	\\
3.96e-09	0	\\
4.06e-09	0	\\
4.16e-09	0	\\
4.27e-09	0	\\
4.37e-09	0	\\
4.47e-09	0	\\
4.57e-09	0	\\
4.68e-09	0	\\
4.78e-09	0	\\
4.89e-09	0	\\
4.99e-09	0	\\
5e-09	0	\\
};
\addplot [color=black!50!green,solid,forget plot]
  table[row sep=crcr]{
0	0	\\
1.1e-10	0	\\
2.2e-10	0	\\
3.3e-10	0	\\
4.4e-10	0	\\
5.4e-10	0	\\
6.5e-10	0	\\
7.5e-10	0	\\
8.6e-10	0	\\
9.6e-10	0	\\
1.07e-09	0	\\
1.18e-09	0	\\
1.28e-09	0	\\
1.38e-09	0	\\
1.49e-09	0	\\
1.59e-09	0	\\
1.69e-09	0	\\
1.8e-09	0	\\
1.9e-09	0	\\
2.01e-09	0	\\
2.11e-09	0	\\
2.21e-09	0	\\
2.32e-09	0	\\
2.42e-09	0	\\
2.52e-09	0	\\
2.63e-09	0	\\
2.73e-09	0	\\
2.83e-09	0	\\
2.93e-09	0	\\
3.04e-09	0	\\
3.14e-09	0	\\
3.24e-09	0	\\
3.34e-09	0	\\
3.45e-09	0	\\
3.55e-09	0	\\
3.65e-09	0	\\
3.75e-09	0	\\
3.86e-09	0	\\
3.96e-09	0	\\
4.06e-09	0	\\
4.16e-09	0	\\
4.27e-09	0	\\
4.37e-09	0	\\
4.47e-09	0	\\
4.57e-09	0	\\
4.68e-09	0	\\
4.78e-09	0	\\
4.89e-09	0	\\
4.99e-09	0	\\
5e-09	0	\\
};
\addplot [color=red,solid,forget plot]
  table[row sep=crcr]{
0	0	\\
1.1e-10	0	\\
2.2e-10	0	\\
3.3e-10	0	\\
4.4e-10	0	\\
5.4e-10	0	\\
6.5e-10	0	\\
7.5e-10	0	\\
8.6e-10	0	\\
9.6e-10	0	\\
1.07e-09	0	\\
1.18e-09	0	\\
1.28e-09	0	\\
1.38e-09	0	\\
1.49e-09	0	\\
1.59e-09	0	\\
1.69e-09	0	\\
1.8e-09	0	\\
1.9e-09	0	\\
2.01e-09	0	\\
2.11e-09	0	\\
2.21e-09	0	\\
2.32e-09	0	\\
2.42e-09	0	\\
2.52e-09	0	\\
2.63e-09	0	\\
2.73e-09	0	\\
2.83e-09	0	\\
2.93e-09	0	\\
3.04e-09	0	\\
3.14e-09	0	\\
3.24e-09	0	\\
3.34e-09	0	\\
3.45e-09	0	\\
3.55e-09	0	\\
3.65e-09	0	\\
3.75e-09	0	\\
3.86e-09	0	\\
3.96e-09	0	\\
4.06e-09	0	\\
4.16e-09	0	\\
4.27e-09	0	\\
4.37e-09	0	\\
4.47e-09	0	\\
4.57e-09	0	\\
4.68e-09	0	\\
4.78e-09	0	\\
4.89e-09	0	\\
4.99e-09	0	\\
5e-09	0	\\
};
\addplot [color=mycolor1,solid,forget plot]
  table[row sep=crcr]{
0	0	\\
1.1e-10	0	\\
2.2e-10	0	\\
3.3e-10	0	\\
4.4e-10	0	\\
5.4e-10	0	\\
6.5e-10	0	\\
7.5e-10	0	\\
8.6e-10	0	\\
9.6e-10	0	\\
1.07e-09	0	\\
1.18e-09	0	\\
1.28e-09	0	\\
1.38e-09	0	\\
1.49e-09	0	\\
1.59e-09	0	\\
1.69e-09	0	\\
1.8e-09	0	\\
1.9e-09	0	\\
2.01e-09	0	\\
2.11e-09	0	\\
2.21e-09	0	\\
2.32e-09	0	\\
2.42e-09	0	\\
2.52e-09	0	\\
2.63e-09	0	\\
2.73e-09	0	\\
2.83e-09	0	\\
2.93e-09	0	\\
3.04e-09	0	\\
3.14e-09	0	\\
3.24e-09	0	\\
3.34e-09	0	\\
3.45e-09	0	\\
3.55e-09	0	\\
3.65e-09	0	\\
3.75e-09	0	\\
3.86e-09	0	\\
3.96e-09	0	\\
4.06e-09	0	\\
4.16e-09	0	\\
4.27e-09	0	\\
4.37e-09	0	\\
4.47e-09	0	\\
4.57e-09	0	\\
4.68e-09	0	\\
4.78e-09	0	\\
4.89e-09	0	\\
4.99e-09	0	\\
5e-09	0	\\
};
\addplot [color=mycolor2,solid,forget plot]
  table[row sep=crcr]{
0	0	\\
1.1e-10	0	\\
2.2e-10	0	\\
3.3e-10	0	\\
4.4e-10	0	\\
5.4e-10	0	\\
6.5e-10	0	\\
7.5e-10	0	\\
8.6e-10	0	\\
9.6e-10	0	\\
1.07e-09	0	\\
1.18e-09	0	\\
1.28e-09	0	\\
1.38e-09	0	\\
1.49e-09	0	\\
1.59e-09	0	\\
1.69e-09	0	\\
1.8e-09	0	\\
1.9e-09	0	\\
2.01e-09	0	\\
2.11e-09	0	\\
2.21e-09	0	\\
2.32e-09	0	\\
2.42e-09	0	\\
2.52e-09	0	\\
2.63e-09	0	\\
2.73e-09	0	\\
2.83e-09	0	\\
2.93e-09	0	\\
3.04e-09	0	\\
3.14e-09	0	\\
3.24e-09	0	\\
3.34e-09	0	\\
3.45e-09	0	\\
3.55e-09	0	\\
3.65e-09	0	\\
3.75e-09	0	\\
3.86e-09	0	\\
3.96e-09	0	\\
4.06e-09	0	\\
4.16e-09	0	\\
4.27e-09	0	\\
4.37e-09	0	\\
4.47e-09	0	\\
4.57e-09	0	\\
4.68e-09	0	\\
4.78e-09	0	\\
4.89e-09	0	\\
4.99e-09	0	\\
5e-09	0	\\
};
\addplot [color=mycolor3,solid,forget plot]
  table[row sep=crcr]{
0	0	\\
1.1e-10	0	\\
2.2e-10	0	\\
3.3e-10	0	\\
4.4e-10	0	\\
5.4e-10	0	\\
6.5e-10	0	\\
7.5e-10	0	\\
8.6e-10	0	\\
9.6e-10	0	\\
1.07e-09	0	\\
1.18e-09	0	\\
1.28e-09	0	\\
1.38e-09	0	\\
1.49e-09	0	\\
1.59e-09	0	\\
1.69e-09	0	\\
1.8e-09	0	\\
1.9e-09	0	\\
2.01e-09	0	\\
2.11e-09	0	\\
2.21e-09	0	\\
2.32e-09	0	\\
2.42e-09	0	\\
2.52e-09	0	\\
2.63e-09	0	\\
2.73e-09	0	\\
2.83e-09	0	\\
2.93e-09	0	\\
3.04e-09	0	\\
3.14e-09	0	\\
3.24e-09	0	\\
3.34e-09	0	\\
3.45e-09	0	\\
3.55e-09	0	\\
3.65e-09	0	\\
3.75e-09	0	\\
3.86e-09	0	\\
3.96e-09	0	\\
4.06e-09	0	\\
4.16e-09	0	\\
4.27e-09	0	\\
4.37e-09	0	\\
4.47e-09	0	\\
4.57e-09	0	\\
4.68e-09	0	\\
4.78e-09	0	\\
4.89e-09	0	\\
4.99e-09	0	\\
5e-09	0	\\
};
\addplot [color=darkgray,solid,forget plot]
  table[row sep=crcr]{
0	0	\\
1.1e-10	0	\\
2.2e-10	0	\\
3.3e-10	0	\\
4.4e-10	0	\\
5.4e-10	0	\\
6.5e-10	0	\\
7.5e-10	0	\\
8.6e-10	0	\\
9.6e-10	0	\\
1.07e-09	0	\\
1.18e-09	0	\\
1.28e-09	0	\\
1.38e-09	0	\\
1.49e-09	0	\\
1.59e-09	0	\\
1.69e-09	0	\\
1.8e-09	0	\\
1.9e-09	0	\\
2.01e-09	0	\\
2.11e-09	0	\\
2.21e-09	0	\\
2.32e-09	0	\\
2.42e-09	0	\\
2.52e-09	0	\\
2.63e-09	0	\\
2.73e-09	0	\\
2.83e-09	0	\\
2.93e-09	0	\\
3.04e-09	0	\\
3.14e-09	0	\\
3.24e-09	0	\\
3.34e-09	0	\\
3.45e-09	0	\\
3.55e-09	0	\\
3.65e-09	0	\\
3.75e-09	0	\\
3.86e-09	0	\\
3.96e-09	0	\\
4.06e-09	0	\\
4.16e-09	0	\\
4.27e-09	0	\\
4.37e-09	0	\\
4.47e-09	0	\\
4.57e-09	0	\\
4.68e-09	0	\\
4.78e-09	0	\\
4.89e-09	0	\\
4.99e-09	0	\\
5e-09	0	\\
};
\addplot [color=blue,solid,forget plot]
  table[row sep=crcr]{
0	0	\\
1.1e-10	0	\\
2.2e-10	0	\\
3.3e-10	0	\\
4.4e-10	0	\\
5.4e-10	0	\\
6.5e-10	0	\\
7.5e-10	0	\\
8.6e-10	0	\\
9.6e-10	0	\\
1.07e-09	0	\\
1.18e-09	0	\\
1.28e-09	0	\\
1.38e-09	0	\\
1.49e-09	0	\\
1.59e-09	0	\\
1.69e-09	0	\\
1.8e-09	0	\\
1.9e-09	0	\\
2.01e-09	0	\\
2.11e-09	0	\\
2.21e-09	0	\\
2.32e-09	0	\\
2.42e-09	0	\\
2.52e-09	0	\\
2.63e-09	0	\\
2.73e-09	0	\\
2.83e-09	0	\\
2.93e-09	0	\\
3.04e-09	0	\\
3.14e-09	0	\\
3.24e-09	0	\\
3.34e-09	0	\\
3.45e-09	0	\\
3.55e-09	0	\\
3.65e-09	0	\\
3.75e-09	0	\\
3.86e-09	0	\\
3.96e-09	0	\\
4.06e-09	0	\\
4.16e-09	0	\\
4.27e-09	0	\\
4.37e-09	0	\\
4.47e-09	0	\\
4.57e-09	0	\\
4.68e-09	0	\\
4.78e-09	0	\\
4.89e-09	0	\\
4.99e-09	0	\\
5e-09	0	\\
};
\addplot [color=black!50!green,solid,forget plot]
  table[row sep=crcr]{
0	0	\\
1.1e-10	0	\\
2.2e-10	0	\\
3.3e-10	0	\\
4.4e-10	0	\\
5.4e-10	0	\\
6.5e-10	0	\\
7.5e-10	0	\\
8.6e-10	0	\\
9.6e-10	0	\\
1.07e-09	0	\\
1.18e-09	0	\\
1.28e-09	0	\\
1.38e-09	0	\\
1.49e-09	0	\\
1.59e-09	0	\\
1.69e-09	0	\\
1.8e-09	0	\\
1.9e-09	0	\\
2.01e-09	0	\\
2.11e-09	0	\\
2.21e-09	0	\\
2.32e-09	0	\\
2.42e-09	0	\\
2.52e-09	0	\\
2.63e-09	0	\\
2.73e-09	0	\\
2.83e-09	0	\\
2.93e-09	0	\\
3.04e-09	0	\\
3.14e-09	0	\\
3.24e-09	0	\\
3.34e-09	0	\\
3.45e-09	0	\\
3.55e-09	0	\\
3.65e-09	0	\\
3.75e-09	0	\\
3.86e-09	0	\\
3.96e-09	0	\\
4.06e-09	0	\\
4.16e-09	0	\\
4.27e-09	0	\\
4.37e-09	0	\\
4.47e-09	0	\\
4.57e-09	0	\\
4.68e-09	0	\\
4.78e-09	0	\\
4.89e-09	0	\\
4.99e-09	0	\\
5e-09	0	\\
};
\addplot [color=red,solid,forget plot]
  table[row sep=crcr]{
0	0	\\
1.1e-10	0	\\
2.2e-10	0	\\
3.3e-10	0	\\
4.4e-10	0	\\
5.4e-10	0	\\
6.5e-10	0	\\
7.5e-10	0	\\
8.6e-10	0	\\
9.6e-10	0	\\
1.07e-09	0	\\
1.18e-09	0	\\
1.28e-09	0	\\
1.38e-09	0	\\
1.49e-09	0	\\
1.59e-09	0	\\
1.69e-09	0	\\
1.8e-09	0	\\
1.9e-09	0	\\
2.01e-09	0	\\
2.11e-09	0	\\
2.21e-09	0	\\
2.32e-09	0	\\
2.42e-09	0	\\
2.52e-09	0	\\
2.63e-09	0	\\
2.73e-09	0	\\
2.83e-09	0	\\
2.93e-09	0	\\
3.04e-09	0	\\
3.14e-09	0	\\
3.24e-09	0	\\
3.34e-09	0	\\
3.45e-09	0	\\
3.55e-09	0	\\
3.65e-09	0	\\
3.75e-09	0	\\
3.86e-09	0	\\
3.96e-09	0	\\
4.06e-09	0	\\
4.16e-09	0	\\
4.27e-09	0	\\
4.37e-09	0	\\
4.47e-09	0	\\
4.57e-09	0	\\
4.68e-09	0	\\
4.78e-09	0	\\
4.89e-09	0	\\
4.99e-09	0	\\
5e-09	0	\\
};
\addplot [color=mycolor1,solid,forget plot]
  table[row sep=crcr]{
0	0	\\
1.1e-10	0	\\
2.2e-10	0	\\
3.3e-10	0	\\
4.4e-10	0	\\
5.4e-10	0	\\
6.5e-10	0	\\
7.5e-10	0	\\
8.6e-10	0	\\
9.6e-10	0	\\
1.07e-09	0	\\
1.18e-09	0	\\
1.28e-09	0	\\
1.38e-09	0	\\
1.49e-09	0	\\
1.59e-09	0	\\
1.69e-09	0	\\
1.8e-09	0	\\
1.9e-09	0	\\
2.01e-09	0	\\
2.11e-09	0	\\
2.21e-09	0	\\
2.32e-09	0	\\
2.42e-09	0	\\
2.52e-09	0	\\
2.63e-09	0	\\
2.73e-09	0	\\
2.83e-09	0	\\
2.93e-09	0	\\
3.04e-09	0	\\
3.14e-09	0	\\
3.24e-09	0	\\
3.34e-09	0	\\
3.45e-09	0	\\
3.55e-09	0	\\
3.65e-09	0	\\
3.75e-09	0	\\
3.86e-09	0	\\
3.96e-09	0	\\
4.06e-09	0	\\
4.16e-09	0	\\
4.27e-09	0	\\
4.37e-09	0	\\
4.47e-09	0	\\
4.57e-09	0	\\
4.68e-09	0	\\
4.78e-09	0	\\
4.89e-09	0	\\
4.99e-09	0	\\
5e-09	0	\\
};
\addplot [color=mycolor2,solid,forget plot]
  table[row sep=crcr]{
0	0	\\
1.1e-10	0	\\
2.2e-10	0	\\
3.3e-10	0	\\
4.4e-10	0	\\
5.4e-10	0	\\
6.5e-10	0	\\
7.5e-10	0	\\
8.6e-10	0	\\
9.6e-10	0	\\
1.07e-09	0	\\
1.18e-09	0	\\
1.28e-09	0	\\
1.38e-09	0	\\
1.49e-09	0	\\
1.59e-09	0	\\
1.69e-09	0	\\
1.8e-09	0	\\
1.9e-09	0	\\
2.01e-09	0	\\
2.11e-09	0	\\
2.21e-09	0	\\
2.32e-09	0	\\
2.42e-09	0	\\
2.52e-09	0	\\
2.63e-09	0	\\
2.73e-09	0	\\
2.83e-09	0	\\
2.93e-09	0	\\
3.04e-09	0	\\
3.14e-09	0	\\
3.24e-09	0	\\
3.34e-09	0	\\
3.45e-09	0	\\
3.55e-09	0	\\
3.65e-09	0	\\
3.75e-09	0	\\
3.86e-09	0	\\
3.96e-09	0	\\
4.06e-09	0	\\
4.16e-09	0	\\
4.27e-09	0	\\
4.37e-09	0	\\
4.47e-09	0	\\
4.57e-09	0	\\
4.68e-09	0	\\
4.78e-09	0	\\
4.89e-09	0	\\
4.99e-09	0	\\
5e-09	0	\\
};
\addplot [color=mycolor3,solid,forget plot]
  table[row sep=crcr]{
0	0	\\
1.1e-10	0	\\
2.2e-10	0	\\
3.3e-10	0	\\
4.4e-10	0	\\
5.4e-10	0	\\
6.5e-10	0	\\
7.5e-10	0	\\
8.6e-10	0	\\
9.6e-10	0	\\
1.07e-09	0	\\
1.18e-09	0	\\
1.28e-09	0	\\
1.38e-09	0	\\
1.49e-09	0	\\
1.59e-09	0	\\
1.69e-09	0	\\
1.8e-09	0	\\
1.9e-09	0	\\
2.01e-09	0	\\
2.11e-09	0	\\
2.21e-09	0	\\
2.32e-09	0	\\
2.42e-09	0	\\
2.52e-09	0	\\
2.63e-09	0	\\
2.73e-09	0	\\
2.83e-09	0	\\
2.93e-09	0	\\
3.04e-09	0	\\
3.14e-09	0	\\
3.24e-09	0	\\
3.34e-09	0	\\
3.45e-09	0	\\
3.55e-09	0	\\
3.65e-09	0	\\
3.75e-09	0	\\
3.86e-09	0	\\
3.96e-09	0	\\
4.06e-09	0	\\
4.16e-09	0	\\
4.27e-09	0	\\
4.37e-09	0	\\
4.47e-09	0	\\
4.57e-09	0	\\
4.68e-09	0	\\
4.78e-09	0	\\
4.89e-09	0	\\
4.99e-09	0	\\
5e-09	0	\\
};
\addplot [color=darkgray,solid,forget plot]
  table[row sep=crcr]{
0	0	\\
1.1e-10	0	\\
2.2e-10	0	\\
3.3e-10	0	\\
4.4e-10	0	\\
5.4e-10	0	\\
6.5e-10	0	\\
7.5e-10	0	\\
8.6e-10	0	\\
9.6e-10	0	\\
1.07e-09	0	\\
1.18e-09	0	\\
1.28e-09	0	\\
1.38e-09	0	\\
1.49e-09	0	\\
1.59e-09	0	\\
1.69e-09	0	\\
1.8e-09	0	\\
1.9e-09	0	\\
2.01e-09	0	\\
2.11e-09	0	\\
2.21e-09	0	\\
2.32e-09	0	\\
2.42e-09	0	\\
2.52e-09	0	\\
2.63e-09	0	\\
2.73e-09	0	\\
2.83e-09	0	\\
2.93e-09	0	\\
3.04e-09	0	\\
3.14e-09	0	\\
3.24e-09	0	\\
3.34e-09	0	\\
3.45e-09	0	\\
3.55e-09	0	\\
3.65e-09	0	\\
3.75e-09	0	\\
3.86e-09	0	\\
3.96e-09	0	\\
4.06e-09	0	\\
4.16e-09	0	\\
4.27e-09	0	\\
4.37e-09	0	\\
4.47e-09	0	\\
4.57e-09	0	\\
4.68e-09	0	\\
4.78e-09	0	\\
4.89e-09	0	\\
4.99e-09	0	\\
5e-09	0	\\
};
\addplot [color=blue,solid,forget plot]
  table[row sep=crcr]{
0	0	\\
1.1e-10	0	\\
2.2e-10	0	\\
3.3e-10	0	\\
4.4e-10	0	\\
5.4e-10	0	\\
6.5e-10	0	\\
7.5e-10	0	\\
8.6e-10	0	\\
9.6e-10	0	\\
1.07e-09	0	\\
1.18e-09	0	\\
1.28e-09	0	\\
1.38e-09	0	\\
1.49e-09	0	\\
1.59e-09	0	\\
1.69e-09	0	\\
1.8e-09	0	\\
1.9e-09	0	\\
2.01e-09	0	\\
2.11e-09	0	\\
2.21e-09	0	\\
2.32e-09	0	\\
2.42e-09	0	\\
2.52e-09	0	\\
2.63e-09	0	\\
2.73e-09	0	\\
2.83e-09	0	\\
2.93e-09	0	\\
3.04e-09	0	\\
3.14e-09	0	\\
3.24e-09	0	\\
3.34e-09	0	\\
3.45e-09	0	\\
3.55e-09	0	\\
3.65e-09	0	\\
3.75e-09	0	\\
3.86e-09	0	\\
3.96e-09	0	\\
4.06e-09	0	\\
4.16e-09	0	\\
4.27e-09	0	\\
4.37e-09	0	\\
4.47e-09	0	\\
4.57e-09	0	\\
4.68e-09	0	\\
4.78e-09	0	\\
4.89e-09	0	\\
4.99e-09	0	\\
5e-09	0	\\
};
\addplot [color=black!50!green,solid,forget plot]
  table[row sep=crcr]{
0	0	\\
1.1e-10	0	\\
2.2e-10	0	\\
3.3e-10	0	\\
4.4e-10	0	\\
5.4e-10	0	\\
6.5e-10	0	\\
7.5e-10	0	\\
8.6e-10	0	\\
9.6e-10	0	\\
1.07e-09	0	\\
1.18e-09	0	\\
1.28e-09	0	\\
1.38e-09	0	\\
1.49e-09	0	\\
1.59e-09	0	\\
1.69e-09	0	\\
1.8e-09	0	\\
1.9e-09	0	\\
2.01e-09	0	\\
2.11e-09	0	\\
2.21e-09	0	\\
2.32e-09	0	\\
2.42e-09	0	\\
2.52e-09	0	\\
2.63e-09	0	\\
2.73e-09	0	\\
2.83e-09	0	\\
2.93e-09	0	\\
3.04e-09	0	\\
3.14e-09	0	\\
3.24e-09	0	\\
3.34e-09	0	\\
3.45e-09	0	\\
3.55e-09	0	\\
3.65e-09	0	\\
3.75e-09	0	\\
3.86e-09	0	\\
3.96e-09	0	\\
4.06e-09	0	\\
4.16e-09	0	\\
4.27e-09	0	\\
4.37e-09	0	\\
4.47e-09	0	\\
4.57e-09	0	\\
4.68e-09	0	\\
4.78e-09	0	\\
4.89e-09	0	\\
4.99e-09	0	\\
5e-09	0	\\
};
\addplot [color=red,solid,forget plot]
  table[row sep=crcr]{
0	0	\\
1.1e-10	0	\\
2.2e-10	0	\\
3.3e-10	0	\\
4.4e-10	0	\\
5.4e-10	0	\\
6.5e-10	0	\\
7.5e-10	0	\\
8.6e-10	0	\\
9.6e-10	0	\\
1.07e-09	0	\\
1.18e-09	0	\\
1.28e-09	0	\\
1.38e-09	0	\\
1.49e-09	0	\\
1.59e-09	0	\\
1.69e-09	0	\\
1.8e-09	0	\\
1.9e-09	0	\\
2.01e-09	0	\\
2.11e-09	0	\\
2.21e-09	0	\\
2.32e-09	0	\\
2.42e-09	0	\\
2.52e-09	0	\\
2.63e-09	0	\\
2.73e-09	0	\\
2.83e-09	0	\\
2.93e-09	0	\\
3.04e-09	0	\\
3.14e-09	0	\\
3.24e-09	0	\\
3.34e-09	0	\\
3.45e-09	0	\\
3.55e-09	0	\\
3.65e-09	0	\\
3.75e-09	0	\\
3.86e-09	0	\\
3.96e-09	0	\\
4.06e-09	0	\\
4.16e-09	0	\\
4.27e-09	0	\\
4.37e-09	0	\\
4.47e-09	0	\\
4.57e-09	0	\\
4.68e-09	0	\\
4.78e-09	0	\\
4.89e-09	0	\\
4.99e-09	0	\\
5e-09	0	\\
};
\addplot [color=mycolor1,solid,forget plot]
  table[row sep=crcr]{
0	0	\\
1.1e-10	0	\\
2.2e-10	0	\\
3.3e-10	0	\\
4.4e-10	0	\\
5.4e-10	0	\\
6.5e-10	0	\\
7.5e-10	0	\\
8.6e-10	0	\\
9.6e-10	0	\\
1.07e-09	0	\\
1.18e-09	0	\\
1.28e-09	0	\\
1.38e-09	0	\\
1.49e-09	0	\\
1.59e-09	0	\\
1.69e-09	0	\\
1.8e-09	0	\\
1.9e-09	0	\\
2.01e-09	0	\\
2.11e-09	0	\\
2.21e-09	0	\\
2.32e-09	0	\\
2.42e-09	0	\\
2.52e-09	0	\\
2.63e-09	0	\\
2.73e-09	0	\\
2.83e-09	0	\\
2.93e-09	0	\\
3.04e-09	0	\\
3.14e-09	0	\\
3.24e-09	0	\\
3.34e-09	0	\\
3.45e-09	0	\\
3.55e-09	0	\\
3.65e-09	0	\\
3.75e-09	0	\\
3.86e-09	0	\\
3.96e-09	0	\\
4.06e-09	0	\\
4.16e-09	0	\\
4.27e-09	0	\\
4.37e-09	0	\\
4.47e-09	0	\\
4.57e-09	0	\\
4.68e-09	0	\\
4.78e-09	0	\\
4.89e-09	0	\\
4.99e-09	0	\\
5e-09	0	\\
};
\addplot [color=mycolor2,solid,forget plot]
  table[row sep=crcr]{
0	0	\\
1.1e-10	0	\\
2.2e-10	0	\\
3.3e-10	0	\\
4.4e-10	0	\\
5.4e-10	0	\\
6.5e-10	0	\\
7.5e-10	0	\\
8.6e-10	0	\\
9.6e-10	0	\\
1.07e-09	0	\\
1.18e-09	0	\\
1.28e-09	0	\\
1.38e-09	0	\\
1.49e-09	0	\\
1.59e-09	0	\\
1.69e-09	0	\\
1.8e-09	0	\\
1.9e-09	0	\\
2.01e-09	0	\\
2.11e-09	0	\\
2.21e-09	0	\\
2.32e-09	0	\\
2.42e-09	0	\\
2.52e-09	0	\\
2.63e-09	0	\\
2.73e-09	0	\\
2.83e-09	0	\\
2.93e-09	0	\\
3.04e-09	0	\\
3.14e-09	0	\\
3.24e-09	0	\\
3.34e-09	0	\\
3.45e-09	0	\\
3.55e-09	0	\\
3.65e-09	0	\\
3.75e-09	0	\\
3.86e-09	0	\\
3.96e-09	0	\\
4.06e-09	0	\\
4.16e-09	0	\\
4.27e-09	0	\\
4.37e-09	0	\\
4.47e-09	0	\\
4.57e-09	0	\\
4.68e-09	0	\\
4.78e-09	0	\\
4.89e-09	0	\\
4.99e-09	0	\\
5e-09	0	\\
};
\addplot [color=mycolor3,solid,forget plot]
  table[row sep=crcr]{
0	0	\\
1.1e-10	0	\\
2.2e-10	0	\\
3.3e-10	0	\\
4.4e-10	0	\\
5.4e-10	0	\\
6.5e-10	0	\\
7.5e-10	0	\\
8.6e-10	0	\\
9.6e-10	0	\\
1.07e-09	0	\\
1.18e-09	0	\\
1.28e-09	0	\\
1.38e-09	0	\\
1.49e-09	0	\\
1.59e-09	0	\\
1.69e-09	0	\\
1.8e-09	0	\\
1.9e-09	0	\\
2.01e-09	0	\\
2.11e-09	0	\\
2.21e-09	0	\\
2.32e-09	0	\\
2.42e-09	0	\\
2.52e-09	0	\\
2.63e-09	0	\\
2.73e-09	0	\\
2.83e-09	0	\\
2.93e-09	0	\\
3.04e-09	0	\\
3.14e-09	0	\\
3.24e-09	0	\\
3.34e-09	0	\\
3.45e-09	0	\\
3.55e-09	0	\\
3.65e-09	0	\\
3.75e-09	0	\\
3.86e-09	0	\\
3.96e-09	0	\\
4.06e-09	0	\\
4.16e-09	0	\\
4.27e-09	0	\\
4.37e-09	0	\\
4.47e-09	0	\\
4.57e-09	0	\\
4.68e-09	0	\\
4.78e-09	0	\\
4.89e-09	0	\\
4.99e-09	0	\\
5e-09	0	\\
};
\addplot [color=darkgray,solid,forget plot]
  table[row sep=crcr]{
0	0	\\
1.1e-10	0	\\
2.2e-10	0	\\
3.3e-10	0	\\
4.4e-10	0	\\
5.4e-10	0	\\
6.5e-10	0	\\
7.5e-10	0	\\
8.6e-10	0	\\
9.6e-10	0	\\
1.07e-09	0	\\
1.18e-09	0	\\
1.28e-09	0	\\
1.38e-09	0	\\
1.49e-09	0	\\
1.59e-09	0	\\
1.69e-09	0	\\
1.8e-09	0	\\
1.9e-09	0	\\
2.01e-09	0	\\
2.11e-09	0	\\
2.21e-09	0	\\
2.32e-09	0	\\
2.42e-09	0	\\
2.52e-09	0	\\
2.63e-09	0	\\
2.73e-09	0	\\
2.83e-09	0	\\
2.93e-09	0	\\
3.04e-09	0	\\
3.14e-09	0	\\
3.24e-09	0	\\
3.34e-09	0	\\
3.45e-09	0	\\
3.55e-09	0	\\
3.65e-09	0	\\
3.75e-09	0	\\
3.86e-09	0	\\
3.96e-09	0	\\
4.06e-09	0	\\
4.16e-09	0	\\
4.27e-09	0	\\
4.37e-09	0	\\
4.47e-09	0	\\
4.57e-09	0	\\
4.68e-09	0	\\
4.78e-09	0	\\
4.89e-09	0	\\
4.99e-09	0	\\
5e-09	0	\\
};
\addplot [color=blue,solid,forget plot]
  table[row sep=crcr]{
0	0	\\
1.1e-10	0	\\
2.2e-10	0	\\
3.3e-10	0	\\
4.4e-10	0	\\
5.4e-10	0	\\
6.5e-10	0	\\
7.5e-10	0	\\
8.6e-10	0	\\
9.6e-10	0	\\
1.07e-09	0	\\
1.18e-09	0	\\
1.28e-09	0	\\
1.38e-09	0	\\
1.49e-09	0	\\
1.59e-09	0	\\
1.69e-09	0	\\
1.8e-09	0	\\
1.9e-09	0	\\
2.01e-09	0	\\
2.11e-09	0	\\
2.21e-09	0	\\
2.32e-09	0	\\
2.42e-09	0	\\
2.52e-09	0	\\
2.63e-09	0	\\
2.73e-09	0	\\
2.83e-09	0	\\
2.93e-09	0	\\
3.04e-09	0	\\
3.14e-09	0	\\
3.24e-09	0	\\
3.34e-09	0	\\
3.45e-09	0	\\
3.55e-09	0	\\
3.65e-09	0	\\
3.75e-09	0	\\
3.86e-09	0	\\
3.96e-09	0	\\
4.06e-09	0	\\
4.16e-09	0	\\
4.27e-09	0	\\
4.37e-09	0	\\
4.47e-09	0	\\
4.57e-09	0	\\
4.68e-09	0	\\
4.78e-09	0	\\
4.89e-09	0	\\
4.99e-09	0	\\
5e-09	0	\\
};
\addplot [color=black!50!green,solid,forget plot]
  table[row sep=crcr]{
0	0	\\
1.1e-10	0	\\
2.2e-10	0	\\
3.3e-10	0	\\
4.4e-10	0	\\
5.4e-10	0	\\
6.5e-10	0	\\
7.5e-10	0	\\
8.6e-10	0	\\
9.6e-10	0	\\
1.07e-09	0	\\
1.18e-09	0	\\
1.28e-09	0	\\
1.38e-09	0	\\
1.49e-09	0	\\
1.59e-09	0	\\
1.69e-09	0	\\
1.8e-09	0	\\
1.9e-09	0	\\
2.01e-09	0	\\
2.11e-09	0	\\
2.21e-09	0	\\
2.32e-09	0	\\
2.42e-09	0	\\
2.52e-09	0	\\
2.63e-09	0	\\
2.73e-09	0	\\
2.83e-09	0	\\
2.93e-09	0	\\
3.04e-09	0	\\
3.14e-09	0	\\
3.24e-09	0	\\
3.34e-09	0	\\
3.45e-09	0	\\
3.55e-09	0	\\
3.65e-09	0	\\
3.75e-09	0	\\
3.86e-09	0	\\
3.96e-09	0	\\
4.06e-09	0	\\
4.16e-09	0	\\
4.27e-09	0	\\
4.37e-09	0	\\
4.47e-09	0	\\
4.57e-09	0	\\
4.68e-09	0	\\
4.78e-09	0	\\
4.89e-09	0	\\
4.99e-09	0	\\
5e-09	0	\\
};
\addplot [color=red,solid,forget plot]
  table[row sep=crcr]{
0	0	\\
1.1e-10	0	\\
2.2e-10	0	\\
3.3e-10	0	\\
4.4e-10	0	\\
5.4e-10	0	\\
6.5e-10	0	\\
7.5e-10	0	\\
8.6e-10	0	\\
9.6e-10	0	\\
1.07e-09	0	\\
1.18e-09	0	\\
1.28e-09	0	\\
1.38e-09	0	\\
1.49e-09	0	\\
1.59e-09	0	\\
1.69e-09	0	\\
1.8e-09	0	\\
1.9e-09	0	\\
2.01e-09	0	\\
2.11e-09	0	\\
2.21e-09	0	\\
2.32e-09	0	\\
2.42e-09	0	\\
2.52e-09	0	\\
2.63e-09	0	\\
2.73e-09	0	\\
2.83e-09	0	\\
2.93e-09	0	\\
3.04e-09	0	\\
3.14e-09	0	\\
3.24e-09	0	\\
3.34e-09	0	\\
3.45e-09	0	\\
3.55e-09	0	\\
3.65e-09	0	\\
3.75e-09	0	\\
3.86e-09	0	\\
3.96e-09	0	\\
4.06e-09	0	\\
4.16e-09	0	\\
4.27e-09	0	\\
4.37e-09	0	\\
4.47e-09	0	\\
4.57e-09	0	\\
4.68e-09	0	\\
4.78e-09	0	\\
4.89e-09	0	\\
4.99e-09	0	\\
5e-09	0	\\
};
\addplot [color=mycolor1,solid,forget plot]
  table[row sep=crcr]{
0	0	\\
1.1e-10	0	\\
2.2e-10	0	\\
3.3e-10	0	\\
4.4e-10	0	\\
5.4e-10	0	\\
6.5e-10	0	\\
7.5e-10	0	\\
8.6e-10	0	\\
9.6e-10	0	\\
1.07e-09	0	\\
1.18e-09	0	\\
1.28e-09	0	\\
1.38e-09	0	\\
1.49e-09	0	\\
1.59e-09	0	\\
1.69e-09	0	\\
1.8e-09	0	\\
1.9e-09	0	\\
2.01e-09	0	\\
2.11e-09	0	\\
2.21e-09	0	\\
2.32e-09	0	\\
2.42e-09	0	\\
2.52e-09	0	\\
2.63e-09	0	\\
2.73e-09	0	\\
2.83e-09	0	\\
2.93e-09	0	\\
3.04e-09	0	\\
3.14e-09	0	\\
3.24e-09	0	\\
3.34e-09	0	\\
3.45e-09	0	\\
3.55e-09	0	\\
3.65e-09	0	\\
3.75e-09	0	\\
3.86e-09	0	\\
3.96e-09	0	\\
4.06e-09	0	\\
4.16e-09	0	\\
4.27e-09	0	\\
4.37e-09	0	\\
4.47e-09	0	\\
4.57e-09	0	\\
4.68e-09	0	\\
4.78e-09	0	\\
4.89e-09	0	\\
4.99e-09	0	\\
5e-09	0	\\
};
\addplot [color=mycolor2,solid,forget plot]
  table[row sep=crcr]{
0	0	\\
1.1e-10	0	\\
2.2e-10	0	\\
3.3e-10	0	\\
4.4e-10	0	\\
5.4e-10	0	\\
6.5e-10	0	\\
7.5e-10	0	\\
8.6e-10	0	\\
9.6e-10	0	\\
1.07e-09	0	\\
1.18e-09	0	\\
1.28e-09	0	\\
1.38e-09	0	\\
1.49e-09	0	\\
1.59e-09	0	\\
1.69e-09	0	\\
1.8e-09	0	\\
1.9e-09	0	\\
2.01e-09	0	\\
2.11e-09	0	\\
2.21e-09	0	\\
2.32e-09	0	\\
2.42e-09	0	\\
2.52e-09	0	\\
2.63e-09	0	\\
2.73e-09	0	\\
2.83e-09	0	\\
2.93e-09	0	\\
3.04e-09	0	\\
3.14e-09	0	\\
3.24e-09	0	\\
3.34e-09	0	\\
3.45e-09	0	\\
3.55e-09	0	\\
3.65e-09	0	\\
3.75e-09	0	\\
3.86e-09	0	\\
3.96e-09	0	\\
4.06e-09	0	\\
4.16e-09	0	\\
4.27e-09	0	\\
4.37e-09	0	\\
4.47e-09	0	\\
4.57e-09	0	\\
4.68e-09	0	\\
4.78e-09	0	\\
4.89e-09	0	\\
4.99e-09	0	\\
5e-09	0	\\
};
\addplot [color=mycolor3,solid,forget plot]
  table[row sep=crcr]{
0	0	\\
1.1e-10	0	\\
2.2e-10	0	\\
3.3e-10	0	\\
4.4e-10	0	\\
5.4e-10	0	\\
6.5e-10	0	\\
7.5e-10	0	\\
8.6e-10	0	\\
9.6e-10	0	\\
1.07e-09	0	\\
1.18e-09	0	\\
1.28e-09	0	\\
1.38e-09	0	\\
1.49e-09	0	\\
1.59e-09	0	\\
1.69e-09	0	\\
1.8e-09	0	\\
1.9e-09	0	\\
2.01e-09	0	\\
2.11e-09	0	\\
2.21e-09	0	\\
2.32e-09	0	\\
2.42e-09	0	\\
2.52e-09	0	\\
2.63e-09	0	\\
2.73e-09	0	\\
2.83e-09	0	\\
2.93e-09	0	\\
3.04e-09	0	\\
3.14e-09	0	\\
3.24e-09	0	\\
3.34e-09	0	\\
3.45e-09	0	\\
3.55e-09	0	\\
3.65e-09	0	\\
3.75e-09	0	\\
3.86e-09	0	\\
3.96e-09	0	\\
4.06e-09	0	\\
4.16e-09	0	\\
4.27e-09	0	\\
4.37e-09	0	\\
4.47e-09	0	\\
4.57e-09	0	\\
4.68e-09	0	\\
4.78e-09	0	\\
4.89e-09	0	\\
4.99e-09	0	\\
5e-09	0	\\
};
\addplot [color=darkgray,solid,forget plot]
  table[row sep=crcr]{
0	0	\\
1.1e-10	0	\\
2.2e-10	0	\\
3.3e-10	0	\\
4.4e-10	0	\\
5.4e-10	0	\\
6.5e-10	0	\\
7.5e-10	0	\\
8.6e-10	0	\\
9.6e-10	0	\\
1.07e-09	0	\\
1.18e-09	0	\\
1.28e-09	0	\\
1.38e-09	0	\\
1.49e-09	0	\\
1.59e-09	0	\\
1.69e-09	0	\\
1.8e-09	0	\\
1.9e-09	0	\\
2.01e-09	0	\\
2.11e-09	0	\\
2.21e-09	0	\\
2.32e-09	0	\\
2.42e-09	0	\\
2.52e-09	0	\\
2.63e-09	0	\\
2.73e-09	0	\\
2.83e-09	0	\\
2.93e-09	0	\\
3.04e-09	0	\\
3.14e-09	0	\\
3.24e-09	0	\\
3.34e-09	0	\\
3.45e-09	0	\\
3.55e-09	0	\\
3.65e-09	0	\\
3.75e-09	0	\\
3.86e-09	0	\\
3.96e-09	0	\\
4.06e-09	0	\\
4.16e-09	0	\\
4.27e-09	0	\\
4.37e-09	0	\\
4.47e-09	0	\\
4.57e-09	0	\\
4.68e-09	0	\\
4.78e-09	0	\\
4.89e-09	0	\\
4.99e-09	0	\\
5e-09	0	\\
};
\addplot [color=blue,solid,forget plot]
  table[row sep=crcr]{
0	0	\\
1.1e-10	0	\\
2.2e-10	0	\\
3.3e-10	0	\\
4.4e-10	0	\\
5.4e-10	0	\\
6.5e-10	0	\\
7.5e-10	0	\\
8.6e-10	0	\\
9.6e-10	0	\\
1.07e-09	0	\\
1.18e-09	0	\\
1.28e-09	0	\\
1.38e-09	0	\\
1.49e-09	0	\\
1.59e-09	0	\\
1.69e-09	0	\\
1.8e-09	0	\\
1.9e-09	0	\\
2.01e-09	0	\\
2.11e-09	0	\\
2.21e-09	0	\\
2.32e-09	0	\\
2.42e-09	0	\\
2.52e-09	0	\\
2.63e-09	0	\\
2.73e-09	0	\\
2.83e-09	0	\\
2.93e-09	0	\\
3.04e-09	0	\\
3.14e-09	0	\\
3.24e-09	0	\\
3.34e-09	0	\\
3.45e-09	0	\\
3.55e-09	0	\\
3.65e-09	0	\\
3.75e-09	0	\\
3.86e-09	0	\\
3.96e-09	0	\\
4.06e-09	0	\\
4.16e-09	0	\\
4.27e-09	0	\\
4.37e-09	0	\\
4.47e-09	0	\\
4.57e-09	0	\\
4.68e-09	0	\\
4.78e-09	0	\\
4.89e-09	0	\\
4.99e-09	0	\\
5e-09	0	\\
};
\addplot [color=black!50!green,solid,forget plot]
  table[row sep=crcr]{
0	0	\\
1.1e-10	0	\\
2.2e-10	0	\\
3.3e-10	0	\\
4.4e-10	0	\\
5.4e-10	0	\\
6.5e-10	0	\\
7.5e-10	0	\\
8.6e-10	0	\\
9.6e-10	0	\\
1.07e-09	0	\\
1.18e-09	0	\\
1.28e-09	0	\\
1.38e-09	0	\\
1.49e-09	0	\\
1.59e-09	0	\\
1.69e-09	0	\\
1.8e-09	0	\\
1.9e-09	0	\\
2.01e-09	0	\\
2.11e-09	0	\\
2.21e-09	0	\\
2.32e-09	0	\\
2.42e-09	0	\\
2.52e-09	0	\\
2.63e-09	0	\\
2.73e-09	0	\\
2.83e-09	0	\\
2.93e-09	0	\\
3.04e-09	0	\\
3.14e-09	0	\\
3.24e-09	0	\\
3.34e-09	0	\\
3.45e-09	0	\\
3.55e-09	0	\\
3.65e-09	0	\\
3.75e-09	0	\\
3.86e-09	0	\\
3.96e-09	0	\\
4.06e-09	0	\\
4.16e-09	0	\\
4.27e-09	0	\\
4.37e-09	0	\\
4.47e-09	0	\\
4.57e-09	0	\\
4.68e-09	0	\\
4.78e-09	0	\\
4.89e-09	0	\\
4.99e-09	0	\\
5e-09	0	\\
};
\addplot [color=red,solid,forget plot]
  table[row sep=crcr]{
0	0	\\
1.1e-10	0	\\
2.2e-10	0	\\
3.3e-10	0	\\
4.4e-10	0	\\
5.4e-10	0	\\
6.5e-10	0	\\
7.5e-10	0	\\
8.6e-10	0	\\
9.6e-10	0	\\
1.07e-09	0	\\
1.18e-09	0	\\
1.28e-09	0	\\
1.38e-09	0	\\
1.49e-09	0	\\
1.59e-09	0	\\
1.69e-09	0	\\
1.8e-09	0	\\
1.9e-09	0	\\
2.01e-09	0	\\
2.11e-09	0	\\
2.21e-09	0	\\
2.32e-09	0	\\
2.42e-09	0	\\
2.52e-09	0	\\
2.63e-09	0	\\
2.73e-09	0	\\
2.83e-09	0	\\
2.93e-09	0	\\
3.04e-09	0	\\
3.14e-09	0	\\
3.24e-09	0	\\
3.34e-09	0	\\
3.45e-09	0	\\
3.55e-09	0	\\
3.65e-09	0	\\
3.75e-09	0	\\
3.86e-09	0	\\
3.96e-09	0	\\
4.06e-09	0	\\
4.16e-09	0	\\
4.27e-09	0	\\
4.37e-09	0	\\
4.47e-09	0	\\
4.57e-09	0	\\
4.68e-09	0	\\
4.78e-09	0	\\
4.89e-09	0	\\
4.99e-09	0	\\
5e-09	0	\\
};
\addplot [color=mycolor1,solid,forget plot]
  table[row sep=crcr]{
0	0	\\
1.1e-10	0	\\
2.2e-10	0	\\
3.3e-10	0	\\
4.4e-10	0	\\
5.4e-10	0	\\
6.5e-10	0	\\
7.5e-10	0	\\
8.6e-10	0	\\
9.6e-10	0	\\
1.07e-09	0	\\
1.18e-09	0	\\
1.28e-09	0	\\
1.38e-09	0	\\
1.49e-09	0	\\
1.59e-09	0	\\
1.69e-09	0	\\
1.8e-09	0	\\
1.9e-09	0	\\
2.01e-09	0	\\
2.11e-09	0	\\
2.21e-09	0	\\
2.32e-09	0	\\
2.42e-09	0	\\
2.52e-09	0	\\
2.63e-09	0	\\
2.73e-09	0	\\
2.83e-09	0	\\
2.93e-09	0	\\
3.04e-09	0	\\
3.14e-09	0	\\
3.24e-09	0	\\
3.34e-09	0	\\
3.45e-09	0	\\
3.55e-09	0	\\
3.65e-09	0	\\
3.75e-09	0	\\
3.86e-09	0	\\
3.96e-09	0	\\
4.06e-09	0	\\
4.16e-09	0	\\
4.27e-09	0	\\
4.37e-09	0	\\
4.47e-09	0	\\
4.57e-09	0	\\
4.68e-09	0	\\
4.78e-09	0	\\
4.89e-09	0	\\
4.99e-09	0	\\
5e-09	0	\\
};
\addplot [color=mycolor2,solid,forget plot]
  table[row sep=crcr]{
0	0	\\
1.1e-10	0	\\
2.2e-10	0	\\
3.3e-10	0	\\
4.4e-10	0	\\
5.4e-10	0	\\
6.5e-10	0	\\
7.5e-10	0	\\
8.6e-10	0	\\
9.6e-10	0	\\
1.07e-09	0	\\
1.18e-09	0	\\
1.28e-09	0	\\
1.38e-09	0	\\
1.49e-09	0	\\
1.59e-09	0	\\
1.69e-09	0	\\
1.8e-09	0	\\
1.9e-09	0	\\
2.01e-09	0	\\
2.11e-09	0	\\
2.21e-09	0	\\
2.32e-09	0	\\
2.42e-09	0	\\
2.52e-09	0	\\
2.63e-09	0	\\
2.73e-09	0	\\
2.83e-09	0	\\
2.93e-09	0	\\
3.04e-09	0	\\
3.14e-09	0	\\
3.24e-09	0	\\
3.34e-09	0	\\
3.45e-09	0	\\
3.55e-09	0	\\
3.65e-09	0	\\
3.75e-09	0	\\
3.86e-09	0	\\
3.96e-09	0	\\
4.06e-09	0	\\
4.16e-09	0	\\
4.27e-09	0	\\
4.37e-09	0	\\
4.47e-09	0	\\
4.57e-09	0	\\
4.68e-09	0	\\
4.78e-09	0	\\
4.89e-09	0	\\
4.99e-09	0	\\
5e-09	0	\\
};
\addplot [color=mycolor3,solid,forget plot]
  table[row sep=crcr]{
0	0	\\
1.1e-10	0	\\
2.2e-10	0	\\
3.3e-10	0	\\
4.4e-10	0	\\
5.4e-10	0	\\
6.5e-10	0	\\
7.5e-10	0	\\
8.6e-10	0	\\
9.6e-10	0	\\
1.07e-09	0	\\
1.18e-09	0	\\
1.28e-09	0	\\
1.38e-09	0	\\
1.49e-09	0	\\
1.59e-09	0	\\
1.69e-09	0	\\
1.8e-09	0	\\
1.9e-09	0	\\
2.01e-09	0	\\
2.11e-09	0	\\
2.21e-09	0	\\
2.32e-09	0	\\
2.42e-09	0	\\
2.52e-09	0	\\
2.63e-09	0	\\
2.73e-09	0	\\
2.83e-09	0	\\
2.93e-09	0	\\
3.04e-09	0	\\
3.14e-09	0	\\
3.24e-09	0	\\
3.34e-09	0	\\
3.45e-09	0	\\
3.55e-09	0	\\
3.65e-09	0	\\
3.75e-09	0	\\
3.86e-09	0	\\
3.96e-09	0	\\
4.06e-09	0	\\
4.16e-09	0	\\
4.27e-09	0	\\
4.37e-09	0	\\
4.47e-09	0	\\
4.57e-09	0	\\
4.68e-09	0	\\
4.78e-09	0	\\
4.89e-09	0	\\
4.99e-09	0	\\
5e-09	0	\\
};
\addplot [color=darkgray,solid,forget plot]
  table[row sep=crcr]{
0	0	\\
1.1e-10	0	\\
2.2e-10	0	\\
3.3e-10	0	\\
4.4e-10	0	\\
5.4e-10	0	\\
6.5e-10	0	\\
7.5e-10	0	\\
8.6e-10	0	\\
9.6e-10	0	\\
1.07e-09	0	\\
1.18e-09	0	\\
1.28e-09	0	\\
1.38e-09	0	\\
1.49e-09	0	\\
1.59e-09	0	\\
1.69e-09	0	\\
1.8e-09	0	\\
1.9e-09	0	\\
2.01e-09	0	\\
2.11e-09	0	\\
2.21e-09	0	\\
2.32e-09	0	\\
2.42e-09	0	\\
2.52e-09	0	\\
2.63e-09	0	\\
2.73e-09	0	\\
2.83e-09	0	\\
2.93e-09	0	\\
3.04e-09	0	\\
3.14e-09	0	\\
3.24e-09	0	\\
3.34e-09	0	\\
3.45e-09	0	\\
3.55e-09	0	\\
3.65e-09	0	\\
3.75e-09	0	\\
3.86e-09	0	\\
3.96e-09	0	\\
4.06e-09	0	\\
4.16e-09	0	\\
4.27e-09	0	\\
4.37e-09	0	\\
4.47e-09	0	\\
4.57e-09	0	\\
4.68e-09	0	\\
4.78e-09	0	\\
4.89e-09	0	\\
4.99e-09	0	\\
5e-09	0	\\
};
\addplot [color=blue,solid,forget plot]
  table[row sep=crcr]{
0	0	\\
1.1e-10	0	\\
2.2e-10	0	\\
3.3e-10	0	\\
4.4e-10	0	\\
5.4e-10	0	\\
6.5e-10	0	\\
7.5e-10	0	\\
8.6e-10	0	\\
9.6e-10	0	\\
1.07e-09	0	\\
1.18e-09	0	\\
1.28e-09	0	\\
1.38e-09	0	\\
1.49e-09	0	\\
1.59e-09	0	\\
1.69e-09	0	\\
1.8e-09	0	\\
1.9e-09	0	\\
2.01e-09	0	\\
2.11e-09	0	\\
2.21e-09	0	\\
2.32e-09	0	\\
2.42e-09	0	\\
2.52e-09	0	\\
2.63e-09	0	\\
2.73e-09	0	\\
2.83e-09	0	\\
2.93e-09	0	\\
3.04e-09	0	\\
3.14e-09	0	\\
3.24e-09	0	\\
3.34e-09	0	\\
3.45e-09	0	\\
3.55e-09	0	\\
3.65e-09	0	\\
3.75e-09	0	\\
3.86e-09	0	\\
3.96e-09	0	\\
4.06e-09	0	\\
4.16e-09	0	\\
4.27e-09	0	\\
4.37e-09	0	\\
4.47e-09	0	\\
4.57e-09	0	\\
4.68e-09	0	\\
4.78e-09	0	\\
4.89e-09	0	\\
4.99e-09	0	\\
5e-09	0	\\
};
\addplot [color=black!50!green,solid,forget plot]
  table[row sep=crcr]{
0	0	\\
1.1e-10	0	\\
2.2e-10	0	\\
3.3e-10	0	\\
4.4e-10	0	\\
5.4e-10	0	\\
6.5e-10	0	\\
7.5e-10	0	\\
8.6e-10	0	\\
9.6e-10	0	\\
1.07e-09	0	\\
1.18e-09	0	\\
1.28e-09	0	\\
1.38e-09	0	\\
1.49e-09	0	\\
1.59e-09	0	\\
1.69e-09	0	\\
1.8e-09	0	\\
1.9e-09	0	\\
2.01e-09	0	\\
2.11e-09	0	\\
2.21e-09	0	\\
2.32e-09	0	\\
2.42e-09	0	\\
2.52e-09	0	\\
2.63e-09	0	\\
2.73e-09	0	\\
2.83e-09	0	\\
2.93e-09	0	\\
3.04e-09	0	\\
3.14e-09	0	\\
3.24e-09	0	\\
3.34e-09	0	\\
3.45e-09	0	\\
3.55e-09	0	\\
3.65e-09	0	\\
3.75e-09	0	\\
3.86e-09	0	\\
3.96e-09	0	\\
4.06e-09	0	\\
4.16e-09	0	\\
4.27e-09	0	\\
4.37e-09	0	\\
4.47e-09	0	\\
4.57e-09	0	\\
4.68e-09	0	\\
4.78e-09	0	\\
4.89e-09	0	\\
4.99e-09	0	\\
5e-09	0	\\
};
\addplot [color=red,solid,forget plot]
  table[row sep=crcr]{
0	0	\\
1.1e-10	0	\\
2.2e-10	0	\\
3.3e-10	0	\\
4.4e-10	0	\\
5.4e-10	0	\\
6.5e-10	0	\\
7.5e-10	0	\\
8.6e-10	0	\\
9.6e-10	0	\\
1.07e-09	0	\\
1.18e-09	0	\\
1.28e-09	0	\\
1.38e-09	0	\\
1.49e-09	0	\\
1.59e-09	0	\\
1.69e-09	0	\\
1.8e-09	0	\\
1.9e-09	0	\\
2.01e-09	0	\\
2.11e-09	0	\\
2.21e-09	0	\\
2.32e-09	0	\\
2.42e-09	0	\\
2.52e-09	0	\\
2.63e-09	0	\\
2.73e-09	0	\\
2.83e-09	0	\\
2.93e-09	0	\\
3.04e-09	0	\\
3.14e-09	0	\\
3.24e-09	0	\\
3.34e-09	0	\\
3.45e-09	0	\\
3.55e-09	0	\\
3.65e-09	0	\\
3.75e-09	0	\\
3.86e-09	0	\\
3.96e-09	0	\\
4.06e-09	0	\\
4.16e-09	0	\\
4.27e-09	0	\\
4.37e-09	0	\\
4.47e-09	0	\\
4.57e-09	0	\\
4.68e-09	0	\\
4.78e-09	0	\\
4.89e-09	0	\\
4.99e-09	0	\\
5e-09	0	\\
};
\addplot [color=mycolor1,solid,forget plot]
  table[row sep=crcr]{
0	0	\\
1.1e-10	0	\\
2.2e-10	0	\\
3.3e-10	0	\\
4.4e-10	0	\\
5.4e-10	0	\\
6.5e-10	0	\\
7.5e-10	0	\\
8.6e-10	0	\\
9.6e-10	0	\\
1.07e-09	0	\\
1.18e-09	0	\\
1.28e-09	0	\\
1.38e-09	0	\\
1.49e-09	0	\\
1.59e-09	0	\\
1.69e-09	0	\\
1.8e-09	0	\\
1.9e-09	0	\\
2.01e-09	0	\\
2.11e-09	0	\\
2.21e-09	0	\\
2.32e-09	0	\\
2.42e-09	0	\\
2.52e-09	0	\\
2.63e-09	0	\\
2.73e-09	0	\\
2.83e-09	0	\\
2.93e-09	0	\\
3.04e-09	0	\\
3.14e-09	0	\\
3.24e-09	0	\\
3.34e-09	0	\\
3.45e-09	0	\\
3.55e-09	0	\\
3.65e-09	0	\\
3.75e-09	0	\\
3.86e-09	0	\\
3.96e-09	0	\\
4.06e-09	0	\\
4.16e-09	0	\\
4.27e-09	0	\\
4.37e-09	0	\\
4.47e-09	0	\\
4.57e-09	0	\\
4.68e-09	0	\\
4.78e-09	0	\\
4.89e-09	0	\\
4.99e-09	0	\\
5e-09	0	\\
};
\addplot [color=mycolor2,solid,forget plot]
  table[row sep=crcr]{
0	0	\\
1.1e-10	0	\\
2.2e-10	0	\\
3.3e-10	0	\\
4.4e-10	0	\\
5.4e-10	0	\\
6.5e-10	0	\\
7.5e-10	0	\\
8.6e-10	0	\\
9.6e-10	0	\\
1.07e-09	0	\\
1.18e-09	0	\\
1.28e-09	0	\\
1.38e-09	0	\\
1.49e-09	0	\\
1.59e-09	0	\\
1.69e-09	0	\\
1.8e-09	0	\\
1.9e-09	0	\\
2.01e-09	0	\\
2.11e-09	0	\\
2.21e-09	0	\\
2.32e-09	0	\\
2.42e-09	0	\\
2.52e-09	0	\\
2.63e-09	0	\\
2.73e-09	0	\\
2.83e-09	0	\\
2.93e-09	0	\\
3.04e-09	0	\\
3.14e-09	0	\\
3.24e-09	0	\\
3.34e-09	0	\\
3.45e-09	0	\\
3.55e-09	0	\\
3.65e-09	0	\\
3.75e-09	0	\\
3.86e-09	0	\\
3.96e-09	0	\\
4.06e-09	0	\\
4.16e-09	0	\\
4.27e-09	0	\\
4.37e-09	0	\\
4.47e-09	0	\\
4.57e-09	0	\\
4.68e-09	0	\\
4.78e-09	0	\\
4.89e-09	0	\\
4.99e-09	0	\\
5e-09	0	\\
};
\addplot [color=mycolor3,solid,forget plot]
  table[row sep=crcr]{
0	0	\\
1.1e-10	0	\\
2.2e-10	0	\\
3.3e-10	0	\\
4.4e-10	0	\\
5.4e-10	0	\\
6.5e-10	0	\\
7.5e-10	0	\\
8.6e-10	0	\\
9.6e-10	0	\\
1.07e-09	0	\\
1.18e-09	0	\\
1.28e-09	0	\\
1.38e-09	0	\\
1.49e-09	0	\\
1.59e-09	0	\\
1.69e-09	0	\\
1.8e-09	0	\\
1.9e-09	0	\\
2.01e-09	0	\\
2.11e-09	0	\\
2.21e-09	0	\\
2.32e-09	0	\\
2.42e-09	0	\\
2.52e-09	0	\\
2.63e-09	0	\\
2.73e-09	0	\\
2.83e-09	0	\\
2.93e-09	0	\\
3.04e-09	0	\\
3.14e-09	0	\\
3.24e-09	0	\\
3.34e-09	0	\\
3.45e-09	0	\\
3.55e-09	0	\\
3.65e-09	0	\\
3.75e-09	0	\\
3.86e-09	0	\\
3.96e-09	0	\\
4.06e-09	0	\\
4.16e-09	0	\\
4.27e-09	0	\\
4.37e-09	0	\\
4.47e-09	0	\\
4.57e-09	0	\\
4.68e-09	0	\\
4.78e-09	0	\\
4.89e-09	0	\\
4.99e-09	0	\\
5e-09	0	\\
};
\addplot [color=darkgray,solid,forget plot]
  table[row sep=crcr]{
0	0	\\
1.1e-10	0	\\
2.2e-10	0	\\
3.3e-10	0	\\
4.4e-10	0	\\
5.4e-10	0	\\
6.5e-10	0	\\
7.5e-10	0	\\
8.6e-10	0	\\
9.6e-10	0	\\
1.07e-09	0	\\
1.18e-09	0	\\
1.28e-09	0	\\
1.38e-09	0	\\
1.49e-09	0	\\
1.59e-09	0	\\
1.69e-09	0	\\
1.8e-09	0	\\
1.9e-09	0	\\
2.01e-09	0	\\
2.11e-09	0	\\
2.21e-09	0	\\
2.32e-09	0	\\
2.42e-09	0	\\
2.52e-09	0	\\
2.63e-09	0	\\
2.73e-09	0	\\
2.83e-09	0	\\
2.93e-09	0	\\
3.04e-09	0	\\
3.14e-09	0	\\
3.24e-09	0	\\
3.34e-09	0	\\
3.45e-09	0	\\
3.55e-09	0	\\
3.65e-09	0	\\
3.75e-09	0	\\
3.86e-09	0	\\
3.96e-09	0	\\
4.06e-09	0	\\
4.16e-09	0	\\
4.27e-09	0	\\
4.37e-09	0	\\
4.47e-09	0	\\
4.57e-09	0	\\
4.68e-09	0	\\
4.78e-09	0	\\
4.89e-09	0	\\
4.99e-09	0	\\
5e-09	0	\\
};
\addplot [color=blue,solid,forget plot]
  table[row sep=crcr]{
0	0	\\
1.1e-10	0	\\
2.2e-10	0	\\
3.3e-10	0	\\
4.4e-10	0	\\
5.4e-10	0	\\
6.5e-10	0	\\
7.5e-10	0	\\
8.6e-10	0	\\
9.6e-10	0	\\
1.07e-09	0	\\
1.18e-09	0	\\
1.28e-09	0	\\
1.38e-09	0	\\
1.49e-09	0	\\
1.59e-09	0	\\
1.69e-09	0	\\
1.8e-09	0	\\
1.9e-09	0	\\
2.01e-09	0	\\
2.11e-09	0	\\
2.21e-09	0	\\
2.32e-09	0	\\
2.42e-09	0	\\
2.52e-09	0	\\
2.63e-09	0	\\
2.73e-09	0	\\
2.83e-09	0	\\
2.93e-09	0	\\
3.04e-09	0	\\
3.14e-09	0	\\
3.24e-09	0	\\
3.34e-09	0	\\
3.45e-09	0	\\
3.55e-09	0	\\
3.65e-09	0	\\
3.75e-09	0	\\
3.86e-09	0	\\
3.96e-09	0	\\
4.06e-09	0	\\
4.16e-09	0	\\
4.27e-09	0	\\
4.37e-09	0	\\
4.47e-09	0	\\
4.57e-09	0	\\
4.68e-09	0	\\
4.78e-09	0	\\
4.89e-09	0	\\
4.99e-09	0	\\
5e-09	0	\\
};
\addplot [color=black!50!green,solid,forget plot]
  table[row sep=crcr]{
0	0	\\
1.1e-10	0	\\
2.2e-10	0	\\
3.3e-10	0	\\
4.4e-10	0	\\
5.4e-10	0	\\
6.5e-10	0	\\
7.5e-10	0	\\
8.6e-10	0	\\
9.6e-10	0	\\
1.07e-09	0	\\
1.18e-09	0	\\
1.28e-09	0	\\
1.38e-09	0	\\
1.49e-09	0	\\
1.59e-09	0	\\
1.69e-09	0	\\
1.8e-09	0	\\
1.9e-09	0	\\
2.01e-09	0	\\
2.11e-09	0	\\
2.21e-09	0	\\
2.32e-09	0	\\
2.42e-09	0	\\
2.52e-09	0	\\
2.63e-09	0	\\
2.73e-09	0	\\
2.83e-09	0	\\
2.93e-09	0	\\
3.04e-09	0	\\
3.14e-09	0	\\
3.24e-09	0	\\
3.34e-09	0	\\
3.45e-09	0	\\
3.55e-09	0	\\
3.65e-09	0	\\
3.75e-09	0	\\
3.86e-09	0	\\
3.96e-09	0	\\
4.06e-09	0	\\
4.16e-09	0	\\
4.27e-09	0	\\
4.37e-09	0	\\
4.47e-09	0	\\
4.57e-09	0	\\
4.68e-09	0	\\
4.78e-09	0	\\
4.89e-09	0	\\
4.99e-09	0	\\
5e-09	0	\\
};
\addplot [color=red,solid,forget plot]
  table[row sep=crcr]{
0	0	\\
1.1e-10	0	\\
2.2e-10	0	\\
3.3e-10	0	\\
4.4e-10	0	\\
5.4e-10	0	\\
6.5e-10	0	\\
7.5e-10	0	\\
8.6e-10	0	\\
9.6e-10	0	\\
1.07e-09	0	\\
1.18e-09	0	\\
1.28e-09	0	\\
1.38e-09	0	\\
1.49e-09	0	\\
1.59e-09	0	\\
1.69e-09	0	\\
1.8e-09	0	\\
1.9e-09	0	\\
2.01e-09	0	\\
2.11e-09	0	\\
2.21e-09	0	\\
2.32e-09	0	\\
2.42e-09	0	\\
2.52e-09	0	\\
2.63e-09	0	\\
2.73e-09	0	\\
2.83e-09	0	\\
2.93e-09	0	\\
3.04e-09	0	\\
3.14e-09	0	\\
3.24e-09	0	\\
3.34e-09	0	\\
3.45e-09	0	\\
3.55e-09	0	\\
3.65e-09	0	\\
3.75e-09	0	\\
3.86e-09	0	\\
3.96e-09	0	\\
4.06e-09	0	\\
4.16e-09	0	\\
4.27e-09	0	\\
4.37e-09	0	\\
4.47e-09	0	\\
4.57e-09	0	\\
4.68e-09	0	\\
4.78e-09	0	\\
4.89e-09	0	\\
4.99e-09	0	\\
5e-09	0	\\
};
\addplot [color=mycolor1,solid,forget plot]
  table[row sep=crcr]{
0	0	\\
1.1e-10	0	\\
2.2e-10	0	\\
3.3e-10	0	\\
4.4e-10	0	\\
5.4e-10	0	\\
6.5e-10	0	\\
7.5e-10	0	\\
8.6e-10	0	\\
9.6e-10	0	\\
1.07e-09	0	\\
1.18e-09	0	\\
1.28e-09	0	\\
1.38e-09	0	\\
1.49e-09	0	\\
1.59e-09	0	\\
1.69e-09	0	\\
1.8e-09	0	\\
1.9e-09	0	\\
2.01e-09	0	\\
2.11e-09	0	\\
2.21e-09	0	\\
2.32e-09	0	\\
2.42e-09	0	\\
2.52e-09	0	\\
2.63e-09	0	\\
2.73e-09	0	\\
2.83e-09	0	\\
2.93e-09	0	\\
3.04e-09	0	\\
3.14e-09	0	\\
3.24e-09	0	\\
3.34e-09	0	\\
3.45e-09	0	\\
3.55e-09	0	\\
3.65e-09	0	\\
3.75e-09	0	\\
3.86e-09	0	\\
3.96e-09	0	\\
4.06e-09	0	\\
4.16e-09	0	\\
4.27e-09	0	\\
4.37e-09	0	\\
4.47e-09	0	\\
4.57e-09	0	\\
4.68e-09	0	\\
4.78e-09	0	\\
4.89e-09	0	\\
4.99e-09	0	\\
5e-09	0	\\
};
\addplot [color=mycolor2,solid,forget plot]
  table[row sep=crcr]{
0	0	\\
1.1e-10	0	\\
2.2e-10	0	\\
3.3e-10	0	\\
4.4e-10	0	\\
5.4e-10	0	\\
6.5e-10	0	\\
7.5e-10	0	\\
8.6e-10	0	\\
9.6e-10	0	\\
1.07e-09	0	\\
1.18e-09	0	\\
1.28e-09	0	\\
1.38e-09	0	\\
1.49e-09	0	\\
1.59e-09	0	\\
1.69e-09	0	\\
1.8e-09	0	\\
1.9e-09	0	\\
2.01e-09	0	\\
2.11e-09	0	\\
2.21e-09	0	\\
2.32e-09	0	\\
2.42e-09	0	\\
2.52e-09	0	\\
2.63e-09	0	\\
2.73e-09	0	\\
2.83e-09	0	\\
2.93e-09	0	\\
3.04e-09	0	\\
3.14e-09	0	\\
3.24e-09	0	\\
3.34e-09	0	\\
3.45e-09	0	\\
3.55e-09	0	\\
3.65e-09	0	\\
3.75e-09	0	\\
3.86e-09	0	\\
3.96e-09	0	\\
4.06e-09	0	\\
4.16e-09	0	\\
4.27e-09	0	\\
4.37e-09	0	\\
4.47e-09	0	\\
4.57e-09	0	\\
4.68e-09	0	\\
4.78e-09	0	\\
4.89e-09	0	\\
4.99e-09	0	\\
5e-09	0	\\
};
\addplot [color=mycolor3,solid,forget plot]
  table[row sep=crcr]{
0	0	\\
1.1e-10	0	\\
2.2e-10	0	\\
3.3e-10	0	\\
4.4e-10	0	\\
5.4e-10	0	\\
6.5e-10	0	\\
7.5e-10	0	\\
8.6e-10	0	\\
9.6e-10	0	\\
1.07e-09	0	\\
1.18e-09	0	\\
1.28e-09	0	\\
1.38e-09	0	\\
1.49e-09	0	\\
1.59e-09	0	\\
1.69e-09	0	\\
1.8e-09	0	\\
1.9e-09	0	\\
2.01e-09	0	\\
2.11e-09	0	\\
2.21e-09	0	\\
2.32e-09	0	\\
2.42e-09	0	\\
2.52e-09	0	\\
2.63e-09	0	\\
2.73e-09	0	\\
2.83e-09	0	\\
2.93e-09	0	\\
3.04e-09	0	\\
3.14e-09	0	\\
3.24e-09	0	\\
3.34e-09	0	\\
3.45e-09	0	\\
3.55e-09	0	\\
3.65e-09	0	\\
3.75e-09	0	\\
3.86e-09	0	\\
3.96e-09	0	\\
4.06e-09	0	\\
4.16e-09	0	\\
4.27e-09	0	\\
4.37e-09	0	\\
4.47e-09	0	\\
4.57e-09	0	\\
4.68e-09	0	\\
4.78e-09	0	\\
4.89e-09	0	\\
4.99e-09	0	\\
5e-09	0	\\
};
\addplot [color=darkgray,solid,forget plot]
  table[row sep=crcr]{
0	0	\\
1.1e-10	0	\\
2.2e-10	0	\\
3.3e-10	0	\\
4.4e-10	0	\\
5.4e-10	0	\\
6.5e-10	0	\\
7.5e-10	0	\\
8.6e-10	0	\\
9.6e-10	0	\\
1.07e-09	0	\\
1.18e-09	0	\\
1.28e-09	0	\\
1.38e-09	0	\\
1.49e-09	0	\\
1.59e-09	0	\\
1.69e-09	0	\\
1.8e-09	0	\\
1.9e-09	0	\\
2.01e-09	0	\\
2.11e-09	0	\\
2.21e-09	0	\\
2.32e-09	0	\\
2.42e-09	0	\\
2.52e-09	0	\\
2.63e-09	0	\\
2.73e-09	0	\\
2.83e-09	0	\\
2.93e-09	0	\\
3.04e-09	0	\\
3.14e-09	0	\\
3.24e-09	0	\\
3.34e-09	0	\\
3.45e-09	0	\\
3.55e-09	0	\\
3.65e-09	0	\\
3.75e-09	0	\\
3.86e-09	0	\\
3.96e-09	0	\\
4.06e-09	0	\\
4.16e-09	0	\\
4.27e-09	0	\\
4.37e-09	0	\\
4.47e-09	0	\\
4.57e-09	0	\\
4.68e-09	0	\\
4.78e-09	0	\\
4.89e-09	0	\\
4.99e-09	0	\\
5e-09	0	\\
};
\addplot [color=blue,solid,forget plot]
  table[row sep=crcr]{
0	0	\\
1.1e-10	0	\\
2.2e-10	0	\\
3.3e-10	0	\\
4.4e-10	0	\\
5.4e-10	0	\\
6.5e-10	0	\\
7.5e-10	0	\\
8.6e-10	0	\\
9.6e-10	0	\\
1.07e-09	0	\\
1.18e-09	0	\\
1.28e-09	0	\\
1.38e-09	0	\\
1.49e-09	0	\\
1.59e-09	0	\\
1.69e-09	0	\\
1.8e-09	0	\\
1.9e-09	0	\\
2.01e-09	0	\\
2.11e-09	0	\\
2.21e-09	0	\\
2.32e-09	0	\\
2.42e-09	0	\\
2.52e-09	0	\\
2.63e-09	0	\\
2.73e-09	0	\\
2.83e-09	0	\\
2.93e-09	0	\\
3.04e-09	0	\\
3.14e-09	0	\\
3.24e-09	0	\\
3.34e-09	0	\\
3.45e-09	0	\\
3.55e-09	0	\\
3.65e-09	0	\\
3.75e-09	0	\\
3.86e-09	0	\\
3.96e-09	0	\\
4.06e-09	0	\\
4.16e-09	0	\\
4.27e-09	0	\\
4.37e-09	0	\\
4.47e-09	0	\\
4.57e-09	0	\\
4.68e-09	0	\\
4.78e-09	0	\\
4.89e-09	0	\\
4.99e-09	0	\\
5e-09	0	\\
};
\addplot [color=black!50!green,solid,forget plot]
  table[row sep=crcr]{
0	0	\\
1.1e-10	0	\\
2.2e-10	0	\\
3.3e-10	0	\\
4.4e-10	0	\\
5.4e-10	0	\\
6.5e-10	0	\\
7.5e-10	0	\\
8.6e-10	0	\\
9.6e-10	0	\\
1.07e-09	0	\\
1.18e-09	0	\\
1.28e-09	0	\\
1.38e-09	0	\\
1.49e-09	0	\\
1.59e-09	0	\\
1.69e-09	0	\\
1.8e-09	0	\\
1.9e-09	0	\\
2.01e-09	0	\\
2.11e-09	0	\\
2.21e-09	0	\\
2.32e-09	0	\\
2.42e-09	0	\\
2.52e-09	0	\\
2.63e-09	0	\\
2.73e-09	0	\\
2.83e-09	0	\\
2.93e-09	0	\\
3.04e-09	0	\\
3.14e-09	0	\\
3.24e-09	0	\\
3.34e-09	0	\\
3.45e-09	0	\\
3.55e-09	0	\\
3.65e-09	0	\\
3.75e-09	0	\\
3.86e-09	0	\\
3.96e-09	0	\\
4.06e-09	0	\\
4.16e-09	0	\\
4.27e-09	0	\\
4.37e-09	0	\\
4.47e-09	0	\\
4.57e-09	0	\\
4.68e-09	0	\\
4.78e-09	0	\\
4.89e-09	0	\\
4.99e-09	0	\\
5e-09	0	\\
};
\addplot [color=red,solid,forget plot]
  table[row sep=crcr]{
0	0	\\
1.1e-10	0	\\
2.2e-10	0	\\
3.3e-10	0	\\
4.4e-10	0	\\
5.4e-10	0	\\
6.5e-10	0	\\
7.5e-10	0	\\
8.6e-10	0	\\
9.6e-10	0	\\
1.07e-09	0	\\
1.18e-09	0	\\
1.28e-09	0	\\
1.38e-09	0	\\
1.49e-09	0	\\
1.59e-09	0	\\
1.69e-09	0	\\
1.8e-09	0	\\
1.9e-09	0	\\
2.01e-09	0	\\
2.11e-09	0	\\
2.21e-09	0	\\
2.32e-09	0	\\
2.42e-09	0	\\
2.52e-09	0	\\
2.63e-09	0	\\
2.73e-09	0	\\
2.83e-09	0	\\
2.93e-09	0	\\
3.04e-09	0	\\
3.14e-09	0	\\
3.24e-09	0	\\
3.34e-09	0	\\
3.45e-09	0	\\
3.55e-09	0	\\
3.65e-09	0	\\
3.75e-09	0	\\
3.86e-09	0	\\
3.96e-09	0	\\
4.06e-09	0	\\
4.16e-09	0	\\
4.27e-09	0	\\
4.37e-09	0	\\
4.47e-09	0	\\
4.57e-09	0	\\
4.68e-09	0	\\
4.78e-09	0	\\
4.89e-09	0	\\
4.99e-09	0	\\
5e-09	0	\\
};
\addplot [color=mycolor1,solid,forget plot]
  table[row sep=crcr]{
0	0	\\
1.1e-10	0	\\
2.2e-10	0	\\
3.3e-10	0	\\
4.4e-10	0	\\
5.4e-10	0	\\
6.5e-10	0	\\
7.5e-10	0	\\
8.6e-10	0	\\
9.6e-10	0	\\
1.07e-09	0	\\
1.18e-09	0	\\
1.28e-09	0	\\
1.38e-09	0	\\
1.49e-09	0	\\
1.59e-09	0	\\
1.69e-09	0	\\
1.8e-09	0	\\
1.9e-09	0	\\
2.01e-09	0	\\
2.11e-09	0	\\
2.21e-09	0	\\
2.32e-09	0	\\
2.42e-09	0	\\
2.52e-09	0	\\
2.63e-09	0	\\
2.73e-09	0	\\
2.83e-09	0	\\
2.93e-09	0	\\
3.04e-09	0	\\
3.14e-09	0	\\
3.24e-09	0	\\
3.34e-09	0	\\
3.45e-09	0	\\
3.55e-09	0	\\
3.65e-09	0	\\
3.75e-09	0	\\
3.86e-09	0	\\
3.96e-09	0	\\
4.06e-09	0	\\
4.16e-09	0	\\
4.27e-09	0	\\
4.37e-09	0	\\
4.47e-09	0	\\
4.57e-09	0	\\
4.68e-09	0	\\
4.78e-09	0	\\
4.89e-09	0	\\
4.99e-09	0	\\
5e-09	0	\\
};
\addplot [color=mycolor2,solid,forget plot]
  table[row sep=crcr]{
0	0	\\
1.1e-10	0	\\
2.2e-10	0	\\
3.3e-10	0	\\
4.4e-10	0	\\
5.4e-10	0	\\
6.5e-10	0	\\
7.5e-10	0	\\
8.6e-10	0	\\
9.6e-10	0	\\
1.07e-09	0	\\
1.18e-09	0	\\
1.28e-09	0	\\
1.38e-09	0	\\
1.49e-09	0	\\
1.59e-09	0	\\
1.69e-09	0	\\
1.8e-09	0	\\
1.9e-09	0	\\
2.01e-09	0	\\
2.11e-09	0	\\
2.21e-09	0	\\
2.32e-09	0	\\
2.42e-09	0	\\
2.52e-09	0	\\
2.63e-09	0	\\
2.73e-09	0	\\
2.83e-09	0	\\
2.93e-09	0	\\
3.04e-09	0	\\
3.14e-09	0	\\
3.24e-09	0	\\
3.34e-09	0	\\
3.45e-09	0	\\
3.55e-09	0	\\
3.65e-09	0	\\
3.75e-09	0	\\
3.86e-09	0	\\
3.96e-09	0	\\
4.06e-09	0	\\
4.16e-09	0	\\
4.27e-09	0	\\
4.37e-09	0	\\
4.47e-09	0	\\
4.57e-09	0	\\
4.68e-09	0	\\
4.78e-09	0	\\
4.89e-09	0	\\
4.99e-09	0	\\
5e-09	0	\\
};
\addplot [color=mycolor3,solid,forget plot]
  table[row sep=crcr]{
0	0	\\
1.1e-10	0	\\
2.2e-10	0	\\
3.3e-10	0	\\
4.4e-10	0	\\
5.4e-10	0	\\
6.5e-10	0	\\
7.5e-10	0	\\
8.6e-10	0	\\
9.6e-10	0	\\
1.07e-09	0	\\
1.18e-09	0	\\
1.28e-09	0	\\
1.38e-09	0	\\
1.49e-09	0	\\
1.59e-09	0	\\
1.69e-09	0	\\
1.8e-09	0	\\
1.9e-09	0	\\
2.01e-09	0	\\
2.11e-09	0	\\
2.21e-09	0	\\
2.32e-09	0	\\
2.42e-09	0	\\
2.52e-09	0	\\
2.63e-09	0	\\
2.73e-09	0	\\
2.83e-09	0	\\
2.93e-09	0	\\
3.04e-09	0	\\
3.14e-09	0	\\
3.24e-09	0	\\
3.34e-09	0	\\
3.45e-09	0	\\
3.55e-09	0	\\
3.65e-09	0	\\
3.75e-09	0	\\
3.86e-09	0	\\
3.96e-09	0	\\
4.06e-09	0	\\
4.16e-09	0	\\
4.27e-09	0	\\
4.37e-09	0	\\
4.47e-09	0	\\
4.57e-09	0	\\
4.68e-09	0	\\
4.78e-09	0	\\
4.89e-09	0	\\
4.99e-09	0	\\
5e-09	0	\\
};
\addplot [color=darkgray,solid,forget plot]
  table[row sep=crcr]{
0	0	\\
1.1e-10	0	\\
2.2e-10	0	\\
3.3e-10	0	\\
4.4e-10	0	\\
5.4e-10	0	\\
6.5e-10	0	\\
7.5e-10	0	\\
8.6e-10	0	\\
9.6e-10	0	\\
1.07e-09	0	\\
1.18e-09	0	\\
1.28e-09	0	\\
1.38e-09	0	\\
1.49e-09	0	\\
1.59e-09	0	\\
1.69e-09	0	\\
1.8e-09	0	\\
1.9e-09	0	\\
2.01e-09	0	\\
2.11e-09	0	\\
2.21e-09	0	\\
2.32e-09	0	\\
2.42e-09	0	\\
2.52e-09	0	\\
2.63e-09	0	\\
2.73e-09	0	\\
2.83e-09	0	\\
2.93e-09	0	\\
3.04e-09	0	\\
3.14e-09	0	\\
3.24e-09	0	\\
3.34e-09	0	\\
3.45e-09	0	\\
3.55e-09	0	\\
3.65e-09	0	\\
3.75e-09	0	\\
3.86e-09	0	\\
3.96e-09	0	\\
4.06e-09	0	\\
4.16e-09	0	\\
4.27e-09	0	\\
4.37e-09	0	\\
4.47e-09	0	\\
4.57e-09	0	\\
4.68e-09	0	\\
4.78e-09	0	\\
4.89e-09	0	\\
4.99e-09	0	\\
5e-09	0	\\
};
\addplot [color=blue,solid,forget plot]
  table[row sep=crcr]{
0	0	\\
1.1e-10	0	\\
2.2e-10	0	\\
3.3e-10	0	\\
4.4e-10	0	\\
5.4e-10	0	\\
6.5e-10	0	\\
7.5e-10	0	\\
8.6e-10	0	\\
9.6e-10	0	\\
1.07e-09	0	\\
1.18e-09	0	\\
1.28e-09	0	\\
1.38e-09	0	\\
1.49e-09	0	\\
1.59e-09	0	\\
1.69e-09	0	\\
1.8e-09	0	\\
1.9e-09	0	\\
2.01e-09	0	\\
2.11e-09	0	\\
2.21e-09	0	\\
2.32e-09	0	\\
2.42e-09	0	\\
2.52e-09	0	\\
2.63e-09	0	\\
2.73e-09	0	\\
2.83e-09	0	\\
2.93e-09	0	\\
3.04e-09	0	\\
3.14e-09	0	\\
3.24e-09	0	\\
3.34e-09	0	\\
3.45e-09	0	\\
3.55e-09	0	\\
3.65e-09	0	\\
3.75e-09	0	\\
3.86e-09	0	\\
3.96e-09	0	\\
4.06e-09	0	\\
4.16e-09	0	\\
4.27e-09	0	\\
4.37e-09	0	\\
4.47e-09	0	\\
4.57e-09	0	\\
4.68e-09	0	\\
4.78e-09	0	\\
4.89e-09	0	\\
4.99e-09	0	\\
5e-09	0	\\
};
\addplot [color=black!50!green,solid,forget plot]
  table[row sep=crcr]{
0	0	\\
1.1e-10	0	\\
2.2e-10	0	\\
3.3e-10	0	\\
4.4e-10	0	\\
5.4e-10	0	\\
6.5e-10	0	\\
7.5e-10	0	\\
8.6e-10	0	\\
9.6e-10	0	\\
1.07e-09	0	\\
1.18e-09	0	\\
1.28e-09	0	\\
1.38e-09	0	\\
1.49e-09	0	\\
1.59e-09	0	\\
1.69e-09	0	\\
1.8e-09	0	\\
1.9e-09	0	\\
2.01e-09	0	\\
2.11e-09	0	\\
2.21e-09	0	\\
2.32e-09	0	\\
2.42e-09	0	\\
2.52e-09	0	\\
2.63e-09	0	\\
2.73e-09	0	\\
2.83e-09	0	\\
2.93e-09	0	\\
3.04e-09	0	\\
3.14e-09	0	\\
3.24e-09	0	\\
3.34e-09	0	\\
3.45e-09	0	\\
3.55e-09	0	\\
3.65e-09	0	\\
3.75e-09	0	\\
3.86e-09	0	\\
3.96e-09	0	\\
4.06e-09	0	\\
4.16e-09	0	\\
4.27e-09	0	\\
4.37e-09	0	\\
4.47e-09	0	\\
4.57e-09	0	\\
4.68e-09	0	\\
4.78e-09	0	\\
4.89e-09	0	\\
4.99e-09	0	\\
5e-09	0	\\
};
\addplot [color=red,solid,forget plot]
  table[row sep=crcr]{
0	0	\\
1.1e-10	0	\\
2.2e-10	0	\\
3.3e-10	0	\\
4.4e-10	0	\\
5.4e-10	0	\\
6.5e-10	0	\\
7.5e-10	0	\\
8.6e-10	0	\\
9.6e-10	0	\\
1.07e-09	0	\\
1.18e-09	0	\\
1.28e-09	0	\\
1.38e-09	0	\\
1.49e-09	0	\\
1.59e-09	0	\\
1.69e-09	0	\\
1.8e-09	0	\\
1.9e-09	0	\\
2.01e-09	0	\\
2.11e-09	0	\\
2.21e-09	0	\\
2.32e-09	0	\\
2.42e-09	0	\\
2.52e-09	0	\\
2.63e-09	0	\\
2.73e-09	0	\\
2.83e-09	0	\\
2.93e-09	0	\\
3.04e-09	0	\\
3.14e-09	0	\\
3.24e-09	0	\\
3.34e-09	0	\\
3.45e-09	0	\\
3.55e-09	0	\\
3.65e-09	0	\\
3.75e-09	0	\\
3.86e-09	0	\\
3.96e-09	0	\\
4.06e-09	0	\\
4.16e-09	0	\\
4.27e-09	0	\\
4.37e-09	0	\\
4.47e-09	0	\\
4.57e-09	0	\\
4.68e-09	0	\\
4.78e-09	0	\\
4.89e-09	0	\\
4.99e-09	0	\\
5e-09	0	\\
};
\addplot [color=mycolor1,solid,forget plot]
  table[row sep=crcr]{
0	0	\\
1.1e-10	0	\\
2.2e-10	0	\\
3.3e-10	0	\\
4.4e-10	0	\\
5.4e-10	0	\\
6.5e-10	0	\\
7.5e-10	0	\\
8.6e-10	0	\\
9.6e-10	0	\\
1.07e-09	0	\\
1.18e-09	0	\\
1.28e-09	0	\\
1.38e-09	0	\\
1.49e-09	0	\\
1.59e-09	0	\\
1.69e-09	0	\\
1.8e-09	0	\\
1.9e-09	0	\\
2.01e-09	0	\\
2.11e-09	0	\\
2.21e-09	0	\\
2.32e-09	0	\\
2.42e-09	0	\\
2.52e-09	0	\\
2.63e-09	0	\\
2.73e-09	0	\\
2.83e-09	0	\\
2.93e-09	0	\\
3.04e-09	0	\\
3.14e-09	0	\\
3.24e-09	0	\\
3.34e-09	0	\\
3.45e-09	0	\\
3.55e-09	0	\\
3.65e-09	0	\\
3.75e-09	0	\\
3.86e-09	0	\\
3.96e-09	0	\\
4.06e-09	0	\\
4.16e-09	0	\\
4.27e-09	0	\\
4.37e-09	0	\\
4.47e-09	0	\\
4.57e-09	0	\\
4.68e-09	0	\\
4.78e-09	0	\\
4.89e-09	0	\\
4.99e-09	0	\\
5e-09	0	\\
};
\addplot [color=mycolor2,solid,forget plot]
  table[row sep=crcr]{
0	0	\\
1.1e-10	0	\\
2.2e-10	0	\\
3.3e-10	0	\\
4.4e-10	0	\\
5.4e-10	0	\\
6.5e-10	0	\\
7.5e-10	0	\\
8.6e-10	0	\\
9.6e-10	0	\\
1.07e-09	0	\\
1.18e-09	0	\\
1.28e-09	0	\\
1.38e-09	0	\\
1.49e-09	0	\\
1.59e-09	0	\\
1.69e-09	0	\\
1.8e-09	0	\\
1.9e-09	0	\\
2.01e-09	0	\\
2.11e-09	0	\\
2.21e-09	0	\\
2.32e-09	0	\\
2.42e-09	0	\\
2.52e-09	0	\\
2.63e-09	0	\\
2.73e-09	0	\\
2.83e-09	0	\\
2.93e-09	0	\\
3.04e-09	0	\\
3.14e-09	0	\\
3.24e-09	0	\\
3.34e-09	0	\\
3.45e-09	0	\\
3.55e-09	0	\\
3.65e-09	0	\\
3.75e-09	0	\\
3.86e-09	0	\\
3.96e-09	0	\\
4.06e-09	0	\\
4.16e-09	0	\\
4.27e-09	0	\\
4.37e-09	0	\\
4.47e-09	0	\\
4.57e-09	0	\\
4.68e-09	0	\\
4.78e-09	0	\\
4.89e-09	0	\\
4.99e-09	0	\\
5e-09	0	\\
};
\addplot [color=mycolor3,solid,forget plot]
  table[row sep=crcr]{
0	0	\\
1.1e-10	0	\\
2.2e-10	0	\\
3.3e-10	0	\\
4.4e-10	0	\\
5.4e-10	0	\\
6.5e-10	0	\\
7.5e-10	0	\\
8.6e-10	0	\\
9.6e-10	0	\\
1.07e-09	0	\\
1.18e-09	0	\\
1.28e-09	0	\\
1.38e-09	0	\\
1.49e-09	0	\\
1.59e-09	0	\\
1.69e-09	0	\\
1.8e-09	0	\\
1.9e-09	0	\\
2.01e-09	0	\\
2.11e-09	0	\\
2.21e-09	0	\\
2.32e-09	0	\\
2.42e-09	0	\\
2.52e-09	0	\\
2.63e-09	0	\\
2.73e-09	0	\\
2.83e-09	0	\\
2.93e-09	0	\\
3.04e-09	0	\\
3.14e-09	0	\\
3.24e-09	0	\\
3.34e-09	0	\\
3.45e-09	0	\\
3.55e-09	0	\\
3.65e-09	0	\\
3.75e-09	0	\\
3.86e-09	0	\\
3.96e-09	0	\\
4.06e-09	0	\\
4.16e-09	0	\\
4.27e-09	0	\\
4.37e-09	0	\\
4.47e-09	0	\\
4.57e-09	0	\\
4.68e-09	0	\\
4.78e-09	0	\\
4.89e-09	0	\\
4.99e-09	0	\\
5e-09	0	\\
};
\addplot [color=darkgray,solid,forget plot]
  table[row sep=crcr]{
0	0	\\
1.1e-10	0	\\
2.2e-10	0	\\
3.3e-10	0	\\
4.4e-10	0	\\
5.4e-10	0	\\
6.5e-10	0	\\
7.5e-10	0	\\
8.6e-10	0	\\
9.6e-10	0	\\
1.07e-09	0	\\
1.18e-09	0	\\
1.28e-09	0	\\
1.38e-09	0	\\
1.49e-09	0	\\
1.59e-09	0	\\
1.69e-09	0	\\
1.8e-09	0	\\
1.9e-09	0	\\
2.01e-09	0	\\
2.11e-09	0	\\
2.21e-09	0	\\
2.32e-09	0	\\
2.42e-09	0	\\
2.52e-09	0	\\
2.63e-09	0	\\
2.73e-09	0	\\
2.83e-09	0	\\
2.93e-09	0	\\
3.04e-09	0	\\
3.14e-09	0	\\
3.24e-09	0	\\
3.34e-09	0	\\
3.45e-09	0	\\
3.55e-09	0	\\
3.65e-09	0	\\
3.75e-09	0	\\
3.86e-09	0	\\
3.96e-09	0	\\
4.06e-09	0	\\
4.16e-09	0	\\
4.27e-09	0	\\
4.37e-09	0	\\
4.47e-09	0	\\
4.57e-09	0	\\
4.68e-09	0	\\
4.78e-09	0	\\
4.89e-09	0	\\
4.99e-09	0	\\
5e-09	0	\\
};
\addplot [color=blue,solid,forget plot]
  table[row sep=crcr]{
0	0	\\
1.1e-10	0	\\
2.2e-10	0	\\
3.3e-10	0	\\
4.4e-10	0	\\
5.4e-10	0	\\
6.5e-10	0	\\
7.5e-10	0	\\
8.6e-10	0	\\
9.6e-10	0	\\
1.07e-09	0	\\
1.18e-09	0	\\
1.28e-09	0	\\
1.38e-09	0	\\
1.49e-09	0	\\
1.59e-09	0	\\
1.69e-09	0	\\
1.8e-09	0	\\
1.9e-09	0	\\
2.01e-09	0	\\
2.11e-09	0	\\
2.21e-09	0	\\
2.32e-09	0	\\
2.42e-09	0	\\
2.52e-09	0	\\
2.63e-09	0	\\
2.73e-09	0	\\
2.83e-09	0	\\
2.93e-09	0	\\
3.04e-09	0	\\
3.14e-09	0	\\
3.24e-09	0	\\
3.34e-09	0	\\
3.45e-09	0	\\
3.55e-09	0	\\
3.65e-09	0	\\
3.75e-09	0	\\
3.86e-09	0	\\
3.96e-09	0	\\
4.06e-09	0	\\
4.16e-09	0	\\
4.27e-09	0	\\
4.37e-09	0	\\
4.47e-09	0	\\
4.57e-09	0	\\
4.68e-09	0	\\
4.78e-09	0	\\
4.89e-09	0	\\
4.99e-09	0	\\
5e-09	0	\\
};
\addplot [color=black!50!green,solid,forget plot]
  table[row sep=crcr]{
0	0	\\
1.1e-10	0	\\
2.2e-10	0	\\
3.3e-10	0	\\
4.4e-10	0	\\
5.4e-10	0	\\
6.5e-10	0	\\
7.5e-10	0	\\
8.6e-10	0	\\
9.6e-10	0	\\
1.07e-09	0	\\
1.18e-09	0	\\
1.28e-09	0	\\
1.38e-09	0	\\
1.49e-09	0	\\
1.59e-09	0	\\
1.69e-09	0	\\
1.8e-09	0	\\
1.9e-09	0	\\
2.01e-09	0	\\
2.11e-09	0	\\
2.21e-09	0	\\
2.32e-09	0	\\
2.42e-09	0	\\
2.52e-09	0	\\
2.63e-09	0	\\
2.73e-09	0	\\
2.83e-09	0	\\
2.93e-09	0	\\
3.04e-09	0	\\
3.14e-09	0	\\
3.24e-09	0	\\
3.34e-09	0	\\
3.45e-09	0	\\
3.55e-09	0	\\
3.65e-09	0	\\
3.75e-09	0	\\
3.86e-09	0	\\
3.96e-09	0	\\
4.06e-09	0	\\
4.16e-09	0	\\
4.27e-09	0	\\
4.37e-09	0	\\
4.47e-09	0	\\
4.57e-09	0	\\
4.68e-09	0	\\
4.78e-09	0	\\
4.89e-09	0	\\
4.99e-09	0	\\
5e-09	0	\\
};
\addplot [color=red,solid,forget plot]
  table[row sep=crcr]{
0	0	\\
1.1e-10	0	\\
2.2e-10	0	\\
3.3e-10	0	\\
4.4e-10	0	\\
5.4e-10	0	\\
6.5e-10	0	\\
7.5e-10	0	\\
8.6e-10	0	\\
9.6e-10	0	\\
1.07e-09	0	\\
1.18e-09	0	\\
1.28e-09	0	\\
1.38e-09	0	\\
1.49e-09	0	\\
1.59e-09	0	\\
1.69e-09	0	\\
1.8e-09	0	\\
1.9e-09	0	\\
2.01e-09	0	\\
2.11e-09	0	\\
2.21e-09	0	\\
2.32e-09	0	\\
2.42e-09	0	\\
2.52e-09	0	\\
2.63e-09	0	\\
2.73e-09	0	\\
2.83e-09	0	\\
2.93e-09	0	\\
3.04e-09	0	\\
3.14e-09	0	\\
3.24e-09	0	\\
3.34e-09	0	\\
3.45e-09	0	\\
3.55e-09	0	\\
3.65e-09	0	\\
3.75e-09	0	\\
3.86e-09	0	\\
3.96e-09	0	\\
4.06e-09	0	\\
4.16e-09	0	\\
4.27e-09	0	\\
4.37e-09	0	\\
4.47e-09	0	\\
4.57e-09	0	\\
4.68e-09	0	\\
4.78e-09	0	\\
4.89e-09	0	\\
4.99e-09	0	\\
5e-09	0	\\
};
\addplot [color=mycolor1,solid,forget plot]
  table[row sep=crcr]{
0	0	\\
1.1e-10	0	\\
2.2e-10	0	\\
3.3e-10	0	\\
4.4e-10	0	\\
5.4e-10	0	\\
6.5e-10	0	\\
7.5e-10	0	\\
8.6e-10	0	\\
9.6e-10	0	\\
1.07e-09	0	\\
1.18e-09	0	\\
1.28e-09	0	\\
1.38e-09	0	\\
1.49e-09	0	\\
1.59e-09	0	\\
1.69e-09	0	\\
1.8e-09	0	\\
1.9e-09	0	\\
2.01e-09	0	\\
2.11e-09	0	\\
2.21e-09	0	\\
2.32e-09	0	\\
2.42e-09	0	\\
2.52e-09	0	\\
2.63e-09	0	\\
2.73e-09	0	\\
2.83e-09	0	\\
2.93e-09	0	\\
3.04e-09	0	\\
3.14e-09	0	\\
3.24e-09	0	\\
3.34e-09	0	\\
3.45e-09	0	\\
3.55e-09	0	\\
3.65e-09	0	\\
3.75e-09	0	\\
3.86e-09	0	\\
3.96e-09	0	\\
4.06e-09	0	\\
4.16e-09	0	\\
4.27e-09	0	\\
4.37e-09	0	\\
4.47e-09	0	\\
4.57e-09	0	\\
4.68e-09	0	\\
4.78e-09	0	\\
4.89e-09	0	\\
4.99e-09	0	\\
5e-09	0	\\
};
\addplot [color=mycolor2,solid,forget plot]
  table[row sep=crcr]{
0	0	\\
1.1e-10	0	\\
2.2e-10	0	\\
3.3e-10	0	\\
4.4e-10	0	\\
5.4e-10	0	\\
6.5e-10	0	\\
7.5e-10	0	\\
8.6e-10	0	\\
9.6e-10	0	\\
1.07e-09	0	\\
1.18e-09	0	\\
1.28e-09	0	\\
1.38e-09	0	\\
1.49e-09	0	\\
1.59e-09	0	\\
1.69e-09	0	\\
1.8e-09	0	\\
1.9e-09	0	\\
2.01e-09	0	\\
2.11e-09	0	\\
2.21e-09	0	\\
2.32e-09	0	\\
2.42e-09	0	\\
2.52e-09	0	\\
2.63e-09	0	\\
2.73e-09	0	\\
2.83e-09	0	\\
2.93e-09	0	\\
3.04e-09	0	\\
3.14e-09	0	\\
3.24e-09	0	\\
3.34e-09	0	\\
3.45e-09	0	\\
3.55e-09	0	\\
3.65e-09	0	\\
3.75e-09	0	\\
3.86e-09	0	\\
3.96e-09	0	\\
4.06e-09	0	\\
4.16e-09	0	\\
4.27e-09	0	\\
4.37e-09	0	\\
4.47e-09	0	\\
4.57e-09	0	\\
4.68e-09	0	\\
4.78e-09	0	\\
4.89e-09	0	\\
4.99e-09	0	\\
5e-09	0	\\
};
\addplot [color=mycolor3,solid,forget plot]
  table[row sep=crcr]{
0	0	\\
1.1e-10	0	\\
2.2e-10	0	\\
3.3e-10	0	\\
4.4e-10	0	\\
5.4e-10	0	\\
6.5e-10	0	\\
7.5e-10	0	\\
8.6e-10	0	\\
9.6e-10	0	\\
1.07e-09	0	\\
1.18e-09	0	\\
1.28e-09	0	\\
1.38e-09	0	\\
1.49e-09	0	\\
1.59e-09	0	\\
1.69e-09	0	\\
1.8e-09	0	\\
1.9e-09	0	\\
2.01e-09	0	\\
2.11e-09	0	\\
2.21e-09	0	\\
2.32e-09	0	\\
2.42e-09	0	\\
2.52e-09	0	\\
2.63e-09	0	\\
2.73e-09	0	\\
2.83e-09	0	\\
2.93e-09	0	\\
3.04e-09	0	\\
3.14e-09	0	\\
3.24e-09	0	\\
3.34e-09	0	\\
3.45e-09	0	\\
3.55e-09	0	\\
3.65e-09	0	\\
3.75e-09	0	\\
3.86e-09	0	\\
3.96e-09	0	\\
4.06e-09	0	\\
4.16e-09	0	\\
4.27e-09	0	\\
4.37e-09	0	\\
4.47e-09	0	\\
4.57e-09	0	\\
4.68e-09	0	\\
4.78e-09	0	\\
4.89e-09	0	\\
4.99e-09	0	\\
5e-09	0	\\
};
\addplot [color=darkgray,solid,forget plot]
  table[row sep=crcr]{
0	0	\\
1.1e-10	0	\\
2.2e-10	0	\\
3.3e-10	0	\\
4.4e-10	0	\\
5.4e-10	0	\\
6.5e-10	0	\\
7.5e-10	0	\\
8.6e-10	0	\\
9.6e-10	0	\\
1.07e-09	0	\\
1.18e-09	0	\\
1.28e-09	0	\\
1.38e-09	0	\\
1.49e-09	0	\\
1.59e-09	0	\\
1.69e-09	0	\\
1.8e-09	0	\\
1.9e-09	0	\\
2.01e-09	0	\\
2.11e-09	0	\\
2.21e-09	0	\\
2.32e-09	0	\\
2.42e-09	0	\\
2.52e-09	0	\\
2.63e-09	0	\\
2.73e-09	0	\\
2.83e-09	0	\\
2.93e-09	0	\\
3.04e-09	0	\\
3.14e-09	0	\\
3.24e-09	0	\\
3.34e-09	0	\\
3.45e-09	0	\\
3.55e-09	0	\\
3.65e-09	0	\\
3.75e-09	0	\\
3.86e-09	0	\\
3.96e-09	0	\\
4.06e-09	0	\\
4.16e-09	0	\\
4.27e-09	0	\\
4.37e-09	0	\\
4.47e-09	0	\\
4.57e-09	0	\\
4.68e-09	0	\\
4.78e-09	0	\\
4.89e-09	0	\\
4.99e-09	0	\\
5e-09	0	\\
};
\addplot [color=blue,solid,forget plot]
  table[row sep=crcr]{
0	0	\\
1.1e-10	0	\\
2.2e-10	0	\\
3.3e-10	0	\\
4.4e-10	0	\\
5.4e-10	0	\\
6.5e-10	0	\\
7.5e-10	0	\\
8.6e-10	0	\\
9.6e-10	0	\\
1.07e-09	0	\\
1.18e-09	0	\\
1.28e-09	0	\\
1.38e-09	0	\\
1.49e-09	0	\\
1.59e-09	0	\\
1.69e-09	0	\\
1.8e-09	0	\\
1.9e-09	0	\\
2.01e-09	0	\\
2.11e-09	0	\\
2.21e-09	0	\\
2.32e-09	0	\\
2.42e-09	0	\\
2.52e-09	0	\\
2.63e-09	0	\\
2.73e-09	0	\\
2.83e-09	0	\\
2.93e-09	0	\\
3.04e-09	0	\\
3.14e-09	0	\\
3.24e-09	0	\\
3.34e-09	0	\\
3.45e-09	0	\\
3.55e-09	0	\\
3.65e-09	0	\\
3.75e-09	0	\\
3.86e-09	0	\\
3.96e-09	0	\\
4.06e-09	0	\\
4.16e-09	0	\\
4.27e-09	0	\\
4.37e-09	0	\\
4.47e-09	0	\\
4.57e-09	0	\\
4.68e-09	0	\\
4.78e-09	0	\\
4.89e-09	0	\\
4.99e-09	0	\\
5e-09	0	\\
};
\addplot [color=black!50!green,solid,forget plot]
  table[row sep=crcr]{
0	0	\\
1.1e-10	0	\\
2.2e-10	0	\\
3.3e-10	0	\\
4.4e-10	0	\\
5.4e-10	0	\\
6.5e-10	0	\\
7.5e-10	0	\\
8.6e-10	0	\\
9.6e-10	0	\\
1.07e-09	0	\\
1.18e-09	0	\\
1.28e-09	0	\\
1.38e-09	0	\\
1.49e-09	0	\\
1.59e-09	0	\\
1.69e-09	0	\\
1.8e-09	0	\\
1.9e-09	0	\\
2.01e-09	0	\\
2.11e-09	0	\\
2.21e-09	0	\\
2.32e-09	0	\\
2.42e-09	0	\\
2.52e-09	0	\\
2.63e-09	0	\\
2.73e-09	0	\\
2.83e-09	0	\\
2.93e-09	0	\\
3.04e-09	0	\\
3.14e-09	0	\\
3.24e-09	0	\\
3.34e-09	0	\\
3.45e-09	0	\\
3.55e-09	0	\\
3.65e-09	0	\\
3.75e-09	0	\\
3.86e-09	0	\\
3.96e-09	0	\\
4.06e-09	0	\\
4.16e-09	0	\\
4.27e-09	0	\\
4.37e-09	0	\\
4.47e-09	0	\\
4.57e-09	0	\\
4.68e-09	0	\\
4.78e-09	0	\\
4.89e-09	0	\\
4.99e-09	0	\\
5e-09	0	\\
};
\addplot [color=red,solid,forget plot]
  table[row sep=crcr]{
0	0	\\
1.1e-10	0	\\
2.2e-10	0	\\
3.3e-10	0	\\
4.4e-10	0	\\
5.4e-10	0	\\
6.5e-10	0	\\
7.5e-10	0	\\
8.6e-10	0	\\
9.6e-10	0	\\
1.07e-09	0	\\
1.18e-09	0	\\
1.28e-09	0	\\
1.38e-09	0	\\
1.49e-09	0	\\
1.59e-09	0	\\
1.69e-09	0	\\
1.8e-09	0	\\
1.9e-09	0	\\
2.01e-09	0	\\
2.11e-09	0	\\
2.21e-09	0	\\
2.32e-09	0	\\
2.42e-09	0	\\
2.52e-09	0	\\
2.63e-09	0	\\
2.73e-09	0	\\
2.83e-09	0	\\
2.93e-09	0	\\
3.04e-09	0	\\
3.14e-09	0	\\
3.24e-09	0	\\
3.34e-09	0	\\
3.45e-09	0	\\
3.55e-09	0	\\
3.65e-09	0	\\
3.75e-09	0	\\
3.86e-09	0	\\
3.96e-09	0	\\
4.06e-09	0	\\
4.16e-09	0	\\
4.27e-09	0	\\
4.37e-09	0	\\
4.47e-09	0	\\
4.57e-09	0	\\
4.68e-09	0	\\
4.78e-09	0	\\
4.89e-09	0	\\
4.99e-09	0	\\
5e-09	0	\\
};
\addplot [color=mycolor1,solid,forget plot]
  table[row sep=crcr]{
0	0	\\
1.1e-10	0	\\
2.2e-10	0	\\
3.3e-10	0	\\
4.4e-10	0	\\
5.4e-10	0	\\
6.5e-10	0	\\
7.5e-10	0	\\
8.6e-10	0	\\
9.6e-10	0	\\
1.07e-09	0	\\
1.18e-09	0	\\
1.28e-09	0	\\
1.38e-09	0	\\
1.49e-09	0	\\
1.59e-09	0	\\
1.69e-09	0	\\
1.8e-09	0	\\
1.9e-09	0	\\
2.01e-09	0	\\
2.11e-09	0	\\
2.21e-09	0	\\
2.32e-09	0	\\
2.42e-09	0	\\
2.52e-09	0	\\
2.63e-09	0	\\
2.73e-09	0	\\
2.83e-09	0	\\
2.93e-09	0	\\
3.04e-09	0	\\
3.14e-09	0	\\
3.24e-09	0	\\
3.34e-09	0	\\
3.45e-09	0	\\
3.55e-09	0	\\
3.65e-09	0	\\
3.75e-09	0	\\
3.86e-09	0	\\
3.96e-09	0	\\
4.06e-09	0	\\
4.16e-09	0	\\
4.27e-09	0	\\
4.37e-09	0	\\
4.47e-09	0	\\
4.57e-09	0	\\
4.68e-09	0	\\
4.78e-09	0	\\
4.89e-09	0	\\
4.99e-09	0	\\
5e-09	0	\\
};
\addplot [color=mycolor2,solid,forget plot]
  table[row sep=crcr]{
0	0	\\
1.1e-10	0	\\
2.2e-10	0	\\
3.3e-10	0	\\
4.4e-10	0	\\
5.4e-10	0	\\
6.5e-10	0	\\
7.5e-10	0	\\
8.6e-10	0	\\
9.6e-10	0	\\
1.07e-09	0	\\
1.18e-09	0	\\
1.28e-09	0	\\
1.38e-09	0	\\
1.49e-09	0	\\
1.59e-09	0	\\
1.69e-09	0	\\
1.8e-09	0	\\
1.9e-09	0	\\
2.01e-09	0	\\
2.11e-09	0	\\
2.21e-09	0	\\
2.32e-09	0	\\
2.42e-09	0	\\
2.52e-09	0	\\
2.63e-09	0	\\
2.73e-09	0	\\
2.83e-09	0	\\
2.93e-09	0	\\
3.04e-09	0	\\
3.14e-09	0	\\
3.24e-09	0	\\
3.34e-09	0	\\
3.45e-09	0	\\
3.55e-09	0	\\
3.65e-09	0	\\
3.75e-09	0	\\
3.86e-09	0	\\
3.96e-09	0	\\
4.06e-09	0	\\
4.16e-09	0	\\
4.27e-09	0	\\
4.37e-09	0	\\
4.47e-09	0	\\
4.57e-09	0	\\
4.68e-09	0	\\
4.78e-09	0	\\
4.89e-09	0	\\
4.99e-09	0	\\
5e-09	0	\\
};
\addplot [color=mycolor3,solid,forget plot]
  table[row sep=crcr]{
0	0	\\
1.1e-10	0	\\
2.2e-10	0	\\
3.3e-10	0	\\
4.4e-10	0	\\
5.4e-10	0	\\
6.5e-10	0	\\
7.5e-10	0	\\
8.6e-10	0	\\
9.6e-10	0	\\
1.07e-09	0	\\
1.18e-09	0	\\
1.28e-09	0	\\
1.38e-09	0	\\
1.49e-09	0	\\
1.59e-09	0	\\
1.69e-09	0	\\
1.8e-09	0	\\
1.9e-09	0	\\
2.01e-09	0	\\
2.11e-09	0	\\
2.21e-09	0	\\
2.32e-09	0	\\
2.42e-09	0	\\
2.52e-09	0	\\
2.63e-09	0	\\
2.73e-09	0	\\
2.83e-09	0	\\
2.93e-09	0	\\
3.04e-09	0	\\
3.14e-09	0	\\
3.24e-09	0	\\
3.34e-09	0	\\
3.45e-09	0	\\
3.55e-09	0	\\
3.65e-09	0	\\
3.75e-09	0	\\
3.86e-09	0	\\
3.96e-09	0	\\
4.06e-09	0	\\
4.16e-09	0	\\
4.27e-09	0	\\
4.37e-09	0	\\
4.47e-09	0	\\
4.57e-09	0	\\
4.68e-09	0	\\
4.78e-09	0	\\
4.89e-09	0	\\
4.99e-09	0	\\
5e-09	0	\\
};
\addplot [color=darkgray,solid,forget plot]
  table[row sep=crcr]{
0	0	\\
1.1e-10	0	\\
2.2e-10	0	\\
3.3e-10	0	\\
4.4e-10	0	\\
5.4e-10	0	\\
6.5e-10	0	\\
7.5e-10	0	\\
8.6e-10	0	\\
9.6e-10	0	\\
1.07e-09	0	\\
1.18e-09	0	\\
1.28e-09	0	\\
1.38e-09	0	\\
1.49e-09	0	\\
1.59e-09	0	\\
1.69e-09	0	\\
1.8e-09	0	\\
1.9e-09	0	\\
2.01e-09	0	\\
2.11e-09	0	\\
2.21e-09	0	\\
2.32e-09	0	\\
2.42e-09	0	\\
2.52e-09	0	\\
2.63e-09	0	\\
2.73e-09	0	\\
2.83e-09	0	\\
2.93e-09	0	\\
3.04e-09	0	\\
3.14e-09	0	\\
3.24e-09	0	\\
3.34e-09	0	\\
3.45e-09	0	\\
3.55e-09	0	\\
3.65e-09	0	\\
3.75e-09	0	\\
3.86e-09	0	\\
3.96e-09	0	\\
4.06e-09	0	\\
4.16e-09	0	\\
4.27e-09	0	\\
4.37e-09	0	\\
4.47e-09	0	\\
4.57e-09	0	\\
4.68e-09	0	\\
4.78e-09	0	\\
4.89e-09	0	\\
4.99e-09	0	\\
5e-09	0	\\
};
\addplot [color=blue,solid,forget plot]
  table[row sep=crcr]{
0	0	\\
1.1e-10	0	\\
2.2e-10	0	\\
3.3e-10	0	\\
4.4e-10	0	\\
5.4e-10	0	\\
6.5e-10	0	\\
7.5e-10	0	\\
8.6e-10	0	\\
9.6e-10	0	\\
1.07e-09	0	\\
1.18e-09	0	\\
1.28e-09	0	\\
1.38e-09	0	\\
1.49e-09	0	\\
1.59e-09	0	\\
1.69e-09	0	\\
1.8e-09	0	\\
1.9e-09	0	\\
2.01e-09	0	\\
2.11e-09	0	\\
2.21e-09	0	\\
2.32e-09	0	\\
2.42e-09	0	\\
2.52e-09	0	\\
2.63e-09	0	\\
2.73e-09	0	\\
2.83e-09	0	\\
2.93e-09	0	\\
3.04e-09	0	\\
3.14e-09	0	\\
3.24e-09	0	\\
3.34e-09	0	\\
3.45e-09	0	\\
3.55e-09	0	\\
3.65e-09	0	\\
3.75e-09	0	\\
3.86e-09	0	\\
3.96e-09	0	\\
4.06e-09	0	\\
4.16e-09	0	\\
4.27e-09	0	\\
4.37e-09	0	\\
4.47e-09	0	\\
4.57e-09	0	\\
4.68e-09	0	\\
4.78e-09	0	\\
4.89e-09	0	\\
4.99e-09	0	\\
5e-09	0	\\
};
\addplot [color=black!50!green,solid,forget plot]
  table[row sep=crcr]{
0	0	\\
1.1e-10	0	\\
2.2e-10	0	\\
3.3e-10	0	\\
4.4e-10	0	\\
5.4e-10	0	\\
6.5e-10	0	\\
7.5e-10	0	\\
8.6e-10	0	\\
9.6e-10	0	\\
1.07e-09	0	\\
1.18e-09	0	\\
1.28e-09	0	\\
1.38e-09	0	\\
1.49e-09	0	\\
1.59e-09	0	\\
1.69e-09	0	\\
1.8e-09	0	\\
1.9e-09	0	\\
2.01e-09	0	\\
2.11e-09	0	\\
2.21e-09	0	\\
2.32e-09	0	\\
2.42e-09	0	\\
2.52e-09	0	\\
2.63e-09	0	\\
2.73e-09	0	\\
2.83e-09	0	\\
2.93e-09	0	\\
3.04e-09	0	\\
3.14e-09	0	\\
3.24e-09	0	\\
3.34e-09	0	\\
3.45e-09	0	\\
3.55e-09	0	\\
3.65e-09	0	\\
3.75e-09	0	\\
3.86e-09	0	\\
3.96e-09	0	\\
4.06e-09	0	\\
4.16e-09	0	\\
4.27e-09	0	\\
4.37e-09	0	\\
4.47e-09	0	\\
4.57e-09	0	\\
4.68e-09	0	\\
4.78e-09	0	\\
4.89e-09	0	\\
4.99e-09	0	\\
5e-09	0	\\
};
\addplot [color=red,solid,forget plot]
  table[row sep=crcr]{
0	0	\\
1.1e-10	0	\\
2.2e-10	0	\\
3.3e-10	0	\\
4.4e-10	0	\\
5.4e-10	0	\\
6.5e-10	0	\\
7.5e-10	0	\\
8.6e-10	0	\\
9.6e-10	0	\\
1.07e-09	0	\\
1.18e-09	0	\\
1.28e-09	0	\\
1.38e-09	0	\\
1.49e-09	0	\\
1.59e-09	0	\\
1.69e-09	0	\\
1.8e-09	0	\\
1.9e-09	0	\\
2.01e-09	0	\\
2.11e-09	0	\\
2.21e-09	0	\\
2.32e-09	0	\\
2.42e-09	0	\\
2.52e-09	0	\\
2.63e-09	0	\\
2.73e-09	0	\\
2.83e-09	0	\\
2.93e-09	0	\\
3.04e-09	0	\\
3.14e-09	0	\\
3.24e-09	0	\\
3.34e-09	0	\\
3.45e-09	0	\\
3.55e-09	0	\\
3.65e-09	0	\\
3.75e-09	0	\\
3.86e-09	0	\\
3.96e-09	0	\\
4.06e-09	0	\\
4.16e-09	0	\\
4.27e-09	0	\\
4.37e-09	0	\\
4.47e-09	0	\\
4.57e-09	0	\\
4.68e-09	0	\\
4.78e-09	0	\\
4.89e-09	0	\\
4.99e-09	0	\\
5e-09	0	\\
};
\addplot [color=mycolor1,solid,forget plot]
  table[row sep=crcr]{
0	0	\\
1.1e-10	0	\\
2.2e-10	0	\\
3.3e-10	0	\\
4.4e-10	0	\\
5.4e-10	0	\\
6.5e-10	0	\\
7.5e-10	0	\\
8.6e-10	0	\\
9.6e-10	0	\\
1.07e-09	0	\\
1.18e-09	0	\\
1.28e-09	0	\\
1.38e-09	0	\\
1.49e-09	0	\\
1.59e-09	0	\\
1.69e-09	0	\\
1.8e-09	0	\\
1.9e-09	0	\\
2.01e-09	0	\\
2.11e-09	0	\\
2.21e-09	0	\\
2.32e-09	0	\\
2.42e-09	0	\\
2.52e-09	0	\\
2.63e-09	0	\\
2.73e-09	0	\\
2.83e-09	0	\\
2.93e-09	0	\\
3.04e-09	0	\\
3.14e-09	0	\\
3.24e-09	0	\\
3.34e-09	0	\\
3.45e-09	0	\\
3.55e-09	0	\\
3.65e-09	0	\\
3.75e-09	0	\\
3.86e-09	0	\\
3.96e-09	0	\\
4.06e-09	0	\\
4.16e-09	0	\\
4.27e-09	0	\\
4.37e-09	0	\\
4.47e-09	0	\\
4.57e-09	0	\\
4.68e-09	0	\\
4.78e-09	0	\\
4.89e-09	0	\\
4.99e-09	0	\\
5e-09	0	\\
};
\addplot [color=mycolor2,solid,forget plot]
  table[row sep=crcr]{
0	0	\\
1.1e-10	0	\\
2.2e-10	0	\\
3.3e-10	0	\\
4.4e-10	0	\\
5.4e-10	0	\\
6.5e-10	0	\\
7.5e-10	0	\\
8.6e-10	0	\\
9.6e-10	0	\\
1.07e-09	0	\\
1.18e-09	0	\\
1.28e-09	0	\\
1.38e-09	0	\\
1.49e-09	0	\\
1.59e-09	0	\\
1.69e-09	0	\\
1.8e-09	0	\\
1.9e-09	0	\\
2.01e-09	0	\\
2.11e-09	0	\\
2.21e-09	0	\\
2.32e-09	0	\\
2.42e-09	0	\\
2.52e-09	0	\\
2.63e-09	0	\\
2.73e-09	0	\\
2.83e-09	0	\\
2.93e-09	0	\\
3.04e-09	0	\\
3.14e-09	0	\\
3.24e-09	0	\\
3.34e-09	0	\\
3.45e-09	0	\\
3.55e-09	0	\\
3.65e-09	0	\\
3.75e-09	0	\\
3.86e-09	0	\\
3.96e-09	0	\\
4.06e-09	0	\\
4.16e-09	0	\\
4.27e-09	0	\\
4.37e-09	0	\\
4.47e-09	0	\\
4.57e-09	0	\\
4.68e-09	0	\\
4.78e-09	0	\\
4.89e-09	0	\\
4.99e-09	0	\\
5e-09	0	\\
};
\addplot [color=mycolor3,solid,forget plot]
  table[row sep=crcr]{
0	0	\\
1.1e-10	0	\\
2.2e-10	0	\\
3.3e-10	0	\\
4.4e-10	0	\\
5.4e-10	0	\\
6.5e-10	0	\\
7.5e-10	0	\\
8.6e-10	0	\\
9.6e-10	0	\\
1.07e-09	0	\\
1.18e-09	0	\\
1.28e-09	0	\\
1.38e-09	0	\\
1.49e-09	0	\\
1.59e-09	0	\\
1.69e-09	0	\\
1.8e-09	0	\\
1.9e-09	0	\\
2.01e-09	0	\\
2.11e-09	0	\\
2.21e-09	0	\\
2.32e-09	0	\\
2.42e-09	0	\\
2.52e-09	0	\\
2.63e-09	0	\\
2.73e-09	0	\\
2.83e-09	0	\\
2.93e-09	0	\\
3.04e-09	0	\\
3.14e-09	0	\\
3.24e-09	0	\\
3.34e-09	0	\\
3.45e-09	0	\\
3.55e-09	0	\\
3.65e-09	0	\\
3.75e-09	0	\\
3.86e-09	0	\\
3.96e-09	0	\\
4.06e-09	0	\\
4.16e-09	0	\\
4.27e-09	0	\\
4.37e-09	0	\\
4.47e-09	0	\\
4.57e-09	0	\\
4.68e-09	0	\\
4.78e-09	0	\\
4.89e-09	0	\\
4.99e-09	0	\\
5e-09	0	\\
};
\addplot [color=darkgray,solid,forget plot]
  table[row sep=crcr]{
0	0	\\
1.1e-10	0	\\
2.2e-10	0	\\
3.3e-10	0	\\
4.4e-10	0	\\
5.4e-10	0	\\
6.5e-10	0	\\
7.5e-10	0	\\
8.6e-10	0	\\
9.6e-10	0	\\
1.07e-09	0	\\
1.18e-09	0	\\
1.28e-09	0	\\
1.38e-09	0	\\
1.49e-09	0	\\
1.59e-09	0	\\
1.69e-09	0	\\
1.8e-09	0	\\
1.9e-09	0	\\
2.01e-09	0	\\
2.11e-09	0	\\
2.21e-09	0	\\
2.32e-09	0	\\
2.42e-09	0	\\
2.52e-09	0	\\
2.63e-09	0	\\
2.73e-09	0	\\
2.83e-09	0	\\
2.93e-09	0	\\
3.04e-09	0	\\
3.14e-09	0	\\
3.24e-09	0	\\
3.34e-09	0	\\
3.45e-09	0	\\
3.55e-09	0	\\
3.65e-09	0	\\
3.75e-09	0	\\
3.86e-09	0	\\
3.96e-09	0	\\
4.06e-09	0	\\
4.16e-09	0	\\
4.27e-09	0	\\
4.37e-09	0	\\
4.47e-09	0	\\
4.57e-09	0	\\
4.68e-09	0	\\
4.78e-09	0	\\
4.89e-09	0	\\
4.99e-09	0	\\
5e-09	0	\\
};
\addplot [color=blue,solid,forget plot]
  table[row sep=crcr]{
0	0	\\
1.1e-10	0	\\
2.2e-10	0	\\
3.3e-10	0	\\
4.4e-10	0	\\
5.4e-10	0	\\
6.5e-10	0	\\
7.5e-10	0	\\
8.6e-10	0	\\
9.6e-10	0	\\
1.07e-09	0	\\
1.18e-09	0	\\
1.28e-09	0	\\
1.38e-09	0	\\
1.49e-09	0	\\
1.59e-09	0	\\
1.69e-09	0	\\
1.8e-09	0	\\
1.9e-09	0	\\
2.01e-09	0	\\
2.11e-09	0	\\
2.21e-09	0	\\
2.32e-09	0	\\
2.42e-09	0	\\
2.52e-09	0	\\
2.63e-09	0	\\
2.73e-09	0	\\
2.83e-09	0	\\
2.93e-09	0	\\
3.04e-09	0	\\
3.14e-09	0	\\
3.24e-09	0	\\
3.34e-09	0	\\
3.45e-09	0	\\
3.55e-09	0	\\
3.65e-09	0	\\
3.75e-09	0	\\
3.86e-09	0	\\
3.96e-09	0	\\
4.06e-09	0	\\
4.16e-09	0	\\
4.27e-09	0	\\
4.37e-09	0	\\
4.47e-09	0	\\
4.57e-09	0	\\
4.68e-09	0	\\
4.78e-09	0	\\
4.89e-09	0	\\
4.99e-09	0	\\
5e-09	0	\\
};
\addplot [color=black!50!green,solid,forget plot]
  table[row sep=crcr]{
0	0	\\
1.1e-10	0	\\
2.2e-10	0	\\
3.3e-10	0	\\
4.4e-10	0	\\
5.4e-10	0	\\
6.5e-10	0	\\
7.5e-10	0	\\
8.6e-10	0	\\
9.6e-10	0	\\
1.07e-09	0	\\
1.18e-09	0	\\
1.28e-09	0	\\
1.38e-09	0	\\
1.49e-09	0	\\
1.59e-09	0	\\
1.69e-09	0	\\
1.8e-09	0	\\
1.9e-09	0	\\
2.01e-09	0	\\
2.11e-09	0	\\
2.21e-09	0	\\
2.32e-09	0	\\
2.42e-09	0	\\
2.52e-09	0	\\
2.63e-09	0	\\
2.73e-09	0	\\
2.83e-09	0	\\
2.93e-09	0	\\
3.04e-09	0	\\
3.14e-09	0	\\
3.24e-09	0	\\
3.34e-09	0	\\
3.45e-09	0	\\
3.55e-09	0	\\
3.65e-09	0	\\
3.75e-09	0	\\
3.86e-09	0	\\
3.96e-09	0	\\
4.06e-09	0	\\
4.16e-09	0	\\
4.27e-09	0	\\
4.37e-09	0	\\
4.47e-09	0	\\
4.57e-09	0	\\
4.68e-09	0	\\
4.78e-09	0	\\
4.89e-09	0	\\
4.99e-09	0	\\
5e-09	0	\\
};
\addplot [color=red,solid,forget plot]
  table[row sep=crcr]{
0	0	\\
1.1e-10	0	\\
2.2e-10	0	\\
3.3e-10	0	\\
4.4e-10	0	\\
5.4e-10	0	\\
6.5e-10	0	\\
7.5e-10	0	\\
8.6e-10	0	\\
9.6e-10	0	\\
1.07e-09	0	\\
1.18e-09	0	\\
1.28e-09	0	\\
1.38e-09	0	\\
1.49e-09	0	\\
1.59e-09	0	\\
1.69e-09	0	\\
1.8e-09	0	\\
1.9e-09	0	\\
2.01e-09	0	\\
2.11e-09	0	\\
2.21e-09	0	\\
2.32e-09	0	\\
2.42e-09	0	\\
2.52e-09	0	\\
2.63e-09	0	\\
2.73e-09	0	\\
2.83e-09	0	\\
2.93e-09	0	\\
3.04e-09	0	\\
3.14e-09	0	\\
3.24e-09	0	\\
3.34e-09	0	\\
3.45e-09	0	\\
3.55e-09	0	\\
3.65e-09	0	\\
3.75e-09	0	\\
3.86e-09	0	\\
3.96e-09	0	\\
4.06e-09	0	\\
4.16e-09	0	\\
4.27e-09	0	\\
4.37e-09	0	\\
4.47e-09	0	\\
4.57e-09	0	\\
4.68e-09	0	\\
4.78e-09	0	\\
4.89e-09	0	\\
4.99e-09	0	\\
5e-09	0	\\
};
\addplot [color=mycolor1,solid,forget plot]
  table[row sep=crcr]{
0	0	\\
1.1e-10	0	\\
2.2e-10	0	\\
3.3e-10	0	\\
4.4e-10	0	\\
5.4e-10	0	\\
6.5e-10	0	\\
7.5e-10	0	\\
8.6e-10	0	\\
9.6e-10	0	\\
1.07e-09	0	\\
1.18e-09	0	\\
1.28e-09	0	\\
1.38e-09	0	\\
1.49e-09	0	\\
1.59e-09	0	\\
1.69e-09	0	\\
1.8e-09	0	\\
1.9e-09	0	\\
2.01e-09	0	\\
2.11e-09	0	\\
2.21e-09	0	\\
2.32e-09	0	\\
2.42e-09	0	\\
2.52e-09	0	\\
2.63e-09	0	\\
2.73e-09	0	\\
2.83e-09	0	\\
2.93e-09	0	\\
3.04e-09	0	\\
3.14e-09	0	\\
3.24e-09	0	\\
3.34e-09	0	\\
3.45e-09	0	\\
3.55e-09	0	\\
3.65e-09	0	\\
3.75e-09	0	\\
3.86e-09	0	\\
3.96e-09	0	\\
4.06e-09	0	\\
4.16e-09	0	\\
4.27e-09	0	\\
4.37e-09	0	\\
4.47e-09	0	\\
4.57e-09	0	\\
4.68e-09	0	\\
4.78e-09	0	\\
4.89e-09	0	\\
4.99e-09	0	\\
5e-09	0	\\
};
\addplot [color=mycolor2,solid,forget plot]
  table[row sep=crcr]{
0	0	\\
1.1e-10	0	\\
2.2e-10	0	\\
3.3e-10	0	\\
4.4e-10	0	\\
5.4e-10	0	\\
6.5e-10	0	\\
7.5e-10	0	\\
8.6e-10	0	\\
9.6e-10	0	\\
1.07e-09	0	\\
1.18e-09	0	\\
1.28e-09	0	\\
1.38e-09	0	\\
1.49e-09	0	\\
1.59e-09	0	\\
1.69e-09	0	\\
1.8e-09	0	\\
1.9e-09	0	\\
2.01e-09	0	\\
2.11e-09	0	\\
2.21e-09	0	\\
2.32e-09	0	\\
2.42e-09	0	\\
2.52e-09	0	\\
2.63e-09	0	\\
2.73e-09	0	\\
2.83e-09	0	\\
2.93e-09	0	\\
3.04e-09	0	\\
3.14e-09	0	\\
3.24e-09	0	\\
3.34e-09	0	\\
3.45e-09	0	\\
3.55e-09	0	\\
3.65e-09	0	\\
3.75e-09	0	\\
3.86e-09	0	\\
3.96e-09	0	\\
4.06e-09	0	\\
4.16e-09	0	\\
4.27e-09	0	\\
4.37e-09	0	\\
4.47e-09	0	\\
4.57e-09	0	\\
4.68e-09	0	\\
4.78e-09	0	\\
4.89e-09	0	\\
4.99e-09	0	\\
5e-09	0	\\
};
\addplot [color=mycolor3,solid,forget plot]
  table[row sep=crcr]{
0	0	\\
1.1e-10	0	\\
2.2e-10	0	\\
3.3e-10	0	\\
4.4e-10	0	\\
5.4e-10	0	\\
6.5e-10	0	\\
7.5e-10	0	\\
8.6e-10	0	\\
9.6e-10	0	\\
1.07e-09	0	\\
1.18e-09	0	\\
1.28e-09	0	\\
1.38e-09	0	\\
1.49e-09	0	\\
1.59e-09	0	\\
1.69e-09	0	\\
1.8e-09	0	\\
1.9e-09	0	\\
2.01e-09	0	\\
2.11e-09	0	\\
2.21e-09	0	\\
2.32e-09	0	\\
2.42e-09	0	\\
2.52e-09	0	\\
2.63e-09	0	\\
2.73e-09	0	\\
2.83e-09	0	\\
2.93e-09	0	\\
3.04e-09	0	\\
3.14e-09	0	\\
3.24e-09	0	\\
3.34e-09	0	\\
3.45e-09	0	\\
3.55e-09	0	\\
3.65e-09	0	\\
3.75e-09	0	\\
3.86e-09	0	\\
3.96e-09	0	\\
4.06e-09	0	\\
4.16e-09	0	\\
4.27e-09	0	\\
4.37e-09	0	\\
4.47e-09	0	\\
4.57e-09	0	\\
4.68e-09	0	\\
4.78e-09	0	\\
4.89e-09	0	\\
4.99e-09	0	\\
5e-09	0	\\
};
\addplot [color=darkgray,solid,forget plot]
  table[row sep=crcr]{
0	0	\\
1.1e-10	0	\\
2.2e-10	0	\\
3.3e-10	0	\\
4.4e-10	0	\\
5.4e-10	0	\\
6.5e-10	0	\\
7.5e-10	0	\\
8.6e-10	0	\\
9.6e-10	0	\\
1.07e-09	0	\\
1.18e-09	0	\\
1.28e-09	0	\\
1.38e-09	0	\\
1.49e-09	0	\\
1.59e-09	0	\\
1.69e-09	0	\\
1.8e-09	0	\\
1.9e-09	0	\\
2.01e-09	0	\\
2.11e-09	0	\\
2.21e-09	0	\\
2.32e-09	0	\\
2.42e-09	0	\\
2.52e-09	0	\\
2.63e-09	0	\\
2.73e-09	0	\\
2.83e-09	0	\\
2.93e-09	0	\\
3.04e-09	0	\\
3.14e-09	0	\\
3.24e-09	0	\\
3.34e-09	0	\\
3.45e-09	0	\\
3.55e-09	0	\\
3.65e-09	0	\\
3.75e-09	0	\\
3.86e-09	0	\\
3.96e-09	0	\\
4.06e-09	0	\\
4.16e-09	0	\\
4.27e-09	0	\\
4.37e-09	0	\\
4.47e-09	0	\\
4.57e-09	0	\\
4.68e-09	0	\\
4.78e-09	0	\\
4.89e-09	0	\\
4.99e-09	0	\\
5e-09	0	\\
};
\addplot [color=blue,solid,forget plot]
  table[row sep=crcr]{
0	0	\\
1.1e-10	0	\\
2.2e-10	0	\\
3.3e-10	0	\\
4.4e-10	0	\\
5.4e-10	0	\\
6.5e-10	0	\\
7.5e-10	0	\\
8.6e-10	0	\\
9.6e-10	0	\\
1.07e-09	0	\\
1.18e-09	0	\\
1.28e-09	0	\\
1.38e-09	0	\\
1.49e-09	0	\\
1.59e-09	0	\\
1.69e-09	0	\\
1.8e-09	0	\\
1.9e-09	0	\\
2.01e-09	0	\\
2.11e-09	0	\\
2.21e-09	0	\\
2.32e-09	0	\\
2.42e-09	0	\\
2.52e-09	0	\\
2.63e-09	0	\\
2.73e-09	0	\\
2.83e-09	0	\\
2.93e-09	0	\\
3.04e-09	0	\\
3.14e-09	0	\\
3.24e-09	0	\\
3.34e-09	0	\\
3.45e-09	0	\\
3.55e-09	0	\\
3.65e-09	0	\\
3.75e-09	0	\\
3.86e-09	0	\\
3.96e-09	0	\\
4.06e-09	0	\\
4.16e-09	0	\\
4.27e-09	0	\\
4.37e-09	0	\\
4.47e-09	0	\\
4.57e-09	0	\\
4.68e-09	0	\\
4.78e-09	0	\\
4.89e-09	0	\\
4.99e-09	0	\\
5e-09	0	\\
};
\addplot [color=black!50!green,solid,forget plot]
  table[row sep=crcr]{
0	0	\\
1.1e-10	0	\\
2.2e-10	0	\\
3.3e-10	0	\\
4.4e-10	0	\\
5.4e-10	0	\\
6.5e-10	0	\\
7.5e-10	0	\\
8.6e-10	0	\\
9.6e-10	0	\\
1.07e-09	0	\\
1.18e-09	0	\\
1.28e-09	0	\\
1.38e-09	0	\\
1.49e-09	0	\\
1.59e-09	0	\\
1.69e-09	0	\\
1.8e-09	0	\\
1.9e-09	0	\\
2.01e-09	0	\\
2.11e-09	0	\\
2.21e-09	0	\\
2.32e-09	0	\\
2.42e-09	0	\\
2.52e-09	0	\\
2.63e-09	0	\\
2.73e-09	0	\\
2.83e-09	0	\\
2.93e-09	0	\\
3.04e-09	0	\\
3.14e-09	0	\\
3.24e-09	0	\\
3.34e-09	0	\\
3.45e-09	0	\\
3.55e-09	0	\\
3.65e-09	0	\\
3.75e-09	0	\\
3.86e-09	0	\\
3.96e-09	0	\\
4.06e-09	0	\\
4.16e-09	0	\\
4.27e-09	0	\\
4.37e-09	0	\\
4.47e-09	0	\\
4.57e-09	0	\\
4.68e-09	0	\\
4.78e-09	0	\\
4.89e-09	0	\\
4.99e-09	0	\\
5e-09	0	\\
};
\addplot [color=red,solid,forget plot]
  table[row sep=crcr]{
0	0	\\
1.1e-10	0	\\
2.2e-10	0	\\
3.3e-10	0	\\
4.4e-10	0	\\
5.4e-10	0	\\
6.5e-10	0	\\
7.5e-10	0	\\
8.6e-10	0	\\
9.6e-10	0	\\
1.07e-09	0	\\
1.18e-09	0	\\
1.28e-09	0	\\
1.38e-09	0	\\
1.49e-09	0	\\
1.59e-09	0	\\
1.69e-09	0	\\
1.8e-09	0	\\
1.9e-09	0	\\
2.01e-09	0	\\
2.11e-09	0	\\
2.21e-09	0	\\
2.32e-09	0	\\
2.42e-09	0	\\
2.52e-09	0	\\
2.63e-09	0	\\
2.73e-09	0	\\
2.83e-09	0	\\
2.93e-09	0	\\
3.04e-09	0	\\
3.14e-09	0	\\
3.24e-09	0	\\
3.34e-09	0	\\
3.45e-09	0	\\
3.55e-09	0	\\
3.65e-09	0	\\
3.75e-09	0	\\
3.86e-09	0	\\
3.96e-09	0	\\
4.06e-09	0	\\
4.16e-09	0	\\
4.27e-09	0	\\
4.37e-09	0	\\
4.47e-09	0	\\
4.57e-09	0	\\
4.68e-09	0	\\
4.78e-09	0	\\
4.89e-09	0	\\
4.99e-09	0	\\
5e-09	0	\\
};
\addplot [color=mycolor1,solid,forget plot]
  table[row sep=crcr]{
0	0	\\
1.1e-10	0	\\
2.2e-10	0	\\
3.3e-10	0	\\
4.4e-10	0	\\
5.4e-10	0	\\
6.5e-10	0	\\
7.5e-10	0	\\
8.6e-10	0	\\
9.6e-10	0	\\
1.07e-09	0	\\
1.18e-09	0	\\
1.28e-09	0	\\
1.38e-09	0	\\
1.49e-09	0	\\
1.59e-09	0	\\
1.69e-09	0	\\
1.8e-09	0	\\
1.9e-09	0	\\
2.01e-09	0	\\
2.11e-09	0	\\
2.21e-09	0	\\
2.32e-09	0	\\
2.42e-09	0	\\
2.52e-09	0	\\
2.63e-09	0	\\
2.73e-09	0	\\
2.83e-09	0	\\
2.93e-09	0	\\
3.04e-09	0	\\
3.14e-09	0	\\
3.24e-09	0	\\
3.34e-09	0	\\
3.45e-09	0	\\
3.55e-09	0	\\
3.65e-09	0	\\
3.75e-09	0	\\
3.86e-09	0	\\
3.96e-09	0	\\
4.06e-09	0	\\
4.16e-09	0	\\
4.27e-09	0	\\
4.37e-09	0	\\
4.47e-09	0	\\
4.57e-09	0	\\
4.68e-09	0	\\
4.78e-09	0	\\
4.89e-09	0	\\
4.99e-09	0	\\
5e-09	0	\\
};
\addplot [color=mycolor2,solid,forget plot]
  table[row sep=crcr]{
0	0	\\
1.1e-10	0	\\
2.2e-10	0	\\
3.3e-10	0	\\
4.4e-10	0	\\
5.4e-10	0	\\
6.5e-10	0	\\
7.5e-10	0	\\
8.6e-10	0	\\
9.6e-10	0	\\
1.07e-09	0	\\
1.18e-09	0	\\
1.28e-09	0	\\
1.38e-09	0	\\
1.49e-09	0	\\
1.59e-09	0	\\
1.69e-09	0	\\
1.8e-09	0	\\
1.9e-09	0	\\
2.01e-09	0	\\
2.11e-09	0	\\
2.21e-09	0	\\
2.32e-09	0	\\
2.42e-09	0	\\
2.52e-09	0	\\
2.63e-09	0	\\
2.73e-09	0	\\
2.83e-09	0	\\
2.93e-09	0	\\
3.04e-09	0	\\
3.14e-09	0	\\
3.24e-09	0	\\
3.34e-09	0	\\
3.45e-09	0	\\
3.55e-09	0	\\
3.65e-09	0	\\
3.75e-09	0	\\
3.86e-09	0	\\
3.96e-09	0	\\
4.06e-09	0	\\
4.16e-09	0	\\
4.27e-09	0	\\
4.37e-09	0	\\
4.47e-09	0	\\
4.57e-09	0	\\
4.68e-09	0	\\
4.78e-09	0	\\
4.89e-09	0	\\
4.99e-09	0	\\
5e-09	0	\\
};
\addplot [color=mycolor3,solid,forget plot]
  table[row sep=crcr]{
0	0	\\
1.1e-10	0	\\
2.2e-10	0	\\
3.3e-10	0	\\
4.4e-10	0	\\
5.4e-10	0	\\
6.5e-10	0	\\
7.5e-10	0	\\
8.6e-10	0	\\
9.6e-10	0	\\
1.07e-09	0	\\
1.18e-09	0	\\
1.28e-09	0	\\
1.38e-09	0	\\
1.49e-09	0	\\
1.59e-09	0	\\
1.69e-09	0	\\
1.8e-09	0	\\
1.9e-09	0	\\
2.01e-09	0	\\
2.11e-09	0	\\
2.21e-09	0	\\
2.32e-09	0	\\
2.42e-09	0	\\
2.52e-09	0	\\
2.63e-09	0	\\
2.73e-09	0	\\
2.83e-09	0	\\
2.93e-09	0	\\
3.04e-09	0	\\
3.14e-09	0	\\
3.24e-09	0	\\
3.34e-09	0	\\
3.45e-09	0	\\
3.55e-09	0	\\
3.65e-09	0	\\
3.75e-09	0	\\
3.86e-09	0	\\
3.96e-09	0	\\
4.06e-09	0	\\
4.16e-09	0	\\
4.27e-09	0	\\
4.37e-09	0	\\
4.47e-09	0	\\
4.57e-09	0	\\
4.68e-09	0	\\
4.78e-09	0	\\
4.89e-09	0	\\
4.99e-09	0	\\
5e-09	0	\\
};
\addplot [color=darkgray,solid,forget plot]
  table[row sep=crcr]{
0	0	\\
1.1e-10	0	\\
2.2e-10	0	\\
3.3e-10	0	\\
4.4e-10	0	\\
5.4e-10	0	\\
6.5e-10	0	\\
7.5e-10	0	\\
8.6e-10	0	\\
9.6e-10	0	\\
1.07e-09	0	\\
1.18e-09	0	\\
1.28e-09	0	\\
1.38e-09	0	\\
1.49e-09	0	\\
1.59e-09	0	\\
1.69e-09	0	\\
1.8e-09	0	\\
1.9e-09	0	\\
2.01e-09	0	\\
2.11e-09	0	\\
2.21e-09	0	\\
2.32e-09	0	\\
2.42e-09	0	\\
2.52e-09	0	\\
2.63e-09	0	\\
2.73e-09	0	\\
2.83e-09	0	\\
2.93e-09	0	\\
3.04e-09	0	\\
3.14e-09	0	\\
3.24e-09	0	\\
3.34e-09	0	\\
3.45e-09	0	\\
3.55e-09	0	\\
3.65e-09	0	\\
3.75e-09	0	\\
3.86e-09	0	\\
3.96e-09	0	\\
4.06e-09	0	\\
4.16e-09	0	\\
4.27e-09	0	\\
4.37e-09	0	\\
4.47e-09	0	\\
4.57e-09	0	\\
4.68e-09	0	\\
4.78e-09	0	\\
4.89e-09	0	\\
4.99e-09	0	\\
5e-09	0	\\
};
\addplot [color=blue,solid,forget plot]
  table[row sep=crcr]{
0	0	\\
1.1e-10	0	\\
2.2e-10	0	\\
3.3e-10	0	\\
4.4e-10	0	\\
5.4e-10	0	\\
6.5e-10	0	\\
7.5e-10	0	\\
8.6e-10	0	\\
9.6e-10	0	\\
1.07e-09	0	\\
1.18e-09	0	\\
1.28e-09	0	\\
1.38e-09	0	\\
1.49e-09	0	\\
1.59e-09	0	\\
1.69e-09	0	\\
1.8e-09	0	\\
1.9e-09	0	\\
2.01e-09	0	\\
2.11e-09	0	\\
2.21e-09	0	\\
2.32e-09	0	\\
2.42e-09	0	\\
2.52e-09	0	\\
2.63e-09	0	\\
2.73e-09	0	\\
2.83e-09	0	\\
2.93e-09	0	\\
3.04e-09	0	\\
3.14e-09	0	\\
3.24e-09	0	\\
3.34e-09	0	\\
3.45e-09	0	\\
3.55e-09	0	\\
3.65e-09	0	\\
3.75e-09	0	\\
3.86e-09	0	\\
3.96e-09	0	\\
4.06e-09	0	\\
4.16e-09	0	\\
4.27e-09	0	\\
4.37e-09	0	\\
4.47e-09	0	\\
4.57e-09	0	\\
4.68e-09	0	\\
4.78e-09	0	\\
4.89e-09	0	\\
4.99e-09	0	\\
5e-09	0	\\
};
\addplot [color=black!50!green,solid,forget plot]
  table[row sep=crcr]{
0	0	\\
1.1e-10	0	\\
2.2e-10	0	\\
3.3e-10	0	\\
4.4e-10	0	\\
5.4e-10	0	\\
6.5e-10	0	\\
7.5e-10	0	\\
8.6e-10	0	\\
9.6e-10	0	\\
1.07e-09	0	\\
1.18e-09	0	\\
1.28e-09	0	\\
1.38e-09	0	\\
1.49e-09	0	\\
1.59e-09	0	\\
1.69e-09	0	\\
1.8e-09	0	\\
1.9e-09	0	\\
2.01e-09	0	\\
2.11e-09	0	\\
2.21e-09	0	\\
2.32e-09	0	\\
2.42e-09	0	\\
2.52e-09	0	\\
2.63e-09	0	\\
2.73e-09	0	\\
2.83e-09	0	\\
2.93e-09	0	\\
3.04e-09	0	\\
3.14e-09	0	\\
3.24e-09	0	\\
3.34e-09	0	\\
3.45e-09	0	\\
3.55e-09	0	\\
3.65e-09	0	\\
3.75e-09	0	\\
3.86e-09	0	\\
3.96e-09	0	\\
4.06e-09	0	\\
4.16e-09	0	\\
4.27e-09	0	\\
4.37e-09	0	\\
4.47e-09	0	\\
4.57e-09	0	\\
4.68e-09	0	\\
4.78e-09	0	\\
4.89e-09	0	\\
4.99e-09	0	\\
5e-09	0	\\
};
\addplot [color=red,solid,forget plot]
  table[row sep=crcr]{
0	0	\\
1.1e-10	0	\\
2.2e-10	0	\\
3.3e-10	0	\\
4.4e-10	0	\\
5.4e-10	0	\\
6.5e-10	0	\\
7.5e-10	0	\\
8.6e-10	0	\\
9.6e-10	0	\\
1.07e-09	0	\\
1.18e-09	0	\\
1.28e-09	0	\\
1.38e-09	0	\\
1.49e-09	0	\\
1.59e-09	0	\\
1.69e-09	0	\\
1.8e-09	0	\\
1.9e-09	0	\\
2.01e-09	0	\\
2.11e-09	0	\\
2.21e-09	0	\\
2.32e-09	0	\\
2.42e-09	0	\\
2.52e-09	0	\\
2.63e-09	0	\\
2.73e-09	0	\\
2.83e-09	0	\\
2.93e-09	0	\\
3.04e-09	0	\\
3.14e-09	0	\\
3.24e-09	0	\\
3.34e-09	0	\\
3.45e-09	0	\\
3.55e-09	0	\\
3.65e-09	0	\\
3.75e-09	0	\\
3.86e-09	0	\\
3.96e-09	0	\\
4.06e-09	0	\\
4.16e-09	0	\\
4.27e-09	0	\\
4.37e-09	0	\\
4.47e-09	0	\\
4.57e-09	0	\\
4.68e-09	0	\\
4.78e-09	0	\\
4.89e-09	0	\\
4.99e-09	0	\\
5e-09	0	\\
};
\addplot [color=mycolor1,solid,forget plot]
  table[row sep=crcr]{
0	0	\\
1.1e-10	0	\\
2.2e-10	0	\\
3.3e-10	0	\\
4.4e-10	0	\\
5.4e-10	0	\\
6.5e-10	0	\\
7.5e-10	0	\\
8.6e-10	0	\\
9.6e-10	0	\\
1.07e-09	0	\\
1.18e-09	0	\\
1.28e-09	0	\\
1.38e-09	0	\\
1.49e-09	0	\\
1.59e-09	0	\\
1.69e-09	0	\\
1.8e-09	0	\\
1.9e-09	0	\\
2.01e-09	0	\\
2.11e-09	0	\\
2.21e-09	0	\\
2.32e-09	0	\\
2.42e-09	0	\\
2.52e-09	0	\\
2.63e-09	0	\\
2.73e-09	0	\\
2.83e-09	0	\\
2.93e-09	0	\\
3.04e-09	0	\\
3.14e-09	0	\\
3.24e-09	0	\\
3.34e-09	0	\\
3.45e-09	0	\\
3.55e-09	0	\\
3.65e-09	0	\\
3.75e-09	0	\\
3.86e-09	0	\\
3.96e-09	0	\\
4.06e-09	0	\\
4.16e-09	0	\\
4.27e-09	0	\\
4.37e-09	0	\\
4.47e-09	0	\\
4.57e-09	0	\\
4.68e-09	0	\\
4.78e-09	0	\\
4.89e-09	0	\\
4.99e-09	0	\\
5e-09	0	\\
};
\addplot [color=mycolor2,solid,forget plot]
  table[row sep=crcr]{
0	0	\\
1.1e-10	0	\\
2.2e-10	0	\\
3.3e-10	0	\\
4.4e-10	0	\\
5.4e-10	0	\\
6.5e-10	0	\\
7.5e-10	0	\\
8.6e-10	0	\\
9.6e-10	0	\\
1.07e-09	0	\\
1.18e-09	0	\\
1.28e-09	0	\\
1.38e-09	0	\\
1.49e-09	0	\\
1.59e-09	0	\\
1.69e-09	0	\\
1.8e-09	0	\\
1.9e-09	0	\\
2.01e-09	0	\\
2.11e-09	0	\\
2.21e-09	0	\\
2.32e-09	0	\\
2.42e-09	0	\\
2.52e-09	0	\\
2.63e-09	0	\\
2.73e-09	0	\\
2.83e-09	0	\\
2.93e-09	0	\\
3.04e-09	0	\\
3.14e-09	0	\\
3.24e-09	0	\\
3.34e-09	0	\\
3.45e-09	0	\\
3.55e-09	0	\\
3.65e-09	0	\\
3.75e-09	0	\\
3.86e-09	0	\\
3.96e-09	0	\\
4.06e-09	0	\\
4.16e-09	0	\\
4.27e-09	0	\\
4.37e-09	0	\\
4.47e-09	0	\\
4.57e-09	0	\\
4.68e-09	0	\\
4.78e-09	0	\\
4.89e-09	0	\\
4.99e-09	0	\\
5e-09	0	\\
};
\addplot [color=mycolor3,solid,forget plot]
  table[row sep=crcr]{
0	0	\\
1.1e-10	0	\\
2.2e-10	0	\\
3.3e-10	0	\\
4.4e-10	0	\\
5.4e-10	0	\\
6.5e-10	0	\\
7.5e-10	0	\\
8.6e-10	0	\\
9.6e-10	0	\\
1.07e-09	0	\\
1.18e-09	0	\\
1.28e-09	0	\\
1.38e-09	0	\\
1.49e-09	0	\\
1.59e-09	0	\\
1.69e-09	0	\\
1.8e-09	0	\\
1.9e-09	0	\\
2.01e-09	0	\\
2.11e-09	0	\\
2.21e-09	0	\\
2.32e-09	0	\\
2.42e-09	0	\\
2.52e-09	0	\\
2.63e-09	0	\\
2.73e-09	0	\\
2.83e-09	0	\\
2.93e-09	0	\\
3.04e-09	0	\\
3.14e-09	0	\\
3.24e-09	0	\\
3.34e-09	0	\\
3.45e-09	0	\\
3.55e-09	0	\\
3.65e-09	0	\\
3.75e-09	0	\\
3.86e-09	0	\\
3.96e-09	0	\\
4.06e-09	0	\\
4.16e-09	0	\\
4.27e-09	0	\\
4.37e-09	0	\\
4.47e-09	0	\\
4.57e-09	0	\\
4.68e-09	0	\\
4.78e-09	0	\\
4.89e-09	0	\\
4.99e-09	0	\\
5e-09	0	\\
};
\addplot [color=darkgray,solid,forget plot]
  table[row sep=crcr]{
0	0	\\
1.1e-10	0	\\
2.2e-10	0	\\
3.3e-10	0	\\
4.4e-10	0	\\
5.4e-10	0	\\
6.5e-10	0	\\
7.5e-10	0	\\
8.6e-10	0	\\
9.6e-10	0	\\
1.07e-09	0	\\
1.18e-09	0	\\
1.28e-09	0	\\
1.38e-09	0	\\
1.49e-09	0	\\
1.59e-09	0	\\
1.69e-09	0	\\
1.8e-09	0	\\
1.9e-09	0	\\
2.01e-09	0	\\
2.11e-09	0	\\
2.21e-09	0	\\
2.32e-09	0	\\
2.42e-09	0	\\
2.52e-09	0	\\
2.63e-09	0	\\
2.73e-09	0	\\
2.83e-09	0	\\
2.93e-09	0	\\
3.04e-09	0	\\
3.14e-09	0	\\
3.24e-09	0	\\
3.34e-09	0	\\
3.45e-09	0	\\
3.55e-09	0	\\
3.65e-09	0	\\
3.75e-09	0	\\
3.86e-09	0	\\
3.96e-09	0	\\
4.06e-09	0	\\
4.16e-09	0	\\
4.27e-09	0	\\
4.37e-09	0	\\
4.47e-09	0	\\
4.57e-09	0	\\
4.68e-09	0	\\
4.78e-09	0	\\
4.89e-09	0	\\
4.99e-09	0	\\
5e-09	0	\\
};
\addplot [color=blue,solid,forget plot]
  table[row sep=crcr]{
0	0	\\
1.1e-10	0	\\
2.2e-10	0	\\
3.3e-10	0	\\
4.4e-10	0	\\
5.4e-10	0	\\
6.5e-10	0	\\
7.5e-10	0	\\
8.6e-10	0	\\
9.6e-10	0	\\
1.07e-09	0	\\
1.18e-09	0	\\
1.28e-09	0	\\
1.38e-09	0	\\
1.49e-09	0	\\
1.59e-09	0	\\
1.69e-09	0	\\
1.8e-09	0	\\
1.9e-09	0	\\
2.01e-09	0	\\
2.11e-09	0	\\
2.21e-09	0	\\
2.32e-09	0	\\
2.42e-09	0	\\
2.52e-09	0	\\
2.63e-09	0	\\
2.73e-09	0	\\
2.83e-09	0	\\
2.93e-09	0	\\
3.04e-09	0	\\
3.14e-09	0	\\
3.24e-09	0	\\
3.34e-09	0	\\
3.45e-09	0	\\
3.55e-09	0	\\
3.65e-09	0	\\
3.75e-09	0	\\
3.86e-09	0	\\
3.96e-09	0	\\
4.06e-09	0	\\
4.16e-09	0	\\
4.27e-09	0	\\
4.37e-09	0	\\
4.47e-09	0	\\
4.57e-09	0	\\
4.68e-09	0	\\
4.78e-09	0	\\
4.89e-09	0	\\
4.99e-09	0	\\
5e-09	0	\\
};
\addplot [color=black!50!green,solid,forget plot]
  table[row sep=crcr]{
0	0	\\
1.1e-10	0	\\
2.2e-10	0	\\
3.3e-10	0	\\
4.4e-10	0	\\
5.4e-10	0	\\
6.5e-10	0	\\
7.5e-10	0	\\
8.6e-10	0	\\
9.6e-10	0	\\
1.07e-09	0	\\
1.18e-09	0	\\
1.28e-09	0	\\
1.38e-09	0	\\
1.49e-09	0	\\
1.59e-09	0	\\
1.69e-09	0	\\
1.8e-09	0	\\
1.9e-09	0	\\
2.01e-09	0	\\
2.11e-09	0	\\
2.21e-09	0	\\
2.32e-09	0	\\
2.42e-09	0	\\
2.52e-09	0	\\
2.63e-09	0	\\
2.73e-09	0	\\
2.83e-09	0	\\
2.93e-09	0	\\
3.04e-09	0	\\
3.14e-09	0	\\
3.24e-09	0	\\
3.34e-09	0	\\
3.45e-09	0	\\
3.55e-09	0	\\
3.65e-09	0	\\
3.75e-09	0	\\
3.86e-09	0	\\
3.96e-09	0	\\
4.06e-09	0	\\
4.16e-09	0	\\
4.27e-09	0	\\
4.37e-09	0	\\
4.47e-09	0	\\
4.57e-09	0	\\
4.68e-09	0	\\
4.78e-09	0	\\
4.89e-09	0	\\
4.99e-09	0	\\
5e-09	0	\\
};
\addplot [color=red,solid,forget plot]
  table[row sep=crcr]{
0	0	\\
1.1e-10	0	\\
2.2e-10	0	\\
3.3e-10	0	\\
4.4e-10	0	\\
5.4e-10	0	\\
6.5e-10	0	\\
7.5e-10	0	\\
8.6e-10	0	\\
9.6e-10	0	\\
1.07e-09	0	\\
1.18e-09	0	\\
1.28e-09	0	\\
1.38e-09	0	\\
1.49e-09	0	\\
1.59e-09	0	\\
1.69e-09	0	\\
1.8e-09	0	\\
1.9e-09	0	\\
2.01e-09	0	\\
2.11e-09	0	\\
2.21e-09	0	\\
2.32e-09	0	\\
2.42e-09	0	\\
2.52e-09	0	\\
2.63e-09	0	\\
2.73e-09	0	\\
2.83e-09	0	\\
2.93e-09	0	\\
3.04e-09	0	\\
3.14e-09	0	\\
3.24e-09	0	\\
3.34e-09	0	\\
3.45e-09	0	\\
3.55e-09	0	\\
3.65e-09	0	\\
3.75e-09	0	\\
3.86e-09	0	\\
3.96e-09	0	\\
4.06e-09	0	\\
4.16e-09	0	\\
4.27e-09	0	\\
4.37e-09	0	\\
4.47e-09	0	\\
4.57e-09	0	\\
4.68e-09	0	\\
4.78e-09	0	\\
4.89e-09	0	\\
4.99e-09	0	\\
5e-09	0	\\
};
\addplot [color=mycolor1,solid,forget plot]
  table[row sep=crcr]{
0	0	\\
1.1e-10	0	\\
2.2e-10	0	\\
3.3e-10	0	\\
4.4e-10	0	\\
5.4e-10	0	\\
6.5e-10	0	\\
7.5e-10	0	\\
8.6e-10	0	\\
9.6e-10	0	\\
1.07e-09	0	\\
1.18e-09	0	\\
1.28e-09	0	\\
1.38e-09	0	\\
1.49e-09	0	\\
1.59e-09	0	\\
1.69e-09	0	\\
1.8e-09	0	\\
1.9e-09	0	\\
2.01e-09	0	\\
2.11e-09	0	\\
2.21e-09	0	\\
2.32e-09	0	\\
2.42e-09	0	\\
2.52e-09	0	\\
2.63e-09	0	\\
2.73e-09	0	\\
2.83e-09	0	\\
2.93e-09	0	\\
3.04e-09	0	\\
3.14e-09	0	\\
3.24e-09	0	\\
3.34e-09	0	\\
3.45e-09	0	\\
3.55e-09	0	\\
3.65e-09	0	\\
3.75e-09	0	\\
3.86e-09	0	\\
3.96e-09	0	\\
4.06e-09	0	\\
4.16e-09	0	\\
4.27e-09	0	\\
4.37e-09	0	\\
4.47e-09	0	\\
4.57e-09	0	\\
4.68e-09	0	\\
4.78e-09	0	\\
4.89e-09	0	\\
4.99e-09	0	\\
5e-09	0	\\
};
\addplot [color=mycolor2,solid,forget plot]
  table[row sep=crcr]{
0	0	\\
1.1e-10	0	\\
2.2e-10	0	\\
3.3e-10	0	\\
4.4e-10	0	\\
5.4e-10	0	\\
6.5e-10	0	\\
7.5e-10	0	\\
8.6e-10	0	\\
9.6e-10	0	\\
1.07e-09	0	\\
1.18e-09	0	\\
1.28e-09	0	\\
1.38e-09	0	\\
1.49e-09	0	\\
1.59e-09	0	\\
1.69e-09	0	\\
1.8e-09	0	\\
1.9e-09	0	\\
2.01e-09	0	\\
2.11e-09	0	\\
2.21e-09	0	\\
2.32e-09	0	\\
2.42e-09	0	\\
2.52e-09	0	\\
2.63e-09	0	\\
2.73e-09	0	\\
2.83e-09	0	\\
2.93e-09	0	\\
3.04e-09	0	\\
3.14e-09	0	\\
3.24e-09	0	\\
3.34e-09	0	\\
3.45e-09	0	\\
3.55e-09	0	\\
3.65e-09	0	\\
3.75e-09	0	\\
3.86e-09	0	\\
3.96e-09	0	\\
4.06e-09	0	\\
4.16e-09	0	\\
4.27e-09	0	\\
4.37e-09	0	\\
4.47e-09	0	\\
4.57e-09	0	\\
4.68e-09	0	\\
4.78e-09	0	\\
4.89e-09	0	\\
4.99e-09	0	\\
5e-09	0	\\
};
\addplot [color=mycolor3,solid,forget plot]
  table[row sep=crcr]{
0	0	\\
1.1e-10	0	\\
2.2e-10	0	\\
3.3e-10	0	\\
4.4e-10	0	\\
5.4e-10	0	\\
6.5e-10	0	\\
7.5e-10	0	\\
8.6e-10	0	\\
9.6e-10	0	\\
1.07e-09	0	\\
1.18e-09	0	\\
1.28e-09	0	\\
1.38e-09	0	\\
1.49e-09	0	\\
1.59e-09	0	\\
1.69e-09	0	\\
1.8e-09	0	\\
1.9e-09	0	\\
2.01e-09	0	\\
2.11e-09	0	\\
2.21e-09	0	\\
2.32e-09	0	\\
2.42e-09	0	\\
2.52e-09	0	\\
2.63e-09	0	\\
2.73e-09	0	\\
2.83e-09	0	\\
2.93e-09	0	\\
3.04e-09	0	\\
3.14e-09	0	\\
3.24e-09	0	\\
3.34e-09	0	\\
3.45e-09	0	\\
3.55e-09	0	\\
3.65e-09	0	\\
3.75e-09	0	\\
3.86e-09	0	\\
3.96e-09	0	\\
4.06e-09	0	\\
4.16e-09	0	\\
4.27e-09	0	\\
4.37e-09	0	\\
4.47e-09	0	\\
4.57e-09	0	\\
4.68e-09	0	\\
4.78e-09	0	\\
4.89e-09	0	\\
4.99e-09	0	\\
5e-09	0	\\
};
\addplot [color=darkgray,solid,forget plot]
  table[row sep=crcr]{
0	0	\\
1.1e-10	0	\\
2.2e-10	0	\\
3.3e-10	0	\\
4.4e-10	0	\\
5.4e-10	0	\\
6.5e-10	0	\\
7.5e-10	0	\\
8.6e-10	0	\\
9.6e-10	0	\\
1.07e-09	0	\\
1.18e-09	0	\\
1.28e-09	0	\\
1.38e-09	0	\\
1.49e-09	0	\\
1.59e-09	0	\\
1.69e-09	0	\\
1.8e-09	0	\\
1.9e-09	0	\\
2.01e-09	0	\\
2.11e-09	0	\\
2.21e-09	0	\\
2.32e-09	0	\\
2.42e-09	0	\\
2.52e-09	0	\\
2.63e-09	0	\\
2.73e-09	0	\\
2.83e-09	0	\\
2.93e-09	0	\\
3.04e-09	0	\\
3.14e-09	0	\\
3.24e-09	0	\\
3.34e-09	0	\\
3.45e-09	0	\\
3.55e-09	0	\\
3.65e-09	0	\\
3.75e-09	0	\\
3.86e-09	0	\\
3.96e-09	0	\\
4.06e-09	0	\\
4.16e-09	0	\\
4.27e-09	0	\\
4.37e-09	0	\\
4.47e-09	0	\\
4.57e-09	0	\\
4.68e-09	0	\\
4.78e-09	0	\\
4.89e-09	0	\\
4.99e-09	0	\\
5e-09	0	\\
};
\addplot [color=blue,solid,forget plot]
  table[row sep=crcr]{
0	0	\\
1.1e-10	0	\\
2.2e-10	0	\\
3.3e-10	0	\\
4.4e-10	0	\\
5.4e-10	0	\\
6.5e-10	0	\\
7.5e-10	0	\\
8.6e-10	0	\\
9.6e-10	0	\\
1.07e-09	0	\\
1.18e-09	0	\\
1.28e-09	0	\\
1.38e-09	0	\\
1.49e-09	0	\\
1.59e-09	0	\\
1.69e-09	0	\\
1.8e-09	0	\\
1.9e-09	0	\\
2.01e-09	0	\\
2.11e-09	0	\\
2.21e-09	0	\\
2.32e-09	0	\\
2.42e-09	0	\\
2.52e-09	0	\\
2.63e-09	0	\\
2.73e-09	0	\\
2.83e-09	0	\\
2.93e-09	0	\\
3.04e-09	0	\\
3.14e-09	0	\\
3.24e-09	0	\\
3.34e-09	0	\\
3.45e-09	0	\\
3.55e-09	0	\\
3.65e-09	0	\\
3.75e-09	0	\\
3.86e-09	0	\\
3.96e-09	0	\\
4.06e-09	0	\\
4.16e-09	0	\\
4.27e-09	0	\\
4.37e-09	0	\\
4.47e-09	0	\\
4.57e-09	0	\\
4.68e-09	0	\\
4.78e-09	0	\\
4.89e-09	0	\\
4.99e-09	0	\\
5e-09	0	\\
};
\addplot [color=black!50!green,solid,forget plot]
  table[row sep=crcr]{
0	0	\\
1.1e-10	0	\\
2.2e-10	0	\\
3.3e-10	0	\\
4.4e-10	0	\\
5.4e-10	0	\\
6.5e-10	0	\\
7.5e-10	0	\\
8.6e-10	0	\\
9.6e-10	0	\\
1.07e-09	0	\\
1.18e-09	0	\\
1.28e-09	0	\\
1.38e-09	0	\\
1.49e-09	0	\\
1.59e-09	0	\\
1.69e-09	0	\\
1.8e-09	0	\\
1.9e-09	0	\\
2.01e-09	0	\\
2.11e-09	0	\\
2.21e-09	0	\\
2.32e-09	0	\\
2.42e-09	0	\\
2.52e-09	0	\\
2.63e-09	0	\\
2.73e-09	0	\\
2.83e-09	0	\\
2.93e-09	0	\\
3.04e-09	0	\\
3.14e-09	0	\\
3.24e-09	0	\\
3.34e-09	0	\\
3.45e-09	0	\\
3.55e-09	0	\\
3.65e-09	0	\\
3.75e-09	0	\\
3.86e-09	0	\\
3.96e-09	0	\\
4.06e-09	0	\\
4.16e-09	0	\\
4.27e-09	0	\\
4.37e-09	0	\\
4.47e-09	0	\\
4.57e-09	0	\\
4.68e-09	0	\\
4.78e-09	0	\\
4.89e-09	0	\\
4.99e-09	0	\\
5e-09	0	\\
};
\addplot [color=red,solid,forget plot]
  table[row sep=crcr]{
0	0	\\
1.1e-10	0	\\
2.2e-10	0	\\
3.3e-10	0	\\
4.4e-10	0	\\
5.4e-10	0	\\
6.5e-10	0	\\
7.5e-10	0	\\
8.6e-10	0	\\
9.6e-10	0	\\
1.07e-09	0	\\
1.18e-09	0	\\
1.28e-09	0	\\
1.38e-09	0	\\
1.49e-09	0	\\
1.59e-09	0	\\
1.69e-09	0	\\
1.8e-09	0	\\
1.9e-09	0	\\
2.01e-09	0	\\
2.11e-09	0	\\
2.21e-09	0	\\
2.32e-09	0	\\
2.42e-09	0	\\
2.52e-09	0	\\
2.63e-09	0	\\
2.73e-09	0	\\
2.83e-09	0	\\
2.93e-09	0	\\
3.04e-09	0	\\
3.14e-09	0	\\
3.24e-09	0	\\
3.34e-09	0	\\
3.45e-09	0	\\
3.55e-09	0	\\
3.65e-09	0	\\
3.75e-09	0	\\
3.86e-09	0	\\
3.96e-09	0	\\
4.06e-09	0	\\
4.16e-09	0	\\
4.27e-09	0	\\
4.37e-09	0	\\
4.47e-09	0	\\
4.57e-09	0	\\
4.68e-09	0	\\
4.78e-09	0	\\
4.89e-09	0	\\
4.99e-09	0	\\
5e-09	0	\\
};
\addplot [color=mycolor1,solid,forget plot]
  table[row sep=crcr]{
0	0	\\
1.1e-10	0	\\
2.2e-10	0	\\
3.3e-10	0	\\
4.4e-10	0	\\
5.4e-10	0	\\
6.5e-10	0	\\
7.5e-10	0	\\
8.6e-10	0	\\
9.6e-10	0	\\
1.07e-09	0	\\
1.18e-09	0	\\
1.28e-09	0	\\
1.38e-09	0	\\
1.49e-09	0	\\
1.59e-09	0	\\
1.69e-09	0	\\
1.8e-09	0	\\
1.9e-09	0	\\
2.01e-09	0	\\
2.11e-09	0	\\
2.21e-09	0	\\
2.32e-09	0	\\
2.42e-09	0	\\
2.52e-09	0	\\
2.63e-09	0	\\
2.73e-09	0	\\
2.83e-09	0	\\
2.93e-09	0	\\
3.04e-09	0	\\
3.14e-09	0	\\
3.24e-09	0	\\
3.34e-09	0	\\
3.45e-09	0	\\
3.55e-09	0	\\
3.65e-09	0	\\
3.75e-09	0	\\
3.86e-09	0	\\
3.96e-09	0	\\
4.06e-09	0	\\
4.16e-09	0	\\
4.27e-09	0	\\
4.37e-09	0	\\
4.47e-09	0	\\
4.57e-09	0	\\
4.68e-09	0	\\
4.78e-09	0	\\
4.89e-09	0	\\
4.99e-09	0	\\
5e-09	0	\\
};
\addplot [color=mycolor2,solid,forget plot]
  table[row sep=crcr]{
0	0	\\
1.1e-10	0	\\
2.2e-10	0	\\
3.3e-10	0	\\
4.4e-10	0	\\
5.4e-10	0	\\
6.5e-10	0	\\
7.5e-10	0	\\
8.6e-10	0	\\
9.6e-10	0	\\
1.07e-09	0	\\
1.18e-09	0	\\
1.28e-09	0	\\
1.38e-09	0	\\
1.49e-09	0	\\
1.59e-09	0	\\
1.69e-09	0	\\
1.8e-09	0	\\
1.9e-09	0	\\
2.01e-09	0	\\
2.11e-09	0	\\
2.21e-09	0	\\
2.32e-09	0	\\
2.42e-09	0	\\
2.52e-09	0	\\
2.63e-09	0	\\
2.73e-09	0	\\
2.83e-09	0	\\
2.93e-09	0	\\
3.04e-09	0	\\
3.14e-09	0	\\
3.24e-09	0	\\
3.34e-09	0	\\
3.45e-09	0	\\
3.55e-09	0	\\
3.65e-09	0	\\
3.75e-09	0	\\
3.86e-09	0	\\
3.96e-09	0	\\
4.06e-09	0	\\
4.16e-09	0	\\
4.27e-09	0	\\
4.37e-09	0	\\
4.47e-09	0	\\
4.57e-09	0	\\
4.68e-09	0	\\
4.78e-09	0	\\
4.89e-09	0	\\
4.99e-09	0	\\
5e-09	0	\\
};
\addplot [color=mycolor3,solid,forget plot]
  table[row sep=crcr]{
0	0	\\
1.1e-10	0	\\
2.2e-10	0	\\
3.3e-10	0	\\
4.4e-10	0	\\
5.4e-10	0	\\
6.5e-10	0	\\
7.5e-10	0	\\
8.6e-10	0	\\
9.6e-10	0	\\
1.07e-09	0	\\
1.18e-09	0	\\
1.28e-09	0	\\
1.38e-09	0	\\
1.49e-09	0	\\
1.59e-09	0	\\
1.69e-09	0	\\
1.8e-09	0	\\
1.9e-09	0	\\
2.01e-09	0	\\
2.11e-09	0	\\
2.21e-09	0	\\
2.32e-09	0	\\
2.42e-09	0	\\
2.52e-09	0	\\
2.63e-09	0	\\
2.73e-09	0	\\
2.83e-09	0	\\
2.93e-09	0	\\
3.04e-09	0	\\
3.14e-09	0	\\
3.24e-09	0	\\
3.34e-09	0	\\
3.45e-09	0	\\
3.55e-09	0	\\
3.65e-09	0	\\
3.75e-09	0	\\
3.86e-09	0	\\
3.96e-09	0	\\
4.06e-09	0	\\
4.16e-09	0	\\
4.27e-09	0	\\
4.37e-09	0	\\
4.47e-09	0	\\
4.57e-09	0	\\
4.68e-09	0	\\
4.78e-09	0	\\
4.89e-09	0	\\
4.99e-09	0	\\
5e-09	0	\\
};
\addplot [color=darkgray,solid,forget plot]
  table[row sep=crcr]{
0	0	\\
1.1e-10	0	\\
2.2e-10	0	\\
3.3e-10	0	\\
4.4e-10	0	\\
5.4e-10	0	\\
6.5e-10	0	\\
7.5e-10	0	\\
8.6e-10	0	\\
9.6e-10	0	\\
1.07e-09	0	\\
1.18e-09	0	\\
1.28e-09	0	\\
1.38e-09	0	\\
1.49e-09	0	\\
1.59e-09	0	\\
1.69e-09	0	\\
1.8e-09	0	\\
1.9e-09	0	\\
2.01e-09	0	\\
2.11e-09	0	\\
2.21e-09	0	\\
2.32e-09	0	\\
2.42e-09	0	\\
2.52e-09	0	\\
2.63e-09	0	\\
2.73e-09	0	\\
2.83e-09	0	\\
2.93e-09	0	\\
3.04e-09	0	\\
3.14e-09	0	\\
3.24e-09	0	\\
3.34e-09	0	\\
3.45e-09	0	\\
3.55e-09	0	\\
3.65e-09	0	\\
3.75e-09	0	\\
3.86e-09	0	\\
3.96e-09	0	\\
4.06e-09	0	\\
4.16e-09	0	\\
4.27e-09	0	\\
4.37e-09	0	\\
4.47e-09	0	\\
4.57e-09	0	\\
4.68e-09	0	\\
4.78e-09	0	\\
4.89e-09	0	\\
4.99e-09	0	\\
5e-09	0	\\
};
\addplot [color=blue,solid,forget plot]
  table[row sep=crcr]{
0	0	\\
1.1e-10	0	\\
2.2e-10	0	\\
3.3e-10	0	\\
4.4e-10	0	\\
5.4e-10	0	\\
6.5e-10	0	\\
7.5e-10	0	\\
8.6e-10	0	\\
9.6e-10	0	\\
1.07e-09	0	\\
1.18e-09	0	\\
1.28e-09	0	\\
1.38e-09	0	\\
1.49e-09	0	\\
1.59e-09	0	\\
1.69e-09	0	\\
1.8e-09	0	\\
1.9e-09	0	\\
2.01e-09	0	\\
2.11e-09	0	\\
2.21e-09	0	\\
2.32e-09	0	\\
2.42e-09	0	\\
2.52e-09	0	\\
2.63e-09	0	\\
2.73e-09	0	\\
2.83e-09	0	\\
2.93e-09	0	\\
3.04e-09	0	\\
3.14e-09	0	\\
3.24e-09	0	\\
3.34e-09	0	\\
3.45e-09	0	\\
3.55e-09	0	\\
3.65e-09	0	\\
3.75e-09	0	\\
3.86e-09	0	\\
3.96e-09	0	\\
4.06e-09	0	\\
4.16e-09	0	\\
4.27e-09	0	\\
4.37e-09	0	\\
4.47e-09	0	\\
4.57e-09	0	\\
4.68e-09	0	\\
4.78e-09	0	\\
4.89e-09	0	\\
4.99e-09	0	\\
5e-09	0	\\
};
\addplot [color=black!50!green,solid,forget plot]
  table[row sep=crcr]{
0	0	\\
1.1e-10	0	\\
2.2e-10	0	\\
3.3e-10	0	\\
4.4e-10	0	\\
5.4e-10	0	\\
6.5e-10	0	\\
7.5e-10	0	\\
8.6e-10	0	\\
9.6e-10	0	\\
1.07e-09	0	\\
1.18e-09	0	\\
1.28e-09	0	\\
1.38e-09	0	\\
1.49e-09	0	\\
1.59e-09	0	\\
1.69e-09	0	\\
1.8e-09	0	\\
1.9e-09	0	\\
2.01e-09	0	\\
2.11e-09	0	\\
2.21e-09	0	\\
2.32e-09	0	\\
2.42e-09	0	\\
2.52e-09	0	\\
2.63e-09	0	\\
2.73e-09	0	\\
2.83e-09	0	\\
2.93e-09	0	\\
3.04e-09	0	\\
3.14e-09	0	\\
3.24e-09	0	\\
3.34e-09	0	\\
3.45e-09	0	\\
3.55e-09	0	\\
3.65e-09	0	\\
3.75e-09	0	\\
3.86e-09	0	\\
3.96e-09	0	\\
4.06e-09	0	\\
4.16e-09	0	\\
4.27e-09	0	\\
4.37e-09	0	\\
4.47e-09	0	\\
4.57e-09	0	\\
4.68e-09	0	\\
4.78e-09	0	\\
4.89e-09	0	\\
4.99e-09	0	\\
5e-09	0	\\
};
\addplot [color=red,solid,forget plot]
  table[row sep=crcr]{
0	0	\\
1.1e-10	0	\\
2.2e-10	0	\\
3.3e-10	0	\\
4.4e-10	0	\\
5.4e-10	0	\\
6.5e-10	0	\\
7.5e-10	0	\\
8.6e-10	0	\\
9.6e-10	0	\\
1.07e-09	0	\\
1.18e-09	0	\\
1.28e-09	0	\\
1.38e-09	0	\\
1.49e-09	0	\\
1.59e-09	0	\\
1.69e-09	0	\\
1.8e-09	0	\\
1.9e-09	0	\\
2.01e-09	0	\\
2.11e-09	0	\\
2.21e-09	0	\\
2.32e-09	0	\\
2.42e-09	0	\\
2.52e-09	0	\\
2.63e-09	0	\\
2.73e-09	0	\\
2.83e-09	0	\\
2.93e-09	0	\\
3.04e-09	0	\\
3.14e-09	0	\\
3.24e-09	0	\\
3.34e-09	0	\\
3.45e-09	0	\\
3.55e-09	0	\\
3.65e-09	0	\\
3.75e-09	0	\\
3.86e-09	0	\\
3.96e-09	0	\\
4.06e-09	0	\\
4.16e-09	0	\\
4.27e-09	0	\\
4.37e-09	0	\\
4.47e-09	0	\\
4.57e-09	0	\\
4.68e-09	0	\\
4.78e-09	0	\\
4.89e-09	0	\\
4.99e-09	0	\\
5e-09	0	\\
};
\addplot [color=mycolor1,solid,forget plot]
  table[row sep=crcr]{
0	0	\\
1.1e-10	0	\\
2.2e-10	0	\\
3.3e-10	0	\\
4.4e-10	0	\\
5.4e-10	0	\\
6.5e-10	0	\\
7.5e-10	0	\\
8.6e-10	0	\\
9.6e-10	0	\\
1.07e-09	0	\\
1.18e-09	0	\\
1.28e-09	0	\\
1.38e-09	0	\\
1.49e-09	0	\\
1.59e-09	0	\\
1.69e-09	0	\\
1.8e-09	0	\\
1.9e-09	0	\\
2.01e-09	0	\\
2.11e-09	0	\\
2.21e-09	0	\\
2.32e-09	0	\\
2.42e-09	0	\\
2.52e-09	0	\\
2.63e-09	0	\\
2.73e-09	0	\\
2.83e-09	0	\\
2.93e-09	0	\\
3.04e-09	0	\\
3.14e-09	0	\\
3.24e-09	0	\\
3.34e-09	0	\\
3.45e-09	0	\\
3.55e-09	0	\\
3.65e-09	0	\\
3.75e-09	0	\\
3.86e-09	0	\\
3.96e-09	0	\\
4.06e-09	0	\\
4.16e-09	0	\\
4.27e-09	0	\\
4.37e-09	0	\\
4.47e-09	0	\\
4.57e-09	0	\\
4.68e-09	0	\\
4.78e-09	0	\\
4.89e-09	0	\\
4.99e-09	0	\\
5e-09	0	\\
};
\addplot [color=mycolor2,solid,forget plot]
  table[row sep=crcr]{
0	0	\\
1.1e-10	0	\\
2.2e-10	0	\\
3.3e-10	0	\\
4.4e-10	0	\\
5.4e-10	0	\\
6.5e-10	0	\\
7.5e-10	0	\\
8.6e-10	0	\\
9.6e-10	0	\\
1.07e-09	0	\\
1.18e-09	0	\\
1.28e-09	0	\\
1.38e-09	0	\\
1.49e-09	0	\\
1.59e-09	0	\\
1.69e-09	0	\\
1.8e-09	0	\\
1.9e-09	0	\\
2.01e-09	0	\\
2.11e-09	0	\\
2.21e-09	0	\\
2.32e-09	0	\\
2.42e-09	0	\\
2.52e-09	0	\\
2.63e-09	0	\\
2.73e-09	0	\\
2.83e-09	0	\\
2.93e-09	0	\\
3.04e-09	0	\\
3.14e-09	0	\\
3.24e-09	0	\\
3.34e-09	0	\\
3.45e-09	0	\\
3.55e-09	0	\\
3.65e-09	0	\\
3.75e-09	0	\\
3.86e-09	0	\\
3.96e-09	0	\\
4.06e-09	0	\\
4.16e-09	0	\\
4.27e-09	0	\\
4.37e-09	0	\\
4.47e-09	0	\\
4.57e-09	0	\\
4.68e-09	0	\\
4.78e-09	0	\\
4.89e-09	0	\\
4.99e-09	0	\\
5e-09	0	\\
};
\addplot [color=mycolor3,solid,forget plot]
  table[row sep=crcr]{
0	0	\\
1.1e-10	0	\\
2.2e-10	0	\\
3.3e-10	0	\\
4.4e-10	0	\\
5.4e-10	0	\\
6.5e-10	0	\\
7.5e-10	0	\\
8.6e-10	0	\\
9.6e-10	0	\\
1.07e-09	0	\\
1.18e-09	0	\\
1.28e-09	0	\\
1.38e-09	0	\\
1.49e-09	0	\\
1.59e-09	0	\\
1.69e-09	0	\\
1.8e-09	0	\\
1.9e-09	0	\\
2.01e-09	0	\\
2.11e-09	0	\\
2.21e-09	0	\\
2.32e-09	0	\\
2.42e-09	0	\\
2.52e-09	0	\\
2.63e-09	0	\\
2.73e-09	0	\\
2.83e-09	0	\\
2.93e-09	0	\\
3.04e-09	0	\\
3.14e-09	0	\\
3.24e-09	0	\\
3.34e-09	0	\\
3.45e-09	0	\\
3.55e-09	0	\\
3.65e-09	0	\\
3.75e-09	0	\\
3.86e-09	0	\\
3.96e-09	0	\\
4.06e-09	0	\\
4.16e-09	0	\\
4.27e-09	0	\\
4.37e-09	0	\\
4.47e-09	0	\\
4.57e-09	0	\\
4.68e-09	0	\\
4.78e-09	0	\\
4.89e-09	0	\\
4.99e-09	0	\\
5e-09	0	\\
};
\addplot [color=darkgray,solid,forget plot]
  table[row sep=crcr]{
0	0	\\
1.1e-10	0	\\
2.2e-10	0	\\
3.3e-10	0	\\
4.4e-10	0	\\
5.4e-10	0	\\
6.5e-10	0	\\
7.5e-10	0	\\
8.6e-10	0	\\
9.6e-10	0	\\
1.07e-09	0	\\
1.18e-09	0	\\
1.28e-09	0	\\
1.38e-09	0	\\
1.49e-09	0	\\
1.59e-09	0	\\
1.69e-09	0	\\
1.8e-09	0	\\
1.9e-09	0	\\
2.01e-09	0	\\
2.11e-09	0	\\
2.21e-09	0	\\
2.32e-09	0	\\
2.42e-09	0	\\
2.52e-09	0	\\
2.63e-09	0	\\
2.73e-09	0	\\
2.83e-09	0	\\
2.93e-09	0	\\
3.04e-09	0	\\
3.14e-09	0	\\
3.24e-09	0	\\
3.34e-09	0	\\
3.45e-09	0	\\
3.55e-09	0	\\
3.65e-09	0	\\
3.75e-09	0	\\
3.86e-09	0	\\
3.96e-09	0	\\
4.06e-09	0	\\
4.16e-09	0	\\
4.27e-09	0	\\
4.37e-09	0	\\
4.47e-09	0	\\
4.57e-09	0	\\
4.68e-09	0	\\
4.78e-09	0	\\
4.89e-09	0	\\
4.99e-09	0	\\
5e-09	0	\\
};
\addplot [color=blue,solid,forget plot]
  table[row sep=crcr]{
0	0	\\
1.1e-10	0	\\
2.2e-10	0	\\
3.3e-10	0	\\
4.4e-10	0	\\
5.4e-10	0	\\
6.5e-10	0	\\
7.5e-10	0	\\
8.6e-10	0	\\
9.6e-10	0	\\
1.07e-09	0	\\
1.18e-09	0	\\
1.28e-09	0	\\
1.38e-09	0	\\
1.49e-09	0	\\
1.59e-09	0	\\
1.69e-09	0	\\
1.8e-09	0	\\
1.9e-09	0	\\
2.01e-09	0	\\
2.11e-09	0	\\
2.21e-09	0	\\
2.32e-09	0	\\
2.42e-09	0	\\
2.52e-09	0	\\
2.63e-09	0	\\
2.73e-09	0	\\
2.83e-09	0	\\
2.93e-09	0	\\
3.04e-09	0	\\
3.14e-09	0	\\
3.24e-09	0	\\
3.34e-09	0	\\
3.45e-09	0	\\
3.55e-09	0	\\
3.65e-09	0	\\
3.75e-09	0	\\
3.86e-09	0	\\
3.96e-09	0	\\
4.06e-09	0	\\
4.16e-09	0	\\
4.27e-09	0	\\
4.37e-09	0	\\
4.47e-09	0	\\
4.57e-09	0	\\
4.68e-09	0	\\
4.78e-09	0	\\
4.89e-09	0	\\
4.99e-09	0	\\
5e-09	0	\\
};
\addplot [color=black!50!green,solid,forget plot]
  table[row sep=crcr]{
0	0	\\
1.1e-10	0	\\
2.2e-10	0	\\
3.3e-10	0	\\
4.4e-10	0	\\
5.4e-10	0	\\
6.5e-10	0	\\
7.5e-10	0	\\
8.6e-10	0	\\
9.6e-10	0	\\
1.07e-09	0	\\
1.18e-09	0	\\
1.28e-09	0	\\
1.38e-09	0	\\
1.49e-09	0	\\
1.59e-09	0	\\
1.69e-09	0	\\
1.8e-09	0	\\
1.9e-09	0	\\
2.01e-09	0	\\
2.11e-09	0	\\
2.21e-09	0	\\
2.32e-09	0	\\
2.42e-09	0	\\
2.52e-09	0	\\
2.63e-09	0	\\
2.73e-09	0	\\
2.83e-09	0	\\
2.93e-09	0	\\
3.04e-09	0	\\
3.14e-09	0	\\
3.24e-09	0	\\
3.34e-09	0	\\
3.45e-09	0	\\
3.55e-09	0	\\
3.65e-09	0	\\
3.75e-09	0	\\
3.86e-09	0	\\
3.96e-09	0	\\
4.06e-09	0	\\
4.16e-09	0	\\
4.27e-09	0	\\
4.37e-09	0	\\
4.47e-09	0	\\
4.57e-09	0	\\
4.68e-09	0	\\
4.78e-09	0	\\
4.89e-09	0	\\
4.99e-09	0	\\
5e-09	0	\\
};
\addplot [color=red,solid,forget plot]
  table[row sep=crcr]{
0	0	\\
1.1e-10	0	\\
2.2e-10	0	\\
3.3e-10	0	\\
4.4e-10	0	\\
5.4e-10	0	\\
6.5e-10	0	\\
7.5e-10	0	\\
8.6e-10	0	\\
9.6e-10	0	\\
1.07e-09	0	\\
1.18e-09	0	\\
1.28e-09	0	\\
1.38e-09	0	\\
1.49e-09	0	\\
1.59e-09	0	\\
1.69e-09	0	\\
1.8e-09	0	\\
1.9e-09	0	\\
2.01e-09	0	\\
2.11e-09	0	\\
2.21e-09	0	\\
2.32e-09	0	\\
2.42e-09	0	\\
2.52e-09	0	\\
2.63e-09	0	\\
2.73e-09	0	\\
2.83e-09	0	\\
2.93e-09	0	\\
3.04e-09	0	\\
3.14e-09	0	\\
3.24e-09	0	\\
3.34e-09	0	\\
3.45e-09	0	\\
3.55e-09	0	\\
3.65e-09	0	\\
3.75e-09	0	\\
3.86e-09	0	\\
3.96e-09	0	\\
4.06e-09	0	\\
4.16e-09	0	\\
4.27e-09	0	\\
4.37e-09	0	\\
4.47e-09	0	\\
4.57e-09	0	\\
4.68e-09	0	\\
4.78e-09	0	\\
4.89e-09	0	\\
4.99e-09	0	\\
5e-09	0	\\
};
\addplot [color=mycolor1,solid,forget plot]
  table[row sep=crcr]{
0	0	\\
1.1e-10	0	\\
2.2e-10	0	\\
3.3e-10	0	\\
4.4e-10	0	\\
5.4e-10	0	\\
6.5e-10	0	\\
7.5e-10	0	\\
8.6e-10	0	\\
9.6e-10	0	\\
1.07e-09	0	\\
1.18e-09	0	\\
1.28e-09	0	\\
1.38e-09	0	\\
1.49e-09	0	\\
1.59e-09	0	\\
1.69e-09	0	\\
1.8e-09	0	\\
1.9e-09	0	\\
2.01e-09	0	\\
2.11e-09	0	\\
2.21e-09	0	\\
2.32e-09	0	\\
2.42e-09	0	\\
2.52e-09	0	\\
2.63e-09	0	\\
2.73e-09	0	\\
2.83e-09	0	\\
2.93e-09	0	\\
3.04e-09	0	\\
3.14e-09	0	\\
3.24e-09	0	\\
3.34e-09	0	\\
3.45e-09	0	\\
3.55e-09	0	\\
3.65e-09	0	\\
3.75e-09	0	\\
3.86e-09	0	\\
3.96e-09	0	\\
4.06e-09	0	\\
4.16e-09	0	\\
4.27e-09	0	\\
4.37e-09	0	\\
4.47e-09	0	\\
4.57e-09	0	\\
4.68e-09	0	\\
4.78e-09	0	\\
4.89e-09	0	\\
4.99e-09	0	\\
5e-09	0	\\
};
\addplot [color=mycolor2,solid,forget plot]
  table[row sep=crcr]{
0	0	\\
1.1e-10	0	\\
2.2e-10	0	\\
3.3e-10	0	\\
4.4e-10	0	\\
5.4e-10	0	\\
6.5e-10	0	\\
7.5e-10	0	\\
8.6e-10	0	\\
9.6e-10	0	\\
1.07e-09	0	\\
1.18e-09	0	\\
1.28e-09	0	\\
1.38e-09	0	\\
1.49e-09	0	\\
1.59e-09	0	\\
1.69e-09	0	\\
1.8e-09	0	\\
1.9e-09	0	\\
2.01e-09	0	\\
2.11e-09	0	\\
2.21e-09	0	\\
2.32e-09	0	\\
2.42e-09	0	\\
2.52e-09	0	\\
2.63e-09	0	\\
2.73e-09	0	\\
2.83e-09	0	\\
2.93e-09	0	\\
3.04e-09	0	\\
3.14e-09	0	\\
3.24e-09	0	\\
3.34e-09	0	\\
3.45e-09	0	\\
3.55e-09	0	\\
3.65e-09	0	\\
3.75e-09	0	\\
3.86e-09	0	\\
3.96e-09	0	\\
4.06e-09	0	\\
4.16e-09	0	\\
4.27e-09	0	\\
4.37e-09	0	\\
4.47e-09	0	\\
4.57e-09	0	\\
4.68e-09	0	\\
4.78e-09	0	\\
4.89e-09	0	\\
4.99e-09	0	\\
5e-09	0	\\
};
\addplot [color=mycolor3,solid,forget plot]
  table[row sep=crcr]{
0	0	\\
1.1e-10	0	\\
2.2e-10	0	\\
3.3e-10	0	\\
4.4e-10	0	\\
5.4e-10	0	\\
6.5e-10	0	\\
7.5e-10	0	\\
8.6e-10	0	\\
9.6e-10	0	\\
1.07e-09	0	\\
1.18e-09	0	\\
1.28e-09	0	\\
1.38e-09	0	\\
1.49e-09	0	\\
1.59e-09	0	\\
1.69e-09	0	\\
1.8e-09	0	\\
1.9e-09	0	\\
2.01e-09	0	\\
2.11e-09	0	\\
2.21e-09	0	\\
2.32e-09	0	\\
2.42e-09	0	\\
2.52e-09	0	\\
2.63e-09	0	\\
2.73e-09	0	\\
2.83e-09	0	\\
2.93e-09	0	\\
3.04e-09	0	\\
3.14e-09	0	\\
3.24e-09	0	\\
3.34e-09	0	\\
3.45e-09	0	\\
3.55e-09	0	\\
3.65e-09	0	\\
3.75e-09	0	\\
3.86e-09	0	\\
3.96e-09	0	\\
4.06e-09	0	\\
4.16e-09	0	\\
4.27e-09	0	\\
4.37e-09	0	\\
4.47e-09	0	\\
4.57e-09	0	\\
4.68e-09	0	\\
4.78e-09	0	\\
4.89e-09	0	\\
4.99e-09	0	\\
5e-09	0	\\
};
\addplot [color=darkgray,solid,forget plot]
  table[row sep=crcr]{
0	0	\\
1.1e-10	0	\\
2.2e-10	0	\\
3.3e-10	0	\\
4.4e-10	0	\\
5.4e-10	0	\\
6.5e-10	0	\\
7.5e-10	0	\\
8.6e-10	0	\\
9.6e-10	0	\\
1.07e-09	0	\\
1.18e-09	0	\\
1.28e-09	0	\\
1.38e-09	0	\\
1.49e-09	0	\\
1.59e-09	0	\\
1.69e-09	0	\\
1.8e-09	0	\\
1.9e-09	0	\\
2.01e-09	0	\\
2.11e-09	0	\\
2.21e-09	0	\\
2.32e-09	0	\\
2.42e-09	0	\\
2.52e-09	0	\\
2.63e-09	0	\\
2.73e-09	0	\\
2.83e-09	0	\\
2.93e-09	0	\\
3.04e-09	0	\\
3.14e-09	0	\\
3.24e-09	0	\\
3.34e-09	0	\\
3.45e-09	0	\\
3.55e-09	0	\\
3.65e-09	0	\\
3.75e-09	0	\\
3.86e-09	0	\\
3.96e-09	0	\\
4.06e-09	0	\\
4.16e-09	0	\\
4.27e-09	0	\\
4.37e-09	0	\\
4.47e-09	0	\\
4.57e-09	0	\\
4.68e-09	0	\\
4.78e-09	0	\\
4.89e-09	0	\\
4.99e-09	0	\\
5e-09	0	\\
};
\addplot [color=blue,solid,forget plot]
  table[row sep=crcr]{
0	0	\\
1.1e-10	0	\\
2.2e-10	0	\\
3.3e-10	0	\\
4.4e-10	0	\\
5.4e-10	0	\\
6.5e-10	0	\\
7.5e-10	0	\\
8.6e-10	0	\\
9.6e-10	0	\\
1.07e-09	0	\\
1.18e-09	0	\\
1.28e-09	0	\\
1.38e-09	0	\\
1.49e-09	0	\\
1.59e-09	0	\\
1.69e-09	0	\\
1.8e-09	0	\\
1.9e-09	0	\\
2.01e-09	0	\\
2.11e-09	0	\\
2.21e-09	0	\\
2.32e-09	0	\\
2.42e-09	0	\\
2.52e-09	0	\\
2.63e-09	0	\\
2.73e-09	0	\\
2.83e-09	0	\\
2.93e-09	0	\\
3.04e-09	0	\\
3.14e-09	0	\\
3.24e-09	0	\\
3.34e-09	0	\\
3.45e-09	0	\\
3.55e-09	0	\\
3.65e-09	0	\\
3.75e-09	0	\\
3.86e-09	0	\\
3.96e-09	0	\\
4.06e-09	0	\\
4.16e-09	0	\\
4.27e-09	0	\\
4.37e-09	0	\\
4.47e-09	0	\\
4.57e-09	0	\\
4.68e-09	0	\\
4.78e-09	0	\\
4.89e-09	0	\\
4.99e-09	0	\\
5e-09	0	\\
};
\addplot [color=black!50!green,solid,forget plot]
  table[row sep=crcr]{
0	0	\\
1.1e-10	0	\\
2.2e-10	0	\\
3.3e-10	0	\\
4.4e-10	0	\\
5.4e-10	0	\\
6.5e-10	0	\\
7.5e-10	0	\\
8.6e-10	0	\\
9.6e-10	0	\\
1.07e-09	0	\\
1.18e-09	0	\\
1.28e-09	0	\\
1.38e-09	0	\\
1.49e-09	0	\\
1.59e-09	0	\\
1.69e-09	0	\\
1.8e-09	0	\\
1.9e-09	0	\\
2.01e-09	0	\\
2.11e-09	0	\\
2.21e-09	0	\\
2.32e-09	0	\\
2.42e-09	0	\\
2.52e-09	0	\\
2.63e-09	0	\\
2.73e-09	0	\\
2.83e-09	0	\\
2.93e-09	0	\\
3.04e-09	0	\\
3.14e-09	0	\\
3.24e-09	0	\\
3.34e-09	0	\\
3.45e-09	0	\\
3.55e-09	0	\\
3.65e-09	0	\\
3.75e-09	0	\\
3.86e-09	0	\\
3.96e-09	0	\\
4.06e-09	0	\\
4.16e-09	0	\\
4.27e-09	0	\\
4.37e-09	0	\\
4.47e-09	0	\\
4.57e-09	0	\\
4.68e-09	0	\\
4.78e-09	0	\\
4.89e-09	0	\\
4.99e-09	0	\\
5e-09	0	\\
};
\addplot [color=red,solid,forget plot]
  table[row sep=crcr]{
0	0	\\
1.1e-10	0	\\
2.2e-10	0	\\
3.3e-10	0	\\
4.4e-10	0	\\
5.4e-10	0	\\
6.5e-10	0	\\
7.5e-10	0	\\
8.6e-10	0	\\
9.6e-10	0	\\
1.07e-09	0	\\
1.18e-09	0	\\
1.28e-09	0	\\
1.38e-09	0	\\
1.49e-09	0	\\
1.59e-09	0	\\
1.69e-09	0	\\
1.8e-09	0	\\
1.9e-09	0	\\
2.01e-09	0	\\
2.11e-09	0	\\
2.21e-09	0	\\
2.32e-09	0	\\
2.42e-09	0	\\
2.52e-09	0	\\
2.63e-09	0	\\
2.73e-09	0	\\
2.83e-09	0	\\
2.93e-09	0	\\
3.04e-09	0	\\
3.14e-09	0	\\
3.24e-09	0	\\
3.34e-09	0	\\
3.45e-09	0	\\
3.55e-09	0	\\
3.65e-09	0	\\
3.75e-09	0	\\
3.86e-09	0	\\
3.96e-09	0	\\
4.06e-09	0	\\
4.16e-09	0	\\
4.27e-09	0	\\
4.37e-09	0	\\
4.47e-09	0	\\
4.57e-09	0	\\
4.68e-09	0	\\
4.78e-09	0	\\
4.89e-09	0	\\
4.99e-09	0	\\
5e-09	0	\\
};
\addplot [color=mycolor1,solid,forget plot]
  table[row sep=crcr]{
0	0	\\
1.1e-10	0	\\
2.2e-10	0	\\
3.3e-10	0	\\
4.4e-10	0	\\
5.4e-10	0	\\
6.5e-10	0	\\
7.5e-10	0	\\
8.6e-10	0	\\
9.6e-10	0	\\
1.07e-09	0	\\
1.18e-09	0	\\
1.28e-09	0	\\
1.38e-09	0	\\
1.49e-09	0	\\
1.59e-09	0	\\
1.69e-09	0	\\
1.8e-09	0	\\
1.9e-09	0	\\
2.01e-09	0	\\
2.11e-09	0	\\
2.21e-09	0	\\
2.32e-09	0	\\
2.42e-09	0	\\
2.52e-09	0	\\
2.63e-09	0	\\
2.73e-09	0	\\
2.83e-09	0	\\
2.93e-09	0	\\
3.04e-09	0	\\
3.14e-09	0	\\
3.24e-09	0	\\
3.34e-09	0	\\
3.45e-09	0	\\
3.55e-09	0	\\
3.65e-09	0	\\
3.75e-09	0	\\
3.86e-09	0	\\
3.96e-09	0	\\
4.06e-09	0	\\
4.16e-09	0	\\
4.27e-09	0	\\
4.37e-09	0	\\
4.47e-09	0	\\
4.57e-09	0	\\
4.68e-09	0	\\
4.78e-09	0	\\
4.89e-09	0	\\
4.99e-09	0	\\
5e-09	0	\\
};
\addplot [color=mycolor2,solid,forget plot]
  table[row sep=crcr]{
0	0	\\
1.1e-10	0	\\
2.2e-10	0	\\
3.3e-10	0	\\
4.4e-10	0	\\
5.4e-10	0	\\
6.5e-10	0	\\
7.5e-10	0	\\
8.6e-10	0	\\
9.6e-10	0	\\
1.07e-09	0	\\
1.18e-09	0	\\
1.28e-09	0	\\
1.38e-09	0	\\
1.49e-09	0	\\
1.59e-09	0	\\
1.69e-09	0	\\
1.8e-09	0	\\
1.9e-09	0	\\
2.01e-09	0	\\
2.11e-09	0	\\
2.21e-09	0	\\
2.32e-09	0	\\
2.42e-09	0	\\
2.52e-09	0	\\
2.63e-09	0	\\
2.73e-09	0	\\
2.83e-09	0	\\
2.93e-09	0	\\
3.04e-09	0	\\
3.14e-09	0	\\
3.24e-09	0	\\
3.34e-09	0	\\
3.45e-09	0	\\
3.55e-09	0	\\
3.65e-09	0	\\
3.75e-09	0	\\
3.86e-09	0	\\
3.96e-09	0	\\
4.06e-09	0	\\
4.16e-09	0	\\
4.27e-09	0	\\
4.37e-09	0	\\
4.47e-09	0	\\
4.57e-09	0	\\
4.68e-09	0	\\
4.78e-09	0	\\
4.89e-09	0	\\
4.99e-09	0	\\
5e-09	0	\\
};
\addplot [color=mycolor3,solid,forget plot]
  table[row sep=crcr]{
0	0	\\
1.1e-10	0	\\
2.2e-10	0	\\
3.3e-10	0	\\
4.4e-10	0	\\
5.4e-10	0	\\
6.5e-10	0	\\
7.5e-10	0	\\
8.6e-10	0	\\
9.6e-10	0	\\
1.07e-09	0	\\
1.18e-09	0	\\
1.28e-09	0	\\
1.38e-09	0	\\
1.49e-09	0	\\
1.59e-09	0	\\
1.69e-09	0	\\
1.8e-09	0	\\
1.9e-09	0	\\
2.01e-09	0	\\
2.11e-09	0	\\
2.21e-09	0	\\
2.32e-09	0	\\
2.42e-09	0	\\
2.52e-09	0	\\
2.63e-09	0	\\
2.73e-09	0	\\
2.83e-09	0	\\
2.93e-09	0	\\
3.04e-09	0	\\
3.14e-09	0	\\
3.24e-09	0	\\
3.34e-09	0	\\
3.45e-09	0	\\
3.55e-09	0	\\
3.65e-09	0	\\
3.75e-09	0	\\
3.86e-09	0	\\
3.96e-09	0	\\
4.06e-09	0	\\
4.16e-09	0	\\
4.27e-09	0	\\
4.37e-09	0	\\
4.47e-09	0	\\
4.57e-09	0	\\
4.68e-09	0	\\
4.78e-09	0	\\
4.89e-09	0	\\
4.99e-09	0	\\
5e-09	0	\\
};
\addplot [color=darkgray,solid,forget plot]
  table[row sep=crcr]{
0	0	\\
1.1e-10	0	\\
2.2e-10	0	\\
3.3e-10	0	\\
4.4e-10	0	\\
5.4e-10	0	\\
6.5e-10	0	\\
7.5e-10	0	\\
8.6e-10	0	\\
9.6e-10	0	\\
1.07e-09	0	\\
1.18e-09	0	\\
1.28e-09	0	\\
1.38e-09	0	\\
1.49e-09	0	\\
1.59e-09	0	\\
1.69e-09	0	\\
1.8e-09	0	\\
1.9e-09	0	\\
2.01e-09	0	\\
2.11e-09	0	\\
2.21e-09	0	\\
2.32e-09	0	\\
2.42e-09	0	\\
2.52e-09	0	\\
2.63e-09	0	\\
2.73e-09	0	\\
2.83e-09	0	\\
2.93e-09	0	\\
3.04e-09	0	\\
3.14e-09	0	\\
3.24e-09	0	\\
3.34e-09	0	\\
3.45e-09	0	\\
3.55e-09	0	\\
3.65e-09	0	\\
3.75e-09	0	\\
3.86e-09	0	\\
3.96e-09	0	\\
4.06e-09	0	\\
4.16e-09	0	\\
4.27e-09	0	\\
4.37e-09	0	\\
4.47e-09	0	\\
4.57e-09	0	\\
4.68e-09	0	\\
4.78e-09	0	\\
4.89e-09	0	\\
4.99e-09	0	\\
5e-09	0	\\
};
\addplot [color=blue,solid,forget plot]
  table[row sep=crcr]{
0	0	\\
1.1e-10	0	\\
2.2e-10	0	\\
3.3e-10	0	\\
4.4e-10	0	\\
5.4e-10	0	\\
6.5e-10	0	\\
7.5e-10	0	\\
8.6e-10	0	\\
9.6e-10	0	\\
1.07e-09	0	\\
1.18e-09	0	\\
1.28e-09	0	\\
1.38e-09	0	\\
1.49e-09	0	\\
1.59e-09	0	\\
1.69e-09	0	\\
1.8e-09	0	\\
1.9e-09	0	\\
2.01e-09	0	\\
2.11e-09	0	\\
2.21e-09	0	\\
2.32e-09	0	\\
2.42e-09	0	\\
2.52e-09	0	\\
2.63e-09	0	\\
2.73e-09	0	\\
2.83e-09	0	\\
2.93e-09	0	\\
3.04e-09	0	\\
3.14e-09	0	\\
3.24e-09	0	\\
3.34e-09	0	\\
3.45e-09	0	\\
3.55e-09	0	\\
3.65e-09	0	\\
3.75e-09	0	\\
3.86e-09	0	\\
3.96e-09	0	\\
4.06e-09	0	\\
4.16e-09	0	\\
4.27e-09	0	\\
4.37e-09	0	\\
4.47e-09	0	\\
4.57e-09	0	\\
4.68e-09	0	\\
4.78e-09	0	\\
4.89e-09	0	\\
4.99e-09	0	\\
5e-09	0	\\
};
\addplot [color=black!50!green,solid,forget plot]
  table[row sep=crcr]{
0	0	\\
1.1e-10	0	\\
2.2e-10	0	\\
3.3e-10	0	\\
4.4e-10	0	\\
5.4e-10	0	\\
6.5e-10	0	\\
7.5e-10	0	\\
8.6e-10	0	\\
9.6e-10	0	\\
1.07e-09	0	\\
1.18e-09	0	\\
1.28e-09	0	\\
1.38e-09	0	\\
1.49e-09	0	\\
1.59e-09	0	\\
1.69e-09	0	\\
1.8e-09	0	\\
1.9e-09	0	\\
2.01e-09	0	\\
2.11e-09	0	\\
2.21e-09	0	\\
2.32e-09	0	\\
2.42e-09	0	\\
2.52e-09	0	\\
2.63e-09	0	\\
2.73e-09	0	\\
2.83e-09	0	\\
2.93e-09	0	\\
3.04e-09	0	\\
3.14e-09	0	\\
3.24e-09	0	\\
3.34e-09	0	\\
3.45e-09	0	\\
3.55e-09	0	\\
3.65e-09	0	\\
3.75e-09	0	\\
3.86e-09	0	\\
3.96e-09	0	\\
4.06e-09	0	\\
4.16e-09	0	\\
4.27e-09	0	\\
4.37e-09	0	\\
4.47e-09	0	\\
4.57e-09	0	\\
4.68e-09	0	\\
4.78e-09	0	\\
4.89e-09	0	\\
4.99e-09	0	\\
5e-09	0	\\
};
\addplot [color=red,solid,forget plot]
  table[row sep=crcr]{
0	0	\\
1.1e-10	0	\\
2.2e-10	0	\\
3.3e-10	0	\\
4.4e-10	0	\\
5.4e-10	0	\\
6.5e-10	0	\\
7.5e-10	0	\\
8.6e-10	0	\\
9.6e-10	0	\\
1.07e-09	0	\\
1.18e-09	0	\\
1.28e-09	0	\\
1.38e-09	0	\\
1.49e-09	0	\\
1.59e-09	0	\\
1.69e-09	0	\\
1.8e-09	0	\\
1.9e-09	0	\\
2.01e-09	0	\\
2.11e-09	0	\\
2.21e-09	0	\\
2.32e-09	0	\\
2.42e-09	0	\\
2.52e-09	0	\\
2.63e-09	0	\\
2.73e-09	0	\\
2.83e-09	0	\\
2.93e-09	0	\\
3.04e-09	0	\\
3.14e-09	0	\\
3.24e-09	0	\\
3.34e-09	0	\\
3.45e-09	0	\\
3.55e-09	0	\\
3.65e-09	0	\\
3.75e-09	0	\\
3.86e-09	0	\\
3.96e-09	0	\\
4.06e-09	0	\\
4.16e-09	0	\\
4.27e-09	0	\\
4.37e-09	0	\\
4.47e-09	0	\\
4.57e-09	0	\\
4.68e-09	0	\\
4.78e-09	0	\\
4.89e-09	0	\\
4.99e-09	0	\\
5e-09	0	\\
};
\addplot [color=mycolor1,solid,forget plot]
  table[row sep=crcr]{
0	0	\\
1.1e-10	0	\\
2.2e-10	0	\\
3.3e-10	0	\\
4.4e-10	0	\\
5.4e-10	0	\\
6.5e-10	0	\\
7.5e-10	0	\\
8.6e-10	0	\\
9.6e-10	0	\\
1.07e-09	0	\\
1.18e-09	0	\\
1.28e-09	0	\\
1.38e-09	0	\\
1.49e-09	0	\\
1.59e-09	0	\\
1.69e-09	0	\\
1.8e-09	0	\\
1.9e-09	0	\\
2.01e-09	0	\\
2.11e-09	0	\\
2.21e-09	0	\\
2.32e-09	0	\\
2.42e-09	0	\\
2.52e-09	0	\\
2.63e-09	0	\\
2.73e-09	0	\\
2.83e-09	0	\\
2.93e-09	0	\\
3.04e-09	0	\\
3.14e-09	0	\\
3.24e-09	0	\\
3.34e-09	0	\\
3.45e-09	0	\\
3.55e-09	0	\\
3.65e-09	0	\\
3.75e-09	0	\\
3.86e-09	0	\\
3.96e-09	0	\\
4.06e-09	0	\\
4.16e-09	0	\\
4.27e-09	0	\\
4.37e-09	0	\\
4.47e-09	0	\\
4.57e-09	0	\\
4.68e-09	0	\\
4.78e-09	0	\\
4.89e-09	0	\\
4.99e-09	0	\\
5e-09	0	\\
};
\addplot [color=mycolor2,solid,forget plot]
  table[row sep=crcr]{
0	0	\\
1.1e-10	0	\\
2.2e-10	0	\\
3.3e-10	0	\\
4.4e-10	0	\\
5.4e-10	0	\\
6.5e-10	0	\\
7.5e-10	0	\\
8.6e-10	0	\\
9.6e-10	0	\\
1.07e-09	0	\\
1.18e-09	0	\\
1.28e-09	0	\\
1.38e-09	0	\\
1.49e-09	0	\\
1.59e-09	0	\\
1.69e-09	0	\\
1.8e-09	0	\\
1.9e-09	0	\\
2.01e-09	0	\\
2.11e-09	0	\\
2.21e-09	0	\\
2.32e-09	0	\\
2.42e-09	0	\\
2.52e-09	0	\\
2.63e-09	0	\\
2.73e-09	0	\\
2.83e-09	0	\\
2.93e-09	0	\\
3.04e-09	0	\\
3.14e-09	0	\\
3.24e-09	0	\\
3.34e-09	0	\\
3.45e-09	0	\\
3.55e-09	0	\\
3.65e-09	0	\\
3.75e-09	0	\\
3.86e-09	0	\\
3.96e-09	0	\\
4.06e-09	0	\\
4.16e-09	0	\\
4.27e-09	0	\\
4.37e-09	0	\\
4.47e-09	0	\\
4.57e-09	0	\\
4.68e-09	0	\\
4.78e-09	0	\\
4.89e-09	0	\\
4.99e-09	0	\\
5e-09	0	\\
};
\addplot [color=mycolor3,solid,forget plot]
  table[row sep=crcr]{
0	0	\\
1.1e-10	0	\\
2.2e-10	0	\\
3.3e-10	0	\\
4.4e-10	0	\\
5.4e-10	0	\\
6.5e-10	0	\\
7.5e-10	0	\\
8.6e-10	0	\\
9.6e-10	0	\\
1.07e-09	0	\\
1.18e-09	0	\\
1.28e-09	0	\\
1.38e-09	0	\\
1.49e-09	0	\\
1.59e-09	0	\\
1.69e-09	0	\\
1.8e-09	0	\\
1.9e-09	0	\\
2.01e-09	0	\\
2.11e-09	0	\\
2.21e-09	0	\\
2.32e-09	0	\\
2.42e-09	0	\\
2.52e-09	0	\\
2.63e-09	0	\\
2.73e-09	0	\\
2.83e-09	0	\\
2.93e-09	0	\\
3.04e-09	0	\\
3.14e-09	0	\\
3.24e-09	0	\\
3.34e-09	0	\\
3.45e-09	0	\\
3.55e-09	0	\\
3.65e-09	0	\\
3.75e-09	0	\\
3.86e-09	0	\\
3.96e-09	0	\\
4.06e-09	0	\\
4.16e-09	0	\\
4.27e-09	0	\\
4.37e-09	0	\\
4.47e-09	0	\\
4.57e-09	0	\\
4.68e-09	0	\\
4.78e-09	0	\\
4.89e-09	0	\\
4.99e-09	0	\\
5e-09	0	\\
};
\addplot [color=darkgray,solid,forget plot]
  table[row sep=crcr]{
0	0	\\
1.1e-10	0	\\
2.2e-10	0	\\
3.3e-10	0	\\
4.4e-10	0	\\
5.4e-10	0	\\
6.5e-10	0	\\
7.5e-10	0	\\
8.6e-10	0	\\
9.6e-10	0	\\
1.07e-09	0	\\
1.18e-09	0	\\
1.28e-09	0	\\
1.38e-09	0	\\
1.49e-09	0	\\
1.59e-09	0	\\
1.69e-09	0	\\
1.8e-09	0	\\
1.9e-09	0	\\
2.01e-09	0	\\
2.11e-09	0	\\
2.21e-09	0	\\
2.32e-09	0	\\
2.42e-09	0	\\
2.52e-09	0	\\
2.63e-09	0	\\
2.73e-09	0	\\
2.83e-09	0	\\
2.93e-09	0	\\
3.04e-09	0	\\
3.14e-09	0	\\
3.24e-09	0	\\
3.34e-09	0	\\
3.45e-09	0	\\
3.55e-09	0	\\
3.65e-09	0	\\
3.75e-09	0	\\
3.86e-09	0	\\
3.96e-09	0	\\
4.06e-09	0	\\
4.16e-09	0	\\
4.27e-09	0	\\
4.37e-09	0	\\
4.47e-09	0	\\
4.57e-09	0	\\
4.68e-09	0	\\
4.78e-09	0	\\
4.89e-09	0	\\
4.99e-09	0	\\
5e-09	0	\\
};
\addplot [color=blue,solid,forget plot]
  table[row sep=crcr]{
0	0	\\
1.1e-10	0	\\
2.2e-10	0	\\
3.3e-10	0	\\
4.4e-10	0	\\
5.4e-10	0	\\
6.5e-10	0	\\
7.5e-10	0	\\
8.6e-10	0	\\
9.6e-10	0	\\
1.07e-09	0	\\
1.18e-09	0	\\
1.28e-09	0	\\
1.38e-09	0	\\
1.49e-09	0	\\
1.59e-09	0	\\
1.69e-09	0	\\
1.8e-09	0	\\
1.9e-09	0	\\
2.01e-09	0	\\
2.11e-09	0	\\
2.21e-09	0	\\
2.32e-09	0	\\
2.42e-09	0	\\
2.52e-09	0	\\
2.63e-09	0	\\
2.73e-09	0	\\
2.83e-09	0	\\
2.93e-09	0	\\
3.04e-09	0	\\
3.14e-09	0	\\
3.24e-09	0	\\
3.34e-09	0	\\
3.45e-09	0	\\
3.55e-09	0	\\
3.65e-09	0	\\
3.75e-09	0	\\
3.86e-09	0	\\
3.96e-09	0	\\
4.06e-09	0	\\
4.16e-09	0	\\
4.27e-09	0	\\
4.37e-09	0	\\
4.47e-09	0	\\
4.57e-09	0	\\
4.68e-09	0	\\
4.78e-09	0	\\
4.89e-09	0	\\
4.99e-09	0	\\
5e-09	0	\\
};
\addplot [color=black!50!green,solid,forget plot]
  table[row sep=crcr]{
0	0	\\
1.1e-10	0	\\
2.2e-10	0	\\
3.3e-10	0	\\
4.4e-10	0	\\
5.4e-10	0	\\
6.5e-10	0	\\
7.5e-10	0	\\
8.6e-10	0	\\
9.6e-10	0	\\
1.07e-09	0	\\
1.18e-09	0	\\
1.28e-09	0	\\
1.38e-09	0	\\
1.49e-09	0	\\
1.59e-09	0	\\
1.69e-09	0	\\
1.8e-09	0	\\
1.9e-09	0	\\
2.01e-09	0	\\
2.11e-09	0	\\
2.21e-09	0	\\
2.32e-09	0	\\
2.42e-09	0	\\
2.52e-09	0	\\
2.63e-09	0	\\
2.73e-09	0	\\
2.83e-09	0	\\
2.93e-09	0	\\
3.04e-09	0	\\
3.14e-09	0	\\
3.24e-09	0	\\
3.34e-09	0	\\
3.45e-09	0	\\
3.55e-09	0	\\
3.65e-09	0	\\
3.75e-09	0	\\
3.86e-09	0	\\
3.96e-09	0	\\
4.06e-09	0	\\
4.16e-09	0	\\
4.27e-09	0	\\
4.37e-09	0	\\
4.47e-09	0	\\
4.57e-09	0	\\
4.68e-09	0	\\
4.78e-09	0	\\
4.89e-09	0	\\
4.99e-09	0	\\
5e-09	0	\\
};
\addplot [color=red,solid,forget plot]
  table[row sep=crcr]{
0	0	\\
1.1e-10	0	\\
2.2e-10	0	\\
3.3e-10	0	\\
4.4e-10	0	\\
5.4e-10	0	\\
6.5e-10	0	\\
7.5e-10	0	\\
8.6e-10	0	\\
9.6e-10	0	\\
1.07e-09	0	\\
1.18e-09	0	\\
1.28e-09	0	\\
1.38e-09	0	\\
1.49e-09	0	\\
1.59e-09	0	\\
1.69e-09	0	\\
1.8e-09	0	\\
1.9e-09	0	\\
2.01e-09	0	\\
2.11e-09	0	\\
2.21e-09	0	\\
2.32e-09	0	\\
2.42e-09	0	\\
2.52e-09	0	\\
2.63e-09	0	\\
2.73e-09	0	\\
2.83e-09	0	\\
2.93e-09	0	\\
3.04e-09	0	\\
3.14e-09	0	\\
3.24e-09	0	\\
3.34e-09	0	\\
3.45e-09	0	\\
3.55e-09	0	\\
3.65e-09	0	\\
3.75e-09	0	\\
3.86e-09	0	\\
3.96e-09	0	\\
4.06e-09	0	\\
4.16e-09	0	\\
4.27e-09	0	\\
4.37e-09	0	\\
4.47e-09	0	\\
4.57e-09	0	\\
4.68e-09	0	\\
4.78e-09	0	\\
4.89e-09	0	\\
4.99e-09	0	\\
5e-09	0	\\
};
\addplot [color=mycolor1,solid,forget plot]
  table[row sep=crcr]{
0	0	\\
1.1e-10	0	\\
2.2e-10	0	\\
3.3e-10	0	\\
4.4e-10	0	\\
5.4e-10	0	\\
6.5e-10	0	\\
7.5e-10	0	\\
8.6e-10	0	\\
9.6e-10	0	\\
1.07e-09	0	\\
1.18e-09	0	\\
1.28e-09	0	\\
1.38e-09	0	\\
1.49e-09	0	\\
1.59e-09	0	\\
1.69e-09	0	\\
1.8e-09	0	\\
1.9e-09	0	\\
2.01e-09	0	\\
2.11e-09	0	\\
2.21e-09	0	\\
2.32e-09	0	\\
2.42e-09	0	\\
2.52e-09	0	\\
2.63e-09	0	\\
2.73e-09	0	\\
2.83e-09	0	\\
2.93e-09	0	\\
3.04e-09	0	\\
3.14e-09	0	\\
3.24e-09	0	\\
3.34e-09	0	\\
3.45e-09	0	\\
3.55e-09	0	\\
3.65e-09	0	\\
3.75e-09	0	\\
3.86e-09	0	\\
3.96e-09	0	\\
4.06e-09	0	\\
4.16e-09	0	\\
4.27e-09	0	\\
4.37e-09	0	\\
4.47e-09	0	\\
4.57e-09	0	\\
4.68e-09	0	\\
4.78e-09	0	\\
4.89e-09	0	\\
4.99e-09	0	\\
5e-09	0	\\
};
\addplot [color=mycolor2,solid,forget plot]
  table[row sep=crcr]{
0	0	\\
1.1e-10	0	\\
2.2e-10	0	\\
3.3e-10	0	\\
4.4e-10	0	\\
5.4e-10	0	\\
6.5e-10	0	\\
7.5e-10	0	\\
8.6e-10	0	\\
9.6e-10	0	\\
1.07e-09	0	\\
1.18e-09	0	\\
1.28e-09	0	\\
1.38e-09	0	\\
1.49e-09	0	\\
1.59e-09	0	\\
1.69e-09	0	\\
1.8e-09	0	\\
1.9e-09	0	\\
2.01e-09	0	\\
2.11e-09	0	\\
2.21e-09	0	\\
2.32e-09	0	\\
2.42e-09	0	\\
2.52e-09	0	\\
2.63e-09	0	\\
2.73e-09	0	\\
2.83e-09	0	\\
2.93e-09	0	\\
3.04e-09	0	\\
3.14e-09	0	\\
3.24e-09	0	\\
3.34e-09	0	\\
3.45e-09	0	\\
3.55e-09	0	\\
3.65e-09	0	\\
3.75e-09	0	\\
3.86e-09	0	\\
3.96e-09	0	\\
4.06e-09	0	\\
4.16e-09	0	\\
4.27e-09	0	\\
4.37e-09	0	\\
4.47e-09	0	\\
4.57e-09	0	\\
4.68e-09	0	\\
4.78e-09	0	\\
4.89e-09	0	\\
4.99e-09	0	\\
5e-09	0	\\
};
\addplot [color=mycolor3,solid,forget plot]
  table[row sep=crcr]{
0	0	\\
1.1e-10	0	\\
2.2e-10	0	\\
3.3e-10	0	\\
4.4e-10	0	\\
5.4e-10	0	\\
6.5e-10	0	\\
7.5e-10	0	\\
8.6e-10	0	\\
9.6e-10	0	\\
1.07e-09	0	\\
1.18e-09	0	\\
1.28e-09	0	\\
1.38e-09	0	\\
1.49e-09	0	\\
1.59e-09	0	\\
1.69e-09	0	\\
1.8e-09	0	\\
1.9e-09	0	\\
2.01e-09	0	\\
2.11e-09	0	\\
2.21e-09	0	\\
2.32e-09	0	\\
2.42e-09	0	\\
2.52e-09	0	\\
2.63e-09	0	\\
2.73e-09	0	\\
2.83e-09	0	\\
2.93e-09	0	\\
3.04e-09	0	\\
3.14e-09	0	\\
3.24e-09	0	\\
3.34e-09	0	\\
3.45e-09	0	\\
3.55e-09	0	\\
3.65e-09	0	\\
3.75e-09	0	\\
3.86e-09	0	\\
3.96e-09	0	\\
4.06e-09	0	\\
4.16e-09	0	\\
4.27e-09	0	\\
4.37e-09	0	\\
4.47e-09	0	\\
4.57e-09	0	\\
4.68e-09	0	\\
4.78e-09	0	\\
4.89e-09	0	\\
4.99e-09	0	\\
5e-09	0	\\
};
\addplot [color=darkgray,solid,forget plot]
  table[row sep=crcr]{
0	0	\\
1.1e-10	0	\\
2.2e-10	0	\\
3.3e-10	0	\\
4.4e-10	0	\\
5.4e-10	0	\\
6.5e-10	0	\\
7.5e-10	0	\\
8.6e-10	0	\\
9.6e-10	0	\\
1.07e-09	0	\\
1.18e-09	0	\\
1.28e-09	0	\\
1.38e-09	0	\\
1.49e-09	0	\\
1.59e-09	0	\\
1.69e-09	0	\\
1.8e-09	0	\\
1.9e-09	0	\\
2.01e-09	0	\\
2.11e-09	0	\\
2.21e-09	0	\\
2.32e-09	0	\\
2.42e-09	0	\\
2.52e-09	0	\\
2.63e-09	0	\\
2.73e-09	0	\\
2.83e-09	0	\\
2.93e-09	0	\\
3.04e-09	0	\\
3.14e-09	0	\\
3.24e-09	0	\\
3.34e-09	0	\\
3.45e-09	0	\\
3.55e-09	0	\\
3.65e-09	0	\\
3.75e-09	0	\\
3.86e-09	0	\\
3.96e-09	0	\\
4.06e-09	0	\\
4.16e-09	0	\\
4.27e-09	0	\\
4.37e-09	0	\\
4.47e-09	0	\\
4.57e-09	0	\\
4.68e-09	0	\\
4.78e-09	0	\\
4.89e-09	0	\\
4.99e-09	0	\\
5e-09	0	\\
};
\addplot [color=blue,solid,forget plot]
  table[row sep=crcr]{
0	0	\\
1.1e-10	0	\\
2.2e-10	0	\\
3.3e-10	0	\\
4.4e-10	0	\\
5.4e-10	0	\\
6.5e-10	0	\\
7.5e-10	0	\\
8.6e-10	0	\\
9.6e-10	0	\\
1.07e-09	0	\\
1.18e-09	0	\\
1.28e-09	0	\\
1.38e-09	0	\\
1.49e-09	0	\\
1.59e-09	0	\\
1.69e-09	0	\\
1.8e-09	0	\\
1.9e-09	0	\\
2.01e-09	0	\\
2.11e-09	0	\\
2.21e-09	0	\\
2.32e-09	0	\\
2.42e-09	0	\\
2.52e-09	0	\\
2.63e-09	0	\\
2.73e-09	0	\\
2.83e-09	0	\\
2.93e-09	0	\\
3.04e-09	0	\\
3.14e-09	0	\\
3.24e-09	0	\\
3.34e-09	0	\\
3.45e-09	0	\\
3.55e-09	0	\\
3.65e-09	0	\\
3.75e-09	0	\\
3.86e-09	0	\\
3.96e-09	0	\\
4.06e-09	0	\\
4.16e-09	0	\\
4.27e-09	0	\\
4.37e-09	0	\\
4.47e-09	0	\\
4.57e-09	0	\\
4.68e-09	0	\\
4.78e-09	0	\\
4.89e-09	0	\\
4.99e-09	0	\\
5e-09	0	\\
};
\addplot [color=black!50!green,solid,forget plot]
  table[row sep=crcr]{
0	0	\\
1.1e-10	0	\\
2.2e-10	0	\\
3.3e-10	0	\\
4.4e-10	0	\\
5.4e-10	0	\\
6.5e-10	0	\\
7.5e-10	0	\\
8.6e-10	0	\\
9.6e-10	0	\\
1.07e-09	0	\\
1.18e-09	0	\\
1.28e-09	0	\\
1.38e-09	0	\\
1.49e-09	0	\\
1.59e-09	0	\\
1.69e-09	0	\\
1.8e-09	0	\\
1.9e-09	0	\\
2.01e-09	0	\\
2.11e-09	0	\\
2.21e-09	0	\\
2.32e-09	0	\\
2.42e-09	0	\\
2.52e-09	0	\\
2.63e-09	0	\\
2.73e-09	0	\\
2.83e-09	0	\\
2.93e-09	0	\\
3.04e-09	0	\\
3.14e-09	0	\\
3.24e-09	0	\\
3.34e-09	0	\\
3.45e-09	0	\\
3.55e-09	0	\\
3.65e-09	0	\\
3.75e-09	0	\\
3.86e-09	0	\\
3.96e-09	0	\\
4.06e-09	0	\\
4.16e-09	0	\\
4.27e-09	0	\\
4.37e-09	0	\\
4.47e-09	0	\\
4.57e-09	0	\\
4.68e-09	0	\\
4.78e-09	0	\\
4.89e-09	0	\\
4.99e-09	0	\\
5e-09	0	\\
};
\addplot [color=red,solid,forget plot]
  table[row sep=crcr]{
0	0	\\
1.1e-10	0	\\
2.2e-10	0	\\
3.3e-10	0	\\
4.4e-10	0	\\
5.4e-10	0	\\
6.5e-10	0	\\
7.5e-10	0	\\
8.6e-10	0	\\
9.6e-10	0	\\
1.07e-09	0	\\
1.18e-09	0	\\
1.28e-09	0	\\
1.38e-09	0	\\
1.49e-09	0	\\
1.59e-09	0	\\
1.69e-09	0	\\
1.8e-09	0	\\
1.9e-09	0	\\
2.01e-09	0	\\
2.11e-09	0	\\
2.21e-09	0	\\
2.32e-09	0	\\
2.42e-09	0	\\
2.52e-09	0	\\
2.63e-09	0	\\
2.73e-09	0	\\
2.83e-09	0	\\
2.93e-09	0	\\
3.04e-09	0	\\
3.14e-09	0	\\
3.24e-09	0	\\
3.34e-09	0	\\
3.45e-09	0	\\
3.55e-09	0	\\
3.65e-09	0	\\
3.75e-09	0	\\
3.86e-09	0	\\
3.96e-09	0	\\
4.06e-09	0	\\
4.16e-09	0	\\
4.27e-09	0	\\
4.37e-09	0	\\
4.47e-09	0	\\
4.57e-09	0	\\
4.68e-09	0	\\
4.78e-09	0	\\
4.89e-09	0	\\
4.99e-09	0	\\
5e-09	0	\\
};
\addplot [color=mycolor1,solid,forget plot]
  table[row sep=crcr]{
0	0	\\
1.1e-10	0	\\
2.2e-10	0	\\
3.3e-10	0	\\
4.4e-10	0	\\
5.4e-10	0	\\
6.5e-10	0	\\
7.5e-10	0	\\
8.6e-10	0	\\
9.6e-10	0	\\
1.07e-09	0	\\
1.18e-09	0	\\
1.28e-09	0	\\
1.38e-09	0	\\
1.49e-09	0	\\
1.59e-09	0	\\
1.69e-09	0	\\
1.8e-09	0	\\
1.9e-09	0	\\
2.01e-09	0	\\
2.11e-09	0	\\
2.21e-09	0	\\
2.32e-09	0	\\
2.42e-09	0	\\
2.52e-09	0	\\
2.63e-09	0	\\
2.73e-09	0	\\
2.83e-09	0	\\
2.93e-09	0	\\
3.04e-09	0	\\
3.14e-09	0	\\
3.24e-09	0	\\
3.34e-09	0	\\
3.45e-09	0	\\
3.55e-09	0	\\
3.65e-09	0	\\
3.75e-09	0	\\
3.86e-09	0	\\
3.96e-09	0	\\
4.06e-09	0	\\
4.16e-09	0	\\
4.27e-09	0	\\
4.37e-09	0	\\
4.47e-09	0	\\
4.57e-09	0	\\
4.68e-09	0	\\
4.78e-09	0	\\
4.89e-09	0	\\
4.99e-09	0	\\
5e-09	0	\\
};
\addplot [color=mycolor2,solid,forget plot]
  table[row sep=crcr]{
0	0	\\
1.1e-10	0	\\
2.2e-10	0	\\
3.3e-10	0	\\
4.4e-10	0	\\
5.4e-10	0	\\
6.5e-10	0	\\
7.5e-10	0	\\
8.6e-10	0	\\
9.6e-10	0	\\
1.07e-09	0	\\
1.18e-09	0	\\
1.28e-09	0	\\
1.38e-09	0	\\
1.49e-09	0	\\
1.59e-09	0	\\
1.69e-09	0	\\
1.8e-09	0	\\
1.9e-09	0	\\
2.01e-09	0	\\
2.11e-09	0	\\
2.21e-09	0	\\
2.32e-09	0	\\
2.42e-09	0	\\
2.52e-09	0	\\
2.63e-09	0	\\
2.73e-09	0	\\
2.83e-09	0	\\
2.93e-09	0	\\
3.04e-09	0	\\
3.14e-09	0	\\
3.24e-09	0	\\
3.34e-09	0	\\
3.45e-09	0	\\
3.55e-09	0	\\
3.65e-09	0	\\
3.75e-09	0	\\
3.86e-09	0	\\
3.96e-09	0	\\
4.06e-09	0	\\
4.16e-09	0	\\
4.27e-09	0	\\
4.37e-09	0	\\
4.47e-09	0	\\
4.57e-09	0	\\
4.68e-09	0	\\
4.78e-09	0	\\
4.89e-09	0	\\
4.99e-09	0	\\
5e-09	0	\\
};
\addplot [color=mycolor3,solid,forget plot]
  table[row sep=crcr]{
0	0	\\
1.1e-10	0	\\
2.2e-10	0	\\
3.3e-10	0	\\
4.4e-10	0	\\
5.4e-10	0	\\
6.5e-10	0	\\
7.5e-10	0	\\
8.6e-10	0	\\
9.6e-10	0	\\
1.07e-09	0	\\
1.18e-09	0	\\
1.28e-09	0	\\
1.38e-09	0	\\
1.49e-09	0	\\
1.59e-09	0	\\
1.69e-09	0	\\
1.8e-09	0	\\
1.9e-09	0	\\
2.01e-09	0	\\
2.11e-09	0	\\
2.21e-09	0	\\
2.32e-09	0	\\
2.42e-09	0	\\
2.52e-09	0	\\
2.63e-09	0	\\
2.73e-09	0	\\
2.83e-09	0	\\
2.93e-09	0	\\
3.04e-09	0	\\
3.14e-09	0	\\
3.24e-09	0	\\
3.34e-09	0	\\
3.45e-09	0	\\
3.55e-09	0	\\
3.65e-09	0	\\
3.75e-09	0	\\
3.86e-09	0	\\
3.96e-09	0	\\
4.06e-09	0	\\
4.16e-09	0	\\
4.27e-09	0	\\
4.37e-09	0	\\
4.47e-09	0	\\
4.57e-09	0	\\
4.68e-09	0	\\
4.78e-09	0	\\
4.89e-09	0	\\
4.99e-09	0	\\
5e-09	0	\\
};
\addplot [color=darkgray,solid,forget plot]
  table[row sep=crcr]{
0	0	\\
1.1e-10	0	\\
2.2e-10	0	\\
3.3e-10	0	\\
4.4e-10	0	\\
5.4e-10	0	\\
6.5e-10	0	\\
7.5e-10	0	\\
8.6e-10	0	\\
9.6e-10	0	\\
1.07e-09	0	\\
1.18e-09	0	\\
1.28e-09	0	\\
1.38e-09	0	\\
1.49e-09	0	\\
1.59e-09	0	\\
1.69e-09	0	\\
1.8e-09	0	\\
1.9e-09	0	\\
2.01e-09	0	\\
2.11e-09	0	\\
2.21e-09	0	\\
2.32e-09	0	\\
2.42e-09	0	\\
2.52e-09	0	\\
2.63e-09	0	\\
2.73e-09	0	\\
2.83e-09	0	\\
2.93e-09	0	\\
3.04e-09	0	\\
3.14e-09	0	\\
3.24e-09	0	\\
3.34e-09	0	\\
3.45e-09	0	\\
3.55e-09	0	\\
3.65e-09	0	\\
3.75e-09	0	\\
3.86e-09	0	\\
3.96e-09	0	\\
4.06e-09	0	\\
4.16e-09	0	\\
4.27e-09	0	\\
4.37e-09	0	\\
4.47e-09	0	\\
4.57e-09	0	\\
4.68e-09	0	\\
4.78e-09	0	\\
4.89e-09	0	\\
4.99e-09	0	\\
5e-09	0	\\
};
\addplot [color=blue,solid,forget plot]
  table[row sep=crcr]{
0	0	\\
1.1e-10	0	\\
2.2e-10	0	\\
3.3e-10	0	\\
4.4e-10	0	\\
5.4e-10	0	\\
6.5e-10	0	\\
7.5e-10	0	\\
8.6e-10	0	\\
9.6e-10	0	\\
1.07e-09	0	\\
1.18e-09	0	\\
1.28e-09	0	\\
1.38e-09	0	\\
1.49e-09	0	\\
1.59e-09	0	\\
1.69e-09	0	\\
1.8e-09	0	\\
1.9e-09	0	\\
2.01e-09	0	\\
2.11e-09	0	\\
2.21e-09	0	\\
2.32e-09	0	\\
2.42e-09	0	\\
2.52e-09	0	\\
2.63e-09	0	\\
2.73e-09	0	\\
2.83e-09	0	\\
2.93e-09	0	\\
3.04e-09	0	\\
3.14e-09	0	\\
3.24e-09	0	\\
3.34e-09	0	\\
3.45e-09	0	\\
3.55e-09	0	\\
3.65e-09	0	\\
3.75e-09	0	\\
3.86e-09	0	\\
3.96e-09	0	\\
4.06e-09	0	\\
4.16e-09	0	\\
4.27e-09	0	\\
4.37e-09	0	\\
4.47e-09	0	\\
4.57e-09	0	\\
4.68e-09	0	\\
4.78e-09	0	\\
4.89e-09	0	\\
4.99e-09	0	\\
5e-09	0	\\
};
\addplot [color=black!50!green,solid,forget plot]
  table[row sep=crcr]{
0	0	\\
1.1e-10	0	\\
2.2e-10	0	\\
3.3e-10	0	\\
4.4e-10	0	\\
5.4e-10	0	\\
6.5e-10	0	\\
7.5e-10	0	\\
8.6e-10	0	\\
9.6e-10	0	\\
1.07e-09	0	\\
1.18e-09	0	\\
1.28e-09	0	\\
1.38e-09	0	\\
1.49e-09	0	\\
1.59e-09	0	\\
1.69e-09	0	\\
1.8e-09	0	\\
1.9e-09	0	\\
2.01e-09	0	\\
2.11e-09	0	\\
2.21e-09	0	\\
2.32e-09	0	\\
2.42e-09	0	\\
2.52e-09	0	\\
2.63e-09	0	\\
2.73e-09	0	\\
2.83e-09	0	\\
2.93e-09	0	\\
3.04e-09	0	\\
3.14e-09	0	\\
3.24e-09	0	\\
3.34e-09	0	\\
3.45e-09	0	\\
3.55e-09	0	\\
3.65e-09	0	\\
3.75e-09	0	\\
3.86e-09	0	\\
3.96e-09	0	\\
4.06e-09	0	\\
4.16e-09	0	\\
4.27e-09	0	\\
4.37e-09	0	\\
4.47e-09	0	\\
4.57e-09	0	\\
4.68e-09	0	\\
4.78e-09	0	\\
4.89e-09	0	\\
4.99e-09	0	\\
5e-09	0	\\
};
\addplot [color=red,solid,forget plot]
  table[row sep=crcr]{
0	0	\\
1.1e-10	0	\\
2.2e-10	0	\\
3.3e-10	0	\\
4.4e-10	0	\\
5.4e-10	0	\\
6.5e-10	0	\\
7.5e-10	0	\\
8.6e-10	0	\\
9.6e-10	0	\\
1.07e-09	0	\\
1.18e-09	0	\\
1.28e-09	0	\\
1.38e-09	0	\\
1.49e-09	0	\\
1.59e-09	0	\\
1.69e-09	0	\\
1.8e-09	0	\\
1.9e-09	0	\\
2.01e-09	0	\\
2.11e-09	0	\\
2.21e-09	0	\\
2.32e-09	0	\\
2.42e-09	0	\\
2.52e-09	0	\\
2.63e-09	0	\\
2.73e-09	0	\\
2.83e-09	0	\\
2.93e-09	0	\\
3.04e-09	0	\\
3.14e-09	0	\\
3.24e-09	0	\\
3.34e-09	0	\\
3.45e-09	0	\\
3.55e-09	0	\\
3.65e-09	0	\\
3.75e-09	0	\\
3.86e-09	0	\\
3.96e-09	0	\\
4.06e-09	0	\\
4.16e-09	0	\\
4.27e-09	0	\\
4.37e-09	0	\\
4.47e-09	0	\\
4.57e-09	0	\\
4.68e-09	0	\\
4.78e-09	0	\\
4.89e-09	0	\\
4.99e-09	0	\\
5e-09	0	\\
};
\addplot [color=mycolor1,solid,forget plot]
  table[row sep=crcr]{
0	0	\\
1.1e-10	0	\\
2.2e-10	0	\\
3.3e-10	0	\\
4.4e-10	0	\\
5.4e-10	0	\\
6.5e-10	0	\\
7.5e-10	0	\\
8.6e-10	0	\\
9.6e-10	0	\\
1.07e-09	0	\\
1.18e-09	0	\\
1.28e-09	0	\\
1.38e-09	0	\\
1.49e-09	0	\\
1.59e-09	0	\\
1.69e-09	0	\\
1.8e-09	0	\\
1.9e-09	0	\\
2.01e-09	0	\\
2.11e-09	0	\\
2.21e-09	0	\\
2.32e-09	0	\\
2.42e-09	0	\\
2.52e-09	0	\\
2.63e-09	0	\\
2.73e-09	0	\\
2.83e-09	0	\\
2.93e-09	0	\\
3.04e-09	0	\\
3.14e-09	0	\\
3.24e-09	0	\\
3.34e-09	0	\\
3.45e-09	0	\\
3.55e-09	0	\\
3.65e-09	0	\\
3.75e-09	0	\\
3.86e-09	0	\\
3.96e-09	0	\\
4.06e-09	0	\\
4.16e-09	0	\\
4.27e-09	0	\\
4.37e-09	0	\\
4.47e-09	0	\\
4.57e-09	0	\\
4.68e-09	0	\\
4.78e-09	0	\\
4.89e-09	0	\\
4.99e-09	0	\\
5e-09	0	\\
};
\addplot [color=mycolor2,solid,forget plot]
  table[row sep=crcr]{
0	0	\\
1.1e-10	0	\\
2.2e-10	0	\\
3.3e-10	0	\\
4.4e-10	0	\\
5.4e-10	0	\\
6.5e-10	0	\\
7.5e-10	0	\\
8.6e-10	0	\\
9.6e-10	0	\\
1.07e-09	0	\\
1.18e-09	0	\\
1.28e-09	0	\\
1.38e-09	0	\\
1.49e-09	0	\\
1.59e-09	0	\\
1.69e-09	0	\\
1.8e-09	0	\\
1.9e-09	0	\\
2.01e-09	0	\\
2.11e-09	0	\\
2.21e-09	0	\\
2.32e-09	0	\\
2.42e-09	0	\\
2.52e-09	0	\\
2.63e-09	0	\\
2.73e-09	0	\\
2.83e-09	0	\\
2.93e-09	0	\\
3.04e-09	0	\\
3.14e-09	0	\\
3.24e-09	0	\\
3.34e-09	0	\\
3.45e-09	0	\\
3.55e-09	0	\\
3.65e-09	0	\\
3.75e-09	0	\\
3.86e-09	0	\\
3.96e-09	0	\\
4.06e-09	0	\\
4.16e-09	0	\\
4.27e-09	0	\\
4.37e-09	0	\\
4.47e-09	0	\\
4.57e-09	0	\\
4.68e-09	0	\\
4.78e-09	0	\\
4.89e-09	0	\\
4.99e-09	0	\\
5e-09	0	\\
};
\addplot [color=mycolor3,solid,forget plot]
  table[row sep=crcr]{
0	0	\\
1.1e-10	0	\\
2.2e-10	0	\\
3.3e-10	0	\\
4.4e-10	0	\\
5.4e-10	0	\\
6.5e-10	0	\\
7.5e-10	0	\\
8.6e-10	0	\\
9.6e-10	0	\\
1.07e-09	0	\\
1.18e-09	0	\\
1.28e-09	0	\\
1.38e-09	0	\\
1.49e-09	0	\\
1.59e-09	0	\\
1.69e-09	0	\\
1.8e-09	0	\\
1.9e-09	0	\\
2.01e-09	0	\\
2.11e-09	0	\\
2.21e-09	0	\\
2.32e-09	0	\\
2.42e-09	0	\\
2.52e-09	0	\\
2.63e-09	0	\\
2.73e-09	0	\\
2.83e-09	0	\\
2.93e-09	0	\\
3.04e-09	0	\\
3.14e-09	0	\\
3.24e-09	0	\\
3.34e-09	0	\\
3.45e-09	0	\\
3.55e-09	0	\\
3.65e-09	0	\\
3.75e-09	0	\\
3.86e-09	0	\\
3.96e-09	0	\\
4.06e-09	0	\\
4.16e-09	0	\\
4.27e-09	0	\\
4.37e-09	0	\\
4.47e-09	0	\\
4.57e-09	0	\\
4.68e-09	0	\\
4.78e-09	0	\\
4.89e-09	0	\\
4.99e-09	0	\\
5e-09	0	\\
};
\addplot [color=darkgray,solid,forget plot]
  table[row sep=crcr]{
0	0	\\
1.1e-10	0	\\
2.2e-10	0	\\
3.3e-10	0	\\
4.4e-10	0	\\
5.4e-10	0	\\
6.5e-10	0	\\
7.5e-10	0	\\
8.6e-10	0	\\
9.6e-10	0	\\
1.07e-09	0	\\
1.18e-09	0	\\
1.28e-09	0	\\
1.38e-09	0	\\
1.49e-09	0	\\
1.59e-09	0	\\
1.69e-09	0	\\
1.8e-09	0	\\
1.9e-09	0	\\
2.01e-09	0	\\
2.11e-09	0	\\
2.21e-09	0	\\
2.32e-09	0	\\
2.42e-09	0	\\
2.52e-09	0	\\
2.63e-09	0	\\
2.73e-09	0	\\
2.83e-09	0	\\
2.93e-09	0	\\
3.04e-09	0	\\
3.14e-09	0	\\
3.24e-09	0	\\
3.34e-09	0	\\
3.45e-09	0	\\
3.55e-09	0	\\
3.65e-09	0	\\
3.75e-09	0	\\
3.86e-09	0	\\
3.96e-09	0	\\
4.06e-09	0	\\
4.16e-09	0	\\
4.27e-09	0	\\
4.37e-09	0	\\
4.47e-09	0	\\
4.57e-09	0	\\
4.68e-09	0	\\
4.78e-09	0	\\
4.89e-09	0	\\
4.99e-09	0	\\
5e-09	0	\\
};
\addplot [color=blue,solid,forget plot]
  table[row sep=crcr]{
0	0	\\
1.1e-10	0	\\
2.2e-10	0	\\
3.3e-10	0	\\
4.4e-10	0	\\
5.4e-10	0	\\
6.5e-10	0	\\
7.5e-10	0	\\
8.6e-10	0	\\
9.6e-10	0	\\
1.07e-09	0	\\
1.18e-09	0	\\
1.28e-09	0	\\
1.38e-09	0	\\
1.49e-09	0	\\
1.59e-09	0	\\
1.69e-09	0	\\
1.8e-09	0	\\
1.9e-09	0	\\
2.01e-09	0	\\
2.11e-09	0	\\
2.21e-09	0	\\
2.32e-09	0	\\
2.42e-09	0	\\
2.52e-09	0	\\
2.63e-09	0	\\
2.73e-09	0	\\
2.83e-09	0	\\
2.93e-09	0	\\
3.04e-09	0	\\
3.14e-09	0	\\
3.24e-09	0	\\
3.34e-09	0	\\
3.45e-09	0	\\
3.55e-09	0	\\
3.65e-09	0	\\
3.75e-09	0	\\
3.86e-09	0	\\
3.96e-09	0	\\
4.06e-09	0	\\
4.16e-09	0	\\
4.27e-09	0	\\
4.37e-09	0	\\
4.47e-09	0	\\
4.57e-09	0	\\
4.68e-09	0	\\
4.78e-09	0	\\
4.89e-09	0	\\
4.99e-09	0	\\
5e-09	0	\\
};
\addplot [color=black!50!green,solid,forget plot]
  table[row sep=crcr]{
0	0	\\
1.1e-10	0	\\
2.2e-10	0	\\
3.3e-10	0	\\
4.4e-10	0	\\
5.4e-10	0	\\
6.5e-10	0	\\
7.5e-10	0	\\
8.6e-10	0	\\
9.6e-10	0	\\
1.07e-09	0	\\
1.18e-09	0	\\
1.28e-09	0	\\
1.38e-09	0	\\
1.49e-09	0	\\
1.59e-09	0	\\
1.69e-09	0	\\
1.8e-09	0	\\
1.9e-09	0	\\
2.01e-09	0	\\
2.11e-09	0	\\
2.21e-09	0	\\
2.32e-09	0	\\
2.42e-09	0	\\
2.52e-09	0	\\
2.63e-09	0	\\
2.73e-09	0	\\
2.83e-09	0	\\
2.93e-09	0	\\
3.04e-09	0	\\
3.14e-09	0	\\
3.24e-09	0	\\
3.34e-09	0	\\
3.45e-09	0	\\
3.55e-09	0	\\
3.65e-09	0	\\
3.75e-09	0	\\
3.86e-09	0	\\
3.96e-09	0	\\
4.06e-09	0	\\
4.16e-09	0	\\
4.27e-09	0	\\
4.37e-09	0	\\
4.47e-09	0	\\
4.57e-09	0	\\
4.68e-09	0	\\
4.78e-09	0	\\
4.89e-09	0	\\
4.99e-09	0	\\
5e-09	0	\\
};
\addplot [color=red,solid,forget plot]
  table[row sep=crcr]{
0	0	\\
1.1e-10	0	\\
2.2e-10	0	\\
3.3e-10	0	\\
4.4e-10	0	\\
5.4e-10	0	\\
6.5e-10	0	\\
7.5e-10	0	\\
8.6e-10	0	\\
9.6e-10	0	\\
1.07e-09	0	\\
1.18e-09	0	\\
1.28e-09	0	\\
1.38e-09	0	\\
1.49e-09	0	\\
1.59e-09	0	\\
1.69e-09	0	\\
1.8e-09	0	\\
1.9e-09	0	\\
2.01e-09	0	\\
2.11e-09	0	\\
2.21e-09	0	\\
2.32e-09	0	\\
2.42e-09	0	\\
2.52e-09	0	\\
2.63e-09	0	\\
2.73e-09	0	\\
2.83e-09	0	\\
2.93e-09	0	\\
3.04e-09	0	\\
3.14e-09	0	\\
3.24e-09	0	\\
3.34e-09	0	\\
3.45e-09	0	\\
3.55e-09	0	\\
3.65e-09	0	\\
3.75e-09	0	\\
3.86e-09	0	\\
3.96e-09	0	\\
4.06e-09	0	\\
4.16e-09	0	\\
4.27e-09	0	\\
4.37e-09	0	\\
4.47e-09	0	\\
4.57e-09	0	\\
4.68e-09	0	\\
4.78e-09	0	\\
4.89e-09	0	\\
4.99e-09	0	\\
5e-09	0	\\
};
\addplot [color=mycolor1,solid,forget plot]
  table[row sep=crcr]{
0	0	\\
1.1e-10	0	\\
2.2e-10	0	\\
3.3e-10	0	\\
4.4e-10	0	\\
5.4e-10	0	\\
6.5e-10	0	\\
7.5e-10	0	\\
8.6e-10	0	\\
9.6e-10	0	\\
1.07e-09	0	\\
1.18e-09	0	\\
1.28e-09	0	\\
1.38e-09	0	\\
1.49e-09	0	\\
1.59e-09	0	\\
1.69e-09	0	\\
1.8e-09	0	\\
1.9e-09	0	\\
2.01e-09	0	\\
2.11e-09	0	\\
2.21e-09	0	\\
2.32e-09	0	\\
2.42e-09	0	\\
2.52e-09	0	\\
2.63e-09	0	\\
2.73e-09	0	\\
2.83e-09	0	\\
2.93e-09	0	\\
3.04e-09	0	\\
3.14e-09	0	\\
3.24e-09	0	\\
3.34e-09	0	\\
3.45e-09	0	\\
3.55e-09	0	\\
3.65e-09	0	\\
3.75e-09	0	\\
3.86e-09	0	\\
3.96e-09	0	\\
4.06e-09	0	\\
4.16e-09	0	\\
4.27e-09	0	\\
4.37e-09	0	\\
4.47e-09	0	\\
4.57e-09	0	\\
4.68e-09	0	\\
4.78e-09	0	\\
4.89e-09	0	\\
4.99e-09	0	\\
5e-09	0	\\
};
\addplot [color=mycolor2,solid,forget plot]
  table[row sep=crcr]{
0	0	\\
1.1e-10	0	\\
2.2e-10	0	\\
3.3e-10	0	\\
4.4e-10	0	\\
5.4e-10	0	\\
6.5e-10	0	\\
7.5e-10	0	\\
8.6e-10	0	\\
9.6e-10	0	\\
1.07e-09	0	\\
1.18e-09	0	\\
1.28e-09	0	\\
1.38e-09	0	\\
1.49e-09	0	\\
1.59e-09	0	\\
1.69e-09	0	\\
1.8e-09	0	\\
1.9e-09	0	\\
2.01e-09	0	\\
2.11e-09	0	\\
2.21e-09	0	\\
2.32e-09	0	\\
2.42e-09	0	\\
2.52e-09	0	\\
2.63e-09	0	\\
2.73e-09	0	\\
2.83e-09	0	\\
2.93e-09	0	\\
3.04e-09	0	\\
3.14e-09	0	\\
3.24e-09	0	\\
3.34e-09	0	\\
3.45e-09	0	\\
3.55e-09	0	\\
3.65e-09	0	\\
3.75e-09	0	\\
3.86e-09	0	\\
3.96e-09	0	\\
4.06e-09	0	\\
4.16e-09	0	\\
4.27e-09	0	\\
4.37e-09	0	\\
4.47e-09	0	\\
4.57e-09	0	\\
4.68e-09	0	\\
4.78e-09	0	\\
4.89e-09	0	\\
4.99e-09	0	\\
5e-09	0	\\
};
\addplot [color=mycolor3,solid,forget plot]
  table[row sep=crcr]{
0	0	\\
1.1e-10	0	\\
2.2e-10	0	\\
3.3e-10	0	\\
4.4e-10	0	\\
5.4e-10	0	\\
6.5e-10	0	\\
7.5e-10	0	\\
8.6e-10	0	\\
9.6e-10	0	\\
1.07e-09	0	\\
1.18e-09	0	\\
1.28e-09	0	\\
1.38e-09	0	\\
1.49e-09	0	\\
1.59e-09	0	\\
1.69e-09	0	\\
1.8e-09	0	\\
1.9e-09	0	\\
2.01e-09	0	\\
2.11e-09	0	\\
2.21e-09	0	\\
2.32e-09	0	\\
2.42e-09	0	\\
2.52e-09	0	\\
2.63e-09	0	\\
2.73e-09	0	\\
2.83e-09	0	\\
2.93e-09	0	\\
3.04e-09	0	\\
3.14e-09	0	\\
3.24e-09	0	\\
3.34e-09	0	\\
3.45e-09	0	\\
3.55e-09	0	\\
3.65e-09	0	\\
3.75e-09	0	\\
3.86e-09	0	\\
3.96e-09	0	\\
4.06e-09	0	\\
4.16e-09	0	\\
4.27e-09	0	\\
4.37e-09	0	\\
4.47e-09	0	\\
4.57e-09	0	\\
4.68e-09	0	\\
4.78e-09	0	\\
4.89e-09	0	\\
4.99e-09	0	\\
5e-09	0	\\
};
\addplot [color=darkgray,solid,forget plot]
  table[row sep=crcr]{
0	0	\\
1.1e-10	0	\\
2.2e-10	0	\\
3.3e-10	0	\\
4.4e-10	0	\\
5.4e-10	0	\\
6.5e-10	0	\\
7.5e-10	0	\\
8.6e-10	0	\\
9.6e-10	0	\\
1.07e-09	0	\\
1.18e-09	0	\\
1.28e-09	0	\\
1.38e-09	0	\\
1.49e-09	0	\\
1.59e-09	0	\\
1.69e-09	0	\\
1.8e-09	0	\\
1.9e-09	0	\\
2.01e-09	0	\\
2.11e-09	0	\\
2.21e-09	0	\\
2.32e-09	0	\\
2.42e-09	0	\\
2.52e-09	0	\\
2.63e-09	0	\\
2.73e-09	0	\\
2.83e-09	0	\\
2.93e-09	0	\\
3.04e-09	0	\\
3.14e-09	0	\\
3.24e-09	0	\\
3.34e-09	0	\\
3.45e-09	0	\\
3.55e-09	0	\\
3.65e-09	0	\\
3.75e-09	0	\\
3.86e-09	0	\\
3.96e-09	0	\\
4.06e-09	0	\\
4.16e-09	0	\\
4.27e-09	0	\\
4.37e-09	0	\\
4.47e-09	0	\\
4.57e-09	0	\\
4.68e-09	0	\\
4.78e-09	0	\\
4.89e-09	0	\\
4.99e-09	0	\\
5e-09	0	\\
};
\addplot [color=blue,solid,forget plot]
  table[row sep=crcr]{
0	0	\\
1.1e-10	0	\\
2.2e-10	0	\\
3.3e-10	0	\\
4.4e-10	0	\\
5.4e-10	0	\\
6.5e-10	0	\\
7.5e-10	0	\\
8.6e-10	0	\\
9.6e-10	0	\\
1.07e-09	0	\\
1.18e-09	0	\\
1.28e-09	0	\\
1.38e-09	0	\\
1.49e-09	0	\\
1.59e-09	0	\\
1.69e-09	0	\\
1.8e-09	0	\\
1.9e-09	0	\\
2.01e-09	0	\\
2.11e-09	0	\\
2.21e-09	0	\\
2.32e-09	0	\\
2.42e-09	0	\\
2.52e-09	0	\\
2.63e-09	0	\\
2.73e-09	0	\\
2.83e-09	0	\\
2.93e-09	0	\\
3.04e-09	0	\\
3.14e-09	0	\\
3.24e-09	0	\\
3.34e-09	0	\\
3.45e-09	0	\\
3.55e-09	0	\\
3.65e-09	0	\\
3.75e-09	0	\\
3.86e-09	0	\\
3.96e-09	0	\\
4.06e-09	0	\\
4.16e-09	0	\\
4.27e-09	0	\\
4.37e-09	0	\\
4.47e-09	0	\\
4.57e-09	0	\\
4.68e-09	0	\\
4.78e-09	0	\\
4.89e-09	0	\\
4.99e-09	0	\\
5e-09	0	\\
};
\addplot [color=black!50!green,solid,forget plot]
  table[row sep=crcr]{
0	0	\\
1.1e-10	0	\\
2.2e-10	0	\\
3.3e-10	0	\\
4.4e-10	0	\\
5.4e-10	0	\\
6.5e-10	0	\\
7.5e-10	0	\\
8.6e-10	0	\\
9.6e-10	0	\\
1.07e-09	0	\\
1.18e-09	0	\\
1.28e-09	0	\\
1.38e-09	0	\\
1.49e-09	0	\\
1.59e-09	0	\\
1.69e-09	0	\\
1.8e-09	0	\\
1.9e-09	0	\\
2.01e-09	0	\\
2.11e-09	0	\\
2.21e-09	0	\\
2.32e-09	0	\\
2.42e-09	0	\\
2.52e-09	0	\\
2.63e-09	0	\\
2.73e-09	0	\\
2.83e-09	0	\\
2.93e-09	0	\\
3.04e-09	0	\\
3.14e-09	0	\\
3.24e-09	0	\\
3.34e-09	0	\\
3.45e-09	0	\\
3.55e-09	0	\\
3.65e-09	0	\\
3.75e-09	0	\\
3.86e-09	0	\\
3.96e-09	0	\\
4.06e-09	0	\\
4.16e-09	0	\\
4.27e-09	0	\\
4.37e-09	0	\\
4.47e-09	0	\\
4.57e-09	0	\\
4.68e-09	0	\\
4.78e-09	0	\\
4.89e-09	0	\\
4.99e-09	0	\\
5e-09	0	\\
};
\addplot [color=red,solid,forget plot]
  table[row sep=crcr]{
0	0	\\
1.1e-10	0	\\
2.2e-10	0	\\
3.3e-10	0	\\
4.4e-10	0	\\
5.4e-10	0	\\
6.5e-10	0	\\
7.5e-10	0	\\
8.6e-10	0	\\
9.6e-10	0	\\
1.07e-09	0	\\
1.18e-09	0	\\
1.28e-09	0	\\
1.38e-09	0	\\
1.49e-09	0	\\
1.59e-09	0	\\
1.69e-09	0	\\
1.8e-09	0	\\
1.9e-09	0	\\
2.01e-09	0	\\
2.11e-09	0	\\
2.21e-09	0	\\
2.32e-09	0	\\
2.42e-09	0	\\
2.52e-09	0	\\
2.63e-09	0	\\
2.73e-09	0	\\
2.83e-09	0	\\
2.93e-09	0	\\
3.04e-09	0	\\
3.14e-09	0	\\
3.24e-09	0	\\
3.34e-09	0	\\
3.45e-09	0	\\
3.55e-09	0	\\
3.65e-09	0	\\
3.75e-09	0	\\
3.86e-09	0	\\
3.96e-09	0	\\
4.06e-09	0	\\
4.16e-09	0	\\
4.27e-09	0	\\
4.37e-09	0	\\
4.47e-09	0	\\
4.57e-09	0	\\
4.68e-09	0	\\
4.78e-09	0	\\
4.89e-09	0	\\
4.99e-09	0	\\
5e-09	0	\\
};
\addplot [color=mycolor1,solid,forget plot]
  table[row sep=crcr]{
0	0	\\
1.1e-10	0	\\
2.2e-10	0	\\
3.3e-10	0	\\
4.4e-10	0	\\
5.4e-10	0	\\
6.5e-10	0	\\
7.5e-10	0	\\
8.6e-10	0	\\
9.6e-10	0	\\
1.07e-09	0	\\
1.18e-09	0	\\
1.28e-09	0	\\
1.38e-09	0	\\
1.49e-09	0	\\
1.59e-09	0	\\
1.69e-09	0	\\
1.8e-09	0	\\
1.9e-09	0	\\
2.01e-09	0	\\
2.11e-09	0	\\
2.21e-09	0	\\
2.32e-09	0	\\
2.42e-09	0	\\
2.52e-09	0	\\
2.63e-09	0	\\
2.73e-09	0	\\
2.83e-09	0	\\
2.93e-09	0	\\
3.04e-09	0	\\
3.14e-09	0	\\
3.24e-09	0	\\
3.34e-09	0	\\
3.45e-09	0	\\
3.55e-09	0	\\
3.65e-09	0	\\
3.75e-09	0	\\
3.86e-09	0	\\
3.96e-09	0	\\
4.06e-09	0	\\
4.16e-09	0	\\
4.27e-09	0	\\
4.37e-09	0	\\
4.47e-09	0	\\
4.57e-09	0	\\
4.68e-09	0	\\
4.78e-09	0	\\
4.89e-09	0	\\
4.99e-09	0	\\
5e-09	0	\\
};
\addplot [color=mycolor2,solid,forget plot]
  table[row sep=crcr]{
0	0	\\
1.1e-10	0	\\
2.2e-10	0	\\
3.3e-10	0	\\
4.4e-10	0	\\
5.4e-10	0	\\
6.5e-10	0	\\
7.5e-10	0	\\
8.6e-10	0	\\
9.6e-10	0	\\
1.07e-09	0	\\
1.18e-09	0	\\
1.28e-09	0	\\
1.38e-09	0	\\
1.49e-09	0	\\
1.59e-09	0	\\
1.69e-09	0	\\
1.8e-09	0	\\
1.9e-09	0	\\
2.01e-09	0	\\
2.11e-09	0	\\
2.21e-09	0	\\
2.32e-09	0	\\
2.42e-09	0	\\
2.52e-09	0	\\
2.63e-09	0	\\
2.73e-09	0	\\
2.83e-09	0	\\
2.93e-09	0	\\
3.04e-09	0	\\
3.14e-09	0	\\
3.24e-09	0	\\
3.34e-09	0	\\
3.45e-09	0	\\
3.55e-09	0	\\
3.65e-09	0	\\
3.75e-09	0	\\
3.86e-09	0	\\
3.96e-09	0	\\
4.06e-09	0	\\
4.16e-09	0	\\
4.27e-09	0	\\
4.37e-09	0	\\
4.47e-09	0	\\
4.57e-09	0	\\
4.68e-09	0	\\
4.78e-09	0	\\
4.89e-09	0	\\
4.99e-09	0	\\
5e-09	0	\\
};
\addplot [color=mycolor3,solid,forget plot]
  table[row sep=crcr]{
0	0	\\
1.1e-10	0	\\
2.2e-10	0	\\
3.3e-10	0	\\
4.4e-10	0	\\
5.4e-10	0	\\
6.5e-10	0	\\
7.5e-10	0	\\
8.6e-10	0	\\
9.6e-10	0	\\
1.07e-09	0	\\
1.18e-09	0	\\
1.28e-09	0	\\
1.38e-09	0	\\
1.49e-09	0	\\
1.59e-09	0	\\
1.69e-09	0	\\
1.8e-09	0	\\
1.9e-09	0	\\
2.01e-09	0	\\
2.11e-09	0	\\
2.21e-09	0	\\
2.32e-09	0	\\
2.42e-09	0	\\
2.52e-09	0	\\
2.63e-09	0	\\
2.73e-09	0	\\
2.83e-09	0	\\
2.93e-09	0	\\
3.04e-09	0	\\
3.14e-09	0	\\
3.24e-09	0	\\
3.34e-09	0	\\
3.45e-09	0	\\
3.55e-09	0	\\
3.65e-09	0	\\
3.75e-09	0	\\
3.86e-09	0	\\
3.96e-09	0	\\
4.06e-09	0	\\
4.16e-09	0	\\
4.27e-09	0	\\
4.37e-09	0	\\
4.47e-09	0	\\
4.57e-09	0	\\
4.68e-09	0	\\
4.78e-09	0	\\
4.89e-09	0	\\
4.99e-09	0	\\
5e-09	0	\\
};
\addplot [color=darkgray,solid,forget plot]
  table[row sep=crcr]{
0	0	\\
1.1e-10	0	\\
2.2e-10	0	\\
3.3e-10	0	\\
4.4e-10	0	\\
5.4e-10	0	\\
6.5e-10	0	\\
7.5e-10	0	\\
8.6e-10	0	\\
9.6e-10	0	\\
1.07e-09	0	\\
1.18e-09	0	\\
1.28e-09	0	\\
1.38e-09	0	\\
1.49e-09	0	\\
1.59e-09	0	\\
1.69e-09	0	\\
1.8e-09	0	\\
1.9e-09	0	\\
2.01e-09	0	\\
2.11e-09	0	\\
2.21e-09	0	\\
2.32e-09	0	\\
2.42e-09	0	\\
2.52e-09	0	\\
2.63e-09	0	\\
2.73e-09	0	\\
2.83e-09	0	\\
2.93e-09	0	\\
3.04e-09	0	\\
3.14e-09	0	\\
3.24e-09	0	\\
3.34e-09	0	\\
3.45e-09	0	\\
3.55e-09	0	\\
3.65e-09	0	\\
3.75e-09	0	\\
3.86e-09	0	\\
3.96e-09	0	\\
4.06e-09	0	\\
4.16e-09	0	\\
4.27e-09	0	\\
4.37e-09	0	\\
4.47e-09	0	\\
4.57e-09	0	\\
4.68e-09	0	\\
4.78e-09	0	\\
4.89e-09	0	\\
4.99e-09	0	\\
5e-09	0	\\
};
\addplot [color=blue,solid,forget plot]
  table[row sep=crcr]{
0	0	\\
1.1e-10	0	\\
2.2e-10	0	\\
3.3e-10	0	\\
4.4e-10	0	\\
5.4e-10	0	\\
6.5e-10	0	\\
7.5e-10	0	\\
8.6e-10	0	\\
9.6e-10	0	\\
1.07e-09	0	\\
1.18e-09	0	\\
1.28e-09	0	\\
1.38e-09	0	\\
1.49e-09	0	\\
1.59e-09	0	\\
1.69e-09	0	\\
1.8e-09	0	\\
1.9e-09	0	\\
2.01e-09	0	\\
2.11e-09	0	\\
2.21e-09	0	\\
2.32e-09	0	\\
2.42e-09	0	\\
2.52e-09	0	\\
2.63e-09	0	\\
2.73e-09	0	\\
2.83e-09	0	\\
2.93e-09	0	\\
3.04e-09	0	\\
3.14e-09	0	\\
3.24e-09	0	\\
3.34e-09	0	\\
3.45e-09	0	\\
3.55e-09	0	\\
3.65e-09	0	\\
3.75e-09	0	\\
3.86e-09	0	\\
3.96e-09	0	\\
4.06e-09	0	\\
4.16e-09	0	\\
4.27e-09	0	\\
4.37e-09	0	\\
4.47e-09	0	\\
4.57e-09	0	\\
4.68e-09	0	\\
4.78e-09	0	\\
4.89e-09	0	\\
4.99e-09	0	\\
5e-09	0	\\
};
\addplot [color=black!50!green,solid,forget plot]
  table[row sep=crcr]{
0	0	\\
1.1e-10	0	\\
2.2e-10	0	\\
3.3e-10	0	\\
4.4e-10	0	\\
5.4e-10	0	\\
6.5e-10	0	\\
7.5e-10	0	\\
8.6e-10	0	\\
9.6e-10	0	\\
1.07e-09	0	\\
1.18e-09	0	\\
1.28e-09	0	\\
1.38e-09	0	\\
1.49e-09	0	\\
1.59e-09	0	\\
1.69e-09	0	\\
1.8e-09	0	\\
1.9e-09	0	\\
2.01e-09	0	\\
2.11e-09	0	\\
2.21e-09	0	\\
2.32e-09	0	\\
2.42e-09	0	\\
2.52e-09	0	\\
2.63e-09	0	\\
2.73e-09	0	\\
2.83e-09	0	\\
2.93e-09	0	\\
3.04e-09	0	\\
3.14e-09	0	\\
3.24e-09	0	\\
3.34e-09	0	\\
3.45e-09	0	\\
3.55e-09	0	\\
3.65e-09	0	\\
3.75e-09	0	\\
3.86e-09	0	\\
3.96e-09	0	\\
4.06e-09	0	\\
4.16e-09	0	\\
4.27e-09	0	\\
4.37e-09	0	\\
4.47e-09	0	\\
4.57e-09	0	\\
4.68e-09	0	\\
4.78e-09	0	\\
4.89e-09	0	\\
4.99e-09	0	\\
5e-09	0	\\
};
\addplot [color=red,solid,forget plot]
  table[row sep=crcr]{
0	0	\\
1.1e-10	0	\\
2.2e-10	0	\\
3.3e-10	0	\\
4.4e-10	0	\\
5.4e-10	0	\\
6.5e-10	0	\\
7.5e-10	0	\\
8.6e-10	0	\\
9.6e-10	0	\\
1.07e-09	0	\\
1.18e-09	0	\\
1.28e-09	0	\\
1.38e-09	0	\\
1.49e-09	0	\\
1.59e-09	0	\\
1.69e-09	0	\\
1.8e-09	0	\\
1.9e-09	0	\\
2.01e-09	0	\\
2.11e-09	0	\\
2.21e-09	0	\\
2.32e-09	0	\\
2.42e-09	0	\\
2.52e-09	0	\\
2.63e-09	0	\\
2.73e-09	0	\\
2.83e-09	0	\\
2.93e-09	0	\\
3.04e-09	0	\\
3.14e-09	0	\\
3.24e-09	0	\\
3.34e-09	0	\\
3.45e-09	0	\\
3.55e-09	0	\\
3.65e-09	0	\\
3.75e-09	0	\\
3.86e-09	0	\\
3.96e-09	0	\\
4.06e-09	0	\\
4.16e-09	0	\\
4.27e-09	0	\\
4.37e-09	0	\\
4.47e-09	0	\\
4.57e-09	0	\\
4.68e-09	0	\\
4.78e-09	0	\\
4.89e-09	0	\\
4.99e-09	0	\\
5e-09	0	\\
};
\addplot [color=mycolor1,solid,forget plot]
  table[row sep=crcr]{
0	0	\\
1.1e-10	0	\\
2.2e-10	0	\\
3.3e-10	0	\\
4.4e-10	0	\\
5.4e-10	0	\\
6.5e-10	0	\\
7.5e-10	0	\\
8.6e-10	0	\\
9.6e-10	0	\\
1.07e-09	0	\\
1.18e-09	0	\\
1.28e-09	0	\\
1.38e-09	0	\\
1.49e-09	0	\\
1.59e-09	0	\\
1.69e-09	0	\\
1.8e-09	0	\\
1.9e-09	0	\\
2.01e-09	0	\\
2.11e-09	0	\\
2.21e-09	0	\\
2.32e-09	0	\\
2.42e-09	0	\\
2.52e-09	0	\\
2.63e-09	0	\\
2.73e-09	0	\\
2.83e-09	0	\\
2.93e-09	0	\\
3.04e-09	0	\\
3.14e-09	0	\\
3.24e-09	0	\\
3.34e-09	0	\\
3.45e-09	0	\\
3.55e-09	0	\\
3.65e-09	0	\\
3.75e-09	0	\\
3.86e-09	0	\\
3.96e-09	0	\\
4.06e-09	0	\\
4.16e-09	0	\\
4.27e-09	0	\\
4.37e-09	0	\\
4.47e-09	0	\\
4.57e-09	0	\\
4.68e-09	0	\\
4.78e-09	0	\\
4.89e-09	0	\\
4.99e-09	0	\\
5e-09	0	\\
};
\addplot [color=mycolor2,solid,forget plot]
  table[row sep=crcr]{
0	0	\\
1.1e-10	0	\\
2.2e-10	0	\\
3.3e-10	0	\\
4.4e-10	0	\\
5.4e-10	0	\\
6.5e-10	0	\\
7.5e-10	0	\\
8.6e-10	0	\\
9.6e-10	0	\\
1.07e-09	0	\\
1.18e-09	0	\\
1.28e-09	0	\\
1.38e-09	0	\\
1.49e-09	0	\\
1.59e-09	0	\\
1.69e-09	0	\\
1.8e-09	0	\\
1.9e-09	0	\\
2.01e-09	0	\\
2.11e-09	0	\\
2.21e-09	0	\\
2.32e-09	0	\\
2.42e-09	0	\\
2.52e-09	0	\\
2.63e-09	0	\\
2.73e-09	0	\\
2.83e-09	0	\\
2.93e-09	0	\\
3.04e-09	0	\\
3.14e-09	0	\\
3.24e-09	0	\\
3.34e-09	0	\\
3.45e-09	0	\\
3.55e-09	0	\\
3.65e-09	0	\\
3.75e-09	0	\\
3.86e-09	0	\\
3.96e-09	0	\\
4.06e-09	0	\\
4.16e-09	0	\\
4.27e-09	0	\\
4.37e-09	0	\\
4.47e-09	0	\\
4.57e-09	0	\\
4.68e-09	0	\\
4.78e-09	0	\\
4.89e-09	0	\\
4.99e-09	0	\\
5e-09	0	\\
};
\addplot [color=mycolor3,solid,forget plot]
  table[row sep=crcr]{
0	0	\\
1.1e-10	0	\\
2.2e-10	0	\\
3.3e-10	0	\\
4.4e-10	0	\\
5.4e-10	0	\\
6.5e-10	0	\\
7.5e-10	0	\\
8.6e-10	0	\\
9.6e-10	0	\\
1.07e-09	0	\\
1.18e-09	0	\\
1.28e-09	0	\\
1.38e-09	0	\\
1.49e-09	0	\\
1.59e-09	0	\\
1.69e-09	0	\\
1.8e-09	0	\\
1.9e-09	0	\\
2.01e-09	0	\\
2.11e-09	0	\\
2.21e-09	0	\\
2.32e-09	0	\\
2.42e-09	0	\\
2.52e-09	0	\\
2.63e-09	0	\\
2.73e-09	0	\\
2.83e-09	0	\\
2.93e-09	0	\\
3.04e-09	0	\\
3.14e-09	0	\\
3.24e-09	0	\\
3.34e-09	0	\\
3.45e-09	0	\\
3.55e-09	0	\\
3.65e-09	0	\\
3.75e-09	0	\\
3.86e-09	0	\\
3.96e-09	0	\\
4.06e-09	0	\\
4.16e-09	0	\\
4.27e-09	0	\\
4.37e-09	0	\\
4.47e-09	0	\\
4.57e-09	0	\\
4.68e-09	0	\\
4.78e-09	0	\\
4.89e-09	0	\\
4.99e-09	0	\\
5e-09	0	\\
};
\addplot [color=darkgray,solid,forget plot]
  table[row sep=crcr]{
0	0	\\
1.1e-10	0	\\
2.2e-10	0	\\
3.3e-10	0	\\
4.4e-10	0	\\
5.4e-10	0	\\
6.5e-10	0	\\
7.5e-10	0	\\
8.6e-10	0	\\
9.6e-10	0	\\
1.07e-09	0	\\
1.18e-09	0	\\
1.28e-09	0	\\
1.38e-09	0	\\
1.49e-09	0	\\
1.59e-09	0	\\
1.69e-09	0	\\
1.8e-09	0	\\
1.9e-09	0	\\
2.01e-09	0	\\
2.11e-09	0	\\
2.21e-09	0	\\
2.32e-09	0	\\
2.42e-09	0	\\
2.52e-09	0	\\
2.63e-09	0	\\
2.73e-09	0	\\
2.83e-09	0	\\
2.93e-09	0	\\
3.04e-09	0	\\
3.14e-09	0	\\
3.24e-09	0	\\
3.34e-09	0	\\
3.45e-09	0	\\
3.55e-09	0	\\
3.65e-09	0	\\
3.75e-09	0	\\
3.86e-09	0	\\
3.96e-09	0	\\
4.06e-09	0	\\
4.16e-09	0	\\
4.27e-09	0	\\
4.37e-09	0	\\
4.47e-09	0	\\
4.57e-09	0	\\
4.68e-09	0	\\
4.78e-09	0	\\
4.89e-09	0	\\
4.99e-09	0	\\
5e-09	0	\\
};
\addplot [color=blue,solid,forget plot]
  table[row sep=crcr]{
0	0	\\
1.1e-10	0	\\
2.2e-10	0	\\
3.3e-10	0	\\
4.4e-10	0	\\
5.4e-10	0	\\
6.5e-10	0	\\
7.5e-10	0	\\
8.6e-10	0	\\
9.6e-10	0	\\
1.07e-09	0	\\
1.18e-09	0	\\
1.28e-09	0	\\
1.38e-09	0	\\
1.49e-09	0	\\
1.59e-09	0	\\
1.69e-09	0	\\
1.8e-09	0	\\
1.9e-09	0	\\
2.01e-09	0	\\
2.11e-09	0	\\
2.21e-09	0	\\
2.32e-09	0	\\
2.42e-09	0	\\
2.52e-09	0	\\
2.63e-09	0	\\
2.73e-09	0	\\
2.83e-09	0	\\
2.93e-09	0	\\
3.04e-09	0	\\
3.14e-09	0	\\
3.24e-09	0	\\
3.34e-09	0	\\
3.45e-09	0	\\
3.55e-09	0	\\
3.65e-09	0	\\
3.75e-09	0	\\
3.86e-09	0	\\
3.96e-09	0	\\
4.06e-09	0	\\
4.16e-09	0	\\
4.27e-09	0	\\
4.37e-09	0	\\
4.47e-09	0	\\
4.57e-09	0	\\
4.68e-09	0	\\
4.78e-09	0	\\
4.89e-09	0	\\
4.99e-09	0	\\
5e-09	0	\\
};
\addplot [color=black!50!green,solid,forget plot]
  table[row sep=crcr]{
0	0	\\
1.1e-10	0	\\
2.2e-10	0	\\
3.3e-10	0	\\
4.4e-10	0	\\
5.4e-10	0	\\
6.5e-10	0	\\
7.5e-10	0	\\
8.6e-10	0	\\
9.6e-10	0	\\
1.07e-09	0	\\
1.18e-09	0	\\
1.28e-09	0	\\
1.38e-09	0	\\
1.49e-09	0	\\
1.59e-09	0	\\
1.69e-09	0	\\
1.8e-09	0	\\
1.9e-09	0	\\
2.01e-09	0	\\
2.11e-09	0	\\
2.21e-09	0	\\
2.32e-09	0	\\
2.42e-09	0	\\
2.52e-09	0	\\
2.63e-09	0	\\
2.73e-09	0	\\
2.83e-09	0	\\
2.93e-09	0	\\
3.04e-09	0	\\
3.14e-09	0	\\
3.24e-09	0	\\
3.34e-09	0	\\
3.45e-09	0	\\
3.55e-09	0	\\
3.65e-09	0	\\
3.75e-09	0	\\
3.86e-09	0	\\
3.96e-09	0	\\
4.06e-09	0	\\
4.16e-09	0	\\
4.27e-09	0	\\
4.37e-09	0	\\
4.47e-09	0	\\
4.57e-09	0	\\
4.68e-09	0	\\
4.78e-09	0	\\
4.89e-09	0	\\
4.99e-09	0	\\
5e-09	0	\\
};
\addplot [color=red,solid,forget plot]
  table[row sep=crcr]{
0	0	\\
1.1e-10	0	\\
2.2e-10	0	\\
3.3e-10	0	\\
4.4e-10	0	\\
5.4e-10	0	\\
6.5e-10	0	\\
7.5e-10	0	\\
8.6e-10	0	\\
9.6e-10	0	\\
1.07e-09	0	\\
1.18e-09	0	\\
1.28e-09	0	\\
1.38e-09	0	\\
1.49e-09	0	\\
1.59e-09	0	\\
1.69e-09	0	\\
1.8e-09	0	\\
1.9e-09	0	\\
2.01e-09	0	\\
2.11e-09	0	\\
2.21e-09	0	\\
2.32e-09	0	\\
2.42e-09	0	\\
2.52e-09	0	\\
2.63e-09	0	\\
2.73e-09	0	\\
2.83e-09	0	\\
2.93e-09	0	\\
3.04e-09	0	\\
3.14e-09	0	\\
3.24e-09	0	\\
3.34e-09	0	\\
3.45e-09	0	\\
3.55e-09	0	\\
3.65e-09	0	\\
3.75e-09	0	\\
3.86e-09	0	\\
3.96e-09	0	\\
4.06e-09	0	\\
4.16e-09	0	\\
4.27e-09	0	\\
4.37e-09	0	\\
4.47e-09	0	\\
4.57e-09	0	\\
4.68e-09	0	\\
4.78e-09	0	\\
4.89e-09	0	\\
4.99e-09	0	\\
5e-09	0	\\
};
\addplot [color=mycolor1,solid,forget plot]
  table[row sep=crcr]{
0	0	\\
1.1e-10	0	\\
2.2e-10	0	\\
3.3e-10	0	\\
4.4e-10	0	\\
5.4e-10	0	\\
6.5e-10	0	\\
7.5e-10	0	\\
8.6e-10	0	\\
9.6e-10	0	\\
1.07e-09	0	\\
1.18e-09	0	\\
1.28e-09	0	\\
1.38e-09	0	\\
1.49e-09	0	\\
1.59e-09	0	\\
1.69e-09	0	\\
1.8e-09	0	\\
1.9e-09	0	\\
2.01e-09	0	\\
2.11e-09	0	\\
2.21e-09	0	\\
2.32e-09	0	\\
2.42e-09	0	\\
2.52e-09	0	\\
2.63e-09	0	\\
2.73e-09	0	\\
2.83e-09	0	\\
2.93e-09	0	\\
3.04e-09	0	\\
3.14e-09	0	\\
3.24e-09	0	\\
3.34e-09	0	\\
3.45e-09	0	\\
3.55e-09	0	\\
3.65e-09	0	\\
3.75e-09	0	\\
3.86e-09	0	\\
3.96e-09	0	\\
4.06e-09	0	\\
4.16e-09	0	\\
4.27e-09	0	\\
4.37e-09	0	\\
4.47e-09	0	\\
4.57e-09	0	\\
4.68e-09	0	\\
4.78e-09	0	\\
4.89e-09	0	\\
4.99e-09	0	\\
5e-09	0	\\
};
\addplot [color=mycolor2,solid,forget plot]
  table[row sep=crcr]{
0	0	\\
1.1e-10	0	\\
2.2e-10	0	\\
3.3e-10	0	\\
4.4e-10	0	\\
5.4e-10	0	\\
6.5e-10	0	\\
7.5e-10	0	\\
8.6e-10	0	\\
9.6e-10	0	\\
1.07e-09	0	\\
1.18e-09	0	\\
1.28e-09	0	\\
1.38e-09	0	\\
1.49e-09	0	\\
1.59e-09	0	\\
1.69e-09	0	\\
1.8e-09	0	\\
1.9e-09	0	\\
2.01e-09	0	\\
2.11e-09	0	\\
2.21e-09	0	\\
2.32e-09	0	\\
2.42e-09	0	\\
2.52e-09	0	\\
2.63e-09	0	\\
2.73e-09	0	\\
2.83e-09	0	\\
2.93e-09	0	\\
3.04e-09	0	\\
3.14e-09	0	\\
3.24e-09	0	\\
3.34e-09	0	\\
3.45e-09	0	\\
3.55e-09	0	\\
3.65e-09	0	\\
3.75e-09	0	\\
3.86e-09	0	\\
3.96e-09	0	\\
4.06e-09	0	\\
4.16e-09	0	\\
4.27e-09	0	\\
4.37e-09	0	\\
4.47e-09	0	\\
4.57e-09	0	\\
4.68e-09	0	\\
4.78e-09	0	\\
4.89e-09	0	\\
4.99e-09	0	\\
5e-09	0	\\
};
\addplot [color=mycolor3,solid,forget plot]
  table[row sep=crcr]{
0	0	\\
1.1e-10	0	\\
2.2e-10	0	\\
3.3e-10	0	\\
4.4e-10	0	\\
5.4e-10	0	\\
6.5e-10	0	\\
7.5e-10	0	\\
8.6e-10	0	\\
9.6e-10	0	\\
1.07e-09	0	\\
1.18e-09	0	\\
1.28e-09	0	\\
1.38e-09	0	\\
1.49e-09	0	\\
1.59e-09	0	\\
1.69e-09	0	\\
1.8e-09	0	\\
1.9e-09	0	\\
2.01e-09	0	\\
2.11e-09	0	\\
2.21e-09	0	\\
2.32e-09	0	\\
2.42e-09	0	\\
2.52e-09	0	\\
2.63e-09	0	\\
2.73e-09	0	\\
2.83e-09	0	\\
2.93e-09	0	\\
3.04e-09	0	\\
3.14e-09	0	\\
3.24e-09	0	\\
3.34e-09	0	\\
3.45e-09	0	\\
3.55e-09	0	\\
3.65e-09	0	\\
3.75e-09	0	\\
3.86e-09	0	\\
3.96e-09	0	\\
4.06e-09	0	\\
4.16e-09	0	\\
4.27e-09	0	\\
4.37e-09	0	\\
4.47e-09	0	\\
4.57e-09	0	\\
4.68e-09	0	\\
4.78e-09	0	\\
4.89e-09	0	\\
4.99e-09	0	\\
5e-09	0	\\
};
\addplot [color=darkgray,solid,forget plot]
  table[row sep=crcr]{
0	0	\\
1.1e-10	0	\\
2.2e-10	0	\\
3.3e-10	0	\\
4.4e-10	0	\\
5.4e-10	0	\\
6.5e-10	0	\\
7.5e-10	0	\\
8.6e-10	0	\\
9.6e-10	0	\\
1.07e-09	0	\\
1.18e-09	0	\\
1.28e-09	0	\\
1.38e-09	0	\\
1.49e-09	0	\\
1.59e-09	0	\\
1.69e-09	0	\\
1.8e-09	0	\\
1.9e-09	0	\\
2.01e-09	0	\\
2.11e-09	0	\\
2.21e-09	0	\\
2.32e-09	0	\\
2.42e-09	0	\\
2.52e-09	0	\\
2.63e-09	0	\\
2.73e-09	0	\\
2.83e-09	0	\\
2.93e-09	0	\\
3.04e-09	0	\\
3.14e-09	0	\\
3.24e-09	0	\\
3.34e-09	0	\\
3.45e-09	0	\\
3.55e-09	0	\\
3.65e-09	0	\\
3.75e-09	0	\\
3.86e-09	0	\\
3.96e-09	0	\\
4.06e-09	0	\\
4.16e-09	0	\\
4.27e-09	0	\\
4.37e-09	0	\\
4.47e-09	0	\\
4.57e-09	0	\\
4.68e-09	0	\\
4.78e-09	0	\\
4.89e-09	0	\\
4.99e-09	0	\\
5e-09	0	\\
};
\addplot [color=blue,solid,forget plot]
  table[row sep=crcr]{
0	0	\\
1.1e-10	0	\\
2.2e-10	0	\\
3.3e-10	0	\\
4.4e-10	0	\\
5.4e-10	0	\\
6.5e-10	0	\\
7.5e-10	0	\\
8.6e-10	0	\\
9.6e-10	0	\\
1.07e-09	0	\\
1.18e-09	0	\\
1.28e-09	0	\\
1.38e-09	0	\\
1.49e-09	0	\\
1.59e-09	0	\\
1.69e-09	0	\\
1.8e-09	0	\\
1.9e-09	0	\\
2.01e-09	0	\\
2.11e-09	0	\\
2.21e-09	0	\\
2.32e-09	0	\\
2.42e-09	0	\\
2.52e-09	0	\\
2.63e-09	0	\\
2.73e-09	0	\\
2.83e-09	0	\\
2.93e-09	0	\\
3.04e-09	0	\\
3.14e-09	0	\\
3.24e-09	0	\\
3.34e-09	0	\\
3.45e-09	0	\\
3.55e-09	0	\\
3.65e-09	0	\\
3.75e-09	0	\\
3.86e-09	0	\\
3.96e-09	0	\\
4.06e-09	0	\\
4.16e-09	0	\\
4.27e-09	0	\\
4.37e-09	0	\\
4.47e-09	0	\\
4.57e-09	0	\\
4.68e-09	0	\\
4.78e-09	0	\\
4.89e-09	0	\\
4.99e-09	0	\\
5e-09	0	\\
};
\addplot [color=black!50!green,solid,forget plot]
  table[row sep=crcr]{
0	0	\\
1.1e-10	0	\\
2.2e-10	0	\\
3.3e-10	0	\\
4.4e-10	0	\\
5.4e-10	0	\\
6.5e-10	0	\\
7.5e-10	0	\\
8.6e-10	0	\\
9.6e-10	0	\\
1.07e-09	0	\\
1.18e-09	0	\\
1.28e-09	0	\\
1.38e-09	0	\\
1.49e-09	0	\\
1.59e-09	0	\\
1.69e-09	0	\\
1.8e-09	0	\\
1.9e-09	0	\\
2.01e-09	0	\\
2.11e-09	0	\\
2.21e-09	0	\\
2.32e-09	0	\\
2.42e-09	0	\\
2.52e-09	0	\\
2.63e-09	0	\\
2.73e-09	0	\\
2.83e-09	0	\\
2.93e-09	0	\\
3.04e-09	0	\\
3.14e-09	0	\\
3.24e-09	0	\\
3.34e-09	0	\\
3.45e-09	0	\\
3.55e-09	0	\\
3.65e-09	0	\\
3.75e-09	0	\\
3.86e-09	0	\\
3.96e-09	0	\\
4.06e-09	0	\\
4.16e-09	0	\\
4.27e-09	0	\\
4.37e-09	0	\\
4.47e-09	0	\\
4.57e-09	0	\\
4.68e-09	0	\\
4.78e-09	0	\\
4.89e-09	0	\\
4.99e-09	0	\\
5e-09	0	\\
};
\addplot [color=red,solid,forget plot]
  table[row sep=crcr]{
0	0	\\
1.1e-10	0	\\
2.2e-10	0	\\
3.3e-10	0	\\
4.4e-10	0	\\
5.4e-10	0	\\
6.5e-10	0	\\
7.5e-10	0	\\
8.6e-10	0	\\
9.6e-10	0	\\
1.07e-09	0	\\
1.18e-09	0	\\
1.28e-09	0	\\
1.38e-09	0	\\
1.49e-09	0	\\
1.59e-09	0	\\
1.69e-09	0	\\
1.8e-09	0	\\
1.9e-09	0	\\
2.01e-09	0	\\
2.11e-09	0	\\
2.21e-09	0	\\
2.32e-09	0	\\
2.42e-09	0	\\
2.52e-09	0	\\
2.63e-09	0	\\
2.73e-09	0	\\
2.83e-09	0	\\
2.93e-09	0	\\
3.04e-09	0	\\
3.14e-09	0	\\
3.24e-09	0	\\
3.34e-09	0	\\
3.45e-09	0	\\
3.55e-09	0	\\
3.65e-09	0	\\
3.75e-09	0	\\
3.86e-09	0	\\
3.96e-09	0	\\
4.06e-09	0	\\
4.16e-09	0	\\
4.27e-09	0	\\
4.37e-09	0	\\
4.47e-09	0	\\
4.57e-09	0	\\
4.68e-09	0	\\
4.78e-09	0	\\
4.89e-09	0	\\
4.99e-09	0	\\
5e-09	0	\\
};
\addplot [color=mycolor1,solid,forget plot]
  table[row sep=crcr]{
0	0	\\
1.1e-10	0	\\
2.2e-10	0	\\
3.3e-10	0	\\
4.4e-10	0	\\
5.4e-10	0	\\
6.5e-10	0	\\
7.5e-10	0	\\
8.6e-10	0	\\
9.6e-10	0	\\
1.07e-09	0	\\
1.18e-09	0	\\
1.28e-09	0	\\
1.38e-09	0	\\
1.49e-09	0	\\
1.59e-09	0	\\
1.69e-09	0	\\
1.8e-09	0	\\
1.9e-09	0	\\
2.01e-09	0	\\
2.11e-09	0	\\
2.21e-09	0	\\
2.32e-09	0	\\
2.42e-09	0	\\
2.52e-09	0	\\
2.63e-09	0	\\
2.73e-09	0	\\
2.83e-09	0	\\
2.93e-09	0	\\
3.04e-09	0	\\
3.14e-09	0	\\
3.24e-09	0	\\
3.34e-09	0	\\
3.45e-09	0	\\
3.55e-09	0	\\
3.65e-09	0	\\
3.75e-09	0	\\
3.86e-09	0	\\
3.96e-09	0	\\
4.06e-09	0	\\
4.16e-09	0	\\
4.27e-09	0	\\
4.37e-09	0	\\
4.47e-09	0	\\
4.57e-09	0	\\
4.68e-09	0	\\
4.78e-09	0	\\
4.89e-09	0	\\
4.99e-09	0	\\
5e-09	0	\\
};
\addplot [color=mycolor2,solid,forget plot]
  table[row sep=crcr]{
0	0	\\
1.1e-10	0	\\
2.2e-10	0	\\
3.3e-10	0	\\
4.4e-10	0	\\
5.4e-10	0	\\
6.5e-10	0	\\
7.5e-10	0	\\
8.6e-10	0	\\
9.6e-10	0	\\
1.07e-09	0	\\
1.18e-09	0	\\
1.28e-09	0	\\
1.38e-09	0	\\
1.49e-09	0	\\
1.59e-09	0	\\
1.69e-09	0	\\
1.8e-09	0	\\
1.9e-09	0	\\
2.01e-09	0	\\
2.11e-09	0	\\
2.21e-09	0	\\
2.32e-09	0	\\
2.42e-09	0	\\
2.52e-09	0	\\
2.63e-09	0	\\
2.73e-09	0	\\
2.83e-09	0	\\
2.93e-09	0	\\
3.04e-09	0	\\
3.14e-09	0	\\
3.24e-09	0	\\
3.34e-09	0	\\
3.45e-09	0	\\
3.55e-09	0	\\
3.65e-09	0	\\
3.75e-09	0	\\
3.86e-09	0	\\
3.96e-09	0	\\
4.06e-09	0	\\
4.16e-09	0	\\
4.27e-09	0	\\
4.37e-09	0	\\
4.47e-09	0	\\
4.57e-09	0	\\
4.68e-09	0	\\
4.78e-09	0	\\
4.89e-09	0	\\
4.99e-09	0	\\
5e-09	0	\\
};
\addplot [color=mycolor3,solid,forget plot]
  table[row sep=crcr]{
0	0	\\
1.1e-10	0	\\
2.2e-10	0	\\
3.3e-10	0	\\
4.4e-10	0	\\
5.4e-10	0	\\
6.5e-10	0	\\
7.5e-10	0	\\
8.6e-10	0	\\
9.6e-10	0	\\
1.07e-09	0	\\
1.18e-09	0	\\
1.28e-09	0	\\
1.38e-09	0	\\
1.49e-09	0	\\
1.59e-09	0	\\
1.69e-09	0	\\
1.8e-09	0	\\
1.9e-09	0	\\
2.01e-09	0	\\
2.11e-09	0	\\
2.21e-09	0	\\
2.32e-09	0	\\
2.42e-09	0	\\
2.52e-09	0	\\
2.63e-09	0	\\
2.73e-09	0	\\
2.83e-09	0	\\
2.93e-09	0	\\
3.04e-09	0	\\
3.14e-09	0	\\
3.24e-09	0	\\
3.34e-09	0	\\
3.45e-09	0	\\
3.55e-09	0	\\
3.65e-09	0	\\
3.75e-09	0	\\
3.86e-09	0	\\
3.96e-09	0	\\
4.06e-09	0	\\
4.16e-09	0	\\
4.27e-09	0	\\
4.37e-09	0	\\
4.47e-09	0	\\
4.57e-09	0	\\
4.68e-09	0	\\
4.78e-09	0	\\
4.89e-09	0	\\
4.99e-09	0	\\
5e-09	0	\\
};
\addplot [color=darkgray,solid,forget plot]
  table[row sep=crcr]{
0	0	\\
1.1e-10	0	\\
2.2e-10	0	\\
3.3e-10	0	\\
4.4e-10	0	\\
5.4e-10	0	\\
6.5e-10	0	\\
7.5e-10	0	\\
8.6e-10	0	\\
9.6e-10	0	\\
1.07e-09	0	\\
1.18e-09	0	\\
1.28e-09	0	\\
1.38e-09	0	\\
1.49e-09	0	\\
1.59e-09	0	\\
1.69e-09	0	\\
1.8e-09	0	\\
1.9e-09	0	\\
2.01e-09	0	\\
2.11e-09	0	\\
2.21e-09	0	\\
2.32e-09	0	\\
2.42e-09	0	\\
2.52e-09	0	\\
2.63e-09	0	\\
2.73e-09	0	\\
2.83e-09	0	\\
2.93e-09	0	\\
3.04e-09	0	\\
3.14e-09	0	\\
3.24e-09	0	\\
3.34e-09	0	\\
3.45e-09	0	\\
3.55e-09	0	\\
3.65e-09	0	\\
3.75e-09	0	\\
3.86e-09	0	\\
3.96e-09	0	\\
4.06e-09	0	\\
4.16e-09	0	\\
4.27e-09	0	\\
4.37e-09	0	\\
4.47e-09	0	\\
4.57e-09	0	\\
4.68e-09	0	\\
4.78e-09	0	\\
4.89e-09	0	\\
4.99e-09	0	\\
5e-09	0	\\
};
\addplot [color=blue,solid,forget plot]
  table[row sep=crcr]{
0	0	\\
1.1e-10	0	\\
2.2e-10	0	\\
3.3e-10	0	\\
4.4e-10	0	\\
5.4e-10	0	\\
6.5e-10	0	\\
7.5e-10	0	\\
8.6e-10	0	\\
9.6e-10	0	\\
1.07e-09	0	\\
1.18e-09	0	\\
1.28e-09	0	\\
1.38e-09	0	\\
1.49e-09	0	\\
1.59e-09	0	\\
1.69e-09	0	\\
1.8e-09	0	\\
1.9e-09	0	\\
2.01e-09	0	\\
2.11e-09	0	\\
2.21e-09	0	\\
2.32e-09	0	\\
2.42e-09	0	\\
2.52e-09	0	\\
2.63e-09	0	\\
2.73e-09	0	\\
2.83e-09	0	\\
2.93e-09	0	\\
3.04e-09	0	\\
3.14e-09	0	\\
3.24e-09	0	\\
3.34e-09	0	\\
3.45e-09	0	\\
3.55e-09	0	\\
3.65e-09	0	\\
3.75e-09	0	\\
3.86e-09	0	\\
3.96e-09	0	\\
4.06e-09	0	\\
4.16e-09	0	\\
4.27e-09	0	\\
4.37e-09	0	\\
4.47e-09	0	\\
4.57e-09	0	\\
4.68e-09	0	\\
4.78e-09	0	\\
4.89e-09	0	\\
4.99e-09	0	\\
5e-09	0	\\
};
\addplot [color=black!50!green,solid,forget plot]
  table[row sep=crcr]{
0	0	\\
1.1e-10	0	\\
2.2e-10	0	\\
3.3e-10	0	\\
4.4e-10	0	\\
5.4e-10	0	\\
6.5e-10	0	\\
7.5e-10	0	\\
8.6e-10	0	\\
9.6e-10	0	\\
1.07e-09	0	\\
1.18e-09	0	\\
1.28e-09	0	\\
1.38e-09	0	\\
1.49e-09	0	\\
1.59e-09	0	\\
1.69e-09	0	\\
1.8e-09	0	\\
1.9e-09	0	\\
2.01e-09	0	\\
2.11e-09	0	\\
2.21e-09	0	\\
2.32e-09	0	\\
2.42e-09	0	\\
2.52e-09	0	\\
2.63e-09	0	\\
2.73e-09	0	\\
2.83e-09	0	\\
2.93e-09	0	\\
3.04e-09	0	\\
3.14e-09	0	\\
3.24e-09	0	\\
3.34e-09	0	\\
3.45e-09	0	\\
3.55e-09	0	\\
3.65e-09	0	\\
3.75e-09	0	\\
3.86e-09	0	\\
3.96e-09	0	\\
4.06e-09	0	\\
4.16e-09	0	\\
4.27e-09	0	\\
4.37e-09	0	\\
4.47e-09	0	\\
4.57e-09	0	\\
4.68e-09	0	\\
4.78e-09	0	\\
4.89e-09	0	\\
4.99e-09	0	\\
5e-09	0	\\
};
\addplot [color=red,solid,forget plot]
  table[row sep=crcr]{
0	0	\\
1.1e-10	0	\\
2.2e-10	0	\\
3.3e-10	0	\\
4.4e-10	0	\\
5.4e-10	0	\\
6.5e-10	0	\\
7.5e-10	0	\\
8.6e-10	0	\\
9.6e-10	0	\\
1.07e-09	0	\\
1.18e-09	0	\\
1.28e-09	0	\\
1.38e-09	0	\\
1.49e-09	0	\\
1.59e-09	0	\\
1.69e-09	0	\\
1.8e-09	0	\\
1.9e-09	0	\\
2.01e-09	0	\\
2.11e-09	0	\\
2.21e-09	0	\\
2.32e-09	0	\\
2.42e-09	0	\\
2.52e-09	0	\\
2.63e-09	0	\\
2.73e-09	0	\\
2.83e-09	0	\\
2.93e-09	0	\\
3.04e-09	0	\\
3.14e-09	0	\\
3.24e-09	0	\\
3.34e-09	0	\\
3.45e-09	0	\\
3.55e-09	0	\\
3.65e-09	0	\\
3.75e-09	0	\\
3.86e-09	0	\\
3.96e-09	0	\\
4.06e-09	0	\\
4.16e-09	0	\\
4.27e-09	0	\\
4.37e-09	0	\\
4.47e-09	0	\\
4.57e-09	0	\\
4.68e-09	0	\\
4.78e-09	0	\\
4.89e-09	0	\\
4.99e-09	0	\\
5e-09	0	\\
};
\addplot [color=mycolor1,solid,forget plot]
  table[row sep=crcr]{
0	0	\\
1.1e-10	0	\\
2.2e-10	0	\\
3.3e-10	0	\\
4.4e-10	0	\\
5.4e-10	0	\\
6.5e-10	0	\\
7.5e-10	0	\\
8.6e-10	0	\\
9.6e-10	0	\\
1.07e-09	0	\\
1.18e-09	0	\\
1.28e-09	0	\\
1.38e-09	0	\\
1.49e-09	0	\\
1.59e-09	0	\\
1.69e-09	0	\\
1.8e-09	0	\\
1.9e-09	0	\\
2.01e-09	0	\\
2.11e-09	0	\\
2.21e-09	0	\\
2.32e-09	0	\\
2.42e-09	0	\\
2.52e-09	0	\\
2.63e-09	0	\\
2.73e-09	0	\\
2.83e-09	0	\\
2.93e-09	0	\\
3.04e-09	0	\\
3.14e-09	0	\\
3.24e-09	0	\\
3.34e-09	0	\\
3.45e-09	0	\\
3.55e-09	0	\\
3.65e-09	0	\\
3.75e-09	0	\\
3.86e-09	0	\\
3.96e-09	0	\\
4.06e-09	0	\\
4.16e-09	0	\\
4.27e-09	0	\\
4.37e-09	0	\\
4.47e-09	0	\\
4.57e-09	0	\\
4.68e-09	0	\\
4.78e-09	0	\\
4.89e-09	0	\\
4.99e-09	0	\\
5e-09	0	\\
};
\addplot [color=mycolor2,solid,forget plot]
  table[row sep=crcr]{
0	0	\\
1.1e-10	0	\\
2.2e-10	0	\\
3.3e-10	0	\\
4.4e-10	0	\\
5.4e-10	0	\\
6.5e-10	0	\\
7.5e-10	0	\\
8.6e-10	0	\\
9.6e-10	0	\\
1.07e-09	0	\\
1.18e-09	0	\\
1.28e-09	0	\\
1.38e-09	0	\\
1.49e-09	0	\\
1.59e-09	0	\\
1.69e-09	0	\\
1.8e-09	0	\\
1.9e-09	0	\\
2.01e-09	0	\\
2.11e-09	0	\\
2.21e-09	0	\\
2.32e-09	0	\\
2.42e-09	0	\\
2.52e-09	0	\\
2.63e-09	0	\\
2.73e-09	0	\\
2.83e-09	0	\\
2.93e-09	0	\\
3.04e-09	0	\\
3.14e-09	0	\\
3.24e-09	0	\\
3.34e-09	0	\\
3.45e-09	0	\\
3.55e-09	0	\\
3.65e-09	0	\\
3.75e-09	0	\\
3.86e-09	0	\\
3.96e-09	0	\\
4.06e-09	0	\\
4.16e-09	0	\\
4.27e-09	0	\\
4.37e-09	0	\\
4.47e-09	0	\\
4.57e-09	0	\\
4.68e-09	0	\\
4.78e-09	0	\\
4.89e-09	0	\\
4.99e-09	0	\\
5e-09	0	\\
};
\addplot [color=mycolor3,solid,forget plot]
  table[row sep=crcr]{
0	0	\\
1.1e-10	0	\\
2.2e-10	0	\\
3.3e-10	0	\\
4.4e-10	0	\\
5.4e-10	0	\\
6.5e-10	0	\\
7.5e-10	0	\\
8.6e-10	0	\\
9.6e-10	0	\\
1.07e-09	0	\\
1.18e-09	0	\\
1.28e-09	0	\\
1.38e-09	0	\\
1.49e-09	0	\\
1.59e-09	0	\\
1.69e-09	0	\\
1.8e-09	0	\\
1.9e-09	0	\\
2.01e-09	0	\\
2.11e-09	0	\\
2.21e-09	0	\\
2.32e-09	0	\\
2.42e-09	0	\\
2.52e-09	0	\\
2.63e-09	0	\\
2.73e-09	0	\\
2.83e-09	0	\\
2.93e-09	0	\\
3.04e-09	0	\\
3.14e-09	0	\\
3.24e-09	0	\\
3.34e-09	0	\\
3.45e-09	0	\\
3.55e-09	0	\\
3.65e-09	0	\\
3.75e-09	0	\\
3.86e-09	0	\\
3.96e-09	0	\\
4.06e-09	0	\\
4.16e-09	0	\\
4.27e-09	0	\\
4.37e-09	0	\\
4.47e-09	0	\\
4.57e-09	0	\\
4.68e-09	0	\\
4.78e-09	0	\\
4.89e-09	0	\\
4.99e-09	0	\\
5e-09	0	\\
};
\addplot [color=darkgray,solid,forget plot]
  table[row sep=crcr]{
0	0	\\
1.1e-10	0	\\
2.2e-10	0	\\
3.3e-10	0	\\
4.4e-10	0	\\
5.4e-10	0	\\
6.5e-10	0	\\
7.5e-10	0	\\
8.6e-10	0	\\
9.6e-10	0	\\
1.07e-09	0	\\
1.18e-09	0	\\
1.28e-09	0	\\
1.38e-09	0	\\
1.49e-09	0	\\
1.59e-09	0	\\
1.69e-09	0	\\
1.8e-09	0	\\
1.9e-09	0	\\
2.01e-09	0	\\
2.11e-09	0	\\
2.21e-09	0	\\
2.32e-09	0	\\
2.42e-09	0	\\
2.52e-09	0	\\
2.63e-09	0	\\
2.73e-09	0	\\
2.83e-09	0	\\
2.93e-09	0	\\
3.04e-09	0	\\
3.14e-09	0	\\
3.24e-09	0	\\
3.34e-09	0	\\
3.45e-09	0	\\
3.55e-09	0	\\
3.65e-09	0	\\
3.75e-09	0	\\
3.86e-09	0	\\
3.96e-09	0	\\
4.06e-09	0	\\
4.16e-09	0	\\
4.27e-09	0	\\
4.37e-09	0	\\
4.47e-09	0	\\
4.57e-09	0	\\
4.68e-09	0	\\
4.78e-09	0	\\
4.89e-09	0	\\
4.99e-09	0	\\
5e-09	0	\\
};
\addplot [color=blue,solid,forget plot]
  table[row sep=crcr]{
0	0	\\
1.1e-10	0	\\
2.2e-10	0	\\
3.3e-10	0	\\
4.4e-10	0	\\
5.4e-10	0	\\
6.5e-10	0	\\
7.5e-10	0	\\
8.6e-10	0	\\
9.6e-10	0	\\
1.07e-09	0	\\
1.18e-09	0	\\
1.28e-09	0	\\
1.38e-09	0	\\
1.49e-09	0	\\
1.59e-09	0	\\
1.69e-09	0	\\
1.8e-09	0	\\
1.9e-09	0	\\
2.01e-09	0	\\
2.11e-09	0	\\
2.21e-09	0	\\
2.32e-09	0	\\
2.42e-09	0	\\
2.52e-09	0	\\
2.63e-09	0	\\
2.73e-09	0	\\
2.83e-09	0	\\
2.93e-09	0	\\
3.04e-09	0	\\
3.14e-09	0	\\
3.24e-09	0	\\
3.34e-09	0	\\
3.45e-09	0	\\
3.55e-09	0	\\
3.65e-09	0	\\
3.75e-09	0	\\
3.86e-09	0	\\
3.96e-09	0	\\
4.06e-09	0	\\
4.16e-09	0	\\
4.27e-09	0	\\
4.37e-09	0	\\
4.47e-09	0	\\
4.57e-09	0	\\
4.68e-09	0	\\
4.78e-09	0	\\
4.89e-09	0	\\
4.99e-09	0	\\
5e-09	0	\\
};
\addplot [color=black!50!green,solid,forget plot]
  table[row sep=crcr]{
0	0	\\
1.1e-10	0	\\
2.2e-10	0	\\
3.3e-10	0	\\
4.4e-10	0	\\
5.4e-10	0	\\
6.5e-10	0	\\
7.5e-10	0	\\
8.6e-10	0	\\
9.6e-10	0	\\
1.07e-09	0	\\
1.18e-09	0	\\
1.28e-09	0	\\
1.38e-09	0	\\
1.49e-09	0	\\
1.59e-09	0	\\
1.69e-09	0	\\
1.8e-09	0	\\
1.9e-09	0	\\
2.01e-09	0	\\
2.11e-09	0	\\
2.21e-09	0	\\
2.32e-09	0	\\
2.42e-09	0	\\
2.52e-09	0	\\
2.63e-09	0	\\
2.73e-09	0	\\
2.83e-09	0	\\
2.93e-09	0	\\
3.04e-09	0	\\
3.14e-09	0	\\
3.24e-09	0	\\
3.34e-09	0	\\
3.45e-09	0	\\
3.55e-09	0	\\
3.65e-09	0	\\
3.75e-09	0	\\
3.86e-09	0	\\
3.96e-09	0	\\
4.06e-09	0	\\
4.16e-09	0	\\
4.27e-09	0	\\
4.37e-09	0	\\
4.47e-09	0	\\
4.57e-09	0	\\
4.68e-09	0	\\
4.78e-09	0	\\
4.89e-09	0	\\
4.99e-09	0	\\
5e-09	0	\\
};
\addplot [color=red,solid,forget plot]
  table[row sep=crcr]{
0	0	\\
1.1e-10	0	\\
2.2e-10	0	\\
3.3e-10	0	\\
4.4e-10	0	\\
5.4e-10	0	\\
6.5e-10	0	\\
7.5e-10	0	\\
8.6e-10	0	\\
9.6e-10	0	\\
1.07e-09	0	\\
1.18e-09	0	\\
1.28e-09	0	\\
1.38e-09	0	\\
1.49e-09	0	\\
1.59e-09	0	\\
1.69e-09	0	\\
1.8e-09	0	\\
1.9e-09	0	\\
2.01e-09	0	\\
2.11e-09	0	\\
2.21e-09	0	\\
2.32e-09	0	\\
2.42e-09	0	\\
2.52e-09	0	\\
2.63e-09	0	\\
2.73e-09	0	\\
2.83e-09	0	\\
2.93e-09	0	\\
3.04e-09	0	\\
3.14e-09	0	\\
3.24e-09	0	\\
3.34e-09	0	\\
3.45e-09	0	\\
3.55e-09	0	\\
3.65e-09	0	\\
3.75e-09	0	\\
3.86e-09	0	\\
3.96e-09	0	\\
4.06e-09	0	\\
4.16e-09	0	\\
4.27e-09	0	\\
4.37e-09	0	\\
4.47e-09	0	\\
4.57e-09	0	\\
4.68e-09	0	\\
4.78e-09	0	\\
4.89e-09	0	\\
4.99e-09	0	\\
5e-09	0	\\
};
\addplot [color=mycolor1,solid,forget plot]
  table[row sep=crcr]{
0	0	\\
1.1e-10	0	\\
2.2e-10	0	\\
3.3e-10	0	\\
4.4e-10	0	\\
5.4e-10	0	\\
6.5e-10	0	\\
7.5e-10	0	\\
8.6e-10	0	\\
9.6e-10	0	\\
1.07e-09	0	\\
1.18e-09	0	\\
1.28e-09	0	\\
1.38e-09	0	\\
1.49e-09	0	\\
1.59e-09	0	\\
1.69e-09	0	\\
1.8e-09	0	\\
1.9e-09	0	\\
2.01e-09	0	\\
2.11e-09	0	\\
2.21e-09	0	\\
2.32e-09	0	\\
2.42e-09	0	\\
2.52e-09	0	\\
2.63e-09	0	\\
2.73e-09	0	\\
2.83e-09	0	\\
2.93e-09	0	\\
3.04e-09	0	\\
3.14e-09	0	\\
3.24e-09	0	\\
3.34e-09	0	\\
3.45e-09	0	\\
3.55e-09	0	\\
3.65e-09	0	\\
3.75e-09	0	\\
3.86e-09	0	\\
3.96e-09	0	\\
4.06e-09	0	\\
4.16e-09	0	\\
4.27e-09	0	\\
4.37e-09	0	\\
4.47e-09	0	\\
4.57e-09	0	\\
4.68e-09	0	\\
4.78e-09	0	\\
4.89e-09	0	\\
4.99e-09	0	\\
5e-09	0	\\
};
\addplot [color=mycolor2,solid,forget plot]
  table[row sep=crcr]{
0	0	\\
1.1e-10	0	\\
2.2e-10	0	\\
3.3e-10	0	\\
4.4e-10	0	\\
5.4e-10	0	\\
6.5e-10	0	\\
7.5e-10	0	\\
8.6e-10	0	\\
9.6e-10	0	\\
1.07e-09	0	\\
1.18e-09	0	\\
1.28e-09	0	\\
1.38e-09	0	\\
1.49e-09	0	\\
1.59e-09	0	\\
1.69e-09	0	\\
1.8e-09	0	\\
1.9e-09	0	\\
2.01e-09	0	\\
2.11e-09	0	\\
2.21e-09	0	\\
2.32e-09	0	\\
2.42e-09	0	\\
2.52e-09	0	\\
2.63e-09	0	\\
2.73e-09	0	\\
2.83e-09	0	\\
2.93e-09	0	\\
3.04e-09	0	\\
3.14e-09	0	\\
3.24e-09	0	\\
3.34e-09	0	\\
3.45e-09	0	\\
3.55e-09	0	\\
3.65e-09	0	\\
3.75e-09	0	\\
3.86e-09	0	\\
3.96e-09	0	\\
4.06e-09	0	\\
4.16e-09	0	\\
4.27e-09	0	\\
4.37e-09	0	\\
4.47e-09	0	\\
4.57e-09	0	\\
4.68e-09	0	\\
4.78e-09	0	\\
4.89e-09	0	\\
4.99e-09	0	\\
5e-09	0	\\
};
\addplot [color=mycolor3,solid,forget plot]
  table[row sep=crcr]{
0	0	\\
1.1e-10	0	\\
2.2e-10	0	\\
3.3e-10	0	\\
4.4e-10	0	\\
5.4e-10	0	\\
6.5e-10	0	\\
7.5e-10	0	\\
8.6e-10	0	\\
9.6e-10	0	\\
1.07e-09	0	\\
1.18e-09	0	\\
1.28e-09	0	\\
1.38e-09	0	\\
1.49e-09	0	\\
1.59e-09	0	\\
1.69e-09	0	\\
1.8e-09	0	\\
1.9e-09	0	\\
2.01e-09	0	\\
2.11e-09	0	\\
2.21e-09	0	\\
2.32e-09	0	\\
2.42e-09	0	\\
2.52e-09	0	\\
2.63e-09	0	\\
2.73e-09	0	\\
2.83e-09	0	\\
2.93e-09	0	\\
3.04e-09	0	\\
3.14e-09	0	\\
3.24e-09	0	\\
3.34e-09	0	\\
3.45e-09	0	\\
3.55e-09	0	\\
3.65e-09	0	\\
3.75e-09	0	\\
3.86e-09	0	\\
3.96e-09	0	\\
4.06e-09	0	\\
4.16e-09	0	\\
4.27e-09	0	\\
4.37e-09	0	\\
4.47e-09	0	\\
4.57e-09	0	\\
4.68e-09	0	\\
4.78e-09	0	\\
4.89e-09	0	\\
4.99e-09	0	\\
5e-09	0	\\
};
\addplot [color=darkgray,solid,forget plot]
  table[row sep=crcr]{
0	0	\\
1.1e-10	0	\\
2.2e-10	0	\\
3.3e-10	0	\\
4.4e-10	0	\\
5.4e-10	0	\\
6.5e-10	0	\\
7.5e-10	0	\\
8.6e-10	0	\\
9.6e-10	0	\\
1.07e-09	0	\\
1.18e-09	0	\\
1.28e-09	0	\\
1.38e-09	0	\\
1.49e-09	0	\\
1.59e-09	0	\\
1.69e-09	0	\\
1.8e-09	0	\\
1.9e-09	0	\\
2.01e-09	0	\\
2.11e-09	0	\\
2.21e-09	0	\\
2.32e-09	0	\\
2.42e-09	0	\\
2.52e-09	0	\\
2.63e-09	0	\\
2.73e-09	0	\\
2.83e-09	0	\\
2.93e-09	0	\\
3.04e-09	0	\\
3.14e-09	0	\\
3.24e-09	0	\\
3.34e-09	0	\\
3.45e-09	0	\\
3.55e-09	0	\\
3.65e-09	0	\\
3.75e-09	0	\\
3.86e-09	0	\\
3.96e-09	0	\\
4.06e-09	0	\\
4.16e-09	0	\\
4.27e-09	0	\\
4.37e-09	0	\\
4.47e-09	0	\\
4.57e-09	0	\\
4.68e-09	0	\\
4.78e-09	0	\\
4.89e-09	0	\\
4.99e-09	0	\\
5e-09	0	\\
};
\addplot [color=blue,solid,forget plot]
  table[row sep=crcr]{
0	0	\\
1.1e-10	0	\\
2.2e-10	0	\\
3.3e-10	0	\\
4.4e-10	0	\\
5.4e-10	0	\\
6.5e-10	0	\\
7.5e-10	0	\\
8.6e-10	0	\\
9.6e-10	0	\\
1.07e-09	0	\\
1.18e-09	0	\\
1.28e-09	0	\\
1.38e-09	0	\\
1.49e-09	0	\\
1.59e-09	0	\\
1.69e-09	0	\\
1.8e-09	0	\\
1.9e-09	0	\\
2.01e-09	0	\\
2.11e-09	0	\\
2.21e-09	0	\\
2.32e-09	0	\\
2.42e-09	0	\\
2.52e-09	0	\\
2.63e-09	0	\\
2.73e-09	0	\\
2.83e-09	0	\\
2.93e-09	0	\\
3.04e-09	0	\\
3.14e-09	0	\\
3.24e-09	0	\\
3.34e-09	0	\\
3.45e-09	0	\\
3.55e-09	0	\\
3.65e-09	0	\\
3.75e-09	0	\\
3.86e-09	0	\\
3.96e-09	0	\\
4.06e-09	0	\\
4.16e-09	0	\\
4.27e-09	0	\\
4.37e-09	0	\\
4.47e-09	0	\\
4.57e-09	0	\\
4.68e-09	0	\\
4.78e-09	0	\\
4.89e-09	0	\\
4.99e-09	0	\\
5e-09	0	\\
};
\addplot [color=black!50!green,solid,forget plot]
  table[row sep=crcr]{
0	0	\\
1.1e-10	0	\\
2.2e-10	0	\\
3.3e-10	0	\\
4.4e-10	0	\\
5.4e-10	0	\\
6.5e-10	0	\\
7.5e-10	0	\\
8.6e-10	0	\\
9.6e-10	0	\\
1.07e-09	0	\\
1.18e-09	0	\\
1.28e-09	0	\\
1.38e-09	0	\\
1.49e-09	0	\\
1.59e-09	0	\\
1.69e-09	0	\\
1.8e-09	0	\\
1.9e-09	0	\\
2.01e-09	0	\\
2.11e-09	0	\\
2.21e-09	0	\\
2.32e-09	0	\\
2.42e-09	0	\\
2.52e-09	0	\\
2.63e-09	0	\\
2.73e-09	0	\\
2.83e-09	0	\\
2.93e-09	0	\\
3.04e-09	0	\\
3.14e-09	0	\\
3.24e-09	0	\\
3.34e-09	0	\\
3.45e-09	0	\\
3.55e-09	0	\\
3.65e-09	0	\\
3.75e-09	0	\\
3.86e-09	0	\\
3.96e-09	0	\\
4.06e-09	0	\\
4.16e-09	0	\\
4.27e-09	0	\\
4.37e-09	0	\\
4.47e-09	0	\\
4.57e-09	0	\\
4.68e-09	0	\\
4.78e-09	0	\\
4.89e-09	0	\\
4.99e-09	0	\\
5e-09	0	\\
};
\addplot [color=red,solid,forget plot]
  table[row sep=crcr]{
0	0	\\
1.1e-10	0	\\
2.2e-10	0	\\
3.3e-10	0	\\
4.4e-10	0	\\
5.4e-10	0	\\
6.5e-10	0	\\
7.5e-10	0	\\
8.6e-10	0	\\
9.6e-10	0	\\
1.07e-09	0	\\
1.18e-09	0	\\
1.28e-09	0	\\
1.38e-09	0	\\
1.49e-09	0	\\
1.59e-09	0	\\
1.69e-09	0	\\
1.8e-09	0	\\
1.9e-09	0	\\
2.01e-09	0	\\
2.11e-09	0	\\
2.21e-09	0	\\
2.32e-09	0	\\
2.42e-09	0	\\
2.52e-09	0	\\
2.63e-09	0	\\
2.73e-09	0	\\
2.83e-09	0	\\
2.93e-09	0	\\
3.04e-09	0	\\
3.14e-09	0	\\
3.24e-09	0	\\
3.34e-09	0	\\
3.45e-09	0	\\
3.55e-09	0	\\
3.65e-09	0	\\
3.75e-09	0	\\
3.86e-09	0	\\
3.96e-09	0	\\
4.06e-09	0	\\
4.16e-09	0	\\
4.27e-09	0	\\
4.37e-09	0	\\
4.47e-09	0	\\
4.57e-09	0	\\
4.68e-09	0	\\
4.78e-09	0	\\
4.89e-09	0	\\
4.99e-09	0	\\
5e-09	0	\\
};
\addplot [color=mycolor1,solid,forget plot]
  table[row sep=crcr]{
0	0	\\
1.1e-10	0	\\
2.2e-10	0	\\
3.3e-10	0	\\
4.4e-10	0	\\
5.4e-10	0	\\
6.5e-10	0	\\
7.5e-10	0	\\
8.6e-10	0	\\
9.6e-10	0	\\
1.07e-09	0	\\
1.18e-09	0	\\
1.28e-09	0	\\
1.38e-09	0	\\
1.49e-09	0	\\
1.59e-09	0	\\
1.69e-09	0	\\
1.8e-09	0	\\
1.9e-09	0	\\
2.01e-09	0	\\
2.11e-09	0	\\
2.21e-09	0	\\
2.32e-09	0	\\
2.42e-09	0	\\
2.52e-09	0	\\
2.63e-09	0	\\
2.73e-09	0	\\
2.83e-09	0	\\
2.93e-09	0	\\
3.04e-09	0	\\
3.14e-09	0	\\
3.24e-09	0	\\
3.34e-09	0	\\
3.45e-09	0	\\
3.55e-09	0	\\
3.65e-09	0	\\
3.75e-09	0	\\
3.86e-09	0	\\
3.96e-09	0	\\
4.06e-09	0	\\
4.16e-09	0	\\
4.27e-09	0	\\
4.37e-09	0	\\
4.47e-09	0	\\
4.57e-09	0	\\
4.68e-09	0	\\
4.78e-09	0	\\
4.89e-09	0	\\
4.99e-09	0	\\
5e-09	0	\\
};
\addplot [color=mycolor2,solid,forget plot]
  table[row sep=crcr]{
0	0	\\
1.1e-10	0	\\
2.2e-10	0	\\
3.3e-10	0	\\
4.4e-10	0	\\
5.4e-10	0	\\
6.5e-10	0	\\
7.5e-10	0	\\
8.6e-10	0	\\
9.6e-10	0	\\
1.07e-09	0	\\
1.18e-09	0	\\
1.28e-09	0	\\
1.38e-09	0	\\
1.49e-09	0	\\
1.59e-09	0	\\
1.69e-09	0	\\
1.8e-09	0	\\
1.9e-09	0	\\
2.01e-09	0	\\
2.11e-09	0	\\
2.21e-09	0	\\
2.32e-09	0	\\
2.42e-09	0	\\
2.52e-09	0	\\
2.63e-09	0	\\
2.73e-09	0	\\
2.83e-09	0	\\
2.93e-09	0	\\
3.04e-09	0	\\
3.14e-09	0	\\
3.24e-09	0	\\
3.34e-09	0	\\
3.45e-09	0	\\
3.55e-09	0	\\
3.65e-09	0	\\
3.75e-09	0	\\
3.86e-09	0	\\
3.96e-09	0	\\
4.06e-09	0	\\
4.16e-09	0	\\
4.27e-09	0	\\
4.37e-09	0	\\
4.47e-09	0	\\
4.57e-09	0	\\
4.68e-09	0	\\
4.78e-09	0	\\
4.89e-09	0	\\
4.99e-09	0	\\
5e-09	0	\\
};
\addplot [color=mycolor3,solid,forget plot]
  table[row sep=crcr]{
0	0	\\
1.1e-10	0	\\
2.2e-10	0	\\
3.3e-10	0	\\
4.4e-10	0	\\
5.4e-10	0	\\
6.5e-10	0	\\
7.5e-10	0	\\
8.6e-10	0	\\
9.6e-10	0	\\
1.07e-09	0	\\
1.18e-09	0	\\
1.28e-09	0	\\
1.38e-09	0	\\
1.49e-09	0	\\
1.59e-09	0	\\
1.69e-09	0	\\
1.8e-09	0	\\
1.9e-09	0	\\
2.01e-09	0	\\
2.11e-09	0	\\
2.21e-09	0	\\
2.32e-09	0	\\
2.42e-09	0	\\
2.52e-09	0	\\
2.63e-09	0	\\
2.73e-09	0	\\
2.83e-09	0	\\
2.93e-09	0	\\
3.04e-09	0	\\
3.14e-09	0	\\
3.24e-09	0	\\
3.34e-09	0	\\
3.45e-09	0	\\
3.55e-09	0	\\
3.65e-09	0	\\
3.75e-09	0	\\
3.86e-09	0	\\
3.96e-09	0	\\
4.06e-09	0	\\
4.16e-09	0	\\
4.27e-09	0	\\
4.37e-09	0	\\
4.47e-09	0	\\
4.57e-09	0	\\
4.68e-09	0	\\
4.78e-09	0	\\
4.89e-09	0	\\
4.99e-09	0	\\
5e-09	0	\\
};
\addplot [color=darkgray,solid,forget plot]
  table[row sep=crcr]{
0	0	\\
1.1e-10	0	\\
2.2e-10	0	\\
3.3e-10	0	\\
4.4e-10	0	\\
5.4e-10	0	\\
6.5e-10	0	\\
7.5e-10	0	\\
8.6e-10	0	\\
9.6e-10	0	\\
1.07e-09	0	\\
1.18e-09	0	\\
1.28e-09	0	\\
1.38e-09	0	\\
1.49e-09	0	\\
1.59e-09	0	\\
1.69e-09	0	\\
1.8e-09	0	\\
1.9e-09	0	\\
2.01e-09	0	\\
2.11e-09	0	\\
2.21e-09	0	\\
2.32e-09	0	\\
2.42e-09	0	\\
2.52e-09	0	\\
2.63e-09	0	\\
2.73e-09	0	\\
2.83e-09	0	\\
2.93e-09	0	\\
3.04e-09	0	\\
3.14e-09	0	\\
3.24e-09	0	\\
3.34e-09	0	\\
3.45e-09	0	\\
3.55e-09	0	\\
3.65e-09	0	\\
3.75e-09	0	\\
3.86e-09	0	\\
3.96e-09	0	\\
4.06e-09	0	\\
4.16e-09	0	\\
4.27e-09	0	\\
4.37e-09	0	\\
4.47e-09	0	\\
4.57e-09	0	\\
4.68e-09	0	\\
4.78e-09	0	\\
4.89e-09	0	\\
4.99e-09	0	\\
5e-09	0	\\
};
\addplot [color=blue,solid,forget plot]
  table[row sep=crcr]{
0	0	\\
1.1e-10	0	\\
2.2e-10	0	\\
3.3e-10	0	\\
4.4e-10	0	\\
5.4e-10	0	\\
6.5e-10	0	\\
7.5e-10	0	\\
8.6e-10	0	\\
9.6e-10	0	\\
1.07e-09	0	\\
1.18e-09	0	\\
1.28e-09	0	\\
1.38e-09	0	\\
1.49e-09	0	\\
1.59e-09	0	\\
1.69e-09	0	\\
1.8e-09	0	\\
1.9e-09	0	\\
2.01e-09	0	\\
2.11e-09	0	\\
2.21e-09	0	\\
2.32e-09	0	\\
2.42e-09	0	\\
2.52e-09	0	\\
2.63e-09	0	\\
2.73e-09	0	\\
2.83e-09	0	\\
2.93e-09	0	\\
3.04e-09	0	\\
3.14e-09	0	\\
3.24e-09	0	\\
3.34e-09	0	\\
3.45e-09	0	\\
3.55e-09	0	\\
3.65e-09	0	\\
3.75e-09	0	\\
3.86e-09	0	\\
3.96e-09	0	\\
4.06e-09	0	\\
4.16e-09	0	\\
4.27e-09	0	\\
4.37e-09	0	\\
4.47e-09	0	\\
4.57e-09	0	\\
4.68e-09	0	\\
4.78e-09	0	\\
4.89e-09	0	\\
4.99e-09	0	\\
5e-09	0	\\
};
\addplot [color=black!50!green,solid,forget plot]
  table[row sep=crcr]{
0	0	\\
1.1e-10	0	\\
2.2e-10	0	\\
3.3e-10	0	\\
4.4e-10	0	\\
5.4e-10	0	\\
6.5e-10	0	\\
7.5e-10	0	\\
8.6e-10	0	\\
9.6e-10	0	\\
1.07e-09	0	\\
1.18e-09	0	\\
1.28e-09	0	\\
1.38e-09	0	\\
1.49e-09	0	\\
1.59e-09	0	\\
1.69e-09	0	\\
1.8e-09	0	\\
1.9e-09	0	\\
2.01e-09	0	\\
2.11e-09	0	\\
2.21e-09	0	\\
2.32e-09	0	\\
2.42e-09	0	\\
2.52e-09	0	\\
2.63e-09	0	\\
2.73e-09	0	\\
2.83e-09	0	\\
2.93e-09	0	\\
3.04e-09	0	\\
3.14e-09	0	\\
3.24e-09	0	\\
3.34e-09	0	\\
3.45e-09	0	\\
3.55e-09	0	\\
3.65e-09	0	\\
3.75e-09	0	\\
3.86e-09	0	\\
3.96e-09	0	\\
4.06e-09	0	\\
4.16e-09	0	\\
4.27e-09	0	\\
4.37e-09	0	\\
4.47e-09	0	\\
4.57e-09	0	\\
4.68e-09	0	\\
4.78e-09	0	\\
4.89e-09	0	\\
4.99e-09	0	\\
5e-09	0	\\
};
\addplot [color=red,solid,forget plot]
  table[row sep=crcr]{
0	0	\\
1.1e-10	0	\\
2.2e-10	0	\\
3.3e-10	0	\\
4.4e-10	0	\\
5.4e-10	0	\\
6.5e-10	0	\\
7.5e-10	0	\\
8.6e-10	0	\\
9.6e-10	0	\\
1.07e-09	0	\\
1.18e-09	0	\\
1.28e-09	0	\\
1.38e-09	0	\\
1.49e-09	0	\\
1.59e-09	0	\\
1.69e-09	0	\\
1.8e-09	0	\\
1.9e-09	0	\\
2.01e-09	0	\\
2.11e-09	0	\\
2.21e-09	0	\\
2.32e-09	0	\\
2.42e-09	0	\\
2.52e-09	0	\\
2.63e-09	0	\\
2.73e-09	0	\\
2.83e-09	0	\\
2.93e-09	0	\\
3.04e-09	0	\\
3.14e-09	0	\\
3.24e-09	0	\\
3.34e-09	0	\\
3.45e-09	0	\\
3.55e-09	0	\\
3.65e-09	0	\\
3.75e-09	0	\\
3.86e-09	0	\\
3.96e-09	0	\\
4.06e-09	0	\\
4.16e-09	0	\\
4.27e-09	0	\\
4.37e-09	0	\\
4.47e-09	0	\\
4.57e-09	0	\\
4.68e-09	0	\\
4.78e-09	0	\\
4.89e-09	0	\\
4.99e-09	0	\\
5e-09	0	\\
};
\addplot [color=mycolor1,solid,forget plot]
  table[row sep=crcr]{
0	0	\\
1.1e-10	0	\\
2.2e-10	0	\\
3.3e-10	0	\\
4.4e-10	0	\\
5.4e-10	0	\\
6.5e-10	0	\\
7.5e-10	0	\\
8.6e-10	0	\\
9.6e-10	0	\\
1.07e-09	0	\\
1.18e-09	0	\\
1.28e-09	0	\\
1.38e-09	0	\\
1.49e-09	0	\\
1.59e-09	0	\\
1.69e-09	0	\\
1.8e-09	0	\\
1.9e-09	0	\\
2.01e-09	0	\\
2.11e-09	0	\\
2.21e-09	0	\\
2.32e-09	0	\\
2.42e-09	0	\\
2.52e-09	0	\\
2.63e-09	0	\\
2.73e-09	0	\\
2.83e-09	0	\\
2.93e-09	0	\\
3.04e-09	0	\\
3.14e-09	0	\\
3.24e-09	0	\\
3.34e-09	0	\\
3.45e-09	0	\\
3.55e-09	0	\\
3.65e-09	0	\\
3.75e-09	0	\\
3.86e-09	0	\\
3.96e-09	0	\\
4.06e-09	0	\\
4.16e-09	0	\\
4.27e-09	0	\\
4.37e-09	0	\\
4.47e-09	0	\\
4.57e-09	0	\\
4.68e-09	0	\\
4.78e-09	0	\\
4.89e-09	0	\\
4.99e-09	0	\\
5e-09	0	\\
};
\addplot [color=mycolor2,solid,forget plot]
  table[row sep=crcr]{
0	0	\\
1.1e-10	0	\\
2.2e-10	0	\\
3.3e-10	0	\\
4.4e-10	0	\\
5.4e-10	0	\\
6.5e-10	0	\\
7.5e-10	0	\\
8.6e-10	0	\\
9.6e-10	0	\\
1.07e-09	0	\\
1.18e-09	0	\\
1.28e-09	0	\\
1.38e-09	0	\\
1.49e-09	0	\\
1.59e-09	0	\\
1.69e-09	0	\\
1.8e-09	0	\\
1.9e-09	0	\\
2.01e-09	0	\\
2.11e-09	0	\\
2.21e-09	0	\\
2.32e-09	0	\\
2.42e-09	0	\\
2.52e-09	0	\\
2.63e-09	0	\\
2.73e-09	0	\\
2.83e-09	0	\\
2.93e-09	0	\\
3.04e-09	0	\\
3.14e-09	0	\\
3.24e-09	0	\\
3.34e-09	0	\\
3.45e-09	0	\\
3.55e-09	0	\\
3.65e-09	0	\\
3.75e-09	0	\\
3.86e-09	0	\\
3.96e-09	0	\\
4.06e-09	0	\\
4.16e-09	0	\\
4.27e-09	0	\\
4.37e-09	0	\\
4.47e-09	0	\\
4.57e-09	0	\\
4.68e-09	0	\\
4.78e-09	0	\\
4.89e-09	0	\\
4.99e-09	0	\\
5e-09	0	\\
};
\addplot [color=mycolor3,solid,forget plot]
  table[row sep=crcr]{
0	0	\\
1.1e-10	0	\\
2.2e-10	0	\\
3.3e-10	0	\\
4.4e-10	0	\\
5.4e-10	0	\\
6.5e-10	0	\\
7.5e-10	0	\\
8.6e-10	0	\\
9.6e-10	0	\\
1.07e-09	0	\\
1.18e-09	0	\\
1.28e-09	0	\\
1.38e-09	0	\\
1.49e-09	0	\\
1.59e-09	0	\\
1.69e-09	0	\\
1.8e-09	0	\\
1.9e-09	0	\\
2.01e-09	0	\\
2.11e-09	0	\\
2.21e-09	0	\\
2.32e-09	0	\\
2.42e-09	0	\\
2.52e-09	0	\\
2.63e-09	0	\\
2.73e-09	0	\\
2.83e-09	0	\\
2.93e-09	0	\\
3.04e-09	0	\\
3.14e-09	0	\\
3.24e-09	0	\\
3.34e-09	0	\\
3.45e-09	0	\\
3.55e-09	0	\\
3.65e-09	0	\\
3.75e-09	0	\\
3.86e-09	0	\\
3.96e-09	0	\\
4.06e-09	0	\\
4.16e-09	0	\\
4.27e-09	0	\\
4.37e-09	0	\\
4.47e-09	0	\\
4.57e-09	0	\\
4.68e-09	0	\\
4.78e-09	0	\\
4.89e-09	0	\\
4.99e-09	0	\\
5e-09	0	\\
};
\addplot [color=darkgray,solid,forget plot]
  table[row sep=crcr]{
0	0	\\
1.1e-10	0	\\
2.2e-10	0	\\
3.3e-10	0	\\
4.4e-10	0	\\
5.4e-10	0	\\
6.5e-10	0	\\
7.5e-10	0	\\
8.6e-10	0	\\
9.6e-10	0	\\
1.07e-09	0	\\
1.18e-09	0	\\
1.28e-09	0	\\
1.38e-09	0	\\
1.49e-09	0	\\
1.59e-09	0	\\
1.69e-09	0	\\
1.8e-09	0	\\
1.9e-09	0	\\
2.01e-09	0	\\
2.11e-09	0	\\
2.21e-09	0	\\
2.32e-09	0	\\
2.42e-09	0	\\
2.52e-09	0	\\
2.63e-09	0	\\
2.73e-09	0	\\
2.83e-09	0	\\
2.93e-09	0	\\
3.04e-09	0	\\
3.14e-09	0	\\
3.24e-09	0	\\
3.34e-09	0	\\
3.45e-09	0	\\
3.55e-09	0	\\
3.65e-09	0	\\
3.75e-09	0	\\
3.86e-09	0	\\
3.96e-09	0	\\
4.06e-09	0	\\
4.16e-09	0	\\
4.27e-09	0	\\
4.37e-09	0	\\
4.47e-09	0	\\
4.57e-09	0	\\
4.68e-09	0	\\
4.78e-09	0	\\
4.89e-09	0	\\
4.99e-09	0	\\
5e-09	0	\\
};
\addplot [color=blue,solid,forget plot]
  table[row sep=crcr]{
0	0	\\
1.1e-10	0	\\
2.2e-10	0	\\
3.3e-10	0	\\
4.4e-10	0	\\
5.4e-10	0	\\
6.5e-10	0	\\
7.5e-10	0	\\
8.6e-10	0	\\
9.6e-10	0	\\
1.07e-09	0	\\
1.18e-09	0	\\
1.28e-09	0	\\
1.38e-09	0	\\
1.49e-09	0	\\
1.59e-09	0	\\
1.69e-09	0	\\
1.8e-09	0	\\
1.9e-09	0	\\
2.01e-09	0	\\
2.11e-09	0	\\
2.21e-09	0	\\
2.32e-09	0	\\
2.42e-09	0	\\
2.52e-09	0	\\
2.63e-09	0	\\
2.73e-09	0	\\
2.83e-09	0	\\
2.93e-09	0	\\
3.04e-09	0	\\
3.14e-09	0	\\
3.24e-09	0	\\
3.34e-09	0	\\
3.45e-09	0	\\
3.55e-09	0	\\
3.65e-09	0	\\
3.75e-09	0	\\
3.86e-09	0	\\
3.96e-09	0	\\
4.06e-09	0	\\
4.16e-09	0	\\
4.27e-09	0	\\
4.37e-09	0	\\
4.47e-09	0	\\
4.57e-09	0	\\
4.68e-09	0	\\
4.78e-09	0	\\
4.89e-09	0	\\
4.99e-09	0	\\
5e-09	0	\\
};
\addplot [color=black!50!green,solid,forget plot]
  table[row sep=crcr]{
0	0	\\
1.1e-10	0	\\
2.2e-10	0	\\
3.3e-10	0	\\
4.4e-10	0	\\
5.4e-10	0	\\
6.5e-10	0	\\
7.5e-10	0	\\
8.6e-10	0	\\
9.6e-10	0	\\
1.07e-09	0	\\
1.18e-09	0	\\
1.28e-09	0	\\
1.38e-09	0	\\
1.49e-09	0	\\
1.59e-09	0	\\
1.69e-09	0	\\
1.8e-09	0	\\
1.9e-09	0	\\
2.01e-09	0	\\
2.11e-09	0	\\
2.21e-09	0	\\
2.32e-09	0	\\
2.42e-09	0	\\
2.52e-09	0	\\
2.63e-09	0	\\
2.73e-09	0	\\
2.83e-09	0	\\
2.93e-09	0	\\
3.04e-09	0	\\
3.14e-09	0	\\
3.24e-09	0	\\
3.34e-09	0	\\
3.45e-09	0	\\
3.55e-09	0	\\
3.65e-09	0	\\
3.75e-09	0	\\
3.86e-09	0	\\
3.96e-09	0	\\
4.06e-09	0	\\
4.16e-09	0	\\
4.27e-09	0	\\
4.37e-09	0	\\
4.47e-09	0	\\
4.57e-09	0	\\
4.68e-09	0	\\
4.78e-09	0	\\
4.89e-09	0	\\
4.99e-09	0	\\
5e-09	0	\\
};
\addplot [color=red,solid,forget plot]
  table[row sep=crcr]{
0	0	\\
1.1e-10	0	\\
2.2e-10	0	\\
3.3e-10	0	\\
4.4e-10	0	\\
5.4e-10	0	\\
6.5e-10	0	\\
7.5e-10	0	\\
8.6e-10	0	\\
9.6e-10	0	\\
1.07e-09	0	\\
1.18e-09	0	\\
1.28e-09	0	\\
1.38e-09	0	\\
1.49e-09	0	\\
1.59e-09	0	\\
1.69e-09	0	\\
1.8e-09	0	\\
1.9e-09	0	\\
2.01e-09	0	\\
2.11e-09	0	\\
2.21e-09	0	\\
2.32e-09	0	\\
2.42e-09	0	\\
2.52e-09	0	\\
2.63e-09	0	\\
2.73e-09	0	\\
2.83e-09	0	\\
2.93e-09	0	\\
3.04e-09	0	\\
3.14e-09	0	\\
3.24e-09	0	\\
3.34e-09	0	\\
3.45e-09	0	\\
3.55e-09	0	\\
3.65e-09	0	\\
3.75e-09	0	\\
3.86e-09	0	\\
3.96e-09	0	\\
4.06e-09	0	\\
4.16e-09	0	\\
4.27e-09	0	\\
4.37e-09	0	\\
4.47e-09	0	\\
4.57e-09	0	\\
4.68e-09	0	\\
4.78e-09	0	\\
4.89e-09	0	\\
4.99e-09	0	\\
5e-09	0	\\
};
\addplot [color=mycolor1,solid,forget plot]
  table[row sep=crcr]{
0	0	\\
1.1e-10	0	\\
2.2e-10	0	\\
3.3e-10	0	\\
4.4e-10	0	\\
5.4e-10	0	\\
6.5e-10	0	\\
7.5e-10	0	\\
8.6e-10	0	\\
9.6e-10	0	\\
1.07e-09	0	\\
1.18e-09	0	\\
1.28e-09	0	\\
1.38e-09	0	\\
1.49e-09	0	\\
1.59e-09	0	\\
1.69e-09	0	\\
1.8e-09	0	\\
1.9e-09	0	\\
2.01e-09	0	\\
2.11e-09	0	\\
2.21e-09	0	\\
2.32e-09	0	\\
2.42e-09	0	\\
2.52e-09	0	\\
2.63e-09	0	\\
2.73e-09	0	\\
2.83e-09	0	\\
2.93e-09	0	\\
3.04e-09	0	\\
3.14e-09	0	\\
3.24e-09	0	\\
3.34e-09	0	\\
3.45e-09	0	\\
3.55e-09	0	\\
3.65e-09	0	\\
3.75e-09	0	\\
3.86e-09	0	\\
3.96e-09	0	\\
4.06e-09	0	\\
4.16e-09	0	\\
4.27e-09	0	\\
4.37e-09	0	\\
4.47e-09	0	\\
4.57e-09	0	\\
4.68e-09	0	\\
4.78e-09	0	\\
4.89e-09	0	\\
4.99e-09	0	\\
5e-09	0	\\
};
\addplot [color=mycolor2,solid,forget plot]
  table[row sep=crcr]{
0	0	\\
1.1e-10	0	\\
2.2e-10	0	\\
3.3e-10	0	\\
4.4e-10	0	\\
5.4e-10	0	\\
6.5e-10	0	\\
7.5e-10	0	\\
8.6e-10	0	\\
9.6e-10	0	\\
1.07e-09	0	\\
1.18e-09	0	\\
1.28e-09	0	\\
1.38e-09	0	\\
1.49e-09	0	\\
1.59e-09	0	\\
1.69e-09	0	\\
1.8e-09	0	\\
1.9e-09	0	\\
2.01e-09	0	\\
2.11e-09	0	\\
2.21e-09	0	\\
2.32e-09	0	\\
2.42e-09	0	\\
2.52e-09	0	\\
2.63e-09	0	\\
2.73e-09	0	\\
2.83e-09	0	\\
2.93e-09	0	\\
3.04e-09	0	\\
3.14e-09	0	\\
3.24e-09	0	\\
3.34e-09	0	\\
3.45e-09	0	\\
3.55e-09	0	\\
3.65e-09	0	\\
3.75e-09	0	\\
3.86e-09	0	\\
3.96e-09	0	\\
4.06e-09	0	\\
4.16e-09	0	\\
4.27e-09	0	\\
4.37e-09	0	\\
4.47e-09	0	\\
4.57e-09	0	\\
4.68e-09	0	\\
4.78e-09	0	\\
4.89e-09	0	\\
4.99e-09	0	\\
5e-09	0	\\
};
\addplot [color=mycolor3,solid,forget plot]
  table[row sep=crcr]{
0	0	\\
1.1e-10	0	\\
2.2e-10	0	\\
3.3e-10	0	\\
4.4e-10	0	\\
5.4e-10	0	\\
6.5e-10	0	\\
7.5e-10	0	\\
8.6e-10	0	\\
9.6e-10	0	\\
1.07e-09	0	\\
1.18e-09	0	\\
1.28e-09	0	\\
1.38e-09	0	\\
1.49e-09	0	\\
1.59e-09	0	\\
1.69e-09	0	\\
1.8e-09	0	\\
1.9e-09	0	\\
2.01e-09	0	\\
2.11e-09	0	\\
2.21e-09	0	\\
2.32e-09	0	\\
2.42e-09	0	\\
2.52e-09	0	\\
2.63e-09	0	\\
2.73e-09	0	\\
2.83e-09	0	\\
2.93e-09	0	\\
3.04e-09	0	\\
3.14e-09	0	\\
3.24e-09	0	\\
3.34e-09	0	\\
3.45e-09	0	\\
3.55e-09	0	\\
3.65e-09	0	\\
3.75e-09	0	\\
3.86e-09	0	\\
3.96e-09	0	\\
4.06e-09	0	\\
4.16e-09	0	\\
4.27e-09	0	\\
4.37e-09	0	\\
4.47e-09	0	\\
4.57e-09	0	\\
4.68e-09	0	\\
4.78e-09	0	\\
4.89e-09	0	\\
4.99e-09	0	\\
5e-09	0	\\
};
\addplot [color=darkgray,solid,forget plot]
  table[row sep=crcr]{
0	0	\\
1.1e-10	0	\\
2.2e-10	0	\\
3.3e-10	0	\\
4.4e-10	0	\\
5.4e-10	0	\\
6.5e-10	0	\\
7.5e-10	0	\\
8.6e-10	0	\\
9.6e-10	0	\\
1.07e-09	0	\\
1.18e-09	0	\\
1.28e-09	0	\\
1.38e-09	0	\\
1.49e-09	0	\\
1.59e-09	0	\\
1.69e-09	0	\\
1.8e-09	0	\\
1.9e-09	0	\\
2.01e-09	0	\\
2.11e-09	0	\\
2.21e-09	0	\\
2.32e-09	0	\\
2.42e-09	0	\\
2.52e-09	0	\\
2.63e-09	0	\\
2.73e-09	0	\\
2.83e-09	0	\\
2.93e-09	0	\\
3.04e-09	0	\\
3.14e-09	0	\\
3.24e-09	0	\\
3.34e-09	0	\\
3.45e-09	0	\\
3.55e-09	0	\\
3.65e-09	0	\\
3.75e-09	0	\\
3.86e-09	0	\\
3.96e-09	0	\\
4.06e-09	0	\\
4.16e-09	0	\\
4.27e-09	0	\\
4.37e-09	0	\\
4.47e-09	0	\\
4.57e-09	0	\\
4.68e-09	0	\\
4.78e-09	0	\\
4.89e-09	0	\\
4.99e-09	0	\\
5e-09	0	\\
};
\addplot [color=blue,solid,forget plot]
  table[row sep=crcr]{
0	0	\\
1.1e-10	0	\\
2.2e-10	0	\\
3.3e-10	0	\\
4.4e-10	0	\\
5.4e-10	0	\\
6.5e-10	0	\\
7.5e-10	0	\\
8.6e-10	0	\\
9.6e-10	0	\\
1.07e-09	0	\\
1.18e-09	0	\\
1.28e-09	0	\\
1.38e-09	0	\\
1.49e-09	0	\\
1.59e-09	0	\\
1.69e-09	0	\\
1.8e-09	0	\\
1.9e-09	0	\\
2.01e-09	0	\\
2.11e-09	0	\\
2.21e-09	0	\\
2.32e-09	0	\\
2.42e-09	0	\\
2.52e-09	0	\\
2.63e-09	0	\\
2.73e-09	0	\\
2.83e-09	0	\\
2.93e-09	0	\\
3.04e-09	0	\\
3.14e-09	0	\\
3.24e-09	0	\\
3.34e-09	0	\\
3.45e-09	0	\\
3.55e-09	0	\\
3.65e-09	0	\\
3.75e-09	0	\\
3.86e-09	0	\\
3.96e-09	0	\\
4.06e-09	0	\\
4.16e-09	0	\\
4.27e-09	0	\\
4.37e-09	0	\\
4.47e-09	0	\\
4.57e-09	0	\\
4.68e-09	0	\\
4.78e-09	0	\\
4.89e-09	0	\\
4.99e-09	0	\\
5e-09	0	\\
};
\addplot [color=black!50!green,solid,forget plot]
  table[row sep=crcr]{
0	0	\\
1.1e-10	0	\\
2.2e-10	0	\\
3.3e-10	0	\\
4.4e-10	0	\\
5.4e-10	0	\\
6.5e-10	0	\\
7.5e-10	0	\\
8.6e-10	0	\\
9.6e-10	0	\\
1.07e-09	0	\\
1.18e-09	0	\\
1.28e-09	0	\\
1.38e-09	0	\\
1.49e-09	0	\\
1.59e-09	0	\\
1.69e-09	0	\\
1.8e-09	0	\\
1.9e-09	0	\\
2.01e-09	0	\\
2.11e-09	0	\\
2.21e-09	0	\\
2.32e-09	0	\\
2.42e-09	0	\\
2.52e-09	0	\\
2.63e-09	0	\\
2.73e-09	0	\\
2.83e-09	0	\\
2.93e-09	0	\\
3.04e-09	0	\\
3.14e-09	0	\\
3.24e-09	0	\\
3.34e-09	0	\\
3.45e-09	0	\\
3.55e-09	0	\\
3.65e-09	0	\\
3.75e-09	0	\\
3.86e-09	0	\\
3.96e-09	0	\\
4.06e-09	0	\\
4.16e-09	0	\\
4.27e-09	0	\\
4.37e-09	0	\\
4.47e-09	0	\\
4.57e-09	0	\\
4.68e-09	0	\\
4.78e-09	0	\\
4.89e-09	0	\\
4.99e-09	0	\\
5e-09	0	\\
};
\addplot [color=red,solid,forget plot]
  table[row sep=crcr]{
0	0	\\
1.1e-10	0	\\
2.2e-10	0	\\
3.3e-10	0	\\
4.4e-10	0	\\
5.4e-10	0	\\
6.5e-10	0	\\
7.5e-10	0	\\
8.6e-10	0	\\
9.6e-10	0	\\
1.07e-09	0	\\
1.18e-09	0	\\
1.28e-09	0	\\
1.38e-09	0	\\
1.49e-09	0	\\
1.59e-09	0	\\
1.69e-09	0	\\
1.8e-09	0	\\
1.9e-09	0	\\
2.01e-09	0	\\
2.11e-09	0	\\
2.21e-09	0	\\
2.32e-09	0	\\
2.42e-09	0	\\
2.52e-09	0	\\
2.63e-09	0	\\
2.73e-09	0	\\
2.83e-09	0	\\
2.93e-09	0	\\
3.04e-09	0	\\
3.14e-09	0	\\
3.24e-09	0	\\
3.34e-09	0	\\
3.45e-09	0	\\
3.55e-09	0	\\
3.65e-09	0	\\
3.75e-09	0	\\
3.86e-09	0	\\
3.96e-09	0	\\
4.06e-09	0	\\
4.16e-09	0	\\
4.27e-09	0	\\
4.37e-09	0	\\
4.47e-09	0	\\
4.57e-09	0	\\
4.68e-09	0	\\
4.78e-09	0	\\
4.89e-09	0	\\
4.99e-09	0	\\
5e-09	0	\\
};
\addplot [color=mycolor1,solid,forget plot]
  table[row sep=crcr]{
0	0	\\
1.1e-10	0	\\
2.2e-10	0	\\
3.3e-10	0	\\
4.4e-10	0	\\
5.4e-10	0	\\
6.5e-10	0	\\
7.5e-10	0	\\
8.6e-10	0	\\
9.6e-10	0	\\
1.07e-09	0	\\
1.18e-09	0	\\
1.28e-09	0	\\
1.38e-09	0	\\
1.49e-09	0	\\
1.59e-09	0	\\
1.69e-09	0	\\
1.8e-09	0	\\
1.9e-09	0	\\
2.01e-09	0	\\
2.11e-09	0	\\
2.21e-09	0	\\
2.32e-09	0	\\
2.42e-09	0	\\
2.52e-09	0	\\
2.63e-09	0	\\
2.73e-09	0	\\
2.83e-09	0	\\
2.93e-09	0	\\
3.04e-09	0	\\
3.14e-09	0	\\
3.24e-09	0	\\
3.34e-09	0	\\
3.45e-09	0	\\
3.55e-09	0	\\
3.65e-09	0	\\
3.75e-09	0	\\
3.86e-09	0	\\
3.96e-09	0	\\
4.06e-09	0	\\
4.16e-09	0	\\
4.27e-09	0	\\
4.37e-09	0	\\
4.47e-09	0	\\
4.57e-09	0	\\
4.68e-09	0	\\
4.78e-09	0	\\
4.89e-09	0	\\
4.99e-09	0	\\
5e-09	0	\\
};
\addplot [color=mycolor2,solid,forget plot]
  table[row sep=crcr]{
0	0	\\
1.1e-10	0	\\
2.2e-10	0	\\
3.3e-10	0	\\
4.4e-10	0	\\
5.4e-10	0	\\
6.5e-10	0	\\
7.5e-10	0	\\
8.6e-10	0	\\
9.6e-10	0	\\
1.07e-09	0	\\
1.18e-09	0	\\
1.28e-09	0	\\
1.38e-09	0	\\
1.49e-09	0	\\
1.59e-09	0	\\
1.69e-09	0	\\
1.8e-09	0	\\
1.9e-09	0	\\
2.01e-09	0	\\
2.11e-09	0	\\
2.21e-09	0	\\
2.32e-09	0	\\
2.42e-09	0	\\
2.52e-09	0	\\
2.63e-09	0	\\
2.73e-09	0	\\
2.83e-09	0	\\
2.93e-09	0	\\
3.04e-09	0	\\
3.14e-09	0	\\
3.24e-09	0	\\
3.34e-09	0	\\
3.45e-09	0	\\
3.55e-09	0	\\
3.65e-09	0	\\
3.75e-09	0	\\
3.86e-09	0	\\
3.96e-09	0	\\
4.06e-09	0	\\
4.16e-09	0	\\
4.27e-09	0	\\
4.37e-09	0	\\
4.47e-09	0	\\
4.57e-09	0	\\
4.68e-09	0	\\
4.78e-09	0	\\
4.89e-09	0	\\
4.99e-09	0	\\
5e-09	0	\\
};
\addplot [color=mycolor3,solid,forget plot]
  table[row sep=crcr]{
0	0	\\
1.1e-10	0	\\
2.2e-10	0	\\
3.3e-10	0	\\
4.4e-10	0	\\
5.4e-10	0	\\
6.5e-10	0	\\
7.5e-10	0	\\
8.6e-10	0	\\
9.6e-10	0	\\
1.07e-09	0	\\
1.18e-09	0	\\
1.28e-09	0	\\
1.38e-09	0	\\
1.49e-09	0	\\
1.59e-09	0	\\
1.69e-09	0	\\
1.8e-09	0	\\
1.9e-09	0	\\
2.01e-09	0	\\
2.11e-09	0	\\
2.21e-09	0	\\
2.32e-09	0	\\
2.42e-09	0	\\
2.52e-09	0	\\
2.63e-09	0	\\
2.73e-09	0	\\
2.83e-09	0	\\
2.93e-09	0	\\
3.04e-09	0	\\
3.14e-09	0	\\
3.24e-09	0	\\
3.34e-09	0	\\
3.45e-09	0	\\
3.55e-09	0	\\
3.65e-09	0	\\
3.75e-09	0	\\
3.86e-09	0	\\
3.96e-09	0	\\
4.06e-09	0	\\
4.16e-09	0	\\
4.27e-09	0	\\
4.37e-09	0	\\
4.47e-09	0	\\
4.57e-09	0	\\
4.68e-09	0	\\
4.78e-09	0	\\
4.89e-09	0	\\
4.99e-09	0	\\
5e-09	0	\\
};
\addplot [color=darkgray,solid,forget plot]
  table[row sep=crcr]{
0	0	\\
1.1e-10	0	\\
2.2e-10	0	\\
3.3e-10	0	\\
4.4e-10	0	\\
5.4e-10	0	\\
6.5e-10	0	\\
7.5e-10	0	\\
8.6e-10	0	\\
9.6e-10	0	\\
1.07e-09	0	\\
1.18e-09	0	\\
1.28e-09	0	\\
1.38e-09	0	\\
1.49e-09	0	\\
1.59e-09	0	\\
1.69e-09	0	\\
1.8e-09	0	\\
1.9e-09	0	\\
2.01e-09	0	\\
2.11e-09	0	\\
2.21e-09	0	\\
2.32e-09	0	\\
2.42e-09	0	\\
2.52e-09	0	\\
2.63e-09	0	\\
2.73e-09	0	\\
2.83e-09	0	\\
2.93e-09	0	\\
3.04e-09	0	\\
3.14e-09	0	\\
3.24e-09	0	\\
3.34e-09	0	\\
3.45e-09	0	\\
3.55e-09	0	\\
3.65e-09	0	\\
3.75e-09	0	\\
3.86e-09	0	\\
3.96e-09	0	\\
4.06e-09	0	\\
4.16e-09	0	\\
4.27e-09	0	\\
4.37e-09	0	\\
4.47e-09	0	\\
4.57e-09	0	\\
4.68e-09	0	\\
4.78e-09	0	\\
4.89e-09	0	\\
4.99e-09	0	\\
5e-09	0	\\
};
\addplot [color=blue,solid,forget plot]
  table[row sep=crcr]{
0	0	\\
1.1e-10	0	\\
2.2e-10	0	\\
3.3e-10	0	\\
4.4e-10	0	\\
5.4e-10	0	\\
6.5e-10	0	\\
7.5e-10	0	\\
8.6e-10	0	\\
9.6e-10	0	\\
1.07e-09	0	\\
1.18e-09	0	\\
1.28e-09	0	\\
1.38e-09	0	\\
1.49e-09	0	\\
1.59e-09	0	\\
1.69e-09	0	\\
1.8e-09	0	\\
1.9e-09	0	\\
2.01e-09	0	\\
2.11e-09	0	\\
2.21e-09	0	\\
2.32e-09	0	\\
2.42e-09	0	\\
2.52e-09	0	\\
2.63e-09	0	\\
2.73e-09	0	\\
2.83e-09	0	\\
2.93e-09	0	\\
3.04e-09	0	\\
3.14e-09	0	\\
3.24e-09	0	\\
3.34e-09	0	\\
3.45e-09	0	\\
3.55e-09	0	\\
3.65e-09	0	\\
3.75e-09	0	\\
3.86e-09	0	\\
3.96e-09	0	\\
4.06e-09	0	\\
4.16e-09	0	\\
4.27e-09	0	\\
4.37e-09	0	\\
4.47e-09	0	\\
4.57e-09	0	\\
4.68e-09	0	\\
4.78e-09	0	\\
4.89e-09	0	\\
4.99e-09	0	\\
5e-09	0	\\
};
\addplot [color=black!50!green,solid,forget plot]
  table[row sep=crcr]{
0	0	\\
1.1e-10	0	\\
2.2e-10	0	\\
3.3e-10	0	\\
4.4e-10	0	\\
5.4e-10	0	\\
6.5e-10	0	\\
7.5e-10	0	\\
8.6e-10	0	\\
9.6e-10	0	\\
1.07e-09	0	\\
1.18e-09	0	\\
1.28e-09	0	\\
1.38e-09	0	\\
1.49e-09	0	\\
1.59e-09	0	\\
1.69e-09	0	\\
1.8e-09	0	\\
1.9e-09	0	\\
2.01e-09	0	\\
2.11e-09	0	\\
2.21e-09	0	\\
2.32e-09	0	\\
2.42e-09	0	\\
2.52e-09	0	\\
2.63e-09	0	\\
2.73e-09	0	\\
2.83e-09	0	\\
2.93e-09	0	\\
3.04e-09	0	\\
3.14e-09	0	\\
3.24e-09	0	\\
3.34e-09	0	\\
3.45e-09	0	\\
3.55e-09	0	\\
3.65e-09	0	\\
3.75e-09	0	\\
3.86e-09	0	\\
3.96e-09	0	\\
4.06e-09	0	\\
4.16e-09	0	\\
4.27e-09	0	\\
4.37e-09	0	\\
4.47e-09	0	\\
4.57e-09	0	\\
4.68e-09	0	\\
4.78e-09	0	\\
4.89e-09	0	\\
4.99e-09	0	\\
5e-09	0	\\
};
\addplot [color=red,solid,forget plot]
  table[row sep=crcr]{
0	0	\\
1.1e-10	0	\\
2.2e-10	0	\\
3.3e-10	0	\\
4.4e-10	0	\\
5.4e-10	0	\\
6.5e-10	0	\\
7.5e-10	0	\\
8.6e-10	0	\\
9.6e-10	0	\\
1.07e-09	0	\\
1.18e-09	0	\\
1.28e-09	0	\\
1.38e-09	0	\\
1.49e-09	0	\\
1.59e-09	0	\\
1.69e-09	0	\\
1.8e-09	0	\\
1.9e-09	0	\\
2.01e-09	0	\\
2.11e-09	0	\\
2.21e-09	0	\\
2.32e-09	0	\\
2.42e-09	0	\\
2.52e-09	0	\\
2.63e-09	0	\\
2.73e-09	0	\\
2.83e-09	0	\\
2.93e-09	0	\\
3.04e-09	0	\\
3.14e-09	0	\\
3.24e-09	0	\\
3.34e-09	0	\\
3.45e-09	0	\\
3.55e-09	0	\\
3.65e-09	0	\\
3.75e-09	0	\\
3.86e-09	0	\\
3.96e-09	0	\\
4.06e-09	0	\\
4.16e-09	0	\\
4.27e-09	0	\\
4.37e-09	0	\\
4.47e-09	0	\\
4.57e-09	0	\\
4.68e-09	0	\\
4.78e-09	0	\\
4.89e-09	0	\\
4.99e-09	0	\\
5e-09	0	\\
};
\addplot [color=mycolor1,solid,forget plot]
  table[row sep=crcr]{
0	0	\\
1.1e-10	0	\\
2.2e-10	0	\\
3.3e-10	0	\\
4.4e-10	0	\\
5.4e-10	0	\\
6.5e-10	0	\\
7.5e-10	0	\\
8.6e-10	0	\\
9.6e-10	0	\\
1.07e-09	0	\\
1.18e-09	0	\\
1.28e-09	0	\\
1.38e-09	0	\\
1.49e-09	0	\\
1.59e-09	0	\\
1.69e-09	0	\\
1.8e-09	0	\\
1.9e-09	0	\\
2.01e-09	0	\\
2.11e-09	0	\\
2.21e-09	0	\\
2.32e-09	0	\\
2.42e-09	0	\\
2.52e-09	0	\\
2.63e-09	0	\\
2.73e-09	0	\\
2.83e-09	0	\\
2.93e-09	0	\\
3.04e-09	0	\\
3.14e-09	0	\\
3.24e-09	0	\\
3.34e-09	0	\\
3.45e-09	0	\\
3.55e-09	0	\\
3.65e-09	0	\\
3.75e-09	0	\\
3.86e-09	0	\\
3.96e-09	0	\\
4.06e-09	0	\\
4.16e-09	0	\\
4.27e-09	0	\\
4.37e-09	0	\\
4.47e-09	0	\\
4.57e-09	0	\\
4.68e-09	0	\\
4.78e-09	0	\\
4.89e-09	0	\\
4.99e-09	0	\\
5e-09	0	\\
};
\addplot [color=mycolor2,solid,forget plot]
  table[row sep=crcr]{
0	0	\\
1.1e-10	0	\\
2.2e-10	0	\\
3.3e-10	0	\\
4.4e-10	0	\\
5.4e-10	0	\\
6.5e-10	0	\\
7.5e-10	0	\\
8.6e-10	0	\\
9.6e-10	0	\\
1.07e-09	0	\\
1.18e-09	0	\\
1.28e-09	0	\\
1.38e-09	0	\\
1.49e-09	0	\\
1.59e-09	0	\\
1.69e-09	0	\\
1.8e-09	0	\\
1.9e-09	0	\\
2.01e-09	0	\\
2.11e-09	0	\\
2.21e-09	0	\\
2.32e-09	0	\\
2.42e-09	0	\\
2.52e-09	0	\\
2.63e-09	0	\\
2.73e-09	0	\\
2.83e-09	0	\\
2.93e-09	0	\\
3.04e-09	0	\\
3.14e-09	0	\\
3.24e-09	0	\\
3.34e-09	0	\\
3.45e-09	0	\\
3.55e-09	0	\\
3.65e-09	0	\\
3.75e-09	0	\\
3.86e-09	0	\\
3.96e-09	0	\\
4.06e-09	0	\\
4.16e-09	0	\\
4.27e-09	0	\\
4.37e-09	0	\\
4.47e-09	0	\\
4.57e-09	0	\\
4.68e-09	0	\\
4.78e-09	0	\\
4.89e-09	0	\\
4.99e-09	0	\\
5e-09	0	\\
};
\addplot [color=mycolor3,solid,forget plot]
  table[row sep=crcr]{
0	0	\\
1.1e-10	0	\\
2.2e-10	0	\\
3.3e-10	0	\\
4.4e-10	0	\\
5.4e-10	0	\\
6.5e-10	0	\\
7.5e-10	0	\\
8.6e-10	0	\\
9.6e-10	0	\\
1.07e-09	0	\\
1.18e-09	0	\\
1.28e-09	0	\\
1.38e-09	0	\\
1.49e-09	0	\\
1.59e-09	0	\\
1.69e-09	0	\\
1.8e-09	0	\\
1.9e-09	0	\\
2.01e-09	0	\\
2.11e-09	0	\\
2.21e-09	0	\\
2.32e-09	0	\\
2.42e-09	0	\\
2.52e-09	0	\\
2.63e-09	0	\\
2.73e-09	0	\\
2.83e-09	0	\\
2.93e-09	0	\\
3.04e-09	0	\\
3.14e-09	0	\\
3.24e-09	0	\\
3.34e-09	0	\\
3.45e-09	0	\\
3.55e-09	0	\\
3.65e-09	0	\\
3.75e-09	0	\\
3.86e-09	0	\\
3.96e-09	0	\\
4.06e-09	0	\\
4.16e-09	0	\\
4.27e-09	0	\\
4.37e-09	0	\\
4.47e-09	0	\\
4.57e-09	0	\\
4.68e-09	0	\\
4.78e-09	0	\\
4.89e-09	0	\\
4.99e-09	0	\\
5e-09	0	\\
};
\addplot [color=darkgray,solid,forget plot]
  table[row sep=crcr]{
0	0	\\
1.1e-10	0	\\
2.2e-10	0	\\
3.3e-10	0	\\
4.4e-10	0	\\
5.4e-10	0	\\
6.5e-10	0	\\
7.5e-10	0	\\
8.6e-10	0	\\
9.6e-10	0	\\
1.07e-09	0	\\
1.18e-09	0	\\
1.28e-09	0	\\
1.38e-09	0	\\
1.49e-09	0	\\
1.59e-09	0	\\
1.69e-09	0	\\
1.8e-09	0	\\
1.9e-09	0	\\
2.01e-09	0	\\
2.11e-09	0	\\
2.21e-09	0	\\
2.32e-09	0	\\
2.42e-09	0	\\
2.52e-09	0	\\
2.63e-09	0	\\
2.73e-09	0	\\
2.83e-09	0	\\
2.93e-09	0	\\
3.04e-09	0	\\
3.14e-09	0	\\
3.24e-09	0	\\
3.34e-09	0	\\
3.45e-09	0	\\
3.55e-09	0	\\
3.65e-09	0	\\
3.75e-09	0	\\
3.86e-09	0	\\
3.96e-09	0	\\
4.06e-09	0	\\
4.16e-09	0	\\
4.27e-09	0	\\
4.37e-09	0	\\
4.47e-09	0	\\
4.57e-09	0	\\
4.68e-09	0	\\
4.78e-09	0	\\
4.89e-09	0	\\
4.99e-09	0	\\
5e-09	0	\\
};
\addplot [color=blue,solid,forget plot]
  table[row sep=crcr]{
0	0	\\
1.1e-10	0	\\
2.2e-10	0	\\
3.3e-10	0	\\
4.4e-10	0	\\
5.4e-10	0	\\
6.5e-10	0	\\
7.5e-10	0	\\
8.6e-10	0	\\
9.6e-10	0	\\
1.07e-09	0	\\
1.18e-09	0	\\
1.28e-09	0	\\
1.38e-09	0	\\
1.49e-09	0	\\
1.59e-09	0	\\
1.69e-09	0	\\
1.8e-09	0	\\
1.9e-09	0	\\
2.01e-09	0	\\
2.11e-09	0	\\
2.21e-09	0	\\
2.32e-09	0	\\
2.42e-09	0	\\
2.52e-09	0	\\
2.63e-09	0	\\
2.73e-09	0	\\
2.83e-09	0	\\
2.93e-09	0	\\
3.04e-09	0	\\
3.14e-09	0	\\
3.24e-09	0	\\
3.34e-09	0	\\
3.45e-09	0	\\
3.55e-09	0	\\
3.65e-09	0	\\
3.75e-09	0	\\
3.86e-09	0	\\
3.96e-09	0	\\
4.06e-09	0	\\
4.16e-09	0	\\
4.27e-09	0	\\
4.37e-09	0	\\
4.47e-09	0	\\
4.57e-09	0	\\
4.68e-09	0	\\
4.78e-09	0	\\
4.89e-09	0	\\
4.99e-09	0	\\
5e-09	0	\\
};
\addplot [color=black!50!green,solid,forget plot]
  table[row sep=crcr]{
0	0	\\
1.1e-10	0	\\
2.2e-10	0	\\
3.3e-10	0	\\
4.4e-10	0	\\
5.4e-10	0	\\
6.5e-10	0	\\
7.5e-10	0	\\
8.6e-10	0	\\
9.6e-10	0	\\
1.07e-09	0	\\
1.18e-09	0	\\
1.28e-09	0	\\
1.38e-09	0	\\
1.49e-09	0	\\
1.59e-09	0	\\
1.69e-09	0	\\
1.8e-09	0	\\
1.9e-09	0	\\
2.01e-09	0	\\
2.11e-09	0	\\
2.21e-09	0	\\
2.32e-09	0	\\
2.42e-09	0	\\
2.52e-09	0	\\
2.63e-09	0	\\
2.73e-09	0	\\
2.83e-09	0	\\
2.93e-09	0	\\
3.04e-09	0	\\
3.14e-09	0	\\
3.24e-09	0	\\
3.34e-09	0	\\
3.45e-09	0	\\
3.55e-09	0	\\
3.65e-09	0	\\
3.75e-09	0	\\
3.86e-09	0	\\
3.96e-09	0	\\
4.06e-09	0	\\
4.16e-09	0	\\
4.27e-09	0	\\
4.37e-09	0	\\
4.47e-09	0	\\
4.57e-09	0	\\
4.68e-09	0	\\
4.78e-09	0	\\
4.89e-09	0	\\
4.99e-09	0	\\
5e-09	0	\\
};
\addplot [color=red,solid,forget plot]
  table[row sep=crcr]{
0	0	\\
1.1e-10	0	\\
2.2e-10	0	\\
3.3e-10	0	\\
4.4e-10	0	\\
5.4e-10	0	\\
6.5e-10	0	\\
7.5e-10	0	\\
8.6e-10	0	\\
9.6e-10	0	\\
1.07e-09	0	\\
1.18e-09	0	\\
1.28e-09	0	\\
1.38e-09	0	\\
1.49e-09	0	\\
1.59e-09	0	\\
1.69e-09	0	\\
1.8e-09	0	\\
1.9e-09	0	\\
2.01e-09	0	\\
2.11e-09	0	\\
2.21e-09	0	\\
2.32e-09	0	\\
2.42e-09	0	\\
2.52e-09	0	\\
2.63e-09	0	\\
2.73e-09	0	\\
2.83e-09	0	\\
2.93e-09	0	\\
3.04e-09	0	\\
3.14e-09	0	\\
3.24e-09	0	\\
3.34e-09	0	\\
3.45e-09	0	\\
3.55e-09	0	\\
3.65e-09	0	\\
3.75e-09	0	\\
3.86e-09	0	\\
3.96e-09	0	\\
4.06e-09	0	\\
4.16e-09	0	\\
4.27e-09	0	\\
4.37e-09	0	\\
4.47e-09	0	\\
4.57e-09	0	\\
4.68e-09	0	\\
4.78e-09	0	\\
4.89e-09	0	\\
4.99e-09	0	\\
5e-09	0	\\
};
\addplot [color=mycolor1,solid,forget plot]
  table[row sep=crcr]{
0	0	\\
1.1e-10	0	\\
2.2e-10	0	\\
3.3e-10	0	\\
4.4e-10	0	\\
5.4e-10	0	\\
6.5e-10	0	\\
7.5e-10	0	\\
8.6e-10	0	\\
9.6e-10	0	\\
1.07e-09	0	\\
1.18e-09	0	\\
1.28e-09	0	\\
1.38e-09	0	\\
1.49e-09	0	\\
1.59e-09	0	\\
1.69e-09	0	\\
1.8e-09	0	\\
1.9e-09	0	\\
2.01e-09	0	\\
2.11e-09	0	\\
2.21e-09	0	\\
2.32e-09	0	\\
2.42e-09	0	\\
2.52e-09	0	\\
2.63e-09	0	\\
2.73e-09	0	\\
2.83e-09	0	\\
2.93e-09	0	\\
3.04e-09	0	\\
3.14e-09	0	\\
3.24e-09	0	\\
3.34e-09	0	\\
3.45e-09	0	\\
3.55e-09	0	\\
3.65e-09	0	\\
3.75e-09	0	\\
3.86e-09	0	\\
3.96e-09	0	\\
4.06e-09	0	\\
4.16e-09	0	\\
4.27e-09	0	\\
4.37e-09	0	\\
4.47e-09	0	\\
4.57e-09	0	\\
4.68e-09	0	\\
4.78e-09	0	\\
4.89e-09	0	\\
4.99e-09	0	\\
5e-09	0	\\
};
\addplot [color=mycolor2,solid,forget plot]
  table[row sep=crcr]{
0	0	\\
1.1e-10	0	\\
2.2e-10	0	\\
3.3e-10	0	\\
4.4e-10	0	\\
5.4e-10	0	\\
6.5e-10	0	\\
7.5e-10	0	\\
8.6e-10	0	\\
9.6e-10	0	\\
1.07e-09	0	\\
1.18e-09	0	\\
1.28e-09	0	\\
1.38e-09	0	\\
1.49e-09	0	\\
1.59e-09	0	\\
1.69e-09	0	\\
1.8e-09	0	\\
1.9e-09	0	\\
2.01e-09	0	\\
2.11e-09	0	\\
2.21e-09	0	\\
2.32e-09	0	\\
2.42e-09	0	\\
2.52e-09	0	\\
2.63e-09	0	\\
2.73e-09	0	\\
2.83e-09	0	\\
2.93e-09	0	\\
3.04e-09	0	\\
3.14e-09	0	\\
3.24e-09	0	\\
3.34e-09	0	\\
3.45e-09	0	\\
3.55e-09	0	\\
3.65e-09	0	\\
3.75e-09	0	\\
3.86e-09	0	\\
3.96e-09	0	\\
4.06e-09	0	\\
4.16e-09	0	\\
4.27e-09	0	\\
4.37e-09	0	\\
4.47e-09	0	\\
4.57e-09	0	\\
4.68e-09	0	\\
4.78e-09	0	\\
4.89e-09	0	\\
4.99e-09	0	\\
5e-09	0	\\
};
\addplot [color=mycolor3,solid,forget plot]
  table[row sep=crcr]{
0	0	\\
1.1e-10	0	\\
2.2e-10	0	\\
3.3e-10	0	\\
4.4e-10	0	\\
5.4e-10	0	\\
6.5e-10	0	\\
7.5e-10	0	\\
8.6e-10	0	\\
9.6e-10	0	\\
1.07e-09	0	\\
1.18e-09	0	\\
1.28e-09	0	\\
1.38e-09	0	\\
1.49e-09	0	\\
1.59e-09	0	\\
1.69e-09	0	\\
1.8e-09	0	\\
1.9e-09	0	\\
2.01e-09	0	\\
2.11e-09	0	\\
2.21e-09	0	\\
2.32e-09	0	\\
2.42e-09	0	\\
2.52e-09	0	\\
2.63e-09	0	\\
2.73e-09	0	\\
2.83e-09	0	\\
2.93e-09	0	\\
3.04e-09	0	\\
3.14e-09	0	\\
3.24e-09	0	\\
3.34e-09	0	\\
3.45e-09	0	\\
3.55e-09	0	\\
3.65e-09	0	\\
3.75e-09	0	\\
3.86e-09	0	\\
3.96e-09	0	\\
4.06e-09	0	\\
4.16e-09	0	\\
4.27e-09	0	\\
4.37e-09	0	\\
4.47e-09	0	\\
4.57e-09	0	\\
4.68e-09	0	\\
4.78e-09	0	\\
4.89e-09	0	\\
4.99e-09	0	\\
5e-09	0	\\
};
\addplot [color=darkgray,solid,forget plot]
  table[row sep=crcr]{
0	0	\\
1.1e-10	0	\\
2.2e-10	0	\\
3.3e-10	0	\\
4.4e-10	0	\\
5.4e-10	0	\\
6.5e-10	0	\\
7.5e-10	0	\\
8.6e-10	0	\\
9.6e-10	0	\\
1.07e-09	0	\\
1.18e-09	0	\\
1.28e-09	0	\\
1.38e-09	0	\\
1.49e-09	0	\\
1.59e-09	0	\\
1.69e-09	0	\\
1.8e-09	0	\\
1.9e-09	0	\\
2.01e-09	0	\\
2.11e-09	0	\\
2.21e-09	0	\\
2.32e-09	0	\\
2.42e-09	0	\\
2.52e-09	0	\\
2.63e-09	0	\\
2.73e-09	0	\\
2.83e-09	0	\\
2.93e-09	0	\\
3.04e-09	0	\\
3.14e-09	0	\\
3.24e-09	0	\\
3.34e-09	0	\\
3.45e-09	0	\\
3.55e-09	0	\\
3.65e-09	0	\\
3.75e-09	0	\\
3.86e-09	0	\\
3.96e-09	0	\\
4.06e-09	0	\\
4.16e-09	0	\\
4.27e-09	0	\\
4.37e-09	0	\\
4.47e-09	0	\\
4.57e-09	0	\\
4.68e-09	0	\\
4.78e-09	0	\\
4.89e-09	0	\\
4.99e-09	0	\\
5e-09	0	\\
};
\addplot [color=blue,solid,forget plot]
  table[row sep=crcr]{
0	0	\\
1.1e-10	0	\\
2.2e-10	0	\\
3.3e-10	0	\\
4.4e-10	0	\\
5.4e-10	0	\\
6.5e-10	0	\\
7.5e-10	0	\\
8.6e-10	0	\\
9.6e-10	0	\\
1.07e-09	0	\\
1.18e-09	0	\\
1.28e-09	0	\\
1.38e-09	0	\\
1.49e-09	0	\\
1.59e-09	0	\\
1.69e-09	0	\\
1.8e-09	0	\\
1.9e-09	0	\\
2.01e-09	0	\\
2.11e-09	0	\\
2.21e-09	0	\\
2.32e-09	0	\\
2.42e-09	0	\\
2.52e-09	0	\\
2.63e-09	0	\\
2.73e-09	0	\\
2.83e-09	0	\\
2.93e-09	0	\\
3.04e-09	0	\\
3.14e-09	0	\\
3.24e-09	0	\\
3.34e-09	0	\\
3.45e-09	0	\\
3.55e-09	0	\\
3.65e-09	0	\\
3.75e-09	0	\\
3.86e-09	0	\\
3.96e-09	0	\\
4.06e-09	0	\\
4.16e-09	0	\\
4.27e-09	0	\\
4.37e-09	0	\\
4.47e-09	0	\\
4.57e-09	0	\\
4.68e-09	0	\\
4.78e-09	0	\\
4.89e-09	0	\\
4.99e-09	0	\\
5e-09	0	\\
};
\addplot [color=black!50!green,solid,forget plot]
  table[row sep=crcr]{
0	0	\\
1.1e-10	0	\\
2.2e-10	0	\\
3.3e-10	0	\\
4.4e-10	0	\\
5.4e-10	0	\\
6.5e-10	0	\\
7.5e-10	0	\\
8.6e-10	0	\\
9.6e-10	0	\\
1.07e-09	0	\\
1.18e-09	0	\\
1.28e-09	0	\\
1.38e-09	0	\\
1.49e-09	0	\\
1.59e-09	0	\\
1.69e-09	0	\\
1.8e-09	0	\\
1.9e-09	0	\\
2.01e-09	0	\\
2.11e-09	0	\\
2.21e-09	0	\\
2.32e-09	0	\\
2.42e-09	0	\\
2.52e-09	0	\\
2.63e-09	0	\\
2.73e-09	0	\\
2.83e-09	0	\\
2.93e-09	0	\\
3.04e-09	0	\\
3.14e-09	0	\\
3.24e-09	0	\\
3.34e-09	0	\\
3.45e-09	0	\\
3.55e-09	0	\\
3.65e-09	0	\\
3.75e-09	0	\\
3.86e-09	0	\\
3.96e-09	0	\\
4.06e-09	0	\\
4.16e-09	0	\\
4.27e-09	0	\\
4.37e-09	0	\\
4.47e-09	0	\\
4.57e-09	0	\\
4.68e-09	0	\\
4.78e-09	0	\\
4.89e-09	0	\\
4.99e-09	0	\\
5e-09	0	\\
};
\addplot [color=red,solid,forget plot]
  table[row sep=crcr]{
0	0	\\
1.1e-10	0	\\
2.2e-10	0	\\
3.3e-10	0	\\
4.4e-10	0	\\
5.4e-10	0	\\
6.5e-10	0	\\
7.5e-10	0	\\
8.6e-10	0	\\
9.6e-10	0	\\
1.07e-09	0	\\
1.18e-09	0	\\
1.28e-09	0	\\
1.38e-09	0	\\
1.49e-09	0	\\
1.59e-09	0	\\
1.69e-09	0	\\
1.8e-09	0	\\
1.9e-09	0	\\
2.01e-09	0	\\
2.11e-09	0	\\
2.21e-09	0	\\
2.32e-09	0	\\
2.42e-09	0	\\
2.52e-09	0	\\
2.63e-09	0	\\
2.73e-09	0	\\
2.83e-09	0	\\
2.93e-09	0	\\
3.04e-09	0	\\
3.14e-09	0	\\
3.24e-09	0	\\
3.34e-09	0	\\
3.45e-09	0	\\
3.55e-09	0	\\
3.65e-09	0	\\
3.75e-09	0	\\
3.86e-09	0	\\
3.96e-09	0	\\
4.06e-09	0	\\
4.16e-09	0	\\
4.27e-09	0	\\
4.37e-09	0	\\
4.47e-09	0	\\
4.57e-09	0	\\
4.68e-09	0	\\
4.78e-09	0	\\
4.89e-09	0	\\
4.99e-09	0	\\
5e-09	0	\\
};
\addplot [color=mycolor1,solid,forget plot]
  table[row sep=crcr]{
0	0	\\
1.1e-10	0	\\
2.2e-10	0	\\
3.3e-10	0	\\
4.4e-10	0	\\
5.4e-10	0	\\
6.5e-10	0	\\
7.5e-10	0	\\
8.6e-10	0	\\
9.6e-10	0	\\
1.07e-09	0	\\
1.18e-09	0	\\
1.28e-09	0	\\
1.38e-09	0	\\
1.49e-09	0	\\
1.59e-09	0	\\
1.69e-09	0	\\
1.8e-09	0	\\
1.9e-09	0	\\
2.01e-09	0	\\
2.11e-09	0	\\
2.21e-09	0	\\
2.32e-09	0	\\
2.42e-09	0	\\
2.52e-09	0	\\
2.63e-09	0	\\
2.73e-09	0	\\
2.83e-09	0	\\
2.93e-09	0	\\
3.04e-09	0	\\
3.14e-09	0	\\
3.24e-09	0	\\
3.34e-09	0	\\
3.45e-09	0	\\
3.55e-09	0	\\
3.65e-09	0	\\
3.75e-09	0	\\
3.86e-09	0	\\
3.96e-09	0	\\
4.06e-09	0	\\
4.16e-09	0	\\
4.27e-09	0	\\
4.37e-09	0	\\
4.47e-09	0	\\
4.57e-09	0	\\
4.68e-09	0	\\
4.78e-09	0	\\
4.89e-09	0	\\
4.99e-09	0	\\
5e-09	0	\\
};
\addplot [color=mycolor2,solid,forget plot]
  table[row sep=crcr]{
0	0	\\
1.1e-10	0	\\
2.2e-10	0	\\
3.3e-10	0	\\
4.4e-10	0	\\
5.4e-10	0	\\
6.5e-10	0	\\
7.5e-10	0	\\
8.6e-10	0	\\
9.6e-10	0	\\
1.07e-09	0	\\
1.18e-09	0	\\
1.28e-09	0	\\
1.38e-09	0	\\
1.49e-09	0	\\
1.59e-09	0	\\
1.69e-09	0	\\
1.8e-09	0	\\
1.9e-09	0	\\
2.01e-09	0	\\
2.11e-09	0	\\
2.21e-09	0	\\
2.32e-09	0	\\
2.42e-09	0	\\
2.52e-09	0	\\
2.63e-09	0	\\
2.73e-09	0	\\
2.83e-09	0	\\
2.93e-09	0	\\
3.04e-09	0	\\
3.14e-09	0	\\
3.24e-09	0	\\
3.34e-09	0	\\
3.45e-09	0	\\
3.55e-09	0	\\
3.65e-09	0	\\
3.75e-09	0	\\
3.86e-09	0	\\
3.96e-09	0	\\
4.06e-09	0	\\
4.16e-09	0	\\
4.27e-09	0	\\
4.37e-09	0	\\
4.47e-09	0	\\
4.57e-09	0	\\
4.68e-09	0	\\
4.78e-09	0	\\
4.89e-09	0	\\
4.99e-09	0	\\
5e-09	0	\\
};
\addplot [color=mycolor3,solid,forget plot]
  table[row sep=crcr]{
0	0	\\
1.1e-10	0	\\
2.2e-10	0	\\
3.3e-10	0	\\
4.4e-10	0	\\
5.4e-10	0	\\
6.5e-10	0	\\
7.5e-10	0	\\
8.6e-10	0	\\
9.6e-10	0	\\
1.07e-09	0	\\
1.18e-09	0	\\
1.28e-09	0	\\
1.38e-09	0	\\
1.49e-09	0	\\
1.59e-09	0	\\
1.69e-09	0	\\
1.8e-09	0	\\
1.9e-09	0	\\
2.01e-09	0	\\
2.11e-09	0	\\
2.21e-09	0	\\
2.32e-09	0	\\
2.42e-09	0	\\
2.52e-09	0	\\
2.63e-09	0	\\
2.73e-09	0	\\
2.83e-09	0	\\
2.93e-09	0	\\
3.04e-09	0	\\
3.14e-09	0	\\
3.24e-09	0	\\
3.34e-09	0	\\
3.45e-09	0	\\
3.55e-09	0	\\
3.65e-09	0	\\
3.75e-09	0	\\
3.86e-09	0	\\
3.96e-09	0	\\
4.06e-09	0	\\
4.16e-09	0	\\
4.27e-09	0	\\
4.37e-09	0	\\
4.47e-09	0	\\
4.57e-09	0	\\
4.68e-09	0	\\
4.78e-09	0	\\
4.89e-09	0	\\
4.99e-09	0	\\
5e-09	0	\\
};
\addplot [color=darkgray,solid,forget plot]
  table[row sep=crcr]{
0	0	\\
1.1e-10	0	\\
2.2e-10	0	\\
3.3e-10	0	\\
4.4e-10	0	\\
5.4e-10	0	\\
6.5e-10	0	\\
7.5e-10	0	\\
8.6e-10	0	\\
9.6e-10	0	\\
1.07e-09	0	\\
1.18e-09	0	\\
1.28e-09	0	\\
1.38e-09	0	\\
1.49e-09	0	\\
1.59e-09	0	\\
1.69e-09	0	\\
1.8e-09	0	\\
1.9e-09	0	\\
2.01e-09	0	\\
2.11e-09	0	\\
2.21e-09	0	\\
2.32e-09	0	\\
2.42e-09	0	\\
2.52e-09	0	\\
2.63e-09	0	\\
2.73e-09	0	\\
2.83e-09	0	\\
2.93e-09	0	\\
3.04e-09	0	\\
3.14e-09	0	\\
3.24e-09	0	\\
3.34e-09	0	\\
3.45e-09	0	\\
3.55e-09	0	\\
3.65e-09	0	\\
3.75e-09	0	\\
3.86e-09	0	\\
3.96e-09	0	\\
4.06e-09	0	\\
4.16e-09	0	\\
4.27e-09	0	\\
4.37e-09	0	\\
4.47e-09	0	\\
4.57e-09	0	\\
4.68e-09	0	\\
4.78e-09	0	\\
4.89e-09	0	\\
4.99e-09	0	\\
5e-09	0	\\
};
\addplot [color=blue,solid,forget plot]
  table[row sep=crcr]{
0	0	\\
1.1e-10	0	\\
2.2e-10	0	\\
3.3e-10	0	\\
4.4e-10	0	\\
5.4e-10	0	\\
6.5e-10	0	\\
7.5e-10	0	\\
8.6e-10	0	\\
9.6e-10	0	\\
1.07e-09	0	\\
1.18e-09	0	\\
1.28e-09	0	\\
1.38e-09	0	\\
1.49e-09	0	\\
1.59e-09	0	\\
1.69e-09	0	\\
1.8e-09	0	\\
1.9e-09	0	\\
2.01e-09	0	\\
2.11e-09	0	\\
2.21e-09	0	\\
2.32e-09	0	\\
2.42e-09	0	\\
2.52e-09	0	\\
2.63e-09	0	\\
2.73e-09	0	\\
2.83e-09	0	\\
2.93e-09	0	\\
3.04e-09	0	\\
3.14e-09	0	\\
3.24e-09	0	\\
3.34e-09	0	\\
3.45e-09	0	\\
3.55e-09	0	\\
3.65e-09	0	\\
3.75e-09	0	\\
3.86e-09	0	\\
3.96e-09	0	\\
4.06e-09	0	\\
4.16e-09	0	\\
4.27e-09	0	\\
4.37e-09	0	\\
4.47e-09	0	\\
4.57e-09	0	\\
4.68e-09	0	\\
4.78e-09	0	\\
4.89e-09	0	\\
4.99e-09	0	\\
5e-09	0	\\
};
\addplot [color=black!50!green,solid,forget plot]
  table[row sep=crcr]{
0	0	\\
1.1e-10	0	\\
2.2e-10	0	\\
3.3e-10	0	\\
4.4e-10	0	\\
5.4e-10	0	\\
6.5e-10	0	\\
7.5e-10	0	\\
8.6e-10	0	\\
9.6e-10	0	\\
1.07e-09	0	\\
1.18e-09	0	\\
1.28e-09	0	\\
1.38e-09	0	\\
1.49e-09	0	\\
1.59e-09	0	\\
1.69e-09	0	\\
1.8e-09	0	\\
1.9e-09	0	\\
2.01e-09	0	\\
2.11e-09	0	\\
2.21e-09	0	\\
2.32e-09	0	\\
2.42e-09	0	\\
2.52e-09	0	\\
2.63e-09	0	\\
2.73e-09	0	\\
2.83e-09	0	\\
2.93e-09	0	\\
3.04e-09	0	\\
3.14e-09	0	\\
3.24e-09	0	\\
3.34e-09	0	\\
3.45e-09	0	\\
3.55e-09	0	\\
3.65e-09	0	\\
3.75e-09	0	\\
3.86e-09	0	\\
3.96e-09	0	\\
4.06e-09	0	\\
4.16e-09	0	\\
4.27e-09	0	\\
4.37e-09	0	\\
4.47e-09	0	\\
4.57e-09	0	\\
4.68e-09	0	\\
4.78e-09	0	\\
4.89e-09	0	\\
4.99e-09	0	\\
5e-09	0	\\
};
\addplot [color=red,solid,forget plot]
  table[row sep=crcr]{
0	0	\\
1.1e-10	0	\\
2.2e-10	0	\\
3.3e-10	0	\\
4.4e-10	0	\\
5.4e-10	0	\\
6.5e-10	0	\\
7.5e-10	0	\\
8.6e-10	0	\\
9.6e-10	0	\\
1.07e-09	0	\\
1.18e-09	0	\\
1.28e-09	0	\\
1.38e-09	0	\\
1.49e-09	0	\\
1.59e-09	0	\\
1.69e-09	0	\\
1.8e-09	0	\\
1.9e-09	0	\\
2.01e-09	0	\\
2.11e-09	0	\\
2.21e-09	0	\\
2.32e-09	0	\\
2.42e-09	0	\\
2.52e-09	0	\\
2.63e-09	0	\\
2.73e-09	0	\\
2.83e-09	0	\\
2.93e-09	0	\\
3.04e-09	0	\\
3.14e-09	0	\\
3.24e-09	0	\\
3.34e-09	0	\\
3.45e-09	0	\\
3.55e-09	0	\\
3.65e-09	0	\\
3.75e-09	0	\\
3.86e-09	0	\\
3.96e-09	0	\\
4.06e-09	0	\\
4.16e-09	0	\\
4.27e-09	0	\\
4.37e-09	0	\\
4.47e-09	0	\\
4.57e-09	0	\\
4.68e-09	0	\\
4.78e-09	0	\\
4.89e-09	0	\\
4.99e-09	0	\\
5e-09	0	\\
};
\addplot [color=mycolor1,solid,forget plot]
  table[row sep=crcr]{
0	0	\\
1.1e-10	0	\\
2.2e-10	0	\\
3.3e-10	0	\\
4.4e-10	0	\\
5.4e-10	0	\\
6.5e-10	0	\\
7.5e-10	0	\\
8.6e-10	0	\\
9.6e-10	0	\\
1.07e-09	0	\\
1.18e-09	0	\\
1.28e-09	0	\\
1.38e-09	0	\\
1.49e-09	0	\\
1.59e-09	0	\\
1.69e-09	0	\\
1.8e-09	0	\\
1.9e-09	0	\\
2.01e-09	0	\\
2.11e-09	0	\\
2.21e-09	0	\\
2.32e-09	0	\\
2.42e-09	0	\\
2.52e-09	0	\\
2.63e-09	0	\\
2.73e-09	0	\\
2.83e-09	0	\\
2.93e-09	0	\\
3.04e-09	0	\\
3.14e-09	0	\\
3.24e-09	0	\\
3.34e-09	0	\\
3.45e-09	0	\\
3.55e-09	0	\\
3.65e-09	0	\\
3.75e-09	0	\\
3.86e-09	0	\\
3.96e-09	0	\\
4.06e-09	0	\\
4.16e-09	0	\\
4.27e-09	0	\\
4.37e-09	0	\\
4.47e-09	0	\\
4.57e-09	0	\\
4.68e-09	0	\\
4.78e-09	0	\\
4.89e-09	0	\\
4.99e-09	0	\\
5e-09	0	\\
};
\addplot [color=mycolor2,solid,forget plot]
  table[row sep=crcr]{
0	0	\\
1.1e-10	0	\\
2.2e-10	0	\\
3.3e-10	0	\\
4.4e-10	0	\\
5.4e-10	0	\\
6.5e-10	0	\\
7.5e-10	0	\\
8.6e-10	0	\\
9.6e-10	0	\\
1.07e-09	0	\\
1.18e-09	0	\\
1.28e-09	0	\\
1.38e-09	0	\\
1.49e-09	0	\\
1.59e-09	0	\\
1.69e-09	0	\\
1.8e-09	0	\\
1.9e-09	0	\\
2.01e-09	0	\\
2.11e-09	0	\\
2.21e-09	0	\\
2.32e-09	0	\\
2.42e-09	0	\\
2.52e-09	0	\\
2.63e-09	0	\\
2.73e-09	0	\\
2.83e-09	0	\\
2.93e-09	0	\\
3.04e-09	0	\\
3.14e-09	0	\\
3.24e-09	0	\\
3.34e-09	0	\\
3.45e-09	0	\\
3.55e-09	0	\\
3.65e-09	0	\\
3.75e-09	0	\\
3.86e-09	0	\\
3.96e-09	0	\\
4.06e-09	0	\\
4.16e-09	0	\\
4.27e-09	0	\\
4.37e-09	0	\\
4.47e-09	0	\\
4.57e-09	0	\\
4.68e-09	0	\\
4.78e-09	0	\\
4.89e-09	0	\\
4.99e-09	0	\\
5e-09	0	\\
};
\addplot [color=mycolor3,solid,forget plot]
  table[row sep=crcr]{
0	0	\\
1.1e-10	0	\\
2.2e-10	0	\\
3.3e-10	0	\\
4.4e-10	0	\\
5.4e-10	0	\\
6.5e-10	0	\\
7.5e-10	0	\\
8.6e-10	0	\\
9.6e-10	0	\\
1.07e-09	0	\\
1.18e-09	0	\\
1.28e-09	0	\\
1.38e-09	0	\\
1.49e-09	0	\\
1.59e-09	0	\\
1.69e-09	0	\\
1.8e-09	0	\\
1.9e-09	0	\\
2.01e-09	0	\\
2.11e-09	0	\\
2.21e-09	0	\\
2.32e-09	0	\\
2.42e-09	0	\\
2.52e-09	0	\\
2.63e-09	0	\\
2.73e-09	0	\\
2.83e-09	0	\\
2.93e-09	0	\\
3.04e-09	0	\\
3.14e-09	0	\\
3.24e-09	0	\\
3.34e-09	0	\\
3.45e-09	0	\\
3.55e-09	0	\\
3.65e-09	0	\\
3.75e-09	0	\\
3.86e-09	0	\\
3.96e-09	0	\\
4.06e-09	0	\\
4.16e-09	0	\\
4.27e-09	0	\\
4.37e-09	0	\\
4.47e-09	0	\\
4.57e-09	0	\\
4.68e-09	0	\\
4.78e-09	0	\\
4.89e-09	0	\\
4.99e-09	0	\\
5e-09	0	\\
};
\addplot [color=darkgray,solid,forget plot]
  table[row sep=crcr]{
0	0	\\
1.1e-10	0	\\
2.2e-10	0	\\
3.3e-10	0	\\
4.4e-10	0	\\
5.4e-10	0	\\
6.5e-10	0	\\
7.5e-10	0	\\
8.6e-10	0	\\
9.6e-10	0	\\
1.07e-09	0	\\
1.18e-09	0	\\
1.28e-09	0	\\
1.38e-09	0	\\
1.49e-09	0	\\
1.59e-09	0	\\
1.69e-09	0	\\
1.8e-09	0	\\
1.9e-09	0	\\
2.01e-09	0	\\
2.11e-09	0	\\
2.21e-09	0	\\
2.32e-09	0	\\
2.42e-09	0	\\
2.52e-09	0	\\
2.63e-09	0	\\
2.73e-09	0	\\
2.83e-09	0	\\
2.93e-09	0	\\
3.04e-09	0	\\
3.14e-09	0	\\
3.24e-09	0	\\
3.34e-09	0	\\
3.45e-09	0	\\
3.55e-09	0	\\
3.65e-09	0	\\
3.75e-09	0	\\
3.86e-09	0	\\
3.96e-09	0	\\
4.06e-09	0	\\
4.16e-09	0	\\
4.27e-09	0	\\
4.37e-09	0	\\
4.47e-09	0	\\
4.57e-09	0	\\
4.68e-09	0	\\
4.78e-09	0	\\
4.89e-09	0	\\
4.99e-09	0	\\
5e-09	0	\\
};
\addplot [color=blue,solid,forget plot]
  table[row sep=crcr]{
0	0	\\
1.1e-10	0	\\
2.2e-10	0	\\
3.3e-10	0	\\
4.4e-10	0	\\
5.4e-10	0	\\
6.5e-10	0	\\
7.5e-10	0	\\
8.6e-10	0	\\
9.6e-10	0	\\
1.07e-09	0	\\
1.18e-09	0	\\
1.28e-09	0	\\
1.38e-09	0	\\
1.49e-09	0	\\
1.59e-09	0	\\
1.69e-09	0	\\
1.8e-09	0	\\
1.9e-09	0	\\
2.01e-09	0	\\
2.11e-09	0	\\
2.21e-09	0	\\
2.32e-09	0	\\
2.42e-09	0	\\
2.52e-09	0	\\
2.63e-09	0	\\
2.73e-09	0	\\
2.83e-09	0	\\
2.93e-09	0	\\
3.04e-09	0	\\
3.14e-09	0	\\
3.24e-09	0	\\
3.34e-09	0	\\
3.45e-09	0	\\
3.55e-09	0	\\
3.65e-09	0	\\
3.75e-09	0	\\
3.86e-09	0	\\
3.96e-09	0	\\
4.06e-09	0	\\
4.16e-09	0	\\
4.27e-09	0	\\
4.37e-09	0	\\
4.47e-09	0	\\
4.57e-09	0	\\
4.68e-09	0	\\
4.78e-09	0	\\
4.89e-09	0	\\
4.99e-09	0	\\
5e-09	0	\\
};
\addplot [color=black!50!green,solid,forget plot]
  table[row sep=crcr]{
0	0	\\
1.1e-10	0	\\
2.2e-10	0	\\
3.3e-10	0	\\
4.4e-10	0	\\
5.4e-10	0	\\
6.5e-10	0	\\
7.5e-10	0	\\
8.6e-10	0	\\
9.6e-10	0	\\
1.07e-09	0	\\
1.18e-09	0	\\
1.28e-09	0	\\
1.38e-09	0	\\
1.49e-09	0	\\
1.59e-09	0	\\
1.69e-09	0	\\
1.8e-09	0	\\
1.9e-09	0	\\
2.01e-09	0	\\
2.11e-09	0	\\
2.21e-09	0	\\
2.32e-09	0	\\
2.42e-09	0	\\
2.52e-09	0	\\
2.63e-09	0	\\
2.73e-09	0	\\
2.83e-09	0	\\
2.93e-09	0	\\
3.04e-09	0	\\
3.14e-09	0	\\
3.24e-09	0	\\
3.34e-09	0	\\
3.45e-09	0	\\
3.55e-09	0	\\
3.65e-09	0	\\
3.75e-09	0	\\
3.86e-09	0	\\
3.96e-09	0	\\
4.06e-09	0	\\
4.16e-09	0	\\
4.27e-09	0	\\
4.37e-09	0	\\
4.47e-09	0	\\
4.57e-09	0	\\
4.68e-09	0	\\
4.78e-09	0	\\
4.89e-09	0	\\
4.99e-09	0	\\
5e-09	0	\\
};
\addplot [color=red,solid,forget plot]
  table[row sep=crcr]{
0	0	\\
1.1e-10	0	\\
2.2e-10	0	\\
3.3e-10	0	\\
4.4e-10	0	\\
5.4e-10	0	\\
6.5e-10	0	\\
7.5e-10	0	\\
8.6e-10	0	\\
9.6e-10	0	\\
1.07e-09	0	\\
1.18e-09	0	\\
1.28e-09	0	\\
1.38e-09	0	\\
1.49e-09	0	\\
1.59e-09	0	\\
1.69e-09	0	\\
1.8e-09	0	\\
1.9e-09	0	\\
2.01e-09	0	\\
2.11e-09	0	\\
2.21e-09	0	\\
2.32e-09	0	\\
2.42e-09	0	\\
2.52e-09	0	\\
2.63e-09	0	\\
2.73e-09	0	\\
2.83e-09	0	\\
2.93e-09	0	\\
3.04e-09	0	\\
3.14e-09	0	\\
3.24e-09	0	\\
3.34e-09	0	\\
3.45e-09	0	\\
3.55e-09	0	\\
3.65e-09	0	\\
3.75e-09	0	\\
3.86e-09	0	\\
3.96e-09	0	\\
4.06e-09	0	\\
4.16e-09	0	\\
4.27e-09	0	\\
4.37e-09	0	\\
4.47e-09	0	\\
4.57e-09	0	\\
4.68e-09	0	\\
4.78e-09	0	\\
4.89e-09	0	\\
4.99e-09	0	\\
5e-09	0	\\
};
\addplot [color=mycolor1,solid,forget plot]
  table[row sep=crcr]{
0	0	\\
1.1e-10	0	\\
2.2e-10	0	\\
3.3e-10	0	\\
4.4e-10	0	\\
5.4e-10	0	\\
6.5e-10	0	\\
7.5e-10	0	\\
8.6e-10	0	\\
9.6e-10	0	\\
1.07e-09	0	\\
1.18e-09	0	\\
1.28e-09	0	\\
1.38e-09	0	\\
1.49e-09	0	\\
1.59e-09	0	\\
1.69e-09	0	\\
1.8e-09	0	\\
1.9e-09	0	\\
2.01e-09	0	\\
2.11e-09	0	\\
2.21e-09	0	\\
2.32e-09	0	\\
2.42e-09	0	\\
2.52e-09	0	\\
2.63e-09	0	\\
2.73e-09	0	\\
2.83e-09	0	\\
2.93e-09	0	\\
3.04e-09	0	\\
3.14e-09	0	\\
3.24e-09	0	\\
3.34e-09	0	\\
3.45e-09	0	\\
3.55e-09	0	\\
3.65e-09	0	\\
3.75e-09	0	\\
3.86e-09	0	\\
3.96e-09	0	\\
4.06e-09	0	\\
4.16e-09	0	\\
4.27e-09	0	\\
4.37e-09	0	\\
4.47e-09	0	\\
4.57e-09	0	\\
4.68e-09	0	\\
4.78e-09	0	\\
4.89e-09	0	\\
4.99e-09	0	\\
5e-09	0	\\
};
\addplot [color=mycolor2,solid,forget plot]
  table[row sep=crcr]{
0	0	\\
1.1e-10	0	\\
2.2e-10	0	\\
3.3e-10	0	\\
4.4e-10	0	\\
5.4e-10	0	\\
6.5e-10	0	\\
7.5e-10	0	\\
8.6e-10	0	\\
9.6e-10	0	\\
1.07e-09	0	\\
1.18e-09	0	\\
1.28e-09	0	\\
1.38e-09	0	\\
1.49e-09	0	\\
1.59e-09	0	\\
1.69e-09	0	\\
1.8e-09	0	\\
1.9e-09	0	\\
2.01e-09	0	\\
2.11e-09	0	\\
2.21e-09	0	\\
2.32e-09	0	\\
2.42e-09	0	\\
2.52e-09	0	\\
2.63e-09	0	\\
2.73e-09	0	\\
2.83e-09	0	\\
2.93e-09	0	\\
3.04e-09	0	\\
3.14e-09	0	\\
3.24e-09	0	\\
3.34e-09	0	\\
3.45e-09	0	\\
3.55e-09	0	\\
3.65e-09	0	\\
3.75e-09	0	\\
3.86e-09	0	\\
3.96e-09	0	\\
4.06e-09	0	\\
4.16e-09	0	\\
4.27e-09	0	\\
4.37e-09	0	\\
4.47e-09	0	\\
4.57e-09	0	\\
4.68e-09	0	\\
4.78e-09	0	\\
4.89e-09	0	\\
4.99e-09	0	\\
5e-09	0	\\
};
\addplot [color=mycolor3,solid,forget plot]
  table[row sep=crcr]{
0	0	\\
1.1e-10	0	\\
2.2e-10	0	\\
3.3e-10	0	\\
4.4e-10	0	\\
5.4e-10	0	\\
6.5e-10	0	\\
7.5e-10	0	\\
8.6e-10	0	\\
9.6e-10	0	\\
1.07e-09	0	\\
1.18e-09	0	\\
1.28e-09	0	\\
1.38e-09	0	\\
1.49e-09	0	\\
1.59e-09	0	\\
1.69e-09	0	\\
1.8e-09	0	\\
1.9e-09	0	\\
2.01e-09	0	\\
2.11e-09	0	\\
2.21e-09	0	\\
2.32e-09	0	\\
2.42e-09	0	\\
2.52e-09	0	\\
2.63e-09	0	\\
2.73e-09	0	\\
2.83e-09	0	\\
2.93e-09	0	\\
3.04e-09	0	\\
3.14e-09	0	\\
3.24e-09	0	\\
3.34e-09	0	\\
3.45e-09	0	\\
3.55e-09	0	\\
3.65e-09	0	\\
3.75e-09	0	\\
3.86e-09	0	\\
3.96e-09	0	\\
4.06e-09	0	\\
4.16e-09	0	\\
4.27e-09	0	\\
4.37e-09	0	\\
4.47e-09	0	\\
4.57e-09	0	\\
4.68e-09	0	\\
4.78e-09	0	\\
4.89e-09	0	\\
4.99e-09	0	\\
5e-09	0	\\
};
\addplot [color=darkgray,solid,forget plot]
  table[row sep=crcr]{
0	0	\\
1.1e-10	0	\\
2.2e-10	0	\\
3.3e-10	0	\\
4.4e-10	0	\\
5.4e-10	0	\\
6.5e-10	0	\\
7.5e-10	0	\\
8.6e-10	0	\\
9.6e-10	0	\\
1.07e-09	0	\\
1.18e-09	0	\\
1.28e-09	0	\\
1.38e-09	0	\\
1.49e-09	0	\\
1.59e-09	0	\\
1.69e-09	0	\\
1.8e-09	0	\\
1.9e-09	0	\\
2.01e-09	0	\\
2.11e-09	0	\\
2.21e-09	0	\\
2.32e-09	0	\\
2.42e-09	0	\\
2.52e-09	0	\\
2.63e-09	0	\\
2.73e-09	0	\\
2.83e-09	0	\\
2.93e-09	0	\\
3.04e-09	0	\\
3.14e-09	0	\\
3.24e-09	0	\\
3.34e-09	0	\\
3.45e-09	0	\\
3.55e-09	0	\\
3.65e-09	0	\\
3.75e-09	0	\\
3.86e-09	0	\\
3.96e-09	0	\\
4.06e-09	0	\\
4.16e-09	0	\\
4.27e-09	0	\\
4.37e-09	0	\\
4.47e-09	0	\\
4.57e-09	0	\\
4.68e-09	0	\\
4.78e-09	0	\\
4.89e-09	0	\\
4.99e-09	0	\\
5e-09	0	\\
};
\addplot [color=blue,solid,forget plot]
  table[row sep=crcr]{
0	0	\\
1.1e-10	0	\\
2.2e-10	0	\\
3.3e-10	0	\\
4.4e-10	0	\\
5.4e-10	0	\\
6.5e-10	0	\\
7.5e-10	0	\\
8.6e-10	0	\\
9.6e-10	0	\\
1.07e-09	0	\\
1.18e-09	0	\\
1.28e-09	0	\\
1.38e-09	0	\\
1.49e-09	0	\\
1.59e-09	0	\\
1.69e-09	0	\\
1.8e-09	0	\\
1.9e-09	0	\\
2.01e-09	0	\\
2.11e-09	0	\\
2.21e-09	0	\\
2.32e-09	0	\\
2.42e-09	0	\\
2.52e-09	0	\\
2.63e-09	0	\\
2.73e-09	0	\\
2.83e-09	0	\\
2.93e-09	0	\\
3.04e-09	0	\\
3.14e-09	0	\\
3.24e-09	0	\\
3.34e-09	0	\\
3.45e-09	0	\\
3.55e-09	0	\\
3.65e-09	0	\\
3.75e-09	0	\\
3.86e-09	0	\\
3.96e-09	0	\\
4.06e-09	0	\\
4.16e-09	0	\\
4.27e-09	0	\\
4.37e-09	0	\\
4.47e-09	0	\\
4.57e-09	0	\\
4.68e-09	0	\\
4.78e-09	0	\\
4.89e-09	0	\\
4.99e-09	0	\\
5e-09	0	\\
};
\addplot [color=black!50!green,solid,forget plot]
  table[row sep=crcr]{
0	0	\\
1.1e-10	0	\\
2.2e-10	0	\\
3.3e-10	0	\\
4.4e-10	0	\\
5.4e-10	0	\\
6.5e-10	0	\\
7.5e-10	0	\\
8.6e-10	0	\\
9.6e-10	0	\\
1.07e-09	0	\\
1.18e-09	0	\\
1.28e-09	0	\\
1.38e-09	0	\\
1.49e-09	0	\\
1.59e-09	0	\\
1.69e-09	0	\\
1.8e-09	0	\\
1.9e-09	0	\\
2.01e-09	0	\\
2.11e-09	0	\\
2.21e-09	0	\\
2.32e-09	0	\\
2.42e-09	0	\\
2.52e-09	0	\\
2.63e-09	0	\\
2.73e-09	0	\\
2.83e-09	0	\\
2.93e-09	0	\\
3.04e-09	0	\\
3.14e-09	0	\\
3.24e-09	0	\\
3.34e-09	0	\\
3.45e-09	0	\\
3.55e-09	0	\\
3.65e-09	0	\\
3.75e-09	0	\\
3.86e-09	0	\\
3.96e-09	0	\\
4.06e-09	0	\\
4.16e-09	0	\\
4.27e-09	0	\\
4.37e-09	0	\\
4.47e-09	0	\\
4.57e-09	0	\\
4.68e-09	0	\\
4.78e-09	0	\\
4.89e-09	0	\\
4.99e-09	0	\\
5e-09	0	\\
};
\addplot [color=red,solid,forget plot]
  table[row sep=crcr]{
0	0	\\
1.1e-10	0	\\
2.2e-10	0	\\
3.3e-10	0	\\
4.4e-10	0	\\
5.4e-10	0	\\
6.5e-10	0	\\
7.5e-10	0	\\
8.6e-10	0	\\
9.6e-10	0	\\
1.07e-09	0	\\
1.18e-09	0	\\
1.28e-09	0	\\
1.38e-09	0	\\
1.49e-09	0	\\
1.59e-09	0	\\
1.69e-09	0	\\
1.8e-09	0	\\
1.9e-09	0	\\
2.01e-09	0	\\
2.11e-09	0	\\
2.21e-09	0	\\
2.32e-09	0	\\
2.42e-09	0	\\
2.52e-09	0	\\
2.63e-09	0	\\
2.73e-09	0	\\
2.83e-09	0	\\
2.93e-09	0	\\
3.04e-09	0	\\
3.14e-09	0	\\
3.24e-09	0	\\
3.34e-09	0	\\
3.45e-09	0	\\
3.55e-09	0	\\
3.65e-09	0	\\
3.75e-09	0	\\
3.86e-09	0	\\
3.96e-09	0	\\
4.06e-09	0	\\
4.16e-09	0	\\
4.27e-09	0	\\
4.37e-09	0	\\
4.47e-09	0	\\
4.57e-09	0	\\
4.68e-09	0	\\
4.78e-09	0	\\
4.89e-09	0	\\
4.99e-09	0	\\
5e-09	0	\\
};
\addplot [color=mycolor1,solid,forget plot]
  table[row sep=crcr]{
0	0	\\
1.1e-10	0	\\
2.2e-10	0	\\
3.3e-10	0	\\
4.4e-10	0	\\
5.4e-10	0	\\
6.5e-10	0	\\
7.5e-10	0	\\
8.6e-10	0	\\
9.6e-10	0	\\
1.07e-09	0	\\
1.18e-09	0	\\
1.28e-09	0	\\
1.38e-09	0	\\
1.49e-09	0	\\
1.59e-09	0	\\
1.69e-09	0	\\
1.8e-09	0	\\
1.9e-09	0	\\
2.01e-09	0	\\
2.11e-09	0	\\
2.21e-09	0	\\
2.32e-09	0	\\
2.42e-09	0	\\
2.52e-09	0	\\
2.63e-09	0	\\
2.73e-09	0	\\
2.83e-09	0	\\
2.93e-09	0	\\
3.04e-09	0	\\
3.14e-09	0	\\
3.24e-09	0	\\
3.34e-09	0	\\
3.45e-09	0	\\
3.55e-09	0	\\
3.65e-09	0	\\
3.75e-09	0	\\
3.86e-09	0	\\
3.96e-09	0	\\
4.06e-09	0	\\
4.16e-09	0	\\
4.27e-09	0	\\
4.37e-09	0	\\
4.47e-09	0	\\
4.57e-09	0	\\
4.68e-09	0	\\
4.78e-09	0	\\
4.89e-09	0	\\
4.99e-09	0	\\
5e-09	0	\\
};
\addplot [color=mycolor2,solid,forget plot]
  table[row sep=crcr]{
0	0	\\
1.1e-10	0	\\
2.2e-10	0	\\
3.3e-10	0	\\
4.4e-10	0	\\
5.4e-10	0	\\
6.5e-10	0	\\
7.5e-10	0	\\
8.6e-10	0	\\
9.6e-10	0	\\
1.07e-09	0	\\
1.18e-09	0	\\
1.28e-09	0	\\
1.38e-09	0	\\
1.49e-09	0	\\
1.59e-09	0	\\
1.69e-09	0	\\
1.8e-09	0	\\
1.9e-09	0	\\
2.01e-09	0	\\
2.11e-09	0	\\
2.21e-09	0	\\
2.32e-09	0	\\
2.42e-09	0	\\
2.52e-09	0	\\
2.63e-09	0	\\
2.73e-09	0	\\
2.83e-09	0	\\
2.93e-09	0	\\
3.04e-09	0	\\
3.14e-09	0	\\
3.24e-09	0	\\
3.34e-09	0	\\
3.45e-09	0	\\
3.55e-09	0	\\
3.65e-09	0	\\
3.75e-09	0	\\
3.86e-09	0	\\
3.96e-09	0	\\
4.06e-09	0	\\
4.16e-09	0	\\
4.27e-09	0	\\
4.37e-09	0	\\
4.47e-09	0	\\
4.57e-09	0	\\
4.68e-09	0	\\
4.78e-09	0	\\
4.89e-09	0	\\
4.99e-09	0	\\
5e-09	0	\\
};
\addplot [color=mycolor3,solid,forget plot]
  table[row sep=crcr]{
0	0	\\
1.1e-10	0	\\
2.2e-10	0	\\
3.3e-10	0	\\
4.4e-10	0	\\
5.4e-10	0	\\
6.5e-10	0	\\
7.5e-10	0	\\
8.6e-10	0	\\
9.6e-10	0	\\
1.07e-09	0	\\
1.18e-09	0	\\
1.28e-09	0	\\
1.38e-09	0	\\
1.49e-09	0	\\
1.59e-09	0	\\
1.69e-09	0	\\
1.8e-09	0	\\
1.9e-09	0	\\
2.01e-09	0	\\
2.11e-09	0	\\
2.21e-09	0	\\
2.32e-09	0	\\
2.42e-09	0	\\
2.52e-09	0	\\
2.63e-09	0	\\
2.73e-09	0	\\
2.83e-09	0	\\
2.93e-09	0	\\
3.04e-09	0	\\
3.14e-09	0	\\
3.24e-09	0	\\
3.34e-09	0	\\
3.45e-09	0	\\
3.55e-09	0	\\
3.65e-09	0	\\
3.75e-09	0	\\
3.86e-09	0	\\
3.96e-09	0	\\
4.06e-09	0	\\
4.16e-09	0	\\
4.27e-09	0	\\
4.37e-09	0	\\
4.47e-09	0	\\
4.57e-09	0	\\
4.68e-09	0	\\
4.78e-09	0	\\
4.89e-09	0	\\
4.99e-09	0	\\
5e-09	0	\\
};
\addplot [color=darkgray,solid,forget plot]
  table[row sep=crcr]{
0	0	\\
1.1e-10	0	\\
2.2e-10	0	\\
3.3e-10	0	\\
4.4e-10	0	\\
5.4e-10	0	\\
6.5e-10	0	\\
7.5e-10	0	\\
8.6e-10	0	\\
9.6e-10	0	\\
1.07e-09	0	\\
1.18e-09	0	\\
1.28e-09	0	\\
1.38e-09	0	\\
1.49e-09	0	\\
1.59e-09	0	\\
1.69e-09	0	\\
1.8e-09	0	\\
1.9e-09	0	\\
2.01e-09	0	\\
2.11e-09	0	\\
2.21e-09	0	\\
2.32e-09	0	\\
2.42e-09	0	\\
2.52e-09	0	\\
2.63e-09	0	\\
2.73e-09	0	\\
2.83e-09	0	\\
2.93e-09	0	\\
3.04e-09	0	\\
3.14e-09	0	\\
3.24e-09	0	\\
3.34e-09	0	\\
3.45e-09	0	\\
3.55e-09	0	\\
3.65e-09	0	\\
3.75e-09	0	\\
3.86e-09	0	\\
3.96e-09	0	\\
4.06e-09	0	\\
4.16e-09	0	\\
4.27e-09	0	\\
4.37e-09	0	\\
4.47e-09	0	\\
4.57e-09	0	\\
4.68e-09	0	\\
4.78e-09	0	\\
4.89e-09	0	\\
4.99e-09	0	\\
5e-09	0	\\
};
\addplot [color=blue,solid,forget plot]
  table[row sep=crcr]{
0	0	\\
1.1e-10	0	\\
2.2e-10	0	\\
3.3e-10	0	\\
4.4e-10	0	\\
5.4e-10	0	\\
6.5e-10	0	\\
7.5e-10	0	\\
8.6e-10	0	\\
9.6e-10	0	\\
1.07e-09	0	\\
1.18e-09	0	\\
1.28e-09	0	\\
1.38e-09	0	\\
1.49e-09	0	\\
1.59e-09	0	\\
1.69e-09	0	\\
1.8e-09	0	\\
1.9e-09	0	\\
2.01e-09	0	\\
2.11e-09	0	\\
2.21e-09	0	\\
2.32e-09	0	\\
2.42e-09	0	\\
2.52e-09	0	\\
2.63e-09	0	\\
2.73e-09	0	\\
2.83e-09	0	\\
2.93e-09	0	\\
3.04e-09	0	\\
3.14e-09	0	\\
3.24e-09	0	\\
3.34e-09	0	\\
3.45e-09	0	\\
3.55e-09	0	\\
3.65e-09	0	\\
3.75e-09	0	\\
3.86e-09	0	\\
3.96e-09	0	\\
4.06e-09	0	\\
4.16e-09	0	\\
4.27e-09	0	\\
4.37e-09	0	\\
4.47e-09	0	\\
4.57e-09	0	\\
4.68e-09	0	\\
4.78e-09	0	\\
4.89e-09	0	\\
4.99e-09	0	\\
5e-09	0	\\
};
\addplot [color=black!50!green,solid,forget plot]
  table[row sep=crcr]{
0	0	\\
1.1e-10	0	\\
2.2e-10	0	\\
3.3e-10	0	\\
4.4e-10	0	\\
5.4e-10	0	\\
6.5e-10	0	\\
7.5e-10	0	\\
8.6e-10	0	\\
9.6e-10	0	\\
1.07e-09	0	\\
1.18e-09	0	\\
1.28e-09	0	\\
1.38e-09	0	\\
1.49e-09	0	\\
1.59e-09	0	\\
1.69e-09	0	\\
1.8e-09	0	\\
1.9e-09	0	\\
2.01e-09	0	\\
2.11e-09	0	\\
2.21e-09	0	\\
2.32e-09	0	\\
2.42e-09	0	\\
2.52e-09	0	\\
2.63e-09	0	\\
2.73e-09	0	\\
2.83e-09	0	\\
2.93e-09	0	\\
3.04e-09	0	\\
3.14e-09	0	\\
3.24e-09	0	\\
3.34e-09	0	\\
3.45e-09	0	\\
3.55e-09	0	\\
3.65e-09	0	\\
3.75e-09	0	\\
3.86e-09	0	\\
3.96e-09	0	\\
4.06e-09	0	\\
4.16e-09	0	\\
4.27e-09	0	\\
4.37e-09	0	\\
4.47e-09	0	\\
4.57e-09	0	\\
4.68e-09	0	\\
4.78e-09	0	\\
4.89e-09	0	\\
4.99e-09	0	\\
5e-09	0	\\
};
\addplot [color=red,solid,forget plot]
  table[row sep=crcr]{
0	0	\\
1.1e-10	0	\\
2.2e-10	0	\\
3.3e-10	0	\\
4.4e-10	0	\\
5.4e-10	0	\\
6.5e-10	0	\\
7.5e-10	0	\\
8.6e-10	0	\\
9.6e-10	0	\\
1.07e-09	0	\\
1.18e-09	0	\\
1.28e-09	0	\\
1.38e-09	0	\\
1.49e-09	0	\\
1.59e-09	0	\\
1.69e-09	0	\\
1.8e-09	0	\\
1.9e-09	0	\\
2.01e-09	0	\\
2.11e-09	0	\\
2.21e-09	0	\\
2.32e-09	0	\\
2.42e-09	0	\\
2.52e-09	0	\\
2.63e-09	0	\\
2.73e-09	0	\\
2.83e-09	0	\\
2.93e-09	0	\\
3.04e-09	0	\\
3.14e-09	0	\\
3.24e-09	0	\\
3.34e-09	0	\\
3.45e-09	0	\\
3.55e-09	0	\\
3.65e-09	0	\\
3.75e-09	0	\\
3.86e-09	0	\\
3.96e-09	0	\\
4.06e-09	0	\\
4.16e-09	0	\\
4.27e-09	0	\\
4.37e-09	0	\\
4.47e-09	0	\\
4.57e-09	0	\\
4.68e-09	0	\\
4.78e-09	0	\\
4.89e-09	0	\\
4.99e-09	0	\\
5e-09	0	\\
};
\addplot [color=mycolor1,solid,forget plot]
  table[row sep=crcr]{
0	0	\\
1.1e-10	0	\\
2.2e-10	0	\\
3.3e-10	0	\\
4.4e-10	0	\\
5.4e-10	0	\\
6.5e-10	0	\\
7.5e-10	0	\\
8.6e-10	0	\\
9.6e-10	0	\\
1.07e-09	0	\\
1.18e-09	0	\\
1.28e-09	0	\\
1.38e-09	0	\\
1.49e-09	0	\\
1.59e-09	0	\\
1.69e-09	0	\\
1.8e-09	0	\\
1.9e-09	0	\\
2.01e-09	0	\\
2.11e-09	0	\\
2.21e-09	0	\\
2.32e-09	0	\\
2.42e-09	0	\\
2.52e-09	0	\\
2.63e-09	0	\\
2.73e-09	0	\\
2.83e-09	0	\\
2.93e-09	0	\\
3.04e-09	0	\\
3.14e-09	0	\\
3.24e-09	0	\\
3.34e-09	0	\\
3.45e-09	0	\\
3.55e-09	0	\\
3.65e-09	0	\\
3.75e-09	0	\\
3.86e-09	0	\\
3.96e-09	0	\\
4.06e-09	0	\\
4.16e-09	0	\\
4.27e-09	0	\\
4.37e-09	0	\\
4.47e-09	0	\\
4.57e-09	0	\\
4.68e-09	0	\\
4.78e-09	0	\\
4.89e-09	0	\\
4.99e-09	0	\\
5e-09	0	\\
};
\addplot [color=mycolor2,solid,forget plot]
  table[row sep=crcr]{
0	0	\\
1.1e-10	0	\\
2.2e-10	0	\\
3.3e-10	0	\\
4.4e-10	0	\\
5.4e-10	0	\\
6.5e-10	0	\\
7.5e-10	0	\\
8.6e-10	0	\\
9.6e-10	0	\\
1.07e-09	0	\\
1.18e-09	0	\\
1.28e-09	0	\\
1.38e-09	0	\\
1.49e-09	0	\\
1.59e-09	0	\\
1.69e-09	0	\\
1.8e-09	0	\\
1.9e-09	0	\\
2.01e-09	0	\\
2.11e-09	0	\\
2.21e-09	0	\\
2.32e-09	0	\\
2.42e-09	0	\\
2.52e-09	0	\\
2.63e-09	0	\\
2.73e-09	0	\\
2.83e-09	0	\\
2.93e-09	0	\\
3.04e-09	0	\\
3.14e-09	0	\\
3.24e-09	0	\\
3.34e-09	0	\\
3.45e-09	0	\\
3.55e-09	0	\\
3.65e-09	0	\\
3.75e-09	0	\\
3.86e-09	0	\\
3.96e-09	0	\\
4.06e-09	0	\\
4.16e-09	0	\\
4.27e-09	0	\\
4.37e-09	0	\\
4.47e-09	0	\\
4.57e-09	0	\\
4.68e-09	0	\\
4.78e-09	0	\\
4.89e-09	0	\\
4.99e-09	0	\\
5e-09	0	\\
};
\addplot [color=mycolor3,solid,forget plot]
  table[row sep=crcr]{
0	0	\\
1.1e-10	0	\\
2.2e-10	0	\\
3.3e-10	0	\\
4.4e-10	0	\\
5.4e-10	0	\\
6.5e-10	0	\\
7.5e-10	0	\\
8.6e-10	0	\\
9.6e-10	0	\\
1.07e-09	0	\\
1.18e-09	0	\\
1.28e-09	0	\\
1.38e-09	0	\\
1.49e-09	0	\\
1.59e-09	0	\\
1.69e-09	0	\\
1.8e-09	0	\\
1.9e-09	0	\\
2.01e-09	0	\\
2.11e-09	0	\\
2.21e-09	0	\\
2.32e-09	0	\\
2.42e-09	0	\\
2.52e-09	0	\\
2.63e-09	0	\\
2.73e-09	0	\\
2.83e-09	0	\\
2.93e-09	0	\\
3.04e-09	0	\\
3.14e-09	0	\\
3.24e-09	0	\\
3.34e-09	0	\\
3.45e-09	0	\\
3.55e-09	0	\\
3.65e-09	0	\\
3.75e-09	0	\\
3.86e-09	0	\\
3.96e-09	0	\\
4.06e-09	0	\\
4.16e-09	0	\\
4.27e-09	0	\\
4.37e-09	0	\\
4.47e-09	0	\\
4.57e-09	0	\\
4.68e-09	0	\\
4.78e-09	0	\\
4.89e-09	0	\\
4.99e-09	0	\\
5e-09	0	\\
};
\addplot [color=darkgray,solid,forget plot]
  table[row sep=crcr]{
0	0	\\
1.1e-10	0	\\
2.2e-10	0	\\
3.3e-10	0	\\
4.4e-10	0	\\
5.4e-10	0	\\
6.5e-10	0	\\
7.5e-10	0	\\
8.6e-10	0	\\
9.6e-10	0	\\
1.07e-09	0	\\
1.18e-09	0	\\
1.28e-09	0	\\
1.38e-09	0	\\
1.49e-09	0	\\
1.59e-09	0	\\
1.69e-09	0	\\
1.8e-09	0	\\
1.9e-09	0	\\
2.01e-09	0	\\
2.11e-09	0	\\
2.21e-09	0	\\
2.32e-09	0	\\
2.42e-09	0	\\
2.52e-09	0	\\
2.63e-09	0	\\
2.73e-09	0	\\
2.83e-09	0	\\
2.93e-09	0	\\
3.04e-09	0	\\
3.14e-09	0	\\
3.24e-09	0	\\
3.34e-09	0	\\
3.45e-09	0	\\
3.55e-09	0	\\
3.65e-09	0	\\
3.75e-09	0	\\
3.86e-09	0	\\
3.96e-09	0	\\
4.06e-09	0	\\
4.16e-09	0	\\
4.27e-09	0	\\
4.37e-09	0	\\
4.47e-09	0	\\
4.57e-09	0	\\
4.68e-09	0	\\
4.78e-09	0	\\
4.89e-09	0	\\
4.99e-09	0	\\
5e-09	0	\\
};
\addplot [color=blue,solid,forget plot]
  table[row sep=crcr]{
0	0	\\
1.1e-10	0	\\
2.2e-10	0	\\
3.3e-10	0	\\
4.4e-10	0	\\
5.4e-10	0	\\
6.5e-10	0	\\
7.5e-10	0	\\
8.6e-10	0	\\
9.6e-10	0	\\
1.07e-09	0	\\
1.18e-09	0	\\
1.28e-09	0	\\
1.38e-09	0	\\
1.49e-09	0	\\
1.59e-09	0	\\
1.69e-09	0	\\
1.8e-09	0	\\
1.9e-09	0	\\
2.01e-09	0	\\
2.11e-09	0	\\
2.21e-09	0	\\
2.32e-09	0	\\
2.42e-09	0	\\
2.52e-09	0	\\
2.63e-09	0	\\
2.73e-09	0	\\
2.83e-09	0	\\
2.93e-09	0	\\
3.04e-09	0	\\
3.14e-09	0	\\
3.24e-09	0	\\
3.34e-09	0	\\
3.45e-09	0	\\
3.55e-09	0	\\
3.65e-09	0	\\
3.75e-09	0	\\
3.86e-09	0	\\
3.96e-09	0	\\
4.06e-09	0	\\
4.16e-09	0	\\
4.27e-09	0	\\
4.37e-09	0	\\
4.47e-09	0	\\
4.57e-09	0	\\
4.68e-09	0	\\
4.78e-09	0	\\
4.89e-09	0	\\
4.99e-09	0	\\
5e-09	0	\\
};
\addplot [color=black!50!green,solid,forget plot]
  table[row sep=crcr]{
0	0	\\
1.1e-10	0	\\
2.2e-10	0	\\
3.3e-10	0	\\
4.4e-10	0	\\
5.4e-10	0	\\
6.5e-10	0	\\
7.5e-10	0	\\
8.6e-10	0	\\
9.6e-10	0	\\
1.07e-09	0	\\
1.18e-09	0	\\
1.28e-09	0	\\
1.38e-09	0	\\
1.49e-09	0	\\
1.59e-09	0	\\
1.69e-09	0	\\
1.8e-09	0	\\
1.9e-09	0	\\
2.01e-09	0	\\
2.11e-09	0	\\
2.21e-09	0	\\
2.32e-09	0	\\
2.42e-09	0	\\
2.52e-09	0	\\
2.63e-09	0	\\
2.73e-09	0	\\
2.83e-09	0	\\
2.93e-09	0	\\
3.04e-09	0	\\
3.14e-09	0	\\
3.24e-09	0	\\
3.34e-09	0	\\
3.45e-09	0	\\
3.55e-09	0	\\
3.65e-09	0	\\
3.75e-09	0	\\
3.86e-09	0	\\
3.96e-09	0	\\
4.06e-09	0	\\
4.16e-09	0	\\
4.27e-09	0	\\
4.37e-09	0	\\
4.47e-09	0	\\
4.57e-09	0	\\
4.68e-09	0	\\
4.78e-09	0	\\
4.89e-09	0	\\
4.99e-09	0	\\
5e-09	0	\\
};
\addplot [color=red,solid,forget plot]
  table[row sep=crcr]{
0	0	\\
1.1e-10	0	\\
2.2e-10	0	\\
3.3e-10	0	\\
4.4e-10	0	\\
5.4e-10	0	\\
6.5e-10	0	\\
7.5e-10	0	\\
8.6e-10	0	\\
9.6e-10	0	\\
1.07e-09	0	\\
1.18e-09	0	\\
1.28e-09	0	\\
1.38e-09	0	\\
1.49e-09	0	\\
1.59e-09	0	\\
1.69e-09	0	\\
1.8e-09	0	\\
1.9e-09	0	\\
2.01e-09	0	\\
2.11e-09	0	\\
2.21e-09	0	\\
2.32e-09	0	\\
2.42e-09	0	\\
2.52e-09	0	\\
2.63e-09	0	\\
2.73e-09	0	\\
2.83e-09	0	\\
2.93e-09	0	\\
3.04e-09	0	\\
3.14e-09	0	\\
3.24e-09	0	\\
3.34e-09	0	\\
3.45e-09	0	\\
3.55e-09	0	\\
3.65e-09	0	\\
3.75e-09	0	\\
3.86e-09	0	\\
3.96e-09	0	\\
4.06e-09	0	\\
4.16e-09	0	\\
4.27e-09	0	\\
4.37e-09	0	\\
4.47e-09	0	\\
4.57e-09	0	\\
4.68e-09	0	\\
4.78e-09	0	\\
4.89e-09	0	\\
4.99e-09	0	\\
5e-09	0	\\
};
\addplot [color=mycolor1,solid,forget plot]
  table[row sep=crcr]{
0	0	\\
1.1e-10	0	\\
2.2e-10	0	\\
3.3e-10	0	\\
4.4e-10	0	\\
5.4e-10	0	\\
6.5e-10	0	\\
7.5e-10	0	\\
8.6e-10	0	\\
9.6e-10	0	\\
1.07e-09	0	\\
1.18e-09	0	\\
1.28e-09	0	\\
1.38e-09	0	\\
1.49e-09	0	\\
1.59e-09	0	\\
1.69e-09	0	\\
1.8e-09	0	\\
1.9e-09	0	\\
2.01e-09	0	\\
2.11e-09	0	\\
2.21e-09	0	\\
2.32e-09	0	\\
2.42e-09	0	\\
2.52e-09	0	\\
2.63e-09	0	\\
2.73e-09	0	\\
2.83e-09	0	\\
2.93e-09	0	\\
3.04e-09	0	\\
3.14e-09	0	\\
3.24e-09	0	\\
3.34e-09	0	\\
3.45e-09	0	\\
3.55e-09	0	\\
3.65e-09	0	\\
3.75e-09	0	\\
3.86e-09	0	\\
3.96e-09	0	\\
4.06e-09	0	\\
4.16e-09	0	\\
4.27e-09	0	\\
4.37e-09	0	\\
4.47e-09	0	\\
4.57e-09	0	\\
4.68e-09	0	\\
4.78e-09	0	\\
4.89e-09	0	\\
4.99e-09	0	\\
5e-09	0	\\
};
\addplot [color=mycolor2,solid,forget plot]
  table[row sep=crcr]{
0	0	\\
1.1e-10	0	\\
2.2e-10	0	\\
3.3e-10	0	\\
4.4e-10	0	\\
5.4e-10	0	\\
6.5e-10	0	\\
7.5e-10	0	\\
8.6e-10	0	\\
9.6e-10	0	\\
1.07e-09	0	\\
1.18e-09	0	\\
1.28e-09	0	\\
1.38e-09	0	\\
1.49e-09	0	\\
1.59e-09	0	\\
1.69e-09	0	\\
1.8e-09	0	\\
1.9e-09	0	\\
2.01e-09	0	\\
2.11e-09	0	\\
2.21e-09	0	\\
2.32e-09	0	\\
2.42e-09	0	\\
2.52e-09	0	\\
2.63e-09	0	\\
2.73e-09	0	\\
2.83e-09	0	\\
2.93e-09	0	\\
3.04e-09	0	\\
3.14e-09	0	\\
3.24e-09	0	\\
3.34e-09	0	\\
3.45e-09	0	\\
3.55e-09	0	\\
3.65e-09	0	\\
3.75e-09	0	\\
3.86e-09	0	\\
3.96e-09	0	\\
4.06e-09	0	\\
4.16e-09	0	\\
4.27e-09	0	\\
4.37e-09	0	\\
4.47e-09	0	\\
4.57e-09	0	\\
4.68e-09	0	\\
4.78e-09	0	\\
4.89e-09	0	\\
4.99e-09	0	\\
5e-09	0	\\
};
\addplot [color=mycolor3,solid,forget plot]
  table[row sep=crcr]{
0	0	\\
1.1e-10	0	\\
2.2e-10	0	\\
3.3e-10	0	\\
4.4e-10	0	\\
5.4e-10	0	\\
6.5e-10	0	\\
7.5e-10	0	\\
8.6e-10	0	\\
9.6e-10	0	\\
1.07e-09	0	\\
1.18e-09	0	\\
1.28e-09	0	\\
1.38e-09	0	\\
1.49e-09	0	\\
1.59e-09	0	\\
1.69e-09	0	\\
1.8e-09	0	\\
1.9e-09	0	\\
2.01e-09	0	\\
2.11e-09	0	\\
2.21e-09	0	\\
2.32e-09	0	\\
2.42e-09	0	\\
2.52e-09	0	\\
2.63e-09	0	\\
2.73e-09	0	\\
2.83e-09	0	\\
2.93e-09	0	\\
3.04e-09	0	\\
3.14e-09	0	\\
3.24e-09	0	\\
3.34e-09	0	\\
3.45e-09	0	\\
3.55e-09	0	\\
3.65e-09	0	\\
3.75e-09	0	\\
3.86e-09	0	\\
3.96e-09	0	\\
4.06e-09	0	\\
4.16e-09	0	\\
4.27e-09	0	\\
4.37e-09	0	\\
4.47e-09	0	\\
4.57e-09	0	\\
4.68e-09	0	\\
4.78e-09	0	\\
4.89e-09	0	\\
4.99e-09	0	\\
5e-09	0	\\
};
\addplot [color=darkgray,solid,forget plot]
  table[row sep=crcr]{
0	0	\\
1.1e-10	0	\\
2.2e-10	0	\\
3.3e-10	0	\\
4.4e-10	0	\\
5.4e-10	0	\\
6.5e-10	0	\\
7.5e-10	0	\\
8.6e-10	0	\\
9.6e-10	0	\\
1.07e-09	0	\\
1.18e-09	0	\\
1.28e-09	0	\\
1.38e-09	0	\\
1.49e-09	0	\\
1.59e-09	0	\\
1.69e-09	0	\\
1.8e-09	0	\\
1.9e-09	0	\\
2.01e-09	0	\\
2.11e-09	0	\\
2.21e-09	0	\\
2.32e-09	0	\\
2.42e-09	0	\\
2.52e-09	0	\\
2.63e-09	0	\\
2.73e-09	0	\\
2.83e-09	0	\\
2.93e-09	0	\\
3.04e-09	0	\\
3.14e-09	0	\\
3.24e-09	0	\\
3.34e-09	0	\\
3.45e-09	0	\\
3.55e-09	0	\\
3.65e-09	0	\\
3.75e-09	0	\\
3.86e-09	0	\\
3.96e-09	0	\\
4.06e-09	0	\\
4.16e-09	0	\\
4.27e-09	0	\\
4.37e-09	0	\\
4.47e-09	0	\\
4.57e-09	0	\\
4.68e-09	0	\\
4.78e-09	0	\\
4.89e-09	0	\\
4.99e-09	0	\\
5e-09	0	\\
};
\addplot [color=blue,solid,forget plot]
  table[row sep=crcr]{
0	0	\\
1.1e-10	0	\\
2.2e-10	0	\\
3.3e-10	0	\\
4.4e-10	0	\\
5.4e-10	0	\\
6.5e-10	0	\\
7.5e-10	0	\\
8.6e-10	0	\\
9.6e-10	0	\\
1.07e-09	0	\\
1.18e-09	0	\\
1.28e-09	0	\\
1.38e-09	0	\\
1.49e-09	0	\\
1.59e-09	0	\\
1.69e-09	0	\\
1.8e-09	0	\\
1.9e-09	0	\\
2.01e-09	0	\\
2.11e-09	0	\\
2.21e-09	0	\\
2.32e-09	0	\\
2.42e-09	0	\\
2.52e-09	0	\\
2.63e-09	0	\\
2.73e-09	0	\\
2.83e-09	0	\\
2.93e-09	0	\\
3.04e-09	0	\\
3.14e-09	0	\\
3.24e-09	0	\\
3.34e-09	0	\\
3.45e-09	0	\\
3.55e-09	0	\\
3.65e-09	0	\\
3.75e-09	0	\\
3.86e-09	0	\\
3.96e-09	0	\\
4.06e-09	0	\\
4.16e-09	0	\\
4.27e-09	0	\\
4.37e-09	0	\\
4.47e-09	0	\\
4.57e-09	0	\\
4.68e-09	0	\\
4.78e-09	0	\\
4.89e-09	0	\\
4.99e-09	0	\\
5e-09	0	\\
};
\addplot [color=black!50!green,solid,forget plot]
  table[row sep=crcr]{
0	0	\\
1.1e-10	0	\\
2.2e-10	0	\\
3.3e-10	0	\\
4.4e-10	0	\\
5.4e-10	0	\\
6.5e-10	0	\\
7.5e-10	0	\\
8.6e-10	0	\\
9.6e-10	0	\\
1.07e-09	0	\\
1.18e-09	0	\\
1.28e-09	0	\\
1.38e-09	0	\\
1.49e-09	0	\\
1.59e-09	0	\\
1.69e-09	0	\\
1.8e-09	0	\\
1.9e-09	0	\\
2.01e-09	0	\\
2.11e-09	0	\\
2.21e-09	0	\\
2.32e-09	0	\\
2.42e-09	0	\\
2.52e-09	0	\\
2.63e-09	0	\\
2.73e-09	0	\\
2.83e-09	0	\\
2.93e-09	0	\\
3.04e-09	0	\\
3.14e-09	0	\\
3.24e-09	0	\\
3.34e-09	0	\\
3.45e-09	0	\\
3.55e-09	0	\\
3.65e-09	0	\\
3.75e-09	0	\\
3.86e-09	0	\\
3.96e-09	0	\\
4.06e-09	0	\\
4.16e-09	0	\\
4.27e-09	0	\\
4.37e-09	0	\\
4.47e-09	0	\\
4.57e-09	0	\\
4.68e-09	0	\\
4.78e-09	0	\\
4.89e-09	0	\\
4.99e-09	0	\\
5e-09	0	\\
};
\addplot [color=red,solid,forget plot]
  table[row sep=crcr]{
0	0	\\
1.1e-10	0	\\
2.2e-10	0	\\
3.3e-10	0	\\
4.4e-10	0	\\
5.4e-10	0	\\
6.5e-10	0	\\
7.5e-10	0	\\
8.6e-10	0	\\
9.6e-10	0	\\
1.07e-09	0	\\
1.18e-09	0	\\
1.28e-09	0	\\
1.38e-09	0	\\
1.49e-09	0	\\
1.59e-09	0	\\
1.69e-09	0	\\
1.8e-09	0	\\
1.9e-09	0	\\
2.01e-09	0	\\
2.11e-09	0	\\
2.21e-09	0	\\
2.32e-09	0	\\
2.42e-09	0	\\
2.52e-09	0	\\
2.63e-09	0	\\
2.73e-09	0	\\
2.83e-09	0	\\
2.93e-09	0	\\
3.04e-09	0	\\
3.14e-09	0	\\
3.24e-09	0	\\
3.34e-09	0	\\
3.45e-09	0	\\
3.55e-09	0	\\
3.65e-09	0	\\
3.75e-09	0	\\
3.86e-09	0	\\
3.96e-09	0	\\
4.06e-09	0	\\
4.16e-09	0	\\
4.27e-09	0	\\
4.37e-09	0	\\
4.47e-09	0	\\
4.57e-09	0	\\
4.68e-09	0	\\
4.78e-09	0	\\
4.89e-09	0	\\
4.99e-09	0	\\
5e-09	0	\\
};
\addplot [color=mycolor1,solid,forget plot]
  table[row sep=crcr]{
0	0	\\
1.1e-10	0	\\
2.2e-10	0	\\
3.3e-10	0	\\
4.4e-10	0	\\
5.4e-10	0	\\
6.5e-10	0	\\
7.5e-10	0	\\
8.6e-10	0	\\
9.6e-10	0	\\
1.07e-09	0	\\
1.18e-09	0	\\
1.28e-09	0	\\
1.38e-09	0	\\
1.49e-09	0	\\
1.59e-09	0	\\
1.69e-09	0	\\
1.8e-09	0	\\
1.9e-09	0	\\
2.01e-09	0	\\
2.11e-09	0	\\
2.21e-09	0	\\
2.32e-09	0	\\
2.42e-09	0	\\
2.52e-09	0	\\
2.63e-09	0	\\
2.73e-09	0	\\
2.83e-09	0	\\
2.93e-09	0	\\
3.04e-09	0	\\
3.14e-09	0	\\
3.24e-09	0	\\
3.34e-09	0	\\
3.45e-09	0	\\
3.55e-09	0	\\
3.65e-09	0	\\
3.75e-09	0	\\
3.86e-09	0	\\
3.96e-09	0	\\
4.06e-09	0	\\
4.16e-09	0	\\
4.27e-09	0	\\
4.37e-09	0	\\
4.47e-09	0	\\
4.57e-09	0	\\
4.68e-09	0	\\
4.78e-09	0	\\
4.89e-09	0	\\
4.99e-09	0	\\
5e-09	0	\\
};
\addplot [color=mycolor2,solid,forget plot]
  table[row sep=crcr]{
0	0	\\
1.1e-10	0	\\
2.2e-10	0	\\
3.3e-10	0	\\
4.4e-10	0	\\
5.4e-10	0	\\
6.5e-10	0	\\
7.5e-10	0	\\
8.6e-10	0	\\
9.6e-10	0	\\
1.07e-09	0	\\
1.18e-09	0	\\
1.28e-09	0	\\
1.38e-09	0	\\
1.49e-09	0	\\
1.59e-09	0	\\
1.69e-09	0	\\
1.8e-09	0	\\
1.9e-09	0	\\
2.01e-09	0	\\
2.11e-09	0	\\
2.21e-09	0	\\
2.32e-09	0	\\
2.42e-09	0	\\
2.52e-09	0	\\
2.63e-09	0	\\
2.73e-09	0	\\
2.83e-09	0	\\
2.93e-09	0	\\
3.04e-09	0	\\
3.14e-09	0	\\
3.24e-09	0	\\
3.34e-09	0	\\
3.45e-09	0	\\
3.55e-09	0	\\
3.65e-09	0	\\
3.75e-09	0	\\
3.86e-09	0	\\
3.96e-09	0	\\
4.06e-09	0	\\
4.16e-09	0	\\
4.27e-09	0	\\
4.37e-09	0	\\
4.47e-09	0	\\
4.57e-09	0	\\
4.68e-09	0	\\
4.78e-09	0	\\
4.89e-09	0	\\
4.99e-09	0	\\
5e-09	0	\\
};
\addplot [color=mycolor3,solid,forget plot]
  table[row sep=crcr]{
0	0	\\
1.1e-10	0	\\
2.2e-10	0	\\
3.3e-10	0	\\
4.4e-10	0	\\
5.4e-10	0	\\
6.5e-10	0	\\
7.5e-10	0	\\
8.6e-10	0	\\
9.6e-10	0	\\
1.07e-09	0	\\
1.18e-09	0	\\
1.28e-09	0	\\
1.38e-09	0	\\
1.49e-09	0	\\
1.59e-09	0	\\
1.69e-09	0	\\
1.8e-09	0	\\
1.9e-09	0	\\
2.01e-09	0	\\
2.11e-09	0	\\
2.21e-09	0	\\
2.32e-09	0	\\
2.42e-09	0	\\
2.52e-09	0	\\
2.63e-09	0	\\
2.73e-09	0	\\
2.83e-09	0	\\
2.93e-09	0	\\
3.04e-09	0	\\
3.14e-09	0	\\
3.24e-09	0	\\
3.34e-09	0	\\
3.45e-09	0	\\
3.55e-09	0	\\
3.65e-09	0	\\
3.75e-09	0	\\
3.86e-09	0	\\
3.96e-09	0	\\
4.06e-09	0	\\
4.16e-09	0	\\
4.27e-09	0	\\
4.37e-09	0	\\
4.47e-09	0	\\
4.57e-09	0	\\
4.68e-09	0	\\
4.78e-09	0	\\
4.89e-09	0	\\
4.99e-09	0	\\
5e-09	0	\\
};
\addplot [color=darkgray,solid,forget plot]
  table[row sep=crcr]{
0	0	\\
1.1e-10	0	\\
2.2e-10	0	\\
3.3e-10	0	\\
4.4e-10	0	\\
5.4e-10	0	\\
6.5e-10	0	\\
7.5e-10	0	\\
8.6e-10	0	\\
9.6e-10	0	\\
1.07e-09	0	\\
1.18e-09	0	\\
1.28e-09	0	\\
1.38e-09	0	\\
1.49e-09	0	\\
1.59e-09	0	\\
1.69e-09	0	\\
1.8e-09	0	\\
1.9e-09	0	\\
2.01e-09	0	\\
2.11e-09	0	\\
2.21e-09	0	\\
2.32e-09	0	\\
2.42e-09	0	\\
2.52e-09	0	\\
2.63e-09	0	\\
2.73e-09	0	\\
2.83e-09	0	\\
2.93e-09	0	\\
3.04e-09	0	\\
3.14e-09	0	\\
3.24e-09	0	\\
3.34e-09	0	\\
3.45e-09	0	\\
3.55e-09	0	\\
3.65e-09	0	\\
3.75e-09	0	\\
3.86e-09	0	\\
3.96e-09	0	\\
4.06e-09	0	\\
4.16e-09	0	\\
4.27e-09	0	\\
4.37e-09	0	\\
4.47e-09	0	\\
4.57e-09	0	\\
4.68e-09	0	\\
4.78e-09	0	\\
4.89e-09	0	\\
4.99e-09	0	\\
5e-09	0	\\
};
\addplot [color=blue,solid,forget plot]
  table[row sep=crcr]{
0	0	\\
1.1e-10	0	\\
2.2e-10	0	\\
3.3e-10	0	\\
4.4e-10	0	\\
5.4e-10	0	\\
6.5e-10	0	\\
7.5e-10	0	\\
8.6e-10	0	\\
9.6e-10	0	\\
1.07e-09	0	\\
1.18e-09	0	\\
1.28e-09	0	\\
1.38e-09	0	\\
1.49e-09	0	\\
1.59e-09	0	\\
1.69e-09	0	\\
1.8e-09	0	\\
1.9e-09	0	\\
2.01e-09	0	\\
2.11e-09	0	\\
2.21e-09	0	\\
2.32e-09	0	\\
2.42e-09	0	\\
2.52e-09	0	\\
2.63e-09	0	\\
2.73e-09	0	\\
2.83e-09	0	\\
2.93e-09	0	\\
3.04e-09	0	\\
3.14e-09	0	\\
3.24e-09	0	\\
3.34e-09	0	\\
3.45e-09	0	\\
3.55e-09	0	\\
3.65e-09	0	\\
3.75e-09	0	\\
3.86e-09	0	\\
3.96e-09	0	\\
4.06e-09	0	\\
4.16e-09	0	\\
4.27e-09	0	\\
4.37e-09	0	\\
4.47e-09	0	\\
4.57e-09	0	\\
4.68e-09	0	\\
4.78e-09	0	\\
4.89e-09	0	\\
4.99e-09	0	\\
5e-09	0	\\
};
\addplot [color=black!50!green,solid,forget plot]
  table[row sep=crcr]{
0	0	\\
1.1e-10	0	\\
2.2e-10	0	\\
3.3e-10	0	\\
4.4e-10	0	\\
5.4e-10	0	\\
6.5e-10	0	\\
7.5e-10	0	\\
8.6e-10	0	\\
9.6e-10	0	\\
1.07e-09	0	\\
1.18e-09	0	\\
1.28e-09	0	\\
1.38e-09	0	\\
1.49e-09	0	\\
1.59e-09	0	\\
1.69e-09	0	\\
1.8e-09	0	\\
1.9e-09	0	\\
2.01e-09	0	\\
2.11e-09	0	\\
2.21e-09	0	\\
2.32e-09	0	\\
2.42e-09	0	\\
2.52e-09	0	\\
2.63e-09	0	\\
2.73e-09	0	\\
2.83e-09	0	\\
2.93e-09	0	\\
3.04e-09	0	\\
3.14e-09	0	\\
3.24e-09	0	\\
3.34e-09	0	\\
3.45e-09	0	\\
3.55e-09	0	\\
3.65e-09	0	\\
3.75e-09	0	\\
3.86e-09	0	\\
3.96e-09	0	\\
4.06e-09	0	\\
4.16e-09	0	\\
4.27e-09	0	\\
4.37e-09	0	\\
4.47e-09	0	\\
4.57e-09	0	\\
4.68e-09	0	\\
4.78e-09	0	\\
4.89e-09	0	\\
4.99e-09	0	\\
5e-09	0	\\
};
\addplot [color=red,solid,forget plot]
  table[row sep=crcr]{
0	0	\\
1.1e-10	0	\\
2.2e-10	0	\\
3.3e-10	0	\\
4.4e-10	0	\\
5.4e-10	0	\\
6.5e-10	0	\\
7.5e-10	0	\\
8.6e-10	0	\\
9.6e-10	0	\\
1.07e-09	0	\\
1.18e-09	0	\\
1.28e-09	0	\\
1.38e-09	0	\\
1.49e-09	0	\\
1.59e-09	0	\\
1.69e-09	0	\\
1.8e-09	0	\\
1.9e-09	0	\\
2.01e-09	0	\\
2.11e-09	0	\\
2.21e-09	0	\\
2.32e-09	0	\\
2.42e-09	0	\\
2.52e-09	0	\\
2.63e-09	0	\\
2.73e-09	0	\\
2.83e-09	0	\\
2.93e-09	0	\\
3.04e-09	0	\\
3.14e-09	0	\\
3.24e-09	0	\\
3.34e-09	0	\\
3.45e-09	0	\\
3.55e-09	0	\\
3.65e-09	0	\\
3.75e-09	0	\\
3.86e-09	0	\\
3.96e-09	0	\\
4.06e-09	0	\\
4.16e-09	0	\\
4.27e-09	0	\\
4.37e-09	0	\\
4.47e-09	0	\\
4.57e-09	0	\\
4.68e-09	0	\\
4.78e-09	0	\\
4.89e-09	0	\\
4.99e-09	0	\\
5e-09	0	\\
};
\addplot [color=mycolor1,solid,forget plot]
  table[row sep=crcr]{
0	0	\\
1.1e-10	0	\\
2.2e-10	0	\\
3.3e-10	0	\\
4.4e-10	0	\\
5.4e-10	0	\\
6.5e-10	0	\\
7.5e-10	0	\\
8.6e-10	0	\\
9.6e-10	0	\\
1.07e-09	0	\\
1.18e-09	0	\\
1.28e-09	0	\\
1.38e-09	0	\\
1.49e-09	0	\\
1.59e-09	0	\\
1.69e-09	0	\\
1.8e-09	0	\\
1.9e-09	0	\\
2.01e-09	0	\\
2.11e-09	0	\\
2.21e-09	0	\\
2.32e-09	0	\\
2.42e-09	0	\\
2.52e-09	0	\\
2.63e-09	0	\\
2.73e-09	0	\\
2.83e-09	0	\\
2.93e-09	0	\\
3.04e-09	0	\\
3.14e-09	0	\\
3.24e-09	0	\\
3.34e-09	0	\\
3.45e-09	0	\\
3.55e-09	0	\\
3.65e-09	0	\\
3.75e-09	0	\\
3.86e-09	0	\\
3.96e-09	0	\\
4.06e-09	0	\\
4.16e-09	0	\\
4.27e-09	0	\\
4.37e-09	0	\\
4.47e-09	0	\\
4.57e-09	0	\\
4.68e-09	0	\\
4.78e-09	0	\\
4.89e-09	0	\\
4.99e-09	0	\\
5e-09	0	\\
};
\addplot [color=mycolor2,solid,forget plot]
  table[row sep=crcr]{
0	0	\\
1.1e-10	0	\\
2.2e-10	0	\\
3.3e-10	0	\\
4.4e-10	0	\\
5.4e-10	0	\\
6.5e-10	0	\\
7.5e-10	0	\\
8.6e-10	0	\\
9.6e-10	0	\\
1.07e-09	0	\\
1.18e-09	0	\\
1.28e-09	0	\\
1.38e-09	0	\\
1.49e-09	0	\\
1.59e-09	0	\\
1.69e-09	0	\\
1.8e-09	0	\\
1.9e-09	0	\\
2.01e-09	0	\\
2.11e-09	0	\\
2.21e-09	0	\\
2.32e-09	0	\\
2.42e-09	0	\\
2.52e-09	0	\\
2.63e-09	0	\\
2.73e-09	0	\\
2.83e-09	0	\\
2.93e-09	0	\\
3.04e-09	0	\\
3.14e-09	0	\\
3.24e-09	0	\\
3.34e-09	0	\\
3.45e-09	0	\\
3.55e-09	0	\\
3.65e-09	0	\\
3.75e-09	0	\\
3.86e-09	0	\\
3.96e-09	0	\\
4.06e-09	0	\\
4.16e-09	0	\\
4.27e-09	0	\\
4.37e-09	0	\\
4.47e-09	0	\\
4.57e-09	0	\\
4.68e-09	0	\\
4.78e-09	0	\\
4.89e-09	0	\\
4.99e-09	0	\\
5e-09	0	\\
};
\addplot [color=mycolor3,solid,forget plot]
  table[row sep=crcr]{
0	0	\\
1.1e-10	0	\\
2.2e-10	0	\\
3.3e-10	0	\\
4.4e-10	0	\\
5.4e-10	0	\\
6.5e-10	0	\\
7.5e-10	0	\\
8.6e-10	0	\\
9.6e-10	0	\\
1.07e-09	0	\\
1.18e-09	0	\\
1.28e-09	0	\\
1.38e-09	0	\\
1.49e-09	0	\\
1.59e-09	0	\\
1.69e-09	0	\\
1.8e-09	0	\\
1.9e-09	0	\\
2.01e-09	0	\\
2.11e-09	0	\\
2.21e-09	0	\\
2.32e-09	0	\\
2.42e-09	0	\\
2.52e-09	0	\\
2.63e-09	0	\\
2.73e-09	0	\\
2.83e-09	0	\\
2.93e-09	0	\\
3.04e-09	0	\\
3.14e-09	0	\\
3.24e-09	0	\\
3.34e-09	0	\\
3.45e-09	0	\\
3.55e-09	0	\\
3.65e-09	0	\\
3.75e-09	0	\\
3.86e-09	0	\\
3.96e-09	0	\\
4.06e-09	0	\\
4.16e-09	0	\\
4.27e-09	0	\\
4.37e-09	0	\\
4.47e-09	0	\\
4.57e-09	0	\\
4.68e-09	0	\\
4.78e-09	0	\\
4.89e-09	0	\\
4.99e-09	0	\\
5e-09	0	\\
};
\addplot [color=darkgray,solid,forget plot]
  table[row sep=crcr]{
0	0	\\
1.1e-10	0	\\
2.2e-10	0	\\
3.3e-10	0	\\
4.4e-10	0	\\
5.4e-10	0	\\
6.5e-10	0	\\
7.5e-10	0	\\
8.6e-10	0	\\
9.6e-10	0	\\
1.07e-09	0	\\
1.18e-09	0	\\
1.28e-09	0	\\
1.38e-09	0	\\
1.49e-09	0	\\
1.59e-09	0	\\
1.69e-09	0	\\
1.8e-09	0	\\
1.9e-09	0	\\
2.01e-09	0	\\
2.11e-09	0	\\
2.21e-09	0	\\
2.32e-09	0	\\
2.42e-09	0	\\
2.52e-09	0	\\
2.63e-09	0	\\
2.73e-09	0	\\
2.83e-09	0	\\
2.93e-09	0	\\
3.04e-09	0	\\
3.14e-09	0	\\
3.24e-09	0	\\
3.34e-09	0	\\
3.45e-09	0	\\
3.55e-09	0	\\
3.65e-09	0	\\
3.75e-09	0	\\
3.86e-09	0	\\
3.96e-09	0	\\
4.06e-09	0	\\
4.16e-09	0	\\
4.27e-09	0	\\
4.37e-09	0	\\
4.47e-09	0	\\
4.57e-09	0	\\
4.68e-09	0	\\
4.78e-09	0	\\
4.89e-09	0	\\
4.99e-09	0	\\
5e-09	0	\\
};
\addplot [color=blue,solid,forget plot]
  table[row sep=crcr]{
0	0	\\
1.1e-10	0	\\
2.2e-10	0	\\
3.3e-10	0	\\
4.4e-10	0	\\
5.4e-10	0	\\
6.5e-10	0	\\
7.5e-10	0	\\
8.6e-10	0	\\
9.6e-10	0	\\
1.07e-09	0	\\
1.18e-09	0	\\
1.28e-09	0	\\
1.38e-09	0	\\
1.49e-09	0	\\
1.59e-09	0	\\
1.69e-09	0	\\
1.8e-09	0	\\
1.9e-09	0	\\
2.01e-09	0	\\
2.11e-09	0	\\
2.21e-09	0	\\
2.32e-09	0	\\
2.42e-09	0	\\
2.52e-09	0	\\
2.63e-09	0	\\
2.73e-09	0	\\
2.83e-09	0	\\
2.93e-09	0	\\
3.04e-09	0	\\
3.14e-09	0	\\
3.24e-09	0	\\
3.34e-09	0	\\
3.45e-09	0	\\
3.55e-09	0	\\
3.65e-09	0	\\
3.75e-09	0	\\
3.86e-09	0	\\
3.96e-09	0	\\
4.06e-09	0	\\
4.16e-09	0	\\
4.27e-09	0	\\
4.37e-09	0	\\
4.47e-09	0	\\
4.57e-09	0	\\
4.68e-09	0	\\
4.78e-09	0	\\
4.89e-09	0	\\
4.99e-09	0	\\
5e-09	0	\\
};
\addplot [color=black!50!green,solid,forget plot]
  table[row sep=crcr]{
0	0	\\
1.1e-10	0	\\
2.2e-10	0	\\
3.3e-10	0	\\
4.4e-10	0	\\
5.4e-10	0	\\
6.5e-10	0	\\
7.5e-10	0	\\
8.6e-10	0	\\
9.6e-10	0	\\
1.07e-09	0	\\
1.18e-09	0	\\
1.28e-09	0	\\
1.38e-09	0	\\
1.49e-09	0	\\
1.59e-09	0	\\
1.69e-09	0	\\
1.8e-09	0	\\
1.9e-09	0	\\
2.01e-09	0	\\
2.11e-09	0	\\
2.21e-09	0	\\
2.32e-09	0	\\
2.42e-09	0	\\
2.52e-09	0	\\
2.63e-09	0	\\
2.73e-09	0	\\
2.83e-09	0	\\
2.93e-09	0	\\
3.04e-09	0	\\
3.14e-09	0	\\
3.24e-09	0	\\
3.34e-09	0	\\
3.45e-09	0	\\
3.55e-09	0	\\
3.65e-09	0	\\
3.75e-09	0	\\
3.86e-09	0	\\
3.96e-09	0	\\
4.06e-09	0	\\
4.16e-09	0	\\
4.27e-09	0	\\
4.37e-09	0	\\
4.47e-09	0	\\
4.57e-09	0	\\
4.68e-09	0	\\
4.78e-09	0	\\
4.89e-09	0	\\
4.99e-09	0	\\
5e-09	0	\\
};
\addplot [color=red,solid,forget plot]
  table[row sep=crcr]{
0	0	\\
1.1e-10	0	\\
2.2e-10	0	\\
3.3e-10	0	\\
4.4e-10	0	\\
5.4e-10	0	\\
6.5e-10	0	\\
7.5e-10	0	\\
8.6e-10	0	\\
9.6e-10	0	\\
1.07e-09	0	\\
1.18e-09	0	\\
1.28e-09	0	\\
1.38e-09	0	\\
1.49e-09	0	\\
1.59e-09	0	\\
1.69e-09	0	\\
1.8e-09	0	\\
1.9e-09	0	\\
2.01e-09	0	\\
2.11e-09	0	\\
2.21e-09	0	\\
2.32e-09	0	\\
2.42e-09	0	\\
2.52e-09	0	\\
2.63e-09	0	\\
2.73e-09	0	\\
2.83e-09	0	\\
2.93e-09	0	\\
3.04e-09	0	\\
3.14e-09	0	\\
3.24e-09	0	\\
3.34e-09	0	\\
3.45e-09	0	\\
3.55e-09	0	\\
3.65e-09	0	\\
3.75e-09	0	\\
3.86e-09	0	\\
3.96e-09	0	\\
4.06e-09	0	\\
4.16e-09	0	\\
4.27e-09	0	\\
4.37e-09	0	\\
4.47e-09	0	\\
4.57e-09	0	\\
4.68e-09	0	\\
4.78e-09	0	\\
4.89e-09	0	\\
4.99e-09	0	\\
5e-09	0	\\
};
\addplot [color=mycolor1,solid,forget plot]
  table[row sep=crcr]{
0	0	\\
1.1e-10	0	\\
2.2e-10	0	\\
3.3e-10	0	\\
4.4e-10	0	\\
5.4e-10	0	\\
6.5e-10	0	\\
7.5e-10	0	\\
8.6e-10	0	\\
9.6e-10	0	\\
1.07e-09	0	\\
1.18e-09	0	\\
1.28e-09	0	\\
1.38e-09	0	\\
1.49e-09	0	\\
1.59e-09	0	\\
1.69e-09	0	\\
1.8e-09	0	\\
1.9e-09	0	\\
2.01e-09	0	\\
2.11e-09	0	\\
2.21e-09	0	\\
2.32e-09	0	\\
2.42e-09	0	\\
2.52e-09	0	\\
2.63e-09	0	\\
2.73e-09	0	\\
2.83e-09	0	\\
2.93e-09	0	\\
3.04e-09	0	\\
3.14e-09	0	\\
3.24e-09	0	\\
3.34e-09	0	\\
3.45e-09	0	\\
3.55e-09	0	\\
3.65e-09	0	\\
3.75e-09	0	\\
3.86e-09	0	\\
3.96e-09	0	\\
4.06e-09	0	\\
4.16e-09	0	\\
4.27e-09	0	\\
4.37e-09	0	\\
4.47e-09	0	\\
4.57e-09	0	\\
4.68e-09	0	\\
4.78e-09	0	\\
4.89e-09	0	\\
4.99e-09	0	\\
5e-09	0	\\
};
\addplot [color=mycolor2,solid,forget plot]
  table[row sep=crcr]{
0	0	\\
1.1e-10	0	\\
2.2e-10	0	\\
3.3e-10	0	\\
4.4e-10	0	\\
5.4e-10	0	\\
6.5e-10	0	\\
7.5e-10	0	\\
8.6e-10	0	\\
9.6e-10	0	\\
1.07e-09	0	\\
1.18e-09	0	\\
1.28e-09	0	\\
1.38e-09	0	\\
1.49e-09	0	\\
1.59e-09	0	\\
1.69e-09	0	\\
1.8e-09	0	\\
1.9e-09	0	\\
2.01e-09	0	\\
2.11e-09	0	\\
2.21e-09	0	\\
2.32e-09	0	\\
2.42e-09	0	\\
2.52e-09	0	\\
2.63e-09	0	\\
2.73e-09	0	\\
2.83e-09	0	\\
2.93e-09	0	\\
3.04e-09	0	\\
3.14e-09	0	\\
3.24e-09	0	\\
3.34e-09	0	\\
3.45e-09	0	\\
3.55e-09	0	\\
3.65e-09	0	\\
3.75e-09	0	\\
3.86e-09	0	\\
3.96e-09	0	\\
4.06e-09	0	\\
4.16e-09	0	\\
4.27e-09	0	\\
4.37e-09	0	\\
4.47e-09	0	\\
4.57e-09	0	\\
4.68e-09	0	\\
4.78e-09	0	\\
4.89e-09	0	\\
4.99e-09	0	\\
5e-09	0	\\
};
\addplot [color=mycolor3,solid,forget plot]
  table[row sep=crcr]{
0	0	\\
1.1e-10	0	\\
2.2e-10	0	\\
3.3e-10	0	\\
4.4e-10	0	\\
5.4e-10	0	\\
6.5e-10	0	\\
7.5e-10	0	\\
8.6e-10	0	\\
9.6e-10	0	\\
1.07e-09	0	\\
1.18e-09	0	\\
1.28e-09	0	\\
1.38e-09	0	\\
1.49e-09	0	\\
1.59e-09	0	\\
1.69e-09	0	\\
1.8e-09	0	\\
1.9e-09	0	\\
2.01e-09	0	\\
2.11e-09	0	\\
2.21e-09	0	\\
2.32e-09	0	\\
2.42e-09	0	\\
2.52e-09	0	\\
2.63e-09	0	\\
2.73e-09	0	\\
2.83e-09	0	\\
2.93e-09	0	\\
3.04e-09	0	\\
3.14e-09	0	\\
3.24e-09	0	\\
3.34e-09	0	\\
3.45e-09	0	\\
3.55e-09	0	\\
3.65e-09	0	\\
3.75e-09	0	\\
3.86e-09	0	\\
3.96e-09	0	\\
4.06e-09	0	\\
4.16e-09	0	\\
4.27e-09	0	\\
4.37e-09	0	\\
4.47e-09	0	\\
4.57e-09	0	\\
4.68e-09	0	\\
4.78e-09	0	\\
4.89e-09	0	\\
4.99e-09	0	\\
5e-09	0	\\
};
\addplot [color=darkgray,solid,forget plot]
  table[row sep=crcr]{
0	0	\\
1.1e-10	0	\\
2.2e-10	0	\\
3.3e-10	0	\\
4.4e-10	0	\\
5.4e-10	0	\\
6.5e-10	0	\\
7.5e-10	0	\\
8.6e-10	0	\\
9.6e-10	0	\\
1.07e-09	0	\\
1.18e-09	0	\\
1.28e-09	0	\\
1.38e-09	0	\\
1.49e-09	0	\\
1.59e-09	0	\\
1.69e-09	0	\\
1.8e-09	0	\\
1.9e-09	0	\\
2.01e-09	0	\\
2.11e-09	0	\\
2.21e-09	0	\\
2.32e-09	0	\\
2.42e-09	0	\\
2.52e-09	0	\\
2.63e-09	0	\\
2.73e-09	0	\\
2.83e-09	0	\\
2.93e-09	0	\\
3.04e-09	0	\\
3.14e-09	0	\\
3.24e-09	0	\\
3.34e-09	0	\\
3.45e-09	0	\\
3.55e-09	0	\\
3.65e-09	0	\\
3.75e-09	0	\\
3.86e-09	0	\\
3.96e-09	0	\\
4.06e-09	0	\\
4.16e-09	0	\\
4.27e-09	0	\\
4.37e-09	0	\\
4.47e-09	0	\\
4.57e-09	0	\\
4.68e-09	0	\\
4.78e-09	0	\\
4.89e-09	0	\\
4.99e-09	0	\\
5e-09	0	\\
};
\addplot [color=blue,solid,forget plot]
  table[row sep=crcr]{
0	0	\\
1.1e-10	0	\\
2.2e-10	0	\\
3.3e-10	0	\\
4.4e-10	0	\\
5.4e-10	0	\\
6.5e-10	0	\\
7.5e-10	0	\\
8.6e-10	0	\\
9.6e-10	0	\\
1.07e-09	0	\\
1.18e-09	0	\\
1.28e-09	0	\\
1.38e-09	0	\\
1.49e-09	0	\\
1.59e-09	0	\\
1.69e-09	0	\\
1.8e-09	0	\\
1.9e-09	0	\\
2.01e-09	0	\\
2.11e-09	0	\\
2.21e-09	0	\\
2.32e-09	0	\\
2.42e-09	0	\\
2.52e-09	0	\\
2.63e-09	0	\\
2.73e-09	0	\\
2.83e-09	0	\\
2.93e-09	0	\\
3.04e-09	0	\\
3.14e-09	0	\\
3.24e-09	0	\\
3.34e-09	0	\\
3.45e-09	0	\\
3.55e-09	0	\\
3.65e-09	0	\\
3.75e-09	0	\\
3.86e-09	0	\\
3.96e-09	0	\\
4.06e-09	0	\\
4.16e-09	0	\\
4.27e-09	0	\\
4.37e-09	0	\\
4.47e-09	0	\\
4.57e-09	0	\\
4.68e-09	0	\\
4.78e-09	0	\\
4.89e-09	0	\\
4.99e-09	0	\\
5e-09	0	\\
};
\addplot [color=black!50!green,solid,forget plot]
  table[row sep=crcr]{
0	0	\\
1.1e-10	0	\\
2.2e-10	0	\\
3.3e-10	0	\\
4.4e-10	0	\\
5.4e-10	0	\\
6.5e-10	0	\\
7.5e-10	0	\\
8.6e-10	0	\\
9.6e-10	0	\\
1.07e-09	0	\\
1.18e-09	0	\\
1.28e-09	0	\\
1.38e-09	0	\\
1.49e-09	0	\\
1.59e-09	0	\\
1.69e-09	0	\\
1.8e-09	0	\\
1.9e-09	0	\\
2.01e-09	0	\\
2.11e-09	0	\\
2.21e-09	0	\\
2.32e-09	0	\\
2.42e-09	0	\\
2.52e-09	0	\\
2.63e-09	0	\\
2.73e-09	0	\\
2.83e-09	0	\\
2.93e-09	0	\\
3.04e-09	0	\\
3.14e-09	0	\\
3.24e-09	0	\\
3.34e-09	0	\\
3.45e-09	0	\\
3.55e-09	0	\\
3.65e-09	0	\\
3.75e-09	0	\\
3.86e-09	0	\\
3.96e-09	0	\\
4.06e-09	0	\\
4.16e-09	0	\\
4.27e-09	0	\\
4.37e-09	0	\\
4.47e-09	0	\\
4.57e-09	0	\\
4.68e-09	0	\\
4.78e-09	0	\\
4.89e-09	0	\\
4.99e-09	0	\\
5e-09	0	\\
};
\addplot [color=red,solid,forget plot]
  table[row sep=crcr]{
0	0	\\
1.1e-10	0	\\
2.2e-10	0	\\
3.3e-10	0	\\
4.4e-10	0	\\
5.4e-10	0	\\
6.5e-10	0	\\
7.5e-10	0	\\
8.6e-10	0	\\
9.6e-10	0	\\
1.07e-09	0	\\
1.18e-09	0	\\
1.28e-09	0	\\
1.38e-09	0	\\
1.49e-09	0	\\
1.59e-09	0	\\
1.69e-09	0	\\
1.8e-09	0	\\
1.9e-09	0	\\
2.01e-09	0	\\
2.11e-09	0	\\
2.21e-09	0	\\
2.32e-09	0	\\
2.42e-09	0	\\
2.52e-09	0	\\
2.63e-09	0	\\
2.73e-09	0	\\
2.83e-09	0	\\
2.93e-09	0	\\
3.04e-09	0	\\
3.14e-09	0	\\
3.24e-09	0	\\
3.34e-09	0	\\
3.45e-09	0	\\
3.55e-09	0	\\
3.65e-09	0	\\
3.75e-09	0	\\
3.86e-09	0	\\
3.96e-09	0	\\
4.06e-09	0	\\
4.16e-09	0	\\
4.27e-09	0	\\
4.37e-09	0	\\
4.47e-09	0	\\
4.57e-09	0	\\
4.68e-09	0	\\
4.78e-09	0	\\
4.89e-09	0	\\
4.99e-09	0	\\
5e-09	0	\\
};
\addplot [color=mycolor1,solid,forget plot]
  table[row sep=crcr]{
0	0	\\
1.1e-10	0	\\
2.2e-10	0	\\
3.3e-10	0	\\
4.4e-10	0	\\
5.4e-10	0	\\
6.5e-10	0	\\
7.5e-10	0	\\
8.6e-10	0	\\
9.6e-10	0	\\
1.07e-09	0	\\
1.18e-09	0	\\
1.28e-09	0	\\
1.38e-09	0	\\
1.49e-09	0	\\
1.59e-09	0	\\
1.69e-09	0	\\
1.8e-09	0	\\
1.9e-09	0	\\
2.01e-09	0	\\
2.11e-09	0	\\
2.21e-09	0	\\
2.32e-09	0	\\
2.42e-09	0	\\
2.52e-09	0	\\
2.63e-09	0	\\
2.73e-09	0	\\
2.83e-09	0	\\
2.93e-09	0	\\
3.04e-09	0	\\
3.14e-09	0	\\
3.24e-09	0	\\
3.34e-09	0	\\
3.45e-09	0	\\
3.55e-09	0	\\
3.65e-09	0	\\
3.75e-09	0	\\
3.86e-09	0	\\
3.96e-09	0	\\
4.06e-09	0	\\
4.16e-09	0	\\
4.27e-09	0	\\
4.37e-09	0	\\
4.47e-09	0	\\
4.57e-09	0	\\
4.68e-09	0	\\
4.78e-09	0	\\
4.89e-09	0	\\
4.99e-09	0	\\
5e-09	0	\\
};
\addplot [color=mycolor2,solid,forget plot]
  table[row sep=crcr]{
0	0	\\
1.1e-10	0	\\
2.2e-10	0	\\
3.3e-10	0	\\
4.4e-10	0	\\
5.4e-10	0	\\
6.5e-10	0	\\
7.5e-10	0	\\
8.6e-10	0	\\
9.6e-10	0	\\
1.07e-09	0	\\
1.18e-09	0	\\
1.28e-09	0	\\
1.38e-09	0	\\
1.49e-09	0	\\
1.59e-09	0	\\
1.69e-09	0	\\
1.8e-09	0	\\
1.9e-09	0	\\
2.01e-09	0	\\
2.11e-09	0	\\
2.21e-09	0	\\
2.32e-09	0	\\
2.42e-09	0	\\
2.52e-09	0	\\
2.63e-09	0	\\
2.73e-09	0	\\
2.83e-09	0	\\
2.93e-09	0	\\
3.04e-09	0	\\
3.14e-09	0	\\
3.24e-09	0	\\
3.34e-09	0	\\
3.45e-09	0	\\
3.55e-09	0	\\
3.65e-09	0	\\
3.75e-09	0	\\
3.86e-09	0	\\
3.96e-09	0	\\
4.06e-09	0	\\
4.16e-09	0	\\
4.27e-09	0	\\
4.37e-09	0	\\
4.47e-09	0	\\
4.57e-09	0	\\
4.68e-09	0	\\
4.78e-09	0	\\
4.89e-09	0	\\
4.99e-09	0	\\
5e-09	0	\\
};
\addplot [color=mycolor3,solid,forget plot]
  table[row sep=crcr]{
0	0	\\
1.1e-10	0	\\
2.2e-10	0	\\
3.3e-10	0	\\
4.4e-10	0	\\
5.4e-10	0	\\
6.5e-10	0	\\
7.5e-10	0	\\
8.6e-10	0	\\
9.6e-10	0	\\
1.07e-09	0	\\
1.18e-09	0	\\
1.28e-09	0	\\
1.38e-09	0	\\
1.49e-09	0	\\
1.59e-09	0	\\
1.69e-09	0	\\
1.8e-09	0	\\
1.9e-09	0	\\
2.01e-09	0	\\
2.11e-09	0	\\
2.21e-09	0	\\
2.32e-09	0	\\
2.42e-09	0	\\
2.52e-09	0	\\
2.63e-09	0	\\
2.73e-09	0	\\
2.83e-09	0	\\
2.93e-09	0	\\
3.04e-09	0	\\
3.14e-09	0	\\
3.24e-09	0	\\
3.34e-09	0	\\
3.45e-09	0	\\
3.55e-09	0	\\
3.65e-09	0	\\
3.75e-09	0	\\
3.86e-09	0	\\
3.96e-09	0	\\
4.06e-09	0	\\
4.16e-09	0	\\
4.27e-09	0	\\
4.37e-09	0	\\
4.47e-09	0	\\
4.57e-09	0	\\
4.68e-09	0	\\
4.78e-09	0	\\
4.89e-09	0	\\
4.99e-09	0	\\
5e-09	0	\\
};
\addplot [color=darkgray,solid,forget plot]
  table[row sep=crcr]{
0	0	\\
1.1e-10	0	\\
2.2e-10	0	\\
3.3e-10	0	\\
4.4e-10	0	\\
5.4e-10	0	\\
6.5e-10	0	\\
7.5e-10	0	\\
8.6e-10	0	\\
9.6e-10	0	\\
1.07e-09	0	\\
1.18e-09	0	\\
1.28e-09	0	\\
1.38e-09	0	\\
1.49e-09	0	\\
1.59e-09	0	\\
1.69e-09	0	\\
1.8e-09	0	\\
1.9e-09	0	\\
2.01e-09	0	\\
2.11e-09	0	\\
2.21e-09	0	\\
2.32e-09	0	\\
2.42e-09	0	\\
2.52e-09	0	\\
2.63e-09	0	\\
2.73e-09	0	\\
2.83e-09	0	\\
2.93e-09	0	\\
3.04e-09	0	\\
3.14e-09	0	\\
3.24e-09	0	\\
3.34e-09	0	\\
3.45e-09	0	\\
3.55e-09	0	\\
3.65e-09	0	\\
3.75e-09	0	\\
3.86e-09	0	\\
3.96e-09	0	\\
4.06e-09	0	\\
4.16e-09	0	\\
4.27e-09	0	\\
4.37e-09	0	\\
4.47e-09	0	\\
4.57e-09	0	\\
4.68e-09	0	\\
4.78e-09	0	\\
4.89e-09	0	\\
4.99e-09	0	\\
5e-09	0	\\
};
\addplot [color=blue,solid,forget plot]
  table[row sep=crcr]{
0	0	\\
1.1e-10	0	\\
2.2e-10	0	\\
3.3e-10	0	\\
4.4e-10	0	\\
5.4e-10	0	\\
6.5e-10	0	\\
7.5e-10	0	\\
8.6e-10	0	\\
9.6e-10	0	\\
1.07e-09	0	\\
1.18e-09	0	\\
1.28e-09	0	\\
1.38e-09	0	\\
1.49e-09	0	\\
1.59e-09	0	\\
1.69e-09	0	\\
1.8e-09	0	\\
1.9e-09	0	\\
2.01e-09	0	\\
2.11e-09	0	\\
2.21e-09	0	\\
2.32e-09	0	\\
2.42e-09	0	\\
2.52e-09	0	\\
2.63e-09	0	\\
2.73e-09	0	\\
2.83e-09	0	\\
2.93e-09	0	\\
3.04e-09	0	\\
3.14e-09	0	\\
3.24e-09	0	\\
3.34e-09	0	\\
3.45e-09	0	\\
3.55e-09	0	\\
3.65e-09	0	\\
3.75e-09	0	\\
3.86e-09	0	\\
3.96e-09	0	\\
4.06e-09	0	\\
4.16e-09	0	\\
4.27e-09	0	\\
4.37e-09	0	\\
4.47e-09	0	\\
4.57e-09	0	\\
4.68e-09	0	\\
4.78e-09	0	\\
4.89e-09	0	\\
4.99e-09	0	\\
5e-09	0	\\
};
\addplot [color=black!50!green,solid,forget plot]
  table[row sep=crcr]{
0	0	\\
1.1e-10	0	\\
2.2e-10	0	\\
3.3e-10	0	\\
4.4e-10	0	\\
5.4e-10	0	\\
6.5e-10	0	\\
7.5e-10	0	\\
8.6e-10	0	\\
9.6e-10	0	\\
1.07e-09	0	\\
1.18e-09	0	\\
1.28e-09	0	\\
1.38e-09	0	\\
1.49e-09	0	\\
1.59e-09	0	\\
1.69e-09	0	\\
1.8e-09	0	\\
1.9e-09	0	\\
2.01e-09	0	\\
2.11e-09	0	\\
2.21e-09	0	\\
2.32e-09	0	\\
2.42e-09	0	\\
2.52e-09	0	\\
2.63e-09	0	\\
2.73e-09	0	\\
2.83e-09	0	\\
2.93e-09	0	\\
3.04e-09	0	\\
3.14e-09	0	\\
3.24e-09	0	\\
3.34e-09	0	\\
3.45e-09	0	\\
3.55e-09	0	\\
3.65e-09	0	\\
3.75e-09	0	\\
3.86e-09	0	\\
3.96e-09	0	\\
4.06e-09	0	\\
4.16e-09	0	\\
4.27e-09	0	\\
4.37e-09	0	\\
4.47e-09	0	\\
4.57e-09	0	\\
4.68e-09	0	\\
4.78e-09	0	\\
4.89e-09	0	\\
4.99e-09	0	\\
5e-09	0	\\
};
\addplot [color=red,solid,forget plot]
  table[row sep=crcr]{
0	0	\\
1.1e-10	0	\\
2.2e-10	0	\\
3.3e-10	0	\\
4.4e-10	0	\\
5.4e-10	0	\\
6.5e-10	0	\\
7.5e-10	0	\\
8.6e-10	0	\\
9.6e-10	0	\\
1.07e-09	0	\\
1.18e-09	0	\\
1.28e-09	0	\\
1.38e-09	0	\\
1.49e-09	0	\\
1.59e-09	0	\\
1.69e-09	0	\\
1.8e-09	0	\\
1.9e-09	0	\\
2.01e-09	0	\\
2.11e-09	0	\\
2.21e-09	0	\\
2.32e-09	0	\\
2.42e-09	0	\\
2.52e-09	0	\\
2.63e-09	0	\\
2.73e-09	0	\\
2.83e-09	0	\\
2.93e-09	0	\\
3.04e-09	0	\\
3.14e-09	0	\\
3.24e-09	0	\\
3.34e-09	0	\\
3.45e-09	0	\\
3.55e-09	0	\\
3.65e-09	0	\\
3.75e-09	0	\\
3.86e-09	0	\\
3.96e-09	0	\\
4.06e-09	0	\\
4.16e-09	0	\\
4.27e-09	0	\\
4.37e-09	0	\\
4.47e-09	0	\\
4.57e-09	0	\\
4.68e-09	0	\\
4.78e-09	0	\\
4.89e-09	0	\\
4.99e-09	0	\\
5e-09	0	\\
};
\addplot [color=mycolor1,solid,forget plot]
  table[row sep=crcr]{
0	0	\\
1.1e-10	0	\\
2.2e-10	0	\\
3.3e-10	0	\\
4.4e-10	0	\\
5.4e-10	0	\\
6.5e-10	0	\\
7.5e-10	0	\\
8.6e-10	0	\\
9.6e-10	0	\\
1.07e-09	0	\\
1.18e-09	0	\\
1.28e-09	0	\\
1.38e-09	0	\\
1.49e-09	0	\\
1.59e-09	0	\\
1.69e-09	0	\\
1.8e-09	0	\\
1.9e-09	0	\\
2.01e-09	0	\\
2.11e-09	0	\\
2.21e-09	0	\\
2.32e-09	0	\\
2.42e-09	0	\\
2.52e-09	0	\\
2.63e-09	0	\\
2.73e-09	0	\\
2.83e-09	0	\\
2.93e-09	0	\\
3.04e-09	0	\\
3.14e-09	0	\\
3.24e-09	0	\\
3.34e-09	0	\\
3.45e-09	0	\\
3.55e-09	0	\\
3.65e-09	0	\\
3.75e-09	0	\\
3.86e-09	0	\\
3.96e-09	0	\\
4.06e-09	0	\\
4.16e-09	0	\\
4.27e-09	0	\\
4.37e-09	0	\\
4.47e-09	0	\\
4.57e-09	0	\\
4.68e-09	0	\\
4.78e-09	0	\\
4.89e-09	0	\\
4.99e-09	0	\\
5e-09	0	\\
};
\addplot [color=mycolor2,solid,forget plot]
  table[row sep=crcr]{
0	0	\\
1.1e-10	0	\\
2.2e-10	0	\\
3.3e-10	0	\\
4.4e-10	0	\\
5.4e-10	0	\\
6.5e-10	0	\\
7.5e-10	0	\\
8.6e-10	0	\\
9.6e-10	0	\\
1.07e-09	0	\\
1.18e-09	0	\\
1.28e-09	0	\\
1.38e-09	0	\\
1.49e-09	0	\\
1.59e-09	0	\\
1.69e-09	0	\\
1.8e-09	0	\\
1.9e-09	0	\\
2.01e-09	0	\\
2.11e-09	0	\\
2.21e-09	0	\\
2.32e-09	0	\\
2.42e-09	0	\\
2.52e-09	0	\\
2.63e-09	0	\\
2.73e-09	0	\\
2.83e-09	0	\\
2.93e-09	0	\\
3.04e-09	0	\\
3.14e-09	0	\\
3.24e-09	0	\\
3.34e-09	0	\\
3.45e-09	0	\\
3.55e-09	0	\\
3.65e-09	0	\\
3.75e-09	0	\\
3.86e-09	0	\\
3.96e-09	0	\\
4.06e-09	0	\\
4.16e-09	0	\\
4.27e-09	0	\\
4.37e-09	0	\\
4.47e-09	0	\\
4.57e-09	0	\\
4.68e-09	0	\\
4.78e-09	0	\\
4.89e-09	0	\\
4.99e-09	0	\\
5e-09	0	\\
};
\addplot [color=mycolor3,solid,forget plot]
  table[row sep=crcr]{
0	0	\\
1.1e-10	0	\\
2.2e-10	0	\\
3.3e-10	0	\\
4.4e-10	0	\\
5.4e-10	0	\\
6.5e-10	0	\\
7.5e-10	0	\\
8.6e-10	0	\\
9.6e-10	0	\\
1.07e-09	0	\\
1.18e-09	0	\\
1.28e-09	0	\\
1.38e-09	0	\\
1.49e-09	0	\\
1.59e-09	0	\\
1.69e-09	0	\\
1.8e-09	0	\\
1.9e-09	0	\\
2.01e-09	0	\\
2.11e-09	0	\\
2.21e-09	0	\\
2.32e-09	0	\\
2.42e-09	0	\\
2.52e-09	0	\\
2.63e-09	0	\\
2.73e-09	0	\\
2.83e-09	0	\\
2.93e-09	0	\\
3.04e-09	0	\\
3.14e-09	0	\\
3.24e-09	0	\\
3.34e-09	0	\\
3.45e-09	0	\\
3.55e-09	0	\\
3.65e-09	0	\\
3.75e-09	0	\\
3.86e-09	0	\\
3.96e-09	0	\\
4.06e-09	0	\\
4.16e-09	0	\\
4.27e-09	0	\\
4.37e-09	0	\\
4.47e-09	0	\\
4.57e-09	0	\\
4.68e-09	0	\\
4.78e-09	0	\\
4.89e-09	0	\\
4.99e-09	0	\\
5e-09	0	\\
};
\addplot [color=darkgray,solid,forget plot]
  table[row sep=crcr]{
0	0	\\
1.1e-10	0	\\
2.2e-10	0	\\
3.3e-10	0	\\
4.4e-10	0	\\
5.4e-10	0	\\
6.5e-10	0	\\
7.5e-10	0	\\
8.6e-10	0	\\
9.6e-10	0	\\
1.07e-09	0	\\
1.18e-09	0	\\
1.28e-09	0	\\
1.38e-09	0	\\
1.49e-09	0	\\
1.59e-09	0	\\
1.69e-09	0	\\
1.8e-09	0	\\
1.9e-09	0	\\
2.01e-09	0	\\
2.11e-09	0	\\
2.21e-09	0	\\
2.32e-09	0	\\
2.42e-09	0	\\
2.52e-09	0	\\
2.63e-09	0	\\
2.73e-09	0	\\
2.83e-09	0	\\
2.93e-09	0	\\
3.04e-09	0	\\
3.14e-09	0	\\
3.24e-09	0	\\
3.34e-09	0	\\
3.45e-09	0	\\
3.55e-09	0	\\
3.65e-09	0	\\
3.75e-09	0	\\
3.86e-09	0	\\
3.96e-09	0	\\
4.06e-09	0	\\
4.16e-09	0	\\
4.27e-09	0	\\
4.37e-09	0	\\
4.47e-09	0	\\
4.57e-09	0	\\
4.68e-09	0	\\
4.78e-09	0	\\
4.89e-09	0	\\
4.99e-09	0	\\
5e-09	0	\\
};
\addplot [color=blue,solid,forget plot]
  table[row sep=crcr]{
0	0	\\
1.1e-10	0	\\
2.2e-10	0	\\
3.3e-10	0	\\
4.4e-10	0	\\
5.4e-10	0	\\
6.5e-10	0	\\
7.5e-10	0	\\
8.6e-10	0	\\
9.6e-10	0	\\
1.07e-09	0	\\
1.18e-09	0	\\
1.28e-09	0	\\
1.38e-09	0	\\
1.49e-09	0	\\
1.59e-09	0	\\
1.69e-09	0	\\
1.8e-09	0	\\
1.9e-09	0	\\
2.01e-09	0	\\
2.11e-09	0	\\
2.21e-09	0	\\
2.32e-09	0	\\
2.42e-09	0	\\
2.52e-09	0	\\
2.63e-09	0	\\
2.73e-09	0	\\
2.83e-09	0	\\
2.93e-09	0	\\
3.04e-09	0	\\
3.14e-09	0	\\
3.24e-09	0	\\
3.34e-09	0	\\
3.45e-09	0	\\
3.55e-09	0	\\
3.65e-09	0	\\
3.75e-09	0	\\
3.86e-09	0	\\
3.96e-09	0	\\
4.06e-09	0	\\
4.16e-09	0	\\
4.27e-09	0	\\
4.37e-09	0	\\
4.47e-09	0	\\
4.57e-09	0	\\
4.68e-09	0	\\
4.78e-09	0	\\
4.89e-09	0	\\
4.99e-09	0	\\
5e-09	0	\\
};
\addplot [color=black!50!green,solid,forget plot]
  table[row sep=crcr]{
0	0	\\
1.1e-10	0	\\
2.2e-10	0	\\
3.3e-10	0	\\
4.4e-10	0	\\
5.4e-10	0	\\
6.5e-10	0	\\
7.5e-10	0	\\
8.6e-10	0	\\
9.6e-10	0	\\
1.07e-09	0	\\
1.18e-09	0	\\
1.28e-09	0	\\
1.38e-09	0	\\
1.49e-09	0	\\
1.59e-09	0	\\
1.69e-09	0	\\
1.8e-09	0	\\
1.9e-09	0	\\
2.01e-09	0	\\
2.11e-09	0	\\
2.21e-09	0	\\
2.32e-09	0	\\
2.42e-09	0	\\
2.52e-09	0	\\
2.63e-09	0	\\
2.73e-09	0	\\
2.83e-09	0	\\
2.93e-09	0	\\
3.04e-09	0	\\
3.14e-09	0	\\
3.24e-09	0	\\
3.34e-09	0	\\
3.45e-09	0	\\
3.55e-09	0	\\
3.65e-09	0	\\
3.75e-09	0	\\
3.86e-09	0	\\
3.96e-09	0	\\
4.06e-09	0	\\
4.16e-09	0	\\
4.27e-09	0	\\
4.37e-09	0	\\
4.47e-09	0	\\
4.57e-09	0	\\
4.68e-09	0	\\
4.78e-09	0	\\
4.89e-09	0	\\
4.99e-09	0	\\
5e-09	0	\\
};
\addplot [color=red,solid,forget plot]
  table[row sep=crcr]{
0	0	\\
1.1e-10	0	\\
2.2e-10	0	\\
3.3e-10	0	\\
4.4e-10	0	\\
5.4e-10	0	\\
6.5e-10	0	\\
7.5e-10	0	\\
8.6e-10	0	\\
9.6e-10	0	\\
1.07e-09	0	\\
1.18e-09	0	\\
1.28e-09	0	\\
1.38e-09	0	\\
1.49e-09	0	\\
1.59e-09	0	\\
1.69e-09	0	\\
1.8e-09	0	\\
1.9e-09	0	\\
2.01e-09	0	\\
2.11e-09	0	\\
2.21e-09	0	\\
2.32e-09	0	\\
2.42e-09	0	\\
2.52e-09	0	\\
2.63e-09	0	\\
2.73e-09	0	\\
2.83e-09	0	\\
2.93e-09	0	\\
3.04e-09	0	\\
3.14e-09	0	\\
3.24e-09	0	\\
3.34e-09	0	\\
3.45e-09	0	\\
3.55e-09	0	\\
3.65e-09	0	\\
3.75e-09	0	\\
3.86e-09	0	\\
3.96e-09	0	\\
4.06e-09	0	\\
4.16e-09	0	\\
4.27e-09	0	\\
4.37e-09	0	\\
4.47e-09	0	\\
4.57e-09	0	\\
4.68e-09	0	\\
4.78e-09	0	\\
4.89e-09	0	\\
4.99e-09	0	\\
5e-09	0	\\
};
\addplot [color=mycolor1,solid,forget plot]
  table[row sep=crcr]{
0	0	\\
1.1e-10	0	\\
2.2e-10	0	\\
3.3e-10	0	\\
4.4e-10	0	\\
5.4e-10	0	\\
6.5e-10	0	\\
7.5e-10	0	\\
8.6e-10	0	\\
9.6e-10	0	\\
1.07e-09	0	\\
1.18e-09	0	\\
1.28e-09	0	\\
1.38e-09	0	\\
1.49e-09	0	\\
1.59e-09	0	\\
1.69e-09	0	\\
1.8e-09	0	\\
1.9e-09	0	\\
2.01e-09	0	\\
2.11e-09	0	\\
2.21e-09	0	\\
2.32e-09	0	\\
2.42e-09	0	\\
2.52e-09	0	\\
2.63e-09	0	\\
2.73e-09	0	\\
2.83e-09	0	\\
2.93e-09	0	\\
3.04e-09	0	\\
3.14e-09	0	\\
3.24e-09	0	\\
3.34e-09	0	\\
3.45e-09	0	\\
3.55e-09	0	\\
3.65e-09	0	\\
3.75e-09	0	\\
3.86e-09	0	\\
3.96e-09	0	\\
4.06e-09	0	\\
4.16e-09	0	\\
4.27e-09	0	\\
4.37e-09	0	\\
4.47e-09	0	\\
4.57e-09	0	\\
4.68e-09	0	\\
4.78e-09	0	\\
4.89e-09	0	\\
4.99e-09	0	\\
5e-09	0	\\
};
\addplot [color=mycolor2,solid,forget plot]
  table[row sep=crcr]{
0	0	\\
1.1e-10	0	\\
2.2e-10	0	\\
3.3e-10	0	\\
4.4e-10	0	\\
5.4e-10	0	\\
6.5e-10	0	\\
7.5e-10	0	\\
8.6e-10	0	\\
9.6e-10	0	\\
1.07e-09	0	\\
1.18e-09	0	\\
1.28e-09	0	\\
1.38e-09	0	\\
1.49e-09	0	\\
1.59e-09	0	\\
1.69e-09	0	\\
1.8e-09	0	\\
1.9e-09	0	\\
2.01e-09	0	\\
2.11e-09	0	\\
2.21e-09	0	\\
2.32e-09	0	\\
2.42e-09	0	\\
2.52e-09	0	\\
2.63e-09	0	\\
2.73e-09	0	\\
2.83e-09	0	\\
2.93e-09	0	\\
3.04e-09	0	\\
3.14e-09	0	\\
3.24e-09	0	\\
3.34e-09	0	\\
3.45e-09	0	\\
3.55e-09	0	\\
3.65e-09	0	\\
3.75e-09	0	\\
3.86e-09	0	\\
3.96e-09	0	\\
4.06e-09	0	\\
4.16e-09	0	\\
4.27e-09	0	\\
4.37e-09	0	\\
4.47e-09	0	\\
4.57e-09	0	\\
4.68e-09	0	\\
4.78e-09	0	\\
4.89e-09	0	\\
4.99e-09	0	\\
5e-09	0	\\
};
\addplot [color=mycolor3,solid,forget plot]
  table[row sep=crcr]{
0	0	\\
1.1e-10	0	\\
2.2e-10	0	\\
3.3e-10	0	\\
4.4e-10	0	\\
5.4e-10	0	\\
6.5e-10	0	\\
7.5e-10	0	\\
8.6e-10	0	\\
9.6e-10	0	\\
1.07e-09	0	\\
1.18e-09	0	\\
1.28e-09	0	\\
1.38e-09	0	\\
1.49e-09	0	\\
1.59e-09	0	\\
1.69e-09	0	\\
1.8e-09	0	\\
1.9e-09	0	\\
2.01e-09	0	\\
2.11e-09	0	\\
2.21e-09	0	\\
2.32e-09	0	\\
2.42e-09	0	\\
2.52e-09	0	\\
2.63e-09	0	\\
2.73e-09	0	\\
2.83e-09	0	\\
2.93e-09	0	\\
3.04e-09	0	\\
3.14e-09	0	\\
3.24e-09	0	\\
3.34e-09	0	\\
3.45e-09	0	\\
3.55e-09	0	\\
3.65e-09	0	\\
3.75e-09	0	\\
3.86e-09	0	\\
3.96e-09	0	\\
4.06e-09	0	\\
4.16e-09	0	\\
4.27e-09	0	\\
4.37e-09	0	\\
4.47e-09	0	\\
4.57e-09	0	\\
4.68e-09	0	\\
4.78e-09	0	\\
4.89e-09	0	\\
4.99e-09	0	\\
5e-09	0	\\
};
\addplot [color=darkgray,solid,forget plot]
  table[row sep=crcr]{
0	0	\\
1.1e-10	0	\\
2.2e-10	0	\\
3.3e-10	0	\\
4.4e-10	0	\\
5.4e-10	0	\\
6.5e-10	0	\\
7.5e-10	0	\\
8.6e-10	0	\\
9.6e-10	0	\\
1.07e-09	0	\\
1.18e-09	0	\\
1.28e-09	0	\\
1.38e-09	0	\\
1.49e-09	0	\\
1.59e-09	0	\\
1.69e-09	0	\\
1.8e-09	0	\\
1.9e-09	0	\\
2.01e-09	0	\\
2.11e-09	0	\\
2.21e-09	0	\\
2.32e-09	0	\\
2.42e-09	0	\\
2.52e-09	0	\\
2.63e-09	0	\\
2.73e-09	0	\\
2.83e-09	0	\\
2.93e-09	0	\\
3.04e-09	0	\\
3.14e-09	0	\\
3.24e-09	0	\\
3.34e-09	0	\\
3.45e-09	0	\\
3.55e-09	0	\\
3.65e-09	0	\\
3.75e-09	0	\\
3.86e-09	0	\\
3.96e-09	0	\\
4.06e-09	0	\\
4.16e-09	0	\\
4.27e-09	0	\\
4.37e-09	0	\\
4.47e-09	0	\\
4.57e-09	0	\\
4.68e-09	0	\\
4.78e-09	0	\\
4.89e-09	0	\\
4.99e-09	0	\\
5e-09	0	\\
};
\addplot [color=blue,solid,forget plot]
  table[row sep=crcr]{
0	0	\\
1.1e-10	0	\\
2.2e-10	0	\\
3.3e-10	0	\\
4.4e-10	0	\\
5.4e-10	0	\\
6.5e-10	0	\\
7.5e-10	0	\\
8.6e-10	0	\\
9.6e-10	0	\\
1.07e-09	0	\\
1.18e-09	0	\\
1.28e-09	0	\\
1.38e-09	0	\\
1.49e-09	0	\\
1.59e-09	0	\\
1.69e-09	0	\\
1.8e-09	0	\\
1.9e-09	0	\\
2.01e-09	0	\\
2.11e-09	0	\\
2.21e-09	0	\\
2.32e-09	0	\\
2.42e-09	0	\\
2.52e-09	0	\\
2.63e-09	0	\\
2.73e-09	0	\\
2.83e-09	0	\\
2.93e-09	0	\\
3.04e-09	0	\\
3.14e-09	0	\\
3.24e-09	0	\\
3.34e-09	0	\\
3.45e-09	0	\\
3.55e-09	0	\\
3.65e-09	0	\\
3.75e-09	0	\\
3.86e-09	0	\\
3.96e-09	0	\\
4.06e-09	0	\\
4.16e-09	0	\\
4.27e-09	0	\\
4.37e-09	0	\\
4.47e-09	0	\\
4.57e-09	0	\\
4.68e-09	0	\\
4.78e-09	0	\\
4.89e-09	0	\\
4.99e-09	0	\\
5e-09	0	\\
};
\addplot [color=black!50!green,solid,forget plot]
  table[row sep=crcr]{
0	0	\\
1.1e-10	0	\\
2.2e-10	0	\\
3.3e-10	0	\\
4.4e-10	0	\\
5.4e-10	0	\\
6.5e-10	0	\\
7.5e-10	0	\\
8.6e-10	0	\\
9.6e-10	0	\\
1.07e-09	0	\\
1.18e-09	0	\\
1.28e-09	0	\\
1.38e-09	0	\\
1.49e-09	0	\\
1.59e-09	0	\\
1.69e-09	0	\\
1.8e-09	0	\\
1.9e-09	0	\\
2.01e-09	0	\\
2.11e-09	0	\\
2.21e-09	0	\\
2.32e-09	0	\\
2.42e-09	0	\\
2.52e-09	0	\\
2.63e-09	0	\\
2.73e-09	0	\\
2.83e-09	0	\\
2.93e-09	0	\\
3.04e-09	0	\\
3.14e-09	0	\\
3.24e-09	0	\\
3.34e-09	0	\\
3.45e-09	0	\\
3.55e-09	0	\\
3.65e-09	0	\\
3.75e-09	0	\\
3.86e-09	0	\\
3.96e-09	0	\\
4.06e-09	0	\\
4.16e-09	0	\\
4.27e-09	0	\\
4.37e-09	0	\\
4.47e-09	0	\\
4.57e-09	0	\\
4.68e-09	0	\\
4.78e-09	0	\\
4.89e-09	0	\\
4.99e-09	0	\\
5e-09	0	\\
};
\addplot [color=red,solid,forget plot]
  table[row sep=crcr]{
0	0	\\
1.1e-10	0	\\
2.2e-10	0	\\
3.3e-10	0	\\
4.4e-10	0	\\
5.4e-10	0	\\
6.5e-10	0	\\
7.5e-10	0	\\
8.6e-10	0	\\
9.6e-10	0	\\
1.07e-09	0	\\
1.18e-09	0	\\
1.28e-09	0	\\
1.38e-09	0	\\
1.49e-09	0	\\
1.59e-09	0	\\
1.69e-09	0	\\
1.8e-09	0	\\
1.9e-09	0	\\
2.01e-09	0	\\
2.11e-09	0	\\
2.21e-09	0	\\
2.32e-09	0	\\
2.42e-09	0	\\
2.52e-09	0	\\
2.63e-09	0	\\
2.73e-09	0	\\
2.83e-09	0	\\
2.93e-09	0	\\
3.04e-09	0	\\
3.14e-09	0	\\
3.24e-09	0	\\
3.34e-09	0	\\
3.45e-09	0	\\
3.55e-09	0	\\
3.65e-09	0	\\
3.75e-09	0	\\
3.86e-09	0	\\
3.96e-09	0	\\
4.06e-09	0	\\
4.16e-09	0	\\
4.27e-09	0	\\
4.37e-09	0	\\
4.47e-09	0	\\
4.57e-09	0	\\
4.68e-09	0	\\
4.78e-09	0	\\
4.89e-09	0	\\
4.99e-09	0	\\
5e-09	0	\\
};
\addplot [color=mycolor1,solid,forget plot]
  table[row sep=crcr]{
0	0	\\
1.1e-10	0	\\
2.2e-10	0	\\
3.3e-10	0	\\
4.4e-10	0	\\
5.4e-10	0	\\
6.5e-10	0	\\
7.5e-10	0	\\
8.6e-10	0	\\
9.6e-10	0	\\
1.07e-09	0	\\
1.18e-09	0	\\
1.28e-09	0	\\
1.38e-09	0	\\
1.49e-09	0	\\
1.59e-09	0	\\
1.69e-09	0	\\
1.8e-09	0	\\
1.9e-09	0	\\
2.01e-09	0	\\
2.11e-09	0	\\
2.21e-09	0	\\
2.32e-09	0	\\
2.42e-09	0	\\
2.52e-09	0	\\
2.63e-09	0	\\
2.73e-09	0	\\
2.83e-09	0	\\
2.93e-09	0	\\
3.04e-09	0	\\
3.14e-09	0	\\
3.24e-09	0	\\
3.34e-09	0	\\
3.45e-09	0	\\
3.55e-09	0	\\
3.65e-09	0	\\
3.75e-09	0	\\
3.86e-09	0	\\
3.96e-09	0	\\
4.06e-09	0	\\
4.16e-09	0	\\
4.27e-09	0	\\
4.37e-09	0	\\
4.47e-09	0	\\
4.57e-09	0	\\
4.68e-09	0	\\
4.78e-09	0	\\
4.89e-09	0	\\
4.99e-09	0	\\
5e-09	0	\\
};
\addplot [color=mycolor2,solid,forget plot]
  table[row sep=crcr]{
0	0	\\
1.1e-10	0	\\
2.2e-10	0	\\
3.3e-10	0	\\
4.4e-10	0	\\
5.4e-10	0	\\
6.5e-10	0	\\
7.5e-10	0	\\
8.6e-10	0	\\
9.6e-10	0	\\
1.07e-09	0	\\
1.18e-09	0	\\
1.28e-09	0	\\
1.38e-09	0	\\
1.49e-09	0	\\
1.59e-09	0	\\
1.69e-09	0	\\
1.8e-09	0	\\
1.9e-09	0	\\
2.01e-09	0	\\
2.11e-09	0	\\
2.21e-09	0	\\
2.32e-09	0	\\
2.42e-09	0	\\
2.52e-09	0	\\
2.63e-09	0	\\
2.73e-09	0	\\
2.83e-09	0	\\
2.93e-09	0	\\
3.04e-09	0	\\
3.14e-09	0	\\
3.24e-09	0	\\
3.34e-09	0	\\
3.45e-09	0	\\
3.55e-09	0	\\
3.65e-09	0	\\
3.75e-09	0	\\
3.86e-09	0	\\
3.96e-09	0	\\
4.06e-09	0	\\
4.16e-09	0	\\
4.27e-09	0	\\
4.37e-09	0	\\
4.47e-09	0	\\
4.57e-09	0	\\
4.68e-09	0	\\
4.78e-09	0	\\
4.89e-09	0	\\
4.99e-09	0	\\
5e-09	0	\\
};
\addplot [color=mycolor3,solid,forget plot]
  table[row sep=crcr]{
0	0	\\
1.1e-10	0	\\
2.2e-10	0	\\
3.3e-10	0	\\
4.4e-10	0	\\
5.4e-10	0	\\
6.5e-10	0	\\
7.5e-10	0	\\
8.6e-10	0	\\
9.6e-10	0	\\
1.07e-09	0	\\
1.18e-09	0	\\
1.28e-09	0	\\
1.38e-09	0	\\
1.49e-09	0	\\
1.59e-09	0	\\
1.69e-09	0	\\
1.8e-09	0	\\
1.9e-09	0	\\
2.01e-09	0	\\
2.11e-09	0	\\
2.21e-09	0	\\
2.32e-09	0	\\
2.42e-09	0	\\
2.52e-09	0	\\
2.63e-09	0	\\
2.73e-09	0	\\
2.83e-09	0	\\
2.93e-09	0	\\
3.04e-09	0	\\
3.14e-09	0	\\
3.24e-09	0	\\
3.34e-09	0	\\
3.45e-09	0	\\
3.55e-09	0	\\
3.65e-09	0	\\
3.75e-09	0	\\
3.86e-09	0	\\
3.96e-09	0	\\
4.06e-09	0	\\
4.16e-09	0	\\
4.27e-09	0	\\
4.37e-09	0	\\
4.47e-09	0	\\
4.57e-09	0	\\
4.68e-09	0	\\
4.78e-09	0	\\
4.89e-09	0	\\
4.99e-09	0	\\
5e-09	0	\\
};
\addplot [color=darkgray,solid,forget plot]
  table[row sep=crcr]{
0	0	\\
1.1e-10	0	\\
2.2e-10	0	\\
3.3e-10	0	\\
4.4e-10	0	\\
5.4e-10	0	\\
6.5e-10	0	\\
7.5e-10	0	\\
8.6e-10	0	\\
9.6e-10	0	\\
1.07e-09	0	\\
1.18e-09	0	\\
1.28e-09	0	\\
1.38e-09	0	\\
1.49e-09	0	\\
1.59e-09	0	\\
1.69e-09	0	\\
1.8e-09	0	\\
1.9e-09	0	\\
2.01e-09	0	\\
2.11e-09	0	\\
2.21e-09	0	\\
2.32e-09	0	\\
2.42e-09	0	\\
2.52e-09	0	\\
2.63e-09	0	\\
2.73e-09	0	\\
2.83e-09	0	\\
2.93e-09	0	\\
3.04e-09	0	\\
3.14e-09	0	\\
3.24e-09	0	\\
3.34e-09	0	\\
3.45e-09	0	\\
3.55e-09	0	\\
3.65e-09	0	\\
3.75e-09	0	\\
3.86e-09	0	\\
3.96e-09	0	\\
4.06e-09	0	\\
4.16e-09	0	\\
4.27e-09	0	\\
4.37e-09	0	\\
4.47e-09	0	\\
4.57e-09	0	\\
4.68e-09	0	\\
4.78e-09	0	\\
4.89e-09	0	\\
4.99e-09	0	\\
5e-09	0	\\
};
\addplot [color=blue,solid,forget plot]
  table[row sep=crcr]{
0	0	\\
1.1e-10	0	\\
2.2e-10	0	\\
3.3e-10	0	\\
4.4e-10	0	\\
5.4e-10	0	\\
6.5e-10	0	\\
7.5e-10	0	\\
8.6e-10	0	\\
9.6e-10	0	\\
1.07e-09	0	\\
1.18e-09	0	\\
1.28e-09	0	\\
1.38e-09	0	\\
1.49e-09	0	\\
1.59e-09	0	\\
1.69e-09	0	\\
1.8e-09	0	\\
1.9e-09	0	\\
2.01e-09	0	\\
2.11e-09	0	\\
2.21e-09	0	\\
2.32e-09	0	\\
2.42e-09	0	\\
2.52e-09	0	\\
2.63e-09	0	\\
2.73e-09	0	\\
2.83e-09	0	\\
2.93e-09	0	\\
3.04e-09	0	\\
3.14e-09	0	\\
3.24e-09	0	\\
3.34e-09	0	\\
3.45e-09	0	\\
3.55e-09	0	\\
3.65e-09	0	\\
3.75e-09	0	\\
3.86e-09	0	\\
3.96e-09	0	\\
4.06e-09	0	\\
4.16e-09	0	\\
4.27e-09	0	\\
4.37e-09	0	\\
4.47e-09	0	\\
4.57e-09	0	\\
4.68e-09	0	\\
4.78e-09	0	\\
4.89e-09	0	\\
4.99e-09	0	\\
5e-09	0	\\
};
\addplot [color=black!50!green,solid,forget plot]
  table[row sep=crcr]{
0	0	\\
1.1e-10	0	\\
2.2e-10	0	\\
3.3e-10	0	\\
4.4e-10	0	\\
5.4e-10	0	\\
6.5e-10	0	\\
7.5e-10	0	\\
8.6e-10	0	\\
9.6e-10	0	\\
1.07e-09	0	\\
1.18e-09	0	\\
1.28e-09	0	\\
1.38e-09	0	\\
1.49e-09	0	\\
1.59e-09	0	\\
1.69e-09	0	\\
1.8e-09	0	\\
1.9e-09	0	\\
2.01e-09	0	\\
2.11e-09	0	\\
2.21e-09	0	\\
2.32e-09	0	\\
2.42e-09	0	\\
2.52e-09	0	\\
2.63e-09	0	\\
2.73e-09	0	\\
2.83e-09	0	\\
2.93e-09	0	\\
3.04e-09	0	\\
3.14e-09	0	\\
3.24e-09	0	\\
3.34e-09	0	\\
3.45e-09	0	\\
3.55e-09	0	\\
3.65e-09	0	\\
3.75e-09	0	\\
3.86e-09	0	\\
3.96e-09	0	\\
4.06e-09	0	\\
4.16e-09	0	\\
4.27e-09	0	\\
4.37e-09	0	\\
4.47e-09	0	\\
4.57e-09	0	\\
4.68e-09	0	\\
4.78e-09	0	\\
4.89e-09	0	\\
4.99e-09	0	\\
5e-09	0	\\
};
\addplot [color=red,solid,forget plot]
  table[row sep=crcr]{
0	0	\\
1.1e-10	0	\\
2.2e-10	0	\\
3.3e-10	0	\\
4.4e-10	0	\\
5.4e-10	0	\\
6.5e-10	0	\\
7.5e-10	0	\\
8.6e-10	0	\\
9.6e-10	0	\\
1.07e-09	0	\\
1.18e-09	0	\\
1.28e-09	0	\\
1.38e-09	0	\\
1.49e-09	0	\\
1.59e-09	0	\\
1.69e-09	0	\\
1.8e-09	0	\\
1.9e-09	0	\\
2.01e-09	0	\\
2.11e-09	0	\\
2.21e-09	0	\\
2.32e-09	0	\\
2.42e-09	0	\\
2.52e-09	0	\\
2.63e-09	0	\\
2.73e-09	0	\\
2.83e-09	0	\\
2.93e-09	0	\\
3.04e-09	0	\\
3.14e-09	0	\\
3.24e-09	0	\\
3.34e-09	0	\\
3.45e-09	0	\\
3.55e-09	0	\\
3.65e-09	0	\\
3.75e-09	0	\\
3.86e-09	0	\\
3.96e-09	0	\\
4.06e-09	0	\\
4.16e-09	0	\\
4.27e-09	0	\\
4.37e-09	0	\\
4.47e-09	0	\\
4.57e-09	0	\\
4.68e-09	0	\\
4.78e-09	0	\\
4.89e-09	0	\\
4.99e-09	0	\\
5e-09	0	\\
};
\addplot [color=mycolor1,solid,forget plot]
  table[row sep=crcr]{
0	0	\\
1.1e-10	0	\\
2.2e-10	0	\\
3.3e-10	0	\\
4.4e-10	0	\\
5.4e-10	0	\\
6.5e-10	0	\\
7.5e-10	0	\\
8.6e-10	0	\\
9.6e-10	0	\\
1.07e-09	0	\\
1.18e-09	0	\\
1.28e-09	0	\\
1.38e-09	0	\\
1.49e-09	0	\\
1.59e-09	0	\\
1.69e-09	0	\\
1.8e-09	0	\\
1.9e-09	0	\\
2.01e-09	0	\\
2.11e-09	0	\\
2.21e-09	0	\\
2.32e-09	0	\\
2.42e-09	0	\\
2.52e-09	0	\\
2.63e-09	0	\\
2.73e-09	0	\\
2.83e-09	0	\\
2.93e-09	0	\\
3.04e-09	0	\\
3.14e-09	0	\\
3.24e-09	0	\\
3.34e-09	0	\\
3.45e-09	0	\\
3.55e-09	0	\\
3.65e-09	0	\\
3.75e-09	0	\\
3.86e-09	0	\\
3.96e-09	0	\\
4.06e-09	0	\\
4.16e-09	0	\\
4.27e-09	0	\\
4.37e-09	0	\\
4.47e-09	0	\\
4.57e-09	0	\\
4.68e-09	0	\\
4.78e-09	0	\\
4.89e-09	0	\\
4.99e-09	0	\\
5e-09	0	\\
};
\addplot [color=mycolor2,solid,forget plot]
  table[row sep=crcr]{
0	0	\\
1.1e-10	0	\\
2.2e-10	0	\\
3.3e-10	0	\\
4.4e-10	0	\\
5.4e-10	0	\\
6.5e-10	0	\\
7.5e-10	0	\\
8.6e-10	0	\\
9.6e-10	0	\\
1.07e-09	0	\\
1.18e-09	0	\\
1.28e-09	0	\\
1.38e-09	0	\\
1.49e-09	0	\\
1.59e-09	0	\\
1.69e-09	0	\\
1.8e-09	0	\\
1.9e-09	0	\\
2.01e-09	0	\\
2.11e-09	0	\\
2.21e-09	0	\\
2.32e-09	0	\\
2.42e-09	0	\\
2.52e-09	0	\\
2.63e-09	0	\\
2.73e-09	0	\\
2.83e-09	0	\\
2.93e-09	0	\\
3.04e-09	0	\\
3.14e-09	0	\\
3.24e-09	0	\\
3.34e-09	0	\\
3.45e-09	0	\\
3.55e-09	0	\\
3.65e-09	0	\\
3.75e-09	0	\\
3.86e-09	0	\\
3.96e-09	0	\\
4.06e-09	0	\\
4.16e-09	0	\\
4.27e-09	0	\\
4.37e-09	0	\\
4.47e-09	0	\\
4.57e-09	0	\\
4.68e-09	0	\\
4.78e-09	0	\\
4.89e-09	0	\\
4.99e-09	0	\\
5e-09	0	\\
};
\addplot [color=mycolor3,solid,forget plot]
  table[row sep=crcr]{
0	0	\\
1.1e-10	0	\\
2.2e-10	0	\\
3.3e-10	0	\\
4.4e-10	0	\\
5.4e-10	0	\\
6.5e-10	0	\\
7.5e-10	0	\\
8.6e-10	0	\\
9.6e-10	0	\\
1.07e-09	0	\\
1.18e-09	0	\\
1.28e-09	0	\\
1.38e-09	0	\\
1.49e-09	0	\\
1.59e-09	0	\\
1.69e-09	0	\\
1.8e-09	0	\\
1.9e-09	0	\\
2.01e-09	0	\\
2.11e-09	0	\\
2.21e-09	0	\\
2.32e-09	0	\\
2.42e-09	0	\\
2.52e-09	0	\\
2.63e-09	0	\\
2.73e-09	0	\\
2.83e-09	0	\\
2.93e-09	0	\\
3.04e-09	0	\\
3.14e-09	0	\\
3.24e-09	0	\\
3.34e-09	0	\\
3.45e-09	0	\\
3.55e-09	0	\\
3.65e-09	0	\\
3.75e-09	0	\\
3.86e-09	0	\\
3.96e-09	0	\\
4.06e-09	0	\\
4.16e-09	0	\\
4.27e-09	0	\\
4.37e-09	0	\\
4.47e-09	0	\\
4.57e-09	0	\\
4.68e-09	0	\\
4.78e-09	0	\\
4.89e-09	0	\\
4.99e-09	0	\\
5e-09	0	\\
};
\addplot [color=darkgray,solid,forget plot]
  table[row sep=crcr]{
0	0	\\
1.1e-10	0	\\
2.2e-10	0	\\
3.3e-10	0	\\
4.4e-10	0	\\
5.4e-10	0	\\
6.5e-10	0	\\
7.5e-10	0	\\
8.6e-10	0	\\
9.6e-10	0	\\
1.07e-09	0	\\
1.18e-09	0	\\
1.28e-09	0	\\
1.38e-09	0	\\
1.49e-09	0	\\
1.59e-09	0	\\
1.69e-09	0	\\
1.8e-09	0	\\
1.9e-09	0	\\
2.01e-09	0	\\
2.11e-09	0	\\
2.21e-09	0	\\
2.32e-09	0	\\
2.42e-09	0	\\
2.52e-09	0	\\
2.63e-09	0	\\
2.73e-09	0	\\
2.83e-09	0	\\
2.93e-09	0	\\
3.04e-09	0	\\
3.14e-09	0	\\
3.24e-09	0	\\
3.34e-09	0	\\
3.45e-09	0	\\
3.55e-09	0	\\
3.65e-09	0	\\
3.75e-09	0	\\
3.86e-09	0	\\
3.96e-09	0	\\
4.06e-09	0	\\
4.16e-09	0	\\
4.27e-09	0	\\
4.37e-09	0	\\
4.47e-09	0	\\
4.57e-09	0	\\
4.68e-09	0	\\
4.78e-09	0	\\
4.89e-09	0	\\
4.99e-09	0	\\
5e-09	0	\\
};
\addplot [color=blue,solid,forget plot]
  table[row sep=crcr]{
0	0	\\
1.1e-10	0	\\
2.2e-10	0	\\
3.3e-10	0	\\
4.4e-10	0	\\
5.4e-10	0	\\
6.5e-10	0	\\
7.5e-10	0	\\
8.6e-10	0	\\
9.6e-10	0	\\
1.07e-09	0	\\
1.18e-09	0	\\
1.28e-09	0	\\
1.38e-09	0	\\
1.49e-09	0	\\
1.59e-09	0	\\
1.69e-09	0	\\
1.8e-09	0	\\
1.9e-09	0	\\
2.01e-09	0	\\
2.11e-09	0	\\
2.21e-09	0	\\
2.32e-09	0	\\
2.42e-09	0	\\
2.52e-09	0	\\
2.63e-09	0	\\
2.73e-09	0	\\
2.83e-09	0	\\
2.93e-09	0	\\
3.04e-09	0	\\
3.14e-09	0	\\
3.24e-09	0	\\
3.34e-09	0	\\
3.45e-09	0	\\
3.55e-09	0	\\
3.65e-09	0	\\
3.75e-09	0	\\
3.86e-09	0	\\
3.96e-09	0	\\
4.06e-09	0	\\
4.16e-09	0	\\
4.27e-09	0	\\
4.37e-09	0	\\
4.47e-09	0	\\
4.57e-09	0	\\
4.68e-09	0	\\
4.78e-09	0	\\
4.89e-09	0	\\
4.99e-09	0	\\
5e-09	0	\\
};
\addplot [color=black!50!green,solid,forget plot]
  table[row sep=crcr]{
0	0	\\
1.1e-10	0	\\
2.2e-10	0	\\
3.3e-10	0	\\
4.4e-10	0	\\
5.4e-10	0	\\
6.5e-10	0	\\
7.5e-10	0	\\
8.6e-10	0	\\
9.6e-10	0	\\
1.07e-09	0	\\
1.18e-09	0	\\
1.28e-09	0	\\
1.38e-09	0	\\
1.49e-09	0	\\
1.59e-09	0	\\
1.69e-09	0	\\
1.8e-09	0	\\
1.9e-09	0	\\
2.01e-09	0	\\
2.11e-09	0	\\
2.21e-09	0	\\
2.32e-09	0	\\
2.42e-09	0	\\
2.52e-09	0	\\
2.63e-09	0	\\
2.73e-09	0	\\
2.83e-09	0	\\
2.93e-09	0	\\
3.04e-09	0	\\
3.14e-09	0	\\
3.24e-09	0	\\
3.34e-09	0	\\
3.45e-09	0	\\
3.55e-09	0	\\
3.65e-09	0	\\
3.75e-09	0	\\
3.86e-09	0	\\
3.96e-09	0	\\
4.06e-09	0	\\
4.16e-09	0	\\
4.27e-09	0	\\
4.37e-09	0	\\
4.47e-09	0	\\
4.57e-09	0	\\
4.68e-09	0	\\
4.78e-09	0	\\
4.89e-09	0	\\
4.99e-09	0	\\
5e-09	0	\\
};
\addplot [color=red,solid,forget plot]
  table[row sep=crcr]{
0	0	\\
1.1e-10	0	\\
2.2e-10	0	\\
3.3e-10	0	\\
4.4e-10	0	\\
5.4e-10	0	\\
6.5e-10	0	\\
7.5e-10	0	\\
8.6e-10	0	\\
9.6e-10	0	\\
1.07e-09	0	\\
1.18e-09	0	\\
1.28e-09	0	\\
1.38e-09	0	\\
1.49e-09	0	\\
1.59e-09	0	\\
1.69e-09	0	\\
1.8e-09	0	\\
1.9e-09	0	\\
2.01e-09	0	\\
2.11e-09	0	\\
2.21e-09	0	\\
2.32e-09	0	\\
2.42e-09	0	\\
2.52e-09	0	\\
2.63e-09	0	\\
2.73e-09	0	\\
2.83e-09	0	\\
2.93e-09	0	\\
3.04e-09	0	\\
3.14e-09	0	\\
3.24e-09	0	\\
3.34e-09	0	\\
3.45e-09	0	\\
3.55e-09	0	\\
3.65e-09	0	\\
3.75e-09	0	\\
3.86e-09	0	\\
3.96e-09	0	\\
4.06e-09	0	\\
4.16e-09	0	\\
4.27e-09	0	\\
4.37e-09	0	\\
4.47e-09	0	\\
4.57e-09	0	\\
4.68e-09	0	\\
4.78e-09	0	\\
4.89e-09	0	\\
4.99e-09	0	\\
5e-09	0	\\
};
\addplot [color=mycolor1,solid,forget plot]
  table[row sep=crcr]{
0	0	\\
1.1e-10	0	\\
2.2e-10	0	\\
3.3e-10	0	\\
4.4e-10	0	\\
5.4e-10	0	\\
6.5e-10	0	\\
7.5e-10	0	\\
8.6e-10	0	\\
9.6e-10	0	\\
1.07e-09	0	\\
1.18e-09	0	\\
1.28e-09	0	\\
1.38e-09	0	\\
1.49e-09	0	\\
1.59e-09	0	\\
1.69e-09	0	\\
1.8e-09	0	\\
1.9e-09	0	\\
2.01e-09	0	\\
2.11e-09	0	\\
2.21e-09	0	\\
2.32e-09	0	\\
2.42e-09	0	\\
2.52e-09	0	\\
2.63e-09	0	\\
2.73e-09	0	\\
2.83e-09	0	\\
2.93e-09	0	\\
3.04e-09	0	\\
3.14e-09	0	\\
3.24e-09	0	\\
3.34e-09	0	\\
3.45e-09	0	\\
3.55e-09	0	\\
3.65e-09	0	\\
3.75e-09	0	\\
3.86e-09	0	\\
3.96e-09	0	\\
4.06e-09	0	\\
4.16e-09	0	\\
4.27e-09	0	\\
4.37e-09	0	\\
4.47e-09	0	\\
4.57e-09	0	\\
4.68e-09	0	\\
4.78e-09	0	\\
4.89e-09	0	\\
4.99e-09	0	\\
5e-09	0	\\
};
\addplot [color=mycolor2,solid,forget plot]
  table[row sep=crcr]{
0	0	\\
1.1e-10	0	\\
2.2e-10	0	\\
3.3e-10	0	\\
4.4e-10	0	\\
5.4e-10	0	\\
6.5e-10	0	\\
7.5e-10	0	\\
8.6e-10	0	\\
9.6e-10	0	\\
1.07e-09	0	\\
1.18e-09	0	\\
1.28e-09	0	\\
1.38e-09	0	\\
1.49e-09	0	\\
1.59e-09	0	\\
1.69e-09	0	\\
1.8e-09	0	\\
1.9e-09	0	\\
2.01e-09	0	\\
2.11e-09	0	\\
2.21e-09	0	\\
2.32e-09	0	\\
2.42e-09	0	\\
2.52e-09	0	\\
2.63e-09	0	\\
2.73e-09	0	\\
2.83e-09	0	\\
2.93e-09	0	\\
3.04e-09	0	\\
3.14e-09	0	\\
3.24e-09	0	\\
3.34e-09	0	\\
3.45e-09	0	\\
3.55e-09	0	\\
3.65e-09	0	\\
3.75e-09	0	\\
3.86e-09	0	\\
3.96e-09	0	\\
4.06e-09	0	\\
4.16e-09	0	\\
4.27e-09	0	\\
4.37e-09	0	\\
4.47e-09	0	\\
4.57e-09	0	\\
4.68e-09	0	\\
4.78e-09	0	\\
4.89e-09	0	\\
4.99e-09	0	\\
5e-09	0	\\
};
\addplot [color=mycolor3,solid,forget plot]
  table[row sep=crcr]{
0	0	\\
1.1e-10	0	\\
2.2e-10	0	\\
3.3e-10	0	\\
4.4e-10	0	\\
5.4e-10	0	\\
6.5e-10	0	\\
7.5e-10	0	\\
8.6e-10	0	\\
9.6e-10	0	\\
1.07e-09	0	\\
1.18e-09	0	\\
1.28e-09	0	\\
1.38e-09	0	\\
1.49e-09	0	\\
1.59e-09	0	\\
1.69e-09	0	\\
1.8e-09	0	\\
1.9e-09	0	\\
2.01e-09	0	\\
2.11e-09	0	\\
2.21e-09	0	\\
2.32e-09	0	\\
2.42e-09	0	\\
2.52e-09	0	\\
2.63e-09	0	\\
2.73e-09	0	\\
2.83e-09	0	\\
2.93e-09	0	\\
3.04e-09	0	\\
3.14e-09	0	\\
3.24e-09	0	\\
3.34e-09	0	\\
3.45e-09	0	\\
3.55e-09	0	\\
3.65e-09	0	\\
3.75e-09	0	\\
3.86e-09	0	\\
3.96e-09	0	\\
4.06e-09	0	\\
4.16e-09	0	\\
4.27e-09	0	\\
4.37e-09	0	\\
4.47e-09	0	\\
4.57e-09	0	\\
4.68e-09	0	\\
4.78e-09	0	\\
4.89e-09	0	\\
4.99e-09	0	\\
5e-09	0	\\
};
\addplot [color=darkgray,solid,forget plot]
  table[row sep=crcr]{
0	0	\\
1.1e-10	0	\\
2.2e-10	0	\\
3.3e-10	0	\\
4.4e-10	0	\\
5.4e-10	0	\\
6.5e-10	0	\\
7.5e-10	0	\\
8.6e-10	0	\\
9.6e-10	0	\\
1.07e-09	0	\\
1.18e-09	0	\\
1.28e-09	0	\\
1.38e-09	0	\\
1.49e-09	0	\\
1.59e-09	0	\\
1.69e-09	0	\\
1.8e-09	0	\\
1.9e-09	0	\\
2.01e-09	0	\\
2.11e-09	0	\\
2.21e-09	0	\\
2.32e-09	0	\\
2.42e-09	0	\\
2.52e-09	0	\\
2.63e-09	0	\\
2.73e-09	0	\\
2.83e-09	0	\\
2.93e-09	0	\\
3.04e-09	0	\\
3.14e-09	0	\\
3.24e-09	0	\\
3.34e-09	0	\\
3.45e-09	0	\\
3.55e-09	0	\\
3.65e-09	0	\\
3.75e-09	0	\\
3.86e-09	0	\\
3.96e-09	0	\\
4.06e-09	0	\\
4.16e-09	0	\\
4.27e-09	0	\\
4.37e-09	0	\\
4.47e-09	0	\\
4.57e-09	0	\\
4.68e-09	0	\\
4.78e-09	0	\\
4.89e-09	0	\\
4.99e-09	0	\\
5e-09	0	\\
};
\addplot [color=blue,solid,forget plot]
  table[row sep=crcr]{
0	0	\\
1.1e-10	0	\\
2.2e-10	0	\\
3.3e-10	0	\\
4.4e-10	0	\\
5.4e-10	0	\\
6.5e-10	0	\\
7.5e-10	0	\\
8.6e-10	0	\\
9.6e-10	0	\\
1.07e-09	0	\\
1.18e-09	0	\\
1.28e-09	0	\\
1.38e-09	0	\\
1.49e-09	0	\\
1.59e-09	0	\\
1.69e-09	0	\\
1.8e-09	0	\\
1.9e-09	0	\\
2.01e-09	0	\\
2.11e-09	0	\\
2.21e-09	0	\\
2.32e-09	0	\\
2.42e-09	0	\\
2.52e-09	0	\\
2.63e-09	0	\\
2.73e-09	0	\\
2.83e-09	0	\\
2.93e-09	0	\\
3.04e-09	0	\\
3.14e-09	0	\\
3.24e-09	0	\\
3.34e-09	0	\\
3.45e-09	0	\\
3.55e-09	0	\\
3.65e-09	0	\\
3.75e-09	0	\\
3.86e-09	0	\\
3.96e-09	0	\\
4.06e-09	0	\\
4.16e-09	0	\\
4.27e-09	0	\\
4.37e-09	0	\\
4.47e-09	0	\\
4.57e-09	0	\\
4.68e-09	0	\\
4.78e-09	0	\\
4.89e-09	0	\\
4.99e-09	0	\\
5e-09	0	\\
};
\addplot [color=black!50!green,solid,forget plot]
  table[row sep=crcr]{
0	0	\\
1.1e-10	0	\\
2.2e-10	0	\\
3.3e-10	0	\\
4.4e-10	0	\\
5.4e-10	0	\\
6.5e-10	0	\\
7.5e-10	0	\\
8.6e-10	0	\\
9.6e-10	0	\\
1.07e-09	0	\\
1.18e-09	0	\\
1.28e-09	0	\\
1.38e-09	0	\\
1.49e-09	0	\\
1.59e-09	0	\\
1.69e-09	0	\\
1.8e-09	0	\\
1.9e-09	0	\\
2.01e-09	0	\\
2.11e-09	0	\\
2.21e-09	0	\\
2.32e-09	0	\\
2.42e-09	0	\\
2.52e-09	0	\\
2.63e-09	0	\\
2.73e-09	0	\\
2.83e-09	0	\\
2.93e-09	0	\\
3.04e-09	0	\\
3.14e-09	0	\\
3.24e-09	0	\\
3.34e-09	0	\\
3.45e-09	0	\\
3.55e-09	0	\\
3.65e-09	0	\\
3.75e-09	0	\\
3.86e-09	0	\\
3.96e-09	0	\\
4.06e-09	0	\\
4.16e-09	0	\\
4.27e-09	0	\\
4.37e-09	0	\\
4.47e-09	0	\\
4.57e-09	0	\\
4.68e-09	0	\\
4.78e-09	0	\\
4.89e-09	0	\\
4.99e-09	0	\\
5e-09	0	\\
};
\addplot [color=red,solid,forget plot]
  table[row sep=crcr]{
0	0	\\
1.1e-10	0	\\
2.2e-10	0	\\
3.3e-10	0	\\
4.4e-10	0	\\
5.4e-10	0	\\
6.5e-10	0	\\
7.5e-10	0	\\
8.6e-10	0	\\
9.6e-10	0	\\
1.07e-09	0	\\
1.18e-09	0	\\
1.28e-09	0	\\
1.38e-09	0	\\
1.49e-09	0	\\
1.59e-09	0	\\
1.69e-09	0	\\
1.8e-09	0	\\
1.9e-09	0	\\
2.01e-09	0	\\
2.11e-09	0	\\
2.21e-09	0	\\
2.32e-09	0	\\
2.42e-09	0	\\
2.52e-09	0	\\
2.63e-09	0	\\
2.73e-09	0	\\
2.83e-09	0	\\
2.93e-09	0	\\
3.04e-09	0	\\
3.14e-09	0	\\
3.24e-09	0	\\
3.34e-09	0	\\
3.45e-09	0	\\
3.55e-09	0	\\
3.65e-09	0	\\
3.75e-09	0	\\
3.86e-09	0	\\
3.96e-09	0	\\
4.06e-09	0	\\
4.16e-09	0	\\
4.27e-09	0	\\
4.37e-09	0	\\
4.47e-09	0	\\
4.57e-09	0	\\
4.68e-09	0	\\
4.78e-09	0	\\
4.89e-09	0	\\
4.99e-09	0	\\
5e-09	0	\\
};
\addplot [color=mycolor1,solid,forget plot]
  table[row sep=crcr]{
0	0	\\
1.1e-10	0	\\
2.2e-10	0	\\
3.3e-10	0	\\
4.4e-10	0	\\
5.4e-10	0	\\
6.5e-10	0	\\
7.5e-10	0	\\
8.6e-10	0	\\
9.6e-10	0	\\
1.07e-09	0	\\
1.18e-09	0	\\
1.28e-09	0	\\
1.38e-09	0	\\
1.49e-09	0	\\
1.59e-09	0	\\
1.69e-09	0	\\
1.8e-09	0	\\
1.9e-09	0	\\
2.01e-09	0	\\
2.11e-09	0	\\
2.21e-09	0	\\
2.32e-09	0	\\
2.42e-09	0	\\
2.52e-09	0	\\
2.63e-09	0	\\
2.73e-09	0	\\
2.83e-09	0	\\
2.93e-09	0	\\
3.04e-09	0	\\
3.14e-09	0	\\
3.24e-09	0	\\
3.34e-09	0	\\
3.45e-09	0	\\
3.55e-09	0	\\
3.65e-09	0	\\
3.75e-09	0	\\
3.86e-09	0	\\
3.96e-09	0	\\
4.06e-09	0	\\
4.16e-09	0	\\
4.27e-09	0	\\
4.37e-09	0	\\
4.47e-09	0	\\
4.57e-09	0	\\
4.68e-09	0	\\
4.78e-09	0	\\
4.89e-09	0	\\
4.99e-09	0	\\
5e-09	0	\\
};
\addplot [color=mycolor2,solid,forget plot]
  table[row sep=crcr]{
0	0	\\
1.1e-10	0	\\
2.2e-10	0	\\
3.3e-10	0	\\
4.4e-10	0	\\
5.4e-10	0	\\
6.5e-10	0	\\
7.5e-10	0	\\
8.6e-10	0	\\
9.6e-10	0	\\
1.07e-09	0	\\
1.18e-09	0	\\
1.28e-09	0	\\
1.38e-09	0	\\
1.49e-09	0	\\
1.59e-09	0	\\
1.69e-09	0	\\
1.8e-09	0	\\
1.9e-09	0	\\
2.01e-09	0	\\
2.11e-09	0	\\
2.21e-09	0	\\
2.32e-09	0	\\
2.42e-09	0	\\
2.52e-09	0	\\
2.63e-09	0	\\
2.73e-09	0	\\
2.83e-09	0	\\
2.93e-09	0	\\
3.04e-09	0	\\
3.14e-09	0	\\
3.24e-09	0	\\
3.34e-09	0	\\
3.45e-09	0	\\
3.55e-09	0	\\
3.65e-09	0	\\
3.75e-09	0	\\
3.86e-09	0	\\
3.96e-09	0	\\
4.06e-09	0	\\
4.16e-09	0	\\
4.27e-09	0	\\
4.37e-09	0	\\
4.47e-09	0	\\
4.57e-09	0	\\
4.68e-09	0	\\
4.78e-09	0	\\
4.89e-09	0	\\
4.99e-09	0	\\
5e-09	0	\\
};
\addplot [color=mycolor3,solid,forget plot]
  table[row sep=crcr]{
0	0	\\
1.1e-10	0	\\
2.2e-10	0	\\
3.3e-10	0	\\
4.4e-10	0	\\
5.4e-10	0	\\
6.5e-10	0	\\
7.5e-10	0	\\
8.6e-10	0	\\
9.6e-10	0	\\
1.07e-09	0	\\
1.18e-09	0	\\
1.28e-09	0	\\
1.38e-09	0	\\
1.49e-09	0	\\
1.59e-09	0	\\
1.69e-09	0	\\
1.8e-09	0	\\
1.9e-09	0	\\
2.01e-09	0	\\
2.11e-09	0	\\
2.21e-09	0	\\
2.32e-09	0	\\
2.42e-09	0	\\
2.52e-09	0	\\
2.63e-09	0	\\
2.73e-09	0	\\
2.83e-09	0	\\
2.93e-09	0	\\
3.04e-09	0	\\
3.14e-09	0	\\
3.24e-09	0	\\
3.34e-09	0	\\
3.45e-09	0	\\
3.55e-09	0	\\
3.65e-09	0	\\
3.75e-09	0	\\
3.86e-09	0	\\
3.96e-09	0	\\
4.06e-09	0	\\
4.16e-09	0	\\
4.27e-09	0	\\
4.37e-09	0	\\
4.47e-09	0	\\
4.57e-09	0	\\
4.68e-09	0	\\
4.78e-09	0	\\
4.89e-09	0	\\
4.99e-09	0	\\
5e-09	0	\\
};
\addplot [color=darkgray,solid,forget plot]
  table[row sep=crcr]{
0	0	\\
1.1e-10	0	\\
2.2e-10	0	\\
3.3e-10	0	\\
4.4e-10	0	\\
5.4e-10	0	\\
6.5e-10	0	\\
7.5e-10	0	\\
8.6e-10	0	\\
9.6e-10	0	\\
1.07e-09	0	\\
1.18e-09	0	\\
1.28e-09	0	\\
1.38e-09	0	\\
1.49e-09	0	\\
1.59e-09	0	\\
1.69e-09	0	\\
1.8e-09	0	\\
1.9e-09	0	\\
2.01e-09	0	\\
2.11e-09	0	\\
2.21e-09	0	\\
2.32e-09	0	\\
2.42e-09	0	\\
2.52e-09	0	\\
2.63e-09	0	\\
2.73e-09	0	\\
2.83e-09	0	\\
2.93e-09	0	\\
3.04e-09	0	\\
3.14e-09	0	\\
3.24e-09	0	\\
3.34e-09	0	\\
3.45e-09	0	\\
3.55e-09	0	\\
3.65e-09	0	\\
3.75e-09	0	\\
3.86e-09	0	\\
3.96e-09	0	\\
4.06e-09	0	\\
4.16e-09	0	\\
4.27e-09	0	\\
4.37e-09	0	\\
4.47e-09	0	\\
4.57e-09	0	\\
4.68e-09	0	\\
4.78e-09	0	\\
4.89e-09	0	\\
4.99e-09	0	\\
5e-09	0	\\
};
\addplot [color=blue,solid,forget plot]
  table[row sep=crcr]{
0	0	\\
1.1e-10	0	\\
2.2e-10	0	\\
3.3e-10	0	\\
4.4e-10	0	\\
5.4e-10	0	\\
6.5e-10	0	\\
7.5e-10	0	\\
8.6e-10	0	\\
9.6e-10	0	\\
1.07e-09	0	\\
1.18e-09	0	\\
1.28e-09	0	\\
1.38e-09	0	\\
1.49e-09	0	\\
1.59e-09	0	\\
1.69e-09	0	\\
1.8e-09	0	\\
1.9e-09	0	\\
2.01e-09	0	\\
2.11e-09	0	\\
2.21e-09	0	\\
2.32e-09	0	\\
2.42e-09	0	\\
2.52e-09	0	\\
2.63e-09	0	\\
2.73e-09	0	\\
2.83e-09	0	\\
2.93e-09	0	\\
3.04e-09	0	\\
3.14e-09	0	\\
3.24e-09	0	\\
3.34e-09	0	\\
3.45e-09	0	\\
3.55e-09	0	\\
3.65e-09	0	\\
3.75e-09	0	\\
3.86e-09	0	\\
3.96e-09	0	\\
4.06e-09	0	\\
4.16e-09	0	\\
4.27e-09	0	\\
4.37e-09	0	\\
4.47e-09	0	\\
4.57e-09	0	\\
4.68e-09	0	\\
4.78e-09	0	\\
4.89e-09	0	\\
4.99e-09	0	\\
5e-09	0	\\
};
\addplot [color=black!50!green,solid,forget plot]
  table[row sep=crcr]{
0	0	\\
1.1e-10	0	\\
2.2e-10	0	\\
3.3e-10	0	\\
4.4e-10	0	\\
5.4e-10	0	\\
6.5e-10	0	\\
7.5e-10	0	\\
8.6e-10	0	\\
9.6e-10	0	\\
1.07e-09	0	\\
1.18e-09	0	\\
1.28e-09	0	\\
1.38e-09	0	\\
1.49e-09	0	\\
1.59e-09	0	\\
1.69e-09	0	\\
1.8e-09	0	\\
1.9e-09	0	\\
2.01e-09	0	\\
2.11e-09	0	\\
2.21e-09	0	\\
2.32e-09	0	\\
2.42e-09	0	\\
2.52e-09	0	\\
2.63e-09	0	\\
2.73e-09	0	\\
2.83e-09	0	\\
2.93e-09	0	\\
3.04e-09	0	\\
3.14e-09	0	\\
3.24e-09	0	\\
3.34e-09	0	\\
3.45e-09	0	\\
3.55e-09	0	\\
3.65e-09	0	\\
3.75e-09	0	\\
3.86e-09	0	\\
3.96e-09	0	\\
4.06e-09	0	\\
4.16e-09	0	\\
4.27e-09	0	\\
4.37e-09	0	\\
4.47e-09	0	\\
4.57e-09	0	\\
4.68e-09	0	\\
4.78e-09	0	\\
4.89e-09	0	\\
4.99e-09	0	\\
5e-09	0	\\
};
\addplot [color=red,solid,forget plot]
  table[row sep=crcr]{
0	0	\\
1.1e-10	0	\\
2.2e-10	0	\\
3.3e-10	0	\\
4.4e-10	0	\\
5.4e-10	0	\\
6.5e-10	0	\\
7.5e-10	0	\\
8.6e-10	0	\\
9.6e-10	0	\\
1.07e-09	0	\\
1.18e-09	0	\\
1.28e-09	0	\\
1.38e-09	0	\\
1.49e-09	0	\\
1.59e-09	0	\\
1.69e-09	0	\\
1.8e-09	0	\\
1.9e-09	0	\\
2.01e-09	0	\\
2.11e-09	0	\\
2.21e-09	0	\\
2.32e-09	0	\\
2.42e-09	0	\\
2.52e-09	0	\\
2.63e-09	0	\\
2.73e-09	0	\\
2.83e-09	0	\\
2.93e-09	0	\\
3.04e-09	0	\\
3.14e-09	0	\\
3.24e-09	0	\\
3.34e-09	0	\\
3.45e-09	0	\\
3.55e-09	0	\\
3.65e-09	0	\\
3.75e-09	0	\\
3.86e-09	0	\\
3.96e-09	0	\\
4.06e-09	0	\\
4.16e-09	0	\\
4.27e-09	0	\\
4.37e-09	0	\\
4.47e-09	0	\\
4.57e-09	0	\\
4.68e-09	0	\\
4.78e-09	0	\\
4.89e-09	0	\\
4.99e-09	0	\\
5e-09	0	\\
};
\addplot [color=mycolor1,solid,forget plot]
  table[row sep=crcr]{
0	0	\\
1.1e-10	0	\\
2.2e-10	0	\\
3.3e-10	0	\\
4.4e-10	0	\\
5.4e-10	0	\\
6.5e-10	0	\\
7.5e-10	0	\\
8.6e-10	0	\\
9.6e-10	0	\\
1.07e-09	0	\\
1.18e-09	0	\\
1.28e-09	0	\\
1.38e-09	0	\\
1.49e-09	0	\\
1.59e-09	0	\\
1.69e-09	0	\\
1.8e-09	0	\\
1.9e-09	0	\\
2.01e-09	0	\\
2.11e-09	0	\\
2.21e-09	0	\\
2.32e-09	0	\\
2.42e-09	0	\\
2.52e-09	0	\\
2.63e-09	0	\\
2.73e-09	0	\\
2.83e-09	0	\\
2.93e-09	0	\\
3.04e-09	0	\\
3.14e-09	0	\\
3.24e-09	0	\\
3.34e-09	0	\\
3.45e-09	0	\\
3.55e-09	0	\\
3.65e-09	0	\\
3.75e-09	0	\\
3.86e-09	0	\\
3.96e-09	0	\\
4.06e-09	0	\\
4.16e-09	0	\\
4.27e-09	0	\\
4.37e-09	0	\\
4.47e-09	0	\\
4.57e-09	0	\\
4.68e-09	0	\\
4.78e-09	0	\\
4.89e-09	0	\\
4.99e-09	0	\\
5e-09	0	\\
};
\addplot [color=mycolor2,solid,forget plot]
  table[row sep=crcr]{
0	0	\\
1.1e-10	0	\\
2.2e-10	0	\\
3.3e-10	0	\\
4.4e-10	0	\\
5.4e-10	0	\\
6.5e-10	0	\\
7.5e-10	0	\\
8.6e-10	0	\\
9.6e-10	0	\\
1.07e-09	0	\\
1.18e-09	0	\\
1.28e-09	0	\\
1.38e-09	0	\\
1.49e-09	0	\\
1.59e-09	0	\\
1.69e-09	0	\\
1.8e-09	0	\\
1.9e-09	0	\\
2.01e-09	0	\\
2.11e-09	0	\\
2.21e-09	0	\\
2.32e-09	0	\\
2.42e-09	0	\\
2.52e-09	0	\\
2.63e-09	0	\\
2.73e-09	0	\\
2.83e-09	0	\\
2.93e-09	0	\\
3.04e-09	0	\\
3.14e-09	0	\\
3.24e-09	0	\\
3.34e-09	0	\\
3.45e-09	0	\\
3.55e-09	0	\\
3.65e-09	0	\\
3.75e-09	0	\\
3.86e-09	0	\\
3.96e-09	0	\\
4.06e-09	0	\\
4.16e-09	0	\\
4.27e-09	0	\\
4.37e-09	0	\\
4.47e-09	0	\\
4.57e-09	0	\\
4.68e-09	0	\\
4.78e-09	0	\\
4.89e-09	0	\\
4.99e-09	0	\\
5e-09	0	\\
};
\addplot [color=mycolor3,solid,forget plot]
  table[row sep=crcr]{
0	0	\\
1.1e-10	0	\\
2.2e-10	0	\\
3.3e-10	0	\\
4.4e-10	0	\\
5.4e-10	0	\\
6.5e-10	0	\\
7.5e-10	0	\\
8.6e-10	0	\\
9.6e-10	0	\\
1.07e-09	0	\\
1.18e-09	0	\\
1.28e-09	0	\\
1.38e-09	0	\\
1.49e-09	0	\\
1.59e-09	0	\\
1.69e-09	0	\\
1.8e-09	0	\\
1.9e-09	0	\\
2.01e-09	0	\\
2.11e-09	0	\\
2.21e-09	0	\\
2.32e-09	0	\\
2.42e-09	0	\\
2.52e-09	0	\\
2.63e-09	0	\\
2.73e-09	0	\\
2.83e-09	0	\\
2.93e-09	0	\\
3.04e-09	0	\\
3.14e-09	0	\\
3.24e-09	0	\\
3.34e-09	0	\\
3.45e-09	0	\\
3.55e-09	0	\\
3.65e-09	0	\\
3.75e-09	0	\\
3.86e-09	0	\\
3.96e-09	0	\\
4.06e-09	0	\\
4.16e-09	0	\\
4.27e-09	0	\\
4.37e-09	0	\\
4.47e-09	0	\\
4.57e-09	0	\\
4.68e-09	0	\\
4.78e-09	0	\\
4.89e-09	0	\\
4.99e-09	0	\\
5e-09	0	\\
};
\addplot [color=darkgray,solid,forget plot]
  table[row sep=crcr]{
0	0	\\
1.1e-10	0	\\
2.2e-10	0	\\
3.3e-10	0	\\
4.4e-10	0	\\
5.4e-10	0	\\
6.5e-10	0	\\
7.5e-10	0	\\
8.6e-10	0	\\
9.6e-10	0	\\
1.07e-09	0	\\
1.18e-09	0	\\
1.28e-09	0	\\
1.38e-09	0	\\
1.49e-09	0	\\
1.59e-09	0	\\
1.69e-09	0	\\
1.8e-09	0	\\
1.9e-09	0	\\
2.01e-09	0	\\
2.11e-09	0	\\
2.21e-09	0	\\
2.32e-09	0	\\
2.42e-09	0	\\
2.52e-09	0	\\
2.63e-09	0	\\
2.73e-09	0	\\
2.83e-09	0	\\
2.93e-09	0	\\
3.04e-09	0	\\
3.14e-09	0	\\
3.24e-09	0	\\
3.34e-09	0	\\
3.45e-09	0	\\
3.55e-09	0	\\
3.65e-09	0	\\
3.75e-09	0	\\
3.86e-09	0	\\
3.96e-09	0	\\
4.06e-09	0	\\
4.16e-09	0	\\
4.27e-09	0	\\
4.37e-09	0	\\
4.47e-09	0	\\
4.57e-09	0	\\
4.68e-09	0	\\
4.78e-09	0	\\
4.89e-09	0	\\
4.99e-09	0	\\
5e-09	0	\\
};
\addplot [color=blue,solid,forget plot]
  table[row sep=crcr]{
0	0	\\
1.1e-10	0	\\
2.2e-10	0	\\
3.3e-10	0	\\
4.4e-10	0	\\
5.4e-10	0	\\
6.5e-10	0	\\
7.5e-10	0	\\
8.6e-10	0	\\
9.6e-10	0	\\
1.07e-09	0	\\
1.18e-09	0	\\
1.28e-09	0	\\
1.38e-09	0	\\
1.49e-09	0	\\
1.59e-09	0	\\
1.69e-09	0	\\
1.8e-09	0	\\
1.9e-09	0	\\
2.01e-09	0	\\
2.11e-09	0	\\
2.21e-09	0	\\
2.32e-09	0	\\
2.42e-09	0	\\
2.52e-09	0	\\
2.63e-09	0	\\
2.73e-09	0	\\
2.83e-09	0	\\
2.93e-09	0	\\
3.04e-09	0	\\
3.14e-09	0	\\
3.24e-09	0	\\
3.34e-09	0	\\
3.45e-09	0	\\
3.55e-09	0	\\
3.65e-09	0	\\
3.75e-09	0	\\
3.86e-09	0	\\
3.96e-09	0	\\
4.06e-09	0	\\
4.16e-09	0	\\
4.27e-09	0	\\
4.37e-09	0	\\
4.47e-09	0	\\
4.57e-09	0	\\
4.68e-09	0	\\
4.78e-09	0	\\
4.89e-09	0	\\
4.99e-09	0	\\
5e-09	0	\\
};
\addplot [color=black!50!green,solid,forget plot]
  table[row sep=crcr]{
0	0	\\
1.1e-10	0	\\
2.2e-10	0	\\
3.3e-10	0	\\
4.4e-10	0	\\
5.4e-10	0	\\
6.5e-10	0	\\
7.5e-10	0	\\
8.6e-10	0	\\
9.6e-10	0	\\
1.07e-09	0	\\
1.18e-09	0	\\
1.28e-09	0	\\
1.38e-09	0	\\
1.49e-09	0	\\
1.59e-09	0	\\
1.69e-09	0	\\
1.8e-09	0	\\
1.9e-09	0	\\
2.01e-09	0	\\
2.11e-09	0	\\
2.21e-09	0	\\
2.32e-09	0	\\
2.42e-09	0	\\
2.52e-09	0	\\
2.63e-09	0	\\
2.73e-09	0	\\
2.83e-09	0	\\
2.93e-09	0	\\
3.04e-09	0	\\
3.14e-09	0	\\
3.24e-09	0	\\
3.34e-09	0	\\
3.45e-09	0	\\
3.55e-09	0	\\
3.65e-09	0	\\
3.75e-09	0	\\
3.86e-09	0	\\
3.96e-09	0	\\
4.06e-09	0	\\
4.16e-09	0	\\
4.27e-09	0	\\
4.37e-09	0	\\
4.47e-09	0	\\
4.57e-09	0	\\
4.68e-09	0	\\
4.78e-09	0	\\
4.89e-09	0	\\
4.99e-09	0	\\
5e-09	0	\\
};
\addplot [color=red,solid,forget plot]
  table[row sep=crcr]{
0	0	\\
1.1e-10	0	\\
2.2e-10	0	\\
3.3e-10	0	\\
4.4e-10	0	\\
5.4e-10	0	\\
6.5e-10	0	\\
7.5e-10	0	\\
8.6e-10	0	\\
9.6e-10	0	\\
1.07e-09	0	\\
1.18e-09	0	\\
1.28e-09	0	\\
1.38e-09	0	\\
1.49e-09	0	\\
1.59e-09	0	\\
1.69e-09	0	\\
1.8e-09	0	\\
1.9e-09	0	\\
2.01e-09	0	\\
2.11e-09	0	\\
2.21e-09	0	\\
2.32e-09	0	\\
2.42e-09	0	\\
2.52e-09	0	\\
2.63e-09	0	\\
2.73e-09	0	\\
2.83e-09	0	\\
2.93e-09	0	\\
3.04e-09	0	\\
3.14e-09	0	\\
3.24e-09	0	\\
3.34e-09	0	\\
3.45e-09	0	\\
3.55e-09	0	\\
3.65e-09	0	\\
3.75e-09	0	\\
3.86e-09	0	\\
3.96e-09	0	\\
4.06e-09	0	\\
4.16e-09	0	\\
4.27e-09	0	\\
4.37e-09	0	\\
4.47e-09	0	\\
4.57e-09	0	\\
4.68e-09	0	\\
4.78e-09	0	\\
4.89e-09	0	\\
4.99e-09	0	\\
5e-09	0	\\
};
\addplot [color=mycolor1,solid,forget plot]
  table[row sep=crcr]{
0	0	\\
1.1e-10	0	\\
2.2e-10	0	\\
3.3e-10	0	\\
4.4e-10	0	\\
5.4e-10	0	\\
6.5e-10	0	\\
7.5e-10	0	\\
8.6e-10	0	\\
9.6e-10	0	\\
1.07e-09	0	\\
1.18e-09	0	\\
1.28e-09	0	\\
1.38e-09	0	\\
1.49e-09	0	\\
1.59e-09	0	\\
1.69e-09	0	\\
1.8e-09	0	\\
1.9e-09	0	\\
2.01e-09	0	\\
2.11e-09	0	\\
2.21e-09	0	\\
2.32e-09	0	\\
2.42e-09	0	\\
2.52e-09	0	\\
2.63e-09	0	\\
2.73e-09	0	\\
2.83e-09	0	\\
2.93e-09	0	\\
3.04e-09	0	\\
3.14e-09	0	\\
3.24e-09	0	\\
3.34e-09	0	\\
3.45e-09	0	\\
3.55e-09	0	\\
3.65e-09	0	\\
3.75e-09	0	\\
3.86e-09	0	\\
3.96e-09	0	\\
4.06e-09	0	\\
4.16e-09	0	\\
4.27e-09	0	\\
4.37e-09	0	\\
4.47e-09	0	\\
4.57e-09	0	\\
4.68e-09	0	\\
4.78e-09	0	\\
4.89e-09	0	\\
4.99e-09	0	\\
5e-09	0	\\
};
\addplot [color=mycolor2,solid,forget plot]
  table[row sep=crcr]{
0	0	\\
1.1e-10	0	\\
2.2e-10	0	\\
3.3e-10	0	\\
4.4e-10	0	\\
5.4e-10	0	\\
6.5e-10	0	\\
7.5e-10	0	\\
8.6e-10	0	\\
9.6e-10	0	\\
1.07e-09	0	\\
1.18e-09	0	\\
1.28e-09	0	\\
1.38e-09	0	\\
1.49e-09	0	\\
1.59e-09	0	\\
1.69e-09	0	\\
1.8e-09	0	\\
1.9e-09	0	\\
2.01e-09	0	\\
2.11e-09	0	\\
2.21e-09	0	\\
2.32e-09	0	\\
2.42e-09	0	\\
2.52e-09	0	\\
2.63e-09	0	\\
2.73e-09	0	\\
2.83e-09	0	\\
2.93e-09	0	\\
3.04e-09	0	\\
3.14e-09	0	\\
3.24e-09	0	\\
3.34e-09	0	\\
3.45e-09	0	\\
3.55e-09	0	\\
3.65e-09	0	\\
3.75e-09	0	\\
3.86e-09	0	\\
3.96e-09	0	\\
4.06e-09	0	\\
4.16e-09	0	\\
4.27e-09	0	\\
4.37e-09	0	\\
4.47e-09	0	\\
4.57e-09	0	\\
4.68e-09	0	\\
4.78e-09	0	\\
4.89e-09	0	\\
4.99e-09	0	\\
5e-09	0	\\
};
\addplot [color=mycolor3,solid,forget plot]
  table[row sep=crcr]{
0	0	\\
1.1e-10	0	\\
2.2e-10	0	\\
3.3e-10	0	\\
4.4e-10	0	\\
5.4e-10	0	\\
6.5e-10	0	\\
7.5e-10	0	\\
8.6e-10	0	\\
9.6e-10	0	\\
1.07e-09	0	\\
1.18e-09	0	\\
1.28e-09	0	\\
1.38e-09	0	\\
1.49e-09	0	\\
1.59e-09	0	\\
1.69e-09	0	\\
1.8e-09	0	\\
1.9e-09	0	\\
2.01e-09	0	\\
2.11e-09	0	\\
2.21e-09	0	\\
2.32e-09	0	\\
2.42e-09	0	\\
2.52e-09	0	\\
2.63e-09	0	\\
2.73e-09	0	\\
2.83e-09	0	\\
2.93e-09	0	\\
3.04e-09	0	\\
3.14e-09	0	\\
3.24e-09	0	\\
3.34e-09	0	\\
3.45e-09	0	\\
3.55e-09	0	\\
3.65e-09	0	\\
3.75e-09	0	\\
3.86e-09	0	\\
3.96e-09	0	\\
4.06e-09	0	\\
4.16e-09	0	\\
4.27e-09	0	\\
4.37e-09	0	\\
4.47e-09	0	\\
4.57e-09	0	\\
4.68e-09	0	\\
4.78e-09	0	\\
4.89e-09	0	\\
4.99e-09	0	\\
5e-09	0	\\
};
\addplot [color=darkgray,solid,forget plot]
  table[row sep=crcr]{
0	0	\\
1.1e-10	0	\\
2.2e-10	0	\\
3.3e-10	0	\\
4.4e-10	0	\\
5.4e-10	0	\\
6.5e-10	0	\\
7.5e-10	0	\\
8.6e-10	0	\\
9.6e-10	0	\\
1.07e-09	0	\\
1.18e-09	0	\\
1.28e-09	0	\\
1.38e-09	0	\\
1.49e-09	0	\\
1.59e-09	0	\\
1.69e-09	0	\\
1.8e-09	0	\\
1.9e-09	0	\\
2.01e-09	0	\\
2.11e-09	0	\\
2.21e-09	0	\\
2.32e-09	0	\\
2.42e-09	0	\\
2.52e-09	0	\\
2.63e-09	0	\\
2.73e-09	0	\\
2.83e-09	0	\\
2.93e-09	0	\\
3.04e-09	0	\\
3.14e-09	0	\\
3.24e-09	0	\\
3.34e-09	0	\\
3.45e-09	0	\\
3.55e-09	0	\\
3.65e-09	0	\\
3.75e-09	0	\\
3.86e-09	0	\\
3.96e-09	0	\\
4.06e-09	0	\\
4.16e-09	0	\\
4.27e-09	0	\\
4.37e-09	0	\\
4.47e-09	0	\\
4.57e-09	0	\\
4.68e-09	0	\\
4.78e-09	0	\\
4.89e-09	0	\\
4.99e-09	0	\\
5e-09	0	\\
};
\addplot [color=blue,solid,forget plot]
  table[row sep=crcr]{
0	0	\\
1.1e-10	0	\\
2.2e-10	0	\\
3.3e-10	0	\\
4.4e-10	0	\\
5.4e-10	0	\\
6.5e-10	0	\\
7.5e-10	0	\\
8.6e-10	0	\\
9.6e-10	0	\\
1.07e-09	0	\\
1.18e-09	0	\\
1.28e-09	0	\\
1.38e-09	0	\\
1.49e-09	0	\\
1.59e-09	0	\\
1.69e-09	0	\\
1.8e-09	0	\\
1.9e-09	0	\\
2.01e-09	0	\\
2.11e-09	0	\\
2.21e-09	0	\\
2.32e-09	0	\\
2.42e-09	0	\\
2.52e-09	0	\\
2.63e-09	0	\\
2.73e-09	0	\\
2.83e-09	0	\\
2.93e-09	0	\\
3.04e-09	0	\\
3.14e-09	0	\\
3.24e-09	0	\\
3.34e-09	0	\\
3.45e-09	0	\\
3.55e-09	0	\\
3.65e-09	0	\\
3.75e-09	0	\\
3.86e-09	0	\\
3.96e-09	0	\\
4.06e-09	0	\\
4.16e-09	0	\\
4.27e-09	0	\\
4.37e-09	0	\\
4.47e-09	0	\\
4.57e-09	0	\\
4.68e-09	0	\\
4.78e-09	0	\\
4.89e-09	0	\\
4.99e-09	0	\\
5e-09	0	\\
};
\addplot [color=black!50!green,solid,forget plot]
  table[row sep=crcr]{
0	0	\\
1.1e-10	0	\\
2.2e-10	0	\\
3.3e-10	0	\\
4.4e-10	0	\\
5.4e-10	0	\\
6.5e-10	0	\\
7.5e-10	0	\\
8.6e-10	0	\\
9.6e-10	0	\\
1.07e-09	0	\\
1.18e-09	0	\\
1.28e-09	0	\\
1.38e-09	0	\\
1.49e-09	0	\\
1.59e-09	0	\\
1.69e-09	0	\\
1.8e-09	0	\\
1.9e-09	0	\\
2.01e-09	0	\\
2.11e-09	0	\\
2.21e-09	0	\\
2.32e-09	0	\\
2.42e-09	0	\\
2.52e-09	0	\\
2.63e-09	0	\\
2.73e-09	0	\\
2.83e-09	0	\\
2.93e-09	0	\\
3.04e-09	0	\\
3.14e-09	0	\\
3.24e-09	0	\\
3.34e-09	0	\\
3.45e-09	0	\\
3.55e-09	0	\\
3.65e-09	0	\\
3.75e-09	0	\\
3.86e-09	0	\\
3.96e-09	0	\\
4.06e-09	0	\\
4.16e-09	0	\\
4.27e-09	0	\\
4.37e-09	0	\\
4.47e-09	0	\\
4.57e-09	0	\\
4.68e-09	0	\\
4.78e-09	0	\\
4.89e-09	0	\\
4.99e-09	0	\\
5e-09	0	\\
};
\addplot [color=red,solid,forget plot]
  table[row sep=crcr]{
0	0	\\
1.1e-10	0	\\
2.2e-10	0	\\
3.3e-10	0	\\
4.4e-10	0	\\
5.4e-10	0	\\
6.5e-10	0	\\
7.5e-10	0	\\
8.6e-10	0	\\
9.6e-10	0	\\
1.07e-09	0	\\
1.18e-09	0	\\
1.28e-09	0	\\
1.38e-09	0	\\
1.49e-09	0	\\
1.59e-09	0	\\
1.69e-09	0	\\
1.8e-09	0	\\
1.9e-09	0	\\
2.01e-09	0	\\
2.11e-09	0	\\
2.21e-09	0	\\
2.32e-09	0	\\
2.42e-09	0	\\
2.52e-09	0	\\
2.63e-09	0	\\
2.73e-09	0	\\
2.83e-09	0	\\
2.93e-09	0	\\
3.04e-09	0	\\
3.14e-09	0	\\
3.24e-09	0	\\
3.34e-09	0	\\
3.45e-09	0	\\
3.55e-09	0	\\
3.65e-09	0	\\
3.75e-09	0	\\
3.86e-09	0	\\
3.96e-09	0	\\
4.06e-09	0	\\
4.16e-09	0	\\
4.27e-09	0	\\
4.37e-09	0	\\
4.47e-09	0	\\
4.57e-09	0	\\
4.68e-09	0	\\
4.78e-09	0	\\
4.89e-09	0	\\
4.99e-09	0	\\
5e-09	0	\\
};
\addplot [color=mycolor1,solid,forget plot]
  table[row sep=crcr]{
0	0	\\
1.1e-10	0	\\
2.2e-10	0	\\
3.3e-10	0	\\
4.4e-10	0	\\
5.4e-10	0	\\
6.5e-10	0	\\
7.5e-10	0	\\
8.6e-10	0	\\
9.6e-10	0	\\
1.07e-09	0	\\
1.18e-09	0	\\
1.28e-09	0	\\
1.38e-09	0	\\
1.49e-09	0	\\
1.59e-09	0	\\
1.69e-09	0	\\
1.8e-09	0	\\
1.9e-09	0	\\
2.01e-09	0	\\
2.11e-09	0	\\
2.21e-09	0	\\
2.32e-09	0	\\
2.42e-09	0	\\
2.52e-09	0	\\
2.63e-09	0	\\
2.73e-09	0	\\
2.83e-09	0	\\
2.93e-09	0	\\
3.04e-09	0	\\
3.14e-09	0	\\
3.24e-09	0	\\
3.34e-09	0	\\
3.45e-09	0	\\
3.55e-09	0	\\
3.65e-09	0	\\
3.75e-09	0	\\
3.86e-09	0	\\
3.96e-09	0	\\
4.06e-09	0	\\
4.16e-09	0	\\
4.27e-09	0	\\
4.37e-09	0	\\
4.47e-09	0	\\
4.57e-09	0	\\
4.68e-09	0	\\
4.78e-09	0	\\
4.89e-09	0	\\
4.99e-09	0	\\
5e-09	0	\\
};
\addplot [color=mycolor2,solid,forget plot]
  table[row sep=crcr]{
0	0	\\
1.1e-10	0	\\
2.2e-10	0	\\
3.3e-10	0	\\
4.4e-10	0	\\
5.4e-10	0	\\
6.5e-10	0	\\
7.5e-10	0	\\
8.6e-10	0	\\
9.6e-10	0	\\
1.07e-09	0	\\
1.18e-09	0	\\
1.28e-09	0	\\
1.38e-09	0	\\
1.49e-09	0	\\
1.59e-09	0	\\
1.69e-09	0	\\
1.8e-09	0	\\
1.9e-09	0	\\
2.01e-09	0	\\
2.11e-09	0	\\
2.21e-09	0	\\
2.32e-09	0	\\
2.42e-09	0	\\
2.52e-09	0	\\
2.63e-09	0	\\
2.73e-09	0	\\
2.83e-09	0	\\
2.93e-09	0	\\
3.04e-09	0	\\
3.14e-09	0	\\
3.24e-09	0	\\
3.34e-09	0	\\
3.45e-09	0	\\
3.55e-09	0	\\
3.65e-09	0	\\
3.75e-09	0	\\
3.86e-09	0	\\
3.96e-09	0	\\
4.06e-09	0	\\
4.16e-09	0	\\
4.27e-09	0	\\
4.37e-09	0	\\
4.47e-09	0	\\
4.57e-09	0	\\
4.68e-09	0	\\
4.78e-09	0	\\
4.89e-09	0	\\
4.99e-09	0	\\
5e-09	0	\\
};
\addplot [color=mycolor3,solid,forget plot]
  table[row sep=crcr]{
0	0	\\
1.1e-10	0	\\
2.2e-10	0	\\
3.3e-10	0	\\
4.4e-10	0	\\
5.4e-10	0	\\
6.5e-10	0	\\
7.5e-10	0	\\
8.6e-10	0	\\
9.6e-10	0	\\
1.07e-09	0	\\
1.18e-09	0	\\
1.28e-09	0	\\
1.38e-09	0	\\
1.49e-09	0	\\
1.59e-09	0	\\
1.69e-09	0	\\
1.8e-09	0	\\
1.9e-09	0	\\
2.01e-09	0	\\
2.11e-09	0	\\
2.21e-09	0	\\
2.32e-09	0	\\
2.42e-09	0	\\
2.52e-09	0	\\
2.63e-09	0	\\
2.73e-09	0	\\
2.83e-09	0	\\
2.93e-09	0	\\
3.04e-09	0	\\
3.14e-09	0	\\
3.24e-09	0	\\
3.34e-09	0	\\
3.45e-09	0	\\
3.55e-09	0	\\
3.65e-09	0	\\
3.75e-09	0	\\
3.86e-09	0	\\
3.96e-09	0	\\
4.06e-09	0	\\
4.16e-09	0	\\
4.27e-09	0	\\
4.37e-09	0	\\
4.47e-09	0	\\
4.57e-09	0	\\
4.68e-09	0	\\
4.78e-09	0	\\
4.89e-09	0	\\
4.99e-09	0	\\
5e-09	0	\\
};
\addplot [color=darkgray,solid,forget plot]
  table[row sep=crcr]{
0	0	\\
1.1e-10	0	\\
2.2e-10	0	\\
3.3e-10	0	\\
4.4e-10	0	\\
5.4e-10	0	\\
6.5e-10	0	\\
7.5e-10	0	\\
8.6e-10	0	\\
9.6e-10	0	\\
1.07e-09	0	\\
1.18e-09	0	\\
1.28e-09	0	\\
1.38e-09	0	\\
1.49e-09	0	\\
1.59e-09	0	\\
1.69e-09	0	\\
1.8e-09	0	\\
1.9e-09	0	\\
2.01e-09	0	\\
2.11e-09	0	\\
2.21e-09	0	\\
2.32e-09	0	\\
2.42e-09	0	\\
2.52e-09	0	\\
2.63e-09	0	\\
2.73e-09	0	\\
2.83e-09	0	\\
2.93e-09	0	\\
3.04e-09	0	\\
3.14e-09	0	\\
3.24e-09	0	\\
3.34e-09	0	\\
3.45e-09	0	\\
3.55e-09	0	\\
3.65e-09	0	\\
3.75e-09	0	\\
3.86e-09	0	\\
3.96e-09	0	\\
4.06e-09	0	\\
4.16e-09	0	\\
4.27e-09	0	\\
4.37e-09	0	\\
4.47e-09	0	\\
4.57e-09	0	\\
4.68e-09	0	\\
4.78e-09	0	\\
4.89e-09	0	\\
4.99e-09	0	\\
5e-09	0	\\
};
\addplot [color=blue,solid,forget plot]
  table[row sep=crcr]{
0	0	\\
1.1e-10	0	\\
2.2e-10	0	\\
3.3e-10	0	\\
4.4e-10	0	\\
5.4e-10	0	\\
6.5e-10	0	\\
7.5e-10	0	\\
8.6e-10	0	\\
9.6e-10	0	\\
1.07e-09	0	\\
1.18e-09	0	\\
1.28e-09	0	\\
1.38e-09	0	\\
1.49e-09	0	\\
1.59e-09	0	\\
1.69e-09	0	\\
1.8e-09	0	\\
1.9e-09	0	\\
2.01e-09	0	\\
2.11e-09	0	\\
2.21e-09	0	\\
2.32e-09	0	\\
2.42e-09	0	\\
2.52e-09	0	\\
2.63e-09	0	\\
2.73e-09	0	\\
2.83e-09	0	\\
2.93e-09	0	\\
3.04e-09	0	\\
3.14e-09	0	\\
3.24e-09	0	\\
3.34e-09	0	\\
3.45e-09	0	\\
3.55e-09	0	\\
3.65e-09	0	\\
3.75e-09	0	\\
3.86e-09	0	\\
3.96e-09	0	\\
4.06e-09	0	\\
4.16e-09	0	\\
4.27e-09	0	\\
4.37e-09	0	\\
4.47e-09	0	\\
4.57e-09	0	\\
4.68e-09	0	\\
4.78e-09	0	\\
4.89e-09	0	\\
4.99e-09	0	\\
5e-09	0	\\
};
\addplot [color=black!50!green,solid,forget plot]
  table[row sep=crcr]{
0	0	\\
1.1e-10	0	\\
2.2e-10	0	\\
3.3e-10	0	\\
4.4e-10	0	\\
5.4e-10	0	\\
6.5e-10	0	\\
7.5e-10	0	\\
8.6e-10	0	\\
9.6e-10	0	\\
1.07e-09	0	\\
1.18e-09	0	\\
1.28e-09	0	\\
1.38e-09	0	\\
1.49e-09	0	\\
1.59e-09	0	\\
1.69e-09	0	\\
1.8e-09	0	\\
1.9e-09	0	\\
2.01e-09	0	\\
2.11e-09	0	\\
2.21e-09	0	\\
2.32e-09	0	\\
2.42e-09	0	\\
2.52e-09	0	\\
2.63e-09	0	\\
2.73e-09	0	\\
2.83e-09	0	\\
2.93e-09	0	\\
3.04e-09	0	\\
3.14e-09	0	\\
3.24e-09	0	\\
3.34e-09	0	\\
3.45e-09	0	\\
3.55e-09	0	\\
3.65e-09	0	\\
3.75e-09	0	\\
3.86e-09	0	\\
3.96e-09	0	\\
4.06e-09	0	\\
4.16e-09	0	\\
4.27e-09	0	\\
4.37e-09	0	\\
4.47e-09	0	\\
4.57e-09	0	\\
4.68e-09	0	\\
4.78e-09	0	\\
4.89e-09	0	\\
4.99e-09	0	\\
5e-09	0	\\
};
\addplot [color=red,solid,forget plot]
  table[row sep=crcr]{
0	0	\\
1.1e-10	0	\\
2.2e-10	0	\\
3.3e-10	0	\\
4.4e-10	0	\\
5.4e-10	0	\\
6.5e-10	0	\\
7.5e-10	0	\\
8.6e-10	0	\\
9.6e-10	0	\\
1.07e-09	0	\\
1.18e-09	0	\\
1.28e-09	0	\\
1.38e-09	0	\\
1.49e-09	0	\\
1.59e-09	0	\\
1.69e-09	0	\\
1.8e-09	0	\\
1.9e-09	0	\\
2.01e-09	0	\\
2.11e-09	0	\\
2.21e-09	0	\\
2.32e-09	0	\\
2.42e-09	0	\\
2.52e-09	0	\\
2.63e-09	0	\\
2.73e-09	0	\\
2.83e-09	0	\\
2.93e-09	0	\\
3.04e-09	0	\\
3.14e-09	0	\\
3.24e-09	0	\\
3.34e-09	0	\\
3.45e-09	0	\\
3.55e-09	0	\\
3.65e-09	0	\\
3.75e-09	0	\\
3.86e-09	0	\\
3.96e-09	0	\\
4.06e-09	0	\\
4.16e-09	0	\\
4.27e-09	0	\\
4.37e-09	0	\\
4.47e-09	0	\\
4.57e-09	0	\\
4.68e-09	0	\\
4.78e-09	0	\\
4.89e-09	0	\\
4.99e-09	0	\\
5e-09	0	\\
};
\addplot [color=mycolor1,solid,forget plot]
  table[row sep=crcr]{
0	0	\\
1.1e-10	0	\\
2.2e-10	0	\\
3.3e-10	0	\\
4.4e-10	0	\\
5.4e-10	0	\\
6.5e-10	0	\\
7.5e-10	0	\\
8.6e-10	0	\\
9.6e-10	0	\\
1.07e-09	0	\\
1.18e-09	0	\\
1.28e-09	0	\\
1.38e-09	0	\\
1.49e-09	0	\\
1.59e-09	0	\\
1.69e-09	0	\\
1.8e-09	0	\\
1.9e-09	0	\\
2.01e-09	0	\\
2.11e-09	0	\\
2.21e-09	0	\\
2.32e-09	0	\\
2.42e-09	0	\\
2.52e-09	0	\\
2.63e-09	0	\\
2.73e-09	0	\\
2.83e-09	0	\\
2.93e-09	0	\\
3.04e-09	0	\\
3.14e-09	0	\\
3.24e-09	0	\\
3.34e-09	0	\\
3.45e-09	0	\\
3.55e-09	0	\\
3.65e-09	0	\\
3.75e-09	0	\\
3.86e-09	0	\\
3.96e-09	0	\\
4.06e-09	0	\\
4.16e-09	0	\\
4.27e-09	0	\\
4.37e-09	0	\\
4.47e-09	0	\\
4.57e-09	0	\\
4.68e-09	0	\\
4.78e-09	0	\\
4.89e-09	0	\\
4.99e-09	0	\\
5e-09	0	\\
};
\addplot [color=mycolor2,solid,forget plot]
  table[row sep=crcr]{
0	0	\\
1.1e-10	0	\\
2.2e-10	0	\\
3.3e-10	0	\\
4.4e-10	0	\\
5.4e-10	0	\\
6.5e-10	0	\\
7.5e-10	0	\\
8.6e-10	0	\\
9.6e-10	0	\\
1.07e-09	0	\\
1.18e-09	0	\\
1.28e-09	0	\\
1.38e-09	0	\\
1.49e-09	0	\\
1.59e-09	0	\\
1.69e-09	0	\\
1.8e-09	0	\\
1.9e-09	0	\\
2.01e-09	0	\\
2.11e-09	0	\\
2.21e-09	0	\\
2.32e-09	0	\\
2.42e-09	0	\\
2.52e-09	0	\\
2.63e-09	0	\\
2.73e-09	0	\\
2.83e-09	0	\\
2.93e-09	0	\\
3.04e-09	0	\\
3.14e-09	0	\\
3.24e-09	0	\\
3.34e-09	0	\\
3.45e-09	0	\\
3.55e-09	0	\\
3.65e-09	0	\\
3.75e-09	0	\\
3.86e-09	0	\\
3.96e-09	0	\\
4.06e-09	0	\\
4.16e-09	0	\\
4.27e-09	0	\\
4.37e-09	0	\\
4.47e-09	0	\\
4.57e-09	0	\\
4.68e-09	0	\\
4.78e-09	0	\\
4.89e-09	0	\\
4.99e-09	0	\\
5e-09	0	\\
};
\addplot [color=mycolor3,solid,forget plot]
  table[row sep=crcr]{
0	0	\\
1.1e-10	0	\\
2.2e-10	0	\\
3.3e-10	0	\\
4.4e-10	0	\\
5.4e-10	0	\\
6.5e-10	0	\\
7.5e-10	0	\\
8.6e-10	0	\\
9.6e-10	0	\\
1.07e-09	0	\\
1.18e-09	0	\\
1.28e-09	0	\\
1.38e-09	0	\\
1.49e-09	0	\\
1.59e-09	0	\\
1.69e-09	0	\\
1.8e-09	0	\\
1.9e-09	0	\\
2.01e-09	0	\\
2.11e-09	0	\\
2.21e-09	0	\\
2.32e-09	0	\\
2.42e-09	0	\\
2.52e-09	0	\\
2.63e-09	0	\\
2.73e-09	0	\\
2.83e-09	0	\\
2.93e-09	0	\\
3.04e-09	0	\\
3.14e-09	0	\\
3.24e-09	0	\\
3.34e-09	0	\\
3.45e-09	0	\\
3.55e-09	0	\\
3.65e-09	0	\\
3.75e-09	0	\\
3.86e-09	0	\\
3.96e-09	0	\\
4.06e-09	0	\\
4.16e-09	0	\\
4.27e-09	0	\\
4.37e-09	0	\\
4.47e-09	0	\\
4.57e-09	0	\\
4.68e-09	0	\\
4.78e-09	0	\\
4.89e-09	0	\\
4.99e-09	0	\\
5e-09	0	\\
};
\addplot [color=darkgray,solid,forget plot]
  table[row sep=crcr]{
0	0	\\
1.1e-10	0	\\
2.2e-10	0	\\
3.3e-10	0	\\
4.4e-10	0	\\
5.4e-10	0	\\
6.5e-10	0	\\
7.5e-10	0	\\
8.6e-10	0	\\
9.6e-10	0	\\
1.07e-09	0	\\
1.18e-09	0	\\
1.28e-09	0	\\
1.38e-09	0	\\
1.49e-09	0	\\
1.59e-09	0	\\
1.69e-09	0	\\
1.8e-09	0	\\
1.9e-09	0	\\
2.01e-09	0	\\
2.11e-09	0	\\
2.21e-09	0	\\
2.32e-09	0	\\
2.42e-09	0	\\
2.52e-09	0	\\
2.63e-09	0	\\
2.73e-09	0	\\
2.83e-09	0	\\
2.93e-09	0	\\
3.04e-09	0	\\
3.14e-09	0	\\
3.24e-09	0	\\
3.34e-09	0	\\
3.45e-09	0	\\
3.55e-09	0	\\
3.65e-09	0	\\
3.75e-09	0	\\
3.86e-09	0	\\
3.96e-09	0	\\
4.06e-09	0	\\
4.16e-09	0	\\
4.27e-09	0	\\
4.37e-09	0	\\
4.47e-09	0	\\
4.57e-09	0	\\
4.68e-09	0	\\
4.78e-09	0	\\
4.89e-09	0	\\
4.99e-09	0	\\
5e-09	0	\\
};
\addplot [color=blue,solid,forget plot]
  table[row sep=crcr]{
0	0	\\
1.1e-10	0	\\
2.2e-10	0	\\
3.3e-10	0	\\
4.4e-10	0	\\
5.4e-10	0	\\
6.5e-10	0	\\
7.5e-10	0	\\
8.6e-10	0	\\
9.6e-10	0	\\
1.07e-09	0	\\
1.18e-09	0	\\
1.28e-09	0	\\
1.38e-09	0	\\
1.49e-09	0	\\
1.59e-09	0	\\
1.69e-09	0	\\
1.8e-09	0	\\
1.9e-09	0	\\
2.01e-09	0	\\
2.11e-09	0	\\
2.21e-09	0	\\
2.32e-09	0	\\
2.42e-09	0	\\
2.52e-09	0	\\
2.63e-09	0	\\
2.73e-09	0	\\
2.83e-09	0	\\
2.93e-09	0	\\
3.04e-09	0	\\
3.14e-09	0	\\
3.24e-09	0	\\
3.34e-09	0	\\
3.45e-09	0	\\
3.55e-09	0	\\
3.65e-09	0	\\
3.75e-09	0	\\
3.86e-09	0	\\
3.96e-09	0	\\
4.06e-09	0	\\
4.16e-09	0	\\
4.27e-09	0	\\
4.37e-09	0	\\
4.47e-09	0	\\
4.57e-09	0	\\
4.68e-09	0	\\
4.78e-09	0	\\
4.89e-09	0	\\
4.99e-09	0	\\
5e-09	0	\\
};
\addplot [color=black!50!green,solid,forget plot]
  table[row sep=crcr]{
0	0	\\
1.1e-10	0	\\
2.2e-10	0	\\
3.3e-10	0	\\
4.4e-10	0	\\
5.4e-10	0	\\
6.5e-10	0	\\
7.5e-10	0	\\
8.6e-10	0	\\
9.6e-10	0	\\
1.07e-09	0	\\
1.18e-09	0	\\
1.28e-09	0	\\
1.38e-09	0	\\
1.49e-09	0	\\
1.59e-09	0	\\
1.69e-09	0	\\
1.8e-09	0	\\
1.9e-09	0	\\
2.01e-09	0	\\
2.11e-09	0	\\
2.21e-09	0	\\
2.32e-09	0	\\
2.42e-09	0	\\
2.52e-09	0	\\
2.63e-09	0	\\
2.73e-09	0	\\
2.83e-09	0	\\
2.93e-09	0	\\
3.04e-09	0	\\
3.14e-09	0	\\
3.24e-09	0	\\
3.34e-09	0	\\
3.45e-09	0	\\
3.55e-09	0	\\
3.65e-09	0	\\
3.75e-09	0	\\
3.86e-09	0	\\
3.96e-09	0	\\
4.06e-09	0	\\
4.16e-09	0	\\
4.27e-09	0	\\
4.37e-09	0	\\
4.47e-09	0	\\
4.57e-09	0	\\
4.68e-09	0	\\
4.78e-09	0	\\
4.89e-09	0	\\
4.99e-09	0	\\
5e-09	0	\\
};
\addplot [color=red,solid,forget plot]
  table[row sep=crcr]{
0	0	\\
1.1e-10	0	\\
2.2e-10	0	\\
3.3e-10	0	\\
4.4e-10	0	\\
5.4e-10	0	\\
6.5e-10	0	\\
7.5e-10	0	\\
8.6e-10	0	\\
9.6e-10	0	\\
1.07e-09	0	\\
1.18e-09	0	\\
1.28e-09	0	\\
1.38e-09	0	\\
1.49e-09	0	\\
1.59e-09	0	\\
1.69e-09	0	\\
1.8e-09	0	\\
1.9e-09	0	\\
2.01e-09	0	\\
2.11e-09	0	\\
2.21e-09	0	\\
2.32e-09	0	\\
2.42e-09	0	\\
2.52e-09	0	\\
2.63e-09	0	\\
2.73e-09	0	\\
2.83e-09	0	\\
2.93e-09	0	\\
3.04e-09	0	\\
3.14e-09	0	\\
3.24e-09	0	\\
3.34e-09	0	\\
3.45e-09	0	\\
3.55e-09	0	\\
3.65e-09	0	\\
3.75e-09	0	\\
3.86e-09	0	\\
3.96e-09	0	\\
4.06e-09	0	\\
4.16e-09	0	\\
4.27e-09	0	\\
4.37e-09	0	\\
4.47e-09	0	\\
4.57e-09	0	\\
4.68e-09	0	\\
4.78e-09	0	\\
4.89e-09	0	\\
4.99e-09	0	\\
5e-09	0	\\
};
\addplot [color=mycolor1,solid,forget plot]
  table[row sep=crcr]{
0	0	\\
1.1e-10	0	\\
2.2e-10	0	\\
3.3e-10	0	\\
4.4e-10	0	\\
5.4e-10	0	\\
6.5e-10	0	\\
7.5e-10	0	\\
8.6e-10	0	\\
9.6e-10	0	\\
1.07e-09	0	\\
1.18e-09	0	\\
1.28e-09	0	\\
1.38e-09	0	\\
1.49e-09	0	\\
1.59e-09	0	\\
1.69e-09	0	\\
1.8e-09	0	\\
1.9e-09	0	\\
2.01e-09	0	\\
2.11e-09	0	\\
2.21e-09	0	\\
2.32e-09	0	\\
2.42e-09	0	\\
2.52e-09	0	\\
2.63e-09	0	\\
2.73e-09	0	\\
2.83e-09	0	\\
2.93e-09	0	\\
3.04e-09	0	\\
3.14e-09	0	\\
3.24e-09	0	\\
3.34e-09	0	\\
3.45e-09	0	\\
3.55e-09	0	\\
3.65e-09	0	\\
3.75e-09	0	\\
3.86e-09	0	\\
3.96e-09	0	\\
4.06e-09	0	\\
4.16e-09	0	\\
4.27e-09	0	\\
4.37e-09	0	\\
4.47e-09	0	\\
4.57e-09	0	\\
4.68e-09	0	\\
4.78e-09	0	\\
4.89e-09	0	\\
4.99e-09	0	\\
5e-09	0	\\
};
\addplot [color=mycolor2,solid,forget plot]
  table[row sep=crcr]{
0	0	\\
1.1e-10	0	\\
2.2e-10	0	\\
3.3e-10	0	\\
4.4e-10	0	\\
5.4e-10	0	\\
6.5e-10	0	\\
7.5e-10	0	\\
8.6e-10	0	\\
9.6e-10	0	\\
1.07e-09	0	\\
1.18e-09	0	\\
1.28e-09	0	\\
1.38e-09	0	\\
1.49e-09	0	\\
1.59e-09	0	\\
1.69e-09	0	\\
1.8e-09	0	\\
1.9e-09	0	\\
2.01e-09	0	\\
2.11e-09	0	\\
2.21e-09	0	\\
2.32e-09	0	\\
2.42e-09	0	\\
2.52e-09	0	\\
2.63e-09	0	\\
2.73e-09	0	\\
2.83e-09	0	\\
2.93e-09	0	\\
3.04e-09	0	\\
3.14e-09	0	\\
3.24e-09	0	\\
3.34e-09	0	\\
3.45e-09	0	\\
3.55e-09	0	\\
3.65e-09	0	\\
3.75e-09	0	\\
3.86e-09	0	\\
3.96e-09	0	\\
4.06e-09	0	\\
4.16e-09	0	\\
4.27e-09	0	\\
4.37e-09	0	\\
4.47e-09	0	\\
4.57e-09	0	\\
4.68e-09	0	\\
4.78e-09	0	\\
4.89e-09	0	\\
4.99e-09	0	\\
5e-09	0	\\
};
\addplot [color=mycolor3,solid,forget plot]
  table[row sep=crcr]{
0	0	\\
1.1e-10	0	\\
2.2e-10	0	\\
3.3e-10	0	\\
4.4e-10	0	\\
5.4e-10	0	\\
6.5e-10	0	\\
7.5e-10	0	\\
8.6e-10	0	\\
9.6e-10	0	\\
1.07e-09	0	\\
1.18e-09	0	\\
1.28e-09	0	\\
1.38e-09	0	\\
1.49e-09	0	\\
1.59e-09	0	\\
1.69e-09	0	\\
1.8e-09	0	\\
1.9e-09	0	\\
2.01e-09	0	\\
2.11e-09	0	\\
2.21e-09	0	\\
2.32e-09	0	\\
2.42e-09	0	\\
2.52e-09	0	\\
2.63e-09	0	\\
2.73e-09	0	\\
2.83e-09	0	\\
2.93e-09	0	\\
3.04e-09	0	\\
3.14e-09	0	\\
3.24e-09	0	\\
3.34e-09	0	\\
3.45e-09	0	\\
3.55e-09	0	\\
3.65e-09	0	\\
3.75e-09	0	\\
3.86e-09	0	\\
3.96e-09	0	\\
4.06e-09	0	\\
4.16e-09	0	\\
4.27e-09	0	\\
4.37e-09	0	\\
4.47e-09	0	\\
4.57e-09	0	\\
4.68e-09	0	\\
4.78e-09	0	\\
4.89e-09	0	\\
4.99e-09	0	\\
5e-09	0	\\
};
\addplot [color=darkgray,solid,forget plot]
  table[row sep=crcr]{
0	0	\\
1.1e-10	0	\\
2.2e-10	0	\\
3.3e-10	0	\\
4.4e-10	0	\\
5.4e-10	0	\\
6.5e-10	0	\\
7.5e-10	0	\\
8.6e-10	0	\\
9.6e-10	0	\\
1.07e-09	0	\\
1.18e-09	0	\\
1.28e-09	0	\\
1.38e-09	0	\\
1.49e-09	0	\\
1.59e-09	0	\\
1.69e-09	0	\\
1.8e-09	0	\\
1.9e-09	0	\\
2.01e-09	0	\\
2.11e-09	0	\\
2.21e-09	0	\\
2.32e-09	0	\\
2.42e-09	0	\\
2.52e-09	0	\\
2.63e-09	0	\\
2.73e-09	0	\\
2.83e-09	0	\\
2.93e-09	0	\\
3.04e-09	0	\\
3.14e-09	0	\\
3.24e-09	0	\\
3.34e-09	0	\\
3.45e-09	0	\\
3.55e-09	0	\\
3.65e-09	0	\\
3.75e-09	0	\\
3.86e-09	0	\\
3.96e-09	0	\\
4.06e-09	0	\\
4.16e-09	0	\\
4.27e-09	0	\\
4.37e-09	0	\\
4.47e-09	0	\\
4.57e-09	0	\\
4.68e-09	0	\\
4.78e-09	0	\\
4.89e-09	0	\\
4.99e-09	0	\\
5e-09	0	\\
};
\addplot [color=blue,solid,forget plot]
  table[row sep=crcr]{
0	0	\\
1.1e-10	0	\\
2.2e-10	0	\\
3.3e-10	0	\\
4.4e-10	0	\\
5.4e-10	0	\\
6.5e-10	0	\\
7.5e-10	0	\\
8.6e-10	0	\\
9.6e-10	0	\\
1.07e-09	0	\\
1.18e-09	0	\\
1.28e-09	0	\\
1.38e-09	0	\\
1.49e-09	0	\\
1.59e-09	0	\\
1.69e-09	0	\\
1.8e-09	0	\\
1.9e-09	0	\\
2.01e-09	0	\\
2.11e-09	0	\\
2.21e-09	0	\\
2.32e-09	0	\\
2.42e-09	0	\\
2.52e-09	0	\\
2.63e-09	0	\\
2.73e-09	0	\\
2.83e-09	0	\\
2.93e-09	0	\\
3.04e-09	0	\\
3.14e-09	0	\\
3.24e-09	0	\\
3.34e-09	0	\\
3.45e-09	0	\\
3.55e-09	0	\\
3.65e-09	0	\\
3.75e-09	0	\\
3.86e-09	0	\\
3.96e-09	0	\\
4.06e-09	0	\\
4.16e-09	0	\\
4.27e-09	0	\\
4.37e-09	0	\\
4.47e-09	0	\\
4.57e-09	0	\\
4.68e-09	0	\\
4.78e-09	0	\\
4.89e-09	0	\\
4.99e-09	0	\\
5e-09	0	\\
};
\addplot [color=black!50!green,solid,forget plot]
  table[row sep=crcr]{
0	0	\\
1.1e-10	0	\\
2.2e-10	0	\\
3.3e-10	0	\\
4.4e-10	0	\\
5.4e-10	0	\\
6.5e-10	0	\\
7.5e-10	0	\\
8.6e-10	0	\\
9.6e-10	0	\\
1.07e-09	0	\\
1.18e-09	0	\\
1.28e-09	0	\\
1.38e-09	0	\\
1.49e-09	0	\\
1.59e-09	0	\\
1.69e-09	0	\\
1.8e-09	0	\\
1.9e-09	0	\\
2.01e-09	0	\\
2.11e-09	0	\\
2.21e-09	0	\\
2.32e-09	0	\\
2.42e-09	0	\\
2.52e-09	0	\\
2.63e-09	0	\\
2.73e-09	0	\\
2.83e-09	0	\\
2.93e-09	0	\\
3.04e-09	0	\\
3.14e-09	0	\\
3.24e-09	0	\\
3.34e-09	0	\\
3.45e-09	0	\\
3.55e-09	0	\\
3.65e-09	0	\\
3.75e-09	0	\\
3.86e-09	0	\\
3.96e-09	0	\\
4.06e-09	0	\\
4.16e-09	0	\\
4.27e-09	0	\\
4.37e-09	0	\\
4.47e-09	0	\\
4.57e-09	0	\\
4.68e-09	0	\\
4.78e-09	0	\\
4.89e-09	0	\\
4.99e-09	0	\\
5e-09	0	\\
};
\addplot [color=red,solid,forget plot]
  table[row sep=crcr]{
0	0	\\
1.1e-10	0	\\
2.2e-10	0	\\
3.3e-10	0	\\
4.4e-10	0	\\
5.4e-10	0	\\
6.5e-10	0	\\
7.5e-10	0	\\
8.6e-10	0	\\
9.6e-10	0	\\
1.07e-09	0	\\
1.18e-09	0	\\
1.28e-09	0	\\
1.38e-09	0	\\
1.49e-09	0	\\
1.59e-09	0	\\
1.69e-09	0	\\
1.8e-09	0	\\
1.9e-09	0	\\
2.01e-09	0	\\
2.11e-09	0	\\
2.21e-09	0	\\
2.32e-09	0	\\
2.42e-09	0	\\
2.52e-09	0	\\
2.63e-09	0	\\
2.73e-09	0	\\
2.83e-09	0	\\
2.93e-09	0	\\
3.04e-09	0	\\
3.14e-09	0	\\
3.24e-09	0	\\
3.34e-09	0	\\
3.45e-09	0	\\
3.55e-09	0	\\
3.65e-09	0	\\
3.75e-09	0	\\
3.86e-09	0	\\
3.96e-09	0	\\
4.06e-09	0	\\
4.16e-09	0	\\
4.27e-09	0	\\
4.37e-09	0	\\
4.47e-09	0	\\
4.57e-09	0	\\
4.68e-09	0	\\
4.78e-09	0	\\
4.89e-09	0	\\
4.99e-09	0	\\
5e-09	0	\\
};
\addplot [color=mycolor1,solid,forget plot]
  table[row sep=crcr]{
0	0	\\
1.1e-10	0	\\
2.2e-10	0	\\
3.3e-10	0	\\
4.4e-10	0	\\
5.4e-10	0	\\
6.5e-10	0	\\
7.5e-10	0	\\
8.6e-10	0	\\
9.6e-10	0	\\
1.07e-09	0	\\
1.18e-09	0	\\
1.28e-09	0	\\
1.38e-09	0	\\
1.49e-09	0	\\
1.59e-09	0	\\
1.69e-09	0	\\
1.8e-09	0	\\
1.9e-09	0	\\
2.01e-09	0	\\
2.11e-09	0	\\
2.21e-09	0	\\
2.32e-09	0	\\
2.42e-09	0	\\
2.52e-09	0	\\
2.63e-09	0	\\
2.73e-09	0	\\
2.83e-09	0	\\
2.93e-09	0	\\
3.04e-09	0	\\
3.14e-09	0	\\
3.24e-09	0	\\
3.34e-09	0	\\
3.45e-09	0	\\
3.55e-09	0	\\
3.65e-09	0	\\
3.75e-09	0	\\
3.86e-09	0	\\
3.96e-09	0	\\
4.06e-09	0	\\
4.16e-09	0	\\
4.27e-09	0	\\
4.37e-09	0	\\
4.47e-09	0	\\
4.57e-09	0	\\
4.68e-09	0	\\
4.78e-09	0	\\
4.89e-09	0	\\
4.99e-09	0	\\
5e-09	0	\\
};
\addplot [color=mycolor2,solid,forget plot]
  table[row sep=crcr]{
0	0	\\
1.1e-10	0	\\
2.2e-10	0	\\
3.3e-10	0	\\
4.4e-10	0	\\
5.4e-10	0	\\
6.5e-10	0	\\
7.5e-10	0	\\
8.6e-10	0	\\
9.6e-10	0	\\
1.07e-09	0	\\
1.18e-09	0	\\
1.28e-09	0	\\
1.38e-09	0	\\
1.49e-09	0	\\
1.59e-09	0	\\
1.69e-09	0	\\
1.8e-09	0	\\
1.9e-09	0	\\
2.01e-09	0	\\
2.11e-09	0	\\
2.21e-09	0	\\
2.32e-09	0	\\
2.42e-09	0	\\
2.52e-09	0	\\
2.63e-09	0	\\
2.73e-09	0	\\
2.83e-09	0	\\
2.93e-09	0	\\
3.04e-09	0	\\
3.14e-09	0	\\
3.24e-09	0	\\
3.34e-09	0	\\
3.45e-09	0	\\
3.55e-09	0	\\
3.65e-09	0	\\
3.75e-09	0	\\
3.86e-09	0	\\
3.96e-09	0	\\
4.06e-09	0	\\
4.16e-09	0	\\
4.27e-09	0	\\
4.37e-09	0	\\
4.47e-09	0	\\
4.57e-09	0	\\
4.68e-09	0	\\
4.78e-09	0	\\
4.89e-09	0	\\
4.99e-09	0	\\
5e-09	0	\\
};
\addplot [color=mycolor3,solid,forget plot]
  table[row sep=crcr]{
0	0	\\
1.1e-10	0	\\
2.2e-10	0	\\
3.3e-10	0	\\
4.4e-10	0	\\
5.4e-10	0	\\
6.5e-10	0	\\
7.5e-10	0	\\
8.6e-10	0	\\
9.6e-10	0	\\
1.07e-09	0	\\
1.18e-09	0	\\
1.28e-09	0	\\
1.38e-09	0	\\
1.49e-09	0	\\
1.59e-09	0	\\
1.69e-09	0	\\
1.8e-09	0	\\
1.9e-09	0	\\
2.01e-09	0	\\
2.11e-09	0	\\
2.21e-09	0	\\
2.32e-09	0	\\
2.42e-09	0	\\
2.52e-09	0	\\
2.63e-09	0	\\
2.73e-09	0	\\
2.83e-09	0	\\
2.93e-09	0	\\
3.04e-09	0	\\
3.14e-09	0	\\
3.24e-09	0	\\
3.34e-09	0	\\
3.45e-09	0	\\
3.55e-09	0	\\
3.65e-09	0	\\
3.75e-09	0	\\
3.86e-09	0	\\
3.96e-09	0	\\
4.06e-09	0	\\
4.16e-09	0	\\
4.27e-09	0	\\
4.37e-09	0	\\
4.47e-09	0	\\
4.57e-09	0	\\
4.68e-09	0	\\
4.78e-09	0	\\
4.89e-09	0	\\
4.99e-09	0	\\
5e-09	0	\\
};
\addplot [color=darkgray,solid,forget plot]
  table[row sep=crcr]{
0	0	\\
1.1e-10	0	\\
2.2e-10	0	\\
3.3e-10	0	\\
4.4e-10	0	\\
5.4e-10	0	\\
6.5e-10	0	\\
7.5e-10	0	\\
8.6e-10	0	\\
9.6e-10	0	\\
1.07e-09	0	\\
1.18e-09	0	\\
1.28e-09	0	\\
1.38e-09	0	\\
1.49e-09	0	\\
1.59e-09	0	\\
1.69e-09	0	\\
1.8e-09	0	\\
1.9e-09	0	\\
2.01e-09	0	\\
2.11e-09	0	\\
2.21e-09	0	\\
2.32e-09	0	\\
2.42e-09	0	\\
2.52e-09	0	\\
2.63e-09	0	\\
2.73e-09	0	\\
2.83e-09	0	\\
2.93e-09	0	\\
3.04e-09	0	\\
3.14e-09	0	\\
3.24e-09	0	\\
3.34e-09	0	\\
3.45e-09	0	\\
3.55e-09	0	\\
3.65e-09	0	\\
3.75e-09	0	\\
3.86e-09	0	\\
3.96e-09	0	\\
4.06e-09	0	\\
4.16e-09	0	\\
4.27e-09	0	\\
4.37e-09	0	\\
4.47e-09	0	\\
4.57e-09	0	\\
4.68e-09	0	\\
4.78e-09	0	\\
4.89e-09	0	\\
4.99e-09	0	\\
5e-09	0	\\
};
\addplot [color=blue,solid,forget plot]
  table[row sep=crcr]{
0	0	\\
1.1e-10	0	\\
2.2e-10	0	\\
3.3e-10	0	\\
4.4e-10	0	\\
5.4e-10	0	\\
6.5e-10	0	\\
7.5e-10	0	\\
8.6e-10	0	\\
9.6e-10	0	\\
1.07e-09	0	\\
1.18e-09	0	\\
1.28e-09	0	\\
1.38e-09	0	\\
1.49e-09	0	\\
1.59e-09	0	\\
1.69e-09	0	\\
1.8e-09	0	\\
1.9e-09	0	\\
2.01e-09	0	\\
2.11e-09	0	\\
2.21e-09	0	\\
2.32e-09	0	\\
2.42e-09	0	\\
2.52e-09	0	\\
2.63e-09	0	\\
2.73e-09	0	\\
2.83e-09	0	\\
2.93e-09	0	\\
3.04e-09	0	\\
3.14e-09	0	\\
3.24e-09	0	\\
3.34e-09	0	\\
3.45e-09	0	\\
3.55e-09	0	\\
3.65e-09	0	\\
3.75e-09	0	\\
3.86e-09	0	\\
3.96e-09	0	\\
4.06e-09	0	\\
4.16e-09	0	\\
4.27e-09	0	\\
4.37e-09	0	\\
4.47e-09	0	\\
4.57e-09	0	\\
4.68e-09	0	\\
4.78e-09	0	\\
4.89e-09	0	\\
4.99e-09	0	\\
5e-09	0	\\
};
\addplot [color=black!50!green,solid,forget plot]
  table[row sep=crcr]{
0	0	\\
1.1e-10	0	\\
2.2e-10	0	\\
3.3e-10	0	\\
4.4e-10	0	\\
5.4e-10	0	\\
6.5e-10	0	\\
7.5e-10	0	\\
8.6e-10	0	\\
9.6e-10	0	\\
1.07e-09	0	\\
1.18e-09	0	\\
1.28e-09	0	\\
1.38e-09	0	\\
1.49e-09	0	\\
1.59e-09	0	\\
1.69e-09	0	\\
1.8e-09	0	\\
1.9e-09	0	\\
2.01e-09	0	\\
2.11e-09	0	\\
2.21e-09	0	\\
2.32e-09	0	\\
2.42e-09	0	\\
2.52e-09	0	\\
2.63e-09	0	\\
2.73e-09	0	\\
2.83e-09	0	\\
2.93e-09	0	\\
3.04e-09	0	\\
3.14e-09	0	\\
3.24e-09	0	\\
3.34e-09	0	\\
3.45e-09	0	\\
3.55e-09	0	\\
3.65e-09	0	\\
3.75e-09	0	\\
3.86e-09	0	\\
3.96e-09	0	\\
4.06e-09	0	\\
4.16e-09	0	\\
4.27e-09	0	\\
4.37e-09	0	\\
4.47e-09	0	\\
4.57e-09	0	\\
4.68e-09	0	\\
4.78e-09	0	\\
4.89e-09	0	\\
4.99e-09	0	\\
5e-09	0	\\
};
\addplot [color=red,solid,forget plot]
  table[row sep=crcr]{
0	0	\\
1.1e-10	0	\\
2.2e-10	0	\\
3.3e-10	0	\\
4.4e-10	0	\\
5.4e-10	0	\\
6.5e-10	0	\\
7.5e-10	0	\\
8.6e-10	0	\\
9.6e-10	0	\\
1.07e-09	0	\\
1.18e-09	0	\\
1.28e-09	0	\\
1.38e-09	0	\\
1.49e-09	0	\\
1.59e-09	0	\\
1.69e-09	0	\\
1.8e-09	0	\\
1.9e-09	0	\\
2.01e-09	0	\\
2.11e-09	0	\\
2.21e-09	0	\\
2.32e-09	0	\\
2.42e-09	0	\\
2.52e-09	0	\\
2.63e-09	0	\\
2.73e-09	0	\\
2.83e-09	0	\\
2.93e-09	0	\\
3.04e-09	0	\\
3.14e-09	0	\\
3.24e-09	0	\\
3.34e-09	0	\\
3.45e-09	0	\\
3.55e-09	0	\\
3.65e-09	0	\\
3.75e-09	0	\\
3.86e-09	0	\\
3.96e-09	0	\\
4.06e-09	0	\\
4.16e-09	0	\\
4.27e-09	0	\\
4.37e-09	0	\\
4.47e-09	0	\\
4.57e-09	0	\\
4.68e-09	0	\\
4.78e-09	0	\\
4.89e-09	0	\\
4.99e-09	0	\\
5e-09	0	\\
};
\addplot [color=mycolor1,solid,forget plot]
  table[row sep=crcr]{
0	0	\\
1.1e-10	0	\\
2.2e-10	0	\\
3.3e-10	0	\\
4.4e-10	0	\\
5.4e-10	0	\\
6.5e-10	0	\\
7.5e-10	0	\\
8.6e-10	0	\\
9.6e-10	0	\\
1.07e-09	0	\\
1.18e-09	0	\\
1.28e-09	0	\\
1.38e-09	0	\\
1.49e-09	0	\\
1.59e-09	0	\\
1.69e-09	0	\\
1.8e-09	0	\\
1.9e-09	0	\\
2.01e-09	0	\\
2.11e-09	0	\\
2.21e-09	0	\\
2.32e-09	0	\\
2.42e-09	0	\\
2.52e-09	0	\\
2.63e-09	0	\\
2.73e-09	0	\\
2.83e-09	0	\\
2.93e-09	0	\\
3.04e-09	0	\\
3.14e-09	0	\\
3.24e-09	0	\\
3.34e-09	0	\\
3.45e-09	0	\\
3.55e-09	0	\\
3.65e-09	0	\\
3.75e-09	0	\\
3.86e-09	0	\\
3.96e-09	0	\\
4.06e-09	0	\\
4.16e-09	0	\\
4.27e-09	0	\\
4.37e-09	0	\\
4.47e-09	0	\\
4.57e-09	0	\\
4.68e-09	0	\\
4.78e-09	0	\\
4.89e-09	0	\\
4.99e-09	0	\\
5e-09	0	\\
};
\addplot [color=mycolor2,solid,forget plot]
  table[row sep=crcr]{
0	0	\\
1.1e-10	0	\\
2.2e-10	0	\\
3.3e-10	0	\\
4.4e-10	0	\\
5.4e-10	0	\\
6.5e-10	0	\\
7.5e-10	0	\\
8.6e-10	0	\\
9.6e-10	0	\\
1.07e-09	0	\\
1.18e-09	0	\\
1.28e-09	0	\\
1.38e-09	0	\\
1.49e-09	0	\\
1.59e-09	0	\\
1.69e-09	0	\\
1.8e-09	0	\\
1.9e-09	0	\\
2.01e-09	0	\\
2.11e-09	0	\\
2.21e-09	0	\\
2.32e-09	0	\\
2.42e-09	0	\\
2.52e-09	0	\\
2.63e-09	0	\\
2.73e-09	0	\\
2.83e-09	0	\\
2.93e-09	0	\\
3.04e-09	0	\\
3.14e-09	0	\\
3.24e-09	0	\\
3.34e-09	0	\\
3.45e-09	0	\\
3.55e-09	0	\\
3.65e-09	0	\\
3.75e-09	0	\\
3.86e-09	0	\\
3.96e-09	0	\\
4.06e-09	0	\\
4.16e-09	0	\\
4.27e-09	0	\\
4.37e-09	0	\\
4.47e-09	0	\\
4.57e-09	0	\\
4.68e-09	0	\\
4.78e-09	0	\\
4.89e-09	0	\\
4.99e-09	0	\\
5e-09	0	\\
};
\addplot [color=mycolor3,solid,forget plot]
  table[row sep=crcr]{
0	0	\\
1.1e-10	0	\\
2.2e-10	0	\\
3.3e-10	0	\\
4.4e-10	0	\\
5.4e-10	0	\\
6.5e-10	0	\\
7.5e-10	0	\\
8.6e-10	0	\\
9.6e-10	0	\\
1.07e-09	0	\\
1.18e-09	0	\\
1.28e-09	0	\\
1.38e-09	0	\\
1.49e-09	0	\\
1.59e-09	0	\\
1.69e-09	0	\\
1.8e-09	0	\\
1.9e-09	0	\\
2.01e-09	0	\\
2.11e-09	0	\\
2.21e-09	0	\\
2.32e-09	0	\\
2.42e-09	0	\\
2.52e-09	0	\\
2.63e-09	0	\\
2.73e-09	0	\\
2.83e-09	0	\\
2.93e-09	0	\\
3.04e-09	0	\\
3.14e-09	0	\\
3.24e-09	0	\\
3.34e-09	0	\\
3.45e-09	0	\\
3.55e-09	0	\\
3.65e-09	0	\\
3.75e-09	0	\\
3.86e-09	0	\\
3.96e-09	0	\\
4.06e-09	0	\\
4.16e-09	0	\\
4.27e-09	0	\\
4.37e-09	0	\\
4.47e-09	0	\\
4.57e-09	0	\\
4.68e-09	0	\\
4.78e-09	0	\\
4.89e-09	0	\\
4.99e-09	0	\\
5e-09	0	\\
};
\addplot [color=darkgray,solid,forget plot]
  table[row sep=crcr]{
0	0	\\
1.1e-10	0	\\
2.2e-10	0	\\
3.3e-10	0	\\
4.4e-10	0	\\
5.4e-10	0	\\
6.5e-10	0	\\
7.5e-10	0	\\
8.6e-10	0	\\
9.6e-10	0	\\
1.07e-09	0	\\
1.18e-09	0	\\
1.28e-09	0	\\
1.38e-09	0	\\
1.49e-09	0	\\
1.59e-09	0	\\
1.69e-09	0	\\
1.8e-09	0	\\
1.9e-09	0	\\
2.01e-09	0	\\
2.11e-09	0	\\
2.21e-09	0	\\
2.32e-09	0	\\
2.42e-09	0	\\
2.52e-09	0	\\
2.63e-09	0	\\
2.73e-09	0	\\
2.83e-09	0	\\
2.93e-09	0	\\
3.04e-09	0	\\
3.14e-09	0	\\
3.24e-09	0	\\
3.34e-09	0	\\
3.45e-09	0	\\
3.55e-09	0	\\
3.65e-09	0	\\
3.75e-09	0	\\
3.86e-09	0	\\
3.96e-09	0	\\
4.06e-09	0	\\
4.16e-09	0	\\
4.27e-09	0	\\
4.37e-09	0	\\
4.47e-09	0	\\
4.57e-09	0	\\
4.68e-09	0	\\
4.78e-09	0	\\
4.89e-09	0	\\
4.99e-09	0	\\
5e-09	0	\\
};
\addplot [color=blue,solid,forget plot]
  table[row sep=crcr]{
0	0	\\
1.1e-10	0	\\
2.2e-10	0	\\
3.3e-10	0	\\
4.4e-10	0	\\
5.4e-10	0	\\
6.5e-10	0	\\
7.5e-10	0	\\
8.6e-10	0	\\
9.6e-10	0	\\
1.07e-09	0	\\
1.18e-09	0	\\
1.28e-09	0	\\
1.38e-09	0	\\
1.49e-09	0	\\
1.59e-09	0	\\
1.69e-09	0	\\
1.8e-09	0	\\
1.9e-09	0	\\
2.01e-09	0	\\
2.11e-09	0	\\
2.21e-09	0	\\
2.32e-09	0	\\
2.42e-09	0	\\
2.52e-09	0	\\
2.63e-09	0	\\
2.73e-09	0	\\
2.83e-09	0	\\
2.93e-09	0	\\
3.04e-09	0	\\
3.14e-09	0	\\
3.24e-09	0	\\
3.34e-09	0	\\
3.45e-09	0	\\
3.55e-09	0	\\
3.65e-09	0	\\
3.75e-09	0	\\
3.86e-09	0	\\
3.96e-09	0	\\
4.06e-09	0	\\
4.16e-09	0	\\
4.27e-09	0	\\
4.37e-09	0	\\
4.47e-09	0	\\
4.57e-09	0	\\
4.68e-09	0	\\
4.78e-09	0	\\
4.89e-09	0	\\
4.99e-09	0	\\
5e-09	0	\\
};
\addplot [color=black!50!green,solid,forget plot]
  table[row sep=crcr]{
0	0	\\
1.1e-10	0	\\
2.2e-10	0	\\
3.3e-10	0	\\
4.4e-10	0	\\
5.4e-10	0	\\
6.5e-10	0	\\
7.5e-10	0	\\
8.6e-10	0	\\
9.6e-10	0	\\
1.07e-09	0	\\
1.18e-09	0	\\
1.28e-09	0	\\
1.38e-09	0	\\
1.49e-09	0	\\
1.59e-09	0	\\
1.69e-09	0	\\
1.8e-09	0	\\
1.9e-09	0	\\
2.01e-09	0	\\
2.11e-09	0	\\
2.21e-09	0	\\
2.32e-09	0	\\
2.42e-09	0	\\
2.52e-09	0	\\
2.63e-09	0	\\
2.73e-09	0	\\
2.83e-09	0	\\
2.93e-09	0	\\
3.04e-09	0	\\
3.14e-09	0	\\
3.24e-09	0	\\
3.34e-09	0	\\
3.45e-09	0	\\
3.55e-09	0	\\
3.65e-09	0	\\
3.75e-09	0	\\
3.86e-09	0	\\
3.96e-09	0	\\
4.06e-09	0	\\
4.16e-09	0	\\
4.27e-09	0	\\
4.37e-09	0	\\
4.47e-09	0	\\
4.57e-09	0	\\
4.68e-09	0	\\
4.78e-09	0	\\
4.89e-09	0	\\
4.99e-09	0	\\
5e-09	0	\\
};
\addplot [color=red,solid,forget plot]
  table[row sep=crcr]{
0	0	\\
1.1e-10	0	\\
2.2e-10	0	\\
3.3e-10	0	\\
4.4e-10	0	\\
5.4e-10	0	\\
6.5e-10	0	\\
7.5e-10	0	\\
8.6e-10	0	\\
9.6e-10	0	\\
1.07e-09	0	\\
1.18e-09	0	\\
1.28e-09	0	\\
1.38e-09	0	\\
1.49e-09	0	\\
1.59e-09	0	\\
1.69e-09	0	\\
1.8e-09	0	\\
1.9e-09	0	\\
2.01e-09	0	\\
2.11e-09	0	\\
2.21e-09	0	\\
2.32e-09	0	\\
2.42e-09	0	\\
2.52e-09	0	\\
2.63e-09	0	\\
2.73e-09	0	\\
2.83e-09	0	\\
2.93e-09	0	\\
3.04e-09	0	\\
3.14e-09	0	\\
3.24e-09	0	\\
3.34e-09	0	\\
3.45e-09	0	\\
3.55e-09	0	\\
3.65e-09	0	\\
3.75e-09	0	\\
3.86e-09	0	\\
3.96e-09	0	\\
4.06e-09	0	\\
4.16e-09	0	\\
4.27e-09	0	\\
4.37e-09	0	\\
4.47e-09	0	\\
4.57e-09	0	\\
4.68e-09	0	\\
4.78e-09	0	\\
4.89e-09	0	\\
4.99e-09	0	\\
5e-09	0	\\
};
\addplot [color=mycolor1,solid,forget plot]
  table[row sep=crcr]{
0	0	\\
1.1e-10	0	\\
2.2e-10	0	\\
3.3e-10	0	\\
4.4e-10	0	\\
5.4e-10	0	\\
6.5e-10	0	\\
7.5e-10	0	\\
8.6e-10	0	\\
9.6e-10	0	\\
1.07e-09	0	\\
1.18e-09	0	\\
1.28e-09	0	\\
1.38e-09	0	\\
1.49e-09	0	\\
1.59e-09	0	\\
1.69e-09	0	\\
1.8e-09	0	\\
1.9e-09	0	\\
2.01e-09	0	\\
2.11e-09	0	\\
2.21e-09	0	\\
2.32e-09	0	\\
2.42e-09	0	\\
2.52e-09	0	\\
2.63e-09	0	\\
2.73e-09	0	\\
2.83e-09	0	\\
2.93e-09	0	\\
3.04e-09	0	\\
3.14e-09	0	\\
3.24e-09	0	\\
3.34e-09	0	\\
3.45e-09	0	\\
3.55e-09	0	\\
3.65e-09	0	\\
3.75e-09	0	\\
3.86e-09	0	\\
3.96e-09	0	\\
4.06e-09	0	\\
4.16e-09	0	\\
4.27e-09	0	\\
4.37e-09	0	\\
4.47e-09	0	\\
4.57e-09	0	\\
4.68e-09	0	\\
4.78e-09	0	\\
4.89e-09	0	\\
4.99e-09	0	\\
5e-09	0	\\
};
\addplot [color=mycolor2,solid,forget plot]
  table[row sep=crcr]{
0	0	\\
1.1e-10	0	\\
2.2e-10	0	\\
3.3e-10	0	\\
4.4e-10	0	\\
5.4e-10	0	\\
6.5e-10	0	\\
7.5e-10	0	\\
8.6e-10	0	\\
9.6e-10	0	\\
1.07e-09	0	\\
1.18e-09	0	\\
1.28e-09	0	\\
1.38e-09	0	\\
1.49e-09	0	\\
1.59e-09	0	\\
1.69e-09	0	\\
1.8e-09	0	\\
1.9e-09	0	\\
2.01e-09	0	\\
2.11e-09	0	\\
2.21e-09	0	\\
2.32e-09	0	\\
2.42e-09	0	\\
2.52e-09	0	\\
2.63e-09	0	\\
2.73e-09	0	\\
2.83e-09	0	\\
2.93e-09	0	\\
3.04e-09	0	\\
3.14e-09	0	\\
3.24e-09	0	\\
3.34e-09	0	\\
3.45e-09	0	\\
3.55e-09	0	\\
3.65e-09	0	\\
3.75e-09	0	\\
3.86e-09	0	\\
3.96e-09	0	\\
4.06e-09	0	\\
4.16e-09	0	\\
4.27e-09	0	\\
4.37e-09	0	\\
4.47e-09	0	\\
4.57e-09	0	\\
4.68e-09	0	\\
4.78e-09	0	\\
4.89e-09	0	\\
4.99e-09	0	\\
5e-09	0	\\
};
\addplot [color=mycolor3,solid,forget plot]
  table[row sep=crcr]{
0	0	\\
1.1e-10	0	\\
2.2e-10	0	\\
3.3e-10	0	\\
4.4e-10	0	\\
5.4e-10	0	\\
6.5e-10	0	\\
7.5e-10	0	\\
8.6e-10	0	\\
9.6e-10	0	\\
1.07e-09	0	\\
1.18e-09	0	\\
1.28e-09	0	\\
1.38e-09	0	\\
1.49e-09	0	\\
1.59e-09	0	\\
1.69e-09	0	\\
1.8e-09	0	\\
1.9e-09	0	\\
2.01e-09	0	\\
2.11e-09	0	\\
2.21e-09	0	\\
2.32e-09	0	\\
2.42e-09	0	\\
2.52e-09	0	\\
2.63e-09	0	\\
2.73e-09	0	\\
2.83e-09	0	\\
2.93e-09	0	\\
3.04e-09	0	\\
3.14e-09	0	\\
3.24e-09	0	\\
3.34e-09	0	\\
3.45e-09	0	\\
3.55e-09	0	\\
3.65e-09	0	\\
3.75e-09	0	\\
3.86e-09	0	\\
3.96e-09	0	\\
4.06e-09	0	\\
4.16e-09	0	\\
4.27e-09	0	\\
4.37e-09	0	\\
4.47e-09	0	\\
4.57e-09	0	\\
4.68e-09	0	\\
4.78e-09	0	\\
4.89e-09	0	\\
4.99e-09	0	\\
5e-09	0	\\
};
\addplot [color=darkgray,solid,forget plot]
  table[row sep=crcr]{
0	0	\\
1.1e-10	0	\\
2.2e-10	0	\\
3.3e-10	0	\\
4.4e-10	0	\\
5.4e-10	0	\\
6.5e-10	0	\\
7.5e-10	0	\\
8.6e-10	0	\\
9.6e-10	0	\\
1.07e-09	0	\\
1.18e-09	0	\\
1.28e-09	0	\\
1.38e-09	0	\\
1.49e-09	0	\\
1.59e-09	0	\\
1.69e-09	0	\\
1.8e-09	0	\\
1.9e-09	0	\\
2.01e-09	0	\\
2.11e-09	0	\\
2.21e-09	0	\\
2.32e-09	0	\\
2.42e-09	0	\\
2.52e-09	0	\\
2.63e-09	0	\\
2.73e-09	0	\\
2.83e-09	0	\\
2.93e-09	0	\\
3.04e-09	0	\\
3.14e-09	0	\\
3.24e-09	0	\\
3.34e-09	0	\\
3.45e-09	0	\\
3.55e-09	0	\\
3.65e-09	0	\\
3.75e-09	0	\\
3.86e-09	0	\\
3.96e-09	0	\\
4.06e-09	0	\\
4.16e-09	0	\\
4.27e-09	0	\\
4.37e-09	0	\\
4.47e-09	0	\\
4.57e-09	0	\\
4.68e-09	0	\\
4.78e-09	0	\\
4.89e-09	0	\\
4.99e-09	0	\\
5e-09	0	\\
};
\addplot [color=blue,solid,forget plot]
  table[row sep=crcr]{
0	0	\\
1.1e-10	0	\\
2.2e-10	0	\\
3.3e-10	0	\\
4.4e-10	0	\\
5.4e-10	0	\\
6.5e-10	0	\\
7.5e-10	0	\\
8.6e-10	0	\\
9.6e-10	0	\\
1.07e-09	0	\\
1.18e-09	0	\\
1.28e-09	0	\\
1.38e-09	0	\\
1.49e-09	0	\\
1.59e-09	0	\\
1.69e-09	0	\\
1.8e-09	0	\\
1.9e-09	0	\\
2.01e-09	0	\\
2.11e-09	0	\\
2.21e-09	0	\\
2.32e-09	0	\\
2.42e-09	0	\\
2.52e-09	0	\\
2.63e-09	0	\\
2.73e-09	0	\\
2.83e-09	0	\\
2.93e-09	0	\\
3.04e-09	0	\\
3.14e-09	0	\\
3.24e-09	0	\\
3.34e-09	0	\\
3.45e-09	0	\\
3.55e-09	0	\\
3.65e-09	0	\\
3.75e-09	0	\\
3.86e-09	0	\\
3.96e-09	0	\\
4.06e-09	0	\\
4.16e-09	0	\\
4.27e-09	0	\\
4.37e-09	0	\\
4.47e-09	0	\\
4.57e-09	0	\\
4.68e-09	0	\\
4.78e-09	0	\\
4.89e-09	0	\\
4.99e-09	0	\\
5e-09	0	\\
};
\addplot [color=black!50!green,solid,forget plot]
  table[row sep=crcr]{
0	0	\\
1.1e-10	0	\\
2.2e-10	0	\\
3.3e-10	0	\\
4.4e-10	0	\\
5.4e-10	0	\\
6.5e-10	0	\\
7.5e-10	0	\\
8.6e-10	0	\\
9.6e-10	0	\\
1.07e-09	0	\\
1.18e-09	0	\\
1.28e-09	0	\\
1.38e-09	0	\\
1.49e-09	0	\\
1.59e-09	0	\\
1.69e-09	0	\\
1.8e-09	0	\\
1.9e-09	0	\\
2.01e-09	0	\\
2.11e-09	0	\\
2.21e-09	0	\\
2.32e-09	0	\\
2.42e-09	0	\\
2.52e-09	0	\\
2.63e-09	0	\\
2.73e-09	0	\\
2.83e-09	0	\\
2.93e-09	0	\\
3.04e-09	0	\\
3.14e-09	0	\\
3.24e-09	0	\\
3.34e-09	0	\\
3.45e-09	0	\\
3.55e-09	0	\\
3.65e-09	0	\\
3.75e-09	0	\\
3.86e-09	0	\\
3.96e-09	0	\\
4.06e-09	0	\\
4.16e-09	0	\\
4.27e-09	0	\\
4.37e-09	0	\\
4.47e-09	0	\\
4.57e-09	0	\\
4.68e-09	0	\\
4.78e-09	0	\\
4.89e-09	0	\\
4.99e-09	0	\\
5e-09	0	\\
};
\addplot [color=red,solid,forget plot]
  table[row sep=crcr]{
0	0	\\
1.1e-10	0	\\
2.2e-10	0	\\
3.3e-10	0	\\
4.4e-10	0	\\
5.4e-10	0	\\
6.5e-10	0	\\
7.5e-10	0	\\
8.6e-10	0	\\
9.6e-10	0	\\
1.07e-09	0	\\
1.18e-09	0	\\
1.28e-09	0	\\
1.38e-09	0	\\
1.49e-09	0	\\
1.59e-09	0	\\
1.69e-09	0	\\
1.8e-09	0	\\
1.9e-09	0	\\
2.01e-09	0	\\
2.11e-09	0	\\
2.21e-09	0	\\
2.32e-09	0	\\
2.42e-09	0	\\
2.52e-09	0	\\
2.63e-09	0	\\
2.73e-09	0	\\
2.83e-09	0	\\
2.93e-09	0	\\
3.04e-09	0	\\
3.14e-09	0	\\
3.24e-09	0	\\
3.34e-09	0	\\
3.45e-09	0	\\
3.55e-09	0	\\
3.65e-09	0	\\
3.75e-09	0	\\
3.86e-09	0	\\
3.96e-09	0	\\
4.06e-09	0	\\
4.16e-09	0	\\
4.27e-09	0	\\
4.37e-09	0	\\
4.47e-09	0	\\
4.57e-09	0	\\
4.68e-09	0	\\
4.78e-09	0	\\
4.89e-09	0	\\
4.99e-09	0	\\
5e-09	0	\\
};
\addplot [color=mycolor1,solid,forget plot]
  table[row sep=crcr]{
0	0	\\
1.1e-10	0	\\
2.2e-10	0	\\
3.3e-10	0	\\
4.4e-10	0	\\
5.4e-10	0	\\
6.5e-10	0	\\
7.5e-10	0	\\
8.6e-10	0	\\
9.6e-10	0	\\
1.07e-09	0	\\
1.18e-09	0	\\
1.28e-09	0	\\
1.38e-09	0	\\
1.49e-09	0	\\
1.59e-09	0	\\
1.69e-09	0	\\
1.8e-09	0	\\
1.9e-09	0	\\
2.01e-09	0	\\
2.11e-09	0	\\
2.21e-09	0	\\
2.32e-09	0	\\
2.42e-09	0	\\
2.52e-09	0	\\
2.63e-09	0	\\
2.73e-09	0	\\
2.83e-09	0	\\
2.93e-09	0	\\
3.04e-09	0	\\
3.14e-09	0	\\
3.24e-09	0	\\
3.34e-09	0	\\
3.45e-09	0	\\
3.55e-09	0	\\
3.65e-09	0	\\
3.75e-09	0	\\
3.86e-09	0	\\
3.96e-09	0	\\
4.06e-09	0	\\
4.16e-09	0	\\
4.27e-09	0	\\
4.37e-09	0	\\
4.47e-09	0	\\
4.57e-09	0	\\
4.68e-09	0	\\
4.78e-09	0	\\
4.89e-09	0	\\
4.99e-09	0	\\
5e-09	0	\\
};
\addplot [color=mycolor2,solid,forget plot]
  table[row sep=crcr]{
0	0	\\
1.1e-10	0	\\
2.2e-10	0	\\
3.3e-10	0	\\
4.4e-10	0	\\
5.4e-10	0	\\
6.5e-10	0	\\
7.5e-10	0	\\
8.6e-10	0	\\
9.6e-10	0	\\
1.07e-09	0	\\
1.18e-09	0	\\
1.28e-09	0	\\
1.38e-09	0	\\
1.49e-09	0	\\
1.59e-09	0	\\
1.69e-09	0	\\
1.8e-09	0	\\
1.9e-09	0	\\
2.01e-09	0	\\
2.11e-09	0	\\
2.21e-09	0	\\
2.32e-09	0	\\
2.42e-09	0	\\
2.52e-09	0	\\
2.63e-09	0	\\
2.73e-09	0	\\
2.83e-09	0	\\
2.93e-09	0	\\
3.04e-09	0	\\
3.14e-09	0	\\
3.24e-09	0	\\
3.34e-09	0	\\
3.45e-09	0	\\
3.55e-09	0	\\
3.65e-09	0	\\
3.75e-09	0	\\
3.86e-09	0	\\
3.96e-09	0	\\
4.06e-09	0	\\
4.16e-09	0	\\
4.27e-09	0	\\
4.37e-09	0	\\
4.47e-09	0	\\
4.57e-09	0	\\
4.68e-09	0	\\
4.78e-09	0	\\
4.89e-09	0	\\
4.99e-09	0	\\
5e-09	0	\\
};
\end{axis}
\end{tikzpicture}%
	\label{fig:wheel-torque-labcar}
	\caption{Breakdown of the delivered powers.}
\end{figure}

\begin{figure}[H]
	\centering
	\setlength\figureheight{4cm}
    	\setlength\figurewidth{0.8\linewidth}
	% This file was created by matlab2tikz v0.4.6 running on MATLAB 8.2.
% Copyright (c) 2008--2014, Nico Schlömer <nico.schloemer@gmail.com>
% All rights reserved.
% Minimal pgfplots version: 1.3
% 
% The latest updates can be retrieved from
%   http://www.mathworks.com/matlabcentral/fileexchange/22022-matlab2tikz
% where you can also make suggestions and rate matlab2tikz.
% 
%
% defining custom colors
\definecolor{mycolor1}{rgb}{0.00000,0.75000,0.75000}%
\definecolor{mycolor2}{rgb}{0.75000,0.00000,0.75000}%
\definecolor{mycolor3}{rgb}{0.75000,0.75000,0.00000}%
%
\begin{tikzpicture}

\begin{axis}[%
width=\figurewidth,
height=\figureheight,
scale only axis,
xmin=0,
xmax=5e-09,
xlabel={t (s)},
ymin=0,
ymax=100,
ylabel={$\eta\text{ (\%)}$}
]
\addplot [color=blue,solid,forget plot]
  table[row sep=crcr]{
0	0	\\
1.2e-10	0	\\
2.4e-10	0	\\
3.6e-10	0	\\
4.8e-10	0	\\
6e-10	0	\\
7.2e-10	0	\\
8.4e-10	0	\\
9.6e-10	0	\\
1.08e-09	0	\\
1.2e-09	82.3025925643359	\\
1.32e-09	82.3025925643359	\\
1.44e-09	82.3025925643359	\\
1.56e-09	82.3025925643359	\\
1.68e-09	82.3025925643359	\\
1.71e-09	0	\\
1.83e-09	0	\\
1.95e-09	0	\\
2.07e-09	0	\\
2.19e-09	0	\\
2.31e-09	0	\\
2.43e-09	0	\\
2.55e-09	0	\\
2.67e-09	0	\\
2.79e-09	0	\\
2.91e-09	0	\\
3.03e-09	0	\\
3.15e-09	0	\\
3.27e-09	0	\\
3.39e-09	0	\\
3.51e-09	0	\\
3.6e-09	97.9778643177519	\\
3.72e-09	97.9778643177519	\\
3.84e-09	97.9778643177519	\\
3.96e-09	97.9778643177519	\\
4.08e-09	97.9778643177519	\\
4.11e-09	0	\\
4.23e-09	0	\\
4.35e-09	0	\\
4.47e-09	0	\\
4.59e-09	0	\\
4.71e-09	0	\\
4.83e-09	0	\\
4.95e-09	0	\\
4.98e-09	0	\\
};
\addplot [color=black!50!green,solid,forget plot]
  table[row sep=crcr]{
0	0	\\
1.2e-10	0	\\
2.4e-10	0	\\
3.6e-10	0	\\
4.8e-10	0	\\
6e-10	0	\\
7.2e-10	0	\\
8.4e-10	0	\\
9.6e-10	0	\\
1.08e-09	0	\\
1.2e-09	0	\\
1.32e-09	0	\\
1.44e-09	0	\\
1.56e-09	0	\\
1.68e-09	0	\\
1.8e-09	0	\\
1.92e-09	0	\\
2.04e-09	0	\\
2.16e-09	0	\\
2.28e-09	0	\\
2.4e-09	0	\\
2.52e-09	0	\\
2.64e-09	0	\\
2.76e-09	0	\\
2.88e-09	0	\\
3e-09	0	\\
3.12e-09	0	\\
3.24e-09	0	\\
3.36e-09	0	\\
3.48e-09	0	\\
3.6e-09	0	\\
3.72e-09	0	\\
3.84e-09	0	\\
3.96e-09	0	\\
4.08e-09	0	\\
4.2e-09	0	\\
4.32e-09	0	\\
4.44e-09	0	\\
4.56e-09	0	\\
4.68e-09	0	\\
4.8e-09	0	\\
4.92e-09	0	\\
4.98e-09	0	\\
};
\addplot [color=red,solid,forget plot]
  table[row sep=crcr]{
0	0	\\
1.2e-10	0	\\
2.4e-10	0	\\
3.6e-10	0	\\
4.8e-10	0	\\
6e-10	0	\\
7.2e-10	0	\\
8.4e-10	0	\\
9.6e-10	0	\\
1.08e-09	0	\\
1.2e-09	0	\\
1.32e-09	0	\\
1.44e-09	0	\\
1.56e-09	0	\\
1.68e-09	0	\\
1.8e-09	0	\\
1.92e-09	0	\\
2.04e-09	0	\\
2.16e-09	0	\\
2.28e-09	0	\\
2.4e-09	0	\\
2.52e-09	0	\\
2.64e-09	0	\\
2.76e-09	0	\\
2.88e-09	0	\\
3e-09	0	\\
3.12e-09	0	\\
3.24e-09	0	\\
3.36e-09	0	\\
3.48e-09	0	\\
3.6e-09	0	\\
3.72e-09	0	\\
3.84e-09	0	\\
3.96e-09	0	\\
4.08e-09	0	\\
4.2e-09	0	\\
4.32e-09	0	\\
4.44e-09	0	\\
4.56e-09	0	\\
4.68e-09	0	\\
4.8e-09	0	\\
4.92e-09	0	\\
4.98e-09	0	\\
};
\addplot [color=mycolor1,solid,forget plot]
  table[row sep=crcr]{
0	0	\\
1.2e-10	0	\\
2.4e-10	0	\\
3.6e-10	0	\\
4.8e-10	0	\\
6e-10	0	\\
7.2e-10	0	\\
8.4e-10	0	\\
9.6e-10	0	\\
1.08e-09	0	\\
1.2e-09	0	\\
1.32e-09	0	\\
1.44e-09	0	\\
1.56e-09	0	\\
1.68e-09	0	\\
1.8e-09	0	\\
1.92e-09	0	\\
2.04e-09	0	\\
2.16e-09	0	\\
2.28e-09	0	\\
2.4e-09	0	\\
2.52e-09	0	\\
2.64e-09	0	\\
2.76e-09	0	\\
2.88e-09	0	\\
3e-09	0	\\
3.12e-09	0	\\
3.24e-09	0	\\
3.36e-09	0	\\
3.48e-09	0	\\
3.6e-09	0	\\
3.72e-09	0	\\
3.84e-09	0	\\
3.96e-09	0	\\
4.08e-09	0	\\
4.2e-09	0	\\
4.32e-09	0	\\
4.44e-09	0	\\
4.56e-09	0	\\
4.68e-09	0	\\
4.8e-09	0	\\
4.92e-09	0	\\
4.98e-09	0	\\
};
\addplot [color=mycolor2,solid,forget plot]
  table[row sep=crcr]{
0	0	\\
1.2e-10	0	\\
2.4e-10	0	\\
3.6e-10	0	\\
4.8e-10	0	\\
6e-10	0	\\
7.2e-10	0	\\
8.4e-10	0	\\
9.6e-10	0	\\
1.08e-09	0	\\
1.2e-09	0	\\
1.32e-09	0	\\
1.44e-09	0	\\
1.56e-09	0	\\
1.68e-09	0	\\
1.8e-09	0	\\
1.92e-09	0	\\
2.04e-09	0	\\
2.16e-09	0	\\
2.28e-09	0	\\
2.4e-09	0	\\
2.52e-09	0	\\
2.64e-09	0	\\
2.76e-09	0	\\
2.88e-09	0	\\
3e-09	0	\\
3.12e-09	0	\\
3.24e-09	0	\\
3.36e-09	0	\\
3.48e-09	0	\\
3.6e-09	0	\\
3.72e-09	0	\\
3.84e-09	0	\\
3.96e-09	0	\\
4.08e-09	0	\\
4.2e-09	0	\\
4.32e-09	0	\\
4.44e-09	0	\\
4.56e-09	0	\\
4.68e-09	0	\\
4.8e-09	0	\\
4.92e-09	0	\\
4.98e-09	0	\\
};
\addplot [color=mycolor3,solid,forget plot]
  table[row sep=crcr]{
0	0	\\
1.2e-10	0	\\
2.4e-10	0	\\
3.6e-10	0	\\
4.8e-10	0	\\
6e-10	0	\\
7.2e-10	0	\\
8.4e-10	0	\\
9.6e-10	0	\\
1.08e-09	0	\\
1.2e-09	0	\\
1.32e-09	0	\\
1.44e-09	0	\\
1.56e-09	0	\\
1.68e-09	0	\\
1.8e-09	0	\\
1.92e-09	0	\\
2.04e-09	0	\\
2.16e-09	0	\\
2.28e-09	0	\\
2.4e-09	0	\\
2.52e-09	0	\\
2.64e-09	0	\\
2.76e-09	0	\\
2.88e-09	0	\\
3e-09	0	\\
3.12e-09	0	\\
3.24e-09	0	\\
3.36e-09	0	\\
3.48e-09	0	\\
3.6e-09	0	\\
3.72e-09	0	\\
3.84e-09	0	\\
3.96e-09	0	\\
4.08e-09	0	\\
4.2e-09	0	\\
4.32e-09	0	\\
4.44e-09	0	\\
4.56e-09	0	\\
4.68e-09	0	\\
4.8e-09	0	\\
4.92e-09	0	\\
4.98e-09	0	\\
};
\addplot [color=darkgray,solid,forget plot]
  table[row sep=crcr]{
0	0	\\
1.2e-10	0	\\
2.4e-10	0	\\
3.6e-10	0	\\
4.8e-10	0	\\
6e-10	0	\\
7.2e-10	0	\\
8.4e-10	0	\\
9.6e-10	0	\\
1.08e-09	0	\\
1.2e-09	0	\\
1.32e-09	0	\\
1.44e-09	0	\\
1.56e-09	0	\\
1.68e-09	0	\\
1.8e-09	0	\\
1.92e-09	0	\\
2.04e-09	0	\\
2.16e-09	0	\\
2.28e-09	0	\\
2.4e-09	0	\\
2.52e-09	0	\\
2.64e-09	0	\\
2.76e-09	0	\\
2.88e-09	0	\\
3e-09	0	\\
3.12e-09	0	\\
3.24e-09	0	\\
3.36e-09	0	\\
3.48e-09	0	\\
3.6e-09	0	\\
3.72e-09	0	\\
3.84e-09	0	\\
3.96e-09	0	\\
4.08e-09	0	\\
4.2e-09	0	\\
4.32e-09	0	\\
4.44e-09	0	\\
4.56e-09	0	\\
4.68e-09	0	\\
4.8e-09	0	\\
4.92e-09	0	\\
4.98e-09	0	\\
};
\addplot [color=blue,solid,forget plot]
  table[row sep=crcr]{
0	0	\\
1.2e-10	0	\\
2.4e-10	0	\\
3.6e-10	0	\\
4.8e-10	0	\\
6e-10	0	\\
7.2e-10	0	\\
8.4e-10	0	\\
9.6e-10	0	\\
1.08e-09	0	\\
1.2e-09	0	\\
1.32e-09	0	\\
1.44e-09	0	\\
1.56e-09	0	\\
1.68e-09	0	\\
1.8e-09	0	\\
1.92e-09	0	\\
2.04e-09	0	\\
2.16e-09	0	\\
2.28e-09	0	\\
2.4e-09	0	\\
2.52e-09	0	\\
2.64e-09	0	\\
2.76e-09	0	\\
2.88e-09	0	\\
3e-09	0	\\
3.12e-09	0	\\
3.24e-09	0	\\
3.36e-09	0	\\
3.48e-09	0	\\
3.6e-09	0	\\
3.72e-09	0	\\
3.84e-09	0	\\
3.96e-09	0	\\
4.08e-09	0	\\
4.2e-09	0	\\
4.32e-09	0	\\
4.44e-09	0	\\
4.56e-09	0	\\
4.68e-09	0	\\
4.8e-09	0	\\
4.92e-09	0	\\
4.98e-09	0	\\
};
\addplot [color=black!50!green,solid,forget plot]
  table[row sep=crcr]{
0	0	\\
1.2e-10	0	\\
2.4e-10	0	\\
3.6e-10	0	\\
4.8e-10	0	\\
6e-10	0	\\
7.2e-10	0	\\
8.4e-10	0	\\
9.6e-10	0	\\
1.08e-09	0	\\
1.2e-09	0	\\
1.32e-09	0	\\
1.44e-09	0	\\
1.56e-09	0	\\
1.68e-09	0	\\
1.8e-09	0	\\
1.92e-09	0	\\
2.04e-09	0	\\
2.16e-09	0	\\
2.28e-09	0	\\
2.4e-09	0	\\
2.52e-09	0	\\
2.64e-09	0	\\
2.76e-09	0	\\
2.88e-09	0	\\
3e-09	0	\\
3.12e-09	0	\\
3.24e-09	0	\\
3.36e-09	0	\\
3.48e-09	0	\\
3.6e-09	0	\\
3.72e-09	0	\\
3.84e-09	0	\\
3.96e-09	0	\\
4.08e-09	0	\\
4.2e-09	0	\\
4.32e-09	0	\\
4.44e-09	0	\\
4.56e-09	0	\\
4.68e-09	0	\\
4.8e-09	0	\\
4.92e-09	0	\\
4.98e-09	0	\\
};
\addplot [color=red,solid,forget plot]
  table[row sep=crcr]{
0	0	\\
1.2e-10	0	\\
2.4e-10	0	\\
3.6e-10	0	\\
4.8e-10	0	\\
6e-10	0	\\
7.2e-10	0	\\
8.4e-10	0	\\
9.6e-10	0	\\
1.08e-09	0	\\
1.2e-09	0	\\
1.32e-09	0	\\
1.44e-09	0	\\
1.56e-09	0	\\
1.68e-09	0	\\
1.8e-09	0	\\
1.92e-09	0	\\
2.04e-09	0	\\
2.16e-09	0	\\
2.28e-09	0	\\
2.4e-09	0	\\
2.52e-09	0	\\
2.64e-09	0	\\
2.76e-09	0	\\
2.88e-09	0	\\
3e-09	0	\\
3.12e-09	0	\\
3.24e-09	0	\\
3.36e-09	0	\\
3.48e-09	0	\\
3.6e-09	0	\\
3.72e-09	0	\\
3.84e-09	0	\\
3.96e-09	0	\\
4.08e-09	0	\\
4.2e-09	0	\\
4.32e-09	0	\\
4.44e-09	0	\\
4.56e-09	0	\\
4.68e-09	0	\\
4.8e-09	0	\\
4.92e-09	0	\\
4.98e-09	0	\\
};
\addplot [color=mycolor1,solid,forget plot]
  table[row sep=crcr]{
0	0	\\
1.2e-10	0	\\
2.4e-10	0	\\
3.6e-10	0	\\
4.8e-10	0	\\
6e-10	0	\\
7.2e-10	0	\\
8.4e-10	0	\\
9.6e-10	0	\\
1.08e-09	0	\\
1.2e-09	0	\\
1.32e-09	0	\\
1.44e-09	0	\\
1.56e-09	0	\\
1.68e-09	0	\\
1.8e-09	0	\\
1.92e-09	0	\\
2.04e-09	0	\\
2.16e-09	0	\\
2.28e-09	0	\\
2.4e-09	0	\\
2.52e-09	0	\\
2.64e-09	0	\\
2.76e-09	0	\\
2.88e-09	0	\\
3e-09	0	\\
3.12e-09	0	\\
3.24e-09	0	\\
3.36e-09	0	\\
3.48e-09	0	\\
3.6e-09	0	\\
3.72e-09	0	\\
3.84e-09	0	\\
3.96e-09	0	\\
4.08e-09	0	\\
4.2e-09	0	\\
4.32e-09	0	\\
4.44e-09	0	\\
4.56e-09	0	\\
4.68e-09	0	\\
4.8e-09	0	\\
4.92e-09	0	\\
4.98e-09	0	\\
};
\addplot [color=mycolor2,solid,forget plot]
  table[row sep=crcr]{
0	0	\\
1.2e-10	0	\\
2.4e-10	0	\\
3.6e-10	0	\\
4.8e-10	0	\\
6e-10	0	\\
7.2e-10	0	\\
8.4e-10	0	\\
9.6e-10	0	\\
1.08e-09	0	\\
1.2e-09	0	\\
1.32e-09	0	\\
1.44e-09	0	\\
1.56e-09	0	\\
1.68e-09	0	\\
1.8e-09	0	\\
1.92e-09	0	\\
2.04e-09	0	\\
2.16e-09	0	\\
2.28e-09	0	\\
2.4e-09	0	\\
2.52e-09	0	\\
2.64e-09	0	\\
2.76e-09	0	\\
2.88e-09	0	\\
3e-09	0	\\
3.12e-09	0	\\
3.24e-09	0	\\
3.36e-09	0	\\
3.48e-09	0	\\
3.6e-09	0	\\
3.72e-09	0	\\
3.84e-09	0	\\
3.96e-09	0	\\
4.08e-09	0	\\
4.2e-09	0	\\
4.32e-09	0	\\
4.44e-09	0	\\
4.56e-09	0	\\
4.68e-09	0	\\
4.8e-09	0	\\
4.92e-09	0	\\
4.98e-09	0	\\
};
\addplot [color=mycolor3,solid,forget plot]
  table[row sep=crcr]{
0	0	\\
1.2e-10	0	\\
2.4e-10	0	\\
3.6e-10	0	\\
4.8e-10	0	\\
6e-10	0	\\
7.2e-10	0	\\
8.4e-10	0	\\
9.6e-10	0	\\
1.08e-09	0	\\
1.2e-09	0	\\
1.32e-09	0	\\
1.44e-09	0	\\
1.56e-09	0	\\
1.68e-09	0	\\
1.8e-09	0	\\
1.92e-09	0	\\
2.04e-09	0	\\
2.16e-09	0	\\
2.28e-09	0	\\
2.4e-09	0	\\
2.52e-09	0	\\
2.64e-09	0	\\
2.76e-09	0	\\
2.88e-09	0	\\
3e-09	0	\\
3.12e-09	0	\\
3.24e-09	0	\\
3.36e-09	0	\\
3.48e-09	0	\\
3.6e-09	0	\\
3.72e-09	0	\\
3.84e-09	0	\\
3.96e-09	0	\\
4.08e-09	0	\\
4.2e-09	0	\\
4.32e-09	0	\\
4.44e-09	0	\\
4.56e-09	0	\\
4.68e-09	0	\\
4.8e-09	0	\\
4.92e-09	0	\\
4.98e-09	0	\\
};
\addplot [color=darkgray,solid,forget plot]
  table[row sep=crcr]{
0	0	\\
1.2e-10	0	\\
2.4e-10	0	\\
3.6e-10	0	\\
4.8e-10	0	\\
6e-10	0	\\
7.2e-10	0	\\
8.4e-10	0	\\
9.6e-10	0	\\
1.08e-09	0	\\
1.2e-09	0	\\
1.32e-09	0	\\
1.44e-09	0	\\
1.56e-09	0	\\
1.68e-09	0	\\
1.8e-09	0	\\
1.92e-09	0	\\
2.04e-09	0	\\
2.16e-09	0	\\
2.28e-09	0	\\
2.4e-09	0	\\
2.52e-09	0	\\
2.64e-09	0	\\
2.76e-09	0	\\
2.88e-09	0	\\
3e-09	0	\\
3.12e-09	0	\\
3.24e-09	0	\\
3.36e-09	0	\\
3.48e-09	0	\\
3.6e-09	0	\\
3.72e-09	0	\\
3.84e-09	0	\\
3.96e-09	0	\\
4.08e-09	0	\\
4.2e-09	0	\\
4.32e-09	0	\\
4.44e-09	0	\\
4.56e-09	0	\\
4.68e-09	0	\\
4.8e-09	0	\\
4.92e-09	0	\\
4.98e-09	0	\\
};
\addplot [color=blue,solid,forget plot]
  table[row sep=crcr]{
0	0	\\
1.2e-10	0	\\
2.4e-10	0	\\
3.6e-10	0	\\
4.8e-10	0	\\
6e-10	0	\\
7.2e-10	0	\\
8.4e-10	0	\\
9.6e-10	0	\\
1.08e-09	0	\\
1.2e-09	0	\\
1.32e-09	0	\\
1.44e-09	0	\\
1.56e-09	0	\\
1.68e-09	0	\\
1.8e-09	0	\\
1.92e-09	0	\\
2.04e-09	0	\\
2.16e-09	0	\\
2.28e-09	0	\\
2.4e-09	0	\\
2.52e-09	0	\\
2.64e-09	0	\\
2.76e-09	0	\\
2.88e-09	0	\\
3e-09	0	\\
3.12e-09	0	\\
3.24e-09	0	\\
3.36e-09	0	\\
3.48e-09	0	\\
3.6e-09	0	\\
3.72e-09	0	\\
3.84e-09	0	\\
3.96e-09	0	\\
4.08e-09	0	\\
4.2e-09	0	\\
4.32e-09	0	\\
4.44e-09	0	\\
4.56e-09	0	\\
4.68e-09	0	\\
4.8e-09	0	\\
4.92e-09	0	\\
4.98e-09	0	\\
};
\addplot [color=black!50!green,solid,forget plot]
  table[row sep=crcr]{
0	0	\\
1.2e-10	0	\\
2.4e-10	0	\\
3.6e-10	0	\\
4.8e-10	0	\\
6e-10	0	\\
7.2e-10	0	\\
8.4e-10	0	\\
9.6e-10	0	\\
1.08e-09	0	\\
1.2e-09	0	\\
1.32e-09	0	\\
1.44e-09	0	\\
1.56e-09	0	\\
1.68e-09	0	\\
1.8e-09	0	\\
1.92e-09	0	\\
2.04e-09	0	\\
2.16e-09	0	\\
2.28e-09	0	\\
2.4e-09	0	\\
2.52e-09	0	\\
2.64e-09	0	\\
2.76e-09	0	\\
2.88e-09	0	\\
3e-09	0	\\
3.12e-09	0	\\
3.24e-09	0	\\
3.36e-09	0	\\
3.48e-09	0	\\
3.6e-09	0	\\
3.72e-09	0	\\
3.84e-09	0	\\
3.96e-09	0	\\
4.08e-09	0	\\
4.2e-09	0	\\
4.32e-09	0	\\
4.44e-09	0	\\
4.56e-09	0	\\
4.68e-09	0	\\
4.8e-09	0	\\
4.92e-09	0	\\
4.98e-09	0	\\
};
\addplot [color=red,solid,forget plot]
  table[row sep=crcr]{
0	0	\\
1.2e-10	0	\\
2.4e-10	0	\\
3.6e-10	0	\\
4.8e-10	0	\\
6e-10	0	\\
7.2e-10	0	\\
8.4e-10	0	\\
9.6e-10	0	\\
1.08e-09	0	\\
1.2e-09	0	\\
1.32e-09	0	\\
1.44e-09	0	\\
1.56e-09	0	\\
1.68e-09	0	\\
1.8e-09	0	\\
1.92e-09	0	\\
2.04e-09	0	\\
2.16e-09	0	\\
2.28e-09	0	\\
2.4e-09	0	\\
2.52e-09	0	\\
2.64e-09	0	\\
2.76e-09	0	\\
2.88e-09	0	\\
3e-09	0	\\
3.12e-09	0	\\
3.24e-09	0	\\
3.36e-09	0	\\
3.48e-09	0	\\
3.6e-09	0	\\
3.72e-09	0	\\
3.84e-09	0	\\
3.96e-09	0	\\
4.08e-09	0	\\
4.2e-09	0	\\
4.32e-09	0	\\
4.44e-09	0	\\
4.56e-09	0	\\
4.68e-09	0	\\
4.8e-09	0	\\
4.92e-09	0	\\
4.98e-09	0	\\
};
\addplot [color=mycolor1,solid,forget plot]
  table[row sep=crcr]{
0	0	\\
1.2e-10	0	\\
2.4e-10	0	\\
3.6e-10	0	\\
4.8e-10	0	\\
6e-10	0	\\
7.2e-10	0	\\
8.4e-10	0	\\
9.6e-10	0	\\
1.08e-09	0	\\
1.2e-09	0	\\
1.32e-09	0	\\
1.44e-09	0	\\
1.56e-09	0	\\
1.68e-09	0	\\
1.8e-09	0	\\
1.92e-09	0	\\
2.04e-09	0	\\
2.16e-09	0	\\
2.28e-09	0	\\
2.4e-09	0	\\
2.52e-09	0	\\
2.64e-09	0	\\
2.76e-09	0	\\
2.88e-09	0	\\
3e-09	0	\\
3.12e-09	0	\\
3.24e-09	0	\\
3.36e-09	0	\\
3.48e-09	0	\\
3.6e-09	0	\\
3.72e-09	0	\\
3.84e-09	0	\\
3.96e-09	0	\\
4.08e-09	0	\\
4.2e-09	0	\\
4.32e-09	0	\\
4.44e-09	0	\\
4.56e-09	0	\\
4.68e-09	0	\\
4.8e-09	0	\\
4.92e-09	0	\\
4.98e-09	0	\\
};
\addplot [color=mycolor2,solid,forget plot]
  table[row sep=crcr]{
0	0	\\
1.2e-10	0	\\
2.4e-10	0	\\
3.6e-10	0	\\
4.8e-10	0	\\
6e-10	0	\\
7.2e-10	0	\\
8.4e-10	0	\\
9.6e-10	0	\\
1.08e-09	0	\\
1.2e-09	0	\\
1.32e-09	0	\\
1.44e-09	0	\\
1.56e-09	0	\\
1.68e-09	0	\\
1.8e-09	0	\\
1.92e-09	0	\\
2.04e-09	0	\\
2.16e-09	0	\\
2.28e-09	0	\\
2.4e-09	0	\\
2.52e-09	0	\\
2.64e-09	0	\\
2.76e-09	0	\\
2.88e-09	0	\\
3e-09	0	\\
3.12e-09	0	\\
3.24e-09	0	\\
3.36e-09	0	\\
3.48e-09	0	\\
3.6e-09	0	\\
3.72e-09	0	\\
3.84e-09	0	\\
3.96e-09	0	\\
4.08e-09	0	\\
4.2e-09	0	\\
4.32e-09	0	\\
4.44e-09	0	\\
4.56e-09	0	\\
4.68e-09	0	\\
4.8e-09	0	\\
4.92e-09	0	\\
4.98e-09	0	\\
};
\addplot [color=mycolor3,solid,forget plot]
  table[row sep=crcr]{
0	0	\\
1.2e-10	0	\\
2.4e-10	0	\\
3.6e-10	0	\\
4.8e-10	0	\\
6e-10	0	\\
7.2e-10	0	\\
8.4e-10	0	\\
9.6e-10	0	\\
1.08e-09	0	\\
1.2e-09	0	\\
1.32e-09	0	\\
1.44e-09	0	\\
1.56e-09	0	\\
1.68e-09	0	\\
1.8e-09	0	\\
1.92e-09	0	\\
2.04e-09	0	\\
2.16e-09	0	\\
2.28e-09	0	\\
2.4e-09	0	\\
2.52e-09	0	\\
2.64e-09	0	\\
2.76e-09	0	\\
2.88e-09	0	\\
3e-09	0	\\
3.12e-09	0	\\
3.24e-09	0	\\
3.36e-09	0	\\
3.48e-09	0	\\
3.6e-09	0	\\
3.72e-09	0	\\
3.84e-09	0	\\
3.96e-09	0	\\
4.08e-09	0	\\
4.2e-09	0	\\
4.32e-09	0	\\
4.44e-09	0	\\
4.56e-09	0	\\
4.68e-09	0	\\
4.8e-09	0	\\
4.92e-09	0	\\
4.98e-09	0	\\
};
\addplot [color=darkgray,solid,forget plot]
  table[row sep=crcr]{
0	0	\\
1.2e-10	0	\\
2.4e-10	0	\\
3.6e-10	0	\\
4.8e-10	0	\\
6e-10	0	\\
7.2e-10	0	\\
8.4e-10	0	\\
9.6e-10	0	\\
1.08e-09	0	\\
1.2e-09	0	\\
1.32e-09	0	\\
1.44e-09	0	\\
1.56e-09	0	\\
1.68e-09	0	\\
1.8e-09	0	\\
1.92e-09	0	\\
2.04e-09	0	\\
2.16e-09	0	\\
2.28e-09	0	\\
2.4e-09	0	\\
2.52e-09	0	\\
2.64e-09	0	\\
2.76e-09	0	\\
2.88e-09	0	\\
3e-09	0	\\
3.12e-09	0	\\
3.24e-09	0	\\
3.36e-09	0	\\
3.48e-09	0	\\
3.6e-09	0	\\
3.72e-09	0	\\
3.84e-09	0	\\
3.96e-09	0	\\
4.08e-09	0	\\
4.2e-09	0	\\
4.32e-09	0	\\
4.44e-09	0	\\
4.56e-09	0	\\
4.68e-09	0	\\
4.8e-09	0	\\
4.92e-09	0	\\
4.98e-09	0	\\
};
\addplot [color=blue,solid,forget plot]
  table[row sep=crcr]{
0	0	\\
1.2e-10	0	\\
2.4e-10	0	\\
3.6e-10	0	\\
4.8e-10	0	\\
6e-10	0	\\
7.2e-10	0	\\
8.4e-10	0	\\
9.6e-10	0	\\
1.08e-09	0	\\
1.2e-09	0	\\
1.32e-09	0	\\
1.44e-09	0	\\
1.56e-09	0	\\
1.68e-09	0	\\
1.8e-09	0	\\
1.92e-09	0	\\
2.04e-09	0	\\
2.16e-09	0	\\
2.28e-09	0	\\
2.4e-09	0	\\
2.52e-09	0	\\
2.64e-09	0	\\
2.76e-09	0	\\
2.88e-09	0	\\
3e-09	0	\\
3.12e-09	0	\\
3.24e-09	0	\\
3.36e-09	0	\\
3.48e-09	0	\\
3.6e-09	0	\\
3.72e-09	0	\\
3.84e-09	0	\\
3.96e-09	0	\\
4.08e-09	0	\\
4.2e-09	0	\\
4.32e-09	0	\\
4.44e-09	0	\\
4.56e-09	0	\\
4.68e-09	0	\\
4.8e-09	0	\\
4.92e-09	0	\\
4.98e-09	0	\\
};
\addplot [color=black!50!green,solid,forget plot]
  table[row sep=crcr]{
0	0	\\
1.2e-10	0	\\
2.4e-10	0	\\
3.6e-10	0	\\
4.8e-10	0	\\
6e-10	0	\\
7.2e-10	0	\\
8.4e-10	0	\\
9.6e-10	0	\\
1.08e-09	0	\\
1.2e-09	0	\\
1.32e-09	0	\\
1.44e-09	0	\\
1.56e-09	0	\\
1.68e-09	0	\\
1.8e-09	0	\\
1.92e-09	0	\\
2.04e-09	0	\\
2.16e-09	0	\\
2.28e-09	0	\\
2.4e-09	0	\\
2.52e-09	0	\\
2.64e-09	0	\\
2.76e-09	0	\\
2.88e-09	0	\\
3e-09	0	\\
3.12e-09	0	\\
3.24e-09	0	\\
3.36e-09	0	\\
3.48e-09	0	\\
3.6e-09	0	\\
3.72e-09	0	\\
3.84e-09	0	\\
3.96e-09	0	\\
4.08e-09	0	\\
4.2e-09	0	\\
4.32e-09	0	\\
4.44e-09	0	\\
4.56e-09	0	\\
4.68e-09	0	\\
4.8e-09	0	\\
4.92e-09	0	\\
4.98e-09	0	\\
};
\addplot [color=red,solid,forget plot]
  table[row sep=crcr]{
0	0	\\
1.2e-10	0	\\
2.4e-10	0	\\
3.6e-10	0	\\
4.8e-10	0	\\
6e-10	0	\\
7.2e-10	0	\\
8.4e-10	0	\\
9.6e-10	0	\\
1.08e-09	0	\\
1.2e-09	0	\\
1.32e-09	0	\\
1.44e-09	0	\\
1.56e-09	0	\\
1.68e-09	0	\\
1.8e-09	0	\\
1.92e-09	0	\\
2.04e-09	0	\\
2.16e-09	0	\\
2.28e-09	0	\\
2.4e-09	0	\\
2.52e-09	0	\\
2.64e-09	0	\\
2.76e-09	0	\\
2.88e-09	0	\\
3e-09	0	\\
3.12e-09	0	\\
3.24e-09	0	\\
3.36e-09	0	\\
3.48e-09	0	\\
3.6e-09	0	\\
3.72e-09	0	\\
3.84e-09	0	\\
3.96e-09	0	\\
4.08e-09	0	\\
4.2e-09	0	\\
4.32e-09	0	\\
4.44e-09	0	\\
4.56e-09	0	\\
4.68e-09	0	\\
4.8e-09	0	\\
4.92e-09	0	\\
4.98e-09	0	\\
};
\addplot [color=mycolor1,solid,forget plot]
  table[row sep=crcr]{
0	0	\\
1.2e-10	0	\\
2.4e-10	0	\\
3.6e-10	0	\\
4.8e-10	0	\\
6e-10	0	\\
7.2e-10	0	\\
8.4e-10	0	\\
9.6e-10	0	\\
1.08e-09	0	\\
1.2e-09	0	\\
1.32e-09	0	\\
1.44e-09	0	\\
1.56e-09	0	\\
1.68e-09	0	\\
1.8e-09	0	\\
1.92e-09	0	\\
2.04e-09	0	\\
2.16e-09	0	\\
2.28e-09	0	\\
2.4e-09	0	\\
2.52e-09	0	\\
2.64e-09	0	\\
2.76e-09	0	\\
2.88e-09	0	\\
3e-09	0	\\
3.12e-09	0	\\
3.24e-09	0	\\
3.36e-09	0	\\
3.48e-09	0	\\
3.6e-09	0	\\
3.72e-09	0	\\
3.84e-09	0	\\
3.96e-09	0	\\
4.08e-09	0	\\
4.2e-09	0	\\
4.32e-09	0	\\
4.44e-09	0	\\
4.56e-09	0	\\
4.68e-09	0	\\
4.8e-09	0	\\
4.92e-09	0	\\
4.98e-09	0	\\
};
\addplot [color=mycolor2,solid,forget plot]
  table[row sep=crcr]{
0	0	\\
1.2e-10	0	\\
2.4e-10	0	\\
3.6e-10	0	\\
4.8e-10	0	\\
6e-10	0	\\
7.2e-10	0	\\
8.4e-10	0	\\
9.6e-10	0	\\
1.08e-09	0	\\
1.2e-09	0	\\
1.32e-09	0	\\
1.44e-09	0	\\
1.56e-09	0	\\
1.68e-09	0	\\
1.8e-09	0	\\
1.92e-09	0	\\
2.04e-09	0	\\
2.16e-09	0	\\
2.28e-09	0	\\
2.4e-09	0	\\
2.52e-09	0	\\
2.64e-09	0	\\
2.76e-09	0	\\
2.88e-09	0	\\
3e-09	0	\\
3.12e-09	0	\\
3.24e-09	0	\\
3.36e-09	0	\\
3.48e-09	0	\\
3.6e-09	0	\\
3.72e-09	0	\\
3.84e-09	0	\\
3.96e-09	0	\\
4.08e-09	0	\\
4.2e-09	0	\\
4.32e-09	0	\\
4.44e-09	0	\\
4.56e-09	0	\\
4.68e-09	0	\\
4.8e-09	0	\\
4.92e-09	0	\\
4.98e-09	0	\\
};
\addplot [color=mycolor3,solid,forget plot]
  table[row sep=crcr]{
0	0	\\
1.2e-10	0	\\
2.4e-10	0	\\
3.6e-10	0	\\
4.8e-10	0	\\
6e-10	0	\\
7.2e-10	0	\\
8.4e-10	0	\\
9.6e-10	0	\\
1.08e-09	0	\\
1.2e-09	0	\\
1.32e-09	0	\\
1.44e-09	0	\\
1.56e-09	0	\\
1.68e-09	0	\\
1.8e-09	0	\\
1.92e-09	0	\\
2.04e-09	0	\\
2.16e-09	0	\\
2.28e-09	0	\\
2.4e-09	0	\\
2.52e-09	0	\\
2.64e-09	0	\\
2.76e-09	0	\\
2.88e-09	0	\\
3e-09	0	\\
3.12e-09	0	\\
3.24e-09	0	\\
3.36e-09	0	\\
3.48e-09	0	\\
3.6e-09	0	\\
3.72e-09	0	\\
3.84e-09	0	\\
3.96e-09	0	\\
4.08e-09	0	\\
4.2e-09	0	\\
4.32e-09	0	\\
4.44e-09	0	\\
4.56e-09	0	\\
4.68e-09	0	\\
4.8e-09	0	\\
4.92e-09	0	\\
4.98e-09	0	\\
};
\addplot [color=darkgray,solid,forget plot]
  table[row sep=crcr]{
0	0	\\
1.2e-10	0	\\
2.4e-10	0	\\
3.6e-10	0	\\
4.8e-10	0	\\
6e-10	0	\\
7.2e-10	0	\\
8.4e-10	0	\\
9.6e-10	0	\\
1.08e-09	0	\\
1.2e-09	0	\\
1.32e-09	0	\\
1.44e-09	0	\\
1.56e-09	0	\\
1.68e-09	0	\\
1.8e-09	0	\\
1.92e-09	0	\\
2.04e-09	0	\\
2.16e-09	0	\\
2.28e-09	0	\\
2.4e-09	0	\\
2.52e-09	0	\\
2.64e-09	0	\\
2.76e-09	0	\\
2.88e-09	0	\\
3e-09	0	\\
3.12e-09	0	\\
3.24e-09	0	\\
3.36e-09	0	\\
3.48e-09	0	\\
3.6e-09	0	\\
3.72e-09	0	\\
3.84e-09	0	\\
3.96e-09	0	\\
4.08e-09	0	\\
4.2e-09	0	\\
4.32e-09	0	\\
4.44e-09	0	\\
4.56e-09	0	\\
4.68e-09	0	\\
4.8e-09	0	\\
4.92e-09	0	\\
4.98e-09	0	\\
};
\addplot [color=blue,solid,forget plot]
  table[row sep=crcr]{
0	0	\\
1.2e-10	0	\\
2.4e-10	0	\\
3.6e-10	0	\\
4.8e-10	0	\\
6e-10	0	\\
7.2e-10	0	\\
8.4e-10	0	\\
9.6e-10	0	\\
1.08e-09	0	\\
1.2e-09	0	\\
1.32e-09	0	\\
1.44e-09	0	\\
1.56e-09	0	\\
1.68e-09	0	\\
1.8e-09	0	\\
1.92e-09	0	\\
2.04e-09	0	\\
2.16e-09	0	\\
2.28e-09	0	\\
2.4e-09	0	\\
2.52e-09	0	\\
2.64e-09	0	\\
2.76e-09	0	\\
2.88e-09	0	\\
3e-09	0	\\
3.12e-09	0	\\
3.24e-09	0	\\
3.36e-09	0	\\
3.48e-09	0	\\
3.6e-09	0	\\
3.72e-09	0	\\
3.84e-09	0	\\
3.96e-09	0	\\
4.08e-09	0	\\
4.2e-09	0	\\
4.32e-09	0	\\
4.44e-09	0	\\
4.56e-09	0	\\
4.68e-09	0	\\
4.8e-09	0	\\
4.92e-09	0	\\
4.98e-09	0	\\
};
\addplot [color=black!50!green,solid,forget plot]
  table[row sep=crcr]{
0	0	\\
1.2e-10	0	\\
2.4e-10	0	\\
3.6e-10	0	\\
4.8e-10	0	\\
6e-10	0	\\
7.2e-10	0	\\
8.4e-10	0	\\
9.6e-10	0	\\
1.08e-09	0	\\
1.2e-09	0	\\
1.32e-09	0	\\
1.44e-09	0	\\
1.56e-09	0	\\
1.68e-09	0	\\
1.8e-09	0	\\
1.92e-09	0	\\
2.04e-09	0	\\
2.16e-09	0	\\
2.28e-09	0	\\
2.4e-09	0	\\
2.52e-09	0	\\
2.64e-09	0	\\
2.76e-09	0	\\
2.88e-09	0	\\
3e-09	0	\\
3.12e-09	0	\\
3.24e-09	0	\\
3.36e-09	0	\\
3.48e-09	0	\\
3.6e-09	0	\\
3.72e-09	0	\\
3.84e-09	0	\\
3.96e-09	0	\\
4.08e-09	0	\\
4.2e-09	0	\\
4.32e-09	0	\\
4.44e-09	0	\\
4.56e-09	0	\\
4.68e-09	0	\\
4.8e-09	0	\\
4.92e-09	0	\\
4.98e-09	0	\\
};
\addplot [color=red,solid,forget plot]
  table[row sep=crcr]{
0	0	\\
1.2e-10	0	\\
2.4e-10	0	\\
3.6e-10	0	\\
4.8e-10	0	\\
6e-10	0	\\
7.2e-10	0	\\
8.4e-10	0	\\
9.6e-10	0	\\
1.08e-09	0	\\
1.2e-09	0	\\
1.32e-09	0	\\
1.44e-09	0	\\
1.56e-09	0	\\
1.68e-09	0	\\
1.8e-09	0	\\
1.92e-09	0	\\
2.04e-09	0	\\
2.16e-09	0	\\
2.28e-09	0	\\
2.4e-09	0	\\
2.52e-09	0	\\
2.64e-09	0	\\
2.76e-09	0	\\
2.88e-09	0	\\
3e-09	0	\\
3.12e-09	0	\\
3.24e-09	0	\\
3.36e-09	0	\\
3.48e-09	0	\\
3.6e-09	0	\\
3.72e-09	0	\\
3.84e-09	0	\\
3.96e-09	0	\\
4.08e-09	0	\\
4.2e-09	0	\\
4.32e-09	0	\\
4.44e-09	0	\\
4.56e-09	0	\\
4.68e-09	0	\\
4.8e-09	0	\\
4.92e-09	0	\\
4.98e-09	0	\\
};
\addplot [color=mycolor1,solid,forget plot]
  table[row sep=crcr]{
0	0	\\
1.2e-10	0	\\
2.4e-10	0	\\
3.6e-10	0	\\
4.8e-10	0	\\
6e-10	0	\\
7.2e-10	0	\\
8.4e-10	0	\\
9.6e-10	0	\\
1.08e-09	0	\\
1.2e-09	0	\\
1.32e-09	0	\\
1.44e-09	0	\\
1.56e-09	0	\\
1.68e-09	0	\\
1.8e-09	0	\\
1.92e-09	0	\\
2.04e-09	0	\\
2.16e-09	0	\\
2.28e-09	0	\\
2.4e-09	0	\\
2.52e-09	0	\\
2.64e-09	0	\\
2.76e-09	0	\\
2.88e-09	0	\\
3e-09	0	\\
3.12e-09	0	\\
3.24e-09	0	\\
3.36e-09	0	\\
3.48e-09	0	\\
3.6e-09	0	\\
3.72e-09	0	\\
3.84e-09	0	\\
3.96e-09	0	\\
4.08e-09	0	\\
4.2e-09	0	\\
4.32e-09	0	\\
4.44e-09	0	\\
4.56e-09	0	\\
4.68e-09	0	\\
4.8e-09	0	\\
4.92e-09	0	\\
4.98e-09	0	\\
};
\addplot [color=mycolor2,solid,forget plot]
  table[row sep=crcr]{
0	0	\\
1.2e-10	0	\\
2.4e-10	0	\\
3.6e-10	0	\\
4.8e-10	0	\\
6e-10	0	\\
7.2e-10	0	\\
8.4e-10	0	\\
9.6e-10	0	\\
1.08e-09	0	\\
1.2e-09	0	\\
1.32e-09	0	\\
1.44e-09	0	\\
1.56e-09	0	\\
1.68e-09	0	\\
1.8e-09	0	\\
1.92e-09	0	\\
2.04e-09	0	\\
2.16e-09	0	\\
2.28e-09	0	\\
2.4e-09	0	\\
2.52e-09	0	\\
2.64e-09	0	\\
2.76e-09	0	\\
2.88e-09	0	\\
3e-09	0	\\
3.12e-09	0	\\
3.24e-09	0	\\
3.36e-09	0	\\
3.48e-09	0	\\
3.6e-09	0	\\
3.72e-09	0	\\
3.84e-09	0	\\
3.96e-09	0	\\
4.08e-09	0	\\
4.2e-09	0	\\
4.32e-09	0	\\
4.44e-09	0	\\
4.56e-09	0	\\
4.68e-09	0	\\
4.8e-09	0	\\
4.92e-09	0	\\
4.98e-09	0	\\
};
\addplot [color=mycolor3,solid,forget plot]
  table[row sep=crcr]{
0	0	\\
1.2e-10	0	\\
2.4e-10	0	\\
3.6e-10	0	\\
4.8e-10	0	\\
6e-10	0	\\
7.2e-10	0	\\
8.4e-10	0	\\
9.6e-10	0	\\
1.08e-09	0	\\
1.2e-09	0	\\
1.32e-09	0	\\
1.44e-09	0	\\
1.56e-09	0	\\
1.68e-09	0	\\
1.8e-09	0	\\
1.92e-09	0	\\
2.04e-09	0	\\
2.16e-09	0	\\
2.28e-09	0	\\
2.4e-09	0	\\
2.52e-09	0	\\
2.64e-09	0	\\
2.76e-09	0	\\
2.88e-09	0	\\
3e-09	0	\\
3.12e-09	0	\\
3.24e-09	0	\\
3.36e-09	0	\\
3.48e-09	0	\\
3.6e-09	0	\\
3.72e-09	0	\\
3.84e-09	0	\\
3.96e-09	0	\\
4.08e-09	0	\\
4.2e-09	0	\\
4.32e-09	0	\\
4.44e-09	0	\\
4.56e-09	0	\\
4.68e-09	0	\\
4.8e-09	0	\\
4.92e-09	0	\\
4.98e-09	0	\\
};
\addplot [color=darkgray,solid,forget plot]
  table[row sep=crcr]{
0	0	\\
1.2e-10	0	\\
2.4e-10	0	\\
3.6e-10	0	\\
4.8e-10	0	\\
6e-10	0	\\
7.2e-10	0	\\
8.4e-10	0	\\
9.6e-10	0	\\
1.08e-09	0	\\
1.2e-09	0	\\
1.32e-09	0	\\
1.44e-09	0	\\
1.56e-09	0	\\
1.68e-09	0	\\
1.8e-09	0	\\
1.92e-09	0	\\
2.04e-09	0	\\
2.16e-09	0	\\
2.28e-09	0	\\
2.4e-09	0	\\
2.52e-09	0	\\
2.64e-09	0	\\
2.76e-09	0	\\
2.88e-09	0	\\
3e-09	0	\\
3.12e-09	0	\\
3.24e-09	0	\\
3.36e-09	0	\\
3.48e-09	0	\\
3.6e-09	0	\\
3.72e-09	0	\\
3.84e-09	0	\\
3.96e-09	0	\\
4.08e-09	0	\\
4.2e-09	0	\\
4.32e-09	0	\\
4.44e-09	0	\\
4.56e-09	0	\\
4.68e-09	0	\\
4.8e-09	0	\\
4.92e-09	0	\\
4.98e-09	0	\\
};
\addplot [color=blue,solid,forget plot]
  table[row sep=crcr]{
0	0	\\
1.2e-10	0	\\
2.4e-10	0	\\
3.6e-10	0	\\
4.8e-10	0	\\
6e-10	0	\\
7.2e-10	0	\\
8.4e-10	0	\\
9.6e-10	0	\\
1.08e-09	0	\\
1.2e-09	0	\\
1.32e-09	0	\\
1.44e-09	0	\\
1.56e-09	0	\\
1.68e-09	0	\\
1.8e-09	0	\\
1.92e-09	0	\\
2.04e-09	0	\\
2.16e-09	0	\\
2.28e-09	0	\\
2.4e-09	0	\\
2.52e-09	0	\\
2.64e-09	0	\\
2.76e-09	0	\\
2.88e-09	0	\\
3e-09	0	\\
3.12e-09	0	\\
3.24e-09	0	\\
3.36e-09	0	\\
3.48e-09	0	\\
3.6e-09	0	\\
3.72e-09	0	\\
3.84e-09	0	\\
3.96e-09	0	\\
4.08e-09	0	\\
4.2e-09	0	\\
4.32e-09	0	\\
4.44e-09	0	\\
4.56e-09	0	\\
4.68e-09	0	\\
4.8e-09	0	\\
4.92e-09	0	\\
4.98e-09	0	\\
};
\addplot [color=black!50!green,solid,forget plot]
  table[row sep=crcr]{
0	0	\\
1.2e-10	0	\\
2.4e-10	0	\\
3.6e-10	0	\\
4.8e-10	0	\\
6e-10	0	\\
7.2e-10	0	\\
8.4e-10	0	\\
9.6e-10	0	\\
1.08e-09	0	\\
1.2e-09	0	\\
1.32e-09	0	\\
1.44e-09	0	\\
1.56e-09	0	\\
1.68e-09	0	\\
1.8e-09	0	\\
1.92e-09	0	\\
2.04e-09	0	\\
2.16e-09	0	\\
2.28e-09	0	\\
2.4e-09	0	\\
2.52e-09	0	\\
2.64e-09	0	\\
2.76e-09	0	\\
2.88e-09	0	\\
3e-09	0	\\
3.12e-09	0	\\
3.24e-09	0	\\
3.36e-09	0	\\
3.48e-09	0	\\
3.6e-09	0	\\
3.72e-09	0	\\
3.84e-09	0	\\
3.96e-09	0	\\
4.08e-09	0	\\
4.2e-09	0	\\
4.32e-09	0	\\
4.44e-09	0	\\
4.56e-09	0	\\
4.68e-09	0	\\
4.8e-09	0	\\
4.92e-09	0	\\
4.98e-09	0	\\
};
\addplot [color=red,solid,forget plot]
  table[row sep=crcr]{
0	0	\\
1.2e-10	0	\\
2.4e-10	0	\\
3.6e-10	0	\\
4.8e-10	0	\\
6e-10	0	\\
7.2e-10	0	\\
8.4e-10	0	\\
9.6e-10	0	\\
1.08e-09	0	\\
1.2e-09	0	\\
1.32e-09	0	\\
1.44e-09	0	\\
1.56e-09	0	\\
1.68e-09	0	\\
1.8e-09	0	\\
1.92e-09	0	\\
2.04e-09	0	\\
2.16e-09	0	\\
2.28e-09	0	\\
2.4e-09	0	\\
2.52e-09	0	\\
2.64e-09	0	\\
2.76e-09	0	\\
2.88e-09	0	\\
3e-09	0	\\
3.12e-09	0	\\
3.24e-09	0	\\
3.36e-09	0	\\
3.48e-09	0	\\
3.6e-09	0	\\
3.72e-09	0	\\
3.84e-09	0	\\
3.96e-09	0	\\
4.08e-09	0	\\
4.2e-09	0	\\
4.32e-09	0	\\
4.44e-09	0	\\
4.56e-09	0	\\
4.68e-09	0	\\
4.8e-09	0	\\
4.92e-09	0	\\
4.98e-09	0	\\
};
\addplot [color=mycolor1,solid,forget plot]
  table[row sep=crcr]{
0	0	\\
1.2e-10	0	\\
2.4e-10	0	\\
3.6e-10	0	\\
4.8e-10	0	\\
6e-10	0	\\
7.2e-10	0	\\
8.4e-10	0	\\
9.6e-10	0	\\
1.08e-09	0	\\
1.2e-09	0	\\
1.32e-09	0	\\
1.44e-09	0	\\
1.56e-09	0	\\
1.68e-09	0	\\
1.8e-09	0	\\
1.92e-09	0	\\
2.04e-09	0	\\
2.16e-09	0	\\
2.28e-09	0	\\
2.4e-09	0	\\
2.52e-09	0	\\
2.64e-09	0	\\
2.76e-09	0	\\
2.88e-09	0	\\
3e-09	0	\\
3.12e-09	0	\\
3.24e-09	0	\\
3.36e-09	0	\\
3.48e-09	0	\\
3.6e-09	0	\\
3.72e-09	0	\\
3.84e-09	0	\\
3.96e-09	0	\\
4.08e-09	0	\\
4.2e-09	0	\\
4.32e-09	0	\\
4.44e-09	0	\\
4.56e-09	0	\\
4.68e-09	0	\\
4.8e-09	0	\\
4.92e-09	0	\\
4.98e-09	0	\\
};
\addplot [color=mycolor2,solid,forget plot]
  table[row sep=crcr]{
0	0	\\
1.2e-10	0	\\
2.4e-10	0	\\
3.6e-10	0	\\
4.8e-10	0	\\
6e-10	0	\\
7.2e-10	0	\\
8.4e-10	0	\\
9.6e-10	0	\\
1.08e-09	0	\\
1.2e-09	0	\\
1.32e-09	0	\\
1.44e-09	0	\\
1.56e-09	0	\\
1.68e-09	0	\\
1.8e-09	0	\\
1.92e-09	0	\\
2.04e-09	0	\\
2.16e-09	0	\\
2.28e-09	0	\\
2.4e-09	0	\\
2.52e-09	0	\\
2.64e-09	0	\\
2.76e-09	0	\\
2.88e-09	0	\\
3e-09	0	\\
3.12e-09	0	\\
3.24e-09	0	\\
3.36e-09	0	\\
3.48e-09	0	\\
3.6e-09	0	\\
3.72e-09	0	\\
3.84e-09	0	\\
3.96e-09	0	\\
4.08e-09	0	\\
4.2e-09	0	\\
4.32e-09	0	\\
4.44e-09	0	\\
4.56e-09	0	\\
4.68e-09	0	\\
4.8e-09	0	\\
4.92e-09	0	\\
4.98e-09	0	\\
};
\addplot [color=mycolor3,solid,forget plot]
  table[row sep=crcr]{
0	0	\\
1.2e-10	0	\\
2.4e-10	0	\\
3.6e-10	0	\\
4.8e-10	0	\\
6e-10	0	\\
7.2e-10	0	\\
8.4e-10	0	\\
9.6e-10	0	\\
1.08e-09	0	\\
1.2e-09	0	\\
1.32e-09	0	\\
1.44e-09	0	\\
1.56e-09	0	\\
1.68e-09	0	\\
1.8e-09	0	\\
1.92e-09	0	\\
2.04e-09	0	\\
2.16e-09	0	\\
2.28e-09	0	\\
2.4e-09	0	\\
2.52e-09	0	\\
2.64e-09	0	\\
2.76e-09	0	\\
2.88e-09	0	\\
3e-09	0	\\
3.12e-09	0	\\
3.24e-09	0	\\
3.36e-09	0	\\
3.48e-09	0	\\
3.6e-09	0	\\
3.72e-09	0	\\
3.84e-09	0	\\
3.96e-09	0	\\
4.08e-09	0	\\
4.2e-09	0	\\
4.32e-09	0	\\
4.44e-09	0	\\
4.56e-09	0	\\
4.68e-09	0	\\
4.8e-09	0	\\
4.92e-09	0	\\
4.98e-09	0	\\
};
\addplot [color=darkgray,solid,forget plot]
  table[row sep=crcr]{
0	0	\\
1.2e-10	0	\\
2.4e-10	0	\\
3.6e-10	0	\\
4.8e-10	0	\\
6e-10	0	\\
7.2e-10	0	\\
8.4e-10	0	\\
9.6e-10	0	\\
1.08e-09	0	\\
1.2e-09	0	\\
1.32e-09	0	\\
1.44e-09	0	\\
1.56e-09	0	\\
1.68e-09	0	\\
1.8e-09	0	\\
1.92e-09	0	\\
2.04e-09	0	\\
2.16e-09	0	\\
2.28e-09	0	\\
2.4e-09	0	\\
2.52e-09	0	\\
2.64e-09	0	\\
2.76e-09	0	\\
2.88e-09	0	\\
3e-09	0	\\
3.12e-09	0	\\
3.24e-09	0	\\
3.36e-09	0	\\
3.48e-09	0	\\
3.6e-09	0	\\
3.72e-09	0	\\
3.84e-09	0	\\
3.96e-09	0	\\
4.08e-09	0	\\
4.2e-09	0	\\
4.32e-09	0	\\
4.44e-09	0	\\
4.56e-09	0	\\
4.68e-09	0	\\
4.8e-09	0	\\
4.92e-09	0	\\
4.98e-09	0	\\
};
\addplot [color=blue,solid,forget plot]
  table[row sep=crcr]{
0	0	\\
1.2e-10	0	\\
2.4e-10	0	\\
3.6e-10	0	\\
4.8e-10	0	\\
6e-10	0	\\
7.2e-10	0	\\
8.4e-10	0	\\
9.6e-10	0	\\
1.08e-09	0	\\
1.2e-09	0	\\
1.32e-09	0	\\
1.44e-09	0	\\
1.56e-09	0	\\
1.68e-09	0	\\
1.8e-09	0	\\
1.92e-09	0	\\
2.04e-09	0	\\
2.16e-09	0	\\
2.28e-09	0	\\
2.4e-09	0	\\
2.52e-09	0	\\
2.64e-09	0	\\
2.76e-09	0	\\
2.88e-09	0	\\
3e-09	0	\\
3.12e-09	0	\\
3.24e-09	0	\\
3.36e-09	0	\\
3.48e-09	0	\\
3.6e-09	0	\\
3.72e-09	0	\\
3.84e-09	0	\\
3.96e-09	0	\\
4.08e-09	0	\\
4.2e-09	0	\\
4.32e-09	0	\\
4.44e-09	0	\\
4.56e-09	0	\\
4.68e-09	0	\\
4.8e-09	0	\\
4.92e-09	0	\\
4.98e-09	0	\\
};
\addplot [color=black!50!green,solid,forget plot]
  table[row sep=crcr]{
0	0	\\
1.2e-10	0	\\
2.4e-10	0	\\
3.6e-10	0	\\
4.8e-10	0	\\
6e-10	0	\\
7.2e-10	0	\\
8.4e-10	0	\\
9.6e-10	0	\\
1.08e-09	0	\\
1.2e-09	0	\\
1.32e-09	0	\\
1.44e-09	0	\\
1.56e-09	0	\\
1.68e-09	0	\\
1.8e-09	0	\\
1.92e-09	0	\\
2.04e-09	0	\\
2.16e-09	0	\\
2.28e-09	0	\\
2.4e-09	0	\\
2.52e-09	0	\\
2.64e-09	0	\\
2.76e-09	0	\\
2.88e-09	0	\\
3e-09	0	\\
3.12e-09	0	\\
3.24e-09	0	\\
3.36e-09	0	\\
3.48e-09	0	\\
3.6e-09	0	\\
3.72e-09	0	\\
3.84e-09	0	\\
3.96e-09	0	\\
4.08e-09	0	\\
4.2e-09	0	\\
4.32e-09	0	\\
4.44e-09	0	\\
4.56e-09	0	\\
4.68e-09	0	\\
4.8e-09	0	\\
4.92e-09	0	\\
4.98e-09	0	\\
};
\addplot [color=red,solid,forget plot]
  table[row sep=crcr]{
0	0	\\
1.2e-10	0	\\
2.4e-10	0	\\
3.6e-10	0	\\
4.8e-10	0	\\
6e-10	0	\\
7.2e-10	0	\\
8.4e-10	0	\\
9.6e-10	0	\\
1.08e-09	0	\\
1.2e-09	0	\\
1.32e-09	0	\\
1.44e-09	0	\\
1.56e-09	0	\\
1.68e-09	0	\\
1.8e-09	0	\\
1.92e-09	0	\\
2.04e-09	0	\\
2.16e-09	0	\\
2.28e-09	0	\\
2.4e-09	0	\\
2.52e-09	0	\\
2.64e-09	0	\\
2.76e-09	0	\\
2.88e-09	0	\\
3e-09	0	\\
3.12e-09	0	\\
3.24e-09	0	\\
3.36e-09	0	\\
3.48e-09	0	\\
3.6e-09	0	\\
3.72e-09	0	\\
3.84e-09	0	\\
3.96e-09	0	\\
4.08e-09	0	\\
4.2e-09	0	\\
4.32e-09	0	\\
4.44e-09	0	\\
4.56e-09	0	\\
4.68e-09	0	\\
4.8e-09	0	\\
4.92e-09	0	\\
4.98e-09	0	\\
};
\addplot [color=mycolor1,solid,forget plot]
  table[row sep=crcr]{
0	0	\\
1.2e-10	0	\\
2.4e-10	0	\\
3.6e-10	0	\\
4.8e-10	0	\\
6e-10	0	\\
7.2e-10	0	\\
8.4e-10	0	\\
9.6e-10	0	\\
1.08e-09	0	\\
1.2e-09	0	\\
1.32e-09	0	\\
1.44e-09	0	\\
1.56e-09	0	\\
1.68e-09	0	\\
1.8e-09	0	\\
1.92e-09	0	\\
2.04e-09	0	\\
2.16e-09	0	\\
2.28e-09	0	\\
2.4e-09	0	\\
2.52e-09	0	\\
2.64e-09	0	\\
2.76e-09	0	\\
2.88e-09	0	\\
3e-09	0	\\
3.12e-09	0	\\
3.24e-09	0	\\
3.36e-09	0	\\
3.48e-09	0	\\
3.6e-09	0	\\
3.72e-09	0	\\
3.84e-09	0	\\
3.96e-09	0	\\
4.08e-09	0	\\
4.2e-09	0	\\
4.32e-09	0	\\
4.44e-09	0	\\
4.56e-09	0	\\
4.68e-09	0	\\
4.8e-09	0	\\
4.92e-09	0	\\
4.98e-09	0	\\
};
\addplot [color=mycolor2,solid,forget plot]
  table[row sep=crcr]{
0	0	\\
1.2e-10	0	\\
2.4e-10	0	\\
3.6e-10	0	\\
4.8e-10	0	\\
6e-10	0	\\
7.2e-10	0	\\
8.4e-10	0	\\
9.6e-10	0	\\
1.08e-09	0	\\
1.2e-09	0	\\
1.32e-09	0	\\
1.44e-09	0	\\
1.56e-09	0	\\
1.68e-09	0	\\
1.8e-09	0	\\
1.92e-09	0	\\
2.04e-09	0	\\
2.16e-09	0	\\
2.28e-09	0	\\
2.4e-09	0	\\
2.52e-09	0	\\
2.64e-09	0	\\
2.76e-09	0	\\
2.88e-09	0	\\
3e-09	0	\\
3.12e-09	0	\\
3.24e-09	0	\\
3.36e-09	0	\\
3.48e-09	0	\\
3.6e-09	0	\\
3.72e-09	0	\\
3.84e-09	0	\\
3.96e-09	0	\\
4.08e-09	0	\\
4.2e-09	0	\\
4.32e-09	0	\\
4.44e-09	0	\\
4.56e-09	0	\\
4.68e-09	0	\\
4.8e-09	0	\\
4.92e-09	0	\\
4.98e-09	0	\\
};
\addplot [color=mycolor3,solid,forget plot]
  table[row sep=crcr]{
0	0	\\
1.2e-10	0	\\
2.4e-10	0	\\
3.6e-10	0	\\
4.8e-10	0	\\
6e-10	0	\\
7.2e-10	0	\\
8.4e-10	0	\\
9.6e-10	0	\\
1.08e-09	0	\\
1.2e-09	0	\\
1.32e-09	0	\\
1.44e-09	0	\\
1.56e-09	0	\\
1.68e-09	0	\\
1.8e-09	0	\\
1.92e-09	0	\\
2.04e-09	0	\\
2.16e-09	0	\\
2.28e-09	0	\\
2.4e-09	0	\\
2.52e-09	0	\\
2.64e-09	0	\\
2.76e-09	0	\\
2.88e-09	0	\\
3e-09	0	\\
3.12e-09	0	\\
3.24e-09	0	\\
3.36e-09	0	\\
3.48e-09	0	\\
3.6e-09	0	\\
3.72e-09	0	\\
3.84e-09	0	\\
3.96e-09	0	\\
4.08e-09	0	\\
4.2e-09	0	\\
4.32e-09	0	\\
4.44e-09	0	\\
4.56e-09	0	\\
4.68e-09	0	\\
4.8e-09	0	\\
4.92e-09	0	\\
4.98e-09	0	\\
};
\addplot [color=darkgray,solid,forget plot]
  table[row sep=crcr]{
0	0	\\
1.2e-10	0	\\
2.4e-10	0	\\
3.6e-10	0	\\
4.8e-10	0	\\
6e-10	0	\\
7.2e-10	0	\\
8.4e-10	0	\\
9.6e-10	0	\\
1.08e-09	0	\\
1.2e-09	0	\\
1.32e-09	0	\\
1.44e-09	0	\\
1.56e-09	0	\\
1.68e-09	0	\\
1.8e-09	0	\\
1.92e-09	0	\\
2.04e-09	0	\\
2.16e-09	0	\\
2.28e-09	0	\\
2.4e-09	0	\\
2.52e-09	0	\\
2.64e-09	0	\\
2.76e-09	0	\\
2.88e-09	0	\\
3e-09	0	\\
3.12e-09	0	\\
3.24e-09	0	\\
3.36e-09	0	\\
3.48e-09	0	\\
3.6e-09	0	\\
3.72e-09	0	\\
3.84e-09	0	\\
3.96e-09	0	\\
4.08e-09	0	\\
4.2e-09	0	\\
4.32e-09	0	\\
4.44e-09	0	\\
4.56e-09	0	\\
4.68e-09	0	\\
4.8e-09	0	\\
4.92e-09	0	\\
4.98e-09	0	\\
};
\addplot [color=blue,solid,forget plot]
  table[row sep=crcr]{
0	0	\\
1.2e-10	0	\\
2.4e-10	0	\\
3.6e-10	0	\\
4.8e-10	0	\\
6e-10	0	\\
7.2e-10	0	\\
8.4e-10	0	\\
9.6e-10	0	\\
1.08e-09	0	\\
1.2e-09	0	\\
1.32e-09	0	\\
1.44e-09	0	\\
1.56e-09	0	\\
1.68e-09	0	\\
1.8e-09	0	\\
1.92e-09	0	\\
2.04e-09	0	\\
2.16e-09	0	\\
2.28e-09	0	\\
2.4e-09	0	\\
2.52e-09	0	\\
2.64e-09	0	\\
2.76e-09	0	\\
2.88e-09	0	\\
3e-09	0	\\
3.12e-09	0	\\
3.24e-09	0	\\
3.36e-09	0	\\
3.48e-09	0	\\
3.6e-09	0	\\
3.72e-09	0	\\
3.84e-09	0	\\
3.96e-09	0	\\
4.08e-09	0	\\
4.2e-09	0	\\
4.32e-09	0	\\
4.44e-09	0	\\
4.56e-09	0	\\
4.68e-09	0	\\
4.8e-09	0	\\
4.92e-09	0	\\
4.98e-09	0	\\
};
\addplot [color=black!50!green,solid,forget plot]
  table[row sep=crcr]{
0	0	\\
1.2e-10	0	\\
2.4e-10	0	\\
3.6e-10	0	\\
4.8e-10	0	\\
6e-10	0	\\
7.2e-10	0	\\
8.4e-10	0	\\
9.6e-10	0	\\
1.08e-09	0	\\
1.2e-09	0	\\
1.32e-09	0	\\
1.44e-09	0	\\
1.56e-09	0	\\
1.68e-09	0	\\
1.8e-09	0	\\
1.92e-09	0	\\
2.04e-09	0	\\
2.16e-09	0	\\
2.28e-09	0	\\
2.4e-09	0	\\
2.52e-09	0	\\
2.64e-09	0	\\
2.76e-09	0	\\
2.88e-09	0	\\
3e-09	0	\\
3.12e-09	0	\\
3.24e-09	0	\\
3.36e-09	0	\\
3.48e-09	0	\\
3.6e-09	0	\\
3.72e-09	0	\\
3.84e-09	0	\\
3.96e-09	0	\\
4.08e-09	0	\\
4.2e-09	0	\\
4.32e-09	0	\\
4.44e-09	0	\\
4.56e-09	0	\\
4.68e-09	0	\\
4.8e-09	0	\\
4.92e-09	0	\\
4.98e-09	0	\\
};
\addplot [color=red,solid,forget plot]
  table[row sep=crcr]{
0	0	\\
1.2e-10	0	\\
2.4e-10	0	\\
3.6e-10	0	\\
4.8e-10	0	\\
6e-10	0	\\
7.2e-10	0	\\
8.4e-10	0	\\
9.6e-10	0	\\
1.08e-09	0	\\
1.2e-09	0	\\
1.32e-09	0	\\
1.44e-09	0	\\
1.56e-09	0	\\
1.68e-09	0	\\
1.8e-09	0	\\
1.92e-09	0	\\
2.04e-09	0	\\
2.16e-09	0	\\
2.28e-09	0	\\
2.4e-09	0	\\
2.52e-09	0	\\
2.64e-09	0	\\
2.76e-09	0	\\
2.88e-09	0	\\
3e-09	0	\\
3.12e-09	0	\\
3.24e-09	0	\\
3.36e-09	0	\\
3.48e-09	0	\\
3.6e-09	0	\\
3.72e-09	0	\\
3.84e-09	0	\\
3.96e-09	0	\\
4.08e-09	0	\\
4.2e-09	0	\\
4.32e-09	0	\\
4.44e-09	0	\\
4.56e-09	0	\\
4.68e-09	0	\\
4.8e-09	0	\\
4.92e-09	0	\\
4.98e-09	0	\\
};
\addplot [color=mycolor1,solid,forget plot]
  table[row sep=crcr]{
0	0	\\
1.2e-10	0	\\
2.4e-10	0	\\
3.6e-10	0	\\
4.8e-10	0	\\
6e-10	0	\\
7.2e-10	0	\\
8.4e-10	0	\\
9.6e-10	0	\\
1.08e-09	0	\\
1.2e-09	0	\\
1.32e-09	0	\\
1.44e-09	0	\\
1.56e-09	0	\\
1.68e-09	0	\\
1.8e-09	0	\\
1.92e-09	0	\\
2.04e-09	0	\\
2.16e-09	0	\\
2.28e-09	0	\\
2.4e-09	0	\\
2.52e-09	0	\\
2.64e-09	0	\\
2.76e-09	0	\\
2.88e-09	0	\\
3e-09	0	\\
3.12e-09	0	\\
3.24e-09	0	\\
3.36e-09	0	\\
3.48e-09	0	\\
3.6e-09	0	\\
3.72e-09	0	\\
3.84e-09	0	\\
3.96e-09	0	\\
4.08e-09	0	\\
4.2e-09	0	\\
4.32e-09	0	\\
4.44e-09	0	\\
4.56e-09	0	\\
4.68e-09	0	\\
4.8e-09	0	\\
4.92e-09	0	\\
4.98e-09	0	\\
};
\addplot [color=mycolor2,solid,forget plot]
  table[row sep=crcr]{
0	0	\\
1.2e-10	0	\\
2.4e-10	0	\\
3.6e-10	0	\\
4.8e-10	0	\\
6e-10	0	\\
7.2e-10	0	\\
8.4e-10	0	\\
9.6e-10	0	\\
1.08e-09	0	\\
1.2e-09	0	\\
1.32e-09	0	\\
1.44e-09	0	\\
1.56e-09	0	\\
1.68e-09	0	\\
1.8e-09	0	\\
1.92e-09	0	\\
2.04e-09	0	\\
2.16e-09	0	\\
2.28e-09	0	\\
2.4e-09	0	\\
2.52e-09	0	\\
2.64e-09	0	\\
2.76e-09	0	\\
2.88e-09	0	\\
3e-09	0	\\
3.12e-09	0	\\
3.24e-09	0	\\
3.36e-09	0	\\
3.48e-09	0	\\
3.6e-09	0	\\
3.72e-09	0	\\
3.84e-09	0	\\
3.96e-09	0	\\
4.08e-09	0	\\
4.2e-09	0	\\
4.32e-09	0	\\
4.44e-09	0	\\
4.56e-09	0	\\
4.68e-09	0	\\
4.8e-09	0	\\
4.92e-09	0	\\
4.98e-09	0	\\
};
\addplot [color=mycolor3,solid,forget plot]
  table[row sep=crcr]{
0	0	\\
1.2e-10	0	\\
2.4e-10	0	\\
3.6e-10	0	\\
4.8e-10	0	\\
6e-10	0	\\
7.2e-10	0	\\
8.4e-10	0	\\
9.6e-10	0	\\
1.08e-09	0	\\
1.2e-09	0	\\
1.32e-09	0	\\
1.44e-09	0	\\
1.56e-09	0	\\
1.68e-09	0	\\
1.8e-09	0	\\
1.92e-09	0	\\
2.04e-09	0	\\
2.16e-09	0	\\
2.28e-09	0	\\
2.4e-09	0	\\
2.52e-09	0	\\
2.64e-09	0	\\
2.76e-09	0	\\
2.88e-09	0	\\
3e-09	0	\\
3.12e-09	0	\\
3.24e-09	0	\\
3.36e-09	0	\\
3.48e-09	0	\\
3.6e-09	0	\\
3.72e-09	0	\\
3.84e-09	0	\\
3.96e-09	0	\\
4.08e-09	0	\\
4.2e-09	0	\\
4.32e-09	0	\\
4.44e-09	0	\\
4.56e-09	0	\\
4.68e-09	0	\\
4.8e-09	0	\\
4.92e-09	0	\\
4.98e-09	0	\\
};
\addplot [color=darkgray,solid,forget plot]
  table[row sep=crcr]{
0	0	\\
1.2e-10	0	\\
2.4e-10	0	\\
3.6e-10	0	\\
4.8e-10	0	\\
6e-10	0	\\
7.2e-10	0	\\
8.4e-10	0	\\
9.6e-10	0	\\
1.08e-09	0	\\
1.2e-09	0	\\
1.32e-09	0	\\
1.44e-09	0	\\
1.56e-09	0	\\
1.68e-09	0	\\
1.8e-09	0	\\
1.92e-09	0	\\
2.04e-09	0	\\
2.16e-09	0	\\
2.28e-09	0	\\
2.4e-09	0	\\
2.52e-09	0	\\
2.64e-09	0	\\
2.76e-09	0	\\
2.88e-09	0	\\
3e-09	0	\\
3.12e-09	0	\\
3.24e-09	0	\\
3.36e-09	0	\\
3.48e-09	0	\\
3.6e-09	0	\\
3.72e-09	0	\\
3.84e-09	0	\\
3.96e-09	0	\\
4.08e-09	0	\\
4.2e-09	0	\\
4.32e-09	0	\\
4.44e-09	0	\\
4.56e-09	0	\\
4.68e-09	0	\\
4.8e-09	0	\\
4.92e-09	0	\\
4.98e-09	0	\\
};
\addplot [color=blue,solid,forget plot]
  table[row sep=crcr]{
0	0	\\
1.2e-10	0	\\
2.4e-10	0	\\
3.6e-10	0	\\
4.8e-10	0	\\
6e-10	0	\\
7.2e-10	0	\\
8.4e-10	0	\\
9.6e-10	0	\\
1.08e-09	0	\\
1.2e-09	0	\\
1.32e-09	0	\\
1.44e-09	0	\\
1.56e-09	0	\\
1.68e-09	0	\\
1.8e-09	0	\\
1.92e-09	0	\\
2.04e-09	0	\\
2.16e-09	0	\\
2.28e-09	0	\\
2.4e-09	0	\\
2.52e-09	0	\\
2.64e-09	0	\\
2.76e-09	0	\\
2.88e-09	0	\\
3e-09	0	\\
3.12e-09	0	\\
3.24e-09	0	\\
3.36e-09	0	\\
3.48e-09	0	\\
3.6e-09	0	\\
3.72e-09	0	\\
3.84e-09	0	\\
3.96e-09	0	\\
4.08e-09	0	\\
4.2e-09	0	\\
4.32e-09	0	\\
4.44e-09	0	\\
4.56e-09	0	\\
4.68e-09	0	\\
4.8e-09	0	\\
4.92e-09	0	\\
4.98e-09	0	\\
};
\addplot [color=black!50!green,solid,forget plot]
  table[row sep=crcr]{
0	0	\\
1.2e-10	0	\\
2.4e-10	0	\\
3.6e-10	0	\\
4.8e-10	0	\\
6e-10	0	\\
7.2e-10	0	\\
8.4e-10	0	\\
9.6e-10	0	\\
1.08e-09	0	\\
1.2e-09	0	\\
1.32e-09	0	\\
1.44e-09	0	\\
1.56e-09	0	\\
1.68e-09	0	\\
1.8e-09	0	\\
1.92e-09	0	\\
2.04e-09	0	\\
2.16e-09	0	\\
2.28e-09	0	\\
2.4e-09	0	\\
2.52e-09	0	\\
2.64e-09	0	\\
2.76e-09	0	\\
2.88e-09	0	\\
3e-09	0	\\
3.12e-09	0	\\
3.24e-09	0	\\
3.36e-09	0	\\
3.48e-09	0	\\
3.6e-09	0	\\
3.72e-09	0	\\
3.84e-09	0	\\
3.96e-09	0	\\
4.08e-09	0	\\
4.2e-09	0	\\
4.32e-09	0	\\
4.44e-09	0	\\
4.56e-09	0	\\
4.68e-09	0	\\
4.8e-09	0	\\
4.92e-09	0	\\
4.98e-09	0	\\
};
\addplot [color=red,solid,forget plot]
  table[row sep=crcr]{
0	0	\\
1.2e-10	0	\\
2.4e-10	0	\\
3.6e-10	0	\\
4.8e-10	0	\\
6e-10	0	\\
7.2e-10	0	\\
8.4e-10	0	\\
9.6e-10	0	\\
1.08e-09	0	\\
1.2e-09	0	\\
1.32e-09	0	\\
1.44e-09	0	\\
1.56e-09	0	\\
1.68e-09	0	\\
1.8e-09	0	\\
1.92e-09	0	\\
2.04e-09	0	\\
2.16e-09	0	\\
2.28e-09	0	\\
2.4e-09	0	\\
2.52e-09	0	\\
2.64e-09	0	\\
2.76e-09	0	\\
2.88e-09	0	\\
3e-09	0	\\
3.12e-09	0	\\
3.24e-09	0	\\
3.36e-09	0	\\
3.48e-09	0	\\
3.6e-09	0	\\
3.72e-09	0	\\
3.84e-09	0	\\
3.96e-09	0	\\
4.08e-09	0	\\
4.2e-09	0	\\
4.32e-09	0	\\
4.44e-09	0	\\
4.56e-09	0	\\
4.68e-09	0	\\
4.8e-09	0	\\
4.92e-09	0	\\
4.98e-09	0	\\
};
\addplot [color=mycolor1,solid,forget plot]
  table[row sep=crcr]{
0	0	\\
1.2e-10	0	\\
2.4e-10	0	\\
3.6e-10	0	\\
4.8e-10	0	\\
6e-10	0	\\
7.2e-10	0	\\
8.4e-10	0	\\
9.6e-10	0	\\
1.08e-09	0	\\
1.2e-09	0	\\
1.32e-09	0	\\
1.44e-09	0	\\
1.56e-09	0	\\
1.68e-09	0	\\
1.8e-09	0	\\
1.92e-09	0	\\
2.04e-09	0	\\
2.16e-09	0	\\
2.28e-09	0	\\
2.4e-09	0	\\
2.52e-09	0	\\
2.64e-09	0	\\
2.76e-09	0	\\
2.88e-09	0	\\
3e-09	0	\\
3.12e-09	0	\\
3.24e-09	0	\\
3.36e-09	0	\\
3.48e-09	0	\\
3.6e-09	0	\\
3.72e-09	0	\\
3.84e-09	0	\\
3.96e-09	0	\\
4.08e-09	0	\\
4.2e-09	0	\\
4.32e-09	0	\\
4.44e-09	0	\\
4.56e-09	0	\\
4.68e-09	0	\\
4.8e-09	0	\\
4.92e-09	0	\\
4.98e-09	0	\\
};
\addplot [color=mycolor2,solid,forget plot]
  table[row sep=crcr]{
0	0	\\
1.2e-10	0	\\
2.4e-10	0	\\
3.6e-10	0	\\
4.8e-10	0	\\
6e-10	0	\\
7.2e-10	0	\\
8.4e-10	0	\\
9.6e-10	0	\\
1.08e-09	0	\\
1.2e-09	0	\\
1.32e-09	0	\\
1.44e-09	0	\\
1.56e-09	0	\\
1.68e-09	0	\\
1.8e-09	0	\\
1.92e-09	0	\\
2.04e-09	0	\\
2.16e-09	0	\\
2.28e-09	0	\\
2.4e-09	0	\\
2.52e-09	0	\\
2.64e-09	0	\\
2.76e-09	0	\\
2.88e-09	0	\\
3e-09	0	\\
3.12e-09	0	\\
3.24e-09	0	\\
3.36e-09	0	\\
3.48e-09	0	\\
3.6e-09	0	\\
3.72e-09	0	\\
3.84e-09	0	\\
3.96e-09	0	\\
4.08e-09	0	\\
4.2e-09	0	\\
4.32e-09	0	\\
4.44e-09	0	\\
4.56e-09	0	\\
4.68e-09	0	\\
4.8e-09	0	\\
4.92e-09	0	\\
4.98e-09	0	\\
};
\addplot [color=mycolor3,solid,forget plot]
  table[row sep=crcr]{
0	0	\\
1.2e-10	0	\\
2.4e-10	0	\\
3.6e-10	0	\\
4.8e-10	0	\\
6e-10	0	\\
7.2e-10	0	\\
8.4e-10	0	\\
9.6e-10	0	\\
1.08e-09	0	\\
1.2e-09	0	\\
1.32e-09	0	\\
1.44e-09	0	\\
1.56e-09	0	\\
1.68e-09	0	\\
1.8e-09	0	\\
1.92e-09	0	\\
2.04e-09	0	\\
2.16e-09	0	\\
2.28e-09	0	\\
2.4e-09	0	\\
2.52e-09	0	\\
2.64e-09	0	\\
2.76e-09	0	\\
2.88e-09	0	\\
3e-09	0	\\
3.12e-09	0	\\
3.24e-09	0	\\
3.36e-09	0	\\
3.48e-09	0	\\
3.6e-09	0	\\
3.72e-09	0	\\
3.84e-09	0	\\
3.96e-09	0	\\
4.08e-09	0	\\
4.2e-09	0	\\
4.32e-09	0	\\
4.44e-09	0	\\
4.56e-09	0	\\
4.68e-09	0	\\
4.8e-09	0	\\
4.92e-09	0	\\
4.98e-09	0	\\
};
\addplot [color=darkgray,solid,forget plot]
  table[row sep=crcr]{
0	0	\\
1.2e-10	0	\\
2.4e-10	0	\\
3.6e-10	0	\\
4.8e-10	0	\\
6e-10	0	\\
7.2e-10	0	\\
8.4e-10	0	\\
9.6e-10	0	\\
1.08e-09	0	\\
1.2e-09	0	\\
1.32e-09	0	\\
1.44e-09	0	\\
1.56e-09	0	\\
1.68e-09	0	\\
1.8e-09	0	\\
1.92e-09	0	\\
2.04e-09	0	\\
2.16e-09	0	\\
2.28e-09	0	\\
2.4e-09	0	\\
2.52e-09	0	\\
2.64e-09	0	\\
2.76e-09	0	\\
2.88e-09	0	\\
3e-09	0	\\
3.12e-09	0	\\
3.24e-09	0	\\
3.36e-09	0	\\
3.48e-09	0	\\
3.6e-09	0	\\
3.72e-09	0	\\
3.84e-09	0	\\
3.96e-09	0	\\
4.08e-09	0	\\
4.2e-09	0	\\
4.32e-09	0	\\
4.44e-09	0	\\
4.56e-09	0	\\
4.68e-09	0	\\
4.8e-09	0	\\
4.92e-09	0	\\
4.98e-09	0	\\
};
\addplot [color=blue,solid,forget plot]
  table[row sep=crcr]{
0	0	\\
1.2e-10	0	\\
2.4e-10	0	\\
3.6e-10	0	\\
4.8e-10	0	\\
6e-10	0	\\
7.2e-10	0	\\
8.4e-10	0	\\
9.6e-10	0	\\
1.08e-09	0	\\
1.2e-09	0	\\
1.32e-09	0	\\
1.44e-09	0	\\
1.56e-09	0	\\
1.68e-09	0	\\
1.8e-09	0	\\
1.92e-09	0	\\
2.04e-09	0	\\
2.16e-09	0	\\
2.28e-09	0	\\
2.4e-09	0	\\
2.52e-09	0	\\
2.64e-09	0	\\
2.76e-09	0	\\
2.88e-09	0	\\
3e-09	0	\\
3.12e-09	0	\\
3.24e-09	0	\\
3.36e-09	0	\\
3.48e-09	0	\\
3.6e-09	0	\\
3.72e-09	0	\\
3.84e-09	0	\\
3.96e-09	0	\\
4.08e-09	0	\\
4.2e-09	0	\\
4.32e-09	0	\\
4.44e-09	0	\\
4.56e-09	0	\\
4.68e-09	0	\\
4.8e-09	0	\\
4.92e-09	0	\\
4.98e-09	0	\\
};
\addplot [color=black!50!green,solid,forget plot]
  table[row sep=crcr]{
0	0	\\
1.2e-10	0	\\
2.4e-10	0	\\
3.6e-10	0	\\
4.8e-10	0	\\
6e-10	0	\\
7.2e-10	0	\\
8.4e-10	0	\\
9.6e-10	0	\\
1.08e-09	0	\\
1.2e-09	0	\\
1.32e-09	0	\\
1.44e-09	0	\\
1.56e-09	0	\\
1.68e-09	0	\\
1.8e-09	0	\\
1.92e-09	0	\\
2.04e-09	0	\\
2.16e-09	0	\\
2.28e-09	0	\\
2.4e-09	0	\\
2.52e-09	0	\\
2.64e-09	0	\\
2.76e-09	0	\\
2.88e-09	0	\\
3e-09	0	\\
3.12e-09	0	\\
3.24e-09	0	\\
3.36e-09	0	\\
3.48e-09	0	\\
3.6e-09	0	\\
3.72e-09	0	\\
3.84e-09	0	\\
3.96e-09	0	\\
4.08e-09	0	\\
4.2e-09	0	\\
4.32e-09	0	\\
4.44e-09	0	\\
4.56e-09	0	\\
4.68e-09	0	\\
4.8e-09	0	\\
4.92e-09	0	\\
4.98e-09	0	\\
};
\addplot [color=red,solid,forget plot]
  table[row sep=crcr]{
0	0	\\
1.2e-10	0	\\
2.4e-10	0	\\
3.6e-10	0	\\
4.8e-10	0	\\
6e-10	0	\\
7.2e-10	0	\\
8.4e-10	0	\\
9.6e-10	0	\\
1.08e-09	0	\\
1.2e-09	0	\\
1.32e-09	0	\\
1.44e-09	0	\\
1.56e-09	0	\\
1.68e-09	0	\\
1.8e-09	0	\\
1.92e-09	0	\\
2.04e-09	0	\\
2.16e-09	0	\\
2.28e-09	0	\\
2.4e-09	0	\\
2.52e-09	0	\\
2.64e-09	0	\\
2.76e-09	0	\\
2.88e-09	0	\\
3e-09	0	\\
3.12e-09	0	\\
3.24e-09	0	\\
3.36e-09	0	\\
3.48e-09	0	\\
3.6e-09	0	\\
3.72e-09	0	\\
3.84e-09	0	\\
3.96e-09	0	\\
4.08e-09	0	\\
4.2e-09	0	\\
4.32e-09	0	\\
4.44e-09	0	\\
4.56e-09	0	\\
4.68e-09	0	\\
4.8e-09	0	\\
4.92e-09	0	\\
4.98e-09	0	\\
};
\addplot [color=mycolor1,solid,forget plot]
  table[row sep=crcr]{
0	0	\\
1.2e-10	0	\\
2.4e-10	0	\\
3.6e-10	0	\\
4.8e-10	0	\\
6e-10	0	\\
7.2e-10	0	\\
8.4e-10	0	\\
9.6e-10	0	\\
1.08e-09	0	\\
1.2e-09	0	\\
1.32e-09	0	\\
1.44e-09	0	\\
1.56e-09	0	\\
1.68e-09	0	\\
1.8e-09	0	\\
1.92e-09	0	\\
2.04e-09	0	\\
2.16e-09	0	\\
2.28e-09	0	\\
2.4e-09	0	\\
2.52e-09	0	\\
2.64e-09	0	\\
2.76e-09	0	\\
2.88e-09	0	\\
3e-09	0	\\
3.12e-09	0	\\
3.24e-09	0	\\
3.36e-09	0	\\
3.48e-09	0	\\
3.6e-09	0	\\
3.72e-09	0	\\
3.84e-09	0	\\
3.96e-09	0	\\
4.08e-09	0	\\
4.2e-09	0	\\
4.32e-09	0	\\
4.44e-09	0	\\
4.56e-09	0	\\
4.68e-09	0	\\
4.8e-09	0	\\
4.92e-09	0	\\
4.98e-09	0	\\
};
\addplot [color=mycolor2,solid,forget plot]
  table[row sep=crcr]{
0	0	\\
1.2e-10	0	\\
2.4e-10	0	\\
3.6e-10	0	\\
4.8e-10	0	\\
6e-10	0	\\
7.2e-10	0	\\
8.4e-10	0	\\
9.6e-10	0	\\
1.08e-09	0	\\
1.2e-09	0	\\
1.32e-09	0	\\
1.44e-09	0	\\
1.56e-09	0	\\
1.68e-09	0	\\
1.8e-09	0	\\
1.92e-09	0	\\
2.04e-09	0	\\
2.16e-09	0	\\
2.28e-09	0	\\
2.4e-09	0	\\
2.52e-09	0	\\
2.64e-09	0	\\
2.76e-09	0	\\
2.88e-09	0	\\
3e-09	0	\\
3.12e-09	0	\\
3.24e-09	0	\\
3.36e-09	0	\\
3.48e-09	0	\\
3.6e-09	0	\\
3.72e-09	0	\\
3.84e-09	0	\\
3.96e-09	0	\\
4.08e-09	0	\\
4.2e-09	0	\\
4.32e-09	0	\\
4.44e-09	0	\\
4.56e-09	0	\\
4.68e-09	0	\\
4.8e-09	0	\\
4.92e-09	0	\\
4.98e-09	0	\\
};
\addplot [color=mycolor3,solid,forget plot]
  table[row sep=crcr]{
0	0	\\
1.2e-10	0	\\
2.4e-10	0	\\
3.6e-10	0	\\
4.8e-10	0	\\
6e-10	0	\\
7.2e-10	0	\\
8.4e-10	0	\\
9.6e-10	0	\\
1.08e-09	0	\\
1.2e-09	0	\\
1.32e-09	0	\\
1.44e-09	0	\\
1.56e-09	0	\\
1.68e-09	0	\\
1.8e-09	0	\\
1.92e-09	0	\\
2.04e-09	0	\\
2.16e-09	0	\\
2.28e-09	0	\\
2.4e-09	0	\\
2.52e-09	0	\\
2.64e-09	0	\\
2.76e-09	0	\\
2.88e-09	0	\\
3e-09	0	\\
3.12e-09	0	\\
3.24e-09	0	\\
3.36e-09	0	\\
3.48e-09	0	\\
3.6e-09	0	\\
3.72e-09	0	\\
3.84e-09	0	\\
3.96e-09	0	\\
4.08e-09	0	\\
4.2e-09	0	\\
4.32e-09	0	\\
4.44e-09	0	\\
4.56e-09	0	\\
4.68e-09	0	\\
4.8e-09	0	\\
4.92e-09	0	\\
4.98e-09	0	\\
};
\addplot [color=darkgray,solid,forget plot]
  table[row sep=crcr]{
0	0	\\
1.2e-10	0	\\
2.4e-10	0	\\
3.6e-10	0	\\
4.8e-10	0	\\
6e-10	0	\\
7.2e-10	0	\\
8.4e-10	0	\\
9.6e-10	0	\\
1.08e-09	0	\\
1.2e-09	0	\\
1.32e-09	0	\\
1.44e-09	0	\\
1.56e-09	0	\\
1.68e-09	0	\\
1.8e-09	0	\\
1.92e-09	0	\\
2.04e-09	0	\\
2.16e-09	0	\\
2.28e-09	0	\\
2.4e-09	0	\\
2.52e-09	0	\\
2.64e-09	0	\\
2.76e-09	0	\\
2.88e-09	0	\\
3e-09	0	\\
3.12e-09	0	\\
3.24e-09	0	\\
3.36e-09	0	\\
3.48e-09	0	\\
3.6e-09	0	\\
3.72e-09	0	\\
3.84e-09	0	\\
3.96e-09	0	\\
4.08e-09	0	\\
4.2e-09	0	\\
4.32e-09	0	\\
4.44e-09	0	\\
4.56e-09	0	\\
4.68e-09	0	\\
4.8e-09	0	\\
4.92e-09	0	\\
4.98e-09	0	\\
};
\addplot [color=blue,solid,forget plot]
  table[row sep=crcr]{
0	0	\\
1.2e-10	0	\\
2.4e-10	0	\\
3.6e-10	0	\\
4.8e-10	0	\\
6e-10	0	\\
7.2e-10	0	\\
8.4e-10	0	\\
9.6e-10	0	\\
1.08e-09	0	\\
1.2e-09	0	\\
1.32e-09	0	\\
1.44e-09	0	\\
1.56e-09	0	\\
1.68e-09	0	\\
1.8e-09	0	\\
1.92e-09	0	\\
2.04e-09	0	\\
2.16e-09	0	\\
2.28e-09	0	\\
2.4e-09	0	\\
2.52e-09	0	\\
2.64e-09	0	\\
2.76e-09	0	\\
2.88e-09	0	\\
3e-09	0	\\
3.12e-09	0	\\
3.24e-09	0	\\
3.36e-09	0	\\
3.48e-09	0	\\
3.6e-09	0	\\
3.72e-09	0	\\
3.84e-09	0	\\
3.96e-09	0	\\
4.08e-09	0	\\
4.2e-09	0	\\
4.32e-09	0	\\
4.44e-09	0	\\
4.56e-09	0	\\
4.68e-09	0	\\
4.8e-09	0	\\
4.92e-09	0	\\
4.98e-09	0	\\
};
\addplot [color=black!50!green,solid,forget plot]
  table[row sep=crcr]{
0	0	\\
1.2e-10	0	\\
2.4e-10	0	\\
3.6e-10	0	\\
4.8e-10	0	\\
6e-10	0	\\
7.2e-10	0	\\
8.4e-10	0	\\
9.6e-10	0	\\
1.08e-09	0	\\
1.2e-09	0	\\
1.32e-09	0	\\
1.44e-09	0	\\
1.56e-09	0	\\
1.68e-09	0	\\
1.8e-09	0	\\
1.92e-09	0	\\
2.04e-09	0	\\
2.16e-09	0	\\
2.28e-09	0	\\
2.4e-09	0	\\
2.52e-09	0	\\
2.64e-09	0	\\
2.76e-09	0	\\
2.88e-09	0	\\
3e-09	0	\\
3.12e-09	0	\\
3.24e-09	0	\\
3.36e-09	0	\\
3.48e-09	0	\\
3.6e-09	0	\\
3.72e-09	0	\\
3.84e-09	0	\\
3.96e-09	0	\\
4.08e-09	0	\\
4.2e-09	0	\\
4.32e-09	0	\\
4.44e-09	0	\\
4.56e-09	0	\\
4.68e-09	0	\\
4.8e-09	0	\\
4.92e-09	0	\\
4.98e-09	0	\\
};
\addplot [color=red,solid,forget plot]
  table[row sep=crcr]{
0	0	\\
1.2e-10	0	\\
2.4e-10	0	\\
3.6e-10	0	\\
4.8e-10	0	\\
6e-10	0	\\
7.2e-10	0	\\
8.4e-10	0	\\
9.6e-10	0	\\
1.08e-09	0	\\
1.2e-09	0	\\
1.32e-09	0	\\
1.44e-09	0	\\
1.56e-09	0	\\
1.68e-09	0	\\
1.8e-09	0	\\
1.92e-09	0	\\
2.04e-09	0	\\
2.16e-09	0	\\
2.28e-09	0	\\
2.4e-09	0	\\
2.52e-09	0	\\
2.64e-09	0	\\
2.76e-09	0	\\
2.88e-09	0	\\
3e-09	0	\\
3.12e-09	0	\\
3.24e-09	0	\\
3.36e-09	0	\\
3.48e-09	0	\\
3.6e-09	0	\\
3.72e-09	0	\\
3.84e-09	0	\\
3.96e-09	0	\\
4.08e-09	0	\\
4.2e-09	0	\\
4.32e-09	0	\\
4.44e-09	0	\\
4.56e-09	0	\\
4.68e-09	0	\\
4.8e-09	0	\\
4.92e-09	0	\\
4.98e-09	0	\\
};
\addplot [color=mycolor1,solid,forget plot]
  table[row sep=crcr]{
0	0	\\
1.2e-10	0	\\
2.4e-10	0	\\
3.6e-10	0	\\
4.8e-10	0	\\
6e-10	0	\\
7.2e-10	0	\\
8.4e-10	0	\\
9.6e-10	0	\\
1.08e-09	0	\\
1.2e-09	0	\\
1.32e-09	0	\\
1.44e-09	0	\\
1.56e-09	0	\\
1.68e-09	0	\\
1.8e-09	0	\\
1.92e-09	0	\\
2.04e-09	0	\\
2.16e-09	0	\\
2.28e-09	0	\\
2.4e-09	0	\\
2.52e-09	0	\\
2.64e-09	0	\\
2.76e-09	0	\\
2.88e-09	0	\\
3e-09	0	\\
3.12e-09	0	\\
3.24e-09	0	\\
3.36e-09	0	\\
3.48e-09	0	\\
3.6e-09	0	\\
3.72e-09	0	\\
3.84e-09	0	\\
3.96e-09	0	\\
4.08e-09	0	\\
4.2e-09	0	\\
4.32e-09	0	\\
4.44e-09	0	\\
4.56e-09	0	\\
4.68e-09	0	\\
4.8e-09	0	\\
4.92e-09	0	\\
4.98e-09	0	\\
};
\addplot [color=mycolor2,solid,forget plot]
  table[row sep=crcr]{
0	0	\\
1.2e-10	0	\\
2.4e-10	0	\\
3.6e-10	0	\\
4.8e-10	0	\\
6e-10	0	\\
7.2e-10	0	\\
8.4e-10	0	\\
9.6e-10	0	\\
1.08e-09	0	\\
1.2e-09	0	\\
1.32e-09	0	\\
1.44e-09	0	\\
1.56e-09	0	\\
1.68e-09	0	\\
1.8e-09	0	\\
1.92e-09	0	\\
2.04e-09	0	\\
2.16e-09	0	\\
2.28e-09	0	\\
2.4e-09	0	\\
2.52e-09	0	\\
2.64e-09	0	\\
2.76e-09	0	\\
2.88e-09	0	\\
3e-09	0	\\
3.12e-09	0	\\
3.24e-09	0	\\
3.36e-09	0	\\
3.48e-09	0	\\
3.6e-09	0	\\
3.72e-09	0	\\
3.84e-09	0	\\
3.96e-09	0	\\
4.08e-09	0	\\
4.2e-09	0	\\
4.32e-09	0	\\
4.44e-09	0	\\
4.56e-09	0	\\
4.68e-09	0	\\
4.8e-09	0	\\
4.92e-09	0	\\
4.98e-09	0	\\
};
\addplot [color=mycolor3,solid,forget plot]
  table[row sep=crcr]{
0	0	\\
1.2e-10	0	\\
2.4e-10	0	\\
3.6e-10	0	\\
4.8e-10	0	\\
6e-10	0	\\
7.2e-10	0	\\
8.4e-10	0	\\
9.6e-10	0	\\
1.08e-09	0	\\
1.2e-09	0	\\
1.32e-09	0	\\
1.44e-09	0	\\
1.56e-09	0	\\
1.68e-09	0	\\
1.8e-09	0	\\
1.92e-09	0	\\
2.04e-09	0	\\
2.16e-09	0	\\
2.28e-09	0	\\
2.4e-09	0	\\
2.52e-09	0	\\
2.64e-09	0	\\
2.76e-09	0	\\
2.88e-09	0	\\
3e-09	0	\\
3.12e-09	0	\\
3.24e-09	0	\\
3.36e-09	0	\\
3.48e-09	0	\\
3.6e-09	0	\\
3.72e-09	0	\\
3.84e-09	0	\\
3.96e-09	0	\\
4.08e-09	0	\\
4.2e-09	0	\\
4.32e-09	0	\\
4.44e-09	0	\\
4.56e-09	0	\\
4.68e-09	0	\\
4.8e-09	0	\\
4.92e-09	0	\\
4.98e-09	0	\\
};
\addplot [color=darkgray,solid,forget plot]
  table[row sep=crcr]{
0	0	\\
1.2e-10	0	\\
2.4e-10	0	\\
3.6e-10	0	\\
4.8e-10	0	\\
6e-10	0	\\
7.2e-10	0	\\
8.4e-10	0	\\
9.6e-10	0	\\
1.08e-09	0	\\
1.2e-09	0	\\
1.32e-09	0	\\
1.44e-09	0	\\
1.56e-09	0	\\
1.68e-09	0	\\
1.8e-09	0	\\
1.92e-09	0	\\
2.04e-09	0	\\
2.16e-09	0	\\
2.28e-09	0	\\
2.4e-09	0	\\
2.52e-09	0	\\
2.64e-09	0	\\
2.76e-09	0	\\
2.88e-09	0	\\
3e-09	0	\\
3.12e-09	0	\\
3.24e-09	0	\\
3.36e-09	0	\\
3.48e-09	0	\\
3.6e-09	0	\\
3.72e-09	0	\\
3.84e-09	0	\\
3.96e-09	0	\\
4.08e-09	0	\\
4.2e-09	0	\\
4.32e-09	0	\\
4.44e-09	0	\\
4.56e-09	0	\\
4.68e-09	0	\\
4.8e-09	0	\\
4.92e-09	0	\\
4.98e-09	0	\\
};
\addplot [color=blue,solid,forget plot]
  table[row sep=crcr]{
0	0	\\
1.2e-10	0	\\
2.4e-10	0	\\
3.6e-10	0	\\
4.8e-10	0	\\
6e-10	0	\\
7.2e-10	0	\\
8.4e-10	0	\\
9.6e-10	0	\\
1.08e-09	0	\\
1.2e-09	0	\\
1.32e-09	0	\\
1.44e-09	0	\\
1.56e-09	0	\\
1.68e-09	0	\\
1.8e-09	0	\\
1.92e-09	0	\\
2.04e-09	0	\\
2.16e-09	0	\\
2.28e-09	0	\\
2.4e-09	0	\\
2.52e-09	0	\\
2.64e-09	0	\\
2.76e-09	0	\\
2.88e-09	0	\\
3e-09	0	\\
3.12e-09	0	\\
3.24e-09	0	\\
3.36e-09	0	\\
3.48e-09	0	\\
3.6e-09	0	\\
3.72e-09	0	\\
3.84e-09	0	\\
3.96e-09	0	\\
4.08e-09	0	\\
4.2e-09	0	\\
4.32e-09	0	\\
4.44e-09	0	\\
4.56e-09	0	\\
4.68e-09	0	\\
4.8e-09	0	\\
4.92e-09	0	\\
4.98e-09	0	\\
};
\addplot [color=black!50!green,solid,forget plot]
  table[row sep=crcr]{
0	0	\\
1.2e-10	0	\\
2.4e-10	0	\\
3.6e-10	0	\\
4.8e-10	0	\\
6e-10	0	\\
7.2e-10	0	\\
8.4e-10	0	\\
9.6e-10	0	\\
1.08e-09	0	\\
1.2e-09	0	\\
1.32e-09	0	\\
1.44e-09	0	\\
1.56e-09	0	\\
1.68e-09	0	\\
1.8e-09	0	\\
1.92e-09	0	\\
2.04e-09	0	\\
2.16e-09	0	\\
2.28e-09	0	\\
2.4e-09	0	\\
2.52e-09	0	\\
2.64e-09	0	\\
2.76e-09	0	\\
2.88e-09	0	\\
3e-09	0	\\
3.12e-09	0	\\
3.24e-09	0	\\
3.36e-09	0	\\
3.48e-09	0	\\
3.6e-09	0	\\
3.72e-09	0	\\
3.84e-09	0	\\
3.96e-09	0	\\
4.08e-09	0	\\
4.2e-09	0	\\
4.32e-09	0	\\
4.44e-09	0	\\
4.56e-09	0	\\
4.68e-09	0	\\
4.8e-09	0	\\
4.92e-09	0	\\
4.98e-09	0	\\
};
\addplot [color=red,solid,forget plot]
  table[row sep=crcr]{
0	0	\\
1.2e-10	0	\\
2.4e-10	0	\\
3.6e-10	0	\\
4.8e-10	0	\\
6e-10	0	\\
7.2e-10	0	\\
8.4e-10	0	\\
9.6e-10	0	\\
1.08e-09	0	\\
1.2e-09	0	\\
1.32e-09	0	\\
1.44e-09	0	\\
1.56e-09	0	\\
1.68e-09	0	\\
1.8e-09	0	\\
1.92e-09	0	\\
2.04e-09	0	\\
2.16e-09	0	\\
2.28e-09	0	\\
2.4e-09	0	\\
2.52e-09	0	\\
2.64e-09	0	\\
2.76e-09	0	\\
2.88e-09	0	\\
3e-09	0	\\
3.12e-09	0	\\
3.24e-09	0	\\
3.36e-09	0	\\
3.48e-09	0	\\
3.6e-09	0	\\
3.72e-09	0	\\
3.84e-09	0	\\
3.96e-09	0	\\
4.08e-09	0	\\
4.2e-09	0	\\
4.32e-09	0	\\
4.44e-09	0	\\
4.56e-09	0	\\
4.68e-09	0	\\
4.8e-09	0	\\
4.92e-09	0	\\
4.98e-09	0	\\
};
\addplot [color=mycolor1,solid,forget plot]
  table[row sep=crcr]{
0	0	\\
1.2e-10	0	\\
2.4e-10	0	\\
3.6e-10	0	\\
4.8e-10	0	\\
6e-10	0	\\
7.2e-10	0	\\
8.4e-10	0	\\
9.6e-10	0	\\
1.08e-09	0	\\
1.2e-09	0	\\
1.32e-09	0	\\
1.44e-09	0	\\
1.56e-09	0	\\
1.68e-09	0	\\
1.8e-09	0	\\
1.92e-09	0	\\
2.04e-09	0	\\
2.16e-09	0	\\
2.28e-09	0	\\
2.4e-09	0	\\
2.52e-09	0	\\
2.64e-09	0	\\
2.76e-09	0	\\
2.88e-09	0	\\
3e-09	0	\\
3.12e-09	0	\\
3.24e-09	0	\\
3.36e-09	0	\\
3.48e-09	0	\\
3.6e-09	0	\\
3.72e-09	0	\\
3.84e-09	0	\\
3.96e-09	0	\\
4.08e-09	0	\\
4.2e-09	0	\\
4.32e-09	0	\\
4.44e-09	0	\\
4.56e-09	0	\\
4.68e-09	0	\\
4.8e-09	0	\\
4.92e-09	0	\\
4.98e-09	0	\\
};
\addplot [color=mycolor2,solid,forget plot]
  table[row sep=crcr]{
0	0	\\
1.2e-10	0	\\
2.4e-10	0	\\
3.6e-10	0	\\
4.8e-10	0	\\
6e-10	0	\\
7.2e-10	0	\\
8.4e-10	0	\\
9.6e-10	0	\\
1.08e-09	0	\\
1.2e-09	0	\\
1.32e-09	0	\\
1.44e-09	0	\\
1.56e-09	0	\\
1.68e-09	0	\\
1.8e-09	0	\\
1.92e-09	0	\\
2.04e-09	0	\\
2.16e-09	0	\\
2.28e-09	0	\\
2.4e-09	0	\\
2.52e-09	0	\\
2.64e-09	0	\\
2.76e-09	0	\\
2.88e-09	0	\\
3e-09	0	\\
3.12e-09	0	\\
3.24e-09	0	\\
3.36e-09	0	\\
3.48e-09	0	\\
3.6e-09	0	\\
3.72e-09	0	\\
3.84e-09	0	\\
3.96e-09	0	\\
4.08e-09	0	\\
4.2e-09	0	\\
4.32e-09	0	\\
4.44e-09	0	\\
4.56e-09	0	\\
4.68e-09	0	\\
4.8e-09	0	\\
4.92e-09	0	\\
4.98e-09	0	\\
};
\addplot [color=mycolor3,solid,forget plot]
  table[row sep=crcr]{
0	0	\\
1.2e-10	0	\\
2.4e-10	0	\\
3.6e-10	0	\\
4.8e-10	0	\\
6e-10	0	\\
7.2e-10	0	\\
8.4e-10	0	\\
9.6e-10	0	\\
1.08e-09	0	\\
1.2e-09	0	\\
1.32e-09	0	\\
1.44e-09	0	\\
1.56e-09	0	\\
1.68e-09	0	\\
1.8e-09	0	\\
1.92e-09	0	\\
2.04e-09	0	\\
2.16e-09	0	\\
2.28e-09	0	\\
2.4e-09	0	\\
2.52e-09	0	\\
2.64e-09	0	\\
2.76e-09	0	\\
2.88e-09	0	\\
3e-09	0	\\
3.12e-09	0	\\
3.24e-09	0	\\
3.36e-09	0	\\
3.48e-09	0	\\
3.6e-09	0	\\
3.72e-09	0	\\
3.84e-09	0	\\
3.96e-09	0	\\
4.08e-09	0	\\
4.2e-09	0	\\
4.32e-09	0	\\
4.44e-09	0	\\
4.56e-09	0	\\
4.68e-09	0	\\
4.8e-09	0	\\
4.92e-09	0	\\
4.98e-09	0	\\
};
\addplot [color=darkgray,solid,forget plot]
  table[row sep=crcr]{
0	0	\\
1.2e-10	0	\\
2.4e-10	0	\\
3.6e-10	0	\\
4.8e-10	0	\\
6e-10	0	\\
7.2e-10	0	\\
8.4e-10	0	\\
9.6e-10	0	\\
1.08e-09	0	\\
1.2e-09	0	\\
1.32e-09	0	\\
1.44e-09	0	\\
1.56e-09	0	\\
1.68e-09	0	\\
1.8e-09	0	\\
1.92e-09	0	\\
2.04e-09	0	\\
2.16e-09	0	\\
2.28e-09	0	\\
2.4e-09	0	\\
2.52e-09	0	\\
2.64e-09	0	\\
2.76e-09	0	\\
2.88e-09	0	\\
3e-09	0	\\
3.12e-09	0	\\
3.24e-09	0	\\
3.36e-09	0	\\
3.48e-09	0	\\
3.6e-09	0	\\
3.72e-09	0	\\
3.84e-09	0	\\
3.96e-09	0	\\
4.08e-09	0	\\
4.2e-09	0	\\
4.32e-09	0	\\
4.44e-09	0	\\
4.56e-09	0	\\
4.68e-09	0	\\
4.8e-09	0	\\
4.92e-09	0	\\
4.98e-09	0	\\
};
\addplot [color=blue,solid,forget plot]
  table[row sep=crcr]{
0	0	\\
1.2e-10	0	\\
2.4e-10	0	\\
3.6e-10	0	\\
4.8e-10	0	\\
6e-10	0	\\
7.2e-10	0	\\
8.4e-10	0	\\
9.6e-10	0	\\
1.08e-09	0	\\
1.2e-09	0	\\
1.32e-09	0	\\
1.44e-09	0	\\
1.56e-09	0	\\
1.68e-09	0	\\
1.8e-09	0	\\
1.92e-09	0	\\
2.04e-09	0	\\
2.16e-09	0	\\
2.28e-09	0	\\
2.4e-09	0	\\
2.52e-09	0	\\
2.64e-09	0	\\
2.76e-09	0	\\
2.88e-09	0	\\
3e-09	0	\\
3.12e-09	0	\\
3.24e-09	0	\\
3.36e-09	0	\\
3.48e-09	0	\\
3.6e-09	0	\\
3.72e-09	0	\\
3.84e-09	0	\\
3.96e-09	0	\\
4.08e-09	0	\\
4.2e-09	0	\\
4.32e-09	0	\\
4.44e-09	0	\\
4.56e-09	0	\\
4.68e-09	0	\\
4.8e-09	0	\\
4.92e-09	0	\\
4.98e-09	0	\\
};
\addplot [color=black!50!green,solid,forget plot]
  table[row sep=crcr]{
0	0	\\
1.2e-10	0	\\
2.4e-10	0	\\
3.6e-10	0	\\
4.8e-10	0	\\
6e-10	0	\\
7.2e-10	0	\\
8.4e-10	0	\\
9.6e-10	0	\\
1.08e-09	0	\\
1.2e-09	0	\\
1.32e-09	0	\\
1.44e-09	0	\\
1.56e-09	0	\\
1.68e-09	0	\\
1.8e-09	0	\\
1.92e-09	0	\\
2.04e-09	0	\\
2.16e-09	0	\\
2.28e-09	0	\\
2.4e-09	0	\\
2.52e-09	0	\\
2.64e-09	0	\\
2.76e-09	0	\\
2.88e-09	0	\\
3e-09	0	\\
3.12e-09	0	\\
3.24e-09	0	\\
3.36e-09	0	\\
3.48e-09	0	\\
3.6e-09	0	\\
3.72e-09	0	\\
3.84e-09	0	\\
3.96e-09	0	\\
4.08e-09	0	\\
4.2e-09	0	\\
4.32e-09	0	\\
4.44e-09	0	\\
4.56e-09	0	\\
4.68e-09	0	\\
4.8e-09	0	\\
4.92e-09	0	\\
4.98e-09	0	\\
};
\addplot [color=red,solid,forget plot]
  table[row sep=crcr]{
0	0	\\
1.2e-10	0	\\
2.4e-10	0	\\
3.6e-10	0	\\
4.8e-10	0	\\
6e-10	0	\\
7.2e-10	0	\\
8.4e-10	0	\\
9.6e-10	0	\\
1.08e-09	0	\\
1.2e-09	0	\\
1.32e-09	0	\\
1.44e-09	0	\\
1.56e-09	0	\\
1.68e-09	0	\\
1.8e-09	0	\\
1.92e-09	0	\\
2.04e-09	0	\\
2.16e-09	0	\\
2.28e-09	0	\\
2.4e-09	0	\\
2.52e-09	0	\\
2.64e-09	0	\\
2.76e-09	0	\\
2.88e-09	0	\\
3e-09	0	\\
3.12e-09	0	\\
3.24e-09	0	\\
3.36e-09	0	\\
3.48e-09	0	\\
3.6e-09	0	\\
3.72e-09	0	\\
3.84e-09	0	\\
3.96e-09	0	\\
4.08e-09	0	\\
4.2e-09	0	\\
4.32e-09	0	\\
4.44e-09	0	\\
4.56e-09	0	\\
4.68e-09	0	\\
4.8e-09	0	\\
4.92e-09	0	\\
4.98e-09	0	\\
};
\addplot [color=mycolor1,solid,forget plot]
  table[row sep=crcr]{
0	0	\\
1.2e-10	0	\\
2.4e-10	0	\\
3.6e-10	0	\\
4.8e-10	0	\\
6e-10	0	\\
7.2e-10	0	\\
8.4e-10	0	\\
9.6e-10	0	\\
1.08e-09	0	\\
1.2e-09	0	\\
1.32e-09	0	\\
1.44e-09	0	\\
1.56e-09	0	\\
1.68e-09	0	\\
1.8e-09	0	\\
1.92e-09	0	\\
2.04e-09	0	\\
2.16e-09	0	\\
2.28e-09	0	\\
2.4e-09	0	\\
2.52e-09	0	\\
2.64e-09	0	\\
2.76e-09	0	\\
2.88e-09	0	\\
3e-09	0	\\
3.12e-09	0	\\
3.24e-09	0	\\
3.36e-09	0	\\
3.48e-09	0	\\
3.6e-09	0	\\
3.72e-09	0	\\
3.84e-09	0	\\
3.96e-09	0	\\
4.08e-09	0	\\
4.2e-09	0	\\
4.32e-09	0	\\
4.44e-09	0	\\
4.56e-09	0	\\
4.68e-09	0	\\
4.8e-09	0	\\
4.92e-09	0	\\
4.98e-09	0	\\
};
\addplot [color=mycolor2,solid,forget plot]
  table[row sep=crcr]{
0	0	\\
1.2e-10	0	\\
2.4e-10	0	\\
3.6e-10	0	\\
4.8e-10	0	\\
6e-10	0	\\
7.2e-10	0	\\
8.4e-10	0	\\
9.6e-10	0	\\
1.08e-09	0	\\
1.2e-09	0	\\
1.32e-09	0	\\
1.44e-09	0	\\
1.56e-09	0	\\
1.68e-09	0	\\
1.8e-09	0	\\
1.92e-09	0	\\
2.04e-09	0	\\
2.16e-09	0	\\
2.28e-09	0	\\
2.4e-09	0	\\
2.52e-09	0	\\
2.64e-09	0	\\
2.76e-09	0	\\
2.88e-09	0	\\
3e-09	0	\\
3.12e-09	0	\\
3.24e-09	0	\\
3.36e-09	0	\\
3.48e-09	0	\\
3.6e-09	0	\\
3.72e-09	0	\\
3.84e-09	0	\\
3.96e-09	0	\\
4.08e-09	0	\\
4.2e-09	0	\\
4.32e-09	0	\\
4.44e-09	0	\\
4.56e-09	0	\\
4.68e-09	0	\\
4.8e-09	0	\\
4.92e-09	0	\\
4.98e-09	0	\\
};
\addplot [color=mycolor3,solid,forget plot]
  table[row sep=crcr]{
0	0	\\
1.2e-10	0	\\
2.4e-10	0	\\
3.6e-10	0	\\
4.8e-10	0	\\
6e-10	0	\\
7.2e-10	0	\\
8.4e-10	0	\\
9.6e-10	0	\\
1.08e-09	0	\\
1.2e-09	0	\\
1.32e-09	0	\\
1.44e-09	0	\\
1.56e-09	0	\\
1.68e-09	0	\\
1.8e-09	0	\\
1.92e-09	0	\\
2.04e-09	0	\\
2.16e-09	0	\\
2.28e-09	0	\\
2.4e-09	0	\\
2.52e-09	0	\\
2.64e-09	0	\\
2.76e-09	0	\\
2.88e-09	0	\\
3e-09	0	\\
3.12e-09	0	\\
3.24e-09	0	\\
3.36e-09	0	\\
3.48e-09	0	\\
3.6e-09	0	\\
3.72e-09	0	\\
3.84e-09	0	\\
3.96e-09	0	\\
4.08e-09	0	\\
4.2e-09	0	\\
4.32e-09	0	\\
4.44e-09	0	\\
4.56e-09	0	\\
4.68e-09	0	\\
4.8e-09	0	\\
4.92e-09	0	\\
4.98e-09	0	\\
};
\addplot [color=darkgray,solid,forget plot]
  table[row sep=crcr]{
0	0	\\
1.2e-10	0	\\
2.4e-10	0	\\
3.6e-10	0	\\
4.8e-10	0	\\
6e-10	0	\\
7.2e-10	0	\\
8.4e-10	0	\\
9.6e-10	0	\\
1.08e-09	0	\\
1.2e-09	0	\\
1.32e-09	0	\\
1.44e-09	0	\\
1.56e-09	0	\\
1.68e-09	0	\\
1.8e-09	0	\\
1.92e-09	0	\\
2.04e-09	0	\\
2.16e-09	0	\\
2.28e-09	0	\\
2.4e-09	0	\\
2.52e-09	0	\\
2.64e-09	0	\\
2.76e-09	0	\\
2.88e-09	0	\\
3e-09	0	\\
3.12e-09	0	\\
3.24e-09	0	\\
3.36e-09	0	\\
3.48e-09	0	\\
3.6e-09	0	\\
3.72e-09	0	\\
3.84e-09	0	\\
3.96e-09	0	\\
4.08e-09	0	\\
4.2e-09	0	\\
4.32e-09	0	\\
4.44e-09	0	\\
4.56e-09	0	\\
4.68e-09	0	\\
4.8e-09	0	\\
4.92e-09	0	\\
4.98e-09	0	\\
};
\addplot [color=blue,solid,forget plot]
  table[row sep=crcr]{
0	0	\\
1.2e-10	0	\\
2.4e-10	0	\\
3.6e-10	0	\\
4.8e-10	0	\\
6e-10	0	\\
7.2e-10	0	\\
8.4e-10	0	\\
9.6e-10	0	\\
1.08e-09	0	\\
1.2e-09	0	\\
1.32e-09	0	\\
1.44e-09	0	\\
1.56e-09	0	\\
1.68e-09	0	\\
1.8e-09	0	\\
1.92e-09	0	\\
2.04e-09	0	\\
2.16e-09	0	\\
2.28e-09	0	\\
2.4e-09	0	\\
2.52e-09	0	\\
2.64e-09	0	\\
2.76e-09	0	\\
2.88e-09	0	\\
3e-09	0	\\
3.12e-09	0	\\
3.24e-09	0	\\
3.36e-09	0	\\
3.48e-09	0	\\
3.6e-09	0	\\
3.72e-09	0	\\
3.84e-09	0	\\
3.96e-09	0	\\
4.08e-09	0	\\
4.2e-09	0	\\
4.32e-09	0	\\
4.44e-09	0	\\
4.56e-09	0	\\
4.68e-09	0	\\
4.8e-09	0	\\
4.92e-09	0	\\
4.98e-09	0	\\
};
\addplot [color=black!50!green,solid,forget plot]
  table[row sep=crcr]{
0	0	\\
1.2e-10	0	\\
2.4e-10	0	\\
3.6e-10	0	\\
4.8e-10	0	\\
6e-10	0	\\
7.2e-10	0	\\
8.4e-10	0	\\
9.6e-10	0	\\
1.08e-09	0	\\
1.2e-09	0	\\
1.32e-09	0	\\
1.44e-09	0	\\
1.56e-09	0	\\
1.68e-09	0	\\
1.8e-09	0	\\
1.92e-09	0	\\
2.04e-09	0	\\
2.16e-09	0	\\
2.28e-09	0	\\
2.4e-09	0	\\
2.52e-09	0	\\
2.64e-09	0	\\
2.76e-09	0	\\
2.88e-09	0	\\
3e-09	0	\\
3.12e-09	0	\\
3.24e-09	0	\\
3.36e-09	0	\\
3.48e-09	0	\\
3.6e-09	0	\\
3.72e-09	0	\\
3.84e-09	0	\\
3.96e-09	0	\\
4.08e-09	0	\\
4.2e-09	0	\\
4.32e-09	0	\\
4.44e-09	0	\\
4.56e-09	0	\\
4.68e-09	0	\\
4.8e-09	0	\\
4.92e-09	0	\\
4.98e-09	0	\\
};
\addplot [color=red,solid,forget plot]
  table[row sep=crcr]{
0	0	\\
1.2e-10	0	\\
2.4e-10	0	\\
3.6e-10	0	\\
4.8e-10	0	\\
6e-10	0	\\
7.2e-10	0	\\
8.4e-10	0	\\
9.6e-10	0	\\
1.08e-09	0	\\
1.2e-09	0	\\
1.32e-09	0	\\
1.44e-09	0	\\
1.56e-09	0	\\
1.68e-09	0	\\
1.8e-09	0	\\
1.92e-09	0	\\
2.04e-09	0	\\
2.16e-09	0	\\
2.28e-09	0	\\
2.4e-09	0	\\
2.52e-09	0	\\
2.64e-09	0	\\
2.76e-09	0	\\
2.88e-09	0	\\
3e-09	0	\\
3.12e-09	0	\\
3.24e-09	0	\\
3.36e-09	0	\\
3.48e-09	0	\\
3.6e-09	0	\\
3.72e-09	0	\\
3.84e-09	0	\\
3.96e-09	0	\\
4.08e-09	0	\\
4.2e-09	0	\\
4.32e-09	0	\\
4.44e-09	0	\\
4.56e-09	0	\\
4.68e-09	0	\\
4.8e-09	0	\\
4.92e-09	0	\\
4.98e-09	0	\\
};
\addplot [color=mycolor1,solid,forget plot]
  table[row sep=crcr]{
0	0	\\
1.2e-10	0	\\
2.4e-10	0	\\
3.6e-10	0	\\
4.8e-10	0	\\
6e-10	0	\\
7.2e-10	0	\\
8.4e-10	0	\\
9.6e-10	0	\\
1.08e-09	0	\\
1.2e-09	0	\\
1.32e-09	0	\\
1.44e-09	0	\\
1.56e-09	0	\\
1.68e-09	0	\\
1.8e-09	0	\\
1.92e-09	0	\\
2.04e-09	0	\\
2.16e-09	0	\\
2.28e-09	0	\\
2.4e-09	0	\\
2.52e-09	0	\\
2.64e-09	0	\\
2.76e-09	0	\\
2.88e-09	0	\\
3e-09	0	\\
3.12e-09	0	\\
3.24e-09	0	\\
3.36e-09	0	\\
3.48e-09	0	\\
3.6e-09	0	\\
3.72e-09	0	\\
3.84e-09	0	\\
3.96e-09	0	\\
4.08e-09	0	\\
4.2e-09	0	\\
4.32e-09	0	\\
4.44e-09	0	\\
4.56e-09	0	\\
4.68e-09	0	\\
4.8e-09	0	\\
4.92e-09	0	\\
4.98e-09	0	\\
};
\addplot [color=mycolor2,solid,forget plot]
  table[row sep=crcr]{
0	0	\\
1.2e-10	0	\\
2.4e-10	0	\\
3.6e-10	0	\\
4.8e-10	0	\\
6e-10	0	\\
7.2e-10	0	\\
8.4e-10	0	\\
9.6e-10	0	\\
1.08e-09	0	\\
1.2e-09	0	\\
1.32e-09	0	\\
1.44e-09	0	\\
1.56e-09	0	\\
1.68e-09	0	\\
1.8e-09	0	\\
1.92e-09	0	\\
2.04e-09	0	\\
2.16e-09	0	\\
2.28e-09	0	\\
2.4e-09	0	\\
2.52e-09	0	\\
2.64e-09	0	\\
2.76e-09	0	\\
2.88e-09	0	\\
3e-09	0	\\
3.12e-09	0	\\
3.24e-09	0	\\
3.36e-09	0	\\
3.48e-09	0	\\
3.6e-09	0	\\
3.72e-09	0	\\
3.84e-09	0	\\
3.96e-09	0	\\
4.08e-09	0	\\
4.2e-09	0	\\
4.32e-09	0	\\
4.44e-09	0	\\
4.56e-09	0	\\
4.68e-09	0	\\
4.8e-09	0	\\
4.92e-09	0	\\
4.98e-09	0	\\
};
\addplot [color=mycolor3,solid,forget plot]
  table[row sep=crcr]{
0	0	\\
1.2e-10	0	\\
2.4e-10	0	\\
3.6e-10	0	\\
4.8e-10	0	\\
6e-10	0	\\
7.2e-10	0	\\
8.4e-10	0	\\
9.6e-10	0	\\
1.08e-09	0	\\
1.2e-09	0	\\
1.32e-09	0	\\
1.44e-09	0	\\
1.56e-09	0	\\
1.68e-09	0	\\
1.8e-09	0	\\
1.92e-09	0	\\
2.04e-09	0	\\
2.16e-09	0	\\
2.28e-09	0	\\
2.4e-09	0	\\
2.52e-09	0	\\
2.64e-09	0	\\
2.76e-09	0	\\
2.88e-09	0	\\
3e-09	0	\\
3.12e-09	0	\\
3.24e-09	0	\\
3.36e-09	0	\\
3.48e-09	0	\\
3.6e-09	0	\\
3.72e-09	0	\\
3.84e-09	0	\\
3.96e-09	0	\\
4.08e-09	0	\\
4.2e-09	0	\\
4.32e-09	0	\\
4.44e-09	0	\\
4.56e-09	0	\\
4.68e-09	0	\\
4.8e-09	0	\\
4.92e-09	0	\\
4.98e-09	0	\\
};
\addplot [color=darkgray,solid,forget plot]
  table[row sep=crcr]{
0	0	\\
1.2e-10	0	\\
2.4e-10	0	\\
3.6e-10	0	\\
4.8e-10	0	\\
6e-10	0	\\
7.2e-10	0	\\
8.4e-10	0	\\
9.6e-10	0	\\
1.08e-09	0	\\
1.2e-09	0	\\
1.32e-09	0	\\
1.44e-09	0	\\
1.56e-09	0	\\
1.68e-09	0	\\
1.8e-09	0	\\
1.92e-09	0	\\
2.04e-09	0	\\
2.16e-09	0	\\
2.28e-09	0	\\
2.4e-09	0	\\
2.52e-09	0	\\
2.64e-09	0	\\
2.76e-09	0	\\
2.88e-09	0	\\
3e-09	0	\\
3.12e-09	0	\\
3.24e-09	0	\\
3.36e-09	0	\\
3.48e-09	0	\\
3.6e-09	0	\\
3.72e-09	0	\\
3.84e-09	0	\\
3.96e-09	0	\\
4.08e-09	0	\\
4.2e-09	0	\\
4.32e-09	0	\\
4.44e-09	0	\\
4.56e-09	0	\\
4.68e-09	0	\\
4.8e-09	0	\\
4.92e-09	0	\\
4.98e-09	0	\\
};
\addplot [color=blue,solid,forget plot]
  table[row sep=crcr]{
0	0	\\
1.2e-10	0	\\
2.4e-10	0	\\
3.6e-10	0	\\
4.8e-10	0	\\
6e-10	0	\\
7.2e-10	0	\\
8.4e-10	0	\\
9.6e-10	0	\\
1.08e-09	0	\\
1.2e-09	0	\\
1.32e-09	0	\\
1.44e-09	0	\\
1.56e-09	0	\\
1.68e-09	0	\\
1.8e-09	0	\\
1.92e-09	0	\\
2.04e-09	0	\\
2.16e-09	0	\\
2.28e-09	0	\\
2.4e-09	0	\\
2.52e-09	0	\\
2.64e-09	0	\\
2.76e-09	0	\\
2.88e-09	0	\\
3e-09	0	\\
3.12e-09	0	\\
3.24e-09	0	\\
3.36e-09	0	\\
3.48e-09	0	\\
3.6e-09	0	\\
3.72e-09	0	\\
3.84e-09	0	\\
3.96e-09	0	\\
4.08e-09	0	\\
4.2e-09	0	\\
4.32e-09	0	\\
4.44e-09	0	\\
4.56e-09	0	\\
4.68e-09	0	\\
4.8e-09	0	\\
4.92e-09	0	\\
4.98e-09	0	\\
};
\addplot [color=black!50!green,solid,forget plot]
  table[row sep=crcr]{
0	0	\\
1.2e-10	0	\\
2.4e-10	0	\\
3.6e-10	0	\\
4.8e-10	0	\\
6e-10	0	\\
7.2e-10	0	\\
8.4e-10	0	\\
9.6e-10	0	\\
1.08e-09	0	\\
1.2e-09	0	\\
1.32e-09	0	\\
1.44e-09	0	\\
1.56e-09	0	\\
1.68e-09	0	\\
1.8e-09	0	\\
1.92e-09	0	\\
2.04e-09	0	\\
2.16e-09	0	\\
2.28e-09	0	\\
2.4e-09	0	\\
2.52e-09	0	\\
2.64e-09	0	\\
2.76e-09	0	\\
2.88e-09	0	\\
3e-09	0	\\
3.12e-09	0	\\
3.24e-09	0	\\
3.36e-09	0	\\
3.48e-09	0	\\
3.6e-09	0	\\
3.72e-09	0	\\
3.84e-09	0	\\
3.96e-09	0	\\
4.08e-09	0	\\
4.2e-09	0	\\
4.32e-09	0	\\
4.44e-09	0	\\
4.56e-09	0	\\
4.68e-09	0	\\
4.8e-09	0	\\
4.92e-09	0	\\
4.98e-09	0	\\
};
\addplot [color=red,solid,forget plot]
  table[row sep=crcr]{
0	0	\\
1.2e-10	0	\\
2.4e-10	0	\\
3.6e-10	0	\\
4.8e-10	0	\\
6e-10	0	\\
7.2e-10	0	\\
8.4e-10	0	\\
9.6e-10	0	\\
1.08e-09	0	\\
1.2e-09	0	\\
1.32e-09	0	\\
1.44e-09	0	\\
1.56e-09	0	\\
1.68e-09	0	\\
1.8e-09	0	\\
1.92e-09	0	\\
2.04e-09	0	\\
2.16e-09	0	\\
2.28e-09	0	\\
2.4e-09	0	\\
2.52e-09	0	\\
2.64e-09	0	\\
2.76e-09	0	\\
2.88e-09	0	\\
3e-09	0	\\
3.12e-09	0	\\
3.24e-09	0	\\
3.36e-09	0	\\
3.48e-09	0	\\
3.6e-09	0	\\
3.72e-09	0	\\
3.84e-09	0	\\
3.96e-09	0	\\
4.08e-09	0	\\
4.2e-09	0	\\
4.32e-09	0	\\
4.44e-09	0	\\
4.56e-09	0	\\
4.68e-09	0	\\
4.8e-09	0	\\
4.92e-09	0	\\
4.98e-09	0	\\
};
\addplot [color=mycolor1,solid,forget plot]
  table[row sep=crcr]{
0	0	\\
1.2e-10	0	\\
2.4e-10	0	\\
3.6e-10	0	\\
4.8e-10	0	\\
6e-10	0	\\
7.2e-10	0	\\
8.4e-10	0	\\
9.6e-10	0	\\
1.08e-09	0	\\
1.2e-09	0	\\
1.32e-09	0	\\
1.44e-09	0	\\
1.56e-09	0	\\
1.68e-09	0	\\
1.8e-09	0	\\
1.92e-09	0	\\
2.04e-09	0	\\
2.16e-09	0	\\
2.28e-09	0	\\
2.4e-09	0	\\
2.52e-09	0	\\
2.64e-09	0	\\
2.76e-09	0	\\
2.88e-09	0	\\
3e-09	0	\\
3.12e-09	0	\\
3.24e-09	0	\\
3.36e-09	0	\\
3.48e-09	0	\\
3.6e-09	0	\\
3.72e-09	0	\\
3.84e-09	0	\\
3.96e-09	0	\\
4.08e-09	0	\\
4.2e-09	0	\\
4.32e-09	0	\\
4.44e-09	0	\\
4.56e-09	0	\\
4.68e-09	0	\\
4.8e-09	0	\\
4.92e-09	0	\\
4.98e-09	0	\\
};
\addplot [color=mycolor2,solid,forget plot]
  table[row sep=crcr]{
0	0	\\
1.2e-10	0	\\
2.4e-10	0	\\
3.6e-10	0	\\
4.8e-10	0	\\
6e-10	0	\\
7.2e-10	0	\\
8.4e-10	0	\\
9.6e-10	0	\\
1.08e-09	0	\\
1.2e-09	0	\\
1.32e-09	0	\\
1.44e-09	0	\\
1.56e-09	0	\\
1.68e-09	0	\\
1.8e-09	0	\\
1.92e-09	0	\\
2.04e-09	0	\\
2.16e-09	0	\\
2.28e-09	0	\\
2.4e-09	0	\\
2.52e-09	0	\\
2.64e-09	0	\\
2.76e-09	0	\\
2.88e-09	0	\\
3e-09	0	\\
3.12e-09	0	\\
3.24e-09	0	\\
3.36e-09	0	\\
3.48e-09	0	\\
3.6e-09	0	\\
3.72e-09	0	\\
3.84e-09	0	\\
3.96e-09	0	\\
4.08e-09	0	\\
4.2e-09	0	\\
4.32e-09	0	\\
4.44e-09	0	\\
4.56e-09	0	\\
4.68e-09	0	\\
4.8e-09	0	\\
4.92e-09	0	\\
4.98e-09	0	\\
};
\addplot [color=mycolor3,solid,forget plot]
  table[row sep=crcr]{
0	0	\\
1.2e-10	0	\\
2.4e-10	0	\\
3.6e-10	0	\\
4.8e-10	0	\\
6e-10	0	\\
7.2e-10	0	\\
8.4e-10	0	\\
9.6e-10	0	\\
1.08e-09	0	\\
1.2e-09	0	\\
1.32e-09	0	\\
1.44e-09	0	\\
1.56e-09	0	\\
1.68e-09	0	\\
1.8e-09	0	\\
1.92e-09	0	\\
2.04e-09	0	\\
2.16e-09	0	\\
2.28e-09	0	\\
2.4e-09	0	\\
2.52e-09	0	\\
2.64e-09	0	\\
2.76e-09	0	\\
2.88e-09	0	\\
3e-09	0	\\
3.12e-09	0	\\
3.24e-09	0	\\
3.36e-09	0	\\
3.48e-09	0	\\
3.6e-09	0	\\
3.72e-09	0	\\
3.84e-09	0	\\
3.96e-09	0	\\
4.08e-09	0	\\
4.2e-09	0	\\
4.32e-09	0	\\
4.44e-09	0	\\
4.56e-09	0	\\
4.68e-09	0	\\
4.8e-09	0	\\
4.92e-09	0	\\
4.98e-09	0	\\
};
\addplot [color=darkgray,solid,forget plot]
  table[row sep=crcr]{
0	0	\\
1.2e-10	0	\\
2.4e-10	0	\\
3.6e-10	0	\\
4.8e-10	0	\\
6e-10	0	\\
7.2e-10	0	\\
8.4e-10	0	\\
9.6e-10	0	\\
1.08e-09	0	\\
1.2e-09	0	\\
1.32e-09	0	\\
1.44e-09	0	\\
1.56e-09	0	\\
1.68e-09	0	\\
1.8e-09	0	\\
1.92e-09	0	\\
2.04e-09	0	\\
2.16e-09	0	\\
2.28e-09	0	\\
2.4e-09	0	\\
2.52e-09	0	\\
2.64e-09	0	\\
2.76e-09	0	\\
2.88e-09	0	\\
3e-09	0	\\
3.12e-09	0	\\
3.24e-09	0	\\
3.36e-09	0	\\
3.48e-09	0	\\
3.6e-09	0	\\
3.72e-09	0	\\
3.84e-09	0	\\
3.96e-09	0	\\
4.08e-09	0	\\
4.2e-09	0	\\
4.32e-09	0	\\
4.44e-09	0	\\
4.56e-09	0	\\
4.68e-09	0	\\
4.8e-09	0	\\
4.92e-09	0	\\
4.98e-09	0	\\
};
\addplot [color=blue,solid,forget plot]
  table[row sep=crcr]{
0	0	\\
1.2e-10	0	\\
2.4e-10	0	\\
3.6e-10	0	\\
4.8e-10	0	\\
6e-10	0	\\
7.2e-10	0	\\
8.4e-10	0	\\
9.6e-10	0	\\
1.08e-09	0	\\
1.2e-09	0	\\
1.32e-09	0	\\
1.44e-09	0	\\
1.56e-09	0	\\
1.68e-09	0	\\
1.8e-09	0	\\
1.92e-09	0	\\
2.04e-09	0	\\
2.16e-09	0	\\
2.28e-09	0	\\
2.4e-09	0	\\
2.52e-09	0	\\
2.64e-09	0	\\
2.76e-09	0	\\
2.88e-09	0	\\
3e-09	0	\\
3.12e-09	0	\\
3.24e-09	0	\\
3.36e-09	0	\\
3.48e-09	0	\\
3.6e-09	0	\\
3.72e-09	0	\\
3.84e-09	0	\\
3.96e-09	0	\\
4.08e-09	0	\\
4.2e-09	0	\\
4.32e-09	0	\\
4.44e-09	0	\\
4.56e-09	0	\\
4.68e-09	0	\\
4.8e-09	0	\\
4.92e-09	0	\\
4.98e-09	0	\\
};
\addplot [color=black!50!green,solid,forget plot]
  table[row sep=crcr]{
0	0	\\
1.2e-10	0	\\
2.4e-10	0	\\
3.6e-10	0	\\
4.8e-10	0	\\
6e-10	0	\\
7.2e-10	0	\\
8.4e-10	0	\\
9.6e-10	0	\\
1.08e-09	0	\\
1.2e-09	0	\\
1.32e-09	0	\\
1.44e-09	0	\\
1.56e-09	0	\\
1.68e-09	0	\\
1.8e-09	0	\\
1.92e-09	0	\\
2.04e-09	0	\\
2.16e-09	0	\\
2.28e-09	0	\\
2.4e-09	0	\\
2.52e-09	0	\\
2.64e-09	0	\\
2.76e-09	0	\\
2.88e-09	0	\\
3e-09	0	\\
3.12e-09	0	\\
3.24e-09	0	\\
3.36e-09	0	\\
3.48e-09	0	\\
3.6e-09	0	\\
3.72e-09	0	\\
3.84e-09	0	\\
3.96e-09	0	\\
4.08e-09	0	\\
4.2e-09	0	\\
4.32e-09	0	\\
4.44e-09	0	\\
4.56e-09	0	\\
4.68e-09	0	\\
4.8e-09	0	\\
4.92e-09	0	\\
4.98e-09	0	\\
};
\addplot [color=red,solid,forget plot]
  table[row sep=crcr]{
0	0	\\
1.2e-10	0	\\
2.4e-10	0	\\
3.6e-10	0	\\
4.8e-10	0	\\
6e-10	0	\\
7.2e-10	0	\\
8.4e-10	0	\\
9.6e-10	0	\\
1.08e-09	0	\\
1.2e-09	0	\\
1.32e-09	0	\\
1.44e-09	0	\\
1.56e-09	0	\\
1.68e-09	0	\\
1.8e-09	0	\\
1.92e-09	0	\\
2.04e-09	0	\\
2.16e-09	0	\\
2.28e-09	0	\\
2.4e-09	0	\\
2.52e-09	0	\\
2.64e-09	0	\\
2.76e-09	0	\\
2.88e-09	0	\\
3e-09	0	\\
3.12e-09	0	\\
3.24e-09	0	\\
3.36e-09	0	\\
3.48e-09	0	\\
3.6e-09	0	\\
3.72e-09	0	\\
3.84e-09	0	\\
3.96e-09	0	\\
4.08e-09	0	\\
4.2e-09	0	\\
4.32e-09	0	\\
4.44e-09	0	\\
4.56e-09	0	\\
4.68e-09	0	\\
4.8e-09	0	\\
4.92e-09	0	\\
4.98e-09	0	\\
};
\addplot [color=mycolor1,solid,forget plot]
  table[row sep=crcr]{
0	0	\\
1.2e-10	0	\\
2.4e-10	0	\\
3.6e-10	0	\\
4.8e-10	0	\\
6e-10	0	\\
7.2e-10	0	\\
8.4e-10	0	\\
9.6e-10	0	\\
1.08e-09	0	\\
1.2e-09	0	\\
1.32e-09	0	\\
1.44e-09	0	\\
1.56e-09	0	\\
1.68e-09	0	\\
1.8e-09	0	\\
1.92e-09	0	\\
2.04e-09	0	\\
2.16e-09	0	\\
2.28e-09	0	\\
2.4e-09	0	\\
2.52e-09	0	\\
2.64e-09	0	\\
2.76e-09	0	\\
2.88e-09	0	\\
3e-09	0	\\
3.12e-09	0	\\
3.24e-09	0	\\
3.36e-09	0	\\
3.48e-09	0	\\
3.6e-09	0	\\
3.72e-09	0	\\
3.84e-09	0	\\
3.96e-09	0	\\
4.08e-09	0	\\
4.2e-09	0	\\
4.32e-09	0	\\
4.44e-09	0	\\
4.56e-09	0	\\
4.68e-09	0	\\
4.8e-09	0	\\
4.92e-09	0	\\
4.98e-09	0	\\
};
\addplot [color=mycolor2,solid,forget plot]
  table[row sep=crcr]{
0	0	\\
1.2e-10	0	\\
2.4e-10	0	\\
3.6e-10	0	\\
4.8e-10	0	\\
6e-10	0	\\
7.2e-10	0	\\
8.4e-10	0	\\
9.6e-10	0	\\
1.08e-09	0	\\
1.2e-09	0	\\
1.32e-09	0	\\
1.44e-09	0	\\
1.56e-09	0	\\
1.68e-09	0	\\
1.8e-09	0	\\
1.92e-09	0	\\
2.04e-09	0	\\
2.16e-09	0	\\
2.28e-09	0	\\
2.4e-09	0	\\
2.52e-09	0	\\
2.64e-09	0	\\
2.76e-09	0	\\
2.88e-09	0	\\
3e-09	0	\\
3.12e-09	0	\\
3.24e-09	0	\\
3.36e-09	0	\\
3.48e-09	0	\\
3.6e-09	0	\\
3.72e-09	0	\\
3.84e-09	0	\\
3.96e-09	0	\\
4.08e-09	0	\\
4.2e-09	0	\\
4.32e-09	0	\\
4.44e-09	0	\\
4.56e-09	0	\\
4.68e-09	0	\\
4.8e-09	0	\\
4.92e-09	0	\\
4.98e-09	0	\\
};
\addplot [color=mycolor3,solid,forget plot]
  table[row sep=crcr]{
0	0	\\
1.2e-10	0	\\
2.4e-10	0	\\
3.6e-10	0	\\
4.8e-10	0	\\
6e-10	0	\\
7.2e-10	0	\\
8.4e-10	0	\\
9.6e-10	0	\\
1.08e-09	0	\\
1.2e-09	0	\\
1.32e-09	0	\\
1.44e-09	0	\\
1.56e-09	0	\\
1.68e-09	0	\\
1.8e-09	0	\\
1.92e-09	0	\\
2.04e-09	0	\\
2.16e-09	0	\\
2.28e-09	0	\\
2.4e-09	0	\\
2.52e-09	0	\\
2.64e-09	0	\\
2.76e-09	0	\\
2.88e-09	0	\\
3e-09	0	\\
3.12e-09	0	\\
3.24e-09	0	\\
3.36e-09	0	\\
3.48e-09	0	\\
3.6e-09	0	\\
3.72e-09	0	\\
3.84e-09	0	\\
3.96e-09	0	\\
4.08e-09	0	\\
4.2e-09	0	\\
4.32e-09	0	\\
4.44e-09	0	\\
4.56e-09	0	\\
4.68e-09	0	\\
4.8e-09	0	\\
4.92e-09	0	\\
4.98e-09	0	\\
};
\addplot [color=darkgray,solid,forget plot]
  table[row sep=crcr]{
0	0	\\
1.2e-10	0	\\
2.4e-10	0	\\
3.6e-10	0	\\
4.8e-10	0	\\
6e-10	0	\\
7.2e-10	0	\\
8.4e-10	0	\\
9.6e-10	0	\\
1.08e-09	0	\\
1.2e-09	0	\\
1.32e-09	0	\\
1.44e-09	0	\\
1.56e-09	0	\\
1.68e-09	0	\\
1.8e-09	0	\\
1.92e-09	0	\\
2.04e-09	0	\\
2.16e-09	0	\\
2.28e-09	0	\\
2.4e-09	0	\\
2.52e-09	0	\\
2.64e-09	0	\\
2.76e-09	0	\\
2.88e-09	0	\\
3e-09	0	\\
3.12e-09	0	\\
3.24e-09	0	\\
3.36e-09	0	\\
3.48e-09	0	\\
3.6e-09	0	\\
3.72e-09	0	\\
3.84e-09	0	\\
3.96e-09	0	\\
4.08e-09	0	\\
4.2e-09	0	\\
4.32e-09	0	\\
4.44e-09	0	\\
4.56e-09	0	\\
4.68e-09	0	\\
4.8e-09	0	\\
4.92e-09	0	\\
4.98e-09	0	\\
};
\addplot [color=blue,solid,forget plot]
  table[row sep=crcr]{
0	0	\\
1.2e-10	0	\\
2.4e-10	0	\\
3.6e-10	0	\\
4.8e-10	0	\\
6e-10	0	\\
7.2e-10	0	\\
8.4e-10	0	\\
9.6e-10	0	\\
1.08e-09	0	\\
1.2e-09	0	\\
1.32e-09	0	\\
1.44e-09	0	\\
1.56e-09	0	\\
1.68e-09	0	\\
1.8e-09	0	\\
1.92e-09	0	\\
2.04e-09	0	\\
2.16e-09	0	\\
2.28e-09	0	\\
2.4e-09	0	\\
2.52e-09	0	\\
2.64e-09	0	\\
2.76e-09	0	\\
2.88e-09	0	\\
3e-09	0	\\
3.12e-09	0	\\
3.24e-09	0	\\
3.36e-09	0	\\
3.48e-09	0	\\
3.6e-09	0	\\
3.72e-09	0	\\
3.84e-09	0	\\
3.96e-09	0	\\
4.08e-09	0	\\
4.2e-09	0	\\
4.32e-09	0	\\
4.44e-09	0	\\
4.56e-09	0	\\
4.68e-09	0	\\
4.8e-09	0	\\
4.92e-09	0	\\
4.98e-09	0	\\
};
\addplot [color=black!50!green,solid,forget plot]
  table[row sep=crcr]{
0	0	\\
1.2e-10	0	\\
2.4e-10	0	\\
3.6e-10	0	\\
4.8e-10	0	\\
6e-10	0	\\
7.2e-10	0	\\
8.4e-10	0	\\
9.6e-10	0	\\
1.08e-09	0	\\
1.2e-09	0	\\
1.32e-09	0	\\
1.44e-09	0	\\
1.56e-09	0	\\
1.68e-09	0	\\
1.8e-09	0	\\
1.92e-09	0	\\
2.04e-09	0	\\
2.16e-09	0	\\
2.28e-09	0	\\
2.4e-09	0	\\
2.52e-09	0	\\
2.64e-09	0	\\
2.76e-09	0	\\
2.88e-09	0	\\
3e-09	0	\\
3.12e-09	0	\\
3.24e-09	0	\\
3.36e-09	0	\\
3.48e-09	0	\\
3.6e-09	0	\\
3.72e-09	0	\\
3.84e-09	0	\\
3.96e-09	0	\\
4.08e-09	0	\\
4.2e-09	0	\\
4.32e-09	0	\\
4.44e-09	0	\\
4.56e-09	0	\\
4.68e-09	0	\\
4.8e-09	0	\\
4.92e-09	0	\\
4.98e-09	0	\\
};
\addplot [color=red,solid,forget plot]
  table[row sep=crcr]{
0	0	\\
1.2e-10	0	\\
2.4e-10	0	\\
3.6e-10	0	\\
4.8e-10	0	\\
6e-10	0	\\
7.2e-10	0	\\
8.4e-10	0	\\
9.6e-10	0	\\
1.08e-09	0	\\
1.2e-09	0	\\
1.32e-09	0	\\
1.44e-09	0	\\
1.56e-09	0	\\
1.68e-09	0	\\
1.8e-09	0	\\
1.92e-09	0	\\
2.04e-09	0	\\
2.16e-09	0	\\
2.28e-09	0	\\
2.4e-09	0	\\
2.52e-09	0	\\
2.64e-09	0	\\
2.76e-09	0	\\
2.88e-09	0	\\
3e-09	0	\\
3.12e-09	0	\\
3.24e-09	0	\\
3.36e-09	0	\\
3.48e-09	0	\\
3.6e-09	0	\\
3.72e-09	0	\\
3.84e-09	0	\\
3.96e-09	0	\\
4.08e-09	0	\\
4.2e-09	0	\\
4.32e-09	0	\\
4.44e-09	0	\\
4.56e-09	0	\\
4.68e-09	0	\\
4.8e-09	0	\\
4.92e-09	0	\\
4.98e-09	0	\\
};
\addplot [color=mycolor1,solid,forget plot]
  table[row sep=crcr]{
0	0	\\
1.2e-10	0	\\
2.4e-10	0	\\
3.6e-10	0	\\
4.8e-10	0	\\
6e-10	0	\\
7.2e-10	0	\\
8.4e-10	0	\\
9.6e-10	0	\\
1.08e-09	0	\\
1.2e-09	0	\\
1.32e-09	0	\\
1.44e-09	0	\\
1.56e-09	0	\\
1.68e-09	0	\\
1.8e-09	0	\\
1.92e-09	0	\\
2.04e-09	0	\\
2.16e-09	0	\\
2.28e-09	0	\\
2.4e-09	0	\\
2.52e-09	0	\\
2.64e-09	0	\\
2.76e-09	0	\\
2.88e-09	0	\\
3e-09	0	\\
3.12e-09	0	\\
3.24e-09	0	\\
3.36e-09	0	\\
3.48e-09	0	\\
3.6e-09	0	\\
3.72e-09	0	\\
3.84e-09	0	\\
3.96e-09	0	\\
4.08e-09	0	\\
4.2e-09	0	\\
4.32e-09	0	\\
4.44e-09	0	\\
4.56e-09	0	\\
4.68e-09	0	\\
4.8e-09	0	\\
4.92e-09	0	\\
4.98e-09	0	\\
};
\addplot [color=mycolor2,solid,forget plot]
  table[row sep=crcr]{
0	0	\\
1.2e-10	0	\\
2.4e-10	0	\\
3.6e-10	0	\\
4.8e-10	0	\\
6e-10	0	\\
7.2e-10	0	\\
8.4e-10	0	\\
9.6e-10	0	\\
1.08e-09	0	\\
1.2e-09	0	\\
1.32e-09	0	\\
1.44e-09	0	\\
1.56e-09	0	\\
1.68e-09	0	\\
1.8e-09	0	\\
1.92e-09	0	\\
2.04e-09	0	\\
2.16e-09	0	\\
2.28e-09	0	\\
2.4e-09	0	\\
2.52e-09	0	\\
2.64e-09	0	\\
2.76e-09	0	\\
2.88e-09	0	\\
3e-09	0	\\
3.12e-09	0	\\
3.24e-09	0	\\
3.36e-09	0	\\
3.48e-09	0	\\
3.6e-09	0	\\
3.72e-09	0	\\
3.84e-09	0	\\
3.96e-09	0	\\
4.08e-09	0	\\
4.2e-09	0	\\
4.32e-09	0	\\
4.44e-09	0	\\
4.56e-09	0	\\
4.68e-09	0	\\
4.8e-09	0	\\
4.92e-09	0	\\
4.98e-09	0	\\
};
\addplot [color=mycolor3,solid,forget plot]
  table[row sep=crcr]{
0	0	\\
1.2e-10	0	\\
2.4e-10	0	\\
3.6e-10	0	\\
4.8e-10	0	\\
6e-10	0	\\
7.2e-10	0	\\
8.4e-10	0	\\
9.6e-10	0	\\
1.08e-09	0	\\
1.2e-09	0	\\
1.32e-09	0	\\
1.44e-09	0	\\
1.56e-09	0	\\
1.68e-09	0	\\
1.8e-09	0	\\
1.92e-09	0	\\
2.04e-09	0	\\
2.16e-09	0	\\
2.28e-09	0	\\
2.4e-09	0	\\
2.52e-09	0	\\
2.64e-09	0	\\
2.76e-09	0	\\
2.88e-09	0	\\
3e-09	0	\\
3.12e-09	0	\\
3.24e-09	0	\\
3.36e-09	0	\\
3.48e-09	0	\\
3.6e-09	0	\\
3.72e-09	0	\\
3.84e-09	0	\\
3.96e-09	0	\\
4.08e-09	0	\\
4.2e-09	0	\\
4.32e-09	0	\\
4.44e-09	0	\\
4.56e-09	0	\\
4.68e-09	0	\\
4.8e-09	0	\\
4.92e-09	0	\\
4.98e-09	0	\\
};
\addplot [color=darkgray,solid,forget plot]
  table[row sep=crcr]{
0	0	\\
1.2e-10	0	\\
2.4e-10	0	\\
3.6e-10	0	\\
4.8e-10	0	\\
6e-10	0	\\
7.2e-10	0	\\
8.4e-10	0	\\
9.6e-10	0	\\
1.08e-09	0	\\
1.2e-09	0	\\
1.32e-09	0	\\
1.44e-09	0	\\
1.56e-09	0	\\
1.68e-09	0	\\
1.8e-09	0	\\
1.92e-09	0	\\
2.04e-09	0	\\
2.16e-09	0	\\
2.28e-09	0	\\
2.4e-09	0	\\
2.52e-09	0	\\
2.64e-09	0	\\
2.76e-09	0	\\
2.88e-09	0	\\
3e-09	0	\\
3.12e-09	0	\\
3.24e-09	0	\\
3.36e-09	0	\\
3.48e-09	0	\\
3.6e-09	0	\\
3.72e-09	0	\\
3.84e-09	0	\\
3.96e-09	0	\\
4.08e-09	0	\\
4.2e-09	0	\\
4.32e-09	0	\\
4.44e-09	0	\\
4.56e-09	0	\\
4.68e-09	0	\\
4.8e-09	0	\\
4.92e-09	0	\\
4.98e-09	0	\\
};
\addplot [color=blue,solid,forget plot]
  table[row sep=crcr]{
0	0	\\
1.2e-10	0	\\
2.4e-10	0	\\
3.6e-10	0	\\
4.8e-10	0	\\
6e-10	0	\\
7.2e-10	0	\\
8.4e-10	0	\\
9.6e-10	0	\\
1.08e-09	0	\\
1.2e-09	0	\\
1.32e-09	0	\\
1.44e-09	0	\\
1.56e-09	0	\\
1.68e-09	0	\\
1.8e-09	0	\\
1.92e-09	0	\\
2.04e-09	0	\\
2.16e-09	0	\\
2.28e-09	0	\\
2.4e-09	0	\\
2.52e-09	0	\\
2.64e-09	0	\\
2.76e-09	0	\\
2.88e-09	0	\\
3e-09	0	\\
3.12e-09	0	\\
3.24e-09	0	\\
3.36e-09	0	\\
3.48e-09	0	\\
3.6e-09	0	\\
3.72e-09	0	\\
3.84e-09	0	\\
3.96e-09	0	\\
4.08e-09	0	\\
4.2e-09	0	\\
4.32e-09	0	\\
4.44e-09	0	\\
4.56e-09	0	\\
4.68e-09	0	\\
4.8e-09	0	\\
4.92e-09	0	\\
4.98e-09	0	\\
};
\addplot [color=black!50!green,solid,forget plot]
  table[row sep=crcr]{
0	0	\\
1.2e-10	0	\\
2.4e-10	0	\\
3.6e-10	0	\\
4.8e-10	0	\\
6e-10	0	\\
7.2e-10	0	\\
8.4e-10	0	\\
9.6e-10	0	\\
1.08e-09	0	\\
1.2e-09	0	\\
1.32e-09	0	\\
1.44e-09	0	\\
1.56e-09	0	\\
1.68e-09	0	\\
1.8e-09	0	\\
1.92e-09	0	\\
2.04e-09	0	\\
2.16e-09	0	\\
2.28e-09	0	\\
2.4e-09	0	\\
2.52e-09	0	\\
2.64e-09	0	\\
2.76e-09	0	\\
2.88e-09	0	\\
3e-09	0	\\
3.12e-09	0	\\
3.24e-09	0	\\
3.36e-09	0	\\
3.48e-09	0	\\
3.6e-09	0	\\
3.72e-09	0	\\
3.84e-09	0	\\
3.96e-09	0	\\
4.08e-09	0	\\
4.2e-09	0	\\
4.32e-09	0	\\
4.44e-09	0	\\
4.56e-09	0	\\
4.68e-09	0	\\
4.8e-09	0	\\
4.92e-09	0	\\
4.98e-09	0	\\
};
\addplot [color=red,solid,forget plot]
  table[row sep=crcr]{
0	0	\\
1.2e-10	0	\\
2.4e-10	0	\\
3.6e-10	0	\\
4.8e-10	0	\\
6e-10	0	\\
7.2e-10	0	\\
8.4e-10	0	\\
9.6e-10	0	\\
1.08e-09	0	\\
1.2e-09	0	\\
1.32e-09	0	\\
1.44e-09	0	\\
1.56e-09	0	\\
1.68e-09	0	\\
1.8e-09	0	\\
1.92e-09	0	\\
2.04e-09	0	\\
2.16e-09	0	\\
2.28e-09	0	\\
2.4e-09	0	\\
2.52e-09	0	\\
2.64e-09	0	\\
2.76e-09	0	\\
2.88e-09	0	\\
3e-09	0	\\
3.12e-09	0	\\
3.24e-09	0	\\
3.36e-09	0	\\
3.48e-09	0	\\
3.6e-09	0	\\
3.72e-09	0	\\
3.84e-09	0	\\
3.96e-09	0	\\
4.08e-09	0	\\
4.2e-09	0	\\
4.32e-09	0	\\
4.44e-09	0	\\
4.56e-09	0	\\
4.68e-09	0	\\
4.8e-09	0	\\
4.92e-09	0	\\
4.98e-09	0	\\
};
\addplot [color=mycolor1,solid,forget plot]
  table[row sep=crcr]{
0	0	\\
1.2e-10	0	\\
2.4e-10	0	\\
3.6e-10	0	\\
4.8e-10	0	\\
6e-10	0	\\
7.2e-10	0	\\
8.4e-10	0	\\
9.6e-10	0	\\
1.08e-09	0	\\
1.2e-09	0	\\
1.32e-09	0	\\
1.44e-09	0	\\
1.56e-09	0	\\
1.68e-09	0	\\
1.8e-09	0	\\
1.92e-09	0	\\
2.04e-09	0	\\
2.16e-09	0	\\
2.28e-09	0	\\
2.4e-09	0	\\
2.52e-09	0	\\
2.64e-09	0	\\
2.76e-09	0	\\
2.88e-09	0	\\
3e-09	0	\\
3.12e-09	0	\\
3.24e-09	0	\\
3.36e-09	0	\\
3.48e-09	0	\\
3.6e-09	0	\\
3.72e-09	0	\\
3.84e-09	0	\\
3.96e-09	0	\\
4.08e-09	0	\\
4.2e-09	0	\\
4.32e-09	0	\\
4.44e-09	0	\\
4.56e-09	0	\\
4.68e-09	0	\\
4.8e-09	0	\\
4.92e-09	0	\\
4.98e-09	0	\\
};
\addplot [color=mycolor2,solid,forget plot]
  table[row sep=crcr]{
0	0	\\
1.2e-10	0	\\
2.4e-10	0	\\
3.6e-10	0	\\
4.8e-10	0	\\
6e-10	0	\\
7.2e-10	0	\\
8.4e-10	0	\\
9.6e-10	0	\\
1.08e-09	0	\\
1.2e-09	0	\\
1.32e-09	0	\\
1.44e-09	0	\\
1.56e-09	0	\\
1.68e-09	0	\\
1.8e-09	0	\\
1.92e-09	0	\\
2.04e-09	0	\\
2.16e-09	0	\\
2.28e-09	0	\\
2.4e-09	0	\\
2.52e-09	0	\\
2.64e-09	0	\\
2.76e-09	0	\\
2.88e-09	0	\\
3e-09	0	\\
3.12e-09	0	\\
3.24e-09	0	\\
3.36e-09	0	\\
3.48e-09	0	\\
3.6e-09	0	\\
3.72e-09	0	\\
3.84e-09	0	\\
3.96e-09	0	\\
4.08e-09	0	\\
4.2e-09	0	\\
4.32e-09	0	\\
4.44e-09	0	\\
4.56e-09	0	\\
4.68e-09	0	\\
4.8e-09	0	\\
4.92e-09	0	\\
4.98e-09	0	\\
};
\addplot [color=mycolor3,solid,forget plot]
  table[row sep=crcr]{
0	0	\\
1.2e-10	0	\\
2.4e-10	0	\\
3.6e-10	0	\\
4.8e-10	0	\\
6e-10	0	\\
7.2e-10	0	\\
8.4e-10	0	\\
9.6e-10	0	\\
1.08e-09	0	\\
1.2e-09	0	\\
1.32e-09	0	\\
1.44e-09	0	\\
1.56e-09	0	\\
1.68e-09	0	\\
1.8e-09	0	\\
1.92e-09	0	\\
2.04e-09	0	\\
2.16e-09	0	\\
2.28e-09	0	\\
2.4e-09	0	\\
2.52e-09	0	\\
2.64e-09	0	\\
2.76e-09	0	\\
2.88e-09	0	\\
3e-09	0	\\
3.12e-09	0	\\
3.24e-09	0	\\
3.36e-09	0	\\
3.48e-09	0	\\
3.6e-09	0	\\
3.72e-09	0	\\
3.84e-09	0	\\
3.96e-09	0	\\
4.08e-09	0	\\
4.2e-09	0	\\
4.32e-09	0	\\
4.44e-09	0	\\
4.56e-09	0	\\
4.68e-09	0	\\
4.8e-09	0	\\
4.92e-09	0	\\
4.98e-09	0	\\
};
\addplot [color=darkgray,solid,forget plot]
  table[row sep=crcr]{
0	0	\\
1.2e-10	0	\\
2.4e-10	0	\\
3.6e-10	0	\\
4.8e-10	0	\\
6e-10	0	\\
7.2e-10	0	\\
8.4e-10	0	\\
9.6e-10	0	\\
1.08e-09	0	\\
1.2e-09	0	\\
1.32e-09	0	\\
1.44e-09	0	\\
1.56e-09	0	\\
1.68e-09	0	\\
1.8e-09	0	\\
1.92e-09	0	\\
2.04e-09	0	\\
2.16e-09	0	\\
2.28e-09	0	\\
2.4e-09	0	\\
2.52e-09	0	\\
2.64e-09	0	\\
2.76e-09	0	\\
2.88e-09	0	\\
3e-09	0	\\
3.12e-09	0	\\
3.24e-09	0	\\
3.36e-09	0	\\
3.48e-09	0	\\
3.6e-09	0	\\
3.72e-09	0	\\
3.84e-09	0	\\
3.96e-09	0	\\
4.08e-09	0	\\
4.2e-09	0	\\
4.32e-09	0	\\
4.44e-09	0	\\
4.56e-09	0	\\
4.68e-09	0	\\
4.8e-09	0	\\
4.92e-09	0	\\
4.98e-09	0	\\
};
\addplot [color=blue,solid,forget plot]
  table[row sep=crcr]{
0	0	\\
1.2e-10	0	\\
2.4e-10	0	\\
3.6e-10	0	\\
4.8e-10	0	\\
6e-10	0	\\
7.2e-10	0	\\
8.4e-10	0	\\
9.6e-10	0	\\
1.08e-09	0	\\
1.2e-09	0	\\
1.32e-09	0	\\
1.44e-09	0	\\
1.56e-09	0	\\
1.68e-09	0	\\
1.8e-09	0	\\
1.92e-09	0	\\
2.04e-09	0	\\
2.16e-09	0	\\
2.28e-09	0	\\
2.4e-09	0	\\
2.52e-09	0	\\
2.64e-09	0	\\
2.76e-09	0	\\
2.88e-09	0	\\
3e-09	0	\\
3.12e-09	0	\\
3.24e-09	0	\\
3.36e-09	0	\\
3.48e-09	0	\\
3.6e-09	0	\\
3.72e-09	0	\\
3.84e-09	0	\\
3.96e-09	0	\\
4.08e-09	0	\\
4.2e-09	0	\\
4.32e-09	0	\\
4.44e-09	0	\\
4.56e-09	0	\\
4.68e-09	0	\\
4.8e-09	0	\\
4.92e-09	0	\\
4.98e-09	0	\\
};
\addplot [color=black!50!green,solid,forget plot]
  table[row sep=crcr]{
0	0	\\
1.2e-10	0	\\
2.4e-10	0	\\
3.6e-10	0	\\
4.8e-10	0	\\
6e-10	0	\\
7.2e-10	0	\\
8.4e-10	0	\\
9.6e-10	0	\\
1.08e-09	0	\\
1.2e-09	0	\\
1.32e-09	0	\\
1.44e-09	0	\\
1.56e-09	0	\\
1.68e-09	0	\\
1.8e-09	0	\\
1.92e-09	0	\\
2.04e-09	0	\\
2.16e-09	0	\\
2.28e-09	0	\\
2.4e-09	0	\\
2.52e-09	0	\\
2.64e-09	0	\\
2.76e-09	0	\\
2.88e-09	0	\\
3e-09	0	\\
3.12e-09	0	\\
3.24e-09	0	\\
3.36e-09	0	\\
3.48e-09	0	\\
3.6e-09	0	\\
3.72e-09	0	\\
3.84e-09	0	\\
3.96e-09	0	\\
4.08e-09	0	\\
4.2e-09	0	\\
4.32e-09	0	\\
4.44e-09	0	\\
4.56e-09	0	\\
4.68e-09	0	\\
4.8e-09	0	\\
4.92e-09	0	\\
4.98e-09	0	\\
};
\addplot [color=red,solid,forget plot]
  table[row sep=crcr]{
0	0	\\
1.2e-10	0	\\
2.4e-10	0	\\
3.6e-10	0	\\
4.8e-10	0	\\
6e-10	0	\\
7.2e-10	0	\\
8.4e-10	0	\\
9.6e-10	0	\\
1.08e-09	0	\\
1.2e-09	0	\\
1.32e-09	0	\\
1.44e-09	0	\\
1.56e-09	0	\\
1.68e-09	0	\\
1.8e-09	0	\\
1.92e-09	0	\\
2.04e-09	0	\\
2.16e-09	0	\\
2.28e-09	0	\\
2.4e-09	0	\\
2.52e-09	0	\\
2.64e-09	0	\\
2.76e-09	0	\\
2.88e-09	0	\\
3e-09	0	\\
3.12e-09	0	\\
3.24e-09	0	\\
3.36e-09	0	\\
3.48e-09	0	\\
3.6e-09	0	\\
3.72e-09	0	\\
3.84e-09	0	\\
3.96e-09	0	\\
4.08e-09	0	\\
4.2e-09	0	\\
4.32e-09	0	\\
4.44e-09	0	\\
4.56e-09	0	\\
4.68e-09	0	\\
4.8e-09	0	\\
4.92e-09	0	\\
4.98e-09	0	\\
};
\addplot [color=mycolor1,solid,forget plot]
  table[row sep=crcr]{
0	0	\\
1.2e-10	0	\\
2.4e-10	0	\\
3.6e-10	0	\\
4.8e-10	0	\\
6e-10	0	\\
7.2e-10	0	\\
8.4e-10	0	\\
9.6e-10	0	\\
1.08e-09	0	\\
1.2e-09	0	\\
1.32e-09	0	\\
1.44e-09	0	\\
1.56e-09	0	\\
1.68e-09	0	\\
1.8e-09	0	\\
1.92e-09	0	\\
2.04e-09	0	\\
2.16e-09	0	\\
2.28e-09	0	\\
2.4e-09	0	\\
2.52e-09	0	\\
2.64e-09	0	\\
2.76e-09	0	\\
2.88e-09	0	\\
3e-09	0	\\
3.12e-09	0	\\
3.24e-09	0	\\
3.36e-09	0	\\
3.48e-09	0	\\
3.6e-09	0	\\
3.72e-09	0	\\
3.84e-09	0	\\
3.96e-09	0	\\
4.08e-09	0	\\
4.2e-09	0	\\
4.32e-09	0	\\
4.44e-09	0	\\
4.56e-09	0	\\
4.68e-09	0	\\
4.8e-09	0	\\
4.92e-09	0	\\
4.98e-09	0	\\
};
\addplot [color=mycolor2,solid,forget plot]
  table[row sep=crcr]{
0	0	\\
1.2e-10	0	\\
2.4e-10	0	\\
3.6e-10	0	\\
4.8e-10	0	\\
6e-10	0	\\
7.2e-10	0	\\
8.4e-10	0	\\
9.6e-10	0	\\
1.08e-09	0	\\
1.2e-09	0	\\
1.32e-09	0	\\
1.44e-09	0	\\
1.56e-09	0	\\
1.68e-09	0	\\
1.8e-09	0	\\
1.92e-09	0	\\
2.04e-09	0	\\
2.16e-09	0	\\
2.28e-09	0	\\
2.4e-09	0	\\
2.52e-09	0	\\
2.64e-09	0	\\
2.76e-09	0	\\
2.88e-09	0	\\
3e-09	0	\\
3.12e-09	0	\\
3.24e-09	0	\\
3.36e-09	0	\\
3.48e-09	0	\\
3.6e-09	0	\\
3.72e-09	0	\\
3.84e-09	0	\\
3.96e-09	0	\\
4.08e-09	0	\\
4.2e-09	0	\\
4.32e-09	0	\\
4.44e-09	0	\\
4.56e-09	0	\\
4.68e-09	0	\\
4.8e-09	0	\\
4.92e-09	0	\\
4.98e-09	0	\\
};
\addplot [color=mycolor3,solid,forget plot]
  table[row sep=crcr]{
0	0	\\
1.2e-10	0	\\
2.4e-10	0	\\
3.6e-10	0	\\
4.8e-10	0	\\
6e-10	0	\\
7.2e-10	0	\\
8.4e-10	0	\\
9.6e-10	0	\\
1.08e-09	0	\\
1.2e-09	0	\\
1.32e-09	0	\\
1.44e-09	0	\\
1.56e-09	0	\\
1.68e-09	0	\\
1.8e-09	0	\\
1.92e-09	0	\\
2.04e-09	0	\\
2.16e-09	0	\\
2.28e-09	0	\\
2.4e-09	0	\\
2.52e-09	0	\\
2.64e-09	0	\\
2.76e-09	0	\\
2.88e-09	0	\\
3e-09	0	\\
3.12e-09	0	\\
3.24e-09	0	\\
3.36e-09	0	\\
3.48e-09	0	\\
3.6e-09	0	\\
3.72e-09	0	\\
3.84e-09	0	\\
3.96e-09	0	\\
4.08e-09	0	\\
4.2e-09	0	\\
4.32e-09	0	\\
4.44e-09	0	\\
4.56e-09	0	\\
4.68e-09	0	\\
4.8e-09	0	\\
4.92e-09	0	\\
4.98e-09	0	\\
};
\addplot [color=darkgray,solid,forget plot]
  table[row sep=crcr]{
0	0	\\
1.2e-10	0	\\
2.4e-10	0	\\
3.6e-10	0	\\
4.8e-10	0	\\
6e-10	0	\\
7.2e-10	0	\\
8.4e-10	0	\\
9.6e-10	0	\\
1.08e-09	0	\\
1.2e-09	0	\\
1.32e-09	0	\\
1.44e-09	0	\\
1.56e-09	0	\\
1.68e-09	0	\\
1.8e-09	0	\\
1.92e-09	0	\\
2.04e-09	0	\\
2.16e-09	0	\\
2.28e-09	0	\\
2.4e-09	0	\\
2.52e-09	0	\\
2.64e-09	0	\\
2.76e-09	0	\\
2.88e-09	0	\\
3e-09	0	\\
3.12e-09	0	\\
3.24e-09	0	\\
3.36e-09	0	\\
3.48e-09	0	\\
3.6e-09	0	\\
3.72e-09	0	\\
3.84e-09	0	\\
3.96e-09	0	\\
4.08e-09	0	\\
4.2e-09	0	\\
4.32e-09	0	\\
4.44e-09	0	\\
4.56e-09	0	\\
4.68e-09	0	\\
4.8e-09	0	\\
4.92e-09	0	\\
4.98e-09	0	\\
};
\addplot [color=blue,solid,forget plot]
  table[row sep=crcr]{
0	0	\\
1.2e-10	0	\\
2.4e-10	0	\\
3.6e-10	0	\\
4.8e-10	0	\\
6e-10	0	\\
7.2e-10	0	\\
8.4e-10	0	\\
9.6e-10	0	\\
1.08e-09	0	\\
1.2e-09	0	\\
1.32e-09	0	\\
1.44e-09	0	\\
1.56e-09	0	\\
1.68e-09	0	\\
1.8e-09	0	\\
1.92e-09	0	\\
2.04e-09	0	\\
2.16e-09	0	\\
2.28e-09	0	\\
2.4e-09	0	\\
2.52e-09	0	\\
2.64e-09	0	\\
2.76e-09	0	\\
2.88e-09	0	\\
3e-09	0	\\
3.12e-09	0	\\
3.24e-09	0	\\
3.36e-09	0	\\
3.48e-09	0	\\
3.6e-09	0	\\
3.72e-09	0	\\
3.84e-09	0	\\
3.96e-09	0	\\
4.08e-09	0	\\
4.2e-09	0	\\
4.32e-09	0	\\
4.44e-09	0	\\
4.56e-09	0	\\
4.68e-09	0	\\
4.8e-09	0	\\
4.92e-09	0	\\
4.98e-09	0	\\
};
\addplot [color=black!50!green,solid,forget plot]
  table[row sep=crcr]{
0	0	\\
1.2e-10	0	\\
2.4e-10	0	\\
3.6e-10	0	\\
4.8e-10	0	\\
6e-10	0	\\
7.2e-10	0	\\
8.4e-10	0	\\
9.6e-10	0	\\
1.08e-09	0	\\
1.2e-09	0	\\
1.32e-09	0	\\
1.44e-09	0	\\
1.56e-09	0	\\
1.68e-09	0	\\
1.8e-09	0	\\
1.92e-09	0	\\
2.04e-09	0	\\
2.16e-09	0	\\
2.28e-09	0	\\
2.4e-09	0	\\
2.52e-09	0	\\
2.64e-09	0	\\
2.76e-09	0	\\
2.88e-09	0	\\
3e-09	0	\\
3.12e-09	0	\\
3.24e-09	0	\\
3.36e-09	0	\\
3.48e-09	0	\\
3.6e-09	0	\\
3.72e-09	0	\\
3.84e-09	0	\\
3.96e-09	0	\\
4.08e-09	0	\\
4.2e-09	0	\\
4.32e-09	0	\\
4.44e-09	0	\\
4.56e-09	0	\\
4.68e-09	0	\\
4.8e-09	0	\\
4.92e-09	0	\\
4.98e-09	0	\\
};
\addplot [color=red,solid,forget plot]
  table[row sep=crcr]{
0	0	\\
1.2e-10	0	\\
2.4e-10	0	\\
3.6e-10	0	\\
4.8e-10	0	\\
6e-10	0	\\
7.2e-10	0	\\
8.4e-10	0	\\
9.6e-10	0	\\
1.08e-09	0	\\
1.2e-09	0	\\
1.32e-09	0	\\
1.44e-09	0	\\
1.56e-09	0	\\
1.68e-09	0	\\
1.8e-09	0	\\
1.92e-09	0	\\
2.04e-09	0	\\
2.16e-09	0	\\
2.28e-09	0	\\
2.4e-09	0	\\
2.52e-09	0	\\
2.64e-09	0	\\
2.76e-09	0	\\
2.88e-09	0	\\
3e-09	0	\\
3.12e-09	0	\\
3.24e-09	0	\\
3.36e-09	0	\\
3.48e-09	0	\\
3.6e-09	0	\\
3.72e-09	0	\\
3.84e-09	0	\\
3.96e-09	0	\\
4.08e-09	0	\\
4.2e-09	0	\\
4.32e-09	0	\\
4.44e-09	0	\\
4.56e-09	0	\\
4.68e-09	0	\\
4.8e-09	0	\\
4.92e-09	0	\\
4.98e-09	0	\\
};
\addplot [color=mycolor1,solid,forget plot]
  table[row sep=crcr]{
0	0	\\
1.2e-10	0	\\
2.4e-10	0	\\
3.6e-10	0	\\
4.8e-10	0	\\
6e-10	0	\\
7.2e-10	0	\\
8.4e-10	0	\\
9.6e-10	0	\\
1.08e-09	0	\\
1.2e-09	0	\\
1.32e-09	0	\\
1.44e-09	0	\\
1.56e-09	0	\\
1.68e-09	0	\\
1.8e-09	0	\\
1.92e-09	0	\\
2.04e-09	0	\\
2.16e-09	0	\\
2.28e-09	0	\\
2.4e-09	0	\\
2.52e-09	0	\\
2.64e-09	0	\\
2.76e-09	0	\\
2.88e-09	0	\\
3e-09	0	\\
3.12e-09	0	\\
3.24e-09	0	\\
3.36e-09	0	\\
3.48e-09	0	\\
3.6e-09	0	\\
3.72e-09	0	\\
3.84e-09	0	\\
3.96e-09	0	\\
4.08e-09	0	\\
4.2e-09	0	\\
4.32e-09	0	\\
4.44e-09	0	\\
4.56e-09	0	\\
4.68e-09	0	\\
4.8e-09	0	\\
4.92e-09	0	\\
4.98e-09	0	\\
};
\addplot [color=mycolor2,solid,forget plot]
  table[row sep=crcr]{
0	0	\\
1.2e-10	0	\\
2.4e-10	0	\\
3.6e-10	0	\\
4.8e-10	0	\\
6e-10	0	\\
7.2e-10	0	\\
8.4e-10	0	\\
9.6e-10	0	\\
1.08e-09	0	\\
1.2e-09	0	\\
1.32e-09	0	\\
1.44e-09	0	\\
1.56e-09	0	\\
1.68e-09	0	\\
1.8e-09	0	\\
1.92e-09	0	\\
2.04e-09	0	\\
2.16e-09	0	\\
2.28e-09	0	\\
2.4e-09	0	\\
2.52e-09	0	\\
2.64e-09	0	\\
2.76e-09	0	\\
2.88e-09	0	\\
3e-09	0	\\
3.12e-09	0	\\
3.24e-09	0	\\
3.36e-09	0	\\
3.48e-09	0	\\
3.6e-09	0	\\
3.72e-09	0	\\
3.84e-09	0	\\
3.96e-09	0	\\
4.08e-09	0	\\
4.2e-09	0	\\
4.32e-09	0	\\
4.44e-09	0	\\
4.56e-09	0	\\
4.68e-09	0	\\
4.8e-09	0	\\
4.92e-09	0	\\
4.98e-09	0	\\
};
\addplot [color=mycolor3,solid,forget plot]
  table[row sep=crcr]{
0	0	\\
1.2e-10	0	\\
2.4e-10	0	\\
3.6e-10	0	\\
4.8e-10	0	\\
6e-10	0	\\
7.2e-10	0	\\
8.4e-10	0	\\
9.6e-10	0	\\
1.08e-09	0	\\
1.2e-09	0	\\
1.32e-09	0	\\
1.44e-09	0	\\
1.56e-09	0	\\
1.68e-09	0	\\
1.8e-09	0	\\
1.92e-09	0	\\
2.04e-09	0	\\
2.16e-09	0	\\
2.28e-09	0	\\
2.4e-09	0	\\
2.52e-09	0	\\
2.64e-09	0	\\
2.76e-09	0	\\
2.88e-09	0	\\
3e-09	0	\\
3.12e-09	0	\\
3.24e-09	0	\\
3.36e-09	0	\\
3.48e-09	0	\\
3.6e-09	0	\\
3.72e-09	0	\\
3.84e-09	0	\\
3.96e-09	0	\\
4.08e-09	0	\\
4.2e-09	0	\\
4.32e-09	0	\\
4.44e-09	0	\\
4.56e-09	0	\\
4.68e-09	0	\\
4.8e-09	0	\\
4.92e-09	0	\\
4.98e-09	0	\\
};
\addplot [color=darkgray,solid,forget plot]
  table[row sep=crcr]{
0	0	\\
1.2e-10	0	\\
2.4e-10	0	\\
3.6e-10	0	\\
4.8e-10	0	\\
6e-10	0	\\
7.2e-10	0	\\
8.4e-10	0	\\
9.6e-10	0	\\
1.08e-09	0	\\
1.2e-09	0	\\
1.32e-09	0	\\
1.44e-09	0	\\
1.56e-09	0	\\
1.68e-09	0	\\
1.8e-09	0	\\
1.92e-09	0	\\
2.04e-09	0	\\
2.16e-09	0	\\
2.28e-09	0	\\
2.4e-09	0	\\
2.52e-09	0	\\
2.64e-09	0	\\
2.76e-09	0	\\
2.88e-09	0	\\
3e-09	0	\\
3.12e-09	0	\\
3.24e-09	0	\\
3.36e-09	0	\\
3.48e-09	0	\\
3.6e-09	0	\\
3.72e-09	0	\\
3.84e-09	0	\\
3.96e-09	0	\\
4.08e-09	0	\\
4.2e-09	0	\\
4.32e-09	0	\\
4.44e-09	0	\\
4.56e-09	0	\\
4.68e-09	0	\\
4.8e-09	0	\\
4.92e-09	0	\\
4.98e-09	0	\\
};
\addplot [color=blue,solid,forget plot]
  table[row sep=crcr]{
0	0	\\
1.2e-10	0	\\
2.4e-10	0	\\
3.6e-10	0	\\
4.8e-10	0	\\
6e-10	0	\\
7.2e-10	0	\\
8.4e-10	0	\\
9.6e-10	0	\\
1.08e-09	0	\\
1.2e-09	0	\\
1.32e-09	0	\\
1.44e-09	0	\\
1.56e-09	0	\\
1.68e-09	0	\\
1.8e-09	0	\\
1.92e-09	0	\\
2.04e-09	0	\\
2.16e-09	0	\\
2.28e-09	0	\\
2.4e-09	0	\\
2.52e-09	0	\\
2.64e-09	0	\\
2.76e-09	0	\\
2.88e-09	0	\\
3e-09	0	\\
3.12e-09	0	\\
3.24e-09	0	\\
3.36e-09	0	\\
3.48e-09	0	\\
3.6e-09	0	\\
3.72e-09	0	\\
3.84e-09	0	\\
3.96e-09	0	\\
4.08e-09	0	\\
4.2e-09	0	\\
4.32e-09	0	\\
4.44e-09	0	\\
4.56e-09	0	\\
4.68e-09	0	\\
4.8e-09	0	\\
4.92e-09	0	\\
4.98e-09	0	\\
};
\addplot [color=black!50!green,solid,forget plot]
  table[row sep=crcr]{
0	0	\\
1.2e-10	0	\\
2.4e-10	0	\\
3.6e-10	0	\\
4.8e-10	0	\\
6e-10	0	\\
7.2e-10	0	\\
8.4e-10	0	\\
9.6e-10	0	\\
1.08e-09	0	\\
1.2e-09	0	\\
1.32e-09	0	\\
1.44e-09	0	\\
1.56e-09	0	\\
1.68e-09	0	\\
1.8e-09	0	\\
1.92e-09	0	\\
2.04e-09	0	\\
2.16e-09	0	\\
2.28e-09	0	\\
2.4e-09	0	\\
2.52e-09	0	\\
2.64e-09	0	\\
2.76e-09	0	\\
2.88e-09	0	\\
3e-09	0	\\
3.12e-09	0	\\
3.24e-09	0	\\
3.36e-09	0	\\
3.48e-09	0	\\
3.6e-09	0	\\
3.72e-09	0	\\
3.84e-09	0	\\
3.96e-09	0	\\
4.08e-09	0	\\
4.2e-09	0	\\
4.32e-09	0	\\
4.44e-09	0	\\
4.56e-09	0	\\
4.68e-09	0	\\
4.8e-09	0	\\
4.92e-09	0	\\
4.98e-09	0	\\
};
\addplot [color=red,solid,forget plot]
  table[row sep=crcr]{
0	0	\\
1.2e-10	0	\\
2.4e-10	0	\\
3.6e-10	0	\\
4.8e-10	0	\\
6e-10	0	\\
7.2e-10	0	\\
8.4e-10	0	\\
9.6e-10	0	\\
1.08e-09	0	\\
1.2e-09	0	\\
1.32e-09	0	\\
1.44e-09	0	\\
1.56e-09	0	\\
1.68e-09	0	\\
1.8e-09	0	\\
1.92e-09	0	\\
2.04e-09	0	\\
2.16e-09	0	\\
2.28e-09	0	\\
2.4e-09	0	\\
2.52e-09	0	\\
2.64e-09	0	\\
2.76e-09	0	\\
2.88e-09	0	\\
3e-09	0	\\
3.12e-09	0	\\
3.24e-09	0	\\
3.36e-09	0	\\
3.48e-09	0	\\
3.6e-09	0	\\
3.72e-09	0	\\
3.84e-09	0	\\
3.96e-09	0	\\
4.08e-09	0	\\
4.2e-09	0	\\
4.32e-09	0	\\
4.44e-09	0	\\
4.56e-09	0	\\
4.68e-09	0	\\
4.8e-09	0	\\
4.92e-09	0	\\
4.98e-09	0	\\
};
\addplot [color=mycolor1,solid,forget plot]
  table[row sep=crcr]{
0	0	\\
1.2e-10	0	\\
2.4e-10	0	\\
3.6e-10	0	\\
4.8e-10	0	\\
6e-10	0	\\
7.2e-10	0	\\
8.4e-10	0	\\
9.6e-10	0	\\
1.08e-09	0	\\
1.2e-09	0	\\
1.32e-09	0	\\
1.44e-09	0	\\
1.56e-09	0	\\
1.68e-09	0	\\
1.8e-09	0	\\
1.92e-09	0	\\
2.04e-09	0	\\
2.16e-09	0	\\
2.28e-09	0	\\
2.4e-09	0	\\
2.52e-09	0	\\
2.64e-09	0	\\
2.76e-09	0	\\
2.88e-09	0	\\
3e-09	0	\\
3.12e-09	0	\\
3.24e-09	0	\\
3.36e-09	0	\\
3.48e-09	0	\\
3.6e-09	0	\\
3.72e-09	0	\\
3.84e-09	0	\\
3.96e-09	0	\\
4.08e-09	0	\\
4.2e-09	0	\\
4.32e-09	0	\\
4.44e-09	0	\\
4.56e-09	0	\\
4.68e-09	0	\\
4.8e-09	0	\\
4.92e-09	0	\\
4.98e-09	0	\\
};
\addplot [color=mycolor2,solid,forget plot]
  table[row sep=crcr]{
0	0	\\
1.2e-10	0	\\
2.4e-10	0	\\
3.6e-10	0	\\
4.8e-10	0	\\
6e-10	0	\\
7.2e-10	0	\\
8.4e-10	0	\\
9.6e-10	0	\\
1.08e-09	0	\\
1.2e-09	0	\\
1.32e-09	0	\\
1.44e-09	0	\\
1.56e-09	0	\\
1.68e-09	0	\\
1.8e-09	0	\\
1.92e-09	0	\\
2.04e-09	0	\\
2.16e-09	0	\\
2.28e-09	0	\\
2.4e-09	0	\\
2.52e-09	0	\\
2.64e-09	0	\\
2.76e-09	0	\\
2.88e-09	0	\\
3e-09	0	\\
3.12e-09	0	\\
3.24e-09	0	\\
3.36e-09	0	\\
3.48e-09	0	\\
3.6e-09	0	\\
3.72e-09	0	\\
3.84e-09	0	\\
3.96e-09	0	\\
4.08e-09	0	\\
4.2e-09	0	\\
4.32e-09	0	\\
4.44e-09	0	\\
4.56e-09	0	\\
4.68e-09	0	\\
4.8e-09	0	\\
4.92e-09	0	\\
4.98e-09	0	\\
};
\addplot [color=mycolor3,solid,forget plot]
  table[row sep=crcr]{
0	0	\\
1.2e-10	0	\\
2.4e-10	0	\\
3.6e-10	0	\\
4.8e-10	0	\\
6e-10	0	\\
7.2e-10	0	\\
8.4e-10	0	\\
9.6e-10	0	\\
1.08e-09	0	\\
1.2e-09	0	\\
1.32e-09	0	\\
1.44e-09	0	\\
1.56e-09	0	\\
1.68e-09	0	\\
1.8e-09	0	\\
1.92e-09	0	\\
2.04e-09	0	\\
2.16e-09	0	\\
2.28e-09	0	\\
2.4e-09	0	\\
2.52e-09	0	\\
2.64e-09	0	\\
2.76e-09	0	\\
2.88e-09	0	\\
3e-09	0	\\
3.12e-09	0	\\
3.24e-09	0	\\
3.36e-09	0	\\
3.48e-09	0	\\
3.6e-09	0	\\
3.72e-09	0	\\
3.84e-09	0	\\
3.96e-09	0	\\
4.08e-09	0	\\
4.2e-09	0	\\
4.32e-09	0	\\
4.44e-09	0	\\
4.56e-09	0	\\
4.68e-09	0	\\
4.8e-09	0	\\
4.92e-09	0	\\
4.98e-09	0	\\
};
\addplot [color=darkgray,solid,forget plot]
  table[row sep=crcr]{
0	0	\\
1.2e-10	0	\\
2.4e-10	0	\\
3.6e-10	0	\\
4.8e-10	0	\\
6e-10	0	\\
7.2e-10	0	\\
8.4e-10	0	\\
9.6e-10	0	\\
1.08e-09	0	\\
1.2e-09	0	\\
1.32e-09	0	\\
1.44e-09	0	\\
1.56e-09	0	\\
1.68e-09	0	\\
1.8e-09	0	\\
1.92e-09	0	\\
2.04e-09	0	\\
2.16e-09	0	\\
2.28e-09	0	\\
2.4e-09	0	\\
2.52e-09	0	\\
2.64e-09	0	\\
2.76e-09	0	\\
2.88e-09	0	\\
3e-09	0	\\
3.12e-09	0	\\
3.24e-09	0	\\
3.36e-09	0	\\
3.48e-09	0	\\
3.6e-09	0	\\
3.72e-09	0	\\
3.84e-09	0	\\
3.96e-09	0	\\
4.08e-09	0	\\
4.2e-09	0	\\
4.32e-09	0	\\
4.44e-09	0	\\
4.56e-09	0	\\
4.68e-09	0	\\
4.8e-09	0	\\
4.92e-09	0	\\
4.98e-09	0	\\
};
\addplot [color=blue,solid,forget plot]
  table[row sep=crcr]{
0	0	\\
1.2e-10	0	\\
2.4e-10	0	\\
3.6e-10	0	\\
4.8e-10	0	\\
6e-10	0	\\
7.2e-10	0	\\
8.4e-10	0	\\
9.6e-10	0	\\
1.08e-09	0	\\
1.2e-09	0	\\
1.32e-09	0	\\
1.44e-09	0	\\
1.56e-09	0	\\
1.68e-09	0	\\
1.8e-09	0	\\
1.92e-09	0	\\
2.04e-09	0	\\
2.16e-09	0	\\
2.28e-09	0	\\
2.4e-09	0	\\
2.52e-09	0	\\
2.64e-09	0	\\
2.76e-09	0	\\
2.88e-09	0	\\
3e-09	0	\\
3.12e-09	0	\\
3.24e-09	0	\\
3.36e-09	0	\\
3.48e-09	0	\\
3.6e-09	0	\\
3.72e-09	0	\\
3.84e-09	0	\\
3.96e-09	0	\\
4.08e-09	0	\\
4.2e-09	0	\\
4.32e-09	0	\\
4.44e-09	0	\\
4.56e-09	0	\\
4.68e-09	0	\\
4.8e-09	0	\\
4.92e-09	0	\\
4.98e-09	0	\\
};
\addplot [color=black!50!green,solid,forget plot]
  table[row sep=crcr]{
0	0	\\
1.2e-10	0	\\
2.4e-10	0	\\
3.6e-10	0	\\
4.8e-10	0	\\
6e-10	0	\\
7.2e-10	0	\\
8.4e-10	0	\\
9.6e-10	0	\\
1.08e-09	0	\\
1.2e-09	0	\\
1.32e-09	0	\\
1.44e-09	0	\\
1.56e-09	0	\\
1.68e-09	0	\\
1.8e-09	0	\\
1.92e-09	0	\\
2.04e-09	0	\\
2.16e-09	0	\\
2.28e-09	0	\\
2.4e-09	0	\\
2.52e-09	0	\\
2.64e-09	0	\\
2.76e-09	0	\\
2.88e-09	0	\\
3e-09	0	\\
3.12e-09	0	\\
3.24e-09	0	\\
3.36e-09	0	\\
3.48e-09	0	\\
3.6e-09	0	\\
3.72e-09	0	\\
3.84e-09	0	\\
3.96e-09	0	\\
4.08e-09	0	\\
4.2e-09	0	\\
4.32e-09	0	\\
4.44e-09	0	\\
4.56e-09	0	\\
4.68e-09	0	\\
4.8e-09	0	\\
4.92e-09	0	\\
4.98e-09	0	\\
};
\addplot [color=red,solid,forget plot]
  table[row sep=crcr]{
0	0	\\
1.2e-10	0	\\
2.4e-10	0	\\
3.6e-10	0	\\
4.8e-10	0	\\
6e-10	0	\\
7.2e-10	0	\\
8.4e-10	0	\\
9.6e-10	0	\\
1.08e-09	0	\\
1.2e-09	0	\\
1.32e-09	0	\\
1.44e-09	0	\\
1.56e-09	0	\\
1.68e-09	0	\\
1.8e-09	0	\\
1.92e-09	0	\\
2.04e-09	0	\\
2.16e-09	0	\\
2.28e-09	0	\\
2.4e-09	0	\\
2.52e-09	0	\\
2.64e-09	0	\\
2.76e-09	0	\\
2.88e-09	0	\\
3e-09	0	\\
3.12e-09	0	\\
3.24e-09	0	\\
3.36e-09	0	\\
3.48e-09	0	\\
3.6e-09	0	\\
3.72e-09	0	\\
3.84e-09	0	\\
3.96e-09	0	\\
4.08e-09	0	\\
4.2e-09	0	\\
4.32e-09	0	\\
4.44e-09	0	\\
4.56e-09	0	\\
4.68e-09	0	\\
4.8e-09	0	\\
4.92e-09	0	\\
4.98e-09	0	\\
};
\addplot [color=mycolor1,solid,forget plot]
  table[row sep=crcr]{
0	0	\\
1.2e-10	0	\\
2.4e-10	0	\\
3.6e-10	0	\\
4.8e-10	0	\\
6e-10	0	\\
7.2e-10	0	\\
8.4e-10	0	\\
9.6e-10	0	\\
1.08e-09	0	\\
1.2e-09	0	\\
1.32e-09	0	\\
1.44e-09	0	\\
1.56e-09	0	\\
1.68e-09	0	\\
1.8e-09	0	\\
1.92e-09	0	\\
2.04e-09	0	\\
2.16e-09	0	\\
2.28e-09	0	\\
2.4e-09	0	\\
2.52e-09	0	\\
2.64e-09	0	\\
2.76e-09	0	\\
2.88e-09	0	\\
3e-09	0	\\
3.12e-09	0	\\
3.24e-09	0	\\
3.36e-09	0	\\
3.48e-09	0	\\
3.6e-09	0	\\
3.72e-09	0	\\
3.84e-09	0	\\
3.96e-09	0	\\
4.08e-09	0	\\
4.2e-09	0	\\
4.32e-09	0	\\
4.44e-09	0	\\
4.56e-09	0	\\
4.68e-09	0	\\
4.8e-09	0	\\
4.92e-09	0	\\
4.98e-09	0	\\
};
\addplot [color=mycolor2,solid,forget plot]
  table[row sep=crcr]{
0	0	\\
1.2e-10	0	\\
2.4e-10	0	\\
3.6e-10	0	\\
4.8e-10	0	\\
6e-10	0	\\
7.2e-10	0	\\
8.4e-10	0	\\
9.6e-10	0	\\
1.08e-09	0	\\
1.2e-09	0	\\
1.32e-09	0	\\
1.44e-09	0	\\
1.56e-09	0	\\
1.68e-09	0	\\
1.8e-09	0	\\
1.92e-09	0	\\
2.04e-09	0	\\
2.16e-09	0	\\
2.28e-09	0	\\
2.4e-09	0	\\
2.52e-09	0	\\
2.64e-09	0	\\
2.76e-09	0	\\
2.88e-09	0	\\
3e-09	0	\\
3.12e-09	0	\\
3.24e-09	0	\\
3.36e-09	0	\\
3.48e-09	0	\\
3.6e-09	0	\\
3.72e-09	0	\\
3.84e-09	0	\\
3.96e-09	0	\\
4.08e-09	0	\\
4.2e-09	0	\\
4.32e-09	0	\\
4.44e-09	0	\\
4.56e-09	0	\\
4.68e-09	0	\\
4.8e-09	0	\\
4.92e-09	0	\\
4.98e-09	0	\\
};
\addplot [color=mycolor3,solid,forget plot]
  table[row sep=crcr]{
0	0	\\
1.2e-10	0	\\
2.4e-10	0	\\
3.6e-10	0	\\
4.8e-10	0	\\
6e-10	0	\\
7.2e-10	0	\\
8.4e-10	0	\\
9.6e-10	0	\\
1.08e-09	0	\\
1.2e-09	0	\\
1.32e-09	0	\\
1.44e-09	0	\\
1.56e-09	0	\\
1.68e-09	0	\\
1.8e-09	0	\\
1.92e-09	0	\\
2.04e-09	0	\\
2.16e-09	0	\\
2.28e-09	0	\\
2.4e-09	0	\\
2.52e-09	0	\\
2.64e-09	0	\\
2.76e-09	0	\\
2.88e-09	0	\\
3e-09	0	\\
3.12e-09	0	\\
3.24e-09	0	\\
3.36e-09	0	\\
3.48e-09	0	\\
3.6e-09	0	\\
3.72e-09	0	\\
3.84e-09	0	\\
3.96e-09	0	\\
4.08e-09	0	\\
4.2e-09	0	\\
4.32e-09	0	\\
4.44e-09	0	\\
4.56e-09	0	\\
4.68e-09	0	\\
4.8e-09	0	\\
4.92e-09	0	\\
4.98e-09	0	\\
};
\addplot [color=darkgray,solid,forget plot]
  table[row sep=crcr]{
0	0	\\
1.2e-10	0	\\
2.4e-10	0	\\
3.6e-10	0	\\
4.8e-10	0	\\
6e-10	0	\\
7.2e-10	0	\\
8.4e-10	0	\\
9.6e-10	0	\\
1.08e-09	0	\\
1.2e-09	0	\\
1.32e-09	0	\\
1.44e-09	0	\\
1.56e-09	0	\\
1.68e-09	0	\\
1.8e-09	0	\\
1.92e-09	0	\\
2.04e-09	0	\\
2.16e-09	0	\\
2.28e-09	0	\\
2.4e-09	0	\\
2.52e-09	0	\\
2.64e-09	0	\\
2.76e-09	0	\\
2.88e-09	0	\\
3e-09	0	\\
3.12e-09	0	\\
3.24e-09	0	\\
3.36e-09	0	\\
3.48e-09	0	\\
3.6e-09	0	\\
3.72e-09	0	\\
3.84e-09	0	\\
3.96e-09	0	\\
4.08e-09	0	\\
4.2e-09	0	\\
4.32e-09	0	\\
4.44e-09	0	\\
4.56e-09	0	\\
4.68e-09	0	\\
4.8e-09	0	\\
4.92e-09	0	\\
4.98e-09	0	\\
};
\addplot [color=blue,solid,forget plot]
  table[row sep=crcr]{
0	0	\\
1.2e-10	0	\\
2.4e-10	0	\\
3.6e-10	0	\\
4.8e-10	0	\\
6e-10	0	\\
7.2e-10	0	\\
8.4e-10	0	\\
9.6e-10	0	\\
1.08e-09	0	\\
1.2e-09	0	\\
1.32e-09	0	\\
1.44e-09	0	\\
1.56e-09	0	\\
1.68e-09	0	\\
1.8e-09	0	\\
1.92e-09	0	\\
2.04e-09	0	\\
2.16e-09	0	\\
2.28e-09	0	\\
2.4e-09	0	\\
2.52e-09	0	\\
2.64e-09	0	\\
2.76e-09	0	\\
2.88e-09	0	\\
3e-09	0	\\
3.12e-09	0	\\
3.24e-09	0	\\
3.36e-09	0	\\
3.48e-09	0	\\
3.6e-09	0	\\
3.72e-09	0	\\
3.84e-09	0	\\
3.96e-09	0	\\
4.08e-09	0	\\
4.2e-09	0	\\
4.32e-09	0	\\
4.44e-09	0	\\
4.56e-09	0	\\
4.68e-09	0	\\
4.8e-09	0	\\
4.92e-09	0	\\
4.98e-09	0	\\
};
\addplot [color=black!50!green,solid,forget plot]
  table[row sep=crcr]{
0	0	\\
1.2e-10	0	\\
2.4e-10	0	\\
3.6e-10	0	\\
4.8e-10	0	\\
6e-10	0	\\
7.2e-10	0	\\
8.4e-10	0	\\
9.6e-10	0	\\
1.08e-09	0	\\
1.2e-09	0	\\
1.32e-09	0	\\
1.44e-09	0	\\
1.56e-09	0	\\
1.68e-09	0	\\
1.8e-09	0	\\
1.92e-09	0	\\
2.04e-09	0	\\
2.16e-09	0	\\
2.28e-09	0	\\
2.4e-09	0	\\
2.52e-09	0	\\
2.64e-09	0	\\
2.76e-09	0	\\
2.88e-09	0	\\
3e-09	0	\\
3.12e-09	0	\\
3.24e-09	0	\\
3.36e-09	0	\\
3.48e-09	0	\\
3.6e-09	0	\\
3.72e-09	0	\\
3.84e-09	0	\\
3.96e-09	0	\\
4.08e-09	0	\\
4.2e-09	0	\\
4.32e-09	0	\\
4.44e-09	0	\\
4.56e-09	0	\\
4.68e-09	0	\\
4.8e-09	0	\\
4.92e-09	0	\\
4.98e-09	0	\\
};
\addplot [color=red,solid,forget plot]
  table[row sep=crcr]{
0	0	\\
1.2e-10	0	\\
2.4e-10	0	\\
3.6e-10	0	\\
4.8e-10	0	\\
6e-10	0	\\
7.2e-10	0	\\
8.4e-10	0	\\
9.6e-10	0	\\
1.08e-09	0	\\
1.2e-09	0	\\
1.32e-09	0	\\
1.44e-09	0	\\
1.56e-09	0	\\
1.68e-09	0	\\
1.8e-09	0	\\
1.92e-09	0	\\
2.04e-09	0	\\
2.16e-09	0	\\
2.28e-09	0	\\
2.4e-09	0	\\
2.52e-09	0	\\
2.64e-09	0	\\
2.76e-09	0	\\
2.88e-09	0	\\
3e-09	0	\\
3.12e-09	0	\\
3.24e-09	0	\\
3.36e-09	0	\\
3.48e-09	0	\\
3.6e-09	0	\\
3.72e-09	0	\\
3.84e-09	0	\\
3.96e-09	0	\\
4.08e-09	0	\\
4.2e-09	0	\\
4.32e-09	0	\\
4.44e-09	0	\\
4.56e-09	0	\\
4.68e-09	0	\\
4.8e-09	0	\\
4.92e-09	0	\\
4.98e-09	0	\\
};
\addplot [color=mycolor1,solid,forget plot]
  table[row sep=crcr]{
0	0	\\
1.2e-10	0	\\
2.4e-10	0	\\
3.6e-10	0	\\
4.8e-10	0	\\
6e-10	0	\\
7.2e-10	0	\\
8.4e-10	0	\\
9.6e-10	0	\\
1.08e-09	0	\\
1.2e-09	0	\\
1.32e-09	0	\\
1.44e-09	0	\\
1.56e-09	0	\\
1.68e-09	0	\\
1.8e-09	0	\\
1.92e-09	0	\\
2.04e-09	0	\\
2.16e-09	0	\\
2.28e-09	0	\\
2.4e-09	0	\\
2.52e-09	0	\\
2.64e-09	0	\\
2.76e-09	0	\\
2.88e-09	0	\\
3e-09	0	\\
3.12e-09	0	\\
3.24e-09	0	\\
3.36e-09	0	\\
3.48e-09	0	\\
3.6e-09	0	\\
3.72e-09	0	\\
3.84e-09	0	\\
3.96e-09	0	\\
4.08e-09	0	\\
4.2e-09	0	\\
4.32e-09	0	\\
4.44e-09	0	\\
4.56e-09	0	\\
4.68e-09	0	\\
4.8e-09	0	\\
4.92e-09	0	\\
4.98e-09	0	\\
};
\addplot [color=mycolor2,solid,forget plot]
  table[row sep=crcr]{
0	0	\\
1.2e-10	0	\\
2.4e-10	0	\\
3.6e-10	0	\\
4.8e-10	0	\\
6e-10	0	\\
7.2e-10	0	\\
8.4e-10	0	\\
9.6e-10	0	\\
1.08e-09	0	\\
1.2e-09	0	\\
1.32e-09	0	\\
1.44e-09	0	\\
1.56e-09	0	\\
1.68e-09	0	\\
1.8e-09	0	\\
1.92e-09	0	\\
2.04e-09	0	\\
2.16e-09	0	\\
2.28e-09	0	\\
2.4e-09	0	\\
2.52e-09	0	\\
2.64e-09	0	\\
2.76e-09	0	\\
2.88e-09	0	\\
3e-09	0	\\
3.12e-09	0	\\
3.24e-09	0	\\
3.36e-09	0	\\
3.48e-09	0	\\
3.6e-09	0	\\
3.72e-09	0	\\
3.84e-09	0	\\
3.96e-09	0	\\
4.08e-09	0	\\
4.2e-09	0	\\
4.32e-09	0	\\
4.44e-09	0	\\
4.56e-09	0	\\
4.68e-09	0	\\
4.8e-09	0	\\
4.92e-09	0	\\
4.98e-09	0	\\
};
\addplot [color=mycolor3,solid,forget plot]
  table[row sep=crcr]{
0	0	\\
1.2e-10	0	\\
2.4e-10	0	\\
3.6e-10	0	\\
4.8e-10	0	\\
6e-10	0	\\
7.2e-10	0	\\
8.4e-10	0	\\
9.6e-10	0	\\
1.08e-09	0	\\
1.2e-09	0	\\
1.32e-09	0	\\
1.44e-09	0	\\
1.56e-09	0	\\
1.68e-09	0	\\
1.8e-09	0	\\
1.92e-09	0	\\
2.04e-09	0	\\
2.16e-09	0	\\
2.28e-09	0	\\
2.4e-09	0	\\
2.52e-09	0	\\
2.64e-09	0	\\
2.76e-09	0	\\
2.88e-09	0	\\
3e-09	0	\\
3.12e-09	0	\\
3.24e-09	0	\\
3.36e-09	0	\\
3.48e-09	0	\\
3.6e-09	0	\\
3.72e-09	0	\\
3.84e-09	0	\\
3.96e-09	0	\\
4.08e-09	0	\\
4.2e-09	0	\\
4.32e-09	0	\\
4.44e-09	0	\\
4.56e-09	0	\\
4.68e-09	0	\\
4.8e-09	0	\\
4.92e-09	0	\\
4.98e-09	0	\\
};
\addplot [color=darkgray,solid,forget plot]
  table[row sep=crcr]{
0	0	\\
1.2e-10	0	\\
2.4e-10	0	\\
3.6e-10	0	\\
4.8e-10	0	\\
6e-10	0	\\
7.2e-10	0	\\
8.4e-10	0	\\
9.6e-10	0	\\
1.08e-09	0	\\
1.2e-09	0	\\
1.32e-09	0	\\
1.44e-09	0	\\
1.56e-09	0	\\
1.68e-09	0	\\
1.8e-09	0	\\
1.92e-09	0	\\
2.04e-09	0	\\
2.16e-09	0	\\
2.28e-09	0	\\
2.4e-09	0	\\
2.52e-09	0	\\
2.64e-09	0	\\
2.76e-09	0	\\
2.88e-09	0	\\
3e-09	0	\\
3.12e-09	0	\\
3.24e-09	0	\\
3.36e-09	0	\\
3.48e-09	0	\\
3.6e-09	0	\\
3.72e-09	0	\\
3.84e-09	0	\\
3.96e-09	0	\\
4.08e-09	0	\\
4.2e-09	0	\\
4.32e-09	0	\\
4.44e-09	0	\\
4.56e-09	0	\\
4.68e-09	0	\\
4.8e-09	0	\\
4.92e-09	0	\\
4.98e-09	0	\\
};
\addplot [color=blue,solid,forget plot]
  table[row sep=crcr]{
0	0	\\
1.2e-10	0	\\
2.4e-10	0	\\
3.6e-10	0	\\
4.8e-10	0	\\
6e-10	0	\\
7.2e-10	0	\\
8.4e-10	0	\\
9.6e-10	0	\\
1.08e-09	0	\\
1.2e-09	0	\\
1.32e-09	0	\\
1.44e-09	0	\\
1.56e-09	0	\\
1.68e-09	0	\\
1.8e-09	0	\\
1.92e-09	0	\\
2.04e-09	0	\\
2.16e-09	0	\\
2.28e-09	0	\\
2.4e-09	0	\\
2.52e-09	0	\\
2.64e-09	0	\\
2.76e-09	0	\\
2.88e-09	0	\\
3e-09	0	\\
3.12e-09	0	\\
3.24e-09	0	\\
3.36e-09	0	\\
3.48e-09	0	\\
3.6e-09	0	\\
3.72e-09	0	\\
3.84e-09	0	\\
3.96e-09	0	\\
4.08e-09	0	\\
4.2e-09	0	\\
4.32e-09	0	\\
4.44e-09	0	\\
4.56e-09	0	\\
4.68e-09	0	\\
4.8e-09	0	\\
4.92e-09	0	\\
4.98e-09	0	\\
};
\addplot [color=black!50!green,solid,forget plot]
  table[row sep=crcr]{
0	0	\\
1.2e-10	0	\\
2.4e-10	0	\\
3.6e-10	0	\\
4.8e-10	0	\\
6e-10	0	\\
7.2e-10	0	\\
8.4e-10	0	\\
9.6e-10	0	\\
1.08e-09	0	\\
1.2e-09	0	\\
1.32e-09	0	\\
1.44e-09	0	\\
1.56e-09	0	\\
1.68e-09	0	\\
1.8e-09	0	\\
1.92e-09	0	\\
2.04e-09	0	\\
2.16e-09	0	\\
2.28e-09	0	\\
2.4e-09	0	\\
2.52e-09	0	\\
2.64e-09	0	\\
2.76e-09	0	\\
2.88e-09	0	\\
3e-09	0	\\
3.12e-09	0	\\
3.24e-09	0	\\
3.36e-09	0	\\
3.48e-09	0	\\
3.6e-09	0	\\
3.72e-09	0	\\
3.84e-09	0	\\
3.96e-09	0	\\
4.08e-09	0	\\
4.2e-09	0	\\
4.32e-09	0	\\
4.44e-09	0	\\
4.56e-09	0	\\
4.68e-09	0	\\
4.8e-09	0	\\
4.92e-09	0	\\
4.98e-09	0	\\
};
\addplot [color=red,solid,forget plot]
  table[row sep=crcr]{
0	0	\\
1.2e-10	0	\\
2.4e-10	0	\\
3.6e-10	0	\\
4.8e-10	0	\\
6e-10	0	\\
7.2e-10	0	\\
8.4e-10	0	\\
9.6e-10	0	\\
1.08e-09	0	\\
1.2e-09	0	\\
1.32e-09	0	\\
1.44e-09	0	\\
1.56e-09	0	\\
1.68e-09	0	\\
1.8e-09	0	\\
1.92e-09	0	\\
2.04e-09	0	\\
2.16e-09	0	\\
2.28e-09	0	\\
2.4e-09	0	\\
2.52e-09	0	\\
2.64e-09	0	\\
2.76e-09	0	\\
2.88e-09	0	\\
3e-09	0	\\
3.12e-09	0	\\
3.24e-09	0	\\
3.36e-09	0	\\
3.48e-09	0	\\
3.6e-09	0	\\
3.72e-09	0	\\
3.84e-09	0	\\
3.96e-09	0	\\
4.08e-09	0	\\
4.2e-09	0	\\
4.32e-09	0	\\
4.44e-09	0	\\
4.56e-09	0	\\
4.68e-09	0	\\
4.8e-09	0	\\
4.92e-09	0	\\
4.98e-09	0	\\
};
\addplot [color=mycolor1,solid,forget plot]
  table[row sep=crcr]{
0	0	\\
1.2e-10	0	\\
2.4e-10	0	\\
3.6e-10	0	\\
4.8e-10	0	\\
6e-10	0	\\
7.2e-10	0	\\
8.4e-10	0	\\
9.6e-10	0	\\
1.08e-09	0	\\
1.2e-09	0	\\
1.32e-09	0	\\
1.44e-09	0	\\
1.56e-09	0	\\
1.68e-09	0	\\
1.8e-09	0	\\
1.92e-09	0	\\
2.04e-09	0	\\
2.16e-09	0	\\
2.28e-09	0	\\
2.4e-09	0	\\
2.52e-09	0	\\
2.64e-09	0	\\
2.76e-09	0	\\
2.88e-09	0	\\
3e-09	0	\\
3.12e-09	0	\\
3.24e-09	0	\\
3.36e-09	0	\\
3.48e-09	0	\\
3.6e-09	0	\\
3.72e-09	0	\\
3.84e-09	0	\\
3.96e-09	0	\\
4.08e-09	0	\\
4.2e-09	0	\\
4.32e-09	0	\\
4.44e-09	0	\\
4.56e-09	0	\\
4.68e-09	0	\\
4.8e-09	0	\\
4.92e-09	0	\\
4.98e-09	0	\\
};
\addplot [color=mycolor2,solid,forget plot]
  table[row sep=crcr]{
0	0	\\
1.2e-10	0	\\
2.4e-10	0	\\
3.6e-10	0	\\
4.8e-10	0	\\
6e-10	0	\\
7.2e-10	0	\\
8.4e-10	0	\\
9.6e-10	0	\\
1.08e-09	0	\\
1.2e-09	0	\\
1.32e-09	0	\\
1.44e-09	0	\\
1.56e-09	0	\\
1.68e-09	0	\\
1.8e-09	0	\\
1.92e-09	0	\\
2.04e-09	0	\\
2.16e-09	0	\\
2.28e-09	0	\\
2.4e-09	0	\\
2.52e-09	0	\\
2.64e-09	0	\\
2.76e-09	0	\\
2.88e-09	0	\\
3e-09	0	\\
3.12e-09	0	\\
3.24e-09	0	\\
3.36e-09	0	\\
3.48e-09	0	\\
3.6e-09	0	\\
3.72e-09	0	\\
3.84e-09	0	\\
3.96e-09	0	\\
4.08e-09	0	\\
4.2e-09	0	\\
4.32e-09	0	\\
4.44e-09	0	\\
4.56e-09	0	\\
4.68e-09	0	\\
4.8e-09	0	\\
4.92e-09	0	\\
4.98e-09	0	\\
};
\addplot [color=mycolor3,solid,forget plot]
  table[row sep=crcr]{
0	0	\\
1.2e-10	0	\\
2.4e-10	0	\\
3.6e-10	0	\\
4.8e-10	0	\\
6e-10	0	\\
7.2e-10	0	\\
8.4e-10	0	\\
9.6e-10	0	\\
1.08e-09	0	\\
1.2e-09	0	\\
1.32e-09	0	\\
1.44e-09	0	\\
1.56e-09	0	\\
1.68e-09	0	\\
1.8e-09	0	\\
1.92e-09	0	\\
2.04e-09	0	\\
2.16e-09	0	\\
2.28e-09	0	\\
2.4e-09	0	\\
2.52e-09	0	\\
2.64e-09	0	\\
2.76e-09	0	\\
2.88e-09	0	\\
3e-09	0	\\
3.12e-09	0	\\
3.24e-09	0	\\
3.36e-09	0	\\
3.48e-09	0	\\
3.6e-09	0	\\
3.72e-09	0	\\
3.84e-09	0	\\
3.96e-09	0	\\
4.08e-09	0	\\
4.2e-09	0	\\
4.32e-09	0	\\
4.44e-09	0	\\
4.56e-09	0	\\
4.68e-09	0	\\
4.8e-09	0	\\
4.92e-09	0	\\
4.98e-09	0	\\
};
\end{axis}
\end{tikzpicture}%
	\label{fig:wheel-torque-labcar}
	\caption{Power efficiency.}
\end{figure}
\fi

\chapter*{Appendix A}
\includecode[matlab]{task-3.m}{C:/Users/Sander/Documents/GitHub/EV2020/docs/skill5/sander-erwin/resources/matlab/Skill_5_Assignment_1_Task_3.m}{lst:task-3.m}

\end{document}