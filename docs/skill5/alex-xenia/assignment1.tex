%!TEX program = xelatex+makeindex+bibtex
\documentclass[final]{scrreprt} %scrreprt of scrartcl
\input{../../library/preamble.tex}
\input{../../library/style.tex}
\addbibresource{../../library/bibliography.bib}
\begin{document}

\chapter*{Assignment 1}
\label{ch:sk5-ass1-task1}
\section*{Task 1}
\begin{center}
$Z_s = \dfrac{1}{j \omega C}$\,\,\,\, $Z_p = j \omega L$

$Z_0 = \dfrac{Z_s}{2} + \sqrt{\dfrac{(Z_s)^2}{4} + Z_s \cdot Z_p}$

$Z_0 = \dfrac{1}{2j \omega C} + \sqrt{\dfrac{(\dfrac{1}{j \omega C})^2}{4} + \dfrac{1}{j \omega C} \cdot j \omega L}$

\begin{equation}
Z_0 = \dfrac{1}{2j \omega C} + \sqrt{\dfrac{L}{C} - \dfrac{1}{\omega^2 C^2 4}}
\end{equation}
\end{center}
For $\lim{\Delta  L\to 0}$ the values of the series capacitance $\Delta C$ and shunt inductance $\Delta L$ decrease. The ratio $\Delta L$ and $\Delta C$ will remain constant while the fractions beneath and infront of the square root will approach zero. 
This results in the characteristic impedance:
\begin{equation}
Z_0 = \sqrt{\dfrac{\Delta L}{\Delta C}}
\end{equation}
The cutoff frequency is:
\begin{equation}
W_0 = \sqrt{\dfrac{1}{4LC}}
\end{equation}\\
\\
To determine how the network behaves filter-wise we use the following equation: $V_n = \gamma^n U$. 
This equation gives the voltage at the n-th section of the ladder network.\\
The factor $\gamma$ is equal to $1 - \dfrac{Z_s}{Z_0}$.
With $Z_s = \dfrac{1}{j \omega c}$ we work out the equation to end up at:
\begin{align}
\gamma = \dfrac{\sqrt{\dfrac{L}{C} - \dfrac{1}{\omega^2 C^2 4}} - \dfrac{1}{2j \omega C}}{\sqrt{\dfrac{L}{C} - \dfrac{1}{\omega^2 C^2 4}} + \dfrac{1}{2j \omega C}}
\end{align}
\\For $\lim{\omega \to \infty}$ $\gamma$ approaches 1 and thus the network behaves as a high-pass filter.

\label{ch:sk5-ass1-task2}
\section*{Task 2}
The space-time domain of the voltage across the channel equals:
\begin{align}
\label{eq:space-time}
V(z,t) = U(t-\gamma z) + \Gamma U(t + \gamma(z-2l)) - \Gamma U(t-\gamma(z+2l)) - \Gamma^2U(t+\gamma(z-4l))....
\end{align}
With $\Gamma = \dfrac{Z_l - Z_0}{Z_l + Z_0}$ and $z = l$. 
In case the terminal is open there will be an infinite load impedance i.e. $Z_l = \infty$. 
As a consequence $\Gamma = 1$. 
Thus resulting in the following space-time expression 
\begin{align*}
V(z,t) = U(t-\gamma l) + U(t + \gamma(-l)) - U(t-\gamma(3l)) - U(t+\gamma(-3l))....
\end{align*}
Similarly, if the terminal is short-circuited $Z_l = 0$ and $\Gamma = -1$. 
Resulting in the following space-time expression:
\begin{align*}
V(z,t) = U(t-\gamma l) + -U(t + \gamma(-l)) + U(t-\gamma(3l)) - U(t+\gamma(-3l))....
\end{align*}

A positive reflection coefficient points to a positively reflected wave while a negative reflection coefficient points to a negatively reflected wave.

\label{ch:sk5-ass1-task3}
\section*{Task 3}
To gain insight in the behavior of the voltage and current in a transmission line a space-time evolution is set up.
The transmission line has the following parameters: $L_0 = 0,5 \mu H, C_0 = 2 pF, Z_l = 100 \Omega$.
Accordingly the charactersistic impedance is equal to:
\begin{equation}
Z_0 = \sqrt{\dfrac{L_0}{C_0}} = 500 \Omega
\end{equation}
The transmission line is excited by a rectangular pulse with the following expression:
\begin{equation}
U(t) = V_{max} \cdot \{\epsilon(t) - \epsilon(t-t_w)\}
\end{equation}
For the space-time evolution equation \ref{eq:space-time} is implemented in MatLab and WORDT NOG GEFIXT. Consequently the space-time evolution can be plotted with help of the Bscan-plot file provided as seen in \ref{app:bscan} and the MatLab code written in \ref{app:task3}.

\begin{figure}[H]
	\centering
	\includegraphics[width=\linewidth]{resources/space-time-voltage.png}
	\caption{Space time evolution of the voltage.}
	\label{fig:space-time-voltage}
\end{figure}
As seen in figure \ref{fig:space-time-voltage} the signal propagates through the transmission channel and reflects negatively because the reflection coefficient is a negative number. Because this coefficient is bigger than -1 the reflected signal decreases in amplitude.

\end{document}