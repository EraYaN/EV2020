%!TEX program = xelatex+makeindex+bibtex
\documentclass[final]{scrreprt} %scrreprt of scrartcl
% Include all project wide packages here.
\usepackage{fullpage}
\usepackage{polyglossia}
\setmainlanguage{english}
\usepackage{csquotes}
\usepackage{graphicx}
\usepackage{epstopdf}
\usepackage{pdfpages}
\usepackage{caption}
\usepackage[list=true]{subcaption}
\usepackage{float}
\usepackage{standalone}
\usepackage{import}
\usepackage{tocloft}
\usepackage{wrapfig}
\usepackage{authblk}
\usepackage{array}
\usepackage{booktabs}
\usepackage[toc,page,title,titletoc]{appendix}
\usepackage{xunicode}
\usepackage{fontspec}
\usepackage{pgfplots}
\usepackage{SIunits}
\usepackage{units}
\pgfplotsset{compat=newest}
\pgfplotsset{plot coordinates/math parser=false}
\newlength\figureheight 
\newlength\figurewidth
\usepackage{amsmath}
\usepackage{mathtools}
\usepackage{unicode-math}
\usepackage[
    backend=bibtexu,
	texencoding=utf8,
bibencoding=utf8,
    style=ieee,
    sortlocale=en_US,
    language=auto
]{biblatex}
\usepackage{listings}
\newcommand{\includecode}[3][c]{\lstinputlisting[caption=#2, escapechar=, style=#1]{#3}}
\newcommand{\superscript}[1]{\ensuremath{^{\textrm{#1}}}}
\newcommand{\subscript}[1]{\ensuremath{_{\textrm{#1}}}}


\newcommand{\chapternumber}{\thechapter}
\renewcommand{\appendixname}{Bijlage}
\renewcommand{\appendixtocname}{Bijlagen}
\renewcommand{\appendixpagename}{Bijlagen}

\usepackage[hidelinks]{hyperref} %<--------ALTIJD ALS LAATSTE

\renewcommand{\familydefault}{\sfdefault}

\setmainfont[Ligatures=TeX]{Myriad Pro}
\setmathfont{Asana Math}
\setmonofont{Lucida Console}

\usepackage{titlesec, blindtext, color}
\definecolor{gray75}{gray}{0.75}
\newcommand{\hsp}{\hspace{20pt}}
\titleformat{\chapter}[hang]{\Huge\bfseries}{\chapternumber\hsp\textcolor{gray75}{|}\hsp}{0pt}{\Huge\bfseries}
\renewcommand{\familydefault}{\sfdefault}
\renewcommand{\arraystretch}{1.2}
\setlength\parindent{0pt}

%For code listings
\definecolor{black}{rgb}{0,0,0}
\definecolor{browntags}{rgb}{0.65,0.1,0.1}
\definecolor{bluestrings}{rgb}{0,0,1}
\definecolor{graycomments}{rgb}{0.4,0.4,0.4}
\definecolor{redkeywords}{rgb}{1,0,0}
\definecolor{bluekeywords}{rgb}{0.13,0.13,0.8}
\definecolor{greencomments}{rgb}{0,0.5,0}
\definecolor{redstrings}{rgb}{0.9,0,0}
\definecolor{purpleidentifiers}{rgb}{0.01,0,0.01}


\lstdefinestyle{csharp}{
language=[Sharp]C,
showspaces=false,
showtabs=false,
breaklines=true,
showstringspaces=false,
breakatwhitespace=true,
escapeinside={(*@}{@*)},
columns=fullflexible,
commentstyle=\color{greencomments},
keywordstyle=\color{bluekeywords}\bfseries,
stringstyle=\color{redstrings},
identifierstyle=\color{purpleidentifiers},
basicstyle=\ttfamily\small}

\lstdefinestyle{c}{
language=C,
showspaces=false,
showtabs=false,
breaklines=true,
showstringspaces=false,
breakatwhitespace=true,
escapeinside={(*@}{@*)},
columns=fullflexible,
commentstyle=\color{greencomments},
keywordstyle=\color{bluekeywords}\bfseries,
stringstyle=\color{redstrings},
identifierstyle=\color{purpleidentifiers},
}

\lstdefinestyle{matlab}{
language=Matlab,
showspaces=false,
showtabs=false,
breaklines=true,
showstringspaces=false,
breakatwhitespace=true,
escapeinside={(*@}{@*)},
columns=fullflexible,
commentstyle=\color{greencomments},
keywordstyle=\color{bluekeywords}\bfseries,
stringstyle=\color{redstrings},
identifierstyle=\color{purpleidentifiers}
}

\lstdefinestyle{vhdl}{
language=VHDL,
showspaces=false,
showtabs=false,
breaklines=true,
showstringspaces=false,
breakatwhitespace=true,
escapeinside={(*@}{@*)},
columns=fullflexible,
commentstyle=\color{greencomments},
keywordstyle=\color{bluekeywords}\bfseries,
stringstyle=\color{redstrings},
identifierstyle=\color{purpleidentifiers}
}

\lstdefinestyle{xaml}{
language=XML,
showspaces=false,
showtabs=false,
breaklines=true,
showstringspaces=false,
breakatwhitespace=true,
escapeinside={(*@}{@*)},
columns=fullflexible,
commentstyle=\color{greencomments},
keywordstyle=\color{redkeywords},
stringstyle=\color{bluestrings},
tagstyle=\color{browntags},
morestring=[b]",
  morecomment=[s]{<?}{?>},
  morekeywords={xmlns,version,typex:AsyncRecords,x:Arguments,x:Boolean,x:Byte,x:Char,x:Class,x:ClassAttributes,x:ClassModifier,x:Code,x:ConnectionId,x:Decimal,x:Double,x:FactoryMethod,x:FieldModifier,x:Int16,x:Int32,x:Int64,x:Key,x:Members,x:Name,x:Object,x:Property,x:Shared,x:Single,x:String,x:Subclass,x:SynchronousMode,x:TimeSpan,x:TypeArguments,x:Uid,x:Uri,x:XData,Grid.Column,Grid.ColumnSpan,Click,ClipToBounds,Content,DropDownOpened,FontSize,Foreground,Header,Height,HorizontalAlignment,HorizontalContentAlignment,IsCancel,IsDefault,IsEnabled,IsSelected,Margin,MinHeight,MinWidth,Padding,SnapsToDevicePixels,Target,TextWrapping,Title,VerticalAlignment,VerticalContentAlignment,Width,WindowStartupLocation,Binding,Mode,OneWay,xmlns:x}
}

%defaults
\lstset{
basicstyle=\ttfamily\small,
extendedchars=false,
numbers=left,
numberstyle=\ttfamily\tiny,
stepnumber=1,
tabsize=4,
numbersep=5pt
}
\addbibresource{../../library/bibliography.bib}
\begin{document}

\chapter{Assignment 1: One dimensional wave propagation - the transmission channel model}
\label{ch:sk5-ass1-task1}
\section*{Task 1}
\begin{center}
$Z_s = \dfrac{1}{j \omega C}$\,\,\,\, $Z_p = j \omega L$

$Z_0 = \dfrac{Z_s}{2} + \sqrt{\dfrac{(Z_s)^2}{4} + Z_s \cdot Z_p}$

$Z_0 = \dfrac{1}{2j \omega C} + \sqrt{\dfrac{(\dfrac{1}{j \omega C})^2}{4} + \dfrac{1}{j \omega C} \cdot j \omega L}$

$Z_0 = \dfrac{1}{2j \omega C} + \sqrt{\dfrac{L}{C} - \dfrac{1}{\omega^2 C^2 4}}$
\end{center}
For $\lim{\Delta  L\to 0}$ the values of the series capacitance $\Delta C$ and shunt inductance $\Delta L$ decrease. The ratio $\Delta L$ and $\Delta C$ will remain constant while the fractions beneath and infront of the square root will approach zero. This results in $Z_0 = \sqrt{\dfrac{\Delta L}{\Delta C}}$.

The cutoff frequency is $W_0 = \sqrt{\dfrac{1}{4LC}}$.\\
\\
To determine how the network behaves filter-wise we use the following equation: $V_n = \gamma^n U$. Which gives the voltage at the n-th section of the ladder network. The factor $\gamma$ is equal to $1 - \dfrac{Z_s}{Z_0}$.
Working out this equation we end up at 
\begin{align*}
\gamma = \dfrac{\sqrt{\dfrac{L}{C} - \dfrac{1}{\omega^2 C^2 4}} - \dfrac{1}{2j \omega C}}{\sqrt{\dfrac{L}{C} - \dfrac{1}{\omega^2 C^2 4}} + \dfrac{1}{2j \omega C}}
\end{align*}
\\For $\lim{\omega \to \infty}$ $\gamma$ approaches 1 and thus the network is a high-pass filter.

\label{ch:sk5-ass1-task2}
\section*{Task 2}
The space-time domain of the voltage accros the channel equals:
\begin{align*}
V(z,t) = U(t-\gamma z) + \Gamma U(t + \gamma(z-2l)) - \Gamma U(t-\gamma(z+2l)) - \Gamma^2U(t+\gamma(z-4l))....
\end{align*}
With $\Gamma = \dfrac{Z_l - Z_0}{Z_l + Z_0}$ and $z = l$. In case the terminal is open there will be an infinite load impedance i.e. $Z_l = \infty$. As a consequence $\Gamma = 1$. Thus resulting in the following space-time expression 
\begin{align*}
V(z,t) = U(t-\gamma l) + U(t + \gamma(-l)) - U(t-\gamma(3l)) - U(t+\gamma(-3l))....
\end{align*}
Similarly, if the terminal is short-circuited $Z_l = 0$ and $\Gamma = -1$. Resulting in the following space-time expression:
\begin{align*}
V(z,t) = U(t-\gamma l) + -U(t + \gamma(-l)) + U(t-\gamma(3l)) - U(t+\gamma(-3l))....
\end{align*}

\end{document}