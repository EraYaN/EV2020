%!TEX program = xelatex+makeindex+bibtex
\documentclass[final]{scrreprt} %scrreprt of scrartcl
\input{../../library/preamble.tex}
\input{../../library/style.tex}
\addbibresource{../../library/bibliography.bib}
\begin{document}

\chapter{Assignment 1: One dimensional wave propagation - the transmission channel model}
\label{ch:sk5-ass1-task1}
\section*{Task 1}
\begin{center}
$Z_s = \dfrac{1}{j \omega C}$\,\,\,\, $Z_p = j \omega L$

$Z_0 = \dfrac{Z_s}{2} + \sqrt{\dfrac{(Z_s)^2}{4} + Z_s \cdot Z_p}$

$Z_0 = \dfrac{1}{2j \omega C} + \sqrt{\dfrac{(\dfrac{1}{j \omega C})^2}{4} + \dfrac{1}{j \omega C} \cdot j \omega L}$

$Z_0 = \dfrac{1}{2j \omega C} + \sqrt{\dfrac{L}{C} - \dfrac{1}{\omega^2 C^2 4}}$
\end{center}
For $\lim{\Delta  L\to 0}$ the values of the series capacitance $\Delta C$ and shunt inductance $\Delta L$ decrease. The ratio $\Delta L$ and $\Delta C$ will remain constant while the fractions beneath and infront of the square root will approach zero. This results in $Z_0 = \sqrt{\dfrac{\Delta L}{\Delta C}}$.

The cutoff frequency is $W_0 = \sqrt{\dfrac{1}{4LC}}$.\\
\\
To determine how the network behaves filter-wise we use the following equation: $V_n = \gamma^n U$. Which gives the voltage at the n-th section of the ladder network. The factor $\gamma$ is equal to $1 - \dfrac{Z_s}{Z_0}$.
Working out this equation we end up at 
\begin{align*}
\gamma = \dfrac{\sqrt{\dfrac{L}{C} - \dfrac{1}{\omega^2 C^2 4}} - \dfrac{1}{2j \omega C}}{\sqrt{\dfrac{L}{C} - \dfrac{1}{\omega^2 C^2 4}} + \dfrac{1}{2j \omega C}}
\end{align*}
\\For $\lim{\omega \to \infty}$ $\gamma$ approaches 1 and thus the network is a high-pass filter.

\label{ch:sk5-ass1-task2}
\section*{Task 2}
The space-time domain of the voltage accros the channel equals:
\begin{align*}
V(z,t) = U(t-\gamma z) + \Gamma U(t + \gamma(z-2l)) - \Gamma U(t-\gamma(z+2l)) - \Gamma^2U(t+\gamma(z-4l))....
\end{align*}
With $\Gamma = \dfrac{Z_l - Z_0}{Z_l + Z_0}$ and $z = l$. In case the terminal is open there will be an infinite load impedance i.e. $Z_l = \infty$. As a consequence $\Gamma = 1$. Thus resulting in the following space-time expression 
\begin{align*}
V(z,t) = U(t-\gamma l) + U(t + \gamma(-l)) - U(t-\gamma(3l)) - U(t+\gamma(-3l))....
\end{align*}
Similarly, if the terminal is short-circuited $Z_l = 0$ and $\Gamma = -1$. Resulting in the following space-time expression:
\begin{align*}
V(z,t) = U(t-\gamma l) + -U(t + \gamma(-l)) + U(t-\gamma(3l)) - U(t+\gamma(-3l))....
\end{align*}

\end{document}