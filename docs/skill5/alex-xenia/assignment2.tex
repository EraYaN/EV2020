%!TEX program = xelatex+makeindex+bibtex
\documentclass[final]{scrreprt} %scrreprt of scrartcl
\input{../../library/preamble.tex}
\input{../../library/style.tex}
\addbibresource{../../library/bibliography.bib}
\begin{document}

\chapter*{Assignment 2}
\label{ch:sk5-ass2}
\section*{Task 1}
Given is that the input impedance $Z_{in}$ is equal to:
\begin{align}
Z_{in}(-l) = Z_0 \cdot \dfrac{Z_l + jZ_0 \tan (\beta l)}{Z_0 + jZ_l \tan (\beta l)}
\end{align}
We now solve for $l$:
\begin{equation}
Z_{in} \cdot (Z_0 + jZ_l \tan (\beta l)) = Z_0 \cdot (Z_l + jZ_0 \tan (\beta l))
\end{equation}
\begin{equation}
\tan( \beta l) \cdot (jZ_{in} Z_l - jZ_0^2) = Z_0(Z_l - Z_{in})
\end{equation}
\begin{equation}
\tan (\beta l) = \dfrac{Z_0(Z_l - Z_{in})}{jZ_{in} Z_l - jZ_0^2}
\end{equation}
\begin{equation}
\label{eq:beta}
l = \dfrac{1}{\beta} \cdot \arctan (Z_0 \dfrac{Z_l - Z_{in}}{jZ_{in} Z_l - jZ_0^2})
\end{equation}
Beta denotes the wave number and reads:
\begin{equation}
\label{eq:range}
\beta = \dfrac{2 \pi}{\gamma_{med}} = \dfrac{\omega}{c_{med}}
\end{equation}

Because the tangent is periodic, the measured input impedance is the same for different lengths. 
Therefore the determined length will not be unambiguous. However $\beta L$ has period $\pi$ and if $\beta L$ is between $\dfrac{- \pi}{2}$ and $\dfrac{\pi}{2}$ the solutions for L will be unambiguous. 
Thus the unambiguous range is
\begin{equation}
\dfrac{- \pi}{2 \beta} < L < \dfrac{\pi}{2 \beta}
\end{equation}

Measuring at two frequencies ($f1 < f2$) will again yield undetermined lengths. However, the unambiguous range now changes as can be derived from equation \ref{eq:range}. In this case the shortest period determines the unambiguous range.
\end{document}
