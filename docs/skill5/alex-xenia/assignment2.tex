%!TEX program = xelatex+makeindex+bibtex
\documentclass[final]{scrreprt} %scrreprt of scrartcl
\input{../../library/preamble.tex}
\input{../../library/style.tex}
\addbibresource{../../library/bibliography.bib}
\documentclass{article}
\usepackage{listings}
\usepackage{color} %red, green, blue, yellow, cyan, magenta, black, white
\definecolor{mygreen}{RGB}{28,172,0} % color values Red, Green, Blue
\definecolor{mylilas}{RGB}{170,55,241}

\usepackage[numbered]{mcode}
\begin{document}

\lstset{language=Matlab,%
    %basicstyle=\color{red},
    breaklines=true,%
    morekeywords={matlab2tikz},
    keywordstyle=\color{blue},%
    morekeywords=[2]{1}, keywordstyle=[2]{\color{black}},
    identifierstyle=\color{black},%
    stringstyle=\color{mylilas},
    commentstyle=\color{mygreen},%
    showstringspaces=false,%without this there will be a symbol in the places where there is a space
    numbers=left,%
    numberstyle={\tiny \color{black}},% size of the numbers
    numbersep=9pt, % this defines how far the numbers are from the text
    emph=[1]{for,end,break},emphstyle=[1]\color{red}, %some words to emphasise
    %emph=[2]{word1,word2}, emphstyle=[2]{style},    
}

\chapter{Assignment 2}
\label{ch:sk5-ass2}
\section{Task 1}
Given is that the input impedance $Z_{in}$ is equal to:
\begin{align}
Z_{in}(-l) = Z_0 \cdot \dfrac{Z_l + jZ_0 \tan (\beta l)}{Z_0 + jZ_l \tan (\beta l)}
\end{align}
We now solve for $l$:
\begin{equation}
Z_{in} \cdot (Z_0 + jZ_l \tan (\beta l)) = Z_0 \cdot (Z_l + jZ_0 \tan (\beta l))
\end{equation}
\begin{equation}
\tan( \beta l) \cdot (jZ_{in} Z_l - jZ_0^2) = Z_0(Z_l - Z_{in})
\end{equation}
\begin{equation}
\tan (\beta l) = \dfrac{Z_0(Z_l - Z_{in})}{jZ_{in} Z_l - jZ_0^2}
\end{equation}
\begin{equation}
\label{eq:beta}
l = \dfrac{1}{\beta} \cdot \arctan (Z_0 \dfrac{Z_l - Z_{in}}{jZ_{in} Z_l - jZ_0^2})
\end{equation}
Beta denotes the wave number and reads:
\begin{equation}
\label{eq:range}
\beta = \dfrac{2 \pi}{\gamma_{med}} = \dfrac{\omega}{c_{med}}
\end{equation}

Because the tangent is periodic, the measured input impedance is the same for different lengths. 
Therefore the determined length will not be unambiguous. However $\beta L$ has period $\pi$ and if $\beta L$ is between $\dfrac{- \pi}{2}$ and $\dfrac{\pi}{2}$ the solutions for L will be unambiguous. 
Thus the unambiguous range is
\begin{equation}
\dfrac{- \pi}{2 \beta} < L < \dfrac{\pi}{2 \beta}
\end{equation}

Measuring at two frequencies ($f1 < f2$) will again yield ambiguous lengths. The length of the transmission line is the point where the two ambiguous lengths for the different frequencies are the same. In this case the frequency with the largest period determines the unambiguous range.


\section{Task 2}

With MatLab the lengths are calculated for different lengths. According to the manual the length of the line is estimated to be at most 3m. This will help in case some calculated lengths are greater than 3 meter. The code can be found in \ref{app:ass2task2}.
For the frequencies and impedancse stated in the file on BlackBoard. The found lengths were: 0.0043m, -0.0012m and -0.0026m.

\end{document}
