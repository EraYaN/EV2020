%!TEX program = xelatex+makeindex+bibtex
\documentclass[final]{scrreprt} %scrreprt of scrartcl
\input{../../library/preamble.tex}
\input{../../library/style.tex}
\addbibresource{../../library/bibliography.bib}
\begin{document}

\chapter{Assignment 1}
\label{ch:mod2-ass1}
Before you can connect to KITT you have to pair the bluetooth chip on the KITT control board and the bluetooth chip in your computer.
You can use the windows bluetooth settings to do so.
There was a precompiled communications library provided on blackboard.
We opted to non use MATLAB for our communication because matlab is not so well suited for this kind of task.
I, Erwin, have developed a neat and working communications module for our EPO-2 project.
It was easy and straight forward to reuse this, because it contains all nessecary elements.
It is written in C\# and uses WPF as it's UI library, it is available from my \href{https://github.com/EraYaN/MiDeRP}{EPO-2 repository}.
The serial communication used there was completely byte based, while the serial communication with KITT is string based.
So some changes were made to optimize the handling of the data for strings.
With the C\# implementation of the serial ports we have had no problems what so ever, in stark contrast to the pure MATLAB implementation.
MATLAB is not capable to correctly handle closing the serial port.
To circumvent any problems in the future with provided MATLAB scripts for use in the final project, we added a MATLAB interface to the software.
MATLAB exposes a COM interface on Windows, the performance of this interface is quite good.
So no need to compile the MATLAB code into an Assembly of a pure c-dll.

We added a simulated joystick control to the UI to control KITT (instead of using keyboard controls) to have greater control over the car.
Retrieving information from KITT proved to be a problem with the original EPO-2 implementation of the communications library, so we fixed that.
We define the ping time as the time from the last byte send to the first byte recieved.
This to eliminate the differences in processing time and packet length.
We also saw that you can not send to many commands to the car.
The main MCU then just can't handle it and doesn't do anything or sends back garbled data.
One command per "ping time" seems to be about the maximum.

The distance sensors are faily accure close by, the further away you get the less accurate the measurements become.
In the next chapter (Chapter \ref{ch:mod2-ass2}) we will go into more detail what the actual parameters of the car are.


\end{document}