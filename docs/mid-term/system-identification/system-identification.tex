%!TEX program = xelatex+makeindex+bibtex
\documentclass[final]{scrreprt} %scrreprt of scrartcl
% Include all project wide packages here.
\usepackage{fullpage}
\usepackage{polyglossia}
\setmainlanguage{english}
\usepackage{csquotes}
\usepackage{graphicx}
\usepackage{epstopdf}
\usepackage{pdfpages}
\usepackage{caption}
\usepackage[list=true]{subcaption}
\usepackage{float}
\usepackage{standalone}
\usepackage{import}
\usepackage{tocloft}
\usepackage{wrapfig}
\usepackage{authblk}
\usepackage{array}
\usepackage{booktabs}
\usepackage[toc,page,title,titletoc]{appendix}
\usepackage{xunicode}
\usepackage{fontspec}
\usepackage{pgfplots}
\usepackage{SIunits}
\usepackage{units}
\pgfplotsset{compat=newest}
\pgfplotsset{plot coordinates/math parser=false}
\newlength\figureheight 
\newlength\figurewidth
\usepackage{amsmath}
\usepackage{mathtools}
\usepackage{unicode-math}
\usepackage[
    backend=bibtexu,
	texencoding=utf8,
bibencoding=utf8,
    style=ieee,
    sortlocale=en_US,
    language=auto
]{biblatex}
\usepackage{listings}
\newcommand{\includecode}[3][c]{\lstinputlisting[caption=#2, escapechar=, style=#1]{#3}}
\newcommand{\superscript}[1]{\ensuremath{^{\textrm{#1}}}}
\newcommand{\subscript}[1]{\ensuremath{_{\textrm{#1}}}}


\newcommand{\chapternumber}{\thechapter}
\renewcommand{\appendixname}{Bijlage}
\renewcommand{\appendixtocname}{Bijlagen}
\renewcommand{\appendixpagename}{Bijlagen}

\usepackage[hidelinks]{hyperref} %<--------ALTIJD ALS LAATSTE

\renewcommand{\familydefault}{\sfdefault}

\setmainfont[Ligatures=TeX]{Myriad Pro}
\setmathfont{Asana Math}
\setmonofont{Lucida Console}

\usepackage{titlesec, blindtext, color}
\definecolor{gray75}{gray}{0.75}
\newcommand{\hsp}{\hspace{20pt}}
\titleformat{\chapter}[hang]{\Huge\bfseries}{\chapternumber\hsp\textcolor{gray75}{|}\hsp}{0pt}{\Huge\bfseries}
\renewcommand{\familydefault}{\sfdefault}
\renewcommand{\arraystretch}{1.2}
\setlength\parindent{0pt}

%For code listings
\definecolor{black}{rgb}{0,0,0}
\definecolor{browntags}{rgb}{0.65,0.1,0.1}
\definecolor{bluestrings}{rgb}{0,0,1}
\definecolor{graycomments}{rgb}{0.4,0.4,0.4}
\definecolor{redkeywords}{rgb}{1,0,0}
\definecolor{bluekeywords}{rgb}{0.13,0.13,0.8}
\definecolor{greencomments}{rgb}{0,0.5,0}
\definecolor{redstrings}{rgb}{0.9,0,0}
\definecolor{purpleidentifiers}{rgb}{0.01,0,0.01}


\lstdefinestyle{csharp}{
language=[Sharp]C,
showspaces=false,
showtabs=false,
breaklines=true,
showstringspaces=false,
breakatwhitespace=true,
escapeinside={(*@}{@*)},
columns=fullflexible,
commentstyle=\color{greencomments},
keywordstyle=\color{bluekeywords}\bfseries,
stringstyle=\color{redstrings},
identifierstyle=\color{purpleidentifiers},
basicstyle=\ttfamily\small}

\lstdefinestyle{c}{
language=C,
showspaces=false,
showtabs=false,
breaklines=true,
showstringspaces=false,
breakatwhitespace=true,
escapeinside={(*@}{@*)},
columns=fullflexible,
commentstyle=\color{greencomments},
keywordstyle=\color{bluekeywords}\bfseries,
stringstyle=\color{redstrings},
identifierstyle=\color{purpleidentifiers},
}

\lstdefinestyle{matlab}{
language=Matlab,
showspaces=false,
showtabs=false,
breaklines=true,
showstringspaces=false,
breakatwhitespace=true,
escapeinside={(*@}{@*)},
columns=fullflexible,
commentstyle=\color{greencomments},
keywordstyle=\color{bluekeywords}\bfseries,
stringstyle=\color{redstrings},
identifierstyle=\color{purpleidentifiers}
}

\lstdefinestyle{vhdl}{
language=VHDL,
showspaces=false,
showtabs=false,
breaklines=true,
showstringspaces=false,
breakatwhitespace=true,
escapeinside={(*@}{@*)},
columns=fullflexible,
commentstyle=\color{greencomments},
keywordstyle=\color{bluekeywords}\bfseries,
stringstyle=\color{redstrings},
identifierstyle=\color{purpleidentifiers}
}

\lstdefinestyle{xaml}{
language=XML,
showspaces=false,
showtabs=false,
breaklines=true,
showstringspaces=false,
breakatwhitespace=true,
escapeinside={(*@}{@*)},
columns=fullflexible,
commentstyle=\color{greencomments},
keywordstyle=\color{redkeywords},
stringstyle=\color{bluestrings},
tagstyle=\color{browntags},
morestring=[b]",
  morecomment=[s]{<?}{?>},
  morekeywords={xmlns,version,typex:AsyncRecords,x:Arguments,x:Boolean,x:Byte,x:Char,x:Class,x:ClassAttributes,x:ClassModifier,x:Code,x:ConnectionId,x:Decimal,x:Double,x:FactoryMethod,x:FieldModifier,x:Int16,x:Int32,x:Int64,x:Key,x:Members,x:Name,x:Object,x:Property,x:Shared,x:Single,x:String,x:Subclass,x:SynchronousMode,x:TimeSpan,x:TypeArguments,x:Uid,x:Uri,x:XData,Grid.Column,Grid.ColumnSpan,Click,ClipToBounds,Content,DropDownOpened,FontSize,Foreground,Header,Height,HorizontalAlignment,HorizontalContentAlignment,IsCancel,IsDefault,IsEnabled,IsSelected,Margin,MinHeight,MinWidth,Padding,SnapsToDevicePixels,Target,TextWrapping,Title,VerticalAlignment,VerticalContentAlignment,Width,WindowStartupLocation,Binding,Mode,OneWay,xmlns:x}
}

%defaults
\lstset{
basicstyle=\ttfamily\small,
extendedchars=false,
numbers=left,
numberstyle=\ttfamily\tiny,
stepnumber=1,
tabsize=4,
numbersep=5pt
}
\addbibresource{../../library/bibliography.bib}

\pgfplotsset{every axis/.append style={
        legend style={ legend cell align=left, font=\small }}}

\begin{document}

\chapter{System Identification}

\section*{System model}
Using Equation \ref{eq:simple}, a state-space model can be derived.
This model is displayed in Equation \ref{eq:simple_matrix}, in which $x$ corresponds with the position and $\dot{x}$ with the velocity.
This model is of second order.
\begin{equation}
	m \ddot{x} = -\rho_f \dot{x} + u
	\label{eq:simple}
\end{equation}

\begin{equation}
	\begin{bmatrix}
		\dot{x} \\
		\ddot{x}
	\end{bmatrix} =
	\begin{bmatrix}
		0 & 1 \\
		0 & -\frac{\rho_f}{m}
	\end{bmatrix}
	\begin{bmatrix}
		x \\
		\dot{x}
	\end{bmatrix} +
	\begin{bmatrix}
		0 \\
		\frac{1}{m}
	\end{bmatrix}
	u
	\label{eq:simple_matrix}
\end{equation}

The characteristic equation can be derived using Equation \ref{eq:char_eq}, which leads to Equation \ref{eq:char_eq_final} for the model in Equation \ref{eq:simple_matrix}.
This leads to the eigenvalues of $0$ and $-\frac{\rho_f}{m}$, the first being on the imaginary axis and the second on the left half of the complex plane.
Both poles should be on the left half of the complex plane to be stable, but since one is not, it is not stable.
\\
\begin{equation}
	\left| I \lambda - A \right| = 0
	\label{eq:char_eq}
\end{equation}
\begin{equation}
	\lambda(\lambda + \frac{\rho_f}{m}) = 0
	\label{eq:char_eq_final}
\end{equation}
\\
To find out about the controllability we set up a controllability matrix as seen in Equation \ref{eq:controllability}.
\begin{equation}
C =
\begin{bmatrix}
B & AB
\end{bmatrix}
\end{equation}
\begin{equation}
C =
\begin{bmatrix}
0 & \frac{1}{m}\\
\frac{1}{m} & \frac{\rho_f}{m^2}
\end{bmatrix}\label{eq:controllability}
\end{equation}

The preceding matrix is rank two, which is full-rank for a second order system.
Hence the system is controllable, which by definition makes the system stabilizable too.
\\
A similar procedure holds to find out the observability.
First we set up the observability matrix:
\begin{equation}
C =
\begin{bmatrix}
C\\
CA
\end{bmatrix}
\end{equation}
\begin{equation}
C =
\begin{bmatrix}
1 & 0\\
0 & 1
\end{bmatrix}\label{eq:observability}
\end{equation}
The observability matrix is rank two and thus full rank.
Therefore the system is observable.
\\
By adding the engine model to the car model, a better understanding of the controllability of the whole system can be achieved.
The motor model can be solved to Equation \ref{eq:motor}, which can be used in the original model of Equation \ref{eq:simple}.
This yields a third order system, as displayed in Equation \ref{eq:total_system}, but by neglecting the motor coil inductance $L$, it can be simplified to the second order system of Equation \ref{eq:total_system_simple}.
Equation \ref{eq:total_system_matrix} is the state-space model of this simplified model.

\begin{equation}
	V = \frac{Lm\ddot{x}}{k_g k_w k_t} + k_g k_w k_t \dot{x} + \frac{uR}{k_g k_w k_t}
	\label{eq:motor}
\end{equation}

\begin{equation}
	\dddot{x} = -\frac{\ddot{x}}{L} -(\frac{k_g^2 k_w^2 k_t^2}{mRL} + \frac{\rho_f}{mL})\dot{x} + \frac{k_g k_w k_t}{mRL}V
	\label{eq:total_system}
\end{equation}

\begin{equation}
	\ddot{x} = -(\frac{k_g^2 k_w^2 k_t^2}{mR} + \frac{\rho_f}{m})\dot{x} + \frac{k_g k_w k_t}{mR}V
	\label{eq:total_system_simple}
\end{equation}

\begin{equation}
	\begin{bmatrix}
		\dot{x} \\
		\ddot{x}
	\end{bmatrix} =
	\begin{bmatrix}
		0 & 1 \\
		0 & -\frac{\rho_f}{m} - \frac{k_g k_w k_t}{mR}
	\end{bmatrix}
	\begin{bmatrix}
		x \\
		\dot{x}
	\end{bmatrix} +
	\begin{bmatrix}
		0 \\
		\frac{k_g k_w k_t}{mR}
	\end{bmatrix}
	V
	\label{eq:total_system_matrix}
\end{equation}

The same way as calculated with equations 1.3 to 1.6 we determined that the system is not stable.
However, the system is controllable (and therefore stabilizable) and observable.

\section*{Simulink model}
To test the different compensator possibilities, a Simulink model was designed.
The model can be found in Figure \ref{app:model}.
This model was made to optimally simulate the car and the compensator.
In this model, the Bluetooth connection delay is also included since it could cause the system to oscillate when compensating for an outdated system state.
These delays, to and from the car, were both set to \unit[50]{ms}, which is about half the average ping time as found in Module 2.
The output of the system is clamped between -15 and 15, and will always be an integer value instead of the default double type.

\section*{Task 3: Identifying the models' parameters}
Now that a model has been constructed we need to find the values for the $A$ and $B$ matrix.
To find these we sent the car a signal and measured the output of the ultrasound sensors.
In Figure \ref{fig:identification} the results are on display.
The blue signal represents the input signal and periodically goes above $x=0$ and beneath it.
In between those transitions there will be a small delay where the input remains zero for a while.
The model output will react accordingly whereby the postion increases for a positive input and decreases for a negative input.
The output signal from KITT shows a phase shift which might be contributed to the car's acceleration that is not instantaneous, there will be a certain delay.

\begin{figure} [H]
\centering
\begin{tikzpicture}
\begin{axis}[
    height=7cm,
    width=0.8\linewidth,
    %title={Model identification},
    axis y line*=left,
    xlabel={Time (\unit{s})},
    ylabel={Distance (\unit{cm})},
    xmin=0, xmax=20,
    ymin=-200, ymax=200,
    xtick={0,2,4,6,8,10,12,14,16,18,20},
    ytick={-200,-150,-100,-50,0,50,100,150,200},
    legend pos=south west,
    ymajorgrids=true,
    grid style=dashed,
]

\addplot [color=blue, mark=none] table [col sep=comma] {resources/csv/identification/kitt.csv}; \label{KITT}
\addplot [color=red, mark=none] table [col sep=comma] {resources/csv/identification/model.csv}; \label{Model}
 
\end{axis}

\begin{axis}[
    height=7cm,
    width=0.8\linewidth,
    axis y line*=right,
    ylabel={Drive},
    xmin=0, xmax=20,
    hide x axis,
    ymin=-10, ymax=6,
    ytick={-10,-8,-6,-4,-2,0,2,4,6},
    legend pos=south west,
]
\addlegendimage{/pgfplots/refstyle=KITT}
\addlegendentry{KITT}
\addlegendimage{/pgfplots/refstyle=Model}
\addlegendentry{Model}

\addplot [color=gray, mark=none] table [col sep=comma] {resources/csv/identification/drive.csv}; \label{Drive}
\addlegendentry{Drive}
 
\end{axis}
\end{tikzpicture}
\caption{The model's and the car's response}
\label{fig:identification}
\end{figure}

The measurements resulted in the following $A$ matrix and associated eigenvalues.

\begin{equation}
A=
\begin{bmatrix}
  0 & 1 \\
  -0.2968 & -1.7820
 \end{bmatrix}
\qquad
eig(A)=
\begin{bmatrix}
  -0.1860 \\
  -1.5960
 \end{bmatrix}
\end{equation}

However, since Matlab tries to fit the model as good as possible, having no actual clue what it is calculating, errors might occur.
One of these errors is the value of $A_{21}$, which is not equal to zero as expected.
According to this value of $A_{21}$, the acceleration depends on the location of the car, which makes no physical sense in the given situation.
Therefore, the value of $A_{21}$ is an error, and thus replaced by zero.
This resulted in the following new $A$ matrix and its associated eigenvalues.

\begin{equation}
A=
\begin{bmatrix}
  0 & 1 \\
  0 & -1.7820
 \end{bmatrix}
\qquad
eig(A)=
\begin{bmatrix}
   0 \\
  -1.7820
 \end{bmatrix}
\end{equation}

The system is not stable since not both eigenvalues are in the left half of the complex plane.
Generally speaking the simulation is a pretty good approximation whereby the phase shift should be kept in mind.

\section*{Task 4: Observer Design}
In order to get accurate information about the velocity of the car, an observer is constructed.
Both poles have been set to -10, which were determined by ajusting the poles untill the observer followed the actual velocity as good as possible.
The main requirement is that the poles are located in the left complex halfplane.
The more negative the real part of a pole is, the stronger the damping.
However choosing the poles too far to the left is not practical since there are delays in the system.
These delays could cause an oscillation when the correction is overly performed, resulting in large amounts of noise on the observer's distance and velocity.
Using MatLab and setting the poles at -10 we calculated the following $L$ matrix:
\begin{equation}
L=
\begin{bmatrix}
  18.2180 \\
  67.5355
 \end{bmatrix}
\end{equation}

The observer was tested in an altered version of the Simulink model.
This version replaced the input from the compensator to the car with a sequence of random driving commands in the range of -5 to 5, which updated every \unit[100]{ms}.
The objective was to follow the velocity as good as possible by measuring only the position of the car.
The altered model kept the realistic delay of \unit[50]{ms} between the car and the observer.
The results can be seen in Figures \ref{fig:observer_distance} and \ref{fig:observer_velocity} where the blue lines represent the actual distance and velocity and the red lines the observed distance and velocity.
Both the distance and velocity are followed very good by the observer, even when the car started with a velocity of 5 m/s, which was unknown to the observer.

\begin{figure} [H]
\centering
\begin{tikzpicture}
\begin{axis}[
    height=7cm,
    width=0.8\linewidth,
    %title={Observer distance},
    axis y line*=left,
    xlabel={Time (\unit{s})},
    ylabel={Distance (\unit{cm})},
    xmin=0, xmax=10,
    ymin=280, ymax=305,
    xtick={0,1,2,3,4,5,6,7,8,9,10},
    %ytick={280, 290, 300, 310, 320},
    legend pos=north east,
    ymajorgrids=true,
    grid style=dashed,
]

\addplot [color=blue, mark=none] table [col sep=comma] {resources/csv/observer/distances_kitt.csv}; \label{KITT}
\addplot [color=red, mark=none] table [col sep=comma] {resources/csv/observer/distances_observer.csv}; \label{Observer}
\legend{KITT, Observer}

\end{axis}
\end{tikzpicture}
\caption{Observer performance, distance compare}
\label{fig:observer_distance}
\end{figure}

\begin{figure} [H]
\centering
\begin{tikzpicture}
\begin{axis}[
    height=7cm,
    width=0.8\linewidth,
    %title={Observer velocity},
    axis y line*=left,
    xlabel={Time (\unit{s})},
    ylabel={Velocity (\unit{cm/s})},
    xmin=0, xmax=10,
    ymin=-15, ymax=10,
    xtick={0,1,2,3,4,5,6,7,8,9,10},
    %ytick={280, 290, 300, 310, 320},
    legend pos=north east,
    ymajorgrids=true,
    grid style=dashed,
]

\addplot [color=blue, mark=none] table [col sep=comma] {resources/csv/observer/velocities_kitt.csv}; \label{KITT}
\addplot [color=red, mark=none] table [col sep=comma] {resources/csv/observer/velocities_observer.csv}; \label{Observer}
\legend{KITT, Observer}

\end{axis}
\end{tikzpicture}
\caption{Observer performance, velocity compare}
\label{fig:observer_velocity}
\end{figure}

Within a half second, the observer finds and follows the actual velocity as seen in Figure \ref{fig:observer_velocity}.
This is in agreement with the requirements for an observer; the difference $x(t)-\hat{x}$ converges to zero and after a certain time point $t_i$ where $x(t) = \hat{x}$, this equation remains satisfied for $t$ \geq $~t_i$.
Without the delay, the preceding statement would be exactly implemented, resulting in an ideal observer.
But since there is a delay, this ideal observer can not exist.

\section*{Task 5: Controller design}
The controller was designed to both drive to the target position and reduce the velocity to zero.
These two objectives oppose each other when the car is not on the target position and standing still.
Therefore, a good ratio between the two must be found.
This can be achieved by placing the poles of the $K$ matrix, where multiple criteria must be fulfilled.
Firstly, both poles must be located in the left half of the complex plane ($\operatorname{Re}(pole) < 0$).
Secondly, the system must not be underdamped.
Crittically damped would be the best option for an ideal system, but since the system experiences large delays, this might not be the perfect damping.
Overdamping the system creates a more stable situation since the actual state of the system is not defined errorless.
Also, the system parameters such as the $B$ matrix are variable and not well-defined.
When the supercap voltage is below the default \unit[20]{V}, the forward force on the car caused by a given drive command will be lower.
The $K$ matrix poles were set to $-4$ and $-1.5$, resulting in the following matrix:

\begin{equation}
\centering
K = 
\begin{bmatrix}
  -0.1502 & -0.0931
\end{bmatrix}
\end{equation}

When simulating with this $K$ matrix, Figure \ref{fig:distances} displays the distance to wall, which is set to a target of \unit[6]{cm}.
The blue line which represends KITT ends at 6 cm with a maximum variation of \unit[1]{cm} as for all other target locations.
The red line will always ends near zero, since it represends the observer state's distance.
Since this distance is defined as KITT's actual distance minus the target location, the observer's distance always needs to converge to zero.

\begin{figure} [H]
\centering
\begin{tikzpicture}
\begin{axis}[
    height=7cm,
    width=0.8\linewidth,
    %title={Driving to target point: 6 \unit{cm}},
    axis y line*=left,
    xlabel={Time (\unit{s})},
    ylabel={Distance (\unit{cm})},
    xmin=0, xmax=5,
    ymin=-50, ymax=350,
    xtick={0,1,2,3,4,5},
    ytick={-50,0,50,100,150,200,250,300,350},
    legend pos=north east,
    ymajorgrids=true,
    grid style=dashed,
]

\addplot [color=blue, mark=none] table [col sep=comma] {resources/csv/final/distances_kitt.csv}; \label{KITT}
\addplot [color=red, mark=none] table [col sep=comma] {resources/csv/final/distances_compensator.csv}; \label{Compensation}
 
\end{axis}

\begin{axis}[
    height=7cm,
    width=0.8\linewidth,
    axis y line*=right,
    ylabel={Drive},
    xmin=0, xmax=5,
    hide x axis,
    ymin=-6, ymax=18,
    ytick={-6,-3,0,3,6,9,12,15,18},
]
\addlegendimage{/pgfplots/refstyle=KITT}
\addlegendentry{KITT}
\addlegendimage{/pgfplots/refstyle=Compensation}
\addlegendentry{Compensation}

\addplot [color=gray, mark=none] table [col sep=comma] {resources/csv/final/drive.csv}; \label{Drive}
\addlegendentry{Drive}
 
\end{axis}
\end{tikzpicture}
\caption{Driving to target location: \unit[6]{cm}}
\label{fig:distances}
\end{figure}

\end{document}