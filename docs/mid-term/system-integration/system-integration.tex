%!TEX program = xelatex+makeindex+bibtex
\documentclass[final]{scrreprt} %scrreprt of scrartcl
\input{../../library/preamble.tex}
\input{../../library/style.tex}
\addbibresource{../../library/bibliography.bib}
\begin{document}

\chapter{System Integration}
\label{ch:system-integration}
We use a program written in C\# to control the car.
The full source in included in Appendix \ref{app:source}.
First, we will quickly explain the different components of the program.
After that we will go into some extra detail on why we choose this approach.
A system diagram is shown in figure \ref{fig:system-diagram}.
For any implementation details please consult the source code.
\begin{figure}[H]
	\centering    	
    	\includegraphics[width=\textwidth]{resources/system-diagram.pdf}
    	\caption{The system in diagram form}
    	\label{fig:system-diagram}
\end{figure}
Here follows a rundown of all the components.
\section{GUI (Visualization)}
This part handles everything to do with the GUI and the Databindingsa to the underlying parts.
The associated classes and files are:
\begin{itemize}
\item App (Appendix \ref{appsec:App.xaml.cs} \& \ref{appsec:App.xaml})
\item MainWindow (Appendix \ref{appsec:MainWindow.xaml.cs} \& \ref{appsec:MainWindow.xaml})
\item Data (Appendix \ref{appsec:Data.cs})
\item Arrow (Appendix \ref{appsec:Arrow.cs})
\item Visualization (Appendix \ref{appsec:Visualization.cs})
\item Databindings (Appendix \ref{appsec:Databindings.cs})
\end{itemize}
\section{Controller}
This part is the "brain", here the serial events are handled and commands sent.
The associated classes and files are:
\begin{itemize}
\item Controller (Appendix \ref{appsec:Controller.cs})
\item Data (Appendix \ref{appsec:Data.cs})
\end{itemize}
\section{Navigation}
This is not yet in use. Now the Target field on the Controller class serves this function.
The associated classes, interfaces and files are:
\begin{itemize}
\item StandardNavigation (Appendix \ref{appsec:StandardNavigation.cs})
\item INavigator (Appendix \ref{appsec:INavigator.cs})
\end{itemize}
\section{Observer}
This part is the gateway to the implemented model.
The associated classes and files are:
\begin{itemize}
\item Observer (Appendix \ref{appsec:Observer.cs})
\end{itemize}
\section{Model}
This part constsist of a standarized interface and a couple of implemented models.
Those are drop-in replacements of each other.
Our currently used model is PDEpicModel.
The associated classes, interfaces and files are:
\begin{itemize}
\item IModel (Appendix \ref{appsec:IModel.cs})
\item PDEpicModel (Appendix \ref{appsec:PDEpicModel.cs})
\item PDDefaultModel (Appendix \ref{appsec:PDDefaultModel.cs})
\item PIDModel (Appendix \ref{appsec:PIDModel.cs})
\end{itemize}
\section{Serial}
This part handles all serial communication asynchronously and drives the receive events.
The associated classes, interfaces and files are:
\begin{itemize}
\item ISerial (Appendix \ref{appsec:ISerial.cs})
\item SerialDataEvent (Appendix \ref{appsec:SerialDataEvent.cs})
\item SerialInterface (Appendix \ref{appsec:SerialInterface.cs})
\end{itemize}
\section{Bluetooth Chipset}
The adapter we use most of the time is an Intel(R) Centrino(R) Wireless Bluetooth(R) 4.0 + High Speed Adapter embedded together with the WiFi module in one of our own laptops.

\end{document}